\hypertarget{section}{%
\section{1}\label{section}}

\bibleverse{1} Paulus, berufen zum Apostel Jesu Christi durch den Willen
Gottes, und Bruder Sosthenes \bibleverse{2} der Gemeinde zu Korinth, den
Geheiligten in Christo Jesu, den berufenen Heiligen samt allen denen,
die anrufen den Namen unseres Herrn Jesu Christi an allen ihren und
unseren Orten: \footnote{\textbf{1:2} 1Kor 6,11; Apg 9,14; Apg 18,1-7}
\bibleverse{3} Gnade sei mit euch und Friede von Gott, unserem Vater,
und dem Herrn Jesus Christus! \bibleverse{4} Ich danke meinem Gott
allezeit eurethalben für die Gnade Gottes, die euch gegeben ist in
Christo Jesu, \bibleverse{5} dass ihr seid durch ihn an allen Stücken
reich gemacht, an aller Lehre und in aller Erkenntnis; \bibleverse{6}
wie denn die Predigt von Christus in euch kräftig geworden ist,
\bibleverse{7} also dass ihr keinen Mangel habt an irgendeiner Gabe und
wartet nur auf die Offenbarung unseres Herrn Jesu Christi,
\bibleverse{8} welcher auch wird euch fest erhalten bis ans Ende, dass
ihr unsträflich seid auf den Tag unseres Herrn Jesu Christi. \footnote{\textbf{1:8}
  Phil 1,6; 1Thes 3,13} \bibleverse{9} Denn Gott ist treu, durch welchen
ihr berufen seid zur Gemeinschaft seines Sohnes Jesu Christi, unseres
Herrn. \footnote{\textbf{1:9} 1Thes 5,24} \bibleverse{10} Ich ermahne
euch aber, liebe Brüder, durch den Namen unseres Herrn Jesu Christi,
dass ihr allzumal einerlei Rede führet und lasset nicht Spaltungen unter
euch sein, sondern haltet fest aneinander in einem Sinne und in einerlei
Meinung. \footnote{\textbf{1:10} 1Kor 11,18; Röm 15,5; Phil 2,2}
\bibleverse{11} Denn es ist vor mich gekommen, liebe Brüder, durch die
aus Chloes Gesinde von euch, dass Zank unter euch sei. \bibleverse{12}
Ich sage aber davon, dass unter euch einer spricht: Ich bin paulisch,
der andere: Ich bin apollisch, der dritte: Ich bin kephisch, der vierte:
Ich bin christisch. \bibleverse{13} Wie? Ist Christus nun zertrennt? Ist
denn Paulus für euch gekreuzigt? Oder seid ihr auf des Paulus Namen
getauft? \bibleverse{14} Ich danke Gott, dass ich niemand unter euch
getauft habe außer Krispus und Gajus, \footnote{\textbf{1:14} Apg 18,8;
  Röm 16,23} \bibleverse{15} dass nicht jemand sagen möge, ich hätte auf
meinen Namen getauft. \bibleverse{16} Ich habe aber auch getauft des
Stephanas Hausgesinde; weiter weiß ich nicht, ob ich etliche andere
getauft habe. \bibleverse{17} Denn Christus hat mich nicht gesandt, zu
taufen, sondern das Evangelium zu predigen, nicht mit klugen Worten, auf
dass nicht das Kreuz Christi zunichte werde. \bibleverse{18} Denn das
Wort vom Kreuz ist eine Torheit denen, die verloren werden; uns aber,
die wir selig werden ist's eine Gotteskraft. \footnote{\textbf{1:18}
  2Kor 4,3; Röm 1,16} \bibleverse{19} Denn es steht geschrieben: „Ich
will zunichte machen die Weisheit der Weisen, und den Verstand der
Verständigen will ich verwerfen.`` \bibleverse{20} Wo sind die Klugen?
Wo sind die Schriftgelehrten? Wo sind die Weltweisen? Hat nicht Gott die
Weisheit dieser Welt zur Torheit gemacht? \bibleverse{21} Denn dieweil
die Welt durch ihre Weisheit Gott in seiner Weisheit nicht erkannte,
gefiel es Gott wohl, durch törichte Predigt selig zu machen die, die
daran glauben. \bibleverse{22} Sintemal die Juden Zeichen fordern und
die Griechen nach Weisheit fragen, \footnote{\textbf{1:22} Mt 12,38; Joh
  4,48; Apg 17,18-21} \bibleverse{23} wir aber predigen den gekreuzigten
Christus, den Juden ein Ärgernis und den Griechen eine Torheit;
\footnote{\textbf{1:23} 1Kor 2,14; Gal 5,11; Apg 17,32} \bibleverse{24}
denen aber, die berufen sind, Juden und Griechen, predigen wir Christum,
göttliche Kraft und göttliche Weisheit. \footnote{\textbf{1:24} Kol 2,3}
\bibleverse{25} Denn die göttliche Torheit ist weiser, als die Menschen
sind; und die göttliche Schwachheit ist stärker, als die Menschen sind.
\bibleverse{26} Sehet an, liebe Brüder, eure Berufung: nicht viel Weise
nach dem Fleisch, nicht viel Gewaltige, nicht viel Edle sind berufen.
\bibleverse{27} Sondern was töricht ist vor der Welt, das hat Gott
erwählt, dass er die Weisen zu Schanden mache; und was schwach ist vor
der Welt, das hat Gott erwählt, dass er zu Schanden mache, was stark
ist; \bibleverse{28} und das Unedle vor der Welt und das Verachtete hat
Gott erwählt, und das da nichts ist, dass er zunichte mache, was etwas
ist, \bibleverse{29} auf dass sich vor ihm kein Fleisch rühme.
\footnote{\textbf{1:29} Röm 3,27; Eph 2,9} \bibleverse{30} Von ihm kommt
auch ihr her in Christo Jesu, welcher uns gemacht ist von Gott zur
Weisheit und zur Gerechtigkeit und zur Heiligung und zur Erlösung,
\footnote{\textbf{1:30} Jer 23,5-6; 2Kor 5,21; Joh 17,19; Mt 20,28}
\bibleverse{31} auf dass (wie geschrieben steht), „wer sich rühmt, der
rühme sich des Herrn!{}`` \footnote{\textbf{1:31} 2Kor 10,17}

\hypertarget{section-1}{%
\section{2}\label{section-1}}

\bibleverse{1} Und ich, liebe Brüder, da ich zu euch kam, kam ich nicht
mit hohen Worten oder hoher Weisheit, euch zu verkündigen die göttliche
Predigt. \bibleverse{2} Denn ich hielt mich nicht dafür, dass ich etwas
wüsste unter euch, als allein Jesum Christum, den Gekreuzigten.
\bibleverse{3} Und ich war bei euch mit Schwachheit und mit Furcht und
mit großem Zittern; \footnote{\textbf{2:3} Apg 18,9; 2Kor 10,1; Gal 4,13}
\bibleverse{4} und mein Wort und meine Predigt war nicht in vernünftigen
Reden menschlicher Weisheit, sondern in Beweisung des Geistes und der
Kraft, \footnote{\textbf{2:4} Mt 10,20} \bibleverse{5} auf dass euer
Glaube bestehe nicht auf Menschenweisheit, sondern auf Gottes Kraft.
\footnote{\textbf{2:5} 1Thes 1,5} \bibleverse{6} Wovon wir aber reden,
das ist dennoch Weisheit bei den Vollkommenen; nicht eine Weisheit
dieser Welt, auch nicht der Obersten dieser Welt, welche vergehen.
\bibleverse{7} Sondern wir reden von der heimlichen, verborgenen
Weisheit Gottes, welche Gott verordnet hat vor der Welt zu unserer
Herrlichkeit, \bibleverse{8} welche keiner von den Obersten dieser Welt
erkannt hat; denn wenn sie die erkannt hätten, hätten sie den Herrn der
Herrlichkeit nicht gekreuzigt. \bibleverse{9} Sondern wie geschrieben
steht: „Was kein Auge gesehen hat und kein Ohr gehört hat und in keines
Menschen Herz gekommen ist, was Gott bereitet hat denen, die ihn
lieben.`` \bibleverse{10} Uns aber hat es Gott offenbart durch seinen
Geist; denn der Geist erforscht alle Dinge, auch die Tiefen der
Gottheit. \footnote{\textbf{2:10} Mt 13,11; Kol 1,26} \bibleverse{11}
Denn welcher Mensch weiß, was im Menschen ist, als der Geist des
Menschen, der in ihm ist? Also auch weiß niemand, was in Gott ist, als
der Geist Gottes. \bibleverse{12} Wir aber haben nicht empfangen den
Geist der Welt, sondern den Geist aus Gott, dass wir wissen können, was
uns von Gott gegeben ist; \bibleverse{13} welches wir auch reden, nicht
mit Worten, welche menschliche Weisheit lehren kann, sondern mit Worten,
die der heilige Geist lehrt, und richten geistliche Sachen geistlich.
\bibleverse{14} Der natürliche Mensch aber vernimmt nichts vom Geist
Gottes; es ist ihm eine Torheit, und er kann es nicht erkennen; denn es
muss geistlich gerichtet sein. \footnote{\textbf{2:14} 1Kor 1,23; Joh
  8,47} \bibleverse{15} Der geistliche aber richtet alles, und wird von
niemand gerichtet. \bibleverse{16} Denn „wer hat des Herrn Sinn erkannt,
oder wer will ihn unterweisen?{}`` Wir aber haben Christi Sinn.

\hypertarget{section-2}{%
\section{3}\label{section-2}}

\bibleverse{1} Und ich, liebe Brüder, konnte nicht mit euch reden als
mit Geistlichen, sondern als mit Fleischlichen, wie mit jungen Kindern
in Christo. \footnote{\textbf{3:1} Joh 16,12} \bibleverse{2} Milch habe
ich euch zu trinken gegeben, und nicht Speise; denn ihr konntet noch
nicht. Auch könnt ihr jetzt noch nicht, \footnote{\textbf{3:2} 1Petr 2,2}
\bibleverse{3} dieweil ihr noch fleischlich seid. Denn sintemal Eifer
und Zank und Zwietracht unter euch sind, seid ihr nicht fleischlich und
wandelt nach menschlicher Weise? \footnote{\textbf{3:3} 1Kor 1,10-11;
  1Kor 11,18} \bibleverse{4} Denn so einer sagt: Ich bin paulisch, der
andere aber: Ich bin apollisch, -- seid ihr nicht fleischlich?
\footnote{\textbf{3:4} 1Kor 1,12} \bibleverse{5} Wer ist nun Paulus? Wer
ist Apollos? Diener sind sie, durch welche ihr seid gläubig geworden,
und das, wie der Herr einem jeglichen gegeben hat. \bibleverse{6} Ich
habe gepflanzt, Apollos hat begossen; aber Gott hat das Gedeihen
gegeben. \footnote{\textbf{3:6} Apg 18,24-28} \bibleverse{7} So ist nun
weder der da pflanzt noch der da begießt, etwas, sondern Gott, der das
Gedeihen gibt. \bibleverse{8} Der aber pflanzt und der da begießt, ist
einer wie der andere. Ein jeglicher aber wird seinen Lohn empfangen nach
seiner Arbeit. \bibleverse{9} Denn wir sind Gottes Mitarbeiter; ihr seid
Gottes Ackerwerk und Gottes Bau. \footnote{\textbf{3:9} Mt 13,3-9; Eph
  2,20} \bibleverse{10} Ich nach Gottes Gnade, die mir gegeben ist, habe
den Grund gelegt als ein weiser Baumeister; ein anderer baut darauf. Ein
jeglicher aber sehe zu, wie er darauf baue. \bibleverse{11} Einen
anderen Grund kann niemand legen außer dem, der gelegt ist, welcher ist
Jesus Christus. \bibleverse{12} So aber jemand auf diesen Grund baut
Gold, Silber, edle Steine, Holz, Heu, Stoppeln, \bibleverse{13} so wird
eines jeglichen Werk offenbar werden: der Tag wird's klar machen. Denn
es wird durchs Feuer offenbar werden; und welcherlei eines jeglichen
Werk sei, wird das Feuer bewähren. \footnote{\textbf{3:13} 1Kor 4,5}
\bibleverse{14} Wird jemandes Werk bleiben, das er darauf gebaut hat, so
wird er Lohn empfangen. \bibleverse{15} Wird aber jemandes Werk
verbrennen, so wird er Schaden leiden; er selbst aber wird selig werden,
so doch wie durchs Feuer. \bibleverse{16} Wisset ihr nicht, dass ihr
Gottes Tempel seid und der Geist Gottes in euch wohnt? \bibleverse{17}
Wenn jemand den Tempel Gottes verderbt, den wird Gott verderben; denn
der Tempel Gottes ist heilig, -- der seid ihr. \bibleverse{18} Niemand
betrüge sich selbst. Welcher sich unter euch dünkt weise zu sein, der
werde ein Narr in dieser Welt, dass er möge weise sein. \footnote{\textbf{3:18}
  Offb 3,17-18} \bibleverse{19} Denn dieser Welt Weisheit ist Torheit
bei Gott. Denn es steht geschrieben: „Die Weisen erhascht er in ihrer
Klugheit.`` \bibleverse{20} Und abermals: „Der Herr weiß der Weisen
Gedanken, dass sie eitel sind.`` \bibleverse{21} Darum rühme sich
niemand eines Menschen. Es ist alles euer: \bibleverse{22} es sei Paulus
oder Apollos, es sei Kephas oder die Welt, es sei das Leben oder der
Tod, es sei das Gegenwärtige oder das Zukünftige, -- alles ist euer;
\bibleverse{23} ihr aber seid Christi, Christus aber ist Gottes.

\hypertarget{section-3}{%
\section{4}\label{section-3}}

\bibleverse{1} Dafür halte uns jedermann: für Christi Diener und
Haushalter über Gottes Geheimnisse. \footnote{\textbf{4:1} Tit 1,7}
\bibleverse{2} Nun sucht man nicht mehr an den Haushaltern, denn dass
sie treu erfunden werden. \footnote{\textbf{4:2} Lk 12,42}
\bibleverse{3} Mir aber ist's ein Geringes, dass ich von euch gerichtet
werde oder von einem menschlichen Tage; auch richte ich mich selbst
nicht. \bibleverse{4} Denn ich bin mir nichts bewusst, aber darin bin
ich nicht gerechtfertigt; der Herr ist's aber, der mich richtet.
\bibleverse{5} Darum richtet nicht vor der Zeit, bis der Herr komme,
welcher auch wird ans Licht bringen, was im Finstern verborgen ist, und
den Rat der Herzen offenbaren; alsdann wird einem jeglichen von Gott Lob
widerfahren. \footnote{\textbf{4:5} 1Kor 3,8} \bibleverse{6} Solches
aber, liebe Brüder, habe ich auf mich und Apollos gedeutet um
euretwillen, dass ihr an uns lernet, dass niemand höher von sich halte,
denn geschrieben ist, auf dass sich nicht einer wider den anderen um
jemandes willen aufblase. \footnote{\textbf{4:6} Röm 12,3}
\bibleverse{7} Denn wer hat dich vorgezogen? Was hast du aber, dass du
nicht empfangen hast? Wenn du es aber empfangen hast, was rühmst du dich
denn, als ob du es nicht empfangen hättest? \bibleverse{8} Ihr seid
schon satt geworden, ihr seid schon reich geworden, ihr herrschet ohne
uns; und wollte Gott, ihr herrschtet, auf dass auch wir mit euch
herrschen möchten! \footnote{\textbf{4:8} Offb 3,17; Offb 3,21}
\bibleverse{9} Ich halte aber dafür, Gott habe uns Apostel für die
Allergeringsten dargestellt, als dem Tode übergeben. Denn wir sind ein
Schauspiel geworden der Welt und den Engeln und den Menschen.
\footnote{\textbf{4:9} Röm 8,36; Hebr 10,33} \bibleverse{10} Wir sind
Narren um Christi willen, ihr aber seid klug in Christo; wir schwach,
ihr aber stark; ihr herrlich, wir aber verachtet. \footnote{\textbf{4:10}
  1Kor 3,18} \bibleverse{11} Bis auf diese Stunde leiden wir Hunger und
Durst und sind nackt und werden geschlagen und haben keine gewisse
Stätte \footnote{\textbf{4:11} 2Kor 11,23-27} \bibleverse{12} und
arbeiten und wirken mit unseren eigenen Händen. Man schilt uns, so
segnen wir; man verfolgt uns, so dulden wir's; man lästert uns, so
flehen wir; \footnote{\textbf{4:12} 1Kor 9,15; Apg 18,3; Mt 5,44; Röm
  12,14} \bibleverse{13} wir sind stets wie ein Fluch der Welt und ein
Fegopfer aller Leute. \bibleverse{14} Nicht schreibe ich solches, dass
ich euch beschäme; sondern ich vermahne euch als meine lieben Kinder.
\bibleverse{15} Denn ob ihr gleich ihr zehntausend Zuchtmeister hättet
in Christo, so habt ihr doch nicht viele Väter; denn ich habe euch
gezeugt in Christo Jesu durchs Evangelium. \bibleverse{16} Darum ermahne
ich euch: Seid meine Nachfolger! \footnote{\textbf{4:16} 1Kor 11,1}
\bibleverse{17} Aus derselben Ursache habe ich Timotheus zu euch
gesandt, welcher ist mein lieber und getreuer Sohn in dem Herrn, dass er
euch erinnere meiner Wege, die in Christo sind, gleichwie ich an allen
Enden in allen Gemeinden lehre. \footnote{\textbf{4:17} Apg 16,1-3}
\bibleverse{18} Es blähen sich etliche auf, als würde ich nicht zu euch
kommen. \bibleverse{19} Ich werde aber gar bald zu euch kommen, so der
Herr will, und kennen lernen nicht die Worte der Aufgeblasenen, sondern
die Kraft. \bibleverse{20} Denn das Reich Gottes steht nicht in Worten,
sondern in Kraft. \footnote{\textbf{4:20} 1Kor 2,4} \bibleverse{21} Was
wollt ihr? Soll ich mit der Rute zu euch kommen oder mit Liebe und
sanftmütigem Geist? \footnote{\textbf{4:21} 2Kor 10,2}

\hypertarget{section-4}{%
\section{5}\label{section-4}}

\bibleverse{1} Es geht eine gemeine Rede, dass Hurerei unter euch ist,
und eine solche Hurerei, davon auch die Heiden nicht zu sagen wissen:
dass einer seines Vaters Weib habe. \bibleverse{2} Und ihr seid
aufgeblasen und habt nicht vielmehr Leid getragen, auf dass, der das
Werk getan hat, von euch getan würde? \footnote{\textbf{5:2} 1Kor 4,6-8}
\bibleverse{3} Ich zwar, der ich mit dem Leibe nicht da bin, doch mit
dem Geist gegenwärtig, habe schon, als sei ich gegenwärtig, beschlossen
über den, der solches also getan hat: \footnote{\textbf{5:3} Kol 2,5}
\bibleverse{4} in dem Namen unseres Herrn Jesu Christi, in eurer
Versammlung mit meinem Geist und mit der Kraft unseres Herrn Jesu
Christi, \footnote{\textbf{5:4} Mt 16,19; Mt 18,18; 2Kor 13,10}
\bibleverse{5} ihn zu übergeben dem Satan zum Verderben des Fleisches,
auf dass der Geist selig werde am Tage des Herrn Jesu. \footnote{\textbf{5:5}
  1Tim 1,20} \bibleverse{6} Euer Ruhm ist nicht fein. Wisset ihr nicht,
dass ein wenig Sauerteig den ganzen Teig versäuert? \footnote{\textbf{5:6}
  Gal 5,9} \bibleverse{7} Darum feget den alten Sauerteig aus, auf dass
ihr ein neuer Teig seid, gleichwie ihr ungesäuert seid. Denn wir haben
auch ein Osterlamm, das ist Christus, für uns geopfert. \footnote{\textbf{5:7}
  2Mo 12,3-20; 2Mo 13,7; Jes 53,7; 1Petr 1,19} \bibleverse{8} Darum
lasset uns Ostern halten nicht im alten Sauerteig, auch nicht im
Sauerteig der Bosheit und Schalkheit, sondern in dem Süßteig der
Lauterkeit und der Wahrheit. \bibleverse{9} Ich habe euch geschrieben in
dem Briefe, dass ihr nichts sollt zu schaffen haben mit den Hurern.
\bibleverse{10} Das meine ich gar nicht von den Hurern in dieser Welt
oder von den Geizigen oder von den Räubern oder von den Abgöttischen;
sonst müsstet ihr die Welt räumen. \bibleverse{11} Nun aber habe ich
euch geschrieben, ihr sollt nichts mit ihnen zu schaffen haben, wenn
jemand sich lässt einen Bruder nennen, und ist ein Hurer oder ein
Geiziger oder ein Abgöttischer oder ein Lästerer oder ein Trunkenbold
oder ein Räuber; mit dem sollt ihr auch nicht essen. \footnote{\textbf{5:11}
  2Thes 3,6} \bibleverse{12} Denn was gehen mich die draußen an, dass
ich sie sollte richten? Richtet ihr nicht, die drinnen sind?
\bibleverse{13} Gott aber wird, die draußen sind, richten. Tut von euch
selbst hinaus, wer da böse ist.

\hypertarget{section-5}{%
\section{6}\label{section-5}}

\bibleverse{1} Wie darf jemand unter euch, wenn er einen Handel hat mit
einem anderen, hadern vor den Ungerechten und nicht vor den Heiligen?
\bibleverse{2} Wisset ihr nicht, dass die Heiligen die Welt richten
werden? So nun die Welt von euch soll gerichtet werden, seid ihr denn
nicht gut genug, geringe Sachen zu richten? \footnote{\textbf{6:2} Mt
  19,28} \bibleverse{3} Wisset ihr nicht, dass wir über die Engel
richten werden? Wie viel mehr über die zeitlichen Güter. \bibleverse{4}
Ihr aber, wenn ihr über zeitlichen Gütern Sachen habt, so nehmt ihr die,
die bei der Gemeinde verachtet sind, und setzet sie zu Richtern.
\bibleverse{5} Euch zur Schande muss ich das sagen: Ist so gar kein
Weiser unter euch, auch nicht einer, der da könnte richten zwischen
Bruder und Bruder? \bibleverse{6} sondern ein Bruder hadert mit dem
anderen, dazu vor den Ungläubigen. \bibleverse{7} Es ist schon ein Fehl
unter euch, dass ihr miteinander rechtet. Warum lasst ihr euch nicht
lieber Unrecht tun? warum lasst ihr euch nicht lieber übervorteilen?
\bibleverse{8} Sondern ihr tut Unrecht und übervorteilt, und solches an
den Brüdern! \bibleverse{9} Wisset ihr nicht, dass die Ungerechten das
Reich Gottes nicht ererben werden? Lasset euch nicht verführen! Weder
die Hurer noch die Abgöttischen noch die Ehebrecher noch die Weichlinge
noch die Knabenschänder \footnote{\textbf{6:9} 1Tim 1,9-11; Gal 5,19-21}
\bibleverse{10} noch die Diebe noch die Geizigen noch die Trunkenbolde
noch die Lästerer noch die Räuber werden das Reich Gottes ererben.
\bibleverse{11} Und solche sind euer etliche gewesen; aber ihr seid
abgewaschen, ihr seid geheiligt, ihr seid gerecht geworden durch den
Namen des Herrn Jesu und durch den Geist unseres Gottes. \bibleverse{12}
Ich habe es alles Macht; es frommt aber nicht alles. Ich habe es alles
Macht; es soll mich aber nichts gefangen nehmen. \footnote{\textbf{6:12}
  1Kor 10,23} \bibleverse{13} Die Speise dem Bauche und der Bauch der
Speise; aber Gott wird diesen und jene zunichte machen. Der Leib aber
nicht der Hurerei, sondern dem Herrn, und der Herr dem Leibe.
\footnote{\textbf{6:13} 1Kor 1,3-5} \bibleverse{14} Gott aber hat den
Herrn auferweckt und wird uns auch auferwecken durch seine Kraft.
\footnote{\textbf{6:14} 1Kor 15,20; 2Kor 4,14} \bibleverse{15} Wisset
ihr nicht, dass eure Leiber Christi Glieder sind? Sollte ich nun die
Glieder Christi nehmen und Hurenglieder daraus machen? Das sei ferne!
\bibleverse{16} Oder wisset ihr nicht, dass, wer an der Hure hangt, der
ist ein Leib mit ihr? Denn „es werden``, spricht er, „die zwei ein
Fleisch sein.`` \bibleverse{17} Wer aber dem Herrn anhangt, der ist ein
Geist mit ihm. \bibleverse{18} Fliehet die Hurerei! Alle Sünden, die der
Mensch tut, sind außer seinem Leibe; wer aber hurt, der sündigt an
seinem eigenen Leibe. \bibleverse{19} Oder wisset ihr nicht, dass euer
Leib ein Tempel des heiligen Geistes ist, der in euch ist, welchen ihr
habt von Gott, und seid nicht euer selbst. \footnote{\textbf{6:19} 1Kor
  3,16} \bibleverse{20} Denn ihr seid teuer erkauft; darum so preiset
Gott an eurem Leibe und in eurem Geiste, welche sind Gottes. \footnote{\textbf{6:20}
  1Kor 7,23; 1Petr 1,18-19; Phil 1,20}

\hypertarget{section-6}{%
\section{7}\label{section-6}}

\bibleverse{1} Wovon ihr aber mir geschrieben habt, darauf antworte ich:
Es ist dem Menschen gut, dass er kein Weib berühre. \bibleverse{2} Aber
um der Hurerei willen habe ein jeglicher sein eigen Weib, und eine
jegliche habe ihren eigenen Mann. \bibleverse{3} Der Mann leiste dem
Weib die schuldige Freundschaft, desgleichen das Weib dem Manne.
\bibleverse{4} Das Weib ist ihres Leibes nicht mächtig, sondern der
Mann. Desgleichen der Mann ist seines Leibes nicht mächtig, sondern das
Weib. \bibleverse{5} Entziehe sich nicht eins dem anderen, es sei denn
aus beider Bewilligung eine Zeitlang, dass ihr zum Fasten und Beten Muße
habt; und kommt wiederum zusammen, auf dass euch der Satan nicht
versuche um eurer Unkeuschheit willen. \bibleverse{6} Solches sage ich
aber aus Vergunst und nicht aus Gebot. \bibleverse{7} Ich wollte aber
lieber, alle Menschen wären, wie ich bin; aber ein jeglicher hat seine
eigene Gabe von Gott, einer so, der andere so. \footnote{\textbf{7:7} Mt
  19,22} \bibleverse{8} Ich sage zwar den Ledigen und Witwen: Es ist
ihnen gut, wenn sie auch bleiben wie ich. \bibleverse{9} So sie aber
sich nicht mögen enthalten, so lass sie freien; es ist besser freien
denn Brunst leiden. \bibleverse{10} Den Ehelichen aber gebiete nicht
ich, sondern der Herr, dass sich das Weib nicht scheide von dem Manne --
\footnote{\textbf{7:10} Mt 5,32} \bibleverse{11} wenn sie sich aber
scheidet, dass sie ohne Ehe bleibe oder sich mit dem Manne versöhne --
und dass der Mann das Weib nicht von sich lasse. \bibleverse{12} Den
anderen aber sage ich, nicht der Herr: So ein Bruder ein ungläubiges
Weib hat, und sie lässt es sich gefallen, bei ihm zu wohnen, der scheide
sich nicht von ihr. \bibleverse{13} Und so ein Weib einen ungläubigen
Mann hat, und er lässt es sich gefallen, bei ihr zu wohnen, die scheide
sich nicht von ihm. \bibleverse{14} Denn der ungläubige Mann ist
geheiligt durchs Weib, und das ungläubige Weib ist geheiligt durch den
Mann. Sonst wären eure Kinder unrein; nun aber sind sie heilig.
\bibleverse{15} So aber der Ungläubige sich scheidet, so lass ihn sich
scheiden. Es ist der Bruder oder die Schwester nicht gefangen in solchen
Fällen. Im Frieden aber hat uns Gott berufen. \footnote{\textbf{7:15}
  Röm 14,19} \bibleverse{16} Denn was weißt du, Weib, ob du den Mann
werdest selig machen? Oder du, Mann, was weißt du, ob du das Weib
werdest selig machen? \footnote{\textbf{7:16} 1Petr 3,1} \bibleverse{17}
Doch wie einem jeglichen Gott hat ausgeteilt, wie einen jeglichen der
Herr berufen hat, also wandle er. Und also schaffe ich's in allen
Gemeinden. \bibleverse{18} Ist jemand beschnitten berufen, der halte an
der Beschneidung. Ist jemand unbeschnitten berufen, der lasse sich nicht
beschneiden. \bibleverse{19} Beschnitten sein ist nichts, und
unbeschnitten sein ist nichts, sondern Gottes Gebote halten. \footnote{\textbf{7:19}
  Gal 5,6; Gal 6,15} \bibleverse{20} Ein jeglicher bleibe in dem Beruf,
darin er berufen ist. \bibleverse{21} Bist du als Knecht berufen, sorge
dich nicht; doch, kannst du frei werden, so brauche es viel lieber.
\bibleverse{22} Denn wer als Knecht berufen ist in dem Herrn, der ist
ein Freigelassener des Herrn; desgleichen, wer als Freier berufen ist,
der ist ein Knecht Christi. \bibleverse{23} Ihr seid teuer erkauft;
werdet nicht der Menschen Knechte. \footnote{\textbf{7:23} 1Kor 6,20}
\bibleverse{24} Ein jeglicher, liebe Brüder, worin er berufen ist, darin
bleibe er bei Gott. \bibleverse{25} Von den Jungfrauen aber habe ich
kein Gebot des Herrn; ich sage aber meine Meinung, als der ich
Barmherzigkeit erlangt habe vom Herrn, treu zu sein. \bibleverse{26} So
meine ich nun, solches sei gut um der gegenwärtigen Not willen, es sei
dem Menschen gut, also zu sein. \bibleverse{27} Bist du an ein Weib
gebunden, so suche nicht los zu werden; bist du los vom Weibe, so suche
kein Weib. \bibleverse{28} Wenn du aber freist, sündigst du nicht; und
so eine Jungfrau freit, sündigt sie nicht. Doch werden solche leibliche
Trübsal haben; ich verschonte euch aber gern. \bibleverse{29} Das sage
ich aber, liebe Brüder: Die Zeit ist kurz. Weiter ist das die Meinung:
Die da Weiber haben, dass sie seien, als hätten sie keine; und die da
weinen, als weinten sie nicht; \footnote{\textbf{7:29} Röm 13,11; Lk
  14,26} \bibleverse{30} und die sich freuen, als freuten sie sich
nicht; und die da kaufen, als besäßen sie es nicht; \bibleverse{31} und
die diese Welt gebrauchen, dass sie dieselbe nicht missbrauchen. Denn
das Wesen dieser Welt vergeht. \bibleverse{32} Ich wollte aber, dass ihr
ohne Sorge wäret. Wer ledig ist, der sorgt, was dem Herrn angehört, wie
er dem Herrn gefalle; \bibleverse{33} wer aber freit, der sorgt, was der
Welt angehört, wie er dem Weibe gefalle. Es ist ein Unterschied zwischen
einem Weibe und einer Jungfrau: \footnote{\textbf{7:33} Lk 14,20}
\bibleverse{34} welche nicht freit, die sorgt, was dem Herrn angehört,
dass sie heilig sei am Leib und auch am Geist; die aber freit, die
sorgt, was der Welt angehört, wie sie dem Manne gefalle. \bibleverse{35}
Solches aber sage ich zu eurem Nutzen; nicht, dass ich euch einen Strick
um den Hals werfe, sondern dazu, dass es fein zugehe und ihr stets
unverhindert dem Herrn dienen könnet. \bibleverse{36} So aber jemand
sich lässt dünken, es wolle sich nicht schicken mit seiner Jungfrau,
weil sie eben wohl mannbar ist, und es will nicht anders sein, so tue
er, was er will; er sündigt nicht, er lasse sie freien. \bibleverse{37}
Wenn einer aber sich fest vornimmt, weil er ungezwungen ist und seinen
freien Willen hat, und beschließt solches in seinem Herzen, seine
Jungfrau also bleiben zu lassen, der tut wohl. \bibleverse{38} Demnach,
welcher verheiratet, der tut wohl; welcher aber nicht verheiratet, der
tut besser. \bibleverse{39} Ein Weib ist gebunden durch das Gesetz,
solange ihr Mann lebt; so aber ihr Mann entschläft, ist sie frei, zu
heiraten, wen sie will, nur, dass es in dem Herrn geschehe.
\bibleverse{40} Seliger ist sie aber, wo sie also bleibt, nach meiner
Meinung. Ich halte aber dafür, ich habe auch den Geist Gottes.

\hypertarget{section-7}{%
\section{8}\label{section-7}}

\bibleverse{1} „Von dem Götzenopfer aber wissen wir; denn wir haben alle
das Wissen.`` -- Das Wissen bläst auf, aber die Liebe bessert.
\footnote{\textbf{8:1} Apg 15,29} \bibleverse{2} So aber jemand sich
dünken lässt, er wisse etwas, der weiß noch nichts, wie er wissen soll.
\footnote{\textbf{8:2} Gal 6,3} \bibleverse{3} So aber jemand Gott
liebt, der ist von ihm erkannt. \footnote{\textbf{8:3} Gal 4,9; 1Kor
  13,12} \bibleverse{4} So wissen wir nun von der Speise des
Götzenopfers, dass ein Götze nichts in der Welt sei und dass kein
anderer Gott sei als der eine. \footnote{\textbf{8:4} 5Mo 6,4}
\bibleverse{5} Und wiewohl welche sind, die Götter genannt werden, es
sei im Himmel oder auf Erden (sintemal es sind viele Götter und viele
Herren), \footnote{\textbf{8:5} 1Kor 10,19-20; Ps 136,2-3; Röm 8,38-39}
\bibleverse{6} so haben wir doch nur einen Gott, den Vater, von welchem
alle Dinge sind und wir zu ihm; und einen Herrn, Jesus Christus, durch
welchen alle Dinge sind und wir durch ihn. \footnote{\textbf{8:6} 1Kor
  12,5-6; Kol 1,16; Eph 4,5-6; Mal 2,10; Joh 1,3} \bibleverse{7} Es hat
aber nicht jedermann das Wissen. Denn etliche machen sich noch ein
Gewissen über dem Götzen und essen's für Götzenopfer; damit wird ihr
Gewissen, weil es so schwach ist, befleckt. \footnote{\textbf{8:7} 1Kor
  10,28} \bibleverse{8} Aber die Speise fördert uns vor Gott nicht:
essen wir, so werden wir darum nicht besser sein; essen wir nicht, so
werden wir darum nichts weniger sein. \footnote{\textbf{8:8} Röm 14,17}
\bibleverse{9} Sehet aber zu, dass diese eure Freiheit nicht gerate zu
einem Anstoß der Schwachen! \footnote{\textbf{8:9} Gal 5,13}
\bibleverse{10} Denn wenn dich, der du die Erkenntnis hast, jemand sähe
zu Tische sitzen im Götzenhause, wird nicht sein Gewissen, obwohl er
schwach ist, ermutigt, das Götzenopfer zu essen? \bibleverse{11} Und
also wird über deiner Erkenntnis der schwache Bruder umkommen, um des
willen doch Christus gestorben ist. \bibleverse{12} Wenn ihr aber also
sündigt an den Brüdern und schlagt ihr schwaches Gewissen, so sündigt
ihr an Christo. \bibleverse{13} Darum, wenn die Speise meinen Bruder
ärgert, wollte ich nimmermehr Fleisch essen, auf dass ich meinen Bruder
nicht ärgere. \footnote{\textbf{8:13} Röm 14,21}

\hypertarget{section-8}{%
\section{9}\label{section-8}}

\bibleverse{1} Bin ich nicht ein Apostel? Bin ich nicht frei? Habe ich
nicht unseren Herrn Jesus Christus gesehen? Seid nicht ihr mein Werk in
dem Herrn? -- \footnote{\textbf{9:1} 1Kor 15,8; Apg 9,3-5; Apg 9,15}
\bibleverse{2} Bin ich anderen nicht ein Apostel, so bin ich doch euer
Apostel; denn das Siegel meines Apostelamts seid ihr in dem Herrn.
\footnote{\textbf{9:2} 1Kor 4,15; 2Kor 3,2-3} \bibleverse{3} Also
antworte ich, wenn man mich fragt. \bibleverse{4} Haben wir nicht Macht,
zu essen und zu trinken? \bibleverse{5} Haben wir nicht auch Macht, eine
Schwester zum Weibe mit umherzuführen wie die anderen Apostel und des
Herrn Brüder und Kephas? \footnote{\textbf{9:5} Joh 1,42; Mt 8,14}
\bibleverse{6} Oder haben allein ich und Barnabas keine Macht, nicht zu
arbeiten? \footnote{\textbf{9:6} Apg 4,36; 2Thes 3,7-9} \bibleverse{7}
Wer zieht jemals in den Krieg auf seinen eigenen Sold? Wer pflanzt einen
Weinberg, und isst nicht von seiner Frucht? Oder wer weidet eine Herde,
und nährt sich nicht von der Milch der Herde? \bibleverse{8} Rede ich
aber solches auf Menschenweise? Sagt nicht solches das Gesetz auch?
\bibleverse{9} Denn im Gesetz Moses steht geschrieben: „Du sollst dem
Ochsen nicht das Maul verbinden, der da drischt.`` Sorgt Gott für die
Ochsen? \footnote{\textbf{9:9} 1Tim 5,18} \bibleverse{10} Oder sagt er's
nicht vielmehr um unseretwillen? Denn es ist ja um unseretwillen
geschrieben. Denn der da pflügt, soll auf Hoffnung pflügen; und der da
drischt, soll auf Hoffnung dreschen, dass er seiner Hoffnung teilhaftig
werde. \bibleverse{11} So wir euch das Geistliche säen, ist's ein großes
Ding, wenn wir euer Leibliches ernten? \bibleverse{12} So andere dieser
Macht an euch teilhaftig sind, warum nicht viel mehr wir? Aber wir haben
solche Macht nicht gebraucht, sondern ertragen allerlei, dass wir nicht
dem Evangelium Christi ein Hindernis machen. \footnote{\textbf{9:12} Apg
  20,33-35; 2Kor 11,9} \bibleverse{13} Wisset ihr nicht, dass, die da
opfern, essen vom Opfer, und die am Altar dienen, vom Altar Genuss
haben? \footnote{\textbf{9:13} 4Mo 18,18-19; 4Mo 18,31; 5Mo 18,1-3}
\bibleverse{14} Also hat auch der Herr befohlen, dass, die das
Evangelium verkündigen, sollen sich vom Evangelium nähren. \footnote{\textbf{9:14}
  Gal 6,6; Lk 10,7} \bibleverse{15} Ich aber habe der keines gebraucht.
Ich schreibe auch nicht darum davon, dass es mit mir also sollte
gehalten werden. Es wäre mir lieber, ich stürbe, denn dass mir jemand
meinen Ruhm sollte zunichte machen. \footnote{\textbf{9:15} Apg 18,3}
\bibleverse{16} Denn dass ich das Evangelium predige, darf ich mich
nicht rühmen; denn ich muss es tun. Und wehe mir, wenn ich das
Evangelium nicht predigte! \footnote{\textbf{9:16} Jer 20,9}
\bibleverse{17} Tue ich's gern, so wird mir gelohnt; tu ich's aber
ungern, so ist mir das Amt doch befohlen. \footnote{\textbf{9:17} 1Kor
  4,1} \bibleverse{18} Was ist denn nun mein Lohn? Dass ich predige das
Evangelium Christi und tue das frei umsonst, auf dass ich nicht meine
Freiheit missbrauche am Evangelium. \bibleverse{19} Denn wiewohl ich
frei bin von jedermann, habe ich doch mich selbst jedermann zum Knechte
gemacht, auf dass ich ihrer viele gewinne. \footnote{\textbf{9:19} Mt
  20,27; Röm 15,2} \bibleverse{20} Den Juden bin ich geworden wie ein
Jude, auf dass ich die Juden gewinne. Denen, die unter dem Gesetz sind,
bin ich geworden wie unter dem Gesetz, auf dass ich die, die unter dem
Gesetz sind, gewinne. \footnote{\textbf{9:20} 1Kor 10,33; Apg 16,3; Apg
  21,20-26} \bibleverse{21} Denen, die ohne Gesetz sind, bin ich wie
ohne Gesetz geworden (obwohl ich doch nicht ohne Gesetz bin vor Gott,
sondern bin in dem Gesetz Christi), auf dass ich die, die ohne Gesetz
sind, gewinne. \footnote{\textbf{9:21} Gal 2,3} \bibleverse{22} Den
Schwachen bin ich geworden wie ein Schwacher, auf dass ich die Schwachen
gewinne. Ich bin jedermann allerlei geworden, auf dass ich allenthalben
ja etliche selig mache. \footnote{\textbf{9:22} Röm 11,14}
\bibleverse{23} Solches aber tue ich um des Evangeliums willen, auf dass
ich sein teilhaftig werde. \bibleverse{24} Wisset ihr nicht, dass die,
die in den Schranken laufen, die laufen alle, aber einer erlangt das
Kleinod? Laufet nun also, dass ihr es ergreifet! \footnote{\textbf{9:24}
  Phil 3,14; 2Tim 4,7} \bibleverse{25} Ein jeglicher aber, der da
kämpft, enthält sich alles Dinges; jene also, dass sie eine vergängliche
Krone empfangen, wir aber eine unvergängliche. \footnote{\textbf{9:25}
  2Tim 2,4-5; 1Petr 5,4} \bibleverse{26} Ich laufe aber also, nicht als
aufs Ungewisse; ich fechte also, nicht als der in die Luft streicht;
\bibleverse{27} sondern ich betäube meinen Leib und zähme ihn, dass ich
nicht den anderen predige, und selbst verwerflich werde. \footnote{\textbf{9:27}
  Röm 13,14}

\hypertarget{section-9}{%
\section{10}\label{section-9}}

\bibleverse{1} Ich will euch aber, liebe Brüder, nicht verhalten, dass
unsere Väter sind alle unter der Wolke gewesen und sind alle durchs Meer
gegangen \footnote{\textbf{10:1} 2Mo 13,21; 2Mo 14,22} \bibleverse{2}
und sind alle auf Mose getauft mit der Wolke und mit dem Meer
\footnote{\textbf{10:2} 2Mo 16,4; 2Mo 16,35; 5Mo 8,3} \bibleverse{3} und
haben alle einerlei geistliche Speise gegessen \bibleverse{4} und haben
alle einerlei geistlichen Trank getrunken; sie tranken aber von dem
geistlichen Fels, der mitfolgte, welcher war Christus. \bibleverse{5}
Aber an ihrer vielen hatte Gott kein Wohlgefallen; denn sie wurden
niedergeschlagen in der Wüste. \footnote{\textbf{10:5} 4Mo 14,22-32}
\bibleverse{6} Das ist aber uns zum Vorbilde geschehen, dass wir nicht
uns gelüsten lassen des Bösen, gleichwie jene gelüstet hat. \footnote{\textbf{10:6}
  4Mo 11,4} \bibleverse{7} Werdet auch nicht Abgöttische, gleichwie
jener etliche wurden, wie geschrieben steht: „Das Volk setzte sich
nieder, zu essen und zu trinken, und stand auf, zu spielen.``
\bibleverse{8} Auch lasset uns nicht Hurerei treiben, wie etliche unter
jenen Hurerei trieben, und fielen auf einen Tag dreiundzwanzigtausend.
\footnote{\textbf{10:8} 4Mo 25,1; 4Mo 25,9} \bibleverse{9} Lasset uns
aber auch Christum nicht versuchen, wie etliche von jenen ihn versuchten
und wurden von den Schlangen umgebracht. \footnote{\textbf{10:9} 4Mo
  21,4-6} \bibleverse{10} Murret auch nicht, gleichwie jener etliche
murrten und wurden umgebracht durch den Verderber. \footnote{\textbf{10:10}
  4Mo 14,2; 4Mo 14,35-36; Hebr 3,11; Hebr 3,17} \bibleverse{11} Solches
alles widerfuhr jenen zum Vorbilde; es ist aber geschrieben uns zur
Warnung, auf welche das Ende der Welt gekommen ist. \footnote{\textbf{10:11}
  1Petr 4,7} \bibleverse{12} Darum, wer sich lässt dünken, er stehe, mag
wohl zusehen, dass er nicht falle. \bibleverse{13} Es hat euch noch
keine denn menschliche Versuchung betreten; aber Gott ist getreu, der
euch nicht lässt versuchen über euer Vermögen, sondern macht, dass die
Versuchung so ein Ende gewinne, dass ihr's könnet ertragen. \footnote{\textbf{10:13}
  2Petr 2,9} \bibleverse{14} Darum, meine Liebsten, fliehet von dem
Götzendienst! \footnote{\textbf{10:14} 1Jo 5,21} \bibleverse{15} Als mit
den Klugen rede ich; richtet ihr, was ich sage. \bibleverse{16} Der
gesegnete Kelch, welchen wir segnen, ist der nicht die Gemeinschaft des
Blutes Christi? Das Brot, das wir brechen, ist das nicht die
Gemeinschaft des Leibes Christi? \footnote{\textbf{10:16} 1Kor 11,23-26;
  Mt 26,27; Apg 2,42} \bibleverse{17} Denn ein Brot ist's, so sind wir
viele ein Leib, dieweil wir alle eines Brotes teilhaftig sind.
\footnote{\textbf{10:17} 1Kor 12,27; Röm 12,5} \bibleverse{18} Sehet an
das Israel nach dem Fleisch! Welche die Opfer essen, sind die nicht in
der Gemeinschaft des Altars? \footnote{\textbf{10:18} 3Mo 7,6}
\bibleverse{19} Was soll ich denn nun sagen? Soll ich sagen, dass der
Götze etwas sei oder dass das Götzenopfer etwas sei? \footnote{\textbf{10:19}
  1Kor 8,4} \bibleverse{20} Aber ich sage: Was die Heiden opfern, das
opfern sie den Teufeln, und nicht Gott. Nun will ich nicht, dass ihr in
der Teufel Gemeinschaft sein sollt. \bibleverse{21} Ihr könnt nicht
zugleich trinken des Herrn Kelch und der Teufel Kelch; ihr könnt nicht
zugleich teilhaftig sein des Tisches des Herrn und des Tisches der
Teufel. \footnote{\textbf{10:21} Mt 6,24; 2Kor 6,15-16} \bibleverse{22}
Oder wollen wir dem Herrn trotzen? Sind wir stärker denn er?
\bibleverse{23} Ich habe es zwar alles Macht; aber es frommt nicht
alles. Ich habe es alles Macht; aber es bessert nicht alles.
\bibleverse{24} Niemand suche das Seine, sondern ein jeglicher, was des
anderen ist. \footnote{\textbf{10:24} Röm 15,2; Phil 2,4}
\bibleverse{25} Alles, was feil ist auf dem Fleischmarkt, das esset, und
forschet nicht, auf dass ihr das Gewissen verschonet. \footnote{\textbf{10:25}
  Röm 14,2-10; Röm 14,22} \bibleverse{26} Denn „die Erde ist des Herrn
und was darinnen ist.`` \bibleverse{27} So aber jemand von den
Ungläubigen euch ladet und ihr wollt hingehen, so esset alles, was euch
vorgetragen wird, und forschet nicht, auf dass ihr das Gewissen
verschonet. \bibleverse{28} Wo aber jemand würde zu euch sagen: „Das ist
Götzenopfer``, so esset nicht, um des willen, der es anzeigte, auf dass
ihr das Gewissen verschonet. \footnote{\textbf{10:28} 1Kor 8,7}
\bibleverse{29} Ich sage aber vom Gewissen, nicht deiner selbst, sondern
des anderen. Denn warum sollte ich meine Freiheit lassen richten von
eines anderen Gewissen? \bibleverse{30} So ich's mit Danksagung genieße,
was sollte ich denn verlästert werden über dem, dafür ich danke?
\bibleverse{31} Ihr esset nun oder trinket oder was ihr tut, so tut es
alles zu Gottes Ehre. \footnote{\textbf{10:31} Kol 3,17} \bibleverse{32}
Gebet kein Ärgernis weder den Juden noch den Griechen noch der Gemeinde
Gottes; \footnote{\textbf{10:32} Röm 14,13} \bibleverse{33} gleichwie
ich auch jedermann in allerlei mich gefällig mache und suche nicht, was
mir, sondern was vielen frommt, dass sie selig werden. \footnote{\textbf{10:33}
  1Kor 9,20-22}

\hypertarget{section-10}{%
\section{11}\label{section-10}}

\bibleverse{1} Seid meine Nachfolger, gleichwie ich Christi!
\bibleverse{2} Ich lobe euch, liebe Brüder, dass ihr an mich gedenket in
allen Stücken und haltet die Weise, wie ich sie euch gegeben habe.
\bibleverse{3} Ich lasse euch aber wissen, dass Christus ist eines
jeglichen Mannes Haupt; der Mann aber ist des Weibes Haupt; Gott aber
ist Christi Haupt. \bibleverse{4} Ein jeglicher Mann, der da betet oder
weissagt und hat etwas auf dem Haupt, der schändet sein Haupt.
\bibleverse{5} Ein Weib aber, das da betet oder weissagt mit unbedecktem
Haupt, die schändet ihr Haupt; denn es ist ebensoviel, als wäre sie
geschoren. \bibleverse{6} Will sie sich nicht bedecken, so schneide man
ihr auch das Haar ab. Nun es aber übel steht, dass ein Weib
verschnittenes Haar habe und geschoren sei, so lasset sie das Haupt
bedecken. \bibleverse{7} Der Mann aber soll das Haupt nicht bedecken,
sintemal er ist Gottes Bild und Ehre; das Weib aber ist des Mannes Ehre.
\bibleverse{8} Denn der Mann ist nicht vom Weibe, sondern das Weib vom
Manne. \footnote{\textbf{11:8} 1Mo 2,21-23} \bibleverse{9} Und der Mann
ist nicht geschaffen um des Weibes willen, sondern das Weib um des
Mannes willen. \footnote{\textbf{11:9} 1Mo 2,18} \bibleverse{10} Darum
soll das Weib eine Macht auf dem Haupt haben, um der Engel willen.
\bibleverse{11} Doch ist weder der Mann ohne das Weib, noch das Weib
ohne den Mann in dem Herrn; \bibleverse{12} denn wie das Weib vom Manne,
also kommt auch der Mann durchs Weib; aber alles von Gott.
\bibleverse{13} Richtet bei euch selbst, ob's wohl steht, dass ein Weib
unbedeckt vor Gott bete. \bibleverse{14} Oder lehrt euch auch nicht die
Natur, dass es einem Manne eine Unehre ist, wenn er das Haar lang
wachsen lässt, \bibleverse{15} und dem Weibe eine Ehre, wenn sie langes
Haar hat? Das Haar ist ihr zur Decke gegeben. \bibleverse{16} Ist aber
jemand unter euch, der Lust zu zanken hat, der wisse, dass wir solche
Weise nicht haben, die Gemeinden Gottes auch nicht. \bibleverse{17} Ich
muss aber dies befehlen: Ich kann's nicht loben, dass ihr nicht auf
bessere Weise, sondern auf ärgere Weise zusammenkommt. \bibleverse{18}
Zum ersten, wenn ihr zusammenkommt in der Gemeinde, höre ich, es seien
Spaltungen unter euch; und zum Teil glaube ich's. \footnote{\textbf{11:18}
  1Kor 1,12; 1Kor 3,3} \bibleverse{19} Denn es müssen Parteien unter
euch sein, auf dass die, die rechtschaffen sind, offenbar unter euch
werden. \footnote{\textbf{11:19} Mt 18,7; 1Jo 2,19} \bibleverse{20} Wenn
ihr nun zusammenkommt, so hält man da nicht des Herrn Abendmahl.
\bibleverse{21} Denn wenn man das Abendmahl halten soll, nimmt ein
jeglicher sein eigenes vorhin, und einer ist hungrig, der andere ist
trunken. \footnote{\textbf{11:21} Jud 1,12} \bibleverse{22} Habt ihr
aber nicht Häuser, da ihr essen und trinken könnt? Oder verachtet ihr
die Gemeinde Gottes und beschämet die, die da nichts haben? Was soll ich
euch sagen? Soll ich euch loben? Hierin lobe ich euch nicht. \footnote{\textbf{11:22}
  Jak 2,5; Jak 1,2-6} \bibleverse{23} Ich habe es von dem Herrn
empfangen, das ich euch gegeben habe. Denn der Herr Jesus in der Nacht,
da er verraten ward, nahm das Brot, \footnote{\textbf{11:23} Mt
  26,26-28; Mk 14,22-24; Lk 22,19-20} \bibleverse{24} dankte und brach's
und sprach: Nehmet, esset, das ist mein Leib, der für euch gebrochen
wird; solches tut zu meinem Gedächtnis. \bibleverse{25} Desgleichen auch
den Kelch nach dem Abendmahl und sprach: Dieser Kelch ist das neue
Testament in meinem Blut; solches tut, so oft ihr's trinket, zu meinem
Gedächtnis. \bibleverse{26} Denn so oft ihr von diesem Brot esset und
von diesem Kelch trinket, sollt ihr des Herrn Tod verkündigen, bis dass
er kommt. \bibleverse{27} Welcher nun unwürdig von diesem Brot isset
oder von dem Kelch des Herrn trinket, der ist schuldig an dem Leib und
Blut des Herrn. \footnote{\textbf{11:27} 1Kor 11,21-22} \bibleverse{28}
Der Mensch prüfe aber sich selbst, und also esse er von diesem Brot und
trinke von diesem Kelch. \footnote{\textbf{11:28} Mt 26,22}
\bibleverse{29} Denn welcher unwürdig isset und trinket, der isset und
trinket sich selber zum Gericht, damit, dass er nicht unterscheidet den
Leib des Herrn. \footnote{\textbf{11:29} 1Kor 10,16-17} \bibleverse{30}
Darum sind auch viele Schwache und Kranke unter euch, und ein gut Teil
schlafen. \bibleverse{31} Denn so wir uns selber richteten, so würden
wir nicht gerichtet. \bibleverse{32} Wenn wir aber gerichtet werden, so
werden wir von dem Herrn gezüchtigt, auf dass wir nicht samt der Welt
verdammt werden. \bibleverse{33} Darum, meine lieben Brüder, wenn ihr
zusammenkommt, zu essen, so harre einer des anderen. \bibleverse{34}
Hungert aber jemand, der esse daheim, auf dass ihr nicht euch zum
Gericht zusammenkommt. -- Das andere will ich ordnen, wenn ich komme.

\hypertarget{section-11}{%
\section{12}\label{section-11}}

\bibleverse{1} Von den geistlichen Gaben aber will ich euch, liebe
Brüder, nicht verhalten. \bibleverse{2} Ihr wisset, dass ihr Heiden seid
gewesen und hingegangen zu den stummen Götzen, wie ihr geführt wurdet.
\footnote{\textbf{12:2} Hab 2,18-19} \bibleverse{3} Darum tue ich euch
kund, dass niemand Jesum verflucht, der durch den Geist Gottes redet;
und niemand kann Jesum einen Herrn heißen außer durch den heiligen
Geist. \footnote{\textbf{12:3} Mk 9,39; 1Jo 4,2; 1Jo 1,4-3}
\bibleverse{4} Es sind mancherlei Gaben; aber es ist ein Geist.
\footnote{\textbf{12:4} Röm 12,6; Eph 4,4-999} \bibleverse{5} Und es
sind mancherlei Ämter; aber es ist ein Herr. \footnote{\textbf{12:5}
  1Kor 12,28} \bibleverse{6} Und es sind mancherlei Kräfte; aber es ist
ein Gott, der da wirket alles in allen. \bibleverse{7} In einem
jeglichen erzeigen sich die Gaben des Geistes zum gemeinen Nutzen.
\footnote{\textbf{12:7} 1Kor 14,26} \bibleverse{8} Einem wird gegeben
durch den Geist, zu reden von der Weisheit; dem anderen wird gegeben, zu
reden von der Erkenntnis nach demselben Geist; \bibleverse{9} einem
anderen der Glaube in demselben Geist; einem anderen die Gabe, gesund zu
machen in demselben Geist; \bibleverse{10} einem anderen, Wunder zu tun;
einem anderen Weissagung; einem anderen, Geister zu unterscheiden; einem
anderen mancherlei Sprachen; einem anderen, die Sprachen auszulegen.
\bibleverse{11} Dies aber alles wirkt derselbe eine Geist und teilt
einem jeglichen seines zu, nach dem er will. \footnote{\textbf{12:11}
  Röm 12,3; Eph 4,7} \bibleverse{12} Denn gleichwie ein Leib ist, und
hat doch viele Glieder, alle Glieder aber des Leibes, wiewohl ihrer viel
sind, doch ein Leib sind: also auch Christus. \bibleverse{13} Denn wir
sind durch einen Geist alle zu einem Leibe getauft, wir seien Juden oder
Griechen, Knechte oder Freie, und sind alle zu einem Geist getränkt.
\bibleverse{14} Denn auch der Leib ist nicht ein Glied, sondern viele.
\bibleverse{15} Wenn aber der Fuß spräche: Ich bin keine Hand, darum bin
ich des Leibes Glied nicht, -- sollte er um deswillen nicht des Leibes
Glied sein? \bibleverse{16} Und wenn das Ohr spräche: Ich bin kein Auge,
darum bin ich nicht des Leibes Glied, -- sollte es um deswillen nicht
des Leibes Glied sein? \bibleverse{17} Wenn der ganze Leib Auge wäre, wo
bliebe das Gehör? Wenn er ganz Gehör wäre, wo bliebe der Geruch?
\bibleverse{18} Nun aber hat Gott die Glieder gesetzt, ein jegliches
sonderlich am Leibe, wie er gewollt hat. \bibleverse{19} Wenn aber alle
Glieder ein Glied wären, wo bliebe der Leib? \bibleverse{20} Nun aber
sind der Glieder viele; aber der Leib ist einer. \bibleverse{21} Es kann
das Auge nicht sagen zur Hand: Ich bedarf dein nicht; oder wiederum das
Haupt zu den Füßen: Ich bedarf euer nicht. \bibleverse{22} Sondern
vielmehr die Glieder des Leibes, die uns dünken die schwächsten zu sein,
sind die nötigsten; \bibleverse{23} und die uns dünken am wenigsten
ehrbar zu sein, denen legen wir am meisten Ehre an; und die uns übel
anstehen, die schmückt man am meisten. \bibleverse{24} Denn die uns wohl
anstehen, die bedürfen's nicht. Aber Gott hat den Leib also vermengt und
dem dürftigen Glied am meisten Ehre gegeben, \bibleverse{25} auf dass
nicht eine Spaltung im Leibe sei, sondern die Glieder füreinander gleich
sorgen. \bibleverse{26} Und so ein Glied leidet, so leiden alle Glieder
mit; und so ein Glied wird herrlich gehalten, so freuen sich alle
Glieder mit. \bibleverse{27} Ihr seid aber der Leib Christi und Glieder,
ein jeglicher nach seinem Teil. \footnote{\textbf{12:27} Röm 12,5}
\bibleverse{28} Und Gott hat gesetzt in der Gemeinde aufs erste die
Apostel, aufs andere die Propheten, aufs dritte die Lehrer, darnach die
Wundertäter, darnach die Gaben, gesund zu machen, Helfer, Regierer,
mancherlei Sprachen. \footnote{\textbf{12:28} Eph 4,11-12}
\bibleverse{29} Sind sie alle Apostel? Sind sie alle Propheten? Sind sie
alle Lehrer? Sind sie alle Wundertäter? \bibleverse{30} Haben sie alle
Gaben, gesund zu machen? Reden sie alle mancherlei Sprachen? Können sie
alle auslegen? \bibleverse{31} Strebet aber nach den besten Gaben! Und
ich will euch noch einen köstlicheren Weg zeigen. \footnote{\textbf{12:31}
  1Kor 14,1; 1Kor 14,12}

\hypertarget{section-12}{%
\section{13}\label{section-12}}

\bibleverse{1} Wenn ich mit Menschen- und mit Engelzungen redete, und
hätte der Liebe nicht, so wäre ich ein tönend Erz oder eine klingende
Schelle. \bibleverse{2} Und wenn ich weissagen könnte und wüsste alle
Geheimnisse und alle Erkenntnis und hätte allen Glauben, also dass ich
Berge versetzte, und hätte der Liebe nicht, so wäre ich nichts.
\bibleverse{3} Und wenn ich alle meine Habe den Armen gäbe und ließe
meinen Leib brennen, und hätte der Liebe nicht, so wäre mir's nichts
nütze. \footnote{\textbf{13:3} Mt 6,2} \bibleverse{4} Die Liebe ist
langmütig und freundlich, die Liebe eifert nicht, die Liebe treibt nicht
Mutwillen, sie blähet sich nicht, \bibleverse{5} sie stellet sich nicht
ungebärdig, sie suchet nicht das Ihre, sie lässt sich nicht erbittern,
sie rechnet das Böse nicht zu, \bibleverse{6} sie freuet sich nicht der
Ungerechtigkeit, sie freuet sich aber der Wahrheit; \footnote{\textbf{13:6}
  Röm 12,9} \bibleverse{7} sie verträgt alles, sie glaubet alles, sie
hoffet alles, sie duldet alles. \footnote{\textbf{13:7} Mt 18,21-22; Spr
  10,12; Röm 15,1} \bibleverse{8} Die Liebe höret nimmer auf, so doch
die Weissagungen aufhören werden und die Sprachen aufhören werden und
die Erkenntnis aufhören wird. \bibleverse{9} Denn unser Wissen ist
Stückwerk, und unser Weissagen ist Stückwerk. \bibleverse{10} Wenn aber
kommen wird das Vollkommene, so wird das Stückwerk aufhören.
\bibleverse{11} Da ich ein Kind war, da redete ich wie ein Kind und war
klug wie ein Kind und hatte kindische Anschläge; da ich aber ein Mann
ward, tat ich ab, was kindisch war. \bibleverse{12} Wir sehen jetzt
durch einen Spiegel in einem dunklen Wort; dann aber von Angesicht zu
Angesicht. Jetzt erkenne ich's stückweise; dann aber werde ich erkennen,
gleichwie ich erkannt bin. \footnote{\textbf{13:12} 1Kor 8,3; 4Mo 12,8;
  2Kor 5,7} \bibleverse{13} Nun aber bleibt Glaube, Hoffnung, Liebe,
diese drei; aber die Liebe ist die größte unter ihnen. \footnote{\textbf{13:13}
  1Thes 1,3; 1Jo 4,16}

\hypertarget{section-13}{%
\section{14}\label{section-13}}

\bibleverse{1} Strebet nach der Liebe! Fleißiget euch der geistlichen
Gaben, am meisten aber, dass ihr weissagen möget! \bibleverse{2} Denn
der mit Zungen redet, der redet nicht den Menschen, sondern Gott; denn
ihm hört niemand zu, im Geist aber redet er die Geheimnisse. \footnote{\textbf{14:2}
  Apg 2,4; Apg 10,46} \bibleverse{3} Wer aber weissagt, der redet den
Menschen zur Besserung und zur Ermahnung und zur Tröstung.
\bibleverse{4} Wer mit Zungen redet, der bessert sich selbst; wer aber
weissagt, der bessert die Gemeinde. \bibleverse{5} Ich wollte, dass ihr
alle mit Zungen reden könntet; aber viel mehr, dass ihr weissagtet. Denn
der da weissagt, ist größer, als der mit Zungen redet; es sei denn, dass
er's auch auslege, dass die Gemeinde davon gebessert werde.
\bibleverse{6} Nun aber, liebe Brüder, wenn ich zu euch käme und redete
mit Zungen, was wäre es euch nütze, wenn ich nicht mit euch redete
entweder durch Offenbarung oder durch Erkenntnis oder durch Weissagung
oder durch Lehre? \footnote{\textbf{14:6} 1Kor 12,8} \bibleverse{7}
Verhält sich's doch auch also mit den Dingen, die da lauten, und doch
nicht leben; es sei eine Pfeife oder eine Harfe: wenn sie nicht
unterschiedene Töne von sich geben, wie kann man erkennen, was gepfiffen
oder geharft wird? \bibleverse{8} Und wenn die Posaune einen
undeutlichen Ton gibt, wer wird sich zum Streit rüsten? \bibleverse{9}
Also auch ihr, wenn ihr mit Zungen redet, so ihr nicht eine deutliche
Rede gebet, wie kann man wissen, was geredet ist? Denn ihr werdet in den
Wind reden. \bibleverse{10} Es ist mancherlei Art der Stimmen in der
Welt, und derselben keine ist undeutlich. \bibleverse{11} So ich nun
nicht weiß der Stimme Bedeutung, werde ich unverständlich sein dem, der
da redet, und der da redet, wird mir unverständlich sein.
\bibleverse{12} Also auch ihr, sintemal ihr euch fleißiget der
geistlichen Gaben, trachtet darnach, dass ihr alles reichlich habet, auf
dass ihr die Gemeinde bessert. \bibleverse{13} Darum, welcher mit Zungen
redet, der bete also, dass er's auch auslege. \bibleverse{14} Denn wenn
ich mit Zungen bete, so betet mein Geist; aber mein Sinn bringt niemand
Frucht. \bibleverse{15} Wie soll es aber denn sein? Ich will beten mit
dem Geist und will beten auch im Sinn; ich will Psalmen singen im Geist
und will auch Psalmen singen mit dem Sinn. \footnote{\textbf{14:15} Eph
  5,19} \bibleverse{16} Wenn du aber segnest im Geist, wie soll der, der
an des Laien Statt steht, Amen sagen auf deine Danksagung, sintemal er
nicht weiß, was du sagst? \bibleverse{17} Du danksagest wohl fein, aber
der andere wird nicht davon gebessert. \bibleverse{18} Ich danke meinem
Gott, dass ich mehr mit Zungen rede denn ihr alle. \bibleverse{19} Aber
ich will in der Gemeinde lieber fünf Worte reden mit meinem Sinn, auf
dass ich auch andere unterweise, denn zehntausend Worte mit Zungen.
\bibleverse{20} Liebe Brüder, werdet nicht Kinder an dem Verständnis;
sondern an der Bosheit seid Kinder, an dem Verständnis aber seid
vollkommen. \bibleverse{21} Im Gesetz steht geschrieben: „Ich will mit
anderen Zungen und mit anderen Lippen reden zu diesem Volk, und sie
werden mich auch also nicht hören, spricht der Herr.`` \bibleverse{22}
Darum sind die Zungen zum Zeichen nicht den Gläubigen, sondern den
Ungläubigen; die Weissagung aber nicht den Ungläubigen, sondern den
Gläubigen. \bibleverse{23} Wenn nun die ganze Gemeinde zusammenkäme an
einen Ort und redeten alle mit Zungen, es kämen aber hinein Laien oder
Ungläubige, würden sie nicht sagen, ihr wäret unsinnig? \bibleverse{24}
Wenn sie aber alle weissagten und käme dann ein Ungläubiger oder Laie
hinein, der würde von ihnen allen gestraft und von allen gerichtet;
\bibleverse{25} und also würde das Verborgene seines Herzens offenbar,
und er würde also fallen auf sein Angesicht, Gott anbeten und bekennen,
dass Gott wahrhaftig in euch sei. \footnote{\textbf{14:25} Joh 16,8}
\bibleverse{26} Wie ist es denn nun, liebe Brüder? Wenn ihr
zusammenkommt, so hat ein jeglicher Psalmen, er hat eine Lehre, er hat
Zungen, er hat Offenbarung, er hat Auslegung. Lasset es alles geschehen
zur Besserung! \footnote{\textbf{14:26} 1Kor 12,8-10; Eph 4,12}
\bibleverse{27} Wenn jemand mit Zungen redet, so seien es ihrer zwei
oder aufs meiste drei, und einer um den anderen; und einer lege es aus.
\bibleverse{28} Ist aber kein Ausleger da, so schweige er in der
Gemeinde, rede aber sich selber und Gott. \bibleverse{29} Weissager aber
lasset reden zwei oder drei, und die anderen lasset richten. \footnote{\textbf{14:29}
  1Thes 5,21; Apg 17,11} \bibleverse{30} Wenn aber eine Offenbarung
geschieht einem anderen, der da sitzt, so schweige der erste.
\bibleverse{31} Ihr könnt wohl alle weissagen, einer nach dem anderen,
auf dass sie alle lernen und alle ermahnt werden. \bibleverse{32} Und
die Geister der Propheten sind den Propheten untertan. \bibleverse{33}
Denn Gott ist nicht ein Gott der Unordnung, sondern des Friedens.
\bibleverse{34} Wie in allen Gemeinden der Heiligen lasset eure Weiber
schweigen in der Gemeinde; denn es soll ihnen nicht zugelassen werden,
dass sie reden, sondern sie sollen untertan sein, wie auch das Gesetz
sagt. \footnote{\textbf{14:34} 1Tim 2,11-12; 1Mo 3,16} \bibleverse{35}
Wollen sie aber etwas lernen, so lasset sie daheim ihre Männer fragen.
Es steht den Weibern übel an, in der Gemeinde zu reden. \bibleverse{36}
Oder ist das Wort Gottes von euch ausgekommen? Oder ist's allein zu euch
gekommen? \bibleverse{37} Wenn sich jemand lässt dünken, er sei ein
Prophet oder geistlich, der erkenne, was ich euch schreibe; denn es sind
des Herrn Gebote. \bibleverse{38} Ist aber jemand unwissend, der sei
unwissend. \bibleverse{39} Darum, liebe Brüder, fleißiget euch des
Weissagens und wehret nicht, mit Zungen zu reden. \bibleverse{40} Lasset
alles ehrbar und ordentlich zugehen. \footnote{\textbf{14:40} 1Kor
  14,33; Kol 2,5}

\hypertarget{section-14}{%
\section{15}\label{section-14}}

\bibleverse{1} Ich erinnere euch aber, liebe Brüder, des Evangeliums,
das ich euch verkündigt habe, welches ihr auch angenommen habt, in
welchem ihr auch stehet, \bibleverse{2} durch welches ihr auch selig
werdet: welchergestalt ich es euch verkündigt habe, so ihr's behalten
habt; es wäre denn, dass ihr umsonst geglaubt hättet. \bibleverse{3}
Denn ich habe euch zuvörderst gegeben, was ich auch empfangen habe: dass
Christus gestorben sei für unsere Sünden nach der Schrift,
\bibleverse{4} und dass er begraben sei, und dass er auferstanden sei am
dritten Tage nach der Schrift, \footnote{\textbf{15:4} Lk 24,27; Lk
  24,44-46} \bibleverse{5} und dass er gesehen worden ist von Kephas,
darnach von den Zwölfen. \footnote{\textbf{15:5} Joh 20,19; Joh 20,26;
  Lk 23,34} \bibleverse{6} Darnach ist er gesehen worden von mehr denn
fünfhundert Brüdern auf einmal, deren noch viele leben, etliche aber
sind entschlafen. \bibleverse{7} Darnach ist er gesehen worden von
Jakobus, darnach von allen Aposteln. \footnote{\textbf{15:7} Lk 24,50}
\bibleverse{8} Am letzten nach allen ist er auch von mir, als einer
unzeitigen Geburt, gesehen worden. \footnote{\textbf{15:8} 1Kor 9,1; Apg
  9,3-6} \bibleverse{9} Denn ich bin der geringste unter den Aposteln,
der ich nicht wert bin, dass ich ein Apostel heiße, darum dass ich die
Gemeinde Gottes verfolgt habe. \footnote{\textbf{15:9} Apg 8,3; Eph 3,8}
\bibleverse{10} Aber von Gottes Gnade bin ich, was ich bin. Und seine
Gnade an mir ist nicht vergeblich gewesen, sondern ich habe vielmehr
gearbeitet denn sie alle; nicht aber ich, sondern Gottes Gnade, die mit
mir ist. \footnote{\textbf{15:10} 2Kor 11,5; 2Kor 11,23} \bibleverse{11}
Es sei nun ich oder jene: also predigen wir, und also habt ihr geglaubt.
\bibleverse{12} So aber Christus gepredigt wird, dass er sei von den
Toten auferstanden, wie sagen denn etliche unter euch, die Auferstehung
der Toten sei nichts? \bibleverse{13} Ist aber die Auferstehung der
Toten nichts, so ist auch Christus nicht auferstanden. \bibleverse{14}
Ist aber Christus nicht auferstanden, so ist unsere Predigt vergeblich,
so ist auch euer Glaube vergeblich. \bibleverse{15} Wir würden aber auch
erfunden als falsche Zeugen Gottes, dass wir wider Gott gezeugt hätten,
er hätte Christum auferweckt, den er nicht auferweckt hätte, wenn doch
die Toten nicht auferstehen. \footnote{\textbf{15:15} Apg 1,22}
\bibleverse{16} Denn wenn die Toten nicht auferstehen, so ist auch
Christus nicht auferstanden. \bibleverse{17} Ist aber Christus nicht
auferstanden, so ist euer Glaube eitel, so seid ihr noch in euren
Sünden. \bibleverse{18} So sind auch die, die in Christo entschlafen
sind, verloren. \bibleverse{19} Hoffen wir allein in diesem Leben auf
Christum, so sind wir die elendesten unter allen Menschen.
\bibleverse{20} Nun aber ist Christus auferstanden von den Toten und der
Erstling geworden unter denen, die da schlafen. \bibleverse{21} Sintemal
durch einen Menschen der Tod und durch einen Menschen die Auferstehung
der Toten kommt. \footnote{\textbf{15:21} 1Mo 3,17-19; Röm 5,18}
\bibleverse{22} Denn gleichwie sie in Adam alle sterben, also werden sie
in Christo alle lebendig gemacht werden. \bibleverse{23} Ein jeglicher
aber in seiner Ordnung: der Erstling Christus; darnach die Christo
angehören, wenn er kommen wird; \bibleverse{24} darnach das Ende, wenn
er das Reich Gott und dem Vater überantworten wird, wenn er aufheben
wird alle Herrschaft und alle Obrigkeit und Gewalt. \footnote{\textbf{15:24}
  Röm 8,38} \bibleverse{25} Er muss aber herrschen, bis dass er „alle
seine Feinde unter seine Füße lege``. \footnote{\textbf{15:25} Mt 22,44}
\bibleverse{26} Der letzte Feind, der aufgehoben wird, ist der Tod.
\footnote{\textbf{15:26} Offb 20,14; Offb 21,4} \bibleverse{27} Denn „er
hat ihm alles unter seine Füße getan``. Wenn er aber sagt, dass es alles
untertan sei, ist's offenbar, dass ausgenommen ist, der ihm alles
untergetan hat. \bibleverse{28} Wenn aber alles ihm untertan sein wird,
alsdann wird auch der Sohn selbst untertan sein dem, der ihm alles
untergetan hat, auf dass Gott sei alles in allen. \bibleverse{29} Was
machen sonst, die sich taufen lassen über den Toten, wenn überhaupt die
Toten nicht auferstehen? Was lassen sie sich taufen über den Toten?
\bibleverse{30} Und was stehen wir alle Stunde in der Gefahr?
\bibleverse{31} Bei unserem Ruhm, den ich habe in Christo Jesu, unserem
Herrn, ich sterbe täglich. \footnote{\textbf{15:31} 2Kor 4,10}
\bibleverse{32} Habe ich nach menschlicher Meinung zu Ephesus mit wilden
Tieren gefochten, was hilft's mir? Wenn die Toten nicht auferstehen,
„lasset uns essen und trinken; denn morgen sind wir tot!{}``
\bibleverse{33} Lasset euch nicht verführen! Böse Geschwätze verderben
gute Sitten. \bibleverse{34} Werdet doch einmal recht nüchtern und
sündiget nicht! Denn etliche wissen nichts von Gott; das sage ich euch
zur Schande. \bibleverse{35} Möchte aber jemand sagen: Wie werden die
Toten auferstehen, und mit welcherlei Leibe werden sie kommen?
\bibleverse{36} Du Narr: was du säst, wird nicht lebendig, es sterbe
denn. \footnote{\textbf{15:36} Joh 12,24} \bibleverse{37} Und was du
säst, ist ja nicht der Leib, der werden soll, sondern ein bloßes Korn,
etwa Weizen oder der anderen eines. \bibleverse{38} Gott aber gibt ihm
einen Leib, wie er will, und einem jeglichen von den Samen seinen
eigenen Leib. \bibleverse{39} Nicht ist alles Fleisch einerlei Fleisch;
sondern ein anderes Fleisch ist der Menschen, ein anderes des Viehs, ein
anderes der Fische, ein anderes der Vögel. \bibleverse{40} Und es sind
himmlische Körper und irdische Körper; aber eine andere Herrlichkeit
haben die himmlischen Körper und eine andere die irdischen.
\bibleverse{41} Eine andere Klarheit hat die Sonne, eine andere Klarheit
hat der Mond, eine andere Klarheit haben die Sterne; denn ein Stern
übertrifft den anderen an Klarheit. \bibleverse{42} Also auch die
Auferstehung der Toten. Es wird gesät verweslich, und wird auferstehen
unverweslich. \bibleverse{43} Es wird gesät in Unehre, und wird
auferstehen in Herrlichkeit. Es wird gesät in Schwachheit, und wird
auferstehen in Kraft. \footnote{\textbf{15:43} Phil 3,21; Kol 3,4}
\bibleverse{44} Es wird gesät ein natürlicher Leib, und wird auferstehen
ein geistlicher Leib. Ist ein natürlicher Leib, so ist auch ein
geistlicher Leib. \bibleverse{45} Wie es geschrieben steht: Der erste
Mensch, Adam, „ward zu einer lebendigen Seele``, und der letzte Adam zum
Geist, der da lebendig macht. \bibleverse{46} Aber der geistliche Leib
ist nicht der erste, sondern der natürliche; darnach der geistliche.
\bibleverse{47} Der erste Mensch ist von der Erde und irdisch; der
andere Mensch ist der Herr vom Himmel. \bibleverse{48} Welcherlei der
irdische ist, solcherlei sind auch die irdischen; und welcherlei der
himmlische ist, solcherlei sind auch die himmlischen. \bibleverse{49}
Und wie wir getragen haben das Bild des irdischen, also werden wir auch
tragen das Bild des himmlischen. \footnote{\textbf{15:49} 1Mo 5,3}
\bibleverse{50} Das sage ich aber, liebe Brüder, dass Fleisch und Blut
nicht können das Reich Gottes ererben; auch wird das Verwesliche nicht
erben das Unverwesliche. \bibleverse{51} Siehe, ich sage euch ein
Geheimnis: Wir werden nicht alle entschlafen, wir werden aber alle
verwandelt werden; \bibleverse{52} und dasselbe plötzlich, in einem
Augenblick, zur Zeit der letzten Posaune. Denn es wird die Posaune
schallen, und die Toten werden auferstehen unverweslich, und wir werden
verwandelt werden. \footnote{\textbf{15:52} Mt 24,31} \bibleverse{53}
Denn dies Verwesliche muss anziehen die Unverweslichkeit, und dies
Sterbliche muss anziehen die Unsterblichkeit. \footnote{\textbf{15:53}
  2Kor 5,4} \bibleverse{54} Wenn aber dies Verwesliche wird anziehen die
Unverweslichkeit, und dies Sterbliche wird anziehen die Unsterblichkeit,
dann wird erfüllt werden das Wort, das geschrieben steht:
\bibleverse{55} „Der Tod ist verschlungen in den Sieg. Tod, wo ist dein
Stachel? Hölle, wo ist dein Sieg?{}`` \bibleverse{56} Aber der Stachel
des Todes ist die Sünde; die Kraft aber der Sünde ist das Gesetz.
\footnote{\textbf{15:56} Röm 7,8; Röm 7,11; Röm 7,13} \bibleverse{57}
Gott aber sei Dank, der uns den Sieg gegeben hat durch unseren Herrn
Jesus Christus! \footnote{\textbf{15:57} 1Jo 5,4} \bibleverse{58} Darum,
meine lieben Brüder, seid fest, unbeweglich, und nehmet immer zu in dem
Werk des Herrn, sintemal ihr wisset, dass eure Arbeit nicht vergeblich
ist in dem Herrn. \footnote{\textbf{15:58} 2Chr 15,7}

\hypertarget{section-15}{%
\section{16}\label{section-15}}

\bibleverse{1} Was aber die Steuer anlangt, die den Heiligen geschieht;
wie ich den Gemeinden in Galatien geordnet habe, also tut auch ihr.
\^{}\^{} \bibleverse{2} An jeglichem ersten Tag der Woche lege bei sich
selbst ein jeglicher unter euch und sammle, was ihn gut dünkt, auf dass
nicht, wenn ich komme, dann allererst die Steuer zu sammeln sei.
\^{}\^{} \bibleverse{3} Wenn ich aber gekommen bin, so will ich die,
welche ihr dafür anseht, mit Briefen senden, dass sie hinbringen eure
Wohltat gen Jerusalem. \bibleverse{4} So es aber wert ist, dass ich auch
hinreise, sollen sie mit mir reisen. \bibleverse{5} Ich will aber zu
euch kommen, wenn ich durch Mazedonien gezogen bin; denn durch
Mazedonien werde ich ziehen. \bibleverse{6} Bei euch aber werde ich
vielleicht bleiben oder auch überwintern, auf dass ihr mich geleitet, wo
ich hin ziehen werde. \bibleverse{7} Ich will euch jetzt nicht sehen im
Vorüberziehen; denn ich hoffe, ich werde etliche Zeit bei euch bleiben,
so es der Herr zulässt. \^{}\^{} \bibleverse{8} Ich werde aber zu
Ephesus bleiben bis Pfingsten. \^{}\^{} \bibleverse{9} Denn mir ist eine
große Tür aufgetan, die viel Frucht wirkt, und sind viel Widersacher da.
\^{}\^{} \bibleverse{10} So Timotheus kommt, so sehet zu, dass er ohne
Furcht bei euch sei; denn er treibt auch das Werk des Herrn wie ich.
\^{}\^{} \bibleverse{11} Dass ihn nun nicht jemand verachte! Geleitet
ihn aber im Frieden, dass er zu mir komme; denn ich warte sein mit den
Brüdern. \bibleverse{12} Von Apollos, dem Bruder, aber wisset, dass ich
ihn sehr viel ermahnt habe, dass er zu euch käme mit den Brüdern; und es
war durchaus sein Wille nicht, dass er jetzt käme; er wird aber kommen,
wenn es ihm gelegen sein wird. \^{}\^{} \bibleverse{13} Wachet, stehet
im Glauben, seid männlich und seid stark! \^{}\^{} \bibleverse{14} Alle
eure Dinge lasset in der Liebe geschehen! \bibleverse{15} Ich ermahne
euch aber, liebe Brüder: Ihr kennet das Haus des Stephanas, dass sie
sind die Erstlinge in Achaja und haben sich selbst verordnet zum Dienst
den Heiligen; \^{}\^{} \bibleverse{16} dass auch ihr solchen untertan
seid und allen, die mitwirken und arbeiten. \^{}\^{} \bibleverse{17} Ich
freue mich über die Ankunft des Stephanas und Fortunatus und Achaikus;
denn wo ich an euch Mangel hatte, das haben sie erstattet.
\bibleverse{18} Sie haben erquickt meinen und euren Geist. Erkennet die
an, die solche sind! \^{}\^{} \bibleverse{19} Es grüßen euch die
Gemeinden in Asien. Es grüßt euch sehr in dem Herrn Aquila und Priscilla
samt der Gemeinde in ihrem Hause. \^{}\^{} \bibleverse{20} Es grüßen
euch alle Brüder. Grüßet euch untereinander mit dem heiligen Kuss.
\bibleverse{21} Ich, Paulus, grüße euch mit meiner Hand. \^{}\^{}
\bibleverse{22} Wenn jemand den Herrn Jesus Christus nicht liebhat, der
sei anathema. Maran atha! (d.~h.: der sei verflucht. Unser Herr kommt!)
\^{}\^{} \bibleverse{23} Die Gnade des Herrn Jesu Christi sei mit euch!
\bibleverse{24} Meine Liebe sei mit euch allen in Christo Jesu! Amen.
