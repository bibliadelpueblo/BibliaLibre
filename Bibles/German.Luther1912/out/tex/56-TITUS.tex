\hypertarget{section}{%
\section{1}\label{section}}

\bibverse{1} Paulus, ein Knecht Gottes und ein Apostel Jesu Christi,
nach dem Glauben der Auserwählten Gottes und der Erkenntnis der Wahrheit
zur Gottseligkeit, \bibverse{2} auf Hoffnung des ewigen Lebens, welches
verheißen hat, der nicht lügt, Gott, vor den Zeiten der Welt,
\footnote{\textbf{1:2} 2Tim 2,13} \bibverse{3} aber zu seiner Zeit hat
er offenbart sein Wort durch die Predigt, die mir vertrauet ist nach dem
Befehl Gottes, unseres Heilandes, \footnote{\textbf{1:3} Eph 1,9-10}
\bibverse{4} dem Titus, meinem rechtschaffenen Sohn nach unser beider
Glauben: Gnade, Barmherzigkeit, Friede von Gott, dem Vater, und dem
Herrn Jesus Christus, unserem Heiland! \footnote{\textbf{1:4} 1Tim 1,2}

\bibverse{5} Derhalben ließ ich dich in Kreta, dass du solltest vollends
ausrichten, was ich gelassen habe, und besetzen die Städte hin und her
mit Ältesten, wie ich dir befohlen haben; \footnote{\textbf{1:5} Apg
  14,23} \bibverse{6} wo einer ist untadelig, eines Weibes Mann, der
gläubige Kinder habe, nicht berüchtigt, dass sie Schwelger und
ungehorsam sind. \footnote{\textbf{1:6} 1Tim 3,1-7} \bibverse{7} Denn
ein Bischof soll untadelig sein als ein Haushalter Gottes, nicht
eigensinnig, nicht zornig, nicht ein Weinsäufer, nicht raufen, nicht
unehrliche Hantierung treiben; \footnote{\textbf{1:7} 1Kor 4,1; 2Tim
  2,24} \bibverse{8} sondern gastfrei, gütig, züchtig, gerecht, heilig,
keusch, \bibverse{9} und haltend ob dem Wort, das gewiss ist, und
lehrhaft, auf dass er mächtig sei, zu ermahnen durch die heilsame Lehre
und zu strafen die Widersprecher.

\bibverse{10} Denn es sind viel freche und unnütze Schwätzer und
Verführer, sonderlich die aus den Juden, \bibverse{11} welchen man muss
das Maul stopfen, die da ganze Häuser verkehren und lehren, was nicht
taugt, um schändlichen Gewinns willen. \bibverse{12} Es hat einer aus
ihnen gesagt, ihr eigener Prophet: „Die Kreter sind immer Lügner, böse
Tiere und faule Bäuche.`` \bibverse{13} Dies Zeugnis ist wahr. Um der
Sache willen strafe sie scharf, auf dass sie gesund seien im Glauben
\footnote{\textbf{1:13} 2Tim 4,2} \bibverse{14} und nicht achten auf die
jüdischen Fabeln und Gebote von Menschen, welche sich von der Wahrheit
abwenden. \footnote{\textbf{1:14} 1Tim 4,7; 2Tim 4,4} \bibverse{15} Den
Reinen ist alles rein; den Unreinen aber und Ungläubigen ist nichts
rein, sondern unrein ist ihr Sinn sowohl als ihr Gewissen. \footnote{\textbf{1:15}
  Mt 15,11; Röm 14,20} \bibverse{16} Sie sagen, sie erkennen Gott; aber
mit den Werken verleugnen sie es, sintemal sie es sind, an welchen Gott
Gräuel hat, und gehorchen nicht und sind zu allem guten Werk untüchtig.
\footnote{\textbf{1:16} 2Tim 3,5}

\hypertarget{section-1}{%
\section{2}\label{section-1}}

\bibverse{1} Du aber rede, wie sich's ziemt nach der heilsamen Lehre:
\footnote{\textbf{2:1} 2Tim 1,13} \bibverse{2} den Alten sage, dass sie
nüchtern seien, ehrbar, züchtig, gesund im Glauben, in der Liebe, in der
Geduld; \footnote{\textbf{2:2} 1Tim 5,1} \bibverse{3} den alten Weibern
desgleichen, dass sie sich halten, wie den Heiligen ziemt, nicht
Lästerinnen seien, nicht Weinsäuferinnen, gute Lehrerinnen; \footnote{\textbf{2:3}
  1Tim 3,11} \bibverse{4} dass sie die jungen Weiber lehren züchtig
sein, ihre Männer lieben, Kinder lieben, \bibverse{5} sittig sein,
keusch, häuslich, gütig, ihren Männern untertan, auf dass nicht das Wort
Gottes verlästert werde.

\bibverse{6} Desgleichen die jungen Männer ermahne, dass sie züchtig
seien. \bibverse{7} Allenthalben aber stelle dich selbst zum Vorbilde
guter Werke, mit unverfälschter Lehre, mit Ehrbarkeit, \footnote{\textbf{2:7}
  1Tim 4,12; 1Petr 5,3; 2Tim 2,15; 1Petr 2,15} \bibverse{8} mit
heilsamem und untadeligem Wort, auf dass der Widersacher sich schäme und
nichts habe, dass er von uns möge Böses sagen.

\bibverse{9} Den Knechten sage, dass sie ihren Herren untertänig seien,
in allen Dingen zu Gefallen tun, nicht widerbellen, \bibverse{10} nicht
veruntreuen, sondern alle gute Treue erzeigen, auf dass sie die Lehre
Gottes, unseres Heilandes, zieren in allen Stücken. \bibverse{11} Denn
es ist erschienen die heilsame Gnade Gottes allen Menschen \footnote{\textbf{2:11}
  Tit 3,4} \bibverse{12} und züchtigt uns, dass wir sollen verleugnen
das ungöttliche Wesen und die weltlichen Lüste, und züchtig, gerecht und
gottselig leben in dieser Welt \bibverse{13} und warten auf die selige
Hoffnung und Erscheinung der Herrlichkeit des großen Gottes und unseres
Heilandes, Jesu Christi, \bibverse{14} der sich selbst für uns gegeben
hat, auf dass er uns erlöste von aller Ungerechtigkeit und reinigte sich
selbst ein Volk zum Eigentum, das fleißig wäre zu guten Werken.
\footnote{\textbf{2:14} Gal 1,4; 1Tim 2,6; 2Mo 19,5; Eph 2,10}

\bibverse{15} Solches rede und ermahne und strafe mit ganzem Ernst. Lass
dich niemand verachten. \footnote{\textbf{2:15} 1Tim 4,12}

\hypertarget{section-2}{%
\section{3}\label{section-2}}

\bibverse{1} Erinnere sie, dass sie den Fürsten und der Obrigkeit
untertan und gehorsam seien, zu allem guten Werk bereit seien,
\footnote{\textbf{3:1} Röm 13,1; 1Petr 2,13} \bibverse{2} niemand
lästern, nicht hadern, gelinde seien, alle Sanftmütigkeit beweisen gegen
alle Menschen. \footnote{\textbf{3:2} Phil 4,5} \bibverse{3} Denn wir
waren auch vordem unweise, ungehorsam, verirrt, dienend den Begierden
und mancherlei Wollüsten, und wandelten in Bosheit und Neid, waren
verhasst und hassten uns untereinander. \footnote{\textbf{3:3} 1Kor
  6,11; Eph 2,2; Eph 5,8; 1Petr 4,3} \bibverse{4} Da aber erschien die
Freundlichkeit und Leutseligkeit Gottes, unseres Heilandes, --
\footnote{\textbf{3:4} Tit 2,11} \bibverse{5} nicht um der Werke willen
der Gerechtigkeit, die wir getan hatten, sondern nach seiner
Barmherzigkeit machte er uns selig durch das Bad der Wiedergeburt und
Erneuerung des heiligen Geistes, \footnote{\textbf{3:5} 2Tim 1,9; Joh
  3,5; Eph 5,26} \bibverse{6} welchen er ausgegossen hat über uns
reichlich durch Jesum Christum, unseren Heiland, \footnote{\textbf{3:6}
  Joe 3,1} \bibverse{7} auf dass wir durch desselben Gnade gerecht und
Erben seien des ewigen Lebens nach der Hoffnung. \footnote{\textbf{3:7}
  Röm 3,26} \bibverse{8} Das ist gewisslich wahr; solches will ich, dass
du fest lehrest, auf dass die, die an Gott gläubig sind geworden, in
einem Stand guter Werke gefunden werden. Solches ist gut und nütze den
Menschen. \bibverse{9} Der törichten Fragen aber, der
Geschlechtsregister, des Zankes und Streites über das Gesetz entschlage
dich; denn sie sind unnütz und eitel. \bibverse{10} Einen ketzerischen
Menschen meide, wenn er einmal und abermals ermahnt ist, \footnote{\textbf{3:10}
  Mt 18,15-17; 2Jo 1,10} \bibverse{11} und wisse, dass ein solcher
verkehrt ist und sündigt, als der sich selbst verurteilt hat.
\footnote{\textbf{3:11} 1Tim 6,4-5}

\bibverse{12} Wenn ich zu dir senden werde Artemas oder Tychikus, so
komm eilend zu mir gen Nikopolis; denn daselbst habe ich beschlossen den
Winter zu bleiben. \footnote{\textbf{3:12} Eph 6,21} \bibverse{13}
Zenas, den Schriftgelehrten, und Apollos fertige ab mit Fleiß, auf dass
ihnen nichts gebreche. \footnote{\textbf{3:13} Apg 18,24; 1Kor 3,5-6}
\bibverse{14} Lass aber auch die Unseren lernen, dass sie im Stand guter
Werke sich finden lassen, wo man ihrer bedarf, auf dass sie nicht
unfruchtbar seien. \footnote{\textbf{3:14} Tit 2,14; Mt 7,19}

\bibverse{15} Es grüßen dich alle, die mit mir sind. Grüße alle, die uns
lieben im Glauben. Die Gnade sei mit euch allen! Amen.
