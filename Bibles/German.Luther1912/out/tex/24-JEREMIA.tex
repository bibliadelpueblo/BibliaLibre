\hypertarget{section}{%
\section{1}\label{section}}

\bibverse{1} Dies sind die Reden Jeremias, des Sohnes Hilkias, aus den
Priestern zu Anathoth im Lande Benjamin, \bibverse{2} zu welchem geschah
das Wort des HErrn zur Zeit Josias, des Sohnes Amons, des Königs in
Juda, im dreizehnten Jahr seines Königreichs, \footnote{\textbf{1:2} 2Kö
  21,24} \bibverse{3} und hernach zur Zeit des Königs in Juda, Jojakims,
des Sohnes Josias, bis ans Ende des elften Jahres Zedekias, des Sohnes
Josias, des Königs in Juda, bis auf die Gefangenschaft Jerusalems im
fünften Monat. \footnote{\textbf{1:3} 2Kö 23,34; 2Kö 24,17; 2Kö 25,2;
  2Kö 25,8} \bibverse{4} Und des HErrn Wort geschah zu mir und sprach:
\bibverse{5} Ich kannte dich, ehe denn ich dich im Mutterleibe
bereitete, und sonderte dich aus, ehe denn du von der Mutter geboren
wurdest, und stellte dich zum Propheten unter die Völker. \footnote{\textbf{1:5}
  Jes 49,1; Gal 1,15}

\bibverse{6} Ich aber sprach: Ach Herr HErr, ich tauge nicht, zu
predigen; denn ich bin zu jung. \footnote{\textbf{1:6} 2Mo 3,11; Jes
  6,5-8}

\bibverse{7} Der HErr sprach aber zu mir: Sage nicht: „Ich bin zu
jung``; sondern du sollst gehen, wohin ich dich sende, und predigen, was
ich dich heiße. \bibverse{8} Fürchte dich nicht vor ihnen; denn ich bin
bei dir und will dich erretten, spricht der HErr.

\bibverse{9} Und der HErr reckte seine Hand aus und rührte meinen Mund
an und sprach zu mir: Siehe, ich lege meine Worte in deinen Mund.
\footnote{\textbf{1:9} 5Mo 18,18} \bibverse{10} Siehe, ich setze dich
heute dieses Tages über Völker und Königreiche, dass du ausreißen,
zerbrechen, verstören und verderben sollst und bauen und pflanzen.
\footnote{\textbf{1:10} Jer 18,7-10}

\bibverse{11} Und es geschah des HErrn Wort zu mir und sprach: Jeremia,
was siehst du? Ich sprach: Ich sehe einen erwachenden Zweig.

\bibverse{12} Und der HErr sprach zu mir: Du hast recht gesehen; denn
ich will wachen über mein Wort, dass ich's tue. \footnote{\textbf{1:12}
  Jer 31,28}

\bibverse{13} Und es geschah des HErrn Wort zum andernmal zu mir und
sprach: Was siehst du? Ich sprach: Ich sehe einen heißsiedenden Topf von
Mitternacht her.

\bibverse{14} Und der HErr sprach zu mir: Von Mitternacht wird das
Unglück ausbrechen über alle, die im Lande wohnen.

\bibverse{15} Denn siehe, ich will rufen alle Fürsten in den
Königreichen gegen Mitternacht, spricht der HErr, dass sie kommen sollen
und ihre Stühle setzen vor die Tore zu Jerusalem und rings um die Mauern
her und vor alle Städte Judas.

\bibverse{16} Und ich will das Recht lassen über sie gehen um all ihrer
Bosheit willen, dass sie mich verlassen und räuchern anderen Göttern und
beten an ihrer Hände Werk.

\bibverse{17} So begürte nun deine Lenden und mache dich auf und predige
ihnen alles, was ich dich heiße. Erschrick nicht vor ihnen, auf dass ich
dich nicht erschrecke vor ihnen; \bibverse{18} denn ich will dich heute
zur festen Stadt, zur eisernen Säule, zur ehernen Mauer machen im ganzen
Lande, wider die Könige Judas, wider ihre Fürsten, wider ihre Priester,
wider das Volk im Lande, \bibverse{19} dass, wenn sie gleich wider dich
streiten, sie dennoch nicht sollen wider dich siegen; denn ich bin bei
dir, spricht der HErr, dass ich dich errette. \# 2 \bibverse{1} Und des
HErrn Wort geschah zu mir und sprach: \bibverse{2} Gehe hin und predige
öffentlich zu Jerusalem und sprich: So spricht der HErr: Ich gedenke, da
du eine freundliche, junge Dirne und eine liebe Braut warst, da du mir
folgtest in der Wüste, in dem Lande, da man nichts sät, \bibverse{3} da
Israel des HErrn eigen war und seine erste Frucht. Wer sie fressen
wollte, musste Schuld haben, und Unglück musste über ihn kommen, spricht
der HErr.

\bibverse{4} Höret des HErrn Wort, ihr vom Hause Jakob und alle
Geschlechter vom Hause Israel. \bibverse{5} So spricht der HErr: Was
haben doch eure Väter Unrechtes an mir gefunden, dass sie von mir wichen
und hingen an den unnützen Götzen, da sie doch nichts erlangten?
\footnote{\textbf{2:5} Mi 6,3-6} \bibverse{6} und dachten nie einmal: Wo
ist der HErr, der uns aus Ägyptenland führte und leitete uns in der
Wüste, im wilden, ungebahnten Lande, im dürren und finsteren Lande, in
dem Lande, da niemand wandelte noch ein Mensch wohnte? \bibverse{7} Und
ich brachte euch in ein gutes Land, dass ihr äßet seine Früchte und
Güter. Und da ihr hineinkamt, verunreinigtet ihr mein Land und machtet
mir mein Erbe zum Gräuel. \bibverse{8} Die Priester gedachten nicht: Wo
ist der HErr? und die das Gesetz treiben, achteten mein nicht, und die
Hirten führten die Leute von mir, und die Propheten weissagten durch
Baal und hingen an den unnützen Götzen. \bibverse{9} Darum muss ich noch
immer mit euch und mit euren Kindeskindern hadern, spricht der HErr.
\bibverse{10} Gehet hin in die Inseln Chittim und schauet, und sendet
nach Kedar und merket mit Fleiß und schauet, ob's daselbst so zugeht!
\bibverse{11} ob die Heiden ihre Götter ändern, wiewohl sie doch nicht
Götter sind! Und mein Volk hat doch seine Herrlichkeit verändert um
einen unnützen Götzen. \bibverse{12} Sollte sich doch der Himmel davor
entsetzen, erschrecken und sehr erbeben, spricht der HErr. \bibverse{13}
Denn mein Volk tut eine zwiefache Sünde: mich, die lebendige Quelle,
verlassen sie und machen sich hier und da ausgehauene Brunnen, die doch
löcherig sind und kein Wasser geben. \footnote{\textbf{2:13} Jer 17,13;
  Ps 36,10} \bibverse{14} Ist denn Israel ein Knecht oder leibeigen,
dass er jedermanns Raub sein muss? \bibverse{15} Denn Löwen brüllen über
ihn und schreien und verwüsten sein Land, und seine Städte werden
verbrannt, dass niemand darin wohnt. \bibverse{16} Dazu zerschlagen die
von Noph und Thachpanhes dir den Kopf. \bibverse{17} Solches machst du
dir selbst, weil du den HErrn, deinen Gott, verlässest, so oft er dich
den rechten Weg leiten will. \footnote{\textbf{2:17} Hos 13,9}
\bibverse{18} Was hilft's dir, dass du nach Ägypten ziehst und willst
vom Wasser Sihor trinken? Und was hilft's dir, dass du nach Assyrien
ziehst und willst vom Wasser Euphrat trinken? \bibverse{19} Es ist
deiner Bosheit Schuld, dass du so gestäupt wirst, und deines
Ungehorsams, dass du so gestraft wirst. Also musst du innewerden und
erfahren, was es für Jammer und Herzeleid bringt, den HErrn, deinen
Gott, verlassen und ihn nicht fürchten, spricht der Herr HErr Zebaoth.
\bibverse{20} Denn du hast immerdar dein Joch zerbrochen und deine Bande
zerrissen und gesagt: Ich will nicht so unterworfen sein! sondern auf
allen hohen Hügeln und unter allen grünen Bäumen liefst du den Götzen
nach. \bibverse{21} Ich aber hatte dich gepflanzt zu einem süßen
Weinstock, einen ganz rechtschaffenen Samen. Wie bist du mir denn
geraten zu einem bitteren, wilden Weinstock? \footnote{\textbf{2:21} Jes
  5,1-4} \bibverse{22} Und wenn du dich gleich mit Lauge wüschest und
nähmest viel Seife dazu, so gleißt doch deine Untugend desto mehr vor
mir, spricht der Herr HErr. \bibverse{23} Wie darfst du denn sagen: Ich
bin nicht unrein, ich hänge nicht an den Baalim? Siehe an, wie du es
treibst im Tal, und bedenke, wie du es ausgerichtet hast. \bibverse{24}
Du läufst umher wie eine Kamelstute in der Brunst, und wie ein Wild in
der Wüste pflegt, wenn es vor großer Brunst lechzt und läuft, dass es
niemand aufhalten kann. Wer's wissen will, darf nicht weit laufen; am
Feiertage sieht man es wohl. \bibverse{25} Schone doch deiner Füße, dass
sie nicht bloß, und deines Halses, dass er nicht durstig werde. Aber du
sprichst: Da wird nichts draus; ich muss mit den Fremden buhlen und
ihnen nachlaufen. \bibverse{26} Wie ein Dieb zu Schanden wird, wenn er
ergriffen wird, also wird das Haus Israel zu Schanden werden samt ihren
Königen, Fürsten, Priestern und Propheten, \bibverse{27} die zum Holz
sagen: Du bist mein Vater, -- und zum Stein: Du hast mich gezeugt. Denn
sie kehren mir den Rücken zu und nicht das Angesicht. Aber wenn die Not
hergeht, sprechen sie: Auf, und hilf uns! \bibverse{28} Wo sind aber
dann deine Götter, die du dir gemacht hast? Heiße sie aufstehen; lass
sehen, ob sie dir helfen können in deiner Not! Denn so manche Stadt, so
manchen Gott hast du, Juda. \bibverse{29} Was wollt ihr noch recht haben
wider mich? Ihr seid alle von mir abgefallen, spricht der HErr.
\bibverse{30} Alle Schläge sind verloren an euren Kindern; sie lassen
sich doch nicht ziehen. Denn euer Schwert frisst eure Propheten wie ein
wütiger Löwe. \footnote{\textbf{2:30} Jes 1,5} \bibverse{31} Du böse
Art, merke auf des HErrn Wort! Bin ich denn für Israel eine Wüste oder
ödes Land? Warum spricht denn mein Volk: Wir sind die Herren und müssen
dir nicht nachlaufen? \bibverse{32} Vergisst doch eine Jungfrau ihres
Schmuckes nicht noch eine Braut ihres Schleiers; aber mein Volk vergisst
mein ewiglich. \bibverse{33} Was beschönst du viel dein Tun, dass ich
dir gnädig sein soll? Unter solchem Schein treibst du je mehr und mehr
Bosheit. \bibverse{34} Überdas findet man Blut der armen und
unschuldigen Seelen bei dir an allen Orten, und das ist nicht heimlich,
sondern offenbar an diesen Orten. \bibverse{35} Doch sprichst du: Ich
bin unschuldig; er wende seinen Zorn von mir. Siehe, ich will mit dir
rechten, dass du sprichst: Ich habe nicht gesündigt. \bibverse{36} Wie
weichst du doch so gern und läufst jetzt dahin, jetzt hierher! Aber du
wirst an Ägypten zu Schanden werden, wie du an Assyrien zu Schanden
geworden bist. \bibverse{37} Denn du musst von dort auch wegziehen und
deine Hände über dem Haupt zusammenschlagen; denn der HErr wird deine
Hoffnung trügen lassen, und nichts wird dir bei ihnen gelingen. \# 3
\bibverse{1} Und er spricht: Wenn sich ein Mann von seinem Weibe
scheidet, und sie zieht von ihm und nimmt einen anderen Mann, darf er
sie auch wieder annehmen? Ist's nicht also, dass das Land verunreinigt
würde? Du aber hast mit vielen Buhlen gehurt; doch komm wieder zu mir;
spricht der HErr. \footnote{\textbf{3:1} 5Mo 24,1-4}

\bibverse{2} Hebe deine Augen auf zu den Höhen und siehe, wie du
allenthalben Hurerei treibst. An den Straßen sitzest du und lauerst auf
sie wie ein Araber in der Wüste und verunreinigst das Land mit deiner
Hurerei und Bosheit. \bibverse{3} Darum muss auch der Frühregen
ausbleiben und kein Spätregen kommen. Du hast eine Hurenstirn, du willst
dich nicht mehr schämen \bibverse{4} und schreist gleichwohl zu mir:
„Lieber Vater, du Meister meiner Jugend!

\bibverse{5} willst du denn ewiglich zürnen und nicht vom Grimm
lassen?{}`` Siehe, so redest du, und tust Böses und lässest dir nicht
steuern.

\bibverse{6} Und der HErr sprach zu mir zur Zeit des Königs Josia: Hast
du auch gesehen, was Israel, die Abtrünnige, tat? Sie ging hin auf alle
hohen Berge und unter alle grünen Bäume und trieb daselbst Hurerei.
\bibverse{7} Und ich sprach, da sie solches alles getan hatte: Bekehre
dich zu mir. Aber sie bekehrte sich nicht. Und obwohl ihre Schwester
Juda, die Verstockte, gesehen hat, \bibverse{8} wie ich der Abtrünnigen
Israel Ehebruch gestraft und sie verlassen und ihr einen Scheidebrief
gegeben habe: dennoch fürchtet sich ihre Schwester, die verstockte Juda,
nicht, sondern geht hin und treibt auch Hurerei. \footnote{\textbf{3:8}
  2Kö 17,18-19; Hes 23,2-11} \bibverse{9} Und von dem Geschrei ihrer
Hurerei ist das Land verunreinigt; denn sie treibt Ehebruch mit Stein
und Holz. \bibverse{10} Und in diesem allem bekehrt sich die verstockte
Juda, ihre Schwester, nicht zu mir von ganzem Herzen, sondern heuchelt
also, spricht der HErr.

\bibverse{11} Und der HErr sprach zu mir: Die abtrünnige Israel ist
fromm gegen die verstockte Juda. \bibverse{12} Gehe hin und rufe diese
Worte gegen die Mitternacht und sprich: Kehre wieder, du abtrünnige
Israel, spricht der HErr, so will ich mein Antlitz nicht gegen euch
verstellen. Denn ich bin barmherzig, spricht der HErr, und will nicht
ewiglich zürnen. \bibverse{13} Allein erkenne deine Missetat, dass du
wider den HErrn, deinen Gott, gesündigt hast und bist hin und wieder
gelaufen zu den fremden Göttern unter allen grünen Bäumen und habt
meiner Stimme nicht gehorcht, spricht der HErr. \bibverse{14} Bekehret
euch, ihr abtrünnigen Kinder, spricht der HErr; denn ich will euch mir
vertrauen und will euch holen, einen aus einer Stadt und zwei aus einem
Geschlecht, und will euch bringen gen Zion \footnote{\textbf{3:14} Hos
  2,21; Jes 6,13} \bibverse{15} und will euch Hirten geben nach meinem
Herzen, die euch weiden sollen mit Lehre und Weisheit. \footnote{\textbf{3:15}
  Jer 23,4} \bibverse{16} Und es soll geschehen, wenn ihr gewachsen seid
und euer viel geworden sind im Lande, so soll man, spricht der HErr, zur
selben Zeit nicht mehr sagen von der Bundeslade des HErrn, auch ihrer
nicht mehr gedenken noch davon predigen noch nach ihr fragen, und sie
wird nicht wieder gemacht werden; \bibverse{17} sondern zur selben Zeit
wird man Jerusalem heißen „des HErrn Thron``, und werden sich dahin
sammeln alle Heiden um des Namens des HErrn willen zu Jerusalem und
werden nicht mehr wandeln nach den Gedanken ihres bösen Herzens.
\footnote{\textbf{3:17} Jes 2,2-4; Jes 65,2} \bibverse{18} Zu der Zeit
wird das Haus Juda gehen zum Hause Israel, und sie werden miteinander
kommen von Mitternacht in das Land, das ich euren Vätern zum Erbe
gegeben habe. \footnote{\textbf{3:18} Jes 11,11-13}

\bibverse{19} Und ich sagte dir zu: Wie will ich dir so viele Kinder
geben und das liebe Land, das allerschönste Erbe unter den Völkern! Und
ich sagte dir zu: Du wirst alsdann mich nennen „lieber Vater!{}`` und
nicht von mir weichen. \footnote{\textbf{3:19} Jer 3,4}

\bibverse{20} Aber das Haus Israel achtete mich nicht, gleichwie ein
Weib ihren Buhlen nicht mehr achtet, spricht der HErr. \bibverse{21}
Darum hört man ein klägliches Heulen und Weinen der Kinder Israel auf
den Höhen, dafür dass sie übel getan und des HErrn, ihres Gottes,
vergessen haben. \bibverse{22} So kehret nun wieder, ihr abtrünnigen
Kinder, so will ich euch heilen von eurem Ungehorsam. Siehe, wir kommen
zu dir; denn du bist der HErr, unser Gott.

\bibverse{23} Wahrlich, es ist eitel Betrug mit Hügeln und mit allen
Bergen. Wahrlich, es hat Israel keine Hilfe denn am HErrn, unserem Gott.
\bibverse{24} Und die Schande hat gefressen unserer Väter Arbeit von
unserer Jugend auf samt ihren Schafen, Rindern, Söhnen und Töchtern.
\bibverse{25} Denn worauf wir uns verließen, das ist uns jetzt eitel
Schande, und wessen wir uns trösteten, des müssen wir uns jetzt schämen.
Denn wir sündigten damit wider den HErrn, unseren Gott, beide, wir und
unsere Väter, von unserer Jugend auf, auch bis auf diesen heutigen Tag,
und gehorchten nicht der Stimme des HErrn, unseres Gottes. \# 4
\bibverse{1} Willst du dich, Israel, bekehren, spricht der HErr, so
bekehre dich zu mir. Und wenn du deine Gräuel wegtust von meinem
Angesicht, so sollst du nicht vertrieben werden. \bibverse{2} Alsdann
wirst du ohne Heuchelei recht und heilig schwören: So wahr der HErr
lebt! und die Heiden werden in ihm gesegnet werden und sich sein rühmen.
\footnote{\textbf{4:2} Jer 12,16; Jes 65,16}

\bibverse{3} Denn so spricht der HErr zu denen in Juda und zu Jerusalem:
Pflüget ein Neues und säet nicht unter die Hecken. \footnote{\textbf{4:3}
  Hos 10,12} \bibverse{4} Beschneidet euch dem HErrn und tut weg die
Vorhaut eures Herzens, ihr Männer in Juda und ihr Leute zu Jerusalem,
auf dass nicht mein Grimm ausfahre wie Feuer und brenne, dass niemand
löschen könne, um eurer Bosheit willen. \footnote{\textbf{4:4} Jer 9,25;
  5Mo 10,16} \bibverse{5} Verkündiget in Juda und schreiet laut zu
Jerusalem und sprecht: „Blaset die Drommete im Lande!{}`` Ruft mit
voller Stimme und sprecht: „Sammelt euch und lasst uns in die festen
Städte ziehen!{}`` \bibverse{6} Werft zu Zion ein Panier auf; fliehet
und säumet nicht! Denn ich bringe ein Unglück herzu von Mitternacht und
einen großen Jammer.

\bibverse{7} Es fährt daher der Löwe aus seiner Hecke, und der Verstörer
der Heiden zieht einher aus seinem Ort, dass er dein Land verwüste und
deine Städte ausbrenne, dass niemand darin wohne. \bibverse{8} Darum
ziehet Säcke an, klaget und heulet; denn der grimmige Zorn des HErrn
will sich nicht wenden von uns. \bibverse{9} Zu der Zeit, spricht der
HErr, wird dem König und den Fürsten das Herz entfallen; die Priester
werden bestürzt und die Propheten erschrocken sein.

\bibverse{10} Ich aber sprach: Ach Herr HErr! du hast's diesem Volk und
Jerusalem weit fehlgehen lassen, da sie sagten: „Es wird Friede bei euch
sein``, so doch das Schwert bis an die Seele reicht. \footnote{\textbf{4:10}
  Jer 6,14}

\bibverse{11} Zu derselben Zeit wird man diesem Volk und Jerusalem
sagen: „Es kommt ein dürrer Wind über das Gebirge her aus der Wüste, des
Weges zu der Tochter meines Volks, nicht zum Worfeln noch zu
Schwingen.`` \bibverse{12} Ja, ein Wind kommt, der ihnen zu stark sein
wird; da will ich denn auch mit ihnen rechten.

\bibverse{13} „Siehe, er fährt daher wie Wolken, und seine Wagen sind
wie ein Sturmwind, seine Rosse sind schneller denn Adler. Weh uns! wir
müssen verstört werden.`` \bibverse{14} So wasche nun, Jerusalem, dein
Herz von der Bosheit, auf dass dir geholfen werde. Wie lange wollen bei
dir bleiben deine leidigen Gedanken? \bibverse{15} Denn es kommt ein
Geschrei von Dan her und eine böse Botschaft vom Gebirge Ephraim.
\bibverse{16} Saget an den Heiden, verkündiget in Jerusalem, dass Hüter
kommen aus fernen Landen und werden schreien wider die Städte Judas.
\bibverse{17} Sie werden sich um sie her lagern wie die Hüter auf dem
Felde; denn sie haben mich erzürnt, spricht der HErr. \footnote{\textbf{4:17}
  Jer 1,15; Jer 6,3} \bibverse{18} Das hast du zum Lohn für dein Wesen
und dein Tun. Dann wird dein Herz fühlen, wie deine Bosheit so groß ist.

\bibverse{19} Wie ist mir so herzlich weh! Mein Herz pocht mir im Leibe,
und habe keine Ruhe; denn meine Seele hört der Posaune Hall und eine
Feldschlacht \bibverse{20} und einen Mordschrei über den anderen; denn
das ganze Land wird verheert, plötzlich werden meine Hütten und meine
Gezelte verstört. \bibverse{21} Wie lange soll ich doch das Panier sehen
und der Posaune Hall hören?

\bibverse{22} Aber mein Volk ist toll, und sie glauben mir nicht;
töricht sind sie und achten's nicht. Weise sind sie genug, Übles zu tun;
aber wohltun wollen sie nicht lernen. \bibverse{23} Ich schaute das Land
an, siehe, das war wüst und öde, und den Himmel, und er war finster.
\bibverse{24} Ich sah die Berge an, und siehe, die bebten, und alle
Hügel zitterten. \bibverse{25} Ich sah, und siehe, da war kein Mensch,
und alle Vögel unter dem Himmel waren weggeflogen. \bibverse{26} Ich
sah, und siehe, das Gefilde war eine Wüste; und alle Städte darin waren
zerbrochen vor dem HErrn und vor seinem grimmigen Zorn. \bibverse{27}
Denn so spricht der HErr: Das ganze Land soll wüst werden, und ich
will's doch nicht gar aus machen. \footnote{\textbf{4:27} Jer 5,10; Jer
  5,18} \bibverse{28} Darum wird das Land betrübt und der Himmel droben
traurig sein; denn ich habe es geredet, ich habe es beschlossen, und es
soll mich nicht reuen, will auch nicht davon ablassen.

\bibverse{29} Aus allen Städten werden sie vor dem Geschrei der Reiter
und Schützen fliehen und in die dicken Wälder laufen und in die Felsen
kriechen; alle Städte werden verlassen stehen, dass niemand darin wohnt.
\bibverse{30} Was willst du alsdann tun, du Verstörte? Wenn du dich
schon mit Purpur kleiden und mit goldenen Kleinoden schmücken und dein
Angesicht schminken würdest, so schmückst du dich doch vergeblich; die
Buhlen werden dich verachten, sie werden dir nach dem Leben trachten.
\bibverse{31} Denn ich höre ein Geschrei als einer Gebärerin, eine Angst
als einer, die in den ersten Kindesnöten ist, ein Geschrei der Tochter
Zion, die da klagt und die Hände auswirft: „Ach, wehe mir! Ich muss
schier vergehen vor den Würgern.`` \# 5 \bibverse{1} Gehet durch die
Gassen zu Jerusalem und schauet und erfahret und suchet auf ihrer
Straße, ob ihr jemand findet, der recht tue und nach dem Glauben frage,
so will ich dir gnädig sein. \bibverse{2} Und wenn sie schon sprechen:
„Bei dem lebendigen Gott!{}``, so schwören sie doch falsch.

\bibverse{3} HErr, deine Augen sehen nach dem Glauben. Du schlägst sie,
aber sie fühlen's nicht; du machst es schier aus mit ihnen, aber sie
bessern sich nicht. Sie haben ein härter Angesicht denn ein Fels und
wollen sich nicht bekehren.

\bibverse{4} Ich dachte aber: Wohlan, der arme Haufe ist unverständig,
weiß nichts um des HErrn Weg und um ihres Gottes Recht. \bibverse{5} Ich
will zu den Gewaltigen gehen und mit ihnen reden; die werden um des
HErrn Weg und ihres Gottes Recht wissen. -- Aber sie allesamt hatten das
Joch zerbrochen und die Seile zerrissen. \footnote{\textbf{5:5} Jer 2,20}
\bibverse{6} Darum wird sie auch der Löwe, der aus dem Walde kommt,
zerreißen, und der Wolf aus der Wüste wird sie verderben, und der Parder
wird um ihre Städte lauern; alle, die daselbst herausgehen, wird er
fressen. Denn ihrer Sünden sind zuviel, und sie bleiben verstockt in
ihrem Ungehorsam. \footnote{\textbf{5:6} 3Mo 26,22}

\bibverse{7} Wie soll ich dir denn gnädig sein, weil mich deine Kinder
verlassen und schwören bei dem, der nicht Gott ist? und nun ich ihnen
vollauf gegeben habe, treiben sie Ehebruch und laufen ins Hurenhaus.
\bibverse{8} Ein jeglicher wiehert nach seines Nächsten Weibe wie die
vollen, müßigen Hengste. \bibverse{9} Und ich sollte sie um solches
nicht heimsuchen? spricht der HErr, und meine Seele sollte sich nicht
rächen an solchem Volk, wie dies ist? \footnote{\textbf{5:9} Jer 5,29}

\bibverse{10} Stürmet ihre Mauern und werfet sie um, und macht's nicht
gar aus! Führet ihre Reben weg, denn sie sind nicht des HErrn;
\footnote{\textbf{5:10} Jer 4,27} \bibverse{11} sondern sie verachten
mich, beide, das Haus Israel und das Haus Juda, spricht der HErr.

\bibverse{12} Sie verleugnen den HErrn und sprechen: „Das ist er nicht,
und so übel wird es uns nicht gehen; Schwert und Hunger werden wir nicht
sehen. \bibverse{13} Ja, die Propheten sind Schwätzer und haben auch
Gottes Wort nicht; es gehe über sie selbst also!{}``

\bibverse{14} Darum spricht der HErr, der Gott Zebaoth: Weil ihr solche
Rede treibt, siehe, so will ich meine Worte in deinem Munde zu Feuer
machen und dieses Volk zu Holz, und es soll sie verzehren. \bibverse{15}
Siehe, ich will über euch vom Hause Israel, spricht der HErr, ein Volk
von ferne bringen, ein mächtiges Volk, ein Volk von alters her, ein
Volk, dessen Sprache du nicht verstehst, und kannst nicht vernehmen, was
sie reden. \footnote{\textbf{5:15} Jer 6,22} \bibverse{16} Seine Köcher
sind offene Gräber; es sind eitel Helden. \bibverse{17} Sie werden deine
Ernte und dein Brot verzehren; sie werden deine Söhne und Töchter
fressen; sie werden deine Schafe und Rinder verschlingen; sie werden
deine Weinstöcke und Feigenbäume verzehren; deine festen Städte, darauf
du dich verlässest, werden sie mit dem Schwert verderben.

\bibverse{18} Doch will ich's, spricht der HErr, zur selben Zeit mit
euch nicht gar aus machen. \bibverse{19} Und ob sie würden sagen: „Warum
tut uns der HErr, unser Gott, solches alles?{}``, sollst du ihnen
antworten: Wie ihr mich verlasst und fremden Göttern dient in eurem
eigenen Lande, also sollt ihr auch Fremden dienen in einem Lande, das
nicht euer ist.

\bibverse{20} Solches sollt ihr verkündigen im Hause Jakob und predigen
in Juda und sprechen: \bibverse{21} Höret zu, ihr tolles Volk, das
keinen Verstand hat, die da Augen haben, und sehen nicht, Ohren haben,
und hören nicht! \bibverse{22} Wollt ihr mich nicht fürchten? spricht
der HErr, und vor mir nicht erschrecken, der ich dem Meer den Sand zum
Ufer setze, darin es allezeit bleiben muss, darüber es nicht gehen darf?
Und ob's schon wallet, so vermag's doch nichts; und ob seine Wellen
schon toben, so dürfen sie doch nicht darüberfahren. \footnote{\textbf{5:22}
  Hi 38,8-11}

\bibverse{23} Aber dieses Volk hat ein abtrünniges, ungehorsames Herz;
sie bleiben abtrünnig und gehen immerfort weg \bibverse{24} und sprechen
nicht einmal in ihrem Herzen: Lasset uns doch den HErrn, unseren Gott,
fürchten, der uns Frühregen und Spätregen zu rechter Zeit gibt und uns
die Ernte treulich und jährlich behütet.

\bibverse{25} Aber eure Missetaten hindern solches, und eure Sünden
wenden das Gute von euch. \footnote{\textbf{5:25} Jes 59,2}
\bibverse{26} Denn man findet unter meinem Volk Gottlose, die den Leuten
nachstellen und Fallen zurichten, sie zu fangen, wie die Vogler tun.
\bibverse{27} Und ihre Häuser sind voller Tücke, wie ein Vogelbauer
voller Lockvögel ist. Daher werden sie gewaltig und reich, fett und
glatt. \bibverse{28} Sie gehen mit bösen Stücken um; sie halten kein
Recht, der Waisen Sache fördern sie nicht, dass auch sie Glück hätten,
und helfen den Armen nicht zum Recht.

\bibverse{29} Sollte ich denn solches nicht heimsuchen, spricht der
HErr, und meine Seele sollte sich nicht rächen an solchem Volk, wie dies
ist? \footnote{\textbf{5:29} Jer 5,9}

\bibverse{30} Es steht gräulich und schrecklich im Lande. \bibverse{31}
Die Propheten weissagen falsch, und die Priester herrschen in ihrem Amt,
und mein Volk hat's gern also. Wie will es euch zuletzt darob gehen? \#
6 \bibverse{1} Fliehet, ihr Kinder Benjamin, aus Jerusalem und blaset
die Drommete auf der Warte Thekoa und werft auf ein Panier über der
Warte Beth-Cherem! denn es geht daher ein Unglück von Mitternacht und
ein großer Jammer. \bibverse{2} Die Tochter Zion ist wie eine schöne und
lustige Aue. \bibverse{3} Aber es werden Hirten über sie kommen mit
ihren Herden, die werden Gezelte rings um sie her aufschlagen und weiden
ein jeglicher an seinem Ort (und sprechen): \footnote{\textbf{6:3} Jer
  4,17}

\bibverse{4} „Rüstet euch zum Krieg wider sie! Wohlauf, lasst uns
hinaufziehen, weil es noch hoch Tag ist! Ei, es will Abend werden, und
die Schatten werden groß! \bibverse{5} Wohlan, so lasst uns auf sein,
und sollten wir bei Nacht hinaufziehen und ihre Paläste verderben!{}``
\bibverse{6} Denn also spricht der HErr Zebaoth: Fället Bäume und werfet
einen Wall auf wider Jerusalem; denn sie ist eine Stadt, die heimgesucht
werden soll. Ist doch eitel Unrecht darin. \bibverse{7} Denn gleichwie
ein Born sein Wasser quillt, also quillt auch ihre Bosheit. Ihr Frevel
und Gewalt schreit über sie, und ihr Morden und Schlagen treiben sie
täglich vor mir. \bibverse{8} Bessere dich, Jerusalem, ehe sich mein
Herz von dir wende und ich dich zum wüsten Lande mache, darin niemand
wohne!

\bibverse{9} So spricht der HErr Zebaoth: Was übriggeblieben ist von
Israel, das muss nachgelesen werden wie am Weinstock. Der Weinleser wird
eins nach dem anderen in die Butten werfen.

\bibverse{10} Ach, mit wem soll ich doch reden und zeugen? Dass doch
jemand hören wollte! Aber ihre Ohren sind unbeschnitten; sie können's
nicht hören. Siehe, sie halten des HErrn Wort für einen Spott und wollen
es nicht. \bibverse{11} Darum bin ich von des HErrn Drohen so voll, dass
ich's nicht lassen kann. Schütte es aus über die Kinder auf der Gasse
und über die Mannschaft im Rat miteinander; denn es sollen beide, Mann
und Weib, Alte und der Wohlbetagte, gefangen werden. \bibverse{12} Ihre
Häuser sollen den Fremden zuteil werden samt den Äckern und Weibern;
denn ich will meine Hand ausstrecken, spricht der HErr, über des Landes
Einwohner. \bibverse{13} Denn sie geizen allesamt, klein und groß; und
beide, Propheten und Priester, gehen allesamt mit Lügen um \footnote{\textbf{6:13}
  Jer 8,10-12} \bibverse{14} und trösten mein Volk in seinem Unglück,
dass sie es gering achten sollen, und sagen: „Friede! Friede!{}``, und
ist doch nicht Friede. \footnote{\textbf{6:14} Hes 13,10; Hes 13,16;
  1Thes 5,3} \bibverse{15} Darum werden sie mit Schanden bestehen, dass
sie solche Gräuel treiben; wiewohl sie wollen ungeschändet sein und
wollen sich nicht schämen. Darum müssen sie fallen auf einen Haufen; und
wenn ich sie heimsuchen werde, sollen sie stürzen, spricht der HErr.

\bibverse{16} So spricht der HErr: Tretet auf die Wege und schauet und
fraget nach den vorigen Wegen, welches der gute Weg sei, und wandelt
darin, so werdet ihr Ruhe finden für eure Seele! Aber sie sprechen: Wir
wollen's nicht tun! \footnote{\textbf{6:16} Mt 11,29; Jer 44,16}
\bibverse{17} Ich habe Wächter über dich gesetzt: Merket auf die Stimme
der Drommete! Aber sie sprechen: Wir wollen's nicht tun! \footnote{\textbf{6:17}
  Jes 52,8; Hes 3,17} \bibverse{18} Darum so höret, ihr Heiden, und
merket samt euren Leuten! \bibverse{19} Du, Erde, höre zu! Siehe, ich
will ein Unglück über dieses Volk bringen, ihren verdienten Lohn, darum
dass sie auf meine Worte nicht achten und mein Gesetz verwerfen.
\footnote{\textbf{6:19} 5Mo 32,1; Jes 1,2} \bibverse{20} Was frage ich
nach dem Weihrauch aus Reicharabien und nach den guten Zimtrinden, die
aus fernen Landen kommen? Eure Brandopfer sind mir nicht angenehm, und
eure Opfer gefallen mir nicht. \footnote{\textbf{6:20} Jes 1,11}

\bibverse{21} Darum spricht der HErr also: Siehe, ich will diesem Volk
einen Anstoß in den Weg stellen, daran sich Väter und Kinder miteinander
stoßen und ein Nachbar mit dem anderen umkommen sollen. \bibverse{22} So
spricht der HErr: Siehe, es wird ein Volk kommen von Mitternacht, und
ein großes Volk wird sich erregen vom Ende der Erde, \footnote{\textbf{6:22}
  Jer 5,15; 5Mo 28,49} \bibverse{23} die Bogen und Lanze führen. Es ist
grausam und ohne Barmherzigkeit; sie brausen daher wie ein ungestümes
Meer und reiten auf Rossen, gerüstet wie Kriegsleute, wider dich, du
Tochter Zion. \footnote{\textbf{6:23} Jer 50,42}

\bibverse{24} Wenn wir von ihnen hören werden, so werden uns die Fäuste
entsinken; es wird uns angst und weh werden wie einer Gebärerin.
\bibverse{25} Es gehe ja niemand hinaus auf den Acker, niemand gehe über
Feld; denn es ist allenthalben unsicher vor dem Schwert des Feindes.
\bibverse{26} O Tochter meines Volks, zieh Säcke an und lege dich in die
Asche; trage Leid wie um einen einzigen Sohn und klage wie die, die hoch
betrübt sind! denn der Verderber kommt über uns plötzlich. \footnote{\textbf{6:26}
  Am 8,10}

\bibverse{27} Ich habe dich zum Schmelzer gesetzt unter mein Volk, das
so hart ist, dass du ihr Wesen erfahren und prüfen sollst. \bibverse{28}
Sie sind allzumal Abtrünnige und wandeln verräterisch, sind Erz und
Eisen; alle sind sie verderbt. \bibverse{29} Der Blasebalg ist
verbrannt, das Blei verschwindet; das Schmelzen ist umsonst, denn das
Böse ist nicht davon geschieden. \bibverse{30} Darum heißen sie auch ein
verworfenes Silber; denn der HErr hat sie verworfen. \footnote{\textbf{6:30}
  Jes 1,22}

\hypertarget{section-1}{%
\section{7}\label{section-1}}

\bibverse{1} Dies ist das Wort, welches geschah zu Jeremia vom HErrn,
und sprach: \bibverse{2} Tritt ins Tor im Hause des HErrn und predige
daselbst dieses Wort und sprich: Höret des HErrn Wort, ihr alle von
Juda, die ihr zu diesen Toren eingehet, den HErrn anzubeten!

\bibverse{3} So spricht der HErr Zebaoth, der Gott Israels: Bessert euer
Leben und Wesen, so will ich bei euch wohnen an diesem Ort. \bibverse{4}
Verlasst euch nicht auf die Lügen, wenn sie sagen: Hier ist des HErrn
Tempel, hier ist des HErrn Tempel, hier ist des HErrn Tempel!
\bibverse{5} sondern bessert euer Leben und Wesen, dass ihr recht tut
einer gegen den anderen \bibverse{6} und den Fremdlingen, Waisen und
Witwen keine Gewalt tut und nicht unschuldiges Blut vergießt an diesem
Ort, und folgt nicht nach anderen Göttern zu eurem eigenen Schaden:
\footnote{\textbf{7:6} 2Mo 22,20-21} \bibverse{7} so will ich immer und
ewiglich bei euch wohnen an diesem Ort, in dem Lande, das ich euren
Vätern gegeben habe. \bibverse{8} Aber nun verlasset ihr euch auf Lügen,
die nichts nütze sind. \bibverse{9} Daneben seid ihr Diebe, Mörder,
Ehebrecher und Meineidige und räuchert dem Baal und folgt fremden
Göttern nach, die ihr nicht kennt. \bibverse{10} Darnach kommt ihr dann
und tretet vor mich in diesem Hause, das nach meinem Namen genannt ist,
und sprecht: Es hat keine Not mit uns, weil wir solche Gräuel tun.
\bibverse{11} Haltet ihr denn dieses Haus, das nach meinem Namen genannt
ist, für eine Mördergrube? Siehe, ich sehe es wohl, spricht der HErr.
\footnote{\textbf{7:11} Mt 21,13}

\bibverse{12} Gehet hin an meinen Ort zu Silo, da vormals mein Name
gewohnt hat, und schauet, was ich daselbst getan habe um der Bosheit
willen meines Volks Israel. \footnote{\textbf{7:12} Joh 18,1; 1Sam 4,12;
  Ps 78,60} \bibverse{13} Weil ihr denn alle solche Stücke treibt,
spricht der HErr, und ich stets euch predigen lasse, und ihr wollt nicht
hören, ich rufe euch, und ihr wollt nicht antworten: \footnote{\textbf{7:13}
  Spr 1,24; Jes 65,12} \bibverse{14} so will ich dem Hause, das nach
meinem Namen genannt ist, darauf ihr euch verlasset, und dem Ort, den
ich euren Vätern gegeben habe, eben tun, wie ich Silo getan habe,
\footnote{\textbf{7:14} Jer 26,6} \bibverse{15} und will euch von meinem
Angesicht wegwerfen, wie ich weggeworfen habe alle eure Brüder, den
ganzen Samen Ephraims. \footnote{\textbf{7:15} 2Kö 17,18; 2Kö 17,20; 2Kö
  17,23}

\bibverse{16} Und du sollst für dieses Volk nicht bitten und sollst für
sie keine Klage noch Gebet vorbringen, auch nicht sie vertreten vor mir;
denn ich will dich nicht hören. \footnote{\textbf{7:16} Jer 11,14; Jer
  14,11} \bibverse{17} Denn siehst du nicht, was sie tun in den Städten
Judas und auf den Gassen zu Jerusalem? \bibverse{18} Die Kinder lesen
Holz, so zünden die Väter das Feuer an, und die Weiber kneten den Teig,
dass sie der Himmelskönigin Kuchen backen, und geben Trankopfer den
fremden Göttern, dass sie mir Verdruss tun. \footnote{\textbf{7:18} Jer
  44,17} \bibverse{19} Aber sie sollen nicht mir damit, spricht der
HErr, sondern sich selbst Verdruss tun und müssen zu Schanden werden.

\bibverse{20} Darum spricht der Herr HErr: Siehe, mein Zorn und mein
Grimm ist ausgeschüttet über diesen Ort, über Menschen und Vieh, über
Bäume auf dem Felde und über die Früchte des Landes; und der soll
brennen, dass niemand löschen kann.

\bibverse{21} So spricht der HErr Zebaoth, der Gott Israels: Tut eure
Brandopfer und anderen Opfer zuhauf und esset Fleisch. \bibverse{22}
Denn ich habe euren Vätern des Tages, da ich sie aus Ägyptenland führte,
weder gesagt noch geboten von Brandopfern und anderen Opfern;
\footnote{\textbf{7:22} Mi 6,6-8; 1Sam 15,22} \bibverse{23} sondern dies
gebot ich ihnen und sprach: Gehorchet meinem Wort, so will ich euer Gott
sein, und ihr sollt mein Volk sein; und wandelt auf allen Wegen, die ich
euch gebiete, auf dass es euch wohl gehe. \footnote{\textbf{7:23} 2Mo
  19,5} \bibverse{24} Aber sie wollen nicht hören noch ihre Ohren
zuneigen, sondern wandelten nach ihrem eigenen Rat und nach ihres bösen
Herzens Gedünken und gingen hinter sich und nicht vor sich. \footnote{\textbf{7:24}
  Jer 11,8; Jes 65,2} \bibverse{25} Ja, von dem Tage an, da ich eure
Väter aus Ägyptenland geführt habe, bis auf diesen Tag habe ich stets zu
euch gesandt alle meine Knechte, die Propheten. \bibverse{26} Aber sie
wollen mich nicht hören noch ihre Ohren neigen, sondern waren
halsstarrig und machten's ärger denn ihre Väter.

\bibverse{27} Und wenn du ihnen dies alles schon sagst, so werden sie
dich doch nicht hören; rufst du ihnen, so werden sie dir nicht
antworten. \bibverse{28} Darum sprich zu ihnen: Dies ist das Volk, das
den HErrn, seinen Gott, nicht hören noch sich bessern will. Der Glaube
ist untergegangen und ausgerottet von ihrem Munde. \footnote{\textbf{7:28}
  Jer 5,1} \bibverse{29} Schneide deine Haare ab und wirf sie von dir
und wehklage auf den Höhen; denn der HErr hat dies Geschlecht, über das
er zornig ist, verworfen und verstoßen.

\bibverse{30} Denn die Kinder Juda tun übel vor meinen Augen, spricht
der HErr. Sie setzen ihre Gräuel in das Haus, das nach meinem Namen
genannt ist, dass sie es verunreinigen, \bibverse{31} und bauen die
Altäre des Thopheth im Tal Ben-Hinnom, dass sie ihre Söhne und Töchter
verbrennen, was ich nie geboten noch in den Sinn genommen habe.
\footnote{\textbf{7:31} 2Kö 23,10; 3Mo 18,21} \bibverse{32} Darum siehe,
es kommt die Zeit, spricht der HErr, dass man's nicht mehr heißen soll
Thopheth und das Tal Ben-Hinnom, sondern Würgetal; und man wird im
Thopheth müssen begraben, weil sonst kein Raum mehr sein wird.
\footnote{\textbf{7:32} Jer 19,6} \bibverse{33} Und die Leichname dieses
Volks sollen den Vögeln des Himmels und den Tieren auf Erden zur Speise
werden, davon sie niemand scheuchen wird. \footnote{\textbf{7:33} Jer
  19,7; Jer 9,21} \bibverse{34} Und ich will in den Städten Judas und
auf den Gassen zu Jerusalem wegnehmen das Geschrei der Freude und Wonne
und die Stimme des Bräutigams und der Braut; denn das Land soll wüst
sein. \footnote{\textbf{7:34} Jer 16,9}

\hypertarget{section-2}{%
\section{8}\label{section-2}}

\bibverse{1} Zu derselben Zeit, spricht der HErr, wird man die Gebeine
der Könige Judas, die Gebeine ihrer Fürsten, die Gebeine der Priester,
die Gebeine der Propheten, die Gebeine der Bürger zu Jerusalem aus ihren
Gräbern werfen; \bibverse{2} und wird sie hinstreuen unter Sonne, Mond
und alles Heer des Himmels, welche sie geliebt und denen sie gedient
haben, denen sie nachgefolgt sind und die sie gesucht und angebetet
haben. Sie sollen nicht wieder aufgelesen und begraben werden, sondern
Kot auf der Erde sein. \footnote{\textbf{8:2} 5Mo 4,19; Jer 14,16}
\bibverse{3} Und alle Übrigen von diesem bösen Volk, an welchem Ort sie
sein werden, dahin ich sie verstoßen habe, werden lieber tot denn
lebendig sein wollen, spricht der HErr Zebaoth. \bibverse{4} Darum
sprich zu ihnen: So spricht der HErr: Wo ist jemand, so er fällt, der
nicht gerne wieder aufstünde? Wo ist jemand, wenn er irregeht, der nicht
gerne wieder zurechtkäme? \bibverse{5} Dennoch will ja dieses Volk zu
Jerusalem irregehen für und für. Sie halten so hart an dem falschen
Gottesdienst, dass sie sich nicht wollen abwenden lassen. \bibverse{6}
Ich sehe und höre, dass sie nichts Rechtes reden. Keiner ist, dem seine
Bosheit leid wäre und der spräche: Was mache ich doch! Sie laufen alle
ihren Lauf wie ein grimmiger Hengst im Streit. \bibverse{7} Ein Storch
unter dem Himmel weiß seine Zeit, eine Turteltaube, Kranich und Schwalbe
merken ihre Zeit, wann sie wiederkommen sollen, aber mein Volk will das
Recht des HErrn nicht wissen. \bibverse{8} Wie mögt ihr doch sagen: „Wir
wissen, was recht ist, und haben die heilige Schrift vor uns``? Ist's
doch eitel Lüge, was die Schriftgelehrten setzen. \bibverse{9} Darum
müssen solche Lehrer zu Schanden, erschreckt und gefangen werden; denn
was können sie Gutes lehren, weil sie des HErrn Wort verwerfen?
\bibverse{10} Darum will ich ihre Weiber den Fremden geben und ihre
Äcker denen, die sie verjagen werden. Denn sie geizen allesamt, beide,
klein und groß; und beide, Priester und Propheten, gehen mit Lügen um
\footnote{\textbf{8:10} Jer 6,13-15; Jes 56,11} \bibverse{11} und
trösten mein Volk in ihrem Unglück, dass sie es gering achten sollen,
und sagen: „Friede! Friede!{}``, und ist doch nicht Friede.
\bibverse{12} Darum werden sie mit Schanden bestehen, dass sie solche
Gräuel treiben; wiewohl sie wollen ungeschändet sein und wollen sich
nicht schämen. Darum müssen sie fallen auf einen Haufen; und wenn ich
sie heimsuchen werde, sollen sie stürzen, spricht der HErr.
\bibverse{13} Ich will sie also ablesen, spricht der HErr, dass keine
Trauben am Weinstock und keine Feigen am Feigenbaum bleiben, ja auch die
Blätter wegfallen sollen; und was ich ihnen gegeben habe, das soll ihnen
genommen werden. \bibverse{14} Wo werden wir dann wohnen? Ja, sammelt
euch dann und lasst uns in die festen Städte ziehen, dass wir daselbst
umkommen. Denn der HErr, unser Gott, wird uns umkommen lassen und
tränken mit einem bitteren Trunk, dass wir so gesündigt haben wider den
HErrn. \bibverse{15} Wir hofften, es sollte Friede werden, so kommt
nichts Gutes; wir hofften, wir sollten heil werden, aber siehe, so ist
mehr Schaden da. \footnote{\textbf{8:15} Jer 14,19} \bibverse{16} Man
hört ihre Rosse schnauben von Dan her; vom Wiehern ihrer Gäule erbebt
das ganze Land. Und sie fahren daher und werden das Land auffressen mit
allem, was darin ist, die Stadt samt allen, die darin wohnen.
\bibverse{17} Denn siehe, ich will Schlangen und Basilisken unter euch
senden, die nicht zu beschwören sind; die sollen euch stechen, spricht
der HErr. \bibverse{18} Was mag mich in meinem Jammer erquicken? Mein
Herz in mir ist krank. \footnote{\textbf{8:18} Jer 4,19} \bibverse{19}
Siehe, die Tochter meines Volks wird schreien aus fernem Lande her:
„Will denn der HErr nicht mehr Gott sein zu Zion, oder soll sie keinen
König mehr haben?{}`` Ja, warum haben sie mich so erzürnt durch ihre
Bilder und fremde, unnütze Gottesdienste? \bibverse{20} „Die Ernte ist
vergangen, der Sommer ist dahin, und uns ist keine Hilfe gekommen.``
\bibverse{21} Mich jammert herzlich, dass mein Volk so verderbt ist; ich
gräme mich und gehabe mich übel. \bibverse{22} Ist denn keine Salbe in
Gilead, oder ist kein Arzt da? Warum ist denn die Tochter meines Volks
nicht geheilt? \bibverse{23} Ach, dass ich Wasser genug hätte in meinem
Haupte und meine Augen Tränenquellen wären, dass ich Tag und Nacht
beweinen möchte die Erschlagenen in meinem Volk! \footnote{\textbf{8:23}
  Jer 13,17; Kla 1,16}

\hypertarget{section-3}{%
\section{9}\label{section-3}}

\bibverse{1} Ach dass ich eine Herberge hätte in der Wüste, so wollte
ich mein Volk verlassen und von ihnen ziehen! Denn es sind eitel
Ehebrecher und ein frecher Haufe. \bibverse{2} Sie schießen mit ihren
Zungen eitel Lüge und keine Wahrheit und treiben's mit Gewalt im Lande
und gehen von einer Bosheit zur anderen und achten mich nicht, spricht
der HErr. \bibverse{3} Ein jeglicher hüte sich vor seinem Freunde und
traue auch seinem Bruder nicht; denn ein Bruder unterdrückt den anderen,
und ein Freund verrät den anderen. \bibverse{4} Ein Freund täuscht den
anderen und reden kein wahres Wort; sie fleißigen sich darauf, wie einer
den anderen betrüge, und ist ihnen leid, dass sie es nicht ärger machen
können. \bibverse{5} Es ist allenthalben eitel Trügerei unter ihnen, und
vor Trügerei wollen sie mich nicht kennen, spricht der HErr.
\bibverse{6} Darum spricht der HErr Zebaoth also: Siehe, ich will sie
schmelzen und prüfen. Denn was soll ich sonst tun, wenn ich ansehe die
Tochter meines Volks?

\bibverse{7} Ihre falschen Zungen sind mörderische Pfeile; mit ihrem
Munde reden sie freundlich gegen den Nächsten, aber im Herzen lauern sie
auf ihn. \bibverse{8} Sollte ich nun solches nicht heimsuchen an ihnen,
spricht der HErr, und meine Seele sollte sich nicht rächen an solchem
Volk, wie dies ist? \footnote{\textbf{9:8} Jer 5,9} \bibverse{9} Ich
muss auf den Bergen weinen und heulen und bei den Hürden in der Wüste
klagen; denn sie sind so gar verheert, dass niemand da wandelt und man
auch nicht ein Vieh schreien hört. Es ist beides, Vögel des Himmels und
das Vieh, alles weg. \footnote{\textbf{9:9} Jer 4,25; Jer 12,4}
\bibverse{10} Und ich will Jerusalem zum Steinhaufen und zur Wohnung der
Schakale machen und will die Städte Judas wüst machen, dass niemand
darin wohnen soll. \footnote{\textbf{9:10} Jer 26,18} \bibverse{11} Wer
nun weise wäre und ließe es sich zu Herzen gehen und verkündigte, was
des HErrn Mund zu ihm sagt, warum das Land verderbt und verheert wird
wie eine Wüste, da niemand wandelt! \footnote{\textbf{9:11} 5Mo 32,29}

\bibverse{12} Und der HErr sprach: Darum dass sie mein Gesetz verlassen,
dass ich ihnen vorgelegt habe, und gehorchen meiner Rede nicht, leben
auch nicht darnach,

\bibverse{13} sondern folgen ihres Herzens Gedünken und den Baalim, wie
sie ihre Väter gelehrt haben: \footnote{\textbf{9:13} Jer 7,24}
\bibverse{14} darum spricht der HErr Zebaoth, der Gott Israels, also:
Siehe, ich will dieses Volk mit Wermut speisen und mit Galle tränken;
\footnote{\textbf{9:14} Jer 23,15} \bibverse{15} ich will sie unter die
Heiden zerstreuen, welche weder sie noch ihre Väter gekannt haben, und
will das Schwert hinter sie schicken, bis dass es aus mit ihnen sei.
\footnote{\textbf{9:15} 3Mo 26,33} \bibverse{16} So spricht der HErr
Zebaoth: Schaffet und bestellet Klageweiber, dass sie kommen, und
schickt nach denen, die es wohl können,

\bibverse{17} dass sie eilend um uns klagen, dass unsere Augen von
Tränen rinnen und unsere Augenlider von Wasser fließen, \bibverse{18}
dass man ein klägliches Geschrei höre zu Zion: Ach, wie sind wir so gar
verstört und zu Schanden geworden! Wir müssen das Land räumen; denn sie
haben unsere Wohnungen geschleift. \bibverse{19} So höret nun, ihr
Weiber, des HErrn Wort und nehmet zu Ohren seines Mundes Rede; lehret
eure Töchter weinen, und eine lehre die andere klagen: \bibverse{20} Der
Tod ist zu unseren Fenstern eingefallen und in unsere Paläste gekommen,
die Kinder zu würgen auf der Gasse und die Jünglinge auf der Straße.
\bibverse{21} So spricht der HErr: Sage: Der Menschen Leichname sollen
liegen wie der Mist auf dem Felde und wie Garben hinter dem Schnitter,
die niemand sammelt.

\bibverse{22} So spricht der HErr: Ein Weiser rühme sich nicht seiner
Weisheit, ein Starker rühme sich nicht seiner Stärke, ein Reicher rühme
sich nicht seines Reichtums;

\bibverse{23} sondern wer sich rühmen will, der rühme sich des, dass er
mich wisse und kenne, dass ich der HErr bin, der Barmherzigkeit, Recht
und Gerechtigkeit übt auf Erden; denn solches gefällt mir, spricht der
HErr. \footnote{\textbf{9:23} 1Kor 1,31; 2Kor 10,17} \bibverse{24}
Siehe, es kommt die Zeit, spricht der HErr, dass ich heimsuchen werde
alle, die Beschnittenen mit den Unbeschnittenen:

\bibverse{25} Ägypten, Juda, Edom, die Kinder Ammon, Moab und alle, die
das Haar rundumher abschneiden, die in der Wüste wohnen. Denn alle
Heiden haben unbeschnittene Vorhaut; aber das ganze Israel hat ein
unbeschnittenes Herz. \# 10 \bibverse{1} Höret, was der HErr zu euch vom
Hause Israel redet. \bibverse{2} So spricht der HErr: Ihr sollt nicht
der Heiden Weise lernen und sollt euch nicht fürchten vor den Zeichen
des Himmels, wie die Heiden sich fürchten. \bibverse{3} Denn der Heiden
Satzungen sind lauter Nichts. Denn sie hauen im Walde einen Baum, und
der Werkmeister macht Götter mit dem Beil \footnote{\textbf{10:3} Jes
  44,10-20} \bibverse{4} und schmückt sie mit Silber und Gold und heftet
sie mit Nägeln und Hämmern, dass sie nicht umfallen. \bibverse{5} Es
sind ja nichts als überzogene Säulen. Sie können nicht reden; so muss
man sie auch tragen, denn sie können nicht gehen. Darum sollt ihr euch
nicht vor ihnen fürchten: denn sie können weder helfen noch Schaden tun.
\bibverse{6} Aber dir, HErr, ist niemand gleich; du bist groß, und dein
Name ist groß, und kannst es mit der Tat beweisen. \bibverse{7} Wer
sollte dich nicht fürchten, du König der Heiden? Dir sollte man
gehorchen; denn es ist unter allen Weisen der Heiden und in allen
Königreichen deinesgleichen nicht. \bibverse{8} Sie sind allzumal Narren
und Toren; denn ein Holz muss ja ein nichtiger Gottesdienst sein.
\bibverse{9} Silbernes Blech bringt man aus Tharsis, Gold aus Uphas,
durch den Meister und Goldschmied zugerichtet; blauen und roten Purpur
zieht man ihm an, und ist alles der Weisen Werk. \bibverse{10} Aber der
HErr ist ein rechter Gott, ein lebendiger Gott, ein ewiger König. Vor
seinem Zorn bebt die Erde, und die Heiden können sein Drohen nicht
ertragen.

\bibverse{11} So sprecht nun zu ihnen also: Die Götter, die den Himmel
und Erde nicht gemacht haben, müssen vertilgt werden von der Erde und
unter dem Himmel. \bibverse{12} Er hat aber die Erde durch seine Kraft
gemacht und den Weltkreis bereitet durch seine Weisheit und den Himmel
ausgebreitet durch seinen Verstand. \bibverse{13} Wenn er donnert, so
ist des Wassers die Menge unter dem Himmel, und er zieht die Nebel auf
vom Ende der Erde; er macht die Blitze im Regen und lässt den Wind
kommen aus seinen Vorratskammern. \footnote{\textbf{10:13} Ps 135,7; Hi
  38,24-30} \bibverse{14} Alle Menschen sind Narren mit ihrer Kunst, und
alle Goldschmiede bestehen mit Schanden mit ihren Bildern; denn ihre
Götzen sind Trügerei und haben kein Leben. \bibverse{15} Es ist eitel
Nichts und ein verführerisches Werk; sie müssen umkommen, wenn sie
heimgesucht werden. \bibverse{16} Aber also ist der nicht, der Jakobs
Schatz ist; sondern er ist's, der alles geschaffen hat, und Israel ist
sein Erbteil. Er heißt HErr Zebaoth. \bibverse{17} Tue deinen Kram weg
aus dem Lande, die du wohnest in der Feste. \bibverse{18} Denn so
spricht der HErr: Siehe, ich will die Einwohner des Landes auf diesmal
wegschleudern und will sie ängsten, dass sie es fühlen sollen.
\bibverse{19} Ach mein Jammer und mein Herzeleid! Ich denke aber: Es ist
meine Plage; ich muss sie leiden. \footnote{\textbf{10:19} Ps 77,11}
\bibverse{20} Meine Hütte ist zerstört, und alle meine Seile sind
zerrissen. Meine Kinder sind von mir gegangen und nicht mehr da. Niemand
ist, der meine Hütte wieder aufrichte und mein Gezelt aufschlage.
\bibverse{21} Denn die Hirten sind zu Narren geworden und fragen nach
dem HErrn nicht; darum können sie auch nichts Rechtes lehren, und ihre
ganze Herde ist zerstreut. \bibverse{22} Siehe, es kommt ein Geschrei
daher und ein großes Beben aus dem Lande von Mitternacht, dass die
Städte Judas verwüstet und zur Wohnung der Schakale werden sollen.
\bibverse{23} Ich weiß, HErr, dass des Menschen Tun steht nicht in
seiner Gewalt, und steht in niemands Macht, wie er wandle oder seinen
Gang richte. \bibverse{24} Züchtige mich, HErr, -- doch mit Maßen und
nicht in deinem Grimm, auf dass du mich nicht aufreibest. \footnote{\textbf{10:24}
  Jer 46,28; Ps 6,2; Hab 1,12} \bibverse{25} Schütte aber deinen Zorn
über die Heiden, so dich nicht kennen, und über die Geschlechter, so
deinen Namen nicht anrufen. Denn sie haben Jakob aufgefressen und
verschlungen; sie haben ihn weggeräumt und seine Wohnung verwüstet.
\footnote{\textbf{10:25} Ps 79,6}

\hypertarget{section-4}{%
\section{11}\label{section-4}}

\bibverse{1} Dies ist das Wort, das zu Jeremia geschah vom HErrn, und
sprach: \bibverse{2} Höret die Worte dieses Bundes, dass ihr sie denen
in Juda und den Bürgern zu Jerusalem saget. \bibverse{3} Und sprich zu
ihnen: So spricht der HErr, der Gott Israels: Verflucht sei, wer nicht
gehorcht den Worten dieses Bundes, \footnote{\textbf{11:3} 5Mo 27,26}
\bibverse{4} den ich euren Vätern gebot des Tages, da ich sie aus
Ägyptenland führte, aus dem eisernen Ofen, und sprach: Gehorchet meiner
Stimme und tut, wie ich euch geboten habe, so sollt ihr mein Volk sein,
und ich will euer Gott sein, \bibverse{5} auf dass ich den Eid halten
möge, den ich euren Vätern geschworen habe, ihnen zu geben ein Land,
darin Milch und Honig fließt, wie es denn heutigestages steht. Ich
antwortete und sprach: HErr, ja, es sei also!

\bibverse{6} Und der HErr sprach zu mir: Predige alle diese Worte in den
Städten Judas und auf den Gassen zu Jerusalem und sprich: Höret die
Worte dieses Bundes und tut darnach!

\bibverse{7} Denn ich habe euren Vätern gezeugt von dem Tage an, da ich
sie aus Ägyptenland führte, bis auf den heutigen Tag und zeugte stets
und sprach: Gehorchet meiner Stimme! \bibverse{8} Aber sie gehorchten
nicht, neigten auch ihre Ohren nicht; sondern ein jeglicher ging nach
seines bösen Herzens Gedünken. Darum habe ich auch über sie kommen
lassen alle Worte dieses Bundes, den ich geboten habe zu tun, und nach
dem sie doch nicht getan haben. \footnote{\textbf{11:8} Jer 7,24; Jer
  7,26}

\bibverse{9} Und der HErr sprach zu mir: Ich weiß wohl, wie sie in Juda
und zu Jerusalem sich rotten. \bibverse{10} Sie kehren sich eben zu den
Sünden ihrer Väter, die vormals waren, welche auch nicht gehorchen
wollten meinen Worten und folgten auch anderen Göttern nach und dienten
ihnen. Also hat das Haus Israel und das Haus Juda meinen Bund gebrochen,
den ich mit ihren Vätern gemacht habe. \bibverse{11} Darum siehe,
spricht der HErr, ich will ein Unglück über sie gehen lassen, dem sie
nicht sollen entgehen können; und wenn sie zu mir schreien, will ich sie
nicht hören. \bibverse{12} So lass denn die Städte Judas und die Bürger
zu Jerusalem hingehen und zu den Göttern schreien, denen sie geräuchert
haben; aber sie werden ihnen nicht helfen in ihrer Not. \footnote{\textbf{11:12}
  Jer 2,28; 5Mo 32,37-38} \bibverse{13} Denn so manche Stadt, so manche
Götter hast du, Juda; und so manche Gassen zu Jerusalem sind, so manchen
Schandaltar habt ihr aufgerichtet, dem Baal zu räuchern.

\bibverse{14} So bitte du nun nicht für dieses Volk und tue kein Flehen
noch Gebet für sie; denn ich will sie nicht hören, wenn sie zu mir
schreien in ihrer Not. \bibverse{15} Was haben meine Freunde in meinem
Hause zu schaffen? Sie treiben alle Schalkheit und meinen, das heilige
Fleisch soll es von ihnen nehmen; und wenn sie übeltun, sind sie guter
Dinge darüber. \bibverse{16} Der HErr nannte dich einen grünen, schönen,
fruchtbaren Ölbaum; aber nun hat er mit einem großen Mordgeschrei ein
Feuer um ihn lassen anzünden, dass seine Äste verderben müssen.
\bibverse{17} Denn der HErr Zebaoth, der dich gepflanzt hat, hat dir ein
Unglück gedroht um der Bosheit willen des Hauses Israel und des Hauses
Juda, welche sie treiben, dass sie mich erzürnen mit ihrem Räuchern, das
sie dem Baal tun.

\bibverse{18} Der HErr hat mir's offenbart, dass ich's weiß, und zeigte
mir ihr Vornehmen, \bibverse{19} nämlich, dass sie mich wie ein armes
Schaf zur Schlachtbank führen wollen. Denn ich wusste nicht, dass sie
wider mich beratschlagt hatten und gesagt: Lasst uns den Baum mit seinen
Früchten verderben und ihn aus dem Lande der Lebendigen ausrotten, dass
seines Namens nimmermehr gedacht werde. \footnote{\textbf{11:19} Jes
  53,7} \bibverse{20} Aber du, HErr Zebaoth, du gerechter Richter, der
du Nieren und Herzen prüfst, lass mich deine Rache über sie sehen; denn
ich habe dir meine Sache befohlen. \footnote{\textbf{11:20} Ps 7,10}

\bibverse{21} Darum spricht der HErr also wider die Männer zu Anathoth,
die dir nach deinem Leben stehen und sprechen: Weissage uns nicht im
Namen des HErrn, willst du anders nicht von unseren Händen sterben!
\footnote{\textbf{11:21} Jer 1,1} \bibverse{22} darum spricht der HErr
Zebaoth also: Siehe, ich will sie heimsuchen; ihre junge Mannschaft soll
mit dem Schwert getötet werden, und ihre Söhne und Töchter sollen
Hungers sterben, dass nichts von ihnen übrigbleibe; \bibverse{23} denn
ich will über die Männer zu Anathoth Unglück kommen lassen des Jahres,
wenn sie heimgesucht werden sollen. \# 12 \bibverse{1} HErr, wenn ich
gleich mit dir rechten wollte, so behältst du doch recht; dennoch muss
ich vom Recht mit dir reden. Warum geht's doch den Gottlosen so wohl und
die Verächter haben alles die Fülle? \bibverse{2} Du pflanzest sie, dass
sie wurzeln und wachsen und Frucht bringen. Nahe bist du in ihrem Munde,
aber ferne von ihrem Herzen; \bibverse{3} mich aber, HErr, kennst du und
siehst mich und prüfst mein Herz vor dir. Reiße sie weg wie Schafe, dass
sie geschlachtet werden; sondere sie aus, dass sie gewürgt werden.
\bibverse{4} Wie lange soll doch das Land so jämmerlich stehen und das
Gras auf dem Felde allenthalben verdorren um der Einwohner Bosheit
willen, dass beide, Vieh und Vögel, nimmer da sind? denn sie sprechen:
Ja, er weiß viel, wie es uns gehen wird. \footnote{\textbf{12:4} Jer 9,9}
\bibverse{5} Wenn dich die müde machen, die zu Fuße gehen, wie will
dir's gehen, wenn du mit den Reitern laufen sollst? Und wenn du in dem
Lande, da es Friede ist, Sicherheit suchst, was will mit dir werden bei
dem stolzen Jordan? \bibverse{6} Denn es verachten dich auch deine
Brüder und deines Vaters Haus und schreien zeter! über dich. Darum
vertraue du ihnen nicht, wenn sie gleich freundlich mit dir reden.
\bibverse{7} Ich habe mein Haus verlassen müssen und mein Erbe meiden,
und was meine Seele liebt, in der Feinde Hand geben. \bibverse{8} Mein
Erbe ist mir geworden wie ein Löwe im Walde und brüllt wider mich; darum
bin ich ihm gram geworden. \bibverse{9} Mein Erbe ist wie der sprenklige
Vogel, um welchen sich die Vögel sammeln. Wohlauf, und sammelt euch,
alle Feldtiere, kommet und fresset. \bibverse{10} Es haben Hirten, und
deren viel, meinen Weinberg verderbt und meinen Acker zertreten; sie
haben meinen schönen Acker zur Wüste gemacht, sie haben's öde gemacht.
\bibverse{11} Ich sehe bereits, wie es so jämmerlich verwüstet ist; ja
das ganze Land ist wüst. Aber es will's niemand zu Herzen nehmen.
\bibverse{12} Denn die Verstörer fahren daher über alle Hügel der Wüste,
und das fressende Schwert des HErrn von einem Ende des Landes bis zum
anderen; und kein Fleisch wird Frieden haben. \bibverse{13} Sie säen
Weizen, aber Disteln werden sie ernten; sie lassen's sich sauer werden,
aber sie werden's nicht genießen; sie werden ihres Einkommens nicht froh
werden vor dem grimmigen Zorn des HErrn. \bibverse{14} So spricht der
HErr wider alle meine bösen Nachbarn, die das Erbteil antasten, das ich
meinem Volk Israel ausgeteilt habe: Siehe, ich will sie aus ihrem Lande
ausreißen und das Haus Juda aus ihrer Mitte reißen. \bibverse{15} Und
wenn ich sie nun ausgerissen habe, will ich mich wiederum über sie
erbarmen und will einen jeglichen zu seinem Erbteil und in sein Land
wiederbringen. \bibverse{16} Und soll geschehen, wo sie von meinem Volk
lernen werden, dass sie schwören bei meinem Namen: „So wahr der HErr
lebt!{}``, wie sie zuvor mein Volk gelehrt haben schwören bei Baal, so
sollen sie unter meinem Volk erbaut werden. \footnote{\textbf{12:16} Jer
  4,2; 5Mo 6,13} \bibverse{17} Wo sie aber nicht hören wollen, so will
ich solches Volk ausreißen und umbringen, spricht der HErr. \# 13
\bibverse{1} So spricht der HErr zu mir: Gehe hin und kaufe dir einen
leinenen Gürtel und gürte damit deine Lenden und mache ihn nicht nass.

\bibverse{2} Und ich kaufte einen Gürtel nach dem Befehl des HErrn und
gürtete ihn um meine Lenden.

\bibverse{3} Da geschah des HErrn Wort zum andernmal zu mir und sprach:
\bibverse{4} Nimm den Gürtel, den du gekauft und um deine Lenden
gegürtet hast, und mache dich auf und gehe hin an den Euphrat und
verstecke ihn daselbst in einen Steinritz.

\bibverse{5} Ich ging hin und versteckte ihn am Euphrat, wie mir der
HErr geboten hatte.

\bibverse{6} Nach langer Zeit aber sprach der HErr zu mir: Mache dich
auf und gehe hin an den Euphrat und hole den Gürtel wieder, den ich dich
hieß daselbst verstecken.

\bibverse{7} Ich ging hin an den Euphrat und grub auf und nahm den
Gürtel von dem Ort, dahin ich ihn versteckt hatte; und siehe, der Gürtel
war verdorben, dass er nichts mehr taugte.

\bibverse{8} Da geschah des HErrn Wort zu mir und sprach: \bibverse{9}
So spricht der HErr: Eben also will ich auch verderben die große Hoffart
Judas und Jerusalems. \bibverse{10} Das böse Volk, das meine Worte nicht
hören will, sondern gehen hin nach Gedünken ihres Herzens und folgen
anderen Göttern, dass sie ihnen dienen und sie anbeten: sie sollen
werden wie der Gürtel, der nichts mehr taugt. \bibverse{11} Denn
gleichwie ein Mann den Gürtel um seine Lenden bindet, also habe ich,
spricht der HErr, das ganze Haus Israel und das ganze Haus Juda um mich
gegürtet, dass sie mein Volk sein sollten, mir zu einem Namen, zu Lob
und Ehren; aber sie wollen nicht hören.

\bibverse{12} So sage ihnen nun dieses Wort: So spricht der HErr, der
Gott Israels: Es sollen alle Krüge mit Wein gefüllt werden. So werden
sie zu dir sagen: Wer weiß das nicht, dass man alle Krüge mit Wein
füllen soll? \bibverse{13} So sprich zu ihnen: So spricht der HErr:
Siehe, ich will alle, die in diesem Lande wohnen, die Könige, die auf
dem Stuhl Davids sitzen, die Priester und Propheten und alle Einwohner
zu Jerusalem füllen, dass sie trunken werden sollen; \footnote{\textbf{13:13}
  Jer 25,15-18; Jes 51,17} \bibverse{14} und will einen mit dem anderen,
die Väter samt den Kindern, verstreuen, spricht der HErr; und will weder
schonen noch übersehen noch barmherzig sein über ihrem Verderben.
\bibverse{15} So höret nun und merket auf und trotzet nicht; denn der
HErr hat's geredet. \bibverse{16} Gebet dem HErrn, eurem Gott, die Ehre,
ehe denn es finster werde, und ehe eure Füße sich an den dunklen Bergen
stoßen, dass ihr des Lichts wartet, wenn er's doch gar finster und
dunkel machen wird. \bibverse{17} Wollt ihr aber solches nicht hören, so
muss meine Seele heimlich weinen über solche Hoffart; meine Augen müssen
von Tränen fließen, dass des HErrn Herde gefangen wird. \bibverse{18}
Sage dem König und der Königin: Setzt euch herunter; denn die Krone der
Herrlichkeit ist euch von eurem Haupt gefallen. \footnote{\textbf{13:18}
  Kla 5,16} \bibverse{19} Die Städte gegen Mittag sind verschlossen, und
ist niemand, der sie auftue; das ganze Juda ist rein weggeführt.
\bibverse{20} Hebet eure Augen auf und sehet, wie sie von Mitternacht
daherkommen. Wo ist nun die Herde, so dir befohlen war, deine herrliche
Herde? \bibverse{21} Was willst du sagen, wenn er dich so heimsuchen
wird? Denn du hast sie so gewöhnt wider dich, dass sie Fürsten und
Häupter sein wollen. Was gilt's? es wird dich Angst ankommen wie ein
Weib in Kindsnöten. \bibverse{22} Und wenn du in deinem Herzen sagen
willst: „Warum begegnet doch mir solches?{}`` Um der Menge willen deiner
Missetaten sind dir deine Säume aufgedeckt und ist deinen Fersen Gewalt
geschehen. \bibverse{23} Kann auch ein Mohr seine Haut wandeln oder ein
Parder seine Flecken? So könnt ihr auch Gutes tun, die ihr des Bösen
gewohnt seid. \footnote{\textbf{13:23} Ps 55,20} \bibverse{24} Darum
will ich sie zerstreuen wie Stoppeln, die vor dem Winde aus der Wüste
verweht werden. \bibverse{25} Das soll dein Lohn sein und dein Teil, den
ich dir zugemessen habe, spricht der HErr. Darum dass du mein vergessen
hast und verlässest dich auf Lügen, \bibverse{26} so will ich auch deine
Säume hoch aufdecken, dass man deine Schande sehen muss. \bibverse{27}
Denn ich habe gesehen deine Ehebrecherei, dein Geilheit, deine freche
Hurerei, ja, deine Gräuel auf Hügeln und auf Äckern. Weh dir, Jerusalem!
Wann wirst du doch endlich rein werden? \# 14 \bibverse{1} Dies ist das
Wort, das der HErr zu Jeremia sagte von der teuren Zeit: \bibverse{2}
Juda liegt jämmerlich, ihre Tore stehen elend; es stehet kläglich auf
dem Lande, und ist zu Jerusalem ein groß Geschrei. \bibverse{3} Die
Großen schicken die Kleinen nach Wasser; aber wenn sie zum Brunnen
kommen, finden sie kein Wasser und bringen ihre Gefäße leer wieder; sie
gehen traurig und betrübt und verhüllen ihre Häupter. \bibverse{4} Darum
dass die Erde lechzet, weil es nicht regnet auf die Erde, gehen die
Ackerleute traurig und verhüllen ihre Häupter. \footnote{\textbf{14:4}
  Joe 1,11} \bibverse{5} Denn auch die Hinden, die auf dem Felde werfen,
verlassen die Jungen, weil kein Gras wächst. \bibverse{6} Das Wild steht
auf den Hügeln und schnappt nach der Luft wie die Drachen und
verschmachtet, weil kein Kraut wächst. \bibverse{7} Ach HErr, unsere
Missetaten haben's ja verdient; aber hilf doch um deines Namens willen!
denn unser Ungehorsam ist groß, damit wir wider dich gesündigt haben.
\bibverse{8} Du bist der Trost Israels und sein Nothelfer; warum stellst
du dich, als wärest du ein Gast im Lande und ein Fremder, der nur über
Nacht darin bleibt? \bibverse{9} Warum stellst du dich wie ein Held, der
verzagt ist, und wie ein Riese, der nicht helfen kann? Du bist ja doch
unter uns, HErr, und wir heißen nach deinem Namen; verlass uns nicht!
\footnote{\textbf{14:9} Jer 15,16; Jes 43,7}

\bibverse{10} So spricht der HErr von diesem Volk: Sie laufen gern hin
und wieder und bleiben nicht gern daheim; darum will sie der HErr nicht,
sondern er denkt nun an ihre Missetat und will ihre Sünden heimsuchen.

\bibverse{11} Und der HErr sprach zu mir: Du sollst nicht für dieses
Volk um Gnade bitten. \bibverse{12} Denn ob sie gleich fasten, so will
ich doch ihr Flehen nicht hören; und ob sie Brandopfer und Speisopfer
bringen, so gefallen sie mir doch nicht, sondern ich will sie mit
Schwert, Hunger und Pestilenz aufreiben. \footnote{\textbf{14:12} Jes
  58,3; Jer 6,20}

\bibverse{13} Da sprach ich: Ach Herr HErr, siehe, die Propheten sagen
ihnen: Ihr werdet kein Schwert sehen und keine Teuerung bei euch haben;
sondern ich will euch guten Frieden geben an diesem Ort.

\bibverse{14} Und der HErr sprach zu mir: Die Propheten weissagen falsch
in meinem Namen; ich habe sie nicht gesandt und ihnen nichts befohlen
und nichts mit ihnen geredet. Sie predigen euch falsche Gesichte,
Deutungen, Abgötterei und ihres Herzens Trügerei. \bibverse{15} Darum so
spricht der HErr von den Propheten, die in meinem Namen weissagen,
obwohl ich sie doch nicht gesandt habe, und die dennoch predigen, es
werde kein Schwert noch Teuerung in dieses Land kommen: Solche Propheten
sollen sterben durch Schwert und Hunger. \footnote{\textbf{14:15} 5Mo
  18,20} \bibverse{16} Und die Leute, denen sie weissagen, sollen vom
Schwert und Hunger auf den Gassen zu Jerusalem hin und her liegen, dass
sie niemand begraben wird, also auch ihre Weiber, Söhne und Töchter; und
ich will ihre Bosheit über sie schütten. \footnote{\textbf{14:16} Jer
  8,2}

\bibverse{17} Und du sollst zu ihnen sagen dieses Wort: Meine Augen
fließen von Tränen Tag und Nacht und hören nicht auf; denn die Jungfrau,
die Tochter meines Volks, ist gräulich zerplagt und jämmerlich
geschlagen. \footnote{\textbf{14:17} Jer 8,23} \bibverse{18} Gehe ich
hinaus aufs Feld, siehe, so liegen da Erschlagene mit dem Schwert; komme
ich in die Stadt, so liegen da vor Hunger Verschmachtete. Denn es müssen
auch die Propheten, dazu auch die Priester, in ein Land ziehen, das sie
nicht kennen. \bibverse{19} Hast du denn Juda verworfen, oder hat deine
Seele einen Ekel an Zion? Warum hast du uns denn so geschlagen, dass uns
niemand heilen kann? Wir hofften, es sollte Friede werden; so kommt
nichts Gutes. Wir hofften, wir sollten heil werden; aber siehe, so ist
mehr Schaden da. \bibverse{20} HErr, wir erkennen unser gottlos Wesen
und unserer Väter Missetat; denn wir haben wider dich gesündigt.
\bibverse{21} Aber um deines Namens willen lass uns nicht geschändet
werden; lass den Thron deiner Herrlichkeit nicht verspottet werden;
gedenke doch und lass deinen Bund mit uns nicht aufhören. \bibverse{22}
Es ist doch ja unter der Heiden Götzen keiner, der Regen könnte geben;
auch der Himmel kann nicht regnen. Du bist doch ja der HErr, unser Gott,
auf den wir hoffen; denn du kannst solches alles tun. \# 15 \bibverse{1}
Und der HErr sprach zu mir: Und wenngleich Mose und Samuel vor mir
stünden, so habe ich doch kein Herz zu diesem Volk; treibe sie weg von
mir und lass sie hinfahren! \footnote{\textbf{15:1} Ps 99,6; Hes 14,14}
\bibverse{2} Und wenn sie zu dir sagen: Wo sollen wir hin? so sprich zu
ihnen: So spricht der HErr: Wen der Tod trifft, den treffe er; wen das
Schwert trifft, den treffe es; wen der Hunger trifft, den treffe er; wen
das Gefängnis trifft, den treffe es. \footnote{\textbf{15:2} Jer 43,11;
  Sach 11,9}

\bibverse{3} Denn ich will sie heimsuchen mit vielerlei Plagen, spricht
der HErr: mit dem Schwert, dass sie erwürgt werden; mit Hunden, die sie
schleifen sollen; mit den Vögeln des Himmels und mit Tieren auf Erden,
dass sie gefressen und vertilgt werden sollen. \footnote{\textbf{15:3}
  Hes 14,21} \bibverse{4} Und ich will sie in allen Königreichen auf
Erden hin und her treiben lassen um Manasses willen, des Sohnes Hiskias,
des Königs in Juda, um deswillen, was er zu Jerusalem begangen hat.
\footnote{\textbf{15:4} 2Kö 21,11-16; 2Kö 23,26} \bibverse{5} Wer will
denn sich dein erbarmen, Jerusalem? Wer wird denn Mitleiden mit dir
haben? Wer wird denn hingehen und dir Frieden wünschen? \bibverse{6} Du
hast mich verlassen, spricht der HErr, und bist von mir abgefallen;
darum habe ich meine Hand ausgestreckt wider dich, dass ich dich
verderben will; ich bin des Erbarmens müde. \bibverse{7} Ich will sie
mit der Wurfschaufel zum Lande hinausworfeln und will mein Volk, das von
seinem Wesen sich nicht bekehren will, zu eitel Waisen machen und
umbringen. \footnote{\textbf{15:7} Mt 3,12} \bibverse{8} Es sollen mir
mehr Witwen unter ihnen werden, denn Sand am Meer ist. Ich will über die
Mutter der jungen Mannschaft kommen lassen einen offenbaren Verderber
und die Stadt damit plötzlich und unversehens überfallen lassen,
\bibverse{9} dass die, die sieben Kinder hat, soll elend sein und von
Herzen seufzen. Denn ihre Sonne soll bei hohem Tage untergehen, dass ihr
Ruhm und ihre Freude ein Ende haben soll. Und die Übrigen will ich ins
Schwert geben vor ihren Feinden, spricht der HErr. \bibverse{10} Ach,
meine Mutter, dass du mich geboren hast, wider den jedermann hadert und
zankt im ganzen Lande! Habe ich doch weder auf Wucher geliehen noch
genommen; doch flucht mir jedermann. \footnote{\textbf{15:10} Jer 20,14}

\bibverse{11} Der HErr sprach: Wohlan, ich will euer etliche
übrigbehalten, denen es soll wieder wohl gehen, und will euch zu Hilfe
kommen in der Not und Angst unter den Feinden. \bibverse{12} Meinst du
nicht, dass etwa ein Eisen sei, welches könnte das Eisen und Erz von
Mitternacht zerschlagen? \bibverse{13} Ich will aber zuvor euer Gut und
eure Schätze zum Raube geben, dass ihr nichts dafür kriegen sollt, und
das um aller eurer Sünden willen, die ihr in allen euren Grenzen
begangen habt. \bibverse{14} Und ich will euch zu euren Feinden bringen
in ein Land, das ihr nicht kennet; denn es ist das Feuer in meinem Zorn
über euch angegangen. \bibverse{15} Ach HErr, du weißt es; gedenke an
mich und nimm dich meiner an und räche mich an meinen Verfolgern. Nimm
mich auf und verzieh nicht deinen Zorn über sie; denn du weißt, dass ich
um deinetwillen geschmäht werde. \bibverse{16} Dein Wort ward mir
Speise, da ich's empfing; und dein Wort ist meines Herzens Freude und
Trost; denn ich bin ja nach deinem Namen genannt; HErr, Gott Zebaoth.
\bibverse{17} Ich habe mich nicht zu den Spöttern gesellt noch mich mit
ihnen gefreut, sondern bin allein geblieben vor deiner Hand; denn du
hattest mich gefüllt mit deinem Grimm. \bibverse{18} Warum währt doch
mein Leiden so lange, und meine Wunden sind so gar böse, dass sie
niemand heilen kann? Du bist mir geworden wie ein Born, der nicht mehr
quellen will. \footnote{\textbf{15:18} Jer 30,12}

\bibverse{19} Darum spricht der HErr also: Wo du dich zu mir hältst, so
will ich mich zu dir halten, und sollst mein Prediger bleiben. Und wo du
die Frommen lehrest, sich sondern von den bösen Leuten, so sollst du
mein Mund sein. Und ehe du solltest zu ihnen fallen, so müssen sie eher
zu dir fallen. \bibverse{20} Denn ich habe dich wider dieses Volk zur
festen, ehernen Mauer gemacht; ob sie wider dich streiten, sollen sie
dir doch nichts anhaben; denn ich bin bei dir, dass ich dir helfe und
dich errette, spricht der HErr, \bibverse{21} und will dich erretten aus
der Hand der Bösen und erlösen aus der Hand der Tyrannen. \# 16
\bibverse{1} Und des HErrn Wort geschah zu mir und sprach: \bibverse{2}
Du sollst kein Weib nehmen und weder Söhne noch Töchter zeugen an diesem
Ort. \bibverse{3} Denn so spricht der HErr von den Söhnen und Töchtern,
die an diesem Ort geboren werden, dazu von ihren Müttern, die sie
gebären, und von ihren Vätern, die sie zeugen in diesem Lande:
\bibverse{4} Sie sollen an Krankheiten sterben und weder beklagt noch
begraben werden, sondern sollen Dung werden auf dem Lande, dazu durch
Schwert und Hunger umkommen, und ihre Leichname sollen der Vögel des
Himmels und der Tiere auf Erden Speise sein.

\bibverse{5} Denn so spricht der HErr: Du sollst nicht zum Trauerhaus
gehen und sollst auch nirgend hin zu klagen gehen noch Mitleiden über
sie haben; denn ich habe meinen Frieden von diesem Volk weggenommen,
spricht der HErr, samt meiner Gnade und Barmherzigkeit, \bibverse{6}
dass beide, groß und klein, sollen in diesem Lande sterben und nicht
begraben noch beklagt werden, und niemand wird sich über sie zerritzen
noch kahl scheren. \bibverse{7} Und man wird auch nicht unter sie Brot
austeilen bei der Klage, sie zu trösten über die Leiche, und ihnen auch
nicht aus dem Trostbecher zu trinken geben über Vater und Mutter.

\bibverse{8} Du sollst auch in kein Trinkhaus gehen, bei ihnen zu
sitzen, weder zu essen noch zu trinken. \bibverse{9} Denn so spricht der
HErr Zebaoth, der Gott Israels: Siehe, ich will an diesem Ort wegnehmen
vor euren Augen und bei eurem Leben die Stimme der Freude und Wonne, die
Stimme des Bräutigams und der Braut. \footnote{\textbf{16:9} Jer 7,34}
\bibverse{10} Und wenn du solches alles diesem Volk gesagt hast und sie
zu dir sprechen werden: Warum redet der HErr über uns all dies große
Unglück? welches ist die Missetat und Sünde, damit wir wider den HErrn,
unseren Gott, gesündigt haben? \bibverse{11} sollst du ihnen sagen:
Darum dass eure Väter mich verlassen haben, spricht der HErr, und
anderen Göttern gefolgt sind, ihnen gedient und sie angebetet, mich aber
verlassen und mein Gesetz nicht gehalten haben \bibverse{12} und ihr
noch ärger tut als eure Väter. Denn siehe, ein jeglicher lebt nach
seines bösen Herzens Gedünken, dass er mir nicht gehorche. \bibverse{13}
Darum will ich euch aus diesem Lande stoßen in ein Land, davon weder ihr
noch eure Väter gewusst haben; daselbst sollt ihr anderen Göttern dienen
Tag und Nacht, dieweil ich euch keine Gnade erzeigen will.

\bibverse{14} Darum siehe, es kommt die Zeit, spricht der HErr, dass man
nicht mehr sagen wird: So wahr der HErr lebt, der die Kinder Israel aus
Ägyptenland geführt hat! \footnote{\textbf{16:14} Jer 23,7-8}
\bibverse{15} sondern: So wahr der HErr lebt, der die Kinder Israel
geführt hat aus dem Lande der Mitternacht und aus allen Ländern, dahin
er sie verstoßen hatte! Denn ich will sie wiederbringen in das Land, das
ich ihren Vätern gegeben habe.

\bibverse{16} Siehe, ich will viele Fischer aussenden, spricht der HErr,
die sollen sie fischen; und darnach will ich viele Jäger aussenden, die
sollen sie fangen auf allen Bergen und auf allen Hügeln und in allen
Steinritzen. \bibverse{17} Denn meine Augen sehen auf alle ihre Wege,
dass sie vor mir sich nicht verhehlen können; und ihre Missetat ist vor
meinen Augen unverborgen. \bibverse{18} Aber zuvor will ich ihre
Missetat und Sünde zwiefach bezahlen, darum dass sie mein Land mit den
Leichen ihrer Abgötterei verunreinigt und mein Erbe mit ihren Gräueln
angefüllt haben. \bibverse{19} HErr, du bist meine Stärke und Kraft und
meine Zuflucht in der Not. Die Heiden werden zu dir kommen von der Welt
Enden und sagen: Unsere Väter haben falsche und nichtige Götter gehabt,
die nichts nützen können. \bibverse{20} Wie kann ein Mensch Götter
machen, die doch nicht Götter sind? \bibverse{21} Darum siehe, nun will
ich sie lehren und meine Hand und Gewalt ihnen kundtun, dass sie
erfahren sollen, ich heiße der HErr. \# 17 \bibverse{1} Die Sünde Judas
ist geschrieben mit eisernen Griffeln und spitzigen Demanten geschrieben
und auf die Tafel ihres Herzens gegraben und auf die Hörner an ihren
Altären, \bibverse{2} dass ihre Kinder gedenken sollen derselben Altäre
und Ascherabilder bei den grünen Bäumen, auf den hohen Bergen.
\bibverse{3} Aber ich will deine Höhen, beide, auf Bergen und Feldern,
samt deiner Habe und allen deinen Schätzen zum Raube geben um der Sünde
willen, in allen deinen Grenzen begangen. \bibverse{4} Und du sollst aus
deinem Erbe verstoßen werden, das ich dir gegeben habe, und will dich zu
Knechten deiner Feinde machen in einem Lande, das du nicht kennst; denn
ihr habt ein Feuer meines Zorns angezündet, das ewiglich brennen wird.

\bibverse{5} So spricht der HErr: Verflucht ist der Mann, der sich auf
Menschen verlässt und hält Fleisch für seinen Arm und mit seinem Herzen
vom HErrn weicht. \footnote{\textbf{17:5} Ps 118,8; Ps 146,3}
\bibverse{6} Der wird sein wie die Heide in der Wüste und wird nicht
sehen den zukünftigen Trost, sondern wird bleiben in der Dürre, in der
Wüste, in einem unfruchtbaren Lande, da niemand wohnt. \footnote{\textbf{17:6}
  Jer 48,6} \bibverse{7} Gesegnet aber ist der Mann, der sich auf den
HErrn verlässt und des Zuversicht der HErr ist. \footnote{\textbf{17:7}
  Ps 146,5} \bibverse{8} Der ist wie ein Baum, am Wasser gepflanzt und
am Bach gewurzelt. Denn obgleich eine Hitze kommt, fürchtet er sich doch
nicht, sondern seine Blätter bleiben grün, und sorgt nicht, wenn ein
dürres Jahr kommt sondern er bringt ohne Aufhören Früchte. \bibverse{9}
Es ist das Herz ein trotzig und verzagt Ding; wer kann es ergründen?
\bibverse{10} Ich, der HErr, kann das Herz ergründen und die Nieren
prüfen und gebe einem jeglichen nach seinem Tun, nach den Früchten
seiner Werke. \footnote{\textbf{17:10} Ps 7,10; Röm 2,6} \bibverse{11}
Denn gleichwie ein Vogel, der sich über Eier setzt und brütet sie nicht
aus, also ist der, der unrecht Gut sammelt; denn er muss davon, wenn
er's am wenigsten achtet, und muss doch zuletzt Spott dazu haben.
\footnote{\textbf{17:11} Ps 39,7} \bibverse{12} Aber die Stätte unseres
Heiligtums, der Thron göttlicher Ehre, ist allezeit fest geblieben.
\bibverse{13} Denn, HErr, du bist die Hoffnung Israels. Alle, die dich
verlassen, müssen zu Schanden werden, und die Abtrünnigen müssen in die
Erde geschrieben werden; denn sie verlassen den HErrn, die Quelle des
lebendigen Wassers. \footnote{\textbf{17:13} Jer 2,13} \bibverse{14}
Heile du mich, HErr, so werde ich heil; hilf du mir, so ist mir
geholfen; denn du bist mein Ruhm. \bibverse{15} Siehe, sie sprechen zu
mir: Wo ist denn des HErrn Wort? Lass es doch kommen! \bibverse{16} Aber
ich bin nicht von dir geflohen, dass ich nicht dein Hirte wäre; so habe
ich den bösen Tag nicht begehrt, das weißt du; was ich gepredigt habe,
das ist recht vor dir. \bibverse{17} Sei du mir nur nicht schrecklich,
meine Zuversicht in der Not! \bibverse{18} Lass sie zu Schanden werden,
die mich verfolgen, und mich nicht; lass sie erschrecken, und mich
nicht; lass den Tag des Unglücks über sie kommen und zerschlage sie
zwiefach!

\bibverse{19} So spricht der HErr zu mir: Gehe hin und tritt unter das
Tor des Volks, dadurch die Könige Judas aus und ein gehen, und unter
alle Tore zu Jerusalem \bibverse{20} und sprich zu ihnen: Höret des
HErrn Wort, ihr Könige Judas und ganz Juda und alle Einwohner zu
Jerusalem, so zu diesem Tor eingehen. \bibverse{21} So spricht der HErr:
Hütet euch und tragt keine Last am Sabbattage durch die Tore hinein zu
Jerusalem \bibverse{22} und führet keine Last am Sabbattage aus euren
Häusern und tut keine Arbeit, sondern heiliget den Sabbattag, wie ich
euren Vätern geboten habe. \footnote{\textbf{17:22} Jes 56,2; Jes 58,13}
\bibverse{23} Aber sie hören nicht und neigen ihre Ohren nicht, sondern
bleiben halsstarrig, dass sie mich ja nicht hören noch sich ziehen
lassen. \footnote{\textbf{17:23} Jer 11,8} \bibverse{24} So ihr mich
hören werdet, spricht der HErr, dass ihr keine Last traget des
Sabbattages durch dieser Stadt Tore ein, sondern ihn heiliget, dass ihr
keine Arbeit an demselben Tage tut: \bibverse{25} so sollen auch durch
dieser Stadt Tore aus und ein gehen Könige und Fürsten, die auf dem
Stuhl Davids sitzen, und reiten und fahren, auf Wagen und Rossen, sie
und ihre Fürsten samt allen, die in Juda und Jerusalem wohnen; und soll
diese Stadt ewiglich bewohnt werden; \bibverse{26} und sollen kommen aus
den Städten Judas, und die um Jerusalem her liegen, und aus dem Lande
Benjamin, aus den Gründen und von den Gebirgen und vom Mittag, die da
bringen Brandopfer, Schlachtopfer, Speisopfer und Weihrauch zum Hause
des HErrn. \bibverse{27} Werdet ihr mich aber nicht hören, dass ihr den
Sabbattag heiliget und keine Last traget durch die Tore zu Jerusalem ein
am Sabbattage, so will ich ein Feuer unter ihren Toren anzünden, das die
Häuser zu Jerusalem verzehren und nicht gelöscht werden soll. \# 18
\bibverse{1} Dies ist das Wort, das geschah vom HErrn zu Jeremia, und
sprach: \bibverse{2} Mache dich auf und gehe hinab in des Töpfers Haus;
daselbst will ich dich meine Worte hören lassen.

\bibverse{3} Und ich ging hinab in des Töpfers Haus, und siehe, er
arbeitete eben auf der Scheibe. \bibverse{4} Und der Topf, den er aus
dem Ton machte, missriet ihm unter den Händen. Da machte er einen
anderen Topf daraus, wie es ihm gefiel.

\bibverse{5} Da geschah des HErrn Wort zu mir und sprach: \bibverse{6}
Kann ich nicht auch also mit euch umgehen, ihr vom Hause Israel, wie
dieser Töpfer? spricht der HErr. Siehe, wie der Ton ist in des Töpfers
Hand, also seid auch ihr vom Hause Israel in meiner Hand. \footnote{\textbf{18:6}
  Jes 45,9; Röm 9,21} \bibverse{7} Plötzlich rede ich wider ein Volk und
Königreich, dass ich es ausrotten, zerbrechen und verderben wolle.
\footnote{\textbf{18:7} Jer 1,10} \bibverse{8} Wenn sich's aber bekehrt
von seiner Bosheit, dawider ich rede, so soll mich auch reuen das
Unglück, das ich ihm gedachte zu tun. \footnote{\textbf{18:8} Jer 26,3;
  Jer 26,19; Jon 3,10} \bibverse{9} Und plötzlich rede ich von einem
Volk und Königreich, dass ich's bauen und pflanzen wolle. \bibverse{10}
So es aber Böses tut vor meinen Augen, dass es meiner Stimme nicht
gehorcht, so soll mich auch reuen das Gute, das ich ihm verheißen hatte
zu tun.

\bibverse{11} So sprich nun zu denen in Juda und zu den Bürgern zu
Jerusalem: So spricht der HErr: Siehe, ich bereite euch ein Unglück zu
und habe Gedanken wider euch: darum kehre sich ein jeglicher von seinem
bösen Wesen und bessert euer Wesen und Tun. \bibverse{12} Aber sie
sprachen: Daraus wird nichts; wir wollen nach unseren Gedanken wandeln
und ein jeglicher tun nach Gedünken seines bösen Herzens. \footnote{\textbf{18:12}
  Jer 6,16; Jer 3,17}

\bibverse{13} Darum spricht der HErr: Fragt doch unter den Heiden. Wer
hat je desgleichen gehört? Dass die Jungfrau Israel so gar gräuliche
Dinge tut! \bibverse{14} Bleibt doch der Schnee länger auf den Steinen
im Felde, wenn's vom Libanon herab schneit, und das Regenwasser
verschießt nicht so bald, wie mein Volk vergisst. \bibverse{15} Sie
räuchern den Göttern und richten Ärgernis an auf ihren Wegen für und für
und gehen auf ungebahnten Straßen, \bibverse{16} auf dass ihr Land zur
Wüste werde, ihnen zur ewigen Schande, dass, wer vorübergeht, sich
verwundere und den Kopf schüttle. \bibverse{17} Denn ich will sie wie
durch einen Ostwind zerstreuen vor ihren Feinden; ich will ihnen den
Rücken und nicht das Antlitz zeigen, wenn sie verderben.

\bibverse{18} Aber sie sprechen: Kommt und lasst uns wider Jeremia
ratschlagen; denn die Priester können nicht irren im Gesetz, und die
Weisen können nicht fehlen mit Raten, und die Propheten können nicht
unrecht lehren! Kommt her, lasst uns ihn mit der Zunge totschlagen und
nichts geben auf alle seine Rede! \bibverse{19} HErr, habe Acht auf mich
und höre die Stimme meiner Widersacher! \bibverse{20} Ist's recht, dass
man Gutes mit Bösem vergilt? Denn sie haben meiner Seele eine Grube
gegraben. Gedenke doch, wie ich vor dir gestanden bin, dass ich ihr
Bestes redete und deinen Grimm von ihnen wendete. \footnote{\textbf{18:20}
  Ps 35,7} \bibverse{21} So strafe nun ihre Kinder mit Hunger und lass
sie ins Schwert fallen, dass ihre Weiber ohne Kinder und Witwen seien
und ihre Männer zu Tode geschlagen und ihre junge Mannschaft im Streit
durchs Schwert erwürgt werde; \bibverse{22} dass ein Geschrei aus ihren
Häusern gehört werde, wie du plötzlich habest Kriegsvolk über sie kommen
lassen. Denn sie haben eine Grube gegraben, mich zu fangen, und meinen
Füßen Stricke gelegt. \bibverse{23} Und weil du, HErr, weißt alle ihre
Anschläge wider mich, dass sie mich töten wollen, so vergib ihnen ihre
Missetat nicht und lass ihre Sünde vor dir nicht ausgetilgt werden. Lass
sie vor dir gestürzt werden und handle mit ihnen nach deinem Zorn. \# 19
\bibverse{1} So spricht der HErr: Gehe hin und kaufe dir einen irdenen
Krug vom Töpfer, samt etlichen von den Ältesten des Volks und von den
Ältesten der Priester, \bibverse{2} und gehe hinaus ins Tal Ben-Hinnom,
das vor dem Ziegeltor liegt, und predige daselbst die Worte, die ich dir
sage, \footnote{\textbf{19:2} Jer 19,11; Jer 7,31} \bibverse{3} und
sprich: Höret des HErrn Wort, ihr Könige Judas und Bürger zu Jerusalem!
So spricht der HErr Zebaoth, der Gott Israels: Siehe, ich will ein solch
Unglück über diese Stätte gehen lassen, dass, wer es hören wird, dem die
Ohren klingen sollen, \footnote{\textbf{19:3} 1Sam 3,11; 2Kö 21,12}
\bibverse{4} darum dass sie mich verlassen und diese Stätte einem
fremden Gott gegeben haben und anderen Göttern darin geräuchert haben,
die weder sie noch ihre Väter noch die Könige Judas gekannt haben, und
haben die Stätte voll unschuldigen Bluts gemacht \bibverse{5} und haben
dem Baal Höhen gebaut, ihre Kinder zu verbrennen, dem Baal zu
Brandopfern, was ich ihnen weder geboten noch davon geredet habe, was
auch in mein Herz nie gekommen ist. \footnote{\textbf{19:5} Jer 7,31-32}
\bibverse{6} Darum siehe, es wird die Zeit kommen, spricht der HErr,
dass man diese Stätte nicht mehr Thopheth noch das Tal Ben-Hinnom,
sondern Würgetal heißen wird.

\bibverse{7} Und ich will den Gottesdienst Judas und Jerusalems an
diesem Ort zerstören und will sie durchs Schwert fallen lassen vor ihren
Feinden, unter der Hand derer, die nach ihrem Leben stehen, und will
ihre Leichname den Vögeln des Himmels und den Tieren auf Erden zu
fressen geben \bibverse{8} und will diese Stadt wüst machen und zum
Spott, dass alle, die vorübergehen, werden sich verwundern über alle
ihre Plage und ihrer spotten. \footnote{\textbf{19:8} Jer 18,16}
\bibverse{9} Ich will sie lassen ihrer Söhne und Töchter Fleisch
fressen, und einer soll des anderen Fleisch fressen in der Not und
Angst, damit sie ihre Feinde und die, die nach ihrem Leben stehen,
bedrängen werden. \footnote{\textbf{19:9} 5Mo 28,53}

\bibverse{10} Und du sollst den Krug zerbrechen vor den Männern, die mit
dir gegangen sind, \bibverse{11} und sprich zu ihnen: So spricht der
HErr Zebaoth: Eben wie man eines Töpfers Gefäß zerbricht, das nicht kann
wieder ganz werden, so will ich dieses Volk und diese Stadt auch
zerbrechen; und sie sollen dazu im Thopheth begraben werden, weil sonst
kein Raum sein wird, zu begraben. \footnote{\textbf{19:11} Jes 30,14;
  Jer 7,32} \bibverse{12} So will ich mit dieser Stätte, spricht der
HErr, und ihren Einwohnern umgehen, dass diese Stadt werden soll gleich
wie das Thopheth. \bibverse{13} Dazu sollen die Häuser zu Jerusalem und
die Häuser der Könige Judas ebenso unrein werden wie die Stätte
Thopheth, ja, alle Häuser, wo sie auf den Dächern geräuchert haben allem
Heer des Himmels und anderen Göttern Trankopfer geopfert haben.

\bibverse{14} Und da Jeremia wieder vom Thopheth kam, dahin ihn der HErr
gesandt hatte, zu weissagen, trat er in den Vorhof am Hause des HErrn
und sprach zu allem Volk: \bibverse{15} So spricht der HErr Zebaoth, der
Gott Israels: Siehe, ich will über diese Stadt und über alle ihre Städte
all das Unglück kommen lassen, das ich wider sie geredet habe, darum
dass sie halsstarrig sind und meine Worte nicht hören wollen. \# 20
\bibverse{1} Da aber Pashur, ein Sohn Immers, der Priester, der zum
Obersten im Hause des HErrn gesetzt war, Jeremia hörte solche Worte
weissagen, \bibverse{2} schlug er den Propheten Jeremia und legte ihn in
den Stock unter dem Obertor Benjamin, welches am Hause des HErrn ist.
\bibverse{3} Und da es Morgen ward, zog Pashur Jeremia aus dem Stock. Da
sprach Jeremia zu ihm: Der HErr heißt dich nicht Pashur, sondern
Schrecken um und um. \bibverse{4} Denn so spricht der HErr: Siehe, ich
will dich zum Schrecken machen dir selbst und allen deinen Freunden, und
sie sollen fallen durchs Schwert ihrer Feinde; das sollst du mit deinen
Augen sehen. Und will das ganze Juda in die Hand des Königs zu Babel
übergeben; der soll sie wegführen gen Babel und mit dem Schwert töten.
\bibverse{5} Auch will ich alle Güter dieser Stadt samt allem, was sie
gearbeitet, und alle Kleinode und alle Schätze der Könige Judas in ihrer
Feinde Hand geben, dass sie dieselben rauben, nehmen und gen Babel
bringen. \footnote{\textbf{20:5} Jes 39,6} \bibverse{6} Und du, Pashur,
sollst mit allen deinen Hausgenossen gefangen gehen und gen Babel
kommen; daselbst sollst du sterben und begraben werden samt allen deinen
Freunden, welchen du Lügen predigst. \bibverse{7} HErr, du hast mich
überredet, und ich habe mich überreden lassen; du bist mir zu stark
gewesen und hast gewonnen; aber ich bin darüber zum Spott geworden
täglich, und jedermann verlacht mich. \bibverse{8} Denn seit ich
geredet, gerufen und gepredigt habe von der Plage und Verstörung, ist
mir des HErrn Wort zum Hohn und Spott geworden täglich. \footnote{\textbf{20:8}
  Jes 49,4} \bibverse{9} Da dachte ich: Wohlan, ich will sein nicht mehr
gedenken und nicht mehr in seinem Namen predigen. Aber es ward in meinem
Herzen wie ein brennendes Feuer, in meinen Gebeinen verschlossen, dass
ich's nicht leiden konnte, und wäre schier vergangen. \bibverse{10} Denn
ich höre, wie mich viele schelten und schrecken um und um. „Hui,
verklagt ihn! Wir wollen ihn verklagen!{}`` sprechen alle meine Freunde
und Gesellen, „ob wir ihn übervorteilen und ihm beikommen mögen und uns
an ihm rächen.`` \bibverse{11} Aber der HErr ist bei mir wie ein starker
Held; darum werden meine Verfolger fallen und nicht obliegen, sondern
sollen sehr zu Schanden werden, darum dass sie so töricht handeln; ewig
wird die Schande sein, deren man nicht vergessen wird. \footnote{\textbf{20:11}
  Jer 1,8; Jer 1,19; Jer 15,20} \bibverse{12} Und nun, HErr Zebaoth, der
du die Gerechten prüfst, Nieren und Herz siehst, lass mich deine Rache
an ihnen sehen; denn ich habe dir meine Sache befohlen. \footnote{\textbf{20:12}
  Jer 11,20} \bibverse{13} Singet dem HErrn, rühmt den HErrn, der des
Armen Leben aus der Boshaften Händen errettet! \bibverse{14} Verflucht
sei der Tag, darin ich geboren bin; der Tag müsse ungesegnet sein, darin
mich meine Mutter geboren hat! \footnote{\textbf{20:14} Jer 15,10; Hi
  3,1-10; Hi 10,18} \bibverse{15} Verflucht sei der, der meinem Vater
gute Botschaft brachte und sprach: „Du hast einen jungen Sohn``, dass er
ihn fröhlich machen wollte! \bibverse{16} Der Mann müsse sein wie die
Städte, die der HErr umgekehrt und ihn nicht gereut hat; und müsse des
Morgens hören ein Geschrei und des Mittags ein Heulen! \bibverse{17}
Dass du mich doch nicht getötet hast im Mutterleibe, dass meine Mutter
mein Grab gewesen und ihr Leib ewig schwanger geblieben wäre!
\bibverse{18} Warum bin ich doch aus Mutterleibe hervorgekommen, dass
ich solchen Jammer und Herzeleid sehen muss und meine Tage mit Schanden
zubringen! \# 21 \bibverse{1} Dies ist das Wort, das vom HErrn geschah
zu Jeremia, da der König Zedekia zu ihm sandte Pashur, den Sohn
Malchias, und Zephanja, den Sohn Maasejas, den Priester, und ließ ihm
sagen: \footnote{\textbf{21:1} Jer 29,25} \bibverse{2} Frage doch den
HErrn für uns. Denn Nebukadnezar, der König zu Babel, streitet wider
uns; dass der HErr doch mit uns tun wolle nach allen seinen Wundern,
damit er von uns abzöge.

\bibverse{3} Jeremia sprach zu ihnen: So saget Zedekia: \bibverse{4} Das
spricht der HErr, der Gott Israels: Siehe, ich will die Waffen
zurückwenden, die ihr in euren Händen habt, womit ihr streitet wider den
König zu Babel und wider die Chaldäer, welche euch draußen an der Mauer
belagert haben; und will sie zuhauf sammeln mitten in dieser Stadt.
\bibverse{5} Und ich will wider euch streiten mit ausgereckter Hand, mit
starkem Arm, mit Zorn, Grimm und großer Ungnade. \bibverse{6} Und will
die Bürger dieser Stadt schlagen, die Menschen und das Vieh, dass sie
sterben sollen durch eine große Pestilenz. \bibverse{7} Und darnach,
spricht der HErr, will ich Zedekia, den König Judas, samt seinen
Knechten und dem Volk, das in dieser Stadt vor der Pestilenz, vor
Schwert und Hunger übrigbleiben wird, geben in die Hände Nebukadnezars,
des Königs zu Babel, und in die Hände ihrer Feinde, und in die Hände
derer, die ihnen nach dem Leben stehen, dass er sie mit der Schärfe des
Schwerts also schlage, dass kein Schonen noch Gnade noch Barmherzigkeit
da sei.

\bibverse{8} Und sage diesem Volk: So spricht der HErr: Siehe, ich lege
euch vor den Weg zum Leben und den Weg zum Tode. \bibverse{9} Wer in
dieser Stadt bleibt, der wird sterben müssen durch Schwert, Hunger und
Pestilenz; wer aber sich hinausbegibt zu den Chaldäern, die euch
belagern, der soll lebendig bleiben und soll sein Leben als eine
Ausbeute behalten. \footnote{\textbf{21:9} Jer 38,2} \bibverse{10} Denn
ich habe mein Angesicht über diese Stadt gerichtet zum Unglück und zu
keinem Guten, spricht der HErr. Sie soll dem König zu Babel übergeben
werden, dass er sie mit Feuer verbrenne.

\bibverse{11} Und höret des HErrn Wort, ihr vom Hause des Königs in
Juda! \bibverse{12} Du Haus David, so spricht der HErr: Haltet des
Morgens Gericht und errettet den Beraubten aus des Frevlers Hand, auf
dass mein Grimm nicht ausfahre wie ein Feuer und brenne also, dass
niemand löschen könne, um eures bösen Wesens willen. \bibverse{13}
Siehe, spricht der HErr, ich will an dich, die du wohnest im Grunde, auf
dem Felsen der Ebene und sprichst: Wer will uns überfallen oder in
unsere Feste kommen? \bibverse{14} Ich will euch heimsuchen, spricht der
HErr, nach der Frucht eures Tuns; ich will ein Feuer anzünden in ihrem
Walde, das soll alles umher verzehren. \# 22 \bibverse{1} So spricht der
HErr: Gehe hinab in das Haus des Königs in Juda und rede daselbst dieses
Wort \bibverse{2} und sprich: Höre des HErrn Wort, du König Judas, der
du auf dem Stuhl Davids sitzest, du und deine Knechte und dein Volk, die
zu diesen Toren eingehen. \bibverse{3} So spricht der HErr: Haltet Recht
und Gerechtigkeit, und errettet den Beraubten von des Frevlers Hand, und
schindet nicht die Fremdlinge, Waisen und Witwen, und tut niemand
Gewalt, und vergießt nicht unschuldig Blut an dieser Stätte. \footnote{\textbf{22:3}
  Jer 21,12} \bibverse{4} Werdet ihr solches tun, so sollen durch die
Tore dieses Hauses einziehen Könige, die auf Davids Stuhl sitzen, zu
Wagen und zu Rosse, samt ihren Knechten und ihrem Volk. \footnote{\textbf{22:4}
  Jer 17,25} \bibverse{5} Werdet ihr aber solchem nicht gehorchen, so
habe ich bei mir selbst geschworen, spricht der HErr, dieses Haus soll
verstört werden.

\bibverse{6} Denn so spricht der HErr von dem Hause des Königs in Juda:
Ein Gilead bist du mir, ein Haupt im Libanon. Was gilt's? ich will dich
zur Wüste und die Städte ohne Einwohner machen. \bibverse{7} Denn ich
habe Verderber über dich bestellt, einen jeglichen mit seinen Waffen;
die sollen deine auserwählten Zedern umhauen und ins Feuer werfen.

\bibverse{8} So werden viele Heiden vor dieser Stadt vorübergehen und
untereinander sagen: Warum hat der HErr mit dieser großen Stadt also
gehandelt? \bibverse{9} Und man wird antworten: Darum dass sie den Bund
des HErrn, ihres Gottes, verlassen und andere Götter angebetet und ihnen
gedient haben. \bibverse{10} Weinet nicht über die Toten und grämet euch
nicht darum; weinet aber über den, der dahinzieht; denn er wird nimmer
wiederkommen, dass er sein Vaterland sehen möchte. \bibverse{11} Denn so
spricht der HErr von Sallum, dem Sohn Josias, des Königs in Juda,
welcher König ist anstatt seines Vaters Josia, der von dieser Stätte
hinausgezogen ist: Er wird nicht wieder herkommen, \footnote{\textbf{22:11}
  2Chr 36,3-4} \bibverse{12} sondern muss sterben an dem Ort, dahin er
gefangen geführt ist, und wird dies Land nicht mehr sehen. \bibverse{13}
Weh dem, der sein Haus mit Sünden baut und seine Gemächer mit Unrecht,
der seinen Nächsten umsonst arbeiten lässt und gibt ihm seinen Lohn
nicht \bibverse{14} und denkt: „Wohlan, ich will mir ein großes Haus
bauen und weite Gemächer!{}`` und lässt sich Fenster drein hauen und es
mit Zedern täfeln und rot malen! \bibverse{15} Meinst du, du wollest
König sein, weil du mit Zedern prangst? Hat dein Vater nicht auch
gegessen und getrunken und hielt dennoch Recht und Gerechtigkeit, und es
ging ihm wohl? \bibverse{16} Er half dem Elenden und Armen zum Recht,
und es ging ihm wohl. Ist's nicht also, dass solches heißt, mich recht
erkennen? spricht der HErr. \bibverse{17} Aber deine Augen und dein Herz
stehen nicht also, sondern auf deinen Geiz, auf unschuldig Blut zu
vergießen, zu freveln und unterzustoßen. \bibverse{18} Darum spricht der
HErr von Jojakim, dem Sohn Josias, dem König Judas: Man wird ihn nicht
beklagen: „Ach Bruder! ach Schwester!{}``, man wird ihn nicht beklagen:
„Ach Herr! ach Edler!{}`` \footnote{\textbf{22:18} Jer 34,5}
\bibverse{19} Er soll wie ein Esel begraben werden, zerschleift und
hinausgeworfen vor die Tore Jerusalems. \footnote{\textbf{22:19} Jes
  14,19} \bibverse{20} Gehe hinauf auf den Libanon und schreie und lass
dich hören zu Basan und schreie von Abarim; denn alle deine Liebhaber
sind zunichte gemacht. \bibverse{21} Ich habe dir's vorhergesagt, da es
noch wohl um dich stand; aber du sprachst: „Ich will nicht hören.`` Also
hast du dein Lebtage getan, dass du meiner Stimme nicht gehorchtest.
\bibverse{22} Alle deine Hirten wird der Wind weiden, und deine
Liebhaber ziehen gefangen dahin; da musst du zum Spott und zu Schanden
werden um aller deiner Bosheit willen. \footnote{\textbf{22:22} Jer
  25,9; Jer 25,18} \bibverse{23} Die du jetzt auf dem Libanon wohnest
und in Zedern nistest, wie schön wirst du sehen, wenn dir Schmerzen und
Wehen kommen werden wie einer in Kindsnöten! \footnote{\textbf{22:23}
  Jer 13,21}

\bibverse{24} So wahr ich lebe, spricht der HErr, wenn Chonja, der Sohn
Jojakims, der König Judas, ein Siegelring wäre an meiner rechten Hand,
so wollte ich dich doch abreißen \footnote{\textbf{22:24} Jer 24,1}
\bibverse{25} und in die Hände geben derer, die nach deinem Leben stehen
und vor welchen du dich fürchtest, in die Hände Nebukadnezars, des
Königs zu Babel, und der Chaldäer. \bibverse{26} Und ich will dich und
deine Mutter, die dich geboren hat, in ein anderes Land treiben, das
nicht euer Vaterland ist, und sollt daselbst sterben. \bibverse{27} Und
in das Land, da sie von Herzen gern wieder hin wären, sollen sie nicht
wiederkommen. \bibverse{28} Wie ein elender, verachteter, verstoßener
Mann ist doch Chonja! ein unwertes Gefäß! Ach wie ist er doch samt
seinem Samen so vertrieben und in ein unbekanntes Land geworfen!
\bibverse{29} O Land, Land, Land, höre des HErrn Wort! \bibverse{30} So
spricht der HErr: Schreibet an diesen Mann als einen, der ohne Kinder
ist, einen Mann, dem es sein Lebtage nicht gelingt. Denn er wird das
Glück nicht haben, dass jemand seines Samens auf dem Stuhl Davids sitze
und fürder in Juda herrsche. \# 23 \bibverse{1} Weh euch Hirten, die ihr
die Herde meiner Weide umbringet und zerstreuet! spricht der HErr.
\footnote{\textbf{23:1} Hes 13,2-16; Hes 34,1-34; Sach 11,5}
\bibverse{2} Darum spricht der HErr, der Gott Israels, von den Hirten,
die mein Volk weiden: Ihr habt meine Herde zerstreut und verstoßen und
nicht besucht. Siehe, ich will euch heimsuchen um eures bösen Wesens
willen, spricht der HErr. \bibverse{3} Und ich will die Übrigen meiner
Herde sammeln aus allen Ländern, dahin ich sie verstoßen habe, und will
sie wiederbringen zu ihren Hürden, dass sie sollen wachsen und viel
werden. \bibverse{4} Und ich will Hirten über sie setzen, die sie weiden
sollen, dass sie sich nicht mehr sollen fürchten noch erschrecken noch
heimgesucht werden, spricht der HErr. \bibverse{5} Siehe, es kommt die
Zeit, spricht der HErr, dass ich dem David ein gerechtes Gewächs
erwecken will, und soll ein König sein, der wohl regieren wird und Recht
und Gerechtigkeit auf Erden anrichten. \footnote{\textbf{23:5} Sach 3,8;
  Sach 6,12; Jes 21,1} \bibverse{6} Zu seiner Zeit soll Juda geholfen
werden und Israel sicher wohnen. Und dies wird sein Name sein, dass man
ihn nennen wird: Der HErr unsere Gerechtigkeit. \footnote{\textbf{23:6}
  Jer 33,16} \bibverse{7} Darum siehe, es wird die Zeit kommen, spricht
der HErr, dass man nicht mehr sagen wird: So wahr der HErr lebt, der die
Kinder Israel aus Ägyptenland geführt hat! \footnote{\textbf{23:7} Jer
  16,14-15} \bibverse{8} sondern: So wahr der HErr lebt, der den Samen
des Hauses Israel hat herausgeführt und gebracht aus dem Lande der
Mitternacht und aus allen Landen, dahin ich sie verstoßen hatte, dass
sie in ihrem Lande wohnen sollen!

\bibverse{9} Wider die Propheten. Mein Herz will mir in meinem Leibe
brechen, alle meine Gebeine zittern; mir ist wie einem trunkenen Mann
und wie einem, der vom Wein taumelt, vor dem HErrn und vor seinen
heiligen Worten; \bibverse{10} dass das Land so voll Ehebrecher ist,
dass das Land so jämmerlich steht, dass es so verflucht ist und die Auen
in der Wüste verdorren; und ihr Leben ist böse, und ihr Regiment taugt
nicht. \bibverse{11} Denn beide, Propheten und Priester, sind Schälke;
und auch in meinem Hause finde ich ihre Bosheit, spricht der HErr.
\bibverse{12} Darum ist ihr Weg wie ein glatter Weg im Finstern, darauf
sie gleiten und fallen; denn ich will Unglück über sie kommen lassen,
das Jahr ihrer Heimsuchung, spricht der HErr. \bibverse{13} Zwar bei den
Propheten zu Samaria sah ich Torheit, dass sie weissagten durch Baal und
verführten mein Volk Israel; \bibverse{14} aber bei den Propheten zu
Jerusalem sehe ich Gräuel, wie sie ehebrechen und gehen mit Lügen um und
stärken die Boshaften, auf dass sich ja niemand bekehre von seiner
Bosheit. Sie sind alle vor mir gleichwie Sodom, und die Bürger zu
Jerusalem wie Gomorra. \footnote{\textbf{23:14} Hes 13,22; Jes 1,10}

\bibverse{15} Darum spricht der HErr Zebaoth von den Propheten also:
Siehe, ich will sie mit Wermut speisen und mit Galle tränken; denn von
den Propheten zu Jerusalem kommt Heuchelei aus ins ganze Land.
\footnote{\textbf{23:15} Jer 9,14}

\bibverse{16} So spricht der HErr Zebaoth: Gehorchet nicht den Worten
der Propheten, so euch weissagen. Sie betrügen euch; denn sie predigen
ihres Herzens Gesicht und nicht aus des HErrn Munde. \footnote{\textbf{23:16}
  Jer 6,14} \bibverse{17} Sie sagen denen, die mich lästern: „Der HErr
hat's gesagt, es wird euch wohl gehen``; und allen, die nach ihres
Herzens Dünkel wandeln, sagen sie: „Es wird kein Unglück über euch
kommen.`` \footnote{\textbf{23:17} Jer 7,24} \bibverse{18} Aber wer ist
im Rat des HErrn gestanden, der sein Wort gesehen und gehört habe? Wer
hat sein Wort vernommen und gehört? \footnote{\textbf{23:18} Jes 40,13}
\bibverse{19} Siehe, es wird ein Wetter des HErrn mit Grimm kommen und
ein schreckliches Ungewitter den Gottlosen auf den Kopf fallen.
\footnote{\textbf{23:19} Jer 30,23} \bibverse{20} Und des HErrn Zorn
wird nicht nachlassen, bis er tue und ausrichte, was er im Sinn hat; zur
letzten Zeit werdet ihr's wohl erfahren. \bibverse{21} Ich sandte die
Propheten nicht, doch liefen sie; ich redete nicht zu ihnen, doch
weissagten sie. \footnote{\textbf{23:21} Jer 14,14} \bibverse{22} Denn
wo sie bei meinem Rat geblieben wären und hätten meine Worte meinem Volk
gepredigt, so hätten sie dasselbe von seinem bösen Wesen und von seinem
bösen Leben bekehrt. \bibverse{23} Bin ich nur ein Gott, der nahe ist,
spricht der HErr, und nicht auch ein Gott von ferneher? \bibverse{24}
Meinst du, dass sich jemand so heimlich verbergen könne, dass ich ihn
nicht sehe? spricht der HErr. Bin ich es nicht, der Himmel und Erde
füllt? spricht der HErr.

\bibverse{25} Ich höre es wohl, was die Propheten predigen und falsch
weissagen in meinem Namen und sprechen: Mir hat geträumt, mir hat
geträumt. \bibverse{26} Wann wollen doch die Propheten aufhören, die
falsch weissagen und ihres Herzens Trügerei weissagen \bibverse{27} und
wollen, dass mein Volk meines Namens vergesse über ihren Träumen, die
einer dem anderen erzählt? gleichwie ihre Väter meines Namens vergaßen
über dem Baal. \bibverse{28} Ein Prophet, der Träume hat, der erzähle
Träume; wer aber mein Wort hat, der predige mein Wort recht. Wie reimen
sich Stroh und Weizen zusammen? spricht der HErr. \bibverse{29} Ist mein
Wort nicht wie Feuer, spricht der HErr, und wie ein Hammer, der Felsen
zerschmeißt?

\bibverse{30} Darum siehe, ich will an die Propheten, spricht der HErr,
die mein Wort stehlen einer dem anderen. \bibverse{31} Siehe, ich will
an die Propheten, spricht der HErr, die ihr eigenes Wort führen und
sprechen: Er hat's gesagt. \bibverse{32} Siehe, ich will an die, die
falsche Träume weissagen, spricht der HErr, und erzählen dieselben und
verführen mein Volk mit ihren Lügen und losen Reden, obwohl ich sie doch
nicht gesandt und ihnen nichts befohlen habe und sie auch diesem Volk
nichts nütze sind, spricht der HErr. \footnote{\textbf{23:32} Jer 23,21}

\bibverse{33} Wenn dich dieses Volk oder ein Prophet oder ein Priester
fragen wird und sagen: Welches ist die Last des HErrn? sollst du zu
ihnen sagen, was die Last sei: Ich will euch hinwerfen, spricht der
HErr. \bibverse{34} Und wo ein Prophet oder Priester oder das Volk wird
sagen: „Das ist die Last des HErrn``, den will ich heimsuchen und sein
Haus dazu. \bibverse{35} Also sollt ihr aber einer mit dem anderen reden
und untereinander sagen: „Was antwortet der HErr, und was sagt der
HErr?{}`` \bibverse{36} Und nennt's nicht mehr „Last des HErrn``; denn
einem jeglichem wird sein eigenes Wort eine „Last`` sein, weil ihr also
die Worte des lebendigen Gottes, des HErrn Zebaoth, unseres Gottes,
verkehrt. \bibverse{37} Darum sollt ihr zum Propheten also sagen: Was
antwortet dir der HErr, und was sagt der HErr? \bibverse{38} Weil ihr
aber sprechet: „Last des HErrn``, darum spricht der HErr also: Nun ihr
dieses Wort eine „Last des HErrn`` nennt und ich zu euch gesandt habe
und sagen lassen, ihr sollt's nicht nennen „Last des HErrn``:
\bibverse{39} siehe, so will ich euch hinwegnehmen und euch samt der
Stadt, die ich euch und euren Vätern gegeben habe, von meinem Angesicht
wegwerfen \bibverse{40} und will euch ewige Schande und ewige Schmach
zufügen, der nimmer vergessen soll werden. \# 24 \bibverse{1} Siehe, der
HErr zeigte mir zwei Feigenkörbe, gestellt vor den Tempel des HErrn,
nachdem der König zu Babel, Nebukadnezar, hatte weggeführt Jechonja, den
Sohn Jojakims, den König Judas, samt den Fürsten Judas und den
Zimmerleuten und Schmieden von Jerusalem und gen Babel gebracht.
\footnote{\textbf{24:1} Jer 29,2; 2Kö 24,14-15} \bibverse{2} In dem
einen Korbe waren sehr gute Feigen, wie die ersten reifen Feigen sind;
im anderen Korbe waren sehr schlechte Feigen, dass man sie nicht essen
konnte, so schlecht waren sie.

\bibverse{3} Und der HErr sprach zu mir: Jeremia, was siehest du? Ich
sprach: Feigen; die guten Feigen sind sehr gut, und die schlechten sind
sehr schlecht, dass man sie nicht essen kann, so schlecht sind sie.

\bibverse{4} Da geschah des HErrn Wort zu mir und sprach:

\bibverse{5} So spricht der HErr, der Gott Israels: Gleichwie diese
Feigen gut sind, also will ich mich gnädig annehmen der Gefangenen aus
Juda, welche ich habe aus dieser Stätte lassen ziehen in der Chaldäer
Land, \bibverse{6} und will sie gnädig ansehen, und will sie wieder in
dieses Land bringen, und will sie bauen und nicht abbrechen; ich will
sie pflanzen und nicht ausraufen, \footnote{\textbf{24:6} Jer 31,28}
\bibverse{7} und will ihnen ein Herz geben, dass sie mich kennen sollen,
dass ich der HErr sei. Und sie sollen mein Volk sein, so will ich ihr
Gott sein; denn sie werden sich von ganzem Herzen zu mir bekehren.
\footnote{\textbf{24:7} Jer 31,33-34}

\bibverse{8} Aber wie die schlechten Feigen so schlecht sind, dass man
sie nicht essen kann, spricht der HErr, also will ich dahingeben
Zedekia, den König Judas, samt seinen Fürsten, und was übrig ist zu
Jerusalem und übrig in diesem Lande und die in Ägyptenland wohnen.
\footnote{\textbf{24:8} Jer 29,17} \bibverse{9} Und will ihnen Unglück
zufügen und sie in keinem Königreich auf Erden bleiben lassen, dass sie
sollen zu Schanden werden, zum Sprichwort, zur Fabel und zum Fluch an
allen Orten, dahin ich sie verstoßen werde; \footnote{\textbf{24:9} Jer
  29,18} \bibverse{10} und will Schwert, Hunger und Pestilenz unter sie
schicken, bis sie umkommen von dem Lande, das ich ihnen und ihren Vätern
gegeben habe. \# 25 \bibverse{1} Dies ist das Wort, welches zu Jeremia
geschah über das ganze Volk Juda im vierten Jahr Jojakims, des Sohnes
Josias, des Königs in Juda (welches ist das erste Jahr Nebukadnezars,
des Königs zu Babel), \bibverse{2} welches auch der Prophet Jeremia
redete zu dem ganzen Volk Juda und zu allen Bürgern zu Jerusalem und
sprach: \bibverse{3} Es ist von dem dreizehnten Jahr an Josias, des
Sohnes Amons, des Königs Judas, des HErrn Wort zu mir geschehen bis auf
diesen Tag, und ich habe euch nun dreiundzwanzig Jahre mit Fleiß
gepredigt; aber ihr habt nie hören wollen.

\bibverse{4} So hat der HErr auch zu euch gesandt alle seine Knechte,
die Propheten, fleißig; aber ihr habt nie hören wollen noch eure Ohren
neigen, dass ihr gehorchtet, \bibverse{5} da er sprach: Bekehret euch,
ein jeglicher von seinem bösen Wege und von eurem bösen Wesen, so sollt
ihr in dem Lande, das der HErr euch und euren Vätern gegeben hat, immer
und ewiglich bleiben. \footnote{\textbf{25:5} Jer 18,11} \bibverse{6}
Folget nicht anderen Göttern, dass ihr ihnen dienet und sie anbetet, auf
dass ihr mich nicht erzürnet durch eurer Hände Werk und ich euch Unglück
zufügen müsse.

\bibverse{7} Aber ihr wolltet mir nicht gehorchen, spricht der HErr, auf
dass ihr mich ja wohl erzürntet durch eurer Hände Werk zu eurem eigenen
Unglück.

\bibverse{8} Darum so spricht der HErr Zebaoth: Weil ihr denn meine
Worte nicht hören wollt, \bibverse{9} siehe, so will ich ausschicken und
kommen lassen alle Völker gegen Mitternacht, spricht der HErr, auch
meinen Knecht Nebukadnezar, den König zu Babel, und will sie bringen
über dieses Land und über die, die darin wohnen, und über alle diese
Völker, die umherliegen, und will sie verbannen und verstören und zum
Spott und zur ewigen Wüste machen, \bibverse{10} und will herausnehmen
allen fröhlichen Gesang, die Stimme des Bräutigams und der Braut, die
Stimme der Mühle und das Licht der Lampe, \footnote{\textbf{25:10} Jer
  16,9} \bibverse{11} dass dieses ganze Land wüst und zerstört liegen
soll. Und sollen diese Völker dem König zu Babel dienen siebzig Jahre.
\footnote{\textbf{25:11} Jer 29,10; 2Chr 36,21; Esr 1,1; Dan 9,2}

\bibverse{12} Wenn aber die siebzig Jahre um sind, will ich den König zu
Babel heimsuchen und dieses Volk, spricht der HErr, um ihre Missetat,
dazu das Land der Chaldäer, und will es zur ewigen Wüste machen.
\bibverse{13} Also will ich über dieses Land bringen alle meine Worte,
die ich geredet habe wider sie (nämlich alles, was in diesem Buch
geschrieben steht, das Jeremia geweissagt hat über alle Völker).
\bibverse{14} Und sie sollen auch großen Völkern und großen Königen
dienen. Also will ich ihnen vergelten nach ihrem Verdienst und nach den
Werken ihrer Hände.

\bibverse{15} Denn also spricht zu mir der HErr, der Gott Israels: Nimm
diesen Becher Wein voll Zorns von meiner Hand und schenke daraus allen
Völkern, zu denen ich dich sende, \footnote{\textbf{25:15} Jer 51,7; Jes
  51,17; Offb 14,10} \bibverse{16} dass sie trinken, taumeln und toll
werden vor dem Schwert, das ich unter sie schicken will.

\bibverse{17} Und ich nahm den Becher von der Hand des HErrn und
schenkte allen Völkern, zu denen mich der HErr sandte, \bibverse{18}
nämlich Jerusalem, den Städten Judas, ihren Königen und Fürsten, dass
sie wüst und zerstört liegen und ein Spott und Fluch sein sollen, wie es
denn heutigestages steht; \bibverse{19} auch Pharao, dem König in
Ägypten, samt seinen Knechten, seinen Fürsten und seinem ganzen Volk;
\bibverse{20} allen Ländern gegen Abend, allen Königen im Lande Uz,
allen Königen in der Philister Lande, samt Askalon, Gaza, Ekron und den
Übrigen zu Asdod; \bibverse{21} denen zu Edom, denen von Moab, den
Kindern Ammon; \bibverse{22} allen Königen zu Tyrus, allen Königen zu
Sidon, den Königen auf den Inseln jenseits des Meeres; \bibverse{23}
denen von Dedan, denen von Thema, denen von Bus und allen, die das Haar
rundumher abschneiden; \bibverse{24} allen Königen in Arabien, allen
Königen gegen Abend, die in der Wüste wohnen; \bibverse{25} allen
Königen in Simri, allen Königen in Elam, allen Königen in Medien;
\bibverse{26} allen Königen gegen Mitternacht, in der Nähe und Ferne,
einem mit dem anderen, und allen Königen auf Erden, die auf dem Erdboden
sind; und der König zu Sesach soll nach diesen trinken.

\bibverse{27} Und sprich zu ihnen: So spricht der HErr Zebaoth, der Gott
Israels: Trinket, dass ihr trunken werdet, speiet und niederfallt und
nicht aufstehen könnt vor dem Schwert, das ich unter euch schicken will.
\bibverse{28} Und wo sie den Becher nicht wollen von deiner Hand nehmen
und trinken, so sprich zu ihnen: Also spricht der HErr Zebaoth: Nun
sollt ihr trinken! \bibverse{29} Denn siehe, in der Stadt, die nach
meinem Namen genannt ist, fange ich an zu Plagen; und ihr solltet
ungestraft bleiben? Ihr sollt nicht ungestraft bleiben; denn ich rufe
das Schwert herbei über alle, die auf Erden wohnen, spricht der HErr
Zebaoth. \footnote{\textbf{25:29} Jer 49,12; 1Petr 4,17}

\bibverse{30} Und du sollst alle diese Worte ihnen weissagen und sprich
zu ihnen: Der HErr wird brüllen aus der Höhe und seinen Donner hören
lassen aus seiner heiligen Wohnung; er wird brüllen über seine Hürden;
er wird singen ein Lied wie die Weintreter über alle Einwohner des
Landes, des Hall erschallen wird bis an der Welt Ende. \footnote{\textbf{25:30}
  Joe 4,16; Am 1,2; Hos 11,10} \bibverse{31} Der HErr hat zu rechten mit
den Heiden und will mit allem Fleisch Gericht halten; die Gottlosen wird
er dem Schwert übergeben, spricht der HErr.

\bibverse{32} So spricht der HErr Zebaoth: Siehe, es wird eine Plage
kommen von einem Volk zum anderen, und ein großes Wetter wird erweckt
werden aus einem fernen Lande. \bibverse{33} Da werden die Erschlagenen
des HErrn zu derselben Zeit liegen von einem Ende der Erde bis ans
andere Ende; die werden nicht beklagt noch aufgehoben noch begraben
werden, sondern müssen auf dem Felde liegen und zu Dung werden.
\footnote{\textbf{25:33} Jer 7,33} \bibverse{34} Heulet nun, ihr Hirten,
und schreiet, wälzet euch in der Asche, ihr Gewaltigen über die Herde;
denn die Zeit ist hier, dass ihr geschlachtet und zerstreut werdet und
zerfallen müsst wie ein köstliches Gefäß. \footnote{\textbf{25:34} Jer
  23,1} \bibverse{35} Und die Hirten werden nicht fliehen können, und
die Gewaltigen über die Herde werden nicht entrinnen können.
\bibverse{36} Da werden die Hirten schreien, und die Gewaltigen über die
Herde werden heulen, dass der HErr ihre Weide so verwüstet hat
\bibverse{37} und ihre Auen, die so wohl standen, verderbt sind vor dem
grimmigen Zorn des HErrn. \bibverse{38} Er hat seine Hütte verlassen wie
ein junger Löwe, und ist also ihr Land zerstört vor dem Zorn des
Tyrannen und vor seinem grimmigen Zorn. \footnote{\textbf{25:38} Jer 4,7}

\hypertarget{section-5}{%
\section{26}\label{section-5}}

\bibverse{1} Im Anfang des Königreichs Jojakims, des Sohnes Josias, des
Königs in Juda, geschah dieses Wort vom HErrn und sprach: \bibverse{2}
So spricht der HErr: Tritt in den Vorhof am Hause des HErrn und predige
allen Städten Judas, die da hereingehen, anzubeten im Hause des HErrn,
alle Worte, die ich dir befohlen habe ihnen zu sagen, und tue nichts
davon; \bibverse{3} ob sie vielleicht hören wollen und sich bekehren,
ein jeglicher von seinem bösen Wesen, damit mich auch reuen möchte das
Übel, das ich gedenke ihnen zu tun um ihres bösen Wandels willen.
\bibverse{4} Und sprich zu ihnen: So spricht der HErr: Werdet ihr mir
nicht gehorchen, dass ihr in meinem Gesetz wandelt, das ich euch
vorgelegt habe, \bibverse{5} dass ihr hört auf die Worte meiner Knechte,
der Propheten, welche ich stets zu euch gesandt habe, und ihr doch nicht
hören wolltet: \footnote{\textbf{26:5} Jer 25,4} \bibverse{6} so will
ich's mit diesem Hause machen wie mit Silo und diese Stadt zum Fluch
allen Heiden auf Erden machen. \footnote{\textbf{26:6} Jer 7,12-14; 1Sam
  4,4; 1Sam 4,12}

\bibverse{7} Da nun die Priester, Propheten und alles Volk hörten
Jeremia, dass er solche Worte redete im Hause des HErrn, \bibverse{8}
und Jeremia nun ausgeredet hatte alles, was ihm der HErr befohlen hatte,
allem Volk zu sagen, griffen ihn die Priester, Propheten und das ganze
Volk und sprachen: Du musst sterben! \bibverse{9} Warum weissagst du im
Namen des HErrn und sagst: Es wird diesem Hause gehen wie Silo, und
diese Stadt soll so wüst werden, dass niemand mehr darin wohne? Und das
ganze Volk sammelte sich im Hause des HErrn wider Jeremia.

\bibverse{10} Da solches hörten die Fürsten Judas, gingen sie aus des
Königs Hause hinauf ins Haus des HErrn und setzten sich vor das neue Tor
des HErrn. \bibverse{11} Und die Priester und Propheten sprachen vor den
Fürsten und allem Volk: Dieser ist des Todes schuldig; denn er hat
geweissagt wider diese Stadt, wie ihr mit euren Ohren gehört habt.
\footnote{\textbf{26:11} Apg 6,13}

\bibverse{12} Aber Jeremia sprach zu allen Fürsten und zu allem Volk:
Der HErr hat mich gesandt, dass ich solches alles, was ihr gehört habt,
sollte weissagen wider dieses Haus und wider diese Stadt. \bibverse{13}
So bessert nun euer Wesen und Wandel und gehorcht der Stimme des HErrn,
eures Gottes, so wird den HErrn auch gereuen das Übel, das er wider euch
geredet hat. \bibverse{14} Siehe, ich bin in euren Händen; ihr mögt es
machen mit mir, wie es euch recht und gut dünkt. \bibverse{15} Doch
sollt ihr wissen: wo ihr mich tötet, so werdet ihr unschuldig Blut laden
auf euch selbst, auf diese Stadt und ihre Einwohner. Denn wahrlich, der
HErr hat mich zu euch gesandt, dass ich solches alles vor euren Ohren
reden soll.

\bibverse{16} Da sprachen die Fürsten und das ganze Volk zu den
Priestern und Propheten: Dieser ist des Todes nicht schuldig; denn er
hat zu uns geredet im Namen des HErrn, unseres Gottes.

\bibverse{17} Und es standen auf etliche der Ältesten im Lande und
sprachen zum ganzen Haufen des Volks: \bibverse{18} Zur Zeit Hiskias,
des Königs in Juda, war ein Prophet, Micha von Moreseth, und sprach zum
ganzen Volk Juda: So spricht der HErr Zebaoth: Zion wird wie ein Acker
gepflügt werden, und Jerusalem wird zum Steinhaufen werden und der Berg
des Tempels zu einer wilden Höhe. \footnote{\textbf{26:18} Mi 3,12}
\bibverse{19} Doch ließ ihn Hiskia, der König Judas, und das ganze Juda
darum nicht töten; ja sie fürchteten vielmehr den HErrn und beteten vor
dem HErrn. Da reute auch den HErrn das Übel, das er wider sie geredet
hatte. Darum täten wir sehr übel wider unsere Seelen. \footnote{\textbf{26:19}
  Jer 18,8}

\bibverse{20} So war auch einer, der im Namen des HErrn weissagte, Uria,
der Sohn Semajas, von Kirjath-Jearim. Derselbe weissagte wider diese
Stadt und wider dieses Land gleichwie Jeremia. \bibverse{21} Da aber der
König Jojakim und alle seine Gewaltigen und die Fürsten seine Worte
hörten, wollte ihn der König töten lassen. Und Uria erfuhr das,
fürchtete sich und floh und zog nach Ägypten. \bibverse{22} Aber der
König Jojakim schickte Leute nach Ägypten, Elnathan, den Sohn Achbors,
und andere mit ihm; \bibverse{23} die führten ihn aus Ägypten und
brachten ihn zum König Jojakim; der ließ ihn mit dem Schwert töten und
ließ seinen Leichnam unter dem gemeinen Pöbel begraben.

\bibverse{24} Aber mit Jeremia war die Hand Ahikams, des Sohnes Saphans,
dass er nicht dem Volk in die Hände kam, dass sie ihn töteten.
\footnote{\textbf{26:24} 2Kö 22,12}

\hypertarget{section-6}{%
\section{27}\label{section-6}}

\bibverse{1} Im Anfang des Königreichs Zedekias, des Sohnes Josias, des
Königs in Juda, geschah dieses Wort vom HErrn zu Jeremia und sprach:
\bibverse{2} So spricht der HErr zu mir: Mache dir ein Joch und hänge es
an deinen Hals \bibverse{3} und schicke es zum König in Edom, zum König
in Moab, zum König der Kinder Ammon, zum König zu Tyrus und zum König zu
Sidon durch die Boten, so zu Zedekia, dem König Judas, gen Jerusalem
gekommen sind, \footnote{\textbf{27:3} Jer 25,21-22} \bibverse{4} und
befiehl ihnen, dass sie ihren Herren sagen: So spricht der HErr Zebaoth,
der Gott Israels: So sollt ihr euren Herren sagen: \bibverse{5} Ich habe
die Erde gemacht und Menschen und Vieh, die auf Erden sind, durch meine
große Kraft und meinen ausgestreckten Arm und gebe sie, wem ich will.
\bibverse{6} Nun aber habe ich alle diese Lande gegeben in die Hand
meines Knechtes Nebukadnezar, des Königs zu Babel, und habe ihm auch die
wilden Tiere auf dem Felde gegeben, dass sie ihm dienen sollen.
\bibverse{7} Und sollen alle Völker dienen ihm und seinem Sohn und
seines Sohnes Sohn, bis dass die Zeit seines Landes auch komme und er
vielen Völkern und großen Königen diene. \footnote{\textbf{27:7} Jer
  25,12}

\bibverse{8} Welches Volk aber und Königreich dem König zu Babel,
Nebukadnezar, nicht dienen will, und wer seinen Hals nicht wird unter
das Joch des Königs zu Babel geben, solch Volk will ich heimsuchen mit
Schwert, Hunger und Pestilenz, spricht der HErr, bis dass ich sie durch
seine Hand umbringe. \bibverse{9} Darum so gehorcht nicht euren
Propheten, Weissagern, Traumdeutern, Tagewählern und Zauberern, die euch
sagen: Ihr werdet nicht dienen müssen dem König zu Babel. \bibverse{10}
Denn sie weissagen euch falsch, auf dass sie euch fern aus eurem Lande
bringen und ich euch ausstoße und ihr umkommt. \bibverse{11} Denn
welches Volk seinen Hals ergibt unter das Joch des Königs zu Babel und
dient ihm, das will ich in seinem Lande lassen, dass es dasselbe baue
und bewohne, spricht der HErr.

\bibverse{12} Und ich redete solches alles zu Zedekia, dem König Judas,
und sprach: Ergebt euren Hals unter das Joch des Königs zu Babel und
dient ihm und seinem Volk, so sollt ihr lebendig bleiben. \bibverse{13}
Warum wollt ihr sterben, du und dein Volk, durch Schwert, Hunger und
Pestilenz, wie denn der HErr geredet hat über das Volk, so dem König zu
Babel nicht dienen will? \bibverse{14} Darum gehorchet nicht den Worten
der Propheten, die euch sagen: „Ihr werdet nicht dienen müssen dem König
zu Babel``! Denn sie weissagen euch falsch, \footnote{\textbf{27:14} Jer
  27,9} \bibverse{15} und ich habe sie nicht gesandt, spricht der HErr;
sondern sie weissagen falsch in meinem Namen, auf dass ich euch ausstoße
und ihr umkommet samt den Propheten, die euch weissagen.

\bibverse{16} Und zu den Priestern und zu allem diesem Volk redete ich
und sprach: So spricht der HErr: Gehorchet nicht den Worten eurer
Propheten, die euch weissagen und sprechen: „Siehe, die Gefäße aus dem
Hause des HErrn werden nun bald von Babel wieder herkommen``! Denn sie
weissagen euch falsch. \bibverse{17} Gehorchet ihnen nicht, sondern
dienet dem König zu Babel, so werdet ihr lebendig bleiben. Warum soll
doch diese Stadt zur Wüste werden? \bibverse{18} Sind sie aber Propheten
und haben sie des HErrn Wort, so lasst sie vom HErrn Zebaoth erbitten,
dass die übrigen Gefäße im Hause des HErrn und im Hause des Königs in
Juda und zu Jerusalem nicht auch gen Babel geführt werden. \bibverse{19}
Denn also spricht der HErr Zebaoth von den Säulen und vom Meer und von
dem Gestühl und von den Gefäßen, die noch übrig sind in dieser Stadt,
\footnote{\textbf{27:19} Jer 52,17} \bibverse{20} welche Nebukadnezar,
der König zu Babel, nicht wegnahm, da er Jechonja, den Sohn Jojakims,
den König Judas, von Jerusalem wegführte gen Babel samt allen Fürsten in
Juda und Jerusalem, \footnote{\textbf{27:20} 2Kö 24,14-15} \bibverse{21}
-- denn so spricht der HErr Zebaoth, der Gott Israels, von den Gefäßen,
die noch übrig sind im Hause des HErrn und im Hause des Königs in Juda
und zu Jerusalem: \bibverse{22} Sie sollen gen Babel geführt werden und
daselbst bleiben bis auf den Tag, da ich sie heimsuche, spricht der
HErr, und ich sie wiederum herauf an diesen Ort bringen lasse.
\footnote{\textbf{27:22} 2Chr 36,22; Esr 1,7-11}

\hypertarget{section-7}{%
\section{28}\label{section-7}}

\bibverse{1} Und in demselben Jahr, im Anfang des Königreiches Zedekias,
des Königs in Juda, im fünften Monat des vierten Jahres, sprach Hananja,
der Sohn Assurs, ein Prophet von Gibeon, zu mir im Hause des HErrn, in
Gegenwart der Priester und alles Volks, und sagte: \bibverse{2} So
spricht der HErr Zebaoth, der Gott Israels: Ich habe das Joch des Königs
zu Babel zerbrochen; \bibverse{3} und ehe zwei Jahre um sind, will ich
alle Gefäße des Hauses des HErrn, welche Nebukadnezar, der König zu
Babel, hat von diesem Ort weggenommen und gen Babel geführt, wiederum an
diesen Ort bringen; \bibverse{4} Dazu Jechonja, den Sohn Jojakims, den
König Judas, samt allen Gefangenen aus Juda, die gen Babel geführt sind,
will ich auch wieder an diesen Ort bringen, spricht der HErr; denn ich
will das Joch des Königs zu Babel zerbrechen. \footnote{\textbf{28:4}
  Jer 27,20}

\bibverse{5} Da sprach der Prophet Jeremia zu dem Propheten Hananja in
der Gegenwart der Priester und des ganzen Volks, die im Hause des HErrn
standen, \bibverse{6} und sagte: Amen! der HErr tue also; der HErr
bestätige dein Wort, das du geweissagt hast, dass er die Gefäße aus dem
Hause des HErrn von Babel wieder bringe an diesen Ort samt allen
Gefangenen. \bibverse{7} Aber doch höre auch dieses Wort, das ich vor
deinen Ohren rede und vor den Ohren des ganzen Volks: \bibverse{8} Die
Propheten, die vor mir und vor dir gewesen sind von alters her, die
haben wider viel Länder und große Königreiche geweissagt von Krieg, von
Unglück und von Pestilenz; \bibverse{9} wenn aber ein Prophet von
Frieden weissagt, den wird man kennen, ob ihn der HErr wahrhaftig
gesandt hat, wenn sein Wort erfüllt wird.

\bibverse{10} Da nahm der Prophet Hananja das Joch vom Halse des
Propheten Jeremia und zerbrach's. \bibverse{11} Und Hananja sprach in
Gegenwart des ganzen Volks: So spricht der HErr: Ebenso will ich
zerbrechen das Joch Nebukadnezars, des Königs zu Babel, ehe zwei Jahre
um kommen, vom Halse aller Völker. Und der Prophet Jeremia ging seines
Weges. \footnote{\textbf{28:11} Jer 28,3}

\bibverse{12} Aber des HErrn Wort geschah zu Jeremia, nachdem der
Prophet Hananja das Joch zerbrochen hatte vom Halse des Propheten
Jeremia, und sprach: \bibverse{13} Gehe hin und sage Hananja: So spricht
der HErr: Du hast das hölzerne Joch zerbrochen und hast nun ein eisernes
Joch an jenes Statt gemacht. \bibverse{14} Denn so spricht der HErr
Zebaoth, der Gott Israels: Ein eisernes Joch habe ich allen diesen
Völkern an den Hals gehängt, damit sie dienen sollen Nebukadnezar, dem
König zu Babel, und müssen ihm dienen; denn ich habe ihm auch die wilden
Tiere gegeben.

\bibverse{15} Und der Prophet Jeremia sprach zum Propheten Hananja: Höre
doch, Hananja! Der HErr hat dich nicht gesandt, und du hast gemacht,
dass dieses Volk auf Lügen sich verlässt. \bibverse{16} Darum spricht
der HErr also: Siehe, ich will dich vom Erdboden nehmen; dieses Jahr
sollst du sterben; denn du hast sie mit deiner Rede vom HErrn
abgewendet. \footnote{\textbf{28:16} Jer 23,14; Jer 29,32}

\bibverse{17} Also starb der Prophet Hananja desselben Jahres im
siebenten Monat. \# 29 \bibverse{1} Dies sind die Worte in dem Brief,
den der Prophet Jeremia sandte von Jerusalem an die übrigen Ältesten,
die weggeführt waren, und an die Priester und Propheten und an das ganze
Volk, das Nebukadnezar von Jerusalem hatte weggeführt gen Babel
\bibverse{2} (nachdem der König Jechonja und die Königin mit den
Kämmerern und Fürsten in Juda und Jerusalem samt den Zimmerleuten und
Schmieden zu Jerusalem weg waren), \bibverse{3} durch Eleasa, den Sohn
Saphans, und Gemarja, den Sohn Hilkias, welche Zedekia, der König Judas,
sandte gen Babel zu Nebukadnezar, dem König zu Babel:

\bibverse{4} So spricht der HErr Zebaoth, der Gott Israels, zu allen
Gefangenen, die ich habe von Jerusalem lassen wegführen gen Babel:
\bibverse{5} Bauet Häuser, darin ihr wohnen mögt, pflanzet Gärten,
daraus ihr die Früchte essen mögt; \bibverse{6} nehmet Weiber und zeuget
Söhne und Töchter; nehmet euren Söhnen Weiber und gebet euren Töchtern
Männern, dass sie Söhne und Töchter zeugen; mehret euch daselbst, dass
euer nicht wenig sei. \bibverse{7} Suchet der Stadt Bestes, dahin ich
euch habe lassen wegführen, und betet für sie zum HErrn; denn wenn's ihr
wohl geht, so geht's euch auch wohl. \bibverse{8} Denn so spricht der
HErr Zebaoth, der Gott Israels: Lasst euch die Propheten, die bei euch
sind, und die Wahrsager nicht betrügen und gehorcht euren Träumen nicht,
die euch träumen. \footnote{\textbf{29:8} Jer 14,14} \bibverse{9} Denn
sie weissagen euch falsch in meinem Namen; ich habe sie nicht gesandt,
spricht der HErr. \bibverse{10} Denn so spricht der HErr: Wenn zu Babel
siebzig Jahre aus sind, so will ich euch besuchen und will mein gnädiges
Wort über euch erwecken, dass ich euch wieder an diesen Ort bringe.
\bibverse{11} Denn ich weiß wohl, was ich für Gedanken über euch habe,
spricht der HErr: Gedanken des Friedens und nicht des Leidens, dass ich
euch gebe das Ende, des ihr wartet. \bibverse{12} Und ihr werdet mich
anrufen und hingehen und mich bitten, und ich will euch erhören.
\bibverse{13} Ihr werdet mich suchen und finden. Denn so ihr mich von
ganzem Herzen suchen werdet, \footnote{\textbf{29:13} 5Mo 4,29; Jes 55,6}
\bibverse{14} so will ich mich von euch finden lassen, spricht der HErr,
und will euer Gefängnis wenden und euch sammeln aus allen Völkern und
von allen Orten, dahin ich euch verstoßen habe, spricht der HErr, und
will euch wiederum an diesen Ort bringen, von dem ich euch habe lassen
wegführen. \footnote{\textbf{29:14} Ps 126,4}

\bibverse{15} Zwar meinet ihr, der HErr habe euch zu Babel Propheten
erweckt. \bibverse{16} Aber also spricht der HErr vom König, der auf
Davids Stuhl sitzt, und von allem Volk, das in dieser Stadt wohnt, von
euren Brüdern, die nicht mit euch gefangen hinausgezogen sind,
\bibverse{17} -- ja, also spricht der HErr Zebaoth: Siehe, ich will
Schwert, Hunger und Pestilenz unter sie schicken und will mit ihnen
umgehen wie mit den schlechten Feigen, davor einen ekelt zu essen,
\footnote{\textbf{29:17} Jer 24,8} \bibverse{18} und will hinter ihnen
her sein mit Schwert, Hunger und Pestilenz und will sie in keinem
Königreich auf Erden bleiben lassen, dass sie sollen zum Fluch, zum
Wunder, zum Hohn und zum Spott unter allen Völkern werden, dahin ich sie
verstoßen werde, \footnote{\textbf{29:18} Jer 24,9-10} \bibverse{19}
darum dass sie meinen Worten nicht gehorchen, spricht der HErr, der ich
meine Knechte, die Propheten, zu euch stets gesandt habe; aber ihr
wolltet nicht hören, spricht der HErr. \footnote{\textbf{29:19} Jer 25,4}

\bibverse{20} Ihr aber alle, die ihr gefangen seid weggeführt, die ich
von Jerusalem habe gen Babel ziehen lassen, höret des HErrn Wort!
\footnote{\textbf{29:20} Jer 29,4} \bibverse{21} So spricht der HErr
Zebaoth, der Gott Israels, wider Ahab, den Sohn Kolajas, und wider
Zedekia, den Sohn Maasejas, die euch falsch weissagen in meinem Namen:
Siehe, ich will sie geben in die Hände Nebukadnezars, des Königs zu
Babel; der soll sie totschlagen lassen vor euren Augen, \footnote{\textbf{29:21}
  Jer 29,8} \bibverse{22} dass man wird aus ihnen einen Fluch machen
unter allen Gefangenen aus Juda, die zu Babel sind, und sagen: Der HErr
tue dir wie Zedekia und Ahab, welche der König zu Babel auf Feuer braten
ließ, \bibverse{23} darum dass sie eine Torheit in Israel begingen und
trieben Ehebruch mit ihrer Nächsten Weibern und predigten falsch in
meinem Namen, was ich ihnen nicht befohlen hatte. Solches weiß ich und
bezeuge es, spricht der HErr.

\bibverse{24} Und wider Semaja von Nehalam sollst du sagen:
\bibverse{25} So spricht der HErr Zebaoth, der Gott Israels: Darum dass
du unter deinem Namen hast Briefe gesandt an alles Volk, das zu
Jerusalem ist, und an den Priester Zephanja, den Sohn Maasejas, und an
alle Priester und gesagt: \bibverse{26} Der HErr hat dich zum Priester
gesetzt anstatt des Priesters Jojada, dass ihr sollt Aufseher sein im
Hause des HErrn über alle Wahnsinnigen und Weissager, dass du sie in den
Kerker und Stock legest. \footnote{\textbf{29:26} Hos 9,7} \bibverse{27}
Nun, warum strafst du denn nicht Jeremia von Anathoth, der euch
weissagt? \bibverse{28} darum dass er zu uns gen Babel geschickt hat und
lassen sagen: Es wird noch lange währen; bauet Häuser, darin ihr wohnet,
und pflanzet Gärten, dass ihr die Früchte davon esset.

\bibverse{29} (Denn Zephanja, der Priester, hatte denselben Brief
gelesen und den Propheten Jeremia lassen zuhören.) \bibverse{30} Darum
geschah des HErrn Wort zu Jeremia und sprach: \bibverse{31} Sende hin zu
allen Gefangenen und lass ihnen sagen: So spricht der HErr wider Semaja
von Nehalam: Darum dass euch Semaja weissagt, und ich habe ihn doch
nicht gesandt, und macht, dass ihr auf Lügen vertrauet, \bibverse{32}
darum spricht der HErr also: Siehe, ich will Semaja von Nehalam
heimsuchen samt seinem Samen, dass der Seinen keiner soll unter diesem
Volk bleiben, und soll das Gute nicht sehen, das ich meinem Volk tun
will, spricht der HErr; denn er hat sie mit seiner Rede vom HErrn
abgewendet. \# 30 \bibverse{1} Dies ist das Wort, das vom HErrn geschah
zu Jeremia: \bibverse{2} So spricht der HErr, der Gott Israels: Schreibe
dir alle Worte in ein Buch, die ich zu dir rede. \bibverse{3} Denn
siehe, es kommt die Zeit, spricht der HErr, dass ich das Gefängnis
meines Volkes Israel und Juda wenden will, spricht der HErr, und will
sie wiederbringen in das Land, das ich ihren Vätern gegeben habe, dass
sie es besitzen sollen. \footnote{\textbf{30:3} Jer 29,14}

\bibverse{4} Dies sind aber die Worte, welche der HErr redet von Israel
und Juda: \bibverse{5} So spricht der HErr: Wir hören ein Geschrei des
Schreckens; es ist eitel Furcht da und kein Friede. \bibverse{6}
Forschet doch und sehet, ob ein Mann gebären könne? Wie geht es denn zu,
dass ich alle Männer sehe ihre Hände auf ihren Hüften haben wie Weiber
in Kindsnöten und alle Angesichter so bleich sind? \bibverse{7} Es ist
ja ein großer Tag, und seinesgleichen ist nicht gewesen, und ist eine
Zeit der Angst in Jakob; doch soll ihm daraus geholfen werden.
\footnote{\textbf{30:7} Joe 2,11; Zeph 1,15} \bibverse{8} Es soll aber
geschehen zu derselben Zeit, spricht der HErr Zebaoth, dass ich sein
Joch von deinem Halse zerbrechen will und deine Bande zerreißen, dass er
nicht mehr den Fremden dienen muss, \footnote{\textbf{30:8} Jer 27,12}
\bibverse{9} sondern sie werden dem HErrn, ihrem Gott, dienen und ihrem
König David, welchen ich ihnen erwecken will. \footnote{\textbf{30:9}
  Jer 23,5; Hes 34,23} \bibverse{10} Darum fürchte du dich nicht, mein
Knecht Jakob, spricht der HErr, und entsetze dich nicht Israel. Denn
siehe, ich will dir helfen aus fernen Landen und deinem Samen aus dem
Lande ihres Gefängnisses, dass Jakob soll wiederkommen, in Frieden leben
und Genüge haben, und niemand soll ihn schrecken. \footnote{\textbf{30:10}
  Jer 46,27; Jes 44,2} \bibverse{11} Denn ich bin bei dir, spricht der
HErr, dass ich dir helfe. Denn ich will mit allen Heiden ein Ende
machen, dahin ich dich zerstreut habe; aber mit dir will ich nicht ein
Ende machen; züchtigen aber will ich dich mit Maßen, dass du dich nicht
für unschuldig haltest. \footnote{\textbf{30:11} Jer 10,24}

\bibverse{12} Denn also spricht der HErr: Dein Schade ist verzweifelt
böse, und deine Wunden sind unheilbar. \footnote{\textbf{30:12} Jer
  15,18} \bibverse{13} Deine Sache behandelt niemand, dass er dich
verbände; es kann dich niemand heilen. \bibverse{14} Alle deine
Liebhaber vergessen dein, fragen nichts darnach. Ich habe dich
geschlagen, wie ich einen Feind schlüge, mit unbarmherziger Staupe um
deiner großen Missetat und um deiner starken Sünden willen.
\bibverse{15} Was schreist du über deinen Schaden und über dein
verzweifelt böses Leiden? Habe ich dir doch solches getan um deiner
großen Missetat und um deiner starken Sünden willen. \bibverse{16} Darum
alle, die dich gefressen haben, sollen gefressen werden, und alle, die
dich geängstet haben, sollen alle gefangen werden; die dich beraubt
haben sollen beraubt werden, und alle, die dich geplündert haben, sollen
geplündert werden. \footnote{\textbf{30:16} Jes 33,1} \bibverse{17} Aber
dich will ich wieder gesund machen und deine Wunden heilen, spricht der
HErr, darum dass man dich nennt die Verstoßene, und Zion, nach der
niemand frage. \footnote{\textbf{30:17} Jer 33,6}

\bibverse{18} So spricht der HErr: Siehe, ich will das Gefängnis der
Hütten Jakobs wenden und mich über seine Wohnungen erbarmen, und die
Stadt soll wieder auf ihre Hügel gebaut werden, und der Tempel soll
stehen nach seiner Weise. \footnote{\textbf{30:18} Jer 30,3}
\bibverse{19} Und soll von dannen herausgehen Lob- und Freudengesang;
denn ich will sie mehren und nicht mindern, ich will sie herrlich machen
und nicht geringer. \bibverse{20} Ihre Söhne sollen sein gleichwie
vormals und ihre Gemeinde vor mir gedeihen; denn ich will heimsuchen
alle, die sie plagen. \bibverse{21} Und ihr Fürst soll aus ihnen
herkommen und ihr Herrscher von ihnen ausgehen, und er soll zu mir
nahen; denn wer ist der, der mit willigem Herzen zu mir naht? spricht
der HErr. \bibverse{22} Und ihr sollt mein Volk sein, und ich will euer
Gott sein. \footnote{\textbf{30:22} Jer 24,7} \bibverse{23} Siehe, es
wird ein Wetter des HErrn mit Grimm kommen; ein schreckliches Ungewitter
wird den Gottlosen auf den Kopf fallen. \footnote{\textbf{30:23} Jer
  23,19} \bibverse{24} Des HErrn grimmiger Zorn wird nicht nachlassen,
bis er tue und ausrichte, was er im Sinn hat; zur letzten Zeit werdet
ihr solches erfahren. \# 31 \bibverse{1} Zu derselben Zeit, spricht der
HErr, will ich aller Geschlechter Israels Gott sein, und sie sollen mein
Volk sein. \footnote{\textbf{31:1} Jer 31,33; Jer 24,7}

\bibverse{2} So spricht der HErr: Das Volk, so übriggeblieben ist vom
Schwert, hat Gnade gefunden in der Wüste; Israel zieht hin zu seiner
Ruhe.

\bibverse{3} Der HErr ist mir erschienen von ferne: Ich habe dich je und
je geliebt; darum habe ich dich zu mir gezogen aus lauter Güte.
\bibverse{4} Wohlan, ich will dich wiederum bauen, dass du sollst gebaut
heißen, du Jungfrau Israel; du sollst noch fröhlich pauken und
herausgehen an den Tanz. \bibverse{5} Du sollst wiederum Weinberge
pflanzen an den Bergen Samarias; pflanzen wird man sie und ihre Früchte
genießen. \bibverse{6} Denn es wird die Zeit noch kommen, dass die Hüter
an dem Gebirge Ephraim werden rufen: Wohlauf, und lasst uns hinaufgehen
gen Zion zu dem HErrn, unserem Gott!

\bibverse{7} Denn also spricht der HErr: Rufet über Jakob mit Freuden
und jauchzet über das Haupt unter den Heiden; rufet laut, rühmet und
sprecht: HErr, hilf deinem Volk, den Übrigen in Israel! \bibverse{8}
Siehe, ich will sie aus dem Lande der Mitternacht bringen und will sie
sammeln aus den Enden der Erde, Blinde und Lahme, Schwangere und
Kindbetterinnen, dass sie in großen Haufen wieder hierher kommen sollen.
\bibverse{9} Sie werden weinend kommen und betend, so will ich sie
leiten; ich will sie leiten an den Wasserbächen auf schlichtem Wege,
dass sie sich nicht stoßen; denn ich bin Israels Vater, so ist Ephraim
mein erstgeborener Sohn. \footnote{\textbf{31:9} 2Kor 6,18}
\bibverse{10} Höret, ihr Heiden, des HErrn Wort und verkündigt es fern
in die Inseln und sprecht: Der Israel zerstreut hat, der wird's auch
wieder sammeln und wird sie hüten wie ein Hirte sein Herde.
\bibverse{11} Denn der HErr wird Jakob erlösen und von der Hand des
Mächtigen erretten. \bibverse{12} Und sie werden kommen und auf der Höhe
Zion zu jauchzen und werden zu den Gaben des HErrn laufen, zum Getreide,
Most, Öl, und jungen Schafen und Ochsen, dass ihre Seele wird sein wie
ein wasserreicher Garten und sie nicht mehr bekümmert sein sollen.
\bibverse{13} Alsdann werden die Jungfrauen fröhlich am Reigen sein,
dazu die junge Mannschaft und die Alten miteinander. Denn ich will ihr
Trauern in Freude verkehren und sie trösten und sie erfreuen nach ihrer
Betrübnis. \bibverse{14} Und ich will der Priester Herz voller Freude
machen, und mein Volk soll meiner Gaben die Fülle haben, spricht der
HErr. \bibverse{15} So spricht der HErr: Man hört eine klägliche Stimme
und bitteres Weinen auf der Höhe; Rahel weint über ihre Kinder und will
sich nicht trösten lassen über ihre Kinder, denn es ist aus mit ihnen.
\footnote{\textbf{31:15} Mt 2,18}

\bibverse{16} Aber der HErr spricht also: Lass dein Schreien und Weinen
und die Tränen deiner Augen; denn deine Arbeit wird wohl belohnt werden,
spricht der HErr. Sie sollen wiederkommen aus dem Lande des Feindes;
\bibverse{17} und deine Nachkommen haben viel Gutes zu erwarten, spricht
der HErr; denn deine Kinder sollen wieder in ihre Grenze kommen.
\bibverse{18} Ich habe wohl gehört, wie Ephraim klagt: „Du hast mich
gezüchtigt, und ich bin auch gezüchtigt wie ein ungebändigtes Kalb;
bekehre mich du, so werde ich bekehrt; denn du, HErr, bist mein Gott.
\bibverse{19} Da ich bekehrt ward, tat ich Buße; denn nachdem ich
gewitzigt bin, schlage ich mich auf die Hüfte. Ich bin zu Schanden
geworden und stehe schamrot; denn ich muss leiden den Hohn meiner
Jugend.`` \bibverse{20} Ist nicht Ephraim mein teurer Sohn und mein
trautes Kind? Denn ich denke noch wohl daran, was ich ihm geredet habe;
darum bricht mir mein Herz gegen ihn, dass ich mich sein erbarmen muss,
spricht der HErr. \bibverse{21} Richte dir Denkmale auf, setze dir
Zeichen und richte dein Herz auf die gebahnte Straße, darauf du
gewandelt hast; kehre wieder, Jungfrau Israel, kehre dich wieder zu
diesen deinen Städten! \bibverse{22} Wie lange willst du in der Irre
gehen, du abtrünnige Tochter? Denn der HErr wird ein Neues im Lande
erschaffen: das Weib wird den Mann umgeben. \bibverse{23} So spricht der
HErr Zebaoth, der Gott Israels: Man wird noch dieses Wort wieder reden
im Lande Juda und in seinen Städten, wenn ich ihr Gefängnis wenden
werde: Der HErr segne dich, du Wohnung der Gerechtigkeit, du heiliger
Berg! \bibverse{24} Und Juda samt allen seinen Städten sollen darin
wohnen, dazu Ackerleute und die mit Herden umherziehen; \bibverse{25}
denn ich will die müden Seelen erquicken und die bekümmerten Seelen
sättigen. \footnote{\textbf{31:25} Mt 11,28}

\bibverse{26} Darüber bin ich aufgewacht und sah auf und hatte so sanft
geschlafen.

\bibverse{27} Siehe, es kommt die Zeit, spricht der HErr, dass ich das
Haus Israel und das Haus Juda besäen will mit Menschen und mit Vieh.
\bibverse{28} Und gleichwie ich über sie gewacht habe, auszureuten, zu
zerreißen, abzubrechen, zu verderben und zu plagen: also will ich über
sie wachen, zu bauen und zu pflanzen, spricht der HErr. \bibverse{29} Zu
derselben Zeit wird man nicht mehr sagen: „Die Väter haben Herlinge
gegessen, und der Kinder Zähne sind stumpf geworden``; \bibverse{30}
sondern ein jeglicher wird um seiner Missetat willen sterben, und
welcher Mensch Herlinge isst, dem sollen seine Zähne stumpf werden.

\bibverse{31} Siehe, es kommt die Zeit, spricht der HErr, da will ich
mit dem Hause Israel und mit dem Hause Juda einen neuen Bund machen;
\footnote{\textbf{31:31} Hebr 8,8-12} \bibverse{32} nicht wie der Bund
gewesen ist, den ich mit ihren Vätern machte, da ich sie bei der Hand
nahm, dass ich sie aus Ägyptenland führte, welchen Bund sie nicht
gehalten haben, und ich sie zwingen musste, spricht der HErr;
\bibverse{33} sondern das soll der Bund sein, den ich mit dem Hause
Israel machen will nach dieser Zeit, spricht der HErr: Ich will mein
Gesetz in ihr Herz geben und in ihren Sinn schreiben; und sie sollen
mein Volk sein, so will ich ihr Gott sein; \bibverse{34} und wird keiner
den anderen noch ein Bruder den anderen lehren und sagen: „Erkenne den
HErrn``, sondern sie sollen mich alle kennen, beide, klein und groß,
spricht der HErr. Denn ich will ihnen ihre Missetat vergeben und ihrer
Sünde nimmermehr gedenken. \footnote{\textbf{31:34} Jer 33,8; Jes 43,25}
\bibverse{35} So spricht der HErr, der die Sonne dem Tage zum Licht gibt
und den Mond und die Sterne nach ihrem Lauf der Nacht zum Licht; der das
Meer bewegt, dass seine Wellen brausen -- HErr Zebaoth ist sein Name --:
\bibverse{36} Wenn solche Ordnungen vergehen vor mir, spricht der HErr,
so soll auch aufhören der Same Israels, dass er nicht mehr ein Volk vor
mir sei ewiglich. \bibverse{37} So spricht der HErr: Wenn man den Himmel
oben kann messen und den Grund der Erde erforschen, so will ich auch
verwerfen den ganzen Samen Israels um alles, was sie tun, spricht der
HErr.

\bibverse{38} Siehe, es kommt die Zeit, spricht der HErr, dass die Stadt
des HErrn soll gebaut werden vom Turm Hananeel an bis ans Ecktor;
\footnote{\textbf{31:38} Sach 14,10} \bibverse{39} und die Richtschnur
wird neben demselben weiter herausgehen bis an den Hügel Gareb und sich
gen Goath wenden; \bibverse{40} und das ganze Tal der Leichen und der
Asche samt dem ganzen Acker bis an den Bach Kidron, bis zu der Ecke am
Rosstor gegen Morgen, wird dem HErrn heilig sein, dass es nimmermehr
zerrissen noch abgebrochen soll werden. \# 32 \bibverse{1} Dies ist das
Wort, das vom HErrn geschah zu Jeremia im zehnten Jahr Zedekias, des
Königs in Juda, welches ist das achtzehnte Jahr Nebukadnezars.
\bibverse{2} Dazumal belagerte das Heer des Königs zu Babel Jerusalem.
Aber der Prophet Jeremia lag gefangen im Vorhof des Gefängnisses am
Hause des Königs in Juda,

\bibverse{3} dahin Zedekia, der König Judas, ihn hatte lassen
verschließen und gesagt: Warum weissagst du und sprichst: So spricht der
HErr: Siehe, ich gebe diese Stadt in die Hände des Königs zu Babel, und
er soll sie gewinnen; \footnote{\textbf{32:3} Jer 21,7; Jer 27,6}
\bibverse{4} und Zedekia, der König Judas, soll den Chaldäern nicht
entrinnen, sondern ich will ihn dem König zu Babel in die Hände geben,
dass er mündlich mit ihm reden und mit seinen Augen ihn sehen soll.
\bibverse{5} Und er wird Zedekia gen Babel führen; da soll er auch
bleiben, bis dass ich ihn heimsuche, spricht der HErr; denn ob ihr schon
wider die Chaldäer streitet, soll euch doch nichts gelingen.

\bibverse{6} Und Jeremia sprach: Es ist des HErrn Wort geschehen zu mir
und spricht: \bibverse{7} Siehe, Hanameel, der Sohn Sallums, deines
Oheims, kommt zu dir und wird sagen: Kaufe du meinen Acker zu Anathoth;
denn du hast das nächste Freundrecht dazu, dass du ihn kaufen sollst.
\footnote{\textbf{32:7} 3Mo 25,25; Rt 4,3-4}

\bibverse{8} Also kam Hanameel, meines Oheims Sohn, wie der HErr gesagt
hatte, zu mir in den Hof des Gefängnisses und sprach zu mir: Kaufe doch
meinen Acker zu Anathoth, der im Lande Benjamin liegt; denn du hast
Erbrecht dazu, und du bist der nächste; kaufe du ihn! Da merkte ich,
dass es des HErrn Wort wäre,

\bibverse{9} und kaufte den Acker von Hanameel, meines Oheims Sohn, zu
Anathoth, und wog ihm das Geld dar, siebzehn Silberlinge. \bibverse{10}
Und ich schrieb einen Brief und versiegelte ihn und nahm Zeugen dazu und
wog das Geld dar auf einer Waage \bibverse{11} und nahm zu mir den
versiegelten Kaufbrief nach Recht und Gewohnheit und eine offene
Abschrift \bibverse{12} und gab den Kaufbrief Baruch, dem Sohn Nerias,
des Sohnes Maasejas, in Gegenwart Hanameels, meines Vetters, und der
Zeugen, die im Kaufbrief geschrieben standen, und aller Juden, die im
Hofe des Gefängnisses saßen,

\bibverse{13} und befahl Baruch vor ihren Augen und sprach:
\bibverse{14} So spricht der HErr Zebaoth, der Gott Israels: Nimm diese
Briefe, den versiegelten Kaufbrief samt dieser offenen Abschrift, und
lege sie in ein irdenes Gefäß, dass sie lange bleiben mögen.
\bibverse{15} Denn so spricht der HErr Zebaoth, der Gott Israels: Noch
soll man Häuser, Äcker und Weinberge kaufen in diesem Lande.

\bibverse{16} Und da ich den Kaufbrief hatte Baruch, dem Sohn Nerias,
gegeben, betete ich zum HErrn und sprach: \bibverse{17} Ach Herr HErr,
siehe, du hast Himmel und Erde gemacht durch deine große Kraft und durch
deinen ausgestreckten Arm, und ist kein Ding vor dir unmöglich;
\bibverse{18} der du wohltust vielen Tausenden und vergiltst die
Missetat der Väter in den Busen ihrer Kinder nach ihnen, du großer und
starker Gott; HErr Zebaoth ist dein Name; \footnote{\textbf{32:18} 2Mo
  20,5-6} \bibverse{19} groß von Rat und mächtig von Tat, und deine
Augen stehen offen über alle Wege der Menschenkinder, dass du einem
jeglichen gebest nach seinem Wandel und nach der Frucht seines Wesens;
\footnote{\textbf{32:19} Röm 2,6} \bibverse{20} der du in Ägyptenland
hast Zeichen und Wunder getan bis auf diesen Tag, an Israel und den
Menschen, und hast dir einen Namen gemacht, wie er heutigestages ist;
\bibverse{21} und hast dein Volk Israel aus Ägyptenland geführt durch
Zeichen und Wunder, durch eine mächtige Hand, durch ausgestreckten Arm
und durch großen Schrecken; \bibverse{22} und hast ihnen dieses Land
gegeben, welches du ihren Vätern geschworen hattest, dass du es ihnen
geben wolltest, ein Land, darin Milch und Honig fließt; \bibverse{23}
und da sie hineinkamen und es besaßen, gehorchten sie deiner Stimme
nicht, wandelten auch nicht nach deinem Gesetz; und alles, was du ihnen
gebotest, dass sie es tun sollten, das ließen sie; darum du auch ihnen
all dies Unglück ließest widerfahren;

\bibverse{24} siehe, diese Stadt ist belagert, dass sie gewonnen und vor
Schwert, Hunger und Pestilenz in der Chaldäer Hände, welche wider sie
streiten, gegeben werden muss; und wie du geredet hast, so geht es, das
siehest du, \bibverse{25} und du sprichst zu mir, Herr HErr: „Kaufe du
einen Acker um Geld und nimm Zeugen dazu``, so doch die Stadt in der
Chaldäer Hände gegeben wird.

\bibverse{26} Und des HErrn Wort geschah zu Jeremia und sprach:
\bibverse{27} Siehe, ich, der HErr, bin ein Gott alles Fleisches; sollte
mir etwas unmöglich sein? \footnote{\textbf{32:27} 4Mo 16,22; Jer 32,17}
\bibverse{28} Darum so spricht der HErr also: Siehe, ich gebe diese
Stadt in der Chaldäer Hände und in die Hand Nebukadnezars, des Königs zu
Babel; und er soll sie gewinnen. \footnote{\textbf{32:28} Jer 32,3}
\bibverse{29} Und die Chaldäer, die wider diese Stadt streiten, werden
hereinkommen und sie mit Feuer anstecken und verbrennen samt den
Häusern, wo sie auf den Dächern Baal geräuchert und anderen Göttern
Trankopfer geopfert haben, auf dass sie mich erzürnten. \footnote{\textbf{32:29}
  Jer 19,13}

\bibverse{30} Denn die Kinder Israel und die Kinder Juda haben von ihrer
Jugend auf getan, was mir übel gefällt; und die Kinder Israel haben mich
erzürnt durch ihrer Hände Werk, spricht der HErr. \bibverse{31} Denn
seitdem diese Stadt gebaut ist, bis auf diesen Tag, hat sie mich zornig
und grimmig gemacht, dass ich sie muss von meinem Angesicht wegtun
\bibverse{32} um all der Bosheit willen der Kinder Israel und der Kinder
Juda, die sie getan haben, dass sie mich erzürnten. Sie, ihre Könige,
Fürsten, Priester und Propheten und die in Juda und Jerusalem wohnen,
\bibverse{33} haben mir den Rücken und nicht das Angesicht zugekehrt,
wiewohl ich sie stets lehren ließ; aber sie wollten nicht hören noch
sich bessern. \bibverse{34} Dazu haben sie ihre Gräuel in das Haus
gesetzt, das von mir den Namen hat, dass sie es verunreinigten,
\bibverse{35} und haben die Höhen des Baal gebaut im Tal Ben-Hinnom,
dass sie ihre Söhne und Töchter dem Moloch verbrennten, davon ich ihnen
nichts befohlen habe und ist mir nie in den Sinn gekommen, dass sie
solchen Gräuel tun sollten, damit sie Juda also zu Sünden brächten.
\footnote{\textbf{32:35} Jer 7,31; Jer 19,5}

\bibverse{36} Und nun um deswillen spricht der HErr, der Gott Israels,
also von dieser Stadt, davon ihr sagt, dass sie werde vor Schwert,
Hunger und Pestilenz in die Hände des Königs zu Babel gegeben:
\bibverse{37} Siehe, ich will sie sammeln aus allen Landen, dahin ich
sie verstoße durch meinen Zorn, Grimm und große Ungnade, und will sie
wiederum an diesen Ort bringen, dass sie sollen sicher wohnen.
\bibverse{38} Und sie sollen mein Volk sein, so will ich ihr Gott sein;
\bibverse{39} und ich will ihnen einerlei Herz und Wesen geben, dass sie
mich fürchten sollen ihr Leben lang, auf dass es ihnen und ihren Kindern
nach ihnen wohl gehe; \footnote{\textbf{32:39} Hes 36,27} \bibverse{40}
und will einen ewigen Bund mit ihnen machen, dass ich nicht will
ablassen, ihnen Gutes zu tun; und will ihnen meine Furcht ins Herz
geben, dass sie nicht von mir weichen; \bibverse{41} und soll meine Lust
sein, dass ich ihnen Gutes tue; und ich will sie in diesem Lande
pflanzen treulich, von ganzem Herzen und von ganzer Seele.

\bibverse{42} Denn so spricht der HErr: Gleichwie ich über dieses Volk
habe kommen lassen all dies große Unglück, also will ich auch alles Gute
über sie kommen lassen, das ich ihnen verheißen habe. \bibverse{43} Und
sollen noch Äcker gekauft werden in diesem Lande, davon ihr sagt, es
werde wüst liegen, dass weder Leute noch Vieh darin bleiben, und es
werde in der Chaldäer Hände gegeben. \bibverse{44} Dennoch wird man
Äcker um Geld kaufen und verbriefen, versiegeln und bezeugen im Lande
Benjamin und um Jerusalem her und in den Städten Judas, in Städten auf
den Gebirgen, in Städten in Gründen und in Städten gegen Mittag; denn
ich will ihr Gefängnis wenden, spricht der HErr. \# 33 \bibverse{1} Und
des HErrn Wort geschah zu Jeremia zum andernmal, da er noch im Vorhof
des Gefängnisses verschlossen war, und sprach: \footnote{\textbf{33:1}
  Jer 32,2} \bibverse{2} So spricht der HErr, der solches macht, tut und
ausrichtet -- HErr ist sein Name --: \bibverse{3} Rufe mich an, so will
ich dir antworten und will dir anzeigen große und gewaltige Dinge, die
du nicht weißt. \bibverse{4} Denn so spricht der HErr, der Gott Israels,
von den Häusern dieser Stadt und von den Häusern der Könige Judas,
welche abgebrochen sind, Bollwerke zu machen zur Wehr, \bibverse{5} und
von denen, die hereingekommen sind, wider die Chaldäer zu streiten, dass
sie diese füllen müssen mit den Leichnamen der Menschen, welche ich in
meinem Zorn und Grimm erschlagen will; denn ich habe mein Angesicht vor
dieser Stadt verborgen um all ihrer Bosheit willen: \bibverse{6} Siehe,
ich will sie heilen und gesund machen und will ihnen Frieden und Treue
die Fülle gewähren. \bibverse{7} Denn ich will das Gefängnis Judas und
das Gefängnis Israels wenden und will sie bauen wie von Anfang
\footnote{\textbf{33:7} Jer 29,14; Jer 30,3} \bibverse{8} und will sie
reinigen von aller Missetat, damit sie wider mich gesündigt haben, und
will ihnen vergeben alle Missetaten, damit sie wider mich gesündigt und
übertreten haben. \footnote{\textbf{33:8} Jer 31,34} \bibverse{9} Und
das soll mir ein fröhlicher Name, Ruhm und Preis sein unter allen Heiden
auf Erden, wenn sie hören werden all das Gute, das ich ihnen tue. Und
sie werden sich verwundern und entsetzen über all dem Guten und über all
dem Frieden, den ich ihnen geben will.

\bibverse{10} So spricht der HErr: An diesem Ort, davon ihr sagt: Er ist
wüst, weil weder Leute noch Vieh in den Städten Judas und auf den Gassen
zu Jerusalem bleiben, die so verwüstet sind, dass weder Leute noch
Bürger noch Vieh darin sind, \bibverse{11} wird man dennoch wiederum
hören Geschrei von Freude und Wonne, die Stimme des Bräutigams und der
Braut und die Stimme derer, die da sagen: „Danket dem HErrn Zebaoth;
denn er ist freundlich, und seine Güte währet ewiglich``, wenn sie
Dankopfer bringen zum Hause des HErrn. Denn ich will des Landes
Gefängnis wenden wie von Anfang, spricht der HErr. \footnote{\textbf{33:11}
  Jer 7,34; Ps 106,1; Esr 3,11}

\bibverse{12} So spricht der HErr Zebaoth: An diesem Ort, der so wüst
ist, dass weder Leute noch Vieh darin sind, und in allen seinen Städten
werden dennoch wiederum Wohnungen sein der Hirten, die da Herden weiden.
\bibverse{13} In Städten auf den Gebirgen und in Städten in Gründen und
in Städten gegen Mittag, im Lande Benjamin und um Jerusalem her und in
Städten Judas sollen dennoch wiederum die Herden gezählt aus und ein
gehen, spricht der HErr.

\bibverse{14} Siehe, es kommt die Zeit, spricht der HErr, dass ich das
gnädige Wort erwecken will, welches ich dem Hause Israel und dem Hause
Juda geredet habe. \bibverse{15} In denselben Tagen und zu derselben
Zeit will ich dem David ein gerechtes Gewächs aufgehen lassen, und er
soll Recht und Gerechtigkeit anrichten auf Erden. \footnote{\textbf{33:15}
  Jer 23,5; Jes 4,2} \bibverse{16} Zu derselben Zeit soll Juda geholfen
werden und Jerusalem sicher wohnen, und man wird sie nennen: Der HErr
unsere Gerechtigkeit. \footnote{\textbf{33:16} Jer 23,6; 5Mo 33,28}

\bibverse{17} Denn so spricht der HErr: Es soll nimmermehr fehlen, es
soll einer von David sitzen auf dem Stuhl des Hauses Israel. \footnote{\textbf{33:17}
  2Sam 7,12; 1Kö 9,5} \bibverse{18} Desgleichen soll's nimmermehr
fehlen, es sollen Priester und Leviten sein vor mir, die da Brandopfer
tun und Speisopfer anzünden und Opfer schlachten ewiglich.

\bibverse{19} Und des HErrn Wort geschah zu Jeremia und sprach:
\bibverse{20} So spricht der HErr: Wenn mein Bund aufhören wird mit Tag
und Nacht, dass nicht Tag und Nacht sei zu seiner Zeit, \bibverse{21} so
wird auch mein Bund aufhören mit meinem Knechte David, dass er nicht
einen Sohn habe zum König auf seinem Stuhl, und mit den Leviten und
Priestern, meinen Dienern. \bibverse{22} Wie man des Himmels Heer nicht
zählen noch den Sand am Meer nicht messen kann, also will ich mehren den
Samen Davids, meines Knechtes, und die Leviten, die mir dienen.
\footnote{\textbf{33:22} 1Mo 15,5; 1Mo 22,17}

\bibverse{23} Und des HErrn Wort geschah zu Jeremia und sprach:
\bibverse{24} Hast du nicht gesehen, was dieses Volk redet und spricht:
„Hat doch der HErr auch die zwei Geschlechter verworfen, welche er
auserwählt hatte``; und lästern mein Volk, als sollten sie nicht mehr
mein Volk sein. \bibverse{25} So spricht der HErr: Halte ich meinen Bund
nicht mit Tag und Nacht noch die Ordnungen des Himmels und der Erde,
\bibverse{26} so will ich auch verwerfen den Samen Jakobs und Davids,
meines Knechtes, dass ich nicht aus ihrem Samen nehme, die da herrschen
über den Samen Abrahams, Isaaks und Jakobs. Denn ich will ihr Gefängnis
wenden und mich über sie erbarmen. \footnote{\textbf{33:26} Jer 32,44}

\hypertarget{section-8}{%
\section{34}\label{section-8}}

\bibverse{1} Dies ist das Wort, das vom HErrn geschah zu Jeremia, da
Nebukadnezar, der König zu Babel, samt allem seinem Heer und allen
Königreichen auf Erden, die unter seiner Gewalt waren, und allen Völkern
stritt wider Jerusalem und alle ihre Städte, und sprach: \bibverse{2} So
spricht der HErr, der Gott Israels: Gehe hin und sage Zedekia, dem König
Judas, und sprich zu ihm: So spricht der HErr: Siehe, ich will diese
Stadt in die Hände des Königs zu Babel geben, und er soll sie mit Feuer
verbrennen. \bibverse{3} Und du sollst seiner Hand nicht entrinnen,
sondern gegriffen und in seine Hand gegeben werden, dass du ihn mit
Augen sehen und mündlich mit ihm reden wirst, und gen Babel kommen.

\bibverse{4} Doch aber höre, Zedekia, du König Judas, des HErrn Wort: So
spricht der HErr von dir: Du sollst nicht durchs Schwert sterben,
\footnote{\textbf{34:4} Jer 52,11} \bibverse{5} sondern du sollst im
Frieden sterben. Und wie deinen Vätern, den vorigen Königen, die vor dir
gewesen sind, so wird man auch dir einen Brand anzünden und dich
beklagen: „Ach Herr!{}`` denn ich habe es geredet, spricht der HErr.
\footnote{\textbf{34:5} 2Chr 16,14; Jer 22,18}

\bibverse{6} Und der Prophet Jeremia redete alle diese Worte zu Zedekia,
dem König Judas, zu Jerusalem, \bibverse{7} da das Heer des Königs zu
Babel schon stritt wider Jerusalem und wider alle übrigen Städte Judas,
nämlich wider Lachis und Aseka; denn diese waren noch übriggeblieben von
den festen Städten Judas. \footnote{\textbf{34:7} 2Kö 25,1}

\bibverse{8} Dies ist das Wort, das vom HErrn geschah zu Jeremia,
nachdem der König Zedekia einen Bund gemacht hatte mit dem ganzen Volk
zu Jerusalem, ein Freijahr auszurufen, \footnote{\textbf{34:8} Jer 34,14}
\bibverse{9} dass ein jeglicher seinen Knecht und ein jeglicher seine
Magd, so Hebräer und Hebräerin wären, sollte freigeben, dass kein Jude
den anderen leibeigen hielte. \bibverse{10} Da gehorchten alle Fürsten
und alles Volk, die solchen Bund eingegangen waren, dass ein jeglicher
sollte seinen Knecht und seine Magd freigeben und sie nicht mehr
leibeigen halten, und gaben sie los. \bibverse{11} Aber darnach kehrten
sie sich um und forderten die Knechte und Mägde wieder zu sich, die sie
freigegeben hatten, und zwangen sie, dass sie Knechte und Mägde sein
mussten.

\bibverse{12} Da geschah des HErrn Wort zu Jeremia vom HErrn und sprach:
\bibverse{13} So spricht der HErr, der Gott Israels: Ich habe einen Bund
gemacht mit euren Vätern, da ich sie aus Ägyptenland, aus dem
Diensthause, führte und sprach: \bibverse{14} Im siebenten Jahr soll ein
jeglicher seinen Bruder, der ein Hebräer ist und sich ihm verkauft und
sechs Jahre gedient hat, frei von sich lassen. Aber eure Väter
gehorchten mir nicht und neigten ihre Ohren nicht. \footnote{\textbf{34:14}
  2Mo 21,2; 5Mo 15,12} \bibverse{15} So habt ihr euch heute bekehrt und
getan, was mir wohl gefiel, dass ihr ein Freijahr ließet ausrufen, ein
jeglicher seinem Nächsten; und habt darüber einen Bund gemacht vor mir
im Hause, das nach meinem Namen genannt ist. \bibverse{16} Aber ihr seid
umgeschlagen und entheiligt meinen Namen; und ein jeglicher fordert
seinen Knecht und seine Magd wieder, die ihr hattet freigegeben, dass
sie ihr selbst eigen wären, und zwingt sie nun, dass sie eure Knechte
und Mägde sein müssen.

\bibverse{17} Darum spricht der HErr also: Ihr gehorchtet mir nicht,
dass ihr ein Freijahr ausriefet ein jeglicher seinem Bruder und seinem
Nächsten; siehe, so rufe ich, spricht der HErr, euch ein Freijahr aus
zum Schwert, zur Pestilenz, zum Hunger, und will euch in keinem
Königreich auf Erden bleiben lassen. \bibverse{18} Und will die Leute,
die meinen Bund übertreten und die Worte des Bundes, den sie vor mir
gemacht haben, nicht halten, so machen wie das Kalb, das sie in zwei
Stücke geteilt haben und sind zwischen den Teilen hingegangen,
\bibverse{19} nämlich die Fürsten Judas, die Fürsten Jerusalems, die
Kämmerer, die Priester und das ganze Volk im Lande, die zwischen des
Kalbes Stücken hingegangen sind. \bibverse{20} Und will sie geben in
ihrer Feinde Hand und derer, die ihnen nach dem Leben stehen, dass ihre
Leichname sollen den Vögeln unter dem Himmel und den Tieren auf Erden
zur Speise werden. \footnote{\textbf{34:20} Jer 7,33}

\bibverse{21} Und Zedekia, den König Judas, und seine Fürsten will ich
geben in die Hände ihrer Feinde und derer, die ihnen nach dem Leben
stehen, und dem Heer des Königs zu Babel, die jetzt von euch abgezogen
sind. \bibverse{22} Denn siehe, ich will ihnen befehlen, spricht der
HErr, und will sie wieder vor diese Stadt bringen, und sollen wider sie
streiten und sie gewinnen und mit Feuer verbrennen; und ich will die
Städte Judas verwüsten, dass niemand mehr da wohnen soll. \# 35
\bibverse{1} Dies ist das Wort, das vom HErrn geschah zu Jeremia zur
Zeit Jojakims, des Sohnes Josias, des Königs in Juda, und sprach:
\bibverse{2} Gehe hin zu dem Hause der Rechabiter und rede mit ihnen und
führe sie in des HErrn Haus, in der Kapellen eine, und schenke ihnen
Wein. \footnote{\textbf{35:2} 1Chr 2,55}

\bibverse{3} Da nahm ich Jaasanja, den Sohn Jeremias, des Sohnes
Habazinjas, samt seinen Brüdern und allen seinen Söhnen und das ganze
Haus der Rechabiter \bibverse{4} und führte sie in des HErrn Haus, in
die Kapelle der Kinder Hanans, des Sohnes Jigdaljas, des Mannes Gottes,
welche neben der Fürstenkapelle ist, über der Kapelle Maasejas, des
Sohnes Sallums, des Torhüters. \bibverse{5} Und ich setzte den Kindern
von der Rechabiter Hause Becher voll Wein und Schalen vor und sprach zu
ihnen: Trinkt Wein!

\bibverse{6} Sie aber antworteten: Wir trinken nicht Wein; denn unser
Vater Jonadab, der Sohn Rechabs, hat uns geboten und gesagt: Ihr und
eure Kinder sollt nimmermehr Wein trinken \bibverse{7} und kein Haus
bauen, keinen Samen säen, keinen Weinberg pflanzen noch haben, sondern
sollt in Hütten wohnen euer Leben lang, auf dass ihr lange lebet in dem
Lande, darin ihr wallet. \bibverse{8} Also gehorchen wir der Stimme
unseres Vaters Jonadab, des Sohnes Rechabs, in allem, was er uns geboten
hat, dass wir keinen Wein trinken unser Leben lang, weder wir noch
unsere Weiber noch Söhne noch Töchter, \bibverse{9} und bauen auch keine
Häuser, darin wir wohnten, und haben weder Weinberge noch Äcker noch
Samen, \bibverse{10} sondern wohnen in Hütten und gehorchen und tun
alles, wie unser Vater Jonadab geboten hat. \bibverse{11} Als aber
Nebukadnezar, der König zu Babel, herauf ins Land zog, sprachen wir:
„Kommt, lasst uns gen Jerusalem ziehen vor dem Heer der Chaldäer und der
Syrer!{}`` und sind also zu Jerusalem geblieben.

\bibverse{12} Da geschah des HErrn Wort zu Jeremia und sprach:
\bibverse{13} So spricht der HErr Zebaoth, der Gott Israels: Gehe hin
und sprich zu denen in Juda und zu den Bürgern zu Jerusalem: Wollt ihr
euch denn nicht bessern, dass ihr meinen Worten gehorchet? spricht der
HErr. \bibverse{14} Die Worte Jonadabs, des Sohnes Rechabs, die er
seinen Kindern geboten hat, dass sie nicht sollen Wein trinken, werden
gehalten, und sie trinken keinen Wein bis auf diesen Tag, darum dass sie
ihres Vaters Gebot gehorchen. Ich aber habe stets euch predigen lassen;
doch gehorchtet ihr mir nicht. \bibverse{15} So habe ich auch stets zu
euch gesandt alle meine Knechte, die Propheten, und lassen sagen:
Bekehret euch, ein jeglicher von seinem bösen Wesen, und bessert euren
Wandel und folget nicht anderen Göttern nach, ihnen zu dienen, so sollt
ihr in dem Lande bleiben, welches ich euch und euren Vätern gegeben
habe. Aber ihr wolltet eure Ohren nicht neigen noch mir gehorchen,
\footnote{\textbf{35:15} Jer 25,4-7} \bibverse{16} so doch die Kinder
Jonadabs, des Sohnes Rechabs, haben ihres Vaters Gebot, das er ihnen
geboten hat, gehalten. Aber dieses Volk gehorchte mir nicht.

\bibverse{17} Darum so spricht der HErr, der Gott Zebaoth und der Gott
Israels: Siehe, ich will über Juda und über alle Bürger zu Jerusalem
kommen lassen all das Unglück, das ich wider sie geredet habe, darum
dass ich zu ihnen geredet habe und sie nicht wollen hören, dass ich
gerufen habe und sie mir nicht wollen antworten.

\bibverse{18} Und zum Hause der Rechabiter sprach Jeremia: So spricht
der HErr Zebaoth, der Gott Israels: Darum dass ihr dem Gebot eures
Vaters Jonadab habt gehorcht und alle seine Gebote gehalten und alles
getan, was er euch geboten hat, \bibverse{19} darum spricht der HErr
Zebaoth, der Gott Israels, also: Es soll dem Jonadab, dem Sohne Rechabs,
nimmer fehlen, es soll jemand von den Seinen allezeit vor mir stehen. \#
36 \bibverse{1} Im vierten Jahr Jojakims, des Sohnes Josias, des Königs
in Juda, geschah dieses Wort zu Jeremia vom HErrn und sprach:
\footnote{\textbf{36:1} Jer 25,1} \bibverse{2} Nimm ein Buch und
schreibe darein alle Reden, die ich zu dir geredet habe über Israel,
über Juda und alle Völker von der Zeit an, da ich zu dir geredet habe,
nämlich von der Zeit Josias an bis auf diesen Tag; \bibverse{3} ob
vielleicht die vom Hause Juda, wo sie hören all das Unglück, das ich
ihnen gedenke zu tun, sich bekehren wollten, ein jeglicher von seinem
bösen Wesen, damit ich ihnen ihre Missetat und Sünde vergeben könnte.

\bibverse{4} Da rief Jeremia Baruch, den Sohn Nerias. Derselbe Baruch
schrieb in ein Buch aus dem Munde Jeremias alle Reden des HErrn, die er
zu ihm geredet hatte. \footnote{\textbf{36:4} Jer 32,12} \bibverse{5}
Und Jeremia gebot Baruch und sprach: Ich bin gefangen, dass ich nicht
kann in des HErrn Haus gehen. \bibverse{6} Du aber gehe hinein und lies
das Buch, darein du des HErrn Reden aus meinem Munde geschrieben hast,
vor dem Volk im Hause des HErrn am Fasttage, und sollst sie auch lesen
vor den Ohren des ganzen Juda, die aus ihren Städten hereinkommen;
\bibverse{7} ob sie vielleicht sich mit Beten vor dem HErrn demütigen
wollten und sich bekehren, ein jeglicher von seinem bösen Wesen; denn
der Zorn und Grimm ist groß, davon der HErr wider dieses Volk geredet
hat.

\bibverse{8} Und Baruch, der Sohn Nerias, tat alles, wie ihm der Prophet
Jeremia befohlen hatte, dass er die Reden des HErrn aus dem Buche läse
im Hause des HErrn. \bibverse{9} Es begab sich aber im fünften Jahr
Jojakims, des Sohnes Josias, des Königs Judas, im neunten Monat, dass
man ein Fasten verkündigte vor dem HErrn allem Volk zu Jerusalem und
allem Volk, das aus den Städten Judas gen Jerusalem kommt. \bibverse{10}
Und Baruch las aus dem Buche die Reden Jeremias im Hause des HErrn, in
der Kapelle Gemarjas, des Sohnes Saphans, des Kanzlers, im oberen
Vorhof, vor dem neuen Tor am Hause des HErrn, vor dem ganzen Volk.

\bibverse{11} Da nun Michaja, der Sohn Gemarjas, des Sohnes Saphans,
alle Reden des HErrn gehört hatte aus dem Buche, \bibverse{12} ging er
hinab in des Königs Haus, in die Kanzlei. Und siehe, daselbst saßen alle
Fürsten: Elisama, der Kanzler, Delaja, der Sohn Semajas, Elnathan, der
Sohn Achbors, Gemarja, der Sohn Saphans, und Zedekia, der Sohn Hananjas,
samt allen Fürsten. \bibverse{13} Und Michaja zeigte ihnen an alle
Reden, die er gehört hatte, da Baruch las aus dem Buche vor den Ohren
des Volks. \bibverse{14} Da sandten alle Fürsten Judi, den Sohn
Nethanjas, des Sohnes Selemjas, des Sohnes Chusis, nach Baruch und
ließen ihm sagen: Nimm das Buch, daraus du vor dem Volk gelesen hast,
mit dir und komme! Und Baruch, der Sohn Nerias, nahm das Buch mit sich
und kam zu ihnen.

\bibverse{15} Und sie sprachen zu ihm: Setze dich und lies, dass wir's
hören! Und Baruch las ihnen vor ihren Ohren.

\bibverse{16} Und da sie alle die Reden hörten, entsetzten sie sich
einer gegen den anderen und sprachen zu Baruch: Wir wollen alle diese
Reden dem König anzeigen.

\bibverse{17} Und sie fragten den Baruch: Sage uns, wie hast du alle
diese Reden aus seinem Munde geschrieben?

\bibverse{18} Baruch sprach zu ihnen: Er sagte vor mir alle diese Reden
aus seinem Munde, und ich schrieb sie mit Tinte ins Buch.

\bibverse{19} Da sprachen die Fürsten zu Baruch: Gehe hin und verbirg
dich mit Jeremia, dass niemand wisse, wo ihr seid.

\bibverse{20} Sie aber gingen hinein zum König in den Vorhof und ließen
das Buch behalten in der Kammer Elisamas, des Kanzlers, und sagten vor
dem König an alle diese Reden. \bibverse{21} Da sandte der König den
Judi, das Buch zu holen. Der nahm es aus der Kammer Elisamas, des
Kanzlers. Und Judi las vor dem König und allen Fürsten, die bei dem
König standen. \bibverse{22} Der König aber saß im Winterhause, im
neunten Monat, vor dem Kamin. \bibverse{23} Wenn aber Judi drei oder
vier Blatt gelesen hatte, zerschnitt er's mit einem Schreibmesser und
warf's ins Feuer, das im Kaminherde war, bis das Buch ganz verbrannte im
Feuer -- \bibverse{24} und niemand entsetzte sich noch zerriss seine
Kleider, weder der König noch seine Knechte, so doch alle diese Reden
gehört hatten. \footnote{\textbf{36:24} 2Kö 22,11} \bibverse{25} Und
wiewohl Elnathan, Delaja und Gemarja den König baten, er wolle das Buch
nicht verbrennen, gehorchte er ihnen doch nicht. \bibverse{26} Dazu
gebot noch der König Jerahmeel, dem Königssohn, und Seraja, dem Sohn
Asriels, und Selemja, dem Sohn Abdeels, sie sollten Baruch, den
Schreiber, und Jeremia, den Propheten, greifen. Aber der HErr hatte sie
verborgen.

\bibverse{27} Da geschah des HErrn Wort zu Jeremia, nachdem der König
das Buch und die Reden, so Baruch hatte geschrieben aus dem Munde
Jeremias, verbrannt hatte, und sprach: \bibverse{28} Nimm dir wiederum
ein anderes Buch und schreib alle vorigen Reden darein, die im ersten
Buch standen, welches Jojakim, der König Judas, verbrannt hat,
\bibverse{29} und sage von Jojakim, dem König Judas: So spricht der
HErr: Du hast dies Buch verbrannt und gesagt: Warum hast du darein
geschrieben, dass der König von Babel werde kommen und dieses Land
verderben und machen, dass weder Leute noch Vieh darin mehr sein werden?
\footnote{\textbf{36:29} Jer 25,9-11; Jer 7,20; Jer 9,9} \bibverse{30}
Darum spricht der HErr von Jojakim, dem König Judas: Es soll keiner von
den Seinen auf dem Stuhl Davids sitzen, und sein Leichnam soll
hingeworfen des Tages in der Hitze und des Nachts im Frost liegen;
\footnote{\textbf{36:30} Jer 22,19} \bibverse{31} und ich will ihn und
seinen Samen und seine Knechte heimsuchen um ihrer Missetat willen; und
ich will über sie und über die Bürger zu Jerusalem und über die in Juda
kommen lassen all das Unglück, davon ich ihnen geredet habe, und sie
gehorchten doch nicht.

\bibverse{32} Da nahm Jeremia ein anderes Buch und gab's Baruch, dem
Sohn Nerias, dem Schreiber. Der schrieb darein aus dem Munde Jeremias
alle die Reden, die in dem Buch standen, das Jojakim, der König Judas,
hatte mit Feuer verbrennen lassen; und zu denselben wurden dergleichen
Reden noch viele hinzugetan. \# 37 \bibverse{1} Und Zedekia, der Sohn
Josias, ward König anstatt Jechonjas, des Sohnes Jojakims; denn
Nebukadnezar, der König zu Babel, machte ihn zum König im Lande Juda.
\footnote{\textbf{37:1} 2Kö 24,17} \bibverse{2} Aber er und seine
Knechte und das Volk im Lande gehorchten nicht des HErrn Worten, die er
durch den Propheten Jeremia redete.

\bibverse{3} Es sandte gleichwohl der König Zedekia Juchal, den Sohn
Selemjas, und Zephanja, den Sohn Maasejas, den Priester, zum Propheten
Jeremia und ließ ihm sagen: Bitte den HErrn, unseren Gott, für uns!

\bibverse{4} Denn Jeremia ging unter dem Volk aus und ein, und niemand
legte ihn ins Gefängnis. \bibverse{5} Es war aber das Heer Pharaos aus
Ägypten gezogen: und die Chaldäer, die vor Jerusalem lagen, da sie solch
Gerücht gehört hatten, waren von Jerusalem abgezogen.

\bibverse{6} Und des HErrn Wort geschah zum Propheten Jeremia und
sprach: \bibverse{7} So spricht der HErr, der Gott Israels: So sagt dem
König Judas, der euch zu mir gesandt hat, mich zu fragen: Siehe, das
Heer Pharaos, das euch zu Hilfe ist ausgezogen, wird wiederum heim nach
Ägypten ziehen; \bibverse{8} und die Chaldäer werden wiederkommen und
wider diese Stadt streiten und sie gewinnen und mit Feuer verbrennen.

\bibverse{9} Darum spricht der HErr also: Betrüget eure Seelen nicht,
dass ihr denkt, die Chaldäer werden von uns abziehen; sie werden nicht
abziehen. \bibverse{10} Und wenn ihr schon schlüget das ganze Heer der
Chaldäer, die wider euch streiten, und blieben ihrer etliche verwundet
übrig, so würden sie doch, ein jeglicher in seinem Gezelt, sich
aufmachen und diese Stadt mit Feuer verbrennen.

\bibverse{11} Als nun der Chaldäer Heer von Jerusalem war abgezogen um
des Heeres willen Pharaos, \bibverse{12} ging Jeremia aus Jerusalem und
wollte ins Land Benjamin gehen, seinen Acker in Besitz zu nehmen unter
dem Volk. \footnote{\textbf{37:12} Jer 32,9} \bibverse{13} Und da er
unter das Tor Benjamin kam, da war einer bestellt zum Torhüter, mit
Namen Jeria, der Sohn Selemjas, des Sohnes Hananjas; der griff den
Propheten Jeremia und sprach: Du willst zu den Chaldäern fallen.

\bibverse{14} Jeremia sprach: Das ist nicht wahr; ich will nicht zu den
Chaldäern fallen. Aber Jeria wollte ihn nicht hören, sondern griff
Jeremia und brachte ihn zu den Fürsten.

\bibverse{15} Und die Fürsten wurden zornig über Jeremia und ließen ihn
schlagen und warfen ihn ins Gefängnis im Hause Jonathans, des
Schreibers; den setzten sie zum Kerkermeister.

\bibverse{16} Also ging Jeremia in die Grube und den Kerker und lag
lange Zeit daselbst. \bibverse{17} Und Zedekia, der König, sandte hin
und ließ ihn holen und fragte ihn heimlich in seinem Hause und sprach:
Ist auch ein Wort vom HErrn vorhanden? Jeremia sprach: Ja; denn du wirst
dem König zu Babel in die Hände gegeben werden. \footnote{\textbf{37:17}
  Jer 34,21}

\bibverse{18} Und Jeremia sprach zum König Zedekia: Was habe ich wider
dich, wider deine Knechte und wider dieses Volk gesündigt, dass sie mich
in den Kerker geworfen haben?

\bibverse{19} Wo sind nun eure Propheten, die euch weissagten und
sprachen: Der König zu Babel wird nicht über euch noch über dieses Land
kommen? \bibverse{20} Und nun, mein Herr König, höre mich und lass meine
Bitte vor dir gelten und lass mich nicht wieder in Jonathans, des
Schreibers, Haus bringen, dass ich nicht sterbe daselbst.

\bibverse{21} Da befahl der König Zedekia, dass man Jeremia im Vorhof
des Gefängnisses behalten sollte und ließ ihm des Tages ein Laiblein
Brot geben aus der Bäckergasse, bis dass alles Brot in der Stadt
aufgezehrt war. Also blieb Jeremia im Vorhof des Gefängnisses. \# 38
\bibverse{1} Es hörten aber Sephatja, der Sohn Matthans, und Gedalja,
der Sohn Pashurs, und Juchal, der Sohn Selemjas, und Pashur, der Sohn
Malchias, die Reden, so Jeremia zu allem Volk redete und sprach:
\footnote{\textbf{38:1} Jer 21,1} \bibverse{2} So spricht der HErr: Wer
in dieser Stadt bleibt, der wird durch Schwert, Hunger und Pestilenz
sterben müssen; wer aber hinausgeht zu den Chaldäern, der soll lebend
bleiben und wird sein Leben wie eine Beute davonbringen. \footnote{\textbf{38:2}
  Jer 21,9} \bibverse{3} Denn also spricht der HErr: Diese Stadt soll
übergeben werden dem Heer des Königs zu Babel, und sie sollen sie
gewinnen.

\bibverse{4} Da sprachen die Fürsten zum König: Lass doch diesen Mann
töten; denn mit der Weise wendet er die Kriegsleute ab, die noch übrig
sind in dieser Stadt, desgleichen das ganze Volk auch, weil er solche
Worte zu ihnen sagt. Denn der Mann sucht nicht, was diesem Volk zum
Frieden, sondern was zum Unglück dient. \footnote{\textbf{38:4} Am 7,10}

\bibverse{5} Der König Zedekia sprach: Siehe, er ist in euren Händen;
denn der König kann nichts wider euch.

\bibverse{6} Da nahmen sie Jeremia und warfen ihn in die Grube Malchias,
des Königssohns, die am Vorhof des Gefängnisses war, und ließen ihn an
Seilen hinab in die Grube, da nicht Wasser, sondern Schlamm war; und
Jeremia sank in den Schlamm.

\bibverse{7} Als aber Ebed-Melech, der Mohr, ein Kämmerer in des Königs
Hause, hörte, dass man Jeremia hatte in die Grube geworfen, und der
König eben saß im Tor Benjamin, \bibverse{8} da ging Ebed-Melech aus des
Königs Hause und redete mit dem König und sprach: \bibverse{9} Mein Herr
König, die Männer handeln übel an dem Propheten Jeremia, dass sie ihn
haben in die Grube geworfen, da er muss Hungers sterben; denn es ist
kein Brot mehr in der Stadt.

\bibverse{10} Da befahl der König Ebed-Melech, dem Mohren, und sprach:
Nimm dreißig Männer mit dir von diesen und zieh den Propheten Jeremia
aus der Grube, ehe denn er sterbe.

\bibverse{11} Und Ebed-Melech nahm die Männer mit sich und ging in des
Königs Haus unter die Schatzkammer und nahm daselbst zerrissene und
vertragene alte Lumpen und ließ sie an einem Seil hinab zu Jeremia in
die Grube. \bibverse{12} Und Ebed-Melech, der Mohr, sprach zu Jeremia:
Lege diese zerrissenen und vertragenen alten Lumpen unter deine Achseln
um das Seil. Und Jeremia tat also.

\bibverse{13} Und sie zogen Jeremia herauf aus der Grube an den
Stricken; und blieb also Jeremia im Vorhof des Gefängnisses.

\bibverse{14} Und der König Zedekia sandte hin und ließ den Propheten
Jeremia zu sich holen unter den dritten Eingang am Hause des HErrn. Und
der König sprach zu Jeremia: Ich will dich etwas fragen; verhalte mir
nichts.

\bibverse{15} Jeremia sprach zu Zedekia: Sage ich dir etwas, so tötest
du mich doch; gebe ich dir aber einen Rat, so gehorchst du mir nicht.

\bibverse{16} Da schwur der König Zedekia dem Jeremia heimlich und
sprach: So wahr der HErr lebt, der uns dieses Leben gegeben hat, so will
ich dich nicht töten noch den Männern in die Hände geben, die dir nach
deinem Leben stehen. \footnote{\textbf{38:16} Jer 38,4-5}

\bibverse{17} Und Jeremia sprach zu Zedekia: So spricht der HErr, der
Gott Zebaoth, der Gott Israels: Wirst du hinausgehen zu den Fürsten des
Königs zu Babel, so sollst du leben bleiben, und diese Stadt soll nicht
verbrannt werden, sondern du und dein Haus sollen am Leben bleiben;
\bibverse{18} wirst du aber nicht hinausgehen zu den Fürsten des Königs
zu Babel, so wird diese Stadt den Chaldäern in die Hände gegeben, und
sie werden sie mit Feuer verbrennen, und du wirst auch nicht ihren
Händen entrinnen.

\bibverse{19} Der König Zedekia sprach zu Jeremia: Ich sorge mich aber,
dass ich den Juden, so zu den Chaldäern gefallen sind, möchte übergeben
werden, dass sie mein spotten.

\bibverse{20} Jeremia sprach: Man wird dich nicht übergeben. Gehorche
doch der Stimme des HErrn, die ich dir sage, so wird dir's wohl gehen,
und du wirst lebend bleiben. \bibverse{21} Wirst du aber nicht
hinausgehen, so ist dies das Wort, das mir der HErr gezeigt hat:
\bibverse{22} Siehe, alle Weiber, die noch vorhanden sind in dem Hause
des Königs in Juda, werden hinaus müssen zu den Fürsten des Königs zu
Babel; diese werden dann sagen: Ach deine Tröster haben dich überredet
und verführt und in Schlamm geführt und lassen dich nun stecken.

\bibverse{23} Also werden dann alle deine Weiber und Kinder hinaus
müssen zu den Chaldäern, und du selbst wirst ihren Händen nicht
entgehen; sondern du wirst vom König zu Babel gegriffen, und diese Stadt
wird mit Feuer verbrannt werden. \footnote{\textbf{38:23} Jer 32,4; Jer
  34,3}

\bibverse{24} Und Zedekia sprach zu Jeremia: Siehe zu, dass niemand
diese Rede erfahre, so wirst du nicht sterben. \bibverse{25} Und wenn's
die Fürsten erführen, dass ich mit dir geredet habe, und kämen zu dir
und sprächen: Sage an, was hast du mit dem König geredet -- leugne es
uns nicht, so wollen wir dich nicht töten --, und was hat der König mit
dir geredet? \bibverse{26} so sprich: Ich habe den König gebeten, dass
er mich nicht wiederum ließe in des Jonathan Haus führen; ich möchte
daselbst sterben.

\bibverse{27} Da kamen alle Fürsten zu Jeremia und fragten ihn; und er
sagte ihnen, wie ihm der König befohlen hatte. Da ließen sie von ihm,
weil sie nichts erfahren konnten.

\bibverse{28} Und Jeremia blieb im Vorhof des Gefängnisses bis auf den
Tag, da Jerusalem gewonnen ward. \# 39 \bibverse{1} Und es geschah, dass
Jerusalem gewonnen ward. Denn im neunten Jahr Zedekias, des Königs in
Juda, im zehnten Monat, kam Nebukadnezar, der König zu Babel, und all
sein Heer vor Jerusalem und belagerten es. \bibverse{2} Und im elften
Jahr Zedekias, am neunten Tage des vierten Monats, brach man in die
Stadt; \bibverse{3} und zogen hinein alle Fürsten des Königs zu Babel
und hielten unter dem Mitteltor, nämlich Nergal-Sarezer, Samgar-Nebo,
Sarsechim, der oberste Kämmerer, Nergal-Sarezer, der Oberste der Weisen,
und alle anderen Fürsten des Königs zu Babel. \bibverse{4} Als sie nun
Zedekia, der König Judas, sah samt seinen Kriegsleuten, flohen sie bei
Nacht zur Stadt hinaus bei des Königs Garten durchs Tor zwischen den
zwei Mauern und zogen des Weges zum blachen Feld.

\bibverse{5} Aber der Chaldäer Kriegsleute jagten ihnen nach und
ergriffen Zedekia im Felde bei Jericho und fingen ihn und brachten ihn
zu Nebukadnezar, dem König zu Babel, gen Ribla, das im Lande Hamath
liegt; der sprach ein Urteil über ihn. \bibverse{6} Und der König zu
Babel ließ die Söhne Zedekias vor seinen Augen töten zu Ribla und tötete
alle Fürsten Judas. \bibverse{7} Aber Zedekia ließ er die Augen
ausstechen und ihn in Ketten binden, dass er ihn gen Babel führte.

\bibverse{8} Und die Chaldäer verbrannten beide, des Königs Haus und der
Bürger Häuser, und zerbrachen die Mauern zu Jerusalem. \bibverse{9} Was
aber noch von Volk in der Stadt war, und was sonst zu ihnen gefallen
war, die führte Nebusaradan, der Hauptmann der Trabanten, alle
miteinander gen Babel gefangen. \bibverse{10} Aber von dem geringen
Volk, das nichts hatte, ließ zu derselben Zeit Nebusaradan, der
Hauptmann, etliche im Lande Juda und gab ihnen Weinberge und Felder.

\bibverse{11} Aber Nebukadnezar, der König zu Babel, hatte Nebusaradan,
dem Hauptmann, befohlen von Jeremia und gesagt: \bibverse{12} Nimm ihn
und lass ihn dir befohlen sein und tu ihm kein Leid; sondern wie er's
von dir begehrt, so mache es mit ihm.

\bibverse{13} Da sandten hin Nebusaradan, der Hauptmann, und Nebusasban,
der oberste Kämmerer, Nergal-Sarezer, der Oberste der Weisen, und alle
Fürsten des Königs zu Babel \bibverse{14} und ließen Jeremia holen aus
dem Vorhof des Gefängnisses und befahlen ihn Gedalja, dem Sohn Ahikams,
des Sohnes Saphans, dass er ihn hinaus in sein Haus führte. Und er blieb
bei dem Volk. \footnote{\textbf{39:14} Jer 38,28; Jer 40,5-6}

\bibverse{15} Es war auch des HErrn Wort geschehen zu Jeremia, als er
noch im Vorhof des Gefängnisses gefangen lag, und hatte gesprochen:
\bibverse{16} Gehe hin und sage Ebed-Melech, dem Mohren: So spricht der
HErr Zebaoth, der Gott Israels: Siehe, ich will meine Worte kommen
lassen über diese Stadt zum Unglück und zu keinem Guten, und du sollst
es sehen zur selben Zeit. \bibverse{17} Aber dich will ich erretten zur
selben Zeit, spricht der HErr, und sollst den Leuten nicht zuteil
werden, vor welchen du dich fürchtest. \bibverse{18} Denn ich will dir
davonhelfen, dass du nicht durchs Schwert fällst, sondern sollst dein
Leben wie eine Beute davonbringen, darum dass du mir vertraut hast,
spricht der HErr. \footnote{\textbf{39:18} Hi 5,20}

\hypertarget{section-9}{%
\section{40}\label{section-9}}

\bibverse{1} Dies ist das Wort, das vom HErrn geschah zu Jeremia, da ihn
Nebusaradan, der Hauptmann, losließ zu Rama; denn er war auch mit Ketten
gebunden unter allen denen, die zu Jerusalem und in Juda gefangen waren,
dass man sie gen Babel wegführen sollte. \footnote{\textbf{40:1} Jer
  39,11-14} \bibverse{2} Da nun der Hauptmann Jeremia zu sich hatte
lassen holen, sprach er zu ihm: Der HErr, dein Gott, hat dies Unglück
über diese Stätte geredet \bibverse{3} und hat's auch kommen lassen und
getan, wie er geredet hat; denn ihr habt gesündigt wider den HErrn und
seiner Stimme nicht gehorcht; darum ist euch solches widerfahren.
\bibverse{4} Und nun siehe, ich habe dich heute losgemacht von den
Ketten, womit deine Hände gebunden waren. Gefällt dir's, mit mir gen
Babel zu ziehen, so komm du sollst mir befohlen sein; gefällt dir's aber
nicht, mit mir gen Babel zu ziehen, so lass es anstehen. Siehe, da hast
du das ganze Land vor dir; wo dich's gut dünkt und dir gefällt, da zieh
hin. \bibverse{5} Denn weiter hinaus wird kein Wiederkehren sein. Darum
magst du umkehren zu Gedalja, dem Sohn Ahikams, des Sohnes Saphans,
welchen der König zu Babel gesetzt hat über die Städte in Juda, und bei
ihm unter dem Volk bleiben; oder gehe, wohin dir's wohl gefällt. Und der
Hauptmann gab ihm Zehrung und Geschenke und ließ ihn gehen. \footnote{\textbf{40:5}
  Jer 39,14}

\bibverse{6} Also kam Jeremia zu Gedalja, dem Sohn Ahikams, gen Mizpa
und blieb bei ihm unter dem Volk, das im Lande noch übrig war.

\bibverse{7} Da nun die Hauptleute, die auf dem Felde sich hielten, samt
ihren Leuten erfuhren, dass der König zu Babel hatte Gedalja, den Sohn
Ahikams, über das Land gesetzt und über die Männer und Weiber, Kinder
und die Geringen im Lande, welche nicht gen Babel geführt waren,
\bibverse{8} kamen sie zu Gedalja gen Mizpa, nämlich Ismael, der Sohn
Nethanjas, Johanan und Jonathan, die Söhne Kareahs, und Seraja, der Sohn
Thanhumeths, und die Söhne Ephais von Netopha und Jesanja, der Sohn
eines Maachathiters, samt ihren Männern. \footnote{\textbf{40:8} Jer
  41,1; Jer 41,11} \bibverse{9} Und Gedalja, der Sohn Ahikams, des
Sohnes Saphans, tat ihnen und ihren Männern einen Eid und sprach:
Fürchtet euch nicht, dass ihr den Chaldäern untertan sein sollt; bleibt
im Lande und seid dem König zu Babel untertan, so wird's euch wohl
gehen. \bibverse{10} Siehe, ich wohne hier zu Mizpa, dass ich den
Chaldäern diene, die zu uns kommen; darum so sammelt ein Wein und Feigen
und Öl und legt's in eure Gefäße und wohnt in euren Städten, die ihr
bekommen habt.

\bibverse{11} Auch allen Juden, die im Lande Moab und der Kinder Ammon
und in Edom und in allen Ländern waren, da sie hörten, dass der König zu
Babel hätte lassen etliche in Juda übrigbleiben und über sie gesetzt
Gedalja, den Sohn Ahikams, des Sohnes Saphans, \bibverse{12} kamen sie
alle wieder von allen Orten dahin sie verstoßen waren, in das Land Juda
zu Gedalja gen Mizpa und sammelten ein sehr viel Wein und Sommerfrüchte.

\bibverse{13} Aber Johanan, der Sohn Kareahs, samt allen den
Hauptleuten, die auf dem Felde sich gehalten hatten, kamen zu Gedalja
gen Mizpa \bibverse{14} und sprachen zu ihm: Weißt du auch, dass Baalis,
der König der Kinder Ammon, gesandt hat Ismael, den Sohn Nethanjas, dass
er dich soll erschlagen? Das wollte ihnen aber Gedalja, der Sohn
Ahikams, nicht glauben.

\bibverse{15} Da sprach Johanan, der Sohn Kareahs, zu Gedalja heimlich
zu Mizpa: Ich will hingehen und Ismael, den Sohn Nethanjas, erschlagen,
dass es niemand erfahren soll. Warum soll er dich erschlagen, dass alle
Juden, so zu dir versammelt sind, zerstreut werden und die noch aus Juda
übriggeblieben sind, umkommen?

\bibverse{16} Aber Gedalja, der Sohn Ahikams, sprach zu Johanan, dem
Sohn Kareahs: Du sollst das nicht tun; es ist nicht wahr, was du von
Ismael sagst. \# 41 \bibverse{1} Aber im siebenten Monat kam Ismael, der
Sohn Nethanjas, des Sohnes Elisamas, aus königlichem Stamm, einer von
den Obersten des Königs, und zehn Männer mit ihm zu Gedalja, dem Sohn
Ahikams, gen Mizpa, und sie aßen daselbst zu Mizpa miteinander.
\bibverse{2} Und Ismael, der Sohn Nethanjas, machte sich auf samt den
zehn Männern, die bei ihm waren, und schlugen Gedalja, den Sohn Ahikams,
des Sohnes Saphans, mit dem Schwert zu Tode, darum dass ihn der König zu
Babel über das Land gesetzt hatte; \footnote{\textbf{41:2} Jer 40,5}
\bibverse{3} dazu alle Juden, die bei Gedalja waren zu Mizpa, und die
Chaldäer, die sie daselbst fanden, alle Kriegsleute, schlug Ismael.

\bibverse{4} Des anderen Tages, nachdem Gedalja erschlagen war und es
noch niemand wusste, \bibverse{5} kamen achtzig Männer von Sichem, von
Silo und von Samaria und hatten die Bärte abgeschoren und ihre Kleider
zerrissen und sich zerritzt und trugen Speisopfer und Weihrauch mit
sich, dass sie es brächten zum Hause des HErrn. \bibverse{6} Und Ismael,
der Sohn Nethanjas, ging heraus von Mizpa ihnen entgegen, ging daher und
weinte. Als er nun an sie kam, sprach er zu ihnen: Ihr sollt zu Gedalja,
dem Sohn Ahikams, kommen. \bibverse{7} Da sie aber mitten in die Stadt
kamen, ermordete sie Ismael, der Sohn Nethanjas, und die Männer, die bei
ihm waren, und warf sie in den Brunnen. \bibverse{8} Aber es waren zehn
Männer darunter, die sprachen zu Ismael: Töte uns nicht; wir haben
Vorrat im Acker liegen von Weizen, Gerste, Öl und Honig. Also ließ er ab
und tötete sie nicht mit den anderen.

\bibverse{9} Der Brunnen aber, darein Ismael die Leichname der Männer
warf, welche er hatte erschlagen samt dem Gedalja, ist der, den der
König Asa machen ließ wider Baesa, den König Israels; den füllte Ismael,
der Sohn Nethanjas, mit den Erschlagenen. \footnote{\textbf{41:9} 1Kö
  15,16; 1Kö 15,22}

\bibverse{10} Und was übriges Volks war zu Mizpa, auch die
Königstöchter, führte Ismael, der Sohn Nethanjas, gefangen weg samt
allem übrigen Volk zu Mizpa, über welche Nebusaradan, der Hauptmann,
hatte gesetzt Gedalja, den Sohn Ahikams, und zog hin und wollte hinüber
zu den Kindern Ammon.

\bibverse{11} Da aber Johanan, der Sohn Kareahs, erfuhr und alle
Hauptleute des Heeres, die bei ihm waren, all das Übel, das Ismael, der
Sohn Nethanjas, begangen hatte, \bibverse{12} nahmen sie zu sich alle
Männer und zogen hin, wider Ismael, den Sohn Nethanjas, zu streiten; und
trafen ihn an dem großen Wasser bei Gibeon. \bibverse{13} Da nun alles
Volk, das bei Ismael war, sah den Johanan, den Sohn Kareahs, und alle
die Hauptleute des Heeres, die bei ihm waren, wurden sie froh.
\bibverse{14} Und das ganze Volk, das Ismael hatte von Mizpa weggeführt,
wandte sich um und kehrte wiederum zu Johanan, dem Sohne Kareahs.
\bibverse{15} Aber Ismael, der Sohn Nethanjas, entrann dem Johanan mit
acht Männern und zog zu den Kindern Ammon.

\bibverse{16} Und Johanan, der Sohn Kareahs, samt allen Hauptleuten des
Heeres, die bei ihm waren, nahmen all das übrige Volk, das sie
wiedergebracht hatten von Ismael, dem Sohn Nethanjas, aus Mizpa zu sich
(weil Gedalja, der Sohn Ahikams, erschlagen war), nämlich die
Kriegsmänner, Weiber und Kinder und Kämmerer, die sie von Gibeon hatten
wiedergebracht; \bibverse{17} und zogen hin und kehrten ein zur Herberge
Chimhams, die bei Bethlehem war, und wollten nach Ägypten ziehen vor den
Chaldäern. \bibverse{18} Denn sie fürchteten sich vor ihnen, weil
Ismael, der Sohn Nethanjas, Gedalja, den Sohn Ahikams, erschlagen hatte,
den der König zu Babel über das Land gesetzt hatte. \# 42 \bibverse{1}
Da traten herzu alle Hauptleute des Heeres, Johanan, der Sohn Kareahs,
Jesanja, der Sohn Hosajas, samt dem ganzen Volk, klein und groß,
\bibverse{2} und sprachen zum Propheten Jeremia: Lass doch unser Gebet
vor dir gelten und bitte für uns den HErrn, deinen Gott, für alle diese
Übrigen (denn unser ist leider wenig geblieben von vielen, wie du uns
selbst siehst mit deinen Augen), \bibverse{3} dass uns der HErr, dein
Gott, wolle anzeigen, wohin wir ziehen und was wir tun sollen.

\bibverse{4} Und der Prophet Jeremia sprach zu ihnen: Wohlan, ich will
gehorchen; und siehe, ich will den HErrn, euren Gott, bitten, wie ihr
gesagt habt; und alles, was euch der HErr antworten wird, das will ich
euch anzeigen und will euch nichts verhalten.

\bibverse{5} Und sie sprachen zu Jeremia: Der HErr sei ein gewisser und
wahrhaftiger Zeuge zwischen uns, wo wir nicht tun werden alles, was dir
der HErr, dein Gott, an uns befehlen wird. \bibverse{6} Es sei Gutes
oder Böses, so wollen wir gehorchen der Stimme des HErrn, unseres
Gottes, zu dem wir dich senden; auf dass es uns wohl gehe, so wir der
Stimme des HErrn, unseres Gottes, gehorchen.

\bibverse{7} Und nach zehn Tagen geschah des HErrn Wort zu Jeremia.
\bibverse{8} Da rief er Johanan, den Sohn Kareahs, und alle Hauptleute
des Heeres, die bei ihm waren, und alles Volk, klein und groß,
\bibverse{9} und sprach zu ihnen: So spricht der HErr, der Gott Israels,
zu dem ihr mich gesandt habt, dass ich euer Gebet vor ihn sollte
bringen: \bibverse{10} Werdet ihr in diesem Lande bleiben, so will ich
euch bauen und nicht zerbrechen; ich will euch pflanzen und nicht
ausreuten; denn es hat mich schon gereut das Übel, das ich euch getan
habe. \bibverse{11} Ihr sollt euch nicht fürchten vor dem König zu
Babel, vor dem ihr euch fürchtet, spricht der HErr; ihr sollt euch vor
ihm nicht fürchten, denn ich will bei euch sein, dass ich euch helfe und
von seiner Hand errette. \bibverse{12} Ich will euch Barmherzigkeit
erzeigen und mich über euch erbarmen und euch wieder in euer Land
bringen.

\bibverse{13} Werdet ihr aber sagen: Wir wollen nicht in diesem Lande
bleiben, damit ihr ja nicht gehorcht der Stimme des HErrn, eures Gottes,
\bibverse{14} sondern sagen: Nein, wir wollen nach Ägyptenland ziehen,
dass wir keinen Krieg sehen noch der Posaune Schall hören und nicht
Hunger Brots halben leiden müssen; daselbst wollen wir bleiben:
\bibverse{15} nun so höret des HErrn Wort, ihr Übrigen aus Juda! So
spricht der HErr Zebaoth, der Gott Israels: Werdet ihr euer Angesicht
richten, nach Ägyptenland zu ziehen, dass ihr daselbst bleiben wollt,
\bibverse{16} so soll euch das Schwert, vor dem ihr euch fürchtet, in
Ägyptenland treffen, und der Hunger, des ihr euch besorgt, soll stets
hinter euch her sein in Ägypten, und sollt daselbst sterben.
\bibverse{17} Denn sie seien, wer sie wollen, die ihr Angesicht richten,
dass sie nach Ägypten ziehen, daselbst zu bleiben, die sollen sterben
durchs Schwert, Hunger und Pestilenz, und soll keiner übrigbleiben noch
entrinnen dem Übel, das ich über sie will kommen lassen. \footnote{\textbf{42:17}
  Jer 29,17-18} \bibverse{18} Denn so spricht der HErr Zebaoth, der Gott
Israels: Gleichwie mein Zorn und Grimm über die Einwohner zu Jerusalem
ausgeschüttet ist, so soll er auch über euch ausgeschüttet werden, wenn
ihr nach Ägypten ziehet, dass ihr zum Fluch, zum Wunder, Schwur und
Schande werdet und diese Stätte nicht mehr sehen sollt.

\bibverse{19} Das Wort des HErrn gilt euch, ihr Übrigen aus Juda, dass
ihr nicht nach Ägypten ziehet. Darum so wisset, dass ich euch heute
bezeuge; \bibverse{20} ihr werdet sonst euer Leben verwahrlosen. Denn
ihr habt mich gesandt zum HErrn, eurem Gott, und gesagt: Bitte den
HErrn, unseren Gott, für uns; und alles, was der HErr, unser Gott, sagen
wird, das zeige uns an, so wollen wir darnach tun. \bibverse{21} Das
habe ich euch heute zu wissen getan; aber ihr wollt der Stimme des
HErrn, eures Gottes, nicht gehorchen noch alle dem, das er mir an euch
befohlen hat. \bibverse{22} So sollt ihr nun wissen, dass ihr durch
Schwert, Hunger und Pestilenz sterben müsst an dem Ort, dahin ihr
gedenkt zu ziehen, dass ihr daselbst wohnen wollt. \# 43 \bibverse{1} Da
Jeremia alle Worte des HErrn, ihres Gottes, hatte ausgeredet zu allem
Volk, wie ihm denn der HErr, ihr Gott, alle diese Worte an sie befohlen
hatte, \bibverse{2} sprachen Asarja, der Sohn Hosajas, und Johanan, der
Sohn Kareahs, und alle frechen Männer zu Jeremia: Du lügst; der HErr,
unser Gott, hat dich nicht zu uns gesandt noch gesagt: Ihr sollt nicht
nach Ägypten ziehen, daselbst zu wohnen; \bibverse{3} sondern Baruch,
der Sohn Nerias, beredet dich, uns zuwider, auf dass wir den Chaldäern
übergeben werden, dass sie uns töten und gen Babel wegführen.

\bibverse{4} Also gehorchten Johanan, der Sohn Kareahs, und alle
Hauptleute des Heeres samt dem ganzen Volk der Stimme des HErrn nicht,
dass sie im Lande Juda wären geblieben; \bibverse{5} sondern Johanan,
der Sohn Kareahs, und alle Hauptleute des Heeres nahmen zu sich alle
Übrigen aus Juda, die von allen Völkern, dahin sie geflohen,
wiedergekommen waren, dass sie im Lande Juda wohnten, \bibverse{6}
nämlich Männer, Weiber und Kinder, dazu die Königstöchter und alle
Seelen, die Nebusaradan, der Hauptmann, bei Gedalja, dem Sohn Ahikams,
des Sohnes Saphans, hatte gelassen, auch den Propheten Jeremia und
Baruch, den Sohn Nerias, \bibverse{7} und zogen nach Ägyptenland, denn
sie wollten der Stimme des HErrn nicht gehorchen, und kamen nach
Thachpanhes. \footnote{\textbf{43:7} 2Kö 25,26}

\bibverse{8} Und des HErrn Wort geschah zu Jeremia zu Thachpanhes und
sprach: \bibverse{9} Nimm große Steine und verscharre sie im Ziegelofen,
der vor der Tür am Hause Pharaos ist zu Thachpanhes, dass die Männer aus
Juda zusehen; \bibverse{10} und sprich zu ihnen: So spricht der HErr
Zebaoth, der Gott Israels: Siehe, ich will hinsenden und meinen Knecht
Nebukadnezar, den König zu Babel, holen lassen und will seinen Stuhl
oben auf diese Steine setzen, die ich verscharrt habe; und er soll sein
Gezelt darüberschlagen. \bibverse{11} Und er soll kommen und Ägyptenland
schlagen, und töten, wen es trifft, gefangen führen, wen es trifft, mit
dem Schwert schlagen, wen es trifft. \footnote{\textbf{43:11} Jer 15,2}
\bibverse{12} Und ich will die Häuser der Götter in Ägypten mit Feuer
anstecken, dass er sie verbrenne, und wegführe. Und er soll sich
Ägyptenland anziehen, wie ein Hirt sein Kleid anzieht, und mit Frieden
von dannen ziehen. \footnote{\textbf{43:12} Jer 46,25} \bibverse{13} Er
soll die Bildsäulen zu Beth-Semes in Ägyptenland zerbrechen und die
Götzentempel in Ägypten mit Feuer verbrennen. \# 44 \bibverse{1} Dies
ist das Wort, das zu Jeremia geschah an alle Juden, die in Ägyptenland
wohnten, nämlich die zu Migdol, zu Thachpanhes, zu Noph und im Lande
Pathros wohnten, und sprach: \footnote{\textbf{44:1} Jer 43,7}
\bibverse{2} So spricht der HErr Zebaoth, der Gott Israels: Ihr habt
gesehen all das Übel, das ich habe kommen lassen über Jerusalem und über
alle Städte in Juda; und siehe, heutigestages sind sie wüst, und wohnt
niemand darin; \bibverse{3} und das um ihrer Bosheit willen, die sie
taten, dass sie mich erzürnten und hingingen und räucherten und dienten
anderen Göttern, welche weder sie noch ihr noch eure Väter kannten.
\bibverse{4} Und ich sandte stets zu euch alle meine Knechte, die
Propheten, und ließ euch sagen: Tut doch nicht solche Gräuel, die ich
hasse. \bibverse{5} Aber sie gehorchten nicht, neigten auch ihre Ohren
nicht, dass sie von ihrer Bosheit sich bekehrt und anderen Göttern nicht
geräuchert hätten. \bibverse{6} Darum ging auch mein Zorn und Grimm an
und entbrannte über die Städte Judas und über die Gassen zu Jerusalem,
dass sie zur Wüste und Öde geworden sind, wie es heutigestages steht.

\bibverse{7} Nun, so spricht der HErr, der Gott Zebaoth, der Gott
Israels: Warum tut ihr doch so großes Übel wider euer eigen Leben, damit
unter euch ausgerottet werden Mann und Weib, Kind und Säugling aus Juda,
und nichts von euch übrigbleibe, \bibverse{8} und erzürnt mich so durch
eurer Hände Werke und räuchert anderen Göttern in Ägyptenland, dahin ihr
gezogen seid, daselbst zu herbergen, auf dass ihr ausgerottet und zum
Fluch und zur Schmach werdet unter allen Heiden auf Erden? \bibverse{9}
Habt ihr vergessen das Unglück eurer Väter, das Unglück der Könige
Judas, das Unglück ihrer Weiber, dazu euer eigenes Unglück und eurer
Weiber Unglück, das euch begegnet ist im Lande Juda und auf den Gassen
zu Jerusalem? \bibverse{10} Noch sind sie bis auf diesen Tag nicht
gedemütigt, fürchten sich auch nicht und wandeln nicht in meinem Gesetz
und den Rechten, die ich euch und euren Vätern vorgestellt habe.

\bibverse{11} Darum spricht der HErr Zebaoth, der Gott Israels, also:
Siehe, ich will mein Angesicht wider euch richten zum Unglück, und ganz
Juda soll ausgerottet werden. \bibverse{12} Und ich will die Übrigen aus
Juda nehmen, die ihr Angesicht gerichtet haben, nach Ägyptenland zu
ziehen, dass sie daselbst herbergen; es soll ein Ende mit ihnen allen
werden in Ägyptenland. Durchs Schwert sollen sie fallen, und durch
Hunger umkommen, beide, klein und groß; sie sollen durch Schwert und
Hunger sterben und sollen ein Schwur, Wunder, Fluch und Schmach werden.
\bibverse{13} Ich will auch die Einwohner in Ägyptenland mit Schwert,
Hunger und Pestilenz heimsuchen, gleichwie ich zu Jerusalem getan habe,
\bibverse{14} dass aus den Übrigen Judas keiner soll entrinnen noch
übrigbleiben, die doch darum hierher gekommen sind nach Ägyptenland zur
Herberge, dass sie wiederum ins Land Juda kommen möchten, dahin sie
gerne wollten wiederkommen und wohnen; aber es soll keiner wieder dahin
kommen, außer, welche von hinnen fliehen.

\bibverse{15} Da antworteten dem Jeremia alle Männer, die da wohl
wussten, dass ihre Weiber anderen Göttern räucherten, und alle Weiber,
die in großem Haufen dastanden, samt allem Volk, die in Ägyptenland
wohnten und in Pathros, und sprachen: \footnote{\textbf{44:15} Jes 11,11}
\bibverse{16} Nach dem Wort, das du im Namen des HErrn uns sagest,
wollen wir dir nicht gehorchen; \bibverse{17} sondern wir wollen tun
nach allem dem Wort, das aus unserem Munde geht, und wollen der
Himmelskönigin räuchern und ihr Trankopfer opfern, wie wir und unsere
Väter, unsere Könige und Fürsten getan haben in den Städten Judas und
auf den Gassen zu Jerusalem. Da hatten wir auch Brot genug und ging uns
wohl und sahen kein Unglück. \bibverse{18} Seit der Zeit aber, dass wir
haben abgelassen, der Himmelskönigin zu räuchern und Trankopfer zu
opfern, haben wir allen Mangel gelitten und sind durch Schwert und
Hunger umgekommen.

\bibverse{19} Auch wenn wir der Himmelskönigin räuchern und Trankopfer
opfern, das tun wir ja nicht ohne unserer Männer Willen, dass wir ihr
Kuchen backen und Trankopfer opfern, auf dass sie sich um uns bekümmere.

\bibverse{20} Da sprach Jeremia zum ganzen Volk, Männern und Weibern und
allem Volk, die ihm so geantwortet hatten: \bibverse{21} Ich meine ja,
der HErr habe gedacht an das Räuchern, das ihr in den Städten Judas und
auf den Gassen zu Jerusalem getrieben habt samt euren Vätern, Königen,
Fürsten und allem Volk im Lande, und hat's zu Herzen genommen,
\bibverse{22} dass er nicht mehr leiden konnte euren bösen Wandel und
die Gräuel, die ihr tatet; daher auch euer Land zur Wüste, zum Wunder
und zum Fluch geworden ist, dass niemand darin wohnt, wie es
heutigestages steht. \bibverse{23} Darum, dass ihr geräuchert habt und
wider den HErrn gesündigt und der Stimme des HErrn nicht gehorchtet und
in seinem Gesetz, seinen Rechten und Zeugnissen nicht gewandelt habt,
darum ist auch euch solches Unglück widerfahren, wie es heutigestages
steht.

\bibverse{24} Und Jeremia sprach zu allem Volk und zu allen Weibern:
Höret des HErrn Wort, alle ihr aus Juda, die in Ägyptenland sind.
\bibverse{25} So spricht der HErr Zebaoth, der Gott Israels: Ihr und
eure Weiber habt mit eurem Munde geredet und mit euren Händen
vollbracht, was ihr sagt: Wir wollen unsere Gelübde halten, die wir
gelobt haben der Himmelskönigin, dass wir ihr räuchern und Trankopfer
opfern. Wohlan, ihr habt eure Gelübde erfüllt und eure Gelübde gehalten.
\footnote{\textbf{44:25} Jer 44,17}

\bibverse{26} So höret nun des HErrn Wort, ihr alle aus Juda, die ihr in
Ägyptenland wohnet: Siehe, ich schwöre bei meinem großen Namen, spricht
der HErr, dass mein Name nicht mehr soll durch irgendeines Menschen Mund
aus Juda genannt werden in ganz Ägyptenland, der da sagt: „So wahr der
Herr HErr lebt!{}``

\bibverse{27} Siehe, ich will über sie wachen zum Unglück und zu keinem
Guten, dass, wer aus Juda in Ägyptenland ist, soll durch Schwert und
Hunger umkommen, bis es ein Ende mit ihnen habe. \bibverse{28} Welche
aber dem Schwert entrinnen, die werden aus Ägyptenland ins Land Juda
wiederkommen müssen als ein geringer Haufe. Und also werden dann alle
die Übrigen aus Juda, die nach Ägyptenland gezogen waren, dass sie
daselbst herbergten, erfahren, wessen Wort wahr geworden sei, meines
oder ihres.

\bibverse{29} Und zum Zeichen, spricht der HErr, dass ich euch an diesem
Ort heimsuchen will, damit ihr wisset, dass mein Wort soll wahr werden
über euch zum Unglück, \bibverse{30} so spricht der HErr also: Siehe,
ich will Pharao Hophra, den König in Ägypten, übergeben in die Hände
seiner Feinde und derer, die ihm nach dem Leben stehen, gleichwie ich
Zedekia, den König Judas, übergeben habe in die Hand Nebukadnezars, des
Königs zu Babel, seines Feindes, und der ihm nach seinem Leben stand.
\footnote{\textbf{44:30} 2Chr 36,13; 2Chr 36,20}

\hypertarget{section-10}{%
\section{45}\label{section-10}}

\bibverse{1} Dies ist das Wort, das der Prophet Jeremia redete zu
Baruch, dem Sohn Nerias, da er diese Reden in ein Buch schrieb aus dem
Munde Jeremias im vierten Jahr Jojakims, des Sohnes Josias, des Königs
in Juda, und sprach: \bibverse{2} So spricht der HErr Zebaoth, der Gott
Israels, von dir, Baruch: \bibverse{3} Du sprichst: Weh mir, wie hat mir
der HErr Jammer zu meinem Schmerz hinzugefügt! Ich seufze mich müde und
finde keine Ruhe.

\bibverse{4} Sage ihm also: So spricht der HErr: Siehe, was ich gebaut
habe, das breche ich ab; und was ich gepflanzt habe, das reute ich aus,
nämlich dieses mein ganzes Land. \bibverse{5} Und du begehrst dir große
Dinge? Begehre es nicht! Denn siehe, ich will Unglück kommen lassen über
alles Fleisch, spricht der HErr; aber deine Seele will ich dir zur Beute
geben, an welchen Ort du ziehest. \footnote{\textbf{45:5} Jer 39,18; Jer
  43,6}

\hypertarget{section-11}{%
\section{46}\label{section-11}}

\bibverse{1} Dies ist das Wort des HErrn, das zu dem Propheten Jeremia
geschehen ist wider alle Heiden.

\bibverse{2} Wider Ägypten. Wider das Heer Pharao Nechos, des Königs in
Ägypten, welches lag am Wasser Euphrat zu Karchemis, das der König zu
Babel, Nebukadnezar, schlug im vierten Jahr Jojakims, des Sohnes Josias,
des Königs in Juda: \bibverse{3} Rüstet Schild und Tartsche und ziehet
in den Streit! \bibverse{4} Spannet Rosse an und lasset Reiter
aufsitzen, setzt die Helme auf und schärft die Spieße und ziehet Panzer
an! \bibverse{5} Wie kommt's aber, dass ich sehe, dass sie verzagt sind
und die Flucht geben und ihre Helden erschlagen sind? Sie fliehen, dass
sie sich auch nicht umsehen. Schrecken ist um und um, spricht der HErr.
\bibverse{6} Der Schnelle kann nicht entfliehen noch der Starke
entrinnen. Gegen Mitternacht am Wasser Euphrat sind sie gefallen und
darniedergelegt. \bibverse{7} Wer ist der, der heraufzieht wie der Nil,
und seine Wellen erheben sich wie Wasserwellen? \bibverse{8} Ägypten
zieht herauf wie der Nil, und seine Wellen erheben sich wie
Wasserwellen, und es spricht: Ich will hinaufziehen, das Land bedecken
und die Stadt verderben samt denen, die darin wohnen. \bibverse{9}
Wohlan, sitzt auf die Rosse, rennt mit den Wagen, lasst die Helden
ausziehen, die Mohren, und aus Put, die den Schild führen, und die
Schützen aus Lud! \bibverse{10} Denn dies ist der Tag des Herrn HErrn
Zebaoth, ein Tag der Rache, dass er sich an seinen Feinden räche, da das
Schwert fressen und von ihrem Blut voll und trunken werden wird. Denn
sie müssen dem Herrn HErrn Zebaoth ein Schlachtopfer werden im Lande
gegen Mitternacht am Wasser Euphrat. \footnote{\textbf{46:10} 5Mo 32,42;
  Jes 34,5} \bibverse{11} Gehe hinauf gen Gilead und hole Salbe,
Jungfrau, Tochter Ägyptens! Aber es ist umsonst, dass du viel arzneiest;
du wirst doch nicht heil! \footnote{\textbf{46:11} Jer 8,22}
\bibverse{12} Deine Schande ist unter die Heiden erschollen, deines
Heulens ist das Land voll; denn ein Held fällt über den anderen und
liegen beide miteinander darnieder. \bibverse{13} Dies ist das Wort des
HErrn, das er zu dem Propheten Jeremia redete, da Nebukadnezar, der
König zu Babel, daherzog, Ägyptenland zu schlagen; \bibverse{14}
Verkündiget in Ägypten und saget's an zu Migdol, saget's an zu Noph und
Thachpanhes und sprechet: Stelle dich zur Wehr! denn das Schwert wird
fressen, was um dich her ist. \bibverse{15} Wie geht's zu, dass deine
Gewaltigen zu Boden fallen und können nicht bestehen? Der HErr hat sie
so gestürzt. \bibverse{16} Er macht, dass ihrer viel fallen, dass einer
mit dem anderen darniederliegt. Da sprachen sie: Wohlauf, lasst uns
wieder zu unserem Volk ziehen, in unser Vaterland vor dem Schwert des
Tyrannen! \bibverse{17} Daselbst schrie man ihnen nach: Pharao, der
König Ägyptens, liegt: er hat sein Gezelt gelassen! \bibverse{18} So
wahr als ich lebe, spricht der König, der HErr Zebaoth heißt: Jener wird
daherziehen so hoch, wie der Berg Thabor unter den Bergen ist und wie
der Karmel am Meer ist. \bibverse{19} Nimm dein Wandergerät, du
Einwohnerin, Tochter Ägyptens; denn Noph wird wüst und verbrannt werden,
dass niemand darin wohnen wird. \bibverse{20} Ägypten ist ein sehr
schönes Kalb; aber es kommt von Mitternacht der Schlächter.
\bibverse{21} Auch die, die darin um Sold dienen, sind wie gemästete
Kälber; aber sie müssen sich dennoch wenden, flüchtig werden miteinander
und werden nicht bestehen; denn der Tag ihres Unfalls wird über sie
kommen, die Zeit ihrer Heimsuchung. \bibverse{22} Man hört sie
davonschleichen wie eine Schlange; denn jene kommen mit Heereskraft und
bringen Äxte über sie wie die Holzhauer. \bibverse{23} Die werden hauen
also in ihrem Wald, spricht der HErr, dass es nicht zu zählen ist; denn
ihrer sind mehr als Heuschrecken, die niemand zählen kann. \bibverse{24}
Die Tochter Ägyptens steht mit Schanden; denn sie ist dem Volk von
Mitternacht in die Hände gegeben.

\bibverse{25} Der HErr Zebaoth, der Gott Israels, spricht: Siehe, ich
will heimsuchen den Amon zu No und den Pharao und Ägypten samt seinen
Göttern und Königen, ja, Pharao mit allen, die sich auf ihn verlassen,
\footnote{\textbf{46:25} Jer 43,12} \bibverse{26} dass ich sie gebe in
die Hände denen, die ihnen nach ihrem Leben stehen, und in die Hände
Nebukadnezars, des Königs zu Babel, und seiner Knechte. Und darnach
sollst du bewohnt werden wie vor alters, spricht der HErr. \bibverse{27}
Aber du, mein Knecht Jakob, fürchte dich nicht, und du, Israel, verzage
nicht! Denn siehe, ich will dir aus fernen Landen und deinem Samen aus
dem Lande seines Gefängnisses helfen, dass Jakob soll wiederkommen und
in Frieden sein und die Fülle haben, und niemand soll ihn schrecken.
\bibverse{28} Darum fürchte dich nicht, du, Jakob, mein Knecht, spricht
der HErr; denn ich bin bei dir. Mit allen Heiden, dahin ich dich
verstoßen habe, will ich ein Ende machen; aber mit dir will ich nicht
ein Ende machen, sondern ich will dich züchtigen mit Maßen, auf dass ich
dich nicht ungestraft lasse. \footnote{\textbf{46:28} Jer 30,11}

\hypertarget{section-12}{%
\section{47}\label{section-12}}

\bibverse{1} Dies ist das Wort des HErrn, das zum Propheten Jeremia
geschah wider die Philister, ehe denn Pharao Gaza schlug. \footnote{\textbf{47:1}
  Jes 14,29-32; Hes 25,15-17}

\bibverse{2} So spricht der HErr: Siehe, es kommen Wasser herauf von
Mitternacht, die eine Flut machen werden und das Land und was darin ist,
die Städte und die, die darin wohnen, wegreißen werden, dass die Leute
werden schreien und alle Einwohner im Lande heulen \bibverse{3} vor dem
Getümmel ihrer starken Rosse, die dahertraben, und vor dem Rasseln ihrer
Wagen und Poltern ihrer Räder; dass sich die Väter nicht werden umsehen
nach den Kindern, so verzagt werden sie sein \bibverse{4} vor dem Tage,
der da kommt, zu verstören alle Philister und auszureuten Tyrus und
Sidon samt ihren anderen Gehilfen. Denn der HErr wird die Philister, die
das Übrige sind aus der Insel Kaphthor, verstören. \bibverse{5} Gaza
wird kahl werden, und Askalon samt den Übrigen in ihren Gründen wird
verderbt. Wie lange ritzest du dich? \footnote{\textbf{47:5} Am 1,6-8;
  Zeph 2,4; Sach 9,5; Jer 41,5; Jer 48,37} \bibverse{6} O du Schwert des
HErrn, wann willst du doch aufhören? Fahre doch in deine Scheide und
ruhe und sei still! \bibverse{7} Aber wie kannst du aufhören, weil der
HErr dir Befehl getan hat wider Askalon und dich wider die Anfurt am
Meer bestellt? \# 48 \bibverse{1} Wider Moab. So spricht der HErr
Zebaoth, der Gott Israels: Weh der Stadt Nebo! denn sie ist zerstört und
liegt elend; Kirjathaim ist gewonnen; die hohe Feste steht elend und ist
zerrissen. \bibverse{2} Der Trotz Moabs ist aus, den sie an Hesbon
hatten; denn man gedenkt Böses wider sie: „Kommt, wir wollen sie
ausrotten, dass sie kein Volk mehr seien.`` Und du, Madmen, musst auch
verderbt werden; das Schwert wird hinter dich kommen. \bibverse{3} Man
hört ein Geschrei zu Horonaim von Verstören und großem Jammer.
\bibverse{4} Moab ist zerschlagen! Man hört ihre Kleinen schreien;
\bibverse{5} denn sie gehen mit Weinen den Weg hinauf gen Luhith, und
die Feinde hören ein Jammergeschrei den Weg von Horonaim herab:
\bibverse{6} „Hebt euch weg und errettet euer Leben!{}`` Aber du wirst
sein wie die Heide in der Wüste. \bibverse{7} Darum dass du dich auf
deine Gebäude verlässest und auf deine Schätze, sollst du auch gewonnen
werden; und Kamos muss hinaus gefangen wegziehen samt seinen Priestern
und Fürsten. \footnote{\textbf{48:7} 1Kö 11,7} \bibverse{8} Denn der
Verstörer wird über alle Städte kommen, dass nicht eine Stadt entrinnen
wird. Es sollen beide, die Gründe verderbt und die Ebenen verstört
werden; denn der HErr hat's gesagt. \bibverse{9} Gebt Moab Federn: er
wird ausgehen, als flöge er; und seine Städte werden wüst liegen, dass
niemand darin wohnen wird. \bibverse{10} Verflucht sei, der des HErrn
Werk lässig tut; verflucht sei, der sein Schwert aufhält, dass es nicht
Blut vergieße! \bibverse{11} Moab ist von seiner Jugend auf sicher
gewesen und auf seinen Hefen stillgelegen und ist nie aus einem Fass ins
andere gegossen und nie ins Gefängnis gezogen; darum ist sein Geschmack
ihm geblieben und sein Geruch nicht verändert worden. \bibverse{12}
Darum siehe, spricht der HErr, es kommt die Zeit, dass ich ihnen will
Schröter schicken, die sie ausschroten sollen und ihre Fässer ausleeren
und ihre Krüge zerschmettern. \bibverse{13} Und Moab soll über dem Kamos
zu Schanden werden, gleichwie das Haus Israel über Beth-El zu Schanden
geworden ist, darauf sie sich doch verließen. \bibverse{14} Wie dürft
ihr sagen: Wir sind die Helden und die rechten Kriegsleute?
\bibverse{15} so doch Moab muss verstört und ihre Städte erstiegen
werden und ihre beste Mannschaft zur Schlachtbank herabgehen muss,
spricht der König, welcher heißt der HErr Zebaoth. \bibverse{16} Denn
der Unfall Moabs wird bald kommen, und ihr Unglück eilt sehr.
\bibverse{17} Habt doch Mitleid mit ihnen alle, die ihr um sie her wohnt
und ihren Namen kennt, und sprecht: „Wie ist die starke Rute und der
herrliche Stab so zerbrochen!{}`` \bibverse{18} Herab von der
Herrlichkeit, du Einwohnerin, Tochter Dibon, und sitze in der Dürre!
Denn der Verstörer Moabs wird zu dir hinaufkommen und deine Festen
zerreißen. \bibverse{19} Tritt auf die Straße und schaue, du Einwohnerin
Aroers; frage die, die da fliehen und entrinnen, und sprich: „Wie
geht's?{}`` \bibverse{20} Ach, Moab ist verwüstet und verderbt! Heulet
und schreiet; sagt's am Arnon, dass Moab verstört sei! \bibverse{21} Die
Strafe ist über das ebene Land gegangen, nämlich über Holon, Jahza,
Mephaath, \bibverse{22} Dibon, Nebo, Beth-Diblathaim, \bibverse{23}
Kirjathaim, Beth-Gamul, Beth-Meon, \bibverse{24} Karioth, Bozra und über
alle Städte im Lande Moab, sie liegen fern oder nahe. \bibverse{25} Das
Horn Moabs ist abgehauen, und sein Arm ist zerbrochen, spricht der HErr.
\bibverse{26} Macht es trunken (denn es hat sich wider den HErrn
erhoben), dass es speien und die Hände ringen müsse, auf dass es auch
zum Gespött werde. \footnote{\textbf{48:26} Jer 25,15} \bibverse{27}
Denn Israel hat dein Gespött sein müssen, als wäre es unter den Dieben
gefunden; und weil du solches wider dasselbe redest, sollst du auch weg
müssen. \bibverse{28} O ihr Einwohner in Moab, verlasst die Städte und
wohnt in den Felsen und tut wie die Tauben, die da nisten in den hohlen
Löchern! \bibverse{29} Man hat immer gesagt von dem stolzen Moab, dass
es sehr stolz sei, hoffärtig, hochmütig, trotzig und übermütig.
\bibverse{30} Aber der HErr spricht: Ich kenne seinen Zorn wohl, dass er
nicht soviel vermag und untersteht sich, mehr zu tun, denn sein Vermögen
ist. \bibverse{31} Darum muss ich über Moab heulen und über das ganze
Moab schreien und über die Leute zu Kir-Heres klagen. \bibverse{32} Mehr
als über Jaser muss ich über dich, du Weinstock zu Sibma, weinen, dessen
Reben über das Meer reichten und bis ans Meer Jaser kamen. Der Verstörer
ist in deine Ernte und Weinlese gefallen; \bibverse{33} Freude und Wonne
ist aus dem Felde weg und aus dem Lande Moab, und man wird keinen Wein
mehr keltern; der Weintreter wird nicht mehr sein Lied singen
\bibverse{34} von des Geschreies wegen zu Hesbon bis gen Eleale, welches
bis gen Jahza erschallt, von Zoar an bis gen Horonaim, bis zum dritten
Eglath; denn auch die Wasser Nimrims sollen versiegen. \bibverse{35} Und
ich will, spricht der HErr, in Moab damit ein Ende machen, dass sie
nicht mehr auf den Höhen opfern und ihren Göttern räuchern sollen.
\bibverse{36} Darum seufzt mein Herz über Moab wie Flöten, und über die
Leute zu Kir-Heres seufzt mein Herz wie Flöten; denn das Gut, das sie
gesammelt, ist zu Grunde gegangen. \bibverse{37} Alle Köpfe werden kahl
sein und alle Bärte abgeschoren, aller Hände zerritzt, und jedermann
wird Säcke anziehen. \footnote{\textbf{48:37} Jer 47,5} \bibverse{38}
Auf allen Dächern und Gassen, allenthalben in Moab, wird man klagen;
denn ich habe Moab zerbrochen wie ein unwertes Gefäß, spricht der HErr.
\bibverse{39} O wie ist es verderbt, wie heulen sie! Wie schändlich
hängen sie die Köpfe! Und Moab ist zum Spott und zum Schrecken geworden
allen, so ringsumher wohnen. \bibverse{40} Denn so spricht der HErr:
Siehe, er fliegt daher wie ein Adler und breitet seine Flügel aus über
Moab. \bibverse{41} Karioth ist gewonnen, und die festen Städte sind
eingenommen; und das Herz der Helden in Moab wird zu derselben Zeit sein
wie einer Frau Herz in Kindsnöten. \bibverse{42} Denn Moab muss vertilgt
werden, dass sie kein Volk mehr seien, darum dass es sich wider den HErr
erhoben hat. \bibverse{43} Schrecken, Grube und Strick kommt über dich,
du Einwohner in Moab, spricht der HErr. \bibverse{44} Wer dem Schrecken
entflieht, der wird in die Grube fallen, und wer aus der Grube kommt,
der wird im Strick gefangen werden; denn ich will über Moab kommen
lassen ein Jahr ihrer Heimsuchung, spricht der HErr. \footnote{\textbf{48:44}
  Jes 24,17-18} \bibverse{45} Die aus der Schlacht entrinnen, werden
Zuflucht suchen zu Hesbon; aber es wird ein Feuer aus Hesbon und eine
Flamme aus Sihon gehen, welche die Örter in Moab und die kriegerischen
Leute verzehren wird. \footnote{\textbf{48:45} 4Mo 21,28-29}
\bibverse{46} Weh dir, Moab! verloren ist das Volk des Kamos; denn man
hat deine Söhne und Töchter genommen und gefangen weggeführt.
\bibverse{47} Aber in der letzten Zeit will ich das Gefängnis Moabs
wenden, spricht der HErr. Das sei gesagt von der Strafe über Moab. \# 49
\bibverse{1} Wider die Kinder Ammon spricht der HErr also: Hat denn
Israel nicht Kinder, oder hat es keinen Erben? Warum besitzt denn Milkom
das Land Gad, und sein Volk wohnt in jener Städten? \footnote{\textbf{49:1}
  Hes 25,2-7; Am 1,13-15; Zeph 2,8-11; 1Kö 11,5} \bibverse{2} Darum
siehe, es kommt die Zeit, spricht der HErr, dass ich will ein
Kriegsgeschrei erschallen lassen über Rabba der Kinder Ammon, dass sie
soll auf einem Haufen wüst liegen und ihre Töchter mit Feuer angesteckt
werden; aber Israel soll besitzen die, von denen sie besessen waren,
spricht der HErr. \bibverse{3} Heule, o Hesbon! denn Ai ist verstört.
Schreiet, ihr Töchter Rabbas, und ziehet Säcke an, klaget und lauft auf
den Mauern herum! denn Milkom wird gefangen weggeführt samt seinen
Priestern und Fürsten. \bibverse{4} Was trotzest du auf deine Auen?
Deine Auen sind ersäuft, du ungehorsame Tochter, die du dich auf deine
Schätze verlässest und sprichst in deinem Herzen: Wer darf sich an mich
machen? \bibverse{5} Siehe, spricht der Herr HErr Zebaoth: Ich will
Furcht über dich kommen lassen von allen, die um dich her wohnen, dass
ein jeglicher seines Weges vor sich hinaus verstoßen werde und niemand
sei, der die Flüchtigen sammle. \bibverse{6} Aber darnach will ich
wieder wenden das Gefängnis der Kinder Ammon, spricht der HErr.
\footnote{\textbf{49:6} Jer 48,47} \bibverse{7} Wider Edom. So spricht
der HErr Zebaoth: Ist denn keine Weisheit mehr zu Theman? ist denn kein
Rat mehr bei den Klugen? ist ihre Weisheit so leer geworden? \footnote{\textbf{49:7}
  Jes 21,11; Jes 34,5-15; Hes 25,12-14; Am 1,11; Am 1,1-12; Ob 1,-1}
\bibverse{8} Fliehet, wendet euch und verkriechet euch tief, ihr Bürger
zu Dedan! denn ich lasse einen Unfall über Esau kommen, die Zeit seiner
Heimsuchung. \bibverse{9} Es sollen Weinleser über dich kommen, die dir
kein Nachlesen lassen; und Diebe des Nachts sollen über dich kommen, die
sollen ihnen genug verderben. \bibverse{10} Denn ich habe Esau entblößt
und seine verborgenen Orte geöffnet, dass er sich nicht verstecken kann;
sein Same, seine Brüder und seine Nachbarn sind verstört, dass ihrer
keiner mehr da ist. \bibverse{11} Doch was übrigbleibt von deinen
Waisen, denen will ich das Leben gönnen, und deine Witwen werden auf
mich hoffen.

\bibverse{12} Denn so spricht der HErr: Siehe, die, die es nicht
verschuldet hatten, den Kelch zu trinken, müssen trinken; und du
solltest ungestraft bleiben? Du sollst nicht ungestraft bleiben, sondern
du musst auch trinken. \footnote{\textbf{49:12} Jer 25,15; Jer 25,21}
\bibverse{13} Denn ich habe bei mir selbst geschworen, spricht der HErr,
dass Bozra soll ein Wunder, Schmach, Wüste und Fluch werden und alle
ihre Städte eine ewige Wüste. \footnote{\textbf{49:13} Jer 44,12}
\bibverse{14} Ich habe gehört vom HErrn, dass eine Botschaft unter die
Heiden gesandt sei: Sammelt euch und kommt her wider sie, macht euch auf
zum Streit! \bibverse{15} Denn siehe, ich habe dich gering gemacht unter
den Heiden und verachtet unter den Menschen. \bibverse{16} Dein Trotz
und deines Herzens Hochmut hat dich betrogen, weil du in Felsenklüften
wohnst und hohe Gebirge innehast. Wenn du denn gleich dein Nest so hoch
machtest wie der Adler, dennoch will ich dich von dort herunterstürzen,
spricht der HErr. \bibverse{17} Also soll Edom wüst werden, dass alle
die, die vorübergehen, sich wundern und pfeifen werden über alle ihre
Plage; \footnote{\textbf{49:17} Jer 50,13} \bibverse{18} gleichwie Sodom
und Gomorra samt ihren Nachbarn umgekehrt ist, spricht der HErr, dass
niemand daselbst wohnen noch kein Mensch darin hausen soll. \footnote{\textbf{49:18}
  Jes 1,9} \bibverse{19} Denn siehe, er kommt herauf wie ein Löwe vom
stolzen Jordan her wider die festen Hürden; denn ich will sie daraus
eilends wegtreiben, und den, der erwählt ist, darübersetzen. Denn wer
ist mir gleich, wer will mich meistern, und wer ist der Hirte, der mir
widerstehen kann? \footnote{\textbf{49:19} Jer 50,44} \bibverse{20} So
höret nun den Ratschlag des HErrn, den er über Edom hat, und seine
Gedanken, die er über die Einwohner in Theman hat. Was gilt's? ob nicht
die Hirtenknaben sie fortschleifen werden und ihre Wohnung zerstören,
\bibverse{21} dass die Erde beben wird, wenn's ineinander fällt, und ihr
Geschrei wird man am Schilfmeer hören. \bibverse{22} Siehe, er fliegt
herauf wie ein Adler und wird seine Flügel ausbreiten über Bozra. Zur
selben Zeit wird das Herz der Helden in Edom sein wie das Herz einer
Frau in Kindsnöten. \bibverse{23} Wider Damaskus. Hamath und Arpad
stehen jämmerlich; sie sind verzagt, denn sie hören ein böses Geschrei;
die am Meer wohnen, sind so erschrocken, dass sie nicht Ruhe haben
können. \footnote{\textbf{49:23} Jes 17,1; Am 1,3-5} \bibverse{24}
Damaskus ist verzagt und gibt die Flucht; sie zappelt und ist in Ängsten
und Schmerzen wie eine Frau in Kindsnöten. \bibverse{25} Wie? ist sie
nun nicht verlassen, die berühmte und fröhliche Stadt? \bibverse{26}
Darum werden ihre junge Mannschaft auf ihren Gassen darniederliegen und
alle ihre Kriegsleute untergehen zur selben Zeit, spricht der HErr
Zebaoth. \bibverse{27} Und ich will in den Mauern von Damaskus ein Feuer
anzünden, dass es die Paläste Benhadads verzehren soll. \bibverse{28}
Wider Kedar und die Königreiche Hazors, welche Nebukadnezar, der König
zu Babel, schlug. So spricht der HErr: Wohlauf, ziehet herauf gegen
Kedar und verstöret die gegen Morgen wohnen! \bibverse{29} Man wird
ihnen ihre Hütten und Herden nehmen; ihr Gezelt, alle Geräte und Kamele
werden sie wegführen; und man wird über sie rufen: Schrecken um und um!
\bibverse{30} Fliehet, hebet euch eilends davon, verkriechet euch tief,
ihr Einwohner in Hazor! spricht der HErr; denn Nebukadnezar, der König
zu Babel, hat etwas im Sinn wider euch und meint euch. \footnote{\textbf{49:30}
  Jer 49,8} \bibverse{31} Wohlauf, ziehet herauf wider ein Volk, das
genug hat und sicher wohnt, spricht der HErr; sie haben weder Tür noch
Riegel und wohnen allein. \bibverse{32} Ihre Kamele sollen geraubt und
die Menge ihres Viehs genommen werden; und ich will sie zerstreuen in
alle Winde, alle, die das Haar rundumher abschneiden; und von allen
Orten her will ich ihr Unglück über sie kommen lassen, spricht der HErr,
\bibverse{33} dass Hazor soll eine Wohnung der Schakale und eine ewige
Wüste werden, dass niemand daselbst wohne und kein Mensch darin hause.
\footnote{\textbf{49:33} Jer 9,10} \bibverse{34} Dies ist das Wort des
HErrn, welches geschah zu Jeremia, dem Propheten, wider Elam im Anfang
des Königreichs Zedekias, des Königs in Juda, und sprach: \footnote{\textbf{49:34}
  Jer 25,25} \bibverse{35} So spricht der HErr Zebaoth: Siehe, ich will
den Bogen Elams zerbrechen, ihre vornehmste Gewalt, \bibverse{36} und
will die vier Winde aus den vier Enden des Himmels über sie kommen
lassen und will sie in alle diese Winde zerstreuen, dass kein Volk sein
soll, dahin nicht Vertriebene aus Elam kommen werden. \bibverse{37} Und
ich will Elam verzagt machen vor ihren Feinden und denen, die ihnen nach
ihrem Leben stehen, und Unglück über sie kommen lassen mit meinem
grimmigen Zorn, spricht der HErr, und will das Schwert hinter ihnen her
schicken, bis ich sie aufreibe. \bibverse{38} Meinen Stuhl will ich in
Elam aufrichten und will beide, den König und die Fürsten, daselbst
umbringen, spricht der HErr. \bibverse{39} Aber in der letzten Zeit will
ich das Gefängnis Elams wieder wenden, spricht der HErr. \footnote{\textbf{49:39}
  Jer 49,6}

\hypertarget{section-13}{%
\section{50}\label{section-13}}

\bibverse{1} Dies ist das Wort, welches der HErr durch den Propheten
Jeremia geredet hat wider Babel und das Land der Chaldäer: \footnote{\textbf{50:1}
  Jes 13,-1; Jes 14,1-14} \bibverse{2} Verkündiget unter den Heiden und
lasst erschallen, werfet ein Panier auf; lasst erschallen, und
verberget's nicht und sprecht: Babel ist gewonnen, Bel steht mit
Schanden, Merodach ist zerschmettert; ihre Götzen stehen mit Schanden,
und ihre Götter sind zerschmettert! \footnote{\textbf{50:2} Jes 46,1}
\bibverse{3} Denn es zieht von Mitternacht ein Volk herauf wider sie,
welches wird ihr Land zur Wüste machen, dass niemand darin wohnen wird,
sondern beide, Leute und Vieh, davonfliehen werden. \bibverse{4} In
denselben Tagen und zur selben Zeit, spricht der HErr, werden kommen die
Kinder Israel samt den Kindern Juda und weinend daherziehen und den
HErrn, ihren Gott, suchen. \bibverse{5} Sie werden forschen nach dem
Wege gen Zion, dahin sich kehren: Kommt, wir wollen uns zum HErrn fügen
mit einem ewigen Bunde, des nimmermehr vergessen werden soll!
\bibverse{6} Denn mein Volk ist wie eine verlorene Herde; ihre Hirten
haben sie verführt und auf den Bergen in der Irre gehen lassen, dass sie
von den Bergen auf die Hügel gegangen sind und ihre Hürden vergessen
haben. \bibverse{7} Es fraßen sie alle, die sie antrafen; und ihre
Feinde sprachen: Wir tun nicht unrecht! darum dass sie sich haben
versündigt an dem HErrn in der Wohnung der Gerechtigkeit und an dem
HErrn, der ihrer Väter Hoffnung ist. \bibverse{8} Fliehet aus Babel und
ziehet aus der Chaldäer Lande und stellet euch als Böcke vor der Herde
her. \footnote{\textbf{50:8} Jer 51,6; Jer 51,45} \bibverse{9} Denn
siehe, ich will große Völker in Haufen aus dem Lande gegen Mitternacht
erwecken und wider Babel heraufbringen, die sich wider sie sollen
rüsten, welche sie auch sollen gewinnen; ihre Pfeile sind wie die eines
guten Kriegers, der nicht fehlt. \bibverse{10} Und das Chaldäerland soll
ein Raub werden, dass alle, die sie berauben, sollen genug davon haben,
spricht der HErr; \bibverse{11} darum dass ihr euch des freuet und
rühmet, dass ihr mein Erbteil geplündert habt, und hüpfet wie die jungen
Kälber und wiehert wie die starken Gäule. \bibverse{12} Eure Mutter
besteht mit großer Schande, und die euch geboren hat, ist zum Spott
geworden; siehe, unter den Heiden ist sie die geringste, wüst, dürr und
öde. \bibverse{13} Denn vor dem Zorn des HErrn muss sie unbewohnt und
ganz wüst bleiben, dass alle, die bei Babel vorübergehen, werden sich
verwundern und pfeifen über all ihre Plage. \footnote{\textbf{50:13} Jer
  51,37; Jer 49,17} \bibverse{14} Rüstet euch wider Babel umher, alle
Schützen, schießet in sie, sparet die Pfeile nicht; denn sie hat wider
den HErrn gesündigt. \bibverse{15} Jauchzet über sie um und um! Sie muss
sich ergeben, ihre Grundfesten sind zerfallen, ihre Mauern sind
abgebrochen; denn das ist des HErrn Rache. Rächet euch an ihr, tut ihr,
wie sie getan hat. \bibverse{16} Rottet aus von Babel beide, den Sämann
und den Schnitter in der Ernte, dass ein jeglicher vor dem Schwert des
Tyrannen sich kehre zu seinem Volk und ein jeglicher fliehe in sein
Land. \bibverse{17} Israel hat müssen sein eine zerstreute Herde, die
die Löwen verscheucht haben. Am ersten fraß sie der König von Assyrien;
darnach überwältigte sie Nebukadnezar, der König zu Babel.

\bibverse{18} Darum spricht der HErr Zebaoth, der Gott Israels, also:
Siehe, ich will den König zu Babel heimsuchen und sein Land, gleichwie
ich den König von Assyrien heimgesucht habe. \bibverse{19} Israel aber
will ich wieder heim zu seiner Wohnung bringen, dass sie auf Karmel und
Basan weiden und ihre Seele auf dem Gebirge Ephraim und Gilead gesättigt
werden soll. \bibverse{20} Zur selben Zeit und in denselben Tagen wird
man die Missetat Israels suchen, spricht der HErr, aber es wird keine da
sein, und die Sünden Judas, aber es wird keine gefunden werden; denn ich
will sie vergeben denen, die ich übrigbleiben lasse. \footnote{\textbf{50:20}
  Jer 31,34; Jer 33,8} \bibverse{21} Zieh hinauf wider das Land, das
alles verbittert hat; zieh hinauf wider die Einwohner der Heimsuchung;
verheere und verbanne ihre Nachkommen, spricht der HErr, und tue alles,
was ich dir befohlen habe! \bibverse{22} Es ist ein Kriegsgeschrei im
Lande und großer Jammer. \bibverse{23} Wie geht's zu, dass der Hammer
der ganzen Welt zerbrochen und zerschlagen ist? Wie geht's zu, dass
Babel eine Wüste geworden ist unter allen Heiden? \bibverse{24} Ich habe
dir nachgestellt, Babel; darum bist du auch gefangen, ehe du dich's
versahst; du bist getroffen und ergriffen, denn du hast dem HErrn
getrotzt. \bibverse{25} Der HErr hat seinen Schatz aufgetan und die
Waffen seines Zorns hervorgebracht; denn der Herr HErr Zebaoth hat etwas
auszurichten in der Chaldäer Lande. \bibverse{26} Kommt her wider sie,
ihr vom Ende, öffnet ihre Kornhäuser, werfet sie in einen Haufen und
verbannet sie, dass ihr nichts übrigbleibe! \bibverse{27} Erwürget alle
ihre Rinder, führt sie hinab zu Schlachtbank! Weh ihnen! denn ihr Tag
ist gekommen, die Zeit ihrer Heimsuchung. \bibverse{28} Man hört ein
Geschrei der Flüchtigen und derer, die entronnen sind aus dem Lande
Babel, auf dass sie verkündigen zu Zion die Rache des HErrn, unseres
Gottes, die Rache seines Tempels. \bibverse{29} Rufet viele wider Babel,
belagert sie um und um, alle Bogenschützen, und lasset keinen
davonkommen! Vergeltet ihr, wie sie verdient hat; wie sie getan hat, so
tut ihr wieder! denn sie hat stolz gehandelt wider den HErrn, den
Heiligen in Israel. \footnote{\textbf{50:29} Jer 50,15} \bibverse{30}
Darum soll ihre junge Mannschaft fallen auf ihren Gassen, und alle
Kriegsleute sollen untergehen zur selben Zeit, spricht der HErr.
\bibverse{31} Siehe, du Stolzer, ich will an dich, spricht der Herr HErr
Zebaoth; denn dein Tag ist gekommen, die Zeit deiner Heimsuchung.
\bibverse{32} Da soll der Stolze stürzen und fallen, dass ihn niemand
aufrichte; ich will seine Städte mit Feuer anstecken, das soll alles,
was um ihn her ist, verzehren. \bibverse{33} So spricht der HErr
Zebaoth: Siehe, die Kinder Israel samt den Kindern Juda müssen Gewalt
und Unrecht leiden; alle, die sie gefangen weggeführt haben, halten sie
und wollen sie nicht loslassen. \bibverse{34} Aber ihr Erlöser ist
stark, der heißt HErr Zebaoth; der wird ihre Sache so ausführen, dass er
das Land bebend und die Einwohner zu Babel zitternd mache. \bibverse{35}
Schwert soll kommen, spricht der HErr, über die Chaldäer und über die
Einwohner zu Babel und über ihre Fürsten und über ihre Weisen!
\bibverse{36} Schwert soll kommen über ihre Weissager, dass sie zu
Narren werden; Schwert soll kommen über ihre Starken, dass sie verzagen!
\bibverse{37} Schwert soll kommen über ihre Rosse und Wagen und alles
fremde Volk, das darin ist, dass sie zu Weibern werden! Schwert soll
kommen über ihre Schätze, dass sie geplündert werden! \bibverse{38}
Trockenheit soll kommen über ihre Wasser, dass sie versiegen! denn es
ist ein Götzenland, und sie trotzen auf ihre schrecklichen Götzen.
\bibverse{39} Darum sollen Wüstentiere und wilde Hunde darin wohnen und
die jungen Strauße; und es soll nimmermehr bewohnt werden und niemand
darin hausen für und für, \bibverse{40} gleichwie Gott Sodom und Gomorra
samt ihren Nachbarn umgekehrt hat, spricht der HErr, dass niemand darin
wohne noch ein Mensch darin hause. \footnote{\textbf{50:40} 1Mo 19,24-25}
\bibverse{41} Siehe, es kommt ein Volk von Mitternacht her; viele Heiden
und viele Könige werden vom Ende der Erde sich aufmachen. \footnote{\textbf{50:41}
  Jer 50,9} \bibverse{42} Die haben Bogen und Lanze; sie sind grausam
und unbarmherzig; ihr Geschrei ist wie das Brausen des Meeres; sie
reiten auf Rossen, gerüstet wie Kriegsmänner wider dich, du Tochter
Babel. \footnote{\textbf{50:42} Jer 6,23} \bibverse{43} Wenn der König
zu Babel ihr Gerücht hören wird, so werden ihm die Fäuste entsinken; ihm
wird so angst und bange werden wie einer Frau in Kindsnöten.
\bibverse{44} Siehe, er kommt herauf wie ein Löwe vom stolzen Jordan
wider die festen Hürden; denn ich will sie daraus eilends wegtreiben,
und den, der erwählt ist, darübersetzen. Denn wer ist mir gleich, wer
will micht meistern, und wer ist der Hirte, der mir widerstehen kann?
\bibverse{45} So höret nun den Ratschlag des HErrn, den er über Babel
hat, und seine Gedanken, die er hat über die Einwohner im Lande der
Chaldäer! Was gilt's? ob nicht die Hirtenknaben sie fortschleifen werden
und ihre Wohnung zerstören. \bibverse{46} Und die Erde wird beben von
dem Geschrei, und es wird unter den Heiden erschallen, wenn Babel
gewonnen wird. \# 51 \bibverse{1} So spricht der HErr: Siehe, ich will
einen scharfen Wind erwecken wider Babel und wider ihre Einwohner, die
sich wider mich gesetzt haben. \bibverse{2} Ich will auch Worfler gen
Babel schicken, die sie worfeln sollen und ihr Land ausfegen, die
allenthalben um sie sein werden am Tage ihres Unglücks; \footnote{\textbf{51:2}
  Jer 15,7} \bibverse{3} denn ihre Schützen werden nicht schießen, und
ihre Geharnischten werden sich nicht wehren können. So verschonet nun
ihre junge Mannschaft nicht, verbannet all ihr Heer, \bibverse{4} dass
die Erschlagenen daliegen im Lande der Chaldäer und die Erstochenen auf
ihren Gassen! \bibverse{5} Denn Israel und Juda sollen nicht Witwen von
ihrem Gott, dem HErrn Zebaoth, gelassen werden. Denn jener Land hat sich
hoch verschuldet am Heiligen in Israel. \bibverse{6} Fliehet aus Babel,
damit ein jeglicher seine Seele errette, dass ihr nicht untergehet in
ihrer Missetat! Denn dies ist die Zeit der Rache des HErrn, der ein
Vergelter ist und will ihnen bezahlen. \footnote{\textbf{51:6} Jer 50,8;
  Offb 18,4; Jes 48,20} \bibverse{7} Ein goldener Kelch, der alle Welt
trunken gemacht hat, war Babel in der Hand des HErrn; alle Heiden haben
von ihrem Wein getrunken, darum sind die Heiden so toll geworden.
\footnote{\textbf{51:7} Jer 25,15; Offb 17,4; Offb 18,3} \bibverse{8}
Wie plötzlich ist Babel gefallen und zerschmettert! Heulet über sie,
nehmet auch Salbe zu ihren Wunden, ob sie vielleicht möchte heil werden!
\footnote{\textbf{51:8} Offb 18,2} \bibverse{9} Wir heilen Babel; aber
sie will nicht heil werden. So lasst sie fahren und lasst uns ein
jeglicher in sein Land ziehen! Denn ihre Strafe reicht bis an den Himmel
und langt hinauf bis an die Wolken. \bibverse{10} Der HErr hat unsere
Gerechtigkeit hervorgebracht; kommt, lasst uns zu Zion erzählen die
Werke des HErrn, unseres Gottes! \bibverse{11} Ja, schärft nun die
Pfeile wohl und rüstet die Schilde! Der HErr hat den Mut der Könige in
Medien erweckt; denn seine Gedanken stehen wider Babel, dass er sie
verderbe. Denn dies ist die Rache des HErrn, die Rache seines Tempels.
\bibverse{12} Ja, steckt nun Panier auf die Mauern zu Babel, nehmt die
Wache ein, setzt Wächter, bestellt die Hut! denn der HErr gedenkt etwas
und wird auch tun, was er wider die Einwohner zu Babel geredet hat.
\bibverse{13} Die du an großen Wassern wohnst und große Schätze hast,
dein Ende ist gekommen, und dein Geiz ist aus! \footnote{\textbf{51:13}
  Offb 17,1} \bibverse{14} Der HErr Zebaoth hat bei seiner Seele
geschworen: Ich will dich mit Menschen füllen, als wären's Käfer; die
sollen dir ein Liedlein singen! \bibverse{15} Er hat die Erde durch
seine Kraft gemacht und den Weltkreis durch seine Weisheit bereitet und
den Himmel ausgebreitet durch seinen Verstand. \bibverse{16} Wenn er
donnert, so ist da Wasser die Menge unter dem Himmel; er zieht die Nebel
auf vom Ende der Erde; er macht die Blitze im Regen und lässt den Wind
kommen aus seinen Vorratskammern. \bibverse{17} Alle Menschen sind
Narren mit ihrer Kunst, und alle Goldschmiede bestehen mit Schanden mit
ihren Bildern; denn ihre Götzen sind Trügerei und haben kein Leben.
\bibverse{18} Es ist eitel Nichts und verführerisches Werk; sie müssen
umkommen, wenn sie heimgesucht werden. \bibverse{19} Aber also ist der
nicht, der Jakobs Schatz ist; sondern der alle Dinge schafft, der ist's,
und Israel ist sein Erbteil. Er heißt HErr Zebaoth. \bibverse{20} Du
bist mein Hammer, meine Kriegswaffe; durch dich zerschmettere ich die
Heiden und zerstöre die Königreiche; \footnote{\textbf{51:20} Jer 50,23;
  Jes 10,5} \bibverse{21} durch dich zerschmettere ich Rosse und Reiter
und zerschmettere Wagen und Fuhrmänner; \bibverse{22} durch dich
zerschmettere ich Männer und Weiber und zerschmettere Alte und Junge und
zerschmettere Jünglinge und Jungfrauen; \bibverse{23} durch dich
zerschmettere ich Hirten und Herden und zerschmettere Bauern und Joche
und zerschmettere Fürsten und Herren.

\bibverse{24} Und ich will Babel und allen Einwohnern in Chaldäa
vergelten alle ihre Bosheit, die sie an Zion begangen haben, vor euren
Augen, spricht der HErr. \bibverse{25} Siehe, ich will an dich, du
schädlicher Berg, der du alle Welt verderbest, spricht der HErr; ich
will meine Hand über dich strecken und dich von den Felsen herabwälzen
und will einen verbrannten Berg aus dir machen, \bibverse{26} dass man
weder Eckstein noch Grundstein aus dir nehmen könne, sondern eine ewige
Wüste sollst du sein, spricht der HErr. \bibverse{27} Werfet Panier auf
im Lande, blaset die Posaune unter den Heiden, heiliget die Heiden wider
sie; rufet wider sie die Königreiche Ararat, Minni und Askenas;
bestellet Hauptleute wider sie; bringet Rosse herauf wie flatternde
Käfer! \footnote{\textbf{51:27} Jes 13,3; 1Mo 10,3} \bibverse{28}
Heiliget die Heiden wider sie, die Könige aus Medien samt allen ihren
Fürsten und Herren und das ganze Land ihrer Herrschaft, \bibverse{29}
dass das Land erbebe und erschrecke; denn die Gedanken des HErrn wollen
erfüllt werden wider Babel, dass er das Land Babel zur Wüste mache,
darin niemand wohne. \bibverse{30} Die Helden zu Babel werden nicht zu
Felde ziehen, sondern müssen in der Festung bleiben, ihre Stärke ist
aus, sie sind Weiber geworden; ihre Wohnungen sind angesteckt und ihre
Riegel zerbrochen. \bibverse{31} Es läuft hier einer und da einer dem
anderen entgegen, und eine Botschaft begegnet der anderen, dem König zu
Babel anzusagen, dass seine Stadt gewonnen sei bis ans Ende
\bibverse{32} und die Furten eingenommen und die Seen ausgebrannt sind
und die Kriegsleute seien blöde geworden.

\bibverse{33} Denn also spricht der HErr Zebaoth, der Gott Israels: „Die
Tochter Babel ist wie eine Tenne, wenn man darauf drischt; es wird ihre
Ernte gar bald kommen.`` \bibverse{34} Nebukadnezar, der König zu Babel,
hat mich gefressen und umgebracht; er hat aus mir ein leeres Gefäß
gemacht; er hat mich verschlungen wie ein Drache; er hat seinen Bauch
gefüllt mit meinem Köstlichsten; er hat mich verstoßen. \bibverse{35}
Nun aber komme über Babel der Frevel, an mir begangen und an meinem
Fleische, spricht die Einwohnerin zu Zion, und mein Blut über die
Einwohner in Chaldäa, spricht Jerusalem. \bibverse{36} Darum spricht der
HErr also: Siehe, ich will dir deine Sache ausführen und dich rächen;
ich will ihr Meer austrocknen und ihre Brunnen versiegen lassen.
\bibverse{37} Und Babel soll zum Steinhaufen und zur Wohnung der
Schakale werden, zum Wunder und zum Anpfeifen, dass niemand darin wohne.
\bibverse{38} Sie sollen miteinander brüllen wie die Löwen und schreien
wie die jungen Löwen. \bibverse{39} Ich will sie mit ihrem Trinken in
die Hitze setzen und will sie trunken machen, dass sie fröhlich werden
und einen ewigen Schlaf schlafen, von dem sie nimmermehr aufwachen
sollen, spricht der HErr. \bibverse{40} Ich will sie herunterführen wie
Lämmer zur Schlachtbank, wie die Widder mit den Böcken. \bibverse{41}
Wie ist Sesach so gewonnen und die Berühmte in aller Welt so
eingenommen! Wie ist Babel so zum Wunder geworden unter den Heiden!
\bibverse{42} Es ist ein Meer über Babel gegangen, und es ist mit seiner
Wellen Menge bedeckt. \bibverse{43} Ihre Städte sind zur Wüste und zu
einem dürren, öden Lande geworden, zu einem Lande, darin niemand wohnt
und darin kein Mensch wandelt. \bibverse{44} Denn ich habe den Bel zu
Babel heimgesucht und habe aus seinem Rachen gerissen, was er
verschlungen hatte; und die Heiden sollen nicht mehr zu ihm laufen; denn
es sind auch die Mauern zu Babel zerfallen. \footnote{\textbf{51:44} Jer
  50,2} \bibverse{45} Ziehet heraus, mein Volk, und errette ein
jeglicher seine Seele vor dem grimmigen Zorn des HErrn! \footnote{\textbf{51:45}
  Jer 51,6} \bibverse{46} Euer Herz möchte sonst weich werden und
verzagen vor dem Geschrei, das man im Lande hören wird; denn es wird ein
Geschrei übers Jahr gehen und darnach im anderen Jahr auch ein Geschrei
über Gewalt im Lande und wird ein Fürst wider den anderen sein.
\bibverse{47} Darum siehe, es kommt die Zeit, dass ich die Götzen zu
Babel heimsuchen will und ihr ganzes Land zu Schanden werden soll und
ihre Erschlagenen darin liegen werden. \bibverse{48} Himmel und Erde und
alles was darin ist, werden jauchzen über Babel, dass ihre Verstörer von
Mitternacht gekommen sind, spricht der HErr. \footnote{\textbf{51:48}
  Offb 18,20} \bibverse{49} Und wie Babel in Israel die Erschlagenen
gefällt hat, also sollen zu Babel die Erschlagenen fallen im ganzen
Lande. \bibverse{50} So ziehet nun hin, die ihr dem Schwert entronnen
seid, und säumet euch nicht! Gedenket des HErrn im fernen Lande und
lasset euch Jerusalem im Herzen sein! \bibverse{51} Wir waren zu
Schanden geworden, da wir die Schmach hören mussten; und die Scham unser
Angesicht bedeckte, da die Fremden über das Heiligtum des Hauses des
HErrn kamen. \bibverse{52} Darum siehe, die Zeit kommt, spricht der
HErr, dass ich ihre Götzen heimsuchen will, und im ganzen Lande sollen
die tödlich Verwundeten seufzen. \bibverse{53} Und wenn Babel gen Himmel
stiege und ihre Macht in der Höhe festmachte, so sollen doch Verstörer
von mir über sie kommen, spricht der HErr. \bibverse{54} Man hört ein
Geschrei zu Babel und einen großen Jammer in der Chaldäer Lande;
\bibverse{55} denn der HErr verstört Babel und verderbt sie mit großem
Getümmel; ihre Wellen brausen wie die großen Wasser, es erschallt ihr
lautes Toben. \bibverse{56} Denn es ist über Babel der Verstörer
gekommen, ihre Helden werden gefangen, ihre Bogen werden zerbrochen;
denn der Gott der Rache, der HErr, bezahlt ihr. \footnote{\textbf{51:56}
  5Mo 32,35} \bibverse{57} Ich will ihre Fürsten, Weisen, Herren und
Hauptleute und Krieger trunken machen, dass sie einen ewigen Schlaf
sollen schlafen, davon sie nimmermehr aufwachen, spricht der König, der
da heißt HErr Zebaoth. \footnote{\textbf{51:57} Jer 51,39}

\bibverse{58} So spricht der HErr Zebaoth: Die Mauern der großen Babel
sollen untergraben und ihre hohen Tore mit Feuer angesteckt werden, dass
der Heiden Arbeit verloren sei, und dass verbrannt werde, was die Völker
mit Mühe erbaut haben. \footnote{\textbf{51:58} Hab 2,13}

\bibverse{59} Dies ist das Wort, das der Prophet Jeremia befahl Seraja
dem Sohn Nerias, des Sohnes Maasejas, da er zog mit Zedekia, dem König
in Juda, gen Babel im vierten Jahr seines Königreichs. Und Seraja war
der Marschall für die Reise. \footnote{\textbf{51:59} Jer 36,4}
\bibverse{60} Und Jeremia schrieb all das Unglück, das über Babel kommen
sollte, in ein Buch, nämlich alle diese Worte, die wider Babel
geschrieben sind. \bibverse{61} Und Jeremia sprach zu Seraja: Wenn du
gen Babel kommst, so schaue zu und lies alle diese Worte \bibverse{62}
und sprich: HErr, du hast geredet wider diese Stätte, dass du sie willst
ausrotten, dass niemand darin wohne, weder Mensch noch Vieh, sondern
dass sie ewiglich wüst sei. \bibverse{63} Und wenn du das Buch hast
ausgelesen, so binde einen Stein daran und wirf's in den Euphrat
\footnote{\textbf{51:63} Offb 18,21} \bibverse{64} und sprich: also soll
Babel versenkt werden und nicht wieder aufkommen von dem Unglück, das
ich über sie bringen will, sondern vergehen. So weit hat Jeremia
geredet. \# 52 \bibverse{1} Zedekia war einundzwanzig Jahre alt, da er
König ward, und regierte elf Jahre zu Jerusalem. Seine Mutter hieß
Hamutal, eine Tochter Jeremias zu Libna. \bibverse{2} Und er tat was dem
HErrn übel gefiel, gleich wie Jojakim getan hatte. \bibverse{3} Denn es
ging des HErrn Zorn über Jerusalem und Juda, bis er sie von seinem
Angesicht verwarf. Und Zedekia fiel ab vom König zu Babel.

\bibverse{4} Aber im neunten Jahr seines Königreichs, am zehnten Tage
des zehnten Monats, kam Nebukadnezar, der König zu Babel, samt all
seinem Heer wider Jerusalem, und sie belagerten es und machten Bollwerke
ringsumher. \bibverse{5} Und blieb also die Stadt belagert bis ins elfte
Jahr des Königs Zedekia.

\bibverse{6} Aber am neunten Tage des vierten Monats nahm der Hunger
überhand in der Stadt, und hatte das Volk vom Lande nichts mehr zu
essen. \bibverse{7} Da brach man in die Stadt; und alle Kriegsleute
gaben die Flucht und zogen zur Stadt hinaus bei der Nacht auf dem Wege
durch das Tor zwischen den zwei Mauern, der zum Garten des Königs geht.
Aber die Chaldäer lagen um die Stadt her. \bibverse{8} Und da diese
zogen des Weges zum blachen Feld, jagte der Chaldäer Heer dem König nach
und ergriffen Zedekia in dem Felde bei Jericho; da zerstreute sich all
sein Heer von ihm. \bibverse{9} Und sie fingen den König und brachten
ihn hinauf zum König zu Babel gen Ribla, das im Lande Hamath liegt; der
sprach ein Urteil über ihn. \bibverse{10} Allda ließ der König zu Babel
die Söhne Zedekias vor seinen Augen erwürgen und erwürgte alle Fürsten
Judas zu Ribla. \bibverse{11} Aber Zedekia ließ er die Augen ausstechen
und ließ ihn mit zwei Ketten binden, und führte ihn also der König zu
Babel gen Babel und legte ihn ins Gefängnis, bis dass er starb.
\footnote{\textbf{52:11} Jer 32,5}

\bibverse{12} Am zehnten Tage des fünften Monats, welches ist das
neunzehnte Jahr Nebukadnezars, des Königs zu Babel, kam Nebusaradan, der
Hauptmann der Trabanten, der stets um den König zu Babel war, gen
Jerusalem \bibverse{13} und verbrannte des HErrn Haus und des Königs
Haus und alle Häuser zu Jerusalem; alle großen Häuser verbrannte er mit
Feuer. \bibverse{14} Und das ganze Heer der Chaldäer, das bei dem
Hauptmann war, riss um alle Mauern zu Jerusalem ringsumher.
\bibverse{15} Aber das arme Volk und andere Volk, das noch übrig war in
der Stadt, und die zum König zu Babel fielen und das übrige
Handwerksvolk führte Nebusaradan, der Hauptmann, gefangen weg.
\bibverse{16} Und vom armen Volk auf dem Lande ließ Nebusaradan, der
Hauptmann, bleiben Weingärtner und Ackerleute.

\bibverse{17} Aber die ehernen Säulen am Hause des HErrn und das Gestühl
und das eherne Meer am Hause des HErrn zerbrachen die Chaldäer und
führten all das Erz davon gen Babel. \bibverse{18} Und die Kessel,
Schaufeln, Messer, Becken, Kellen und alle ehernen Gefäße, die man im
Gottesdienst pflegte zu brauchen, nahmen sie weg. \bibverse{19} Dazu
nahm der Hauptmann, was golden und silbern war an Bechern, Räuchtöpfen,
Becken, Kesseln, Leuchtern, Löffeln und Schalen.

\bibverse{20} Die zwei Säulen, das Meer, die zwölf ehernen Rinder
darunter und die Gestühle, welche der König Salomo hatte lassen machen
zum Hause des HErrn, alles dieses Gerätes Erz war unermesslich viel.
\footnote{\textbf{52:20} 1Kö 7,15-47} \bibverse{21} Der zwei Säulen aber
war eine jegliche achtzehn Ellen hoch, und eine Schnur, zwölf Ellen
lang, reichte um sie her, und war eine jegliche vier Finger dick und
inwendig hohl; \bibverse{22} und stand auf jeglicher ein eherner Knauf,
fünf Ellen hoch, und ein Gitterwerk und Granatäpfel waren an jeglichem
Knauf ringsumher, alles ehern; und war eine Säule wie die andere, die
Granatäpfel auch. \bibverse{23} Es waren der Granatäpfel sechsundneunzig
daran, und aller Granatäpfel waren hundert an einem Gitterwerk
ringsumher.

\bibverse{24} Und der Hauptmann nahm den obersten Priester Seraja und
den Priester Zephanja, den nächsten nach ihm, und die drei Torhüter
\bibverse{25} und einen Kämmerer aus der Stadt, welcher über die
Kriegsleute gesetzt war, und sieben Männer, welche um den König sein
mussten, die in der Stadt gefunden wurden, dazu den Schreiber des
Feldhauptmanns, der das Volk im Lande zum Heer aufbot, dazu sechzig Mann
Landvolks, die in der Stadt gefunden wurden: \bibverse{26} diese nahm
Nebusaradan, der Hauptmann, und brachte sie dem König zu Babel gen
Ribla. \bibverse{27} Und der König zu Babel schlug sie tot zu Ribla, das
im Lande Hamath liegt. Also ward Juda aus seinem Lande weggeführt.

\bibverse{28} Dies ist das Volk, welches Nebukadnezar weggeführt hat: im
siebenten Jahr 3023 Juden; \footnote{\textbf{52:28} 2Kö 24,11-16}
\bibverse{29} Im achtzehnten Jahr aber des Nebukadnezar 832 Seelen aus
Jerusalem; \bibverse{30} und im dreiundzwanzigsten Jahr des Nebukadnezar
führte Nebusaradan, der Hauptmann, 745 Seelen weg aus Juda. Alle Seelen
sind 4600.

\bibverse{31} Aber im siebenunddreißigsten Jahr, nachdem Jojachin, der
König zu Juda, weggeführt war, am fünfundzwanzigsten Tage des zwölften
Monats, erhob Evil-Merodach, der König zu Babel, im Jahr, da er König
ward, das Haupt Jojachins, des Königs in Juda, und ließ ihn aus dem
Gefängnis \bibverse{32} und redete freundlich mit ihm und setzte seinen
Stuhl über der Könige Stühle, die bei ihm zu Babel waren, \bibverse{33}
und wandelte ihm seines Gefängnisses Kleider, dass er vor ihm aß stets
sein Leben lang. \bibverse{34} Und ihm ward stets sein Unterhalt vom
König zu Babel gegeben, wie es ihm verordnet war, sein ganzes Leben lang
bis an sein Ende.
