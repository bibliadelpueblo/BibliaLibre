\hypertarget{section}{%
\section{1}\label{section}}

\bibverse{1} Im dritten Jahr des Reiches Jojakims, des Königs in Juda,
kam Nebukadnezar, der König zu Babel, vor Jerusalem und belagerte es.
\footnote{\textbf{1:1} 2Kö 24,1-2} \bibverse{2} Und der HErr übergab ihm
Jojakim, den König Judas, und etliche Gefäße aus dem Hause Gottes; die
ließ er führen ins Land Sinear in seines Gottes Haus und tat die Gefäße
in seines Gottes Schatzkammer.

\bibverse{3} Und der König sprach zu Aspenas, seinem obersten Kämmerer,
er sollte aus den Kindern Israel vom königlichen Stamm und Herrenkindern
wählen \footnote{\textbf{1:3} 2Kö 20,18} \bibverse{4} Knaben, die nicht
gebrechlich wären, sondern schöne, vernünftige, weise, kluge und
verständige, die da geschickt wären, zu dienen an des Königs Hofe und zu
lernen chaldäische Schrift und Sprache. \bibverse{5} Solchen bestimmte
der König, was man ihnen täglich geben sollte von seiner Speise und von
dem Wein, den er selbst trank, dass sie also drei Jahre auferzogen
würden und darnach vor dem König dienen sollten.

\bibverse{6} Unter diesen waren Daniel, Hananja, Misael und Asarja von
den Kindern Juda. \bibverse{7} Und der oberste Kämmerer gab ihnen Namen
und nannte Daniel Beltsazar und Hananja Sadrach und Misael Mesach und
Asarja Abed-Nego.

\bibverse{8} Aber Daniel setzte sich vor in seinem Herzen, dass er sich
mit des Königs Speise und mit dem Wein, den er selbst trank, nicht
verunreinigen wollte, und bat den obersten Kämmerer, dass er sich nicht
müsste verunreinigen. \bibverse{9} Und Gott gab Daniel, dass ihm der
oberste Kämmerer günstig und gnädig ward. \footnote{\textbf{1:9} 1Mo
  39,21} \bibverse{10} Derselbe sprach zu ihm: Ich fürchte mich vor
meinem Herrn, dem König, der euch eure Speise und Trank bestimmt hat;
wenn er würde sehen, dass eure Angesichter jämmerlicher wären denn der
anderen Knaben eures Alters, so brächtet ihr mich bei dem König um mein
Leben.

\bibverse{11} Da sprach Daniel zu dem Aufseher, welchem der oberste
Kämmerer Daniel, Hananja, Misael und Asarja befohlen hatte:
\bibverse{12} Versuche es doch mit deinen Knechten zehn Tage und lass
uns geben Gemüse zu essen und Wasser zu trinken. \bibverse{13} Und lass
dann vor dir unsere Gestalt und der Knaben, die von des Königs Speise
essen, besehen; und darnach du sehen wirst, darnach schaffe mit deinen
Knechten. \bibverse{14} Und er gehorchte ihnen darin und versuchte es
mit ihnen zehn Tage.

\bibverse{15} Und nach den zehn Tagen waren sie schöner und besser bei
Leibe denn alle Knaben, die von des Königs Speise aßen. \bibverse{16} Da
tat der Aufseher ihre verordnete Speise und Trank weg und gab ihnen
Gemüse.

\bibverse{17} Aber diesen vier Knaben gab Gott Kunst und Verstand in
allerlei Schrift und Weisheit; Daniel aber gab er Verstand in allen
Gesichten und Träumen.

\bibverse{18} Und da die Zeit um war, die der König bestimmt hatte, dass
sie sollten hineingebracht werden, brachte sie der oberste Kämmerer
hinein vor Nebukadnezar. \bibverse{19} Und der König redete mit ihnen,
und ward unter allen niemand gefunden, der Daniel, Hananja, Misael und
Asarja gleich wäre; und sie wurden des Königs Diener. \bibverse{20} Und
der König fand sie in allen Sachen, die er sie fragte, zehnmal klüger
und verständiger denn alle Sternseher und Weisen in seinem ganzen Reich.

\bibverse{21} Und Daniel erlebte das erste Jahr des König Kores.
\footnote{\textbf{1:21} Esr 1,1}

\hypertarget{section-1}{%
\section{2}\label{section-1}}

\bibverse{1} Im zweiten Jahr des Reiches Nebukadnezars hatte
Nebukadnezar einen Traum, davon er erschrak, dass er aufwachte.
\bibverse{2} Und er hieß alle Sternseher und Weisen und Zauberer und
Chaldäer zusammenfordern, dass sie dem König seinen Traum sagen sollten.
Und sie kamen und traten vor den König. \bibverse{3} Und der König
sprach zu ihnen: Ich habe einen Traum gehabt, der hat mich erschreckt;
und ich wollte gern wissen, was es für ein Traum gewesen sei.

\bibverse{4} Da sprachen die Chaldäer zum König auf chaldäisch: Der
König lebe ewiglich! Sage deinen Knechten den Traum, so wollen wir ihn
deuten. \footnote{\textbf{2:4} Dan 3,9}

\bibverse{5} Der König antwortete und sprach zu den Chaldäern: Es ist
mir entfallen. Werdet ihr mir den Traum nicht anzeigen und ihn deuten,
so sollt ihr in Stücke zerhauen und eure Häuser schändlich verstört
werden. \bibverse{6} Werdet ihr mir aber den Traum anzeigen und deuten,
so sollt ihr Geschenke, Gaben und große Ehre von mir haben. Darum so
sagt mir den Traum und seine Deutung.

\bibverse{7} Sie antworteten wiederum und sprachen: Der König sage
seinen Knechten den Traum, so wollen wir ihn deuten.

\bibverse{8} Der König antwortete und sprach: Wahrlich, ich merke es,
dass ihr Frist sucht, weil ihr seht, dass mir's entfallen ist.
\bibverse{9} Aber werdet ihr mir nicht den Traum sagen, so geht das
Recht über euch, als die ihr Lügen und Gedichte vor mir zu reden euch
vorgenommen habt, bis die Zeit vorübergehe. Darum so sagt mir den Traum,
so kann ich merken, dass ihr auch die Deutung trefft.

\bibverse{10} Da antworteten die Chaldäer vor dem König und sprachen zu
ihm: Es ist kein Mensch auf Erden, der sagen könne, was der König
fordert. So ist auch kein König, wie groß oder mächtig er sei, der
solches von irgendeinem Sternseher, Weisen oder Chaldäer fordere.
\bibverse{11} Denn was der König fordert, ist zu hoch, und ist auch
sonst niemand, der es vor dem König sagen könne, ausgenommen die Götter,
die bei den Menschen nicht wohnen.

\bibverse{12} Da ward der König sehr zornig und befahl, alle Weisen zu
Babel umzubringen. \bibverse{13} Und das Urteil ging aus, dass man die
Weisen töten sollte; und Daniel samt seinen Gesellen ward auch gesucht,
dass man sie tötete.

\bibverse{14} Da erwiderte Daniel klug und verständig dem Arioch, dem
obersten Richter des Königs, welcher auszog, zu töten die Weisen zu
Babel. \bibverse{15} Und er fing an und sprach zu des Königs Vogt,
Arioch: Warum ist ein so strenges Urteil vom König ausgegangan? Und
Arioch zeigte es dem Daniel an. \bibverse{16} Da ging Daniel hinein und
bat den König, dass er ihm Frist gäbe, damit er die Deutung dem König
sagen möchte.

\bibverse{17} Und Daniel ging heim und zeigte solches an seinen
Gesellen, Hananja, Misael und Asarja, \bibverse{18} dass sie den Gott
des Himmels um Gnade bäten solches verborgenen Dinges halben, damit
Daniel und seine Gesellen nicht samt den anderen Weisen zu Babel
umkämen. \bibverse{19} Da ward Daniel solch verborgenes Ding durch ein
Gesicht des Nachts offenbart. \footnote{\textbf{2:19} Dan 2,30}
\bibverse{20} Darüber lobte Daniel den Gott des Himmels, fing an und
sprach: Gelobt sei der Name Gottes von Ewigkeit zu Ewigkeit! denn sein
ist beides, Weisheit und Stärke. \bibverse{21} Er ändert Zeit und
Stunde; er setzt Könige ab und setzt Könige ein; er gibt den Weisen ihre
Weisheit und den Verständigen ihren Verstand; \bibverse{22} er
offenbart, was tief und verborgen ist; er weiß, was in der Finsternis
liegt, denn bei ihm ist eitel Licht. \bibverse{23} Ich danke dir und
lobe dich, Gott meiner Väter, dass du mir Weisheit und Stärke verleihest
und jetzt offenbart hast, darum wir dich gebeten haben; denn du hast uns
des Königs Sache offenbart.

\bibverse{24} Da ging Daniel hinein zu Arioch, der vom König Befehl
hatte, die Weisen zu Babel umzubringen, und sprach zu ihm also: Du
sollst die Weisen zu Babel nicht umbringen, sondern führe mich hinein
zum König, ich will dem König die Deutung sagen. \footnote{\textbf{2:24}
  Dan 2,14}

\bibverse{25} Arioch brachte Daniel eilends hinein vor den König und
sprach zu ihm also: Es ist einer gefunden unter den Gefangenen aus Juda,
der dem König die Deutung sagen kann.

\bibverse{26} Der König antwortete und sprach zu Daniel, den sie
Beltsazar hießen: Bist du, der mir den Traum, den ich gesehen habe, und
seine Deutung anzeigen kann?

\bibverse{27} Daniel fing an vor dem König und sprach: Das verborgene
Ding, das der König fordert von den Weisen, Gelehrten, Sternsehern und
Wahrsagern, steht in ihrem Vermögen nicht, dem König zu sagen.
\bibverse{28} Aber es ist ein Gott im Himmel, der kann verborgene Dinge
offenbaren; der hat dem König Nebukadnezar angezeigt, was in künftigen
Zeiten geschehen soll.

\bibverse{29} Mit deinem Traum und deinem Gesichten, da du schliefest,
verhielt sich's also: Du, König, dachtest auf deinem Bette, wie es doch
hernach gehen würde; und der, der verborgene Dinge offenbart, hat dir
angezeigt, wie es gehen werde. \bibverse{30} So ist mir solch
verborgenes Ding offenbart, nicht durch meine Weisheit, als wäre sie
größer denn aller, die da leben; sondern darum, dass dem König die
Deutung angezeigt würde und du deines Herzens Gedanken erführest.
\footnote{\textbf{2:30} 1Mo 41,16}

\bibverse{31} Du, König, sahest, und siehe, ein großes und hohes und
sehr glänzendes Bild stand vor dir, das war schrecklich anzusehen.
\bibverse{32} Des Bildes Haupt war von feinem Golde, seine Brust und
Arme waren von Silber, sein Bauch und seine Lenden waren von Erz,
\bibverse{33} seine Schenkel waren Eisen, seine Füße waren eines Teils
Eisen und eines Teils Ton. \bibverse{34} Solches sahest du, bis dass ein
Stein herabgerissen ward ohne Hände; der schlug das Bild an seine Füße,
die Eisen und Ton waren, und zermalmte sie. \bibverse{35} Da wurden
miteinander zermalmt das Eisen, Ton, Erz, Silber und Gold und wurden wie
Spreu auf der Sommertenne, und der Wind verwehte sie, dass man sie
nirgends mehr finden konnte. Der Stein aber, der das Bild schlug, ward
ein großer Berg, dass er die ganze Welt füllte.

\bibverse{36} Das ist der Traum. Nun wollen wir die Deutung vor dem
König sagen. \bibverse{37} Du, König, bist ein König aller Könige, dem
der Gott des Himmels Königreich, Macht, Stärke und Ehre gegeben hat
\bibverse{38} und alles, da Leute wohnen, dazu die Tiere auf dem Felde
und die Vögel unter dem Himmel in deine Hände gegeben und dir über alles
Gewalt verliehen hat. Du bist das goldene Haupt. \footnote{\textbf{2:38}
  Jer 27,6}

\bibverse{39} Nach dir wird ein anderes Königreich aufkommen, geringer
denn deines. Darnach das dritte Königreich, das ehern ist, welches wird
über alle Lande herrschen. \bibverse{40} Und das vierte wird hart sein
wie Eisen; denn gleichwie Eisen alles zermalmt und zerschlägt, ja, wie
Eisen alles zerbricht, also wird es auch diese alle zermalmen und
zerbrechen. \bibverse{41} Dass du aber gesehen hast die Füße und Zehen
eines Teils Ton und eines Teils Eisen: das wird ein zerteiltes
Königreich sein; doch wird von des Eisens Art darin bleiben, wie du es
denn gesehen hast Eisen mit Ton vermengt. \bibverse{42} Und dass die
Zehen an seinen Füßen eines Teils Eisen und eines Teils Ton sind: wird's
zum Teil ein starkes und zum Teil ein schwaches Reich sein.
\bibverse{43} Und dass du gesehen hast Eisen mit Ton vermengt: werden
sie sich wohl nach Menschengeblüt untereinander mengen, aber sie werden
doch nicht aneinander halten, gleichwie sich Eisen mit Ton nicht mengen
lässt.

\bibverse{44} Aber zur Zeit solcher Königreiche wird der Gott des
Himmels ein Königreich aufrichten, das nimmermehr zerstört wird; und
sein Königreich wird auf kein anderes Volk kommen. Es wird alle diese
Königreiche zermalmen und verstören; aber es selbst wird ewiglich
bleiben; \bibverse{45} wie du denn gesehen hast einen Stein, ohne Hände
vom Berge herabgerissen, der das Eisen, Erz, Ton, Silber und Gold
zermalmte. Also hat der große Gott dem König gezeigt, wie es hernach
gehen werde; und der Traum ist gewiss, und die Deutung ist recht.
\footnote{\textbf{2:45} Dan 2,34}

\bibverse{46} Da fiel der König Nebukadnezar auf sein Angesicht und
betete an vor dem Daniel und befahl, man sollte ihm Speisopfer und
Räuchopfer tun. \bibverse{47} Und der König antwortete Daniel und
sprach: Es ist kein Zweifel, euer Gott ist ein Gott über alle Götter und
ein HErr über alle Könige, der da kann verborgene Dinge offenbaren, weil
du dies verborgene Ding hast können offenbaren.

\bibverse{48} Und der König erhöhte Daniel und gab ihm große und viele
Geschenke und machte ihn zum Fürsten über die ganze Landschaft Babel und
setzte ihn zum Obersten über alle Weisen zu Babel. \footnote{\textbf{2:48}
  Dan 2,6} \bibverse{49} Und Daniel bat vom König, dass er über die
Ämter der Landschaft Babel setzen möchte Sadrach, Mesach und Abed-Nego;
und er, Daniel, blieb bei dem König am Hofe. \footnote{\textbf{2:49} Dan
  3,12}

\hypertarget{section-2}{%
\section{3}\label{section-2}}

\bibverse{1} Der König Nebukadnezar ließ ein goldenes Bild machen,
sechzig Ellen hoch und sechs Ellen breit, und ließ es setzen ins Tal
Dura in der Landschaft Babel. \bibverse{2} Und der König Nebukadnezar
sandte nach den Fürsten, Herren, Landpflegern, Richtern, Vögten, Räten,
Amtleuten und allen Gewaltigen im Lande, dass sie zusammenkommen
sollten, das Bild zu weihen, das der König Nebukadnezar hatte setzen
lassen. \bibverse{3} Da kamen zusammen die Fürsten, Herren, Landpfleger,
Richter, Vögte, Räte, Amtleute und alle Gewaltigen im Lande, das Bild zu
weihen, das der König Nebukadnezar hatte setzen lassen. Und sie mussten
dem Bilde gegenübertreten, das Nebukadnezar hatte setzen lassen.

\bibverse{4} Und der Herold rief überlaut: Das lasst euch gesagt sein,
ihr Völker, Leute und Zungen! \bibverse{5} Wenn ihr hören werdet den
Schall der Posaunen, Drommeten, Harfen, Geigen, Psalter, Lauten und
allerlei Saitenspiel, so sollt ihr niederfallen und das goldene Bild
anbeten, das der König Nebukadnezar hat setzen lassen. \bibverse{6} Wer
aber alsdann nicht niederfällt und anbetet, der soll von Stund an in den
glühenden Ofen geworfen werden.

\bibverse{7} Da sie nun hörten den Schall der Posaunen, Drommeten,
Harfen, Geigen, Psalter und allerlei Saitenspiel, fielen nieder alle
Völker, Leute und Zungen und beteten an das goldene Bild, das der König
Nebukadnezar hatte setzen lassen.

\bibverse{8} Von Stund an traten hinzu etliche chaldäische Männer und
verklagten die Juden, \bibverse{9} fingen an und sprachen zum König
Nebukadnezar: Der König lebe ewiglich! \bibverse{10} Du hast ein Gebot
lassen ausgehen, dass alle Menschen, wenn sie hören würden den Schall
der Posaunen, Drommeten, Harfen, Geigen, Psalter, Lauten und allerlei
Saitenspiel, sollten niederfallen und das goldene Bild anbeten;
\footnote{\textbf{3:10} Dan 6,13} \bibverse{11} wer aber nicht
niederfiele und anbetete, sollte in einen glühenden Ofen geworfen
werden. \bibverse{12} Nun sind da jüdische Männer, welche du über die
Ämter der Landschaft Babel gesetzt hast: Sadrach, Mesach und Abed-Nego;
die verachten dein Gebot und ehren deine Götter nicht und beten nicht an
das goldene Bild, das du hast setzen lassen.

\bibverse{13} Da befahl Nebukadnezar mit Grimm und Zorn, dass man vor
ihn stellte Sadrach, Mesach und Abed-Nego. Und die Männer wurden vor den
König gestellt. \bibverse{14} Da fing Nebukadnezar an und sprach zu
ihnen: Wie? wollt ihr Sadrach, Mesach, Abed-Nego, meinen Gott nicht
ehren und das goldene Bild nicht anbeten, das ich habe setzen lassen?
\bibverse{15} Wohlan, schickt euch! Sobald ihr hören werdet den Schall
der Posaunen, Drommeten, Harfen, Geigen, Psalter, Lauten und allerlei
Saitenspiel, so fallet nieder und betet das Bild an, das ich habe machen
lassen! Werdet ihr's nicht anbeten, so sollt ihr von Stund an in den
glühenden Ofen geworfen werden. Lasst sehen, wer der Gott sei, der euch
aus meiner Hand erretten werde! \footnote{\textbf{3:15} 2Kö 18,35}

\bibverse{16} Da fingen an Sadrach, Mesach, Abed-Nego und sprachen zum
König Nebukadnezar: Es ist nicht not, dass wir dir darauf antworten.
\bibverse{17} Siehe, unser Gott, den wir ehren, kann uns wohl erretten
aus dem glühenden Ofen, dazu auch von deiner Hand erretten.
\bibverse{18} Und wo er's nicht tun will, so sollst du dennoch wissen,
dass wir deine Götter nicht ehren noch das goldene Bild, das du hast
setzen lassen, anbeten wollen. \footnote{\textbf{3:18} 2Mo 20,3-5}

\bibverse{19} Da ward Nebukadnezar voll Grimms, und sein Angesicht
verstellte sich wider Sadrach, Mesach und Abed-Nego, und er befahl, man
sollte den Ofen siebenmal heißer machen, denn man sonst zu tun pflegte.
\bibverse{20} Und befahl den besten Kriegsleuten, die in seinem Heer
waren, dass sie Sadrach, Mesach und Abed-Nego bänden und in den
glühenden Ofen würfen. \bibverse{21} Also wurden diese Männer in ihren
Mänteln, Schuhen, Hüten und anderen Kleidern gebunden und in den
glühenden Ofen geworfen; \bibverse{22} denn des Königs Gebot musste man
eilends tun. Und man schürte das Feuer im Ofen so sehr, dass die Männer,
die den Sadrach, Mesach und Abed-Nego hinaufbrachten, verdarben von des
Feuers Flammen. \bibverse{23} Aber die drei Männer, Sadrach, Mesach und
Abed-Nego fielen hinab in den glühenden Ofen, wie sie gebunden waren.

\bibverse{24} Da entsetzte sich der König Nebukadnezar und fuhr auf und
sprach zu seinen Räten: Haben wir nicht drei Männer gebunden in das
Feuer lassen werfen? Sie antworteten und sprachen zum König: Ja, Herr
König.

\bibverse{25} Er antwortete und sprach: Sehe ich doch vier Männer frei
im Feuer gehen, und sie sind unversehrt; und der vierte ist gleich, als
wäre er ein Sohn der Götter.

\bibverse{26} Und Nebukadnezar trat hinzu vor das Loch des glühenden
Ofens und sprach: Sadrach, Mesach, Abed-Nego, ihr Knechte Gottes des
Höchsten, gehet heraus und kommet her! Da gingen Sadrach, Mesach und
Abed-Nego heraus aus dem Feuer.

\bibverse{27} Und die Fürsten, Herren, Vögte und Räte des Königs kamen
zusammen und sahen, dass das Feuer keine Macht am Leibe dieser Männer
bewiesen hatte und ihr Haupthaar nicht versengt und ihre Mäntel nicht
versehrt waren; ja man konnte keinen Brand an ihnen riechen. \footnote{\textbf{3:27}
  Hebr 11,34}

\bibverse{28} Da fing Nebukadnezar an und sprach: Gelobt sei der Gott
Sadrachs, Mesachs und Abed-Negos, der seinen Engel gesandt und seine
Knechte errettet hat, die ihm vertraut und des Königs Gebot nicht
gehalten, sondern ihren Leib dargegeben haben, dass sie keinen Gott
ehren noch anbeten wollten als allein ihren Gott! \footnote{\textbf{3:28}
  Dan 6,23}

\bibverse{29} So sei nun dies mein Gebot: Welcher unter allen Völkern,
Leuten und Zungen den Gott Sadrachs, Mesachs und Abed-Negos lästert, der
soll in Stücke zerhauen und sein Haus schändlich verstört werden. Denn
es ist kein anderer Gott, der also erretten kann, als dieser.
\footnote{\textbf{3:29} Dan 2,47}

\bibverse{30} Und der König gab Sadrach, Mesach und Abed-Nego große
Gewalt in der Landschaft Babel. \bibverse{31} König Nebukadnezar allen
Völkern, Leuten und Zungen auf der ganzen Erde: Viel Friede zuvor!
\bibverse{32} Ich sehe es für gut an, dass ich verkündige die Zeichen
und Wunder, die Gott der Höchste an mir getan hat. \bibverse{33} Denn
seine Zeichen sind groß, und seine Wunder mächtig, und sein Reich ist
ein ewiges Reich, und seine Herrschaft währet für und für. \# 4
\bibverse{1} Ich, Nebukadnezar, da ich gute Ruhe hatte in meinem Hause
und es wohl stand auf meiner Burg,

\bibverse{2} sah einen Traum und erschrak, und die Gedanken, die ich auf
meinem Bett hatte, und das Gesicht, das ich gesehen hatte, betrübten
mich.

\bibverse{3} Und ich befahl, dass alle Weisen zu Babel vor mich
hereingebracht würden, dass sie mir sagten, was der Traum bedeutete.

\bibverse{4} Da brachte man herein die Sternseher, Weisen, Chaldäer und
Wahrsager, und ich erzählte den Traum vor ihnen; aber sie konnten mir
nicht sagen, was er bedeutete, \footnote{\textbf{4:4} Dan 2,2}

\bibverse{5} bis zuletzt Daniel vor mich kam, welcher Beltsazar heißt
nach dem Namen meines Gottes, der den Geist der heiligen Götter hat. Und
ich erzählte vor ihm den Traum: \footnote{\textbf{4:5} Dan 5,11; Dan
  5,14} \bibverse{6} Beltsazar, du Oberster unter den Sternsehern, von
dem ich weiß, dass du den Geist der heiligen Götter hast und dir nichts
verborgen ist, sage, was das Gesicht meines Traumes, das ich gesehen
habe, bedeutet. \footnote{\textbf{4:6} Hes 28,3} \bibverse{7} Dies ist
aber das Gesicht, das ich gesehen habe auf meinem Bette: Siehe, es stand
ein Baum mitten im Lande, der war sehr hoch. \footnote{\textbf{4:7} Hes
  31,3-14} \bibverse{8} Und er wurde groß und mächtig, und seine Höhe
reichte bis an den Himmel, und er breitete sich aus bis ans Ende der
ganzen Erde.

\bibverse{9} Seine Äste waren schön und trugen viel Früchte, davon alles
zu essen hatte; alle Tiere auf dem Felde fanden Schatten unter ihm, und
die Vögel unter dem Himmel saßen auf seinen Ästen, und alles Fleisch
nährte sich von ihm. \footnote{\textbf{4:9} Dan 4,18; Hes 17,23}
\bibverse{10} Und ich sah ein Gesicht auf meinem Bette, und siehe, ein
heiliger Wächter fuhr vom Himmel herab; \bibverse{11} der rief überlaut
und sprach also: Hauet den Baum um und behaut ihm die Äste und streift
ihm das Laub ab und zerstreuet seine Früchte, dass die Tiere, die unter
ihm liegen, weglaufen und die Vögel von seinen Zweigen fliehen.
\bibverse{12} Doch lasst den Stock mit seinen Wurzeln in der Erde
bleiben; er aber soll in eisernen und ehernen Ketten auf dem Felde im
Grase und unter dem Tau des Himmels liegen und nass werden und soll sich
weiden mit den Tieren von den Kräutern der Erde.

\bibverse{13} Und das menschliche Herz soll von ihm genommen und ein
viehisches Herz ihm gegeben werden, bis dass sieben Zeiten über ihm um
sind. \footnote{\textbf{4:13} Dan 7,25} \bibverse{14} Solches ist im Rat
der Wächter beschlossen und im Gespräch der Heiligen beratschlagt, auf
dass die Lebendigen erkennen, dass der Höchste Gewalt hat über der
Menschen Königreiche und gibt sie, wem er will, und erhöht die Niedrigen
zu denselben. \footnote{\textbf{4:14} Dan 2,21} \bibverse{15} Solchen
Traum habe ich, König Nebukadnezar, gesehen; du aber Beltsazar, sage,
was er bedeutet. Denn alle Weisen in meinem Königreiche können mir nicht
anzeigen, was er bedeute; du aber kannst es wohl, denn der Geist der
heiligen Götter ist bei dir. \bibverse{16} Da entsetzte sich Daniel, der
sonst Beltsazar heißt, bei einer Stunde lang, und seine Gedanken
betrübten ihn. Aber der König sprach: Beltsazar, lass dich den Traum und
seine Deutung nicht betrüben. Beltsazar fing an und sprach: Ach mein
Herr, dass der Traum deinen Feinden und seine Deutung deinen
Widersachern gölte!

\bibverse{17} Der Baum, den du gesehen hast, dass er groß und mächtig
ward und seine Höhe an den Himmel reichte und dass er sich über die
ganze Erde breitete

\bibverse{18} und seine Äste schön waren und seiner Früchte viel, davon
alles zu essen hatte, und dass die Tiere auf dem Felde unter ihm wohnten
und die Vögel des Himmels auf seinen Ästen saßen:

\bibverse{19} das bist du, König, der du so groß und mächtig geworden;
denn deine Macht ist groß und reicht an den Himmel, und deine Gewalt
langt bis an der Welt Ende.

\bibverse{20} Dass aber der König einen heiligen Wächter gesehen hat vom
Himmel herabfahren und sagen: Haut den Baum um und verderbet ihn; doch
den Stock mit seinen Wurzeln lasst in der Erde bleiben; er aber soll in
eisernen und ehernen Ketten auf dem Felde im Grase gehen und unter dem
Tau des Himmels liegen und nass werden und sich mit den Tieren auf dem
Felde weiden, bis über ihm sieben Zeiten um sind, -- \bibverse{21} das
ist die Deutung, Herr König, und solcher Rat des Höchsten geht über
meinen Herrn König: \bibverse{22} Man wird dich von den Leuten stoßen,
und du musst bei den Tieren auf dem Felde bleiben, und man wird dich
Gras essen lassen wie die Ochsen, und wirst unter dem Tau des Himmels
liegen und nass werden, bis über dir sieben Zeiten um sind, auf dass du
erkennest, dass der Höchste Gewalt hat über der Menschen Königreiche und
gibt sie, wem er will.

\bibverse{23} Dass aber gesagt ist, man solle dennoch den Stock des
Baumes mit seinen Wurzeln bleiben lassen: dein Königreich soll dir
bleiben, wenn du erkannt hast die Gewalt im Himmel.

\bibverse{24} Darum, Herr König, lass dir meinen Rat gefallen und mache
dich los von deinen Sünden durch Gerechtigkeit und ledig von deiner
Missetat durch Wohltat an den Armen, so wird dein Glück lange währen.
\footnote{\textbf{4:24} Spr 19,17; Mt 5,7; Mt 19,21} \bibverse{25} Dies
alles widerfuhr dem König Nebukadnezar. \bibverse{26} Denn nach zwölf
Monaten, da der König auf der königlichen Burg zu Babel ging,
\bibverse{27} hob er an und sprach: Das ist die große Babel, die ich
erbaut habe zum königlichen Hause durch meine große Macht, zu Ehren
meiner Herrlichkeit.

\bibverse{28} Ehe der König diese Worte ausgeredet hatte, fiel eine
Stimme vom Himmel: Dir, König Nebukadnezar, wird gesagt: Dein Königreich
soll dir genommen werden; \bibverse{29} und man wird dich von den Leuten
verstoßen, und sollst bei den Tieren, die auf dem Feld gehen, bleiben;
Gras wird man dich essen lassen wie Ochsen, bis dass über dir sieben
Zeiten um sind, -- auf dass du erkennest, dass der Höchste Gewalt hat
über der Menschen Königreiche und gibt sie, wem er will. \footnote{\textbf{4:29}
  Dan 5,21} \bibverse{30} Von Stund an ward das Wort vollbracht über
Nebukadnezar, und er ward verstoßen von den Leuten hinweg, und er aß
Gras wie Ochsen, und sein Leib lag unter dem Tau des Himmels, und er
ward nass, bis sein Haar wuchs so groß wie Adlersfedern und seine Nägel
wie Vogelsklauen wurden.

\bibverse{31} Nach dieser Zeit hob ich, Nebukadnezar, meine Augen auf
gen Himmel und kam wieder zur Vernunft und lobte den Höchsten. Ich pries
und ehrte den, der ewiglich lebt, des Gewalt ewig ist und des Reich für
und für währt, \bibverse{32} gegen welchen alle, die auf Erden wohnen,
als nichts zu rechnen sind. Er macht's, wie er will, mit den Kräften im
Himmel und mit denen, die auf Erden wohnen; und niemand kann seiner Hand
wehren noch zu ihm sagen: Was machst du?

\bibverse{33} Zur selben Zeit kam ich wieder zur Vernunft, auch zu
meinen königlichen Ehren, zu meiner Herrlichkeit und zu meiner Gestalt.
Und meine Räte und Gewaltigen suchten mich, und ich ward wieder in mein
Königreich gesetzt; und ich überkam noch größere Herrlichkeit.

\bibverse{34} Darum lobe ich, Nebukadnezar, und ehre und preise den
König des Himmels; denn all sein Tun ist Wahrheit, und seine Wege sind
recht, und wer stolz ist, den kann er demütigen. \footnote{\textbf{4:34}
  Dan 5,20; Lk 1,51; Lk 18,14}

\hypertarget{section-3}{%
\section{5}\label{section-3}}

\bibverse{1} König Belsazer machte ein herrliches Mahl seinen tausend
Gewaltigen und soff sich voll mit ihnen. \footnote{\textbf{5:1} Dan 7,1}
\bibverse{2} Und da er trunken war, hieß er die goldenen und silbernen
Gefäße herbringen, die sein Vater Nebukadnezar aus dem Tempel zu
Jerusalem weggenommen hatte, dass der König mit seinen Gewaltigen, mit
seinen Weibern und mit seinen Kebsweibern daraus tränken. \footnote{\textbf{5:2}
  Dan 1,2; 2Chr 36,10} \bibverse{3} Also wurden hergebracht die goldenen
Gefäße, die aus dem Tempel, aus dem Haus Gottes zu Jerusalem, genommen
waren; und der König, seine Gewaltigen, seine Weiber und Kebsweiber
tranken daraus. \bibverse{4} Und da sie so soffen, lobten sie die
goldenen, silbernen, ehernen, eisernen, hölzernen und steinernen Götter.
\bibverse{5} Eben zur selben Stunde gingen hervor Finger wie einer
Menschenhand, die schrieben, gegenüber dem Leuchter, auf die getünchte
Wand in dem königlichen Saal; und der König ward gewahr der Hand, die da
schrieb. \bibverse{6} Da entfärbte sich der König, und seine Gedanken
erschreckten ihn, dass ihm die Lenden schütterten und die Beine
zitterten. \bibverse{7} Und der König rief überlaut, dass man die
Weisen, Chaldäer und Wahrsager hereinbringen sollte. Und er ließ den
Weisen zu Babel sagen: Welcher Mensch diese Schrift liest und sagen
kann, was sie bedeute, der soll in Purpur gekleidet werden und eine
goldene Kette am Halse tragen und der dritte Herr sein in meinem
Königreiche. \bibverse{8} Da wurden alle Weisen des Königs
hereingebracht; aber sie konnten weder die Schrift lesen noch die
Deutung dem König anzeigen. \bibverse{9} Darüber erschrak der König
Belsazer noch härter und verlor ganz seine Farbe; und seinen Gewaltigen
ward bange. \bibverse{10} Da ging die Königin um solcher Sache des
Königs und seiner Gewaltigen willen hinein in den Saal und sprach: Der
König lebe ewiglich! Lass dich deine Gedanken nicht so erschrecken und
entfärbe dich nicht also! \bibverse{11} Es ist ein Mann in deinem
Königreich, der den Geist der heiligen Götter hat. Denn zu deines Vaters
Zeit ward bei ihm Erleuchtung gefunden, Klugheit und Weisheit, wie der
Götter Weisheit ist; und dein Vater, König Nebukadnezar, setzte ihn über
die Sternseher, Weisen, Chaldäer und Wahrsager, \footnote{\textbf{5:11}
  Dan 4,5} \bibverse{12} darum dass ein hoher Geist bei ihm gefunden
ward, dazu Verstand und Klugheit, Träume zu deuten, dunkle Sprüche zu
erraten und verborgene Sachen zu offenbaren: nämlich Daniel, den der
König ließ Beltsazar nennen. So rufe man nun Daniel der wird sagen, was
es bedeutet. \footnote{\textbf{5:12} Hes 28,3} \bibverse{13} Da ward
Daniel hinein vor den König gebracht. Und der König sprach zu Daniel:
Bist du der Daniel, der Gefangenen einer aus Juda, die der König, mein
Vater, aus Juda hergebracht hat? \bibverse{14} Ich habe von dir hören
sagen, dass du den Geist der Götter habest und Erleuchtung, Verstand und
hohe Weisheit bei dir gefunden sei. \bibverse{15} Nun habe ich vor mich
fordern lassen die Klugen und Weisen, dass sie mir diese Schrift lesen
und anzeigen sollen, was sie bedeutet: und sie können mir nicht sagen,
was solches bedeutet. \bibverse{16} Von dir aber höre ich, dass du
könnest Deutungen geben und das Verborgene offenbaren. Kannst du nun die
Schrift lesen und mir anzeigen, was sie bedeutet, so sollst du mit
Purpur gekleidet werden und eine golden Kette an deinem Halse tragen und
der dritte Herr sein in meinem Königreiche. \bibverse{17} Da fing Daniel
an und redete vor dem König: Behalte deine Gaben selbst und gib dein
Geschenk einem anderen; ich will dennoch die Schrift dem König lesen und
anzeigen, was sie bedeutet. \bibverse{18} Herr König, Gott der Höchste
hat deinem Vater, Nebukadnezar, Königreich, Macht, Ehre und Herrlichkeit
gegeben. \footnote{\textbf{5:18} Dan 2,37; Dan 4,22} \bibverse{19} Und
vor solcher Macht, die ihm gegeben war, fürchteten und scheuten sich vor
ihm alle Völker, Leute und Zungen. Er tötete, wen er wollte; er ließ
leben, wen er wollte; er erhöhte, wen er wollte; er demütigte, wen er
wollte. \bibverse{20} Da sich aber sein Herz erhob und er stolz und
hochmütig ward, ward er vom königlichen Stuhl gestoßen und verlor seine
Ehre \bibverse{21} und ward verstoßen von den Leuten hinweg, und sein
Herz ward gleich den Tieren, und er musste bei dem Wild laufen und fraß
Gras wie Ochsen, und sein Leib lag unter dem Tau des Himmels, und er
ward nass, bis dass er lernte, dass Gott der Höchste Gewalt hat über der
Menschen Königreiche und gibt sie, wem er will. \bibverse{22} Und du,
Belsazer, sein Sohn, hast dein Herz nicht gedemütigt, ob du wohl solches
alles weißt, \bibverse{23} sondern hast dich wider den HErrn des Himmels
erhoben, und die Gefäße seines Hauses hat man vor dich bringen müssen,
und du, deine Gewaltigen, deine Weiber und deine Kebsweiber habt daraus
getrunken, dazu die silbernen, goldenen, ehernen, eisernen, hölzernen,
steinernen Götter gelobt, die weder sehen noch hören noch fühlen; den
Gott aber, der deinen Odem und alle deine Wege in seiner Hand hat, hast
du nicht geehrt. \footnote{\textbf{5:23} Dan 5,2; Ps 115,4-7}
\bibverse{24} Darum ist von ihm gesandt diese Hand und diese Schrift,
die da verzeichnet steht. \bibverse{25} Das aber ist die Schrift allda
verzeichnet: Mene, mene, tekel, U-pharsin. \bibverse{26} Und sie
bedeutet dies: Mene, das ist: Gott hat dein Königreich gezählt und
vollendet. \bibverse{27} Tekel, das ist: man hat dich in einer Waage
gewogen und zu leicht gefunden. \bibverse{28} Peres, das ist: dein
Königreich ist zerteilt und den Medern und Persern gegeben. --
\bibverse{29} Da befahl Belsazer, dass man Daniel mit Purpur kleiden
sollte und ihm eine goldene Kette an den Hals geben, und ließ von ihm
verkündigen, dass er der dritte Herr sei im Königreich. \bibverse{30}
Aber in derselben Nacht ward der Chaldäer König Belsazer getötet. \# 6
\bibverse{1} Und Darius aus Medien nahm das Reich ein, da er
zweiundsechzig Jahre alt war. \footnote{\textbf{6:1} Dan 9,1; Jes 13,17}
\bibverse{2} Und Darius sah es für gut an, dass er über das ganze
Königreich setzte hundertzwanzig Landvögte. \bibverse{3} Über diese
setzte er drei Fürsten, deren einer Daniel war, welchen die Landvögte
sollten Rechnung tun, dass der König keinen Schaden litte. \bibverse{4}
Daniel aber übertraf die Fürsten und Landvögte alle, denn es war ein
hoher Geist in ihm; darum gedachte der König, ihn über das ganze
Königreich zu setzen. \bibverse{5} Derhalben trachteten die Fürsten und
Landvögte darnach, wie sie eine Sache an Daniel fänden, die wider das
Königreich wäre. Aber sie konnten keine Sache noch Übeltat finden; denn
er war treu, dass man keine Schuld noch Übeltat an ihm finden mochte.
\bibverse{6} Da sprachen die Männer: Wir werden keine Sache an Daniel
finden außer seinem Gottesdienst. \bibverse{7} Da kamen die Fürsten und
Landvögte zuhauf vor den König und sprachen zu ihm also: Der König
Darius lebe ewiglich! \footnote{\textbf{6:7} Dan 3,9; Dan 5,10}
\bibverse{8} Es haben die Fürsten des Königreichs, die Herren, die
Landvögte, die Räte und Hauptleute alle gedacht, dass man einen
königlichen Befehl soll ausgehen lassen und ein strenges Gebot stellen,
dass, wer in dreißig Tagen etwas bitten wird von irgendeinem Gott oder
Menschen außer dir, König, allein, solle zu den Löwen in den Graben
geworfen werden. \bibverse{9} Darum, lieber König, sollst du solch Gebot
bestätigen und dich unterschreiben, auf dass es nicht wieder geändert
werde, nach dem Rechte der Meder und Perser, welches niemand aufheben
darf. \bibverse{10} Also unterschrieb sich der König Darius.
\bibverse{11} Als nun Daniel erfuhr, dass solch Gebot unterschrieben
wäre, ging er hinein in sein Haus (er hatte aber an seinem Söller offene
Fenster gegen Jerusalem); und er fiel des Tages dreimal auf seine Knie,
betete, lobte und dankte seinem Gott, wie er denn bisher zu tun pflegte.
\footnote{\textbf{6:11} 1Kö 8,48; Jer 51,50; Ps 55,18} \bibverse{12} Da
kamen diese Männer zuhauf und fanden Daniel beten und flehen vor seinem
Gott. \bibverse{13} Und traten hinzu und redeten mit dem König von dem
königlichen Gebot: Herr König, hast du nicht ein Gebot unterschrieben,
dass, wer in dreißig Tagen etwas bitten würde von irgendeinem Gott oder
Menschen außer dir, König, allein, solle zu den Löwen in den Graben
geworfen werden? Der König antwortete und sprach: Es ist wahr, und das
Recht der Meder und Perser soll niemand aufheben. \bibverse{14} Sie
antworteten und sprachen vor dem König: Daniel, der Gefangenen aus Juda
einer, der achtet weder dich noch dein Gebot, das du verzeichnet hast;
denn er betet des Tages dreimal. \bibverse{15} Da der König solches
hörte, ward er sehr betrübt und tat großen Fleiß, dass er Daniel
erlöste, und mühte sich, bis die Sonne unterging, dass er ihn errettete.
\bibverse{16} Aber die Männer kamen zuhauf zu dem König und sprachen zu
ihm: Du weißt, Herr König, dass der Meder und Perser Recht ist, dass
alle Gebote und Befehle, die der König beschlossen hat, sollen
unverändert bleiben. \bibverse{17} Da befahl der König, dass man Daniel
herbrächte; und sie warfen ihn zu den Löwen in den Graben. Der König
aber sprach zu Daniel: Dein Gott, dem du ohne Unterlass dienst, der
helfe dir! \footnote{\textbf{6:17} Dan 6,21} \bibverse{18} Und sie
brachten einen Stein, den legten sie vor die Tür am Graben; den
versiegelte der König mit seinem eigenen Ring und mit dem Ring seiner
Gewaltigen, auf dass nichts anderes mit Daniel geschähe. \bibverse{19}
Und der König ging weg in seine Burg und blieb ungegessen und ließ kein
Essen vor sich bringen, konnte auch nicht schlafen. \bibverse{20} Des
Morgens früh, da der Tag anbrach, stand der König auf und ging eilend
zum Graben, da die Löwen waren. \bibverse{21} Und als er zum Graben kam,
rief er Daniel mit kläglicher Stimme. Und der König sprach zu Daniel:
Daniel, du Knecht des lebendigen Gottes, hat dich auch dein Gott, dem du
ohne Unterlass dienest, können von den Löwen erlösen? \bibverse{22}
Daniel aber redete mit dem König: Der König lebe ewiglich! \footnote{\textbf{6:22}
  Dan 6,7} \bibverse{23} Mein Gott hat seinen Engel gesandt, der den
Löwen den Rachen zugehalten hat, dass sie mir kein Leid getan haben;
denn vor ihm bin ich unschuldig erfunden; so habe ich auch wider dich,
Herr König, nichts getan. \footnote{\textbf{6:23} Dan 3,28; Hebr 11,33}
\bibverse{24} Da ward der König sehr froh und hieß Daniel aus dem Graben
ziehen. Und sie zogen Daniel aus dem Graben, und man spürte keinen
Schaden an ihm; denn er hatte seinem Gott vertraut. \footnote{\textbf{6:24}
  Ps 37,40} \bibverse{25} Da hieß der König die Männer, die Daniel
verklagt hatten, herbringen und zu den Löwen in den Graben werfen samt
ihren Kindern und Weibern. Und ehe sie auf den Boden hinabkamen,
ergriffen sie die Löwen und zermalmten alle ihre Gebeine. \bibverse{26}
Da ließ der König Darius schreiben allen Völkern, Leuten und Zungen auf
der ganzen Erde: „Viel Friede zuvor! \bibverse{27} Das ist mein Befehl,
dass man in der ganzen Herrschaft meines Königreichs den Gott Daniels
fürchten und scheuen soll. Denn er ist der lebendige Gott, der ewiglich
bleibt, und sein Königreich ist unvergänglich, und seine Herrschaft hat
kein Ende. \bibverse{28} Er ist ein Erlöser und Nothelfer, und er tut
Zeichen und Wunder im Himmel und auf Erden. Der hat Daniel von den Löwen
erlöst.`` \bibverse{29} Und Daniel ward gewaltig im Königreich des
Darius und auch im Königreich des Kores, des Persers. \footnote{\textbf{6:29}
  Dan 1,21}

\hypertarget{section-4}{%
\section{7}\label{section-4}}

\bibverse{1} Im ersten Jahr Belsazers, des Königs zu Babel, hatte Daniel
einen Traum und Gesichte auf seinem Bett; und er schrieb den Traum auf
und verfasste ihn also: \footnote{\textbf{7:1} Dan 5,1} \bibverse{2}
Ich, Daniel, sah ein Gesicht in der Nacht, und siehe, die vier Winde
unter dem Himmel stürmten widereinander auf dem großen Meer. \footnote{\textbf{7:2}
  Offb 17,15} \bibverse{3} Und vier große Tiere stiegen herauf aus dem
Meer, ein jedes anders denn das andere. \footnote{\textbf{7:3} Offb
  13,1-2} \bibverse{4} Das erste wie ein Löwe und hatte Flügel wie ein
Adler. Ich sah zu, bis dass ihm die Flügel ausgerauft wurden; und es
ward von der Erde aufgehoben, und es stand auf zwei Füßen wie ein
Mensch, und ihm ward ein menschlich Herz gegeben. \footnote{\textbf{7:4}
  Dan 4,31} \bibverse{5} Und siehe, das andere Tier hernach war gleich
einem Bären und stand auf der einen Seite und hatte in seinem Maul unter
seinen Zähnen drei große, lange Zähne. Und man sprach zu ihm: Stehe auf
und friss viel Fleisch! \bibverse{6} Nach diesem sah ich, und siehe, ein
anderes Tier, gleich einem Parder, das hatte vier Flügel wie ein Vogel
auf seinem Rücken, und das Tier hatte vier Köpfe; und ihm ward Gewalt
gegeben. \bibverse{7} Nach diesem sah ich in diesem Gesicht in der
Nacht, und siehe, das vierte Tier war gräulich und schrecklich und sehr
stark und hatte große eiserne Zähne, fraß um sich und zermalmte, und das
Übrige zertrat's mit seinen Füßen; es war auch viel anders denn die
vorigen und hatte zehn Hörner. \bibverse{8} Da ich aber die Hörner
schaute, siehe, da brach hervor zwischen ihnen ein anderes kleines Horn,
vor welchem der vorigen Hörner drei ausgerissen wurden; und siehe,
dasselbe Horn hatte Augen wie Menschenaugen und ein Maul, das redete
große Dinge. \bibverse{9} Solches sah ich, bis dass Stühle gesetzt
wurden; und der Alte setzte sich. Des Kleid war schneeweiß, und das Haar
auf seinem Haupt wie reine Wolle; sein Stuhl war eitel Feuerflammen, und
dessen Räder brannten mit Feuer. \footnote{\textbf{7:9} Ps 90,2}
\bibverse{10} Und von ihm ging aus ein langer feuriger Strahl.
Tausendmal tausend dienten ihm, und zehntausendmal zehntausend standen
vor ihm. Das Gericht ward gehalten, und die Bücher wurden aufgetan.
\footnote{\textbf{7:10} Ps 68,18; Offb 5,11} \bibverse{11} Ich sah zu um
der großen Reden willen, die das Horn redete; ich sah zu bis das Tier
getötet ward und sein Leib umkam und ins Feuer geworfen ward \footnote{\textbf{7:11}
  Offb 19,20} \bibverse{12} und der anderen Tiere Gewalt auch aus war;
denn es war ihnen Zeit und Stunde bestimmt, wie lange ein jegliches
währen sollte. \footnote{\textbf{7:12} Dan 2,21} \bibverse{13} Ich sah
in diesem Gesichte des Nachts, und siehe, es kam einer in des Himmels
Wolken wie eines Menschen Sohn bis zu dem Alten und ward vor ihn
gebracht. \footnote{\textbf{7:13} Lk 21,27} \bibverse{14} Der gab ihm
Gewalt, Ehre und Reich, dass ihm alle Völker, Leute und Zungen dienen
sollten. Seine Gewalt ist ewig, die nicht vergeht, und sein Königreich
hat kein Ende. \bibverse{15} Ich, Daniel, entsetzte mich davor, und
solches Gesicht erschreckte mich. \bibverse{16} Und ich ging zu der
einem, die dastanden, und bat ihn, dass er mir von dem allem gewissen
Bericht gäbe. Und er redete mit mir und zeigte mir, was es bedeutete.
\bibverse{17} Diese vier großen Tiere sind vier Reiche, die auf Erden
kommen werden. \bibverse{18} Aber die Heiligen des Höchsten werden das
Reich einnehmen und werden's immer und ewiglich besitzen. \footnote{\textbf{7:18}
  Dan 7,22} \bibverse{19} Darnach hätte ich gern gewusst gewissen
Bericht von dem vierten Tier, welches gar anders war denn die anderen
alle, sehr gräulich, das eiserne Zähne und eherne Klauen hatte, das um
sich fraß und zermalmte und das Übrige mit seinen Füßen zertrat;
\footnote{\textbf{7:19} Dan 7,7} \bibverse{20} und von den zehn Hörnern
auf seinem Haupt und von dem anderen, das hervorbrach, vor welchem drei
abfielen; und das Horn hatte Augen und ein Maul, das große Dinge redete,
und war größer, denn die neben ihm waren. \bibverse{21} Und ich sah das
Horn streiten wider die Heiligen, und es behielt den Sieg wider sie,
\footnote{\textbf{7:21} Offb 13,7} \bibverse{22} bis der Alte kam und
Gericht hielt für die Heiligen des Höchsten, und die Zeit kam, dass die
Heiligen das Reich einnahmen. \bibverse{23} Er sprach also: Das vierte
Tier wird das vierte Reich auf Erden sein, welches wird gar anders sein
denn alle Reiche; es wird alle Lande fressen, zertreten und zermalmen.
\bibverse{24} Die zehn Hörner bedeuten zehn Könige, die aus dem Reich
entstehen werden. Nach ihnen aber wird ein anderer aufkommen, der wird
gar anders sein denn die vorigen und wird drei Könige demütigen.
\bibverse{25} Er wird den Höchsten lästern und die Heiligen des Höchsten
verstören und wird sich unterstehen, Zeit und Gesetz zu ändern. Sie
werden aber in sein Hand gegeben werden eine Zeit und zwei Zeiten und
eine halbe Zeit. \footnote{\textbf{7:25} Offb 13,5-6; Dan 12,7; Dan 4,13}
\bibverse{26} Darnach wird das Gericht gehalten werden; da wird dann
seine Gewalt weggenommen werden, dass er zu Grund vertilgt und
umgebracht werde. \bibverse{27} Aber das Reich, Gewalt und Macht unter
dem ganzen Himmel wird dem heiligen Volk des Höchsten gegeben werden,
des Reich ewig ist, und alle Gewalt wird ihm dienen und gehorchen.
\bibverse{28} Das war der Rede Ende. Aber ich, Daniel, ward sehr betrübt
in meinen Gedanken, und meine Gestalt verfiel; doch behielt ich die Rede
in meinem Herzen. \# 8 \bibverse{1} Im dritten Jahr des Königreichs des
Königs Belsazer erschien mir, Daniel, ein Gesicht nach dem, so mir
zuerst erschienen war. \bibverse{2} Ich war aber in solchem Gesicht zu
Schloss Susan im Lande Elam, am Wasser Ulai. \bibverse{3} Und ich hob
meine Augen auf und sah, und siehe, ein Widder stand vor dem Wasser, der
hatte zwei hohe Hörner, doch eins höher denn das andere, und das höchste
wuchs am letzten. \bibverse{4} Ich sah, dass der Widder mit den Hörnern
stieß gegen Abend, gegen Mitternacht und gegen Mittag; und kein Tier
konnte vor ihm bestehen noch von seiner Hand errettet werden, sondern er
tat, was er wollte, und ward groß. \bibverse{5} Und indem ich darauf
merkte, siehe, da kommt ein Ziegenbock vom Abend her über die ganze
Erde, dass er die Erde nicht berührte; und der Bock hatte ein
ansehnliches Horn zwischen seinen Augen. \bibverse{6} Und er kam bis zu
dem Widder der zwei Hörner hatte, den ich stehen sah vor dem Wasser, und
er lief in seinem Zorn gewaltig auf ihn zu. \bibverse{7} Und ich sah ihm
zu, dass er hart an den Widder kam, und er ergrimmte über ihn und stieß
den Widder und zerbrach ihm seine zwei Hörner. Und der Widder hatte
keine Kraft, dass er vor ihm hätte können bestehen; sondern er warf ihn
zu Boden und zertrat ihn und niemand konnte den Widder von seiner Hand
erretten. \bibverse{8} Und der Ziegenbock ward sehr groß. Und da er am
stärksten geworden war, zerbrach das große Horn, und wuchsen an seiner
Statt ansehnliche vier gegen die vier Winde des Himmels. \bibverse{9}
Und aus einem wuchs ein kleines Horn; das ward sehr groß gegen Mittag,
gegen Morgen und gegen das werte Land. \footnote{\textbf{8:9} Dan 7,8;
  Dan 11,16} \bibverse{10} Und es wuchs bis an des Himmels Heer und warf
etliche davon und von den Sternen zur Erde und zertrat sie.
\bibverse{11} Ja es wuchs bis an den Fürsten des Heeres und nahm von ihm
weg das tägliche Opfer und verwüstete die Wohnung seines Heiligtums.
\bibverse{12} Es ward ihm aber solche Macht gegeben wider das tägliche
Opfer um der Sünde willen, dass es die Wahrheit zu Boden schlüge, und
was es tat, ihm gelingen musste. \bibverse{13} Ich hörte aber einen
Heiligen reden; und ein Heiliger sprach zu dem, der da redete: Wie lange
soll doch währen solch Gesicht vom täglichen Opfer und von der Sünde, um
welcher willen diese Verwüstung geschieht, dass beide, das Heiligtum und
das Heer, zertreten werden? \bibverse{14} Und er antwortete mir: Bis
zweitausend dreihundert Abende und Morgen um sind; dann wird das
Heiligtum wieder geweiht werden. \bibverse{15} Und da ich, Daniel, solch
Gesicht sah und hätte es gern verstanden, siehe, da stand's vor mir wie
ein Mann. \bibverse{16} Und ich hörte mitten vom Ulai her einen mit
Menschenstimme rufen und sprechen: Gabriel, lege diesem das Gesicht aus,
dass er's verstehe! \footnote{\textbf{8:16} Dan 9,21} \bibverse{17} Und
er trat nahe zu mir. Ich erschrak aber, da er kam, und fiel auf mein
Angesicht. Er aber sprach zu mir: Merke auf, du Menschenkind! denn dies
Gesicht gehört in die Zeit des Endes. \footnote{\textbf{8:17} Dan 10,9}
\bibverse{18} Und da er mit mir redete, sank ich in eine Ohnmacht zur
Erde auf mein Angesicht. Er aber rührte mich an und richtete mich auf,
dass ich stand. \bibverse{19} Und er sprach: Siehe, ich will dir zeigen,
wie es gehen wird zur Zeit des letzten Zorns; denn das Ende hat seine
bestimmte Zeit. \bibverse{20} Der Widder mit den zwei Hörnern, den du
gesehen hast, sind die Könige in Medien und Persien. \bibverse{21} Der
Ziegenbock aber ist der König in Griechenland. Das große Horn zwischen
seinen Augen ist der erste König. \bibverse{22} Dass aber vier an seiner
Statt standen, da es zerbrochen war, bedeutet, dass vier Königreiche aus
dem Volk entstehen werden, aber nicht so mächtig, wie er war.
\bibverse{23} In der letzten Zeit ihres Königreichs, wenn die Übertreter
überhandnehmen, wird aufkommen ein frecher und tückischer König.
\footnote{\textbf{8:23} Dan 11,21; 1Makk 1,11} \bibverse{24} Der wird
mächtig sein, doch nicht durch seine Kraft; er wird gräulich verwüsten,
und es wird ihm gelingen, dass er's ausrichte. Er wird die Starken samt
dem heiligen Volk verstören. \bibverse{25} Und durch seine Klugheit wird
ihm der Betrug geraten, und er wird sich in seinem Herzen erheben, und
mitten im Frieden wird er viele verderben und wird sich auflehnen wider
den Fürsten allen Fürsten; aber er wird ohne Hand zerbrochen werden.
\bibverse{26} Dies Gesicht vom Abend und Morgen, das dir gesagt ist, das
ist wahr; aber du sollst das Gesicht heimlich halten, denn es ist noch
eine lange Zeit dahin. \bibverse{27} Und ich, Daniel, ward schwach und
lag etliche Tage krank. Darnach stand ich auf und richtete aus des
Königs Geschäft. Und verwunderte mich des Gesichts; und niemand war, der
mir's auslegte. \# 9 \bibverse{1} Im ersten Jahr des Darius, des Sohnes
des Ahasveros, aus der Meder Stamm, der über das Königreich der Chaldäer
König ward, \footnote{\textbf{9:1} Dan 6,1} \bibverse{2} in diesem
ersten Jahr seines Königreichs merkte ich, Daniel, in den Büchern auf
die Zahl der Jahre, davon der HErr geredet hatte zum Propheten Jeremia,
dass Jerusalem sollte siebzig Jahre wüst liegen. \footnote{\textbf{9:2}
  Jer 25,11-12} \bibverse{3} Und ich kehrte mich zu Gott dem HErrn, zu
beten und zu flehen mit Fasten im Sack und in der Asche. \bibverse{4}
Ich betete aber zu dem HErrn, meinem Gott, bekannte und sprach: Ach
lieber HErr, du großer und schrecklicher Gott, der du Bund und Gnade
hältst denen, die dich lieben und deine Gebote halten: \bibverse{5} wir
haben gesündigt, unrecht getan, sind gottlos gewesen und abtrünnig
geworden; wir sind von deinen Geboten und Rechten gewichen. \bibverse{6}
Wir gehorchten nicht deinen Knechten, den Propheten, die in deinem Namen
unseren Königen, Fürsten, Vätern und allem Volk im Lande predigten.
\bibverse{7} Du, HErr, bist gerecht, wir aber müssen uns schämen; wie es
denn jetzt geht denen von Juda und denen von Jerusalem und dem ganzen
Israel, denen, die nahe und fern sind in allen Landen, dahin du sie
verstoßen hast um ihrer Missetat willen, die sie an dir begangen haben.
\bibverse{8} Ja, HErr, wir, unsere Könige, unsere Fürsten und unsere
Väter müssen uns schämen, dass wir uns an dir versündigt haben.
\footnote{\textbf{9:8} Jes 43,27} \bibverse{9} Dein aber, HErr, unser
Gott, ist die Barmherzigkeit und Vergebung. Denn wir sind abtrünnig
geworden \footnote{\textbf{9:9} Ps 130,4} \bibverse{10} und gehorchten
nicht der Stimme des HErrn, unseres Gottes, dass wir gewandelt hätten in
seinem Gesetz, welches er uns vorlegte durch seine Knechte, die
Propheten; \bibverse{11} sondern das ganze Israel übertrat dein Gesetz,
und sie wichen ab, dass sie deiner Stimme nicht gehorchten. Darum trifft
uns auch der Fluch und Schwur, der geschrieben steht im Gesetz Moses,
des Knechtes Gottes, weil wir an ihm gesündigt haben. \footnote{\textbf{9:11}
  5Mo 28,15-68; 3Mo 26,14-39} \bibverse{12} Und er hat seine Worte
gehalten, die er geredet hat wider uns und unsere Richter, die uns
richten sollten, dass er so großes Unglück über uns hat gehen lassen,
dass desgleichen unter dem ganzen Himmel nicht geschehen ist, wie über
Jerusalem geschehen ist. \bibverse{13} Gleichwie es geschrieben steht im
Gesetz Moses, so ist all dies große Unglück über uns gegangen. So
beteten wir auch nicht vor dem HErrn, unserem Gott, dass wir uns von den
Sünden bekehrten und auf deine Wahrheit achteten. \bibverse{14} Darum
ist der HErr auch wach gewesen mit diesem Unglück und hat's über uns
gehen lassen. Denn der HErr, unser Gott, ist gerecht in allen seinen
Werken, die er tut; denn wir gehorchten seiner Stimme nicht.
\bibverse{15} Und nun, HErr, unser Gott, der du dein Volk aus
Ägyptenland geführt hast mit starker Hand und hast dir einen Namen
gemacht, wie er jetzt ist: wir haben ja gesündigt und sind leider
gottlos gewesen. \bibverse{16} Ach HErr, um aller deiner Gerechtigkeit
willen wende ab deinen Zorn und Grimm von deiner Stadt Jerusalem und
deinem heiligen Berge. Denn um unserer Sünden willen und um unserer
Väter Missetaten willen trägt Jerusalem und dein Volk Schmach bei allen,
die um uns her sind. \bibverse{17} Und nun, unser Gott, höre das Gebet
deines Knechtes und sein Flehen, und siehe gnädig an dein Heiligtum, das
verstört ist, um des Herrn willen. \bibverse{18} Neige dein Ohr, mein
Gott, und höre, tue deine Augen auf und sieh, wie wir verstört sind und
die ganze Stadt, die nach deinem Namen genannt ist. Denn wir liegen vor
dir mit unserem Gebet, nicht auf unsere Gerechtigkeit, sondern auf deine
große Barmherzigkeit. \footnote{\textbf{9:18} Ps 115,1} \bibverse{19}
Ach Herr, höre, ach Herr, sei gnädig, ach Herr, merke auf und tue es,
und verzieh nicht um deiner selbst willen, mein Gott! denn deine Stadt
und dein Volk ist nach deinem Namen genannt. \footnote{\textbf{9:19} Jer
  14,9} \bibverse{20} Als ich noch so redete und betete und meine und
meines Volks Israel Sünde bekannte und lag mit meinem Gebet vor dem
HErrn, meinem Gott, um den heiligen Berg meines Gottes, \bibverse{21}
eben da ich so redete in meinem Gebet, flog daher der Mann Gabriel, den
ich zuvor gesehen hatte im Gesicht, und rührte mich an um die Zeit des
Abendopfers. \footnote{\textbf{9:21} Dan 8,16} \bibverse{22} Und er
unterrichtete mich und redete mit mir und sprach: Daniel, jetzt bin ich
ausgegangen, dich zu unterrichten. \bibverse{23} Denn da du anfingst zu
beten, ging dieser Befehl aus, und ich komme darum, dass ich dir's
anzeige; denn du bist lieb und wert. So merke nun darauf, dass du das
Gesicht verstehest. \bibverse{24} Siebzig Wochen sind bestimmt über dein
Volk und über deine heilige Stadt, so wird dem Übertreten gewehrt und
die Sünde abgetan und die Missetat versöhnt und die ewige Gerechtigkeit
gebracht und die Gesichte und Weissagung versiegelt und ein Hochheiliges
gesalbt werden. \bibverse{25} So wisse nun und merke: von der Zeit an,
da ausgeht der Befehl, dass Jerusalem soll wiederum gebaut werden, bis
auf den Gesalbten, den Fürsten, sind sieben Wochen; und zweiundsechzig
Wochen, so werden die Gassen und Mauern wieder gebaut werden, wiewohl in
kümmerlicher Zeit. \bibverse{26} Und nach den zweiundsechzig Wochen wird
der Gesalbte ausgerottet werden und nichts mehr sein. Und das Volk eines
Fürsten wird kommen und die Stadt und das Heiligtum verstören, dass es
ein Ende nehmen wird wie durch eine Flut; und bis zum Ende des Streits
wird's wüst bleiben. \bibverse{27} Er wird aber vielen den Bund stärken
eine Woche lang. Und mitten in der Woche wird das Opfer und Speisopfer
aufhören. Und bei den Flügeln werden stehen Gräuel der Verwüstung, bis
das Verderben, welches beschlossen ist, sich über die Verwüstung
ergießen wird. \footnote{\textbf{9:27} Dan 12,11; Mt 24,15}

\hypertarget{section-5}{%
\section{10}\label{section-5}}

\bibverse{1} Im dritten Jahr des Königs Kores aus Persien ward dem
Daniel, der Beltsazar heißt, etwas offenbart, das gewiss ist und von
großen Sachen; und er merkte darauf und verstand das Gesicht wohl.
\footnote{\textbf{10:1} Dan 1,21; Dan 1,7} \bibverse{2} Zur selben Zeit
war ich, Daniel, traurig drei Wochen lang. \bibverse{3} Ich aß keine
leckere Speise, Fleisch und Wein kam nicht in meinen Mund, und salbte
mich auch nie, bis die drei Wochen um waren. \bibverse{4} Und am
vierundzwanzigsten Tage des ersten Monats war ich bei dem großen Wasser
Hiddekkel \bibverse{5} und hob meine Augen auf und sah, und siehe, da
stand ein Mann in Leinwand und hatte einen goldenen Gürtel um seine
Lenden. \footnote{\textbf{10:5} Hes 9,2; Offb 1,13-15} \bibverse{6} Sein
Leib war wie Türkis, sein Antlitz sah wie ein Blitz, seine Augen wie
feurige Fackeln, seine Arme und Füße wie helles, glattes Erz, und seine
Rede war wie ein großes Getön. \bibverse{7} Ich, Daniel, aber sah solch
Gesicht allein, und die Männer, die bei mir waren, sahen's nicht; doch
fiel ein großer Schrecken über sie, dass sie flohen und sich verkrochen.
\bibverse{8} Und ich blieb allein und sah dies große Gesicht. Es blieb
aber keine Kraft in mir, und ich ward sehr entstellt und hatte keine
Kraft mehr. \bibverse{9} Und ich hörte seine Rede; und in dem ich sie
hörte, sank ich ohnmächtig auf mein Angesicht zur Erde. \bibverse{10}
Und siehe, eine Hand rührte mich an und half mir auf die Knie und auf
die Hände, \bibverse{11} und er sprach zu mir: Du, lieber Daniel, merke
auf die Worte, die ich mit dir rede, und richte dich auf; denn ich bin
jetzt zu dir gesandt. Und da er solches mit mir redete, richtete ich
mich auf und zitterte. \bibverse{12} Und er sprach zu mir: Fürchte dich
nicht, Daniel denn von dem ersten Tage an, da du von Herzen begehrtest
zu verstehen und dich kasteitest vor deinem Gott, sind deine Worte
erhört; und ich bin gekommen um deinetwillen. \bibverse{13} Aber der
Fürst des Königreichs im Perserland hat mir einundzwanzig Tage
widerstanden; und siehe, Michael, der vornehmsten Fürsten einer, kam mir
zu Hilfe; da behielt ich den Sieg bei den Königen in Persien.
\footnote{\textbf{10:13} Dan 10,20-21} \bibverse{14} Nun aber komme ich,
dass ich dich unterrichte, wie es deinem Volk hernach gehen wird; denn
das Gesicht wird erst nach etlicher Zeit geschehen. \footnote{\textbf{10:14}
  Dan 9,22} \bibverse{15} Und als er solches mit mir redete, schlug ich
mein Angesicht nieder zur Erde und schwieg still. \bibverse{16} Und
siehe, einer, gleich einem Menschen, rührte meine Lippen an. Da tat ich
meinen Mund auf und redete und sprach zu dem, der vor mir stand: Mein
HErr, meine Gelenke beben mir über dem Gesicht, und ich habe keine Kraft
mehr; \footnote{\textbf{10:16} Jes 6,7; Jer 1,9} \bibverse{17} und wie
kann der Knecht meines Herrn mit meinem Herrn reden, weil nun keine
Kraft mehr in mir ist und ich auch keinen Odem mehr habe? \bibverse{18}
Da rührte einer, gleich wie ein Mensch gestaltet, mich abermals an und
stärkte mich \bibverse{19} und sprach: Fürchte dich nicht, du lieber
Mann! Friede sei mit dir! Und sei getrost, sei getrost! Und als er mit
mir redete, ermannte ich mich und sprach: Mein Herr, rede! denn du hast
mich gestärkt. \bibverse{20} Und er sprach: Weißt du auch, warum ich zu
dir gekommen bin? Jetzt will ich wieder hin und mit dem Fürsten in
Perserland streiten; aber wenn ich wegziehe, siehe, so wird der Fürst
von Griechenland kommen. \footnote{\textbf{10:20} Dan 10,13}
\bibverse{21} Doch ich will dir anzeigen, was geschrieben ist, was
gewiss geschehen wird. Und es ist keiner, der mir hilft wider jene, denn
euer Fürst Michael, \# 11 \bibverse{1} Denn ich stand ihm auch bei im
ersten Jahr des Darius, des Meders, dass ich ihm hülfe und ihn stärkte.
\bibverse{2} Und nun will ich dir anzeigen, was gewiss geschehen soll.
Siehe, es werden noch drei Könige in Persien aufstehen; der vierte aber
wird größeren Reichtum haben denn alle anderen; und wenn er in seinem
Reichtum am mächtigsten ist, wird er alles wider das Königreich in
Griechenland erregen. \bibverse{3} Darnach wird ein mächtiger König
aufstehen und mit großer Macht herrschen, und was er will, wird er
ausrichten. \bibverse{4} Und wenn er aufs Höchste gekommen ist, wird
sein Reich zerbrechen und sich in die vier Winde des Himmels zerteilen,
nicht auf seine Nachkommen, auch nicht mit solcher Macht, wie seine
gewesen ist; denn sein Reich wird ausgerottet und Fremden zuteil werden.
\footnote{\textbf{11:4} Dan 8,8; Dan 8,22} \bibverse{5} Und der König
gegen Mittag, welcher ist seiner Fürsten einer, wird mächtig werden;
aber gegen ihn wird einer auch mächtig sein und herrschen, dessen
Herrschaft wird groß sein. \bibverse{6} Nach etlichen Jahren aber werden
sie sich miteinander befreunden; und die Tochter des Königs gegen Mittag
wird kommen zum König gegen Mitternacht, Einigkeit zu machen. Aber ihr
wird die Macht des Arms nicht bleiben, dazu wird er und sein Arm auch
nicht bestehen bleiben; sondern sie wird übergeben werden samt denen,
die sie gebracht haben, und mit dem, der sie erzeugt hat, und dem, der
sie eine Weile mächtig gemacht hatte. \bibverse{7} Es wird aber der
Zweige einer von ihrem Stamm aufkommen; der wird kommen mit Heereskraft
und dem König gegen Mitternacht in seine Feste fallen und wird's
ausrichten und siegen. \bibverse{8} Auch wird er ihre Götter und Bilder
samt den köstlichen Kleinoden, silbernen und goldenen, wegführen nach
Ägypten und etliche Jahre vor dem König gegen Mitternacht wohl stehen
bleiben. \bibverse{9} Und dieser wird ziehen in das Reich des Königs
gegen Mittag, aber wieder in sein Land umkehren. \bibverse{10} Aber
seine Söhne werden zornig werden und große Heere zusammenbringen; und
der eine wird kommen und wie eine Flut daherfahren und wiederum Krieg
führen bis vor seine Feste. \bibverse{11} Da wird der König gegen Mittag
ergrimmen und ausziehen und mit dem König gegen Mitternacht streiten und
wird solchen großen Haufen zusammenbringen, dass ihm jener Haufe wird in
seine Hand gegeben, \bibverse{12} und wird den Haufen wegführen. Des
wird sich sein Herz überheben, dass er so viele Tausende darniedergelegt
hat; aber damit wird er sein nicht mächtig werden. \bibverse{13} Denn
der König gegen Mitternacht wird wiederum einen größeren Haufen
zusammenbringen, als der vorige war; und nach etlichen Jahren wird er
daherziehen mit großer Heereskraft und mit großem Gut. \bibverse{14} Und
zur selben Zeit werden sich viele wider den König gegen Mittag setzen;
auch werden sich Abtrünnige aus deinem Volk erheben und die Weissagung
erfüllen, und werden fallen. \bibverse{15} Also wird der König gegen
Mitternacht daherziehen und einen Wall aufschütten und eine feste Stadt
gewinnen; und die Mittagsheere werden's nicht können wehren, und sein
bestes Volk wird nicht können widerstehen; \bibverse{16} sondern der an
ihn kommt, wird seinen Willen schaffen, und niemand wird ihm widerstehen
können. Er wird auch in das werte Land kommen und wird's vollenden durch
seine Hand. \bibverse{17} Und wird sein Angesicht richten, dass er mit
der Macht seines ganzen Königreichs komme. Aber er wird sich mit ihm
vertragen und wird ihm seine Tochter zum Weibe geben, dass er ihn
verderbe; aber es wird ihm nicht geraten und wird nichts daraus werden.
\bibverse{18} Darnach wird er sich kehren wider die Inseln und deren
viele gewinnen. Aber ein Fürst wird ihn lehren aufhören mit Schmähen,
dass er ihn nicht mehr schmähe. \bibverse{19} Also wird er sich wiederum
kehren zu den Festen seines Landes und wird sich stoßen und fallen, dass
man ihn nirgend finden wird. \bibverse{20} Und an seiner Statt wird
einer aufkommen, der wird einen Schergen sein herrliches Reich
durchziehen lassen; aber nach wenig Tagen wird er zerbrochen werden,
doch weder durch Zorn noch durch Streit. \bibverse{21} An des Statt wird
aufkommen ein Ungeachteter, welchem die Ehre des Königreichs nicht
zugedacht war; der wird mitten im Frieden kommen und das Königreich mit
süßen Worten einnehmen. \footnote{\textbf{11:21} Dan 8,23} \bibverse{22}
Und die Heere, die wie eine Flut daherfahren, werden von ihm wie mit
einer Flut überfallen und zerbrochen werden, dazu auch der Fürst, mit
dem der Bund gemacht war. \bibverse{23} Denn nachdem er mit ihm
befreundet ist, wird er listig gegen ihn handeln und wird heraufziehen
und mit geringem Volk ihn überwältigen, \bibverse{24} und es wird ihm
gelingen, dass er in die besten Städte des Landes kommen wird; und
wird's also ausrichten, wie es weder seine Väter noch seine Voreltern
tun konnten, mit Rauben, Plündern und Ausbeuten; und wird nach den
allerfestesten Städten trachten, und das eine Zeitlang. \bibverse{25}
Und er wird seine Macht und sein Herz wider den König gegen Mittag
erregen mit großer Heereskraft; Da wird der König gegen Mittag gereizt
werden zum Streit mit einer großen, mächtigen Heereskraft; aber er wird
nicht bestehen, denn es werden Verrätereien wider ihn gemacht.
\bibverse{26} Und eben die sein Brot essen, die werden ihn helfen
verderben und sein Heer unterdrücken, dass gar viele erschlagen werden.
\bibverse{27} Und beider Könige Herz wird denken, wie sie einander
Schaden tun, und werden an einem Tische fälschlich miteinander reden. Es
wird ihnen aber nicht gelingen; denn das Ende ist noch auf eine andere
Zeit bestimmt. \bibverse{28} Darnach wird er wiederum heimziehen mit
großem Gut und sein Herz richten wider den heiligen Bund; da wird er es
ausrichten und also heim in sein Land ziehen. \bibverse{29} Darnach wird
er zu gelegener Zeit wieder gegen Mittag ziehen; aber es wird ihm zum
andernmal nicht geraten wie zum erstenmal. \bibverse{30} Denn es werden
Schiffe aus Chittim wider ihn kommen, dass er verzagen wird und umkehren
muss. Da wird er wider den heiligen Bund ergrimmen und wird's
ausrichten; und wird sich umsehen und an sich ziehen, die den heiligen
Bund verlassen. \bibverse{31} Und es werden seine Heere daselbst stehen;
die werden das Heiligtum in der Feste entweihen und das tägliche Opfer
abtun und einen Gräuel der Verwüstung aufrichten. \footnote{\textbf{11:31}
  Dan 9,27; Dan 12,11; 1Makk 1,57; Mt 24,15} \bibverse{32} Und er wird
heucheln und gute Worte geben den Gottlosen, die den Bund übertreten.
Aber die vom Volk, die ihren Gott kennen, werden sich ermannen und es
ausrichten. \footnote{\textbf{11:32} 1Makk 2,1-6} \bibverse{33} Und die
Verständigen im Volk werden viele andere lehren; darüber werden sie
fallen durch Schwert, Feuer, Gefängnis und Raub eine Zeitlang.
\footnote{\textbf{11:33} Dan 12,3} \bibverse{34} Und wenn sie so fallen,
wird ihnen eine kleine Hilfe geschehen; aber viele werden sich zu ihnen
tun betrüglich. \bibverse{35} Und der Verständigen werden etliche
fallen, auf dass sie bewährt, rein und lauter werden, bis dass es ein
Ende habe; denn es ist noch eine andere Zeit vorhanden.

\bibverse{36} Und der König wird tun, was er will, und wird sich erheben
und aufwerfen wider alles, was Gott ist; und wider den Gott aller Götter
wird er gräulich reden; und es wird ihm gelingen, bis der Zorn aus sei;
denn es muss geschehen, was beschlossen ist.

\bibverse{37} Und die Götter seiner Väter wird er nicht achten; er wird
weder Frauenliebe noch irgendeines Gottes achten; denn er wird sich
wider alles aufwerfen. \footnote{\textbf{11:37} 1Tim 4,3}

\bibverse{38} Aber anstatt dessen wird er den Gott der Festungen ehren;
denn er wird einen Gott, davon seine Väter nichts gewusst haben, ehren
mit Gold, Silber, Edelsteinen und Kleinoden

\bibverse{39} und wird denen, die ihm helfen die Festungen stärken mit
dem fremden Gott, den er erwählt hat, große Ehre tun und sie zu Herren
machen über große Güter und ihnen das Land zum Lohn austeilen.

\bibverse{40} Und am Ende wird sich der König gegen Mittag mit ihm
messen; und der König gegen Mitternacht wird gegen ihn stürmen mit
Wagen, Reitern und vielen Schiffen und wird in die Länder fallen und
verderben und durchziehen

\bibverse{41} und wird in das werte Land fallen, und viele werden
umkommen. Diese aber werden seiner Hand entrinnen: Edom, Moab und die
Vornehmsten der Kinder Ammon.

\bibverse{42} Und er wird seine Hand ausstrecken nach den Ländern, und
Ägypten wird ihm nicht entrinnen;

\bibverse{43} sondern er wird herrschen über die goldenen und silbernen
Schätze und über alle Kleinode Ägyptens; Libyer und Mohren werden in
seinem Zuge sein.

\bibverse{44} Es wird ihn aber ein Geschrei erschrecken von Morgen und
Mitternacht; und er wird mit großem Grimm ausziehen, willens, viele zu
vertilgen und zu verderben.

\bibverse{45} Und er wird den Palast seines Gezeltes aufschlagen
zwischen zwei Meeren um den werten heiligen Berg, bis es mit ihm ein
Ende werde; und niemand wird ihm helfen. \# 12 \bibverse{1} Zur selben
Zeit wird der große Fürst Michael, der für die Kinder deines Volks
steht, sich aufmachen. Denn es wird eine solche trübselige Zeit sein,
wie sie nicht gewesen ist, seitdem Leute gewesen sind bis auf diese
Zeit. Zur selben Zeit wird dein Volk errettet werden, alle, die im Buch
geschrieben stehen. \^{}\^{} \bibverse{2} Und viele, die unter der Erde
schlafen liegen, werden aufwachen: etliche zum ewigen Leben, etliche zu
ewiger Schmach und Schande. \^{}\^{} \bibverse{3} Die Lehrer aber werden
leuchten wie des Himmels Glanz, und die, die viele zur Gerechtigkeit
weisen, wie die Sterne immer und ewiglich. \^{}\^{} \bibverse{4} Und du,
Daniel, verbirg diese Worte und versiegle diese Schrift bis auf die
letzte Zeit; so werden viele darüberkommen und großen Verstand finden.
\^{}\^{} \bibverse{5} Und ich, Daniel, sah, und siehe, es standen zwei
andere da, einer an diesem Ufer des Wassers, der andere an jenem Ufer.
\bibverse{6} Und er sprach zu dem in leinenen Kleidern, der über den
Wassern des Flusses stand: Wann will's denn ein Ende sein mit solchen
Wundern? \^{}\^{} \bibverse{7} Und ich hörte zu dem in leinenen
Kleidern, der über den Wassern des Flusses stand; und er hob seine
rechte und linke Hand auf gen Himmel und schwur bei dem, der ewiglich
lebt, dass es eine Zeit und (zwei) Zeiten und eine halbe Zeit währen
soll; und wenn die Zerstreuung des heiligen Volks ein Ende hat, soll
solches alles geschehen. \^{}\^{} \bibverse{8} Und ich hörte es; aber
ich verstand's nicht und sprach: Mein Herr, was wird darnach werden?
\bibverse{9} Er aber sprach: Gehe hin, Daniel denn es ist verborgen und
versiegelt bis auf die letzte Zeit. \bibverse{10} Viele werden
gereinigt, geläutert und bewährt werden; und die Gottlosen werden
gottlos Wesen führen, und die Gottlosen alle werden's nicht achten; aber
die Verständigen werden's achten. \bibverse{11} Und von der Zeit an,
wenn das tägliche Opfer abgetan und ein Gräuel der Verwüstung
aufgerichtet wird, sind 1290 Tage. \bibverse{12} Wohl dem, der da wartet
und erreicht 1335 Tage! \bibverse{13} Du aber, Daniel, gehe hin, bis das
Ende komme; und ruhe, dass du aufstehest zu deinem Erbteil am Ende der
Tage!
