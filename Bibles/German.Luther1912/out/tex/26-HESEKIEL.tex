\hypertarget{section}{%
\section{1}\label{section}}

\bibverse{1} Im dreißigsten Jahr, am fünften Tage des vierten Monats, da
ich war unter den Gefangenen am Wasser Chebar, tat sich der Himmel auf,
und Gott zeigte mir Gesichte. \footnote{\textbf{1:1} Hes 10,15}

\bibverse{2} Derselbe fünfte Tag des Monats war eben im fünften Jahr,
nachdem Jojachin, der König Judas, war gefangen weggeführt. \footnote{\textbf{1:2}
  2Kö 24,15} \bibverse{3} Da geschah des HErrn Wort zu Hesekiel, dem
Sohn Busis, dem Priester, im Lande der Chaldäer, am Wasser Chebar;
daselbst kam die Hand des HErrn über ihn.

\bibverse{4} Und ich sah, und siehe, es kam ein ungestümer Wind von
Mitternacht her mit einer großen Wolke voll Feuer, das allenthalben
umher glänzte; und mitten in dem Feuer war es lichthell. \footnote{\textbf{1:4}
  Hes 10,-1; Offb 4,6-8} \bibverse{5} Und darin war es gestaltet wie
vier Tiere, und dieselben waren anzusehen wie Menschen. \bibverse{6} Und
ein jegliches hatte vier Angesichter und vier Flügel. \bibverse{7} Und
ihre Beine standen gerade, und ihre Füße waren gleich wie Rinderfüße und
glänzten wie helles glattes Erz. \bibverse{8} Und sie hatten
Menschenhände unter ihren Flügeln an ihren vier Seiten; denn sie hatten
alle vier ihre Angesichter und ihre Flügel. \bibverse{9} Und je einer
der Flügel rührte an den anderen; und wenn sie gingen, mussten sie nicht
herumlenken, sondern wo sie hin gingen, gingen sie stracks vor sich.

\bibverse{10} Ihre Angesichter waren vorn gleich einem Menschen, und zur
rechten Seite gleich einem Löwen bei allen vieren, und zur linken Seite
gleich einem Ochsen bei allen vieren, und hinten gleich einem Adler bei
allen vieren. \bibverse{11} Und ihre Angesichter und Flügel waren
obenher zerteilt, dass je zwei Flügel zusammenschlugen, und mit zwei
Flügeln bedeckten sie ihren Leib. \bibverse{12} Wo sie hin gingen, da
gingen sie stracks vor sich -- sie gingen aber, wo der Geist sie hin
trieb -- und mussten sich nicht herumlenken, wenn sie gingen.
\bibverse{13} Und die Tiere waren anzusehen wie feurige Kohlen, die da
brennen, und wie Fackeln; und das Feuer fuhr hin zwischen den Tieren und
gab einen Glanz von sich, und aus dem Feuer gingen Blitze. \bibverse{14}
Die Tiere aber liefen hin und her wie der Blitz.

\bibverse{15} Als ich die Tiere so sah, siehe, da stand ein Rad auf der
Erde bei den vier Tieren und war anzusehen wie vier Räder. \bibverse{16}
Und die Räder waren wie ein Türkis und waren alle vier eins wie das
andere, und sie waren anzusehen, als wäre ein Rad im anderen.
\bibverse{17} Wenn sie gehen wollten, konnten sie nach allen ihren vier
Seiten gehen und mussten sich nicht herumlenken, wenn sie gingen.
\bibverse{18} Ihre Felgen und Höhe waren schrecklich; und ihre Felgen
waren voller Augen um und um an allen vier Rädern.

\bibverse{19} Auch wenn die vier Tiere gingen, so gingen die Räder auch
neben ihnen; und wenn die Tiere sich von der Erde emporhoben, so hoben
sich die Räder auch empor. \bibverse{20} Wo der Geist sie hin trieb, da
gingen sie hin, und die Räder hoben sich neben ihnen empor; denn es war
der Geist der Tiere in den Rädern. \footnote{\textbf{1:20} Hes 1,12}
\bibverse{21} Wenn sie gingen, so gingen diese auch; wenn sie standen,
so standen diese auch; und wenn sie sich emporhoben von der Erde, so
hoben sich auch die Räder neben ihnen empor; denn es war der Geist der
Tiere in den Rädern.

\bibverse{22} Oben aber über den Tieren war es gestaltet wie ein Himmel,
wie ein Kristall, schrecklich, gerade oben über ihnen ausgebreitet,
\bibverse{23} dass unter dem Himmel ihre Flügel einer stracks gegen den
anderen standen, und eines jeglichen Leib bedeckten zwei Flügel.
\bibverse{24} Und ich hörte die Flügel rauschen wie große Wasser und wie
ein Getön des Allmächtigen, wenn sie gingen, und wie ein Getümmel in
einem Heer. Wenn sie aber stillstanden, so ließen sie die Flügel nieder.

\bibverse{25} Und wenn sie stillstanden und die Flügel niederließen, so
donnerte es in dem Himmel oben über ihnen. \bibverse{26} Und über dem
Himmel, so oben über ihnen war, war es gestaltet wie ein Saphir,
gleichwie ein Stuhl; und auf dem Stuhl saß einer, gleichwie ein Mensch
gestaltet. \footnote{\textbf{1:26} Hes 1,22} \bibverse{27} Und ich sah,
und es war lichthell, und inwendig war es gestaltet wie ein Feuer um und
um. Von seinen Lenden überwärts und unterwärts sah ich's wie Feuer
glänzen um und um. \bibverse{28} Gleichwie der Regenbogen sieht in den
Wolken, wenn es geregnet hat, also glänzte es um und um. Dies war das
Ansehen der Herrlichkeit des HErrn. Und da ich's gesehen hatte, fiel ich
auf mein Angesicht und hörte einen reden. \# 2 \bibverse{1} Und er
sprach zu mir: Du Menschenkind, tritt auf deine Füße, so will ich mit
dir reden. \bibverse{2} Und da er so mit mir redete, ward ich wieder
erquickt und trat auf meine Füße und hörte dem zu, der mit mir redete.

\bibverse{3} Und er sprach zu mir: Du Menschenkind, ich sende dich zu
den Kindern Israel, zu dem abtrünnigen Volk, die von mir abtrünnig
geworden sind. Sie samt ihren Vätern haben bis auf diesen heutigen Tag
wider mich getan. \bibverse{4} Aber die Kinder, zu welchen ich dich
sende, haben harte Köpfe und verstockte Herzen. Zu denen sollst du
sagen: So spricht der Herr HErr! \bibverse{5} Sie gehorchen oder
lassen's. Es ist wohl ein ungehorsames Haus; dennoch sollen sie wissen,
dass ein Prophet unter ihnen ist. \footnote{\textbf{2:5} Hes 3,11; Hes
  3,27} \bibverse{6} Und du Menschenkind, sollst dich vor ihnen nicht
fürchten noch vor ihren Worten fürchten. Es sind wohl widerspenstige und
stachlige Dornen bei dir, und du wohnst unter Skorpionen; aber du sollst
dich nicht fürchten vor ihren Worten noch vor ihrem Angesicht dich
entsetzen, ob sie wohl ein ungehorsames Haus sind, \bibverse{7} sondern
du sollst ihnen meine Worte sagen, sie gehorchen oder lassen's; denn es
ist ein ungehorsames Volk. \bibverse{8} Aber du, Menschenkind, höre du,
was ich dir sage, und sei nicht ungehorsam, wie das ungehorsame Haus
ist. Tue deinen Mund auf und iss, was ich dir geben werde.

\bibverse{9} Und ich sah, und siehe, da war eine Hand gegen mich
ausgereckt, die hatte einen zusammengelegten Brief; \bibverse{10} den
breitete sie aus vor mir, und er war beschrieben auswendig und inwendig,
und stand darin geschrieben Klage, Ach und Wehe. \# 3 \bibverse{1} Und
er sprach zu mir: Du Menschenkind, iss, was vor dir ist, iss diesen
Brief, und gehe hin und predige dem Hause Israel! \footnote{\textbf{3:1}
  Hes 2,9}

\bibverse{2} Da tat ich meinen Mund auf, und er gab mir den Brief zu
essen

\bibverse{3} und sprach zu mir: Du Menschenkind, du musst diesen Brief,
den ich dir gebe, in deinen Leib essen und deinen Bauch damit füllen. Da
aß ich ihn, und er war in meinem Munde so süß wie Honig.

\bibverse{4} Und er sprach zu mir: Du Menschenkind, gehe hin zum Hause
Israel und predige ihnen meine Worte.

\bibverse{5} Denn ich sende dich ja nicht zu einem Volk, das eine fremde
Rede und unbekannte Sprache hat, sondern zum Hause Israel; \bibverse{6}
ja, freilich nicht zu großen Völkern, die fremde Rede und unbekannte
Sprache haben, welcher Worte du nicht verstehen könntest. Und wenn ich
dich gleich zu denselben sendete, würden sie dich doch gern hören.
\bibverse{7} Aber das Haus Israel will dich nicht hören, denn sie wollen
mich selbst nicht hören; denn das ganze Haus Israel hat harte Stirnen
und verstockte Herzen. \bibverse{8} Siehe, ich habe dein Angesicht hart
gemacht gegen ihr Angesicht und deine Stirn gegen ihre Stirn.
\bibverse{9} Ja, ich habe deine Stirn so hart wie einen Demant, der
härter ist denn ein Fels, gemacht. Darum fürchte dich nicht, entsetze
dich auch nicht vor ihnen, dass sie so ein ungehorsames Haus sind.

\bibverse{10} Und er sprach zu mir: Du Menschenkind, alle meine Worte,
die ich dir sage, die fasse zu Herzen und nimm sie zu Ohren!
\bibverse{11} Und gehe hin zu den Gefangenen deines Volks und predige
ihnen und sprich zu ihnen: So spricht der Herr HErr! sie hören's oder
lassen's. \footnote{\textbf{3:11} Hes 8,3; Apg 8,39}

\bibverse{12} Und ein Wind hob mich auf, und ich hörte hinter mir ein
Getön wie eines großen Erdbebens: Gelobt sei die Herrlichkeit des HErrn
an ihrem Ort! \bibverse{13} Und war ein Rauschen von den Flügeln der
Tiere, die aneinander schlugen, und auch das Rasseln der Räder, die hart
bei ihnen waren, und das Getön eines großen Erdbebens. \bibverse{14} Da
hob mich der Wind auf und führte mich weg. Und ich fuhr dahin in
bitterem Grimm, und des HErrn Hand hielt mich fest. \bibverse{15} Und
ich kam zu den Gefangenen, die am Wasser Chebar wohnten, gen Thel-Abib,
und setzte mich zu ihnen, die da saßen, und blieb daselbst unter ihnen
sieben Tage ganz traurig.

\bibverse{16} Und da die sieben Tage um waren, geschah des HErrn Wort zu
mir und sprach: \bibverse{17} Du Menschenkind, ich habe dich zum Wächter
gesetzt über das Haus Israel; du sollst aus meinem Munde das Wort hören
und sie von meinetwegen warnen. \bibverse{18} Wenn ich dem Gottlosen
sage: Du musst des Todes sterben, und du warnst ihn nicht und sagst es
ihm nicht, damit sich der Gottlose vor seinem gottlosen Wesen hüte, auf
dass er lebendig bleibe: so wird der Gottlose um seiner Sünde willen
sterben; aber sein Blut will ich von deiner Hand fordern. \bibverse{19}
Wo du aber den Gottlosen warnst und er sich nicht bekehrt von seinem
gottlosen Wesen und Wege, so wird er um seiner Sünde willen sterben;
aber du hast deine Seele errettet.

\bibverse{20} Und wenn sich ein Gerechter von seiner Gerechtigkeit
wendet und tut Böses, so werde ich ihn lassen anlaufen, dass er muss
sterben. Denn weil du ihn nicht gewarnt hast, wird er um seiner Sünde
willen sterben müssen, und seine Gerechtigkeit, die er getan hat, wird
nicht angesehen werden; aber sein Blut will ich von deiner Hand fordern.
\footnote{\textbf{3:20} Hes 18,24} \bibverse{21} Wo du aber den
Gerechten warnst, dass er nicht sündigen soll, und er sündigt auch
nicht, so soll er leben, denn er hat sich warnen lassen; und du hast
deine Seele errettet.

\bibverse{22} Und daselbst kam des HErrn Hand über mich, und er sprach
zu mir: Mache dich auf und gehe hinaus ins Feld; da will ich mit dir
reden.

\bibverse{23} Und ich machte mich auf und ging hinaus ins Feld; und
siehe, da stand die Herrlichkeit des HErrn daselbst, gleichwie ich sie
am Wasser Chebar gesehen hatte; und ich fiel nieder auf mein Angesicht.

\bibverse{24} Und ich ward erquickt und trat auf meine Füße. Und er
redete mit mir und sprach zu mir: Gehe hin und verschließ dich in deinem
Hause! \footnote{\textbf{3:24} Hes 2,2} \bibverse{25} Und du,
Menschenkind, siehe, man wird dir Stricke anlegen und dich damit binden,
dass du nicht ausgehen sollst unter sie. \bibverse{26} Und ich will dir
die Zunge an deinem Gaumen kleben lassen, dass du verstummen sollst und
nicht mehr sie strafen könnest; denn es ist ein ungehorsames Haus.
\bibverse{27} Wenn ich aber mit dir reden werde, will ich dir den Mund
auftun, dass du zu ihnen sagen sollst: So spricht der Herr HErr! Wer's
hört, der höre es; wer's lässt, der lasse es; denn es ist ein
ungehorsames Haus. \footnote{\textbf{3:27} Hes 3,11}

\hypertarget{section-1}{%
\section{4}\label{section-1}}

\bibverse{1} Und du, Menschenkind, nimm einen Ziegel; den lege vor dich
und entwirf darauf die Stadt Jerusalem \bibverse{2} und mache eine
Belagerung darum und baue ein Bollwerk darum und schütte einen Wall
darum und mache ein Heerlager darum und stelle Sturmböcke rings um sie
her. \bibverse{3} Vor dich aber nimm eine eiserne Pfanne; die lass eine
eiserne Mauer sein zwischen dir und der Stadt, und richte dein Angesicht
gegen sie und belagere sie. Das sei ein Zeichen dem Hause Israel.

\bibverse{4} Du sollst dich auch auf deine linke Seite legen und die
Missetat des Hauses Israel auf dieselbe legen; soviel Tage du darauf
liegst, so lange sollst du auch ihre Missetat tragen. \bibverse{5} Ich
will dir aber die Jahre ihrer Missetat zur Anzahl der Tage machen,
nämlich 390 Tage; so lange sollst du die Missetat des Hauses Israel
tragen.

\bibverse{6} Und wenn du solches ausgerichtet hast, sollst du darnach
dich auf deine rechte Seite legen und sollst tragen die Missetat des
Hauses Juda vierzig Tage lang; denn ich gebe dir hier auch je einen Tag
für ein Jahr. \bibverse{7} Und richte dein Angesicht und deinen bloßen
Arm wider das belagerte Jerusalem und weissage wider dasselbe.
\bibverse{8} Und siehe, ich will dir Stricke anlegen, dass du dich nicht
wenden könnest von einer Seite zur anderen, bis du die Tage deiner
Belagerung vollendet hast.

\bibverse{9} So nimm nun zu dir Weizen, Gerste, Bohnen, Linsen, Hirse
und Spelt und tue alles in ein Fass und mache dir Brot daraus, soviel
Tage du auf deiner Seite liegst, dass du 390 Tage daran zu essen hast,
\bibverse{10} also dass deine Speise, die du täglich essen sollst, sei
zwanzig Lot nach dem Gewicht. Solches sollst du von einer Zeit zur
anderen essen. \bibverse{11} Das Wasser sollst du auch nach dem Maß
trinken, nämlich das sechste Teil vom Hin, und sollst solches auch von
einer Zeit zur anderen trinken. \bibverse{12} Gerstenkuchen sollst du
essen, die du vor ihren Augen auf Menschenmist backen sollst.
\bibverse{13} Und der HErr sprach: Also müssen die Kinder Israel ihr
unreines Brot essen unter den Heiden, dahin ich sie verstoßen werde.

\bibverse{14} Ich aber sprach: Ach Herr HErr! siehe, meine Seele ist
noch nie unrein geworden; denn ich habe von meiner Jugend auf bis auf
diese Zeit kein Aas noch Zerrissenes gegessen, und ist nie unreines
Fleisch in meinen Mund gekommen. \footnote{\textbf{4:14} Apg 10,14}

\bibverse{15} Er aber sprach zu mir: Siehe, ich will dir Kuhmist für
Menschenmist zulassen, darauf du dein Brot machen sollst.

\bibverse{16} Und sprach zu mir: Du Menschenkind, siehe, ich will den
Vorrat des Brots zu Jerusalem wegnehmen, dass sie das Brot essen müssen
nach dem Gewicht und mit Kummer, und das Wasser nach dem Maß mit Kummer
trinken, \bibverse{17} darum dass es an Brot und Wasser mangeln und
einer mit dem anderen trauern wird und sie in ihrer Missetat
verschmachten sollen. \# 5 \bibverse{1} Und du, Menschenkind, nimm ein
Schwert, scharf wie ein Schermesser, und fahr damit über dein Haupt und
deinen Bart und nimm eine Waage und teile das Haar damit. \bibverse{2}
Das eine dritte Teil sollst du mit Feuer verbrennen mitten in der Stadt,
wenn die Tage der Belagerung um sind; das andere dritte Teil nimm und
schlag's mit dem Schwert ringsumher; das letzte dritte Teil streue in
den Wind, dass ich das Schwert hinter ihnen her ausziehe. \bibverse{3}
Nimm aber ein klein wenig davon und binde es in deinen Mantelzipfel.
\bibverse{4} Und nimm wiederum etliches davon und wirf's in ein Feuer
und verbrenne es mit Feuer; von dem soll ein Feuer auskommen über das
ganze Haus Israel.

\bibverse{5} So spricht der Herr HErr: Das ist Jerusalem, das ich mitten
unter die Heiden gesetzt habe und ringsumher Länder. \bibverse{6} Aber
es hat mein Gesetz verwandelt in gottlose Lehre mehr denn die Heiden,
und meine Rechte mehr denn die Länder, die ringsumher liegen. Denn sie
verwerfen mein Gesetz und wollen nicht nach meinen Rechten leben.

\bibverse{7} Darum spricht der Herr HErr also: Weil ihr's mehr macht
denn die Heiden, die um euch her sind, und nach meinen Geboten nicht
lebt und nach meinen Rechten nicht tut, sondern nach der Heiden Weise
tut, die um euch her sind, \bibverse{8} so spricht der Herr HErr also:
Siehe, ich will auch an dich und will Recht über dich gehen lassen, dass
die Heiden zusehen sollen; \bibverse{9} und will also mit dir umgehen,
wie ich nie getan und hinfort nicht tun werde, um aller deiner Gräuel
willen: \bibverse{10} dass in dir die Väter ihre Kinder und die Kinder
ihre Väter fressen sollen; und will solch Recht über dich gehen lassen,
dass alle deine Übrigen sollen in alle Winde zerstreut werden.
\footnote{\textbf{5:10} 5Mo 28,53-55; Kla 4,10} \bibverse{11} Darum, so
wahr als ich lebe, spricht der Herr HErr, weil du mein Heiligtum mit
allen deinen Gräueln und Götzen verunreinigt hast, will ich dich auch
zerschlagen, und mein Auge soll dein nicht schonen, und ich will nicht
gnädig sein. \footnote{\textbf{5:11} Hes 8,6-18} \bibverse{12} Es soll
ein drittes Teil an der Pestilenz sterben und durch Hunger alle werden
in dir, und das andere dritte Teil durchs Schwert fallen rings um dich
her; und das letzte dritte Teil will ich in alle Winde zerstreuen und
das Schwert hinter ihnen her ausziehen. \footnote{\textbf{5:12} Hes 5,2}

\bibverse{13} Also soll mein Zorn vollendet und mein Grimm an ihnen
ausgerichtet werden, dass ich meinen Mut kühle; und sie sollen erfahren,
dass ich, der HErr, in meinem Eifer geredet habe, wenn ich meinen Grimm
an ihnen ausgerichtet habe. \footnote{\textbf{5:13} Hes 16,42}

\bibverse{14} Ich will dich zur Wüste und zur Schmach setzen vor den
Heiden, die um dich her sind, vor den Augen aller, die vorübergehen.
\bibverse{15} Und sollst eine Schmach, Hohn, Beispiel und Wunder sein
allen Heiden, die um dich her sind, wenn ich über dich das Recht gehen
lasse mit Zorn, Grimm und zornigem Schelten (das sage ich, der HErr)
\footnote{\textbf{5:15} Jer 24,9} \bibverse{16} und wenn ich böse Pfeile
des Hungers unter sie schießen werde, die da schädlich sein sollen, und
ich sie ausschießen werde, euch zu verderben, und den Hunger über euch
immer größer werden lasse und den Vorrat des Brots wegnehme. \footnote{\textbf{5:16}
  5Mo 32,23; Hes 4,16} \bibverse{17} Ja, Hunger und böse, wilde Tiere
will ich unter euch schicken, die sollen euch kinderlos machen; und soll
Pestilenz und Blut unter dir umgehen, und ich will das Schwert über dich
bringen. Ich, der HErr, habe es gesagt. \footnote{\textbf{5:17} Hes
  14,21}

\hypertarget{section-2}{%
\section{6}\label{section-2}}

\bibverse{1} Und des HErrn Wort geschah zu mir und sprach: \bibverse{2}
Du Menschenkind, kehre dein Angesicht wider die Berge Israels und
weissage wider sie \bibverse{3} und sprich: Ihr Berge Israels, höret das
Wort des Herrn HErrn! So spricht der Herr HErr zu den Bergen und Hügeln,
zu den Bächen und Tälern: Siehe, ich will das Schwert über euch bringen
und eure Höhen zerstören, \bibverse{4} dass eure Altäre verwüstet und
eure Sonnensäulen zerbrochen werden, und will eure Erschlagenen vor eure
Bilder werfen; \footnote{\textbf{6:4} 3Mo 26,30} \bibverse{5} ja, ich
will die Leichname der Kinder Israel vor ihre Bilder hinwerfen und will
eure Gebeine um eure Altäre her zerstreuen. \bibverse{6} Wo ihr wohnet,
da sollen die Städte wüst und die Höhen zur Einöde werden; denn man wird
eure Altäre wüst und zur Einöde machen und eure Götzen zerbrechen und
zunichte machen und eure Sonnensäulen zerschlagen und eure Machwerke
vertilgen. \bibverse{7} Und sollen Erschlagene unter euch daliegen, dass
ihr erfahret, ich sei der HErr.

\bibverse{8} Ich will aber etliche von euch übrigbleiben lassen, die dem
Schwert entgehen unter den Heiden, wenn ich euch in die Länder zerstreut
habe. \bibverse{9} Diese eure Entronnenen werden dann an mich gedenken
unter den Heiden, da sie gefangen sein müssen, wenn ich ihr abgöttisches
Herz, das von mir gewichen, und ihre abgöttischen Augen, die nach ihren
Götzen gesehen, zerschlagen habe; und es wird sie gereuen die Bosheit,
die sie durch alle ihre Gräuel begangen haben; \bibverse{10} und sie
sollen erfahren, dass ich der HErr sei und nicht umsonst geredet habe,
solches Unglück ihnen zu tun.

\bibverse{11} So spricht der Herr HErr: Schlage deine Hände zusammen und
stampfe mit deinem Fuß und sprich: Weh über alle Gräuel der Bosheit im
Hause Israel, darum sie durch Schwert, Hunger und Pestilenz fallen
müssen! \bibverse{12} Wer fern ist, wird an der Pestilenz sterben, und
wer nahe ist, wird durchs Schwert fallen; wer aber übrigbleibt und davor
behütet ist, wird Hungers sterben. Also will ich meinen Grimm unter
ihnen vollenden, \bibverse{13} dass ihr erfahren sollt, ich sei der
HErr, wenn ihre Erschlagenen unter ihren Götzen liegen werden um ihre
Altäre her, oben auf allen Hügeln und oben auf allen Bergen und unter
allen grünen Bäumen und unter allen dichten Eichen, an welchen Orten sie
allerlei Götzen süßes Räuchopfer taten. \footnote{\textbf{6:13} 1Kö
  14,23} \bibverse{14} Ich will meine Hand wider sie ausstrecken und das
Land wüst und öde machen von der Wüste an bis gen Dibla, überall, wo sie
wohnen; und sie sollen erfahren, dass ich der HErr sei. \footnote{\textbf{6:14}
  Hes 6,7}

\hypertarget{section-3}{%
\section{7}\label{section-3}}

\bibverse{1} Und des HErrn Wort geschah zu mir und sprach: \bibverse{2}
Du Menschenkind, so spricht der Herr HErr vom Lande Israel: Das Ende
kommt, das Ende über alle vier Örter des Landes. \bibverse{3} Nun kommt
das Ende über dich; denn ich will meinen Grimm über dich senden und will
dich richten, wie du verdient hast, und will dir geben, was allen deinen
Gräueln gebührt. \bibverse{4} Mein Auge soll dein nicht schonen noch
übersehen; sondern ich will dir geben, wie du verdient hast, und deine
Gräuel sollen unter dich kommen, dass ihr erfahren sollt, ich sei der
HErr.

\bibverse{5} So spricht der Herr HErr: Siehe, es kommt ein Unglück über
das andere! \bibverse{6} Das Ende kommt, es kommt das Ende, es ist
erwacht über dich; siehe, es kommt! \bibverse{7} Es geht schon auf und
bricht daher über dich, du Einwohner des Landes; die Zeit kommt, der Tag
des Jammers ist nahe, da kein Singen auf den Bergen sein wird.
\footnote{\textbf{7:7} Joe 1,15} \bibverse{8} Nun will ich bald meinen
Grimm über dich schütten und meinen Zorn an dir vollenden und will dich
richten, wie du verdient hast, und dir geben, was deinen Gräueln allen
gebührt. \bibverse{9} Mein Auge soll dein nicht schonen, und ich will
nicht gnädig sein; sondern will dir geben, wie du verdient hast, und
deine Gräuel sollen unter dich kommen, dass ihr erfahren sollt, ich sei
der HErr, der euch schlägt.

\bibverse{10} Siehe, der Tag, siehe, er kommt daher, er bricht an; die
Rute blüht, und der Stolze grünt. \bibverse{11} Der Tyrann hat sich
aufgemacht zur Rute über die Gottlosen, dass nichts von ihnen noch von
ihrem Volk noch von ihrem Haufen Trost haben wird. \bibverse{12} Es
kommt die Zeit, der Tag naht herzu! Der Käufer freue sich nicht, und der
Verkäufer trauere nicht; denn es kommt der Zorn über all ihren Haufen.
\bibverse{13} Darum soll der Verkäufer nach seinem verkauften Gut nicht
wieder trachten; denn wer da lebt, der wird's haben. Denn die Weissagung
über all ihren Haufen wird nicht zurückkehren; keiner wird sein Leben
erhalten, um seiner Missetat willen. \footnote{\textbf{7:13} 3Mo 27,24}
\bibverse{14} Lasst sie die Posaune nur blasen und alles zurüsten, es
wird doch niemand in den Krieg ziehen; denn mein Grimm geht über all
ihren Haufen.

\bibverse{15} Draußen geht das Schwert; drinnen geht Pestilenz und
Hunger. Wer auf dem Felde ist, der wird vom Schwert sterben; wer aber in
der Stadt ist, den wird Pestilenz und Hunger fressen. \bibverse{16} Und
welche unter ihnen entrinnen, die müssen auf den Gebirgen sein, und wie
die Tauben in den Gründen, die alle untereinander girren, ein jeglicher
um seiner Missetat willen. \bibverse{17} Aller Hände werden dahinsinken,
und aller Knie werden so ungewiss stehen wie Wasser; \bibverse{18} und
sie werden Säcke um sich gürten und mit Furcht überschüttet sein, und
aller Angesichter werden jämmerlich sehen und aller Häupter kahl sein.
\bibverse{19} Sie werden ihr Silber hinaus auf die Gassen werfen und ihr
Gold wie Unflat achten; denn ihr Silber und Gold wird sie nicht erretten
am Tage des Zorns des HErrn. Und sie werden ihre Seele davon nicht
sättigen noch ihren Bauch davon füllen; denn es ist ihnen gewesen ein
Anstoß zu ihrer Missetat. \footnote{\textbf{7:19} Spr 11,4; Zeph 1,18}
\bibverse{20} Sie haben aus ihren edlen Kleinoden, damit sie Hoffart
trieben, Bilder ihrer Gräuel und Scheuel gemacht; darum will ich's ihnen
zum Unflat machen \bibverse{21} und will's Fremden in die Hände geben,
dass sie es rauben, und den Gottlosen auf Erden zur Ausbeute, dass sie
es entheiligen sollen. \bibverse{22} Ich will mein Angesicht davon
kehren, dass sie meinen Schatz entheiligen; ja, Räuber sollen
darüberkommen und es entheiligen.

\bibverse{23} Mache Ketten; denn das Land ist voll Blutschulden und die
Stadt voll Frevels. \bibverse{24} So will ich die Ärgsten unter den
Heiden kommen lassen, dass sie sollen ihre Häuser einnehmen, und will
der Hoffart der Gewaltigen ein Ende machen und ihre Heiligtümer
entheiligen. \bibverse{25} Der Ausrotter kommt; da werden sie Frieden
suchen, und wird keiner dasein. \bibverse{26} Ein Unfall wird über den
anderen kommen, ein Gerücht über das andere. So werden sie dann ein
Gesicht bei den Propheten suchen; auch wird weder Gesetz bei den
Priestern noch Rat bei den Alten mehr sein. \bibverse{27} Der König wird
betrübt sein, und die Fürsten werden in Entsetzen gekleidet sein, und
die Hände des Volkes im Lande werden verzagt sein. Ich will mit ihnen
umgehen, wie sie gelebt haben, und will sie richten, wie sie verdient
haben, dass sie erfahren sollen, ich sei der HErr. \# 8 \bibverse{1} Und
es begab sich im sechsten Jahr, am fünften Tage des Sechsten Monats,
dass ich saß in meinem Hause und die Alten aus Juda saßen vor mir;
daselbst fiel die Hand des Herrn HErrn auf mich. \bibverse{2} Und siehe,
ich sah, dass es von seinen Lenden herunterwärts war gleichwie Feuer;
aber oben über seinen Lenden war es lichthell; \bibverse{3} und er
reckte aus gleichwie eine Hand und ergriff mich bei dem Haar meines
Hauptes. Da führte mich ein Wind zwischen Himmel und Erde und brachte
mich gen Jerusalem in einem göttlichen Gesichte zu dem Tor am inneren
Vorhof, das gegen Mitternacht sieht, da stand ein Bild zu Verdruss dem
Hausherrn. \footnote{\textbf{8:3} Hes 3,12} \bibverse{4} Und siehe, da
war die Herrlichkeit des Gottes Israels, wie ich sie zuvor gesehen hatte
im Felde. \footnote{\textbf{8:4} Hes 1,4-28}

\bibverse{5} Und er sprach zu mir: Du Menschenkind, hebe deine Augen auf
gegen Mitternacht. Und da ich meine Augen aufhob gegen Mitternacht,
siehe, da stand gegen Mitternacht das verdrießliche Bild am Tor des
Altars, eben da man hineingeht.

\bibverse{6} Und er sprach zu mir: Du Menschenkind, siehst du auch, was
diese tun? Große Gräuel, die das Haus Israel hier tut, dass sie mich ja
fern von meinem Heiligtum treiben. Aber du wirst noch mehr große Gräuel
sehen.

\bibverse{7} Und er führte mich zur Tür des Vorhofs; da sah ich, und
siehe, da war ein Loch in der Wand.

\bibverse{8} Und er sprach zu mir: Du Menschenkind, grabe durch die
Wand. Und da ich durch die Wand grub, siehe, da war eine Tür.

\bibverse{9} Und er sprach zu mir: Gehe hinein und schaue die bösen
Gräuel, die sie allhier tun.

\bibverse{10} Und da ich hineinkam und sah, siehe, da waren allerlei
Bildnisse der Würmer und Tiere, eitel Scheuel, und allerlei Götzen des
Hauses Israel, allenthalben umher an der Wand gemacht; \footnote{\textbf{8:10}
  Röm 1,23}

\bibverse{11} vor welchen standen siebzig Männer aus den Ältesten des
Hauses Israel, und Jaasanja, der Sohn Saphans, stand auch unter ihnen;
und ein jeglicher hatte sein Räuchfass in der Hand, und ging ein dicker
Nebel auf vom Räuchwerk.

\bibverse{12} Und er sprach zu mir: Menschenkind, siehst du, was die
Ältesten des Hauses Israel tun in der Finsternis, ein jeglicher in
seiner Bilderkammer? Denn sie sagen: Der HErr sieht uns nicht, sondern
der HErr hat das Land verlassen. \bibverse{13} Und er sprach zu mir: Du
sollst noch mehr große Gräuel sehen, die sie tun.

\bibverse{14} Und er führte mich hinein zum Tor an des HErrn Hause, das
gegen Mitternacht steht; und siehe, daselbst saßen Weiber, die weinten
über den Thammus. \bibverse{15} Und er sprach zu: Menschenkind, siehst
du das? Aber du sollst noch größere Gräuel sehen, denn diese sind.

\bibverse{16} Und er führte mich in den inneren Hof am Hause des HErrn;
und siehe, vor der Tür am Tempel des HErrn, zwischen der Halle und dem
Altar, da waren bei fünfundzwanzig Männer, die ihren Rücken gegen den
Tempel des HErrn und ihr Angesicht gegen Morgen gekehrt hatten und
beteten gegen der Sonne Aufgang. \footnote{\textbf{8:16} 2Chr 29,6}

\bibverse{17} Und er sprach zu mir: Menschenkind, siehst du das? Ist's
dem Hause Juda zu wenig, alle solche Gräuel hier zu tun, dass sie auch
sonst im ganzen Lande eitel Gewalt und Unrecht treiben und reizen mich
immer wieder? Und siehe, sie halten die Weinrebe an die Nase.
\bibverse{18} Darum will ich auch wider sie mit Grimm handeln, und mein
Auge soll ihrer nicht verschonen, und ich will nicht gnädig sein; und
wenn sie gleich mit lauter Stimme vor meinen Ohren schreien, will ich
sie doch nicht hören. \# 9 \bibverse{1} Und er rief mit lauter Stimme
vor meinen Ohren und sprach: Lasst herzukommen die Heimsuchung der
Stadt, und ein jeglicher habe eine Mordwaffe in seiner Hand.
\bibverse{2} Und siehe, es kamen sechs Männer auf dem Wege vom Obertor
her, das gegen Mitternacht steht; und ein jeglicher hatte eine
schädliche Waffe in seiner Hand. Aber es war einer unter ihnen, der
hatte Leinwand an und ein Schreibzeug an seiner Seite. Und sie gingen
hinein und traten neben den ehernen Altar. \footnote{\textbf{9:2} Hes
  10,2; Dan 10,5}

\bibverse{3} Und die Herrlichkeit des Gottes Israels erhob sich von dem
Cherub, über dem sie war, zu der Schwelle am Hause und rief dem, der die
Leinwand anhatte und das Schreibzeug an seiner Seite. \footnote{\textbf{9:3}
  Hes 1,4-28} \bibverse{4} Und der HErr sprach zu ihm: Gehe durch die
Stadt Jerusalem und zeichne mit einem Zeichen an die Stirn die Leute,
die da seufzen und jammern über alle Gräuel, die darin geschehen.
\footnote{\textbf{9:4} Offb 7,3; 2Petr 2,8}

\bibverse{5} Zu jenen aber sprach er, dass ich's hörte: Gehet diesem
nach durch die Stadt und schlaget drein; eure Augen sollen nicht schonen
noch übersehen. \bibverse{6} Erwürget Alte, Jünglinge, Jungfrauen,
Kinder und Weiber, alles tot; aber die das Zeichen an sich haben, derer
sollt ihr keinen anrühren. Fanget aber an an meinem Heiligtum! Und sie
fingen an an den alten Leuten, die vor dem Hause waren.

\bibverse{7} Und er sprach zu ihnen: Verunreinigt das Haus und macht die
Vorhöfe voll Erschlagener; gehet heraus! Und sie gingen heraus und
schlugen in der Stadt.

\bibverse{8} Und da sie ausgeschlagen hatten, war ich noch übrig. Und
ich fiel auf mein Angesicht, schrie und sprach: Ach Herr HErr, willst du
denn alle Übrigen in Israel verderben, dass du deinen Zorn so
ausschüttest über Jerusalem? \footnote{\textbf{9:8} Hes 11,13}

\bibverse{9} Und er sprach zu mir: Es ist die Missetat des Hauses Israel
und Juda allzusehr groß; es ist eitel Blutschuld im Lande und Unrecht in
der Stadt. Denn sie sprechen: Der HErr hat das Land verlassen, und der
HErr sieht uns nicht. \footnote{\textbf{9:9} Hes 8,12}

\bibverse{10} Darum soll mein Auge auch nicht schonen, ich will auch
nicht gnädig sein, sondern ihr Tun auf ihren Kopf werfen.

\bibverse{11} Und siehe, der Mann, der die Leinwand anhatte und das
Schreibzeug an seiner Seite, antwortete und sprach: Ich habe getan, wie
du mir geboten hast. \# 10 \bibverse{1} Und ich sah, und siehe, an dem
Himmel über dem Haupt der Cherubim war es gestaltet wie ein Saphir, und
über ihnen war es gleich anzusehen wie ein Thron. \footnote{\textbf{10:1}
  Hes 1,22; Hes 1,26} \bibverse{2} Und er sprach zu dem Mann in der
Leinwand: Gehe hinein zwischen die Räder unter den Cherub und fasse die
Hände voll glühender Kohlen, die zwischen den Cherubim sind, und streue
sie über die Stadt. Und er ging hinein, dass ich's sah, da er
hineinging. \footnote{\textbf{10:2} Hes 9,2; Offb 8,5}

\bibverse{3} Die Cherubim aber standen zur Rechten am Hause, und die
Wolke erfüllte den inneren Vorhof. \bibverse{4} Und die Herrlichkeit des
HErrn erhob sich von dem Cherub zur Schwelle am Hause; und das Haus ward
erfüllt mit der Wolke und der Vorhof voll Glanzes von der Herrlichkeit
des HErrn. \footnote{\textbf{10:4} Hes 1,4-28; Jes 6,4} \bibverse{5} Und
man hörte die Flügel der Cherubim rauschen bis in den äußeren Vorhof wie
eine Stimme des allmächtigen Gottes, wenn er redet.

\bibverse{6} Und da er dem Mann in der Leinwand geboten hatte und
gesagt: Nimm Feuer zwischen den Rädern unter den Cherubim! ging er
hinein und trat neben das Rad. \bibverse{7} Und der Cherub streckte
seine Hand heraus zwischen den Cherubim zum Feuer, das zwischen den
Cherubim war, nahm davon und gab's dem Mann in der Leinwand in die
Hände; der empfing's und ging hinaus. \bibverse{8} Und es erschien an
den Cherubim gleichwie eines Menschen Hand unter ihren Flügeln.

\bibverse{9} Und ich sah, und siehe, vier Räder standen bei den
Cherubim, bei einem jeglichen Cherub ein Rad; und die Räder waren
anzusehen gleichwie ein Türkis \bibverse{10} und waren alle vier eines
wie das andere, als wäre ein Rad im anderen. \bibverse{11} Wenn sie
gehen sollten, so konnten sie nach allen ihren vier Seiten gehen und
mussten sich nicht herumlenken, wenn sie gingen; sondern wohin das erste
ging, da gingen sie nach und mussten sich nicht herumlenken.
\bibverse{12} Und ihr ganzer Leib, Rücken, Hände und Flügel und die
Räder waren voll Augen um und um; alle vier hatten ihre Räder.
\bibverse{13} Und die Räder wurden genannt „der Wirbel``, dass ich's
hörte. \bibverse{14} Ein jeglicher hatte vier Angesichter; das erste
Angesicht war eines Cherubs, das andere eines Menschen, das dritte eines
Löwen, das vierte eines Adlers.

\bibverse{15} Und die Cherubim schwebten empor. Es ist eben das Tier,
das ich sah am Wasser Chebar. \bibverse{16} Wenn die Cherubim gingen, so
gingen die Räder auch neben ihnen; und wenn die Cherubim ihre Flügel
schwangen, dass sie sich von der Erde erhoben, so lenkten sich die Räder
auch nicht von ihnen. \bibverse{17} Wenn jene standen, so standen diese
auch; erhoben sie sich, so erhoben sich diese auch; denn es war der
Geist der Tiere in ihnen.

\bibverse{18} Und die Herrlichkeit des HErrn ging wieder aus von der
Schwelle am Hause und stellte sich über die Cherubim. \bibverse{19} Da
schwangen die Cherubim ihre Flügel und erhoben sich von der Erde vor
meinen Augen; und da sie ausgingen, gingen die Räder neben ihnen. Und
sie traten in das Tor am Hause des HErrn gegen Morgen, und die
Herrlichkeit des Gottes Israels war oben über ihnen. \footnote{\textbf{10:19}
  Hes 10,1}

\bibverse{20} Das ist das Tier, das ich unter dem Gott Israels sah am
Wasser Chebar; und ich merkte, das es Cherubim wären, \bibverse{21} da
ein jegliches vier Angesichter hatte und vier Flügel und unter den
Flügeln gleichwie Menschenhände. \bibverse{22} Es waren ihre Angesichter
gestaltet, wie ich sie am Wasser Chebar sah, und sie gingen stracks vor
sich. \# 11 \bibverse{1} Und mich hob ein Wind auf und brachte mich zum
Tor am Hause des HErrn, das gegen Morgen sieht; und siehe, unter dem Tor
waren fünfundzwanzig Männer; und ich sah unter ihnen Jaasanja, den Sohn
Assurs, und Pelatja, den Sohn Benajas, die Fürsten im Volk. \bibverse{2}
Und er sprach zu mir: Menschenkind, diese Leute haben unselige Gedanken
und schädliche Ratschläge in dieser Stadt; \bibverse{3} denn sie
sprechen: „Es ist nicht so nahe; lasst uns nur Häuser bauen! Sie ist der
Topf, so sind wir das Fleisch.`` \bibverse{4} Darum sollst du,
Menschenkind, wider sie weissagen.

\bibverse{5} Und der Geist des HErrn fiel auf mich, und er sprach zu
mir: Sprich: So sagt der HErr: Ich habe also geredet, ihr vom Hause
Israel; und eures Geistes Gedanken kenne ich wohl. \bibverse{6} Ihr habt
viele erschlagen in dieser Stadt, und ihre Gassen liegen voll Toter.

\bibverse{7} Darum spricht der Herr HErr also: Die ihr darin getötet
habt, die sind das Fleisch, und sie ist der Topf; aber ihr müsset
hinaus. \bibverse{8} Das Schwert, das ihr fürchtet, das will ich über
euch kommen lassen, spricht der Herr HErr. \bibverse{9} Ich will euch
von dort herausstoßen und den Fremden in die Hand geben und will euch
euer Recht tun. \bibverse{10} Ihr sollt durchs Schwert fallen; an der
Grenze Israels will ich euch richten, und sollt erfahren, dass ich der
HErr bin. \footnote{\textbf{11:10} 2Kö 25,20-21} \bibverse{11} Die Stadt
aber soll nicht euer Topf sein noch ihr das Fleisch darin; sondern an
der Grenze Israels will ich euch richten. \bibverse{12} Und ihr sollt
erfahren, dass ich der HErr bin; denn ihr habt nach meinen Geboten nicht
gewandelt und habt meine Rechte nicht gehalten, sondern getan nach der
Heiden Weise, die um euch her sind.

\bibverse{13} Und da ich so weissagte, starb Pelatja, der Sohn Benajas.
Da fiel ich auf mein Angesicht und schrie mit lauter Stimme und sprach:
Ach Herr HErr, du wirst's mit den Übrigen Israels gar aus machen!

\bibverse{14} Da geschah des HErrn Wort zu mir und sprach: \bibverse{15}
Du Menschenkind, zu deinen Brüdern und nahen Freunden und dem ganzen
Haus Israel sprechen wohl die, die noch zu Jerusalem wohnen: Ihr müsset
fern vom HErrn sein, aber wir haben das Land inne.

\bibverse{16} Darum sprich du: So spricht der Herr HErr: Ja, ich habe
sie fern weg unter die Heiden lassen treiben und in die Länder
zerstreut; doch will ich bald ihr Heiland sein in den Ländern, dahin sie
gekommen sind. \footnote{\textbf{11:16} Hes 6,8-10; Jer 24,5-6}

\bibverse{17} Darum sprich: So sagt der Herr HErr: Ich will euch sammeln
aus den Völkern und will euch sammeln aus den Ländern, dahin ihr
zerstreut seid, und will euch das Land Israel geben. \footnote{\textbf{11:17}
  Jer 29,14}

\bibverse{18} Da sollen sie kommen und alle Scheuel und Gräuel daraus
wegtun. \bibverse{19} Und ich will euch ein einträchtiges Herz geben und
einen neuen Geist in euch geben und will das steinerne Herz wegnehmen
aus eurem Leibe und ein fleischernes Herz geben, \footnote{\textbf{11:19}
  Hes 36,26; Jer 24,7} \bibverse{20} auf dass sie in meinen Sitten
wandeln und meine Rechte halten und darnach tun. Und sie sollen mein
Volk sein, so will ich ihr Gott sein. \footnote{\textbf{11:20} Jer 31,33}
\bibverse{21} Denen aber, die nach ihres Herzens Scheueln und Gräueln
wandeln, will ich ihr Tun auf ihren Kopf werfen, spricht der Herr HErr.

\bibverse{22} Da schwangen die Cherubim ihre Flügel, und die Räder
gingen neben ihnen, und die Herrlichkeit des Gottes Israels war oben
über ihnen. \footnote{\textbf{11:22} Hes 1,4-28} \bibverse{23} Und die
Herrlichkeit des HErrn erhob sich aus der Stadt und stellte sich auf den
Berg, der gegen Morgen vor der Stadt liegt. \bibverse{24} Und ein Wind
hob mich auf und brachte mich im Gesicht und im Geist Gottes nach
Chaldäa zu den Gefangenen. Und das Gesicht, das ich gesehen hatte,
verschwand vor mir.

\bibverse{25} Und ich sagte den Gefangenen alle Worte des HErrn, die er
mir gezeigt hatte. \# 12 \bibverse{1} Und des HErrn Wort geschah zu mir
und sprach: \bibverse{2} Du Menschenkind, du wohnst unter einem
ungehorsamen Haus, welches hat wohl Augen, dass sie sehen könnten, und
wollen nicht sehen, Ohren, dass sie hören könnten, und wollen nicht
hören, sondern es ist ein ungehorsames Haus. \footnote{\textbf{12:2} Jes
  6,9-10}

\bibverse{3} Darum, du Menschenkind, nimm dein Wandergerät und zieh am
lichten Tage davon vor ihren Augen. Von deinem Ort sollst du ziehen an
einen anderen Ort vor ihren Augen, ob sie vielleicht merken wollten,
dass sie ein ungehorsames Haus sind. \bibverse{4} Und sollst dein Gerät
heraustun wie Wandergerät bei lichtem Tage vor ihren Augen; und du
sollst ausziehen des Abends vor ihren Augen, gleichwie man auszieht,
wenn man wandern will; \bibverse{5} und du sollst durch die Wand
ausbrechen vor ihren Augen und durch dieselbe ziehen; \bibverse{6} und
du sollst es auf deine Schulter nehmen vor ihren Augen und, wenn es
dunkel geworden ist, hinaustragen; dein Angesicht sollst du verhüllen,
dass du das Land nicht sehest. Denn ich habe dich dem Hause Israel zum
Wunderzeichen gesetzt.

\bibverse{7} Und ich tat wie mir befohlen war, und trug mein Gerät
heraus wie Wandergerät bei lichtem Tage; und am Abend brach ich mit der
Hand durch die Wand; und da es dunkel geworden war, nahm ich's auf die
Schulter und trug's hinaus vor ihren Augen.

\bibverse{8} Und frühmorgens geschah des HErrn Wort zu mir und sprach:
\bibverse{9} Menschenkind, hat das Haus Israel, das ungehorsame Haus,
nicht zu dir gesagt: Was machst du?

\bibverse{10} So sprich zu ihnen: So spricht der Herr HErr: Diese Last
betrifft den Fürsten zu Jerusalem und das ganze Haus Israel, das darin
ist.

\bibverse{11} Sprich: Ich bin euer Wunderzeichen; wie ich getan habe,
also soll ihnen geschehen, dass sie wandern müssen und gefangen geführt
werden. \footnote{\textbf{12:11} Hes 12,6}

\bibverse{12} Ihr Fürst wird seine Habe auf der Schulter tragen im
Dunkel und muss ausziehen durch die Wand, die sie zerbrechen werden,
dass sie dadurch ausziehen; sein Angesicht wird verhüllt werden, dass er
mit keinem Auge das Land sehe. \footnote{\textbf{12:12} Jer 39,7}
\bibverse{13} Ich will auch mein Netz über ihn werfen, dass er in meinem
Garn gefangen werde, und will ihn gen Babel bringen in der Chaldäer
Land, das er doch nicht sehen wird, und er soll daselbst sterben.
\footnote{\textbf{12:13} Hes 17,20; Hes 32,3-6} \bibverse{14} Und alle,
die um ihn her sind, seine Gehilfen und all sein Anhang, will ich unter
alle Winde zerstreuen und das Schwert hinter ihnen her ausziehen.

\bibverse{15} Also sollen sie erfahren, dass ich der HErr sei, wenn ich
sie unter die Heiden verstoße und in die Länder zerstreue. \bibverse{16}
Aber ich will ihrer etliche wenige übrigbleiben lassen vor dem Schwert,
dem Hunger und der Pestilenz; die sollen jener Gräuel erzählen unter den
Heiden, dahin sie kommen werden, und sie sollen erfahren, dass ich der
HErr sei.

\bibverse{17} Und des HErrn Wort geschah zu mir und sprach:
\bibverse{18} Du Menschenkind, du sollst dein Brot essen mit Beben und
dein Wasser trinken mit Zittern und Sorgen. \bibverse{19} Und sprich zum
Volk im Lande: So spricht der Herr HErr von den Einwohnern zu Jerusalem
im Lande Israel: Sie müssen ihr Brot essen in Sorgen und ihr Wasser
trinken im Elend; denn das Land soll wüst werden von allem, was darin
ist, um des Frevels willen aller Einwohner. \bibverse{20} Und die
Städte, die wohl bewohnt sind, sollen verwüstet und das Land öde werden;
also sollt ihr erfahren, dass ich der HErr sei.

\bibverse{21} Und des HErrn Wort geschah zu mir und sprach:
\bibverse{22} Du Menschenkind, was habt ihr für ein Sprichwort im Lande
Israel und sprecht: Weil sich's so lange verzieht, so wird nun hinfort
nichts aus der Weissagung? \footnote{\textbf{12:22} 2Petr 3,4}
\bibverse{23} Darum sprich zu ihnen: So spricht der Herr HErr: Ich will
das Sprichwort aufheben, dass man es nicht mehr führen soll in Israel.
Und rede zu ihnen: Die Zeit ist nahe und alles, was geweissagt ist.
\footnote{\textbf{12:23} Hab 2,3} \bibverse{24} Denn es soll hinfort
kein falsches Gesicht und keine Weissagung mit Schmeichelworten mehr
sein im Hause Israel. \bibverse{25} Denn ich bin der HErr; was ich rede,
das soll geschehen und nicht länger verzogen werden; sondern bei eurer
Zeit, ihr ungehorsames Haus, will ich tun, was ich rede, spricht der
Herr HErr.

\bibverse{26} Und des HErrn Wort geschah zu mir und sprach:
\bibverse{27} Du Menschenkind, siehe, das Haus Israel spricht: Das
Gesicht, das dieser sieht, da ist noch lange hin; und er weissagt auf
die Zeit, die noch ferne ist.

\bibverse{28} Darum sprich zu ihnen: So spricht der Herr HErr: Was ich
rede, soll nicht länger verzogen werden, sondern soll geschehen, spricht
der Herr HErr. \# 13 \bibverse{1} Und des HErrn Wort geschah zu mir und
sprach: \bibverse{2} Du Menschenkind, weissage wider die Propheten
Israels und sprich zu denen, die aus ihrem eigenen Herzen weissagen:
Höret des HErrn Wort! \bibverse{3} So spricht der Herr HErr: Weh den
tollen Propheten, die ihrem eigenen Geist folgen und haben keine
Gesichte! \footnote{\textbf{13:3} Jer 23,21; Jer 23,31} \bibverse{4} O
Israel, deine Propheten sind wie die Füchse in den Wüsten! \bibverse{5}
Sie treten nicht vor die Lücken und machen sich nicht zur Hürde um das
Haus Israel und stehen nicht im Streit am Tage des HErrn. \bibverse{6}
Ihr Gesicht ist nichts, und ihr Weissagen ist eitel Lügen. Sie sprechen:
„Der HErr hat's gesagt``, wenn sie doch der HErr nicht gesandt hat, und
warten, dass ihr Wort bestehe. \footnote{\textbf{13:6} Hes 22,28; Jer
  23,32} \bibverse{7} Ist's nicht also, dass euer Gesicht ist nichts und
euer Weissagen ist eitel Lügen? und ihr sprecht doch: „Der HErr hat's
geredet``, obwohl ich's doch nicht geredet habe.

\bibverse{8} Darum spricht der Herr HErr also: Weil ihr das predigt,
woraus nichts wird, und Lügen weissagt, so will ich an euch, spricht der
Herr HErr. \bibverse{9} Und meine Hand soll kommen über die Propheten,
die das predigen, woraus nichts wird, und Lügen weissagen. Sie sollen in
der Versammlung meines Volks nicht sein und in die Zahl des Hauses
Israel nicht geschrieben werden noch ins Land Israels kommen; und ihr
sollt erfahren, dass ich der Herr HErr bin.

\bibverse{10} Darum dass sie mein Volk verführen und sagen:
„Friede!{}``, so doch kein Friede ist. Das Volk baut die Wand, so
tünchen sie dieselbe mit losem Kalk. \footnote{\textbf{13:10} Jer 6,14}
\bibverse{11} Sprich zu den Tünchern, die mit losem Kalk tünchen, dass
es abfallen wird; denn es wird ein Platzregen kommen und werden große
Hagel fallen und ein Windwirbel wird es zerreißen. \bibverse{12} Siehe,
so wird die Wand einfallen. Was gilt's? dann wird man zu euch sagen: Wo
ist nun das Getünchte, das ihr getüncht habt?

\bibverse{13} So spricht der Herr HErr: Ich will einen Windwirbel reißen
lassen in meinem Grimm und einen Platzregen in meinem Zorn und große
Hagelsteine im Grimm, die sollen alles umstoßen. \bibverse{14} Also will
ich die Wand umwerfen; die ihr mit losem Kalk getüncht habt, und will
sie zu Boden stoßen, dass man ihren Grund sehen soll; so fällt sie, und
ihr sollt darin auch umkommen und erfahren, dass ich der HErr sei.
\bibverse{15} Also will ich meinen Grimm vollenden an der Wand und an
denen, die sie mit losem Kalk tünchen, und will zu euch sagen: Hier ist
weder Wand noch Tüncher. \bibverse{16} Das sind die Propheten Israels,
die Jerusalem weissagen und predigen von Frieden, so doch kein Friede
ist, spricht der Herr HErr.

\bibverse{17} Und du, Menschenkind, richte dein Angesicht wider die
Töchter in deinem Volk, welche weissagen aus ihrem Herzen, und weissage
wider sie \bibverse{18} und sprich: So spricht der Herr HErr: Wehe euch,
die ihr Kissen macht den Leuten unter die Arme und Pfühle zu den
Häuptern, beiden, Jungen und Alten, die Seelen zu fangen. Wenn ihr nun
die Seelen gefangen habt unter meinem Volk, verheißt ihr ihnen das Leben
\bibverse{19} und entheiligt mich in meinem Volk um eine Handvoll Gerste
und einen Bissen Brot, damit dass ihr die Seelen zum Tode verurteilt,
die doch nicht sollten sterben, und urteilt die zum Leben, die doch
nicht leben sollten, durch eure Lügen unter meinem Volk, welches gerne
Lügen hört. \footnote{\textbf{13:19} Jes 5,23; Spr 17,15}

\bibverse{20} Darum spricht der Herr HErr: Siehe, ich will an eure
Kissen, womit ihr Seelen fanget und vertröstet, und will sie von euren
Armen wegreißen und die Seelen, die ihr fanget und vertröstet,
losmachen. \bibverse{21} Und ich will eure Pfühle zerreißen und mein
Volk aus eurer Hand erretten, dass ihr sie nicht mehr fangen sollt; und
ihr sollt erfahren, dass ich der HErr sei. \bibverse{22} Darum dass ihr
das Herz der Gerechten fälschlich betrübet, die ich nicht betrübt habe,
und habt gestärkt die Hände der Gottlosen, dass sie sich von ihrem bösen
Wesen nicht bekehren, damit sie lebendig möchten bleiben: \bibverse{23}
darum sollt ihr nicht mehr unnütze Lehre predigen noch weissagen;
sondern ich will mein Volk aus euren Händen erretten, und ihr sollt
erfahren, dass ich der HErr bin. \# 14 \bibverse{1} Und es kamen etliche
von den Ältesten Israels zu mir und setzten sich vor mir. \footnote{\textbf{14:1}
  Hes 20,1} \bibverse{2} Da geschah des HErrn Wort zu mir und sprach:
\bibverse{3} Menschenkind, diese Leute hangen mit ihrem Herzen an ihren
Götzen und halten an dem Anstoß zu ihrer Missetat; sollte ich denn ihnen
antworten, wenn sie mich fragen? \bibverse{4} Darum rede mit ihnen und
sage zu ihnen: So spricht der Herr HErr: Welcher Mensch vom Hause Israel
mit dem Herzen an seinen Götzen hängt und hält an dem Anstoß zu seiner
Missetat und kommt zum Propheten, dem will ich, der HErr, antworten, wie
er verdient hat mit seiner großen Abgötterei, \bibverse{5} auf dass ich
das Haus Israel fasse an ihrem Herzen, darum dass sie alle von mir
gewichen sind durch Abgötterei.

\bibverse{6} Darum sollst du zum Hause Israel sagen: So spricht der Herr
HErr: Kehret und wendet euch von eurer Abgötterei und wendet euer
Angesicht von allen euren Gräueln. \footnote{\textbf{14:6} Jes 31,6}

\bibverse{7} Denn welcher Mensch vom Hause Israel oder welcher
Fremdling, der in Israel wohnt, von mir weicht und mit seinem Herzen an
seinen Götzen hängt und an dem Ärgernis seiner Abgötterei hält und zum
Propheten kommt, dass er durch ihn mich frage: dem will ich, der HErr,
selbst antworten; \bibverse{8} und will mein Angesicht wider ihn setzen,
dass er soll wüst und zum Zeichen und Sprichwort werden, und will ihn
aus meinem Volk ausrotten, dass ihr erfahren sollt, ich sei der HErr.

\bibverse{9} Wo aber ein Prophet sich betören lässt, etwas zu reden, den
habe ich, der HErr, betört, und will meine Hand über ihn ausstrecken und
ihn aus meinem Volk Israel ausrotten. \footnote{\textbf{14:9} 1Kö
  22,20-23} \bibverse{10} Also sollen sie beide ihre Missetat tragen;
wie die Missetat des Fragers, also soll auch sein die Missetat des
Propheten, \bibverse{11} auf dass das Haus Israel nicht mehr irregehe
von mir und sich nicht mehr verunreinige in aller seiner Übertretung;
sondern sie sollen mein Volk sein, und ich will ihr Gott sein, spricht
der Herr HErr.

\bibverse{12} Und des HErrn Wort geschah zu mir und sprach:
\bibverse{13} Du Menschenkind, wenn ein Land an mir sündigt und dazu
mich verschmäht, so will ich meine Hand über dasselbe ausstrecken und
den Vorrat des Brots wegnehmen und will Teuerung hineinschicken, dass
ich Menschen und Vieh darin ausrotte. \footnote{\textbf{14:13} Hes 5,16}
\bibverse{14} Und wenn dann gleich die drei Männer Noah, Daniel und Hiob
darin wären, so würden sie allein ihre eigene Seele erretten durch ihre
Gerechtigkeit, spricht der Herr HErr. \footnote{\textbf{14:14} Jer 15,1}

\bibverse{15} Und wenn ich böse Tiere in das Land bringen würde, die die
Leute aufräumten und es verwüsteten, dass niemand darin wandeln könnte
vor den Tieren, \footnote{\textbf{14:15} Hes 14,21} \bibverse{16} und
diese drei Männer wären auch darin: so wahr ich lebe, spricht der Herr
HErr, sie würden weder Söhne noch Töchter erretten, sondern allein sich
selbst, und das Land müsste öde werden.

\bibverse{17} Oder wenn ich das Schwert kommen ließe über das Land und
spräche: Schwert, fahre durch das Land! und würde also Menschen und Vieh
ausrotten, \bibverse{18} und die drei Männer wären darin: so wahr ich
lebe, spricht der Herr HErr, sie würden weder Söhne noch Töchter
erretten, sondern sie allein würden errettet sein.

\bibverse{19} Oder wenn ich Pestilenz in das Land schicken und meinen
Grimm über dasselbe ausschütten würde mit Blutvergießen, also dass ich
Menschen und Vieh ausrottete, \bibverse{20} und Noah, Daniel und Hiob
wären darin: so wahr ich lebe, spricht der Herr HErr, würden sie weder
Söhne noch Töchter, sondern allein ihre eigene Seele durch ihre
Gerechtigkeit erretten.

\bibverse{21} Denn so spricht der Herr HErr: So ich meine vier bösen
Strafen, als Schwert, Hunger, böse Tiere und Pestilenz, über Jerusalem
schicken werde, dass ich darin ausrotte Menschen und Vieh, \bibverse{22}
siehe, so sollen etliche Übrige darin davonkommen, die herausgebracht
werden, Söhne und Töchter, und zu euch herkommen, dass ihr sehen werdet
ihr Wesen und Tun und euch trösten über dem Unglück, das ich über
Jerusalem habe kommen lassen samt allem anderen, das ich über sie habe
kommen lassen. \bibverse{23} Sie werden euer Trost sein, wenn ihr sehen
werdet ihr Wesen und Tun; und ihr werdet erfahren, dass ich nicht ohne
Ursache getan habe, was ich darin getan habe, spricht der Herr HErr. \#
15 \bibverse{1} Und des HErrn Wort geschah zu mir und sprach:
\bibverse{2} Du Menschenkind, was ist das Holz vom Weinstock vor anderem
Holz oder eine Rebe vor anderem Holz im Walde? \footnote{\textbf{15:2}
  Jer 2,21} \bibverse{3} Nimmt man es auch und macht etwas daraus? Macht
man auch nur einen Nagel daraus, daran man etwas hängen kann?
\bibverse{4} Siehe, man wirft's ins Feuer, dass es verzehrt wird, dass
das Feuer seine beiden Enden verzehrt und sein Mittles versengt; wozu
sollte es nun taugen? \bibverse{5} Siehe, da es noch ganz war, konnte
man nichts daraus machen; wie viel weniger kann nun hinfort etwas daraus
gemacht werden, so es das Feuer verzehrt und versengt hat!

\bibverse{6} Darum spricht der Herr HErr: Gleichwie ich das Holz vom
Weinstock vor anderem Holz im Walde dem Feuer zu verzehren gebe, also
will ich mit den Einwohnern zu Jerusalem auch umgehen \bibverse{7} und
will mein Angesicht wider sie setzen, dass das Feuer sie fressen soll,
ob sie schon aus dem Feuer herausgekommen sind. Und ihr sollt's
erfahren, dass ich der HErr bin, wenn ich mein Angesicht wider sie setze
\bibverse{8} und das Land wüst mache, darum dass sie mich verschmähen,
spricht der Herr HErr. \# 16 \bibverse{1} Und des HErrn Wort geschah zu
mir und sprach: \bibverse{2} Du Menschenkind, offenbare der Stadt
Jerusalem ihre Gräuel und sprich: \bibverse{3} So spricht der Herr HErr
zu Jerusalem: Dein Geschlecht und deine Geburt ist aus der Kanaaniter
Lande, dein Vater aus den Amoritern und deine Mutter aus den Hethitern.
\bibverse{4} Deine Geburt ist also gewesen: Dein Nabel, da du geboren
wurdest, ist nicht verschnitten; so hat man dich auch mit Wasser nicht
gebadet, dass du sauber würdest, noch mit Salz gerieben noch in Windeln
gewickelt. \bibverse{5} Denn niemand jammerte dein, dass er sich über
dich hätte erbarmt und der Stücke eins dir erzeigt, sondern du wurdest
aufs Feld geworfen. Also verachtet war deine Seele, da du geboren warst.

\bibverse{6} Ich aber ging vor dir vorüber und sah dich in deinem Blut
liegen und sprach zu dir, da du so in deinem Blut lagst: Du sollst
leben! ja, zu dir sprach ich, da du so in deinem Blut lagst: Du sollst
leben! \bibverse{7} Und habe dich erzogen und lassen groß werden wie ein
Gewächs auf dem Felde; und warst nun gewachsen und groß und schön
geworden. Deine Brüste waren gewachsen und hattest schon lange Haare;
aber du warst noch nackt und bloß.

\bibverse{8} Und ich ging vor dir vorüber und sah dich an; und siehe, es
war die Zeit, um dich zu werben. Da breitete ich meinen Mantel über dich
und bedeckte deine Blöße. Und ich gelobte dir's und begab mich mit dir
in einen Bund, spricht der Herr HErr, dass du solltest mein sein.
\footnote{\textbf{16:8} Rt 3,9; 2Mo 19,5}

\bibverse{9} Und ich badete dich mit Wasser und wusch dich von deinem
Blut und salbte dich mit Balsam \bibverse{10} und kleidete dich mit
gestickten Kleidern und zog dir Schuhe von feinem Leder an; ich gab dir
köstliche leinene Kleider und seidene Schleier \bibverse{11} und zierte
dich mit Kleinoden und legte dir Geschmeide an deine Arme und Kettlein
an deinen Hals \bibverse{12} und gab dir ein Haarband an deine Stirn und
Ohrenringe an deine Ohren und eine schöne Krone auf dein Haupt.
\bibverse{13} So warst du geziert mit eitel Gold und Silber und
gekleidet mit eitel Leinwand, Seide und Gesticktem. Du aßest auch eitel
Semmel, Honig und Öl und warst überaus schön und bekamst das Königreich.
\bibverse{14} Und dein Ruhm erscholl unter die Heiden deiner Schöne
halben, welche ganz vollkommen war durch den Schmuck, den ich an dich
gehängt hatte, spricht der Herr HErr.

\bibverse{15} Aber du verließest dich auf deine Schöne; und weil du so
gerühmt warst, triebst du Hurerei, also dass du dich einem jeglichen,
wer vorüberging, gemein machtest und tatest seinen Willen. \footnote{\textbf{16:15}
  2Mo 34,16} \bibverse{16} Und nahmst von deinen Kleidern und machtest
dir bunte Altäre daraus und triebst deine Hurerei darauf, wie nie
geschehen ist noch geschehen wird. \bibverse{17} Du nahmst auch dein
schönes Gerät, das ich dir von meinem Gold und Silber gegeben hatte, und
machtest dir Mannsbilder daraus und triebst deine Hurerei mit ihnen.
\bibverse{18} Und nahmst deine gestickten Kleider und bedecktest sie
damit und mein Öl und Räuchwerk legtest du ihnen vor. \bibverse{19}
Meine Speise, die ich dir zu essen gab, Semmel, Öl, Honig, legtest du
ihnen vor zum süßen Geruch. Ja es kam dahin, spricht der Herr HErr,

\bibverse{20} dass du nahmst deine Söhne und Töchter, die du mir geboren
hattest, und opfertest sie denselben zu fressen. Meinst du denn, dass es
ein Geringes sei um deine Hurerei, \bibverse{21} dass du meine Kinder
schlachtest und lässest sie denselben verbrennen? \bibverse{22} Und in
allen deinen Gräueln und Hurerei hast du nie gedacht an die Zeit deiner
Jugend, wie bloß und nackt du warst und in deinem Blut lagst.
\footnote{\textbf{16:22} Hes 16,6-7}

\bibverse{23} Über alle diese deine Bosheit (ach weh, weh dir! spricht
der Herr HErr) \bibverse{24} bautest du dir Götzenkapellen und machtest
dir Altäre auf allen Gassen; \bibverse{25} und vornan auf allen Straßen
bautest du deine Altäre und machtest deine Schöne zu eitel Gräuel; du
spreiztest deine Beine gegen alle, die vorübergingen, und triebst große
Hurerei. \bibverse{26} Erstlich triebst du Hurerei mit den Kindern
Ägyptens, deinen Nachbarn, die großes Fleisch hatten, und triebst große
Hurerei, mich zu reizen. \bibverse{27} Ich aber streckte meine Hand aus
wider dich und brach dir an deiner Nahrung ab und übergab dich in den
Willen deiner Feinde, der Töchter der Philister, welche sich schämten
vor deinem verruchten Wesen. \bibverse{28} Darnach triebst du Hurerei
mit den Kindern Assur und konntest des nicht satt werden; ja, da du mit
ihnen Hurerei getrieben hattest und des nicht satt werden konntest,
\bibverse{29} machtest du der Hurerei noch mehr bis ins Krämerland
Chaldäa; doch konntest du damit auch nicht satt werden.

\bibverse{30} Wie soll ich dir doch dein Herz beschneiden, spricht der
Herr HErr, weil du solche Werke tust einer großen Erzhure, \bibverse{31}
damit dass du deine Götzenkapellen bautest vornan auf allen Straßen und
deine Altäre machtest auf allen Gassen? Dazu warst du nicht wie eine
andere Hure, die man muss mit Geld kaufen.

\bibverse{32} Du Ehebrecherin, die anstatt ihres Mannes andere zulässt!
\bibverse{33} Denn allen anderen Huren gibt man Geld; du aber gibst
allen deinen Buhlern Geld zu und schenkst ihnen, dass sie zu dir kommen
allenthalben und mit dir Hurerei treiben. \bibverse{34} Und findet sich
an dir das Widerspiel vor anderen Weibern mit deiner Hurerei, weil man
dir nicht nachläuft, sondern du Geld zugibst, und man dir nicht Geld
zugibt; also treibst du das Widerspiel.

\bibverse{35} Darum, du Hure, höre des HErrn Wort! \bibverse{36} So
spricht der Herr HErr: Weil du denn so milde Geld zugibst und deine
Blöße durch deine Hurerei gegen deine Buhlen aufdeckst und gegen alle
Götzen deiner Gräuel und vergießest das Blut deiner Kinder, welche du
ihnen opferst: \bibverse{37} darum, siehe, will ich sammeln alle deine
Buhlen, welchen du wohl gefielst, samt allen, die du für Freunde hältst,
zu deinen Feinden und will sie beide wider dich sammeln allenthalben und
will ihnen deine Blöße aufdecken, dass sie deine Blöße ganz sehen
sollen. \footnote{\textbf{16:37} Jer 13,22; Jer 13,26} \bibverse{38} Und
will das Recht der Ehebrecherinnen und Blutvergießerinnen über dich
gehen und dein Blut vergießen lassen mit Grimm und Eifer. \bibverse{39}
Und will dich in ihre Hände geben, dass sie deine Götzenkapellen
abbrechen und deine Altäre umreißen und dir deine Kleider ausziehen und
dein schönes Gerät dir nehmen und dich nackt und bloß sitzen lassen.
\bibverse{40} Und sie sollen Haufen Leute über dich bringen, die dich
steinigen und mit ihren Schwertern zerhauen \bibverse{41} und deine
Häuser mit Feuer verbrennen und dir dein Recht tun vor den Augen vieler
Weiber. Also will ich deiner Hurerei ein Ende machen, dass du nicht mehr
sollst Geld noch zugeben, \bibverse{42} und will meinen Mut an dir
kühlen und meinen Eifer an dir sättigen, dass ich ruhe und nicht mehr
zürnen müsse.

\bibverse{43} Darum dass du nicht gedacht hast an die Zeit deiner
Jugend, sondern mich mit diesem allem gereizt, darum will ich auch dir
all dein Tun auf den Kopf legen, spricht der Herr HErr, wiewohl ich
damit nicht getan habe nach dem Laster in deinen Gräueln.

\bibverse{44} Siehe, alle die, die Sprichwort pflegen zu üben, werden
von dir dies Sprichwort sagen: „Die Tochter ist wie die Mutter.``
\bibverse{45} Du bist deiner Mutter Tochter, welche Mann und Kinder von
sich stößt, und bist eine Schwester deiner Schwestern, die ihre Männer
und Kinder von sich stoßen. Eure Mutter ist eine von den Hethitern und
euer Vater ein Amoriter. \footnote{\textbf{16:45} Hes 16,3}
\bibverse{46} Samaria ist dein große Schwester mit ihren Töchtern, die
dir zur Linken wohnt; und Sodom ist deine kleine Schwester mit ihren
Töchtern, die zu deiner Rechten wohnt; \footnote{\textbf{16:46} Hes 23,4}
\bibverse{47} wiewohl du dennoch nicht gelebt hast nach ihrem Wesen noch
getan nach ihren Gräueln. Es fehlt nicht viel, dass du es ärger gemacht
hast denn sie in allem deinem Wesen. \bibverse{48} So wahr ich lebe,
spricht der Herr HErr, Sodom, deine Schwester, samt ihren Töchtern hat
nicht so getan wie du und deine Töchter.

\bibverse{49} Siehe, das war deiner Schwester Sodom Missetat: Hoffart
und alles vollauf und guter Friede, den sie und ihre Töchter hatten;
aber den Armen und Dürftigen halfen sie nicht, \bibverse{50} sondern
waren stolz und taten Gräuel vor mir; darum ich sie auch weggetan habe,
da ich begann dareinzusehen. \footnote{\textbf{16:50} 1Mo 18,20}
\bibverse{51} So hat auch Samaria nicht die Hälfte deiner Sünden getan;
sondern du hast deiner Gräuel so viel mehr als sie getan, dass du deine
Schwester fromm gemacht hast gegen alle deine Gräuel, die du getan hast.
\bibverse{52} So trage auch nun deine Schande, die du deiner Schwester
zuerkannt hast. Durch deine Sünden, in welchen du größere Gräuel denn
sie getan hast, machst du sie frömmer, denn du bist. So sei nun auch du
schamrot und trage deine Schande, dass du deine Schwestern fromm gemacht
hast.

\bibverse{53} Ich will aber ihr Gefängnis wenden, nämlich das Gefängnis
dieser Sodom und ihrer Töchter und das Gefängnis dieser Samaria und
ihrer Töchter und das Gefängnis deiner Gefangenen samt ihnen,
\bibverse{54} dass du tragen müssest deine Schande und dich schämest
alles dessen, was du getan hast ihnen zum Troste. \bibverse{55} Und
deine Schwestern, diese Sodom und ihre Töchter, sollen wieder werden,
wie sie zuvor gewesen sind, und Samaria und ihre Töchter sollen wieder
werden, wie sie zuvor gewesen sind; dazu du auch und deine Töchter sollt
wieder werden, wie ihr zuvor gewesen seid. \bibverse{56} Und wirst nicht
mehr die Sodom, deine Schwester, rühmen wie zur Zeit deines Hochmuts,
\bibverse{57} da deine Bosheit noch nicht aufgedeckt war wie zur Zeit,
da dich die Töchter Syriens und die Töchter der Philister allenthalben
schändeten und verachteten dich um und um, \bibverse{58} da ihr musstet
eure Laster und Gräuel tragen, spricht der HErr.

\bibverse{59} Denn also spricht der Herr HErr: Ich will dir tun, wie du
getan hast, dass du den Eid verachtest und brichst den Bund.
\bibverse{60} Ich will aber gedenken an meinen Bund, den ich mit dir
gemacht habe zur Zeit deiner Jugend, und will mit dir einen ewigen Bund
aufrichten. \footnote{\textbf{16:60} 3Mo 26,45; Hos 2,17; Hes 37,26; Jer
  31,31; Jer 31,34} \bibverse{61} Da wirst du an deine Wege gedenken und
dich schämen, wenn du deine großen und kleinen Schwestern zu dir nehmen
wirst, die ich dir zu Töchtern geben werde, aber nicht aus deinem Bund.
\footnote{\textbf{16:61} Hes 20,43} \bibverse{62} Sondern ich will
meinen Bund mit dir aufrichten, dass du erfahren sollst, dass ich der
HErr sei, \bibverse{63} auf dass du daran gedenkest und dich schämest
und vor Schande nicht mehr deinen Mund auftun dürfest, wenn ich dir
alles vergeben werde, was du getan hast, spricht der Herr HErr.
\footnote{\textbf{16:63} Hes 36,31-32}

\hypertarget{section-4}{%
\section{17}\label{section-4}}

\bibverse{1} Und des HErrn Wort geschah zu mir und sprach: \bibverse{2}
Du Menschenkind, lege dem Hause Israel ein Rätsel vor und ein Gleichnis
\bibverse{3} und sprich: So spricht der Herr HErr: Ein großer Adler mit
großen Flügeln und langen Fittichen und voll Federn, die bunt waren, kam
auf den Libanon und nahm den Wipfel von der Zeder \bibverse{4} und brach
das oberste Reis ab und führte es ins Krämerland und setzte es in die
Kaufmannstadt.

\bibverse{5} Er nahm auch vom Samen des Landes und pflanzte es in gutes
Land, da viel Wasser war, und setzte es lose hin. \bibverse{6} Und es
wuchs und ward ein ausgebreiteter Weinstock und niedrigen Stammes; denn
seine Reben bogen sich zu ihm, und seine Wurzeln waren unter ihm; und er
war also ein Weinstock, der Reben kriegte und Zweige.

\bibverse{7} Und da war ein anderer großer Adler mit großen Flügeln und
vielen Federn; und siehe, der Weinstock hatte verlangen an seinen
Wurzeln zu diesem Adler und streckte seine Reben aus gegen ihn, dass er
gewässert würde, vom Platz, da er gepflanzt war. \bibverse{8} Und war
doch auf einen guten Boden an viel Wasser gepflanzt, da er wohl hätte
können Zweige bringen, Früchte tragen und ein herrlicher Weinstock
werden.

\bibverse{9} So sprich nun: Also sagt der Herr HErr: Sollte der geraten?
Ja, man wird seine Wurzeln ausrotten und seine Früchte abreißen, und er
wird verdorren, dass alle Blätter seines Gewächses verdorren werden; und
es wird nicht geschehen durch großen Arm noch viel Volks, dass man ihn
von seinen Wurzeln wegführe. \bibverse{10} Siehe, er ist zwar gepflanzt;
aber sollte er geraten? Ja, sobald der Ostwind an ihn rühren wird, wird
er verdorren auf dem Platz, da er gewachsen ist.

\bibverse{11} Und des HErrn Wort geschah zu mir und sprach:
\bibverse{12} Sprich doch zu dem ungehorsamen Haus: Wisset ihr nicht,
was das ist? Und sprich: Siehe, es kam der König zu Babel gen Jerusalem
und nahm ihren König und ihre Fürsten und führte sie weg zu sich gen
Babel. \footnote{\textbf{17:12} 2Kö 24,10; 2Kö 24,15} \bibverse{13} Und
nahm einen vom königlichen Geschlecht und machte einen Bund mit ihm und
nahm einen Eid von ihm; aber die Gewaltigen im Lande nahm er weg,
\footnote{\textbf{17:13} 2Kö 24,17} \bibverse{14} damit das Königreich
demütig bliebe und sich nicht erhöbe, auf dass sein Bund gehalten würde
und bestünde. \bibverse{15} Aber derselbe fiel von ihm ab und sandte
seine Botschaft nach Ägypten, dass man ihm Rosse und viel Volks schicken
sollte. Sollte es dem geraten? Sollte er davonkommen, der solches tut?
Und sollte der, der den Bund bricht, davonkommen?

\bibverse{16} So wahr ich lebe, spricht der Herr HErr, an dem Ort des
Königs, der ihn zum König gesetzt hat, dessen Eid er verachtet und
dessen Bund er gebrochen hat, da soll er sterben, nämlich zu Babel.
\bibverse{17} Auch wird ihm Pharao nicht beistehen im Kriege mit großem
Heer und vielem Volk, wenn man den Wall aufwerfen wird und die Bollwerke
bauen, dass viel Leute umgebracht werden. \bibverse{18} Denn weil er den
Eid verachtet und den Bund gebrochen hat, darauf er seine Hand gegeben
hat, und solches alles tut, wird er nicht davonkommen.

\bibverse{19} Darum spricht der Herr HErr also; So wahr als ich lebe, so
will ich meinen Eid, den er verachtet hat, und meinen Bund, den er
gebrochen hat, auf seinen Kopf bringen. \bibverse{20} Denn ich will mein
Netz über ihn werfen, und er muss in meinem Garn gefangen werden; und
ich will ihn gen Babel bringen und will daselbst mit ihm rechten über
dem, dass er sich also an mir vergriffen hat. \footnote{\textbf{17:20}
  Hes 12,13} \bibverse{21} Und alle seine Flüchtigen, die ihm anhingen,
sollen durchs Schwert fallen, und ihre Übrigen sollen in alle Winde
zerstreut werden; und ihr sollt's erfahren, dass ich, der HErr, es
geredet habe.

\bibverse{22} So spricht der Herr HErr: Ich will auch von dem Wipfel des
hohen Zedernbaumes nehmen und oben von seinen Zweigen ein zartes Reis
brechen und will's auf einen hohen, erhabenen Berg pflanzen;
\bibverse{23} auf den hohen Berg Israels will ich's pflanzen, dass es
Zweige gewinne und Früchte bringe und ein herrlicher Zedernbaum werde,
also dass allerlei Vögel unter ihm wohnen und allerlei Fliegendes unter
dem Schatten seiner Zweige bleiben möge. \footnote{\textbf{17:23} Hes
  20,40; Dan 4,9; Mt 13,32} \bibverse{24} Und sollen alle Feldbäume
erfahren, dass ich, der HErr, den hohen Baum erniedrigt und den
niedrigen Baum erhöht habe und den grünen Baum ausgedörrt und den dürren
Baum grünend gemacht habe. Ich, der HErr, rede es und tue es auch.
\footnote{\textbf{17:24} Hes 21,31}

\hypertarget{section-5}{%
\section{18}\label{section-5}}

\bibverse{1} Und des HErrn Wort geschah zu mir und sprach: \bibverse{2}
Was treibt ihr unter euch im Lande Israel dies Sprichwort und sprecht:
„Die Väter haben Herlinge gegessen, aber den Kindern sind die Zähne
davon stumpf geworden``? \footnote{\textbf{18:2} Jer 31,29}

\bibverse{3} So wahr als ich lebe, spricht der Herr HErr, solches
Sprichwort soll nicht mehr unter euch gehen in Israel. \bibverse{4} Denn
siehe, alle Seelen sind mein; des Vaters Seele ist sowohl mein als des
Sohnes Seele. Welche Seele sündigt, die soll sterben. \bibverse{5} Wenn
nun einer fromm ist, der recht und wohl tut, \bibverse{6} der auf den
Bergen nicht isset, der seine Augen nicht aufhebt zu den Götzen des
Hauses Israel und seines Nächsten Weib nicht befleckt und liegt nicht
bei der Frau in ihrer Krankheit, \bibverse{7} der niemand beschädigt,
der dem Schuldner sein Pfand wiedergibt, der niemand etwas mit Gewalt
nimmt, der dem Hungrigen sein Brot mitteilt und den Nackten kleidet,
\footnote{\textbf{18:7} Ps 15,3; 5Mo 24,10-13} \bibverse{8} der nicht
wuchert, der nicht Zins nimmt, der seine Hand vom Unrechten kehrt, der
zwischen den Leuten recht urteilt, \footnote{\textbf{18:8} 2Mo 22,24}
\bibverse{9} der nach meinen Rechten wandelt und meine Gebote hält, dass
er ernstlich darnach tue: das ist ein frommer Mann, der soll das Leben
haben, spricht der Herr HErr. \bibverse{10} Wenn er aber einen Sohn
zeugt, und derselbe wird ein Mörder, der Blut vergießt oder dieser
Stücke eins tut, \bibverse{11} und der anderen Stücke keines tut,
sondern auf den Bergen isset und seines Nächsten Weib befleckt,
\bibverse{12} die Armen und Elenden beschädigt, mit Gewalt etwas nimmt,
das Pfand nicht wiedergibt, seine Augen zu den Götzen aufhebt und einen
Gräuel begeht, \bibverse{13} auf Wucher gibt, Zins nimmt: sollte der
leben? Er soll nicht leben, sondern weil er solche Gräuel alle getan
hat, soll er des Todes sterben; sein Blut soll auf ihm sein.

\bibverse{14} Wenn er aber einen Sohn zeugt, der alle solche Sünden
sieht, die sein Vater tut, und sich fürchtet und nicht also tut,
\bibverse{15} isst nicht auf den Bergen, hebt seine Augen nicht auf zu
den Götzen des Hauses Israel, befleckt nicht seines Nächsten Weib,
\bibverse{16} beschädigt niemand, behält das Pfand nicht, nimmt nicht
mit Gewalt etwas, teilt sein Brot mit dem Hungrigen und kleidet den
Nackten, \bibverse{17} der seine Hand vom Unrechten kehrt, keinen Wucher
noch Zins nimmt, sondern meine Gebote hält und nach meinen Rechten lebt:
der soll nicht sterben um seines Vaters Missetat willen, sondern leben.
\bibverse{18} Aber sein Vater, der Gewalt und Unrecht geübt hat und
unter seinem Volk getan hat, was nicht taugt, siehe, der soll sterben um
seiner Missetat willen.

\bibverse{19} So sprecht ihr: Warum soll denn ein Sohn nicht tragen
seines Vaters Missetat? Darum dass er recht und wohl getan und alle
meine Rechte gehalten und getan hat, soll er leben. \bibverse{20} Denn
welche Seele sündigt, die soll sterben. Der Sohn soll nicht tragen die
Missetat des Vaters, und der Vater soll nicht tragen die Missetat des
Sohnes; sondern des Gerechten Gerechtigkeit soll über ihm sein, und des
Ungerechten Ungerechtigkeit soll über ihm sein.. \footnote{\textbf{18:20}
  2Mo 20,5; 4Mo 26,11}

\bibverse{21} Wenn sich aber der Gottlose bekehrt von allen seine
Sünden, die er getan hat, und hält alle meine Rechte und tut recht und
wohl, so soll er leben und nicht sterben. \bibverse{22} Es soll aller
seiner Übertretung, die er begangen hat, nicht gedacht werden; sondern
er soll leben um der Gerechtigkeit willen, die er tut. \bibverse{23}
Meinest du, dass ich Gefallen habe am Tode des Gottlosen, spricht der
HErr, und nicht vielmehr, dass er sich bekehre von seinem Wesen und
lebe? \footnote{\textbf{18:23} Hes 18,32; Hes 33,11}

\bibverse{24} Und wo sich der Gerechte kehrt von seiner Gerechtigkeit
und tut Böses und lebt nach allen Gräueln, die ein Gottloser tut, sollte
der leben? Ja, aller seiner Gerechtigkeit, die er getan hat, soll nicht
gedacht werden; sondern in seiner Übertretung und Sünde, die er getan
hat, soll er sterben. \footnote{\textbf{18:24} Hes 3,20}

\bibverse{25} Doch sprecht ihr: Der HErr handelt nicht recht. So höret
nun, ihr vom Hause Israel: Ist's nicht also, dass ich recht habe und ihr
unrecht habt? \footnote{\textbf{18:25} Hes 33,17-20} \bibverse{26} Denn
wenn der Gerechte sich kehrt von seiner Gerechtigkeit und tut Böses, so
muss er sterben; er muss aber um seiner Bosheit willen, die er getan
hat, sterben. \bibverse{27} Wiederum, wenn sich der Gottlose kehrt von
seiner Ungerechtigkeit, die er getan hat, und tut nun recht und wohl,
der wird seine Seele lebendig erhalten. \bibverse{28} Denn weil er sieht
und bekehrt sich von aller seiner Bosheit, die er getan hat, so soll er
leben und nicht sterben. \bibverse{29} Doch sprechen die vom Hause
Israel: Der HErr handelt nicht recht. Sollte ich Unrecht haben? Ihr vom
Hause Israel habt unrecht.

\bibverse{30} Darum will ich euch richten, ihr vom Hause Israel, einen
jeglichen nach seinem Wesen, spricht der Herr HErr. Darum so bekehret
euch von aller eurer Übertretung, auf dass ihr nicht fallen müsset um
der Missetat willen. \bibverse{31} Werfet von euch alle eure
Übertretung, damit ihr übertreten habt, und machet euch ein neues Herz
und einen neuen Geist. Denn warum willst du sterben, du Haus Israel?
\footnote{\textbf{18:31} Hes 36,26} \bibverse{32} Denn ich habe keinen
Gefallen am Tode des Sterbenden, spricht der Herr HErr. Darum bekehret
euch, so werdet ihr leben. \footnote{\textbf{18:32} Hes 18,23}

\hypertarget{section-6}{%
\section{19}\label{section-6}}

\bibverse{1} Du aber mache eine Wehklage über die Fürsten Israels
\bibverse{2} und sprich: Warum liegt deine Mutter, die Löwin, unter den
Löwen und erzieht ihre Jungen unter den jungen Löwen? \bibverse{3} Deren
eines zog sie auf, und ward ein junger Löwe daraus, der gewöhnte sich,
die Leute zu zerreißen und zu fressen. \bibverse{4} Da das die Heiden
von ihm hörten, fingen sie ihn in ihren Gruben und führten ihn an Ketten
nach Ägyptenland. \footnote{\textbf{19:4} 2Kö 23,30-34} \bibverse{5} Da
nun die Mutter sah, dass ihre Hoffnung verloren war, da sie lange
gehofft hatte, nahm sie ein anderes aus ihren Jungen und machte einen
jungen Löwen daraus. \bibverse{6} Da er unter den Löwen wandelte, ward
er ein junger Löwe; der gewöhnte sich auch, die Leute zu zerreißen und
zu fressen. \bibverse{7} Er verderbte ihre Paläste und verwüstete ihre
Städte, dass das Land und was darin ist, vor der Stimme seines Brüllens
sich entsetzte. \bibverse{8} Da legten sich die Heiden aus allen Ländern
ringsumher und warfen ein Netz über ihn und fingen ihn in ihren Gruben
\bibverse{9} und stießen ihn gebunden in einen Käfig und führten ihn zum
König zu Babel; und man ließ ihn verwahren, dass seine Stimme nicht mehr
gehört würde auf den Bergen Israels. \footnote{\textbf{19:9} 2Kö 24,15}
\bibverse{10} Deine Mutter war wie ein Weinstock, gleich wie du am
Wasser gepflanzt; und seine Frucht und Reben wuchsen von dem großen
Wasser, \footnote{\textbf{19:10} Hes 17,6} \bibverse{11} dass seine
Reben so stark wurden, dass sie zu Herrenzeptern gut waren, und er ward
hoch unter den Reben. Und da man sah, dass er so hoch war und viel Reben
hatte, \bibverse{12} ward er mit Grimm ausgerissen und zu Boden
geworfen; der Ostwind verdorrte seine Frucht, und seine starken Reben
wurden zerbrochen, dass sie verdorrten und verbrannt wurden. \footnote{\textbf{19:12}
  Hes 15,4} \bibverse{13} Nun aber ist er gepflanzt in der Wüste, in
einem dürren, durstigen Lande, \bibverse{14} und ist ein Feuer
ausgegangen von seinen starken Reben, das verzehrte seine Frucht, dass
in ihm keine starke Rebe mehr ist zu einem Herrenzepter. Das ist ein
kläglich und jämmerlich Ding. \# 20 \bibverse{1} Und es begab sich im
siebenten Jahr, am zehnten Tage des fünften Monats, kamen etliche aus
den Ältesten Israels, den HErrn zu fragen, und setzten sich vor mir
nieder.

\bibverse{2} Da geschah des HErrn Wort zu mir und sprach: \bibverse{3}
Du Menschenkind, sage den Ältesten Israels und sprich zu ihnen: So
spricht der Herr HErr: Seid ihr gekommen, mich zu fragen? So wahr ich
lebe, ich will von euch ungefragt sein, spricht der Herr HErr.
\footnote{\textbf{20:3} Hes 14,3}

\bibverse{4} Aber willst du sie strafen, du Menschenkind, so magst du
sie also strafen: zeige ihnen an die Gräuel ihrer Väter \bibverse{5} und
sprich zu ihnen: So spricht der Herr HErr: Zu der Zeit, da ich Israel
erwählte, erhob ich meine Hand zu dem Samen des Hauses Jakob und gab
mich ihnen zu erkennen in Ägyptenland. Ja, ich erhob meine Hand zu ihnen
und sprach: Ich bin der HErr, euer Gott. \bibverse{6} Ich erhob aber zur
selben Zeit meine Hand, dass ich sie führte aus Ägyptenland in ein Land,
das ich ihnen ersehen hatte, das mit Milch und Honig fließt, ein edles
Land vor allen Ländern, \bibverse{7} und sprach zu ihnen: Ein jeglicher
werfe weg die Gräuel vor seinen Augen, und verunreinigt euch nicht an
den Götzen Ägyptens! denn ich bin der HErr, euer Gott. \footnote{\textbf{20:7}
  Jos 24,14; Jos 24,23}

\bibverse{8} Sie aber waren mir ungehorsam und wollten nicht gehorchen
und warf ihrer keiner weg die Gräuel vor seinen Augen und verließen die
Götzen Ägyptens nicht. Da dachte ich meinen Grimm über sie auszuschütten
und all meinen Zorn über sie gehen zu lassen noch in Ägyptenland.
\bibverse{9} Aber ich ließ es um meines Namens willen, dass er nicht
entheiligt würde vor den Heiden, unter denen sie waren und vor denen ich
mich ihnen hatte zu erkennen gegeben, dass ich sie aus Ägyptenland
führen wollte. \bibverse{10} Und da ich sie aus Ägyptenland geführt
hatte und in die Wüste gebracht, \bibverse{11} gab ich ihnen meine
Gebote und lehrte sie meine Rechte, durch welche lebt der Mensch, der
sie hält. \footnote{\textbf{20:11} 3Mo 18,5} \bibverse{12} Ich gab ihnen
auch meine Sabbate zum Zeichen zwischen mir und ihnen, damit sie
lernten, dass ich der HErr sei, der sie heiligt. \footnote{\textbf{20:12}
  2Mo 31,13; 2Mo 31,17}

\bibverse{13} Aber das Haus Israel war mir ungehorsam auch in der Wüste
und lebten nicht nach meinen Geboten und verachteten meine Rechte, durch
welche der Mensch lebt, der sie hält, und entheiligten meine Sabbate
sehr. Da gedachte ich meinem Grimm über sie auszuschütten in der Wüste
und sie ganz umzubringen. \bibverse{14} Aber ich ließ es um meines
Namens willen, auf dass er nicht entheiligt würde vor den Heiden, vor
welchen ich sie hatte ausgeführt. \footnote{\textbf{20:14} Hes 20,9}
\bibverse{15} Und ich hob auch meine Hand auf wider sie in der Wüste,
dass ich sie nicht wollte bringen in das Land, das ich ihnen gegeben
hatte, das mit Milch und Honig fließt, ein edles Land vor allen Ländern,
\footnote{\textbf{20:15} 4Mo 14,12} \bibverse{16} darum dass sie meine
Rechte verachtet und nach meinen Geboten nicht gelebt und meine Sabbate
entheiligt hatten; denn sie wandelten nach den Götzen ihres Herzens.
\bibverse{17} Aber mein Auge verschonte sie, dass ich sie nicht
verderbte noch ganz umbrächte in der Wüste. \bibverse{18} Und ich sprach
zu ihren Kindern in der Wüste: Ihr sollt nach eurer Väter Geboten nicht
leben und ihre Rechte nicht halten und an ihren Götzen euch nicht
verunreinigen. \bibverse{19} Denn ich bin der HErr, euer Gott; nach
meinen Geboten sollt ihr leben, und meine Rechte sollt ihr halten und
darnach tun; \bibverse{20} und meine Sabbate sollt ihr heiligen, dass
sie seien ein Zeichen zwischen mir und euch, damit ihr wisset, das ich,
der HErr, euer Gott bin. \footnote{\textbf{20:20} Hes 20,12}

\bibverse{21} Aber die Kinder waren mir auch ungehorsam, lebten nach
meinen Geboten nicht, hielten auch meine Rechte nicht, dass sie darnach
täten, durch welche der Mensch lebt, der sie hält, und entheiligten
meine Sabbate. Da gedachte ich, meinen Grimm über sie auszuschütten und
allen meinen Zorn über sie gehen zu lassen in der Wüste. \bibverse{22}
Ich wandte aber meine Hand und ließ es um meines Namens willen, auf dass
er nicht entheiligt würde vor den Heiden, vor welchen ich sie hatte
ausgeführt. \bibverse{23} Ich hob auch meine Hand auf wider sie in der
Wüste, dass ich sie zerstreute unter die Heiden und zerstäubte in die
Länder, \bibverse{24} darum dass sie meine Gebote nicht gehalten und
meine Rechte verachtet und meine Sabbate entheiligt hatten und nach den
Götzen ihrer Väter sahen. \bibverse{25} Darum übergab ich sie in die
Lehre, die nicht gut ist, und in Rechte, darin sie kein Leben konnten
haben, \bibverse{26} und ließ sie unrein werden durch ihre Opfer, da sie
alle Erstgeburt durchs Feuer gehen ließen, damit ich sie verstörte und
sie lernen mussten, dass ich der HErr sei. \footnote{\textbf{20:26} Hes
  20,31; 2Chr 33,6}

\bibverse{27} Darum rede, du Menschenkind, mit dem Hause Israel und
sprich zu ihnen: So spricht der Herr HErr: Eure Väter haben mich noch
weiter gelästert und mir getrotzt. \bibverse{28} Denn da ich sie in das
Land gebracht hatte, über welches ich meine Hand aufgehoben hatte, dass
ich's ihnen gäbe: wo sie einen hohen Hügel oder dichten Baum ersahen,
daselbst opferten sie ihre Opfer und brachten dahin ihre verdrießlichen
Gaben und räucherten daselbst ihren süßen Geruch und gossen daselbst
ihre Trankopfer. \bibverse{29} Ich aber sprach zu ihnen: Was soll doch
die Höhe, dahin ihr geht? Und also heißt sie bis auf diesen Tag „die
Höhe``.

\bibverse{30} Darum sprich zum Hause Israel: So spricht der Herr HErr:
Ihr verunreinigt euch in dem Wesen eurer Väter und treibt Abgötterei mit
ihren Gräueln \bibverse{31} und verunreinigt euch an euren Götzen,
welchen ihr eure Gaben opfert und eure Söhne und Töchter durchs Feuer
gehen lasst, bis auf den heutigen Tag; und ich sollte mich von euch,
Haus Israel, fragen lassen? So wahr ich lebe, spricht der Herr HErr, ich
will von euch ungefragt sein.

\bibverse{32} Dazu, was ihr gedenkt: „Wir wollen tun wie die Heiden und
wie andere Leute in den Ländern: Holz und Stein anbeten``, das soll euch
fehlschlagen. \bibverse{33} So wahr ich lebe, spricht der Herr HErr, ich
will über euch herrschen mit starker Hand und ausgestrecktem Arm und mit
ausgeschüttetem Grimm \bibverse{34} und will euch aus den Völkern führen
und aus den Ländern, dahin ihr verstreut seid, sammeln mit starker Hand,
mit ausgestrecktem Arm und mit ausgeschüttetem Grimm, \bibverse{35} und
will euch bringen in die Wüste der Völker und daselbst mit euch rechten
von Angesicht zu Angesicht. \footnote{\textbf{20:35} Hos 2,16}
\bibverse{36} Wie ich mit euren Vätern in der Wüste bei Ägypten
gerechtet habe, ebenso will ich auch mit euch rechten, spricht der Herr
HErr. \footnote{\textbf{20:36} 4Mo 14,22-23} \bibverse{37} Ich will euch
wohl unter die Rute bringen und euch in die Bande des Bundes zwingen
\bibverse{38} und will die Abtrünnigen und die wider mich übertreten,
unter euch ausfegen; ja, aus dem Lande, da ihr jetzt wohnt, will ich sie
führen und ins Land Israel nicht kommen lassen, dass ihr lernen sollt,
ich sei der HErr.

\bibverse{39} Darum, ihr vom Hause Israel, so spricht der Herr HErr:
Weil ihr denn mir ja nicht wollt gehorchen, so fahret hin und diene ein
jeglicher seinen Götzen; aber meinen heiligen Namen lasst hinfort
ungeschändet mit euren Opfern und Götzen. \bibverse{40} Denn so spricht
der Herr HErr: Auf meinem heiligen Berge, auf dem hohen Berge Israel,
daselbst wird mir das ganze Haus Israel, alle die im Lande sind, dienen;
daselbst werden sie mir angenehm sein, und daselbst will ich eure
Hebopfer und Erstlinge eurer Opfer fordern mit allem, was ihr mir
heiligt. \footnote{\textbf{20:40} Hes 17,23} \bibverse{41} Ihr werdet
mir angenehm sein mit dem süßen Geruch, wenn ich euch aus den Völkern
bringen und aus den Ländern sammeln werde, dahin ihr verstreut seid, und
werde in euch geheiligt werden vor den Heiden. \bibverse{42} Und ihr
werdet erfahren, dass ich der HErr bin, wenn ich euch ins Land Israel
gebracht habe, in das Land, darüber ich meine Hand aufhob, dass ich's
euren Vätern gäbe. \bibverse{43} Daselbst werdet ihr gedenken an euer
Wesen und an all euer Tun, darin ihr verunreinigt seid, und werdet
Missfallen haben über alle eure Bosheit, die ihr getan habt.
\bibverse{44} Und werdet erfahren, dass ich der HErr bin, wenn ich mit
euch tue um meines Namens willen und nicht nach eurem bösen Wesen und
schädlichen Tun, du Haus Israel, spricht der Herr HErr. \# 21
\bibverse{1} Und des HErrn Wort geschah zu mir und sprach: \bibverse{2}
Du Menschenkind, richte dein Angesicht gegen den Südwind zu und predige
gegen den Mittag und weissage wider den Wald im Felde gegen Mittag.
\bibverse{3} Und sprich zum Walde gegen Mittag: Höre des HErrn Wort! So
spricht der Herr HErr: Siehe, ich will in dir ein Feuer anzünden, das
soll beide, grüne und dürre Bäume, verzehren, dass man seine Flamme
nicht wird löschen können; sondern es soll verbrannt werden alles, was
vom Mittag gegen Mitternacht steht. \bibverse{4} Und alles Fleisch soll
sehen, dass ich, der HErr, es angezündet habe und niemand löschen kann.
\bibverse{5} Und ich sprach: Ach Herr HErr, sie sagen von mir: Dieser
redet eitel Rätselworte. \bibverse{6} Und des HErrn Wort geschah zu mir
und sprach: \bibverse{7} Du Menschenkind, richte dein Angesicht wider
Jerusalem und predige wider die Heiligtümer und weissage wider das Land
Israel \bibverse{8} und sprich zum Lande Israel: So spricht der Herr
HErr: Siehe, ich will an dich; ich will mein Schwert aus der Scheide
ziehen und will in dir ausrotten beide, Gerechte und Ungerechte.
\bibverse{9} Weil ich denn in dir Gerechte und Ungerechte ausrotte, so
wird mein Schwert aus der Scheide fahren über alles Fleisch, vom Mittag
her bis gen Mitternacht. \bibverse{10} Und soll alles Fleisch erfahren,
dass ich, der HErr, mein Schwert habe aus seiner Scheide gezogen; und es
soll nicht wieder eingesteckt werden. \bibverse{11} Und du,
Menschenkind, sollst seufzen, bis dir die Lenden weh tun, ja, bitterlich
sollst du seufzen, dass sie es sehen. \bibverse{12} Und wenn sie zu dir
sagen werden: Warum seufzest du? sollst du sagen: Um des Geschreies
willen, das da kommt, vor welchem alle Herzen verzagen, und alle Hände
sinken, aller Mut fallen und alle Knie so ungewiss stehen werden wie
Wasser. Siehe, es kommt und wird geschehen, spricht der Herr HErr.
\bibverse{13} Und des HErrn Wort geschah zu mir und sprach:
\bibverse{14} Du Menschenkind, weissage und sprich: So spricht der HErr:
Sprich: Das Schwert, ja, das Schwert ist geschärft und gefegt.
\footnote{\textbf{21:14} Hes 32,20} \bibverse{15} Es ist geschärft, dass
es schlachten soll; es ist gefegt, dass es blinken soll. O wie froh
wollten wir sein, wenn er gleich alle Bäume zu Ruten machte über die
bösen Kinder! \bibverse{16} Aber er hat ein Schwert zu fegen gegeben,
dass man es fassen soll; es ist geschärft und gefegt, dass man's dem
Totschläger in die Hand gebe. \bibverse{17} Schreie und heule, du
Menschenkind; denn es geht über mein Volk und über alle Regenten in
Israel, die dem Schwert samt meinem Volk verfallen sind. Darum schlage
auf deine Lenden. \bibverse{18} Denn er hat sie oft gezüchtigt; was
hat's geholfen? Es will der bösen Kinder Rute nicht helfen, spricht der
Herr HErr. \bibverse{19} Und du, Menschenkind, weissage und schlage
deine Hände zusammen. Denn das Schwert wird zweifach, ja dreifach
kommen, ein Würgeschwert, ein Schwert großer Schlacht, das sie auch
treffen wird in den Kammern, dahin sie fliehen. \bibverse{20} Ich will
das Schwert lassen klingen, dass die Herzen verzagen und viele fallen
sollen an allen ihren Toren. Ach, wie glänzt es und haut daher zur
Schlacht! \bibverse{21} Haue drein, zur Rechten und Linken, was vor dir
ist! \bibverse{22} Da will ich dann mit meinen Händen darob frohlocken
und meinen Zorn gehen lassen. Ich, der HErr, habe es gesagt.
\bibverse{23} Und des HErrn Wort geschah zu mir und sprach:
\bibverse{24} Du Menschenkind, mache zwei Wege, durch welche kommen soll
das Schwert des Königs zu Babel; sie sollen aber alle beide aus einem
Lande gehen. \footnote{\textbf{21:24} Hes 4,1} \bibverse{25} Und stelle
ein Zeichen vorn an den Weg zur Stadt, dahin es weisen soll; und mache
den Weg, dass das Schwert komme gen Rabba der Kinder Ammon und nach
Juda, zu der festen Stadt Jerusalem. \bibverse{26} Denn der König zu
Babel wird sich an die Wegscheide stellen, vorn an den zwei Wegen, dass
er sich wahrsagen lasse, mit den Pfeilen das Los werfe, seinen Abgott
frage und schaue die Leber an. \bibverse{27} Und die Wahrsagung wird auf
die rechte Seite gen Jerusalem deuten, dass er solle Sturmböcke
hinanführen lassen und Löcher machen und mit großem Geschrei sie
überfalle und morde, und dass er Böcke führen soll wider die Tore und da
Wall aufschütte und Bollwerk baue. \bibverse{28} Aber es wird sie
solches Wahrsagen falsch dünken, er schwöre, wie teuer er will. Er aber
wird denken an die Missetat, dass er sie gewinne. \bibverse{29} Darum
spricht der Herr HErr also: Darum dass euer gedacht wird um eure
Missetat und euer Ungehorsam offenbart ist, dass man eure Sünden sieht
in allem eurem Tun, ja, darum dass euer gedacht wird, werdet ihr mit
Gewalt gefangen werden. \bibverse{30} Und du, Fürst in Israel, der du
verdammt und verurteilt bist, dessen Tag daherkommen wird, wenn die
Missetat zum Ende gekommen ist, \bibverse{31} so spricht der Herr HErr:
Tue weg den Hut und hebe ab die Krone! Denn es wird weder der Hut noch
die Krone bleiben; sondern der sich erhöht hat, der soll erniedrigt
werden, und der sich erniedrigt, soll erhöht werden. \bibverse{32} Ich
will die Krone zunichte, zunichte, zunichte machen, bis der komme, der
sie haben soll; dem will ich sie geben. \footnote{\textbf{21:32} 1Mo
  49,10} \bibverse{33} Und du, Menschenkind, weissage und sprich: So
spricht der Herr HErr von den Kindern Ammon und von ihrem Schmähen; und
sprich: Das Schwert, das Schwert ist gezückt, dass es schlachten soll;
es ist gefegt, dass es würgen soll und soll blinken, \footnote{\textbf{21:33}
  Hes 25,2-7} \bibverse{34} darum dass du falsche Gesichte dir sagen
lässest und Lügen weissagen, damit du auch hingegeben werdest unter die
erschlagenen Gottlosen, welchen ihr Tag kam, da die Missetat zum Ende
gekommen war. \bibverse{35} Und ob's schon wieder in die Scheide
gesteckt würde, so will ich dich doch richten an dem Ort, da du
geschaffen, und in dem Lande, da du geboren bist, \bibverse{36} und will
meinen Zorn über dich schütten; ich will das Feuer meines Grimmes über
dich aufblasen und will dich Leuten, die brennen und verderben können,
überantworten. \bibverse{37} Du musst dem Feuer zur Speise werden, und
dein Blut muss im Lande vergossen werden, und man wird dein nicht mehr
gedenken; denn ich, der HErr, habe es geredet. \# 22 \bibverse{1} Und
des HErrn Wort geschah zu mir und sprach: \bibverse{2} Du Menschenkind,
willst du nicht strafen die mörderische Stadt und ihr anzeigen alle ihre
Gräuel? \footnote{\textbf{22:2} Hes 24,6} \bibverse{3} Sprich: So
spricht der Herr HErr: O Stadt, die du der Deinen Blut vergießest, auf
dass deine Zeit komme, und die du Götzen bei dir machst, dadurch du dich
verunreinigst! \bibverse{4} Du verschuldest dich an dem Blut, das du
vergießest, und verunreinigst dich an den Götzen, die du machst; damit
bringst du deine Tage herzu und machst, dass deine Jahre kommen müssen.
Darum will ich dich zum Spott unter den Heiden und zum Hohn in allen
Ländern machen. \bibverse{5} In der Nähe und in der Ferne sollen sie
dein spotten, dass du ein schändlich Gerücht haben und großen Jammer
leiden müssest. \bibverse{6} Siehe, die Fürsten in Israel, ein jeglicher
ist mächtig bei dir, Blut zu vergießen. \bibverse{7} Vater und Mutter
verachten sie, den Fremdlingen tun sie Gewalt und Unrecht, die Witwen
und die Waisen schinden sie. \bibverse{8} Du verachtest meine
Heiligtümer und entheiligst meine Sabbate. \bibverse{9} Verräter sind in
dir, auf dass sie Blut vergießen. Sie essen auf den Bergen und handeln
mutwillig in dir; \bibverse{10} sie decken auf die Blöße der Väter und
nötigen die Weiber in ihrer Krankheit \footnote{\textbf{22:10} 3Mo 18,7;
  3Mo 18,19} \bibverse{11} und treiben untereinander, Freund mit
Freundes Weibe, Gräuel; sie schänden ihre eigene Schwiegertochter mit
allem Mutwillen; sie notzüchtigen ihre eigenen Schwestern, ihres Vaters
Töchter; \footnote{\textbf{22:11} 3Mo 18,9; 3Mo 18,15; 3Mo 18,20}
\bibverse{12} sie nehmen Geschenke, auf dass sie Blut vergießen; sie
wuchern und nehmen Zins voneinander und treiben ihren Geiz wider ihren
Nächsten und tun einander Gewalt und vergessen mein also, spricht der
Herr HErr. \footnote{\textbf{22:12} 2Mo 22,24} \bibverse{13} Siehe, ich
schlage meine Hände zusammen über den Geiz, den du treibst, und über das
Blut, das in dir vergossen ist. \bibverse{14} Meinst du aber, dein Herz
möge es erleiden, oder werden es deine Hände ertragen zu der Zeit, wann
ich mit dir handeln werde? Ich, der HErr, habe es geredet und will's
auch tun \bibverse{15} und will dich zerstreuen unter die Heiden und
dich verstoßen in die Länder und will deinem Unflat ein Ende machen,
\bibverse{16} dass du bei den Heiden musst verflucht geachtet werden und
erfahren, dass ich der HErr sei. \bibverse{17} Und des HErrn Wort
geschah zu mir und sprach: \bibverse{18} Du Menschenkind, das Haus
Israel ist mir zu Schlacken geworden und sind alle Erz, Zinn, Eisen und
Blei im Ofen; ja, zu Silberschlacken sind sie geworden. \bibverse{19}
Darum spricht der Herr HErr also: Weil ihr denn alle Schlacken geworden
seid, siehe, so will ich euch alle gen Jerusalem zusammentun.
\bibverse{20} Wie man Silber, Erz, Eisen, Blei und Zinn zusammentut im
Ofen, dass man ein Feuer darunter aufblase und zerschmelze es, also will
ich euch auch in meinem Zorn und Grimm zusammentun, einlegen und
schmelzen. \bibverse{21} Ja, ich will euch sammeln und das Feuer meines
Zorns unter euch aufblasen, dass ihr drinnen zerschmelzen müsset.
\bibverse{22} Wie das Silber zerschmilzt im Ofen, so sollt ihr auch
darin zerschmelzen und erfahren, dass ich, der HErr, meinen Grimm über
euch ausgeschüttet habe. \bibverse{23} Und des HErrn Wort geschah zu mir
und sprach: \bibverse{24} Du Menschenkind, sprich zu ihnen: Du bist ein
Land, das nicht zu reinigen ist, wie eins, das nicht beregnet wird zur
Zeit des Zorns. \bibverse{25} Die Propheten, die darin sind, haben sich
gerottet, die Seelen zu fressen wie ein brüllender Löwe, wenn er raubt;
sie reißen Gut und Geld an sich und machen der Witwen viel darin.
\footnote{\textbf{22:25} Hes 34,3; Hes 34,8; Ps 14,4; Mt 23,14}
\bibverse{26} Ihre Priester verkehren mein Gesetz freventlich und
entheiligen mein Heiligtum; sie halten unter dem Heiligen und Unheiligen
keinen Unterschied und lehren nicht, was rein oder unrein sei, und
warten meiner Sabbate nicht, und ich werde unter ihnen entheiligt.
\footnote{\textbf{22:26} Zeph 3,4; Hes 44,23} \bibverse{27} Ihre Fürsten
sind darin wie die reißenden Wölfe, Blut zu vergießen und Seelen
umzubringen um ihres Geizes willen. \bibverse{28} Und ihre Propheten
tünchen ihnen mit losem Kalk, predigen loses Gerede und weissagen ihnen
Lügen und sagen: „So spricht der Herr HErr``, obwohl es doch der HErr
nicht geredet hat. \footnote{\textbf{22:28} Hes 13,6} \bibverse{29} Das
Volk im Lande übt Gewalt; rauben getrost und schinden die Armen und
Elenden und tun den Fremdlingen Gewalt und Unrecht. \footnote{\textbf{22:29}
  Hes 22,7} \bibverse{30} Ich suchte unter ihnen, ob jemand sich zur
Mauer machte und wider den Riss stünde vor mir für das Land, dass ich's
nicht verderbte; aber ich fand keinen. \footnote{\textbf{22:30} Hes 13,5}
\bibverse{31} Darum schüttete ich meinen Zorn über sie, und mit dem
Feuer meines Grimmes machte ich mit ihnen ein Ende und gab ihnen also
ihren Verdienst auf ihren Kopf, spricht der Herr HErr. \footnote{\textbf{22:31}
  Hes 21,36}

\hypertarget{section-7}{%
\section{23}\label{section-7}}

\bibverse{1} Und des HErrn Wort geschah zu mir und sprach: \bibverse{2}
Du Menschenkind, es waren zwei Weiber, einer Mutter Töchter.
\bibverse{3} Die trieben Hurerei in Ägypten in ihrer Jugend; daselbst
ließen sie ihre Brüste begreifen und den Busen ihrer Jungfrauschaft
betasten. \bibverse{4} Die große heißt Ohola und ihre Schwester Oholiba.
Und ich nahm sie zur Ehe, und sie gebaren mir Söhne und Töchter. Und
Ohola heißt Samaria und Oholiba Jerusalem. \bibverse{5} Ohola trieb
Hurerei, da ich sie genommen hatte, und brannte gegen ihre Buhlen,
nämlich gegen die Assyrer, die zu ihr kamen, \bibverse{6} gegen die
Fürsten und Herren, die mit Purpur gekleidet waren, und alle junge,
liebliche Gesellen, Reisige, die auf Rossen ritten. \bibverse{7} Und sie
buhlte mit allen schönen Gesellen in Assyrien und verunreinigte sich mit
allen ihren Götzen, wo sie auf einen entbrannte. \bibverse{8} Dazu
verließ sie auch nicht ihre Hurerei mit Ägypten, die bei ihr gelegen
hatten von ihrer Jugend auf und die Brüste ihrer Jungfrauschaft betastet
und große Hurerei mit ihr getrieben hatten. \bibverse{9} Da übergab ich
sie in die Hand ihrer Buhlen, den Kindern Assur, gegen welche sie
brannte vor Lust. \bibverse{10} Die deckten ihre Blöße auf und nahmen
ihre Söhne und Töchter weg; sie aber töteten sie mit dem Schwert. Und es
kam aus unter den Weibern, wie sie gestraft wäre. \footnote{\textbf{23:10}
  Hes 23,29} \bibverse{11} Da es aber ihre Schwester Oholiba sah,
entbrannte sie noch viel ärger denn jene und trieb die Hurerei mehr denn
ihre Schwester; \footnote{\textbf{23:11} Hes 16,51} \bibverse{12} und
entbrannte gegen die Kinder Assur, nämlich die Fürsten und Herren, die
zu ihr kamen wohl gekleidet, Reisige, die auf Rossen ritten, und alle
junge, liebliche Gesellen. \bibverse{13} Da sah ich, dass sie alle beide
gleicherweise verunreinigt waren. \bibverse{14} Aber diese trieb ihre
Hurerei mehr. Denn da sie sah gemalte Männer an der Wand in roter Farbe,
die Bilder der Chaldäer, \bibverse{15} um ihre Lenden gegürtet und bunte
Mützen auf ihren Köpfen, und alle gleich anzusehen wie gewaltige Leute,
wie denn die Kinder Babels, die Chaldäer, tragen in ihrem Vaterlande:
\bibverse{16} entbrannte sie gegen sie, sobald sie ihrer gewahr ward,
und schickte Botschaft zu ihnen nach Chaldäa. \bibverse{17} Als nun die
Kinder Babels zu ihr kamen, bei ihr zu schlafen nach der Liebe,
verunreinigten sie dieselbe mit ihrer Hurerei, und sie verunreinigte
sich mit ihnen, bis sie ihrer müde ward. \bibverse{18} Und da ihre
Hurerei und Schande so gar offenbar war, ward ich ihrer auch
überdrüssig, wie ich ihrer Schwester auch war müde geworden.
\bibverse{19} Sie aber trieb ihre Hurerei immer mehr und gedachte an die
Zeit ihrer Jugend, da sie in Ägyptenland Hurerei getrieben hatte,
\bibverse{20} und entbrannte gegen ihre Buhlen, welcher Brunst war wie
der Esel und der Hengste Brunst. \bibverse{21} Und du bestelltest deine
Unzucht wie in deiner Jugend, da die in Ägypten deine Brüste begriffen
und deinen Busen betasteten. \bibverse{22} Darum, Oholiba, so spricht
der Herr HErr: Siehe, ich will deine Buhlen, deren du müde bist
geworden, wider dich erwecken und will sie ringsumher wider dich
bringen, \bibverse{23} nämlich die Kinder Babels und alle Chaldäer mit
Hauptleuten, Fürsten und Herren und alle Assyrer mit ihnen, die schöne
junge Mannschaft, alle Fürsten und Herren, Ritter und Edle, die alle auf
Rossen reiten. \bibverse{24} Und sie werden über dich kommen, gerüstet
mit Wagen und Rädern und mit großem Haufen Volks, und werden dich
belagern mit Tartschen, Schilden und Helmen um und um. Denen will ich
das Recht befehlen, dass sie dich richten sollen nach ihrem Recht.
\footnote{\textbf{23:24} Lk 19,43} \bibverse{25} Ich will meinen Eifer
über dich gehen lassen, dass sie unbarmherzig mit dir handeln sollen.
Sie sollen dir Nase und Ohren abschneiden; und was übrigbleibt, soll
durchs Schwert fallen. Sie sollen deine Söhne und Töchter wegnehmen und
das Übrige mit Feuer verbrennen. \bibverse{26} Sie sollen dir deine
Kleider ausziehen und deinen Schmuck wegnehmen. \bibverse{27} Also will
ich deiner Unzucht und deiner Hurerei mit Ägyptenland ein Ende machen,
dass du deine Augen nicht mehr nach ihnen aufheben und Ägyptens nicht
mehr gedenken sollst. \bibverse{28} Denn so spricht der Herr HErr:
Siehe, ich will dich überantworten, denen du feind geworden und deren du
müde bist. \bibverse{29} Die sollen wie Feinde mit dir umgehen und alles
nehmen, was du erworben hast, und dich nackt und bloß lassen, dass die
Schande deiner Unzucht und Hurerei offenbar werde. \bibverse{30} Solches
wird dir geschehen um deiner Hurerei willen, die du mit den Heiden
getrieben, an deren Götzen du dich verunreinigt hast. \bibverse{31} Du
bist auf dem Wege deiner Schwester gegangen; darum gebe ich dir auch
deren Kelch in deine Hand. \bibverse{32} So spricht der Herr HErr: Du
musst den Kelch deiner Schwester trinken, so tief und weit er ist; du
sollst zu so großem Spott und Hohn werden, dass es unerträglich sein
wird. \bibverse{33} Du musst dich des starken Tranks und Jammers
vollsaufen; denn der Kelch deiner Schwester Samaria ist ein Kelch des
Jammers und Trauerns. \footnote{\textbf{23:33} Jes 51,17; Jer 25,15; Jer
  25,18} \bibverse{34} Denselben musst du rein austrinken, darnach die
Scherben zerwerfen und deine Brüste zerreißen; denn ich habe es geredet,
spricht der Herr HErr. \bibverse{35} Darum so spricht der Herr HErr:
Darum, dass du mein vergessen und mich hinter deinen Rücken geworfen
hast, so trage auch nun deine Unzucht und deine Hurerei. \bibverse{36}
Und der HErr sprach zu mir: Du Menschenkind, willst du nicht Ohola und
Oholiba strafen und ihnen zeigen ihre Gräuel? \bibverse{37} Wie sie
Ehebrecherei getrieben und Blut vergossen und die Ehe gebrochen haben
mit den Götzen; dazu ihre Kinder, die sie mir geboren hatten,
verbrannten sie denselben zum Opfer. \bibverse{38} Überdas haben sie mir
das getan: sie haben meine Heiligtümer verunreinigt dazumal und meine
Sabbate entheiligt. \bibverse{39} Denn da sie ihre Kinder den Götzen
geschlachtet hatten, gingen sie desselben Tages in mein Heiligtum, es zu
entheiligen. Siehe, solches haben sie in meinem Hause begangen.
\bibverse{40} Sie haben auch Boten geschickt nach Leuten, die aus fernen
Landen kommen sollten; und siehe, da sie kamen, badetest du dich und
schminktest dich und schmücktest dich mit Geschmeide ihnen zu Ehren
\bibverse{41} und saßest auf einem herrlichen Polster, vor welchem stand
ein Tisch zugerichtet; darauf legtest du mein Räuchwerk und mein Öl.
\bibverse{42} Daselbst erhob sich ein großes Freudengeschrei; und es
gaben ihnen die Leute, so allenthalben aus großem Volk und aus der Wüste
gekommen waren, Geschmeide an ihre Arme und schöne Kronen auf ihre
Häupter. \bibverse{43} Ich aber gedachte: Sie ist der Ehebrecherei
gewohnt von alters her; sie kann von der Hurerei nicht lassen.
\bibverse{44} Denn man geht zu ihr ein, wie man zu einer Hure eingeht;
ebenso geht man zu Ohola und Oholiba, den unzüchtigen Weibern.
\bibverse{45} Darum werden sie die Männer strafen, die das Recht
vollbringen, wie man die Ehebrecherinnen und Blutvergießerinnen strafen
soll. Denn sie sind Ehebrecherinnen, und ihre Hände sind voll Blut.
\footnote{\textbf{23:45} 3Mo 20,10} \bibverse{46} Also spricht der Herr
HErr: Führe einen großen Haufen über sie herauf und gib sie zu Raub und
Beute, \bibverse{47} dass die Leute sie steinigen und mit ihren
Schwertern erstechen und ihre Söhne und Töchter erwürgen und ihre Häuser
mit Feuer verbrennen. \bibverse{48} Also will ich der Unzucht im Lande
ein Ende machen, dass alle Weiber sich warnen lassen und nicht nach
solcher Unzucht tun.

\bibverse{49} Und man soll eure Unzucht auf euch legen, und ihr sollt
eurer Götzen Sünden tragen, auf dass ihr erfahret, dass ich der Herr
HErr bin. \# 24 \bibverse{1} Und es geschah das Wort des HErrn zu mir im
neunten Jahr, am zehnten Tage des zehnten Monats, und sprach:
\bibverse{2} Du Menschenkind, schreib diesen Tag an, ja, eben diesen
Tag; denn der König zu Babel hat sich eben an diesem Tage wider
Jerusalem gelagert. \bibverse{3} Und gib dem ungehorsamen Volk ein
Gleichnis und sprich zu ihnen: So spricht der Herr HErr: Setze einen
Topf zu, setze zu und gieß Wasser hinein; \bibverse{4} tue die Stücke
zusammen darein, die hinein sollen, alle besten Stücke, die Lenden und
Schultern, und fülle ihn mit den besten Knochenstücken; \bibverse{5}
nimm das Beste von der Herde und mache ein Feuer darunter, Knochenstücke
zu kochen, und lass es getrost sieden und die Knochenstücke darin wohl
kochen. \bibverse{6} Darum spricht der Herr HErr: O der mörderischen
Stadt, die ein solcher Topf ist, da der Rost daran klebt und nicht
abgehen will! Tue ein Stück nach dem anderen heraus; und darfst nicht
darum losen, welches zuerst heraus soll. \footnote{\textbf{24:6} Hes
  24,9} \bibverse{7} Denn ihr Blut ist darin, das sie auf einen bloßen
Felsen und nicht auf die Erde verschüttet hat, da man's doch hätte mit
Erde können zuscharren.

\bibverse{8} Und ich habe auch darum sie lassen das Blut auf einen
bloßen Felsen schütten, dass es nicht zugescharrt würde, auf dass der
Grimm über sie käme und es gerächt würde. \bibverse{9} Darum spricht der
Herr HErr also: O du mörderische Stadt, welche ich will zu einem großen
Feuer machen! \bibverse{10} Trage nur viel Holz her, zünde das Feuer an,
dass das Fleisch gar werde, und würze es wohl, und die Knochenstücke
sollen anbrennen. \bibverse{11} Lege auch den Topf leer auf die Glut,
auf dass er heiß werde und sein Erz entbrenne, ob seine Unreinigkeit
zerschmelzen und sein Rost abgehen wolle. \bibverse{12} Aber wie sehr er
brennt, will sein Rost doch nicht abgehen, denn es ist zuviel des Rosts;
er muss im Feuer verschmelzen. \bibverse{13} Deine Unreinigkeit ist so
verhärtet, dass, ob ich dich gleich gern reinigen wollte, dennoch du
nicht willst dich reinigen lassen von deiner Unreinigkeit. Darum kannst
du hinfort nicht wieder rein werden, bis mein Grimm sich an dir gekühlt
habe. \footnote{\textbf{24:13} Hes 5,13} \bibverse{14} Ich, der HErr,
habe es geredet! Es soll kommen, ich will's tun und nicht säumen; ich
will nicht schonen noch mich's reuen lassen; sondern sie sollen dich
richten, wie du gelebt und getan hast, spricht der Herr HErr.
\bibverse{15} Und des HErrn Wort geschah zu mir und sprach:
\bibverse{16} Du Menschenkind, siehe, ich will dir deiner Augen Lust
nehmen durch eine Plage. Aber du sollst nicht klagen noch weinen noch
eine Träne lassen. \bibverse{17} Heimlich magst du seufzen, aber keine
Totenklage führen; sondern du sollst deinen Schmuck anlegen und deine
Schuhe anziehen. Du sollst deinen Mund nicht verhüllen und nicht das
Trauerbrot essen.

\bibverse{18} Und da ich des Morgens früh zum Volke geredet hatte, starb
mir am Abend mein Weib. Und ich tat des anderen Morgens, wie mir
befohlen war. \bibverse{19} Und das Volk sprach zu mir: Willst du uns
denn nicht anzeigen, was uns das bedeutet, was du tust? \bibverse{20}
Und ich sprach zu ihnen: Der HErr hat mit mir geredet und gesagt:
\bibverse{21} Sage dem Hause Israel, dass der Herr HErr spricht also:
Siehe, ich will mein Heiligtum, euren höchsten Trost, die Lust eurer
Augen und eures Herzens Wunsch, entheiligen; und eure Söhne und Töchter,
die ihr verlassen musstet, werden durchs Schwert fallen. \bibverse{22}
Und müsset tun, wie ich getan habe: euren Mund sollt ihr nicht verhüllen
und das Trauerbrot nicht essen, \bibverse{23} sondern sollt euren
Schmuck auf euer Haupt setzen und eure Schuhe anziehen. Ihr werdet nicht
klagen noch weinen, sondern über euren Sünden verschmachten und
untereinander seufzen.

\bibverse{24} Und soll also Hesekiel euch ein Wunderzeichen sein, dass
ihr tun müsset, wie er getan hat, wenn es nun kommen wird, damit ihr
erfahret, dass ich der Herr HErr bin. \footnote{\textbf{24:24} Hes
  24,27; Hes 12,11}

\bibverse{25} Und du, Menschenkind, zu der Zeit, wann ich wegnehmen
werde von ihnen ihre Macht und ihren Trost, die Lust ihrer Augen und
ihres Herzens Wunsch, ihre Söhne und Töchter, \bibverse{26} ja, zur
selben Zeit wird einer, der entronnen ist, zu dir kommen und dir's
kundtun. \bibverse{27} Zur selben Zeit wird dein Mund aufgetan werden
samt dem, der entronnen ist, dass du reden sollst und nicht mehr
schweigen; denn du musst ihr Wunderzeichen sein, dass sie erfahren, ich
sei der HErr. \# 25 \bibverse{1} Und des HErrn Wort geschah zu mir und
sprach: \bibverse{2} Du Menschenkind, richte dein Angesicht gegen die
Kinder Ammon und weissage wider sie \footnote{\textbf{25:2} Hes
  21,33-37; Jer 49,1-6} \bibverse{3} und sprich zu den Kindern Ammon:
Höret des Herrn HErrn Wort! So spricht der Herr HErr: Darum dass ihr
über mein Heiligtum sprecht: „Ha! es ist entheiligt!{}`` und über das
Land Israel: „Es ist verwüstet!{}`` und über das Haus Juda: „Es ist
gefangen weggeführt!{}``, \footnote{\textbf{25:3} Hes 36,2; Kla 2,16}
\bibverse{4} darum siehe, ich will dich den Kindern des Morgenlandes
übergeben, dass sie ihre Zeltdörfer in dir bauen und ihre Wohnungen in
dir machen sollen; sie sollen deine Früchte essen und deine Milch
trinken. \bibverse{5} Und will Rabba zum Kamelstall machen und das Land
der Kinder Ammon zu Schafhürden machen; und ihr sollt erfahren, dass ich
der HErr bin.

\bibverse{6} Denn so spricht der Herr HErr: Darum dass du mit deinen
Händen geklatscht und mit den Füßen gescharrt und über das Land Israel
von ganzem Herzen so höhnisch dich gefreut hast, \bibverse{7} darum
siehe, ich will meine Hand über dich ausstrecken und dich den Heiden zur
Beute geben und dich aus den Völkern ausrotten und aus den Ländern
umbringen und dich vertilgen; und sollst erfahren, dass ich der HErr
bin. \bibverse{8} So spricht der Herr HErr: Darum dass Moab und Seir
sprechen: Siehe, das Haus Juda ist eben wie alle Heiden! \footnote{\textbf{25:8}
  Jes 15,-1; Jer 48,-1} \bibverse{9} siehe, so will ich Moab zur Seite
öffnen in seinen Städten und in seinen Grenzen, das edle Land von
Beth-Jesimoth, Baal-Meon und Kirjathaim, \bibverse{10} und will es den
Kindern des Morgenlandes zum Erbe geben samt dem Lande der Kinder Ammon,
dass man der Kinder Ammon nicht mehr gedenken soll unter den Heiden.
\bibverse{11} Und will das Recht gehen lassen über Moab; und sie sollen
erfahren, dass ich der HErr bin. \bibverse{12} So spricht der Herr HErr:
Darum dass sich Edom am Hause Juda gerächt hat und sich verschuldet mit
seinem Rächen,

\bibverse{13} darum spricht der Herr HErr also: Ich will meine Hand
ausstrecken über Edom und will ausrotten von ihm Menschen und Vieh und
will es wüst machen von Theman bis gen Dedan und durchs Schwert fällen;
\bibverse{14} und will mich an Edom rächen durch mein Volk Israel, und
sie sollen mit Edom umgehen nach meinem Zorn und Grimm, dass sie meine
Rache erfahren sollen, spricht der Herr HErr. \bibverse{15} So spricht
der Herr HErr: Darum dass die Philister sich gerächt haben und den alten
Hass gebüßt nach allem ihrem Willen am Schaden meines Volks, \footnote{\textbf{25:15}
  Jes 14,29; Jer 47,-1; Zeph 2,5} \bibverse{16} darum spricht der Herr
HErr also: Siehe, ich will meine Hand ausstrecken über die Philister und
die Kreter ausrotten und will die Übrigen am Ufer des Meeres umbringen;
\footnote{\textbf{25:16} 1Sam 30,14}

\bibverse{17} und will große Rache an ihnen üben und mit Grimm sie
strafen, dass sie erfahren sollen, ich sei der HErr, wenn ich meine
Rache an ihnen geübt habe. \# 26 \bibverse{1} Und es begab sich im
elften Jahr, am ersten Tage des ersten Monats, geschah des HErrn Wort zu
mir und sprach: \bibverse{2} Du Menschenkind, darum dass Tyrus spricht
über Jerusalem: „Ha! die Pforte der Völker ist zerbrochen; es ist zu mir
gewandt; ich werde nun voll werden, weil sie wüst ist!{}``, \footnote{\textbf{26:2}
  Hes 25,3} \bibverse{3} darum spricht der Herr HErr also: Siehe, ich
will an dich, Tyrus, und will viele Heiden über dich heraufbringen,
gleich wie sich ein Meer erhebt mit seinen Wellen. \footnote{\textbf{26:3}
  Jes 23,-1} \bibverse{4} Die sollen die Mauern zu Tyrus verderben und
ihre Türme abbrechen; ja ich will auch ihren Staub von ihr wegfegen und
will einen bloßen Fels aus ihr machen \bibverse{5} und einen Ort am
Meer, darauf man die Fischgarne aufspannt; denn ich habe es geredet,
spricht der Herr HErr, und sie soll den Heiden zum Raub werden.
\bibverse{6} Und ihre Töchter, die auf dem Felde liegen, sollen durchs
Schwert erwürgt werden und sollen erfahren, dass ich der HErr bin.
\bibverse{7} Denn so spricht der Herr HErr: Siehe, ich will über Tyrus
kommen lassen Nebukadnezar, den König zu Babel, von Mitternacht her, der
ein König aller Könige ist, mit Rossen, Wagen, Reitern und mit großem
Haufen Volks. \footnote{\textbf{26:7} Dan 2,37} \bibverse{8} Der soll
deine Töchter, die auf dem Felde liegen, mit dem Schwert erwürgen; aber
wider dich wird er Bollwerke aufschlagen und einen Wall aufschütten und
Schilde wider dich rüsten. \footnote{\textbf{26:8} Hes 26,6}
\bibverse{9} Er wird mit Sturmböcken deine Mauern zerstoßen und deine
Türme mit seinen Werkzeugen umreißen. \bibverse{10} Der Staub von der
Menge seiner Pferde wird dich bedecken; so werden auch deine Mauern
erbeben vor dem Getümmel seiner Rosse, Räder und Reiter, wenn er zu
deinen Toren einziehen wird, wie man pflegt in eine zerrissene Stadt
einzuziehen. \bibverse{11} Er wird mit den Füßen seiner Rosse alle deine
Gassen zertreten. Dein Volk wird er mit dem Schwert erwürgen und deine
starken Säulen zu Boden reißen. \bibverse{12} Sie werden dein Gut rauben
und deinen Handel plündern. Deine Mauern werden sie abbrechen und deine
feinen Häuser umreißen und werden deine Steine, Holz und Staub ins
Wasser werfen. \bibverse{13} Also will ich mit dem Getön deines Gesanges
ein Ende machen, dass man den Klang deiner Harfen nicht mehr hören soll.
\footnote{\textbf{26:13} Jes 14,11} \bibverse{14} Und ich will einen
bloßen Fels aus dir machen und einen Ort, darauf man die Fischgarne
aufspannt, dass du nicht mehr gebaut werdest; denn ich bin der HErr, der
solches redet, spricht der Herr HErr. \bibverse{15} So spricht der Herr
HErr wider Tyrus: Was gilt's? die Inseln werden erbeben, wenn du so
gräulich zerfallen wirst und deine Verwundeten seufzen werden, die in
dir sollen ermordet werden. \bibverse{16} Alle Fürsten am Meer werden
herab von ihren Stühlen steigen und ihre Röcke von sich tun und ihre
gestickten Kleider ausziehen und werden in Trauerkleidern gehen und auf
der Erde sitzen und werden erschrecken und sich entsetzen über deinen
plötzlichen Fall. \bibverse{17} Sie werden über dich wehklagen und von
dir sagen: Ach, wie bist du so gar wüst geworden, du berühmte Stadt, die
du am Meer lagst und so mächtig warst auf dem Meer samt deinen
Einwohnern, dass sich das ganze Land vor dir fürchten musste!
\bibverse{18} Ach, wie entsetzen sich die Inseln über deinen Fall! ja
die Inseln im Meer erschrecken über deinen Untergang. \bibverse{19} Denn
so spricht der Herr HErr: Ich will dich zu einer wüsten Stadt machen wie
andere Städte, darin niemand wohnt, und eine große Flut über dich kommen
lassen, dass dich große Wasser bedecken, \bibverse{20} und will dich
hinunterstoßen zu denen, die in die Grube gefahren sind, zu dem Volk der
Toten. Ich will dich unter die Erde hinabstoßen in die ewigen Wüsten zu
denen, die in die Grube gefahren sind, auf dass niemand in dir wohne.
Ich will dich, du Prächtige im Lande der Lebendigen, \bibverse{21} ja,
zum Schrecken will ich dich machen, dass du nichts mehr seist; und wenn
man nach dir fragt, dass man dich ewiglich nimmer finden könne, spricht
der Herr HErr. \# 27 \bibverse{1} Und des HErrn Wort geschah zu mir und
sprach: \bibverse{2} Du Menschenkind, mache eine Wehklage über Tyrus
\bibverse{3} und sprich zu Tyrus, die da liegt vorn am Meer und mit
vielen Inseln der Völker handelt: So spricht der Herr HErr: O Tyrus, du
sprichst: Ich bin die Allerschönste. \footnote{\textbf{27:3} Hos 9,13}
\bibverse{4} Deine Grenzen sind mitten im Meer und deine Bauleute haben
dich aufs allerschönste zugerichtet. \bibverse{5} Sie haben all dein
Tafelwerk aus Zypressenholz vom Senir gemacht und die Zedern von dem
Libanon führen lassen und deine Mastbäume daraus gemacht \bibverse{6}
und deine Ruder von Eichen aus Basan und deine Bänke von Elfenbein,
gefasst in Buchsbaumholz aus den Inseln der Chittiter. \bibverse{7} Dein
Segel war von gestickter, köstlicher Leinwand aus Ägypten, dass es dein
Panier wäre, und deine Decken von blauem und rotem Purpur aus den Inseln
Elisa. \footnote{\textbf{27:7} 1Mo 10,4} \bibverse{8} Die von Sidon und
Arvad waren deine Ruderknechte, und hattest geschickte Leute zu Tyrus,
zu schiffen. \bibverse{9} Die Ältesten und Klugen von Gebal mussten
deine Risse bessern. Alle Schiffe im Meer und ihre Schiffsleute fand man
bei dir; die hatten ihren Handel in dir. \bibverse{10} Die aus Persien,
Lud und Lybien waren dein Kriegsvolk, die ihre Schilde und Helme in dir
aufhingen und haben dich so schön geschmückt. \bibverse{11} Die von
Arvad waren unter deinem Heer rings um deine Mauern und Wächter auf
deinen Türmen; die haben ihre Schilde allenthalben von deinen Mauern
herabgehängt und dich so schön geschmückt. \bibverse{12} Tharsis hat mit
dir seinem Handel gehabt und allerlei Ware, Silber, Eisen, Zinn und Blei
auf deine Märkte gebracht. \bibverse{13} Javan, Thubal und Mesech haben
mit dir gehandelt und haben dir leibeigene Leute und Geräte von Erz auf
deine Märkte gebracht. \footnote{\textbf{27:13} Hes 38,2} \bibverse{14}
Die von Thogarma haben dir Rosse und Wagenpferde und Maulesel auf deine
Märkte gebracht. \bibverse{15} Die von Dedan sind deine Händler gewesen,
und hast allenthalben in den Inseln gehandelt; die haben dir Elfenbein
und Ebenholz verkauft. \bibverse{16} Die Syrer haben bei dir geholt
deine Arbeit, was du gemacht hast, und Rubine, Purpur, Teppiche, feine
Leinwand und Korallen und Kristalle auf deine Märkte gebracht.
\bibverse{17} Juda und das Land Israel haben auch mit dir gehandelt und
haben dir Weizen von Minnith und Balsam und Honig und Öl und Mastix auf
deine Märkte gebracht. \bibverse{18} Dazu hat auch Damaskus bei dir
geholt deine Arbeit und allerlei Ware um Wein von Helbon und köstliche
Wolle. \bibverse{19} Dan und Javan und Mehusal haben auch auf deine
Märkte gebracht Eisenwerk, Kassia und Kalmus, dass du damit handeltest.
\bibverse{20} Dedan hat mit dir gehandelt mit Decken zum Reiten.
\bibverse{21} Arabien und alle Fürsten von Kedar haben mit dir gehandelt
mit Schafen, Widdern und Böcken. \bibverse{22} Die Kaufleute aus Saba
und Ragma haben mit dir gehandelt und allerlei köstliche Spezerei und
Edelsteine und Gold auf deine Märkte gebracht.

\bibverse{23} Haran und Kanne und Eden samt den Kaufleuten aus Seba,
Assur und Kilmad sind auch deine Händler gewesen. \bibverse{24} Die
haben alle mit dir gehandelt mit köstlichem Gewand, mit purpurnen und
gestickten Tüchern, welche sie in köstlichen Kasten, von Zedern gemacht
und wohl verwahrt, auf deine Märkte geführt haben. \bibverse{25} Aber
die Tharsisschiffe sind die vornehmsten auf deinen Märkten gewesen. Also
bist du sehr reich und prächtig geworden mitten im Meer. \bibverse{26}
Deine Ruderer haben dich auf große Wasser geführt; ein Ostwind wird dich
mitten auf dem Meer zerbrechen, \bibverse{27} also dass dein Reichtum,
dein Kaufgut, deine Ware, deine Schiffsleute, deine Schiffsherren und
die, die deine Risse bessern und die deinen Handel treiben und alle
deine Kriegsleute und alles Volk in dir mitten auf dem Meer umkommen
werden zur Zeit, wann du untergehst; \bibverse{28} dass auch die
Anfurten erbeben werden vor dem Geschrei deiner Schiffsherren.
\bibverse{29} Und alle, die an den Rudern ziehen, samt den
Schiffsknechten und Meistern werden aus ihren Schiffen ans Land treten

\bibverse{30} und laut über dich schreien, bitterlich klagen und werden
Staub auf ihre Häupter werfen und sich in der Asche wälzen.

\bibverse{31} Sie werden sich kahl scheren über dir und Säcke um sich
gürten und von Herzen bitterlich um dich weinen und trauern.

\bibverse{32} Es werden auch ihre Kinder über dich wehklagen: Ach! wer
ist jemals auf dem Meer so still geworden wie du, Tyrus?

\bibverse{33} Da du deinen Handel auf dem Meer triebst, da machtest du
viele Länder reich, ja, mit der Menge deiner Ware und deiner
Kaufmannschaft machtest du reich die Könige auf Erden.

\bibverse{34} Nun aber bist du vom Meer in die rechten, tiefen Wasser
gestürzt, dass dein Handel und all dein Volk in dir umgekommen ist.

\bibverse{35} Alle die auf den Inseln wohnen, erschrecken über dich, und
ihre Könige entsetzen sich und sehen jämmerlich.

\bibverse{36} Die Kaufleute in den Ländern pfeifen dich an, dass du so
plötzlich untergegangen bist und nicht mehr aufkommen kannst.
\footnote{\textbf{27:36} Hes 28,19}

\hypertarget{section-8}{%
\section{28}\label{section-8}}

\bibverse{1} Und des HErrn Wort geschah zu mir und sprach: \bibverse{2}
Du Menschenkind, sage dem Fürsten zu Tyrus: So spricht der Herr HErr:
Darum dass sich dein Herz erhebt und spricht: „Ich bin Gott, ich sitze
auf dem Thron Gottes mitten im Meer``, so du doch ein Mensch und nicht
Gott bist -- doch erhebt sich dein Herz, als wäre es eines Gottes Herz:
\bibverse{3} siehe, du hältst dich für klüger denn Daniel, dass dir
nichts verborgen sei \footnote{\textbf{28:3} Hes 14,14} \bibverse{4} und
habest durch deine Klugheit und deinen Verstand solche Macht zuwege
gebracht und Schätze von Gold und Silber gesammelt \bibverse{5} und
habest durch deine große Weisheit und Hantierung so große Macht
überkommen; davon bist du so stolz geworden, dass du so mächtig bist --;
\bibverse{6} darum spricht der Herr HErr also: Weil sich denn dein Herz
erhebt, als wäre es eines Gottes Herz, \bibverse{7} darum, siehe, ich
will Fremde über dich schicken, nämlich die Tyrannen der Heiden; die
sollen ihr Schwert zücken über deine schöne Weisheit und deine große
Ehre zu Schanden machen. \bibverse{8} Sie sollen dich hinunter in die
Grube stoßen, dass du mitten auf dem Meer sterbest wie die Erschlagenen.
\bibverse{9} Was gilt's, ob du dann vor deinem Totschläger werdest
sagen: „Ich bin Gott``, so du doch nicht Gott, sondern ein Mensch und in
deiner Totschläger Hand bist? \footnote{\textbf{28:9} Hes 28,2}
\bibverse{10} Du sollst sterben wie die Unbeschnittenen von der Hand der
Fremden; denn ich habe es geredet, spricht der Herr HErr. \bibverse{11}
Und des HErrn Wort geschah zu mir und sprach: \bibverse{12} Du
Menschenkind, mache eine Wehklage über den König zu Tyrus und sprich von
ihm: So spricht der Herr HErr: Du bist ein reinliches Siegel, voller
Weisheit und aus der Maßen schön. \bibverse{13} Du bist im Lustgarten
Gottes und mit allerlei Edelsteinen geschmückt: mit Sarder, Topas,
Demant, Türkis, Onyx, Jaspis, Saphir, Amethyst, Smaragd und Gold. Am
Tage, da du geschaffen wurdest, mussten da bereitet sein bei dir deine
Pauken und Pfeifen. \bibverse{14} Du bist wie ein Cherub, der sich weit
ausbreitet und decket; und ich habe dich auf den heiligen Berg Gottes
gesetzt, dass du unter den feurigen Steinen wandelst. \footnote{\textbf{28:14}
  Jes 14,14} \bibverse{15} Du warst ohne Tadel in deinem Tun von dem
Tage an, da du geschaffen wurdest, bis sich deine Missetat gefunden hat.
\bibverse{16} Denn du bist inwendig voll Frevels geworden vor deiner
großen Hantierung und hast dich versündigt. Darum will ich dich
entheiligen von dem Berge Gottes und will dich ausgebreiteten Cherub aus
den feurigen Steinen verstoßen. \bibverse{17} Und weil sich dein Herz
erhebt, dass du so schön bist, und hast dich deine Klugheit lassen
betrügen in deiner Pracht, darum will ich dich zu Boden stürzen und ein
Schauspiel aus dir machen vor den Königen. \bibverse{18} Denn du hast
dein Heiligtum verderbt mit deiner großen Missetat und unrechtem Handel.
Darum will ich ein Feuer aus dir angehen lassen, das dich soll
verzehren, und will dich zu Asche machen auf der Erde, dass alle Welt
zusehen soll. \bibverse{19} Alle, die dich kennen unter den Heiden,
werden sich über dich entsetzen, dass du so plötzlich bist untergegangen
und nimmermehr aufkommen kannst. \bibverse{20} Und des HErrn Wort
geschah zu mir und sprach: \bibverse{21} Du Menschenkind, richte dein
Angesicht wider Sidon und weissage wider sie \footnote{\textbf{28:21}
  Jes 23,2; Jes 23,12} \bibverse{22} und sprich: So spricht der Herr
HErr: Siehe, ich will an dich, Sidon, und will an dir Ehre einlegen,
dass man erfahren soll, dass ich der HErr bin, wenn ich das Recht über
sie gehen lasse und an ihr erzeige, dass ich heilig sei. \footnote{\textbf{28:22}
  2Mo 14,18} \bibverse{23} Und ich will Pestilenz und Blutvergießen
unter sie schicken auf ihren Gassen, und sie sollen tödlich verwundet
drinnen fallen durchs Schwert, welches allenthalben über sie gehen wird;
und sollen erfahren, dass ich der HErr bin. \bibverse{24} Und forthin
sollen allenthalben um das Haus Israel, da ihre Feinde sind, keine
Dornen, die da stechen, noch Stacheln, die da wehe tun, bleiben, dass
sie erfahren, dass ich der Herr HErr bin. \bibverse{25} So spricht der
Herr HErr: Wenn ich das Haus Israel wieder versammeln werde von den
Völkern, dahin sie zerstreut sind, so will ich vor den Heiden an ihnen
erzeigen, dass ich heilig bin. Und sie sollen wohnen in ihrem Lande, das
ich meinem Knecht Jakob gegeben habe; \bibverse{26} und sollen sicher
darin wohnen und Häuser bauen und Weinberge pflanzen; ja, sicher sollen
sie wohnen, wenn ich das Recht gehen lasse über alle ihre Feinde um und
um; und sollen erfahren, dass ich, der HErr, ihr Gott bin. \# 29
\bibverse{1} Im zehnten Jahr, am zwölften Tage des zehnten Monats,
geschah des HErrn Wort zu mir und sprach: \bibverse{2} Du Menschenkind,
richte dein Angesicht wider Pharao, den König in Ägypten, und weissage
wider ihn und wider ganz Ägyptenland. \footnote{\textbf{29:2} Jes 19,-1;
  Jer 46,-1} \bibverse{3} Predige und sprich: So spricht der Herr HErr:
Siehe, ich will an dich, Pharao, du König in Ägypten, du großer Drache,
der du in deinem Wasser liegst und sprichst: Der Strom ist mein, und ich
habe ihn mir gemacht. \footnote{\textbf{29:3} Hes 32,2} \bibverse{4}
Aber ich will dir ein Gebiss ins Maul legen und die Fische in deinen
Wassern an deine Schuppen hängen und will dich aus deinem Strom
herausziehen samt allen Fischen in deinen Wassern, die an deinen
Schuppen hangen. \footnote{\textbf{29:4} Hes 38,4; 2Kö 19,28}
\bibverse{5} Ich will dich mit den Fischen aus deinen Wassern in die
Wüste wegwerfen; du wirst aufs Land fallen und nicht wieder aufgelesen
noch gesammelt werden, sondern den Tieren auf dem Lande und den Vögeln
des Himmels zur Speise werden. \bibverse{6} Und alle, die in Ägypten
wohnen, sollen erfahren, dass ich der HErr bin; darum dass sie dem Hause
Israel ein Rohrstab gewesen sind. \bibverse{7} Wenn sie ihn in die Hand
fassten, so brach er und stach sie durch die Seite; wenn sie sich aber
darauf lehnten, so zerbrach er und stach sie in die Lenden. \bibverse{8}
Darum spricht der Herr HErr also: Siehe, ich will das Schwert über dich
kommen lassen und Leute und Vieh in dir ausrotten. \bibverse{9} Und
Ägyptenland soll zur Wüste und Öde werden, und sie sollen erfahren, dass
ich der HErr sei, darum dass du sprichst: Der Wasserstrom ist mein, und
ich bin's, der's tut. \bibverse{10} Darum, siehe, ich will an dich und
an deine Wasserströme und will Ägyptenland wüst und öde machen von
Migdol bis gen Syene und bis an die Grenze des Mohrenlands,
\bibverse{11} dass weder Vieh noch Leute darin gehen oder da wohnen
sollen vierzig Jahre lang. \bibverse{12} Denn ich will Ägyptenland wüst
machen wie andere wüste Länder und ihre Städte wüst liegen lassen wie
andere wüste Städte vierzig Jahre lang; und will die Ägypter zerstreuen
unter die Heiden, und in die Länder will ich sie verjagen. \bibverse{13}
Doch so spricht der Herr HErr: Wenn die vierzig Jahre aus sein werden,
will ich die Ägypter wieder sammeln aus den Völkern, darunter sie
zerstreut sollen werden, \bibverse{14} und will das Gefängnis Ägyptens
wenden und sie wiederum ins Land Pathros bringen, welches ihr Vaterland
ist; und sie sollen daselbst ein kleines Königreich sein. \bibverse{15}
Denn sie sollen klein sein gegen andere Königreiche und nicht mehr sich
erheben über die Heiden; und ich will sie gering machen, damit sie nicht
über die Heiden herrschen sollen, \bibverse{16} dass sich das Haus
Israel nicht mehr auf sie verlasse und sich damit versündige, wenn sie
sich an sie hängen; und sie sollen erfahren, dass ich der Herr HErr bin.
\bibverse{17} Und es begab sich im siebenundzwanzigsten Jahr, am ersten
Tage des ersten Monats, geschah des HErrn Wort zu mir und sprach:
\bibverse{18} Du Menschenkind, Nebukadnezar, der König zu Babel, hat
sein Heer mit großer Mühe vor Tyrus arbeiten lassen, dass alle Häupter
kahl und alle Schultern wund gerieben waren; und ist doch weder ihm noch
seinem Heer seine Arbeit vor Tyrus belohnt worden. \bibverse{19} Darum
spricht der Herr HErr also: Siehe, ich will Nebukadnezar, dem König zu
Babel, Ägyptenland geben, dass er all ihr Gut wegnehmen und sie berauben
und plündern soll, dass er seinem Heer den Sold gebe. \bibverse{20} Zum
Lohn für seine Arbeit, die er getan hat, will ich ihm das Land Ägypten
geben; denn sie haben mir gedient, spricht der Herr HErr. \footnote{\textbf{29:20}
  Hes 30,24; Jes 10,5} \bibverse{21} Zur selben Zeit will ich das Horn
des Hauses Israel wachsen lassen und will deinen Mund unter ihnen
auftun, dass sie erfahren, dass ich der HErr bin. \# 30 \bibverse{1} Und
des HErrn Wort geschah zu mir und sprach: \bibverse{2} Du Menschenkind,
weissage und sprich: So spricht der Herr HErr: Heulet: „O weh des
Tages!{}`` \bibverse{3} Denn der Tag ist nahe, ja, des HErrn Tag ist
nahe, ein finsterer Tag; die Zeit der Heiden kommt. \bibverse{4} Und das
Schwert soll über Ägypten kommen; und Mohrenland muss erschrecken, wenn
die Erschlagenen in Ägypten fallen werden und sein Volk weggeführt und
seine Grundfesten umgerissen werden. \bibverse{5} Mohrenland und Libyen
und Lud mit allerlei Volk und Chub und die aus dem Lande des Bundes
sind, sollen samt ihnen durchs Schwert fallen. \bibverse{6} So spricht
der HErr: Die Schutzherren Ägyptens müssen fallen, und die Hoffart
seiner Macht muss herunter; von Migdol bis gen Syene sollen sie durchs
Schwert fallen, spricht der Herr HErr. \bibverse{7} Und sie sollen wie
andere wüste Länder wüst werden, und ihre Städte unter anderen wüsten
Städten wüst liegen, \bibverse{8} dass sie erfahren, dass ich der HErr
sei, wenn ich ein Feuer in Ägypten mache, dass alle, die ihnen helfen,
verstört werden. \bibverse{9} Zur selben Zeit werden Boten von mir
ausziehen in Schiffen, Mohrenland zu schrecken, das jetzt so sicher ist;
und wird ein Schrecken unter ihnen sein, gleich wie es Ägypten ging, da
seine Zeit kam; denn siehe, es kommt gewiss. \footnote{\textbf{30:9} Jes
  18,2; Jes 20,2-3} \bibverse{10} So spricht der Herr HErr: Ich will die
Menge in Ägypten wegräumen durch Nebukadnezar, den König zu Babel.
\bibverse{11} Denn er und sein Volk mit ihm, die Tyrannen der Heiden,
sind herzugebracht, das Land zu verderben, und werden ihre Schwerter
ausziehen wider Ägypten, dass das Land allenthalben voll Erschlagener
liege. \bibverse{12} Und ich will die Wasserströme trocken machen und
das Land bösen Leuten verkaufen, und will das Land und was darin ist,
durch Fremde verwüsten. Ich, der HErr, habe es geredet. \bibverse{13} So
spricht der Herr HErr: Ich will die Götzen zu Noph ausrotten und die
Abgötter vertilgen, und Ägypten soll keinen Fürsten mehr haben, und ich
will einen Schrecken in Ägyptenland schicken. \bibverse{14} Ich will
Pathros wüst machen und ein Feuer zu Zoan anzünden und das Recht über No
gehen lassen \bibverse{15} und will meinen Grimm ausschütten über Sin,
die Festung Ägyptens, und will die Menge zu No ausrotten. \bibverse{16}
Ich will ein Feuer in Ägypten anzünden, und Sin soll angst und bange
werden, und No soll zerrissen und Noph täglich geängstet werden.
\bibverse{17} Die junge Mannschaft zu On und Bubastus sollen durchs
Schwert fallen und die Weiber gefangen weggeführt werden. \bibverse{18}
Thachpanhes wird einen finsteren Tag haben, wenn ich das Joch Ägyptens
daselbst zerbrechen werde, dass die Hoffart seiner Macht darin ein Ende
habe; sie wird mit Wolken bedeckt werden, und ihre Töchter werden
gefangen weggeführt werden. \bibverse{19} Und ich will das Recht über
Ägypten gehen lassen, dass sie erfahren, dass ich der HErr sei.
\bibverse{20} Und es begab sich im elften Jahr, am siebenten Tage des
ersten Monats, geschah des HErrn Wort zu mir und sprach: \bibverse{21}
Du Menschenkind, ich habe den Arm Pharaos, des Königs von Ägypten,
zerbrochen; und siehe, er soll nicht verbunden werden, dass er heilen
möge, noch mit Binden zugebunden werden, dass er stark werde und ein
Schwert fassen könne. \bibverse{22} Darum spricht der Herr HErr also:
Siehe, ich will an Pharao, den König von Ägypten, und will seine Arme
zerbrechen, beide, den starken und den zerbrochenen, dass ihm das
Schwert aus seiner Hand entfallen muss; \bibverse{23} und ich will die
Ägypter unter die Heiden zerstreuen und in die Länder verjagen.
\bibverse{24} Aber die Arme des Königs zu Babel will ich stärken und ihm
mein Schwert in seine Hand geben, und will die Arme Pharaos zerbrechen,
dass er vor ihm winseln soll wie ein tödlich Verwundeter. \bibverse{25}
Ja, ich will die Arme des Königs zu Babel stärken, dass die Arme Pharaos
dahinfallen, auf dass sie erfahren, dass ich der HErr sei, wenn ich mein
Schwert dem König zu Babel in die Hand gebe, dass er's über Ägyptenland
zücke, \bibverse{26} und ich die Ägypter unter die Heiden zerstreue und
in die Länder verjage, dass sie erfahren, dass ich der HErr bin. \# 31
\bibverse{1} Und es begab sich im elften Jahr, am ersten Tage des
dritten Monats, geschah des HErrn Wort zu mir und sprach: \bibverse{2}
Du Menschenkind, sage zu Pharao, dem König von Ägypten, und zu allem
seinem Volk: Wem meinst du denn, dass du gleich seist in deiner
Herrlichkeit? \bibverse{3} Siehe, Assur war wie ein Zedernbaum auf dem
Libanon, von schönen Ästen und dick von Laub und sehr hoch, dass sein
Wipfel hoch stand unter großen, dichten Zweigen. \footnote{\textbf{31:3}
  Dan 4,7-11} \bibverse{4} Die Wasser machten, dass er groß ward, und
die Tiefe, dass er hoch wuchs. Ihre Ströme gingen rings um seinen Stamm
her und ihre Bäche zu allen Bäumen im Felde. \bibverse{5} Darum ist er
höher geworden als alle Bäume im Felde und kriegte viel Äste und lange
Zweige; denn er hatte Wasser genug, sich auszubreiten. \bibverse{6} Alle
Vögel des Himmels nisteten auf seinen Ästen, und alle Tiere im Felde
hatten Junge unter seinen Zweigen; und unter seinem Schatten wohnten
alle großen Völker. \bibverse{7} Er hatte schöne, große und lange Äste;
denn seine Wurzeln hatten viel Wasser. \bibverse{8} Und war ihm kein
Zedernbaum gleich in Gottes Garten, und die Tannenbäume waren seinen
Ästen nicht zu vergleichen, und die Kastanienbäume waren nichts gegen
seine Zweige. Ja, er war so schön wie kein Baum im Garten Gottes.
\bibverse{9} Ich hatte ihn so schön gemacht, dass er so viel Äste
kriegte, dass ihn alle lustigen Bäume im Garten Gottes neideten.
\bibverse{10} Darum spricht der Herr HErr also: Weil er so hoch geworden
ist, dass sein Wipfel stand unter großen, hohen, dichten Zweigen, und
sein Herz sich erhob, dass er so hoch geworden war, \bibverse{11} darum
gab ich ihn dem Mächtigsten unter den Heiden in die Hände, dass der mit
ihm umginge und ihn vertriebe, wie er verdient hat mit seinem gottlosen
Wesen, \bibverse{12} dass Fremde ihn ausrotten sollten, nämlich die
Tyrannen der Heiden, und ihn zerstreuen, und seine Äste auf den Bergen
und in allen Tälern liegen mussten und seine Zweige zerbrachen an allen
Bächen im Lande; dass alle Völker auf Erden von seinem Schatten
wegziehen mussten und ihn verlassen; \bibverse{13} und alle Vögel des
Himmels auf seinem umgefallenen Stamm saßen und alle Tiere im Felde sich
legten auf seine Äste; \bibverse{14} auf dass sich forthin kein Baum am
Wasser seiner Höhe überhebe, dass sein Wipfel unter großen, dichten
Zweigen stehe, und kein Baum am Wasser sich erhebe über die anderen;
denn sie müssen alle unter die Erde und dem Tod übergeben werden wie
andere Menschen, die in die Grube fahren. \bibverse{15} So spricht der
Herr HErr: Zu der Zeit, da er hinunter in die Hölle fuhr, da machte ich
ein Trauern, dass ihn die Tiefe bedeckte und seine Ströme stillstehen
mussten und die großen Wasser nicht laufen konnten; und machte, dass der
Libanon um ihn trauerte und alle Feldbäume verdorrten über ihm.
\bibverse{16} Ich erschreckte die Heiden, da sie ihn hörten fallen, da
ich ihn hinunterstieß zur Hölle, zu denen, die in die Grube gefahren
sind. Und alle lustigen Bäume unter der Erde, die edelsten und besten
auf dem Libanon, und alle, die am Wasser gestanden hatten, gönnten's ihm
wohl. \footnote{\textbf{31:16} Hes 31,14} \bibverse{17} Denn sie mussten
auch mit ihm hinunter zur Hölle, zu den Erschlagenen mit dem Schwert,
weil sie unter dem Schatten seines Armes gewohnt hatten unter den
Heiden. \bibverse{18} Wie groß meinst du denn, Pharao, dass du seist mit
deiner Pracht und Herrlichkeit unter den lustigen Bäumen? Denn du musst
mit den lustigen Bäumen unter die Erde hinabfahren und unter den
Unbeschnittenen liegen, die mit dem Schwert erschlagen sind. Also soll
es Pharao gehen samt allem seinem Volk, spricht der Herr HErr. \# 32
\bibverse{1} Und es begab sich im zwölften Jahr, am ersten Tage des
zwölften Monats, geschah des HErrn Wort zu mir und sprach: \bibverse{2}
Du Menschenkind, mache eine Wehklage über Pharao, den König von Ägypten,
und sprich zu ihm: Du bist gleich wie ein Löwe unter den Heiden und wie
ein Meerdrache und springst in deinen Strömen und rührst das Wasser auf
mit deinen Füßen und machst seine Ströme trüb. \bibverse{3} So spricht
der Herr HErr: Ich will mein Netz über dich auswerfen durch einen großen
Haufen Volks, die dich sollen in mein Garn jagen; \footnote{\textbf{32:3}
  Hes 17,20} \bibverse{4} und will dich aufs Land ziehen und aufs Feld
werfen, dass alle Vögel des Himmels auf dir sitzen sollen und alle Tiere
auf Erden von dir satt werden. \bibverse{5} Und will dein Aas auf die
Berge werfen und mit deiner Höhe die Täler ausfüllen. \bibverse{6} Das
Land, darin du schwimmst, will ich von deinem Blut rot machen bis an die
Berge hinan, dass die Bäche von dir voll werden. \bibverse{7} Und wenn
du nun ganz dahin bist, so will ich den Himmel verhüllen und seine
Sterne verfinstern und die Sonne mit Wolken überziehen, und der Mond
soll nicht scheinen. \bibverse{8} Alle Lichter am Himmel will ich über
dir lassen dunkel werden, und will eine Finsternis in deinem Lande
machen, spricht der Herr HErr. \bibverse{9} Dazu will ich vieler Völker
Herz erschreckt machen, wenn ich die Heiden deine Plage erfahren lasse
und viele Länder, die du nicht kennst. \bibverse{10} Viele Völker sollen
sich über dich entsetzen, und ihren Königen soll vor dir grauen, wenn
ich mein Schwert vor ihnen blinken lasse, und sollen plötzlich
erschrecken, dass ihnen das Herz entfallen wird über deinem Fall.
\bibverse{11} Denn so spricht der Herr HErr: Das Schwert des Königs zu
Babel soll dich treffen. \bibverse{12} Und ich will dein Volk fällen
durch das Schwert der Helden, durch allerlei Tyrannen der Heiden; die
sollen die Herrlichkeit Ägyptens verheeren, dass all ihr Volk vertilgt
werde. \bibverse{13} Und ich will alle ihre Tiere umbringen an den
großen Wassern, dass sie keines Menschen Fuß und keines Tieres Klaue
mehr trüb machen soll. \bibverse{14} Alsdann will ich ihre Wasser lauter
machen, dass ihre Ströme fließen wie Öl, spricht der Herr HErr,
\bibverse{15} wenn ich das Land Ägypten verwüstet und alles, was im
Lande ist, öde gemacht und alle, die darin wohnen, erschlagen habe, dass
sie erfahren, dass ich der HErr sei. \bibverse{16} Das wird der Jammer
sein, den man wohl mag klagen; ja, die Töchter der Heiden werden solche
Klage führen; über Ägypten und all ihr Volk wird man klagen, spricht der
Herr HErr. \bibverse{17} Und im zwölften Jahr, am fünfzehnten Tage
desselben Monats, geschah des HErrn Wort zu mir und sprach:
\bibverse{18} Du Menschenkind, beweine das Volk in Ägypten und stoße es
mit den Töchtern der starken Heiden hinab unter die Erde zu denen, die
in die Grube gefahren sind. \footnote{\textbf{32:18} Hes 31,16}
\bibverse{19} Wo ist nun deine Wollust? Hinunter, und lege dich zu den
Unbeschnittenen! \footnote{\textbf{32:19} Jes 14,11-19} \bibverse{20}
Sie werden fallen unter denen, die mit dem Schwert erschlagen sind. Das
Schwert ist schon gefasst und gezückt über ihr ganzes Volk. \footnote{\textbf{32:20}
  Hes 21,14} \bibverse{21} Von ihm werden sagen in der Hölle die starken
Helden mit ihren Gehilfen, die alle hinuntergefahren sind und liegen da
unter den Unbeschnittenen und mit dem Schwert Erschlagenen. \footnote{\textbf{32:21}
  Jes 14,9} \bibverse{22} Daselbst liegt Assur mit allem seinem Volk
umher begraben, die alle erschlagen und durchs Schwert gefallen sind;
\bibverse{23} ihre Gräber sind tief in der Grube, und sein Volk liegt
allenthalben umher begraben, die alle erschlagen und durchs Schwert
gefallen sind, vor denen sich die ganze Welt fürchtete. \footnote{\textbf{32:23}
  Jes 14,15} \bibverse{24} Da liegt auch Elam mit allem seinem Haufen
umher begraben, die alle erschlagen und durchs Schwert gefallen sind und
hinuntergefahren als die Unbeschnittenen unter die Erde, vor denen sich
auch alle Welt fürchtete; und müssen ihre Schande tragen mit denen, die
in die Grube gefahren sind. \bibverse{25} Man hat sie unter die
Erschlagenen gelegt samt allem ihrem Haufen, und liegen umher begraben;
und sind alle, wie die Unbeschnittenen und mit dem Schwert Erschlagenen,
vor denen sich auch alle Welt fürchten musste; und müssen ihre Schande
tragen mit denen, die in die Grube gefahren sind, und unter den
Erschlagenen bleiben. \bibverse{26} Da liegt Mesech und Thubal mit allem
ihrem Haufen umher begraben, die alle unbeschnitten und mit dem Schwert
erschlagen sind, vor denen sich auch die ganze Welt fürchten musste;
\bibverse{27} und alle anderen Helden, die unter den Unbeschnittenen
gefallen und mit ihrer Kriegswehr zur Hölle gefahren sind und ihre
Schwerter unter ihre Häupter haben müssen legen und deren Missetat über
ihre Gebeine gekommen ist, die doch auch gefürchtete Helden waren in der
ganzen Welt; also müssen sie liegen. \bibverse{28} So musst du freilich
auch unter den Unbeschnittenen zerschmettert werden und unter denen, die
mit dem Schwert erschlagen sind, liegen. \bibverse{29} Da liegt Edom mit
seinen Königen und allen seinen Fürsten unter den Unbeschnittenen und
mit dem Schwert Erschlagenen samt anderen, die in die Grube gefahren
sind, die doch mächtig waren. \footnote{\textbf{32:29} Hes 25,12-14}
\bibverse{30} Da sind alle Fürsten von Mitternacht und alle Sidonier,
die mit den Erschlagenen hinabgefahren sind; und ihre schreckliche
Gewalt ist zu Schanden geworden, und müssen liegen unter den
Unbeschnittenen und denen, die mit dem Schwert erschlagen sind, und ihre
Schande tragen samt denen, die in die Grube gefahren sind. \footnote{\textbf{32:30}
  Hes 38,6; Hes 28,21-23} \bibverse{31} Diese wird Pharao sehen und sich
trösten über all sein Volk, die unter ihm mit dem Schwert erschlagen
sind, und über sein ganzes Heer, spricht der Herr HErr. \footnote{\textbf{32:31}
  Jes 14,10} \bibverse{32} Denn es soll sich auch einmal alle Welt vor
mir fürchten, dass Pharao und alle seine Menge liegen unter den
Unbeschnittenen und mit dem Schwert Erschlagenen, spricht der Herr HErr.
\# 33 \bibverse{1} Und des HErrn Wort geschah zu mir und sprach:
\bibverse{2} Du Menschenkind, predige den Kindern deines Volkes und
sprich zu ihnen: Wenn ich ein Schwert über das Land führen würde, und
das Volk im Lande nähme einen Mann unter ihnen und machten ihn zu ihrem
Wächter, \bibverse{3} und er sähe das Schwert kommen über das Land und
bliese die Drommete und warnte das Volk, -- \bibverse{4} wer nun der
Drommete Hall hörte und wollte sich nicht warnen lassen, und das Schwert
käme und nähme ihn weg: desselben Blut sei auf seinem Kopf; \bibverse{5}
denn er hat der Drommete Hall gehört und hat sich dennoch nicht warnen
lassen; darum sei sein Blut auf ihm. Wer sich aber warnen lässt, der
wird sein Leben davonbringen. \bibverse{6} Wo aber der Wächter sähe das
Schwert kommen und die Drommete nicht bliese noch sein Volk warnte, und
das Schwert käme und nähme etliche weg: dieselben würden wohl um ihrer
Sünden willen weggenommen; aber ihr Blut will ich von des Wächters Hand
fordern. \bibverse{7} Und nun, du Menschenkind, ich habe dich zu einem
Wächter gesetzt über das Haus Israel, wenn du etwas aus meinem Munde
hörst, dass du sie von meinetwegen warnen sollst. \bibverse{8} Wenn ich
nun zu dem Gottlosen sage: Du Gottloser musst des Todes sterben! und du
sagst ihm solches nicht, dass sich der Gottlose warnen lasse vor seinem
Wesen, so wird wohl der Gottlose um seines gottlosen Wesens willen
sterben; aber sein Blut will ich von deiner Hand fordern. \bibverse{9}
Warnest du aber den Gottlosen vor seinem Wesen, dass er sich davon
bekehre, und er will sich nicht von seinem Wesen bekehren, so wird er um
seiner Sünde willen sterben, und du hast deine Seele errettet.
\bibverse{10} Darum, du Menschenkind, sage dem Hause Israel: Ihr sprecht
also: Unsere Sünden und Missetaten liegen auf uns, dass wir darunter
vergehen; wie können wir denn leben? \bibverse{11} So sprich zu ihnen:
So wahr als ich lebe, spricht der Herr HErr, ich habe keinen Gefallen am
Tode des Gottlosen, sondern dass sich der Gottlose bekehre von seinem
Wesen und lebe. So bekehret euch doch nun von eurem bösen Wesen. Warum
wollt ihr sterben, ihr vom Hause Israel? \footnote{\textbf{33:11} Hes
  18,23; Hes 18,31-32; Jes 55,7; Joe 2,12-13} \bibverse{12} Und du,
Menschenkind, sprich zu deinem Volk: Wenn ein Gerechter Böses tut, so
wird's ihm nicht helfen, dass er fromm gewesen ist; und wenn ein
Gottloser fromm wird, so soll's ihm nicht schaden, dass er gottlos
gewesen ist. So kann auch der Gerechte nicht leben, wenn er sündigt.
\footnote{\textbf{33:12} Hes 3,20; Hes 18,24} \bibverse{13} Denn wenn
ich zu dem Gerechten spreche, er soll leben, und er verlässt sich auf
seine Gerechtigkeit und tut Böses, so soll aller seiner Frömmigkeit
nicht gedacht werden; sondern er soll sterben in seiner Bosheit, die er
tut. \bibverse{14} Und wenn ich zum Gottlosen spreche, er soll sterben,
und er bekehrt sich von seiner Sünde und tut, was recht und gut ist,
\bibverse{15} also dass der Gottlose das Pfand wiedergibt und bezahlt,
was er geraubt hat, und nach dem Wort des Lebens wandelt, dass er kein
Böses tut: so soll er leben und nicht sterben, \footnote{\textbf{33:15}
  Hes 18,7; Lk 19,8} \bibverse{16} und aller seiner Sünden, die er getan
hat, soll nicht gedacht werden; denn er tut nun, was recht und gut ist;
darum soll er leben. \bibverse{17} Aber dein Volk spricht: Der HErr
urteilt nicht recht, obwohl doch sie unrecht haben. \bibverse{18} Denn
wenn der Gerechte sich kehrt von seiner Gerechtigkeit und tut Böses, so
stirbt er ja billig darum. \bibverse{19} Und wennre sich der Gottlose
bekehrt von seinem gottlosen Wesen und tut, was recht und gut ist, so
soll er ja billig leben. \bibverse{20} Doch sprecht ihr: Der HErr
urteilt nicht recht, obwohl ich doch euch vom Hause Israel einen
jeglichen nach seinem Wesen richte. \bibverse{21} Und es begab sich im
zwölften Jahr unserer Gefangenschaft, am fünften Tage des zehnten
Monats, kam zu mir ein Entronnener von Jerusalem und sprach: Die Stadt
ist geschlagen. \footnote{\textbf{33:21} Hes 24,26} \bibverse{22} Und
die Hand des HErrn war über mir des Abends, ehe der Entronnene kam, und
tat mir meinen Mund auf, bis er zu mir kam des Morgens; und tat mir
meinen Mund auf, also dass ich nicht mehr schweigen musste.
\bibverse{23} Und des HErrn Wort geschah zu mir und sprach:
\bibverse{24} Du Menschenkind, die Einwohner dieser Wüsten im Lande
Israel sprechen also: Abraham war ein einziger Mann und erbte dieses
Land; unser aber sind viele, desto billiger gehört das Land uns zu.
\bibverse{25} Darum sprich zu ihnen: So spricht der Herr HErr: Ihr habt
Blutiges gegessen und eure Augen zu den Götzen aufgehoben und Blut
vergossen: und ihr meint, ihr wollt das Land besitzen? \bibverse{26} Ja,
ihr fahret immer fort mit Morden und übet Gräuel, und einer schändet dem
anderen sein Weib; und ihr meint, ihr wollt das Land besitzen?
\bibverse{27} So sprich zu ihnen: So spricht der Herr HErr: So wahr ich
lebe, sollen alle, die in den Wüsten wohnen, durchs Schwert fallen; und
die auf dem Felde sind, will ich den Tieren zu fressen geben; und die in
den Festungen und Höhlen sind, sollen an der Pestilenz sterben.
\bibverse{28} Denn ich will das Land ganz verwüsten und seiner Hoffart
und Macht ein Ende machen, dass das Gebirge Israel so wüst werde, dass
niemand da durchgehe. \bibverse{29} Und sie sollen erfahren, dass ich
der HErr bin, wenn ich das Land ganz verwüstet habe um aller ihrer
Gräuel willen, die sie üben. \bibverse{30} Und du, Menschenkind, dein
Volk redet über dich an den Wänden und unter den Haustüren, und einer
spricht zum anderen: Kommt doch und lasst uns hören, was der HErr sage!
\bibverse{31} Und sie werden zu dir kommen in die Versammlung und vor
dir sitzen als mein Volk und werden deine Worte hören, aber nicht
darnach tun; sondern sie werden sie gern in ihrem Munde haben, und
gleichwohl fortleben nach ihrem Geiz. \footnote{\textbf{33:31} Jes 53,1;
  Jak 1,22} \bibverse{32} Und siehe, du musst ihnen sein wie ein
liebliches Liedlein, wie einer, der eine schöne Stimme hat und wohl
spielen kann. Also werden sie deine Worte hören und nicht darnach tun.
\bibverse{33} Wenn es aber kommt, was kommen soll, siehe, so werden sie
erfahren, dass ein Prophet unter ihnen gewesen ist. \# 34 \bibverse{1}
Und des HErrn Wort geschah zu mir und sprach: \bibverse{2} Du
Menschenkind, weissage wider die Hirten Israels, weissage und sprich zu
ihnen: So spricht der Herr HErr: Weh den Hirten Israels, die sich selbst
weiden! Sollen nicht die Hirten die Herde weiden? \footnote{\textbf{34:2}
  Hes 13,2; Jer 23,1-6} \bibverse{3} Aber ihr fresset das Fette und
kleidet euch mit der Wolle und schlachtet das Gemästete; aber die Schafe
wollt ihr nicht weiden. \bibverse{4} Der Schwachen wartet ihr nicht, und
die Kranken heilt ihr nicht, das Verwundete verbindet ihr nicht, das
Verirrte holt ihr nicht und das Verlorene sucht ihr nicht; sondern
streng und hart herrschet ihr über sie. \bibverse{5} Und meine Schafe
sind zerstreut, als die keinen Hirten haben, und allen wilden Tieren zur
Speise geworden und gar zerstreut. \footnote{\textbf{34:5} Mt 9,36}
\bibverse{6} Und gehen irre hin und wieder auf den Bergen und auf den
hohen Hügeln und sind auf dem ganzen Lande zerstreut; und ist niemand,
der nach ihnen frage oder ihrer achte. \bibverse{7} Darum höret, ihr
Hirten, des HErrn Wort! \bibverse{8} So wahr ich lebe, spricht der Herr
HErr, weil ihr meine Schafe lasset zum Raub und meine Herde allen wilden
Tieren zur Speise werden, weil sie keinen Hirten haben und meine Hirten
nach meiner Herde nicht fragen, sondern sind solche Hirten, die sich
selbst weiden, aber meine Schafe wollen sie nicht weiden: \bibverse{9}
darum, ihr Hirten, höret des HErrn Wort! \bibverse{10} So spricht der
Herr HErr: Siehe, ich will an die Hirten und will meine Herde von ihren
Händen fordern und will mit ihnen ein Ende machen, dass sie nicht mehr
sollen Hirten sein und sollen sich nicht mehr selbst weiden. Ich will
meine Schafe erretten aus ihrem Maul, dass sie sie forthin nicht mehr
fressen sollen. \bibverse{11} Denn so spricht der Herr HErr: Siehe, ich
will mich meiner Herde selbst annehmen und sie suchen. \bibverse{12} Wie
ein Hirte seine Schafe sucht, wenn sie von seiner Herde verirrt sind,
also will ich meine Schafe suchen und will sie erretten von allen
Örtern, dahin sie zerstreut waren zur Zeit, da es trüb und finster war.
\footnote{\textbf{34:12} Lk 15,4} \bibverse{13} Ich will sie von allen
Völkern ausführen und aus allen Ländern versammeln und will sie in ihr
Land führen und will sie weiden auf den Bergen Israels und in allen Auen
und auf allen Angern des Landes. \bibverse{14} Ich will sie auf die
beste Weide führen, und ihre Hürden werden auf den hohen Bergen in
Israel stehen; daselbst werden sie in sanften Hürden liegen und fette
Weide haben auf den Bergen Israels. \bibverse{15} Ich will selbst meine
Schafe weiden, und ich will sie lagern, spricht der Herr HErr.
\bibverse{16} Ich will das Verlorene wieder suchen und das Verirrte
wiederbringen und das Verwundete verbinden und des Schwachen warten;
aber was fett und stark ist, will ich vertilgen und will es weiden mit
Gericht. \bibverse{17} Aber zu euch, meine Herde, spricht der Herr HErr
also: Siehe, ich will richten zwischen Schaf und Schaf und zwischen
Widdern und Böcken. \footnote{\textbf{34:17} Mt 25,32} \bibverse{18}
Ist's euch nicht genug, so gute Weide zu haben, dass ihr das Übrige mit
Füßen tretet, und so schöne Borne zu trinken, dass ihr auch noch
dareintretet und sie trüb macht, \bibverse{19} dass meine Schafe essen
müssen, was ihr mit euren Füßen zertreten habt, und trinken, was ihr mit
euren Füßen trüb gemacht habt? \bibverse{20} Darum so spricht der Herr
HErr zu ihnen: Siehe, ich will richten zwischen den fetten und mageren
Schafen, \bibverse{21} darum dass ihr mit der Seite und Schulter drängt
und die Schwachen von euch stoßt mit euren Hörnern, bis ihr sie alle von
euch zerstreut. \bibverse{22} Und ich will meiner Herde helfen, dass sie
nicht mehr sollen zum Raub werden, und will richten zwischen Schaf und
Schaf. \bibverse{23} Und ich will ihnen einen einigen Hirten erwecken,
der sie weiden soll, nämlich meinen Knecht David. Der wird sie weiden
und soll ihr Hirte sein, \bibverse{24} und ich, der HErr, will ihr Gott
sein; aber mein Knecht David soll der Fürst unter ihnen sein; das sage
ich, der HErr. \bibverse{25} Und ich will einen Bund des Friedens mit
ihnen machen und alle bösen Tiere aus dem Lande ausrotten, dass sie in
der Wüste sicher wohnen und in den Wäldern schlafen sollen. \footnote{\textbf{34:25}
  Hes 37,26} \bibverse{26} Ich will sie und alles, was um meinen Hügel
her ist, segnen und auf sie regnen lassen zu rechter Zeit; das sollen
gnädige Regen sein, \bibverse{27} dass die Bäume auf dem Felde ihre
Früchte bringen und das Land sein Gewächs geben wird; und sie sollen
sicher auf dem Lande wohnen und sollen erfahren, dass ich der HErr bin,
wenn ich ihr Joch zerbrochen und sie errettet habe von der Hand derer,
denen sie dienen mussten. \bibverse{28} Und sie sollen nicht mehr den
Heiden zum Raub werden, und kein Tier auf Erden soll sie mehr fressen,
sondern sollen sicher wohnen ohne alle Furcht. \bibverse{29} Und ich
will ihnen eine herrliche Pflanzung aufgehen lassen, dass sie nicht mehr
sollen Hunger leiden im Lande und ihre Schmach unter den Heiden nicht
mehr tragen sollen. \bibverse{30} Und sie sollen erfahren, dass ich, der
HErr, ihr Gott, bei ihnen bin und dass sie vom Haus Israel mein Volk
seien, spricht der Herr HErr. \footnote{\textbf{34:30} Hes 11,20}
\bibverse{31} Ja, ihr Menschen sollt die Herde meiner Weide sein, und
ich will euer Gott sein, spricht der Herr HErr. \footnote{\textbf{34:31}
  Ps 100,3}

\hypertarget{section-9}{%
\section{35}\label{section-9}}

\bibverse{1} Und des HErrn Wort geschah zu mir und sprach: \bibverse{2}
Du Menschenkind, richte dein Angesicht wider das Gebirge Seir und
weissage dawider, \footnote{\textbf{35:2} Hes 25,8; Hes 25,12}
\bibverse{3} und sprich zu ihm: So spricht der Herr HErr: Siehe, ich
will an dich, du Berg Seir, und meine Hand wider dich ausstrecken und
will dich gar wüst machen. \bibverse{4} Ich will deine Städte öde
machen, dass du sollst zur Wüste werden und erfahren, dass ich der HErr
bin. \bibverse{5} Darum dass ihr ewige Feindschaft tragt wider die
Kinder Israel und triebet sie ins Schwert zur Zeit, da es ihnen übel
ging und ihre Missetat zum Ende gekommen war, \bibverse{6} darum, so
wahr ich lebe, spricht der Herr HErr, will ich dich auch blutend machen,
und du sollst dem Bluten nicht entrinnen; weil du Lust zum Blut hast,
sollst du dem Bluten nicht entrinnen. \bibverse{7} Und ich will den Berg
Seir wüst und öde machen, dass niemand darauf wandeln noch gehen soll.
\bibverse{8} Und will sein Gebirge und alle Hügel, Täler und alle Gründe
voll Toter machen, die durchs Schwert sollen erschlagen daliegen.
\bibverse{9} Ja, zu einer ewigen Wüste will ich dich machen, dass
niemand in deinen Städten wohnen soll; und ihr sollt erfahren, dass ich
der HErr bin. \bibverse{10} Und darum dass du sprichst: Diese beiden
Völker mit beiden Ländern müssen mein werden, und wir wollen sie
einnehmen -- obgleich der HErr da wohnt --, \bibverse{11} darum, so wahr
ich lebe, spricht der Herr HErr, will ich nach deinem Zorn und Hass mit
dir umgehen, wie du mit ihnen umgegangen bist aus lauter Hass, und will
bei ihnen bekannt werden, wenn ich dich gestraft habe. \bibverse{12} Und
du sollst erfahren, dass ich, der HErr, all dein Lästern gehört habe, so
du geredet hast wider die Berge Israels und gesagt: „Sie sind verwüstet
und uns zu verderben gegeben.`` \bibverse{13} Und ihr habt euch wider
mich gerühmt und heftig wider mich geredet; das habe ich gehört.
\bibverse{14} So spricht nun der Herr HErr: Ich will dich zur Wüste
machen, dass sich alles Land freuen soll. \bibverse{15} Und wie du dich
gefreut hast über das Erbe des Hauses Israel, darum dass es wüst
geworden, ebenso will ich mit dir tun, dass der Berg Seir wüst sein muss
samt dem ganzen Edom; und sie sollen erfahren, dass ich der HErr bin.
\footnote{\textbf{35:15} Hes 25,3; Ps 137,7}

\hypertarget{section-10}{%
\section{36}\label{section-10}}

\bibverse{1} Und du, Menschenkind, weissage den Bergen Israels und
sprich: Höret des HErrn Wort, ihr Berge Israels! \footnote{\textbf{36:1}
  Hes 6,2} \bibverse{2} So spricht der Herr HErr: Darum dass der Feind
über euch rühmt: Ha! die ewigen Höhen sind nun unser Erbe geworden!
\footnote{\textbf{36:2} Hes 25,3} \bibverse{3} darum weissage und
sprich: So spricht der Herr HErr: Weil man euch allenthalben verwüstet
und vertilgt, und ihr seid den übrigen Heiden zuteil geworden und seid
den Leuten ins Maul gekommen und ein böses Geschrei geworden,
\bibverse{4} darum höret, ihr Berge Israels, das Wort des Herrn HErrn!
So spricht der Herr HErr zu den Bergen und Hügeln, zu den Bächen und
Tälern, zu den öden Wüsten und verlassenen Städten, welche den übrigen
Heiden ringsumher zum Raub und Spott geworden sind: \bibverse{5} ja, so
spricht der Herr HErr: Ich habe in meinem feurigen Eifer geredet wider
die übrigen Heiden und wider das ganze Edom, welche mein Land
eingenommen haben mit Freuden von ganzem Herzen und mit Hohnlachen, es
zu verheeren und zu plündern. \bibverse{6} Darum weissage von dem Lande
Israel und sprich zu den Bergen und Hügeln, zu den Bächen und Tälern: So
spricht der Herr HErr: Siehe, ich habe in meinem Eifer und Grimm
geredet, weil ihr solche Schmach von den Heiden tragen müsset.
\bibverse{7} Darum spricht der Herr HErr also: Ich hebe meine Hand auf,
dass eure Nachbarn, die Heiden umher, ihre Schande tragen sollen.
\bibverse{8} Aber ihr Berge Israels sollt wieder grünen und eure Frucht
bringen meinem Volk Israel; und es soll in kurzem geschehen.
\bibverse{9} Denn siehe, ich will mich wieder zu euch wenden und euch
ansehen, dass ihr gebaut und besät werdet; \bibverse{10} und will bei
euch der Leute viel machen, das ganze Israel allzumal; und die Städte
sollen wieder bewohnt und die Wüsten erbaut werden. \bibverse{11} Ja,
ich will bei euch der Leute und des Viehes viel machen, dass sie sich
mehren und wachsen sollen. Und ich will euch wieder bewohnt machen wie
vorher und will euch mehr Gutes tun denn zuvor je; und ihr sollt
erfahren, dass ich der HErr sei. \footnote{\textbf{36:11} Hes 36,3; Hes
  36,38} \bibverse{12} Ich will euch Leute herzubringen, mein Volk
Israel, die werden dich besitzen; und sollst ihr Erbteil sein und sollst
sie nicht mehr ohne Erben machen. \bibverse{13} So spricht der Herr
HErr: Weil man das von euch sagt: Du hast Leute gefressen und hast dein
Volk ohne Erben gemacht, \bibverse{14} darum sollst du nun nicht mehr
Leute fressen noch dein Volk ohne Erben machen, spricht der Herr HErr.
\bibverse{15} Und ich will dich nicht mehr lassen hören die Schmähung
der Heiden, und sollst den Spott der Heiden nicht mehr tragen und sollst
dein Volk nicht mehr ohne Erben machen, spricht der Herr HErr.
\bibverse{16} Und des HErrn Wort geschah weiter zu mir: \bibverse{17} Du
Menschenkind, da das Haus Israel in seinem Lande wohnte und es
verunreinigte mit seinem Wesen und Tun, dass ihr Wesen vor mir war wie
die Unreinigkeit eines Weibes in ihrer Krankheit, \bibverse{18} da
schüttete ich meinen Grimm über sie aus um des Blutes willen, das sie im
Lande vergossen, und weil sie es verunreinigt hatten durch ihre Götzen.
\bibverse{19} Und ich zerstreute sie unter die Heiden und zerstäubte sie
in die Länder und richtete sie nach ihrem Wesen und Tun. \bibverse{20}
Und sie hielten sich wie die Heiden, zu denen sie kamen, und
entheiligten meinen heiligen Namen, dass man von ihnen sagte: Ist das
des HErrn Volk, das aus seinem Lande hat müssen ziehen? \footnote{\textbf{36:20}
  Jes 52,5} \bibverse{21} Aber ich schonte meines heiligen Namens,
welchen das Haus Israel entheiligte unter den Heiden, dahin sie kamen.
\footnote{\textbf{36:21} Hes 20,9} \bibverse{22} Darum sollst du zum
Hause Israel sagen: So spricht der Herr HErr: Ich tue es nicht um
euretwillen, ihr vom Hause Israel, sondern um meines heiligen Namens
willen, welchen ihr entheiligt habt unter den Heiden, zu welchen ihr
gekommen seid. \footnote{\textbf{36:22} Ps 115,1; Jer 14,7}
\bibverse{23} Denn ich will meinen großen Namen, der vor den Heiden
entheiligt ist, den ihr unter ihnen entheiligt habt, heilig machen. Und
die Heiden sollen erfahren, dass ich der HErr sei, spricht der Herr
HErr, wenn ich mich vor ihnen an euch erzeige, dass ich heilig sei.
\footnote{\textbf{36:23} Hes 37,28} \bibverse{24} Denn ich will euch aus
den Heiden holen und euch aus allen Landen versammeln und wieder in euer
Land führen. \bibverse{25} Und will reines Wasser über euch sprengen,
dass ihr rein werdet; von all eurer Unreinigkeit und von allen euren
Götzen will ich euch reinigen. \footnote{\textbf{36:25} Sach 13,1; Hebr
  10,22} \bibverse{26} Und ich will euch ein neues Herz und einen neuen
Geist in euch geben und will das steinerne Herz aus eurem Fleisch
wegnehmen und euch ein fleischernes Herz geben; \footnote{\textbf{36:26}
  Hes 11,19} \bibverse{27} ich will meinen Geist in euch geben und will
solche Leute aus euch machen, die in meinen Geboten wandeln und meine
Rechte halten und darnach tun. \footnote{\textbf{36:27} Hes 37,24; Hes
  39,29; Jes 44,3} \bibverse{28} Und ihr sollt wohnen im Lande, das ich
euren Vätern gegeben habe, und sollt mein Volk sein, und ich will euer
Gott sein. \footnote{\textbf{36:28} Hes 11,20} \bibverse{29} Ich will
euch von aller eurer Unreinigkeit losmachen und will dem Korn rufen und
will es mehren und will euch keine Teuerung kommen lassen. \bibverse{30}
Ich will die Früchte auf den Bäumen und das Gewächs auf dem Felde
mehren, dass euch die Heiden nicht mehr verspotten mit der Teuerung.
\footnote{\textbf{36:30} Joe 2,17; Joe 2,19} \bibverse{31} Alsdann
werdet ihr an euer böses Wesen gedenken und an euer Tun, das nicht gut
war, und wird euch eure Sünde und Abgötterei gereuen. \footnote{\textbf{36:31}
  Hes 16,61; Hes 16,63} \bibverse{32} Solches will ich tun, nicht um
euretwillen, spricht der Herr HErr, dass ihr's wisset; sondern ihr
werdet euch müssen schämen und schamrot werden, ihr vom Hause Israel,
über eurem Wesen. \footnote{\textbf{36:32} Hes 36,22} \bibverse{33} So
spricht der Herr HErr: Zu der Zeit, wenn ich euch reinigen werde von
allen euren Sünden, so will ich die Städte wieder besetzen, und die
Wüsten sollen wieder gebaut werden. \bibverse{34} Das verwüstete Land
soll wieder gepflügt werden, dafür dass es verheert war; dass es sehen
sollen alle, die dadurch gehen,

\bibverse{35} und sagen: Dieses Land war verheert, und jetzt ist's wie
der Garten Eden; und diese Städte waren zerstört, öde und zerrissen, und
stehen nun fest gebaut.

\bibverse{36} Und die Heiden, die um euch her übrigbleiben werden,
sollen erfahren, dass ich der HErr bin, der da baut, was zerrissen ist,
und pflanzt, was verheert war. Ich, der HErr, sage es und tue es auch.
\bibverse{37} So spricht der Herr HErr: Auch darin will ich mich vom
Hause Israel finden lassen, dass ich es ihnen erzeige: ich will die
Menschen bei ihnen mehren wie eine Herde. \footnote{\textbf{36:37} Mi
  2,12} \bibverse{38} Wie eine heilige Herde, wie eine Herde zu
Jerusalem auf ihren Festen, so sollen die verheerten Städte voll
Menschenherden werden und sollen erfahren, dass ich der HErr bin. \# 37
\bibverse{1} Und des HErrn Hand kam über mich, und er führte mich hinaus
im Geist des HErrn und stellte mich auf ein weites Feld, das voller
Totengebeine lag. \bibverse{2} Und er führte mich allenthalben dadurch.
Und siehe, des Gebeins lag sehr viel auf dem Feld; und siehe, sie waren
sehr verdorrt. \bibverse{3} Und er sprach zu mir: Du Menschenkind,
meinst du auch, dass diese Gebeine wieder lebendig werden? Und ich
sprach: Herr HErr, das weißt du wohl. \bibverse{4} Und er sprach zu mir:
Weissage von diesen Gebeinen und sprich zu ihnen: Ihr verdorrten
Gebeine, höret des HErrn Wort! \bibverse{5} So spricht der Herr HErr von
diesen Gebeinen: Siehe, ich will einen Odem in euch bringen, dass ihr
sollt lebendig werden. \bibverse{6} Ich will euch Adern geben und
Fleisch lassen über euch wachsen und euch mit Haut überziehen und will
euch Odem geben, dass ihr wieder lebendig werdet, und ihr sollt
erfahren, dass ich der HErr bin. \footnote{\textbf{37:6} Jes 26,19}
\bibverse{7} Und ich weissagte, wie mir befohlen war; und siehe, da
rauschte es, als ich weissagte, und siehe, es regte sich, und die
Gebeine kamen wieder zusammen, ein jegliches zu seinem Gebein.
\footnote{\textbf{37:7} Hes 37,10} \bibverse{8} Und ich sah, und siehe,
es wuchsen Adern und Fleisch darauf, und sie wurden mit Haut überzogen;
es war aber noch kein Odem in ihnen. \bibverse{9} Und er sprach zu mir:
Weissage zum Winde; weissage, du Menschenkind, und sprich zum Wind: So
spricht der Herr HErr: Wind komm herzu aus den vier Winden und blase
diese Getöteten an, dass sie wieder lebendig werden! \bibverse{10} Und
ich weissagte, wie er mir befohlen hatte. Da kam Odem in sie, und sie
wurden wieder lebendig und richteten sich auf ihre Füße. Und ihrer war
ein sehr großes Heer. \bibverse{11} Und er sprach zu mir: Du
Menschenkind, diese Gebeine sind das ganze Haus Israel. Siehe, jetzt
sprechen sie: Unsere Gebeine sind verdorrt, und unsere Hoffnung ist
verloren, und es ist aus mit uns. \bibverse{12} Darum weissage und
sprich zu ihnen: So spricht der Herr HErr: Siehe, ich will eure Gräber
auftun und will euch, mein Volk, aus denselben herausholen und euch ins
Land Israel bringen; \bibverse{13} und ihr sollt erfahren, dass ich der
HErr bin, wenn ich eure Gräber geöffnet und euch, mein Volk, aus
denselben gebracht habe. \bibverse{14} Und ich will meinen Geist in euch
geben, dass ihr wieder leben sollt, und will euch in euer Land setzen,
und sollt erfahren, dass ich der HErr bin. Ich rede es und tue es auch,
spricht der HErr. \bibverse{15} Und des HErrn Wort geschah zu mir und
sprach: \bibverse{16} Du Menschenkind, nimm dir ein Holz und schreibe
darauf: Des Juda und der Kinder Israel, seiner Zugetanen. Und nimm noch
ein Holz und schreibe darauf: Des Joseph, nämlich das Holz Ephraims, und
des ganzen Hauses Israel, seiner Zugetanen. \footnote{\textbf{37:16} Hes
  4,1} \bibverse{17} Und tue eins zum anderen zusammen, dass es ein Holz
werde in deiner Hand. \bibverse{18} So nun dein Volk zu dir wird sagen
und sprechen: Willst du uns nicht zeigen, was du damit meinst?
\bibverse{19} So sprich zu ihnen: So spricht der Herr HErr: Siehe, ich
will das Holz Josephs, welches ist in Ephraims Hand, nehmen samt seinen
Zugetanen, den Stämmen Israels, und will sie zu dem Holz Judas tun und
ein Holz daraus machen, und sollen eins in meiner Hand sein.
\bibverse{20} Und sollst also die Hölzer, darauf du geschrieben hast, in
deiner Hand halten, dass sie zusehen, \bibverse{21} und sollst zu ihnen
sagen: So spricht der Herr HErr: Siehe, ich will die Kinder Israel holen
aus den Heiden, dahin sie gezogen sind, und will sie allenthalben
sammeln und will sie wieder in ihr Land bringen \bibverse{22} und will
ein Volk aus ihnen machen im Lande auf den Bergen Israels, und sie
sollen allesamt einen König haben und sollen nicht mehr zwei Völker noch
in zwei Königreiche zerteilt sein; \footnote{\textbf{37:22} Jes
  11,12-13; Hos 2,2; Jer 3,18} \bibverse{23} sollen sich auch nicht mehr
verunreinigen mit ihren Götzen und Gräueln und allerlei Sünden. Ich will
ihnen heraushelfen aus allen Örtern, da sie gesündigt haben, und will
sie reinigen; und sie sollen mein Volk sein, und ich will ihr Gott sein.
\footnote{\textbf{37:23} Hes 36,28} \bibverse{24} Und mein Knecht David
soll ihr König und ihrer aller einiger Hirte sein. Und sie sollen
wandeln in meinen Rechten und meine Gebote halten und darnach tun.
\footnote{\textbf{37:24} Hes 34,23; Hes 36,27} \bibverse{25} Und sie
sollen wieder in dem Lande wohnen, das ich meinem Knecht Jakob gegeben
habe, darin eure Väter gewohnt haben. Sie und ihre Kinder und
Kindeskinder sollen darin wohnen ewiglich, und mein Knecht David soll
ewiglich ihr Fürst sein. \bibverse{26} Und ich will mit ihnen einen Bund
des Friedens machen, das soll ein ewiger Bund sein mit ihnen; und will
sie erhalten und mehren, und mein Heiligtum soll unter ihnen sein
ewiglich. \bibverse{27} Und ich will unter ihnen wohnen und will ihr
Gott sein, und sie sollen mein Volk sein, \bibverse{28} dass auch die
Heiden sollen erfahren, dass ich der HErr bin, der Israel heilig macht,
wenn mein Heiligtum ewiglich unter ihnen sein wird. \footnote{\textbf{37:28}
  Hes 36,36}

\hypertarget{section-11}{%
\section{38}\label{section-11}}

\bibverse{1} Und des HErrn Wort geschah zu mir und sprach: \bibverse{2}
Du Menschenkind, wende dich gegen Gog, der im Lande Magog ist und der
oberste Fürst in Mesech und Thubal, und weissage von ihm \bibverse{3}
und sprich: So spricht der Herr HErr: Siehe, ich will an dich Gog! der
du der oberste Fürst bist in Mesech und Thubal. \bibverse{4} Siehe, ich
will dich herumlenken und will dir einen Zaum ins Maul legen und will
dich herausführen mit allem deinem Heer, Ross und Mann, die alle wohl
gekleidet sind; und ist ihrer ein großer Haufe, die alle Tartsche und
Schild und Schwert führen. \bibverse{5} Du führst mit dir Perser, Mohren
und Libyer, die alle Schild und Helm führen, \bibverse{6} dazu Gomer und
all sein Heer samt dem Hause Thogarma, das gegen Mitternacht liegt, mit
allem seinem Heer; ja, du führst ein großes Volk mit dir. \bibverse{7}
Wohlan, rüste dich wohl, du und alle deine Haufen, die bei dir sind, und
sei du ihr Hauptmann! \bibverse{8} Nach langer Zeit sollst du
heimgesucht werden. Zur letzten Zeit wirst du kommen in das Land, das
vom Schwert wiedergebracht und aus vielen Völkern zusammengekommen ist,
nämlich auf die Berge Israels, welche lange Zeit wüst gewesen sind; und
nun ist es ausgeführt aus den Völkern, und wohnen alle sicher.
\bibverse{9} Du wirst heraufziehen und daherkommen mit großem Ungestüm;
und wirst sein wie eine Wolke, das Land zu bedecken, du und all dein
Heer und das große Volk mit dir. \bibverse{10} So spricht der Herr HErr:
Zu der Zeit wirst du dir solches vornehmen und wirst Böses im Sinn haben
\bibverse{11} und gedenken: „Ich will das Land ohne Mauern überfallen
und über die kommen, die still und sicher wohnen, als die alle ohne
Mauern dasitzen und haben weder Riegel noch Tore``, \footnote{\textbf{38:11}
  Sach 2,8} \bibverse{12} auf dass du rauben und plündern mögest und
dein Hand lassen gehen über die verstörten Örter, die wieder bewohnt
sind, und über das Volk, das aus den Heiden zusammengerafft ist und sich
in die Nahrung und Güter geschickt hat und mitten auf der Erde wohnt.
\footnote{\textbf{38:12} Hes 5,5} \bibverse{13} Das reiche Arabien,
Dedan und die Kaufleute von Tharsis und alle Gewaltigen, die daselbst
sind, werden zu dir sagen: Ich meine ja, du seist recht gekommen, zu
rauben, und hast deine Haufen versammelt, zu plündern, auf dass du
wegnehmest Silber und Gold und sammlest Vieh und Güter, und großen Raub
treibest. \bibverse{14} Darum so weissage, du Menschenkind, und sprich
zu Gog: So spricht der Herr HErr: Ist's nicht also, dass du wirst
merken, wenn mein Volk Israel sicher wohnen wird? \bibverse{15} So wirst
du kommen aus deinem Ort, von den Enden gegen Mitternacht, du und großes
Volk mit dir, alle zu Rosse, ein großer Haufe und ein mächtiges Heer,
\bibverse{16} und wirst heraufziehen über mein Volk Israel wie eine
Wolke, das Land zu bedecken. Solches wird zur letzten Zeit geschehen.
Ich will dich aber darum in mein Land kommen lassen, auf dass die Heiden
mich erkennen, wie ich an dir, o Gog, geheiligt werde vor ihren Augen.
\bibverse{17} So spricht der Herr HErr: Du bist's, von dem ich vorzeiten
gesagt habe durch meine Diener, die Propheten in Israel, die zur selben
Zeit weissagten, dass ich dich über sie kommen lassen wollte.
\footnote{\textbf{38:17} Jes 24,21; Jes 63,6; Joe 4,2; Joe 4,12; Zeph
  3,8} \bibverse{18} Und es wird geschehen zu der Zeit, wann Gog kommen
wird über das Land Israel, spricht der Herr HErr, wird heraufziehen mein
Zorn in meinem Grimm. \bibverse{19} Und ich rede solches in meinem Eifer
und im Feuer meines Zorns. Denn zur selben Zeit wird großes Zittern sein
im Lande Israel, \bibverse{20} dass vor meinem Angesicht zittern sollen
die Fische im Meer, die Vögel unter dem Himmel, die Tiere auf dem Felde
und alles, was sich regt und bewegt auf dem Lande, und alle Menschen,
die auf der Erde sind; und sollen die Berge umgekehret werden und die
Felswände und alle Mauern zu Boden fallen. \bibverse{21} Ich will aber
wider ihn herbeirufen das Schwert auf allen meinen Bergen, spricht der
Herr HErr, dass eines jeglichen Schwert soll wider den anderen sein.
\bibverse{22} Und ich will ihn richten mit Pestilenz und Blut und will
regnen lassen Platzregen mit Schloßen, Feuer und Schwefel über ihn und
sein Heer und über das große Volk, das mit ihm ist. \bibverse{23} Also
will ich denn herrlich, heilig und bekannt werden vor vielen Heiden,
dass sie erfahren sollen, dass ich der HErr bin. \footnote{\textbf{38:23}
  Hes 29,6}

\hypertarget{section-12}{%
\section{39}\label{section-12}}

\bibverse{1} Und du, Menschenkind, weissage wider Gog und sprich: Also
spricht der Herr HErr: Siehe, ich will an dich, Gog, der du der oberste
Fürst bist in Mesech und Thubal. \bibverse{2} Siehe, ich will dich
herumlenken und locken und aus den Enden von Mitternacht bringen und auf
die Berge Israels kommen lassen. \bibverse{3} Und ich will dir den Bogen
aus deiner linken Hand schlagen und deine Pfeile aus deiner rechten Hand
werfen. \bibverse{4} Auf den Bergen Israels sollst du niedergelegt
werden, du mit allem deinem Heer und mit dem Volk, das bei dir ist. Ich
will dich den Vögeln, woher sie fliegen, und den Tieren auf dem Felde zu
fressen geben. \bibverse{5} Du sollst auf dem Felde darniederliegen;
denn ich, der Herr HErr, habe es gesagt. \bibverse{6} Und ich will Feuer
werfen über Magog und über die, die in den Inseln sicher wohnen; und
sollen's erfahren, dass ich der HErr bin. \bibverse{7} Denn ich will
meinen heiligen Namen kundmachen unter meinem Volk Israel und will
meinen heiligen Namen nicht länger schänden lassen; sondern die Heiden
sollen erfahren, dass ich der HErr bin, der Heilige in Israel.
\bibverse{8} Siehe, es ist gekommen und geschehen, spricht der Herr
HErr; das ist der Tag, davon ich geredet habe. \bibverse{9} Und die
Bürger in den Städten Israels werden herausgehen und Feuer machen und
verbrennen die Waffen, Schilde, Tartschen, Bogen, Pfeile, Keulen und
langen Spieße; und sie werden sieben Jahre lang Feuer damit machen,
\bibverse{10} dass sie nicht müssen Holz auf dem Felde holen noch im
Walde hauen, sondern von den Waffen werden sie Feuer machen; und sollen
die berauben, von denen sie beraubt sind, und plündern, von denen sie
geplündert sind, spricht der Herr HErr. \bibverse{11} Und soll zu der
Zeit geschehen, da will ich Gog einen Ort geben zum Begräbnis in Israel,
nämlich das Tal, da man geht am Meer gegen Morgen, also dass die, die
vorübergehen, sich davor scheuen werden, weil man daselbst Gog mit
seiner Menge begraben hat; und soll heißen „Gogs Haufental``.
\bibverse{12} Es wird sie aber das Haus Israel begraben sieben Monden
lang, damit das Land gereinigt werde. \bibverse{13} Ja, alles Volk im
Lande wird an ihnen zu begraben haben, und sie werden Ruhm davon haben
des Tages, da ich meine Herrlichkeit erzeige, spricht der Herr HErr.
\bibverse{14} Und sie werden Leute aussondern, die stets im Lande
umhergehen und mit ihnen die Totengräber, zu begraben die Übrigen auf
dem Lande, auf dass es gereinigt werde; nach sieben Monden werden sie
forschen. \bibverse{15} Und die, die im Lande umhergehen und eines
Menschen Gebein sehen, werden dabei ein Mal aufrichten, bis es die
Totengräber auch in Gogs Haufental begraben. \bibverse{16} So soll auch
die Stadt heißen Hamona. Also werden sie das Land reinigen.
\bibverse{17} Nun, du Menschenkind, so spricht der Herr HErr: Sage allen
Vögeln, woher sie fliegen, und allen Tieren auf dem Felde: Sammelt euch
und kommt her, findet euch allenthalben zuhauf zu meinem Schlachtopfer,
das ich euch schlachte -- ein großes Schlachtopfer auf den Bergen
Israels --, und fresset Fleisch und saufet Blut! \footnote{\textbf{39:17}
  Offb 19,17-18} \bibverse{18} Fleisch der Starken sollt ihr fressen,
und Blut der Fürsten auf Erden sollt ihr saufen, der Widder, der Hammel,
der Böcke, der Ochsen, die allzumal feist und wohl gemästet sind.
\bibverse{19} Und sollt das Fett fressen, dass ihr voll werdet, und das
Blut saufen, dass ihr trunken werdet, von dem Schlachtopfer, das ich
euch schlachte. \bibverse{20} Sättiget euch nun an meinem Tisch von
Rossen und Reitern, von Starken und allerlei Kriegsleuten, spricht der
Herr HErr. \bibverse{21} Und ich will meine Herrlichkeit unter die
Heiden bringen, dass alle Heiden sehen sollen mein Urteil, das ich habe
ergehen lassen, und meine Hand, die ich an sie gelegt habe,
\bibverse{22} und also das Haus Israel erfahre, dass ich, der HErr, ihr
Gott bin von dem Tage an und hinfürder, \bibverse{23} und die Heiden
erfahren, wie das Haus Israel um seiner Missetat willen sei weggeführt.
Weil sie sich an mir versündigt hatten, darum habe ich mein Angesicht
vor ihnen verborgen und habe sie übergeben in die Hände ihrer
Widersacher, dass sie allzumal durchs Schwert fallen mussten.
\bibverse{24} Ich habe ihnen getan, wie ihre Sünde und Übertretung
verdient haben, und also mein Angesicht vor ihnen verborgen.
\bibverse{25} Darum so spricht der Herr HErr: Nun will ich das Gefängnis
Jakobs wenden und mich des ganzen Hauses Israel erbarmen und um meinen
heiligen Namen eifern. \footnote{\textbf{39:25} Hes 16,53-63}
\bibverse{26} Sie aber werden ihre Schmach und alle ihre Sünde, damit
sie sich an mir versündigt haben, tragen, wenn sie nun sicher in ihrem
Lande wohnen, dass sie niemand schrecke, \bibverse{27} und ich sie
wieder aus den Völkern gebracht und aus den Landen ihrer Feinde
versammelt habe und ich an ihnen geheiligt worden bin vor den Augen
vieler Heiden. \bibverse{28} Also werden sie erfahren, dass ich, der
HErr, ihr Gott bin, der ich sie habe lassen unter die Heiden wegführen
und wiederum in ihr Land versammeln und nicht einen von ihnen dort
gelassen habe. \bibverse{29} Und ich will mein Angesicht nicht mehr vor
ihnen verbergen; denn ich habe meinen Geist über das Haus Israel
ausgegossen, spricht der Herr HErr. \# 40 \bibverse{1} Im
fünfundzwanzigsten Jahr unserer Gefangenschaft, im Anfang des Jahres, am
zehnten Tage des Monats, im vierzehnten Jahr, nachdem die Stadt
geschlagen war, eben an diesem Tage kam des HErrn Hand über mich und
führte mich dahin. \bibverse{2} Durch göttliche Gesichte führte er mich
ins Land Israel und stellte mich auf einen sehr hohen Berg, darauf war's
wie eine gebaute Stadt gegen Mittag. \bibverse{3} Und da er mich
dahingebracht hatte, siehe, da war ein Mann, des Ansehen war wie Erz;
der hatte eine leinene Schnur und eine Messrute in seiner Hand und stand
unter dem Tor. \footnote{\textbf{40:3} Hes 47,3; Sach 2,5; Offb 21,15}
\bibverse{4} Und er sprach zu mir: Du Menschenkind, siehe und höre
fleißig zu und merke auf alles, was ich dir zeigen will. Denn darum bist
du hergebracht, dass ich dir solches zeige, auf dass du solches alles,
was du hier siehst, verkündigest dem Hause Israel. \footnote{\textbf{40:4}
  Hes 44,5} \bibverse{5} Und siehe, es ging eine Mauer auswendig um das
Haus ringsumher. Und der Mann hatte die Messrute in der Hand, die war
sechs Ellen lang; eine jegliche Elle war eine Handbreit länger denn eine
gemeine Elle. Und er maß das Gebäude in die Breite eine Rute und in die
Höhe auch eine Rute. \bibverse{6} Und er ging ein zum Tor, das gegen
Morgen lag, und ging hinauf auf seinen Stufen und maß die Schwelle am
Tor, nämlich die eine Schwelle, eine Rute breit. \bibverse{7} Und die
Gemächer, die beiderseits neben dem Tor waren, maß er auch, nach der
Länge eine Rute und nach der Breite eine Rute; und der Raum zwischen den
Gemächern war fünf Ellen weit. Und er maß auch die Schwelle am Tor neben
der Halle, die nach dem Hause zu war, eine Rute. \bibverse{8} Und er maß
die Halle am Tor, die nach dem Hause zu war, eine Rute. \bibverse{9} Und
maß die Halle am Tor acht Ellen und ihre Pfeiler zwei Ellen, und die
Halle am Tor war nach dem Hause zu. \bibverse{10} Und der Gemächer waren
auf jeglicher Seite drei am Tor gegen Morgen, je eins so weit wie das
andere, und die Pfeiler auf beiden Seiten waren gleich groß.
\bibverse{11} Darnach maß er die Weite der Tür im Tor zehn Ellen und die
Länge des Tors dreizehn Ellen. \bibverse{12} Und vorn an den Gemächern
war Raum abgegrenzt auf beiden Seiten, je eine Elle; aber die Gemächer
waren je sechs Ellen auf beiden Seiten. \bibverse{13} Dazu maß er das
Tor vom Dach der Gemächer auf der einen Seite bis zum Dach der Gemächer
auf der anderen Seite fünfundzwanzig Ellen breit; und eine Tür stand
gegenüber der anderen. \bibverse{14} Und er machte die Pfeiler sechzig
Ellen, und an den Pfeilern war der Vorhof, am Tor ringsherum.
\bibverse{15} Und vom Tor, da man hineingeht, bis außen vor die Halle an
der inneren Seite des Tors waren fünfzig Ellen. \bibverse{16} Und es
waren enge Fensterlein an den Gemächern und an ihren Pfeilern
hineinwärts am Tor ringsumher. Also waren auch Fenster inwendig an der
Halle herum, und an den Pfeilern war Palmlaubwerk. \bibverse{17} Und er
führte mich weiter zum äußeren Vorhof, und siehe, da waren Kammern und
ein Pflaster gemacht am Vorhofe herum; dreißig Kammern waren auf dem
Pflaster. \bibverse{18} Und es war das Pflaster zur Seite der Tore,
solang die Tore waren, nämlich das untere Pflaster. \bibverse{19} Und er
maß die Breite von dem unteren Tor an bis vor den inneren Hof auswendig
hundert Ellen, gegen Morgen und gegen Mitternacht. \bibverse{20} Er maß
auch das Tor, das gegen Mitternacht lag, am äußeren Vorhof, nach der
Länge und Breite. \bibverse{21} Das hatte auch auf jeder Seite drei
Gemächer und hatte auch seine Pfeiler und Halle, gleich so groß wie am
vorigen Tor, fünfzig Ellen die Länge und fünfundzwanzig Ellen die
Breite. \bibverse{22} Und hatte auch seine Fenster und seine Halle und
sein Palmlaubwerk, gleich wie das Tor gegen Morgen; und hatte sieben
Stufen, da man hinaufging, und hatte seine Halle davor. \bibverse{23}
Und es waren Tore am inneren Vorhof gegenüber den Toren, die gegen
Mitternacht und Morgen standen; und er maß hundert Ellen von einem Tor
zum anderen. \bibverse{24} Darnach führte er mich gegen Mittag, und
siehe, da war auch ein Tor gegen Mittag; und er maß seine Pfeiler und
Halle gleich wie die anderen. \bibverse{25} Und es waren auch Fenster an
ihm und an seiner Halle umher, gleich wie jene Fenster; und es war
fünfzig Ellen lang und fünfundzwanzig Ellen breit. \bibverse{26} Und
waren auch sieben Stufen hinauf und eine Halle davor und Palmlaubwerk an
ihren Pfeilern auf jeglicher Seite. \bibverse{27} Und es war auch ein
Tor am inneren Vorhof gegen Mittag, und er maß hundert Ellen von dem
einen Mittagstor zum anderen. \bibverse{28} Und er führte mich weiter
durchs Mittagstor in den inneren Vorhof und maß dasselbe Tor gleich so
groß wie die anderen, \bibverse{29} mit seinen Gemächern, Pfeilern und
Halle und mit Fenstern an ihm und an seiner Halle, ebenso groß wie jene,
ringsumher; und es war fünfzig Ellen lang und fünfundzwanzig Ellen
breit. \bibverse{30} Und es ging eine Halle herum, fünfundzwanzig Ellen
lang und fünf Ellen breit. \bibverse{31} Und die Halle, die gegen den
äußeren Vorhof stand, hatte auch Palmlaubwerk an den Pfeilern; es waren
aber acht Stufen hinaufzugehen. \bibverse{32} Darnach führte er mich zum
inneren Vorhof gegen Morgen und maß das Tor gleich so groß wie die
anderen, \bibverse{33} mit seinen Gemächern, Pfeilern und Halle, gleich
so groß wie die anderen, und mit Fenstern an ihm und an seiner Halle
ringsumher; und es war fünfzig Ellen lang und fünfundzwanzig Ellen
breit. \bibverse{34} Und seine Halle stand auch gegen den äußeren Vorhof
und Palmlaubwerk an ihren Pfeilern zu beiden Seiten und acht Stufen
hinauf. \bibverse{35} Darnach führte er mich zum Tor gegen Mitternacht;
das maß er gleich so groß wie die anderen, \bibverse{36} mit seinen
Gemächern, Pfeilern und Halle und ihren Fenstern ringsumher, fünfzig
Ellen lang und fünfundzwanzig Ellen breit. \bibverse{37} Und seine Halle
stand auch gegen den äußeren Vorhof und Palmlaubwerk an den Pfeilern zu
beiden Seiten und acht Stufen hinauf. \bibverse{38} Und unten an den
Pfeilern an jedem Tor war eine Kammer mit einer Tür, darin man das
Brandopfer wusch. \bibverse{39} Aber in der Halle des Tors standen auf
jeglicher Seite zwei Tische, darauf man die Brandopfer, Sündopfer und
Schuldopfer schlachten sollte.

\bibverse{40} Und herauswärts zur Seite, da man hinaufgeht zum Tor gegen
Mitternacht, standen auch zwei Tische und an der anderen Seite unter der
Halle des Tors auch zwei Tische. \bibverse{41} Also standen auf jeder
Seite des Tors vier Tische; das sind zusammen acht Tische, darauf man
schlachtete.

\bibverse{42} Und noch vier Tische, zum Brandopfer gemacht, die waren
aus gehauenen Steinen, je anderthalb Ellen lang und breit und eine Elle
hoch, darauf man legte allerlei Geräte, womit man Brandopfer und andere
Opfer schlachtete. \bibverse{43} Und es gingen Leisten herum,
hineinwärts gebogen, eine quere Hand hoch. Und auf die Tische sollte man
das Opferfleisch legen. \bibverse{44} Und außen vor dem inneren Tor
waren zwei Kammern im inneren Vorhofe: eine an der Seite neben dem Tor
zur Mitternacht, die sah gegen Mittag; die andere zur Seite des Tors
gegen Mittag, die sah gegen Mitternacht. \bibverse{45} Und er sprach zu
mir: Die Kammer gegen Mittag gehört den Priestern, die im Hause dienen
sollen;

\bibverse{46} aber die Kammer gegen Mitternacht gehört den Priestern,
die auf dem Altar dienen. Dies sind die Kinder Zadok, welche allein
unter den Kindern Levi vor den HErrn treten sollen, ihm zu dienen.
\footnote{\textbf{40:46} 1Kö 1,8; 1Kö 1,39; 1Chr 5,34} \bibverse{47} Und
er maß den Vorhof, nämlich hundert Ellen lang und hundert Ellen breit
ins Gevierte; und der Altar stand vorn vor dem Tempel. \footnote{\textbf{40:47}
  Hes 43,13}

\bibverse{48} Und er führte mich hinein zur Halle des Tempels und maß
die Pfeiler der Halle fünf Ellen auf jeder Seite und das Tor vierzehn
Ellen, und die Wände zu beiden Seiten an der Tür drei Ellen auf jeder
Seite. \footnote{\textbf{40:48} 1Kö 6,3}

\bibverse{49} Aber die Halle war zwanzig Ellen lang und elf Ellen weit
und hatte Stufen, da man hinaufging; und Säulen standen an den Pfeilern,
auf jeder Seite eine. \footnote{\textbf{40:49} 1Kö 7,21}

\hypertarget{section-13}{%
\section{41}\label{section-13}}

\bibverse{1} Und er führte mich hinein in den Tempel und maß die Pfeiler
an den Wänden; die waren zu jeder Seite sechs Ellen breit, soweit das
Haus war. \bibverse{2} Und die Tür war zehn Ellen weit; aber die Wände
zu beiden Seiten an der Tür waren jede fünf Ellen breit. Und er maß den
Raum im Tempel; der hatte vierzig Ellen in die Länge und zwanzig Ellen
in die Breite. \bibverse{3} Und er ging inwendig hinein und maß die
Pfeiler der Tür, zwei Ellen; und die Tür hatte sechs Ellen, und die
Breite zu beiden Seiten an der Tür je sieben Ellen. \bibverse{4} Und er
maß zwanzig Ellen in die Länge und zwanzig Ellen in die Breite am
Tempel. Und er sprach zu mir: Dies ist das Allerheiligste. \footnote{\textbf{41:4}
  Hes 43,12} \bibverse{5} Und er maß die Wand des Hauses sechs Ellen
dick. Daran waren Gänge allenthalben herum, geteilt in Gemächer, die
waren allenthalben vier Ellen weit. \bibverse{6} Und derselben Gemächer
waren je dreißig, dreimal übereinander, und reichten bis auf die Wand
des Hauses, an der die Gänge waren allenthalben herum, und wurden also
festgehalten, dass sie in des Hauses Wand nicht eingriffen. \bibverse{7}
Und die Gänge rings um das Haus her mit ihren Gemächern waren umso
weiter, je höher sie lagen; und aus dem unteren ging man in den
mittleren und aus dem mittleren in den obersten. \bibverse{8} Und ich
sah am Hause eine Erhöhung ringsumher als Grundlage der Gänge, die hatte
eine volle Rute von sechs Ellen bis an den Rand. \bibverse{9} Und die
Breite der Wand außen an den Gängen war fünf Ellen; und es war ein
freigelassener Raum an den Gemächern am Hause. \bibverse{10} Und die
Breite bis zu den Kammern war zwanzig Ellen um das Haus herum.
\bibverse{11} Und es waren zwei Türen an den Gängen nach dem
freigelassenen Raum, eine gegen Mitternacht, die andere gegen Mittag;
und der freigelassene Raum war fünf Ellen weit ringsumher. \bibverse{12}
Und das Gebäude am Hofraum gegen Abend war siebzig Ellen weit, und die
Mauer des Gebäudes war fünf Ellen breit allenthalben umher, und es war
neunzig Ellen lang. \bibverse{13} Und er maß die Länge des Hauses, die
hatte hundert Ellen; und der Hofraum samt dem Gebäude und seinen Mauern
war auch hundert Ellen lang. \bibverse{14} Und die Weite der vorderen
Seite des Hauses samt dem Hofraum gegen Morgen war auch hundert Ellen.

\bibverse{15} Und er maß die Länge des Gebäudes am Hofraum, welches
hinter ihm liegt, mit seinen Umgängen von einer Seite bis zur anderen
hundert Ellen, und den inneren Tempel und die Hallen im Vorhofe
\bibverse{16} samt den Schwellen, den engen Fenstern und den drei
Umgängen ringsumher; und es war Tafelwerk allenthalben herum.
\bibverse{17} Er maß auch, wie hoch von der Erde bis zu den Fenstern war
und wie breit die Fenster sein sollten; und maß vom Tor bis zum
Allerheiligsten auswendig und inwendig herum.

\bibverse{18} Und am ganzen Hause herum waren Cherubim und Palmlaubwerk
zwischen die Cherubim gemacht.

\bibverse{19} Und ein jeder Cherub hatte zwei Angesichter: auf einer
Seite wie ein Menschenkopf, auf der anderen Seite wie ein Löwenkopf.

\bibverse{20} Vom Boden an bis hinauf über die Tür waren die Cherubim
und die Palmen geschnitzt, desgleichen an der Wand des Tempels.
\bibverse{21} Und die Türpfosten im Tempel waren viereckig, und war
alles artig ineinander gefügt. \bibverse{22} Und der hölzerne Altar war
drei Ellen hoch und zwei Ellen lang und breit, und seine Ecken und alle
seine Seiten waren hölzern. und er sprach zu mir: Das ist der Tisch, der
vor dem HErrn stehen soll. \footnote{\textbf{41:22} 2Mo 30,1-10}
\bibverse{23} Und die Türen am Tempel und am Allerheiligsten
\bibverse{24} hatten zwei Türflügel, und ein jeder derselben hatte zwei
Blätter, die man auf und zu tat.

\bibverse{25} Und waren auch Cherubim und Palmlaubwerk daran wie an den
Wänden. Und ein hölzerner Aufgang war außen vor der Halle.

\bibverse{26} Und es waren enge Fenster und viel Palmlaubwerk herum an
der Halle und an den Wänden. \# 42 \bibverse{1} Und er führte mich
hinaus zum äußeren Vorhof gegen Mitternacht und brachte mich zu den
Kammern, die gegenüber dem Hofraum und gegenüber dem Gebäude nach
Mitternacht zu lagen, \bibverse{2} entlang den hundert Ellen an der Tür
gegen Mitternacht; und ihre Breite war fünfzig Ellen. \bibverse{3}
Gegenüber den zwanzig Ellen des inneren Vorhofs und gegenüber dem
Pflaster im äußeren Vorhof war Umgang an Umgang dreifach. \footnote{\textbf{42:3}
  Hes 41,10; Hes 40,17} \bibverse{4} Und inwendig vor den Kammern war
ein Weg zehn Ellen breit vor den Türen der Kammern; die lagen alle gegen
Mitternacht. \bibverse{5} Und die oberen Kammern waren enger als die
unteren und mittleren Kammern; denn die Umgänge nahmen Raum von ihnen
weg. \bibverse{6} Denn es war drei Gemächer hoch, und sie hatten keine
Säulen, wie die Vorhöfe Säulen hatten. Darum war von den unteren und
mittleren Kammern Raum weggenommen von untenan. \bibverse{7} Und die
Mauer außen vor den Kammern nach dem äußeren Vorhof war fünfzig Ellen
lang. \bibverse{8} Denn die Länge der Kammern nach dem äußeren Vorhof zu
war fünfzig Ellen; aber gegen den Tempel waren es hundert Ellen.
\bibverse{9} Und unten an diesen Kammern war ein Eingang gegen Morgen,
da man aus dem äußeren Vorhof zu ihnen hineinging. \bibverse{10} Und an
der Mauer gegen Mittag waren auch Kammern gegenüber dem Hofraum und
gegenüber dem Gebäude. \bibverse{11} Und war auch ein Weg davor wie vor
jenen Kammern, die gegen Mitternacht lagen; und war alles gleich mit der
Länge, Breite und allem, was daran war, wie droben an jenen.
\bibverse{12} Und wie die Türen jener, also waren auch die Türen der
Kammern gegen Mittag; und am Anfang des Weges war eine Tür, dazu man
kommt von der Mauer, die gegen Morgen liegt. \bibverse{13} Und er sprach
zu mir: Die Kammern gegen Mitternacht und die Kammern gegen Mittag
gegenüber dem Hofraum, das sind die heiligen Kammern, darin die
Priester, welche dem HErrn nahen, die hochheiligen Opfer essen. Und sie
sollen die hochheiligen Opfer, nämlich Speisopfer, Sündopfer und
Schuldopfer, dahineinlegen; denn es ist eine heilige Stätte.
\bibverse{14} Und wenn die Priester hineingehen, sollen sie nicht wieder
aus dem Heiligtum gehen in den äußeren Vorhof, sondern sollen zuvor ihre
Kleider, darin sie gedient haben, in den Kammern weglegen, denn sie sind
heilig; und sollen ihre anderen Kleider anlegen und alsdann heraus
unters Volk gehen.

\bibverse{15} Und da er das Haus inwendig ganz gemessen hatte, führte er
mich heraus zum Tor gegen Morgen und maß von demselben allenthalben
herum.

\bibverse{16} Gegen Morgen maß er 500 Ruten lang;

\bibverse{17} und gegen Mitternacht maß er auch 500 Ruten lang;
\bibverse{18} desgleichen gegen Mittag auch 500 Ruten;

\bibverse{19} und da er kam gegen Abend, maß er auch 500 Ruten lang.

\bibverse{20} Also hatte die Mauer, die er gemessen, ins Gevierte auf
jeder Seite herum 500 Ruten, damit das Heilige von dem Unheiligen
unterschieden wäre. \# 43 \bibverse{1} Und er führte mich wieder zum Tor
gegen Morgen. \bibverse{2} Und siehe, die Herrlichkeit des Gottes
Israels kam von Morgen und brauste, wie ein großes Wasser braust; und es
ward sehr licht auf der Erde von seiner Herrlichkeit. \bibverse{3} Und
es war eben wie das Gesicht, das ich sah, da ich kam, dass die Stadt
sollte zerstört werden, und wie das Gesicht, das ich gesehen hatte am
Wasser Chebar. Da fiel ich nieder auf mein Angesicht. \bibverse{4} Und
die Herrlichkeit des HErrn kam hinein zum Hause durchs Tor gegen Morgen.
\footnote{\textbf{43:4} Hes 11,22-23; Hes 10,19} \bibverse{5} Da hob
mich ein Wind auf und brachte mich in den inneren Vorhof; und siehe, die
Herrlichkeit des HErrn erfüllte das Haus. \footnote{\textbf{43:5} 2Mo
  40,34; 1Kö 8,10-11} \bibverse{6} Und ich hörte einen mit mir reden vom
Hause heraus, und ein Mann stand neben mir. \bibverse{7} Der sprach zu
mir: Du Menschenkind, das ist der Ort meines Throns und die Stätte
meiner Fußsohlen, darin ich ewiglich will wohnen unter den Kindern
Israel. Und das Haus Israel soll nicht mehr meinen heiligen Namen
verunreinigen, weder sie noch ihre Könige, durch ihre Abgötterei und
durch die Leichen ihrer Könige in ihren Höhen, \footnote{\textbf{43:7}
  Ps 132,13-14} \bibverse{8} welche ihre Schwelle an meine Schwelle und
ihre Pfoste an meine Pfoste gesetzt haben, dass nur eine Wand zwischen
mir und ihnen war; und haben also meinen heiligen Namen verunreinigt
durch ihre Gräuel, die sie taten, darum ich sie auch in meinem Zorn
verzehrt habe. \footnote{\textbf{43:8} Hes 8,7-18} \bibverse{9} Nun aber
sollen sie ihre Abgötterei und die Leichen ihrer Könige fern von mir
wegtun; und ich will ewiglich unter ihnen wohnen. \bibverse{10} Und du,
Menschenkind, zeige dem Haus Israel den Tempel an, dass sie sich schämen
ihrer Missetaten, und lass sie ein reinliches Muster davon nehmen.
\footnote{\textbf{43:10} Hes 16,61; Hes 16,63; Hes 36,32} \bibverse{11}
Und wenn sie sich nun alles ihres Tuns schämen, so zeige ihnen die
Gestalt und das Muster des Hauses und seine Ausgänge und Eingänge und
alle seine Weise und alle seine Sitten und alle seine Weise und alle
seine Gesetze; und schreibe es ihnen vor, dass sie alle seine Weise und
alle seine Sitten halten und darnach tun. \bibverse{12} Das soll aber
das Gesetz des Hauses sein: Auf der Höhe des Berges, soweit ihr Umfang
ist, soll das Allerheiligste sein; das ist das Gesetz des Hauses.
\bibverse{13} Das ist aber das Maß des Altars nach der Elle, welche eine
Handbreit länger ist denn eine gemeine Elle: Sein Fuß ist eine Elle hoch
und eine Elle breit; und die Leiste an seinem Rand ist eine Spanne breit
umher.

\bibverse{14} Und das ist seine Höhe: Von dem Fuße auf der Erde bis an
den unteren Absatz sind zwei Ellen hoch und eine Elle breit; aber von
demselben kleineren Absatz bis an den größeren Absatz sind's vier Ellen
hoch und eine Elle breit.

\bibverse{15} Und der Harel (der Gottesberg) vier Ellen hoch, und vom
Ariel (dem Gottesherd) überwärts die vier Hörner.

\bibverse{16} Der Ariel aber war zwölf Ellen lang und zwölf Ellen breit
ins Geviert.

\bibverse{17} Und der oberste Absatz war vierzehn Ellen lang und
vierzehn Ellen breit ins Geviert; und eine Leiste ging allenthalben
umher, eine halbe Elle breit; und sein Fuß war eine Elle hoch, und seine
Stufen waren gegen Morgen.

\bibverse{18} Und er sprach zu mir: Du Menschenkind, so spricht der Herr
HErr: Dies sollen die Sitten des Altars sein des Tages, da er gemacht
ist, dass man Brandopfer darauf lege und Blut darauf sprenge.

\bibverse{19} Und den Priestern von Levi aus dem Samen Zadoks, die da
vor mich treten, dass sie mir dienen, spricht der Herr HErr, sollst du
geben einen jungen Farren zum Sündopfer. \footnote{\textbf{43:19} 2Mo
  29,-1; Hes 40,46}

\bibverse{20} Und von desselben Blut sollst du nehmen und seine vier
Hörner damit besprengen und die vier Ecken an dem obersten Absatz und um
die Leiste herum; damit sollst du ihn entsündigen und versöhnen.

\bibverse{21} Und sollst den Farren des Sündopfers nehmen und ihn
verbrennen an einem Ort am Hause, der dazu verordnet ist außerhalb des
Heiligtums.

\bibverse{22} Aber am anderen Tage sollst du einen Ziegenbock opfern,
der ohne Fehl sei, zu einem Sündopfer und den Altar damit entsündigen,
wie er mit dem Farren entsündigt ist.

\bibverse{23} Und wenn das Entsündigen vollendet ist, sollst du einen
jungen Farren opfern, der ohne Fehl sei, und einen Widder von der Herde
ohne Fehl.

\bibverse{24} Und sollst sie beide vor dem HErrn opfern; und die
Priester sollen Salz darauf streuen und sollen sie also opfern dem HErrn
zum Brandopfer.

\bibverse{25} Also sollst du sieben Tage nacheinander täglich einen Bock
zum Sündopfer opfern; und sie sollen einen jungen Farren und einen
Widder von der Herde, die beide ohne Fehl sind, opfern.

\bibverse{26} Und sollen also sieben Tage lang den Altar versöhnen und
ihn reinigen und ihre Hände füllen.

\bibverse{27} Und nach denselben Tagen sollen die Priester am achten Tag
und hernach für und für auf dem Altar opfern eure Brandopfer und eure
Dankopfer, so will ich euch gnädig sein, spricht der Herr HErr. \# 44
\bibverse{1} Und er führte mich wiederum zu dem äußeren Tor des
Heiligtums gegen Morgen; es war aber zugeschlossen. \bibverse{2} Und der
HErr sprach zu mir: Dies Tor soll zugeschlossen bleiben und nicht
aufgetan werden, und soll niemand dadurchgehen; denn der HErr, der Gott
Israels, ist dadurch eingegangen, darum soll es zugeschlossen bleiben.
\bibverse{3} Doch den Fürsten ausgenommen; denn der Fürst soll
daruntersitzen, das Brot zu essen vor dem HErrn. Durch die Halle des
Tors soll er hineingehen und durch dieselbe wieder herausgehen.
\footnote{\textbf{44:3} Hes 45,7} \bibverse{4} Darnach führte er mich
zum Tor gegen Mitternacht vor das Haus. Und ich sah, und siehe, des
HErrn Haus war voll der Herrlichkeit des HErrn; und ich fiel auf mein
Angesicht. \footnote{\textbf{44:4} Hes 43,5} \bibverse{5} Und der HErr
sprach zu mir: Du Menschenkind, merke darauf und siehe und höre fleißig
auf alles, was ich dir sagen will von allen Sitten und Gesetzen im Haus
des HErrn; und merke, wie man hineingehen soll, und auf alle Ausgänge
des Heiligtums. \bibverse{6} Und sage dem ungehorsamen Hause Israel: So
spricht der Herr HErr: Ihr macht es zuviel, ihr vom Hause Israel, mit
allen euren Gräueln, \bibverse{7} denn ihr führt fremde Leute eines
unbeschnittenen Herzens und unbeschnittenen Fleisches in mein Heiligtum,
dadurch ihr mein Haus entheiligt, wenn ihr mein Brot, Fettes und Blut
opfert, und brecht also meinen Bund mit allen euren Gräueln;
\bibverse{8} und haltet die Sitten meines Heiligtums nicht, sondern
macht euch selbst neue Sitten in meinem Heiligtum. \bibverse{9} Darum
spricht der Herr HErr also: Es soll kein Fremder eines unbeschnittenen
Herzens und unbeschnittenen Fleisches in mein Heiligtum kommen aus allen
Fremdlingen, die unter den Kindern Israel sind;

\bibverse{10} sondern die Leviten, die von mir gewichen sind und samt
Israel von mir irregegangen nach ihren Götzen, die sollen ihre Sünde
tragen, \bibverse{11} und sollen in meinem Heiligtum dienen als Hüter an
den Türen des Hauses und als Diener des Hauses; und sollen nur das
Brandopfer und andere Opfer, die das Volk herzubringt, schlachten und
vor den Leuten stehen, dass sie ihnen dienen. \bibverse{12} Darum dass
sie ihnen gedient vor ihren Götzen und dem Haus Israel einen Anstoß zur
Sünde gegeben haben, darum habe ich meine Hand über sie ausgestreckt,
spricht der Herr HErr, dass sie müssen ihre Sünde tragen.

\bibverse{13} Und sie sollen nicht zu mir nahen, Priesteramt zu führen,
noch kommen zu allen meinen Heiligtümern, zu den hochheiligen Opfern,
sondern sollen ihre Schande tragen und ihre Gräuel, die sie geübt haben.
\bibverse{14} Darum habe ich sie zu Hütern gemacht an allem Dienst des
Hauses und zu allem, was man darin tun soll. \bibverse{15} Aber die
Priester aus den Leviten, die Kinder Zadok, die die Sitten meines
Heiligtums gehalten haben, da die Kinder Israel von mir abfielen, die
sollen vor mich treten und mir dienen und vor mir stehen, dass sie mir
das Fett und Blut opfern, spricht der Herr HErr. \footnote{\textbf{44:15}
  Hes 40,46; Hes 48,11} \bibverse{16} Und sie sollen hineingehen in mein
Heiligtum und vor meinen Tisch treten, mir zu dienen und meine Sitten zu
halten. \bibverse{17} Und wenn sie durch die Tore des inneren Vorhofs
gehen wollen, sollen sie leinene Kleider anziehen und nichts Wollenes
anhaben, wenn sie in den Toren im inneren Vorhofe und im Hause dienen.
\bibverse{18} Und sollen leinenen Schmuck auf ihrem Haupt haben und
leinene Beinkleider um ihre Lenden, und sollen sich nicht im Schweiß
gürten. \bibverse{19} Und wenn sie in den äußeren Vorhof zum Volk
herausgehen, sollen sie die Kleider, darin sie gedient haben, ausziehen
und dieselben in die Kammern des Heiligtums legen und andere Kleider
anziehen und das Volk nicht heiligen in ihren eigenen Kleidern.
\footnote{\textbf{44:19} Hes 42,14} \bibverse{20} Ihr Haupt sollen sie
nicht kahl scheren, und sollen auch nicht die Haare frei wachsen lassen,
sondern sollen die Haare umher verschneiden. \footnote{\textbf{44:20}
  3Mo 19,27; 3Mo 21,5}

\bibverse{21} Und soll auch kein Priester Wein trinken, wenn sie in den
inneren Vorhof gehen sollen. \footnote{\textbf{44:21} 3Mo 10,9}

\bibverse{22} Und sie sollen keine Witwe noch Verstoßene zur Ehe nehmen,
sondern Jungfrauen vom Samen des Hauses Israel oder eines Priesters
nachgelassene Witwe. \footnote{\textbf{44:22} 3Mo 21,7; 3Mo 21,13-14}

\bibverse{23} Und sie sollen mein Volk lehren, dass sie wissen
Unterschied zu halten zwischen Heiligem und Unheiligem und zwischen
Reinem und Unreinem. \footnote{\textbf{44:23} 3Mo 10,10}

\bibverse{24} Und wo eine Sache vor sie kommt, sollen sie stehen und
richten und nach meinen Rechten sprechen und sollen meine Gebote und
Sitten halten und alle meine Feste halten und meine Sabbate heiligen.

\bibverse{25} Und sollen zu keinem Toten gehen und sich verunreinigen,
nur allein zu Vater und Mutter, Sohn oder Tochter, Bruder oder
Schwester, die noch keinen Mann gehabt hat; über denen mögen sie sich
verunreinigen.

\bibverse{26} Und nach seiner Reinigung soll man ihm zählen sieben Tage.

\bibverse{27} Und wenn er wieder hinein zum Heiligtum geht in den
inneren Vorhof, dass er im Heiligtum diene, so soll er sein Sündopfer
opfern, spricht der Herr HErr.

\bibverse{28} Aber das Erbteil, das sie haben sollen, das will ich
selbst sein. Darum sollt ihr ihnen kein eigen Land geben in Israel; denn
ich bin ihr Erbteil. \footnote{\textbf{44:28} 4Mo 18,20}

\bibverse{29} Sie sollen ihre Nahrung haben vom Speisopfer, Sündopfer
und Schuldopfer, und alles Verbannte in Israel soll ihnen gehören.

\bibverse{30} Und alle ersten Früchte und alle Hebopfer von allem, davon
ihr Hebopfer bringt, sollen den Priestern gehören. Ihr sollt auch den
Priestern die Erstlinge eures Teiges geben, damit der Segen in deinem
Hause bleibe.

\bibverse{31} Was aber ein Aas oder zerrissen ist, es sei von Vögeln
oder Tieren, das sollen die Priester nicht essen. \# 45 \bibverse{1}
Wenn ihr nun das Land durchs Los austeilt, so sollt ihr ein Hebopfer vom
Lande absondern, das dem HErrn heilig sein soll, 25.000 Ruten lang und
10.000 breit; der Platz soll heilig sein, soweit er reicht. \bibverse{2}
Und von diesem sollen zum Heiligtum kommen je 500 Ruten ins Gevierte und
dazu ein freier Raum umher 50 Ellen. \bibverse{3} Und auf dem Platz, der
25.000 Ruten lang und 10.000 breit ist, soll das Heiligtum stehen, das
Allerheiligste. \bibverse{4} Das übrige aber vom geheiligten Lande soll
den Priestern gehören, die im Heiligtum dienen und vor den HErrn treten,
ihm zu dienen, dass sie Raum zu Häusern haben, und soll auch heilig
sein. \bibverse{5} Aber die Leviten, die vor dem Hause dienen, sollen
auch 25.000 Ruten lang und 10.000 breit haben zu ihrem Teil, dass sie da
wohnen. \bibverse{6} Und der Stadt sollt ihr auch einen Platz lassen für
das ganze Haus Israel, 5000 Ruten breit und 25.000 lang, neben dem
geheiligten Lande. \bibverse{7} Dem Fürsten aber sollt ihr auch einen
Platz geben zu beiden Seiten, neben dem geheiligten Lande und neben dem
Platz der Stadt, und soll der Platz gegen Abend und gegen Morgen so weit
reichen als die Teile der Stämme. \footnote{\textbf{45:7} Hes 44,3; Hes
  48,21-22} \bibverse{8} Das soll sein eigen Teil sein in Israel, damit
meine Fürsten nicht mehr meinem Volk das Ihre nehmen, sondern sollen das
Land dem Haus Israel lassen für ihre Stämme. \bibverse{9} Denn so
spricht der Herr HErr: Ihr habt's lange genug gemacht, ihr Fürsten
Israels; lasset ab von Frevel und Gewalt und tut, was recht und gut ist,
und tut ab von meinem Volk euer Austreiben, spricht der Herr HErr.
\bibverse{10} Ihr sollt rechtes Gewicht und rechte Scheffel und rechtes
Maß haben. \footnote{\textbf{45:10} 3Mo 19,36; 5Mo 25,15} \bibverse{11}
Epha und Bath sollen gleich sein, dass ein Bath den zehnten Teil vom
Homer habe und das Epha auch den zehnten Teil vom Homer; denn nach dem
Homer soll man sie beide messen. \bibverse{12} Aber ein Lot soll zwanzig
Gera haben; und eine Mina macht zwanzig Lot, fünfundzwanzig Lot und
fünfzehn Lot. \bibverse{13} Das soll nun das Hebopfer sein, das ihr
heben sollt, nämlich den sechsten Teil eines Epha von einem Homer Weizen
und den sechsten Teil eines Epha von einem Homer Gerste. \bibverse{14}
Und vom Öl sollt ihr geben je den zehnten Teil eines Bath vom Kor,
welches zehn Bath oder ein Homer ist; denn zehn Bath machen einen Homer.
\bibverse{15} Und je ein Lamm von zweihundert Schafen aus der Herde auf
der Weide Israels zum Speisopfer und Brandopfer und Dankopfer, zur
Versöhnung für sie, spricht der Herr HErr. \bibverse{16} Alles Volk im
Lande soll solches Hebopfer zum Fürsten in Israel bringen. \bibverse{17}
Und der Fürst soll die Brandopfer, Speisopfer und Trankopfer ausrichten
auf die Feste, Neumonde und Sabbate, auf alle Feiertage des Hauses
Israel; er soll die Sündopfer und Speisopfer, Brandopfer und Dankopfer
tun zur Versöhnung für das Haus Israel. \bibverse{18} So spricht der
Herr HErr: Am ersten Tage des ersten Monats sollst du nehmen einen
jungen Farren, der ohne Fehl sei, und das Heiligtum entsündigen.
\bibverse{19} Und der Priester soll von dem Blut des Sündopfers nehmen
und die Pfosten am Hause damit besprengen und die vier Ecken des
Absatzes am Altar samt den Pfosten am Tor des inneren Vorhofs.
\bibverse{20} Also sollst du auch tun am siebenten Tage des Monats wegen
derer, die geirrt haben oder verführt worden sind, dass ihr das Haus
entsündiget. \bibverse{21} Am vierzehnten Tage des ersten Monats sollt
ihr das Passah halten und sieben Tage feiern und ungesäuertes Brot
essen. \footnote{\textbf{45:21} 3Mo 23,5} \bibverse{22} Und am selben
Tage soll der Fürst für sich und für alles Volk im Lande einen Farren
zum Sündopfer opfern. \bibverse{23} Aber die sieben Tage des Festes soll
er dem HErrn täglich ein Brandopfer tun: je sieben Farren und sieben
Widder, die ohne Fehl seien; und je einen Ziegenbock zum Sündopfer.

\bibverse{24} Zum Speisopfer aber soll er je ein Epha zu einem Farren
und ein Epha zu einem Widder opfern und je ein Hin Öl zu einem Epha.
\footnote{\textbf{45:24} Hes 46,5; 4Mo 15,4; 4Mo 15,6; 4Mo 15,9}
\bibverse{25} Am fünfzehnten Tage des siebenten Monats soll er sieben
Tage nacheinander feiern, gleichwie jene sieben Tage, und es ebenso
halten mit Sündopfer, Brandopfer, Speisopfer samt dem Öl. \footnote{\textbf{45:25}
  3Mo 23,34}

\hypertarget{section-14}{%
\section{46}\label{section-14}}

\bibverse{1} So spricht der Herr HErr: Das Tor am inneren Vorhof
morgenwärts soll die sechs Werktage zugeschlossen sein; aber am
Sabbattage und am Neumonde soll man's auftun. \bibverse{2} Und der Fürst
soll von draußen unter die Halle des Tors treten und bei den Pfosten am
Tor stehen bleiben. Und die Priester sollen sein Brandopfer und
Dankopfer opfern; er aber soll auf der Schwelle des Tors anbeten und
darnach wieder hinausgehen; das Tor aber soll offen bleiben bis an den
Abend. \footnote{\textbf{46:2} Hes 44,3} \bibverse{3} Desgleichen das
Volk im Lande soll an der Tür desselben Tors anbeten vor dem HErrn an
den Sabbaten und Neumonden. \bibverse{4} Das Brandopfer aber, das der
Fürst vor dem HErrn opfern soll am Sabbattage, soll sein sechs Lämmer,
die ohne Fehl seien, und ein Widder ohne Fehl; \bibverse{5} und je ein
Epha zu einem Widder zum Speisopfer, zu den Lämmern aber, soviel seine
Hand gibt, zum Speisopfer, und je ein Hin Öl zu einem Epha. \bibverse{6}
Am Neumonde aber soll er einen jungen Farren opfern, der ohne Fehl sei,
und sechs Lämmer und einen Widder auch ohne Fehl; \bibverse{7} und je
ein Epha zum Farren und je ein Epha zum Widder zum Speisopfer, aber zu
den Lämmern soviel, als er geben mag, und je ein Hin Öl zu einem Epha.
\footnote{\textbf{46:7} Hes 45,24} \bibverse{8} Und wenn der Fürst
hineingeht, soll er durch die Halle des Tors hineingehen und desselben
Weges wieder herausgehen. \bibverse{9} Aber das Volk im Lande, das vor
den HErrn kommt auf die hohen Feste und zum Tor gegen Mitternacht
hineingeht, anzubeten, das soll durch das Tor gegen Mittag wieder
herausgehen; und welche zum Tor gegen Mittag hineingehen, die sollen zum
Tor gegen Mitternacht wieder herausgehen; und sollen nicht wieder zu dem
Tor hinausgehen, dadurch sie hinein sind gegangen, sondern stracks vor
sich hinausgehen. \bibverse{10} Der Fürst aber soll mit ihnen hinein und
heraus gehen. \bibverse{11} Aber an den Feiertagen und hohen Festen soll
man zum Speisopfer je zu einem Farren ein Epha und je zu einem Widder
ein Epha opfern und zu den Lämmern, soviel seine Hand gibt, und je ein
Hin Öl zu einem Epha. \bibverse{12} Wenn aber der Fürst ein freiwilliges
Brandopfer oder Dankopfer dem HErrn tun wollte, so soll man ihm das Tor
gegen Morgen auftun, dass er sein Brandopfer und Dankopfer opfere, wie
er's sonst am Sabbat pflegt zu opfern; und wenn er wieder herausgeht,
soll man das Tor nach ihm zuschließen. \bibverse{13} Und er soll dem
HErrn täglich ein Brandopfer tun, nämlich ein jähriges Lamm ohne Fehl;
dasselbe soll er alle Morgen opfern. \footnote{\textbf{46:13} 4Mo 28,3}
\bibverse{14} Und soll alle Morgen den sechsten Teil von einem Epha zum
Speisopfer darauftun und den dritten Teil von einem Hin Öl, auf das
Semmelmehl zu träufen, dem HErrn zum Speisopfer; das soll ein ewiges
Recht sein vom täglichem Opfer. \bibverse{15} Und also sollen sie das
Lamm samt dem Speisopfer und Öl alle Morgen opfern zum täglichen
Brandopfer. \bibverse{16} So spricht der Herr HErr: Wenn der Fürst
seiner Söhne einem ein Geschenk gibt von seinem Erbe, dasselbe soll
seinen Söhnen bleiben, und sie sollen es erblich besitzen. \bibverse{17}
Wo er aber seiner Knechte einem von seinem Erbteil etwas schenkt, das
sollen sie besitzen bis aufs Freijahr und soll alsdann dem Fürsten
wieder heimfallen; denn sein Teil soll allein auf seine Söhne erben.
\bibverse{18} Es soll auch der Fürst dem Volk nichts nehmen von seinem
Erbteil noch sie aus ihren eigenen Gütern stoßen, sondern soll sein
eigenes Gut auf seine Kinder vererben, auf dass meines Volks nicht
jemand von seinem Eigentum zerstreut werde. \footnote{\textbf{46:18} Hes
  45,8-9} \bibverse{19} Und er führte mich durch den Eingang an der
Seite des Tors gegen Mitternacht zu den Kammern des Heiligtums, die den
Priestern gehörten; und siehe, daselbst war ein Raum in der Ecke gegen
Abend. \bibverse{20} Und er sprach zu mir: Dies ist der Ort, da die
Priester kochen sollen das Schuldopfer und Sündopfer und das Speisopfer
backen, dass sie es nicht hinaus in den äußeren Vorhof tragen müssen,
das Volk zu heiligen. \bibverse{21} Darnach führte er mich hinaus in den
äußeren Vorhof und hieß mich gehen in die vier Ecken des Vorhofs.
\bibverse{22} Und siehe, da war in jeglicher der vier Ecken ein anderes
Vorhöflein, vierzig Ellen lang und dreißig Ellen breit, alle vier
einerlei Maßes. \bibverse{23} Und es ging ein Mäuerlein um ein jegliches
der vier; da waren Herde herum gemacht unten an den Mauern.
\bibverse{24} Und er sprach zu mir: Dies sind die Küchen, darin die
Diener des Hauses kochen sollen, was das Volk opfert. \# 47 \bibverse{1}
Und er führte mich wieder zu der Tür des Tempels. Und siehe, da floss
ein Wasser heraus unter der Schwelle des Tempels gegen Morgen; denn die
vordere Seite des Tempels war gegen Morgen. Und das Wasser lief an der
rechten Seite des Tempels neben dem Altar hin gegen Mittag. \bibverse{2}
Und er führte mich hinaus zum Tor gegen Mitternacht und brachte mich
auswendig herum zum äußeren Tor gegen Morgen; und siehe, das Wasser
sprang heraus von der rechten Seite. \bibverse{3} Und der Mann ging
heraus gegen Morgen und hatte die Messschnur in der Hand; und er maß
tausend Ellen und führte mich durchs Wasser, dass mir's an die Knöchel
ging. \footnote{\textbf{47:3} Hes 40,3} \bibverse{4} Und maß abermals
tausend Ellen und führte mich durchs Wasser, dass mir's an die Knie
ging. Und maß noch tausend Ellen und ließ mich dadurch gehen, dass es
mir an die Lenden ging. \bibverse{5} Da maß er noch tausend Ellen, und
es ward so tief, dass ich nicht mehr gründen konnte; denn das Wasser war
zu hoch, dass man darüber schwimmen musste und es nicht gründen konnte.
\bibverse{6} Und er sprach zu mir: Du Menschenkind, das hast du ja
gesehen. Und er führte mich wieder zurück am Ufer des Bachs.
\bibverse{7} Und siehe, da standen sehr viel Bäume am Ufer auf beiden
Seiten. \bibverse{8} Und er sprach zu mir: Dies Wasser, das da gegen
Morgen herausfließt, wird durchs Blachfeld fließen ins Meer; und wenn's
dahin ins Meer kommt, da sollen desselben Wasser gesund werden.
\bibverse{9} Ja, alles, was darin lebt und webt, dahin diese Ströme
kommen, das soll leben; und es soll sehr viel Fische haben; und soll
alles gesund werden und leben, wo dieser Strom hin kommt. \bibverse{10}
Und es werden die Fischer an demselben stehen; von Engedi bis zu
En-Eglaim wird man die Fischgarne aufspannen; denn es werden daselbst
sehr viel Fische von allerlei Art sein, gleichwie im großen Meer.
\bibverse{11} Aber die Teiche und Lachen daneben werden nicht gesund
werden, sondern gesalzen bleiben. \bibverse{12} Und an demselben Strom,
am Ufer auf beiden Seiten, werden allerlei fruchtbare Bäume wachsen, und
ihre Blätter werden nicht verwelken noch ihre Früchte ausgehen; und sie
werden alle Monate neue Früchte bringen, denn ihr Wasser fließt aus dem
Heiligtum. Ihre Frucht wird zur Speise dienen und ihre Blätter zur
Arznei. \footnote{\textbf{47:12} Offb 22,2} \bibverse{13} So spricht der
Herr HErr: Dies sind die Grenzen, nach denen ihr das Land sollt
austeilen den zwölf Stämmen Israels; denn zwei Teile gehören dem Stamm
Joseph. \footnote{\textbf{47:13} 1Mo 48,5; Jos 17,17} \bibverse{14} Und
ihr sollt's gleich austeilen, einem wie dem anderen; denn ich habe meine
Hand aufgehoben, das Land euren Vätern und euch zum Erbteil zu geben.
\bibverse{15} Dies ist nun die Grenze des Landes gegen Mitternacht: von
dem großen Meer an des Weges nach Hethlon gen Zedad, \footnote{\textbf{47:15}
  4Mo 34,2-12} \bibverse{16} Hamath, Berotha, Sibraim, das an Damaskus
und Hamath grenzt, und Hazar-Thichon, das an Hauran grenzt.
\bibverse{17} Das soll die Grenze sein vom Meer an bis gen Hazar-Enon,
und Damaskus und Hamath sollen das Ende sein. Das sei die Grenze gegen
Mitternacht. \bibverse{18} Aber die Grenze gegen Morgen sollt ihr messen
zwischen Hauran und Damaskus und zwischen Gilead und dem Lande Israel,
am Jordan hinab bis an das Meer gegen Morgen. Das soll die Grenze gegen
Morgen sein. \bibverse{19} Aber die Grenze gegen Mittag ist von Thamar
bis ans Haderwasser zu Kades und den Bach hinab bis an das große Meer.
Das soll die Grenze gegen Mittag sein. \bibverse{20} Und an der Seite
gegen Abend ist das große Meer von der Grenze an bis gegenüber Hamath.
Das sei die Grenze gegen Abend. \bibverse{21} Also sollt ihr das Land
austeilen unter die Stämme Israels. \bibverse{22} Und wenn ihr das Los
werft, das Land unter euch zu teilen, so sollt ihr die Fremdlinge, die
bei euch wohnen und Kinder unter euch zeugen, halten gleich wie die
Einheimischen unter den Kindern Israel; \footnote{\textbf{47:22} 2Mo
  22,20} \bibverse{23} und sie sollen auch ihren Teil im Lande haben,
ein jeglicher unter dem Stamm, dabei er wohnt, spricht der Herr HErr. \#
48 \bibverse{1} Dies sind die Namen der Stämme: von Mitternacht, an dem
Wege nach Hethlon, gen Hamath und Hazar-Enon und von Damaskus gegen
Hamath, das soll Dan für seinen Teil haben, von Morgen bis gen Abend.
\bibverse{2} Neben Dan soll Asser seinen Teil haben, von Morgen bis gen
Abend. \bibverse{3} Neben Asser soll Naphthali seinen Teil haben, von
Morgen bis gen Abend. \bibverse{4} Neben Naphthali soll Manasse seinen
Teil haben, von Morgen bis gen Abend. \bibverse{5} Neben Manasse soll
Ephraim seinen Teil haben, von Morgen bis gen Abend. \bibverse{6} Neben
Ephraim soll Ruben seinen Teil haben, von Morgen bis gen Abend.
\bibverse{7} Neben Ruben soll Juda seinen Teil haben, von Morgen bis gen
Abend. \bibverse{8} Neben Juda aber sollt ihr einen Teil absondern, von
Morgen bis gen Abend, der 25.000 Ruten breit und so lang sei, wie sonst
ein Teil ist von Morgen bis gen Abend; darin soll das Heiligtum stehen.
\footnote{\textbf{48:8} Hes 45,1-8} \bibverse{9} Und davon sollt ihr dem
HErrn einen Teil absondern, 25.000 Ruten lang und 10.000 Ruten breit.
\bibverse{10} Und dieser heilige Teil soll den Priestern gehören,
nämlich 25.000 Ruten lang gegen Mitternacht und gegen Mittag und 10.000
breit gegen Morgen und gegen Abend. Und das Heiligtum des HErrn soll
mittendarin stehen. \bibverse{11} Das soll geheiligt sein den Priestern,
den Kindern Zadok, welche meine Sitten gehalten haben und sind nicht
abgefallen mit den Kindern Israel, wie die Leviten abgefallen sind.
\bibverse{12} Und soll also dieser abgesonderte Teil des geheiligten
Landes ihr eigen sein als Hochheiliges neben der Leviten Grenze.
\bibverse{13} Die Leviten aber sollen neben der Priester Grenze auch
25.000 Ruten in die Länge und 10.000 Ruten in die Breite haben; denn
alle Länge soll 25.000 und die Breite 10.000 Ruten haben. \bibverse{14}
Und sollen nichts davon verkaufen noch verändern, damit des Landes
Erstling nicht wegkomme; denn es ist dem HErrn geheiligt. \bibverse{15}
Aber die übrigen 5000 Ruten in die Breite gegen 25.000 Ruten in die
Länge, das soll gemeines Land sein zur Stadt, darin zu wohnen, und zu
Vorstädten; und die Stadt soll mittendarin stehen. \bibverse{16} Und das
soll ihr Maß sein: 4500 Ruten gegen Mitternacht und gegen Mittag,
desgleichen gegen Morgen und gegen Abend auch 4500. \footnote{\textbf{48:16}
  Offb 21,16} \bibverse{17} Die Vorstadt aber soll haben 250 Ruten gegen
Mitternacht und gegen Mittag, desgleichen auch gegen Morgen und gegen
Abend 250 Ruten. \bibverse{18} Aber das Übrige an der Länge neben dem
Abgesonderten und Geheiligten, nämlich 10.000 Ruten gegen Morgen und
10.000 Ruten gegen Abend, das gehört zum Unterhalt derer, die in der
Stadt arbeiten. \bibverse{19} Und die Arbeiter aus allen Stämmen Israels
sollen in der Stadt arbeiten. \bibverse{20} Also soll die ganze
Absonderung 25.000 Ruten ins Gevierte sein; ein Vierteil der geheiligten
Absonderung sei zu eigen der Stadt. \bibverse{21} Was aber noch übrig
ist auf beiden Seiten neben dem abgesonderten heiligen Teil und neben
der Stadt Teil, nämlich 25.000 Ruten gegen Morgen und gegen Abend neben
den Teilen der Stämme, das soll alles dem Fürsten gehören. Aber der
abgesonderte heilige Teil und das Haus des Heiligtums soll mitteninnen
sein. \bibverse{22} Was aber neben der Leviten Teil und neben der Stadt
Teil zwischen der Grenze Judas und der Grenze Benjamins liegt, das soll
dem Fürsten gehören. \bibverse{23} Darnach sollen die anderen Stämme
sein: Benjamin soll seinen Teil haben, von Morgen bis gen Abend.
\bibverse{24} Aber neben der Grenze Benjamins soll Simeon seinen Teil
haben, von Morgen bis gen Abend. \bibverse{25} Neben der Grenze Simeons
soll Isaschar seinen Teil haben, von Morgen bis gen Abend. \bibverse{26}
Neben der Grenze Isaschars soll Sebulon seinen Teil haben, von Morgen
bis gen Abend. \bibverse{27} Neben der Grenze Sebulons soll Gad seinen
Teil haben, von Morgen bis gen Abend.

\bibverse{28} Aber neben Gad ist die Grenze gegen Mittag von Thamar bis
ans Haderwasser zu Kades und den Bach hinab bis an das große Meer.
\footnote{\textbf{48:28} Hes 47,19}

\bibverse{29} Das ist das Land, das ihr austeilen sollt zum Erbteil
unter die Stämme Israels; und das sollen ihre Erbteile sein, spricht der
Herr HErr.

\bibverse{30} Und so weit soll die Stadt sein: 4500 Ruten gegen
Mitternacht.

\bibverse{31} Und die Tore der Stadt sollen nach den Namen der Stämme
Israels genannt werden, drei Toren gegen Mitternacht: das erste Tor
Ruben, das zweite Juda, das dritte Levi.

\bibverse{32} Also auch gegen Morgen 4500 Ruten und auch drei Tore:
nämlich das erste Tor Joseph, das zweite Benjamin, das dritte Dan.

\bibverse{33} Gegen Mittag auch also 4500 Ruten und auch drei Tore: das
erste Tor Simeon, das zweite Isaschar, das dritte Sebulon.

\bibverse{34} Also auch gegen Abend 4500 Ruten und drei Tore: ein Tor
Gad, das andere Asser, das dritte Naphthali.

\bibverse{35} Also sollen es um und um 18.000 Ruten sein. Und alsdann
soll die Stadt genannt werden: „Hier ist der HErr``.
