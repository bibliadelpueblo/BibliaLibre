\hypertarget{section}{%
\section{1}\label{section}}

\bibverse{1} Nachdem vorzeiten Gott manchmal und mancherleiweise geredet
hat zu den Vätern durch die Propheten, \bibverse{2} hat er am letzten in
diesen Tagen zu uns geredet durch den Sohn, welchen er gesetzt hat zum
Erben über alles, durch welchen er auch die Welt gemacht hat;
\bibverse{3} welcher, sintemal er ist der Glanz seiner Herrlichkeit und
das Ebenbild seines Wesens und trägt alle Dinge mit seinem kräftigen
Wort und hat gemacht die Reinigung unsrer Sünden durch sich selbst, hat
er sich gesetzt zu der Rechten der Majestät in der Höhe \bibverse{4} und
ist so viel besser geworden den die Engel, so viel höher der Name ist,
den er von ihnen ererbt hat. \bibverse{5} Denn zu welchem Engel hat er
jemals gesagt: ``Du bist mein lieber Sohn, heute habe ich dich
gezeugt''? und abermals: ``Ich werde sein Vater sein, und er wird mein
Sohn sein''? \bibverse{6} Und abermals, da er einführt den Erstgeborenen
in die Welt, spricht er: ``Und es sollen ihn alle Engel Gottes
anbeten.'' \bibverse{7} Von den Engeln spricht er zwar: ``Er macht seine
Engel zu Winden und seine Diener zu Feuerflammen'', \bibverse{8} aber
von dem Sohn: ``Gott, dein Stuhl währt von Ewigkeit zu Ewigkeit; das
Zepter deines Reichs ist ein richtiges Zepter. \bibverse{9} Du hast
geliebt die Gerechtigkeit und gehaßt die Ungerechtigkeit; darum hat
dich, o Gott, gesalbt dein Gott mit dem Öl der Freuden über deine
Genossen.'' \bibverse{10} Und: ``Du, HERR, hast von Anfang die Erde
gegründet, und die Himmel sind deiner Hände Werk. \bibverse{11} Sie
werden vergehen, du aber wirst bleiben. Und sie werden alle veralten wie
ein Kleid; \bibverse{12} und wie ein Gewand wirst du sie wandeln, und
sie werden sich verwandeln. Du aber bist derselbe, und deine Jahre
werden nicht aufhören.'' \bibverse{13} Zu welchem Engel aber hat er
jemals gesagt: ``Setze dich zu meiner Rechten, bis ich lege deine Feinde
zum Schemel deiner Füße''? \bibverse{14} Sind sie nicht allzumal
dienstbare Geister, ausgesandt zum Dienst um derer willen, die ererben
sollen die Seligkeit?

\hypertarget{section-1}{%
\section{2}\label{section-1}}

\bibverse{1} Darum sollen wir desto mehr wahrnehmen des Worts, das wir
hören, damit wir nicht dahinfahren. \bibverse{2} Denn so das Wort
festgeworden ist, das durch die Engel geredet ist, und eine jegliche
Übertretung und jeder Ungehorsam seinen rechten Lohn empfangen hat,
\bibverse{3} wie wollen wir entfliehen, so wir eine solche Seligkeit
nicht achten? welche, nachdem sie zuerst gepredigt ist durch den HERRN,
auf uns gekommen ist durch die, so es gehört haben; \bibverse{4} und
Gott hat ihr Zeugnis gegeben mit Zeichen, Wundern und mancherlei Kräften
und mit Austeilung des heiligen Geistes nach seinem Willen. \bibverse{5}
Denn er hat nicht den Engeln untergetan die zukünftige Welt, davon wir
reden. \bibverse{6} Es bezeugt aber einer an einem Ort und spricht:
``Was ist der Mensch, daß du sein gedenkest, und des Menschen Sohn, daß
du auf ihn achtest? \bibverse{7} Du hast ihn eine kleine Zeit niedriger
sein lassen denn die Engel; mit Preis und Ehre hast du ihn gekrönt und
hast ihn gesetzt über die Werke deiner Hände; \bibverse{8} alles hast du
unter seine Füße getan.'' In dem, daß er ihm alles hat untergetan, hat
er nichts gelassen, das ihm nicht untertan sei; jetzt aber sehen wir
noch nicht, daß ihm alles untertan sei. \bibverse{9} Den aber, der eine
kleine Zeit niedriger gewesen ist als die Engel, Jesum, sehen wir durchs
Leiden des Todes gekrönt mit Preis und Ehre, auf daß er von Gottes
Gnaden für alle den Tod schmeckte. \bibverse{10} Denn es ziemte dem, um
deswillen alle Dinge sind und durch den alle Dinge sind, der da viel
Kinder hat zur Herrlichkeit geführt, daß er den Herzog der Seligkeit
durch Leiden vollkommen machte. \bibverse{11} Sintemal sie alle von
einem kommen, beide, der da heiligt und die da geheiligt werden. Darum
schämt er sich auch nicht, sie Brüder zu heißen, \bibverse{12} und
spricht: ``Ich will verkündigen deinen Namen meinen Brüdern und mitten
in der Gemeinde dir lobsingen.'' \bibverse{13} Und abermals: ``Ich will
mein Vertrauen auf ihn setzen.'' und abermals: ``Siehe da, ich und die
Kinder, welche mir Gott gegeben hat.'' \bibverse{14} Nachdem nun die
Kinder Fleisch und Blut haben, ist er dessen gleichermaßen teilhaftig
geworden, auf daß er durch den Tod die Macht nehme dem, der des Todes
Gewalt hatte, das ist dem Teufel, \bibverse{15} und erlöste die, so
durch Furcht des Todes im ganzen Leben Knechte sein mußten.
\bibverse{16} Denn er nimmt sich ja nicht der Engel an, sondern des
Samens Abrahams nimmt er sich an. \bibverse{17} Daher mußte er in allen
Dingen seinen Brüdern gleich werden, auf daß er barmherzig würde und ein
treuer Hoherpriester vor Gott, zu versöhnen die Sünden des Volks.
\bibverse{18} Denn worin er gelitten hat und versucht ist, kann er
helfen denen, die versucht werden.

\hypertarget{section-2}{%
\section{3}\label{section-2}}

\bibverse{1} Derhalben, ihr heiligen Brüder, die ihr mit berufen seid
durch die himmlische Berufung, nehmet wahr des Apostels und
Hohenpriesters, den wir bekennen, Christus Jesus, \bibverse{2} der da
treu ist dem, der ihn gemacht hat, wie auch Mose in seinem ganzen Hause.
\bibverse{3} Dieser aber ist größerer Ehre wert denn Mose, soviel
größere Ehre denn das Haus der hat, der es bereitete. \bibverse{4} Denn
ein jeglich Haus wird von jemand bereitet; der aber alles bereitet hat,
das ist Gott. \bibverse{5} Und Mose war zwar treu in seinem ganzen Hause
als ein Knecht, zum Zeugnis des, das gesagt sollte werden, \bibverse{6}
Christus aber als ein Sohn über sein Haus; des Haus sind wir, so wir
anders das Vertrauen und den Ruhm der Hoffnung bis ans Ende fest
behalten. \bibverse{7} Darum, wie der heilige Geist spricht: ``Heute, so
ihr hören werdet seine Stimme, \bibverse{8} so verstocket eure Herzen
nicht, wie geschah in der Verbitterung am Tage der Versuchung in der
Wüste, \bibverse{9} da mich eure Väter versuchten; sie prüften mich und
sahen meine Werke vierzig Jahre lang. \bibverse{10} Darum ward ich
entrüstet über dies Geschlecht und sprach: Immerdar irren sie mit dem
Herzen! Aber sie erkannten meine Wege nicht, \bibverse{11} daß ich auch
schwur in meinem Zorn, sie sollten zu meiner Ruhe nicht kommen.''
\bibverse{12} Sehet zu, liebe Brüder, daß nicht jemand unter euch ein
arges, ungläubiges Herz habe, das da abtrete von dem lebendigen Gott;
\bibverse{13} sondern ermahnet euch selbst alle Tage, solange es
``heute'' heißt, daß nicht jemand unter euch verstockt werde durch
Betrug der Sünde. \bibverse{14} Denn wir sind Christi teilhaftig
geworden, so wir anders das angefangene Wesen bis ans Ende fest
behalten. \bibverse{15} Indem gesagt wird: ``Heute, so ihr seine Stimme
hören werdet, so verstocket eure Herzen nicht, wie in der Verbitterung
geschah'': \bibverse{16} welche denn hörten sie und richteten eine
Verbitterung an? Waren's nicht alle, die von Ägypten ausgingen durch
Mose? \bibverse{17} Über welche aber ward er entrüstet vierzig Jahre
lang? Ist's nicht über die, so da sündigten, deren Leiber in der Wüste
verfielen? \bibverse{18} Welchen schwur er aber, daß sie nicht zur Ruhe
kommen sollten, wenn nicht den Ungläubigen? \bibverse{19} Und wir sehen,
daß sie nicht haben können hineinkommen um des Unglaubens willen.

\hypertarget{section-3}{%
\section{4}\label{section-3}}

\bibverse{1} So lasset uns nun fürchten, daß wir die Verheißung,
einzukommen zu seiner Ruhe, nicht versäumen und unser keiner dahinten
bleibe. \bibverse{2} Denn es ist uns auch verkündigt gleichwie jenen;
aber das Wort der Predigt half jenen nichts, da nicht glaubten die, so
es hörten. \bibverse{3} Denn wir, die wir glauben, gehen in die Ruhe,
wie er spricht: ``Daß ich schwur in meinem Zorn, sie sollten zu meiner
Ruhe nicht kommen.'' Und zwar, da die Werke von Anbeginn der Welt
gemacht waren, \bibverse{4} sprach er an einem Ort von dem siebenten Tag
also: ``Und Gott ruhte am siebenten Tage von allen seinen Werken;''
\bibverse{5} und hier an diesem Ort abermals: ``Sie sollen nicht kommen
zu meiner Ruhe.'' \bibverse{6} Nachdem es nun noch vorhanden ist, daß
etliche sollen zu ihr kommen, und die, denen es zuerst verkündigt ist,
sind nicht dazu gekommen um des Unglaubens willen, \bibverse{7} bestimmt
er abermals einen Tag nach solcher langen Zeit und sagt durch David:
``Heute,'' wie gesagt ist, ``so ihr seine Stimme hören werdet, so
verstocket eure Herzen nicht.'' \bibverse{8} Denn so Josua hätte sie zur
Ruhe gebracht, würde er nicht hernach von einem andern Tage gesagt
haben. \bibverse{9} Darum ist noch eine Ruhe vorhanden dem Volke Gottes.
\bibverse{10} Denn wer zu seiner Ruhe gekommen ist, der ruht auch von
seinen Werken gleichwie Gott von seinen. \bibverse{11} So lasset uns nun
Fleiß tun, einzukommen zu dieser Ruhe, auf daß nicht jemand falle in
dasselbe Beispiel des Unglaubens. \bibverse{12} Denn das Wort Gottes ist
lebendig und kräftig und schärfer denn kein zweischneidig Schwert, und
dringt durch, bis daß es scheidet Seele und Geist, auch Mark und Bein,
und ist ein Richter der Gedanken und Sinne des Herzens. \bibverse{13}
Und keine Kreatur ist vor ihm unsichtbar, es ist aber alles bloß und
entdeckt vor seinen Augen. Von dem reden wir. \bibverse{14} Dieweil wir
denn einen großen Hohenpriester haben, Jesum, den Sohn Gottes, der gen
Himmel gefahren ist, so lasset uns halten an dem Bekenntnis.
\bibverse{15} Denn wir haben nicht einen Hohenpriester, der nicht könnte
Mitleiden haben mit unsern Schwachheiten, sondern der versucht ist
allenthalben gleichwie wir, doch ohne Sünde. \bibverse{16} Darum laßt
uns hinzutreten mit Freudigkeit zu dem Gnadenstuhl, auf daß wir
Barmherzigkeit empfangen und Gnade finden auf die Zeit, wenn uns Hilfe
not sein wird.

\hypertarget{section-4}{%
\section{5}\label{section-4}}

\bibverse{1} Denn ein jeglicher Hoherpriester, der aus den Menschen
genommen wird, der wird gesetzt für die Menschen gegen Gott, auf daß er
opfere Gaben und Opfer für die Sünden; \bibverse{2} der da könnte
mitfühlen mit denen, die da unwissend sind und irren, dieweil er auch
selbst umgeben ist mit Schwachheit. \bibverse{3} Darum muß er auch,
gleichwie für das Volk, also auch für sich selbst opfern für die Sünden.
\bibverse{4} Und niemand nimmt sich selbst die Ehre, sondern er wird
berufen von Gott gleichwie Aaron. \bibverse{5} Also auch Christus hat
sich nicht selbst in die Ehre gesetzt, daß er Hoherpriester würde,
sondern der zu ihm gesagt hat: ``Du bist mein lieber Sohn, heute habe
ich dich gezeuget.'' \bibverse{6} Wie er auch am andern Ort spricht:
``Du bist ein Priester in Ewigkeit nach der Ordnung Melchisedeks.''
\bibverse{7} Und er hat in den Tagen seines Fleisches Gebet und Flehen
mit starkem Geschrei und Tränen geopfert zu dem, der ihm von dem Tode
konnte aushelfen; und ist auch erhört, darum daß er Gott in Ehren hatte.
\bibverse{8} Und wiewohl er Gottes Sohn war, hat er doch an dem, was er
litt Gehorsam gelernt. \bibverse{9} Und da er vollendet war, ist er
geworden allen, die ihm gehorsam sind, eine Ursache zur ewigen
Seligkeit. \bibverse{10} genannt von Gott ein Hoherpriester nach der
Ordnung Melchisedeks. \bibverse{11} Davon hätten wir wohl viel zu reden;
aber es ist schwer, weil ihr so unverständig seid. \bibverse{12} Und die
ihr solltet längst Meister sein, bedürft wiederum, daß man euch die
ersten Buchstaben der göttlichen Worte lehre und daß man euch Milch gebe
und nicht starke Speise. \bibverse{13} Denn wem man noch Milch geben
muß, der ist unerfahren in dem Wort der Gerechtigkeit; denn er ist ein
junges Kind. \bibverse{14} Den Vollkommenen aber gehört starke Speise,
die durch Gewohnheit haben geübte Sinne zu unterscheiden Gutes und
Böses.

\hypertarget{section-5}{%
\section{6}\label{section-5}}

\bibverse{1} Darum wollen wir die Lehre vom Anfang christlichen Lebens
jetzt lassen und zur Vollkommenheit fahren, nicht abermals Grund legen
von Buße der toten Werke, vom Glauben an Gott, \bibverse{2} von der
Taufe, von der Lehre, vom Händeauflegen, von der Toten Auferstehung und
vom ewigen Gericht. \bibverse{3} Und das wollen wir tun, so es Gott
anders zuläßt. \bibverse{4} Denn es ist unmöglich, die, so einmal
erleuchtet sind und geschmeckt haben die himmlische Gabe und teilhaftig
geworden sind des heiligen Geistes \bibverse{5} und geschmeckt haben das
gütige Wort Gottes und die Kräfte der zukünftigen Welt, \bibverse{6} wo
sie abfallen, wiederum zu erneuern zur Buße, als die sich selbst den
Sohn Gottes wiederum kreuzigen und für Spott halten. \bibverse{7} Denn
die Erde, die den Regen trinkt, der oft über sie kommt, und nützliches
Kraut trägt denen, die sie bauen, empfängt Segen von Gott. \bibverse{8}
Welche aber Dornen und Disteln trägt, die ist untüchtig und dem Fluch
nahe, daß man sie zuletzt verbrennt. \bibverse{9} Wir versehen uns aber,
ihr Liebsten, eines Besseren zu euch und daß die Seligkeit näher sei, ob
wir wohl also reden. \bibverse{10} Denn Gott ist nicht ungerecht, daß er
vergesse eures Werks und der Arbeit der Liebe, die ihr erzeigt habt an
seinem Namen, da ihr den Heiligen dientet und noch dienet. \bibverse{11}
Wir begehren aber, daß euer jeglicher denselben Fleiß beweise, die
Hoffnung festzuhalten bis ans Ende, \bibverse{12} daß ihr nicht träge
werdet, sondern Nachfolger derer, die durch Glauben und Geduld ererben
die Verheißungen. \bibverse{13} Denn als Gott Abraham verhieß, da er bei
keinem Größeren zu schwören hatte, schwur er bei sich selbst
\bibverse{14} und sprach: ``Wahrlich, ich will dich segnen und
vermehren.'' \bibverse{15} Und also trug er Geduld und erlangte die
Verheißung. \bibverse{16} Die Menschen schwören ja bei einem Größeren,
denn sie sind; und der Eid macht ein Ende alles Haders, dabei es fest
bleibt unter ihnen. \bibverse{17} So hat Gott, da er wollte den Erben
der Verheißung überschwenglich beweisen, daß sein Rat nicht wankte,
einen Eid dazu getan, \bibverse{18} auf daß wir durch zwei Stücke, die
nicht wanken (denn es ist unmöglich, daß Gott lüge), einen starken Trost
hätten, die wir Zuflucht haben und halten an der angebotenen Hoffnung,
\bibverse{19} welche wir haben als einen sichern und festen Anker unsrer
Seele, der auch hineingeht in das Inwendige des Vorhangs, \bibverse{20}
dahin der Vorläufer für uns eingegangen, Jesus, ein Hoherpriester
geworden in Ewigkeit nach der Ordnung Melchisedeks.

\hypertarget{section-6}{%
\section{7}\label{section-6}}

\bibverse{1} Dieser Melchisedek aber war ein König von Salem, ein
Priester Gottes, des Allerhöchsten, der Abraham entgegenging, da er von
der Könige Schlacht wiederkam, und segnete ihn; \bibverse{2} welchem
auch Abraham gab den Zehnten aller Güter. Aufs erste wird er
verdolmetscht: ein König der Gerechtigkeit; darnach aber ist er auch ein
König Salems, das ist: ein König des Friedens; \bibverse{3} ohne Vater,
ohne Mutter, ohne Geschlecht und hat weder Anfang der Tage noch Ende des
Lebens: er ist aber verglichen dem Sohn Gottes und bleibt Priester in
Ewigkeit. \bibverse{4} Schauet aber, wie groß ist der, dem auch Abraham,
der Patriarch, den Zehnten gibt von der eroberten Beute! \bibverse{5}
Zwar die Kinder Levi, die das Priestertum empfangen, haben ein Gebot,
den Zehnten vom Volk, das ist von ihren Brüdern, zu nehmen nach dem
Gesetz, wiewohl auch diese aus den Lenden Abrahams gekommen sind.
\bibverse{6} Aber der, des Geschlecht nicht genannt wird unter ihnen,
der nahm den Zehnten von Abraham und segnete den, der die Verheißungen
hatte. \bibverse{7} Nun ist's ohne alles Widersprechen also, daß das
Geringere von dem Besseren gesegnet wird; \bibverse{8} und hier nehmen
die Zehnten die sterbenden Menschen, aber dort einer, dem bezeugt wird,
daß er lebe. \bibverse{9} Und, daß ich also sage, es ist auch Levi, der
den Zehnten nimmt, verzehntet durch Abraham, \bibverse{10} denn er war
ja noch in den Lenden des Vaters, da ihm Melchisedek entgegenging.
\bibverse{11} `1487' Ist nun die Vollkommenheit durch das levitische
Priestertum geschehen (denn unter demselben hat das Volk das Gesetz
empfangen), was ist denn weiter not zu sagen, daß ein anderer Priester
aufkommen solle nach der Ordnung Melchisedeks und nicht nach der Ordnung
Aarons? \bibverse{12} Denn wo das Priestertum verändert wird, da muß
auch das Gesetz verändert werden. \bibverse{13} Denn von dem solches
gesagt ist, der ist von einem andern Geschlecht, aus welchem nie einer
des Altars gewartet hat. \bibverse{14} Denn es ist offenbar, daß von
Juda aufgegangen ist unser HERR, zu welchem Geschlecht Mose nichts
geredet hat vom Priestertum. \bibverse{15} Und es ist noch viel klarer,
so nach der Weise Melchisedeks ein andrer Priester aufkommt,
\bibverse{16} welcher nicht nach dem Gesetz des fleischlichen Gebots
gemacht ist, sondern nach der Kraft des unendlichen Lebens.
\bibverse{17} Denn er bezeugt: ``Du bist ein Priester ewiglich nach der
Ordnung Melchisedeks.'' \bibverse{18} Denn damit wird das vorige Gebot
aufgehoben, darum daß es zu schwach und nicht nütze war \bibverse{19}
(denn das Gesetz konnte nichts vollkommen machen); und wird eingeführt
eine bessere Hoffnung, durch welche wir zu Gott nahen; \bibverse{20} und
dazu, was viel ist, nicht ohne Eid. Denn jene sind ohne Eid Priester
geworden, \bibverse{21} dieser aber mit dem Eid, durch den, der zu ihm
spricht: ``Der HERR hat geschworen, und es wird ihn nicht gereuen: Du
bist ein Priester in Ewigkeit nach der Ordnung Melchisedeks.''
\bibverse{22} Also eines so viel besseren Testaments Ausrichter ist
Jesus geworden. \bibverse{23} Und jener sind viele, die Priester wurden,
darum daß sie der Tod nicht bleiben ließ; \bibverse{24} dieser aber hat
darum, daß er ewiglich bleibt, ein unvergängliches Priestertum.
\bibverse{25} Daher kann er auch selig machen immerdar, die durch ihn zu
Gott kommen, und lebt immerdar und bittet für sie. \bibverse{26} Denn
einen solchen Hohenpriester sollten wir haben, der da wäre heilig,
unschuldig, unbefleckt, von den Sünden abgesondert und höher, denn der
Himmel ist; \bibverse{27} dem nicht täglich not wäre, wie jenen
Hohenpriestern, zuerst für eigene Sünden Opfer zu tun, darnach für des
Volkes Sünden; denn das hat er getan einmal, da er sich selbst opferte.
\bibverse{28} denn das Gesetz macht Menschen zu Hohenpriestern, die da
Schwachheit haben; dies Wort aber des Eides, das nach dem Gesetz gesagt
ward, setzt den Sohn ein, der ewig und vollkommen ist.

\hypertarget{section-7}{%
\section{8}\label{section-7}}

\bibverse{1} Das ist nun die Hauptsache, davon wir reden: Wir haben
einen solchen Hohenpriester, der da sitzt zu der Rechten auf dem Stuhl
der Majestät im Himmel \bibverse{2} und ist ein Pfleger des Heiligen und
der wahrhaften Hütte, welche Gott aufgerichtet hat und kein Mensch.
\bibverse{3} Denn ein jeglicher Hoherpriester wird eingesetzt, zu opfern
Gaben und Opfer. Darum muß auch dieser etwas haben, das er opfere.
\bibverse{4} Wenn er nun auf Erden wäre, so wäre er nicht Priester,
dieweil da Priester sind, die nach dem Gesetz die Gaben opfern,
\bibverse{5} welche dienen dem Vorbilde und dem Schatten des
Himmlischen; wie die göttliche Antwort zu Mose sprach, da er sollte die
Hütte vollenden: ``Schaue zu,'' sprach er, ``daß du machest alles nach
dem Bilde, das dir auf dem Berge gezeigt ist.'' \bibverse{6} Nun aber
hat er ein besseres Amt erlangt, als der eines besseren Testaments
Mittler ist, welches auch auf besseren Verheißungen steht. \bibverse{7}
Denn so jenes, das erste, untadelig gewesen wäre, würde nicht Raum zu
einem andern gesucht. \bibverse{8} Denn er tadelt sie und sagt: ``Siehe,
es kommen die Tage, spricht der HERR, daß ich über das Haus Israel und
über das Haus Juda ein neues Testament machen will; \bibverse{9} nicht
nach dem Testament, das ich gemacht habe mit ihren Vätern an dem Tage,
da ich ihre Hand ergriff, sie auszuführen aus Ägyptenland. Denn sie sind
nicht geblieben in meinem Testament, so habe ich ihrer auch nicht wollen
achten, spricht der HERR. \bibverse{10} Denn das ist das Testament, das
ich machen will dem Hause Israel nach diesen Tagen, spricht der HERR:
Ich will geben mein Gesetz in ihren Sinn, und in ihr Herz will ich es
schreiben, und will ihr Gott sein, und sie sollen mein Volk sein.
\bibverse{11} Und soll nicht lehren jemand seinen Nächsten noch jemand
seinen Bruder und sagen: Erkenne den HERRN! denn sie sollen mich alle
kennen von dem Kleinsten an bis zu dem Größten. \bibverse{12} Denn ich
will gnädig sein ihrer Untugend und ihren Sünden, und ihrer
Ungerechtigkeit will ich nicht mehr gedenken.'' \bibverse{13} Indem er
sagt: ``Ein neues'', macht das erste alt. Was aber alt und überjahrt
ist, das ist nahe bei seinem Ende.

\hypertarget{section-8}{%
\section{9}\label{section-8}}

\bibverse{1} Es hatte zwar auch das erste seine Rechte des
Gottesdienstes und das äußerliche Heiligtum. \bibverse{2} Denn es war da
aufgerichtet das Vorderteil der Hütte, darin der Leuchter war und der
Tisch und die Schaubrote; und dies hieß das Heilige. \bibverse{3} Hinter
dem andern Vorhang aber war die Hütte, die da heißt das Allerheiligste;
\bibverse{4} die hatte das goldene Räuchfaß und die Lade des Testaments
allenthalben mit Gold überzogen, in welcher war der goldene Krug mit dem
Himmelsbrot und die Rute Aarons, die gegrünt hatte, und die Tafeln des
Testaments; \bibverse{5} obendarüber aber waren die Cherubim der
Herrlichkeit, die überschatteten den Gnadenstuhl; von welchen Dingen
jetzt nicht zu sagen ist insonderheit. \bibverse{6} Da nun solches also
zugerichtet war, gingen die Priester allezeit in die vordere Hütte und
richteten aus den Gottesdienst. \bibverse{7} In die andere aber ging nur
einmal im Jahr allein der Hohepriester, nicht ohne Blut, das er opferte
für seine und des Volkes Versehen. \bibverse{8} Damit deutete der
heilige Geist, daß noch nicht offenbart wäre der Weg zum Heiligen,
solange die vordere Hütte stünde, \bibverse{9} welche ist ein Gleichnis
auf die gegenwärtige Zeit, nach welchem Gaben und Opfer geopfert werden,
die nicht können vollkommen machen nach dem Gewissen den, der da
Gottesdienst tut \bibverse{10} allein mit Speise und Trank und
mancherlei Taufen und äußerlicher Heiligkeit, die bis auf die Zeit der
Besserung sind aufgelegt. \bibverse{11} Christus aber ist gekommen, daß
er sei ein Hoherpriester der zukünftigen Güter, und ist durch eine
größere und vollkommenere Hütte, die nicht mit der Hand gemacht, das
ist, die nicht von dieser Schöpfung ist, \bibverse{12} auch nicht der
Böcke oder Kälber Blut, sondern sein eigen Blut einmal in das Heilige
eingegangen und hat eine ewige Erlösung erfunden. \bibverse{13} Denn so
der Ochsen und der Böcke Blut und die Asche von der Kuh, gesprengt,
heiligt die Unreinen zu der leiblichen Reinigkeit, \bibverse{14} wie
viel mehr wird das Blut Christi, der sich selbst ohne allen Fehl durch
den ewigen Geist Gott geopfert hat, unser Gewissen reinigen von den
toten Werken, zu dienen dem lebendigen Gott! \bibverse{15} Und darum ist
er auch ein Mittler des neuen Testaments, auf daß durch den Tod, so
geschehen ist zur Erlösung von den Übertretungen, die unter dem ersten
Testament waren, die, so berufen sind, das verheißene ewige Erbe
empfangen. \bibverse{16} Denn wo ein Testament ist, da muß der Tod
geschehen des, der das Testament machte. \bibverse{17} Denn ein
Testament wird fest durch den Tod; es hat noch nicht Kraft, wenn der
noch lebt, der es gemacht hat. \bibverse{18} Daher auch das erste nicht
ohne Blut gestiftet ward. \bibverse{19} Denn als Mose ausgeredet hatte
von allen Geboten nach dem Gesetz zu allem Volk, nahm er Kälber-und
Bocksblut mit Wasser und Scharlachwolle und Isop und besprengte das Buch
und alles Volk \bibverse{20} und sprach: ``Das ist das Blut des
Testaments, das Gott euch geboten hat.'' \bibverse{21} Und die Hütte und
alles Geräte des Gottesdienstes besprengte er gleicherweise mit Blut.
\bibverse{22} Und es wird fast alles mit Blut gereinigt nach dem Gesetz;
und ohne Blut vergießen geschieht keine Vergebung. \bibverse{23} So
mußten nun der himmlischen Dinge Vorbilder mit solchem gereinigt werden;
aber sie selbst, die himmlischen, müssen bessere Opfer haben, denn jene
waren. \bibverse{24} Denn Christus ist nicht eingegangen in das Heilige,
so mit Händen gemacht ist (welches ist ein Gegenbild des wahrhaftigen),
sondern in den Himmel selbst, nun zu erscheinen vor dem Angesicht Gottes
für uns; \bibverse{25} auch nicht, daß er sich oftmals opfere, gleichwie
der Hohepriester geht alle Jahre in das Heilige mit fremden Blut;
\bibverse{26} sonst hätte er oft müssen leiden von Anfang der Welt her.
Nun aber, am Ende der Welt, ist er einmal erschienen, durch sein eigen
Opfer die Sünde aufzuheben. \bibverse{27} Und wie den Menschen gesetzt
ist, einmal zu sterben, darnach aber das Gericht: \bibverse{28} also ist
auch Christus einmal geopfert, wegzunehmen vieler Sünden; zum andernmal
wird er ohne Sünde erscheinen denen, die auf ihn warten, zur Seligkeit.

\hypertarget{section-9}{%
\section{10}\label{section-9}}

\bibverse{1} Denn das Gesetz hat den Schatten von den zukünftigen
Gütern, nicht das Wesen der Güter selbst; alle Jahre muß man opfern
immer einerlei Opfer, und es kann nicht, die da opfern, vollkommen
machen; \bibverse{2} sonst hätte das Opfern aufgehört, wo die, so am
Gottesdienst sind, kein Gewissen mehr hätten von den Sünden, wenn sie
einmal gereinigt wären; \bibverse{3} sondern es geschieht dadurch nur
ein Gedächtnis der Sünden alle Jahre. \bibverse{4} Denn es ist
unmöglich, durch Ochsen-und Bocksblut Sünden wegzunehmen. \bibverse{5}
Darum, da er in die Welt kommt, spricht er: ``Opfer und Gaben hast du
nicht gewollt; den Leib aber hast du mir bereitet. \bibverse{6}
Brandopfer und Sündopfer gefallen dir nicht. \bibverse{7} Da sprach ich:
Siehe, ich komme (im Buch steht von mir geschrieben), daß ich tue, Gott,
deinen Willen.'' \bibverse{8} Nachdem er weiter oben gesagt hatte:
``Opfer und Gaben, Brandopfer und Sündopfer hast du nicht gewollt, sie
gefallen dir auch nicht'' (welche nach dem Gesetz geopfert werden),
\bibverse{9} da sprach er: ``Siehe, ich komme, zu tun, Gott, deinen
Willen.'' Da hebt er das erste auf, daß er das andere einsetze.
\bibverse{10} In diesem Willen sind wir geheiligt auf einmal durch das
Opfer des Leibes Jesu Christi. \bibverse{11} Und ein jeglicher Priester
ist eingesetzt, daß er täglich Gottesdienst pflege und oftmals einerlei
Opfer tue, welche nimmermehr können die Sünden abnehmen. \bibverse{12}
Dieser aber, da er hat ein Opfer für die Sünden geopfert, das ewiglich
gilt, sitzt nun zur Rechten Gottes \bibverse{13} und wartet hinfort, bis
daß seine Feinde zum Schemel seiner Füße gelegt werden. \bibverse{14}
Denn mit einem Opfer hat er in Ewigkeit vollendet die geheiligt werden.
\bibverse{15} Es bezeugt uns aber das auch der heilige Geist. Denn
nachdem er zuvor gesagt hatte: \bibverse{16} Das ist das Testament, das
ich ihnen machen will nach diesen Tagen'', spricht der HERR: ``Ich will
mein Gesetz in ihr Herz geben, und in ihren Sinn will ich es schreiben,
\bibverse{17} und ihrer Sünden und Ungerechtigkeit will ich nicht mehr
gedenken.'' \bibverse{18} Wo aber derselben Vergebung ist, da ist nicht
mehr Opfer für die Sünde. \bibverse{19} So wir denn nun haben, liebe
Brüder, die Freudigkeit zum Eingang in das Heilige durch das Blut Jesu,
\bibverse{20} welchen er uns bereitet hat zum neuen und lebendigen Wege
durch den Vorhang, das ist durch sein Fleisch, \bibverse{21} und haben
einen Hohenpriester über das Haus Gottes: \bibverse{22} so lasset uns
hinzugehen mit wahrhaftigem Herzen in völligem Glauben, besprengt in
unsern Herzen und los von dem bösen Gewissen und gewaschen am Leibe mit
reinem Wasser; \bibverse{23} und lasset uns halten an dem Bekenntnis der
Hoffnung und nicht wanken; denn er ist treu, der sie verheißen hat;
\bibverse{24} und lasset uns untereinander unser selbst wahrnehmen mit
Reizen zur Liebe und guten Werken \bibverse{25} und nicht verlassen
unsere Versammlung, wie etliche pflegen, sondern einander ermahnen; und
das so viel mehr, soviel ihr sehet, daß sich der Tag naht. \bibverse{26}
Denn so wir mutwillig sündigen, nachdem wir die Erkenntnis der Wahrheit
empfangen haben, haben wir fürder kein anderes Opfer mehr für die
Sünden, \bibverse{27} sondern ein schreckliches Warten des Gerichts und
des Feuereifers, der die Widersacher verzehren wird. \bibverse{28} Wenn
jemand das Gesetz Mose's bricht, der muß sterben ohne Barmherzigkeit
durch zwei oder drei Zeugen. \bibverse{29} Wie viel, meint ihr, ärgere
Strafe wird der verdienen, der den Sohn Gottes mit Füßen tritt und das
Blut des Testaments unrein achtet, durch welches er geheiligt ist, und
den Geist der Gnade schmäht? \bibverse{30} Denn wir kennen den, der da
sagte: ``Die Rache ist mein, ich will vergelten'', und abermals: ``Der
HERR wird sein Volk richten.'' \bibverse{31} Schrecklich ist's, in die
Hände des lebendigen Gottes zu fallen. \bibverse{32} Gedenket aber an
die vorigen Tage, in welchen ihr, nachdem ihr erleuchtet wart, erduldet
habt einen großen Kampf des Leidens \bibverse{33} und zum Teil selbst
durch Schmach und Trübsal ein Schauspiel wurdet, zum Teil Gemeinschaft
hattet mit denen, welchen es also geht. \bibverse{34} Denn ihr habt mit
den Gebundenen Mitleiden gehabt und den Raub eurer Güter mit Freuden
erduldet, als die ihr wisset, daß ihr bei euch selbst eine bessere und
bleibende Habe im Himmel habt. \bibverse{35} Werfet euer Vertrauen nicht
weg, welches eine große Belohnung hat. \bibverse{36} Geduld aber ist
euch not, auf daß ihr den Willen Gottes tut und die Verheißung
empfanget. \bibverse{37} Denn ``noch über eine kleine Weile, so wird
kommen, der da kommen soll, und nicht verziehen. \bibverse{38} Der
Gerechte aber wird des Glaubens leben, Wer aber weichen wird, an dem
wird meine Seele keinen Gefallen haben.'' \bibverse{39} Wir aber sind
nicht von denen, die da weichen und verdammt werden, sondern von denen,
die da glauben und die Seele erretten.

\hypertarget{section-10}{%
\section{11}\label{section-10}}

\bibverse{1} Es ist aber der Glaube eine gewisse Zuversicht des, das man
hofft, und ein Nichtzweifeln an dem, das man nicht sieht. \bibverse{2}
Durch den haben die Alten Zeugnis überkommen. \bibverse{3} Durch den
Glauben merken wir, daß die Welt durch Gottes Wort fertig ist, daß
alles, was man sieht, aus nichts geworden ist. \bibverse{4} Durch den
Glauben hat Abel Gott ein größeres Opfer getan denn Kain; durch welchen
er Zeugnis überkommen hat, daß er gerecht sei, da Gott zeugte von seiner
Gabe; und durch denselben redet er noch, wiewohl er gestorben ist.
\bibverse{5} Durch den Glauben ward Henoch weggenommen, daß er den Tod
nicht sähe, und ward nicht gefunden, darum daß ihn Gott wegnahm; denn
vor seinem Wegnehmen hat er Zeugnis gehabt, daß er Gott gefallen habe.
\bibverse{6} Aber ohne Glauben ist's unmöglich, Gott zu gefallen; denn
wer zu Gott kommen will, der muß glauben, daß er sei und denen, die ihn
suchen, ein Vergelter sein werde. \bibverse{7} Durch den Glauben hat
Noah Gott geehrt und die Arche zubereitet zum Heil seines Hauses, da er
ein göttliches Wort empfing über das, was man noch nicht sah; und
verdammte durch denselben die Welt und hat ererbt die Gerechtigkeit, die
durch den Glauben kommt. \bibverse{8} Durch den Glauben ward gehorsam
Abraham, da er berufen ward, auszugehen in das Land, das er ererben
sollte; und ging aus und wußte nicht wo er hinkäme. \bibverse{9} Durch
den Glauben ist er ein Fremdling gewesen in dem verheißenen Lande als in
einem fremden und wohnte in Hütten mit Isaak und Jakob, den Miterben
derselben Verheißung; \bibverse{10} denn er wartete auf eine Stadt, die
einen Grund hat, der Baumeister und Schöpfer Gott ist. \bibverse{11}
Durch den Glauben empfing auch Sara Kraft, daß sie schwanger ward und
gebar über die Zeit ihres Alters; denn sie achtete ihn treu, der es
verheißen hatte. \bibverse{12} Darum sind auch von einem, wiewohl
erstorbenen Leibes, viele geboren wie die Sterne am Himmel und wie der
Sand am Rande des Meeres, der unzählig ist. \bibverse{13} Diese alle
sind gestorben im Glauben und haben die Verheißung nicht empfangen,
sondern sie von ferne gesehen und sich ihrer getröstet und wohl genügen
lassen und bekannt, daß sie Gäste und Fremdlinge auf Erden wären.
\bibverse{14} Denn die solches sagen, die geben zu verstehen, daß sie
ein Vaterland suchen. \bibverse{15} Und zwar, wo sie das gemeint hätten,
von welchem sie waren ausgezogen, hatten sie ja Zeit, wieder umzukehren.
\bibverse{16} Nun aber begehren sie eines bessern, nämlich eines
himmlischen. Darum schämt sich Gott ihrer nicht, zu heißen ihr Gott;
denn er hat ihnen eine Stadt zubereitet. \bibverse{17} Durch den Glauben
opferte Abraham den Isaak, da er versucht ward, und gab dahin den
Eingeborenen, da er schon die Verheißungen empfangen hatte,
\bibverse{18} von welchem gesagt war: ``In Isaak wird dir dein Same
genannt werden''; \bibverse{19} und dachte, Gott kann auch wohl von den
Toten auferwecken; daher er auch ihn zum Vorbilde wiederbekam.
\bibverse{20} Durch den Glauben segnete Isaak von den zukünftigen Dingen
den Jakob und Esau. \bibverse{21} Durch den Glauben segnete Jakob, da er
starb, beide Söhne Josephs und neigte sich gegen seines Stabes Spitze.
\bibverse{22} Durch den Glauben redete Joseph vom Auszug der Kinder
Israel, da er starb, und tat Befehl von seinen Gebeinen. \bibverse{23}
Durch den Glauben ward Mose, da er geboren war, drei Monate verborgen
von seinen Eltern, darum daß sie sahen, wie er ein schönes Kind war, und
fürchteten sich nicht vor des Königs Gebot. \bibverse{24} Durch den
Glauben wollte Mose, da er groß ward, nicht mehr ein Sohn heißen der
Tochter Pharaos, \bibverse{25} und erwählte viel lieber, mit dem Volk
Gottes Ungemach zu leiden, denn die zeitliche Ergötzung der Sünde zu
haben, \bibverse{26} und achtete die Schmach Christi für größern
Reichtum denn die Schätze Ägyptens; denn er sah an die Belohnung.
\bibverse{27} Durch den Glauben verließ er Ägypten und fürchtete nicht
des Königs Grimm; denn er hielt sich an den, den er nicht sah, als sähe
er ihn. \bibverse{28} Durch den Glauben hielt er Ostern und das
Blutgießen, auf daß, der die Erstgeburten erwürgte, sie nicht träfe.
\bibverse{29} Durch den Glauben gingen sie durchs Rote Meer wie durch
trockenes Land; was die Ägypter auch versuchten, und ersoffen.
\bibverse{30} Durch den Glauben fielen die Mauern Jerichos, da sie
sieben Tage um sie herumgegangen waren. \bibverse{31} Durch den Glauben
ward die Hure Rahab nicht verloren mit den Ungläubigen, da sie die
Kundschafter freundlich aufnahm. \bibverse{32} Und was soll ich mehr
sagen? Die Zeit würde mir zu kurz, wenn ich sollte erzählen von Gideon
und Barak und Simson und Jephthah und David und Samuel und den
Propheten, \bibverse{33} welche haben durch den Glauben Königreiche
bezwungen, Gerechtigkeit gewirkt, Verheißungen erlangt, der Löwen Rachen
verstopft, \bibverse{34} des Feuers Kraft ausgelöscht, sind des
Schwertes Schärfe entronnen, sind kräftig geworden aus der Schwachheit,
sind stark geworden im Streit, haben der Fremden Heere darniedergelegt.
\bibverse{35} Weiber haben ihre Toten durch Auferstehung wiederbekommen.
Andere aber sind zerschlagen und haben keine Erlösung angenommen, auf
daß sie die Auferstehung, die besser ist, erlangten. \bibverse{36}
Etliche haben Spott und Geißeln erlitten, dazu Bande und Gefängnis;
\bibverse{37} sie wurden gesteinigt, zerhackt, zerstochen, durchs
Schwert getötet; sie sind umhergegangen in Schafpelzen und Ziegenfellen,
mit Mangel, mit Trübsal, mit Ungemach \bibverse{38} (deren die Welt
nicht wert war), und sind im Elend umhergeirrt in den Wüsten, auf den
Bergen und in den Klüften und Löchern der Erde. \bibverse{39} Diese alle
haben durch den Glauben Zeugnis überkommen und nicht empfangen die
Verheißung, \bibverse{40} darum daß Gott etwas Besseres für uns zuvor
ersehen hat, daß sie nicht ohne uns vollendet würden.

\hypertarget{section-11}{%
\section{12}\label{section-11}}

\bibverse{1} Darum wir auch, dieweil wir eine solche Wolke von Zeugen um
uns haben, lasset uns ablegen die Sünde, so uns immer anklebt und träge
macht, und lasset uns laufen durch Geduld in dem Kampf, der uns
verordnet ist. \bibverse{2} und aufsehen auf Jesum, den Anfänger und
Vollender des Glaubens; welcher, da er wohl hätte mögen Freude haben,
erduldete das Kreuz und achtete der Schande nicht und hat sich gesetzt
zur Rechten auf den Stuhl Gottes. \bibverse{3} Gedenket an den, der ein
solches Widersprechen von den Sündern wider sich erduldet hat, daß ihr
nicht in eurem Mut matt werdet und ablasset. \bibverse{4} Denn ihr habt
noch nicht bis aufs Blut widerstanden in den Kämpfen wider die Sünde
\bibverse{5} und habt bereits vergessen des Trostes, der zu euch redet
als zu Kindern: ``Mein Sohn, achte nicht gering die Züchtigung des HERRN
und verzage nicht, wenn du von ihm gestraft wirst. \bibverse{6} Denn
welchen der HERR liebhat, den züchtigt er; und stäupt einen jeglichen
Sohn, den er aufnimmt.'' \bibverse{7} So ihr die Züchtigung erduldet, so
erbietet sich euch Gott als Kindern; denn wo ist ein Sohn, den der Vater
nicht züchtigt? \bibverse{8} Seid ihr aber ohne Züchtigung, welcher sind
alle teilhaftig geworden, so seid ihr Bastarde und nicht Kinder.
\bibverse{9} Und so wir haben unsre leiblichen Väter zu Züchtigern
gehabt und sie gescheut, sollten wir denn nicht viel mehr untertan sein
dem Vater der Geister, daß wir leben? \bibverse{10} Denn jene haben uns
gezüchtigt wenig Tage nach ihrem Dünken, dieser aber zu Nutz, auf daß
wir seine Heiligung erlangen. \bibverse{11} Alle Züchtigung aber, wenn
sie da ist, dünkt uns nicht Freude, sondern Traurigkeit zu sein; aber
darnach wird sie geben eine friedsame Frucht der Gerechtigkeit denen,
die dadurch geübt sind. \bibverse{12} Darum richtet wieder auf die
lässigen Hände und die müden Kniee \bibverse{13} und tut gewisse Tritte
mit euren Füßen, daß nicht jemand strauchle wie ein Lahmer, sondern
vielmehr gesund werde. \bibverse{14} Jaget nach dem Frieden gegen
jedermann und der Heiligung, ohne welche wird niemand den HERRN sehen,
\bibverse{15} und sehet darauf, daß nicht jemand Gottes Gnade versäume;
daß nicht etwa eine bittere Wurzel aufwachse und Unfrieden anrichte und
viele durch dieselbe verunreinigt werden; \bibverse{16} daß nicht jemand
sei ein Hurer oder ein Gottloser wie Esau, der um einer Speise willen
seine Erstgeburt verkaufte. \bibverse{17} Wisset aber, daß er hernach,
da er den Segen ererben wollte, verworfen ward; denn er fand keinen Raum
zur Buße, wiewohl er sie mit Tränen suchte. \bibverse{18} Denn ihr seid
nicht gekommen zu dem Berge, den man anrühren konnte und der mit Feuer
brannte, noch zu dem Dunkel und Finsternis und Ungewitter \bibverse{19}
noch zu dem Hall der Posaune und zu der Stimme der Worte, da sich
weigerten, die sie hörten, daß ihnen das Wort ja nicht gesagt würde;
\bibverse{20} denn sie mochten's nicht ertragen, was da gesagt ward:
``Und wenn ein Tier den Berg anrührt, soll es gesteinigt oder mit einem
Geschoß erschossen werden''; \bibverse{21} und also schrecklich war das
Gesicht, daß Mose sprach: Ich bin erschrocken und zittere. \bibverse{22}
Sondern ihr seid gekommen zu dem Berge Zion und zu der Stadt des
lebendigen Gottes, dem himmlischen Jerusalem, und zu einer Menge vieler
tausend Engel \bibverse{23} und zu der Gemeinde der Erstgeborenen, die
im Himmel angeschrieben sind, und zu Gott, dem Richter über alle, und zu
den Geistern der vollendeten Gerechten \bibverse{24} und zu dem Mittler
des neuen Testaments, Jesus, und zu dem Blut der Besprengung, das da
besser redet denn das Abels. \bibverse{25} Sehet zu, daß ihr den nicht
abweiset, der da redet. Denn so jene nicht entflohen sind, die ihn
abwiesen, da er auf Erden redete, viel weniger wir, so wir den abweisen,
der vom Himmel redet; \bibverse{26} dessen Stimme zu der Zeit die Erde
bewegte, nun aber verheißt er und spricht: ``Noch einmal will ich
bewegen nicht allein die Erde sondern auch den Himmel.'' \bibverse{27}
Aber solches ``Noch einmal'' zeigt an, daß das Bewegliche soll
verwandelt werden, als das gemacht ist, auf daß da bleibe das
Unbewegliche. \bibverse{28} Darum, dieweil wir empfangen ein unbeweglich
Reich, haben wir Gnade, durch welche wir sollen Gott dienen, ihm zu
gefallen, mit Zucht und Furcht; \bibverse{29} denn unser Gott ist ein
verzehrend Feuer.

\hypertarget{section-12}{%
\section{13}\label{section-12}}

\bibverse{1} Bleibet fest in der brüderlichen Liebe. \bibverse{2}
Gastfrei zu sein vergesset nicht; denn dadurch haben etliche ohne ihr
Wissen Engel beherbergt. \bibverse{3} Gedenket der Gebundenen als die
Mitgebundenen derer, die in Trübsal leiden, als die ihr auch noch im
Leibe lebet. \bibverse{4} Die Ehe soll ehrlich gehalten werden bei allen
und das Ehebett unbefleckt; die Hurer aber und die Ehebrecher wird Gott
richten. \bibverse{5} Der Wandel sei ohne Geiz; und laßt euch genügen an
dem, was da ist. Denn er hat gesagt: ``Ich will dich nicht verlassen
noch versäumen''; \bibverse{6} also daß wir dürfen sagen: ``Der HERR ist
mein Helfer, ich will mich nicht fürchten; was sollte mir ein Mensch
tun?'' \bibverse{7} Gedenkt an eure Lehrer, die euch das Wort Gottes
gesagt haben; ihr Ende schaut an und folgt ihrem Glauben nach.
\bibverse{8} Jesus Christus gestern und heute und derselbe auch in
Ewigkeit. \bibverse{9} Lasset euch nicht mit mancherlei und fremden
Lehren umtreiben; denn es ist ein köstlich Ding, daß das Herz fest
werde, welches geschieht durch die Gnade, nicht durch Speisen, davon
keinen Nutzen haben, die damit umgehen. \bibverse{10} Wir haben einen
Altar, davon nicht Macht haben zu essen, die der Hütte pflegen.
\bibverse{11} Denn welcher Tiere Blut getragen wird durch den
Hohenpriester in das Heilige für die Sünde, deren Leichname werden
verbrannt außerhalb des Lagers. \bibverse{12} Darum hat auch Jesus, auf
daß er heiligte das Volk durch sein eigen Blut, gelitten draußen vor dem
Tor. \bibverse{13} So laßt uns nun zu ihm hinausgehen aus dem Lager und
seine Schmach tragen. \bibverse{14} Denn wir haben hier keine bleibende
Stadt, sondern die zukünftige suchen wir. \bibverse{15} So lasset uns
nun opfern durch ihn das Lobopfer Gott allezeit, das ist die Frucht der
Lippen, die seinen Namen bekennen. \bibverse{16} Wohlzutun und
mitzuteilen vergesset nicht; denn solche Opfer gefallen Gott wohl.
\bibverse{17} Gehorcht euren Lehrern und folgt ihnen; denn sie wachen
über eure Seelen, als die da Rechenschaft dafür geben sollen; auf daß
sie das mit Freuden tun und nicht mit Seufzen; denn das ist euch nicht
gut. \bibverse{18} Betet für uns. Unser Trost ist der, daß wir ein gutes
Gewissen haben und fleißigen uns, guten Wandel zu führen bei allen.
\bibverse{19} Ich ermahne aber desto mehr, solches zu tun, auf daß ich
umso schneller wieder zu euch komme. \bibverse{20} Der Gott aber des
Friedens, der von den Toten ausgeführt hat den großen Hirten der Schafe
durch das Blut des ewigen Testaments, unsern HERRN Jesus, \bibverse{21}
der mache euch fertig in allem guten Werk, zu tun seinen Willen, und
schaffe in euch, was vor ihm gefällig ist, durch Jesum Christum; welchem
sei Ehre von Ewigkeit zu Ewigkeit! Amen. \bibverse{22} Ich ermahne euch
aber, liebe Brüder, haltet das Wort der Ermahnung zugute; denn ich habe
euch kurz geschrieben. \bibverse{23} Wisset, daß der Bruder Timotheus
wieder frei ist; mit dem, so er bald kommt, will ich euch sehen.
\bibverse{24} Grüßet alle eure Lehrer und alle Heiligen. Es grüßen euch
die Brüder aus Italien. \bibverse{25} Die Gnade sei mit euch allen!
Amen.
