\hypertarget{section}{%
\section{1}\label{section}}

\bibverse{1} Paulus, ein Apostel Jesu Christi durch den Willen Gottes,
und Bruder Timotheus \bibverse{2} den Heiligen zu Kolossä und den
gläubigen Brüdern in Christo: Gnade sei mit euch und Friede von Gott,
unserem Vater, und dem Herrn Jesus Christus!

\bibverse{3} Wir danken Gott und dem Vater unseres Herrn Jesu Christi
und beten allezeit für euch, \bibverse{4} nachdem wir gehört haben von
eurem Glauben an Christum Jesum und von der Liebe zu allen Heiligen,
\bibverse{5} um der Hoffnung willen, die euch beigelegt ist im Himmel,
von welcher ihr zuvor gehört habt durch das Wort der Wahrheit im
Evangelium, \footnote{\textbf{1:5} 1Petr 1,3-4} \bibverse{6} das zu euch
gekommen ist, wie auch in alle Welt, und ist fruchtbar, wie auch in
euch, von dem Tage an, da ihr's gehört habt und erkannt die Gnade Gottes
in der Wahrheit; \bibverse{7} wie ihr denn gelernt habt von Epaphras,
unserem lieben Mitdiener, welcher ist ein treuer Diener Christi für
euch, \bibverse{8} der uns auch eröffnet hat eure Liebe im Geist.

\bibverse{9} Derhalben auch wir von dem Tage an, da wir's gehört haben,
hören wir nicht auf, für euch zu beten und zu bitten, dass ihr erfüllt
werdet mit Erkenntnis seines Willens in allerlei geistlicher Weisheit
und Verständnis, \footnote{\textbf{1:9} Eph 1,15-17} \bibverse{10} dass
ihr wandelt würdig dem Herrn zu allem Gefallen und fruchtbar seid in
allen guten Werken \footnote{\textbf{1:10} Eph 4,1; Phil 1,27}
\bibverse{11} und wachset in der Erkenntnis Gottes und gestärkt werdet
mit aller Kraft nach seiner herrlichen Macht zu aller Geduld und
Langmütigkeit mit Freuden, \bibverse{12} und danksaget dem Vater, der
uns tüchtig gemacht hat zu dem Erbteil der Heiligen im Licht;
\footnote{\textbf{1:12} Eph 1,11; 1Petr 1,4} \bibverse{13} welcher uns
errettet hat von der Obrigkeit der Finsternis und hat uns versetzt in
das Reich seines lieben Sohnes, \footnote{\textbf{1:13} Kol 2,15}
\bibverse{14} an welchem wir haben die Erlösung durch sein Blut, die
Vergebung der Sünden; \footnote{\textbf{1:14} Eph 1,7}

\bibverse{15} welcher ist das Ebenbild des unsichtbaren Gottes, der
Erstgeborene vor allen Kreaturen. \footnote{\textbf{1:15} Hebr 1,3; Offb
  3,14} \bibverse{16} Denn durch ihn ist alles geschaffen, was im Himmel
und auf Erden ist, das Sichtbare und das Unsichtbare, es seien Throne
oder Herrschaften oder Fürstentümer oder Obrigkeiten; es ist alles durch
ihn und zu ihm geschaffen. \footnote{\textbf{1:16} Joh 1,3; Joh 1,10;
  Hebr 1,2} \bibverse{17} Und er ist vor allem, und es besteht alles in
ihm. \footnote{\textbf{1:17} Apg 26,23; 1Kor 15,20; Offb 1,5; Eph 1,22}
\bibverse{18} Und er ist das Haupt des Leibes, nämlich der Gemeinde; er,
welcher ist der Anfang und der Erstgeborene von den Toten, auf dass er
in allen Dingen den Vorrang habe. \bibverse{19} Denn es ist das
Wohlgefallen gewesen, dass in ihm alle Fülle wohnen sollte \footnote{\textbf{1:19}
  Kol 2,9; Joh 1,16; Eph 1,23; 2Kor 5,19} \bibverse{20} und alles durch
ihn versöhnt würde zu ihm selbst, es sei auf Erden oder im Himmel, damit
dass er Frieden machte durch das Blut an seinem Kreuz, durch sich
selbst. \footnote{\textbf{1:20} 1Jo 2,2}

\bibverse{21} Und euch, die ihr vordem Fremde und Feinde waret durch die
Vernunft in bösen Werken, \footnote{\textbf{1:21} Röm 5,10; Eph 2,12-13;
  Eph 4,18} \bibverse{22} hat er nun versöhnt mit dem Leibe seines
Fleisches durch den Tod, auf dass er euch darstellte heilig und
unsträflich und ohne Tadel vor ihm selbst; \footnote{\textbf{1:22} Eph
  5,27} \bibverse{23} so ihr anders bleibet im Glauben, gegründet und
fest und unbeweglich von der Hoffnung des Evangeliums, welches ihr
gehört habt, welches gepredigt ist unter aller Kreatur, die unter dem
Himmel ist, dessen Diener ich, Paulus, geworden bin.

\bibverse{24} Nun freue ich mich in meinem Leiden, das ich für euch
leide, und erstatte an meinem Fleisch, was noch mangelt an Trübsalen in
Christo, für seinen Leib, welcher ist die Gemeinde, \footnote{\textbf{1:24}
  Eph 3,13; 2Tim 2,10} \bibverse{25} deren Diener ich geworden bin nach
dem göttlichen Predigtamt, das mir gegeben ist unter euch, dass ich das
Wort Gottes reichlich predigen soll, \bibverse{26} nämlich das
Geheimnis, das verborgen gewesen ist von der Welt her und von den Zeiten
her, nun aber ist es offenbart seinen Heiligen, \bibverse{27} denen Gott
gewollt hat kundtun, welcher da sei der herrliche Reichtum dieses
Geheimnisses unter den Heiden, welches ist Christus in euch, der da ist
die Hoffnung der Herrlichkeit. \footnote{\textbf{1:27} 1Tim 1,1}
\bibverse{28} Den verkündigen wir und vermahnen alle Menschen und lehren
alle Menschen mit aller Weisheit, auf dass wir darstellen einen
jeglichen Menschen vollkommen in Christo Jesu; \bibverse{29} daran ich
auch arbeite und ringe, nach der Wirkung des, der in mir kräftig wirkt.
\# 2 \bibverse{1} Ich lasse euch aber wissen, welch einen Kampf ich habe
um euch und um die zu Laodizea und alle, die meine Person im Fleisch
nicht gesehen haben, \bibverse{2} auf dass ihre Herzen ermahnt und
zusammengefasst werden in der Liebe und zu allem Reichtum des gewissen
Verständnisses, zu erkennen das Geheimnis Gottes, des Vaters und
Christi, \bibverse{3} in welchem verborgen liegen alle Schätze der
Weisheit und der Erkenntnis. \bibverse{4} Ich sage aber davon, auf dass
euch niemand betrüge mit unvernünftigen Reden. \footnote{\textbf{2:4}
  Röm 16,18} \bibverse{5} Denn ob ich wohl nach dem Fleisch nicht da
bin, so bin ich doch im Geist bei euch, freue mich und sehe eure Ordnung
und euren festen Glauben an Christum. \footnote{\textbf{2:5} 1Kor 14,40}

\bibverse{6} Wie ihr nun angenommen habt den Herrn Christus Jesus, so
wandelt in ihm \bibverse{7} und seid gewurzelt und erbaut in ihm und
fest im Glauben, wie ihr gelehrt seid, und seid in demselben reichlich
dankbar. \footnote{\textbf{2:7} Eph 3,17}

\bibverse{8} Sehet zu, dass euch niemand beraube durch die Philosophie
und lose Verführung nach der Menschen Lehre und nach der Welt Satzungen,
und nicht nach Christo. \bibverse{9} Denn in ihm wohnt die ganze Fülle
der Gottheit leibhaftig, \bibverse{10} und ihr seid vollkommen in ihm,
welcher ist das Haupt aller Fürstentümer und Obrigkeiten; \footnote{\textbf{2:10}
  Eph 1,21} \bibverse{11} in welchem ihr auch beschnitten seid mit der
Beschneidung ohne Hände, durch Ablegung des sündlichen Leibes im
Fleisch, nämlich mit der Beschneidung Christi, \footnote{\textbf{2:11}
  Röm 2,29; Röm 6,5; 1Petr 3,21} \bibverse{12} indem ihr mit ihm
begraben seid durch die Taufe; in welchem ihr auch seid auferstanden
durch den Glauben, den Gott wirkt, welcher ihn auferweckt hat von den
Toten. \footnote{\textbf{2:12} Kol 3,1; Röm 6,4} \bibverse{13} Und er
hat euch auch mit ihm lebendig gemacht, da ihr tot waret in den Sünden
und in eurem unbeschnittenen Fleisch; und hat uns geschenkt alle Sünden
\footnote{\textbf{2:13} Eph 2,1; Eph 2,5} \bibverse{14} und ausgetilgt
die Handschrift, die wider uns war, welche durch Satzungen entstand und
uns entgegen war, und hat sie aus dem Mittel getan und an das Kreuz
geheftet; \footnote{\textbf{2:14} Eph 2,15} \bibverse{15} und hat
ausgezogen die Fürstentümer und die Gewaltigen und sie schaugetragen
öffentlich und einen Triumph aus ihnen gemacht durch sich selbst.
\footnote{\textbf{2:15} Kol 1,13; Eph 4,8}

\bibverse{16} So lasset nun niemand euch Gewissen machen über Speise
oder über Trank oder über bestimmte Feiertage oder Neumonde oder
Sabbate; \footnote{\textbf{2:16} Röm 14,1-12} \bibverse{17} welches ist
der Schatten von dem, das zukünftig war; aber der Körper selbst ist in
Christo. \footnote{\textbf{2:17} Hebr 8,5; Hebr 10,1} \bibverse{18}
Lasst euch niemand das Ziel verrücken, der nach eigener Wahl einhergeht
in Demut und Geistlichkeit der Engel, davon er nie etwas gesehen hat,
und ist ohne Ursache aufgeblasen in seinem fleischlichen Sinn
\bibverse{19} und hält sich nicht an dem Haupt, aus welchem der ganze
Leib durch Gelenke und Fugen Handreichung empfängt und zusammengehalten
wird und also wächst zur göttlichen Größe. \footnote{\textbf{2:19} Eph
  4,15-16}

\bibverse{20} So ihr denn nun abgestorben seid mit Christo den Satzungen
der Welt, was lasset ihr euch denn fangen mit Satzungen, als lebtet ihr
noch in der Welt? \footnote{\textbf{2:20} Gal 4,9-10} \bibverse{21} „Du
sollst``, sagen sie, „das nicht angreifen, du sollst das nicht kosten,
du sollst das nicht anrühren``, \bibverse{22} was sich doch alles unter
den Händen verzehrt; es sind der Menschen Gebote und Lehren, \footnote{\textbf{2:22}
  Jes 29,13; Mt 15,9} \bibverse{23} welche haben einen Schein der
Weisheit durch selbst erwählte Geistlichkeit und Demut und dadurch, dass
sie des Leibes nicht schonen und dem Fleisch nicht seine Ehre tun zu
seiner Notdurft. \footnote{\textbf{2:23} Röm 13,14; 1Tim 4,3}

\hypertarget{section-1}{%
\section{3}\label{section-1}}

\bibverse{1} Seid ihr nun mit Christo auferstanden, so suchet, was
droben ist, da Christus ist, sitzend zu der Rechten Gottes. \footnote{\textbf{3:1}
  Kol 2,12} \bibverse{2} Trachtet nach dem, was droben ist, nicht nach
dem, was auf Erden ist. \footnote{\textbf{3:2} Mt 6,33} \bibverse{3}
Denn ihr seid gestorben, und euer Leben ist verborgen mit Christo in
Gott. \footnote{\textbf{3:3} Röm 6,2} \bibverse{4} Wenn aber Christus,
euer Leben, sich offenbaren wird, dann werdet ihr auch offenbar werden
mit ihm in der Herrlichkeit. \footnote{\textbf{3:4} 1Kor 15,43}

\bibverse{5} So tötet nun eure Glieder, die auf Erden sind, Hurerei,
Unreinigkeit, schändliche Brunst, böse Lust und den Geiz, welcher ist
Abgötterei, \footnote{\textbf{3:5} Eph 5,3} \bibverse{6} um welcher
willen kommt der Zorn Gottes über die Kinder des Unglaubens; \footnote{\textbf{3:6}
  Eph 5,6} \bibverse{7} in welchem auch ihr vordem gewandelt habt, da
ihr darin lebtet. \bibverse{8} Nun aber leget alles ab von euch: den
Zorn, Grimm, Bosheit, Lästerung, schandbare Worte aus eurem Munde.
\footnote{\textbf{3:8} Eph 4,29; Eph 4,31} \bibverse{9} Lüget nicht
untereinander; zieht den alten Menschen mit seinen Werken aus
\footnote{\textbf{3:9} Eph 4,22-25} \bibverse{10} und ziehet den neuen
an, der da erneuert wird zur Erkenntnis nach dem Ebenbilde des, der ihn
geschaffen hat; \footnote{\textbf{3:10} 1Mo 1,27; Eph 4,24}
\bibverse{11} da nicht ist Grieche, Jude, Beschnittener,
Unbeschnittener, Ungrieche, Scythe, Knecht, Freier, sondern alles und in
allen Christus. \footnote{\textbf{3:11} Gal 3,28}

\bibverse{12} So ziehet nun an, als die Auserwählten Gottes, Heiligen
und Geliebten, herzliches Erbarmen, Freundlichkeit, Demut, Sanftmut,
Geduld; \bibverse{13} und vertrage einer den anderen und vergebet euch
untereinander, wenn jemand Klage hat wider den anderen; gleichwie
Christus euch vergeben hat, also auch ihr. \footnote{\textbf{3:13} Mt
  6,14; Eph 4,2; Eph 4,32}

\bibverse{14} Über alles aber ziehet an die Liebe, die da ist das Band
der Vollkommenheit. \footnote{\textbf{3:14} Röm 13,8; Röm 13,10}
\bibverse{15} Und der Friede Gottes regiere in euren Herzen, zu welchem
ihr auch berufen seid in einem Leibe; und seid dankbar! \footnote{\textbf{3:15}
  1Kor 12,13; 1Kor 12,27; Eph 4,3-4; Phil 4,7} \bibverse{16} Lasset das
Wort Christi unter euch reichlich wohnen in aller Weisheit; lehret und
vermahnet euch selbst mit Psalmen und Lobgesängen und geistlichen
lieblichen Liedern und singt dem Herrn in eurem Herzen. \footnote{\textbf{3:16}
  Eph 5,19}

\bibverse{17} Und alles, was ihr tut mit Worten oder mit Werken, das tut
alles in dem Namen des Herrn Jesu, und danket Gott und dem Vater durch
ihn. \footnote{\textbf{3:17} 1Kor 10,31}

\bibverse{18} Ihr Weiber, seid untertan euren Männern in dem Herrn, wie
sich's gebührt. \footnote{\textbf{3:18} Eph 5,22-999}

\bibverse{19} Ihr Männer, liebet eure Weiber und seid nicht bitter gegen
sie. \footnote{\textbf{3:19} 1Petr 3,7}

\bibverse{20} Ihr Kinder, seid gehorsam euren Eltern in allen Dingen;
denn das ist dem Herrn gefällig.

\bibverse{21} Ihr Väter, erbittert eure Kinder nicht, auf dass sie nicht
scheu werden.

\bibverse{22} Ihr Knechte, seid gehorsam in allen Dingen euren
leiblichen Herren, nicht mit Dienst vor Augen, als den Menschen zu
gefallen, sondern mit Einfalt des Herzens und mit Gottesfurcht.
\bibverse{23} Alles, was ihr tut, das tut von Herzen als dem Herrn und
nicht den Menschen, \bibverse{24} und wisset, dass ihr von dem Herrn
empfangen werdet die Vergeltung des Erbes; denn ihr dienet dem Herrn
Christus. \bibverse{25} Wer aber Unrecht tut, der wird empfangen, was er
unrecht getan hat; und gilt kein Ansehen der Person. \# 4 \bibverse{1}
Ihr Herren, was recht und billig ist, das beweiset den Knechten, und
wisset, dass ihr auch einen Herrn im Himmel habt. \footnote{\textbf{4:1}
  3Mo 25,43; 3Mo 25,53}

\bibverse{2} Haltet an am Gebet und wachet in demselben mit Danksagung;
\footnote{\textbf{4:2} Röm 12,12; 1Thes 5,17} \bibverse{3} und betet
zugleich auch für uns, auf dass Gott uns eine Tür des Wortes auftue, zu
reden das Geheimnis Christi, darum ich auch gebunden bin, \footnote{\textbf{4:3}
  Kol 1,26-27; Röm 15,30; 1Kor 16,9; Eph 6,19; 2Thes 3,1} \bibverse{4}
auf dass ich es offenbare, wie ich soll reden.

\bibverse{5} Wandelt weise gegen die, die draußen sind, und kauft die
Zeit aus. \bibverse{6} Eure Rede sei allezeit lieblich und mit Salz
gewürzt, dass ihr wisst, wie ihr einem jeglichen antworten sollt.
\footnote{\textbf{4:6} Mk 9,50; Eph 4,29}

\bibverse{7} Wie es um mich steht, wird euch alles kundtun Tychikus, der
liebe Bruder und getreue Diener und Mitknecht in dem Herrn, \bibverse{8}
welchen ich habe darum zu euch gesandt, dass er erfahre, wie es sich mit
euch verhält, und dass er eure Herzen ermahne, \bibverse{9} samt
Onesimus, dem getreuen und lieben Bruder, welcher von den Euren ist.
Alles, wie es hier steht, werden sie euch kundtun. \footnote{\textbf{4:9}
  Phim 1,10}

\bibverse{10} Es grüßt euch Aristarchus, mein Mitgefangener, und Markus,
der Neffe des Barnabas, über welchen ihr etliche Befehle empfangen habt
(wenn er zu euch kommt, nehmt ihn auf!) \footnote{\textbf{4:10} Apg
  12,12; Apg 12,25; Apg 16,29; Apg 27,2} \bibverse{11} und Jesus, der da
heißt Just, die aus den Juden sind. Diese sind allein meine Gehilfen am
Reich Gottes, die mir ein Trost geworden sind.

\bibverse{12} Es grüßt euch Epaphras, der von den Euren ist, ein Knecht
Christi, und allezeit ringt für euch mit Gebeten, auf dass ihr bestehet
vollkommen und erfüllt mit allem Willen Gottes. \footnote{\textbf{4:12}
  Kol 1,7; Phim 1,23} \bibverse{13} Ich gebe ihm Zeugnis, dass er großen
Fleiß hat um euch und um die zu Laodizea und zu Hierapolis. \footnote{\textbf{4:13}
  Offb 1,11; Offb 3,14} \bibverse{14} Es grüßt euch Lukas, der Arzt, der
Geliebte, und Demas. \footnote{\textbf{4:14} 2Tim 4,10-11; Phim 1,24}
\bibverse{15} Grüßet die Brüder zu Laodizea und den Nymphas und die
Gemeinde in seinem Hause. \bibverse{16} Und wenn der Brief bei euch
gelesen ist, so schafft, dass er auch in der Gemeinde zu Laodizea
gelesen werde und dass ihr den von Laodizea lest. \bibverse{17} Und
saget Archippus: Siehe auf das Amt, das du empfangen hast in dem Herrn,
dass du es ausrichtest!

\bibverse{18} Mein Gruß mit meiner, des Paulus, Hand. Gedenket meiner
Bande! Die Gnade sei mit euch! Amen.
