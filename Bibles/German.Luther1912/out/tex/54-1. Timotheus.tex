\hypertarget{section}{%
\section{1}\label{section}}

\bibleverse{1} Paulus, ein Apostel Jesu Christi nach dem Befehl Gottes,
unseres Heilandes, und des Herrn Jesu Christi, der unsere Hoffnung ist,
\bibleverse{2} dem Timotheus, meinem rechtschaffenen Sohn im Glauben:
Gnade, Barmherzigkeit, Friede von Gott, unserem Vater, und unserem Herrn
Jesus Christus! \footnote{\textbf{1:2} Apg 16,1-2; Tit 1,4}
\bibleverse{3} Wie ich dich ermahnt habe, dass du zu Ephesus bliebest,
da ich nach Mazedonien zog, und gebötest etlichen, dass sie nicht anders
lehrten, \footnote{\textbf{3:4} 1Sam 2,12} \bibleverse{4} und nicht
achthätten auf die Fabeln und Geschlechtsregister, die kein Ende haben
und Fragen aufbringen mehr denn Besserung zu Gott im Glauben;
\footnote{\textbf{1:4} 1Tim 4,7} \bibleverse{5} denn die Hauptsumme des
Gebotes ist Liebe von reinem Herzen und von gutem Gewissen und von
ungefärbtem Glauben; \footnote{\textbf{5:16} Apg 6,1} \bibleverse{6}
wovon etliche sind abgeirrt und haben sich umgewandt zu unnützem
Geschwätz, \footnote{\textbf{1:6} 1Tim 6,4; 1Tim 6,20} \bibleverse{7}
wollen der Schrift Meister sein, und verstehen nicht, was sie sagen oder
was sie setzen. \bibleverse{8} Wir wissen aber, dass das Gesetz gut ist,
so es jemand recht braucht \bibleverse{9} und weiß solches, dass dem
Gerechten kein Gesetz gegeben ist, sondern den Ungerechten und
Ungehorsamen, den Gottlosen und Sündern, den Unheiligen und
Ungeistlichen, den Vatermördern und Muttermördern, den Totschlägern
\textsuperscript{f} \bibleverse{10} den Hurern, den Knabenschändern, den
Menschendieben, den Lügnern, den Meineidigen und wenn etwas mehr der
heilsamen Lehre zuwider ist, \bibleverse{11} nach dem herrlichen
Evangelium des seligen Gottes, welches mir vertrauet ist.
\bibleverse{12} Ich danke unserem Herrn Christus Jesus, der mich stark
gemacht und treu geachtet hat und gesetzt in das Amt, \bibleverse{13}
der ich zuvor war ein Lästerer und ein Verfolger und ein Schmäher; aber
mir ist Barmherzigkeit widerfahren, denn ich habe es unwissend getan im
Unglauben. \bibleverse{14} Es ist aber desto reicher gewesen die Gnade
unseres Herrn samt dem Glauben und der Liebe, die in Christo Jesu ist.
\bibleverse{15} Das ist gewisslich wahr und ein teuer wertes Wort, dass
Christus Jesus gekommen ist in die Welt, die Sünder selig zu machen,
unter welchen ich der vornehmste bin. \footnote{\textbf{1:15} Lk 19,10}
\bibleverse{16} Aber darum ist mir Barmherzigkeit widerfahren, auf dass
an mir vornehmlich Jesus Christus erzeigte alle Geduld, zum Vorbild
denen, die an ihn glauben sollten zum ewigen Leben. \bibleverse{17} Aber
Gott, dem ewigen König, dem Unvergänglichen und Unsichtbaren und allein
Weisen, sei Ehre und Preis in Ewigkeit! Amen. \bibleverse{18} Dies Gebot
befehle ich dir, mein Sohn Timotheus, nach den vorigen Weissagungen über
dich, dass du in ihnen eine gute Ritterschaft übest \bibleverse{19} und
habest den Glauben und gutes Gewissen, welches etliche von sich gestoßen
und am Glauben Schiffbruch erlitten haben; \footnote{\textbf{5:20} Gal
  2,14} \bibleverse{20} unter welchen ist Hymenäus und Alexander, welche
ich habe dem Satan übergeben, dass sie gezüchtigt werden, nicht mehr zu
lästern. \footnote{\textbf{1:20} 1Kor 5,5; 2Tim 2,17}
\footnote{\textbf{2:5} Hebr 9,15}{[}\textbf{1:3} Apg 20,1{]}
\footnote{\textbf{2:7} 2Tim 1,11; Gal 2,7-8}{[}\textbf{1:5} Mt 22,37-40;
Röm 13,10; Gal 5,6{]} \footnote{\textbf{2:9} 1Petr 3,3-5}{[}\textbf{1:9}
1Kor 6,9-11{]} \footnote{\textbf{2:11} Eph 5,22}{[}\textbf{1:19} 1Tim
3,9; 1Tim 6,10{]}

\hypertarget{section-1}{%
\section{2}\label{section-1}}

\bibleverse{1} So ermahne ich nun, dass man vor allen Dingen zuerst tue
Bitte, Gebet, Fürbitte und Danksagung für alle Menschen, \bibleverse{2}
für die Könige und für alle Obrigkeit, auf dass wir ein ruhiges und
stilles Leben führen mögen in aller Gottseligkeit und Ehrbarkeit.
\bibleverse{3} Denn solches ist gut und angenehm vor Gott, unserem
Heiland, \bibleverse{4} welcher will, dass allen Menschen geholfen werde
und sie zur Erkenntnis der Wahrheit kommen. \footnote{\textbf{2:4} Hes
  18,23; Röm 11,32; 2Petr 3,9} \bibleverse{5} Denn es ist ein Gott und
ein Mittler zwischen Gott und den Menschen, nämlich der Mensch Christus
Jesus, \textsuperscript{b} \bibleverse{6} der sich selbst gegeben hat
für alle zur Erlösung, dass solches zu seiner Zeit gepredigt würde;
\textsuperscript{c} \bibleverse{7} dazu ich gesetzt bin als Prediger und
Apostel (ich sage die Wahrheit in Christo und lüge nicht), als Lehrer
der Heiden im Glauben und in der Wahrheit. \textsuperscript{d}
\bibleverse{8} So will ich nun, dass die Männer beten an allen Orten und
aufheben heilige Hände ohne Zorn und Zweifel. \textsuperscript{e}
\bibleverse{9} Desgleichen dass die Weiber in zierlichem Kleide mit
Scham und Zucht sich schmücken, nicht mit Zöpfen oder Gold oder Perlen
oder köstlichem Gewand, \textsuperscript{f} \bibleverse{10} sondern, wie
sich's ziemt den Weibern, die da Gottseligkeit beweisen wollen, durch
gute Werke. \textsuperscript{g} \bibleverse{11} Ein Weib lerne in der
Stille mit aller Untertänigkeit. \textsuperscript{h} \bibleverse{12}
Einem Weibe aber gestatte ich nicht, dass sie lehre, auch nicht, dass
sie des Mannes Herr sei, sondern stille sei. \footnote{\textbf{2:12}
  1Kor 14,34; 1Mo 3,16} \bibleverse{13} Denn Adam ist am ersten gemacht,
darnach Eva. \bibleverse{14} Und Adam ward nicht verführt; das Weib aber
ward verführt und hat die Übertretung eingeführt. \bibleverse{15} Sie
wird aber selig werden durch Kinderzeugen, wenn sie bleiben im Glauben
und in der Liebe und in der Heiligung samt der Zucht. \footnote{\textbf{2:15}
  1Tim 5,14; Tit 2,4; Tit 1,2-5} \footnote{\textbf{3:9} 1Tim 1,19}{[}\textbf{2:6}
Gal 1,4; Gal 2,20; Tit 2,14{]} \footnote{\textbf{3:16} Joh 1,14; Röm
  1,4; Eph 1,20-21; Apg 28,28; Mk 16,19}{[}\textbf{2:8} Jak 1,6{]}
\footnote{\textbf{5:19} 5Mo 19,15; Mt 18,16}{[}\textbf{2:10} 1Tim
5,10{]}

\hypertarget{section-2}{%
\section{3}\label{section-2}}

\bibleverse{1} Das ist gewisslich wahr: Wenn jemand ein Bischofsamt
begehrt, der begehrt ein köstlich Werk. \footnote{\textbf{3:1} Apg
  20,28; Phil 1,1; Tit 1,5-9} \bibleverse{2} Es soll aber ein Bischof
unsträflich sein, eines Weibes Mann, nüchtern, mäßig, sittig, gastfrei,
lehrhaft, \bibleverse{3} nicht ein Weinsäufer, nicht raufen, nicht
unehrliche Hantierung treiben, sondern gelinde, nicht zänkisch, nicht
geizig, \bibleverse{4} der seinem eigenen Hause wohl vorstehe, der
gehorsame Kinder habe mit aller Ehrbarkeit, \textsuperscript{b}
\bibleverse{5} (so aber jemand seinem eigenen Hause nicht weiß
vorzustehen, wie wird er die Gemeinde Gottes versorgen?); \bibleverse{6}
nicht ein Neuling, auf dass er sich nicht aufblase und ins Urteil des
Lästerers falle. \bibleverse{7} Er muss aber auch ein gutes Zeugnis
haben von denen, die draußen sind, auf dass er nicht falle dem Lästerer
in Schmach und Strick. \bibleverse{8} Desgleichen die Diener sollen
ehrbar sein, nicht zweizüngig, nicht Weinsäufer, nicht unehrliche
Hantierung treiben; \bibleverse{9} die das Geheimnis des Glaubens in
reinem Gewissen haben. \footnote{\textbf{5:10} Joh 13,14; Hebr 13,2}
\bibleverse{10} Und diese lasse man zuvor versuchen; darnach lasse man
sie dienen, wenn sie unsträflich sind. \bibleverse{11} Desgleichen ihre
Weiber sollen ehrbar sein, nicht Lästerinnen, nüchtern, treu in allen
Dingen. \bibleverse{12} Die Diener lass einen jeglichen sein eines
Weibes Mann, die ihren Kindern wohl vorstehen und ihren eigenen Häusern.
\bibleverse{13} Welche aber wohl dienen, die erwerben sich selbst eine
gute Stufe und eine große Freudigkeit im Glauben an Christum Jesum.
\bibleverse{14} Solches schreibe ich dir und hoffe, bald zu dir zu
kommen; \bibleverse{15} wenn ich aber verzöge, dass du wissest, wie du
wandeln sollst in dem Hause Gottes, welches ist die Gemeinde des
lebendigen Gottes, ein Pfeiler und eine Grundfeste der Wahrheit.
\textsuperscript{d} \bibleverse{16} Und kündlich groß ist das gottselige
Geheimnis: Gott ist offenbart im Fleisch, gerechtfertigt im Geist,
erschienen den Engeln, gepredigt den Heiden, geglaubt von der Welt,
aufgenommen in die Herrlichkeit. \textsuperscript{e}
\footnote{\textbf{4:8} 1Tim 6,6; Hebr 13,9}{[}\textbf{3:15} Eph
2,19-22{]}

\hypertarget{section-3}{%
\section{4}\label{section-3}}

\bibleverse{1} Der Geist aber sagt deutlich, dass in den letzten Zeiten
werden etliche von dem Glauben abtreten und anhangen den verführerischen
Geistern und Lehren der Teufel \footnote{\textbf{4:1} Mt 24,24; 2Thes
  2,3; 2Tim 3,1; 2Petr 3,3; 1Jo 2,18; Jud 1,18} \bibleverse{2} durch
die, die in Gleisnerei Lügen reden und Brandmal in ihrem Gewissen haben,
\bibleverse{3} die da gebieten, nicht ehelich zu werden und zu meiden
die Speisen, die Gott geschaffen hat zu nehmen mit Danksagung, den
Gläubigen und denen, die die Wahrheit erkennen. \bibleverse{4} Denn alle
Kreatur Gottes ist gut, und nichts ist verwerflich, das mit Danksagung
empfangen wird; \textsuperscript{b} \bibleverse{5} denn es wird
geheiligt durch das Wort Gottes und Gebet. \bibleverse{6} Wenn du den
Brüdern solches vorhältst, so wirst du ein guter Diener Jesu Christi
sein, auferzogen in den Worten des Glaubens und der guten Lehre, bei
welcher du immerdar gewesen bist. \bibleverse{7} Aber der ungeistlichen
Altweiberfabeln entschlage dich; übe dich selbst aber in der
Gottseligkeit. \footnote{\textbf{4:7} 1Tim 6,20; 2Tim 2,16; 2Tim 2,23;
  2Tim 4,4; Tit 1,14; Tit 3,9} \bibleverse{8} Denn die leibliche Übung
ist wenig nütz; aber die Gottseligkeit ist zu allen Dingen nütz und hat
die Verheißung dieses und des zukünftigen Lebens. \textsuperscript{d}
\bibleverse{9} Das ist gewisslich wahr und ein teuer wertes Wort.
\bibleverse{10} Denn dahin arbeiten wir auch und werden geschmäht, dass
wir auf den lebendigen Gott gehofft haben, welcher ist der Heiland aller
Menschen, sonderlich der Gläubigen. \bibleverse{11} Solches gebiete und
lehre. \bibleverse{12} Niemand verachte deine Jugend; sondern sei ein
Vorbild den Gläubigen im Wort, im Wandel, in der Liebe, im Geist, im
Glauben, in der Keuschheit. \textsuperscript{e} \bibleverse{13} Halte an
mit Lesen, mit Ermahnen, mit Lehren, bis ich komme. \bibleverse{14} Lass
nicht aus der Acht die Gabe, die dir gegeben ist durch die Weissagung
mit Handauflegung der Ältesten. \bibleverse{15} Dessen warte, damit gehe
um, auf dass dein Zunehmen in allen Dingen offenbar sei. \bibleverse{16}
Habe acht auf dich selbst und auf die Lehre; beharre in diesen Stücken.
Denn wo du solches tust, wirst du dich selbst selig machen und die dich
hören. \textsuperscript{f} \footnote{\textbf{5:5} Lk 2,37}{[}\textbf{4:4}
1Mo 1,31; Mt 15,11; Apg 10,15{]} \footnote{\textbf{5:17} Apg 14,23; Röm
  12,8}{[}\textbf{4:12} Tit 2,15; 2Tim 2,22{]}
\footnote{\textbf{5:18} 1Kor 9,9; Lk 10,7}{[}\textbf{4:16} Röm 11,14{]}

\hypertarget{section-4}{%
\section{5}\label{section-4}}

\bibleverse{1} Einen Alten schilt nicht, sondern ermahne ihn als einen
Vater, die Jungen als Brüder, \footnote{\textbf{5:1} 3Mo 19,32; Tit 2,2}
\bibleverse{2} die alten Weiber als Mütter, die jungen als Schwestern
mit aller Keuschheit. \bibleverse{3} Ehre die Witwen, welche rechte
Witwen sind. \bibleverse{4} So aber eine Witwe Enkel oder Kinder hat,
solche lass zuvor lernen, ihre eigenen Häuser göttlich regieren und den
Eltern Gleiches vergelten; denn das ist wohl getan und angenehm vor
Gott. \bibleverse{5} Das ist aber eine rechte Witwe, die einsam ist, die
ihre Hoffnung auf Gott stellt und bleibt am Gebet und Flehen Tag und
Nacht. \textsuperscript{b} \bibleverse{6} Welche aber in Wollüsten lebt,
die ist lebendig tot. \bibleverse{7} Solches gebiete, auf dass sie
untadelig seien. \bibleverse{8} So aber jemand die Seinen, sonderlich
seine Hausgenossen, nicht versorgt, der hat den Glauben verleugnet und
ist ärger denn ein Heide. \bibleverse{9} Lass keine Witwe erwählt werden
unter sechzig Jahren, und die da gewesen sei eines Mannes Weib,
\bibleverse{10} und die ein Zeugnis habe guter Werke: wenn sie Kinder
aufgezogen hat, wenn sie gastfrei gewesen ist, wenn sie der Heiligen
Füße gewaschen hat, wenn sie den Trübseligen Handreichung getan hat,
wenn sie allem guten Werk nachgekommen ist. \textsuperscript{c}
\bibleverse{11} Der jungen Witwen aber entschlage dich; denn wenn sie
geil geworden sind wider Christum, so wollen sie freien \bibleverse{12}
und haben ihr Urteil, dass sie den ersten Glauben gebrochen haben.
\bibleverse{13} Daneben sind sie faul und lernen umlaufen durch die
Häuser; nicht allein aber sind sie faul, sondern auch geschwätzig und
vorwitzig und reden, was nicht sein soll. \bibleverse{14} So will ich
nun, dass die jungen Witwen freien, Kinder zeugen, haushalten, dem
Widersacher keine Ursache geben zu schelten. \bibleverse{15} Denn es
sind schon etliche umgewandt dem Satan nach. \bibleverse{16} Wenn aber
ein Gläubiger oder Gläubige Witwen hat, der versorge sie und lasse die
Gemeinde nicht beschwert werden, auf dass die, die rechte Witwen sind,
mögen genug haben. \textsuperscript{d} \bibleverse{17} Die Ältesten, die
wohl vorstehen, die halte man zwiefacher Ehre wert, sonderlich die da
arbeiten im Wort und in der Lehre. \textsuperscript{e} \bibleverse{18}
Denn es spricht die Schrift: „Du sollst nicht dem Ochsen das Maul
verbinden, der da drischt;`` und: „Ein Arbeiter ist seines Lohnes
wert.`` \textsuperscript{f} \bibleverse{19} Wider einen Ältesten nimm
keine Klage an ohne zwei oder drei Zeugen. \textsuperscript{g}
\bibleverse{20} Die da sündigen, die strafe vor allen, auf dass sich
auch die anderen fürchten. \textsuperscript{h} \bibleverse{21} Ich
bezeuge vor Gott und dem Herrn Jesus Christus und den auserwählten
Engeln, dass du solches haltest ohne eigenes Gutdünken und nichts tust
nach Gunst. \bibleverse{22} Die Hände lege niemand zu bald auf, mache
dich auch nicht teilhaftig fremder Sünden. Halte dich selber keusch.
\bibleverse{23} Trinke nicht mehr Wasser, sondern brauche auch ein wenig
Wein um deines Magens willen und weil du oft krank bist. \bibleverse{24}
Etlicher Menschen Sünden sind offenbar, dass man sie zuvor richten kann;
bei etlichen aber werden sie hernach offenbar. \bibleverse{25}
Desgleichen auch etlicher gute Werke sind zuvor offenbar, und die
anderen bleiben auch nicht verborgen.

\hypertarget{section-5}{%
\section{6}\label{section-5}}

\bibleverse{1} Die Knechte, die unter dem Joch sind, sollen ihre Herren
aller Ehre wert halten, auf dass nicht der Name Gottes und die Lehre
verlästert werde. \textsuperscript{a} \bibleverse{2} Welche aber
gläubige Herren haben, sollen sie nicht verachten, weil sie Brüder sind,
sondern sollen viel mehr dienstbar sein, dieweil sie gläubig und geliebt
und der Wohltat teilhaftig sind. Solches lehre und ermahne.
\textsuperscript{b} \bibleverse{3} Wenn jemand anders lehrt und bleibt
nicht bei den heilsamen Worten unseres Herrn Jesu Christi und bei der
Lehre, die gemäß ist der Gottseligkeit, \textsuperscript{c}
\bibleverse{4} der ist aufgeblasen und weiß nichts, sondern hat die
Seuche der Fragen und Wortkriege, aus welchen entspringt Neid, Hader,
Lästerung, böser Argwohn. \textsuperscript{d} \bibleverse{5}
Schulgezänke solcher Menschen, die zerrüttete Sinne haben und der
Wahrheit beraubt sind, die da meinen, Gottseligkeit sei ein Gewerbe. Tue
dich von solchen! \textsuperscript{e} \bibleverse{6} Es ist aber ein
großer Gewinn, wer gottselig ist und lässet sich genügen. \bibleverse{7}
Denn wir haben nichts in die Welt gebracht; darum offenbar ist, wir
werden auch nichts hinausbringen. \bibleverse{8} Wenn wir aber Nahrung
und Kleider haben, so lasset uns genügen. \textsuperscript{f}
\bibleverse{9} Denn die da reich werden wollen, die fallen in Versuchung
und Stricke und viel törichte und schädliche Lüste, welche versenken die
Menschen ins Verderben und Verdammnis. \textsuperscript{g}
\bibleverse{10} Denn Geiz ist eine Wurzel alles Übels; das hat etliche
gelüstet und sind vom Glauben irregegangen und machen sich selbst viel
Schmerzen. \textsuperscript{h} \bibleverse{11} Aber du, Gottesmensch,
fliehe solches! Jage aber nach -- der Gerechtigkeit, der Gottseligkeit,
dem Glauben, der Liebe, der Geduld, der Sanftmut; \textsuperscript{i}
\bibleverse{12} kämpfe den guten Kampf des Glaubens; ergreife das ewige
Leben, dazu du auch berufen bist und bekannt hast ein gutes Bekenntnis
vor vielen Zeugen. \textsuperscript{j} \bibleverse{13} Ich gebiete dir
vor Gott, der alle Dinge lebendig macht, und vor Christo Jesu, der unter
Pontius Pilatus bezeugt hat ein gutes Bekenntnis, \textsuperscript{k}
\bibleverse{14} dass du haltest das Gebot ohne Flecken, untadelig, bis
auf die Erscheinung unseres Herrn Jesu Christi, \bibleverse{15} welche
wird zeigen zu seiner Zeit der Selige und allein Gewaltige, der König
aller Könige und Herr aller Herren, \textsuperscript{l} \bibleverse{16}
der allein Unsterblichkeit hat, der da wohnt in einem Licht, da niemand
zukommen kann, welchen kein Mensch gesehen hat noch sehen kann; dem sei
Ehre und ewiges Reich! Amen. \textsuperscript{m} \bibleverse{17} Den
Reichen von dieser Welt gebiete, dass sie nicht stolz seien, auch nicht
hoffen auf den ungewissen Reichtum, sondern auf den lebendigen Gott, der
uns dargibt reichlich, allerlei zu genießen; \textsuperscript{n}
\bibleverse{18} dass sie Gutes tun, reich werden an guten Werken, gern
geben, behilflich seien, \bibleverse{19} Schätze sammeln, sich selbst
einen guten Grund aufs Zukünftige, dass sie ergreifen das wahre Leben.
\bibleverse{20} O Timotheus! bewahre, was dir vertraut ist, und meide
die ungeistlichen, losen Geschwätze und das Gezänke der falsch berühmten
Kunst, \textsuperscript{o} \bibleverse{21} welche etliche vorgeben und
gehen vom Glauben irre. Die Gnade sei mit dir! Amen.
