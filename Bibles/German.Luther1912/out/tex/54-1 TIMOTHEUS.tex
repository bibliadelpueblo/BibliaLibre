\hypertarget{section}{%
\section{1}\label{section}}

\bibverse{1} Paulus, ein Apostel Jesu Christi nach dem Befehl Gottes,
unsers Heilandes, und des HERRN Jesu Christi, der unsre Hoffnung ist,
\bibverse{2} dem Timotheus, meinem rechtschaffenen Sohn im Glauben:
Gnade, Barmherzigkeit, Friede von Gott, unserm Vater, und unserm HERRN
Jesus Christus! \bibverse{3} Wie ich dich ermahnt habe, daß du zu
Ephesus bliebest, da ich nach Mazedonien zog, und gebötest etlichen, daß
sie nicht anders lehrten, \bibverse{4} und nicht acht hätten auf die
Fabeln und Geschlechtsregister, die kein Ende haben und Fragen
aufbringen mehr denn Besserung zu Gott im Glauben; \bibverse{5} denn die
Hauptsumme des Gebotes ist Liebe von reinem Herzen und von gutem
Gewissen und von ungefärbtem Glauben; \bibverse{6} wovon etliche sind
abgeirrt und haben sich umgewandt zu unnützem Geschwätz, \bibverse{7}
wollen der Schrift Meister sein, und verstehen nicht, was sie sagen oder
was sie setzen. \bibverse{8} Wir wissen aber, daß das Gesetz gut ist, so
es jemand recht braucht \bibverse{9} und weiß solches, daß dem Gerechten
kein Gesetz gegeben ist, sondern den Ungerechten und Ungehorsamen, den
Gottlosen und Sündern, den Unheiligen und Ungeistlichen, den
Vatermördern und Muttermördern, den Totschlägern \bibverse{10} den
Hurern, den Knabenschändern, den Menschendieben, den Lügnern, den
Meineidigen und so etwas mehr der heilsamen Lehre zuwider ist,
\bibverse{11} nach dem herrlichen Evangelium des seligen Gottes, welches
mir anvertrauet ist. \bibverse{12} Ich danke unserm HERR Christus Jesus,
der mich stark gemacht und treu geachtet hat und gesetzt in das Amt,
\bibverse{13} der ich zuvor war ein Lästerer und ein Verfolger und ein
Schmäher; aber mir ist Barmherzigkeit widerfahren, denn ich habe es
unwissend getan im Unglauben. \bibverse{14} Es ist aber desto reicher
gewesen die Gnade unsers HERRN samt dem Glauben und der Liebe, die in
Christo Jesu ist. \bibverse{15} Das ist gewißlich wahr und ein teuer
wertes Wort, daß Christus Jesus gekommen ist in die Welt, die Sünder
selig zu machen, unter welchen ich der vornehmste bin. \bibverse{16}
Aber darum ist mir Barmherzigkeit widerfahren, auf daß an mir
vornehmlich Jesus Christus erzeigte alle Geduld, zum Vorbild denen, die
an ihn glauben sollten zum ewigen Leben. \bibverse{17} Aber Gott, dem
ewigen König, dem Unvergänglichen und Unsichtbaren und allein Weisen,
sei Ehre und Preis in Ewigkeit! Amen. \bibverse{18} Dies Gebot befehle
ich dir, mein Sohn Timotheus, nach den vorherigen Weissagungen über
dich, daß du in ihnen eine gute Ritterschaft übest \bibverse{19} und
habest den Glauben und gutes Gewissen, welches etliche von sich gestoßen
und am Glauben Schiffbruch erlitten haben; \bibverse{20} unter welchen
ist Hymenäus und Alexander, welche ich habe dem Satan übergeben, daß sie
gezüchtigt werden, nicht mehr zu lästern.

\hypertarget{section-1}{%
\section{2}\label{section-1}}

\bibverse{1} So ermahne ich euch nun, daß man vor allen Dingen zuerst
tue Bitte, Gebet, Fürbitte und Danksagung für alle Menschen,
\bibverse{2} für die Könige und alle Obrigkeit, auf daß wir ein ruhiges
und stilles Leben führen mögen in aller Gottseligkeit und Ehrbarkeit.
\bibverse{3} Denn solches ist gut und angenehm vor Gott, unserm Heiland,
\bibverse{4} welcher will, daß allen Menschen geholfen werde und sie zur
Erkenntnis der Wahrheit kommen. \bibverse{5} Denn es ist ein Gott und
ein Mittler zwischen Gott und den Menschen, nämlich der Mensch Christus
Jesus, \bibverse{6} der sich selbst gegeben hat für alle zur Erlösung,
daß solches zu seiner Zeit gepredigt würde; \bibverse{7} dazu ich
gesetzt bin als Prediger und Apostel (ich sage die Wahrheit in Christo
und lüge nicht), als Lehrer der Heiden im Glauben und in der Wahrheit.
\bibverse{8} So will ich nun, daß die Männer beten an allen Orten und
aufheben heilige Hände ohne Zorn und Zweifel. \bibverse{9} Desgleichen
daß die Weiber in zierlichem Kleide mit Scham und Zucht sich schmücken,
nicht mit Zöpfen oder Gold oder Perlen oder köstlichem Gewand,
\bibverse{10} sondern, wie sich's ziemt den Weibern, die da
Gottseligkeit beweisen wollen, durch gute Werke. \bibverse{11} Ein Weib
lerne in der Stille mit aller Untertänigkeit. \bibverse{12} Einem Weibe
aber gestatte ich nicht, daß sie lehre, auch nicht, daß sie des Mannes
Herr sei, sondern stille sei. \bibverse{13} Denn Adam ist am ersten
gemacht, darnach Eva. \bibverse{14} Und Adam ward nicht verführt; das
Weib aber ward verführt und hat die Übertretung eingeführt.
\bibverse{15} Sie wird aber selig werden durch Kinderzeugen, so sie
bleiben im Glauben und in der Liebe und in der Heiligung samt der Zucht.

\hypertarget{section-2}{%
\section{3}\label{section-2}}

\bibverse{1} Das ist gewißlich wahr: So jemand ein Bischofsamt begehrt,
der begehrt ein köstlich Werk. \bibverse{2} Es soll aber ein Bischof
unsträflich sein, eines Weibes Mann, nüchtern, mäßig, sittig, gastfrei,
lehrhaft, \bibverse{3} nicht ein Weinsäufer, nicht raufen, nicht
unehrliche Hantierung treiben, sondern gelinde, nicht zänkisch, nicht
geizig, \bibverse{4} der seinem eigenen Hause wohl vorstehe, der
gehorsame Kinder habe mit aller Ehrbarkeit, \bibverse{5} (so aber jemand
seinem eigenen Hause nicht weiß vorzustehen, wie wird er die Gemeinde
Gottes versorgen?); \bibverse{6} Nicht ein Neuling, auf daß er sich
nicht aufblase und ins Urteil des Lästerers falle. \bibverse{7} Er muß
aber auch ein gutes Zeugnis haben von denen, die draußen sind, auf daß
er nicht falle dem Lästerer in Schmach und Strick. \bibverse{8}
Desgleichen die Diener sollen ehrbar sein, nicht zweizüngig, nicht
Weinsäufer, nicht unehrliche Hantierungen treiben; \bibverse{9} die das
Geheimnis des Glaubens in reinem Gewissen haben. \bibverse{10} Und diese
lasse man zuvor versuchen; darnach lasse man sie dienen, wenn sie
unsträflich sind. \bibverse{11} Desgleichen ihre Weiber sollen ehrbar
sein, nicht Lästerinnen, nüchtern, treu in allen Dingen. \bibverse{12}
Die Diener laß einen jeglichen sein eines Weibes Mann, die ihren Kindern
wohl vorstehen und ihren eigenen Häusern. \bibverse{13} Welche aber wohl
dienen, die erwerben sich selbst eine gute Stufe und eine große
Freudigkeit im Glauben an Christum Jesum. \bibverse{14} Solches schreibe
ich dir und hoffe, bald zu dir zu kommen; \bibverse{15} so ich aber
verzöge, daß du wissest, wie du wandeln sollst in dem Hause Gottes,
welches ist die Gemeinde des lebendigen Gottes, ein Pfeiler und eine
Grundfeste der Wahrheit. \bibverse{16} Und kündlich groß ist das
gottselige Geheimnis: Gott ist offenbart im Fleisch, gerechtfertigt im
Geist, erschienen den Engeln, gepredigt den Heiden, geglaubt von der
Welt, aufgenommen in die Herrlichkeit.

\hypertarget{section-3}{%
\section{4}\label{section-3}}

\bibverse{1} Der Geist aber sagt deutlich, daß in den letzten Zeiten
werden etliche von dem Glauben abtreten und anhangen den verführerischen
Geistern und Lehren der Teufel \bibverse{2} durch die, so in Gleisnerei
Lügen reden und Brandmal in ihrem Gewissen haben, \bibverse{3} die da
gebieten, nicht ehelich zu werden und zu meiden die Speisen, die Gott
geschaffen hat zu nehmen mit Danksagung, den Gläubigen und denen, die
die Wahrheit erkennen. \bibverse{4} Denn alle Kreatur Gottes ist gut,
und nichts ist verwerflich, das mit Danksagung empfangen wird;
\bibverse{5} denn es wird geheiligt durch das Wort Gottes und Gebet.
\bibverse{6} Wenn du den Brüdern solches vorhältst, so wirst du ein
guter Diener Jesu Christi sein, auferzogen in den Worten des Glaubens
und der guten Lehre, bei welcher du immerdar gewesen bist. \bibverse{7}
Aber der ungeistlichen Altweiberfabeln entschlage dich; übe dich selbst
aber in der Gottseligkeit. \bibverse{8} Denn die leibliche Übung ist
wenig nütz; aber die Gottseligkeit ist zu allen Dingen nütz und hat die
Verheißung dieses und des zukünftigen Lebens. \bibverse{9} Das ist
gewißlich wahr und ein teuer wertes Wort. \bibverse{10} Denn dahin
arbeiten wir auch und werden geschmäht, daß wir auf den lebendigen Gott
gehofft haben, welcher ist der Heiland aller Menschen, sonderlich der
Gläubigen. \bibverse{11} Solches gebiete und lehre. \bibverse{12}
Niemand verachte deine Jugend; sondern sei ein Vorbild den Gläubigen im
Wort, im Wandel, in der Liebe, im Geist, im Glauben, in der Keuschheit.
\bibverse{13} Halte an mit Lesen, mit Ermahnen, mit Lehren, bis ich
komme. \bibverse{14} Laß nicht aus der Acht die Gabe, die dir gegeben
ist durch die Weissagung mit Handauflegung der Ältesten. \bibverse{15}
Dessen warte, gehe damit um, auf daß dein Zunehmen in allen Dingen
offenbar sei. \bibverse{16} Habe acht auf dich selbst und auf die Lehre;
beharre in diesen Stücken. Denn wo du solches tust, wirst du dich selbst
selig machen und die dich hören.

\hypertarget{section-4}{%
\section{5}\label{section-4}}

\bibverse{1} Einen Alten schilt nicht, sondern ermahne ihn als einen
Vater, die Jungen als Brüder, \bibverse{2} Die alten Weiber als Mütter,
die jungen als Schwestern mit aller Keuschheit. \bibverse{3} Ehre die
Witwen, welche rechte Witwen sind. \bibverse{4} So aber eine Witwe Enkel
oder Kinder hat, solche laß zuvor lernen, ihre eigenen Häuser göttlich
regieren und den Eltern Gleiches vergelten; denn das ist wohl getan und
angenehm vor Gott. \bibverse{5} Das ist aber die rechte Witwe, die
einsam ist, die ihre Hoffnung auf Gott stellt und bleibt am Gebet und
Flehen Tag und Nacht. \bibverse{6} Welche aber in Wollüsten lebt, die
ist lebendig tot. \bibverse{7} Solches gebiete, auf daß sie untadelig
seien. \bibverse{8} So aber jemand die Seinen, sonderlich seine
Hausgenossen, nicht versorgt, der hat den Glauben verleugnet und ist
ärger denn ein Heide. \bibverse{9} Laß keine Witwe erwählt werden unter
sechzig Jahren, und die da gewesen sei eines Mannes Weib, \bibverse{10}
und die ein Zeugnis habe guter Werke: so sie Kinder aufgezogen hat, so
sie gastfrei gewesen ist, so sie der Heiligen Füße gewaschen hat, so sie
den Trübseligen Handreichung getan hat, so sie in allem guten Werk
nachgekommen ist. \bibverse{11} Der jungen Witwen aber entschlage dich;
denn wenn sie geil geworden sind wider Christum, so wollen sie freien
\bibverse{12} und haben ihr Urteil, daß sie den ersten Glauben gebrochen
haben. \bibverse{13} Daneben sind sie faul und lernen umlaufen durch die
Häuser; nicht allein aber sind sie faul sondern auch geschwätzig und
vorwitzig und reden, was nicht sein soll. \bibverse{14} So will ich nun,
daß die jungen Witwen freien, Kinder zeugen, haushalten, dem Widersacher
keine Ursache geben zu schelten. \bibverse{15} Denn es sind schon
etliche umgewandt dem Satan nach. \bibverse{16} So aber ein Gläubiger
oder Gläubige Witwen hat, der versorge sie und lasse die Gemeinde nicht
beschwert werden, auf daß die, so rechte Witwen sind, mögen genug haben.
\bibverse{17} Die Ältesten, die wohl vorstehen, die halte man zweifacher
Ehre wert, sonderlich die da arbeiten im Wort und in der Lehre.
\bibverse{18} Denn es spricht die Schrift: ``Du sollst dem Ochsen nicht
das Maul verbinden, der da drischt;'' und ``Ein Arbeiter ist seines
Lohnes wert.'' \bibverse{19} Wider einen Ältesten nimm keine Klage an
ohne zwei oder drei Zeugen. \bibverse{20} Die da sündigen, die strafe
vor allen, auf daß sich auch die andern fürchten. \bibverse{21} Ich
bezeuge vor Gott und dem HERRN Jesus Christus und den auserwählten
Engeln, daß du solches haltest ohne eigenes Gutdünken und nichts tust
nach Gunst. \bibverse{22} Die Hände lege niemand zu bald auf, mache dich
auch nicht teilhaftig fremder Sünden. Halte dich selber keusch.
\bibverse{23} Trinke nicht mehr Wasser, sondern auch ein wenig Wein um
deines Magens willen und weil du oft krank bist. \bibverse{24} Etlicher
Menschen Sünden sind offenbar, daß man sie zuvor richten kann; bei
etlichen aber werden sie hernach offenbar. \bibverse{25} Desgleichen
auch etlicher gute Werke sind zuvor offenbar, und die andern bleiben
auch nicht verborgen.

\hypertarget{section-5}{%
\section{6}\label{section-5}}

\bibverse{1} Die Knechte, so unter dem Joch sind, sollen ihre Herren
aller Ehre wert halten, auf daß nicht der Name Gottes und die Lehre
verlästert werde. \bibverse{2} Welche aber gläubige Herren haben, sollen
sie nicht verachten, weil sie Brüder sind, sondern sollen viel mehr
dienstbar sein, dieweil sie gläubig und geliebt und der Wohltat
teilhaftig sind. Solches lehre und ermahne. \bibverse{3} So jemand
anders lehrt und bleibt nicht bei den heilsamen Worten unsers HERRN Jesu
Christi und bei der Lehre, die gemäß ist der Gottseligkeit, \bibverse{4}
der ist aufgeblasen und weiß nichts, sondern hat die Seuche der Fragen
und Wortkriege, aus welchen entspringt Neid, Hader, Lästerung, böser
Argwohn. \bibverse{5} Schulgezänke solcher Menschen, die zerrüttete
Sinne haben und der Wahrheit beraubt sind, die da meinen, Gottseligkeit
sei ein Gewerbe. Tue dich von solchen! \bibverse{6} Es ist aber ein
großer Gewinn, wer gottselig ist und lässet sich genügen. \bibverse{7}
Denn wir haben nichts in die Welt gebracht; darum offenbar ist, wir
werden auch nichts hinausbringen. \bibverse{8} Wenn wir aber Nahrung und
Kleider haben, so lasset uns genügen. \bibverse{9} Denn die da reich
werden wollen, die fallen in Versuchung und Stricke und viel törichte
und schädliche Lüste, welche versenken die Menschen ins Verderben und
Verdammnis. \bibverse{10} Denn Geiz ist eine Wurzel alles Übels; das hat
etliche gelüstet und sind vom Glauben irregegangen und machen sich
selbst viel Schmerzen. \bibverse{11} Aber du, Gottesmensch, fliehe
solches! Jage aber nach der Gerechtigkeit, der Gottseligkeit, dem
Glauben, der Liebe, der Geduld, der Sanftmut; \bibverse{12} kämpfe den
guten Kampf des Glaubens; ergreife das ewige Leben, dazu du auch berufen
bist und bekannt hast ein gutes Bekenntnis vor vielen Zeugen.
\bibverse{13} Ich gebiete dir vor Gott, der alle Dinge lebendig macht,
und vor Christo Jesu, der unter Pontius Pilatus bezeugt hat ein gutes
Bekenntnis, \bibverse{14} daß du haltest das Gebot ohne Flecken,
untadelig, bis auf die Erscheinung unsers HERRN Jesu Christi,
\bibverse{15} welche wird zeigen zu seiner Zeit der Selige und allein
Gewaltige, der König aller Könige und HERR aller Herren. \bibverse{16}
der allein Unsterblichkeit hat, der da wohnt in einem Licht, da niemand
zukommen kann, welchen kein Mensch gesehen hat noch sehen kann; dem sei
Ehre und ewiges Reich! Amen. \bibverse{17} Den Reichen von dieser Welt
gebiete, daß sie nicht stolz seien, auch nicht hoffen auf den ungewissen
Reichtum, sondern auf den lebendigen Gott, der uns dargibt reichlich,
allerlei zu genießen; \bibverse{18} daß sie Gutes tun, reich werden an
guten Werken, gern geben, behilflich seien, \bibverse{19} Schätze
sammeln, sich selbst einen guten Grund aufs Zukünftige, daß sie
ergreifen das wahre Leben. \bibverse{20} O Timotheus! bewahre, was dir
vertraut ist, und meide die ungeistlichen, losen Geschwätze und das
Gezänke der falsch berühmten Kunst, \bibverse{21} welche etliche
vorgeben und gehen vom Glauben irre. Die Gnade sei mit dir! Amen.
