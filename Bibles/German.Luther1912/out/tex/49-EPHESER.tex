\hypertarget{section}{%
\section{1}\label{section}}

\bibverse{1} Paulus, ein Apostel Jesu Christi durch den Willen Gottes,
den Heiligen zu Ephesus und Gläubigen an Christum Jesum: \bibverse{2}
Gnade sei mit euch und Friede von Gott, unserem Vater, und dem Herrn
Jesus Christus!

\bibverse{3} Gelobet sei Gott und der Vater unseres Herrn Jesu Christi,
der uns gesegnet hat mit allerlei geistlichem Segen in himmlischen
Gütern durch Christum; \bibverse{4} wie er uns denn erwählt hat durch
denselben, ehe der Welt Grund gelegt war, dass wir sollten sein heilig
und unsträflich vor ihm in der Liebe; \bibverse{5} und hat uns verordnet
zur Kindschaft gegen sich selbst durch Jesum Christum nach dem
Wohlgefallen seines Willens, \bibverse{6} zu Lob seiner herrlichen
Gnade, durch welche er uns hat angenehm gemacht in dem Geliebten,
\footnote{\textbf{1:6} Mt 3,17} \bibverse{7} an welchem wir haben die
Erlösung durch sein Blut, die Vergebung der Sünden, nach dem Reichtum
seiner Gnade, \footnote{\textbf{1:7} Kol 1,14; Eph 2,7; Eph 3,8; Eph
  3,16} \bibverse{8} welche uns reichlich widerfahren ist durch allerlei
Weisheit und Klugheit; \bibverse{9} und er hat uns wissen lassen das
Geheimnis seines Willens nach seinem Wohlgefallen, das er sich
vorgesetzt hatte in ihm, \footnote{\textbf{1:9} Eph 3,3-999; Röm 16,25;
  Kol 1,26-27} \bibverse{10} dass es ausgeführt würde, da die Zeit
erfüllet war, auf dass alle Dinge zusammengefasst würden in Christo,
beides, das im Himmel und auf Erden ist, durch ihn, \footnote{\textbf{1:10}
  Gal 4,4} \bibverse{11} durch welchen wir auch zum Erbteil gekommen
sind, die wir zuvor verordnet sind nach dem Vorsatz des, der alle Dinge
wirkt nach dem Rat seines Willens, \footnote{\textbf{1:11} Kol 1,12}
\bibverse{12} auf dass wir etwas seien zu Lob seiner Herrlichkeit, die
wir zuvor auf Christum hofften; \bibverse{13} durch welchen auch ihr
gehört habt das Wort der Wahrheit, das Evangelium von eurer Seligkeit;
durch welchen ihr auch, da ihr gläubig wurdet, versiegelt worden seid
mit dem Heiligen Geist der Verheißung, \bibverse{14} welcher ist das
Pfand unseres Erbes zu unserer Erlösung, dass wir sein Eigentum würden
zu Lob seiner Herrlichkeit. \footnote{\textbf{1:14} 2Kor 1,22; 2Kor 5,5}

\bibverse{15} Darum auch ich, nachdem ich gehört habe von dem Glauben
bei euch an den Herrn Jesus und von eurer Liebe zu allen Heiligen,
\bibverse{16} höre ich nicht auf, zu danken für euch, und gedenke euer
in meinem Gebet, \bibverse{17} dass der Gott unseres Herrn Jesus
Christi, der Vater der Herrlichkeit, gebe euch den Geist der Weisheit
und der Offenbarung zu seiner selbst Erkenntnis \bibverse{18} und
erleuchtete Augen eures Verständnisses, dass ihr erkennen möget, welche
da sei die Hoffnung eurer Berufung, und welcher sei der Reichtum seines
herrlichen Erbes bei seinen Heiligen, \bibverse{19} und welche da sei
die überschwengliche Größe seiner Kraft an uns, die wir glauben nach der
Wirkung seiner mächtigen Stärke, \bibverse{20} welche er gewirkt hat in
Christo, da er ihn von den Toten auferweckt hat und gesetzt zu seiner
Rechten im Himmel \footnote{\textbf{1:20} Ps 110,1} \bibverse{21} über
alle Fürstentümer, Gewalt, Macht, Herrschaft und alles, was genannt mag
werden, nicht allein auf dieser Welt, sondern auch in der zukünftigen;
\footnote{\textbf{1:21} Phil 2,9; Kol 2,10; Röm 8,38-39} \bibverse{22}
und hat alle Dinge unter seine Füße getan und hat ihn gesetzt zum Haupt
der Gemeinde über alles, \footnote{\textbf{1:22} Mt 28,18; Eph 4,15}
\bibverse{23} welche da ist sein Leib, nämlich die Fülle des, der alles
in allen erfüllt. \footnote{\textbf{1:23} Eph 5,30; 1Kor 12,27; Kol 1,19}

\hypertarget{section-1}{%
\section{2}\label{section-1}}

\bibverse{1} Und auch euch, da ihr tot waret durch Übertretungen und
Sünden, \footnote{\textbf{2:1} Kol 2,13; Lk 15,24; Lk 15,32}
\bibverse{2} in welchen ihr vordem gewandelt habt nach dem Lauf dieser
Welt und nach dem Fürsten, der in der Luft herrscht, nämlich nach dem
Geist, der zu dieser Zeit sein Werk hat in den Kindern des Unglaubens,
\footnote{\textbf{2:2} Tit 3,3; Eph 6,12; Joh 12,31} \bibverse{3} unter
welchen auch wir alle vordem unseren Wandel gehabt haben in den Lüsten
unseres Fleisches und taten den Willen des Fleisches und der Vernunft
und waren auch Kinder des Zorns von Natur, gleichwie auch die anderen;
\footnote{\textbf{2:3} Kol 3,6; 1Petr 4,3} \bibverse{4} Aber Gott, der
da reich ist an Barmherzigkeit, durch seine große Liebe, damit er uns
geliebt hat, \bibverse{5} da wir tot waren in den Sünden, hat er uns
samt Christo lebendig gemacht (denn aus Gnade seid ihr selig geworden)
\bibverse{6} und hat uns samt ihm auferweckt und samt ihm in das
himmlische Wesen gesetzt in Christo Jesu, \bibverse{7} auf dass er
erzeigte in den zukünftigen Zeiten den überschwenglichen Reichtum seiner
Gnade durch seine Güte gegen uns in Christo Jesu. \footnote{\textbf{2:7}
  Eph 1,7} \bibverse{8} Denn aus Gnade seid ihr selig geworden durch den
Glauben, und das nicht aus euch: Gottes Gabe ist es, \footnote{\textbf{2:8}
  Gal 2,16; Röm 3,23-24} \bibverse{9} nicht aus den Werken, auf dass
sich nicht jemand rühme. \footnote{\textbf{2:9} Röm 3,28; 1Kor 1,19}
\bibverse{10} Denn wir sind sein Werk, geschaffen in Christo Jesu zu
guten Werken, zu welchen Gott uns zuvor bereitet hat, dass wir darin
wandeln sollen. \footnote{\textbf{2:10} Tit 2,14}

\bibverse{11} Darum gedenket daran, dass ihr, die ihr vordem nach dem
Fleisch Heiden gewesen seid und die Unbeschnittenen genannt wurdet von
denen, die genannt sind die Beschneidung nach dem Fleisch, die mit der
Hand geschieht, \footnote{\textbf{2:11} Eph 5,8} \bibverse{12} dass ihr
zu derselben Zeit waret ohne Christum, fremd und außer der Bürgerschaft
Israels und fremd den Testamenten der Verheißung; daher ihr keine
Hoffnung hattet und waret ohne Gott in der Welt. \footnote{\textbf{2:12}
  Röm 9,4; 1Thes 4,13} \bibverse{13} Nun aber seid ihr, die ihr in
Christo Jesu seid und vordem ferne gewesen, nahe geworden durch das Blut
Christi. \bibverse{14} Denn er ist unser Friede, der aus beiden eines
hat gemacht und hat abgebrochen den Zaun, der dazwischen war, indem er
durch sein Fleisch wegnahm die Feindschaft, \footnote{\textbf{2:14} Jes
  9,5; Gal 3,28} \bibverse{15} nämlich das Gesetz, das in Geboten
gestellt war, auf dass er aus zweien einen neuen Menschen in ihm selber
schüfe und Frieden machte, \footnote{\textbf{2:15} Kol 2,14}
\bibverse{16} und dass er beide versöhnte mit Gott in einem Leibe durch
das Kreuz und hat die Feindschaft getötet durch sich selbst.
\bibverse{17} Und er ist gekommen, hat verkündigt im Evangelium den
Frieden euch, die ihr ferne waret, und denen, die nahe waren;
\footnote{\textbf{2:17} Jes 57,19} \bibverse{18} denn durch ihn haben
wir den Zugang alle beide in einem Geiste zum Vater. \footnote{\textbf{2:18}
  Eph 3,12} \bibverse{19} So seid ihr nun nicht mehr Gäste und
Fremdlinge, sondern Bürger mit den Heiligen und Gottes Hausgenossen,
\footnote{\textbf{2:19} Eph 3,6; Hebr 12,22-23} \bibverse{20} erbaut auf
den Grund der Apostel und Propheten, da Jesus Christus der Eckstein ist,
\footnote{\textbf{2:20} Mt 16,18; Jes 28,16; 1Petr 2,4-6} \bibverse{21}
auf welchem der ganze Bau ineinandergefügt wächst zu einem heiligen
Tempel in dem Herrn, \bibverse{22} auf welchem auch ihr mit erbaut
werdet zu einer Behausung Gottes im Geist. \# 3 \bibverse{1} Derhalben
ich, Paulus, der Gefangene Christi Jesu für euch Heiden, \footnote{\textbf{3:1}
  Phil 1,7; Phil 1,13} \bibverse{2} wie ihr ja gehört habt von dem Amt
der Gnade Gottes, die mir an euch gegeben ist, \footnote{\textbf{3:2}
  Gal 2,7} \bibverse{3} dass mir ist kund geworden dieses Geheimnis
durch Offenbarung, wie ich droben aufs kürzeste geschrieben habe,
\footnote{\textbf{3:3} Eph 1,9-10; Gal 1,12} \bibverse{4} daran ihr, so
ihr's leset, merken könnt mein Verständnis des Geheimnisses Christi,
\bibverse{5} welches nicht kundgetan ist in den vorigen Zeiten den
Menschenkindern, wie es nun offenbart ist seinen heiligen Aposteln und
Propheten durch den Geist, \bibverse{6} nämlich, dass die Heiden
Miterben seien und mit eingeleibt und Mitgenossen seiner Verheißung in
Christo durch das Evangelium, \footnote{\textbf{3:6} Eph 2,13; Eph
  2,18-19; Apg 15,7-9} \bibverse{7} dessen Diener ich geworden bin nach
der Gabe aus der Gnade Gottes, die mir nach seiner mächtigen Kraft
gegeben ist; \bibverse{8} mir, dem allergeringsten unter allen Heiligen,
ist gegeben diese Gnade, unter den Heiden zu verkündigen den
unausforschlichen Reichtum Christi \bibverse{9} und zu erleuchten
jedermann, welche da sei die Gemeinschaft des Geheimnisses, das von der
Welt her in Gott verborgen gewesen ist, der alle Dinge geschaffen hat
durch Jesum Christum, \footnote{\textbf{3:9} Eph 1,9-10; Röm 16,25-26;
  Kol 1,16} \bibverse{10} auf dass jetzt kund würde den Fürstentümern
und Herrschaften in dem Himmel an der Gemeinde die mannigfaltige
Weisheit Gottes, \footnote{\textbf{3:10} 1Petr 1,12} \bibverse{11} nach
dem Vorsatz von der Welt her, welche er bewiesen hat in Christo Jesu,
unserem Herrn, \bibverse{12} durch welchen wir haben Freudigkeit und
Zugang in aller Zuversicht durch den Glauben an ihn. \footnote{\textbf{3:12}
  Röm 5,2} \bibverse{13} Darum bitte ich, dass ihr nicht müde werdet um
meiner Trübsal willen, die ich für euch leide, welche euch eine Ehre
sind. \footnote{\textbf{3:13} Kol 1,24}

\bibverse{14} Derhalben beuge ich meine Knie vor dem Vater unseres Herrn
Jesu Christi, \bibverse{15} der der rechte Vater ist über alles, was da
Kinder heißt im Himmel und auf Erden, \bibverse{16} dass er euch Kraft
gebe nach dem Reichtum seiner Herrlichkeit, stark zu werden durch seinen
Geist an dem inwendigen Menschen, \footnote{\textbf{3:16} Eph 1,7; Eph
  6,10; 2Kor 4,16} \bibverse{17} dass Christus wohne durch den Glauben
in euren Herzen und ihr durch die Liebe eingewurzelt und gegründet
werdet, \footnote{\textbf{3:17} Joh 14,23; Kol 2,7} \bibverse{18} auf
dass ihr begreifen möget mit allen Heiligen, welches da sei die Breite
und die Länge und die Tiefe und die Höhe; \bibverse{19} auch erkennen
die Liebe Christi, die doch alle Erkenntnis übertrifft, auf dass ihr
erfüllt werdet mit allerlei Gottesfülle. \footnote{\textbf{3:19} Kol
  2,2-3}

\bibverse{20} Dem aber, der überschwenglich tun kann über alles, das wir
bitten oder verstehen, nach der Kraft, die da in uns wirkt,
\bibverse{21} dem sei Ehre in der Gemeinde, die in Christo Jesu ist, zu
aller Zeit, von Ewigkeit zu Ewigkeit! Amen. \# 4 \bibverse{1} So ermahne
nun euch ich Gefangener in dem Herrn, dass ihr wandelt, wie sich's
gebührt eurer Berufung, mit der ihr berufen seid, \bibverse{2} mit aller
Demut und Sanftmut, mit Geduld, und vertraget einer den anderen in der
Liebe \footnote{\textbf{4:2} Kol 3,12-13} \bibverse{3} und seid fleißig,
zu halten die Einigkeit im Geist durch das Band des Friedens:
\footnote{\textbf{4:3} Phil 2,2; Kol 3,15} \bibverse{4} ein Leib und ein
Geist, wie ihr auch berufen seid auf einerlei Hoffnung eurer Berufung;
\footnote{\textbf{4:4} Röm 12,5} \bibverse{5} ein Herr, ein Glaube, eine
Taufe; \footnote{\textbf{4:5} 1Kor 8,6} \bibverse{6} ein Gott und Vater
unser aller, der da ist über euch allen und durch euch alle und in euch
allen. \footnote{\textbf{4:6} 1Kor 12,6} \bibverse{7} Einem jeglichen
aber unter uns ist gegeben die Gnade nach dem Maß der Gabe Christi.
\footnote{\textbf{4:7} Röm 12,3; Röm 12,6; 1Kor 12,11} \bibverse{8}
Darum heißt es: „Er ist aufgefahren in die Höhe und hat das Gefängnis
gefangengeführt und hat den Menschen Gaben gegeben.`` \footnote{\textbf{4:8}
  Kol 2,15} \bibverse{9} dass er aber aufgefahren ist, was ist's, denn
dass er zuvor ist hinuntergefahren in die untersten Örter der Erde?
\footnote{\textbf{4:9} 1Petr 3,18-22} \bibverse{10} Der hinuntergefahren
ist, das ist derselbe, der aufgefahren ist über alle Himmel, auf dass er
alles erfüllte.

\bibverse{11} Und er hat etliche zu Aposteln gesetzt, etliche aber zu
Propheten, etliche zu Evangelisten, etliche zu Hirten und Lehrern,
\footnote{\textbf{4:11} 1Kor 12,28; Apg 21,8} \bibverse{12} dass die
Heiligen zugerichtet werden zum Werk des Dienstes, dadurch der Leib
Christi erbaut werde, \footnote{\textbf{4:12} 1Kor 14,26; 1Petr 2,5}
\bibverse{13} bis dass wir alle hinankommen zu einerlei Glauben und
Erkenntnis des Sohnes Gottes und ein vollkommener Mann werden, der da
sei im Maße des vollkommenen Alters Christi, \bibverse{14} auf dass wir
nicht mehr Kinder seien und uns bewegen und wiegen lassen von allerlei
Wind der Lehre durch Schalkheit der Menschen und Täuscherei, womit sie
uns erschleichen, uns zu verführen. \footnote{\textbf{4:14} 1Kor 14,20;
  Hebr 13,9} \bibverse{15} Lasset uns aber rechtschaffen sein in der
Liebe und wachsen in allen Stücken an dem, der das Haupt ist, Christus,
\footnote{\textbf{4:15} Eph 1,22; Eph 5,23; Kol 1,18} \bibverse{16} von
welchem aus der ganze Leib zusammengefügt ist und ein Glied am anderen
hanget durch alle Gelenke, dadurch eins dem anderen Handreichung tut
nach dem Werk eines jeglichen Gliedes in seinem Maße und macht, dass der
Leib wächst zu seiner selbst Besserung, und das alles in Liebe.
\footnote{\textbf{4:16} Kol 2,19}

\bibverse{17} So sage ich nun und bezeuge in dem Herrn, dass ihr nicht
mehr wandelt, wie die anderen Heiden wandeln in der Eitelkeit ihres
Sinnes, \footnote{\textbf{4:17} Röm 1,21-24} \bibverse{18} deren
Verstand verfinstert ist, und die entfremdet sind von dem Leben, das aus
Gott ist, durch die Unwissenheit, die in ihnen ist, durch die Blindheit
ihres Herzens; \footnote{\textbf{4:18} Eph 2,12} \bibverse{19} welche
ruchlos sind und ergeben sich der Unzucht und treiben allerlei
Unreinigkeit samt dem Geiz. \bibverse{20} Ihr aber habt Christum nicht
also gelernt, \bibverse{21} so ihr anders von ihm gehört habt und in ihm
gelehrt, wie in Jesu ein rechtschaffenes Wesen ist. \bibverse{22} So
legt nun von euch ab nach dem vorigen Wandel den alten Menschen, der
durch Lüste im Irrtum sich verderbt. \bibverse{23} Erneuert euch aber im
Geist eures Gemüts \footnote{\textbf{4:23} Röm 12,2} \bibverse{24} und
ziehet den neuen Menschen an, der nach Gott geschaffen ist in
rechtschaffener Gerechtigkeit und Heiligkeit. \footnote{\textbf{4:24}
  1Mo 1,26}

\bibverse{25} Darum leget die Lüge ab und redet die Wahrheit, ein
jeglicher mit seinem Nächsten, sintemal wir untereinander Glieder sind.
\footnote{\textbf{4:25} Sach 8,16} \bibverse{26} Zürnet, und sündiget
nicht; lasset die Sonne nicht über eurem Zorn untergehen. \footnote{\textbf{4:26}
  Ps 4,5; Jak 1,19; Jak 1,1-20} \bibverse{27} Gebet auch nicht Raum dem
Lästerer. \bibverse{28} Wer gestohlen hat der stehle nicht mehr, sondern
arbeite und schaffe mit den Händen etwas Gutes, auf dass er habe, zu
geben dem Dürftigen. \footnote{\textbf{4:28} 1Thes 4,11} \bibverse{29}
Lasset kein faul Geschwätz aus eurem Munde gehen, sondern was nützlich
zur Besserung ist, wo es not tut, dass es holdselig sei zu hören.
\footnote{\textbf{4:29} Eph 5,4; Kol 4,6} \bibverse{30} Und betrübet
nicht den heiligen Geist Gottes, mit dem ihr versiegelt seid auf den Tag
der Erlösung. \footnote{\textbf{4:30} Eph 1,13-14; Jes 63,10}
\bibverse{31} Alle Bitterkeit und Grimm und Zorn und Geschrei und
Lästerung sei ferne von euch samt aller Bosheit. \footnote{\textbf{4:31}
  Kol 3,8} \bibverse{32} Seid aber untereinander freundlich, herzlich
und vergebet einer dem anderen, gleichwie Gott euch auch vergeben hat in
Christo. \footnote{\textbf{4:32} Mt 6,14; Mt 18,22-35; Kol 3,13}

\hypertarget{section-2}{%
\section{5}\label{section-2}}

\bibverse{1} So seid nun Gottes Nachfolger als die lieben Kinder
\footnote{\textbf{5:1} Lk 6,36} \bibverse{2} und wandelt in der Liebe,
gleichwie Christus uns hat geliebt und sich selbst dargegeben für uns
als Gabe und Opfer, Gott zu einem süßen Geruch. \footnote{\textbf{5:2}
  2Mo 29,18; Gal 2,20}

\bibverse{3} Hurerei aber und alle Unreinigkeit oder Geiz lasset nicht
von euch gesagt werden, wie den Heiligen zusteht, \footnote{\textbf{5:3}
  Kol 3,5} \bibverse{4} auch nicht schandbare Worte und Narrenteidinge
oder Scherze, welche euch nicht ziemen, sondern vielmehr Danksagung.
\footnote{\textbf{5:4} Eph 4,29}

\bibverse{5} Denn das sollt ihr wissen, dass kein Hurer oder Unreiner
oder Geiziger, welcher ist ein Götzendiener, Erbe hat in dem Reich
Christi und Gottes. \footnote{\textbf{5:5} 1Kor 6,9-10; Offb 21,8; Offb
  22,15}

\bibverse{6} Lasset euch niemand verführen mit vergeblichen Worten; denn
um dieser Dinge willen kommt der Zorn Gottes über die Kinder des
Unglaubens. \footnote{\textbf{5:6} Kol 3,6} \bibverse{7} Darum seid
nicht ihr Mitgenossen. \bibverse{8} Denn ihr waret vordem Finsternis;
nun aber seid ihr ein Licht in dem Herrn. \bibverse{9} Wandelt wie die
Kinder des Lichts, die Frucht des Geistes ist allerlei Gütigkeit und
Gerechtigkeit und Wahrheit, \footnote{\textbf{5:9} Mt 5,14; Joh 12,36;
  Gal 5,22; Phil 1,11; 1Thes 5,5} \bibverse{10} und prüfet, was da sei
wohlgefällig dem Herrn. \footnote{\textbf{5:10} Röm 12,1; Phil 1,10}
\bibverse{11} und habt nicht Gemeinschaft mit den unfruchtbaren Werken
der Finsternis, strafet sie aber vielmehr. \bibverse{12} Denn was
heimlich von ihnen geschieht, das ist auch zu sagen schändlich.
\footnote{\textbf{5:12} Röm 1,24} \bibverse{13} Das alles aber wird
offenbar, wenn's vom Licht gestraft wird; denn alles, was offenbar ist,
das ist Licht. \footnote{\textbf{5:13} Joh 3,10; Joh 3,21} \bibverse{14}
Darum heißt es: „Wache auf, der du schläfst, und stehe auf von den
Toten, so wird dich Christus erleuchten.`` \footnote{\textbf{5:14} Jes
  60,1; Röm 13,11; Joh 8,12}

\bibverse{15} So sehet nun zu, wie ihr vorsichtig wandelt, nicht als die
Unweisen, sondern als die Weisen, \footnote{\textbf{5:15} Mt 10,16; Kol
  4,5} \bibverse{16} und kaufet die Zeit aus; denn es ist böse Zeit.
\bibverse{17} Darum werdet nicht unverständig, sondern verständig, was
da sei des Herrn Wille. \bibverse{18} Und saufet euch nicht voll Wein,
daraus ein unordentlich Wesen folgt, sondern werdet voll Geistes:
\footnote{\textbf{5:18} Lk 21,34} \bibverse{19} redet untereinander in
Psalmen und Lobgesängen und geistlichen Liedern, singet und spielet dem
Herrn in euren Herzen \footnote{\textbf{5:19} Ps 33,2-3; Kol 3,16}
\bibverse{20} und saget Dank allezeit für alles Gott und dem Vater in
dem Namen unseres Herrn Jesu Christi, \footnote{\textbf{5:20} 1Thes 5,18}
\bibverse{21} und seid untereinander untertan in der Furcht Gottes.
\footnote{\textbf{5:21} 1Petr 5,5}

\bibverse{22} Die Weiber seien untertan ihren Männern als dem Herrn.
\footnote{\textbf{5:22} 1Tim 2,11; Tit 2,5; 1Mo 3,16} \bibverse{23} Denn
der Mann ist des Weibes Haupt, gleichwie auch Christus das Haupt ist der
Gemeinde, und er ist seines Leibes Heiland. \footnote{\textbf{5:23} 1Kor
  11,3; Eph 1,22-23} \bibverse{24} Aber wie nun die Gemeinde ist Christo
untertan, also auch die Weiber ihren Männern in allen Dingen.

\bibverse{25} Ihr Männer, liebet eure Weiber, gleichwie Christus auch
geliebt hat die Gemeinde und hat sich selbst für sie gegeben,
\bibverse{26} auf dass er sie heiligte, und hat sie gereinigt durch das
Wasserbad im Wort, \footnote{\textbf{5:26} Tit 3,5; Hebr 10,22}
\bibverse{27} auf dass er sie sich selbst darstellte als eine Gemeinde,
die herrlich sei, die nicht habe einen Flecken oder Runzel oder des
etwas, sondern dass sie heilig sei und unsträflich. \footnote{\textbf{5:27}
  2Kor 11,2; Kol 1,22} \bibverse{28} Also sollen auch die Männer ihre
Weiber lieben wie ihre eigenen Leiber. Wer sein Weib liebt, der liebt
sich selbst. \bibverse{29} Denn niemand hat jemals sein eigen Fleisch
gehasst; sondern er nährt es und pflegt sein, gleichwie auch der Herr
die Gemeinde. \bibverse{30} Denn wir sind die Glieder seines Leibes, von
seinem Fleisch und von seinem Gebein. \footnote{\textbf{5:30} Eph 1,23}
\bibverse{31} „Um deswillen wird ein Mensch verlassen Vater und Mutter
und seinem Weibe anhangen, und werden die zwei ein Fleisch sein.``
\bibverse{32} Das Geheimnis ist groß; ich sage aber von Christo und der
Gemeinde. \bibverse{33} Doch auch ihr, ja ein jeglicher habe lieb sein
Weib als sich selbst; das Weib aber fürchte den Mann. \# 6 \bibverse{1}
Ihr Kinder, seid gehorsam euren Eltern in dem Herrn, denn das ist
billig. \bibverse{2} „Ehre Vater und Mutter,`` das ist das erste Gebot,
das Verheißung hat: \footnote{\textbf{6:2} 2Mo 20,12} \bibverse{3} „auf
dass dir's wohl gehe und du lange lebest auf Erden.``

\bibverse{4} Und ihr Väter, reizet eure Kinder nicht zum Zorn, sondern
zieht sie auf in der Zucht und Vermahnung zum Herrn.

\bibverse{5} Ihr Knechte, seid gehorsam euren leiblichen Herren mit
Furcht und Zittern, in Einfalt eures Herzens, als Christo; \footnote{\textbf{6:5}
  1Tim 6,1-2; Tit 2,9; Tit 1,2-10; 1Petr 2,18} \bibverse{6} nicht mit
Dienst allein vor Augen, als den Menschen zu gefallen, sondern als die
Knechte Christi, dass ihr solchen Willen Gottes tut von Herzen, mit
gutem Willen. \bibverse{7} Lasset euch dünken, dass ihr dem Herrn dienet
und nicht den Menschen, \bibverse{8} und wisset: Was ein jeglicher Gutes
tun wird, das wird er von dem Herrn empfangen, er sei ein Knecht oder
ein Freier.

\bibverse{9} Und ihr Herren, tut auch dasselbe gegen sie und lasset das
Drohen; wisset, dass auch euer Herr im Himmel ist und ist bei ihm kein
Ansehen der Person. \footnote{\textbf{6:9} Pred 5,7}

\bibverse{10} Zuletzt, meine Brüder, seid stark in dem Herrn und in der
Macht seiner Stärke. \footnote{\textbf{6:10} Eph 3,16; 1Kor 16,13; 1Jo
  2,14; 2Tim 2,1} \bibverse{11} Ziehet an den Harnisch Gottes, dass ihr
bestehen könnet gegen die listigen Anläufe des Teufels. \bibverse{12}
Denn wir haben nicht mit Fleisch und Blut zu kämpfen, sondern mit
Fürsten und Gewaltigen, nämlich mit den Herren der Welt, die in der
Finsternis dieser Welt herrschen, mit den bösen Geistern unter dem
Himmel. \footnote{\textbf{6:12} Lk 22,31; Eph 2,2} \bibverse{13} Um
deswillen ergreifet den Harnisch Gottes, auf dass ihr an dem bösen Tage
Widerstand tun und alles wohl ausrichten und das Feld behalten möget.
\bibverse{14} So stehet nun, umgürtet an euren Lenden mit Wahrheit und
angezogen mit dem Panzer der Gerechtigkeit \bibverse{15} und an den
Beinen gestiefelt, als fertig, zu treiben das Evangelium des Friedens.
\bibverse{16} Vor allen Dingen aber ergreifet den Schild des Glaubens,
mit welchem ihr auslöschen könnt alle feurigen Pfeile des Bösewichtes;
\footnote{\textbf{6:16} 1Petr 5,9; 1Jo 5,4} \bibverse{17} und nehmet den
Helm des Heils und das Schwert des Geistes, welches ist das Wort Gottes.
\footnote{\textbf{6:17} 1Thes 5,8; Hebr 4,12} \bibverse{18} Und betet
stets in allem Anliegen mit Bitten und Flehen im Geist, und wachet dazu
mit allem Anhalten und Flehen für alle Heiligen \bibverse{19} und für
mich, auf dass mir gegeben werde das Wort mit freudigem Auftun meines
Mundes, dass ich möge kundmachen das Geheimnis des Evangeliums,
\footnote{\textbf{6:19} Kol 4,3; 2Thes 3,1; Apg 4,29} \bibverse{20}
dessen Bote ich bin in der Kette, auf dass ich darin freudig handeln
möge und reden, wie sich's gebührt. \footnote{\textbf{6:20} Apg 28,31;
  2Kor 5,20}

\bibverse{21} Auf dass aber ihr auch wisset, wie es um mich steht und
was ich schaffe, wird's euch alles kundtun Tychikus, mein lieber Bruder
und getreuer Diener in dem Herrn, \footnote{\textbf{6:21} Apg 20,4; 2Tim
  4,12; Tit 3,12; Kol 4,7-8} \bibverse{22} welchen ich gesandt habe zu
euch um deswillen, dass ihr erfahret, wie es um mich steht, und dass er
eure Herzen tröste.

\bibverse{23} Friede sei den Brüdern und Liebe mit Glauben von Gott, dem
Vater, und dem Herrn Jesus Christus! \bibverse{24} Gnade sei mit euch
allen, die da liebhaben unseren Herrn Jesus Christus unverrückt! Amen.
