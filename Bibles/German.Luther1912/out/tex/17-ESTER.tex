\hypertarget{section}{%
\section{1}\label{section}}

\bibverse{1} Zu den Zeiten Ahasveros (der da König war von Indien bis an
Mohrenland über hundert und siebenundzwanzig Länder) \bibverse{2} und da
er auf seinem königlichen Stuhl saß zu Schloß Susan, \bibverse{3} im
dritten Jahr seines Königreichs, machte er bei sich ein Mahl allen
seinen Fürsten und Knechten, den Gewaltigen in Persien und Medien, den
Landpflegern und Obersten in seinen Ländern, \bibverse{4} daß er sehen
ließe den herrlichen Reichtum seines Königreichs und die köstliche
Pracht seiner Majestät viele Tage lang, hundert und achtzig Tage.
\bibverse{5} Und da die Tage aus waren, machte der König ein Mahl allem
Volk, das zu Schloß Susan war, Großen und Kleinen, sieben Tage lang im
Hofe des Gartens am Hause des Königs. \bibverse{6} Da hingen weiße, rote
und blaue Tücher, mit leinenen und scharlachnen Seilen gefaßt, in
silbernen Ringen auf Marmorsäulen. Die Bänke waren golden und silbern
auf Pflaster von grünem, weißem, gelbem und schwarzen Marmor.
\bibverse{7} Und das Getränk trug man in goldenen Gefäßen und immer
andern und andern Gefäßen, und königlichen Wein die Menge, wie denn der
König vermochte. \bibverse{8} Und man setzte niemand, was er trinken
sollte; denn der König hatte allen Vorstehern befohlen, daß ein
jeglicher sollte tun, wie es ihm wohl gefiel. \bibverse{9} Und die
Königin Vasthi machte auch ein Mahl für die Weiber im königlichen Hause
des Königs Ahasveros. \bibverse{10} Und am siebenten Tage, da der König
gutes Muts war vom Wein, hieß er Mehuman, Bistha, Harbona, Bigtha,
Abagtha, Sethar und Charkas, die sieben Kämmerer, die vor dem König
Ahasveros dienten, \bibverse{11} daß sie die Königin Vasthi holten vor
den König mit der königlichen Krone, daß er den Völkern und Fürsten
zeigte ihre Schöne; denn sie war schön. \bibverse{12} Aber die Königin
Vasthi wollte nicht kommen nach dem Wort des Königs durch seine
Kämmerer. Da ward der König sehr zornig, und sein Grimm entbrannte in
ihm. \bibverse{13} Und der König sprach zu den Weisen, die sich auf die
Zeiten verstanden (denn des Königs Sachen mußten geschehen vor allen,
die sich auf Recht und Händel verstanden; \bibverse{14} die nächsten
aber die bei ihm waren Charsena, Sethar, Admatha, Tharsis, Meres,
Marsena und Memuchan, die sieben Fürsten der Perser und Meder, die das
Angesicht des Königs sahen und saßen obenan im Königreich),
\bibverse{15} was für ein Recht man an der Königin Vasthi tun sollte,
darum daß sie nicht getan hatte nach dem Wort des Königs durch seine
Kämmerer. \bibverse{16} Da sprach Memuchan vor dem König und den
Fürsten: Die Königin Vasthi hat nicht allein an dem König übel getan,
sondern auch an allen Fürsten und an allen Völkern in allen Landen des
Königs Ahasveros. \bibverse{17} Denn es wird solche Tat der Königin
auskommen zu allen Weibern, daß sie ihre Männer verachten vor ihren
Augen und werden sagen: Der König Ahasveros hieß die Königin Vasthi vor
sich kommen; aber sie wollte nicht. \bibverse{18} So werden nun die
Fürstinnen in Persien und Medien auch so sagen zu allen Fürsten des
Königs, wenn sie solche Tat der Königin hören; so wird sich Verachtens
und Zorn genug erheben. \bibverse{19} Gefällt es dem König, so lasse man
ein königlich Gebot von ihm ausgehen und schreiben nach der Perser und
Meder Gesetz, welches man nicht darf übertreten: daß Vasthi nicht mehr
vor den König Ahasveros komme, und der König gebe ihre königliche Würde
einer andern, die besser ist denn sie. \bibverse{20} Und es erschalle
dieser Befehl des Königs, den er geben wird, in sein ganzes Reich,
welches groß ist, daß alle Weiber ihre Männer in Ehren halten, unter
Großen und Kleinen. \bibverse{21} Das gefiel dem König und den Fürsten;
und der König tat nach dem Wort Memuchans. \bibverse{22} Da wurden
Briefe ausgesandt in alle Länder des Königs, in ein jegliches Land nach
seiner Schrift und zu jeglichem Volk nach seiner Sprache: daß ein
jeglicher Mann der Oberherr in seinem Hause sei und ließe reden nach der
Sprache seines Volkes.

\hypertarget{section-1}{%
\section{2}\label{section-1}}

\bibverse{1} Nach diesen Geschichten, da der Grimm des Königs Ahasveros
sich gelegt hatte, gedachte er an Vasthi, was sie getan hatte und was
über sie beschlossen war. \bibverse{2} Da sprachen die Diener des
Königs, die ihm dienten: Man suche dem König junge, schöne Jungfrauen,
\bibverse{3} und der König bestellte Männer in allen Landen seines
Königreichs, daß sie allerlei junge, schöne Jungfrauen zusammenbringen
gen Schloß Susan ins Frauenhaus unter der Hand Hegais, des Königs
Kämmerers, der der Weiber wartet, und man gebe ihnen ihren Schmuck;
\bibverse{4} und welche Dirne dem König gefällt, die werde Königin an
Vasthis Statt. Das gefiel dem König, und er tat also. \bibverse{5} Es
war aber ein jüdischer Mann zu Schloß Susan, der hieß Mardochai, ein
Sohn Jairs, des Sohnes Simeis, des Sohnes des Kis, ein Benjaminiter,
\bibverse{6} der mit weggeführt war von Jerusalem, da Jechonja, der
König Juda's, weggeführt ward, welchen Nebukadnezar, der König zu Babel,
wegführte. \bibverse{7} Und er war ein Vormund der Hadassa, das ist
Esther, eine Tochter seines Oheims; denn sie hatte weder Vater noch
Mutter. Und sie war eine schöne und feine Dirne. Und da ihr Vater und
ihre Mutter starb, nahm sie Mardochai auf zur Tochter. \bibverse{8} Da
nun das Gebot und Gesetz des Königs laut ward und viel Dirnen zuhaufe
gebracht wurden gen Schloß Susan unter die Hand Hegais, ward Esther auch
genommen zu des Königs Hause unter die Hand Hegais, des Hüters der
Weiber. \bibverse{9} Und die Dirne gefiel ihm, und sie fand
Barmherzigkeit vor ihm. Und er eilte mit ihrem Schmuck, daß er ihr ihren
Teil gäbe und sieben feine Dirnen von des Königs Hause dazu. Und er tat
sie mit ihren Dirnen an den besten Ort im Frauenhaus. \bibverse{10} Und
Esther sagte ihm nicht an ihr Volk und ihre Freundschaft; denn Mardochai
hatte ihr geboten, sie sollte es nicht ansagen. \bibverse{11} Und
Mardochai wandelte alle Tage vor dem Hofe am Frauenhaus, daß er erführe,
ob's Esther wohl ginge und was ihr geschehen würde. \bibverse{12} Wenn
aber die bestimmte Zeit einer jeglichen Dirne kam, daß sie zum König
Ahasveros kommen sollte, nachdem sie zwölf Monate im Frauen-Schmücken
gewesen war (denn ihr Schmücken mußte soviel Zeit haben, nämlich sechs
Monate mit Balsam und Myrrhe und sechs Monate mit guter Spezerei, so
waren denn die Weiber geschmückt): \bibverse{13} alsdann ging die Dirne
zum König und alles, was sie wollte, mußte man ihr geben, daß sie damit
vom Frauenhaus zu des Königs Hause ginge. \bibverse{14} Und wenn eine
des Abends hineinkam, die ging des Morgens von ihm in das andere
Frauenhaus unter die Hand des Saasgas, des Königs Kämmerers, des Hüters
der Kebsweiber Und sie durfte nicht wieder zum König kommen, es lüstete
denn den König und er ließ sie mit Namen rufen. \bibverse{15} Da nun die
Zeit Esthers herankam, der Tochter Abihails, des Oheims Mardochais (die
er zur Tochter hatte aufgenommen), daß sie zum König kommen sollte,
begehrte sie nichts, denn was Hegai, des Königs Kämmerer, der Weiber
Hüter, sprach. Und Esther fand Gnade vor allen, die sie ansahen.
\bibverse{16} Es ward aber Esther genommen zum König Ahasveros ins
königliche Haus im zehnten Monat, der da heißt Tebeth, im siebenten Jahr
seines Königreichs. \bibverse{17} Und der König gewann Esther lieb über
alle Weiber, und sie fand Gnade und Barmherzigkeit vor ihm vor allen
Jungfrauen. Und er setzte die königliche Krone auf ihr Haupt und machte
sie zur Königin an Vasthis Statt. \bibverse{18} Und der König machte ein
großes Mahl allen seinen Fürsten und Knechten, das war ein Mahl um
Esthers willen, und ließ die Länder ruhen und gab königliche Geschenke
aus. \bibverse{19} Und da man das anderemal Jungfrauen versammelte, saß
Mardochai im Tor des Königs. \bibverse{20} Und Esther hatte noch nicht
angesagt ihre Freundschaft noch ihr Volk, wie ihr Mardochai geboten
hatte; denn Esther tat nach den Worten Mardochais, gleich als da er ihr
Vormund war. \bibverse{21} Zur selben Zeit, da Mardochai im Tor des
Königs saß, wurden zwei Kämmerer des Königs, Bigthan und Theres, die die
Tür hüteten, zornig und trachteten ihre Hände an den König Ahasveros zu
legen. \bibverse{22} Das ward Mardochai kund, und er sagte es der
Königin Esther, und Esther sagte es dem König in Mardochais Namen.
\bibverse{23} Und da man nachforschte, ward's gefunden, und sie wurden
beide an Bäume gehängt. Und es ward geschrieben in die Chronik vor dem
König.

\hypertarget{section-2}{%
\section{3}\label{section-2}}

\bibverse{1} Nach diesen Geschichten machte der König Ahasveros Haman
groß, den Sohn Hammedathas, den Agagiter, und erhöhte ihn und setzte
seinen Stuhl über alle Fürsten, die bei ihm waren. \bibverse{2} Und alle
Knechte des Königs, die im Tor waren, beugten die Kniee und fielen vor
Haman nieder; denn der König hatte es also geboten. Aber Mardochai
beugte die Kniee nicht und fiel nicht nieder. \bibverse{3} Da sprachen
des Königs Knechte, die im Tor des Königs waren, zu Mardochai: Warum
übertrittst du des Königs Gebot? \bibverse{4} Und da sie solches täglich
zu ihm sagten und er ihnen nicht gehorchte, sagten sie es Haman an, daß
sie sähen, ob solch Tun Mardochais bestehen würde; denn er hatte ihnen
gesagt, daß er ein Jude wäre. \bibverse{5} Und da nun Haman sah, daß
Mardochai ihm nicht die Kniee beugte noch vor ihm niederfiel, ward er
voll Grimms. \bibverse{6} Und verachtete es, daß er an Mardochai allein
sollte die Hand legen, denn sie hatten ihm das Volk Mardochais angesagt;
sondern er trachtete, das Volk Mardochais, alle Juden, so im ganzen
Königreich des Ahasveros waren, zu vertilgen. \bibverse{7} Im ersten
Monat, das ist der Monat Nisan, im zwölften Jahr des Königs Ahasveros,
ward das Pur, das ist das Los, geworfen vor Haman, von einem Tage auf
den andern und von Monat zu Monat bis auf den zwölften, das ist der
Monat Adar. \bibverse{8} Und Haman sprach zum König Ahasveros: Es ist
ein Volk, zerstreut in allen Ländern deines Königreichs, und ihr Gesetz
ist anders denn aller Völker, und tun nicht nach des Königs Gesetzen; es
ziemt dem König nicht, sie also zu lassen. \bibverse{9} Gefällt es dem
König, so lasse er schreiben, daß man sie umbringe; so will ich
zehntausend Zentner Silber darwägen unter die Hand der Amtleute, daß
man's bringt in die Kammer des Königs. \bibverse{10} Da tat der König
seinen Ring von der Hand und gab ihn Haman, dem Sohn Hammadathas, dem
Agagiter, der Juden Feind. \bibverse{11} Und der König sprach zu Haman:
Das Silber sei dir gegeben, dazu das Volk, daß du damit tust, was dir
gefällt. \bibverse{12} Da rief man die Schreiber des Königs am
dreizehnten Tage des ersten Monats; und ward geschrieben, wie Haman
befahl, an die Fürsten des Königs und zu den Landpflegern hin und her in
den Ländern und zu den Hauptleuten eines jeglichen Volks in den Ländern
hin und her, nach der Schrift eines jeglichen Volks und nach ihrer
Sprache, im Namen des Königs Ahasveros und mit des Königs Ring
versiegelt. \bibverse{13} Und die Briefe wurden gesandt durch die Läufer
in alle Länder des Königs, zu vertilgen, zu erwürgen und umzubringen
alle Juden, jung und alt, Kinder und Weiber, auf einen Tag, nämlich auf
den dreizehnten Tag des zwölften Monats, das ist der Monat Adar, und ihr
Gut zu rauben. \bibverse{14} Also war der Inhalt der Schrift: daß ein
Gebot gegeben wäre in allen Ländern, allen Völkern zu eröffnen, daß sie
auf denselben Tag bereit wären. \bibverse{15} Und die Läufer gingen aus
eilend nach des Königs Wort, und zu Schloß Susan ward das Gebot
angeschlagen. Und der König und Haman saßen und tranken; aber die Stadt
Susan ward bestürzt.

\hypertarget{section-3}{%
\section{4}\label{section-3}}

\bibverse{1} Da Mardochai erfuhr alles, was geschehen war, zerriß er
seine Kleider und legte einen Sack an und Asche und ging hinaus mitten
in die Stadt und schrie laut und kläglich. \bibverse{2} Und kam bis vor
das Tor des Königs; denn es durfte niemand zu des Königs Tor eingehen,
der einen Sack anhatte. \bibverse{3} Und in allen Ländern, an welchen
Ort des Königs Wort und Gebot gelangte, war ein großes Klagen unter den
Juden, und viele fasteten, weinten trugen Leid und lagen in Säcken und
in der Asche. \bibverse{4} Da kamen die Dirnen Esthers und ihre Kämmerer
und sagten's ihr an. Da erschrak die Königin sehr. Und sie sandte
Kleider, daß Mardochai sie anzöge und den Sack von sich legte; aber er
nahm sie nicht. \bibverse{5} Da rief Esther Hathach unter des Königs
Kämmerern, der vor ihr stand, und gab ihm Befehl an Mardochai, daß sie
erführe, was das wäre und warum er so täte. \bibverse{6} Da ging Hathach
hinaus zu Mardochai in die Gasse der Stadt, die vor dem Tor des Königs
war. \bibverse{7} Und Mardochai sagte ihm alles, was ihm begegnet wäre,
und die Summe des Silbers, das Haman versprochen hatte in des Königs
Kammer darzuwägen um der Juden willen, sie zu vertilgen, \bibverse{8}
und gab ihm die Abschrift des Gebots, das zu Susan angeschlagen war, sie
zu vertilgen, daß er's Esther zeigte und ihr ansagte und geböte ihr, daß
sie zum König hineinginge und flehte zu ihm und täte eine Bitte an ihn
um ihr Volk. \bibverse{9} Und da Hathach hineinkam und sagte Esther die
Worte Mardochais, \bibverse{10} sprach Esther zu Hathach und gebot ihm
an Mardochai: \bibverse{11} Es wissen alle Knechte des Königs und das
Volk in den Landen des Königs, daß, wer zum König hineingeht inwendig in
den Hof, er sei Mann oder Weib, der nicht gerufen ist, der soll stracks
nach dem Gebot sterben; es sei denn, daß der König das goldene Zepter
gegen ihn recke, damit er lebendig bleibe. Ich aber bin nun in dreißig
Tagen nicht gerufen, zum König hineinzukommen. \bibverse{12} Und da die
Worte Esthers wurden Mardochai angesagt, \bibverse{13} hieß Mardochai
Esther wieder sagen: Gedenke nicht, daß du dein Leben errettest, weil du
im Hause des Königs bist, vor allen Juden; \bibverse{14} denn wo du
wirst zu dieser Zeit schweigen, so wird eine Hilfe und Errettung von
einem andern Ort her den Juden entstehen, und du und deines Vaters Haus
werdet umkommen. Und wer weiß, ob du nicht um dieser Zeit willen zur
königlichen Würde gekommen bist? \bibverse{15} Esther hieß Mardochai
antworten: \bibverse{16} So gehe hin und versammle alle Juden, die zu
Susan vorhanden sind, und fastet für mich, daß ihr nicht esset und
trinket in drei Tagen, weder Tag noch Nacht; ich und meine Dirnen wollen
auch also fasten. Und ich will zum König hineingehen wider das Gebot;
komme ich um, so komme ich um. \bibverse{17} Mardochai ging hin und tat
alles, was ihm Esther geboten hatte.

\hypertarget{section-4}{%
\section{5}\label{section-4}}

\bibverse{1} Und am dritten Tage zog sich Esther königlich an und trat
in den inneren Hof am Hause des Königs gegenüber dem Hause des Königs.
Und der König saß auf seinem königlichen Stuhl im königlichen Hause,
gegenüber der Tür des Hauses. \bibverse{2} Und da der König sah Esther,
die Königin, stehen im Hofe, fand sie Gnade vor seinen Augen. Und der
König reckte das goldene Zepter in seiner Hand gegen Esther. Da trat
Esther herzu und rührte die Spitze des Zepters an. \bibverse{3} Da
sprach der König zu ihr: Was ist dir, Esther, Königin? und was forderst
du? Auch die Hälfte des Königreichs soll dir gegeben werden.
\bibverse{4} Esther sprach: Gefällt es dem König, so komme der König und
Haman heute zu dem Mahl, das ich zugerichtet habe. \bibverse{5} Der
König sprach: Eilet, daß Haman tue, was Esther gesagt hat! Da nun der
König und Haman zu dem Mahl kamen, das Esther zugerichtet hatte,
\bibverse{6} sprach der König zu Esther, da er Wein getrunken hatte: Was
bittest du, Esther? Es soll dir gegeben werden. Und was forderst du?
Auch die Hälfte des Königreichs, es soll geschehen. \bibverse{7} Da
antwortete Esther und sprach: Meine Bitte und Begehr ist: \bibverse{8}
Habe ich Gnade gefunden vor dem König, und so es dem König gefällt mir
zu geben eine Bitte und zu tun mein Begehren, so komme der König und
Haman zu dem Mahl, das ich für sie zurichten will; so will ich morgen
Tun, was der König gesagt hat. \bibverse{9} Da ging Haman des Tages
hinaus fröhlich und gutes Muts. Und da er sah Mardochai im Tor des
Königs, daß er nicht aufstand noch sich vor ihm bewegte, ward er voll
Zorns über Mardochai. \bibverse{10} Aber er hielt an sich. Und da er
heimkam, sandte er hin und ließ holen seine Freunde und sein Weib Seres
\bibverse{11} und zählte ihnen auf die Herrlichkeit seines Reichtums und
die Menge seiner Kinder und alles, wie ihn der König so groß gemacht
hätte und daß er über die Fürsten und Knechte des Königs erhoben wäre.
\bibverse{12} Auch sprach Haman: Und die Königin Esther hat niemand
kommen lassen mit dem König zum Mahl, das sie zugerichtet hat, als mich;
und ich bin morgen auch zu ihr geladen mit dem König. \bibverse{13} Aber
an dem allem habe ich keine Genüge, solange ich sehe den Juden Mardochai
am Königstor sitzen. \bibverse{14} Da sprachen zu ihm sein Weib Seres
und alle Freunde: Man mache einen Baum, fünfzig Ellen hoch, und morgen
sage dem König, daß man Mardochai daran hänge; so kommst du mit dem
König fröhlich zum Mahl. Das gefiel Haman wohl, und er ließ einen Baum
zurichten.

\hypertarget{section-5}{%
\section{6}\label{section-5}}

\bibverse{1} In derselben Nacht konnte der König nicht schlafen und hieß
die Chronik mit den Historien bringen. Da die wurden vor dem König
gelesen, \bibverse{2} fand sich's geschrieben, wie Mardochai hatte
angesagt, daß die zwei Kämmerer des Königs, Bigthan und Theres, die an
der Schwelle hüteten, getrachtet hätten, die Hand an den König Ahasveros
zu legen. \bibverse{3} Und der König sprach: Was haben wir Mardochai
Ehre und Gutes dafür getan? Da sprachen die Diener des Königs, die ihm
dienten: Es ist ihm nichts geschehen. \bibverse{4} Und der König sprach:
Wer ist im Hofe? Haman aber war in den Hof gegangen, draußen vor des
Königs Hause, daß er dem König sagte, Mardochai zu hängen an den Baum,
den er zubereitet hatte. \bibverse{5} Und des Königs Diener sprachen zu
ihm: Siehe, Haman steht im Hofe. Der König sprach: Laßt ihn hereingehen!
\bibverse{6} Und da Haman hineinkam, sprach der König zu ihm: Was soll
man dem Mann tun, den der König gerne wollte ehren? Haman aber gedachte
in seinem Herzen: Wem sollte der König anders gern wollen Ehre tun denn
mir? \bibverse{7} Und Haman sprach zum König: Dem Mann, den der König
gerne wollte ehren, \bibverse{8} soll man königliche Kleider bringen,
die der König pflegt zu tragen, und ein Roß, darauf der König reitet,
und soll eine königliche Krone auf sein Haupt setzen; \bibverse{9} und
man soll solch Kleid und Roß geben in die Hand eines Fürsten des Königs,
daß derselbe den Mann anziehe, den der König gern ehren wollte, und
führe ihn auf dem Roß in der Stadt Gassen und lasse rufen vor ihm her:
So wird man tun dem Mann, den der König gerne ehren will. \bibverse{10}
Der König sprach zu Haman: Eile und nimm das Kleid und Roß, wie du
gesagt hast, und tu also mit Mardochai, dem Juden, der vor dem Tor des
Königs sitzt; und laß nichts fehlen an allem, was du geredet hast!
\bibverse{11} Da nahm Haman das Kleid und Roß und zog Mardochai an und
führte ihn auf der Stadt Gassen und rief vor ihm her: So wird man tun
dem Mann, den der König gerne ehren will. \bibverse{12} Und Mardochai
kam wieder an das Tor des Königs. Haman aber eilte nach Hause, trug Leid
mit verhülltem Kopf \bibverse{13} und erzählte seinem Weibe Seres und
seinen Freunden allen alles, was ihm begegnet war. Da sprachen zu ihm
seine Weisen und sein Weib Seres: Ist Mardochai vom Geschlecht der
Juden, vor dem du zu fallen angehoben hast, so vermagst du nichts an
ihm, sondern du wirst vor ihm fallen. \bibverse{14} Da sie aber noch mit
ihm redeten, kamen herbei des Königs Kämmerer und trieben Haman, zum
Mahl zu kommen, das Esther zugerichtet hatte.

\hypertarget{section-6}{%
\section{7}\label{section-6}}

\bibverse{1} Und da der König mit Haman kam zum Mahl, das die Königin
Esther zugerichtet hatte, \bibverse{2} sprach der König zu Esther auch
des andern Tages, da er Wein getrunken hatte: Was bittest du, Königin
Esther, daß man's dir gebe? Und was forderst du? Auch das halbe
Königreich, es soll geschehen. \bibverse{3} Esther, die Königin,
antwortete und sprach: habe ich Gnade vor dir gefunden, o König, und
gefällt es dem König, so gib mir mein Leben um meiner Bitte willen und
mein Volk um meines Begehrens willen. \bibverse{4} Denn wir sind
verkauft, ich und mein Volk, daß wir vertilgt, erwürgt und umgebracht
werden. Und wären wir doch nur zu Knechten und Mägden verkauft, so
wollte ich schweigen; so würde der Feind doch dem König nicht schaden.
\bibverse{5} Der König Ahasveros redete und sprach zu der Königin
Esther: Wer ist der, oder wo ist der, der solches in seinen Sinn nehmen
dürfe, also zu tun? \bibverse{6} Esther sprach: Der Feind und
Widersacher ist dieser böse Haman. Haman entsetzte sich vor dem König
und der Königin. \bibverse{7} Und der König stand auf vom Mahl und vom
Wein in seinem Grimm und ging in den Garten am Hause. Und Haman stand
auf und bat die Königin Esther um sein Leben; denn er sah, daß ihm ein
Unglück vom König schon bereitet war. \bibverse{8} Und da der König
wieder aus dem Garten am Hause in den Saal, da man gegessen hatte, kam,
lag Haman an der Bank, darauf Esther saß. Da sprach der König: Will er
auch der Königin Gewalt tun bei mir im Hause? Da das Wort aus des Königs
Munde ging, verhüllten sie Haman das Antlitz. \bibverse{9} Und Harbona,
der Kämmerer einer vor dem König, sprach: Siehe, es steht ein Baum im
Hause Haman, fünfzig Ellen hoch, den er Mardochai gemacht hatte, der
Gutes für den König geredet hat. Der König sprach: Laßt ihn daran
hängen! \bibverse{10} Also hängte man Haman an den Baum, den er
Mardochai gemacht hatte. Da legte sich des Königs Zorn.

\hypertarget{section-7}{%
\section{8}\label{section-7}}

\bibverse{1} An dem Tage gab der König Ahasveros der Königin Esther das
Haus Hamans, des Judenfeindes. Und Mardochai kam vor den König; denn
Esther sagte an, wie er ihr zugehörte. \bibverse{2} Und der König tat ab
von seinem Fingerreif, den er von Haman hatte genommen, und gab ihn
Mardochai. Und Esther setzte Mardochai über das Haus Hamans.
\bibverse{3} Und Esther redete weiter vor dem König und fiel ihm zu den
Füßen und weinte und flehte ihn an, daß er zunichte machte die Bosheit
Hamans, des Agagiters, und seine Anschläge, die er wider die Juden
erdacht hatte. \bibverse{4} Und der König reckte das goldene Zepter
gegen Esther. Da stand Esther auf und trat vor den König \bibverse{5}
und sprach: Gefällt es dem König und habe ich Gnade gefunden vor ihm und
ist's gelegen dem König und ich gefalle ihm, so schreibe man, daß die
Briefe Hamans, des Sohnes Hammedathas, des Agagiters, widerrufen werden,
die er geschrieben hat, die Juden umzubringen in allen Landen des
Königs. \bibverse{6} Denn wie kann ich zusehen dem Übel, das mein Volk
treffen würde? Und wie kann ich zusehen, daß mein Geschlecht umkomme?
\bibverse{7} Da sprach der König Ahasveros zur Königin Esther und zu
Mardochai, dem Juden: Siehe, ich habe Esther das Haus Hamans gegeben,
und ihn hat man an einen Baum gehängt, darum daß er seine Hand hat an
die Juden gelegt; \bibverse{8} so schreibt ihr nun für die Juden, wie es
euch gefällt, in des Königs Namen und versiegelt's mit des Königs Ringe.
Denn die Schriften, die in des Königs Namen geschrieben und mit des
Königs Ring versiegelt wurden, durfte niemand widerrufen. \bibverse{9}
Da wurden berufen des Königs Schreiber zu der Zeit im dritten Monat, das
ist der Monat Sivan, am dreiundzwanzigsten Tage, und wurde geschrieben,
wie Mardochai gebot, an die Juden und an die Fürsten, Landpfleger und
Hauptleute in den Landen von Indien bis an das Mohrenland, nämlich
hundert und siebenundzwanzig Länder, einem jeglichen Lande nach seiner
Schrift, einem jeglichen Volk nach seiner Sprache, und den Juden nach
ihrer Schrift und Sprache. \bibverse{10} Und es war geschrieben in des
Königs Ahasveros Namen und mit des Königs Ring versiegelt. Und er sandte
die Briefe durch die reitenden Boten auf jungen Maultieren,
\bibverse{11} darin der König den Juden Macht gab, in welchen Städten
sie auch waren, sich zu versammeln und zu stehen für ihr Leben und zu
vertilgen, zu erwürgen und umzubringen alle Macht des Volkes und Landes,
die sie ängsteten, samt den Kindern und Weibern, und ihr Gut zu rauben
\bibverse{12} auf einen Tag in allen Ländern des Königs Ahasveros,
nämlich am dreizehnten Tage des zwölften Monats, das ist der Monat Adar.
\bibverse{13} Der Inhalt aber der Schrift war, daß ein Gebot gegeben
wäre in allen Landen, zu eröffnen allen Völkern, daß die Juden bereit
sein sollten, sich zu rächen an ihren Feinden. \bibverse{14} Und die
reitenden Boten auf den Maultieren ritten aus schnell und eilend nach
dem Wort des Königs, und das Gebot ward zu Schloß Susan angeschlagen.
\bibverse{15} Mardochai aber ging aus von dem König in königlichen
Kleidern, blau und weiß, und mit einer großen goldenen Krone, angetan
mit einem Leinen-und Purpur-mantel; und die Stadt Susan jauchzte und war
fröhlich. \bibverse{16} Den Juden aber war Licht und Freude und Wonne
und Ehre gekommen. \bibverse{17} Und in allen Landen und Städten, an
welchen Ort des Königs Wort und Gebot gelangte, da war Freude und Wonne
unter den Juden, Wohlleben und gute Tage, daß viele aus den Völkern im
Lande Juden wurden; denn die Furcht vor den Juden war über sie gekommen.

\hypertarget{section-8}{%
\section{9}\label{section-8}}

\bibverse{1} Im zwölften Monat, das ist der Monat Adar, am dreizehnten
Tag, den des Königs Wort und Gebot bestimmt hatte, daß man's tun sollte,
ebendesselben Tages, da die Feinde der Juden hofften, sie zu
überwältigen, wandte sich's, daß die Juden ihre Feinde überwältigen
sollten. \bibverse{2} Da versammelten sich die Juden in ihren Städten in
allen Landen des Königs Ahasveros, daß sie die Hand legten an die, so
ihnen übel wollten. Und niemand konnte ihnen widerstehen; denn ihre
Furcht war über alle Völker gekommen. \bibverse{3} Auch alle Obersten in
den Landen und Fürsten und Landpfleger und Amtleute des Königs halfen
den Juden; denn die Furcht vor Mardochai war über sie gekommen.
\bibverse{4} Denn Mardochai war groß im Hause des Königs, und sein
Gerücht erscholl in allen Ländern, wie er zunähme und groß würde.
\bibverse{5} Also schlugen die Juden an allen ihren Feinden eine
Schwertschlacht und würgten und raubten und brachten um und taten nach
ihrem Willen an denen, die ihnen feind waren. \bibverse{6} Und zu Schloß
Susan erwürgten die Juden und brachten um fünfhundert Mann; \bibverse{7}
dazu erwürgten sie Parsandatha, Dalphon, Aspatha, \bibverse{8} Poratha,
Adalja, Aridatha, \bibverse{9} Parmastha, Arisai, Aridai, Vajesatha,
\bibverse{10} die zehn Söhne Hamans, des Sohne Hammedathas, des
Judenfeindes. Aber an die Güter legten sie ihre Hände nicht.
\bibverse{11} Zu derselben Zeit kam die Zahl der Erwürgten zu Schloß
Susan vor den König. \bibverse{12} Und der König sprach zu der Königin
Esther: Die Juden haben zu Schloß Susan fünfhundert Mann erwürgt und
umgebracht und die zehn Söhne Hamans; was werden sie tun in den andern
Ländern des Königs? was bittest du, daß man dir gebe? und was forderst
du mehr, daß man tue? \bibverse{13} Esther sprach: Gefällt's dem König,
so lasse er auch morgen die Juden tun nach dem heutigen Gebot, und die
zehn Söhne Hamans soll man an den Baum hängen. \bibverse{14} Und der
König hieß also tun. Und das Gebot ward zu Susan angeschlagen, und die
zehn Söhne Haman wurden gehängt. \bibverse{15} Und die Juden zu Susan
versammelten sich auch am vierzehnten Tage des Monats Adar und erwürgten
zu Susan dreihundert Mann; aber an ihre Güter legten sie ihre Hände
nicht. \bibverse{16} Aber die andern Juden in den Ländern des Königs
kamen zusammen und standen für ihr Leben, daß sie Ruhe schafften vor
ihren Feinden, und erwürgten ihrer Feinde fünfundsiebzigtausend; aber an
ihre Güter legten sie ihre Hände nicht. \bibverse{17} Das geschah am
dreizehnten Tage des Monats Adar, und sie ruhten am vierzehnten Tage
desselben Monats; den machte man zum Tage des Wohllebens und der Freude.
\bibverse{18} Aber die Juden zu Susan waren zusammengekommen am
dreizehnten und am vierzehnten Tage und ruhten am fünfzehnten Tag; und
den Tag machte man zum Tage des Wohllebens und der Freude. \bibverse{19}
Darum machten die Juden, die auf den Dörfern und Flecken wohnten, den
vierzehnten Tag des Monats Adar zum Tag des Wohllebens und der Freude,
und sandte einer dem andern Geschenke. \bibverse{20} Und Mardochai
schrieb diese Geschichten auf und sandte Briefe an alle Juden, die in
den Landen des Königs Ahasveros waren, nahen und fernen, \bibverse{21}
daß sie annähmen und hielten den vierzehnten und fünfzehnten Tag des
Monats Adar jährlich, \bibverse{22} nach den Tagen, darin die Juden zur
Ruhe gekommen waren von ihren Feinden und nach dem Monat, darin ihre
Schmerzen in Freude und ihr Leid in gute Tage verkehrt war; daß sie
dieselben halten sollten als Tage des Wohllebens und der Freude und
einer dem andern Geschenke schicken und den Armen mitteilen.
\bibverse{23} Und die Juden nahmen's an, was sie angefangen hatten zu
tun und was Mardochai an sie schrieb: \bibverse{24} wie Haman, der Sohn
Hammedathas, der Agagiter, aller Juden Feind, gedacht hatte, alle Juden
umzubringen, und das Pur, das ist das Los, werfen lassen, sie zu
erschrecken und umzubringen; \bibverse{25} und wie Esther zum König
gegangen war und derselbe durch Briefe geboten hatte, daß seine bösen
Anschläge, die er wider die Juden gedacht, auf seinen Kopf gekehrt
würden; und wie man ihn und seine Söhne an den Baum gehängt hatte.
\bibverse{26} Daher sie diese Tage Purim nannten nach dem Namen des
Loses. Und nach allen Worten dieses Briefes und dem, was sie selbst
gesehen hatten und was an sie gelangt war, \bibverse{27} richteten die
Juden es auf und nahmen's auf sich und auf ihre Nachkommen und auf alle,
die sich zu ihnen taten, daß sie nicht unterlassen wollten, zu halten
diese zwei Tage jährlich, wie die vorgeschrieben und bestimmt waren;
\bibverse{28} daß diese Tage nicht vergessen, sondern zu halten seien
bei Kindeskindern, bei allen Geschlechtern, in allen Ländern und
Städten. Es sind die Tage Purim, welche nicht sollen übergangen werden
unter den Juden, und ihr Gedächtnis soll nicht umkommen bei ihren
Nachkommen. \bibverse{29} Und die Königin Esther, die Tochter Abihails,
und Mardochai, der Jude, schrieben mit ganzem Ernst, um es zu
bestätigen, diesen zweiten Brief von Purim; \bibverse{30} und er sandte
die Briefe zu allen Juden in den hundert und siebenundzwanzig Ländern
des Königreichs des Ahasveros mit freundlichen und treuen Worten:
\bibverse{31} daß sie annähmen die Tage Purim auf die bestimmte Zeit,
wie Mardochai, der Jude, über sie bestätigt hatte und die Königin
Esther, und wie sie für sich selbst und ihre Nachkommen bestätigt hatten
die Geschichte der Fasten und ihres Schreiens. \bibverse{32} Und Esther
befahl, die Geschichte dieser Purim zu bestätigen. Und es ward in ein
Buch geschrieben.

\hypertarget{section-9}{%
\section{10}\label{section-9}}

\bibverse{1} Und der König Ahasveros legte Zins aufs Land und auf die
Inseln im Meer. \bibverse{2} Aber alle Werke seiner Gewalt und Macht und
die große Herrlichkeit Mardochais, die ihm der König gab, siehe, das ist
geschrieben in der Chronik der Könige in Medien und Persien.
\bibverse{3} Denn Mardochai, der Jude, war der nächste nach dem König
Ahasveros und groß unter den Juden und angenehm unter der Menge seiner
Brüder, der für sein Volk Gutes suchte und redete das Beste für sein
ganzes Geschlecht.
