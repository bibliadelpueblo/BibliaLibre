\hypertarget{la-penitencia-introductoria}{%
\subsection{La penitencia
introductoria}\label{la-penitencia-introductoria}}

\hypertarget{section}{%
\section{1}\label{section}}

\bibleverse{1} El Señor envió un mensaje al profeta Zacarías, hijo de
Berequías, hijo de Idó, en el octavo mes del segundo año del reinado de
Darío, diciendo:\footnote{\textbf{1:1} Al comienzo del libro de Zacarías
  hay comillas dentro de las comillas. Si se identificaran todas, el
  resultado sería un conjunto difícil de comillas dentro de otras
  comillas; de hecho, habría cinco grados de comillas. En consecuencia,
  aquí (y en la mayoría de los libros de los profetas menores) hemos
  omitido las comillas de manera general, excepto cuando ayudan a
  identificar a los hablantes.} \footnote{\textbf{1:1} Esd 5,1}
\bibleverse{2} El Señor estuvo muy enojado\footnote{\textbf{1:2}
  ``enojado''. En muchos de los libros proféticos del Antiguo Testamento
  se dice que Dios está enojado. Pero debe tenerse en cuenta que esta es
  una descripción de la oposición y la intensa hostilidad de Dios hacia
  el mal y la rebelión, en lugar de la ira que experimentan los seres
  humanos, que es la que tendemos a tomar como referencia al leer. La
  ira humana está basada en emociones y es incluso irracional. La ira de
  Dios es una respuesta racional a la maldad, y se basa en su deseo de
  salvar y sanar, en lugar de exigir venganza retributiva. Su enfoque es
  asegurarse de que las personas entiendan el terrible peligro en el que
  se encuentran al persistir en el mal.} con sus padres. \bibleverse{3}
Así que dice esto al pueblo: Vuelvan a mi, y yo volveré a ustedes, dice
el Señor Todopoderoso.\footnote{\textbf{1:3} Literalmente, ``El Señor
  Todopoderoso dice: Regresen a mi, declara el Señor Todopoderoso, y yo
  regresaré a ustedes, dice el Señor Todopoderoso''. Hemos omitido las
  repeticiones para facilitar la lectura.} \bibleverse{4} No sean como
sus padres. Ellos recibieron advertencia de los profetas: ¡Abandonen sus
malos caminos y sus malas acciones! Pero no escucharon ni me prestaron
atención, dice el Señor. \footnote{\textbf{1:4} Jer 3,12; Ezeq 33,11}
\bibleverse{5} ¿Dónde están sus padres ahora? ¿Acaso vivieron esos
profetas para siempre? \bibleverse{6} Todas mis instrucciones y
advertencias,\footnote{\textbf{1:6} 1:6 Literalmente, ``regulaciones''.}
que comuniqué a través de mis siervos los profetas, ¿acaso no se
cumplieron en sus padres antepasados? Por eso se arrepintieron y
dijeron: ``Lo que el Señor Todopoderoso quería hacer con nosotros era lo
que merecíamos por nuestros caminos y maldad. Por eso hizo lo que
prometió''.

\hypertarget{los-cuatro-jinetes-en-cuatro-caballos-de-diferentes-colores-ante-el-seuxf1or-el-celo-de-dios-por-sion}{%
\subsection{Los cuatro jinetes en cuatro caballos de diferentes colores
ante el Señor; El celo de Dios por
Sion}\label{los-cuatro-jinetes-en-cuatro-caballos-de-diferentes-colores-ante-el-seuxf1or-el-celo-de-dios-por-sion}}

\bibleverse{7} El Señor envió un mensaje al profeta Zacarías, hijo de
Berequías, hijo de Idó, en el vigesimocuarto día del onceavo mes (el mes
de Sebat) en el segungo año del reinado de Darío: \bibleverse{8} Por la
noche vi a un hombre sentado en un caballo rojo que se paró en medio de
unos mirtos en un valle angosto. Detrás de él había caballos rojos,
marrones y blancos, con sus jinetes.\footnote{\textbf{1:8} ``Con sus
  jinetes''. Implícito. Ver versículo 11.} \bibleverse{9} Yo le
pregunté: ``Mi Señor, ¿quiénes son estos?'' Y el ángel al que le hablé
me respondió: ``Ven, te mostraré''.

\bibleverse{10} Y el hombre que estaba entre los mirtos dijo: ``Estos
son a quienes el Señor ha enviado para vigilar la tierra''.

\bibleverse{11} Los jinetes\footnote{\textbf{1:11} Implícito.} le
informaron al ángel del Señor que estaba entre los mirtos: ``Hemos
estado vigilando la tierra y vimos que toda la tierra ha sido
apaciguada''.\footnote{\textbf{1:11} ``Apaciguada''. En el contexto,
  esta ``paz'' tiene más que ver con ser forzado a la sumisión y la
  derrota que a un tiempo de armonía y tranquilidad. Podría compararse
  con la impuesta ``Pax Romana'' de épocas posteriores en la que los
  romanos afirmaron su control sobre las naciones que habían derrotado
  militarmente y trajeron la ``paz''.}

\bibleverse{12} Entonces el ángel del Señor dijo: ``Dios Todopoderoso,
¿cuánto tiempo pasará antes de que tengas misericordia de Jerusalén y de
las ciudades de Judá con las que has estado airado por los últimos
setenta años?'' \footnote{\textbf{1:12} Sal 102,14; Dan 9,2}

\bibleverse{13} Entonces el Señor le respondió al ángel con el cual yo
hablaba, diciéndole palabras bondadosas y de consuelo. \bibleverse{14}
Entonces el ángel me dijo: ``Esto es lo que debes anunciar. El Señor
Todopoderoso dice así: Yo soy un Dios protector y celoso\footnote{\textbf{1:14}
  ``protector y celoso'': Estar preocupado por los seres amados.} de
Jerusalén y del Monte de Sión, \bibleverse{15} y estoy enojado en gran
manera con las naciones arrogantes que creen que están seguras. Estaba
un poco enojado con mi pueblo,\footnote{\textbf{1:15} ``Con mi pueblo''
  implícito.} pero ellos han hecho que el castigo sea más
severo.\footnote{\textbf{1:15} En otras palabras, Dios había permitido
  que las naciones paganas castigaran a su pueblo por sus pecados, pero
  estas naciones fueron demasiado lejos en sus ataques.} \bibleverse{16}
``Por eso, esto es lo que dice el Señor: He vuelto a ser misericordioso
con Jerusalén. Mi Templo volverá a construirse allí, así como la
ciudad,\footnote{\textbf{1:16} ``Así como la ciudad'': Literalmente,
  ``una línea de medida será extendida sobre Jerusalén''.} declara el
Señor Todopoderoso. \footnote{\textbf{1:16} Zac 8,3}

\bibleverse{17} ``Anuncia también esto, dice el Señor Todopoderoso: La
prosperidad inundará mis ciudades. Yo, el Señor, consolaré a Sión, y
Jerusalén será mi ciudad escogida''. \footnote{\textbf{1:17} Is 40,1-2;
  Is 14,1}

\hypertarget{los-cuatro-cuernos-y-los-cuatro-herreros-el-juicio-de-dios-de-aniquilaciuxf3n-sobre-las-potencias-mundiales-hostiles}{%
\subsection{Los cuatro cuernos y los cuatro herreros; El juicio de Dios
de aniquilación sobre las potencias mundiales
hostiles}\label{los-cuatro-cuernos-y-los-cuatro-herreros-el-juicio-de-dios-de-aniquilaciuxf3n-sobre-las-potencias-mundiales-hostiles}}

\bibleverse{18} Entonces miré y vi cuatro cuernos de
animales.\footnote{\textbf{1:18} Los cuernos en los escritos proféticos
  son símbolo de poderes.} \bibleverse{19} ``¿Qué es esto?'' le pregunté
al ángel con el que hablaba. ``Estos son los cuernos que dispersaron a
Judá, Israel y Jerusalén'', respondió.

\bibleverse{20} Entonces el Señor me mostró a cuatro hombres
artesanos.\footnote{\textbf{1:20} Probablemente herreros o trabajadores
  de metal.} \bibleverse{21} ``¿Qué vienen a hacer estos hombres?'' le
pregunté. El ángel respondió: ``Los cuatro cuernos---o estas
naciones---dispersaron a Judá, humillando al pueblo de tal manera que no
podían levantar sus cabezas. Estos artesanos han venido para aterrorizar
a estas naciones, y para destruirlas, a aquellas naciones que usaron su
poder contra la tierra de Judá, y dispersaron al pueblo''.

\hypertarget{el-hombre-de-la-luxednea-de-mediciuxf3n-restaurando-jerusaluxe9n-a-una-ciudad-rica-y-abierta}{%
\subsection{El hombre de la línea de medición, Restaurando Jerusalén a
una ciudad rica y
abierta}\label{el-hombre-de-la-luxednea-de-mediciuxf3n-restaurando-jerusaluxe9n-a-una-ciudad-rica-y-abierta}}

\hypertarget{section-1}{%
\section{2}\label{section-1}}

\bibleverse{1} Entonces miré otra vez y vi a un hombre con una línea de
medida en su mano. \bibleverse{2} ``¿A dónde vas?'' le pregunté. ``Voy a
Jerusalén a medir su anchura y su longitud'', respondió.

\bibleverse{3} El ángel con el que yo hablaba vino Adelante y otro ángel
vino a su encuentro \bibleverse{4} y le dijo: ``Corre, y dije al
joven\footnote{\textbf{2:4} Refiriéndose al hombre con la línea de
  medida que se menciona en el versículo 2:1.} que Jerusalén tendrán
tantos habitantes y animales que será demasiado grande para tener
muros''. \bibleverse{5} El Señor declara: Yo mismo será un muro de fuego
alrededor de la ciudad, y seré la gloria dentro de ella.

\hypertarget{invitaciuxf3n-a-regresar-a-casa-a-todos-los-camaradas-que-todavuxeda-estuxe1n-en-babilonia}{%
\subsection{Invitación a regresar a casa a todos los camaradas que
todavía están en
Babilonia}\label{invitaciuxf3n-a-regresar-a-casa-a-todos-los-camaradas-que-todavuxeda-estuxe1n-en-babilonia}}

\bibleverse{6} ¡Corre! ¡Corre! Escapa de la tierra del norte, dice el
Señor, porque yo te he dispersado a los cuatro vientos del cielo.
\bibleverse{7} ¡Corre, pueblo de Sión! Todos ustedes que viven en
Babilonia deben escapar.

\hypertarget{tres-proclamas-de-salvaciuxf3n-para-juduxe1}{%
\subsection{Tres proclamas de salvación para
Judá}\label{tres-proclamas-de-salvaciuxf3n-para-juduxe1}}

\bibleverse{8} Porque esto es lo que dice el Señor Todopoderoso:
Después, el glorioso Señor\footnote{\textbf{2:8} ``Después, el glorioso
  Señor'': Este término hebreo no está claro. Literalmente, ``después de
  gloria''.} me envió contra las naciones que te sitiaron. Porque los
que te tocan, es como si tocaran la luz de sus ojos. \footnote{\textbf{2:8}
  Ezeq 38,11} \bibleverse{9} Yo levantaré mi mano contra ellos y sus
antiguos esclavos los squearán. Entonces sabrán que el Señor
Todopoderoso me ha enviado.\footnote{\textbf{2:9} ``Me ha enviado''.
  Zacarías se refiere a sí mismo, y dice que el cumplimiento de esta
  profecía confirmará la verdad de su mensaje.} \footnote{\textbf{2:9}
  Zac 9,8}

\bibleverse{10} Canta y celebra, pueblo de Sión, porque yo vengo a vivir
contigo, declara el Señor. \bibleverse{11} Ese día, muchas naciones
creerán\footnote{\textbf{2:11} ``Creerán'': Literalmente, ``se unirán''.}
en el Señor, y serán mi pueblo. Yo viviré en medio de ustedes, y ustedes
sabrán que el Señor Todopoderoso me ha enviado a ustedes.
\bibleverse{12} El pueblo de Judá será el pueblo especial del Señor en
la tierra santa, y una vez más elegirá a Israel como su ciudad especial.
\footnote{\textbf{2:12} Deut 32,10}

\bibleverse{13} Callen ante el Señor, todos ustedes, porque él se ha
levantado del lugar santo donde habita.

\hypertarget{el-sacerdocio-que-recibiuxf3-el-sumo-sacerdote-josuuxe9-y-que-promete-para-el-futuro}{%
\subsection{El sacerdocio que recibió el sumo sacerdote Josué y que
promete para el
futuro}\label{el-sacerdocio-que-recibiuxf3-el-sumo-sacerdote-josuuxe9-y-que-promete-para-el-futuro}}

\hypertarget{section-2}{%
\section{3}\label{section-2}}

\bibleverse{1} Entonces el Señor\footnote{\textbf{3:1} Literalmente,
  ``él'' que puede referirse ya sea a Señor, o al ángel que ya se ha
  mencionado.} me mostró a Josué, el sumo sacerdote, en pie delante del
ángel del Señor, y Satanás\footnote{\textbf{3:1} Satanás significa ``el
  acusador''.} estaba en pie a su mano derecha, acusándolo. \footnote{\textbf{3:1}
  Ag 1,1; Job 1,9; Apoc 12,10} \bibleverse{2} Y el Señor le dijo a
Satanás: ``El Señor te reprende, Satanás. Yo, el Señor que he escogido a
Jerusalén, te reprendo. ¿Acaso no es como un carbón arrebatado de la
fogata?'' \footnote{\textbf{3:2} Jds 1,9; Am 4,11}

\bibleverse{3} Josué estaba usando ropas sucias mientras estaba en pie
delante del ángel. \bibleverse{4} Y el ángel le dijo a
aquellos\footnote{\textbf{3:4} Se presume que habla de otros ángeles.}
que estaban allí: ``Quiten su ropa sucia''. Y entonces le dijo a Josué:
``Mira como he quitado tus pecados, y ahora te vestiré con ropas
finas''.

\bibleverse{5} Entonces yo dije: ``Pongan un turbante limpio sobre su
cabeza''. Y pusieron un turbante limpio en su cabeza, y ropas, mientras
el ángel del Señor permanecía en pie allí. \footnote{\textbf{3:5} Éxod
  28,39}

\bibleverse{6} Entonces el ángel del Señor le habló solemnemente a
Josué, aconsejándole: \bibleverse{7} ``Esto es lo que el Señor
Todopoderoso dice: Si sigues mis caminos y observas mis mandamientos, tú
gobernarás mi Templo y sus atrios. Yo te dejaré caminar en medio de los
que están aquí en pie. \bibleverse{8} ¡Presta atención, sumo sacerdote
Josué, y todos los sacerdotes a quienes enseñas!\footnote{\textbf{3:8}
  Literalmente, ``amigos que se sientan delante de ti''.} Eres una señal
de las cosas buenas que vendrán. ¡Miren! Yo traeré a mi siervo, la
rama.\footnote{\textbf{3:8} Tanto ``mi siervo'' como ``la rama'' con
  títulos que se refieren al Mesías.} \footnote{\textbf{3:8} Is 8,18;
  Zac 2,12; Jer 23,5; Jer 33,15} \bibleverse{9} Nota que he puesto una
piedra preciosa delante de Josué, una sola piedra con siete ángulos.
Miren que yo mismo tallaré siete ojos en ella, delara el Señor
Todopoderoso, y borraré los pecados de esta tierra en un solo día.
\footnote{\textbf{3:9} Zac 4,10; Apoc 5,6} \bibleverse{10} Ese día,
todos invitarán a sus amigos a sentarse en paz\footnote{\textbf{3:10}
  ``En paz'' implícito.} bajo sus vides e higueras, dice el Señor
Todopoderoso''.\footnote{\textbf{3:10} 1Re 5,5; Miq 4,4}

\hypertarget{el-candelero-de-oro-entre-los-dos-olivos}{%
\subsection{El candelero de oro entre los dos
olivos}\label{el-candelero-de-oro-entre-los-dos-olivos}}

\hypertarget{section-3}{%
\section{4}\label{section-3}}

\bibleverse{1} Entonces el ángel con el que yo hablaba volvió y llamó mi
atención, como cuando despiertan a alguien de su sueño.\footnote{\textbf{4:1}
  Claramente Zacrías no estaba dormido, sino sumido en sus pensamientos.}
\bibleverse{2} ``¿Qué ves?'' me preguntó. ``Veo un candelabro hecho de
oro sólido con un tazón que sostiene siete lámparas sobre él, cada una
con siete labios.\footnote{\textbf{4:2} Los labios eran pequeños canales
  dentro de los cuales estaban las mechas o pabilos de un candelabro.}

\bibleverse{3} También veo árboles de olivos, uno a la derecha y uno a
la izquierda del tazón''.

\bibleverse{4} Entonces le pregunté al ángel con el que hablaba: ``¿Qué
son estos, mi señor?''

\hypertarget{la-interpretaciuxf3n-de-la-visiona}{%
\subsection{La interpretación de la
visiona}\label{la-interpretaciuxf3n-de-la-visiona}}

\bibleverse{5} ``¿No sabes lo que son?'' respondió el ángel. ``No, mi
señor'', respondí.

\bibleverse{6} Entonces me dijo: ``Este es el mensaje del Señor a
Zorobabel: No es con poder, ni con fuerza sino con mi espíritu, dice el
Señor. \bibleverse{7} Aún los obstáculos grandes como montañas serán
aplastados ante Zorobabel. Finalmente traerá la piedra
angular\footnote{\textbf{4:7} Probablemente se refiere a la piedra
  angular del Templo reconstruido.} con gritos de `¡Bendiciones sobre
ella!'\,'' \footnote{\textbf{4:7} Sal 122,6}

\bibleverse{8} Entonces el Señor me dio otro mensaje. \bibleverse{9}
Zorobabel con sus propias manos estableció los cimientos de este Templo,
y será completado de la misma forma. Entonces sabrás\footnote{\textbf{4:9}
  ``sabrás'' es singular y se cree que se refiere a Zorobabel.} que el
Señor Todopoderoso me ha enviado. \bibleverse{10} ¿Acaso quién se atreve
a menospreciar estos tiempos de comienzos pequeños? Serán felices cuando
vean la plomada en la mano de Zorobabel. ``Las siete lámparas
representan los ojos del Señor que ve a todo el mundo''.\footnote{\textbf{4:10}
  El ángel está respondiendo su propia pregunta que aparece en el
  versículo 4:5 respecto al significado de las lámparas.} \footnote{\textbf{4:10}
  Ag 2,3; Zac 3,9}

\bibleverse{11} Entonces le pregunté al ángel: ``¿Que significan los dos
árboles de olivo que están a los lados del candelabro?''

\bibleverse{12} Y también le pregunté: ``¿Que significan las dos ramas
de olvido de las cuales sale el aceite dorado a través de las boquillas
doradas?''

\bibleverse{13} ``¿No lo sabes?'' respondió el ángel. ``No, mi señor'',
le respondí.

\bibleverse{14} ``Estos son los dos que han sido ungidos\footnote{\textbf{4:14}
  Se debate sobre la identidad de estos dos seres. Algunos los ven como
  seres celestiales, y otros los identifican como Josué y Zorobabel.} y
que están junto al Señor de toda la tierra'', respondió.

\hypertarget{el-pergamino-volador}{%
\subsection{El pergamino volador}\label{el-pergamino-volador}}

\hypertarget{section-4}{%
\section{5}\label{section-4}}

\bibleverse{1} Miré una vez más y vi un rollo que volaba. \bibleverse{2}
``¿Qué ves?'' me preguntó el ángel. ``Veo un rollo que vuela'',
respondí. ``Tiene diez metros de largo y quince de ancho''.\footnote{\textbf{5:2}
  Literalmente, ``Veinte codos de largo y diez codos de ancho''.}

\bibleverse{3} Entonces me dijo: ``Esta es la maldición que caerá sobre
todo el mundo. Cualquiera que roba será purgado\footnote{\textbf{5:3} O
  ``eliminado''.} de entre la sociedad, según un lado del rollo.
Cualquiera que jura con engaño, será purgado de entre la Sociedad, según
el otro lado del rollo''.\footnote{\textbf{5:3} Un lado del rollo/el
  otro lado: Este es el referente más común aquí, pero hay otras
  interpretaciones.} \bibleverse{4} ``Yo he enviado esta maldición y
entrará a la casa del ladrón, y a la casa del que jura con mentiras en
mi nombre, declara el Señor Todopoderoso. La maldición permanecerá en
esa casa y destruirá tanto las vigas como los ladrillos''.

\hypertarget{la-mujer-del-gran-barril}{%
\subsection{La mujer del gran barril}\label{la-mujer-del-gran-barril}}

\bibleverse{5} Entonces el ángel con el que yo había estado hablando
vino hacia mi, y me dijo: ``Mira, ¿ves eso que se mueve?''\footnote{\textbf{5:5}
  O ``eso que se acerca''.}

\bibleverse{6} ``¿Qué es?'' le pregunté. ``Lo que ves moverse es un
barril\footnote{\textbf{5:6} Literalmente, ``efa'', un recipiente donde
  se medía el grano. A veces se traduce como ``canasta''. Sin embargo,
  en este caso es claro que debe ser lo suficientemente grande para que
  haya una mujer adentro (5:7), por eso hemos usado la palabra barril
  para esta traducción.} lleno de los pecados\footnote{\textbf{5:6}
  Texto tomado de la septuaginta. El texto hebreo dice ``ojo'', pero es
  difícil entenderlo en context, y el texto revisado solo cambia una
  letra de texto hebreo.} de todos en la nación'',\footnote{\textbf{5:6}
  La nación de Judá.} respondió. \footnote{\textbf{5:6} Miq 6,10}

\bibleverse{7} Entonces la tapa del barril se levantó y había una mujer
sentada adentro. \bibleverse{8} ``Ella representa la maldad'', me dijo,
y la empujó hacia adentro de nuevo, forzando la tapa hasta cerrarla.

\bibleverse{9} Levanté la mirada otra vez y vi dos mujeres que volaban
hacia mi. Sus alas parecían alas de cigüeña. Ellos recogieron el barril
y se fueron volando, muy alto en el cielo. \bibleverse{10} ``¿A dónde lo
llevan?'' le pregunté al ángel con que hablaba.

\bibleverse{11} ``Lo llevan a la tierra de Babilonia\footnote{\textbf{5:11}
  Literalmente, ``Sinar''.} para construir una casa para él. Cuando la
casa esté lista, el barril será puesto sobre su cimiento''.\footnote{\textbf{5:11}
  ``Sobre su cimiento''. Algunos interpretan esto queriendo decir que la
  mujer que representa la maldad será adorada, y que esa ``casa'' en
  realidad es un Templo.}

\hypertarget{la-salida-de-los-cuatro-carros-de-guerra-celestiales}{%
\subsection{La salida de los cuatro carros de guerra
celestiales}\label{la-salida-de-los-cuatro-carros-de-guerra-celestiales}}

\hypertarget{section-5}{%
\section{6}\label{section-5}}

\bibleverse{1} Entonces volví a mirar y vi cuatro carruajes que salían
de en medio de dos montañas que parecían como de bronce. \footnote{\textbf{6:1}
  Zac 1,8; Apoc 6,2-8} \bibleverse{2} Al primer carruaje lo tiraban
caballos rojos, al Segundo caballos negros; \bibleverse{3} al tercero,
caballos blancos, y al cuarto, caballos grises. Todos eran caballos
fuertes. \bibleverse{4} ``¿Qué significa esto, mi señor?'' Le pregunté
al ángel con el que hablaba.

\bibleverse{5} ``Ellos van a los cuatro vientos del cielo,\footnote{\textbf{6:5}
  En otras palabras, todos iban en dorecciones distintas.} después de
haberse presentado al Señor de toda la tierra'', explicó el ángel.
\bibleverse{6} El carruaje tirado por caballos blancos fue al norte; el
carruaje con caballos blancos, fue en dirección al oeste; y el carruaje
que era tirado por caballos grises, se dirigió al sur. \bibleverse{7}
Cuando los caballos fuertes salieron, iban dispuestos y presurosos a
patrullar la tierra. Y él dijo: ``¡Vayan y vigilen la tierra!'' Entonces
los caballos salieron y empezaron a vigilar la tierra. \footnote{\textbf{6:7}
  Zac 1,10}

\bibleverse{8} Entonces el ángel me llamó, diciéndome: ``¡Mira! Los que
se fueron al norte han logrado lo que el Señor quería\footnote{\textbf{6:8}
  ``Lograron lo que el Señor quería'', Literalmente, ``hicieron
  descansar mi espíritu''.} en la tierra del norte''.

\hypertarget{la-fabricaciuxf3n-de-una-corona-para-zorobabel}{%
\subsection{La fabricación de una corona para
Zorobabel}\label{la-fabricaciuxf3n-de-una-corona-para-zorobabel}}

\bibleverse{9} Entonces el Señor me dio otro mensaje: \bibleverse{10}
Toma los regalos traídos por Jeldai, Tobías y Jedaías, los exiliados que
vuelven de Babilonia, y ve de inmediato a la casa de Josías, hijo de
Sofonías. \bibleverse{11} Usa la plata y el oro que trajeron, y manda a
hacer un corona, y ponla sobre la cabeza del sumo sacerdote Josué, hijo
de Josadac. \bibleverse{12} Y dile que esto es lo que dice el Señor:
¡Miren! El hombre que se llama La Rama crecerá\footnote{\textbf{6:12} O
  ``retoño''.} de donde viene y construirá el Templo del Señor.
\bibleverse{13} Él fue quien construyó el Templo del Señor y a él se le
dará el honor de gobernar tanto desde el trono real, como desde el trono
de sumo sacerdote, y habrá paz y comprensión en sus dos funciones.
\footnote{\textbf{6:13} Sal 110,4} \bibleverse{14} La corona se
mantendrá en el Templo del Señor como un recordatorio de Jeldai, Tobías,
Jedaía y Josué\footnote{\textbf{6:14} Literalmente, ``Hen''.} el hijo de
Sofonías.

\bibleverse{15} Los que habitan en tierras lejanas vendrán y construirán
el Templo del Señor, y sabrás que el Señor Todopoderoso me ha enviado
ante ustedes. Esto ocurrirá si escuchan atentamente lo que el Señor les
dice.

\hypertarget{solicitud-de-los-hombres-de-betel-sobre-los-duxedas-de-ayuno}{%
\subsection{Solicitud de los hombres de Betel sobre los días de
ayuno}\label{solicitud-de-los-hombres-de-betel-sobre-los-duxedas-de-ayuno}}

\hypertarget{section-6}{%
\section{7}\label{section-6}}

\bibleverse{1} El Señor envió un mensaje a Zacarías, en el cuarto día
del novena mes, el mes de Quisleu. Esto fue durante el cuarto año del
reinado de Darío. \bibleverse{2} Bethel-Sarezer envió a Regem-Melec y a
sus hombres para pedir la bendición del Señor. \bibleverse{3} Fueron a
preguntar a los sacerdotes del Templo del Dios Todopoderoso y a los
profetas: ``¿Debo seguir de luto y ayuno en el quinto mes como lo he
hecho por muchos años?''

\hypertarget{los-duxedas-de-ayuno-son-algo-externo-sin-sentido-para-dios}{%
\subsection{Los días de ayuno son algo externo, sin sentido para
Dios}\label{los-duxedas-de-ayuno-son-algo-externo-sin-sentido-para-dios}}

\bibleverse{4} Entonces el Señor Todopoderoso me envió un mensaje,
diciendo: \bibleverse{5} Dile a todos en la nación y a los sacerdotes:
Cuando ayunaban y guardaban luto en el quinto y el séptimo mes durante
estos setenta años, ¿lo hacían por mi? \footnote{\textbf{7:5} Zac 8,19;
  Is 58,5} \bibleverse{6} Y cuando comen y beben, ¿acaso no lo hacen
para ustedes mismos?

\hypertarget{a-travuxe9s-de-los-profetas-dios-siempre-solo-ha-exigido-justicia-y-amor-a-su-pueblo}{%
\subsection{A través de los profetas, Dios siempre solo ha exigido
justicia y amor a su
pueblo}\label{a-travuxe9s-de-los-profetas-dios-siempre-solo-ha-exigido-justicia-y-amor-a-su-pueblo}}

\bibleverse{7} ¿No es esto lo que el Señor les dijo a través de los
profetas anteriores, cuando Jerusalén era próspera y deshabitada, y
cuando el pueblo vivía en el Neguev y la Sefelá?\footnote{\textbf{7:7}
  ``El Neguév y la Sefelá'': el área que está al sur y al oeste.}

\bibleverse{8} El Señor Todopoderoso me envió otro mensaje.
\bibleverse{9} Esto es lo que el Señor dice: Juzguen con justicia y
verdad. Tengan misericordia y bondad unos por otros. \bibleverse{10} No
exploten a las viudas ni a los huérfanos, tampoco a los extranjeros ni a
los pobres. No hagan planes sorbre cómo hacerse daño unos a otros.
\footnote{\textbf{7:10} Éxod 22,20-21}

\bibleverse{11} Pero ellos se negaron a oír. Fueron obstinados, dieron
la espalda y cerraron sus oídos. \bibleverse{12} Endurecieron sus
corazones como piedras. Se negaron a oír la ley o lo que el Señor
Todopoderoso les decía por medio de su Espíritu a través de los profetas
anteriores. Por eso el Señor Todopoderoso se enojó con ellos.
\bibleverse{13} Así que como no me oyeron cuando los llamé, yo no
escucharé cuando me llamen ellos, dice el Señor Todopoderoso.
\bibleverse{14} Con los vientos de una tormenta yo dispersé a las
naciones donde vivían como extranjeros. La tierra que abandonaron se
volvió tan desolada que ni siquiera los viajeros pasaban por ella.
Convirtieron la Tierra Prometida en un desierto.

\hypertarget{dios-ama-a-su-pueblo-y-le-permitiruxe1-alcanzar-una-gran-felicidad-pero-siempre-defiende-sus-exigencias-morales}{%
\subsection{Dios ama a su pueblo y le permitirá alcanzar una gran
felicidad, pero siempre defiende sus exigencias
morales}\label{dios-ama-a-su-pueblo-y-le-permitiruxe1-alcanzar-una-gran-felicidad-pero-siempre-defiende-sus-exigencias-morales}}

\hypertarget{section-7}{%
\section{8}\label{section-7}}

\bibleverse{1} Entonces el Señor Todopoderoso me envió otro mensaje.
\bibleverse{2} Esto es lo que dice el Señor Todopoderoso: Soy celoso y
protector del pueblo de Sión. Soy apasionado por ellos en gran manera.
\footnote{\textbf{8:2} Zac 1,14}

\bibleverse{3} Esto es lo que dice el Señor: Yo le regresado a Sión, y
viviré en Jerusalén. Entonces a Jerusalén se le llamará la ``Ciudad
Fiel'', y a la montaña del Señor Todopoderoso se le llamará el ``Santo
Monte''. \footnote{\textbf{8:3} Zac 1,16}

\bibleverse{4} Esto es lo que dice el Señor: Los ancianos podrán
sentarse nuevamente en las calles de Jerusalén, cada uno con sus
bastones\footnote{\textbf{8:4} Literalmente, ``varas''.} que usan por su
edad. \footnote{\textbf{8:4} Is 65,20} \bibleverse{5} Las calles estarán
llenas de niños y niñas jugando felices.

\bibleverse{6} Esto es lo que dice el Señor Todopoderoso: Ahora parece
demasiado bueno para ser cierto\footnote{\textbf{8:6} ``Parece demasiado
  bueno para ser cierto'': o ``puede parecer imposible (o de carácter
  maravilloso)''.} para ustedes, mi pueblo remanente de estos días.
¿Pero acaso es imposible para mi? pregunta e Señor Todopoderoso.

\bibleverse{7} Esto es lo que dice el Señor Todopoderoso: Yo salvaré a
mi pueblo de las naciones del este y del oeste. \bibleverse{8} Los
traeré de regreso y vivirán en Jerusalén, y serán mi pueblo y yo seré y
Dios fiel y verdadero.\footnote{\textbf{8:8} ``Verdadero'':
  Literalmente, ``justo''.} \footnote{\textbf{8:8} Jer 24,7}

\bibleverse{9} Esto es lo que dice el Señor Todopoderoso: Sean fuertes
para que el Templo sea terminado. Todos los que hoy están aquí, están
oyendo las mismas palabras de los profetas que estuvieron presentes en
el día que se fundó y se estableció el Templo del Señor Todopoderoso.
\footnote{\textbf{8:9} Is 35,3} \bibleverse{10} Antes de ese tiempo no
había suficiente\footnote{\textbf{8:10} Este versículo parece referirse
  más a tener suficiente para comer que al dinero, que era una mercancía
  poco común en ese momento.} comida para la gente o los animales. Nadie
podía vivir con normalidad porque no estaban seguros de sus enemigos, y
yo puse a todos los unos contra otros. \bibleverse{11} Pero ahora no
trataré más a mi remanente como los traté antes, declara el Señor
Todopoderoso. \bibleverse{12} Ellos segarán en paz. La vid producirá sus
uvas; el suelo dará cosecha y los cielos enviarán el agua sobre ellos.
Me aseguraré de que esto suceda con el remanente de este pueblo.
\bibleverse{13} Al pueblo de Judá e Israel: Así como fueron considerados
como una maldición entre las naciones, yo los salvaré y se convertirán
en una bendición. ¡No tengan miedo! ¡Sean fuertes! \footnote{\textbf{8:13}
  Gén 12,2}

\bibleverse{14} Porque esto es lo que dice el Señor Todopoderoso: Yo
decidí traer desastre sobre ustedes cuando sus antiguos padres
provocaron mi ira y no cambié mi parecer. \bibleverse{15} Pero ahora lo
he decidido, y hare bien a Jerusalén y al pueblo de Judá. ¡No tengan
miedo! \bibleverse{16} Esto es lo que deben hacer: Díganse la verdad los
unos a los otros. En sus cortes juzguen honestamente y con la verdad,
para lograr la paz. \bibleverse{17} No hagan planes sobre cómo hacer el
mal contra otros. Dejen de amar el engaño. Yo lo aborrezco, declara el
Señor. \footnote{\textbf{8:17} Zac 7,10}

\hypertarget{en-el-futuro-los-duxedas-de-ayuno-seruxe1n-reemplazados-por-festivales-felices-y-el-mundo-pagano-tambiuxe9n-participaruxe1-en-la-gloria-del-pueblo-de-dios}{%
\subsection{En el futuro, los días de ayuno serán reemplazados por
festivales felices, y el mundo pagano también participará en la gloria
del pueblo de
Dios}\label{en-el-futuro-los-duxedas-de-ayuno-seruxe1n-reemplazados-por-festivales-felices-y-el-mundo-pagano-tambiuxe9n-participaruxe1-en-la-gloria-del-pueblo-de-dios}}

\bibleverse{18} El Señor Todopoderoso me dio otro mensaje.
\bibleverse{19} Esto es lo que dice el Señor Todopoderoso: Los ayunos
que hacen el cuarto, quinto, séptimo y décimo mes serán tiempos de
alegría y regocijo para el pueblo de Judá. Y habrá fiestas de
celebración. Pero amen la verdad y la paz.

\bibleverse{20} Esto es lo que dice el Señor Todopoderoso: Las gentes
vendrán a Jerusalén de muchas naciones y cuidades,\footnote{\textbf{8:20}
  ``A Jerusalén'': implícito. Ver 8:22.} \bibleverse{21} e irán de una
ciudad a otra diciendo: ``Permítannos buscar al Señor y pedir la
bendición del Señor Todopoderoso. ¡Yo mismo iré!'' \bibleverse{22} Y
mucha gente y naciones poderosas vendrán a Jerusalén para pedir la
bendición del Señor Todopoderoso y buscar al Señor. \bibleverse{23} Esto
es lo que dice el Señor Todopoderoso: En ese tiempo diez hombres de
diferentes naciones e idiomas sujetarán el dobladillo de la capa de un
hombre judío y rogarán: ``Por favor, llévanos contigo, porque hemos
escuchado que Dios está contigo''.

\hypertarget{el-juicio-de-dios-sobre-los-pueblos-hostiles-en-siria-y-fenicia-y-su-defensa-de-jerusaluxe9n}{%
\subsection{El juicio de Dios sobre los pueblos hostiles en Siria y
Fenicia y su defensa de
Jerusalén}\label{el-juicio-de-dios-sobre-los-pueblos-hostiles-en-siria-y-fenicia-y-su-defensa-de-jerusaluxe9n}}

\hypertarget{section-8}{%
\section{9}\label{section-8}}

\bibleverse{1} Una profecía:\footnote{\textbf{9:1} Literalmente,
  ``carga''.} Un mensaje del Señor a la tierra de Hadrac, y Damasco es
su principal objetivo.\footnote{\textbf{9:1} ``Objetivo principal''
  Literalmente, ``lugar donde se posa''. En otras palabras, esta era la
  ciudad a quien la profecía estaba dirigida.} Porque los ojos de todos
los seres humanos y todas las tribus de Israel están atentos al
Señor,\footnote{\textbf{9:1} Esto también podría traducirse como:
  ``porque el Señor tiene sus ojos puestos sobre la humanidad así como
  sobre todas las tribus de Israel''.} \footnote{\textbf{9:1} Is 17,1}
\bibleverse{2} así también el territorio de Jamat, que está cerca de
Damasco. De igual manera Tiro y Sidón, porque son ciudades muy sabias.
\footnote{\textbf{9:2} Is 23,-1; Jer 47,4; Ezeq 26,1-28} \bibleverse{3}
El pueblo de Tiro construyó un castillo, y acumuló plata como el polvo,
y oro como el mugre en las calles. \bibleverse{4} Pero miren lo que
sucederá: El Señor les quitará todo lo que poseen, y destruirá su
defensa fuerte\footnote{\textbf{9:4} ``Defensa fuerte'', Literalmente,
  ``poder''. También puede referirse al Señor destruyendo el poder
  marítimo de Tiro.} hasta derribarla al mar. La ciudad será consumida
con fuego. \bibleverse{5} El puelo de Ascalón verá todo lo que sucederá
y temerán. Los que están en Gaza andarán de aquí para allá con angustia
como una mujer a punto de dar a luz; y el pueblo de Ecrón también
temblará, porque sus esperanzas se desvanecerán. El rey de Gaza será
asesinado, y Ascalón quedará como un desierto. \footnote{\textbf{9:5}
  Jer 47,-1} \bibleverse{6} Gentes de razas mezcladas vivirán en Asdod,
y yo quitaré el motivo de orgullo de los filisteos. \bibleverse{7}
Arrebataré la carne ensangrentada de sus bocas, y la comida impura de
sus quijadas.\footnote{\textbf{9:7} Los filisteos no seguían las
  regulaciones judías con respecto al sacrificio de animales,
  particularmente al drenar la sangre de la carne. En consecuencia,
  consumían carne que todavía contenía sangre y eso hacía que fuera
  comida impura.} Los que quedan le pertenecerán a nuestro Dios---serán
como una familia de Judá---y los de Ecrón serán parte de mi pueblo, tal
como los Jebusitas. \bibleverse{8} Acamparé en mi Templo para
salvaguardarlo de los invasores, ni habrá opresores que lo conquisten,
porque yo mismo seré el que vigila.

\hypertarget{entrada-y-bendiciuxf3n-del-rey-de-la-paz-en-jerusaluxe9n}{%
\subsection{Entrada y bendición del Rey de la Paz en
Jerusalén}\label{entrada-y-bendiciuxf3n-del-rey-de-la-paz-en-jerusaluxe9n}}

\bibleverse{9} Estén felices y celebren, pueblo de Sión! ¡Grita, pueblo
de Jerusalén! Mira, tu rey viene hacia ti. Él hace lo recto y trae la
salvación;\footnote{\textbf{9:9} ``Trae salvación'', o ``es
  victorioso''.} es humilde, viene montado sobre un asno---en realidad
sobre un potro, que es la cría de un asno. \footnote{\textbf{9:9} Sof
  3,14; Mat 21,5} \bibleverse{10} (Yo destruiré los carruajes de Efraín
y los caballos e Guerra de Jerusalén. Destruiré los arcos que usaron en
batalla). Él proclamará paz a las naciones, y gobernará de mar a mar,
desde el río Éufrates hasta los confines de la tierra. \footnote{\textbf{9:10}
  Miq 5,9}

\hypertarget{liberaciuxf3n-y-regreso-a-casa-de-los-juduxedos-capturados-su-victoria-y-prosperidad}{%
\subsection{Liberación y regreso a casa de los judíos capturados, su
victoria y
prosperidad}\label{liberaciuxf3n-y-regreso-a-casa-de-los-juduxedos-capturados-su-victoria-y-prosperidad}}

\bibleverse{11} Y en cuanto a ti,\footnote{\textbf{9:11} Refiriéndose
  nuevamente al pueblo de Jerusalén, como en el versículo 9:9.} porque
mi acuerdo\footnote{\textbf{9:11} Literalmente, ``pacto''.} contigo,
sellado con sangre, te liberaré del pozo seco.\footnote{\textbf{9:11}
  ``Pozo seco'': refiriéndose al exilio.} \footnote{\textbf{9:11} Éxod
  24,8} \bibleverse{12} ¡Vuelvan a los baluartes,\footnote{\textbf{9:12}
  Se entiende que habla de Jerusalén.} prisioneros con esperanza! Hoy
les prometo que les pagaré el doble de lo que han perdido.\footnote{\textbf{9:12}
  ``Lo que perdiste'' está implícito.} \footnote{\textbf{9:12} Is 61,7}
\bibleverse{13} Usaré a Judá como mi arco, y lo llenaré de Efraín como
mi flecha. Los llamaré a ustedes, hombres de Sión para que peleen contra
Grecia, empuñando la espada como un guerrero. \footnote{\textbf{9:13}
  Dan 8,21-22}

\bibleverse{14} ¡Entonces el Señor aparecerá sobre su pueblo y su flecha
resplandecerá como relámpago! El Señor Dios hará sonar la trompeta y
marchará como un vendaval que viene del sur. \bibleverse{15} El Señor
Todopoderoso los protegerá. Destruirán a sus enemigos y los conquistarán
con hondas. Ellos beberán y gritarán como borrachos. Estarán llenos como
una taza, empapados como las esquinas de un altar.\footnote{\textbf{9:15}
  Esto hace referencia al sistema de sacrificios donde se usaba una taza
  para recoger la sangre del sacrificio y luego esparcir la sangre en
  las esquinas de altar.} \bibleverse{16} Ese día el Señor su Dios los
salvará---su pueblo que son su rebaño---porque ellos resplandecen como
joyas de una corona en su tierra. \bibleverse{17} Cuán hermosos y
esplendorosos serán!\footnote{\textbf{9:17} Algunos creen que estas
  palabras se aplican al Señor, pero el contexto sugiere que se refieren
  a las joyas, al pueblo de Dios.} El grano y el nuevo vino darán fuerza
a los hombres jóvenes y harán florecer a las mujeres jóvenes.

\hypertarget{dios-solo-otorga-todas-las-bendiciones-los-uxeddolos-y-los-profetas-mentirosos-solo-crean-dauxf1o}{%
\subsection{Dios solo otorga todas las bendiciones; los ídolos y los
profetas mentirosos solo crean
daño}\label{dios-solo-otorga-todas-las-bendiciones-los-uxeddolos-y-los-profetas-mentirosos-solo-crean-dauxf1o}}

\hypertarget{section-9}{%
\section{10}\label{section-9}}

\bibleverse{1} Pídele al Señor la lluvia en primavera, porque él es el
que forma las nubes de lluvia y las hace enviar lluvia para hacer crecer
las cosechas de todos. \bibleverse{2} Los ídolos de la casa no dan
ninguna respuesta, los adivinos miente, y los intérpretes de sueños
inventan falsas esperanzas. En consecuencia, el pueblo anda sin rumbo,
como ovejas extraviadas, porque no hay pastor.

\hypertarget{dios-despierta-a-luxedderes-heroicos-a-batallas-victoriosas-para-su-pueblo-a-quien-ahora-se-le-ha-confiado-el-sombrero-de-malos-pastores}{%
\subsection{Dios despierta a líderes heroicos a batallas victoriosas
para su pueblo, a quien ahora se le ha confiado el sombrero de malos
pastores}\label{dios-despierta-a-luxedderes-heroicos-a-batallas-victoriosas-para-su-pueblo-a-quien-ahora-se-le-ha-confiado-el-sombrero-de-malos-pastores}}

\bibleverse{3} Estoy enojado con los pastores, y castigaré a los
líderes.\footnote{\textbf{10:3} El hablante es el Señor.} Porque el
Señor Todopoderoso cuida de su rebaño, del pueblo de Judá, y los
convertirá en su caballo de guerra más valioso. \footnote{\textbf{10:3}
  Zac 11,5} \bibleverse{4} Del pueblo de Judá saldrá la piedra angular,
la estaca de la tienda, el arco usado en batalla, y todos sus líderes
juntos.\footnote{\textbf{10:4} En otras palabras, el pueblo de Judá no
  quedará sometido a gobernantes extranjeros.} \footnote{\textbf{10:4}
  Jer 30,21} \bibleverse{5} Ellos serán como guerreros que irán a la
batalla, tendiendo trampas a sus enemigos en el lodo. Porque el Señor
está con ellos, ellos vencerán al enemigo que viene a caballo.
\bibleverse{6} Yo fortaleceré al pueblo de Judá. Salvaré al pueblo de
José. Los traeré de vuelta a casa porque cuido de ellos. Será como si
nunca los hubiera rechazado, porque yo soy el Señor su Dios y atenderé
sus clamores de ayuda.\footnote{\textbf{10:6} ``Sus clamores de ayuda''
  está implícito.} \bibleverse{7} El pueblo de Efraín se convertirá en
un pueblo de guerreros, y estarán felices como si hubieran bebido vino.
Sus hijos verán lo que sucede y también se alegrarán, gozosos en el
Señor.

\hypertarget{la-gente-esparcida-por-todo-el-mundo-volveruxe1-purificada-especialmente-de-egipto-y-asiria-y-volveruxe1-a-ser-un-pueblo-fuerte-de-dios}{%
\subsection{La gente esparcida por todo el mundo volverá purificada,
especialmente de Egipto y Asiria, y volverá a ser un pueblo fuerte de
Dios}\label{la-gente-esparcida-por-todo-el-mundo-volveruxe1-purificada-especialmente-de-egipto-y-asiria-y-volveruxe1-a-ser-un-pueblo-fuerte-de-dios}}

\bibleverse{8} Los llamaré con sonido de mis labios y vendrán corriendo
hacia mi. Yo los rescataré, y habrá muchos como lo eran
antes.\footnote{\textbf{10:8} Literalmente, ``Se multiplicarán como se
  han multiplicado''.} \bibleverse{9} Los he dispersado como semillas en
medio de las naciones, y desde lugares lejanos se acordarán de mi.
Traerán a sus hijos y regresarán juntos. \footnote{\textbf{10:9} Is
  66,19}

\bibleverse{10} Los traeré de regreso desde la tierra de Egipto, y los
reuniré desde Asiria. Los llevaré a Galaad y al Líbano, y no habrá
espacio para todos ellos. \bibleverse{11} Pasarán a través del mar de la
angustia y golpearán las olas del mar, y las aguas del Nilo se
secarán.\footnote{\textbf{10:11} Esta es una clara referencia al Éxodo,
  incluso al momento en que Moisés golpea la roca para obtener agua.} El
orgullo de Asiria quedará destruido, y Egitpto perderá su poder.
\bibleverse{12} Yo los haré fuertes en el Señor, y ellos seguirán todo
lo que él diga, declara el Señor.

\hypertarget{el-terrible-colapso-de-la-potencia-mundial-hostil}{%
\subsection{El terrible colapso de la potencia mundial
hostil}\label{el-terrible-colapso-de-la-potencia-mundial-hostil}}

\hypertarget{section-10}{%
\section{11}\label{section-10}}

\bibleverse{1} ¡Abre tus puertas, Líbano, para que el fuego pueda
consumir tus cedros! \bibleverse{2} Llora, enebro, porque el cedro ha
caído. Los majestosos árboles están destruidos! ¡Lloren, robles de
Basán, porque el espeso bosque ha sido talado! \bibleverse{3} Escuchen a
los aullidos de los pastores, porque sus pastizales\footnote{\textbf{11:3}
  Literalmente, ``gloria''. El paralelo con la segunda línea del
  versículo indica algo en el mundo natural.} están destruidos. Escuchen
los rugidos de los leoncillos, porque la selva\footnote{\textbf{11:3}
  ``la selva del río Jordán'': Literalmente, ``la majestad del Jordán''.}
del río Jordán ha sido destruida.

\hypertarget{el-llamado-del-profeta-a-pastorear-a-la-gente-infeliz}{%
\subsection{El llamado del profeta a pastorear a la gente
infeliz}\label{el-llamado-del-profeta-a-pastorear-a-la-gente-infeliz}}

\bibleverse{4} Esto es lo que el Señor mi Dios dice: Sé el pastor del
rebaño que está marcado para ser sacrificado. \bibleverse{5} Los que
compran las ovejas para matarlas no sienten culpa por ello; y los que
las venden dicen: ``¡Alabado sea el Señor! ¡Ahora soy rico!'' Ni aún sus
pastores se preocupan por ellos. \footnote{\textbf{11:5} Jer 23,-1; Ezeq
  13,-1; Ezeq 34,1-34} \bibleverse{6} Porque yo no me preocuparé más del
pueblo de la tierra, declara el Señor. Yo voy a convertirlos en víctimas
unos de otros, y del rey. Ellos destruirán la tierra y no salvarán a
ninguno.

\hypertarget{el-infructuoso-oficio-pastoral-del-profeta-su-deshonroso-rechazo-por-parte-de-los-dueuxf1os-del-rebauxf1o}{%
\subsection{El infructuoso oficio pastoral del profeta; su deshonroso
rechazo por parte de los dueños del
rebaño}\label{el-infructuoso-oficio-pastoral-del-profeta-su-deshonroso-rechazo-por-parte-de-los-dueuxf1os-del-rebauxf1o}}

\bibleverse{7} Yo me convertiré en pastor del rebaño que está listo para
ser sacrificado por los comerciantes de ovejas.\footnote{\textbf{11:7}
  El texto dice: ``comerciantes de ovejas'' en lugar de ``ovejas
  oprimidas''.} Entonces tomé dos varas, una llamada Gracia, y la otra
llamada Unión, y yo fui el pastor del rebaño. \bibleverse{8} En un mes
despedí a tres pastores. Mi paciencia con ellos se agotó,\footnote{\textbf{11:8}
  Los académicos están divididos en respecto a la referencia de
  ``ellos''. Algunos lo toman en el sentido de que son los tres
  pastores, otros dicen que son las ovejas, y otros piensan que se
  refiere a los comerciantes de ovejas.} y ellos también me odiaron.
\bibleverse{9} Entonces yo dije: ``No seré su pastor.\footnote{\textbf{11:9}
  Claramente Zacarías ahora está hablando con los comerciantes de ovejas
  para quienes trabajaba como pastor (11: 7)} Si las ovejas mueren,
mueren. Que los que vayan a perecer, perezcan. ¡Que los que queden se
coman unos con otros!'' \footnote{\textbf{11:9} Jer 15,2}
\bibleverse{10} Entonces tomé mi vara llamada Gracia y la rompí,
quebrantando el acuerdo que había hecho con todos los
pueblos.\footnote{\textbf{11:10} Ya que no hay registro de un acuerdo, o
  ``pacto'', con ninguna otra nación, se presume que la palabra
  ``pueblos'' aquí se refiere a los Israelitas.} \bibleverse{11} Fue
quebrantado ese día, y los mercaderes de ovejas que me miraban sabían
que era un mensaje del Señor. \bibleverse{12} Yo les dije: ``Si quieren
pagarme, háganlo. Si no, no lo hagan''. Así que me pagaron: Treinta
piezas de plata. \footnote{\textbf{11:12} Mat 26,15} \bibleverse{13} Y
el Señor me dijo: ``Echa el dinero en la tesorería'',\footnote{\textbf{11:13}
  O ``en el bote''. Sin embargo, como se menciona ``el Templo del
  Señor'' más adelante, este parece ser el sentido más probable.} esa
miserable suma que pensaron que pagaba mi precio. Así que tomé las
treinta piezas de plata y las lanzó en la tesorería del Templo del
Señor. \footnote{\textbf{11:13} Mat 27,9-10} \bibleverse{14} Entonces
rompí mi segunda vara llamada Unión, rompiendo así la unión familiar
entre Judá e Israel. \footnote{\textbf{11:14} Ezeq 37,22}

\hypertarget{otro-llamado-del-profeta-al-oficio-pastoral-y-amenazas-al-pastor-inuxfatil}{%
\subsection{Otro llamado del profeta al oficio pastoral y amenazas al
pastor
inútil}\label{otro-llamado-del-profeta-al-oficio-pastoral-y-amenazas-al-pastor-inuxfatil}}

\bibleverse{15} Y el Señor me dijo: Toma tus implementos de pastor, y sé
como un pastor irresponsable. \bibleverse{16} Porque yo pondré un pastor
a cargo al que no le importarán los que estén muriendo, ni buscará a los
perdidos,\footnote{\textbf{11:16} O ``los jóvenes''.} ni sanará a los
heridos, ni alimentará a las ovejas sanas. Por el contrario, comerá la
carne de las ovejas gordas. Incluso les arrancará las
pezuñas.\footnote{\textbf{11:16} Quizás ilustrando el grado de crueldad
  hacia los animales.} \bibleverse{17} ¡Grande es el desastre que vendrá
sobre este pastor inútil que abandona al rebaño! La espada golpeará su
brazo y su ojo derecho. Su brazo se secará y su ojo derecho quedará
ciego.

\hypertarget{embestida-de-los-gentiles-en-jerusaluxe9n-salvaciuxf3n-de-la-ciudad-por-dios-y-por-el-muxe9rito-de-juduxe1}{%
\subsection{Embestida de los gentiles en Jerusalén; Salvación de la
ciudad por Dios y por el mérito de
Judá}\label{embestida-de-los-gentiles-en-jerusaluxe9n-salvaciuxf3n-de-la-ciudad-por-dios-y-por-el-muxe9rito-de-juduxe1}}

\hypertarget{section-11}{%
\section{12}\label{section-11}}

\bibleverse{1} Una profecía:\footnote{\textbf{12:1} Literalmente,
  ``carga''.} Este mensaje vino del Señor, respecto a Israel, una
declaración del Señor que extendió los cielos, y quien estableció los
cimientos de la tierra y puso aliento de vida\footnote{\textbf{12:1}
  ``Aliento de vida'': o ``espíritu''.} en os seres humanos.
\bibleverse{2} ¡Miren! Yo haré de Jerusalén una copa con bebida
alcohólica que hará tambalear a todas las como borrachos cuando se
acerquen a atacar a Judá y a Jerusalén. \bibleverse{3} Ese día haré que
Jerusalén sea como una roca pesada para todas las personas. Y cualquiera
que trate de levantar la roca quedará muy lastimado. Todas las naciones
se unirán entonces para atacar a Jerusalén. \footnote{\textbf{12:3} Zac
  14,2; Jl 4,12} \bibleverse{4} Ese día haré que los caballos se
atemoricen y que los jinetes se vuelvan locos, declara el Señor, pero yo
cuidaré del pueblo de Judá mientas dejo ciegos a los caballos de sus
enemigos. \bibleverse{5} Entonces las familias de Judá se dirán para sí
mismos: el pueblo de Jerusalén es fuerte gracias a su Dios, el Señor
Todopoderoso.

\bibleverse{6} Ese día haré que las familias de Judá sean como carbones
encendidos en el bosque, o como antorchas ardientes en un campo de pasto
seco. Ellos destruirán con fuego todo lo que encuentren a su paso a
diestra y siniestra, a todos los pueblos vecinos, mientras que el pueblo
de Jerusalén estará seguro en su ciudad.

\bibleverse{7} El Señor le dará la victoria primero a los
soldados\footnote{\textbf{12:7} Literalmente, ``tiendas''.} de Judá,
para que la Gloria de la casa de David y la Gloria de los habitantes de
Jerusalén no sea mayor que la de Judá. \bibleverse{8} Ese día el Señor
pondrá un escudo protector alrededor del pueblo de Jerusalén para que
hasta el más torpe sea un guerrero hábil como David, y la casa de Davod
será como Dios, como el ángel del Señor que los guía. \footnote{\textbf{12:8}
  Is 33,24}

\hypertarget{derramamiento-del-espuxedritu-sobre-jerusaluxe9n-gran-lamento-del-pueblo-por-un-acto-de-sangre-cometido}{%
\subsection{Derramamiento del Espíritu sobre Jerusalén; gran lamento del
pueblo por un acto de sangre
cometido}\label{derramamiento-del-espuxedritu-sobre-jerusaluxe9n-gran-lamento-del-pueblo-por-un-acto-de-sangre-cometido}}

\bibleverse{9} Ese día comenzaré a destruir a todas las naciones que
atacan a Jerusalén. \footnote{\textbf{12:9} Apoc 20,9}

\bibleverse{10} Yo enviaré un espíritu de gracia y oración en la casa de
David y sobe los habitantes de Jerusalén. Ellos mirarán al que han
atravesado, y se lamentarán sobre él, como quien guarda luto por su
único hijo, llorando amargamente por su romogénito. \footnote{\textbf{12:10}
  Jl 3,1; Juan 19,37; Apoc 1,7} \bibleverse{11} Ese día el lamento de
Jerusalén será tan grande como el lamento en Hadad Rimón en el Valle de
Meguido.\footnote{\textbf{12:11} Algunos han vinculado esta referencia
  al luto por el ultimo ``buen'' rey de Judá después de su muerte en la
  batalla de Meguido.} \footnote{\textbf{12:11} 2Cró 35,22-25}

\bibleverse{12} La tierra lamentará, cada familia por separado: la casa
de la familia de David sola y sus mujeres, así como las familias de
Natán, \bibleverse{13} Leví, y Simeí, \bibleverse{14} además las
familias sobrevivientes y sus mujeres, cada grupo llorando amargamente,
todos por separado.

\hypertarget{el-pecado-divino-del-pueblo-eliminaciuxf3n-de-la-idolatruxeda-la-falsa-profecuxeda-y-toda-inmundicia}{%
\subsection{El pecado divino del pueblo; Eliminación de la idolatría, la
falsa profecía y toda
inmundicia}\label{el-pecado-divino-del-pueblo-eliminaciuxf3n-de-la-idolatruxeda-la-falsa-profecuxeda-y-toda-inmundicia}}

\hypertarget{section-12}{%
\section{13}\label{section-12}}

\bibleverse{1} Ese día se abrirá una fuente que brotará\footnote{\textbf{13:1}
  El verbo indica que no es una sola acción, sino que tiene resultados
  continuos.} continuamenteporque la casa de David y el pueblo de
Jerusalén para limpiar su pecado e impureza. \footnote{\textbf{13:1} Is
  12,3; Is 55,1}

\bibleverse{2} Ese día, declara el Señor Todopoderoso, elminaré la
idolatría de la tierra, y no habrá nunca más memoria de los ídolos. Yo
quitaré a los falsos profetas y al espíritu de impureza de la tierra.
\footnote{\textbf{13:2} Miq 5,12} \bibleverse{3} Si alguno sigue
profetizando, su padre o madre le dirán: ``No vivirás, porque has
engañado en nombre del Señor''. Entonces sus padres lo matarán, porque
ha profetizado. \footnote{\textbf{13:3} Deut 13,6} \bibleverse{4} Ese
día, tales profetas sentirán vergüenza de profetizar sus supuestas
visiones. Pra engañar no se pondrán más sus vestiduras de profetas,
hechas de pelo áspero.\footnote{\textbf{13:4} En otras palabras, actúan
  con engaño para continuar con sus prácticas.} \footnote{\textbf{13:4}
  2Re 1,8} \bibleverse{5} Dirán entonces: ``No soy profeta, soy un
granjero. He labrado la tierra desde que era pequeño''. \bibleverse{6} Y
si alguien le pregunta: ``¿Cuál es el motivo de esas heridas en tu
espalda?''\footnote{\textbf{13:6} En hebreo: ``manos''. La razón por la
  que hacían esta pregunta era porque los adoradores paganos a menudo
  practicaban la auto mutilación.} él responderá: ``Fui herido en la
casa de un amigo''.

\hypertarget{el-tribunal-de-purificaciuxf3n}{%
\subsection{El tribunal de
purificación}\label{el-tribunal-de-purificaciuxf3n}}

\bibleverse{7} ¡Levántate, espada mía! ¡Ataca a mi pastor, al hombre que
ha estado junto a mi! declara el Señor. Golpea al pastor y las ovejas
serán dispersas, y yo levantaré mi mano contra los corderos. \footnote{\textbf{13:7}
  Mat 26,31} \bibleverse{8} Dos tercios de los habitants de la tierra
serán destruids, y solo un tercio quedará, dice el Señor. \footnote{\textbf{13:8}
  Is 6,13} \bibleverse{9} Yo pondré este tercio en el fuego, y lo
refinaré como la plata, los probaré como se prueba al oro. Ellos
clamarán por mi ayua, y yo les responderé. Diré: ``Este es mi pueblo'',
y ellos dirán: ``El Señor es mi Dios''.\footnote{\textbf{13:9} Os 2,25}

\hypertarget{el-duxeda-del-seuxf1or-lucha-dificultades-y-salvaciuxf3n-en-jerusaluxe9n}{%
\subsection{El día del Señor: lucha, dificultades y salvación en
Jerusalén}\label{el-duxeda-del-seuxf1or-lucha-dificultades-y-salvaciuxf3n-en-jerusaluxe9n}}

\hypertarget{section-13}{%
\section{14}\label{section-13}}

\bibleverse{1} ¡Cuidado! Porque viene el día del Señor en el cual lo que
te ha sido saqueado será repartido delante de tus ojos. \footnote{\textbf{14:1}
  Is 39,6} \bibleverse{2} Yo reuniré a todas las naciones para que
ataquen a Jerusalén. La ciudad será capturada, las casas saqueadas, y
las mujeres serán violadas. La mitad de la población será llevada en
exilio, pero el resto del pueblo no será quitado de la ciudad.
\footnote{\textbf{14:2} Zac 12,3} \bibleverse{3} Entonces el Señor
saldrá a pelear contra las naciones, como se pelea en tiempos de guerra.
\footnote{\textbf{14:3} Apoc 19,19} \bibleverse{4} Ese día, sus pies
estarán sobre el monte de los olivos, que da la cara a Jerusalén, hacia
el Este. El Monte de los Olivos se partirá en dos, la mitad hacia el
norte, y la otra mitad hacia el sur, creando un valle amplio de Este a
Oeste. \bibleverse{5} Huirás de esta montaña, por el valle que se
extiende hasta Azal.\footnote{\textbf{14:5} Si este es el nombre de un
  lugar, su ubicación es desconocida.} Huirás como lo hicieron en
tiempos del terremoto durante el reinado de Uzías, rey de Judá. Entonces
el Señor vendrá, acompañado de todos sus santos. \footnote{\textbf{14:5}
  Am 1,1}

\hypertarget{los-maravillosos-procesos-de-la-naturaleza-en-el-duxeda-del-seuxf1or}{%
\subsection{Los maravillosos procesos de la naturaleza en el día del
Señor}\label{los-maravillosos-procesos-de-la-naturaleza-en-el-duxeda-del-seuxf1or}}

\bibleverse{6} Ese día no habrá frío ni heladas.\footnote{\textbf{14:6}
  Aún se debaten las palabras usadas aquí y su significado.}
\bibleverse{7} Será un día continuo (solo el Señor sabe cómo esto puede
ocurrir). No habrá ni día ni noche, porque aún en la noche habrá luz.

\bibleverse{8} Ese día saldrá agua viva de Jerusalén, y la mitad irá al
Este, hacia el Mar Muerto, y la mitad irá al Oeste, al mar Mediterráneo,
fluyendo en verano e invierno por igual. \footnote{\textbf{14:8} Ezeq
  47,1-8}

\bibleverse{9} El Señor será el rey sobre toda la tierra. Ese día habrá
un verdadero Señor, y su nombre será el único. \footnote{\textbf{14:9}
  Sal 97,1; Apoc 11,15}

\bibleverse{10} Toda la tierra será transformada en un valle, desde
Gueba hasta Rimón, al sur de Jerusalén.\footnote{\textbf{14:10} Esto
  indica toda la tierra de Judá.} Pero Jerusalén será reconstruida, y
será habitada desde la puerta de Benjamín, hasta donde estaba la Puerta
Antigua, es decir, la Puerta de la Esquina, y desde la Torre de Jananel,
hasta las bodegas del vino del rey.\footnote{\textbf{14:10} Esto
  incluiría toda la ciudad antigua de Jerusalén.} \footnote{\textbf{14:10}
  Jer 31,38} \bibleverse{11} La ciudad será habitada y nunca más
condenada a la destrucción. El pueblo podrá vivir seguro en Jerusalén.
\footnote{\textbf{14:11} Apoc 22,3; Jer 33,16}

\hypertarget{el-juicio-sobre-los-pueblos-que-hicieron-la-guerra-a-jerusaluxe9n}{%
\subsection{El juicio sobre los pueblos que hicieron la guerra a
Jerusalén}\label{el-juicio-sobre-los-pueblos-que-hicieron-la-guerra-a-jerusaluxe9n}}

\bibleverse{12} Esta será la plaga que el Señor usará para azotar a las
acciones que atacaron a Jerusalén: Su carne se pudrirá mientras aún
están en pie. Sus ojos se pudrirán en sus cuencas, y sus lenguas se
pudrirán en sus bocas. \bibleverse{13} Ese día el Señor los golpeará con
un terrible pánico, y se conquistarán y lucharán ente ellos mismos, mano
a mano. \bibleverse{14} Hasta Judá peleará en\footnote{\textbf{14:14}
  Texto tomado de la Septuaginta. En el hebreo dice ``contra''.}
Jerusalén. La riqueza de las naciones vecinas será tomada: Botines de
oro, plata y prendas de vestir.

\bibleverse{15} Una plaga similar azotará a los caballos, mulas,
camellos, asnos y a todos los demás animales de sus campos.

\hypertarget{todos-los-pueblos-deben-adorar-al-seuxf1or-en-jerusaluxe9n}{%
\subsection{Todos los pueblos deben adorar al Señor en
Jerusalén}\label{todos-los-pueblos-deben-adorar-al-seuxf1or-en-jerusaluxe9n}}

\bibleverse{16} Después de esto, cada sobreviviente de las naciones que
atacaron a Jerusalén irán allí a adorar al Rey, al Señor Todopoderoso, y
a celebrar la Fiesta de las Enramadas. \footnote{\textbf{14:16} Zac 14,9}
\bibleverse{17} Si alguno de los pueblos del mundo se niega a ir a
Jerusalén a adorar al Rey, el Señor, Todopoderoso, la lluvia cesará
\bibleverse{18} Si el pueblo de Egipto se niega a ir, entonces el Señor
enviará sobre ellos la misma plaga que a las otras naciones que no
fueron a celebrar la Fiesta de las Enramadas. \bibleverse{19} Este será
el castigo sobre Egipto y sobre todas las naciones que no vayan a
Jerusalén a celebrar.

\hypertarget{en-juduxe1-y-jerusaluxe9n-incluso-los-objetos-de-uso-muxe1s-comunes-seruxe1n-santificados}{%
\subsection{En Judá y Jerusalén, incluso los objetos de uso más comunes
serán
santificados}\label{en-juduxe1-y-jerusaluxe9n-incluso-los-objetos-de-uso-muxe1s-comunes-seruxe1n-santificados}}

\bibleverse{20} Ese día, en los cencerros de los caballos estarán
escritas las palabras ``Santo es el Señor''. Las ollas de la casa usadas
en el Templo del Señor serán tan santos como las ollas usadas en el
altar en la presencia del Señor. \bibleverse{21} Cada olla en Jerusalén
y en Judá será santa para el Señor Todopoderoso, a fin de que todos los
que vengan a hacer sacrificios las tomen y cocinen en ellas las carnes
de sus sacrificios.\footnote{\textbf{14:21} Estos versículos sugieren
  que tantas personas vendrían a adorar al Señor en Jerusalén, que se
  necesitaría cada utensilio de cocina para el sistema de sacrificios.}
Ese día no habrá más comerciantes en el Templo del Señor.
