\hypertarget{la-inutilidad-de-todo-esfuerzo-humano-como-resultado-de-la-constante-monotonuxeda-en-el-ciclo-de-las-cosas.}{%
\subsection{La inutilidad de todo esfuerzo humano como resultado de la
constante monotonía en el ciclo de las
cosas.}\label{la-inutilidad-de-todo-esfuerzo-humano-como-resultado-de-la-constante-monotonuxeda-en-el-ciclo-de-las-cosas.}}

\hypertarget{section}{%
\section{1}\label{section}}

\bibleverse{1} He aquí las palabras del Maestro, el rey de Jerusalén,
hijo de David.

\bibleverse{2} ``¡Todo pasa, todo es tan temporal! Todo es tan difícil
de entender'', dice el Maestro.\footnote{\textbf{1:2} La palabra
  utilizada aquí (cinco veces, y se repite con frecuencia en el libro)
  no significa realmente ``sin sentido'', como se traduce tan a menudo.
  Su significado básico es ``vapor'' o ``aliento'', y se asocia con todo
  lo que es transitorio y fugaz. ``Transitorio'' o ``efímero'' también
  reflejarían el significado: no es que no tenga valor, sino que todo
  pasa muy rápido. Nada dura. La brevedad de la vida es lo que ``no
  tiene sentido'' y provoca una incertidumbre frustrante. La brevedad y
  la naturaleza insustancial de la existencia es lo que al Maestro le
  cuesta entender. Es ``fugaz''.} \bibleverse{3} ¿Qué
provecho\footnote{\textbf{1:3} Esta es otra palabra que se utiliza en un
  sentido especial en el Eclesiastés. Su significado primario es
  ``ganancia'' o ``beneficio'' en un sentido comercial, pero aquí se
  está utilizando más en el sentido de ``beneficio en la vida'' -en
  otras palabras, ¿qué ventaja se obtiene en el sentido de ``el
  significado de la vida'' y cualquier recompensa futura?} obtienesde
trabajar como un esclavo en esta vida?\footnote{\textbf{1:3}
  Literalmente, ``bajo el sol''.} \bibleverse{4} La gente viene y va,
pero la tierra es eterna. \footnote{\textbf{1:4} Sal 90,3}
\bibleverse{5} El sol sale, y el sol se pone, y luego vuelve corriendo a
su lugar para salir de nuevo. \bibleverse{6} El viento sopla hacia el
sur y luego gira hacia el norte. Gira y gira, y finalmente da una vuelta
completa. \bibleverse{7} Todos los arroyos desembocan en el mar, pero el
mar nunca se llena. Los arroyos vuelven al lugar de donde vinieron.
\bibleverse{8} Todo sigue su curso. No puedes decir todo lo que hay por
decir. No puedes ver todo lo que hay para ver. No se puede oír todo lo
que hay para oír.\footnote{\textbf{1:8} Literalmente, ``el hombre no es
  capaz de decir, el ojo no está satisfecho de ver, el oído no está
  saciado de oír''.} \bibleverse{9} Todo lo que fue seguirá siendo; todo
lo que se ha hecho se volverá a hacer. Aquí nunca ocurre nada
nuevo.\footnote{\textbf{1:9} ``Aquí'': Literalmente, ``bajo el sol''.}
\bibleverse{10} No hay nada que pueda señalar y decir: ``¡Mira! Aquí hay
algo nuevo''. De hecho, existe desde hace mucho tiempo, mucho antes de
nuestra época. \bibleverse{11} El problema es que\footnote{\textbf{1:11}
  ``El problema'': implícito.} no recordamos a la gente del pasado, y la
gente en el futuro no recordará a los que vinieron antes.

\hypertarget{la-inutilidad-de-luchar-por-la-sabiduruxeda-y-el-conocimiento-la-vida-humana-resulta-inuxfatil-para-el-espectador}{%
\subsection{La inutilidad de luchar por la sabiduría y el conocimiento;
La vida humana resulta inútil para el
espectador}\label{la-inutilidad-de-luchar-por-la-sabiduruxeda-y-el-conocimiento-la-vida-humana-resulta-inuxfatil-para-el-espectador}}

\bibleverse{12} Yo soy el Maestro, y fui rey sobre Israel, reinando
desde Jerusalén. \footnote{\textbf{1:12} Ecl 1,1} \bibleverse{13} Decidí
enfocar mi mente en explorar, empleando la sabiduría, todo lo que sucede
aquí en la tierra. ¡Esta es una tarea difícil que Dios le ha dado a la
gente para mantenerla ocupada! \bibleverse{14} Examiné todo lo que la
gente hace aquí en la tierra, y descubrí que todo es tan temporal;
¡tratar de entenderlo es como tratar de sujetar el viento!\footnote{\textbf{1:14}
  ``Viento''. Hay un problema de traducción, ya que la misma palabra se
  utiliza en este libro para ``viento'', ``aliento'' o ``espíritu''. Así
  que el proverbio ``persiguiendo el viento'' podría significar
  efectivamente ``persiguiendo el aliento/espíritu'', lo que podría
  interpretarse como la búsqueda del sentido de la vida
  (aliento/espíritu). Por eso la versión Reina-Valera traduce la frase
  como ``aflicción de espíritu''.} \bibleverse{15} No se puede enderezar
lo que está torcido, y no se puede contar lo que no existe.\footnote{\textbf{1:15}
  Probablemente se trata de proverbios cotidianos de la época. En
  realidad dicen que hay que aceptar las cosas como son.}

\hypertarget{la-buxfasqueda-de-un-conocimiento-claro-conduce-a-la-decepciuxf3n}{%
\subsection{La búsqueda de un conocimiento claro conduce a la
decepción}\label{la-buxfasqueda-de-un-conocimiento-claro-conduce-a-la-decepciuxf3n}}

\bibleverse{16} Pensé para mí: ``Me he vuelto muy sabio, más sabio que
todos los reyes de Jerusalén que me precedieron. Mi mente ha adquirido
mucha sabiduría y conocimiento''. \bibleverse{17} Así que decidí usar mi
mente para aprender todo sobre la sabiduría, y también sobre la locura y
la necedad. Pero descubrí que esto es tan difícil como tratar de agarrar
el viento. \bibleverse{18} Porque una gran sabiduría conlleva una gran
frustración. Cuanto mayor es el conocimiento, más grande es la
afflicción.

\hypertarget{la-inutilidad-de-intentar-obtener-satisfacciuxf3n-mediante-los-placeres-sensuales-y-el-disfrute-de-la-vida-o-mediante-la-actividad-creativa}{%
\subsection{La inutilidad de intentar obtener satisfacción mediante los
placeres sensuales y el disfrute de la vida o mediante la actividad
creativa}\label{la-inutilidad-de-intentar-obtener-satisfacciuxf3n-mediante-los-placeres-sensuales-y-el-disfrute-de-la-vida-o-mediante-la-actividad-creativa}}

\hypertarget{section-1}{%
\section{2}\label{section-1}}

\bibleverse{1} Así que me dije: ``Muy bien, déjame probar el placer y
ver lo bueno que es''. Pero esto también resultó ser algo temporal y
pasajero. \bibleverse{2} Llegué a la conclusión de que reírse en la vida
es una estupidez, y el placer, ¿de qué sirve?\footnote{\textbf{2:2} El
  Maestro no está diciendo que no hay que reírse. Se refiere a las
  personas que se ríen de todo, que no se toman la vida en serio.}

\bibleverse{3} Entonces usé mi mente para examinar la
atracción\footnote{\textbf{2:3} La palabra utilizada significa
  ``arrastrar'' o ``atraer''.} del vino en mi cuerpo. Como mente aún me
guiaba con sabiduría, lo tomé hasta que actué como un
insensato,\footnote{\textbf{2:3} Algunos creen que esto significa que el
  Maestro se embriagó en este ``experimento''. La frase es literalmente:
  ``Me aferré (a ella) hasta la locura''.} para ver si esto era bueno
para la gente durante su tiempo aquí. \footnote{\textbf{2:3} Prov 31,4}
\bibleverse{4} Entonces intenté grandes proyectos de construcción.
Construí casas para mí; planté viñedos para mí. \bibleverse{5} Hice para
mí\footnote{\textbf{2:5} La repetición de la palabra ``para mí'' puede
  parecer redundante, pero el hecho de que el Maestro estuviera pensando
  principalmente en sí mismo es seguramente significativo.} jardines y
parques, plantándolos con toda clase de árboles frutales. \bibleverse{6}
Construí para mí embalses para regar todos estos árboles en crecimiento.
\bibleverse{7} Compré esclavos y esclavas, y sus hijos también me
pertenecían. También poseí muchos rebaños y manadas, más que nadie en
Jerusalén antes de mí. \bibleverse{8} Recogí para mí grandes cantidades
de plata y oro, que me pagaban como tributo los reyes y las provincias.
Traje para mí cantantes masculinos y femeninos, y disfruté de muchas
concubinas\ldots{}\footnote{\textbf{2:8} Esta palabra no aparece en
  ninguna otra parte de la Biblia, por lo que el significado se asume a
  partir del contexto.} ¡Todo lo que un hombre pudiera desear!
\bibleverse{9} Me hice grande, más grande que nadie en Jerusalén antes
que yo. Todo el tiempo mi sabiduría permaneció conmigo. \bibleverse{10}
No me detuve en probar todo lo que quería. Todo lo que me apetecía
disfrutar, lo hacía. Incluso disfruté de todo lo que había logrado, como
recompensa por todo mi trabajo. \bibleverse{11} Pero cuando pensaba en
lo mucho que había trabajado para conseguirlo, en todo lo que había
hecho, era tan efímero, tan significativo como alguien que intenta
atrapar el viento. Realmente no hay ningún beneficio duradero aquí en la
tierra. \footnote{\textbf{2:11} Ecl 1,14}

\hypertarget{al-final-la-sabiduruxeda-es-tan-vacuxeda-como-la-locura-porque-el-destino-final-de-los-sabios-y-los-necios-es-el-mismo}{%
\subsection{Al final, la sabiduría es tan vacía como la locura, porque
el destino final de los sabios y los necios es el
mismo}\label{al-final-la-sabiduruxeda-es-tan-vacuxeda-como-la-locura-porque-el-destino-final-de-los-sabios-y-los-necios-es-el-mismo}}

\bibleverse{12} Así que me puse a pensar en la sabiduría, en la locura y
en la insensatez. Porque, ¿qué puede hacer el que viene después del rey
que no se haya hecho ya? \footnote{\textbf{2:12} Ecl 1,17}
\bibleverse{13} Reconocí que la sabiduría es mejor que la locura, así
como la luz es mejor que las tinieblas. \bibleverse{14} Los sabios ven
hacia dónde van,\footnote{\textbf{2:14} Literalmente, ``El sabio tiene
  los ojos en la cabeza''.} pero los insensatos caminan en la oscuridad.
Pero también me di cuenta de que todos llegan al mismo final.
\footnote{\textbf{2:14} Prov 17,24} \bibleverse{15} Entonces me dije:
``Si voy a terminar igual que un insensato, ¿de qué sirve ser tan
sabio?'' . Y me dije: ``¡Esto también es difícil de entender!''.
\bibleverse{16} Nadie se acuerda del sabio ni del insensato por mucho
tiempo: en el futuro todo se olvidará. Tanto los sabios como los necios
mueren.

\bibleverse{17} Así que terminé sintiéndome asqueado\footnote{\textbf{2:17}
  ``Sintiéndose desagradado'': Literalmente ``odió''.} con la vida
porque todo lo que sucede aquí en la tierra es demasiado
angustioso.\footnote{\textbf{2:17} ``Angustioso'': la palabra utilizada
  aquí también significa mal, problema, perjuicio, miseria, etc.} Es tan
incomprensible\footnote{\textbf{2:17} ``Incomprensible'': Una vez más,
  el Maestro no dice que la vida no tenga sentido, sino que es difícil
  hallarle sentido.} como tratar de controlar el viento.

\hypertarget{referencia-al-mal-de-que-el-sabio-debe-dejar-el-beneficio-y-el-disfrute-de-su-laboriosa-labor-a-un-heredero-quizuxe1s-insensato}{%
\subsection{Referencia al mal de que el sabio debe dejar el beneficio y
el disfrute de su laboriosa labor a un heredero quizás
insensato}\label{referencia-al-mal-de-que-el-sabio-debe-dejar-el-beneficio-y-el-disfrute-de-su-laboriosa-labor-a-un-heredero-quizuxe1s-insensato}}

\bibleverse{18} Incluso acabé odiando lo que había conseguido aquí en la
tierra porque tengo que entregarlo a quien venga después de mí.
\footnote{\textbf{2:18} Ecl 2,21; Ecl 2,26; Sal 39,7} \bibleverse{19} ¿Y
quién sabe si será sabio o insensato? Sin embargo, gobernará sobre todo
lo que logré con mi sabiduría aquí en la tierra. ¡Eso es tan frustrante,
tan difícil de entender!\footnote{\textbf{2:19} Este es un buen ejemplo
  de lo que el Maestro está tratando de decir. No está diciendo que ``no
  tiene sentido'' dejar un legado, el problema es que no se puede saber
  cómo se va a utilizar, y que esto es difícil de aceptar.}

\bibleverse{20} Decidí rendirme, con mi mente desesperada por la
importancia de todos los logros de mi vida. \bibleverse{21} Porque se
puede trabajar con sabiduría, conocimiento y destreza, pero ¿quién se
beneficia? Alguien que no ha trabajado para ello. Eso es frustrante y
completamente injusto. \bibleverse{22} ¿Qué obtienes aquí en la tierra
por todo tu esfuerzo y preocupación? \bibleverse{23} Tu vida laboral
está llena de problemas y conflictos; incluso por la noche tus
pensamientos te quitan el sueño. Es difícil de comprender.

\hypertarget{entonces-lo-mejor-para-el-hombre-es-disfrutar-el-momento-en-la-medida-en-que-dios-se-lo-conceda.}{%
\subsection{Entonces lo mejor para el hombre es disfrutar el momento en
la medida en que Dios se lo
conceda.}\label{entonces-lo-mejor-para-el-hombre-es-disfrutar-el-momento-en-la-medida-en-que-dios-se-lo-conceda.}}

\bibleverse{24} Entonces, ¿qué es lo mejor que puedes hacer? Comer,
beber y disfrutar de tu trabajo, reconociendo como yo que estas cosas
nos son dadas por Dios, \bibleverse{25} pues ¿quién puede comer o
disfrutar de la vida sin él? \bibleverse{26} A los buenos, Dios les da
sabiduría, conocimiento y alegría. Pero al pecador Dios le da la tarea
de juntar y recolectar riquezas, sólo para entregarlas a alguien que
agrade a Dios. Esto también muestra lo efímera que es la vida, y lo
difícil que es entenderla, como tratar de entender cómo sopla el
viento.\footnote{\textbf{2:26} Prov 13,22; Prov 28,8}

\hypertarget{todo-tiene-su-tiempo}{%
\subsection{Todo tiene su tiempo}\label{todo-tiene-su-tiempo}}

\hypertarget{section-2}{%
\section{3}\label{section-2}}

\bibleverse{1} Todo tiene su propio tiempo. Hay una hora para todo lo
que sucede aquí:\footnote{\textbf{3:1} Esta es la observación del
  Maestro, no su instrucción. Por esta razón se utiliza el participio
  del verbo, en lugar del infinitivo, ya que el infinitivo podría
  sugerir que esto es lo que debería suceder, por ejemplo, ``un tiempo
  de matar'' (es decir, cuando sucede) en lugar de ``un tiempo para
  matar'' (cuando debería suceder).} \footnote{\textbf{3:1} Ecl 8,6}
\bibleverse{2} Un tiempo de nacer, y un tiempo de morir. Un tiempo de
sembrar, y un tiempo de cosechar. \bibleverse{3} Tiempo de matar, y
tiempo de curar. Tiempo de derribar, y tiempo de edificar.
\bibleverse{4} Tiempo de llorar, y tiempo de reír. Tiempo de llorar, y
tiempo de bailar. \bibleverse{5} Tiempo de lanzar piedras, y tiempo de
recogerlas.\footnote{\textbf{3:5} Esta mención de las piedras y su
  significado son objeto de debate. La tradición judía indica que es un
  eufemismo para hacer el amor.} Tiempo de abrazar, y tiempo de evitar
abrazar. \bibleverse{6} Tiempo de buscar, y tiempo de dejar de buscar.
Tiempo de guardar, y tiempo de botar. \bibleverse{7} Tiempo de romper, y
tiempo de reparar. Tiempo de callar, tiempo de hablar. \bibleverse{8}
Tiempo de amar, y tiempo de odiar. Tiempo de guerra, y tiempo de paz.

\hypertarget{pero-el-hombre-no-conoce-el-tiempo-establecido-por-dios-y-es-impotente-contra-uxe9l}{%
\subsection{Pero el hombre no conoce el tiempo establecido por Dios y es
impotente contra
él}\label{pero-el-hombre-no-conoce-el-tiempo-establecido-por-dios-y-es-impotente-contra-uxe9l}}

\bibleverse{9} ¿Y qué obtienes por todo tu esfuerzo? \bibleverse{10} He
examinado lo que Dios nos da por hacer. \bibleverse{11} Todo lo que Dios
hace está bellamente programado, y aunque también ha puesto la idea de
la eternidad\footnote{\textbf{3:11} ``Eternidad'': la palabra utilizada
  aquí tiene el significado de épocas pasadas y épocas futuras,
  ``continuación hasta el tiempo más lejano''.} en nuestras mentes, no
podemos entender completamente lo que Dios hace de principio a fin.
\footnote{\textbf{3:11} Ecl 8,17} \bibleverse{12} Llegué a la conclusión
de que no hay nada mejor que ser feliz y procurar lo bueno de la vida.
\footnote{\textbf{3:12} Ecl 2,24} \bibleverse{13} Además, todo el mundo
debe comer y beber y disfrutar de su trabajo: esto es un regalo de Dios
para nosotros. \bibleverse{14} También llegué a la conclusión de que
todo lo que Dios hace dura para siempre: no se le puede añadir ni quitar
nada. Dios actúa así para que la gente lo admire. \bibleverse{15} Lo que
fue, es; y lo que será, ha sido, y Dios examina todo el
tiempo.\footnote{\textbf{3:15} Literalmente, ``Dios busca lo que ha sido
  ahuyentado''. Esta frase ha sido interpretada de diversas maneras,
  pero quizás la mejor en el contexto es que el tiempo está abierto a
  Dios: las cosas olvidadas por los seres humanos (``ahuyentadas'')
  siguen siendo accesibles para él y son lo que él examina (``busca'').}

\hypertarget{en-el-mundo-humano-hay-maldad-e-injusticia-pero-dios-es-el-juez-mundial}{%
\subsection{En el mundo humano hay maldad e injusticia, pero Dios es el
juez
mundial}\label{en-el-mundo-humano-hay-maldad-e-injusticia-pero-dios-es-el-juez-mundial}}

\bibleverse{16} También observé que aquí en la tierra había maldad
incluso en el lugar donde se suponía que había justicia; incluso donde
las cosas debían ser correctas, había maldad. \bibleverse{17} Pero
entonces pensé para mí: ``En última instancia, Dios juzgará tanto a los
que hacen el bien como a los que hacen el mal, y a cada obra y acción,
en el momento señalado''. \footnote{\textbf{3:17} Ecl 12,14}

\hypertarget{la-ley-de-la-impermanencia-existe-tanto-para-los-humanos-como-para-los-animales-y-exhorta-a-disfrutar-de-la-vida}{%
\subsection{La ley de la impermanencia existe tanto para los humanos
como para los animales y exhorta a disfrutar de la
vida}\label{la-ley-de-la-impermanencia-existe-tanto-para-los-humanos-como-para-los-animales-y-exhorta-a-disfrutar-de-la-vida}}

\bibleverse{18} También pensé para mí: ``En cuanto a lo que ocurre con
los seres humanos, Dios nos demuestra que no somos mejores que los
animales''.\footnote{\textbf{3:18} Este pensamiento es una reacción a la
  constatación de que la maldad ocupa el lugar de la justicia,
  mencionada en el verso 3:16.} \bibleverse{19} Porque lo que ocurre con
los seres humanos es lo mismo que lo que ocurre con los animales: de la
misma manera que uno muere, el otro también muere. Todos tienen el
aliento de vida, así que en lo que respecta a cualquier ventaja que los
seres humanos tengan sobre los animales, no hay ninguna. Sin duda, esto
es muy difícil de entender!\footnote{\textbf{3:19} Además, dado que esto
  sigue a una discusión sobre el ``aliento'' (que también puede
  traducirse como ``espíritu''), esto puede vincularse con el tema
  continuo de la ``transitoriedad'' y ``difícil de entender'' que, como
  se mencionó antes, viene de la palabra para vapor o aliento.}
\bibleverse{20} Todos acaban en el mismo lugar: todos proceden del polvo
y todos vuelven al polvo. \footnote{\textbf{3:20} Gén 3,19}
\bibleverse{21} ¿Quién sabe realmente si el aliento de vida\footnote{\textbf{3:21}
  ``Aliento de vida'': o ``espíritu''.} de los seres humanos va hacia
arriba, y el aliento de vida de los animales baja a la tierra?

\bibleverse{22} Así que llegué a la conclusión de que no hay nada mejor
que la gente disfrute de su trabajo. Esto es lo que debemos hacer.
Porque ¿quién puede resucitar a alguien de entre los muertos para
mostrarle lo que sucederá después de su muerte?

\hypertarget{la-opresiuxf3n-los-celos-y-el-trabajo-en-parte-inquieto-la-calma-en-parte-lenta-devaluxfaan-la-vida}{%
\subsection{La opresión, los celos y el trabajo en parte inquieto, la
calma en parte lenta devalúan la
vida}\label{la-opresiuxf3n-los-celos-y-el-trabajo-en-parte-inquieto-la-calma-en-parte-lenta-devaluxfaan-la-vida}}

\hypertarget{section-3}{%
\section{4}\label{section-3}}

\bibleverse{1} Entonces me puse a pensar en todas las formas en que la
gente oprime a los demás aquí en la tierra. Miren las lágrimas de los
oprimidos: ¡no hay nadie que los consuele! Los poderosos los oprimen, ¡y
no hay nadie que los consuele! \bibleverse{2} Felicité a los que ya
estaban muertos, porque los muertos están mejor que los que aún viven y
son oprimidos.\footnote{\textbf{4:2} ``Y ser oprimido'': añadido, pero
  este es el contexto para la declaración del Maestro.} \bibleverse{3}
Pero lo mejor de todo son los que nunca han existido: no han visto las
cosas malas que la gente se hace entre sí aquí en la tierra. \footnote{\textbf{4:3}
  Ecl 6,3} \bibleverse{4} Observé que toda habilidad en el trabajo
proviene de la competencia con los demás. Una vez más, esto es difícil
de entender, como tratar de aferrarse al escurridizo viento.

\bibleverse{5} Los insensatos se cruzan de brazos y no hacen nada, así
que al final se agotan. \bibleverse{6} Es mejor ganar un poco\footnote{\textbf{4:6}
  Literalmente, ``un puñado''.} sin estrés que mucho con demasiado
estrés y persiguiendo el viento. \footnote{\textbf{4:6} Prov 15,16}

\hypertarget{el-esfuerzo-de-la-persona-soltera-es-inuxfatil-dos-trabajadores-que-trabajan-juntos-estuxe1n-mejor}{%
\subsection{El esfuerzo de la persona soltera es inútil; dos
trabajadores que trabajan juntos están
mejor}\label{el-esfuerzo-de-la-persona-soltera-es-inuxfatil-dos-trabajadores-que-trabajan-juntos-estuxe1n-mejor}}

\bibleverse{7} Luego me puse a considerar otra cosa aquí en la tierra
que también es frustrantemente difícil de entender. \footnote{\textbf{4:7}
  Ecl 2,12} \bibleverse{8} ¿Qué pasa con alguien que no tiene familia
que le ayude, ni hermano ni hijo, que trabaja todo el tiempo, pero no
está satisfecho con el dinero que gana? ``¿Para quién estoy
trabajando?'' , se pregunta. ``¿Por qué me impido disfrutar de la
vida?'' . Una situación así es difícil de explicar: ¡es un negocio
malvado!

\bibleverse{9} Dos son mejor que uno, pues pueden ayudarse mutuamente en
su trabajo. \bibleverse{10} Si uno de ellos se cae, el otro puede
ayudarlo a levantarse, pero qué triste es el que se cae y no tiene a
nadie que lo ayude a levantarse. \bibleverse{11} Del mismo modo, si dos
personas se acuestan juntas, se abrigan mutuamente, pero uno no puede
calentarse si está solo. \bibleverse{12} Alguien que lucha contra otra
persona puede ganar, pero no si lucha contra dos. Una cuerda hecha de
tres hilos no puede romperse rápidamente.

\hypertarget{comunicaciuxf3n-de-un-evento-histuxf3rico-que-confirma-la-observaciuxf3n-del-predicador-de-que-el-favor-popular-no-es-confiable}{%
\subsection{Comunicación de un evento histórico que confirma la
observación del predicador de que el favor popular no es
confiable}\label{comunicaciuxf3n-de-un-evento-histuxf3rico-que-confirma-la-observaciuxf3n-del-predicador-de-que-el-favor-popular-no-es-confiable}}

\bibleverse{13} Un joven pobre y sabio es mejor que un rey viejo y necio
que ya no acepta consejos. \bibleverse{14} Incluso puede salir de la
cárcel\footnote{\textbf{4:14} Tal vez para ser entendido como ``la
  prisión de las malas circunstancias''.} para reinar sobre su reino,
aunque haya nacido pobre. \footnote{\textbf{4:14} Gén 41,14}

\bibleverse{15} He observado que todos los que están aquí en la tierra
siguen al joven que ocupa su lugar.\footnote{\textbf{4:15} Probablemente
  el joven mencionado en 4:13 que sustituye al viejo rey.}
\bibleverse{16} Está rodeado de una multitud de seguidores, pero la
siguiente generación no está contenta con él. Esto también ilustra la
naturaleza pasajera de la vida cuyo significado es esquivo, como
perseguir el viento para comprender.

\hypertarget{recordatorio-de-tener-cuidado-al-realizar-deberes-religiosos-con-sacrificios-oraciuxf3n-y-votos}{%
\subsection{Recordatorio de tener cuidado al realizar deberes religiosos
(con sacrificios, oración y
votos)}\label{recordatorio-de-tener-cuidado-al-realizar-deberes-religiosos-con-sacrificios-oraciuxf3n-y-votos}}

\hypertarget{section-4}{%
\section{5}\label{section-4}}

\bibleverse{1} Ten cuidado\footnote{\textbf{5:1} Literalmente, ``guarda
  tus pasos''.} cuando entres en la casa de Dios. Es mejor escuchar y
responder en lugar de ofrecer sacrificios sin sentido.\footnote{\textbf{5:1}
  ``Sacrificios sin sentido'': Literalmente, ``los sacrificios de los
  necios''.} La gente que hace tal cosa ni siquiera sabe que está
actuando mal. \footnote{\textbf{5:1} Sant 1,19} \bibleverse{2} No te
precipites, y piensa antes de hablar con Dios. Porque Dios está en el
cielo y tú en la tierra, así que sé breve. \footnote{\textbf{5:2} Ecl
  10,14; Prov 10,19} \bibleverse{3} Cuando te preocupas demasiado,
tienes pesadillas; cuando hablas demasiado, dices tonterías. \footnote{\textbf{5:3}
  Deut 23,22} \bibleverse{4} Cuando haces un voto a Dios, con una
maldición sobre ti si no lo cumples,\footnote{\textbf{5:4} ``Con una
  maldición sobre ti si no lo cumples'', implícito. Con frecuencia los
  votos a Dios incluían una maldición por no cumplir el voto.} no te
demores en cumplirlo, pues no le agrada el comportamiento insensato. Así
que cumple lo que has prometido. \bibleverse{5} Más vale no hacer ningún
voto que hacer un voto que no se cumpla. \footnote{\textbf{5:5} Mal 2,7}
\bibleverse{6} No dejes que tu boca te haga pecar. Y no le digas al
sacerdote\footnote{\textbf{5:6} Literalmente, ``mensajero''. En
  Malaquías 2:7 se identifica a los sacerdotes como mensajeros. En
  cualquier caso, la situación es la de informar a alguien en el Templo
  que la promesa (voto) que hizo fue un error.} que tu voto fue un
error, pues ¿por qué enemistarte con Dios rompiendo tu
promesa?\footnote{\textbf{5:6} Muchas de estas promesas (votos) incluían
  una maldición autoproclamada si no se cumplía, de modo que al no
  cumplir la promesa, el individuo quedaba sujeto a la maldición que
  había pronunciado sobre sí mismo.} Podría hacer caer sobre ti tu
propia maldición. \bibleverse{7} Ciertamente hay muchos sueños,
preguntas sobre su significado y muchas ideas diferentes, pero tú debes
seguir a Dios.

\hypertarget{las-opresiones-en-el-estado-son-lamentables-pero-comprensibles-bendiciones-de-regaluxedas-para-los-estados-agruxedcolas}{%
\subsection{Las opresiones en el estado son lamentables, pero
comprensibles; Bendiciones de regalías para los estados
agrícolas}\label{las-opresiones-en-el-estado-son-lamentables-pero-comprensibles-bendiciones-de-regaluxedas-para-los-estados-agruxedcolas}}

\bibleverse{8} Si ves gente pobre en algún lugar\footnote{\textbf{5:8}
  ``Algún lugar'': Literalmente, ``en la provincia'' -un término
  general.} siendo oprimida, o la verdad y la justicia violada, no te
escandalices por ello, porque cada funcionario es responsable ante otros
más altos, y hay funcionarios por encima de ellos también.\footnote{\textbf{5:8}
  El significado hebreo no está claro. Probablemente está diciendo que
  la corrupción y la injusticia son endémicas.} \bibleverse{9} Sin
embargo, lo que la tierra produce es para todos, incluso el rey se
beneficia de lo que se cultiva.\footnote{\textbf{5:9} Una vez más, el
  significado de este verso es impreciso.}

\hypertarget{la-nulidad-y-las-quejas-de-las-riquezas}{%
\subsection{La nulidad y las quejas de las
riquezas}\label{la-nulidad-y-las-quejas-de-las-riquezas}}

\bibleverse{10} La gente que ama el dinero nunca tiene suficiente
dinero; los que aman la riqueza nunca tienen suficientes ingresos. Esto
también es muy temporal y tiene poco sentido. \bibleverse{11} Cuanto más
ganas, más gastos tienes. ¡Sólo pareciera que tuvieras más!

\bibleverse{12} Los que trabajan duro duermen bien, tengan poco o mucho
que comer, pero los ricos poseen tanto que no descansan.

\bibleverse{13} Aquí he observado algo que es realmente
enfermizo:\footnote{\textbf{5:13} Las dos palabras que suelen traducirse
  como ``mal grave'' se refieren en realidad a una ``enfermedad
  miserable'', o a algo que te pone ``realmente enfermo''. También en el
  versículo 16.} Las personas que acumulan dinero se perjudican a sí
mismas. \bibleverse{14} Ponen su dinero en malas inversiones y lo
pierden todo. Cuando nacemos, no traemos nada al mundo. \footnote{\textbf{5:14}
  Job 1,21; Sal 49,18} \bibleverse{15} Cuando morimos, nos vamos tan
desnudos como cuando nacimos, sin llevarnos nada de todo lo que hemos
trabajado. \bibleverse{16} ¡Esto también me enferma! ¿Qué gana la gente,
trabajando para el viento?\footnote{\textbf{5:16} Como se indica en la
  nota a pie de página de 1:14, la palabra aquí puede significar
  ``viento'', ``aliento'' o ``espíritu''. Así que podría traducirse
  fácilmente como ``¿Qué ganas trabajando por un soplo de aire?'' , etc.}
\bibleverse{17} Viven su vida en la oscuridad, muy frustrados, enfermos
y resentidos.

\hypertarget{recomendaciuxf3n-del-disfrute-de-la-vida-ademuxe1s-del-trabajo-y-la-riqueza}{%
\subsection{Recomendación del disfrute de la vida además del trabajo y
la
riqueza}\label{recomendaciuxf3n-del-disfrute-de-la-vida-ademuxe1s-del-trabajo-y-la-riqueza}}

\bibleverse{18} Pero yo observé que lo bueno y lo correcto es comer,
beber y hallar placer en el trabajo que Dios nos da aquí en esta vida.
Este es el regalo de Dios para nosotros. \bibleverse{19} Además, a todos
los que Dios les da riquezas y posesiones, les da también la capacidad
de disfrutar de estos dones, de estar agradecidos por lo que se les da y
de disfrutar del trabajo que hacen. Esto también es un regalo de Dios
para nosotros. \bibleverse{20} De hecho, estas personas tienen poco
tiempo para pensar en la vida porque Dios las mantiene ocupadas con todo
lo que las hace felices.\footnote{\textbf{5:20} No se indica si esto es
  algo bueno o no.}

\hypertarget{alguien-tiene-bienes-ricos-pero-no-los-disfruta}{%
\subsection{Alguien tiene bienes ricos pero no los
disfruta}\label{alguien-tiene-bienes-ricos-pero-no-los-disfruta}}

\hypertarget{section-5}{%
\section{6}\label{section-5}}

\bibleverse{1} He observado otro mal aquí en la tierra, y tiene un gran
impacto en la humanidad. \bibleverse{2} Dios le da riqueza, posesiones y
honor a alguien. Ellos tienen todo lo que quieren. Pero Dios no les
permite disfrutar de lo que tienen. En su lugar, ¡otro lo hace! Esto es
difícil de entender, y es verdaderamente cruel. \footnote{\textbf{6:2}
  Ecl 2,18}

\bibleverse{3} Un hombre podría tener cien hijos, y envejecer, pero no
importaría lo larga que fuera su vida si no pudiera disfrutarla y al
final recibir un entierro decente. Yo diría que un niño nacido muerto
estaría mejor que él. \bibleverse{4} La forma en que un niño que nace
muerto viene al mundo y luego se va es dolorosamente difícil de entender
-se va en la oscuridad- y nunca se sabe quién habría sido.\footnote{\textbf{6:4}
  Literalmente, ``En las tinieblas se cubrirá su nombre''. Sin embargo,
  se trata de algo más que de quedarse sin nombre. El nombre en el
  pensamiento hebreo se asocia con el carácter y la personalidad, así
  que lo que se dice aquí es que el niño que nace muerto nunca tendrá la
  oportunidad de convertirse en una persona.} \bibleverse{5} Nunca vio
la luz del día ni supo lo que era vivir. Sin embargo, el niño encuentra
el descanso, y este hombre no. \bibleverse{6} Aunque este hombre viviera
mil años dos veces, no sería feliz. ¿Acaso no acabamos todos en el mismo
lugar: la tumba?\footnote{\textbf{6:6} ``La tumba'': implícito.}

\hypertarget{la-insaciabilidad-del-deseo-y-la-buxfasqueda-del-placer}{%
\subsection{La insaciabilidad del deseo y la búsqueda del
placer}\label{la-insaciabilidad-del-deseo-y-la-buxfasqueda-del-placer}}

\bibleverse{7} Todo el mundo trabaja para poder vivir\footnote{\textbf{6:7}
  La palabra es literalmente ``boca'', por lo que generalmente se
  entiende ``comer''. Sin embargo, el trabajo humano es suplir todo tipo
  de necesidades, por lo que se sugiere que la aplicación es más general
  que simplemente comer.} , pero nunca están santisfechos.
\bibleverse{8} Entonces, ¿qué ventaja real tienen los sabios sobre los
insensatos? Y los pobres, ¿ganan realmente algo con saber comportarse
ante los demás? \bibleverse{9} ¡Alégrate de lo que tienes en lugar de
correr detrás de lo que no tienes! Pero esto también es difícil de
hacer, como correr detrás del viento.

\hypertarget{la-impotencia-humana-en-relaciuxf3n-con-la-predestinaciuxf3n-divina-de-todas-las-cosas-especialmente-la-vida-de-personas-individuales}{%
\subsection{La impotencia humana en relación con la predestinación
divina de todas las cosas (especialmente la vida de personas
individuales)}\label{la-impotencia-humana-en-relaciuxf3n-con-la-predestinaciuxf3n-divina-de-todas-las-cosas-especialmente-la-vida-de-personas-individuales}}

\bibleverse{10} Todo lo que existe ya ha sido descrito\footnote{\textbf{6:10}
  ``Descrito'': Literalmente, ``nombrado''. Sin embargo, en el
  pensamiento hebreo ``nombre'' es mucho más que un simple apelativo, es
  descriptivo del objeto o persona.} . Todo el mundo sabe cómo es la
gente, y que no se puede ganar una discusión con un superior.\footnote{\textbf{6:10}
  ``Un superior'': Literalmente, ``más fuerte'': podría referirse a la
  fuerza física o mental. Sin embargo, es probable que se trate de algún
  tipo de argumento, pero la cuestión es que un superior no tiene que
  ``jugar con las reglas'' de la argumentación. Algunos toman el ``más
  fuerte'' para referirse a Dios, en cuyo caso la esencia de la frase
  significa ``no se puede discutir con Dios''. Además, algunos han visto
  en este versículo un argumento a favor de la predestinación, pero el
  texto no lo apoya necesariamente.} \bibleverse{11} Porque cuantas más
palabras se utilizan, más difícil es que tengan sentido. Entonces, ¿qué
sentido tiene? \bibleverse{12} ¿Quién sabe lo que es mejor para nosotros
y nuestras vidas? Durante nuestras cortas vidas, que pasan como sombras,
tenemos muchas preguntas sin respuesta. Y quién puede decirnos qué
pasará cuando nos hayamos ido?\footnote{\textbf{6:12} Esto podría
  significar lo que sucederá en la tierra una vez que la gente muera, o
  lo que le sucederá a la gente después de la muerte. Ambas
  interpretaciones son posibles a partir del texto.}

\hypertarget{advertencias-para-ser-serios-con-la-vida-y-someterse-pacientemente-a-los-decretos-divinos}{%
\subsection{Advertencias para ser serios con la vida y someterse
pacientemente a los decretos
divinos}\label{advertencias-para-ser-serios-con-la-vida-y-someterse-pacientemente-a-los-decretos-divinos}}

\hypertarget{section-6}{%
\section{7}\label{section-6}}

\bibleverse{1} Una buena reputación es mejor que un perfume costoso, y
el día de tu muerte es mejor que el día de tu nacimiento. \footnote{\textbf{7:1}
  Prov 22,1} \bibleverse{2} Es mejor ir a un funeral que a una
fiesta.\footnote{\textbf{7:2} Literalmente, ``Es mejor ir a la casa del
  luto que a la casa del banquete''.} Al final, todo el mundo muere, y
los que aún están vivos deberían pensar en ello. \bibleverse{3} La pena
es mejor que la risa, porque la tragedia nos ayuda a pensar.\footnote{\textbf{7:3}
  Literalmente, ``por la tristeza del semblante el corazón es bueno''.
  En el pensamiento hebreo, el corazón era donde ocurría el pensamiento.}
\bibleverse{4} Los sabios piensan en el impacto de la muerte, mientras
que los necios sólo piensan en divertirse. \bibleverse{5} Es mejor
escuchar la crítica de un sabio que la canción de los necios.
\bibleverse{6} La risa de los necios es como el crujir de las ramas de
espino que se queman debajo de una olla: sin sentido y que se extinguen
rápidamente.\footnote{\textbf{7:6} Las ramitas de espino utilizadas como
  combustible tienen un valor limitado, ya que aunque arden con fuerza,
  las llamas se apagan rápidamente.} \bibleverse{7} Extorsionar a los
demás convierte a los sabios en insensatos, y aceptar sobornos corrompe
la mente. \bibleverse{8} Terminar algo es mejor que empezarlo. Ser
paciente es mejor que ser orgulloso.

\bibleverse{9} No te apresures a enojarte, porque la ira controla la
mente de los insensatos.\footnote{\textbf{7:9} ``La ira controla la
  mente de los insensatos'': Literalmente, ``la ira se aloja en el seno
  de los insensatos''.} \bibleverse{10} No preguntes: ``¿Por qué los
viejos tiempos eran mejores que ahora?'' . Preguntar eso demuestra que
no eres sabio.

\bibleverse{11} La sabiduría es buena, es como recibir una herencia.
Beneficia a todos en la vida.\footnote{\textbf{7:11} ``Beneficia a todos
  en la vida'': Literalmente, ``Es una ventaja para los que ven el
  sol''.} \bibleverse{12} Porque la sabiduría trae seguridad, al igual
que el dinero, pero la ventaja para los que tienen sabiduría es que se
mantienen sanos y salvos. \footnote{\textbf{7:12} Prov 3,2}

\bibleverse{13} Piensa en lo que hace Dios. Si él hace que algo se
doble, ¡no podrás enderezarlo! \bibleverse{14} En un buen día, alégrate.
Cuando llegue un día malo, párate a pensar. Dios hizo cada día, de modo
que no sabes lo que te sucederá después.

\hypertarget{advertencia-contra-todo-exceso-y-amonestaciuxf3n-de-la-verdadera-sabiduruxeda}{%
\subsection{Advertencia contra todo exceso y amonestación de la
verdadera
sabiduría}\label{advertencia-contra-todo-exceso-y-amonestaciuxf3n-de-la-verdadera-sabiduruxeda}}

\bibleverse{15} A lo largo de mi vida he visto muchas cosas que son
difíciles de entender. Gente buena que muere joven a pesar de hacer lo
correcto,\footnote{\textbf{7:15} Aquí el énfasis parece estar en hacer
  lo correcto tal y como lo define la Ley.} y a la gente malvada que
vive una larga vida de maldad. \bibleverse{16} No pienses que puedes
hacer lo correcto con mucha observancia religiosa, y no pretendas ser
tan sabio.\footnote{\textbf{7:16} Literalmente, ``No debes ser justo en
  exceso, y no debes actuar con sabiduría en exceso''. La palabra
  ``excesivamente'' se refiere aquí a la autosuficiencia más que a la
  cantidad.} ¿Quieres destruirte a ti mismo?\footnote{\textbf{7:16} En
  el sentido de tratar de hacer lo correcto, y sabio, por sus propios
  esfuerzos.} \bibleverse{17} Por otro lado,\footnote{\textbf{7:17}
  Implícito.} no te decidas a vivir una vida malvada, ¡no seas
insensato! ¿Por qué morir antes de tiempo? \bibleverse{18} Debes tener
en cuenta estas advertencias. Los que siguen a Dios estarán seguros de
evitar ambas cosas. \bibleverse{19} La sabiduría da a una persona sabia
más poder que diez consejeros de la ciudad. \bibleverse{20} No hay una
sola persona buena en todo el mundo que haga siempre lo correcto y no
peque nunca. \footnote{\textbf{7:20} Sal 14,3} \bibleverse{21} No te
tomes a pecho todo lo que dice la gente, pues de lo contrario podrías
oír a tu siervo hablar mal\footnote{\textbf{7:21} ``Hablar mal'' -esto
  en el sentido de hablar despectivamente y no de maldecir, como
  sugieren algunas traducciones.} de ti, \bibleverse{22} ¡pues sabes
cuántas veces tú mismo has hablado mal de los demás! \bibleverse{23} He
examinado todo esto usando los principios de la sabiduría. Me dije:
``Pensaré con sabiduría''. Pero la sabiduría se me escapó.
\bibleverse{24} Todo lo que existe está fuera de nuestro alcance, es
demasiado profundo para nuestro entendimiento. ¿Quién puede
comprenderlo?

\hypertarget{las-malas-experiencias-del-predicador-con-las-mujeres}{%
\subsection{Las malas experiencias del predicador con las
mujeres}\label{las-malas-experiencias-del-predicador-con-las-mujeres}}

\bibleverse{25} Dirigí mis pensamientos a descubrir, investigar y
averiguar más sobre la sabiduría y lo que tiene sentido. Quería saber
más sobre lo estúpido que es el mal y lo ridículo que es ser un
insensato.

\bibleverse{26} Descubrí algo más horrible\footnote{\textbf{7:26}
  Literalmente, ``amargo''.} que la muerte: una tontería como la
mujer\footnote{\textbf{7:26} ``Mujer'': simboliza la locura, véase
  Proverbios 5 y Proverbios 7.} que trata de atraparte, que quiere usar
su mente y sus manos para capturarte y atarte. Los que siguen a Dios no
serán atrapados, pero los pecadores caerán en su trampa.

\bibleverse{27} Esto es lo que descubrí después de sumar dos y
dos\footnote{\textbf{7:27} Hebreo: ``uno y uno''.} juntos para intentar
averiguar qué significaba todo aquello, dice el Maestro. \bibleverse{28}
Aunque realmente busqué, no encontré lo que buscaba. La gente dice:
``Encontré un hombre entre mil, pero ni una sola mujer''.\footnote{\textbf{7:28}
  Parece que se trata de una especie de proverbio. Su significado exacto
  es incierto.} \bibleverse{29} Pero descubrí esto: Dios hizo al ser
humano para hacer lo que es correcto, pero ellos han seguido sus propias
ideas.\textsuperscript{{[}\textbf{7:29} Literalmente, ``pero han buscado
muchas artimañas''.{]}}{[}\textbf{7:29} Prov 2,7{]}

\hypertarget{la-conducta-del-sabio-hacia-el-gobernante-y-en-duxedas-de-opresiuxf3n}{%
\subsection{La conducta del sabio hacia el gobernante y en días de
opresión}\label{la-conducta-del-sabio-hacia-el-gobernante-y-en-duxedas-de-opresiuxf3n}}

\hypertarget{section-7}{%
\section{8}\label{section-7}}

\bibleverse{1} ¿Quién puede compararse con los verdaderos
sabios?\footnote{\textbf{8:1} En el sentido de que ser sabio es la
  máxima ambición. Literalmente, ``¿Quién es como el sabio?''} ¿Quién
sabe interpretar las cosas? Si tienes sabiduría tu rostro se ilumina y
tu mirada severa se suaviza.

\bibleverse{2} Mi consejo\footnote{\textbf{8:2} Tomando el ``yo''
  inicial de la frase como ``yo digo''.} es hacer lo que dice el rey, ya
que eso es lo que le prometiste a Dios. \bibleverse{3} No te apresures a
abandonar al rey sin pensar lo que haces, y no te involucres con los que
conspiran contra él,\footnote{\textbf{8:3} Literalmente, ``el asunto es
  desagradable''. Se cree que este término se refiere a un complot o una
  rebelión contra un rey.} pues el rey puede hacer lo que le plazca.
\bibleverse{4} Las órdenes del rey tienen autoridad suprema; ¿quién va a
cuestionarle diciendo: ``Qué haces''? \bibleverse{5} Los que siguen sus
órdenes no se verán involucrados en hacer el mal. La gente sabia piensa,
reconociendo que hay un tiempo correcto, y una manera
correcta.\footnote{\textbf{8:5} O bien, ``la gente sabia sabe que hay un
  tiempo de juicio''.}

\hypertarget{impotencia-e-desorientaciuxf3n-del-hombre}{%
\subsection{Impotencia e desorientación del
hombre}\label{impotencia-e-desorientaciuxf3n-del-hombre}}

\bibleverse{6} Porque hay un momento y una forma correcta para todo,
incluso cuando las cosas te van mal.\footnote{\textbf{8:6} ``Incluso
  cuando las cosas te salen mal'': Literalmente, ``aunque los problemas
  pesen sobre el mortal''.} \bibleverse{7} Nadie sabe lo que va a pasar,
así que ¿quién puede decir lo que depara el futuro? \footnote{\textbf{8:7}
  Ecl 10,14} \bibleverse{8} Nadie puede retener el aliento de vida;
nadie puede evitar el día en que muera. No hay manera de escapar de esa
batalla, ¡y los malvados no se salvarán por su maldad!

\hypertarget{justos-y-malvados-generalmente-corren-el-mismo-destino-en-una-sola-violencia-pertenece-cuando-uno-tiene-pautas-para-el-disfrute-de-la-vida-en-el-trabajo}{%
\subsection{Justos y malvados generalmente corren el mismo destino en
una sola violencia; Pertenece cuando uno tiene pautas para el disfrute
de la vida en el
trabajo}\label{justos-y-malvados-generalmente-corren-el-mismo-destino-en-una-sola-violencia-pertenece-cuando-uno-tiene-pautas-para-el-disfrute-de-la-vida-en-el-trabajo}}

\bibleverse{9} Examiné todas estas cosas, y pensé en todo lo que sucede
aquí en la tierra, y en el daño que se causa cuando la gente domina a
los demás. \bibleverse{10} Sí, he visto a gente malvada enterrada con
gran honor.\footnote{\textbf{8:10} El hebreo dice simplemente ``los
  malvados fueron enterrados'', pero como esto no tiene importancia se
  entiende que fueron enterrados con mucho espectáculo y ceremonia.}
Solían ir al lugar santo,\footnote{\textbf{8:10} El texto hebreo no es
  claro. ``Lugar santo'' se referiría al Templo o a la sinagoga.} y
fueron alabados en la misma ciudad donde hicieron su maldad. Esto es
difícil de entender. \bibleverse{11} Cuando la gente no es castigada
rápidamente por sus crímenes, está aún más decidida a hacer el mal.
\bibleverse{12} Aunque un pecador pueda hacer el mal cien veces y vivir
una larga vida, estoy convencido de que los que hacen lo que Dios dice
estarán mejor. \footnote{\textbf{8:12} Sal 73,17-26} \bibleverse{13} De
hecho, los malvados no vivirán mucho tiempo, pasando como una sombra,
porque se niegan a seguir a Dios.

\bibleverse{14} Otra cosa que es difícil de entender es la siguiente:
las personas buenas son tratadas como deben ser los malvados, y los
malvados son tratados como deben ser las personas buenas. Como digo,
¡esto es difícil de comprender! \bibleverse{15} Así que recomiendo
disfrutar de la vida. No hay nada mejor para nosotros aquí en la tierra
que comer y beber y ser felices. Tal actitud nos acompañará mientras
trabajamos y mientras vivimos la vida que Dios nos da aquí en la tierra.
\footnote{\textbf{8:15} Ecl 2,24}

\hypertarget{el-gobierno-de-dios-en-el-gobierno-mundial-es-insondable-para-el-hombre}{%
\subsection{El gobierno de Dios en el gobierno mundial es insondable
para el
hombre}\label{el-gobierno-de-dios-en-el-gobierno-mundial-es-insondable-para-el-hombre}}

\bibleverse{16} Cuando apliqué mi mente a descubrir la sabiduría y a
observar todo lo que la gente hace aquí en la tierra, no pude dormir, ni
de día ni de noche.\footnote{\textbf{8:16} ``No pude dormir, ni de día
  ni de noche'': o bien, ``nadie tiene descanso, ni de día ni de
  noche''.} \bibleverse{17} Entonces estudié todo lo que hace Dios, y me
di cuenta de que nadie puede entender completamente lo que ocurre aquí.
Por mucho que lo intenten, por muy sabios que pretendan ser, no pueden
comprenderlo realmente.

\hypertarget{la-misma-suerte-para-todos-en-la-vida-y-en-la-muerte-impotencia-humana-contra-la-deidad-el-disfrute-piadoso-de-la-vida-antes-de-la-muerte-establece-una-meta-para-todo-gozo-y-actividad}{%
\subsection{La misma suerte para todos en la vida y en la muerte;
impotencia humana contra la deidad; El disfrute piadoso de la vida antes
de la muerte establece una meta para todo gozo y
actividad}\label{la-misma-suerte-para-todos-en-la-vida-y-en-la-muerte-impotencia-humana-contra-la-deidad-el-disfrute-piadoso-de-la-vida-antes-de-la-muerte-establece-una-meta-para-todo-gozo-y-actividad}}

\hypertarget{section-8}{%
\section{9}\label{section-8}}

\bibleverse{1} Tuve en cuenta todo esto en mi mente. Las personas sabias
y buenas y todo lo que hacen están en manos de Dios. El amor o el odio,
quién sabe lo que les sucederá?\footnote{\textbf{9:1} El significado de
  esta última frase es objeto de debate, tal y como demuestra la
  multiplicidad de traducciones. Parece que subraya la incertidumbre de
  la vida en cuanto a lo que se puede experimentar.} \bibleverse{2} Sin
embargo, todos compartimos el mismo destino: los que hacen el bien, los
que hacen el mal, los buenos, los creyentes religiosos y los que no lo
son,\footnote{\textbf{9:2} ``Los religiosamente observantes y los que no
  lo son''. Literalmente, ``los limpios y los impuros''.} los que se
sacrifican y los que no. Los que hacen el bien son como los que pecan,
los que hacen votos a Dios son como los que no los hacen. \footnote{\textbf{9:2}
  Ecl 2,14; Job 9,22} \bibleverse{3} Esto está muy mal: ¡que todos aquí
en la tierra sufran el mismo destino! Además, la mente de la gente está
llena de maldad. Se pasan la vida pensando en estupideces, y luego se
mueren. \footnote{\textbf{9:3} Ecl 8,11} \bibleverse{4} Pero los vivos
aún tienen esperanza: ¡un perro vivo es mejor que un león muerto!
\bibleverse{5} Los vivos son conscientes de que van a morir, pero los
muertos no tienen conciencia de nada. No reciben ningún otro beneficio;
están olvidados. \bibleverse{6} Su amor, su odio y su envidia, todo ha
desaparecido. No tienen más participación en nada de lo que ocurre aquí
en la tierra.

\bibleverse{7} Así que adelante, coman su comida y disfrútenla. Bebe tu
vino con un corazón feliz. Eso es lo que Dios quiere que hagas.
\footnote{\textbf{9:7} Ecl 5,17} \bibleverse{8} Ponte siempre ropa
elegante y ten buen aspecto.\footnote{\textbf{9:8} Literalmente, ``ropa
  blanca y asegúrate de ponerte aceite de oliva en la cabeza''. El
  sentido aquí es estar siempre en un estado de ánimo festivo -la ropa
  blanca se usaba para las fiestas, junto con la práctica de ungir la
  cabeza con aceite de oliva.} \bibleverse{9} Disfruta de la vida con la
esposa que amas -la que Dios te dio- durante todos los días de esta
breve vida, todos estos días que pasan y cuyo significado es tan difícil
de entender mientras trabajas aquí en la tierra. \bibleverse{10} Todo lo
que hagas, hazlo con todas tus fuerzas, porque cuando vayas a la tumba
ya no habrá trabajo ni pensamiento, ni conocimiento ni sabiduría.

\hypertarget{la-dependencia-del-hombre-del-destino}{%
\subsection{La dependencia del hombre del
destino}\label{la-dependencia-del-hombre-del-destino}}

\bibleverse{11} Pensé en otras cosas que ocurren aquí en la tierra. Las
carreras no siempre las gana el más rápido. Las batallas no siempre las
decide el guerrero más fuerte. Además, los sabios no siempre tienen
comida, las personas inteligentes no siempre ganan dinero, y los astutos
no siempre ganan el favor. El tiempo y el azar afectan a todos ellos.
\footnote{\textbf{9:11} Jer 10,23} \bibleverse{12} No puedes predecir
cuándo será tu final\footnote{\textbf{9:12} Literalmente, ``tiempo''.}
vendrá. Al igual que los peces atrapados en una red, o los pájaros
atrapados en una trampa, así las personas son atrapadas repentinamente
por la muerte cuando menos lo esperan.

\hypertarget{muxe1s-experiencias-de-vida-y-dichos-de-sabiduruxeda}{%
\subsection{Más experiencias de vida y dichos de
sabiduría}\label{muxe1s-experiencias-de-vida-y-dichos-de-sabiduruxeda}}

\bibleverse{13} He aquí otro aspecto de la sabiduría que me impresionó
sobre lo que ocurre aquí en la tierra. \bibleverse{14} Había una vez una
pequeña ciudad con pocos habitantes. Llegó un rey poderoso y sitió la
ciudad, construyendo grandes rampas de tierra contra sus muros.
\bibleverse{15} En aquella ciudad vivía un hombre sabio, pero pobre.
Salvó a la ciudad con su sabiduría. Pero nadie se acordó de darle las
gracias\footnote{\textbf{9:15} ``Darle las gracias'': implícito.} ese
pobre hombre. \bibleverse{16} Como siempre he dicho: ``La sabiduría es
mejor que la fuerza''. Sin embargo, la sabiduría de ese pobre hombre fue
desestimada: la gente no prestó atención a lo que decía. \bibleverse{17}
Es mejor escuchar las palabras tranquilas de un sabio que los gritos de
un gobernante de insensatos. \bibleverse{18} Es mejor tener sabiduría
que armas de guerra; pero un pecador puede destruir mucho bien.

\hypertarget{section-9}{%
\section{10}\label{section-9}}

\bibleverse{1} Las moscas muertas pueden hacer que el aceite perfumado
huela mal. Del mismo modo, un poco de insensatez supera a la gran
sabiduría y al honor. \bibleverse{2} La mente del sabio elige el lado
correcto, pero la mente del insensato va hacia la izquierda.
\bibleverse{3} Sólo la forma en que los necios andan por el camino
demuestra que no tienen sentido común, dejando en claro a todos su
estupidez. \bibleverse{4} Si tu superior se enfada contigo, no te rindas
y te vayas. Si mantienes la calma, incluso los errores graves pueden
resolverse.

\bibleverse{5} También me di cuenta de que hay otro mal aquí en la
tierra: los gobernantes cometen un gran error \bibleverse{6} cuando
ponen a los tontos en altos cargos, mientras que los que están
ampliamente cualificados\footnote{\textbf{10:6} ``Ampliamente
  cualificados'': Literalmente quiere decir ``los ricos'' simplemente,
  pero seguramente se trata de algo más que la simple riqueza acumulada.}
son puestos en posiciones bajas. \footnote{\textbf{10:6} Prov 30,21-22}
\bibleverse{7} He visto a esclavos montando a caballo, mientras los
príncipes caminan por el suelo como esclavos.\footnote{\textbf{10:7} En
  esa sociedad, habría sido muy improbable que los esclavos, a menudo
  enemigos capturados, hubieran montado a caballo. En cambio, la imagen
  de los príncipes obligados a caminar es para mostrar una pérdida de su
  dignidad.} \bibleverse{8} Si cavas un pozo, puedes caerte dentro. Si
derribas un muro, te puede morder una serpiente. \bibleverse{9} Si
extraes piedra, puedes lesionarte. Si partes troncos,\footnote{\textbf{10:9}
  O ``cortar árboles''.} podrías herirte. \bibleverse{10} Si tu hacha
está desafilada y no la afilas, tienes que usar mucha más fuerza.
Conclusión:\footnote{\textbf{10:10} Implícito.} ser sabio trae buenos
resultados.

\bibleverse{11} Si la serpiente muerde al encantador de serpientes antes
de ser encantada, ¡no hay beneficio para el encantador de serpientes!
\footnote{\textbf{10:11} Sal 58,5-6} \bibleverse{12} Las palabras sabias
son beneficiosas, pero los necios se destruyen a sí mismos con lo que
dicen. \bibleverse{13} Los insensatos comienzan diciendo tonterías y
terminan diciendo tonterías perversas. \bibleverse{14} Los insensatos no
paran de hablar, pero nadie sabe lo que va a pasar, así que ¿quién puede
decir lo que depara el futuro?\footnote{\textbf{10:14} Véase8:7.}

\bibleverse{15} El trabajo hace que los insensatos se desgasten tanto
que no pueden lograr nada.\footnote{\textbf{10:15} ``No consiguen
  nada'': Literalmente, ``no encuentran el camino a la ciudad'', una
  expresión coloquial que significa que la gente se confunde por lo que
  no tiene éxito.} \bibleverse{16} Estás en problemas si el rey de tu
país es joven, y si tus líderes están ocupados festejando desde la
mañana. \footnote{\textbf{10:16} Is 3,4} \bibleverse{17} Tienes suerte
si tu rey viene de una familia noble, y si tus líderes festejan a la
hora apropiada para darse energía, y no para emborracharse.
\bibleverse{18} La gente perezosa deja que sus techos se derrumben; la
gente ociosa no repara sus casas con goteras. \bibleverse{19} Una buena
comida trae placer; el vino hace la vida agradable; el dinero cubre
todas las necesidades. \footnote{\textbf{10:19} Jue 9,13; Sal 104,15}
\bibleverse{20} No hables mal del rey, ni siquiera en tus pensamientos.
No hables mal de los dirigentes,\footnote{\textbf{10:20} ``Líderes'':
  Literalmente, ``los ricos''.} incluso en la intimidad de tu
habitación. Un pájaro puede oír lo que dices y salir volando para
contarles.\footnote{\textbf{10:20} Éxod 22,27}

\hypertarget{actuaciuxf3n-inteligente-y-rentable-ante-la-incertidumbre-de-todo-lo-terrenal}{%
\subsection{Actuación inteligente y rentable ante la incertidumbre de
todo lo
terrenal}\label{actuaciuxf3n-inteligente-y-rentable-ante-la-incertidumbre-de-todo-lo-terrenal}}

\hypertarget{section-10}{%
\section{11}\label{section-10}}

\bibleverse{1} Echa el pan a la superficie del agua, y muchos días
después lo encontrarás de nuevo.\footnote{\textbf{11:1} Esta afirmación
  proverbial es una llamada a la generosidad, con la implicación de que
  será recompensada.} \footnote{\textbf{11:1} Prov 19,17} \bibleverse{2}
Comparte lo que tienes con siete u ocho personas, porque nunca se sabe
qué desastre puede ocurrir.\footnote{\textbf{11:2} Este es también un
  proverbio que anima a compartir.} \bibleverse{3} Cuando las nubes
están llenas, vierten la lluvia sobre la tierra. Si un árbol cae al
norte o al sur, se queda donde cayó. \bibleverse{4} El agricultor que se
fija en la dirección del viento sabe cuándo no debe sembrar, y
observando las nubes sabe cuándo no debe cosechar.\footnote{\textbf{11:4}
  Estas afirmaciones reflejan el clima local. Los agricultores no
  sembraban cuando soplaban vientos del este, ya que era un viento seco
  del desierto. Los vientos del oeste traían la lluvia y eran un buen
  momento para sembrar. Lo contrario era cierto para la cosecha: se
  requería un clima seco, no nubes que trajeran lluvia. Lo que se
  fomenta aquí es el tema general de la sabiduría a través de la
  observación.} \bibleverse{5} Así como no sabes cómo llega el aliento
de vida al niño en el vientre de su madre, tampoco puedes entender la
obra de Dios, el Creador de todo. \bibleverse{6} Por la mañana, siembra
tu semilla. Por la tarde, no te detengas. Porque no hay manera de saber
qué cosecha crecerá bien: una puede ser rentable, o la otra, o tal vez
ambas. \bibleverse{7} Qué dulce es vivir en la luz, ver salir el sol un
día más.\footnote{\textbf{11:7} Literalmente, ``Dulce es la luz y
  agradable para los ojos ver el sol''. Sin embargo, lo que se quiere
  decir claramente es el aprecio por la vida continua.} \bibleverse{8}
Que vivas muchos años y que los disfrutes todos. Pero recuerda que habrá
muchos días de oscuridad,\footnote{\textbf{11:8} ``Los días de
  oscuridad'' pueden referirse al tiempo que pasa la gente cuando está
  muerta.} y todo lo que está por venir es incierto.

\hypertarget{recordatorio-para-disfrutar-plenamente-de-la-vida-en-la-juventud-pero-agrada-a-dios}{%
\subsection{Recordatorio para disfrutar plenamente de la vida en la
juventud, pero agrada a
Dios}\label{recordatorio-para-disfrutar-plenamente-de-la-vida-en-la-juventud-pero-agrada-a-dios}}

\bibleverse{9} ¡Jóvenes, disfruten de su juventud! ¡Sean felices con lo
que es bueno! Mientras sean jóvenes, dejen que su mente guíe su vida, y
hagan lo que mejor les parezca. Pero recuerden que Dios los juzgará por
todos sus pensamientos y acciones. \footnote{\textbf{11:9} Ecl 8,15}

\bibleverse{10} Así que no permitan que su mente se preocupe, y eviten
las cosas que hacen daño a su cuerpo. ¡Aun así, a pesar de la juventud y
el entusiasmo, la vida sigue siendo muy difícil de entender!

\hypertarget{section-11}{%
\section{12}\label{section-11}}

\bibleverse{1} Acuérdate de tu Creador mientras eres joven, antes de que
lleguen los días de angustia y envejezcas diciendo: ``Ya no disfruto de
la vida''.

\hypertarget{descripciuxf3n-de-las-dolencias-de-la-vejez}{%
\subsection{Descripción de las dolencias de la
vejez}\label{descripciuxf3n-de-las-dolencias-de-la-vejez}}

\bibleverse{2} Antes de\footnote{\textbf{12:2} ``Antes de'': el llamado
  es a recordar al Creador antes de todo lo que sigue en los versos
  siguientes.} la luz se apague -sol, luna y estrellas- y las nubes de
lluvia vuelvan a oscurecer el cielo. \bibleverse{3} Antes de que los
guardianes de la casa tiemblen y los hombres fuertes se dobleguen, los
moledores dejen de trabajar porque sólo quedan unos pocos, y los que
miran por las ventanas sólo vean tenuemente,\footnote{\textbf{12:3} Las
  alusiones a los problemas de la edad avanzada son evidentes.}
\bibleverse{4} y las puertas de la calle estén cerradas. Antes de que el
sonido del molino disminuya, y te despiertes temprano cuando los pájaros
cantan, pero apenas puedas oírlos. \bibleverse{5} Antes de que
desarrolles el miedo a las alturas y te preocupes por salir a la calle;
cuando el almendro florezca, el saltamontes se arrastre y el deseo
falle,\footnote{\textbf{12:5} ``El deseo falle'': esta palabra sólo
  aparece una vez en el Antiguo Testamento. Algunos creen que se refiere
  a la alcaparra, una fruta con fama de afrodisíaca. Todo el verso se
  refiere de nuevo a la experiencia de envejecer y acercarse a la
  muerte.} porque todos tienen que ir a su casa eterna\footnote{\textbf{12:5}
  ``Casa eterna'': refiriéndose a la tumba.} mientras los dolientes
suben y bajan por la calle. \bibleverse{6} Antes de que se rompa el
cordón de plata y se quiebre el cuenco de oro; antes de que se rompa el
cántaro de agua en la fuente, o la polea en el pozo. \bibleverse{7}
Entonces el polvo vuelve a la tierra de la que salió, y el aliento de
vida vuelve a Dios que lo dio.

\bibleverse{8} ``¡Todo pasa! ¡Todo es tan difícil de entender!'' dice el
Maestro.\footnote{\textbf{12:8} Aquí se resume de nueavo el tema
  introducido en el verso 1:2.} \footnote{\textbf{12:8} Ecl 1,2}

\bibleverse{9} El Maestro no sólo era un hombre sabio, sino que también
enseñaba lo que sabía a los demás. Pensaba en muchos proverbios, los
estudiaba y los ordenaba. \footnote{\textbf{12:9} 1Re 5,12}
\bibleverse{10} El Maestro buscaba la mejor manera de explicar las
cosas, escribiendo con verdad y honestidad. \bibleverse{11} Las palabras
de los sabios son como arreadores para el ganado. Sus dichos recopilados
son como clavos bien puestos por un pastor. \footnote{\textbf{12:11} Heb
  4,12}

\hypertarget{advertencia-contra-cavilaciones-inuxfatiles-lista-del-resultado-final}{%
\subsection{Advertencia contra cavilaciones inútiles; Lista del
resultado
final}\label{advertencia-contra-cavilaciones-inuxfatiles-lista-del-resultado-final}}

\bibleverse{12} Además, alumno mío, ten cuidado, porque la escritura de
libros no tiene fin, y el exceso de estudio desgasta.

\bibleverse{13} Resumiendo, ahora que se ha hablado de todo: Respeta a
Dios y guarda sus mandamientos, pues eso es lo que debe hacer todo el
mundo. \bibleverse{14} Dios nos va a juzgar por todo lo que hagamos,
incluso por lo que hagamos en secreto, sea bueno o malo.
