\hypertarget{primer-recuento-de-los-hombres-de-guerra}{%
\subsection{Primer recuento de los hombres de
guerra}\label{primer-recuento-de-los-hombres-de-guerra}}

\hypertarget{section}{%
\section{1}\label{section}}

\bibleverse{1} El Señor le habló a Moisés en el Tabernáculo de Reunión
mientras estaban en el desierto del Sinaí. Esto fue el primer día del
segundo mes, dos años después de que los israelitas salieran de Egipto.
Le dijo: \bibleverse{2} ``Censen a todos los israelitas según su tribu y
su familia. Cuenten a cada hombre y mantengan un registro del nombre de
cada uno. \footnote{\textbf{1:2} Núm 26,2-51; Éxod 30,12} \bibleverse{3}
Tú y Aarón deberán registrar a todos los mayores de veinte años que sean
aptos para prestar el servicio militar según sus divisiones en el
ejército israelita. \bibleverse{4} Para ayudarlos habrá estar con
ustedes un representante de cada tribu, que es el jefe de cada familia:
\bibleverse{5} ``Estos son los nombres de los hombres que trabajarán con
ustedes: De la tribu de Rubén, Elisur, hijo de Sedeur;

\bibleverse{6} de la tribu de Simeón, Selumiel, hijo de Zurisadai;

\bibleverse{7} de la tribu de Judá, Naasón, hijo de Aminadab;

\bibleverse{8} de la tribu de Isacar, Nataanel, hijo de Zuar;

\bibleverse{9} de la tribu de Zabulón, Eliab, hijo de Helón;

\bibleverse{10} de los hijos de José: de la tribu de Efraín, Elisama,
hijo de Amihud; y de la tribu de Manasés, Gamaliel, hijo de Pedasur;
\footnote{\textbf{1:10} 1Cró 7,26}

\bibleverse{11} de la tribu de Benjamín, Abidán, hijo de Gedeoni;

\bibleverse{12} de la tribu de Dan, Ajiezer, hijo de Amisadai;

\bibleverse{13} de la tribu de Aser, Pagiel, hijo de Ocrán;

\bibleverse{14} de la tribu de Gad, Eliasaf, hijo de Deuel;

\bibleverse{15} y de la tribu de Neftalí, Ahira, hijo de Enán''.

\bibleverse{16} Estos fueron los hombres elegidos de la comunidad
israelita. Eran los jefes de las tribus de sus padres; los jefes de las
familias de Israel. \bibleverse{17} Moisés y Aarón convocaron a estos
hombres que habían sido seleccionados por nombre. \bibleverse{18}
Hicieron que todos los israelitas se reunieran el primer día del segundo
mes, y registraron la genealogía del pueblo según su tribu y familia, y
contaron los nombres de todos los que tenían veinte años o más,
\bibleverse{19} como el Señor le había dicho a Moisés que hiciera.
Moisés llevó a cabo este censo en el desierto del Sinaí.

\hypertarget{los-resultados-del-censo}{%
\subsection{Los resultados del censo}\label{los-resultados-del-censo}}

\bibleverse{20} Los descendientes de Rubén, (que era el hijo primogénito
de Israel), hombres de veinte años o más, fueron registrados por nombre
según los registros genealógicos de su tribu y familias. Y todos los
hombres registrados que estaban aptos para servir en el ejército
\bibleverse{21} n la tribu de Rubén sumaron 46. 500.

\bibleverse{22} Los descendientes de Simeón, hombres de veinte años o
más, fueron registrados por nombre según los registros genealógicos de
su tribu y sus familias. Todos hombres registrados que estaban aptos
para servir en el ejército, \bibleverse{23} de la tribu de Simeón,
sumaron 59. 300.

\bibleverse{24} Los descendientes de Gad, hombres de veinte años o más,
fueron registrados por nombre según los registros genealógicos de su
tribu y sus familias. Todos los hombres registrados que estaban aptos
para servir en el ejército, \bibleverse{25} de la tribu de Gad, sumaron
45. 650.

\bibleverse{26} Los descendientes de Judá, hombres de veinte años o más,
fueron registrados por nombre según los registros genealógicos de su
tribu y sus familias. Todos los hombres inscritos, que estaban aptos
para servir en el ejército, \bibleverse{27} de la tribu de Judá, sumaron
74. 600.

\bibleverse{28} Los descendientes de Isacar, hombres de veinte años o
más, fueron registrados por nombre según los registros genealógicos de
su tribu y sus familias. Todos los hombres inscritos que eran aptos para
servir en el ejército, \bibleverse{29} de la tribu de Isacar, sumaron
54. 400.

\bibleverse{30} Los descendientes de Zabulón, hombres de veinte años o
más, fueron registrados por nombre según los registros genealógicos de
su tribu y sus familias. Todos los hombres inscritos que estaban aptos
para servir en el ejército, \bibleverse{31} de la tribu de Zabulón,
sumaron 57. 400.

\bibleverse{32} Los descendientes de José: los descendientes de Efraín,
hombres de veinte años o más, fueron registrados por nombre según los
registros genealógicos de su tribu y sus familias. Todos los hombres
registrados que estaban aptos para servir en el ejército \bibleverse{33}
de la tribu de Efraín, sumaron 40. 500.

\bibleverse{34} Y los descendientes de Manasés, hombres de veinte años o
más, fueron registrados por nombre según los registros genealógicos de
su tribu y sus familias. Todos los hombres registrados que estaban aptos
para servir en el ejército \bibleverse{35} de la tribu de Manasés,
sumaron 32. 200.

\bibleverse{36} Los descendientes de Benjamín, hombres de veinte años o
más, fueron registrados por nombre según los registros genealógicos de
su tribu y sus familias. Todos los hombres registrados que estaban aptos
para servir en el ejército, \bibleverse{37} de la tribu de Benjamín,
totalizaban 35. 400.

\bibleverse{38} Los descendientes de Dan, hombres de veinte años o más,
fueron registrados por nombre según los registros genealógicos de su
tribu y sus familias. Todos los hombres registrados que estaban aptos
para servir en el ejército, \bibleverse{39} de la tribu de Dan, sumaron
62. 700.

\bibleverse{40} Los descendientes de Aser, hombres de veinte años o más,
fueron registrados por nombre según los registros genealógicos de su
tribu y sus familias. Todos los hombres inscritos que estaban aptos para
servir en el ejército, \bibleverse{41} de la tribu de Aser, sumaron 41.
500.

\bibleverse{42} Los descendientes de Neftalí, hombres de veinte años o
más, fueron registrados por nombre según los registros genealógicos de
su tribu y sus familias. Todos los hombres inscritos que estaban aptos
para servir en el ejército, \bibleverse{43} de la tribu de Neftalí,
sumaron 53. 400.

\bibleverse{44} Estos fueron los totales de los hombres contados y
registrados por Moisés y Aarón, con la ayuda de los doce líderes de
Israel, que representaban cada uno a su familia. \bibleverse{45} De esta
manera, todos los hombres israelitas de veinte años o más que pudieron
servir en el ejército de Israel fueron registrados según sus familias.
\bibleverse{46} La suma total de los registrados fue de 603. 550.

\hypertarget{la-posiciuxf3n-excepcional-de-los-levitas}{%
\subsection{La posición excepcional de los
levitas}\label{la-posiciuxf3n-excepcional-de-los-levitas}}

\bibleverse{47} Sin embargo, los levitas no estaban registrados con los
demás según su tribu y sus familias. \bibleverse{48} Esto se debió a que
el Señor le había dicho a Moisés: \bibleverse{49} ``No registres la
tribu de Leví, ni los cuentes en el censo con los otros israelitas.
\footnote{\textbf{1:49} Núm 2,33; Núm 3,15} \bibleverse{50} Pon a los
levitas a cargo del Tabernáculo y del Testimonio,\footnote{\textbf{1:50}
  El Testimonio se refiere a las tablas de piedra de los Diez
  Mandamientos contenidas en el interior del Arca.} así como de todo su
mobiliario y de todo lo que hay en él. Ellos serán los responsables de
llevar el Tabernáculo y todos sus artículos. Deben cuidarlo, y hacer su
campamento alrededor de él. \footnote{\textbf{1:50} Núm 4,-1; Núm
  3,23-38} \bibleverse{51} Cuando llegue el momento de trasladar el
Tabernáculo, los levitas lo bajarán, y cuando llegue el momento de
acampar, los levitas lo levantarán. Cualquier forastero que se acerque
al Tabernáculo debe ser condenado a muerte. \footnote{\textbf{1:51} Núm
  3,10; Núm 3,38} \bibleverse{52} Los israelitas acamparán por tribus,
cada uno estará en su propio campamento, bajo su propia bandera.
\bibleverse{53} Pero los levitas deben levantar su campamento alrededor
del Tabernáculo del Testimonio para evitar que alguien me haga enojar
con los israelitas.\footnote{\textbf{1:53} Presumiblemente impidiendo
  que cualquiera que no fuera sacerdote se acercara demasiado al
  Tabernáculo.} Los levitas son responsables de cuidar el Tabernáculo
del Testimonio''.

\bibleverse{54} Los israelitas hicieron todo lo que el Señor les ordenó
a través de Moisés.

\hypertarget{el-orden-de-acampamiento-de-las-tribus.}{%
\subsection{El orden de acampamiento de las
tribus.}\label{el-orden-de-acampamiento-de-las-tribus.}}

\hypertarget{section-1}{%
\section{2}\label{section-1}}

\bibleverse{1} El Señor les dijo a Moisés y Aarón: \bibleverse{2} ``Los
israelitas deben establecer su campamento alrededor del Tabernáculo de
Reunión pero a cierta distancia de él. Cada miembro de cada tribu
acampará bajo su propia bandera y estandarte familiar.

\bibleverse{3} La tribu de Judá acampará bajo su bandera en el lado
este. Su líder es Naasón, hijo de Aminadab, \bibleverse{4} tiene 74. 600
hombres.

\bibleverse{5} La tribu de Isacar acampará junto a ellos. Su líder es
Natanael, hijo de Zuar, \bibleverse{6} y tiene 54. 400 hombres.

\bibleverse{7} La siguiente es la tribu de Zabulón. Su líder es Eliab,
hijo de Helón, \bibleverse{8} y tiene 57. 400 hombres.

\bibleverse{9} Así que el total de hombres en el territorio de Judá e de
186. 400. Y cuando llegue la hora de marcharse,\footnote{\textbf{2:9}
  ``Cuando llegue la hora de marcharse'': añadido para mayor claridad.}
ellos irán a la cabeza.

\bibleverse{10} ``La tribu de Rubén acampará bajo su bandera en el lado
sur. Su líder es Elisur, hijo de Sedeúr, \bibleverse{11} cuenta con 46.
500 hombres.

\bibleverse{12} La tribu de Simeón acampará junto a ellos. Su líder es
Selumiel, hijo de Zurisadai, \bibleverse{13} y cuenta con 59. 300
hombres.

\bibleverse{14} La siguiente es la tribu de Gad. Su líder es Eliasaph,
hijo de Deuel, \bibleverse{15} y cuenta con 45. 650 hombres.

\bibleverse{16} Así que el número total de hombres en el área del
campamento de Rubén es de 151. 450. Ellos marcharán en segundo lugar.

\bibleverse{17} ``El Tabernáculo de Reunión que está en el centro del
campamento acompañará a los levitas. Deben marchar en el mismo orden en
que levantaron el campamento, cada uno en el lugar que le corresponde,
bajo su bandera.

\bibleverse{18} ``La tribu de Efraín acampará bajo su bandera en el lado
oeste. Su líder es Elisama, hijo de Amiud, \bibleverse{19} y cuenta con
40. 500 hombres.

\bibleverse{20} La tribu de Manasés acampará junto a ellos. Su líder es
Gamaliel, hijo de Pedasur, \bibleverse{21} y cuenta con 32. 200 hombres.

\bibleverse{22} La siguiente es la tribu de Benjamín. Su líder es
Abidán, hijo de Gedeoni, \bibleverse{23} y cuenta con 35. 400 hombres.

\bibleverse{24} Así que el número total de hombres en el área del
campamento de Efraín es de 108. 100. Ellos marcharán en tercer lugar.

\bibleverse{25} ``La tribu de Dan acampará bajo su bandera en el lado
norte. Su líder es Ajiezer, hijo de Amisadai, \bibleverse{26} y cuenta
con 62. 700 hombres.

\bibleverse{27} La tribu de Aser acampará junto a ellos. Su líder es
Pagiel, hijo de Ocrán, \bibleverse{28} y cuenta con 41. 500 hombres.

\bibleverse{29} A continuación estará la tribu de Neftalí. Su líder es
Ajirá, hijo de Enán, \bibleverse{30} y cuenta con 53. 400 hombres.

\bibleverse{31} Así que el total de hombres en el área del campamento de
Dan es de 157. 600. Ellos marcharán en último lugar, con sus banderas''.

\bibleverse{32} Este es un resumen del censo de los israelitas, hecho
por familia. El total final de los contados en los campamentos por
tribus fue de 603. 550. \footnote{\textbf{2:32} Núm 1,46}
\bibleverse{33} Sin embargo, los levitas no fueron contados entre los
demás israelitas, siguiendo las instrucciones que el Señor le dio a
Moisés. \footnote{\textbf{2:33} Núm 1,48-49}

\bibleverse{34} Los israelitas hicieron todo lo que el Señor le ordenó a
Moisés. Establecieron sus campamentos bajo sus banderas en sus
posiciones asignadas, y marchaban en el mismo orden, cada uno con su
propia tribu y familia.\footnote{\textbf{2:34} Núm 2,2}

\hypertarget{los-hijos-de-aaron}{%
\subsection{Los hijos de Aaron}\label{los-hijos-de-aaron}}

\hypertarget{section-2}{%
\section{3}\label{section-2}}

\bibleverse{1} Este es el relato sobre Aarón y Moisés cuando el Señor le
habló a Moisés en el Monte Sinaí. \footnote{\textbf{3:1} Éxod 6,23}
\bibleverse{2} Los nombres de los hijos de Aarón eran: Nadab
(primogénito), Abiú, Eleazar e Itamar.

\bibleverse{3} Estos eran los nombres de los hijos de Aarón que fueron
ungidos y ordenados para servir como sacerdotes. \bibleverse{4} Nadab y
Abiú murieron en la presencia del Señor cuando ofrecieron el fuego
prohibido ante el Señor en el desierto del Sinaí. Como no tenían hijos,
Eleazar e Itamar sirvieron como sacerdotes mientras su padre Aarón
vivía. \footnote{\textbf{3:4} Lev 10,1-2}

\hypertarget{los-levitas-fueron-designados-para-ayudar-a-los-sacerdotes-y-servir-en-el-santuario.}{%
\subsection{Los levitas fueron designados para ayudar a los sacerdotes y
servir en el
santuario.}\label{los-levitas-fueron-designados-para-ayudar-a-los-sacerdotes-y-servir-en-el-santuario.}}

\bibleverse{5} El Señor le dijo a Moisés, \bibleverse{6} ``Reúne a la
tribu de Leví y preséntalos ante el sacerdote Aarón para que le ayuden
en el ministerio. \bibleverse{7} Deben cumplir con sus deberes en su
nombre y en nombre de todos los israelitas en el Tabernáculo de Reunión,
cuidando el servicio del Tabernáculo. \bibleverse{8} Serán responsables
de cuidar todo el mobiliario del Tabernáculo de Reunión, sirviendo a los
israelitas a través de su trabajo en el Tabernáculo. \footnote{\textbf{3:8}
  Núm 4,-1} \bibleverse{9} Los levitas deben trabajar exclusivamente
para Aarón y sus hijos de porque esta es su asignación entre los
israelitas. \bibleverse{10} Tú designarás a Aarón y a sus hijos para que
tengan la responsabilidad del sacerdocio. Cualquier otro que intente
actuar como sacerdote debe ser ejecutado''.

\hypertarget{los-levitas-fueran-designados-para-redimir-al-primoguxe9nito-israelita}{%
\subsection{Los levitas fueran designados para redimir al primogénito
israelita}\label{los-levitas-fueran-designados-para-redimir-al-primoguxe9nito-israelita}}

\bibleverse{11} El Señor le dijo a Moisés: \bibleverse{12} ``He tomado a
los levitas de entre los israelitas en lugar de cada uno de sus
primogénitos. Los levitas me pertenecen \footnote{\textbf{3:12} Núm
  8,16; Éxod 13,2} \bibleverse{13} porque todos los primogénitos son
míos. Cuando maté a cada primogénito en Egipto, separé como sagrado para
mí a todos los primogénitos de Israel, humanos y animales. Son míos. Yo
soy el Señor''.

\hypertarget{conteo-lugar-de-almacenamiento-luxedder-y-reglamentos-de-los-levitas-masculinos}{%
\subsection{Conteo, lugar de almacenamiento, líder y reglamentos de los
levitas
masculinos}\label{conteo-lugar-de-almacenamiento-luxedder-y-reglamentos-de-los-levitas-masculinos}}

\bibleverse{14} El Señor le habló a Moisés en el desierto del Sinaí, y
le dijo: \bibleverse{15} ``Censa a los levitas según la genealogía de su
padre y su familia. Cuenten cada varón de un mes o mayor''.

\bibleverse{16} Entonces Moisés los registró, siguiendo las
instrucciones del Señor, tal como se lo había dicho.

\bibleverse{17} Estos eran los nombres de los hijos de Levi: Gersón,
Coat y Merari.

\bibleverse{18} Estos eran los nombres de los hijos de Gersón por
familia: Libni y Simeí.

\bibleverse{19} Los hijos de Coat por familia eran Amram, Izar, Hebrón y
Uziel.

\bibleverse{20} Los hijos de Merari, por familia, eran Majlí y Musí.
Estas eran las familias de los levitas, según el linaje de su padre.

\bibleverse{21} La familia de Libni y la familia de Simeí procedían de
Gersón. Estas eran las familias de Gersón.

\bibleverse{22} El total de todos los varones de un mes o más era de 7.
500.

\bibleverse{23} El campamento de las familias de Gerson estaba al oeste,
detrás del Tabernáculo.

\bibleverse{24} El líder de las familias de Gerson era Eliasaf, hijo de
Lael. \bibleverse{25} Su responsabilidad asignada para el Tabernáculo de
Reunión era cuidar del Tabernáculo y la tienda, su cubierta, la cortina
de la entrada del Tabernáculo de Reunión, \bibleverse{26} las cortinas
del patio, la cortina de la entrada del patio que rodea el Tabernáculo y
el altar, las cuerdas y todo lo relacionado con su uso.

\bibleverse{27} Las familias de Amram, Izar, Hebrón y Uziel procedían de
Coat. Estas eran las familias de Coat. \bibleverse{28} El total de todos
los varones de un mes o más era de 8. 600. Su responsabilidad asignada
era cuidar del santuario.

\bibleverse{29} El campamento de las familias de Coat estaba en el lado
sur del Tabernáculo. \bibleverse{30} El líder de las familias de Coat
era Elisafán, hijo de Uziel. \footnote{\textbf{3:30} Lev 10,4}
\bibleverse{31} Su responsabilidad asignada era cuidar el Arca, la mesa,
el candelabro, los altares, los artículos del santuario usados con
ellos, el velo, y todo lo relacionado con estos artículos. \footnote{\textbf{3:31}
  Núm 7,9} \bibleverse{32} El jefe de los líderes de los levitas era
Eleazar, hijo del sacerdote Aarón. Él estaba a cargo de los responsables
de servir en el santuario.

\bibleverse{33} La familia de Majlí y la familia de Musíprocedían de
Merari. Estas eran las familias de Merari. \bibleverse{34} El total de
todos los varones de un mes o más era de 6. 200.

\bibleverse{35} El líder de las familias de Merari era Zuriel, hijo de
Abijaíl. Su campamento estaba en el lado norte del Tabernáculo.
\bibleverse{36} Su responsabilidad asignada era cuidar de los marcos del
Tabernáculo, barras transversales, postes, soportes, todo su equipo y
todo lo relacionado con su uso, \bibleverse{37} así como los postes del
patio circundante con sus soportes, estacas y cuerdas.

\bibleverse{38} El campamento de los hijos de Moisés, Aarón y Aarón
estaba al Este del santuario, con vista al amanecer, frente al
Tabernáculo de Reunión. Eran responsables del santuario en nombre de los
israelitas. Cualquier otro que intentara actuar como sacerdote debía ser
ejecutado. \footnote{\textbf{3:38} Núm 3,10} \bibleverse{39} La suma
total de levitas registrados por Moisés y Aarón como el Señor ordenó fue
de 22. 000. Esto incluía a todos los varones de un mes o mayores.

\hypertarget{examen-y-resoluciuxf3n-del-primoguxe9nito-masculino-en-israel}{%
\subsection{Examen y resolución del primogénito masculino en
Israel}\label{examen-y-resoluciuxf3n-del-primoguxe9nito-masculino-en-israel}}

\bibleverse{40} El Señor le dijo a Moisés: ``Haz un censo de todos los
primogénitos varones israelitas de un mes o más, y registra sus nombres.
\bibleverse{41} Aparta a los levitas para mí. Yo soy el Señor. Ellos
están en lugar de todos los primogénitos de los israelitas. El ganado de
los levitas está en lugar de todo el ganado primogénito de los
israelitas''.

\bibleverse{42} Moisés realizó un censo de todos los primogénitos de los
israelitas, tal como el Señor le había instruido. \bibleverse{43} La
suma total de los primogénitos varones de un mes o más, registrados por
nombre, fue de 22. 273.

\bibleverse{44} El Señor habló con Moisés y le dijo: \bibleverse{45}
``Debes tomar a los levitas en lugar de todos los primogénitos de
Israel, y el ganado de los levitas en lugar de su ganado, porque los
levitas me pertenecen. Yo soy el Señor. \bibleverse{46} Para poder
comprar los 273 primogénitos de Israel que son más que el número de
levitas, \footnote{\textbf{3:46} Núm 3,39; Núm 3,43} \bibleverse{47} se
recaudan cinco siclos para cada uno de ellos, (usando la norma del siclo
del santuario de veinte geras). \bibleverse{48} Entregarás el dinero a
Aarón y a sus hijos como precio de redención para cubrir el exceso de
los israelitas que sobran''.

\bibleverse{49} Moisés recaudó el dinero de redención para aquellos
israelitas que excedían el número redimido por los levitas.
\bibleverse{50} Recolectó el dinero dado en nombre de los primogénitos
de los israelitas. Llegó a recolectar 1. 365 siclos, (usando el estándar
del siclo del santuario). \bibleverse{51} Moisés dio este dinero de
redención a Aarón y sus hijos como el Señor se lo había dicho, siguiendo
las instrucciones del Señor.

\hypertarget{examen-de-los-levitas-aptos-para-el-servicio-incluidas-las-normas-de-servicio}{%
\subsection{Examen de los levitas aptos para el servicio, incluidas las
normas de
servicio}\label{examen-de-los-levitas-aptos-para-el-servicio-incluidas-las-normas-de-servicio}}

\hypertarget{section-3}{%
\section{4}\label{section-3}}

\bibleverse{1} El Señor le dijo a Moisés y Aarón: \bibleverse{2}
``Registra a los descendientes de Coat de la tribu de Leví, de acuerdo a
su familia y línea paterna. \bibleverse{3} Cuenta a los hombres de
treinta a cincuenta años y que tengan derecho a hacer el trabajo de
servir en el Tabernáculo de Reunión.

\bibleverse{4} Este trabajo que deben hacer en el Tabernáculo de Reunión
implica cuidar las cosas más sagradas. \bibleverse{5} Cada vez que
muevan el campamento, Aarón y sus hijos entrarán, quitarán el velo y lo
colocarán sobre el Arca del Testimonio. \bibleverse{6} Sobre esto
pondrán una fina cubierta de cuero, extenderán un paño de color azul
sólido sobre ella, y luego insertarán las varas para transportarlo.

\bibleverse{7} Que extiendan también un paño azul sobre la mesa de la
Presencia, y que pongan sobre ella los platos y las copas, así como los
cuencos y las jarras para la ofrenda de bebida. La ofrenda permanente de
pan debe permanecer sobre ella. \bibleverse{8} Sobre todas estas cosas
deben extender un paño carmesí, luego una fina cubierta de cuero, y
luego insertar las varas para transportarla.

\bibleverse{9} ``Con un paño azul cubrirán el candelabro de luz, junto
con sus lámparas, pinzas de mecha y bandejas, así como los frascos de
aceite de oliva que se usan para llenarlos. \footnote{\textbf{4:9} Éxod
  25,31} \bibleverse{10} Luego deben envolverlo junto con todos sus
utensilios dentro de una fina cubierta de cuero y colocarlo enel
bastidor para transportarlo.

\bibleverse{11} Deben extender un paño azul sobre el altar de oro,
cubrirlo con cuero fino, y luego insertar sus varas para transportarlo.

\bibleverse{12} Deben colocar todos los utensilios usados para el
servicio en el santuario en un paño azul, cubrirlos con cuero fino y
colocarlos en el bastidor para transportarlos.

\bibleverse{13} ``Que limpien las cenizas del altar de bronce y
extiendan un paño morado sobre él, \bibleverse{14} y que pongan sobre él
todo el equipo usado en los servicios del altar: los fogones, los
tenedores para la carne, las palas y los aspersores. Extiendan sobre él
una fina cubierta de cuero y luego inserten las varas para
transportarlo.

\bibleverse{15} ``Una vez que Aarón y sus hijos hayan terminado de
cubrir estas cosas sagradas y todo el equipo relacionado con ellas,
cuando el campamento esté listo para moverse, los sacerdotes de la
familia Coat vendrán y las llevarán. Pero tienen prohibido tocar
cualquier cosa sagrada, de lo contrario morirán. Estas son sus
responsabilidades a la hora de trasladar el Tabernáculo de Reunión.

\bibleverse{16} ``Eleazar, hijo del sacerdote Aarón, supervisará la
obtención del aceite de oliva para las lámparas, el incienso aromático,
la ofrenda de grano diaria y el aceite de la unción. Estará a cargo de
todo el Tabernáculo y todo lo que hay en él, todas las cosas sagradas y
el equipo''.

\bibleverse{17} Entonces el Señor le dijo a Moisés y Aarón:
\bibleverse{18} ``Asegúrense de que las familias de Coat no sean
eliminadas entre los levitas. \bibleverse{19} Esto es lo que tienes que
hacer para que vivan y no mueran por acercarse demasiado a un objeto
sagrado: Aarón y sus hijos deben entrar y decirle a cada uno de ellos lo
que tienen que hacer y lo que tienen que llevar. \bibleverse{20} Pero no
deben entrar y mirar las cosas más sagradas, ni siquiera por un momento,
de lo contrario morirán''. \footnote{\textbf{4:20} 1Sam 6,19}

\bibleverse{21} El Señor le dijo a Moisés: \bibleverse{22} ``Registra a
los descendientes de Gersón, según su familia y el linaje paterno.
\bibleverse{23} Cuenta a los hombres de treinta a cincuenta años que
tengan derecho a hacer el trabajo de servir en el Tabernáculo de
Reunión.

\bibleverse{24} Así es como las familias de Gersón servirán en cuanto a
trabajo y el traslado: \bibleverse{25} levarán las cortinas del
Tabernáculocon su fina cubierta de cuero, las cortinas de la entrada del
Tabernáculo de Reunión, \bibleverse{26} las cortinas del patio, la
cortina de la entrada del patio que rodea el Tabernáculo y el altar, las
cuerdas y todo lo relacionado con su uso. Las familias de Gersón son
responsables de todo lo que se requiera en relación con estos artículos.
\bibleverse{27} Todo lo que hagan estará bajo la supervisión de Aarón y
sus hijos, así como todo el trabajo y las tareas que lleven a cabo.
Debes decirles todo lo que deben llevar. \bibleverse{28} Estas son sus
responsabilidades para el traslado del Tabernáculo de Reunión, realizado
bajo la dirección de Itamar, hijo del sacerdote Aarón.

\bibleverse{29} ``Registra los descendientes de Merari, según su familia
y linaje paterno. \bibleverse{30} Cuenta a los hombres de treinta a
cincuenta años que tengan derecho a realizar el trabajo de servir en el
Tabernáculo de Reunión. \bibleverse{31} Así es como servirán en el
manejo del Tabernáculo de Reunión: llevarán los marcos del Tabernáculo
con sus travesaños, postes y soportes, \bibleverse{32} los postes del
patio circundante con sus soportes, estacas y cuerdas, todo su equipo y
todo lo relacionado con su uso. Debes decirles por su nombre lo que cada
uno debe llevar. \bibleverse{33} Estas son sus responsabilidades por
todo su trabajo en el traslado del Tabernáculo de Reunión, realizado
bajo la dirección de Itamar, hijo del sacerdote Aarón''.

\hypertarget{resultados-de-la-inspecciuxf3n}{%
\subsection{Resultados de la
inspección}\label{resultados-de-la-inspecciuxf3n}}

\bibleverse{34} Moisés, Aarón y los líderes israelitas registraron a las
familias de Coat según el linaje de su familia y de su padre.
\bibleverse{35} Contaban a los hombres de treinta a cincuenta años que
tenían derecho a hacer el trabajo de servir en el Tabernáculo de
Reunión. \bibleverse{36} El total por familias fue de 2. 750.
\bibleverse{37} Este fue el total de las familias de Coat, y eran todos
los que tenían derecho a hacer el trabajo de servir en el Tabernáculo de
Reunión. Moisés y Aarón los registraron de acuerdo con las instrucciones
que el Señor le dio a Moisés.

\bibleverse{38} Las familias de Gersón fueron contadas, de acuerdo a su
familia y linaje paterno, \bibleverse{39} hombres de treinta a cincuenta
años de edad todos ellos con derecho a hacer el trabajo de servir en el
Tabernáculo de Reunión. \bibleverse{40} El total por familias y linaje
paterno fue de 2. 630. \bibleverse{41} Este fue el total de las familias
de Gersón, todos los que tenían derecho a hacer el trabajo de servir en
el Tabernáculo de Reunión. Fueron registrados por Moisés y Aarón de
acuerdo con las instrucciones del Señor.

\bibleverse{42} Las familias de Merari fueron contadas, según el linaje
familiar y paterno, \bibleverse{43} hombres de treinta a cincuenta años
de edad, todos ellos con derecho a realizar el trabajo de servir en el
Tabernáculo de Reunión. \bibleverse{44} El total por familias fue de 3.
200. \bibleverse{45} Este fue el total de las familias de Merari
registradas por Moisés y Aarón de acuerdo con las instrucciones del
Señor.

\bibleverse{46} Así es como Moisés, Aarón y los líderes israelitas
registraron a todos los levitas de acuerdo a su familia y linaje
paterno. \bibleverse{47} Contarona los hombres de treinta a cincuenta
años que tenían derecho a hacer el trabajo de servir en el Tabernáculo
de Reunión y llevarlo. \bibleverse{48} La suma total fue de 8. 580.
\bibleverse{49} Fue en respuesta a las instrucciones del Señor que
fueron registrados por Moisés. A cada uno de los inscritos se les dijo
qué hacer y qué llevar, tal como el Señor se lo había ordenado a Moisés.

\hypertarget{extracciuxf3n-de-los-inmundos-del-campamento}{%
\subsection{Extracción de los inmundos del
campamento}\label{extracciuxf3n-de-los-inmundos-del-campamento}}

\hypertarget{section-4}{%
\section{5}\label{section-4}}

\bibleverse{1} Entonces el Señor le dijo a Moisés: \bibleverse{2}
``Ordena a los israelitas que expulsen del campamento a cualquiera que
tenga una enfermedad de la piel, o que tenga una secreción, o que esté
sucio por tocar un cuerpo muerto.\footnote{\textbf{5:2} ``Sucio por
  tocar un cuerpo muerto'': Esta parece ser una exclusión temporal. Ver
  Levítico 11:24.} \bibleverse{3} Ya sea hombre o mujer, debes
expulsarlos para que no ensucien su campamento, porque ahí es donde yo
habito con ellos''. \footnote{\textbf{5:3} Núm 12,14; Núm 35,34}

\bibleverse{4} Los israelitas siguieron estas instrucciones y expulsaron
a esas personas del campamento. Hicieron lo que el Señor le había dicho
a Moisés que debían hacer.

\hypertarget{malversaciuxf3n-y-su-expiaciuxf3n}{%
\subsection{Malversación y su
expiación}\label{malversaciuxf3n-y-su-expiaciuxf3n}}

\bibleverse{5} El Señor le dijo a Moisés: \bibleverse{6} ``Dile a los
israelitas que cuando un hombre o una mujer es infiel al Señor pecando
contra alguien más, son culpables \bibleverse{7} y deben confesar su
pecado. Tienen que pagar el monto total de la compensación más un quinto
de su valor, y darlo a la persona a la que han agraviado. \bibleverse{8}
Sin embargo, si esa persona\footnote{\textbf{5:8} Esta disposición se
  refiere a una situación en la que la persona agraviada ha muerto.} no
tiene un pariente que pueda recibir la compensación, ésta le pertenece
al Señor y será entregada al sacerdote, junto con un carnero de
sacrificio con el que se justifica al culpable. \bibleverse{9} Todas las
ofrendas sagradas que los israelitas traigan al sacerdote, le pertenecen
a él. \footnote{\textbf{5:9} Núm 18,8} \bibleverse{10} Sus santas
ofrendas les pertenecen, pero una vez que se las dan al sacerdote, le
pertenecen a él''.

\hypertarget{sacrificio-de-celo-y-agua-de-maldiciuxf3n-de-una-mujer-sospechosa-de-adulterio}{%
\subsection{Sacrificio de celo y agua de maldición de una mujer
sospechosa de
adulterio}\label{sacrificio-de-celo-y-agua-de-maldiciuxf3n-de-una-mujer-sospechosa-de-adulterio}}

\bibleverse{11} El Señor le dijo a Moisés: \bibleverse{12} ``Dile a los
israelitas que estas son las instrucciones a seguir\footnote{\textbf{5:12}
  ``Estas son las instrucciones a seguir'': añadido para mayor claridad.}
en caso de que la esposa de un hombre tenga una aventura amorosa,
siéndole infiel a él \bibleverse{13} por acostarse con otra persona.
Puede ser que su marido no se entere y que su acto sucio no haya sido
presenciado. No la atraparon. \bibleverse{14} Pero si su marido se pone
celoso y sospecha de su mujer, sea culpable o no, \bibleverse{15} debe
llevarla ante el sacerdote. También debe llevar en su nombre una ofrenda
de un décimo de efa de harina de cebada. También debe llevar para ella
una ofrenda de un efa de harina de cebada. No debe verter aceite de
oliva o poner incienso sobre ella, ya que es una ofrenda de grano por
los celos, una ofrenda recordatoria para recordarle a las personas sobre
el pecado. \bibleverse{16} ``El sacerdote debe guiar a la esposa hacia
adelante y hacer que se presente ante el Señor. \bibleverse{17} Luego
llenará una vasija de barro con agua sagrada y rociará sobre ella polvo
del suelo del Tabernáculo. \bibleverse{18} Una vez que el sacerdote haya
hecho que la mujer se ponga de pie ante el Señor, le soltará el pelo y
le hará sostener la ofrenda de grano recordatoria, la ofrenda de grano
que se usa en casos de celos. El sacerdote sostendrá el agua amarga que
maldice. \bibleverse{19} Pondrá a la mujer bajo juramento y le dirá: `Si
nadie más ha dormido contigo y no has sido infiel ni te has vuelto
impura mientras estabas casada con tu marido, que no te perjudique esta
agua amarga que maldice. \bibleverse{20} Pero si has sido infiel
mientras estabas casada con tu marido y te has vuelto impura y has
tenido relaciones sexuales con otra persona\ldots{}'\,''.
\bibleverse{21} (Aquí el sacerdote pondrá a la mujer bajo juramento de
la maldición como sigue). ``Que el Señor te envíe una maldición que todo
el mundo conoce, haciendo que tus muslos se encojan y tu vientre se
hinche. \bibleverse{22} Que esta agua que maldice entre en tu estómago y
haga que tu vientre se hinche y tus muslos se encojan. ``La mujer debe
responder: `De acuerdo, estoy de acuerdo'.\footnote{\textbf{5:22}
  Literalmente, ``Amén, Amén''.}

\bibleverse{23} ``El sacerdote debe escribir estas maldiciones en un
pergamino y luego lavarlas en el agua amarga. \bibleverse{24} Hará que
la mujer beba el agua amarga que maldice, y le causará un dolor amargo
si es culpable.\footnote{\textbf{5:24} ``Si es culpable'': implícito.}
\bibleverse{25} El sacerdote le quitará la ofrenda de grano por los
celos, la agitará ante el Señor y la llevará al altar. \bibleverse{26}
Entonces el sacerdote tomará un puñado de la ofrenda de grano como
porción de recuerdo y lo quemará en el altar, y hará que la mujer beba
el agua. \bibleverse{27} ``Después de hacerla beber el agua, si ella se
ha hecho impura y ha sido infiel a su marido, entonces el agua que
maldice le causará un dolor amargo. Su vientre se hinchará y sus muslos
se encogerán. Se convertirá en una mujer maldita entre su pueblo.
\bibleverse{28} Pero si la mujer no se ha hecho impura por ser infiel y
está limpia, no experimentará este castigo y aún podrá tener hijos.

\bibleverse{29} ``Esta es la regla a seguir en casos de celos cuando una
mujer tiene una aventura y se hace impura mientras está casada con su
marido, \bibleverse{30} o cuando el marido empieza a sentir celos y
sospecha de su esposa. Su esposa deberá presentarse ante el Señor, y el
sacerdote deberá cumplir cada parte de esta regla. \bibleverse{31} Si es
hallada culpable,\footnote{\textbf{5:31} ``Si es hallada culpable'':
  implícito.} su marido no será responsable. Pero la mujer cargará con
las consecuencias de su pecado''.

\hypertarget{normas-relativas-a-los-nazareos}{%
\subsection{Normas relativas a los
nazareos}\label{normas-relativas-a-los-nazareos}}

\hypertarget{section-5}{%
\section{6}\label{section-5}}

\bibleverse{1} El Señor le dijo a Moisés: \bibleverse{2} ``Dile a los
israelitas: Si un hombre o una mujer hace una promesa especial de
convertirse en nazareo,\footnote{\textbf{6:2} ``Nazareo'': significa
  ``dedicado''.} para dedicarse al Señor, \footnote{\textbf{6:2} 1Sam
  1,11} \bibleverse{3} no deben beber vino u otra bebida alcohólica. No
deben ni siquiera beber vinagre de vino o cualquier otra bebida
alcohólica, o cualquier jugo de uva o comer uvas o pasas. \footnote{\textbf{6:3}
  Luc 1,15} \bibleverse{4} Durante todo el tiempo que estén dedicados al
Señor no deben comer nadaque sea fruto de una vid, ni siquiera las
semillas o las cáscaras de uva.

\bibleverse{5} ``No deben usar una navaja de afeitar sobre sus cabezas
durante todo el tiempo de esta promesa de dedicación. Deben permanecer
santos hasta que su tiempo de dedicación al Señor haya terminado. Deben
dejar crecer su cabello. \footnote{\textbf{6:5} Jue 13,5}

\bibleverse{6} ``Durante este tiempo de dedicación al Señor no deben
acercarse a un cadáver. \bibleverse{7} Incluso si es su padre, madre,
hermano o hermana los que han muerto, no deben ensuciarse, porque su
pelo sin cortar anuncia su dedicación a Dios. \bibleverse{8} Durante
todo el tiempo de su dedicación deben ser santos para el Señor.

\hypertarget{regulaciones-relativas-a-la-contaminaciuxf3n-del-nazareo}{%
\subsection{Regulaciones relativas a la contaminación del
nazareo}\label{regulaciones-relativas-a-la-contaminaciuxf3n-del-nazareo}}

\bibleverse{9} ``Sin embargo, si alguien muere repentinamente cerca de
ellos, convirtiéndolos en inmundos, deben esperar siete días, y al
séptimo día cuando se limpien de nuevo deben afeitarse la cabeza.
\footnote{\textbf{6:9} Núm 19,11} \bibleverse{10} El octavo día llevarán
dos tórtolas o dos pichones al sacerdote que está a la entrada del
Tabernáculo de Reunión. \footnote{\textbf{6:10} Lev 5,7} \bibleverse{11}
El sacerdote ofrecerá una como ofrenda por el pecado y la otra como
holocausto para corregirlas, porque se hicieron culpables por estar
cerca del cadáver. Ese día deben volver a dedicarse y dejar que les
vuelva a crecer el cabello. \bibleverse{12} Deben volver a dedicarse al
Señor por el tiempo completo que prometieron originalmente y traer un
cordero macho de un año como ofrenda por la culpa. Los días anteriores
no cuentan para el tiempo de dedicación porque se volvieron inmundos.

\hypertarget{ordenanzas-sobre-la-ceremonia-del-sacrificio-al-final-del-nazareo}{%
\subsection{Ordenanzas sobre la ceremonia del sacrificio al final del
nazareo}\label{ordenanzas-sobre-la-ceremonia-del-sacrificio-al-final-del-nazareo}}

\bibleverse{13} ``Estas son las reglas que se deben observar cuando el
tiempo de dedicación del nazareo termine. Deben ser llevadas a la
entrada del Tabernáculo de Reunión. \bibleverse{14} Allí deben presentar
una ofrenda al Señor de un cordero macho sin defectos de un año como
holocausto, un cordero hembra sin defectos de un año como ofrenda por el
pecado y un carnero sin defectos como ofrenda de paz. \bibleverse{15}
Además deben traer una cesta de pan sin levadura hecha de la mejor
harina mezclada con aceite de oliva y obleas sin levadura recubiertas
con aceite de oliva así como sus ofrendas de granos y bebidas.
\bibleverse{16} El sacerdote presentará todo esto ante el Señor, así
como el sacrificio de la ofrenda por el pecado y el holocausto.
\bibleverse{17} También sacrificará un carnero como ofrenda de paz al
Señor, junto con la cesta de pan sin levadura. Además el sacerdote
presentará la ofrenda de grano y la ofrenda de bebida. \bibleverse{18}
``Luego los nazareos se afeitarán la cabeza a la entrada del Tabernáculo
de Reunión. Se quitarán el cabello de sus cabezas que fueron dedicadas,
y lo pondrán en el fuego bajo la ofrenda de paz. \footnote{\textbf{6:18}
  Hech 18,18} \bibleverse{19} Una vez que los nazareos se hayan
afeitado, el sacerdote tomará la espaldilla hervida del carnero, un pan
sin levadura de la cesta, y una oblea sin levadura, y los pondrá en sus
manos. \bibleverse{20} El sacerdote los agitará como ofrenda mecida ante
el Señor. Estos artículos son sagrados y pertenecen al sacerdote, así
como el pecho de la ofrenda mecida y el muslo que fue ofrecido. Una vez
que esto termine, los nazareos podrán beber vino.

\bibleverse{21} Estas son las reglas que deben observarse cuando un
nazareo promete dar ofrendas al Señor en relación con su dedicación.
También pueden traer ofrendas adicionales si tienen los medios para
hacerlo. Cada nazareo debe cumplir las promesas que ha hecho cuando se
dedica''.

\hypertarget{orden-de-la-bendiciuxf3n-sacerdotal}{%
\subsection{Orden de la bendición
sacerdotal}\label{orden-de-la-bendiciuxf3n-sacerdotal}}

\bibleverse{22} El Señor le dijo a Moisés: \bibleverse{23} ``Dile a
Aarón y a sus hijos: Así es como debes bendecir a los israelitas. Esto
es lo que deben decir: \footnote{\textbf{6:23} Lev 9,22-23}
\bibleverse{24} ```Que el Señor te bendiga y te cuide. \footnote{\textbf{6:24}
  Sal 121,-1} \bibleverse{25} Que el Señor te sonría y sea
misericordioso contigo. \footnote{\textbf{6:25} Sal 80,4}
\bibleverse{26} Que el Señor te cuide y te dé la paz'. \footnote{\textbf{6:26}
  Sal 69,17-18}

\bibleverse{27} Cuando los sacerdotes bendigan a los israelitas en mi
nombre, yo los bendeciré''.

\hypertarget{los-dones-de-consagraciuxf3n-de-los-jefes-tribales-para-el-santuario}{%
\subsection{Los dones de consagración de los jefes tribales para el
santuario}\label{los-dones-de-consagraciuxf3n-de-los-jefes-tribales-para-el-santuario}}

\hypertarget{section-6}{%
\section{7}\label{section-6}}

\bibleverse{1} El mismo día que Moisés terminó de montar el Tabernáculo,
lo ungió y lo dedicó, junto con todo su mobiliario, el altar y todos sus
utensilios. \footnote{\textbf{7:1} Éxod 40,9-10} \bibleverse{2} Los
líderes israelitas, que eran los jefes de sus familias, vinieron y
dieron una ofrenda. Eran los mismos líderes de las tribus que habían
trabajado en el registro\footnote{\textbf{7:2} Ver el capítulo 1.} de
los israelitas. \bibleverse{3} Trajeron al Señor una ofrenda de seis
carros cubiertos y doce bueyes. Cada líder dio un buey, y dos líderes
compartieron la ofrenda de un carro. Los presentaron frente al
Tabernáculo. \bibleverse{4} El Señor le dijo a Moisés: \bibleverse{5}
``Acepta lo que te dan y úsalo para el trabajo del Tabernáculo de
Reunión. Dáselo a los levitas para que los usen según sea necesario''.

\bibleverse{6} Moisés aceptó las carretas y los bueyes y los entregó a
los levitas. \bibleverse{7} Dio dos carros y cuatro bueyes a las
familias de Gersón para que los usaran según sus necesidades.
\bibleverse{8} Dio cuatro carros y ocho bueyes a las familias de Merari,
para que los usaran según sus necesidades. Todo el trabajo debía hacerse
bajo la dirección de Itamar, hijo del sacerdote Aarón. \bibleverse{9} No
dio carros ni bueyes a los coatitas porque su responsabilidad era llevar
sobre sus hombros los objetos sagrados asignados bajo su cuidado.

\bibleverse{10} El día que el altar fue ungido, los líderes se
presentaron con sus ofrendas dedicatorias, presentándolas delante de él.
\footnote{\textbf{7:10} 2Cró 7,9}

\bibleverse{11} Entonces el Señor le dijo a Moisés: ``Haz que un líder
venga cada día y presente su ofrenda para la dedicación del altar''.
\footnote{\textbf{7:11} Núm 1,4-16; Núm 2,3-29}

\bibleverse{12} El primer día Naasón, hijo de Aminadab, de la tribu de
Judá se adelantó con su ofrenda. \bibleverse{13} Su ofrenda era una
placa de plata que pesaba ciento treinta siclos, y un tazón de plata que
pesaba setenta siclos, (usando la tasación del siclo según el
santuario). Ambos estaban llenos de la mejor harina mezclada con aceite
de oliva como ofrenda de grano.

\bibleverse{14} También presentó un plato de oro que pesaba diez siclos
llenos de incienso. Como sacrificios trajo

\bibleverse{15} un novillo, un carnero y un cordero macho de un año como
holocausto,

\bibleverse{16} una cabra macho como ofrenda por el pecado,

\bibleverse{17} y una ofrenda de paz de dos bueyes, cinco carneros,
cinco cabras macho, y cinco corderos macho de un año. Esta fue la
ofrenda de Naasón, hijo de Aminadab.

\bibleverse{18} El segundo día se presentó Natanael, hijo de Zuar, el
líder de la tribu de Isacar. \bibleverse{19} La ofrenda que presentó fue
una placa de plata que pesaba ciento treinta siclos, y un tazón de plata
que pesaba setenta siclos, (usando la tasación del siclo según el
santuario). Ambos estaban llenos de la mejor harina mezclada con aceite
de oliva como ofrenda de grano.

\bibleverse{20} También presentó un plato de oro que pesaba diez siclos
llenos de incienso. Como sacrificios trajo

\bibleverse{21} un novillo, un carnero y un cordero macho de un año como
holocausto,

\bibleverse{22} una cabra macho como ofrenda por el pecado,

\bibleverse{23} y una ofrenda de paz de dos bueyes, cinco carneros,
cinco cabras macho, y corderos macho de cinco años. Esta fue la ofrenda
de Natanael, hijo de Zuar.

\bibleverse{24} El tercer día se presentó Eliab, hijo de Helón, el líder
de la tribu de Zabulón. \bibleverse{25} La ofrenda que presentó fue una
placa de plata que pesaba ciento treinta siclos, y un cuenco de plata
que pesaba setenta siclos, (usando la tasación del siclo según el
santuario). Ambos estaban llenos de la mejor harina mezclada con aceite
de oliva como ofrenda de grano.

\bibleverse{26} También presentó un plato de oro que pesaba diez siclos
llenos de incienso. Como sacrificios trajo

\bibleverse{27} un novillo, un carnero y un cordero macho de un año como
holocausto,

\bibleverse{28} una cabra macho como ofrenda por el pecado,

\bibleverse{29} y una ofrenda de paz de dos bueyes, cinco carneros,
cinco cabras macho, y corderos macho de cinco años. Esta fue la ofrenda
de Eliab, hijo de Helón.

\bibleverse{30} El cuarto día se presentó Elisur, hijo de Sedeúr, el
líder de la tribu de Rubén. \bibleverse{31} La ofrenda que presentó fue
una placa de plata que pesaba ciento treinta siclos, y un cuenco de
plata que pesaba setenta siclos, (usando la tasación del siclo según el
santuario). Ambos estaban llenos de la mejor harina mezclada con aceite
de oliva como ofrenda de grano.

\bibleverse{32} También presentó un plato de oro que pesaba diez siclos
llenos de incienso. Como sacrificios trajo

\bibleverse{33} un novillo, un carnero y un cordero macho de un año como
holocausto,

\bibleverse{34} una cabra macho como ofrenda por el pecado,

\bibleverse{35} y una ofrenda de paz de dos bueyes, cinco carneros,
cinco cabras macho, y corderos macho de cinco años. Esta fue la ofrenda
de Elisur, hijo de Sedeúr.

\bibleverse{36} El quinto día se presentó Selumiel, hijo de Zurisadai,
el líder de la tribu de Simeón. \bibleverse{37} La ofrenda que presentó
fue una placa de plata que pesaba ciento treinta siclos, y un cuenco de
plata que pesaba setenta siclos, (usando la tasación del siclo según el
santuario). Ambos estaban llenos de la mejor harina mezclada con aceite
de oliva como ofrenda de grano.

\bibleverse{38} También presentó un plato de oro que pesaba diez siclos
llenos de incienso. Como sacrificios trajo

\bibleverse{39} un novillo, un carnero y un cordero macho de un año como
holocausto,

\bibleverse{40} una cabra macho como ofrenda por el pecado,

\bibleverse{41} y una ofrenda de paz de dos bueyes, cinco carneros,
cinco cabras macho, y corderos macho de cinco años. Esta fue la ofrenda
de Selumiel, hijo de Zurisadai.

\bibleverse{42} El sexto día se presentó Eliasaf, hijo de Deuel, el
líder de la tribu de Gad. \bibleverse{43} La ofrenda que presentó fue
una placa de plata que pesaba ciento treinta siclos, y un cuenco de
plata que pesaba setenta siclos, (usando la tasación del siclo según el
santuario). Ambos estaban llenos de la mejor harina mezclada con aceite
de oliva como ofrenda de grano.

\bibleverse{44} También presentó un plato de oro que pesaba diez siclos
llenos de incienso. Como sacrificios trajo

\bibleverse{45} un novillo, un carnero y un cordero macho de un año como
holocausto,

\bibleverse{46} una cabra macho como ofrenda por el pecado,

\bibleverse{47} y una ofrenda de paz de dos bueyes, cinco carneros,
cinco cabras macho, y corderos macho de cinco años. Esta fue la ofrenda
de Eliasaf, hijo de Deuel.

\bibleverse{48} El séptimo día se presentó Elisama, hijo de Ammihud, el
líder de la tribu de Efraín. \bibleverse{49} La ofrenda que presentó fue
una placa de plata que pesaba ciento treinta siclos, y un tazón de plata
que pesaba setenta siclos, (usando la tasación del siclo según el
santuario). Ambos estaban llenos de la mejor harina mezclada con aceite
de oliva como ofrenda de grano.

\bibleverse{50} También presentó un plato de oro que pesaba diez siclos
llenos de incienso. Como sacrificios trajo

\bibleverse{51} un novillo, un carnero y un cordero macho de un año como
holocausto,

\bibleverse{52} una cabra macho como ofrenda por el pecado,

\bibleverse{53} y una ofrenda de paz de dos bueyes, cinco carneros,
cinco cabras macho, y corderos macho de cinco años. Esta fue la ofrenda
de Elishama, hijo de Amiúd.

\bibleverse{54} El octavo día se presentó Gamaliel, hijo de Pedasur, el
líder de la tribu de Manasés. \bibleverse{55} La ofrenda que presentó
fue una placa de plata que pesaba ciento treinta siclos, y un tazón de
plata que pesaba setenta siclos, (usando la tasación del siclo según el
santuario). Ambos estaban llenos de la mejor harina mezclada con aceite
de oliva como ofrenda de grano.

\bibleverse{56} También presentó un plato de oro que pesaba diez siclos
llenos de incienso. Como sacrificios trajo

\bibleverse{57} un novillo, un carnero y un cordero macho de un año como
holocausto,

\bibleverse{58} una cabra macho como ofrenda por el pecado,

\bibleverse{59} y una ofrenda de paz de dos bueyes, cinco carneros,
cinco cabras macho, y corderos macho de cinco años. Esta fue la ofrenda
de Gamaliel, hijo de Pedasur.

\bibleverse{60} El noveno día se presentó Abidán, hijo de Gideoni, el
líder de la tribu de Benjamín. \bibleverse{61} La ofrenda que presentó
fue una placa de plata que pesaba ciento treinta siclos, y un tazón de
plata que pesaba setenta siclos, (usando la tasación del siclo según el
santuario). Ambos estaban llenos de la mejor harina mezclada con aceite
de oliva como ofrenda de grano.

\bibleverse{62} También presentó un plato de oro que pesaba diez siclos
llenos de incienso. Como sacrificios trajo

\bibleverse{63} un novillo, un carnero y un cordero macho de un año como
holocausto,

\bibleverse{64} una cabra macho como ofrenda por el pecado,

\bibleverse{65} y una ofrenda de paz de dos bueyes, cinco carneros,
cinco cabras macho, y corderos macho de cinco años. Esta fue la ofrenda
de Abidán, hijo de Gedeoni.

\bibleverse{66} El décimo día se presentó Ahiezer, hijo de Amisadai, el
líder de la tribu de Dan. \bibleverse{67} La ofrenda que presentó fue
una placa de plata que pesaba ciento treinta siclos, y un tazón de plata
que pesaba setenta siclos, (usando la tasación del siclo según el
santuario). Ambos estaban llenos de la mejor harina mezclada con aceite
de oliva como ofrenda de grano.

\bibleverse{68} También presentó un plato de oro que pesaba diez siclos
llenos de incienso. Como sacrificios trajo

\bibleverse{69} un novillo, un carnero y un cordero macho de un año como
holocausto,

\bibleverse{70} una cabra macho como ofrenda por el pecado,

\bibleverse{71} y una ofrenda de paz de dos bueyes, cinco carneros,
cinco cabras macho, y corderos macho de cinco años. Esta era la ofrenda
de Ahiezer, hijo de Amisadai.

\bibleverse{72} El undécimo día se presentó Pagiel, hijo de Ocrán, el
líder de la tribu de Aser. \bibleverse{73} La ofrenda que presentó fue
una placa de plata que pesaba ciento treinta siclos, y un cuenco de
plata que pesaba setenta siclos, (usando la tasación del siclo según el
santuario). Ambos estaban llenos de la mejor harina mezclada con aceite
de oliva como ofrenda de grano.

\bibleverse{74} También presentó un plato de oro que pesaba diez siclos
llenos de incienso. Como sacrificios trajo

\bibleverse{75} un novillo, un carnero y un cordero macho de un año como
holocausto,

\bibleverse{76} una cabra macho como ofrenda por el pecado,

\bibleverse{77} y una ofrenda de paz de dos bueyes, cinco carneros,
cinco cabras macho, y corderos macho de cinco años. Esta era la ofrenda
de Pagiel, hijo de Ocran.

\bibleverse{78} El duodécimo día se presentó Ahira, hijo de Enán, el
jefe de la tribu de Neftalí. \bibleverse{79} La ofrenda que presentó fue
una placa de plata que pesaba ciento treinta siclos, y un cuenco de
plata que pesaba setenta siclos, (usando la tasación del siclo según el
santuario). Ambos estaban llenos de la mejor harina mezclada con aceite
de oliva como ofrenda de grano.

\bibleverse{80} También presentó un plato de oro que pesaba diez siclos
llenos de incienso. Como sacrificios trajo

\bibleverse{81} un novillo, un carnero y un cordero macho de un año como
holocausto,

\bibleverse{82} una cabra macho como ofrenda por el pecado,

\bibleverse{83} y una ofrenda de paz de dos bueyes, cinco carneros,
cinco cabras macho, y corderos macho de cinco años. Esta fue la ofrenda
de Ahira, hijo de Enan.

\bibleverse{84} Así que el día en que el altar fue ungido, las ofrendas
dedicatorias traídas por los líderes israelitas fueron doce platos de
plata, doce cuencos de plata y doce platos de oro. \bibleverse{85} Cada
plato de plata pesaba ciento treinta siclos, y cada cuenco pesaba
setenta siclos. El peso total de la plata era de dos mil cuatrocientos
siclos, (usando la tasación del siclo según el santuario).
\bibleverse{86} Los doce platos de oro llenos de incienso pesaban diez
siclos cada uno, (usando la tasación del siclo según el santuario). El
peso total del oro era de ciento veinte siclos. \bibleverse{87} Los
animales presentados como holocausto eran doce toros, doce carneros y
doce corderos machos de un año, así como sus ofrendas de grano, y doce
cabras machos como ofrenda por el pecado. \bibleverse{88} Los animales
presentados como ofrenda de paz eran veinticuatro toros, sesenta
carneros, sesenta machos cabríos y sesenta corderos machos de un año.
Esta era la ofrenda de dedicación para el altar una vez que había sido
ungido.

\bibleverse{89} Cada vez que Moisés entraba en el Tabernáculo de Reunión
para hablar con el Señor, oía la voz que le hablaba desde la tapa de
expiación del Arca del Testimonio entre los dos querubines. Así es como
el Señor le habló.\footnote{\textbf{7:89} Éxod 25,21-22; 1Sam 3,3-14}

\hypertarget{las-siete-luxe1mparas-del-candelero}{%
\subsection{Las siete lámparas del
candelero}\label{las-siete-luxe1mparas-del-candelero}}

\hypertarget{section-7}{%
\section{8}\label{section-7}}

\bibleverse{1} El Señor le dijo a Moisés: \bibleverse{2} ``Dile a Aarón,
`Cuando pongas las siete lámparas en el candelabro, asegúrate de que
brillen hacia el frente'\,''.

\bibleverse{3} Así que eso es lo que hizo Aarón. Colocó las lámparas
hacia el frente del candelabro, como el Señor le había ordenado a
Moisés. \bibleverse{4} El candelabro estaba hecho de oro martillado
desde su base hasta los adornos florales de la parte superior, de
acuerdo con el diseño que el Señor había mostrado a Moisés.

\hypertarget{la-consagraciuxf3n-de-los-levitas-como-un-regalo-santo-a-dios}{%
\subsection{La consagración de los levitas como un regalo santo a
Dios}\label{la-consagraciuxf3n-de-los-levitas-como-un-regalo-santo-a-dios}}

\bibleverse{5} El Señor le dijo a Moisés: \bibleverse{6} ``Separa a los
levitas de los demás israelitas y purifícalos. \footnote{\textbf{8:6}
  Mal 3,3} \bibleverse{7} Los purificarás así: Rocíalos con el agua de
la purificación. Deben afeitarse todo el pelo de sus cuerpos y lavar su
ropa para que estén limpios. \footnote{\textbf{8:7} Núm 5,17; Núm 19,9;
  Núm 19,17; Lev 14,8} \bibleverse{8} Haz que traigan un novillo con su
ofrenda de grano de la mejor harina mezclada con aceite de oliva, y
debes traer un segundo novillo como ofrenda por el pecado.
\bibleverse{9} Toma a los levitas y haz que se paren frente al
Tabernáculo de Reunión y llama a todos los israelitas para que se reúnan
allí. \bibleverse{10} Cuando lleves a los levitas al Señor, los
israelitas pondrán sus manos sobre ellos. \bibleverse{11} Aarón
presentará a los levitas a Jehová como ofrenda agitada de los israelitas
para que hagan la obra de Jehová. \footnote{\textbf{8:11} Núm 8,21}

\bibleverse{12} Los levitas pondrán sus manos sobre las cabezas de los
toros. Uno será sacrificado como ofrenda por el pecado al Señor, y el
otro como holocausto para reconciliar a los levitas con el Señor.
\bibleverse{13} Que los levitas se pongan de pie delante de Aarón y sus
hijos y los presenten al Señor como ofrenda de ofrenda. \bibleverse{14}
Así separarás a los levitas del resto de los israelitas, y los levitas
me pertenecerán a mí.

\bibleverse{15} Pueden venir a servir en el Tabernáculo de Reunión una
vez que los hayas purificado y presentado como ofrenda mecida.
\bibleverse{16} ``Los levitas han sido completamente consagrados a mí
por los israelitas. Los he aceptado como míos en lugar de todos los
primogénitos de los israelitas. \footnote{\textbf{8:16} Núm 3,12}
\bibleverse{17} Todo primogénito varón de Israel me pertenece, tanto
humano como animal. Los reservé para mí cuando maté a todos los
primogénitos de Egipto. \footnote{\textbf{8:17} Éxod 13,2}
\bibleverse{18} He tomado a los levitas en lugar de todos los
primogénitos de los israelitas. \bibleverse{19} De todos los israelitas,
los levitas son un regalo mío para Aarón y sus hijos para servir a los
israelitas en el Tabernáculo de Reunión, y en su nombre para
enderezarlos, para que no les pase nada malo cuando vengan al
santuario''. \footnote{\textbf{8:19} Núm 3,9}

\bibleverse{20} Moisés, Aarón y todos los israelitas hicieron todo lo
que el Señor había ordenado a Moisés que hicieran con respecto a los
levitas. \bibleverse{21} Los levitas se purificaron y lavaron sus ropas.
Entonces Aarón los presentó como ofrenda mecida al Señor. Aarón también
presentó el sacrificio para que estuvieran bien con el Señor para que
estuvieran limpios. \bibleverse{22} Después los levitas vinieron a
realizar su servicio en el Tabernáculo de Reunión bajo la dirección de
Aarón y sus hijos. Siguieron todas las instrucciones sobre los levitas
que el Señor había dado a Moisés.

\hypertarget{el-tiempo-del-deber-de-los-levitas}{%
\subsection{El tiempo del deber de los
levitas}\label{el-tiempo-del-deber-de-los-levitas}}

\bibleverse{23} El Señor le dijo a Moisés: \bibleverse{24} ``Esta regla
se aplica a los levitas. Los mayores de veinticinco años servirán en el
Tabernáculo de Reunión. \footnote{\textbf{8:24} Núm 4,3; Núm 4,23; Núm
  4,30; Núm 4,47}

\bibleverse{25} Sin embargo, una vez que alcancen la edad de cincuenta
años deben retirarse del trabajo y no servirán más. \bibleverse{26}
Todavía pueden ayudar a sus compañeros levitas en sus tareas, pero no
deben hacer el trabajo por sí mismos. Estos son los arreglos en el caso
de los levitas''.

\hypertarget{la-celebraciuxf3n-posterior-a-la-pascua-para-los-inmundos-y-los-viajeros-la-pascua-de-los-extrauxf1os}{%
\subsection{La celebración posterior a la Pascua para los inmundos y los
viajeros; la pascua de los
extraños}\label{la-celebraciuxf3n-posterior-a-la-pascua-para-los-inmundos-y-los-viajeros-la-pascua-de-los-extrauxf1os}}

\hypertarget{section-8}{%
\section{9}\label{section-8}}

\bibleverse{1} El Señor le habló a Moisés en el desierto del Sinaí en el
primer mes, dos años después de que Israel dejara Egipto. Le dijo:
\bibleverse{2} ``Los israelitas deben celebrar la Pascua en el momento
designado. \bibleverse{3} La observarán a la hora requerida, en la tarde
después de la puesta del sol del día catorce de este mes, y lo harán de
acuerdo con sus reglas y normas''.

\bibleverse{4} Moisés hizo un llamado a los israelitas para que
observaran la Pascua. \bibleverse{5} Así que celebraron la Pascua en el
desierto del Sinaí, comenzando por la tarde después de la puesta del sol
del día catorce del primer mes. Los israelitas siguieron todas las
instrucciones que el Señor había dado a Moisés. \bibleverse{6} Sin
embargo, había algunos hombres que eran impuros porque habían estado en
contacto con un cadáver, por lo que no podían celebrar la Pascua ese
día. Fueron a ver a Moisés y Aarón el mismo día \footnote{\textbf{9:6}
  Núm 19,11} \bibleverse{7} y le explicaron a Moisés: ``Somos inmundos
por causa de un cadáver, ¿pero por qué eso significa que no podemos dar
nuestra ofrenda al Señor con los demás israelitas en el momento
oportuno?''

\bibleverse{8} ``Quédense aquí mientras averiguo cuáles son las
instrucciones del Señor respecto a ustedes'', respondió Moisés.

\bibleverse{9} Entonces el Señor le dijo a Moisés: \bibleverse{10}
``Dile a los israelitas: 'Si tú o tus descendientes están sucios por
causa de un cadáver, o están viajando, aún pueden celebrar la Pascua del
Señor. \bibleverse{11} La observarán por la tarde, después de la puesta
del sol, en el día catorce del segundo mes. Comerán el cordero con el
pan sin levadura y las hierbas amargas. \bibleverse{12} No deben dejar
nada de él hasta la mañana siguiente y no deben romper ninguno de sus
huesos. Deberán observar la Pascua de acuerdo con todas las normas.
\bibleverse{13} ``Sin embargo, cualquiera que esté ceremonialmente
limpio y no viaje lejos y que no observe la Pascua debe ser expulsado de
su pueblo, porque no presentó la ofrenda del Señor en el momento
apropiado. Ellos serán responsables de las consecuencias de su pecado.

\bibleverse{14} Cualquier extranjero que viva entre ustedes y que quiera
observar la Pascua del Señor puede hacerlo siguiendo las normas y
preceptos de la Pascua. Las mismas reglas se aplican a los extranjeros
como a ustedes'\,''.

\hypertarget{la-apariciuxf3n-de-la-columna-de-nubes-y-fuego-sobre-el-santuario}{%
\subsection{La aparición de la columna de nubes y fuego sobre el
santuario}\label{la-apariciuxf3n-de-la-columna-de-nubes-y-fuego-sobre-el-santuario}}

\bibleverse{15} La nube cubrió la Tienda del Testimonio (el Tabernáculo)
el día en que fue erigida, y se vio como fuego sobre ella desde la noche
hasta la mañana. \bibleverse{16} Siempre era así. La nube cubría el
Tabernáculo durante el día\footnote{\textbf{9:16} ``Durante el día'':
  Tomado de la Septuaginta.} y por la noche parecía fuego.
\bibleverse{17} Cuando la nube se levantaba sobre la Tienda, los
israelitas marchaban, y cuando la nube se detenía, los israelitas
acampaban allí. \bibleverse{18} Los israelitas se movían cuando el Señor
les decía, y levantaban el campamento cuando el Señor les decía.
Mientras la nube permanecía sobre el Tabernáculo, ellos permanecían
acampados allí. \bibleverse{19} Aunque la nube no se moviera durante
mucho tiempo, los israelitas hicieron lo que el Señor les decía y no
seguían adelante. \bibleverse{20} A veces la nube sólo permanecía sobre
el Tabernáculo durante unos pocos días. Como siempre, siguieron la orden
del Señor de acampar o seguir adelante. \bibleverse{21} A veces la nube
sólo se quedaba durante la noche, así que cuando se levantaban por la
mañana seguían avanzando. Cada vez que la nube se levantaba, de día o de
noche, se marchaban. \bibleverse{22} Si la nube se quedaba en un lugar
durante dos días, o un mes, o más tiempo, los israelitas se quedaban
donde estaban y no se iban mientras la nube permaneciera sobre el
Tabernáculo. Sin embargo, una vez que se levantaba, se iban.
\bibleverse{23} Acampaban cuando el Señor les decía, y se iban cuando él
les decía. Ellos seguían las instrucciones del Señor le daba a Moisés.

\hypertarget{ordenanza-sobre-dos-trompetas-de-plata}{%
\subsection{Ordenanza sobre dos trompetas de
plata}\label{ordenanza-sobre-dos-trompetas-de-plata}}

\hypertarget{section-9}{%
\section{10}\label{section-9}}

\bibleverse{1} El Señor le dijo a Moisés: \bibleverse{2} ``Haz dos
trompetas de plata martillada. Se usarán para convocar a los israelitas
y para hacer que el campamento se mueva. \footnote{\textbf{10:2} Núm
  31,6} \bibleverse{3} Cuando se toquen las dos trompetas, todos los
israelitas se reunirán ante ti en la entrada del Tabernáculo de Reunión.
\bibleverse{4} Pero si sólo se toca una, sólo los líderes de la tribu se
reunirán ante ti. \bibleverse{5} ``Cuando se toque la trompeta, que es
la señal de alarma para salir, los campamentos del lado este deben salir
primero. \bibleverse{6} Cuando se toca la trompeta por segunda vez, los
campamentos del lado sur deben marchar. Esa es su señal para empezar a
moverse. \bibleverse{7} Para convocar a la gente, soplen las trompetas
normalmente, no la señal de alarma fuerte.

\bibleverse{8} Los descendientes de Aarón deben tocar las trompetas.
Esta regulación seguirá vigente en todos los tiempos y para todas las
generaciones futuras. \bibleverse{9} ``Cuando estés en tu propia tierra
y tengas que ir a la batalla contra un enemigo que te haya atacado, toca
la señal de alarma y el Señor tu Dios no te olvidará: te salvará de tus
enemigos.

\bibleverse{10} Toquen las trompetas cuando celebren también, en sus
fiestas regulares y al principio de cada mes. Es decir, cuando traigas
tus holocaustos y tus ofrendas de comunión que serán como un
recordatorio para ti ante tu Dios. Yo soy el Señor tu Dios''.

\hypertarget{salida-del-sinauxed-hacia-el-desierto-de-paran}{%
\subsection{Salida del Sinaí hacia el desierto de
Paran}\label{salida-del-sinauxed-hacia-el-desierto-de-paran}}

\bibleverse{11} Entonces la nube se levantó del Tabernáculo del
Testimonio el vigésimo día del segundo mes del segundo año.
\bibleverse{12} Los israelitas abandonaron el desierto del Sinaí y se
desplazaron de un lugar a otro hasta que la nube se detuvo en el
desierto de Parán.

\hypertarget{descripciuxf3n-del-pedido-de-tren}{%
\subsection{Descripción del pedido de
tren}\label{descripciuxf3n-del-pedido-de-tren}}

\bibleverse{13} Esta fue la primera vez que salieron siguiendo el
mandato del Señor a través de Moisés. \footnote{\textbf{10:13} Núm 1,1-4}
\bibleverse{14} Las divisiones de latribu de Judá fueron las primeras en
marchar bajo su bandera, con Naasón, hijo de Aminadab, al mando.
\bibleverse{15} Natanael, hijo de Zuar, estaba a cargo de la tribu de
Isacar, \bibleverse{16} y Eliab, hijo de Helón, estaba a cargo de la
división tribal de Zabulón. \bibleverse{17} Entonces el Tabernáculo fue
desmontado, y los guersonitas y los meraritas que lo llevaban se
pusieron en marcha. \bibleverse{18} Luego vinieron las divisiones de la
tribu de Rubén, quienes marcharon bajo su bandera, con Elisur, hijo de
Sedeur, a cargo. \bibleverse{19} Selumiel, hijo de Zurishaddai, estaba a
cargo de la tribu de Simeón, \bibleverse{20} y Eliasaf, hijo de Deuel,
estaba a cargo de la tribu de Gad.

\bibleverse{21} Entonces los coatitas se pusieron en marcha, llevando
los objetos sagrados. El tabernáculo se colocaría antes de que llegaran.

\bibleverse{22} Luego vinieron las divisiones de la tribu de Efraín, y
marcharon bajo su bandera, con Elisama, hijo de Amihud a cargo.
\bibleverse{23} Gamaliel, hijo de Pedasur, estaba a cargo de la tribu de
Manasés, \bibleverse{24} y Abidán, hijo de Gedeón, estaba a cargo de la
tribu de Benjamín.

\bibleverse{25} Finalmente llegaron las divisiones de Dan que marcharon
bajo su bandera, defendiendo la retaguardia de todos los grupos
tribales, con Ahiezer, hijo de Amisadai, a cargo. \bibleverse{26}
Pagiel, hijo de Ocrán, estaba a cargo de la tribu de Aser,
\bibleverse{27} y Ajirá, hijo de Enán, estaba a cargo de la tribu de
Neftalí. \bibleverse{28} Este era el orden en el se desplazaban las
tribus de Israel.

\hypertarget{moisuxe9s-intenta-ganarse-a-su-cuuxf1ado-hobab-como-guuxeda-para-el-viaje-hacia-adelante}{%
\subsection{Moisés intenta ganarse a su cuñado Hobab como guía para el
viaje hacia
adelante}\label{moisuxe9s-intenta-ganarse-a-su-cuuxf1ado-hobab-como-guuxeda-para-el-viaje-hacia-adelante}}

\bibleverse{29} Moisés le explicó a Hobab, el hijo del suegro de Moisés,
Reuel, el madianita,\footnote{\textbf{10:29} Esto convertía a Hobab en
  el cuñado de Moisés.} ``Nos vamos al lugar que el Señor prometió
diciendo: `Te daré esta tierra'. Ven con nosotros y seremos buenos
contigo, porque el Señor le ha prometido cosas buenas a Israel''.

\bibleverse{30} ``No, no me iré, volveré a mi país y a mi pueblo'',
respondió Hobab.

\bibleverse{31} ``Por favor, no nos abandones ahora'', le dijo Moisés,
``porque tú eres el único que sabe dónde debemos acampar en el desierto
y puedes guiarnos. \bibleverse{32} Si vienes con nosotros, todo lo bueno
que el Señor nos de como bendición lo compartiremos contigo''.

\hypertarget{la-partida-del-monte-de-dios-bajo-la-guuxeda-del-arca}{%
\subsection{La partida del monte de Dios bajo la guía del
arca}\label{la-partida-del-monte-de-dios-bajo-la-guuxeda-del-arca}}

\bibleverse{33} Se fueron de la montaña del Señor para hacer un viaje de
tres días, y El Arca del Pacto del Señor les mostró el camino durante
estos tres días para encontrar un lugar para acampar. \bibleverse{34} La
nube del Señor estuvo sobre ellos durante el día mientras se alejaban
del campamento. \footnote{\textbf{10:34} Éxod 13,21} \bibleverse{35}
Cada vez que el Arca avanzaba, Moisés gritaba: ``Levántate, Señor, y que
tus enemigos se dispersen, y que los que te odian huyan de ti''.
\footnote{\textbf{10:35} Sal 68,2; Sal 132,8}

\bibleverse{36} Cada vez que se detenía, Moisés gritaba: ``Vuelve,
Señor, a los miles y miles del pueblo de Israel''.

\hypertarget{el-murmullo-de-la-gente-y-la-fogata-en-thabera}{%
\subsection{El murmullo de la gente y la fogata en
Thabera}\label{el-murmullo-de-la-gente-y-la-fogata-en-thabera}}

\hypertarget{section-10}{%
\section{11}\label{section-10}}

\bibleverse{1} No pasó mucho tiempo antes de que la gente empezara a
quejarse de lo mucho que estaban sufriendo. Cuando el Señor escuchó lo
que decían, se enfadó. El fuego del Señor los quemó, destruyendo algunos
que iban por los extremos del campamento. \footnote{\textbf{11:1} Lev
  10,2} \bibleverse{2} El pueblo clamó a Moisés por ayuda. Entones él
oró al Señor y el fuego se apagó. \bibleverse{3} Ese lugar se llamó
Taberá,\footnote{\textbf{11:3} ``Taberá'' significa ``arder''.} porque
el fuego del Señor los quemó.

\hypertarget{la-gente-se-queja-de-la-comida}{%
\subsection{La gente se queja de la
comida}\label{la-gente-se-queja-de-la-comida}}

\bibleverse{4} Entonces un grupo de alborotadores\footnote{\textbf{11:4}
  Generalmente asociado con una ``multitud mixta'' que salió de Egipto
  con los Israelitas (ver Éxodo 12:38)} que estaba entre ellos tenían
antojos de comida tan intensos que afectaron a los israelitas que
empezaron a llorar de nuevo, preguntando ``¿Quién va a conseguirnos algo
de carne para comer? \bibleverse{5} Recuerden todo el pescado que
comíamos en Egipto y que no nos costaba nada, así como los pepinos, los
melones, los puerros, las cebollas y el ajo. \bibleverse{6} ¡Pero ahora
nos estamos desvaneciendo aquí! ¡Lo único que vemos es este maná!''
\bibleverse{7} El maná tenía la apariencia de semillas de cilantro, de
color claro como la resina. \footnote{\textbf{11:7} Éxod 16,14-31}
\bibleverse{8} El pueblo salía a recogerlo, lo molíancon un molino o lo
trituraban en un mortero; luego lo hervirían en una olla y lo
convertirían en pan plano. EL sabor era como de pasteles hechos con el
mejor aceite de oliva. \bibleverse{9} Cuando el rocío descendíasobre el
campamento por la noche, el maná bajaba con él.

\hypertarget{el-lamento-de-moisuxe9s-ante-dios}{%
\subsection{El lamento de Moisés ante
Dios}\label{el-lamento-de-moisuxe9s-ante-dios}}

\bibleverse{10} Moisés escuchó a todas las familias llorando a la
entrada de sus tiendas. El Señor se enfadó mucho, y Moisés también se
enfadó. \bibleverse{11} Le preguntó al Señor: ``¿Por qué me has puesto
las cosas tan difíciles a mí, tu siervo? ¿Por qué estás tan descontento
conmigo que me has puesto la pesada responsabilidad de toda esta gente?
\bibleverse{12} ¿Acaso son mis hijos? ¿Los di a luz para que me dijeras:
`Sujétalos en tu pecho como una nodriza que lleva un bebé' y luego tener
que llevarlos a la tierra que les prometiste a sus antepasados?
\bibleverse{13} ¿De dónde se supone que voy a sacar carne para todos
ellos? Se siguen quejando de mí, `¡Consíguenos algo de carne para
comer!' \bibleverse{14} No puedo seguir soportando a todo este pueblo yo
solo. ¡Es demasiado! \bibleverse{15} Si esta es la forma en que me vas a
tratar, entonces por favor mátame ahora para no tener que enfrentarme a
esta depresión que me abruma. Por favor, concédeme esta petición''.

\hypertarget{ordenanza-de-dios-nombramiento-de-setenta-asistentes-de-moisuxe9s-la-promesa-de-dios-de-donaciuxf3n-de-carne-la-respuesta-incruxe9dula-de-moisuxe9s}{%
\subsection{Ordenanza de Dios (nombramiento de setenta asistentes de
Moisés); La promesa de Dios de donación de carne; la respuesta incrédula
de
Moisés}\label{ordenanza-de-dios-nombramiento-de-setenta-asistentes-de-moisuxe9s-la-promesa-de-dios-de-donaciuxf3n-de-carne-la-respuesta-incruxe9dula-de-moisuxe9s}}

\bibleverse{16} Entonces el Señor le dijo a Moisés: ``Trae ante mí
setenta ancianos israelitas que sepas que son respetados como líderes
por el pueblo. Llévalos al Tabernáculo de Reunión. Se quedarán allí
contigo. \footnote{\textbf{11:16} Éxod 18,21; Éxod 24,1} \bibleverse{17}
Yo bajaré y hablaré contigo allí. Tomaré un poco del Espíritu que tienes
y se lo daré. Ellos tomarán parte de la responsabilidad del pueblo para
que no tengas que soportarlo todo tú solo.

\bibleverse{18} ``Dile al pueblo: Purifíquense, porque mañana tendrán
carne para comer, pues se han quejado y el Señor ha oído su petición:
`¿Quién nos va a dar carne para comer? Estábamos mejor en Egipto'. Así
que el Señor va a proveerles carne para comer. \bibleverse{19} La
comerán, no sólo por un día o dos, ni por cinco, diez o veinte días.
\bibleverse{20} La comerán durante un mes entero hasta que vomiten y les
salga por las narices, porque han rechazado al Señor, que está aquí con
ustedes, y se han quejado de él diciendo: `¿Por qué se nos ocurrió salir
de Egipto?'\,''

\bibleverse{21} Pero Moisés respondió: ``Estoy aquí con 600. 000
personas y me dices: `Les voy a dar carne y la comerán durante un mes'?
\bibleverse{22} ún si todos nuestros rebaños y manadas fueran
sacrificados, no sería suficiente para ellos. Incluso si todos los peces
del mar fueran capturados, ¡no sería suficiente para todos ellos!''
\footnote{\textbf{11:22} Juan 6,7}

\bibleverse{23} ``¿No tiene el Señor el poder de hacer eso?'' ,
respondió el Señor. ``¡Ahora vas a ver con tus ojos si lo que he dicho
sucederá o no!'' \footnote{\textbf{11:23} Is 50,2; Is 59,1}

\hypertarget{el-entusiasmo-profuxe9tico-de-los-setenta-ancianos}{%
\subsection{El entusiasmo profético de los setenta
ancianos}\label{el-entusiasmo-profuxe9tico-de-los-setenta-ancianos}}

\bibleverse{24} Entonces Moisés fue y compartió con el pueblo lo que el
Señor dijo. Convocó a setenta ancianos del pueblo y los hizo ponerse de
pie alrededor de la tienda. \bibleverse{25} Entonces el Señor descendió
y le habló. El Señor tomó algo del Espíritu que Moisés tenía y se lo
dio. Ellos profetizaron, pero esto no volvió a suceder. \bibleverse{26}
Sin embargo, dos hombres llamados Eldad y Medad se habían quedado en el
campamento, y el Espíritu vino sobre ellos también. (Habían sido puestos
en la lista de los setenta ancianos, pero no habían ido a la tienda.
Pero profetizaron donde estaban en el campamento de todos modos).
\bibleverse{27} Un joven corrió y le dijo a Moisés: ``Eldad y Medad
están profetizando en el campamento''.

\bibleverse{28} Josué, hijo de Nun, que había sido asistente de Moisés
desde joven, reaccionó diciendo: ``¡Moisés, mi señor, tienes que
detenerlos!'' \footnote{\textbf{11:28} Núm 13,16; Éxod 24,13}

\bibleverse{29} ``¿Estás celoso de mi reputación?'' respondió Moisés.
``¡Deseo que cada uno en el pueblo del Señor sea profeta y que el Señor
les dé su espíritu a todos!'' \footnote{\textbf{11:29} Mar 9,39; Jl 3,1}

\bibleverse{30} Entonces Moisés volvió al campamento con los ancianos de
Israel.

\hypertarget{alimentaciuxf3n-de-codornices-juicio-de-dios-las-tumbas-del-placer}{%
\subsection{Alimentación de codornices; Juicio de Dios; las tumbas del
placer}\label{alimentaciuxf3n-de-codornices-juicio-de-dios-las-tumbas-del-placer}}

\bibleverse{31} El Señor envió un viento que sopló codornices desde el
mar y las hizo caer cerca del campamento. Cubrieron el suelo hasta una
profundidad de unos dos codos y se extendieron durante un día de viaje
en todas direcciones del campamento. \footnote{\textbf{11:31} Éxod 16,13}
\bibleverse{32} Durante todo ese día y noche, y durante todo el día
siguiente, el pueblo siguió recogiendo codornices. Todos recolectaron al
menos diez homers,\footnote{\textbf{11:32} Estimado en un volumen de 220
  litros.} y las repartieron por todo el campamento. \bibleverse{33}
Pero mientras la gente seguía mordiendo la carne, incluso antes de que
la masticaran, el Señor mostró su ardiente ira contra ellos, matando a
algunos de ellos con una grave enfermedad. \bibleverse{34} Llamaron a
ese lugar Quibrot-Hatavá,\footnote{\textbf{11:34} Que significa:
  ``sepulturas de glotonería''.} porque allí enterraron a la gente que
tenía estos intensos antojos de comida.

\bibleverse{35} Luego se trasladaron de Quibrot-Hataváhacia Jazerot,
donde permanecieron durante algún tiempo.

\hypertarget{la-rebeliuxf3n-de-miriam-y-aaruxf3n-contra-moisuxe9s}{%
\subsection{La rebelión de Miriam y Aarón contra
Moisés}\label{la-rebeliuxf3n-de-miriam-y-aaruxf3n-contra-moisuxe9s}}

\hypertarget{section-11}{%
\section{12}\label{section-11}}

\bibleverse{1} Pero Miriam y Aarón criticaban a Moisés por su esposa,
pues Moisés se había casado con una mujer etíope.\footnote{\textbf{12:1}
  ``Etíope'': literalmente, ``Cusita'', refiriéndose a la tierra que
  quedaba en el sureste de Egipto.} \footnote{\textbf{12:1} Éxod 2,21}
\bibleverse{2} ``¿Acaso el Señor solo habla a través de Moisés?'' ,
cuestionaban. ``¿No habla también a través de nosotros?'' Y el Señor
escuchó todo esto. \bibleverse{3} Moisés era un hombre muy humilde, más
que nadie en la tierra.

\hypertarget{dios-estuxe1-defendiendo-a-moisuxe9s-el-castigo-de-miriam}{%
\subsection{Dios está defendiendo a Moisés; El castigo de
miriam}\label{dios-estuxe1-defendiendo-a-moisuxe9s-el-castigo-de-miriam}}

\bibleverse{4} De repente el Señor llamó a Moisés, Aarón y Miriam,
diciéndoles: ``Ustedes tres, vengan al Tabernáculo de Reunión''. Y los
tres lo hicieron.

\bibleverse{5} El Señor bajó en una columna de nube y se paró en la
entrada de la Tienda. Llamó a Aarón y a Miriam y ellos se adelantaron.
\bibleverse{6} ``Escuchen mis palabras, les dijo. Si tuvieran profetas,
yo, el Señor, me revelaría a ellos en visiones; me comunicaría con ellos
en sueños. \bibleverse{7} Pero no es así con mi siervo Moisés, que de
todo mi pueblo es el que me es fiel. \footnote{\textbf{12:7} Heb 3,2}
\bibleverse{8} Yo hablo con él personalmente, cara a cara. Hablo
claramente, y no con acertijos. Él ve la semejanza del Señor. Entonces,
¿por qué no tuvieron miedo al criticar a mi siervo Moisés?'' \footnote{\textbf{12:8}
  Éxod 33,11; Éxod 33,23} \bibleverse{9} Entonces el Señor se enfadó con
ellos, y se fue.

\bibleverse{10} Cuando la nube se elevó sobre la Tienda, la piel de
Miriam se volvió repentinamente blanca por la lepra. Aarón se volvió a
mirar y vio que tenía lepra. \footnote{\textbf{12:10} Deut 24,9}

\hypertarget{la-intercesiuxf3n-de-aaruxf3n-y-moisuxe9s-la-respuesta-de-dios-la-curaciuxf3n-de-miriam-llegada-al-desierto-de-paran}{%
\subsection{La intercesión de Aarón y Moisés; La respuesta de Dios; La
curación de Miriam; Llegada al desierto de
Paran}\label{la-intercesiuxf3n-de-aaruxf3n-y-moisuxe9s-la-respuesta-de-dios-la-curaciuxf3n-de-miriam-llegada-al-desierto-de-paran}}

\bibleverse{11} Le dijo a Moisés: ``Señor mío, por favor no nos
castigues por este pecado que hemos cometido tan estúpidamente.
\bibleverse{12} Por favor, no dejes que me convierta en un moribundo
cuya carne ya se está pudriendo cuando nace!''

\bibleverse{13} Moisés clamó al Señor: ``¡Dios, por favor, cúrala!''

\bibleverse{14} Pero el Señor le respondió a Moisés: ``Si su padre le
hubiera escupido en la cara, ¿no habría sido deshonrosa durante siete
días? Mantenla aislada fuera del campamento durante siete días, y luego
podráregresar''. \footnote{\textbf{12:14} Lev 13,46}

\bibleverse{15} Miriam quedó en aislamiento fuera del campamento durante
siete días, y el pueblo no avanzó hasta que fue llevada de vuelta.
\bibleverse{16} Entonces el pueblo se fue de Jazerot y se instaló en el
desierto de Parán.

\hypertarget{envuxedo-de-los-doce-exploradores}{%
\subsection{Envío de los doce
exploradores}\label{envuxedo-de-los-doce-exploradores}}

\hypertarget{section-12}{%
\section{13}\label{section-12}}

\bibleverse{1} El Señor le dijo a Moisés, \bibleverse{2} ``Envía algunos
hombres a explorar la tierra de Canaán, el país que le doy a los
israelitas. Escoge a uno de los líderes de cada una de las tribus para
que vaya y haga esto''.

\bibleverse{3} Moisés hizo lo que el Señor le había ordenado y envió a
los hombres desde el desierto de Parán. Todos ellos eran líderes de los
israelitas. \bibleverse{4} Sus nombres eran: Samúa hijo de Zacur, de la
tribu de Rubén. \bibleverse{5} Safat, hijo de Hori, de la tribu de
Simeón. \bibleverse{6} Caleb, hijo de Jefone, de la tribu de Judá.
\bibleverse{7} Igal, hijo de José, de la tribu de Isacar. \bibleverse{8}
Oseas,\footnote{\textbf{13:8} También llamado Josué. Ver el versículo
  16.} hijo de Nun, de la tribu de Efraín. \bibleverse{9} Palti hijo de
Raphu, de la tribu de Benjamín. \footnote{\textbf{13:9} Núm 13,16; 1Cró
  7,27} \bibleverse{10} Gaddiel, hijo de Sodi, de la tribu de Zabulón.
\bibleverse{11} Gaddi, hijo de Susi, de la tribu de Manasés (una tribu
de José). \bibleverse{12} Amiel, hijo de Gemalli, de la tribu de Dan.
\bibleverse{13} Sethur, hijo de Miguel, de la tribu de Aser.
\bibleverse{14} Nahbi, hijo de Vophsi, de la tribu de Neftalí.
\bibleverse{15} Geuel, hijo de Machi, de la tribu de Gad.
\bibleverse{16} Estos eran los nombres de los hombres que Moisés envió a
explorar el país. Moisés le puso por nombre Josué a Oseas.
\bibleverse{17} Moisés los envió a explorar la tierra de Canaán,
diciéndoles: ``Pasen por el Néguev y entren en las montañas.

\hypertarget{la-instrucciuxf3n-de-moisuxe9s-a-los-espuxedas}{%
\subsection{La instrucción de Moisés a los
espías}\label{la-instrucciuxf3n-de-moisuxe9s-a-los-espuxedas}}

\bibleverse{18} Vean cómo es el lugar, y averigüen acercade la gente que
vive allí, ¿son fuertes o débiles? ¿Son muchos o pocos? \bibleverse{19}
¿La tierra donde viven es buena o mala? ¿Son sus ciudades como campos
abiertos, o tienen muros defensivos? \bibleverse{20} ¿Es el suelo
productivo o no? ¿Es forestal? Sean valientes y traigan algunos de los
frutos del país''. (Era el comienzo de la vendimia). \bibleverse{21} Así
que los hombres fueron y exploraron la tierra desde el desierto de Zin
hasta Rejob, el oso Lebó Jamat.

\hypertarget{explorando-la-tierra}{%
\subsection{Explorando la tierra}\label{explorando-la-tierra}}

\bibleverse{22} Atravesaron el Néguev y llegaron a Hebrón donde vivían
Ahiman, Seshai y Talmai, los descendientes de Anac. Esta ciudad fue
construida siete años antes que la ciudad egipcia de Zoán.
\bibleverse{23} Cuando llegaron al Valle de Escol cortaron una rama que
tenía un solo racimo de uvas. Tenían que cargarla en un palo sostenido
entre dos hombres. También recogieron algunas granadas e higos.
\bibleverse{24} (El lugar fue llamado el Valle de Escol\footnote{\textbf{13:24}
  ``Escol'' significa ``manojo''.} por el racimo de uvas que tomaron de
allí). \bibleverse{25} Cuarenta días después los hombres regresaron de
explorar el país.

\hypertarget{regreso-e-informe-de-los-emisarios}{%
\subsection{Regreso e informe de los
emisarios}\label{regreso-e-informe-de-los-emisarios}}

\bibleverse{26} Fueron a ver a Moisés y Aarón, y todos los israelitas se
reunieron allí en su campamento en Cades, en el desierto de Parán.
Dieron un informe ante todos y les mostraron los frutos que habían
traído del país. \bibleverse{27} Este es el informe que dieron a Moisés:
``Fuimos y exploramos el país al que nos enviaste, y es definitivamente
muy productivo, como si fluyera leche y miel. ¡Miren algunas de sus
frutas! \bibleverse{28} Pero la gente que vive allí es fuerte, y sus
pueblos son grandes y tienen muros defensivos. También vimos algunos
descendientes de Anac allí. \footnote{\textbf{13:28} Éxod 3,8; Éxod 3,17}
\bibleverse{29} Los amalecitas viven en el Néguev. Los hititas, jebuseos
y amorreos viven en las colinas. Los cananeos viven en la costa del mar
y también al lado del Jordán''.

\bibleverse{30} Entonces Caleb pidió silencio mientras la gente se
paraba delante de Moisés y les decía: ``Vamos a tomar la tierra. Podemos
conquistar el país, ¡sin duda!''

\hypertarget{las-palabras-tranquilizadoras-de-caleb-y-las-palabras-desalentadoras-de-los-otros-exploradores}{%
\subsection{Las palabras tranquilizadoras de Caleb y las palabras
desalentadoras de los otros
exploradores}\label{las-palabras-tranquilizadoras-de-caleb-y-las-palabras-desalentadoras-de-los-otros-exploradores}}

\bibleverse{31} Pero los hombres que habían ido con él no estaban de
acuerdo. ``¡No podemos ir a luchar contra este pueblo! ¡Son mucho más
fuertes que nosotros!'' \bibleverse{32} difundieron un informe negativo
entre los israelitas sobre el país que habían explorado. Le dijeron al
pueblo: ``El país que exploramos destruye a la gente que vive allí.
Además todas las personas que vimos eran muy grandes! \bibleverse{33}
Incluso vimos gigantes allí, ¡son descendientes del gigante Anac!
Comparados con ellos pareceríamos saltamontes, ¡y así debimos parecerles
a ellos también!''\footnote{\textbf{13:33} Deut 9,2}

\hypertarget{el-efecto-del-informe-en-la-gente}{%
\subsection{El efecto del informe en la
gente}\label{el-efecto-del-informe-en-la-gente}}

\hypertarget{section-13}{%
\section{14}\label{section-13}}

\bibleverse{1} Entonces todos los que estaban allí gritaron toda la
noche. \bibleverse{2} Todos los israelitas fueron y se quejaron a Moisés
y Aarón, diciéndoles: ``¡Ojalá hubiéramos muerto en Egipto, o aquí en
este desierto! \bibleverse{3} ¿Por qué el Señor nos lleva a este país
sólo para que nos maten? ¡Nuestras esposas e hijos serán capturados y
llevados como esclavos! ¿No sería mejor que volviéramos a Egipto?''
\footnote{\textbf{14:3} Sal 106,24} \bibleverse{4} Se dijeron unos a
otros: ``Elijamos un nuevo líder y volvamos a Egipto''.

\bibleverse{5} Moisés y Aarón se postraron en el suelo frente a todos
los israelitas reunidos.

\hypertarget{el-intento-fallido-de-apaciguamiento-de-joshua-y-caleb}{%
\subsection{El intento fallido de apaciguamiento de Joshua y
Caleb}\label{el-intento-fallido-de-apaciguamiento-de-joshua-y-caleb}}

\bibleverse{6} Josué, hijo de Nun, y Caleb, hijo de Jefone, estaban
allí. Habían sido parte del grupo que había ido a espiar la tierra. Se
rasgaron la ropa,\footnote{\textbf{14:6} En señal de duelo y emoción
  intensa.} \footnote{\textbf{14:6} Núm 13,16; Núm 13,30} \bibleverse{7}
y les dijeron a los israelitas: ``El país que viajamos y exploramos
tiene muy buena tierra. \bibleverse{8} Si el Señor está contento con
nosotros, nos llevará allí y nos la dará, una tierra tan productiva que
es como si fluyera leche y miel. \bibleverse{9} No se rebelen ni luchen
contra el Señor. No hay que tener miedo de la gente que vive en el
campo, ¡podemos cogerlos fácilmente! Están indefensos y el Señor está
con nosotros. ¡No les tengan miedo!''

\hypertarget{ira-de-dios-la-exitosa-intercesiuxf3n-de-moisuxe9s-el-juicio-divino}{%
\subsection{Ira de Dios; la exitosa intercesión de Moisés; el juicio
divino}\label{ira-de-dios-la-exitosa-intercesiuxf3n-de-moisuxe9s-el-juicio-divino}}

\bibleverse{10} En respuesta, todo el pueblo gritó: ``¡Apedréenlos!''
Pero la gloria del Señor apareció de repente en el Tabernáculo de
Reunión, justo en medio de los israelitas. \footnote{\textbf{14:10} Éxod
  17,4; Éxod 16,10}

\bibleverse{11} El Señor le dijo a Moisés: ``¿Hasta cuándo me va a
rechazar este pueblo? ¿Cuánto tiempo va a rechazar esta gente la
confianza en mí, a pesar de todos los milagros que he hecho delante de
ellos? \bibleverse{12} Voy a enfermarlos con una enfermedad y matarlos.
Entonces los convertiré en una nación más grande y fuerte que ellos''.

\bibleverse{13} Pero Moisés le dijo al Señor: ``¡Los egipcios se
enterarán de esto! Fue por tu poder que sacaste a los israelitas de
entre ellos. \bibleverse{14} Ellos le contarán todo al pueblo que vive
en este país. Ya han oído que tú, Señor, estás con nosotros los
israelitas, que tú, Señor, te muestras cara a cara, que tu nube los
vigila, y que los conduces por una columna de nube durante el día y una
columna de fuego por la noche. \bibleverse{15} Si matas a toda esta
gente de una sola vez, las naciones que han oído hablar de ti dirán:
\bibleverse{16} `El Señor mató a este pueblo en el desierto porque no
pudo llevarlos al país que prometió darles. Los ha matado a todos en el
desierto'. \footnote{\textbf{14:16} Deut 9,28} \bibleverse{17} ``Ahora,
Señor, por favor demuestra el alcance de tu poder tal como lo has dicho:
\bibleverse{18} El Señor es lento para enojarse y está lleno de amor
confiable, perdonando el pecado y la rebelión. Sin embargo, no permitirá
que los culpables queden impunes, trayendo las consecuencias del pecado
de los padres a sus hijos, nietos y bisnietos. \bibleverse{19} Por
favor, perdona el pecado de estas personas ya que tu amor digno de
confianza es tan grande, de la misma manera que los has perdonado desde
que salieron de Egipto hasta ahora''.

\bibleverse{20} ``Los he perdonado como me lo pediste'', respondió el
Señor. \bibleverse{21} ``Pero aún así, juro por mi vida y toda mi gloria
en la tierra, \footnote{\textbf{14:21} Éxod 9,16} \bibleverse{22} ni uno
solo de los que vieron mi gloria y los milagros que hice en Egipto y en
el desierto, sino que me provocaron y se negaron a obedecerme una y otra
vez; \footnote{\textbf{14:22} ``Una y otra vez'': literalmente, ``diez
  veces'', pero se cree que es una expresión que se refiere a múltiples
  ocasiones.} \bibleverse{23} ni uno solo de ellos va a ver el país que
prometí a sus antepasados. Ninguno de los que me rechazaron lo verá.
\bibleverse{24} ``Pero como mi siervo Caleb tiene un espíritu totalmente
diferente y está totalmente comprometido conmigo, lo llevaré al país que
visitó, y sus descendientes serán los dueños. \footnote{\textbf{14:24}
  Jos 14,6; Jos 14,9} \bibleverse{25} Como los amalecitas y los cananeos
viven en los valles, mañana deberán dar la vuelta y volver al desierto,
tomando la ruta hacia el Mar Rojo''.

\hypertarget{el-castigo-de-dios-para-las-personas-y-los-espuxedas-se-especifica-con-muxe1s-detalle}{%
\subsection{El castigo de Dios para las personas y los espías se
especifica con más
detalle}\label{el-castigo-de-dios-para-las-personas-y-los-espuxedas-se-especifica-con-muxe1s-detalle}}

\bibleverse{26} El Señor le dijo a Moisés y Aarón, \bibleverse{27}
``¿Cuánto tiempo más me van a criticar estos malvados? Ya he oído lo que
dicen, haciendo quejas en mi contra. \bibleverse{28} Ve y diles: Juro
por mi propia vida, declara el Señor, que cumpliré sus deseos, ¡créanme!
\bibleverse{29} Todos ustedes morirán en este desierto, todos los que
fueron registrados en el censo que contó a los mayores de veinte años, y
será porque se quejaron contra mi. \bibleverse{30} Ninguno de ustedes
entrará en el país que prometí darles, excepto Caleb, hijo de Jefone, y
Josué, hijo de Nun. \bibleverse{31} Sin embargo, me llevaré a sus hijos,
los que dijeron que serían llevados como botín, al país que ustedes
rechazaron, y ellos sí lo apreciarán. \bibleverse{32} Pero todos ustedes
van a morir en este desierto. \bibleverse{33} Tus hijos vagarán por el
desierto durante cuarenta años, sufriendo por su falta de confianza,
hasta que todos sus cuerpos estén enterrados en el desierto.
\bibleverse{34} ``Así como han explorado el país durante cuarenta días,
su castigo por sus pecados será de cuarenta años, un año por cada día, y
verán lo que ocurre cuando me opongo a ustedes. \bibleverse{35} ¡Yo, el
Señor, así lo he dicho! Verán por ustedes mismos que haré esto con estos
malvados israelitas que se han unido para oponerse a mí. Sus vidas
acabarán en el desierto, y morirán allí''.

\hypertarget{muerte-repentina-de-los-espuxedas-excepto-josuuxe9-y-caleb}{%
\subsection{Muerte repentina de los espías excepto Josué y
Caleb}\label{muerte-repentina-de-los-espuxedas-excepto-josuuxe9-y-caleb}}

\bibleverse{36} Los hombres que Moisés había enviado a explorar el país
-- los que regresaron y porque dieron un mal informe hicieron que todos
los israelitas se quejaran contra el Señor -- \footnote{\textbf{14:36}
  1Cor 10,5; 1Cor 10,10; Jds 1,5} \bibleverse{37} los hombres que dieron
el mal informe murieron ante el Señor de la peste. \bibleverse{38} Los
únicos que vivieron fueron Josué hijo de Nun y Caleb hijo de Jefone de
los que fueron a explorar el país.

\hypertarget{arrepentimiento-del-pueblo-el-intento-fallido-de-penetrar-en-el-pauxeds-enemigo}{%
\subsection{Arrepentimiento del pueblo; el intento fallido de penetrar
en el país
enemigo}\label{arrepentimiento-del-pueblo-el-intento-fallido-de-penetrar-en-el-pauxeds-enemigo}}

\bibleverse{39} Cuando Moisés dijo a los israelitas lo que el Señor
había dicho estaban muy, muy tristes. \bibleverse{40} Se levantaron
temprano a la mañana siguiente planeando ir a las colinas. ``Sí,
realmente pecamos'', dijeron, ``pero ahora estamos aquí e iremos donde
el Señor nos dijo''. \footnote{\textbf{14:40} Núm 13,17}

\bibleverse{41} Pero Moisés se opuso. ``¿Por qué desobedecen la orden
del Señor? ¡No tendrán éxito en su plan! \bibleverse{42} No intenten
irse, porque sus enemigos los matarán, pues el Señor no está con
ustedes. \bibleverse{43} Los amalecitas y cananeos que viven allí los
atacarán, y morirán por espada. Porque rechazaron al Señor, y no les
ayudará''.

\bibleverse{44} Pero ellos fueron arrogantes y subieron a las colinas,
aunque Moisés y el Arca del Pacto del Señor no se movieron del
campamento. \bibleverse{45} Los amalecitas y cananeos que vivían allí en
las colinas bajaron y atacaron a esos israelitas y los derrotaron, y los
persiguieron hasta Jormá.

\hypertarget{regulaciones-con-respecto-a-las-ofrendas-de-comida-y-bebida-como-adiciuxf3n-a-los-holocaustos-y-las-ofrendas-de-salvaciuxf3n}{%
\subsection{Regulaciones con respecto a las ofrendas de comida y bebida
como adición a los holocaustos y las ofrendas de
salvación}\label{regulaciones-con-respecto-a-las-ofrendas-de-comida-y-bebida-como-adiciuxf3n-a-los-holocaustos-y-las-ofrendas-de-salvaciuxf3n}}

\hypertarget{section-14}{%
\section{15}\label{section-14}}

\bibleverse{1} Entonces el Señor le dijo a Moisés: \bibleverse{2} ``Dile
a los israelitas, 'Estas son la instrucciones sobre lo que deben hacer
una vez que lleguen al país que les doy para vivir: \bibleverse{3}
Cuando traiga una ofrenda al Señor de tu ganado o rebaño (ya sea un
holocausto, un sacrificio para cumplir una promesa que hiciste, o una
ofrenda de libre albedrío o de fiesta) que sea aceptable para el Señor,
\footnote{\textbf{15:3} Lev 7,16} \bibleverse{4} entonces también
presentarás una ofrenda de grano de una décima parte de un efa de la
mejor harina mezclada con un cuarto de hin de aceite de oliva.
\bibleverse{5} Añade un cuarto de hin de vino como ofrenda de bebida al
holocausto o al sacrificio de un cordero.

\bibleverse{6} ``Cuando se trate de un carnero, presenta una ofrenda de
grano de dos décimas de efa de la mejor harina mezclada con un tercio de
hin de aceite de oliva, \bibleverse{7} junto con un tercio de hin de
vino como ofrenda de bebida, todo ello para ser aceptable al Señor.
\bibleverse{8} ``Cuando traigas un novillo como holocausto o sacrificio
para cumplir una promesa que hiciste o como ofrenda de paz al Señor,
\bibleverse{9} también llevarás con el novillo una ofrenda de grano de
tres décimas de efa de la mejor harina mezclada con medio hin de aceite
de oliva. \bibleverse{10} Añade medio hin de vino como ofrenda de
bebida. Todo esto es una ofrenda para ser aceptable al Señor.
\bibleverse{11} ``Esto debe hacerse por cada toro, carnero, cordero o
cabra que se traiga como ofrenda.\footnote{\textbf{15:11} ``Que se
  traiga como ofrenda'': añadido para mayor claridad.} \bibleverse{12}
Esto es lo que tienes que hacer para cada uno, sin importar cuántos
sean.

\bibleverse{13} Todo israelita debe seguir estas instrucciones cuando
presente una ofrenda que sea aceptable para el Señor. \bibleverse{14}
Esto también se aplica a todas las generaciones futuras que si un
extranjero que vive entre ustedes o cualquier otra persona entre ustedes
desea presentar una ofrenda aceptable para el Señor: deben hacer
exactamente lo que ustedes hacen. \bibleverse{15} Toda la congregación
debe tener las mismas reglas para ustedes y para el extranjero que vive
entre ustedes. Esta es una ley permanente para todas las generaciones
futuras. Tú y el extranjero deben ser tratados de la misma manera ante
la ley. \footnote{\textbf{15:15} Éxod 12,49} \bibleverse{16} Las mismas
reglas y normas se aplican a ustedes y al extranjero que vive entre
ustedes'\,''.

\hypertarget{disposiciuxf3n-sobre-los-primeros-pasteles}{%
\subsection{Disposición sobre los primeros
pasteles}\label{disposiciuxf3n-sobre-los-primeros-pasteles}}

\bibleverse{17} El Señor le dijo a Moisés: \bibleverse{18} ``Diles a los
israelitas: 'Cuando lleguen al país al que yo los llevo \bibleverse{19}
y cománde los alimentos que allí se producen, darán parte de ellos como
ofrenda al Señor. \bibleverse{20} Darán como ofrenda una parte de la
harina que usen para hacer los panes, y la presentarás como una ofrenda
de la era. \bibleverse{21} Para todas las generaciones futuras, darás al
Señor una ofrenda de la primera de tus harinas.

\hypertarget{reglas-con-respecto-a-las-ofrendas-por-el-pecado-por-obrar-mal-involuntariamente-impunidad-por-transgresiones-intencionales}{%
\subsection{Reglas con respecto a las ofrendas por el pecado por obrar
mal involuntariamente; Impunidad por transgresiones
intencionales}\label{reglas-con-respecto-a-las-ofrendas-por-el-pecado-por-obrar-mal-involuntariamente-impunidad-por-transgresiones-intencionales}}

\bibleverse{22} ``Ahora bien, si pecan colectivamente sin querer y no
siguen todas estas instrucciones que el Señor ha dado a Moisés,
\footnote{\textbf{15:22} Lev 4,2; Lev 4,13} \bibleverse{23} s decir,
todo lo que el Señor les ha ordenado hacer a través de Moisés desde el
momento en que el Señor les dio y para todas las generaciones futuras,
\bibleverse{24} y si se hizo sin querer y sin que todos lo supieran,
entonces toda la congregación debe presentar un novillo como holocausto
para ser aceptado por el Señor, junto con su ofrenda de grano y su
libación presentada según las reglas, así como un macho cabrío como
ofrenda por el pecado. \bibleverse{25} De esta manera el sacerdote debe
hacer que toda la congregación de Israel esté bien con el Señor para que
puedan ser perdonados, porque el pecado fue involuntario y han
presentado al Señor un holocausto y una ofrenda por el pecado, ofrecida
ante el Señor por su pecado involuntario. \bibleverse{26} Entonces toda
la congregación de Israel y los extranjeros que viven entre ellos serán
perdonados, porque el pueblo pecó sin intención.

\bibleverse{27} ``En el caso de un individuo que peca sin intención,
deben presentar una cabra hembra de un año como ofrenda por el pecado.
\bibleverse{28} El sacerdote hará que la persona que pecó sin querer
esté en su derecho ante el Señor. Una vez que hayan sido expiados, serán
perdonados. \bibleverse{29} Aplicarás la misma ley para el que peca por
error a un israelita o a un extranjero que viva entre ustedes.

\bibleverse{30} ``Pero la persona que peca a manera de desafío, ya sea
un israelita o un extranjero, está blasfemando\footnote{\textbf{15:30}
  ``Blasfemando'': en el sentido de abusar deliberadamente del Señor.}
al Señor. Serán expulsados de su pueblo. \footnote{\textbf{15:30} Hech
  13,38; Heb 10,26-27} \bibleverse{31} Deben ser expulsados, porque han
tratado la palabra del Señor con desprecio y han quebrantado su
mandamiento. Son responsables de las consecuencias de su propia
culpa'''.

\hypertarget{informe-de-la-lapidaciuxf3n-de-un-abusador-del-suxe1bado}{%
\subsection{Informe de la lapidación de un abusador del
sábado}\label{informe-de-la-lapidaciuxf3n-de-un-abusador-del-suxe1bado}}

\bibleverse{32} Durante el tiempo en que los israelitas vagaban por el
desierto, un hombre fue sorprendido recogiendo leña en el día Sábado.
\bibleverse{33} Las personas que lo encontraron recogiendo leña lo
llevaron ante Moisés, Aarón y el resto de los israelitas.
\bibleverse{34} Lo pusieron bajo vigilancia porque no estaba claro qué
le iba a pasar.

\bibleverse{35} El Señor le dijo a Moisés: ``Este hombre tiene que ser
ejecutado. Todos los israelitas deben apedrearlo fuera del campamento''.
\bibleverse{36} Así que todos tomaron al hombre fuera del campamento y
lo apedrearon hasta la muerte como el Señor había ordenado a Moisés.

\hypertarget{ordenanza-sobre-las-borlas-para-adherir-a-las-puntas-de-la-ropa}{%
\subsection{Ordenanza sobre las borlas para adherir a las puntas de la
ropa}\label{ordenanza-sobre-las-borlas-para-adherir-a-las-puntas-de-la-ropa}}

\bibleverse{37} Poco después el Señor le dijo a Moisés: \bibleverse{38}
``Diles a los israelitas que para todas las generaciones futuras harán
borlas para los dobladillos de tu ropa y deberán atarlas con un cordón
azul. \footnote{\textbf{15:38} Deut 22,12; Mat 23,5}

\bibleverse{39} Cuando miren estas borlas recordarán que deben guardar
todos los mandamientos del Señor y que no sean infieles, siguiendo sus
propios pensamientos y deseos. \bibleverse{40} De esta manera
serecordarán que deben guardar todos mis mandamientos y serán santos
para Dios. Yo soy el Señor su Dios que los sacó de Egipto para ser su
Dios. \bibleverse{41} ¡Yo soy el Señor su Dios!''

\hypertarget{el-ultraje-de-coruxe9-y-los-rubenitas}{%
\subsection{El ultraje de Coré y los
rubenitas}\label{el-ultraje-de-coruxe9-y-los-rubenitas}}

\hypertarget{section-15}{%
\section{16}\label{section-15}}

\bibleverse{1} Coré,\footnote{\textbf{16:1} Coré era primo de Moisés y
  Aarón, y el celo por su posición pudo haber sido la causa de su
  rebelión.} hijo de Izhar, hijo de Coat, hijo de Levi, trató de asumir
el liderazgo, junto con Datán y Abiram, hijos de Eliab, y On, hijo de
Pelet, que eran de la tribu de Rubén. \bibleverse{2} Istos se rebelaron
contra Moisés, y se les unieron 250 respetados líderes israelitas y
miembros de la asamblea. \footnote{\textbf{16:2} Núm 12,1-2}
\bibleverse{3} Se unieron en oposición a Moisés y Aarón, diciéndoles:
``¡Ustedes se han adueñado del poder! Cada uno de los israelitas es
santo, y el Señor está entre ellos. Entonces, ¿por qué se ponen ustedes
por encima de la asamblea del Señor?''

\hypertarget{moisuxe9s-confronta-al-grupo-de-coruxe9-y-anuncia-un-juicio-divino-en-el-santuario}{%
\subsection{Moisés confronta al grupo de Coré y anuncia un juicio divino
en el
santuario}\label{moisuxe9s-confronta-al-grupo-de-coruxe9-y-anuncia-un-juicio-divino-en-el-santuario}}

\bibleverse{4} Cuando Moisés oyó lo que decían, cayó al suelo boca
abajo. \bibleverse{5} Entonces le dijo a Coré y a todos los que estaban
con él: ``Por la mañana el Señor va a demostrar quién es suyo y quién es
santo, y permitirá que esa persona se acerque a él. Sólo permitirá que
se acerque a él quien él elija. \footnote{\textbf{16:5} 2Tim 2,19}
\bibleverse{6} Esto es lo que tú, Coré, y todos los que están contigo
van a hacer. Toma unos quemadores de incienso, \bibleverse{7} y mañana
pon incienso en ellos y enciéndelo en la presencia del Señor. Entonces
el hombre que el Señor elija es el que es santo. ¡Son ustedes, los
levitas, los que están tomando demasiado poder para ustedes mismos!''

\bibleverse{8} Moisés también le dijo a Coré: ``¡Escuchen, levitas!
\bibleverse{9} ¿Les parece poco que el Dios de Israel los haya elegido
entre todos los demás israelitas y les haya permitido acercarse a él y
realizar la obra en el Tabernáculo del Señor, estar ante los israelitas
y servirles? \bibleverse{10} estedado el privilegio de acercarte a él, a
ti, Coré, a y a todos los demás levitas, ¡pero ahora también quieren
tener el sacerdocio! \bibleverse{11} Así que en realidad tú y los que se
han unido a ti están luchando contra el Señor, porque ¿quién es Aarón
para que murmurencontra él?'' \footnote{\textbf{16:11} Éxod 16,7}

\hypertarget{datuxe1n-y-abiram-se-burlan-de-la-invitaciuxf3n-de-moisuxe9s-moisuxe9s-oraciuxf3n-a-dios}{%
\subsection{Datán y Abiram se burlan de la invitación de Moisés; Moisés
oración a
Dios}\label{datuxe1n-y-abiram-se-burlan-de-la-invitaciuxf3n-de-moisuxe9s-moisuxe9s-oraciuxf3n-a-dios}}

\bibleverse{12} Entonces Moisés convocó a Datán y a Abiram, los hijos de
Eliab, pero ellos respondieron: ``No vamos a comparecer ante
ustedes!\footnote{\textbf{16:12} En otras palabras, se negaron a
  reconocer la autoridad de Moisés para exigirles que comparecieran ante
  él para ser juzgados.} \bibleverse{13} ¿No has hecho suficiente
alejándonos de una tierra que fluye leche y miel para matarnos aquí en
el desierto? ¿También tienes que hacerte un dictador y gobernante?
\bibleverse{14} Además, no nos has llevado a una tierra que fluye leche
y miel ni nos has dado campos y viñedos para que los poseamos. ¿De
verdad crees que puedes engañar a todo el mundo?\footnote{\textbf{16:14}
  La expresión usada aquí ``¿Le sacarás los ojos a estos hombres?'' se
  entiende como algo así como ``¿Les vas a tirar de la lana en los
  ojos?''} ¡No, no asistiremos!''

\bibleverse{15} Moisés se enfadó mucho y le dijo al Señor: ``No aceptes
sus ofrendas. Nunca les he quitado ni un burro ni he tratado mal a
ninguno de ellos''. \footnote{\textbf{16:15} 1Sam 12,3; Hech 20,33}

\hypertarget{moisuxe9s-convoca-a-coruxe9-y-sus-compauxf1eros-para-realizar-el-sacrificio-la-apariciuxf3n-de-la-gloria-de-dios-intercesiuxf3n-de-moisuxe9s}{%
\subsection{Moisés convoca a Coré y sus compañeros para realizar el
sacrificio; La aparición de la gloria de Dios; Intercesión de
Moisés}\label{moisuxe9s-convoca-a-coruxe9-y-sus-compauxf1eros-para-realizar-el-sacrificio-la-apariciuxf3n-de-la-gloria-de-dios-intercesiuxf3n-de-moisuxe9s}}

\bibleverse{16} Moisés le dijo a Coré: ``Tú y todos los que se han unido
a ti deben presentarse ante el Señor mañana, todos ustedes y Aarón
también. \bibleverse{17} Cada uno tomará su quemador de incienso, lo
pondrá en él y lo ofrecerá ante el Señor. Los 250 usarán sus quemadores
de incienso y Aarón también''.

\bibleverse{18} Entonces cada uno tomó su incensario, puso incienso en
él, lo encendió, y se paró junto con Moisés y Aarón a la entrada del
Tabernáculo de Reunión. \bibleverse{19} Cuando Coré reunió a todo su
grupo rebelde a la entrada del Tabernáculo de Reunión, la gloria del
Señor apareció ante toda la congregación.

\bibleverse{20} El Señor dijo a Moisés y Aarón, \bibleverse{21}
``Apártense de estos israelitas y los destruiré enseguida''.

\bibleverse{22} Pero Moisés y Aarón cayeron al suelo boca abajo y
dijeron: ``Dios -- Diosde todo lo que vive -- si es un solo hombre el
que peca, ¿tienes que enfadarte con todos?'' \footnote{\textbf{16:22}
  Job 12,10; 2Sam 24,17}

\bibleverse{23} Entonces el Señor le dijo a Moisés: \bibleverse{24}
``Dile al pueblo que se aleje de las casas de Coré, Datán y Abiram''.

\hypertarget{moisuxe9s-convoca-a-coruxe9-y-sus-compauxf1eros-para-realizar-el-sacrificio-la-apariciuxf3n-de-la-gloria-de-dios-intercesiuxf3n-de-moisuxe9s-1}{%
\subsection{Moisés convoca a Coré y sus compañeros para realizar el
sacrificio; La aparición de la gloria de Dios; Intercesión de
Moisés}\label{moisuxe9s-convoca-a-coruxe9-y-sus-compauxf1eros-para-realizar-el-sacrificio-la-apariciuxf3n-de-la-gloria-de-dios-intercesiuxf3n-de-moisuxe9s-1}}

\bibleverse{25} Entonces Moisés se acercó a Dathan y Abiram, y los
ancianos israelitas de Israel le siguieron. \bibleverse{26} Ordenó al
pueblo: ``Apártense de las tiendas de estos malvados y no toquen nada
que les pertenezca, de lo contrario serán destruidos junto con ellos en
todos sus pecados''.

\bibleverse{27} El pueblo se alejó de las casas de Coré, Datán y Abiram.
Dathan y Abiram salieron y se pararon en las entradas de sus tiendas
junto con sus esposas, hijos y pequeños.

\bibleverse{28} Moisés dijo: ``Así es como sabrán que el Señor me envió
para llevar a cabo todo lo que he hecho, porque no fue nada que surgiera
de mi pernsamiento.\footnote{\textbf{16:28} ``Nada que surgiera de mi
  pensamiento'': literalmente, ``no salió de mi mente'', porque se creía
  que el corazón era el lugar donde se generaban los pensamientos.}
\bibleverse{29} Si estos hombres mueren de muerte natural,
experimentando el destino de cada ser humano, entonces el Señor no me
envió. \bibleverse{30} Pero si el Señor hace algo totalmente diferente,
y la tierra se abre y se los traga junto con todo lo que les pertenece
para que bajen vivos al Seol, entonces sabrán que estos hombres han
actuado con desprecio ante Señor''.

\bibleverse{31} Tan pronto como Moisés terminó de decir todo esto, la
tierra debajo de los rebeldes se abrió, \bibleverse{32} y la tierra se
los tragó así como a sus hogares, y a todos los que estaban allí con
Coré y todo lo que les pertenecía. \bibleverse{33} Bajaron vivos al Seol
con todo lo que tenían. La tierra se cerró sobre ellos, y ya no estaban.
\bibleverse{34} Cuando oyeron sus gritos, todos los israelitas cercanos
salieron corriendo, gritando: ``¡Cuidado! La tierra podría tragarnos a
nosotros también''. \bibleverse{35} Entonces fuego salió del Señor y
quemó a los 250 hombres que ofrecían incienso. \footnote{\textbf{16:35}
  Lev 10,1-2; Sal 106,18}

\hypertarget{el-uso-de-las-250-ollas-humeantes-por-parte-de-coruxe9-y-sus-compauxf1eros-como-cubierta-para-el-altar-de-sacrificios}{%
\subsection{El uso de las 250 ollas humeantes por parte de Coré y sus
compañeros como cubierta para el altar de
sacrificios}\label{el-uso-de-las-250-ollas-humeantes-por-parte-de-coruxe9-y-sus-compauxf1eros-como-cubierta-para-el-altar-de-sacrificios}}

\bibleverse{36} Entonces el Señor dijo a Moisés, \bibleverse{37} ``Dile
a Eleazar, hijo del sacerdote Aarón, que recoja los incensarios sagrados
de entre los que se han quemado, y que esparza las brasas usadas para el
incienso bien lejos del campamento. \bibleverse{38} Haz que los
incensarios de los que pecaron a costa de su propia vida sean
martillados en láminas de metal como cobertura para el altar, porque
fueron ofrecidos ante el Señor, y así se han hecho santos. Serán un
recordatorio para los israelitas de lo que pasó''.

\bibleverse{39} Así que el sacerdote Eleazar recogió los incensarios de
bronce que usaban los quemados y los hizo martillar como cubierta para
el altar, \bibleverse{40} siguiendo las instrucciones que le dio el
Señor a través de Moisés. Esto era para recordar a los israelitas que
nadie que no sea descendiente de Aarón debe venir a ofrecer incienso
ante el Señor, de lo contrario podrían terminar como Coré y los que
están con él.

\hypertarget{castigo-a-la-comunidad-quejuxe1ndose-por-la-desapariciuxf3n-de-los-alborotadores-la-expiaciuxf3n-hecha-por-moisuxe9s-y-aaruxf3n}{%
\subsection{Castigo a la comunidad quejándose por la desaparición de los
alborotadores; la expiación hecha por Moisés y
Aarón}\label{castigo-a-la-comunidad-quejuxe1ndose-por-la-desapariciuxf3n-de-los-alborotadores-la-expiaciuxf3n-hecha-por-moisuxe9s-y-aaruxf3n}}

\bibleverse{41} Al día siguiente todos los israelitas se quejaron a
Moisés y Aarón, diciendo: ``¡Han matado al pueblo del Señor!''

\bibleverse{42} Pero cuando el pueblo se reunió para enfrentarse a
ellos, Moisés y Aarón se acercaron al Tabernáculo de Reunión, y de
repente la nube lo cubrió y apareció la gloria del Señor.
\bibleverse{43} Moisés y Aarón fueron y se pararon al frente del
Tabernáculo de Reunión, \bibleverse{44} y el Señor le dijo a Moisés:
\bibleverse{45} ``Aléjate de este pueblo, pues acabaré con ellos
inmediatamente''. Moisés y Aarón cayeron al suelo boca abajo.

\bibleverse{46} Moisés le dijo a Aarón: ``Pon algunas brasas del altar y
algo de incienso en tu incensario. Luego corre donde está el pueblo y
ponlos delante del Señor, porque el Señor está enojado con ellos y una
plaga ha comenzado''.

\bibleverse{47} Aarón tomó el incensario tal como le había dicho Moisés
y corrió al centro de la asamblea. Vio que la peste había empezado a
afectar al pueblo, así que ofreció el incienso e hizo que el pueblo se
pusiera en pie ante el Señor. \bibleverse{48} Se interpuso entre los
muertos y los vivos, y la peste se detuvo. \bibleverse{49} Sin embargo,
14. 700 personas murieron por la plaga además de los que murieron por
culpa de Coré. \bibleverse{50} Entonces Aarón regresó a Moisés a la
entrada del Tabernáculo de Reunión porque la plaga había sido detenida.

\hypertarget{prueba-del-derecho-sacerdotal-de-aaruxf3n-por-los-maravillosos-peldauxf1os-de-su-cayado}{%
\subsection{Prueba del derecho sacerdotal de Aarón por los maravillosos
peldaños de su
cayado}\label{prueba-del-derecho-sacerdotal-de-aaruxf3n-por-los-maravillosos-peldauxf1os-de-su-cayado}}

\hypertarget{section-16}{%
\section{17}\label{section-16}}

\bibleverse{1} El Señor le dijo a Moisés: \bibleverse{2} ``Dile a los
israelitas que traigan doce bastones, uno del líder de cada tribu.
Escriban el nombre de cada hombre en el bastón, \bibleverse{3} y
escriban el nombre de Aarón en el bastón de la tribu de Leví, porque
tiene que haber un bastón para el jefe de cada tribu. \bibleverse{4}
Coloca los bastones en el Tabernáculo de Reunión frente al
Testimonio\footnote{\textbf{17:4} El testimonio se refería a las dos
  tablas de piedra de los Diez Mandamientos que se guardaban dentro del
  Arca.} donde me encuentro contigo. \bibleverse{5} El bastón que
pertenece al hombre que yo elija brotará ramas, y pondré fin a las
constantes quejas de los israelitas contra ti''.

\bibleverse{6} Moisés explicó esto a los israelitas, y cada uno de sus
líderes le dio un bastón, uno para cada uno de los líderes de sus
tribus. Así que había doce bastones incluyendo el de Aarón.
\bibleverse{7} Moisés colocó los bastones ante el Señor en la Tienda del
Testimonio. \footnote{\textbf{17:7} Núm 14,10}

\bibleverse{8} Al día siguiente Moisés entró en la Tienda del Testimonio
y vio que el bastón de Aarón que representaba a la tribu de Leví, había
brotado y salieron ramas de él, y estaba florecido y había producido
almendras. \bibleverse{9} Moisés tomó todos los bastones de la presencia
del Señor y los mostró a todos los israelitas. Ellos los vieron, y cada
hombre recogió su propio bastón.

\bibleverse{10} El Señor le dijo a Moisés: ``Pon el bastón de Aarón de
nuevo delante del Testimonio, para que lo guardes allí como un
recordatorio para advertir a cualquiera que quiera rebelarse, para que
dejen de quejarse contra mí. De lo contrario, morirán''. \bibleverse{11}
Moisés hizo lo que el Señor le ordenó. \footnote{\textbf{17:11} Éxod
  28,38; Lev 16,13}

\bibleverse{12} Entonces los israelitas vinieron y le dijeron a Moisés:
``¿No ves que todos vamos a morir? ¡Nos van a destruir! ¡Nos van a matar
a todos! \bibleverse{13} El que se atreva a acercarse al Tabernáculo del
Señor morirá. ¿Nos van a aniquilar a todos?''

\hypertarget{ordenanzas-generales-sobre-los-deberes-de-los-sacerdotes-y-sus-ayudantes-los-levitas}{%
\subsection{Ordenanzas generales sobre los deberes de los sacerdotes y
sus ayudantes, los
levitas}\label{ordenanzas-generales-sobre-los-deberes-de-los-sacerdotes-y-sus-ayudantes-los-levitas}}

\hypertarget{section-17}{%
\section{18}\label{section-17}}

\bibleverse{1} El Señor le dijo a Aarón: ``Tú y tus hijos y los otros
levitas son responsables de los pecados relacionados con el santuario.
Sólo tú y tus hijos son responsables de los pecados relacionados con su
sacerdocio. \footnote{\textbf{18:1} Éxod 28,38; Lev 16,32-33}
\bibleverse{2} Haz que tus hermanos de la tribu de Leví, la tribu de tu
padre, se unan a ti para ayudarte a ti y a tus hijos con tu servicio en
la Tienda del Testimonio. \footnote{\textbf{18:2} Núm 3,6-10}
\bibleverse{3} Ellos se encargarán de tus responsabilidades y de las
relacionadas con la Tienda, pero no deben acercarse demasiado a los
objetos sagrados del santuario o del altar, de lo contrario morirán, y
tú también. \bibleverse{4} Deben ayudarte y cuidar de las
responsabilidades del Tabernáculo de Reunión, haciendo todo el trabajo
en la Tienda, pero no se les permite estar contigo durante tu ministerio
sacerdotal.\footnote{\textbf{18:4} ``Durante tu ministerio sacerdotal'':
  añadido para mayor claridad.}

\bibleverse{5} ``Debes llevar a cabo las responsabilidades relacionadas
con el santuario y el altar, para que mi ira no vuelva a caer sobre los
israelitas. \bibleverse{6} Mira, yo mismo he elegido a tus hermanos los
levitas de los israelitas como mi regalo para ti, dedicado al Señor para
hacer el trabajo que relaciona el Tabernáculo de Reunión. \footnote{\textbf{18:6}
  Núm 3,12; Núm 3,45} \bibleverse{7} Pero sólo tú y tus hijos son
responsables de tu sacerdocio, haciendo todo lo que concierne al altar y
está detrás del velo. Sólo tú debes realizar ese servicio. Te estoy
dando el don de tu sacerdocio, pero cualquier otro que se acerque al
santuario debe ser ejecutado''. \footnote{\textbf{18:7} Núm 1,51}

\hypertarget{los-ingresos-de-los-sacerdotes}{%
\subsection{Los ingresos de los
sacerdotes}\label{los-ingresos-de-los-sacerdotes}}

\bibleverse{8} El Señor le dijo a Aarón, ``Escucha, te he puesto a cargo
de oficiar mis ofrendas. Todas las santas contribuciones de los
israelitas que traen están reservadas para ti, y esta es una regla
permanente. \footnote{\textbf{18:8} Lev 2,3; Lev 2,10; Lev 6,9-11; Lev
  6,19-22; Lev 7,6-10} \bibleverse{9} Parte de las ofrendas más sagradas
tomadas de los holocaustos son tuyas. Parte de todas las ofrendas que me
dan como ofrendas sagradas, ya sean ofrendas de grano o de pecado o de
culpa, esa parte pertenece a ti y a tus hijos. \bibleverse{10} Lo
comerás en un lugar santísimo.\footnote{\textbf{18:10} Según lo requería
  la ley levítica: Ver por ejemplo Levítico 6:16; Levítico 16:26;
  Levítico 7:6.} A todo macho se le permite comerlo. Deben considerarlo
como algo sagrado.

\bibleverse{11} ``También te pertenecen los regalos voluntarios y las
ofrendas de los israelitas. Te he dado esto a ti y a tus hijos e hijas
como una regla permanente. Todos los de tu casa que estén
ceremonialmente limpios pueden comerlo.

\bibleverse{12} Les doy el mejor aceite de oliva y el mejor vino y grano
que los israelitas dan como primicias al Señor. \bibleverse{13} Las
primicias de todas las cosechas que produzcan en su tierra y que traigan
al Señor son tuyas. Todos los miembros de tu familia que estén
ceremonialmente limpios pueden comerlas. \footnote{\textbf{18:13} Éxod
  23,19; Deut 18,4}

\bibleverse{14} ``Todo lo que en Israel se dedica al Señor es tuyo.
\footnote{\textbf{18:14} Lev 27,28} \bibleverse{15} Todo primogénito, ya
sea humano o animal, que se ofrezca al Señor es tuyo. Pero debes redimir
todo primogénito y todo primogénito de los animales inmundos.
\footnote{\textbf{18:15} Éxod 13,12-13; Éxod 34,19-20} \bibleverse{16}
Cuando tengan un mes de edad, pagarás el precio de redención de cinco
siclos de plata (usando el estándar de siclos del santuario),
equivalente a veinte gueras.

\bibleverse{17} ``Pero no se te permitirá redimir al primogénito de un
buey, una oveja o una cabra porque son sagrados. Esparcirás su sangre
sobre el altar y quemarás su grasa como holocausto aceptado por el
Señor. \bibleverse{18} Su carne es tuya, de la misma manera que el pecho
y el muslo derecho de la ofrenda ondulada son tuyos. \bibleverse{19}
``Te doy todas las ofrendas voluntarias que los israelitas presentan al
Señor así como a tus hijos e hijas como una regla permanente. Es un
acuerdo permanente de sal\footnote{\textbf{18:19} ``Acuerdo permanente
  de sal'': se refiere a un acuerdo que no puede romperse. La sal se
  usaba como conservante, y las ofrendas al Señor incluían sal (ver
  Levítico 2:13).} ante el Señor para ti y tus descendientes''.

\hypertarget{asignaciuxf3n-del-diezmo-a-los-levitas-por-la-negaciuxf3n-de-la-tierra}{%
\subsection{Asignación del diezmo a los levitas por la negación de la
tierra}\label{asignaciuxf3n-del-diezmo-a-los-levitas-por-la-negaciuxf3n-de-la-tierra}}

\bibleverse{20} ``No tendrás propiedades en su país, y no tendrás una
parte de sus tierras. Yo soy tu parte y tu posesión entre los
israelitas.

\bibleverse{21} En cambio, he dado a los levitas todos los diezmos de
Israel como compensación por el servicio que prestan al hacer el trabajo
en el Tabernáculo de Reunión. \footnote{\textbf{18:21} Lev 27,30}
\bibleverse{22} ``A los israelitas ya no se les permite acercarse al
Tabernáculo de Reunión, de lo contrario cometerán una ofensa y morirán.
\bibleverse{23} Los levitas deben realizar el trabajo en el Tabernáculo
de Reunión, y deben asumir la responsabilidad de los pecados que se
cometan. Esta es una regla permanente para todas las generaciones
futuras. Los levitas no recibirán una parte de la tierra entre los
israelitas. \bibleverse{24} En su lugar, he dado a los levitas como
compensación el diezmo que los israelitas dan al Señor como
contribución. Por eso les dije que no recibirían una parte de la tierra
entre los israelitas''.

\hypertarget{el-diezmo-de-los-ingresos-de-los-levitas-a-los-sacerdotes}{%
\subsection{El diezmo de los ingresos de los levitas a los
sacerdotes}\label{el-diezmo-de-los-ingresos-de-los-levitas-a-los-sacerdotes}}

\bibleverse{25} El Señor le dijo a Moisés: \bibleverse{26} ``Habla con
los levitas y explícales: `Cuando recibas de los israelitas el diezmo
que te he dado como compensación, debes devolver parte de él como
ofrenda al Señor: un diezmo del diezmo. \bibleverse{27} Tu ofrenda será
considerada como si fueran las primicias del grano de tu era o del jugo
de uva del lagar. \bibleverse{28} De este modo, deberás contribuir con
una ofrenda al Señor de cada diezmo que recibas de los israelitas,
entregando la ofrenda del Señor al sacerdote Aarón. \bibleverse{29} De
todas las ofrendas que recibas debes contribuir como ofrenda del Señor
con lo mejor, la parte más sagrada de cada ofrenda'.

\bibleverse{30} ``Así que di a los levitas, `Cuando hayas presentado la
mejor parte, será considerada como tu contribución producida por tu
trilladora o lagar. \bibleverse{31} Ustedes y sus familias pueden
comerla en cualquier sitio porque es la compensación por su servicio en
el Tabernáculo de Reunión. \bibleverse{32} No se considerará que han
pecado si han presentado la mejor parte. Pero si tratan las sagradas
ofrendas de los israelitas con falta de respeto morirán'\,''.

\hypertarget{preparaciuxf3n-y-uso-del-agua-de-limpieza}{%
\subsection{Preparación y uso del agua de
limpieza}\label{preparaciuxf3n-y-uso-del-agua-de-limpieza}}

\hypertarget{section-18}{%
\section{19}\label{section-18}}

\bibleverse{1} El Señor le dijo a Moisés y Aarón: \bibleverse{2} ``Esta
es una norma que el Señor ha ordenado, diciendo, Dile a los israelitas
que te traigan una vaca roja\footnote{\textbf{19:2} ``Vaca'': la palabra
  utilizada aquí se traduce generalmente como ``novilla'' que en se
  refiere a una joven vaca hembra que no ha tenido un ternero. Sin
  embargo, como se desprende de 1 Samuel 6:7, la palabra también se
  utiliza para describir a una vaca que ha tenido un ternero y está
  produciendo leche.} sin defectos, que nunca haya sido uncida.
\footnote{\textbf{19:2} Heb 9,13; Lev 22,20} \bibleverse{3} Entrégasela
al sacerdote Eleazar, y él la llevará fuera del campamento ylamandará a
masacrar. \bibleverse{4} El sacerdote Eleazar pondrá un poco de su
sangre en su dedo y la rociará siete veces hacia la entrada del
Tabernáculo de Reunión. \bibleverse{5} Luego la vaca debe ser quemada
mientras él observa. Todo debe ser quemado: su piel, carne y sangre, así
como sus excrementos. \bibleverse{6} El sacerdote arrojará madera de
cedro, hisopo e hilo carmesí sobre la vaca en llamas. \footnote{\textbf{19:6}
  Lev 14,6} \bibleverse{7} ``Entonces el sacerdote lavará sus ropas y su
cuerpo en agua, y después podrá entrar en el campamento, pero
permanecerá impuro hasta la noche. \footnote{\textbf{19:7} Lev 16,28}
\bibleverse{8} La persona que quemó la vaca también lavará sus ropas y
su cuerpo en agua, y él también permanecerá impuro hasta la noche.

\bibleverse{9} ``El que esté limpio recogerá las cenizas de la vaca y
las guardará en un lugar limpio fuera del campamento. Las guardarán los
israelitas para preparar el agua de purificación que sirve para
purificar del pecado. \bibleverse{10} El hombre que recogió las cenizas
de la vaca lavará también sus ropas, y permanecerá impuro hasta la
noche. Esta es una regla permanente para los israelitas y para el
extranjero que vive con ellos.

\bibleverse{11} ``Si tocas un cadáver serás impuro durante siete días.
\bibleverse{12} Debes purificarte con el agua de la purificación al
tercer día y al séptimo día, y entonces estarás limpio. Pero si no te
purificas en el tercer y séptimo día, no estarás limpio. \bibleverse{13}
Si tocas un cadáver y no te purificas, harás impuro el Tabernáculo del
Señor y deberás ser expulsado de Israel. Sigues siendo impuro porque no
se te ha rociado con el agua de la purificación y tu impureza permanece.
\footnote{\textbf{19:13} Lev 15,31}

\hypertarget{instrucciones-sobre-casos-especuxedficos-de-contaminaciuxf3n-y-su-tratamiento}{%
\subsection{Instrucciones sobre casos específicos de contaminación y su
tratamiento}\label{instrucciones-sobre-casos-especuxedficos-de-contaminaciuxf3n-y-su-tratamiento}}

\bibleverse{14} ``La siguiente norma se aplica cuando una persona muere
en una tienda. Todo el que entre en la tienda y todo el que ya esté en
ella será impuro durante siete días. \bibleverse{15} Cualquier
recipiente abierto que no tenga una tapa cerrada es impuro.

\bibleverse{16} Si estás al aire libre y tocas a alguien que ha muerto
por la espada o que ha muerto de forma natural, o si tocas un hueso
humano o una tumba, entonces serás impuro durante siete días.

\bibleverse{17} ``Este es el proceso para la purificación si eres
impuro. Toma algunas de las cenizas del holocausto para la purificación
y ponlas en un frasco con agua fresca. \bibleverse{18} El hombre que
esté limpio tomará un hisopo y lo mojará en el agua. Luego rociará la
tienda y todo lo que haya dentro de ella, y a todos los que estuvieran
allí. También deberá rociarlo a usted si ha tocado un hueso, o una
tumba, o alguien que ha muerto o ha sido asesinado. \bibleverse{19} ``El
hombre que está limpio debe rociarte tanto al tercer día como al séptimo
día. Después de que te purifiques al séptimo día, debes lavar tu ropa y
a ti mismo en agua, y esa noche estarás limpio. \bibleverse{20} Pero si
no te purificas, serás expulsado de los israelitas, porque has hecho
impuro el Tabernáculo del Señor. El agua de la purificación no ha sido
rociada sobre ti, y sigues siendo impuro. \bibleverse{21} Esta es una
regla permanente para todos. El hombre que rocía el agua de purificación
debe lavar su ropa, y cualquiera que toque el agua de purificación será
impuro hasta la noche.

\bibleverse{22} Todo lo que toque la persona impura será impuro, y
cualquiera que lo toque será impuro hasta la noche''.

\hypertarget{llegada-a-kade-y-muerte-de-miriam-reonovada-queja-del-pueblo-la-fatuxeddicia-doncaciuxf3n-de-agua-de-la-roca-para-moisuxe9s-y-aaruxf3n}{%
\subsection{Llegada a Kade y muerte de Miriam; reonovada queja del
pueblo; la fatídicia doncación de agua de la roca para Moisés y
Aarón}\label{llegada-a-kade-y-muerte-de-miriam-reonovada-queja-del-pueblo-la-fatuxeddicia-doncaciuxf3n-de-agua-de-la-roca-para-moisuxe9s-y-aaruxf3n}}

\hypertarget{section-19}{%
\section{20}\label{section-19}}

\bibleverse{1} Fue durante el primer mes del año que todos los
israelitas llegaron al desierto de Zin y establecieron un campamento en
Cades. (Aquí fue donde Miriam murió y fue enterrada). \bibleverse{2} Sin
embargo, allí no había agua para que nadie bebiera, así que la gente se
reunió para enfrentarse a Moisés y Aarón. \footnote{\textbf{20:2} Éxod
  17,1-7} \bibleverse{3} Discutieron con Moisés, diciendo: ``¡Si
hubiéramos muerto con nuestros parientes en la presencia del Señor!
\bibleverse{4} ¿Por qué has traído al pueblo del Señor a este desierto
para que nosotros y nuestro ganado muramos aquí? \bibleverse{5} ¿Por qué
nos has sacado de Egipto para venir a este horrible lugar? Aquí no crece
nada, ni grano, ni higos, ni viñas, ni granadas. Y no hay agua para
beber''.

\bibleverse{6} Moisés y Aarón dejaron el pueblo y se fueron a la entrada
del Tabernáculo de Reunión. Allí cayeron boca abajo en el suelo, y la
gloria del Señor se les apareció. \bibleverse{7} El Señor le dijo a
Moisés, \bibleverse{8} ``Toma el bastón y haz que la gente se reúna a tu
alrededor. Mientras miran, tú y tu hermano Aarón ordenarán a la roca y
derramará agua. Traerán agua de la roca para que el pueblo y su ganado
puedan beber''.

\bibleverse{9} Moisés recogió el bastón que estaba guardado en la
presencia del Señor, como se le había ordenado. \bibleverse{10} Moisés y
Aarón hicieron que todos se reunieran frente a la roca. Moisés les dijo:
``¡Escuchen, pandilla de rebeldes! ¿Tenemos que sacar agua de esta roca
para ustedes?'' \footnote{\textbf{20:10} Sal 106,33} \bibleverse{11}
Entonces Moisés tomó el bastón y golpeó la roca dos veces. Salieron
chorros de agua para que la gente y su ganado pudieran beber.

\bibleverse{12} Pero el Señor les dijo a Moisés y a Aarón: ``Como no
confiaron en mí lo suficiente para demostrar lo santo que soy a los
israelitas, no serán ustedes los que los lleven al país que les he
dado''.

\bibleverse{13} El lugar donde los israelitas discutían con el Señor se
llamaba las aguas de Meribá, y era donde les revelaba su santidad.
\footnote{\textbf{20:13} Sal 81,8}

\hypertarget{los-edomitas-se-niegan-a-permitir-el-paso-la-muerte-de-aaron}{%
\subsection{Los edomitas se niegan a permitir el paso; La muerte de
Aaron}\label{los-edomitas-se-niegan-a-permitir-el-paso-la-muerte-de-aaron}}

\bibleverse{14} Moisés envió mensajeros desde Cades al rey de Edom,
diciéndole: ``Esto es lo que dice tu hermano Israel. Tú sabes todo sobre
las dificultades que hemos enfrentado. \footnote{\textbf{20:14} Gén
  32,4; Jue 11,17; Deut 23,8}

\bibleverse{15} Nuestros antepasados fueron a Egipto y nosotros vivimos
allí mucho tiempo. Los egipcios nos trataron mal a nosotros y a nuestros
antepasados, \bibleverse{16} así que pedimos ayuda al Señor, y él
escuchó nuestros gritos. Envió un ángel y nos sacó de Egipto.
``Escuchen, ahora estamos en Cades, un pueblo en la frontera de su
territorio. \footnote{\textbf{20:16} Éxod 23,20}

\bibleverse{17} Por favor, permítanos viajar a través de su país. No
cruzaremos ninguno de sus campos o viñedos, ni beberemos agua de ninguno
de sus pozos. Nos quedaremos en la Carretera del Rey; no nos desviaremos
ni a la derecha ni a la izquierda hasta que hayamos pasado por su
país''. \footnote{\textbf{20:17} Núm 21,22}

\bibleverse{18} Pero el rey de Edom respondió: ``Se les prohíbe viajar
por nuestro país, de lo contrario saldremos y los detendremos por la
fuerza''.

\bibleverse{19} ``Nos mantendremos en el camino principal'', insistieron
los israelitas. ``Si nosotros o nuestro ganado bebemos tu agua, te
pagaremos por ella. Eso es todo lo que queremos, sólo pasar a pie''.

\bibleverse{20} Pero el rey de Edom insistió: ``¡Tienen prohibido viajar
por nuestro país!'' Salió con su gran y poderoso ejército para
enfrentarse a los israelitas de frente. \bibleverse{21} Como el rey de
Edom se negó a permitir que Israel viajara por su territorio, los
israelitas tuvieron que volver.

\hypertarget{el-tren-de-kades-al-monte-hor-la-muerte-de-aaron}{%
\subsection{El tren de Kades al monte Hor; La muerte de
Aaron}\label{el-tren-de-kades-al-monte-hor-la-muerte-de-aaron}}

\bibleverse{22} Todos los israelitas dejaron Cades y viajaron al Monte
Hor. \bibleverse{23} En el monte Hor, cerca de la frontera con el país
de Edom, el Señor dijo a Moisés y Aarón, \bibleverse{24} ``Aarón pronto
se unirá a sus antepasados en la muerte. No entrará en el país que he
dado a los israelitas, porque ambos desobedecieron mi orden en las aguas
de Meribá. \bibleverse{25} Que Aarón y su hijo Eleazar se unan a ustedes
y suban juntos al monte Hor. \bibleverse{26} Quítale a Aarón la ropa de
sacerdote y pónsela a su hijo Eleazar. Aarón va a morir allí y se unirá
a sus antepasados en la muerte''. \footnote{\textbf{20:26} Lev 21,10}

\bibleverse{27} Moisés hizo lo que el Señor le ordenó: Subieron al monte
Hor a la vista de todos los israelitas. \bibleverse{28} Moisés se quitó
las ropas sacerdotales que llevaba Aarón y se las puso a su hijo
Eleazar. Aarón murió allí, en la cima del monte. Entonces Moisés y
Eleazar volvieron a bajar. \bibleverse{29} Cuando la gente se dio cuenta
de que Aarón había muerto, todos lloraron por él durante treinta días.

\hypertarget{batalla-victoriosa-con-el-rey-de-arad}{%
\subsection{Batalla victoriosa con el Rey de
Arad}\label{batalla-victoriosa-con-el-rey-de-arad}}

\hypertarget{section-20}{%
\section{21}\label{section-20}}

\bibleverse{1} El rey cananeo de Arad, que vivía en el Néguev, se enteró
de que los israelitas se acercaban por el camino de Atharim. Fue y atacó
a Israel y tomó a algunos de ellos prisioneros. \bibleverse{2} Así que
Israel hizo una promesa solemne al Señor: ``Si nos entregas a esta
gente, nos comprometemos a destruir completamente sus pueblos''.
\footnote{\textbf{21:2} Deut 13,16; Jos 6,17; Jue 1,17; 1Sam 15,3}
\bibleverse{3} El Señor respondió a su invitación y les entregó a los
cananeos. Los israelitas los destruyeron completamente a ellos y a sus
pueblos, y llamaron al lugar Horma.\footnote{\textbf{21:3} ``Horma''
  significa ``destrucción''.}

\hypertarget{murmullos-de-la-gente-las-serpientes-venenosas-y-la-serpiente-de-bronce}{%
\subsection{Murmullos de la gente; las serpientes venenosas y la
serpiente de
bronce}\label{murmullos-de-la-gente-las-serpientes-venenosas-y-la-serpiente-de-bronce}}

\bibleverse{4} Los israelitas dejaron el Monte Hor por el camino que
lleva al Mar Rojo para evitar viajar por el país de Edom. Pero el pueblo
se puso de mal humor en el camino \bibleverse{5} y se quejó contra Dios
y contra Moisés, diciendo: ``¿Por qué nos sacaste de Egipto para morir
en el desierto? No tenemos ni pan ni agua, y odiamos esta horrible
comida!''\footnote{\textbf{21:5} ``Horrible comida'': refiriéndose al
  maná.}

\bibleverse{6} Así que el Señor envió serpientes venenosas para
atacarlos, y muchos israelitas fueron mordidos y murieron. \footnote{\textbf{21:6}
  1Cor 10,9} \bibleverse{7} El pueblo fue a ver a Moisés y le dijo:
``Nos equivocamos al presentar quejas contra el Señor y contra ti. Por
favor, ruega al Señor que nos quite las serpientes de encima''. Moisés
rezó al Señor en su nombre.

\bibleverse{8} El Señor le dijo a Moisés: ``Haz una maqueta de una
serpiente y ponla en un palo. Cuando alguien que haya sido mordido la
mire, vivirá''. \bibleverse{9} Moisés hizo una serpiente de bronce y la
puso en un poste. Aquellos que la miraron vivieron.

\hypertarget{el-tren-al-arnuxf3n-ya-las-estepas-de-los-moabitas-la-canciuxf3n-de-la-fuente}{%
\subsection{El tren al Arnón ya las estepas de los moabitas; la canción
de la
fuente}\label{el-tren-al-arnuxf3n-ya-las-estepas-de-los-moabitas-la-canciuxf3n-de-la-fuente}}

\bibleverse{10} Los israelitas salieron y acamparon en Obot.
\bibleverse{11} Luego se fueron de Obot y acamparon en Iye-abarim en el
desierto en el lado este de Moab. \bibleverse{12} Se fueron de allí y
acamparon en el Valle de Zered. \bibleverse{13} Luego se trasladaron de
allí y acamparon en el lado más alejado del río Arnón, en el desierto
cerca del territorio de Amorite. El río Arnón es la frontera entre Moab
y los amorreos. \bibleverse{14} Por eso el Libro de las Guerras del
Señor se refiere ``al pueblo de Vaheb en Sufa y al barranco de Arnón,
\bibleverse{15} a las laderas del barranco que llegan al pueblo de Ar
que está en la frontera con Moab''.

\bibleverse{16} Desde allí se trasladaron a Beer, el pozo donde el Señor
le dijo a Moisés, ``Haz que el pueblo se reúna para que pueda darles
agua''.

\bibleverse{17} Entonces los israelitas cantaron esta canción: ``¡Echen
agua en el pozo! ¡Cada uno de ustedes, cante! \bibleverse{18} Los jefes
de las tribus cavaron el pozo; sí, los jefes del pueblo cavaron el pozo
con sus varas de autoridad y sus bastones''. Los israelitas dejaron el
desierto y siguieron hasta Matanaá

\bibleverse{19} Desde Mataná viajaron a Nahaliel, de Nahaliel a Bamot,
\bibleverse{20} y de Bamoth al valle en el territorio de Moab donde la
cima del Monte Pisga mira hacia los páramos.

\hypertarget{derrota-del-rey-amorreo-sehuxf3n-y-conquista-de-su-pauxeds-canciuxf3n-de-triunfo-de-los-israelitas}{%
\subsection{Derrota del rey amorreo Sehón y conquista de su país;
Canción de triunfo de los
israelitas}\label{derrota-del-rey-amorreo-sehuxf3n-y-conquista-de-su-pauxeds-canciuxf3n-de-triunfo-de-los-israelitas}}

\bibleverse{21} Entonces Israel envió mensajeros a Sehón, rey de los
amorreos, con la siguiente petición: \footnote{\textbf{21:21} Deut
  2,26-37} \bibleverse{22} ``Por favor, permítanos viajar a través de su
país. No cruzaremos ninguno de sus campos o viñedos, ni beberemos agua
de ninguno de sus pozos. Permaneceremos en la Carretera del Rey hasta
que hayamos pasado por su país''.

\bibleverse{23} Pero Sehón se negó a permitir que los israelitas
viajaran por su territorio. En su lugar, llamó a todo su ejército y
salió al encuentro de los israelitas de frente en el desierto. Cuando
llegó a Jahaz, atacó a los israelitas. \bibleverse{24} Los israelitas
los derrotaron, matándolos con sus espadas. Se apoderaron de su tierra
desde el río Arnón hasta el río Jaboc, pero sólo hasta la frontera de
los amonitas, porque estaba bien defendida. \bibleverse{25} Los
israelitas conquistaron todos los pueblos amorreos y se apoderaron de
ellos, incluyendo Hesbón y sus pueblos vecinos. \bibleverse{26} Hesbón
era la capital de Sehón, rey de los amorreos, que había luchado contra
el anterior rey de Moab y le había quitado todas sus tierras hasta el
río Arnón. \bibleverse{27} Por eso los antiguos compositores
escribieron: ``¡Vengan a Hesbón y hagan que la reconstruyan; restauren
la ciudad de Sehón! \bibleverse{28} Porque un fuego salió de Hesbón, una
llama de la ciudad de Sihón. Quemó a Ar en Moab, donde los gobernantes
viven en las alturas de Arnón. \bibleverse{29} ¡Qué desastre enfrentas,
Moab! ¡Vais a morir todos, pueblo de Quemos!\footnote{\textbf{21:29}
  Quemos era un dios al que se le presentaban sacrificios humanos.}
Entregaste a tus hijos como exiliados y a tus hijas como prisioneras a
Sehón, rey de los amorreos. \bibleverse{30} ¡Pero ahora hemos derrotado
a los amorreos! El gobierno de Heshbon ha sido destruido hasta Dibon.
Los aniquilamos hasta Nofa y hasta Medeba''.

\hypertarget{mayor-avance-de-los-israelitas-derrota-del-rey-og-de-basan}{%
\subsection{Mayor avance de los israelitas; Derrota del rey Og de
Basan}\label{mayor-avance-de-los-israelitas-derrota-del-rey-og-de-basan}}

\bibleverse{31} Los israelitas ocuparon el país de los amorreos.
\bibleverse{32} Moisés envió hombres a explorar Jazer. Los israelitas
conquistaron los pueblos de los alrededores y expulsaron a los amorreos
que vivían allí. \bibleverse{33} Luego continuaron en el camino hacia
Basán. Og, rey de Basán, dirigió a todo su ejército para enfrentarse a
ellos de frente, y luchó contra ellos en Edrei.

\bibleverse{34} El Señor le dijo a Moisés: ``No tienes que temerle,
porque yo te lo he entregado, junto con todo su pueblo y su tierra.
Hazle lo que hiciste con Sehón, rey de los amorreos, que gobernó desde
Hesbón''. \footnote{\textbf{21:34} Sal 136,17-22}

\bibleverse{35} Así que mataron a Og, a sus hijos y a todo su ejército.
Nadie sobrevivió, y los israelitas se apoderaron de su país.

\hypertarget{der-moabiterkuxf6nig-balak-beschlieuxdft-gesandte-an-bileam-zu-schicken}{%
\subsection{Der Moabiterkönig Balak beschließt, Gesandte an Bileam zu
schicken}\label{der-moabiterkuxf6nig-balak-beschlieuxdft-gesandte-an-bileam-zu-schicken}}

\hypertarget{section-21}{%
\section{22}\label{section-21}}

\bibleverse{1} Los israelitas avanzaron y acamparon en las llanuras de
Moab al este del Jordán, frente a Jericó. \bibleverse{2} Balac, hijo de
Zippor, había visto todo lo que los israelitas habían hecho a los
amorreos. \bibleverse{3} Los moabitas estaban aterrorizados de los
israelitas porque eran muchos. Los moabitas temían la llegada de los
israelitas \bibleverse{4} y dijeron a los líderes de Madián, ``Esta
horda se comerá todo lo que tenemos, como un buey se come la hierba del
campo''. (Balac hijo de Zippor, era rey de Moab en ese momento).

\bibleverse{5} Envió mensajeros para llamar a Balaam, hijo de Beor, que
vivía en Petor, cerca del río Éufrates en su propio país. ``Escuchen, ha
llegado aquí un grupo de personas que vinieron de Egipto'', dijo Balac
en su mensaje a Balaam. ``Hay hordas de ellos y representan una
verdadera amenaza para nosotros. \bibleverse{6} Por favor, ven
inmediatamente y maldice a estas personas por mí, porque son más fuertes
que yo. Tal vez entonces pueda atacarlos y expulsarlos de mi país porque
sé que quienquiera que bendiga es bendecido, y quienquiera que maldiga
es maldito''.

\hypertarget{la-primera-embajada-de-balac-a-balaam-sin-uxe9xito-su-mensaje-repetido}{%
\subsection{La primera embajada de Balac a Balaam sin éxito; su mensaje
repetido}\label{la-primera-embajada-de-balac-a-balaam-sin-uxe9xito-su-mensaje-repetido}}

\bibleverse{7} Entonces los líderes moabitas y madianitas partieron,
llevándose el pago de la adivinación con ellos. Cuando llegaron, le
dieron a Balaam el mensaje de Balac. \footnote{\textbf{22:7} 2Pe 2,15}

\bibleverse{8} ``Pasen la noche y les haré saber la respuesta que me da
el Señor'',\footnote{\textbf{22:8} Aunque Balaam no es un israelita, usa
  su nombre para Dios.} les dijo Balaam. Así que los líderes moabitas se
quedaron allí con Balaam.

\bibleverse{9} Dios vino a Balaam y le preguntó: ``¿Quiénes son estos
hombres que están contigo?''

\bibleverse{10} Balaam le dijo a Dios: ``Balac, hijo de Zipor, el rey de
Moab, me envió este mensaje: \bibleverse{11} `Escucha, ha llegado aquí
un grupo de gente que ha venido de Egipto. Hay hordas de ellos. Por
favor, ven inmediatamente y maldice a esta gente por mí. Tal vez así
pueda luchar contra ellos y expulsarlos de mi país'\,''.

\bibleverse{12} Pero Dios le dijo a Balaam, ``No debes volver con ellos.
No debes maldecir a este pueblo porque están bendecidos''.

\bibleverse{13} Por la mañana Balaam se levantó y dijo a los mensajeros
de Balac, ``Vuelve al lugar de donde viniste porque el Señor se ha
negado a permitirme ir contigo''.

\bibleverse{14} Los líderes moabitas se fueron. Volvieron donde Balac y
le dijeron: ``Balaam se negó a volver con nosotros''.

\bibleverse{15} Entonces Balac envió más líderes, unos que eran más
prestigiosos que los anteriores. \bibleverse{16} Cuando llegaron le
dijeron a Balaam: ``Esto es lo que dice Balac hijo de Zipor: `Por favor,
no dejes que nada te impida venir a verme, \bibleverse{17} porque te
pagaré mucho y seguiré todos los consejos que me des. Por favor, ven y
maldice a este pueblo por mí'\,''.

\bibleverse{18} Pero Balaam le dijo a los oficiales de Balac, ``Aunque
Balac me diera todo su palacio lleno de plata y oro, no podría
desobedecer el mandato del Señor mi Dios de ninguna manera.\footnote{\textbf{22:18}
  ``De ninguna manera'': literalmente, ``Ya sea por poco o mucho''.}
\bibleverse{19} Ahora también deberías pasar la noche para ver si el
Señor tiene algo más que decirme''.

\bibleverse{20} Dios vino a Balaam durante la noche y le dijo, ``Ya que
estos hombres han venido por ti, levántate y ve con ellos. Pero sólo haz
lo que yo te diga''.

\bibleverse{21} Por la mañana Balaam se levantó, puso una silla en su
burro y se fue con los líderes moabitas.

\hypertarget{el-viaje-de-balaam-a-moab-y-el-incidente-con-el-burro}{%
\subsection{El viaje de Balaam a Moab y el incidente con el
burro}\label{el-viaje-de-balaam-a-moab-y-el-incidente-con-el-burro}}

\bibleverse{22} Dios se enfadó porque Balaam había decidido irse. El
ángel del Señor vino y se paró en el camino para enfrentarlo. Balaam iba
montado en su burro, y estaba acompañado por sus dos sirvientes.
\bibleverse{23} El burro vio al ángel del Señor de pie en el camino con
una espada desenvainada, así que se apartó del camino y se fue a un
campo. Así que Balaam lo golpeó para que volviera al camino. \footnote{\textbf{22:23}
  Gén 3,24; Jos 5,13} \bibleverse{24} Entonces el ángel del Señor se
paró en una parte estrecha del camino que pasaba entre dos viñedos, con
muros a ambos lados. \bibleverse{25} El burro vio al ángel del Señor e
intentó pasar.\footnote{\textbf{22:25} ``E intentó pasar'': añadido para
  mayor claridad.} Empujó contra la pared y aplastó el pie de Balaam
contra ella. Así que lo golpeó de nuevo.

\bibleverse{26} Entonces el ángel del Señor se adelantó y se paró en un
lugar estrecho donde no había espacio para pasar, ni a la derecha ni a
la izquierda. \bibleverse{27} El burro vio al ángel del Señor y se
acostó bajo Balaam. Se enfadó y lo golpeó con su bastón.

\bibleverse{28} El Señor le dio al burro la habilidad de hablar y le
dijo a Balaam: ``¿Qué te he hecho para que me golpees tres veces?''

\bibleverse{29} ``¡Me has hecho quedar como un estúpido!'' Balaam le
dijo al burro. ``¡Si tuviera una espada, te mataría ahora!''

\bibleverse{30} Pero el burro le preguntó a Balaam, ``¿No soy yo el
burro que has montado toda tu vida hasta hoy? ¿Alguna vez te he tratado
así antes?'' ``No'', admitió.

\bibleverse{31} Entonces el Señor le dio a Balaam la habilidad de ver al
ángel del Señor de pie en el camino con una espada desenvainada. Balaam
se inclinó y cayó al suelo boca abajo. \bibleverse{32} El ángel del
Señor le preguntó: ``¿Por qué golpeaste a tu burro tres veces? Escucha,
he venido a enfrentarme a ti porque estás siendo obstinado.
\bibleverse{33} El burro me vio y me evitó tres veces. Si no me hubiera
evitado, ya te habría matado y dejado vivir al burro''.

\bibleverse{34} ``He pecado porque no me di cuenta de que estabas parado
en el camino para enfrentarme'', dijo Balaam al ángel del Señor, ``Así
que, si esto no es lo que quieres, volveré a casa''.

\bibleverse{35} El ángel del Señor le dijo a Balaam, ``No, puedes ir con
los hombres, pero sólo di lo que yo te diga''. Así que Balaam continuó
con los oficiales de Balac.

\hypertarget{la-llegada-de-balaam-a-balac}{%
\subsection{La llegada de Balaam a
Balac}\label{la-llegada-de-balaam-a-balac}}

\bibleverse{36} Cuando Balac se enteró de que Balaam estaba en camino,
fue a reunirse con él en el pueblo moabita en la frontera del río Arnón,
el punto más alejado de su territorio. \bibleverse{37} Le dijo a Balaam,
``¿No pensaste que mi llamada para que vinieras era urgente? ¿Por qué no
viniste a mí inmediatamente? ¿Pensaste que no podía pagarte lo
suficiente?''

\bibleverse{38} ``Mira, estoy aquí contigo ahora, ¿no?'' Balaam
respondió. ``¿Pero crees que puedo decir cualquier cosa? Sólo puedo
decir las palabras que Dios me da para que las diga''.

\bibleverse{39} Así que Balaam se fue con Balac y llegaron a
Quiriath-huzot. \bibleverse{40} Balac sacrificó ganado y ovejas, y
compartió la carne con Balaam y los líderes que estaban con él.
\bibleverse{41} A la mañana siguiente Balac llevó a Balaam hasta
Bamot-baal.\footnote{\textbf{22:41} ``Bamot-baal'': que significa ``Los
  Altos Lugares de Baal''. Algunos han llegado a la conclusión de que un
  templo pagano a Baal ocupaba este punto alto.} Desde allí pudo ver la
extensión del campamento israelita.\footnote{\textbf{22:41} Núm 23,28}

\hypertarget{los-preparativos-para-la-revelaciuxf3n-divina-el-primer-dicho-de-balaam}{%
\subsection{Los preparativos para la revelación divina; el primer dicho
de
Balaam}\label{los-preparativos-para-la-revelaciuxf3n-divina-el-primer-dicho-de-balaam}}

\hypertarget{section-22}{%
\section{23}\label{section-22}}

\bibleverse{1} Entonces Balaam le dijo a Balac, ``Constrúyeme siete
altares aquí, y prepárame siete toros y siete carneros para un
sacrificio''.

\bibleverse{2} Balac hizo lo que Balaam había dicho, y juntos ofrecieron
un toro y un carnero en cada altar. \bibleverse{3} Balaam le dijo a
Balac, ``Espera aquí junto a tu holocausto mientras voy a ver si quizás
el Señor vendrá y se reunirá conmigo. Cualquier cosa que me revele, la
compartiré contigo''. Entonces Balaam se fue a escalar un peñasco
rocoso.

\bibleverse{4} Dios se encontró con él allí, y Balaam dijo. ``He
construido siete altares y en cada uno de ellos he ofrecido un toro y un
carnero''.

\bibleverse{5} El Señor le dio a Balaam un mensaje para compartir. Le
dijo, ``Vuelve a Balac y esto es lo que debes decirle''.

\bibleverse{6} Así que volvió a Balac, que estaba esperando allí junto a
su holocausto, junto con todos los líderes moabitas.

\hypertarget{balaam-bendice-a-israel-desde-bamot-baal}{%
\subsection{Balaam bendice a Israel desde
Bamot-Baal}\label{balaam-bendice-a-israel-desde-bamot-baal}}

\bibleverse{7} Esta es la declaración que Balaam dio: ``Balac me trajo
de Aram; el rey de Moab me trajo de las montañas del este. Dijo: `¡Ven a
maldecir a Jacob por mí! Ven y condena a Israel'. \bibleverse{8} ``Pero
¿cómo puedo maldecir lo que Dios no ha maldito? ¿Cómo puedo condenar lo
que el Señor no ha condenado? \bibleverse{9} Porque yo los miro desde lo
alto de los peñascos; los observo desde las colinas. Veo un pueblo que
vive por su cuenta, diferente de las otras naciones. \bibleverse{10}
``¿Quién puede contar los descendientes de Jacob? ¡Son tantos que son
como el polvo! ¿Quién puede contar hasta una cuarta parte de los
israelitas? ``¡Me gustaría morir como muere una persona buena! ¡Que el
fin de mi vida sea como el fin de ellos!''

\bibleverse{11} Entonces Balac se quejó a Balaam, ``¿Qué esesto que me
has hecho? Te traje aquí para maldecir a mis enemigos, ¡y ahora mira!
¡Todo lo que has hecho es bendecirlos!''

\bibleverse{12} Pero Balaam respondió: ``¿No crees que debería decir
precisamente lo que el Señor me dice?'' \footnote{\textbf{23:12} Núm
  22,38}

\hypertarget{los-preparativos-para-la-nueva-revelaciuxf3n-divina-el-segundo-dicho-de-balaam}{%
\subsection{Los preparativos para la nueva revelación divina; el segundo
dicho de
Balaam}\label{los-preparativos-para-la-nueva-revelaciuxf3n-divina-el-segundo-dicho-de-balaam}}

\bibleverse{13} Entonces Balac le dijo: ``Por favor, ven conmigo a otro
lugar donde puedas verlos. Pero sólo verás una parte de su campamento,
no los verás a todos. Puedes maldecirlos por mí desde allí''.

\bibleverse{14} Lo llevó al campo de Zofim en la cima del Monte Pisga.
Allí construyó siete altares y ofreció un toro y un carnero en cada
altar. \bibleverse{15} Balaam le dijo a Balac, ``Espera aquí junto a tu
holocausto mientras me encuentro con el Señor allí''.

\bibleverse{16} El Señor se encontró con Balaam y le dio un mensaje para
compartir. Le dijo, ``Vuelve a Balac y esto es lo que debes decirle''.

\bibleverse{17} Así que volvió a Balac, que estaba esperando allí junto
a su holocausto, junto con todos los líderes moabitas. ``¿Qué dijo el
Señor?'' Preguntó Balac.

\hypertarget{balaam-bendice-a-israel-desde-el-monte-pisga}{%
\subsection{Balaam bendice a Israel desde el monte
Pisga}\label{balaam-bendice-a-israel-desde-el-monte-pisga}}

\bibleverse{18} Esta es la profecía que Balaam cumplió: ``¡Levántate,
Balac, y presta atención! ¡Escúchame, hijo de Zipor! \bibleverse{19}
``Dios no es un ser humano que mentiría. No es un simple mortal que
cambia de opinión. ¿Acaso él dice que va a hacer algo, pero no lo hace?
¿Acaso hace promesas que no cumple? \bibleverse{20} ``Mira, se me ha
ordenado dar una bendición. Dios ha bendecido, y no puedo cambiar eso.
\bibleverse{21} No esperes que le pase nada malo a Jacob; no se prevé
ningún problema para Israel. El Señor su Dios está con ellos; lo
celebran como su rey. \bibleverse{22} Dios los sacó de Egipto con gran
poder, tan fuerte como un buey. \bibleverse{23} No se puede lanzar
ningún hechizo contra Jacob; no se puede usar ninguna magia contra
Israel. La gente hablará de Jacob e Israel, diciendo: `¡Qué cosas tan
asombrosas ha hecho Dios por ellos!' \bibleverse{24} ¡Miren! Los
israelitas salen a cazar como una leona; persiguen como un león. No
descansan hasta que comen su presa, y beben la sangre de su víctima
muerta''. \footnote{\textbf{23:24} Núm 24,9}

\bibleverse{25} Entonces Balac le dijo a Balaam, ``¡Si no puedes darles
ninguna maldición, al menos no les des ninguna bendición!''

\bibleverse{26} Pero Balaam respondió: ``¿No te he explicado que tengo
que hacer todo lo que el Señor me diga?''

\hypertarget{los-preparativos-para-la-tercera-revelaciuxf3n-divina-el-tercer-dicho-de-balaam}{%
\subsection{Los preparativos para la tercera revelación divina; el
tercer dicho de
Balaam}\label{los-preparativos-para-la-tercera-revelaciuxf3n-divina-el-tercer-dicho-de-balaam}}

\bibleverse{27} ``Por favor, ven conmigo y te llevaré a otro lugar'',
dijo Balac. ``Tal vez Dios te permita maldecirlos por mí desde allí''.

\bibleverse{28} Balac llevó a Balaam a la cima del Monte Peor, que mira
hacia los páramos. \footnote{\textbf{23:28} Núm 25,3} \bibleverse{29}
Balaam le dijo a Balac, ``Constrúyeme siete altares aquí, y prepárame
siete toros y siete carneros para sacrificar''. \footnote{\textbf{23:29}
  Núm 23,1}

\bibleverse{30} Balac le dijo lo que Balaam le dijo, y ofreció un toro y
un carnero en cada altar.

\hypertarget{balaam-bendice-a-israel-desde-el-monte-peor}{%
\subsection{Balaam bendice a Israel desde el monte
Peor}\label{balaam-bendice-a-israel-desde-el-monte-peor}}

\hypertarget{section-23}{%
\section{24}\label{section-23}}

\bibleverse{1} Cuando Balaam vio que el Señor quería bendecir a Israel,
eligió no usar la adivinación como lo había hecho anteriormente. En su
lugar se volvió hacia el desierto, \bibleverse{2} y al mirar a Israel
acampado allí según sus respectivas tribus, el Espíritu de Dios vino
sobre él. \bibleverse{3} Hizo una declaración, diciendo: \footnote{\textbf{24:3}
  1Sam 9,9} \bibleverse{4} ``Esta es la profecía de Balaam, hijo de
Beor, la profecía de un hombre que ve con los ojos bien
abiertos,\footnote{\textbf{24:4} ``Con los ojos abiertos'': Esta palabra
  sólo aparece aquí y en el versículo 15. Se traduce como ``cerrado'' o
  ``abierto'', sin embargo, el significado es esencialmente claro en que
  Balaam se refiere a la visión profética. La Vulgata Latina tiene
  ``Ojos que están bloqueados'' mientras que la Septuaginta Griega dice:
  ``el que ve realmente''.} la profecía de uno que oye las palabras de
Dios, que ve la visión dada por el Todopoderoso, que se inclina con
respeto con los ojos abiertos. \footnote{\textbf{24:4} Is 50,4}
\bibleverse{5} ``¡Qué bien puestas tus tiendas, Jacob; los lugares donde
vives, Israel! \bibleverse{6} Parecen valles boscosos, como jardines
junto a un río, como árboles de áloe que el Señor ha plantado, como
cedros a la orilla del agua. \bibleverse{7} Los israelitas derramarán
cubos de agua; sus descendientes tendrán mucha agua. Su rey será más
grande que el rey Agag; su reino será glorioso. \bibleverse{8} Dios los
sacó de Egipto con gran poder, tan fuerte como un buey, destruyendo a
las naciones enemigas, rompiéndoles los huesos, atravesándolos con
flechas. \bibleverse{9} Son como un león que se agacha y se acuesta. Son
como una leona que nadie se atreve a molestar. Quienes los bendigan
serán bendecidos; y quienes los maldigan serán malditos''. \footnote{\textbf{24:9}
  Núm 23,24; Gén 49,9; Gén 12,3}

\hypertarget{la-ira-de-balac-y-la-disculpa-de-balaam}{%
\subsection{La ira de Balac y la disculpa de
Balaam}\label{la-ira-de-balac-y-la-disculpa-de-balaam}}

\bibleverse{10} Balac se enfadó con Balaam, y se golpeó los puños. Le
dijo a Balaam, ``Te traje aquí para maldecir a mis enemigos, ¡y ahora
mira! Sigues bendiciéndolos, haciéndolo tres veces. \bibleverse{11}
¡Vete ahora mismo! ¡Vete a casa! Prometí pagarte bien, pero el Señor se
ha asegurado de que no recibirás ningún pago''.

\bibleverse{12} Pero Balaam le dijo a Balac: ``¿No le expliqué ya a los
mensajeros que enviaste \bibleverse{13} que aunque me dieras todo tu
palacio lleno de plata y oro, no podría hacer nada de lo que quisiera ni
desobedecer el mandato del Señor mi Dios de ninguna manera? Sólo puedo
decir lo que el Señor me dice. \bibleverse{14} ¡Escucha! Ahora vuelvo a
casa con mi propio pueblo, pero primero déjame advertirte lo que estos
israelitas van a hacer con tu pueblo en el futuro''.

\hypertarget{cuarto-dicho-de-balaam-la-estrella-de-jacob-cuya-victoria-sobre-moab-y-edom}{%
\subsection{Cuarto dicho de Balaam: la estrella de Jacob; cuya victoria
sobre Moab y
Edom}\label{cuarto-dicho-de-balaam-la-estrella-de-jacob-cuya-victoria-sobre-moab-y-edom}}

\bibleverse{15} Entonces Balaam hizo una declaración, diciendo: ``Esta
es la profecía de Balaam, hijo de Beor, la profecía de un hombre con los
ojos bien abiertos \footnote{\textbf{24:15} Núm 24,3-4} \bibleverse{16}
la profecía de uno que escucha las palabras de Dios, que recibe el
conocimiento del Altísimo, que ve la visión dada por el Todopoderoso,
que se inclina con respeto con los ojos abiertos. \bibleverse{17} ``Lo
veo, pero esto no es ahora. Lo observo, pero esto no está cerca. En el
futuro, un líder como una estrella vendrá de Jacob, un gobernante con un
cetro llegará al poder desde Israel. Aplastará las cabezas de los
moabitas, y destruirá a todo el pueblo de Set.\footnote{\textbf{24:17}
  ``El pueblo de Set'': si esto se tomara literalmente, tal descripción
  también incluiría a los israelitas como descendientes de Set. En el
  contexto de la poesía paralela hebrea aquí probablemente se refiere
  específicamente a los moabitas. En el pasaje paralelo de Jeremías
  48:45 se lee ``pueblo rebelde''.} \bibleverse{18} El país de Edom será
conquistado, su enemigo Seir\footnote{\textbf{24:18} Seir era el nombre
  antiguo de Edom.} serán conquistados, y los israelitas saldrán
victoriosos. \footnote{\textbf{24:18} 2Sam 8,14; Am 9,11; Am 1,9-12}
\bibleverse{19} Un gobernante de Jacob vendrá y destruirá a los que
queden en la ciudad''. \footnote{\textbf{24:19} Miq 5,1; Miq 5,7-8}

\hypertarget{proverbios-sobre-los-amalecitas-ceneos-y-asirios-fin-de-la-historia-de-balaam}{%
\subsection{Proverbios sobre los amalecitas, ceneos y asirios; Fin de la
historia de
Balaam}\label{proverbios-sobre-los-amalecitas-ceneos-y-asirios-fin-de-la-historia-de-balaam}}

\bibleverse{20} Balaam dirigió su atención a los amalecitas y dio esta
declaración sobre ellos, diciendo, ``Amalec fue el primero entre las
naciones, pero terminarán siendo destruidos''. \footnote{\textbf{24:20}
  Éxod 17,14}

\bibleverse{21} Dirigió su atención a los ceneos y dio esta declaración
sobre ellos, diciendo, ``Donde vives está seguro y protegido, como un
nido en la cara de un acantilado. \footnote{\textbf{24:21} 1Sam 15,6;
  Abd 1,3} \bibleverse{22} Pero Kain será quemado cuando Asiria los
conquiste''.

\bibleverse{23} Luego Balaam hizo otra declaración, diciendo: ``¡Es una
tragedia! ¿Quién puede sobrevivir cuando Dios hace esto? \bibleverse{24}
Se enviarán barcos desde Chipre para atacar Asiria y Eber, pero también
serán destruidos permanentemente''. \footnote{\textbf{24:24} 1Macc 1,1}

\bibleverse{25} Entonces Balaam se marchó y volvió a su país, y Balac se
marchó también.

\hypertarget{la-deuda-de-israel-a-travuxe9s-de-la-fornicaciuxf3n-y-la-idolatruxeda}{%
\subsection{La deuda de Israel a través de la fornicación y la
idolatría}\label{la-deuda-de-israel-a-travuxe9s-de-la-fornicaciuxf3n-y-la-idolatruxeda}}

\hypertarget{section-24}{%
\section{25}\label{section-24}}

\bibleverse{1} Cuando los israelitas se alojaban en Sitím los hombres
empezaron a tener sexo con mujeres moabitas \bibleverse{2} que los
invitaban a los sacrificios hechos a sus dioses. Los israelitas comían
las comidas paganas y se inclinaban ante estos dioses. \bibleverse{3} De
esta manera los israelitas se dedicaban a la adoración de Baal de Peor,
y el Señor estaba enojado con ellos. \footnote{\textbf{25:3} Deut 4,3}
\bibleverse{4} El Señor le dijo a Moisés, ``Arresta a todos los líderes
israelitas y mátalos ante el Señor donde todos puedan ver\footnote{\textbf{25:4}
  ``Donde todos puedan ver'': literalmente, ``delante del sol''.} para
alejar la furiosa ira del Señor del pueblo''. \footnote{\textbf{25:4}
  2Sam 21,6; 2Sam 21,9; Deut 21,22-23}

\bibleverse{5} Así que Moisés instruyó a los jueces de
Israel,\footnote{\textbf{25:5} Estos eran ``jueces líderes'' que
  desempeñaban más que un rol legal en la sociedad israelita.} ``Cada
uno de ustedes tiene que matar a todos sus hombres que se han dedicado a
adorar a Baal de Peor''.

\hypertarget{la-intervenciuxf3n-de-phinehas-su-enajenaciuxf3n-de-dios-con-un-sacerdocio-eterno}{%
\subsection{La intervención de Phinehas; su enajenación de Dios con un
sacerdocio
eterno}\label{la-intervenciuxf3n-de-phinehas-su-enajenaciuxf3n-de-dios-con-un-sacerdocio-eterno}}

\bibleverse{6} En ese momento un hombre israelita llevó a una mujer
madianita a la tienda de su familia a la vista de Moisés y de todos los
israelitas mientras lloraban a la entrada del Tabernáculo de Reunión.
\bibleverse{7} Al ver esto, Finees, hijo de Eleazar, hijo del sacerdote
Aarón, abandonó la asamblea, agarró una lanza \bibleverse{8} y siguió al
hombre a su tienda. Allí, Finees atravesó con la lanza a ambos, al
israelita y al estómago de la mujer. Esta acción detuvo la plaga que
había empezado a matar a los israelitas, \bibleverse{9} pero ya habían
muerto 24. 000. \footnote{\textbf{25:9} 1Cor 10,8}

\bibleverse{10} El Señor le dijo a Moisés, \bibleverse{11} ``Finees hijo
de Eleazar, hijo del sacerdote Aarón, ha alejado mi ira de los
israelitas, porque de todos ellos estaba fervorosamente dedicado a mí,
así que no destruí a los israelitas en mi apasionada ira.
\bibleverse{12} Así que anunciad que le concedo mi acuerdo de paz.
\bibleverse{13} Será un acuerdo que asegura un sacerdocio permanente
para él y sus descendientes, porque se dedicó apasionadamente a su Dios
y enderezó a los israelitas''. \footnote{\textbf{25:13} Sal 106,30-31}

\bibleverse{14} El nombre del israelita que fue asesinado con la mujer
madianita era Zimri, hijo de Salu, un líder de la familia de la tribu de
Simeón. \bibleverse{15} El nombre de la mujer madianita que fue
asesinada era Cozbi, hija de Zur, un líder de familia de la tribu de
Madián.

\hypertarget{gottes-gebot-an-den-midianitern-rache-zu-nehmen}{%
\subsection{Gottes Gebot, an den Midianitern Rache zu
nehmen}\label{gottes-gebot-an-den-midianitern-rache-zu-nehmen}}

\bibleverse{16} El Señor le dijo a Moisés: \bibleverse{17} ``Ataca a los
madianitas y mátalos, \footnote{\textbf{25:17} Núm 31,2-10}

\bibleverse{18} porque te atacaron engañosamente, llevándote por mal
camino al usar a Peor y a su mujer Cozbi, la hija del líder madianita --
la mujer que fue asesinada el día que llegó la plaga por su devoción a
Peor --''.

\hypertarget{el-segundo-censo-de-la-gente-en-la-llanura-de-los-moabitas-con-el-propuxf3sito-de-distribuir-la-tierra}{%
\subsection{El segundo censo de la gente en la llanura de los moabitas
con el propósito de distribuir la
tierra}\label{el-segundo-censo-de-la-gente-en-la-llanura-de-los-moabitas-con-el-propuxf3sito-de-distribuir-la-tierra}}

\hypertarget{section-25}{%
\section{26}\label{section-25}}

\bibleverse{1} Después de que la plaga terminó, el Señor le dijo a
Moisés y Eleazar, hijo del sacerdote Aarón, \bibleverse{2} ``Censen a
todos los israelitas por familia, todos aquellos de veinte años o más
que sean elegibles para el servicio militar en el ejército de Israel''.
\bibleverse{3} Allí, en la llanura de Moab, junto al Jordán, frente a
Jericó, Moisés y Eleazar el sacerdote dio la orden, \bibleverse{4}
``Censar a los hombres de veinte años o más, siguiendo las instrucciones
que el Señor dio a Moisés''. El siguiente es el registro genealógico de
los que dejaron la tierra de Egipto.

\hypertarget{los-resultados-del-censo-1}{%
\subsection{Los resultados del censo}\label{los-resultados-del-censo-1}}

\bibleverse{5} Estos eran los descendientes de Rubén, el primogénito de
Israel: Hanoc, antepasado de la familia hanocítica; Falú, antepasado de
la familia faluita; \footnote{\textbf{26:5} Gén 46,8-27; 1Cró 4,1-7}
\bibleverse{6} Hezrón, antepasado de la familia hezronita; y Carmi,
antepasado de la familia carmita. \bibleverse{7} Estas fueron las
familias descendientes de Rubén y fueron 43. 730. \bibleverse{8} El hijo
de Falú era Eliab, \bibleverse{9} y los hijos de Eliab eran Nemuel,
Datán y Abiram. (Fueron Datán y Abiram, líderes escogidos por los
israelitas, los que se unieron a la rebelión contra Moisés y Aarón con
los seguidores de Coré cuando se rebelaron contra el Señor.
\bibleverse{10} La tierra se abrió y se los tragó, junto con Coré. Sus
seguidores murieron cuando el fuego quemó a 250 hombres. Lo que les
sucedió fue una advertencia para los israelitas. \bibleverse{11} Pero
los hijos de Coré no murieron).

\bibleverse{12} Estos fueron los descendientes de Simeón por familia:
Nemuel,\footnote{\textbf{26:12} Or ``Jemuel'', see the parallel lists in
  Génesis 46:10 y Éxodo 6:15.} antepasado de la familia Nemuelita;
Jamin, antepasado de la familia Jaminita; Jacín, antepasado de la
familia Jaquinita; \bibleverse{13} Zera,\footnote{\textbf{26:13} Escrito
  también como ``Zojar'' en las listas paralelas de Génesis 46:10 y
  Éxodo 6:15.} ancestro de la familia Zeraita; y Saul, ancestro de la
familia Saulita. \bibleverse{14} Estas eran las familias descendientes
de Simeón y eran 22. 200.

\bibleverse{15} Estos fueron los descendientes de Gad por familia:
Sefón,\footnote{\textbf{26:15} Escrito también Zefón en Génesis 46:15.}
ancestro de la familia sefonita; Haggi, ancestro de la familia Haggite;
Shuni, ancestro de la familia Shunite; \bibleverse{16} Ozni, ancestro de
la familia Oznite; Eri, ancestro de la familia Erite; \bibleverse{17}
Arod,\footnote{\textbf{26:17} Escrito también como Arodí en Génesis
  46:16.} antepasado de la familia Arodita; Areli, antepasado de la
familia Arelite. \bibleverse{18} Estas eran las familias descendientes
de Gad y eran 40. 500.

\bibleverse{19} Los hijos de Judá que murieron en Canaán fueron Er y
Onan. Estos eran los descendientes de Judá por familia: \footnote{\textbf{26:19}
  Gén 38,7; Gén 38,10} \bibleverse{20} Sela, antepasado de la familia
selaíta; Fares, antepasado de la familia faresita; Zera, antepasado de
la familia zeraíta. \bibleverse{21} Estos fueron los descendientes de
Fares: Hezrón, ancestro de la familia hezronita; y Hamul, ancestro de la
familia hamulita. \bibleverse{22} Estas eran las familias descendientes
de Judá y sumaban 76. 500.

\bibleverse{23} Estos fueron los descendientes de Isacar por familia:
Tola, antepasado de la familia tolaíta; Púa,\footnote{\textbf{26:23}
  Escrito como ``Puah'' en algunas traducciones antiguas.} antepasado de
la familia punita; \bibleverse{24} Jasub, antepasado de la familia
jasubita; y Simrón, antepasado de la familia simronita. \bibleverse{25}
Estas eran las familias descendientes de Isacar y sumaban 64. 300.

\bibleverse{26} Estos eran los descendientes de Zabulón por familia:
Sered, antepasado de la familia seredita; Elón, antepasado de la familia
elonita; y Jahleel, antepasado de la familia jahleelita. \bibleverse{27}
Estas eran las familias descendientes de Zabulón, y eran 60. 500.

\bibleverse{28} Estos fueron los descendientes de José por familia a
través de Manasés y Efraín: \bibleverse{29} Los descendientes de
Manasés: Maquir (era el padre de Galaad), antepasado de la familia
maquirita; y Galaad, antepasado de la familia galaadita. \footnote{\textbf{26:29}
  Jos 17,1-3} \bibleverse{30} Los descendientes de Galaad: Izer,
antepasado de la familia Iezerita; Heled, antepasado de la familia
helequita; \bibleverse{31} Asriel, antepasado de la familia asrielita;
Siquem, antepasado de la familia siquemita; \bibleverse{32} Semida,
antepasado de la familia Semidita; y Hefer, antepasado de la familia
heferita. \bibleverse{33} (Zelofehad, hijo de Hefer, no tuvo hijos, sólo
hijas. Se llamaban Maala, Noa, Hogla, Milca y Tirsa). \bibleverse{34}
Estas eran las familias que descendían de Manasés, y eran 52. 700.

\bibleverse{35} Estos eran los descendientes de Efraín por familia:
Sutela, antepasado de la familia sutelaíta; Bequer, antepasado de la
familia bequerita; y Tahán, antepasado de la familia tahanita.
\bibleverse{36} El descendiente de Suthelah era Erán, ancestro de la
familia eranita. \bibleverse{37} Estas eran las familias descendientes
de Efraín, y sumaban 32. 500. Estas familias eran descendientes de José.

\bibleverse{38} Estos eran los descendientes de Benjamín por familia:
Bela, antepasado de la familia Belaite; Asbel, antepasado de la familia
asbelita; Ahiram, antepasado de la familia ahiramita; \bibleverse{39}
Sufán,\footnote{\textbf{26:39} O ``Sefufán''.} antepasado de la familia
sufamita; y Hufam, antepasado de la familia hufamita. \bibleverse{40}
Los descendientes de Bela fueron Ard, ancestro de la familia de arditas;
y Naamán, ancestro de la familia Naamita. \bibleverse{41} Estas fueron
las familias descendientes de Benjamín, y sumaban 45. 600.

\bibleverse{42} Estos fueron los descendientes de Dan por familia:
Súham, antepasado de las familias Suhamitas. \bibleverse{43} Todas eran
familias suhamitas, y eran 64. 400.

\bibleverse{44} Estos eran los descendientes de Aser por familia: Imnah,
antepasado de la familia imnite; Isvi, antepasado de la familia isvita;
y Bería, antepasado de la familia beriaita. \bibleverse{45} Los
descendientes de Bería fueron Heber, antepasado de la familia heberita;
y Malquiel, antepasado de la familia malquielita. \bibleverse{46} El
nombre de la hija de Aser era Sera. \bibleverse{47} Estas eran las
familias descendientes de Aser, y sumaban 53. 400.

\bibleverse{48} Estos eran los descendientes de Neftalí por familia:
Jahzeel, antepasado de la familia jahzeelita; Guni, antepasado de la
familia gunita; \bibleverse{49} Jezer, antepasado de la familia
jezerita; y Silem, antepasado de la familia silemita. \bibleverse{50}
Estas eran las familias descendientes de Neftalí, y sumaban 45. 400.

\bibleverse{51} El total de todos los contados fue de 601. 730.

\hypertarget{instrucciuxf3n-sobre-distribuciuxf3n-de-tierras}{%
\subsection{Instrucción sobre distribución de
tierras}\label{instrucciuxf3n-sobre-distribuciuxf3n-de-tierras}}

\bibleverse{52} El Señor le dijo a Moisés: \bibleverse{53} ``Divide la
tierra que se va a poseer en función del número de los censados.
\bibleverse{54} Dale una mayor superficie de tierra a las tribus
grandes, y una menor superficie a las tribus más pequeña. Cada tribu
recibirá su asignación de tierra dependiendo de su número contado en el
censo. \bibleverse{55} ``La tierra debe ser dividida por sorteo. Cada
uno recibirá su tierra asignada en función del nombre de la tribu de su
antepasado. \footnote{\textbf{26:55} Núm 33,54; Jos 14,2}
\bibleverse{56} Cada asignación de tierra se dividirá por sorteo entre
las tribus, ya sean grandes o pequeñas''.

\hypertarget{el-conteo-de-los-levitas}{%
\subsection{El conteo de los levitas}\label{el-conteo-de-los-levitas}}

\bibleverse{57} Estos fueron los levitas censados por familia: Gerson,
antepasado de la familia gersonita; Coat, antepasado de la familia
coatita; y Merari, antepasado de la familia merarita. \bibleverse{58}
Las siguientes fueron las familias de los levitas: la familia libnita,
la familia hebronita, la familia mahlita, la familia musita y la familia
coraíta. Coat era el padre de Amram, \bibleverse{59} y el nombre de la
esposa de Amram era Jocabed. Era descendiente de Levi, nacida mientras
los levitas estaban en Egipto. Tuvo hijos con Amram: Aarón, Moisés y su
hermana Miriam. \bibleverse{60} Los hijos de Aarón fueron Nadab, Abihu,
Eleazar e Itamar, \bibleverse{61} pero Nadab y Abihu murieron cuando
ofrecieron fuego prohibido en presencia del Señor. \footnote{\textbf{26:61}
  Lev 10,1-2} \bibleverse{62} El número de los levitas censados ascendía
a 23. 000. Esto incluía a todos los varones de un mes o más. Sin
embargo, no fueron contados con los otros israelitas, porque no se les
dio ninguna asignación de tierras con los otros israelitas.

\bibleverse{63} Este es el registro de los que fueron censados por
Moisés y Eleazar el sacerdote cuando contaron a los israelitas en las
llanuras de Moab al lado del Jordán frente a Jericó. \bibleverse{64} Sin
embargo, no incluyeron ni uno solo que hubiera sido censado previamente
por Moisés y el sacerdote Aarón cuando contaron a los israelitas en el
desierto del Sinaí, \bibleverse{65} porque el Señor les había dicho que
todos morirían en el desierto. No quedó nadie excepto Caleb, hijo de
Jefone, y Josué, hijo de Nun.\footnote{\textbf{26:65} Núm 14,22-38}

\hypertarget{disposiciones-relativas-a-la-propiedad-de-los-herederos}{%
\subsection{Disposiciones relativas a la propiedad de los
herederos}\label{disposiciones-relativas-a-la-propiedad-de-los-herederos}}

\hypertarget{section-26}{%
\section{27}\label{section-26}}

\bibleverse{1} Las hijas de Zelofead vinieron a presentar su
caso.\footnote{\textbf{27:1} Ver también Josué 17:3-6.} Su padre
Zelofehad era hijo de Hefer, hijo de Galaad, hijo de Maquir, hijo de
Manasés, y era de la tribu de Manasés, hijo de José. Los nombres de sus
hijas eran Maala, Noa, Hogla, Milca y Tirsa. Vinieron \footnote{\textbf{27:1}
  Núm 26,33; Núm 36,2; Jos 17,3-6} \bibleverse{2} y se presentaron ante
Moisés, el sacerdote Eleazar, los líderes y todos los israelitas a la
entrada del Tabernáculo de Reunión. Dijeron, \bibleverse{3} ``Nuestro
padre murió en el desierto, pero no era uno de los seguidores de Coré
que se unieron para rebelarse contra el Señor. No, murió por sus propios
pecados, y no tuvo hijos. \footnote{\textbf{27:3} Núm 16,2; Núm 26,65}
\bibleverse{4} ¿Por qué debería perderse el nombre de nuestra familia
simplemente porque no tuvo un hijo? Danos tierra para que la poseamos
junto a nuestros tíos''.

\bibleverse{5} Moisés llevó su caso ante el Señor. \bibleverse{6} El
Señor le dio esta respuesta, \bibleverse{7} Lo que las hijas de
Zelofehad están diciendo es correcto. Debes darles tierra para que la
posean junto a sus tíos, dales lo que se le habría asignado a su padre.
\bibleverse{8} Además, dile a los israelitas: ``Si un hombre muere y no
tiene un hijo, dale su propiedad a su hija. \bibleverse{9} Si no tiene
una hija, da su propiedad a sus hermanos. \bibleverse{10} Si no tiene
hermanos, dé su propiedad a los hermanos de su padre. \bibleverse{11} Si
su padre no tiene hermanos, déle su propiedad a los parientes más
cercanos de su familia para que puedan ser dueños de ella. Esta es una
regulación legal para los israelitas, dada como una orden del Señor a
Moisés''.

\hypertarget{anuncio-de-muerte-inminente-a-moisuxe9s-instalaciuxf3n-de-joshua-como-su-sucesor}{%
\subsection{Anuncio de muerte inminente a Moisés; Instalación de Joshua
como su
sucesor}\label{anuncio-de-muerte-inminente-a-moisuxe9s-instalaciuxf3n-de-joshua-como-su-sucesor}}

\bibleverse{12} El Señor le dijo a Moisés: ``Sube a los montes de Abarim
para que veas la tierra que he dado a los israelitas. \footnote{\textbf{27:12}
  Deut 32,48-49} \bibleverse{13} Después que la hayas visto, también te
unirás a tus antepasados en la muerte, como lo hizo tu hermano Aarón,
\footnote{\textbf{27:13} Núm 20,24; Núm 20,28} \bibleverse{14} porque
cuando los israelitas se quejaron en el desierto de Zin, ambos se
rebelaron contra mis instrucciones de mostrar mi santidad ante ellos en
lo que respecta al suministro de agua''. (Estas fueron las aguas de
Meribá en Cades, en el desierto de Zin). \footnote{\textbf{27:14} Núm
  20,12-13}

\bibleverse{15} Entonces Moisés suplicó al Señor, \bibleverse{16} ``Que
el Señor, el Dios que da la vida a todos los seres vivos, elija un
hombre que guíe a los israelitas \bibleverse{17} que les diga qué hacer
y les muestre dónde ir, para que el pueblo del Señor no sea como ovejas
sin pastor''. \footnote{\textbf{27:17} Mat 9,36}

\bibleverse{18} El Señor le dijo a Moisés: ``Llama a Josué, hijo de Nun,
un hombre que tiene el Espíritu en él, y pon tus manos sobre él.
\footnote{\textbf{27:18} Deut 3,21; Deut 34,9} \bibleverse{19} Haz que
se ponga delante del sacerdote Eleazar y de todos los israelitas, y
dedícalo mientras ellos velan. \bibleverse{20} Entrégale algo de tu
autoridad para que todos los israelitas le obedezcan. \footnote{\textbf{27:20}
  2Re 2,9; 2Re 2,15} \bibleverse{21} Cuando necesite instrucciones
deberá ir ante Eleazar, el sacerdote, quien orará al Señor en su nombre
y consultará la decisión usando el Urim.\footnote{\textbf{27:21}
  Elemento que se usaba para determinar la voluntad del Señor. Ver Éxodo
  28:30, Levítico 8:8.} Josué les dará órdenes a todos los israelitas
sobre todo lo que deben hacer''. \footnote{\textbf{27:21} Éxod 28,30}

\bibleverse{22} Moisés siguió las instrucciones del Señor. Hizo que
Josué viniera y se pusiera delante del sacerdote Eleazar y de todos los
israelitas. \bibleverse{23} Moisés puso sus manos sobre Josué y lo
dedicó, tal como el Señor le había dicho que hiciera.

\hypertarget{normativa-sobre-los-sacrificios-comunitarios-diarios-y-diarios}{%
\subsection{Normativa sobre los sacrificios comunitarios diarios y
diarios}\label{normativa-sobre-los-sacrificios-comunitarios-diarios-y-diarios}}

\hypertarget{section-27}{%
\section{28}\label{section-27}}

\bibleverse{1} El Señor le dijo a Moisés: \bibleverse{2} ``Dales las
siguientes normas a los israelitas:\footnote{\textbf{28:2} Este pasaje
  es paralelo a las instrucciones dadas en Éxodo 29:38-41.} Debes
presentarme en el momento apropiado mis ofrendas de comida para que las
acepte. \footnote{\textbf{28:2} Lev 21,6}

\hypertarget{el-holocausto-diario-de-la-mauxf1ana-y-de-la-tarde}{%
\subsection{El holocausto diario de la mañana y de la
tarde}\label{el-holocausto-diario-de-la-mauxf1ana-y-de-la-tarde}}

\bibleverse{3} Diles que debes presentar al Señor cada día dos corderos
machos de un año como holocausto continua. \footnote{\textbf{28:3} Éxod
  29,38-42} \bibleverse{4} Ofrece un cordero por la mañana y otro por la
tarde antes de que oscurezca, \bibleverse{5} junto con una décima parte
de una efa de la mejor harina para una ofrenda de grano, mezclada con un
cuarto de hin de aceite de oliva prensado. \footnote{\textbf{28:5} Lev
  2,1} \bibleverse{6} ``Este es un holocausto continuo que se inició en
el Monte Sinaí como una ofrenda aceptable para el Señor. \bibleverse{7}
La ofrenda de bebida que acompaña a cada cordero debe ser un cuarto de
hin. Vierte la ofrenda de bebida fermentada al Señor en el santuario.
\bibleverse{8} Ofrecerás el segundo cordero por la tarde antes de que
oscurezca, junto con las mismas ofrendas de grano y bebida que por la
mañana. Es un holocausto aceptable para el Señor.

\hypertarget{la-ofrenda-adicional-del-suxe1bado}{%
\subsection{La ofrenda adicional del
sábado}\label{la-ofrenda-adicional-del-suxe1bado}}

\bibleverse{9} ``En el día de reposo, presentarás los corderos machos de
dos años, sin defectos, junto con una ofrenda de grano de dos décimas de
efa de la mejor harina mezclada con aceite de oliva, y su libación.
\bibleverse{10} Este holocausto debe ser presentado cada sábado además
del holocausto continuo y su libación.

\hypertarget{el-sacrificio-adicional-en-el-duxeda-de-luna-nueva}{%
\subsection{El sacrificio adicional en el día de luna
nueva}\label{el-sacrificio-adicional-en-el-duxeda-de-luna-nueva}}

\bibleverse{11} ``Al comienzo de cada mes, presentarán al Señor un
holocausto de dos novillos, un carnero y siete corderos machos de un
año, todos ellos sin defectos, \footnote{\textbf{28:11} Núm 10,10}
\bibleverse{12} junto con ofrendas de grano que consisten en tres
décimas de una efa de la mejor harina mezclada con aceite de oliva para
cada toro, dos décimas de una efa de la mejor harina mezclada con aceite
de oliva para el carnero, \footnote{\textbf{28:12} Núm 28,20; Núm 28,28;
  Núm 15,2-13} \bibleverse{13} y una décima de una efa de la mejor
harina mezclada con aceite de oliva para cada uno de los corderos. Este
es un holocausto aceptable para el Señor. \bibleverse{14} ``Sus
respectivas libaciones serán medio hin de vino por cada toro, un tercio
de hin por el carnero y un cuarto de hin por cada cordero. Este es el
holocausto mensual que se presentará cada mes durante el año.
\bibleverse{15} Además del holocausto continuo con su libación,
presentar un macho cabrío al Señor como ofrenda por el pecado.
\footnote{\textbf{28:15} Núm 28,22}

\hypertarget{las-ofrendas-adicionales-para-los-siete-duxedas-de-la-fiesta-de-los-panes-sin-levadura}{%
\subsection{Las ofrendas adicionales para los siete días de la Fiesta de
los Panes sin
Levadura}\label{las-ofrendas-adicionales-para-los-siete-duxedas-de-la-fiesta-de-los-panes-sin-levadura}}

\bibleverse{16} ``La Pascua del Señor es el día catorce del primer mes.
\footnote{\textbf{28:16} Lev 23,5-14} \bibleverse{17} Habrá una fiesta a
los quince días de este mes, y durante siete días sólo se comerá pan sin
levadura. \bibleverse{18} Celebrarán una reunión sagrada el primer día
de la fiesta. No hagan ninguna actividad de su trabajo normal.
\footnote{\textbf{28:18} Núm 28,25-26} \bibleverse{19} Preséntense ante
el Señor con las siguientes ofrendas: un holocausto de dos novillos, un
carnero y siete corderos de un año, todos ellos sin defectos.
\bibleverse{20} Sus ofrendas de grano se harán con la mejor harina
mezclada con aceite de oliva: tres décimas de efa para cada toro, dos
décimas de efa para el carnero, \bibleverse{21} y una décima de efa para
cada uno de los siete corderos. \bibleverse{22} Presenten también una
cabra macho como ofrenda por el pecado para hacerte justicia.
\footnote{\textbf{28:22} Núm 28,15} \bibleverse{23} Deben presentar
estas ofrendas además del continuo holocausto de la mañana.
\bibleverse{24} Presenta las mismas ofrendas todos los días durante
siete días como holocausto para ser aceptado por el Señor. Deben ser
ofrecidas con su libación y el continuo holocausto. \bibleverse{25}
Celebren una reunión sagrada el séptimo día del festival. Ese día no
harán su trabajo usual.

\hypertarget{los-sacrificios-adicionales-en-la-fiesta-de-las-primicias}{%
\subsection{Los sacrificios adicionales en la fiesta de las
primicias}\label{los-sacrificios-adicionales-en-la-fiesta-de-las-primicias}}

\bibleverse{26} ``Durante la celebración del Festival de las
Semanas,\footnote{\textbf{28:26} También llamado el ``Festival de la
  Cosecha'' en Éxodo 23:16.} celebrarán una reunión sagrada el día de
las primicias cuando presenten una ofrenda de grano nuevo al Señor. No
hagan ningún tipo de trabajo. \bibleverse{27} Presenten un holocausto de
dos novillos, un carnero y siete corderos de un año para ser aceptados
por el Señor. \bibleverse{28} Deben ir acompañados de sus ofrendas de
grano de la mejor harina mezclada con aceite de oliva: tres décimos de
un efa para cada toro, dos décimos de un efa para el carnero,
\bibleverse{29} y un décimo de un efa para cada uno de los siete
corderos. \bibleverse{30} Presenten también una cabra macho como ofrenda
para que los justifique. \footnote{\textbf{28:30} Núm 28,15}

\bibleverse{31} Presenten estas ofrendas junto con sus libaciones además
del holocausto continuo y su ofrenda de grano. Asegúrate de que los
animales sacrificados no tengan defectos''.

\hypertarget{los-sacrificios-adicionales-el-duxeda-de-auxf1o-nuevo}{%
\subsection{Los sacrificios adicionales el día de Año
Nuevo}\label{los-sacrificios-adicionales-el-duxeda-de-auxf1o-nuevo}}

\hypertarget{section-28}{%
\section{29}\label{section-28}}

\bibleverse{1} ``Celebren una reunión sagrada el primer día del séptimo
mes. No hagas nada de tu trabajo normal. Este es el día en que tocarás
las trompetas. \bibleverse{2} Presenten un holocausto de un novillo, un
carnero y siete corderos machos de un año, todos ellos sin defectos,
como sacrificio aceptable al Señor, \bibleverse{3} junto con sus
ofrendas de grano de la mejor harina mezclada con aceite de oliva: tres
décimos de un efa para el toro, dos décimos de un efa para el carnero,
\bibleverse{4} y un décimo de un efa para cada uno de los siete corderos
machos. \bibleverse{5} Presenten también una cabra macho como ofrenda
por el pecado para hacerte justicia. \bibleverse{6} Estas ofrendas se
suman a los holocaustos mensuales y diarios junto con las ofrendas de
grano y las ofrendas de bebida requeridas. Son ofrendas quemadas
aceptables para el Señor.

\hypertarget{los-sacrificios-adicionales-en-el-gran-duxeda-de-la-expiaciuxf3n}{%
\subsection{Los sacrificios adicionales en el gran día de la
expiación}\label{los-sacrificios-adicionales-en-el-gran-duxeda-de-la-expiaciuxf3n}}

\bibleverse{7} ``Celebrarás una reunión sagrada el décimo día de este
séptimo mes, y practiquen la abnegación. No hagas nada de tu trabajo
normal. \footnote{\textbf{29:7} Lev 23,27-32} \bibleverse{8} Presenta un
holocausto de un novillo, un carnero y siete corderos machos de un año,
todos ellos sin defectos, aceptables para el Señor. \bibleverse{9} Deben
ir acompañados de sus ofrendas de grano de la mejor harina mezclada con
aceite de oliva: tres décimas de efa para el toro, dos décimas de efa
para el carnero, \bibleverse{10} y una décima de efa para cada uno de
los siete corderos. \bibleverse{11} Presenta también un macho cabrío
como ofrenda por el pecado, además de la ofrenda por el pecado para
corregirte y el holocausto continuo con su ofrenda de grano y su
libación.

\hypertarget{las-ofrendas-adicionales-para-los-siete-duxedas-de-la-fiesta-de-los-tabernuxe1culos}{%
\subsection{Las ofrendas adicionales para los siete días de la Fiesta de
los
Tabernáculos}\label{las-ofrendas-adicionales-para-los-siete-duxedas-de-la-fiesta-de-los-tabernuxe1culos}}

\bibleverse{12} ``Celebrea una reunión sagrada el día quince del séptimo
mes. No hagas nada de tu trabajo normal. Debes celebrar un festival
dedicado al Señor durante siete días. \footnote{\textbf{29:12} Lev
  23,34-43} \bibleverse{13} Presenta como holocausto aceptable al Señor:
trece novillos, dos carneros y catorce corderos machos de un año, todos
ellos sin defectos. \bibleverse{14} Se acompañarán con sus ofrendas de
grano de la mejor harina mezclada con aceite de oliva: tres décimas de
efa de la mejor harina mezclada con aceite de oliva por cada uno de los
trece toros, dos décimas de efa por cada uno de los dos carneros,
\bibleverse{15} y una décima de efa por cada uno de los catorce
corderos. \bibleverse{16} También presentarás un macho cabrío como
ofrenda por el pecado además del holocausto continuo con su ofrenda de
grano y su libación.

\bibleverse{17} ``El segundo día, presente doce novillos, dos carneros y
catorce corderos machos de un año, todos ellos sin defectos.
\bibleverse{18} Deben ir acompañados por sus ofrendas de grano y bebidas
para los toros, carneros y corderos, todo según el número requerido.
\bibleverse{19} Presenta también un macho cabrío como ofrenda por el
pecado, además del continuo holocausto con su ofrenda de grano y su
libación.

\bibleverse{20} ``Al tercer día, presenta once novillos, dos carneros y
catorce corderos machos de un año de edad, todos ellos sin defectos.
\bibleverse{21} Deben estar acompañados por sus ofrendas de grano y
libaciones para los toros, carneros y corderos, todo de acuerdo al
número requerido. \bibleverse{22} Presenta también un macho cabrío como
ofrenda por el pecado además del holocausto continuo con su ofrenda de
grano y su libación.

\bibleverse{23} ``Al cuarto día presentarás diez novillos, dos carneros
y catorce corderos machos de un año, todos ellos sin defectos.
\bibleverse{24} Deben ir acompañados por sus ofrendas de grano y
libaciones para los toros, carneros y corderos, todo de acuerdo al
número requerido. \bibleverse{25} También presentarás un macho cabrío
como ofrenda por el pecado, además del continuo holocausto con su
ofrenda de grano y su libación.

\bibleverse{26} ``El quinto día presentarás nueve novillos, dos carneros
y catorce corderos machos de un año, todos ellos sin defectos.
\bibleverse{27} Deben ir acompañados por sus ofrendas de grano y
libaciones para los toros, carneros y corderos, todo de acuerdo con el
número requerido. \bibleverse{28} Presentarás también un macho cabrío
como ofrenda por el pecado, además del continuo holocausto con su
ofrenda de grano y su libación.

\bibleverse{29} ``Al sexto día presentarás ocho novillos, dos carneros y
catorce corderos machos de un año, todos ellos sin defectos.
\bibleverse{30} Deben estar acompañados por sus ofrendas de grano y
libaciones para los toros, carneros y corderos, todo de acuerdo al
número requerido. \bibleverse{31} También se presentará un macho cabrío
como ofrenda por el pecado, además del continuo holocausto con su
ofrenda de grano y su libación.

\bibleverse{32} ``Al séptimo día presentar siete novillos, dos carneros
y catorce corderos machos de un año de edad, todos ellos sin defectos.
\bibleverse{33} Deben estar acompañados por sus ofrendas de grano y
libaciones para los toros, carneros y corderos, todo de acuerdo al
número requerido. \bibleverse{34} También se presentará un macho cabrío
como ofrenda por el pecado, además del continuo holocausto con su
ofrenda de grano y su libación.

\bibleverse{35} ``En el octavo día todos ustedes se reunirán juntos. No
hagan nada de su trabajo normal. \bibleverse{36} Presenta como
holocausto aceptable al Señor: un toro, dos carneros y siete corderos
machos de un año, todos ellos sin defectos. \bibleverse{37} Deben ir
acompañados de sus ofrendas de grano y de las libaciones para los toros,
carneros y corderos, todo según el número requerido. \bibleverse{38}
También se presentará un macho cabrío como ofrenda por el pecado, además
del continuo holocausto con su ofrenda de grano y su libación.

\hypertarget{sentencia-final-de-las-leyes-de-vuxedctimas}{%
\subsection{Sentencia final de las leyes de
víctimas}\label{sentencia-final-de-las-leyes-de-vuxedctimas}}

\bibleverse{39} ``Presenta estas ofrendas al Señor en los momentos en
que se te requiera, además de tus ofrendas para cumplir una promesa y
las ofrendas de libre albedrío, ya sean holocaustos, ofrendas de grano,
libaciones o sacrificios de paz''. \bibleverse{40} Moisés repitió todo
esto a los israelitas como el Señor se lo ordenó.

\hypertarget{section-29}{%
\section{30}\label{section-29}}

\bibleverse{1} Moisés dijo a los jefes de las tribus de Israel: ``Esto
es lo que nos ordena el Señor:

\hypertarget{reglamento-sobre-la-vinculaciuxf3n-o-nulidad-de-los-votos}{%
\subsection{Reglamento sobre la vinculación o nulidad de los
votos}\label{reglamento-sobre-la-vinculaciuxf3n-o-nulidad-de-los-votos}}

\bibleverse{2} Si un hombre hace una promesa solemne al Señor, o promete
hacer algo jurando, no debe romper su promesa. Debe hacer todo lo que
dijo que haría.

\bibleverse{3} ``Si una mujer joven que aún vive en la casa de su padre
hace una promesa solemne al Señor o se compromete a hacer algo mediante
un juramento \bibleverse{4} y su padre se entera de su promesa o
juramento pero no le dice nada, todas las promesas o juramentos que ha
hecho se mantendrán. \bibleverse{5} Pero si su padre las rechaza tan
pronto como se entere, entonces ninguna de sus promesas o juramentos
serán válidos. El Señor la liberará de cumplirlas porque su padre las ha
desautorizado.

\bibleverse{6} ``Si una mujer se casa después de haber hecho una promesa
solemne o un juramento sin pensarlo \bibleverse{7} y su marido se entera
de ello pero no le dice nada inmediatamente, todas las promesas o
juramentos que haya hecho se mantendrán. \bibleverse{8} Pero si su
marido las rechaza cuando se entera de ello, entonces ninguna de sus
promesas o juramentos permanecen válidos y el Señor la liberará de
cumplirlos.

\bibleverse{9} ``Toda promesa solemne hecha por una viuda o una mujer
divorciada debe cumplirse.

\bibleverse{10} ``Si una mujer que vive con su marido hace una promesa
solemne al Señor o se compromete a hacer algo mediante un juramento,
\bibleverse{11} y su marido se entera de su promesa o juramento pero no
le dice nada y no lo desautoriza, entonces ninguna de sus promesas o
juramentos permanecen válidos. \bibleverse{12} Pero si su marido las
rechaza tan pronto como se entera de ello, entonces ninguna de sus
promesas o juramentos siguen siendo válidos. El Señor la liberará de
mantenerlas porque su marido las ha rechazado. \bibleverse{13} ``Su
marido también puede confirmar o rechazar cualquier promesa o juramento
solemne que la mujer haga para negarse a sí misma.

\hypertarget{promulgaciuxf3n-renovada-de-los-derechos-del-marido}{%
\subsection{Promulgación renovada de los derechos del
marido}\label{promulgaciuxf3n-renovada-de-los-derechos-del-marido}}

\bibleverse{14} Pero si su marido no le dice nunca una palabra al
respecto, se supone que ha confirmado todas las promesas y juramentos
solemnes que ella ha hecho. \bibleverse{15} Sin embargo, si él las
rechaza algún tiempo después de enterarse de ellas, entonces él tendrá
la responsabilidad de que ella las rompa''.

\bibleverse{16} Estos son los preceptos que el Señor dio a Moisés sobre
la relación entre un hombre y su esposa, y entre un padre y una hija que
es joven y todavía vive en casa.

\hypertarget{guerra-de-venganza-de-los-israelitas-contra-los-madianitas}{%
\subsection{Guerra de venganza de los israelitas contra los
madianitas}\label{guerra-de-venganza-de-los-israelitas-contra-los-madianitas}}

\hypertarget{section-30}{%
\section{31}\label{section-30}}

\bibleverse{1} El Señor le dijo a Moisés, \bibleverse{2} ``Castiga a los
madianitas por lo que le hicieron a los israelitas. Después de eso te
unirás a tus antepasados en la muerte''. \footnote{\textbf{31:2} Núm
  25,17; Núm 27,13}

\bibleverse{3} Moisés instruyó al pueblo: ``Que algunos de tus hombres
se preparen para la batalla, para que puedan ir a atacar a los
madianitas y llevar a cabo el castigo del Señor sobre ellos.
\bibleverse{4} Debes contribuir con mil hombres de cada tribu
israelita''. \bibleverse{5} Así que se eligieron mil hombres de cada
tribu israelita, haciendo doce mil tropas listas para la batalla.
\bibleverse{6} Moisés los envió a la batalla, mil de cada tribu, junto
con Finees, hijo del sacerdote Eleazar. Llevó consigo los objetos
sagrados del santuario y las trompetas usadas para dar señales.
\bibleverse{7} Atacaron a los madianitas, como el Señor le había dicho a
Moisés, y mataron a todos los hombres. \footnote{\textbf{31:7} Éxod
  20,13} \bibleverse{8} Entre los muertos estaban los cinco reyes de
Madián, Evi, Rekem, Zur, Hur y Reba. También mataron a Balaam, hijo de
Beor, con la espada. \footnote{\textbf{31:8} Jos 13,21-22; Núm 22,5}
\bibleverse{9} Los israelitas capturaron a las mujeres y niños
madianitas, y tomaron como botín todas sus manadas, rebaños y
posesiones. \bibleverse{10} Prendieron fuego a todos los pueblos y
campamentos madianitas donde habían vivido, \bibleverse{11} y se
llevaron todo el saqueo y el botín, incluyendo personas y animales.
\bibleverse{12} Llevaron los prisioneros, el saqueo y el pillaje a
Moisés, al sacerdote Eleazar y al resto de los israelitas donde estaban
acampados en las llanuras de Moab, junto al Jordán, frente a Jericó.

\hypertarget{ordenanza-sobre-la-matanza-de-todos-los-niuxf1os-varones-sobre-el-trato-de-las-reclusas-y-los-niuxf1os-y-sobre-la-limpieza-que-se-debe-realizar-antes-del-regreso}{%
\subsection{Ordenanza sobre la matanza de todos los niños varones, sobre
el trato de las reclusas y los niños y sobre la limpieza que se debe
realizar antes del
regreso}\label{ordenanza-sobre-la-matanza-de-todos-los-niuxf1os-varones-sobre-el-trato-de-las-reclusas-y-los-niuxf1os-y-sobre-la-limpieza-que-se-debe-realizar-antes-del-regreso}}

\bibleverse{13} Moisés, Eleazar el sacerdote y todos los líderes
israelitas salieron del campamento para encontrarse con ellos.
\bibleverse{14} Moisés estaba enfadado con los oficiales del ejército,
los comandantes de miles y los comandantes de cientos, que volvieron de
la batalla. \bibleverse{15} ``¿Por qué dejaste vivir a todas las
mujeres?'' les preguntó. \bibleverse{16} ``¡Noten que estas mujeres
sedujeron a los hombres israelitas, llevándolos a ser infieles al Señor
en Peor, siguiendo el consejo de Balaam! Por eso el pueblo del Señor
sufrió la plaga. \footnote{\textbf{31:16} Núm 25,1; Apoc 2,14}
\bibleverse{17} Así que ve y mata a todos los niños y a todas las
mujeres que se hayan acostado con un hombre. \footnote{\textbf{31:17}
  Jue 21,11} \bibleverse{18} Deja vivir a todas las chicas que son
vírgenes. Son tuyas.

\bibleverse{19} Y todos aquellos que mataron a alguien o tocaron un
cadáver deben permanecer fuera del campamento durante siete días.
Purifíquense y purifiquen a sus prisioneros al tercer y séptimo día.
\footnote{\textbf{31:19} Núm 19,11} \bibleverse{20} También purifiquen
toda su ropa y cualquier cosa hecha de cuero, pelo de cabra o madera''.

\bibleverse{21} El sacerdote Eleazar dijo a los soldados que habían ido
a la batalla: ``Estos son los preceptos legales que el Señor ha ordenado
llevar a cabo a Moisés: \bibleverse{22} Todo lo que esté hecho de oro,
plata, bronce, hierro, estaño y plomo, \bibleverse{23} todo lo que no se
queme, debe ser puesto al fuego para que quede limpio. Pero todavía
tiene que ser purificado usando agua de purificación. Todo lo que se
quema debe ser pasado por el agua. \bibleverse{24} Lava tu ropa en el
séptimo día y estarás limpio. Entonces podrás entrar en el campamento''.

\hypertarget{distribuciuxf3n-de-presas-vivas-humanos-y-ganado-regalo-de-navidad-de-los-luxedderes}{%
\subsection{Distribución de presas vivas (humanos y ganado); Regalo de
Navidad de los
líderes}\label{distribuciuxf3n-de-presas-vivas-humanos-y-ganado-regalo-de-navidad-de-los-luxedderes}}

\bibleverse{25} El Señor le dijo a Moisés, \bibleverse{26} ``Tú, el
sacerdote Eleazar, y los líderes de la familia israelita deben tomar un
registro de las personas y animales que fueron capturados.
\bibleverse{27} Luego divídanlos entre las tropas que entraron en
batalla y el resto de los israelitas. \bibleverse{28} Tomen como
contribución al Señor de lo que se asigna a las tropas que fueron a la
batalla una de cada quinientas personas, ganado, asnos u ovejas.
\bibleverse{29} Tomen esto de su media parte y denlo al sacerdote
Eleazar como ofrenda al Señor. \bibleverse{30} ``De los israelitas; la
mitad de la parte, toma una de cada cincuenta personas, ganado, asnos u
ovejas, u otros animales, y dáselos a los levitas que cuidan del
Tabernáculo del Señor''.

\bibleverse{31} Moisés y el sacerdote Eleazar hicieron lo que el Señor
había ordenado a Moisés.

\bibleverse{32} Esta era la lista de los botines que quedaban y que
habían sido saqueados por las tropas: 675. 000 ovejas, \bibleverse{33}
72. 000 vacas, \bibleverse{34} 61. 000 burros, \bibleverse{35} y 32. 000
vírgenes. \bibleverse{36} Esta era la mitad de los que habían ido a
luchar: 337. 500 ovejas, \bibleverse{37} con una contribución para el
Señor de 675 \bibleverse{38} 36. 000 bovinos, con una contribución para
el Señor de 72, \bibleverse{39} 30. 500 burros, con una contribución
para el Señor de 61, \bibleverse{40} y 16. 000 personas, con una
contribución para el Señor de 32. \bibleverse{41} Moisés dio la
contribución al sacerdote Eleazar como ofrenda al Señor, como el Señor
había ordenado a Moisés. \bibleverse{42} La mitad de la parte de los
israelitas se fue después de que Moisés diera la mitad de la parte a las
tropas que habían ido a luchar, \bibleverse{43} consistió en: 337. 500
ovejas, \bibleverse{44} 36. 000 vacas, \bibleverse{45} 30. 500 burros,
\bibleverse{46} y 16. 000 personas. \bibleverse{47} Moisés tomó de la
mitad de los israelitas una de cada cincuenta personas y animales y les
dio los levitas que cuidan del Tabernáculo del Señor, como el Señor le
había ordenado.

\bibleverse{48} Los oficiales del ejército, los comandantes de millares
y los comandantes de centenas, se acercaron a Moisés \bibleverse{49} y
le dijeron: ``Nosotros, tus siervos, hemos comprobado las tropas que
mandamos y no falta ni un solo hombre. \bibleverse{50} Así que hemos
traído al Señor una ofrenda de los objetos de oro que cada hombre
recibió: brazaletes, pulseras, anillos, pendientes y collares, para que
podamos estar bien ante el Señor''.

\bibleverse{51} El sacerdote Moisés y Eleazar aceptaron de ellos todos
los objetos de oro. \bibleverse{52} El oro que los comandantes de miles
y cientos de personas dieron como ofrenda al Señor pesaba en total 16.
750 siclos. \bibleverse{53} (Los hombres que habían luchado en la
batalla habían tomado cada uno un botín para sí mismos). \bibleverse{54}
Moisés y el sacerdote Eleazar aceptaron el oro de los comandantes de
miles y cientos y lo llevaron al Tabernáculo de Reunión como ofrenda
conmemorativa para los israelitas en presencia del Señor.

\hypertarget{la-peticiuxf3n-de-los-rubenitas-y-gaditas-fue-rechazada-por-moisuxe9s-en-un-discurso-punitivo}{%
\subsection{La petición de los rubenitas y gaditas fue rechazada por
Moisés en un discurso
punitivo}\label{la-peticiuxf3n-de-los-rubenitas-y-gaditas-fue-rechazada-por-moisuxe9s-en-un-discurso-punitivo}}

\hypertarget{section-31}{%
\section{32}\label{section-31}}

\bibleverse{1} Las tribus de Rubén y Gad tenían grandes cantidades de
ganado y vieron que la tierra de Jazer y Galaad era un buen lugar para
criarlos. \bibleverse{2} Entonces vinieron a Moisés, al sacerdote
Eleazar y a los líderes israelitas y dijeron, \bibleverse{3} ``Las
ciudades de Atarot, Dibón, Jazer, Nimra, Hesbón, Eleale,
Sebam,\footnote{\textbf{32:3} También conocido como Sibma en el
  versículo 38.} Nebo y Beón, \bibleverse{4} que el Señor conquistó a la
vista de los israelitas, son adecuados para el ganado que poseemos tus
siervos''. \bibleverse{5} Continuaron: ``Por favor, responde
favorablemente a nuestra petición: danos esta tierra. No nos hagas
cruzar el Jordán''.

\bibleverse{6} En respuesta Moisés preguntó a las tribus de Gad y Rubén:
``¿Esperas que tus hermanos vayan a luchar mientras tú te quedas aquí
sentado? \bibleverse{7} ¿Por qué desanimar a los israelitas para que no
crucen al país que el Señor les ha dado? \bibleverse{8} Esto es lo que
hicieron sus padres cuando los envié desde Cades-barnea a explorar la
tierra. \footnote{\textbf{32:8} Núm 13,-1} \bibleverse{9} Después de que
sus padres viajaron por el valle de Escol y exploraron la tierra,
desalentaron a los israelitas, persuadiéndolos de que no entraran en el
país que el Señor les había dado. \bibleverse{10} Como resultado,
hicieron enojar mucho al Señor ese día, y él hizo este juramento,
\bibleverse{11} `Ni uno solo de los que salvé de Egipto, que tenga
veinte años o más, verá jamás la tierra que prometí con el juramento de
dar a Abraham, Isaac y Jacob, porque no estaban completamente
comprometidos conmigo, \bibleverse{12} nadie excepto Caleb, hijo de
Jefone, el cenesita, y Josué, hijo de Nun, porque estaban completamente
comprometidos conmigo'. \bibleverse{13} El Señor se enojó con Israel y
los hizo vagar por el desierto durante cuarenta años, hasta que murió
toda la generación que había hecho el mal ante sus ojos.

\bibleverse{14} ``¡Miraos ahora, hijos de pecadores que han venido a
ocupar el lugar de sus padres para hacer que el Señor se enfade aún más
con Israel! \bibleverse{15} Si dejas de seguirlo, él volverá a abandonar
a esta gente en el desierto, y su muerte será culpa tuya!''

\hypertarget{la-respuesta-de-los-rubenitas-y-gaditas}{%
\subsection{La respuesta de los rubenitas y
gaditas}\label{la-respuesta-de-los-rubenitas-y-gaditas}}

\bibleverse{16} Entonces las tribus de Gad y Rubén vinieron a Moisés y
le dijeron: ``Planeamos construir muros de piedra para mantener a salvo
nuestro ganado y pueblos para nuestros hijos. \bibleverse{17} Pero aún
así nos prepararemos para la batalla, y estaremos preparados para
liderar a los israelitas hasta que puedan ocupar su tierra con
seguridad. Mientras tanto, nuestros hijos se quedarán atrás, viviendo en
los pueblos fortificados para protegerlos de la población local.
\bibleverse{18} No regresaremos a nuestros hogares hasta que cada
israelita esté en posesión de su tierra asignada. \bibleverse{19} Sin
embargo, no poseeremos ninguna tierra al otro lado del Jordán porque
hemos recibido esta tierra para poseerla en este lado oriental del
Jordán''.

\hypertarget{la-promesa-de-moisuxe9s-declarando-las-condiciones-otorgando-el-pauxeds-al-este-del-jorduxe1n-a-las-tribus-suplicantes}{%
\subsection{La promesa de Moisés, declarando las condiciones; Otorgando
el País al este del Jordán a las tribus
suplicantes}\label{la-promesa-de-moisuxe9s-declarando-las-condiciones-otorgando-el-pauxeds-al-este-del-jorduxe1n-a-las-tribus-suplicantes}}

\bibleverse{20} Moisés respondió: ``Si esto es lo que realmente harán,
si se preparan para la batalla bajo la dirección del Señor, \footnote{\textbf{32:20}
  Jos 1,13-15} \bibleverse{21} y si todas sus tropas cruzan el Jordán
con el Señor hasta que haya expulsado a sus enemigos delante de él,
\bibleverse{22} entonces una vez que el país sea conquistado con la
ayuda del Señor entonces podrán regresar, y habrán cumplido sus
obligaciones con el Señor y con Israel. Serás dueño de esta tierra, que
te ha sido concedida por el Señor.

\bibleverse{23} Pero si no lo haces, claramente estarás pecando contra
el Señor, y las consecuencias de tu pecado te alcanzarán.
\bibleverse{24} Adelante, construye ciudades para tus hijos y muros de
piedra para tus rebaños, pero asegúrate de hacer lo que has prometido''.

\bibleverse{25} Las tribus de Gad y Rubén prometieron a Moisés, ``Señor,
nosotros, tus siervos, haremos lo que tú has ordenado. \bibleverse{26}
Nuestras esposas e hijos, nuestro ganado y todos nuestros animales,
permanecerán aquí en los pueblos de Galaad. \bibleverse{27} Pero
nosotros, tus siervos, estamos listos para la batalla, y todas nuestras
tropas cruzarán para luchar con la ayuda del Señor, tal como tú has
dicho, señor''.

\bibleverse{28} Moisés les dio las siguientes instrucciones sobre ellos
al sacerdote Eleazar, a Josué, hijo de Nun, y a los jefes de familia de
las tribus de Israel. \bibleverse{29} Moisés les dijo: ``Si las tribus
de Gaditas y Rubén cruzan el Jordán contigo, con todas sus tropas listas
para la batalla con la ayuda del Señor, y la tierra es conquistada a
medida que avanzas, entonces dales la tierra de Galaad para que la
posean. \bibleverse{30} Pero si no se preparan para la batalla y cruzan
contigo, entonces deben aceptar su tierra asignada entre ustedes en el
país de Canaán''.

\bibleverse{31} Las tribus de Gad y Rubén respondieron: ``Haremos lo que
el Señor nos ha dicho, como sus siervos. \bibleverse{32} Cruzaremos y
entraremos en el país de Canaán listos para la batalla con la ayuda del
Señor, para que podamos tener nuestra parte de tierra asignada a este
lado del Jordán''.

\bibleverse{33} Moisés dio a las tribus de Gad y Rubén y a la media
tribu de Manasés, hijo de José, el reino de Sehón, rey de los amorreos,
y el reino de Og, rey de Basán. Esta tierra incluía sus ciudades y sus
alrededores. \footnote{\textbf{32:33} Jos 13,8-31}

\hypertarget{resumen-de-las-ciudades-reconstruidas-por-los-gaditas-y-los-rubenitas}{%
\subsection{Resumen de las ciudades reconstruidas por los gaditas y los
rubenitas}\label{resumen-de-las-ciudades-reconstruidas-por-los-gaditas-y-los-rubenitas}}

\bibleverse{34} Los pueblos de Gad reconstruyeron Dibon, Ataroth, Aroer,
\bibleverse{35} Atarot-sofán, Jazer, Jogbeha, \bibleverse{36} Bet-nimra
y Bet-arán como ciudades fortificadas, y construyeron muros de piedra
para sus rebaños. \bibleverse{37} El pueblo de Rubén reconstruyó Hesbón,
Eleale, Quiriataim, \bibleverse{38} así como Nebo y Baal-meón (cambiando
sus nombres), y Sibma. De hecho, cambiaron el nombre de los pueblos que
reconstruyeron.

\hypertarget{los-descendientes-de-manasuxe9s-se-establecieron-en-la-ribera-oriental}{%
\subsection{Los descendientes de Manasés se establecieron en la Ribera
Oriental}\label{los-descendientes-de-manasuxe9s-se-establecieron-en-la-ribera-oriental}}

\bibleverse{39} Los descendientes de Maquir, hijo de Manasés, atacaron a
Galaad y lo capturaron. Expulsaron a los amorreos que vivían allí.
\bibleverse{40} Entonces Moisés entregó a Galaad a la familia de Maquir,
hijo de Manasés, y se establecieron allí. \bibleverse{41} Jair, un
descendiente de Manasés, atacó sus pueblos y los capturó. Los llamó las
Aldeas de Jair. \bibleverse{42} Noba atacó a Kenat y la capturó, junto
con sus aldeas. La nombró Nobah en su honor.

\hypertarget{lista-de-los-campamentos-en-los-que-pasaron-los-israelitas-durante-los-cuarenta-auxf1os-del-desierto}{%
\subsection{Lista de los campamentos en los que pasaron los israelitas
durante los cuarenta años del
desierto}\label{lista-de-los-campamentos-en-los-que-pasaron-los-israelitas-durante-los-cuarenta-auxf1os-del-desierto}}

\hypertarget{section-32}{%
\section{33}\label{section-32}}

\bibleverse{1} Este es un registro de los viajes realizados por los
israelitas al salir de Egipto en sus divisiones tribales lideradas por
Moisés y Aarón. \bibleverse{2} Moisés registró las diferentes partes de
su viaje según las instrucciones del Señor. Estos son los viajes que
hicieron listados en orden desde donde comenzaron: \bibleverse{3} Los
israelitas dejaron Ramsés el día quince del primer mes, el día después
de la Pascua. Salieron triunfantes mientras todos los egipcios
observaban. \footnote{\textbf{33:3} Éxod 1,11; Éxod 14,8} \bibleverse{4}
Los egipcios enterraban a todos sus primogénitos que el Señor había
matado, porque el Señor había hecho caer sus juicios sobre sus dioses.
\footnote{\textbf{33:4} Éxod 12,12} \bibleverse{5} Los israelitas
dejaron Ramsés e instalaron un campamento en Sucot. \footnote{\textbf{33:5}
  Éxod 12,37} \bibleverse{6} Se fueron de Sucot y acamparon en Etam, en
la frontera con el desierto. \footnote{\textbf{33:6} Éxod 13,20}
\bibleverse{7} Se alejaron de Etam, volviendo a Pi-hahiroth, frente a
Baal-zefón, y acamparon cerca de Migdol. \footnote{\textbf{33:7} Éxod
  14,2} \bibleverse{8} Se mudaron de Pi-hahirot\footnote{\textbf{33:8}
  Ver Éxodo 14:2.} y cruzó por el medio del mar hacia el desierto.
Viajaron durante tres días al desierto de Etham y establecieron un
campamento en Marah. \footnote{\textbf{33:8} Éxod 14,22; Éxod 15,23}
\bibleverse{9} Se desplazaron desde Mara y llegaron a Elim, donde había
doce manantiales de agua y setenta palmeras, y acamparon allí.
\footnote{\textbf{33:9} Éxod 15,27} \bibleverse{10} Se trasladaron de
Elim y acamparon al lado del Mar Rojo. \bibleverse{11} Se trasladaron
desde el Mar Rojo y acamparon en el Desierto del Pecado. \bibleverse{12}
Se trasladaron del desierto de Sin y acamparon en Dofca. \bibleverse{13}
Se mudaron de Dofca y acamparon en Alús. \bibleverse{14} Se mudaron de
Alús y acamparon en Refidím. No había agua allí para que la gente
bebiera. \footnote{\textbf{33:14} Éxod 17,1} \bibleverse{15} Se fueron
de Refidim y acamparon en el desierto del Sinaí. \footnote{\textbf{33:15}
  Éxod 19,1} \bibleverse{16} Se fueron del desierto del Sinaí y
acamparon en Kibroth-hataava. \footnote{\textbf{33:16} Núm 11,34}
\bibleverse{17} Se mudaron de Kibroth-hattaavah y acamparon en Hazerot.
\footnote{\textbf{33:17} Núm 11,35} \bibleverse{18} Se trasladaron de
Hazerot y establecieron un campamento en Ritma. \footnote{\textbf{33:18}
  Núm 12,16} \bibleverse{19} Se trasladaron de Ritma y establecieron un
campamento en Rimón-fares. \bibleverse{20} Se trasladaron de
Rimmon-fares y acamparon en Libna. \bibleverse{21} Se trasladaron de
Libna y establecieron un campamento en Rissa. \bibleverse{22} Se
trasladaron de Rissa y establecieron un campamento en Ceelata.
\bibleverse{23} Se trasladaron de Ceelata y acamparon en el Monte Sefer.
\bibleverse{24} Se trasladaron del Monte Sefer y acamparon en Harada.
\bibleverse{25} Se trasladaron de Harada y acamparon en Macelot.
\bibleverse{26} Se trasladaron de Macelot y acamparon en Tahat.
\bibleverse{27} Se fueron de Tahat y acamparon en Tara. \bibleverse{28}
Se mudaron de Tara y acamparon en Mitca. \bibleverse{29} Se mudaron de
Mitca y acamparon en Hasmona. \bibleverse{30} Se fueron de Hasmona y
acamparon en Moserot. \bibleverse{31} Se mudaron de Moserot y acamparon
en Bene-jaacán. \bibleverse{32} Se mudaron de Bene-jaacán y acamparon en
Hor-haggidgad. \bibleverse{33} Se trasladaron de Hor-haggidgad y
acamparon en Jotbata. \footnote{\textbf{33:33} Deut 10,7}
\bibleverse{34} Se mudaron de Jotbata y establecieron un campamento en
Abrona. \bibleverse{35} Se mudaron de Abrona y acamparon en Ezión-geber.
\bibleverse{36} Se trasladaron de Ezion-geber y acamparon en Cades, en
el desierto de Zin. \bibleverse{37} Se trasladaron de Cades y acamparon
en el monte Hor, en la orilla de Edom. \footnote{\textbf{33:37} Núm
  20,22-29} \bibleverse{38} El sacerdote Aarón subió al monte Hor como
el Señor le había ordenado, y murió allí el primer día del quinto mes,
en el cuadragésimo año después de que los israelitas hubieran salido de
Egipto. \bibleverse{39} Aarón tenía 123 años cuando murió en el Monte
Hor. \bibleverse{40} (El rey cananeo de Arad, que vivía en el Néguev en
el país de Canaán, se enteró de que los israelitas estaban en camino).
\bibleverse{41} Los israelitas se trasladaron del Monte Hor y
establecieron un campamento en Zalmona. \bibleverse{42} Se trasladaron
de Zalmona y acamparon en Punón. \bibleverse{43} Se trasladaron de Punón
y acamparon en Obot. \footnote{\textbf{33:43} Núm 21,10} \bibleverse{44}
Se trasladaron de Oboth y acamparon en Iye-abarim, en la frontera de
Moab. \footnote{\textbf{33:44} Núm 21,11} \bibleverse{45} Se mudaron de
Iye-abarim\footnote{\textbf{33:45} Como se escribe en el versículo
  21:11. Aquí el nombre se menciona como ``Iyim''.} y acamparon en
Dibon-gad. \bibleverse{46} Se mudaron de Dibon-gad y acamparon en
Almon-diblataim. \bibleverse{47} Se mudaron de Almon-diblataim y
acamparon en las montañas de Abarim, frente a Nebo. \footnote{\textbf{33:47}
  Núm 21,20} \bibleverse{48} Se trasladaron de las montañas de Abarim y
acamparon en las llanuras de Moab, junto al Jordán, frente a Jericó.
\footnote{\textbf{33:48} Núm 22,1; Deut 32,49} \bibleverse{49} Allí, en
las llanuras de Moab, acamparon al lado del Jordán, desde Beth-jesimot
hasta Abel-sitim. \footnote{\textbf{33:49} Núm 25,1}

\hypertarget{ordenanzas-provisionales-de-dios-con-respecto-a-la-conquista-y-distribuciuxf3n-de-cisjordania-de-canauxe1n}{%
\subsection{Ordenanzas provisionales de Dios con respecto a la conquista
y distribución de Cisjordania de
Canaán}\label{ordenanzas-provisionales-de-dios-con-respecto-a-la-conquista-y-distribuciuxf3n-de-cisjordania-de-canauxe1n}}

\bibleverse{50} Aquí fue donde, en la llanura de Moab junto al Jordán,
frente a Jericó, el Señor le dijo a Moisés, \bibleverse{51} ``Dile a los
israelitas: Tan pronto crucen el Jordán y entren en el país de Canaán,
\bibleverse{52} deben expulsar a todos los que viven en la tierra,
destruir todas sus imágenes talladas e ídolos de metal, y derribar todos
sus templos paganos.\footnote{\textbf{33:52} ``Templos paganos'':
  literalmente, ``lugares altos''.} \bibleverse{53} Debes tomar el país
y establecerte allí, porque te he dado la tierra y te pertenece.
\bibleverse{54} Debes dividir la tierra y asignarla por sorteo a las
diferentes familias tribales. Dale una porción más grande a una familia
más grande, y una porción más pequeña a una familia más pequeña. La
asignación de cada uno se decide por sorteo, y todos ustedes recibirán
una asignación dependiendo de su tribu.

\bibleverse{55} ``Pero si no expulsan a todos los que viven en la
tierra, las personas que dejen permanecer serán como arena en sus ojos y
espinas en sus costados. Les causarán problemas cuando se establezcan en
el país. \footnote{\textbf{33:55} Jos 23,13}

\bibleverse{56} Eventualmente, el castigo que planeé para ellos se los
infligiré a ustedes''.

\hypertarget{establecer-los-luxedmites-de-la-tierra-de-canauxe1n-que-se-tomaruxe1n}{%
\subsection{Establecer los límites de la tierra de Canaán que se
tomarán}\label{establecer-los-luxedmites-de-la-tierra-de-canauxe1n-que-se-tomaruxe1n}}

\hypertarget{section-33}{%
\section{34}\label{section-33}}

\bibleverse{1} El Señor le dijo a Moisés, \bibleverse{2} ``Dales esta
orden a los israelitas: Cuando entren en el país de Canaán, se les
asignarán las posesiones con los siguientes límites:\footnote{\textbf{34:2}
  Otros pasajes que incluyen demarcaciones de límites son: Josué
  13:8-33; Josué 14:1---19:51; Ezequiel 47:13-20.} \bibleverse{3} ``La
extensión sur de su país será desde el desierto de Zin a lo largo de la
frontera de Edom. Su frontera sur correrá hacia el este desde el final
del Mar Muerto, \footnote{\textbf{34:3} Jos 15,1} \bibleverse{4} cruzará
al sur del Paso del Escorpión, hasta Zin, y alcanzará su límite sur al
sur de Cades-barnea. Luego irá a Hazar-addar y a Azmon. \bibleverse{5}
Allí la frontera girará desde Azmon hasta el Wadi de Egipto,\footnote{\textbf{34:5}
  Normalmente se identifica como Wadi El-Arish. No se cree que se
  refiera al Nilo.} terminando en el Mar Mediterráneo.

\bibleverse{6} ``Su frontera occidental será la costa del Mar
Mediterráneo. Este será su límite al oeste.

\bibleverse{7} ``Tu frontera norte irá desde el Mar Mediterráneo hasta
el Monte Hor. \bibleverse{8} Desde el Monte Hor la frontera irá a
Lebo-hamat, luego a Zedad, \bibleverse{9} a Zifrón, terminando en
Hazar-enan. Este será su límite al norte.

\bibleverse{10} ``Su frontera oriental irá directamente de Hazar-enan a
Sefan. \bibleverse{11} Luego la frontera bajará de Sefam a Ribla en el
lado este de Aín. Pasará a lo largo de las laderas al este del Mar de
Galilea. \bibleverse{12} Luego el límite bajará a lo largo del Jordán,
terminando en el Mar Muerto. Esta será su tierra con sus fronteras
circundantes''.

\bibleverse{13} Moisés dio la orden a los israelitas, ``Asignen la
propiedad de esta tierra por sorteo. El Señor ha ordenado que sea
entregada a las nueve tribus y media. \bibleverse{14} Las tribus de
Rubén y Gad, junto con la media tribu de Manasés, ya han recibido su
asignación. \footnote{\textbf{34:14} Núm 32,33} \bibleverse{15} Estas
dos tribus y media han recibido su asignación en el lado este del
Jordán, frente a Jericó''.

\hypertarget{lista-de-hombres-que-se-encargaruxe1n-de-la-distribuciuxf3n-de-la-tierra}{%
\subsection{Lista de hombres que se encargarán de la distribución de la
tierra}\label{lista-de-hombres-que-se-encargaruxe1n-de-la-distribuciuxf3n-de-la-tierra}}

\bibleverse{16} El Señor le dijo a Moisés, \bibleverse{17} ``Estos son
los nombres de los hombres que se encargarán de asignar la propiedad de
la tierra para ustedes: Eleazar el sacerdote y Josué, hijo de Nun.
\bibleverse{18} Que un líder de cada tribu ayude en la distribución de
la tierra. \bibleverse{19} Estos son sus nombres: ``De la tribu de Judá:
Caleb, hijo de Jefone. \footnote{\textbf{34:19} Núm 13,6; Núm 13,30}

\bibleverse{20} De la tribu de Simeón: Semuel, hijo de Amiud.
\bibleverse{21} De la tribu de Benjamín: Elidad, hijo de Quislón.
\bibleverse{22} Un líder de la tribu de Dan: Buqui, hijo de Jogli.
\bibleverse{23} Un líder de la tribu de Manasés, hijo de José: Haniel,
hijo de Efod. \bibleverse{24} Un líder de la tribu de Efraín: Kemuel,
hijo de Siftán. \bibleverse{25} Un líder de la tribu de Zabulón:
Eli-zafán, hijo de Parnac. \bibleverse{26} Un líder de la tribu de
Isacar: Paltiel, hijo de Azán. \bibleverse{27} Un líder de la tribu de
Aser: Ahiud, hijo de Selomi. \bibleverse{28} Un líder de la tribu de
Neftalí: Pedael, hijo de Amiud''. \bibleverse{29} Estos son los nombres
de los que el Señor puso a cargo de la asignación de la propiedad de la
tierra en el país de Canaán.

\hypertarget{regulaciones-relativas-a-las-ciudades-levitas-y-las-seis-ciudades-libres-designadas-para-asesinos}{%
\subsection{Regulaciones relativas a las ciudades levitas y las seis
ciudades libres designadas para
asesinos}\label{regulaciones-relativas-a-las-ciudades-levitas-y-las-seis-ciudades-libres-designadas-para-asesinos}}

\hypertarget{section-34}{%
\section{35}\label{section-34}}

\bibleverse{1} El Señor le habló a Moisés en las llanuras de Moab junto
al Jordán, frente a Jericó. Le dijo, \bibleverse{2} ``Ordena a los
israelitas que provean de sus ciudades de asignación de tierras para que
los levitas vivan y pasten alrededor de las ciudades. \bibleverse{3} Las
ciudades son para que vivan en ellas, y los pastos serán para sus
rebaños y para todo su ganado.

\bibleverse{4} Los pastos alrededor de las ciudades que le des a los
levitas se extenderán desde el muro mil codos por todos lados.
\bibleverse{5} Mide dos mil codos fuera de la ciudad al Este, dos mil al
Sur, dos mil al Oeste y dos mil al Norte, con la ciudad en el medio.
Estas áreas serán sus pastos alrededor de las ciudades.

\bibleverse{6} ``Seis de los pueblos que le das a los levitas serán
pueblos santuarios,\footnote{\textbf{35:6} Ver también Josué 20.} donde
una persona que mata a alguien puede correr para protegerse. Además de
estas ciudades, dale a los levitas cuarenta y dos más . \footnote{\textbf{35:6}
  Éxod 21,13; Deut 4,41; Deut 19,2; Deut 19,9; Jos 20,-1} \bibleverse{7}
El número total de pueblos que le darás a los levitas es de cuarenta y
ocho, junto con sus pastos. \bibleverse{8} Las ciudades que asignes para
ser entregadas a los levitas serán tomadas del territorio de los
israelitas, y tomarás más de las tribus más grandes y menos las más
pequeñas. El número será proporcional al tamaño de la asignación de
tierras de cada tribu''. \bibleverse{9} El Señor le dijo a Moisés,
\bibleverse{10} ``Dile a los israelitas: 'Cuando cruces el Jordán hacia
Canaán, \bibleverse{11} escoge pueblos como tus pueblos de santuario,
para que una persona que mate a alguien por error pueda correr allí.
\bibleverse{12} Estas ciudades serán para ustedes santuario de los que
buscan venganza, para que el asesino no muera hasta que sea juzgado en
un tribunal. \bibleverse{13} ``Las ciudades que elijan serán sus seis
ciudades santuario. \bibleverse{14} Elijan tres ciudades al otro lado
del Jordán y tres en Canaán como ciudades de refugio. \bibleverse{15}
Estas seis ciudades serán lugares de santuario para los israelitas y
para los extranjeros o colonos entre ellos, de modo que cualquiera que
mate a una persona por error pueda correr allí.

\hypertarget{el-castigo-del-asesino}{%
\subsection{El castigo del asesino}\label{el-castigo-del-asesino}}

\bibleverse{16} ``Pero si alguien golpea deliberadamente a alguien con
algo hecho de hierro y lo mata, esa persona es un asesino y debe ser
ejecutado. \bibleverse{17} Si alguien tomaun trozo de piedra que pueda
ser usado como arma y golpea a alguien con ella, y lo mata, esa persona
es un asesino y debe ser ejecutado. \bibleverse{18} Si alguien tomaun
trozo de madera que pueda ser usado como arma y golpea a alguien con
ella, y lo mata, esa persona es un asesino y debe ser ejecutado.
\bibleverse{19} ``El vengador\footnote{\textbf{35:19} ``El vengador:''
  este era el pariente más cercano a la víctima: literalmente, ``el
  vengador de la sangre''.} debe ejecutar al asesino. Cuando encuentre
al asesino, lo matará. \bibleverse{20} De la misma manera, si uno odia
al otro y lo derriba o le tira algo deliberadamente, y lo mata;
\bibleverse{21} o si alguien golpea a otro con su mano y mueren, el que
lo golpeó debe ser ejecutado porque es un asesino. Cuando el vengador
encuentra al asesino, debe matarlo.

\bibleverse{22} ``Pero si alguien derriba a otro sin querer y sin
odiarlo, o le tira algo sin querer hacerle daño, \bibleverse{23} o deja
caer descuidadamente una piedra pesada que lo mata, pero no como enemigo
o con intención de hacerle daño, \bibleverse{24} entonces la comunidad
debe juzgar entre el asesino y el vengador siguiendo este reglamento.
\bibleverse{25} El tribunal debe proteger al asesino de ser atacado por
el vengador y debe devolverlo a la ciudad santuario a la que corrió, y
debe permanecer allí hasta la muerte del sumo sacerdote, que fue ungido
con el óleo santo. \footnote{\textbf{35:25} Lev 21,10}

\bibleverse{26} ``Pero si el asesino sale de los límites de la ciudad
santuario a la que huyó, \bibleverse{27} y el vengador lo encuentra
fuera de su ciudad santuario y lo mata, entonces el vengador no será
culpable de asesinato, \bibleverse{28} porque el asesino tiene que
permanecer en su ciudad santuario hasta la muerte del sumo sacerdote.
Sólo después de la muerte del sumo sacerdote se les permite volver a la
tierra que poseen.

\bibleverse{29} Estas normas se aplican a todas las generaciones futuras
dondequiera que vivan.

\bibleverse{30} ``Si alguien mata a una persona, el asesino debe ser
ejecutado basándose en las pruebas aportadas por los testigos, en
plural. Nadie debe ser ejecutado basándose en la evidencia dada por un
solo testigo.

\bibleverse{31} ``No se aceptará el pago en lugar de ejecutar a un
asesino que ha sido declarado culpable.

\bibleverse{32} Tampoco se le permite aceptar el pago de una persona que
huye a una ciudad santuario y le permite regresar y vivir en su propia
tierra antes de la muerte del sumo sacerdote.

\bibleverse{33} ``No contaminen la tierra donde viven porque el
derramamiento de sangre contamina la tierra, y la tierra donde se
derrama la sangre no puede ser purificada excepto por la sangre de quien
la derrama. \footnote{\textbf{35:33} Gén 9,6} \bibleverse{34} No hagas
impura la tierra donde vives porque yo también vivo allí. Yo soy el
Señor, y vivo con los israelitas'''.\footnote{\textbf{35:34} Éxod 29,45}

\hypertarget{apuxe9ndice-a-la-ley-de-reliquias}{%
\subsection{Apéndice a la ley de
reliquias}\label{apuxe9ndice-a-la-ley-de-reliquias}}

\hypertarget{section-35}{%
\section{36}\label{section-35}}

\bibleverse{1} Los jefes de familia de los descendientes de Galaad, hijo
de Maquir, hijo de Manasés, una de las tribus de José, vinieron y
hablaron ante Moisés y con los líderes israelitas, que eran otros jefes
de familia. \bibleverse{2} Y les dijeron: ``Cuando el Señor te ordenó,
mi señor, que asignaras la propiedad de la tierra a los israelitas por
sorteo, también te ordenó que dieras la parte de nuestro hermano
Zelofehad a sus hijas. \footnote{\textbf{36:2} Núm 26,55; Núm 27,6-7}
\bibleverse{3} Sin embargo, si se casan con hombres de las otras tribus
de Israel, su asignación les quitaría la parte de nuestros padres y se
le añadiría a la tribu de los hombres con los que se casan. Esa parte de
nuestra asignación sería pérdida para nosotros. \bibleverse{4} Así que
cuando llegue el Jubileo para los israelitas, su asignación se añadirá a
la tribu con la que se casen, y se le quitará a la tribu de nuestros
padres''.

\hypertarget{la-nueva-regulaciuxf3n-de-aplicaciuxf3n-general-sobre-el-matrimonio-de-reliquias}{%
\subsection{La nueva regulación de aplicación general sobre el
matrimonio de
reliquias}\label{la-nueva-regulaciuxf3n-de-aplicaciuxf3n-general-sobre-el-matrimonio-de-reliquias}}

\bibleverse{5} Siguiendo lo que el Señor le dijo, Moisés dio estas
órdenes a los israelitas, ``Lo que dice la tribu de los hijos de José es
correcto. \bibleverse{6} Esto es lo que el Señor ha ordenado con
respecto a las hijas de Zelofehad: Pueden casarse con quien quieran
siempre que lo hagan dentro de una familia que pertenezca a la tribu de
su padre. \bibleverse{7} No se podrá pasar ninguna asignación de tierras
en Israel de tribu a tribu, porque cada israelita debe aferrarse a la
asignación de la tribu de su padre. \bibleverse{8} Toda hija que posea
una herencia de cualquier tribu israelita debe casarse dentro de un clan
de la tribu de su padre, de modo que todo israelita poseerá la herencia
de sus padres. \bibleverse{9} No se podrá pasar ninguna asignación de
tierras de una tribu a otra, pues cada tribu israelita debe mantener su
propia asignación''.

\bibleverse{10} Las hijas de Zelofehad siguieron las órdenes del Señor a
través de Moisés. \bibleverse{11} Maala, Tirsa, Hogla, Milca y Noa,
hijas de Zelofehad, primos casados por parte de su padre. \footnote{\textbf{36:11}
  Núm 26,33} \bibleverse{12} Se casaron dentro de las familias de los
descendientes de Manasés, hijo de José, y su asignación de tierras
permaneció dentro de la tribu de su padre.

\bibleverse{13} Estas son las órdenes y normas que el Señor dio a los
israelitas a través de Moisés en las llanuras de Moab, junto al Jordán,
frente a Jericó.
