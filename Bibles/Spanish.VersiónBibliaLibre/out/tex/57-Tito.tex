\hypertarget{bendiciones}{%
\subsection{Bendiciones}\label{bendiciones}}

\hypertarget{section}{%
\section{1}\label{section}}

\bibleverse{1} Esta carta viene de parte de Pablo, siervo de Dios y
apóstol de Jesucristo. Fui enviado para edificar la fe del pueblo
escogido de Dios y para compartir el conocimiento de la verdad que
conduce a vidas dedicadas a Dios. \bibleverse{2} Esto les da la
esperanza de una vida eterna que Dios (quien no puede mentir) prometió
desde edades atrás, \bibleverse{3} pero que a su debido tiempo reveló
por medio de su palabra, en el mensaje que se me encomendó predicar,
siguiendo el mandato de Dios, nuestro Salvador. \footnote{\textbf{1:3}
  Efes 1,9-10} \bibleverse{4} Esta carta va dirigida a Tito, mi
verdadero hijo por medio de la fe en Dios que tenemos en común. Ten
gracia y paz de Dios el Padre, y de Cristo Jesús, nuestro Salvador.
\footnote{\textbf{1:4} 1Tim 1,2}

\hypertarget{regulaciones-que-rigen-el-nombramiento-de-ancianos-como-luxedderes-de-la-iglesia}{%
\subsection{Regulaciones que rigen el nombramiento de ancianos como
líderes de la
iglesia}\label{regulaciones-que-rigen-el-nombramiento-de-ancianos-como-luxedderes-de-la-iglesia}}

\bibleverse{5} La razón por la cual te dejé en Creta fue para que
organizaras lo que era necesario y para designar ancianos en cada
ciudad, como te dije. \footnote{\textbf{1:5} Hech 14,23} \bibleverse{6}
Un anciano debe tener una buena reputación, ser esposo de una mujer, y
tener hijos creyentes y de los cuales no se diga que son rebeldes y
desobedientes. \footnote{\textbf{1:6} 1Tim 3,1-7} \bibleverse{7} Como
líder de Dios, un anciano líder debe tener una buena reputación y no ser
arrogante. No debe tener un mal carácter ni embriagarse; no debe ser
violento ni tener avaricia por el dinero. \footnote{\textbf{1:7} 1Cor
  4,1; 2Tim 2,24} \bibleverse{8} Debe ser hospitalario, amar el bien y
hacer lo recto. Debe vivir una vida dedicada a Dios, tener dominio
propio, \bibleverse{9} y consagrarse al mensaje fiel, tal como se le
enseño. De esta manera podrá animar a otros por medio de la enseñanza
correcta, y convencer a los que se oponen.

\hypertarget{reglas-de-conducta-contra-seductores-maliciosos-y-falsos-maestros-hipuxf3critas}{%
\subsection{Reglas de conducta contra seductores maliciosos y falsos
maestros
hipócritas}\label{reglas-de-conducta-contra-seductores-maliciosos-y-falsos-maestros-hipuxf3critas}}

\bibleverse{10} Pues hay muchos rebeldes por ahí que predican engaños
sin sentido, especialmente los del grupo de la circuncisión.\footnote{\textbf{1:10}
  Refiriéndose a los creyentes judíos que enseñaban que la circuncisión
  era necesaria para la salvación.} \bibleverse{11} Toda su habladuría
debe parar. Pues ellos causan inestabilidad en las familias, enseñando
cosas que no son correctas, por interés de ganar dinero. \bibleverse{12}
Tal como ha dicho uno de su propio pueblo,\footnote{\textbf{1:12} No
  necesariamente se refiere a alguien de la facción que estaba a favor
  de la circuncisión o de cualquier otro grupo disidente, sino alguien
  proveniente de Creta.} un profeta: ``Todos los cretenses son
mentirosos, bestias del mal, perezosas y avaras''. \bibleverse{13} ¡Esto
es muy cierto! Por ello, repréndelos con severidad para que puedan
llegar a tener una fe sana en Dios, \bibleverse{14} dejando de atender
los mandamientos humanos y mitos judíos de aquellos que se desvían de la
verdad. \footnote{\textbf{1:14} 1Tim 4,7; 2Tim 4,4} \bibleverse{15} A
los que tienen mentes puras, todo les parece puro; pero para los que son
corruptos y se niegan a creer en Dios, nada es puro. Porque tanto sus
mentes como sus conciencias están corrompidas. \footnote{\textbf{1:15}
  Mat 15,11; Rom 14,20} \bibleverse{16} Ellos dicen conocer a Dios, pero
con sus actos demuestran que es mentira. Son aborrecibles y
desobedientes, y no sirven para hacer nada bueno.\footnote{\textbf{1:16}
  2Tim 3,5}

\hypertarget{regulaciones-para-las-fincas-individuales-en-la-comunidad}{%
\subsection{Regulaciones para las fincas individuales en la
comunidad}\label{regulaciones-para-las-fincas-individuales-en-la-comunidad}}

\hypertarget{section-1}{%
\section{2}\label{section-1}}

\bibleverse{1} Sin embargo, tú enseña lo que está acorde a las creencias
sanas. \footnote{\textbf{2:1} 2Tim 1,13} \bibleverse{2} Los hombres de
mayor edad deben ser respetables y sensatos,\footnote{\textbf{2:2}
  ``Sensato'', o ``considerado'', ``con dominio propio'', ``decente''.
  También en 2:5, 2:6 y 2:12.} con una fe sana en Dios, amorosos y
pacientes. \footnote{\textbf{2:2} 1Tim 5,1} \bibleverse{3} Del mismo
modo, las mujeres de mayor edad deben comportarse de una manera que
demuestre que tienen vidas dedicadas a Dios. No deben destruir la
reputación de la gente con su hablar, y no deben ser adictas al vino.
\footnote{\textbf{2:3} 1Tim 3,11} \bibleverse{4} Deben ser maestras de
lo bueno, y enseñar a las esposas más jóvenes a amar a sus esposos y a
sus hijos. \bibleverse{5} Deben ser sensatas y puras, hacendosas,
hacedoras del bien y tener oídos prestos a lo que sus esposos les dicen.
De este modo, no habrá nada malo que decir de la palabra de Dios.
\footnote{\textbf{2:5} Efes 5,22}

\bibleverse{6} Del mismo modo, enseña a los hombres jóvenes a ser
sensatos. \bibleverse{7} Tú debes ser ejemplo de cómo hacer el bien en
todas las áreas de la vida: muestra integridad y seriedad en lo que
enseñas, \bibleverse{8} compartiendo creencias sanas que no puedan ser
cuestionadas. Así, los que se oponen, se avergonzarán de sí mismos y no
tendrán nada malo que decir acerca de nosotros.

\bibleverse{9} Enseña a los siervos a que siempre obedezcan a sus amos.
Enséñales que siempre deben procurar agradarles y no hablar mal a sus
espaldas. \footnote{\textbf{2:9} Efes 6,5-6; 1Tim 6,1-2; 1Pe 2,18}
\bibleverse{10} Diles que no deben robar cosas para sí, sino demostrar
que son completamente fieles y que pueden representar correctamente la
verdad acerca de Dios, nuestro Salvador, en todas las formas.

\hypertarget{justificaciuxf3n-de-estos-reglamentos-haciendo-referencia-a-la-gracia-de-dios-que-apareciuxf3-en-el-mundo}{%
\subsection{Justificación de estos reglamentos haciendo referencia a la
gracia de Dios que apareció en el
mundo}\label{justificaciuxf3n-de-estos-reglamentos-haciendo-referencia-a-la-gracia-de-dios-que-apareciuxf3-en-el-mundo}}

\bibleverse{11} Pues la gracia de Dios ha sido revelada, otorgando
salvación a todos. \bibleverse{12} Nos enseña a rechazar el estilo de
vida impío junto a los deseos de este mundo. Por el contrario, debemos
vivir con sensatez, vidas de dominio propio que sean rectas ante Dios,
en presencia del mundo \bibleverse{13} mientras aguardamos la
maravillosa esperanza de la aparición gloriosa de nuestro gran Dios y
Salvador Jesucristo. \footnote{\textbf{2:13} 1Cor 1,7; Fil 3,20; 1Tes
  1,10} \bibleverse{14} Pues él se entregó a sí mismo por nosotros, para
podernos libertar de toda nuestra maldad, y para limpiarnos para él,
como un pueblo que le pertenece, y que está dispuesto a hacer el bien.
\footnote{\textbf{2:14} Gal 1,4; 1Tim 2,6; Éxod 19,5; Efes 2,10}

\bibleverse{15} Tales cosas debes enseñar. Pues tienes autoridad para
animar y corregir en cuanto sea necesario. No permitas que nadie te
menosprecie.\footnote{\textbf{2:15} 1Tim 4,12}

\hypertarget{sobre-el-comportamiento-contra-las-autoridades-paganas-y-los-no-cristianos-y-sobre-el-camino-de-los-cristianos-como-pueblo-renovado}{%
\subsection{Sobre el comportamiento contra las autoridades paganas y los
no cristianos y sobre el camino de los cristianos como pueblo
renovado}\label{sobre-el-comportamiento-contra-las-autoridades-paganas-y-los-no-cristianos-y-sobre-el-camino-de-los-cristianos-como-pueblo-renovado}}

\hypertarget{section-2}{%
\section{3}\label{section-2}}

\bibleverse{1} Recuérdales que deben seguir lo que los gobernantes les
dicen, y que deben obedecer a las autoridades. Siempre deben estar
listos para hacer el bien. \footnote{\textbf{3:1} Rom 13,1; 1Pe 2,13}
\bibleverse{2} Diles que no deben hablar mal de nadie, y que no deben
estar en contiendas. Enséñales a mostrar bondad con todas las personas.
\footnote{\textbf{3:2} Fil 4,5} \bibleverse{3} Pues hubo un tiempo en
que nosotros también fuimos necios y desobedientes. Éramos engañados y
andábamos como esclavos de diversos deseos y placeres. Vivíamos vidas de
maldad, llenas de celos. Estábamos llenos de odio los unos por los
otros. \footnote{\textbf{3:3} 1Cor 6,11; Efes 2,2; Efes 5,8; 1Pe 4,3}
\bibleverse{4} Pero cuando la bondad y el amor de Dios nuestro Salvador
fueron revelados, nos salvó, \footnote{\textbf{3:4} Tit 2,11}
\bibleverse{5} no porque hubiésemos hecho algo bueno, sino por su
misericordia. Lo hizo por medio de la limpieza del nuevo nacimiento y
renovación del Espíritu Santo, \footnote{\textbf{3:5} 2Tim 1,9; Juan
  3,5; Efes 5,26} \bibleverse{6} el cual derramó sobre nosotros
abundantemente por medio de Jesucristo nuestro Salvador. \footnote{\textbf{3:6}
  Jl 3,1} \bibleverse{7} Ahora que estamos justificados por su gracia,
nos hemos convertido en herederos por la esperanza de la vida eterna.
\footnote{\textbf{3:7} Rom 3,26}

\hypertarget{conclusiuxf3n-sobre-el-comportamiento-frente-a-las-aberraciones-doctrinales-y-sus-representantes}{%
\subsection{Conclusión sobre el comportamiento frente a las aberraciones
doctrinales y sus
representantes}\label{conclusiuxf3n-sobre-el-comportamiento-frente-a-las-aberraciones-doctrinales-y-sus-representantes}}

\bibleverse{8} Puedes confiar en lo que te digo, y quiero que hagas
énfasis en estas instrucciones para que los que creen en Dios tomen su
vida con seriedad y sigan haciendo el bien. Ellos son excelentes
personas y siempre están prestos a ayudar a todos. \bibleverse{9} Evita
las discusiones insensatas sobre linajes. No entres en contiendas y
evita las discusiones sobre las leyes judías, pues tales discusiones son
vanas y no sirven para nada. \footnote{\textbf{3:9} 1Tim 1,4; 1Tim 4,7;
  2Tim 2,14} \bibleverse{10} A aquella persona que cause división,
adviértele una vez, y después no le prestes atención, \footnote{\textbf{3:10}
  Mat 18,15-17; 2Jn 1,10} \bibleverse{11} entendiendo que es una persona
perversa y pecadora que ya ha traído su propia condenación. \footnote{\textbf{3:11}
  1Tim 6,4-5}

\hypertarget{comentarios-personales-finales-uxfaltimos-pedidos-y-saludos}{%
\subsection{Comentarios personales finales, últimos pedidos y
saludos}\label{comentarios-personales-finales-uxfaltimos-pedidos-y-saludos}}

\bibleverse{12} Tan pronto envíe a Artemas o a Tíquico donde ti, procura
venir a visitarme a Nicópolis, pues tengo planes de pasar el invierno
allí. \footnote{\textbf{3:12} Efes 6,21} \bibleverse{13} Haz todo lo que
puedas por ayudar a Zenas, el abogado, y a Apolo cuando vayan de camino
para que puedan tener lo que necesitan. \footnote{\textbf{3:13} Hech
  18,24; 1Cor 3,5-6} \bibleverse{14} Ojalá nuestro pueblo aprenda el
hábito de hacer el bien, proveyendo para las necesidades diarias de los
demás. ¡Necesitan ser productivos! \footnote{\textbf{3:14} Tit 2,14; Mat
  7,19}

\bibleverse{15} Todos los que están aquí conmigo envían sus saludos.
Envía mis saludos a quienes nos aman, los que tienen fe en Dios. Que la
gracia esté con todos ustedes.
