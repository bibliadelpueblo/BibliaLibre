\hypertarget{bendiciones}{%
\subsection{Bendiciones}\label{bendiciones}}

\hypertarget{section}{%
\section{1}\label{section}}

\bibleverse{1} Esta carta es enviada por Pablo, prisionero de
Jesucristo, y de nuestro hermano Timoteo, a Filemón, nuestro buen amigo
y compañero de trabajo; \footnote{\textbf{1:1} Efes 3,1} \bibleverse{2}
a nuestra hermana Apia, a Arquipo, quien lucha junto con nosotros, y a
nuestra iglesia que está en tu casa. \footnote{\textbf{1:2} Col 4,17}
\bibleverse{3} Recibe gracia y paz de parte de Dios nuestro Padre y del
Señor Jesucristo.

\hypertarget{gracias-a-dios-e-intercesiuxf3n-por-filemuxf3n}{%
\subsection{Gracias a Dios e intercesión por
Filemón}\label{gracias-a-dios-e-intercesiuxf3n-por-filemuxf3n}}

\bibleverse{4} Siempre le doy gracias a Dios por ti, al recordarte en
mis oraciones, \bibleverse{5} pues escucho sobre tu fe en el Señor Jesús
y tu amor por todos los creyentes. \bibleverse{6} Oro para que esa
generosidad que caracteriza tu fe en Dios puedas ponerla en acción al
reconocer las cosas buenas de las que participamos en Cristo.
\footnote{\textbf{1:6} Fil 1,9} \bibleverse{7} Tu amor, mi querido
hermano, me ha causado mucha felicidad y ánimo. ¡Has reanimado los
corazones de nosotros, los que somos creyentes! \footnote{\textbf{1:7}
  2Cor 7,4}

\hypertarget{defensa-de-onuxe9simo}{%
\subsection{Defensa de Onésimo}\label{defensa-de-onuxe9simo}}

\bibleverse{8} Por eso, aunque soy suficientemente valiente en Cristo
para darte orden de hacer tu trabajo, \bibleverse{9} prefiero pedirte
este favor en nombre del amor. El viejo Pablo, ahora también prisionero
de Cristo Jesús, \bibleverse{10} te ruega en nombre de Onésimo, que ha
venido a ser como mi hijo adoptivo durante mi encarcelamiento.
\footnote{\textbf{1:10} Gal 4,19; 1Cor 4,15} \bibleverse{11} En el
pasado él no fue útil para ti, ¡pero ahora es útil tanto para ti como
para mí! \bibleverse{12} Lo envío, pues, con mis más sinceros
deseos.\footnote{\textbf{1:12} Literalmente ``con aprecio de corazón''.}
\bibleverse{13} Habría preferido que se quedara aquí conmigo para que me
fuera de ayuda como me habrías ayudado tú mientras estoy encadenado por
predicar la buena noticia. \bibleverse{14} Pero decidí no hacer nada sin
tu permiso. No quería obligarte a hacer el bien, sino que lo hicieras de
buen agrado. \footnote{\textbf{1:14} 2Cor 9,7} \bibleverse{15} ¡Quizás
lo perdiste por un tiempo para ahora tenerlo para siempre!
\bibleverse{16} Ya no es más un siervo, porque es más que un siervo. Es
un hermano especialmente amado, principalmente para mí, e incluso más
para ustedes, tanto como persona y también como hermano creyente en el
Señor.\footnote{\textbf{1:16} Literalmente, ``en la carne y en el
  Señor''.} \bibleverse{17} Así que si me consideras un compañero de
trabajo en el Señor,\footnote{\textbf{1:17} ``Un colega que trabaja
  contigo por el Señor''. La palabra griega es ``socio'', pero requiere
  explicación debido a los usos modernos de esta palabra.} recíbelo como
si me recibieras a mí. \bibleverse{18} Y si ha cometido algún error, o
te debe algo, cárgalo a mi cuenta. \bibleverse{19} Yo, Pablo, escribo
esto con mi propia mano: Te pagaré. Sin duda no diré lo que me debes,
¡incluyendo tu propia vida! \bibleverse{20} Sí, hermano, espero este
favor de tu parte en el Señor; por favor, dame esa alegría en Cristo.

\hypertarget{cierre-de-cartas-anuncio-de-visita-saludos-y-bendiciones}{%
\subsection{Cierre de cartas, anuncio de visita, saludos y
bendiciones}\label{cierre-de-cartas-anuncio-de-visita-saludos-y-bendiciones}}

\bibleverse{21} Te escribo sobre esto porque estoy convencido de que
harás lo que te estoy pidiendo. ¡E incluso sé que harás más que eso!
\bibleverse{22} Mientras tanto, por favor, prepara una habitación para
mí, pues espero poder regresar a verte pronto, como respuesta a tus
oraciones. \bibleverse{23} Epafras, que está aquí conmigo en prisión, te
envía su saludo, \bibleverse{24} así como mis colaboradores Marcos,
Aristarco, Demas, y Lucas. \bibleverse{25} Que la gracia de nuestro
Señor Jesucristo esté con todos ustedes.
