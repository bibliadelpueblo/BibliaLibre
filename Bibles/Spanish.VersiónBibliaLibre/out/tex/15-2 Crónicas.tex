\hypertarget{acceso-de-salomuxf3n-al-gobierno-su-ejuxe9rcito-y-su-riqueza}{%
\subsection{Acceso de Salomón al gobierno; su ejército y su
riqueza}\label{acceso-de-salomuxf3n-al-gobierno-su-ejuxe9rcito-y-su-riqueza}}

\hypertarget{section}{%
\section{1}\label{section}}

\bibleverse{1} Salomón, hijo de David, fortaleció su dominio sobre el
reino, y el Señor Dios estuvo con él y lo hizo sumamente poderoso.

\bibleverse{2} Salomón mandó llamar a todos los jefes israelitas, a los
comandantes de millares y de centenas, a los jueces y a todos los jefes
de familia. \bibleverse{3} Salomón se dirigió con toda la asamblea al
lugar alto de Gabaón, pues allí se encontraba la Tienda del Encuentro de
Dios que Moisés, el siervo del Señor, había hecho en el desierto.
\footnote{\textbf{1:3} 1Cró 16,39; 1Cró 21,29} \bibleverse{4} David
había subido el Arca de Dios desde Quiriat-jearim hasta el lugar de
Jerusalén, donde había levantado una tienda para ella. \footnote{\textbf{1:4}
  1Cró 13,6; 1Cró 15,3; 1Cró 15,28; 1Cró 16,1} \bibleverse{5} Sin
embargo, el altar de bronce hecho por Bezalel, hijo de Uri, hijo de Hur,
estaba allí\footnote{\textbf{1:5} En Gibeón.} frente a la Tienda del
Señor, por lo que allí fue Salomón y la asamblea a adorar. \footnote{\textbf{1:5}
  Éxod 38,1-8; 2Cró 1,3} \bibleverse{6} Salomón subió al altar de bronce
ante el Señor, frente a la Tienda del Encuentro. Allí presentó mil
holocaustos.

\hypertarget{la-apariciuxf3n-de-dios-o-sueuxf1o-despuuxe9s-del-sacrificio}{%
\subsection{La aparición de Dios (o sueño) después del
sacrificio}\label{la-apariciuxf3n-de-dios-o-sueuxf1o-despuuxe9s-del-sacrificio}}

\bibleverse{7} Esa noche Dios se le apareció a Salomón y le dijo: ``Pide
lo que quieras que te dé''.

\bibleverse{8} Salomón respondió a Dios: ``Tú mostraste un amor
confiable y sin límites a mi padre David, y me has hecho rey en su
lugar. \bibleverse{9} Señor Dios, por favor, cumple la promesa que le
hiciste a mi padre David. Me has hecho rey de una nación que tiene tanta
gente como el polvo de la tierra. \bibleverse{10} Por favor, dame
sabiduría y conocimiento para dirigir a este pueblo, pues ¿quién puede
gobernar con justicia\footnote{\textbf{1:10} ``Gobernar con justicia'':
  la palabra aquí significa realmente ``juzgar''.} este gran pueblo
tuyo''.

\bibleverse{11} Dios le dijo a Salomón: ``Porque esto es lo que
realmente querías, y no pediste riquezas, posesiones u honores, ni la
muerte de los que te odian, ni una larga vida, sino que pediste
sabiduría y conocimiento para poder gobernar con justicia a mi pueblo
del que te he hecho rey; \bibleverse{12} la sabiduría y el conocimiento
te son dados. También te daré riquezas, posesiones y honor, mucho más de
lo que ha tenido cualquier rey que te haya precedido, o que venga
después de ti''.

\bibleverse{13} Entonces Salomón regresó a Jerusalén desde la Tienda del
Encuentro en Gabaón, y gobernó sobre Israel.

\hypertarget{la-riqueza-y-el-comercio-de-salomuxf3n-en-carros-y-caballos}{%
\subsection{La riqueza y el comercio de Salomón en carros y
caballos}\label{la-riqueza-y-el-comercio-de-salomuxf3n-en-carros-y-caballos}}

\bibleverse{14} Salomón construyó un ejército de carros y caballos.
Tenía 1. 400 carros y 12. 000 caballos, que colocó en las ciudades de
los carros, y también con él en Jerusalén. \footnote{\textbf{1:14} 1Re
  10,26-29} \bibleverse{15} El rey hizo que en Jerusalén abundaran la
plata y el oro como las piedras, y la madera de cedro como los sicómoros
en las estribaciones. \footnote{\textbf{1:15} 2Cró 9,27}

\bibleverse{16} Salomón importó para sí caballos de Egipto y de Koa; los
comerciantes del rey los compraban en Koa. \bibleverse{17} Se podía
importar un carro de Egipto por seiscientos siclos de plata, y un
caballo por ciento cincuenta. De la misma manera los exportaban a todos
los reyes hititas y a los reyes arameos.

\hypertarget{el-tratado-de-salomuxf3n-con-hiram-de-tiro-preparativos-para-la-construcciuxf3n-del-templo}{%
\subsection{El tratado de Salomón con Hiram de Tiro; Preparativos para
la construcción del
templo}\label{el-tratado-de-salomuxf3n-con-hiram-de-tiro-preparativos-para-la-construcciuxf3n-del-templo}}

\hypertarget{section-1}{%
\section{2}\label{section-1}}

\bibleverse{1} Salomón ordenó la construcción de un Templo\footnote{\textbf{2:1}
  Las palabras ``Templo'' y ``palacio'' traducen la palabra habitual de
  ``casa''.} para honrar al Señor y un palacio real para él. \footnote{\textbf{2:1}
  1Cró 14,1} \bibleverse{2} Asignó 70. 000 hombres como obreros, 80. 000
como cortadores de piedra en las montañas y 3. 600 como capataces.

\hypertarget{mensaje-de-salomuxf3n-y-peticiuxf3n-a-hiram}{%
\subsection{Mensaje de Salomón y petición a
Hiram}\label{mensaje-de-salomuxf3n-y-peticiuxf3n-a-hiram}}

\bibleverse{3} Salomón envió un mensaje a Hiram,\footnote{\textbf{2:3}
  ``Hiram'', aquí aparece escrito como ``Huram'', también 2:11. (Véase 1
  Reyes 5).} rey de Tiro, diciéndole: \bibleverse{4} ``Por favor, haz
como hiciste con mi padre David cuando le enviaste madera de cedro para
que construyera un palacio donde vivir. Estoy a punto de empezar a
construir un Templo en honor del Señor, mi Dios, dedicado a él, donde se
le ofrecerá incienso aromático, donde los panes de la proposición
estarán siempre dispuestos en hileras, y donde se harán holocaustos
todas las mañanas y las tardes, en los sábados, en las fiestas de luna
nueva y en las fiestas del Señor, nuestro Dios; esto se hará para
siempre en Israel.

\bibleverse{5} Este Templo que voy a construir debe ser impresionante,
porque nuestro Dios es más grande que todos los dioses. \footnote{\textbf{2:5}
  2Cró 6,18; 1Re 8,27} \bibleverse{6} Pero ¿quién puede construirle un
Templo para que viva en él, pues los cielos, incluso los más altos, no
pueden contenerlo, y quién soy yo para atreverme a construirle una casa,
salvo para quemarle incienso?

\bibleverse{7} ``Así que, por favor, envíame un maestro artesano que
sepa trabajar el oro, la plata, el bronce y el hierro; y las telas de
color púrpura, escarlata y azul. También debe saber grabar, trabajando
junto con mis expertos artesanos de Judea y Jerusalén proporcionados por
mi padre David.

\bibleverse{8} Envíame también madera de cedro, de ciprés y de algum del
Líbano, porque sé que tus obreros son hábiles para cortar los árboles
del Líbano. Enviaré hombres para que ayuden a tus trabajadores
\bibleverse{9} a producir una gran cantidad de madera, porque el Templo
que estoy construyendo será realmente grande y muy impresionante.
\bibleverse{10} Pagaré a tus trabajadores, los cortadores de madera, 20.
000 cors de trigo triturado, 20. 000 cors de cebada, 20. 000 baños de
vino y 20. 000 baños de aceite de oliva''.

\hypertarget{respuesta-y-aceptaciuxf3n-de-hiram}{%
\subsection{Respuesta y aceptación de
Hiram}\label{respuesta-y-aceptaciuxf3n-de-hiram}}

\bibleverse{11} El rey Hiram de Tiro respondió a Salomón por carta: ``Es
porque el Señor ama a su pueblo que te ha hecho su rey''.
\bibleverse{12} Hiram continuó: ``¡Alabado sea el Señor, el Dios de
Israel, que hizo los cielos y la tierra! Él ha dado al rey David un hijo
sabio, con perspicacia y entendimiento, que va a construir un Templo
para el Señor y un palacio real para él.

\bibleverse{13} ``Te envío a Hiram-Abi, un maestro artesano que sabe y
comprende lo que hace. \bibleverse{14} Su madre es de la tribu de Dan y
su padre es de Tiro. Es un experto en trabajar el oro y la plata, el
bronce y el hierro, la piedra y la madera, la tela púrpura, azul y
carmesí, y el lino fino. Sabe hacer todo tipo de grabados y puede
realizar cualquier diseño que se le encargue. Trabajará con tus
artesanos y con los artesanos de mi señor, tu padre David.

\bibleverse{15} ``Ahora, mi señor, por favor, envíanos a sus siervos el
trigo, la cebada, el aceite de oliva y el vino de que habló.
\bibleverse{16} Nosotros cortaremos del Líbano toda la madera que
necesites y te la llevaremos por mar en balsas hasta Jope. Desde allí
podrás transportarla a Jerusalén''.

\hypertarget{salomuxf3n-eleva-a-los-no-israelitas-al-trabajo-esclavo}{%
\subsection{Salomón eleva a los no israelitas al trabajo
esclavo}\label{salomuxf3n-eleva-a-los-no-israelitas-al-trabajo-esclavo}}

\bibleverse{17} Salomón mandó hacer un censo de todos los extranjeros en
la tierra de Israel, como el censo que había hecho su padre David, y
encontró que había 153. 600. \footnote{\textbf{2:17} Jos 9,27}

\bibleverse{18} Asignó 70. 000 como obreros, 80. 000 como canteros en
las montañas y 3. 600 como capataces.

\hypertarget{inicio-de-la-construcciuxf3n-del-templo-los-muebles-del-templo}{%
\subsection{Inicio de la construcción del templo; los muebles del
templo}\label{inicio-de-la-construcciuxf3n-del-templo-los-muebles-del-templo}}

\hypertarget{section-2}{%
\section{3}\label{section-2}}

\bibleverse{1} Entonces Salomón comenzó a construir el Templo del Señor
en Jerusalén, en el monte Moriah, donde el Señor se apareció a su padre
David. Este era el lugar que David había dispuesto: la antigua era de
Ornán el jebuseo. \bibleverse{2} Salomón comenzó la construcción el
segundo día del segundo mes de su cuarto año como rey.

\hypertarget{dimensiones-y-decoraciones-de-la-casa-del-templo}{%
\subsection{Dimensiones y decoraciones de la casa del
templo}\label{dimensiones-y-decoraciones-de-la-casa-del-templo}}

\bibleverse{3} El tamaño de los cimientos que Salomón puso para el
Templo de Dios era de sesenta codos de largo y veinte de ancho, (según
la antigua medida de codos). \bibleverse{4} El pórtico que corría a lo
ancho del Templo tenía veinte codos de largo y veinte\footnote{\textbf{3:4}
  En hebreo se lee ``ciento veinte'', pero seguramente se trata de un
  error de los escribas, ya que la altura del Templo principal, según 1
  Reyes 6:2, era de 30 codos.} codos de altura. Cubrió el interior del
pórtico con oro puro. \bibleverse{5} Recubrió la sala principal con
ciprés recubierto de oro fino, con imágenes de palmeras y cadenas.
\bibleverse{6} Decoró el Templo con hermosas gemas y con oro que importó
de Parvaim. \bibleverse{7} Cubrió de oro las vigas, los umbrales, las
paredes y las puertas del Templo, y esculpió querubines en las paredes.

\hypertarget{equipo-del-lugar-santuxedsimo}{%
\subsection{Equipo del lugar
santísimo}\label{equipo-del-lugar-santuxedsimo}}

\bibleverse{8} Hizo que la sala del Lugar Santísimo se correspondiera
con la anchura del Templo: veinte codos de largo y veinte de ancho.
Cubrió el interior con seiscientos talentos de oro fino. \bibleverse{9}
El peso de los clavos era de un siclo por cada cincuenta siclos de
oro.\footnote{\textbf{3:9} ``Un siclo por cada cincuenta siclos de
  oro'': Tomado de la Septuaginta.}

\bibleverse{10} Hizo para el Lugar Santísimo dos querubines de madera
cubiertos de oro. \bibleverse{11} La envergadura de los querubines
juntos era de veinte codos. Un ala del primer querubín medía cinco codos
y tocaba una de las paredes del Templo, mientras que su otra ala,
también de cinco codos, tocaba el segundo querubín. \bibleverse{12} Del
mismo modo, una de las alas del segundo querubín medía cinco codos y
tocaba una de las paredes del Templo, mientras que su otra ala, que
también medía cinco codos, tocaba al primer querubín. \bibleverse{13}
Así que la envergadura de estos querubines juntos era de veinte codos.
Estaban de pie, de cara a la sala principal. \bibleverse{14} Hizo el
velo\footnote{\textbf{3:14} ``Velo'': cortina que separa el Lugar
  Santísimo de la sala principal.} de bordado azul, púrpura y carmesí
sobre lino fino, con imágenes de querubines. \footnote{\textbf{3:14}
  Éxod 26,31}

\hypertarget{los-dos-pilares-de-bronce-frente-a-la-casa-del-templo}{%
\subsection{Los dos pilares de bronce frente a la casa del
templo}\label{los-dos-pilares-de-bronce-frente-a-la-casa-del-templo}}

\bibleverse{15} Hizo dos columnas para la fachada del Templo, de treinta
y cinco codos, cada una con un capitel de cinco codos de altura.
\bibleverse{16} Hizo cadenas como en el Lugar Santísimo y las colocó
encima de las columnas. También hizo cien granadas ornamentales y las
fijó a cada cadena.\footnote{\textbf{3:16} Parece que había cuatro
  cadenas, cada una de las cuales sostenía cien granadas ornamentales
  (véase 4:13, 1 Reyes 7:42).} \bibleverse{17} Colocó las columnas
frente al Templo, una al sur y otra al norte. A la columna del sur le
puso el nombre de Jaquín, y a la del norte el de Booz.

\hypertarget{fabricaciuxf3n-de-implementos-para-el-templo}{%
\subsection{Fabricación de implementos para el
templo}\label{fabricaciuxf3n-de-implementos-para-el-templo}}

\hypertarget{section-3}{%
\section{4}\label{section-3}}

\bibleverse{1} Salomón hizo un altar de bronce de veinte codos de largo,
veinte de ancho y diez de alto. \bibleverse{2} Hizo un ``Mar'' de metal
fundido,\footnote{\textbf{4:2} Se trata de una gran pila llena de agua.
  El metal utilizado era probablemente el bronce, pero no se identifica
  específicamente como tal en el texto.} diez codos de diámetro, cinco
de altura y treinta de circunferencia. \bibleverse{3} Debajo de ella
había toros ornamentales\footnote{\textbf{4:3} Véase 1 Reyes 7:24 que
  dice ``calabazas''} a su alrededor, diez por codo. Estaban en dos
filas cuando todo estaba fundido. \bibleverse{4} El Mar estaba sostenido
por doce estatuas de toros, tres orientadas al norte, tres al oeste,
tres al sur y tres al este. El Mar estaba colocado sobre ellos, con sus
espaldas hacia el centro. \bibleverse{5} Era tan grueso como el ancho de
una mano, y su borde era como el borde acampanado de una copa o de una
flor de lis. Contenía tres mil baños.\footnote{\textbf{4:5} Véase 1
  Reyes 7:26, donde se indica que el aforo era de dos mil baños.}
\bibleverse{6} También hizo diez pilas sobre carros para lavar. Colocó
cinco en el lado sur y cinco en el norte. Se usaban para limpiar lo que
se usaba en los holocaustos, pero el Mar lo usaban los sacerdotes para
lavarse.

\bibleverse{7} Hizo diez candelabros de oro como se había
especificado,\footnote{\textbf{4:7} Véase 1 Crónicas 28:15.} y las
colocó en el Templo, cinco en el lado sur y cinco en el norte.
\bibleverse{8} Además, hizo diez mesas y las colocó en el Templo, cinco
en el lado sur y cinco en el norte. También hizo cien pilas de oro.
\bibleverse{9} Salomón también construyó el patio de los sacerdotes, el
patio grande y las puertas del patio, y cubrió las puertas con bronce.
\bibleverse{10} Colocó el Mar en el lado sur, junto a la esquina
sureste.

\bibleverse{11} Hiram también hizo las ollas, las palas y las pilas.
Hiram terminó el trabajo que había estado haciendo para el rey Salomón
en el Templo de Dios:

\bibleverse{12} las dos columnas; los dos capiteles en forma de cuenco
en la parte superior de las columnas; los dos juegos de redes\footnote{\textbf{4:12}
  Probablemente una red de cadenas, ya mencionada.} que cubrían las dos
cazoletas de los capiteles de la parte superior de las columnas;
\bibleverse{13} las cuatrocientas granadas ornamentales para los dos
conjuntos de red -dos filas de granadas para cada red que cubrían las
dos cazoletas de los capiteles de la parte superior de las columnas-;
\bibleverse{14} los carros de agua y las pilas de los carros de agua;
\bibleverse{15} el Mar y las doce estatuas de toros que lo sostenían;
las ollas, las palas, los tenedores y todo lo demás. \bibleverse{16}
Todo el trabajo de metal que Hiram hizo para el rey Salomón para la casa
del Señor era de bronce pulido. \bibleverse{17} El rey los fundió en
moldes de arcilla en la llanura del Jordán, entre Sucot y Zereda.
\bibleverse{18} Salomón hizo tantas de estas cosas que no se podía medir
el peso del bronce utilizado.

\bibleverse{19} Salomón hizo también todo lo que se utilizaba en el
Templo de Dios: el altar de oro; las mesas donde se exponía el Pan de la
Presencia; \bibleverse{20} los candelabros de oro puro y sus lámparas
que debían arder delante del Lugar Santísimo, tal como se había
especificado; \bibleverse{21} las flores decorativas, las lámparas y las
pinzas, todo de oro macizo; \bibleverse{22} los adornos para las mechas,
las jofainas, los platos y los incensarios, todo de oro; y las puertas
del Templo: las puertas interiores del Lugar Santísimo y las puertas de
la sala principal, todas cubiertas de oro.

\hypertarget{los-objetos-de-valor-almacenados-en-las-cuxe1maras-del-tesoro.}{%
\subsection{Los objetos de valor almacenados en las cámaras del
tesoro.}\label{los-objetos-de-valor-almacenados-en-las-cuxe1maras-del-tesoro.}}

\hypertarget{section-4}{%
\section{5}\label{section-4}}

\bibleverse{1} Una vez que Salomón hubo terminado todas las obras de la
casa del Señor, trajo los objetos sagrados que su padre David había
dedicado -la plata, el oro y todos los diversos objetos de culto- y los
colocó en los tesoros del Templo de Dios. \footnote{\textbf{5:1} 1Cró
  28,14-18}

\hypertarget{la-transferencia-del-arca-al-lugar-santuxedsimo}{%
\subsection{La transferencia del arca al lugar
santísimo}\label{la-transferencia-del-arca-al-lugar-santuxedsimo}}

\bibleverse{2} Luego Salomón convocó a Jerusalén a los ancianos de
Israel -todos los jefes de las tribus y los jefes de familia de los
israelitas- para que trajeran el Arca del Pacto del Señor desde Sión, la
Ciudad de David. \bibleverse{3} Así que todos los israelitas se
reunieron para estar con el rey en la fiesta que se celebra en el
séptimo mes.\footnote{\textbf{5:3} La fiesta de los tabernáculos.}
\bibleverse{4} Cuando llegaron todos los ancianos de Israel, los levitas
levantaron el Arca. \bibleverse{5} Los sacerdotes y los levitas subieron
el Arca, la Tienda de la Reunión, estaban con él delante del Arca.
\bibleverse{6} ¡Sacrificaron tantas ovejas y reses que no se podían
contar! \bibleverse{7} Entonces los sacerdotes trajeron el Arca del
Pacto del Señor y la colocaron en el santuario interior del Templo, el
Lugar Santísimo, bajo las alas de los querubines. \bibleverse{8} Los
querubines extendían sus alas sobre el lugar donde estaba el Arca, de
modo que los querubines formaban una cubierta sobre el Arca y sus varas.
\bibleverse{9} Los postes eran tan largos que sus extremos podían verse
desde el Lugar Santo, frente al Lugar Santísimo, pero no desde afuera.
Allí están hasta el día de hoy. \bibleverse{10} Dentro del Arca no había
nada más que las dos tablas de piedra que Moisés había colocado en el
Monte Sinaí,\footnote{\textbf{5:10} ``Monte Sinaí'': Literalmente,
  ``Horeb'', un nombre alternativo.} donde el Señor había hecho un
acuerdo con el pueblo de Israel al salir de Egipto. \footnote{\textbf{5:10}
  Heb 9,4}

\hypertarget{la-apariciuxf3n-de-la-gloria-de-dios}{%
\subsection{La aparición de la gloria de
Dios}\label{la-apariciuxf3n-de-la-gloria-de-dios}}

\bibleverse{11} Entonces los sacerdotes salieron del Lugar Santo. Todos
los sacerdotes que estaban allí se habían purificado, cualquiera que
fuera su división. \bibleverse{12} Todos los levitas cantores -- Asaf,
Hemán, Jedutún y sus hijos y parientes -- se pusieron de pie al lado
oriental del altar. Estaban vestidos de lino fino, tocando címbalos,
arpas y liras, y acompañados por ciento veinte sacerdotes que tocaban
las trompetas. \bibleverse{13} Los trompetistas y los cantores se unían
con una sola voz para alabar y dar gracias al Señor. Acompañados por las
trompetas, los címbalos y los instrumentos musicales, los cantores
alzaron sus voces, alabando al Señor: ``Porque él es bueno; su amor
confiable es eterno''. Entonces el Templo, la casa del Señor, se llenó
de una nube. \footnote{\textbf{5:13} 1Cró 16,34} \bibleverse{14} Los
sacerdotes no podían levantarse para continuar con el servicio a causa
de la nube, porque la gloria del Señor había llenado el Templo de
Dios.\footnote{\textbf{5:14} 2Cró 7,1; 2Cró 7,3}

\hypertarget{el-discurso-de-ordenaciuxf3n-y-consagraciuxf3n-del-rey-al-pueblo}{%
\subsection{El discurso de ordenación y consagración del rey al
pueblo}\label{el-discurso-de-ordenaciuxf3n-y-consagraciuxf3n-del-rey-al-pueblo}}

\hypertarget{section-5}{%
\section{6}\label{section-5}}

\bibleverse{1} Entonces Salomón dijo: ``El Señor ha dicho que vive en
las profundas tinieblas. \bibleverse{2} Sin embargo, te he construido un
magnífico Templo, un lugar para que vivas para siempre''.

\bibleverse{3} Entonces el rey se volvió y bendijo a toda la asamblea de
Israel, mientras todos estaban de pie.

\bibleverse{4} Dijo: ``Alabado sea el Señor, Dios de Israel, que ha
cumplido la promesa que le hizo a mi padre David cuando le dijo:
\bibleverse{5} `Desde el día en que saqué a mi pueblo del país de
Egipto, no he elegido una ciudad de ninguna tribu de Israel donde se
pudiera construir un Templo para honrarme, y no he elegido a nadie para
que sea gobernante de mi pueblo Israel. \bibleverse{6} Pero ahora he
elegido a Jerusalén para que se me honre allí, y he elegido a David para
que gobierne a mi pueblo Israel'. \bibleverse{7} ``Mi padre David quería
construir este Templo para honrar al Señor, el Dios de Israel.
\footnote{\textbf{6:7} 2Sam 7,2-13} \bibleverse{8} Pero el Señor le dijo
a mi padre David: `Realmente querías construirme un Templo para
honrarme, y era bueno que quisieras hacerlo. \bibleverse{9} Pero no vas
a construir el Templo. Tu hijo, uno de tus hijos, construirá el Templo
para honrarme'.

\bibleverse{10} ``Ahora el Señor ha cumplido la promesa que hizo. Porque
yo he ocupado el lugar de mi padre y me he sentado en el trono de
Israel, como dijo el Señor, y he construido el Templo para honrar al
Señor, Dios de Israel. \bibleverse{11} He colocado allí el Arca, que
tiene en su interior el acuerdo que el Señor hizo con Los hijos de
Israel''.

\hypertarget{oraciuxf3n-de-consagraciuxf3n-de-salomuxf3n}{%
\subsection{Oración de consagración de
Salomón}\label{oraciuxf3n-de-consagraciuxf3n-de-salomuxf3n}}

\bibleverse{12} Entonces Salomón se puso delante del altar del Señor,
ante toda la asamblea de Israel, y extendió las manos en oración.
\bibleverse{13} Salomón había hecho una plataforma de bronce de cinco
codos de largo, cinco de ancho y tres de alto. La había colocado en
medio del patio, y estaba de pie sobre ella. Entonces se arrodilló ante
toda la asamblea de Israel y extendió sus manos hacia el cielo.
\bibleverse{14} Y dijo: ``Señor, Dios de Israel, no hay ningún dios como
tú en el cielo ni en la tierra, que mantiene tu acuerdo de amor
confiable con tus siervos que te siguen con total devoción.
\bibleverse{15} Tú has cumplido la promesa que hiciste a tu siervo, mi
padre David. Con tu propia boca hiciste esa promesa, y con tus propias
manos la has cumplido hoy.

\bibleverse{16} ``Así que ahora, Señor Dios de Israel, te ruego que
cumplas la promesa que hiciste a tu siervo David, mi padre, cuando le
dijiste: `Si tus descendientes se empeñan en seguir mi camino y en
cumplir mi ley como tú lo has hecho, nunca faltará uno de ellos para
sentarse en el trono de Israel'. \bibleverse{17} Ahora, Señor Dios de
Israel, cumple esta promesa que hiciste a tu siervo David.

\bibleverse{18} ``Pero, ¿realmente vivirá Dios aquí en la tierra entre
la gente? Los cielos, incluso los más altos, no pueden contenerte, ¡y
mucho menos este Templo que he construido! \footnote{\textbf{6:18} 2Cró
  2,5} \bibleverse{19} Por favor, escucha la oración de tu siervo y su
petición, Señor Dios mío. Por favor, escucha las súplicas y las
oraciones que tu siervo presenta ante ti. \bibleverse{20} Que vigiles
este Templo día y noche, cuidando el lugar donde dijiste que serías
honrado. Que escuches la oración que tu siervo eleva hacia este lugar,
\bibleverse{21} y que escuches la petición de tu siervo y de tu pueblo
Israel cuando oran hacia este lugar. Por favor, escucha desde el cielo
donde vives. Que escuches y perdones.

\bibleverse{22} ``Cuando alguien peca contra otro y se le exige un
juramento declarando la verdad\footnote{\textbf{6:22} ``Declarando la
  verdad'': implícito.} ante tu altar en este Templo, \footnote{\textbf{6:22}
  Éxod 22,10} \bibleverse{23} escucha desde el cielo, actúa y juzga a
tus siervos. Repara a los culpables; reivindica y recompensa a los que
hacen el bien.

\bibleverse{24} ``Cuando tu pueblo Israel sea derrotado por un enemigo
porque ha pecado contra ti, y si vuelve arrepentido a ti, orando por el
perdón en este Templo, \bibleverse{25} entonces escucha desde el cielo y
perdona el pecado de tu pueblo Israel, y haz que vuelva a la tierra que
le diste a él y a sus antepasados.

\bibleverse{26} ``Si los cielos se cierran y no llueve porque tu pueblo
ha pecado contra ti, si oran mirando hacia este lugar y si vuelven
arrepentidos a ti, apartándose de su pecado porque los has castigado,
\footnote{\textbf{6:26} Deut 28,23-24} \bibleverse{27} entonces escucha
desde el cielo y perdona el pecado de tus siervos, tu pueblo Israel.
Enséñales el buen camino para que puedan andar por él, y envía la lluvia
sobre la tierra que le has dado a tu pueblo como posesión.

\bibleverse{28} ``Si hay hambre en la tierra, o enfermedad, o tizón o
moho en las cosechas, o si hay langostas u orugas, o si viene un enemigo
a sitiar las ciudades de la tierra -- sea cualquier tipo de plaga o de
enfermedad -- \bibleverse{29} entonces cualquier tipo de oración o
cualquier tipo de apelación que haga cualquiera o todo tu pueblo Israel,
de hecho cualquiera que, consciente de sus problemas y dolores, ore
mirando hacia este Templo, \bibleverse{30} entonces tú escucha desde el
cielo, el lugar donde vives, y perdona. Dales según su modo de vivir,
porque tú sabes cómo son realmente las personas por dentro, y sólo tú
conoces el verdadero carácter de las personas. \bibleverse{31} Entonces
te respetarán y seguirán tus caminos todo el tiempo que vivan en la
tierra que diste a nuestros antepasados.

\bibleverse{32} ``En cuanto a los extranjeros que no pertenecen a tu
pueblo Israel, sino que vienen de una tierra lejana, habiendo oído
hablar de tu carácter y poder, cuando vengan y oren mirando hacia este
Templo, \bibleverse{33} entonces escucha desde el cielo, el lugar donde
vives, y dales lo que piden. De esta manera, todos los habitantes de la
tierra llegarán a conocerte y respetarte, al igual que tu propio pueblo
Israel. También sabrán que este Templo que he construido te honra.

\bibleverse{34} ``Cuando tu pueblo vaya a luchar contra sus enemigos,
dondequiera que lo envíes, y cuando te ore mirando hacia la ciudad que
has elegido y la casa que he construido para honrarte, \footnote{\textbf{6:34}
  Dan 6,11} \bibleverse{35} entonces escucha desde el cielo lo que están
orando y pidiendo, y apoya su causa.

\bibleverse{36} ``Si pecan contra ti -- y no hay nadie que no peque --
puedes enojarte con ellos y entregarlos a un enemigo que los lleve como
prisioneros a una tierra extranjera, cercana o lejana. \bibleverse{37}
Pero si recapacitan en su tierra de cautiverio y se arrepienten y te
piden misericordia, diciendo: `Hemos pecado, hemos hecho mal, hemos
actuado con maldad', \bibleverse{38} y vuelven a ti con total sinceridad
en sus pensamientos y actitudes allí en su tierra de cautiverio; y oran
mirando hacia la tierra que le diste a sus antepasados y la ciudad que
elegiste y el Templo que he construido para honrarte, \bibleverse{39}
entonces escucha desde el cielo, el lugar donde vives, responde y apoya
su causa. Perdona a tu pueblo que ha pecado contra ti.

\bibleverse{40} ``Ahora, Dios mío, abre tus ojos y que tus oídos presten
atención a las oraciones ofrecidas en este lugar.

\bibleverse{41} ``\,'Ven, Señor, y entra en tu casa,\footnote{\textbf{6:41}
  Literalmente, ``lugar de descanso'', pero con el sentido de que se
  trata del lugar habitual donde alguien reside, de ahí lo de ``casa''.}
junto con tu Arca de poder. Que tus sacerdotes lleven la salvación como
un vestido; que tu pueblo fiel grite de alegría por tu bondad.
\footnote{\textbf{6:41} Sal 132,8-9}

\bibleverse{42} ``Señor Dios, no rechaces al rey que has elegido.
Acuérdate de tu amor fiel a tu siervo
David'''.\textsuperscript{{[}\textbf{6:42} Los versos 41 y 42 provienen
del Salmos 132.{]}}{[}\textbf{6:42} 2Sam 7,13{]}

\hypertarget{apariciuxf3n-de-la-gloria-de-dios-salomuxf3n-y-el-pueblo-fiesta-solemne-de-sacrificios-y-asamblea-de-celebraciuxf3n}{%
\subsection{Aparición de la gloria de Dios; Salomón y el pueblo fiesta
solemne de sacrificios y asamblea de
celebración}\label{apariciuxf3n-de-la-gloria-de-dios-salomuxf3n-y-el-pueblo-fiesta-solemne-de-sacrificios-y-asamblea-de-celebraciuxf3n}}

\hypertarget{section-6}{%
\section{7}\label{section-6}}

\bibleverse{1} Cuando Salomón terminó de orar, bajó fuego del cielo y
quemó el holocausto y los sacrificios, y la gloria del Señor llenó el
Templo. \footnote{\textbf{7:1} Lev 9,24; 1Re 18,38; Éxod 40,34}
\bibleverse{2} Los sacerdotes no podían entrar en el Templo del Señor
porque la gloria del Señor llenaba el Templo del Señor. \bibleverse{3}
Cuando todos los israelitas vieron el fuego que descendía y la gloria
del Señor en el Templo, se arrodillaron e inclinaron el rostro hacia el
suelo. Adoraron y alabaron al Señor, diciendo: ``¡Es bueno! Su amor
confiable es eterno''.

\bibleverse{4} Entonces el rey y todo el pueblo ofrecieron sacrificios
al Señor. \bibleverse{5} El rey Salomón ofreció un sacrificio de 22. 000
reses y 120. 000 ovejas. De este modo, el rey y todo el pueblo dedicaron
el Templo de Dios. \bibleverse{6} Los sacerdotes estaban de pie en sus
puestos, y también los levitas, con los instrumentos musicales que el
rey David había hecho para dar alabanzas, y que David había utilizado
para alabar. Cantaban: ``¡Porque su amor fiel es eterno!''. Frente a
ellos, los sacerdotes tocaron las trompetas, y todos los israelitas se
pusieron de pie.

\bibleverse{7} Después Salomón dedicó el centro del patio frente al
Templo del Señor. Allí presentó los holocaustos y la grasa de las
ofrendas de amistad, ya que en el altar de bronce que había hecho no
cabían todos los holocaustos, las ofrendas de grano y la grasa de las
ofrendas. \footnote{\textbf{7:7} 1Re 8,62-66}

\bibleverse{8} Luego, durante los siete días siguientes, Salomón celebró
la fiesta con todo Israel, una gran reunión que llegó desde Lebo-hamat
hasta el Wadi de Egipto.

\bibleverse{9} El octavo día\footnote{\textbf{7:9} El octavo día del
  Festival.} celebraron una asamblea final, pues la dedicación del altar
había durado siete días, y la fiesta otros siete días. \bibleverse{10}
El día veintitrés del mes séptimo, Salomón envió al pueblo a su casa.
Todavía estaban celebrando y muy contentos por la bondad que el Señor
había mostrado a David, a Salomón y a su pueblo Israel.

\hypertarget{la-repetida-apariciuxf3n-de-dios-y-su-respuesta-promesa-y-amenaza-a-la-oraciuxf3n-de-salomuxf3n}{%
\subsection{La repetida aparición de Dios y su respuesta (promesa y
amenaza) a la oración de
Salomón}\label{la-repetida-apariciuxf3n-de-dios-y-su-respuesta-promesa-y-amenaza-a-la-oraciuxf3n-de-salomuxf3n}}

\bibleverse{11} Después de que Salomón hubo terminado el Templo del
Señor y el palacio real, habiendo realizado con éxito todo lo que había
querido hacer para el Templo del Señor y para su propio palacio,

\bibleverse{12} el Señor se le apareció por la noche y le dijo ``He
escuchado tu oración y he elegido este lugar para mí como Templo de
sacrificio. \footnote{\textbf{7:12} Deut 12,5}

\bibleverse{13} Si yo cerrara el cielo para que no lloviera, o mandara a
la langosta a devorar la tierra, o enviara una plaga entre mi pueblo,
\bibleverse{14} y si mi pueblo, llamado por mi nombre, se humillara y
orara y se volviera a mí, y se apartara de sus malos caminos, entonces
yo oiría desde el cielo, perdonaría sus pecados y sanaría su tierra.
\bibleverse{15} Ahora mis ojos estarán abiertos y mis oídos prestarán
atención a las oraciones que se ofrezcan en este lugar, \bibleverse{16}
porque he elegido y consagrado este Templo para que se me honre allí
para siempre. Siempre velaré por él y lo cuidaré, porque me importa
mucho.

\bibleverse{17} ``En cuanto a ti, si sigues mis caminos como lo hizo tu
padre David, haciendo todo lo que te he dicho que hagas, y si guardas
mis leyes y reglamentos, \bibleverse{18} entonces me aseguraré de que tu
reinado sea seguro. Yo hice este acuerdo con tu padre David, diciéndole:
`Siempre tendrás un descendiente que gobierne sobre Israel'. \footnote{\textbf{7:18}
  2Sam 7,12; 2Sam 7,16}

\bibleverse{19} Pero si te alejas e ignoras las leyes y los mandamientos
que te he dado, y si vas a servir y adorar a otros dioses,
\bibleverse{20} entonces te quitaré de la tierra que te di. Desterraré
de mi presencia este Templo que he dedicado a mi honor, y lo convertiré
en objeto de burla entre las naciones. \bibleverse{21} Este Templo que
ahora es tan respetado se estropeará de tal manera que los transeúntes
dirán: `¿Por qué el Señor ha actuado así con esta tierra y este Templo?'
\footnote{\textbf{7:21} Deut 29,23-26; Jer 22,8-9}

\bibleverse{22} La respuesta será: `Porque han abandonado al Señor, el
Dios de sus padres, que los sacó de Egipto, y se han aferrado a otros
dioses, adorándolos y sirviéndolos. Por eso el Señor ha traído sobre
ellos toda esta angustia'''.

\hypertarget{informaciuxf3n-sobre-las-ciudades-y-fortalezas-de-salomuxf3n}{%
\subsection{Información sobre las ciudades y fortalezas de
Salomón}\label{informaciuxf3n-sobre-las-ciudades-y-fortalezas-de-salomuxf3n}}

\hypertarget{section-7}{%
\section{8}\label{section-7}}

\bibleverse{1} Salomón había tardado veinte años en construir el Templo
del Señor y su propio palacio. \bibleverse{2} Salomón reconstruyó las
ciudades que Hiram le había dado y envió israelitas a vivir allí.

\bibleverse{3} Luego Salomón atacó a Hamat-zoba y la capturó.
\bibleverse{4} Edificó Tadmor en el desierto y también construyó todas
las ciudades-almacén de Hamat. \bibleverse{5} Reconstruyó Bet-horón Alto
y Bajo, ciudades fortificadas con murallas y puertas enrejadas,
\bibleverse{6} y también Baalat. Construyó todas las ciudades-almacén
que le pertenecían, y todas las ciudades donde guardaba sus carros y
caballos. Construyó todo lo que quiso en Jerusalén, en el Líbano y en
todo su reino.

\hypertarget{los-obreros-de-salomuxf3n-y-sus-capataces-su-esposa-la-princesa-egipcia-se-traslada-al-palacio-construido-para-ella}{%
\subsection{Los obreros de Salomón y sus capataces; Su esposa, la
princesa egipcia, se traslada al palacio construido para
ella}\label{los-obreros-de-salomuxf3n-y-sus-capataces-su-esposa-la-princesa-egipcia-se-traslada-al-palacio-construido-para-ella}}

\bibleverse{7} Hubo algunos pueblos que permanecieron en la tierra: los
hititas, amorreos, ferezeos, heveos y jebuseos, gente que no era
israelita. \bibleverse{8} Eran los descendientes restantes de los
pueblos que los israelitas no habían destruido. Salomón los hizo
trabajar como mano de obra forzada, como lo hacen hasta el día de hoy.
\bibleverse{9} Pero Salomón no hizo trabajar a ninguno de los israelitas
como esclavos. En cambio, eran sus militares, sus oficiales y los
comandantes de sus carros y jinetes. \bibleverse{10} También eran los
principales oficiales del rey Salomón, 250 hombres que supervisaban al
pueblo.\footnote{\textbf{8:10} ``El pueblo'': o, ``sus trabajadores''.}

\bibleverse{11} Salomón trasladó a la hija del faraón de la Ciudad de
David al palacio que había construido para ella. Porque dijo: ``Mi mujer
no puede vivir en el palacio de David, rey de Israel, porque donde ha
ido el Arca del Señor hay lugares sagrados''.

\hypertarget{orden-de-sacrificio-y-servicio-en-el-templo-de-salomuxf3n}{%
\subsection{Orden de sacrificio y servicio en el templo de
Salomón}\label{orden-de-sacrificio-y-servicio-en-el-templo-de-salomuxf3n}}

\bibleverse{12} Entonces Salomón presentó holocaustos al Señor en el
altar del Señor que había construido frente al pórtico del Templo.
\footnote{\textbf{8:12} 2Cró 1,3-6} \bibleverse{13} Siguió el requisito
de las ofrendas diarias, tal como lo había ordenado Moisés para los
sábados, las lunas nuevas y las tres fiestas anuales: la Fiesta de los
Panes sin Levadura, la Fiesta de las Semanas y la Fiesta de los
Refugios. \footnote{\textbf{8:13} Núm 28,2; Núm 28,9; Núm 28,11; Núm
  28,17; Núm 28,26; Núm 29,12}

\bibleverse{14} Siguiendo las instrucciones de su padre David, asignó
las divisiones de los sacerdotes para su servicio, y a los levitas en
sus responsabilidades para ofrecer alabanzas, y para ayudar a los
sacerdotes en sus tareas diarias. También asignó a los porteros por sus
divisiones en cada puerta, tal como lo había instruido David, el hombre
de Dios. \footnote{\textbf{8:14} 1Cró 23,1-26} \bibleverse{15} Siguieron
exactamente las instrucciones de David en cuanto a los sacerdotes, los
levitas y todo lo relacionado con los tesoros.

\bibleverse{16} Así se llevó a cabo toda la obra de Salomón, desde el
día en que se pusieron los cimientos del Templo del Señor hasta que se
terminó. Así quedó terminado el Templo del Señor.

\hypertarget{paseos-de-ofir-de-salomuxf3n}{%
\subsection{Paseos de Ofir de
Salomón}\label{paseos-de-ofir-de-salomuxf3n}}

\bibleverse{17} Después de esto, Salomón fue a Ezión-geber y a Elot, en
la costa del país de Edom. \bibleverse{18} Hiram le envió barcos al
mando de sus propios oficiales, junto con marineros experimentados.
Fueron con los hombres de Salomón a Ofir, donde cargaron 450 talentos de
oro, que luego llevaron al rey Salomón.

\hypertarget{visita-de-la-reina-de-saba}{%
\subsection{Visita de la Reina de
Saba}\label{visita-de-la-reina-de-saba}}

\hypertarget{section-8}{%
\section{9}\label{section-8}}

\bibleverse{1} La reina de Saba se enteró de la fama de Salomón y vino a
Jerusalén para ponerle a prueba con preguntas difíciles. Trajo consigo
un séquito muy numeroso, con camellos cargados de especias, grandes
cantidades de oro y piedras preciosas. Se acercó a Salomón y le preguntó
todo lo que tenía en mente. \bibleverse{2} Salomón respondió a todas sus
preguntas. No había nada que no pudiera explicarle. \bibleverse{3}
Cuando la reina de Saba vio la sabiduría de Salomón y el palacio que
había construido, \bibleverse{4} la comida que había en la mesa, cómo
vivían sus funcionarios, cómo funcionaban sus sirvientes y cómo estaban
vestidos, las ropas de los camareros y los holocaustos que presentaba en
el Templo del Señor, quedó tan asombrada\footnote{\textbf{9:4} ``Estaba
  tan asombrada'': implícito por la frase (Literalmente) ``ya no había
  aliento en ella''.} que apenas podía respirar.

\bibleverse{5} Le dijo al rey: ``Es cierto lo que he oído en mi país
sobre tus proverbios\footnote{\textbf{9:5} ``Proverbios'': Literalmente,
  ``palabras''.} ¡y tu sabiduría! \bibleverse{6} Pero no creí lo que me
dijeron hasta que vine y lo vi con mis propios ojos. De hecho, no me
contaron ni la mitad: ¡la extensión de tu sabiduría supera con creces lo
que he oído! \bibleverse{7} ¡Qué feliz debe ser tu pueblo! ¡Qué felices
los que trabajan para ti, los que están aquí cada día escuchando tu
sabiduría! \bibleverse{8} Alabado sea el Señor, tu Dios, que tanto se
complace en ti, que te puso en su trono como rey para gobernar en su
nombre. Por el amor de tu Dios a Israel los ha asegurado para siempre, y
te ha hecho rey sobre ellos para que hagas lo justo y lo correcto''.

\bibleverse{9} Presentó al rey ciento veinte talentos de oro, enormes
cantidades de especias y piedras preciosas. Nunca antes había habido
especias como las que la reina de Saba regaló al rey Salomón.

\bibleverse{10} (Hiram y los hombres de Salomón, que trajeron oro de
Ofir, también trajeron madera de algum y piedras preciosas.
\bibleverse{11} El rey utilizó la madera de algum para hacer escalones
para el Templo y para el palacio real, y en liras y arpas para los
músicos. Nunca se había visto nada igual en el país de Judá).
\bibleverse{12} El rey Salomón dio a la reina de Saba todo lo que quiso,
todo lo que pidió. Esto era mucho más de lo que ella había traído al
rey. Luego, ella y sus acompañantes regresaron a su país.

\hypertarget{riqueza-obras-de-arte-y-esplendor-de-salomuxf3n-y-artuxedculos-de-comercio-exterior}{%
\subsection{Riqueza, obras de arte y esplendor de Salomón y artículos de
comercio
exterior}\label{riqueza-obras-de-arte-y-esplendor-de-salomuxf3n-y-artuxedculos-de-comercio-exterior}}

\bibleverse{13} El peso del oro que Salomón recibía cada año era de 666
talentos, \bibleverse{14} sin incluir el que recibía de los comerciantes
y mercaderes. Todos los reyes de Arabia y los gobernadores del país
también le llevaban oro y plata a Salomón. \bibleverse{15} El rey
Salomón hizo doscientos escudos de oro martillado. Cada escudo requería
seiscientos siclos de oro martillado. \bibleverse{16} También hizo
trescientos escudos pequeños de oro martillado. Cada uno de estos
escudos requería trescientas monedas de oro. El rey los colocó en el
Palacio del Bosque del Líbano. \bibleverse{17} El rey hizo también un
gran trono de marfil y lo cubrió de oro puro. \bibleverse{18} El trono
tenía seis escalones, con un escabel de oro adosado. A ambos lados del
asiento había reposabrazos, con leones de pie junto a los reposabrazos.
\bibleverse{19} En los seis escalones había doce leones, uno en cada
extremo de cada escalón. Nunca se había hecho nada parecido para ningún
reino. \bibleverse{20} Todas las copas del rey Salomón eran de oro, y
todos los utensilios del Palacio del Bosque del Líbano eran de oro puro.
No se usó plata, porque no era valorada en los días de Salomón.
\bibleverse{21} El rey tenía una flota de barcos de Tarsis tripulada por
marineros de Hiram. Una vez cada tres años los barcos de Tarsis llegaban
con un cargamento de oro, plata, marfil, monos y pavos reales.

\hypertarget{la-posiciuxf3n-de-poder-de-salomuxf3n-y-la-riqueza-que-promueve}{%
\subsection{La posición de poder de Salomón y la riqueza que
promueve}\label{la-posiciuxf3n-de-poder-de-salomuxf3n-y-la-riqueza-que-promueve}}

\bibleverse{22} El rey Salomón era más grande que cualquier otro rey de
la tierra en riqueza y sabiduría. \bibleverse{23} Todos los reyes de la
tierra querían conocer a Salomón para escuchar la sabiduría que Dios
había puesto en su mente. \bibleverse{24} Año tras año, todos los
visitantes traían regalos: objetos de plata y oro, ropa, armas,
especias, caballos y mulas. \bibleverse{25} Salomón tenía cuatro mil
establos para caballos y carros, y doce mil jinetes.\footnote{\textbf{9:25}
  ``Jinetes'': o ``caballos'' (la palabra en hebreo es la misma). Sin
  embargo, dado que ya se ha dado el número de establos para los
  caballos, es más probable que se refiera a los jinetes.} Los mantuvo
en las ciudades de los carros, y también con él en Jerusalén.
\footnote{\textbf{9:25} 2Cró 1,14-17; 1Re 5,6} \bibleverse{26} Dominó a
todos los reyes desde el río Éufrates hasta el país de los filisteos y
hasta la frontera con Egipto. \bibleverse{27} El rey hizo que en
Jerusalén abundara la plata como las piedras, y la madera de cedro como
los sicómoros en las estribaciones.\footnote{\textbf{9:27} Véase 1:15.}
\bibleverse{28} Los caballos de Salomón fueron importados de Egipto y de
muchas otras tierras.

\hypertarget{las-fuentes-de-la-historia-de-salomuxf3n-su-muerte}{%
\subsection{Las fuentes de la historia de Salomón; su
muerte}\label{las-fuentes-de-la-historia-de-salomuxf3n-su-muerte}}

\bibleverse{29} El resto de los hechos de Salomón, desde el principio
hasta el final, están escritos en las Actas de Natán el Profeta, en la
Profecía de Ahías el Silonita y en las Visiones de Iddo el Vidente
acerca Jeroboam, hijo de Nabat. \bibleverse{30} Salomón gobernó en
Jerusalén sobre todo Israel durante cuarenta años. \bibleverse{31} Luego
Salomón murió y fue enterrado en la ciudad de su padre David. Su hijo
Roboam asumió como rey.

\hypertarget{roboam-y-jeroboam-en-siquem-la-divisiuxf3n-del-imperio}{%
\subsection{Roboam y Jeroboam en Siquem; la división del
imperio}\label{roboam-y-jeroboam-en-siquem-la-divisiuxf3n-del-imperio}}

\hypertarget{section-9}{%
\section{10}\label{section-9}}

\bibleverse{1} Roboam fue a Siquem, porque todos los israelitas habían
ido a Siquem para hacerlo rey. \bibleverse{2} Jeroboam, hijo de Nabat,
todavía estaba en Egipto cuando se enteró de esto. (Había huido a Egipto
para escapar del rey Salomón y estaba viviendo allí). \footnote{\textbf{10:2}
  1Re 11,40} \bibleverse{3} Los líderes israelitas enviaron a buscarlo.
Jereboam y todos los israelitas fueron a hablar con Roboam.
\bibleverse{4} ``Tu padre nos impuso una pesada carga'', le dijeron.
``Pero ahora, si aligeras la carga que tu padre impuso y las pesadas
exigencias que nos impuso, te serviremos''.

\bibleverse{5} Roboam respondió: ``Vuelvan dentro de tres días''. Así
que el pueblo se fue.

\hypertarget{consejeruxeda-de-rehoboams}{%
\subsection{Consejería de Rehoboams}\label{consejeruxeda-de-rehoboams}}

\bibleverse{6} El rey Roboam pidió consejo a los ancianos que habían
servido a su padre Salomón en vida. ``¿Cómo me aconsejan que responda a
esta gente sobre esto?'' , preguntó.

\bibleverse{7} Ellos le respondieron: ``Si tratas bien a este pueblo y
les complaces hablándoles con amabilidad, siempre te servirán''.

\bibleverse{8} Pero Roboam desestimó el consejo de los ancianos. En
cambio, pidió consejo a los jóvenes con los que había crecido y que
estaban cerca de él. \bibleverse{9} Entonces les preguntó: ``¿Qué
respuesta aconsejan ustedes que enviemos a esta gente que me ha dicho:
`Aligera la carga que tu padre puso sobre nosotros'?''

\bibleverse{10} Los jóvenes con los que se había criado le dijeron:
``Esto es lo que tienes que decirles a estas personas que te han dicho:
`Tu padre nos ha hecho pesada la carga, pero tú deberías aligerarla'.
Esto es lo que debes responderles: `Mi dedo meñique es más grueso que la
cintura de mi padre. \bibleverse{11} Mi padre les puso una carga pesada,
y yo la haré aún más pesada. Mi padre te castigó con látigos; yo los
castigaré con escorpiones'\,''.

\hypertarget{descenso-de-las-diez-tribus-elecciuxf3n-de-jeroboam-como-rey-de-israel}{%
\subsection{Descenso de las diez tribus; Elección de Jeroboam como rey
de
Israel}\label{descenso-de-las-diez-tribus-elecciuxf3n-de-jeroboam-como-rey-de-israel}}

\bibleverse{12} Tres días después, Jeroboam y todo el pueblo volvieron a
Roboam, porque el rey les había dicho: ``Vuelvan dentro de tres días''.
\bibleverse{13} El rey les respondió bruscamente. Desechando el consejo
de los ancianos, \bibleverse{14} contestó utilizando el consejo de los
jóvenes. Les dijo: ``Mi padre les impuso una pesada carga, y yo la haré
aún más pesada. Mi padre te castigó con látigos; yo te castigaré con
escorpiones''.

\bibleverse{15} El rey no escuchó lo que el pueblo decía, pues este
cambio de circunstancias venía de Dios, para cumplir lo que el Señor le
había dicho a Jeroboam hijo de Nabat por medio de Ahías el silonita.

\bibleverse{16} Cuando todos los israelitas vieron que el rey no los
escuchaba, le dijeron al rey ``¿Qué parte tenemos en David, y qué parte
tenemos en el hijo de Isaí? ¡Vete a casa, Israel! Estás solo, casa de
David''. Así que todos los israelitas se fueron a casa.

\bibleverse{17} Sin embargo, Roboam seguía gobernando sobre los
israelitas que vivían en Judá. \bibleverse{18} Entonces el rey Roboam
envió a Adoram, encargado de los trabajos forzados,\footnote{\textbf{10:18}
  Fue enviado a sofocar la rebelión.} pero los israelitas lo apedrearon
hasta la muerte. El rey Roboam se subió rápidamente a su carro y corrió
de regreso a Jerusalén. \bibleverse{19} Como resultado, Israel se ha
rebelado contra la casa de David hasta el día de hoy.

\hypertarget{roboam-se-abstiene-de-la-guerra-contra-israel-bajo-la-direcciuxf3n-de-dios}{%
\subsection{Roboam se abstiene de la guerra contra Israel bajo la
dirección de
Dios}\label{roboam-se-abstiene-de-la-guerra-contra-israel-bajo-la-direcciuxf3n-de-dios}}

\hypertarget{section-10}{%
\section{11}\label{section-10}}

\bibleverse{1} Cuando Roboam llegó a Jerusalén, reunió a los hombres de
las familias de Judá y Benjamín - 180. 000 guerreros elegidos - para ir
a luchar contra Israel y devolver el reino a Roboam. \bibleverse{2} Pero
llegó un mensaje del Señor a Semaías, el hombre de Dios, que decía:
\bibleverse{3} ``Dile a Roboam, hijo de Salomón, rey de Judá, y a todos
los israelitas que viven en Judá y Benjamín: \bibleverse{4} `Esto es lo
que dice el Señor. No luches contra tus parientes. Cada uno de ustedes,
váyase a su casa. Porque lo que ha sucedido se debe a mí'\,''. Así que
obedecieron lo que el Señor les dijo y no lucharon contra Jeroboam.

\hypertarget{fortalezas-de-roboam}{%
\subsection{Fortalezas de Roboam}\label{fortalezas-de-roboam}}

\bibleverse{5} Roboam se quedó en Jerusalén y reforzó las defensas de
las ciudades de Judá. \bibleverse{6} Construyó Belén, Etam, Tecoa,
\bibleverse{7} Bet-zur, Soco, Adulam, \bibleverse{8} Gat, Maresa, Zif,
\bibleverse{9} Adoraim, Laquis, Azeca, \bibleverse{10} Zora, Ajalón y
Hebrón. Estas son las ciudades fortificadas de Judá y de Benjamín.
\bibleverse{11} Fortaleció sus fortalezas y puso comandantes a cargo de
ellas, junto con provisiones de alimentos, aceite de oliva y vino.
\bibleverse{12} Almacenó escudos y lanzas en todas las ciudades y las
hizo muy fuertes. Así mantuvo a Judá y a Benjamín bajo su dominio.

\hypertarget{entrada-de-sacerdotes-levitas-y-personas-piadosas-del-reino-de-diez-tribus}{%
\subsection{Entrada de sacerdotes, levitas y personas piadosas del reino
de diez
tribus}\label{entrada-de-sacerdotes-levitas-y-personas-piadosas-del-reino-de-diez-tribus}}

\bibleverse{13} Sin embargo, los sacerdotes y los levitas de todo Israel
decidieron ponerse del lado de Roboam. \bibleverse{14} Los levitas
incluso dejaron sus pastos y propiedades y vinieron a Judá y Jerusalén,
porque Jeroboam y sus hijos se negaron a permitirles servir como
sacerdotes del Señor. \footnote{\textbf{11:14} 2Cró 13,9}
\bibleverse{15} Jeroboam eligió a sus propios sacerdotes para los
lugares altos\footnote{\textbf{11:15} ``Lugares altos'': asociado con
  santuarios paganos.} y para los ídolos de cabra y de becerro que había
hecho. \footnote{\textbf{11:15} 1Re 12,31} \bibleverse{16} Los de todas
las tribus de Israel que estaban comprometidos con el culto a su Dios
seguían a los levitas a Jerusalén para sacrificar al Señor, el Dios de
sus antepasados. \bibleverse{17} Así apoyaron al reino de Judá y durante
tres años fueron leales a Roboam, hijo de Salomón, porque siguieron el
camino de David y Salomón.

\hypertarget{historia-familiar-de-rehaboam}{%
\subsection{Historia familiar de
rehaboam}\label{historia-familiar-de-rehaboam}}

\bibleverse{18} Roboam se casó con Mahalat, que era hija de Jerimot,
hijo de David, y de Abihail, hija\footnote{\textbf{11:18} Probablemente
  la nieta.} de Eliab, hijo de Isaí. \footnote{\textbf{11:18} 1Sam 16,6}

\bibleverse{19} Ella fue la madre de sus hijos Jeús, Samaria y Zaham.
\bibleverse{20} Después de ella se casó con Maaca, hija de
Absalón,\footnote{\textbf{11:20} Probablemente la nieta.} y fue madre de
sus hijos Abías, Atai, Ziza y Selomit. \bibleverse{21} Roboam amaba a
Maaca, la hija de Absalón, más que a todas sus otras esposas y
concubinas. Tuvo en total dieciocho esposas y sesenta concubinas,
veintiocho hijos y sesenta hijas. \bibleverse{22} Roboam nombró a Abías,
hijo de Maacá, príncipe heredero entre sus hermanos, planeando hacerlo
rey. \bibleverse{23} Roboam también tuvo la sabiduría de colocar a
algunos de sus hijos en toda la tierra de Judá y Benjamín, y en todas
las ciudades fortificadas. Les dio abundantes provisiones y les buscó
muchas esposas. Trabajó para conseguirles muchas esposas.

\hypertarget{incursiuxf3n-y-saqueo-del-rey-egipcio-sisak-apariciuxf3n-del-profeta-semeuxedas}{%
\subsection{Incursión y saqueo del rey egipcio Sisak; Aparición del
profeta
Semeías}\label{incursiuxf3n-y-saqueo-del-rey-egipcio-sisak-apariciuxf3n-del-profeta-semeuxedas}}

\hypertarget{section-11}{%
\section{12}\label{section-11}}

\bibleverse{1} Una vez que Roboam se afianzó en el trono y estuvo seguro
de su poder, junto con todos los israelitas abandonó la ley del Señor.
\bibleverse{2} En el quinto año del reinado de Roboam, Sisac, rey de
Egipto, vino y atacó a Jerusalén porque habían sido infieles a Dios.
\bibleverse{3} Vino de Egipto con 1. 200 carros, 60. 000 jinetes y un
ejército que no se podía contar Egipto: libios, siquenos y cusitas.
\bibleverse{4} Conquistó las ciudades fortificadas de Judá y luego se
acercó a Jerusalén. \footnote{\textbf{12:4} 2Cró 11,4-10} \bibleverse{5}
El profeta Semaías se acercó a Roboam y a los dirigentes de Judá que
habían huido para ponerse a salvo en Jerusalén a causa de Sisac. Les
dijo: ``Esto es lo que dice el Señor: `Ustedes me han abandonado, así
que yo los he abandonado a Sisac'\,''.

\bibleverse{6} Los dirigentes de Israel y el rey admitieron que estaban
equivocados y dijeron: ``El Señor tiene razón''.

\bibleverse{7} Cuando el Señor vio que se habían arrepentido, envió un
mensaje a Semaías, diciendo: ``Se han arrepentido. No los destruiré, y
pronto los salvaré. Mi ira no se derramará sobre Jerusalén por medio de
Sisac. \bibleverse{8} Aun así, se convertirán en sus súbditos, para que
aprendan la diferencia entre servirme a mí y servir a los reyes de la
tierra''.

\bibleverse{9} El rey Sisac de Egipto atacó Jerusalén y se llevó los
tesoros del Templo del Señor y los tesoros del palacio real. Se lo llevó
todo, incluidos los escudos de oro que había hecho Salomón.
\bibleverse{10} Más tarde Roboam los sustituyó por escudos de bronce y
los entregó para que los cuidaran los comandantes de la guardia
apostados a la entrada del palacio real. \bibleverse{11} Cada vez que el
rey entraba en el Templo del Señor, los guardias lo acompañaban,
llevando los escudos, y luego los llevaban de vuelta a la sala de
guardia. \bibleverse{12} Como Roboam se arrepintió, la ira del Señor no
cayó sobre él, y el Señor no lo destruyó por completo. Las cosas fueron
bien en Judá.

\hypertarget{conclusiuxf3n-del-gobierno-de-roboam-y-las-fuentes-de-su-historia}{%
\subsection{Conclusión del gobierno de Roboam y las fuentes de su
historia}\label{conclusiuxf3n-del-gobierno-de-roboam-y-las-fuentes-de-su-historia}}

\bibleverse{13} El rey Roboam se hizo poderoso en Jerusalén. Tenía
cuarenta y un años cuando llegó a ser rey, y reinó diecisiete años en
Jerusalén, la ciudad que el Señor había escogido de entre todas las
tribus de Israel donde sería honrado. El nombre de su madre era Naama la
amonita. \bibleverse{14} Pero Roboam hizo lo malo porque no se
comprometió a seguir al Señor.

\bibleverse{15} Lo que hizo Roboam, desde el principio hasta el final,
está escrito en los registros del profeta Semaías y del vidente Iddo que
tratan de las genealogías. Sin embargo, Roboam y Jeroboam siempre
estuvieron en guerra entre sí. \footnote{\textbf{12:15} 2Cró 13,22}

\bibleverse{16} Roboam murió y fue enterrado en la Ciudad de David. Su
hijo Abías tomó el relevo como rey.

\hypertarget{la-guerra-de-abias-con-jeroboam-su-discurso-al-ejuxe9rcito-de-jeroboam}{%
\subsection{La guerra de Abias con Jeroboam; su discurso al ejército de
Jeroboam}\label{la-guerra-de-abias-con-jeroboam-su-discurso-al-ejuxe9rcito-de-jeroboam}}

\hypertarget{section-12}{%
\section{13}\label{section-12}}

\bibleverse{1} Abías llegó a ser rey de Judá en el año dieciocho del
reinado de Jeroboam. \bibleverse{2} Reinó en Jerusalén durante tres
años. Su madre se llamaba Micaías, hija de Uriel, y era de Gabaa. Abías
y Jeroboam estaban en guerra. \bibleverse{3} Abías salió a luchar con un
ejército de 400. 000 valientes guerreros, mientras que Jeroboam se le
opuso con su ejército de 800. 000 guerreros elegidos de gran fuerza.
\bibleverse{4} Abías se paró en el monte Zemaraim, en la región
montañosa de Efraín, y dijo: ``¡Escúchenme, Jeroboam y todo Israel!
\bibleverse{5} ¿No entienden que el Señor, el Dios de Israel, dio el
reino de Israel a David y a sus descendientes para siempre mediante un
acuerdo vinculante?\footnote{\textbf{13:5} ``Acuerdo vinculante'':
  Literalmente, ``un pacto de sal''.} \bibleverse{6} Sin embargo,
Jeroboam, hijo de Nabat, sólo un siervo de Salomón, hijo de David, tuvo
la audacia de rebelarse contra su amo. \bibleverse{7} Entonces algunos
malvados buenos para nada se reunieron a su alrededor y desafiaron a
Roboam, hijo de Salomón, cuando éste era joven e inexperto y no podía
enfrentarse a ellos.

\bibleverse{8} ``Ahora, ¿creen realmente que pueden oponerse al reino
del Señor, en manos de los descendientes de David? Podrán ser una gran
horda, y podrán tener los becerros de oro que Jeroboam les hizo como
dioses. \footnote{\textbf{13:8} 1Re 12,28} \bibleverse{9} ¿Pero acaso no
expulsaron a los sacerdotes del Señor, a los descendientes de Aarón y a
los levitas, y se hicieron sacerdotes como los de otras naciones? Ahora
cualquiera que quiera puede venir y dedicarse, sacrificando un novillo y
siete carneros, y puede hacerse sacerdote de cosas que realmente no son
dioses. \footnote{\textbf{13:9} 2Cró 11,15}

\bibleverse{10} ``¡Pero para nosotros, el Señor es nuestro Dios! No lo
hemos abandonado. Tenemos sacerdotes que sirven al Señor y que son
descendientes de Aarón, y tenemos levitas que los ayudan en su
ministerio. \bibleverse{11} Mañana y tarde presentan holocaustos y
queman incienso aromático al Señor. Colocan las hileras de panes de la
proposición en la mesa purificada, y encienden las lámparas del
candelabro de oro cada noche. Hacemos lo que el Señor, nuestro Dios, nos
ha dicho que hagamos, mientras tú lo has abandonado. \footnote{\textbf{13:11}
  Núm 28,3-8} \bibleverse{12} ¡Dios nos guía! Sus sacerdotes tocan las
trompetas para ir a la batalla contra ustedes. Pueblo de Israel, no
peleen contra el Señor, el Dios de sus padres, porque no ganarán''.
\footnote{\textbf{13:12} Núm 10,9}

\hypertarget{victoria-de-abias-sobre-jeroboam}{%
\subsection{Victoria de Abias sobre
Jeroboam}\label{victoria-de-abias-sobre-jeroboam}}

\bibleverse{13} Pero Jeroboam había enviado tropas para atacar por la
retaguardia, de modo que mientras él y la fuerza principal estaban al
frente de Judá,\footnote{\textbf{13:13} Aquí Judá se refiere al reino
  del sur, e Israel al del norte.} la emboscada estaba detrás de ellos.
\bibleverse{14} Judá se dio la vuelta y se dio cuenta de que tenían que
luchar por delante y por detrás. Clamaron al Señor pidiendo ayuda.
Entonces los sacerdotes tocaron las trompetas, \bibleverse{15} y los
hombres de Judá dieron un fuerte grito. Cuando gritaron, Dios hirió a
Jeroboam y a todo Israel frente a Abías y a Judá. \bibleverse{16} Los
israelitas huyeron de Judá, y Dios los entregó a Judá, derrotados.
\bibleverse{17} Abías y sus hombres los golpearon duramente, y 500. 000
de los mejores guerreros de Israel murieron. \bibleverse{18} Así que los
israelitas fueron sometidos en ese momento, y el pueblo de Judá salió
victorioso porque se apoyó en el Señor, el Dios de sus antepasados.
\bibleverse{19} Abías persiguió a Jeroboam y le capturó algunas
ciudades: Betel, Janá y Efrón, junto con sus aldeas.

\bibleverse{20} Jereboam nunca recuperó su poder durante el reinado de
Abías. Finalmente, el Señor lo abatió y murió.

\hypertarget{conclusiuxf3n-y-fuentes-de-la-historia-de-abias}{%
\subsection{Conclusión y fuentes de la historia de
Abias}\label{conclusiuxf3n-y-fuentes-de-la-historia-de-abias}}

\bibleverse{21} Pero Abías se hizo cada vez más fuerte. Se casó con
catorce esposas y tuvo veintidós hijos y dieciséis hijas.
\bibleverse{22} El resto de lo que hizo Abías -lo que dijo y lo que
logró- está registrado en la historia escrita por el profeta
Ido.\footnote{\textbf{13:22} 2Cró 12,15}

\hypertarget{la-intervenciuxf3n-de-asa-contra-la-idolatruxeda}{%
\subsection{La intervención de Asa contra la
idolatría}\label{la-intervenciuxf3n-de-asa-contra-la-idolatruxeda}}

\hypertarget{section-13}{%
\section{14}\label{section-13}}

\bibleverse{1} Abías murió y fue enterrado en la Ciudad de David. Su
hijo Asa tomó el relevo como rey. Durante diez años de su reinado la
nación estuvo en paz. \footnote{\textbf{14:1} 1Re 15,11-12}
\bibleverse{2} Asa hizo lo que era bueno y correcto a los ojos del
Señor. \bibleverse{3} Derribó los altares y los lugares altos
extranjeros, rompió sus pilares sagrados y cortó los postes de
Asera.\footnote{\textbf{14:3} Imágenes dedicadas a la diosa cananea de
  la fertilidad Asera. Se discute si hay que añadir ``poste''.}
\bibleverse{4} Ordenó a Judá que adorara al Señor, el Dios de sus
antepasados, y que observara la ley y los mandamientos. \bibleverse{5}
También derribó los lugares altos y los altares de incienso de todas las
ciudades de Judá. Bajo su gobierno el reino estaba en paz. \footnote{\textbf{14:5}
  2Cró 15,15}

\hypertarget{eleva-la-fuerza-defensiva-del-imperio}{%
\subsection{Eleva la fuerza defensiva del
imperio}\label{eleva-la-fuerza-defensiva-del-imperio}}

\bibleverse{6} Como el país estaba en paz, pudo reconstruir las ciudades
fortificadas de Judá. No hubo guerras durante estos años porque el Señor
le había concedido la paz. \bibleverse{7} Entonces Asa le dijo al pueblo
de Judá: ``Construyamos estas ciudades y rodeémoslas de murallas y
torres y puertas enrejadas. La tierra sigue siendo nuestra, porque
seguimos adorando al Señor, nuestro Dios. Lo adoramos, y él nos ha dado
la paz de todos nuestros enemigos''. Así que comenzaron los proyectos de
construcción, y los completaron con éxito.

\hypertarget{la-victoria-de-asa-sobre-los-cusitas-serah}{%
\subsection{La victoria de Asa sobre los cusitas
Serah}\label{la-victoria-de-asa-sobre-los-cusitas-serah}}

\bibleverse{8} Asa tenía un ejército compuesto por trescientos mil
hombres de Judá que llevaban grandes escudos y lanzas, y doscientos
ochenta mil hombres de Benjamín que llevaban escudos regulares y arcos.
Todos ellos eran valientes guerreros.

\bibleverse{9} Zeraa el etíope, los atacó con un ejército de mil veces
mil\footnote{\textbf{14:9} ``Mil veces mil'': aunque equivale a un
  millón, puede significar simplemente un número muy grande.} hombres y
trescientos carros, avanzando hasta Maresa. \bibleverse{10} Asa salió a
enfrentarse a él, alineándose para la batalla en el Valle de Cefatá, en
Maresa. \bibleverse{11} Asa pidió ayuda al Señor, su Dios: ``Señor, no
hay nadie fuera de ti que pueda ayudar al impotente contra el poderoso.
Por favor, ayúdanos, Señor, nuestro Dios, porque confiamos en ti. Hemos
venido contra esta horda porque confiamos en ti,\footnote{\textbf{14:11}
  ``Confiamos en ti''. Literalmente, ``en tu nombre''. En otras
  palabras, Asa está diciendo que la batalla es del Señor y no de ellos.}
Señor. Tú eres nuestro Dios. No permitas que un simple ser humano te
venza''.\footnote{\textbf{14:11} ``Te venza'': la palabra aquí significa
  ``contener, frenar''.}

\bibleverse{12} El Señor hirió a los etíopes frente a Asa y Judá, y los
etíopes huyeron. \bibleverse{13} Asa y su ejército los persiguieron
hasta Gerar. Los etíopes murieron; no hubo ninguno que sobreviviera,
pues quedaron atrapados entre el Señor y su ejército. Los hombres de
Judá se llevaron una gran cantidad de botín. \bibleverse{14} También
atacaron todas las ciudades alrededor de Gerar, porque sus habitantes
estaban aterrorizados por el Señor. Los hombres de Judá tomaron una gran
cantidad de botín de todas las ciudades. \bibleverse{15} Luego atacaron
los campamentos de los pastores y tomaron muchas ovejas y camellos.
Luego regresaron a Jerusalén.

\hypertarget{la-amonestaciuxf3n-del-profeta-azaruxedas}{%
\subsection{La amonestación del profeta
Azarías}\label{la-amonestaciuxf3n-del-profeta-azaruxedas}}

\hypertarget{section-14}{%
\section{15}\label{section-14}}

\bibleverse{1} El Espíritu de Dios vino sobre Azarías, hijo de Oded.
\bibleverse{2} Salió al encuentro de Asa y le dijo: ``Escúchame, Asa y
todo Judá y Benjamín. El Señor está con ustedes mientras estén con él.
Si lo buscan, lo encontrarán; pero si lo abandonan, él los abandonará a
ustedes. \bibleverse{3} ``Durante muchos años Israel estuvo sin el
verdadero Dios, sin un sacerdote que les enseñara y sin la ley.
\footnote{\textbf{15:3} Os 3,4} \bibleverse{4} Pero cuando tuvieron
problemas, volvieron al Señor, el Dios de Israel; lo buscaron y lo
encontraron. \footnote{\textbf{15:4} Jer 29,13-14} \bibleverse{5} ``En
aquellos tiempos, viajar era peligroso, pues todos los habitantes de las
tierras estaban muy revueltos. En todas partes la gente tenía terribles
problemas. \bibleverse{6} La nación luchaba contra la nación, y el
pueblo contra el pueblo, pues Dios los sumía en el pánico con toda clase
de problemas. \bibleverse{7} Pero tú tienes que ser fuerte, no débil,
porque serás recompensado por el trabajo que hagas''. \footnote{\textbf{15:7}
  1Cor 15,58}

\hypertarget{renovaciuxf3n-de-asa-del-pacto-con-dios}{%
\subsection{Renovación de Asa del pacto con
Dios}\label{renovaciuxf3n-de-asa-del-pacto-con-dios}}

\bibleverse{8} Cuando Asa escuchó estas palabras proféticas del profeta
Azarías, hijo de Oded, se animó. Quitó los ídolos viles de todo el
territorio de Judá y Benjamín y de las ciudades que había capturado en
la región montañosa de Efraín. Luego reparó el altar del Señor que
estaba frente al pórtico del Templo del Señor. \bibleverse{9} Entonces
Asa convocó a todo Judá y Benjamín, junto con los israelitas de las
tribus de Efraín, Manasés y Simeón que vivían entre ellos, pues mucha
gente había desertado de Israel y se había acercado a Asa al ver que el
Señor, su Dios, estaba con él. \bibleverse{10} Se reunieron en Jerusalén
en el tercer mes del decimoquinto año del reinado de Asa.
\bibleverse{11} Ese día sacrificaron al Señor setecientos bueyes y siete
mil ovejas del botín que habían traído. \bibleverse{12} Luego hicieron
un acuerdo para seguir concienzuda y completamente al Señor, el Dios de
sus antepasados. \bibleverse{13} También acordaron que cualquiera que se
negara a seguir al Señor, el Dios de Israel, sería condenado a muerte,
ya fuera joven o viejo, hombre o mujer. \bibleverse{14} Declararon su
juramento con un fuerte grito, acompañado de trompetas y toques de
cuernos de carnero. \bibleverse{15} Todo Judá se alegró del juramento
que habían hecho a conciencia. Lo buscaron sinceramente, y lo
encontraron. El Señor les dio la paz de todos sus enemigos. \footnote{\textbf{15:15}
  2Cró 14,5-6; 2Cró 20,30}

\bibleverse{16} El rey Asa también destituyó a Maaca de su cargo de
reina madre\footnote{\textbf{15:16} En realidad era la abuela de Asa.}
por hacer un poste de Asera ofensivo. Asa cortó su vil ídolo, lo aplastó
y lo quemó en el valle del Cedrón. \footnote{\textbf{15:16} 1Re 15,13-15}
\bibleverse{17} Mientras los lugares altos no fueron eliminados de
Israel,\footnote{\textbf{15:17} En 14:3 y 14:5 se registra la
  eliminación de los lugares altos. Por supuesto, esto no se refería a
  Israel, el reino del norte, sino sólo al territorio sobre el que Asa
  tenía autoridad.} Asa fue completamente devoto del Señor durante toda
su vida. \bibleverse{18} Llevó al Templo de Dios los artículos de plata
y oro que él y su padre habían dedicado.

\hypertarget{la-guerra-de-asa-con-baesa-de-israel-su-refugio-en-ben-adad-de-siria}{%
\subsection{La guerra de Asa con Baesa de Israel; su refugio en Ben-adad
de
Siria}\label{la-guerra-de-asa-con-baesa-de-israel-su-refugio-en-ben-adad-de-siria}}

\bibleverse{19} No hubo más guerra hasta el año treinta y cinco del
reinado de Asa.

\hypertarget{section-15}{%
\section{16}\label{section-15}}

\bibleverse{1} En el año treinta y seis del reinado de Asa,\footnote{\textbf{16:1}
  Probablemente calculado desde el comienzo del reino del sur, más que
  desde el reinado personal de Asa. Véase 1 Reyes 15. Esto también se
  aplicaría al versículo anterior.} Baasa, rey de Israel, invadió Judá.
Fortificó Ramá para impedir que nadie viniera o fuera a Asa, rey de
Judá.\footnote{\textbf{16:1} Se supone que esta acción fue
  principalmente para evitar el continuo éxodo de personas hacia el
  reino del sur.} \footnote{\textbf{16:1} 1Re 15,16-22} \bibleverse{2}
Asa tomó la plata y el oro de los tesoros del Templo del Señor y del
palacio real y los envió a Ben-hadad, rey de Siria, que vivía en
Damasco, con un mensaje que decía \bibleverse{3} ``Haz una alianza entre
tú y yo como la que hubo entre mi padre y el tuyo. Mira la plata y el
oro que te he enviado. Ve y rompe tu acuerdo con Baasa, rey de Israel,
para que me deje y se vaya a casa''.

\bibleverse{4} El rey Ben-hadad hizo lo que Asa le había pedido, y envió
a sus ejércitos y a sus comandantes a atacar las ciudades de Israel.
Conquistaron Ijón, Dan, Abel-maim y todas las ciudades almacén de
Neftalí. \bibleverse{5} Cuando Baasa se enteró, dejó de fortificar Ramá
y abandonó su proyecto. \bibleverse{6} Entonces el rey Asa fue con todos
los hombres de Judá, y se llevaron de Rama las piedras y los maderos que
Baasa había usado para construir, y con ellos edificó Geba y Mizpa.

\hypertarget{el-discurso-de-castigo-de-hanani-a-asa-tiene-un-efecto-negativo}{%
\subsection{El discurso de castigo de Hanani a Asa tiene un efecto
negativo}\label{el-discurso-de-castigo-de-hanani-a-asa-tiene-un-efecto-negativo}}

\bibleverse{7} Pero en ese momento el vidente Hanani se presentó ante
Asa, rey de Judá, y le dijo: ``Por haber puesto tu confianza en el rey
de Aram y no haber puesto tu confianza en el Señor, tu Dios, tu
oportunidad de destruir el ejército del rey de Aram ha desaparecido.
\bibleverse{8} ¿Acaso los etíopes y los libios no tenían un gran
ejército con muchos carros y jinetes? Sin embargo, como confiaste en el
Señor, él te hizo victorioso sobre ellos. \footnote{\textbf{16:8} 2Cró
  14,8-12} \bibleverse{9} Porque el Señor busca por toda la tierra la
oportunidad de mostrar su poder a favor de los que le son total y
sinceramente devotos. Tú has actuado de forma estúpida al hacer esto.
Así que de ahora en adelante siempre estarás en guerra''.

\bibleverse{10} Asa se enfadó con el vidente. Estaba tan enojado con él
por esto que lo puso en prisión. Al mismo tiempo, Asa comenzó a
maltratar a algunos del pueblo.

\hypertarget{el-fin-de-asa-y-un-entierro-honorable}{%
\subsection{El fin de Asa y un entierro
honorable}\label{el-fin-de-asa-y-un-entierro-honorable}}

\bibleverse{11} El resto de lo que hizo Asa, de principio a fin, está
escrito en el Libro de los Reyes de Judá e Israel. \bibleverse{12} En el
año treinta y nueve de su reinado, Asa tuvo problemas con una enfermedad
en los pies, que se fue agravando. Sin embargo, ni siquiera en su
enfermedad se dirigió al Señor, sino sólo a los médicos. \bibleverse{13}
Asa murió en el año cuarenta y uno de su reinado. \bibleverse{14} Fue
enterrado en la tumba que él mismo había preparado en la Ciudad de
David. Lo colocaron en un lecho lleno de especias, aceites perfumados y
fragancias. Luego hicieron un gran fuego para honrarlo.\footnote{\textbf{16:14}
  2Cró 21,19; Jer 34,5}

\hypertarget{el-gobierno-piadoso-y-feliz-de-josafat}{%
\subsection{El gobierno piadoso y feliz de
Josafat}\label{el-gobierno-piadoso-y-feliz-de-josafat}}

\hypertarget{section-16}{%
\section{17}\label{section-16}}

\bibleverse{1} El hijo de Asa, Josafat, asumió el cargo de rey. Reforzó
las defensas de su país contra Israel. \footnote{\textbf{17:1} 1Re 15,24}
\bibleverse{2} Asignó tropas a cada ciudad fortificada de Judá y colocó
guarniciones en todo Judá y en las ciudades de Efraín que su padre Asa
había capturado. \bibleverse{3} El Señor apoyó a Josafat porque siguió
los caminos de su padre David. No creía en los baales, \bibleverse{4}
sino que adoraba al Dios de su padre y obedecía sus mandamientos, a
diferencia de lo que hacía el reino de Israel. \bibleverse{5} Así, el
Señor aseguró el dominio del reino de Josafat, y todo el pueblo de Judá
le pagó sus cuotas. Como resultado, llegó a ser muy rico y honrado.
\footnote{\textbf{17:5} 2Cró 18,1} \bibleverse{6} Se comprometió
sinceramente con lo que el Señor quería. También eliminó los lugares
altos y los postes de Asera de Judá.

\hypertarget{josafat-instruye-al-pueblo-en-la-ley-del-seuxf1or}{%
\subsection{Josafat instruye al pueblo en la ley del
Señor}\label{josafat-instruye-al-pueblo-en-la-ley-del-seuxf1or}}

\bibleverse{7} En el tercer año de su reinado, Josafat envió a sus
funcionarios Ben-hail, Abdías, Zacarías, Netanel y Micaías a enseñar en
las ciudades de Judá. \bibleverse{8} Envió con ellos a los levitas
llamados Semaías, Netanías, Zebadías, Asael, Semiramot, Jonatán,
Adonías, Tobías y Tobadonías, y con ellos a los sacerdotes Elisama y
Joram. \bibleverse{9} Llevando consigo el Libro de la Ley del Señor,
enseñaban mientras recorrían Judá. Visitaron todas las ciudades de Judá,
enseñando al pueblo.

\hypertarget{la-reputaciuxf3n-de-josafat-entre-los-pueblos-vecinos-y-su-importante-poder-militar}{%
\subsection{La reputación de Josafat entre los pueblos vecinos y su
importante poder
militar}\label{la-reputaciuxf3n-de-josafat-entre-los-pueblos-vecinos-y-su-importante-poder-militar}}

\bibleverse{10} Todos los reinos circundantes estaban atemorizados por
el Señor, de modo que no atacaron a Josafat. \bibleverse{11} Algunos de
los filisteos incluso le trajeron regalos y plata, mientras que los
árabes le trajeron 7. 700 carneros y 7. 700 cabras. \bibleverse{12}
Josafat se hizo cada vez más poderoso, y construyó fortalezas y
ciudades-almacén en Judá. \bibleverse{13} Mantenía una gran cantidad de
provisiones en las ciudades de Judá. También tenía tropas, guerreros
experimentados, en Jerusalén. \bibleverse{14} Este es un recuento de
ellos, según sus líneas familiares: de Judá, los comandantes de miles:
Adná, el comandante, y 300. 000 guerreros poderosos con él;
\bibleverse{15} luego Johanán, el comandante, y 280. 000 con él;
\bibleverse{16} luego Amasías, hijo de Zicrí, que se ofreció como
voluntario para servir al Señor, y 200. 000 guerreros poderosos con él;
\bibleverse{17} de Benjamín, Eliada, poderoso guerrero, y 200. 000 con
él, armados con arcos y escudos; \bibleverse{18} luego Jozabad, y 180.
000 con él, listos para la batalla; \bibleverse{19} estos fueron los
hombres que sirvieron al rey, además de los que asignó a las ciudades
fortificadas en todo Judá.

\hypertarget{josafat-y-acab-unen-fuerzas-en-una-guerra-contra-los-sirios}{%
\subsection{Josafat y Acab unen fuerzas en una guerra contra los
sirios}\label{josafat-y-acab-unen-fuerzas-en-una-guerra-contra-los-sirios}}

\hypertarget{section-17}{%
\section{18}\label{section-17}}

\bibleverse{1} Josafat era muy rico y honrado, e hizo una alianza
matrimonial con Acab. \footnote{\textbf{18:1} 2Cró 17,5} \bibleverse{2}
Algunos años después fue a visitar a Acab en Samaria. Acab sacrificó
muchas ovejas y ganado para él y la gente que lo acompañaba, y lo animó
a atacar Ramot de Galaad. \bibleverse{3} Acab, rey de Israel, le
preguntó a Josafat, rey de Judá: ``¿Quieres ir conmigo contra Ramot de
Galaad?'' Josafat respondió: ``Tú y yo somos como uno, y mis hombres y
los tuyos son como uno. Uniremos nuestras fuerzas contigo en esta
guerra''.

\hypertarget{el-mensaje-favorable-de-los-400-profetas-micha-deberuxeda-ser-entrevistado}{%
\subsection{El mensaje favorable de los 400 profetas; Micha debería ser
entrevistado}\label{el-mensaje-favorable-de-los-400-profetas-micha-deberuxeda-ser-entrevistado}}

\bibleverse{4} Entonces Josafat le dijo al rey de Israel: ``Pero antes,
por favor, averigua lo que dice el Señor''.

\bibleverse{5} Así que el rey de Israel sacó a los profetas
-cuatrocientos- y les preguntó: ``¿Subimos a atacar Ramot de Galaad, o
no lo hacemos?'' . ``Sí, hagámoslo'', le respondieron, ``porque Dios la
entregará al rey''.

\bibleverse{6} Pero Josafat preguntó: ``¿No hay aquí otro profeta del
Señor al que podamos preguntar?''

\bibleverse{7} ``Sí, hay otro hombre que podría consultar al Señor'',
respondió el rey de Israel, ``pero no me gusta porque nunca profetiza
nada bueno para mí, ¡siempre es malo! Se llama Micaías, hijo de Imá''.
``No deberías hablar así'', dijo Josafat.

\bibleverse{8} El rey de Israel llamó a uno de sus funcionarios y le
dijo: ``Tráeme enseguida a Micaías, hijo de Imá''.

\bibleverse{9} Vestidos con sus ropas reales, el rey de Israel y el rey
Josafat de Judá, estaban sentados en sus tronos en la era junto a la
puerta de Samaria, con todos los profetas profetizando frente a ellos.
\bibleverse{10} Uno de ellos, Sedequías, hijo de Quená, se había hecho
unos cuernos de hierro. Anunció: ``Esto es lo que dice el Señor: `¡Con
estos cuernos vas a cornear a los arameos hasta que estén muertos!'\,''

\bibleverse{11} Todos los profetas profetizaban lo mismo, diciendo:
``Adelante, ataquen Ramot de Galaad; tendrán éxito, porque el Señor se
la entregará al rey''.

\hypertarget{la-buena-fortuna-inicial-de-micha-luego-su-anuncio-de-la-perdiciuxf3n}{%
\subsection{La buena fortuna inicial de Micha, luego su anuncio de la
perdición}\label{la-buena-fortuna-inicial-de-micha-luego-su-anuncio-de-la-perdiciuxf3n}}

\bibleverse{12} El mensajero que fue a llamar a Micaías le dijo: ``Mira,
todos los profetas son unánimes en profetizar positivamente al rey. Así
que asegúrate de hablar positivamente como ellos''.

\bibleverse{13} Pero Micaías respondió: ``Vive el Señor, yo sólo puedo
decir lo que mi Dios me dice''.

\bibleverse{14} Cuando llegó ante el rey, éste le preguntó: ``¿Subimos a
atacar Ramot de Galaad, o no?'' ``Sí, sube y vence'', contestó Micaías,
``porque serán entregados al rey''.\footnote{\textbf{18:14} Está claro
  que hay algo en el tono de esta declaración que llevó a Acab a
  responder como lo hizo en el siguiente verso.}

\bibleverse{15} Pero el rey le dijo: ``¿Cuántas veces tengo que hacerte
jurar que sólo me dirás la verdad en nombre del Señor?''

\bibleverse{16} Entonces Micaías respondió: ``Vi a todo Israel disperso
por los montes como ovejas sin pastor. El Señor dijo: `Este pueblo no
tiene dueño;\footnote{\textbf{18:16} ``No tiene dueño'': lo que implica
  que su amo está muerto.} que cada uno se vaya a su casa en paz'\,''.

\bibleverse{17} El rey de Israel le dijo a Josafat: ``¿No te he dicho
que él nunca me profetiza nada bueno, sino sólo malo?''

\bibleverse{18} Micaías continuó diciendo: ``Escucha, pues, lo que dice
el Señor. Vi al Señor sentado en su trono, rodeado de todo el ejército
del cielo que estaba a su derecha y a su izquierda. \bibleverse{19} El
Señor preguntó: `¿Quién engañará a Acab, rey de Israel, para que ataque
a Ramot de Galaad y lo mate allí?' ``Uno dijo esto, otro dijo aquello, y
otro dijo otra cosa. \bibleverse{20} Finalmente vino un espíritu y se
acercó al Señor y dijo: `Yo lo engañaré'. ```¿Cómo vas a hacerlo?'
preguntó el Señor.

\bibleverse{21} ```Iré y seré un espíritu mentiroso y haré que todos sus
profetas digan mentiras', respondió el espíritu. ``El Señor respondió:
`Eso funcionará. Ve y hazlo'.

\bibleverse{22} ``Como ves, el Señor ha puesto un espíritu mentiroso en
estos profetas tuyos, y el Señor ha dictado tu sentencia de muerte''.

\hypertarget{el-maltrato-de-miqueas-por-sedequuxedas-y-su-captura-por-acab}{%
\subsection{El maltrato de Miqueas por Sedequías y su captura por
Acab}\label{el-maltrato-de-miqueas-por-sedequuxedas-y-su-captura-por-acab}}

\bibleverse{23} Entonces Sedequías, hijo de Quená, fue y abofeteó a
Micaías en la cara, y le preguntó: ``¿Por dónde se fue el Espíritu del
Señor cuando me dejó hablar contigo?'' \footnote{\textbf{18:23} 2Cró
  18,10}

\bibleverse{24} ``¡Pronto lo descubrirás cuando intentes encontrar algún
lugar secreto para esconderte!'' respondió Micaías.

\bibleverse{25} El rey de Israel ordenó: ``Pongan a Micaías bajo arresto
y llévenlo a Amón, el gobernador de la ciudad, y a mi hijo Joás.
\bibleverse{26} Diles que estas son las instrucciones del rey: `Pongan a
este hombre en la cárcel. Denle sólo pan y agua hasta mi regreso
seguro'\,''.

\bibleverse{27} ``Si de hecho regresas sano y salvo, entonces el Señor
no ha hablado a través de mí'', declaró Micaías. ``¡Presten atención
todos a todo lo que he dicho!''

\hypertarget{derrota-de-los-aliados-en-ramoth-muerte-de-acab}{%
\subsection{Derrota de los aliados en Ramoth; Muerte de
Acab}\label{derrota-de-los-aliados-en-ramoth-muerte-de-acab}}

\bibleverse{28} El rey de Israel y Josafat, rey de Judá, fueron a atacar
Ramot de Galaad. \bibleverse{29} El rey de Israel le dijo a Josafat:
``Cuando yo vaya a la batalla me disfrazaré, pero tú debes llevar tus
ropas reales''. Así que el rey de Israel se disfrazó y fue a la batalla.
\bibleverse{30} El rey de Aram ya había dado estas órdenes a sus
comandantes de carros ``Diríjanse directamente hacia el rey de Israel
solo. No luchen con nadie más, sea quien sea''.

\bibleverse{31} Así que cuando los comandantes de los carros vieron a
Josafat, gritaron: ``¡Ahí está el rey de Israel!''. Así que se volvieron
para atacarlo, pero Josafat pidió ayuda, y el Señor lo ayudó. Dios los
alejó de él, \bibleverse{32} pues cuando los comandantes de los carros
se dieron cuenta de que no era el rey de Israel, dejaron de perseguirlo.
\bibleverse{33} Sin embargo, un arquero enemigo disparó una flecha al
azar, hiriendo al rey de Israel entre las junturas de su armadura, junto
al peto. El rey le dijo a su auriga: ``¡Da la vuelta y sácame del
combate, porque me han herido!''. \bibleverse{34} La batalla duró todo
el día. El rey de Israel se apuntaló en su carro para enfrentar a los
arameos hasta el atardecer. Pero murió al atardecer.

\hypertarget{discurso-de-castigo-del-profeta-jehuxfa-a-josafat}{%
\subsection{Discurso de castigo del profeta Jehú a
Josafat}\label{discurso-de-castigo-del-profeta-jehuxfa-a-josafat}}

\hypertarget{section-18}{%
\section{19}\label{section-18}}

\bibleverse{1} Una vez que Josafat llegó sano y salvo a su casa en
Jerusalén, \bibleverse{2} Jehú, hijo de Hanani, el vidente, salió a
hacerle frente. Le dijo al rey Josafat: ``¿Por qué ayudas a los
malvados? ¿Por qué amas a los que odian al Señor? El Señor está enojado
contigo por eso. \bibleverse{3} Aun así, has hecho algunas cosas buenas,
como destruir los postes de Asera en todo el país, y te has comprometido
sinceramente a seguir a Dios''. \footnote{\textbf{19:3} 2Cró 17,3-6}

\hypertarget{la-reorganizaciuxf3n-de-la-administraciuxf3n-de-justicia-por-parte-de-josafat}{%
\subsection{La reorganización de la administración de justicia por parte
de
Josafat}\label{la-reorganizaciuxf3n-de-la-administraciuxf3n-de-justicia-por-parte-de-josafat}}

\bibleverse{4} Josafat siguió viviendo en Jerusalén, y una vez más viajó
entre el pueblo, desde Beerseba hasta la región montañosa de Efraín,
para animarlos a servir al Señor, el Dios de sus padres. \bibleverse{5}
Nombró jueces en todo el país, en todas las ciudades fortificadas de
Judá. \bibleverse{6} Les dijo a los jueces: ``Tengan cuidado con lo que
hacen como jueces, porque no buscan la aprobación de la gente, sino la
aprobación del Señor. Él es quien está con ustedes cuando dan su
veredicto. \bibleverse{7} Así que asegúrense de tener reverencia por
Dios, obedeciéndolo y haciendo lo que él quiere, porque Dios no permite
ninguna clase de injusticia, favoritismo o soborno''.

\bibleverse{8} Josafat también nombró en Jerusalén a algunos de los
levitas, sacerdotes y jefes de familia para que actuaran como jueces
respecto a la ley del Señor y para que resolvieran las disputas. Debían
tener sus tribunales en Jerusalén.\footnote{\textbf{19:8} Está claro que
  los sacerdotes y los levitas tendrían jurisdicción sobre la ley
  religiosa, mientras que otros líderes se ocuparían de las disputas
  civiles. La segunda mención de Jerusalén en el versículo, junto con
  los versículos que siguen, sugieren que debían funcionar como un
  tribunal nacional, un tribunal de apelación.} \footnote{\textbf{19:8}
  Deut 17,8-9; Deut 19,17}

\bibleverse{9} Les dio estas órdenes: ``Deben honrar a Dios y actuar con
fidelidad y total compromiso. \bibleverse{10} En todos los casos que se
presenten ante ustedes de su gente que vive en otras ciudades, ya sea
que se trate de un asesinato o de violaciones de la ley, los
mandamientos, los estatutos o las sentencias, deben advertirles que no
ofendan\footnote{\textbf{19:10} Esto incluiría pecados como el perjurio
  o el falso testimonio.} al Señor para que el castigo no caiga sobre ti
y tu pueblo. Si haces esto no serás considerado culpable.
\bibleverse{11} ``Amarías, el sumo sacerdote, tomará la decisión final
por ti en todo lo relacionado con el Señor, y Zebadías, hijo de Ismael,
jefe de la tribu de Judá, en todo lo relacionado con el rey. Los levitas
servirán como oficiales para ayudarte. Sé firme, y que el Señor esté con
los que hacen lo correcto''.

\hypertarget{la-oraciuxf3n-de-josafat-despuuxe9s-de-que-el-enemigo-invadiuxf3}{%
\subsection{La oración de Josafat después de que el enemigo
invadió}\label{la-oraciuxf3n-de-josafat-despuuxe9s-de-que-el-enemigo-invadiuxf3}}

\hypertarget{section-19}{%
\section{20}\label{section-19}}

\bibleverse{1} Después de esto, los moabitas y amonitas, así como
algunos de los meunitas,\footnote{\textbf{20:1} ``Meunitas'': según
  algunos manuscritos de la Septuaginta. El hebreo repite la palabra
  ``amonitas''.} vino a atacar a Josafat. \bibleverse{2} Algunas
personas vinieron y le dijeron a Josafat: ``Un gran ejército viene a
pelear contigo desde Edom,\footnote{\textbf{20:2} ``Desde Edom'', más
  probable que ``de Aram'', como dicen la mayoría de los manuscritos
  hebreos.} desde el otro lado del Mar Muerto. Ya han llegado a
Hazazón-tamar'' (también llamado En-gedi). \bibleverse{3} Josafat tuvo
miedo y fue a preguntar al Señor qué hacer. También ordenó a todos los
habitantes de Judá que ayunaran. \bibleverse{4} Entonces el pueblo de
Judá se reunió en Jerusalén para orar al Señor; de hecho, vinieron de
todas las ciudades de Judá para encomendarse a él.

\bibleverse{5} Josafat se presentó ante el pueblo de Judá y de Jerusalén
reunido en el Templo, frente al patio nuevo, \bibleverse{6} y dijo:
``Señor, Dios de nuestros antepasados, ¿no eres tú el Dios del cielo?
¿No dominas todos los reinos terrestres? Posees fuerza y poder, y nadie
puede enfrentarse a ti. \footnote{\textbf{20:6} 1Cró 29,12; 2Cró 14,10}
\bibleverse{7} Dios nuestro, ¿no expulsaste delante de tu pueblo Israel
a los que vivían en esta tierra? ¿No diste esta tierra a los
descendientes de tu amigo Abrahán para siempre? \bibleverse{8} Ellos
viven en la tierra y han construido aquí un Templo para honrarte,
diciendo: \bibleverse{9} `Si nos sobreviene un desastre, ya sea una
invasión o un juicio, una enfermedad o una hambruna, nos pondremos
delante de este Templo y ante ti, porque este Templo es tuyo. Clamaremos
a ti para que nos ayudes en nuestro sufrimiento, y tú nos escucharás y
nos salvarás'.\footnote{\textbf{20:9} Véase 6:24-30.} \bibleverse{10}
``Mira, aquí vienen los ejércitos de Amón, Moab y el Monte Seir, esos
mismos países que no dejaste que Israel invadiera cuando salieron de
Egipto. Israel los dejó en paz y no los destruyó. \footnote{\textbf{20:10}
  Deut 2,4-5; Deut 2,9; Deut 2,19} \bibleverse{11} ¡Mira cómo nos
recompensan, viniendo a robar la tierra que nos diste a poseer para
siempre! \bibleverse{12} Dios nuestro, ¿no los castigarás, porque no
tenemos poder para enfrentar a un ejército tan grande que marcha contra
nosotros? No sabemos qué hacer. Buscamos tu ayuda''.

\bibleverse{13} Todos los hombres de Judá se pusieron de pie ante el
Señor, junto con sus esposas, hijos y bebés.

\hypertarget{la-respuesta-de-dios-promesa-de-victoria-del-levita-jahaziel}{%
\subsection{La respuesta de Dios: promesa de victoria del levita
Jahaziel}\label{la-respuesta-de-dios-promesa-de-victoria-del-levita-jahaziel}}

\bibleverse{14} Entonces el Espíritu del Señor se apoderó de Jahaziel
mientras estaba de pie en la asamblea. Era hijo de Zacarías, hijo de
Benaía, hijo de Jeiel, hijo de Matanías, un levita de los descendientes
de Asaf. \bibleverse{15} Él dijo: ``Escuchen todos los de Judá, el
pueblo de Jerusalén y el rey Josafat. Esto es lo que el Señor tiene que
decirles: No tengan miedo ni se desanimen por culpa de este gran
ejército. Esta no es su batalla, ¡es la de Dios! \bibleverse{16} Mañana
marchen a enfrentarlos. Los verás subir por el paso de Ziz; los
encontrarás al final del valle, frente al desierto de Jeruel.
\bibleverse{17} Pero no es necesario que luches en esta batalla. Sólo
quédense quietos y observen la victoria del Señor. Él está contigo, Judá
y Jerusalén. ¡No tengan miedo ni se desanimen! Marchen a enfrentarlos,
porque el Señor está con ustedes''.

\bibleverse{18} Josafat se inclinó con el rostro hacia el suelo, y todo
el pueblo de Judá y de Jerusalén se postró ante el Señor, adorándolo.
\bibleverse{19} Entonces los levitas coatitas y corasitas se pusieron de
pie para alabar al Señor, el Dios de Israel, gritando con fuerza.

\hypertarget{la-salida-contra-el-enemigo-autodestrucciuxf3n-del-enemigo-enorme-botuxedn-de-los-juduxedos}{%
\subsection{La salida contra el enemigo; Autodestrucción del enemigo;
enorme botín de los
judíos}\label{la-salida-contra-el-enemigo-autodestrucciuxf3n-del-enemigo-enorme-botuxedn-de-los-juduxedos}}

\bibleverse{20} A la mañana siguiente se levantaron temprano y fueron al
desierto de Tecoa. Al salir, Josafat se levantó y dijo: ``Escúchenme,
pueblo de Judá y de Jerusalén. Confíen en el Señor, su Dios, y serán
reivindicados; confíen en sus profetas, y tendrán éxito''. \footnote{\textbf{20:20}
  Is 28,16}

\bibleverse{21} Después de discutir con el pueblo, designó a unos
cantores para que alabaran al Señor por su gloriosa y santa bondad.
Ellos iban al frente del ejército, cantando: ``¡Alaben al Señor, porque
su amor digno de confianza es eterno!'' \footnote{\textbf{20:21} Sal
  106,1} \bibleverse{22} En cuanto empezaron a cantar y a alabar, el
Señor tendió una emboscada a los hombres de Amón, Moab y el monte Seír
que venían a atacar a Judá, y fueron derrotados. \bibleverse{23} Los
hombres de Amón y Moab se volvieron contra los hombres del monte Seir, y
los mataron a todos. Una vez que terminaron de aniquilar al ejército de
Seir, se volvieron unos contra otros, destruyéndose.\footnote{\textbf{20:23}
  ``They turned on each other, destroying themselves'': Literalmente,
  ``each helped his neighbor to destruction''.} \footnote{\textbf{20:23}
  1Sam 14,20}

\bibleverse{24} Así que cuando los hombres de Judá llegaron a la torre
de vigilancia en el desierto, se asomaron para ver al ejército enemigo y
todo lo que vieron fueron cadáveres tirados en el suelo. Nadie había
escapado. \bibleverse{25} Cuando Josafat y su gente fueron a recoger el
botín, encontraron una gran cantidad de ganado, equipo, ropa,\footnote{\textbf{20:25}
  ``Ropa'': Algunos manuscritos y la Vulgata. La mayoría de los
  manuscritos tienen ``cadáveres''.} y otros artículos de valor, más de
lo que podían llevar. Tardaron tres días en recoger el botín porque era
mucho. \bibleverse{26} Al cuarto día se reunieron en el Valle de la
Bendición. Le pusieron este nombre porque allí bendecían al Señor. Hasta
el día de hoy se le llama el Valle de la Bendición. \bibleverse{27}
Entonces todos los hombres de Judá y Jerusalén celebraron su regreso a
Jerusalén, con Josafat a la cabeza, llenos de alegría por la victoria
del Señor sobre sus enemigos. \bibleverse{28} Entraron en Jerusalén y se
dirigieron directamente al Templo del Señor, acompañados por la música
de arpas, liras y trompetas. \bibleverse{29} Todos los reinos de los
alrededores se asombraron de Dios al oír que el Señor había luchado
contra los enemigos de Israel.\footnote{\textbf{20:29} Véase 17:10.}
\bibleverse{30} Josafat y su reino estaban en paz, pues Dios le dio
descanso; no hubo ataques de ninguna dirección.

\hypertarget{la-conclusiuxf3n-del-gobierno-de-josafat-y-las-fuentes-de-su-historia}{%
\subsection{La conclusión del gobierno de Josafat y las fuentes de su
historia}\label{la-conclusiuxf3n-del-gobierno-de-josafat-y-las-fuentes-de-su-historia}}

\bibleverse{31} Así reinó Josafat sobre Judá, siendo rey a los treinta y
cinco años, y reinó en Jerusalén durante veinticinco años. Su madre se
llamaba Azuba, hija de Silhi. \footnote{\textbf{20:31} 1Re 22,41-51}
\bibleverse{32} Josafat siguió el camino de su padre Asa y no se apartó
de él. Hizo lo que era correcto a los ojos del Señor. \bibleverse{33}
Sin embargo, los lugares altos no fueron eliminados, y el pueblo no se
comprometió con el Dios de sus antepasados.

\bibleverse{34} El resto de lo que hizo Josafat, de principio a fin,
está escrito en las Crónicas de Jehú, hijo de Hanani, registradas en el
Libro de los Reyes de Israel.

\hypertarget{la-alianza-de-josafat-con-ocozuxedas-de-israel-y-su-castigo-su-muerte}{%
\subsection{La alianza de Josafat con Ocozías de Israel y su castigo; su
muerte}\label{la-alianza-de-josafat-con-ocozuxedas-de-israel-y-su-castigo-su-muerte}}

\bibleverse{35} Más adelante en su vida, Josafat, rey de Judá, se alió
con Ocozías, rey de Israel, quien hizo cosas malvadas. \bibleverse{36}
Acordaron trabajar juntos y enviar barcos a Tarsis.\footnote{\textbf{20:36}
  Una empresa comercial conjunta.} Los barcos fueron construidos en
Ezión-geber. \bibleverse{37} Pero Eliezer, hijo de Dodava de Maresa,
profetizó contra Josafat, diciendo: ``Por haber hecho una alianza con
Ocozías, el Señor destruirá lo que estás haciendo''. Entonces los barcos
naufragaron y no pudieron navegar hasta Tarsis.

\hypertarget{el-gobierno-del-rey-joram}{%
\subsection{El gobierno del rey Joram}\label{el-gobierno-del-rey-joram}}

\hypertarget{section-20}{%
\section{21}\label{section-20}}

\bibleverse{1} Josafat murió y fue enterrado con sus antepasados en la
Ciudad de David, y su hijo Joram asumió el cargo de rey.

\hypertarget{asesinato-de-sus-hermanos}{%
\subsection{Asesinato de sus hermanos}\label{asesinato-de-sus-hermanos}}

\bibleverse{2} Sus hermanos, Los hijos de Josafat, fueron Azarías,
Jehiel, Zacarías, Azarías, Miguel y Sefatías. Todos eran hijos de
Josafat, rey de Judá.\footnote{\textbf{21:2} ``Judá'', siguiendo algunos
  manuscritos hebreos, la Septuaginta y la Vulgata. La mayoría de los
  manuscritos hebreos tienen ``Israel'', tal vez reflejando la intención
  del Cronista de que Judá sea visto como el verdadero heredero del
  título ``reino de Israel''. Del mismo modo, el versículo 4.}
\bibleverse{3} Su padre les había dado muchos regalos de plata y oro y
objetos valiosos, así como las ciudades fortificadas de Judá; pero le
dio el reino a Joram porque era el primogénito. \bibleverse{4} Pero una
vez que Joram se aseguró el reino, se aseguró de su posición matando a
todos sus hermanos, junto con algunos de los príncipes de Judá.

\hypertarget{la-posiciuxf3n-de-dios-sobre-la-apostasuxeda-de-joram}{%
\subsection{La posición de Dios sobre la apostasía de
Joram}\label{la-posiciuxf3n-de-dios-sobre-la-apostasuxeda-de-joram}}

\bibleverse{5} Joram tenía treinta y dos años cuando llegó a ser rey, y
reinó en Jerusalén durante ocho años. \bibleverse{6} Siguió los malos
caminos de los reyes de Israel y fue tan malo como Acab, pues se había
casado con una de las hijas de Acab. Hizo lo malo a los ojos del Señor.
\bibleverse{7} Sin embargo, el Señor no quiso destruir el linaje de
David debido al acuerdo que había hecho con él, y había prometido que
los descendientes de David gobernarían para siempre como una lámpara que
siempre arde. \footnote{\textbf{21:7} 2Sam 7,12; 1Re 11,36; Sal 132,17}

\hypertarget{apostasuxeda-de-los-edomitas-y-la-ciudad-de-libna}{%
\subsection{Apostasía de los edomitas y la ciudad de
Libna}\label{apostasuxeda-de-los-edomitas-y-la-ciudad-de-libna}}

\bibleverse{8} Durante el reinado de Joram, Edom se rebeló contra el
gobierno de Judá y eligió su propio rey. \bibleverse{9} Entonces Joram
cruzó a Edom con sus oficiales y todo su ejército de carros. Los
edomitas lo rodearon a él y a sus comandantes de carros, pero él se
abrió paso durante la noche.\footnote{\textbf{21:9} El hebreo no aclara
  si se trató de un ataque nocturno, o simplemente de que Joram escapó.
  En cualquier caso, los siguientes versos revelan que la rebelión no
  fue sofocada por Joram.} \bibleverse{10} A partir de este momento,
Edom se rebeló contra el gobierno de Judá, y lo sigue haciendo hasta el
día de hoy. Al mismo tiempo, Libna también se rebeló contra su gobierno,
porque había abandonado al Señor, el Dios de sus antepasados.

\hypertarget{la-carta-amenazante-del-profeta-eluxedas-a-joram}{%
\subsection{La carta amenazante del profeta Elías a
Joram}\label{la-carta-amenazante-del-profeta-eluxedas-a-joram}}

\bibleverse{11} También construyó lugares altos en los montes de Judá;
hizo que el pueblo de Jerusalén fuera infiel a Dios y alejó a Judá de
él. \bibleverse{12} Joram recibió una carta del profeta Elías que decía:
``Esto es lo que dice el Señor, el Dios de David, tu antepasado: `No has
seguido los caminos de tu padre Josafat, ni de Asa, rey de Judá,
\bibleverse{13} sino que has seguido los caminos de los reyes de Israel,
y has hecho que el pueblo de Jerusalén sea tan infiel como la familia de
Acab. Incluso has matado a tus hermanos, la familia de tu padre, que
eran mejores que tú. \bibleverse{14} Ten cuidado, porque el Señor va a
golpear duramente a tu pueblo: a tus hijos, a tus mujeres y a todo lo
que posees. \bibleverse{15} Tú mismo serás golpeado con una terrible
enfermedad, una enfermedad de los intestinos que empeorará día a día
hasta que salgan'\,''.

\hypertarget{incursiones-filisteas-y-uxe1rabes}{%
\subsection{Incursiones filisteas y
árabes}\label{incursiones-filisteas-y-uxe1rabes}}

\bibleverse{16} El Señor despertó la hostilidad de los filisteos y de
los árabes (que viven cerca de los etíopes) contra Joram.
\bibleverse{17} Vinieron e invadieron Judá, y se llevaron todo lo que
encontraron en el palacio del rey, junto con sus hijos y sus esposas, de
modo que sólo el hijo menor Joacaz\footnote{\textbf{21:17} ``Joacaz'':
  también llamado Azarías.} quedó.

\hypertarget{el-agonizante-final-y-el-entierro-deshonroso-de-joram}{%
\subsection{El agonizante final y el entierro deshonroso de
Joram}\label{el-agonizante-final-y-el-entierro-deshonroso-de-joram}}

\bibleverse{18} Después de todo esto, el Señor hirió a Joram con una
enfermedad de los intestinos para la cual no había cura. \bibleverse{19}
Día tras día se agravaba, hasta que después de dos años completos se le
salieron los intestinos a causa de su enfermedad, y murió en agonía. Su
pueblo no hizo una hoguera para honrarlo como había hecho con sus
antepasados. \bibleverse{20} Joram tenía treinta y dos años cuando fue
rey, y reinó en Jerusalén durante ocho años. Cuando murió, nadie lo
lloró. Fue enterrado en la Ciudad de David, pero no en las tumbas
reales.\footnote{\textbf{21:20} 2Cró 21,5; 2Cró 24,25}

\hypertarget{el-gobierno-del-rey-ochuxf4zuxedas-su-gobierno-desaprobando-a-dios}{%
\subsection{El gobierno del rey Ochôzías; Su gobierno desaprobando a
Dios}\label{el-gobierno-del-rey-ochuxf4zuxedas-su-gobierno-desaprobando-a-dios}}

\hypertarget{section-21}{%
\section{22}\label{section-21}}

\bibleverse{1} El pueblo de Jerusalén nombró a Ocozías, el hijo menor de
Joram, rey en sucesión de su padre, ya que los invasores que habían
entrado en el campamento con los árabes habían matado a todos los hijos
mayores. Así que Ocozías, hijo de Joram, se convirtió en rey de Judá.
\footnote{\textbf{22:1} 2Re 8,25-29} \bibleverse{2} Ocozías tenía
veintidós años\footnote{\textbf{22:2} ``Veintidós'': según 2 Reyes 8:26.
  Aquí su edad se da en el texto hebreo como cuarenta y dos.} cuando
llegó a ser rey, y reinó en Jerusalén durante un año. Su madre se
llamaba Atalía, nieta de Omri. \bibleverse{3} Ocozías también siguió los
malos caminos de la familia de Acab, pues su madre lo animaba a hacer
cosas malas. \bibleverse{4} Hizo lo que era malo a los ojos del Señor,
tal como lo había hecho la familia de Acab. Pues después de la muerte de
su padre ellos fueron sus consejeros, para su ruina.

\hypertarget{su-pacto-con-joram-de-israel-y-su-muerte-por-jehuxfa}{%
\subsection{Su pacto con Joram de Israel y su muerte por
Jehú}\label{su-pacto-con-joram-de-israel-y-su-muerte-por-jehuxfa}}

\bibleverse{5} También siguió su consejo al unirse a Joram, hijo de
Acab, rey de Israel, para atacar a Hazael, rey de Aram, en Ramot de
Galaad. Los arameos hirieron a Joram, \bibleverse{6} y éste regresó a
Jezreel para recuperarse de las heridas que había recibido en Ramá
luchando contra Hazael, rey de Aram. Ocozías, hijo de Joram, rey de
Judá, fue a Jezreel a visitar a Joram, hijo de Acab, porque éste estaba
herido.

\bibleverse{7} La caída de Azarías vino de Dios cuando fue a ver a
Joram. Cuando Azarías llegó allí, fue con Joram a encontrarse con Jehú,
hijo\footnote{\textbf{22:7} En realidad ``nieto''. Su padre era Josafat.}
de Nimri. El Señor había ungido a Jehú para que destruyera a Acab y a su
familia. \footnote{\textbf{22:7} 1Re 19,16; 2Re 9,6} \bibleverse{8}
Mientras Jehú llevaba a cabo el juicio sobre la familia de Acab, se
encontró con los líderes de Judá y los parientes de Azarías\footnote{\textbf{22:8}
  ``Parientes de Azarías'': Lectura de la Septuaginta. Véase también 2
  Reyes 10:13. Hebreo: ``hijos de los hermanos de Ocozías''.} que
ayudaban a Ocozías, y los mató. \footnote{\textbf{22:8} 2Re 10,12-14}
\bibleverse{9} Entonces Jehú fue en busca de Ocozías. Sus hombres lo
encontraron en Samaria y lo capturaron, y lo llevaron a Jehú, donde lo
mataron. Lo enterraron, pues dijeron: ``Es el nieto de Josafat, que se
comprometió completamente a seguir al Señor''. No quedó nadie de la
familia de Ocozías para gobernar el reino. \footnote{\textbf{22:9} 2Re
  9,27-29}

\hypertarget{el-robo-y-el-asesinato-de-ataluxeda-rescate-de-jouxe1s}{%
\subsection{El robo y el asesinato de Atalía; Rescate de
Joás}\label{el-robo-y-el-asesinato-de-ataluxeda-rescate-de-jouxe1s}}

\bibleverse{10} Cuando la madre de Ocozías se enteró de que su hijo
había muerto, procedió a matar a todos los que quedaban de la familia
real de Judá. \bibleverse{11} Pero Josabet, hija del rey Joram, agarró a
Joás, hijo de Ocozías, y lo apartó de los hijos del rey que estaban a
punto de ser asesinados, y lo colocó a él y a su nodriza en un
dormitorio. Como Josabet, hija del rey Joram y esposa del sacerdote
Joiada, era hermana de Ocozías, escondió a Joás de Atalía para que no
pudiera matarlo. \bibleverse{12} Y mantuvieron a Joás escondido con
ellos en el Templo de Dios durante seis años, mientras Atalía gobernaba
el país.

\hypertarget{la-conspiraciuxf3n-de-joiada}{%
\subsection{La conspiración de
Joiada}\label{la-conspiraciuxf3n-de-joiada}}

\hypertarget{section-22}{%
\section{23}\label{section-22}}

\bibleverse{1} Pero en el séptimo año, Joiada tuvo el valor de actuar.
Hizo un compromiso con los comandantes de centenas: Azarías, hijo de
Jeroham, Ismael, hijo de Johanán, Azarías, hijo de Obed, Maasías, hijo
de Adaías, y Elisafat, hijo de Zicri. \bibleverse{2} Recorrieron todo
Judá y reunieron a los levitas de todas las ciudades de Judá y a los
jefes de familia de Israel. Cuando llegaron a Jerusalén, \bibleverse{3}
se reunieron todos en el Templo de Dios e hicieron un acuerdo solemne
con el rey. Joiada les anunció: ``Miren, aquí está el hijo del rey y
debe reinar, tal como el Señor prometió que lo harían los descendientes
de David. \bibleverse{4} Esto es lo que tienen que hacer. Un tercio de
ustedes, sacerdotes y levitas, que entran en sábado, vigilarán las
entradas. \bibleverse{5} Otro tercio irá al palacio del rey, mientras
que el último tercio estará en la Puerta de los Cimientos. Todos los
demás se quedarán en los patios del Templo del Señor. \bibleverse{6}
Nadie debe entrar en el Templo del Señor, excepto los sacerdotes y los
levitas que estén sirviendo. Ellos pueden entrar porque han sido
santificados, pero todos los demás deben seguir los mandatos del Señor.
\bibleverse{7} Los levitas rodearán al rey, con las armas en la mano.
Maten a cualquiera que entre en el Templo. Permanezcan cerca del rey
dondequiera que vaya''.

\hypertarget{captura-y-asesinato-de-athalja-elevaciuxf3n-de-jouxe1s-a-rey}{%
\subsection{Captura y asesinato de Athalja; Elevación de Joás a
rey}\label{captura-y-asesinato-de-athalja-elevaciuxf3n-de-jouxe1s-a-rey}}

\bibleverse{8} Los levitas y todo el pueblo de Judá hicieron todo lo que
les dijo el sacerdote Joiada. Los comandantes trajeron cada uno sus
hombres, tanto los que entraban en servicio el sábado como los que
salían de servicio, pues el sacerdote Joiada no había despedido a
ninguna de las divisiones. \bibleverse{9} El sacerdote Joiada
proporcionó a los comandantes las lanzas y los escudos grandes y
pequeños del rey David que estaban en el Templo de Dios. \bibleverse{10}
Los colocó a todos, con sus armas en la mano, para rodear al rey desde
el lado sur del Templo hasta el lado norte, y cerca del altar y del
Templo. \bibleverse{11} Entonces Joiada y sus hijos sacaron al hijo del
rey, le pusieron la corona y le entregaron un ejemplar de la ley de
Dios,\footnote{\textbf{23:11} ``La ley de Dios'': Literalmente,
  ``testimonio''.} y lo proclamaron rey. Lo ungieron y gritaron: ``¡Viva
el rey!''.

\bibleverse{12} Cuando Atalía oyó el ruido de la gente que corría y
gritaba alabanzas al rey, se apresuró a acercarse a la multitud en el
Templo del Señor. \bibleverse{13} Vio al rey de pie junto a su columna
en la entrada. Los comandantes y los trompetistas estaban con el rey, y
todos celebraban y tocaban las trompetas mientras los cantantes con
instrumentos musicales dirigían las alabanzas. Atalía se rasgó las
vestiduras y gritó: ``¡Traición! Traición!''

\bibleverse{14} Joiada ordenó a los comandantes del ejército: ``Llévenla
ante los hombres que están frente al Templo y maten a cualquiera que la
siga''. Antes, el sacerdote había dejado claro: ``No deben matarla en el
Templo del Señor''. \bibleverse{15} La agarraron y la llevaron a la
entrada de la Puerta de los Caballos del palacio del rey, y allí la
mataron.

\hypertarget{medidas-de-joiada-para-la-gloria-de-dios-coronaciuxf3n-de-jouxe1s}{%
\subsection{Medidas de Joiada para la gloria de Dios; Coronación de
Joás}\label{medidas-de-joiada-para-la-gloria-de-dios-coronaciuxf3n-de-jouxe1s}}

\bibleverse{16} Entonces Joiada hizo un acuerdo solemne entre él y todo
el pueblo y el rey de que serían el pueblo del Señor. \bibleverse{17}
Todos fueron al templo de Baal y derribaron sus altares y destrozaron
los ídolos. Mataron a Matán, el sacerdote de Baal, delante del altar.
\bibleverse{18} Joiada puso la responsabilidad del Templo del Señor en
manos de los sacerdotes levitas. Ellos eran los que David había
designado para que estuvieran a cargo del Templo del Señor y ofrecieran
holocaustos al Señor, como lo exige la Ley de Moisés, con celebraciones
y cantos, según las instrucciones de David. \footnote{\textbf{23:18}
  2Cró 29,30}

\bibleverse{19} Puso porteros en las entradas del Templo del Señor, para
que no entrara nadie impuro por ningún motivo. \bibleverse{20} Junto con
los comandantes, los nobles, los gobernantes del pueblo y todo el
pueblo, hizo bajar al rey en procesión desde el Templo del Señor por la
puerta superior hasta el palacio real. Allí colocaron al rey en el trono
real. \bibleverse{21} En todo el país la gente celebró, y Jerusalén
estaba en paz, porque Atalía había sido muerta a espada.

\hypertarget{el-gobierno-del-rey-jouxe1s}{%
\subsection{El gobierno del rey
Joás}\label{el-gobierno-del-rey-jouxe1s}}

\hypertarget{section-23}{%
\section{24}\label{section-23}}

\bibleverse{1} Joás tenía siete años cuando llegó a ser rey, y reinó en
Jerusalén durante cuarenta años. Su madre se llamaba Sibia de Beerseba.
\bibleverse{2} Joás hizo lo que era correcto a los ojos del Señor
durante la vida del sacerdote Joiada. \bibleverse{3} Joiada hizo que se
casara con dos mujeres, y tuvo hijos e hijas.

\hypertarget{reparando-el-templo-ordenanza-sobre-la-administraciuxf3n-y-el-uso-del-dinero-entrante-para-el-templo}{%
\subsection{Reparando el templo; Ordenanza sobre la administración y el
uso del dinero entrante para el
templo}\label{reparando-el-templo-ordenanza-sobre-la-administraciuxf3n-y-el-uso-del-dinero-entrante-para-el-templo}}

\bibleverse{4} Tiempo después, Joás decidió reparar el Templo del Señor.
\bibleverse{5} Convocó a los sacerdotes y a los levitas y les dijo:
``Vayan a las ciudades de Judá y recojan las cuotas anuales de todos en
Israel para reparar el Templo de su Dios. Háganlo de inmediato''. Pero
los levitas no fueron de inmediato. \bibleverse{6} Entonces el rey llamó
al sumo sacerdote Joiada y le preguntó: ``¿Por qué no has ordenado a los
levitas que recauden de Judá y Jerusalén el impuesto que Moisés, siervo
del Señor, y la asamblea de Israel impusieron para mantener la Tienda de
la Ley?'' \footnote{\textbf{24:6} ``Tienda de la Ley'': o, ``Tienda del
  Testimonio''.} \footnote{\textbf{24:6} Éxod 30,12-13} \bibleverse{7}
(Los partidarios de esa malvada mujer, Atalía, habían irrumpido en el
Templo de Dios y habían robado los objetos sagrados del Templo del Señor
y los habían utilizado para adorar a los baales).\footnote{\textbf{24:7}
  ``Baales'': diferentes dioses paganos.} \footnote{\textbf{24:7} 2Cró
  22,3-4}

\bibleverse{8} El rey ordenó que se hiciera un cofre para la colecta y
que se colocara frente a la entrada del Templo del Señor. \bibleverse{9}
Se proclamó un decreto en toda Judea y Jerusalén para traer al Señor el
impuesto que Moisés, el siervo del Señor, impuso a Israel en el
desierto. \footnote{\textbf{24:9} 2Cró 24,6} \bibleverse{10} Todos los
dirigentes y todo el pueblo se alegraron de hacerlo y trajeron sus
impuestos. Los echaron en el cofre hasta que estuvo lleno.
\bibleverse{11} De vez en cuando los levitas llevaban el cofre a los
funcionarios del rey. Cuando veían que contenía una gran cantidad de
dinero, el secretario del rey y el oficial principal del sumo sacerdote
venían y vaciaban el cofre. Luego lo llevaban de vuelta a su lugar. Lo
hacían todos los días y recogían una gran cantidad de dinero.
\bibleverse{12} Luego el rey y Joiada destinaban el dinero de los que
supervisaban las obras del Templo del Señor a contratar canteros y
carpinteros para restaurar el Templo del Señor y artesanos del hierro y
del bronce para reparar el Templo del Señor. \bibleverse{13} Los hombres
que hacían las reparaciones trabajaron duro y avanzaron mucho.
Restauraron el Templo de Dios a su condición original y lo
fortalecieron. \bibleverse{14} Cuando terminaron, devolvieron el dinero
que quedaba al rey y a Joiada, y con él se hicieron utensilios para el
Templo del Señor, tanto para los servicios de adoración como para los
holocaustos, también copas para el incienso y recipientes de oro y
plata. Los holocaustos se ofrecían regularmente en el Templo del Señor
durante toda la vida de Joiada.

\hypertarget{el-alejamiento-de-jouxe1s-de-dios-despuuxe9s-de-la-muerte-de-joiada-el-discurso-de-zachuxe2ruxedas-y-su-lapidaciuxf3n}{%
\subsection{El alejamiento de Joás de Dios después de la muerte de
Joiada; El discurso de Zachârías y su
lapidación}\label{el-alejamiento-de-jouxe1s-de-dios-despuuxe9s-de-la-muerte-de-joiada-el-discurso-de-zachuxe2ruxedas-y-su-lapidaciuxf3n}}

\bibleverse{15} Joiada envejeció y murió a la edad de 130 años, habiendo
vivido una vida plena. \bibleverse{16} Fue enterrado con los reyes en la
Ciudad de David, por todo el bien que había hecho en Israel por Dios y
su Templo.

\bibleverse{17} Pero después de la muerte de Joiada, los líderes de Judá
vinieron a jurar su lealtad al rey, y él escuchó sus consejos.
\bibleverse{18} Abandonaron el Templo del Señor, el Dios de sus
antepasados, y adoraron postes de Asera e ídolos. Judá y Jerusalén
fueron castigados por su pecado. \bibleverse{19} El Señor envió a los
profetas para que hicieran volver al pueblo a él y les advirtieran, pero
ellos se negaron a escuchar.

\bibleverse{20} Entonces el Espíritu de Dios vino a Zacarías, hijo del
sacerdote Joiada. Se puso de pie ante el pueblo y les dijo: ``Esto es lo
que dice Dios: `¿Por qué quebrantan los mandamientos del Señor para no
tener éxito? Siendo que han abandonado al Señor, él los ha abandonado a
ustedes'\,''.

\bibleverse{21} Entonces los dirigentes tramaron un complot para matar a
Zacarías, y por orden del rey lo apedrearon hasta la muerte en el patio
del Templo del Señor. \bibleverse{22} El rey Joás demostró que había
olvidado la lealtad y el amor que le había demostrado Joiada, el padre
de Zacarías, al matar a su hijo. Al morir, Zacarías gritó: ``¡Que el
Señor vea lo que has hecho y te lo pague!''.

\hypertarget{guerra-desafortunada-con-los-sirios-asesinato-del-rey-por-conspiradores-palabra-final}{%
\subsection{Guerra desafortunada con los sirios; Asesinato del rey por
conspiradores; Palabra
final}\label{guerra-desafortunada-con-los-sirios-asesinato-del-rey-por-conspiradores-palabra-final}}

\bibleverse{23} Al final del año, el ejército arameo vino a atacar a
Joás. Invadieron Judá y Jerusalén y mataron a todos los líderes del
pueblo, y enviaron todo su botín al rey de Damasco. \bibleverse{24}
Aunque el ejército arameo había llegado con pocos hombres, el Señor les
dio la victoria sobre un ejército muy numeroso, porque Judá había
abandonado al Señor, el Dios de sus antepasados. De esta manera
castigaron a Joás.

\bibleverse{25} Cuando los arameos se fueron, dejaron a Joás malherido.
Pero entonces sus propios oficiales conspiraron contra él por haber
asesinado al hijo del sacerdote Joiada, y lo mataron en su lecho. Lo
enterraron en la Ciudad de David, pero no en el cementerio de los reyes.
\footnote{\textbf{24:25} 2Cró 21,20}

\bibleverse{26} Los que conspiraron contra él fueron Zabad, hijo de
Simeat, una mujer amonita, y Jozabad, hijo de Simrit, una mujer moabita.
\bibleverse{27} La historia de los hijos de Joás, así como las numerosas
profecías sobre él y sobre la restauración del Templo de Dios, se
recogen en el Comentario al Libro de los Reyes. Posteriormente su hijo
Amasías le sucedió como rey.

\hypertarget{el-gobierno-del-rey-amasuxedas-buen-comienzo-para-el-gobierno}{%
\subsection{El gobierno del rey Amasías; Buen comienzo para el
gobierno}\label{el-gobierno-del-rey-amasuxedas-buen-comienzo-para-el-gobierno}}

\hypertarget{section-24}{%
\section{25}\label{section-24}}

\bibleverse{1} Amasías tenía veinticinco años cuando llegó a ser rey, y
reinó en Jerusalén durante veintinueve años. Su madre se llamaba Joadán
y era de Jerusalén. \bibleverse{2} E hizo lo que era correcto a los ojos
del Señor, pero no con total compromiso. \bibleverse{3} Después de
asegurarse de que su gobierno era seguro, ejecutó a los oficiales que
habían asesinado a su padre el rey. \bibleverse{4} Sin embargo, no mató
a sus hijos, como está escrito en la Ley, en el libro de Moisés, donde
el Señor ordenó ``Los padres no deben ser ejecutados por sus hijos, y
los hijos no deben ser ejecutados por sus padres. Cada uno debe morir
por su propio pecado''. \footnote{\textbf{25:4} Deut 24,16}

\hypertarget{la-victoria-de-amasuxedas-sobre-los-edomitas-despuuxe9s-de-que-los-mercenarios-israelitas-fueran-devueltos-la-venganza-de-estas-tropas}{%
\subsection{La victoria de Amasías sobre los edomitas después de que los
mercenarios israelitas fueran devueltos; la venganza de estas
tropas}\label{la-victoria-de-amasuxedas-sobre-los-edomitas-despuuxe9s-de-que-los-mercenarios-israelitas-fueran-devueltos-la-venganza-de-estas-tropas}}

\bibleverse{5} Entonces Amasías convocó al pueblo de Judá para el
servicio militar, y los asignó por familias a comandantes de millares y
de centenas. También hizo un censo de los mayores de veinte años en todo
Judá y Benjamín, y descubrió que había 300. 000 hombres de combate de
primera categoría que sabían usar la lanza y el escudo. \bibleverse{6}
También contrató a 100. 000 combatientes de Israel listos para la
batalla por cien talentos de plata. \bibleverse{7} Pero un hombre de
Dios se le acercó y le dijo: ``¡Majestad, no permita que este ejército
de Israel se una a usted, porque el Señor no está con Israel, con estos
hijos de Efraín! \bibleverse{8} Aunque luchen con valentía, Dios los
hará tropezar y caer ante el enemigo, pues Dios tiene el poder de
ayudarlos o dejarlos caer''.

\bibleverse{9} Amasías le preguntó al hombre de Dios: ``¿Pero qué hay de
los cien talentos de plata que pagué al ejército de Israel?'' ``¡El
Señor puede darte mucho más que eso!'', respondió el hombre de Dios.

\bibleverse{10} Entonces Amasías despidió al ejército que había
contratado de Efraín y lo envió a su casa. Ellos se enojaron mucho con
Judá, y regresaron a casa furiosos.

\bibleverse{11} Entonces Amasías condujo valientemente su ejército al
Valle de la Sal, donde atacaron al ejército edomita de Seír y mataron a
diez mil de ellos. \bibleverse{12} El ejército de Judá también capturó a
otros diez mil, los llevó a la cima de un acantilado y los arrojó al
vacío, matándolos a todos. \bibleverse{13} Pero los hombres del ejército
que Amasías envió a casa, negándose a dejarlos ir con él a la batalla,
asaltaron las ciudades de Judá, desde Samaria hasta Bet-horón. Mataron a
tres mil de sus habitantes y se llevaron una gran cantidad de botín.

\hypertarget{amasuxedas-se-aparta-de-dios-advertencia-de-un-profeta}{%
\subsection{Amasías se aparta de Dios; Advertencia de un
profeta}\label{amasuxedas-se-aparta-de-dios-advertencia-de-un-profeta}}

\bibleverse{14} Cuando Amasías regresó de matar a los edomitas, trajo
los dioses de los pueblos de Seír y los erigió como sus propios dioses,
los adoró y les ofreció sacrificios. \bibleverse{15} El Señor se enojó
con Amasías y le envió un profeta que le dijo: ``¿Por qué adoras a los
dioses de un pueblo que ni siquiera pudo salvar a su propio pueblo de
ti?''

\bibleverse{16} Pero mientras él seguía hablando, el rey le dijo: ``¿Te
hemos hecho consejero del rey? ¡Detente ahora mismo! ¿Quieres que te
derriben?'' Entonces el profeta se detuvo, pero dijo: ``Sé que Dios ha
decidido destruirte, porque has actuado así y te has negado a escuchar
mis consejos''.

\hypertarget{la-desafortunada-guerra-de-amasuxedas-con-jouxe1s-de-israel}{%
\subsection{La desafortunada guerra de Amasías con Joás de
Israel}\label{la-desafortunada-guerra-de-amasuxedas-con-jouxe1s-de-israel}}

\bibleverse{17} Entonces Amasías, rey de Judá, se dejó aconsejar por sus
consejeros y envió un mensaje al rey de Israel, Joás, hijo de Joacaz,
hijo de Jehú. ``Ven y enfréntate a mí en la batalla'', lo desafió.

\bibleverse{18} Joás, rey de Israel, respondió a Amasías, rey de Judá:
``Un cardo del Líbano envió un mensaje a un cedro del Líbano, diciendo:
`Dale tu hija a mi hijo por esposa', pero un animal salvaje del Líbano
pasó y pisoteó el cardo. \bibleverse{19} Te dices a ti mismo lo grande
que eres por haber derrotado a Edom, presumiendo de ello. Pero quédate
en casa. ¿Por qué has de suscitar problemas que te harán caer a ti y a
Judá contigo?''

\bibleverse{20} Pero Amasías no escuchó, pues Dios iba a entregarlo a
sus enemigos porque había elegido adorar a los dioses de Edom.
\bibleverse{21} Así que Joás, rey de Israel, se preparó para la batalla.
Él y Amasías, rey de Judá, se enfrentaron en Bet-semes, en Judá.
\bibleverse{22} Judá fue derrotado por Israel; todos huyeron a casa.

\bibleverse{23} Joás, rey de Israel, capturó a Amasías, rey de Judá,
hijo de Joás, hijo de Ocozías, en Bet Semes. Lo llevó a Jerusalén y
derribó la muralla de Jerusalén a lo largo de 400 codos, desde la Puerta
de Efraín hasta la Puerta de la Esquina. \bibleverse{24} Se llevó todo
el oro y la plata, y todos los artículos encontrados en el Templo de
Dios que habían sido cuidados por Obed-edom y en los tesoros del palacio
real, así como algunos rehenes, y luego regresó a Samaria.

\hypertarget{palabra-final-asesinato-del-rey-por-conspiradores}{%
\subsection{Palabra final; Asesinato del rey por
conspiradores}\label{palabra-final-asesinato-del-rey-por-conspiradores}}

\bibleverse{25} Amasías, hijo de Joás, rey de Judá, vivió quince años
después de la muerte de Joás, hijo de Joacaz, rey de Israel.
\bibleverse{26} El resto de lo que hizo Amasías, de principio a fin,
está escrito en el Libro de los Reyes de Judá e Israel. \bibleverse{27}
Después de que Amasías renunció a seguir al Señor, se tramó un complot
contra él en Jerusalén, y huyó a Laquis. Pero los conspiradores enviaron
hombres a Laquis para perseguirlo, y allí lo mataron. \footnote{\textbf{25:27}
  2Cró 24,25}

\bibleverse{28} Lo trajeron de vuelta a caballo y lo enterraron con sus
padres en la ciudad de Judá.

\hypertarget{el-gobierno-del-rey-ussia-buen-comienzo-para-el-gobierno-la-felicidad-de-ussia-en-la-guerra-y-la-paz}{%
\subsection{El gobierno del rey Ussia; Buen comienzo para el gobierno;
La felicidad de Ussia en la guerra y la
paz}\label{el-gobierno-del-rey-ussia-buen-comienzo-para-el-gobierno-la-felicidad-de-ussia-en-la-guerra-y-la-paz}}

\hypertarget{section-25}{%
\section{26}\label{section-25}}

\bibleverse{1} Todo el pueblo de Judá tomó a Uzías, de dieciséis años, y
lo nombró rey en sucesión de su padre Amasías. \bibleverse{2} Él
reconstruyó Elot y la devolvió al reino de Judá después de la muerte de
Amasías. \bibleverse{3} Uzías tenía dieciséis años cuando llegó a ser
rey, y reinó en Jerusalén durante cincuenta y dos años. Su madre se
llamaba Jecolías y era de Jerusalén. \bibleverse{4} Hizo lo que era
correcto a los ojos del Señor, como lo había hecho su padre Amasías.
\footnote{\textbf{26:4} 2Cró 25,2} \bibleverse{5} Adoró a Dios durante
la vida de Zacarías, quien le enseñó a respetar a Dios. Mientras siguió
al Señor, Dios le dio éxito.

\bibleverse{6} Uzías fue a la guerra contra los filisteos, y derribó las
murallas de Gat, Jabne y Asdod. Luego construyó ciudades alrededor de
Asdod y en otras zonas filisteas. \bibleverse{7} Dios lo ayudó contra
los filisteos, contra los árabes que vivían en Gurbaal y contra los
meunitas. \bibleverse{8} Los meunitas\footnote{\textbf{26:8}
  ``Meunitas'': Lectura de la Septuaginta. El hebreo tiene ``amonitas''.}
trajeron regalos como tributo a Uzías. Su fama se extendió hasta la
frontera de Egipto, pues llegó a ser muy poderoso. \bibleverse{9} Uzías
construyó torres defensivas en Jerusalén, en la Puerta de la Esquina y
en la Puerta del Valle, y en la esquina, y las reforzó. \bibleverse{10}
También construyó torres en el desierto y cortó muchas cisternas de agua
en la roca, porque tenía mucho ganado en las colinas y en las llanuras.
Tenía agricultores y viñadores en las colinas y en las tierras bajas
fértiles, porque amaba la tierra.

\hypertarget{la-preocupaciuxf3n-de-ussia-por-un-ejuxe9rcito-capaz-y-por-la-seguridad-del-pauxeds}{%
\subsection{La preocupación de Ussia por un ejército capaz y por la
seguridad del
país}\label{la-preocupaciuxf3n-de-ussia-por-un-ejuxe9rcito-capaz-y-por-la-seguridad-del-pauxeds}}

\bibleverse{11} Uzías tenía un ejército de soldados listos para la
batalla, en divisiones según los números de la lista hecha por el
secretario Jeiel y el funcionario Maasías, bajo la dirección de
Hananías, uno de los comandantes del rey. \bibleverse{12} El número
total de jefes de familia era de 2. 600 combatientes. \bibleverse{13}
Bajo su mando había un ejército de 307. 500 entrenados para la batalla,
que tenían el poder de ayudar al rey a luchar contra el enemigo.
\bibleverse{14} Uzías suministró escudos, lanzas, cascos, armaduras,
arcos y hondas para todo el ejército. \bibleverse{15} También fabricó
máquinas de guerra hábilmente diseñadas para disparar flechas y grandes
piedras desde las torres y las esquinas de la muralla. Su fama se
extendió por todas partes, pues recibió una ayuda extraordinaria hasta
que llegó a ser realmente poderoso.

\hypertarget{la-invasiuxf3n-de-ussia-al-sacerdocio-es-castigada-por-dios-con-lepra}{%
\subsection{La invasión de Ussia al sacerdocio es castigada por Dios con
lepra}\label{la-invasiuxf3n-de-ussia-al-sacerdocio-es-castigada-por-dios-con-lepra}}

\bibleverse{16} Pero por ser poderoso se volvió arrogante, y esto lo
llevó a la ruina. Porque fue infiel al Señor, su Dios, y él mismo entró
en el Templo del Señor para quemar incienso en el altar del incienso.
\bibleverse{17} El sacerdote Azarías entró tras él, con ochenta
valientes sacerdotes del Señor. \bibleverse{18} Se enfrentaron a él y le
dijeron: ``No es tu lugar quemar incienso al Señor. Sólo los sacerdotes,
los descendientes de Aarón, que han sido apartados como santos, pueden
quemar incienso. Sal del santuario, porque has pecado, y el Señor Dios
no te bendecirá''. \footnote{\textbf{26:18} Núm 18,7}

\bibleverse{19} Uzías, que tenía un incensario en la mano para ofrecer
incienso, se puso furioso. Pero mientras se enfurecía con los sacerdotes
en el Templo del Señor, frente al altar del incienso, le apareció la
lepra en la frente. \bibleverse{20} Cuando el sumo sacerdote Azarías y
todos los sacerdotes lo miraron y vieron la lepra en su frente, lo
sacaron corriendo. De hecho, él también tenía prisa por salir, porque el
Señor lo había golpeado. \bibleverse{21} El rey Uzías fue leproso hasta
el día de su muerte. Vivió solo como leproso, sin poder entrar en el
Templo del Señor, mientras su hijo Jotam se encargaba de los asuntos del
rey y gobernaba el país.

\hypertarget{muerte-y-entierro-de-ussia}{%
\subsection{Muerte y entierro de
Ussia}\label{muerte-y-entierro-de-ussia}}

\bibleverse{22} El resto de lo que hizo Uzías, desde el principio hasta
el final, fue escrito por el profeta Isaías, hijo de Amoz. \footnote{\textbf{26:22}
  2Re 15,5-7; Is 1,1; Is 6,1}

\bibleverse{23} Uzías murió y fue enterrado cerca de ellos en un
cementerio de los reyes, porque la gente decía: ``Era un leproso''. Su
hijo Jotam tomó el relevo coamo rey.

\hypertarget{el-gobierno-del-rey-jotam-gobierno-bueno-y-feliz-edificios-y-guerras-exitosas}{%
\subsection{El gobierno del rey Jotam; Gobierno bueno y feliz; Edificios
y guerras
exitosas}\label{el-gobierno-del-rey-jotam-gobierno-bueno-y-feliz-edificios-y-guerras-exitosas}}

\hypertarget{section-26}{%
\section{27}\label{section-26}}

\bibleverse{1} Jotam tenía veinticinco años cuando llegó a ser rey, y
reinó por Jerusalén dieciséis años. Su madre se llamaba Jerusá, hija de
Sadoc. \bibleverse{2} Hizo lo que era correcto a los ojos de Jehová,
como lo había hecho su padre Uzías, y no entró en el Templo de Jehová
como lo había hecho su padre.\footnote{\textbf{27:2} ``Como lo había
  hecho su padre'': implícito.} Pero el pueblo seguía actuando de forma
corrupta. \bibleverse{3} Jotam reconstruyó la Puerta Superior del Templo
del Señor e hizo grandes obras en la muralla de Ofel. \bibleverse{4}
Construyó ciudades en la región montañosa de Judá, y fortalezas y torres
en las montañas y los bosques. \footnote{\textbf{27:4} 2Cró 26,10}

\bibleverse{5} Jotam luchó con el rey de los amonitas y los derrotó. Los
amonitas le dieron cada año durante tres años cien talentos de plata, y
diez mil coros de trigo y diez mil de cebada. \bibleverse{6} Jotam se
hizo poderoso porque se aseguró de que lo que hacía seguía los caminos
del Señor, su Dios. \bibleverse{7} El resto de lo que hizo Jotam, sus
guerras y otros sucesos, fueron escritos en el Libro de los Reyes de
Israel y Judá. \bibleverse{8} Tenía veinticinco años cuando llegó a ser
rey, y reinó en Jerusalén durante dieciséis años. \bibleverse{9} Jotam
murió y fue enterrado en la Ciudad de David. Su hijo Acaz tomó el relevo
como rey.

\hypertarget{el-reinado-del-rey-acaz-las-abominaciones-paganas-de-acaz}{%
\subsection{El reinado del rey Acaz; Las abominaciones paganas de
Acaz}\label{el-reinado-del-rey-acaz-las-abominaciones-paganas-de-acaz}}

\hypertarget{section-27}{%
\section{28}\label{section-27}}

\bibleverse{1} Acaz tenía veinte años cuando se convirtió en rey, y
reinó en Jerusalén durante dieciséis años. No hizo lo que era correcto a
los ojos del Señor como lo había hecho su antepasado David.
\bibleverse{2} Siguió el ejemplo de los reyes de Israel y también fundió
ídolos de metal para adorar a los baales. \bibleverse{3} Quemó
sacrificios en el valle de Ben Hinnom y sacrificó a sus hijos en el
fuego, siguiendo las prácticas repugnantes de los pueblos que el Señor
había expulsado antes de los israelitas. \footnote{\textbf{28:3} Deut
  18,9-10; Deut 18,12} \bibleverse{4} Presentó sacrificios y quemó
ofrendas de incienso en los lugares altos, en las cimas de las montañas
y debajo de todo árbol viviente. \footnote{\textbf{28:4} 1Re 14,23}

\hypertarget{visitaciones-severas-de-sirios-e-israelitas}{%
\subsection{Visitaciones severas de sirios e
israelitas}\label{visitaciones-severas-de-sirios-e-israelitas}}

\bibleverse{5} Como resultado, el Señor, su Dios, permitió que el rey de
Aram conquistara a Acaz. Los arameos lo atacaron y capturaron a muchos
de su pueblo, llevándolos a Damasco. Acaz también fue derrotado por el
rey de Israel en un ataque masivo. \bibleverse{6} En un solo día, Peka,
hijo de Remalías, mató a 120. 000 combatientes en Judá, porque habían
abandonado al Señor, el Dios de sus padres. \bibleverse{7} Zicri, un
guerrero de Efraín, mató a Maasías, el hijo del rey; a Azricam, el
gobernador del palacio, y a Elcana, el segundo al mando del rey.
\bibleverse{8} Los israelitas capturaron a 200. 000 de sus
``hermanos''\footnote{\textbf{28:8} ``Hermanos'': el texto dice en
  realidad hermanos, para señalar que los pueblos de Israel y Judá
  estaban emparentados entre sí.} ---mujeres, hijos e hijas. También
tomaron una gran cantidad de botín y lo llevaron a Samaria.

\hypertarget{liberaciuxf3n-de-los-prisioneros-de-guerra-de-judea-en-samaria-siguiendo-la-amonestaciuxf3n-del-profeta-oded}{%
\subsection{Liberación de los prisioneros de guerra de Judea en Samaria
siguiendo la amonestación del profeta
Oded}\label{liberaciuxf3n-de-los-prisioneros-de-guerra-de-judea-en-samaria-siguiendo-la-amonestaciuxf3n-del-profeta-oded}}

\bibleverse{9} Pero un profeta del Señor llamado Oded estaba allí en
Samaria, y salió al encuentro del ejército que regresaba. Les dijo:
``Fue porque el Señor, el Dios de sus padres, estaba enojado con Judá
que permitió que ustedes los derrotaran. Pero ustedes los han matado con
tal furia que ha trastornado el cielo. \footnote{\textbf{28:9} Gén
  18,21; Esd 9,6} \bibleverse{10} Ahora planeas convertir a esta gente
de Judá y Jerusalén en esclavos. ¿Pero no eres tú también culpable de
pecar contra el Señor, tu Dios? \bibleverse{11} ¡Escúchame! Devuelve los
prisioneros que has tomado de tus hermanos, ¡la feroz ira del Señor está
cayendo sobre ti!'' \bibleverse{12} Algunos de los líderes del pueblo de
Efraín\footnote{\textbf{28:12} ``Efraín'': significa Israel.}
---Azarías, hijo de Johanán, Berequías, hijo de Meshillemot, Jehizcías,
hijo de Salum, y Amasa, hijo de Hadlai, se opusieron a los que
regresaban de la guerra. \bibleverse{13} ``¡No traigan a esos
prisioneros aquí!'', les dijeron. ``Si lo hacen, no lograrán más que
aumentar nuestros pecados y nuestra maldad contra el Señor. Nuestra
culpa ya es grande, y su feroz ira está cayendo sobre Israel''.

\bibleverse{14} Así que los hombres armados dejaron los prisioneros y el
botín ante los líderes y todo el pueblo allí reunido. \bibleverse{15}
Los hombres mencionados se levantaron y llevaron ropa del botín a los
que no tenían, les dieron sandalias para que se las pusieran, y comida y
bebida, y les pusieron aceite de oliva en las heridas. A los que ya no
podían caminar los montaron en burros, y los llevaron a todos a Jericó,
la ciudad de las palmeras, para que estuvieran cerca del pueblo de
Judá.\footnote{\textbf{28:15} ``El pueblo de Judá'' se añadió para mayor
  claridad. El hebreo dice simplemente ``sus hermanos''.}

\hypertarget{fuertes-visitaciones-de-los-edomitas-filisteos-y-asirios}{%
\subsection{Fuertes visitaciones de los edomitas, filisteos y
asirios}\label{fuertes-visitaciones-de-los-edomitas-filisteos-y-asirios}}

\bibleverse{16} Fue entonces cuando el rey Acaz pidió ayuda al rey de
Asiria. \bibleverse{17} Los ejércitos de Edom habían invadido de nuevo
Judá y habían hecho prisioneros a sus habitantes, \bibleverse{18}
mientras que los filisteos habían atacado las ciudades del pie de monte
y del Néguev de Judá. Habían capturado y ocupado Bet-semesh, Aijalón,
Gederoth, junto con Soco, Timnah y Gimzo y sus aldeas. \bibleverse{19}
El Señor había hecho caer a Judá porque Acaz, rey de Israel, estaba
fuera de control en Judá, pecando terriblemente contra el Señor.
\bibleverse{20} Entonces Tiglat-pileser, rey de Asiria, vino a Acaz,
pero lo atacó en vez de ayudarlo. \bibleverse{21} Acaz tomó lo que era
valioso del Templo del Señor, del palacio del rey y de sus funcionarios
y se lo entregó al rey de Asiria como tributo. Pero no le sirvió de
nada.

\hypertarget{la-creciente-maldad-de-acaz-palabra-final}{%
\subsection{La creciente maldad de Acaz; Palabra
final}\label{la-creciente-maldad-de-acaz-palabra-final}}

\bibleverse{22} Incluso en esta época en que tenía tantos problemas, el
rey Acaz pecaba cada vez más contra el Señor. \bibleverse{23} Sacrificó
a los dioses de Damasco, cuyo ejército lo había derrotado, pues se dijo:
``Ya que los dioses de los reyes de Aram los ayudaron, les sacrificaré a
ellos para que me ayuden''. Pero esto llevó a la ruina a Acaz y a todo
Israel. \bibleverse{24} Acaz tomó los objetos sagrados del Templo del
Señor y los hizo pedazos. Encerró las puertas del Templo del Señor y
levantó altares paganos en cada esquina de Jerusalén. \bibleverse{25} En
todas las ciudades de Judá levantó lugares altos para hacer ofrendas a
dioses paganos, enojando al Señor, el Dios de sus antepasados.

\bibleverse{26} El resto de lo que hizo Acaz, de principio a fin, está
escrito en el Libro de los Reyes de Judá e Israel. \bibleverse{27} Acaz
murió y lo enterraron en la ciudad, en Jerusalén. No lo enterraron en
las tumbas de los reyes de Israel. Su hijo Ezequías tomó el relevo como
rey.\footnote{\textbf{28:27} 2Cró 21,20}

\hypertarget{el-gobierno-del-rey-ezechuxeeas-restauraciuxf3n-del-templo-y-adoraciuxf3n-pura}{%
\subsection{El gobierno del rey Ezechîas; Restauración del templo y
adoración
pura}\label{el-gobierno-del-rey-ezechuxeeas-restauraciuxf3n-del-templo-y-adoraciuxf3n-pura}}

\hypertarget{section-28}{%
\section{29}\label{section-28}}

\bibleverse{1} Ezequías tenía veinticinco años cuando se convirtió en
rey, y reinó en Jerusalén durante veintinueve años. Su madre se llamaba
Abías, hija de Zacarías. \footnote{\textbf{29:1} 2Re 18,1-3}
\bibleverse{2} Hizo lo que era correcto a los ojos del Señor, tal como
lo había hecho su antepasado David.

\hypertarget{la-exhortaciuxf3n-de-ezechuxeeas-a-los-sacerdotes-y-levitas}{%
\subsection{La exhortación de Ezechîas a los sacerdotes y
levitas}\label{la-exhortaciuxf3n-de-ezechuxeeas-a-los-sacerdotes-y-levitas}}

\bibleverse{3} En el primer mes del primer año de su reinado, Ezequías
abrió las puertas del Templo del Señor y las reparó. \bibleverse{4}
Convocó a los sacerdotes y a los levitas, y los hizo reunirse en la
plaza al este.\footnote{\textbf{29:4} Al este del Templo.}
\bibleverse{5} Les dijo: ``Escuchadme, levitas. Purifíquense ahora y
purifiquen el Templo del Señor, el Dios de sus antepasados. Quiten del
Lugar Santo todo lo que esté sucio.\footnote{\textbf{29:5} ``Suciedad'':
  no se trata tanto de limpiar, sino de eliminar todo lo relacionado con
  el culto a los ídolos.} \bibleverse{6} Porque nuestros padres fueron
pecadores e hicieron lo que era malo a los ojos del Señor. Lo
abandonaron y no prestaron atención al Templo del Señor, dándole la
espalda. \bibleverse{7} Cerraron las puertas de la entrada del Templo y
apagaron las lámparas. No quemaron incienso ni presentaron holocaustos
en el santuario del Dios de Israel. \footnote{\textbf{29:7} 2Cró 28,24}
\bibleverse{8} ``Así que la ira del Señor cayó sobre Judá y Jerusalén, y
los convirtió en algo espantoso, aterrador y ridículo, como pueden ver
ustedes mismos. \bibleverse{9} Como resultado, nuestros padres han
muerto en la batalla, y nuestros hijos, nuestras hijas y nuestras
esposas han sido capturados. \bibleverse{10} Pero ahora voy a hacer un
acuerdo con el Señor, el Dios de Israel, para que su feroz ira no caiga
más sobre nosotros. \bibleverse{11} Hijos míos, no descuiden sus
responsabilidades, porque el Señor los ha escogido para estar en su
presencia y servirle, y para ser sus ministros presentando
holocaustos''.

\hypertarget{purificaciuxf3n-del-templo-por-los-levitas}{%
\subsection{Purificación del templo por los
levitas}\label{purificaciuxf3n-del-templo-por-los-levitas}}

\bibleverse{12} Entonces los levitas se pusieron a trabajar. Eran Mahat,
hijo de Amasai, y Joel, hijo de Azarías, de los coatitas; Cis, hijo de
Abdi, y Azarías, hijo de Jehallelel, de los meraritas; Joá, hijo de
Zima, y Edén, hijo de Joá, de los gersonitas; \bibleverse{13} de los
hijos de Elizafán, Simri y Jeiel; y de los hijos de Asaf, Zacarías y
Matanías; \bibleverse{14} de los hijos de Hemán, Jehiel y Simei; y de
los hijos de Jedutún, Semaías y Uziel. \bibleverse{15} Convocaron a los
demás levitas y todos se purificaron. Luego entraron a limpiar el Templo
del Señor, tal como el rey lo había ordenado, siguiendo las
instrucciones que el Señor les exigía. \bibleverse{16} Los sacerdotes
entraron en el santuario interior del Templo del Señor para limpiarlo.
Sacaron todas las cosas impuras que encontraron en el Templo del Señor y
las colocaron en el patio del Templo. Luego los levitas las sacaron y
las llevaron al Valle del Cedrón. \bibleverse{17} Comenzaron el trabajo
de purificación el primer día del primer mes, y al octavo día del mes
habían llegado al pórtico del Templo. Durante ocho días más trabajaron
en la purificación del Templo, y terminaron el día dieciséis del primer
mes. \bibleverse{18} Entonces entraron a decirle al rey Ezequías:
``Hemos limpiado todo el Templo del Señor, el altar del holocausto con
todos sus utensilios y la mesa de los panes de la proposición con todos
sus utensilios. \bibleverse{19} Hemos recuperado y purificado todos los
objetos que el rey Acaz tiró durante su reinado cuando fue infiel. Ahora
están ante el altar del Señor''.

\hypertarget{la-nueva-consagraciuxf3n-del-templo-con-sacrificios-oraciuxf3n-y-cuxe1nticos}{%
\subsection{La nueva consagración del templo con sacrificios, oración y
cánticos}\label{la-nueva-consagraciuxf3n-del-templo-con-sacrificios-oraciuxf3n-y-cuxe1nticos}}

\bibleverse{20} El rey Ezequías se levantó temprano, convocó a los
funcionarios de la ciudad y fue al Templo del Señor. \bibleverse{21}
Trajeron siete toros, siete carneros, siete corderos y siete machos
cabríos como ofrenda por el pecado por el reino, por el santuario y por
Judá. El rey ordenó a los sacerdotes, descendientes de Aarón, que los
ofrecieran sobre el altar del Señor. \bibleverse{22} Así que mataron los
toros, y los sacerdotes tomaron la sangre y la rociaron sobre el altar.
Mataron los carneros y rociaron la sangre sobre el altar. Mataron los
corderos y rociaron la sangre sobre el altar. \bibleverse{23} Luego
trajeron los machos cabríos para la ofrenda por el pecado ante el rey y
la asamblea, quienes pusieron sus manos sobre ellos. \bibleverse{24}
Luego los sacerdotes mataron los machos cabríos y pusieron su sangre
sobre el altar como ofrenda por el pecado, para hacer expiación por todo
Israel, porque el rey había ordenado que el holocausto y la ofrenda por
el pecado fueran por todo Israel.

\bibleverse{25} Ezequías hizo que los levitas estuvieran de pie en el
Templo del Señor con címbalos, arpas y liras, siguiendo las
instrucciones de David, de Gad, el vidente del rey, y de Natán, el
profeta. Las instrucciones habían venido del Señor a través de sus
profetas. \footnote{\textbf{29:25} 1Cró 25,1} \bibleverse{26} Los
levitas estaban de pie con los instrumentos musicales proporcionados por
David, y los sacerdotes sostenían sus trompetas. \bibleverse{27}
Entonces Ezequías dio la orden de que se ofreciera el holocausto sobre
el altar. Al comenzar el holocausto, comenzó al mismo tiempo el canto
del Señor, sonaron las trompetas y se tocó música con los instrumentos
de David, el que fuera rey de Israel. \bibleverse{28} Todo el pueblo en
la asamblea adoraba, los cantantes cantaban y los trompetistas tocaban.
Esto continuó hasta que se terminó el holocausto.

\bibleverse{29} Una vez terminadas las ofrendas, el rey y todos los que
estaban allí con él se inclinaron y adoraron. \bibleverse{30} Entonces
el rey Ezequías y sus funcionarios ordenaron a los levitas que cantaran
alabanzas al Señor con las palabras de David y del vidente Asaf. Así que
cantaron alabanzas con alegría, e inclinaron la cabeza y adoraron.

\bibleverse{31} Entonces Ezequías les dijo: ``Ahora que se han
consagrado al Señor, vengan y traigan sus sacrificios y ofrendas de
agradecimiento al Templo del Señor''. Así que la gente de la asamblea
trajo sus sacrificios y ofrendas de agradecimiento, y todos los que
quisieron trajeron holocaustos. \bibleverse{32} El número total de
holocaustos que trajeron fue de setenta toros, cien carneros y
doscientos corderos; todo esto debía ser un holocausto para el Señor.
\bibleverse{33} Además había ofrendas dedicadas de seiscientos toros y
tres mil ovejas. \bibleverse{34} Como no había suficientes sacerdotes
para desollar todos los holocaustos, sus parientes levitas los ayudaban
hasta que el trabajo estaba terminado y los sacerdotes se habían
purificado. (Los levitas habían sido más concienzudos en la purificación
que los sacerdotes). \footnote{\textbf{29:34} 2Cró 30,3; 2Cró 30,16-17}
\bibleverse{35} Además del gran número de holocaustos, estaba la grasa
de las ofrendas de amistad, así como las libaciones que acompañaban a
los holocaustos. De esta manera se restauró el servicio del Templo del
Señor. \footnote{\textbf{29:35} Lev 3,16-17; Núm 15,5; Núm 15,7; Núm
  15,10}

\bibleverse{36} Ezequías y todos los presentes estaban muy contentos por
lo que Dios había hecho por el pueblo, porque todo se había logrado tan
rápidamente.

\hypertarget{celebraciuxf3n-de-la-pascua-de-ezechuxeeas}{%
\subsection{Celebración de la Pascua de
Ezechîas}\label{celebraciuxf3n-de-la-pascua-de-ezechuxeeas}}

\hypertarget{section-29}{%
\section{30}\label{section-29}}

\bibleverse{1} Entonces Ezequías envió un anuncio a todos en Israel y
Judá, y también envió cartas a Efraín y Manasés,\footnote{\textbf{30:1}
  ``Efraín y Manasés'': en esta época, el reino del norte de Israel ya
  no existía, pues había sido destruido y su pueblo había sido tomado
  como prisionero por el rey asirio Salmanar. Con su invitación,
  Ezequías hace un llamamiento a los que quedan en el norte para que
  ``vuelvan a casa''.} invitándolos a venir al Templo del Señor en
Jerusalén para celebrar la Pascua del Señor, el Dios de Israel.
\footnote{\textbf{30:1} 2Cró 35,1} \bibleverse{2} El rey y sus
funcionarios y toda la asamblea de Jerusalén habían decidido celebrar la
Pascua en el segundo mes,\footnote{\textbf{30:2} El aplazamiento de la
  Pascua en circunstancias especiales estaba permitido, véase Números
  9:6-11.} \footnote{\textbf{30:2} 2Cró 30,15} \bibleverse{3} porque no
habían podido celebrarlo a la hora habitual, ya que no se habían
purificado suficientes sacerdotes y el pueblo no había tenido tiempo de
llegar a Jerusalén. \bibleverse{4} El plan les pareció bien al rey y a
toda la asamblea. \bibleverse{5} Así que decidieron enviar un anuncio a
todo Israel, desde Beerseba hasta Dan, invitando a la gente a venir a
celebrar la Pascua al Señor, el Dios de Israel, en Jerusalén, pues
muchos no habían hecho lo que exigía la Ley.

\bibleverse{6} Así que los mensajeros fueron a todo Israel y Judá
llevando cartas del rey y de sus funcionarios y con la autorización del
rey. Decían: ``Hijos de Israel, vuelvan al Señor, el Dios de Abrahán, de
Isaac y de Israel, para que él los devuelva a ustedes, que han escapado
de la opresión de los reyes de Asiria. \bibleverse{7} No sean como sus
padres y los que pecaron contra el Señor, el Dios de sus antepasados,
que los convirtió en algo espantoso, como pueden ver. \bibleverse{8} No
sean, pues, orgullosos y obstinados como sus padres, sino entréguense al
Señor y vengan a su santuario, que él ha santificado para siempre, y
sirvan al Señor, su Dios, para que no caiga más sobre ustedes su feroz
ira. \bibleverse{9} ``Si vuelven al Señor, sus parientes e hijos
recibirán la misericordia de sus captores y volverán a esta tierra.
Porque el Señor, tu Dios, es clemente y misericordioso. No los rechazará
si vuelven a él''.

\bibleverse{10} Los mensajeros fueron de pueblo en pueblo por toda la
tierra de Efraín y Manasés hasta Zabulón; pero la gente se reía de ellos
y se burlaba. \bibleverse{11} Sólo algunos hombres de Aser, Manasés y
Zabulón no fueron demasiado orgullosos para ir a Jerusalén.
\bibleverse{12} En ese momento el poder de Dios ayudaba a que el pueblo
de Judá tuviera todos el mismo deseo de seguir las órdenes del rey y de
sus funcionarios, tal como lo indicaba la palabra del Señor.

\hypertarget{curso-de-pascua-en-la-primera-semana}{%
\subsection{Curso de Pascua en la primera
semana}\label{curso-de-pascua-en-la-primera-semana}}

\bibleverse{13} Mucha gente se reunió en Jerusalén para celebrar la
Fiesta de los Panes sin Levadura en el segundo mes, una multitud
realmente grande. \bibleverse{14} Fueron y quitaron los altares paganos
de Jerusalén, así como los altares de incienso, y los arrojaron al valle
del Cedrón. \bibleverse{15} El día catorce del segundo mes mataron el
cordero de la Pascua. Los sacerdotes y los levitas se
avergonzaron,\footnote{\textbf{30:15} ``Se avergonzaron'': quizás porque
  no habían sido suficientes en la celebración anterior, y que no se
  habían tomado en serio la responsabilidad de purificarse. Otra
  posibilidad es que la devoción mostrada por los que asistían a la
  Fiesta de los Panes sin Levadura haya impulsado a los sacerdotes y
  levitas a actuar.} y se purificaron y trajeron holocaustos al Templo
del Señor. \footnote{\textbf{30:15} Núm 9,11} \bibleverse{16} Se
colocaron en sus puestos asignados, según la ley de Moisés, el hombre de
Dios. Los sacerdotes rociaban la sangre de los sacrificios, que los
levitas les entregaban. \footnote{\textbf{30:16} 2Cró 29,34}
\bibleverse{17} Como mucha gente de la asamblea no se había purificado,
los levitas tenían que matar los corderos de la Pascua en nombre de
todos los impuros para dedicar los corderos al Señor. \bibleverse{18} La
mayoría del pueblo, muchos de los de Efraín, Manasés, Isacar y Zabulón,
no se habían purificado. Sin embargo, comieron la cena de la Pascua a
pesar de que no era lo que exigía la Ley, pues Ezequías había orado por
ellos, diciendo: ``Que el buen Señor perdone a todos los \footnote{\textbf{30:18}
  Éxod 12,-1} \bibleverse{19} que sinceramente quieren seguir al Señor
Dios, el Dios de sus antepasados, aunque no estén limpios según los
requisitos del santuario''.

\bibleverse{20} El Señor aceptó la oración de Ezequías y les permitió
esta violación.\footnote{\textbf{30:20} ``Les permitió esta violación'':
  la palabra aquí es ``curó'', pero esto es en un sentido metafórico, ya
  que no estaban ``enfermos''. Es en respuesta a la petición de Ezequías
  de que no se les considerara culpables por infringir la Ley
  ceremonial.} \bibleverse{21} El pueblo de Israel que estaba allí, en
Jerusalén, celebró con gran entusiasmo la Fiesta de los Panes sin
Levadura durante siete días, y cada día los levitas y los sacerdotes
alababan al Señor, acompañados de fuertes instrumentos. \bibleverse{22}
Ezequías hablaba positivamente a todos los levitas que mostraban un buen
entendimiento con el Señor. Durante siete días comieron la comida que se
les había asignado, presentaron ofrendas de amistad y dieron gracias al
Señor, el Dios de sus antepasados.

\hypertarget{continuaciuxf3n-de-la-celebraciuxf3n-en-la-segunda-semana}{%
\subsection{Continuación de la celebración en la segunda
semana}\label{continuaciuxf3n-de-la-celebraciuxf3n-en-la-segunda-semana}}

\bibleverse{23} Entonces todos acordaron seguir celebrando la fiesta
durante siete días más. Así que durante otros siete días celebraron,
llenos de alegría. \bibleverse{24} Ezequías, rey de Judá, dio mil toros
y siete mil ovejas como ofrendas en nombre de la asamblea. Los
funcionarios, a su vez, dieron mil toros y diez mil ovejas como ofrendas
en nombre de la asamblea. Un gran número de sacerdotes se purificó.
\footnote{\textbf{30:24} 2Cró 35,7}

\bibleverse{25} Toda la asamblea de Judá celebró, junto con los
sacerdotes y los levitas, y también con toda la asamblea que había
venido de Israel, incluidos los extranjeros de Israel y los que vivían
en Judá. \bibleverse{26} Había una tremenda alegría en Jerusalén, pues
desde los tiempos de Salomón, hijo de David, rey de Israel, no había
ocurrido nada parecido en la ciudad. \bibleverse{27} Los sacerdotes y
los levitas se levantaron para bendecir al pueblo, y Dios los escuchó:
su oración ascendió hasta donde él vivía en el cielo.

\hypertarget{limpiando-la-tierra-de-la-idolatruxeda}{%
\subsection{Limpiando la tierra de la
idolatría}\label{limpiando-la-tierra-de-la-idolatruxeda}}

\hypertarget{section-30}{%
\section{31}\label{section-30}}

\bibleverse{1} Cuando todo esto terminó, los israelitas que estaban allí
fueron a las ciudades de Judá y destrozaron las columnas paganas,
cortaron los postes de Asera y destruyeron los lugares altos y los
altares en todo Judá y Benjamín, así como en Efraín y Manasés, hasta que
los demolieron todos por completo. Después de eso, todos se fueron a sus
respectivas ciudades.

\hypertarget{cuidado-exitoso-de-los-ingresos-de-los-sacerdotes-y-levitas}{%
\subsection{Cuidado exitoso de los ingresos de los sacerdotes y
levitas}\label{cuidado-exitoso-de-los-ingresos-de-los-sacerdotes-y-levitas}}

\bibleverse{2} Entonces Ezequías reasignó las divisiones de los
sacerdotes y levitas, cada uno según su servicio: presentar holocaustos
y ofrendas de amistad, servir, dar gracias y cantar alabanzas en las
entradas del Templo del Señor. \bibleverse{3} El rey contribuía
personalmente a los holocaustos matutinos y vespertinos, y a los
holocaustos de los sábados, las lunas nuevas y las fiestas especiales,
como lo exige la Ley del Señor. \footnote{\textbf{31:3} Núm 28,-1; Núm
  29,1-29} \bibleverse{4} También ordenó al pueblo que vivía en
Jerusalén que mantuviera a los sacerdotes y a los levitas para que
pudieran dedicarse a estudiar y enseñar la Ley del Señor. \bibleverse{5}
En cuanto se difundió el mensaje, los israelitas dieron generosamente
las primicias del grano, del vino nuevo, del aceite de oliva y de la
miel, así como de todas las cosechas. Trajeron abundancia, el diezmo de
todo. \bibleverse{6} El pueblo de Israel que ahora vivía en Judá, y el
pueblo de Judá, trajeron el diezmo de sus vacas y rebaños. También
trajeron el diezmo de lo que habían dedicado al Señor su Dios, y lo
amontonaron.

\bibleverse{7} Comenzaron a hacerlo en el tercer mes, y terminaron en el
séptimo. \bibleverse{8} Cuando Ezequías y sus funcionarios llegaron y
vieron lo que se había recogido, dieron gracias al Señor y a su pueblo
Israel. \bibleverse{9} Ezequías preguntó a los sacerdotes y a los
levitas sobre lo que se había recogido. \bibleverse{10} Azarías, el jefe
de los sacerdotes de la familia de Sadoc, respondió: ``Desde que el
pueblo comenzó a traer sus contribuciones al Templo del Señor, hemos
tenido suficiente para comer y de sobra. Como el Señor ha bendecido a su
pueblo, nos ha sobrado mucho''.

\bibleverse{11} Ezequías ordenó la construcción de almacenes en el
Templo del Señor. Una vez que estuvieron listos, \bibleverse{12} el
pueblo trajo fielmente sus ofrendas, diezmos y regalos dedicados. El
levita Conanías era el responsable de ellos, y su hermano Simei era el
segundo al mando. \bibleverse{13} Estaban a cargo de los siguientes
oficiales: Jehiel, Azazías, Nahat, Asael, Jerimot, Jozabad, Eliel,
Ismaquías, Mahat y Benaía eran supervisores que ayudaban a Conanías y a
su hermano Simei. Fueron nombrados por el rey Ezequías y Azarías, el
jefe del Templo de Dios. \bibleverse{14} El levita Coré, hijo de Imna,
guardián de la puerta oriental, era el encargado de recibir las ofrendas
voluntarias que se daban a Dios. También distribuía las ofrendas
entregadas al Señor, junto con los dones consagrados. \bibleverse{15}
Bajo su mando estaban sus ayudantes Edén, Miniamín, Jesúa, Semaías,
Amarías y Secanías. Ellos hacían fielmente las asignaciones a sus
compañeros levitas en sus ciudades, según las divisiones sacerdotales,
compartiendo por igual con los ancianos y los jóvenes. \bibleverse{16}
También daban asignaciones a los varones que figuraban en la genealogía
y que tenían tres años o más, a todos los que entraban en el Templo del
Señor para cumplir con sus deberes diarios de servicio según las
responsabilidades de sus divisiones.

\hypertarget{elaboraciuxf3n-de-listas-de-sacerdotes-y-levitas-palabra-final}{%
\subsection{Elaboración de listas de sacerdotes y levitas; Palabra
final}\label{elaboraciuxf3n-de-listas-de-sacerdotes-y-levitas-palabra-final}}

\bibleverse{17} También dieron asignaciones a los sacerdotes que
figuraban por familia en la genealogía, y a los levitas de veinte años o
más, según las responsabilidades de sus divisiones. \bibleverse{18} La
genealogía incluía a todos los bebés, las esposas, los hijos y las hijas
de toda la comunidad, pues eran fieles al asegurarse de que se dedicaban
a la santidad. \bibleverse{19} En el caso de los sacerdotes, los
descendientes de Aarón, los que vivían en las tierras de labranza
alrededor de sus pueblos, se designaron hombres por nombre en todos los
pueblos para distribuir una asignación a cada varón entre los sacerdotes
y a cada levita según la lista de las genealogías.

\bibleverse{20} Esto es lo que hizo Ezequías en todo Judá. Hizo lo que
era bueno, correcto y verdadero ante el Señor, su Dios. \bibleverse{21}
En todo lo que hizo al trabajar para el Templo de Dios y al seguir las
leyes y los mandamientos de Dios, Ezequías fue sincero en su compromiso
con Dios. Por eso tuvo éxito en todo lo que hizo.\footnote{\textbf{31:21}
  Sal 1,3}

\hypertarget{la-incursiuxf3n-de-senaquerib-y-el-resto-de-ezechuxeeas}{%
\subsection{La incursión de Senaquerib y el resto de
Ezechîas}\label{la-incursiuxf3n-de-senaquerib-y-el-resto-de-ezechuxeeas}}

\hypertarget{section-31}{%
\section{32}\label{section-31}}

\bibleverse{1} Después de la fiel labor de Ezequías, Senaquerib, rey de
Asiria, invadió Judá y atacó sus ciudades fortificadas, planeando
conquistarlas para sí. \footnote{\textbf{32:1} 2Cró 31,20}
\bibleverse{2} Cuando Ezequías vio que Senaquerib había venido a atacar
Jerusalén, \bibleverse{3} habló con los comandantes de su ejército para
que bloquearan las fuentes de agua que se encontraban fuera de la
ciudad. Esto es lo que hicieron. \bibleverse{4} Dirigieron a un gran
grupo de trabajadores para que bloquearan todos los manantiales, así
como el arroyo que fluía en las cercanías. ``¿Por qué han de venir aquí
los reyes de Asiria y encontrar agua en abundancia?'' , preguntaron.

\bibleverse{5} Ezequías se puso a trabajar y reconstruyó todas las
partes de la muralla que se habían caído y construyó torres en ella.
También construyó otro muro fuera del primero. Reforzó el Milo
\footnote{\textbf{32:5} ``Milo'': el significado de este término es
  incierto, quizás signifique ``terrazas''. Véase 1 Crónicas 11:8.} en
la ciudad de David. También hizo una gran cantidad de armas y escudos.
\footnote{\textbf{32:5} 2Cró 25,23} \bibleverse{6} Ezequías puso a los
comandantes del ejército a cargo del pueblo. Luego convocó al pueblo
para que se reuniera en la plaza de la puerta de la ciudad. Les habló
con confianza, diciéndoles: \footnote{\textbf{32:6} 2Cró 30,22}
\bibleverse{7} ``¡Sean fuertes y valientes! No tengan miedo ni se
desanimen por culpa del rey de Asiria con su gran ejército, porque hay
más de nuestro lado que del suyo. \footnote{\textbf{32:7} 2Re 6,16}
\bibleverse{8} Él tiene ayuda humana, pero nosotros tenemos al Señor
Dios de nuestro lado para ayudarnos y librar nuestras batallas''. El
pueblo se animó con este discurso de Ezequías, rey de Judá. \footnote{\textbf{32:8}
  Jer 17,5; Jer 17,7}

\hypertarget{la-solicitud-de-senaquerib-de-entregar-la-ciudad-a-lachis}{%
\subsection{La solicitud de Senaquerib de entregar la ciudad a
Lachis}\label{la-solicitud-de-senaquerib-de-entregar-la-ciudad-a-lachis}}

\bibleverse{9} Algún tiempo después, cuando Senaquerib estaba atacando
la ciudad de Laquis con sus ejércitos, envió a sus oficiales a Jerusalén
con este mensaje para Ezequías, rey de Judá, y para todos los de Judá
que vivían allí. \bibleverse{10} ``Esto es lo que dice Senaquerib, rey
de Asiria. ¿En qué vas a confiar para sobrevivir cuando venga a atacar
Jerusalén? \bibleverse{11} ¿No ven que en realidad Ezequías les está
diciendo que morirán de hambre y de sed cuando les dice: `El Señor,
nuestro Dios, nos salvará del rey de Asiria'? \bibleverse{12} ¿No fue
Ezequías quien destruyó los lugares altos y los altares de este dios y
les dijo a Judá y a Jerusalén,\footnote{\textbf{32:12} Algunos han
  argumentado que los asirios malinterpretaron la naturaleza de las
  reformas religiosas de Ezequías. Sin embargo, puede ser que la
  eliminación de los ``santuarios'' locales, incluso los dedicados al
  Dios verdadero, no fuera apreciada por algunos, ya que en lugar de
  rendir culto localmente todos tenían que ir a Jerusalén. Esto puede
  haber provocado un resentimiento contra Ezequías que los asirios
  intentan capitalizar.} `Deben adorar en este único altar, y ofrecer
sacrificios en él solamente'? \bibleverse{13} ``¿No sabes lo que yo y
mis padres hemos hecho a todas las naciones de la tierra? Ninguno de sus
dioses pudo salvarlos a ellos ni a sus tierras de mí. \bibleverse{14}
¿Cuál de todos los dioses de estas naciones que mis padres destruyeron
ha podido salvarlos de mí? Entonces, ¿por qué creen que su dios puede
salvarlos de mí? \bibleverse{15} Así que no te dejes engañar por
Ezequías, ni permitas que te engañe de esta manera. No confíes en él,
porque ningún dios de ninguna nación o reino ha podido salvar a su
pueblo de mí o de mis padres. Así que menos aún es posible que tu dios
te salve de mí''.

\hypertarget{senaquerib-y-la-arrogancia-de-sus-embajadores}{%
\subsection{Senaquerib y la arrogancia de sus
embajadores}\label{senaquerib-y-la-arrogancia-de-sus-embajadores}}

\bibleverse{16} Los oficiales de Senaquerib siguieron criticando al
Señor Dios y a su siervo Ezequías. \bibleverse{17} Senaquerib también
escribió cartas insultando al Señor, el Dios de Israel, burlándose de él
al decir: ``Así como los dioses de las naciones no salvaron a su pueblo
de mí, el dios de Ezequías tampoco salvará a su pueblo de mí''.
\bibleverse{18} Los asirios también gritaron esto en hebreo al pueblo de
Jerusalén, de pie sobre la muralla, para atemorizarlo y aterrorizarlo a
fin de que la ciudad se rindiera. \bibleverse{19} Hablaban del Dios de
Jerusalén como lo hacían de los dioses de las otras naciones, dioses
hechos por seres humanos.

\hypertarget{oraciuxf3n-de-ezequuxedas-la-ayuda-de-dios-la-destrucciuxf3n-el-retiro-y-la-muerte-de-senaquerib}{%
\subsection{Oración de Ezequías; La ayuda de Dios: la destrucción, el
retiro y la muerte de
Senaquerib}\label{oraciuxf3n-de-ezequuxedas-la-ayuda-de-dios-la-destrucciuxf3n-el-retiro-y-la-muerte-de-senaquerib}}

\bibleverse{20} El rey Ezequías y el profeta Isaías, hijo de Amoz,
apelaron sobre esto en oración al Dios del cielo.

\bibleverse{21} El Señor envió un ángel que aniquiló a todos los
guerreros, jefes y comandantes del campamento del rey asirio. Así que
regresó a su casa en desgracia. Cuando entró en el templo de su dios,
algunos de sus propios hijos lo mataron con sus espadas. \bibleverse{22}
El Señor salvó a Ezequías y al pueblo de Jerusalén del rey Senaquerib de
Asiria y de todos los demás enemigos, dándoles paz en todos los
sentidos. \bibleverse{23} Desde entonces fue muy respetado por todas las
naciones, y muchos trajeron a Jerusalén ofrendas para el Señor y regalos
valiosos para Ezequías, rey de Judá.

\hypertarget{la-enfermedad-la-arrogancia-y-la-penitencia-de-ezechuxeeas}{%
\subsection{La enfermedad, la arrogancia y la penitencia de
Ezechîas}\label{la-enfermedad-la-arrogancia-y-la-penitencia-de-ezechuxeeas}}

\bibleverse{24} Por aquel entonces Ezequías cayó enfermo y estuvo a
punto de morir. Entonces oró al Señor, quien le respondió sanándolo y
dándole una señal milagrosa.\footnote{\textbf{32:24} La sombra del reloj
  de sol se mueve hacia atrás, véase 2 Reyes 20:8-11.} \bibleverse{25}
Pero como se había vuelto orgulloso, Ezequías no reconoció el don que se
le había dado. Así que la ira del Señor cayó sobre él, y sobre Judá y
Jerusalén. \footnote{\textbf{32:25} 2Cró 26,16} \bibleverse{26} Entonces
Ezequías se disculpó por su arrogancia, al igual que el pueblo de
Jerusalén, y la ira del Señor ya no cayó sobre ellos durante la vida de
Ezequías.

\hypertarget{la-riqueza-de-ezechuxeeas-abastecimiento-de-agua-a-jerusaluxe9n-y-tentaciuxf3n-de-la-embajada-de-babilonia}{%
\subsection{La riqueza de Ezechîas; Abastecimiento de agua a Jerusalén y
tentación de la embajada de
Babilonia}\label{la-riqueza-de-ezechuxeeas-abastecimiento-de-agua-a-jerusaluxe9n-y-tentaciuxf3n-de-la-embajada-de-babilonia}}

\bibleverse{27} Ezequías era muy rico y gozaba de mucha honra, y
construyó almacenes de tesorería para guardar plata, oro, piedras
preciosas, especias, escudos y toda clase de cosas valiosas.
\bibleverse{28} Construyó edificios para almacenar provisiones de grano,
vino nuevo y aceite de oliva, y establos para toda clase de animales,
incluyendo ganado vacuno y ovino. \bibleverse{29} Construyó muchas
ciudades, y poseía grandes rebaños de ganado y de ovejas, porque Dios lo
había hecho muy rico. \bibleverse{30} Ezequías bloqueó la salida del
manantial superior de Gihón e hizo que el agua fluyera hacia el lado
occidental de la ciudad de David. Ezequías tuvo éxito en todo lo que
hizo.

\bibleverse{31} Pero cuando los embajadores de los gobernantes de
Babilonia se acercaron a él para preguntar por la señal
milagrosa\footnote{\textbf{32:31} Véase el versículo 24.} que había
sucedido en el país, Dios lo dejó para que lo pusiera a prueba, para
conocer el verdadero pensamiento de Ezequías.\footnote{\textbf{32:31}
  Esto seguramente se refiere al orgullo anterior de Ezequías y a su
  incapacidad para reconocer el milagro de Dios en su favor Aquí, en
  lenguaje humano, vemos a Dios asegurándose de que Ezequías deje de ser
  orgulloso.}

\hypertarget{termina-la-historia-de-ezechuxeeas}{%
\subsection{Termina la historia de
Ezechîas}\label{termina-la-historia-de-ezechuxeeas}}

\bibleverse{32} El resto de lo que hizo Ezequías, incluidos sus actos de
lealtad, están registrados en la visión del profeta Isaías, hijo de
Amoz, en el Libro de los Reyes de Judá e Israel. \bibleverse{33}
Ezequías murió y fue enterrado en el cementerio superior de los
descendientes de David. Todo Judá y el pueblo de Jerusalén lo honraron a
su muerte. Su hijo Manasés tomó el relevo como rey.

\hypertarget{manasuxe9s-rey-de-juduxe1-idolatruxeda-manasuxe9s}{%
\subsection{Manasés rey de Judá; Idolatría
manasés}\label{manasuxe9s-rey-de-juduxe1-idolatruxeda-manasuxe9s}}

\hypertarget{section-32}{%
\section{33}\label{section-32}}

\bibleverse{1} Manasés tenía doce años cuando llegó a ser rey, y reinó
por Jerusalén cincuenta y cinco años. \bibleverse{2} Hizo lo malo a los
ojos del Señor, al seguir las repugnantes prácticas religiosas de las
naciones que el Señor había expulsado ante los israelitas. \footnote{\textbf{33:2}
  Deut 18,9} \bibleverse{3} Reconstruyó los lugares altos que su padre
Ezequías había destruido, e hizo altares para los baales y levantó
postes de Asera. Adoró al sol, a la luna y a las estrellas y les rindió
culto. \footnote{\textbf{33:3} 2Re 18,4} \bibleverse{4} Construyó
altares en el Templo del Señor, sobre el cual el Señor había dicho:
``Seré honrado en Jerusalén para siempre''. \footnote{\textbf{33:4} Deut
  12,5; Deut 12,11; 1Re 9,3} \bibleverse{5} Construyó estos altares para
adorar al sol, la luna y las estrellas en los dos patios del Templo del
Señor. \bibleverse{6} Sacrificó a sus hijos quemándolos hasta la muerte
en el Valle de Ben-hinnom. Practicaba la hechicería, la adivinación y la
brujería, y visitaba a médiums y espiritistas. Hizo mucho mal a los ojos
del Señor, haciéndolo enojar. \bibleverse{7} Tomó un ídolo pagano que
había fabricado y lo colocó en el Templo de Dios, sobre el cual Dios
había dicho a David y a su hijo Salomón: ``Seré honrado para siempre en
este Templo y en Jerusalén, que he elegido entre todas las tribus de
Israel. \bibleverse{8} Si los israelitas tienen cuidado de seguir todo
lo que les he ordenado hacer -todas las leyes, mandamientos y
reglamentos, dados por medio de Moisés- entonces no los haré abandonar
la tierra que les concedí a sus antepasados''. \bibleverse{9} Pero
Manasés sedujo a Judá y al pueblo de Jerusalén, llevándolos a cometer
pecados aún peores que los de las naciones que el Señor había destruido
antes de los israelitas.

\bibleverse{10} El Señor advirtió a Manasés y a su pueblo, pero no le
hicieron caso.

\hypertarget{la-gira-del-prisionero-de-manasuxe9s-a-babilonia-su-arrepentimiento-y-regreso-a-casa}{%
\subsection{La gira del prisionero de Manasés a Babilonia, su
arrepentimiento y regreso a
casa}\label{la-gira-del-prisionero-de-manasuxe9s-a-babilonia-su-arrepentimiento-y-regreso-a-casa}}

\bibleverse{11} Entonces el Señor envió a los ejércitos de Asiria con
sus comandantes para que los atacaran. Los asirios capturaron a Manasés,
le pusieron un garfio en la nariz, le pusieron grilletes de bronce y se
lo llevaron a Babilonia.

\bibleverse{12} En su miseria, pidió ayuda al Señor Dios, arrepentido de
su arrogancia ante el Dios de sus antepasados. \bibleverse{13} Oró y
oró, y el Señor escuchó sus súplicas, por lo que el Señor hizo regresar
a Manasés a Jerusalén y a su reino. Entonces Manasés se convenció de que
el Señor es Dios.

\hypertarget{manasuxe9s-construyendo-muros-y-esfuerzos-para-eliminar-la-idolatruxeda}{%
\subsection{Manasés construyendo muros y esfuerzos para eliminar la
idolatría}\label{manasuxe9s-construyendo-muros-y-esfuerzos-para-eliminar-la-idolatruxeda}}

\bibleverse{14} Después de esto, Manasés reconstruyó la muralla exterior
de la Ciudad de David desde el oeste de Gihón, en el valle, hasta la
Puerta del Pescado, y alrededor de la colina de Ofel, y la hizo mucho
más alta. También asignó comandantes del ejército a todas las ciudades
fortificadas de Judá. \bibleverse{15} Se deshizo de los dioses
extranjeros y del ídolo del Templo del Señor, junto con todos los
altares que había construido en la colina del Templo y en Jerusalén,
arrojándolos todos fuera de la ciudad. \bibleverse{16} Luego restauró el
altar del Señor y sacrificó en él ofrendas de amistad y de
agradecimiento, e instruyó a Judá para que adorara al Señor, el Dios de
Israel. \bibleverse{17} Pero el pueblo seguía sacrificando en los
lugares altos, pero sólo al Señor, su Dios.

\bibleverse{18} El resto de lo que hizo Manasés, junto con su oración a
su Dios y lo que le dijeron los videntes que hablaban en nombre del
Señor, están registrados en el Libro de los Reyes de Israel. \footnote{\textbf{33:18}
  2Re 21,17-18} \bibleverse{19} Su oración y la forma en que Dios le
respondió, así como todos sus pecados e infidelidades, y dónde construyó
lugares altos y levantó postes de Asera e ídolos antes de admitir que
estaba equivocado, están registrados en los Registros de los Videntes.

\hypertarget{amuxf3n-rey-de-juduxe1}{%
\subsection{Amón Rey de Judá}\label{amuxf3n-rey-de-juduxe1}}

\bibleverse{20} Manasés murió y fue enterrado en su palacio. Su hijo
Amón asumió como rey. \bibleverse{21} Amón tenía veintidós años cuando
se convirtió en rey, y reinó en Jerusalén durante dos años.
\bibleverse{22} Hizo el mal a los ojos del Señor, tal como lo había
hecho su padre Manasés. Amón adoraba y sacrificaba a todos los ídolos
que su padre Manasés había hecho. \bibleverse{23} Sin embargo, no
admitió su orgullo ante el Señor como lo había hecho su padre Manasés;
de hecho, Amón empeoró aún más su culpa. \footnote{\textbf{33:23} 2Cró
  33,12}

\bibleverse{24} Entonces los funcionarios de Amón conspiraron contra él
y lo mataron en su palacio. \bibleverse{25} Pero la gente de la
nación\footnote{\textbf{33:25} ``La gente de la nación''. No se da nada
  más específico, pero podría referirse a los nobles propietarios de
  tierras. Si es así, probablemente fueron regentes durante la época en
  que Josías era un niño.} mató a todos los que habían conspirado contra
el rey Amón, e hicieron rey a su hijo Josías.

\hypertarget{el-gobierno-del-rey-josuxedas}{%
\subsection{El gobierno del rey
Josías}\label{el-gobierno-del-rey-josuxedas}}

\hypertarget{section-33}{%
\section{34}\label{section-33}}

\bibleverse{1} Josías tenía ocho años cuando se convirtió en rey, y
reinó en Jerusalén durante treinta y un años. \bibleverse{2} Hizo lo que
era correcto a los ojos del Señor y siguió los caminos de su antepasado
David; no se desvió ni a la derecha ni a la izquierda. \footnote{\textbf{34:2}
  2Re 22,1-2; 2Cró 29,2}

\hypertarget{restauraciuxf3n-del-culto-puro}{%
\subsection{Restauración del culto
puro}\label{restauraciuxf3n-del-culto-puro}}

\bibleverse{3} En el octavo año de su reinado, siendo aún joven, Josías
comenzó a adorar públicamente al Dios de David, su antepasado, y en el
duodécimo año de su reinado comenzó a limpiar Judá y Jerusalén, quitando
los lugares altos, los postes de Asera, los ídolos tallados y las
imágenes de metal. \footnote{\textbf{34:3} 2Re 23,4-20} \bibleverse{4}
Hizo derribar los altares de Baal que estaban frente a él, y los altares
de incienso que estaban sobre ellos fueron cortados. Además, los postes
de Asera, los ídolos tallados y las imágenes de metal fueron hechos
pedazos y esparcidos sobre las tumbas de los que les habían ofrecido
sacrificios. \footnote{\textbf{34:4} 2Cró 14,4; Lev 26,30}
\bibleverse{5} Quemó los huesos de los sacerdotes idólatras en sus
altares. De esta manera purificó a Judá y a Jerusalén. \footnote{\textbf{34:5}
  1Re 13,2} \bibleverse{6} Josías repitió esto en las ciudades de
Manasés, Efraín y Simeón, hasta Neftalí, y en sus
alrededores.\footnote{\textbf{34:6} ``Y en sus alrededores'': como una
  versión antigua. Otra posibilidad es ``y en sus ruinas'', debido a la
  destrucción causada por la invasión asiria.} \bibleverse{7} Derribó
los altares y redujo a polvo los postes de Asera y las imágenes, y cortó
todos los altares de incienso en toda la tierra de Israel. Luego regresó
a Jerusalén.

\hypertarget{explicar-los-procedimientos-que-se-siguen-para-restaurar-y-mantener-el-templo}{%
\subsection{Explicar los procedimientos que se siguen para restaurar y
mantener el
templo}\label{explicar-los-procedimientos-que-se-siguen-para-restaurar-y-mantener-el-templo}}

\bibleverse{8} En el año dieciocho de su reinado, una vez que terminó de
limpiar la tierra y el Templo, Josías envió a Safán, hijo de Azalías, a
Maasías, el gobernador de la ciudad, y a Joá, hijo de Joacaz, el
guardián de los registros, a reparar el Templo del Señor su Dios.
\footnote{\textbf{34:8} 2Re 22,3-6} \bibleverse{9} Fueron a ver al sumo
sacerdote Hilcías y le dieron el dinero que se había llevado al Templo
de Dios. Los levitas de las entradas habían recogido este dinero del
pueblo de Manasés y Efraín, de lo que quedaba del pueblo de Israel, así
como las contribuciones de Judá, Benjamín y el pueblo de Jerusalén.
\bibleverse{10} Lo entregaron a los que supervisaban los trabajos de
reparación del Templo del Señor, que a su vez pagaban a los obreros que
hacían la restauración y la reparación. \bibleverse{11} También pagaron
a los carpinteros y constructores para que compraran piedra cortada, así
como madera para viguetas y vigas para los edificios que los reyes de
Judá habían dejado deteriorar. \bibleverse{12} Los hombres hicieron un
trabajo bueno y honesto. Al frente de ellos estaban Jahat y Abdías,
levitas de los hijos de Merari; y Zacarías y Mesulam, levitas de los
hijos de los coatitas. Los levitas, todos ellos hábiles músicos,
\bibleverse{13} estaban a cargo de los obreros y dirigían a todos los
involucrados, según lo que se requiriera. Algunos de los levitas eran
escribas, otros oficiales y otros porteros.

\hypertarget{informe-sobre-el-descubrimiento-del-cuxf3digo-y-su-primer-efecto}{%
\subsection{Informe sobre el descubrimiento del código y su primer
efecto}\label{informe-sobre-el-descubrimiento-del-cuxf3digo-y-su-primer-efecto}}

\bibleverse{14} En el proceso de sacar el dinero donado al Templo del
Señor, el sacerdote Hilcías descubrió el Libro de la Ley del Señor
escrito por Moisés. \bibleverse{15} Hilcías le dijo al escriba Safán:
``He encontrado el Libro de la Ley en el Templo del Señor''. Se lo dio a
Safán.

\bibleverse{16} Safán llevó el libro al rey y le dijo: ``Nosotros, tus
siervos, estamos haciendo todo lo que se nos ordenó. \bibleverse{17} El
dinero recaudado en el Templo del Señor ha sido entregado a los que
están supervisando a los trabajadores, pagándoles para que hagan las
reparaciones''. \bibleverse{18} El escriba Safán le dijo al rey: ``El
sacerdote Hilcías me dio este libro''. Safán se lo leyó al rey.

\bibleverse{19} Cuando el rey oyó lo que decía la Ley, se rasgó las
vestiduras.\footnote{\textbf{34:19} ``Se rasgó las vestiduras'': una
  demostración de gran angustia emocional.} \bibleverse{20} Entonces dio
las siguientes órdenes a Hilcías, a Ajicam, hijo de Safán, a Abdón, hijo
de Miqueas, al escriba Safán y a Asaías, ayudante del rey:
\bibleverse{21} ``Vayan y hablen con el Señor por mí, y también por los
que aún viven en Israel y en Judá, sobre lo que dice el libro que se ha
encontrado. Porque el Señor debe estar realmente enojado con nosotros
porque nuestros antepasados no han obedecido las instrucciones del Señor
siguiendo todo lo que está escrito en este libro''.

\hypertarget{interrogatorio-y-respuesta-de-la-profetisa-hulda}{%
\subsection{Interrogatorio y respuesta de la profetisa
Hulda}\label{interrogatorio-y-respuesta-de-la-profetisa-hulda}}

\bibleverse{22} Hilcías y los que el rey había seleccionado fueron a
hablar con la profetisa Hulda, esposa de Salum, hijo de Tojat, hijo de
Hasra, guardián del armario.\footnote{\textbf{34:22} ``Armario'': ya sea
  las vestiduras del rey, o del (los) sacerdote (s).} Vivía en
Jerusalén, en el segundo barrio de la ciudad.

\bibleverse{23} Hulda les dijo: ``Esto es lo que dice el Señor, el Dios
de Israel: Díganle al hombre que los envió a verme: \bibleverse{24} Esto
es lo que dice el Señor: Estoy a punto de hacer caer el desastre sobre
este lugar y sobre su pueblo, de acuerdo con todas las maldiciones
escritas en el libro que se le ha leído al rey de Judá. \bibleverse{25}
Me han abandonado y han ofrecido sacrificios a otros dioses, haciéndome
enojar por todo lo que han hecho. Mi ira se derramará sobre este lugar y
no se detendrá. \bibleverse{26} ``Pero dile al rey de Judá que te envió
a preguntar al Señor, dile que esto es lo que dice el Señor, el Dios de
Israel: En cuanto a lo que oíste que te leyeron, \bibleverse{27} porque
fuiste receptivo y te arrepentiste ante Dios cuando oíste sus
advertencias contra este lugar y contra su pueblo, y porque te has
arrepentido, rasgando tus vestiduras y llorando ante mí, yo también los
he oído\footnote{\textbf{34:27} ``Oído'': en el sentido de una respuesta
  positiva.} , declara el Señor. \footnote{\textbf{34:27} 2Cró 33,12}
\bibleverse{28} Todo esto no sucederá hasta después de tu muerte, y
morirás en paz.\footnote{\textbf{34:28} ``Morirás en paz'': por
  supuesto, esto no ocurrió, porque Josías decidió enfrentarse al faraón
  egipcio en la batalla y fue asesinado. Véase 35:20-24.} No verás todo
el desastre que voy a hacer caer sobre este lugar y sus habitantes''.
Volvieron al rey y le dieron su respuesta.

\hypertarget{josuxedas-concluye-el-nuevo-pacto-de-dios-en-asociaciuxf3n-con-los-ancianos-del-pueblo}{%
\subsection{Josías concluye el nuevo pacto de Dios en asociación con los
ancianos del
pueblo}\label{josuxedas-concluye-el-nuevo-pacto-de-dios-en-asociaciuxf3n-con-los-ancianos-del-pueblo}}

\bibleverse{29} Entonces el rey convocó a todos los ancianos de Judá y
de Jerusalén. \bibleverse{30} Fue al Templo del Señor con todo el pueblo
de Judá y de Jerusalén, junto con los sacerdotes y los levitas, todo el
pueblo desde el más pequeño hasta el más grande, y les leyó todo el
Libro del Acuerdo que había sido descubierto en el Templo del Señor.
\bibleverse{31} El rey se puso de pie junto a la columna e hizo un
acuerdo solemne ante el Señor de seguirlo y de cumplir sus mandamientos,
leyes y reglamentos con total dedicación, y de observar los requisitos
del acuerdo tal como estaban escritos en el libro. \footnote{\textbf{34:31}
  2Cró 15,12; Jos 24,25} \bibleverse{32} Entonces hizo que todos los
presentes de Jerusalén y de Benjamín se pusieran de pie para mostrar que
estaban de acuerdo. Así todo el pueblo de Jerusalén aceptó y siguió el
acuerdo con Dios, el Dios de sus antepasados. \footnote{\textbf{34:32}
  2Re 23,3}

\bibleverse{33} Josías demolió todos los ídolos viles de todo el
territorio perteneciente a los israelitas, e hizo que todos en Israel
sirvieran al Señor, su Dios. Durante su reinado no dejaron de adorar al
Señor, el Dios de sus padres.

\hypertarget{la-estricta-celebraciuxf3n-de-la-pascua-de-josuxedas}{%
\subsection{La estricta celebración de la Pascua de
Josías}\label{la-estricta-celebraciuxf3n-de-la-pascua-de-josuxedas}}

\hypertarget{section-34}{%
\section{35}\label{section-34}}

\bibleverse{1} Josías celebró una Pascua para el Señor en Jerusalén, y
el cordero de la Pascua se mataba el día catorce del primer mes.
\bibleverse{2} Asignó a los sacerdotes sus respectivos deberes y los
animó en su ministerio en el Templo del Señor. \bibleverse{3} Josías
dijo a los levitas que enseñaban a todo Israel y eran santos para el
Señor: ``Colocad el Arca sagrada en el Templo construido por Salomón,
hijo de David, rey de Israel. Ya no es necesario que la lleven sobre sus
hombros. Tu responsabilidad ahora es servir al Señor, tu Dios, y a su
pueblo Israel. \footnote{\textbf{35:3} 1Re 6,1} \bibleverse{4}
Prepárense para el servicio en sus divisiones, por familias, según las
instrucciones dadas por David, rey de Israel, y su hijo Salomón.
\bibleverse{5} ``Luego, deberán estar en el santuario para asistir a los
laicos según las divisiones familiares, siguiendo las asignaciones de
acuerdo con sus divisiones familiares de los levitas. \bibleverse{6}
Sacrifiquen los corderos de la Pascua, purifícate y prepárate para
ayudar al pueblo que viene a cumplir los requisitos dados por el Señor a
través de Moisés''.

\bibleverse{7} Josías aportó como ofrendas pascuales para todo el pueblo
presente 30. 000 corderos y cabras, y 3. 000 toros, todos de sus propios
rebaños y manadas. \bibleverse{8} Sus funcionarios contribuyeron
libremente con el pueblo, los sacerdotes y los levitas. Hilcías,
Zacarías y Jehiel, que estaban a cargo del Templo de Dios, dieron a los
sacerdotes como ofrendas de Pascua 2. 600 corderos de Pascua y 300
toros. \bibleverse{9} Los jefes de los levitas, Conanías, Semaías y
Netanel, sus hermanos, y Hasabías, Jeiel y Jozabad, dieron a los levitas
como ofrendas de Pascua 5. 000 corderos y cabritos y 500 toros.

\bibleverse{10} Una vez terminados los preparativos, los sacerdotes se
colocaron en el lugar que se les había asignado y los levitas ocuparon
sus puestos en sus divisiones, tal como lo había ordenado el rey.
\bibleverse{11} Mataron los corderos de la Pascua, los sacerdotes
rociaron la sangre que se les dio en el altar, mientras los levitas
desollaban los sacrificios. \bibleverse{12} Dejaron a un lado los
holocaustos que debían entregarse a las familias de los laicos, por
división, para que los ofrecieran al Señor, como lo exige el Libro de
Moisés. Lo mismo hicieron con los toros. \bibleverse{13} Asaron al fuego
los sacrificios de la Pascua, como se requiere, y cocieron las ofrendas
sagradas en ollas, calderos y sartenes, y las llevaron rápidamente a los
laicos. \bibleverse{14} Después preparaban la comida de las ofrendas
para ellos y para los sacerdotes, porque los sacerdotes, descendientes
de Aarón, estaban ocupados presentando holocaustos y grasa hasta que
llegaba la noche. Así que los levitas hacían este trabajo para sí mismos
y para los sacerdotes, los descendientes de Aarón. \bibleverse{15} Los
cantores, descendientes de Asaf, estaban en sus puestos siguiendo las
instrucciones dadas por David, Asaf, Hemán y Jedutún, el vidente del
rey. Los porteros encargados de cada puerta no necesitaban salir, porque
sus compañeros levitas los proveían. \footnote{\textbf{35:15} 1Cró 25,1;
  1Cró 26,1}

\bibleverse{16} Aquel día se celebró todo el servicio de la Pascua del
Señor, incluida la presentación de los holocaustos en el altar del
Señor, tal como lo había ordenado el rey Josías. \bibleverse{17} Los
israelitas que estaban allí también celebraron la Pascua en ese momento,
y también la Fiesta de los Panes sin Levadura durante los siete días
siguientes. \bibleverse{18} No se había celebrado una Pascua como ésta
en Israel desde los tiempos del profeta Samuel. Ninguno de los reyes de
Israel había celebrado una Pascua como la que Josías observó con los
sacerdotes, los levitas, todo Judá, los israelitas que estaban allí y el
pueblo de Jerusalén. \bibleverse{19} Esta Pascua se celebró en el año
dieciocho del reinado de Josías.

\hypertarget{necao-de-egipto-y-la-muerte-de-josuxedas-dolor-por-el}{%
\subsection{Necao de Egipto y la muerte de Josías; Dolor por
el}\label{necao-de-egipto-y-la-muerte-de-josuxedas-dolor-por-el}}

\bibleverse{20} Después de todo este trabajo que Josías había realizado
en la restauración del Templo, el rey Neco de Egipto dirigía su ejército
para luchar en Carquemis, cerca del Éufrates, y Josías fue a enfrentarse
a él. \bibleverse{21} Neco le envió mensajeros diciendo: ``¿Qué
discusión hay entre tú y yo, rey de Judá? No he venido a atacarte hoy,
porque estoy luchando con otro reino. Dios me dijo que debía
apresurarme, así que deja de obstruir a Dios, que está conmigo, ¡o te
destruirá!''

\bibleverse{22} Pero Josías no se dio la vuelta y se marchó. En cambio,
se disfrazó para poder luchar contra Neco en la batalla. Ignoró el
mensaje de Neco que venía de Dios, y fue a luchar contra él en la
llanura de Meguido. \bibleverse{23} Allí los arqueros dispararon al rey
Josías. Él llamó a los que estaban a su lado: ``¡Sáquenme de la batalla,
porque estoy malherido!''.

\bibleverse{24} Así que lo sacaron de su carro y lo llevaron en su
segundo carro a Jerusalén, donde murió. Josías fue enterrado en la tumba
de sus antepasados. Todo Judá y Jerusalén lo lloraron. \bibleverse{25}
Entonces Jeremías escribió un lamento sobre Josías, y hasta hoy los
coros de hombres y mujeres cantan canciones tristes sobre Josías. Se han
convertido en parte de lo que se canta regularmente en Israel, y están
registradas en el Libro de los Lamentos. \footnote{\textbf{35:25} Jer
  22,10-11}

\bibleverse{26} El resto de lo que hizo Josías, junto con sus actos de
lealtad siguiendo lo que está escrito en la Ley del Señor,
\bibleverse{27} todas sus acciones, de principio a fin, están
registradas en el Libro de los Reyes de Israel y Judá.

\hypertarget{joachuxe2z-rey-de-juduxe1}{%
\subsection{Joachâz rey de Judá}\label{joachuxe2z-rey-de-juduxe1}}

\hypertarget{section-35}{%
\section{36}\label{section-35}}

\bibleverse{1} Entonces la gente de la nación tomó a Joacaz, hijo de
Josías, y lo hizo rey en Jerusalén en sucesión de su padre.
\bibleverse{2} Joacaz tenía veintitrés años cuando llegó a ser rey, y
reinó en Jerusalén durante tres meses. \bibleverse{3} Entonces el rey de
Egipto lo destituyó del trono en Jerusalén y le impuso a Judá un
impuesto de cien talentos de plata y un talento de oro.

\hypertarget{joacim-kuxf6nig-von-juda}{%
\subsection{Joacim König von Juda}\label{joacim-kuxf6nig-von-juda}}

\bibleverse{4} Neco, rey de Egipto, nombró a Eliaquim, hermano de
Joacaz, rey de Judá y de Jerusalén, y cambió el nombre de Eliaquim por
el de Joacim. Neco se llevó a Egipto al hermano de Eliaquim, Joacaz.

\bibleverse{5} Joacim tenía veinticinco años cuando llegó a ser rey, y
reinó en Jerusalén durante once años. Hizo lo malo ante los ojos del
Señor, su Dios. \bibleverse{6} Entonces Nabucodonosor, rey de Babilonia,
atacó a Joacim. Lo capturó\footnote{\textbf{36:6} ``Lo capturó'':
  implícito.} y le puso grilletes de bronce, y lo llevó a Babilonia.
\bibleverse{7} Nabucodonosor también tomó algunos objetos del Templo del
Señor, y los puso en su templo\footnote{\textbf{36:7} ``Templo'': o,
  ``palacio''.} en Babilonia. \footnote{\textbf{36:7} Esd 1,7}
\bibleverse{8} El resto de lo que hizo Joacim, los repugnantes pecados
que cometió y todas las pruebas contra él, están escritos en el Libro de
los Reyes de Israel y Judá. Su hijo Joaquín tomó el relevo como rey.

\hypertarget{joachuxeen-rey-de-juduxe1}{%
\subsection{Joachîn rey de Judá}\label{joachuxeen-rey-de-juduxe1}}

\bibleverse{9} Joaquín tenía dieciocho años cuando llegó a ser rey, y
reinó en Jerusalén durante tres meses y diez días. Hizo el mal a los
ojos del Señor. \bibleverse{10} En la primavera del año, el rey
Nabucodonosor lo llamó y lo llevó a Babilonia, junto con objetos
valiosos del Templo del Señor, e hizo que el tío de Joaquín\footnote{\textbf{36:10}
  ``Tío'': véase 2 Reyes 24:17.} Sedequías rey sobre Judá y Jerusalén.

\hypertarget{sedecuxedas-rey-de-juduxe1-la-ruina-de-uxe9l-y-de-su-gente}{%
\subsection{Sedecías, rey de Judá; la ruina de él y de su
gente}\label{sedecuxedas-rey-de-juduxe1-la-ruina-de-uxe9l-y-de-su-gente}}

\bibleverse{11} Sedequías tenía veintiún años cuando llegó a ser rey, y
reinó en Jerusalén durante once años. \footnote{\textbf{36:11} Jer
  52,1-27} \bibleverse{12} Hizo lo malo ante los ojos del Señor, su
Dios, y se negó a admitir su orgullo cuando el profeta Jeremías le
advirtió directamente de parte del Señor. \footnote{\textbf{36:12} Jer
  37,-1; Jer 38,1-38} \bibleverse{13} También se rebeló contra el rey
Nabucodonosor, quien le había hecho jurar lealtad a Dios. Sedequías era
arrogante y de corazón duro, y se negó a volver al Señor, el Dios de
Israel. \bibleverse{14} Y todos los dirigentes de los sacerdotes y del
pueblo eran también totalmente infieles y pecadores, y seguían todas las
prácticas repugnantes de las naciones paganas. Profanaron el Templo del
Señor, que él había consagrado como santo en Jerusalén. \footnote{\textbf{36:14}
  Deut 18,9}

\bibleverse{15} Una y otra vez el Señor, el Dios de sus padres, advirtió
a su pueblo por medio de sus profetas, porque quería mostrar
misericordia con ellos y con su Templo. \bibleverse{16} Pero ellos
ridiculizaban a los mensajeros de Dios, despreciaban sus advertencias y
se burlaban de sus profetas, hasta que la ira del Señor contra su pueblo
fue provocada a tal punto que no pudo ser contenida.

\hypertarget{destrucciuxf3n-del-imperio-por-nabucodonosor-el-cautiverio-babiluxf3nico}{%
\subsection{Destrucción del imperio por Nabucodonosor; el cautiverio
babilónico}\label{destrucciuxf3n-del-imperio-por-nabucodonosor-el-cautiverio-babiluxf3nico}}

\bibleverse{17} Entonces el Señor hizo que el rey de Babilonia los
atacara. Su ejército mató a espada a sus mejores jóvenes incluso en el
santuario. Los babilonios no perdonaron a los jóvenes ni a las mujeres,
ni a los enfermos, ni a los ancianos. Dios los entregó a todos en manos
de Nabucodonosor. \bibleverse{18} Se llevó a Babilonia todos los
artículos, grandes y pequeños, del Templo de Dios, del tesoro del
Templo, del rey y de sus funcionarios. \bibleverse{19} Luego los
babilonios quemaron el Templo de Dios y demolieron las murallas de
Jerusalén. Incendiaron todos los palacios y destruyeron todo lo que
tenía algún valor. \bibleverse{20} Nabucodonosor llevó al exilio en
Babilonia a los que no habían sido asesinados. Fueron esclavos para él y
sus hijos, hasta que el reino de Persia tomó el control. \bibleverse{21}
Así que para cumplir la profecía del Señor dada por medio de Jeremías,
la tierra disfrutó de sus sábados como descanso durante todo el tiempo
que estuvo desolada, guardando el sábado hasta que se cumplieron setenta
años. \footnote{\textbf{36:21} Lev 26,34; Jer 25,8-11}

\hypertarget{el-permiso-para-regresar-a-casa-del-rey-persa-ciro}{%
\subsection{El permiso para regresar a casa del rey persa
Ciro}\label{el-permiso-para-regresar-a-casa-del-rey-persa-ciro}}

\bibleverse{22} En el primer año de Ciro, rey de Persia, para cumplir la
profecía del Señor dada por medio de Jeremías, el Señor animó a Ciro,
rey de Persia, a que emitiera una proclama en todo su reino y también a
que la pusiera por escrito, diciendo: \bibleverse{23} ``Esto es lo que
dice Ciro, rey de Persia: `El Señor, el Dios del cielo, que me ha dado
todos los reinos de la tierra, me ha dado la responsabilidad de
construir un Templo para él en Jerusalén, en Judá. Cualquiera de ustedes
que pertenezca a su pueblo puede ir allí. Que el Señor, su Dios, esté
con ustedes'\,''.\footnote{\textbf{36:23} Esta proclamación de Ciro se
  encuentra también al principio de Esdras 1.}
