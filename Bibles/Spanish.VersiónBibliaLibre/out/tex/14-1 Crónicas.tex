\hypertarget{los-antepasados-hasta-el-diluvio}{%
\subsection{Los antepasados \hspace{0pt}\hspace{0pt}hasta el
diluvio}\label{los-antepasados-hasta-el-diluvio}}

\hypertarget{section}{%
\section{1}\label{section}}

\bibleverse{1} Adán, Set, Enós,\footnote{\textbf{1:1} El libro comienza
  con la lista de nombres que puede parecer extraña para un lector
  moderno, pero al brindar esta línea genealógica el escritor de
  Crónicas está resumiendo la historia. En lugar de intentar
  proporcionar información sobre quiénes fueron todos estos individuos,
  se recomienda que la información relevante se encuentre en los libros
  históricos de la Biblia desde el Génesis en adelante.} \bibleverse{2}
Quenán, Malalel, Jared, \bibleverse{3} Enoc, Matusalén, Lamec, Noé.
\bibleverse{4} Los hijos de Noé:\footnote{\textbf{1:4} Tomado de la
  Septuaginta: Esta línea está ausente en el texto hebreo.} Sem, Cam, y
Jafet.

\hypertarget{los-descendientes-de-nouxe9-excepto-abraham-los-jafetitas}{%
\subsection{Los descendientes de Noé excepto Abraham; Los
jafetitas}\label{los-descendientes-de-nouxe9-excepto-abraham-los-jafetitas}}

\bibleverse{5} Los hijos\footnote{\textbf{1:5} Como se ha señalado en
  otras partes, ``hijos'' puede significar ``descendientes''.} de Jafet:
Gómer, Magog, Madai, Javan, Tubal, Mésec, and Tirás. \footnote{\textbf{1:5}
  Gén 10,2-5} \bibleverse{6} Los hijos de Gomer: Asquenaz,
Rifat,\footnote{\textbf{1:6} O Difat.} y Togarma. \bibleverse{7} Los
hijos de Javán: Elisá, Tarsis, Quitín, Rodanín.

\hypertarget{los-camitas}{%
\subsection{Los camitas}\label{los-camitas}}

\bibleverse{8} Los hijos de Cam: Cus,\footnote{\textbf{1:8} O
  ``Sudán/Etiopía''.} Mizrayin,\footnote{\textbf{1:8} O ``Egipto''.}
Fut, y Canaán. \bibleverse{9} Los hijos de Cus: Seba, Javilá, Sabta,
Ragama y Sabteca. Los hijos de Ragama: Sabá y Dedán. \bibleverse{10} Cus
fue el padre de Nimrod, que se convirtió en el primer tirano del mundo.
\bibleverse{11} Mizrayin fue el padre de los ludeos, anameos, leabitas,
naftuitas, \bibleverse{12} patruseos, caslujitas y los caftoritas
(quienes eran antepasados de los filisteos). \bibleverse{13} Canaán fue
el padre Sidón, su primogénito, y de los hititas, \bibleverse{14}
jebuseos, amorreos, gergeseos, \bibleverse{15} heveos, araceos, sineos,
\bibleverse{16} arvadeos, zemareos y jamatitas.

\hypertarget{los-semitas}{%
\subsection{Los semitas}\label{los-semitas}}

\bibleverse{17} Los hijos de Sem: Elam, Asur, Arfaxad, Lud y Aram. Los
hijos de Aram:\footnote{\textbf{1:17} Algunos manuscritos de la
  Septuaginta: esta línea está ausente en la mayoría de los manuscritos
  hebreos. Véase Génesis 10:23.} Uz, Hul, Guéter, y Mésec. \footnote{\textbf{1:17}
  Gén 10,21-31} \bibleverse{18} Arfaxad fue el padre de Selá, y Selá el
padre de Éber. \bibleverse{19} Éber tuvo dos hijos. Uno se llamaba
Peleg,\footnote{\textbf{1:19} La palabra significa ``dividido''.} porque
en su tiempo la tierra fue dividida; el nombre de su hermano fue Joctán.
\bibleverse{20} Joctán fue el padre de Almodad, Sélef, Jazar Mávet,
Yeraj, \bibleverse{21} Adoram, Uzal, Diclá, \bibleverse{22}
Obal,\footnote{\textbf{1:22} La mayoría de los manuscritos lo llaman
  Ebal, pero véase Génesis 10:28.} Abimael, Sabá, \bibleverse{23} Ofir,
Javilá y Jobab. Todos estos fueron hijos de Joctán.

\hypertarget{la-luxednea-recta-de-sem-a-abraham}{%
\subsection{La línea recta de Sem a
Abraham}\label{la-luxednea-recta-de-sem-a-abraham}}

\bibleverse{24} Sem, Arfaxad,\footnote{\textbf{1:24} Algunos manuscritos
  de la Septuaginta añaden aquí ``Cainán''.} Selá, \bibleverse{25} Éber,
Peleg, Reú, \bibleverse{26} Serug, Najor, Téraj, \bibleverse{27} y Abram
(también llamado Abrahán).

\hypertarget{los-ismaelitas}{%
\subsection{Los ismaelitas}\label{los-ismaelitas}}

\bibleverse{28} Los hijos de Abrahán: Isaac e Ismael. \footnote{\textbf{1:28}
  Gén 21,3; Gén 16,16} \bibleverse{29} Estos fueron sus descendientes:
Nebayot, quien fue el hijo primogénito de Ismael, Cedar, Adbeel, Mibsam,
\footnote{\textbf{1:29} Gén 25,13-16} \bibleverse{30} Mismá, Dumá, Masá,
Hadad, Temá, \bibleverse{31} Jetur, Nafis y Cedema. Estos fueron los
hijos de Ismael.

\hypertarget{los-descendientes-de-ketura}{%
\subsection{Los descendientes de
Ketura}\label{los-descendientes-de-ketura}}

\bibleverse{32} Los hijos que le nacieron a Cetura, la concubina de
Abrahán. Ella dio a luz a: Zimrán, Jocsán, Medán, Madián, Isbac y Súah.
Los hijos de Jocsán: Sabá y Dedán. \footnote{\textbf{1:32} Gén 25,1-3}
\bibleverse{33} Los hijos de Madián: Efá, Éfer, Janoc, Abidá y Eldá.
Todos ellos fueron descendientes de Cetura.

\hypertarget{los-descendientes-de-esauxfa}{%
\subsection{Los descendientes de
Esaú}\label{los-descendientes-de-esauxfa}}

\bibleverse{34} Abrahán fue el padre de Isaac. Los hijos de Isaac fueron
Esaú e Israel. \bibleverse{35} Los hijos de Esaú: Elifaz, Reuel, Jeús,
Jalán y Coré. \footnote{\textbf{1:35} Gén 36,10-19} \bibleverse{36} Los
hijos de Elifaz: Temán, Omar, Zefo,\footnote{\textbf{1:36} La mayoría de
  los manuscritos hebreos tienen ``Zefi'', pero véase Génesis 36:11.}
Gatán y Quenaz; además Amalec por medio de Timná.\footnote{\textbf{1:36}
  Según algunos manuscritos de la Septuaginta, Timna era la concubina de
  Elifaz (véase Génesis 36:12).} \bibleverse{37} Los hijos de Reuel:
Najat, Zera, Sama y Mizá.

\bibleverse{38} Los hijos de Seír: Lotán, Sobal, Zibeón, Aná, Disón,
Ezer y Disán. \bibleverse{39} Los hijos de Lotán: Horí y Homán. La
hermana de Lotán era Timná. \bibleverse{40} Los hijos de Sobal:
Alván,\footnote{\textbf{1:40} En la mayoría de los manuscritos hebreos
  dice ``Alian'', pero algunos manuscritos hebreos y de la Septuaginta
  lo presentan como ``Alván''. Véase Génesis 36:23.} Manajat, Ebal, Sefó
y Onam. Los hijos de Zibeón: Aja y Aná. \bibleverse{41} El hijo de Aná
fue Disón. Los hijos de Dishón fueron Hemdán,\footnote{\textbf{1:41} En
  la mayoría de los manuscritos hebreos dice ``Hamran'', pero algunos
  manuscritos hebreos y de la Septuaginta dicen ``Hemdán''. Véase
  Génesis 36:26.} Esbán, Itrán y Querán. \bibleverse{42} Los hijos de
Ezer: Bilán, Zaván y Acán.\footnote{\textbf{1:42} En la mayoría de los
  manuscritos hebreos aparece como ``Zaván'' o ``Jacán'', pero algunos
  manuscritos hebreos y de la Septuaginta lo tienen como ``Acán''. Véase
  Génesis 36:27.} Los hijos de Disán:\footnote{\textbf{1:42} O
  ``Disón''.} Uz y Arán.

\hypertarget{los-reyes-y-jefes-edomitas}{%
\subsection{Los reyes y jefes
edomitas}\label{los-reyes-y-jefes-edomitas}}

\bibleverse{43} Estos fueron los reyes que reinaron sobre Edom antes de
que cualquier rey israelita reinara sobre ellos: Bela hijo de Beor, cuya
ciudad se llamaba Dinaba. \footnote{\textbf{1:43} Gén 36,31-43}

\bibleverse{44} Cuando murió Bela, Jobab hijo de Zera, proveniente de
Bosra, asumió el reinado. \bibleverse{45} Tras la muerte de Jobab, Husam
asumió el reinado entonces, y era proveniente de la tierra de los
Temanitas. \bibleverse{46} Cuando murió Husam, Hadad, hijo de Bedad,
asumió el reinado. Él fue quien derrotó a Madián en el país de Moab. El
nombre de su ciudad era Avit. \bibleverse{47} Cuando murió Hadad, Samá,
de Masreca, asumió el reinado. \bibleverse{48} Cuando murió Samá, Saúl,
proveniente de Rehobot del río\footnote{\textbf{1:48} Probablemente el
  Río Éufrates. Véase Génesis 10:11.} asumió el reinado. \bibleverse{49}
Cuando murió Saúl, Baal-Hanán, hijo de Acbor, asumió el reinado.
\bibleverse{50} Cuando Baal-Hanán murió, Hadad reinó en su lugar. El
nombre de su ciudad era Pau.\footnote{\textbf{1:50} En la mayoría de los
  manuscritos hebreos ``Pai'', pero algunos manuscritos hebreos y de la
  Septuaginta tienen ``Pau''. Véase Génesis 36:39.} El nombre de su
esposa era Mehetabel, hija de Matred, nieta de Me-Zahab. \bibleverse{51}
Entonces murió Hadad. Los jefes de Edom\footnote{\textbf{1:51} La lista
  de nombres cambia de reyes a jefes, ya que después de esta época Edom
  estaba bajo el dominio de Israel y por lo tanto no tenía su propio
  rey.} eran: Timná, Alva, Jetet, \bibleverse{52} Aholibama, Ela, Pinón,
\bibleverse{53} Quenaz, Temán, Mibzar, \bibleverse{54} Magdiel, e Iram.
Estos eran los jefes de Edom.

\hypertarget{los-hijos-de-jacob-israel-y-las-familias-de-la-tribu-de-juduxe1}{%
\subsection{Los hijos de Jacob Israel y las familias de la tribu de
Judá}\label{los-hijos-de-jacob-israel-y-las-familias-de-la-tribu-de-juduxe1}}

\hypertarget{section-1}{%
\section{2}\label{section-1}}

\bibleverse{1} Estos fueron Los hijos de Israel: Rubén, Simeón, Leví,
Judá, Isacar, Zabulón, \bibleverse{2} Dan, José, Benjamín, Neftalí, Gad
y Aser.

\hypertarget{de-juduxe1-a-hezruxf3n}{%
\subsection{De Judá a Hezrón}\label{de-juduxe1-a-hezruxf3n}}

\bibleverse{3} Los hijos de Judá: Er, Onán y Selá: a estos tres los dio
a luz la hija de Súa, una mujer cananea. Er, el primogénito de Judá, era
malvado ante los ojos del Señor, por lo que le quitó la vida.
\footnote{\textbf{2:3} Gén 38,1-7} \bibleverse{4} Tamar era la nuera de
Judá, y le dio a luz a Fares y a Zera. Judá tuvo un total de cinco
hijos. \footnote{\textbf{2:4} Gén 38,29-30}

\bibleverse{5} Los hijos de Fares: Hezrón y Hamul. \footnote{\textbf{2:5}
  Gén 46,12} \bibleverse{6} Los hijos de Zera: Zimri, Etán, Hemán,
Calcol y Darda\footnote{\textbf{2:6} En la mayoría de los manuscritos
  hebreos ``Dara'', pero algunos manuscritos de la Septuaginta tienen
  ``Darda''. Véase 1 Reyes 4:31.} para un total de cinco. \bibleverse{7}
El hijo de Carmi: Acar,\footnote{\textbf{2:7} En el libro de Josué se le
  llama Acán. Véase Josué 7.} que le causó problemas a Israel al ser
infiel y tomar lo que estaba consagrado para el Señor. \bibleverse{8} El
hijo de Etán: Azarías.

\hypertarget{de-hezron-a-david-la-luxednea-ram}{%
\subsection{De Hezron a David (la línea
Ram)}\label{de-hezron-a-david-la-luxednea-ram}}

\bibleverse{9} Los hijos que le nacieron a Hezrón: Jerameel, Ram y
Caleb.\footnote{\textbf{2:9} Literalmente, ``Quelubai''.} \footnote{\textbf{2:9}
  Rut 4,19-22; Mat 1,3; 1Cró 2,18; 1Cró 2,42} \bibleverse{10} Ram fue el
padre de Aminadab, y Aminadab fue el padre de Naasón, un líder de los
descendientes de Judá. \bibleverse{11} Naasón fue el padre de Salmón,
\footnote{\textbf{2:11} Lectura de la Septuaginta. El hebreo es
  ``Salma'', pero véase Rut 4:21.} Salmón fue el padre de Booz,
\bibleverse{12} Booz fue el padre de Obed, y Obed fue el padre de Isaí.
\bibleverse{13} Isaí fue el padre de su hijo primogénito Eliab; el
segundo hijo fue Abinadab, el tercero Simea, \bibleverse{14} el cuarto
Netanel, el quinto Raddai, \bibleverse{15} el sexto Ozem y el séptimo
David. \footnote{\textbf{2:15} 1Sam 17,12} \bibleverse{16} Sus hermanas
fueron Zeruiah y Abigail. Los hijos de Sarvia fueron Abisai, Joab y
Asael, tres en total. \footnote{\textbf{2:16} 2Sam 2,18} \bibleverse{17}
Abigail dio a luz a Amasa, y el padre de Amasa fue Jeter el ismaelita.
\footnote{\textbf{2:17} 2Sam 17,25}

\hypertarget{la-luxednea-caleb}{%
\subsection{La línea Caleb}\label{la-luxednea-caleb}}

\bibleverse{18} Caleb hijo de Hezrón tuvo hijos de su esposa Azuba, y
también de Jeriot. Estos fueron sus hijos Jeser, Sobab y Ardón.
\footnote{\textbf{2:18} 1Cró 2,9; 1Cró 2,42} \bibleverse{19} Cuando
Azuba murió, Caleb tomó a Efrat\footnote{\textbf{2:19} También llamada
  Efrata en 2:50, 4:4.} para que fuera su esposa, y ella le dio a luz a
Hur. \footnote{\textbf{2:19} 1Cró 2,50} \bibleverse{20} Hur fue el padre
de Uri, y Uri fue el padre de Bezalel. \footnote{\textbf{2:20} Éxod 31,2}

\bibleverse{21} Más tarde, Hezrón se acostó con la hija de Maquir, padre
de Galaad, con quien se casó cuando tenía sesenta años, y ella le dio a
luz a Segub. \bibleverse{22} Segub fue el padre de Jair, que tenía
veintitrés ciudades en Galaad. \footnote{\textbf{2:22} Jue 10,3}
\bibleverse{23} Pero Gesur y Aram les quitaron las ciudades de Havvoth
Jair, junto con Kenat y sus ciudades, para un total de sesenta ciudades.
Todos ellos eran descendientes de Maquir, el padre de Galaad.
\footnote{\textbf{2:23} 1Re 4,13} \bibleverse{24} Después de la muerte
de Hezrón en Caleb Efrata, su esposa Abías dio a luz a Asur, padre de
Tecoa. \footnote{\textbf{2:24} 1Cró 4,5}

\hypertarget{la-luxednea-jerameel}{%
\subsection{La línea Jerameel}\label{la-luxednea-jerameel}}

\bibleverse{25} Los hijos de Jerajmeel, primogénito de Hezrón: Ram
(primogénito), Bunah, Oren, Ozem y Ahías. \footnote{\textbf{2:25} 1Cró
  2,9} \bibleverse{26} Jerajmeel tuvo otra esposa llamada Atara. Ella
fue la madre de Onam. \bibleverse{27} Los hijos de Ram el primogénito de
Jerajmeel: Maaz, Jamín y Equer. \bibleverse{28} Los hijos de Onam:
Samaiy Jada. Los hijos de Samai: Nadab y Abisur. \bibleverse{29} La
mujer de Abisur se llamaba Abihail, y dio a luz a Ahbán y Molid.
\bibleverse{30} Los hijos de Nadab: Seled y Appaim. Seled murió sin
tener hijos. \bibleverse{31} El hijo de Apaim: Isi, el padre de Sesán.
Sesán fue el padre de Ahlai. \bibleverse{32} Los hijos de Jada, el
hermano de Samai: Jeter y Jonathan. Jeter murió sin tener hijos.
\bibleverse{33} Los hijos de Jonatán: Pelet y Zaza. Estos son todos los
descendientes de Jerajmeel. \bibleverse{34} Sesán no tenía hijos, sino
que sólo tenía hijas, pero tenía un siervo egipcio llamada Jarha.
\bibleverse{35} Así que Sesán dio su hija en matrimonio a su siervo
Jarha, y ella le dio a luz a Atai. \bibleverse{36} Atai fue el padre de
Natán. Natán fue el padre de Zabad, \bibleverse{37} Zabad fue el padre
de Eflal, Eflal fue el padre de Obed, \bibleverse{38} Obed fue el padre
de Jehú, Jehú fue el padre de Azarías, \bibleverse{39} Azarías fue el
padre de Heles, Heles fue el padre de Eleasá, \bibleverse{40} Eleasá fue
el padre de Sismai, Sismai fue el padre de Salum, \bibleverse{41} Salum
fue el padre de Jecamías, y Jecamías fue el padre de Elisama.

\hypertarget{la-luxednea-caleb-1}{%
\subsection{La línea Caleb}\label{la-luxednea-caleb-1}}

\bibleverse{42} Los hijos de Caleb, hermano de Jerameel: Mesha, su
primogénito, que fue el padre de Zif, y su hijo Maresa, que fue el padre
de Hebrón. \footnote{\textbf{2:42} 1Cró 2,18} \bibleverse{43} Los hijos
de Hebrón: Coré, Tapuá, Requem y Sema. \bibleverse{44} Sema fue el padre
de Raham, y Raham el padre de Jorcoam. Requem fue el padre de Samai.
\bibleverse{45} El hijo de Samai fue Maón, y Maón fue el padre de Bet
Sur. \bibleverse{46} Efá, concubina de Caleb, fue la madre de Harán,
Mosa y Gazez. Harán fue el padre de Gazez. \bibleverse{47} Los hijos de
Jahdai: Regem, Jotam, Gesam, Pelet, Efá y Saaf. \bibleverse{48} Maaca,
concubina de Caleb, fue madre de Seber y de Tirhana. \bibleverse{49}
También fue madre de Saaf, padre de Madmaná, y de Seva, padre de Macbena
y Gibea. La hija de Caleb fue Acsa.

\bibleverse{50} Estos fueron todos los descendientes de Caleb. Los hijos
de Hur, primogénito de Efrata: Sobal, padre de Quiriat Jearim,
\footnote{\textbf{2:50} 1Cró 2,19} \bibleverse{51} Salma, padre de
Belén, y Haref, padre de Bet Gader. \bibleverse{52} Los descendientes de
Sobal, padre de Quiriat Jearim, fueron: Haroe, la mitad de los
manahetitas, \bibleverse{53} y las familias de Quiriat Jearim: los
itritas, los futitas, los sumatitas y los misraítas. De ellos
descendieron los zoratitas y los estaolitas. \bibleverse{54} Los
descendientes de Salma: Belén, los netofatitas, Atrot Bet Joab, la mitad
de los manaítas, los zoritas, \footnote{\textbf{2:54} 1Cró 9,16}
\bibleverse{55} y las familias de escribas que vivían en Jabes: los
tirateos, los simeateos y los sucateos. Estos fueron los ceneos que
descendían de Hamat, el padre de la casa de Recab.\footnote{\textbf{2:55}
  Jue 1,16; Jer 35,-1}

\hypertarget{los-hijos-de-david}{%
\subsection{Los hijos de David}\label{los-hijos-de-david}}

\hypertarget{section-2}{%
\section{3}\label{section-2}}

\bibleverse{1} Estos fueron los hijos de David que le nacieron en
Hebrón: El primogénito fue Amnón, cuya madre fue Ahinoam de Jezreel. El
segundo fue Daniel, cuya madre fue Abigail de Carmelo. \footnote{\textbf{3:1}
  2Sam 3,2-5} \bibleverse{2} El tercero fue Absalón, cuya madre fue
Maaca, hija de Talmai, rey de Gesur. El cuarto fue Adonías, cuya madre
fue Haguit. \bibleverse{3} El quinto fue Sefatías, cuya madre fue
Abital. El sexto fue Itream, cuya madre fue Egla, esposa de David.
\bibleverse{4} Esos fueron los seis hijos que le nacieron a David en
Hebrón, donde reinó siete años y seis meses. David reinó en Jerusalén
treinta y tres años más, \bibleverse{5} y estos fueron los hijos que le
nacieron allí: Samúa,\footnote{\textbf{3:5} En realidad Simea, una
  escritura diferente de Samúa.} Sobab, Natán y Salomón. La madre de
ellos fue Betsabé,\footnote{\textbf{3:5} En realidad Batsúa, una
  escritura diferente de Betsabé.} hija de Ammiel. \bibleverse{6} Además
estaban también Ibhar, Elisúa,\footnote{\textbf{3:6} En realidad
  Elisama, una escritura diferente de Elisúa.} Elifelet, \bibleverse{7}
Noga, Nefeg, Jafía, \bibleverse{8} Elisama, Eliada y Elifelet, un total
de nueve. \bibleverse{9} Todos estos fueron los hijos de David, aparte
de sus hijos de sus concubinas. Su hermana era Tamar. \footnote{\textbf{3:9}
  2Sam 13,1}

\hypertarget{los-reyes-davuxeddicos-desde-salomuxf3n-hasta-la-destrucciuxf3n-de-jerusaluxe9n}{%
\subsection{Los reyes davídicos desde Salomón hasta la destrucción de
Jerusalén}\label{los-reyes-davuxeddicos-desde-salomuxf3n-hasta-la-destrucciuxf3n-de-jerusaluxe9n}}

\bibleverse{10} El linaje masculino\footnote{\textbf{3:10} ``Linaje
  masculino'': se utiliza este término en lugar de decir repetidamente
  ``su hijo''.} desde Salomón fue: Roboam, Abías, Asa, \footnote{\textbf{3:10}
  Mat 1,7-12} \bibleverse{11} Joram,\footnote{\textbf{3:11} De hecho
  dice Joram, una escritura diferente de Jehoram.} Ocozías, Joás,
\bibleverse{12} Amasías, Azarías, Jotam, \bibleverse{13} Acaz, Ezequías,
Manasés, \bibleverse{14} Amón, Josías. \bibleverse{15} Los hijos de
Josías: Johanán (primogénito), Joaquín (el segundo), Sedequías (el
tercero), y Salum (el cuarto). \bibleverse{16} Los hijos de Joacim:
Joaquín\footnote{\textbf{3:16} De hecho dice Jeconías, una ortografía
  diferente de Joaquín.} y Sedecías.

\hypertarget{los-otros-descendientes-de-david-desde-jechonja-en-adelante}{%
\subsection{Los otros descendientes de David (desde Jechonja en
adelante)}\label{los-otros-descendientes-de-david-desde-jechonja-en-adelante}}

\bibleverse{17} Los hijos de Joaquín que fueron llevados al cautiverio:
Sealtiel, \footnote{\textbf{3:17} 2Cró 36,10}

\bibleverse{18} Malquiram, Pedaías, Senaza, Jecamías, Hosama y Nedabías.
\bibleverse{19} Los hijos de Pedaías: Zorobabel y Simei. Los hijos de
Zorobabel: Mesulam y Hananías. Su hermana era Selomit. \bibleverse{20}
Otros cinco hijos fueron: Hasuba, Ohel, Berequías, Hasadías y
Jushab-Hesed. \bibleverse{21} Los hijos de Hananías: Pelatías y Jesaías,
y Los hijos de Refaías, Los hijos de Arnán, Los hijos de Abdías, y Los
hijos de Secanías.\footnote{\textbf{3:21} El texto tiene dificultades de
  interpretación.} \bibleverse{22} Los hijos de Secanías: Semaías y sus
hijos: Hatús, Igal, Barías, Nearías y Safat. En total eran seis.
\bibleverse{23} Los hijos de Nearías: Elioenai, Ezequías y Azricam. Eran
tres en total. \bibleverse{24} Los hijos de Elioenai: Hodavías, Eliasib,
Pelaías, Acub, Johanán, Dalaías, y Anani. En total eran un total de
siete.

\hypertarget{muxe1s-informaciuxf3n-sobre-las-familias-de-la-tribu-de-juduxe1}{%
\subsection{Más información sobre las familias de la tribu de
Judá}\label{muxe1s-informaciuxf3n-sobre-las-familias-de-la-tribu-de-juduxe1}}

\hypertarget{section-3}{%
\section{4}\label{section-3}}

\bibleverse{1} Los hijos de Judá fueron Fares, Hezrón, Carmi, Hur y
Sobal. \footnote{\textbf{4:1} 1Cró 2,4-5; 1Cró 2,7; 1Cró 2,19; 1Cró 2,50}
\bibleverse{2} Reaía, hijo de Sobal, fue el padre de Jahath. Jahat fue
el padre de Ahumai y Lahad. Estas fueron las familias de los zoratitas.
\footnote{\textbf{4:2} 1Cró 2,53} \bibleverse{3} Estos fueron los
hijos\footnote{\textbf{4:3} ``Hijos'': el texto hebreo dice ``padre'',
  pero algunos manuscritos de la Septuaginta y la Vulgata dicen
  ``hijos''.} de Etam: Jezreel, Isma e Ibdas. Su hermana se llamaba
Haze-lelponi. \bibleverse{4} Penuel fue el padre de Gedor, y Ezer fue el
padre de Husa. Estos fueron los descendientes de Hur, primogénito de
Efrata y padre\footnote{\textbf{4:4} ``Padre'': probablemente en el
  sentido de ``fundador''.} de Belén. \footnote{\textbf{4:4} 1Cró 2,19;
  1Cró 2,50} \bibleverse{5} Asur fue el padre de Tecoa y tuvo dos
esposas, Helá y Naara. \bibleverse{6} Naara fue la madre de Ahuzam,
Hefer, Temeni y Ahastari. Estos fueron los hijos de Naara.
\bibleverse{7} Los hijos de Hela: Zeret, Zohar, Etnán, \bibleverse{8} y
Cos, que fue el padre de Anub y Zobeba, y de las familias de Aharhel,
hijo de Harum.

\bibleverse{9} Jabes fue más fiel a Dios\footnote{\textbf{4:9} ``Más
  fiel a Dios'': Literalmente, ``más honorable'', pero esto no conlleva
  el significado de una mejor relación con Dios.} que sus hermanos. Su
madre le había puesto el nombre de Jabes, diciendo: ``Lo di a luz con
dolor''.

\bibleverse{10} Jabes suplicó al Dios de Israel: ``¡Por favor, bendíceme
y amplía mis fronteras!\footnote{\textbf{4:10} ``Amplía mis fronteras'':
  o, ``extiende mi territorio''. Aunque esto puede ser visto como una
  simple petición de mayor propiedad de la tierra, es quizás mejor
  entender esta petición de que Dios expanda todo lo que Jabes tenía,
  incluyendo los aspectos espirituales.} Acompáñame y mantenme a salvo
de cualquier daño para que no tenga dolor''.\footnote{\textbf{4:10}
  ``Dolor'': parte de la oración es el deseo de que, a pesar del nombre
  que le dio su madre, no se le maldiga para que sufra dolor.} Y Dios le
dio lo que pidió.

\bibleverse{11} Quelub, hermano de Súa, fue el padre de Mehir, quien a
su vez fue el padre de Estón. \bibleverse{12} Estón fue el padre de
Bet-Rafa, Paseah y Tehina, el padre\footnote{\textbf{4:12} ``Padre'':
  probablemente en el sentido de ``fundador''. Ir-Nahas significa
  ``ciudad de la serpiente''} de Ir-Nahas. Estos fueron los hombres de
Reca.\footnote{\textbf{4:12} ``Reca''. En algunos manuscritos se lee
  ``Recab'', en cuyo caso se referiría a los mencionados en 2:55.}
\bibleverse{13} Los hijos de Kenaz: Otoniel y Seraías. Los hijos de
Otoniel: Hatat y Meonothai.\footnote{\textbf{4:13} ``Meonothai'':
  algunos manuscritos de la Septuaginta y la Vulgata. El texto hebreo
  actual no tiene la palabra, probablemente perdida porque ocurre como
  la primera palabra del siguiente verso.} \footnote{\textbf{4:13} Jos
  15,17; Jue 1,13} \bibleverse{14} Meonothai fue el padre de Ofra.
Seraías fue el padre de Joab, el padre\footnote{\textbf{4:14} ``Padre'':
  probablemente en el sentido de ``fundador''. Gue-Jarasim significa
  ``valle de los artesanos''.} de Gue-Harashim, llamado así porque allí
vivían artesanos. \bibleverse{15} Los hijos de Caleb hijo de Jefone:
Iru, Ela y Naam. El hijo de Elah: Kenaz. \bibleverse{16} Los hijos de
Jehalelel: Zif, Zifa, Tirías y Asarel. \bibleverse{17} Los hijos de
Esdras: Jeter, Mered, Efer y Jalón. Una de las esposas de
Mered\footnote{\textbf{4:17} ``Mered'': Se asume por el versículo
  siguiente.} fue la madre de Miriam, Samai e Ishbah el padre de
Estemoa.\footnote{\textbf{4:17} ``Padre'': en el sentido de ``fundador''
  de la ciudad de ese nombre.} \bibleverse{18} (Otra esposa que vino de
Judá fue la madre de Jered el padre de Gedor, Heber el padre de Soco, y
Jecuthiel el padre de Zanoa.\footnote{\textbf{4:18} ``Padre'': cada uno
  se refiere al ``fundador'' de las respectivas ciudades. Véase Josué
  15.} ) Estos eran Los hijos de Bitia, la hija del Faraón, con quien
Mered se había casado.\footnote{\textbf{4:18} Es de suponer que se
  refiere a los hijos mencionados en el verso anterior.} \bibleverse{19}
Los hijos de la esposa de Hodías, hermana de Natán: un hijo fue el padre
de Keila la Garmita, y otro el padre de Estemoa la Maacatea.
\bibleverse{20} Los hijos de Simón: Amnón, Rinah, Ben-Hanan y Tilón. Los
hijos de Isi: Zohet y Ben-Zohet. \bibleverse{21} Los hijos de Selá hijo
de Judá: Er, que fue el padre de Leca; Laada, que fue el padre de
Maresa; las familias de los trabajadores del lino en Beth Asbea;
\footnote{\textbf{4:21} 1Cró 2,3} \bibleverse{22} Jaocim, los hombres de
Cozeba, y Joas y Saraf, que gobernaron sobre Moab y Jasubi-Lehem.
\bibleverse{23} Eran alfareros, habitantes de Netaim y Gedera, que
vivían allí y trabajaban para el rey.

\hypertarget{informaciuxf3n-sobre-los-descendientes-de-simeuxf3n}{%
\subsection{Información sobre los descendientes de
Simeón}\label{informaciuxf3n-sobre-los-descendientes-de-simeuxf3n}}

\bibleverse{24} Los hijos de Simeón: Nemuel, Jamín, Jarib, Zera y Saúl.
\bibleverse{25} Salum era hijo de Saúl, Mibsam su hijo y Mismá su hijo.
\bibleverse{26} Los hijos de Mismá: Hamuel su hijo, Zacur su hijo y
Simei su hijo. \bibleverse{27} Simei tuvo dieciséis hijos y seis hijas,
pero sus hermanos no tuvieron muchos hijos, por lo que su tribu no fue
tan numerosa como la de Judá.

\hypertarget{las-residencias-muxe1s-antiguas-de-la-tribu}{%
\subsection{Las residencias más antiguas de la
tribu}\label{las-residencias-muxe1s-antiguas-de-la-tribu}}

\bibleverse{28} Vivían en Beerseba, Molada, Hazar Sual, \footnote{\textbf{4:28}
  Jos 19,2-8} \bibleverse{29} Bilha, Ezem, Tolad, \bibleverse{30}
Betuel, Horma, Ziclag, \bibleverse{31} Bet Marcabot, Hazar Susim, Bet
Birai y Saaraim. Estas fueron sus ciudades hasta que David llegó a ser
rey. \bibleverse{32} También vivían en Etam, Ain, Rimón, Toquén y Asán,
un total de cinco ciudades, \bibleverse{33} junto con todas las aldeas
de los alrededores hasta Baal.\footnote{\textbf{4:33} Véase Josué 19:8.}
Estos fueron los lugares donde vivieron y registraron su genealogía.

\hypertarget{indicaciuxf3n-de-otros-jefes-de-familia-simeonitas-las-dos-conquistas-de-los-simeonitas}{%
\subsection{Indicación de otros jefes de familia simeonitas; las dos
conquistas de los
simeonitas}\label{indicaciuxf3n-de-otros-jefes-de-familia-simeonitas-las-dos-conquistas-de-los-simeonitas}}

\bibleverse{34} Mesobab, Jamlec, Josá, hijo de Amasías, \bibleverse{35}
Joel, Jehú, hijo de Josibías, hijo de Seraías, hijo de Asiel,
\bibleverse{36} Elioenai, Jaacoba, Jesohaía, Asaías, Adiel, Jesimiel,
Benaía, \bibleverse{37} y Ziza, hijo de Sifi, hijo de Alón, hijo de
Jedaiah, hijo de Shimri, hijo de Semaías. \bibleverse{38} Estos fueron
los nombres de los jefes de sus familias, cuyo linaje aumentó
considerablemente.

\bibleverse{39} Llegaron hasta la frontera de Gedor, en el lado oriental
del valle, para buscar pastos para sus rebaños. \bibleverse{40} Allí
encontraron buenos pastos, y la zona era abierta, tranquila y apacible,
pues los que vivían allí eran descendientes de Cam.\footnote{\textbf{4:40}
  ``Descendientes de Cam'': es decir, los antiguos habitantes cananeos.}
\bibleverse{41} En la época de Ezequías, rey de Judá, los líderes
mencionados por su nombre vinieron y atacaron a estos descendientes de
Cam donde vivían, junto con los meunitas de allí y los destruyeron
totalmente, como está claro hasta el día de hoy. Luego se establecieron
allí, porque había pastizales para sus rebaños. \footnote{\textbf{4:41}
  2Re 18,1}

\bibleverse{42} Algunos de estos simeonitas invadieron el monte de Seir:
quinientos hombres dirigidos por Pelatías, Nearías, Refaías y Uziel, los
hijos de Isi. \bibleverse{43} Destruyeron al resto de los amalecitas que
habían escapado. Ellos han vivido allí hasta el día de hoy.

\hypertarget{informaciuxf3n-sobre-rubuxe9n-y-sus-descendientes}{%
\subsection{Información sobre Rubén y sus
descendientes}\label{informaciuxf3n-sobre-rubuxe9n-y-sus-descendientes}}

\hypertarget{section-4}{%
\section{5}\label{section-4}}

\bibleverse{1} Los hijos de Rubén, el primogénito de Israel. (Aunque era
el primogénito, su primogenitura fue entregada a los hijos de José hijo
de Israel porque había profanado el lecho de su padre.\footnote{\textbf{5:1}
  Rubén se había acostado con Bilhá, la concubina de Jacob. Génesis
  35:22, Génesis 49:4.} Por eso Rubén no figura en la genealogía según
la primogenitura, \footnote{\textbf{5:1} Gén 35,22; Gén 49,4}
\bibleverse{2} y aunque Judá llegó a ser el más fuerte de sus hermanos y
de él salió un gobernante, la primogenitura le pertenecía a José).
\footnote{\textbf{5:2} Gén 49,8; Gén 49,10; Gén 49,22; Deut 33,7; Deut
  33,13-17} \bibleverse{3} Los hijos de Rubén el primogénito de Israel:
Hanoc, Falú, Hezrón y Carmi. \footnote{\textbf{5:3} Éxod 6,14}
\bibleverse{4} Los hijos de Joel: Semaías su hijo, Gog su hijo, Simei su
hijo, \bibleverse{5} Miqueas su hijo, Reaías su hijo, Baal su hijo,
\bibleverse{6} y Beera su hijo, el que Tiglat-Pileser el rey de Asiria
llevó al exilio. Él (Beera) era un líder de los rubenitas.
\bibleverse{7} Los parientes de Beera son, según sus registros
genealógicos: por familia Jeiel (el jefe de familia), Zacarías,
\bibleverse{8} y Bela de Azaz, hijo de Sema, hijo de Joel. Ellos
habitaban en la zona que iba desde Aroer hasta Nebo y Baal Meón.

\hypertarget{informaciuxf3n-histuxf3rica-sobre-bela}{%
\subsection{Información histórica sobre
Bela}\label{informaciuxf3n-histuxf3rica-sobre-bela}}

\bibleverse{9} Por el lado oriental se extendieron por la tierra hasta
el borde del desierto que continúa hasta el río Éufrates, porque sus
rebaños habían crecido mucho en Galaad.

\bibleverse{10} En los tiempos de Saúl fueron a la guerra contra los
agarenos, y los derrotaron. Se apoderaron de los lugares donde habían
vivido los agarenos en todas las regiones al este de Galaad.

\hypertarget{informaciuxf3n-sobre-la-estirpe-y-lugares-de-residencia-asuxed-como-sobre-la-valoraciuxf3n-de-los-gaditas.}{%
\subsection{Información sobre la estirpe y lugares de residencia, así
como sobre la valoración de los
gaditas.}\label{informaciuxf3n-sobre-la-estirpe-y-lugares-de-residencia-asuxed-como-sobre-la-valoraciuxf3n-de-los-gaditas.}}

\bibleverse{11} Junto a ellos, los descendientes de Gad vivían en Basa,
hasta Salca. \bibleverse{12} Joel (el jefe de familia), Safam (el
segundo), y Janai y Safat, en Basán. \bibleverse{13} Sus parientes,
según la familia, eran: Miguel, Mesulán, Sabá, Jorai, Jacán, Zía y Éber,
siendo un total de siete. \bibleverse{14} Estos fueron Los hijos de
Abihail, hijo de Huri, hijo de Jaroa, hijo de Galaad, hijo de Miguel,
hijo de Jesisai, hijo de Jahdo, hijo de Buz. \bibleverse{15} Ahi hijo de
Abdiel, hijo de Guni, era su jefe de familia. \bibleverse{16} Vivían en
Galaad, en Basán y sus ciudades, y en los pastizales de Sarón hasta sus
fronteras. \bibleverse{17} Todos ellos fueron registrados en la
genealogía durante el tiempo de Jotam, rey de Judá, y de Jeroboam, rey
de Israel. \footnote{\textbf{5:17} 2Re 15,32; 2Re 14,23}

\hypertarget{la-lucha-de-las-tres-tribus-de-transjordania-con-los-agaritascon-los-agaritas}{%
\subsection{La lucha de las tres tribus de Transjordania con los
agaritascon los
agaritas}\label{la-lucha-de-las-tres-tribus-de-transjordania-con-los-agaritascon-los-agaritas}}

\bibleverse{18} La tribu de Rubén, la tribu de Gad y la media tribu de
Manasés tenían 44. 760 guerreros fuertes y listos para la batalla,
capaces de usar escudos, espadas y arcos. \bibleverse{19} Fueron a la
guerra contra los agarenos, los jetureos, los nafiseos y los nodabitas.
\bibleverse{20} Recibieron ayuda en la lucha contra estos enemigos
porque invocaron a Dios durante las batallas. Así pudieron derrotar a
los agarenos y a todos los que estaban con ellos. Dios respondió a sus
oraciones porque confiaron en él. \bibleverse{21} Capturaron el ganado
de sus enemigos: cincuenta mil camellos, doscientas cincuenta mil ovejas
y dos mil asnos. También capturaron a cien mil personas, \bibleverse{22}
y muchas otras murieron porque la batalla era de Dios. Se apoderaron de
la tierra y vivieron allí hasta el exilio.

\hypertarget{las-residencias-y-la-divisiuxf3n-de-guxe9nero-de-los-manasitas}{%
\subsection{Las residencias y la división de género de los
manasitas}\label{las-residencias-y-la-divisiuxf3n-de-guxe9nero-de-los-manasitas}}

\bibleverse{23} La media tribu de Manasés había crecido mucho. Vivían en
la tierra desde Basán hasta Baal Hermón, (también conocida como Senir y
Monte Hermón). \bibleverse{24} Estos eran los jefes de familia: Efer,
Isi, Eliel, Azriel, Jeremías, Hodavías y Jahdiel. Eran fuertes
guerreros, hombres famosos, jefes de sus familias.

\hypertarget{castigo-por-la-apostasuxeda-de-las-tres-tribus-de-jordania-oriental-por-los-reyes-asirios}{%
\subsection{Castigo por la apostasía de las tres tribus de Jordania
Oriental por los reyes
asirios}\label{castigo-por-la-apostasuxeda-de-las-tres-tribus-de-jordania-oriental-por-los-reyes-asirios}}

\bibleverse{25} Pero fueron infieles al Dios de sus antepasados. Se
prostituyeron siguiendo a los dioses de los pueblos de la tierra, los
que Dios había destruido antes que ellos. \bibleverse{26} Así que el
Dios de Israel animó a Pul, rey de Asiria, (también conocido como
Tiglat-Pileser, rey de Asiria), para que invadiera la tierra. Llevó al
exilio a los rubenitas, a los gaditas y a la media tribu de Manasés. Los
llevó a Halah, a Habor, a Hara y al río de Gozán, donde permanecen hasta
el día de hoy.\footnote{\textbf{5:26} 2Re 15,19; 2Re 15,29}

\hypertarget{de-levi-a-los-hijos-de-aaruxf3n}{%
\subsection{De Levi a los hijos de
Aarón}\label{de-levi-a-los-hijos-de-aaruxf3n}}

\hypertarget{section-5}{%
\section{6}\label{section-5}}

\bibleverse{1} Los hijos de Levi: Gersón, Coat y Merari. \footnote{\textbf{6:1}
  1Cró 5,27; Éxod 6,16-19} \bibleverse{2} Los hijos de Coat: Amram,
Izhar, Hebrón y Uziel. \bibleverse{3} Los hijos de Amram: Aarón, Moisés
y Miriam. Los hijos de Aarón: Nadab, Abiú, Eleazar e Itamar.

\hypertarget{la-luxednea-de-sumo-sacerdote-desde-eleazar-hasta-el-cautiverio-babiluxf3nico}{%
\subsection{La línea de sumo sacerdote desde Eleazar hasta el cautiverio
babilónico}\label{la-luxednea-de-sumo-sacerdote-desde-eleazar-hasta-el-cautiverio-babiluxf3nico}}

\bibleverse{4} Eleazar fue el padre de Finehas. Finehas fue el padre de
Abisúa. \bibleverse{5} Abisúa fue el padre de Buqui. Buqui fue el padre
de Uzi; \bibleverse{6} Uzi fue el padre de Zeraías. Zeraías fue el padre
de Meraioth. \bibleverse{7} Meraiot fue el padre de Amarías. Amarías fue
el padre de Ajitub. \footnote{\textbf{6:7} Éxod 6,24} \bibleverse{8}
Ajitub fue el padre de Sadok. Sadok fue el padre de Ahimaas.
\bibleverse{9} Ahimaas fue el padre de Azarías. Azarías fue el padre de
Johanán. \bibleverse{10} Johanán fue el padre de Azarías (quien sirvió
como sacerdote cuando Salomón construyó el Templo en Jerusalén).
\bibleverse{11} Azarías fue el padre de Amarías. Amarías fue el padre de
Ajitub. \bibleverse{12} Ajitub fue el padre de Sadoc. Sadoc fue el padre
de Salum. \bibleverse{13} Salum fue el padre de Hilcías. Hilcías fue el
padre de Azarías. \footnote{\textbf{6:13} 1Sam 8,2} \bibleverse{14}
Azarías fue el padre de Seraías, y Seraías fue el padre de Josadac.
\bibleverse{15} Josadac fue llevado al cautiverio cuando el Señor usó a
Nabucodonosor para enviar a Judá y a Jerusalén al exilio.

\hypertarget{los-descendientes-de-levi}{%
\subsection{Los descendientes de Levi}\label{los-descendientes-de-levi}}

\bibleverse{16} Los hijos de Leví: Gersón, Coat y Merari.
\bibleverse{17} Estos son los nombres de los hijos de Gersón: Libni y
Simei. \bibleverse{18} Los hijos de Coat: Amram, Izhar, Hebrón y Uziel.
\bibleverse{19} Los hijos de Merari: Mahli y Musi. Estas son las
familias de los levitas, que estaban ordenadas según sus padres:
\bibleverse{20} Los descendientes de Gersón: Libni su hijo, Jehat su
hijo, Zima su hijo, \bibleverse{21} Joa su hijo, Iddo su hijo, Zera su
hijo y Jeatherai su hijo. \bibleverse{22} Los descendientes de Coat:
Aminadab su hijo; Coré su hijo; Asir su hijo, \bibleverse{23} Elcana su
hijo; Ebiasaf su hijo; Asir su hijo; \bibleverse{24} Tahat su hijo;
Uriel su hijo; Uzías su hijo, y Saúl su hijo. \footnote{\textbf{6:24}
  1Cró 15,17} \bibleverse{25} Los descendientes de Elcana: Amasai,
Ahimot, \bibleverse{26} Elcana su hijo; Zofai su hijo; Nahat su hijo;
\bibleverse{27} Eliab su hijo, Jeroham su hijo; Elcana su hijo, y Samuel
su hijo.\footnote{\textbf{6:27} ``Y Samuel, su hijo'': según algunos
  manuscritos de la Septuaginta. El texto hebreo omite estas palabras.
  Véase 1 Samuel 1:19-20 y 1 Crónicas 6:33-34.} \bibleverse{28} Los
hijos de Samuel: Joel\footnote{\textbf{6:28} ``Joel'': según algunos
  manuscritos de la Septuaginta. El texto hebreo omite esta palabra.
  Véase 1 Samuel 8:2 y 1 Crónicas 6:33.} (primogénito) y Abías (el
segundo). \bibleverse{29} Los descendientes de Merari: Mahli, su hijo
Libni, su hijo Simei, su hijo Uza, \bibleverse{30} su hijo Simea, su
hijo Haguía y su hijo Asaías.

\hypertarget{las-tres-familias-de-cantantes-levuxedticos-hemuxe1n-asaf-y-etuxe1n}{%
\subsection{Las tres familias de cantantes Levíticos, Hemán, Asaf y
Etán}\label{las-tres-familias-de-cantantes-levuxedticos-hemuxe1n-asaf-y-etuxe1n}}

\bibleverse{31} Estos son los músicos que David designó para dirigir la
música en la casa del Señor una vez que el Arca fuera colocada allí.
\bibleverse{32} Ellos dirigieron la música y el canto ante el
Tabernáculo, la Tienda de Reunión, hasta que Salomón construyó el Templo
del Señor en Jerusalén. Servían siguiendo el reglamento que se les había
dado. \bibleverse{33} Estos son los hombres que servían, junto con sus
hijos: De los coatitas Hemán, el cantor, el hijo de Joel, el hijo de
Samuel, \bibleverse{34} hijo de Elcana, hijo de Jeroham, hijo de Eliel,
hijo de Toa, \footnote{\textbf{6:34} Éxod 28,1; Lev 16,-1}
\bibleverse{35} hijo de Zuf, ehijo de Elcana, hijo de Mahat, hijo de
Amasai, \footnote{\textbf{6:35} 1Cró 5,29-34} \bibleverse{36} hijo de
Elcana, hijo de Joel, hijo de Azarías, hijo de Sofonías, \bibleverse{37}
hijo de Tahat, hijo de Asir, hijo de Ebiasaf, hijo de Coré,
\bibleverse{38} hijo de Izhar, hijo de Coat, hijo de Leví, hijo de
Israel.

\bibleverse{39} Asaf, pariente de Hemán, que servía junto a él a la
derecha: Asaf hijo de Berequías, hijo de Simea, \bibleverse{40} hijo de
Miguel, hijo de Baasías, hijo de Malaquías, \bibleverse{41} hijo de
Etni, hijo de Zera, hijo de Adaías, \bibleverse{42} hijo de Etán, hijo
de Zima, hijo de Simei, \bibleverse{43} hijo de Jahat, hijo de Gersón,
hijo de Leví.

\bibleverse{44} A la izquierda de Hemán sirvieron los hijos de Merari:
Etán hijo de Quisi, el hijo de Abdi, hijo de Malluch, \bibleverse{45}
hijo de Hasabiah, hijo de Amasías, hijo de Hilcías, \bibleverse{46} hijo
de Amzi, hijo de Bani, hijo de Semer, \footnote{\textbf{6:46} 1Cró
  6,51-55} \bibleverse{47} hijo de Mahli, hijo de Musi, hijo de Merari,
hijo de Levi. \footnote{\textbf{6:47} 1Cró 6,56-61}

\hypertarget{los-levitas-y-los-aaronitas-en-el-servicio-del-templo}{%
\subsection{Los levitas y los aaronitas en el servicio del
templo}\label{los-levitas-y-los-aaronitas-en-el-servicio-del-templo}}

\bibleverse{48} Los demás levitas desempeñaban todas las demás funciones
en el Tabernáculo, la casa de Dios. \footnote{\textbf{6:48} 1Cró 6,62-66}
\bibleverse{49} Sin embargo, eran Aarón y sus descendientes quienes
daban ofrendas en el altar de los holocaustos y en el altar del incienso
y hacían todo el trabajo en el Lugar Santísimo, haciendo la expiación
por Israel según todo lo que había ordenado Moisés, el siervo de Dios.

\hypertarget{segunda-luxednea-de-sumos-sacerdotes-desde-aaruxf3n-hasta-ahimaas}{%
\subsection{Segunda línea de sumos sacerdotes desde Aarón hasta
Ahimaas}\label{segunda-luxednea-de-sumos-sacerdotes-desde-aaruxf3n-hasta-ahimaas}}

\bibleverse{50} Los descendientes de Aarón fueron: Eleazar su hijo,
Finees su hijo, Abisúa su hijo, \bibleverse{51} Buqui su hijo, Uzi su
hijo, Zeraías su hijo, \bibleverse{52} Meraioth su hijo, Amariah su
hijo, Ahitub su hijo, \bibleverse{53} Zadok su hijo, y su hijo Ahimaas.

\hypertarget{las-ciudades-levitas}{%
\subsection{Las ciudades levitas}\label{las-ciudades-levitas}}

\bibleverse{54} Estos fueron los lugares que se les dieron para vivir
como territorio asignado a los descendientes de Aarón, comenzando por el
clan coatita, porque el suyo fue el primer lote: \bibleverse{55}
Recibieron Hebrón, en Judá, junto con los pastos que la rodean.
\bibleverse{56} Pero los campos y las aldeas cercanas a la ciudad fueron
entregados a Caleb hijo de Jefone. \bibleverse{57} Así, los
descendientes de Aarón recibieron Hebrón, una ciudad de refugio, Libna,
Jatir, Estemoa, \bibleverse{58} Hilén, Debir, \bibleverse{59} Asán,
Juta\footnote{\textbf{6:59} Esta ciudad no aparece en esta lista, pero
  está incluida en Josué 21:16.} y Bet Semes, junto con sus pastizales.
\bibleverse{60} De la tribu de Benjamín recibieron Gabaón,\footnote{\textbf{6:60}
  Esta ciudad no aparece en esta lista, pero está incluida en Josué
  21:17.} Geba, Alemeth y Anathoth, junto con sus pastizales. Tenían un
total de trece ciudades entre sus familias.

\bibleverse{61} Los demás descendientes de Coat recibieron por sorteo
diez ciudades de la media tribu de Manasés. \bibleverse{62} Los
descendientes de Gersón, por familia, recibieron trece ciudades de las
tribus de Isacar, Aser y Neftalí, y de la media tribu de Manasés en
Basán. \bibleverse{63} Los descendientes de Merari, por familia,
recibieron doce ciudades de las tribus de Rubén, Gad y Zabulón.
\bibleverse{64} El pueblo de Israel dio a los levitas estas ciudades y
sus pastizales. \bibleverse{65} A las ciudades ya mencionadas les
asignaron por nombre las de las tribus de Judá, Simeón y Benjamín.

\bibleverse{66} Algunas de las familias coatitas recibieron como
territorio ciudades de la tribu de Efraín. \bibleverse{67} Se les dio
Siquem, una ciudad de refugio, en la región montañosa de Efraín,
Gezer,\footnote{\textbf{6:67} Aquí también se incluye Gezer como ciudad
  de refugio, pero véase Josué 21:21.} \bibleverse{68} Jocmeán,
Bet-Horon, \bibleverse{69} Aijalon y Gat Rimmon, junto con sus
pastizales. \bibleverse{70} De la mitad de la tribu de Manasés, el
pueblo de Israel dio Aner y Bileam, junto con sus pastizales, al resto
de las familias coatitas.

\bibleverse{71} Los descendientes de Gersón recibieron lo siguiente. De
la familia de la media tribu de Manasés Golán en Basán, y Astarot, junto
con sus pastizales; \bibleverse{72} de la tribu de Isacar: Cedes,
Daberat, \bibleverse{73} Ramot y Anem, junto con sus pastizales;
\bibleverse{74} de la tribu de Aser: Masal, Abdón, \bibleverse{75} Hucoc
y Rehob, junto con sus pastizales; \bibleverse{76} y de la tribu de
Neftalí: Cedes en Galilea, Hamón y Quiriatím, con sus pastizales.

\bibleverse{77} Los demás descendientes de Merari recibieron lo
siguiente. De la tribu de Zabulón Jocneam, Carta,\footnote{\textbf{6:77}
  Jocneam y Carta no están incluidos en la lista aquí, pero véase Josué
  21:34.} Rimón y Tabor, con sus pastizales; \bibleverse{78} de la tribu
de Rubén, al este del Jordán, frente a Jericó: Beser (en el desierto),
Jahzá, \bibleverse{79} Cedemot y Mefat, con sus pastizales
\bibleverse{80} y de la tribu de Gad: Ramot de Galaad, Mahanaim,
\bibleverse{81} Hesbón y Jazer, con sus pastizales.

\hypertarget{la-tribu-de-isacar}{%
\subsection{La tribu de Isacar}\label{la-tribu-de-isacar}}

\hypertarget{section-6}{%
\section{7}\label{section-6}}

\bibleverse{1} Los hijos de Isacar: Tola, Púa, Jasub y Simrón, un total
de cuatro. \footnote{\textbf{7:1} Gén 46,13; Núm 26,23-24}
\bibleverse{2} Los hijos de Tola: Uzi, Refaías, Jeriel, Jahmai, Ibsam y
Samuel, quienes eran jefes de sus familias. En la época de David, los
descendientes de Tola enumeraban en su genealogía un total de 22. 600
guerreros. \bibleverse{3} El hijo de Uzi: Israhías. Los hijos de
Israhías: Miguel, Obadías, Joel e Isías. Los cinco eran jefes de
familia. \bibleverse{4} Tenían muchas esposas e hijos, por lo que en su
genealogía figuran 36. 000 hombres de combate listos para la batalla.
\bibleverse{5} Los parientes guerreros de todas las familias de Isacar,
según su genealogía, eran 87. 000 en total.

\hypertarget{la-tribu-de-benjamuxedn}{%
\subsection{La tribu de Benjamín}\label{la-tribu-de-benjamuxedn}}

\bibleverse{6} Tres hijos de Benjamín: Bela, Bequer y Jediael.
\bibleverse{7} Los hijos de Bela: Ezbón, Uzi, Uziel, Jerimot e Iri,
quienes eran jefes de sus familias, y eran un total de cinco. Tenían 22.
034 combatientes según su genealogía. \bibleverse{8} Los hijos de
Bequer: Zemira, Joás, Eliezer, Elioenai, Omrí, Jerimot, Abías, Anatot y
Alemet. Todos ellos fueron los hijos de Bequer. \bibleverse{9} Su
genealogía incluía a los jefes de familia y a 20. 200 combatientes.
\bibleverse{10} El hijo de Jediael: Bilhán. Los hijos de Bilhán: Jeús,
Benjamín, Aod, Quenaana, Zetán, Tarsis y Ahisahar. \bibleverse{11} Todos
estos hijos de Jediael eran jefes de sus familias. Tenían 17. 200
guerreros listos para la batalla. \bibleverse{12} Supim y Hupim eran los
hijos de Ir, y Husim era hijo de Aher.

\hypertarget{la-tribu-de-neftaluxed}{%
\subsection{La tribu de Neftalí}\label{la-tribu-de-neftaluxed}}

\bibleverse{13} Los hijos de Neftalí: Jahziel, Guni, Jezer y
Salum,\footnote{\textbf{7:13} ``Salum'': o ``Silem''.} quienes eran los
descendientes de Bilha. \footnote{\textbf{7:13} Gén 46,24}

\hypertarget{la-tribu-de-manasuxe9s}{%
\subsection{La tribu de Manasés}\label{la-tribu-de-manasuxe9s}}

\bibleverse{14} Los hijos de Manasés: Asriel, cuya madre era su
concubina aramea. También fue la madre de Maquir, el padre de Galaad.
\footnote{\textbf{7:14} Núm 26,29-33} \bibleverse{15} Maquir encontró
una esposa para Hupim y otra para Suppim. Su hermana se llamaba Maaca.
La segunda se llamaba Zelofehad. Él solo tuvo hijas.\footnote{\textbf{7:15}
  El texto hebreo de este verso es muy poco claro.} \footnote{\textbf{7:15}
  Núm 27,1} \bibleverse{16} Maaca, la esposa de Maquir, tuvo un hijo y
lo llamó Peres. Su hermano se llamaba Seres, y sus hijos fueron Ulam y
Raquem. \bibleverse{17} El hijo de Ulam: Bedan. Todos estos fueron los
hijos de Galaad, hijo de Maquir, hijo de Manasés. \bibleverse{18} Su
hermana Hamolequet fue la madre de Isod, Abiezer y Mahala.
\bibleverse{19} Los hijos de Semida fueron: Ahian, Siquem, Likhi y
Aniam.

\hypertarget{la-tribu-de-ephraim}{%
\subsection{La tribu de Ephraim}\label{la-tribu-de-ephraim}}

\bibleverse{20} Los descendientes de Efraín fueron: Sutela, su hijo
Bered, su hijo Tahat, su hijo Elead, su hijo Tahat, \bibleverse{21} su
hijo Zabad y su hijo Sutela. Ezer y Elead fueron asesinados por los
hombres que vivían en Gat cuando fueron allí a tratar de robar su
ganado. \bibleverse{22} Su padre Efraín los lloró durante mucho tiempo,
y sus parientes fueron a consolarlo. \bibleverse{23} Luego volvió a
acostarse con su mujer. Ella quedó embarazada y dio a luz un hijo, al
que llamó Bería por esta tragedia familiar. \bibleverse{24} Seera, su
hija, fundó la parte baja y alta de Bet Horon junto con Uzen-Seera.
\bibleverse{25} Sus desciendientes fueron: Refa su hioj, Resef su
hijo,\footnote{\textbf{7:25} ``Su hijo'': Lectura de la Septuaginta.
  Falta en el texto hebreo.} Telah su hijo, Tahan su hijo,
\bibleverse{26} Ladan su hijo, Amiud su hijo, Elisama su hijo,
\footnote{\textbf{7:26} Núm 1,10} \bibleverse{27} Nun su hijo y Josué su
hijo. \footnote{\textbf{7:27} Núm 13,8}

\hypertarget{residencias-de-la-tribu}{%
\subsection{Residencias de la tribu}\label{residencias-de-la-tribu}}

\bibleverse{28} La tierra que poseían y los lugares donde vivían
incluían Betel y las ciudades cercanas, desde Naarán al este hasta Gezer
y sus ciudades al oeste, y Siquem y sus ciudades hasta Aya y sus
ciudades. \footnote{\textbf{7:28} Jos 16,1; Jos 16,10} \bibleverse{29}
En la frontera con Manasés estaban Bet-San, Taanac, Meguido y Dor, junto
con sus ciudades. Estas eran las ciudades donde vivían los descendientes
de José hijo de Israel. \footnote{\textbf{7:29} Jos 17,11}

\hypertarget{la-tribu-de-asser}{%
\subsection{La tribu de Asser}\label{la-tribu-de-asser}}

\bibleverse{30} Los hijos de Aser: Imna, Isúa, Isúi y Bería. Su hermana
era Sera. \footnote{\textbf{7:30} Gén 46,17}

\bibleverse{31} Los hijos de Bería: Heber y Malquiel, el padre de
Birzavit. \bibleverse{32} Heber fue el padre de Jaflet, Somer y Hotam, y
de su hermana Súa. \bibleverse{33} Los hijos de Jaflet: Pasac, Bimhal y
Asvat. Todos estos fueron Los hijos de Jaflet. \bibleverse{34} Los hijos
de Somer: Ahi,\footnote{\textbf{7:34} ``Los hijos de Somer: Ahí:'' o
  ``Los hijos de Somer, su hermano:''} Rohga, Jeúba y Aram.
\bibleverse{35} Los hijos de su hermano Helem: Zofa, Imna, Seles y Amal.
\bibleverse{36} Los hijos de Zofa: Súa, Harnefer, Súal, Beri, Imra,
\bibleverse{37} Beser, Hod, Sama, Silsa, Itrán y Beera. \bibleverse{38}
Los hijos de Jeter fueron Jefone, Pispa y Ara. \bibleverse{39} Los hijos
de Ula fueron Ara, Haniel y Rezia. \bibleverse{40} Todos ellos eran
descendientes de los jefes de familia de Aser, hombres selectos, fuertes
guerreros y grandes líderes. Según su genealogía, tenían 26. 000
guerreros listos para la batalla.

\hypertarget{hijos-y-descendientes-de-benjamuxedn-a-travuxe9s-de-bela}{%
\subsection{Hijos y descendientes de Benjamín a través de
Bela}\label{hijos-y-descendientes-de-benjamuxedn-a-travuxe9s-de-bela}}

\hypertarget{section-7}{%
\section{8}\label{section-7}}

\bibleverse{1} Benjamín fue el padre de Bela (su primogénito), Asbel (el
segundo), Ahara (el tercero), \bibleverse{2} Noé (el cuarto) y Rafa (el
quinto). \bibleverse{3} Los hijos de Bela fueron: Adar, Gera, Abiud,
\bibleverse{4} Abisúa, Naamán, Ahoa, \bibleverse{5} Gera, Sefufán y
Huram.

\hypertarget{los-hijos-de-ehud}{%
\subsection{Los hijos de Ehud}\label{los-hijos-de-ehud}}

\bibleverse{6} Estos fueron los hijos de Aod, jefes de familia que
vivían en Geba, y fueron desterrados a Manahat: \bibleverse{7} Naamán,
Ahías y Gera. Gera fue quien los exilió. Era el padre de Uzz y Ahiud.

\hypertarget{la-familia-de-saharaim}{%
\subsection{La familia de Saharaim}\label{la-familia-de-saharaim}}

\bibleverse{8} Saharaim tuvo hijos en Moab después de divorciarse de sus
esposas Husim y Baara. \bibleverse{9} Se casó con Hodes y tuvo a Jobab,
Zibia, Mesa, Malcam, \bibleverse{10} Jeuz, Saquías y Mirma. Todos estos
fueron sus hijos, jefes de familia. \bibleverse{11} También tuvo hijos
con Husim: Abitob y Elpal. \bibleverse{12} Los hijos de Elpal: Éber,
Misham, Shemed (construyó Ono y Lod con sus ciudades cercanas),

\hypertarget{cinco-familias-benjaminitas-en-ajaluxf3n-y-jerusaluxe9n}{%
\subsection{Cinco familias benjaminitas en Ajalón y
Jerusalén}\label{cinco-familias-benjaminitas-en-ajaluxf3n-y-jerusaluxe9n}}

\bibleverse{13} y Bería y Sema, que eran jefes de familia que vivían en
Ajalón y que expulsaron a los que vivían en Gat. \bibleverse{14} Ahío,
Sasac, Jeremot, \bibleverse{15} Zebadías, Arad, Eder, \bibleverse{16}
Micael, Ispa y Joha eran los hijos de Bería. \bibleverse{17} Zebadías,
Mesulám, Hizqui, Heber, \bibleverse{18} Ismerai, Jezlías y Jobab fueron
los hijos de Elpaal. \bibleverse{19} Jacim, Zicri, Zabdi,
\bibleverse{20} Elienai, Ziletai, Eliel, \bibleverse{21} Adaías,
Beraías, y Simrat fueron los hijos de Simei. \bibleverse{22} Ispán,
Heber, Eliel, \bibleverse{23} Abdón, Zicri, Hanán, \bibleverse{24}
Hananías, Elam, Antotías, \bibleverse{25} Ifdaías y Peniel fueron los
hijos de Sasac. \bibleverse{26} Samsherai, Seharías, Atalías,
\bibleverse{27} Jaresías, Elías, y Zicri fueron los hijos de Jeroham.
\bibleverse{28} Todos ellos eran jefes de familia, según su genealogía.
Y vivían en Jerusalén.

\hypertarget{la-familia-del-rey-sauxfal}{%
\subsection{La familia del rey Saúl}\label{la-familia-del-rey-sauxfal}}

\bibleverse{29} Jeiel\footnote{\textbf{8:29} Según algunos manuscritos
  de la Septuaginta y también 1 Crónicas 9:35. Su nombre falta en el
  texto hebreo.} fundó Gabaón y vivió allí. Su mujer se llamaba Maaca.
\footnote{\textbf{8:29} 1Cró 9,35-44} \bibleverse{30} Su hijo
primogénito fue Abdón, luego Zur, Cis, Baal, Ner,\footnote{\textbf{8:30}
  Según algunos manuscritos de la Septuaginta y también de 1 Crónicas
  9:36. Su nombre falta en el texto hebreo.} Nadab, \bibleverse{31}
Gedor, Ahío, Zequer, \bibleverse{32} y Miclot. Miclot fue el padre de
Simea. También vivían cerca de sus parientes en Jerusalén.
\bibleverse{33} Ner fue el padre de Cis, Cis fue el padre de Saúl, y
Saúl fue el padre de Jonatán, Malquisúa, Abinadab y Es-Baal.\footnote{\textbf{8:33}
  En otros lugares se le conoce como ``Is-boset'', para evitar incluir
  en su nombre al dios pagano Baal. ``Boset'' significa ``vergüenza''.}
\bibleverse{34} El hijo de Jonatán fue Merib-Baal,\footnote{\textbf{8:34}
  Del mismo modo, también se le conoce como Mefi-boset.} que fue el
padre de Miqueas. \bibleverse{35} Los hijos de Miqueas fueron Pitón,
Melec, Tarea y Acaz. \bibleverse{36} Acaz fue el padre de Jada; Jada fue
el padre de Alemet, Azmavet y Zimri; y Zimri fue el padre de Mosa.
\bibleverse{37} Mosa fue el padre de Bina. Su hijo fue Rafa, padre de
Elasa, padre de Azel. \bibleverse{38} Azel tuvo seis hijos. Estos fueron
sus nombres: Azricam, Bocru, Ismael, Seraías, Obadías y Hanán. Todos
ellos eran los hijos de Azel. \bibleverse{39} Los hijos de su hermano
Esec fueron Ulam (primogénito), Jeús (el segundo) y Elifelet (el
tercero). \bibleverse{40} Los hijos de Ulam eran fuertes guerreros y
hábiles arqueros. Tuvieron muchos hijos y nietos, un total de 150. Todos
ellos fueron los hijos de Benjamín.\footnote{\textbf{8:40} 1Cró 12,2}

\hypertarget{directorio-de-residentes-destacados-de-jerusaluxe9n-en-el-peruxedodo-posterior-al-cautiverio}{%
\subsection{Directorio de residentes destacados de Jerusalén (en el
período posterior al
cautiverio)}\label{directorio-de-residentes-destacados-de-jerusaluxe9n-en-el-peruxedodo-posterior-al-cautiverio}}

\hypertarget{section-8}{%
\section{9}\label{section-8}}

\bibleverse{1} Todo el pueblo de Israel quedó registrado en los
registros de las genealogías del libro de los reyes de Israel. Pero el
pueblo de Judá fue llevado al cautiverio en Babilonia porque había sido
infiel.\footnote{\textbf{9:1} Claramente el autor de Crónicas está
  escribiendo después del cautiverio, y atribuye este evento al fracaso
  de la nación en seguir al verdadero Dios.} \footnote{\textbf{9:1} 2Re
  24,15-16} \bibleverse{2} Los primeros en regresar y reclamar sus
propiedades y vivir en sus ciudades fueron algunos israelitas,
sacerdotes, levitas y servidores del Templo. \footnote{\textbf{9:2} Jos
  9,23; Esd 8,20}

\hypertarget{el-pueblo-de-jerusaluxe9n}{%
\subsection{El pueblo de Jerusalén}\label{el-pueblo-de-jerusaluxe9n}}

\bibleverse{3} Algunos de los miembros de las tribus de Judá, Benjamín,
Efraín y Manasés volvieron a vivir en Jerusalén. Entre ellos estaban:
\footnote{\textbf{9:3} Neh 11,3-19}

\bibleverse{4} Utaí hijo de Ammihud, hijo de Omrí, hijo de Imrí, hijo de
Baní, descendiente de Fares, hijo de Judá. \bibleverse{5} De los
silonitas: Asaías (el primogénito) y sus hijos. \bibleverse{6} De los
zeraítas: Jeuel y sus parientes, con un total de 690.

\bibleverse{7} De los benjamitas: Salú, hijo de Mesulam, hijo de
Hodavías, hijo de Asenúa; \bibleverse{8} Ibneías, hijo de Jeroham; Ela,
hijo de Uzi, hijo de Micrí; y Mesulam, hijo de Sefatías, hijo de Reuel,
hijo de Ibniás. \bibleverse{9} Todos ellos eran jefes de familia, según
consta en sus genealogías, y en total sumaban 956.

\bibleverse{10} De los sacerdotes: Jedaías, Joiarib, Yacín,
\bibleverse{11} Azarías hijo de Hilcías, hijo de Mesulam, hijo de Sadoc,
hijo de Meraiot, hijo de Ajitub. (Azarías era el funcionario principal a
cargo de la casa de Dios). \footnote{\textbf{9:11} 1Cró 5,39}
\bibleverse{12} También Adaías hijo de Jeroham, hijo de Pasur, hijo de
Malquías, y Masai hijo de Adiel, hijo de Jazera, hijo de Mesulam, hijo
de Mesilemit, hijo de Imer. \bibleverse{13} Los sacerdotes, que eran
jefes de familia, sumaban 1. 760. Eran hombres fuertes y capaces que
servían en la casa de Dios.

\bibleverse{14} De los levitas: Semaías, hijo de Jasub, hijo de Azricam,
hijo de Hasabías, descendiente de Merari; \bibleverse{15} Bacbacar,
Heres, Galal y Matanías, hijo de Mica, hijo de Zicri, hijo de Asaf;
\bibleverse{16} Abdías, hijo de Semaías, hijo de Galal, hijo de Jedutún;
y Berequías, hijo de Asa, hijo de Elcana, que vivían en las aldeas de
los netofatitas.

\hypertarget{los-porteros-y-sus-servicios}{%
\subsection{Los porteros y sus
servicios}\label{los-porteros-y-sus-servicios}}

\bibleverse{17} Los guardas de la puerta del Templo:\footnote{\textbf{9:17}
  ``Del Templo'': implícito.} Salum, Acub, Talmón, Ahimán y sus
parientes. Salum era el jefe de los guardianes de la puerta del Templo.
\bibleverse{18} Ellos tenían la responsabilidad hasta ahora de cuidar la
Puerta del Rey en el lado Este. Eran los guardianes de los campamentos
de los levitas. \bibleverse{19} Salum era hijo de Coré, hijo de Ebiasaf,
hijo de Coré. Él y sus parientes, los coreítas, eran responsables de
vigilar las entradas del santuario\footnote{\textbf{9:19} ``Santuario'':
  Literalmente, ``tienda'', aunque ahora se refiera al edificio del
  Templo.} de la misma manera que sus padres se habían encargado de
vigilar la entrada de la casa de campaña\footnote{\textbf{9:19} ``La
  casa de campaña'', o ``Tabernáculo''.} del Señor. \footnote{\textbf{9:19}
  Núm 4,18-20} \bibleverse{20} Anteriormente, Finees hijo de Eleazar,
había sido el líder de los porteros. Y el Señor estaba con él.
\footnote{\textbf{9:20} Núm 25,7-13} \bibleverse{21} Posteriormente,
Zacarías hijo de Meselemías fue el portero de la entrada de la tienda de
reunión. \bibleverse{22} En total hubo 212 elegidos para ser porteros en
las entradas. Registraron sus genealogías en sus ciudades de origen.
David y el profeta Samuel habían seleccionado a sus antepasados por su
fidelidad. \footnote{\textbf{9:22} 1Sam 9,9; 1Sam 9,11} \bibleverse{23}
Ellos y sus descendientes eran responsables de vigilar la entrada de la
casa del Señor, también cuando era una tienda. \bibleverse{24} Los
porteros estaban ubicados en cuatro lados: al este, al oeste, al norte y
al sur. \bibleverse{25} Sus parientes en sus pueblos venían cada siete
días a ciertas horas para ayudarlos.

\hypertarget{informaciuxf3n-sobre-los-deberes-oficiales-de-los-levitas}{%
\subsection{Información sobre los deberes oficiales de los
levitas}\label{informaciuxf3n-sobre-los-deberes-oficiales-de-los-levitas}}

\bibleverse{26} Los cuatro porteros principales, que eran levitas,
tenían la responsabilidad de cuidar las habitaciones y los tesoros de la
casa de Dios. \bibleverse{27} Pasaban la noche alrededor de la casa de
Dios porque tenían que vigilarla y tenían la llave para abrirla por la
mañana.

\bibleverse{28} Algunos de los porteros eran responsables de los
artículos que se utilizaban en los servicios de culto. Llevaban un
inventario de lo que se traía y de lo que se sacaba. \bibleverse{29} A
otros se les asignaba la tarea de cuidar el mobiliario y el equipo
utilizado en el santuario, así como la harina especial, el vino, el
aceite de oliva, el incienso y las especias.

\bibleverse{30} (Sin embargo, algunos de los sacerdotes eran los
encargados de mezclar las especias). \bibleverse{31} A Matatías, un
levita, que era el hijo primogénito de Salum el coreíta, se le dio la
responsabilidad de hornear el pan usado en las ofrendas. \bibleverse{32}
Otros coatitas se encargaban de preparar el pan que se ponía en la mesa
cada sábado. \footnote{\textbf{9:32} Lev 24,5; Lev 24,8}

\hypertarget{informaciuxf3n-sobre-los-cantantes-del-templo-palabra-final}{%
\subsection{Información sobre los cantantes del templo; Palabra
final}\label{informaciuxf3n-sobre-los-cantantes-del-templo-palabra-final}}

\bibleverse{33} Los músicos, jefes de familias de levitas, vivían en las
habitaciones del Templo y no debían realizar otras tareas porque estaban
de servicio día y noche. \footnote{\textbf{9:33} 1Cró 9,14-16}
\bibleverse{34} Todos ellos eran jefes de familias de levitas, líderes
según sus genealogías, y vivían en Jerusalén.

\hypertarget{apuxe9ndice-los-habitantes-de-gabauxf3n-y-una-segunda-genealoguxeda-de-la-casa-de-sauxfal}{%
\subsection{Apéndice: Los habitantes de Gabaón y una segunda genealogía
de la casa de
Saúl}\label{apuxe9ndice-los-habitantes-de-gabauxf3n-y-una-segunda-genealoguxeda-de-la-casa-de-sauxfal}}

\bibleverse{35} Jeiel\footnote{\textbf{9:35} Véase 8:29.} era el padre
de Gabaón y vivía en Gabaón. Su mujer se llamaba Maaca. \footnote{\textbf{9:35}
  1Cró 8,29-38}

\bibleverse{36} Su hijo primogénito fue Abdón, luego Zur, Cis, Baal,
Ner, Nadab, \bibleverse{37} Gedor, Ahío, Zacarías y Miclot.
\bibleverse{38} Miclot fue el padre de Simeam. Ellos también vivían
cerca de sus parientes en Jerusalén. \bibleverse{39} Ner fue el padre de
Cis, Cis fue el padre de Saúl, y Saúl fue el padre de Jonatán,
Malquisúa, Abinadab y Es-Baal.\footnote{\textbf{9:39} Véase la nota a
  pie de página de 8:33.} \bibleverse{40} El hijo de Jonathan:
Merib-Baal,\footnote{\textbf{9:40} Véase la nota a pie de página de
  8:34.} que fue el padre de Miqueas. \bibleverse{41} Los hijos de
Miqueas: Pitón, Melec, Tarea y Acaz.\footnote{\textbf{9:41} ``Y Acaz'':
  Tomado de la Septuaginta y del verso 8:35. En el texto hebreo, el
  nombre no aparece aquí.} \bibleverse{42} Acaz fue el padre de
Jada,\footnote{\textbf{9:42} Según la Septuaginta y el verso 8:36. En
  hebreo se escribe ``Jara''.} Jada fue el padre de Alemet, Azmavet y
Zimri, y Zimri fue el padre de Mosa. \bibleverse{43} Mosa fue el padre
de Bina; Refaías fue su hijo, Elasa su hijo, y Azel su hijo.
\bibleverse{44} Azel tuvo seis hijos. Sus nombres eran: Azricam, Bocru,
Ismael, Searías, Obadías y Hanán. Estos fueron los hijos de Azel.

\hypertarget{israel-derrotado-por-los-filisteos-en-el-monte-gilboa-muerte-de-sauxfal-y-sus-tres-hijos}{%
\subsection{Israel derrotado por los filisteos en el monte Gilboa;
Muerte de Saúl y sus tres
hijos}\label{israel-derrotado-por-los-filisteos-en-el-monte-gilboa-muerte-de-sauxfal-y-sus-tres-hijos}}

\hypertarget{section-9}{%
\section{10}\label{section-9}}

\bibleverse{1} Los filisteos atacaron a Israel y todos los soldados
israelitas huyeron de ellos. Muchos israelitas fueron abatidos en el
monte Gilboa. \bibleverse{2} Los filisteos persiguieron a Saúl y a sus
hijos. Mataron a los hijos de Saúl: Jonatán, Abinadab y Malquisúa.
\bibleverse{3} La batalla se desató intensamente alrededor de Saúl.
Luego los arqueros enemigos vieron dónde estaba y lo hirieron.
\bibleverse{4} Entonces Saúl le dijo a su escudero: ``Saca tu espada y
mátame antes de que estos paganos vengan a torturarme''. Pero su
escudero se negó, pues tenía demasiado miedo de hacerlo. Así que Saúl
tomó su propia espada y se hizo caer sobre ella.

\bibleverse{5} Al ver que Saúl estaba muerto, su escudero también se
cayó sobre su espada y murió. \bibleverse{6} Así que Saúl y tres de sus
hijos murieron allí, junto con su línaje real.\footnote{\textbf{10:6}
  Literalmente, ``todos los de su casa murieron juntos'', sin embargo,
  esto debe tomarse en el contexto de que ninguno de sus hijos le
  sucedió, pues su hijo Is-boset sí sobrevivió.} \bibleverse{7} Cuando
todos los israelitas del valle vieron que su ejército había huido y que
Saúl y sus hijos habían muerto, abandonaron sus ciudades y también
huyeron. Entonces los filisteos llegaron y las ocuparon.

\hypertarget{el-destino-de-los-caduxe1veres-de-sauxfal-y-sus-hijos}{%
\subsection{El destino de los cadáveres de Saúl y sus
hijos}\label{el-destino-de-los-caduxe1veres-de-sauxfal-y-sus-hijos}}

\bibleverse{8} Al día siguiente, cuando los filisteos fueron a despojar
a los muertos, descubrieron los cuerpos de Saúl y de sus hijos en el
monte Gilboa. \bibleverse{9} Lo desnudaron, le cortaron la cabeza y se
llevaron su armadura. Luego enviaron la noticia a toda la tierra de
Filistea, a sus ídolos y a su pueblo. \bibleverse{10} Pusieron la
armadura de Saúl en el templo de sus ídolos y fijaron su cabeza en el
templo de Dagón. \bibleverse{11} Sin embargo, cuando todos en Jabes de
Galaad se enteraron de todo lo que los filisteos habían hecho con Saúl,
\bibleverse{12} todos sus combatientes fueron a recuperar los cuerpos de
Saúl y de sus hijos. Entonces los trajeron de vuelta y los enterraron
bajo el gran árbol de Jabes. Luego ayunaron durante siete días.

\hypertarget{revisiuxf3n-de-la-deuda-de-sauxfal-con-dios}{%
\subsection{Revisión de la deuda de Saúl con
Dios}\label{revisiuxf3n-de-la-deuda-de-sauxfal-con-dios}}

\bibleverse{13} Saúl murió porque le fue infiel al Señor. No cumplió los
mandatos del Señor, y además fue a consultar a una médium. \footnote{\textbf{10:13}
  1Sam 15,11; 1Sam 28,8}

\bibleverse{14} No consultó al Señor, así que el Señor le dio muerte y
le entregó el reinado a David, hijo de Isaí.

\hypertarget{la-unciuxf3n-de-david-en-hebruxf3n-y-la-conquista-de-jerusaluxe9n}{%
\subsection{La unción de David en Hebrón y la conquista de
Jerusalén}\label{la-unciuxf3n-de-david-en-hebruxf3n-y-la-conquista-de-jerusaluxe9n}}

\hypertarget{section-10}{%
\section{11}\label{section-10}}

\bibleverse{1} Entonces todos los israelitas se reunieron con David en
Hebrón. Y le dijeron: ``Somos tu carne y tu sangre.\footnote{\textbf{11:1}
  ``Tu carne y tu sangre'': Literalmente, ``huesos y carne''.}
\bibleverse{2} En los últimos tiempos, aunque Saúl era el rey, tú eras
el verdadero líder de Israel.\footnote{\textbf{11:2} ``Verdadero líder
  de Israel'': Literalmente, ``Tú guiaste y trajiste a Israel''.} El
Señor, tu Dios, te ha dicho: `Tú serás el pastor de mi pueblo, y tú
serás el jefe de mi pueblo Israel'\,''.

\bibleverse{3} Todos los ancianos de Israel acudieron ante el rey en
Hebrón, y David hizo un acuerdo solemne\footnote{\textbf{11:3} ``Acuerdo
  solemne'': o ``pacto''.} con ellos ante el Señor. Allí ungieron a
David como rey de Israel, tal como el Señor lo había prometido por medio
de Samuel. \footnote{\textbf{11:3} 1Sam 16,1; 1Sam 16,3; 1Sam 16,12}

\bibleverse{4} Entonces David y todos los israelitas fueron a Jerusalén
(antes conocida como Jebús), donde vivían los jebuseos. \bibleverse{5}
Pero los jebuseos le dijeron a David: ``No entrarás aquí''. Sin embargo,
David capturó la fortaleza de Sión, ahora conocida como la Ciudad de
David. \bibleverse{6} Y David había dicho: ``El primero que ataque a los
jebuseos será mi comandante en jefe''. Como Joab, hijo de Sarvia, fue el
primero, se convirtió en el comandante en jefe. \bibleverse{7} David
decidió habitar en la fortaleza, por eso la llamaron Ciudad de David.
\bibleverse{8} Entonces construyó la ciudad a su alrededor, desde el
Milo\footnote{\textbf{11:8} El significado de esta palabra es incierto.}
e hizo un circuito alrededor, mientras que Joab reparaba el resto de la
ciudad. \bibleverse{9} David se hizo cada vez más poderoso,\footnote{\textbf{11:9}
  ``Más y más poderoso'': literalmente, ``aumentaba y aumentaba''.}
porque el Señor Todopoderoso estaba con él.

\hypertarget{directorio-y-hazauxf1as-de-los-guerreros-de-david}{%
\subsection{Directorio y hazañas de los guerreros de
David}\label{directorio-y-hazauxf1as-de-los-guerreros-de-david}}

\bibleverse{10} Estos fueron los líderes de los poderosos guerreros de
David que, junto con todos los israelitas, le dieron un fuerte apoyo
para que se convirtiera en rey, tal como el Señor había prometido que
sucedería con Israel.

\bibleverse{11} Esta es la lista de los principales guerreros que
apoyaron a David: Jasobeam, hijo de Hacmoni, líder de los Tres. Con su
lanza, una vez mató a 300 hombres en una sola batalla. \footnote{\textbf{11:11}
  1Cró 27,2} \bibleverse{12} Después de él vino Eleazar, hijo de Dodo,
descendiente de Ahoha, uno de los Tres guerreros principales.
\footnote{\textbf{11:12} 1Cró 27,4} \bibleverse{13} Estaba con David en
Pasdamin cuando los filisteos se reunieron para la batalla que tuvo
lugar en un campo de cebada. El ejército israelita huyó cuando los
filisteos atacaron, \bibleverse{14} pero David y Eleazar se apostaron en
medio del campo, defendiendo su terreno y matando a los filisteos. El
Señor los salvó dándoles una gran victoria.

\hypertarget{wagnis-dreier-helden}{%
\subsection{Wagnis dreier Helden}\label{wagnis-dreier-helden}}

\bibleverse{15} En otra ocasión,\footnote{\textbf{11:15} Implícito.} los
Tres, que formaban parte de los Treinta guerreros principales, bajaron a
recibir a David cuando estaba en la cueva de Adulam. El ejército
filisteo estaba acampado en el valle de Refaim. \footnote{\textbf{11:15}
  1Sam 22,1} \bibleverse{16} En ese momento David estaba en la
fortaleza, y la guarnición filistea estaba en Belén. \bibleverse{17}
David tenía mucha sed y dijo: ``¡Ojalá alguien pudiera traerme un trago
de agua del pozo que está junto a la puerta de la entrada de Belén!''

\bibleverse{18} Los Tres atravesaron las defensas filisteas, tomaron un
poco de agua del pozo de la puerta de Belén y se la llevaron a David.
Pero David se negó a beberla y la vertió como ofrenda al Señor.
\bibleverse{19} ``¡Dios me libre de hacer esto!'', dijo. ``Sería como
beber la sangre de esos hombres que arriesgaron sus vidas. Ellos
arriesgaron sus vidas para traerme una bebida''. Así que no la bebió.
Estas son algunas de las cosas que hicieron los tres guerreros
principales.

\hypertarget{abisai-y-benauxedas}{%
\subsection{Abisai y Benaías}\label{abisai-y-benauxedas}}

\bibleverse{20} Abisai, hermano de Joab, era el líder de los segundos
Tres.\footnote{\textbf{11:20} Sin embargo, ya se ha mencionado a
  Jasobeam como líder de los Tres (11:11), y también se ha mencionado la
  muerte de 300 con su lanza. Algunos sugieren una confusión de nombres
  o una ortografía alternativa, o que esto se refiere a otra persona en
  conjunto como líder no de los Tres sino de los Treinta, o que había
  otro ``Tres''.} Usando su lanza, una vez mató a 300 hombres, y se hizo
famoso entre los Tres. \bibleverse{21} Era el más apreciado de los Tres
y era su comandante, aunque no fue uno de los primeros Tres.\footnote{\textbf{11:21}
  Identificar un primer y un segundo Tres parece ser la solución más
  sencilla a lo que son versos confusos.}

\bibleverse{22} Benaía, hijo de Joiada, un fuerte guerrero de Cabseel,
hizo muchas cosas sorprendentes. Mató a dos hijos de Ariel de
Moab.\footnote{\textbf{11:22} Entendido en la Septuaginta; puede
  referirse a dos campeones combatientes de Moab.} También fue tras un
león a un pozo en la nieve y lo mató. \bibleverse{23} En otra ocasión
mató a un egipcio, un hombre enorme que medía dos metros y
medio.\footnote{\textbf{11:23} Literalmente ``cinco codos''.} El egipcio
tenía una lanza cuyo asta era tan gruesa como la vara de un tejedor.
Benaía lo atacó sólo con un garrote, pero pudo agarrar la lanza de la
mano del egipcio, y lo mató con su propia lanza. \bibleverse{24} Este
fue el tipo de cosas que hizo Benaía y que lo hicieron tan famoso como
los tres guerreros principales. \footnote{\textbf{11:24} 1Cró 27,5-6}
\bibleverse{25} Era el más apreciado de los Treinta, aunque no era uno
de los Tres. David lo puso a cargo de su guardia personal.

\hypertarget{una-lista-de-otros-huxe9roes-de-david}{%
\subsection{Una lista de otros héroes de
David}\label{una-lista-de-otros-huxe9roes-de-david}}

\bibleverse{26} Otros guerreros principales eran: Asael, hermano de
Joab; Elhanán, hijo de Dodo, de Belén; \bibleverse{27} Sama el harodita;
Heles el pelonita; \bibleverse{28} Ira, hijo de Iques, de Tecoa;
Abiezer, de Anatot; \bibleverse{29} Sibecai el husatita; Ilai el
ahohita; \bibleverse{30} Maharai, de Netofa; Heled, hijo de Baana, de
Netofa; \bibleverse{31} Itai, hijo de Ribai, de Guibeá, de los
benjamitas; Benaía el Piratonita; \bibleverse{32} Hurai de los valles de
Gaas; Abiel de Arabá; \bibleverse{33} Azmavet de Bahurim; Eliaba el
Saalbonita; \bibleverse{34} Los hijos de Jasén el Gizonita; Jonatán,
hijo de Sage el Ararita; \bibleverse{35} Ahíam, hijo de Sacar el
Ararita; Elifal, hijo de Ur; \bibleverse{36} Hefer de Mequer; Ahías el
pelonita; \bibleverse{37} Hezro el Carmelita; Naarai, hijo de Ezbai;
\bibleverse{38} Joel, hermano de Natán; Mibhar, hijo de Hagri;
\bibleverse{39} Zelek, el amonita; Naharai, de Beerot; el escudero de
Joab, hijo de Sarvia; \bibleverse{40} Ira, de Jatir; Gareb, de Jatir;
\bibleverse{41} Urías, el hitita; Zabad, hijo de Ahlai; \footnote{\textbf{11:41}
  2Sam 11,3}

\bibleverse{42} Adina, hijo de Siza, rubenita, jefe de los rubenitas, y
los treinta que estaban con él; \bibleverse{43} Hanán, hijo de Maaca;
Josafat mitnita; \bibleverse{44} Uzías de Astarot; Sama y Jeiel, los
hijos de Hotam de Aroer; \bibleverse{45} Jediael, hijo de Simri, y su
hermano, Joha el tizita; \bibleverse{46} Eliel de Mahava; Jerebai y
Josavía, los hijos de Elnaam; Itma el moabita; \bibleverse{47} Eliel;
Obed y Jaasiel, todos ellos de Soba.

\hypertarget{los-seguidores-de-david-en-siclag-y-adullam-mientras-sauxfal-todavuxeda-estaba-vivo}{%
\subsection{Los seguidores de David en Siclag y Adullam mientras Saúl
todavía estaba
vivo}\label{los-seguidores-de-david-en-siclag-y-adullam-mientras-sauxfal-todavuxeda-estaba-vivo}}

\hypertarget{section-11}{%
\section{12}\label{section-11}}

\bibleverse{1} La siguiente es una lista de los hombres que se unieron a
David cuando estaba en Siclag, todavía escondiéndose de Saúl, hijo de
Cis. Eran algunos de los principales guerreros que lucharon del lado de
David. \bibleverse{2} Todos ellos eran hábiles arqueros, y podían
disparar flechas o hondas con la mano derecha o con la izquierda. Eran
parientes de Saúl de la tribu de Benjamín. \footnote{\textbf{12:2} 1Cró
  8,40} \bibleverse{3} Ahiezer era su líder, luego lo fue Joás; los
hijos de Semaa de Gugibeá; Jeziel y Pelet los hijos de Azmavet; Beraca;
Jehú de Anatot; \bibleverse{4} Ismaías de Gabaón, (un fuerte guerrero
entre los Treinta, y líder sobre los Treinta); Jeremías; Jahaziel;
Johanán; Jozabad de Gedera; \bibleverse{5} Eluzai; Jerimot; Bealías;
Semarías; Sefatías de Haruf. \bibleverse{6} Elcana, Isías, Azareel,
Joezer y Jasobeam (quienes eran Coreítas); \bibleverse{7} Joela y
Zebadías, los hijos de Jeroham de Gedor.

\bibleverse{8} Algunos guerreros de la tribu de Gad se pasaron al lado
de David cuando éste estaba en la fortaleza del desierto. Eran guerreros
fuertes y experimentados, curtidos en la batalla, expertos en el uso de
escudos y lanzas. Sus rostros parecían tan fieros como los de los
leones, y corrían tan rápido como las gacelas en las montañas.
\footnote{\textbf{12:8} 2Sam 2,18} \bibleverse{9} Ezer el era el líder,
Obadías (el segundo), Eliab (el tercero), \bibleverse{10} Mismaná (el
cuarto), Jeremías (el quinto), \bibleverse{11} Atai (el sexto), Eliel
(el séptimo), \bibleverse{12} Johanán (el octavo), Elzabad (el noveno),
\bibleverse{13} Jeremías (el décimo), Macbanai (el undécimo).
\bibleverse{14} Estos guerreros de Gad eran oficiales del ejército. El
menos hábil de ellos estaba a cargo de 100 hombres; y el mejor estaba a
cargo de 1. 000. \bibleverse{15} Estos eran los que cruzaban el río
Jordán en el primer mes del año, cuando se desborda. Expulsaron a todos
los pueblos que vivían en el valle, tanto al este como al oeste.

\bibleverse{16} Otros de las tribus de Benjamín y Judá también vinieron
a unirse a David en la fortaleza. \bibleverse{17} David salió a
recibirlos y les dijo: ``Si han venido en son de paz para ayudarme,
podemos ser amigos.\footnote{\textbf{12:17} ``Podemos ser amigos'':
  Literalmente, ``mi corazón estará junto a ustedes''.} Pero si han
venido a entregarme a mis enemigos, aunque no he hecho nada malo, que el
Dios de nuestros padres vea lo que hacen y los condene''.
\bibleverse{18} Entonces el Espíritu vino sobre\footnote{\textbf{12:18}
  ``Vino sobre'': Literalmente, ``vistió''.} Amasai, el líder de los
Treinta. ``¡Somos tuyos, David, y estamos contigo, hijo de Isaí! Que la
paz, la prosperidad y el éxito\footnote{\textbf{12:18} ``Paz, la
  prosperidad y el éxito'': Literalmente, ``Shalom, shalom a ti, y
  shalom a quien te ayude''.} sean tuyos y de los que te ayuden, porque
Dios es el que te ayuda''. Así que David les permitió unirse a él, y los
puso al frente de su ejército.

\bibleverse{19} Otros se pasaron al lado de David desde la tribu de
Manasés y se unieron a él cuando fue con los filisteos a atacar a Saúl.
Sin embargo, los jefes filisteos decidieron finalmente despedirlos,
diciéndose: ``Nos costará la cabeza si nos abandona y se une a su amo
Saúl''.

\bibleverse{20} La siguiente es una lista de los hombres de Manasés que
se pasaron al lado de David cuando éste regresó a Siclag: Adnas,
Jozabad, Jediael, Micael, Jozabad, Eliú y Ziletai, jefes de millares en
la tribu de Manasés. \bibleverse{21} Ellos ayudaron a David contra los
asaltantes, pues todos eran guerreros fuertes y experimentados y
comandantes del ejército. \bibleverse{22} Cada día llegaban hombres para
ayudar a David, hasta que tuvo un gran ejército, como el ejército de
Dios.

\bibleverse{23} Esta es la lista del número de guerreros armados que
vinieron y se unieron a David en Hebrón para entregarle el reino de
Saúl, como había dicho el Señor.

\hypertarget{nuxfamero-de-guerreros-en-la-elecciuxf3n-de-david-como-rey-en-hebruxf3n}{%
\subsection{Número de guerreros en la elección de David como rey en
Hebrón}\label{nuxfamero-de-guerreros-en-la-elecciuxf3n-de-david-como-rey-en-hebruxf3n}}

\bibleverse{24} De la tribu de Judá, 6. 800 guerreros con escudos y
lanzas. \bibleverse{25} De la tribu de Simeón, 7. 100 guerreros fuertes.
\bibleverse{26} De la tribu de Leví, 4. 600, \bibleverse{27} incluyendo
a Joiada, jefe de la familia de Aarón, y con él 3. 700, \bibleverse{28}
y Sadoc, un joven guerrero fuerte, con 22 miembros de su familia, todos
oficiales. \footnote{\textbf{12:28} 2Sam 15,24; 1Cró 5,34}
\bibleverse{29} De la tribu de Benjamín, de entre los parientes de Saúl,
3. 000, la mayoría de los cuales habían permanecido leales a Saúl hasta
ese momento. \bibleverse{30} De la tribu de Efraín, 20. 800 guerreros
fuertes, cada uno de ellos muy apreciado en su propio clan.
\bibleverse{31} De la media tribu de Manasés, 18. 000 hombres designados
por su nombre para venir a hacer rey a David. \bibleverse{32} De la
tribu de Isacar vinieron líderes que conocían y podían entender los
signos de los tiempos y lo que Israel debía hacer: un total de 200
líderes de la tribu junto con sus parientes. \bibleverse{33} De la tribu
de Zabulón, 50. 000 guerreros. Estaban completamente armados y listos
para la batalla, y totalmente dedicados. \bibleverse{34} De la tribu de
Neftalí, 1. 000 oficiales y 37. 000 guerreros con escudos y lanzas.
\bibleverse{35} De la tribu de Dan, 28. 600 guerreros, todos preparados
para la batalla. \bibleverse{36} De la tribu de Aser, 40. 000 guerreros
experimentados, todos listos para la batalla. \bibleverse{37} De la
parte oriental del río Jordán, de las tribus de Rubén, Gad y la media
tribu de Manasés, 120. 000 guerreros con todo tipo de armas.

\bibleverse{38} Todos estos hombres llegaron a Hebrón vestidos para la
batalla, completamente decididos a hacer rey a David. Todo Israel estaba
de acuerdo en que David debía ser rey. \bibleverse{39} Se quedaron allí
tres días, comiendo y bebiendo juntos, pues sus parientes les habían
proporcionado provisiones. \bibleverse{40} Sus vecinos, incluso de
lugares tan lejanos como Isacar, Zabulón y Neftalí, llegaron trayendo
comida en burros, camellos, mulas y bueyes. Tenían mucha harina, tortas
de higos, racimos de pasas, vino, aceite de oliva, ganado y ovejas,
porque Israel estaba muy contento.

\hypertarget{movilizaciuxf3n-de-todo-el-pueblo-con-fines-de-recuperaciuxf3n.}{%
\subsection{Movilización de todo el pueblo con fines de
recuperación.}\label{movilizaciuxf3n-de-todo-el-pueblo-con-fines-de-recuperaciuxf3n.}}

\hypertarget{section-12}{%
\section{13}\label{section-12}}

\bibleverse{1} David tuvo discusiones con todos sus líderes, incluyendo
los comandantes del ejército de miles y cientos.\footnote{\textbf{13:1}
  ``Miles y cientos'': en referencia a la estructura del ejército, en el
  que unos estaban a cargo de 1000 hombres y otros de 100.}
\bibleverse{2} Luego se dirigió a toda la asamblea de Israel, diciendo:
``Si están de acuerdo, y si Dios lo aprueba, enviemos una invitación a
todos los israelitas de la tierra, incluidos los sacerdotes y levitas en
sus ciudades y pastos, para que vengan a unirse a nosotros.
\bibleverse{3} Traigamos de vuelta el Arca de nuestro Dios\footnote{\textbf{13:3}
  ``Traigamos de vuelta'': Curiosamente la raíz del verbo tiene el
  significado básico de ``rodear''.} a nosotros, porque lo habíamos
olvidado en tiempos de Saúl''.

\bibleverse{4} Toda la asamblea se alegró de la propuesta, y estuvo de
acuerdo en que sería una buena cosa. \bibleverse{5} Así que David
convocó a todo Israel, desde el río Sihor de Egipto hasta Lebo-hamat,
para que ayudaran a traer el Arca desde Quiriat-jearim.

\hypertarget{fracaso-del-plan}{%
\subsection{Fracaso del plan}\label{fracaso-del-plan}}

\bibleverse{6} Así pues, David y todo Israel fueron a Baalá (llamada
también Quiriat-jearim), en Judá, para traer de vuelta el Arca de Dios
el Señor, cuyo trono está entre los querubines y que es llamado por el
Nombre. \bibleverse{7} Cargaron el Arca de Dios en una carreta nueva y
la trajeron desde la casa de Adinadab, con Uza y Ahio dirigiéndola.
\bibleverse{8} David y todo Israel estaban celebrando ante el Señor lo
más alto posible, cantando canciones y tocando música con liras, arpas,
panderetas, címbalos y trompetas.

\bibleverse{9} Pero cuando llegaron a la era de Quidón, los bueyes
tropezaron y Uzza extendió la mano para evitar que el Arca se cayera.
\bibleverse{10} El Señor se enfadó con Uza por atreverse a tocar el Arca
de esa manera, así que lo abatió, y Uza murió allí ante el Señor.
\bibleverse{11} David se enfadó con el Señor por su violento arrebato
contra Uza. Llamó al lugar Fares-uza,\footnote{\textbf{13:11} Fares-uza
  significa ``arrebato contra Uzza''.} y aún hoy se le llama así.

\hypertarget{el-cajuxf3n-se-encuentra-en-la-casa-de-obed-edom}{%
\subsection{El cajón se encuentra en la casa de
Obed-Edom}\label{el-cajuxf3n-se-encuentra-en-la-casa-de-obed-edom}}

\bibleverse{12} Ese día, David tuvo miedo de Dios. ``¿Cómo podré
devolver el Arca de Dios a mi casa?'' , se preguntó. \bibleverse{13} Así
que David no trasladó el Arca de Dios para que estuviera con él en la
Ciudad de David. En lugar de eso, hizo que la llevaran a la casa de
Obed-edom de Gat. \bibleverse{14} El Arca de Dios permaneció en la casa
de Obed-edom durante tres meses, y el Señor bendijo la casa de Obed-edom
y todo lo que tenía.

\hypertarget{el-edificio-del-palacio-de-david-y-los-nuevos-matrimonios-sus-guerras-victoriosas-con-los-filisteos}{%
\subsection{El edificio del palacio de David y los nuevos matrimonios;
sus guerras victoriosas con los
filisteos}\label{el-edificio-del-palacio-de-david-y-los-nuevos-matrimonios-sus-guerras-victoriosas-con-los-filisteos}}

\hypertarget{section-13}{%
\section{14}\label{section-13}}

\bibleverse{1} Entonces Hiram, rey de Tiro, envió mensajeros a David
junto con madera de cedro, canteros y carpinteros para que le
construyeran un palacio. \bibleverse{2} De esta manera David se dio
cuenta de que el Señor lo había colocado en el trono como rey de Israel
y había bendecido apoyando su reino por el bien del pueblo del Señor,
Israel.

\hypertarget{los-hijos-de-david-nacidos-en-jerusaluxe9n}{%
\subsection{Los hijos de David nacidos en
Jerusalén}\label{los-hijos-de-david-nacidos-en-jerusaluxe9n}}

\bibleverse{3} David se casó con más esposas en Jerusalén y tuvo más
hijos e hijas. \bibleverse{4} Esta es una lista de los nombres de los
hijos que tuvo en Jerusalén: Samúa, Sobab, Natán, Salomón,
\bibleverse{5} Ibhar, Elisúa, Elpelet, \footnote{\textbf{14:5} 2Sam
  5,17-25} \bibleverse{6} Noga, Nefeg, Jafía, \bibleverse{7} Elisama,
Beeliada y Elifelet.

\hypertarget{dos-batallas-victoriosas-entre-david-y-los-filisteos}{%
\subsection{Dos batallas victoriosas entre David y los
filisteos}\label{dos-batallas-victoriosas-entre-david-y-los-filisteos}}

\bibleverse{8} Cuando los filisteos se enteraron de que David había sido
ungido rey de todo Israel, reunieron todo su ejército para ir tras él.
Pero David oyó que venían y salió a enfrentarlos. \bibleverse{9} Los
filisteos llegaron y asaltaron el valle de Refaim. \bibleverse{10} David
consultó a Dios y le preguntó: ``¿Debo ir a atacar a los filisteos? ¿Me
harás victorioso sobre ellos?'' . ``Adelante'', le dijo el Señor, ``yo
te haré victorioso sobre ellos''.

\bibleverse{11} Así que David atacó y los derrotó allí en Baal-perazim.
``Dios me utilizó para derrotar a mis enemigos como un torrente de agua
que brota'', declaró. Por eso el lugar se llamó Baal-perazim.\footnote{\textbf{14:11}
  Baal-perazim significa ``el Señor irrumpe''.} \bibleverse{12} Los
filisteos habían dejado sus dioses, así que David dio órdenes de que los
quemaran.

\bibleverse{13} Sin embargo, los filisteos regresaron y realizaron otra
incursión en el valle. \bibleverse{14} David volvió a consultar a Dios.
``No hagas un ataque frontal'', le dijo Dios. ``En lugar de eso, ve por
detrás de ellos y atácalos frente a los árboles de bálsamo.
\bibleverse{15} En cuanto oigas el ruido de la marcha en las copas de
los bálsamos, ve y ataca, porque el Señor ha ido delante de ti para
derribar al ejército filisteo''.

\bibleverse{16} Así que David hizo lo que Dios le dijo, derribando al
ejército filisteo desde Gabaón hasta Gezer. \bibleverse{17} Como
resultado, la reputación de David se extendió por todas partes, y el
Señor hizo que todas las naciones tuvieran miedo de David.

\hypertarget{preparativos-para-el-traslado-del-arca-sagrada-designaciuxf3n-e-instrucciuxf3n-de-los-levitas-a-cargo}{%
\subsection{Preparativos para el traslado del arca sagrada; Designación
e instrucción de los levitas a
cargo}\label{preparativos-para-el-traslado-del-arca-sagrada-designaciuxf3n-e-instrucciuxf3n-de-los-levitas-a-cargo}}

\hypertarget{section-14}{%
\section{15}\label{section-14}}

\bibleverse{1} Una vez que David terminó de construirse casas en la
Ciudad de David, hizo un lugar para el Arca de Dios y levantó allí una
tienda. \bibleverse{2} Luego dio órdenes: ``Nadie debe llevar el Arca de
Dios, excepto los levitas, porque el Señor mismo los eligió para llevar
el Arca del Señor y servirle para siempre''.

\bibleverse{3} Entonces David convocó a todo Israel a Jerusalén para
llevar el Arca del Señor al lugar que había preparado para ella.
\bibleverse{4} Esta es la lista de los levitas, Los hijos de Aarón, que
David convocó para asistir: \bibleverse{5} De los hijos de Coat, Uriel
(el jefe de familia), y 120 de sus parientes; \bibleverse{6} de los
hijos de Merari, Asaías (el jefe de familia), con 220 de sus parientes;
\bibleverse{7} de los hijos de Gersón, Joel (el jefe de familia), con
130 de sus parientes; \bibleverse{8} de los hijos de Elizafán, Semaías
(el jefe de familia), con 200 de sus parientes; \bibleverse{9} de los
hijos de Hebrón, Eliel (el jefe de familia), con 80 de sus parientes;
\bibleverse{10} de los hijos de Uziel, Aminadab (el jefe de familia),
con 112 de sus parientes.

\bibleverse{11} Entonces David convocó a los sacerdotes Sadoc y Abiatar,
y a los levitas Uriel, Asaías, Joel, Semaías, Eliel y Aminadab.
\footnote{\textbf{15:11} 2Sam 15,29} \bibleverse{12} Les dijo: ``Ustedes
son los jefes de las familias de los levitas. Ustedes mismos y sus
parientes deben estar ceremonialmente limpios y puros\footnote{\textbf{15:12}
  ``Ceremonialmente limpio y puro'': seguir las normas y requisitos
  religiosos.} antes de que traigas de vuelta el Arca de Dios, el Señor
de Israel al lugar que he hecho para ella. \bibleverse{13} Por no haber
estado allí la primera vez para llevar el Arca, el Señor, nuestro Dios,
estalló en violencia contra nosotros. No la tratamos de acuerdo con sus
instrucciones''.

\bibleverse{14} Así que los sacerdotes y los levitas se purificaron para
poder traer de vuelta el Arca del Señor, el Dios de Israel.
\bibleverse{15} Entonces los levitas llevaron el Arca de Dios de la
manera que Moisés había ordenado, según lo que Dios había dicho: sobre
sus hombros, usando las varas especiales para transportarla. \footnote{\textbf{15:15}
  Éxod 25,14; Núm 4,15}

\hypertarget{orden-de-los-cantantes-muxfasicos-y-porteadores-levuxedticos}{%
\subsection{Orden de los cantantes, músicos y porteadores
levíticos}\label{orden-de-los-cantantes-muxfasicos-y-porteadores-levuxedticos}}

\bibleverse{16} David también dio instrucciones a los jefes de los
levitas para que asignaran de entre sus parientes a cantores que
cantaran con alegría, acompañados por músicos que tocaran liras, arpas y
címbalos. \bibleverse{17} Así que los levitas asignaron a Hemán, hijo de
Joel, y de sus parientes a Asaf, hijo de Berequías, y de los hijos de
Merari, sus parientes, a Etán, hijo de Cusaías. \bibleverse{18} El
segundo grupo de levitas estaba formado por Zacarías, Jaaziel,
Semiramot, Jehiel, Uni, Eliab, Benaía, Maaseías, Matatías, Elifelehu y
Micnías; y los porteros Obed-edom y Jeiel. \bibleverse{19} Los músicos
Hemán, Asaf y Etán debían tocar los címbalos de bronce; \bibleverse{20}
Zacarías, Aziel, Semiramot, Jehiel, Uni, Eliab, Maasé y Benaía debían
tocar las arpas ``según alamot'', \bibleverse{21} mientras que Matatías,
Elifelehu, Micnías, Obed-edom, Jeiel y Azazías debían dirigir la música
con liras ``según seminit''. \bibleverse{22} Quenanías, el líder de los
levitas en el canto, fue elegido para dirigir la música debido a su
habilidad. \bibleverse{23} Berequías y Elcana fueron designados para
custodiar el Arca. \bibleverse{24} Los sacerdotes Sebanías, Josafat,
Natanel, Amasai, Zacarías, Benaía y Eliezer debían tocar las trompetas
delante del Arca de Dios. Obed-edom y Jeías también fueron designados
para custodiar el Arca.

\hypertarget{la-participaciuxf3n-personal-de-david-en-la-transferencia-la-fiesta-del-sacrificio-y-la-acciuxf3n-de-gracias}{%
\subsection{La participación personal de David en la transferencia; la
fiesta del sacrificio y la acción de
gracias}\label{la-participaciuxf3n-personal-de-david-en-la-transferencia-la-fiesta-del-sacrificio-y-la-acciuxf3n-de-gracias}}

\bibleverse{25} Luego David, los ancianos de Israel y los comandantes
del ejército de mayor rango,\footnote{\textbf{15:25} ``De mayor rango'':
  Literalmente, ``comandantes de miles''.} fueron con gran celebración a
traer el Arca del Pacto del Señor desde la casa de Obed-Edom.
\footnote{\textbf{15:25} 2Sam 6,12-16}

\bibleverse{26} Como Dios ayudó a los levitas que llevaban el Arca del
Pacto del Señor, sacrificaron siete toros y siete carneros.
\bibleverse{27} David se vistió con una túnica de lino fino, al igual
que todos los levitas que llevaban el Arca, y los cantores y Quenanías,
el líder de la música y los cantores. David también se puso un efod de
lino.\footnote{\textbf{15:27} ``Efod de lino'': ropa especial que llevan
  los sacerdotes.} \bibleverse{28} Así que todo Israel trajo de vuelta
el Arca del Pacto del Señor con mucha gritería, acompañada de cuernos,
trompetas y címbalos, y música tocada con arpas y liras. \bibleverse{29}
Pero cuando el Arca del Pacto del Señor entró en la Ciudad de David, la
hija de Saúl, Mical, miró desde una ventana. Al ver al rey David
saltando y bailando de alegría, se llenó de desprecio por él.

\hypertarget{section-15}{%
\section{16}\label{section-15}}

\bibleverse{1} Trajeron el Arca de Dios y la colocaron en la tienda que
David había preparado para ella. Presentaron holocaustos y ofrendas de
amistad a Dios. \bibleverse{2} Cuando David terminó de presentar los
holocaustos y las ofrendas de amistad, bendijo al pueblo en nombre del
Señor. \bibleverse{3} Luego repartió a cada israelita, a cada hombre y a
cada mujer, una hogaza de pan, una torta de dátiles y una torta de
pasas.

\hypertarget{orden-del-servicio-de-canto-y-muxfasica-en-el-arca}{%
\subsection{Orden del servicio de canto y música en el
Arca}\label{orden-del-servicio-de-canto-y-muxfasica-en-el-arca}}

\bibleverse{4} David asignó a algunos de los levitas para que sirvieran
de ministros ante el Arca del Señor, para recordar, agradecer y alabar
al Señor, el Dios de Israel. \bibleverse{5} Asaf era el encargado,
Zacarías era el segundo, luego Jeiel, Semiramot, Jehiel, Matatías,
Eliab, Benaía, Obed-edom y Jeiel. Tocaban arpas y liras, y Asaf tocaba
los címbalos, \bibleverse{6} y los sacerdotes Benaía y Jahaziel tocaban
continuamente las trompetas delante del Arca del Pacto de Dios.

\hypertarget{canto-de-agradecimiento-y-alabanza-de-david}{%
\subsection{Canto de agradecimiento y alabanza de
David}\label{canto-de-agradecimiento-y-alabanza-de-david}}

\bibleverse{7} Este fue el día en que David instruyó por primera vez a
Asaf y a sus parientes para que dieran gracias al Señor de esta
manera:\footnote{\textbf{16:7} ``De esta manera'': implícito. Lo que
  sigue es una selección de Salmos 105, Salmos 96, Salmos 107, and
  Salmos 106} \bibleverse{8} Denle gracias al Señor, adoren su
naturaleza maravillosa, ¡hagan saber lo que ha hecho! \footnote{\textbf{16:8}
  Sal 105,1-15} \bibleverse{9} Cántenle, canten sus alabanzas; cuéntenle
a todos las grandes cosas que ha hecho. \bibleverse{10} Enorgullézcanse
de su carácter santo; ¡alégrense todos los que se acercan al Señor!
\bibleverse{11} Busquen al Señor y su fuerza; busquen siempre estar con
él. \bibleverse{12} Recuerden todas las maravillas que ha hecho, los
milagros que ha realizado y los juicios que ha pronunciado,
\bibleverse{13} descendientes de Israel, hijos de Jacob, su pueblo
elegido. \bibleverse{14} Él es el Señor, nuestro Dios, sus juicios
cubren toda la tierra. \bibleverse{15} Él se acuerda de su acuerdo para
siempre, la promesa que hizo dura mil generaciones \bibleverse{16} el
acuerdo que hizo con Abrahán, el voto que hizo a Isaac. \bibleverse{17}
El Señor lo confirmó legalmente con Jacob, hizo este acuerdo eterno con
Israel \bibleverse{18} diciendo: ``Les daré la tierra de Canaán para que
la posean''. \bibleverse{19} Dijo esto cuando sólo eran unos pocos, un
pequeño grupo de extranjeros en la tierra. \bibleverse{20} Iban de un
país a otro, de un reino a otro. \bibleverse{21} No permitió que nadie
los tratara mal; advirtió a los reyes que los dejaran en paz:
\bibleverse{22} ``¡No toquen a mi pueblo elegido, no hagan daño a mis
profetas!'' \bibleverse{23} ¡Cántenle al Señor! ¡Que toda la tierra le
cante al Señor! ¡Que cada día todos oigan de su salvación! \footnote{\textbf{16:23}
  Sal 96,-1} \bibleverse{24} Anuncien sus actos gloriosos entre las
naciones, las maravillas que hace entre todos los pueblos.
\bibleverse{25} Porque el Señor es grande y merece la mejor alabanza. Él
debe ser respetado con temor por encima de todos los dioses.
\bibleverse{26} Porque todos los dioses de las demás naciones son
ídolos, pero el Señor hizo los cielos. \bibleverse{27} Suyos son el
esplendor y la majestuosidad; en su santuario hay poder y gloria.
\bibleverse{28} Reconozcan al Señor, naciones del mundo, dénle la gloria
y el poder. \bibleverse{29} Dénle al Señor la gloria que se merece;
traigan una ofrenda y preséntense ante él. Adoren al Señor en su
magnífica santidad. \bibleverse{30} Que todo el mundo en la tierra
tiemble ante su presencia. El mundo se mantiene unido con firmeza; no
puede romperse. \bibleverse{31} Que los cielos canten de alegría, que la
tierra se alegre. Digan a las naciones: ``¡El Señor está al mando!''
\bibleverse{32} ¡Que el mar y todo lo que hay en él griten de alabanza!
Que los campos y todo lo que hay en ellos celebren; \bibleverse{33} Que
todos los árboles del bosque canten de alegría, porque él viene a juzgar
la tierra. \bibleverse{34} Denle gracias al Señor, porque es bueno. Su
amor es eterno. \footnote{\textbf{16:34} Sal 106,47-48} \bibleverse{35}
Griten: ``¡Sálvanos, Señor, Dios nuestro! Reúnenos de entre las
naciones, rescátanos, para que podamos darte gracias y alabar lo
magnífico y santo que eres''. \bibleverse{36} ¡Qué maravilloso es el
Señor, el Dios de Israel, que vive por los siglos de los siglos!
Entonces todo el pueblo dijo: ``¡Amén!'' y ``¡Alaben al Señor!''.

\hypertarget{establecimiento-del-servicio-de-portero-sacerdote-y-cantante-en-el-arca-fin-del-festival}{%
\subsection{Establecimiento del servicio de portero, sacerdote y
cantante en el arca; Fin del
festival}\label{establecimiento-del-servicio-de-portero-sacerdote-y-cantante-en-el-arca-fin-del-festival}}

\bibleverse{37} Entonces David se aseguró de que Asaf y sus hermanos
ministraran continuamente ante el Arca del Pacto del Señor, realizando
los servicios que fueran necesarios cada día, \bibleverse{38} así como
Obed-edom y sus sesenta y ocho parientes. Obed-edom, hijo de Jedutún, y
Hosa, eran porteros. \bibleverse{39} David puso al sacerdote Sadoc y a
sus compañeros sacerdotes a cargo del Arca del Señor en el lugar alto de
Gabaón \footnote{\textbf{16:39} 1Cró 21,29} \bibleverse{40} para que
presentaran holocaustos al Señor en el altar de los holocaustos, por la
mañana y por la tarde, según todo lo que estaba escrito en la ley del
Señor que él había ordenado seguir a Israel. \footnote{\textbf{16:40}
  Éxod 29,38-39}

\bibleverse{41} Los acompañaban Hemán, Jedutún y el resto de los
elegidos e identificados por su nombre para dar gracias al Señor, porque
``su amor confiable es eterno''. \bibleverse{42} Hemán y Jedutún usaron
sus trompetas y címbalos para hacer música que acompañara los cantos de
Dios. Los hijos de Jedutún custodiaban la puerta. \bibleverse{43} Luego
todo el pueblo se fue a su casa, y David fue a bendecir a su familia.

\hypertarget{natuxe1n-aprueba-el-plan-de-david-para-construir-el-templo}{%
\subsection{Natán aprueba el plan de David para construir el
templo}\label{natuxe1n-aprueba-el-plan-de-david-para-construir-el-templo}}

\hypertarget{section-16}{%
\section{17}\label{section-16}}

\bibleverse{1} Una vez que David se instaló en su palacio, habló con el
profeta Natán. ``Mira'', le dijo David, ``¡Vivo en un palacio de cedro
mientras que el Arca del Pacto del Señor se guarda en una tienda!''.

\bibleverse{2} ``Haz lo que creas que debes hacer, porque el Dios está
contigo'', respondió Natán.

\hypertarget{dios-rechaza-el-plan-el-discurso-profuxe9tico-de-nathan-el-templo-seruxe1-construido-por-el-hijo-de-david}{%
\subsection{Dios rechaza el plan; El discurso profético de Nathan; el
templo será construido por el hijo de
David}\label{dios-rechaza-el-plan-el-discurso-profuxe9tico-de-nathan-el-templo-seruxe1-construido-por-el-hijo-de-david}}

\bibleverse{3} Pero esa noche Dios le dijo a Natán: \bibleverse{4} ``Ve
y habla con mi siervo David. Dile que esto es lo que dice el Señor: No
debes construir una casa para que yo viva en ella. \bibleverse{5} No he
vivido en una casa desde que saqué a Israel de Egipto\footnote{\textbf{17:5}
  ``De Egipto'': implícito. Estas palabras no aparecen en el texto
  hebreo.} hasta ahora. He vivido en tiendas, moviéndome de un lugar a
otro. \bibleverse{6} Pero en todos esos viajes con todo Israel nunca le
pregunté a ningún jefe israelita al que le hubiera ordenado cuidar de mi
pueblo: `¿Por qué no has construido una casa de cedro para mí?'

\bibleverse{7} Entonces, ve y dile a mi siervo David que esto es lo que
dice el Señor Todopoderoso. Fui yo quien te sacó del campo, del cuidado
de las ovejas, para convertirte en jefe de mi pueblo Israel.
\bibleverse{8} He estado contigo dondequiera que hayas ido. He derribado
a todos tus enemigos delante de ti, y haré que tu reputación sea tan
grande como la de las personas más famosas de la tierra. \bibleverse{9}
Elegiré un lugar para mi pueblo Israel. Allí los asentaré y ya no serán
molestados. Los malvados no los perseguirán como antes, \bibleverse{10}
desde que puse jueces a cargo de mi pueblo. Derrotaré a todos sus
enemigos. ``También quiero dejar claro que yo, el Señor, les construiré
una casa.\footnote{\textbf{17:10} En otras palabras, el Señor
  construiría una ``casa'' para David en el sentido de establecer una
  dinastía real.} \bibleverse{11} Porque cuando llegues al final de tu
vida y te unas a tus antepasados en la muerte, llevaré al poder a uno de
tus descendientes, a uno de tus hijos, y me aseguraré de que su reino
tenga éxito. \bibleverse{12} Él será quien me construya una casa, y me
aseguraré de que su reino dure para siempre. \footnote{\textbf{17:12}
  1Cró 22,10; 1Cró 28,6}

\hypertarget{la-gran-proclamaciuxf3n-de-salvaciuxf3n-de-dios-a-david-con-respecto-a-la-duraciuxf3n-eterna-de-su-casa}{%
\subsection{La gran proclamación de salvación de Dios a David con
respecto a la duración eterna de su
casa}\label{la-gran-proclamaciuxf3n-de-salvaciuxf3n-de-dios-a-david-con-respecto-a-la-duraciuxf3n-eterna-de-su-casa}}

\bibleverse{13} Yo seré un padre para él, y él será un hijo para mí.
Nunca le quitaré mi bondad y mi amor, como hice con el que gobernó antes
que tú. \bibleverse{14} Lo pondré al frente de mi casa y de mi reino
para siempre, y su dinastía durará para siempre''.

\hypertarget{acciuxf3n-de-gracias-y-suxfaplica-de-david}{%
\subsection{Acción de gracias y súplica de
David}\label{acciuxf3n-de-gracias-y-suxfaplica-de-david}}

\bibleverse{15} Esto es lo que Natán le explicó a David, todo lo que se
le dijo en esta revelación divina.

\bibleverse{16} Entonces el rey David fue y se sentó en presencia del
Señor. Oró: ``¿Quién soy yo, Señor Dios, y qué importancia alguna tiene
mi familia, para que me hayas traído hasta este lugar? \bibleverse{17}
Dios, hablas como si esto fuera poco a tus ojos, y también has hablado
del futuro de mi casa, de la dinastía de mi familia.\footnote{\textbf{17:17}
  ``La dinastía de mi familia'': explica el significado de ``casa'' en
  este contexto.} Tú también me ves como alguien muy importante, Señor
Dios. \bibleverse{18} ``¿Qué más puedo decir yo, David, para que me
honres de esta manera? ¡Tú conoces muy bien a tu siervo! \bibleverse{19}
Señor, haces todo esto por mí, tu siervo, y porque es lo que quieres:
hacer todas estas cosas increíbles y que la gente las conozca.
\bibleverse{20} ``Señor, realmente no hay nadie como tú; no hay otro
Dios, sólo tú. Nunca hemos oído hablar de ningún otro. \bibleverse{21}
¿Quién más es tan afortunado como tu pueblo Israel? ¿A quién más en la
tierra fue Dios a redimir para hacer su propio pueblo? Te ganaste una
maravillosa reputación por todas las cosas tremendas y asombrosas que
hiciste al expulsar a otras naciones ante tu pueblo cuando lo redimiste
de Egipto. \bibleverse{22} Hiciste tuyo a tu pueblo Israel para siempre,
y tú, Señor, te has convertido en su Dios. \bibleverse{23} ``Así que
ahora, Señor, haz que lo que has dicho de mí y de mi casa se cumpla y
dure para siempre. Por favor, haz lo que has prometido, \bibleverse{24}
y que tu verdadera naturaleza sea reconocida y honrada para siempre, y
que la gente declare: `¡El Señor Todopoderoso, el Dios de Israel, es el
Dios de Israel!' Que la casa de tu siervo David siga estando en tu
presencia. \bibleverse{25} ``Tú, Dios mío, me has explicado a mí, tu
siervo, que me construirás una casa. Por eso tu siervo ha tenido el
valor de orar a ti. \bibleverse{26} Porque tú, Señor, eres Dios. Tú eres
quien ha prometido todas estas cosas buenas a tu siervo. \bibleverse{27}
Así que ahora, por favor, bendice la casa de tu siervo para que continúe
en tu presencia para siempre. Porque cuando bendices, Señor, queda
bendecida para siempre''.

\hypertarget{las-victorias-de-david-sobre-los-filisteos-moabitas-sirios-y-edomitas}{%
\subsection{Las victorias de David sobre los filisteos, moabitas, sirios
y
edomitas}\label{las-victorias-de-david-sobre-los-filisteos-moabitas-sirios-y-edomitas}}

\hypertarget{section-17}{%
\section{18}\label{section-17}}

\bibleverse{1} Tiempo después, David derrotó a los filisteos y los
sometió, y capturó Gat y sus ciudades cercanas a los filisteos.
\bibleverse{2} David también derrotó a los moabitas, sometiéndolos y
obligándolos a pagar impuestos.

\hypertarget{las-victorias-de-david-sobre-los-sirios-el-uso-del-botuxedn-felicitaciones-del-rey-tou}{%
\subsection{Las victorias de David sobre los sirios; el uso del botín;
Felicitaciones del rey
Tou}\label{las-victorias-de-david-sobre-los-sirios-el-uso-del-botuxedn-felicitaciones-del-rey-tou}}

\bibleverse{3} Luego David derrotó a Hadad-Ezer, rey de Soba, cerca de
Hamat, cuando intentaba imponer su control a lo largo del río Éufrates.
\bibleverse{4} David le capturó 1. 000 carros, 7. 000 auriculares y 20.
000 soldados de a pie. David ató a todos los caballos de los carros,
pero guardó lo suficiente para 100 carros. \bibleverse{5} Cuando los
arameos de Damasco vinieron a ayudar a Hadad-Ezer, rey de Soba, David
mató a 22. 000 de ellos. \bibleverse{6} David puso fuerzas\footnote{\textbf{18:6}
  En el texto hebreo no se especifica qué colocó David. Según el texto,
  parece tratarse de unidades del ejército o guarniciones, como sugieren
  las traducciones de la Septuaginta y la Vulgata, y se confirma en el
  pasaje paralelo de 2 Samuel 8:6.} en la ciudad aramea de Damasco, y
también los sometió y les exigió el pago de impuestos. El Señor le dio a
David victorias dondequiera que fuera. \bibleverse{7} David tomó los
escudos de oro que llevaban los oficiales de Hadad-Ezer y los llevó a
Jerusalén. \bibleverse{8} David también tomó una gran cantidad de bronce
de Tibhat y de Cun, ciudades que habían pertenecido a Hadad-Ezer.
Salomón utilizó ese bronce para hacer el mar de bronce, las columnas y
los diversos objetos de bronce.\footnote{\textbf{18:8} Objetos
  utilizados en el Templo.} \footnote{\textbf{18:8} 1Re 7,23; 1Re 7,15}

\bibleverse{9} Cuando Tou, rey de Hamat, se enteró de que David había
destruido todo el ejército de Hadad-Ezer, rey de Soba, \bibleverse{10}
envió a su hijo Adoram donde David para que se hiciera amigo de él y lo
felicitara por su victoria en la batalla sobre Hadad-Ezer. Tou y
Hadad-Ezer habían estado a menudo en guerra. Adoram trajo regalos de
oro, plata y bronce. \bibleverse{11} El rey David dedicó estos regalos
al Señor, junto con la plata y el oro que había tomado de todas las
naciones siguientes: Edom, Moab, los amonitas, los filisteos y los
amalecitas.

\hypertarget{derrota-y-subyugaciuxf3n-de-los-edomitas}{%
\subsection{Derrota y subyugación de los
edomitas}\label{derrota-y-subyugaciuxf3n-de-los-edomitas}}

\bibleverse{12} Abisai,\footnote{\textbf{18:12} En el pasaje paralelo de
  2 Samuel 8:13 se atribuye a David esta victoria.} hijo de Sarvia, mató
a 18. 000 edomitas en el Valle de la Sal. \bibleverse{13} Estableció
puestos militares en Edom, y todos los edomitas se sometieron a David.
El Señor le dio a David victorias dondequiera que fuera.

\hypertarget{los-altos-funcionarios-de-david}{%
\subsection{Los altos funcionarios de
David}\label{los-altos-funcionarios-de-david}}

\bibleverse{14} David gobernó sobre todo Israel. Hizo lo que era justo y
correcto para todo su pueblo. \bibleverse{15} Joab, hijo de
Sarvia,\footnote{\textbf{18:15} Zeruiah era la hermana de David (2:16).}
era el comandante del ejército, mientras que Josafat, hijo de Ahilud,
llevaba los registros oficiales. \bibleverse{16} Sadoc, hijo de Ahitob,
y Ahimelec, hijo de Abiatar, eran los sacerdotes, mientras que Savsha
era el secretario. \bibleverse{17} Benaía, hijo de Joiada, estaba a
cargo de los queretanos y peletanos;\footnote{\textbf{18:17} ``Los
  queretanos y peletanos'': eran la guardia del rey (2 Samuel 15:18).} y
los hijos de David estaban al lado del rey, sirviendo como sus
principales funcionarios.

\hypertarget{el-vergonzoso-crimen-de-los-amonitas-contra-el-mensajero-de-david}{%
\subsection{El vergonzoso crimen de los amonitas contra el mensajero de
David}\label{el-vergonzoso-crimen-de-los-amonitas-contra-el-mensajero-de-david}}

\hypertarget{section-18}{%
\section{19}\label{section-18}}

\bibleverse{1} Algún tiempo después, Nahas, rey de los amonitas, murió y
su hijo lo sucedió. \bibleverse{2} Entonces David dijo: ``Seré bondadoso
con Hanún, hijo de Nahas, porque su padre fue bondadoso conmigo''. Así
que David envió mensajeros para consolarle por la muerte de su padre.
Los embajadores de David llegaron a la tierra de los amonitas y fueron a
consolar a Hanún.

\bibleverse{3} Pero los príncipes amonitas le dijeron a Hanún: ``¿De
verdad crees que David honra a tu padre enviándote a estos hombres para
consolarte? ¿Acaso no crees que han venido sólo a espiar la tierra para
encontrar la manera de conquistarla?'' \footnote{\textbf{19:3} 1Sam 3,18}
\bibleverse{4} Entonces Hanún detuvo a los embajadores de David y los
mandó a afeitar, y además les cortó la túnica a la altura de las
nalgas.\footnote{\textbf{19:4} Para humillarlos y avergonzarlos, y para
  enviar un mensaje de desafío a David.} Entonces los envió de vuelta.
\bibleverse{5} Luego informaron a David de lo que había sucedido con
estos hombres. Entonces David envió mensajeros a los hombres para
decirles: ``Quédense en Jericó hasta que les crezca la barba, y entonces
podrán regresar''.

\hypertarget{comienzo-de-la-guerra-primeros-trabajos-ganados}{%
\subsection{Comienzo de la guerra; primeros trabajos
ganados}\label{comienzo-de-la-guerra-primeros-trabajos-ganados}}

\bibleverse{6} Entonces los amonitas se dieron cuenta de que realmente
habían sido ofensivos con David. Así que Hanún y los amonitas enviaron
mil talentos de plata para contratar carros y sus conductores de
Aram-naharaim, Aram-maaca y Soba. \bibleverse{7} También contrataron 32.
000 carros y al rey de Maaca con su ejército. Vinieron a acampar cerca
de Medeba. Los amonitas también fueron llamados desde sus ciudades y se
prepararon para la batalla. \bibleverse{8} Cuando David se enteró de
esto, envió a Joab y a todo el ejército a enfrentarlos. \bibleverse{9}
Los amonitas establecieron sus líneas de batalla cerca de la entrada de
la ciudad, mientras que los otros reyes que se les habían unido tomaron
posiciones en los campos abiertos.

\bibleverse{10} Joab se dio cuenta de que tendría que luchar tanto
delante como detrás de él, así que escogió algunas de las mejores tropas
de Israel y se puso al frente de ellas para dirigir el ataque a los
arameos. \bibleverse{11} Puso al resto del ejército bajo el mando de
Abisai, su hermano. Debían atacar a los amonitas. \bibleverse{12} Joab
le dijo: ``Si los arameos son más fuertes que yo, ven a ayudarme. Si los
amonitas son más fuertes que tú, yo vendré a ayudarte. \bibleverse{13}
Sé valiente y lucha lo mejor que puedas por nuestro pueblo y las
ciudades de nuestro Dios. Que el Señor haga lo que considere bueno''.

\bibleverse{14} Joab atacó con sus fuerzas a los arameos y éstos huyeron
de él. \bibleverse{15} Cuando los amonitas vieron que los arameos habían
huido, también huyeron de Abisai, el hermano de Joab, y se retiraron a
la ciudad. Entonces Joab regresó a Jerusalén.

\hypertarget{david-personalmente-en-el-campo-su-victoria-sobre-los-sirios-aliados-con-los-amonitas}{%
\subsection{David personalmente en el campo; su victoria sobre los
sirios aliados con los
amonitas}\label{david-personalmente-en-el-campo-su-victoria-sobre-los-sirios-aliados-con-los-amonitas}}

\bibleverse{16} En cuanto los arameos vieron que habían sido derrotados
por los israelitas, enviaron a buscar refuerzos del otro lado del río
Éufrates, bajo el mando de Sobac, comandante del ejército de Hadad-Ezer.
\bibleverse{17} Cuando le informaron de esto a David, reunió a todo
Israel. Atravesó el Jordán y se acercó al ejército arameo, poniendo sus
fuerzas en línea de batalla contra ellos. Cuando David entró en combate
con ellos, ellos lucharon con él. \bibleverse{18} Pero el ejército
arameo huyó de los israelitas, y David mató a 7. 000 aurigas y 40. 000
soldados de infantería, así como a Sobac, su comandante. \bibleverse{19}
Cuando los aliados de Hadad-Ezer se dieron cuenta de que habían sido
derrotados por Israel, hicieron la paz con David y se sometieron a él.
Como resultado, los arameos no quisieron ayudar más a los amonitas.

\hypertarget{joab-conquista-rabuxe1-el-triunfo-de-david-y-el-castigo-de-los-amonitas}{%
\subsection{Joab conquista Rabá; El triunfo de David y el castigo de los
amonitas}\label{joab-conquista-rabuxe1-el-triunfo-de-david-y-el-castigo-de-los-amonitas}}

\hypertarget{section-19}{%
\section{20}\label{section-19}}

\bibleverse{1} En primavera, en la época del año en que los reyes salen
a hacer la guerra, Joab dirigió el ejército israelita en los ataques
contra el país de los amonitas, asediando también Rabá. Sin embargo,
David se quedó en Jerusalén. Joab atacó Rabá y la destruyó.
\bibleverse{2} David tomó la corona de la cabeza de su ídolo
Milcom.\footnote{\textbf{20:2} ``Milcom'': o ``su rey''.} Era de oro y
estaba engastado con gemas. Pesaba un talentob y fue colocado sobre la
cabeza de David. David también tomó una gran cantidad de botín de la
ciudad. \bibleverse{3} David hizo trabajar a la gente de allí con
sierras, picos de hierro y hachas. También hizo lo mismo con todas las
ciudades amonitas. Luego David y todo su ejército regresaron a
Jerusalén.

\hypertarget{algunas-hazauxf1as-de-los-guerreros-de-david-en-las-guerras-filisteas}{%
\subsection{Algunas hazañas de los guerreros de David en las guerras
filisteas}\label{algunas-hazauxf1as-de-los-guerreros-de-david-en-las-guerras-filisteas}}

\bibleverse{4} Algún tiempo después de esto estalló un conflicto con los
filisteos en Gezer. Pero entonces Sibecai de Husa mató a Sipai, un
descendiente de los refaítas,\footnote{\textbf{20:4} ``Refaim'': una
  raza de gigantes. Una palabra similar se utiliza en 20:8.} y los
filisteos se vieron obligados a someterse.

\bibleverse{5} En otra batalla con los filisteos, Elhanán, hijo de Jair,
mató a Lahmi, hermano de Goliat de Gat. El asta de su lanza era tan
gruesa como una vara de tejedor. \bibleverse{6} En otra batalla en Gat,
había un hombre gigantesco, que tenía seis dedos en cada mano y seis
dedos en cada pie, haciendo un total de veinticuatro. También él
descendía de los gigantes. \bibleverse{7} Pero cuando insultó a Israel,
Jonatán, hijo de Simea, hermano de David, lo mató. \footnote{\textbf{20:7}
  1Sam 17,10}

\bibleverse{8} Estos eran los descendientes de los gigantes en Gat, pero
todos fueron muertos por David y sus hombres.

\hypertarget{david-a-instigaciuxf3n-de-satanuxe1s-completa-el-censo-a-pesar-de-la-advertencia-de-joab-resultado-del-recuento}{%
\subsection{David, a instigación de Satanás, completa el censo a pesar
de la advertencia de Joab; Resultado del
recuento}\label{david-a-instigaciuxf3n-de-satanuxe1s-completa-el-censo-a-pesar-de-la-advertencia-de-joab-resultado-del-recuento}}

\hypertarget{section-20}{%
\section{21}\label{section-20}}

\bibleverse{1} Satanás interfirió para causar problemas a Israel.
Entonces provocó a David para que hiciera un censo de Israel.
\bibleverse{2} David les dijo a Joab y a los comandantes del ejército:
``Vayan a contar a los israelitas desde Beerseba hasta Dan. Luego
infórmenme para que tenga un número total''.

\bibleverse{3} Pero Joab respondió: ``Que el Señor multiplique su pueblo
cien veces. Su Majestad, ¿no son todos sus súbditos? ¿Por qué quieres
hacer esto? ¿Por qué culparás a Israel?''

\bibleverse{4} Pero el rey se mostró inflexible, así que Joab se marchó
y recorrió todo Israel. Finalmente regresó a Jerusalén, \bibleverse{5} y
le dio a David el número de personas censadas. En Israel había 1. 100.
000 hombres combatientes que podían manejar una espada, y 470. 000 en
Judá. \bibleverse{6} Sin embargo, Joab no incluyó a Leví y Benjamín en
el total del censo, porque no estaba de acuerdo con lo que el rey había
ordenado.

\hypertarget{el-arrepentimiento-de-david-intervenciuxf3n-del-profeta-gad-david-elige-una-muerte-popular-para-expiar-su-culpa}{%
\subsection{El arrepentimiento de David; Intervención del profeta Gad;
David elige una muerte popular para expiar su
culpa}\label{el-arrepentimiento-de-david-intervenciuxf3n-del-profeta-gad-david-elige-una-muerte-popular-para-expiar-su-culpa}}

\bibleverse{7} El Señor consideró que el censo era algo malo y castigó a
Israel por ello. \footnote{\textbf{21:7} 1Cró 27,24} \bibleverse{8}
Entonces David le dijo a Dios: ``He cometido un terrible pecado al hacer
esto. Por favor, quita la culpa de tu siervo, porque he sido muy
estúpido''.

\bibleverse{9} El Señor le dijo a Gad, el vidente de David,
\bibleverse{10} ``Ve y dile a David que esto es lo que dice el Señor:
`Te doy tres opciones. Elige una de ellas, y eso es lo que te haré'\,''.

\bibleverse{11} Gad fue y le dijo a David: ``Esto es lo que dice el
Señor: `Elige: \bibleverse{12} o tres años de hambre; o tres meses de
devastación, huyendo de las espadas de tus enemigos; o tres días de la
espada del Señor, es decir, tres días de plaga en la tierra, con un
ángel del Señor causando la destrucción en todo Israel'. Ahora tienes
que decidir cómo debo responder al que me ha enviado''.

\bibleverse{13} David respondió a Gad: ``¡Esta es una situación terrible
para mí! Por favor, deja que el Señor decida mi castigo,\footnote{\textbf{21:13}
  ``Deja que el Señor decida mi castigo'': Literalmente, ``déjame caer
  en las manos del Señor''. También al final del verso, ``no me dejes
  caer en manos humanas''.} porque es muy misericordioso. No permitas
que la gente me castigue''.

\hypertarget{el-juicio-divino-la-penitencia-y-la-suxfaplica-de-david}{%
\subsection{El juicio divino; La penitencia y la súplica de
David}\label{el-juicio-divino-la-penitencia-y-la-suxfaplica-de-david}}

\bibleverse{14} Entonces el Señor lanzó una plaga sobre Israel, y
murieron 70. 000 israelitas. \bibleverse{15} Dios también envió un ángel
para destruir Jerusalén. Pero justo cuando el ángel estaba a punto de
destruirla, el Señor lo vio, y renunció a causar tal desastre. Le dijo
al ángel destructor: ``Es suficiente. Ya puedes parar''. Justo en ese
momento el ángel del Señor estaba junto a la era de Ornán el jebuseo.
\bibleverse{16} Cuando David levantó la vista y vio al ángel del Señor
de pie entre la tierra y el cielo, con su espada desenvainada extendida
sobre Jerusalén, David y los ancianos, vestidos de saco, cayeron sobre
sus rostros.

\bibleverse{17} David le dijo a Dios: ``¿No fui yo quien ordenó el censo
del pueblo? Yo soy el que ha pecado y actuado con maldad. Pero estas
ovejas, ¿qué han hecho? Señor, Dios mío, por favor, castígame a mí y a
mi familia, pero no castigues a tu pueblo con esta plaga''.

\hypertarget{david-adquiere-la-era-de-ornuxe1n-y-la-dedica-a-un-lugar-de-sacrificio-y-templo-fin-de-la-plaga}{%
\subsection{David adquiere la era de Ornán y la dedica a un lugar de
sacrificio y templo; Fin de la
plaga}\label{david-adquiere-la-era-de-ornuxe1n-y-la-dedica-a-un-lugar-de-sacrificio-y-templo-fin-de-la-plaga}}

\bibleverse{18} Entonces el ángel del Señor le dijo a Gad que le dijera
a David que fuera a construir un altar al Señor en la era de Ornán el
jebuseo. \bibleverse{19} Así que David fue e hizo lo que Gad le había
dicho en nombre del Señor.

\bibleverse{20} Ornán estaba ocupado trillando trigo. Se volvió y vio al
ángel; y sus cuatro hijos que estaban con él fueron a esconderse.
\bibleverse{21} Cuando llegó David, Ornán se asomó y vio a David.
Abandonó la era y se inclinó ante David con el rostro en tierra.

\bibleverse{22} David le dijo a Ornán: ``Por favor, déjame la era. La
compraré por su precio completo. Así podré construir aquí un altar al
Señor para que cese la plaga del pueblo''.

\bibleverse{23} ``Tómala, y tu majestad podrá hacer lo que quiera con
ella'', le dijo Ornán a David. ``Puedes tener los bueyes para los
holocaustos, las tablas de trillar para la leña y el trigo para la
ofrenda de grano. Te lo daré todo''.

\bibleverse{24} ``No, insisto, pagaré el precio completo'', respondió el
rey David. ``No tomaré para el Señor lo que es tuyo ni presentaré
holocaustos que no me costaron nada''.

\bibleverse{25} Así que David pagó a Ornán seiscientos siclos de oro por
el lugar. \bibleverse{26} David construyó allí un altar al Señor y
presentó holocaustos y ofrendas de amistad. Invocó al Señor en oración,
y el Señor le respondió con fuego del cielo sobre el altar del
holocausto. \footnote{\textbf{21:26} 1Re 18,24}

\bibleverse{27} Entonces el Señor le dijo al ángel que volviera a
enfundar su espada.

\bibleverse{28} Cuando David vio que el Señor le había respondido en la
era de Ornán el jebuseo, ofreció allí sacrificios. \bibleverse{29} En
aquel tiempo, la tienda del Señor que Moisés había hecho en el desierto
y el altar del holocausto estaban en el lugar alto de Gabaón.
\bibleverse{30} Pero David no quiso ir allí a pedir la voluntad de
Dios,\footnote{\textbf{21:30} ``Pedir la voluntad de Dios'':
  Literalmente, ``preguntar a Dios''.} porque tenía miedo de la espada
del ángel del Señor.\footnote{\textbf{21:30} 1Cró 21,16}

\hypertarget{section-21}{%
\section{22}\label{section-21}}

\bibleverse{1} Entonces David dijo: ``Aquí estará la casa del Señor
Dios, y este es el lugar para el altar de los holocaustos para Israel''.
\footnote{\textbf{22:1} 2Cró 3,1}

\hypertarget{los-preparativos-de-david-para-la-construcciuxf3n-del-templo-colecciuxf3n-de-materiales-de-construcciuxf3n}{%
\subsection{Los preparativos de David para la construcción del templo;
Colección de materiales de
construcción}\label{los-preparativos-de-david-para-la-construcciuxf3n-del-templo-colecciuxf3n-de-materiales-de-construcciuxf3n}}

\bibleverse{2} Entonces David dio órdenes de convocar a los extranjeros
que vivían en la tierra de Israel, y asignó a canteros para que
prepararan piedras labradas para construir la casa de Dios. \footnote{\textbf{22:2}
  2Cró 2,16} \bibleverse{3} David proporcionó mucho hierro para hacer
los clavos de las puertas de entrada y de los soportes, así como más
bronce del que se podía pesar. \bibleverse{4} Proporcionó más troncos de
cedro de los que se podían contar, porque la gente de Sidón y Tiro le
había traído a David una enorme cantidad de troncos de cedro.
\bibleverse{5} David se dijo: ``Mi hijo Salomón es todavía joven e
inexperto, y la casa que va a construir para el Señor debe ser realmente
magnífica, famosa y gloriosa en todo el mundo. Tengo que empezar a
prepararla''. Así que David se aseguró de tener listos muchos materiales
de construcción antes de morir.

\hypertarget{instrucciones-de-david-a-su-hijo-salomuxf3n}{%
\subsection{Instrucciones de David a su hijo
Salomón}\label{instrucciones-de-david-a-su-hijo-salomuxf3n}}

\bibleverse{6} Luego mandó llamar a su hijo Salomón y le encargó que
construyera una casa para el Señor, el Dios de Israel. \bibleverse{7}
David le dijo a Salomón: ``Hijo mío, siempre había querido construir una
casa para honrar al Señor, mi Dios. \footnote{\textbf{22:7} 1Cró
  17,1-14; 1Cró 28,2-7} \bibleverse{8} Pero el Señor me dijo: `Has
derramado mucha sangre y has participado en muchas guerras. No debes
construir una casa para honrarme porque te he visto derramar mucha
sangre en la tierra. \bibleverse{9} Pero tendrás un hijo que será un
hombre de paz. Le daré la paz de todos sus enemigos en las naciones de
alrededor. Salomón será su nombre, y concederé paz y tranquilidad a
Israel durante su reinado. \bibleverse{10} Él es quien construirá una
casa para honrarme. Él será mi hijo, y yo seré su padre. Y me aseguraré
de que el trono de su reino sobre Israel dure para siempre'.
\bibleverse{11} ``Ahora, hijo mío, que el Señor te acompañe para que
logres construir la casa del Señor, tu Dios, tal como él dijo que lo
harías. \bibleverse{12} Que el Señor te dé inteligencia y entendimiento
cuando te ponga al frente de Israel, para que cumplas la ley del Señor,
tu Dios. \bibleverse{13} Entonces tendrás éxito, siempre y cuando sigas
las leyes y los reglamentos que el Señor, a través de Moisés, le ordenó
a Israel. ¡Sé fuerte y valiente! ¡No tengas miedo ni te desanimes!
\bibleverse{14} ``Mira, me he tomado muchas molestias para proveer la
casa del Señor: 100. 000 talentos de oro, 1. 000. 000 de talentos de
plata, y bronce y hierro, más de lo que se puede pesar. \footnote{\textbf{22:14}
  1Cró 29,2} \bibleverse{15} También he proporcionado madera y piedra,
pero tendrás que añadir más. \bibleverse{16} Tienes muchos trabajadores,
como canteros, albañiles, carpinteros y toda clase de artesanos del oro,
la plata, el bronce y el hierro, sin límite. Así que ponte en marcha, y
que el Señor te acompañe''.

\hypertarget{la-amonestaciuxf3n-de-david-a-los-pruxedncipes-de-israel}{%
\subsection{La amonestación de David a los príncipes de
Israel}\label{la-amonestaciuxf3n-de-david-a-los-pruxedncipes-de-israel}}

\bibleverse{17} David también ordenó a todos los dirigentes de Israel
que ayudaran a su hijo Salomón. \bibleverse{18} ``¿No está el Señor Dios
contigo? ¿No te ha dado la paz en todas tus fronteras?'' , preguntó.
``¿Por qué? Porque ha puesto a los habitantes de la tierra bajo mi
poder, y ahora están sometidos al Señor y a su pueblo. \bibleverse{19}
Ahora, con toda tu mente y tu corazón, toma la decisión definitiva de
adorar siempre al Señor, tu Dios. Comienza a construir el santuario del
Señor Dios, Entonces podrás llevar el Arca del Pacto del Señor y las
cosas sagradas de Dios a la casa que se va a construir para honrar al
Señor''.

\hypertarget{contando-y-ejecutando-los-levitas}{%
\subsection{Contando y ejecutando los
levitas}\label{contando-y-ejecutando-los-levitas}}

\hypertarget{section-22}{%
\section{23}\label{section-22}}

\bibleverse{1} Cuando David envejeció, habiendo vivido una larga vida,
nombró a su hijo Salomón rey de Israel. \footnote{\textbf{23:1} 1Re
  1,28-40} \bibleverse{2} También convocó a todos los jefes de Israel, a
los sacerdotes y a los levitas. \bibleverse{3} Se contaron los levitas
mayores de treinta años, y fueron 38. 000 en total. \bibleverse{4} ``De
ellos, 24. 000 estarán a cargo de las obras de la casa del Señor,
mientras que 6. 000 serán oficiales y jueces'', instruyó David.
\bibleverse{5} ``Y 4. 000 serán porteros, mientras que 4. 000 alabarán
al Señor usando los instrumentos musicales que he proporcionado para la
alabanza''.

\hypertarget{clasificaciuxf3n-de-los-levitas-seguxfan-gerson-kehath-y-merari}{%
\subsection{Clasificación de los levitas según Gerson, Kehath y
Merari}\label{clasificaciuxf3n-de-los-levitas-seguxfan-gerson-kehath-y-merari}}

\bibleverse{6} David los dividió en secciones correspondientes a Los
hijos de Levi: Gersón, Coat y Merari.

\bibleverse{7} Los hijos de Gersón: Ladán y Simei. \bibleverse{8} Los
hijos de Ladan: Jehiel (el jefe de familia), Zetham y Joel, tres en
total. \footnote{\textbf{23:8} 1Cró 26,21} \bibleverse{9} Los hijos de
Simei: Selomit, Haziel y Harán, tres en total. Estos eran los jefes de
las familias de Ladán. \bibleverse{10} Los hijos de Simei: Jahat,
Ziza,\footnote{\textbf{23:10} ``Ziza'': según la Septuaginta y la
  Vulgata, en hebreo se lee ``Zina'' (pero nótese el siguiente verso).}
Jeús y Bería, cuatro en total. \bibleverse{11} Jahat (el jefe de
familia), y Ziza (el segundo); pero como Jeús y Bería no tuvieron muchos
hijos, fueron contados como una sola familia.

\bibleverse{12} Los hijos de Coat: Amram, Izhar, Hebrón y Uziel-un total
de cuatro. \bibleverse{13} Los hijos de Amram: Aarón y Moisés. Aarón
estaba dedicado al servicio con las cosas más sagradas, para que él y
sus hijos presentaran siempre ofrendas al Señor, y ministraran ante él,
y dieran bendiciones en su nombre para siempre. \footnote{\textbf{23:13}
  1Cró 6,34; Heb 5,4; Deut 10,8} \bibleverse{14} En cuanto a Moisés, el
hombre de Dios, sus hijos estaban incluidos en la tribu de Leví.
\footnote{\textbf{23:14} Deut 33,1} \bibleverse{15} Los hijos de Moisés:
Gersón y Eliezer. \footnote{\textbf{23:15} Éxod 18,3-4} \bibleverse{16}
Los hijos de Gersón: Sebuel (el jefe de familia). \footnote{\textbf{23:16}
  1Cró 26,24} \bibleverse{17} Los hijos de Eliezer: Rehabías (el jefe de
familia). Eliezer no tuvo más hijos, pero Rehabías tuvo muchos hijos.
\footnote{\textbf{23:17} 1Cró 24,21-30} \bibleverse{18} Los hijos de
Izhar: Selomit (el jefe de familia). \bibleverse{19} Los hijos de
Hebrón: Jeria (el jefe de familia), Amarías (el segundo), Jahaziel (el
tercero) y Jecamán (el cuarto). \bibleverse{20} Los hijos de Uziel:
Micaías (el jefe de familia) e Isías (el segundo).

\bibleverse{21} Los hijos de Merari: Mahli y Musi. Los hijos de Mahli:
Eleazar y Cis. \bibleverse{22} Eleazar murió sin tener hijos, sólo
hijas. Sus primos, Los hijos de Cis, se casaron con ellas.
\bibleverse{23} Los hijos de Musi: Mahli, Eder y Jeremot, tres en total.

\hypertarget{instrucciones-oficiales-para-los-levitas}{%
\subsection{Instrucciones oficiales para los
levitas}\label{instrucciones-oficiales-para-los-levitas}}

\bibleverse{24} Estos eran los descendientes de Leví por familias, los
jefes de familia enumerados individualmente por su nombre: los que
tenían veinte años o más y servían en la casa del Señor. \bibleverse{25}
Porque David dijo: ``El Señor, el Dios de Israel, ha dado la paz a su
pueblo, y vivirá en Jerusalén para siempre. \footnote{\textbf{23:25} Jl
  4,21}

\bibleverse{26} Así que los levitas ya no necesitan llevar la Tienda ni
nada necesario para su servicio''. \bibleverse{27} De acuerdo con las
instrucciones finales de David, se contaron los levitas de veinte años o
más. \bibleverse{28} Su misión era ayudar a los descendientes de Aarón
en el servicio de la casa del Señor. Eran responsables de los patios y
las habitaciones, de la limpieza de todas las cosas sagradas y del
trabajo del servicio de la casa de Dios. \bibleverse{29} También eran
responsables del pan de la proposición que se ponía sobre la mesa, de la
harina especial para las ofrendas de grano, de los panes sin levadura,
de la cocción, de la mezcla y del manejo de todas las cantidades y
medidas. \bibleverse{30} También debían ponerse de pie todas las mañanas
para dar gracias y alabar al Señor, y hacer lo mismo al atardecer,
\bibleverse{31} y siempre que se presentaran holocaustos al Señor, ya
fuera en los sábados, lunas nuevas y días festivos. Debían servir
regularmente ante el Señor según el número que se les exigiera.
\bibleverse{32} Así, los levitas debían cumplir con la responsabilidad
de cuidar la Tienda del Encuentro y el santuario, y con sus hermanos los
descendientes de Aarón, servían a la casa del Señor.

\hypertarget{el-dibujo-de-las-24-clases-sacerdotales}{%
\subsection{El dibujo de las 24 clases
sacerdotales}\label{el-dibujo-de-las-24-clases-sacerdotales}}

\hypertarget{section-23}{%
\section{24}\label{section-23}}

\bibleverse{1} Los hijos de Aarón fueron colocados en divisiones de la
siguiente manera. Los hijos de Aarón eran Nadab, Abiú, Eleazar e Itamar.
\footnote{\textbf{24:1} 1Cró 23,6; 1Cró 5,29} \bibleverse{2} Pero Nadab
y Abiú murieron antes que su padre, y no tuvieron hijos. Sólo Eleazar e
Itamar continuaron como sacerdotes. \footnote{\textbf{24:2} Lev 10,1-2;
  Lev 10,12} \bibleverse{3} Con la ayuda de Sadoc, descendiente de
Eleazar, y de Itamar, descendiente de Ahimelec, David los colocó en
divisiones según sus funciones asignadas. \footnote{\textbf{24:3} 2Cró
  8,14} \bibleverse{4} Como los descendientes de Eleazar tenían más
jefes que los de Itamar, se dividieron así: dieciséis jefes de familia
de los descendientes de Eleazar, y ocho de los descendientes de Itamar.
\bibleverse{5} Se dividieron echando suertes, sin preferencia, porque
había oficiales del santuario y oficiales de Dios tanto de los hijos de
Eleazar como de los hijos de Itamar. \bibleverse{6} Semaías hijo de
Netanel, un levita, era el secretario. Anotó los nombres y las
asignaciones en presencia del rey, de los funcionarios, del sacerdote
Sadoc, de Ahimelec hijo de Abiatar y de los jefes de familia de los
sacerdotes y levitas. Una familia de Eleazar y otra de Itamar fueron
elegidas por turno.

\bibleverse{7} La primera suerte recayó en Joiarib. El segundo a
Jedaías. \bibleverse{8} La tercera a Harim. El cuarto a Seorim.
\bibleverse{9} La quinta a Malquías. La sexta a Mijamín. \bibleverse{10}
La séptima a Cos. La octava a Abías. \footnote{\textbf{24:10} Luc 1,5}
\bibleverse{11} La novena a Jesúa. La décima por Secanías.
\bibleverse{12} La undécima por Eliasib. La duodécima a Jacim.
\bibleverse{13} La decimotercera por Hupah. La decimocuarta por
Jeshebeab. \bibleverse{14} La decimoquinta por Bilga. El decimosexto a
Immer. \bibleverse{15} El decimoséptimo a Hezir. El decimoctavo a
Afisés. \bibleverse{16} La decimonovena a Petaías. El vigésimo a
Hezequiel. \bibleverse{17} El vigésimo primero a Jaquín. El vigésimo
segundo a Gamul. \bibleverse{18} El vigésimo tercero a Delaía. El
vigésimo cuarto a Maazías. \bibleverse{19} Este era el orden en que cada
grupo debía servir cuando entraba en la casa del Señor, siguiendo el
procedimiento que les había definido su antepasado Aarón, según las
instrucciones del Señor, el Dios de Israel.

\hypertarget{las-clases-levitas-y-sus-luxedderes}{%
\subsection{Las clases levitas y sus
líderes}\label{las-clases-levitas-y-sus-luxedderes}}

\bibleverse{20} Estos fueron el resto de los hijos de Leví: de Los hijos
de Amram: Shubael; de Los hijos de Shubael: Jehdeiah. \bibleverse{21}
Para Rehabía, de sus hijos Isías (el primogénito). \bibleverse{22} De
los Izharitas: Shelomoth; de Los hijos de Shelomoth: Jahat.
\bibleverse{23} Los hijos de Hebrón: Jeriah (el mayor), Amariah (el
segundo), Jahaziel (el tercero) y Jecamán (el cuarto). \bibleverse{24}
El hijo de Uziel: Miqueas; de Los hijos de Miqueas: Shamir.
\bibleverse{25} El hermano de Micaías: Isías; de Los hijos de Isías:
Zacarías. \bibleverse{26} Los hijos de Merari: Mahli y Musi. El hijo de
Jaaziah: Beno. \bibleverse{27} Los hijos de Merari: de Jaaziah: Beno,
Shoham, Zaccur e Ibri. \bibleverse{28} De Mahli: Eleazar, que no tuvo
hijos. \bibleverse{29} De Cis: el hijo de Cis, Jerajmeel.
\bibleverse{30} Los hijos de Musi: Mahli, Eder y Jerimot. Estos eran los
hijos de los levitas, según sus familias. \bibleverse{31} También
echaron suertes de la misma manera que sus parientes los descendientes
de Aarón. Lo hicieron en presencia del rey David, de Sadoc, de Ahimelec
y de los jefes de familia de los sacerdotes y de los levitas, tanto de
los jefes de familia como de sus hermanos menores.\footnote{\textbf{24:31}
  1Cró 25,8}

\hypertarget{el-sorteo-de-las-24-divisiones-de-los-cantantes-y-muxfasicos-sagrados}{%
\subsection{El sorteo de las 24 divisiones de los cantantes y músicos
sagrados}\label{el-sorteo-de-las-24-divisiones-de-los-cantantes-y-muxfasicos-sagrados}}

\hypertarget{section-24}{%
\section{25}\label{section-24}}

\bibleverse{1} David y los líderes de los levitas\footnote{\textbf{25:1}
  ``Líderes de los levitas'': Muchas traducciones lo traducen como
  ``comandantes del ejército'', lo que parece una función extraña para
  ellos aquí. Sin embargo, la palabra también se utiliza para designar a
  los líderes de una reunión de levitas (véase, por ejemplo, Números
  4:3; Números 8:24-25). Véase también 15:16 en este libro para una
  descripción similar.} eligió a hombres de las familias de Asaf, Hemán
y Jedutún para que sirvieran profetizando acompañados de liras, arpas y
címbalos. Esta es la lista de los que realizaron este servicio:
\footnote{\textbf{25:1} 1Cró 15,19} \bibleverse{2} De los hijos de Asaf:
Zaccur, José, Netanías y Asarela. Estos hijos de Asaf estaban bajo la
supervisión de Asaf, quien profetizaba bajo la supervisión del rey.
\bibleverse{3} De los hijos de Jedutún: Gedalías, Zeri, Jesaías, Simei,
Hasabías y Matatías, seis en total, bajo la supervisión de su padre
Jedutún, que profetizaban acompañados del arpa, dando gracias y alabando
al Señor. \bibleverse{4} De los hijos de Hemán: Buquías, Matanías,
Uziel, Sebuel, Jerimot, Hananías, Hanani, Eliatá, Giddalti,
Romamti-ezer, Josbecasa, Maloti, Hotir y Mahaziot. \bibleverse{5} Todos
estos hijos de Hemán, el vidente del rey, le fueron dados por las
promesas de Dios de honrarlo, pues Dios le dio a Hemán catorce hijos y
tres hijas. \footnote{\textbf{25:5} 1Cró 21,9; 2Cró 35,15}
\bibleverse{6} Todos ellos estaban bajo la supervisión de sus padres
para la música de la casa del Señor con címbalos, arpas y liras, para el
servicio de la casa de Dios. Asaf, Jedutún y Hemán estaban bajo la
supervisión del rey. \bibleverse{7} Junto con sus parientes, todos ellos
entrenados y hábiles en el canto al Señor, sumaban 288. \bibleverse{8}
Echaron suertes para cualquier responsabilidad que tuvieran, el menos
importante igual al más importante, el maestro al alumno.

\bibleverse{9} La primera suerte, que era para Asaf, recayó en José, sus
hijos y su hermano, 12 en total. La segunda recayó en Gedalías, sus
hijos y sus hermanos, 12 en total. \bibleverse{10} La tercera cayó en
manos de Zacur, sus hijos y sus hermanos, 12 en total. \bibleverse{11}
La cuarta cayó en manos de Izri, sus hijos y sus hermanos, 12 en total.
\bibleverse{12} La quinta cayó en manos de Netanías, sus hijos y sus
hermanos, 12 en total. \bibleverse{13} La sexta cayó en manos de
Buquías, sus hijos y sus hermanos, 12 en total. \bibleverse{14} La
séptima cayó en manos de Jesarela, sus hijos y sus hermanos, 12 en
total. \bibleverse{15} La octava cayó en manos de Jesaías, sus hijos y
sus hermanos, 12 en total. \bibleverse{16} La novena cayó en manos de
Matanías, sus hijos y sus hermanos, 12 en total. \bibleverse{17} La
décima cayó en manos de Simei, sus hijos y sus hermanos, 12 en total.
\bibleverse{18} La undécima cayó en manos de Azarel, sus hijos y sus
hermanos, 12 en total. \bibleverse{19} La duodécima cayó en manos de
Hasabías, sus hijos y sus hermanos, 12 en total. \bibleverse{20} La
decimotercera cayó en manos de Subael, sus hijos y sus hermanos, 12 en
total. \bibleverse{21} El decimocuarto cayó en manos de Matatías, sus
hijos y sus hermanos, 12 en total. \bibleverse{22} La decimoquinta cayó
en manos de Jerimot, sus hijos y sus hermanos, 12 en total.
\bibleverse{23} La decimosexta cayó en manos de Hananías, sus hijos y
sus hermanos, 12 en total. \bibleverse{24} La decimoséptima cayó en
manos de Josbecasa, sus hijos y sus hermanos, 12 en total.
\bibleverse{25} La decimoctava cayó en manos de Hanani sus hijos y sus
hermanos, 12 en total. \bibleverse{26} La decimonovena cayó en manos de
Maloti, sus hijos y sus hermanos, 12 en total. \bibleverse{27} La
vigésima cayó en manos de Eliata, sus hijos y sus hermanos, 12 en total.
\bibleverse{28} El vigésimo primero cayó en manos de Hotir, sus hijos y
sus hermanos, 12 en total. \bibleverse{29} El vigésimo segundo cayó en
manos de Gidalti, sus hijos y sus hermanos, 12 en total. \bibleverse{30}
El vigésimo tercero cayó en manos de Mahaziot, sus hijos y sus hermanos,
12 en total. \bibleverse{31} El vigésimo cuarto cayó en manos de
Romanti-Ezer, sus hijos y sus hermanos, 12 en total.

\hypertarget{divisiones-de-los-porteros-levuxedticos}{%
\subsection{Divisiones de los porteros
levíticos}\label{divisiones-de-los-porteros-levuxedticos}}

\hypertarget{section-25}{%
\section{26}\label{section-25}}

\bibleverse{1} Esta es una lista de las divisiones de los porteros. De
los corasitas: Meselemías hijo de Coré, uno de Los hijos de Asaf.
\footnote{\textbf{26:1} 2Cró 8,14; 2Cró 35,15} \bibleverse{2} Los hijos
de Meselemías: Zacarías (el mayor), Jediael (el segundo), Zebadías (el
tercero), Jatniel (el cuarto), \bibleverse{3} Elam (el quinto), Johanán
(el sexto) y Elioenai (el séptimo). \bibleverse{4} Los hijos de
Obed-edom: Semaías (el mayor), Jozabad (el segundo), Joa (el tercero),
Sacar (el cuarto), Natanel (el quinto), \bibleverse{5} Ammiel (el
sexto), Isacar (el séptimo) y Peuletai (el octavo), porque Dios había
bendecido a Obed-edom. \bibleverse{6} Semaías, hijo de Obed-edom, tenía
hijos que eran líderes capaces y tenían gran autoridad en la familia de
su padre. \bibleverse{7} Los hijos de Semaías: Othni, Refael, Obed y
Elzabad. Sus parientes, Elihú y Semaquías, también eran hombres capaces.
\bibleverse{8} Todos estos descendientes de Obed-edom, más sus hijos y
nietos, un total de sesenta y dos, eran hombres capaces, bien
calificados para su servicio. \bibleverse{9} Los dieciocho hijos y
hermanos de Meselemías también eran hombres capaces. \bibleverse{10}
Hosa, uno de los hijos de Merari, puso a Simri como líder entre sus
hijos, aunque no era el primogénito. \bibleverse{11} Entre sus otros
hijos estaban Hilcías (el segundo), Tebalías (el tercero) y Zacarías (el
cuarto). El total de los hijos y parientes de Osa era de trece.

\hypertarget{la-distribuciuxf3n-de-los-porteros-a-las-diferentes-localizaciones}{%
\subsection{La distribución de los porteros a las diferentes
localizaciones}\label{la-distribuciuxf3n-de-los-porteros-a-las-diferentes-localizaciones}}

\bibleverse{12} Estas divisiones de los porteros, a través de sus jefes
de familia, servían en la casa del Señor, al igual que sus hermanos.
\bibleverse{13} Cada puerta fue asignada por sorteo a diferentes
familias, la menos importante igual a la más importante. \bibleverse{14}
La suerte de la puerta oriental recayó en Meselemías.\footnote{\textbf{26:14}
  Véase 26:2. El hebreo aquí se lee ``Selemías''.} Entonces echaron
suertes sobre su hijo Zacarías, consejero sabio y perspicaz, y la suerte
de la puerta del norte le correspondió a él. \bibleverse{15} La suerte
de la puerta del sur correspondió a Obed-edom, y la del almacén a sus
hijos. \bibleverse{16} Supim y Hosa recibieron la puerta del oeste y la
puerta de Salet en el camino que sube. Estaban siempre
vigiladas.\footnote{\textbf{26:16} Literalmente, ``guardia de guardia''.
  El significado es incierto.} \bibleverse{17} Había seis levitas de
servicio cada día en la puerta oriental, cuatro en la puerta norte,
cuatro en la puerta sur y dos a la vez en el almacén. \bibleverse{18}
Seis estaban de servicio todos los días en la puerta oeste, cuatro en el
camino principal y dos en el patio. \bibleverse{19} Estas eran las
divisiones de los porteros de Los hijos de Coré y Los hijos de Merari.

\hypertarget{los-tesoreros-levuxedticos-y-los-funcionarios-de-la-administraciuxf3n}{%
\subsection{Los tesoreros levíticos y los funcionarios de la
administración}\label{los-tesoreros-levuxedticos-y-los-funcionarios-de-la-administraciuxf3n}}

\bibleverse{20} Otros levitas bajo el mando de Ahías estaban a cargo de
los tesoros de la casa de Dios y de los tesoros de lo que había sido
dedicado a Dios. \bibleverse{21} De Los hijos de Ladán, que eran los
descendientes de los gersonitas a través de Ladán, y eran los jefes de
familia de Ladán el gersonita: Jehieli. \footnote{\textbf{26:21} 1Cró
  23,8} \bibleverse{22} Los hijos de Jehieli, Zetam y su hermano Joel,
estaban a cargo de los tesoros de la casa del Señor. \bibleverse{23} De
los amramitas, los izaritas, los hebronitas y los uzielitas:
\bibleverse{24} Sebuel, descendiente de Gersón, hijo de Moisés, era el
principal encargado de los tesoros. \bibleverse{25} Sus parientes por
parte de Eliezer fueron Rehabías, Jesaías, Joram, Zicri y Selomot.
\footnote{\textbf{26:25} 1Cró 23,17} \bibleverse{26} Selomot y sus
parientes estaban a cargo de todos los tesoros de todo lo que había sido
dedicado por el rey David, por los jefes de familia que eran los
comandantes de millares y de centenas, y por los comandantes del
ejército. \bibleverse{27} Dedicaron una parte del botín que habían
ganado en la batalla para ayudar a mantener la casa del Señor.
\bibleverse{28} Selomot y sus parientes también se encargaron de las
ofrendas dedicadas al Señor por Samuel el vidente, Saúl hijo de Cis,
Abner hijo de Ner y Joab hijo de Sarvia. Todas las ofrendas dedicadas
eran responsabilidad de Selomot y sus parientes.

\bibleverse{29} De los izaritas: Quenanías y sus hijos recibieron
funciones externas como funcionarios y jueces sobre Israel.
\bibleverse{30} De los hebronitas: Hasabías y sus parientes, 1. 700
hombres capaces, fueron puestos a cargo del Israel al oeste del Jordán,
responsables de todo lo relacionado con la obra del Señor y el servicio
del rey. \bibleverse{31} También de los hebronitas salió Jericó, el
líder de los hebronitas según las genealogías familiares. En el año
cuarenta del reinado de David se examinaron los registros, y se
descubrieron hombres de gran capacidad en Jazer de Galaad.
\bibleverse{32} Entre los parientes de Jericó había 2. 700 hombres
capaces que eran líderes familiares. El rey David los puso a cargo de
las tribus de Rubén y Gad y de la media tribu de Manasés. Eran
responsables de todo lo relacionado con la obra del Señor y el servicio
del rey.

\hypertarget{los-doce-jefes-militares-los-caudillos-y-los-demuxe1s-altos-funcionarios-de-david-la-divisiuxf3n-del-ejuxe9rcito-en-doce}{%
\subsection{Los doce jefes militares, los caudillos y los demás altos
funcionarios de David; La división del ejército en
doce}\label{los-doce-jefes-militares-los-caudillos-y-los-demuxe1s-altos-funcionarios-de-david-la-divisiuxf3n-del-ejuxe9rcito-en-doce}}

\hypertarget{section-26}{%
\section{27}\label{section-26}}

\bibleverse{1} Esta es una lista de los israelitas, de los jefes de
familia, de los comandantes de millares y de los comandantes de
centenas, y de sus oficiales que servían al rey en todo lo relacionado
con las divisiones que estaban de servicio cada mes durante el año.
Había 24. 000 hombres en cada división.

\bibleverse{2} Al mando de la primera división para el primer mes,
estaba Jashobeam, hijo de Zabdiel. Tenía 24. 000 hombres en su división.
\bibleverse{3} Era descendiente de Fares y estaba a cargo de todos los
oficiales del ejército durante el primer mes. \bibleverse{4} Al mando de
la división para el segundo mes estaba Dodai el ahohita. Miclot era su
jefe de división. Tenía 24. 000 hombres en su división. \bibleverse{5}
El tercer comandante del ejército para el tercer mes era Benaía, hijo
del sacerdote Joiada. Era el jefe y había 24. 000 hombres en su
división. \bibleverse{6} Este era el mismo Benaía que era un gran
guerrero entre los Treinta, y estaba a cargo de los Treinta. Su hijo
Amizabad era el jefe de su división. \bibleverse{7} El cuarto, para el
cuarto mes, era Asael, hermano de Joab. Su hijo Zebadías fue su sucesor.
Tenía 24. 000 hombres en su división. \bibleverse{8} El quinto, para el
quinto mes, era el comandante del ejército Shamhuth el Izrahita. Tenía
24. 000 hombres en su división. \bibleverse{9} El sexto, para el sexto
mes, era Ira, hijo de Iqués de Tecoa. Tenía 24. 000 hombres en su
división. \bibleverse{10} El séptimo, para el séptimo mes, era Heles el
pelonita de la tribu de Efraín. Tenía 24. 000 hombres en su división.
\bibleverse{11} El octavo, para el octavo mes, era Sibecai de Husa, de
la tribu de Zera. Tenía 24. 000 hombres en su división. \bibleverse{12}
El noveno, para el noveno mes, era Abiezer, de Anatot, de la tribu de
Benjamín. Tenía 24. 000 hombres en su división. \bibleverse{13} El
décimo, para el décimo mes, era Maharai de Netofa, de la tribu de Zera.
Tenía 24. 000 hombres en su división. \bibleverse{14} El undécimo, para
el undécimo mes, era Benaía, de Piratón, de la tribu de Efraín. Tenía
24. 000 hombres en su división. \bibleverse{15} El duodécimo, para el
duodécimo mes, era Heldai de Netofa, de la familia de Otoniel. Tenía 24.
000 hombres en su división.

\hypertarget{los-doce-pruxedncipes-tribales-de-israel}{%
\subsection{Los doce príncipes tribales de
Israel}\label{los-doce-pruxedncipes-tribales-de-israel}}

\bibleverse{16} Esta es la lista de los jefes para las tribus de Israel:
para los rubenitas Eliezer, hijo de Zicri; para los simeonitas:
Sefatías, hijo de Maaca; \bibleverse{17} para Leví: Hasabías, hijo de
Quemuel; para Aarón: Sadoc; \bibleverse{18} para Judá: Eliú, hermano de
David; por Isacar Omri, hijo de Miguel; \bibleverse{19} para Zabulón
Ismaías, hijo de Abdías; por Neftalí: Jerimot, hijo de Azriel;
\bibleverse{20} por los efraimitas: Oseas, hijo de Azazías; por la media
tribu de Manasés Joel, hijo de Pedaías; \bibleverse{21} para la media
tribu de Manasés en Galaad Iddo, hijo de Zacarías; por Benjamín:
Jaasiel, hijo de Abner; \bibleverse{22} por Dan: Azarel, hijo de
Jeroham. Estos fueron los oficiales para las tribus de Israel.

\hypertarget{comentar-el-censo-incompleto}{%
\subsection{Comentar el censo
incompleto}\label{comentar-el-censo-incompleto}}

\bibleverse{23} David no censó a los hombres menores de veinte años
porque el Señor había dicho que haría a Israel tan numeroso como las
estrellas del cielo. \footnote{\textbf{27:23} Gén 22,17} \bibleverse{24}
Joab, hijo de Sarvia, había comenzado el censo, pero no lo terminó.
Israel fue castigado a causa de este censo, y los resultados no fueron
registrados en la cuenta oficial del rey David.\footnote{\textbf{27:24}
  Sin embargo, se registraron de forma resumida. Véase 21:5.}
\footnote{\textbf{27:24} 1Cró 21,14}

\hypertarget{los-administradores-de-la-propiedad-real-tesorero-y-maestro-de-alquileres}{%
\subsection{Los administradores de la propiedad real (tesorero y maestro
de
alquileres)}\label{los-administradores-de-la-propiedad-real-tesorero-y-maestro-de-alquileres}}

\bibleverse{25} Azmavet, hijo de Adiel, estaba a cargo de los almacenes
del rey, mientras que Jonatán, hijo de Uzías, estaba a cargo de los del
campo, las ciudades, las aldeas y las torres de vigilancia.
\bibleverse{26} Ezri, hijo de Quelub, estaba a cargo de los campesinos
que trabajaban la tierra. \bibleverse{27} Simei, el ramatita, estaba a
cargo de las viñas. Zabdi el sifmita estaba a cargo del producto de las
viñas para las bodegas. \bibleverse{28} Baal-Hanán el gederita estaba a
cargo de los olivos y los sicómoros de las colinas. Joás estaba a cargo
de los almacenes de aceite de oliva. \bibleverse{29} Sitrai de Sarón
estaba a cargo del ganado en los pastos de Sarón. Safat, hijo de Adlai,
estaba a cargo del ganado en los valles. \bibleverse{30} Obil el
ismaelita estaba a cargo de los camellos. Jehedías de Meronot estaba a
cargo de los asnos. \bibleverse{31} Jaziz el agareno estaba a cargo de
las ovejas y las cabras. Todos estos eran funcionarios a cargo de lo que
pertenecía al rey David.

\hypertarget{los-muxe1s-altos-funcionarios-imperiales-consejeros-del-rey}{%
\subsection{Los más altos funcionarios imperiales (consejeros del
rey)}\label{los-muxe1s-altos-funcionarios-imperiales-consejeros-del-rey}}

\bibleverse{32} Jonatán, tío de David, era un consejero, un hombre
perspicaz y un escriba. Jehiel, hijo de Hacmoni, cuidaba de los hijos
del rey. \bibleverse{33} Ahitofel era el consejero del rey y Husai, el
arquita, era el amigo del rey. \footnote{\textbf{27:33} 2Sam 15,12; 2Sam
  15,37} \bibleverse{34} Después de Ahitofel vino Joiada, hijo de Benaía
y de Abiatar. Joab era el comandante del ejército real.\footnote{\textbf{27:34}
  2Sam 8,16}

\hypertarget{el-discurso-de-david-a-los-jefes-de-israel}{%
\subsection{El discurso de David a los jefes de
Israel}\label{el-discurso-de-david-a-los-jefes-de-israel}}

\hypertarget{section-27}{%
\section{28}\label{section-27}}

\bibleverse{1} David convocó a Jerusalén a todos los dirigentes de
Israel: los jefes de las tribus, los comandantes de las divisiones del
ejército al servicio del rey, los comandantes de millares y los
comandantes de centenas, y los funcionarios encargados de todas las
propiedades y el ganado del rey y de sus hijos, junto con los
funcionarios de la corte, los guerreros y todos los mejores
combatientes.

\hypertarget{david-presenta-al-superior-del-pueblo-a-salomuxf3n-como-su-sucesor}{%
\subsection{David presenta al superior del pueblo a Salomón como su
sucesor}\label{david-presenta-al-superior-del-pueblo-a-salomuxf3n-como-su-sucesor}}

\bibleverse{2} El rey David se puso en pie y dijo: ``¡Escúchenme,
hermanos míos y pueblo! Yo quería construir una casa como lugar de
descanso para el Arca del Pacto del Señor, como escabel para nuestro
Dios. Así que hice planes para construirla. \footnote{\textbf{28:2} 1Cró
  22,7-10} \bibleverse{3} Pero Dios me dijo: `No debes construir una
casa para honrarme, porque eres un hombre de guerra que ha derramado
sangre'. \footnote{\textbf{28:3} 2Sam 7,5} \bibleverse{4} ``Sin embargo,
el Señor, el Dios de Israel, me eligió de entre toda la familia de mi
padre para ser rey de Israel para siempre. Porque eligió a Judá como
tribu principal, y de entre las familias de Judá eligió a la familia de
mi padre. De entre los hijos de mi padre se complació en elegirme rey de
todo Israel. \footnote{\textbf{28:4} Gén 49,10; 1Sam 16,1; 1Sam 16,12}
\bibleverse{5} De entre todos mis hijos (porque el Señor me dio muchos)
el Señor ha elegido a mi hijo Salomón para que se siente en el trono y
gobierne el reino del Señor, Israel. \bibleverse{6} Me dijo: 'Tu hijo
Salomón es el que construirá mi casa y mis atrios, porque lo he elegido
como hijo mío, y yo seré su padre. \bibleverse{7} Me aseguraré de que su
reino sea eterno si cumple con mis mandamientos y normas como lo hace
hoy.

\bibleverse{8} ``Así que ahora, a la vista de todo Israel, de la
asamblea del Señor, y mientras Dios te escucha, asegúrate de obedecer
todos los mandamientos del Señor, tu Dios, para que sigas poseyendo esta
buena tierra y puedas transmitirla como herencia a tus descendientes
para siempre.

\hypertarget{las-instrucciones-de-david-y-su-contribuciuxf3n-a-salomuxf3n}{%
\subsection{Las instrucciones de David y su contribución a
Salomón}\label{las-instrucciones-de-david-y-su-contribuciuxf3n-a-salomuxf3n}}

\bibleverse{9} ``Salomón, hijo mío, conoce al Dios de tu padre. Sírvele
con total dedicación y con una mente dispuesta, porque el Señor examina
cada motivación y entiende la intención de cada pensamiento. Si lo
buscas, lo encontrarás; pero si lo abandonas, te rechazará para siempre.
\footnote{\textbf{28:9} Sal 7,10} \bibleverse{10} Presta atención ahora,
porque el Señor te ha elegido para construir una casa para el santuario.
Sé fuerte y haz el trabajo'''.

\hypertarget{david-le-da-a-salomuxf3n-el-modelo-de-la-casa-del-templo-y-los-tesoros-recolectados-para-su-construcciuxf3n}{%
\subsection{David le da a Salomón el modelo de la casa del templo y los
tesoros recolectados para su
construcción}\label{david-le-da-a-salomuxf3n-el-modelo-de-la-casa-del-templo-y-los-tesoros-recolectados-para-su-construcciuxf3n}}

\bibleverse{11} Entonces David le dio a su hijo Salomón los planos del
pórtico del Templo, de sus edificios, de los almacenes, de las salas
superiores, de las salas interiores y de la sala para el ``lugar de
expiación''. \bibleverse{12} También le dio todo lo que había planeado
para los atrios de la casa del Señor, para todas las habitaciones
circundantes, para los tesoros de la casa de Dios y de las cosas que
habían sido dedicadas. \bibleverse{13} Además, le dio instrucciones
sobre las divisiones de los sacerdotes y de los levitas, para todo el
trabajo de servicio de la casa del Señor y para todo lo que se utilizaba
para el culto en la casa del Señor. \bibleverse{14} También estableció
la cantidad de oro y plata que debía emplearse en la fabricación de los
diferentes objetos utilizados en todo tipo de servicio,\footnote{\textbf{28:14}
  En los siguientes versos hay muchas repeticiones, por lo que la
  traducción se ha simplificado para mayor claridad.} \bibleverse{15} el
peso de los candelabros de oro y de plata y de sus lámparas, según el
uso de cada candelabro; \bibleverse{16} el peso del oro para cada mesa
de los panes de la proposición, y el peso de la plata para las mesas de
plata, \bibleverse{17} el peso del oro puro para los tenedores, las
jofainas y las copas; el peso de cada plato de oro; el peso de cada
cuenco de plata; \bibleverse{18} el peso del oro refinado para el altar
del incienso; y, por último, los planos de un carro de oro con
querubines que despliegan sus alas, cubriendo el Arca del Pacto del
Señor. \bibleverse{19} ``Todo esto está por escrito de la mano del
Señor, que me ha sido dado como instrucciones: cada detalle de este
plan'', dijo David.

\bibleverse{20} Entonces David también le dijo a Salomón: ``¡Sé fuerte,
sé valiente y actúa! No tengas miedo ni te desanimes, porque el Señor,
mi Dios, está contigo. Él no te dejará ni te abandonará. Él se encargará
de que todo el trabajo para el servicio de la casa del Señor esté
terminado. \footnote{\textbf{28:20} 1Cró 22,13; Deut 31,6}

\bibleverse{21} Las divisiones de los sacerdotes y los levitas están
preparadas para todo el servicio de la casa de Dios. La gente estará
dispuesta a usar sus diferentes habilidades para ayudarte en todo el
trabajo; los funcionarios y todo el pueblo harán lo que tú les digas''.

\hypertarget{la-contribuciuxf3n-de-los-pruxedncipes-a-la-construcciuxf3n-del-templo-siguiendo-la-amonestaciuxf3n-de-david}{%
\subsection{La contribución de los príncipes a la construcción del
templo siguiendo la amonestación de
David}\label{la-contribuciuxf3n-de-los-pruxedncipes-a-la-construcciuxf3n-del-templo-siguiendo-la-amonestaciuxf3n-de-david}}

\hypertarget{section-28}{%
\section{29}\label{section-28}}

\bibleverse{1} Entonces el rey David dijo a todos los allí reunidos:
``Mi hijo Salomón, elegido sólo por Dios, es joven e inexperto, y el
trabajo a realizar es grande porque este Templo\footnote{\textbf{29:1}
  ``Templo'': la palabra también puede traducirse como ``palacio'' o
  ``fortaleza''} no será para el hombre, sino para el Señor Dios.
\bibleverse{2} Con todos mis medios he provisto para la casa de mi Dios:
oro para los artículos de oro, plata para la plata, bronce para el
bronce, hierro para el hierro y madera para la madera; piedras de ónice
y piedras para los engastes: turquesa, piedras de diferentes colores,
toda clase de piedras preciosas; y mucho mármol. \bibleverse{3}
``Además, por mi devoción a la casa de mi Dios, ahora doy mi fortuna
personal de oro y plata, además de todo lo que he provisto para esta
santa casa. \bibleverse{4} 3. 000 talentos de oro -el oro de Ofir- y 7.
000 talentos de plata refinada irán a cubrir las paredes de los
edificios, \bibleverse{5} el oro para la orfebrería, y la plata para la
platería, y para todo el trabajo de los artesanos. ¿Quién quiere
comprometerse de buen grado a dar hoy al Señor?'' \footnote{\textbf{29:5}
  Éxod 35,5}

\bibleverse{6} Dieron de buena gana: los jefes de familia, los
responsables de las tribus de Israel, los comandantes de millares y de
centenas, y los funcionarios encargados de la obra del rey.
\bibleverse{7} Dieron para el servicio de la casa de Dios 5. 000
talentos y 10. 000 dáricos\footnote{\textbf{29:7} Un dárico era una
  moneda persa.} de oro, 10. 000 talentos de plata, 18. 000 talentos de
bronce y 100. 000 talentos de hierro. \bibleverse{8} Los que tenían
piedras preciosas las entregaron al tesoro de la casa del Señor, bajo la
supervisión de Jehiel el gersonita. \bibleverse{9} El pueblo celebró
porque sus líderes habían estado tan dispuestos a dar al Señor,
libremente y de todo corazón. El rey David también se alegró mucho.

\hypertarget{oraciuxf3n-final-de-david}{%
\subsection{Oración final de David}\label{oraciuxf3n-final-de-david}}

\bibleverse{10} Entonces David alabó al Señor ante toda la asamblea:
``¡Alabado seas, Señor, Dios de Israel, nuestro padre, por los siglos de
los siglos! \bibleverse{11} Señor, tuyos son la grandeza, el poder, la
gloria, el esplendor y la majestad, porque todo lo que hay en el cielo y
en la tierra es tuyo. Señor, tuyo es el reino, y eres admirado como
gobernante de todo. \footnote{\textbf{29:11} Apoc 4,11; Apoc 5,13}
\bibleverse{12} Las riquezas y el honor provienen de ti y tú reinas de
forma suprema. Tú posees el poder y la fuerza, y tienes la capacidad de
engrandecer a las personas y de dar fuerza a todos. \footnote{\textbf{29:12}
  2Cró 20,6} \bibleverse{13} ``Ahora, nuestro Dios, te damos gracias y
te alabamos a ti y a tu glorioso carácter. \bibleverse{14} Pero, ¿quién
soy yo y quién es mi pueblo, para que seamos capaces de dar de tan buena
gana? Porque todo lo que tenemos viene de ti; sólo te devolvemos lo que
tú nos has dado. \bibleverse{15} A tus ojos somos extranjeros y
forasteros, como nuestros antepasados. Nuestro tiempo aquí en la tierra
pasa como una sombra, no tenemos esperanza de quedarnos mucho tiempo.
\footnote{\textbf{29:15} Sal 39,13; Heb 11,13; Job 14,2} \bibleverse{16}
``Señor, Dios nuestro, incluso toda esta riqueza que hemos proporcionado
para construirte una casa para tu santo nombre proviene de lo que tú
das, y todo te pertenece. \bibleverse{17} Yo sé, Dios mío, que tú miras
por dentro y te alegras cuando vivimos bien. Todo lo he dado de buena
gana y con un corazón honesto, y ahora he visto a tu pueblo aquí dando
felizmente y de buena gana para ti. \bibleverse{18} Señor, el Dios de
Abrahán, de Isaac, de Israel y de nuestros antepasados, por favor,
mantén estos pensamientos y compromisos en la mente de tu pueblo para
siempre, y haz que permanezcan leales\footnote{\textbf{29:18} ``Haz que
  permanezcan leales''. Literalmente, ``de corazón''} a ti.
\bibleverse{19} Por favor, dale también a mi hijo Salomón el deseo de
cumplir de todo corazón tus mandamientos, decretos y estatutos, y de
hacer todo lo posible para construir tu Templo que yo he dispuesto''.

\hypertarget{final-solemne-de-la-reuniuxf3n-la-unciuxf3n-de-salomuxf3n-como-rey-fin-del-reinado-de-david}{%
\subsection{Final solemne de la reunión; La unción de Salomón como rey;
Fin del reinado de
David}\label{final-solemne-de-la-reuniuxf3n-la-unciuxf3n-de-salomuxf3n-como-rey-fin-del-reinado-de-david}}

\bibleverse{20} Entonces David dijo a todos los presentes: ``¡Alaben al
Señor, su Dios!'' Y todos alabaron al Señor, el Dios de sus padres. Se
inclinaron en reverencia ante el Señor y ante el rey.

\bibleverse{21} Al día siguiente presentaron sacrificios y holocaustos
al Señor: mil toros, mil carneros y mil corderos, con sus libaciones y
abundantes sacrificios para todo Israel. \bibleverse{22} Entonces
comieron y bebieron en presencia del Señor con gran alegría aquel día.
Hicieron rey por segunda vez a Salomón, hijo de David, y lo ungieron
como gobernante del Señor, y ungieron a Sadoc como sacerdote.
\footnote{\textbf{29:22} 1Cró 23,1}

\bibleverse{23} Así Salomón ocupó el trono del Señor como rey en lugar
de David, su padre. Tuvo éxito, y todos los israelitas le obedecieron.
\footnote{\textbf{29:23} 1Cró 28,5; 1Re 1,35; 1Re 1,39} \bibleverse{24}
Todos los funcionarios y guerreros, así como todos los hijos del rey
David, hicieron una promesa solemne de lealtad al rey Salomón.
\bibleverse{25} El Señor hizo que Salomón fuera muy respetado en todo
Israel, y le dio mayor majestad real que la que se le había dado a
cualquier otro rey de Israel antes de él. \footnote{\textbf{29:25} 2Cró
  1,1}

\hypertarget{el-final-de-david-y-las-fuentes-de-su-historia}{%
\subsection{El final de David y las fuentes de su
historia}\label{el-final-de-david-y-las-fuentes-de-su-historia}}

\bibleverse{26} Así, David, hijo de Isaí, gobernó sobre todo Israel.
\bibleverse{27} Gobernó sobre Israel cuarenta años: siete en Hebrón y
treinta y tres en Jerusalén. \bibleverse{28} David murió a una buena
edad, habiendo vivido una larga vida bendecida con riquezas y honor.
Entonces su hijo Salomón tomó el relevo y gobernó en su lugar.
\bibleverse{29} Todo lo que hizo el rey David, desde el principio hasta
el final, está escrito en las Actas de Samuel el Vidente, las Actas de
Natán el Profeta y las Actas de Gad el Vidente. \bibleverse{30} Estos
incluyen todos los detalles de su reinado, su poder y lo que le sucedió
a él, a Israel y a todos los reinos de los países vecinos.
