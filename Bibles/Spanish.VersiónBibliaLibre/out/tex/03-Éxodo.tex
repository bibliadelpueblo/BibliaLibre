\hypertarget{los-hijos-de-israel-multiplicaron}{%
\subsection{Los hijos de Israel
multiplicaron}\label{los-hijos-de-israel-multiplicaron}}

\hypertarget{section}{%
\section{1}\label{section}}

\bibleverse{1} Estos fueron los nombres de los hijos de Israel (Jacob)
que vinieron con él y sus familias a Egipto: \footnote{\textbf{1:1} Gén
  46,8} \bibleverse{2} Rubén, Simeón, Leví y Judá; \bibleverse{3}
Isacar, Zabulón y Benjamín; \bibleverse{4} Dan y Neftalí, Gad y Aser.
\bibleverse{5} Allí Jacob tuvo 70 descendientes, incluyendo a José, que
ya estaba en Egipto. \bibleverse{6} Finalmente José, todos sus hermanos,
y toda esa generación murieron. \footnote{\textbf{1:6} Gén 50,26}
\bibleverse{7} Sin embargo, los israelitas tenían muchos hijos y su
número aumentaba rápidamente. De hecho, eran tantos que se volvieron muy
poderosos, y el país estaba lleno de ellos. \footnote{\textbf{1:7} Hech
  7,17}

\bibleverse{8} Entonces subió al poder un nuevo rey que no tenía ningún
conocimiento acerca de José.\footnote{\textbf{1:8} Se cree que se
  refiere a una dinastía egipcia diferente.} \bibleverse{9} Este rey se
reunió con sus compatriotas egipcios y les dijo: ``Debemos tener cuidado
con estos israelitas, pues son más numerosos y más poderosos que
nosotros. \bibleverse{10} Tenemos que hacer un plan para evitar que
sigan multiplicándose, porque que si llega a haber una guerra, se
pondrán del lado de nuestros enemigos, lucharán contra nosotros, y
huirán del país''. \bibleverse{11} Entonceslos egipcios comenzaron a
obligarlos a hacer trabajos forzados y asignaron capataces para que
estuvieran a cargo de ellos. Los usaron para construir las ciudades de
almacenamiento de Pitón y Ramsés. \footnote{\textbf{1:11} Gén 15,13; Gén
  47,11} \bibleverse{12} Pero cuanto más maltrataban a los israelitas,
más se multiplicaban y se extendían, y también los egipcios los
detestaban\footnote{\textbf{1:12} ``Detestaban'' o ``temían''.} aún más.
\bibleverse{13} Los egipcios trataban a los israelitas con violencia,
\bibleverse{14} haciendo de sus vidas una miseria. Los obligaban a hacer
trabajos duros, construyendo con mortero y ladrillos, y haciendo todo
tipo de trabajo pesado en los campos. Y en medio de todo este trabajo
duro los trataban con crueldad.

\hypertarget{el-temor-de-dios-de-los-dos-parteras}{%
\subsection{El temor de Dios de los dos
parteras}\label{el-temor-de-dios-de-los-dos-parteras}}

\bibleverse{15} Entonces el rey les dio órdenes a las parteras hebreas
Sifra y Fúa. \bibleverse{16} Y les dijo: ``Cuando ayuden a las mujeres
hebreas durante el parto, si ven que es un niño, mátenlo; pero si es una
niña, déjenla vivir''. \bibleverse{17} Pero como las parteras respetaban
a Dios, no hicieron lo que el rey de Egipto les había ordenado, sinoque
dejaban vivir a los niños también. \bibleverse{18} Entonces el rey de
Egipto llamó a las parteras y les preguntó: ``¿Por qué han hecho esto, y
han dejado vivir a los niños varones?''

\bibleverse{19} ``Las mujeres hebreas no son como las egipcias'', le
dijeron las parteras al Faraón. ``Dan a luz más fácilmente, y tienen a
sus hijos antes de que lleguen las parteras''.

\bibleverse{20} Y Dios trató bien a las parteras, y el pueblo aumentó en
número, así que había aún muchos más de ellos. \bibleverse{21} Y como
las parteras reverenciaban a Dios, él les dio familias propias.
\bibleverse{22} Entonces el Faraón emitió esta orden a todo su pueblo:
``Arrojen al Nilo a todo niño hebreo que nazca, y por el contrario,
dejen vivir a las niñas''.

\hypertarget{nacimiento-y-abandono-salvaciuxf3n-y-educaciuxf3n-de-moisuxe9s}{%
\subsection{Nacimiento y abandono, salvación y educación de
Moisés}\label{nacimiento-y-abandono-salvaciuxf3n-y-educaciuxf3n-de-moisuxe9s}}

\hypertarget{section-1}{%
\section{2}\label{section-1}}

\bibleverse{1} Fue por esta época que un hombre de la tribu de Levi se
casó con una mujer, también levita. \bibleverse{2} Ella quedó embarazada
y tuvo un hijo. Y viendo que era un bebé precioso, lo escondió durante
tres meses. \footnote{\textbf{2:2} Hech 7,20; Heb 11,23} \bibleverse{3}
Pero cuando ya no pudo esconderlo más, cogió una cesta de papiro y la
cubrió con alquitrán. Luego puso a su bebé en la cesta y lo colocó entre
los juncos, a lo largo de la orilla del Nilo. \bibleverse{4} Yla hermana
del niño esperaba a cierta distancia, vigilándolo. \bibleverse{5}
Entonces la hija del Faraón llegó para bañarse en el Nilo. Sus criadas
caminaban por la orilla del río, y cuando ella vio la cesta entre los
juncos, envió a su criada a buscarla y traérsela. \bibleverse{6} Al
abrirla, vio al niño que lloraba y sintió pesar por él. ``Este debe ser
uno de los niños hebreos'', dijo.

\bibleverse{7} Entoncesla hermana del niño le preguntó a la hija del
Faraón: ``¿Desea que vaya a buscar a una de las mujeres hebreas para que
lo cuide por usted?''

\bibleverse{8} ``Sí, ve y hazlo'', respondió ella. Así que la niña fue y
llamó a la madre del bebé para que viniera.

\bibleverse{9} ``Toma a este niño y amamántalo por mí'', le dijo la hija
del Faraóna la madre del niño. ``Yo misma te pagaré''. Así que su madre
se lo llevó a casa y lo cuidó.

\bibleverse{10} Cuando el niño creció, se lo llevó a la hija del Faraón,
quien lo adoptó como su hijo. Ella lo llamó Moisés,\footnote{\textbf{2:10}
  ``Moisés'' suena como la palabra hebrea ``sacar''. En egipcio es una
  abreviatura que significa ``hijo de''.} porque dijo: ``Yo lo saqué del
agua''.

\hypertarget{moisuxe9s-matuxf3-al-egipcio}{%
\subsection{Moisés mató al Egipcio}\label{moisuxe9s-matuxf3-al-egipcio}}

\bibleverse{11} Más tarde, cuando Moisés había crecido, fue a visitar a
su pueblo, los hebreos. Los vio haciendo trabajos forzados. También vio
a un egipcio golpeando a un hebreo, uno de su propio pueblo. \footnote{\textbf{2:11}
  Heb 11,24-25} \bibleverse{12} Entonces miró a su alrededor para
asegurarse de que nadie estuviera mirando, yluego mató al egipcio y
enterró su cuerpo en la arena. \footnote{\textbf{2:12} Hech 7,24}

\bibleverse{13} Al día siguiente, regresó y vio a dos hebreos peleando
entre sí. Entonces le dijo al culpable: ``¿Por qué golpeas a uno de los
tuyos?''

\bibleverse{14} ``¿Quién te ha encargado como juez sobre nosotros?'' ,
respondió el hombre. ``¿Acaso vas a matarme como lo hiciste con el
egipcio?'' Entonces Moisés se asustó por esto y se dijo a sí mismo:
``¡La gente sabe lo que he hecho!'' \footnote{\textbf{2:14} Hech
  7,27-28; Hech 7,35}

\hypertarget{moisuxe9s-huyuxf3-a-madiuxe1n-y-se-casa-con-suxe9phora}{%
\subsection{Moisés huyó a Madián y se casa con
Séphora}\label{moisuxe9s-huyuxf3-a-madiuxe1n-y-se-casa-con-suxe9phora}}

\bibleverse{15} Cuando el Faraón se enteró, trató de mandar a matar a
Moisés, pero Moisés huyó del Faraón y se fue a vivir a Madián. Un día,
mientras estaba sentado junto a un pozo, \footnote{\textbf{2:15} Heb
  11,27}

\bibleverse{16} las siete hijas del sacerdote de Madiánvinieron a buscar
agua para llenar los bebederos a fin de que el rebaño de su padre
pudiera beber. \footnote{\textbf{2:16} Éxod 3,1} \bibleverse{17}
Entonces llegaron unos pastores y las echaron de allí, pero Moisés
intervino y las puso a salvo, y le dio de beber a su rebaño. \footnote{\textbf{2:17}
  Gén 29,10} \bibleverse{18} Cuando llegaron a casa, su padre Reuel les
preguntó: ``¿Cómo es que hoy han regresado tan rápido?''

\bibleverse{19} ``Un egipcio nos rescató de unos pastores que nos
atacaron'', respondieron. ``Incluso nos trajo agua para que el rebaño
pudiera beber''.

\bibleverse{20} ``¿Y dónde está?'' le preguntó Reuel a su hija. ``No lo
dejaste allí, ¿verdad? ¡Ve e invítalo a comer con nosotros!''

\bibleverse{21} Y Moisés aceptó quedarse con el hombre, quien arregló
que su hija Séfora se casara con Moisés. \bibleverse{22} Ella tuvo un
hijo, y Moisés le puso el nombre de Gersón,\footnote{\textbf{2:22}
  ``Gersón'' suena como ``Allí hay un extranjero''.} porque dijo: ``Soy
un exiliado que vive en un país extranjero''. \footnote{\textbf{2:22}
  Éxod 18,3}

\hypertarget{dios-escucha-las-aflicciones-de-los-israelitas-oprimidos}{%
\subsection{Dios escucha las aflicciones de los israelitas
oprimidos}\label{dios-escucha-las-aflicciones-de-los-israelitas-oprimidos}}

\bibleverse{23} Años más tarde, el rey de Egipto murió. Pero los
israelitas seguían gimiendo por su duro trabajo. Su clamor pidiendo
ayuda en medio de sus dificultades llegó hasta Dios. \footnote{\textbf{2:23}
  Éxod 3,7} \bibleverse{24} Dios escuchó sus gemidos y se acordó de su
pacto con Abraham, Isaac, y Jacob. \footnote{\textbf{2:24} Gén 15,18;
  Gén 26,3; Gén 28,13-14}

\bibleverse{25} Además Dios mirabacon compasión a los israelitas, y se
preocupaba por ellos.\footnote{\textbf{2:25} ``Se preocupaba por
  ellos'': literalmente, ``sabía''.}

\hypertarget{dios-se-revela-a-moisuxe9s-en-la-zarza-como-el-yo-soy}{%
\subsection{Dios se revela a Moisés en la zarza como el ``Yo
soy''}\label{dios-se-revela-a-moisuxe9s-en-la-zarza-como-el-yo-soy}}

\hypertarget{section-2}{%
\section{3}\label{section-2}}

\bibleverse{1} Moisés era un pastor que cuidaba el rebaño de
Jetro,\footnote{\textbf{3:1} ``Jetro'': Otro nombre de Reuel.} su
suegro, el sacerdote de Madián. Condujo el rebaño lejos en el desierto
hasta que llegó al monte de Dios, el monte Horeb.\footnote{\textbf{3:1}
  ``Monte Horeb'': Otro nombre para el Monte Sinaí.} \bibleverse{2} Allí
el ángel del Señor se le apareció en una llama de fuego desde dentro de
un arbusto. Moisés miró con atención y vio que, aunque la zarza estaba
ardiendo, no se estaba quemando. \bibleverse{3} ``Iré a echar un
vistazo'', se dijo a sí mismo Moisés. ``Es muy extraño ver un arbusto
que no se queme''.

\bibleverse{4} Cuando el Señor vio que Moisés venía a echar un vistazo,
Dios le llamó desde dentro de la zarza: ``¡Moisés! ¡Moisés!'' ``Aquí
estoy'', respondió Moisés.

\bibleverse{5} ``¡No te acerques más!'' le dijo Dios. ``Quítate las
sandalias porque estás parado en tierra sagrada''. \footnote{\textbf{3:5}
  Jos 5,15; Gén 28,17} \bibleverse{6} Luego dijo: ``Soy el Dios de tu
padre, el Dios de Abraham, el Dios de Isaac y el Dios de Jacob''. Moisés
se cubrió el rostro, porque tuvo miedo de mirar a Dios. \footnote{\textbf{3:6}
  Mat 22,32}

\bibleverse{7} ``Soyplenamente consciente de la miseria de mi pueblo en
Egipto'', le dijo el Señor. ``Los he escuchado gemir por culpa de sus
capataces. Sé cuánto están sufriendo. \footnote{\textbf{3:7} Éxod 2,23}
\bibleverse{8} Por eso he descendido para rescatarlos de la opresión
egipcia y para llevarlos desde ese país a una tierra fértil y amplia,
una tierra donde fluye leche y miel, donde actualmente viven los
cananeos, los hititas, los amorreos, los ferezeos, los heveos y los
jebuseos. \bibleverse{9} Escucha ahora: El clamor de los israelitas ha
llegado hasta mí, y he visto cómo los egipcios los maltratan.
\bibleverse{10} Ahora debes irte, porque yo te envío donde el Faraón
para que saques a mi pueblo Israel de Egipto''.

\bibleverse{11} Pero Moisés le dijo a Dios: ``¿Por qué yo? ¡Yo soy un
don nadie! ¡No podré ir ante el Faraón y sacar a los israelitas de
Egipto!''

\bibleverse{12} ``Yo estaré contigo'', respondió el Señor, ``y esta será
la señal de que soy yo quien te envía: cuando hayas sacado al pueblo de
Egipto, adorarás a Dios en este mismo monte''.

\hypertarget{la-revelacion-del-nombre-de-dios}{%
\subsection{La revelacion del nombre de
Dios}\label{la-revelacion-del-nombre-de-dios}}

\bibleverse{13} Entonces Moisés dijo a Dios: ``Mira, si yo fuera donde
los israelitas y les dijera: `El Dios de sus padres me ha enviado a
ustedes', y ellos me preguntaran: `¿Cómo se llama?', ¿qué les diré
entonces?'' .

\bibleverse{14} Dios le respondió a Moisés: ``\,`Yo soy' el que soy.
Dile esto a los israelitas: `Yo soy' me ha enviado a ustedes''.
\footnote{\textbf{3:14} Apoc 1,4; Apoc 1,8} \bibleverse{15} Entonces
Dios le dijo a Moisés: ``Diles a los israelitas: `El Señor, el Dios de
sus padres, el Dios de Abraham, el Dios de Isaac y el Dios de Jacob, me
ha enviado a ustedes. Este es mi nombre para siempre, el nombre con el
que me llamarás en todas las generaciones futuras'. \footnote{\textbf{3:15}
  Éxod 6,2-3; Is 42,8}

\hypertarget{el-llamado-de-dios-y-su-promesa-por-moisuxe9s}{%
\subsection{El llamado de Dios y su promesa por
Moisés}\label{el-llamado-de-dios-y-su-promesa-por-moisuxe9s}}

\bibleverse{16} ``Ve y llama a todos los ancianos de Israel para que se
reúnan contigo. Diles: `El Señor, el Dios de tus padres, se me ha
aparecido, el Dios de Abraham, Isaac y Jacob. Él dijo: He prestado mucha
atención a lo que te ha pasado en Egipto. \bibleverse{17} He decidido
sacarlos de la miseria que están teniendo en Egipto y llevarlos a la
tierra de los cananeos, hititas, amorreos, ferezeos, heveos y jebuseos,
una tierra que fluye leche y miel'\,''. \bibleverse{18} ``Los ancianos
de Israel aceptarán lo que tú digas. Entonces debes ir con ellos al rey
de Egipto y decirle: `El Señor, el Dios de los hebreos se nos ha
revelado. Así que, por favor, hagamos un viaje de tres días al desierto
para poder ofrecer sacrificios al Señor nuestro Dios allí'. \footnote{\textbf{3:18}
  Éxod 5,1; Éxod 5,3}

\bibleverse{19} Pero sé que el rey de Egipto no te dejará ir a menos que
se vea obligado a hacerlo por un poder más fuerte que él.\footnote{\textbf{3:19}
  ``Un poder más fuerte que él'': literalmente, ``una mano poderosa''.}
\bibleverse{20} Así que usaré mi poder para infligir a Egipto todas las
cosas aterradoras que estoy a punto de hacerles. Después de eso los
dejará ir. \bibleverse{21} Haré que los egipcios los traten bien como
pueblo, para que cuando se vayan, no se vayan con las manos vacías.
\bibleverse{22} Toda mujer pedirá a su vecina, así como a cualquier
mujer que viva en su casa, joyas y ropa de plata y oro, y se las pondrá
a sus hijos e hijas. De esta manera se llevarán la riqueza de los
egipcios con ustedes''.

\hypertarget{los-milagros-de-la-autenticaciuxf3n}{%
\subsection{Los milagros de la
autenticación}\label{los-milagros-de-la-autenticaciuxf3n}}

\hypertarget{section-3}{%
\section{4}\label{section-3}}

\bibleverse{1} ``Pero, ¿qué pasa si no me creen, o no escuchan lo que
digo?'' Preguntó Moisés. ``Podrían decir: `El Señor no se te
apareció'\,''.

\bibleverse{2} El Señor le preguntó: ``¿Qué tienes en la mano?'' ``Un
bastón'', respondió Moisés.

\bibleverse{3} ``Tíralo al suelo'', le dijo a Moisés. Así lo hizo
Moisés. Se transformó en una serpiente y Moisés huía de ella.
\footnote{\textbf{4:3} Éxod 7,10}

\bibleverse{4} ``Ahora extiende la mano y agárrala por la cola'', le
dijo el Señor a Moisés. Moisés lo hizo y se convirtió en un bastón en su
mano.

\bibleverse{5} ``Debes hacer esto para que crean que yo, el Señor, me
aparecí delante de ti. E Dios de sus padres, el Dios de Abraham, Isaac y
Jacob''. \bibleverse{6} El Señor le dijo: ``Pon tu mano dentro de tus
ropas cerca de tu pecho''. Así que Moisés hizo lo que se le dijo. Cuando
sacó su mano, estaba blanca como la nieve, con una enfermedad de la
piel.

\bibleverse{7} ``Vuelve a meter la mano dentro de tu ropa'', dijo el
Señor. Y Moisés lo hizo. Cuando la sacó de nuevo, su mano había vuelto a
la normalidad.\footnote{\textbf{4:7} ``A la normalidad'': literalmente,
  ``como su carne''.}

\bibleverse{8} ``Si no te creen y no les convence la primera señal,
creerán por la segunda señal'', explicó el Señor. \bibleverse{9} ``Pero
si todavía no te creen o no te escuchan debido a estos dos signos,
entonces debes tomar un poco de agua del Nilo y ponerla en el suelo. El
agua del Nilo se convertirá en sangre en el suelo''.

\hypertarget{nuevas-objeciones-de-moisuxe9s-nombramiento-de-aaruxf3n-como-orador}{%
\subsection{Nuevas objeciones de Moisés; Nombramiento de Aarón como
orador}\label{nuevas-objeciones-de-moisuxe9s-nombramiento-de-aaruxf3n-como-orador}}

\bibleverse{10} Entonces Moisés dijo al Señor: ``Discúlpame, pero no soy
bueno con las palabras, ni lo he sido en el pasado, ni desde que
comenzaste a hablar conmigo, tu siervo. Soy de hablar lento y no digo
las cosas bien''.\footnote{\textbf{4:10} ``Soy de hablar lento y no digo
  las cosas bien'': literalmente: ``Me pesa la boca y la lengua''.}
\footnote{\textbf{4:10} Éxod 3,11; Éxod 6,12; Éxod 6,30}

\bibleverse{11} ``¿Quién le dio la boca a la gente?'' le preguntó el
Señor. ``¿Quién hace a la gente sorda o muda, capaz de ver o ciega? Soy
yo, el Señor, quien lo hace. \footnote{\textbf{4:11} Sal 94,9}
\bibleverse{12} Ahora ve, y yo mismo seré tu boca, y te diré lo que
debes decir''. \footnote{\textbf{4:12} Mat 10,19}

\bibleverse{13} ``Por favor, Señor, ¡envía a otra persona!'' respondió
Moisés.

\bibleverse{14} El Señor se enojó con Moisés y le dijo: ``Ahí está tu
hermano Aarón, el levita. Sé que habla bien. Viene camino para
encontrarse contigo y se alegrará mucho de verte. \bibleverse{15} Habla
con él y dile qué decir. Yo seré tu boca y la suya, y te diré lo que
debes hacer. \bibleverse{16} Aarón hablará en tu nombre al pueblo, como
si fuera tu boca, y tú estarás en el lugar de Dios para él.
\bibleverse{17} Asegúrate de llevar tu bastón contigo para que puedas
usarlo para hacer la señales''.

\hypertarget{moisuxe9s-despidiuxe9ndose-de-su-suegro-jetro-instrucciuxf3n-de-dios}{%
\subsection{Moisés despidiéndose de su suegro Jetro; Instrucción de
Dios}\label{moisuxe9s-despidiuxe9ndose-de-su-suegro-jetro-instrucciuxf3n-de-dios}}

\bibleverse{18} Entonces Moisés regresó donde Jetro su suegro y le dijo:
``Por favor, permíteme volver con mi propio pueblo en Egipto para ver si
alguno de ellos sigue vivo''. ``Ve con mi bendición'', respondió Jetro.
\footnote{\textbf{4:18} Éxod 3,1}

\bibleverse{19} Mientras Moisés estaba en Madián, el Señor le dijo:
``Vuelve a Egipto porque todos los que querían matarte han muerto''.
\footnote{\textbf{4:19} Mat 2,20}

\bibleverse{20} Moisés puso a su esposa e hijos sobre un asno y regresó
a Egipto, llevando el bastón que Dios había usado para hacer
milagros.\footnote{\textbf{4:20} ``El bastón que Dios había usado para
  hacer milagros:'' literalmente, ``el bastón de Dios''. Esta
  interpretación se refiere a los milagros que se registran en los
  versículos 3 y 4.} \footnote{\textbf{4:20} Éxod 18,3-4}
\bibleverse{21} El Señor le dijo a Moisés: ``Cuando regreses a Egipto,
asegúrate de ir al Faraón y realizar los milagros que te he enseñado
para que los hagas. Lo volveré terco\footnote{\textbf{4:21} ``Terco'':
  literalmente, ``endureceré su corazón'', traducido de manera similar a
  lo largo del libro. La misma experiencia se describe como acto de
  Dios, también como una acción del propio Faraón, y también en voz
  pasiva sin un agente identificado.} y no dejará ir al pueblo.
\footnote{\textbf{4:21} Éxod 7,3; Éxod 7,13; Éxod 8,11; Éxod 8,15; Éxod
  8,28; Éxod 9,12; Éxod 9,23; Éxod 10,1; Éxod 10,20; Éxod 10,27; Éxod
  11,10; Éxod 14,4; Éxod 14,17} \bibleverse{22} Pero esto es lo que
debes decirle al Faraón: `Esto es lo que dice el Señor: Israel es mi
hijo primogénito. \footnote{\textbf{4:22} Jer 31,9; Os 11,1}
\bibleverse{23} Te ordené que dejaras ir a mi hijo para que pueda
adorarme. Pero te negaste a liberarlo, así que ahora mataré a tu hijo
primogénito'\,''. \footnote{\textbf{4:23} Éxod 11,5; Éxod 12,29}

\hypertarget{dios-quiso-matar-a-moisuxe9s-en-la-noche}{%
\subsection{Dios quiso matar a Moisés en la
noche}\label{dios-quiso-matar-a-moisuxe9s-en-la-noche}}

\bibleverse{24} Pero mientras iban de camino, el Señor llegó al lugar
donde se encontraban, queriendo matar a Moisés. \footnote{\textbf{4:24}
  Gén 17,14} \bibleverse{25} Sin embargo, Séfora usó un cuchillo de
pedernal para cortar el prepucio de su hijo. Le tocó los pies con él y
le dijo: ``Para mí eres un marido de sangre''. \footnote{\textbf{4:25}
  Jos 5,2}

\bibleverse{26} (Llamarlo marido de sangre se refiere a la
circuncisión).\footnote{\textbf{4:26} El término utilizado aquí no está
  muy claro. Puede significar algo como: ``A través de esta sangre que
  he derramado, ahora estás emparentado conmigo a través del
  matrimonio''. Algunos intérpretes creen que la palabra significa
  ``alguien que está circuncidado''.} Después de esto el Señor dejó a
Moisés tranquilo.

\hypertarget{moisuxe9s-y-aaruxf3n-encontraron-fe-entre-los-israelitas-en-egipto}{%
\subsection{Moisés y Aarón encontraron fe entre los israelitas en
Egipto}\label{moisuxe9s-y-aaruxf3n-encontraron-fe-entre-los-israelitas-en-egipto}}

\bibleverse{27} El Señor le había dicho a Aarón: ``Ve a encontrarte con
Moisés en el desierto''. Así que Aarón fue y se encontró con Moisés en
el monte de Dios y lo saludó con un beso.

\bibleverse{28} Entonces Moisés le explicó a Aarón todo lo que el Señor
le había mandado a decir, y todos los milagros que le había ordenado
hacer. \bibleverse{29} Moisés y Aarón viajaron a Egipto. Allí reunieron
a todos los ancianos israelitas. \bibleverse{30} Aarón compartió con
ellos todo lo que el Señor le había dicho a Moisés, y Moisés realizó los
milagros para que pudieran verlos. \bibleverse{31} Los israelitas
estaban convencidos. Cuando oyeron que el Señor había venido a ellos, y
que había sido tocado por su sufrimiento, inclinaron sus cabezas y
adoraron.\footnote{\textbf{4:31} Éxod 3,16}

\hypertarget{la-primera-negociaciuxf3n-fallida-con-el-farauxf3n}{%
\subsection{La primera negociación fallida con el
faraón}\label{la-primera-negociaciuxf3n-fallida-con-el-farauxf3n}}

\hypertarget{section-4}{%
\section{5}\label{section-4}}

\bibleverse{1} Después de esto, Moisés y Aarón fueron donde el Faraón y
le dijeron: ``Esto es lo que el Señor, el Dios de Israel dice: `Deja ir
a mi pueblo para que me haga una fiesta religiosa en el desierto'\,''.
\footnote{\textbf{5:1} Éxod 3,18; Éxod 7,16; Éxod 7,26; Éxod 8,16; Éxod
  9,1; Éxod 9,13}

\bibleverse{2} ``¿Quién es este `Señor' para que yo escuche su petición
de dejar ir a Israel?'' respondió El Faraón. ``¡No conozco al Señor y
ciertamente no dejaré que Israel se vaya!'' \footnote{\textbf{5:2} Dan
  3,15}

\bibleverse{3} ``El Dios de los hebreos vino a nosotros'', añadieron.
``Por favor, permítenos hacer un viaje de tres días al desierto y
ofrecer sacrificios al Señor nuestro Dios. De lo contrario nos matará
por enfermedad o por espada''.

\bibleverse{4} ``Moisés y Aarón, ¿por qué intentan distraer al pueblo de
su trabajo?'' Preguntó el Faraón. ``¡Vuelvan a su trabajo!'' ordenó.
\bibleverse{5} ``Mira aquí'', continuó. ``Hay mucha gente tuya aquí en
nuestro país y estás impidiendo que hagan el trabajo que se les ha
asignado''.

\hypertarget{el-pueblo-estuxe1-auxfan-muxe1s-oprimida-los-israelitas-reprochan-amargamente-a-moisuxe9s-y-a-aaruxf3n}{%
\subsection{El pueblo está aún más oprimida; los israelitas reprochan
amargamente a Moisés y a
Aarón}\label{el-pueblo-estuxe1-auxfan-muxe1s-oprimida-los-israelitas-reprochan-amargamente-a-moisuxe9s-y-a-aaruxf3n}}

\bibleverse{6} Ese mismo día ordenó a los capataces y a los encargados
del pueblo: \bibleverse{7} ``No les den más paja para hacer ladrillos
como antes. Que vayan ellos mismos a recoger la paja. \bibleverse{8}
Pero que sigan produciendo la misma cantidad de ladrillos que antes. Son
un pueblo perezoso, por eso gritan, pidiendo: `Por favor, déjanos ir y
ofrecer sacrificios a nuestro dios'. \bibleverse{9} ¡Hagan que su
trabajo sea más duro, para que puedan seguir trabajando y no se
distraigan con todas estas mentiras!''

\bibleverse{10} Así que los capataces salieron y le dijeron al pueblo de
Israel: ``Esto es lo que el Faraón ha ordenado: `No lesproveeré más
paja. \bibleverse{11} Vayan y recojan la paja donde puedan encontrarla,
porque su cuota de trabajo no se reducirá'\,''. \bibleverse{12} Así que
la gente iba por todo Egipto recogiendo rastrojos para la paja.
\bibleverse{13} Loscapataces seguían forzándolos, diciendo: ``¡Todavía
tienen que hacer el mismo trabajo que hacían cuando recibían la paja!''
\bibleverse{14} Golpeaban a los supervisores israelitas que ellos habían
puesto a cargo, gritándoles: ``¿Por qué no han cumplido con su cuota de
ladrillos como lo hicieron antes?''

\bibleverse{15} Los supervisores israelitas fueron a quejarse al Faraón,
diciendo: ``¿Por qué nos tratas así a tus siervos? \bibleverse{16} No
nos das nada de paja, pero tus capataces exigen que hagamos ladrillos,
¡y nos golpean! ¡Tu pueblo nos trata mal!'' \footnote{\textbf{5:16} 1Re
  1,21}

\bibleverse{17} ``¡No, ustedes solo son unos vagos, unos perezosos!''
respondió el Faraón. ``Por eso siguenrogando: `Por favor, déjanos ir y
ofrecer sacrificios al Señor'. \bibleverse{18} ¡Ahora salgan de aquí y
vayan a trabajar! ¡No se les dará paja, pero aún así tendrán que
producir la cuota completa de ladrillos!''

\bibleverse{19} Los supervisores israelitas se dieron cuenta de que
estaban en problemas cuando les dijeron: ``No deben reducir la
producción diaria de ladrillos''.

\bibleverse{20} Se acercaron a Moisés y Aarón que los esperaban después
de su encuentro con el Faraón, \bibleverse{21} y dijeron: ``¡Que el
Señor vea lo que han hecho y los juzgue por ello! Han hecho que el
Faraón y sus oficiales se enojen con nosotros. ¡Han puesto una espada en
sus manos para matarnos!''

\hypertarget{el-lamento-de-moisuxe9s-y-la-promesa-de-dios}{%
\subsection{El lamento de Moisés y la promesa de
Dios}\label{el-lamento-de-moisuxe9s-y-la-promesa-de-dios}}

\bibleverse{22} Moisés volvió donde el Señor y se quejó: ``¿Por qué le
has causado todos estos problemas a tu propio pueblo, Señor? ¿Fue para
esto que me enviaste? \bibleverse{23} ``Desde que fui a ver al Faraón
para hablar en tu nombre, él ha sido aún más duro con tu pueblo, ¡y no
has hecho nada para salvarlo!''

\hypertarget{section-5}{%
\section{6}\label{section-5}}

\bibleverse{1} Pero el Señor le dijo a Moisés: ``Ahora verás lo que le
haré al Faraón. Con mi gran fuerza lo obligaré a dejarlos ir; por mi
poder los enviará fuera de su país''. \footnote{\textbf{6:1} Éxod 11,1;
  Éxod 12,33}

\hypertarget{dios-se-revela-a-suxed-mismo-a-moisuxe9s-de-nuevo-y-promete-la-salvaciuxf3n-del-pueblo}{%
\subsection{Dios se revela a sí mismo a Moisés de nuevo y promete la
salvación del
pueblo}\label{dios-se-revela-a-suxed-mismo-a-moisuxe9s-de-nuevo-y-promete-la-salvaciuxf3n-del-pueblo}}

\bibleverse{2} Dios habló a Moisés y le dijo: ``¡Yo soy
Yahvé!\footnote{\textbf{6:2} ``Yahvé'': Este término suele traducirse
  como ``Señor'', pero dado que se identifica específicamente por su
  nombre, parece apropiado utilizar ``Yahvé'' aquí.} \bibleverse{3} Me
revelé como Dios Todopoderoso a Abraham, a Isaac y a Jacob, pero ellos
no conocían mi nombre, `Yahvé'. \bibleverse{4} También confirmé mi
acuerdo solemne con ellos de darles la tierra de Canaán, el país donde
vivían como extranjeros. \footnote{\textbf{6:4} Gén 12,7} \bibleverse{5}
Además he escuchado los gemidos de los israelitas que los egipcios
tratan como esclavos, y no he olvidado el acuerdo que les prometí.
\bibleverse{6} ``Así que di a los israelitas: `Yo soy el Señor y los
salvaré del trabajo forzoso que les imponen los egipcios; yo los
liberaré su esclavitud. Los rescataré usando mi poder e imponiendo
fuertes castigos. \bibleverse{7} Yo los convertiré en mi propio pueblo.
Entonces sabrán que soy el Señor su Dios, que los rescató de la
esclavitud en Egipto. \bibleverse{8} Los llevaré a la tierra que prometí
solemnemente darles a Abraham, Isaac y Jacob. Se las daré y será de
ellos. ¡Yo soy el Señor!'\,''

\hypertarget{moisuxe9s-rechazado-por-su-pueblo-desesperado-recibe-nuevas-instrucciones-de-dios}{%
\subsection{Moisés, rechazado por su pueblo desesperado, recibe nuevas
instrucciones de
Dios}\label{moisuxe9s-rechazado-por-su-pueblo-desesperado-recibe-nuevas-instrucciones-de-dios}}

\bibleverse{9} Moisés le explicó esto a los israelitas, pero ellos no lo
escucharon, porque estaban muy desanimados, y por el duro trabajo que se
veían obligados a hacer.

\bibleverse{10} El Señor le dijo a Moisés: \bibleverse{11} ``Ve y habla
con el Faraón, rey de Egipto. Dile que deje a los israelitas salir de su
país''.

\bibleverse{12} Pero Moisés respondió: ``Ni siquiera mi propio pueblo me
escucha. ¿Por qué me escucharía el Faraón, sobre todo si soy tan mal
orador?'' \footnote{\textbf{6:12} Éxod 6,30; Éxod 4,10} \bibleverse{13}
Pero el Señor les habló a Moisés y a Aarón, y les dijo lo que debían
hacer con respecto al pueblo de Israel y con Faraón, rey de Egipto, para
sacar a los israelitas de Egipto.

\hypertarget{uxe1rbol-genealuxf3gico-de-aarons-y-moisuxe9s}{%
\subsection{Árbol genealógico de Aarons y
Moisés}\label{uxe1rbol-genealuxf3gico-de-aarons-y-moisuxe9s}}

\bibleverse{14} Estos eran los jefes de la familia de Israel: Los hijos
de Rubén, el primogénito de Israel, eran Hanok y Pallu, Hezrón y Karmi.
Estas fueron las familias de Rubén. \bibleverse{15} Los hijos de Simeón
fueron Jemuel, Jamín, Oad, Joaquín, Zojar y Saúl (hijo de una mujer
cananea). Estas fueron las familias de Simeón. \bibleverse{16}
Estosfueron los nombres de los hijos de Leví según sus registros
genealógicos: Gersón, Coaty Merari. Leví vivió durante 137 años.
\footnote{\textbf{6:16} 1Cró 5,27-30; 1Cró 6,1-4} \bibleverse{17} Los
hijos de Gersón, por familias, fueron Libní y Simí. \bibleverse{18} Los
hijos de Coat fueron Amram, Izar, Hebrón y Uziel. Coat vivió durante 133
años. \bibleverse{19} Los hijos de Merari fueron Majli y Mushi. Estas
eran las familias de los levitas según sus registros genealógicos.
\bibleverse{20} Amram se casó con la hermana de su padre, Jocabed, y
ella tuvo sus hijos Aarón y Moisés. Amram vivió durante 137 años.
\bibleverse{21} Los hijos de Izar fueron Coré, Neferón y Zicrí.
\footnote{\textbf{6:21} Núm 16,1} \bibleverse{22} Los hijos de Uziel
fueron Misael, Elzafán y Sitri. \footnote{\textbf{6:22} Lev 10,4}
\bibleverse{23} Aarón se casó con Elisabet, hija de Aminadab y hermana
de Nasón. Ella tuvo sus hijos Nadab y Abiú, Eleazar e Itamar.
\footnote{\textbf{6:23} Éxod 28,1} \bibleverse{24} Los hijos de Coré
fueron Asir, Elcana y Abiasaf. Estas eran las familias de Coré.
\footnote{\textbf{6:24} Núm 25,7} \bibleverse{25} Eleazar, hijo de
Aarón, se casó con una de las hijas de Futiel, y tuvo su hijo Finés.
Estos son los ancestros de las familias levitas, listados según sus
clanes. Eleazar hijo de Aarón se casó con una de las hijas de Futiel, y
ella dio a luz a su hijo, Finés. Estos son los jefes de las familias
levitas, listados por familia. \bibleverse{26} Aarón y Moisés
mencionados aquí son los que el Señor dijo, ``Saquen a los israelitas de
Egipto, divididos en sus respectivas tribus''. \bibleverse{27} Moisés y
Aarón también fueron los que fueron a hablar con el Faraón, rey de
Egipto, sobre la salida de los israelitas de Egipto.

\hypertarget{la-nueva-misiuxf3n-de-moisuxe9s-y-aaruxf3n-ante-el-farauxf3n}{%
\subsection{La nueva misión de Moisés y Aarón ante el
faraón}\label{la-nueva-misiuxf3n-de-moisuxe9s-y-aaruxf3n-ante-el-farauxf3n}}

\bibleverse{28} Cuando el Señor habló a Moisés en Egipto,
\bibleverse{29} le dijo: ``Yo soy el Señor. Dile al Faraón, rey de
Egipto, todo lo que te digo''

\bibleverse{30} Pero Moisés respondió: ``No soy un buen orador, ¿por qué
me escucharía el Faraón?''\footnote{\textbf{6:30} Éxod 6,12}

\hypertarget{section-6}{%
\section{7}\label{section-6}}

\bibleverse{1} Entonces el Señor le dijo a Moisés: ``Mira, te haré
parecer como Dios ante el Faraón, y tu hermano Aarón será tu profeta.
\footnote{\textbf{7:1} Éxod 4,16} \bibleverse{2} Debes repetir todo lo
que te digo, y tu hermano Aarón debe repetirlo al Faraón para que deje
salir a los israelitas de su país. \bibleverse{3} Pero le daré al Faraón
una actitud terca, y aunque haré muchas señales y milagros en Egipto, no
te escuchará. \footnote{\textbf{7:3} Éxod 4,21} \bibleverse{4}
Entoncesatacaré\footnote{\textbf{7:4} Literalmente, ``pondré mi mano
  sobre''.} a Egipto, imponiéndoles fuertes castigos, y sacaré a mi
pueblo, los israelitas, tribu por tribu. \bibleverse{5} De esta manera
los egipcios sabrán que yo soy el Señor cuando actúe contra Egipto y
saque a los israelitas del país''.

\bibleverse{6} Moisés y Aarón hicieron exactamente lo que el Señor había
ordenado. \bibleverse{7} Moisés tenía ochenta y Aarón ochenta y tres
años cuando fueron a hablar con el Faraón.

\hypertarget{transformaciuxf3n-del-bastuxf3n-en-serpiente}{%
\subsection{Transformación del bastón en
serpiente}\label{transformaciuxf3n-del-bastuxf3n-en-serpiente}}

\bibleverse{8} El Señor les dijo a Moisés y a Aarón: \bibleverse{9}
``Cuando el Faraón te pregunte: `¿Por qué no haces un milagro,
entonces?' dile a Aarón: `Toma tu bastón y tíralo delante del Faraón', y
se convertirá en una serpiente''. \footnote{\textbf{7:9} Éxod 4,3}

\bibleverse{10} Moisés y Aarón fueron a ver al Faraón e hicieron lo que
el Señor había ordenado. Aarón arrojó su bastón delante del Faraón y sus
oficiales, y se convirtió en una serpiente. \bibleverse{11} Pero el
Faraón llamó a sabios y hechiceros, y estos magos egipcios hicieron lo
mismo usando sus artes mágicas. \bibleverse{12} Cada uno de ellos arrojó
su bastón y también se convirtieron en serpientes, pero el bastón de
Aarón se tragó todos sus bastones. \bibleverse{13} Sin embargo, el
Faraón tenía una actitud dura y terca, y no los escuchaba, como el Señor
había predicho. \footnote{\textbf{7:13} Éxod 4,21}

\hypertarget{la-primer-plaga-transformaciuxf3n-de-las-aguas-del-nilo-en-sangre}{%
\subsection{La primer plaga: Transformación de las aguas del Nilo en
sangre}\label{la-primer-plaga-transformaciuxf3n-de-las-aguas-del-nilo-en-sangre}}

\bibleverse{14} El Señor le dijo a Moisés: ``Faraón tiene una actitud
obstinada, se niega a dejar ir al pueblo. \bibleverse{15} Así que mañana
por la mañana ve a Faraón mientras camina hacia el río. Espera para
encontrarte con él en la orilla del Nilo. Lleva contigo el bastón que se
convirtió en una serpiente. \bibleverse{16} Dile: El Señor, el Dios de
los hebreos, me ha enviado a decirte: `Deja ir a mi pueblo para que me
adoren en el desierto. Pero no me has escuchado hasta ahora.
\bibleverse{17} Esto es lo que el Señor te dice ahora: Así es como
sabrán que yo soy el Señor'\,''. ``¡Miren! Con el bastón que tengo en la
mano, voy a golpear el agua del Nilo, y se convertirá en sangre.
\footnote{\textbf{7:17} Éxod 4,9} \bibleverse{18} Los peces del Nilo
morirán, el río tendrá mal olor, y los egipcios no podrán beber nada de
su agua''. \bibleverse{19} El Señor le dijo a Moisés: ``Dile a Aarón:
`Toma tu bastón en tu mano y sostenlo sobre las aguas de Egipto, sobre
sus ríos y canales y estanques y albercas, para que se conviertan en
sangre. Habrá sangre por todo Egipto, incluso en los recipientes de
madera y piedra'\,''.

\bibleverse{20} Y Moisés y Aarón hicieron exactamente lo que el Señor
les dijo. Mientras el Faraón y todos sus oficiales miraban, Aarón
levantó su bastón y golpeó el agua del Nilo. ¡Inmediatamente todo el río
se convirtió en sangre! \bibleverse{21} Los peces del Nilo murieron, y
el río olía tan mal que los egipcios no podían beber su agua. ¡Había
sangre por todo Egipto! \bibleverse{22} Pero los magos egipcios hicieron
lo mismo usando sus artes mágicas. El Faraón mantuvo su actitud terca y
no quiso escuchar a Moisés y Aarón, tal como el Señor había predicho.
\footnote{\textbf{7:22} Éxod 7,11}

\bibleverse{23} Entonces el Faraón volvió a su palacio y no prestó
atención a lo que había sucedido \bibleverse{24} Todos los egipcios
cavaron a lo largo del Nilo porque no podían beber su agua.
\bibleverse{25} Siete días pasaron después de que el Señor llegara al
Nilo.

\hypertarget{la-segunda-plaga-de-las-ranas}{%
\subsection{La segunda plaga de las
ranas}\label{la-segunda-plaga-de-las-ranas}}

\hypertarget{section-7}{%
\section{8}\label{section-7}}

\bibleverse{1} El Señor le dijo a Moisés: ``Ve a ver al Faraón y dile:
`Esto es lo que dice el Señor: Deja ir a mi pueblo, para que me adoren.
\bibleverse{2} Si te niegas a dejarlos ir, enviaré una plaga de ranas
por todo tu país. \bibleverse{3} Saldrán en enjambre del Nilo, y
entrarán en tu palacio y se meterán en tu dormitorio y saltarán a tu
cama. Entrarán en las casas de tus funcionarios y saltarán alrededor de
tu gente, incluso en tus hornos y tazones de pan. \footnote{\textbf{8:3}
  Éxod 7,11} \bibleverse{4} Ranas saltarán sobre ti, tu pueblo y todos
tus oficiales'\,''. \footnote{\textbf{8:4} Éxod 8,24; Éxod 9,28; Éxod
  10,17} \bibleverse{5} El Señor le dijo a Moisés: ``Dile a Aarón:
`Extiende tu bastón en tu mano sobre los ríos, canales y estanques, y
haz que las ranas se extiendan por todo Egipto'\,''. \bibleverse{6}
Aarón extendió su mano sobre las aguas de Egipto, y las ranas subieron y
cubrieron la tierra. \footnote{\textbf{8:6} Éxod 9,14; Éxod 15,11}
\bibleverse{7} Pero los magos egipcios hicieron lo mismo usando sus
artes mágicas. Criaron ranas en Egipto.

\bibleverse{8} El Faraón llamó a Moisés y a Aarón y les suplicó: ``Oren
al Señor y pídanle que me quite las ranas a mí y a mi pueblo. Entonces
dejaré ir a tu pueblo para que pueda ofrecer sacrificios al Señor''.

\bibleverse{9} ``Ustedes tendrán el honor de decidir\footnote{\textbf{8:9}
  ``Ustedes tendrán el honor de decidir'': Literalmente, ``Glorifícate a
  ti mismo sobre mí''.} cuándo oraré por ustedes, sus funcionarios y su
pueblo para que os quiten las ranas a ustedes y a sus casas.
Permanecerán sólo en el Nilo''.

\bibleverse{10} ``Hazlo mañana'', respondió el Faraón. Moisés dijo:
``Sucederá como has pedido para que sepas que no hay nadie como el Señor
nuestro Dios.

\bibleverse{11} Las ranas los dejarán y abandonarán sus casas, las casas
de tus funcionarios y detodo tu pueblo, y sólo permanecerán en el
Nilo''.

\bibleverse{12} Moisés y Aarón dejaron al Faraón, y Moisés le suplicó al
Señor por las ranas que había enviado contra el Faraón. \bibleverse{13}
El Señor hizo lo que Moisés le pidió. Las ranas de las casas, los patios
y los campos murieron. \bibleverse{14} El pueblo las recogió montón tras
montón, y todo el país olía fatal. \footnote{\textbf{8:14} Éxod 9,11}
\bibleverse{15} Pero cuando el Faraón se dio cuenta de que la plaga
había pasado, decidió volver a ser duro y terco, y no quiso escuchar a
Moisés y Aarón, tal como el Señor había predicho. \footnote{\textbf{8:15}
  Éxod 14,25; Éxod 4,21}

\hypertarget{la-tercera-plaga-mosquitos}{%
\subsection{La tercera plaga:
Mosquitos}\label{la-tercera-plaga-mosquitos}}

\bibleverse{16} Entonces el Señor le dijo a Moisés: ``Dile a Aarón:
`Recoge tu bastón y golpea el polvo del suelo. El polvo se convertirá en
un enjambrede mosquitos\footnote{\textbf{8:16} El nombre exacto del
  insecto que se menciona aquí no se conoce con certeza. El hebreo
  sugiere ``insecto molesto'', y ha sido traducido como piojos,
  mosquitos o pulgas además de mosquitos. Sin embargo, alguna forma de
  pequeño insecto volador que muerde como un mosquito encajaría mejor en
  el contexto de ``polvo''.} por todo Egipto'\,''. \footnote{\textbf{8:16}
  Éxod 5,1} \bibleverse{17} Así que hicieron lo que el Señor dijo.
Cuando Aarón levantó su bastón y golpeó el polvo de la tierra, los
mosquitos pululaban sobre las personas y los animales. El polvo de todo
Egipto se convirtió en mosquitos. \bibleverse{18} Los magos también
trataron de hacer mosquitos usando sus artes mágicas, pero no pudieron.
Los mosquitos se mantuvieron tanto sobre las personas como sobre los
animales. \bibleverse{19} ``Este es un acto de Dios'', le dijeron los
magos al Faraón. Pero el Faraón eligió ser obstinado y duro de corazón,
y no quiso escuchar a Moisés y Aarón, como el Señor había predicho.

\hypertarget{cuarto-plaga-moscas-del-perro}{%
\subsection{Cuarto plaga: moscas del
perro}\label{cuarto-plaga-moscas-del-perro}}

\bibleverse{20} El Señor le dijo a Moisés: ``Mañana por la mañana
levántate temprano y bloquea el camino del Faraón mientras baja al río.
Dile: `Esto es lo que dice el Señor: Deja ir a mi pueblo, para que me
adoren. \bibleverse{21} Si no dejas que mi pueblo se vaya, enviaré
enjambres de moscas sobre ti y tus funcionarios, y sobre tu pueblo y tus
casas. Todas las casas egipcias, e incluso el suelo sobre el que se
levantan, se llenarán de enjambres de moscas. \bibleverse{22} Sin
embargo, en esta ocasión trataré a la tierra de Gosén de manera
diferente, que es donde vive mi pueblo, y no habrá allí ningún enjambre
de moscas. Así es como sabrán que yo, el Señor, estoy aquí en su país.
\footnote{\textbf{8:22} Gén 43,32} \bibleverse{23}
Distinguiré\footnote{\textbf{8:23} El hebreo menciona aquí
  ``redención'', pero parece ser una errata. En este versículo hemos
  usado el término que propone la Septuaginta.} a mi pueblo de su
pueblo. Verás esta señal que lo confirma mañana'\,''. \footnote{\textbf{8:23}
  Éxod 3,18} \bibleverse{24} Y el Señor hizo lo que había dicho. Enormes
enjambres de moscas entraron en el palacio del Faraón y en las casas de
sus oficiales. Todo Egipto fue devastado por estos enjambres de moscas.
\footnote{\textbf{8:24} Éxod 8,4}

\bibleverse{25} El Faraón llamó a Moisés y a Aarón y les dijo: ``Vayan y
ofrezcan sacrificios a su Dios aquí dentro de este país''.

\bibleverse{26} ``No, eso no sería lo correcto'', respondió Moisés.
``Los sacrificios que ofrecemos al Señor nuestro Dios serían ofensivos
para los egipcios. ¡Si nos adelantáramos y ofreciéramos sacrificios
ofensivos a los egipcios, nos apedrearían! \bibleverse{27} Debemos hacer
un viaje de tres días al desierto y ofrecer allí los sacrificios al
Señor nuestro Dios como nos ha dicho''.

\bibleverse{28} ``Los dejaré ir para que ofrezcan sacrificios al Señor
su Dios en el desierto, pero no vayan muy lejos'', respondió el Faraón.
``Ahora ora por mí para que esta plaga termine''.\footnote{\textbf{8:28}
  ``Que esta plaga termine''. Idea implícita.}

\bibleverse{29} ``Tan pronto como te deje, oraré al Señor'', respondió
Moisés, ``y mañana las moscas dejarán al Faraón y a sus oficiales y a su
pueblo. Pero el Faraón debe asegurarse de no volver a ser mentiroso,
negándose después a dejar que el pueblo vaya a ofrecerle sacrificios al
Señor''. \bibleverse{30} Moisés dejó al Faraón y oró al Señor,
\bibleverse{31} y el Señor hizo lo que Moisés le pidió, y quitó los
enjambres de moscas del Faraón y sus funcionarios y su pueblo. No quedó
ni una sola mosca. \bibleverse{32} Pero una vez más el Faraón eligió ser
obstinado y duro de corazón y no dejó que el pueblo se fuera.

\hypertarget{quinta-plaga-plaga-del-ganado}{%
\subsection{Quinta plaga: plaga del
ganado}\label{quinta-plaga-plaga-del-ganado}}

\hypertarget{section-8}{%
\section{9}\label{section-8}}

\bibleverse{1} El Señor le dijo a Moisés: ``Ve y habla con el Faraón.
Dile: `Esto es lo que dice el Señor: Deja ir a mi pueblo, para que me
adoren. \footnote{\textbf{9:1} Éxod 5,1} \bibleverse{2} Si te niegas a
dejarlos ir y sigues deteniéndolos, \bibleverse{3} te castigaré trayendo
una plaga muy severa sobre tu ganado: en tus caballos, tus asnos, tus
camellos, así como tus rebaños y manadas. \bibleverse{4} Pero el Señor
distinguirá entre el ganado de los israelitas y el de los egipcios, de
modo que ninguno de los que pertenecen a los israelitas morirá.
\bibleverse{5} El Señor ha fijado un tiempo, diciendo: Mañana esto es lo
que va a pasar aquí en el país'\,''. \bibleverse{6} Al día siguiente el
Señor hizo lo que había dicho. Todo el ganado de los egipcios murió,
pero no murió ni un solo animal de los israelitas. \bibleverse{7} El
Faraón envió a los oficiales y descubrió que no había muerto ni un solo
animal de los israelitas. Pero el Faraón fue terco y no dejó que el
pueblo se fuera. \footnote{\textbf{9:7} Éxod 4,21}

\hypertarget{la-sexta-plaga-tumores}{%
\subsection{La sexta plaga: tumores}\label{la-sexta-plaga-tumores}}

\bibleverse{8} El Señor les dijo a Moisés y a Aarón: ``Vayan y saquen
unos puñados de hollín de un horno. Luego Moisés deberá arrojarlo al
aire delante del Faraón. \bibleverse{9} Se esparcirá como polvo fino por
todo el país de Egipto, y aparecerán llagas abiertas en la gente y en
los animales de toda la tierra''.

\bibleverse{10} Entonces sacaron hollín de un horno y fueron a ver al
Faraón. Moisés lo arrojó al aire, y se comenzaron a abrir llagas en las
personas y los animales. \footnote{\textbf{9:10} Apoc 16,2}
\bibleverse{11} Los magos no pudieron venir nicomparecer ante Moisés,
porque ellos y todos los demás egipcios estaban cubiertos de llagas.
\bibleverse{12} Pero el Señor puso en el Faraón una actitud obstinada, y
el Faraón no los escuchó, tal como el Señor le había dicho a Moisés.

\hypertarget{la-suxe9ptima-plaga-el-granizo}{%
\subsection{La séptima plaga: el
granizo}\label{la-suxe9ptima-plaga-el-granizo}}

\bibleverse{13} El Señor le dijo a Moisés: ``Mañana por la mañana
levántate temprano y ve al Faraón, y dile que esto es lo que el Señor,
el Dios de los hebreos, dice: `Deja ir a mi pueblo para que me adore'.
\footnote{\textbf{9:13} Éxod 5,1} \bibleverse{14} Esta vez dirigiré
todas mis plagas contra ti y tus funcionarios y tu pueblo, para que te
des cuenta de que no hay nadie como yo en toda la tierra. \footnote{\textbf{9:14}
  Éxod 8,6} \bibleverse{15} A estas alturas ya podría haber extendido mi
mano para atacarte a ti y a tu pueblo con una plaga que te habría
destruido por completo.\footnote{\textbf{9:15} ``Te habría destruído por
  completo'': Literalmente, ``Habrías perecido en la tierra''.}
\bibleverse{16} Sin embargo, te he dejado vivir para que veas mi poder,
y para que mi reputación sea conocida por toda la tierra. \footnote{\textbf{9:16}
  Éxod 7,3; Éxod 14,4; Rom 9,17} \bibleverse{17} Pero en tu orgullo
sigues tiranizando a mi pueblo, y te niegas a dejar que se vaya.
\bibleverse{18} ¡Así que ten cuidado! Mañana a esta hora enviaré la peor
granizada que haya caído sobre Egipto, desde el principio de su historia
hasta ahora. \bibleverse{19} Así que haz guardar a tu ganado y todo lo
que tienes en el campo. Porque toda persona y todo animal que
permanezcan fuera morirán cuando el granizo caiga sobre ellos''.

\bibleverse{20} Aquellos oficiales del Faraón que tomaron en serio lo
que el Señor dijo, se apresuraron a traer a sus sirvientes y a su ganado
adentro. \bibleverse{21} Pero aquellos a los que no les importó lo que
el Señor decía, dejaron a sus sirvientes y ganado afuera.

\bibleverse{22} El Señor le dijo a Moisés: ``Levanta tu mano hacia el
cielo para que caiga una tormenta de granizo sobre todo Egipto, sobre la
gente y sobre los animales, y sobre todo lo que crece en los campos de
Egipto''.

\bibleverse{23} Moisés levantó su bastón hacia el cielo, y el Señor
envió truenos y granizo, e hizo caer rayos al suelo. Así es como el
Señor hizo llover granizo sobre Egipto. \footnote{\textbf{9:23} Apoc
  16,21} \bibleverse{24} Cuando el granizo cayó, vino acompañado de
relámpagos por todas partes. El granizo que cayó fue tan severo como
nunca se había visto en todo Egipto desde los comienzos de su historia.
\bibleverse{25} A lo largo de todo Egipto el granizo golpeó todo en los
campos, tanto a las personas como a los animales. Derribó todo lo que
crecía en los campos, y desnudó todos los árboles. \bibleverse{26} Sólo
en la tierra de Gosén, donde vivían los israelitas, no había granizo.

\bibleverse{27} El Faraón llamó a Moisés y a Aarón y les dijo: ``Admito
que esta vez he pecado. ¡El Señor tiene razón, y yo y mi pueblo estamos
equivocados! \bibleverse{28} Rueguen al Señor por nosotros, porque ya ha
habido más que suficiente de los truenos y granizos de Dios. Dejaré que
se vayan. No necesitan quedarse más tiempo aquí''. \footnote{\textbf{9:28}
  Éxod 8,4}

\bibleverse{29} ``Una vez que haya dejado la ciudad, oraré al Señor por
ti'', le dijo Moisés. ``Los truenos cesarán y no habrá más granizo, para
que te des cuenta de que la tierra pertenece al Señor. \bibleverse{30}
Pero sé que tú y tus funcionarios aún no respetan realmente al Señor
nuestro Dios''.

\bibleverse{31} (El lino y la cebada fueron destruidos, porque la cebada
estaba madura y el lino estaba floreciendo. \bibleverse{32} Sin embargo,
el trigo y la escanda no fueron destruidos porque crecen más tarde).
\bibleverse{33} Moisés dejó al Faraón y salió de la ciudad, y oró al
Señor. Los truenos y el granizo se detuvieron, y la tormenta de lluvia
terminó. \bibleverse{34} Cuando el Faraón vio que la lluvia, el granizo
y los truenos habían cesado, volvió a pecar, y eligió volver a ser
obstinado, junto con sus funcionarios. \bibleverse{35} Debido a su
actitud terca, el Faraón no permitió que los israelitas se fueran, tal
como el Señor había predicho a través de Moisés.

\hypertarget{la-octava-plaga-langostas}{%
\subsection{La octava plaga:
langostas}\label{la-octava-plaga-langostas}}

\hypertarget{section-9}{%
\section{10}\label{section-9}}

\bibleverse{1} El Señor le dijo a Moisés: ``Ve a ver al Faraón, porque
fui yo quien le dio a él y a sus oficiales una actitud obstinada para
que yo pudiera hacer mis milagros ante ellos. \bibleverse{2} Esto es
para que puedas contar a tus hijos y nietos cómo hice que los egipcios
parecieran tontos\footnote{\textbf{10:2} ``Parecieran tontos'': La
  palabra sugiere que el Señor se está burlando de los egipcios, y esto
  sería principalmente por su devoción a ídolos inútiles.} haciendo
estos milagros entre ellos, y para que sepas que yo soy el Señor''.
\footnote{\textbf{10:2} Éxod 6,2-7}

\bibleverse{3} Moisés y Aarón fueron a ver al Faraón y le dijeron:
``Esto es lo que dice el Señor, el Dios de los hebreos: `¿Hasta cuándo
te negarás a humillarte ante mí? Deja ir a mi pueblo, para que me adore.
\footnote{\textbf{10:3} Éxod 5,3} \bibleverse{4} Sino dejas que mi
pueblo se vaya, mañana enviaré una plaga de langostas a tu país.
\bibleverse{5} Habrá tantas que cubrirán el suelo para que nadie pueda
verlas. Comerán los cultivos que haya dejado el granizo, así como todos
los árboles que crezcan en tus campos. \bibleverse{6} Entrarán en
enjambres en tus casas y en las casas de todos tus funcionarios, de
hecho en las casas de todos los egipcios. Esto es algo que ninguno de
tus antepasados ha visto desde que llegaron a este país'\,''. Entonces
Moisés y Aarón se volvieron y dejaron al Faraón.

\bibleverse{7} Los oficiales del Faraón se acercaron a él y le
preguntaron: ``¿Cuánto tiempo vas a dejar que este hombre nos cause
problemas?\footnote{\textbf{10:7} ``Nos cause problemas'': Literalmente,
  ``sea una trampa para nosotros''.} Deja que esta gente se vaya para
que puedan adorar al Señor su Dios. ¿No te das cuenta de que Egipto ha
quedado destruido?''

\bibleverse{8} Moisés y Aarón fueron traídos nuevamente ante el Faraón.
``Vayan y adoren al Señor su Dios'', les dijo. ``Pero ¿quién de ustedes
irá?''

\bibleverse{9} ``Todos iremos'', respondió Moisés. ``Jóvenes y viejos,
hijos e hijas, y llevaremos nuestros rebaños y manadas con nosotros,
porque vamos a celebrar unafiesta para el Señor''. \footnote{\textbf{10:9}
  Éxod 5,1}

\bibleverse{10} ``¡El Señor realmente tendrá que estar con ustedes si
dejo que sus hijos se vayan!'' respondió el Faraón. ``¡Claramente estás
planeando algún tipo de truco maligno! \bibleverse{11} ¡Así que no! Sólo
los hombres pueden ir y adorar al Señor, porque eso es lo que has estado
pidiendo''. Entonces hizo que echaran a Moisés y a Aarón.

\bibleverse{12} El Señor le dijo a Moisés, ``Levanta tu mano sobre
Egipto, para aparezcan las langostas y se coman todas las plantas del
país, todo lo que haya dejado el granizo''. \bibleverse{13} Moisés
extendió su bastón sobre Egipto, y durante todo ese día y noche el Señor
envió un viento del este que soplaba sobre la tierra. Cuando llegó la
mañana, el viento del Este había traído las langostas. \bibleverse{14}
Las langostas pululaban por toda la tierra y se asentaron en cada parte
del país. Nunca había habido tal enjambre de langostas, y no lo habrá
nunca más. \bibleverse{15} Cubrieron el suelo hasta que se vio negro, y
se comieron todas las plantas de los campos y todos los frutos de los
árboles que había dejado el granizo. No quedó ni una sola hoja verde en
ningún árbol o planta en ningún lugar de Egipto. \bibleverse{16} El
Faraón llamó urgentemente a Moisés y a Aarón y dijo: ``He pecado contra
el Señor tu Dios y contra ti. \footnote{\textbf{10:16} Éxod 9,27}
\bibleverse{17} Así que, por favor, perdona mi pecado sólo esta vez y
suplica al Señor tu Dios, pidiéndole que al menos me quite esta plaga
mortal''. \footnote{\textbf{10:17} Éxod 8,4; 1Sam 12,19}

\bibleverse{18} Moisés dejó al Faraón y rezó al Señor. \footnote{\textbf{10:18}
  Núm 11,2} \bibleverse{19} El Señor cambió la dirección del viento, de
modo que un fuerte viento del Oeste arrastró a las langostas hasta el
Mar Rojo. No quedó ni una sola langosta en ningún lugar de Egipto.
\bibleverse{20} Pero el Señor hizo que el Faraón se obstinara y no
dejara ir a los israelitas.

\hypertarget{la-novena-plaga-oscuridad}{%
\subsection{La novena plaga:
oscuridad}\label{la-novena-plaga-oscuridad}}

\bibleverse{21} El Señor le dijo a Moisés: ``Levanta tu mano hacia el
cielo para que caiga la oscuridad sobre Egipto, una oscuridad tan espesa
que se pueda sentir''. \bibleverse{22} Moisés levantó su mano hacia el
cielo, y todo Egipto quedó completamente a oscuras durante tres días.
\bibleverse{23} Nadie podía ver a nadie más, y nadie se movió de donde
estaba durante tres días. Pero todavía había luz donde vivían todos los
israelitas.

\bibleverse{24} Finalmente el Faraón llamó a Moisés. ``Vayan y adoren al
Señor'', dijo. ``Dejen a sus rebaños y manadas aquí. Incluso puedes
llevarte a tus hijos contigo''. \footnote{\textbf{10:24} Éxod 10,10}

\bibleverse{25} Pero Moisés respondió: ``También debes dejarnos animales
para los sacrificios y los holocaustos, para que podamos ofrecerlos al
Señor nuestro Dios. \bibleverse{26} Nuestro ganado tiene que ir con
nosotros también. No se dejará ni un solo animal. Necesitaremos algunos
para adorar al Señor nuestro Dios, y no sabremos cómo debemos adorar al
Señor hasta que lleguemos allí''.

\bibleverse{27} Pero el Señor hizo que el Faraón se obstinara, y no los
dejó ir. \bibleverse{28} El Faraón le gritó a Moisés: ``¡Fuera de aquí!
¡No quiero volver a verte nunca más! ¡Si te vuelvo a ver, morirás!''

\bibleverse{29} ``Que sea como tú dices'', respondió Moisés. ``No
volveré a verte''

\hypertarget{anuncio-de-la-duxe9cima-plaga-la-muerte-del-primoguxe9nito}{%
\subsection{Anuncio de la décima plaga, la muerte del
primogénito}\label{anuncio-de-la-duxe9cima-plaga-la-muerte-del-primoguxe9nito}}

\hypertarget{section-10}{%
\section{11}\label{section-10}}

\bibleverse{1} El Señor le dijo a Moisés: ``Hay una última plaga que
derribaré sobre el Faraón y sobre Egipto. Después de eso os dejará
marchar, pero cuando lo haga, os expulsará a todos del país.
\bibleverse{2} Ahora ve y dile a los israelitas, tanto hombres como
mujeres, que pidan a sus vecinos egipcios objetos de plata y oro''.
\footnote{\textbf{11:2} Éxod 3,21-22} \bibleverse{3} El Señor hizo que
los egipcios miraran favorablemente a los israelitas. De hecho, el
propio Moisés era muy respetado en Egipto tanto por los oficiales del
Faraón como por la gente común.

\bibleverse{4} Moisés dijo: ``Esto es lo que dice el Señor: Alrededor de
la medianoche recorreré todo Egipto. \bibleverse{5} Todo primogénito en
la tierra de Egipto morirá, desde el primogénito del Faraón sentado en
su trono hasta el primogénito de la sirvienta que trabaja con un molino
de mano, así como todo primogénito del ganado. \bibleverse{6}
Habráfuertes gritos de luto en todo Egipto, como nunca antes se ha
hecho, y nunca más se hará. \bibleverse{7} Pero entre todos los
israelitas ni siquiera el ladrar de un perro molestará a las personas o
a sus animales. Así sabrán que el Señor distingue entre Egipto e Israel.
\footnote{\textbf{11:7} Éxod 9,4; Éxod 9,26} \bibleverse{8} Todos tus
oficiales vendrán a mí, se inclinarán ante mí y me dirán: `¡Vete y
llévate a todos tus seguidores!' Después de eso me iré''. Moisés se
enfadó mucho y se fue de la presencia del Faraón.

\bibleverse{9} El Señor le dijo a Moisés: ``El Faraón se niega a
escucharte para que pueda hacer más milagros en Egipto''.
\bibleverse{10} Moisés y Aarón hicieron estos milagros ante el Faraón,
pero el Señor le dio al Faraón una actitud obstinada, y no dejó que los
israelitas salieran de su país.

\hypertarget{instituciuxf3n-de-pascua}{%
\subsection{Institución de Pascua}\label{instituciuxf3n-de-pascua}}

\hypertarget{section-11}{%
\section{12}\label{section-11}}

\bibleverse{1} El Señor le dijo a Moisés y a Aarón cuando aún estaban en
Egipto: \bibleverse{2} ``Este mes será para ti el primer mes, el primer
mes de tu año. \footnote{\textbf{12:2} Éxod 13,4} \bibleverse{3} Diles a
todos los israelitas que el décimo día de este mes, cada hombre debe
elegir un cordero\footnote{\textbf{12:3} ``Cordero'': O una cabra joven.
  La palabra usada aquí se aplica a ambos.} para su familia, uno para
cada hogar. \bibleverse{4} Sin embargo, si la casa es demasiado pequeña
para un cordero entero, entonces él y su vecino más cercano pueden
elegir un cordero según el número total de personas. Dividirán el
cordero según lo que cada uno pueda comer. \bibleverse{5} El cordero
debe ser un macho de un año sin ningún defecto, y puede ser tomado del
rebaño de ovejas o del rebaño de cabras. \bibleverse{6} ``Guárdalo hasta
el día catorce del mes, cuando todos los israelitas sacrificarán los
animales después de la puesta del sol y antes de que oscurezca.
\bibleverse{7} Tomarán un poco de sangre y la pondrán a los lados y en
la parte superior de los marcos de las puertas de las casas en las que
coman. \footnote{\textbf{12:7} Éxod 12,13; Éxod 12,22} \bibleverse{8}
Asarán la carne en el fuego y la comerán esa noche, junto con pan sin
levadura y hierbas amargas. \bibleverse{9} No deben comer la carne cruda
o hervida en agua. Todo debe ser asado sobre el fuego, incluyendo la
cabeza, las piernas y los intestinos. \bibleverse{10} Asegúrense de que
no quede nada hasta la mañana. Si sobra algo, deben quemarlo por la
mañana. \bibleverse{11} ``Así es como deben comer la comida. Deben estar
vestidosy listos para viajar, con las sandalias en los pies y el bastón
en la mano. Deben comer rápido, pues es la Pascua del Señor.
\bibleverse{12} Esa misma noche recorreré todo Egipto y mataré a todos
los primogénitos de las personas y los animales, y traeré la condenación
a todos los dioses de Egipto. Yo soy el Señor. \footnote{\textbf{12:12}
  Núm 33,4} \bibleverse{13} Marcaré las casas con sangre, y cuando vea
la sangre, pasaré de largo. Ninguna plaga mortal caerá sobre ustedesni
los destruirá cuando ataque a Egipto. \footnote{\textbf{12:13} Heb 11,28}

\hypertarget{arreglos-para-la-fiesta-de-los-panes-sin-levadura}{%
\subsection{Arreglos para la fiesta de los panes sin
levadura}\label{arreglos-para-la-fiesta-de-los-panes-sin-levadura}}

\bibleverse{14} ``Este será para ustedes un día para recordar. Lo
celebrarán como un festival para el Señor por las generaciones futuras.
Observarán esto por todos los tiempos venideros.

\bibleverse{15} Durante siete días sólo comerán pan hecho sin levadura.
El primer día deben deshacerse de la levadura de sus casas. Cualquiera
que coma algo con levadura desde el primer día hasta el séptimo debe ser
excluido de la comunidad israelita. \footnote{\textbf{12:15} Éxod 13,7}
\bibleverse{16} Tanto el primer como el séptimo día deben tener una
reunión sagrada. No deben trabajar en esos días, excepto para preparar
la comida. Eso es lo único que pueden hacer. \bibleverse{17}
``Celebrarán la fiesta de los panes sin levadura porque en este mismo
día yo saqué a sus tribus de Egipto. Deben observar este día de aquí en
adelante. \bibleverse{18} En el primer mes deberán comer pan sin
levadura desde la tarde del día catorce hasta la tarde del día
veintiuno. \bibleverse{19} Durante siete días no debe haber levadura en
sus casas. Si alguien come algo con levadura, debe ser excluido de la
comunidad israelita, sea extranjero o nativo de la tierra.
\bibleverse{20} No comerán nada que contenga levadura. Coman sólo pan
sin levadura en todas sus casas''.

\hypertarget{moisuxe9s-enseuxf1a-a-los-ancianos-los-preceptos-sobre-la-pascua}{%
\subsection{Moisés enseña a los ancianos los preceptos sobre la
Pascua}\label{moisuxe9s-enseuxf1a-a-los-ancianos-los-preceptos-sobre-la-pascua}}

\bibleverse{21} Entonces Moisés convocó a todos los ancianos de Israel y
les dijo: ``Vayan enseguida y elijan un cordero para cada una de sus
familias y maten el cordero de la Pascua. \bibleverse{22} Cojan un
manojo de hisopo, mójenlo en la sangre de la palangana y pónganlo en la
parte superior y en los lados del marco de la puerta. Ninguno de ustedes
saldrá por la puerta de la casa hasta la mañana. \bibleverse{23}
``Cuando el Señor pase a castigar a los egipcios, verá la sangre en la
parte superior y en los lados del marco de la puerta. Pasará porencima
de la puerta y no permitirá que el destructor entre en sus casas y los
mate. \bibleverse{24} Ustedes y sus descendientes deberán recordar estas
instrucciones para el futuro. \bibleverse{25} Cuando entren en la tierra
que el Señor prometió darles, celebrarán esta ceremonia. \bibleverse{26}
Cuando sus hijos vengan y les pregunten: `¿Por qué es importante esta
ceremonia para ustedes?' \bibleverse{27} deben decirles: `Este es el
sacrificio de Pascua para el Señor. Él fue quien pasó por encima de las
casas de los israelitas en Egipto cuando mató a los egipcios, pero
perdonó a nuestras familias'\,''. El pueblo se inclinó en adoración.

\bibleverse{28} Entonces los israelitas fueron e hicieron lo que el
Señor les había dicho a Moisés y a Aarón.

\hypertarget{la-duxe9cima-plaga-la-muerte-del-primoguxe9nito-egipcio}{%
\subsection{La décima plaga: la muerte del primogénito
egipcio}\label{la-duxe9cima-plaga-la-muerte-del-primoguxe9nito-egipcio}}

\bibleverse{29} A medianoche el Señor mató a todo primogénito varón en
la tierra de Egipto, desde el primogénito del Faraón, que estaba sentado
en su trono, hasta el primogénito del prisionero en la cárcel, y también
todo el primogénito del ganado \footnote{\textbf{12:29} Éxod 4,23}
\bibleverse{30} El Faraón se levantó durante la noche, así como todos
sus oficiales y todos los egipcios. Hubo fuertes gritos de agonía en
todo Egipto, porque no había una sola casa en la que no hubiera muerto
alguien. \bibleverse{31} El Faraón llamó a Moisés y a Aarón durante la
noche y les dijo: ``¡Fuera de aquí! ¡Dejen a mi pueblo, ustedes dos y
los israelitas! Váyanse, para que puedan adorar al Señor como lo han
pedido. \bibleverse{32} ¡Llévense también a sus rebaños y manadas, como
lo dijeron antes y váyanse! Oh, y bendíceme a mí también''.

\bibleverse{33} Los egipcios instaron a los israelitas a dejar su país
lo más rápido posible, diciendo: ``¡Si no, moriremos todos!''
\footnote{\textbf{12:33} Éxod 6,1} \bibleverse{34} Así que los
israelitas recogieron su masa antes de que se levantara y la llevaron
sobre sus hombros en tazones de amasar envueltos en ropa \bibleverse{35}
Además, los israelitas hicieron lo que Moisés les había dicho y pidieron
a los egipcios objetos de plata y oro, y ropa. \bibleverse{36} El Señor
había hecho que los egipcios miraran tan favorablemente a los israelitas
que aceptaron su petición. De esta manera se llevaron las riquezas de
los egipcios. \footnote{\textbf{12:36} Éxod 3,21}

\hypertarget{el-uxe9xodo-de-israel-de-egipto}{%
\subsection{El éxodo de Israel de
Egipto}\label{el-uxe9xodo-de-israel-de-egipto}}

\bibleverse{37} Los israelitas partieron a pie desde Ramsés hacia Sucot
y fueron unos 600. 000 hombres, así como mujeres y niños.\footnote{\textbf{12:37}
  ``Mujeres y niños'': Literalmente, ``dependientes''.} \bibleverse{38}
Además, muchos extranjeros se les unieron. También se llevaron consigo
grandes rebaños y manadas de ganado. \bibleverse{39} Como su masa de pan
no tenía levadura, los israelitas cocinaron lo que habían sacado de
Egipto en panes sin levadura. Esto se debió a que cuando fueron
expulsados de Egipto tuvieron que salir de prisa y no tuvieron tiempo de
prepararse la comida. \bibleverse{40} Los israelitas habían vivido en
Egipto durante 430 años. \bibleverse{41} El mismo día en que terminaron
los 430 años, todas las tribus del Señor, por sus respectivas
divisiones, salieron de Egipto. \bibleverse{42} Siendo que el Señor veló
esa noche para sacarlos de la tierra de Egipto, ustedes deben velar esa
misma noche como una observancia para honrar al Señor, que será guardada
por todos los israelitas para las generaciones futuras.

\hypertarget{ordenanza-de-la-pascua-santificaciuxf3n-del-primoguxe9nito}{%
\subsection{Ordenanza de la Pascua; Santificación del
primogénito}\label{ordenanza-de-la-pascua-santificaciuxf3n-del-primoguxe9nito}}

\bibleverse{43} El Señor les dijo a Moisés y a Aarón: ``Esta es la
ceremonia de la Pascua. Ningún extranjero puede comerla. \bibleverse{44}
Pero cualquier esclavo que haya sido comprado puede comerla cuando lo
hayas circuncidado. \bibleverse{45} Los visitantes extranjeros o los
contratados de otras naciones no podrán comer la Pascua. \bibleverse{46}
Se debe comer dentro de la casa. No se permite sacar nada de la carne
fuera de la casa, ni romper ningún hueso. \footnote{\textbf{12:46} Juan
  19,36}

\bibleverse{47} Todos los israelitas deben celebrarla. \bibleverse{48}
Si hay un extranjero que vive con ustedes y quiere celebrar la Pascua
del Señor, todos los varones de su casa tienen que ser circuncidados.
Entonces podrán venir a celebrar y ser tratados como nativosdel país.
Pero ningún hombre que no esté circuncidado puede comerla.
\bibleverse{49} La misma regla se aplica tanto al nativo como al
extranjero que vive entre ustedes''. \bibleverse{50} Entonces todos los
israelitas siguieron estas instrucciones. Hicieron exactamente lo que el
Señor había ordenado a Moisés y Aarón. \bibleverse{51} Ese mismo día el
Señor sacó a las tribus israelitas de Egipto, una por una.

\hypertarget{section-12}{%
\section{13}\label{section-12}}

\bibleverse{1} Entonces el Señor le dijo a Moisés: \bibleverse{2} ``Todo
varón primogénito será dedicado a mí. El primogénito de cada familia
israelita me pertenece, y también cada animal primogénito''. \footnote{\textbf{13:2}
  Núm 8,17-18; Núm 18,15; Luc 2,23}

\hypertarget{el-reglamento-sobre-la-celebraciuxf3n-de-la-fiesta-de-los-panes-sin-levadura}{%
\subsection{El reglamento sobre la celebración de la Fiesta de los Panes
sin
Levadura}\label{el-reglamento-sobre-la-celebraciuxf3n-de-la-fiesta-de-los-panes-sin-levadura}}

\bibleverse{3} Así que Moisés le dijo al pueblo: ``Recuerden que este es
el día en que dejaron Egipto, la tierra de su esclavitud, porque el
Señor los sacó de allí con su asombroso poder. (Nada con levadura en él
será comido). \bibleverse{4} Hoyustedes están en camino, este día en el
mes de Abib. \bibleverse{5} El Señor los llevará a la tierra de los
cananeos, hititas, amorreos, heveos y jebuseos, la tierra que le
prometió a sus antepasados, una tierra que fluye leche y miel. Así que
deben observar esta ceremonia en este mes. \footnote{\textbf{13:5} Gén
  17,8} \bibleverse{6} Durante siete días sólo comerán pan sin levadura,
y el séptimo día celebrarán una fiesta religiosa para honrar al Señor.
\footnote{\textbf{13:6} Éxod 12,15-16} \bibleverse{7} Durante esos siete
días solo podrán comer pan sin levadura. No deben tener levadura; de
hecho, no debe haber levadura en ningún lugar donde vivan. \footnote{\textbf{13:7}
  1Cor 5,8} \bibleverse{8} ``Ese día digan a sus hijos: `Esto es por
causa de lo que el Señor hizo por mí cuando salí de Egipto'.
\bibleverse{9} Cuando celebren esta ceremonia\footnote{\textbf{13:9}
  ``Cuando celebren esta ceremonia'': añadido para mayor claridad.} será
como una señal en su mano y un recordatorio en la frente de que esta
enseñanza del Señor debe ser contada con regularidad. Porque el Señor
los sacó de Egipto con su gran poder. \bibleverse{10} Es por eso que
deben observar esta ceremonia año tras año, en esta fecha.

\hypertarget{santificaciuxf3n-del-primoguxe9nito}{%
\subsection{Santificación del
primogénito}\label{santificaciuxf3n-del-primoguxe9nito}}

\bibleverse{11} Una vez que el Señor los lleve a la tierra de los
cananeos y se las entregue, como se los prometió a ustedes y a sus
antepasados, \bibleverse{12} deben presentar al Señor todos los
primogénitos varones, humanos o animales. Todos los primogénitos de su
ganado le pertenecen al Señor. \bibleverse{13} Deben rescatara cada asno
primogénito a cambio de un cordero, y si no lo hacen, deberán romperle
el cuello. Deberán rescatar a cada primogénito de sus hijos.
\bibleverse{14} ``Cuando en el futuro sus hijos vengan a ustedes y les
pregunten: `¿Por qué es importante esta ceremonia' deberán decirles: `El
Señor nos sacó de Egipto, la tierra de nuestra esclavitud, mediante su
asombroso poder. \footnote{\textbf{13:14} Éxod 12,26} \bibleverse{15} El
Faraón se negó obstinadamente a dejarnos ir, así que el Señor mató a
todos los primogénitos de la tierra de Egipto, tanto humanos como
animales. Por eso sacrificamos al Señor el primogénito de cada animal, y
compramos todos los primogénitos de nuestros hijos'. \footnote{\textbf{13:15}
  Éxod 12,29} \bibleverse{16} De esta manera, será como una señal en la
mano y un recordatorio en la frente, porque el Señor nos sacó de Egipto
por su asombroso poder''.

\hypertarget{el-tren-al-desierto-y-el-mar-rojo-a-etham}{%
\subsection{El tren al desierto y el Mar Rojo a
Etham}\label{el-tren-al-desierto-y-el-mar-rojo-a-etham}}

\bibleverse{17} Cuando el Faraón dejó salir a los israelitas, Dios no
los llevó por la tierra de los filisteos, aunque era un camino más
corto. Porque Dios dijo, ``Si se ven obligados a luchar, podrían cambiar
de opinión y volver a Egipto''. \bibleverse{18} Así que Dios los llevó
por el camino más largo a través del desierto hacia el Mar Rojo. Cuando
los israelitas dejaron la tierra de Egipto eran como un ejército listo
para la batalla. \bibleverse{19} Moisés llevó los huesos de José con él
porque José leshabía hecho a los hijos de Israel una promesa solemne,
diciendo: ``Dios definitivamente cuidará de ustedes, y entonces deben
llevarse mis huesos cuando salgan de aquí''. \footnote{\textbf{13:19}
  Gén 50,25; Jos 24,32}

\bibleverse{20} Viajaron desde Sucot y acamparon en Etam, a la entrada
del desierto. \bibleverse{21} El Señor iba delante de ellos como una
columna de nubes para mostrarles el camino durante el día, y como una
columna de fuego para proporcionarles luz por la noche. Así podían
viajar de día o de noche. \bibleverse{22} La columna de nubes durante el
día y la columna de fuego por la noche iban siempre delante del pueblo.

\hypertarget{dios-ordena-el-cambio-de-direcciuxf3n}{%
\subsection{Dios ordena el cambio de
dirección}\label{dios-ordena-el-cambio-de-direcciuxf3n}}

\hypertarget{section-13}{%
\section{14}\label{section-13}}

\bibleverse{1} Entonces el Señor le dijo a Moisés: \bibleverse{2}
``Diles a los israelitas que vuelvan y acampen cerca de Pi-Ajirot, entre
Migdol y el mar. Deben acampar junto al mar, frente a Baal-Zefón.
\bibleverse{3} El Faraónsacará su conclusión respecto a los israelitas:
`Están vagando por el país con gran confusión, y el desierto les ha
impedido salir'. \bibleverse{4} Daré a Faraón una actitud terca para que
los persiga a fin derecuperarlos.\footnote{\textbf{14:4} ``A fin de
  recuperarlos'': añadido para mayor claridad.} Pero ganaré honra por lo
que le sucederá al Faraón y a todo su ejército, y los egipcios sabrán
que yo soy el Señor''. Así que los israelitas hicieron lo que se les
ordenó. \footnote{\textbf{14:4} Éxod 4,21; Éxod 9,16; Ezeq 28,22}

\hypertarget{el-farauxf3n-persigue-a-los-israelitas-y-los-alcanza}{%
\subsection{El faraón persigue a los israelitas y los
alcanza}\label{el-farauxf3n-persigue-a-los-israelitas-y-los-alcanza}}

\bibleverse{5} Cuando el rey de Egipto se enteró de que los israelitas
se habían marchado apresuradamente, el Faraón y sus oficiales cambiaron
de opinión sobre lo que había sucedido y dijeron: ``¿Qué hemos hecho?
Hemos dejado ir a todos estos esclavos israelitas''. \bibleverse{6} Así
que el Faraón hizo preparar su carro y se puso en marcha con su
ejército. \bibleverse{7} Tomó 600 de sus mejores carros junto con todos
los demás carros de Egipto, cada uno con su oficial a cargo.
\bibleverse{8} El Señor le dio al Faraón, rey de Egipto, una actitud
terca, así que persiguió a los israelitas, que salían con los puños
levantados en triunfo. \bibleverse{9} Los egipcios salieron en
persecución, con todos los caballos y carros del Faraón, así como
jinetes y soldados. Alcanzaron a los israelitas mientras estaban
acampandojunto al mar cerca de Pi-Ajirot, frente a Baal-Zefón.

\hypertarget{moisuxe9s-anima-a-los-israelitas-desanimados-la-intervenciuxf3n-de-dios}{%
\subsection{Moisés anima a los israelitas desanimados; La intervención
de
Dios}\label{moisuxe9s-anima-a-los-israelitas-desanimados-la-intervenciuxf3n-de-dios}}

\bibleverse{10} Los israelitas miraron hacia atrás y vieron al Faraón y
al ejército egipcio acercándose. Estaban absolutamente aterrorizados y
pidieron ayuda al Señor. \bibleverse{11} Se quejaron a Moisés: ``¿No
había tumbas en Egipto que nos tuvieras que traer aquí en el desierto
para morir? ¿Qué nos has hecho al hacernos salir de Egipto?
\bibleverse{12} ¿Acaso no te dijimos en Egipto: `Déjanos en paz para que
sigamos siendo esclavos de los egipcios'? ¡Hubiera sido mejor para
nosotros ser esclavos de los egipcios que morir aquí en el desierto!''

\bibleverse{13} Pero Moisés le dijo al pueblo: ``No tengan miedo.
Quédense donde están y verán cómo el Señor nos salvará hoy. Los egipcios
que ven ahora, ¡no los volverán a ver nunca más! \bibleverse{14} El
Señor va a luchar por ustedes, así que no necesitan hacer nada''.
\footnote{\textbf{14:14} Deut 1,30; 2Cró 20,15; Is 30,15}

\bibleverse{15} El Señor le dijo a Moisés: ``¿Por qué clamas a mi con
gritos? Dile a los israelitas que sigan adelante. \bibleverse{16} Debes
tomar tu bastón y sostenerlo en tu mano sobre el mar. Divídelo para que
los israelitas puedan caminar por el mar en tierra seca. \bibleverse{17}
Pondré en los egipcios una actitud obstinada y dura para que los
persigan. Entonces me ganaré su honra por lo que le sucederá al Faraón y
a todo su ejército, así como a sus carros y jinetes. \bibleverse{18} Los
egipcios sabrán que soy el Señor cuando me gane su respeto a través del
Faraón, sus carros y su caballería''. \bibleverse{19} El ángel de Dios,
que había estado guiando a los israelitas, se movía detrás de ellos,
\footnote{\textbf{14:19} Éxod 13,21} \bibleverse{20} posicionándose
entre los campos de los egipcios y de los israelitas. La nube estaba
oscurapor un lado, pero iluminaba la noche por el otro. Nadie de ninguno
de los dos campamentos se acercaba al otro durante la noche.

\hypertarget{paso-de-los-israelitas-por-el-mar-rojo-cauxedda-de-los-egipcios}{%
\subsection{Paso de los israelitas por el Mar Rojo; Caída de los
egipcios}\label{paso-de-los-israelitas-por-el-mar-rojo-cauxedda-de-los-egipcios}}

\bibleverse{21} Entonces Moisés extendió su mano sobre el mar, y durante
toda la noche el Señor hizo retroceder el mar con un fuerte viento del
este, y convirtió el fondo del mar en tierra firme. Así que el agua se
dividió, \bibleverse{22} y los israelitas caminaron por el mar en tierra
seca, con muros de agua a su derecha y a su izquierda. \bibleverse{23}
Los egipcios los persiguieron, con todos los caballos, carros y jinetes
del Faraón. Siguieron a los israelitas hasta el mar. \footnote{\textbf{14:23}
  Éxod 15,19} \bibleverse{24} Pero al final de la noche el Señor miró al
ejército egipcio desde la columna de fuego y nube, y les causó pánico.
\footnote{\textbf{14:24} Sal 34,17; Sal 104,32} \bibleverse{25} Hizo que
las ruedas de sus carros se atascaran, por lo que les resultaba difícil
conducir. Los egipcios gritaron: ``¡Retírense! ¡Debemos huir de los
israelitas porque el Señor está luchando en favor de ellos contra
nosotros!'' \footnote{\textbf{14:25} Éxod 14,14; Sal 64,10}

\bibleverse{26} Entonces el Señor le dijo a Moisés: ``Extiende tu mano
sobre el mar, para que el agua caiga sobre los egipcios, sus carros y
jinetes''. \bibleverse{27} Entonces Moisés extendió su mano sobre el
mar, y al amanecer el mar volvió a la normalidad. Mientras los egipcios
se retiraban, el Señor los arrastró al mar. \bibleverse{28} El agua cayó
sobre ellos y cubrió los carros y los jinetes, así como todo el ejército
del Faraón que había perseguido a los israelitas hasta el mar. Ni uno
solo de ellos sobrevivió. \bibleverse{29} Pero los israelitas habían
caminado por el mar en tierra seca, con muros de agua a su derecha y a
su izquierda. \bibleverse{30} El Señor salvó a los israelitas de la
amenaza de los egipcios. Y los israelitas vieron a los egipcios muertos
en la orilla. \bibleverse{31} Cuandovieron el gran poder que el Señor
había usado contra los egipcios, los israelitas se quedaron asombrados
del Señor y confiaron en él y en su siervo Moisés.\footnote{\textbf{14:31}
  Éxod 19,9; 2Cró 20,20}

\hypertarget{canciuxf3n-de-victoria-de-los-israelitas-en-el-mar-rojo}{%
\subsection{Canción de victoria de los israelitas en el Mar
Rojo}\label{canciuxf3n-de-victoria-de-los-israelitas-en-el-mar-rojo}}

\hypertarget{section-14}{%
\section{15}\label{section-14}}

\bibleverse{1} Entonces Moisés y los israelitas cantaron esta canción al
Señor: ¡Cantaré al Señor, porque él es supremo! Ha arrojado al mar a los
caballos y a sus jinetes. \footnote{\textbf{15:1} Apoc 15,3}
\bibleverse{2} El Señor me da fuerza. Él es el tema de mi canción. Él me
salva. Él es mi Dios, y yo lo alabaré. Él es el Dios de mi padre, y yo
lo honraré. El Señor me da fuerza. Él es el tema de mi canción. Él me
salva. Él es mi Dios, y yo lo alabaré. Él es el Dios de mi padre, y yo
lo honraré. \footnote{\textbf{15:2} Sal 118,14; Is 12,2} \bibleverse{3}
El Señor es como un guerrero. Su nombre es el Señor. \footnote{\textbf{15:3}
  Éxod 14,14; Sal 46,10; Éxod 3,15} \bibleverse{4} Arrojó los carros del
Faraón y su ejército al mar. Los mejores oficiales del Faraón se
ahogaron en el Mar Rojo. \bibleverse{5} El agua los cubrió como una
inundación. Cayeron a las profundidades como una piedra. \bibleverse{6}
Tu poder, Señor, es verdaderamente asombroso. Tu poder, Señor, aplastó
al enemigo. \bibleverse{7} Con tu majestuoso poder destruiste a los que
se te oponían. Tu cólera ardió y los quemó como un rastrojo. \footnote{\textbf{15:7}
  Is 47,14} \bibleverse{8} Túsoplaste\footnote{\textbf{15:8}
  Literalmente, ``por el aliento de tu nariz''.} y el mar se amontonó.
Las olas se alzaron como un muro. Las profundidades del océano se
volvieron sólidas. \bibleverse{9} El enemigo se jactó: ``Los perseguiré
y los alcanzaré. Dividiré el botín. Los comeré vivos. Bailaré con
laespada. Con mi mano los destruiré''. \bibleverse{10} Pero tú soplaste
con tu aliento y el mar los arrastró. Se hundieron como el plomo en las
aguas revueltas. \bibleverse{11} ¿Quién es como tú entre los dioses,
Señor? ¿Quién es como tú, glorioso en santidad, asombroso y maravilloso,
que hace milagros? \bibleverse{12} Tú actuaste, y la tierra se tragó a
los egipcios. \bibleverse{13} Guiaste a las personas que salvaste con tu
confiable amor. Los guiarás en tu fuerza a tu santo hogar.
\bibleverse{14} Las naciones oirán lo que ha sucedido y temblarán de
miedo. El pueblo que vive en Filistea experimentará una angustia
agonizante. \footnote{\textbf{15:14} Jos 2,9-11} \bibleverse{15} Los
jefes edomitas estarán aterrorizados. Los líderes moabitas temblarán. La
gente que vive en Canaán se derretirá en pánico. \bibleverse{16} El
terror y el miedo caerán sobre ellos. Señor, debido a tu gran poder,
estarán quietos como una piedra hasta que tu pueblo pase, hasta que pase
el pueblo que compraste. \bibleverse{17} Tomarás a tu pueblo y lo
plantarás en el monte que tú posees, el lugar que tú, Señor, has
preparado como tu casa, el \bibleverse{18} ¡El Señor reinará por siempre
y para siempre!

\bibleverse{19} Cuando los caballos, carros y jinetes del Faraón
entraron en el mar, el Señor hizo que el agua se precipitara sobre
ellos. Pero los israelitas caminaron por el mar en tierra seca.
\footnote{\textbf{15:19} Éxod 14,22-29}

\hypertarget{participaciuxf3n-de-mujeres-en-la-alabanza-del-seuxf1or}{%
\subsection{Participación de mujeres en la alabanza del
Señor}\label{participaciuxf3n-de-mujeres-en-la-alabanza-del-seuxf1or}}

\bibleverse{20} La profeta Miriam, hermana de Aarón, cogió una pandereta
y todas las mujeres la siguieron bailando y tocando la pandereta.
\footnote{\textbf{15:20} Sal 68,26} \bibleverse{21} Miriam les cantó:
``¡Canten al Señor, porque él es supremo! Ha arrojado al mar a los
caballos y a sus jinetes''. \footnote{\textbf{15:21} Éxod 15,1}

\hypertarget{el-agua-amarga-de-mara-se-volviuxf3-apetecible-la-llegada-a-la-encantadora-elim}{%
\subsection{El agua amarga de Mara se volvió apetecible; la llegada a la
encantadora
Elim}\label{el-agua-amarga-de-mara-se-volviuxf3-apetecible-la-llegada-a-la-encantadora-elim}}

\bibleverse{22} Entonces Moisés llevó a Israel lejos del Mar Rojo y al
desierto de Sur. Durante tres días caminaron por el desierto pero no
encontraron agua. \bibleverse{23} Cuando llegaron a Mara, el agua allí
era demasiado amarga para beber. (Por eso el lugar se llama Mara).
\bibleverse{24} Entonces el pueblo se quejó a Moisés, preguntando:
``¿Qué vamos a beber?'' \bibleverse{25} Moisés le pidió ayuda al Señor,
y el Señor le mostró un trozo de madera. Cuando lo arrojó al agua, se
volvió dulce. Allí el Señor les dio reglas e instrucciones y también
puso a prueba su lealtad hacia él.\footnote{\textbf{15:25} ``Lealtad
  hacia él'': añadido para mayor claridad.} \bibleverse{26} Les dijo:
``Si prestan atención a lo que dice el Señor su Dios, hagan lo que es
correcto ante sus ojos, obedezcan sus órdenes y cumplan todos sus
reglamentos, entonces no les haré sufrir ninguna de las enfermedades que
les di a los egipcios porque yo soy el Señor que los sana''.

\bibleverse{27} Luego viajaron a Elim, que tenía doce manantiales de
agua y setenta palmeras. Allí acamparon junto al agua.

\hypertarget{el-murmullo-del-pueblo-la-exaltaciuxf3n-divina-mediante-la-donaciuxf3n-de-codornices-y-manuxe1}{%
\subsection{El murmullo del pueblo; la exaltación divina mediante la
donación de codornices y
maná}\label{el-murmullo-del-pueblo-la-exaltaciuxf3n-divina-mediante-la-donaciuxf3n-de-codornices-y-manuxe1}}

\hypertarget{section-15}{%
\section{16}\label{section-15}}

\bibleverse{1} Toda la comunidad israelita dejó Elim y se fue al
desierto de pecado, entre Elim y Sinaí. Esto fue el día quince del
segundo mes después de que dejaran la tierra de Egipto. \bibleverse{2}
Allí, en el desierto, se quejaron a Moisés y a Aarón. \footnote{\textbf{16:2}
  Éxod 17,2} \bibleverse{3} ``¡El Señor debería habernos matado en
Egipto!'' les dijeron los israelitas. ``Al menos allí podíamos sentarnos
junto a ollas de carne y comer pan hasta que estuviéramos llenos. ¡Pero
tenías que traernos a todos aquí en el desierto para matarnos de
hambre!'' \footnote{\textbf{16:3} Éxod 14,11}

\bibleverse{4} El Señor le dijo a Moisés: ``Ahora haré llover pan del
cielo para ustedes. Cada día la gente debe salir y recoger lo suficiente
para ese día. Voy a ponerlos a prueba con esto para saber si seguirán
mis instrucciones o no. \footnote{\textbf{16:4} Juan 6,31; 1Cor 10,3}
\bibleverse{5} El sexto día deben recoger el doble de lo habitual y
prepararlo''.

\bibleverse{6} Entonces Moisés y Aarón explicaron a todos los
israelitas: ``Esta tarde tendrán la prueba de que el Señor fue quien los
sacó de Egipto \bibleverse{7} y por la mañana verán la gloria del Señor
desplegada al responder a las quejas que los ha oído hacer contra él.
¿Por qué debería quejarse con nosotros? ¡No somos nadie!''
\bibleverse{8} Entonces Moisés continuó: ``El Señor les dará esta tarde
carne para comer y por la mañana todo el pan que quieran, porque ha oído
sus quejas contra él. ¿Por qué se queja ante nosotros, nadie? Tus quejas
no están dirigidas contra nosotros, sino contra el Señor''.
\bibleverse{9} Entonces Moisés dijo a Aarón: ``Dile a toda la comunidad
israelita: `Preséntense ante el Señor, porque ha oído sus quejas'\,''.
\bibleverse{10} Mientras Aarón aún hablaba a todos los israelitas,
miraron hacia el desierto y vieron aparecer la gloria del Señor en una
nube. \bibleverse{11} Entonces el Señor le dijo a Moisés:
\bibleverse{12} ``He oído las quejas de los israelitas. Diles: `Por la
tarde comerás carne, y por la mañana tendrás todo el pan que quieras'.
Entonces sabrán que yo soy el Señor su Dios''.

\bibleverse{13} Esa noche las codornices volaron y aterrizaron, llenando
el campamento. Por la mañana, el rocío cubrió el suelo alrededor del
campamento. \footnote{\textbf{16:13} Núm 11,31} \bibleverse{14} Una vez
que el rocío se había ido, había una capa delgada y escamosa en el
desierto, que parecía cristales de escarcha en el suelo. \bibleverse{15}
Cuando los israelitas lo vieron, se preguntaron ``¿Qué es?'' porque no
tenían ni idea de lo que era. Así que Moisés les explicó, ``Es el pan
que el Señor ha provisto para que coman.

\hypertarget{reglas-sobre-recolecciuxf3n-de-manuxe1-moisuxe9s-explica-un-fenuxf3meno-milagroso-que-ocurriuxf3}{%
\subsection{Reglas sobre recolección de maná; Moisés explica un fenómeno
milagroso que
ocurrió}\label{reglas-sobre-recolecciuxf3n-de-manuxe1-moisuxe9s-explica-un-fenuxf3meno-milagroso-que-ocurriuxf3}}

\bibleverse{16} Esto es lo que el Señor les ha ordenado hacer: `Todos
ustedes recogerán lacantidad que les sea necesaria. Tomen un gómer por
cada persona en su tienda'''. \bibleverse{17} Los israelitas hicieron lo
que se les dijo. Algunos recolectaron más, mientras que otros
recolectaron menos. \bibleverse{18} Pero cuando lo midieron en gomeres,
a los que habían recogido mucho no les sobraba nada, mientras que a los
que sólo habían recogido un poco les sobraba. Cada persona recolectó
tanto como necesitaba para comer. \footnote{\textbf{16:18} 2Cor 8,15}
\bibleverse{19} Entonces Moisés les dijo: ``Nadie debe dejar nada para
mañana''. \footnote{\textbf{16:19} Mat 6,34; Luc 11,3} \bibleverse{20}
Pero algunos no escucharon a Moisés. Dejaron un poco para el día
siguiente, y estaba lleno de gusanos y olía mal. Y Moisés se enfadó con
ellos. \bibleverse{21} Así que cada mañana todos recogían todo lo que
necesitaban, y cuando el sol se calentaba, se desvanecía.
\bibleverse{22} Sin embargo, en el sexto día, recogieron el doble de
esta comida, dos gomeres por cada persona. Todos los líderes israelitas
vinieron y le dijeron a Moisés lo que habían hecho. \bibleverse{23}
Moisés respondió: ``Estas son las instrucciones del Señor: `Mañana es un
día especial de descanso, un sábado santo para honrar al Señor. Así que
horneen lo que quieran, y hiervan lo que quieran. Luego aparten lo que
quede y guárdenlo hasta la mañana'\,''. \footnote{\textbf{16:23} Gén
  2,2-3; Éxod 20,8} \bibleverse{24} Así que lo guardaron hasta la mañana
como Moisés había ordenado, y no olía mal ni tenía gusanos.
\bibleverse{25} Moisés les dijo: ``Coman hoy, porque hoy es un sábado
para honrar al Señor. Hoy no encontrarán nada ahí fuera. \bibleverse{26}
Pueden salir a recolectar durante seis días, pero el séptimo día, el
sábado, no habrá nada que puedan recolectar''. \bibleverse{27} Aún así,
el séptimo día algunas personas todavía salieron a recolectar, pero no
encontraron nada. \bibleverse{28} El Señor le dijo a Moisés: ``¿Cuánto
tiempo te negarás a obedecer mis órdenes e instrucciones?
\bibleverse{29} Debes entender que el Señor te ha dado el sábado, así
que el sexto día te dará comida para dos días. El séptimo día, todos
tienen que quedarse donde están, y nadie tiene que salir''
\bibleverse{30} Así que el pueblo no hizo ningún trabajo en el séptimo
día.

\bibleverse{31} Los israelitas llamaron a esta comida maná.\footnote{\textbf{16:31}
  Que significa, ``¿Qué es esto?'' Ver versículo 15.} Era blanca como la
semilla de cilantro y sabía a obleas con miel. \bibleverse{32} Moisés
dijo: ``Esto es lo que el Señor ha ordenado: `Guarda un gomer de maná
como recordatorio para las generaciones futuras, para que puedan ver la
comida que usé para alimentarlos en el desierto cuando los saqué de
Egipto'\,''. \bibleverse{33} Así que Moisés le dijo a Aarón: ``Toma un
frasco y pon un gomer de maná en él. Luego ponlo ante el Señor para que
lo guarde como un recordatorio para las generaciones futuras''.
\bibleverse{34} Aarón lo hizo y colocó la jarra delante del
Testimonio,\footnote{\textbf{16:34} El significado de este término en el
  contexto es incierto. Normalmente se refiere a las dos tablas de los
  Diez Mandamientos (ver 25:16, 40:20 etc.) El recipiente con maná fue
  finalmente colocado dentro del Arca del Pacto junto con las tablas de
  piedra de los Diez Mandamientos, pero ni el arca ni las tablas
  existían todavía (ver capítulos 25 y 26).} para que se conservara tal
y como el Señor se lo había ordenado a Moisés. \bibleverse{35} Los
israelitas comieron maná durante cuarenta años, hasta que llegaron a la
tierra en la que se asentarían; comieron maná hasta que llegaron a la
frontera de Canaán. \footnote{\textbf{16:35} Jos 5,12}

\bibleverse{36} (Un gómer es una décima parte de una efa).

\hypertarget{la-maravillosa-donaciuxf3n-de-agua-de-la-roca-cerca-de-massa-y-meriba}{%
\subsection{La maravillosa donación de agua de la roca cerca de Massa y
Meriba}\label{la-maravillosa-donaciuxf3n-de-agua-de-la-roca-cerca-de-massa-y-meriba}}

\hypertarget{section-16}{%
\section{17}\label{section-16}}

\bibleverse{1} Todos los israelitas dejaron el desierto de Sin, yendo de
un lugar hacia otro, según las órdenes del Señor. Acamparon en Refidim,
pero no había agua para que el pueblo la bebiera. \bibleverse{2} Algunos
de ellos vinieron y se quejaron a Moisés, diciendo: ``¡Danos agua para
beber!'' Moisés respondió, ``¿Por qué se quejas conmigo?'' Preguntó
Moisés. ``¿Por qué intentan desafiar al Señor?''

\bibleverse{3} Pero el pueblo estaba tan sediento de agua que se quejó a
Moisés, diciendo: ``¿Por qué tuviste que sacarnos de Egipto? ¿Intentas
matarnos a nosotros y a nuestros hijos y ganado de sed?''

\bibleverse{4} Moisés le gritó al Señor: ``¿Qué voy a hacer con esta
gente? ¡Un poco más de esto y me apedrearán!'' \footnote{\textbf{17:4}
  Núm 14,10}

\bibleverse{5} El Señor le dijo a Moisés: ``Ve delante del pueblo y
llévate a algunos de los ancianos de Israel contigo. Lleva contigo el
bastón que usaste para golpear el Nilo, y sigue adelante. \footnote{\textbf{17:5}
  Éxod 7,20} \bibleverse{6} Mira, me pararé a tu lado junto a la roca en
Horeb. Cuando golpees la roca, el agua se derramará para que la gente
beba''. Así que Moisés hizo esto mientras los ancianos de Israel
observaban. \footnote{\textbf{17:6} Núm 20,11; 1Cor 10,4} \bibleverse{7}
Llamó al lugar Masá y Meribá\footnote{\textbf{17:7} Masá significa
  ``prueba'' y Meribá significa ``queja''.} porque los israelitas
discutieron allí, y porque desafiaron al Señor, diciendo: ``¿Está el
Señor con nosotros o no?'' \footnote{\textbf{17:7} Sal 95,8-9}

\hypertarget{la-victoria-sobre-los-amalecitas-en-refidim}{%
\subsection{La victoria sobre los amalecitas en
Refidim}\label{la-victoria-sobre-los-amalecitas-en-refidim}}

\bibleverse{8} Entonces vinieron unos amalecitas y atacaron a los
israelitas en Refidim. \bibleverse{9} Moisés le dijo a Josué: ``Escoge
algunos hombres y sal a combatir a los amalecitas. Mañana me pararé en
la cima de esta colina con el bastón de Dios''. \footnote{\textbf{17:9}
  Núm 13,8; Núm 13,16}

\bibleverse{10} Josué hizo lo que le dijo Moisés y luchó contra los
amalecitas, mientras que Moisés, Aarón y Hur subieron a la cima de la
colina. \bibleverse{11} Mientras Moisés sostenía el bastón\footnote{\textbf{17:11}
  ``El bastón'': implícito.} con sus manos, los israelitas eran los que
ganaban, pero cuando los bajaba, eran los amalecitas. \bibleverse{12} A
sí que cuando las manos de Moisés se volvieron pesadas, los otros
tomaron una piedra y la pusieron debajo de él para que se sentara. Aarón
y Hur se pararon a cada lado de Moisés y le levantaron las manos. De
esta manera sus manos se mantuvieron firmes hasta que el sol se puso.
\bibleverse{13} Como resultado, Josué derrotó al ejército amalecita.
\bibleverse{14} El Señor le dijo a Moisés: ``Escribe todo esto en un
pergamino como recordatorio y léeselo en voz alta a Josué, porque voy a
eliminar por completo a los amalecitas para que nadie en la tierra se
acuerde de ellos''. \bibleverse{15} Moisés construyó un altar y lo llamó
``el Señor es mi bandera de la victoria''. \bibleverse{16} ``¡Levanten
el estandarte de la victoria del Señor!'', declaró Moisés. ``¡El Señor
seguirá luchando contra los amalecitas por todas las generaciones!''

\hypertarget{la-visita-de-jetro-a-moisuxe9s-en-el-monte-de-dios-nombramiento-de-jueces}{%
\subsection{La visita de Jetro a Moisés en el monte de Dios;
Nombramiento de
jueces}\label{la-visita-de-jetro-a-moisuxe9s-en-el-monte-de-dios-nombramiento-de-jueces}}

\hypertarget{section-17}{%
\section{18}\label{section-17}}

\bibleverse{1} Entonces Jetro,\footnote{\textbf{18:1} También llamado
  Reuel en el capítulo 2.} el suegro de Moisés y sacerdote de Madián,
escuchó todo lo que Dios había hecho por Moisés y su pueblo, los
israelitas, y cómo el Señor los había sacado de Egipto. \footnote{\textbf{18:1}
  Éxod 3,1} \bibleverse{2} Cuando Moisés envió a casa a su esposa
Séfora, su suegro Jetro la acogió, \footnote{\textbf{18:2} Éxod 4,20}
\bibleverse{3} junto con sus dos hijos. Uno de los hijos se llamaba
Gersón,\footnote{\textbf{18:3} Ver 2:22.} ya que Moisés había dicho:
``He sido un extranjero en tierra extranjera''. \footnote{\textbf{18:3}
  Éxod 2,22} \bibleverse{4} El otro hijo se llamaba Eliezer,\footnote{\textbf{18:4}
  Que significa, ``mi Dios es mi ayuda''.} porque Moisés había dicho:
``El Dios de mi padre fue mi ayuda, y me salvó de la muerte de la mano
del Faraón''. \bibleverse{5} El suegro de Moisés, Jetro, junto con la
esposa y los hijos de Moisés, fue a verlo en el desierto en el
campamento cerca de la montaña de Dios. \bibleverse{6} A Moisés se le
dijo de antemano: ``Yo, tu suegro Jetro, vengo a verte junto con tu
esposa y sus dos hijos''.

\bibleverse{7} Moisés salió al encuentro de su suegro y se inclinó y le
besó. Se preguntaron cómo estaban y luego entraron en la tienda.
\bibleverse{8} Moisés le contó a su suegro todo lo que el Señor había
hecho al Faraón y a los egipcios en favor de los israelitas, todos los
problemas que habían experimentado en el camino y cómo el Señor los
había salvado. \bibleverse{9} Jetro se alegró de escuchar todas las
cosas buenas que el Señor había hecho por Israel cuando los había
salvado de los egipcios. \bibleverse{10} Jetro anunció: ``Bendito sea el
Señor, que te salvó de los egipcios y del Faraón. \bibleverse{11} Esto
me convence de que el Señor es más grande que todos los demás dioses,
porque salvó al pueblo de los egipcios cuando actuaron tan
arrogantemente con los israelitas''. \bibleverse{12} Entonces Jetro
presentó un holocausto y sacrificios a Dios, y Aarón vino con todos los
ancianos de Israel para comer con él en presencia de Dios.

\hypertarget{la-reorganizaciuxf3n-del-poder-judicial}{%
\subsection{La reorganización del poder
judicial}\label{la-reorganizaciuxf3n-del-poder-judicial}}

\bibleverse{13} Al día siguiente Moisés se sentó como juez del pueblo, y
le presentaron sus casos desde la mañana hasta la noche. \bibleverse{14}
Cuando su suegro vio todo lo que Moisés estaba haciendo por el pueblo,
preguntó: ``¿Qué es todo esto que estás haciendo por el pueblo? ¿Por qué
te sientas solo como juez, con todo el mundo presentándote sus casos de
la mañana a la noche?''

\bibleverse{15} ``Porque el pueblo viene a mí para consultar la decisión
de Dios'', respondió Moisés. \bibleverse{16} ``Cuando discuten sobre
algo, el caso se presenta ante mí para decidir entre uno de ellos, y les
explico las leyes y reglamentos de Dios''. \bibleverse{17} Jetro le
dijo: ``Lo que estás haciendo no es lo mejor. \bibleverse{18} Tú y los
que vienen a ti se van a agotar, porque la carga de trabajo es demasiado
pesada. No pueden manejarlo solos. \footnote{\textbf{18:18} Núm 11,14;
  Deut 1,9} \bibleverse{19} Así que, por favor, escúchame. Voy a darte
un consejo, y Dios estará contigo. Sí, debes continuar siendo el
representante del pueblo ante Dios, y llevarle sus casos a él.
\bibleverse{20} Sigue enseñándoles las leyes y los reglamentos.
Muéstrales cómo vivir y el trabajo que deben hacer. \bibleverse{21} Pero
ahora debes elegir entre el pueblo hombres competentes, hombres que
respeten a Dios y que sean dignos de confianza y no corruptos. Ponlos a
cargo del pueblo como líderes de miles, cientos, cincuenta y decenas.
\bibleverse{22} Estos hombres deben juzgar al pueblo de manera continua.
Pueden traertelos asuntos más grandes, pero podrán decidirpor sí mismos
respecto a todos los asuntos pequeños. De esta manera su carga se hará
más ligera a medida que la compartan. \bibleverse{23} Si sigues mi
consejo, y si es lo que Dios te dice que hagas, entonces podrás
sobrevivir, y toda esta gente podrá volver a casa satisfecha de que sus
casos han sido escuchados''.\footnote{\textbf{18:23} ``Satisfecha de que
  sus casos han sido escuchados'': Literalmente, ``en paz''. La palabra
  shalom, sin embargo, significa más que paz, pues también tiene el
  significado de bienestar y armonía dentro de la comunidad.}

\bibleverse{24} Moisés escuchó lo que dijo su suegro y siguió todos sus
consejos. \bibleverse{25} Así que Moisés eligió hombres competentes de
todo Israel y los puso a cargo del pueblo como líderes de miles,
cientos, cincuenta y decenas. \bibleverse{26} Y actuaron como jueces del
pueblo de manera continua. Llevaban los casos difíciles a Moisés, pero
juzgaban los pequeños asuntos por sí mismos. \bibleverse{27} Entonces
Moisés envió a Jetrode camino, y regresó a su propio país.

\hypertarget{llegada-del-pueblo-al-sinauxed-elaboraciuxf3n-de-legislaciuxf3n}{%
\subsection{Llegada del pueblo al Sinaí; Elaboración de
legislación}\label{llegada-del-pueblo-al-sinauxed-elaboraciuxf3n-de-legislaciuxf3n}}

\hypertarget{section-18}{%
\section{19}\label{section-18}}

\bibleverse{1} Dos meses después del día\footnote{\textbf{19:1} ``Dos
  meses después del día'': Literalmente, ``El día de la tercera luna
  nueva''.} en que habían salido de Egipto, los israelitas llegaron al
desierto del Sinaí. \bibleverse{2} Habían partido de Refidim, y después
de entrar en el desierto del Sinaí acamparon allí frente a la montaña.
\bibleverse{3} Moisés subió al monte de Dios. Y el Señor habló con
Moisés desde la montaña y le dijo: ``Esto es lo que debes decirles a los
descendientes de Jacob, los israelitas: \bibleverse{4} `Vieron con sus
propios ojos lo que hice con los egipcios, y cómo los llevé sobre alas
de águila, y cómo los traje hacia mí. \bibleverse{5} Ahora bien, si
realmente obedecen lo que digo y cumplen el acuerdo conmigo, entonces,
de todas las naciones, serán mi pueblo especial. Aunque que el mundo
entero es mío, \footnote{\textbf{19:5} Deut 7,6} \bibleverse{6} para mí
serán un reino de sacerdotes, una nación santa'. Esto es lo que debes
decirles a los israelitas''. \footnote{\textbf{19:6} 1Pe 2,9; Apoc 1,6;
  Lev 19,2}

\bibleverse{7} Entonces Moisés bajó, convocó a los ancianos del pueblo y
les presentó todo lo que el Señor le había ordenado decir.
\bibleverse{8} Todos respondieron: ``Prometemos hacer todo lo que el
Señor diga''. Entonces Moisés llevó la respuesta del pueblo al Señor.

\bibleverse{9} El Señor le dijo a Moisés: ``Voy a ir hacia ti en una
nube espesa para que el pueblo me oiga hablar contigo y así siempre
confiarán en ti''. Entonces Moisés le informó al Señor lo que el pueblo
había dicho. \bibleverse{10} El Señor le dijo a Moisés: ``Baja y
prepáralos espiritualmente\footnote{\textbf{19:10} ``Prepáralos
  espiritualmente'': Literalmente, ``conságralos, apártalos'', quizás
  mediante algún ritual. Ver también versículos 14 y 22.} hoy y mañana.
Deben lavar sus ropas \bibleverse{11} y estar listos al tercer día
porque es cuando el Señor descenderá al Monte Sinaí a la vista de todos.
\bibleverse{12} Establezcan un límite alrededor de la montaña y
adviértanles: `Tengan cuidado y no intenten subir a la montaña, ¡ni
siquiera la toquen! Porque cualquiera que toque la montaña seguramente
morirá. No toquen a ninguna persona o animal que haya tocado la montaña.
\footnote{\textbf{19:12} Éxod 34,3} \bibleverse{13} Asegúrate de que
sean apedreados o disparados con flechas, pues no se les debe permitir
vivir'. Sólo cuando escuchen un fuerte sonido de cuerno de carnero, el
pueblo podrá subir a la montaña''. \footnote{\textbf{19:13} Heb 12,18-20}

\bibleverse{14} Moisés bajó de la montaña y preparó al pueblo
espiritualmente y lavó sus ropas. \bibleverse{15} Luego instruyó al
pueblo: ``Prepárense para el tercer día, y no tengan relación íntima
con\footnote{\textbf{19:15} ``No tengan relación íntima con'':
  Literalmente, ``no se acerquen a mujer alguna''.} una mujer''.
\footnote{\textbf{19:15} 1Cor 7,5}

\hypertarget{la-aterradora-apariciuxf3n-de-dios-en-el-sinauxed}{%
\subsection{La aterradora aparición de Dios en el
Sinaí}\label{la-aterradora-apariciuxf3n-de-dios-en-el-sinauxed}}

\bibleverse{16} Cuando llegó la mañana del tercer día hubo truenos y
relámpagos, y una nube espesa cubrió la montaña. Hubo un fuerte sonido
de cuerno de carnero, y todos en el campamento temblaron de miedo.
\footnote{\textbf{19:16} Heb 12,21} \bibleverse{17} Moisés condujo al
pueblo fuera del campamento para encontrarse con Dios. Se pararon al pie
de la montaña \bibleverse{18} El humo se derramó sobre todo el Monte
Sinaí porque la presencia del Señor había descendido como el fuego. El
humo se elevó como el humo de un horno, y toda la montaña tembló
furiosamente. \bibleverse{19} A medida que el sonido del cuerno de
carnero se hacía cada vez más fuerte, Moisés hablaba, y Dios le
respondía con una voz fuerte y atronadora. \footnote{\textbf{19:19} Hech
  7,38} \bibleverse{20} El Señor descendió a la cima del Monte Sinaí, y
llamó a Moisés para que subiera allí. Así que Moisés subió,

\bibleverse{21} y el Señor le dijo: ``Vuelve a bajar, y adviértele al
pueblo que no se esfuercen en cruzar el límite para intentar subir donde
está el Señor o morirán. \bibleverse{22} Incluso los sacerdotes, que
vienen ante el Señor, deben prepararse espiritualmente, para que el
Señor no los castigue''.

\bibleverse{23} Pero Moisés le dijo al Señor: ``El pueblo no puede subir
al monte Sinaí. Tu mismo nos advertiste diciendo: `Establezcan un límite
alrededor de la montaña, y considérenla como sagrada'\,''.\footnote{\textbf{19:23}
  ``Trátenla como sagrada'': se utiliza la misma palabra que para
  preparar o consagrar al pueblo espiritualmente. Sin embargo, es
  evidente que un objeto inanimado como una montaña no puede ser
  ``consagrada'' de la misma manera que una persona.}

\bibleverse{24} El Señor le dijo: ``Baja y trae a Aarón contigo. Pero
los sacerdotes y el pueblo no deben tratar de subir donde estáel Señor,
o él los castigará''

\bibleverse{25} Entonces Moisés bajó y le explicó al pueblo lo que el
Señor había dicho.\footnote{\textbf{19:25} ``Lo que el Señor le había
  dicho'': añadido para mayor claridad.}

\hypertarget{la-proclamaciuxf3n-de-los-diez-mandamientos}{%
\subsection{La proclamación de los Diez
Mandamientos}\label{la-proclamaciuxf3n-de-los-diez-mandamientos}}

\hypertarget{section-19}{%
\section{20}\label{section-19}}

\bibleverse{1} Dios dijo todas las siguientes palabras: \bibleverse{2}
``Yo soy el Señor tu Dios, que te sacó de Egipto, de la tierra de tu
esclavitud.

\bibleverse{3} ``No tendrás a otros dioses aparte de mi. \footnote{\textbf{20:3}
  Deut 6,4-5; 1Cor 8,6}

\bibleverse{4} ``No harás ningún tipo de ídolo, ya sea que se parezca a
algo arriba en los cielos, o abajo en la tierra, ni debajo en las aguas.
\footnote{\textbf{20:4} Lev 26,1; Deut 27,15; Sal 97,7; Is 40,18-26; Rom
  1,23} \bibleverse{5} No debes inclinarte ante ellos ni adorarlos,
porque yo soy el Señor tu Dios y soy celosamente exclusivo. Yo pongo las
consecuencias del pecado de los que me odian sobre sus hijos, sus nietos
y sus bisnietos; \footnote{\textbf{20:5} Éxod 34,7; Jer 31,29-30; Ezeq
  18,2-3; Ezeq 18,20} \bibleverse{6} pero muestro mi amor fiel a las
miles de generaciones que me aman y guardan mis mandamientos.

\bibleverse{7} ``No debes usar mal el nombre del Señor tu Dios, porque
el Señor no perdonará a nadie que use su nombre de forma incorrecta.

\bibleverse{8} ``Recuerda el sábado para santificarlo. \footnote{\textbf{20:8}
  Éxod 16,25; Ezeq 20,12; Mar 2,27-28; Col 2,16-17} \bibleverse{9}
Tienes seis días para trabajar y ganarte el sustento, \bibleverse{10}
pero el séptimo día es el sábado para honrar al Señor tu Dios. En este
día no debes hacer ningún trabajo, ni tú, ni tu hijo o hija, ni tu
esclavo o esclava, ni el ganado, ni el extranjero que esté contigo.
\bibleverse{11} Porqueen seis días el Señor hizo los cielos y la tierra,
el mar y todo lo que hay en ellos, y luego descansó en el séptimo día.
Por eso el Señor bendijo el día de reposo y lo hizo santo.

\bibleverse{12} ``Honra a tu padre y a tu madre, para que vivas mucho
tiempo en la tierra que el Señor tu Dios te da. \footnote{\textbf{20:12}
  Mat 15,4; Efes 6,2-3}

\bibleverse{13} ``No cometerás asesinato. \footnote{\textbf{20:13} Éxod
  21,12; Gén 9,5-6; Sant 2,11}

\bibleverse{14} ``No cometerás adulterio. \footnote{\textbf{20:14} Lev
  20,10; Efes 5,3-5}

\bibleverse{15} ``No robarás. \footnote{\textbf{20:15} Lev 19,11; Efes
  4,28}

\bibleverse{16} ``No darás falso testimonio contra otros. \footnote{\textbf{20:16}
  Éxod 23,1; Efes 4,25}

\bibleverse{17} ``No desearás tener la casa de otro. No desearás a su
esposa, nia su esclavo o esclava, ni a su buey o asno, ni cualquier otra
cosa que le pertenezca''. \footnote{\textbf{20:17} Rom 7,7; Rom 13,9}

\hypertarget{el-efecto-de-la-apariciuxf3n-divina-en-la-gente-discurso-tranquilizador-de-moisuxe9s}{%
\subsection{El efecto de la aparición divina en la gente; Discurso
tranquilizador de
Moisés}\label{el-efecto-de-la-apariciuxf3n-divina-en-la-gente-discurso-tranquilizador-de-moisuxe9s}}

\bibleverse{18} Cuando todo el pueblo oyó el trueno y el sonido de la
trompeta, y vio el relámpago y el humo de la montaña, temblaron de miedo
y se alejaron. \bibleverse{19} ``Habla con nosotros y te escucharemos'',
le dijeron a Moisés. ``Pero no dejes que Dios nos hable, o moriremos''.

\bibleverse{20} Moisés les dijo: ``No teman, porque Dios sólo ha venido
a probarlos. Quiere que le tengan miedo para que no pequen''.
\bibleverse{21} Entonces el pueblo se alejó mucho cuando Moisés se
acercó a la espesa y oscura nube donde estaba Dios. \footnote{\textbf{20:21}
  Heb 12,18}

\hypertarget{orden-provisional-de-culto}{%
\subsection{Orden provisional de
culto}\label{orden-provisional-de-culto}}

\bibleverse{22} El Señor le dijo a Moisés: ``Esto es lo que les debes
decir a los israelitas: 'Vieron con sus propios ojos que les hablé desde
el cielo. \bibleverse{23} Noharán ningún ídolo de plata o de oro ni lo
adorarán aparte de mí. \bibleverse{24} Háganme un altar de tierra y
sacrifiquen sobre él sus holocaustos y ofrendas de paz, sus ovejas, sus
cabras y su ganado. Dondequiera que decida que me adoren, vendé a
ustedes y los bendeciré. \bibleverse{25} Ahora bien, si me hacen un
altar de piedras, no lo construyas con piedras cortadas, porque si usan
un cincel para cortar la piedra, dejan de ser sagradas. \footnote{\textbf{20:25}
  Deut 27,5; Jos 8,31}

\bibleverse{26} Además, no deben subir a mi altar con escalones, para
que no se vean sus partes privadas'\,''.

\hypertarget{los-derechos-de-los-esclavos-hebreos}{%
\subsection{Los derechos de los esclavos
hebreos}\label{los-derechos-de-los-esclavos-hebreos}}

\hypertarget{section-20}{%
\section{21}\label{section-20}}

\bibleverse{1} ``Estos son los reglamentos que debe presentarles:

\bibleverse{2} ``Si compran un esclavo hebreo, debe trabajar para
ustedes durante seis años. Pero en el séptimo año, debe ser liberado sin
tener que pagar nada. \bibleverse{3} Si era soltero cuando llegó, debe
irse soltero. Si tenía una esposa cuando llegó, ella debe irse con él.
\bibleverse{4} Si su amo le da una esposa y ella tiene hijos con él, la
mujer y sus hijos pertenecerán a su amo, y sólo el hombre será liberado.
\bibleverse{5} ``Sin embargo, si el esclavo declara formalmente: `Amo a
mi señor, a mi esposa y a mis hijos; no quiero ser liberado'
\bibleverse{6} entonces su señor lo llevará ante los jueces.\footnote{\textbf{21:6}
  La palabra utilizada aquí también puede referirse a Dios, pero en este
  contexto parece que se está hablando de un tribunal civil. Ver también
  22:8,9.} Luego lo pondrá de pie contra la puerta o el poste de la
puerta y usará una herramienta de metal para hacerle un agujero en la
oreja. Entonces trabajará para su amo de por vida. \footnote{\textbf{21:6}
  Éxod 22,7-8; Éxod 22,27; Deut 1,17}

\bibleverse{7} ``Si un hombre vende a su hija como esclava, no será
liberada de la misma manera que los esclavos. \footnote{\textbf{21:7}
  Éxod 21,2} \bibleverse{8} Si el hombre que la eligió para
sí\footnote{\textbf{21:8} ``La eligió para sí'': probablemente
  refiriéndose a ella como concubina.} no está satisfecho con ella, debe
dejar que sea comprada de nuevo. No podrá venderla a los extranjeros, ya
que ha sido injusto con ella. \bibleverse{9} Si decide dársela a su
hijo, debe tratarla como a una hija. \bibleverse{10} Si toma a otra
mujer, no debe reducir los subsidios de comida y ropa, ni los derechos
maritales de la primera. \bibleverse{11} Si no le da estas tres cosas,
ella es libre de irse sin pagar nada.

\hypertarget{disposiciones-para-la-protecciuxf3n-de-la-vida-humana}{%
\subsection{Disposiciones para la protección de la vida
humana}\label{disposiciones-para-la-protecciuxf3n-de-la-vida-humana}}

\bibleverse{12} ``Todo aquel que golpee y mate a otra persona debe ser
ejecutado. \footnote{\textbf{21:12} Gén 9,6; Éxod 20,13; Mat 5,21-22}
\bibleverse{13} Sin embargo, si no fue intencional y Dios permitió que
sucediera, entonces arreglaré un lugar para ustedes donde puedan correr
y estar seguros. \footnote{\textbf{21:13} Núm 35,6-29; Deut 19,4-13}
\bibleverse{14} Pero si alguien planea deliberadamente y mata a
propósito a otro, debe alejarlo de mi altar\footnote{\textbf{21:14} ``De
  mi altar'': donde la gente iba considerándolo santuario.} y
ejecutarlo. \footnote{\textbf{21:14} 1Re 2,29; 1Re 2,31}

\bibleverse{15} ``Cualquiera que golpee a su padre o madre debe ser
ejecutado.

\bibleverse{16} Cualquiera que secuestre a alguien más debe ser
ejecutado, ya sea que la víctima sea vendida o que aún esté en su
posesión.

\bibleverse{17} ``Cualquiera que desprecie a su padre o a su madre debe
ser ejecutado. \footnote{\textbf{21:17} Deut 27,16; Prov 20,20; Mat 15,4}

\bibleverse{18} ``Si los hombres están peleando y uno golpea al otro con
una piedra o con el puño, y el hombre herido no muere pero tiene que
permanecer en cama, \bibleverse{19} y luego se levanta y camina afuera
con su bastón, entonces el que lo golpeó no será castigado. Aún así,
debe compensar al hombre por el tiempo perdido de su trabajo y
asegurarse de que esté completamente curado.

\bibleverse{20} ``Cualquiera que golpee a su esclavo o esclava con una
vara, y el esclavo muera como resultado, debe ser castigado.
\bibleverse{21} Sin embargo, si después de un día o dos el esclavo
mejora, el dueño no será castigado porque el esclavo es de su propiedad.

\bibleverse{22} ``Si los hombres que están peleando golpean a una mujer
embarazada para que dé a luz prematuramente,\footnote{\textbf{21:22}
  ``Dé a luz prematuramente'': o, ``tiene un aborto espontáneo''.} pero
no se produce ninguna lesión grave, debe ser multado con la cantidad que
el marido de la mujer demande y según lo permitan los jueces.
\bibleverse{23} Pero si se produce una lesión grave, entonces debe pagar
una vida por otra vida; \bibleverse{24} ojo por ojo, diente por diente,
mano por mano, pie por pie, \bibleverse{25} quemadura por quemadura,
herida por herida y moretón por moretón.

\bibleverse{26} ``El que golpee a su esclavo o esclava en el ojo y lo
ciegue, debe liberar al esclavo como compensación por el ojo.
\bibleverse{27} El que golpee el diente de su esclavo o esclava debe
liberar al esclavo como compensación por el diente.

\hypertarget{indemnizaciuxf3n-por-muerte-o-lesiones-a-una-persona-por-animales}{%
\subsection{Indemnización por muerte o lesiones a una persona por
animales}\label{indemnizaciuxf3n-por-muerte-o-lesiones-a-una-persona-por-animales}}

\bibleverse{28} ``Si un buey usa sus cuernos para matar a un hombre o
una mujer, el buey debe ser apedreado hasta morir, y su carne no debe
ser comida. Pero el dueño del buey no será castigado. \bibleverse{29}
Pero si el buey ha herido repetidamente a la gente con sus cuernos, y su
dueño ha sido advertido pero aún no lo tiene bajo control, y mata a un
hombre o una mujer, entonces el buey debe ser apedreado hasta morir y su
dueño también debe ser ejecutado. \footnote{\textbf{21:29} Gén 9,5}
\bibleverse{30} Pero si en lugar de ello se exige el pago de una
indemnización, el propietario puede compensar su vida pagando la
totalidad de la indemnización exigida. \bibleverse{31} Pero si en lugar
de ello se exige el pago de una indemnización, el propietario puede
salvar su vida pagando la totalidad de la indemnización exigida.
\bibleverse{32} Si el buey usa sus cuernos y mata a un esclavo o
esclava, el propietario del buey debe pagar treinta siclos de plata al
amo del esclavo, y el buey debe ser apedreado hasta la muerte.

\hypertarget{disposiciones-para-la-protecciuxf3n-de-la-propiedad}{%
\subsection{Disposiciones para la protección de la
propiedad}\label{disposiciones-para-la-protecciuxf3n-de-la-propiedad}}

\bibleverse{33} ``Si alguien quita la tapa de una cisterna o cava una y
no la cubre, y un buey o un asno cae en ella, \bibleverse{34} el dueño
de la fosa debe pagar una compensación al dueño del animal y quedarse
con el animal muerto.

\bibleverse{35} ``Si el buey de alguien hiere al de otro y éste muere,
debe vender al buey vivo y compartir el dinero recibido; también debe
compartir al animal muerto. \bibleverse{36} Pero si se sabe que el buey
ha herido repetidamente a personas con sus cuernos, y su dueño ha sido
advertido pero aún no lo tiene bajo control, debe pagar una compensación
completa, buey por buey; pero el dueño puede quedarse con el animal
muerto''.

\hypertarget{section-21}{%
\section{22}\label{section-21}}

\bibleverse{1} ``Quien robe un buey o una oveja y la mate o la venda,
deberá devolver cinco bueyes por un buey y cuatro ovejas por una oveja.
\bibleverse{2} ``Si se descubre a un ladrón entrando en la casa de
alguien y es golpeado hasta la muerte, nadie será culpable de asesinato.
\bibleverse{3} Pero si ocurre durante el día, entonces alguien es
culpable de asesinato. El ladrón debe devolver todo lo robado. Si no
tiene nada, entonces debe ser vendido para pagar lo que fue robado.
\bibleverse{4} Si lo que fue robado es un animal vivo que todavía tiene,
ya sea un buey, un asno o una oveja, debe devolver el doble.

\bibleverse{5} ``Si el ganado pasta en un campo o en un viñedo y su
dueño lo deja vagar para que pasten en el campo de otro, el dueño debe
pagar una compensación con lo mejor de sus propios campos o viñedos.

\bibleverse{6} ``Si se inicia un incendio que se extiende a los arbustos
espinosos y luego quema el grano apilado o en pie, o incluso todo el
campo, la persona que inició el fuego debe pagar una compensación
completa.

\hypertarget{apropiaciuxf3n-indebida-puxe9rdida-o-dauxf1o-de-bienes-que-se-le-hayan-confiado}{%
\subsection{Apropiación indebida, pérdida o daño de bienes que se le
hayan
confiado}\label{apropiaciuxf3n-indebida-puxe9rdida-o-dauxf1o-de-bienes-que-se-le-hayan-confiado}}

\bibleverse{7} ``Si alguien le da a su vecino dinero o posesiones para
que las guarde y se las roban de la casa del vecino, si el ladrón es
atrapado debe pagar el doble. \bibleverse{8} Si el ladrón no es
atrapado, el propietario de la casa debe comparecer ante los jueces para
averiguar si se llevó la propiedad de su vecino. \bibleverse{9} ``Si hay
una discusión sobre la propiedad de un buey, un asno, una oveja, una
prenda de vestir, o cualquier cosa que se haya perdido y alguien dice:
`Esto es mío', ambas partes deben llevar su caso ante los jueces. Aquel
al que los jueces encuentren culpable debe devolverle el doble al otro.

\bibleverse{10} ``Si alguien pide a un vecino que cuide un asno, un
buey, una oveja o cualquier otro animal, pero éste muere o se lesiona o
es robado sin que nadie se dé cuenta, \bibleverse{11} entonces se debe
prestar un juramento ante el Señor para decidir si el vecino ha tomado
la propiedad del dueño. El propietario debe aceptar el juramento y no
exigir una compensación. \bibleverse{12} ``Sin embargo, si el animal fue
realmente robado al vecino, debe compensar al propietario. \footnote{\textbf{22:12}
  Gén 31,39} \bibleverse{13} Si fue matado y despedazado por un animal
salvaje, el vecino deberá presentar el cadáver como prueba y no necesita
pagar indemnización.

\bibleverse{14} ``Si alguien toma prestado un animal del vecino y éste
resulta herido o muere mientras su dueño no está presente, debe pagar
una indemnización en su totalidad. \bibleverse{15} Si el propietario
estaba presente, no se pagará ninguna compensación. Si el animal fue
alquilado, sólo se debe pagar el precio del alquiler.

\hypertarget{varias-normas-relativas-a-dios-la-moral-y-la-caridad}{%
\subsection{Varias normas relativas a Dios, la moral y la
caridad}\label{varias-normas-relativas-a-dios-la-moral-y-la-caridad}}

\bibleverse{16} ``Si un hombre seduce a una virgen no comprometida para
casarse y se acuesta con ella, debe pagar el precio completo de la novia
para que se convierta en su esposa. \bibleverse{17} Si el padre de ella
se niega rotundamente a dársela, el hombre debe pagar la misma cantidad
que el precio de la novia por una virgen. \footnote{\textbf{22:17} Lev
  20,6; Lev 20,27; Deut 18,10; 1Sam 28,9}

\bibleverse{18} ``No se debe permitir que viva una mujer que practique
la brujería. \footnote{\textbf{22:18} Lev 18,23; Deut 27,21}

\bibleverse{19} ``Todo aquel que tenga relaciones sexuales con un animal
debe ser ejecutado. \footnote{\textbf{22:19} Deut 13,7-19; Deut 17,2-7}

\bibleverse{20} ``Cualquiera que se sacrifique a cualquier otro dios que
no sea el Señor debe ser apartado y ejecutado.\footnote{\textbf{22:20}
  ``Apartado y ejecutado'': el término usado aquí significa ``dedicados
  a la destrucción'' en el sentido de que ahora sufrirán el castigo de
  Dios.} \footnote{\textbf{22:20} Éxod 23,9; Lev 19,33-34; Deut
  10,18-19; Deut 24,17-18; Deut 27,19}

\bibleverse{21} ``No se debe explotar o maltratar a un extranjero.
Recuerden que ustedes mismos fueron una vez extranjeros en Egipto.
\footnote{\textbf{22:21} Is 1,17}

\bibleverse{22} ``No se aprovechen de ninguna viuda o huérfano.
\bibleverse{23} Si los maltratan, y ellos me piden ayuda, responderé a
su clamor. \bibleverse{24} Me enfadaré y mataré a quien se aproveche de
ellos con espada. Entonces sus esposas se convertirán en viudas y sus
hijos quedarán huérfanos.

\bibleverse{25} ``Si le prestas dinero a mi pueblo porque son pobres, no
te comportes como un prestamista con ellos. No debes cobrarles ningún
interés. \footnote{\textbf{22:25} Deut 24,12-13} \bibleverse{26} ``Si
necesitas la capa de tu vecino como garantía de un préstamo, debes
devolvérsela antes de la puesta de sol, \bibleverse{27} porque es la
única ropa que tiene para su cuerpo. ¿De otro modo, con qué dormirá? Y
si me pide ayuda, le escucharé, porque soy misericordioso.

\bibleverse{28} ``No desprecies a Dios ni maldigas al líder de tu
pueblo. \footnote{\textbf{22:28} Deut 18,4; Éxod 13,2; Éxod 13,13}

\bibleverse{29} ``No retengas las ofrendas requeridas de tus productos,
aceite de oliva y vino.\footnote{\textbf{22:29} ``Aceite de oliva y
  vino'': Literalmente, ``mejores vendimias''.} Debes darme el
primogénito de tus hijos. \footnote{\textbf{22:29} Lev 22,27}

\bibleverse{30} También debes darme el primogénito de tus vacas, ovejas
y cabras. Podrás dejarlos con sus madres durante los primeros siete
días, pero debes darmelos al octavo día. \footnote{\textbf{22:30} Lev
  7,24; Lev 11,40; Lev 17,15; Lev 22,8; Deut 14,21; Ezeq 44,31}

\bibleverse{31} ``Ustedes deben ser un pueblo santo para mí. No coman
ningún cadáver de animal que encuentren en el campo y que haya sido
asesinado por animales salvajes. Láncenlos a los perros para que se lo
coman''.

\hypertarget{comportamiento-veraz-y-honesto-especialmente-en-la-corte}{%
\subsection{Comportamiento veraz y honesto, especialmente en la
corte}\label{comportamiento-veraz-y-honesto-especialmente-en-la-corte}}

\hypertarget{section-22}{%
\section{23}\label{section-22}}

\bibleverse{1} ``No ayudes a difundir historias que son mentiras. No
ayudes a la gente mala dando mal testimonio.

\bibleverse{2} ``No sigas a la multitud haciendo el mal. Cuando
testifiques en un juicio, no corrompas la justicia poniéndote del lado
de la mayoría. \bibleverse{3} Tampoco\footnote{\textbf{23:3}
  ``Tampoco'': añadido para mayor claridad. La justicia tiene que ser
  imparcial, así que mostrar favoritismo a cualquier parte está mal. Sin
  embargo, el problema más usual es la negación de la justicia a los
  pobres (ver por ejemplo el versículo 6).} muestres favoritismo hacia
los pobres en sus casos legales. \footnote{\textbf{23:3} Lev 19,15}

\bibleverse{4} ``Si te encuentras con el buey o asno de tu enemigo que
se ha extraviado, devuélveselo. \footnote{\textbf{23:4} Luc 6,27}
\bibleverse{5} Si ves el asno de alguien que te odia y que ha caído
porel peso de su carga, no lo dejes ahí. Debes detenerte y ayudarle.

\bibleverse{6} ``No debes impedir que los pobres obtengan justicia en
sus demandas. \footnote{\textbf{23:6} Deut 27,19}

\bibleverse{7} No tengas nada que ver con hacer falsas acusaciones. No
maten a los inocentes ni a los que hacen el bien, porque no dejaré que
los culpables queden impunes.

\bibleverse{8} ``No aceptes sobornos, porque un soborno ciega a los que
pueden ver y socava las pruebas de los honestos.

\bibleverse{9} ``No abusen de los extranjeros que viven entre ustedes,
pues ustedes saben muy bien lo que es ser extranjeros, ya que una vez
fueron extranjeros en Egipto. \footnote{\textbf{23:9} Éxod 22,20}

\hypertarget{disposiciones-para-los-auxf1os-sabuxe1ticos-las-fiestas-y-los-sacrificios}{%
\subsection{Disposiciones para los años sabáticos, las fiestas y los
sacrificios}\label{disposiciones-para-los-auxf1os-sabuxe1ticos-las-fiestas-y-los-sacrificios}}

\bibleverse{10} ``Seis años deben sembrar la tierra y cosechar los
cultivos, \footnote{\textbf{23:10} Lev 25,-1; Deut 15,1-11}
\bibleverse{11} pero en el séptimo año deben dejarla descansar y dejarla
sin cultivar, para que los pobres puedan comer lo que crece
naturalmente\footnote{\textbf{23:11} ``Lo que crece naturalmente'':
  añadido para mayor claridad.} en el campo y los animales salvajes
puedan terminar lo que queda. Sigan el mismo procedimiento para sus
viñedos y olivares.

\bibleverse{12} ``Tendrán seis días para hacer su trabajo, pero el
séptimo día deben dejar de trabajar, para que su buey y su asno puedan
descansar, y las familias de sus esclavos puedan recuperar el aliento,
así como los extranjeros que viven entre ustedes. \footnote{\textbf{23:12}
  Éxod 20,8-11}

\bibleverse{13} ``Asegúrate de prestar atención a todo lo que te he
dicho. Que no pase por tu mente invocar el nombre de otros dioses, ni
siquiera debes mencionarlos. \footnote{\textbf{23:13} Jos 23,7}

\bibleverse{14} ``Tres veces al año celebrarán una fiesta dedicada a mí.
\bibleverse{15} Deben observar el Festival de los Panes sin Levadura
como se los he instruído.\footnote{\textbf{23:15} Ver el capítulo 13.}
Deben comer pan sin levadura durante siete días en el momento apropiado
en el mes de Abib, porque ese fue el mes en que saliste de Egipto. Nadie
puede venir delante mí sin traer una ofrenda. \footnote{\textbf{23:15}
  Éxod 12,15} \bibleverse{16} ``Y también observarán el Festival de las
Cosechas cuando presenten las primicias de los productos de lo que hayan
sembrado en los campos. Por último, deben observar el Festival de la
Cosecha\footnote{\textbf{23:16} El nombre más familiar, dado más tarde,
  es el Festival de los tabernáculos.} al final del año, cuando recojan
la cosecha del resto de tus cultivos en el campo. \bibleverse{17} Todo
varón israelita debe presentarse ante el Señor Dios en estas tres
ocasiones cada año.

\bibleverse{18} ``No ofrecerán la sangre de mis sacrificios junto con
nada que contenga levadura, y la grasa de las ofrendas presentadas en mi
festival no debe dejarse hasta la mañana.

\bibleverse{19} ``Traigan las mejores primicias de sus cosechas a la
casa del Señor su Dios. No cocinarán a un cabrito en la leche de su
madre. \footnote{\textbf{23:19} Deut 26,1-11; Éxod 22,29; Deut 14,21}

\hypertarget{advertencia-final-sobre-la-expulsiuxf3n-de-los-cananeos-promesa-de-asistencia-y-bendiciones-en-caso-de-obediencia-fiel}{%
\subsection{Advertencia final sobre la expulsión de los cananeos;
Promesa de asistencia y bendiciones en caso de obediencia
fiel}\label{advertencia-final-sobre-la-expulsiuxf3n-de-los-cananeos-promesa-de-asistencia-y-bendiciones-en-caso-de-obediencia-fiel}}

\bibleverse{20} ``Yo envío un ángel delante de ti para que te proteja en
el camino y te lleve al lugar que te he preparado. \footnote{\textbf{23:20}
  Éxod 14,19} \bibleverse{21} Asegúrate de prestarle atención y hacer lo
que te diga. No te opongas a él, porque no perdonará la rebelión, pues
lleva mi autoridad.\footnote{\textbf{23:21} ``Lleva mi autoridad'':
  Literalmente, ``mi nombre está en medio de él''.} \footnote{\textbf{23:21}
  Is 63,9-10} \bibleverse{22} ``Sin embargo, si le escuchas atentamente
y haces todo lo que te digo, entonces seré enemigo de tus enemigos y
lucharé contra los que luchan contra ti. \bibleverse{23} Porque mi ángel
irá delante de ti y te llevará a la tierra de los amorreos, hititas,
ferezeos, cananeos, heveos y jebuseos, y los aniquilaré. \bibleverse{24}
No debes inclinarte ante sus dioses ni adorarlos, ni seguir sus
prácticas paganas. Más bien destruirás sus ídolos y derribarás sus
altares. \bibleverse{25} ``Adorarás al Señor tu Dios, y él bendecirá tu
comida y tu agua. Me aseguraré de que ninguno de ustedes se enferme.
\footnote{\textbf{23:25} Éxod 15,26} \bibleverse{26} Ninguna mujer
tendrá un aborto espontáneo ni se quedará sin hijos. Me aseguraré de que
vivan una larga vida. \bibleverse{27} ``Enviaré un terror sobre mí
delante de ustedes que hará que todas las naciones que los conozcan
entren en pánico. Haré que todos sus enemigos se den la vuelta y huyan.
\bibleverse{28} Y enviaré avispones\footnote{\textbf{23:28}
  ``Avispones'': El significado de la palabra utilizada aquí aún se
  debate. Algunos lo ven de manera similar al ``terror'' del verso
  anterior que causa pánico.} delante de ti para expulsar a los heveos,
cananeos e hititas. \bibleverse{29} No los expulsaré en un año, porque
la tierra se volvería desolada y tendrías que enfrentarte a un mayor
número de animales salvajes. \bibleverse{30} Poco a poco los expulsaré
delante de ti, hasta que haya suficientes para tomar posesión de la
tierra. \bibleverse{31} ``Fijaré sus fronteras desde el Mar Rojo hasta
el Mar de los Filisteos,\footnote{\textbf{23:31} ``Mar de los
  filisteos'': El mediterráneo.} y desde el desierto hasta el río
Éufrates. Te entregaré los habitantes de la tierra y tú los expulsarás.
\footnote{\textbf{23:31} Gén 15,18} \bibleverse{32} No debes hacer
ningún acuerdo con ellos ni con sus dioses. \footnote{\textbf{23:32}
  Éxod 34,12; Deut 7,2} \bibleverse{33} No se les debe permitir
permanecer en tu tierra, de lo contrario te llevarán a pecar contra mí.
Porque si adoras a sus dioses, definitivamente se convertirán en una
trampa para ti''.\footnote{\textbf{23:33} Jue 2,3}

\hypertarget{conclusiuxf3n-solemne-del-pacto-redacciuxf3n-y-lectura-del-libro-federal-el-sacrificio-del-pacto-y-la-aspersiuxf3n-de-sangre}{%
\subsection{Conclusión solemne del pacto; Redacción y lectura del libro
federal; el sacrificio del pacto y la aspersión de
sangre}\label{conclusiuxf3n-solemne-del-pacto-redacciuxf3n-y-lectura-del-libro-federal-el-sacrificio-del-pacto-y-la-aspersiuxf3n-de-sangre}}

\hypertarget{section-23}{%
\section{24}\label{section-23}}

\bibleverse{1} El Señor le dijo a Moisés: ``Subana la presencia del
Señor, tú y Aarón, Nadab y Abiú, y setenta de los ancianos de Israel.
Deben adorar a distancia. \footnote{\textbf{24:1} Núm 11,16}
\bibleverse{2} Sólo Moisés puede acercarse al Señor, los demás no deben
acercarse. El pueblo no puede subir al monte\footnote{\textbf{24:2} ``Al
  monte'': añadido para mayor claridad.} con él''.

\bibleverse{3} Moisés fue y le dijo al pueblo todas las instrucciones y
reglamentos del Señor. Todos respondieron juntos: ``¡Haremos todo lo que
el Señor diga!'' \footnote{\textbf{24:3} Éxod 19,8}

\bibleverse{4} Moisés escribió todo lo que el Señor había dicho. Se
levantó temprano a la mañana siguiente y construyó un altar al pie de la
montaña, y levantó doce pilares para cada una de las doce tribus de
Israel. \footnote{\textbf{24:4} Éxod 34,27; 1Re 18,31} \bibleverse{5}
Luego envió a algunos jóvenes israelitas que fueron y ofrecieron
holocaustos y sacrificaron toros jóvenes como ofrendas de paz al Señor.
\footnote{\textbf{24:5} Éxod 3,12} \bibleverse{6} Moisés puso la mitad
de la sangre en tazones y roció la otra mitad en el altar.
\bibleverse{7} Luego tomó el Libro del Acuerdo y se lo leyó al pueblo.
Ellos respondieron: ``Haremos todo lo que el Señor diga. Obedeceremos''.

\bibleverse{8} Entonces Moisés tomó la sangre, la roció sobre el pueblo
y dijo: ``Mira, esta es la sangre del pacto que el Señor ha hecho
contigo siguiendo estos términos''. \footnote{\textbf{24:8} Heb 9,19-22}

\hypertarget{los-setenta-ancianos-de-los-israelitas-en-el-sinauxed-ante-dios}{%
\subsection{Los setenta ancianos de los israelitas en el Sinaí ante
Dios}\label{los-setenta-ancianos-de-los-israelitas-en-el-sinauxed-ante-dios}}

\bibleverse{9} Entonces Moisés y Aarón, Nadab y Abiú, y setenta de los
ancianos de Israel subieron al monte, \bibleverse{10} y vieron al Dios
de Israel. Bajo sus pies había algo así como un pavimento de azulejos
hecho de lapislázuli, tan azul claro como el propio cielo.
\bibleverse{11} Pero Dios no hirió\footnote{\textbf{24:11} ``Hirió'':
  esto se debió a la expectativa de que cualquiera que viera a Dios
  moriría (Génesis 32:30; Jueces 6:22), respaldado por el mismo Dios
  (33:20)} a los líderes de Israel. Ellos lo vieron, y luego comieron y
bebieron una comida sagrada.\footnote{\textbf{24:11} ``Una comida
  sagrada'': añadido para mayor claridad.} \footnote{\textbf{24:11} Éxod
  33,20-23}

\hypertarget{moisuxe9s-permanece-en-el-sinauxed-durante-cuarenta-duxedas}{%
\subsection{Moisés permanece en el Sinaí durante cuarenta
días}\label{moisuxe9s-permanece-en-el-sinauxed-durante-cuarenta-duxedas}}

\bibleverse{12} Entonces el Señor le dijo a Moisés: ``Sube a mí al monte
y quédate aquí, para que te dé las tablas de piedra, con las
instrucciones y órdenes que he escrito para que las aprendan''.
\footnote{\textbf{24:12} Éxod 31,18}

\bibleverse{13} Así que Moisés se fue con Josué su ayudante y subió a la
montaña de Dios. \bibleverse{14} Les dijo a los ancianos: ``Quédense
aquí y esperen a que volvamos. Aarón y Hur están contigo. Si alguien
tiene un problema, puede hablar con ellos''.

\bibleverse{15} Cuando Moisés subió a la montaña, la nube la cubrió.
\bibleverse{16} La gloria del Señor descendió sobre el Monte Sinaí,
cubriéndolo durante seis días. En el séptimo día, el Señor llamó a
Moisés desde dentro de la nube. \footnote{\textbf{24:16} Éxod 16,10}
\bibleverse{17} Para los israelitas la gloria del Señor parecía un fuego
ardiente en la cima de la montaña. \footnote{\textbf{24:17} Deut 4,24;
  Deut 9,3; Heb 12,29} \bibleverse{18} Moisés subió a la nube cuando
subió a la montaña, y permaneció en la montaña durante cuarenta días y
noches.\footnote{\textbf{24:18} Éxod 34,28}

\hypertarget{regulaciones-sobre-la-construcciuxf3n-y-equipamiento-del-tabernuxe1culo}{%
\subsection{Regulaciones sobre la construcción y equipamiento del
tabernáculo}\label{regulaciones-sobre-la-construcciuxf3n-y-equipamiento-del-tabernuxe1culo}}

\hypertarget{section-24}{%
\section{25}\label{section-24}}

\bibleverse{1} Entonces el Señor le dijo a Moisés: \bibleverse{2}
``Ordena a los israelitas que me traigan una ofrenda. Recibirás mi
ofrenda de todos los que quieran darla. \bibleverse{3} ``Estos son los
artículos que debes aceptar de ellos como contribuciones: oro, plata y
bronce; \bibleverse{4} hilos azules, púrpura y carmesí; lino y pelo de
cabra finamente hilados; \bibleverse{5} pieles de carnero curtidas y
cuero fino; madera de acacia; \bibleverse{6} aceite de oliva para las
lámparas; especias para el aceite de oliva usado en la unción y para el
incienso fragante; \bibleverse{7} y piedras de ónix y otras gemas para
ser usadas en la fabricación del efod y el pectoral. \bibleverse{8} ``Me
harán un santuario para que pueda vivir entre ellos. \bibleverse{9}
Debes hacer el Tabernáculo\footnote{\textbf{25:9} La palabra
  ``Tabernáculo'' viene del latín para ``tienda de campaña'', y traduce
  el hebreo que se refiere a una morada, o lugar donde se habita.} y
todos sus muebles según el diseño que te voy a mostrar. \footnote{\textbf{25:9}
  Éxod 25,40}

\hypertarget{instrucciones-para-hacer-los-implementos-sagrados}{%
\subsection{Instrucciones para hacer los implementos
sagrados}\label{instrucciones-para-hacer-los-implementos-sagrados}}

\bibleverse{10} ``Deben hacer un Arca de madera de acacia que mida dos
codos y medio de largo por codo y medio de ancho por codo y medio de
alto. \bibleverse{11} Cúbranla con oro puro por dentro y por fuera, y
hagan un adorno de oro para rodearla. \bibleverse{12} Fundirán cuatro
anillos de oro y fijarlos a sus cuatro pies, dos en un lado y dos en el
otro. \bibleverse{13} Harán palos de madera de acacia y cubrirlos con
oro. \bibleverse{14} Colocarán las varas en los anillos de los lados del
Arca, para que pueda ser transportada. \bibleverse{15} Las varas deben
permanecer en los anillos del Arca; no las saques. \bibleverse{16}
Pongan dentro del Arca el testimonio que os voy a dar. \bibleverse{17}
``Harás una tapa de expiación\footnote{\textbf{25:17} ``Cubierta de
  expiación'': la palabra usada aquí significa ``cubrir'', en el sentido
  de tratar con los pecados. La traducción tradicional de
  ``propiciatorio'' se originó en Martín Lutero. Desde un punto de vista
  físico era la ``tapa'' del Arca.} de oro puro, de dos codos y medio de
largo por codo y medio de ancho. \footnote{\textbf{25:17} Heb 4,16}
\bibleverse{18} Haz dos querubines\footnote{\textbf{25:18} Una clase de
  ángel.} de oro forjado para los extremos de la cubierta de la
expiación, \bibleverse{19} y pon un querubín en cada extremo. Todo esto
debe ser hecho a partir de una sola pieza de oro. \bibleverse{20} Los
querubines deben ser diseñados con alas extendidas apuntando hacia
arriba, cubriendo la cubierta de expiación. Los querubines se colocarán
uno frente al otro, mirando hacia abajo, hacia la cubierta de expiación.
\bibleverse{21} Pondrán la cubiertade expiación encima del Arca, y
también podrán el testimonio que les daré dentro del Arca.
\bibleverse{22} Me reuniré contigo allí como está dispuesto sobre la
tapa de la expiación, entre los dos querubines que están de pie sobre el
Arca del Testimonio, y hablaré contigo sobre todas las órdenes que daré
a los israelitas. \footnote{\textbf{25:22} Núm 7,89}

\bibleverse{23} ``Entonces harás una mesa de madera de acacia de dos
codos de largo por un codo de ancho por un codo y medio de alto.
\bibleverse{24} Cúbrela con oro puro y haz un adorno de oro para
rodearla. \bibleverse{25} Haz un borde a su alrededor del ancho de una
mano y pon un ribete de oro en el borde. \bibleverse{26} Haz cuatro
anillos de oro para la mesa y sujétalos a las cuatro esquinas de la mesa
por las patas. \bibleverse{27} Los anillos deben estar cerca del borde
para sostener los palos usados para llevar la mesa. \bibleverse{28}
Haránlas varas de madera de acacia para llevar la mesa y las cubrirán
con oro. \bibleverse{29} Harán platos y fuentes para la mesa, así como
jarras y tazones para verter las ofrendas de bebida. Todos serán de oro
puro. \bibleverse{30} Pongan el Pan de la Presencia sobre la mesa para
que esté siempre en mi presencia.

\bibleverse{31} ``Haz un candelabro de oro puro, modelado con martillo.
Todo debe ser hecho de una sola pieza: su base, su eje, sus copas, sus
capullos y sus flores. \bibleverse{32} Debe tener seis ramas que salgan
de los lados del candelabro, tres en cada lado. \bibleverse{33} Tiene
tres tazas en forma de flores de almendra en la primera rama, cada una
con capullos y pétalos, tres en la siguiente rama. Cada una de las seis
ramas que salen tendrá tres tazas en forma de flores de almendra, todas
con brotes y pétalos. \bibleverse{34} ``En el eje principal del
candelabro se harán cuatro tazas en forma de flores de almendra, con
capullos y pétalos. \bibleverse{35} En las seis ramas que salen del
candelabro, colocarás un capullo bajo el primer par de ramas, un capullo
bajo el segundo par y un capullo bajo el tercer par. \bibleverse{36} Los
brotes y las ramas deben hacerse con el candelabro como una sola pieza,
modelada con martillo en oro puro. \bibleverse{37} Hagan siete lámparas
y colóquenlas en el candelabro para que iluminen el área que está
delante de él. \bibleverse{38} Las pinzas de la mecha y sus bandejas
deben ser de oro puro. \bibleverse{39} El candelabro y todos estos
utensilios requerirán un talento de oro puro. \bibleverse{40} Asegúrate
de hacer todo de acuerdo con el diseño que te mostré en la
montaña''.\footnote{\textbf{25:40} Éxod 26,30; Hech 7,44; Heb 8,5}

\hypertarget{instrucciones-para-hacer-el-apartamento-los-cuatro-techos.}{%
\subsection{Instrucciones para hacer el apartamento: Los cuatro
techos.}\label{instrucciones-para-hacer-el-apartamento-los-cuatro-techos.}}

\hypertarget{section-25}{%
\section{26}\label{section-25}}

\bibleverse{1} Harás diez cortinas para el Tabernáculo de lino finamente
hilado, usando hilos azules, púrpura y carmesí. Háganlas bordar con
querubines por alguien que sea hábil en el bordado. \bibleverse{2} Cada
cortina debe medir 28 codos de largo por 4 codos de ancho, y todas las
cortinas deben ser del mismo tamaño. \bibleverse{3} Junta cinco de las
cortinas y haz lo mismo con las otras cinco. \bibleverse{4} Usa material
azul para hacer lazos en el borde de la última cortina de ambos juegos.
\bibleverse{5} Haz cincuenta lazos en una cortina y cincuenta lazos en
la última cortina del segundo juego, alineando los lazos entre sí.
\bibleverse{6} Luego haz cincuenta ganchos de oro y une las cortinas con
los ganchos, para que el Tabernáculo sea una sola estructura.

\bibleverse{7} Haz once cortinas de pelo de cabra como una tienda de
campaña para cubrir el Tabernáculo. \bibleverse{8} Cada una de las once
cortinas debe ser del mismo tamaño: 30 codos de largo por 4 codos de
ancho. \bibleverse{9} Unirás cinco de las cortinas como un conjunto y
las otras seis como otro conjunto. Luego dobla la sexta cortina en dos
en la parte delantera de la tienda. \bibleverse{10} Haz cincuenta lazos
en el borde de la última cortina del primer juego, y cincuenta lazos a
lo largo del borde de la última cortina del segundo juego.
\bibleverse{11} Harás cincuenta ganchos de bronce y póngalos en los
lazos para unir la tienda como una sola cubierta. \bibleverse{12} La
media cortina extra de esta cubierta de la tienda se dejará colgada en
la parte trasera del Tabernáculo. \bibleverse{13} Las cortinas de la
tienda serán un codo más largas en cada lado, y la longitud extra
colgará sobre los lados del Tabernáculo para que quede todo cubierto.
\bibleverse{14} Harás una cubierta para la tienda con pelo de cabra y
pieles de carnero curtidas, y colocarás una cubierta extra de cuero fino
sobre ella.

\hypertarget{el-marco-de-madera-que-consta-de-tablas-y-cinco-barras}{%
\subsection{El marco de madera, que consta de tablas y cinco
barras}\label{el-marco-de-madera-que-consta-de-tablas-y-cinco-barras}}

\bibleverse{15} Hagan un marco vertical de madera de acacia para el
Tabernáculo. \bibleverse{16} Cada estructura debe tener diez codos de
largo por uno y medio de ancho. \bibleverse{17} Cada marco tendrá dos
clavijas para que los marcos puedan ser conectados entre sí. Hagan todos
los marcos del Tabernáculo así. \bibleverse{18} Haz veinte marcos para
el lado sur del Tabernáculo. \bibleverse{19} Haz cuarenta soportes de
plata como apoyo para los veinte marcos usando dos soportes por marco,
uno debajo de cada clavija del marco. \bibleverse{20} De manera similar
para el lado norte del Tabernáculo, harás veinte marcos \bibleverse{21}
y cuarenta soportes de plata, dos soportes por marco. \bibleverse{22}
Harás seis marcos para la parte trasera (lado oeste) del Tabernáculo,
\bibleverse{23} junto con dos marcos para sus dos esquinas traseras.
\bibleverse{24} Unirás estos marcos de las esquinas en la parte inferior
y en la parte superior cerca del primer anillo. Así es como debes hacer
los dos marcos de las esquinas. \bibleverse{25} En total habrá ocho
marcos y dieciséis soportes de plata, dos debajo de cada marco.

\bibleverse{26} Haz cinco barras transversales de madera de acacia para
unir los marcos del lado sur del Tabernáculo, \bibleverse{27} cinco para
los del norte y cinco para los de la parte trasera del Tabernáculo, al
oeste. \bibleverse{28} El travesaño central que se coloca a mitad de
camino de los marcos irá de un extremo al otro. \bibleverse{29} Cubrid
los marcos con oro, y haced anillos de oro para sujetar los travesaños
en su sitio. Cubrir los travesaños con oro también. \bibleverse{30}
Ensambla el Tabernáculo siguiendo el diseño que te mostré en la montaña.

\hypertarget{die-beiden-vorhuxe4nge-und-die-innere-ausstattung-der-wohnung}{%
\subsection{Die beiden Vorhänge und die innere Ausstattung der
Wohnung}\label{die-beiden-vorhuxe4nge-und-die-innere-ausstattung-der-wohnung}}

\bibleverse{31} Haz un velo de hilo azul, púrpura y carmesí, y de lino
finamente hilado, bordado con querubines por alguien que sea hábil en el
bordado. \footnote{\textbf{26:31} Mat 27,51} \bibleverse{32} Con ganchos
de oro, cuélgalo de cuatro postes de madera de acacia cubiertos de oro,
sostenidos por cuatro soportes de plata. \bibleverse{33} Coloca el velo
bajo el gancho y pon el Arca del Testimonio dentro, detrás del velo. El
velo separará el Lugar Santo del Lugar Santísimo. \bibleverse{34} Pon la
cubierta de expiación en el Arca del Testimonio en el Lugar Santísimo.
\footnote{\textbf{26:34} Éxod 25,21} \bibleverse{35} Pon la mesa fuera
del velo en el lado norte del Tabernáculo y pon el candelabro enfrente
en el lado sur. \footnote{\textbf{26:35} Éxod 40,22}

\bibleverse{36} Haz una pantalla para la entrada de la tienda usando
hilos azules, púrpura y carmesí, y lino finamente hilado y hazlo
bordado. \bibleverse{37} Haz cinco postes de madera de acacia con
ganchos de oro para colgar el biombo, y funde cinco soportes de bronce
para sujetarlos.

\hypertarget{instrucciones-sobre-el-altar-de-los-holocaustos-la-explanada-y-la-entrega-de-aceite-para-el-candelero}{%
\subsection{Instrucciones sobre el altar de los holocaustos, la
explanada y la entrega de aceite para el
candelero}\label{instrucciones-sobre-el-altar-de-los-holocaustos-la-explanada-y-la-entrega-de-aceite-para-el-candelero}}

\hypertarget{section-26}{%
\section{27}\label{section-26}}

\bibleverse{1} Haz un altar de madera de acacia. Debe ser cuadrado y
debe medir cinco codos de largo por cinco codos de ancho por tres codos
de alto. \bibleverse{2} Haráscuernos para cada una de sus esquinas,
todos de una sola pieza con el altar, y cubrirás todo el altar con
bronce. \bibleverse{3} Harás todos sus utensilios de bronce: cubos para
quitar las cenizas, palas, tazones para rociar, tenedores para la carne
y cacerolas. \bibleverse{4} Hagan una rejilla de malla de bronce para él
con un anillo de bronce en cada una de sus esquinas. \bibleverse{5}
Coloquen la rejilla bajo el saliente del altar, de modo que la malla
llegue hasta la mitad del altar. \bibleverse{6} Haz postes de madera de
acacia para el altar y cúbrelos con bronce. \bibleverse{7} Las varas
deben ser colocadas en los anillos para que las varas estén a cada lado
del altar cuando sea llevado. \bibleverse{8} Hagan el altar hueco,
usando tablas, tal como te lo mostré en la montaña. \footnote{\textbf{27:8}
  Éxod 26,30}

\bibleverse{9} Haz un patio para el Tabernáculo. Para el lado sur del
patio haz cortinas de lino finamente hilado, de cien codos de largo por
un lado, \bibleverse{10} con veinte postes y veinte soportes de bronce,
con ganchos y bandas de plata en los postes. \bibleverse{11} Del mismo
modo, en el lado norte se colocarán cortinas en una disposición
idéntica. \bibleverse{12} Las cortinas del lado oeste del patio tendrán
cincuenta codos de ancho, con diez postes y diez soportes.
\bibleverse{13} El lado este del patio que da al amanecer tendrá 50
codos de ancho. \bibleverse{14} Las cortinas de un lado deben tener
quince codos de largo, con tres postes y tres soportes, \bibleverse{15}
y las cortinas del otro lado deben ser iguales. \bibleverse{16} La
entrada al patio debe tener veinte codos de ancho, con una cortina
bordada con hilos azules, púrpura y carmesí, y lino finamente hilado,
sostenida por cuatro postes y cuatro soportes. \bibleverse{17} Todos los
postes alrededor del patio tendrán bandas de plata, ganchos de plata y
soportes de bronce. \bibleverse{18} Todo el patio tendrá cien codos de
largo y cincuenta de ancho, con cortinas de lino finamente hilado de
cinco codos de alto, y con soportes de bronce. \bibleverse{19} Todo el
resto del equipo usado en el Tabernáculo, incluyendo las estacas de la
tienda y las del patio, serán de bronce.

\bibleverse{20} Debes ordenar a los israelitas que te traigan aceite de
oliva puro, prensado a mano, para las lámparas, para que puedan seguir
encendidas, dando luz. \bibleverse{21} Enel Tabernáculo de Reunión,
fuera del velo delante del Testimonio, Aarón y sus hijos mantendrán las
lámparas encendidas en presencia del Señor desde la tarde hasta la
mañana. Este requisito debe ser observado por los israelitas durante
todas las generaciones.

\hypertarget{instrucciones-sobre-la-vestimenta-sacerdotal-de-aaruxf3n-y-sus-hijos}{%
\subsection{Instrucciones sobre la vestimenta sacerdotal de Aarón y sus
hijos}\label{instrucciones-sobre-la-vestimenta-sacerdotal-de-aaruxf3n-y-sus-hijos}}

\hypertarget{section-27}{%
\section{28}\label{section-27}}

\bibleverse{1} Haz que tu hermano Aarón venga a ti, junto con sus hijos
Nadab, Abihu, Eleazar e Itamar. Ellos, de todos los israelitas, me
servirán como sacerdotes. \footnote{\textbf{28:1} 1Cró 23,13; Éxod 6,23}
\bibleverse{2} Harás que se hagan ropas sagradas para tu hermano Aarón
para que se vea espléndido y digno. \bibleverse{3} Debes dar
instrucciones a todos los obreros hábiles, a los que han recibido de mí
sus habilidades, sobre cómo hacer la ropa para la dedicación de Aarón,
para que pueda servirme como sacerdote. \bibleverse{4} Estas son las
ropas que deben hacer: un pectoral, un efod, una túnica, una túnica
plisada, un turbante y una faja. Estos son los vestidos sagrados que
harán para tu hermano Aarón y sus hijos para que puedan servirme como
sacerdote. \bibleverse{5} Los trabajadores usarán hilo de oro, junto con
hilo azul, púrpura y carmesí, y lino finamente hilado.

\hypertarget{el-vestido-de-hombro-ephod}{%
\subsection{El vestido de hombro
(ephod)}\label{el-vestido-de-hombro-ephod}}

\bibleverse{6} Harán el efod de lino finamente tejido y bordado con oro,
y con hilos azules, púrpura y carmesí, hábilmente trabajado.
\bibleverse{7} Dos piezas de hombro deben ser unidas a las piezas
delanteras y traseras. \bibleverse{8} La cintura del efod será una pieza
hecha de la misma manera, usando hilo de oro, con hilo azul, púrpura y
carmesí, y con lino finamente tejido. \bibleverse{9} Escribe en dos
piedras de ónice los nombres de las tribus de Israel, \bibleverse{10}
seis nombres en una piedra, y seis en la otra, en orden de
nacimiento.\footnote{\textbf{28:10} ``En orden de nacimiento'':
  Literalmente, ``según su generación''.} \bibleverse{11} Escribe los
nombres en las dos piedras de la misma manera que un joyero graba un
sello personal. Luego coloque las piedras en un adorno de oro.
\bibleverse{12} Ata ambas piedras a las piezas del hombro del efod como
recordatorio para las tribus israelitas. Aarón debe llevar sus nombres
en sus dos hombros para recordar a los israelitas que los representa
cuando va a la presencia del Señor. \bibleverse{13} Hagan adornos de oro
\bibleverse{14} y dos cadenas trenzadas de oro puro, y sujetar estas
cadenas a los adornos.

\hypertarget{el-peto-con-accesorios}{%
\subsection{El peto con accesorios}\label{el-peto-con-accesorios}}

\bibleverse{15} También debe hacer un pectoral para las
decisiones\footnote{\textbf{28:15} ``Para las decisiones'': el pectoral
  debía sostener el Urim y el Tumim utilizados para determinar la
  voluntad del Señor y las decisiones sobre diferentes cuestiones (véase
  el versículo 30).} de la misma manera hábil que el efod, para ser
usado en la determinación de la voluntad del Señor. Háganlo usando hilo
de oro, con hilo azul, púrpura y carmesí, y con lino finamente tejido.
\bibleverse{16} Tiene que ser cuadrado cuando se pliega, midiendo
alrededor de nueve pulgadas\footnote{\textbf{28:16} ``Nueve pulgadas'':
  Literalmente, ``un espacio'', la distancia entre el pulgar y el dedo
  meñique cuando la mano está estirada.} de largo y ancho.
\bibleverse{17} Adjunta un arreglo de piedras preciosas en cuatro filas
como sigue:\footnote{\textbf{28:17} Ninguna de las siguientes piedras ha
  sido identificada con certeza.} En la primera fila cornalina, peridoto
y esmeralda. \bibleverse{18} En la segunda fila turquesa, lapislázuli y
sardónice. \bibleverse{19} En la tercera fila jacinto, ágata y amatista.
\bibleverse{20} En la cuarta fila topacio, berilo y jaspe. Coloca estas
piedras en los adornos de oro. \bibleverse{21} Cada una de las doce
piedras se grabará como un sello personal con el nombre de una de las
doce tribus israelitas y las representará. \bibleverse{22} Haz cordones
de cadenas trenzadas de oro puro para sujetar el pectoral.
\bibleverse{23} Harás dos anillos de oro y sujételos a las dos esquinas
superiores del pectoral. \bibleverse{24} Ata las dos cadenas de oro a
los dos anillos de oro de las esquinas del pectoral, \bibleverse{25} y
luego ata los extremos opuestos de las dos cadenas a los adornos de oro
de los hombros de la parte delantera del efod. \bibleverse{26} Haz dos
anillos de oro más y fíjelos a las dos esquinas inferiores del pectoral,
en el borde interior junto al efod. \bibleverse{27} Haz dos anillos de
oro más y póngalos en la parte inferior de las dos hombreras de la parte
delantera del efod, cerca de donde se une a su cintura tejida.
\bibleverse{28} Ata los anillos del pectoral a los anillos del efod con
un cordón de hilo azul, para que el pectoral no se suelte del efod.
\bibleverse{29} Así, cada vez que Aarón entre en el Lugar Santo, llevará
los nombres de las tribus israelitas sobre su corazón en el pectoral,
como un recordatorio constante ante el Señor. \bibleverse{30} Coloca el
Urim y Tumim en el pectoral de la decisión, para que ellos también estén
sobre el corazón de Aarón siempre que venga a la presencia del Señor.
Aarón llevará continuamente los medios de decisión sobre su corazón ante
el Señor. \footnote{\textbf{28:30} Lev 8,8; Núm 27,21; Deut 33,8}

\hypertarget{la-prenda-superior-para-el-vestido-de-hombros}{%
\subsection{La prenda superior para el vestido de
hombros}\label{la-prenda-superior-para-el-vestido-de-hombros}}

\bibleverse{31} Haz la túnica que va con el efod exclusivamente de tela
azul, \bibleverse{32} con una abertura en el medio en la parte superior.
Cose un cuello tejido alrededor de la abertura para fortalecerla y que
no se rompa. \bibleverse{33} Haz las granadas con los hilos azul,
púrpura y carmesí y pégalas alrededor de su dobladillo, con campanas de
oro entre ellas, \bibleverse{34} teniendo las campanas de oro y las
granadas alternadas. \bibleverse{35} Aarón debe llevar la túnica siempre
que sirva, y el sonido que haga se oirá cuando entre o salga del
santuario al entrar en la presencia del Señor, para que no muera.

\hypertarget{frente-ropa-interior-diadema-y-cinturuxf3n}{%
\subsection{Frente, ropa interior, diadema y
cinturón}\label{frente-ropa-interior-diadema-y-cinturuxf3n}}

\bibleverse{36} Haz una placa de oro puro y grabad en ella como un
sello, ``Consagradoal Señor''. \bibleverse{37} Pónganlo en la parte
delantera del turbante con un cordón azul. \bibleverse{38} Aarón lo
llevará en la frente, para que se responsabilice de la culpa de las
ofrendas que hagan los israelitas, y esto se aplica a todas sus santas
ofrendas. Debe permanecer siempre en su frente para que el pueblo sea
aceptado en la presencia del Señor. \bibleverse{39} Teje la túnica con
lino finamente hilado y haz el turbante del mismo material, y también
haz la faja y con bordado.

\hypertarget{la-ropa-de-los-hijos-de-aaruxf3n}{%
\subsection{La ropa de los hijos de
Aarón}\label{la-ropa-de-los-hijos-de-aaruxf3n}}

\bibleverse{40} Haz túnicas, fajas y tocados para los hijos de Aarón,
para que tengan un aspecto espléndido y digno. \bibleverse{41} Haz que
tu hermano Aarón y sus hijos vistan esta ropa y luego úngelos y
ordénalos. Dedícalos para que puedan servirme como sacerdotes.
\footnote{\textbf{28:41} Lev 8,12; Éxod 29,9; Éxod 29,24}

\bibleverse{42} Elabora calzoncillos de lino para cubrir sus cuerpos
desnudos, desde la cintura hasta el muslo. \bibleverse{43} Aarón y sus
hijos deben usarlos cuando entren a el Tabernáculo de Reunión o cuando
se acerquen al altar para servir en el Lugar Santo, para que no
seanhallados culpables y mueran. Esta es una ley para Aarón y sus
descendientes para siempre.

\hypertarget{instrucciuxf3n-para-la-ordenaciuxf3n-de-sacerdotes}{%
\subsection{Instrucción para la ordenación de
sacerdotes}\label{instrucciuxf3n-para-la-ordenaciuxf3n-de-sacerdotes}}

\hypertarget{section-28}{%
\section{29}\label{section-28}}

\bibleverse{1} Así es como debes proceder para dedicarlos y que me
sirvan como sacerdotes. Coge un novillo y dos carneros sin defectos.
\bibleverse{2} Luego, con la mejor harina de trigo, haced lo siguiente
sin levadura: pan, pasteles mezclados con aceite de oliva y barquillos
espolvoreados con aceite de oliva. \bibleverse{3} Ponlos todos en una
cesta y tráelos como ofrenda, junto con el toro y los dos carneros.
\bibleverse{4} Lleva a Aarón y a sus hijos a la entrada del Tabernáculo
de Reunión y lávalos con agua.\footnote{\textbf{29:4} Esta era una
  limpieza ceremonial, no era como la limpieza diaria.} \bibleverse{5}
Toma los vestidos y pónselos a Aarón: la túnica, el manto del efod, el
efod mismo y el pectoral. Ata el efod sobre él con su cinturón.
\bibleverse{6} Envuelve el turbante en la cabeza y ata la corona sagrada
al turbante. \footnote{\textbf{29:6} Éxod 28,36; Éxod 39,30}
\bibleverse{7} Luego usa el aceite de la unción para ungirlo,
vertiéndolo sobre su cabeza. \footnote{\textbf{29:7} Éxod 30,25}
\bibleverse{8} Luego que vengan sus hijos y les pongan las túnicas.
\bibleverse{9} Ata las fajas alrededor de Aarón y sus hijos y ponles los
tocados. El sacerdocio les pertenece para siempre. Así es como debes
ordenar a Aarón y a sus hijos. \footnote{\textbf{29:9} Éxod 28,41}

\bibleverse{10} Lleva el toro al frente del Tabernáculo de Reunión, y
Aarón y sus hijos deben poner sus manos sobre su cabeza. \bibleverse{11}
Luego mata el toro en presencia del Señor a la entrada del Tabernáculo
de Reunión. \bibleverse{12} Toma un poco de la sangre del toro y
úntasela con el dedo en los cuernos del altar. Luego vierte el resto de
la sangre en la base del altar. \bibleverse{13} Tomen toda la grasa que
cubre los intestinos, las mejores partes\footnote{\textbf{29:13}
  ``Mejores partes'': Se cree que se refiere al epiplón.} del hígado y
los dos riñones con su grasa, y quemadlos en el altar. \bibleverse{14}
Pero quema la carne del toro, su piel y sus excrementos fuera del
campamento, pues es una ofrenda por el pecado. \footnote{\textbf{29:14}
  Lev 4,11-12}

\bibleverse{15} A continuación, que Aarón y sus hijos pongan sus manos
en la cabeza de uno de los carneros. \bibleverse{16} Sacrifiquen el
carnero, tomen su sangre y salpicaalrededor del altar. \bibleverse{17}
Corta el carnero en pedazos, lava los intestinos y las piernas, y ponlos
con los otros pedazos y con la cabeza. \bibleverse{18} Luego quema todo
el carnero en el altar. Es una ofrenda quemada al Señor para ser
aceptada por él.

\bibleverse{19} Entonces haz que Aarón y sus hijos coloquen sus manos
sobre la cabeza del otro carnero. \bibleverse{20} Luegosacrifica el
carnero y pon un poco de su sangre en los lóbulos de las orejas derechas
de Aarón y sus hijos, en los pulgares de sus manos derechas y en los
dedos gordos de sus pies derechos. Salpica el resto de su sangre
alrededor del altar. \bibleverse{21} Toma un poco de la sangre del altar
y un poco del aceite de la unción y rociadlo sobre Aarón y sus ropas, y
sobre sus hijos y sus ropas. Entonces él y sus ropas serán sagradas, así
como sus hijos y sus ropas. \bibleverse{22} Toma la grasa del carnero,
incluyendo la grasa de su amplio rabo, la grasa que cubre los
intestinos, las mejores partes del hígado, los dos riñones con su grasa,
así como el muslo derecho (porque este es un carnero para la
ordenación). \bibleverse{23} Toma también una barra de pan, una torta de
pan hecha con aceite de oliva y una oblea de la cesta de pan hecho sin
levadura que está en la presencia del Señor. \bibleverse{24} Dáselos
todos a Aarón y a sus hijos para que los mezan\footnote{\textbf{29:24}
  Algunos estudiosos creen que en lugar de ``mecer'' la ofrenda ante el
  Señor, la elevaban hasta él. Sin embargo, esto parecería ser lo mismo
  que lo que tradicionalmente se llama la ofrenda ``levantada''.} ante
el Señor como ofrendamecida. \bibleverse{25} Luego toma los diferentes
panes y quémalos en el altar sobre el holocausto, para que sean
agradables para el Señor.

\bibleverse{26} Toma el pecho del carnero de la ordenación de Aarón y
mécelo ante el Señor como ofrenda mecida. Esta es la parte que puedes
guardar.\footnote{\textbf{29:26} De aquí en adelante esta porción estaba
  reservada para los sacerdotes.} \bibleverse{27} Separa para Aarón y
sus hijos el pecho de la ofrenda mecida y el muslo de la ofrenda mecida,
ambos tomados del carnero de la ordenación. \footnote{\textbf{29:27} Núm
  18,18} \bibleverse{28} De ahora en adelante, cuando los israelitas
levanten las ofrendas de paz al Señor, estas partes pertenecerán a Aarón
y a sus hijos para siempre como una parte regular de los israelitas.

\bibleverse{29} Las vestiduras sagradas que tiene Aarón serán
transmitidas a sus descendientes, para que las lleven cuando sean
ungidos y ordenados. \bibleverse{30} El descendiente que le suceda como
sacerdote y entre al Tabernáculo de Reunión para servir en el Lugar
Santo deberá llevarlas durante los siete días de su
ordenación.\footnote{\textbf{29:30} ``De su ordenación'': añadido para
  mayor claridad.}

\bibleverse{31} Toma el carnero de la ordenación y hierve su carne en un
lugar sagrado. \bibleverse{32} Aarón y sus hijos comerán la carne del
carnero y el pan que está en la cesta, a la entrada del Tabernáculo de
Reunión, \bibleverse{33} Comerán la carne y el pan que formaban parte de
las ofrendas que simbolizaban el perdón requerido\footnote{\textbf{29:33}
  ``Que simbolizaban el perdón requerido'': añadido para mayor claridad.
  La palabra hebrea es sencillamente ``cubrir sobre'', y se usa para
  describir perdón y reconciliación.} para su ordenación y dedicación.
Nadie más puede comerlos, porque son sagrados. \bibleverse{34} Si alguna
de las carnes de la ordenación o algún pan permanece hasta la mañana
siguiente, quemen lo que sobre. No debe ser comido, porque es sagrado.

\bibleverse{35} Este es el proceso que debes seguir para Aarón y sus
hijos, observando todas las instrucciones que les he dado. La ordenación
durará siete días.

\hypertarget{la-santificaciuxf3n-y-unciuxf3n-del-altar-del-holocausto}{%
\subsection{La santificación y unción del altar del
holocausto}\label{la-santificaciuxf3n-y-unciuxf3n-del-altar-del-holocausto}}

\bibleverse{36} Cada día debes sacrificar un toro como ofrenda para el
perdón por el pecado. Al hacer esto, el altar necesita ser purificado.
Úngelo para hacerlo sagrado. \bibleverse{37} Durante siete días
purificarás el altar y lo consagrarás. Entonces el altar se volverá
completamente santo, y todo lo que toque el altar se volverá santo.

\hypertarget{la-ofrenda-diaria-de-quema-bebida-y-comida-por-la-mauxf1ana-y-por-la-noche}{%
\subsection{La ofrenda diaria de quema, bebida y comida por la mañana y
por la
noche}\label{la-ofrenda-diaria-de-quema-bebida-y-comida-por-la-mauxf1ana-y-por-la-noche}}

\bibleverse{38} Ofrecerás dos corderos de un año en el altar, diaria y
continuamente. \bibleverse{39} Por la mañana ofrece un cordero y por la
tarde, antes de que oscurezca, ofrece el otro.\footnote{\textbf{29:39}
  ``Por la tarde, antes de que oscurezca'': Literalmente, ``entre las
  noches''.} \footnote{\textbf{29:39} Sal 141,2}

\bibleverse{40} Con el primer cordero ofrece también una décima parte de
una efa de harina de la mejor calidad, mezclada con un cuarto de hin de
aceite de oliva, y una libación de un cuarto de hin de vino.
\bibleverse{41} Entonces ofrece el segundo cordero por la tarde, con las
mismas ofrendas de grano y bebida que por la mañana, un holocausto al
Señor y aceptado por él. \bibleverse{42} Estos holocaustos se harán
continuamente por todas las generaciones a la entrada del Tabernáculo de
Reunión en presencia del Señor. Allí me reuniré para hablar con ustedes.
\bibleverse{43} Me reuniré con los israelitas allí, y ese lugar será
sagrado por mi gloria. \bibleverse{44} De esta manera dedicaré el
Tabernáculo de Reunión y el altar, y dedicaré a Aarón y sus hijos a
servirme como sacerdotes. \bibleverse{45} Entonces viviré con los
israelitas y seré su Dios. \bibleverse{46} Ellos sabrán que soy el Señor
su Dios, que los sacó de Egipto, para poder vivir con ellos. Yo soy el
Señor su Dios.

\hypertarget{regulaciones-sobre-el-altar-humeante}{%
\subsection{Regulaciones sobre el altar
humeante}\label{regulaciones-sobre-el-altar-humeante}}

\hypertarget{section-29}{%
\section{30}\label{section-29}}

\bibleverse{1} ``Haz un altar de madera de acacia\footnote{\textbf{30:1}
  Esta es una adición al altar que se menciona en el capítulo 27.} para
quemar incienso. \footnote{\textbf{30:1} Éxod 37,25-28} \bibleverse{2}
Será cuadrado, medirá un codo por codo, de dos codos de alto, con
cuernos en sus esquinas que son todos de una sola pieza con el altar.
\bibleverse{3} Cubre su parte superior, su lado y sus cuernos con oro
puro, y hace un adorno de oro para rodearlo. \bibleverse{4} Hagan dos
anillos de oro para el altar y pónganlos debajo de la moldura, dos a
ambos lados, para sostener las varas para llevarlo. \bibleverse{5} Haz
las varas de madera de acacia y cúbrelas con oro. \bibleverse{6} Pon el
altar delante del velo que cuelga delante del Arca del Testimonio y la
tapa de expiación que está sobre el Testimonio\footnote{\textbf{30:6}
  ``Testimonio'': se refiere a las tablas de piedra donde se escribieron
  los Diez Mandamientos.} donde me reuniré con ustedes. \bibleverse{7}
``Aarón debe quemar incienso fragante en el altar cada mañana cuando
cuida las lámparas. \footnote{\textbf{30:7} Sal 141,2; Apoc 5,8}
\bibleverse{8} Cuando enciendas las lámparas por la noche, se debe
quemar incienso de nuevo para que hay incienso siempre en la presencia
del Señor por las generaciones futuras. \bibleverse{9} No ofrezcas en
este altar ningún incienso no aprobado,\footnote{\textbf{30:9}
  ``Incienso no aprobado'': En otras palabras, incienso no preparado
  según las instrucciones dadas en los versículos 34-38.} ni ningún
holocausto ni ofrenda de grano, y no derrames sobre él ninguna libación.
\bibleverse{10} ``Una vez al año, Aarón debe realizar el ritual de
expiación poniendo en los cuernos del altar la sangre de la ofrenda por
el pecado para la expiación. Este ritual anual de expiación debe ser
llevado a cabo por las generaciones futuras. Este es el altar sagrado
del Señor''. \footnote{\textbf{30:10} Lev 16,18; Éxod 29,37}

\hypertarget{regulaciones-relativas-a-la-recaudaciuxf3n-de-un-impuesto-de-capitaciuxf3n-en-el-santuario-en-la-reuniuxf3n-del-pueblo}{%
\subsection{Regulaciones relativas a la recaudación de un impuesto de
capitación en el santuario en la reunión del
pueblo}\label{regulaciones-relativas-a-la-recaudaciuxf3n-de-un-impuesto-de-capitaciuxf3n-en-el-santuario-en-la-reuniuxf3n-del-pueblo}}

\bibleverse{11} El Señor le dijo a Moisés: \bibleverse{12} ``Cuando
hagas un censo de los israelitas, cada hombre debe pagarle al Señor el
rescate por su vida cuando sea contado. Así no sufrirán la plaga cuando
sean contados. \bibleverse{13} Cada uno que pase a esos condados debe
dar medio siclo, (usando el estandarte del siclo del santuario, que pesa
veinte geras). Este medio siclo es una ofrenda al Señor. \bibleverse{14}
Esta ofrenda al Señor se exige a todos los que tengan veinte años o más.
\bibleverse{15} Cuando ofrezcan esta ofrenda como rescatepor sus vidas,
los ricos no deben dar más de medio siclo y los pobres no deben dar
menos. \bibleverse{16} Tomen este dinero pagado por los israelitas y
úsenlo para los gastos de los servicios del Tabernáculo de Reunión.
Servirá como recordatorio para que los israelitas hagan expiación por
sus vidas en presencia del Señor''.

\hypertarget{normativa-sobre-el-fregadero-de-cobre-para-los-sacerdotes}{%
\subsection{Normativa sobre el fregadero de cobre para los
sacerdotes}\label{normativa-sobre-el-fregadero-de-cobre-para-los-sacerdotes}}

\bibleverse{17} Y el Señor le dijo a Moisés: \bibleverse{18} ``Haz una
palangana de bronce con un soporte de bronce para lavar. Colócalo entre
el Tabernáculo de Reunión y el altar, y pon agua en él. \bibleverse{19}
Aarón y sus hijos la usarán para lavarse las manos y los pies.
\bibleverse{20} Cada vez que entren en el Tabernáculo de Reunión, se
lavarán con agua para no morir. Cuando se acerquen al altar para
presentar los holocaustos al Señor, \bibleverse{21} también deben
lavarse para no morir. Este requisito debe ser observado por ellos y sus
descendientes por todas las generaciones''.

\hypertarget{preparaciuxf3n-y-uso-del-aceite-de-la-unciuxf3n-sagrada}{%
\subsection{Preparación y uso del aceite de la unción
sagrada}\label{preparaciuxf3n-y-uso-del-aceite-de-la-unciuxf3n-sagrada}}

\bibleverse{22} Entonces el Señor le dijo a Moisés: \bibleverse{23}
``Toma las especias de mejor calidad: 500 siclos de mirra líquida, 250
siclos de canela de olor dulce, 250 siclos de caña aromática,
\bibleverse{24} 500 siclos de casia, (pesos usando el estándar del siclo
del santuario), y un hin de aceite de oliva. \bibleverse{25} Mezcla todo
esto en el aceite de la unción sagrada, una mezcla aromática como el
producto de un experto perfumista. Úsalo como aceite de la unción
sagrada. \footnote{\textbf{30:25} Éxod 37,29} \bibleverse{26} Úsalo para
ungir el Tabernáculo de Reunión, el Arca del Testimonio, \bibleverse{27}
la mesa y todo su equipo, el candelabro y su equipo, el altar de
incienso, \bibleverse{28} el altar de los holocaustos y todos sus
utensilios, y la vasija más su soporte. \bibleverse{29} Dedícalos para
que sean especialmente santos. Todo lo que los toque será sagrado.
\bibleverse{30} ``Unjan a Aarón y a sus hijos también y dedíquenlos para
que sirvan como sacerdotes para mí. \footnote{\textbf{30:30} Éxod 29,7}
\bibleverse{31} Diles a los israelitas: `Este será mi aceite santo de
unción para todas las generaciones futuras. \bibleverse{32} No lo usen
en la gente común y no hagan nada parecido usando la misma fórmula. Es
santo, y debes tratarlo como si fuera santo. \bibleverse{33} Cualquiera
que mezcle aceite de unción como éste, o lo ponga sobre alguien que no
sea un sacerdote,\footnote{\textbf{30:33} ``Alguien que no sea
  sacerdote'': Literalmente, ``un extraño''.} será expulsado de su
pueblo'\,''.

\hypertarget{preparaciuxf3n-y-uso-del-incienso-sagrado}{%
\subsection{Preparación y uso del incienso
sagrado}\label{preparaciuxf3n-y-uso-del-incienso-sagrado}}

\bibleverse{34} El Señor le dijo a Moisés: ``Toma cantidades iguales de
estas especias aromáticas: resina de bálsamo, perfume, gálbano e
incienso puro. \bibleverse{35} Añade un poco de sal y haz incienso puro
y santo mezclado como el producto de un experto perfumista.
\bibleverse{36} Muele un poco en polvo y colóquelo delante del Arca del
Testimonio en el Terbenáculo de Reunión, donde me reuniré contigo. Será
especialmente sagrado para ti. \footnote{\textbf{30:36} Éxod 30,6}

\bibleverse{37} Nopreparen ningún incienso como éste usando la misma
fórmula. Deben considerar este incienso como sagrado para el Señor.
\bibleverse{38} Cualquiera que se haga un incienso como este para su
propio deleite será expulsado de su pueblo''.

\hypertarget{nombramiento-de-dos-capataces-y-sus-ayudantes}{%
\subsection{Nombramiento de dos capataces y sus
ayudantes}\label{nombramiento-de-dos-capataces-y-sus-ayudantes}}

\hypertarget{section-30}{%
\section{31}\label{section-30}}

\bibleverse{1} El Señor le dijo a Moisés: \bibleverse{2} ``He escogido
por nombre a Bezalel, hijo de Uri, hijo de Hur, de la tribu de Judá.
\bibleverse{3} Lo he llenado con el Espíritu de Dios dándole habilidad,
creatividad y experiencia en todo tipo de artesanías. \footnote{\textbf{31:3}
  1Re 7,14} \bibleverse{4} Puede producir diseños en oro, plata y
bronce, \bibleverse{5} puede tallar piedras preciosas para colocarlas en
los marcos, y puede tallar madera. Es un maestro de todas las artes.
\bibleverse{6} ``También he elegido a Aholiab, hijo de Ahisamac, de la
tribu de Dan, para que le ayude. También he dado a todos los artesanos
las habilidades necesarias para hacer todo lo que te he ordenado hacer:
\bibleverse{7} ``El Tabernáculo de Reunión, el Arca del Testimonio y su
tapa de expiación; y todos los demás muebles de la Tienda:
\bibleverse{8} la mesa con su equipamiento, el candelabro de oro puro
con todo su equipo, el altar de incienso, \bibleverse{9} el altar del
holocausto con todos sus utensilios, y la palangana más su soporte;
\bibleverse{10} así como las ropas tejidas tanto para Aarón el sacerdote
como para sus hijos para servir como sacerdotes, \bibleverse{11} así
como el aceite de unción y el incienso fragante para el Lugar Santo.
Deben hacerlos siguiendo todas las instrucciones que les he dado''.

\hypertarget{promulgaciuxf3n-del-mandamiento-del-suxe1bado}{%
\subsection{Promulgación del mandamiento del
sábado}\label{promulgaciuxf3n-del-mandamiento-del-suxe1bado}}

\bibleverse{12} El Señor le dijo a Moisés: \bibleverse{13} ``Dile a los
israelitas, `Es absolutamente esencial que guarden mis sábados. El
sábado será una señal entre ustedes y yo para las generaciones futuras,
para que sepan que yo soy el Señor que los santifica. \footnote{\textbf{31:13}
  Éxod 20,8} \bibleverse{14} Guardarán el sábado porque es santo para
ustedes. Cualquiera que lo deshonre debe ser asesinado. Cualquiera que
trabaje en ese día debe ser cortado de su pueblo. \footnote{\textbf{31:14}
  Núm 15,32-35} \bibleverse{15} Seis días podrán trabajar, pero el
séptimo día será un día de descanso, santo para el Señor. Cualquiera que
trabaje en el día de descanso debe ser asesinado. \bibleverse{16} Los
israelitas deben guardar el sábado, observando el sábado como un acuerdo
eterno para las generaciones futuras. \bibleverse{17} Es una señal entre
los israelitas y yo para siempre, porque el Señor hizo los cielos y la
tierra en seis días, pero en el séptimo día se detuvo y descansó'\,''.
\footnote{\textbf{31:17} Gén 2,2}

\hypertarget{moisuxe9s-recibe-las-tablas-de-la-ley}{%
\subsection{Moisés recibe las tablas de la
ley}\label{moisuxe9s-recibe-las-tablas-de-la-ley}}

\bibleverse{18} Cuando el Señor terminó de hablar con Moisés en el Monte
Sinaí, le dio las dos tablas del Testimonio, tablas de piedra escritas
por el dedo de Dios.\footnote{\textbf{31:18} Éxod 32,15-16; Éxod 34,28;
  Deut 4,13; Deut 5,19; Deut 9,10; Deut 10,4}

\hypertarget{hacer-y-adorar-la-imagen-dorada-del-toro}{%
\subsection{Hacer y adorar la imagen dorada del
toro}\label{hacer-y-adorar-la-imagen-dorada-del-toro}}

\hypertarget{section-31}{%
\section{32}\label{section-31}}

\bibleverse{1} Cuando el pueblo se dio cuenta de cuánto tiempo tardaba
Moisés en bajar de la montaña, fueron juntos a ver a Aarón. Le dijeron:
``¡Levántate! Haznos unos dioses que nos guíen porque este hombre,
Moisés, que nos sacó de la tierra de Egipto, no sabemos qué le ha
pasado''.

\bibleverse{2} ``Tráiganme los pendientes de oro que llevan sus esposas,
hijos e hijas'', respondió Aarón.

\bibleverse{3} Así que todos se quitaron los pendientes de oro que
llevaban puestos y se los llevaron a Aarón. \bibleverse{4} Él tomó lo
que le dieron y usando una herramienta moldeó un ídolo con forma de
becerro. Gritaron: ``Israel, estos son los dioses que te sacaron de la
tierra de Egipto''. \footnote{\textbf{32:4} Sal 106,19-20; 1Re 12,28;
  Hech 7,41}

\bibleverse{5} Cuando Aarón vio esto, edificó un altar frente al becerro
de oro y gritó: ``¡Mañana será una fiesta para honrar al Señor!''

\bibleverse{6} Al día siguiente, temprano, sacrificaron ofrendas
quemadas y presentaron ofrendas de paz. Luego se sentaron a celebrar con
comida y bebida. Luego se levantaron para bailar, y se convirtió en una
orgía.\footnote{\textbf{32:6} La palabra utilizada en este sentido, no
  era una especie de juego de fiestas. Los matices sexuales están claros
  por su uso en el Génesis 26:8 donde se refiere a las ``caricias'' de
  la intimidad entre Isaac y su esposa Rebeca. Tal resultado final de un
  festival que incluía la indulgencia en la comida y la bebida era
  habitual en las ceremonias paganas.}

\hypertarget{moisuxe9s-informado-por-dios-de-la-apostasuxeda-del-pueblo-desciende-del-monte}{%
\subsection{Moisés, informado por Dios de la apostasía del pueblo,
desciende del
monte}\label{moisuxe9s-informado-por-dios-de-la-apostasuxeda-del-pueblo-desciende-del-monte}}

\bibleverse{7} Entonces el Señor le dijo a Moisés, ``Baja, porque tu
pueblo, el que sacaste de Egipto está actuando inmoralmente.
\bibleverse{8} Han abandonado rápidamente el camino que les ordené
seguir. Se han hecho un ídolo de metal con forma de becerro,
inclinándose ante él en adoración y ofreciéndole sacrificios. Dicen:
`Estos son los dioses que los sacaron de la tierra de Egipto'\,''.
\footnote{\textbf{32:8} Éxod 20,4; Éxod 20,23; Éxod 32,4}

\bibleverse{9} ``Sé cómo es este pueblo'', continuó diciendo el Señor a
Moisés. ``¡Son tan rebeldes!\footnote{\textbf{32:9} ``Rebeldes'' o
  ``perversos'': la imagen es de un caballo siendo tirado por las
  riendas en una dirección pero deliberadamente yendo en la dirección
  opuesta. Esto significa más que simplemente ser obstinado, sino que
  trata de hacer lo opuesto.} \bibleverse{10} ¡Ahora déjame! Estoy
enfadado con ellos\ldots{} ¡Déjame acabar con ellos! Te convertiré en
una gran nación''.

\bibleverse{11} Pero Moisés suplicó al Señor su Dios, diciendo: ``¿Por
qué estás enojado con el pueblo que sacaste de la tierra de Egipto con
tremendo poder y gran fuerza? \bibleverse{12} ¿Por qué permitirás que
los egipcios digan `los sacó con el malvado propósito de matarlos en las
montañas, borrándolos de la faz de la tierra'? Apártate de tu feroz ira.
Por favor, arrepiéntete de esta amenaza contra tu pueblo.
\bibleverse{13} Recuerda que juraste una promesa a tus siervos Abraham,
Isaac y Jacob,\footnote{\textbf{32:13} ``Jacob'': literalmente,
  ``Israel''.} diciéndoles: `Haré que tu descendencia sea tan numerosa
como las estrellas del cielo, y te daré toda la tierra que les prometí,
y la poseerán para siempre'\,''. \footnote{\textbf{32:13} Gén 22,16-17;
  Gén 26,4; Gén 28,14}

\bibleverse{14} El Señor se arrepintió sobre el desastre que amenazó con
causar a su pueblo.

\bibleverse{15} Moisés se volvió y bajó del monte, llevando las dos
tablas de piedra de la Ley escritas a ambos lados. \bibleverse{16} Dios
había hecho las tablas, y Dios mismo había grabado la escritura.

\bibleverse{17} Cuando Josué escuchó todos los gritos del campamento, le
dijo a Moisés: ``¡Suena como una pelea en el campamento!''

\bibleverse{18} Pero Moisés respondió: ``Estos no son los gritos de la
victoria o de la derrota. ¡Lo que oigo es gente que está de fiesta!''

\hypertarget{el-celo-de-moisuxe9s-por-dios-castiga-al-pueblo-a-travuxe9s-de-los-levitas}{%
\subsection{El celo de Moisés por Dios; castiga al pueblo a través de
los
levitas}\label{el-celo-de-moisuxe9s-por-dios-castiga-al-pueblo-a-travuxe9s-de-los-levitas}}

\bibleverse{19} Al acercarse al campamento vio el ídolo del becerro y el
baile. Se enfadó tanto que tiró las tablas de piedra y las rompió allí
al pie de la montaña. \bibleverse{20} Tomó el becerro, lo quemó y lo
molió en polvo. Luego mezcló esto con agua e hizo que los israelitas la
bebieran.

\bibleverse{21} Entonces Moisés le preguntó a Aarón: ``¿Qué te hizo esta
gente para que los hicieras pecar tan mal?''

\bibleverse{22} ``Por favor, no te enfades conmigo, mi señor'',
respondió Aarón. ``Tú mismo sabes cuánto mal es capaz de hacer este
pueblo. \bibleverse{23} Me dijeron: `Haznos unos dioses que nos guíen
porque este hombre, Moisés, que nos sacó de la tierra de Egipto, no
sabemos qué le ha pasado'. \bibleverse{24} Entonces les dije: `El que
tenga joyas de oro, que se las quite y me las dé'. Eché el oro en el
horno y salió este becerro''.

\bibleverse{25} Moisés vio al pueblo enloqueciendo completamente porque
Aarón lo había permitido, y que esto les había traído el ridículo de sus
enemigos. \bibleverse{26} Así que fue y se paró a la entrada del
campamento, y gritó: ``¡Quien esté del lado del Señor, que venga y se
una a mí!'' Y todos los levitas se reunieron a su alrededor.

\bibleverse{27} Moisés les dijo: ``Esto es lo que dice el Señor, el Dios
de Israel: Cada uno amárrese su espada. Luego recorran todo el
campamento de un extremo a otro y maten a sus hermanos, amigos y
vecinos''. \bibleverse{28} Los levitas hicieron lo que Moisés les había
dicho, y ese día alrededor de 3. 000 hombres fueron asesinados.
\bibleverse{29} Moisés les dijo a los levitas: ``Hoy han sido dedicados
al Señor porque hanactuado contra sus hijos y hermanos. Hoy han ganado
una bendición para ustedes mismos''. \footnote{\textbf{32:29} Éxod
  28,41; Núm 3,6-10; Deut 33,8-11}

\hypertarget{intercesiuxf3n-de-moisuxe9s-por-el-pueblo-el-respiro-de-dios}{%
\subsection{Intercesión de Moisés por el pueblo; El respiro de
dios}\label{intercesiuxf3n-de-moisuxe9s-por-el-pueblo-el-respiro-de-dios}}

\bibleverse{30} Al día siguiente Moisés habló al pueblo diciendo: ``Han
pecado muy mal. Pero ahora subiré al Señor. Tal vez pueda conseguir que
perdone su pecado''.

\bibleverse{31} Así que Moisés volvió al Señor. Y dijo: ``Por favor, el
pueblo ha pecado muy mal al hacerse dioses de oro para sí mismos.
\bibleverse{32} Pero ahora, si quieres, perdona sus pecados. Si no,
bórrame del pergamino en el que guardas tus registros''.

\bibleverse{33} Pero el Señor respondió a Moisés: ``Los que pecaron
contra mí son los que serán borrados de mi pergamino. \bibleverse{34}
Ahora ve y conduce al pueblo al lugar del que te hablé. Mi ángel irá
delante de ti, pero en el momento en que decida castigarlos, los
castigaré por su pecado''. \footnote{\textbf{32:34} Éxod 33,2; Éxod
  33,12; Éxod 33,14}

\bibleverse{35} El Señor trajo una plaga sobre el pueblo porque hicieron
que Aarón hiciera el becerro.

\hypertarget{el-mandato-de-dios-de-ir-a-la-tierra-prometida-el-dolor-de-la-gente-por-el-rechazo-de-dios}{%
\subsection{El mandato de Dios de ir a la tierra prometida; El dolor de
la gente por el rechazo de
Dios}\label{el-mandato-de-dios-de-ir-a-la-tierra-prometida-el-dolor-de-la-gente-por-el-rechazo-de-dios}}

\hypertarget{section-32}{%
\section{33}\label{section-32}}

\bibleverse{1} Entonces el Señor le dijo a Moisés: ``Deja este lugar, tú
y el pueblo que sacaste de Egipto, y ve a la tierra que prometí con
juramento dar a Abraham, Isaac y Jacob, diciéndoles: `Daré esta tierra a
tu descendencia'. \bibleverse{2} Enviaré un ángel delante de ti y
expulsaré a los cananeos, amorreos, hititas, ferezeos, heveos y
jebuseos. \footnote{\textbf{33:2} Éxod 32,34} \bibleverse{3} Entra en
una tierra que fluye leche y miel, pero no te acompañaré porque eres un
pueblo rebelde. De lo contrario, te destruiría en el camino''.
\footnote{\textbf{33:3} Éxod 32,9-10}

\bibleverse{4} Cuando el pueblo escuchó estas palabras de crítica, se
pusieron de luto y no se pusieron sus joyas.

\bibleverse{5} Porque el Señor ya le había dicho a Moisés: ``Dile al
pueblo de Israel: `Tú eres un pueblo rebelde. Si estuviera contigo un
momento, te aniquilaría. Ahora quítate las joyas, y yo decidiré qué
hacer contigo'\,''.

\bibleverse{6} Así que los israelitas se quitaron las joyas desde que
dejaron el Monte Sinaí.\footnote{\textbf{33:6} ``Monte Sinaí'':
  Literalmente, ``Monte Horeb'', otro nombre para este mismo monte.}
\footnote{\textbf{33:6} Jon 3,6}

\hypertarget{la-fundaciuxf3n-y-uso-de-la-tienda-de-revelaciuxf3n-frente-al-campamento.}{%
\subsection{La fundación y uso de la tienda de revelación frente al
campamento.}\label{la-fundaciuxf3n-y-uso-de-la-tienda-de-revelaciuxf3n-frente-al-campamento.}}

\bibleverse{7} Moisés solía montar el Tabernáculo de Reunión en las
afueras del campamento. Cualquiera que quisiera preguntarle algo al
Señor podía ir a el Tabernáculo de Reunión. \footnote{\textbf{33:7} Éxod
  29,42} \bibleverse{8} Cada vez que Moisés salía a la tienda, todo el
pueblo iba y se paraba a la entrada de sus tiendas. Lo observaban hasta
que entraba. \bibleverse{9} Tan pronto como Moisés entraba en la tienda,
la columna de nubes descendía y se quedaba en la entrada mientras el
Señor hablaba con Moisés. \footnote{\textbf{33:9} Éxod 13,21}
\bibleverse{10} Cuandoel pueblo veía la columna de nubes de pie en la
puerta de la tienda, todos se levantaban y se inclinaban en adoración a
la entrada de sus tiendas. \bibleverse{11} Moisés hablaba con el Señor
cara a cara como si fuera un amigo, y luego regresaba al campamento. Sin
embargo, su joven ayudante Josué, hijo de Nun, se quedó en la Tienda.

\hypertarget{nuevas-negociaciones-entre-moisuxe9s-y-dios-sobre-la-direcciuxf3n-divina-adicional-del-pueblo}{%
\subsection{Nuevas negociaciones entre Moisés y Dios sobre la dirección
divina adicional del
pueblo}\label{nuevas-negociaciones-entre-moisuxe9s-y-dios-sobre-la-direcciuxf3n-divina-adicional-del-pueblo}}

\bibleverse{12} Moisés le dijo al Señor: ``Mira, me has estado diciendo:
`Ve y dirige a estepueblo', pero no me has hecho saber a quién vas a
enviar conmigo. Y sin embargo has declarado: `Te conozco personalmente,
x y estoy feliz contigo'. \footnote{\textbf{33:12} Éxod 33,2-5}
\bibleverse{13} Ahora bien, si es cierto que eres feliz conmigo, por
favor, enséñame tus caminos para que pueda conocerte y seguir
agradándote. Recuerda que la gente de esta nación es tuya''. \footnote{\textbf{33:13}
  Sal 103,7}

\bibleverse{14} El Señor respondió: ``Yo mismo iré contigo y te
apoyaré''.\footnote{\textbf{33:14} ``Te apoyaré'': Literalmente, ``te
  daré descanso''.}

\bibleverse{15} ``Si no vas con nosotros, por favor no nos saques de
aquí'', respondió Moisés. \bibleverse{16} ``¿Cómo sabrán los demás que
eres feliz conmigo y con tu pueblo si no nos acompañas? ¿Cómo podría
alguien separarnos a mí y a tu pueblo de todos los demás pueblos que
viven en la tierra?'' \footnote{\textbf{33:16} Deut 4,6-8}

\bibleverse{17} El Señor le dijo a Moisés: ``Prometo hacer lo que me
pidas, porque soy feliz contigo y te conozco personalmente''.
\footnote{\textbf{33:17} Éxod 33,12; 2Tim 2,19}

\hypertarget{dios-le-promete-a-moisuxe9s-que-veruxe1-su-gloria-como-una-muestra-de-gracia}{%
\subsection{Dios le promete a Moisés que verá su gloria como una muestra
de
gracia}\label{dios-le-promete-a-moisuxe9s-que-veruxe1-su-gloria-como-una-muestra-de-gracia}}

\bibleverse{18} ``Ahora, por favor, revélame tu gloria'', pidió Moisés.

\bibleverse{19} ``Haré pasar toda la bondad de mi carácter delante de
ti, gritaré el nombre `Yahvé',\footnote{\textbf{33:19} ``Yahvé'': esta
  es la palabra normalmente traducida como ``el Señor'', por lo que en
  los siguientes versos se observa que ``Yahvé'' y ``el Señor'' son lo
  mismo.} mostraré gracia a los que les quiero mostrar gracia, y
mostraré misericordia a los que les quiero mostrar misericordia.
\footnote{\textbf{33:19} Rom 9,15} \bibleverse{20} Pero no podrás ver mi
rostro, porque nadie puede ver mi rostro y vivir''. \footnote{\textbf{33:20}
  Gén 32,31; Is 6,5; 1Tim 6,16} \bibleverse{21} ``Ven aquí y quédate a
mi lado en esta roca'', continuó el Señor, \footnote{\textbf{33:21} 1Re
  19,8-13} \bibleverse{22} ``y a medida que pase mi gloria te pondré en
una grieta de la roca y te cubriré con mi mano hasta que haya pasado.
\footnote{\textbf{33:22} Éxod 34,5-6; Éxod 24,11}

\bibleverse{23} Entonces quitaré mi mano y verás mi espalda; pero no
verás mi cara''.

\hypertarget{por-orden-de-dios-moisuxe9s-sube-al-sinauxed-con-dos-tablas-de-piedra-en-blanco}{%
\subsection{Por orden de Dios, Moisés sube al Sinaí con dos tablas de
piedra en
blanco}\label{por-orden-de-dios-moisuxe9s-sube-al-sinauxed-con-dos-tablas-de-piedra-en-blanco}}

\hypertarget{section-33}{%
\section{34}\label{section-33}}

\bibleverse{1} El Señor le dijo a Moisés: ``Corta dos tablas de piedra
como las primeras, y escribiré en ellas de nuevo las mismas palabras que
estaban en las primeras tablas, las que tú rompiste. \footnote{\textbf{34:1}
  Éxod 32,19} \bibleverse{2} Prepárate por la mañana, y luego sube al
Monte Sinaí. Ponte delante de mí en la cima de la montaña.
\bibleverse{3} Nadie más puede subir contigo. No quiero ver a nadie en
ningún lugar de la montaña, y ningún rebaño o manada debe pastar al pie
de la montaña''.

\bibleverse{4} Entonces Moisés cortó dos tablas de piedra como las
anteriores y subió al monte Sinaí por la mañana temprano como el Señor
le había ordenado, llevando consigo las dos tablas de piedra.

\hypertarget{la-apariciuxf3n-de-dios-y-la-intercesiuxf3n-de-moisuxe9s}{%
\subsection{La aparición de Dios y la intercesión de
Moisés}\label{la-apariciuxf3n-de-dios-y-la-intercesiuxf3n-de-moisuxe9s}}

\bibleverse{5} El Señor descendió en una nube, se puso de pie con él, y
llamó el nombre ``Yahvé''. \footnote{\textbf{34:5} Éxod 33,19}
\bibleverse{6} El Señor pasó por delante de él, gritando: ``¡Yahvé!
Yahvé! ¡Soy el Dios de la gracia y la misericordia! Soy lento para
enojarme, lleno de amor eterno, siempre fiel. \footnote{\textbf{34:6}
  Núm 14,18; Sal 103,8; 1Jn 4,16} \bibleverse{7} Sigo mostrando mi amor
fiel a miles de personas, perdonando la culpa, la rebelión y el pecado.
Pero no dejaré a los culpables impunes, el impacto del pecado afectará
no sólo a los padres, sino también a sus hijos y nietos, hasta la
tercera y cuarta generación''. \footnote{\textbf{34:7} Éxod 20,5-6}

\bibleverse{8} Moisés se inclinó rápidamente hasta el suelo y adoró.
\bibleverse{9} Dijo: ``Señor, si es verdad que eres feliz conmigo, por
favor acompáñanos. Es cierto que este es un pueblo rebelde, pero por
favor perdona nuestra culpa y nuestro pecado. Acéptanos como algo que te
pertenece especialmente''.

\hypertarget{dios-consiente-la-renovaciuxf3n-de-la-relaciuxf3n-del-pacto-con-advertencias-serias}{%
\subsection{Dios consiente la renovación de la relación del pacto con
advertencias
serias}\label{dios-consiente-la-renovaciuxf3n-de-la-relaciuxf3n-del-pacto-con-advertencias-serias}}

\bibleverse{10} El Señor dijo: ``Verás que estoy haciendo un pacto
contigo. Frente a todos ustedes haré milagros que nunca se han hecho, ni
entre nadie en ningún lugar de la tierra. Todos aquí y los que están
alrededor verán al Señor trabajando, porque lo que voy a hacer por
ustedes será increíble. \bibleverse{11} Pero deben seguir cuidadosamente
lo que les digo que hagan hoy. ¡Presten atención! Voy a expulsar delante
de ustedes a los amorreos, cananeos, hititas, ferezeos, heveos y
jebuseos. \bibleverse{12} Asegúrense de no acordar un tratado de
paz\footnote{\textbf{34:12} ``Acordar tratado de paz'': La palabra es la
  misma que ``pacto'' con Dios en el versículo 10. También ``acuerdo''
  en el versículo 15.} con el pueblo que habite en la tierra a la que
van. De lo contrario, se convertirán en una trampa para ustedes.
\bibleverse{13} Porque deben derribar sus altares, derribar sus pilares
idólatras y cortar sus postes de Asera, \footnote{\textbf{34:13} Éxod
  23,24} \bibleverse{14} porque no debes adorar a ningún otro dios que
no sea el Señor. Su nombre significa exclusivo,\footnote{\textbf{34:14}
  ``Ser exclusivo'': Literalmente ``celoso''. Sin embargo, esto en
  términos humanos se asocia con la envidia y el resentimiento. Dios es
  ``celoso'' al querer ser el único Dios que es adorado.} porque es un
Dios que exige una relación exclusiva. \footnote{\textbf{34:14} Éxod
  20,3; Éxod 20,5}

\bibleverse{15} ``Asegúrense de no hacer un acuerdode paz con el pueblo
que habita enesa tierra, porque cuando se prostituyen adorando y
sacrificándose a sus dioses, los invitarán a unirse a ellos, y comerás
de sus sacrificios paganos. \bibleverse{16} Cuando hagas que sus hijas
se casen con tus hijos y esas hijas se prostituyan con sus dioses, harán
que tus hijos adoren a sus dioses de la misma manera. \footnote{\textbf{34:16}
  Deut 7,3; Jue 3,6; 1Re 11,2}

\hypertarget{las-nuevas-regulaciones-federales-sobre-el-debido-culto-a-dios}{%
\subsection{Las nuevas regulaciones federales sobre el debido culto a
Dios}\label{las-nuevas-regulaciones-federales-sobre-el-debido-culto-a-dios}}

\bibleverse{17} Nohagan ningún ídolo. \footnote{\textbf{34:17} Éxod
  20,23}

\bibleverse{18} ``Guardarán el Festival de los Panes sin Levadura.
Durante siete días comerán panes sin levadura, como se los he ordenado.
Lo harán en el momento indicado en el mes de Abib, porque ese fue el mes
en que salieron de Egipto. \footnote{\textbf{34:18} Éxod 23,14-19}

\bibleverse{19} Todo primogénito es mío. Eso incluye a todos los
primogénitos de su ganado, de sus manadas y rebaños. \bibleverse{20}
Pueden redimir el primogénito de un asnoa cambio de un cordero, pero si
no lo hacen, deberán romperle el cuello. Todos tus primogénitos deben
ser redimidos. Nadie debe presentarse ante mí sin una ofrenda.

\bibleverse{21} Trabajarás durante seis días, pero descansarás el
séptimo día. Incluso durante el tiempo de la siembra y la cosecha
descansarás.

\bibleverse{22} ``Guarden el Festival de las Semanas cuando ofrezcan las
primicias de la cosecha de trigo, y el Festival de la Cosecha al final
del año agrícola. \bibleverse{23} Tres veces al año todos tus varones
deben presentarse ante el Señor Yahvé, el Dios de Israel.
\bibleverse{24} Expulsaré las naciones que están delante de ti y
ampliaré tus fronteras, y nadie vendrá a tomar tu tierra cuando vayas
tres veces al año a presentarte ante el Señor tu Dios.

\bibleverse{25} No ofrezcas pan hecho con levadura cuando me presentes
un sacrificio, ni guardes ningún sacrificio de la fiesta de la Pascua
hasta la mañana siguiente.

\bibleverse{26} Cuandosiembrestus cosechas, llevalas primicias a la casa
del Señor tu Dios. ``No cocines un cabrito joven en la leche de su
madre''.

\hypertarget{mose-schreibt-die-bundesgebote-auf-gott-erneuert-die-gesetzestafeln}{%
\subsection{Mose schreibt die Bundesgebote auf; Gott erneuert die
Gesetzestafeln}\label{mose-schreibt-die-bundesgebote-auf-gott-erneuert-die-gesetzestafeln}}

\bibleverse{27} Entonces el Señor le dijo a Moisés: ``Escribe estas
palabras, porque son la base del acuerdo que he hecho contigo y con
Israel''. \footnote{\textbf{34:27} Éxod 24,4}

\bibleverse{28} Moisés pasó allí cuarenta días y cuarenta noches con el
Señor sin comer pan ni beber agua. Escribió en las tablas las palabras
del acuerdo, los Diez Mandamientos. \footnote{\textbf{34:28} Éxod 24,18;
  Mat 4,2; Éxod 31,18}

\hypertarget{el-descenso-de-moisuxe9s-el-brillo-de-la-piel-de-su-rostro}{%
\subsection{El descenso de Moisés; el brillo de la piel de su
rostro}\label{el-descenso-de-moisuxe9s-el-brillo-de-la-piel-de-su-rostro}}

\bibleverse{29} Cuando Moisés bajó del Monte Sinaí llevando las dos
tablas de la Ley, no se dio cuenta de que su rostro brillaba con fuerza
porque había estado hablando con el Señor. \bibleverse{30} Cuando Aarón
y los israelitas vieron a Moisés con su rostro tan brillante que se
asustaron al acercarse a él. \footnote{\textbf{34:30} 2Cor 3,7-18}

\bibleverse{31} Pero Moisés los llamó, así que Aarón y todos los líderes
de la comunidad se acercaron a él y él habló con ellos. \bibleverse{32}
Después todos los israelitas se acercaron y él les dio todas las
instrucciones del Señor que había recibido en el Monte Sinaí.
\bibleverse{33} Cuando Moisés terminó de hablar con ellos, se puso un
velo sobre su rostro. \bibleverse{34} Sin embargo, cada vez que Moisés
entraba a hablar con el Señor, se quitaba el velo hasta que volvía a
salir. Entonces les decía a los israelitas las instrucciones del Señor,
\bibleverse{35} y los israelitas veían su rostro brillar con fuerza. Así
que se ponía el velo en la cara hasta la próxima vez que fuera a hablar
con el Señor.

\hypertarget{comunicaciuxf3n-del-mandamiento-del-suxe1bado-invitaciuxf3n-a-contribuir-al-tabernuxe1culo}{%
\subsection{Comunicación del mandamiento del sábado; Invitación a
contribuir al
tabernáculo}\label{comunicaciuxf3n-del-mandamiento-del-suxe1bado-invitaciuxf3n-a-contribuir-al-tabernuxe1culo}}

\hypertarget{section-34}{%
\section{35}\label{section-34}}

\bibleverse{1} Moisés convocó a todos los israelitas y les dijo: ``Esto
es lo que el Señor nos ha ordenado hacer: \bibleverse{2} Seis días
pueden trabajar, pero el séptimo día debe ser un santo sábado de
descanso para el Señor. Cualquiera que haga cualquier trabajo en el día
de reposo debe ser asesinado. \footnote{\textbf{35:2} Éxod 20,8-11; Éxod
  31,12-17} \bibleverse{3} Noenciendan fuego en ninguna de sus casas en
el día de reposo''.

\bibleverse{4} Moisés también les dijo a todos los israelitas: ``Esto es
lo que el Señor ha ordenado: \bibleverse{5} Recojan una ofrenda al Señor
de lo que poseen. Todo el que quiera debe traer una ofrenda al Señor:
oro, plata y bronce; \bibleverse{6} hilos azules, púrpura y carmesí;
lino y pelo de cabra finamente tejidos; \bibleverse{7} pieles de carnero
curtidas y cuero fino; madera de acacia; \bibleverse{8} aceite de oliva
para las lámparas; especias para el aceite de la unción y para el
incienso aromático \bibleverse{9} y piedras de ónice y gemas para hacer
el efod y el pectoral.

\bibleverse{10} ``Todos tus artesanos vendrán a hacer todo lo que el
Señor ha ordenado: \bibleverse{11} el Tabernáculo con su tienda y su
cubierta, sus pinzas y sus marcos, sus travesaños, postes y soportes;
\footnote{\textbf{35:11} Éxod 31,7-11} \bibleverse{12} el Arca con sus
varas y su cubierta de expiación, y el velo para colgarla;
\bibleverse{13} la mesa con sus varas, todo su equipo y el Pan de la
Presencia; \bibleverse{14} el candelabro de luz con su equipo y lámparas
y aceite de oliva para alumbrar; \bibleverse{15} el altar de incienso
con sus varas; el aceite de la unción y el incienso aromático; la
pantalla para la entrada del Tabernáculo y todos sus accesorios;
\bibleverse{16} el altar del holocausto con su reja de bronce, sus varas
y todos sus utensilios; el lavabo más su soporte; \bibleverse{17} las
cortinas del patio con sus postes y bases, y la cortina para la entrada
del patio; \bibleverse{18} las estacas de la tienda para el Tabernáculo
y para el patio, así como sus cuerdas; \bibleverse{19} y las ropas
tejidas para servir en el lugar santo: la ropa sagrada para el sacerdote
Aarón y para sus hijos para servir como sacerdotes''.

\hypertarget{la-gente-muestra-su-disposiciuxf3n}{%
\subsection{La gente muestra su
disposición}\label{la-gente-muestra-su-disposiciuxf3n}}

\bibleverse{20} Los israelitas se fueron y dejaron a Moisés.
\bibleverse{21} Y todos aquellos que se sintieron movidos a hacerlo y
que tenían un espíritu dispuesto vinieron y trajeron una ofrenda al
Señor por el trabajo de hacer el Tabernáculo de Reunión, por todo lo que
se requería para sus servicios, y por las ropas sagradas.
\bibleverse{22} Así que todos los que quisieron, tanto hombres como
mujeres, vinieron y presentaron su oro como ofrenda de agradecimiento al
Señor, incluyendo broches, pendientes, anillos y collares, todo tipo de
joyas de oro. \bibleverse{23} Todos los que tenían hilos azules, púrpura
y carmesí, lino finamente tejido, pelo de cabra, pieles de carnero
curtidas y cuero fino, los trajeron. \bibleverse{24} Los que podían
presentar una ofrenda de plata o bronce la traían como regalo al Señor.
Todos los que tenían madera de acacia para cualquier parte del trabajo,
la donaban. \bibleverse{25} Toda mujer hábil en el hilado con sus manos
traía lo que había hilado: hilo azul, púrpura o carmesí, o lino
finamente tejido. \bibleverse{26} Todas las mujeres que estaban
dispuestas a usar sus habilidades hilaban el pelo de cabra.
\bibleverse{27} Los jefes trajeron piedras de ónix y gemas para hacer el
efod y el pectoral, \bibleverse{28} así como especias y aceite de oliva
para el alumbrado, para el aceite de la unción y para el incienso
aromático. \bibleverse{29} Todos los hombres y mujeres israelitas que
estaban dispuestos trajeron una ofrenda voluntaria al Señor por todo el
trabajo de hacer lo que el Señor, a través de Moisés, les había ordenado
hacer.

\hypertarget{nombramiento-de-los-capataces-y-demuxe1s-artesanos-abundantes-donaciones-y-servicios-voluntarios-por-parte-del-pueblo}{%
\subsection{Nombramiento de los capataces y demás artesanos; abundantes
donaciones y servicios voluntarios por parte del
pueblo}\label{nombramiento-de-los-capataces-y-demuxe1s-artesanos-abundantes-donaciones-y-servicios-voluntarios-por-parte-del-pueblo}}

\bibleverse{30} Entonces Moisés dijo a los israelitas: ``El Señor
escogió el nombre de Bezaleel, hijo de Uri, hijo de Hur, de la tribu de
Judá. \bibleverse{31} Lo ha llenado del Espíritu de Dios dándole
habilidad, creatividad y experiencia en todo tipo de artesanía.
\bibleverse{32} Puede producir diseños en oro, plata y bronce,
\bibleverse{33} puede tallar piedras preciosas para colocarlas en los
marcos, y puede tallar madera. Es un maestro de todas las artesanías.
\bibleverse{34} El Señor también le ha dado a él y a Aholiab, hijo de
Ahisamac, de la tribu de Dan, la habilidad de enseñar a otros.
\bibleverse{35} Los ha dotado de habilidad para hacer todo tipo de
trabajos como grabadores, diseñadores, bordadores en hilo azul, púrpura
y carmesí, y en lino finamente tejido, y como tejedores, de hecho como
hábiles diseñadores en todo tipo de artesanía''.

\hypertarget{section-35}{%
\section{36}\label{section-35}}

\bibleverse{1} ``Así que Bezalel, Aholiab, y todos los demás artesanos
con la experiencia necesaria y con la habilidad y la capacidad dadas por
el Señor, deben trabajar para llevar a cabo todo el trabajo de
construcción del santuario como lo ordenó el Señor''.

\bibleverse{2} Moisés convocó a Bezalel, a Aholiab y a todos los
artesanos a los que el Señor les había dado habilidades especiales, para
que vinieran a hacer el trabajo. \bibleverse{3} Moisés les dio todo lo
que los israelitas habían contribuido para llevar a cabo el trabajo de
construcción del santuario. Mientras tanto el pueblo siguió trayendo
ofrendas voluntarias cada mañana, \bibleverse{4} tanto que todos los
artesanos que trabajaban en el santuario dejaron lo que estaban haciendo
\bibleverse{5} y fueron a decirle a Moisés: ``El pueblo ya ha traído lo
suficiente para completar el trabajo que el Señor nos ha ordenado
hacer''.

\bibleverse{6} Moisés dio la orden, y se hizo un anuncio en todo el
campamento: ``Hombres y mujeres, no traigan nada más como ofrenda para
el santuario''. Así que se impidió que el pueblo trajera nada más,
\bibleverse{7} puesto que ya había más que suficiente para hacer todo el
trabajo necesario.

\hypertarget{la-fabricaciuxf3n-de-las-cuatro-cubiertas-de-la-tienda}{%
\subsection{La fabricación de las cuatro cubiertas de la
tienda}\label{la-fabricaciuxf3n-de-las-cuatro-cubiertas-de-la-tienda}}

\bibleverse{8} Los hábiles artesanos entre los trabajadores hicieron las
diez cortinas para el Tabernáculo. Estaban hechas de lino finamente
hilado junto con hilos azules, púrpura y carmesí, bordadas con
querubines. \footnote{\textbf{36:8} Éxod 26,1-14} \bibleverse{9} Cada
cortina tenía 28 codos de largo por 4 codos de ancho, y todas eran del
mismo tamaño. \bibleverse{10} Unieron cinco de las cortinas como un
conjunto, y las otras cinco las unió como un segundo conjunto.
\bibleverse{11} Utilizaron material azul para hacer lazos en el borde de
la última cortina de ambos juegos. \bibleverse{12} Hicieron cincuenta
lazos en una cortina y cincuenta lazos en la última cortina del segundo
juego, alineando los lazos entre sí. \bibleverse{13} También hicieron
cincuenta ganchos de oro y unieron las cortinas con los ganchos, de modo
que el Tabernáculo era una sola estructura.

\bibleverse{14} Hicieron once cortinas de pelo de cabra como una tienda
de campaña para cubrir el Tabernáculo. \bibleverse{15} Cada una de las
once cortinas era del mismo tamaño, 30 codos de largo por 4 codos de
ancho. \bibleverse{16} Unieron cinco de las cortinas como un conjunto y
las otras seis como otro conjunto. \bibleverse{17} Confeccionaron
cincuenta lazos en el borde de la última cortina del primer juego, y
cincuenta lazos a lo largo del borde de la última cortina del segundo
juego. \bibleverse{18} Hicieron cincuenta ganchos de bronce para unir la
tienda como una sola cubierta. \bibleverse{19} Elaboraron una cubierta
para la tienda con pelo de cabra y pieles de carnero curtidas, y
colocaron una cubierta extra de cuero fino sobre ella.

\hypertarget{fabricaciuxf3n-del-marco-de-madera}{%
\subsection{Fabricación del marco de
madera}\label{fabricaciuxf3n-del-marco-de-madera}}

\bibleverse{20} Hicieron un marco vertical de madera de acacia para el
Tabernáculo. \bibleverse{21} Cada marco tenía diez codos de largo por un
codo y medio de ancho. \bibleverse{22} Cada marco tenía dos clavijas
para que los marcos pudieran conectarse entre sí. Hicieron todos los
marcos del Tabernáculo así. \bibleverse{23} Hicieron veinte marcos para
el lado sur del Tabernáculo. \bibleverse{24} Hicieron cuarenta soportes
de plata como apoyo para los veinte marcos usando dos soportes por
marco, uno debajo de cada clavija del marco. \bibleverse{25} De manera
similar para el lado norte del Tabernáculo, hicieron veinte marcos
\bibleverse{26} y cuarenta soportes de plata, dos soportes por marco.
\bibleverse{27} Hicieron seis marcos para la parte trasera (lado oeste)
del Tabernáculo, \bibleverse{28} junto con dos marcos para sus dos
esquinas traseras. \bibleverse{29} Unieron estos marcos de las esquinas
en la parte inferior y en la parte superior cerca del primer anillo. Así
es como hicieron los dos marcos angulares. \bibleverse{30} En total
había ocho marcos y dieciséis soportes de plata, dos debajo de cada
marco.

\bibleverse{31} Fabicaron cinco barras transversales de madera de acacia
para sostener los marcos en el lado sur del Tabernáculo, \footnote{\textbf{36:31}
  Éxod 26,26-30} \bibleverse{32} cinco para los del norte y cinco para
los de la parte trasera del Tabernáculo, al oeste. \bibleverse{33}
Hicieron el travesaño central que se colocó a la mitad de los marcos y
corrió de un extremo al otro. \bibleverse{34} Cubrieron los marcos con
oro, e hicieron anillos de oro para sostener las barras transversales en
su lugar. También cubrieron los travesaños con oro.

\hypertarget{fabricaciuxf3n-de-las-dos-cortinas}{%
\subsection{Fabricación de las dos
cortinas}\label{fabricaciuxf3n-de-las-dos-cortinas}}

\bibleverse{35} Confeccionaron un velo de hilo azul, púrpura y carmesí,
y de lino finamente hilado, bordado con querubines por alguien que era
hábil en este arte. \bibleverse{36} Fabricaron cuatro postes de madera
de acacia para ello y los cubrieron con oro. Hicieron ganchos de oro
para los postes y fundieron sus cuatro soportes de plata.
\bibleverse{37} Hicieron un biombo para la entrada de la tienda usando
hilos azules, púrpura y carmesí, y lino finamente hilado, y lo hicieron
bordar. \bibleverse{38} También hicieron cinco postes de madera de
acacia con ganchos para colgar el biombo. Cubrieron la parte superior de
los postes y sus bandas con oro, y sus cinco soportes eran de bronce.

\hypertarget{el-cajuxf3n-con-la-placa-de-cubierta}{%
\subsection{El cajón con la placa de
cubierta}\label{el-cajuxf3n-con-la-placa-de-cubierta}}

\hypertarget{section-36}{%
\section{37}\label{section-36}}

\bibleverse{1} Bezalel hizo el Arca de madera de acacia que mide dos
codos y medio de largo por un codo y medio de ancho por un codo y medio
de alto. \footnote{\textbf{37:1} Éxod 25,10-22} \bibleverse{2} La cubrió
con oro puro por dentro y por fuera, e hizo un adorno de oro para
rodearla. \bibleverse{3} Fundió cuatro anillos de oro y los unió a sus
cuatro pies, dos en un lado y dos en el otro. \bibleverse{4} Hizo palos
de madera de acacia y los cubrió con oro. \bibleverse{5} Colocó las
varas en los anillos de los lados del Arca, para que pudiera ser
transportada. \bibleverse{6} Hizo la tapa de expiación de oro puro, de
dos codos y medio de largo por un codo y medio de ancho. \bibleverse{7}
Hizo dos querubines de oro martillado para los extremos de la tapa de
expiación, \bibleverse{8} y puso un querubín en cada extremo. Todo esto
fue hecho de una sola pieza de oro. \bibleverse{9} Los querubines fueron
diseñados con alas extendidas apuntando hacia arriba, cubriendo la
cubierta de expiación. Los querubines se colocaron uno frente al otro,
mirando hacia la cubierta de expiación.

\hypertarget{la-mesa-para-los-panes-de-la-proposiciuxf3n-y-las-libaciones}{%
\subsection{La mesa para los panes de la proposición y las
libaciones}\label{la-mesa-para-los-panes-de-la-proposiciuxf3n-y-las-libaciones}}

\bibleverse{10} Luego hizo la mesa de madera de acacia de dos codos de
largo por un codo de ancho por un codo y medio de alto. \bibleverse{11}
La cubrió con oro puro e hizo un adorno de oro para rodearla.
\bibleverse{12} Hizo un borde a su alrededor del ancho de una mano y
puso un adorno de oro en el borde. \bibleverse{13} Fundió cuatro anillos
de oro para la mesa y los sujetó a las cuatro esquinas de la mesa por
las patas. \bibleverse{14} Los anillos estaban cerca del borde para
sujetar los palos usados para llevar la mesa. \bibleverse{15} Fabricó
las varas de madera de acacia para llevar la mesa y las cubrió con oro.
\bibleverse{16} Elaboró utensilios para la mesa de oro puro: platos y
fuentes, tazones y jarras para verter las ofrendas de bebida.

\hypertarget{el-candelero-de-oro}{%
\subsection{El candelero de oro}\label{el-candelero-de-oro}}

\bibleverse{17} Hizo el candelabro de oro puro, martillado. Todo el
conjunto estaba hecho de una sola pieza: su base, el fuste, las tazas,
los capullos y las flores. \footnote{\textbf{37:17} Éxod 25,31-39}
\bibleverse{18} Tenía seis ramas que salían de los lados del candelabro,
tres en cada lado. Tenía tres tazas en forma de flores de almendra en la
primera rama, cada una con brotes y pétalos, tres en la siguiente rama.
\bibleverse{19} Cada una de las seis ramas que salían tenía tres copas
en forma de flores de almendra, todas con brotes y pétalos.
\bibleverse{20} En el eje principal del candelabro hizo cuatro tazas en
forma de flores de almendra, con capullos y pétalos. \bibleverse{21} En
las seis ramas que salían de él, colocó un brote bajo el primer par de
ramas, un brote bajo el segundo par, y un brote bajo el tercer par.
\bibleverse{22} Los brotes y las ramas deben ser hechos con el
candelabro como una sola pieza, martillado en oro puro. \bibleverse{23}
Hizo siete lámparas, así como pinzas de mecha y sus bandejas de oro
puro. \bibleverse{24} El candelabro y todos estos utensilios requerían
un talento de oro puro.

\hypertarget{el-altar-del-incienso}{%
\subsection{El altar del incienso}\label{el-altar-del-incienso}}

\bibleverse{25} Hizo el altar para quemar incienso de madera de acacia.
Era cuadrado, medía un codo por codo, por dos codo de alto, con cuernos
en sus esquinas que eran todos de una sola pieza con el altar.
\bibleverse{26} Cubrió su parte superior, su costado y sus cuernos con
oro puro, e hizo un adorno de oro para rodearlo. \bibleverse{27} Hizo
dos anillos de oro para el altar y los colocó debajo del adorno, dos a
ambos lados, para sostener los palos para llevarlo. \bibleverse{28} Hizo
las varas de madera de acacia y las cubrió con oro. \bibleverse{29} Hizo
el aceite de la santa unción y el incienso puro y aromático como el
producto de un experto perfumista.\footnote{\textbf{37:29} Éxod 30,25-35}

\hypertarget{el-altar-de-los-holocaustos.-la-cuenca-de-cobre}{%
\subsection{El altar de los holocaustos. La cuenca de
cobre}\label{el-altar-de-los-holocaustos.-la-cuenca-de-cobre}}

\hypertarget{section-37}{%
\section{38}\label{section-37}}

\bibleverse{1} Bezalelpresentó la ofrenda quemada en el altar hecho con
madera de acacia. Era cuadrado y medía cinco codos de largo por cinco de
ancho por tres de alto. \footnote{\textbf{38:1} Éxod 27,1-8}
\bibleverse{2} Hizo cuernos para cada una de sus esquinas, todos de una
sola pieza con el altar, y cubrió todo el altar con bronce.
\bibleverse{3} Elaboró todos sus utensilios: cubos para quitar las
cenizas, palas, tazones para rociar, tenedores para la carne y
cacerolas. Todos sus utensilios los hizo de bronce. \bibleverse{4}
Fabricóuna rejilla de malla de bronce para el altar y la colocó bajo el
saliente del altar, de modo que la malla llegara hasta la mitad del
altar. \bibleverse{5} Fundió cuatro anillos de bronce para las cuatro
esquinas de la rejilla como soportes para los postes. \bibleverse{6}
Elaboró postes de madera de acacia para el altar y los cubrió con
bronce. \bibleverse{7} Puso las varas a través de los anillos a cada
lado del altar para que pudiera ser transportado. Hizo el altar hueco,
usando tablas.

\bibleverse{8} Hizo la palangana de bronce con su soporte con bronce de
los espejos de las mujeres que servían en la entrada de la Carpa del
Encuentro. \footnote{\textbf{38:8} Éxod 30,18-21}

\hypertarget{el-atrio}{%
\subsection{El atrio}\label{el-atrio}}

\bibleverse{9} Luegoconstruyó un patio. Para el lado sur del patio hizo
cortinas de lino finamente hilado, de cien codos de largo por un lado,
\footnote{\textbf{38:9} Éxod 27,9-19} \bibleverse{10} con veinte postes
y veinte soportes de bronce, con ganchos y bandas de plata en los
postes. \bibleverse{11} De manera similar hizo cortinas colocadas en el
lado norte en una disposición idéntica. \bibleverse{12} Confeccionó
cortinas para el lado oeste del patio de cincuenta codos de ancho, con
diez postes y diez soportes. \bibleverse{13} El lado este del patio que
da al amanecer tenía cincuenta codos de ancho. \bibleverse{14} Diseñó
las cortinas de un lado de quince codos de largo, con tres postes y tres
soportes, \bibleverse{15} y las cortinas del otro lado de la misma
manera. \bibleverse{16} Todas las cortinas que rodeaban el patio eran de
lino finamente tejido. \bibleverse{17} Las gradas de los postes eran de
bronce, los ganchos y las bandas eran de plata, y la parte superior de
los postes estaba cubierta de plata. Todos los postes alrededor del
patio tenían bandas de plata. \bibleverse{18} La cortina de la entrada
al patio estaba bordada con hilos azules, púrpura y carmesí, y con lino
finamente hilado. Tenía 20 codos de largo por 5 codos de alto, la misma
altura que las cortinas del patio. \bibleverse{19} Estaba sostenido por
cuatro postes y cuatro soportes. Los postes tenían ganchos, tapas y
bandas de plata. \bibleverse{20} Todas las estacas de la tienda para el
Tabernáculo y para el patio circundante eran de bronce.

\hypertarget{el-cuxe1lculo-del-costo-de-los-metales-utilizados-para-el-santuario}{%
\subsection{El cálculo del costo de los metales utilizados para el
santuario}\label{el-cuxe1lculo-del-costo-de-los-metales-utilizados-para-el-santuario}}

\bibleverse{21} Lo siguiente es lo que se usó para el Tabernáculo del
Testimonio, registrado bajo la dirección de Moisés por los levitas bajo
la supervisión de Itamar, hijo del sacerdote Aarón. \footnote{\textbf{38:21}
  Núm 4,28} \bibleverse{22} Bezaleel, hijo de Uri, hijo de Hur, de la
tribu de Judá, hizo todo lo que el Señor había ordenado a Moisés.
\footnote{\textbf{38:22} Éxod 31,1-11} \bibleverse{23} Fue asistido por
Aholiab, hijo de Ahisamac, de la tribu de Dan, un grabador, diseñador y
bordador que usaba hilos azules, púrpura y carmesí y lino finamente
tejido.

\bibleverse{24} La cantidad total de oro de la ofrenda que se utilizó
para el trabajo en el santuario fue de 29 talentos y 730 siclos, (usando
el estándar de siclos del santuario). \footnote{\textbf{38:24} Éxod
  30,13}

\bibleverse{25} La cantidad total de plata de los que habían sido
contados en el censo era de 100 talentos y 1. 775 siclos (usando el
estándar del siclo del santuario). \bibleverse{26} Esto representa un
beka por persona, o medio siclo, (usando el estándar del siclo del
santuario) de cada persona de veinte años o más que había sido censada,
un total de 603. 550 hombres. \bibleverse{27} Los cien talentos de plata
se usaron para fundir los soportes del santuario y los soportes de las
cortinas, 100 bases de los 100 talentos, o un talento por base.
\bibleverse{28} Bezalel usó los 1. 775 siclos de plata para hacer los
ganchos de los postes, cubrir sus tapas y hacer bandas para ellos.
\bibleverse{29} La cantidad total de bronce de la ofrenda fue de 70
talentos y 2. 400 siclos. \bibleverse{30} Bezalel lo usó para hacer las
gradas para la entrada a el Tabernáculo de Reunión, el altar de bronce y
su rejilla de bronce, todos los utensilios para el altar,
\bibleverse{31} las gradas para el patio y su entrada, y todas las
estacas de la tienda para el Tabernáculo y el patio.

\hypertarget{confecciuxf3n-de-la-ropa-sacerdotal}{%
\subsection{Confección de la ropa
sacerdotal}\label{confecciuxf3n-de-la-ropa-sacerdotal}}

\hypertarget{section-38}{%
\section{39}\label{section-38}}

\bibleverse{1} Estos hombres\footnote{\textbf{39:1} Refiriéndose a los
  artesanos.} confeccionaron ropa tejida con hilos azules, púrpura y
carmesí para servir en el santuario. También hicieron vestimentas
sagradas para Aarón, como el Señor le había ordenado a Moisés.

\hypertarget{el-vestido-de-hombro-ephod-1}{%
\subsection{El vestido de hombro
(ephod)}\label{el-vestido-de-hombro-ephod-1}}

\bibleverse{2} Hicieron el efod de lino finamente tejido bordado con
oro, y con hilos azules, púrpura y carmesí. \bibleverse{3} Martillaron
finas láminas de oro y cortaron hilos para tejerlos con los hilos azul,
púrpura y escarlata, junto con lino fino, todo hábilmente trabajado.
\bibleverse{4} Dos piezas de hombro fueron unidas a las piezas
delanteras y traseras \bibleverse{5} La cintura del efod era una pieza
hecha de la misma manera, usando hilo de oro, con hilo azul, púrpura y
carmesí, y con lino fino, como el Señor había ordenado a Moisés.

\bibleverse{6} Colocaron las piedras de ónix en engastes de oro
ornamental, grabando los nombres de las tribus israelitas de la misma
manera que un joyero graba un sello personal. \bibleverse{7} Pusieron
ambas piedras en los hombros del efod como recordatorio para las tribus
israelitas, como el Señor le había ordenado a Moisés.

\hypertarget{el-adorno-del-pecho}{%
\subsection{El adorno del pecho}\label{el-adorno-del-pecho}}

\bibleverse{8} También hicieron un pectoral para las decisiones de la
misma manera hábil que el efod, para ser usado en la determinación de la
voluntad del Señor. Lo hicieron usando hilo de oro, con hilos azules,
púrpura y carmesí, y con lino finamente tejido. \bibleverse{9} Era
cuadrado cuando se doblaba, midiendo alrededor de nueve
pulgadas\footnote{\textbf{39:9} ``Nueve pulgadas'': Literalmente, ``un
  espacio'', la distancia entre el pulgar y el dedo meñique cuando la
  mano está estirada.} de largo y ancho. \bibleverse{10} Adjuntaron un
arreglo de piedras preciosas en cuatro filas como sigue.\footnote{\textbf{39:10}
  Ninguna de las siguientes piedras ha sido identificada con certeza.}
En la primera fila cornalina, peridoto y esmeralda. \bibleverse{11} En
la segunda fila turquesa, lapislázuli y sardónice. \bibleverse{12} En la
tercera fila jacinto, ágata y amatista. \bibleverse{13} En la cuarta
fila topacio, berilo y jaspe. Todos ellos fueron colocados en un marco
de oro ornamental. \bibleverse{14} Cada una de las doce piedras estaba
grabada como un sello personal con el nombre de una de las doce tribus
israelitas y las representaba. \bibleverse{15} Confeccionaron cordones
de cadenas trenzadas de oro puro para sujetar el pectoral.
\bibleverse{16} Hicieron dos ajustes de oro y dos anillos de oro y
sujetaron los anillos a las dos esquinas superiores del pectoral.
\bibleverse{17} Fijaron las dos cadenas de oro a los dos anillos de oro
de las esquinas del pectoral, \bibleverse{18} y luego sujetaron los
extremos opuestos de las dos cadenas a los adornos de oro de los hombros
de la parte delantera del efod. \bibleverse{19} Hicieron dos anillos de
oro más y los fijaron a las dos esquinas inferiores del pectoral, en el
borde interior junto al efod. \bibleverse{20} Hicieron dos anillos de
oro más y los fijaron en la parte inferior de las dos hombreras de la
parte delantera del efod, cerca de donde se une a su cintura tejida.
\bibleverse{21} Ataron los anillos del pectoral a los anillos del efod
con un cordón de hilo azul, para que el pectoral no se soltara del efod,
como el Señor había ordenado a Moisés.

\hypertarget{la-prenda-superior-para-el-vestido-de-hombros-1}{%
\subsection{La prenda superior para el vestido de
hombros}\label{la-prenda-superior-para-el-vestido-de-hombros-1}}

\bibleverse{22} Hicieron la túnica que acompaña al efod exclusivamente
de tela azul tejida, \footnote{\textbf{39:22} Éxod 28,31-35}
\bibleverse{23} con una abertura en el centro en la parte superior.
Cosieron un cuello tejido alrededor de la abertura para reforzarla y que
no se rompiera. \bibleverse{24} Hicieron granadas usando hilos azules,
púrpura y carmesí y lino finamente tejido y las unieron alrededor de su
dobladillo. \bibleverse{25} Hicieron campanas de oro puro y las unieron
entre las granadas alrededor de su dobladillo, \bibleverse{26} haciendo
que las campanas y las granadas se alternaran. La túnica debía ser usada
para el servicio sacerdotal, como el Señor le había ordenado a Moisés.

\hypertarget{la-ropa-oficial-restante-de-los-sacerdotes.}{%
\subsection{La ropa oficial restante de los
sacerdotes.}\label{la-ropa-oficial-restante-de-los-sacerdotes.}}

\bibleverse{27} Confeccionaron túnicas con lino finamente hilado hechas
por un tejedor para Aarón y sus hijos. \bibleverse{28} Tambiénelaboraron
turbantes, tocados y diademas de lino fino, y calzoncillos de lino
finamente tejidos, \bibleverse{29} así como fajas de lino finamente
tejidas bordadas con hilos azules, púrpura y carmesí, como el Señor
había ordenado a Moisés.

\hypertarget{el-rostro-del-sumo-sacerdote}{%
\subsection{El rostro del sumo
sacerdote}\label{el-rostro-del-sumo-sacerdote}}

\bibleverse{30} Diseñaron la placa de la corona santa de oro puro y
escribieron en ella, grabada como un sello, ``Consagrado al Señor''.
\footnote{\textbf{39:30} Éxod 29,6; Lev 8,9} \bibleverse{31} Le ataron
un cordón azul para atarlo a la parte delantera del turbante, como el
Señor le había ordenado a Moisés. \footnote{\textbf{39:31} Éxod 28,36-38}

\hypertarget{entrega-de-los-artuxedculos-terminados-a-moisuxe9s}{%
\subsection{Entrega de los artículos terminados a
Moisés}\label{entrega-de-los-artuxedculos-terminados-a-moisuxe9s}}

\bibleverse{32} Así que todo el trabajo para el Tabernáculo, el
Tarbernáculo de Reunión, estaba terminado. Los israelitas hicieron todo
lo que el Señor le había ordenado a Moisés. \bibleverse{33} Luego
presentaron el Tabernáculo a Moisés: la tienda con todos sus muebles,
sus pinzas, sus marcos, sus travesaños y sus postes y soportes;
\bibleverse{34} la cubierta de pieles de carnero curtidas, la cubierta
de cuero fino y el velo; \bibleverse{35} el Arca del Testimonio con sus
varas y la cubierta de expiación; \bibleverse{36} la mesa con todos sus
equipos y el Pan de la Presencia; \bibleverse{37} el candelabro de oro
puro con sus lámparas puestas en fila, y todos sus equipos, así como el
aceite de oliva para las lámparas; \bibleverse{38} el altar de oro, el
aceite de la unción, el incienso aromático y el biombo para la entrada
de la tienda; \bibleverse{39} el altar de bronce con su reja de bronce,
sus postes y todos sus utensilios; la palangana más su soporte;
\bibleverse{40} las cortinas del patio y sus postes y soportes; la
cortina para la entrada del patio, sus cuerdas y estacas de la tienda, y
todo el equipo para los servicios del Tabernáculo, el Tabernáculo de
Reunión; \bibleverse{41} y las vestimentas tejidas para servir en el
santuario, las ropas sagradas para el sacerdote Aarón y para sus hijos
para servir como sacerdotes. \bibleverse{42} Los israelitas hicieron
todo el trabajo que el Señor había ordenado a Moisés. \bibleverse{43}
Moisés inspeccionó todo el trabajo y se aseguró de que lo habían hecho
como el Señor se los había indicado. Entonces Moisés los bendijo.

\hypertarget{establecimiento-y-dedicaciuxf3n-del-santuario}{%
\subsection{Establecimiento y dedicación del
santuario}\label{establecimiento-y-dedicaciuxf3n-del-santuario}}

\hypertarget{section-39}{%
\section{40}\label{section-39}}

\bibleverse{1} El Señor le dijo a Moisés: \footnote{\textbf{40:1} Éxod
  25,1-31} \bibleverse{2} ``Levanta el Tabernáculo de Reunión, el primer
día del primer mes del año. \bibleverse{3} Coloca el Arca del Testimonio
dentro de ella. Asegúrate de que el Arca esté detrás del velo.
\bibleverse{4} Trae la mesa y pon sobre ella lo que sea necesario. Trae
también el candelabro y coloca sus lámparas. \bibleverse{5} Pon el altar
de oro del incienso delante del Arca del Testimonio, y pon el velo a la
entrada del Tabernáculo.

\bibleverse{6} Coloca el altar de los holocaustos frente a la entrada
del Tabernáculo, el Tabernáculo de Reunión. \bibleverse{7} Coloca la
palangana entre el Tabernáculo de Reunión y el altar, y pon agua en
ella. \bibleverse{8} ``Prepara el patio que lo rodea y pon la cortina
para la entrada del patio.

\bibleverse{9} ``Usa el aceite de la unción para ungir el Tabernáculo y
todo lo que hay en él. Dedícalo y todos sus muebles para hacerlo
sagrado. \bibleverse{10} Unge el altar de los holocaustos y todos sus
utensilios. Dedica el altar y será especialmente santo. \bibleverse{11}
Ungirás y dedicarás la pila con su soporte.

\bibleverse{12} ``Lleva a Aarón y a sus hijos a la entrada del
Tabernáculo de Reunión y lávalos allí con agua. \bibleverse{13}
Luegovistea Aarón con los vestidos sagrados, úngelo y dedícalo, para que
me sirva de sacerdote. \bibleverse{14} Que sus hijos se acerquen y los
vistan con túnicas. \bibleverse{15} Úngelos de la misma manera que
ungiste a su padre, para que también me sirvan como sacerdotes. Su
unción hará que su linaje de sacerdotes sea eterno, para las
generaciones futuras''.

\hypertarget{la-ejecuciuxf3n-del-mandato-divino}{%
\subsection{La ejecución del mandato
divino}\label{la-ejecuciuxf3n-del-mandato-divino}}

\bibleverse{16} Moisés llevó a cabo todas las instrucciones del Señor.

\bibleverse{17} El tabernáculo se levantó el primer día del primer mes
del segundo año.\footnote{\textbf{40:17} En otras palabras, hacía un año
  que habían salido de Egipto.} \bibleverse{18} Cuando Moisés levantó el
tabernáculo,\footnote{\textbf{40:18} Es evidente que Moisés no hizo todo
  este trabajo por sí mismo, sino que lo supervisaba.} colocó sus
soportes, fijó sus marcos, conectó sus travesaños y erigió sus postes.
\bibleverse{19} Luego extendió la tienda sobre el tabernáculo y colocó
la cubierta sobre la tienda, como el Señor le había ordenado.
\bibleverse{20} Moisés tomó el testimonio\footnote{\textbf{40:20} Las
  dos tablas inscritas con los Diez Mandamientos.} y lo puso en el arca.
Ató los postes al Arca, y colocó la tapa de expiación en la parte
superior del Arca. \bibleverse{21} Luego llevó el Arca al Tabernáculo.
Levantó el velo y se aseguró de que el Arca del Testimonio estuviera
detrás de ella, como el Señor le había ordenado. \bibleverse{22} Moisés
colocó la mesa dentro del Tabernáculo de Reunión en el lado norte del
Tabernáculo, fuera del velo. \bibleverse{23} Puso el pan sobre ella en
presencia del Señor, como el Señor le había ordenado. \bibleverse{24}
Colocó el candelabro en la tienda de la Reunión, frente a la mesa, en el
lado sur del Tabernáculo \bibleverse{25} y levantó las lámparas en
presencia del Señor, como el Señor le había ordenado. \bibleverse{26}
Moisés levantó el altar de oro en el Tabernáculo de Reunión, frente al
velo, \bibleverse{27} y quemó incienso aromático en él, como el Señor le
había ordenado. \bibleverse{28} Luego levantó el velo a la entrada del
Tabernáculo. \bibleverse{29} Levantó el altar del holocausto cerca de la
entrada del Tabernáculo de Reunión, y presentó el holocausto y la
ofrenda de grano, como el Señor le había ordenado. \bibleverse{30} Puso
la palangana entre el Tabernáculo de Reunión y el altar y puso agua para
lavar. \bibleverse{31} Moisés, Aarón y sus hijos la usaron para lavarse
las manos y los pies. \bibleverse{32} Se lavaban cada vez que entraban
en el Tabernáculo de Reunión o se acercaban al altar, como el Señor le
había ordenado a Moisés. \bibleverse{33} Moisés levantó el patio
alrededor del Tabernáculo y del altar, y puso la cortina para la entrada
del patio. Esto marcó el final del trabajo hecho por Moisés.

\hypertarget{la-gloria-del-seuxf1or-se-apodera-de-la-morada}{%
\subsection{La gloria del Señor se apodera de la
morada}\label{la-gloria-del-seuxf1or-se-apodera-de-la-morada}}

\bibleverse{34} Entonces la nube cubrió la Tienda de la Reunión, y la
gloria del Señor llenó el Tabernáculo. \bibleverse{35} Moisés no pudo
entrar en el Tabernáculo de Reunión porque la nube permaneció sobre
ella, y la gloria del Señor llenó el Tabernáculo. \bibleverse{36} Cada
vez que la nube se levantaba del Tabernáculo, los israelitas se ponían
en marcha de nuevo en su viaje. \bibleverse{37} Si la nube no se
levantaba, no se ponían en marcha hasta que la nube se levantara.
\bibleverse{38} La nube del Señor permanecía sobre el Tabernáculo
durante el día, y el fuego ardía dentro de la nube durante la noche, de
modo que podía ser visto por todos los israelitas dondequiera que
viajaran.
