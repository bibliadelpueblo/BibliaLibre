\hypertarget{apariciuxf3n-y-eficacia-de-juan-el-bautista}{%
\subsection{Aparición y eficacia de Juan el
Bautista}\label{apariciuxf3n-y-eficacia-de-juan-el-bautista}}

\hypertarget{section}{%
\section{1}\label{section}}

\bibleverse{1} Este es el principio de la Buena Noticia sobre
Jesucristo, el Hijo de Dios.

\bibleverse{2} Tal como escribió el profeta Isaías: ``Yo enviaré a mi
mensajero antes de ti para que prepare tu camino. \bibleverse{3} Una voz
que clama en el desierto: `¡Preparen el camino del Señor! Enderecen su
senda'\,''.\footnote{\textbf{1:3} Ver Malaquías 3:1; Isaías 40:3.}

\bibleverse{4} Juan vino bautizando en el desierto, anunciando un
bautismo de arrepentimiento\footnote{\textbf{1:4} Arrepentimiento quiere
  decir un ``cambio de pensamiento''.} para perdón de pecados.
\bibleverse{5} Todas las personas de Judea y de Jerusalén iban a él, y
confesaban públicamente sus pecados y eran bautizadas en el río Jordán.
\bibleverse{6} Juan usaba vestiduras hechas de pelo de camello, con un
cinturón de cuero. Comía langostas\footnote{\textbf{1:6} Lo más probable
  es que se refiera a algarrobas y no al insecto.} y miel silvestre.
\bibleverse{7} Esto es lo que él decía: ``Después de mi viene uno que es
más grande que yo. Yo no soy digno de desatar sus sandalias.
\bibleverse{8} Yo a ustedes los bautizaba con agua, pero él los
bautizará con el Espíritu Santo''.

\hypertarget{el-bautismo-y-la-tentaciuxf3n-de-jesuxfas}{%
\subsection{El bautismo y la tentación de
Jesús}\label{el-bautismo-y-la-tentaciuxf3n-de-jesuxfas}}

\bibleverse{9} Entonces Jesús vino de Nazaret, en Galilea, y fue
bautizado por Juan en el río Jordán. \footnote{\textbf{1:9} Luc 2,51}
\bibleverse{10} Cuando Jesús salió del agua, vio que los cielos se
abrieron y vio al Espíritu que descendía sobre Jesús como una paloma.
\bibleverse{11} Entonces una voz del cielo dijo: ``Tú eres mi hijo, al
que amo. Estoy muy agradado de ti''.

\bibleverse{12} Justo después de esto, el Espíritu lo envió al desierto,
\bibleverse{13} donde fue tentado por Satanás durante cuarenta días.
Allí estaba con los animales salvajes y los ángeles cuidaban de él.

\hypertarget{primera-apariciuxf3n-de-jesuxfas-en-galilea}{%
\subsection{Primera aparición de Jesús en
Galilea}\label{primera-apariciuxf3n-de-jesuxfas-en-galilea}}

\bibleverse{14} Más adelante, después que Juan fue arrestado, Jesús fue
a Galilea, anunciando la Buena Noticia de Dios. \bibleverse{15} ``El
tiempo que estaba predicho ha llegado'', decía. ``El reino de Dios está
cerca. Arrepiéntanse y crean en la Buena Noticia''. \footnote{\textbf{1:15}
  Gal 4,4}

\hypertarget{llamando-a-los-primeros-cuatro-discuxedpulos}{%
\subsection{Llamando a los primeros cuatro
discípulos}\label{llamando-a-los-primeros-cuatro-discuxedpulos}}

\bibleverse{16} Mientras caminaba junto al Mar de Galilea, Jesús vio a
Simón y a su hermano Andrés lanzando una red al agua, pues ellos se
ganaban la vida como pescadores. \bibleverse{17} ``Vengan y síganme'',
les dijo, ``y yo haré que sean pescadores de personas''.

\bibleverse{18} Entonces ellos dejaron inmediatamente a un lado sus
redes y lo siguieron.

\bibleverse{19} Entonces Jesús caminó un poco más y vio a Santiago y a
su hermano Juan, los hijos de Zebedeo. Ellos estaban en una barca
arreglando sus redes. \bibleverse{20} De inmediato los llamó para que lo
siguieran,\footnote{\textbf{1:20} Implícito.} y ellos dejaron a su padre
Zebedeo en la barca con los trabajadores, y siguieron a Jesús.

\hypertarget{el-primer-sermuxf3n-de-jesuxfas-y-la-curaciuxf3n-de-un-hombre-poseuxeddo-en-la-sinagoga-de-capernaum}{%
\subsection{El primer sermón de Jesús y la curación de un hombre poseído
en la sinagoga de
Capernaum}\label{el-primer-sermuxf3n-de-jesuxfas-y-la-curaciuxf3n-de-un-hombre-poseuxeddo-en-la-sinagoga-de-capernaum}}

\bibleverse{21} Partieron de allí hacia Capernaúm, y el sábado Jesús
entró a la sinagoga y enseñaba allí. \bibleverse{22} La gente estaba
maravillada de su enseñanza, porque él hablaba con autoridad, no como
los maestros religiosos.\footnote{\textbf{1:22} ``Maestros religiosos''
  o ``escribas''. Estos eran más que ``escribas'' en cuanto a escribir
  solamente. Estos eran hombres que tenían autoridad de interpretar las
  Escrituras y dedicaban tiempo para instruir al pueblo en cuanto a las
  exigencias de las leyes religiosas.} \bibleverse{23} De repente, allí
en la sinagoga, un hombre con un espíritu malo comenzó a gritar.
\bibleverse{24} ``Jesús de Nazaret, ¿por qué nos molestas?'' gritaba.
``¿Has venido a destruirnos? ¡Yo sé quién eres! ¡Tú eres el Santo de
Dios!'' \footnote{\textbf{1:24} Mar 5,7}

\bibleverse{25} Jesús interrumpió al espíritu malo diciéndole:
``¡Cállate! Sal de él''.

\bibleverse{26} El espíritu malo gritaba, hizo convulsionar al hombre, y
salió de él. \bibleverse{27} Todos estaban asombrados ante lo que había
ocurrido. ``¿Qué es esto?'' se preguntaban unos a otros. ``¿Qué
enseñanza nueva es esta, que tiene tanta autoridad? ¡Incluso los
espíritus malos hacen lo que él les ordena!'' \bibleverse{28} Y la
noticia acerca de Jesús se esparció rápidamente por toda la región de
Galilea.

\hypertarget{sanaciuxf3n-de-la-suegra-de-simuxf3n-y-otros-enfermos-en-capernaum}{%
\subsection{Sanación de la suegra de Simón y otros enfermos en
Capernaum}\label{sanaciuxf3n-de-la-suegra-de-simuxf3n-y-otros-enfermos-en-capernaum}}

\bibleverse{29} Entonces ellos salieron de la sinagoga y se fueron a la
casa de Simón y Andrés, junto con Santiago y Juan. \bibleverse{30} Pero
la suegra de Simón estaba enferma, acostada en la cama, y con fiebre.
Entonces se lo dijeron a Jesús. \bibleverse{31} Jesús fue donde ella
estaba, la tomó de la mano y la ayudó a levantarse. De inmediato la
fiebre se le quitó. Entonces ella les preparó una comida.

\bibleverse{32} Esa tarde, después de la puesta del sol, trajeron
delante de Jesús muchos enfermos y endemoniados. \bibleverse{33} Toda la
ciudad se reunió afuera. \bibleverse{34} Él sanaba a muchas personas con
diferentes enfermedades y expulsaba muchos demonios. Jesús no permitía
que los demonios hablaran, porque ellos sabían quién era él. \footnote{\textbf{1:34}
  Hech 16,17-18}

\hypertarget{jesuxfas-deja-capernaum-su-sermuxf3n-errante-y-actividad-curativa-en-galilea}{%
\subsection{Jesús deja Capernaum; su sermón errante y actividad curativa
en
Galilea}\label{jesuxfas-deja-capernaum-su-sermuxf3n-errante-y-actividad-curativa-en-galilea}}

\bibleverse{35} Muy temprano en la mañana, mientras aún era oscuro,
Jesús se levantó y se fue a solas a un lugar tranquilo para orar.
\footnote{\textbf{1:35} Mat 14,23; Mat 26,36; Luc 5,16; Luc 11,1}
\bibleverse{36} Simón y los otros discípulos fueron a buscarlo.
\bibleverse{37} Cuando lo encontraron, le dijeron: ``Todos te están
buscando''.

\bibleverse{38} Pero Jesús respondió: ``Tenemos que ir a otras ciudades
cercanas para contarles la Buena Noticia a ellos también, pues por eso
vine'', les dijo. \bibleverse{39} Así que Jesús se fue por toda Galilea,
hablando en las sinagogas y expulsando demonios.

\hypertarget{jesuxfas-sana-a-un-leproso-y-escapa-a-la-soledad}{%
\subsection{Jesús sana a un leproso y escapa a la
soledad}\label{jesuxfas-sana-a-un-leproso-y-escapa-a-la-soledad}}

\bibleverse{40} Entonces un leproso vino a él pidiéndole ayuda. El
hombre se arrodilló delante de Jesús y le dijo: ``¡Por favor, si
quieres, puedes sanarme!''

\bibleverse{41} Jesús se extendió hacia él con compasión y lo tocó, y le
dijo: ``Quiero. ¡Queda sano!'' \bibleverse{42} Entonces la lepra se fue
por completo de su cuerpo, y quedó sano. \bibleverse{43} Jesús lo envió
de regreso con una advertencia muy importante: \footnote{\textbf{1:43}
  Mar 3,12; Mar 7,36} \bibleverse{44} ``Asegúrate de no decirle a nadie
acerca de esto'', le dijo. ``Ve donde el sacerdote y preséntate delante
de él. Da la ofrenda que exige la ley de Moisés por tal limpieza, para
que el pueblo tenga una prueba de ello''.\footnote{\textbf{1:44} Ver
  Levítico 14.} \footnote{\textbf{1:44} Lev 14,2-32}

\bibleverse{45} Pero el hombre que había sido sanado se fue y le contó a
todos lo que había ocurrido. Como resultado de ello, Jesús ya no podía
ir más a las ciudades abiertamente, sino que tenía que quedarse en el
campo, donde las personas venían a buscarlo desde todas partes.

\hypertarget{curaciuxf3n-de-un-paraluxedtico-en-capernaum-jesuxfas-perdona-los-pecados}{%
\subsection{Curación de un paralítico en Capernaum; Jesús perdona los
pecados}\label{curaciuxf3n-de-un-paraluxedtico-en-capernaum-jesuxfas-perdona-los-pecados}}

\hypertarget{section-1}{%
\section{2}\label{section-1}}

\bibleverse{1} Unos pocos días después, Jesús regresó a su casa en
Capernaúm, y entonces se difundió la noticia de que él estaba allí.
\bibleverse{2} Muchas personas se amontonaron dentro de la casa y esta
estaba llena, incluso hasta fuera de la puerta, y Jesús les hablaba de
la Buena Noticia. \bibleverse{3} Cuatro hombres habían traído a un
hombre que estaba paralítico, \bibleverse{4} pero no pudieron acercarse
a Jesús por la multitud que estaba allí. Así que subieron al techo y lo
abrieron. Después que hicieron una abertura sobre el sitio donde estaba
Jesús, bajaron la camilla con el hombre paralítico en ella.
\bibleverse{5} Cuando Jesús vio la confianza que tuvieron estos hombres,
le dijo al hombre paralítico: ``Amigo, tus pecados están perdonados''.

\bibleverse{6} Entonces algunos de los maestros religiosos que estaban
sentados allí dijeron para sí: \bibleverse{7} ``¿Por qué habla él de
esta manera? ¡Está blasfemando! ¿Quién puede perdonar pecados? ¡Solo
Dios puede hacer eso!'' \footnote{\textbf{2:7} Sal 130,4; Is 43,25}

\bibleverse{8} Jesús supo inmediatamente lo que ellos estaban pensando.
Entonces les dijo: ``¿Por qué piensan así? \bibleverse{9} ¿Qué es más
fácil: decirle al paralítico `tus pecados están perdonados,' o
`levántate, toma tu camilla y camina'? \bibleverse{10} Pero para
convencerlos a ustedes de que el Hijo del hombre tiene autoridad para
perdonar pecados, \bibleverse{11} yo te digo (dirigiéndose al
paralítico), `Levántate, recoge tu camilla y vete a casa'\,''.

\bibleverse{12} Entonces el paralítico se levantó, recogió su camilla y
caminó frente a todos los que estaban allí. Y todos estaban asombrados,
y alababan a Dios, diciendo: ``¡Nunca hemos visto algo así!''

\hypertarget{llamando-al-recaudador-de-impuestos-levi-jesuxfas-como-compauxf1ero-de-mesa-para-recaudadores-de-impuestos-y-pecadores}{%
\subsection{Llamando al recaudador de impuestos Levi; Jesús como
compañero de mesa para recaudadores de impuestos y
pecadores}\label{llamando-al-recaudador-de-impuestos-levi-jesuxfas-como-compauxf1ero-de-mesa-para-recaudadores-de-impuestos-y-pecadores}}

\bibleverse{13} Jesús salió y se ubicó junto al mar una vez más y le
enseñaba a las multitudes que venían a él. \bibleverse{14} Mientras
caminaba, vio a Leví, el hijo de Alfeo, sentado en la mesa de los
recaudadores de impuestos. ``Sígueme'', le dijo Jesús. Entonces Leví se
levantó y siguió a Jesús.

\bibleverse{15} Esa noche Jesús cenó en la casa de Leví. Muchos
recaudadores de impuestos y ``pecadores''\footnote{\textbf{2:15}
  ``Pecadores'' se refería a quienes no se les veía que fueran estrictos
  en el cumplimiento de las leyes religiosas tanto como sí lo hacían los
  maestros religiosos y los Fariseos.} se unieron a Jesús y sus
discípulos para la cena, porque muchos de ellos lo seguían.
\bibleverse{16} Cuando los líderes religiosos de los Fariseos vieron a
Jesús comiendo con tales personas, le preguntaron a los discípulos de
Jesús: ``¿Por qué Jesús come con los recaudadores de impuestos y
pecadores?''

\bibleverse{17} Cuando Jesús escuchó esto, les dijo: ``No son las
personas sanas las que necesitan de un médico, sino las que están
enfermas. No he venido a invitar a los que hacen lo correcto, sino a
quienes no lo hacen, a los pecadores''.

\hypertarget{la-pregunta-del-ayuno-de-los-discuxedpulos-de-juan-y-los-fariseos}{%
\subsection{La pregunta del ayuno de los discípulos de Juan y los
fariseos}\label{la-pregunta-del-ayuno-de-los-discuxedpulos-de-juan-y-los-fariseos}}

\bibleverse{18} Aconteció que los discípulos de Juan y los Fariseos
estaban ayunando.\footnote{\textbf{2:18} Ayunar: elegir no comer ciertos
  días por motivos religiosos.} Algunos de ellos vinieron donde Jesús
estaba y le preguntaron: ``¿Por qué los discípulos de Juan y los
Fariseos ayunan, pero tus discípulos no lo hacen?''

\bibleverse{19} ``¿Acaso los invitados a una fiesta de bodas ayunan
mientras el novio está con ellos?'' preguntó Jesús. ``No.~Mientras el
novio está con ellos, ellos no pueden ayunar. \bibleverse{20} Pero viene
el día en que el novio será arrebatado de en medio de ellos, y entonces
ellos ayunarán. \bibleverse{21} Nadie coloca un parche nuevo en ropas
viejas. De lo contrario el parche nuevo se encogería y se despegaría de
la ropa vieja y la rasgadura sería peor. \bibleverse{22} Nadie echa vino
nuevo en odres viejos. De ser así, el vino rompería los odres y se
dañaría tanto el vino como los odres. No.~La gente echa el vino nuevo en
odres nuevos''.

\hypertarget{el-arranco-de-espigas-de-los-discuxedpulos-en-suxe1bado-la-primera-disputa-de-jesuxfas-con-los-fariseos-sobre-la-santificaciuxf3n-del-duxeda-de-reposo}{%
\subsection{El arranco de espigas de los discípulos en sábado; La
primera disputa de Jesús con los fariseos sobre la santificación del día
de
reposo}\label{el-arranco-de-espigas-de-los-discuxedpulos-en-suxe1bado-la-primera-disputa-de-jesuxfas-con-los-fariseos-sobre-la-santificaciuxf3n-del-duxeda-de-reposo}}

\bibleverse{23} Sucedió que un día sábado, mientras Jesús caminaba por
los campos de trigo, sus discípulos comenzaron a recoger espigas por el
camino. \bibleverse{24} Los Fariseos le preguntaron entonces a Jesús:
``Mira, ¿por qué ellos están haciendo algo que no está permitido hacer
en sábado?''

\bibleverse{25} ``¿Acaso ustedes no han leído lo que hizo David cuando
él y sus hombres tuvieron hambre?'' les preguntó Jesús. \bibleverse{26}
``Él entró a la casa de Dios cuando Abiatar era el sumo sacerdote, y
comió del pan de la consagración, del cual no podía comer nadie, excepto
los sacerdotes, y lo dio a comer a sus hombres también''.

\bibleverse{27} ``El sábado fue hecho para beneficio de ustedes, y no
ustedes para beneficio del sábado'', les dijo. \footnote{\textbf{2:27}
  Deut 5,14}

\bibleverse{28} ``Así que el Hijo del hombre es Señor incluso del
sábado''.

\hypertarget{sanaciuxf3n-del-hombre-con-el-brazo-paralizado-en-suxe1bado-el-segundo-argumento-sobre-la-observancia-del-suxe1bado}{%
\subsection{Sanación del hombre con el brazo paralizado en sábado; el
segundo argumento sobre la observancia del
sábado}\label{sanaciuxf3n-del-hombre-con-el-brazo-paralizado-en-suxe1bado-el-segundo-argumento-sobre-la-observancia-del-suxe1bado}}

\hypertarget{section-2}{%
\section{3}\label{section-2}}

\bibleverse{1} Una vez más Jesús fue a la sinagoga. Allí estaba un
hombre que tenía una mano lisiada. \bibleverse{2} Algunos de los que
estaban allí estaban observando si Jesús lo sanaría en sábado, pues
estaban buscando un motivo para acusarlo de quebrantar la ley.
\bibleverse{3} Jesús le dijo al hombre con la mano lisiada: ``Ven y
párate aquí frente a todos''. \bibleverse{4} ``¿Es lícito hacer el bien
en sábado, o hacer el mal? ¿Debemos salvar vidas o matar?'' les
preguntó. Pero ellos no dijeron ni una palabra. \bibleverse{5} Jesús los
miró con exasperación, muy molesto por la dureza de sus corazones.
Entonces le dijo al hombre: ``Extiende tu mano''. Y el hombre extendió
su mano, y le fue sanada. \bibleverse{6} Los Fariseos salieron, e
inmediatamente comenzaron a conspirar con los aliados de Herodes sobre
cómo podrían matar a Jesús.

\hypertarget{afluencia-de-personas-muchas-curaciones-en-el-lago}{%
\subsection{Afluencia de personas; muchas curaciones en el
lago}\label{afluencia-de-personas-muchas-curaciones-en-el-lago}}

\bibleverse{7} Mientras tanto, Jesús regresó al Mar,\footnote{\textbf{3:7}
  De Galilea.} y una gran multitud lo seguía. Había gente de Galilea,
\bibleverse{8} de Judea, de Idumea, de Transjordania, y de las regiones
de Tiro y Sidón. Muchas personas venían a verlo porque habían escuchado
todo lo que él hacía. \bibleverse{9} Jesús les dijo a sus discípulos que
tuvieran una barca pequeña en caso de que la multitud comenzara a
aglomerarse sobre él, \bibleverse{10} porque había sanado a tantas
personas que todos los enfermos seguían tratando de amontonarse y
empujarse para poder tocarlo. \bibleverse{11} Cada vez que los espíritus
malos lo veian, caían frente a él y comenzaban a gritar: ``¡Tú eres el
Hijo de Dios!'' \footnote{\textbf{3:11} Luc 4,41} \bibleverse{12} Pero
Jesús les ordenaba que no revelasen quién era él. \footnote{\textbf{3:12}
  Mar 1,43}

\hypertarget{berufung-und-namen-der-zwuxf6lf-juxfcnger}{%
\subsection{Berufung und Namen der zwölf
Jünger}\label{berufung-und-namen-der-zwuxf6lf-juxfcnger}}

\bibleverse{13} Entonces Jesús se fue al monte. Llamó a los que quería
que lo acompañaran, y ellos fueron con él. \bibleverse{14} Eligió a doce
para que estuvieran con él, y los llamó apóstoles. Ellos estarían con
él, y él los enviaría a anunciar la Buena Noticia, \bibleverse{15}
dándoles autoridad para expulsar demonios. \bibleverse{16} Estos son los
doce que él escogió: Simón (a quien llamó Pedro), \bibleverse{17}
Santiago, hijo de Zebedeo y su hermano Juan (a quienes llamó Boanerges,
que quiere decir ``hijos del trueno''), \footnote{\textbf{3:17} Luc 9,54}
\bibleverse{18} Andrés, Felipe, Bartolomé, Mateo, Tomás, Santiago hijo
de Alfeo, Tadeo, Simón el revolucionario,\footnote{\textbf{3:18} Ver
  Lucas 6:15.} \bibleverse{19} y Judas Iscariote (quien lo entregó).

\hypertarget{el-crecimiento-del-movimiento}{%
\subsection{El crecimiento del
movimiento}\label{el-crecimiento-del-movimiento}}

\bibleverse{20} Jesús se fue a casa, pero la gran multitud se volvió a
reunir y él y sus discípulos ni siquiera tenían tiempo para comer.
\bibleverse{21} Cuando la familia de Jesús\footnote{\textbf{3:21}
  Literalmente, ``los que andaban con él''.} escuchó acerca de esto,
fueron a buscarlo para llevárselo, porque decían, ``¡se ha vuelto
loco!''

\hypertarget{jesuxfas-se-defiende-de-la-blasfemia-de-beelzebul-de-los-escribas.-del-pecado-contra-el-espuxedritu-santo}{%
\subsection{Jesús se defiende de la blasfemia de Beelzebul de los
escribas. Del pecado contra el espíritu
santo}\label{jesuxfas-se-defiende-de-la-blasfemia-de-beelzebul-de-los-escribas.-del-pecado-contra-el-espuxedritu-santo}}

\bibleverse{22} Pero los líderes religiosos de Jerusalén, decían: ``¡Él
está poseído por Belcebú! ¡Es en nombre del príncipe de los demonios que
los expulsa!'' \footnote{\textbf{3:22} Mat 9,34}

\bibleverse{23} Pero Jesús los llamó para que se acercaran a él. Y a
través de ilustraciones\footnote{\textbf{3:23} O ``parábolas'', es
  decir, analogías, comparaciones o ilustraciones.} les preguntó:
``¿Cómo puede Satanás expulsar a Satanás? \bibleverse{24} Un reino que
pelea contra sí mismo no puede mantenerse. \bibleverse{25} Una casa
dividida está destinada a la destrucción. \bibleverse{26} Si Satanás
está dividido y pelea contra sí mismo, no durará y pronto llegará a su
fin. \bibleverse{27} Sin duda, si alguien entra a robar a la casa de un
hombre fuerte y trata de llevarse sus pertenencias, no lo logrará a
menos que ate al hombre fuerte primero''.

\bibleverse{28} ``Les digo la verdad: los pecados y las blasfemias
pueden ser perdonados, \bibleverse{29} pero si alguno blasfema
rechazando al Espíritu Santo, no podrá ser perdonado, porque es culpable
de un pecado eterno''. \bibleverse{30} (Jesús dijo esto\footnote{\textbf{3:30}
  Implícito.} porque ellos decían: ``Él tiene un espíritu maligno'').
\footnote{\textbf{3:30} Juan 10,20}

\hypertarget{los-verdaderos-parientes-de-jesuxfas}{%
\subsection{Los verdaderos parientes de
Jesús}\label{los-verdaderos-parientes-de-jesuxfas}}

\bibleverse{31} Entonces la madre de Jesús y sus hermanos llegaron. Lo
esperaron afuera y mandaron a alguien para que le pidiera que saliera.
\bibleverse{32} La multitud que estaba sentada afuera le dijo: ``Tu
madre y tus hermanos están allá afuera preguntando por ti''.

\bibleverse{33} ``¿Quién es mi madre? ¿Quiénes son mis hermanos?''
respondió él. \bibleverse{34} Y mirando alrededor a todos los que
estaban sentados, les dijo: ``¡Aquí está mi madre! ¡Aquí están mis
hermanos! \bibleverse{35} Todo aquél que hace la voluntad de Dios, ese
es mi hermano, mi hermana y mi madre''.

\hypertarget{paruxe1bola-del-sembrador-y-cuatro-tipos-de-campos}{%
\subsection{Parábola del sembrador y cuatro tipos de
campos}\label{paruxe1bola-del-sembrador-y-cuatro-tipos-de-campos}}

\hypertarget{section-3}{%
\section{4}\label{section-3}}

\bibleverse{1} Jesús comenzó a enseñar junto al mar una vez más.
Vinieron tantas personas a escucharlo que tuvo que montarse en una barca
y se sentó en ella, en el agua, mientras la multitud lo oía desde la
orilla. \bibleverse{2} Él ilustraba sus enseñanzas por medio de relatos.
\bibleverse{3} ``Escuchen'', les dijo él. ``Un sembrador salió a
sembrar. \bibleverse{4} Sucedió que cuando estaba esparciendo las
semillas, algunas cayeron en el camino, y las aves vinieron y se las
comieron. \bibleverse{5} Otras semillas cayeron en terreno rocoso donde
no había mucha tierra. En ese suelo sin profundidad las plantas
comenzaron a crecer con rapidez, pero como el suelo no era
suficientemente profundo, \bibleverse{6} se quemaron apenas salió el
sol. Y como no tenían raíces profundas, pronto se marchitaron.
\bibleverse{7} Otras semillas cayeron entre espinos. Estos crecieron y
ahogaron las semillas que germinaban, así que no dieron fruto.
\bibleverse{8} Otras semillas cayeron en buen suelo y allí germinaron y
crecieron. Produjeron cosecha treinta, sesenta, y algunas hasta cien
veces lo que había sido sembrado. \bibleverse{9} Si tienen oídos para
oír, oigan lo que les digo''.

\hypertarget{analice-el-significado-y-el-propuxf3sito-de-las-paruxe1bolas}{%
\subsection{Analice el significado y el propósito de las
parábolas}\label{analice-el-significado-y-el-propuxf3sito-de-las-paruxe1bolas}}

\bibleverse{10} Estando a solas con Jesús, sus doce discípulos y otras
personas que estaban con él le preguntaron lo que significaban tales
ilustraciones. \bibleverse{11} Entonces les dijo: ``El misterio del
reino de Dios ha sido entregado a ustedes para que entiendan. Pero los
incrédulos solo escuchan las historias, \bibleverse{12} de modo que
aunque pueden ver, en realidad no `ven,' y aunque pueden oír, no
entienden, de lo contrario podrían convertirse y ser
perdonados''.\footnote{\textbf{4:12} Citando Isaías 6:9-10.}

\bibleverse{13} ``¿No entienden este relato?'' les preguntó Jesús. ``Si
no pueden entenderlo, ¿cómo podrán entender todos los demás?''

\hypertarget{interpretaciuxf3n-de-la-paruxe1bola-del-sembrador}{%
\subsection{Interpretación de la parábola del
sembrador}\label{interpretaciuxf3n-de-la-paruxe1bola-del-sembrador}}

\bibleverse{14} ``El sembrador siembra la palabra.\footnote{\textbf{4:14}
  Palabra o ``mensaje'', la Palabra de Dios que vino a traer Jesús. (Ver
  también Juan 1:1).} \bibleverse{15} Las semillas en el camino donde se
siembra la palabra representan a aquellos que escuchan el mensaje, pero
inmediatamente Satanás llega y se lleva la palabra que ha sido sembrada
en ellos. \bibleverse{16} De la misma manera, las semillas en el suelo
rocoso representan a los que oyen la palabra y la aceptan inmediatamente
con felicidad. \bibleverse{17} Pero como no tienen raíces profundas,
solo permanecen por un tiempo, hasta que llega la persecución, y pronto
se apartan. \bibleverse{18} Aquellas semillas sembradas entre los
espinos representan a quienes oyen la palabra, \bibleverse{19} pero las
preocupaciones de este mundo, la tentación por las riquezas, y otras
distracciones ahogan el crecimiento de la palabra, y se vuelve
infructuosa. \bibleverse{20} Pero las semillas que fueron sembradas en
el buen suelo representan a aquellos que escuchan la palabra, la
aceptan, y produce fruto, produciendo treinta, sesenta y hasta cien
veces más lo que originalmente se sembró.

\bibleverse{21} ``¿Quién pone una lámpara debajo de un balde, o bajo la
cama?'' les preguntó Jesús. ``No, una lámpara se coloca sobre un
candelabro. \footnote{\textbf{4:21} Mat 5,15} \bibleverse{22} Todo lo
que está oculto, será revelado, y todo lo que está en secreto, saldrá a
la luz. \footnote{\textbf{4:22} Mat 10,26-27; Luc 12,2} \bibleverse{23}
Si tienen oídos para oír, oigan lo que les digo.

\bibleverse{24} Presten atención a lo que están oyendo'', les dijo,
``pues se les dará en la medida que ustedes quieran recibir, medida por
medida. \footnote{\textbf{4:24} Mat 7,2} \bibleverse{25} Se le dará más
a los que ya tienen entendimiento, pero los que no quieren saber, el
poco entendimiento que tengan se les quitará. \footnote{\textbf{4:25}
  Mat 13,12-13}

\hypertarget{paruxe1bolas-de-la-semilla-que-crece-tranquilamente-por-suxed-misma-y-de-la-semilla-de-mostaza}{%
\subsection{Parábolas de la semilla que crece tranquilamente por sí
misma y de la semilla de
mostaza}\label{paruxe1bolas-de-la-semilla-que-crece-tranquilamente-por-suxed-misma-y-de-la-semilla-de-mostaza}}

\bibleverse{26} ``El reino de Dios es como un hombre que siembra las
semillas en el suelo'', dijo Jesús. \bibleverse{27} ``Este hombre va a
dormir y se levanta cada día, pero no sabe cómo germinarán y crecerán
las semillas. \footnote{\textbf{4:27} Sant 5,7} \bibleverse{28} La
tierra produce la cosecha por sí sola. Primero aparece un brote, luego
se ve el grano, luego el grano maduro. \bibleverse{29} Cuando el grano
está maduro, el granjero lo siega con una hoz, pues la cosecha está
lista.\footnote{\textbf{4:29} Posiblemente una referencia a Joel 3:13.}

\bibleverse{30} ``¿Con qué podríamos comparar el reino de Dios? ¿Qué
ilustración podríamos usar?'' preguntó. \bibleverse{31} ``Es como una
semilla de mostaza, la más pequeña de todas las semillas.
\bibleverse{32} Pero cuando se siembra, crece y se convierte en un árbol
que es más grande que las demás plantas. Y tiene ramas tan grandes que
las aves pueden posarse bajo su sombra''.

\bibleverse{33} Jesús usaba muchos de estos relatos ilustrados cuando
hablaba a la gente a fin de que pudieran entender cuanto fuera posible.
\bibleverse{34} De hecho, cuando hablaba públicamente solo usaba
relatos, pero en privado él les explicaba todas las cosas a sus
discípulos.

\hypertarget{jesuxfas-apacigua-la-tormenta-del-mar}{%
\subsection{Jesús apacigua la tormenta del
mar}\label{jesuxfas-apacigua-la-tormenta-del-mar}}

\bibleverse{35} Ese mismo día por la noche, él les dijo a sus
discípulos: ``Vayamos y crucemos hasta el otro lado del Mar''.
\bibleverse{36} Y abandonando la multitud, los discípulos se subieron
con Jesús en una barca. Y otras embarcaciones iban con ellos.
\bibleverse{37} De pronto, comenzó a soplar una fuerte tormenta, y las
olas chocaban contra la barca, llenándola de agua. \bibleverse{38} Jesús
estaba dormido en la parte trasera de la barca, con su cabeza recostada
sobre un almohadón. Entonces los discípulos lo despertaron, gritándole:
``¡Maestro! ¿No te preocupa que estamos a punto de ahogarnos?''

\bibleverse{39} Jesús se despertó. Entonces le dijo al viento que se
calmara y a las olas les dijo: ``¡Cállense! Estén quietas''. Entonces el
viento se calmó y el agua se quedó completamente tranquila.
\bibleverse{40} ``¿Por qué están tan asustados?\footnote{\textbf{4:40}
  La palabra que se usa aquí se refiere a cobardes.} ¿No han aprendido a
confiar en mí?'' les preguntó.

\bibleverse{41} Ellos estaban aturdidos y aterrorizados.\footnote{\textbf{4:41}
  Aunque a menudo en las traducciones se enfatiza el aspecto del temor,
  el texto indica que ya estaban asustados previamente. Ahora estaban
  asombrados, estaban impresionados por lo que había ocurrido, aunque
  sin duda estaban aún aterrorizados.} Se preguntaban unos a otros,
``¿Quién es este? ¡Hasta el viento y las olas le obedecen!''

\hypertarget{jesuxfas-sana-a-los-poseuxeddos-en-la-tierra-de-los-gerasenos}{%
\subsection{Jesús sana a los poseídos en la tierra de los
gerasenos}\label{jesuxfas-sana-a-los-poseuxeddos-en-la-tierra-de-los-gerasenos}}

\hypertarget{section-4}{%
\section{5}\label{section-4}}

\bibleverse{1} Entonces llegaron al otro lado del lago, a la región de
los Gerasenes. \bibleverse{2} Cuando Jesús bajó de la barca, un hombre
con un espíritu maligno salió del cementerio a su encuentro.
\bibleverse{3} Este hombre vivía entre las tumbas, y ya era imposible
hacerle más ataduras, incluso con una cadena. \bibleverse{4} A menudo
había sido atado con cadenas y grilletes, pero fácilmente rompía las
cadenas y hacía pedazos los grilletes. Nadie tenía la fuerza suficiente
para dominarlo. \bibleverse{5} Siempre estaba gritando, día y noche,
entre las tumbas y en las colinas cercanas, cortándose con piedras
filosas. \bibleverse{6} Al ver a Jesús desde la distancia, corrió y se
arrodilló frente a él. \bibleverse{7} Y con voz alta gritó: ``¿Qué
tienes que ver conmigo, Jesús, hijo del Dios Todopoderoso? ¡Jura por
Dios que no me torturarás!'' \bibleverse{8} Pues Jesús ya le había dicho
al espíritu maligno que saliera del hombre.

\bibleverse{9} Entonces Jesús le preguntó: ``¿Cuál es tu nombre?'' ``Mi
nombre es Legión, ¡porque somos muchos!'' le respondió.

\bibleverse{10} Además le imploraba a Jesús repetidas veces que no los
enviara lejos.\footnote{\textbf{5:10} Literalmente, ``fuera de la
  región''.} \bibleverse{11} Un gran rebaño de cerdos se alimentaba en
la ladera que estaba cerca. \bibleverse{12} Entonces los espíritus
malignos le imploraron: ``Envíanos a los cerdos para que entremos en
ellos''.

\bibleverse{13} Y Jesús permitió que lo hicieran. Entonces los espíritus
malignos salieron de aquél hombre y se fueron hacia el lugar donde
estaban los cerdos. Y todo el rebaño, cerca de dos mil cerdos, salió
corriendo cuesta abajo por un precipicio hacia el mar y se ahogaron.
\bibleverse{14} Los hombres que cuidaban el rebaño de cerdos salieron
corriendo, y difundieron la noticia por toda la ciudad y en el pueblo.
La gente vino a ver lo que había pasado.

\bibleverse{15} Cuando encontraron a Jesús, vieron al hombre endemoniado
sentado allí, vestido, y en su sano juicio---y se asustaron.
\bibleverse{16} Los que habían visto lo que había ocurrido con el hombre
poseído por el demonio y con los cerdos lo contaron a los demás.
\bibleverse{17} Comenzaron a suplicarle a Jesús que se fuera de su
región.

\bibleverse{18} Cuando Jesús subió a la barca, el hombre que había
estado poseído por el demonio le rogó que lo dejara ir con él.
\bibleverse{19} Pero Jesús no aceptó, y le dijo: ``Ve a tu casa, a tu
propio pueblo, y cuéntales todo lo que el Señor ha hecho por ti y cuán
misericordioso ha sido contigo''.

\bibleverse{20} Así que el hombre siguió su propio camino y comenzó a
contarle a la gente de las Diez Ciudades todo lo que Jesús había hecho
por él, y todos estaban asombrados. \footnote{\textbf{5:20} Mar 7,31}

\hypertarget{jesuxfas-sana-a-la-mujer-ensangrentada-en-capernaum-y-despierta-a-la-hija-de-jairo}{%
\subsection{Jesús sana a la mujer ensangrentada en Capernaum y despierta
a la hija de
Jairo}\label{jesuxfas-sana-a-la-mujer-ensangrentada-en-capernaum-y-despierta-a-la-hija-de-jairo}}

\bibleverse{21} Jesús regresó nuevamente en la barca al otro lado del
lago donde había una gran multitud reunida a su alrededor en la orilla.
\bibleverse{22} Un líder, llamado Jairo, de una de las sinagogas vino
donde él estaba. Cuando vio a Jesús, cayó a sus pies \bibleverse{23} y
le suplicó diciendo: ``Mi hijita está a punto de morir. Por favor, ven y
coloca tus manos sobre ella para que sea sanada y viva''.

\bibleverse{24} Entonces Jesús fue con él. Todos lo seguían, al tiempo
que lo empujaban y se arrimaban sobre él. \bibleverse{25} Allí había una
mujer que había estado enferma por causa de un sangrado durante doce
años. \bibleverse{26} Había sufrido mucho bajo el cuidado de muchos
médicos, y había gastado todo lo que tenía. Pero nada había sido útil,
de hecho, había empeorado. \bibleverse{27} Ella había escuchado sobre
Jesús, así que se levantó para ir tras él, en medio de la multitud, y
tocó su manto. \bibleverse{28} Pues ella pensaba dentro de sí: ``Si tan
solo logro tocar su manto, seré sanada''. \bibleverse{29} El sangrado se
detuvo de inmediato, y ella sintió que su cuerpo quedó sano de su
enfermedad.

\bibleverse{30} Jesús, al percibir que de él había salido poder, se dio
la vuelta en medio de la multitud y preguntó, ``¿quién tocó mi manto?''

\bibleverse{31} ``Mira la multitud que te empuja. ¿Qué quieres decir con
eso de `quién me tocó?'\,'' respondieron los discípulos.

\bibleverse{32} Pero Jesús seguía mirando la multitud a su alrededor
para ver quién lo había hecho. \bibleverse{33} Entonces la mujer, al
comprender lo que le había sucedido, vino y se postró delante de él, y
le dijo toda la verdad.

\bibleverse{34} ``Hija mía, tu confianza en mí te ha sanado. Vete en
paz. Has sido completamente sanada de tu enfermedad'', le dijo Jesús.

\bibleverse{35} Mientras aún hablaba, algunas personas vinieron de la
casa del líder de la sinagoga. ``Tu hija murió'', le dijeron. ``Ya no
necesitas molestar más al Maestro''.

\bibleverse{36} Pero Jesús no prestó atención a lo que ellos dijeron.
Entonces le dijo al líder de la Sinagoga: ``No temas, confía en
mí''.\footnote{\textbf{5:36} ``En mi'': Implícito.} \bibleverse{37} Él
no dejó que ninguno fuera con él, excepto Pedro, Santiago, y Juan, que
era el hermano de Santiago. \footnote{\textbf{5:37} Mat 17,1}
\bibleverse{38} Cuando llegaron a la casa del líder de la sinagoga,
Jesús vio toda la conmoción de las personas que lloraban y gemían.
\bibleverse{39} Jesús entró y les preguntó: ``¿Por qué están haciendo
tanto alboroto con tanto llanto? La niña no está muerta, ella solamente
está durmiendo''.

\bibleverse{40} Entonces todos se rieron de él con menosprecio. Jesús
mandó a todos salir. Entonces entró a la habitación donde estabala niña,
llevando consigo al padre y a la madre de la niña y a tres discípulos.
\bibleverse{41} Luego sostuvo la mano de la niña y dijo: ``Talitha
koum'', que quiere decir: ``Pequeña niña, ¡levántate!'' \footnote{\textbf{5:41}
  Luc 7,14; Hech 9,40}

\bibleverse{42} La niña, que tenía doce años, se levantó de inmediato y
comenzó a caminar. Todos estaban completamente asombrados de lo que
había ocurrido. \bibleverse{43} Entonces él les dio orden estricta de no
contárselo a nadie, y les dijo que le dieran algo de comer a la niña.

\hypertarget{rechazo-y-fracaso-de-jesuxfas-en-su-natal-nazaret}{%
\subsection{Rechazo y fracaso de Jesús en su natal
Nazaret}\label{rechazo-y-fracaso-de-jesuxfas-en-su-natal-nazaret}}

\hypertarget{section-5}{%
\section{6}\label{section-5}}

\bibleverse{1} Jesús se fue de allí y se dirigió a Nazaret con sus
discípulos. \bibleverse{2} El sábado, comenzó a enseñar en la sinagoga,
y muchos de los que estaban allí escuchando estaban sorprendidos. ``¿De
dónde saca tales ideas?'' preguntaban. ``¿De dónde recibe tal sabiduría?
¿De dónde saca el poder para hacer milagros? \bibleverse{3} ¿Acaso no es
este el carpintero, el hijo de María, hermano de Santiago, José, Judas y
Simón? ¿No viven sus hermanas entre nosotros?'' Entonces se sintieron
ofendidos y lo rechazaron.\footnote{\textbf{6:3} ``Y lo rechazaron'':
  implicado en la idea de ofenderse.} \footnote{\textbf{6:3} Juan 6,42}

\bibleverse{4} ``Un profeta es tratado con respeto excepto en su propia
ciudad, entre sus familiares, y en su propia familia'', les dijo Jesús.
\bibleverse{5} El resultado fue que Jesús no podía hacer milagros allí,
sino apenas sanar a algunas personas enfermas. \bibleverse{6} Y estaba
sorprendido por su falta de fe. Jesús viajaba por las ciudades cercanas,
enseñando a su paso.

\hypertarget{enviar-e-instruir-a-los-doce-discuxedpulos}{%
\subsection{Enviar e instruir a los doce
discípulos}\label{enviar-e-instruir-a-los-doce-discuxedpulos}}

\bibleverse{7} Reunió a los doce discípulos y comenzó a enviarlos de dos
en dos, dándoles autoridad sobre los malos espíritus. \bibleverse{8} Les
dijo que no llevaran nada con ellos, excepto un bastón. No llevarían
pan, ni bolsas, ni dinero en sus cinturones. \bibleverse{9} Podían usar
sandalias, pero no debían llevar ropa adicional. \bibleverse{10}
``Cuando sean invitados a una casa, quédense allí hasta marcharse'', les
dijo. \bibleverse{11} ``Si no son bien recibidos ni escuchados, entonces
sacudan el polvo de sus pies al salir como señal de que han desistido de
ellos''.\footnote{\textbf{6:11} Literalmente, ``testigo de ellos''. El
  acto de sacudirse de los pies incluso el polvo de un lugar, indicaba
  el completo abandono del mismo.}

\bibleverse{12} Así que los discípulos iban por todos lados llamando a
las personas al arrepentimiento. \bibleverse{13} Expulsaron muchos
demonios, y sanaron a muchos que estaban enfermos, ungiéndolos con
aceite. \footnote{\textbf{6:13} Sant 5,14; Sant 1,5-15}

\hypertarget{el-juicio-de-herodes-sobre-jesuxfas-el-fin-de-juan-el-bautista}{%
\subsection{El juicio de Herodes sobre Jesús; el fin de Juan el
Bautista}\label{el-juicio-de-herodes-sobre-jesuxfas-el-fin-de-juan-el-bautista}}

\bibleverse{14} El Rey Herodes llegó a escuchar sobre Jesús desde que se
había vuelto reconocido. Algunos decían: ``Este es Juan el Bautista que
ha resucitado de entre los muertos. Por eso tiene tales poderes
milagrosos''. \bibleverse{15} Otros decían: ``Es Elías''. Y otros
también decían: ``Es un profeta, como los profetas del pasado''.
\bibleverse{16} Pero cuando Herodes escuchó esto, dijo: ``¡Es Juan, el
que yo decapité! ¡Ha regresado de entre los muertos!'' \bibleverse{17}
Pues Herodes había dado órdenes de arrestar y encarcelar a Juan por
causa de Herodías, la esposa de su hermano Felipe, con la cual él se
había casado. \bibleverse{18} Juan le había dicho a Herodes: ``Es contra
la ley casarte con la esposa de tu hermano''. \bibleverse{19} De modo
que Herodías tenía resentimiento contra Juan y quería que lo mataran.
Pero ella no era capaz de mandar a hacerlo \bibleverse{20} porque
Herodes sabía que Juan era un hombre santo que hacía lo recto. Herodes
protegía a Juan y, aunque lo que Juan le decía lo inquietaba, aun así
Herodes se complacía en escuchar lo que él decía.

\bibleverse{21} Herodías tuvo su oportunidad en ocasión del cumpleaños
de Herodes. Él ofreció un banquete para los nobles, los oficiales
militares y líderes importantes de Galilea. \bibleverse{22} Entonces la
hija de Herodías llegó y comenzó a danzar para ellos. Herodes y los que
estaban comiendo con él estaban deleitados por su presentación, así que
le dijo a la niña: \bibleverse{23} ``Pídeme lo que quieras, y te lo
daré''. Y confirmó su promesa con un juramento, ``Te daré hasta la mitad
de mi reino''. \footnote{\textbf{6:23} Est 5,3; Est 5,6}

\bibleverse{24} Ella salió y le preguntó a su madre: ``¿Qué debo
pedir?'' ``La cabeza de Juan el Bautista'', respondió ella.

\bibleverse{25} La joven se apresuró para regresar y le dijo al rey:
``Quiero que me des ahora la cabeza de Juan el Bautista en un plato''.

\bibleverse{26} El rey estaba muy descontento por esto, pero como había
hecho juramento frente a sus invitados, no quiso negarse a lo que ella
pidió. \bibleverse{27} Así que inmediatamente envió a un verdugo para
que le trajera la cabeza de Juan. Después de decapitarlo en la prisión,
\bibleverse{28} el verdugo trajo la cabeza de Juan en un plato y se lo
entregó a la niña, y ella se lo entregó a su madre.

\bibleverse{29} Cuando los discípulos de Jesús oyeron lo que había
ocurrido, vinieron y tomaron su cuerpo y lo colocaron en una tumba.

\hypertarget{regreso-de-los-doce-apuxf3stoles-jesuxfas-escapa-a-la-soledad-alimentando-a-los-cinco-mil}{%
\subsection{Regreso de los doce apóstoles; Jesús escapa a la soledad;
Alimentando a los cinco
mil}\label{regreso-de-los-doce-apuxf3stoles-jesuxfas-escapa-a-la-soledad-alimentando-a-los-cinco-mil}}

\bibleverse{30} Los apóstoles regresaron\footnote{\textbf{6:30} De su
  recorrido por las ciudades, predicando la Buena Noticia.} y se
reunieron alrededor de Jesús. Le contaron todo lo que habían hecho y lo
que habían enseñado. \bibleverse{31} ``Vengan conmigo, solo ustedes.
Iremos a un lugar tranquilo, y descansaremos un poco'', les dijo Jesús,
pues surgían tantas cosas por todas partes que ni siquiera tenían tiempo
de comer. \bibleverse{32} Así que se fueron en una barca a un lugar
tranquilo para estar a solas. \bibleverse{33} Pero la gente los vio
partir y los reconocieron. Así que las personas de todas las ciudades
vecinas se apresuraron para seguirlos. \bibleverse{34} Cuando Jesús
descendió de la barca, vio una gran multitud, y entonces sintió
compasión de ellos, porque estaban como ovejas sin pastor. Así que
comenzó a enseñarles sobre muchas cosas. \footnote{\textbf{6:34} Mat
  9,36} \bibleverse{35} Ya se hacía tarde ese día y los discípulos de
Jesús vinieron donde él estaba. Y le dijeron: ``Estamos a millas de
distancia y es muy tarde. \footnote{\textbf{6:35} Mar 8,1-9}
\bibleverse{36} Deberías decirles a las personas que se vayan y compren
alimentos en las aldeas y pueblos cercanos''.

\bibleverse{37} Pero Jesús respondió: ``Denles ustedes de comer''.
``¿Qué? Necesitaríamos más de seis meses de salario\footnote{\textbf{6:37}
  Literalmente, ``200 denarios''.} para comprar pan para alimentar a
todas estas personas'', respondieron los discípulos.

\bibleverse{38} ``Bueno, ¿cuánto pan tienen allí?'' preguntó Jesús.
``Vayan y vean''. Entonces ellos fueron y revisaron, y le dijeron:
``Cinco panes, y un par de peces''.

\bibleverse{39} Jesús ordenó a todos que se sentaran en grupos sobre la
hierba verde. \bibleverse{40} Ellos se sentaron en grupos de cien y de
cincuenta. \bibleverse{41} Entonces Jesús tomó los cinco panes y los dos
peces. Mirando al cielo bendijo el alimento y partió el pan en pedazos.
Entonces lo entregó a los discípulos para que lo repartieran entre las
personas, y dividió los peces entre todos ellos. \footnote{\textbf{6:41}
  Mar 7,34} \bibleverse{42} Todos comieron hasta que quedaron saciados.
\bibleverse{43} Entonces recogieron las sobras de los panes y los peces:
doce canastas. \bibleverse{44} Un total de cinco mil hombres además de
sus familias, comieron de esa comida.

\hypertarget{regrese-a-travuxe9s-del-lago-por-la-noche-el-caminar-de-jesuxfas-sobre-el-lago-el-desembarco-en-gennesaret}{%
\subsection{Regrese a través del lago por la noche; el caminar de Jesús
sobre el lago; el desembarco en
Gennesaret}\label{regrese-a-travuxe9s-del-lago-por-la-noche-el-caminar-de-jesuxfas-sobre-el-lago-el-desembarco-en-gennesaret}}

\bibleverse{45} Inmediatamente después de esto, Jesús dio órdenes a sus
discípulos de regresar a la barca. Irían a Betsaida, al otro lado del
lago, mientras él despedía a las personas para que se fueran.
\bibleverse{46} Cuando terminó de despedirse, subió a las montañas para
orar.

\bibleverse{47} Ya tarde en la noche la barca estaba en medio del lago,
mientras Jesús aún estaba a solas en tierra. \bibleverse{48} Pudo ver
que estaban a merced del mar mientras remaban, pues el viento soplaba
contra ellos. En las primeras horas de la mañana, Jesús se acercó a
ellos, caminando sobre el agua. Ya los iba a alcanzar, \bibleverse{49}
pero cuando lo vieron caminando sobre el agua, los discípulos pensaron
que era un fantasma. Entonces gritaron \bibleverse{50} porque todos
podían verlo y estaban completamente horrorizados. Inmediatamente Jesús
les dijo: ``No se preocupen, soy yo. ¡No tengan miedo!'' \bibleverse{51}
Entonces se dirigió hacia ellos y subió a la barca, y el viento se
detuvo. Todos estaban sorprendidos, \bibleverse{52} pues no habían
entendido el significado del milagro de la alimentación de la multitud
por su terquedad y por la dureza de sus corazones. \footnote{\textbf{6:52}
  Mar 8,17}

\bibleverse{53} Tras haber cruzado el Mar, llegaron a Genezaret y allí
anclaron la barca. \bibleverse{54} Mientras subían, la gente enseguida
reconoció a Jesús. \bibleverse{55} Y corrían por todas partes en la
región para traer los enfermos en sus camillas hasta el lugar donde
Jesús estaba. \bibleverse{56} Dondequiera que él iba, en las aldeas, en
las ciudades o en los campos, la gente ponía a los enfermos en las
plazas del mercado y le rogaban a Jesús que permitiera que los enfermos
tocaran aunque fuera la punta de sus vestiduras. Y todos los que lo
tocaban quedaban sanos.

\hypertarget{pelea-con-los-oponentes-sobre-el-lavado-de-manos-advertencia-de-estatutos-humanos-y-marcado-de-verdadera-impureza}{%
\subsection{Pelea con los oponentes sobre el lavado de manos;
Advertencia de estatutos humanos y marcado de verdadera
impureza}\label{pelea-con-los-oponentes-sobre-el-lavado-de-manos-advertencia-de-estatutos-humanos-y-marcado-de-verdadera-impureza}}

\hypertarget{section-6}{%
\section{7}\label{section-6}}

\bibleverse{1} Los Fariseos y líderes religiosos que habían descendido
desde Jerusalén para conocer a Jesús \bibleverse{2} se dieron cuenta de
que algunos de sus discípulos comían con las manos ``impuras'' (es
decir, sin lavar). \bibleverse{3} (Los Fariseos y los judíos no comen
sin lavarse las manos previamente, siguiendo la tradición de sus
ancestros. \bibleverse{4} De la misma manera, no comen cuando regresan
del mercado hasta que no se hayan lavado. Observan muchos otros
rituales, como lavar las tazas, las ollas y los recipientes\footnote{\textbf{7:4}
  Aunque es un hábito higiénico, el enfoque estaba en asegurarse de que
  todo estuviera ceremonialmente limpio.} ). \footnote{\textbf{7:4} Mat
  23,25} \bibleverse{5} Entonces los Fariseos y los líderes religiosos
le preguntaron a Jesús: ``¿Por qué tus discípulos no siguen la tradición
de nuestros ancestros? Pues ellos comen los alimentos\footnote{\textbf{7:5}
  Literalmente, ``pan''.} con las manos impuras''.

\bibleverse{6} Jesús respondió: ``Isaías tenía la razón sobre ustedes,
pueblo hipócrita, cuando dijo: `Este pueblo dice que me honra, pero sus
mentes están lejos de mí. \bibleverse{7} Su adoración no tiene sentido,
pues lo que enseñan como doctrinas son solamente normas
humanas'.\footnote{\textbf{7:7} Citando Isaías 29:13.}

\bibleverse{8} Ustedes ignoran la ley de Dios, y por el contrario
observan con mucho cuidado las tradiciones humanas'', les dijo.
\bibleverse{9} ``¡Con cuanta astucia ustedes han dejado a un lado la ley
de Dios para poder fundamentar sus tradiciones! \bibleverse{10} Moisés
dijo: `Honra a tu padre y a tu madre;' y también dijo: `Cualquiera que
maldice a su padre o a su madre, debe morir'.\footnote{\textbf{7:10}
  Citando Éxodo 20:12; Éxodo 21:17.} \bibleverse{11} Pero ustedes dicen
`está bien si alguien dice a su padre y a su madre, Todo lo que recibían
de mí ahora es Corbán', (es decir, dedicado a Dios), \bibleverse{12} y
entonces ustedes no permiten que ellos hagan nada más en favor de su
padre o su madre. \bibleverse{13} Y por medio de esta tradición que
ustedes han transmitido, anulan e invalidan la palabra de Dios. Y
ustedes hacen muchas otras cosas como estas''.

\bibleverse{14} Una vez más Jesús llamó a la multitud para que vinieran
donde él estaba y les dijo: ``Por favor, escúchenme todos y entiendan.
\bibleverse{15} No es lo que está afuera y entra por su boca lo que los
hace impuros. Es lo que sale de ella lo que los hace impuros''.
\bibleverse{16} \footnote{\textbf{7:16} Los primeros manuscritos no
  incluían el versículo16.}

\bibleverse{17} Entonces Jesús entró para alejarse de la multitud, y sus
discípulos le preguntaron sobre esta ilustración. \bibleverse{18}
``¿Ustedes tampoco lo entienden?'' les preguntó. ``¿No ven que lo que
ustedes comen no es lo que los vuelve impuros? \bibleverse{19} Lo que
comen no entra en sus mentes, sino en sus estómagos, y luego sale del
cuerpo. Así que todos los alimentos están ceremonialmente
`limpios'.\footnote{\textbf{7:19} Algunos eruditos creen que esta
  oración se agregó mucho después.} \bibleverse{20} ``Lo que sale de
ustedes es lo que los hace impuros. \bibleverse{21} Es desde adentro,
desde la mente de las personas, que salen los malos pensamientos: la
inmoralidad sexual, los robos, los asesinatos, el adulterio,
\bibleverse{22} la glotonería, la malicia, el engaño, la indecencia, la
envidia, la calumnia, el orgullo y la inconsciencia. \bibleverse{23}
Todos estos males vienen desde adentro y contaminan a las personas''.

\hypertarget{jesuxfas-y-la-sirofenicia-en-el-uxe1rea-de-tiro-y-siduxf3n}{%
\subsection{Jesús y la sirofenicia en el área de Tiro y
Sidón}\label{jesuxfas-y-la-sirofenicia-en-el-uxe1rea-de-tiro-y-siduxf3n}}

\bibleverse{24} Entonces Jesús se fue de allí hacia la región de Tiro.
Él no quería que nadie supiera que estaba quedándose allí en una casa,
pero no pudo mantenerlo en secreto. \bibleverse{25} Tan pronto como una
mujer, cuya hija tenía un espíritu maligno, escuchó acerca de él, vino y
se postró a sus pies. \bibleverse{26} La mujer era griega, nacida en
Sirofenicia. Ella le rogó a Jesús que sacara ese demonio de su hija.
\bibleverse{27} ``Deja que los hijos coman primero hasta que estén
satisfechos'', respondió Jesús. ``No está bien tomar el alimento de los
hijos para tirárselo a los perros''.

\bibleverse{28} ``Eso es cierto, señor'', dijo ella, ``pero aún los
perros que están debajo de la mesa comen de las migajas que los hijos
dejan''.

\bibleverse{29} Entonces Jesús le dijo: ``Por semejante respuesta que
has dado, puedes irte, el demonio ha salido de tu hija''.

\bibleverse{30} Y ella se fue a casa y encontró a la niña acostada en la
cama, y el demonio se había ido.

\hypertarget{el-regreso-de-jesuxfas-a-galilea-en-la-orilla-oriental-del-lago-sanando-a-un-sordomudo}{%
\subsection{El regreso de Jesús a Galilea en la orilla oriental del
lago; Sanando a un
sordomudo}\label{el-regreso-de-jesuxfas-a-galilea-en-la-orilla-oriental-del-lago-sanando-a-un-sordomudo}}

\bibleverse{31} Al salir de la región de Tiro, Jesús pasó por Sidón y
luego por el Mar de Galilea y por el territorio de las Diez Ciudades.
\footnote{\textbf{7:31} Mar 5,20; Mat 15,29-31}

\bibleverse{32} Allí le trajeron a un hombre sordo que tampoco podía
hablar bien. Ellos le pidieron a Jesús que tocara al hombre con su mano
y lo sanara. \bibleverse{33} Y después de llevarlo aparte, lejos de la
multitud, Jesús puso sus dedos en los oídos del hombre sordo. Entonces
tocó la lengua del hombre con saliva. \bibleverse{34} Luego miró al
cielo, y con un suspiro dijo: ``Efata'',\footnote{\textbf{7:34} Este es
  el término arameo que significa ``haz que se abra''.} que quiere
decir, ``¡ábrete!'' \bibleverse{35} Y los oídos del hombre se abrieron,
y ya no tenía impedimento para hablar, y comenzó a hablar con claridad.
\bibleverse{36} Entonces Jesús dio órdenes estrictas de no contarlo a
nadie, pero cuanto más él decía esto, tanto más la gente difundía la
noticia. \bibleverse{37} La gente estaba completamente asombrada y
decían: ``Todo lo que él hace es maravilloso. Incluso hace que los
sordos oigan y que los mudos puedan hablar''.

\hypertarget{alimentando-a-los-cuatro-mil}{%
\subsection{Alimentando a los cuatro
mil}\label{alimentando-a-los-cuatro-mil}}

\hypertarget{section-7}{%
\section{8}\label{section-7}}

\bibleverse{1} En esos días se reunió otra gran multitud y de nuevo no
tenían nada para comer. Entonces Jesús reunió a los discípulos y les
dijo: \bibleverse{2} ``Me da gran pesar por ellos porque ya han estado
aquí conmigo por tres días y no tienen nada para comer. \footnote{\textbf{8:2}
  Mar 6,34-44} \bibleverse{3} Si los despido sin comer, se desmayarán en
el camino. Y algunos han venido desde muy lejos''.

\bibleverse{4} ``¿Dónde podría alguien encontrar suficiente pan para
alimentarlos aquí, en este desierto?'' respondieron sus discípulos.

\bibleverse{5} ``¿Cuántos panes tienen?'' preguntó Jesús. ``Siete'',
respondieron ellos.

\bibleverse{6} Entonces pidió a la multitud que se sentaran en el suelo.
Luego tomó los siete panes y dio gracias. Partió el pan y entregó los
trozos de pan a sus discípulos para que los dieran a la multitud.
\bibleverse{7} También tenían un pescado, así que después de bendecirlo,
dijo: ``Tomen estos y compártanlos también''. \bibleverse{8} Y comieron
hasta que quedaron saciados, y luego recogieron siete canastas con lo
que había sobrado. \bibleverse{9} Había allí cuatro mil personas. Y
después de despedirlos,

\hypertarget{el-rechazo-de-jesuxfas-a-la-demanda-de-seuxf1ales-de-los-fariseos}{%
\subsection{El rechazo de Jesús a la demanda de señales de los
fariseos}\label{el-rechazo-de-jesuxfas-a-la-demanda-de-seuxf1ales-de-los-fariseos}}

\bibleverse{10} Jesús subió a una barca con sus discípulos y se dirigió
a la región de Dalmanuta. \bibleverse{11} Los Fariseos llegaron y
comenzaron a discutir con él, queriendo que les mostrara alguna señal
milagrosa del cielo, tratando así de probarlo. \bibleverse{12} Entonces
Jesús suspiró profundamente y preguntó: ``¿Por qué la gente\footnote{\textbf{8:12}
  Literalmente, ``esta generación''.} quiere una señal? Les digo la
verdad: No les daré una señal''.

\bibleverse{13} Entonces los dejó allí, subió a la barca, y volvió a
cruzar el lago.

\hypertarget{advertencia-de-la-levadura-de-los-fariseos-y-la-de-herodes}{%
\subsection{Advertencia de la levadura de los fariseos y la de
Herodes}\label{advertencia-de-la-levadura-de-los-fariseos-y-la-de-herodes}}

\bibleverse{14} Pero los discípulos habían olvidado llevar pan. Lo único
que tenían en la barca era un solo pan. \bibleverse{15} ``¡Estén alerta
y cuídense de la levadura de los Fariseos y de Herodes!'' les advirtió.
\footnote{\textbf{8:15} Luc 12,1; Mar 3,6}

\bibleverse{16} ``Él lo dice porque no trajimos pan'', concluyeron
ellos.

\bibleverse{17} Pero Jesús sabía lo que ellos estaban diciendo y dijo:
``¿Por qué están hablando acerca del pan que no trajeron? ¿Aún no están
pensando ni están entendiendo? ¿Han cerrado sus mentes?\footnote{\textbf{8:17}
  Literalmente, ``¿Han endurecido sus corazones?''} \bibleverse{18}
¿Acaso no tienen ojos para ver y oídos para oír?\footnote{\textbf{8:18}
  Una expresión del Antiguo Testamento: ver Deuteronomio 29:4, Isaías
  42:20, Jeremías 5:21 y Ezequiel 12:2.} \footnote{\textbf{8:18} Mat
  13,13; Mat 13,16} \bibleverse{19} ¿No recuerdan que repartí cinco
panes entre cinco mil personas? ¿Cuántas canastas sobrantes
recogieron?'' ``Doce'', respondieron ellos \footnote{\textbf{8:19} Mar
  6,41-44}

\bibleverse{20} ``Y los siete panes que se repartieron entre cuatro mil.
¿Cuántas canastas sobrantes recogieron ustedes?'' ``Siete'',
respondieron.

\bibleverse{21} ``¿Aún no entienden?'' les preguntó.

\hypertarget{curaciuxf3n-de-ciegos-en-betsaida}{%
\subsection{Curación de ciegos en
Betsaida}\label{curaciuxf3n-de-ciegos-en-betsaida}}

\bibleverse{22} Entonces partieron hacia Betsaida, donde unas personas
trajeron a un hombre ciego ante Jesús. Ellos le rogaban a Jesús que lo
tocara y lo sanara. \footnote{\textbf{8:22} Mar 6,56} \bibleverse{23}
Entonces Jesús tomó al hombre ciego por la mano y lo llevó fuera de la
aldea. Luego escupió en los ojos del hombre y lo tocó con sus manos.
Entonces Jesús le preguntó: ``¿Puedes ver?'' \footnote{\textbf{8:23}
  Juan 9,6}

\bibleverse{24} El hombre miró a su alrededor, y dijo: ``Puedo ver a la
gente, pero lucen como árboles que caminan''.

\bibleverse{25} Entonces Jesús tocó una vez más los ojos del hombre, y
pudo ver claramente. Había sido curado y su vista estaba clara.
\bibleverse{26} Entonces Jesús envió al hombre a su casa, y le dijo:
``No pases de regreso por la aldea''.\footnote{\textbf{8:26} En otras
  palabras, no difundan la noticia de lo que ha pasado.} \footnote{\textbf{8:26}
  Mar 7,36}

\hypertarget{la-confesiuxf3n-de-pedro-del-mesuxedas}{%
\subsection{La confesión de Pedro del
Mesías}\label{la-confesiuxf3n-de-pedro-del-mesuxedas}}

\bibleverse{27} Jesús y sus discípulos se marcharon para ir a las aldeas
de Cesarea de Filipo. Y cuando iban de camino, le preguntó a sus
discípulos: ``¿Quién dice la gente que soy?''

\bibleverse{28} ``Algunos dicen que eres Juan el Bautista, otros dicen
que eres Elías, y otros dicen que eres uno de los profetas'',
respondieron ellos.

\bibleverse{29} ``¿Pero quién dicen ustedes que soy yo?'' les preguntó.
``¡Tú eres el Mesías!'' respondió Pedro.

\bibleverse{30} Jesús les advirtió acerca de no contarle a nadie sobre
él. \footnote{\textbf{8:30} Mar 9,9}

\hypertarget{el-primer-anuncio-del-sufrimiento-de-jesuxfas}{%
\subsection{El primer anuncio del sufrimiento de
Jesús}\label{el-primer-anuncio-del-sufrimiento-de-jesuxfas}}

\bibleverse{31} Entonces comenzó a explicarles que el Hijo del hombre
sufriría muchas cosas y sería rechazado por los ancianos, por los sumos
sacerdotes, y por los maestros religiosos. Sería llevado a la muerte,
pero tres días después se levantaría de nuevo. \bibleverse{32} Jesús les
explicaba esto de manera muy clara. Pero Pedro lo llevó aparte y comenzó
a amonestarlo por decir tales cosas. \bibleverse{33} Entonces Jesús se
dio vuelta y mirando a sus discípulos, reprendió a Pedro. ``Apártate de
mí, Satanás'', dijo. ``No estás pensando como Dios piensa, sino como
piensan los humanos''.

\hypertarget{proverbios-sobre-el-seguimiento-de-los-discuxedpulos-en-el-sufrimiento}{%
\subsection{Proverbios sobre el seguimiento de los discípulos en el
sufrimiento}\label{proverbios-sobre-el-seguimiento-de-los-discuxedpulos-en-el-sufrimiento}}

\bibleverse{34} Jesús entonces llamó a la multitud y a sus discípulos
para que se acercaran a él, y les dijo: ``Si alguno quiere seguirme,
debe renunciar a sí mismo, cargar su cruz y entonces seguirme.
\bibleverse{35} Si alguno quiere salvar su vida, la perderá, pero si
alguno pierde su vida por mi causa y por causa de la Buena Noticia, la
salvará. \bibleverse{36} ``¿De qué le servirá a alguien ganar todo en el
mundo, y perder su vida? \bibleverse{37} ¿Qué darían ustedes a cambio de
su vida? \bibleverse{38} Si ustedes sienten vergüenza de reconocerme a
mí\footnote{\textbf{8:38} ``Avergonzados de reconocerme'', o, ``no se
  declaran de mi parte''.} y lo que yo digo entre este pueblo infiel y
pecaminoso,\footnote{\textbf{8:38} Literalmente, ``generación''.}
entonces el Hijo del hombre se avergonzará de ustedes cuando venga con
la gloria de su Padre, con los santos ángeles''.\footnote{\textbf{8:38}
  Mat 10,33}

\hypertarget{section-8}{%
\section{9}\label{section-8}}

\bibleverse{1} Jesús les dijo: ``Les digo la verdad: algunos de los que
están aquí no morirán antes de que vean venir al reino de Dios con
poder''.

\hypertarget{la-transfiguraciuxf3n-de-jesuxfas-en-la-montauxf1a-y-su-conversaciuxf3n-con-los-discuxedpulos-en-el-descenso}{%
\subsection{La transfiguración de Jesús en la montaña y su conversación
con los discípulos en el
descenso}\label{la-transfiguraciuxf3n-de-jesuxfas-en-la-montauxf1a-y-su-conversaciuxf3n-con-los-discuxedpulos-en-el-descenso}}

\bibleverse{2} Seis días más tarde, Jesús llevó consigo a Pedro,
Santiago y Juan, y los condujo a lo alto de una montaña para estar allí
a solas. Su apariencia cambió por completo. \bibleverse{3} Sus
vestiduras brillaban de lo blancas que estaban, más blancas de lo que
cualquier persona sobre la tierra podría emblanquecer. \bibleverse{4}
Entonces Elías y Moisés se aparecieron frente a ellos también, y
hablaban con Jesús.

\bibleverse{5} Pedro alzó la voz y dijo: ``¡Rabí, para nosotros es
maravilloso estar aquí! Deberíamos preparar tres albergues. Cada uno de
nosotros haría uno: para ti, para Moisés y para Elías''. \bibleverse{6}
(¡En realidad, él no sabía qué decir porque los tres discípulos estaban
muy asustados!)

\bibleverse{7} Entonces una nube los cubrió,\footnote{\textbf{9:7} O,
  ``hizo sombra''.} y de la nube salió una voz que decía: ``Este es mi
Hijo, al que amo. Escúchenlo''.

\bibleverse{8} De repente, mientras los discípulos miraban, ya no había
nadie. Solo Jesús estaba con ellos.

\bibleverse{9} Mientras descendían de la montaña, Jesús les dio
instrucciones de no contarle a nadie lo que habían visto, hasta que el
Hijo del hombre se hubiese levantado de entre los muertos. \footnote{\textbf{9:9}
  Mar 8,30} \bibleverse{10} Ellos guardaron esto para sí, pero discutían
sobre qué significaba eso de levantarse de entre los muertos.

\bibleverse{11} ``¿Por qué los maestros religiosos afirman que Elías
tiene que venir primero?'' le preguntaron.

\bibleverse{12} ``Es cierto que Elías viene primero para prepararlo
todo'', respondió Jesús. ``Pero, ¿por qué, entonces, dicen las
Escrituras que el Hijo del hombre tiene que sufrir mucho y ser tratado
con deprecio? \bibleverse{13} Sin embargo, les digo que Elías vino, y
ellos lo maltrataron de todas las formas que quisieron, tal como las
Escrituras dicen que lo harían''. \footnote{\textbf{9:13} Mat 11,14; 1Re
  19,2; 1Re 19,10}

\hypertarget{curaciuxf3n-de-un-niuxf1o-epiluxe9ptico-la-incapacidad-de-los-discuxedpulos}{%
\subsection{Curación de un niño epiléptico; la incapacidad de los
discípulos}\label{curaciuxf3n-de-un-niuxf1o-epiluxe9ptico-la-incapacidad-de-los-discuxedpulos}}

\bibleverse{14} Cuando regresaron donde estaban los demás discípulos,
vieron que estaban rodeados de una gran multitud y había allí algunos
maestros religiosos discutiendo con ellos. \bibleverse{15} Pero tan
pronto como la multitud vio a Jesús, se asombraron, y corrieron a
recibirlo. \bibleverse{16} ``¿Sobre qué están discutiendo ustedes con
ellos?'' les preguntó Jesús.

\bibleverse{17} Una de las personas de la multitud respondió: ``Maestro,
te traje a mi hijo. Él tiene un espíritu malo que no lo deja hablar.
\bibleverse{18} Cada vez que lo ataca, lo tira al suelo, y lo hace botar
espuma por la boca, cruje sus dientes y su cuerpo se pone rígido. Le
pedí a tus discípulos que sacaran este demonio de él, pero ellos no
pudieron hacerlo''.

\bibleverse{19} ``¡Pueblo incrédulo!'' respondió Jesús. ``¿Por cuánto
tiempo debo permanecer aquí con ustedes? ¿Por cuánto tiempo tengo que
soportarlos? ¡Tráiganmelo aquí!''

\bibleverse{20} Así que ellos lo trajeron donde Jesús. Cuando el
espíritu malo vio a Jesús, de inmediato le produjo convulsiones al joven
y lo lanzó al suelo, donde este comenzó a rodar de un lado al otro y a
botar espuma por la boca.

\bibleverse{21} ``¿Por cuánto tiempo ha tenido esto?'' le preguntó Jesús
al padre del joven. ``Desde que era pequeño'', respondió el padre.

\bibleverse{22} ``A menudo lo lanza al fuego para quemarlo y matarlo, o
lo lanza al agua para ahogarlo. Por favor, ten misericordia de nosotros
y ayúdanos, si puedes''.

\bibleverse{23} ``¿Por qué dices,\footnote{\textbf{9:23} Implícito.
  Jesús estaba preguntando por qué el hombre le había dicho ``si
  puedes'', lo cual podría sugerir que había dudas respecto a lo que
  Jesús podía hacer.} `si puedes?'\,'' respondió Jesús. ``¡Todo es
posible para el que cree!''

\bibleverse{24} ``Yo creo en ti'', gritó el hombre de inmediato.
``Ayúdame a no desconfiar de ti''.

\bibleverse{25} Jesús, viendo que la multitud se aproximaba
más,\footnote{\textbf{9:25} O, ``corrían todos a la vez''. Esta palabra
  se usa en el Nuevo Testamento una sola vez.} le dio orden al espíritu
malo: ``Espíritu que causa sordera y mudez, te ordeno que salgas de él y
no regreses más''.

\bibleverse{26} El espíritu gritó y lanzó al joven al piso, causándole
graves convulsiones. Entonces salió del joven y lo dejó casi muerto, al
punto que muchas de las personas decían: ``Está muerto''.
\bibleverse{27} Pero Jesús tomó al joven por la mano y lo ayudó a
levantarse, y éste se puso en pie.

\bibleverse{28} Después, cuando Jesús estaba en casa, sus discípulos le
preguntaron en privado: ``¿Por qué nosotros no pudimos sacar al
espíritu?''

\bibleverse{29} ``Este tipo de espíritu no puede sacarse si no es con
oración'', les dijo Jesús.

\hypertarget{segundo-anuncio-de-sufrimiento}{%
\subsection{Segundo anuncio de
sufrimiento}\label{segundo-anuncio-de-sufrimiento}}

\bibleverse{30} Entonces se marcharon y pasaron por Galilea. Jesús no
quería que nadie supiera donde estaba él \bibleverse{31} porque estaba
enseñándole a sus discípulos.\footnote{\textbf{9:31} En otras palabras,
  él quería dedicar tiempo para enseñarle a los discípulos.} ``El Hijo
del hombre será entregado a las autoridades humanas'', les dijo. ``Ellos
lo matarán, pero tres días después se levantará de nuevo''. \footnote{\textbf{9:31}
  Mar 8,31; Mar 10,32-34}

\bibleverse{32} Pero ellos no entendieron lo que él quiso decir y tenían
mucho miedo como para preguntarle al respecto. \footnote{\textbf{9:32}
  Luc 18,34}

\hypertarget{controversia-entre-discuxedpulos-la-exhortaciuxf3n-de-jesuxfas-a-la-humildad}{%
\subsection{Controversia entre discípulos; La exhortación de Jesús a la
humildad}\label{controversia-entre-discuxedpulos-la-exhortaciuxf3n-de-jesuxfas-a-la-humildad}}

\bibleverse{33} Llegaron a Capernaum, y cuando estaban dentro de la casa
donde se hospedaban, Jesús les preguntó: ``¿De qué venían hablando
durante el camino?''

\bibleverse{34} Pero ellos no dijeron nada porque habían estado
discutiendo sobre quién de ellos era el más importante.

\bibleverse{35} Entonces Jesús se sentó y reunió a sus discípulos. ``Si
alguno quiere ser el primero, tendrá que ser el último, el siervo de
todos los demás'', les dijo. \footnote{\textbf{9:35} Mar 10,44; Mat
  20,27} \bibleverse{36} Luego tomó a un niño pequeño y lo hizo sentarse
justo en medio de ellos. Entonces tomó al niño y lo abrazó, y les dijo:
\bibleverse{37} ``Cualquiera que recibe a un niño como este en mi
nombre, me recibe a mí, y cualquiera que me recibe a mí, no me recibe a
mí, sino al que me envió''.

\hypertarget{enseuxf1ar-sobre-la-tolerancia}{%
\subsection{Enseñar sobre la
tolerancia}\label{enseuxf1ar-sobre-la-tolerancia}}

\bibleverse{38} Juan le dijo a Jesús: ``Maestro, vimos a alguien sacando
demonios en tu nombre. Nosotros tratamos de detenerlo, porque no era uno
de nosotros''. \footnote{\textbf{9:38} Núm 11,27-28}

\bibleverse{39} ``No lo detengan'', respondió Jesús. ``Porque ninguno
que esté haciendo milagros en mi nombre, puede maldecir al mismo tiempo.
\footnote{\textbf{9:39} 1Cor 12,3} \bibleverse{40} El que no está contra
nosotros, está a favor de nosotros. \footnote{\textbf{9:40} Mat 12,30;
  Luc 11,23} \bibleverse{41} Todo el que les brinde un vaso de agua a
ustedes en mi nombre, porque ustedes pertenecen a Cristo, créanme que no
perderá su recompensa. \footnote{\textbf{9:41} Mat 10,42}

\hypertarget{advertencia-de-engauxf1o-a-la-incredulidad-y-al-pecado-dichos-de-sal}{%
\subsection{Advertencia de engaño (a la incredulidad y al pecado);
Dichos de
sal}\label{advertencia-de-engauxf1o-a-la-incredulidad-y-al-pecado-dichos-de-sal}}

\bibleverse{42} ``Pero si cualquiera conduce a uno de estos pequeños que
creen en mí, a pecar, mejor sería que fuera lanzado al mar con una
piedra de molino atada en su cuello. \bibleverse{43} Si una mano te hace
pecar, -¡córtala! Es mejor entrar a la vida eterna como un lisiado que
ir con ambas manos al Gehena,\footnote{\textbf{9:43} La palabra usada
  aquí es literalmente ``Gehena'', que a veces se traduce como
  ``infierno'' o ``llamas del infierno''. Gehena era el lugar que estaba
  a las afueras de Jerusalén, en donde se prendía fuego para quemar la
  basura. El concepto de ``Infierno'' se deriva de la mitología nórdica
  y anglosajona y no expresa apropiadamente el significado de este
  texto. Ver nota en Mateo 5:22.} al fuego que no puede apagarse.
\footnote{\textbf{9:43} Mat 5,30} \bibleverse{44} \footnote{\textbf{9:44}
  El versículo 44 no aparece en los primeros manuscritos.}
\bibleverse{45} Si el pie te hace pecar, ¡córtalo! Es mejor entrar a la
vida eterna cojo, que teniendo ambos pies y aun así ser lanzado al
Gehena. \bibleverse{46} \footnote{\textbf{9:46} El versículo 46 no
  aparece en los primeros manuscritos.} \bibleverse{47} Si el ojo te
hace pecar, ¡sácalo! Es mejor entrar al reino de Dios con un solo ojo
que ser lanzado al Gehena con ambos ojos, \bibleverse{48} donde los
gusanos no mueren y el fuego no se apaga. \bibleverse{49} Todos serán
`salados' con fuego. \footnote{\textbf{9:49} Lev 2,13} \bibleverse{50}
La sal es buena, pero si pierde su sabor, ¿cómo podría alguien salarla
de nuevo? Ustedes necesitan ser como la sal: vivan en paz unos con
otros''.\footnote{\textbf{9:50} Mat 5,13; Luc 14,34; Col 4,6}

\hypertarget{jesuxfas-en-judea-y-transjordania-conversaciones-sobre-matrimonio-y-divorcio}{%
\subsection{Jesús en Judea y Transjordania; Conversaciones sobre
matrimonio y
divorcio}\label{jesuxfas-en-judea-y-transjordania-conversaciones-sobre-matrimonio-y-divorcio}}

\hypertarget{section-9}{%
\section{10}\label{section-9}}

\bibleverse{1} Jesús partió de Capernaúm y se fue a la región de Judea y
Transjordania. Una vez más la gente se amontonó para verlo, y él les
enseñaba como de costumbre.

\bibleverse{2} Entonces algunos Fariseos vinieron a verlo. Trataron de
probarlo haciéndole la pregunta: ``¿Es legal el divorcio?''

\bibleverse{3} ``¿Qué les dijo Moisés que hicieran?'' les preguntó como
respuesta.

\bibleverse{4} ``Moisés permitía que un hombre escribiera un certificado
de divorcio y desechara a la esposa'',\footnote{\textbf{10:4} Ver
  Deuteronomio 24:1.} respondieron ellos. \footnote{\textbf{10:4} Deut
  24,1; Mat 5,31-32}

\bibleverse{5} Entonces Jesús les dijo: ``Moisés solo escribió esta
regla para ustedes por la actitud dura de sus corazones. \bibleverse{6}
Sin embargo, en el principio, desde la creación, Dios los creó hombre y
mujer. \bibleverse{7} Esa es la razón por la que el hombre deja a su
padre y a su madre y se une en matrimonio con su esposa, \footnote{\textbf{10:7}
  Gén 2,24} \bibleverse{8} y los dos se vuelven un solo cuerpo. Ya no
son más dos, sino uno.\footnote{\textbf{10:8} Ver Génesis 2:24.}
\bibleverse{9} Que nadie separe lo que Dios ha unido''.

\bibleverse{10} Cuando volvieron a estar adentro, los discípulos
comenzaron a preguntarle sobre esto. \bibleverse{11} ``Todo hombre que
se divorcie de su esposa y vuelva a casarse, comete adulterio contra
ella'', les dijo. \bibleverse{12} ``Y si la esposa se divorcia de su
esposo y se casa nuevamente, comete adulterio''.

\hypertarget{jesuxfas-bendice-a-los-niuxf1os}{%
\subsection{Jesús bendice a los
niños}\label{jesuxfas-bendice-a-los-niuxf1os}}

\bibleverse{13} Aconteció que algunas personas trajeron a sus hijos
donde estaba Jesús para que los bendijera, pero los discípulos los
echaban y trataban de mantener a los niños lejos de Jesús.
\bibleverse{14} Pero cuando Jesús vio lo que estaban haciendo, se
molestó mucho y les dijo: ``¡Dejen a los niños venir hacia mí! No se los
impidan, porque el reino de los cielos pertenece a todos los que son
como estos niños. \bibleverse{15} Les digo la verdad, y es que todo
aquél que no reciba el reino de Dios como un niño, no entrará en él''.
\footnote{\textbf{10:15} Mat 18,3} \bibleverse{16} Y Jesús abrazaba a
los niños y colocaba sus manos sobre ellos, y los bendecía. \footnote{\textbf{10:16}
  Mar 9,36}

\hypertarget{la-conversaciuxf3n-de-jesuxfas-con-los-ricos-y-su-referencia-al-peligro-de-las-riquezas}{%
\subsection{La conversación de Jesús con los ricos y su referencia al
peligro de las
riquezas}\label{la-conversaciuxf3n-de-jesuxfas-con-los-ricos-y-su-referencia-al-peligro-de-las-riquezas}}

\bibleverse{17} Cuando Jesús se dispuso a seguir su camino,\footnote{\textbf{10:17}
  Hacia Jerusalén, ver 11:1.} vino un hombre y se arrodilló delante de
él. ``Maestro bueno, ¿qué debo hacer para asegurarme de que tendré la
vida eterna?'' le preguntó.

\bibleverse{18} ``¿Por qué me llamas bueno?'' le preguntó Jesús. ``Nadie
es bueno, solo Dios. \bibleverse{19} Ya conoces los mandamientos: no
matarás, no cometerás adulterio, no robarás, no darás falso testimonio,
no engañarás, honra a tu padre y a tu madre\ldots{}''\footnote{\textbf{10:19}
  Citando Éxodo 20:12-16 o Deuteronomio 5:16-20.} \footnote{\textbf{10:19}
  Éxod 20,12-17}

\bibleverse{20} ``Maestro'', respondió el hombre, ``ya he obedecido
todos esos mandamientos desde que estaba pequeño''.

\bibleverse{21} Jesús lo miró con amor y dijo: ``Solo te falta una cosa.
Ve y vende todo lo que posees, da el dinero a los pobres, y tendrás
tesoro en el cielo. Entonces ven y sígueme''.

\bibleverse{22} Ante esto, el rostro del hombre decayó, y se fue
sintiéndose muy triste, porque era muy rico.

\bibleverse{23} Entonces Jesús miró a su alrededor y les dijo a sus
discípulos: ``¡Será muy difícil para los ricos entrar al reino de
Dios!''

\bibleverse{24} Los discípulos quedaron impresionados por esto. Pero
Jesús siguió: ``Amigos míos, es difícil entrar al reino de Dios.
\footnote{\textbf{10:24} Sal 62,11; 1Tim 6,17} \bibleverse{25} Es más
fácil que un camello pase por el ojo de una aguja, que un rico entre en
el reino de Dios''.

\bibleverse{26} Y los discípulos estaban aún más confundidos. ``Entonces
¿quién podrá ser salvo en toda la tierra?'' se preguntaban unos a otros.

\bibleverse{27} Mirándolos, Jesús respondió: ``Desde un punto de vista
humano, es imposible, pero no con la ayuda de Dios. Con Dios todo es
posible''.

\hypertarget{la-recompensa-de-seguir-a-jesuxfas-y-la-renuncia}{%
\subsection{La recompensa de seguir a Jesús y la
renuncia}\label{la-recompensa-de-seguir-a-jesuxfas-y-la-renuncia}}

\bibleverse{28} Pedro levantó la voz y dijo: ``Nosotros lo hemos dejado
todo para seguirte\ldots{}''

\bibleverse{29} ``Les digo la verdad'', respondió Jesús, ``cualquiera
que deje todo por mi causa, y por causa de la Buena Noticia, su casa o
sus hermanos, sus hermanas o a su padre y su madre, sus hijos o sus
tierras, \bibleverse{30} recibirá como recompensa cien veces tantas
casas y hermanos, hermanas e hijos y tierras, mas persecución. En el
mundo por venir recibirán vida eterna. \bibleverse{31} Sin embargo,
muchos de los primeros serán los últimos, y los últimos serán los
primeros''.

\hypertarget{salida-hacia-jerusaluxe9n-tercer-anuncio-del-sufrimiento-de-jesuxfas}{%
\subsection{Salida hacia Jerusalén; tercer anuncio del sufrimiento de
Jesús}\label{salida-hacia-jerusaluxe9n-tercer-anuncio-del-sufrimiento-de-jesuxfas}}

\bibleverse{32} Ellos siguieron su camino hacia Jerusalén, mientras
Jesús iba adelante. Los discípulos estaban ansiosos y los otros
seguidores estaban asustados. Así que Jesús llevó a los discípulos
aparte para explicarles lo que estaba a punto de ocurrirle.
\bibleverse{33} ``Vamos a Jerusalén'', les dijo, ``y el Hijo del hombre
será entregado a los jefes de los sacerdotes y a los maestros
religiosos. Ellos lo condenarán a muerte y lo entregarán en manos de los
extranjeros.\footnote{\textbf{10:33} En este contexto, está refiriéndose
  a los romanos.} \bibleverse{34} Se burlarán de él, lo escupirán, lo
azotarán y lo matarán. Pero tres días después, él se levantará de
nuevo''.

\hypertarget{solicitud-ambiciosa-de-los-dos-hijos-de-zebedeo}{%
\subsection{Solicitud ambiciosa de los dos hijos de
Zebedeo}\label{solicitud-ambiciosa-de-los-dos-hijos-de-zebedeo}}

\bibleverse{35} Santiago y Juan, los hijos de Zebedeo, vinieron a verlo.
``Maestro'', dijeron ellos, ``queremos que hagas por nosotros lo que te
pidamos''.

\bibleverse{36} ``¿Qué quieren que haga por ustedes?'' respondió Jesús.

\bibleverse{37} ``Cuando estés victorioso y sentado en tu
trono,\footnote{\textbf{10:37} Implícito.} haznos sentar a tu lado, uno
a la derecha y el otro a la izquierda'', le dijeron.

\bibleverse{38} ``Ustedes no saben lo que están pidiendo'', respondió
Jesús. ``¿Pueden ustedes beber la copa que yo bebo? ¿Pueden ustedes ser
bautizados con el bautismo de dolor que yo voy a sufrir?'' \footnote{\textbf{10:38}
  Mar 14,36; Luc 12,50}

\bibleverse{39} ``Sí, podemos'', respondieron ellos. ``Ustedes beberán
la copa que yo bebo, y serán bautizados con el mismo bautismo que yo'',
les dijo Jesús. \footnote{\textbf{10:39} Hech 12,2; Apoc 1,9}

\bibleverse{40} ``Pero no me corresponde a mí decidir quién se sentará a
mi derecha o a mi izquierda. Esos lugares están guardados para aquellos
para quienes han sido preparados''.

\bibleverse{41} Cuando los otros diez discípulos escucharon sobre esto,
comenzaron a sentirse molestos con Santiago y Juan.

\bibleverse{42} Jesús reunió a los discípulos y les dijo: ``Ustedes
pueden darse cuenta de que aquellos que afirman gobernar a las naciones
oprimen a su pueblo. Los gobernantes actúan como tiranos. \footnote{\textbf{10:42}
  Luc 22,25-27} \bibleverse{43} Pero para ustedes no será así.
Cualquiera de ustedes que quiera ser gobernante, tendrá que ser siervo
de todos, \footnote{\textbf{10:43} Mar 9,35; 1Pe 5,3} \bibleverse{44} y
todo aquel que quiera ser el primero entre ustedes, debe ser el esclavo
de todos. \bibleverse{45} Porque incluso el Hijo del hombre no vino para
que lo sirvieran sino para servir, y para dar su vida en rescate para
muchos''.

\hypertarget{curaciuxf3n-del-ciego-bartimeo-cerca-de-jericuxf3}{%
\subsection{Curación del ciego Bartimeo cerca de
Jericó}\label{curaciuxf3n-del-ciego-bartimeo-cerca-de-jericuxf3}}

\bibleverse{46} Entonces pasaron por Jericó. Y cuando Jesús y sus
discípulos salían de la ciudad junto con una gran multitud, Bartimeo, un
indigente ciego, estaba sentado a un lado del camino. \bibleverse{47}
Cuando este escuchó que era Jesús de Nazaret, comenzó a gritar:
``¡Jesús, hijo de David, por favor, ten misericordia de mí!''
\bibleverse{48} Y muchas personas le decían que se callara, pero eso
solo lograba que él gritara aún más, ``¡Jesús, hijo de David, por favor,
ten misericordia de mí!''

\bibleverse{49} Jesús se detuvo y dijo: ``Díganle que venga''. Entonces
lo llamaron, diciéndole: ``¡Buenas noticias! Levántate. Él te llama''.

\bibleverse{50} Bartimeo se levantó de un salto, tiró su abrigo al
suelo, y se apresuró a llegar donde estaba Jesús.

\bibleverse{51} ``¿Qué quieres que haga por ti?'' le preguntó Jesús.
``Maestro'', le dijo a Jesús, ``¡Quiero ver!''

\bibleverse{52} ``Puedes irte. Tu confianza en mí te ha
sanado''.\footnote{\textbf{10:52} O ``salvado''. La palabra puede
  significar ambas cosas: ``salvar'' y ``sanar''.} De inmediato Bartimeo
pudo ver y siguió a Jesús por el camino que iba.

\hypertarget{la-entrada-de-jesuxfas-a-jerusaluxe9n}{%
\subsection{La entrada de Jesús a
Jerusalén}\label{la-entrada-de-jesuxfas-a-jerusaluxe9n}}

\hypertarget{section-10}{%
\section{11}\label{section-10}}

\bibleverse{1} Cuando se acercaban a Jerusalén, estando cerca de Betfagé
y Betania, Jesús envió a dos discípulos para que siguieran adelante.
\footnote{\textbf{11:1} Juan 2,13} \bibleverse{2} Y les dijo: ``Vayan a
la aldea que sigue, y tan pronto como entren allí, encontrarán un
pollino atado, el cual ninguno ha montado todavía. Desátenlo y tráiganlo
aquí. \bibleverse{3} Si alguno les pregunta qué están haciendo,
díganles: `El Señor lo necesita y lo devolverá pronto'\,''.

\bibleverse{4} Entonces ellos partieron de allí, y encontraron un
pollino atado a una puerta, afuera en la calle, y lo desataron.
\bibleverse{5} Y algunos de los que estaban allí cerca de ellos les
preguntaron: ``¿Qué hacen desatando a ese potro''? \bibleverse{6}
Entonces los discípulos respondieron tal como Jesús les había dicho, y
las personas los dejaron ir.

\bibleverse{7} Ellos entonces trajeron a Jesús el pollino, le pusieron
sus abrigos encima y entonces Jesús se sentó sobre él. \bibleverse{8} Y
muchas personas extendieron sus abrigos por el camino, mientras otros
colocaban ramas que habían cortado en los campos. \bibleverse{9} Los que
iban al frente y los que seguían atrás, todos gritaban:
``¡Hosanna!\footnote{\textbf{11:9} ``¡Hosanna!'' Esta es sencillamente
  una transliteración de la palabra aramea que se usa para decir
  ``¡Salve!''} Bendito el que viene en el nombre del Señor.
\bibleverse{10} ¡Bendito el reino de nuestro padre David que ya se
acerca! ¡Hosanna en las alturas!''\footnote{\textbf{11:10} Citando
  Salmos 118:26.}

\bibleverse{11} Jesús llegó a Jerusalén y entró al Templo. Allí comenzó
a mirar a su alrededor, observando cada cosa, y entonces, como se hacía
tarde, regresó a Betania con los doce discípulos.

\hypertarget{la-maldiciuxf3n-de-una-higuera-estuxe9ril}{%
\subsection{La maldición de una higuera
estéril}\label{la-maldiciuxf3n-de-una-higuera-estuxe9ril}}

\bibleverse{12} Al día siguiente, después de salir de Betania, Jesús
tuvo hambre. \bibleverse{13} Y desde cierta distancia, vio una higuera
con hojas, así que fue hacia ella para ver si tenía algún fruto. Pero
cuando llegó allí, se dio cuenta de que solo tenía hojas, porque no era
la temporada de higos. \bibleverse{14} Entonces le dijo a la higuera:
``Que de ti no vuelva a salir más fruto''. Y sus discípulos escucharon
sus palabras.

\hypertarget{la-limpieza-del-templo}{%
\subsection{La limpieza del templo}\label{la-limpieza-del-templo}}

\bibleverse{15} Llegaron nuevamente a Jerusalén, y Jesús entró al
Templo. Y comenzó a sacar a las personas que estaban comprando y
vendiendo dentro del Templo. Volteó las mesas de los cambistas y las
sillas de los que vendían palomas. \footnote{\textbf{11:15} Juan 2,14-16}
\bibleverse{16} Detuvo a todos los que llevaban cosas por el Templo.
\bibleverse{17} Y les explicó: ``¿Acaso no dice la Escritura: `Mi casa
será llamada casa de oración para todas las naciones'?\footnote{\textbf{11:17}
  Citando Isaías 56:7.} ¡Pero ustedes la han convertido en refugio de
ladrones!''\footnote{\textbf{11:17} Citando Jeremías 7:11.}

\bibleverse{18} Los jefes de los sacerdotes y maestros religiosos
escucharon lo que había ocurrido, y trataban de encontrar la manera de
matar a Jesús. Pero le tenían miedo, porque todos estaban muy
impresionados por sus enseñanzas.

\bibleverse{19} Cuando llegó la noche, Jesús y sus discípulos se
marcharon de la ciudad.

\hypertarget{repaso-de-la-higuera-seca-con-posterior-referencia-al-poder-de-la-fe-y-la-oraciuxf3n-advertencia}{%
\subsection{Repaso de la higuera seca con posterior referencia al poder
de la fe y la oración;
advertencia}\label{repaso-de-la-higuera-seca-con-posterior-referencia-al-poder-de-la-fe-y-la-oraciuxf3n-advertencia}}

\bibleverse{20} A la mañana siguiente regresaron, vieron la higuera, y
se había marchitado toda desde la raíz. \bibleverse{21} Y Pedro recordó
lo que Jesús había hecho, y le dijo: ``Maestro, mira, la higuera que
maldijiste se ha marchitado''.

\bibleverse{22} ``Crean en Dios'', respondió Jesús. \bibleverse{23}
``Créanme cuando les digo que si ustedes le dijeran a esta montaña:
`Vete de aquí y lánzate al mar,' y no dudan en sus corazones, sino que
están convencidos de lo que están pidiendo, ¡entonces así pasará!
\footnote{\textbf{11:23} Mar 9,23; Mat 17,20} \bibleverse{24} Les estoy
diciendo que todo aquello por lo que oren, todo lo que pidan, crean que
lo han recibido, y así será. \footnote{\textbf{11:24} Mat 7,7; Juan
  14,13; 1Jn 5,14; 1Jn 1,5-15} \bibleverse{25} Pero cuando estén orando,
si tienen algo contra alguien, perdónenle, para que así el Padre, que
está en el cielo, también pueda perdonar los pecados de ustedes''.
\footnote{\textbf{11:25} Mat 5,23} \bibleverse{26} \footnote{\textbf{11:26}
  Los primeros manuscritos no contienen el versículo 26, el cual se ha
  agregado del texto de Mateo 6:15.} \footnote{\textbf{11:26} Mat
  6,14-15}

\hypertarget{la-pregunta-del-sumo-consejo-sobre-la-autoridad-de-jesuxfas}{%
\subsection{La pregunta del sumo consejo sobre la autoridad de
Jesús}\label{la-pregunta-del-sumo-consejo-sobre-la-autoridad-de-jesuxfas}}

\bibleverse{27} Entonces regresaron a Jerusalén, y mientras caminaba en
el Templo, los jefes de los sacerdotes, los maestros religiosos y los
líderes se acercaron a él. \bibleverse{28} ``¿Con qué autoridad estás
haciendo todo esto?'' le reclamaron. ``¿Quién te dio ese derecho?''

\bibleverse{29} ``Déjenme hacerles una pregunta'', les dijo Jesús. ``Si
ustedes me responden, yo les diré con qué autoridad hago estas cosas.
\bibleverse{30} El bautismo de Juan, ¿provenía del cielo, o de los
hombres?''

\bibleverse{31} Entonces ellos debatían entre ellos mismos. Y decían:
``Si decimos que venía del cielo, el responderá `¿Por qué no creyeron en
él?' \bibleverse{32} Pero si decimos que era de origen humano,
pues\ldots{}'' Y tenían miedo de la multitud, porque todos creían que
Juan era un verdadero profeta. \footnote{\textbf{11:32} Luc 7,29-30}

\bibleverse{33} Entonces le respondieron a Jesús: ``No sabemos''.
``Entonces yo no les diré quién me dio la autoridad de hacer estas
cosas'', respondió Jesús.

\hypertarget{paruxe1bola-de-los-viticultores-infieles}{%
\subsection{Parábola de los viticultores
infieles}\label{paruxe1bola-de-los-viticultores-infieles}}

\hypertarget{section-11}{%
\section{12}\label{section-11}}

\bibleverse{1} Entonces Jesús comenzó a hablarles usando relatos
ilustrados.\footnote{\textbf{12:1} Ver 3:23.} ``Un hombre plantó una
viña. Colocó un cerco a su alrededor, cavó un hueco para que hubiera un
lagar, y construyó una torre de vigilancia. Entonces la alquiló a unos
granjeros, y se fue de viaje. \bibleverse{2} ``Cuando llegó el tiempo de
la cosecha, envió a uno de sus siervos donde los granjeros a quienes
había alquilado su viña, para que recolectaran las uvas de la viña.
\bibleverse{3} Pero ellos lo agarraron y lo golpearon, y lo enviaron de
regreso sin nada. \bibleverse{4} Entonces el propietario envió a otro
siervo. Ellos lo golpearon en la cabeza y lo maltrataron. \bibleverse{5}
Entonces envió a otro siervo, y a este lo mataron. El propietario envió
a muchos otros siervos, y a unos los golpearon y a otros los mataron.
\bibleverse{6} Al final, el único que quedaba era el hijo a quien amaba,
y lo envió, pensando: `ellos respetarán a mi hijo'. \bibleverse{7} Pero
los granjeros pensaron para sí: `Aquí viene el heredero del propietario,
¡si lo matamos, podremos quedarnos con toda su herencia!' \bibleverse{8}
Así que lo tomaron y lo mataron, y lo lanzaron fuera de la viña.
\footnote{\textbf{12:8} Heb 13,12} \bibleverse{9} ¿Qué hará ahora el
dueño de la viña? Vendrá y matará a esos granjeros, y entonces alquilará
su viña a otros. \bibleverse{10} ``¿No han leído la Escritura que dice
`la piedra rechazada por los constructores se ha convertido en la piedra
angular. \bibleverse{11} ¡Esto viene del Señor, y desde nuestro punto de
vista es maravilloso!'?''\footnote{\textbf{12:11} Citado de Salmos
  118:22-23.}

\bibleverse{12} Los líderes judíos trataban de atraparlo porque se
dieron cuenta de que la ilustración estaba dirigida a ellos, pero tenían
miedo de la multitud. Así que lo dejaron solo y se fueron.

\hypertarget{la-cuestiuxf3n-fiscal-de-los-fariseos}{%
\subsection{La cuestión fiscal de los
fariseos}\label{la-cuestiuxf3n-fiscal-de-los-fariseos}}

\bibleverse{13} Luego le enviaron a unos Fariseos y a otros que estaban
a favor de Herodes para tratar de atraparlo en las cosas que decía.
\bibleverse{14} Llegaron y dijeron: ``Maestro, sabemos que eres una
persona honesta y que no buscas aprobación, porque no te interesa el
estatus o la posición.\footnote{\textbf{12:14} Literalmente, ``Tú no te
  preocupas por nadie porque no te fijas en la cara de los hombres''.
  Sin embargo, esta traducción literal podría hacer parecer que Jesús
  era desconsiderado o indiferente.} Por el contrario, enseñas el camino
de Dios conforme a la verdad. Dinos entonces, ¿es correcto pagar o no el
tributo al césar? \bibleverse{15} ¿Deberíamos pagarlo, o deberíamos
negarnos a hacerlo?'' Pero Jesús, dándose cuenta de lo hipócritas que
eran, les preguntó: ``¿Por qué intentan atraparme en algo? Muéstrenme
una moneda''.

\bibleverse{16} Ellos le dieron una moneda. ``¿De quién es esta imagen y
la inscripción en ella?'' les preguntó Jesús. ``Es del césar'',
respondieron ellos.

\bibleverse{17} ``Entonces devuelvan al césar lo que le pertenece al
césar, y a Dios lo que le pertenece a Dios'', les dijo Jesús. Y ellos
estaban sorprendidos de su respuesta.

\hypertarget{la-pregunta-sobre-la-resurrecciuxf3n-de-los-muertos}{%
\subsection{La pregunta sobre la resurrección de los
muertos}\label{la-pregunta-sobre-la-resurrecciuxf3n-de-los-muertos}}

\bibleverse{18} Entonces los Saduceos, quienes no creen en la
resurrección, vinieron también y le hicieron una pregunta:
\bibleverse{19} ``Maestro, Moisés nos enseñó que si un hombre muere y
deja a su viuda sin hijos, entonces su hermano debe casarse con ella, y
darle hijos por él.\footnote{\textbf{12:19} Ver Deuteronomio 25:5.}
\bibleverse{20} Digamos que había siete hermanos. El primero se casó y
murió sin tener hijos. \bibleverse{21} El segundo se casó con la viuda,
y murió, sin tener hijos. El tercero hizo lo mismo. \bibleverse{22} De
hecho, los siete murieron sin tener hijos. Al final, la mujer también
murió. \bibleverse{23} En la resurrección, ¿cuál de todos será su
esposo, siendo que ella fue esposa de los siete hermanos?''

\bibleverse{24} Jesús les dijo: ``Esto demuestra que ustedes están
equivocados, y que no conocen las Escrituras o el poder de Dios.
\bibleverse{25} Cuando los muertos se levanten, no se casarán, y no se
darán en casamiento. Serán como los ángeles que están en el cielo.
\bibleverse{26} Pero respecto a la resurrección, ¿no han leído en los
escritos de Moisés el relato de la zarza ardiente, donde Dios habló con
Moisés y le dijo: `Yo soy el Dios de Abrahán, y el Dios de Isaac, y el
Dios de Jacob?'\footnote{\textbf{12:26} Ver Éxodo 3:2-6.}
\bibleverse{27} Él no es Dios de los muertos, sino de los vivos.
¡Ustedes están completamente equivocados!''

\hypertarget{la-pregunta-de-un-escriba-sobre-el-mandamiento-muxe1s-noble}{%
\subsection{La pregunta de un escriba sobre el mandamiento más
noble}\label{la-pregunta-de-un-escriba-sobre-el-mandamiento-muxe1s-noble}}

\bibleverse{28} Uno de los maestros religiosos vino y los escuchó
discutiendo. Este reconoció que Jesús les había dado una buena
respuesta. Así que le preguntó: ``¿Cuál es el mandamiento más importante
de todos?''

\bibleverse{29} Jesús respondió: ``El primer mandamiento es: `Escucha,
oh, Israel, el Señor nuestro Dios es uno. \bibleverse{30} Ama al Señor
tu Dios con todo tu corazón, con todo tu espíritu, con toda tu mente y
con toda tu fuerza'.\footnote{\textbf{12:30} Citando Deuteronomio 6:4.}
\bibleverse{31} El segundo es: `Ama a tu prójimo como a ti
mismo'.\footnote{\textbf{12:31} Citando Levítico 19:18.} Ningún otro
mandamiento es más importante que estos''.

\bibleverse{32} ``Eso es correcto, Maestro'', respondió el hombre. ``Es
cierto lo que dices, que Dios es uno y no hay otro. \bibleverse{33}
Debemos amarlo con todo nuestro corazón, con todo nuestro entendimiento,
y con toda nuestra fuerza, y debemos amar a nuestro prójimo como a
nosotros mismos. Esto es mucho más importante que las ofrendas y los
sacrificios''. \footnote{\textbf{12:33} 1Sam 15,22; Os 6,6}

\bibleverse{34} Jesús se dio cuenta de que el hombre había dado una
respuesta pertinente, y dijo: ``No estás lejos del reino de Dios''.
Después de esto, nadie tuvo la valentía para hacerle más preguntas.
\footnote{\textbf{12:34} Hech 26,27-29}

\hypertarget{la-contrapregunta-de-jesuxfas-sobre-el-mesuxedas-como-hijo-de-david}{%
\subsection{La contrapregunta de Jesús sobre el Mesías como hijo de
David}\label{la-contrapregunta-de-jesuxfas-sobre-el-mesuxedas-como-hijo-de-david}}

\bibleverse{35} Mientras Jesús enseñaba en el Templo, preguntó: ``¿Por
qué los maestros religiosos afirman que Cristo es el hijo de David?
\footnote{\textbf{12:35} Is 11,1; Rom 1,3} \bibleverse{36} Pues el mismo
David dijo, inspirado por el Espíritu Santo: `El Señor dijo a mi Señor:
``Siéntate a mi diestra hasta que ponga a tus enemigos bajo tus
pies''\,'.\footnote{\textbf{12:36} Citando Salmos 110:1.} \footnote{\textbf{12:36}
  2Sam 23,2}

\bibleverse{37} Pues, si David mismo lo llama Señor, ¿cómo puede él ser
el hijo de David?'' Y la gran multitud estaba oyendo con mucho deleite
lo que Jesús decía.

\hypertarget{la-advertencia-de-jesuxfas-sobre-la-ambiciuxf3n-y-la-codicia-de-los-escribas}{%
\subsection{La advertencia de Jesús sobre la ambición y la codicia de
los
escribas}\label{la-advertencia-de-jesuxfas-sobre-la-ambiciuxf3n-y-la-codicia-de-los-escribas}}

\bibleverse{38} Y Jesús seguía enseñándoles, diciendo: ``¡Tengan cuidado
con los maestros religiosos! A ellos les encanta caminar por ahí con
batas largas, y que los saluden con respeto en las plazas.
\bibleverse{39} Les encanta tener los asientos más importantes en las
sinagogas, y los mejores lugares en los banquetes. \bibleverse{40}
Engañan a las viudas y les quitan lo que poseen,\footnote{\textbf{12:40}
  Literalmente, ``devoran las casas de las viudas''.} y encubren el tipo
de personas que son realmente, con oraciones extensas y llenas de
palabrerías. Ellos recibirán una condenación severa en el juicio''.
\footnote{\textbf{12:40} Sant 1,27}

\hypertarget{jesuxfas-alaba-las-dos-blancas-de-la-viuda-pobre}{%
\subsection{Jesús alaba las dos blancas de la viuda
pobre}\label{jesuxfas-alaba-las-dos-blancas-de-la-viuda-pobre}}

\bibleverse{41} Jesús se sentó al otro lado de la alcancía de la
tesorería del Templo, mientras veía a la gente echando las monedas.
Muchos ricos que estaban allí daban mucho dinero, de manera
extravagante. \footnote{\textbf{12:41} 2Re 12,10}

\bibleverse{42} Entonces una viuda pobre vino y echó solo dos monedas
pequeñas.\footnote{\textbf{12:42} Literalmente, ``dos lepta'', eran de
  poco valor.} \bibleverse{43} Entonces él llamó a sus discípulos y les
dijo: ``Les digo la verdad: esa pobre viuda ha dado más que todos los
demás juntos. \bibleverse{44} Todos ellos dieron lo que tenían de sus
riquezas, pero ella dio de su pobreza lo que no tenía. Ella dio todo lo
que tenía para vivir''.

\hypertarget{los-primeros-signos-del-fin-de-los-tiempos}{%
\subsection{Los primeros signos del fin de los
tiempos}\label{los-primeros-signos-del-fin-de-los-tiempos}}

\hypertarget{section-12}{%
\section{13}\label{section-12}}

\bibleverse{1} Cuando Jesús salía del Templo, uno de sus discípulos le
dijo: ``¡Maestro, mira toda esa cantidad de piedras y esas magníficas
edificaciones!''

\bibleverse{2} ``¿Ves todas estas edificaciones?'' respondió Jesús. ``No
quedará piedra sobre piedra. Todo será derribado''.

\bibleverse{3} Al sentarse en el Monte de los Olivos, mirando el Templo,
Pedro, Santiago, Juan y Andrés le preguntaron en privado: \footnote{\textbf{13:3}
  Mat 17,1} \bibleverse{4} ``Dinos cuándo ocurrirá esto. ¿Cuál es la
señal de que todo esto está a punto de cumplirse?''\footnote{\textbf{13:4}
  Siguiendo la comprensión de los discípulos, esto quiere decir que la
  respuesta de Jesús combina los aspectos de la destrucción del Templo y
  el tiempo final.}

\hypertarget{los-primeros-signos-del-fin-de-los-tiempos-1}{%
\subsection{Los primeros signos del fin de los
tiempos}\label{los-primeros-signos-del-fin-de-los-tiempos-1}}

\bibleverse{5} Jesús comenzó a decirles: ``No dejen que nadie los
engañe. \bibleverse{6} Muchos vendrán en mi nombre, diciendo: `Yo soy el
Cristo'. Ellos engañarán a muchas personas.

\bibleverse{7} No se atribulen cuando escuchen de guerras aquí y allá.
Estas cosas deben suceder pero este no es el fin. \bibleverse{8} Las
naciones pelearán unas contra otras, y los reinos unos contra otros.
Habrá terremotos en diferentes lugares, y hambrunas. Estos son los
comienzos de los dolores de parto que sufrirá el mundo.

\hypertarget{la-persecuciuxf3n-de-los-discuxedpulos}{%
\subsection{La persecución de los
discípulos}\label{la-persecuciuxf3n-de-los-discuxedpulos}}

\bibleverse{9} ``¡Cuídense! Ellos los entregarán a ustedes a las cortes
para ser juzgados. Ustedes serán golpeados en las sinagogas. Y por mi
causa ustedes tendrán que estar en pie frente a gobernantes y reyes, y
ustedes les testificarán. \bibleverse{10} ``Y es necesario que primero
se anuncie la Buena Noticia en toda nación. \footnote{\textbf{13:10} Mar
  16,15} \bibleverse{11} Cuando ellos vengan a arrestarlos y juzgarlos,
no se preocupen por lo que vayan a decir. Digan lo que se les diga en
ese momento, porque no serán ustedes los que hablen, sino el Espíritu
Santo.

\bibleverse{12} ``El hermano entregará a su hermano a la muerte, y el
padre entregará a su hijo. Los hijos se volverán en contra de sus padres
y harán que los condenen a muerte. \bibleverse{13} Ustedes serán odiados
por todos, por mi causa, pero todo el que persevere hasta el fin será
salvo.

\hypertarget{el-cluxedmax-de-la-tribulaciuxf3n-en-judea}{%
\subsection{El clímax de la tribulación en
Judea}\label{el-cluxedmax-de-la-tribulaciuxf3n-en-judea}}

\bibleverse{14} ``Pero cuando vean la `abominación
desoladora'\footnote{\textbf{13:14} O ``la abominación que causa
  desolación''. Ver Daniel 9:27, Daniel 11:31 y Daniel 12:11} en el
lugar donde no debe estar (el que lee, que entienda), entonces los que
estén en Judea deben correr a las montañas. \footnote{\textbf{13:14} Dan
  9,27; Dan 11,31} \bibleverse{15} Los que estén en el techo, no entren
de regreso a la casa a buscar nada. \bibleverse{16} Los que están
afuera, en los campos, no vayan a la casa a buscar un abrigo.
\bibleverse{17} ¡Cuán difícil será para las que estén embarazadas o
lactando en esos días! \bibleverse{18} Oren para que esto no ocurra
durante el invierno. \bibleverse{19} Porque estos serán días de
tribulación como nunca ha habido desde el principio de la creación de
Dios hasta ahora, y nunca más habrán. \bibleverse{20} Si Dios no
acortase esos días, nadie sobreviviría. Sin embargo, por causa de los
que Dios ha escogido, él ha acortado esos días.

\hypertarget{profecuxeda-sobre-los-falsos-profetas}{%
\subsection{Profecía sobre los falsos
profetas}\label{profecuxeda-sobre-los-falsos-profetas}}

\bibleverse{21} ``De modo que si alguno les dice: `miren, aquí está el
Mesías,' o `miren, está allá,' no lo crean. \bibleverse{22} Porque
aparecerán falsos Mesías y falsos profetas, y harán milagros y
maravillas para engañar, si fuese posible, a los escogidos de Dios.
\bibleverse{23} ¡Tengan cuidado! Yo les he dicho todo antes de que
suceda''.

\hypertarget{los-uxfaltimos-augurios-y-la-apariciuxf3n-del-hijo-del-hombre-en-el-uxfaltimo-duxeda}{%
\subsection{Los últimos augurios y la aparición del Hijo del Hombre en
el último
día}\label{los-uxfaltimos-augurios-y-la-apariciuxf3n-del-hijo-del-hombre-en-el-uxfaltimo-duxeda}}

\bibleverse{24} ``Esto es lo que ocurrirá después de esas tribulaciones:
`el sol se oscurecerá, la luna no brillará, \bibleverse{25} las
estrellas caerán del cielo, y los poderes en los cielos serán
conmovidos'.\footnote{\textbf{13:25} Ver Isaías 13:10.} \footnote{\textbf{13:25}
  Heb 12,26} \bibleverse{26} ``Entonces verán al Hijo del hombre venir
en las nubes, con gran poder y gloria.\footnote{\textbf{13:26} Ver
  Daniel 7:13-14.} \bibleverse{27} Él enviará a los ángeles, y reunirá a
todos sus escogidos desde donde estén,\footnote{\textbf{13:27}
  Literalmente, ``por los cuatro vientos''.} desde las partes más
lejanas de la tierra hasta el punto más lejano del cielo.

\bibleverse{28} ``Aprendan la lección de la higuera: cuando sus ramas
crecen suaves y se caen sus hojas, ya saben que el verano está cerca.
\bibleverse{29} De la misma manera, cuando vean suceder estas cosas, ya
sabrán que está cerca, ¡justo a las puertas! \bibleverse{30} Les digo la
verdad, esta generación no llegará a su fin hasta que estas cosas hayan
ocurrido. \bibleverse{31} El cielo y la tierra llegarán a su fin, pero
mis enseñanzas no.

\bibleverse{32} ``Nadie sabe el día ni la hora en que esto ocurrirá, ni
siquiera los ángeles que están en el cielo, ni siquiera el Hijo; solo el
Padre lo sabe.

\hypertarget{exhortaciuxf3n-final-a-los-discuxedpulos-a-estar-alerta}{%
\subsection{Exhortación final a los discípulos a estar
alerta}\label{exhortaciuxf3n-final-a-los-discuxedpulos-a-estar-alerta}}

\bibleverse{33} ¡Estén atentos! ¡Estén despiertos! Porque ustedes no
saben cuándo sucederá esto. \footnote{\textbf{13:33} Luc 12,35-40}

\bibleverse{34} Es como un hombre que se fue de viaje. Se fue de la casa
y le dio autoridad a cada uno de sus sirvientes para hacer lo que él les
había dicho. Pero al portero le dijo que se mantuviera despierto.
\bibleverse{35} ``Así que estén vigilantes, porque no saben en qué
momento regresará el dueño de la casa. Puede ser al anochecer, a la
media noche, antes del amanecer, o en la mañana. \bibleverse{36} Más
vale que no estén durmiendo si el dueño regresa sorpresivamente.
\bibleverse{37} Lo que les digo a ustedes, lo digo a todos: ¡Estén
vigilantes!''

\hypertarget{intento-de-asesinato-por-parte-de-los-luxedderes-del-pueblo}{%
\subsection{Intento de asesinato por parte de los líderes del
pueblo}\label{intento-de-asesinato-por-parte-de-los-luxedderes-del-pueblo}}

\hypertarget{section-13}{%
\section{14}\label{section-13}}

\bibleverse{1} Faltaban dos días para la Pascua y para la Fiesta de los
panes sin levadura. Los jefes de los sacerdotes y los líderes religiosos
estaban tratando de encontrar alguna manera oculta de arrestar a Jesús y
mandarlo a matar. \bibleverse{2} ``Pero no será durante la Pascua'',
pensaban ellos, ``de lo contrario el pueblo podría amotinarse''.

\hypertarget{unciuxf3n-de-jesuxfas-en-betania}{%
\subsection{Unción de Jesús en
Betania}\label{unciuxf3n-de-jesuxfas-en-betania}}

\bibleverse{3} Mientras tanto, Jesús estaba en Betania, cenando en la
casa de Simón, el leproso. Una mujer entró con un frasco de alabastro
que contenía un costoso perfume de nardo puro.\footnote{\textbf{14:3}
  Nardo: un aceite esencial derivado de las raíces de la planta de
  nardo, la cual es originaria de China e India.} Ella quebró el frasco
y derramó el perfume sobre la cabeza de Jesús. \footnote{\textbf{14:3}
  Juan 12,1-8} \bibleverse{4} Y algunos de los que estaban allí se
molestaron y dijeron: ``¿Por qué desperdiciar este perfume?
\bibleverse{5} Podría haberse vendido por el salario de un
año\footnote{\textbf{14:5} Literalmente, ``300 denarios''.} y luego se
habría dado ese dinero a los pobres'' Y estaban muy molestos con ella.

\bibleverse{6} Pero Jesús respondió: ``¡Déjenla en paz! ¿Por qué la
critican por hacer algo hermoso por mí? \bibleverse{7} Ustedes siempre
tendrán a los pobres entre ustedes\footnote{\textbf{14:7} Ver
  Deuteronomio 15:11.} y podrán ayudarlos cuando quieran. Pero no
siempre me tendrán a mí aquí con ustedes. \bibleverse{8} Ella hizo lo
que pudo: ungió mi cuerpo en anticipación para mi sepultura.
\bibleverse{9} Les digo la verdad: dondequiera que se predique la Buena
Noticia, la gente recordará lo que ella hizo''.

\hypertarget{traiciuxf3n-de-judas}{%
\subsection{Traición de Judas}\label{traiciuxf3n-de-judas}}

\bibleverse{10} Entonces Judas Iscariote, uno de los doce discípulos,
fue donde los jefes de los sacerdotes y llegó con ellos a un acuerdo
para entregarles a Jesús. \bibleverse{11} Cuando ellos oyeron esto, se
alegraron, y prometieron pagarle. Así que Judas comenzó a buscar una
oportunidad para entregar a Jesús.

\hypertarget{preparaciuxf3n-de-la-comida-pascual}{%
\subsection{Preparación de la comida
pascual}\label{preparaciuxf3n-de-la-comida-pascual}}

\bibleverse{12} El primer día de la Fiesta de los panes sin levadura, el
tiempo cuando se sacrifica el cordero de la Pascua, los discípulos de
Jesús le preguntaron: ``¿Dónde quieres que vayamos a preparar la cena de
la Pascua para ti?''

\bibleverse{13} Entonces él envió a dos de sus discípulos, diciéndoles:
``Entren a la ciudad y allí conocerán a un hombre que llevará una olla
de agua. Síganlo \bibleverse{14} y cuando él entre a una casa,
pregúntenle al dueño dónde puedo yo celebrar con mis discípulos la
Pascua. \footnote{\textbf{14:14} Mar 11,3} \bibleverse{15} Él los
llevará a un salón en el piso de arriba, que ya está arreglado y listo.
Allí pueden hacer los preparativos para nosotros''.

\bibleverse{16} Entonces los discípulos fueron a la ciudad, y
encontraron las cosas tal como él las había descrito. Prepararon la cena
de la Pascua,

\hypertarget{la-uxfaltima-cena-de-jesuxfas-en-el-cuxedrculo-de-los-discuxedpulos-anuncio-de-la-traiciuxf3n-de-judas-instituciuxf3n-de-la-santa-comuniuxf3n}{%
\subsection{La última cena de Jesús en el círculo de los discípulos;
Anuncio de la traición de Judas; Institución de la santa
comunión}\label{la-uxfaltima-cena-de-jesuxfas-en-el-cuxedrculo-de-los-discuxedpulos-anuncio-de-la-traiciuxf3n-de-judas-instituciuxf3n-de-la-santa-comuniuxf3n}}

\bibleverse{17} y en la noche Jesús fue allí con los doce discípulos.
\bibleverse{18} Mientras estaban sentados y comiendo, Jesús dijo: ``Les
digo la verdad: uno de ustedes va a entregarme, uno que está comiendo
ahora conmigo''.

\bibleverse{19} Ellos estaban sorprendidos, y cada uno preguntaba: ``No
soy yo, ¿cierto?''

\bibleverse{20} ``Es uno de los doce, uno de ustedes y que está
compartiendo esta comida conmigo. \bibleverse{21} El Hijo del hombre
morirá, tal como lo dijeron las Escrituras. ¡Pero cuán terrible será
para quien entregue al Hijo del hombre! Mejor sería que ese hombre no
hubiera nacido''.

\bibleverse{22} Mientras comían, Jesús tomó el pan con sus manos. Luego
lo bendijo y lo dio a los discípulos. ``Tomen. Este es mi cuerpo'', les
dijo. \footnote{\textbf{14:22} 1Cor 11,23-25}

\bibleverse{23} Entonces tomó la copa en sus manos. La bendijo y la dio
a los discípulos. Y todos bebieron de ella. \bibleverse{24} ``Esta es mi
sangre'', les dijo, ``el pacto\footnote{\textbf{14:24} Queriendo decir
  ``acuerdo'' o ``promesa''.} que es vertido por muchos. \bibleverse{25}
Les digo la verdad: no beberé más del fruto de la vid hasta el día en
que lo beba nuevamente en el reino de Dios''.

\hypertarget{camina-a-getsemanuxed}{%
\subsection{Camina a Getsemaní}\label{camina-a-getsemanuxed}}

\bibleverse{26} Después de haber cantado un salmo, se fueron hacia el
Monte de los Olivos. \footnote{\textbf{14:26} Sal 113,1-118}

\bibleverse{27} ``Todos ustedes me abandonarán'', les dijo Jesús.
``Porque como dicen las Escrituras, `Yo atacaré al pastor, y las ovejas
estarán totalmente dispersas'.\footnote{\textbf{14:27} Citando Zacarías
  13:7.} \footnote{\textbf{14:27} Juan 16,32} \bibleverse{28} Pero
después que yo me haya levantado de entre los muertos, iré delante de
ustedes a Galilea''. \footnote{\textbf{14:28} Mar 16,7}

\bibleverse{29} ``Yo no te abandonaré aunque todos los demás lo hagan'',
respondió Pedro.

\bibleverse{30} Jesús le respondió: ``Te digo la verdad hoy: esta misma
noche, antes de que el gallo cante dos veces, tres veces negarás que me
conoces''.

\bibleverse{31} Pero Pedro, insistentemente dijo: ``Aun si me toca morir
contigo, nunca te negaré''. Y todos ellos dijeron lo mismo.

\hypertarget{el-conflicto-y-la-oraciuxf3n-de-jesuxfas-en-getsemanuxed-debilidad-de-los-discuxedpulos}{%
\subsection{El conflicto y la oración de Jesús en Getsemaní; Debilidad
de los
discípulos}\label{el-conflicto-y-la-oraciuxf3n-de-jesuxfas-en-getsemanuxed-debilidad-de-los-discuxedpulos}}

\bibleverse{32} Llegaron entonces a un lugar llamado
Getsemaní,\footnote{\textbf{14:32} Que significa, ``lagar de olivos''.}
donde Jesús le dijo a sus discípulos: ``Siéntense aquí mientras yo voy a
orar''. \bibleverse{33} Y llevó consigo a Pedro, Santiago y Juan.
Entonces su espíritu se turbó y estaba muy afligido. \footnote{\textbf{14:33}
  Mat 17,1} \bibleverse{34} Jesús les dijo: ``Mi agonía es tan dolorosa
que siento que muero. Por favor, quédense aquí y estén despiertos''.
\footnote{\textbf{14:34} Juan 12,27}

\bibleverse{35} Se fue un poco más lejos y se postró en el suelo. Oraba,
pidiendo que el momento\footnote{\textbf{14:35} Literalmente, ``la
  hora''.} que estaba por llegar pudiera evitarse, si fuera posible.
\bibleverse{36} ``¡Abba, Padre! Tú puedes hacerlo todo'', decía. ``Por
favor, quítame esta copa de sufrimiento. Pero que no sea como yo quiero,
sino como tú quieres''. \footnote{\textbf{14:36} Mar 10,38}

\bibleverse{37} Entonces Jesús regresó y encontró a los discípulos
dormidos. ``Simón, ¿estás durmiendo?'' le preguntó a Pedro. ``¿No
pudieron estar despiertos conmigo tan solo una hora? \bibleverse{38}
Quédense despiertos, y oren para que no caigan en tentación. El espíritu
está dispuesto, pero el cuerpo es débil''.

\bibleverse{39} Una vez más los dejó allí, y oraba, pidiendo lo mismo
otra vez. \bibleverse{40} Entonces regresó y una vez más los encontró
durmiendo porque no podía mantener sus ojos abiertos.\footnote{\textbf{14:40}
  Literalmente, ``sus ojos estaban pesados''.} Ellos no sabían qué
responder. \bibleverse{41} Entonces Jesús regresó por tercera vez, y les
preguntó, ``¿aún están dormidos? ¿Aún están descansando? ¡Pues ya es
suficiente,\footnote{\textbf{14:41} El significado de la palabra griega
  aquí no está claro. Por lo general, significa ``pagado en su
  totalidad''.} porque ha llegado el momento! Miren, el Hijo del hombre
está a punto de ser entregado en manos de pecadores. \bibleverse{42}
¡Levántense! ¡Vamos! Miren, aquí viene el que me entrega''.

\hypertarget{encarcelamiento-de-jesuxfas-escape-de-los-discuxedpulos}{%
\subsection{Encarcelamiento de Jesús; Escape de los
discípulos}\label{encarcelamiento-de-jesuxfas-escape-de-los-discuxedpulos}}

\bibleverse{43} Justo cuando decía esto, Judas---uno de los doce
discípulos---llegó con una turba que traía espadas y palos, enviados por
los jefes de los sacerdotes, líderes religiosos y ancianos.
\bibleverse{44} Aconteció que el que entregaba a Jesús se había puesto
de acuerdo con ellos en una señal: ``El que yo bese, ese es. Arréstenlo,
y llévenselo bajo custodia''. \bibleverse{45} Judas llegó directamente
donde estaba Jesús. ``Maestro'', le dijo, y lo besó. \bibleverse{46}
Entonces ellos lo agarraron y lo arrestaron. \bibleverse{47} Pero uno de
los que estaba allí sacó su espada e hirió al siervo del sumo sacerdote,
cortándole la oreja.

\bibleverse{48} ``¿Acaso soy alguna clase de rebelde, que han venido a
arrestarme con espadas y palos?'' les preguntó Jesús. \bibleverse{49}
``Estuve allí en el Templo, enseñándoles cada día. ¿Por qué no me
arrestaron en ese momento? Pero esto ha ocurrido para cumplir las
Escrituras''.

\bibleverse{50} Entonces todos los discípulos de Jesús lo abandonaron y
huyeron. \bibleverse{51} (Uno de sus seguidores era un joven que usaba
solamente un vestido de lino. \bibleverse{52} Ellos lo sujetaron por la
fuerza, pero él salió corriendo desnudo, dejando allí tirado su
vestido).

\hypertarget{el-interrogatorio-la-confesiuxf3n-y-la-condena-de-jesuxfas-ante-el-sumo-sacerdote-y-el-concilio}{%
\subsection{El interrogatorio, la confesión y la condena de Jesús ante
el sumo sacerdote y el
concilio}\label{el-interrogatorio-la-confesiuxf3n-y-la-condena-de-jesuxfas-ante-el-sumo-sacerdote-y-el-concilio}}

\bibleverse{53} Entonces llevaron a Jesús a la casa del sumo sacerdote,
donde se habían reunido todos los jefes de los sacerdotes, ancianos y
líderes religiosos.

\bibleverse{54} Pedro lo seguía a la distancia, y entró al patio de la
casa del sumo sacerdote. Se sentó allí con los guardias y se calentaba
cerca de la fogata. \bibleverse{55} Dento de la casa, los jefes de los
sacerdotes y todo el concilio de gobierno\footnote{\textbf{14:55}
  Literalmente, ``Sanedrín''.} trataba de encontrar alguna evidencia
para mandar a matar a Jesús, pero no podían encontrar nada.
\bibleverse{56} Había muchos dando falso testimonio contra él, pero sus
afirmaciones no concordaban. \bibleverse{57} Algunos de ellos se
levantaron a decir cosas falsas sobre Jesús. \bibleverse{58} ``Lo oímos
decir: `Destruiré este Templo que han construido manos humanas, y en
tres días lo volveré a construir sin usar las manos'\,''.
\bibleverse{59} Pero aun así sus testimonios no concordaban.

\bibleverse{60} Entonces el sumo sacerdote se puso en pie frente al
concilio, y le preguntó a Jesús: ``¿No tienes nada que decir en
respuesta a las acusaciones que se han hecho en tu contra?''
\bibleverse{61} Pero Jesús permanecía en silencio y no respondía. Así
que el sumo sacerdote le preguntó de nuevo: ``¿eres el Mesías, el Hijo
del Dios bendito?'' \footnote{\textbf{14:61} Mar 15,5; Is 53,7}

\bibleverse{62} ``Lo soy'', respondió Jesús, ``y ustedes verán al Hijo
del hombre sentado a la diestra del Todopoderoso, y viniendo en las
nubes del cielo''.\footnote{\textbf{14:62} Ver Salmos 110:1 y Daniel
  7:13.} \footnote{\textbf{14:62} Dan 7,13-14}

\bibleverse{63} Entonces el sumo sacerdote rasgó sus vestidos\footnote{\textbf{14:63}
  En esos tiempos, esa era una expresión de gran angustia.} y preguntó:
``¿Por qué necesitamos más testigos? \bibleverse{64} ¡Ustedes mismos han
escuchado la blasfemia! ¿Qué tienen para decir?'' Y todos lo hallaron
culpable y lo condenaron a muerte. \footnote{\textbf{14:64} Juan 19,7}
\bibleverse{65} Entonces algunos de ellos comenzaron a escupirlo. Le
vendaron los ojos, lo golpeaban en la cabeza y le decían: ``¡¿Por qué no
profetizas `profeta'?!'' Y los guardias se lo llevaron y lo golpearon.

\hypertarget{negaciuxf3n-y-arrepentimiento-de-pedro}{%
\subsection{Negación y arrepentimiento de
Pedro}\label{negaciuxf3n-y-arrepentimiento-de-pedro}}

\bibleverse{66} Mientras tanto Pedro estaba abajo en el patio. Y una de
las criadas del sumo sacerdote pasaba por allí, \bibleverse{67} y al ver
a Pedro que se calentaba junto a la fogata, lo miró fijamente y dijo:
``¡Tú también estabas con Jesús de Nazaret!''

\bibleverse{68} Pero él lo negó. ``No sé de qué hablas, ni qué quieres
decir'', respondió. Entonces salió al patio delantero, y en ese momento
un gallo cantó.\footnote{\textbf{14:68} ``Y un gallo cantó''. Esta frase
  no aparece en los primeros manuscritos.}

\bibleverse{69} Al verlo, la criada volvió a decir delante de los que
estaban allí: ``¡Este hombre es uno de ellos!'' \bibleverse{70} Y una
vez más Pedro lo negó. Al cabo de un rato, ellos volvieron a decirle a
Pedro: ``¡Definitivamente eres uno de ellos porque eres galileo
también!'' \bibleverse{71} Pedro comenzó a decir maldiciones respecto de
sí mismo y juró: ``No conozco a este hombre del cual hablan ustedes''.

\bibleverse{72} E inmediatamente el gallo cantó la segunda vez. Entonces
Pedro se acordó de lo que Jesús le había dicho: ``Antes de que el gallo
cante dos veces, me negarás tres veces''. Cuando se dio cuenta de lo que
había hecho, rompió a llorar.

\hypertarget{el-interrogatorio-de-jesuxfas-ante-el-gobernador-romano-poncio-pilato-su-condenaciuxf3n-y-flagelaciuxf3n}{%
\subsection{El interrogatorio de Jesús ante el gobernador romano Poncio
Pilato; su condenación y
flagelación}\label{el-interrogatorio-de-jesuxfas-ante-el-gobernador-romano-poncio-pilato-su-condenaciuxf3n-y-flagelaciuxf3n}}

\hypertarget{section-14}{%
\section{15}\label{section-14}}

\bibleverse{1} Temprano, a la mañana siguiente, los jefes de los
sacerdotes, los ancianos y los maestros religiosos---todo el concilio de
gobierno---tomaron una decisión. Mandaron que fuera atado y entregado a
Pilato. \bibleverse{2} Pilato le preguntó: ``¿Eres tú el rey de los
judíos?'' ``Tú lo has dicho'', respondió Jesús.

\bibleverse{3} El jefe de los sacerdotes presentó muchas acusaciones
contra él. \bibleverse{4} Y Pilato le preguntó una vez más: ``¿No vas a
responder? ¡Mira cuántas acusaciones están presentando contra ti!''

\bibleverse{5} pero Jesús no dio más respuestas, para sorpresa de
Pilato.

\bibleverse{6} Y Pilato tenía la costumbre de liberar a un prisionero
durante la Pascua, a quien el pueblo pidiera. \bibleverse{7} Y uno de
los prisioneros era un hombre llamado Barrabás, quien pertenecía a un
grupo de rebeldes que había cometido asesinatos durante una revuelta.
\bibleverse{8} La multitud fue donde Pilato y le pidieron que liberara a
un prisionero como era su costumbre. \bibleverse{9} ``¿Quieren que
libere al Rey de los judíos?'' les preguntó, \bibleverse{10} pues él se
había dado cuenta de que era por celos que los jefes de los sacerdotes
le habían entregado a Jesús. \footnote{\textbf{15:10} Juan 11,48}
\bibleverse{11} Pero los jefes de los sacerdotes causaron revuelo entre
la multitud para que pidieran la liberación de Barrabás. \bibleverse{12}
``¿Entonces qué debo hacer con el que ustedes llaman Rey de los
judíos?'' les preguntó.

\bibleverse{13} ``¡Crucifícalo!'' gritaron en respuesta.

\bibleverse{14} ``¿Por qué? ¿Qué crimen ha cometido?'' les preguntó
Pilato. ``¡Crucifícalo!'' respondieron, gritando más fuerte aún.

\bibleverse{15} Por complacer a la turba, Pilato liberó a Barrabás.
Primero mandó a azotar a Jesús y luego lo entregó para que lo
crucificaran.

\hypertarget{la-burla-y-el-maltrato-de-jesuxfas-por-parte-de-los-soldados-romanos}{%
\subsection{La burla y el maltrato de Jesús por parte de los soldados
romanos}\label{la-burla-y-el-maltrato-de-jesuxfas-por-parte-de-los-soldados-romanos}}

\bibleverse{16} Entonces los soldados se lo llevaron al patio del
Pretorio,\footnote{\textbf{15:16} La residencia oficial del gobernador
  romano.} donde llamaron a toda la cohorte.\footnote{\textbf{15:16} Una
  cohorte era un grupo de aproximadamente unos 600 soldados.}
\bibleverse{17} Le pusieron una bata de color púrpura e hicieron una
corona de espinos, la cual pusieron sobre su cabeza. \bibleverse{18}
Entonces lo saludaban, diciendo: ``¡Salve, Rey de los judíos!''
\bibleverse{19} Y repetidas veces lo golpearon en la cabeza con una
vara, lo escupían y se arrodillaban delante de él como si lo adorasen.

\hypertarget{el-curso-de-la-muerte-de-jesuxfas-despuuxe9s-del-guxf3lgota-su-crucifixiuxf3n-y-su-muerte}{%
\subsection{El curso de la muerte de Jesús después del Gólgota, su
crucifixión y su
muerte}\label{el-curso-de-la-muerte-de-jesuxfas-despuuxe9s-del-guxf3lgota-su-crucifixiuxf3n-y-su-muerte}}

\bibleverse{20} Después que terminaron de burlarse de él, le quitaron la
bata púrpura, y le volvieron a colocar su ropa. Entonces se lo llevaron
para crucificarlo.

\bibleverse{21} Obligaron a un hombre que pasaba por allí, llamado Simón
de Cirene, quien venía del campo, para que cargara la cruz de Jesús.
Simón era el Padre de Alejandro y Rufo. \bibleverse{22} Y trajeron a
Jesús al lugar llamado Gólgota, que significa ``el lugar de la
Calavera''. \bibleverse{23} Allí le ofrecieron vino mezclado con mirra,
pero él se negó a beberlo. \footnote{\textbf{15:23} Sal 69,22}

\bibleverse{24} Entonces lo crucificaron. Repartieron sus prendas de
vestir, y lanzaron dados para decidir quién se llevaría cada
cosa.\footnote{\textbf{15:24} Ver Salmos 22:18.} \footnote{\textbf{15:24}
  Sal 22,19} \bibleverse{25} Eran las nueve de la mañana cuando lo
crucificaron. \bibleverse{26} La acusación escrita en su contra decía:
``El Rey de los judíos''. \bibleverse{27} Con él crucificaron a dos
criminales, uno a su izquierda y otro a su derecha. \bibleverse{28}
\footnote{\textbf{15:28} El versículo 28 no aparece en los manuscritos
  originales.}

\bibleverse{29} Y la gente que pasaba por allí le gritaba con insultos,
sacudiendo sus cabezas y diciendo: ``¡Oye, tu! Tú que decías que ibas a
destruir el Templo y reconstruirlo en tres días: \footnote{\textbf{15:29}
  Mar 14,58} \bibleverse{30} ¡Sálvate a ti mismo y baja de la cruz!''

\bibleverse{31} De igual modo los jefes de los sacerdotes y los maestros
religiosos se burlaban de él, diciendo: ``Salvó a otros, pero no puede
salvarse a sí mismo. \bibleverse{32} Si él realmente es el Mesías, el
Rey de Israel, ¿por qué, entonces, no baja de la cruz para que podemos
ver y creerle?'' Incluso los que estaban crucificados con él lo
insultaban.

\hypertarget{la-muerte-de-jesuxfas-el-signo-milagroso-de-su-muerte}{%
\subsection{La muerte de Jesús; el signo milagroso de su
muerte}\label{la-muerte-de-jesuxfas-el-signo-milagroso-de-su-muerte}}

\bibleverse{33} A medio día hubo una gran oscuridad en toda la tierra
que duró hasta las tres de la tarde. \bibleverse{34} A las tres de la
tarde Jesús gritó: ``Elí, Elí, lema sabactani'', que quiere decir:
``Dios mío, Dios mío, ¿por qué me has abandonado?''\footnote{\textbf{15:34}
  Citando Salmos 22:1.} \footnote{\textbf{15:34} Sal 22,2}

\bibleverse{35} Y algunos de los que estaban allí oyeron esto y dijeron:
``Está llamando a Elías''.

\bibleverse{36} Un hombre corrió y llenó una esponja con vinagre, la
puso en un palo y trató de dárselo a Jesús para que lo
bebiera.\footnote{\textbf{15:36} Ver Salmos 69:21.} ``¡Déjenlo en
paz!'', dijo. ``Veamos si Elías vendrá a bajarlo de ahí''.

\bibleverse{37} Entonces Jesús gimió fuertemente, y murió.
\bibleverse{38} El velo del Templo se rompió de arriba a abajo.
\bibleverse{39} Cuando el centurión que estaba frente a Jesús vio cómo
murió, dijo: ``Este hombre era de verdad el Hijo de Dios''.

\bibleverse{40} Había algunas mujeres mirando a la distancia, incluyendo
a María Magdalena, María la madre de Santiago (el menor) y José, y
Salomé. \bibleverse{41} Ellas habían seguido a Jesús y habían cuidado de
él mientras estuvo en Galilea. Muchas otras mujeres que habían venido
con él a Jerusalén también estaban allí.

\hypertarget{entierro-de-jesuxfas}{%
\subsection{Entierro de Jesús}\label{entierro-de-jesuxfas}}

\bibleverse{42} Era viernes,\footnote{\textbf{15:42} Literalmente, ``el
  día de preparación''.} el día antes del sábado. Cuando llegó la noche,
\bibleverse{43} José de Arimatea, quien era miembro del concilio de
gobierno, y que esperaba el reino de Dios, tuvo la valentía de ir donde
Pilato y pedirle el cuerpo de Jesús. \bibleverse{44} Pilato se
sorprendió al saber que Jesús había muerto tan pronto, así que mandó a
llamar al centurión y le preguntó si Jesús ya había muerto.
\bibleverse{45} Después de tener la confirmación del centurión, Pilato
le dio permiso a José de tomar el cuerpo. \bibleverse{46} José compró
una sábana de lino. Luego bajó el cuerpo de Jesús de la cruz y lo
envolvió en la sábana, y lo colocó en una tumba que había sido elaborada
a partir de una piedra. Después rodó y colocó una piedra pesada en la
entrada de la tumba. \bibleverse{47} María Magdalena y María, la madre
de José, estaban mirando dónde habían colocado a Jesús.

\hypertarget{descubrimiento-de-la-tumba-vacuxeda-en-la-mauxf1ana-de-pascua-la-revelaciuxf3n-del-uxe1ngel-a-las-mujeres}{%
\subsection{Descubrimiento de la tumba vacía en la mañana de Pascua; la
revelación del ángel a las
mujeres}\label{descubrimiento-de-la-tumba-vacuxeda-en-la-mauxf1ana-de-pascua-la-revelaciuxf3n-del-uxe1ngel-a-las-mujeres}}

\hypertarget{section-15}{%
\section{16}\label{section-15}}

\bibleverse{1} Cuando terminó el Sábado, María Magdalena, María la madre
de Santiago y Salomé compraron ungüentos aromáticos para ir a ungir el
cuerpo de Jesús. \bibleverse{2} Y muy temprano, el domingo\footnote{\textbf{16:2}
  Literalmente, ``el primer día de la semana''.} por la mañana, cuando
apenas salía el sol, fueron a la tumba. \bibleverse{3} Se preguntaban
unas a otras: ``¿Quién rodará por nosotras la piedra que está en la
entrada de la tumba?'' \bibleverse{4} Pero cuando llegaron, vieron que
la piedra enorme y pesada ya estaba rodada de su lugar.

\bibleverse{5} Luego entraron a la tumba y vieron a un joven sentado a
la derecha, usaba una bata blanca y larga, y estaban asustadas.
\bibleverse{6} ``No tengan miedo'', les dijo. ``Ustedes buscan a Jesús
el Nazareno, el que fue crucificado. Él se ha levantado de entre los
muertos. No está aquí. \bibleverse{7} Miren, este es el lugar donde lo
pusieron para que descansara. Ahora vayan, y díganles a los discípulos y
a Pedro que él va delante de ustedes a Galilea. Lo verán allí, tal como
les dijo''. \footnote{\textbf{16:7} Mar 14,28}

\bibleverse{8} Ellas se fueron corriendo de la tumba, estaban temblando
y confundidas. No le dijeron a nadie porque estaban muy
asustadas.\footnote{\textbf{16:8} Muchos de los primeros manuscritos del
  libro de Marcos terminan aquí. Como podemos ver, otros continuaron.}

\hypertarget{jesuxfas-se-aparece-a-maruxeda-magdalena-y-a-los-dos-discuxedpulos-de-emauxfas}{%
\subsection{Jesús se aparece a María Magdalena y a los dos discípulos de
Emaús}\label{jesuxfas-se-aparece-a-maruxeda-magdalena-y-a-los-dos-discuxedpulos-de-emauxfas}}

\bibleverse{9} Cuando Jesús se levantó de entre los muertos el domingo
por la mañana, se le apareció primero a María Magdalena, de quien había
expulsado siete demonios. \bibleverse{10} Ella fue y le contó a los que
habían estado con él, cuando ellos estaban llorando y lamentando la
muerte de Jesús. \bibleverse{11} Pero cuando oyeron que Jesús estaba
vivo y que ella lo había visto, no creyeron.

\bibleverse{12} Sin embargo, más tarde Jesús se le apareció de una
manera distinta a otros dos discípulos que se habían ido al campo.
\footnote{\textbf{16:12} Luc 24,13-35} \bibleverse{13} Entonces ellos
regresaron y le contaron a los otros discípulos, pero ellos no les
creyeron.

\hypertarget{la-apariciuxf3n-de-jesuxfas-a-los-once-apuxf3stoles-y-su-mandato-misional}{%
\subsection{La aparición de Jesús a los once apóstoles y su mandato
misional}\label{la-apariciuxf3n-de-jesuxfas-a-los-once-apuxf3stoles-y-su-mandato-misional}}

\bibleverse{14} Después de esto se le apareció a los once discípulos
mientras comían. Jesús los reprendió por su falta de confianza y
terquedad, porque no le habían creído a los que lo habían visto después
que haber resucitado. \bibleverse{15} Entonces les dijo: ``Vayan por
todo el mundo, y anuncien la Buena Noticia a todos.\footnote{\textbf{16:15}
  Literalmente, ``toda la creación''.} \footnote{\textbf{16:15} Mar
  13,10; Mat 28,18-20} \bibleverse{16} Todo el que crea y sea bautizado
será salvo, pero todo el que elija no creer, será condenado. \footnote{\textbf{16:16}
  Hech 2,38; Hech 16,31; Hech 16,33} \bibleverse{17} Estas señales
acompañarán a todos los que creen en mí: expulsarán demonios en mi
nombre, hablarán nuevos idiomas, \footnote{\textbf{16:17} Hech 16,18;
  Hech 10,46; Hech 19,6} \bibleverse{18} y podrán manipular serpientes,
y si toman algo venenoso no les hará daño alguno; pondrán sus manos
sobre los enfermos y estos serán sanados''. \footnote{\textbf{16:18} Luc
  10,19; Hech 28,3-6; Sant 5,14; Sant 1,5-15}

\hypertarget{ascensiuxf3n-de-jesuxfas}{%
\subsection{Ascensión de Jesús}\label{ascensiuxf3n-de-jesuxfas}}

\bibleverse{19} Entonces, el Señor Jesús, cuando terminó de hablarles,
fue llevado hacia el cielo, donde se sentó a la diestra de Dios.
\bibleverse{20} Los discípulos salieron y predicaron la Buena Noticia en
todos lados, y el Señor obraba por medio de ellos, confirmando el
mensaje por medio de muchos milagros.
