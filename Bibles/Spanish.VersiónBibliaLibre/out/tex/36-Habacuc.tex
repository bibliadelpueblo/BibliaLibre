\hypertarget{el-lamento-del-profeta-por-las-puxe9simas-condiciones-de-su-tiempo-y-la-depravaciuxf3n-de-su-pueblo}{%
\subsection{El lamento del Profeta por las pésimas condiciones de su
tiempo y la depravación de su
pueblo}\label{el-lamento-del-profeta-por-las-puxe9simas-condiciones-de-su-tiempo-y-la-depravaciuxf3n-de-su-pueblo}}

\hypertarget{section}{%
\section{1}\label{section}}

\bibleverse{1} Este es el mensaje que Habacuc vio en visión.
\bibleverse{2} Señor, ¿hasta cuándo tendré que clamar por tu ayuda sin
que me escuches? Clamo a ti y digo: ``¡Violencia!'' pero tú no nos
libras de ella. \bibleverse{3} ¿Por qué me obligas a ver esta maldad y
sufrimiento? ¿Por qué te quedas allí simplemente observando la
destrucción y la violencia? ¡Hay riñas y pleitos frente ante mis propios
ojos! \bibleverse{4} Por eso es que la ley está paralizada, y nunca gana
la justicia. Los malvados son más numerosos que los que hacen el bien, y
por eso manipulan la justicia.

\hypertarget{anuncio-del-pruxf3ximo-tribunal-penal-descripciuxf3n-del-terrible-enemigo-que-se-acerca}{%
\subsection{Anuncio del próximo tribunal penal; Descripción del terrible
enemigo que se
acerca}\label{anuncio-del-pruxf3ximo-tribunal-penal-descripciuxf3n-del-terrible-enemigo-que-se-acerca}}

\bibleverse{5} Mira a tu alrededor las naciones; observa y te
sorprenderás.\footnote{\textbf{1:5} Este es el comienzo de la respuesta
  del Señor.} Sucederá algo en tu tiempo que no lo creerías al oírlo.
\footnote{\textbf{1:5} Hech 13,41} \bibleverse{6} ¡Mira! Yo levantaré a
Babilonia,\footnote{\textbf{1:6} Literalmente, ``a los caldeos''.} y
serán un pueblo cruel y salvaje que andará por el mundo conquistando
otras tierras. \bibleverse{7} Son temibles y espantosos, y están tan
llenos de orgullo que solo siguen sus propias reglas.\footnote{\textbf{1:7}
  En otras palabras, hacen lo que quieren.} \bibleverse{8} Sus caballos
son más rápidos que leopardos y más feroces que lobos hambrientos. Sus
jinetes vienen a gran velocidad desde muy lejos.\footnote{\textbf{1:8}
  El texto masorético dice: ``sus jinetes, sí, sus jinetes''. El pesher
  (comentario) de Habacuc en el Qumran (1QpHab) es la base para este
  texto.} Son como águilas que descienden en picada para comerse a su
presa. \bibleverse{9} Aquí vienen, con toda la intención de causar
violencia. Sus ejércitos avanzan para atacar por el frente tan
rápidamente como el viento del desierto, y son como la arena cuando
salen a caprturar a los prisioneros. \bibleverse{10} Se ríen de los
reyes se burlan en la cara de los gobernantes. Se ríen con desprecio de
los castillos, y amontonan rampas de tierra para sitiarlos.
\bibleverse{11} Luego desaparecen como el viento y se van. Son culpables
porque han hecho de su propia fuerza su dios.

\hypertarget{el-profeta-le-preguntuxf3-a-dios-con-reproche-por-quuxe9-permitiruxeda-que-el-enemigo-hiciera-algo-tan-terrible}{%
\subsection{El profeta le preguntó a Dios con reproche por qué
permitiría que el enemigo hiciera algo tan
terrible}\label{el-profeta-le-preguntuxf3-a-dios-con-reproche-por-quuxe9-permitiruxeda-que-el-enemigo-hiciera-algo-tan-terrible}}

\bibleverse{12} ¿No has existido desde la eternidad pasada? Tú eres el
Señor mi Dios, mi Santo, y no mueres. Señor, tu los nombraste para dar
juicio; Dios, nuestra Roca, tú los enviaste para castigarnos.
\footnote{\textbf{1:12} Jer 10,24}

\bibleverse{13} Tus ojos son demasiado puros para ver el mal. No toleras
ver el mal. ¿Por qué has soportado a personas infieles? ¿Por qué guardas
silencio mientas los malvados destruyen a los que hacen menos mal que
ellos? \bibleverse{14} Tú haces que las personas se vuelvan como peces
en el mar, o como insectos que se arrastran, que no tienen quien los
gobierne. \bibleverse{15} Ellos\footnote{\textbf{1:15} Los babilonios.}
arrastran a todos con ganchos, los sacan con redes y los atrapan. Luego
celebran felices. \bibleverse{16} Adoran sus redes como si fueran sus
dioses, hacienda sacrificios y quemando incienso para ellos, porque con
sus redes pueden vivir en medio de lujos, comiendo comida rica.
\bibleverse{17} ¿Seguirán acaso sacando sus espadas\footnote{\textbf{1:17}
  ``Sacando sus espadas'': 1QpHab reading.} para siempre, matando a las
naciones sin piedad?

\hypertarget{habacuc-espera-y-recibe-la-respuesta-de-dios-a-su-queja}{%
\subsection{Habacuc espera y recibe la respuesta de Dios a su
queja}\label{habacuc-espera-y-recibe-la-respuesta-de-dios-a-su-queja}}

\hypertarget{section-1}{%
\section{2}\label{section-1}}

\bibleverse{1} Subiré a mi torre de vigilancia y ocuparé mi lugar en la
muralla de la ciudad. Vigilaré y veré qué me va a decir, y cómo
responderá a mis quejas.

\bibleverse{2} Entonces el Señor me dijo: Escribe la visión, escríbela
en tablas para que puedan leerla fácilmente.\footnote{\textbf{2:2} 2:2
  Literalmente, ``para que la pueda leer el que corre''.} \bibleverse{3}
Porque la visión es para un tiempo futuro. Es sobre el fin y no miente.
¡Si parece demorarse en su cumplimiento, espera, porque sin duda llegará
y no tardará! \bibleverse{4} ¡Mira a los orgullosos!\footnote{\textbf{2:4}
  Una vez más, esto se aplica al tema principal de la visión, al pueblo
  de Babilonia.} No viven con rectitud. Pero los que viven con rectitud
lo hacen mediante su confianza en Dios. \footnote{\textbf{2:4} Is 48,22;
  Rom 1,17; Gal 3,11; Heb 10,38}

\hypertarget{la-revelaciuxf3n-real-de-dios-contra-el-conquistador-salvaje-en-cinco-ayes}{%
\subsection{La revelación real de Dios contra el conquistador salvaje en
cinco
ayes}\label{la-revelaciuxf3n-real-de-dios-contra-el-conquistador-salvaje-en-cinco-ayes}}

\bibleverse{5} Además, la riqueza no brinda seguridad.\footnote{\textbf{2:5}
  ``La riqueza no brinda seguridad'': siguiendo una comprensión del
  pesher Habbakuk (comentario) de Qumran. El texto masorético dice: ``el
  vino es engañoso''.} Los arrogantes nunca tienen paz. Sus bocas
codiciosas están abiertas como una tumba,\footnote{\textbf{2:5}
  Literalmente, ``Seol'', o el lugar de los muertos.} y como la muerte
nunca están satisfechos. Reúnen a las naciones como su fueran su
propiedad, tragándose muchos pueblos.

\bibleverse{6} ¿Acaso no se burlarán de todos estos pueblos? Los
ridiculizarán diciéndoles: ``¡Grande es el desastre que viene sobre
ustedes los que amontonan cosas que no les pertenecen! ¡Se enriquecen
ibligando a sus deudores a pagar! ¿Hasta cuándo podrás seguir haciendo
esto?'' \bibleverse{7} ¿Acaso crees que tus deudores no harán nada?
¿Acaso no aprovecharán la situación para hacerte temblar? ¡Serás
saqueado por ellos! \bibleverse{8} Y como has saqueado a muchas
naciones, los que quedan te saquearán a ti, por la sangre humana que has
derramado y la destrucción que has causado en las naciones y ciudades, y
en los que allí habitaban.

\bibleverse{9} ¡Grande es el desastre que viene sobre ti, que construyes
casas con ganancias deshonestas! Tú crees que puedes poner tu ``nido''
muy alto y que estarás seguro del desastre. \bibleverse{10} Tus planes
malvados han traído vergüenza sobre tus familias, y al destruir muchas
naciones has perdido la vida de los tuyos. \bibleverse{11} Hasta las
piedras en la pared gritan en medio de su condena, y las vigas de madera
se les unen.

\bibleverse{12} ¡Grande es el desastre que viene sobre ti, que
construyes ciudades con derramamiento de sangre, y fundas naciones sobre
los pilares de la maldad! \footnote{\textbf{2:12} Jer 22,13; Miq 3,10}
\bibleverse{13} ¿No ha decidido el Señor que tales naciones serán
destruidas con fuego y que tales naciones se desgastan trabajando por
nada? \footnote{\textbf{2:13} Jer 51,58} \bibleverse{14} Por que la
tierra será llena del conocimiento de la gloria del Señor así como las
aguas llenan el mar. \footnote{\textbf{2:14} Is 11,9}

\bibleverse{15} ¡Grande es el desastre que viene sobre ti, que
emborrachas a tus vecinos! Tú fuerzas tu copa de ira\footnote{\textbf{2:15}
  Or ``veneno''.} sobre ellos y los haces beber para ver su desnudez.
\bibleverse{16} En tu momento te llenarás de vergüenza en lugar de
gloria. Bebe tú mismo y expón tu desnudez!\footnote{\textbf{2:16}
  ``Expose your nakedness'': or ``stagger'' (1QpHab reading).} La copa
que el Señor sostiene en su mano derecha te será entregada y tu gloria
se convertirá en vergüenza. \bibleverse{17} Así como destruíste los
bosques del Líbano, también serás destruido; cazaste a los aniumales
allí y ahora ellos te cazarán a ti.\footnote{\textbf{2:17} Literalmente,
  ``terrify''.} Porque derramaste sangre humana y destruiste naciones y
ciudades con sus habitantes.

\bibleverse{18} ¿De qué sirve un ídolo de madera tallado con manos
humanas, o una imagen de metal que enseña mentira? ¿De qué sirve que sus
creadores confíen en su propia obra, creando ídolos que no pueden habla?
\footnote{\textbf{2:18} Is 44,10} \bibleverse{19} Grande es el desastre
que viene sobre ti, que le dices a un objeto de madera: ``¡Levántate!''
o a una piedra inerte: ``¡Ponte de pie!'' ¿Puede acaso enseñarte algo?
¡Míralo! Está cubierto en oro y plata, pero no hay vida en su interior.
\footnote{\textbf{2:19} Sal 115,4-8} \bibleverse{20} Pero el Señor está
en su santo Templo. Que toda la tierra calle ante su
presencia.\footnote{\textbf{2:20} Sal 11,4; Sal 76,9; Zac 2,17; Apoc 8,1}

\hypertarget{canciuxf3n-a-la-apariciuxf3n-del-seuxf1or-en-el-juicio}{%
\subsection{Canción a la aparición del Señor en el
juicio}\label{canciuxf3n-a-la-apariciuxf3n-del-seuxf1or-en-el-juicio}}

\hypertarget{section-2}{%
\section{3}\label{section-2}}

\bibleverse{1} Esta es una oración cantada por el profeta Habacuc. Con
Sigonot.\footnote{\textbf{3:1} ``Con Sigionot'': el significado es
  desconocido. Puede referirse a un instrumento musical.}

\bibleverse{2} He oído lo que se dice de ti, Señor. Me impresiona tu
obra. Señor, revívela en nuestros tiempos; haz que en nuestro tiempo sea
conocida tu obra. En tu ira, por favor, acuérdate de tu misericordia.

\bibleverse{3} Dios vino desde Temán. El Santo del Monte de
Parán.\footnote{\textbf{3:3} Temán es la tierra de Edom, mientas que el
  Monte de Parán se encuentra en la Península del Sinaí.}
Selah.\footnote{\textbf{3:3} ``Selah'': es un término desconocido que se
  usa a menudo en los salmos.} Su Gloria cubrió los cielos. La tierra se
llenó de su alabanza. \bibleverse{4} Su brillo es como un relámpago. De
su mano salen rayos, y en ellas guarda su poder. \bibleverse{5} Delante
de él viene la plaga, y la enfermedad\footnote{\textbf{3:5} O ``rayos de
  fuego''.} sigue a sus pies. \bibleverse{6} La tierra tiembla
dondequiera que él se queda en pie. Cuando mira, las naciones tiemblan.
Las antiguas montañas y colinas se sacuden y colapsan, pero sus caminos
son eternos. \footnote{\textbf{3:6} Sal 104,32} \bibleverse{7} Vi el
sufriemiento de las tiendas de Cusán, y las cortinas de las tiendas en
la tierra de Madián tiemblan,\footnote{\textbf{3:7} Con esto Habacuc
  probablemente se refiere al pueblo que vivía en estas tiendas.}
\bibleverse{8} ¿Quemaste los ríos con tu ira, Señor? ¿Estabas enojado
con los ríos? ¿Estabas furioso con el mar cuando montaste tus caballos y
carruajes de salvación? \bibleverse{9} Desenfundaste tu arco y llenaste
con flechas tu aljaba. Selah. Tú dividiste la tierra con los ríos.
\bibleverse{10} Las montañas te vieron y se estremecieron. Salió el agua
y se derramó por todo el lugar. Las profundidades salieron a la luz,
formando enormes y altas olas.\footnote{\textbf{3:10} Literalmente,
  ``manos''.} \bibleverse{11} El sol y la luna se detuvieron en el cielo
mientras tus flechas volaban y tus lanzas emanaban luz. \footnote{\textbf{3:11}
  Jos 10,13} \bibleverse{12} Enfurecido, marchaste por la tierra,
pisoteando a las naciones con tu enojo. \bibleverse{13} Saliste a salvar
a tu pueblo, a salvar a tu pueblo escogido. Destruiste la cabeza de los
malvados, despojándolos hasta los huesos.\footnote{\textbf{3:13} Este
  versículo se ha interpretado de muchas formas.} \bibleverse{14} Con
sus propias flechas atravesaste las cabezas de sus guerreros, los que
vinieron en medio de un torbellino para dispersarme, y que se
regocijaban como los que abusan de los pobres en secreto.
\bibleverse{15} Pisoteaste el mar con tus caballos, agitando las
poderosas aguas.

\hypertarget{efecto-aterrador-y-al-mismo-tiempo-alentador-de-la-apariciuxf3n-de-dios-en-el-profeta}{%
\subsection{Efecto aterrador y al mismo tiempo alentador de la aparición
de Dios en el
profeta}\label{efecto-aterrador-y-al-mismo-tiempo-alentador-de-la-apariciuxf3n-de-dios-en-el-profeta}}

\bibleverse{16} Me sacudí por dentro cuando oí esto. Mis labios
temblaron ante el sonido. Mis huesos se volvieron gelatina y temblé allí
donde estaba en pie. Espero en silencio el día en que vendrá la
tribulación sobre aquellos que nos atacaron.

\bibleverse{17} Aunque no haya flores en las higueras ni uvas en los
viñedos; aunque no crezca la cosecha de olivo, ni haya animales en el
corral, o ganado en los establos; \bibleverse{18} aún así me alegraré en
el Señor, gozoso en el Dios de mi salvación. \bibleverse{19} El Señor
Dios es mi fuerza. Él me hace caminar sobre montes altos, con la
seguridad de un ciervo. (Al director musical: con instrumentos de
cuerda).
