\hypertarget{el-amor-de-dios-por-israel-en-contraste-con-su-trato-al-pueblo-hermano-de-edom}{%
\subsection{El amor de Dios por Israel en contraste con su trato al
pueblo hermano de
Edom}\label{el-amor-de-dios-por-israel-en-contraste-con-su-trato-al-pueblo-hermano-de-edom}}

\hypertarget{section}{%
\section{1}\label{section}}

\bibleverse{1} Una profecía:\footnote{\textbf{1:1} Literalmente,
  ``carga''.} Este mensaje vino del Señor respecto a Israel, a través de
Malaquías.

\bibleverse{2} Yo te he amado,\footnote{\textbf{1:2} El tiempo verbal
  indica que no es solo una acción pasada, sino una acción que continúa
  en el presente.} dice el Señor. Pero tú preguntas: ``¿Cómo nos has
amado?'' ¿Acaso no era Esaú el hermano de José? responde el Señor. Pero
yo amé a José

\bibleverse{3} y desprecié a Esaú. He destruido las montañas de Esaú y
transformé su heredad en un desierto para chacales. \footnote{\textbf{1:3}
  Is 34,13} \bibleverse{4} El pueblo de Edom podrá decir: ``Hemos sido
golpeados hasta el polvo, pero reconstruiremos las ruinas''. Pero esto
es lo que dice el Señor Todopoderoso: Aunque lo intenten y construyan,
yo los volveré a destruir. Serán llamados tierra de maldad, y el pueblo
de los que enojan\footnote{\textbf{1:4} La ira divina no puede
  compararse con la ira humana. En este caso tiene el sentido de
  hostilidad de parte de Dios hacia el mal, y no una reacción emotiva
  egoísta.} al Señor para siempre.

\bibleverse{5} Tú\footnote{\textbf{1:5} El pueblo de Israel.} verás esta
destrucción con tus propios ojos, y dirás: ``El Señor es grande, incluso
más allá de los límites de Israel''.

\hypertarget{demostraciuxf3n-de-la-falta-de-amor-y-el-descuido-del-deber-del-sacerdocio-para-con-dios}{%
\subsection{Demostración de la falta de amor y el descuido del deber del
sacerdocio para con
Dios}\label{demostraciuxf3n-de-la-falta-de-amor-y-el-descuido-del-deber-del-sacerdocio-para-con-dios}}

\bibleverse{6} Un hijo honra a su padre, y un siervo respeta a su amo.
Si yo soy su padre, ¿dónde está mi honra? Si soy su amo, ¿dónde está el
respeto que me merezco? dice el Señor Todopoderoso entre tus sacerdotes
que han mostrado desprecio por mi. Pero tú preguntas: ``¿Cómo hemos
mostrado desprecio por ti?'' \bibleverse{7} Presentando ofrendas
contaminadas\footnote{\textbf{1:7} ``Contaminado'': El concepto aquí es
  que los sacerdotes no han seguido las instrucciones del Señor sobre
  cómo debían ofrecerse los sacrificios, demostrando así su falta de
  cuidado y desprecio por el culto al Señor.} en mi altar. Entonces
ustedes preguntan: ``¿Cómo te hemos contaminado?'' Diciendo:\footnote{\textbf{1:7}
  Es posible que los sacerdotes no lo hayan dicho de forma audible, pero
  sus acciones demuestran que lo decían para sí mismos.} la mesa del
Señor no merece respeto. \bibleverse{8} Cuando presentan un animal ciego
como sacrificio, ¿está mal? O cuando ofrecen animals tullidos o
enfermos, ¿no está mal? ¿Presentarían tales ofrendas a un gobernante?
¿Se agradaría él con ustedes? ¿Sería amable y les mostraría su favor?
Pregunta el Señor Todopoderoso. \footnote{\textbf{1:8} Lev 22,20; Lev
  22,23}

\bibleverse{9} ¿Por qué, entonces, no tratan de agradar a Dios, y piden
su misericordia?\footnote{\textbf{1:9} Esta línea a menudo se interpreta
  con un sentido irónico.} Pero cuando traen tales ofrendas, ¿por qué
debería él mostrarles su favor? Pregunta el Señor Todopoderoso.

\bibleverse{10} Desearía que uno de ustedes cerrara las puertas del
Templo, y así cesen sus hogueras\footnote{\textbf{1:10} Las hogueras se
  prendían sobre los altares para quemar los sacrificios. Dios dice aquí
  que no desea tales sacrificios. No tienen sentido porque no
  representan ningún arrepentimiento de parte de los adoradores.} sin
sentido sobre mi altar! No estoy agradado con ustedes, dice el Señor
Todopoderoso, y no aceptaré ofrendas de parte de ustedes.
\bibleverse{11} Me siento honrado por las naciones desde el extremo más
oriental, hasta el extremo más oriental. En todas partes la gente me
trae ofrendas de incienso y sacrificios puros. Soy honrado entre las
naciones, dice el Señor Todopoderoso. \footnote{\textbf{1:11} Is 60,1-7}
\bibleverse{12} Pero ustedes me deshonran cuando dicen que la mesa del
Señor no merece respeto, y que su comida puede ser tratada con
desprecio. \bibleverse{13} Ustedes dicen: ``¡Esto es demasiada
molestia!'' y huelen la comida con repulsión, dice el Señor
Todopoderoso. Pero cuando ustedes me traen como sacrificio animales que
son robados, o están lisiados o enfermos, ¿debería yo aceptar tales
ofrendas? Pregunta el Señor.

\bibleverse{14} Malditos son los que engañan y juran traer un carnero
como sacrificio, y luego traen un animal imperfecto ante el Señor.
¡Porque yo soy un Rey grande, dice el Señor Todopoderoso, y soy
respetado\footnote{\textbf{1:14} Literalmente, ``temido'', pero en este
  ejemplo está vinculado al respeto que se le debe, y que se menciona en
  el versículo 1:6.} entre las naciones!

\hypertarget{advertencia-y-amenaza-de-castigo-a-los-sacerdotes-levi-entonces-y-ahora}{%
\subsection{Advertencia y amenaza de castigo a los sacerdotes; Levi
entonces y
ahora}\label{advertencia-y-amenaza-de-castigo-a-los-sacerdotes-levi-entonces-y-ahora}}

\hypertarget{section-1}{%
\section{2}\label{section-1}}

\bibleverse{1} ¡Ahora este mandamiento\footnote{\textbf{2:1}
  ``Mandamiento'' en el sentido de las instrucciones que debían seguir,
  o una advertencia.} es para tus sacerdotes! \bibleverse{2} Si no
escuchan y si no disponen su corazón para honrarme, dice el Señor
Todopoderoso, yo enviaré maldición sobre ti, y maldeciré tus
bendiciones. De hecho, ya las he maldecido porque ustedes no han abierto
sus corazones para oír mi palabra. \footnote{\textbf{2:2} Deut 28,15}
\bibleverse{3} ¡Tengan cuidado! Yo voy a enviar castigo\footnote{\textbf{2:3}
  O ``reprensión''.} a tus descendientes. Untaré en sus caras el
estiércol de los animales que traen como sacrificio, el estiércol de sus
fiestas religiosas, y ustedes serán expulsados con el estiércol también.
\bibleverse{4} Entonces sabrán que yo mismo les he enviado este
mandamiento, para que mi acuerdo con Leví\footnote{\textbf{2:4} ``Leví''
  no solo se refiere a Leví, sino a su descendencia de sacerdotes.}
pueda permanecer, dice el Señor Todopoderoso. \bibleverse{5} Mi acuerdo
con él era de vida y paz, lo cual le otorgué, y también había respeto.
Él me respetaba. Se maravillaba de mi. \bibleverse{6} Le enseñó al
pueblo la verdad, y no había nada falso en su enseñanza. Él caminó
onmigo en paz e hizo lo recto, y ayudó a muchos a alejarse del pecado.
\bibleverse{7} Un sacerdote debe enseñar la verdad acerca de
Dios,\footnote{\textbf{2:7} ``Un sacerdote debe enseñar la verdad acerca
  de Dios'': Literalmente, ``Un sacerdote debe ser guardián del
  conocimiento''.} las personas deben acudir a él para aprender, porque
él es el mensajero del Señor Todopoderoso. \bibleverse{8} Pero ustedes
se han desviado de mi camino. Han hecho caer a muchos en el pecado. Con
su enseñanza han quebrantado el acuerdo con Leví, dice el Señor
Todopoderoso. \bibleverse{9} Por eso yo he destruido su reputación, y
los he humillado ante todo el pueblo. Porque ustedes no han seguido mis
caminos, y han mostrado preferencia en sus enseñanzas.\footnote{\textbf{2:9}
  O ``No han traído bendiciones al pueblo por medio de sus enseñanzas''.}

\hypertarget{contra-los-matrimonios-con-mujeres-extranjeras-y-contra-los-divorcios-recordatorio-de-fidelidad-matrimonial}{%
\subsection{Contra los matrimonios con mujeres extranjeras y contra los
divorcios; Recordatorio de fidelidad
matrimonial}\label{contra-los-matrimonios-con-mujeres-extranjeras-y-contra-los-divorcios-recordatorio-de-fidelidad-matrimonial}}

\bibleverse{10} ¿Acaso no tenemos todos un mismo Padre? ¿No nos creó a
todos el mismo Dios? ¿Por qué, entonces, somos desleales unos con otros,
violando el acuerdo que hicieron nuestros antiguos padres? \footnote{\textbf{2:10}
  Mal 1,6; Job 31,15} \bibleverse{11} El pueblo de Judá ha sido desleal
y ha cometido pecado repugnante\footnote{\textbf{2:11} ``Un pecado
  repugnante'': o ``una cosa abominable''.} en Israel y en Jerusalén.
Porque los hombres de Judá han contaminado el Templo del
Señor\footnote{\textbf{2:11} ``Templo'': Literalmente, ``santidad''.}
(su Templo amado) al casase con mujeres que adoran ídolos. \footnote{\textbf{2:11}
  Esd 9,2} \bibleverse{12} ¡Que el Señor expulse a la familia de
cualquier hombre de la nación de Israel que haga esto! ¡Que no quede ni
uno solo de ellos que pueda traer ofrenda al Señor
Todopoderoso!\footnote{\textbf{2:12} Existe gran debate sobre el
  significado del testo hebreo que se usa aquí.}

\bibleverse{13} Otra cosa que haces es que derramas lágrimas sobre el
altar del Señor, llorando y lamentándote porque ya el Señor no presta
atención a tus ofrendas o no quiere aceptarlas. \footnote{\textbf{2:13}
  Mal 1,10} \bibleverse{14} ``¿Por qué no?'' preguntas. Porque el Señor
fue testigo de los votos que tú hiciste con tu esposa cuando eran
jóvenes.\footnote{\textbf{2:14} Algunos creen que estos sacerdotes no
  solo se divorciaban de sus esposas, sino que después se casaban con
  mujeres extranjeras.} Pero le fuiste infiel a ella, tu esposa y pareja
que se unió a ti en contrato matrimonial. \bibleverse{15} ¿Acaso no los
hizo uno solo, y les dio de su Espíritu? ¿Y qué es lo que pide de
ustedes? Hijos de Dios.\footnote{\textbf{2:15} Este es uno de los
  versículos más oscuros del Antiguo Testamento y, en consecuencia, hay
  muchas interpretaciones muy diferentes. Algunos toman esta línea para
  referirse a que el hombre y la esposa se convierten en una sola carne,
  como se señala en Génesis. Otros ven al ``no uno'' como el sujeto de
  la oración, por lo que otra posible traducción sería algo así como:
  ``Nadie habría actuado así si hubiera tenido un remanente del
  espíritu''. Algunos han visto al ``uno'' como refiriéndose a Abraham
  como el padre de Israel, y el hecho de que él se divorció de Agar pudo
  haber citado como un precedente para algunos de sus divorcios. La
  respuesta entonces sería que Abraham estaba protegiendo a los ``hijos
  de Dios'' a través de Isaac. Cualquiera sea el caso, parece mejor
  dejar algo de la ambigüedad presente, aunque en el contexto de las
  relaciones matrimoniales parecería que el diseño original para el
  matrimonio en el Edén sería un aspecto relevante que Malaquías quiere
  mencionar.} Así que tengan cuidado con lo que hacen, y no sean
desleales a la esposa con la que se casaron cuando eran jóvenes.
\bibleverse{16} Porque yo aborrezco el divorcio, dice el Señor, Dios de
Israel, porque es un ataque violento contra la esposa,\footnote{\textbf{2:16}
  ``Un ataque violento contra la esposa'': Literalmente, ``cubre sus
  vestidos con violencia''.} dice el Señor Todopoderoso. Así que anden
con cuidado y no sean infieles. \footnote{\textbf{2:16} Deut 24,1}

\hypertarget{reprende-a-los-que-dudan-de-la-justicia-de-dios-descripciuxf3n-del-curso-del-juicio-infalible}{%
\subsection{Reprende a los que dudan de la justicia de Dios; Descripción
del curso del juicio
infalible}\label{reprende-a-los-que-dudan-de-la-justicia-de-dios-descripciuxf3n-del-curso-del-juicio-infalible}}

\bibleverse{17} Ustedes han cansado al Señor con sus
palabras.\footnote{\textbf{2:17} Esta puede ser una referencia a la
  repetición inconsciente de oraciones.} ``¿Cómo lo hemos cansado?''
preguntan ustedes. Al decir que todos los que hacen el mal son Buenos a
la vista del Señor, y que él está a gusto con ellos, o también al
preguntar ¿dónde está la justicia del Señor?\footnote{\textbf{2:17} Mal
  3,13-14}

\hypertarget{section-2}{%
\section{3}\label{section-2}}

\bibleverse{1} ¡Miren! Yo envío a mi mensajero,\footnote{\textbf{3:1}
  Malaquías significa ``mi mensajero''} y él preparará un camino para
mi. El Señor que ustedes buscan\footnote{\textbf{3:1} A la luz de los
  versículos anteriores, la idea de que están buscando a Dios es
  irónica.} llegará de repente a su Templo. El mensajero del acuerdo que
ustedes dicen que está a gusto con ustedes\footnote{\textbf{3:1} ``Que
  ustedes dicen que está a gusto con ustedes''---refiriéndose de nuevo
  al versículo 2:17.} viene pronto, dice el Señor Todopoderoso.
\footnote{\textbf{3:1} Mat 11,10; Mar 1,2; Luc 1,17} \bibleverse{2}
¿Quién podrá sobrevivir en el día de su venida? ¿Quién puede permanecer
en pie delante de él cuando aparezca? Porque él será como un horno
ardiente que refina el metal, o como el jabón fuerte que limpia las
manchas. \footnote{\textbf{3:2} Is 1,25} \bibleverse{3} Él se sentará
como el que refina y purifica la plata; así purificará a los
descendientes de Leví, y los refinará como oro, para que pueda traer
ofrendas puras al Señor. \footnote{\textbf{3:3} Zac 13,9} \bibleverse{4}
Entonces las ofrendas de Judá y de Jerusalén agradarán al Señor, como en
los días de antaño.

\bibleverse{5} Yo vendré y los probaré. Estoy listo para ser testigo
contra: los que cometen hechicería los que cometen adulterio dicen
mentiras dicen falso testimonio engañan a sus empleados oprimen a las
viudas y huérfanos abusan de los extranjeros y contra los que no me
respetan, dice el Señor Todopoderoso.

\hypertarget{la-mala-situaciuxf3n-actual-de-la-gente-es-autoinfligida-a-travuxe9s-de-muxfaltiples-delitos}{%
\subsection{La mala situación actual de la gente es autoinfligida a
través de múltiples
delitos}\label{la-mala-situaciuxf3n-actual-de-la-gente-es-autoinfligida-a-travuxe9s-de-muxfaltiples-delitos}}

\bibleverse{6} Porque yo soy el Señor, y no he cambiado, y ustedes no
han dejad de ser descendientes de Jacob.\footnote{\textbf{3:6} Hay
  debate sobre el significado de esta segunda parte del versículo. Se
  podría interpretar que es debido a la naturaleza inmutable de Dios que
  los descendientes de Jacob no han sido destruidos. Sin embargo, en el
  contexto, parece más probable que Dios diga ``No he cambiado, y tú
  tampoco has cambiado, eres igual que tu antepasado Jacob, que también
  fue un engañador\ldots{}'' La palabra ``cesó'' puede significar
  ``terminó'' como en ``destruido'', o simplemente ``se detuvo''.}
\bibleverse{7} Desde el tiempo de sus antiguos padres y hasta ahora,
ustedes se han apartado de mis leyes y no las han respetado. Vuelvan a
mi, y yo volveré a ustedes, dice el Señor Todopoderoso. Pero ustedes
preguntan: ``¿Cómo debemos volver?''\footnote{\textbf{3:7} El sentido
  parece ser que las personas no ven la necesidad de regresar, y no
  reconocen que han hecho algo malo.}

\bibleverse{8} ¿Debe el pueblo defraudar a Dios? ¡Aún así ustedes me han
defraudado!\footnote{\textbf{3:8} ``Defraudar'': Este término se acerca
  más al significado original que el verbo robar. Además presenta
  continuidad en el tema del pueblo que es descendiente de Jacob, quien
  defraudó a su hermano de su derecho desde el nacimiento.} Pero ustedes
me preguntan ``¿Cómo te hemos defraudado?'' En los diezmos y en las
ofrendas. \bibleverse{9} Ustedes están bajo maldición, porque ustedes y
toda la nación me están defraudando. \footnote{\textbf{3:9} Ag 1,6}
\bibleverse{10} Traigan todo el diezmo a la tesorería para que haya
alimento en mi Templo. Pruébenme en esto, dice el Señor Todopoderoso, y
yo abriré las ventanas de los cielos y haré que sobreabunden las
bendiciones tanto que no tendrán lugar para ellas. \bibleverse{11} Yo
evitaré que vengan plagas de langostas\footnote{\textbf{3:11}
  Literalmente, ``la devoradora''.} en tus cosechas, y tus viñedos no
cesarán de dar fruto, dice el Señor Todopoderoso. \bibleverse{12} Todas
las naciones los llamarán benditos porque vives en una tierra
maravillosa, dice el Señor Todopoderoso.

\hypertarget{reproche-y-correcciuxf3n-de-los-piadosos-descontentos-y-dudosos-promesa-consoladora-a-los-justos}{%
\subsection{Reproche y corrección de los piadosos descontentos y
dudosos; promesa consoladora a los
justos}\label{reproche-y-correcciuxf3n-de-los-piadosos-descontentos-y-dudosos-promesa-consoladora-a-los-justos}}

\bibleverse{13} Ustedes han oído sobre mi, dice el Señor. Pero ustedes
dicen: ``¿Qué hemos dicho contra ti?'' \bibleverse{14} Ustedes han
dicho: ``¿Qué sentido tiene servir a Dios? ¿Qué beneficio hay en guardar
sus mandamientos o en presentarse ante el Señor Todopoderoso con caras
largas?\footnote{\textbf{3:14} ``caras largas'': Literalmente, ``como
  enlutados'' No obstante, no parce que la gente estuviera sinceramente
  arrepentida.} \bibleverse{15} Desde ahora diremos que los orgullosos
son benditos. Los malvados hacen el bien y nada pasa cuando retan a Dios
para que los castigue''.

\bibleverse{16} Entonces los que de verdad respetaban al Señor hablaron
entre ellos y el Señor escuchó lo que dijeron. Un rollo de memorias fue
escrito en su presencia con los nombres de aquellos que respetaban al
Señor y que le prestaron atención.\footnote{\textbf{3:16} ``le prestaron
  atención'': Literalmente, ``Meditaron en su nombre''.} \bibleverse{17}
Ellos serán míos, dice el Señor Todopoderoso, serán mi especial tesoro
en el día en que actúe. Y los trataré con bondad, como un padre trata a
un hijo obediente. \footnote{\textbf{3:17} Éxod 19,5}

\bibleverse{18} Entonces ustedes podrán volver a distinguir a los que
hacen el bien de los que hacen el mal, y también a los que le sirven de
los que no le sirven.

\hypertarget{section-3}{%
\section{4}\label{section-3}}

\bibleverse{1} ¡Tengan cuidado! Viene el día---ardiente como un
horno---en el que los arrogantes y malvados serán quemados como la paja.
Cuando llegue ese día, serán quemados por complete, desde la raíz hasta
las ramas, dice el Señor Todopoderoso. \bibleverse{2} Pero para los que
tienen reverencia por mi, el sol de la salvación de Dios brillará con
curación en sus alas, y serán liberados, saltando como terneros que han
sido liberados de sus establos. \bibleverse{3} Ustedes pisotearán a los
malvados como ceniza bajo sus pies, en el día que yo actúe, dice el
Señor Todopoderoso.

\hypertarget{advertencia-final-y-promesa-de-la-misiuxf3n-del-profeta-eluxedas}{%
\subsection{Advertencia final y promesa de la misión del profeta
Elías}\label{advertencia-final-y-promesa-de-la-misiuxf3n-del-profeta-eluxedas}}

\bibleverse{4} Recuerden la ley de Moisés mi siervo que yo le di a él y
a Israel para que la siguieran. Todas las instrucciones y ceremonias las
enseñé en el Monte Sinaí.\footnote{\textbf{4:4} Literalmente, ``Horeb''.}

\bibleverse{5} ¡Miren! Yo enviaré a Elías el profeta antes de que llegue
el día del Señor, ese día grande y terrible. \bibleverse{6} Él
restaurará la armonía entre padres e hijos, y si eso no ocurre, yo
vendré y azotaré la tierra con una maldición.
