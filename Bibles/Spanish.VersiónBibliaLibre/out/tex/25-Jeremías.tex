\hypertarget{llamando-al-profeta}{%
\subsection{Llamando al profeta}\label{llamando-al-profeta}}

\hypertarget{section}{%
\section{1}\label{section}}

\bibleverse{1} Estas son las palabras de Jeremías hijo de Hilcías, uno
de los sacerdotes que vivía en Anatot, en el territorio de Benjamín.
\bibleverse{2} El mensaje del Señor llegó a Jeremías a partir del año
trece del reinado de Josías, hijo de Amón, rey de Judá, \footnote{\textbf{1:2}
  2Re 21,24} \bibleverse{3} y desde el tiempo de Joacim, hijo de Josías,
rey de Judá, hasta el quinto mes del undécimo año de Sedequías, hijo de
Josías, rey de Judá, que fue cuando el pueblo de Jerusalén partió al
exilio. \footnote{\textbf{1:3} 2Re 23,34; 2Re 24,17; 2Re 25,2; 2Re 25,8}

\hypertarget{el-llamado-y-la-ordenaciuxf3n-de-jeremuxedas-al-oficio-de-profetas}{%
\subsection{El llamado y la ordenación de Jeremías al oficio de
profetas}\label{el-llamado-y-la-ordenaciuxf3n-de-jeremuxedas-al-oficio-de-profetas}}

\bibleverse{4} El Señor vino y me dijo: \bibleverse{5} ``Yo sabía
exactamente quién serías antes de crearte en el vientre de tu madre; te
elegí antes de que nacieras para que fueras profeta de las naciones''.
\footnote{\textbf{1:5} Is 49,1; Gal 1,15}

\bibleverse{6} ``¡Oh, no, Señor Dios!'' respondí. ``¡De verdad que no sé
hablar en público porque todavía soy demasiado joven!''. \footnote{\textbf{1:6}
  Éxod 3,11; Is 6,5-8}

\bibleverse{7} ``No digas que eres demasiado joven'', me dijo el Señor.
``Ve a todos los lugares que yo te mande. Diles todo lo que te ordeno
que digas. \bibleverse{8} No les tengas miedo, porque yo iré contigo y
te cuidaré. Esta es la promesa del Señor''.

\bibleverse{9} El Señor extendió su mano, me tocó la boca y me dijo
``Mira, he puesto mis palabras en tu boca. \footnote{\textbf{1:9} Deut
  18,18} \bibleverse{10} Hoy te he puesto sobre naciones y reinos para
arrancar y derribar, para destruir y demoler, para construir y
plantar''. \footnote{\textbf{1:10} Jer 18,7-10}

\hypertarget{dos-caras-que-llaman-el-uxe1rbol-de-vigilia-y-el-caldero-hirviendo}{%
\subsection{Dos caras que llaman (el árbol de vigilia y el caldero
hirviendo)}\label{dos-caras-que-llaman-el-uxe1rbol-de-vigilia-y-el-caldero-hirviendo}}

\bibleverse{11} El mensaje del Señor llegó a mí, preguntando:
``Jeremías, ¿qué ves?'' ``Veo una ramita de un almendro'', respondí.

\bibleverse{12} ``Así es, porque estoy vigilante\footnote{\textbf{1:12}
  En hebreo, el almendro se llama ``vigilante'' porque es el primer
  árbol que florece en primavera.} para que se cumpla lo que yo digo'',
dijo el Señor. \footnote{\textbf{1:12} Jer 31,28}

\bibleverse{13} El mensaje del Señor llegó de nuevo a mí, preguntando:
``¿Qué ves?'' ``Veo una olla que está hirviendo'', respondí, ``y se está
inclinando en esta dirección desde el norte''.

\bibleverse{14} Entonces el Señor me dijo: ``Los problemas que se están
gestando desde el norte arrasarán con todos los que viven en el país.
\bibleverse{15} ¡Presta atención! Voy a convocar a todas las naciones y
a los reyes del norte'', declara el Señor. ``Cada uno de estos reyes
vendrá y pondrá sus tronos justo a la entrada de las puertas de
Jerusalén, y atacará todas sus fortificaciones y todas las ciudades de
Judá. \bibleverse{16} Cumpliré mi sentencia contra los habitantes por
toda su maldad, porque me abandonaron para ir quemar incienso a los
dioses paganos, para adorar a los ídolos que ellos mismos fabricaron.

\hypertarget{a-trabajar-y-a-la-oficina}{%
\subsection{¡A trabajar y a la
oficina!}\label{a-trabajar-y-a-la-oficina}}

\bibleverse{17} ``Tienes que prepararte. Vas a presentarte ante el
pueblo y a decirles todo lo que yo te ordene. No tengas miedo de ellos,
o yo te asustaré delante de ellos. \bibleverse{18} ¡Presta atención! Hoy
te he convertido en una ciudad fortificada, en una columna de hierro, en
una muralla de bronce, para que te enfrentes a todo el país: contra los
reyes de Judá, sus funcionarios, sus sacerdotes y toda la gente del
país. \bibleverse{19} Ellos lucharán contra ti, pero no te derrotarán,
porque yo estaré allí para rescatarte'', declara el Señor.

\hypertarget{la-lealtad-inicial-de-israel-a-su-dios-y-su-posterior-apostasuxeda-desdeuxf1osa-con-sus-desastrosas-consecuencias}{%
\subsection{La lealtad inicial de Israel a su Dios y su posterior
apostasía desdeñosa con sus desastrosas
consecuencias}\label{la-lealtad-inicial-de-israel-a-su-dios-y-su-posterior-apostasuxeda-desdeuxf1osa-con-sus-desastrosas-consecuencias}}

\hypertarget{section-1}{%
\section{2}\label{section-1}}

\bibleverse{1} Me llegó el mensaje del Señor, diciendo: \bibleverse{2}
Ve y anuncia al pueblo de Jerusalén que esto es lo que dice el
Señor:\footnote{\textbf{2:2} Como es habitual en esta traducción, cuando
  un profeta habla en nombre de Dios no se utilizan comillas. Así se
  evitan demasiadas comillas dentro de comillas, y además a veces
  resulta problemático diferenciar entre las palabras de Dios y las del
  propio profeta. Además, hay muchas ocasiones en las que se repite con
  frecuencia la frase ``Esto dice el Señor'', que normalmente debería
  iniciar un conjunto de comillas, lo que termina con un gran número de
  citas ``anidadas''. Una situación similar se da con la frase ``declara
  el Señor'', que también debería ir fuera de las comillas, pero
  aumentaría considerablemente su número. Por esta razón, las comillas
  se reducen al mínimo y, por lo general, sólo se utilizan cuando es
  necesario representar a otros interlocutores que no sean Jeremías o el
  Señor.} Recuerdo cuán devoto a mi eras cuando eras joven. Recuerdo
cómo me amabas cuando eras mi novia. Recuerdo cómo me seguiste en el
desierto, en una tierra donde no se cultiva nada. \bibleverse{3} Israel
era sagrado para el Señor, las primicias de su cosecha. Cualquiera que
comiera esta cosecha era culpable de pecado, y experimentaba los
resultados desastrosos, declara el Señor.

\bibleverse{4} Escuchen el mensaje del Señor, descendientes de Jacob,
todos ustedes israelitas. \bibleverse{5} Esto es lo que dice el Señor:
¿Qué les pareció a sus antepasados que se alejaron tanto de mí? Se
alejaron para adorar ídolos inútiles, y como resultado se volvieron
inútiles ellos mismos. \footnote{\textbf{2:5} Miq 6,3-6} \bibleverse{6}
No se preguntaron: ``¿Dónde está el Señor que nos sacó de Egipto, que
nos condujo a través del desierto, a través de una tierra de desiertos y
barrancos, una tierra de sequía y oscuridad, una tierra que nadie
recorre y donde nadie vive?'' \bibleverse{7} Los llevé a una tierra
productiva para que comieran todo lo bueno que allí crece. Pero ustedes
vinieron y ensuciaron mi tierra, haciéndola ofensiva para mí.
\bibleverse{8} Sus sacerdotes no preguntaron: ``¿Dónde está el Señor?''
Sus maestros de la ley ya no creyeron en mí, y sus dirigentes se
rebelaron contra mí. Sus profetas profetizaron invocando a Baal y
siguieron a ídolos inútiles. \bibleverse{9} Así que voy a confrontarte
de nuevo, declara el Señor, y presentaré cargos contra los hijos de tus
hijos.

\hypertarget{el-comportamiento-de-israel-es-inaudito-y-sin-precedentes}{%
\subsection{El comportamiento de Israel es inaudito y sin
precedentes}\label{el-comportamiento-de-israel-es-inaudito-y-sin-precedentes}}

\bibleverse{10} Viajen a las islas de Chipre\footnote{\textbf{2:10}
  Hebreo: ``Quitim''. El significado es ir al oeste más lejano.} y echen
un vistazo; vayan a la tierra de Cedar\footnote{\textbf{2:10} Cedar
  estaba muy al este.} y examinen cuidadosamente para ver si algo así ha
sucedido antes. \bibleverse{11} ¿Ha cambiado alguna vez una nación sus
dioses? ¡Aunque no sean ni siquiera dioses en absoluto! Sin embargo, mi
pueblo ha cambiado a su glorioso Dios por ídolos inútiles.
\bibleverse{12} ¡Los cielos deberían estar espantados, escandalizados y
horrorizados! declara el Señor. \bibleverse{13} Porque mi pueblo ha
hecho dos cosas malas. Me han abandonado a mí, la fuente de agua viva, y
han cavado sus propias cisternas: cisternas rotas que no pueden retener
el agua. \footnote{\textbf{2:13} Jer 17,13; Sal 36,10}

\hypertarget{las-terribles-consecuencias-expiaciuxf3n}{%
\subsection{Las terribles consecuencias;
Expiación}\label{las-terribles-consecuencias-expiaciuxf3n}}

\bibleverse{14} ¿Son los israelitas esclavos? ¿Han nacido en la
esclavitud? ¿Por qué se han convertido en víctimas? \bibleverse{15} Los
leones jóvenes han rugido contra ustedes; han gruñido con fuerza. Han
devastado tu país; tus ciudades yacen en ruinas. Nadie vive allí.
\bibleverse{16} Los hombres de Menfis\footnote{\textbf{2:16} ``Menfis'':
  Tomado de la Septuaginta. En hebreo, ``Tof''. Menfis y Tafnes eran
  ciudades de Egipto. El afeitado de cabezas era una humillación
  infligida a un pueblo capturado.} y Tafnes les han afeitado la cabeza.
\bibleverse{17} ¿No te lo has buscado tú mismo al abandonar al Señor, tu
Dios, cuando te guiaba por el camino correcto? \footnote{\textbf{2:17}
  Os 13,9} \bibleverse{18} Ahora bien, ¿en qué te beneficiarás cuando
vuelvas a Egipto a beber las aguas del río Sihor?\footnote{\textbf{2:18}
  El río Sihor era un brazo del río Nilo.} ¿Qué ganarás en tu camino a
Asiria para beber las aguas del río Éufrates? \bibleverse{19} Tu propia
maldad te disciplinará; tu propia desobediencia te dará una lección.
Piénsalo y reconocerás qué amargo mal es para ti abandonar al Señor tu
Dios y no respetarme, declara el Señor Dios Todopoderoso.

\hypertarget{el-servicio-pernicioso-de-baal-y-la-tendencia-desenfrenada-por-la-idolatruxeda}{%
\subsection{El servicio pernicioso de Baal y la tendencia desenfrenada
por la
idolatría}\label{el-servicio-pernicioso-de-baal-y-la-tendencia-desenfrenada-por-la-idolatruxeda}}

\bibleverse{20} Hace tiempo que rompiste tu yugo y te arrancaste las
cadenas. ``¡No te adoraré!'', declaraste. Por el contrario, te acostaste
como una prostituta en toda colina alta y bajo todo árbol verde.
\bibleverse{21} Yo fui quien te plantó como la mejor cepa, cultivada a
partir de la mejor semilla. ¿Cómo pudiste degenerar en una inútil vid
silvestre? \footnote{\textbf{2:21} Is 5,1-4} \bibleverse{22} Ni siquiera
la lejía y el jabón en abundancia pueden eliminar tus manchas de culpa.
Todavía las veo, declara el Señor Dios. \bibleverse{23} ¿Cómo te atreves
a decir: ``¡No estoy impuro! No he ido a adorar a los baales''. Mira lo
que has hecho en el valle. ¡Admite lo que has hecho! Eres un camello
hembra joven, que corre por todas partes. \bibleverse{24} Eres una burra
que vive en el desierto, olfateando el viento en busca de pareja porque
está en celo. Nadie puede controlarla en la época de celo. Todos los que
la buscan no tendrán problemas para encontrarla cuando esté en celo.
\bibleverse{25} No hace falta que corra descalza ni que se le seque la
garganta. Pero tú respondes: ``¡No, es imposible! Estoy enamorado de los
dioses extranjeros, debo ir a ellos''.

\hypertarget{la-idolatruxeda-indigna-y-la-apostasuxeda-incomprensible-la-mala-administraciuxf3n-de-justicia-y-el-mal-gobierno}{%
\subsection{La idolatría indigna y la apostasía incomprensible, la mala
administración de justicia y el mal
gobierno}\label{la-idolatruxeda-indigna-y-la-apostasuxeda-incomprensible-la-mala-administraciuxf3n-de-justicia-y-el-mal-gobierno}}

\bibleverse{26} De la misma manera que un ladrón se siente culpable
cuando es atrapado, así el pueblo de Israel ha sido avergonzado. Todos
ellos: sus reyes, sus funcionarios, sus sacerdotes y sus profetas.
\bibleverse{27} Le dicen a un ídolo de madera: ``Tú eres mi padre'', y a
uno de piedra: ``Tú me diste a luz''. Me dan la espalda y me ocultan el
rostro. Pero cuando están en apuros vienen a suplicarme, diciendo:
``¡Por favor, ven a salvarnos!''. \bibleverse{28} Entonces, ¿dónde están
esos ``dioses'' suyos que se han fabricado? ¡Que vengan a ayudarlos
cuando estén en apuros! Que los salven si pueden, porque ustedes,
israelitas, tienen tantos dioses como pueblos. \bibleverse{29} ¿Por qué
se quejan ante mí? Son todos ustedes los que se han rebelado contra mí!
declara el Señor. \bibleverse{30} Fue inútil que castigara a tus hijos
porque se negaron a aceptar cualquier disciplina. Usaste tus propias
espadas para matar a tus profetas, destruyéndolos como un león feroz.
\footnote{\textbf{2:30} Is 1,5} \bibleverse{31} Pueblo de hoy, piensa en
lo que dice el Señor: Israel, ¿te he tratado como un desierto vacío, o
como una tierra de densas tinieblas? ¿Por qué dice mi pueblo: ``¡Podemos
ir donde queramos! Ya no tenemos que venir a adorarte''? \bibleverse{32}
¿Acaso una muchacha olvida sus joyas o una novia su vestido de
novia?\footnote{\textbf{2:32} ``Vestido'': probablemente un ``fajón''.}
Sin embargo, mi pueblo me ha olvidado durante demasiados años para
contarlos.

\bibleverse{33} ¡Cuán astutamente buscas a tus amantes! ¡Hasta las
prostitutas podrían aprender algo de ti! \bibleverse{34} Además, tus
ropas están manchadas con la sangre de los pobres y de los inocentes. No
es que los hayas matado entrando en tus casas. A pesar de todo esto,
\bibleverse{35} sigues diciendo: ``¡Soy inocente! Ciertamente no puedes
seguir enfadado conmigo''. ¡Escucha con atención! Te voy a castigar
porque sigues diciendo: ``Yo no he pecado''.

\bibleverse{36} ¡Eres tan inconstante que sigues cambiando de opinión!
Terminarás tan decepcionado por tu alianza con Egipto como lo estuviste
con Asiria. \bibleverse{37} De hecho, irás al exilio con las manos en la
cabeza como los prisioneros, porque el Señor no tendrá nada que ver con
aquellos en los que ustedes confían; y ellos no les servirán de ayuda.

\hypertarget{es-posible-aceptar-a-las-personas-que-han-sido-violadas-por-la-idolatruxeda}{%
\subsection{¿Es posible aceptar a las personas que han sido violadas por
la
idolatría?}\label{es-posible-aceptar-a-las-personas-que-han-sido-violadas-por-la-idolatruxeda}}

\hypertarget{section-2}{%
\section{3}\label{section-2}}

\bibleverse{1} Si un hombre se divorcia de su mujer y ella se va y se
casa con otro, ¿podría este hombre volver con ella? ¿No quedaría el país
totalmente impuro por ello? Pero ustedes han hecho algo peor al
prostituirse con muchos amantes, ¿y ahora quieren volver a mí? declara
el Señor. \footnote{\textbf{3:1} Deut 24,1-4}

\bibleverse{2} Miren hacia las cumbres desnudas. ¿Hay algún lugar donde
no hayan tenido relaciones sexuales? Se han sentado al borde del camino,
como un errante en el desierto, esperando que pasen sus amantes. Han
ensuciado la tierra con su prostitución y su maldad. \bibleverse{3} Por
eso no se ha enviado rocío ni han caído lluvias de primavera. Pero tú te
limitas a comportarte como una prostituta; te niegas a aceptar que has
hecho algo malo. \bibleverse{4} ¿No me acabas de decir: ``Padre mío, has
sido un gran amigo mío desde que era pequeño.

\bibleverse{5} No te enfadarás conmigo durante mucho tiempo, ¿verdad?
¿No seguirás así siempre?'' Esto es lo que has dicho, pero sigues
haciendo todo el mal posible.

\hypertarget{israel-y-juduxe1-como-hermanas-renegadas-recordatorio-para-arrepentirse-y-regresar-a-casa}{%
\subsection{Israel y Judá como hermanas renegadas; Recordatorio para
arrepentirse y regresar a
casa}\label{israel-y-juduxe1-como-hermanas-renegadas-recordatorio-para-arrepentirse-y-regresar-a-casa}}

\bibleverse{6} Durante el reinado del rey Josías, el Señor me dijo: ¿Has
visto lo que ha hecho el infiel Israel? Se ha prostituido en todo monte
alto y bajo todo árbol verde. \bibleverse{7} Esperaba que, después de
hacer todo esto, volviera a mí. Pero no volvió, y su hermana infiel,
Judá, vio lo que pasó. \bibleverse{8} Ella\footnote{\textbf{3:8} Tomado
  de los Rollos del Mar Muerto. El texto masorético dice ``Yo''.} vio
que por todo lo que había hecho la infiel Israel al cometer adulterio,
la rechacé, dándole un certificado de divorcio. Pero su hermana infiel
Judá no tuvo miedo y se prostituyó también. \footnote{\textbf{3:8} 2Re
  17,18-19; Ezeq 23,2-11} \bibleverse{9} A Israel no le importó la
inmoralidad, pues se ensució a sí misma y a la tierra, cometiendo
adulterio al rendirle culto a las piedras y a los árboles.
\bibleverse{10} A pesar de todo esto, su infiel hermana Judá no volvió a
mí con sinceridad. Sólo fingió hacerlo, declara el Señor.

\bibleverse{11} El Señor me dijo: La infiel Israel demostró que no era
tan culpable como la infiel Judá. \bibleverse{12} Ahora ve y anuncia
este mensaje al norte:\footnote{\textbf{3:12} Las diez tribus del norte
  habían sido llevadas al norte al exilio en Asiria.} Vuelve, Israel
infiel, declara el Señor. No me enfadaré más contigo, porque soy
misericordioso, declara el Señor. No me enfadaré para siempre.
\bibleverse{13} Reconoce que hiciste mal, que te rebelaste contra el
Señor, tu Dios. Te dispersaste, cometiendo adulterio al adorar a dioses
extranjeros bajo cualquier árbol verde, negándote a hacer lo que te
dije, declara el Señor.

\hypertarget{llama-al-arrepentimiento-palabra-de-salvaciuxf3n-sobre-jerusaluxe9n-y-juduxe1}{%
\subsection{Llama al arrepentimiento; Palabra de salvación sobre
Jerusalén y
Judá}\label{llama-al-arrepentimiento-palabra-de-salvaciuxf3n-sobre-jerusaluxe9n-y-juduxe1}}

\bibleverse{14} Vuelvan, hijos infieles, declara el Señor, porque estoy
casado con ustedes. Los tomaré, uno de un pueblo y dos de una familia, y
los llevaré a Sión. \footnote{\textbf{3:14} Os 2,21; Is 6,13}
\bibleverse{15} Os daré pastores que sean como yo, que os alimentarán
con sabiduría y entendimiento. \footnote{\textbf{3:15} Jer 23,4}
\bibleverse{16} En ese momento, a medida que ustedes aumenten en número
en el país, declara el Señor, ya nadie hablará del Arca del Acuerdo del
Señor. La gente no necesitará pensar en ella, ni recordarla, ni
preguntarse qué pasó con ella; y ciertamente no necesitará hacer una
nueva. \bibleverse{17} Cuando llegue ese momento, Jerusalén será llamada
el Trono del Señor, y todas las naciones se reunirán en Jerusalén para
honrar al Señor. Ya no serán tercos ni malvados. \footnote{\textbf{3:17}
  Is 2,2-4; Is 65,2} \bibleverse{18} En ese momento el pueblo de Judá se
unirá al pueblo de Israel, y volverán de la tierra del norte al país que
les di a sus antepasados para que lo poseyeran. \footnote{\textbf{3:18}
  Is 11,11-13}

\hypertarget{nostuxe1lgica-mirada-atruxe1s-a-la-infidelidad-de-la-gente}{%
\subsection{Nostálgica mirada atrás a la infidelidad de la
gente}\label{nostuxe1lgica-mirada-atruxe1s-a-la-infidelidad-de-la-gente}}

\bibleverse{19} Me dije: Quiero que sean mis hijos, y darles el mejor
país, el lugar más hermoso de cualquier nación. Esperaba que me
llamarais ``Padre'' y que nunca dejarais de seguirme. \footnote{\textbf{3:19}
  Jer 3,4}

\bibleverse{20} Pero al igual que una esposa puede traicionar a su
marido, ustedes me han traicionado, pueblo de Israel, declara el Señor.

\hypertarget{canciuxf3n-de-arrepentimiento-del-pueblo}{%
\subsection{Canción de arrepentimiento del
pueblo}\label{canciuxf3n-de-arrepentimiento-del-pueblo}}

\bibleverse{21} Hay voces que claman en las cimas de los montes: los
israelitas lloran y piden misericordia, porque se han extraviado y se
han olvidado del Señor, su Dios. \bibleverse{22} Volved, hijos infieles,
y yo curaré vuestra infidelidad. ``¡Ya estamos aquí! Sí, volvemos a ti,
porque tú eres el Señor, nuestro Dios''.

\bibleverse{23} No hay duda de que el culto pagano de las colinas es
pura mentira; la idolatría que viene de las montañas es sólo
ruido.\footnote{\textbf{3:23} El hebreo de este verso es impreciso.} La
salvación de Israel está sólo en el Señor, nuestro Dios. \bibleverse{24}
Durante toda nuestra vida, la idolatría pagana ha destruido lo que
nuestros padres tanto trabajaron: sus rebaños y manadas, sus hijos e
hijas. \bibleverse{25} Deberíamos acostarnos avergonzados, y que nuestra
desgracia nos sepulte. Hemos pecado contra el Señor, nuestro Dios,
nosotros y nuestros padres. Desde que éramos jóvenes hasta ahora no
hemos obedecido lo que el Señor, nuestro Dios, nos dijo que hiciéramos.

\hypertarget{promesa-de-aceptaciuxf3n-despuuxe9s-de-un-arrepentimiento-sincero}{%
\subsection{Promesa de aceptación después de un arrepentimiento
sincero}\label{promesa-de-aceptaciuxf3n-despuuxe9s-de-un-arrepentimiento-sincero}}

\hypertarget{section-3}{%
\section{4}\label{section-3}}

\bibleverse{1} Israel, si quieres volver, vuelve a mí, declara el Señor.
Si te deshaces de esos ídolos desagradables que veo, y no te alejas,
\bibleverse{2} y si cuando hagas tus votos, lo haces sólo a mí, con
sinceridad, verdad y honestidad, entonces serán bendecidas las naciones
por mí, y me alabarán. \footnote{\textbf{4:2} Jer 12,16; Is 65,16}

\bibleverse{3} Esto es lo que el Señor dice al pueblo de Judá y de
Jerusalén: Siembren su tierra sin arar, y no siembren entre los
espinos.\footnote{\textbf{4:3} No se trata de un consejo agrícola, sino
  de una invitación a dejar de ser duros y obstinados y a abrirse al
  Señor para ser espiritualmente productivos.} \footnote{\textbf{4:3} Os
  10,12} \bibleverse{4} Dedíquense al Señor; comprométanse totalmente
con él,\footnote{\textbf{4:4} La imagen utilizada aquí es la de la
  ``circuncisión espiritual''.} pueblo de Judá y Jerusalén. De lo
contrario, mi ira arderá como el fuego, ardiendo con tanta fuerza que
nadie podrá apagarla a causa del mal que has hecho. \footnote{\textbf{4:4}
  Jer 9,25; Deut 10,16}

\hypertarget{guerra-la-apariciuxf3n-del-terrible-enemigo-del-norte}{%
\subsection{¡Guerra! La aparición del terrible enemigo del
norte}\label{guerra-la-apariciuxf3n-del-terrible-enemigo-del-norte}}

\bibleverse{5} ¡Anuncien esta advertencia por todo Judá y Jerusalén!
Díganles: ¡Toquen la trompeta en todo el país! Griten: ``¡Rápido!
Corramos hacia las ciudades fortificadas para protegernos''.
\bibleverse{6} ¡Icen la bandera del peligro; vayan a Sión! ¡Busquen un
lugar seguro! ¡No duden! Traigo enemigos del norte que causarán una
terrible destrucción.

\bibleverse{7} Un león ha salido de su escondite; un destructor de
naciones ha salido. Ha salido de su guarida para venir a convertir tu
país en un páramo. Tus ciudades serán demolidas, y nadie vivirá en
ellas. \bibleverse{8} Vistan ropas de cilicio, lloren y lamenten,
gritando: ``La furia del Señor contra nosotros no ha cesado''.

\bibleverse{9} Cuando eso ocurra, declara el Señor, el rey y los
funcionarios desesperarán, los sacerdotes quedarán abatidos y los
profetas se escandalizarán.

\bibleverse{10} Entonces dije: ``Oh, Señor Dios, has engañado
completamente al pueblo de Jerusalén diciéndole: `Tendrás paz', mientras
nos pones una espada en la garganta''. \footnote{\textbf{4:10} Jer 6,14}

\bibleverse{11} En ese momento se le dirá al pueblo de Jerusalén: ``Un
viento ardiente de las colinas desnudas del desierto está soplando hacia
Jerusalén, pero no para llevarse la paja o el polvo. \bibleverse{12} No,
este viento es demasiado fuerte para eso, y viene de mí. Ahora también
voy a decirles cómo los voy a castigar''.

\bibleverse{13} Mira, se precipita como nubes de tormenta; sus carros
son como un torbellino. Sus caballos son más rápidos que las águilas.
``¡Qué desastre! Estamos arruinados!''

\bibleverse{14} Limpia el mal de tu corazón, Jerusalén, para que puedas
salvarte. ¿Hasta cuándo te aferrarás a tus malos pensamientos?

\bibleverse{15} Las noticias llegan a gritos desde Dan, anunciando el
desastre desde las colinas de Efraín. \bibleverse{16} ``¡Que se enteren
las naciones! ¡Miren lo que está sucediendo! Anuncien esto a Jerusalén:
Un ejército está viniendo a asediarte desde un país lejano; dando gritos
de guerra contra las ciudades de Judá. \bibleverse{17} La rodean como
hombres que cuidan un campo, porque se ha rebelado contra mí, declara el
Señor. \footnote{\textbf{4:17} Jer 1,15; Jer 6,3} \bibleverse{18} Tú
misma provocaste esto con tus propias actitudes y acciones. Este es tu
castigo, y es tan doloroso que es como si te apuñalaran en el corazón''.

\bibleverse{19} ``Estoy\footnote{\textbf{4:19} Jeremías es quien habla
  aquí.} en agonía, ¡en absoluta agonía! ¡Mi corazón se está rompiendo!
¡Late salvajemente en mi pecho! Mi corazón late dentro de mí; no puedo
callar porque he oído la trompeta, la señal de batalla. \bibleverse{20}
``Las noticias de una catástrofe tras otra llegan a raudales, pues todo
el país está en ruinas. Mi propia casa se destruye en un instante, y
también todo lo que hay dentro. \bibleverse{21} ¿Hasta cuándo tengo que
ver las banderas de guerra y oír las trompetas de batalla?''

\bibleverse{22} ``Mi\footnote{\textbf{4:22} El Señor es quien habla
  aquí.} gente es estúpida; no me conocen. Son niños tontos que no
entienden. Son expertos en hacer el mal, pero no saben hacer el bien''.
\bibleverse{23} Miré la tierra, y estaba sin forma y vacía;\footnote{\textbf{4:23}
  Cita directa de Génesis 1:2.} Miré a los cielos, y su luz había
desaparecido. \bibleverse{24} Miré a las montañas y vi que temblaban;
todas las colinas se agitaban de un lado a otro. \bibleverse{25} Miré, y
no quedaba nadie; todas las aves habían volado. \bibleverse{26} Miré, y
los campos fértiles eran un desierto. Todas las ciudades fueron
demolidas por la furia del Señor. \bibleverse{27} Esto es lo que dice el
Señor: ``Todo el país será devastado, pero no lo haré completamente.
\footnote{\textbf{4:27} Jer 5,10; Jer 5,18} \bibleverse{28} La tierra se
enlutará y los cielos se oscurecerán. Yo he hablado; esto es lo que he
ordenado. No me detendré ni cambiaré de opinión''.

\bibleverse{29} Habitantes de todos los pueblos: huyan cuando oigan
venir a los jinetes y arqueros enemigos. Escóndanse en el bosque y entre
las rocas. Todas las ciudades están abandonadas; nadie vive en ellas.
\bibleverse{30} Tú, Jerusalén, ahora desolada, ¿qué vas a hacer? Aunque
te vistas con ropas de color escarlata, y te pongas joyas de oro, y te
maquilles los ojos, ¡todo tu adorno es inútil! Tus amantes te odian;
¡quieren matarte! \bibleverse{31} Oigo los gritos de una mujer que está
dando a luz, los gemidos agónicos de una mujer que da a luz a su primer
hijo. Son los gritos de la Hija de Sión, que jadea y extiende las manos
diciendo: ``¡Por favor, ayúdenme, me están matando!''

\hypertarget{la-terrible-corrupciuxf3n-de-todo-el-pueblo-de-jerusaluxe9n-obliga-al-seuxf1or-a-ejecutar-sin-piedad-el-castigo-del-terrible-enemigo}{%
\subsection{La terrible corrupción de todo el pueblo de Jerusalén obliga
al Señor a ejecutar sin piedad el castigo del terrible
enemigo}\label{la-terrible-corrupciuxf3n-de-todo-el-pueblo-de-jerusaluxe9n-obliga-al-seuxf1or-a-ejecutar-sin-piedad-el-castigo-del-terrible-enemigo}}

\hypertarget{section-4}{%
\section{5}\label{section-4}}

\bibleverse{1} Ve a todas partes por las calles de Jerusalén. Busca y
presta atención. Busca por todas las plazas de su ciudad a ver si
encuentras aunque sea una sola persona que haga lo correcto, alguien que
sea fiel, y yo perdonaré a la ciudad. \bibleverse{2} Pueden hacer
promesas en mi nombre, pero no son sinceras.

\bibleverse{3} Señor, ¿no buscas siempre la fidelidad? Los derrotaste,
pero no les importó. Estuviste a punto de destruirlos, pero se negaron a
aceptar tu disciplina. Eran tercos, duros como una roca, y no se
arrepentían.

\bibleverse{4} Entonces me dije: ``Esta gente no es más que los pobres;
son sólo tontos que no conocen nada mejor. Ciertamente no saben lo que
quiere el Señor, la manera correcta de vivir de Dios. \bibleverse{5}
Déjame ir a hablar con los que mandan. Ellos seguramente sabrán lo que
quiere el Señor, la forma correcta de vivir de Dios''. Pero todos habían
roto también el yugo, y arrancado las cadenas. \footnote{\textbf{5:5}
  Jer 2,20} \bibleverse{6} Como resultado, un león del bosque los
atacará; un lobo del desierto los desgarrará. Un leopardo los acechará
cerca de sus ciudades, listo para despedazar a cualquiera que salga.
Porque no dejan de rebelarse y se alejan de mí tantas veces. \footnote{\textbf{5:6}
  Lev 26,22}

\hypertarget{comienzo-del-discurso-amenazador-de-dios}{%
\subsection{Comienzo del discurso amenazador de
Dios}\label{comienzo-del-discurso-amenazador-de-dios}}

\bibleverse{7} ¿Por qué habría de perdonarlos? Tus hijos me han
abandonado y creen en dioses que no son dioses. Les he dado todo lo que
necesitan, y sin embargo han ido a cometer adulterio, reuniéndose en
casas de prostitutas. \bibleverse{8} Son como sementales viriles con
ganas de sexo, cada uno de ellos relinchando de lujuria tras la mujer de
su vecino. \bibleverse{9} ¿No debería yo castigarlos por todo esto?
declara el Señor. ¿No debo tomar represalias por lo que ha hecho esta
nación? \footnote{\textbf{5:9} Jer 5,29}

\hypertarget{el-juicio-divino-para-el-apuxf3stata-e-incruxe9dulo-desafiante}{%
\subsection{El juicio divino para el apóstata e incrédulo
desafiante}\label{el-juicio-divino-para-el-apuxf3stata-e-incruxe9dulo-desafiante}}

\bibleverse{10} Atraviesa sus viñedos y destrúyelos, pero no los
destruyas por completo. Arranca sus ramas, porque no le pertenecen al
Señor. \footnote{\textbf{5:10} Jer 4,27} \bibleverse{11} El pueblo de
Israel y de Judá me ha traicionado completamente, declara el Señor.

\bibleverse{12} Han mentido acerca del Señor, diciendo: ``Él no hará
nada. No nos ocurrirá nada malo. No tendremos guerra ni hambre.
\bibleverse{13} Los profetas son como el viento. El Señor no habla a
través de ellos. Lo que predicen puede ocurrirles a ellos''.

\bibleverse{14} Esta es la respuesta del Señor Dios Todopoderoso: Por lo
que has dicho, haré que mis palabras sean como un fuego en tu boca y que
tú seas como la leña que quema.

\hypertarget{el-terrible-enemigo}{%
\subsection{El terrible enemigo}\label{el-terrible-enemigo}}

\bibleverse{15} ¡Mira! Traigo una nación de muy lejos para atacarte,
pueblo de Israel, declara el Señor. Es una nación poderosa que existe
desde hace mucho tiempo; es una nación cuya lengua no conoces, y cuando
habla no puedes entenderla. \footnote{\textbf{5:15} Jer 6,22}
\bibleverse{16} Sus flechas traen la muerte;\footnote{\textbf{5:16}
  ``Sus flechas traen la muerte'': literalmente, ``sus aljabas son como
  una tumba abierta''.} todos ellos son fuertes guerreros.
\bibleverse{17} Consumirán tu cosecha y tu comida; destruirán a tus
hijos y a tus hijas; se comerán tus rebaños y tus manadas; se
alimentarán de tus viñas y de tus higueras. Atacarán y destruirán las
ciudades fortificadas en las que tanto confías.

\hypertarget{la-causa-del-castigo-divino-es-decir-del-exilio}{%
\subsection{La causa del castigo divino, es decir, del
exilio}\label{la-causa-del-castigo-divino-es-decir-del-exilio}}

\bibleverse{18} Pero ni siquiera en ese momento te destruiré por
completo, declara el Señor. \bibleverse{19} Cuando la gente te pregunte,
Jeremías, ``¿Por qué el Señor, nuestro Dios, nos ha hecho todas estas
cosas?'' , les dirás: ``De la misma manera que ustedes me han abandonado
y han servido a dioses extranjeros aquí en su país, así servirán a
extranjeros en un país que no es el suyo''.

\hypertarget{la-ignorancia-del-pueblo-la-codicia-de-las-clases-altas-y-la-deshonestidad-del-clero}{%
\subsection{La ignorancia del pueblo, la codicia de las clases altas y
la deshonestidad del
clero}\label{la-ignorancia-del-pueblo-la-codicia-de-las-clases-altas-y-la-deshonestidad-del-clero}}

\bibleverse{20} Anuncia esto al pueblo de Jacob y de Judá:
\bibleverse{21} Escuchen esto, pueblo necio y estúpido, que tiene ojos y
no ve, que tiene oídos y no oye. \bibleverse{22} ¿No tienen miedo de lo
que puedo hacer? declara el Señor. ¿No creen que deberían temblar en mi
presencia? Yo soy el que puso la orilla como límite del mar, un límite
eterno que no puede cruzar. Las olas chocan contra ella, pero no pueden
vencerla. Rugen, pero no pueden cruzar la barrera. \footnote{\textbf{5:22}
  Job 38,8-11}

\bibleverse{23} Pero ustedes tienen una actitud obstinada y rebelde. Me
han dejado y se han ido por su propia cuenta. \bibleverse{24} No han
pensado ni siquiera en decir: ``Debemos apreciar al Señor, nuestro Dios,
que envía las lluvias de otoño y primavera en el momento oportuno, que
hace que podamos tener una cosecha cada año''.

\bibleverse{25} Tus malas acciones te han quitado estos beneficios; tus
pecados te han privado de mis bendiciones. \footnote{\textbf{5:25} Is
  59,2} \bibleverse{26} Porque hay hombres malvados en mi pueblo. Son
como cazadores de pájaros, que vigilan en secreto y esperan atrapar a la
gente en su trampa. \bibleverse{27} Sus casas están llenas de sus
ganancias mal habidas, como jaulas llenas de pájaros. Por eso se han
hecho poderosos y ricos. \bibleverse{28} Han engordado y se han hecho
expertos en el mal. Niegan la justicia a los huérfanos, y no defienden
los derechos de los necesitados.

\bibleverse{29} ¿No debería yo castigarlos por todo esto? declara el
Señor. ¿No debo tomar represalias por lo que ha hecho esta nación?
\footnote{\textbf{5:29} Jer 5,9}

\bibleverse{30} Algo horrible, algo terrible ha ocurrido en este país.
\bibleverse{31} Los profetas dan falsas profecías; los sacerdotes
gobiernan a su antojo. Mi pueblo lo quiere así, pero ¿qué hará cuando
todo se derrumbe?

\hypertarget{anuncio-renovado-de-la-guerra-inminente-nueva-descripciuxf3n-del-tamauxf1o-del-dauxf1o-interno}{%
\subsection{Anuncio renovado de la guerra inminente; nueva descripción
del tamaño del daño
interno}\label{anuncio-renovado-de-la-guerra-inminente-nueva-descripciuxf3n-del-tamauxf1o-del-dauxf1o-interno}}

\hypertarget{section-5}{%
\section{6}\label{section-5}}

\bibleverse{1} Corran y escóndanse, descendientes de
Benjamín,\footnote{\textbf{6:1} Jerusalén formaba parte del territorio
  original de Benjamín.} ¡salgan de Jerusalén! Toquen la trompeta en
Tecoa; enciendan una señal de fuego en Bet-hacquerem, porque el desastre
y la terrible destrucción están llegando desde el norte. \bibleverse{2}
Aunque sea bonita y encantadora, destruiré\footnote{\textbf{6:2} La
  palabra utilizada aquí suele significar ``parecerse'', pero se utiliza
  en el sentido de ``destruir'' en Oseas 4:5.} a la hija de Sión.
\bibleverse{3} ``Los pastores'' y sus ``rebaños''\footnote{\textbf{6:3}
  Claramente una alusión a los ejércitos invasores con sus generales.}
vendrán a atacarla; instalarán sus tiendas alrededor de ella, cada uno
cuidando la suya. \footnote{\textbf{6:3} Jer 4,17}

\bibleverse{4} Se preparan para la batalla contra ella, diciendo:
``¡Vamos, atacaremos al mediodía! Oh, no, el día está a punto de
terminar, las sombras de la tarde se alargan. \bibleverse{5} ¡Vamos,
atacaremos de noche y destruiremos sus fortalezas!'' \bibleverse{6} Esto
es lo que dice el Señor Todopoderoso: Corten los árboles y hagan una
rampa de asedio para usarla contra Jerusalén. Esta ciudad necesita ser
castigada porque está llena de gente que se maltrata. \bibleverse{7}
Como un manantial que rebosa\footnote{\textbf{6:7} ``Rebosa'': o
  ``refresca''.} de agua, por lo que vierte su maldad. Los sonidos de la
violencia y el abuso resuenan en su interior. Veo gente enferma y herida
por todas partes. \bibleverse{8} Te advierto, pueblo de Jerusalén, que
voy a abandonarte con disgusto. Te destruiré y dejaré tu país
deshabitado.

\hypertarget{santa-ira-del-profeta-por-la-depravaciuxf3n-de-todas-las-personas-especialmente-los-luxedderes-amenaza-de-desastre}{%
\subsection{Santa ira del profeta por la depravación de todas las
personas, especialmente los líderes; Amenaza de
desastre}\label{santa-ira-del-profeta-por-la-depravaciuxf3n-de-todas-las-personas-especialmente-los-luxedderes-amenaza-de-desastre}}

\bibleverse{9} Esto es lo que dice el Señor Todopoderoso: Incluso los
que queden en Israel serán tomados, como las uvas que quedan en una vid
son tomadas por el que cosecha las uvas que vuelve a revisar las ramas.

\bibleverse{10} ¿A quién puedo dar esta advertencia? ¿Quién va a
escucharme? ¿No ves que se niegan a escuchar?\footnote{\textbf{6:10}
  ``Se niegan a escuchar'': literalmente, ``Tienen oídos
  incircuncisos''.} No pueden escuchar lo que estoy diciendo. Vean lo
ofensivo que es el mensaje del Señor para ellos. No les gusta en
absoluto. \bibleverse{11} Pero en cuanto a mí, estoy lleno de la ira del
Señor; me cuesta mucho contenerla. El Señor responde,\footnote{\textbf{6:11}
  ``El Señor responde'': añadido para mayor claridad.} Derrámalo sobre
los niños en la calle, y sobre los grupos de jóvenes, porque tanto el
marido como la mujer van a ser capturados; son todos, y no importa la
edad que tengan. \bibleverse{12} Sus casas serán entregadas a otros, sus
campos y sus esposas también, porque voy a castigar a todos los que
viven en este país, declara el Señor. \bibleverse{13} Todos engañan
porque son codiciosos, tanto los pobres como los ricos. Incluso los
profetas y los sacerdotes: ¡todos son unos mentirosos deshonestos!
\footnote{\textbf{6:13} Jer 8,10-12} \bibleverse{14} Le dan a mis
heridos los primeros auxilios, pero en realidad no se preocupan por
ellos. Les dicen: ``¡No se preocupen! Tenemos paz!'', aun cuando la
guerra se acerca. \footnote{\textbf{6:14} Ezeq 13,10; Ezeq 13,16; 1Tes
  5,3} \bibleverse{15} ¿Se avergonzaron de las cosas repugnantes que
hicieron? No, no se avergonzaron en absoluto, ni siquiera pudieron
sonrojarse. Por eso caerán como los demás, cuando los castigue; caerán
muertos, dice el Señor.

\hypertarget{todo-el-trabajo-de-dios-ha-sido-en-vano-el-juicio-es-inevitable}{%
\subsection{Todo el trabajo de Dios ha sido en vano; el juicio es
inevitable}\label{todo-el-trabajo-de-dios-ha-sido-en-vano-el-juicio-es-inevitable}}

\bibleverse{16} Esto es lo que dice el Señor: Ve y párate donde se
dividen los caminos, y mira. Averigua cuáles son los caminos antiguos.
Pregunta: ``¿Cuál es el camino correcto?'' . Luego síguelo y estarás
contento.\footnote{\textbf{6:16} ``Estarán contentos:'' literalmente,
  ``encontrarán descanso para sus almas''.} Pero os negasteis, diciendo:
``¡No iremos por ahí!''. \footnote{\textbf{6:16} Mat 11,29; Jer 44,16}
\bibleverse{17} Puse vigilantes a cargo de ustedes y les dije que se
aseguraran de escuchar el llamado de la trompeta que les advertía del
peligro. Pero ustedes respondieron: ``¡No escucharemos!''. \footnote{\textbf{6:17}
  Is 52,8; Ezeq 3,17} \bibleverse{18} Así que ahora ustedes, otras
naciones, pueden escuchar y averiguar lo que les va a pasar.
\bibleverse{19} Tierra, ¡escucha tú también! Estoy haciendo caer el
desastre sobre este pueblo, el resultado final de lo que ellos mismos
planearon. Es porque no prestaron atención a lo que dije y rechazaron
mis instrucciones. \footnote{\textbf{6:19} Deut 32,1; Is 1,2}
\bibleverse{20} ¿De qué sirve ofrecerme incienso de Saba o cálamo dulce
de una tierra lejana? No acepto sus holocaustos; no me agradan sus
sacrificios. \footnote{\textbf{6:20} Is 1,11}

\bibleverse{21} Así que esto es lo que dice el Señor: Voy a poner
bloques delante de esta gente para hacerla tropezar. Padres e hijos
caerán muertos, amigos y vecinos también.

\hypertarget{el-terrible-enemigo-del-norte-ante-el-cual-juduxe1-se-hunde-en-dolor-y-ruina}{%
\subsection{El terrible enemigo del norte, ante el cual Judá se hunde en
dolor y
ruina}\label{el-terrible-enemigo-del-norte-ante-el-cual-juduxe1-se-hunde-en-dolor-y-ruina}}

\bibleverse{22} Esto es lo que dice el Señor: ¡Mira! Un ejército invade
desde el norte; una nación poderosa se prepara para atacar desde los
confines de la tierra. \footnote{\textbf{6:22} Jer 5,15; Deut 28,49}
\bibleverse{23} Recogen sus arcos y sus lanzas. Son crueles y no tienen
piedad. Sus gritos de guerra son como el rugido del mar, y montan
caballos alineados listos para atacarte, hija de Sion. \footnote{\textbf{6:23}
  Jer 50,42}

\bibleverse{24} El pueblo responde,\footnote{\textbf{6:24} ``El pueblo
  responde'': añadido para mayor claridad.} ``Nos hemos enterado de la
noticia y nuestras manos están inmovilizadas por la conmoción. Nos
invade la agonía y sufrimos dolores como una parturienta.
\bibleverse{25} ¡No vayas al campo! ¡No caminen por el camino! ¡El
enemigo está armado con espadas! El terror está en todas partes''.
\bibleverse{26} Oh, pueblo mío, vístete de cilicio y revuélcate en
cenizas. Llora y llora amargamente como lo harías por un hijo único,
porque el destructor descenderá sobre ti de repente. \footnote{\textbf{6:26}
  Am 8,10}

\hypertarget{jeremuxedas-como-probador-de-metales-su-trabajo-inuxfatil-en-su-pueblo}{%
\subsection{Jeremías como probador de metales: su trabajo inútil en su
pueblo}\label{jeremuxedas-como-probador-de-metales-su-trabajo-inuxfatil-en-su-pueblo}}

\bibleverse{27} Jeremías, te he hecho probador de metales para que
pruebes a mi pueblo como si fuera metal, para que sepas de qué está
hecho y cómo actúa. \bibleverse{28} Son unos rebeldes obstinados que van
por ahí diciendo calumnias. Son duros como el bronce y el hierro; están
todos corrompidos. \bibleverse{29} Los fuelles del horno del refinador
soplan con fuerza, quemando el plomo. Pero esta refinación es inútil,
porque los impíos no están purificados. \bibleverse{30} Son
identificados como plata impura que hay que rechazar, porque el Señor
los ha rechazado.\footnote{\textbf{6:30} Is 1,22}

\hypertarget{contra-el-culto-puramente-externo-y-la-obvia-desobediencia-del-pueblo}{%
\subsection{Contra el culto puramente externo y la obvia desobediencia
del
pueblo}\label{contra-el-culto-puramente-externo-y-la-obvia-desobediencia-del-pueblo}}

\hypertarget{section-6}{%
\section{7}\label{section-6}}

\bibleverse{1} Este es el mensaje que le llegó a Jeremías de parte del
Señor: \bibleverse{2} Ve y ponte a la entrada del Templo del Señor, y
entrega este mensaje: Escuchen lo que el Señor tiene que decir, todos
ustedes de Judá que entran por estas puertas para adorar al Señor.

\bibleverse{3} Esto es lo que dice el Señor Todopoderoso, el Dios de
Israel: Cambien sus costumbres y hagan lo correcto, y los dejaré seguir
viviendo aquí. \bibleverse{4} No creas en los que intentan engañarte
repitiendo: ``El Templo del Señor está aquí, el Templo del Señor está
aquí, el Templo del Señor está aquí''.\footnote{\textbf{7:4} En otras
  palabras, como el Templo del Señor estaba ubicado en Jerusalén, el
  Señor nunca permitiría que la ciudad fuera conquistada.}
\bibleverse{5} Si cambian con sinceridad su manera de actuar y hacen lo
que es correcto, si se tratan con justicia unos a otros, \bibleverse{6}
si dejan de maltratar a los extranjeros, a los huérfanos y a las viudas,
y si dejan de asesinar a gente inocente y de hacerse daño a sí mismos
con sus cultos, \footnote{\textbf{7:6} Éxod 22,20-21} \bibleverse{7}
entonces les dejaré seguir viviendo aquí, en el país que les di a sus
antepasados, por los siglos de los siglos. \bibleverse{8} ¡Pero mírense!
Seguís creyendo en estos engaños, en estas palabras sin valor.
\bibleverse{9} ¿Realmente van a seguir robando, asesinando, cometiendo
adulterio y mintiendo, quemando incienso a Baal y adorando a otros
dioses de los que no saben nada, \bibleverse{10} y luego vienen a
pararse frente a mí en mi propio Templo y dicen: ``Estamos a salvo, así
que podemos seguir haciendo todas estas cosas ofensivas''?
\bibleverse{11} ¿Consideran que esta casa, mi propio Templo, es una
cueva de ladrones? Pues eso es lo que me parece a mí también, declara el
Señor. \footnote{\textbf{7:11} Mat 21,13}

\hypertarget{el-templo-seruxe1-destruido-tanto-como-lo-fue-antes-el-santuario-en-silo-si-la-gente-continuxfaa-transgrediendo}{%
\subsection{El templo será destruido tanto como lo fue antes el
santuario en Silo si la gente continúa
transgrediendo}\label{el-templo-seruxe1-destruido-tanto-como-lo-fue-antes-el-santuario-en-silo-si-la-gente-continuxfaa-transgrediendo}}

\bibleverse{12} Entonces, ¿por qué no van a Silo\footnote{\textbf{7:12}
  Véase Salmos 78:60.} donde me hice por primera vez un lugar para vivir
contigo, y mira lo que le hice por el mal que hizo mi pueblo Israel?
\footnote{\textbf{7:12} Juan 18,1; 1Sam 4,12; Sal 78,60} \bibleverse{13}
Te he advertido una y otra vez sobre todas estas cosas que has hecho,
pero no has querido escuchar, declara el Señor. Te he llamado, pero no
has querido responderme. \footnote{\textbf{7:13} Prov 1,24; Is 65,12}
\bibleverse{14} Así que ahora voy a hacer con mi Templo lo que hice con
Silo. Este es el Templo en el que pusiste tu fe, el lugar que les di a
ti y a tus antepasados. \footnote{\textbf{7:14} Jer 26,6}
\bibleverse{15} Te expulsaré de mi presencia, así como expulsé a todos
tus parientes israelitas, a todos los descendientes de
Efraín.\footnote{\textbf{7:15} Refiriéndose a la captura y el exilio de
  las diez tribus del norte.} \footnote{\textbf{7:15} 2Re 17,18; 2Re
  17,20; 2Re 17,23}

\hypertarget{rechazo-de-la-intercesiuxf3n-del-profeta-el-culto-iduxf3latra-de-la-reina-pagana-del-cielo}{%
\subsection{Rechazo de la intercesión del Profeta; el culto idólatra de
la reina pagana del
cielo}\label{rechazo-de-la-intercesiuxf3n-del-profeta-el-culto-iduxf3latra-de-la-reina-pagana-del-cielo}}

\bibleverse{16} Tú, Jeremías, no debes orar por esta gente. No me clames
en oración por ellos, no me ruegues en su favor, porque no te escucharé.
\footnote{\textbf{7:16} Jer 11,14; Jer 14,11} \bibleverse{17} ¿No ves
cómo se comportan en las ciudades de Judá y en las calles de Jerusalén?
\bibleverse{18} Los niños recogen la leña, los padres encienden el fuego
y las mujeres amasan la masa para hacer tortas para la Reina del Cielo,
y derraman libaciones a otros dioses para hacerme enojar y herir.
\footnote{\textbf{7:18} Jer 44,17} \bibleverse{19} Pero, ¿es a mí a
quien realmente hieren? declara el Señor. ¿No se están lastimando a sí
mismos y se están avergonzando?

\bibleverse{20} Esto es lo que dice el Señor: ¡Mira! Mi ira se derramará
sobre este país, sobre las personas y los animales, sobre los huertos y
las cosechas del campo. Arderá y nadie podrá apagarlo.

\hypertarget{dios-quiere-obediencia-no-sacrificio-y-no-adoraciuxf3n-elegida-por-uno-mismo}{%
\subsection{Dios quiere obediencia, no sacrificio y no adoración elegida
por uno
mismo}\label{dios-quiere-obediencia-no-sacrificio-y-no-adoraciuxf3n-elegida-por-uno-mismo}}

\bibleverse{21} Esto es lo que dice el Señor Todopoderoso, el Dios de
Israel: Pueden agregar sus holocaustos a sus otros sacrificios y comer
toda la carne ustedes mismos!\footnote{\textbf{7:21} Esto, por supuesto,
  no estaba permitido en la ley levítica. Sin embargo, lo que el Señor
  está diciendo es que, ya que no aceptará sus sacrificios, también
  podrían comer toda la carne ellos mismos.} \bibleverse{22} Cuando
saqué a tus antepasados de Egipto no sólo les di instrucciones sobre
holocaustos y sacrificios, \footnote{\textbf{7:22} Miq 6,6-8; 1Sam 15,22}
\bibleverse{23} Este es el mandamiento que les di: Obedézcanme, y yo
seré su Dios, y ustedes serán mi pueblo. Sigan todo lo que les he
mandado hacer, para que todo les vaya bien. \footnote{\textbf{7:23} Éxod
  19,5} \bibleverse{24} Pero no quisieron escuchar ni prestar atención.
En lugar de ello, siguieron los deseos de su propio pensamiento
obstinado y malvado, por lo que terminaron retrocediendo y no avanzando.
\footnote{\textbf{7:24} Jer 11,8; Is 65,2} \bibleverse{25} Desde que sus
antepasados salieron de Egipto hasta ahora, les he enviado una y otra
vez a mis siervos los profetas. \bibleverse{26} Pero ustedes no
quisieron escuchar ni prestarles atención. Por el contrario, se
volvieron más tercos y rebeldes que sus antepasados.

\bibleverse{27} Cuando les dices todo esto, no te escuchan. Cuando los
llamas, no responden. \bibleverse{28} Así que tienes que decirles:
``Esta es la nación que se negó a escuchar lo que dijo el Señor, su
Dios, y no quiso aceptar la disciplina del Señor. La verdad se ha
extinguido; la gente ni siquiera habla de ella. \footnote{\textbf{7:28}
  Jer 5,1}

\hypertarget{la-aborrecible-idolatruxeda-encontraruxe1-una-terrible-expiaciuxf3n}{%
\subsection{La aborrecible idolatría encontrará una terrible
expiación}\label{la-aborrecible-idolatruxeda-encontraruxe1-una-terrible-expiaciuxf3n}}

\bibleverse{29} Córtense el pelo y tírenlo.\footnote{\textbf{7:29} Ya
  sea en señal de luto (lo cual estaba prohibido en Deuteronomio 14:1,
  presumiblemente como una práctica pagana), o como señal de que habían
  roto su voto a Dios como si fueran un nazareno (Números 6:5).} Canten
una canción de duelo en las colinas desnudas, porque el Señor ha
rechazado y abandonado a la generación que lo hizo enojar''.

\bibleverse{30} Porque el pueblo de Judá ha hecho el mal a mis ojos,
declara el Señor. Han colocado sus ídolos ofensivos en mi propio Templo,
volviéndolo impuro. \bibleverse{31} Han construido santuarios paganos en
Tofet, en el Valle de Hinom, para poder sacrificar a sus hijos e hijas
quemándolos en el fuego. Esto es algo que nunca ordené. Nunca pensé en
algo así. \footnote{\textbf{7:31} 2Re 23,10; Lev 18,21} \bibleverse{32}
¡Así que cuidado! Se acerca el tiempo, declara el Señor, en que en lugar
de Tofet y el Valle de Hinom este lugar se llamará Valle de la Matanza.
La gente enterrará a sus muertos en Tofet hasta que se llene.
\footnote{\textbf{7:32} Jer 19,6} \bibleverse{33} Los cadáveres de este
pueblo serán alimento para las aves de rapiña y los animales salvajes, y
no habrá nadie que los espante. \footnote{\textbf{7:33} Jer 19,7; Jer
  9,21} \bibleverse{34} Pondré fin a los sonidos alegres de la
celebración y a las voces felices de los novios de las ciudades de Judá
y de las calles de Jerusalén, porque el país se convertirá en un
desierto.\footnote{\textbf{7:34} Jer 16,9}

\hypertarget{el-vergonzoso-destino-de-los-iduxf3latras-tras-la-conquista-de-la-tierra}{%
\subsection{El vergonzoso destino de los idólatras tras la conquista de
la
tierra}\label{el-vergonzoso-destino-de-los-iduxf3latras-tras-la-conquista-de-la-tierra}}

\hypertarget{section-7}{%
\section{8}\label{section-7}}

\bibleverse{1} Cuando eso suceda, declara el Señor, los huesos de los
reyes de Judá, los huesos de los funcionarios, los huesos de los
sacerdotes, los huesos de los profetas y los huesos del pueblo de
Jerusalén serán sacados de sus tumbas. \bibleverse{2} Yacerán expuestos
al sol y a la luna, y a todas las estrellas que amaron, a las que
sirvieron, a las que siguieron, a las que consultaron y a las que
adoraron.\footnote{\textbf{8:2} La repetición enfatiza la incapacidad de
  estos ``dioses'' de hacer algo para proteger incluso los huesos de sus
  adoradores.} Sus huesos no se recogerán ni se volverán a enterrar,
sino que se quedarán como estiércol tirado en el suelo. \footnote{\textbf{8:2}
  Deut 4,19; Jer 14,16} \bibleverse{3} Los que queden de esta familia
malvada preferirán morir antes que vivir en todos los lugares donde los
he dispersado, declara el Señor Todopoderoso.

\hypertarget{contra-la-gente-impenitente-y-la-arrogancia-de-los-luxedderes-espirituales-los-horrores-del-juicio-inminente}{%
\subsection{Contra la gente impenitente y la arrogancia de los líderes
espirituales; los horrores del juicio
inminente}\label{contra-la-gente-impenitente-y-la-arrogancia-de-los-luxedderes-espirituales-los-horrores-del-juicio-inminente}}

\bibleverse{4} Diles que esto es lo que dice el Señor: Cuando la gente
se cae, ¿no se levanta de nuevo? Cuando la gente se equivoca de camino,
¿no se regresa? \bibleverse{5} Entonces, ¿por qué este pueblo de
Jerusalén se ha equivocado de camino? ¿Por qué se niegan a arrepentirse
de sus repetidas traiciones, aferrándose a todas sus mentiras?
\bibleverse{6} He oído exactamente lo que han dicho, pero no dicen la
verdad. Nadie se arrepiente de haber hecho el mal, preguntando: ``¿Qué
he hecho?'' . Cada uno elige su propio camino, como un caballo que se
lanza a la batalla. \bibleverse{7} Incluso las cigüeñas en lo alto del
cielo saben cuándo es el momento de emigrar. Las tórtolas, los vencejos
y los pájaros cantores saben cuándo volar en el momento adecuado del
año. Pero mi pueblo no conoce las leyes del Señor.

\bibleverse{8} ¿Cómo pueden decir: ``Somos sabios y tenemos la Ley del
Señor''? ¿No ves que los escritos de tus maestros de la Ley la han
convertido en mentira? \bibleverse{9} Los sabios se mostrarán necios; se
escandalizarán al ser descubiertos. ¿No ven que han rechazado lo que
dice el Señor? ¿Acaso tienen alguna sabiduría?

\hypertarget{amenaza-contra-los-luxedderes-espirituales-del-pueblo}{%
\subsection{Amenaza contra los líderes espirituales del
pueblo}\label{amenaza-contra-los-luxedderes-espirituales-del-pueblo}}

\bibleverse{10} Voy a entregar sus esposas a otros, y sus campos a
diferentes dueños, ya que todos mienten porque son codiciosos, tanto los
pobres como los ricos. Incluso los profetas y los sacerdotes: ¡todos son
unos mentirosos deshonestos! \footnote{\textbf{8:10} Jer 6,13-15; Is
  56,11} \bibleverse{11} Le dan a mis heridos los primeros auxilios,
pero en realidad no se preocupan por ellos. Les dicen: ``¡No te
preocupes! Tenemos paz!'', aunque se acerque la guerra. \bibleverse{12}
¿Se avergüenzan de las cosas repugnantes que hicieron? No, no se
avergüenzan en absoluto, ni siquiera son capaces de sonrojarse. Por eso
caerán como los demás, cuando los castigue; caerán muertos, dice el
Señor. \bibleverse{13} Voy a destruirlos, declara el Señor. No quedarán
uvas en las vides, ni higos en los árboles; hasta las hojas se
marchitarán. Perderán todo lo que les di.

\hypertarget{los-horrores-del-juicio-de-dios-que-se-lleva-a-cabo-mediante-el-acercamiento-del-enemigo-del-norte}{%
\subsection{Los horrores del juicio de Dios, que se lleva a cabo
mediante el acercamiento del enemigo del
norte}\label{los-horrores-del-juicio-de-dios-que-se-lleva-a-cabo-mediante-el-acercamiento-del-enemigo-del-norte}}

\bibleverse{14} La gente dice: ``¿Por qué estamos sentados aquí?
Juntémonos y corramos a las ciudades fortificadas. Allí podemos morir,
porque el Señor, nuestro Dios, nos está matando dándonos a beber agua
envenenada, porque pecamos contra él. \bibleverse{15} Esperábamos la
paz, pero en lugar de eso no ha llegado nada bueno; esperábamos un
tiempo de curación, pero en lugar de eso sólo ha habido terror
repentino''. \footnote{\textbf{8:15} Jer 14,19} \bibleverse{16} El
bramido de los caballos enemigos se oye desde Dan.\footnote{\textbf{8:16}
  Dan estaba en el norte del país y sería el primero en experimentar la
  invasión.} Todo el país se estremece de miedo al oír los relinchos de
estos fuertes sementales. Han venido a destruir el país y todo lo que
hay en él; Jerusalén y todos los que viven en ella. \bibleverse{17}
¡Cuidado! Estoy enviando serpientes entre ustedes, víboras que no pueden
ser encantadas. Vendrán a morderte, declara el Señor.

\hypertarget{la-desesperanza-del-profeta-y-su-dolor-por-la-alteraciuxf3n-moral-del-pueblo}{%
\subsection{La desesperanza del profeta y su dolor por la alteración
moral del
pueblo}\label{la-desesperanza-del-profeta-y-su-dolor-por-la-alteraciuxf3n-moral-del-pueblo}}

\bibleverse{18} Nada me consuela\footnote{\textbf{8:18} Jeremías es
  quien habla aquí.} en medio de mi sufrimiento;\footnote{\textbf{8:18}
  El hebreo de esta línea es confuso.} Me siento terrible por dentro.
\footnote{\textbf{8:18} Jer 4,19} \bibleverse{19} Escucha a mi pueblo
clamando por ayuda desde una tierra lejana, preguntando: ``¿Ya no está
presente el Señor en Sión? ¿Se ha ido su Rey?'' ¿Por qué me han
hecho\footnote{\textbf{8:19} El Señor es quien habla aquí.} enojar,
adorando sus imágenes esculpidas y sus inútiles ídolos extranjeros?
\bibleverse{20} ``La cosecha ha terminado, el verano ha acabado, pero no
estamos salvados'', dice la gente.\footnote{\textbf{8:20} ``Dice la
  gente'': añadido para mayor claridad}

\bibleverse{21} Estoy abatido por las heridas sufridas por mi pueblo;
estoy de luto por ellos. ¡Estoy horrorizado por lo que ha sucedido!
\bibleverse{22} ¿No hay ningún ungüento de Galaad que ayude a curarlos?
¿No hay médicos allí? ¿Por qué mi pueblo no se ha curado de sus heridas?

\hypertarget{jeremuxedas-se-desespera-de-su-pueblo-moralmente-quebrantado}{%
\subsection{Jeremías se desespera de su pueblo moralmente
quebrantado}\label{jeremuxedas-se-desespera-de-su-pueblo-moralmente-quebrantado}}

\hypertarget{section-8}{%
\section{9}\label{section-8}}

\bibleverse{1} Cómo quisiera que mi cabeza fuera un manantial de agua, y
mis ojos una fuente de lágrimas. Entonces lloraría día y noche por todo
mi pueblo que ha sido asesinado. \bibleverse{2} Ojalá tuviera un refugio
temporal en el desierto; renunciaría a mi pueblo y lo abandonaría,
porque todos son adúlteros, una banda de traidores. \bibleverse{3} Sus
palabras son como flechas lanzadas desde un arco. La mentira se impone a
la verdad en todo el país. Van de mal en peor y se olvidan de mí,
declara el Señor. \bibleverse{4} ¡Cuidado con tus amigos! ¡Ni siquiera
confíes en tu hermano! Todo hermano es engañoso, y todo amigo calumnia a
los demás. \bibleverse{5} Todos traicionan a sus amigos; nadie dice la
verdad. Se han convertido en expertos mentirosos; se cansan de hacer el
mal. \bibleverse{6} Todos se explotan mutuamente, y en medio de todas
sus mentiras no quieren conocerme, declara el Señor.

\bibleverse{7} Así que esto es lo que dice el Señor Todopoderoso: Mira,
voy a probarlos y a purificarlos como el metal en un horno. ¿Qué más
puedo hacer por lo que ha hecho mi pueblo? \bibleverse{8} Sus palabras
son flechas que matan; siempre dicen mentiras. Por fuera son amables con
sus amigos, pero por dentro conspiran contra ellos. \footnote{\textbf{9:8}
  Jer 5,9}

\hypertarget{el-lamento-del-profeta-juicio-de-dios-lamentaciones}{%
\subsection{El lamento del profeta; Juicio de Dios;
Lamentaciones}\label{el-lamento-del-profeta-juicio-de-dios-lamentaciones}}

\bibleverse{9} ¿No debo castigarlos por todo esto? declara el Señor. ¿No
debo tomar represalias por lo que ha hecho esta nación? \footnote{\textbf{9:9}
  Jer 4,25; Jer 12,4} \bibleverse{10} Lloraré y me lamentaré por los
montes, cantaré un canto fúnebre sobre los pastos del campo, porque han
quedado tan quemados que nadie puede pasar por ellos, y no hay ganado
que haga ruido. Las aves han volado y los animales salvajes han huido.
\footnote{\textbf{9:10} Jer 26,18}

\hypertarget{la-pregunta-del-profeta-y-el-juicio-de-dios-como-respuesta}{%
\subsection{La pregunta del profeta y el juicio de Dios como
respuesta}\label{la-pregunta-del-profeta-y-el-juicio-de-dios-como-respuesta}}

\bibleverse{11} Voy a convertir a Jerusalén en un montón de escombros,
en un lugar donde viven los chacales. Destruiré las ciudades de Judá, y
las dejaré vacías. \footnote{\textbf{9:11} Deut 32,29}

\bibleverse{12} ¿Quién es tan sabio como para entender esto? ¿Le ha
dicho el Señor esto a alguien para que pueda explicar lo que ha
sucedido? ¿Por qué la tierra ha sido destruida y quemada hasta quedar
como un desierto, para que nadie pueda pasar por ella?

\bibleverse{13} El Señor respondió: Es porque han dejado de cumplir mis
leyes que les puse enfrente. No las han seguido; no han hecho lo que les
dije. \footnote{\textbf{9:13} Jer 7,24} \bibleverse{14} Por el
contrario, han seguido su propia y obstinada manera de pensar, y han ido
a adorar a los baales, tal como sus antepasados les enseñaron.
\footnote{\textbf{9:14} Jer 23,15} \bibleverse{15} Así que esto es lo
que dice el Señor Todopoderoso, el Dios de Israel: ¡Cuidado! Voy a dar a
esta gente ajenjo para comer y agua envenenada para beber. \footnote{\textbf{9:15}
  Lev 26,33} \bibleverse{16} Estoy a punto de dispersarlos entre
naciones desconocidas para ellos y para sus antepasados, y enviaré
enemigos con espadas para que los persigan hasta que los haya
aniquilado.

\hypertarget{lamentaciones-contra-el-pueblo}{%
\subsection{Lamentaciones contra el
pueblo}\label{lamentaciones-contra-el-pueblo}}

\bibleverse{17} Esto es lo que dice el Señor Todopoderoso: Estén atentos
a lo que sucede. Convoquen a las mujeres dolientes profesionales, y
pidan lo mejor de ellas. \bibleverse{18} Haz que vengan cuanto antes y
canten un canto fúnebre sobre nosotros, para que lloremos a mares, para
que nuestras lágrimas fluyan como torrentes. \bibleverse{19} El sonido
del llanto viene de Sión: ``¡Estamos completamente devastados! Estamos
totalmente avergonzados, porque hemos tenido que abandonar nuestro país,
porque nuestras casas han sido demolidas''. \bibleverse{20} Mujeres,
escuchen el mensaje del Señor, oigan lo que tiene que decir. Enséñenle a
sus hijas a llorar y a cantar cantos de tristeza. \bibleverse{21} La
muerte se ha colado por nuestras ventanas; ha entrado en nuestras
fortalezas. Ha matado a los niños que juegan en las calles y a los
jóvenes que se reúnen en las plazas.

\bibleverse{22} Díganle a todos que esto es lo que dice el Señor: Los
cadáveres quedarán donde caen como el estiércol en los campos, tirados
allí como tallos de grano recién cortado detrás del segador, sin que
nadie los recoja.

\hypertarget{la-verdadera-auto-fama-circuncisiuxf3n-derecha}{%
\subsection{La verdadera auto-fama; circuncisión
derecha}\label{la-verdadera-auto-fama-circuncisiuxf3n-derecha}}

\bibleverse{23} Esto es lo que dice el Señor: El sabio no debe jactarse
de su sabiduría. El fuerte no debe presumir de su fuerza. El rico no
debe presumir de sus riquezas. \footnote{\textbf{9:23} 1Cor 1,31; 2Cor
  10,17} \bibleverse{24} El que quiera vanagloriarse, que se jacte de
que me conoce y me entiende de verdad, reconociendo que soy el Señor que
actúa con amor fiel, que muestra equidad y que hace lo correcto en toda
la tierra, porque esto es lo más importante para mí, declara el Señor.

\hypertarget{israel-es-incircunciso-de-corazuxf3n}{%
\subsection{Israel es incircunciso de
corazón}\label{israel-es-incircunciso-de-corazuxf3n}}

\bibleverse{25} Cuidado, porque se acerca el momento, declara el Señor,
en que castigaré a todos los que sólo se circuncidan físicamente.
\bibleverse{26} Egipto, Judá, Edom, Amón, Moab y todos los pueblos del
desierto que se cortan el pelo a los lados de la cabeza: todas estas
naciones son incircuncisas, y todos los israelitas son incircuncisos
espirituales.

\hypertarget{la-nada-de-los-uxeddolos-y-la-majestad-del-uxfanico-dios-verdadero}{%
\subsection{La nada de los ídolos y la majestad del único Dios
verdadero}\label{la-nada-de-los-uxeddolos-y-la-majestad-del-uxfanico-dios-verdadero}}

\hypertarget{section-9}{%
\section{10}\label{section-9}}

\bibleverse{1} Escuchen el mensaje que el Señor les envía, pueblo de
Israel. \bibleverse{2} Esto es lo que dice el Señor: No adopten las
prácticas de otras naciones. No se asusten como ellos por las señales en
los cielos que interpretan como una predicción de desastre.
\bibleverse{3} Las creencias religiosas de los pueblos no tienen
sentido. Cortan un árbol en el bosque y un artesano talla la madera con
una herramienta para hacer un ídolo. \footnote{\textbf{10:3} Is 44,10-20}
\bibleverse{4} Lo decoran con plata y oro, y lo clavan con un martillo
para que no se caiga. \bibleverse{5} Al igual que un espantapájaros en
un campo de pepinos, sus ídolos no pueden hablar. Hay que llevarlos en
brazos porque no pueden caminar. No hay que tenerles miedo porque no
pueden hacerte daño y tampoco pueden hacerte ningún bien.

\hypertarget{la-majestad-de-dios-sobre-el-desprecio-de-los-uxeddolos}{%
\subsection{La majestad de Dios sobre el desprecio de los
ídolos}\label{la-majestad-de-dios-sobre-el-desprecio-de-los-uxeddolos}}

\bibleverse{6} ¡No hay nadie como tú, Señor! ¡Eres tan grande! ¡Eres
increíblemente poderoso! \bibleverse{7} Todo el mundo debería
respetarte, Rey de las naciones. Así es como deben tratarte. No hay
nadie como tú entre todos los sabios de todas las naciones y reinos.
\bibleverse{8} ¡Sin embargo, estos ``sabios'' son completamente tontos y
estúpidos, porque piensan que pueden ser enseñados por inútiles ídolos
hechos de madera! \bibleverse{9} Desde Tarsis se envían láminas de plata
martillada, y oro de Ufaz, para que lo utilicen los artesanos y los
metalistas. Estos ídolos se visten con ropas de azul y púrpura hechas
por expertos. \bibleverse{10} Pero el Señor es el único Dios verdadero.
Él es el Dios vivo y el Rey eterno. La tierra tiembla cuando él se
enoja; las naciones no pueden resistir su furia.

\bibleverse{11} Esto es lo que deben decir a las naciones: ``Estos
dioses, que no hicieron ni los cielos ni la tierra, serán borrados de
esta tierra y de debajo de estos cielos''.\footnote{\textbf{10:11} Este
  verso dirigido a las naciones extranjeras está escrito en arameo, la
  lengua común de la época.} \bibleverse{12} Fue Dios quien hizo la
tierra con su poder. Él creó el mundo con su sabiduría y con su
entendimiento puso los cielos en su lugar. \bibleverse{13} Las aguas de
los cielos llueven con estruendo por orden suya. Él hace que las nubes
se eleven por toda la tierra. Hace que el rayo acompañe a la lluvia, y
envía el viento desde sus almacenes. \footnote{\textbf{10:13} Sal 135,7;
  Job 38,24-30} \bibleverse{14} Todos son estúpidos; no saben nada.
Todos los trabajadores del metal se avergüenzan de los ídolos que
fabrican. Porque sus imágenes hechas de metal fundido son fraudulentas:
¡no están vivas! \bibleverse{15} Son inútiles, un objeto de risa. Serán
destruidos en el momento de su castigo. \bibleverse{16} El Dios de Jacob
no es como esos ídolos, porque él es el Creador de todo, e Israel es la
tribu que le pertenece. El Señor Todopoderoso es su nombre.

\hypertarget{la-necesidad-de-las-personas-destinadas-al-exilio-su-queja-sobre-el-juicio-sentenciado-y-su-humilde-sumisiuxf3n-a-la-voluntad-de-dios}{%
\subsection{La necesidad de las personas destinadas al exilio; su queja
sobre el juicio sentenciado y su humilde sumisión a la voluntad de
Dios}\label{la-necesidad-de-las-personas-destinadas-al-exilio-su-queja-sobre-el-juicio-sentenciado-y-su-humilde-sumisiuxf3n-a-la-voluntad-de-dios}}

\bibleverse{17} Ustedes habitantes de Jerusalén\footnote{\textbf{10:17}
  ``Habitantes de Jerusalén'': añadido para mayor claridad.} bajo
asedio, reúnan todas sus cosas y prepárense para salir, \bibleverse{18}
porque esto es lo que dice el Señor: ¡Mira! Ahora mismo estoy a punto de
echar a la gente que vive en este país, trayendo problemas que realmente
sentirán.\footnote{\textbf{10:18} El significado del hebreo de esta
  última cláusula es impreciso.}

\bibleverse{19} El pueblo de Jerusalén respondió,\footnote{\textbf{10:19}
  ``El pueblo de Jerusalén respondió'': añadido para mayor claridad.}
``Estamos sufriendo mucho porque nos hemos hecho mucho daño, nuestras
lesiones son realmente graves. Pensábamos que no sería tan grave y que
podríamos soportarlo. \footnote{\textbf{10:19} Sal 77,11}
\bibleverse{20} Nuestras tiendas\footnote{\textbf{10:20} ``Tiendas'' son
  símbolo de las casas de Jerusalén.} han sido destruidos; todas
nuestras cuerdas se han roto. Nos han quitado a nuestros hijos y ya no
están. No nos queda nadie para armar nuestras tiendas o colgar nuestras
cortinas''. \bibleverse{21} Los ``pastores''\footnote{\textbf{10:21}
  ``Pastores'': refiriéndose a los líderes de la nación.} se han vuelto
estúpidos: no le piden consejo al Señor. Por eso han fracasado, y todo
su rebaño se ha dispersado.

\bibleverse{22} Escuchen la noticia de que un ejército ruidoso está
invadiendo desde un país del norte. Las ciudades de Judá serán
derribadas, serán lugares donde sólo viven chacales.

\hypertarget{oraciuxf3n-del-pueblo-por-la-gracia-de-dios-y-por-el-castigo-de-los-paganos-rudos}{%
\subsection{Oración del pueblo por la gracia de Dios y por el castigo de
los paganos
rudos}\label{oraciuxf3n-del-pueblo-por-la-gracia-de-dios-y-por-el-castigo-de-los-paganos-rudos}}

\bibleverse{23} Me doy cuenta\footnote{\textbf{10:23} Aquí habla
  Jeremías.} , Señor, de que la gente no controla su propia vida; nadie
sabe elegir su camino. \bibleverse{24} Por favor, disciplíname con
justicia, Señor, pero no mientras estés enojado, pues de lo contrario me
matarás. \footnote{\textbf{10:24} Jer 46,28; Sal 6,2; Hab 1,12}
\bibleverse{25} Derrama tu furia sobre las naciones que no te reconocen
como Dios, y sobre sus familias que no te adoran. Porque han destruido
completamente a los israelitas, aniquilándonos. Han devastado nuestro
país.\footnote{\textbf{10:25} Sal 79,6}

\hypertarget{el-vuxednculo-entre-dios-y-el-pueblo-se-ha-roto}{%
\subsection{El vínculo entre Dios y el pueblo se ha
roto}\label{el-vuxednculo-entre-dios-y-el-pueblo-se-ha-roto}}

\hypertarget{section-10}{%
\section{11}\label{section-10}}

\bibleverse{1} Este es el mensaje del Señor que llegó a Jeremías:
\bibleverse{2} Escucha los términos de este acuerdo, y luego repítelos
al pueblo de Judá y de Jerusalén. \bibleverse{3} Diles que esto es lo
que dice el Señor, el Dios de Israel: Ustedes están malditos si no
obedecen los términos de este acuerdo. \footnote{\textbf{11:3} Deut
  27,26} \bibleverse{4} Yo hice este acuerdo con sus antepasados cuando
los saqué de Egipto, del horno de hierro, diciendo: ``Obedézcanme y
hagan todo lo que les ordeno, y serán mi pueblo y yo seré su Dios''.
\bibleverse{5} Lo hice para cumplir lo que prometí a sus antepasados:
darles una tierra que mana leche y miel, como sigue siendo hoy. Amén,
Señor, respondí.

\bibleverse{6} Entonces el Señor me dijo: Ve y anuncia públicamente todo
este mensaje en las ciudades de Judá y en las calles de Jerusalén,
diciendo Presta atención a los términos de este acuerdo y haz lo que
dicen. \bibleverse{7} Desde que saqué a sus antepasados de Egipto hasta
ahora, les advertí seriamente una y otra vez, diciendo: ``¡Hagan lo que
les digo!'' \bibleverse{8} Pero se negaron a obedecer, no quisieron
escuchar. En cambio, cada uno de ellos siguió su propio pensamiento
obstinado y malvado. Les había ordenado que siguieran el acuerdo, pero
no lo hicieron. Así que hice caer sobre ellos todas las maldiciones
contenidas en el acuerdo. \footnote{\textbf{11:8} Jer 7,24; Jer 7,26}

\hypertarget{el-incumplimiento-del-pacto-y-el-rechazo-del-pueblo}{%
\subsection{El incumplimiento del pacto y el rechazo del
pueblo}\label{el-incumplimiento-del-pacto-y-el-rechazo-del-pueblo}}

\bibleverse{9} El Señor me dijo: Está ocurriendo una rebelión entre el
pueblo de Judá y los que viven en Jerusalén. \bibleverse{10} Han vuelto
a los pecados de sus antepasados, que se negaron a obedecer lo que yo
decía. Han ido a adorar a otros dioses. El pueblo de Israel y de Judá ha
roto el acuerdo que hice con sus antepasados. \bibleverse{11} Así que
esto es lo que dice el Señor: Voy a traer sobre ellos un desastre del
que no podrán escapar. Me pedirán ayuda a gritos, pero no los escucharé.
\bibleverse{12} Entonces los habitantes de las ciudades de Judá y
Jerusalén irán a pedir ayuda a los dioses a los que han estado quemando
incienso, pero estos dioses no podrán hacer nada para salvarlos en su
momento de angustia. \footnote{\textbf{11:12} Jer 2,28; Deut 32,37-38}
\bibleverse{13} ¡Ciertamente tienes tantos dioses como ciudades, Judá!
Has construido altares vergonzosos, altares para quemar incienso a Baal.
Tienes tantos altares como las calles de Jerusalén.

\bibleverse{14} Jeremías, no ores por este pueblo. No clames por ayuda
ni ofrezcas una oración en su favor, porque no los escucharé cuando
clamen a mí en el momento de su angustia.

\hypertarget{los-sacrificios-y-las-ofrendas-elegidas-por-uno-mismo-no-evitan-la-cauxedda-del-pueblo-que-alguna-vez-fue-amado-por-dios}{%
\subsection{Los sacrificios y las ofrendas elegidas por uno mismo no
evitan la caída del pueblo que alguna vez fue amado por
Dios}\label{los-sacrificios-y-las-ofrendas-elegidas-por-uno-mismo-no-evitan-la-cauxedda-del-pueblo-que-alguna-vez-fue-amado-por-dios}}

\bibleverse{15} ¿Qué derecho tienen las personas que amo a estar en mi
Templo cuando han hecho tantas cosas malas? ¿Creen que la carne de los
sacrificios los salvará? Cuando ocurra el desastre, ¿te
alegrarás?\footnote{\textbf{11:15} El significado de este verso es
  objeto de diferentes interpretaciones.} \bibleverse{16} En un tiempo
el Señor dijo que eras un olivo sano, lleno de hojas y que daba hermosos
frutos. Pero con gran ruido le prenderá fuego, destruyendo sus ramas.
\bibleverse{17} Yo, el Señor Todopoderoso, fui quien te plantó, pero he
anunciado que serás destruido a causa de la maldad que ha cometido el
pueblo de Israel y de Judá, enojándome al quemar incienso a Baal.

\hypertarget{el-intento-de-asesinato-de-los-habitantes-de-anathoth-contra-jeremuxedas-y-su-castigo}{%
\subsection{El intento de asesinato de los habitantes de Anathoth contra
Jeremías y su
castigo}\label{el-intento-de-asesinato-de-los-habitantes-de-anathoth-contra-jeremuxedas-y-su-castigo}}

\bibleverse{18} El Señor me lo ha comunicado,\footnote{\textbf{11:18}
  Presumiblemente refiriéndose a la rebelión mencionada en el verso 9.}
para que lo sepa. Luego me mostró lo que realmente estaban haciendo.
\bibleverse{19} Yo era como una ovejita confiada a la que llevan al
matadero. No sabía que habían conspirado contra mí. Dijeron:
``Destruyamos el árbol junto con todo lo que produce. Matémoslo para que
nadie recuerde su nombre''. \footnote{\textbf{11:19} Is 53,7}
\bibleverse{20} Apelo a ti, Señor Todopoderoso, tú que juzgas con
justicia y examinas los pensamientos y sentimientos de la gente, déjame
ver cómo los castigas, porque he dejado mi caso en tus manos.
\footnote{\textbf{11:20} Sal 7,10}

\bibleverse{21} Esto es lo que dice el Señor acerca de la gente de
Anatot que trata de matarte, diciéndote: ``No profetices en nombre del
Señor, o te mataremos''. \footnote{\textbf{11:21} Jer 1,1}

\bibleverse{22} Esta es la respuesta del Señor Todopoderoso: Los
castigaré. Sus jóvenes morirán a espada, sus hijos e hijas morirán de
hambre. \bibleverse{23} No quedará nadie porque traeré el desastre sobre
el pueblo de Anatot en el momento en que sea castigado.

\hypertarget{la-pregunta-de-jeremuxedas-a-dios-sobre-la-felicidad-de-los-malvados}{%
\subsection{La pregunta de Jeremías a Dios sobre la felicidad de los
malvados}\label{la-pregunta-de-jeremuxedas-a-dios-sobre-la-felicidad-de-los-malvados}}

\hypertarget{section-11}{%
\section{12}\label{section-11}}

\bibleverse{1} Señor, cuando me quejo ante ti, siempre demuestras tener
la razón. Aun así, quiero presentarte mi caso. ¿Por qué les va tan bien
a los malvados? ¿Por qué viven tan cómodamente los que te son infieles?
\bibleverse{2} Tú los plantaste, y han echado raíces, han crecido y han
dado fruto. Siempre hablan de ti, pero no piensan en ti, ni siquiera por
un momento. \bibleverse{3} Pero tú me conoces, Señor, me ves, y examinas
lo que pienso de ti. Arrastra a esta gente como si fueran ovejas para
ser sacrificadas; apártalas para el momento de su muerte. \bibleverse{4}
¿Hasta cuándo tendrá que lamentarse la tierra y secarse la hierba de
todos los campos a causa de la maldad de la gente que la habita? Los
animales y las aves se han extinguido porque la gente ha dicho:
``El\footnote{\textbf{12:4} ``Él'' se puede aplicar al Señor o a
  Jeremías, ya sea refutando la presciencia de Dios o las afirmaciones
  proféticas de Jeremías.} no sabe lo que nos va a pasar''. \footnote{\textbf{12:4}
  Jer 9,9}

\hypertarget{la-respuesta-divina}{%
\subsection{La respuesta divina}\label{la-respuesta-divina}}

\bibleverse{5} El Señor dice,\footnote{\textbf{12:5} ``El Señor dice'':
  añadido para mayor claridad.} Si te desgastas en una carrera a pie
contra los hombres, ¿cómo ganarías una carrera contra los caballos? Si
tropiezas en terreno abierto, ¿cómo lo harías en la enmarañada maleza
junto al Jordán? \bibleverse{6} Incluso tus propios hermanos y la
familia de tu padre te han traicionado; te han criticado públicamente.
No te fíes de ellos cuando te hablen bien.

\hypertarget{lamento-de-dios-por-su-tierra-devastada-por-los-pueblos-vecinos}{%
\subsection{Lamento de Dios por su tierra devastada por los pueblos
vecinos}\label{lamento-de-dios-por-su-tierra-devastada-por-los-pueblos-vecinos}}

\bibleverse{7} He renunciado a mi pueblo; he abandonado la nación que
elegí. He entregado a sus enemigos a los que verdaderamente amo.
\bibleverse{8} Se han convertido en un león salvaje que ruge contra mí;
por eso los odio. \bibleverse{9} Mi pueblo es como un ave de rapiña
manchada\footnote{\textbf{12:9} ``Ave de rapiña manchada'': o ``hiena''.}
a mí con otras aves de rapiña dando vueltas para atacarlo. Ve y trae a
todos los animales salvajes para que se coman el cadáver.
\bibleverse{10} Muchos pastores\footnote{\textbf{12:10} ``Pastores'': se
  refiere a los líderes de los ejércitos invasores.} han venido y han
destruido mi viña; han pisoteado las cosechas de mi campo. Han
convertido mi tierra agradable en un páramo vacío. \bibleverse{11} La
han convertido en un desierto; está de luto ante mí, desolada. Todo el
país es un páramo, pero a nadie le importa. \bibleverse{12} Los
ejércitos destructores han atravesado todas las colinas desnudas del
desierto, porque la espada del Señor destruye de un extremo a otro del
país. Nadie tiene paz. \bibleverse{13} Mi pueblo sembró trigo pero
cosechó espinas. Se desgastaron, pero no obtuvieron ningún beneficio.
Deberían avergonzarse de una cosecha tan pobre, causada por la furia del
Señor.

\hypertarget{anuncio-de-juicio-y-salvaciuxf3n-para-los-pueblos-paganos-vecinos}{%
\subsection{Anuncio de juicio y salvación para los pueblos paganos
vecinos}\label{anuncio-de-juicio-y-salvaciuxf3n-para-los-pueblos-paganos-vecinos}}

\bibleverse{14} Esto es lo que dice el Señor: Cuando vengan esas
naciones malvadas cercanas que atacan el país que le di a mi pueblo
Israel, voy a desarraigarlos de su tierra. También voy a desarraigar al
pueblo de Judá de entre ellos. \bibleverse{15} Sin embargo, una vez que
los haya desarraigado, volveré a tener misericordia de ellos y haré que
cada uno vuelva a su propiedad y a su tierra. \bibleverse{16} Si
aprenden honestamente los caminos de mi pueblo y me respetan, haciendo
sus votos por mí, tal como una vez enseñaron a mi pueblo a jurar por
Baal, entonces les irá bien entre mi pueblo. \footnote{\textbf{12:16}
  Jer 4,2; Deut 6,13}

\bibleverse{17} Pero si se niegan a obedecer, entonces no sólo
desarraigaré a esa nación, sino que la destruiré por completo, declara
el Señor.

\hypertarget{paruxe1bola-del-cinturuxf3n-de-lino}{%
\subsection{Parábola del cinturón de
lino}\label{paruxe1bola-del-cinturuxf3n-de-lino}}

\hypertarget{section-12}{%
\section{13}\label{section-12}}

\bibleverse{1} Esto es lo que el Señor me dijo que hiciera: Ve y
cómprate un taparrabos de lino y póntelo, pero no lo laves.

\bibleverse{2} Así que fui y compré un taparrabos como el Señor me había
indicado, y me lo puse.

\bibleverse{3} Entonces el Señor me dio otro mensaje: \bibleverse{4}
Toma el taparrabos que compraste y póntelo, y ve inmediatamente al río
Perat\footnote{\textbf{13:4} El río Parath en hebreo suele traducirse
  como el Éufrates. Sin embargo, esto significaría que Jeremías haría
  dos viajes de ida y vuelta de unas 700 millas cada uno. Algunos han
  sugerido que el río en cuestión era uno con un nombre similar situado
  cerca de Anathoth. En la medida en que se trata de una parábola
  actuada, parece probable que otros debían ver lo que sucedía y
  entender lo que significaba, el Éufrates real parece una ubicación
  poco probable. Sin embargo, el aspecto simbólico debe incluirse ya que
  los invasores vendrían de Babilonia por el Éufrates.} y escóndela allí
en un agujero entre las rocas.

\bibleverse{5} Fui, pues, y lo escondí junto al río Perat, como me había
dicho el Señor.

\bibleverse{6} Mucho tiempo después, el Señor me dijo: Ve a Perat y trae
el taparrabos que te ordené esconder allí.

\bibleverse{7} Fui a Perat, desenterré el taparrabos y lo saqué de donde
lo había escondido. Obviamente, estaba arruinado, completamente
inservible.

\hypertarget{la-interpretaciuxf3n-de-la-paruxe1bola}{%
\subsection{La interpretación de la
parábola}\label{la-interpretaciuxf3n-de-la-paruxe1bola}}

\bibleverse{8} Entonces me llegó un mensaje del Señor: \bibleverse{9}
Esto es lo que dice el Señor: Voy a arruinar la arrogancia de Judá y la
gran arrogancia de Jerusalén exactamente de la misma manera.
\bibleverse{10} Esta gente malvada se niega a escuchar lo que les digo.
Siguen su propio pensamiento obstinado y malvado y corren a adorar a
otros dioses; serán como este taparrabos, completamente inútil.
\bibleverse{11} Así como el taparrabos se adhiere al cuerpo, así hice
que todo el pueblo de Israel y de Judá se adhiriera a mí, declara el
Señor. Así podrían haber sido mi pueblo, representándome, dándome honor
y alabanza. Pero se negaron a escuchar.

\hypertarget{la-paruxe1bola-de-las-jarras-de-vino-llenas-y-rotas-llamada-de-advertencia}{%
\subsection{La parábola de las jarras de vino llenas y rotas; Llamada de
advertencia}\label{la-paruxe1bola-de-las-jarras-de-vino-llenas-y-rotas-llamada-de-advertencia}}

\bibleverse{12} Así que diles que esto es lo que dice el Señor, el Dios
de Israel: Toda jarra de vino se llenará de vino. Cuando respondan:
``¿No lo sabemos ya? Claro que toda jarra de vino debe llenarse de
vino!'' \bibleverse{13} entonces diles que esto es lo que dice el Señor:
Voy a emborrachar a todos los que viven en esta tierra: a los reyes que
se sientan en el trono de David, a los sacerdotes, a los profetas y a
todo el pueblo de Jerusalén. \footnote{\textbf{13:13} Jer 25,15-18; Is
  51,17} \bibleverse{14} Voy a aplastarlos unos contra otros como si
fueran tinajas de vino,\footnote{\textbf{13:14} ``Como las tinajas de
  vino'': añadido para mayor claridad.} tanto a los padres como a los
hijos, declara el Señor. No dejaré que ninguna misericordia, piedad o
compasión me impida destruirlos.

\hypertarget{advertencia-de-autoconfianza}{%
\subsection{Advertencia de
autoconfianza}\label{advertencia-de-autoconfianza}}

\bibleverse{15} Escuchen y presten atención. No seas arrogante, porque
el Señor ha hablado. \bibleverse{16} Honra al Señor, tu Dios, antes de
que traiga la oscuridad, antes de que tropieces y caigas en el
crepúsculo de las montañas. Tú anhelas que llegue la luz, pero él sólo
envía tinieblas y oscuridad total. \bibleverse{17} Pero si te niegas a
escuchar, lloraré secretamente por dentro a causa de tu orgullo. Mis
lágrimas se derraman porque el rebaño del Señor ha sido capturado.

\hypertarget{discurso-amenazador-para-el-rey-y-la-reina-madre}{%
\subsection{Discurso amenazador para el rey y la reina
madre}\label{discurso-amenazador-para-el-rey-y-la-reina-madre}}

\bibleverse{18} Dile al rey y a la reina madre: Bajen de sus tronos,
porque sus espléndidas coronas han caído de sus cabezas. \footnote{\textbf{13:18}
  Lam 5,16} \bibleverse{19} Las ciudades del Néguev están rodeadas;
nadie puede pasar por ellas. Todo Judá ha sido llevado al exilio, todos
han sido desterrados.

\hypertarget{lamento-y-ay-de-jerusaluxe9n}{%
\subsection{Lamento y ay de
Jerusalén}\label{lamento-y-ay-de-jerusaluxe9n}}

\bibleverse{20} Miren hacia arriba y verán a los invasores que vienen
del norte. ¿Dónde está el rebaño que se te dio para que lo cuidaras?
¿Dónde están las ovejas de las que estabas tan orgulloso?
\bibleverse{21} ¿Qué vas a decir cuando ponga a tus enemigos a cargo de
ti, gente que antes considerabas tus amigos? ¿No sufrirás dolores como
una mujer de parto? \bibleverse{22} Si te dices a ti mismo: ¿Por qué me
ha pasado esto? es porque has sido muy malvado. Por eso te han quitado
las faldas y te han violado. \bibleverse{23} ¿Pueden los etíopes cambiar
el color de su piel? ¿Puede un leopardo cambiar sus manchas? De la misma
manera tú no puedes cambiar y hacer el bien porque estás muy
acostumbrada a hacer el mal. \footnote{\textbf{13:23} Sal 55,20}

\bibleverse{24} Voy a dispersarte como el tamo que se lleva el viento
del desierto. \bibleverse{25} Esto es lo que te va a pasar; esto es lo
que he decidido hacer contigo, declara el Señor, porque te has olvidado
de mí y has creído en la mentira. \bibleverse{26} Te subiré las faldas
sobre la cara, para que te vean desnuda y avergonzada. \bibleverse{27}
He visto sus actos de adulterio y lujuria, cómo se prostituyeron
descaradamente, adorando a los ídolos en las colinas y en los campos.
Sí, vi las cosas repugnantes que hicisteis. El desastre viene hacia ti,
Jerusalén. ¿Cuánto tiempo vas a seguir siendo impura?

\hypertarget{descripciuxf3n-de-la-gran-sequuxeda}{%
\subsection{Descripción de la gran
sequía}\label{descripciuxf3n-de-la-gran-sequuxeda}}

\hypertarget{section-13}{%
\section{14}\label{section-13}}

\bibleverse{1} Este es un mensaje del Señor que llegó a Jeremías en
relación con la sequía: \bibleverse{2} Judá está de luto; sus ciudades
se están consumiendo. Su pueblo llora por la tierra, y de Jerusalén
llega un grito de auxilio. \bibleverse{3} Los ricos envían a sus siervos
a buscar agua. Van a las cisternas, pero no encuentran agua. Regresan
con las tinajas vacías, decepcionados y avergonzados, cubriendo sus
cabezas. \bibleverse{4} La tierra se ha secado porque no ha llovido en
el país. Los campesinos se avergüenzan y se cubren la cabeza.
\footnote{\textbf{14:4} Jl 1,11} \bibleverse{5} Hasta la cierva abandona
a su cervatillo recién nacido porque no hay hierba. \bibleverse{6} Los
asnos salvajes se paran en las colinas desnudas, jadeando como chacales.
Les falla la vista porque no tienen nada que comer.

\hypertarget{solicitud-de-la-gente}{%
\subsection{Solicitud de la gente}\label{solicitud-de-la-gente}}

\bibleverse{7} Aunque nuestros pecados nos delatan, Señor, por favor,
haz algo por nosotros gracias a tu bondad. Sí, nos hemos rebelado contra
ti muchas veces; hemos pecado contra ti. \bibleverse{8} Tú eres la
esperanza de Israel, nuestro Salvador en tiempos de angustia. ¿Por qué
actúas como un extranjero en nuestro país, como un viajero que sólo se
queda una noche? \bibleverse{9} ¿Por qué te comportas como alguien
sorprendido, como un guerrero poderoso que no puede ayudar? Tú estás
aquí entre nosotros, Señor, y nosotros somos conocidos como tu pueblo.
¡Por favor, no nos abandones! \footnote{\textbf{14:9} Jer 15,16; Is 43,7}

\hypertarget{dios-rechaza-la-intercesiuxf3n-del-profeta-y-amenaza-a-los-falsos-profetas-y-a-todo-el-pueblo-con-dificultades-auxfan-mayores}{%
\subsection{Dios rechaza la intercesión del profeta y amenaza a los
falsos profetas y a todo el pueblo con dificultades aún
mayores}\label{dios-rechaza-la-intercesiuxf3n-del-profeta-y-amenaza-a-los-falsos-profetas-y-a-todo-el-pueblo-con-dificultades-auxfan-mayores}}

\bibleverse{10} Esto es lo que el Señor dice de su pueblo: Les encanta
alejarse de mí; ni siquiera intentan evitarlo. Por eso el Señor se niega
a aceptarlos. Ahora se acordará de sus acciones culpables y los
castigará por sus pecados.

\bibleverse{11} El Señor me dijo: No reces por el bienestar de este
pueblo. \bibleverse{12} Aunque ayunen, no escucharé su clamor. Aunque
ofrezcan holocaustos y ofrendas de grano, no los aceptaré. Por el
contrario, los exterminaré con la espada, el hambre y la peste.
\footnote{\textbf{14:12} Is 58,3; Jer 6,20}

\bibleverse{13} ``¡Oh, Señor Dios!'' Respondí: ``Mira lo que les dicen
los profetas, que dicen hablar en tu nombre:\footnote{\textbf{14:13}
  ``Que dicen hablar en tu nombre'': suministrado para mayor claridad.}
`No verán la guerra ni sufrirán el hambre, sino que les daré una paz
duradera en este lugar'\,''.

\bibleverse{14} Los profetas están profetizando mentiras en mi nombre,
respondió el Señor. Yo no los envié, ni los elegí, ni les hablé. Es una
visión mentirosa, una predicción vacía, un producto engañoso de sus
propias mentes lo que te están profetizando. \bibleverse{15} Así que
esto es lo que dice el Señor sobre esos profetas que profetizan en mi
nombre: Yo no los envié, pero aun así dicen: ``Este país no sufrirá
guerra ni hambre''. Esos mismos profetas morirán de guerra o de hambre.
\footnote{\textbf{14:15} Deut 18,20} \bibleverse{16} Los cadáveres de la
gente a la que profetizaron serán arrojados a las calles de Jerusalén a
causa del hambre y la guerra. No habrá nadie que los entierre, ni a sus
esposas, ni a sus hijos, ni a sus hijas. Derramaré sobre ellos su propio
mal. \footnote{\textbf{14:16} Jer 8,2}

\hypertarget{jeremuxedas-llora-por-la-gran-angustia-de-juduxe1}{%
\subsection{Jeremías llora por la gran angustia de
Judá}\label{jeremuxedas-llora-por-la-gran-angustia-de-juduxe1}}

\bibleverse{17} Esto es lo que debes decirles: Las lágrimas brotan de
mis ojos sin cesar, de día y de noche, porque mi pueblo ha sido
aplastado por un duro golpe, una herida realmente grave. \footnote{\textbf{14:17}
  Jer 8,23} \bibleverse{18} Si salgo al campo, veo a los muertos por la
espada; si voy a la ciudad, veo a los muertos por el hambre. Tanto los
profetas como los sacerdotes vagan por el campo; no saben lo que hacen.

\hypertarget{queja-renovada-y-suxfaplica-urgente-del-profeta}{%
\subsection{Queja renovada y súplica urgente del
Profeta}\label{queja-renovada-y-suxfaplica-urgente-del-profeta}}

\bibleverse{19} ¿Realmente has rechazado a Judá? ¿Odias tanto a Sión?
¿Por qué nos has herido tanto que no podemos curarnos? Esperábamos la
paz, pero en lugar de ello no ha llegado nada bueno; esperábamos un
tiempo de curación, pero en lugar de ello sólo ha habido terror
repentino. \bibleverse{20} Señor, reconocemos nuestra maldad, la culpa
de nuestros antepasados y nuestros propios pecados contra ti.
\bibleverse{21} Por tu propia reputación, por favor no nos odies; no
traigas deshonra a tu glorioso trono. Por favor, recuerda tu acuerdo con
nosotros; no lo rompas. \bibleverse{22} ¿Pueden los falsos dioses de las
otras naciones hacer llover? ¿Pueden los cielos mismos enviar lluvias?
No, eres tú, Señor, nuestro Dios. Por eso ponemos nuestra esperanza en
ti, porque sólo tú puedes hacer todo esto.

\hypertarget{otro-rechazo-a-una-intercesiuxf3n-de-jeremuxedas-y-una-nueva-amenaza-de-dios}{%
\subsection{Otro rechazo a una intercesión de Jeremías y una nueva
amenaza de
Dios}\label{otro-rechazo-a-una-intercesiuxf3n-de-jeremuxedas-y-una-nueva-amenaza-de-dios}}

\hypertarget{section-14}{%
\section{15}\label{section-14}}

\bibleverse{1} El Señor me dijo: Aunque Moisés y Samuel estuvieran
delante de mí suplicándome en nombre de este pueblo, no me darían
lástima. Envíalos lejos de mí. Haz que se vayan. \footnote{\textbf{15:1}
  Sal 99,6; Ezeq 14,14} \bibleverse{2} Si te preguntan: ``¿Adónde
iremos?'' , diles que esto es lo que dice el Señor: Los que vayan a
morir por la peste, a la peste; los que vayan a morir por la espada, a
la espada; los que vayan a morir de hambre, al hambre; y los que vayan a
morir en el cautiverio, al cautiverio. \footnote{\textbf{15:2} Jer
  43,11; Zac 11,9}

\bibleverse{3} Pondré a cargo de ellos cuatro clases de destructores,
declara el Señor: espadas para matar, perros para arrastrar sus cuerpos,
y aves de rapiña y animales salvajes para que los devoren y los
destruyan. \footnote{\textbf{15:3} Ezeq 14,21} \bibleverse{4} Haré que
todos los reinos del mundo se horroricen de ellos, a causa de las
maldades que Manasés, hijo de Ezequías, rey de Judá, hizo en Jerusalén.
\footnote{\textbf{15:4} 2Re 21,11-16; 2Re 23,26}

\hypertarget{el-lamento-de-jeremuxedas-por-los-severos-sufrimientos-de-la-guerra-que-han-afligido-a-jerusaluxe9n-y-auxfan-no-han-sucedido}{%
\subsection{El lamento de Jeremías por los severos sufrimientos de la
guerra que han afligido a Jerusalén y aún no han
sucedido}\label{el-lamento-de-jeremuxedas-por-los-severos-sufrimientos-de-la-guerra-que-han-afligido-a-jerusaluxe9n-y-auxfan-no-han-sucedido}}

\bibleverse{5} ¿Quién se lamentará por ti, Jerusalén? ¿Quién se
lamentará por ti? ¿Quién se detendrá a preguntarte cómo estás?
\bibleverse{6} Me has abandonado, declara el Señor. Me has dado la
espalda. Por eso actuaré contra ti y te destruiré; me he cansado de
mostrarte misericordia. \bibleverse{7} Te dispersaré con una hoz de
segar\footnote{\textbf{15:7} ``Horquilla de aventar'': herramienta
  utilizada para separar la paja del grano.} de todos los pueblos del
país. Destruiré a mi pueblo y me llevaré a sus hijos porque se niegan a
abandonar sus malos caminos. \footnote{\textbf{15:7} Mat 3,12}
\bibleverse{8} Habrá más viudas que la arena del mar. Traeré un
destructor al mediodía y las madres perderán a sus hijos pequeños. De
repente experimentarán agonía y conmoción. \bibleverse{9} Una madre de
siete hijos se derrumbará; jadeará para respirar. Su sol se pondrá
cuando aún sea de día; se sentirá avergonzada y humillada. Dejaré que
los enemigos maten al resto de ellos, declara el Señor.

\hypertarget{jeremuxedas-agotado-en-sus-fuerzas-y-loco-por-dios}{%
\subsection{Jeremías agotado en sus fuerzas y loco por
Dios}\label{jeremuxedas-agotado-en-sus-fuerzas-y-loco-por-dios}}

\bibleverse{10} ¡Qué triste estoy, madre mía, por el hecho de que me
hayas dado a luz! Soy víctima de discusiones y conflictos por donde
quiera que voy en el país. Nunca le he prestado nada a nadie, ni he
pedido nada prestado, pero aun así todos me maldicen. \footnote{\textbf{15:10}
  Jer 20,14}

\bibleverse{11} Pero el Señor me dijo: No te preocupes, voy a quitarte
los problemas para que puedas hacer el bien. Haré que tus enemigos te
supliquen cada vez que tengan problemas o sufran.\footnote{\textbf{15:11}
  O ``Hablaré con tus enemigos por ti siempre que tengas problemas,
  siempre que sufras''. El hebreo es ambiguo.} \bibleverse{12} ¿Puede
alguien romper el hierro, el hierro del norte o el bronce?
\bibleverse{13} Regalaré sus riquezas y posesiones valiosas. Se
convertirán en botín para sus enemigos a causa de todos los pecados que
cometieron en todo su país. \bibleverse{14} Entonces haré que sus
enemigos los conviertan en sus esclavos\footnote{\textbf{15:14}
  ``Convertirlos en sus esclavos'': o ``Conducirlos a la esclavitud''.}
en un país desconocido, porque me enfadaré tanto que será como encender
un fuego que te quemará. \bibleverse{15} Tú sabes lo que me pasa, Señor.
Por favor, acuérdate de mí y cuida de mí. Castiga a mis perseguidores.
Por favor, ten paciencia, ¡no me dejes morir! Tú sabes que soporto las
críticas porque quiero honrarte. \bibleverse{16} Cuando recibí tus
mensajes, los devoré. Lo que dijiste me hizo muy feliz, me encantó. Te
pertenezco, Señor Dios Todopoderoso. \bibleverse{17} No me uní a un
grupo de burlones mientras se divertían. Me quedé solo porque me has
llamado, y me has llenado de indignación.\footnote{\textbf{15:17} ``Me
  llenó de indignación'': ante las acciones de los fiesteros burlones y
  de la nación en general.} \bibleverse{18} ¿Por qué mi dolor no cesa
nunca? ¿Por qué mi herida es incurable? ¿Por qué no se puede curar?
Realmente te has convertido en un arroyo estacional para mí, una fuente
de agua poco fiable. \footnote{\textbf{15:18} Jer 30,12}

\hypertarget{la-correcciuxf3n-y-aceptaciuxf3n-del-profeta-por-dios}{%
\subsection{La corrección y aceptación del profeta por
Dios}\label{la-correcciuxf3n-y-aceptaciuxf3n-del-profeta-por-dios}}

\bibleverse{19} Así que esto es lo que dice el Señor: Si vuelves a mí,
te aceptaré de nuevo y volverás a servirme. Si lo que hablas son
palabras que valen la pena y no tonterías, serás mi portavoz,
Jeremías.\footnote{\textbf{15:19} ``Jeremías'': añadido para mayor
  claridad.} Ellos deben ser los que te sigan; tú no debes seguirlos.
\bibleverse{20} Entonces te convertiré en un muro para esa gente, un
fuerte muro de bronce. Lucharán contra ti, pero no te
vencerán.\footnote{\textbf{15:20} Véase también 1:18-19.} Yo estoy
contigo para salvarte y rescatarte, declara el Señor. \bibleverse{21} Te
liberaré del poder de los malvados y te libraré de las garras de los
crueles.

\hypertarget{se-supone-que-jeremuxedas-no-debe-tener-familia}{%
\subsection{Se supone que Jeremías no debe tener
familia}\label{se-supone-que-jeremuxedas-no-debe-tener-familia}}

\hypertarget{section-15}{%
\section{16}\label{section-15}}

\bibleverse{1} Un mensaje del Señor que vino a mí, diciendo:
\bibleverse{2} No te cases ni tengas hijos aquí. \bibleverse{3} Esto es
lo que dice el Señor sobre los niños que nacen aquí, y sobre sus madres
y padres, es decir, sus padres aquí en este país: \bibleverse{4} Morirán
de enfermedades mortales. Nadie los llorará. Sus cuerpos no serán
enterrados, sino que yacerán en el suelo como el estiércol. Serán
destruidos por la guerra y el hambre, y sus cuerpos serán alimento para
las aves de rapiña y los animales salvajes.

\hypertarget{jeremuxedas-debe-mantenerse-alejado-de-las-ceremonias-fuxfanebres-y-las-fiestas-felices}{%
\subsection{Jeremías debe mantenerse alejado de las ceremonias fúnebres
y las fiestas
felices}\label{jeremuxedas-debe-mantenerse-alejado-de-las-ceremonias-fuxfanebres-y-las-fiestas-felices}}

\bibleverse{5} Esto es lo que dice el Señor: No entres en una casa donde
la gente esté celebrando una comida fúnebre. No los visites para llorar
ni para darles el pésame, porque les he quitado mi paz, mi amor fiel y
mi misericordia, declara el Señor. \bibleverse{6} Todos, desde el más
importante hasta el más insignificante, morirán en este país. No se les
enterrará ni se les llorará; no habrá ritos para los muertos, como el
inmolarse o afeitarse la cabeza. \bibleverse{7} No se celebrarán
recepciones fúnebres para consolar a los que lloran; ni siquiera se
ofrecerá una bebida reconfortante ante la pérdida de un padre o una
madre.

\bibleverse{8} No entres en una casa donde la gente está de fiesta ni te
sientes con ellos a comer y beber. \bibleverse{9} Esto es lo que dice el
Señor Todopoderoso, el Dios de Israel: Voy a poner fin aquí mismo,
mientras tú observas, a cualquier sonido de celebración y alegría, a las
voces alegres de los novios. \footnote{\textbf{16:9} Jer 7,34}

\hypertarget{justificaciuxf3n-de-estas-visitas-y-anuncio-de-que-la-gente-seruxe1-llevada-al-cautiverio}{%
\subsection{Justificación de estas visitas y anuncio de que la gente
será llevada al
cautiverio}\label{justificaciuxf3n-de-estas-visitas-y-anuncio-de-que-la-gente-seruxe1-llevada-al-cautiverio}}

\bibleverse{10} Cuando les expliques todo esto, te preguntarán: ``¿Por
qué ha ordenado el Señor que nos ocurra un desastre tan terrible? ¿Qué
hemos hecho mal? ¿Qué pecado hemos cometido contra el Señor, nuestro
Dios?'' \bibleverse{11} Contéstales: Es porque sus antepasados me
abandonaron, declara el Señor. Se fueron y siguieron a otros dioses,
sirviéndolos y adorándolos. Me abandonaron y no cumplieron mis leyes.
\bibleverse{12} Ustedes, sin embargo, han hecho aún más mal que sus
antepasados. Miren cómo todos ustedes siguieron su propio y obstinado
pensamiento malvado en lugar de obedecerme. \bibleverse{13} Así que voy
a expulsarlos de este país y a exiliarlos a un país desconocido para
ustedes y sus antepasados. Allí servirán a otros dioses día y noche,
porque yo no los ayudaré en nada.

\hypertarget{profecuxeda-de-salvaciuxf3n-insertada}{%
\subsection{Profecía de salvación
insertada}\label{profecuxeda-de-salvaciuxf3n-insertada}}

\bibleverse{14} ¡Pero escuchen! Se acerca el tiempo, declara el Señor,
en que la gente ya no hará votos, diciendo. ``Por la vida del Señor, que
sacó a los israelitas de Egipto''. \footnote{\textbf{16:14} Jer 23,7-8}
\bibleverse{15} En cambio, dirán: ``Por la vida del Señor, que hizo
regresar a los israelitas del país del norte y de todos los demás países
donde los había exiliado''. Los haré regresar al país que les di a sus
antepasados.

\hypertarget{los-pescadores-y-cazadores-enviados-por-dios-pronto-perseguiruxe1n-cruelmente-al-pueblo}{%
\subsection{Los pescadores y cazadores enviados por Dios pronto
perseguirán cruelmente al
pueblo}\label{los-pescadores-y-cazadores-enviados-por-dios-pronto-perseguiruxe1n-cruelmente-al-pueblo}}

\bibleverse{16} Pero por el momento voy a enviar por muchos pescadores y
ellos los pescarán, declara el Señor. Luego voy a enviar a muchos
cazadores, y los cazarán en todas las montañas y colinas, incluso desde
sus escondites en las rocas. \bibleverse{17} Yo veo todo lo que hacen.
No pueden esconderse de mí, y sus pecados tampoco están ocultos para mí.
\bibleverse{18} Primero voy a pagarles el doble por su maldad y su
pecado, porque han ensuciado mi tierra con los cuerpos sin vida de sus
repugnantes ídolos, llenando mi país especial con sus ofensivas imágenes
paganas.

\hypertarget{los-gentiles-reconocen-al-uxfanico-dios}{%
\subsection{Los gentiles reconocen al único
Dios}\label{los-gentiles-reconocen-al-uxfanico-dios}}

\bibleverse{19} Señor, tú eres mi fuerza y mi fortaleza, mi lugar seguro
en el tiempo de angustia. Vendrán a ti naciones de toda la tierra, y
dirán: ``¡La religión de nuestros antepasados era una total mentira! Los
ídolos que adoraban eran inútiles, no servían para nada. \bibleverse{20}
¿Cómo puede la gente hacerse dioses para sí misma? Estos no son
dioses!'' \bibleverse{21} ¡Ahora verán! Les mostraré, y entonces
reconocerán mi poder y mi fuerza. Entonces sabrán que yo soy el Señor!

\hypertarget{la-culpa-imperdonable-de-juduxe1-y-el-severo-castigo-de-dios}{%
\subsection{La culpa imperdonable de Judá y el severo castigo de
Dios}\label{la-culpa-imperdonable-de-juduxe1-y-el-severo-castigo-de-dios}}

\hypertarget{section-16}{%
\section{17}\label{section-16}}

\bibleverse{1} El pecado de Judá está inscrito con un punzón de hierro,
grabado con una punta de diamante,\footnote{\textbf{17:1} ``Punta de
  diamante'': piedra extremadamente dura que se utiliza de la misma
  manera que el diamante hoy en día. Los diamantes eran desconocidos en
  Israel en aquella época.} en sus mentes y en las esquinas de sus
altares donde adoran. \bibleverse{2} Incluso sus hijos se acuerdan de
adorar en sus altares paganos y en sus postes de Asera, erigidos junto a
los árboles verdes y en las colinas altas, \bibleverse{3} en mi montaña,
en los campos. Entregaré sus riquezas y todas sus posesiones valiosas
como botín, a causa del pecado cometido en sus lugares altos paganos
dentro de su país. \bibleverse{4} Tendrás que renunciar a la tierra que
te di. Haré que tus enemigos te conviertan en sus esclavos en un país
desconocido, porque has hecho arder mi ira, que arderá para siempre.

\hypertarget{confianza-falsa-en-las-personas-y-confianza-correcta-en-dios-y-sus-frutos}{%
\subsection{Confianza falsa en las personas y confianza correcta en Dios
y sus
frutos}\label{confianza-falsa-en-las-personas-y-confianza-correcta-en-dios-y-sus-frutos}}

\bibleverse{5} Esto es lo que dice el Señor: Malditos los que ponen su
confianza en las personas, los que confían en las fuerzas humanas y
dejan de confiar en el Señor. \footnote{\textbf{17:5} Sal 118,8; Sal
  146,3} \bibleverse{6} Serán como un arbusto solitario en el desierto
que ni siquiera se da cuenta cuando suceden cosas buenas. Sólo sigue
viviendo en el desierto seco, en un salar deshabitado. \footnote{\textbf{17:6}
  Jer 48,6} \bibleverse{7} Dichosos los que confían en el Señor, los que
ponen su confianza en él. \footnote{\textbf{17:7} Sal 146,5}
\bibleverse{8} Son como árboles plantados junto al agua, que echan
raíces hacia la corriente. No se asustan cuando hace calor; sus hojas
están siempre verdes. No se preocupan en tiempos de sequía, sino que
siguen dando fruto.

\hypertarget{dos-dichos-de-la-experiencia-de-vida-el-corazuxf3n-humano-y-la-incertidumbre-de-la-ganancia-injusta}{%
\subsection{Dos dichos de la experiencia de vida: el corazón humano y la
incertidumbre de la ganancia
injusta}\label{dos-dichos-de-la-experiencia-de-vida-el-corazuxf3n-humano-y-la-incertidumbre-de-la-ganancia-injusta}}

\bibleverse{9} La mente es más engañosa que cualquier otra cosa: ¡está
incurablemente enferma! ¿Quién puede entenderla? \bibleverse{10} Pero
yo, el Señor, veo lo que la gente piensa. Examino sus mentes, para poder
recompensarlas según sus actitudes y su forma de comportarse.
\footnote{\textbf{17:10} Sal 7,10; Rom 2,6} \bibleverse{11} Como una
perdiz que empolla huevos que no puso es alguien que hace una fortuna
engañando a los demás. Sus riquezas volarán al mediodía, y al final
quedarán como un tonto. \footnote{\textbf{17:11} Sal 39,7}

\hypertarget{la-gloriosa-posesiuxf3n-de-israel}{%
\subsection{La gloriosa posesión de
Israel}\label{la-gloriosa-posesiuxf3n-de-israel}}

\bibleverse{12} Nuestro Templo es un trono de gloria, levantado en alto
desde el principio. \bibleverse{13} Señor, tú eres la esperanza de
Israel, cualquiera que te abandone será deshonrado. Cualquiera que te dé
la espalda se desvanecerá como nombres escritos en el polvo, porque ha
abandonado al Señor, la fuente de agua viva. \footnote{\textbf{17:13}
  Jer 2,13}

\hypertarget{la-oraciuxf3n-de-venganza-de-jeremuxedas-contra-los-burladores-y-oponentes}{%
\subsection{La oración de venganza de Jeremías contra los burladores y
oponentes}\label{la-oraciuxf3n-de-venganza-de-jeremuxedas-contra-los-burladores-y-oponentes}}

\bibleverse{14} Sáname, Señor, y seré curado; sálvame, y seré salvado,
porque a ti te alabo. \bibleverse{15} Mira cómo siguen diciéndome:
``¿Dónde está el desastre que el Señor ha predicho? ¿Va a ocurrir alguna
vez?'' \footnote{\textbf{17:15} Una expresión coloquial sería:
  ``¡Adelante!''.} \bibleverse{16} Pero no he tenido prisa por dejar de
ser tu pastor. No he querido que llegara el tiempo de los problemas.
Sabes que todo lo que he dicho lo he dicho delante de ti.
\bibleverse{17} ¡Por favor, no seas tú quien me aterrorice! Tú eres mi
protección en el tiempo de la angustia. \bibleverse{18} Avergüenza a mis
perseguidores, pero no a mí. Aterrorízalos a ellos, pero no a mí, haz
que experimenten el tiempo de la angustia, y hazlos pedazos.

\hypertarget{ampliaciuxf3n-de-la-observancia-del-suxe1bado-con-las-correspondientes-promesas-y-amenazas}{%
\subsection{Ampliación de la observancia del sábado con las
correspondientes promesas y
amenazas}\label{ampliaciuxf3n-de-la-observancia-del-suxe1bado-con-las-correspondientes-promesas-y-amenazas}}

\bibleverse{19} Esto es lo que me dijo el Señor: Ve y ponte en la puerta
principal de la ciudad, la que usan los reyes de Judá, y haz lo mismo en
todas las demás puertas de Jerusalén. \bibleverse{20} Diles: Escuchen el
mensaje del Señor, reyes de Judá, y todos ustedes, pueblo de Judá y de
Jerusalén, que entran por estas puertas. \bibleverse{21} Esto es lo que
dice el Señor: ¡Presten atención, si valoran sus vidas! No lleven carga
en el día de reposo, ni la introduzcan por las puertas de Jerusalén.
\bibleverse{22} No saquen carga de sus casas ni hagan ningún trabajo en
el día de reposo. Santifiquen el día de reposo, tal como se lo ordené a
sus antepasados. \footnote{\textbf{17:22} Is 56,2; Is 58,13}
\bibleverse{23} Sin embargo, se negaron a escuchar o prestar atención.
Por el contrario, fueron tercos y se negaron a obedecer o a aceptar la
instrucción. \footnote{\textbf{17:23} Jer 11,8}

\bibleverse{24} Escúchenme bien, dice el Señor, y no introduzcan ninguna
carga por las puertas de esta ciudad en el día de reposo, y santifiquen
el día de reposo, y no hagan ningún trabajo en él. \bibleverse{25}
Entonces reyes y príncipes entrarán por las puertas de esta ciudad. Se
sentarán en el trono de David. Montarán en carros y en caballos con sus
funcionarios, acompañados por el pueblo de Judá y los que viven en
Jerusalén, y esta ciudad estará habitada para siempre. \bibleverse{26}
Vendrá gente de las ciudades de Judá y de todos los alrededores de
Jerusalén, de la tierra de Benjamín y de las tierras bajas, de la región
montañosa y del Néguev. Traerán holocaustos y sacrificios, ofrendas de
grano e incienso, y ofrendas de agradecimiento al Templo del Señor.
\bibleverse{27} Pero si se niegan a escucharme y a santificar el día de
reposo no llevando carga al entrar por las puertas de Jerusalén en el
día de reposo, entonces incendiaré sus puertas con un fuego imposible de
apagar, y quemará las fortalezas de Jerusalén.

\hypertarget{la-obra-del-alfarero-como-suxedmbolo-del-dominio-divino-sobre-el-destino-de-los-pueblos}{%
\subsection{La obra del alfarero como símbolo del dominio divino sobre
el destino de los
pueblos}\label{la-obra-del-alfarero-como-suxedmbolo-del-dominio-divino-sobre-el-destino-de-los-pueblos}}

\hypertarget{section-17}{%
\section{18}\label{section-17}}

\bibleverse{1} Este mensaje llegó a Jeremías de parte del Señor:
\bibleverse{2} Baja enseguida a la casa del alfarero. Allí te daré mi
mensaje.

\bibleverse{3} Bajé a la casa del alfarero y lo vi trabajando en su
torno. \bibleverse{4} Pero la vasija que estaba haciendo con la arcilla
estaba mal. Así que la convirtió en algo diferente, como mejor le
pareció.

\bibleverse{5} Me llegó el mensaje del Señor, diciendo: \bibleverse{6}
Pueblo de Israel, declara el Señor, ¿no puedo tratar con ustedes como
este alfarero hace con su arcilla? Los tengo en mi mano como la arcilla
en la mano del alfarero, pueblo de Israel. \footnote{\textbf{18:6} Is
  45,9; Rom 9,21} \bibleverse{7} En un momento dado puede suceder que yo
anuncie que una nación o un reino va a ser desarraigado, derribado y
destruido. \footnote{\textbf{18:7} Jer 1,10} \bibleverse{8} Sin embargo,
si esa nación a la que advertí abandona sus malos caminos, entonces
cambiaré de opinión respecto al desastre que iba a traer. \footnote{\textbf{18:8}
  Jer 26,3; Jer 26,19; Jon 3,10} \bibleverse{9} En otro momento podría
anunciar que voy a edificar y dar poder a una nación o a un reino.
\bibleverse{10} Pero si hace el mal ante mis ojos y se niega a escuchar
mi voz, entonces cambiaré de opinión con respecto al bien que había
planeado para ella.

\bibleverse{11} Así que dile al pueblo de Judá y a los que viven en
Jerusalén que esto es lo que dice el Señor: ¡Cuidado! Estoy preparando
un desastre para ustedes, y elaborando un plan contra ustedes. Abandonen
todos ustedes sus malos caminos. ¡Vivan bien y actúen bien!
\bibleverse{12} Pero ellos dirán: ``¡No podemos! Haremos lo que nos dé
la gana. Cada uno de nosotros seguirá obstinadamente su propio
pensamiento malvado''. \footnote{\textbf{18:12} Jer 6,16; Jer 3,17}

\hypertarget{la-apostasuxeda-del-pueblo-es-incomprensible-y-antinatural}{%
\subsection{La apostasía del pueblo es incomprensible y
antinatural}\label{la-apostasuxeda-del-pueblo-es-incomprensible-y-antinatural}}

\bibleverse{13} En consecuencia, esto es lo que dice el Señor: Pregunten
en las naciones: ¿alguien ha escuchado algo así? La virgen Israel ha
actuado muy mal. \bibleverse{14} ¿Acaso la nieve del Líbano desaparece
alguna vez de sus cimas rocosas? ¿Se secan alguna vez sus aguas frescas
que fluyen de fuentes tan lejanas? \bibleverse{15} ¡Pero mi pueblo me ha
rechazado! Queman incienso a ídolos inútiles que los hacen tropezar,
haciéndolos abandonar los viejos caminos para andar por senderos sin
hacer en lugar de la carretera. \bibleverse{16} Han convertido su país
en un horrible páramo, un lugar que siempre será tratado con
desprecio.\footnote{\textbf{18:16} ``Tratado con desprecio'':
  literalmente, ``abucheado''.} La gente que pase por allí se
escandalizará y sacudirá la cabeza con incredulidad. \bibleverse{17}
Como un fuerte viento del este, los dispersaré ante el enemigo. Les daré
la espalda y no los miraré cuando llegue su tiempo de angustia.

\hypertarget{ataques-hostiles-de-sacerdotes-y-profetas-impenitentes-contra-jeremuxedas-oraciuxf3n-de-venganza-del-profeta}{%
\subsection{Ataques hostiles de sacerdotes y profetas impenitentes
contra Jeremías; Oración de venganza del
profeta}\label{ataques-hostiles-de-sacerdotes-y-profetas-impenitentes-contra-jeremuxedas-oraciuxf3n-de-venganza-del-profeta}}

\bibleverse{18} Algunos decidieron: ``Necesitamos un plan para lidiar
con Jeremías. Todavía habrá sacerdotes para explicar la ley, todavía
habrá sabios para dar consejos, y todavía habrá profetas para dar
profecías. Organicemos una campaña de desprestigio\footnote{\textbf{18:18}
  ``Organicemos una campaña de desprestigio'': literalmente,
  ``golpeémosle con la lengua''.} contra él para no tener que escuchar
una palabra de lo que dice''. \bibleverse{19} ¡Señor, por favor, presta
atención a lo que me pasa! ¡Escucha lo que dicen mis acusadores!
\bibleverse{20} ¿Hay que pagar el bien con el mal? Sin embargo, ¡han
cavado una fosa para atraparme! ¿Recuerdas cómo me presenté ante ti para
abogar por ellos, para que no te enfadaras con ellos? \footnote{\textbf{18:20}
  Sal 35,7}

\bibleverse{21} Pero ahora que sus hijos se mueran de hambre; que los
maten a espada. Que sus mujeres pierdan a sus hijos y a sus maridos; que
sus maridos mueran de enfermedad; que sus jóvenes mueran en la batalla.
\bibleverse{22} Que se oigan gritos de agonía desde sus casas cuando de
repente traigas invasores que los ataquen, porque han cavado una fosa
para capturarme y han escondido trampas para atraparme mientras camino.
\bibleverse{23} Pero, Señor, tú conoces todas sus conspiraciones para
intentar matarme. No perdones su maldad; no borres su pecado.
Derríbalos. Trata con ellos cuando estés enojado!

\hypertarget{la-abominaciuxf3n-de-thopheth-la-destrucciuxf3n-de-juduxe1-simbuxf3licamente-anunciada-por-el-rompimiento-de-una-jarra}{%
\subsection{La abominación de Thopheth; la destrucción de Judá
simbólicamente anunciada por el rompimiento de una
jarra}\label{la-abominaciuxf3n-de-thopheth-la-destrucciuxf3n-de-juduxe1-simbuxf3licamente-anunciada-por-el-rompimiento-de-una-jarra}}

\hypertarget{section-18}{%
\section{19}\label{section-18}}

\bibleverse{1} Esto es lo que dice el Señor: Ve y compra una vasija de
barro a un alfarero. Lleva contigo a algunos de los ancianos del pueblo
y a los jefes de los sacerdotes, \bibleverse{2} y pasa por la Puerta de
la Cerámica Rota hasta el valle de Ben-hinom. Anuncia este mensaje que
te doy. \footnote{\textbf{19:2} Jer 19,11; Jer 7,31} \bibleverse{3}
Díganles: Escuchen lo que dice el Señor, reyes de Judá y pueblo que vive
en Jerusalén. Esto es lo que dice el Señor Todopoderoso, el Dios de
Israel: Voy a hacer caer sobre este lugar un desastre tal que hará
retumbar los oídos de cualquiera que lo oiga. \footnote{\textbf{19:3}
  1Sam 3,11; 2Re 21,12} \bibleverse{4} Mi pueblo me ha abandonado y ha
hecho de éste un lugar donde se adoran dioses extranjeros. Han quemado
en él incienso a otros dioses que ni ellos ni sus antepasados ni los
reyes de Judá conocían. Han llenado este lugar con la sangre de gente
inocente. \bibleverse{5} Han construido santuarios paganos a Baal donde
queman a sus hijos en el fuego como ofrendas a Baal. Esto es algo que
nunca ordené ni siquiera mencioné. Nunca pensé en algo así. \footnote{\textbf{19:5}
  Jer 7,31-32} \bibleverse{6} Así que ¡cuidado! Se acerca el momento,
declara el Señor, en que en lugar de Tofet y el Valle de Hinom este
lugar será llamado el Valle de la Matanza.

\bibleverse{7} Aquí mismo, en este lugar, voy a echar a perder los
planes de Judá y Jerusalén. Dejaré que sus enemigos que quieren matarlos
vengan y hagan exactamente eso. Sus cadáveres serán alimento para las
aves de rapiña y los animales salvajes. \bibleverse{8} Voy a hacer de
esta ciudad un lugar desolado y burlado. Todos los que pasen por allí se
horrorizarán, se escandalizarán de todo su daño. \footnote{\textbf{19:8}
  Jer 18,16} \bibleverse{9} El asedio de sus enemigos que quieren
matarlos será tan terrible que haré que se coman unos a otros, incluso a
sus propios hijos e hijas. \footnote{\textbf{19:9} Deut 28,53}

\bibleverse{10} Entonces rompe la vasija delante de la gente que está
contigo. \bibleverse{11} Diles: Esto es lo que dice el Señor
Todopoderoso: Voy a destrozar esta nación y esta ciudad, como se
destroza una vasija de barro para que no se pueda reparar jamás. La
gente enterrará a sus muertos en Tofet hasta que se llene. \footnote{\textbf{19:11}
  Is 30,14; Jer 7,32} \bibleverse{12} Esto es lo que voy a hacer con
este lugar y con la gente que vive aquí, declara el Señor. Convertiré
esta ciudad en Tofet. \bibleverse{13} Todas las casas de Jerusalén y los
palacios de los reyes de Judá se volverán inmundos como Tofet, porque
todas son casas en cuyos tejados se quemaba incienso al sol, a la luna y
a las estrellas, y se derramaban libaciones a otros dioses.

\hypertarget{el-discurso-de-jeremuxedas-ante-el-pueblo-en-el-patio-del-templo-su-maltrato-por-parte-del-coronel-del-templo-pashur}{%
\subsection{El discurso de Jeremías ante el pueblo en el patio del
templo; su maltrato por parte del coronel del templo
Pashur}\label{el-discurso-de-jeremuxedas-ante-el-pueblo-en-el-patio-del-templo-su-maltrato-por-parte-del-coronel-del-templo-pashur}}

\bibleverse{14} Jeremías regresó de Tofet, adonde el Señor lo había
enviado a entregar este mensaje. Fue y se puso de pie en el patio del
Templo del Señor y anunció a todos: \bibleverse{15} Esto es lo que dice
el Señor Todopoderoso, el Dios de Israel: ¡Cuidado! Estoy a punto de
hacer caer sobre esta ciudad y sobre todos sus pueblos circundantes
todos los desastres de los que les advertí, porque se han negado
obstinadamente a escuchar lo que digo.

\hypertarget{section-19}{%
\section{20}\label{section-19}}

\bibleverse{1} Pasur, hijo de Imer, era un sacerdote y el
funcionario\footnote{\textbf{20:1} Probablemente a cargo de los guardias
  del Templo, y claramente no el sumo sacerdote.} encargado del Templo
del Señor. Cuando oyó que Jeremías profetizaba estas cosas,
\bibleverse{2} golpeó al profeta Jeremías y lo hizo poner en el calabozo
de la Puerta Superior de Benjamín, cerca del Templo del Señor.
\bibleverse{3} Al día siguiente, cuando Pasur hizo que soltaran a
Jeremías del cepo, éste le dijo: ``El Señor no te llama Pasur
(despedazar), sino Magor-misabib (el terror está en todas partes).
\bibleverse{4} Porque esto es lo que dice el Señor: Voy a aterrorizarte
a ti y a todos los que amas. Los enemigos los matarán mientras tú miras.
Entregaré a Judá al rey de Babilonia. Matará a algunos, y al resto se lo
llevará al exilio en Babilonia. \bibleverse{5} ``Lo entregaré todo.
Todas las riquezas de esta ciudad, todos los resultados del trabajo
duro, todos los objetos de valor, todas las joyas de la corona de los
reyes de Judá: voy a entregárselas a sus enemigos, que las tomarán como
botín y se las llevarán a Babilonia. \footnote{\textbf{20:5} Is 39,6}
\bibleverse{6} ``Tú, Pasur, y todos los que viven contigo, irán al
cautiverio. Irás a Babilonia. Morirás allí y serás enterrado, tú y todos
los que amas, aquellos a los que les profetizaste mentiras''.

\hypertarget{el-amargo-lamento-del-profeta-por-los-sufrimientos-de-su-profesiuxf3n-sus-luchas-internas-y-su-consuelo}{%
\subsection{El amargo lamento del profeta por los sufrimientos de su
profesión; sus luchas internas y su
consuelo}\label{el-amargo-lamento-del-profeta-por-los-sufrimientos-de-su-profesiuxf3n-sus-luchas-internas-y-su-consuelo}}

\bibleverse{7} Me engañaste, Señor, y me dejé engañar!\footnote{\textbf{20:7}
  Aquí es Jeremías quien expresa esto.} Eres más fuerte que yo: ¡has
ganado! Me he convertido en un chiste del que la gente se ríe todo el
día. Todo el mundo se burla de mí. \bibleverse{8} Esto se debe a que
cada vez que abro la boca tengo que gritar amenazas de violencia y
destrucción. El mensaje del Señor se ha convertido en la razón por la
que la gente me critica y me ridiculiza todo el tiempo. \footnote{\textbf{20:8}
  Is 49,4} \bibleverse{9} Si me digo a mí mismo: ``No hablaré más de él,
ni siquiera mencionaré su nombre'', entonces su mensaje es como un fuego
atrapado dentro de mí, que me quema por dentro. Me estoy cansando de
aguantar. Simplemente no puedo ganar. \bibleverse{10} He oído a mucha
gente murmurar: ``¡Él es el que dice que el terror está en todas partes!
¡Hay que denunciarle! Denunciar lo que hace!''\footnote{\textbf{20:10}
  La idea es que Jeremías está creando pánico y debe ser reportado a los
  líderes del país.} Todos mis buenos amigos están esperando que cometa
un error. ``Tal vez cometa un error para que podamos derrotarlo y
vengarnos de él'', dicen.

\bibleverse{11} Pero el Señor está a mi lado como un poderoso guerrero.
Por eso, los que me atacan caerán. No ganarán. Al no tener éxito
quedarán totalmente deshonrados. Su vergüenza no se olvidará jamás.
\footnote{\textbf{20:11} Jer 1,8; Jer 1,19; Jer 15,20} \bibleverse{12}
Señor Todopoderoso, tú sabes sin lugar a dudas quién vive bien. Tú
examinas los pensamientos y sentimientos de la gente. Así que deja que
tu castigo caiga sobre ellos, porque he confiado en ti para que juzgues
mi caso. \footnote{\textbf{20:12} Jer 11,20} \bibleverse{13} ¡Canten al
Señor! ¡Alaben al Señor! Porque él salva a los pobres del poder de los
malvados.

\hypertarget{jeremuxedas-maldice-su-vida}{%
\subsection{Jeremías maldice su
vida}\label{jeremuxedas-maldice-su-vida}}

\bibleverse{14} ¡Que se maldiga el día en que nací! ¡Que el día en que
mi madre me dio a luz nunca sea bendecido! \footnote{\textbf{20:14} Jer
  15,10; Job 3,1-10; Job 10,18}

\bibleverse{15} Que sea maldito el hombre que le trajo a mi padre la
noticia que lo alegró mucho, diciendo: ``Tienes un hijo''.
\bibleverse{16} Que ese hombre sea como las ciudades que el Señor
destruyó sin piedad. Que oiga gritos de alarma por la mañana y gritos de
guerra al mediodía, \bibleverse{17} porque debió matarme en el vientre
para que mi madre fuera mi tumba, quedando embarazada para siempre.
\bibleverse{18} ¿Por qué nací sólo para ver problemas y tristeza, y para
terminar mi vida en la vergüenza?

\hypertarget{unheilvolle-antwort-jeremias-an-zedekias-gesandte-und-mahnungen-an-das-volk-wuxe4hrend-der-belagerung-jerusalems}{%
\subsection{Unheilvolle Antwort Jeremias an Zedekias Gesandte und
Mahnungen an das Volk während der Belagerung
Jerusalems}\label{unheilvolle-antwort-jeremias-an-zedekias-gesandte-und-mahnungen-an-das-volk-wuxe4hrend-der-belagerung-jerusalems}}

\hypertarget{section-20}{%
\section{21}\label{section-20}}

\bibleverse{1} Este es el mensaje que le llegó a Jeremías de parte del
Señor cuando el rey Sedequías envió a Pasur, hijo de Malquías, y al
sacerdote Sofonías, hijo de Maasías, a hablar con él. Le dijeron:
\footnote{\textbf{21:1} Jer 29,25} \bibleverse{2} ``Por favor, habla con
el Señor en nuestro favor porque Nabucodonosor, rey de Babilonia, nos
está atacando. Tal vez el Señor haga algún milagro por nosotros como
todos los que solía hacer, para que Nabucodonosor se retire de
nosotros''.

\bibleverse{3} Pero Jeremías respondió: ``Dile esto a Sedequías:
\bibleverse{4} Esto es lo que dice el Señor, el Dios de Israel: Voy a
volver contra ti las armas que tienes en la mano, las armas que usas
para luchar contra el rey de Babilonia y el ejército babilónico fuera
del muro que te asedia. Voy a llevarlas al centro de esta ciudad.
\bibleverse{5} Yo mismo lucharé contra ti con todo mi poder y fuerza,
con toda la fuerza de mi furiosa ira. \bibleverse{6} Mataré a los que
viven en esta ciudad, a los seres humanos y a los animales. Morirán a
causa de una terrible plaga. \bibleverse{7} ``Después de eso, declara el
Señor, voy a entregarte a ti, Sedequías, rey de Judá, así como a tus
oficiales y a la gente que quede en esta ciudad después de la peste, la
guerra y el hambre, a Nabucodonosor, rey de Babilonia, y a tus enemigos
que te quieren muerto. Él te atacará; no te perdonará ni tendrá piedad
ni misericordia.

\hypertarget{asesoramiento-a-la-gente}{%
\subsection{Asesoramiento a la gente}\label{asesoramiento-a-la-gente}}

\bibleverse{8} ``Dile también al pueblo esto: Esto es lo que dice el
Señor: Mira, pongo ante ti el camino de la vida y el de la muerte.
\bibleverse{9} Si se quedan en esta ciudad, morirán por la espada, el
hambre y la peste, pero si se van y se rinden a los babilonios que los
están sitiando, vivirán. De hecho, será como ganar tu vida como si fuera
botín de guerra. \footnote{\textbf{21:9} Jer 38,2} \bibleverse{10}
Porque estoy decidido a traer un desastre sobre esta ciudad, y no
bendiciones, declara el Señor. Será entregada al rey de Babilonia, que
la destruirá con fuego.

\hypertarget{advertencia-a-la-familia-real}{%
\subsection{Advertencia a la familia
real}\label{advertencia-a-la-familia-real}}

\bibleverse{11} ``Además, dile a la familia real del rey de Judá que
escuche el mensaje del Señor: \bibleverse{12} Descendientes de David,
esto es lo que dice el Señor: Asegúrense de juzgar con justicia cada
día. Protejan a los que son tratados injustamente de esa gente corrupta,
pues de lo contrario, a causa de sus malas acciones, mi ira arderá como
un fuego que no se puede apagar.

\hypertarget{anuncio-de-sentencia-para-la-ciudad-de-jerusaluxe9n}{%
\subsection{Anuncio de sentencia para la ciudad de
Jerusalén}\label{anuncio-de-sentencia-para-la-ciudad-de-jerusaluxe9n}}

\bibleverse{13} Tengan cuidado, porque voy a luchar contra ustedes, que
viven sobre el valle, en lo alto de una roca plana,\footnote{\textbf{21:13}
  Dado que este mensaje estaba dirigido a la familia real de Judá, se
  entiende que esta referencia se refiere a las fortalezas reales de la
  Ciudad de David sobre Jerusalén.} declara el Señor. Tú dices: `¿Quién
puede atacarnos? ¿Quién puede derribar nuestras defensas?'
\bibleverse{14} Voy a castigarte como te mereces por lo que has hecho,
declara el Señor. Pondré tu bosque\footnote{\textbf{21:14} El palacio de
  Salomón se denominaba ``El Palacio del Bosque del Líbano''. 1 Reyes
  7:2; 1 Reyes 10:21.} en el fuego y quemará todo lo que te rodea''.

\hypertarget{advertencia-a-la-casa-real-de-juduxe1-anuncio-del-juicio-sobre-los-reyes-salum-joacaz-y-joacim}{%
\subsection{Advertencia a la casa real de Judá; Anuncio del juicio sobre
los reyes Salum (Joacaz) y
Joacim}\label{advertencia-a-la-casa-real-de-juduxe1-anuncio-del-juicio-sobre-los-reyes-salum-joacaz-y-joacim}}

\hypertarget{section-21}{%
\section{22}\label{section-21}}

\bibleverse{1} Esto es lo que dice el Señor: Vayan al palacio del rey de
Judá y den este mensaje. \bibleverse{2} Díganles: Oye lo que el Señor
quiere decirte, rey de Judá, sentado en el trono de David, a ti y a tus
funcionarios y al pueblo que está aquí contigo. \bibleverse{3} Esto es
lo que dice el Señor: Haz lo que es justo y correcto. Protege a los que
son tratados injustamente por gente corrupta. No hagas nada malo a los
extranjeros, a los huérfanos o a las viudas. No uses la violencia contra
ellos. No mates a los inocentes. \footnote{\textbf{22:3} Jer 21,12}
\bibleverse{4} Si haces honestamente lo que te digo, los reyes que se
sientan en el trono de David pasarán en carros y caballos con sus
funcionarios por las puertas de este palacio. Los acompañará el pueblo
de Judá y los que viven en Jerusalén. \footnote{\textbf{22:4} Jer 17,25}
\bibleverse{5} Pero si te niegas a obedecer lo que digo, entonces juro
por mí, declara el Señor, que este palacio será convertido en escombros.

\hypertarget{maldiciuxf3n-en-el-palacio-real}{%
\subsection{Maldición en el palacio
real}\label{maldiciuxf3n-en-el-palacio-real}}

\bibleverse{6} Esto es lo que dice el Señor sobre la familia real del
rey de Judá: Ustedes son tan estimados para mí como los bosques de
Galaad y los montes del Líbano. Pero te convertiré en un desierto, en
ciudades donde nadie vive. \bibleverse{7} Escogeré hombres que vengan a
destruirte, cada uno con su propia hacha. Cortarán tus hermosos
cedros\footnote{\textbf{22:7} El palacio estaba hecho de muchos cedros
  grandes. Ver 21:14.} y los arrojarán al fuego.

\bibleverse{8} Extranjeros de muchas naciones pasarán por esta ciudad y
se preguntarán unos a otros: ``¿Por qué el Señor ha hecho cosas tan
terribles a esta gran ciudad?'' \bibleverse{9} La gente responderá:
``Porque rompieron el acuerdo del Señor, su Dios. Fueron a adorar a
otros dioses''.

\hypertarget{unas-palabras-de-puxe9same-para-el-desafortunado-sallum-joacaz}{%
\subsection{Unas palabras de pésame para el desafortunado Sallum
(Joacaz)}\label{unas-palabras-de-puxe9same-para-el-desafortunado-sallum-joacaz}}

\bibleverse{10} No lloren por el rey que murió. No lloren por él. En
cambio, lloren por el rey que está exiliado, que nunca regresará, que
nunca volverá a ver su patria. \bibleverse{11} Esto es lo que dice el
Señor sobre Joacaz\footnote{\textbf{22:11} Aquí se llama ``Shallum''.}
de Josías, rey de Judá. Sucedió a su padre Josías, pero se lo llevaron.
Nunca regresará. \footnote{\textbf{22:11} 2Cró 36,3-4} \bibleverse{12}
Morirá en el exilio; no volverá a ver este país.

\hypertarget{graves-acusaciones-y-amenazas-de-castigo-contra-el-rey-jojakim}{%
\subsection{Graves acusaciones y amenazas de castigo contra el rey
Jojakim}\label{graves-acusaciones-y-amenazas-de-castigo-contra-el-rey-jojakim}}

\bibleverse{13} A Joacim le llegan los problemas\footnote{\textbf{22:13}
  En realidad, no se menciona a Joaquín por su nombre hasta el versículo
  18.} porque maltrata a otros en la construcción de su palacio,
tratando injustamente a los que construyen los pisos superiores. Hace
trabajar a los suyos a cambio de nada: no les paga ningún salario.
\bibleverse{14} Se dice a sí mismo: ``Voy a construirme un gran palacio,
con grandes habitaciones superiores''. Hace colocar ventanas, pone
paneles de cedro y lo pinta de rojo brillante con bermellón.
\bibleverse{15} ¿Acaso te hace rey el hecho de tener más cedro que
nadie? Tu padre tenía comida y bebida, ¿no es así? Gobernó con justicia
y honestidad, y por eso tuvo una buena vida. \bibleverse{16} Defendía a
los pobres y a los necesitados, y así las cosas iban bien. ¿No es esto
lo que significa realmente conocerme? declara el Señor. \bibleverse{17}
Pero lo único que buscas, lo único en lo que piensas, es en conseguir lo
que quieres, aunque sea de forma deshonesta. Matas a los inocentes,
maltratas con violencia y explotas a tu pueblo. \bibleverse{18} Esto es
lo que dice el Señor sobre Joaquín, hijo de Josías, rey de Judá: No
harán duelo por él, diciendo: ``¡Qué triste, hermano mío! Qué tristeza,
hermana mía!'' No llorarán por él, diciendo: ``¡Qué triste, mi señor!
Qué triste, su majestad!'' \footnote{\textbf{22:18} Jer 34,5}
\bibleverse{19} Su entierro será el de un burro. Lo arrastrarán y lo
tirarán fuera de las puertas de Jerusalén. \footnote{\textbf{22:19} Is
  14,19}

\hypertarget{proclamaciuxf3n-de-la-condenaciuxf3n-de-jerusaluxe9n-y-su-rey-joaquuxedn}{%
\subsection{Proclamación de la condenación de Jerusalén y su rey
Joaquín}\label{proclamaciuxf3n-de-la-condenaciuxf3n-de-jerusaluxe9n-y-su-rey-joaquuxedn}}

\bibleverse{20} ¡Vayan al Líbano y griten pidiendo ayuda! ¡Grita en
Basán! ¡Grita desde Abarim! Porque todos tus amantes han sido
destruidos.\footnote{\textbf{22:20} Todos los lugares mencionados son
  montañas, por lo que la imagen es de una llamada que se grita desde
  allí. Lo más probable es que los amantes se refieran a las naciones
  con las que se ha establecido una alianza.} \bibleverse{21} Te advertí
cuando pensabas que estabas sano y salvo. Pero tú respondiste: ``¡No voy
a hacer caso!''. Esa ha sido tu actitud desde que eras joven: nunca
hiciste lo que te dije. \bibleverse{22} El viento se llevará a todos tus
``pastores'',\footnote{\textbf{22:22} ``Pastores'': Una referencia a los
  líderes de la nación.} y tus amantes irán al exilio. Entonces serás
avergonzada y deshonrada por todas las cosas malas que has hecho.
\footnote{\textbf{22:22} Jer 25,9; Jer 25,18} \bibleverse{23} Tú que
vives en el ``Líbano'' en tu nido de cedro,\footnote{\textbf{22:23} Otra
  referencia al palacio de cedro de Salomón.} cuánto vas a gemir cuando
los dolores agónicos te golpeen como a una mujer de parto. \footnote{\textbf{22:23}
  Jer 13,21}

\hypertarget{tres-dichos-sobre-el-rey-conuxedas-joaquuxedn}{%
\subsection{Tres dichos sobre el rey Conías
(Joaquín)}\label{tres-dichos-sobre-el-rey-conuxedas-joaquuxedn}}

\bibleverse{24} El Señor dijo a Joaquín,\footnote{\textbf{22:24} Aquí se
  llama ``Conías''. Igual que en el versículo 28.} hijo de Joacim, rey
de Judá: Vivo yo, declara el Señor, que aunque fueras un anillo de sello
en un dedo de mi mano derecha, te arrancaría. \footnote{\textbf{22:24}
  Jer 24,1}

\bibleverse{25} Te voy a entregar a los que te aterrorizan y quieren
matarte, a Nabucodonosor, rey de Babilonia, y a los babilonios.
\bibleverse{26} Te voy a echar -a ti y a la madre que te dio a luz- y te
voy a enviar a otro país. Ninguno de ustedes nació allí, pero ambos
morirán allí. \bibleverse{27} Jamás volverán al país que tanto aman.
\bibleverse{28} ¿Quién es este hombre Joaquín? ¿Una vasija rota que ha
sido desechada, algo que nadie quiere? ¿Por qué lo han echado a él y a
sus hijos, exiliados en un país desconocido? \bibleverse{29} ¡Mi país,
mi país, mi país! ¡Escucha lo que dice el Señor! \bibleverse{30} Esto es
lo que dice el Señor: Anota a este hombre como si no tuviera hijos. Es
un hombre que no tendrá éxito en toda su vida. Ninguno de sus hijos
tendrá éxito tampoco. Ninguno de ellos se sentará en el trono de David
ni será rey en Judá.

\hypertarget{ay-de-los-pastores-infieles-y-la-promesa-del-verdadero-pastor-de-la-casa-de-david}{%
\subsection{Ay de los pastores infieles y la promesa del verdadero
pastor de la casa de
David}\label{ay-de-los-pastores-infieles-y-la-promesa-del-verdadero-pastor-de-la-casa-de-david}}

\hypertarget{section-22}{%
\section{23}\label{section-22}}

\bibleverse{1} Qué desgracia les espera a los pastores que destruyen y
dispersan las ovejas de mi prado! declara el Señor. \footnote{\textbf{23:1}
  Ezeq 13,2-16; Ezeq 34,1-34; Zac 11,5} \bibleverse{2} Esto es lo que el
Señor, el Dios de Israel, dice de los pastores que debían cuidar a mi
pueblo: Ustedes han dispersado mi rebaño. Los han ahuyentado y no los
han cuidado, así que ahora me ocuparé de ustedes por todo el mal que han
hecho, declara el Señor. \bibleverse{3} Yo mismo reuniré lo que queda de
mi rebaño de todos los países donde lo desterré, y lo devolveré a sus
pastos, donde aumentará en número. \bibleverse{4} Pondré al frente de
ellas a pastores que las cuidarán, y ya no tendrán miedo ni se
desanimarán, y no faltará ninguna, declara el Señor.

\hypertarget{promesa-del-brote-de-david}{%
\subsection{Promesa del brote de
David}\label{promesa-del-brote-de-david}}

\bibleverse{5} Mira, se acerca el momento, declara el Señor, en que
elegiré a un descendiente de David que haga lo correcto. Como rey
gobernará con sabiduría y hará lo que es justo y correcto en todo el
país. \footnote{\textbf{23:5} Zac 3,8; Zac 6,12; Is 21,1} \bibleverse{6}
Cuando sea rey, Judá se salvará e Israel vivirá en seguridad. Este es el
nombre que recibirá: El Señor que nos hace justos. \footnote{\textbf{23:6}
  Jer 33,16} \bibleverse{7} Mira, se acerca el tiempo, declara el Señor,
en que la gente ya no dirá: ``Por la vida del Señor, que sacó a los
israelitas de Egipto''. \footnote{\textbf{23:7} Jer 16,14-15}
\bibleverse{8} En cambio, dirán: ``Por la vida del Señor, que hizo
regresar a los israelitas del país del norte y de todos los demás países
donde los había exiliado''. Entonces vivirán en su propio país.

\hypertarget{lamentar-la-corrupciuxf3n-y-depravaciuxf3n-general-de-los-luxedderes-espirituales}{%
\subsection{Lamentar la corrupción y depravación general de los líderes
espirituales}\label{lamentar-la-corrupciuxf3n-y-depravaciuxf3n-general-de-los-luxedderes-espirituales}}

\bibleverse{9} Cuando se trata de los profetas: Estoy realmente
perturbado: ¡estoy temblando por dentro! Me tambaleo como un borracho,
como alguien que ha tomado demasiado vino, por lo que es el Señor, por
sus santas palabras.\footnote{\textbf{23:9} Jeremías está alarmado por
  el contraste entre lo que conoce de la naturaleza del Señor y el
  estado degradado de la nación, y lo que esto significa en términos del
  desastre que se avecina.} \bibleverse{10} Porque el país está lleno de
gente que comete adulterio, por lo que está bajo una maldición. La
tierra está de luto y los pastos del desierto se han secado. La gente
vive mal, usando su energía para hacer el mal. \bibleverse{11} Tanto los
profetas como los sacerdotes no me respetan. Veo la maldad incluso en mi
Templo, declara el Señor. \bibleverse{12} Por eso su camino se volverá
resbaladizo; serán perseguidos en la oscuridad y caerán. Voy a traer el
desastre sobre ellos en el momento en que sean castigados, declara el
Señor.

\hypertarget{dichos-sobre-los-falsos-profetas}{%
\subsection{Dichos sobre los falsos
profetas}\label{dichos-sobre-los-falsos-profetas}}

\bibleverse{13} Vi a los profetas de Samaria haciendo algo realmente
ofensivo: Profetizaban en nombre de Baal y llevaban a mi pueblo Israel a
pecar. \bibleverse{14} Pero ahora veo a los profetas de Jerusalén
haciendo algo aún más repugnante: Cometen adulterio y sus vidas son una
mentira. Apoyan a los malvados, para que nadie deje de pecar. Para mí
todos son como Sodoma; la gente de Jerusalén es como Gomorra.
\footnote{\textbf{23:14} Ezeq 13,22; Is 1,10}

\bibleverse{15} Esto es lo que dice el Señor Todopoderoso sobre los
profetas: Les daré ajenjo para comer y agua envenenada para beber,
porque el mal se ha extendido por todo el país desde los profetas de
Jerusalén. \footnote{\textbf{23:15} Jer 9,14}

\bibleverse{16} Esto es lo que dice el Señor Todopoderoso: No presten
atención a lo que dicen estos profetas cuando les profetizan. Te están
engañando con visiones que inventan en su propia mente. No vienen de mí.
\footnote{\textbf{23:16} Jer 6,14} \bibleverse{17} Se la pasan diciendo
a la gente que no me respeta: ``El Señor dice que ustedes vivirán en
paz'', y a todos los que siguen su propia actitud obstinada: ``Nada malo
les va a pasar''. \footnote{\textbf{23:17} Jer 7,24} \bibleverse{18}
Pero ¿quién de ellos ha asistido al consejo del Señor para escuchar y
entender lo que dice? ¿Quién ha prestado atención a sus instrucciones y
las ha seguido? \footnote{\textbf{23:18} Is 40,13} \bibleverse{19}
¡Cuidado! El Señor ha enviado una furiosa tormenta, un tornado que gira
en torno a las cabezas de los malvados. \footnote{\textbf{23:19} Jer
  30,23} \bibleverse{20} La ira del Señor no se desvanecerá hasta que
termine de hacer todo lo que quiere. Sólo entonces entenderá realmente.
\bibleverse{21} Yo no envié a estos profetas, sino que ellos corren a
entregar sus mensajes. Yo no les dije que dijeran nada, pero aun así
siguen profetizando. \footnote{\textbf{23:21} Jer 14,14} \bibleverse{22}
Ahora bien, si hubieran asistido a mi consejo, habrían entregado mis
instrucciones a mi pueblo y lo habrían hecho retroceder de su mala forma
de vida, de sus malas acciones.

\bibleverse{23} ¿Acaso soy sólo un Dios local y no un Dios que actúa
ampliamente? pregunta el Señor. \bibleverse{24} ¿Puede la gente
esconderse en lugares secretos donde yo no pueda verlos? pregunta el
Señor. ¿No actúo en todos los lugares del cielo y de la tierra? pregunta
el Señor.

\hypertarget{advertencia-de-los-sueuxf1os-de-los-profetas-mentirosos}{%
\subsection{Advertencia de los sueños de los profetas
mentirosos}\label{advertencia-de-los-sueuxf1os-de-los-profetas-mentirosos}}

\bibleverse{25} He escuchado a los profetas que profetizan mentiras en
mi nombre. Dicen: ``¡He tenido un sueño! He tenido un sueño!''
\bibleverse{26} ¿Hasta cuándo seguirá esto? ¿Hasta cuándo seguirán estos
profetas profetizando estas mentiras que no son más que el producto de
sus propias mentes engañadas? \bibleverse{27} Creen que los sueños que
se repiten unos a otros harán que mi pueblo se olvide de mí, como sus
antepasados se olvidaron de mí al adorar a Baal. \bibleverse{28} El
profeta que tenga un sueño debe decir que es sólo un sueño, pero
cualquier persona a la que le haya hablado debe entregar mi mensaje
fielmente. ¿Qué es la paja en comparación con el grano? pregunta el
Señor. \bibleverse{29} ¿No arde mi palabra como el fuego? pregunta el
Señor. ¿No es como un martillo que rompe una roca?

\bibleverse{30} Presta atención a esto, declara el Señor. Me opongo a
los profetas que se roban las palabras unos a otros y luego dicen que es
un mensaje mío. \bibleverse{31} Presten atención a esto, declara el
Señor. Me opongo a los profetas que se inventan sus propias
historias\footnote{\textbf{23:31} ``Se inventan sus propias historias'':
  literalmente, ``toman sus propias lenguas''.} y luego anuncia: ``Esto
es lo que dice el Señor''. \bibleverse{32} Presten atención a esto
declara el Señor, me opongo a los que profetizan sueños ficticios. Los
cuentan para llevar a mi pueblo al pecado con sus mentiras
descabelladas. Yo no los envié ni les di instrucciones, y no le hacen
ningún bien a nadie, declara el Señor. \footnote{\textbf{23:32} Jer
  23,21}

\hypertarget{advertencia-contra-la-expresiuxf3n-incorrecta-carga-del-seuxf1or}{%
\subsection{Advertencia contra la expresión incorrecta ``carga del
Señor''}\label{advertencia-contra-la-expresiuxf3n-incorrecta-carga-del-seuxf1or}}

\bibleverse{33} Por eso, cuando venga un profeta, un sacerdote o
cualquier otra persona y les pregunte: ``¿Cuál es la carga del Señor?''
\footnote{\textbf{23:33} ``La carga del Señor'' en el sentido de algún
  requisito que el Señor impone al pueblo. Claramente el pueblo veía las
  diversas leyes y regulaciones como ``cargas'' y se relacionaba con el
  Señor como este ``dador de cargas'' legalista. Aunque vivían mal,
  parece que pensaban que si observaban tales ``cargas'', incluyendo las
  nuevas, entonces el Señor estaría satisfecho.} diles, no te estoy
dando una carga. Me desentiendo de ustedes, declara el Señor.
\bibleverse{34} Si un profeta o un sacerdote o cualquier otra persona
afirma: ``Esta es la carga del Señor'', castigaré a esa persona y a su
familia. \bibleverse{35} Esto es lo que todos deben decir a sus amigos y
parientes: ``¿Qué respuesta ha dado el Señor?'' o, ``¿Qué ha dicho el
Señor?'' \bibleverse{36} No hablen más de ``la carga del Señor'', porque
todos tienen ideas diferentes sobre esta ``carga'', pervirtiendo las
palabras del Dios vivo, el Señor Todopoderoso, nuestro Dios.
\bibleverse{37} Esto es lo que debes decir pregúntale a cualquier
profeta: ``¿Qué mensaje te ha dado el Señor?'' y ``¿Qué te ha dicho el
Señor?'' \bibleverse{38} Si dicen: ``Ésta es la carga del Señor'', ésta
es la respuesta del Señor: Porque dijiste: ``Esta es la carga del
Señor'', y yo te advertí que no lo hicieras, \bibleverse{39} ahora te
voy a recoger como una carga y te voy a tirar, a ti y a la ciudad que te
di a ti y a tus antepasados. \bibleverse{40} Te deshonraré para siempre,
tu vergüenza nunca será olvidada.

\hypertarget{la-cara-de-las-dos-cestas-de-higos-y-el-significado-de-las-cestas}{%
\subsection{La cara de las dos cestas de higos y el significado de las
cestas}\label{la-cara-de-las-dos-cestas-de-higos-y-el-significado-de-las-cestas}}

\hypertarget{section-23}{%
\section{24}\label{section-23}}

\bibleverse{1} El Señor me mostró en visión dos cestas de higos
colocadas delante del Templo del Señor. Esto sucedió después de que
Nabucodonosor, rey de Babilonia, llevara a Babilonia a
Joaquín,\footnote{\textbf{24:1} Aquí se llama ``Jeconías''.} hijo de
Joacim, rey de Judá, así como los dirigentes de Judá y los artesanos y
metalúrgicos de Jerusalén. \footnote{\textbf{24:1} Jer 29,2; 2Re
  24,14-15} \bibleverse{2} Una cesta estaba llena de higos muy buenos,
como los que maduran pronto, pero en la otra cesta sólo había higos muy
malos, tan malos que no se podían comer.

\bibleverse{3} ``Jeremías'', preguntó el Señor, ``¿qué ves?'' ``¡Veo
higos!'' Respondí. ``Los higos buenos parecen muy buenos, pero los higos
malos parecen muy malos, tan malos que no se pueden comer''.

\bibleverse{4} Entonces me llegó un mensaje del Señor, que decía:
\bibleverse{5} Esto es lo que dice el Señor, el Dios de Israel: Los
higos buenos representan para mí a los exiliados de Judá, a los que he
enviado de aquí al país de Babilonia. \bibleverse{6} Yo velaré por ellos
y los haré volver a este país. Los edificaré y no los derribaré; los
plantaré y no los desarraigaré. \footnote{\textbf{24:6} Jer 31,28}
\bibleverse{7} Les daré el deseo de conocerme, de saber que yo soy el
Señor. Serán mi pueblo, y yo seré su Dios, porque volverán a estar
completamente comprometidos conmigo. \footnote{\textbf{24:7} Jer
  31,33-34}

\bibleverse{8} Pero los higos malos, tan malos que no se pueden comer,
dice el Señor, representan la forma en que trataré a Sedequías, rey de
Judá, a sus funcionarios y a los que quedan de Jerusalén, así como a los
que quedan en este país y a los que viven en Egipto. \footnote{\textbf{24:8}
  Jer 29,17} \bibleverse{9} Voy a hacer de ellos un ejemplo que
horrorizará y ofenderá a todos en la tierra. Serán deshonrados,
burlados, ridiculizados y maldecidos en todos los lugares a los que los
he exiliado. \footnote{\textbf{24:9} Jer 29,18}

\bibleverse{10} Voy a atacarlos con guerras, hambre y plagas, hasta que
sean completamente eliminados del país que les di a ellos y a sus
antepasados.

\hypertarget{indicaciuxf3n-de-tiempo-la-referencia-de-jeremuxedas-a-su-efectividad-fallida-de-23-auxf1os}{%
\subsection{Indicación de tiempo; La referencia de Jeremías a su
efectividad fallida de 23
años}\label{indicaciuxf3n-de-tiempo-la-referencia-de-jeremuxedas-a-su-efectividad-fallida-de-23-auxf1os}}

\hypertarget{section-24}{%
\section{25}\label{section-24}}

\bibleverse{1} Este es el mensaje que llegó a Jeremías en el cuarto año
de Joaquín, hijo de Josías, rey de Judá, que era el primer año de
Nabucodonosor, rey de Babilonia. Se refería a todo el pueblo de Judá.
\bibleverse{2} Entonces el profeta Jeremías fue y habló a todo el pueblo
de Judá y a toda la gente que vivía en Jerusalén, diciéndoles
\bibleverse{3} Desde el año trece del reinado de Josías, hijo de Amón,
rey de Judá, hasta ahora, veintitrés años en total, me han llegado
mensajes del Señor, y les he dicho lo que él decía una y otra vez, pero
ustedes no han escuchado.

\bibleverse{4} A pesar de que el Señor les ha enviado una y otra vez a
todos sus siervos los profetas, ustedes no se molestan en escuchar ni en
prestar atención. \bibleverse{5} El mensaje constante ha sido: Dejen sus
malos caminos y las cosas malas que están haciendo para que puedan vivir
en el país que el Señor les ha dado a ustedes y a sus antepasados para
siempre. \footnote{\textbf{25:5} Jer 18,11} \bibleverse{6} No sigan a
otros dioses ni los adoren, y no me enojen al construir
ídolos.\footnote{\textbf{25:6} ``Al construir ídolos'': literalmente,
  ``con las obras de tus manos''.} Entonces no haré nada que os
perjudique.

\bibleverse{7} Pero ustedes se han perjudicado a sí mismos al no
escucharme, declara el Señor, porque me enojaron haciendo ídolos.

\hypertarget{anuncio-de-la-aniquilaciuxf3n-de-juduxe1-asuxed-como-de-los-setenta-auxf1os-de-cautiverio-en-babilonia-y-el-posterior-castigo-de-los-caldeos}{%
\subsection{Anuncio de la aniquilación de Judá, así como de los setenta
años de cautiverio en Babilonia y el posterior castigo de los
caldeos}\label{anuncio-de-la-aniquilaciuxf3n-de-juduxe1-asuxed-como-de-los-setenta-auxf1os-de-cautiverio-en-babilonia-y-el-posterior-castigo-de-los-caldeos}}

\bibleverse{8} Así que esto es lo que dice el Señor Todopoderoso: Como
no han obedecido lo que les dije, \bibleverse{9} miren cómo convoco a
todo el pueblo del norte, declara el Señor. Voy a enviar a mi siervo
Nabucodonosor, rey de Babilonia, para que ataque a este país y a la
gente que vive aquí, y a todas las naciones de los alrededores. Los
destinaré a la destrucción.\footnote{\textbf{25:9} Término utilizado
  para una acción que dedica algo al Señor en términos de destrucción.
  Véase, por ejemplo, Josué 6:17.} Voy a destruirte totalmente, y la
gente se horrorizará de lo que te ha ocurrido y se burlará de ti.
\bibleverse{10} También pondré fin a los sonidos alegres de la
celebración y a las voces felices de los novios. No habrá ruido de las
piedras de molino que se usen; no se encenderán las lámparas.
\footnote{\textbf{25:10} Jer 16,9} \bibleverse{11} Todo este país se
convertirá en un páramo vacío, y Judá y estas otras naciones servirán al
rey de Babilonia durante setenta años. \footnote{\textbf{25:11} Jer
  29,10; 2Cró 36,21; Esd 1,1; Dan 9,2}

\bibleverse{12} Sin embargo, cuando terminen estos setenta años, voy a
castigar al rey de Babilonia y a esa nación, el país de Babilonia, por
su pecado, declara el Señor. Los destruiré por completo. \bibleverse{13}
Haré caer sobre ese país todo lo que he amenazado hacer, todo lo que
está escrito en este libro que Jeremías profetizó contra todas las
diferentes naciones. \bibleverse{14} Muchas naciones y reyes poderosos
se harán esclavos de ellos, de los babilonios, y yo les pagaré el mal
que han hecho.

\hypertarget{la-copa-de-la-ira-y-la-espada-de-dios-para-todos-los-pueblos}{%
\subsection{La copa de la ira y la espada de Dios para todos los
pueblos}\label{la-copa-de-la-ira-y-la-espada-de-dios-para-todos-los-pueblos}}

\bibleverse{15} Esto es lo que me dijo el Señor, el Dios de Israel: Toma
esta copa que te entrego. Contiene el vino de mi ira. Debes hacer que
todas las naciones que te envío beban de ella. \footnote{\textbf{25:15}
  Jer 51,7; Is 51,17; Apoc 14,10} \bibleverse{16} Beberán y tropezarán y
enloquecerán a causa de la guerra que traen los ejércitos que envío a
atacarlos.\footnote{\textbf{25:16} Literalmente, ``a causa de la espada
  que envío entre ellos''.}

\bibleverse{17} Tomé la copa que el Señor me entregó e hice beber de
ella a todas las naciones que me envió: \bibleverse{18} a Jerusalén y a
las ciudades de Judá, a sus reyes y a sus funcionarios, destruyéndolos
de tal manera que la gente se horrorizaba de lo que les ocurría y se
burlaba de ellos y los maldecía (y todavía hoy están así);
\bibleverse{19} al Faraón, rey de Egipto, y a sus funcionarios,
dirigentes, a todo su pueblo \bibleverse{20} y a todos los extranjeros
que vivían allí; a todos los reyes del país de Uz; a todos los reyes de
los filisteos: Ascalón, Gaza, Ecrón y lo que queda de Asdod;
\bibleverse{21} a Edom, Moab y los amonitas; \bibleverse{22} a todos los
reyes de Tiro y Sidón; a los reyes de la costa del mar Mediterráneo;
\bibleverse{23} a Dedán, Tema, Buz y a todos los que se recortan el pelo
a los lados de la cabeza \bibleverse{24} a todos los reyes de Arabia, y
a todos los reyes de las diferentes tribus que viven en el desierto;
\bibleverse{25} a todos los reyes de Zimri, Elam y Media;
\bibleverse{26} a todos los reyes del norte; de hecho, a todos los
reinos de la tierra, ya sean cercanos o lejanos, uno tras otro. Después
de todos ellos, el rey de Babilonia\footnote{\textbf{25:26}
  ``Babilonia'': literalmente, ``Sesac'', nombre en clave de Babilonia.}
la beberán también.

\bibleverse{27} Diles que esto es lo que dice el Señor Todopoderoso, el
Dios de Israel: Bebed, emborrachaos y vomitad. A causa de la guerra
morirán, cayendo para no volver a levantarse. \bibleverse{28} Si se
niegan a tomar la copa y a beber de ella, diles que esto es lo que dice
el Señor Todopoderoso: No pueden evitar beberlo; tienen que hacerlo.
\bibleverse{29} ¿No ves que estoy a punto de hacer caer el desastre
sobre mi propia ciudad, así que realmente crees que no serás castigado
también? No quedarás impune, porque estoy trayendo la guerra a todo el
mundo en la tierra, declara el Señor Todopoderoso. \footnote{\textbf{25:29}
  Jer 49,12; 1Pe 4,17}

\hypertarget{dios-aparece-en-el-juicio-final-aniquilaciuxf3n-de-los-pueblos}{%
\subsection{Dios aparece en el Juicio Final; Aniquilación de los
pueblos}\label{dios-aparece-en-el-juicio-final-aniquilaciuxf3n-de-los-pueblos}}

\bibleverse{30} Da todo este mensaje como una profecía contra ellos.
Diles: El Señor tronará desde lo alto. Tronará con fuerza desde el lugar
santo donde vive. Dará un gran rugido contra los rediles.\footnote{\textbf{25:30}
  ``Rediles'': o ``casas de los pastores''. También el versículo 37.}
Dará un fuerte grito como de gente que pisa las uvas, asustando a todos
los que viven en la tierra. \footnote{\textbf{25:30} Jl 4,16; Am 1,2; Os
  11,10} \bibleverse{31} El sonido llegará a todos los rincones de la
tierra porque el Señor está acusando a las naciones. Está juzgando a
todos, ejecutando a los malvados, declara el Señor.

\bibleverse{32} Esto es lo que dice el Señor Todopoderoso: ¡Cuidado! El
desastre está cayendo sobre una nación tras otra; una inmensa tormenta
se está formando en la distancia. \bibleverse{33} Los muertos por el
Señor en ese momento cubrirán la tierra de un extremo a otro. Nadie los
llorará, ni los recogerá, ni los enterrará. Serán como montones de
estiércol tirados en el suelo. \footnote{\textbf{25:33} Jer 7,33}
\bibleverse{34} ¡Griten y lloren, pastores! Arrástrense por el suelo con
luto, jefes del rebaño. Ha llegado la hora de que los maten; caerán
destrozados como la mejor cerámica. \footnote{\textbf{25:34} Jer 23,1}
\bibleverse{35} Los pastores no podrán huir; los jefes del rebaño no
escaparán. \bibleverse{36} Escuchen los gritos de los pastores, el
llanto de los jefes del rebaño, porque el Señor está destruyendo sus
pastos. \bibleverse{37} Los pacíficos apriscos han sido arruinados por
la feroz ira del Señor. \bibleverse{38} El Señor ha salido de su guarida
como un león, porque su país ha sido devastado por los ejércitos
invasores, y a causa de la feroz ira del Señor.\footnote{\textbf{25:38}
  Jer 4,7}

\hypertarget{introducciuxf3n-indicaciuxf3n-del-contenido-principal-del-discurso-captura-de-jeremuxedas}{%
\subsection{Introducción; Indicación del contenido principal del
discurso; Captura de
Jeremías}\label{introducciuxf3n-indicaciuxf3n-del-contenido-principal-del-discurso-captura-de-jeremuxedas}}

\hypertarget{section-25}{%
\section{26}\label{section-25}}

\bibleverse{1} Este mensaje vino del Señor al comienzo del reinado de
Joaquín, hijo de Josías, rey de Judá, \bibleverse{2} Esto es lo que dice
el Señor: Ve y ponte de pie en el patio del Templo y entrega todo el
mensaje que te he ordenado dar a todos los que vengan de todos los
pueblos de Judá a adorar allí. No omitas ni una sola palabra.
\bibleverse{3} Tal vez te escuchen, y cada uno de ellos renuncie a sus
malas costumbres, para que yo no tenga que llevar a cabo el desastre que
pienso hacer caer sobre ellos a causa de las cosas malas que hacen.
\bibleverse{4} Diles que esto es lo que dice el Señor: Si no me escuchan
y no siguen mi ley, que yo les he dado, \bibleverse{5} y si no escuchan
los mensajes de mis siervos los profetas -los he enviado a ustedes una y
otra vez, pero se negaron a escuchar- \footnote{\textbf{26:5} Jer 25,4}
\bibleverse{6} entonces destruiré este Templo como lo hice con Silo, y
haré de esta ciudad una palabra de maldición usada por todos en la
tierra. \footnote{\textbf{26:6} Jer 7,12-14; 1Sam 4,4; 1Sam 4,12}

\bibleverse{7} Los sacerdotes, los profetas y todo el pueblo escucharon
a Jeremías pronunciar este mensaje en el Templo del Señor.
\bibleverse{8} En cuanto terminó de decir todo lo que el Señor le había
ordenado, los sacerdotes y profetas y todo el pueblo lo agarraron,
gritando: ``¡Morirás por esto! \bibleverse{9} ¿Cómo te atreves a hablar
en nombre del Señor aquí en el Templo y a declarar que será destruido
como Silo, y que esta ciudad quedará vacía y abandonada?'' Todos se
agolparon alrededor de Jeremías amenazándolo en el Templo del Señor.

\hypertarget{el-juicio-ante-los-superiores-la-absoluciuxf3n-de-jeremuxedas-la-intercesiuxf3n-de-algunos-ancianos-por-uxe9l}{%
\subsection{El juicio ante los superiores; La absolución de Jeremías; la
intercesión de algunos ancianos por
él}\label{el-juicio-ante-los-superiores-la-absoluciuxf3n-de-jeremuxedas-la-intercesiuxf3n-de-algunos-ancianos-por-uxe9l}}

\bibleverse{10} Cuando los dirigentes de Judá se enteraron de lo
sucedido, vinieron del palacio del rey al Templo del Señor y se sentaron
a la entrada de la Puerta Nueva del Templo para juzgar el caso.
\bibleverse{11} Los sacerdotes y los profetas se quejaron ante los
dirigentes y todo el pueblo: ``Este hombre merece la pena de muerte
porque ha cometido traición\footnote{\textbf{26:11} ``Cometido
  traición'': Añadido para mayor claridad.} profetizando contra esta
ciudad. Ustedes mismos lo oyeron''. \footnote{\textbf{26:11} Hech 6,13}

\bibleverse{12} Jeremías se dirigió a todos los dirigentes y a todo el
pueblo, diciendo: ``El Señor me ha enviado a pronunciar cada una de las
palabras de esta profecía contra este Templo, como ustedes han oído.
\bibleverse{13} Así que cambien su forma de actuar y hagan lo que el
Señor, su Dios, les diga, para que no tenga que llevar a cabo el
desastre que ha anunciado que hará caer sobre ustedes. \bibleverse{14}
Por lo que a mí respecta, estoy en tus manos; haz conmigo lo que te
parezca bueno y correcto. \bibleverse{15} Pero tengan cuidado, porque
deben saber que si me matan, se harán culpables de asesinato a ustedes
mismos, a esta ciudad y a todos los que viven aquí, porque es cierto que
el Señor me envió a decirles todo lo que dijo''.

\bibleverse{16} Entonces los dirigentes y todo el pueblo dijeron a los
sacerdotes y a los profetas: ``Este hombre no merece la pena de muerte,
porque hablaba en nombre del Señor, nuestro Dios''.

\bibleverse{17} Algunos de los ancianos del país se levantaron y se
dirigieron a todos los allí reunidos \bibleverse{18} ``Miqueas de
Moreset profetizó durante el reinado de Ezequías, rey de Judá. Dijo a
todo el pueblo de Judá que esto es lo que dice el Señor Todopoderoso:
`Sión se convertirá en un campo arado; Jerusalén acabará siendo un
montón de escombros, y el monte del Templo estará cubierto de
árboles'.\footnote{\textbf{26:18} Véase Miqueas 3:12.} \footnote{\textbf{26:18}
  Miq 3,12} \bibleverse{19} ``¿Acaso Ezequías, rey de Judá, o cualquier
otra persona del país, lo hizo matar? ¿No respetó Ezequías al Señor y le
suplicó? ¿No cambió el Señor de opinión sobre el desastre que había
anunciado contra ellos? Pero nosotros estamos a punto de provocar un
gran desastre''. \footnote{\textbf{26:19} Jer 18,8}

\hypertarget{el-siniestro-destino-del-profeta-uruxedas}{%
\subsection{El siniestro destino del profeta
Urías}\label{el-siniestro-destino-del-profeta-uruxedas}}

\bibleverse{20} Por aquel entonces había otro hombre que profetizaba en
nombre del Señor, Urías, hijo de Semaías, de Quiriat-jearim. Profetizó
contra Jerusalén y contra el país igual que Jeremías. \bibleverse{21} El
rey Joaquín y todos sus oficiales militares y funcionarios oyeron lo que
decía, y el rey quiso ejecutarlo. Pero cuando Urías se enteró, se asustó
y huyó a Egipto. \bibleverse{22} Pero el rey Joaquín envió a Elnatán,
hijo de Acbor, junto con otros. \bibleverse{23} Ellos trajeron a Urías
de Egipto y lo llevaron ante el rey Joaquín. El rey lo mató con una
espada y mandó arrojar su cuerpo al cementerio público.

\bibleverse{24} Sin embargo, Ahicam, hijo de Safán, se puso del lado de
Jeremías para que no lo entregaran al pueblo para que lo
mataran.\footnote{\textbf{26:24} 2Re 22,12}

\hypertarget{jeremuxedas-yugo-en-el-cuello-advierte-a-los-enviados-de-algunos-estados-extranjeros}{%
\subsection{Jeremías, yugo en el cuello, advierte a los enviados de
algunos estados
extranjeros}\label{jeremuxedas-yugo-en-el-cuello-advierte-a-los-enviados-de-algunos-estados-extranjeros}}

\hypertarget{section-26}{%
\section{27}\label{section-26}}

\bibleverse{1} Este mensaje llegó a Jeremías de parte del Señor al
comienzo del reinado de Sedequías,\footnote{\textbf{27:1} La mayoría de
  los manuscritos hebreos tienen el nombre de ``Joaquín'' en lugar de
  Sedequías, pero esto no encaja con el resto del capítulo (Sedequías
  está claramente identificado en los versículos 3 y 12).} hijo de
Josías, rey de Judá. \bibleverse{2} Esto es lo que me dijo el Señor:
Hazte un arnés y un yugo y átalo a tu cuello \bibleverse{3} Envía un
mensaje a los reyes de Edom, Moab, Amón, Tiro y Sidón por medio de los
embajadores que han venido a Jerusalén a ver a Sedequías, rey de Judá.
\footnote{\textbf{27:3} Jer 25,21-22} \bibleverse{4} Dales esta orden
del Señor Todopoderoso, el Dios de Israel, para que la transmitan a sus
señores: \bibleverse{5} Por mi fuerza y mi poder creador hice la tierra
y los seres humanos y los animales que la habitan, y la entrego a los
que son rectos a mis ojos. \bibleverse{6} Ahora he puesto a mi siervo
Nabucodonosor, rey de Babilonia, a cargo de todos estos países. Incluso
le he dado el control de los animales salvajes. \bibleverse{7} Todas las
naciones le servirán a él, a su hijo y a su nieto, hasta el momento en
que su propia tierra quede bajo el control de otras naciones y de reyes
poderosos. \footnote{\textbf{27:7} Jer 25,12}

\bibleverse{8} Cualquier nación o reino que no sirva a Nabucodonosor,
rey de Babilonia, y no se someta a él\footnote{\textbf{27:8} ``Que se
  someta a él'': literalmente, ``pongan su cuello bajo su yugo''.
  También el versículo 11.} Castigaré a esa nación con guerra, hambre y
peste, declara el Señor, hasta que deje que Nabucodonosor la destruya
por completo. \bibleverse{9} No escuches a tus profetas, a tus adivinos,
a tus intérpretes de sueños, a tus médiums o a tus magos cuando te
digan: ``No servirás al rey de Babilonia''. \bibleverse{10} Ellos te
están profetizando una mentira que te llevará a la expulsión de tu país.
Te expulsaré y morirás. \bibleverse{11} Pero a la nación que se someta
al rey de Babilonia y le sirva, la dejaré en su propia tierra, para que
la cultive y viva en ella, declara el Señor.

\hypertarget{jeremuxedas-dirige-la-misma-advertencia-al-rey-juduxedo-sedechuxeeas}{%
\subsection{Jeremías dirige la misma advertencia al rey judío
Sedechîas}\label{jeremuxedas-dirige-la-misma-advertencia-al-rey-juduxedo-sedechuxeeas}}

\bibleverse{12} El mismo mensaje le di a Sedequías, rey de Judá:
Sométete al rey de Babilonia; sírvele a él y a su pueblo, y vive.
\bibleverse{13} ¿Por qué han de morir tú y tu pueblo a causa de la
guerra, el hambre y la peste, como el Señor ha dicho que traería contra
cualquier nación que no sirva al rey de Babilonia? \bibleverse{14} No
escuches los mensajes de los profetas que dicen: ``No servirás al rey de
Babilonia'', porque te están profetizando una mentira. \footnote{\textbf{27:14}
  Jer 27,9} \bibleverse{15} Yo no los envié, declara el Señor, y sin
embargo están dando falsas profecías en mi nombre. Por eso los expulsaré
y morirán, ustedes y los profetas que les profetizan.

\hypertarget{la-advertencia-de-jeremuxedas-a-los-sacerdotes-y-al-pueblo}{%
\subsection{La advertencia de Jeremías a los sacerdotes y al
pueblo}\label{la-advertencia-de-jeremuxedas-a-los-sacerdotes-y-al-pueblo}}

\bibleverse{16} Entonces dije a los sacerdotes y a todo el pueblo: Esto
es lo que dice el Señor: No escuchen las palabras de sus profetas que
les profetizan diciendo: ``¡Miren! Los objetos del Templo del Señor
volverán pronto de Babilonia''. Te están profetizando una mentira.
\bibleverse{17} No los escuches. Sirvan al rey de Babilonia y vivan.
¿Por qué debe ser destruida esta ciudad? \bibleverse{18} Si realmente
son profetas y tienen la palabra del Señor consigo, deberían estar
suplicando ahora al Señor Todopoderoso que lo que queda en el Templo del
Señor, en el palacio del rey de Judá y en Jerusalén, no sea llevado a
Babilonia. \bibleverse{19} Esto dice el Señor Todopoderoso sobre las
columnas, el mar de bronce, las bases y el resto de los objetos que
quedan en Jerusalén: \footnote{\textbf{27:19} Jer 52,17} \bibleverse{20}
todo lo que Nabucodonosor, rey de Babilonia, no se llevó cuando tomó a
Joaquín \footnote{\textbf{27:20} Aquí aparece como Jeconías.} hijo de
Joacim, rey de Judá, al exilio de Jerusalén a Babilonia, junto con todos
los nobles de Judá y de Jerusalén. \footnote{\textbf{27:20} 2Re 24,14-15}
\bibleverse{21} De nuevo, esto es lo que dice el Señor Todopoderoso, el
Dios de Israel, sobre los objetos que quedaron en el Templo del Señor,
en el palacio del rey de Judá y en Jerusalén: \bibleverse{22} Serán
llevados a Babilonia y se quedarán allí hasta el momento en que vuelva a
verlos, declara el Señor. Sólo entonces los traeré de vuelta para que
vuelvan a estar en Jerusalén.\footnote{\textbf{27:22} 2Cró 36,22; Esd
  1,7-11}

\hypertarget{el-dicho-de-hananya-y-la-respuesta-de-jeremuxedas}{%
\subsection{El dicho de Hananya y la respuesta de
Jeremías}\label{el-dicho-de-hananya-y-la-respuesta-de-jeremuxedas}}

\hypertarget{section-27}{%
\section{28}\label{section-27}}

\bibleverse{1} Esto es lo que sucedió al principio del reinado del rey
Sedequías de Judá, en el quinto mes de ese mismo año, el cuarto año. El
profeta Ananías, hijo de Azzur, que era de Gabaón, me dijo en el Templo
del Señor, delante de los sacerdotes y de todo el pueblo: \bibleverse{2}
``Esto es lo que dice el Señor Todopoderoso, el Dios de Israel: He roto
el yugo del rey de Babilonia. \bibleverse{3} Antes de que pasen dos años
voy a traer de vuelta a Jerusalén todos los objetos del Templo que
Nabucodonosor, rey de Babilonia, quitó y se llevó a Babilonia.
\bibleverse{4} También haré volver a Jerusalén a Joaquín, hijo de
Joaquín, rey de Judá, junto con todos los exiliados de Judá que fueron
llevados a Babilonia, declara el Señor, porque voy a romper el yugo del
rey de Babilonia''. \footnote{\textbf{28:4} Jer 27,20}

\bibleverse{5} Entonces el profeta Jeremías respondió al profeta Ananías
delante de los sacerdotes y de todo el pueblo que estaba de pie en el
Templo del Señor. \bibleverse{6} ``¡Amén!'', dijo Jeremías. ``¡Deseo que
el Señor haga precisamente eso! Ojalá el Señor cumpliera tus palabras
proféticas y trajera de vuelta a Jerusalén los objetos del Templo y a
todos los exiliados de Babilonia. \bibleverse{7} ``Pero aun así, presten
atención a este mensaje que les voy a decir a ustedes y a todos los
presentes. \bibleverse{8} Los profetas de antaño que vinieron antes que
tú y yo profetizaron guerra, desastre y enfermedad contra muchos países
y grandes reinos. \bibleverse{9} Cuando se trata de un profeta que
profetiza la paz, vean si sus profecías se hacen realidad. Sólo eso
probará que son realmente enviados del Señor''.

\hypertarget{las-acciones-violentas-de-hananya-y-el-veredicto-divino-de-jeremuxedas-sentencia-de-muerte-sobre-uxe9l}{%
\subsection{Las acciones violentas de Hananya y el veredicto divino de
Jeremías (sentencia de muerte) sobre
él}\label{las-acciones-violentas-de-hananya-y-el-veredicto-divino-de-jeremuxedas-sentencia-de-muerte-sobre-uxe9l}}

\bibleverse{10} Entonces el profeta Hananías quitó el yugo del cuello
del profeta Jeremías y lo rompió. \bibleverse{11} Hananías anunció
delante de todos: ``Esto es lo que dice el Señor: Así, antes de que
pasen dos años, romperé el yugo de Nabucodonosor, rey de Babilonia, del
cuello de todas las naciones''. El profeta Jeremías se fue. \footnote{\textbf{28:11}
  Jer 28,3}

\bibleverse{12} Sin embargo, justo después de que el profeta Hananías
rompiera el yugo de su cuello, llegó a Jeremías un mensaje del Señor:
\bibleverse{13} ``Ve y dile a Hananías que esto es lo que dice el Señor:
Has roto un yugo de madera, pero lo has sustituido por un yugo de
hierro. \bibleverse{14} Esto es lo que dice el Señor Todopoderoso, el
Dios de Israel: He atado yugos de hierro al cuello de todas estas
naciones para obligarlas a servir a Nabucodonosor, rey de Babilonia, y
le servirán. Incluso le he dado el control sobre los animales
salvajes''.

\bibleverse{15} Entonces el profeta Jeremías le dijo al profeta
Hananías: ``¡Escucha esto, Hananías! El Señor no te envió a ti, pero tú
has convencido a este pueblo de creer en una mentira. \bibleverse{16}
Así que esto es lo que dice el Señor: Voy a deshacerme de ti de la
tierra. Morirás este año porque has promovido la rebelión contra el
Señor''. \footnote{\textbf{28:16} Jer 23,14; Jer 29,32}

\bibleverse{17} El profeta Ananías murió en el séptimo mes de ese mismo
año.

\hypertarget{la-carta-de-jeremuxedas-a-los-juduxedos-encarcelados-en-babilonia-mal-resultado-de-dos-falsos-profetas-en-babilonia}{%
\subsection{La carta de Jeremías a los judíos encarcelados en Babilonia;
mal resultado de dos falsos profetas en
Babilonia}\label{la-carta-de-jeremuxedas-a-los-juduxedos-encarcelados-en-babilonia-mal-resultado-de-dos-falsos-profetas-en-babilonia}}

\hypertarget{section-28}{%
\section{29}\label{section-28}}

\bibleverse{1} El profeta Jeremías escribió esta carta y la envió desde
Jerusalén a los ancianos que habían quedado entre los exiliados, a los
sacerdotes, a los profetas y a todos los demás que habían sido
desterrados de Jerusalén a Babilonia por Nabucodonosor. \bibleverse{2}
Esto sucedió después de que el rey Joaquín, la reina madre, los
funcionarios de la corte, los dirigentes de Judá y Jerusalén, los
artesanos y los metalúrgicos habían sido desterrados de Jerusalén.
\bibleverse{3} Elasá, hijo de Safán, y Gemarías, hijo de Hilcías,
llevaron la carta cuando Sedequías, rey de Judá, los envió al rey
Nabucodonosor en Babilonia. En la carta Jeremías escribió

\hypertarget{texto-de-la-carta-de-jeremuxedas}{%
\subsection{Texto de la carta de
Jeremías}\label{texto-de-la-carta-de-jeremuxedas}}

\bibleverse{4} Esto es lo que el Señor Todopoderoso, el Dios de Israel,
dice a todos los exiliados que fueron llevados de Jerusalén a Babilonia:
\bibleverse{5} Construyan allí casas para vivir. Planten jardines y
cultiven alimentos para comer. \bibleverse{6} Cásense y tengan hijos.
Hagan arreglos para que sus hijos se casen y puedan tener hijos también.
Aumenten en número, no disminuyan. \bibleverse{7} Ayudad a hacer más
próspera la ciudad a la que os he desterrado. Ruega al Señor por ella,
ya que, según prospere, tú también lo harás. \bibleverse{8} Esto es lo
que dice el Señor Todopoderoso, el Dios de Israel: No te dejes engañar
por tus profetas y adivinos, y no escuches ningún sueño que te
interpreten. \footnote{\textbf{29:8} Jer 14,14} \bibleverse{9} Ellos les
están profetizando mentiras en mi nombre; yo no los he enviado, declara
el Señor. \bibleverse{10} Esto es lo que dice el Señor: Cuando terminen
los setenta años de exilio en Babilonia, me ocuparé de ustedes y
cumpliré mi promesa de hacerlos regresar a Jerusalén. \bibleverse{11} Yo
sé lo que pienso hacer por ustedes, declara el Señor. Planeo cosas
buenas para ti y no malas. Voy a darte un futuro y una esperanza.
\bibleverse{12} Entonces pedirás mi ayuda, vendrás a orar a mí, y yo te
responderé. \bibleverse{13} Me buscarás y me encontrarás cuando te
empeñes en buscarme. \footnote{\textbf{29:13} Deut 4,29; Is 55,6}
\bibleverse{14} Dejaré que me encuentres, declara el Señor. Acabaré con
tu cautiverio, reuniéndote de todas las naciones y lugares donde te
dispersé, declara el Señor. Los haré volver a casa, al lugar desde donde
os envié al exilio. \footnote{\textbf{29:14} Sal 126,4}

\hypertarget{la-triste-situaciuxf3n-de-la-gente-que-se-queduxf3-en-jerusaluxe9n-insultar-a-dos-profetas-mentirosos-aduxfalteros-en-babilonia}{%
\subsection{La triste situación de la gente que se quedó en Jerusalén;
Insultar a dos profetas mentirosos adúlteros en
Babilonia}\label{la-triste-situaciuxf3n-de-la-gente-que-se-queduxf3-en-jerusaluxe9n-insultar-a-dos-profetas-mentirosos-aduxfalteros-en-babilonia}}

\bibleverse{15} Pero si ustedes argumentan: ``El Señor nos ha provisto
de profetas en Babilonia'', \bibleverse{16} esto es lo que dice el Señor
sobre el rey que se sienta en el trono de David y todos los que quedan
en Jerusalén, tus conciudadanos que no fueron llevados contigo al
exilio. \bibleverse{17} Esto es lo que dice el Señor Todopoderoso: Voy a
enviar contra ellos guerra, hambre y enfermedad. Los haré como higos
podridos, tan malos que no se pueden comer. \footnote{\textbf{29:17} Jer
  24,8} \bibleverse{18} Los perseguiré con guerra, hambre y enfermedad.
Haré que todos los reinos de la tierra se horroricen de ellos. Se
convertirán en una palabra de maldición, totalmente arruinados, gente de
la que se burlarán y criticarán entre todas las naciones donde los
disperse. \footnote{\textbf{29:18} Jer 24,9-10} \bibleverse{19} Voy a
hacer esto porque no han obedecido a mis palabras, declara el Señor, que
les envié una y otra vez por medio de mis siervos los profetas. Ustedes,
los exiliados, tampoco me han obedecido, declara el Señor. \footnote{\textbf{29:19}
  Jer 25,4}

\bibleverse{20} Así que escuchen la palabra del Señor, todos los
exiliados que envié de Jerusalén a Babilonia. \footnote{\textbf{29:20}
  Jer 29,4} \bibleverse{21} Esto es lo que dice el Señor Todopoderoso,
el Dios de Israel, sobre Acab hijo de Colaías y Sedequías hijo de
Maasías, que les están profetizando mentiras en mi nombre. Voy a
entregarlos a Nabucodonosor, rey de Babilonia, y él los matará ante tus
ojos. \footnote{\textbf{29:21} Jer 29,8} \bibleverse{22} Por lo que les
suceda, todos los exiliados de Judá en Babilonia maldecirán a los demás
de esta manera ``¡Que el Señor los trate como a Sedequías y Acab,
quemados vivos por el rey de Babilonia!'' \bibleverse{23} Hicieron cosas
escandalosas en Israel: cometieron adulterio con las esposas de sus
vecinos y dijeron mentiras en mi nombre. Yo no les dije que dijeran
nada. Yo soy el que sabe lo que hicieron, y puedo dar testimonio de
ello, declara el Señor.

\hypertarget{la-queja-de-semaja-sobre-la-carta-de-jeremuxedas-al-sacerdocio-en-jerusaluxe9n-el-castigo-amenazado-por-dios}{%
\subsection{La queja de Semaja sobre la carta de Jeremías al sacerdocio
en Jerusalén; el castigo amenazado por
dios}\label{la-queja-de-semaja-sobre-la-carta-de-jeremuxedas-al-sacerdocio-en-jerusaluxe9n-el-castigo-amenazado-por-dios}}

\bibleverse{24} Dile a Semaías el nehelamita \bibleverse{25} que esto es
lo que dice el Señor Todopoderoso, el Dios de Israel: Con tu propia
autoridad enviaste cartas a todo el pueblo de Jerusalén, al sacerdote
Sofonías, hijo de Maasías, y a todos los sacerdotes, diciendo:
\bibleverse{26} ``Sofonías,\footnote{\textbf{29:26} Nombre añadido para
  mayor claridad.} el Señor te ha elegido como sacerdote para reemplazar
a Joiada, para estar a cargo del Templo del Señor. En calidad de tal,
estás obligado a poner en el cepo y en los grilletes a cualquier loco
que pretenda ser profeta. \footnote{\textbf{29:26} Os 9,7}
\bibleverse{27} Entonces, ¿por qué no has castigado a Jeremías de
Anatot, que dice ser profeta entre ustedes? \bibleverse{28} Debiste
haberlo hecho porque\footnote{\textbf{29:28} ``Debiste haberlo hecho
  porque'': suministrado para mayor claridad.} nos ha enviado una carta
aquí en Babilonia, diciendo: `El exilio durará mucho tiempo. Así que
construyan allí casas para vivir. Planten jardines y cultiven alimentos
para comer'\,''.

\bibleverse{29} Sin embargo, el sacerdote Sofonías leyó esta carta al
profeta Jeremías. \bibleverse{30} Entonces el Señor le dijo a Jeremías
\bibleverse{31} Envía este mensaje a todos los exiliados: Esto es lo que
dice el Señor sobre Semaías el nehelamita. Ya que Semaías les ha
profetizado, aunque yo no lo envié, y los ha convencido de creer en una
mentira, \bibleverse{32} esto es lo que dice el Señor: Voy a castigar a
Semaías el nehelamita y a sus descendientes. No le quedará familia en
este pueblo, y no experimentará las cosas buenas que voy a hacer por mi
pueblo, declara el Señor, porque ha promovido la rebelión contra el
Señor.

\hypertarget{introducciuxf3n-jeremuxedas-debe-registrar-todas-las-palabras-de-dios-que-le-han-llegado}{%
\subsection{Introducción: Jeremías debe registrar todas las palabras de
Dios que le han
llegado}\label{introducciuxf3n-jeremuxedas-debe-registrar-todas-las-palabras-de-dios-que-le-han-llegado}}

\hypertarget{section-29}{%
\section{30}\label{section-29}}

\bibleverse{1} Este es el mensaje que llegó a Jeremías de parte del
Señor: \bibleverse{2} Esto es lo que dice el Señor, el Dios de Israel:
Escribe en un libro todo lo que te he dicho. \bibleverse{3} Mira, se
acerca el tiempo, declara el Señor, en que haré volver a mi pueblo
Israel y Judá del cautiverio, declara el Señor. Los haré volver al país
que les di a sus antepasados, y volverán a ser dueños de él. \footnote{\textbf{30:3}
  Jer 29,14}

\hypertarget{el-terrible-punto-de-inflexiuxf3n}{%
\subsection{El terrible punto de
inflexión}\label{el-terrible-punto-de-inflexiuxf3n}}

\bibleverse{4} Esto es lo que ha dicho el Señor sobre Israel y Judá.
\bibleverse{5} Esto es lo que dice el Señor: Oigan los gritos de pánico,
gritos de miedo, no de paz. \bibleverse{6} ¡Piensa en ello! ¿Pueden los
hombres dar a luz? No.~Entonces, ¿por qué veo a todos los hombres
sujetándose el vientre con las manos como una mujer de parto? ¿Por qué
todos los rostros están blancos como una sábana? \bibleverse{7} ¡Qué día
tan terrible será, un día como nunca antes! Este es el tiempo de la
angustia para los descendientes de Jacob, pero serán rescatados de ella.
\footnote{\textbf{30:7} Jl 2,11; Sof 1,15}

\hypertarget{redenciuxf3n-de-problemas}{%
\subsection{Redención de problemas}\label{redenciuxf3n-de-problemas}}

\bibleverse{8} En ese día, declara el Señor Todopoderoso, romperé el
yugo de sus cuellos y arrancaré sus cadenas. Los extranjeros ya no los
harán esclavos. \footnote{\textbf{30:8} Jer 27,12} \bibleverse{9}
Servirán al Señor, su Dios, y a su rey, el descendiente de David que yo
les daré. \footnote{\textbf{30:9} Jer 23,5; Ezeq 34,23} \bibleverse{10}
Por lo que a ti respecta, siervo mío Jacob, no temas, declara el Señor,
Israel, no te desanimes. Prometo salvarte de tus lejanos lugares de
exilio, a tus descendientes de los países donde están cautivos. Los
descendientes de Jacob volverán a casa con una vida tranquila y cómoda,
libres de cualquier amenaza. \footnote{\textbf{30:10} Jer 46,27; Is 44,2}
\bibleverse{11} Yo estoy con ustedes y los salvaré, declara el Señor.
Aunque voy a destruir por completo a todas las naciones donde te he
dispersado, no te destruiré por completo. Sin embargo, te disciplinaré
como te mereces, y puedes estar seguro de que no te dejaré sin castigo.
\footnote{\textbf{30:11} Jer 10,24}

\hypertarget{la-cauxedda-de-israel-como-resultado-de-sus-pecados}{%
\subsection{La caída de Israel como resultado de sus
pecados}\label{la-cauxedda-de-israel-como-resultado-de-sus-pecados}}

\bibleverse{12} Esto es lo que dice el Señor: Tienes una herida que no
se puede curar, tienes una lesión terrible. \footnote{\textbf{30:12} Jer
  15,18} \bibleverse{13} No hay nadie que se ocupe de tu caso, no hay
cura para tus llagas, no hay curación para ti. \bibleverse{14} Todos tus
amantes se han olvidado de ti; ya no se molestan en buscarte, porque te
he golpeado como si fuera tu enemigo, la disciplina de una persona
cruel, por lo malvado que eres, por tus muchos pecados. \bibleverse{15}
¿Por qué lloras por tu herida? Tu dolor no se puede curar. Yo te hice
esto por lo malvado que eres, por tus muchos pecados.

\hypertarget{la-restauraciuxf3n-del-pueblo-y-el-pauxeds}{%
\subsection{La restauración del pueblo y el
país}\label{la-restauraciuxf3n-del-pueblo-y-el-pauxeds}}

\bibleverse{16} Aun así, todo el que te destruya será destruido. Todos
tus enemigos, hasta el último, serán enviados al exilio. Los que te
saquearon serán saqueados, y todos los que te robaron serán robados.
\footnote{\textbf{30:16} Is 33,1} \bibleverse{17} Pero yo te devolveré
la salud y sanaré tus heridas, declara el Señor, porque la gente dice
que has sido abandonada y que nadie se preocupa por ti, Sión.
\footnote{\textbf{30:17} Jer 33,6}

\bibleverse{18} Esto es lo que dice el Señor: Haré que los descendientes
de Jacob vuelvan a sus hogares y me apiadaré de sus familias. La ciudad
será reconstruida sobre sus ruinas, y el palacio volverá a estar en pie
donde debe estar. \footnote{\textbf{30:18} Jer 30,3} \bibleverse{19} La
gente cantará canciones de agradecimiento, sonidos de celebración.
Aumentaré su número, no disminuirá. Los honraré: no serán tratados como
insignificantes. \bibleverse{20} Sus hijos serán atendidos como antes.
Haré que su nación vuelva a ser fuerte, y castigaré a cualquiera que los
ataque. \bibleverse{21} Su líder será de su propio país, su gobernante
será elegido de entre ellos. Lo invitaré a acercarse a mí, y lo hará,
porque ¿se atreverá alguien a acercarse a mí sin que se lo pida? declara
el Señor. \bibleverse{22} Ustedes serán mi pueblo y yo seré su Dios.
\footnote{\textbf{30:22} Jer 24,7} \bibleverse{23} ¡Cuidado! El Señor ha
enviado una tormenta furiosa, un tornado que gira en torno a las cabezas
de los malvados. \footnote{\textbf{30:23} Jer 23,19}

\bibleverse{24} La ira del Señor no se desvanecerá hasta que termine de
hacer todo lo que quiere. Sólo entonces comprenderá realmente.

\hypertarget{el-encuentro-de-dios-e-israel-en-el-desierto-las-esperanzadas-palabras-de-saludo}{%
\subsection{El encuentro de Dios e Israel en el desierto; las
esperanzadas palabras de
saludo}\label{el-encuentro-de-dios-e-israel-en-el-desierto-las-esperanzadas-palabras-de-saludo}}

\hypertarget{section-30}{%
\section{31}\label{section-30}}

\bibleverse{1} En ese momento, yo seré el Dios de todas las familias de
Israel, y ellos serán mi pueblo, declara el Señor. \footnote{\textbf{31:1}
  Jer 31,33; Jer 24,7}

\bibleverse{2} Esto es lo que dice el Señor: Los israelitas que
sobreviven a la muerte por la espada fueron bendecidos por el Señor en
el desierto cuando buscaban la paz y la tranquilidad.

\bibleverse{3} Hace tiempo, el Señor vino y nos dijo: Mi amor por
ustedes durará para siempre. Los mantengo cerca de mí con mi amor
infinito. \bibleverse{4} Voy a reconstruirte, y así será. Serás
reconstruida, Virgen Israel. Volverás a coger tus panderetas y saldrás a
bailar con alegría. \bibleverse{5} Volverás a plantar viñedos en las
colinas de Samaria; los que planten y disfruten de las uvas.
\bibleverse{6} Se acerca un día en que los vigías gritarán desde las
colinas de Efraín: ``¡Vamos, subamos a Sión para adorar al Señor,
nuestro Dios!''

\hypertarget{el-regreso-a-casa}{%
\subsection{El regreso a casa}\label{el-regreso-a-casa}}

\bibleverse{7} Esto es lo que dice el Señor: ¡Canten con alegría por los
descendientes de Jacob; griten por la más grande de todas las naciones!
¡Que todo el mundo lo sepa! Alaben y griten: ``¡Señor, salva a tu
pueblo, a los que quedan de Israel!'' \bibleverse{8} Estén atentos,
porque los haré volver del país del norte y los reuniré de los confines
de la tierra. Todos volverán, incluso los ciegos y los cojos, las
mujeres embarazadas, incluso las madres que den a luz, una gran reunión
que volverá a casa, \bibleverse{9} Volverán con lágrimas en los ojos, y
estarán orando mientras los llevo a casa. Los guiaré junto a corrientes
de agua, por caminos llanos donde no tropezarán. Porque yo soy el Padre
de Israel; Efraín\footnote{\textbf{31:9} Efraín no era literalmente el
  primogénito, sino que se utiliza como una descripción más amplia de
  Israel. Más que al orden de nacimiento, el primogénito se refiere a
  los derechos y privilegios relacionados con este estatus.} es mi
primogénito. \footnote{\textbf{31:9} 2Cor 6,18}

\hypertarget{en-la-patria}{%
\subsection{En la patria}\label{en-la-patria}}

\bibleverse{10} Escuchen, naciones, lo que el Señor tiene que decir, y
háganlo saber a otros en países lejanos: El Señor, que dispersó a
Israel, lo reunirá y lo mantendrá a salvo, como un pastor cuida de su
rebaño. \bibleverse{11} El Señor ha redimido a los descendientes de
Jacob y los ha rescatado de sus enemigos que los habían derrotado.
\bibleverse{12} Volverán y celebrarán con gritos de alegría en el monte
Sión; sus rostros resplandecerán ante los maravillosos regalos del
Señor: el grano, el vino nuevo y el aceite de oliva, y las crías de sus
rebaños y manadas. Su vida será como un jardín bien regado; y no
volverán a deprimirse. \bibleverse{13} Las muchachas bailarán en la
celebración; los jóvenes y los ancianos también se unirán. Convertiré su
dolor en alegría, y los consolaré y cambiaré su tristeza en felicidad.
\bibleverse{14} Daré a mis sacerdotes todo lo que necesitan y más, y mi
pueblo estará más que satisfecho de mi bondad para con ellos, declara el
Señor.

\hypertarget{raquel-llora-por-sus-hijos-y-dios-la-consuela}{%
\subsection{Raquel llora por sus hijos y Dios la
consuela}\label{raquel-llora-por-sus-hijos-y-dios-la-consuela}}

\bibleverse{15} Esto es lo que dice el Señor: El sonido de un terrible
llanto y de un lamento se oye en Ramá. Es Raquel que llora por sus
hijos. Están muertos, y ella no puede ser consolada \footnote{\textbf{31:15}
  Mat 2,18}

\bibleverse{16} Esto es lo que dice el Señor: No llores más, no llores
más, porque vas a ser recompensada por lo que has hecho, declara el
Señor. Tus hijos volverán del país de tus enemigos. \bibleverse{17} Así
podrás tener esperanza en el futuro, declara el Señor. Tus hijos
volverán a su país.

\hypertarget{el-arrepentimiento-de-efrauxedn-y-la-gracia-de-dios}{%
\subsection{El arrepentimiento de Efraín y la gracia de
Dios}\label{el-arrepentimiento-de-efrauxedn-y-la-gracia-de-dios}}

\bibleverse{18} No te preocupes, he oído los gemidos de Efraín, que
dice: ``Me has disciplinado muy duramente, como si fuera un ternero que
no ha sido adiestrado. Por favor, hazme volver, déjame regresar, porque
tú eres el Señor, mi Dios. \bibleverse{19} Cuando volví a ti me
arrepentí, y una vez que comprendí, me sujeté la cabeza con
tristeza.\footnote{\textbf{31:19} ``Me sujeté la cabeza con tristeza'':
  literalmente, ``me golpeé el muslo''.} Me avergoncé y me sonrojé,
avergonzado por lo que había hecho cuando era joven''. \bibleverse{20}
¿Pero no sigue siendo Efraín mi hijo precioso, mi hijo adorable? Aunque
a menudo tenga que regañarlo, no puedo olvidarlo. Por eso me desgarro
por dentro con anhelo, queriendo demostrar lo mucho que me importa!
declara el Señor.

\hypertarget{el-llamado-de-dios-a-israel-para-que-regrese}{%
\subsection{El llamado de Dios a Israel para que
regrese}\label{el-llamado-de-dios-a-israel-para-que-regrese}}

\bibleverse{21} Pongan indicadores en el camino; háganse señales. Tienen
que estar seguros de poder encontrar de nuevo el camino por el que han
viajado. Vuelve, Virgen Israel, vuelve a tus pueblos. \bibleverse{22}
¿Hasta cuándo vas a vacilar en tu decisión, hija infiel? Porque el Señor
ha hecho que aquí ocurra algo nuevo: una mujer va a proteger a un
hombre.

\hypertarget{la-bendiciuxf3n-de-dios-sobre-el-regreso-de-los-esparcidos}{%
\subsection{La bendición de Dios sobre el regreso de los
esparcidos}\label{la-bendiciuxf3n-de-dios-sobre-el-regreso-de-los-esparcidos}}

\bibleverse{23} Esto es lo que dice el Señor Todopoderoso, el Dios de
Israel: Cuando los traiga de vuelta a casa desde el exilio, volverán a
decir en la tierra de Judá y en sus ciudades ``Que el Señor te bendiga,
monte santo de Jerusalén, hogar de lo bueno y lo justo. \bibleverse{24}
El pueblo de Judá y todas sus ciudades vivirán juntos en la tierra, los
agricultores y los que se desplazan con sus rebaños, \bibleverse{25}
porque voy a dar descanso a los que están cansados y a dar fuerza a
todos los que están débiles''. \footnote{\textbf{31:25} Mat 11,28}

\bibleverse{26} Al oír esto me desperté y miré a mi alrededor. Había
tenido un sueño muy placentero.\footnote{\textbf{31:26} No está claro
  cómo encaja este versículo en el conjunto de la narración. Algunos lo
  ven como un comentario de Jeremías, otros como las reacciones de los
  recién mencionados.}

\hypertarget{dios-quiere-construir-y-plantar}{%
\subsection{Dios quiere construir y
plantar}\label{dios-quiere-construir-y-plantar}}

\bibleverse{27} ¡Mira! Se acerca el tiempo, dice el Señor, en que haré
crecer el número de personas y de ganado en Israel y en Judá.
\bibleverse{28} Yo me ocupé de ellos desarraigándolos y derribándolos,
aniquilándolos, destruyéndolos y llevándolos al desastre. Ahora me
ocuparé de ellos construyéndolos y ayudándolos a crecer, declara el
Señor.

\hypertarget{el-proverbio-sobre-los-uvas-agraces-deberuxeda-quedar-fuera-de-uso}{%
\subsection{El proverbio sobre los uvas agraces debería quedar fuera de
uso}\label{el-proverbio-sobre-los-uvas-agraces-deberuxeda-quedar-fuera-de-uso}}

\bibleverse{29} En ese tiempo la gente no repetirá este proverbio: ``Los
padres comieron las uvas inmaduras, pero sus hijos obtuvieron el sabor
agrio''. \bibleverse{30} No.~Cada persona morirá por sus propios
pecados. Si alguien come uvas sin madurar, él mismo obtendrá el sabor
agrio.

\hypertarget{el-nuevo-pacto-de-gracia-con-ambas-casas-de-israel}{%
\subsection{El nuevo pacto de gracia con ambas casas de
Israel}\label{el-nuevo-pacto-de-gracia-con-ambas-casas-de-israel}}

\bibleverse{31} ¡Mira! Se acerca el momento, dice el Señor, en que haré
un nuevo acuerdo con el pueblo de Israel y de Judá. \footnote{\textbf{31:31}
  Heb 8,8-12} \bibleverse{32} No será como el acuerdo que hice con sus
antepasados cuando los tomé de la mano y los saqué de Egipto. Ellos
rompieron ese acuerdo, aunque yo les fui fiel como un esposo, declara el
Señor. \bibleverse{33} Pero este es el acuerdo que voy a hacer con el
pueblo de Israel en ese momento, declara el Señor. Pondré mis leyes
dentro de ellos y las escribiré en sus mentes. Yo seré su Dios y ellos
serán mi pueblo. \bibleverse{34} Nadie tendrá que enseñar a su vecino o
a su hermano, diciéndole: ``Debes conocer al Señor''. Porque todos me
conocerán, desde el más pequeño hasta el más grande. Los perdonaré
cuando hagan el mal, y me olvidaré de sus pecados. \footnote{\textbf{31:34}
  Jer 33,8; Is 43,25}

\hypertarget{la-eterna-existencia-de-la-salvaciuxf3n}{%
\subsection{La eterna existencia de la
salvación}\label{la-eterna-existencia-de-la-salvaciuxf3n}}

\bibleverse{35} Esto es lo que dice el Señor, que dispone el sol para
alumbrar durante el día, que pone en orden la luna y las estrellas para
alumbrar por la noche, que hace que el mar se agite para que sus olas
rujan; su nombre es el Señor Todopoderoso: \bibleverse{36} Sólo si yo
permitiera que este orden se desmoronara, declara el Señor, los
descendientes de Israel dejarían de ser mi pueblo. \bibleverse{37} Esto
es lo que dice el Señor: Sólo si se pudieran medir los cielos de arriba
y se pudieran investigar los cimientos de la tierra de abajo, rechazaría
yo a todos los descendientes de Israel por todo lo que han hecho,
declara el Señor.

\hypertarget{restaurar-jerusaluxe9n-a-una-ciudad-perfectamente-santa}{%
\subsection{Restaurar Jerusalén a una ciudad perfectamente
santa}\label{restaurar-jerusaluxe9n-a-una-ciudad-perfectamente-santa}}

\bibleverse{38} Se acerca el tiempo, declara el Señor, en que esta
ciudad será reconstruida para el Señor, desde la torre de Hananel hasta
la Puerta de la Esquina. \footnote{\textbf{31:38} Zac 14,10}

\bibleverse{39} La línea de medición del constructor volverá a
extenderse directamente hasta la colina de Gareb y luego girará hacia
Goa. \bibleverse{40} Todo el valle, donde se entierra a los muertos y se
arroja la basura, y todos los campos desde el valle del Cedrón hasta la
Puerta de los Caballos, al este, serán sagrados para el Señor. Jerusalén
nunca más será derribada o destruida.

\hypertarget{como-prisionero-jeremuxedas-compra-un-campo-en-anatot-seguxfan-la-direcciuxf3n-de-dios}{%
\subsection{Como prisionero, Jeremías compra un campo en Anatot según la
dirección de
Dios}\label{como-prisionero-jeremuxedas-compra-un-campo-en-anatot-seguxfan-la-direcciuxf3n-de-dios}}

\hypertarget{section-31}{%
\section{32}\label{section-31}}

\bibleverse{1} Este es el mensaje del Señor que llegó a Jeremías en el
décimo año del reinado de Sedequías, rey de Judá, que era el decimoctavo
año del reinado de Nabucodonosor. \bibleverse{2} Esto ocurría cuando el
ejército del rey de Babilonia estaba sitiando Jerusalén. El profeta
Jeremías estaba preso en el patio de la guardia, que formaba parte del
palacio del rey de Judá.

\bibleverse{3} Sedequías, rey de Judá, lo había encarcelado, diciéndole
``¿Por qué tienes que profetizar así? Dices que el Señor está diciendo:
`Mira, voy a entregar esta ciudad al rey de Babilonia, y él la
capturará. \footnote{\textbf{32:3} Jer 21,7; Jer 27,6} \bibleverse{4}
Sedequías, rey de Judá, no escapará de los babilonios. Será capturado y
llevado ante el rey de Babilonia para hablar con él personalmente y
verlo cara a cara. \bibleverse{5} Se llevará a Sedequías a Babilonia,
donde permanecerá hasta que yo trate con él, declara el Señor. No
tendrás éxito si luchas contra los babilonios'\,''.

\bibleverse{6} Jeremías respondió: ``El Señor me dio un mensaje,
diciendo \bibleverse{7} Tu primo Hanamel, hijo de Salum, viene a
decirte: `¿Por qué no compras mi campo en Anatot, porque tienes derecho
a rescatarlo y comprarlo?' \footnote{\textbf{32:7} Lev 25,25; Rut 4,3-4}

\bibleverse{8} ``Tal como había dicho el Señor, mi primo Hanamel vino a
verme al patio de la guardia y me pidió: `Por favor, compra mi campo en
Anatot, en la tierra de Benjamín, porque tienes el derecho de propiedad
familiar para redimirlo.\footnote{\textbf{32:8} Para mantener la
  propiedad de la tierra en la familia, los parientes tenían el derecho
  de ``primer rechazo'' cuando otro miembro de la familia se veía
  obligado a vender su tierra. Véase Levítico 25:25-28.} Deberías
comprarlo para ti'\,''. Esto me convenció de que era un mensaje del
Señor.

\bibleverse{9} Así que compré el campo en Anatot a mi primo Hanamel.
Pesé diecisiete siclos de plata para pagarle. \bibleverse{10} Firmé la
escritura y la sellé, hice que la atestiguaran y pesé la plata con la
balanza. \bibleverse{11} Luego tomé la escritura de venta, tanto el
original sellado que contenía los términos y condiciones, como la copia
sin sellar, \bibleverse{12} y se los entregué a Baruc hijo de Nerías,
hijo de Maseías. Hice esto en presencia de mi primo Hanamel, de los
testigos que habían firmado la escritura de venta, y de todo el pueblo
de Judá que estaba sentado allí en el patio de la guardia.

\bibleverse{13} Le di a Baruc estas instrucciones delante de ellos:
\bibleverse{14} ``Esto es lo que dice el Señor Todopoderoso, el Dios de
Israel: Poned estas escrituras de venta, el original sellado y la copia
abierta, en una vasija de barro para que se mantengan a salvo durante
mucho tiempo. \bibleverse{15} Porque esto es lo que dice el Señor
Todopoderoso, el Dios de Israel: Llegará el momento en que de nuevo se
comprarán casas, campos y viñedos en este país''.

\hypertarget{oraciuxf3n-de-jeremuxedas-y-solicitud-de-aclaraciuxf3n}{%
\subsection{Oración de Jeremías y solicitud de
aclaración}\label{oraciuxf3n-de-jeremuxedas-y-solicitud-de-aclaraciuxf3n}}

\bibleverse{16} Después de entregar la escritura de venta a Baruc hijo
de Nerías, oré al Señor: \bibleverse{17} ``¡Ah, Señor Dios! Tú creaste
los cielos y la tierra con tu gran fuerza y poder. ¡Nada es demasiado
difícil para ti! \bibleverse{18} Tú das tu amor confiable a miles de
personas, pero castigas los pecados de los padres las consecuencias
afectan también a sus hijos, Dios grande y poderoso cuyo nombre es Señor
Todopoderoso, \footnote{\textbf{32:18} Éxod 20,5-6} \bibleverse{19} tú
eres el que es supremamente sabio y el que hace cosas increíbles. Tú
vigilas lo que hace cada uno, y lo recompensas según su forma de vivir y
lo que merecen sus acciones. \footnote{\textbf{32:19} Rom 2,6}
\bibleverse{20} ``Tú realizaste señales y milagros en Egipto, y lo
sigues haciendo hoy, tanto aquí en Israel como entre todos los pueblos
del mundo. Gracias a ello te ganaste una gran reputación, y esto sigue
siendo así hoy. \bibleverse{21} Sacaste a tu pueblo Israel de Egipto con
señales y milagros, con tu gran poder y fuerza que aterrorizaba a la
gente. \bibleverse{22} Les diste esta tierra que habías prometido a sus
antepasados, una tierra que mana leche y miel. \bibleverse{23}
``Vinieron y se apoderaron de ella, pero no hicieron lo que dijiste ni
siguieron tus leyes. No hicieron todo lo que les ordenaste, y por eso
has hecho caer sobre ellos todo este desastre.

\bibleverse{24} ¡Mira las rampas de asedio apiladas contra la ciudad
para capturarla! Mediante la guerra, el hambre y las enfermedades, la
ciudad será tomada por los babilonios que la están atacando. Ya ves que
todo lo que dijiste que pasaría ha sucedido. \bibleverse{25} ``Sin
embargo, Señor Dios, me has dicho: `¡Compra tú mismo el campo con plata
delante de los testigos, aunque la ciudad haya sido entregada a los
babilonios!'\,''

\hypertarget{la-respuesta-de-dios-proclamaciuxf3n-de-iluminaciuxf3n-y-salvaciuxf3n}{%
\subsection{La respuesta de Dios (proclamación de iluminación y
salvación)}\label{la-respuesta-de-dios-proclamaciuxf3n-de-iluminaciuxf3n-y-salvaciuxf3n}}

\bibleverse{26} Entonces el Señor le dio a Jeremías este mensaje
\bibleverse{27} ¡Mira! Yo soy el Señor, el Dios de todos. ¿Hay algo que
sea demasiado difícil para mí? \footnote{\textbf{32:27} Núm 16,22; Jer
  32,17} \bibleverse{28} ``Esto es lo que dice el Señor: ¡Escucha! Voy a
entregar esta ciudad al rey de Babilonia y a los babilonios, y ellos la
capturarán. \footnote{\textbf{32:28} Jer 32,3} \bibleverse{29} Los
babilonios que están atacando la ciudad van a venir y la van a
incendiar. La quemarán, incluso las casas de la gente que me hizo enojar
quemando incienso a Baal en sus azoteas, y derramando libaciones en
adoración de otros dioses. \footnote{\textbf{32:29} Jer 19,13}

\bibleverse{30} ``Desde sus primeros días, todo lo que ha hecho el
pueblo de Israel y de Judá ha sido malo a mis ojos. De hecho, todo lo
que han hecho es para enfurecerme con sus acciones, declara el Señor.
\bibleverse{31} Esta ciudad ha sido una fuente de ira y frustración
desde que fue construida hasta ahora. Así que voy a deshacerme de ella,
\bibleverse{32} por todas las cosas malas que hizo el pueblo de Israel y
de Judá y que me hicieron enojar: sus reyes y funcionarios, sus
sacerdotes y profetas, todos los que viven en Judá y Jerusalén, todos
\bibleverse{33} Me han dado la espalda. Ni siquiera me miraron. A pesar
de que seguí tratando de enseñarles, se negaron a escuchar o a aceptar
la instrucción. \bibleverse{34} ``Han puesto sus repugnantes ídolos en
mi Templo, haciéndolo impuro. \bibleverse{35} Han construido santuarios
paganos a Baal en el Valle de Hinom para poder sacrificar a sus hijos e
hijas quemándolos en el fuego. Esto es algo que nunca ordené. Nunca se
me ocurrió hacer algo tan horrible y hacer al pueblo de Judá culpable de
pecado. \footnote{\textbf{32:35} Jer 7,31; Jer 19,5}

\bibleverse{36} ``Ahora sobre esta ciudad. Usted está diciendo
correctamente: `Va a ser entregado al rey de Babilonia a través de la
guerra y el hambre y la enfermedad'. Sin embargo, esto es lo que dice el
Señor, el Dios de Israel: \bibleverse{37} Prometo reunir a mi pueblo de
todas las tierras a las que lo desterré porque me hizo enfadar mucho.
Los traeré de vuelta aquí y vivirán con seguridad. \bibleverse{38} Ellos
serán mi pueblo y yo seré su Dios. \bibleverse{39} Me aseguraré de que
piensen de la misma manera y actúen en armonía, para que siempre me
honren y todo sea bueno para ellos y sus descendientes. \footnote{\textbf{32:39}
  Ezeq 36,27} \bibleverse{40} ``Haré un acuerdo eterno con ellos: Nunca
dejaré de hacerles el bien y les ayudaré a respetarme para que nunca me
abandonen. \bibleverse{41} Me encantará tratarlos bien, y me
comprometeré con todo mi ser a ayudarlos a crecer como nación en esta
tierra.

\bibleverse{42} ``Esto es lo que dice el Señor: Así como ciertamente he
hecho caer todo este desastre sobre mi pueblo, así también voy a darles
todas las cosas buenas que he prometido. \bibleverse{43} Se volverán a
comprar campos en este país que describes, diciendo: `Ha sido
completamente destruido; no quedan personas ni animales. Ha sido
entregado a los babilonios'. \bibleverse{44} La gente volverá a comprar
campos con plata, las escrituras serán firmadas, selladas y
atestiguadas. Esto sucederá aquí, en la tierra de Benjamín, en los
alrededores de Jerusalén y en todas las ciudades de Judá -incluyendo las
ciudades de la región montañosa, las estribaciones y el Néguev-, porque
yo haré regresar al pueblo del exilio, declara el Señor''.

\hypertarget{profecuxedas-de-salvaciuxf3n-para-jerusaluxe9n-y-juduxe1}{%
\subsection{Profecías de salvación para Jerusalén y
Judá}\label{profecuxedas-de-salvaciuxf3n-para-jerusaluxe9n-y-juduxe1}}

\hypertarget{section-32}{%
\section{33}\label{section-32}}

\bibleverse{1} Un segundo mensaje vino del Señor a Jeremías mientras
seguía detenido en el patio de la guardia \footnote{\textbf{33:1} Jer
  32,2} \bibleverse{2} Esto es lo que dice el Señor, el Señor que hizo
la tierra, el Señor que le dio forma y la puso en su lugar, el Señor es
su nombre: \bibleverse{3} Clama a mí, y yo te responderé, explicándote
cosas sorprendentes y ocultas de las que no tienes idea. \bibleverse{4}
Porque esto es lo que dice el Señor, el Dios de Israel, sobre las casas
de Jerusalén y los palacios de los reyes de Judá que fueron demolidos
para obtener materiales de defensa contra las rampas de asedio y los
ataques del enemigo. \bibleverse{5} Vienen a luchar contra los
babilonios, pero sólo llenarán esas casas con los cadáveres de los que
voy a matar en mi furiosa ira. He renunciado a esta ciudad a causa de
toda su maldad. \bibleverse{6} Pero aun así, en el futuro la restauraré
y repararé, y sanaré a su pueblo y le daré paz y seguridad duraderas.
\bibleverse{7} Haré que Judá e Israel vuelvan del exilio y los haré tan
fuertes como antes. \footnote{\textbf{33:7} Jer 29,14; Jer 30,3}
\bibleverse{8} Lavaré todos sus pecados que cometieron contra mí, y
perdonaré toda su culpa desde que pecaron al rebelarse contra mí.
\footnote{\textbf{33:8} Jer 31,34} \bibleverse{9} Entonces esta ciudad
me dará una reputación gloriosa, celebrada y alabada por todas las
naciones de la tierra que se enteren de todas las cosas buenas que hago
por ella. Temblarán, asombrados de todo el bien que he hecho por ella,
de cómo la he hecho tan próspera.

\bibleverse{10} Esto es lo que dice el Señor: Ustedes llaman a este
lugar ``un páramo donde no hay gente ni animales''. Pues bien, aquí, en
las ciudades de Judá y en las calles vacías de Jerusalén, donde no viven
ni personas ni animales, un día \bibleverse{11} volverán a oírse allí
los sonidos de la alegría y la fiesta, las voces alegres de los novios y
los gritos de alabanza de los que traen ofrendas de agradecimiento al
Templo del Señor, diciendo: ``¡Gracias al Señor Todopoderoso! Porque el
Señor es bueno; su amor confiable perdura para siempre''. Porque yo
también haré volver a la tierra de su ``cautiverio'', dice el Señor.
\footnote{\textbf{33:11} Jer 7,34; Sal 106,1; Esd 3,11}

\bibleverse{12} Esto es lo que dice el Señor Todopoderoso: En este
páramo donde no hay gente ni animales, y en todas sus ciudades, volverá
a haber pastos donde los pastores puedan llevar sus rebaños.
\bibleverse{13} En todos los pueblos, ya sea en la región montañosa, en
las estribaciones, en el Néguev, en la tierra de Benjamín, en los
pueblos alrededor de Jerusalén o en todas las ciudades de Judá, los
rebaños volverán a ser contados por sus pastores, dice el Señor.

\hypertarget{profecuxeda-del-brote-de-david-del-futuro-y-de-la-existencia-eterna-del-pueblo-la-realeza-davuxeddica-y-el-sacerdocio-levuxedtico}{%
\subsection{Profecía del brote de David del futuro y de la existencia
eterna del pueblo, la realeza davídica y el sacerdocio
levítico}\label{profecuxeda-del-brote-de-david-del-futuro-y-de-la-existencia-eterna-del-pueblo-la-realeza-davuxeddica-y-el-sacerdocio-levuxedtico}}

\bibleverse{14} ¡Mira! Se acerca el momento, dice el Señor, en que
cumpliré mi promesa de hacer el bien al pueblo de Israel y de Judá.
\bibleverse{15} En ese momento, allí mismo, les daré un buen rey del
linaje de David.\footnote{\textbf{33:15} Literalmente, ``haré brotar
  para David una rama justa''.} Él hará lo que es justo y correcto en
todo el país. \footnote{\textbf{33:15} Jer 23,5; Is 4,2} \bibleverse{16}
Entonces se salvará Judá, y el pueblo de Jerusalén vivirá con seguridad.
Este es el nombre que recibirá: El Señor que nos hace justicia.
\footnote{\textbf{33:16} Jer 23,6; Deut 33,28}

\bibleverse{17} Esto es lo que dice el Señor: David tendrá siempre un
descendiente que será rey de Israel, \footnote{\textbf{33:17} 2Sam 7,12;
  1Re 9,5} \bibleverse{18} y los sacerdotes levitas tendrán siempre un
descendiente que me presente holocaustos, ofrendas de grano y
sacrificios.

\bibleverse{19} Un mensaje del Señor llegó a Jeremías: \bibleverse{20}
Esto es lo que dice el Señor: Si fueras capaz de romper mi acuerdo con
el día y con la noche, para que no llegaran a su hora, \bibleverse{21}
sólo entonces se rompería mi acuerdo con David, mi siervo, y con los
levitas que sirven como mis sacerdotes, para que David no tuviera un
descendiente que reinara en su trono. \bibleverse{22} De la misma manera
que no se pueden contar las estrellas del cielo ni se puede medir la
arena de la orilla del mar, así multiplicaré el número de los
descendientes de mi siervo David y de los levitas que me sirven.
\footnote{\textbf{33:22} Gén 15,5; Gén 22,17}

\bibleverse{23} Otro mensaje del Señor llegó a Jeremías: \bibleverse{24}
¿Has oído lo que dice la gente? ``El Señor eligió a dos familias, pero
ahora las ha rechazado''? Por eso desprecian a mi pueblo y no lo
consideran digno de ser llamado\footnote{\textbf{33:24} ``Digno de ser
  llamado'' se ha añadido para mayor claridad.} una nación.
\bibleverse{25} Esto es lo que dice el Señor: Así como no puedo romper
mi acuerdo con el día y la noche y las leyes que regulan el cielo y la
tierra, \bibleverse{26} tampoco puedo rechazar a los descendientes de
Jacob y de mi siervo David, y no puedo dejar de hacer que sus
descendientes sean gobernantes sobre los descendientes de Abraham, Isaac
y Jacob. Los haré volver del exilio y seré benévolo con
ellos.\footnote{\textbf{33:26} Jer 32,44}

\hypertarget{anunciando-el-destino-de-sedechuxeeas}{%
\subsection{Anunciando el destino de
Sedechîas}\label{anunciando-el-destino-de-sedechuxeeas}}

\hypertarget{section-33}{%
\section{34}\label{section-33}}

\bibleverse{1} Este es el mensaje del Señor que llegó a Jeremías cuando
Nabucodonosor, rey de Babilonia, y todo su ejército, junto con las
tropas de todos los países que gobernaba y de otras naciones, estaban
atacando Jerusalén y todas sus ciudades cercanas: \bibleverse{2} Esto es
lo que dice el Señor, el Dios de Israel: Ve a hablar con Sedequías, rey
de Judá, y dile que esto es lo que dice el Señor: ¡Escucha! Estoy a
punto de entregar esta ciudad al rey de Babilonia, y él la va a
incendiar. \bibleverse{3} Tú mismo no escaparás de ser capturado por él.
Ciertamente serás tomado prisionero y llevado ante él para hablar con él
personalmente y verlo cara a cara. Serás llevado a Babilonia.

\bibleverse{4} Escucha lo que el Señor te dice, Sedequías, rey de Judá.
Esto es lo que el Señor dice de ti: No te matarán; \footnote{\textbf{34:4}
  Jer 52,11} \bibleverse{5} morirás en paz. Tendrás un funeral apropiado
con incienso quemado para ti como lo hicieron con tus antepasados, los
reyes que gobernaron antes que tú. Llorarán por ti, gritando: ``El rey
ha muerto''. Yo mismo te lo digo, declara el Señor. \footnote{\textbf{34:5}
  2Cró 16,14; Jer 22,18}

\bibleverse{6} El profeta Jeremías le dijo todo esto a Sedequías, rey de
Judá, allí en Jerusalén. \bibleverse{7} En ese momento el ejército del
rey de Babilonia estaba atacando la ciudad y las ciudades de Judea de
Laquis y Azeca. Estas eran las únicas ciudades fortificadas que aún no
habían sido conquistadas en Judá. \footnote{\textbf{34:7} 2Re 25,1}

\hypertarget{el-discurso-de-jeremuxedas-y-la-amenaza-de-castigo-de-dios-por-la-violaciuxf3n-de-la-fidelidad-cometida-con-los-esclavos-hebreos-liberados-en-jerusaluxe9n}{%
\subsection{El discurso de Jeremías y la amenaza de castigo de Dios por
la violación de la fidelidad cometida con los esclavos hebreos liberados
en
Jerusalén}\label{el-discurso-de-jeremuxedas-y-la-amenaza-de-castigo-de-dios-por-la-violaciuxf3n-de-la-fidelidad-cometida-con-los-esclavos-hebreos-liberados-en-jerusaluxe9n}}

\bibleverse{8} Un mensaje del Señor llegó a Jeremías después de que el
rey Sedequías había acordado con todos en Jerusalén anunciar una
proclamación de libertad. \footnote{\textbf{34:8} Jer 34,14}
\bibleverse{9} Esto significaba que todo propietario de esclavos debía
liberar a sus esclavos hebreos, tanto hombres como mujeres. Nadie debía
obligar a sus conciudadanos a seguir siendo esclavos. \bibleverse{10}
Todos los funcionarios y todo el pueblo que aceptaron este acuerdo
hicieron lo que dijeron. Liberaron a sus esclavos y esclavas, sin
obligarlos a seguir siendo esclavos. Obedecieron y los dejaron libres.
\bibleverse{11} Sin embargo, más tarde cambiaron de opinión y volvieron
a tomar a los esclavos y esclavas que habían liberado, obligándolos a
volver a la esclavitud.

\hypertarget{la-palabra-de-juicio-de-dios}{%
\subsection{La palabra de juicio de
Dios}\label{la-palabra-de-juicio-de-dios}}

\bibleverse{12} Un mensaje del Señor llegó a Jeremías, diciendo:
\bibleverse{13} Esto es lo que dice el Señor, el Dios de Israel: Hice un
acuerdo con sus antepasados cuando los saqué de Egipto, de la cárcel de
la esclavitud, diciendo: \bibleverse{14} Cada siete años, cada uno de
ustedes deberá liberar a todos los compañeros hebreos que se hayan
vendido a ustedes. Pueden servirte durante seis años, pero luego debes
liberarlos. Pero tus antepasados no prestaron atención y no obedecieron
lo que les dije. \footnote{\textbf{34:14} Éxod 21,2; Deut 15,12}
\bibleverse{15} Hace poco tiempo ustedes decidieron hacer lo que es
correcto, lo que me hizo feliz. Todos ustedes anunciaron que liberarían
a sus esclavos. Hicieron un acuerdo ante mí en mi Templo.
\bibleverse{16} Pero ahora han cambiado de parecer y me han deshonrado.
Cada uno de ustedes recuperó a los esclavos y esclavas que habían
liberado para que hicieran lo que quisieran. Los obligaron a volver a
ser sus esclavos.

\bibleverse{17} Esto es lo que dice el Señor: No me has obedecido. No
has anunciado la libertad para tus esclavos, tu propio pueblo. Así que
ahora les anuncio la ``libertad'', declara el Señor: ¡Libertad para ser
asesinados por la guerra, por la enfermedad y por el hambre! Haré que
todos los reinos del mundo se horroricen de ustedes. \bibleverse{18}
Ellos han roto mi acuerdo, y no han cumplido los términos del acuerdo
que prometieron ante mí. Así que los voy a despedazar como al ternero
que cortaron por la mitad para pasar entre sus dos trozos.\footnote{\textbf{34:18}
  Se refiere a la forma en que se hizo un acuerdo. Véase Génesis 15.}
\bibleverse{19} Los entregaré a sus enemigos que intentan matarlos. Esto
incluye a los líderes de Judá y Jerusalén, a los funcionarios de la
corte, a los sacerdotes y a todos los que pasaron entre las piezas del
becerro. \bibleverse{20} Sus cadáveres se convertirán en alimento para
las aves de rapiña y los animales salvajes. \footnote{\textbf{34:20} Jer
  7,33}

\bibleverse{21} Entregaré a Sedequías, rey de Judá, y a sus funcionarios
a sus enemigos que intentan matarlos, al ejército del rey de Babilonia
que había detenido su ataque contra ustedes.\footnote{\textbf{34:21} El
  ejército babilónico abandonó temporalmente Jerusalén para hacer frente
  a un ejército egipcio que avanzaba. Esto había hecho creer a los
  dirigentes de Jerusalén que el peligro había pasado y que el ejército
  egipcio les ayudaría. Ver Jeremías 37.} \bibleverse{22} ¡Escuchen! Yo
daré la orden, declara el Señor, y los haré volver a Jerusalén. La
atacarán, la capturarán y la quemarán. Voy a destruir las ciudades de
Judá para que nadie viva allí.

\hypertarget{por-mandato-divino-jeremuxedas-prueba-la-fidelidad-de-los-recabitas}{%
\subsection{Por mandato divino, Jeremías prueba la fidelidad de los
recabitas}\label{por-mandato-divino-jeremuxedas-prueba-la-fidelidad-de-los-recabitas}}

\hypertarget{section-34}{%
\section{35}\label{section-34}}

\bibleverse{1} Este es el mensaje que le llegó a Jeremías de parte del
Señor durante el reinado de Joacim hijo de Josías, rey de Judá:
\bibleverse{2} Ve a donde los recabitas\footnote{\textbf{35:2} Véase 2
  Reyes 10:15-31; 1 Crónicas 2:55.} en vivo. Invítalos a venir contigo a
una de las salas del Templo del Señor y ofréceles vino para beber.
\footnote{\textbf{35:2} 1Cró 2,55}

\bibleverse{3} Fui, pues, a visitar a Jaazanías hijo de Jeremías, hijo
de Habazzinías, y a sus hermanos y a todos sus hijos: toda la familia
recabita. \bibleverse{4} Luego los llevé al Templo del Señor, a una sala
que usaban los hijos de Hanán, hijo de Igdaliah, un hombre de Dios.
Estaba junto a la que usaban los funcionarios, que estaba encima de la
habitación de Maasías hijo de Salum, que era el portero del Templo.
\bibleverse{5} Coloqué unas jarras llenas de vino y unas copas delante
de los recabitas y les dije: ``Tomad vino''.

\bibleverse{6} ``No bebemos vino'', dijeron, ``porque nuestro antepasado
Jonadab hijo de Recab nos dio estas órdenes: `Tú y tus descendientes no
deben beber nunca vino. \bibleverse{7} No construyan nunca casas, ni
siembren cosechas, ni planten viñedos. No lo hagan. En cambio, vivan
siempre en tiendas para que tengan una larga vida mientras se desplazan
de un lugar a otro del país'. \bibleverse{8} ``Hemos hecho exactamente
lo que nos dijo nuestro antepasado Jonadab hijo de Recab. Ninguno de
nosotros ha bebido nunca vino, y eso incluye a nuestras esposas y a
nuestros hijos e hijas, así como a nosotros. \bibleverse{9} No hemos
construido casas para vivir, ni hemos tenido viñas ni campos ni hemos
cultivado nada. \bibleverse{10} Hemos vivido en tiendas de campaña y
hemos obedecido a nuestro antepasado Jonadab, siguiendo todo lo que nos
ordenaba. \bibleverse{11} ``Por eso, cuando Nabucodonosor, rey de
Babilonia, invadió el país, decidimos: `Vamos, entremos en Jerusalén
para ponernos a salvo de los ejércitos de los caldeos,\footnote{\textbf{35:11}
  ``Caldeos'': suele traducirse como ``babilonios'', pero los babilonios
  también se incluyen aquí por separado. Los caldeos formaban parte de
  Babilonia y procedían del sur de la región.} babilonios y arameos'.
Por eso nos hemos quedado en Jerusalén''.

\hypertarget{el-discurso-de-jeremuxedas-a-los-transeuxfantes}{%
\subsection{El discurso de Jeremías a los
transeúntes}\label{el-discurso-de-jeremuxedas-a-los-transeuxfantes}}

\bibleverse{12} Entonces llegó a Jeremías un mensaje del Señor:
\bibleverse{13} Esto es lo que dice el Señor Todopoderoso, el Dios de
Israel: Ve y dile a los hombres de Judá y a los habitantes de Jerusalén:
¿Por qué no aceptan mis instrucciones y obedecen lo que les digo?
pregunta el Señor. \bibleverse{14} Se han seguido las instrucciones de
Jonadab hijo de Recab. Él ordenó a sus descendientes que no bebieran
vino, y no lo han bebido hasta hoy porque han obedecido el mandato de su
antepasado. Pero yo les he dicho una y otra vez lo que deben hacer, ¡y
sin embargo se niegan a obedecerme! \bibleverse{15} Una y otra vez les
he enviado a muchos de mis siervos los profetas para decirles ¡Todos,
dejen sus malos caminos y hagan lo que es correcto! No sigan a otros
dioses ni los adoren. Vivan en la tierra que les di a ustedes y a sus
padres. Pero no me han hecho caso ni me han obedecido. \footnote{\textbf{35:15}
  Jer 25,4-7} \bibleverse{16} Estos descendientes de Jonadab hijo de
Recab han seguido el mandato que les dio su antepasado, pero este pueblo
no me ha obedecido.

\bibleverse{17} Así que esto es lo que dice el Señor Dios Todopoderoso,
el Dios de Israel: Mira cómo hago caer sobre Judá y sobre todo el pueblo
que vive en Jerusalén todos los desastres que he amenazado hacerles,
porque les he dicho lo que deben hacer y no han obedecido; les he
apelado y no han respondido.

\bibleverse{18} Entonces Jeremías dijo a los recabitas Esto es lo que
dice el Señor Todopoderoso, el Dios de Israel: Como ustedes han
obedecido las instrucciones de su antepasado Jonadab y han seguido sus
órdenes y han hecho todo lo que él les dijo que hicieran,
\bibleverse{19} esto es lo que dice el Señor Todopoderoso, el Dios de
Israel: Jonadab hijo de Recab tendrá siempre a alguien que estará en mi
presencia sirviéndome.

\hypertarget{hacer-el-libro-y-leerlo-a-la-gente-y-a-los-superiores}{%
\subsection{Hacer el libro y leerlo a la gente y a los
superiores}\label{hacer-el-libro-y-leerlo-a-la-gente-y-a-los-superiores}}

\hypertarget{section-35}{%
\section{36}\label{section-35}}

\bibleverse{1} Este mensaje del Señor llegó a Jeremías en el cuarto año
del reinado de Joaquín hijo de Josías, rey de Judá: \footnote{\textbf{36:1}
  Jer 25,1} \bibleverse{2} Toma un rollo y escribe todo lo que te he
dicho condenando a Israel, a Judá y a todas las demás naciones, desde
que te hablé por primera vez durante el reinado de Josías hasta ahora.
\bibleverse{3} Tal vez cuando el pueblo de Judá se entere de todos los
desastres que pienso hacer caer sobre ellos, todos dejarán de hacer sus
malas acciones. Entonces perdonaré su culpa y su pecado.

\bibleverse{4} Entonces Jeremías llamó a Baruc hijo de Nerías para que
viniera, y mientras Jeremías dictaba, Baruc escribió en un pergamino
todo lo que el Señor le había dicho a Jeremías. \footnote{\textbf{36:4}
  Jer 32,12} \bibleverse{5} Luego Jeremías le dio a Baruc estas
instrucciones: ``Estoy prisionero aquí, así que no puedo entrar en el
Templo del Señor. \bibleverse{6} Así que tienes que ir al Templo del
Señor en un día en que la gente esté ayunando, y leerles los mensajes
del Señor del rollo que te dicté. Léelos a todo el pueblo de Judá que
venga de sus ciudades. \bibleverse{7} Tal vez vengan y pidan perdón al
Señor, y todos ellos dejen de actuar con maldad, porque el Señor amenaza
con una ira terrible contra ellos''.

\bibleverse{8} Baruc hijo de Nerías hizo exactamente lo que el profeta
Jeremías le había dicho que hiciera. Fue y leyó el mensaje del Señor del
rollo en el Templo. \bibleverse{9} Así sucedió.\footnote{\textbf{36:9}
  Añadido para mayor claridad.} Se declaró un ayuno para honrar al Señor
en el que participó todo el pueblo de Jerusalén y todos los que habían
llegado allí desde las ciudades de Judá. Esto ocurrió en el noveno mes
del quinto año del reinado de Joacim hijo de Josías, rey de Judá.
\bibleverse{10} Baruc leyó del pergamino lo que Jeremías había dictado
para que todos pudieran oírlo. Lo leyó desde la habitación del escriba
Gemarías hijo de Safán. Ésta se encontraba en el patio superior del
Templo, a la entrada de la Puerta Nueva.

\bibleverse{11} Cuando Micaías hijo de Gemarías, hijo de Safán, escuchó
todos los mensajes del Señor leídos en el rollo, \bibleverse{12} bajó a
la habitación del secretario real en el palacio del rey, donde se habían
reunido todos los funcionarios. Estaban allí el secretario Elisama,
Delaías hijo de Semaías, Elnatán hijo de Acbor, Gemarías hijo de Safán,
Sedequías hijo de Jananías y todos los demás funcionarios.
\bibleverse{13} Micaías les dio un informe de todo lo que había oído
leer a Baruc del rollo al pueblo. \bibleverse{14} Los funcionarios
enviaron a Jehudí hijo de Netanías, hijo de Selemías, hijo de Cusi, a
convocar a Baruc, diciéndole: ``Trae el rollo que has leído al pueblo y
ven aquí''. Así que Baruc fue a verlos llevando el pergamino.

\bibleverse{15} ``Por favor, siéntate y léenoslo'', le dijeron. Así que
Baruc se lo leyó.

\bibleverse{16} Después de oírlo todo, se asustaron y se miraron unos a
otros. Le dijeron a Baruc: ``Tenemos que contarle todo esto al rey''.
\bibleverse{17} Entonces le preguntaron a Baruc: ``Ahora dinos, ¿cómo
llegaste a escribir todo esto? ¿Te lo dictó Jeremías?''

\bibleverse{18} ``Sí, me lo dictó'', respondió Baruc. ``Yo escribí con
tinta en el pergamino todo lo que me dijo''.

\bibleverse{19} Los funcionarios le dijeron a Baruc: ``Tú y Jeremías van
a tener que esconderse. No le digan a nadie dónde están''.

\hypertarget{el-rey-joacim-corta-y-quema-el-libro-de-adivinaciuxf3n-de-jeremuxedas}{%
\subsection{El rey Joacim corta y quema el libro de adivinación de
Jeremías}\label{el-rey-joacim-corta-y-quema-el-libro-de-adivinaciuxf3n-de-jeremuxedas}}

\bibleverse{20} Entonces los funcionarios fueron a ver al rey al patio.
Habían guardado el pergamino a buen recaudo en la habitación de Elisama,
el secretario, mientras daban un informe completo al rey.
\bibleverse{21} El rey envió a Jehudí a buscar el pergamino. Fue y lo
sacó de la habitación de Elisama el secretario. Entonces Jehudí se lo
leyó al rey y a todos los funcionarios que estaban allí de pie junto a
él. \bibleverse{22} Era el mes noveno y el rey estaba sentado frente al
fuego en sus aposentos de invierno. \bibleverse{23} Cada vez que Jehudí
terminaba de leer tres o cuatro columnas, Joaquín las cortaba con un
cuchillo de escriba y las arrojaba al fuego. Finalmente, todo el
pergamino se quemó por completo. \bibleverse{24} A pesar de escuchar
todos estos mensajes, el rey y sus asistentes no se asustaron ni se
rasgaron las vestiduras por el remordimiento. \footnote{\textbf{36:24}
  2Re 22,11} \bibleverse{25} Incluso cuando Elnatán, Delaías y Gemarías
le suplicaron al rey que no quemara el rollo, éste se negó a
escucharlos. \bibleverse{26} De hecho, el rey ordenó a Jerajmeel, uno de
los príncipes,\footnote{\textbf{36:26} ``Uno de los príncipes'':
  literalmente, ``un hijo del rey'', pero Joacim habría sido demasiado
  joven para tener un hijo adulto.} así como a Seraías hijo de Azriel y
a Selemías hijo de Abdeel, para ir a detener a Baruc y a Jeremías. Pero
el Señor los escondió.

\hypertarget{renovaciuxf3n-del-libro-y-la-amenaza-de-dios-a-joacim}{%
\subsection{Renovación del libro y la amenaza de Dios a
Joacim}\label{renovaciuxf3n-del-libro-y-la-amenaza-de-dios-a-joacim}}

\bibleverse{27} Después de que el rey quemó el pergamino que Jeremías
había dictado a Baruc, llegó a Jeremías un mensaje del Señor:
\bibleverse{28} Consigue otro pergamino y escribe todo lo que había en
el primer pergamino que quemó Joacim, rey de Judá. \bibleverse{29}
Respecto a Joacim, rey de Judá, anuncia que esto es lo que dice el
Señor: Has quemado el pergamino y has preguntado: ``¿Por qué has escrito
que el rey de Babilonia va a venir a destruir este país y a matar a toda
su gente y a sus animales?'' \footnote{\textbf{36:29} Jer 25,9-11; Jer
  7,20; Jer 9,9} \bibleverse{30} Esto es lo que dice el Señor acerca de
Joacim, rey de Judá: No tendrá a nadie que lo suceda como rey, sentado
en el trono de David. Su cuerpo será arrojado para que repose en el
calor del día y en el frío de la noche. \footnote{\textbf{36:30} Jer
  22,19} \bibleverse{31} Voy a castigarlo a él y a sus descendientes y
funcionarios por sus pecados. Haré caer sobre ellos y sobre el pueblo
que vive en Jerusalén y en Judá, todos los desastres de los que les
advertí, pero se negaron a escuchar.

\bibleverse{32} Jeremías tomó otro rollo y se lo dio a Baruc. Jeremías
le dictó todo lo que había en el pergamino que Joacim había quemado en
el fuego y Baruc lo escribió. Se añadieron aún más mensajes de tipo
similar.

\hypertarget{respuesta-de-jeremuxedas-a-la-embajada-de-sedechuxeeas}{%
\subsection{Respuesta de Jeremías a la embajada de
Sedechîas}\label{respuesta-de-jeremuxedas-a-la-embajada-de-sedechuxeeas}}

\hypertarget{section-36}{%
\section{37}\label{section-36}}

\bibleverse{1} Nabucodonosor, rey de Babilonia, sustituyó a
Joaquín\footnote{\textbf{37:1} ``Joaquín'': aquí llamado ``Conías''.}
hijo de Joacim con Sedequías hijo de Josías como rey gobernante de Judá.
\footnote{\textbf{37:1} 2Re 24,17} \bibleverse{2} Pero Sedequías y sus
oficiales y todos los demás en el país se negaron a obedecer lo que el
Señor había dicho por medio del profeta Jeremías.

\bibleverse{3} Sin embargo, el rey Sedequías envió a Jehucal\footnote{\textbf{37:3}
  O ``Jucal''. Jeremías 38:1.} hijo de Selemías y el sacerdote Sofonías,
hijo de Maasías, al profeta Jeremías con el mensaje: ``¡Por favor, ruega
al Señor nuestro Dios por nosotros!''

\bibleverse{4} (En ese momento Jeremías podía ir y venir libremente,
porque todavía no lo habían encarcelado). \bibleverse{5} El ejército del
faraón avanzaba desde Egipto, y cuando el ejército babilónico se enteró,
se alejó de Jerusalén.

\bibleverse{6} Entonces llegó un mensaje del Señor al profeta Jeremías:
\bibleverse{7} Esto es lo que el Señor, el Dios de Israel, te manda
decir al rey de Judá, que te ha enviado a pedirme ayuda: ¡Mira! El
ejército del faraón, que salió en tu ayuda, va a regresar a Egipto.
\bibleverse{8} Entonces los babilonios volverán y atacarán Jerusalén. La
capturarán y la incendiarán.

\bibleverse{9} Esto es lo que dice el Señor: No se engañen diciendo:
``Los babilonios se han ido para siempre'', porque no es así.
\bibleverse{10} De hecho, aunque ustedes pudieran matar a todo el
ejército babilónico que los ataca, dejando sólo a los hombres heridos en
sus tiendas, igual se levantarían y quemarían esta ciudad.

\hypertarget{arresto-y-encarcelamiento-de-jeremuxedas-por-un-oficial-militar}{%
\subsection{Arresto y encarcelamiento de Jeremías por un oficial
militar}\label{arresto-y-encarcelamiento-de-jeremuxedas-por-un-oficial-militar}}

\bibleverse{11} Cuando el ejército babilónico se alejó de Jerusalén
debido a la amenaza del ejército del faraón, \bibleverse{12} Jeremías
estaba saliendo de Jerusalén para ir a su casa en el territorio de
Benjamín a reclamar su parte de la propiedad de su familia. \footnote{\textbf{37:12}
  Jer 32,9} \bibleverse{13} Sin embargo, cuando llegó a la puerta de
Benjamín, el capitán de la guardia, que se llamaba Irías, hijo de
Selemías, hijo de Jananías, lo detuvo diciendo: ``¡Desertas a los
babilonios!''

\bibleverse{14} ``Eso no es cierto'', respondió Jeremías. ``¡No estoy
desertando a los babilonios!'' Pero Irías se negó a escucharlo. Arrestó
a Jeremías y lo llevó ante los oficiales.\footnote{\textbf{37:14}
  Probablemente los mismos oficiales mencionados en el verso 2.}

\bibleverse{15} Los oficiales estaban furiosos con Jeremías. Hicieron
que lo golpearan y lo encerraran en la casa del escriba Jonatán, que
había sido convertida en prisión.

\hypertarget{jeremuxedas-interrogado-nuevamente-por-el-rey-y-llevado-de-la-cuxe1rcel-al-patio-de-la-guardia}{%
\subsection{Jeremías interrogado nuevamente por el rey y llevado de la
cárcel al patio de la
guardia}\label{jeremuxedas-interrogado-nuevamente-por-el-rey-y-llevado-de-la-cuxe1rcel-al-patio-de-la-guardia}}

\bibleverse{16} Jeremías fue colocado en una celda del calabozo
subterráneo y permaneció allí durante mucho tiempo. \bibleverse{17} Un
tiempo después, el rey Sedequías lo mandó llamar en secreto y lo hizo
llevar al palacio real, donde le preguntó: ``¿Hay un mensaje del Señor
para mí?'' ``Sí lo hay'', respondió Jeremías. ``Vas a ser entregado al
rey de Babilonia''. \footnote{\textbf{37:17} Jer 34,21}

\bibleverse{18} Entonces Jeremías le preguntó al rey Sedequías: ``¿Qué
mal te he hecho a ti, a tus siervos o a este pueblo, para que me pongas
en prisión? \bibleverse{19} ¿Dónde están ahora tus profetas, los que te
profetizaron diciendo: `El rey de Babilonia no vendrá a atacarte a ti y
a este país'? \bibleverse{20} Ahora, por favor, escúchame, mi señor el
rey, y responde positivamente a mi petición. No me envíes de nuevo a la
prisión en la casa del escriba Jonatán, pues de lo contrario moriré
allí''.

\bibleverse{21} El rey Sedequías dio la orden de que Jeremías fuera
recluido en el patio de la guardia y que se le proporcionara una hogaza
de pan cada día de una panadería hasta que no quedara pan en la ciudad.
Así que Jeremías se quedó en el patio de la guardia.

\hypertarget{jeremuxedas-arrojado-a-una-cisterna-como-traidor-por-los-superiores}{%
\subsection{Jeremías arrojado a una cisterna como traidor por los
superiores}\label{jeremuxedas-arrojado-a-una-cisterna-como-traidor-por-los-superiores}}

\hypertarget{section-37}{%
\section{38}\label{section-37}}

\bibleverse{1} Sefatías hijo de Matán, Gedalías hijo de Pasur,
Jucal\footnote{\textbf{38:1} O ``Jucal''. Jeremías 37:3.} hijo de
Selemías, y Pasur hijo de Malquías oyeron lo que Jeremías decía a todos:
\footnote{\textbf{38:1} Jer 21,1} \bibleverse{2} Esto es lo que dice el
Señor: El que se quede en esta ciudad morirá de guerra, de hambre y de
enfermedad, pero el que se pase a los babilonios vivirá. Su recompensa
será salvar su vida. \footnote{\textbf{38:2} Jer 21,9} \bibleverse{3}
Esto es lo que dice el Señor: Estén seguros de esto: Jerusalén será
entregada al ejército del rey de Babilonia. Él va a capturarla.

\bibleverse{4} Los oficiales le dijeron al rey: ``Este hombre merece
morir porque está desmoralizando a los defensores que quedan en la
ciudad, y también a todo el pueblo, al decirles esto. Este hombre no
trata de ayudar a esta gente, sólo va a destruirla''. \footnote{\textbf{38:4}
  Am 7,10}

\bibleverse{5} ``Bueno, puedes hacer lo que quieras con él'', respondió
el rey Sedequías. ``No puedo detenerte''.

\bibleverse{6} Así que tomaron a Jeremías y lo pusieron en la cisterna
que pertenecía a Malquías, el hijo del rey, que estaba en el patio de la
guardia. Bajaron a Jeremías con cuerdas a la cisterna. No tenía agua,
sólo barro, y Jeremías se hundió en el barro.

\hypertarget{el-rescate-de-jeremuxedas-por-el-etuxedope-ebedmelec}{%
\subsection{El rescate de Jeremías por el etíope
Ebedmelec}\label{el-rescate-de-jeremuxedas-por-el-etuxedope-ebedmelec}}

\bibleverse{7} Ebed-melec el cusita,\footnote{\textbf{38:7} Se suele
  pensar que ``Cus'' se refiere al alto Egipto. ``Ebed-melec'' significa
  ``siervo del rey''.} un funcionario real del palacio del rey,
descubrió que habían metido a Jeremías en la cisterna. El rey estaba
sentado en la puerta de Benjamín,\footnote{\textbf{38:7} Esto
  probablemente significa que el rey estaba decidiendo casos legales.}
\bibleverse{8} Entonces Ebed-melec salió del palacio y fue a hablar con
el rey \bibleverse{9} ``Mi señor el rey, todas estas cosas terribles que
estos hombres le han hecho al profeta Jeremías son malas. Lo han metido
en la cisterna, y allí morirá de hambre porque ya no queda pan en la
ciudad''.

\bibleverse{10} Entonces el rey dio la orden a Ebed-melec el cusita:
``Toma treinta hombres contigo y ve a sacar al profeta Jeremías de la
cisterna antes de que muera''.

\bibleverse{11} Ebed-melec tomó a los hombres y se dirigió al almacén
bajo el palacio. Tomó de allí algunos trapos y ropas viejas y se dirigió
a la cisterna donde los bajó con cuerdas a Jeremías. \bibleverse{12}
Ebed-melec, el cusita, llamó a Jeremías: ``Ponte estos trapos y ropas
viejas bajo los brazos para protegerte de las cuerdas''. Jeremías así lo
hizo,

\bibleverse{13} y con las cuerdas lo levantaron y lo sacaron de la
cisterna. Jeremías se quedó allí en el patio de la guardia.

\hypertarget{el-uxfaltimo-encuentro-secreto-de-jeremuxedas-con-el-rey}{%
\subsection{El último encuentro secreto de Jeremías con el
rey}\label{el-uxfaltimo-encuentro-secreto-de-jeremuxedas-con-el-rey}}

\bibleverse{14} Entonces el rey Sedequías mandó llamar al profeta
Jeremías y se reunió con él en la tercera entrada del Templo. ``Necesito
pedirte algo'', le dijo el rey a Jeremías. ``No debes ocultarme nada''.

\bibleverse{15} ``Si te lo digo, seguro que haces que me maten'',
respondió Jeremías. ``Aunque te diera un consejo, de todos modos no me
escucharías''.

\bibleverse{16} El rey Sedequías le prometió solemnemente a Jeremías en
privado: ``Vive el Señor, que nos dio esta vida, que no te haré matar,
ni te entregaré a los que quieren matarte''. \footnote{\textbf{38:16}
  Jer 38,4-5}

\bibleverse{17} Entonces Jeremías le dijo a Sedequías: ``Esto es lo que
dice el Señor Dios Todopoderoso, el Dios de Israel: `Si te entregas a
los oficiales del rey de Babilonia, entonces vivirás. Jerusalén no será
incendiada, y tú y tu familia sobrevivirán. \bibleverse{18} Pero si no
te entregas a los oficiales del rey de Babilonia, esta ciudad será
entregada a los babilonios. Ellos la incendiarán, y tú mismo no
escaparás de ser capturado'\,''.

\bibleverse{19} Pero el rey Sedequías le dijo a Jeremías: ``Tengo miedo
de la gente de Judá que se ha pasado a los babilonios, porque los
babilonios podrían entregarme a ellos para que abusen de mí''.

\bibleverse{20} ``No te entregarán'', respondió Jeremías. ``Si obedeces
lo que dice el Señor haciendo lo que te digo, las cosas te irán bien y
vivirás. \bibleverse{21} Pero si te niegas a entregarte, esto es lo que
me ha dicho el Señor: \bibleverse{22} Todas las mujeres que queden en el
palacio del rey de Judá serán sacadas y entregadas a los funcionarios
del rey de Babilonia Esas mujeres dirán: `¡Esos buenos amigos tuyos! Te
han acogido y te han conquistado. Tus pies se atascaron en el medio, por
lo que te abandonaron'.

\bibleverse{23} Todas tus esposas e hijos serán entregados a los
caldeos. Y tú mismo no escaparás, pues serás capturado por el rey de
Babilonia, y Jerusalén será incendiada''. \footnote{\textbf{38:23} Jer
  32,4; Jer 34,3}

\hypertarget{por-orden-del-rey-jeremuxedas-oculta-el-contenido-de-la-conversaciuxf3n-a-los-superiores}{%
\subsection{Por orden del rey, Jeremías oculta el contenido de la
conversación a los
superiores}\label{por-orden-del-rey-jeremuxedas-oculta-el-contenido-de-la-conversaciuxf3n-a-los-superiores}}

\bibleverse{24} Sedequías advirtió a Jeremías: ``Nadie puede enterarse
de esta conversación, pues de lo contrario morirás. \bibleverse{25} Si
los oficiales se enteran de que he hablado contigo, y vienen a
preguntarte: `¡Dinos de qué hablaron tú y el rey! No nos ocultes nada, o
te mataremos;' \bibleverse{26} entonces les dirás: `Le estaba pidiendo
al rey que me concediera mi petición de no devolverme a la casa de
Jonatán para morir allí'\,''.

\bibleverse{27} Cuando todos los oficiales vinieron a Jeremías queriendo
saber, él les repitió exactamente lo que el rey le había dicho que
dijera. Entonces no le preguntaron nada más, porque nadie había oído lo
que se había dicho.

\bibleverse{28} Jeremías permaneció allí, en el patio de la guardia,
hasta el día en que Jerusalén fue capturada.

\hypertarget{la-fortuna-de-jeremuxedas-al-conquistar-jerusaluxe9n-el-destino-de-sedequuxedas-asuxed-como-la-ciudad-y-el-pauxeds}{%
\subsection{La fortuna de Jeremías al conquistar Jerusalén; El destino
de Sedequías, así como la ciudad y el
país}\label{la-fortuna-de-jeremuxedas-al-conquistar-jerusaluxe9n-el-destino-de-sedequuxedas-asuxed-como-la-ciudad-y-el-pauxeds}}

\hypertarget{section-38}{%
\section{39}\label{section-38}}

\bibleverse{1} En el décimo mes del noveno año del reinado de Sedequías,
rey de Judá, Nabucodonosor, rey de Babilonia, y todo su ejército
llegaron a Jerusalén y la sitiaron. \bibleverse{2} El noveno día del
cuarto mes del undécimo año del reinado de Sedequías, la muralla de la
ciudad fue atravesada. \bibleverse{3} Todos los funcionarios del rey de
Babilonia entraron y se apoderaron de la ciudad, estableciendo su
cuartel general en la Puerta del Medio. Eran Nergal-sharezer de Samgar,
Nebo-sarsekim de Rabsaris, Nergal-sharezer de Rabmag,\footnote{\textbf{39:3}
  ``Rabsaris'' y ``Rabmag'' son funciones que no están claras. Sin
  embargo, se refieren claramente a altos funcionarios reales.} y todos
los demás funcionarios del rey de Babilonia. \bibleverse{4} Cuando
Sedequías, rey de Judá, y todos los defensores los vieron allí, huyeron.
Escaparon de la ciudad durante la noche por el jardín del rey, pasando
por la puerta entre las dos murallas, y tomaron el camino del
Arabá.\footnote{\textbf{39:4} ``El Arabá'': el Valle del Jordán.}

\bibleverse{5} Pero el ejército babilónico los persiguió y alcanzó a
Sedequías en las llanuras de Jericó. Lo capturaron y lo llevaron ante
Nabucodonosor, rey de Babilonia, en Ribla, en la tierra de Hamat, donde
lo juzgó y lo castigó. \bibleverse{6} El rey de Babilonia mandó matar a
los hijos de Sedequías mientras éste miraba, y también ejecutó a todos
los dirigentes de Judá allí en Riblá. \bibleverse{7} Luego hizo que le
sacaran los ojos a Sedequías, lo ataron con cadenas de bronce y lo
llevaron a Babilonia.

\bibleverse{8} Los babilonios quemaron el palacio del rey y las casas
del pueblo, y demolieron las murallas de Jerusalén. \bibleverse{9}
Entonces Nabuzaradán, el comandante de la guardia, se llevó a Babilonia
al resto del pueblo que se había quedado en la ciudad, junto con los que
habían desertado y se habían pasado a él. \bibleverse{10} Pero dejó en
la tierra de Judá a algunos de los más pobres que no tenían ninguna
propiedad. Les dio viñedos y campos en ese momento.

\hypertarget{buen-trato-a-jeremuxedas-por-parte-de-los-caldeos}{%
\subsection{Buen trato a Jeremías por parte de los
caldeos}\label{buen-trato-a-jeremuxedas-por-parte-de-los-caldeos}}

\bibleverse{11} Nabucodonosor, rey de Babilonia, había dado órdenes a
Nabuzaradán, comandante de la guardia, respecto a Jeremías, diciendo:
\bibleverse{12} ``Ve a buscar a Jeremías y vigila que no le pase nada
malo. Haz lo que él quiera''.

\bibleverse{13} Así que Nabuzaradán, el comandante de la guardia,
Nabushazban el Rabsaris, Nergal-sharezer el Rabmag, y todos los
capitanes del ejército del rey de Babilonia \bibleverse{14} sacaron a
Jeremías del patio de la guardia, y lo entregaron a Gedalías hijo de
Ahicam, hijo de Safán, para que lo llevara a su casa. Jeremías se quedó
allí con los suyos. \footnote{\textbf{39:14} Jer 38,28; Jer 40,5-6}

\hypertarget{hechizo-de-salvaciuxf3n-para-el-etuxedope-ebedmelech}{%
\subsection{Hechizo de salvación para el etíope
Ebedmelech}\label{hechizo-de-salvaciuxf3n-para-el-etuxedope-ebedmelech}}

\bibleverse{15} Durante el tiempo que Jeremías estuvo prisionero en el
patio de la guardia, le llegó un mensaje del Señor \bibleverse{16} ``Ve
y dile a Ebed-melec, el cusita, que esto es lo que dice el Señor
Todopoderoso, el Dios de Israel: Estoy a punto de cumplir la promesa que
hice contra esta ciudad -de perjudicarla y no ayudarla- lo verás por ti
mismo cuando suceda. \bibleverse{17} Pero cuando llegue ese día, voy a
salvarte, declara el Señor. No serás entregado a la gente a la que
temes. \bibleverse{18} Prometo rescatarte para que no te maten. Tu
recompensa será tu vida, porque confiaste en mí, declara el
Señor''.\footnote{\textbf{39:18} Job 5,20}

\hypertarget{liberaciuxf3n-de-jeremuxedas-del-cautiverio-caldeo-y-regreso-al-gobernador-gedaluxedas}{%
\subsection{Liberación de Jeremías del cautiverio caldeo y regreso al
gobernador
Gedalías}\label{liberaciuxf3n-de-jeremuxedas-del-cautiverio-caldeo-y-regreso-al-gobernador-gedaluxedas}}

\hypertarget{section-39}{%
\section{40}\label{section-39}}

\bibleverse{1} Este es el mensaje del Señor que llegó a Jeremías después
de que Nabuzaradán, el comandante de la guardia, lo liberara en Ramá
Nabuzaradán había descubierto a Jeremías atado con cadenas junto con
todos los prisioneros de Jerusalén y de Judá que eran llevados al exilio
en Babilonia. \footnote{\textbf{40:1} Jer 39,11-14} \bibleverse{2}
Cuando el comandante de la guardia encontró a Jeremías y le dijo: ``El
Señor, tu Dios, anunció que el desastre llegaría a este lugar,
\bibleverse{3} y ahora el Señor lo ha hecho; ha hecho justo lo que dijo
que haría. Esto les sucedió a ustedes porque pecaron contra el Señor y
no obedecieron lo que él dijo. \bibleverse{4} Pero fíjense que ahora les
quito las cadenas de las muñecas y los libero. Si quieren venir conmigo
a Babilonia, pueden venir, y yo los cuidaré. Pero si piensas que es una
mala idea venir conmigo a Babilonia, no tienes que ir más lejos. Mira,
eres libre de ir a cualquier parte del país. Ve a donde te convenga; haz
lo que creas que es correcto''. \bibleverse{5} Como Jeremías no
respondió de inmediato, Nabuzaradán continuó: ``Vuelve a Gedalías hijo
de Ahicam, hijo de Safán. Ha sido nombrado gobernador de Judá por el rey
de Babilonia. Puedes quedarte con él con tu gente, o puedes ir a donde
quieras''. El comandante de la guardia le dio una asignación de comida y
algo de dinero y lo dejó ir. \footnote{\textbf{40:5} Jer 39,14}

\bibleverse{6} Así que Jeremías se dirigió a Gedalías hijo de Ahicam, en
Mizpa, y se quedó con él con la gente que aún quedaba en el país.

\hypertarget{gedalja-reuxfane-a-los-juduxedos-en-una-colonia-en-mizpah}{%
\subsection{Gedalja reúne a los judíos en una colonia en
Mizpah}\label{gedalja-reuxfane-a-los-juduxedos-en-una-colonia-en-mizpah}}

\bibleverse{7} Los comandantes del ejército de Judea y sus hombres que
aún estaban en el campo se enteraron de que el rey de Babilonia había
nombrado a Gedalías hijo de Ahicam como gobernador del país y lo había
puesto a cargo de la gente más pobre del país: los hombres, las mujeres
y los niños que no habían sido exiliados a Babilonia. \bibleverse{8} Así
que ellos, junto con sus hombres, vinieron a Gedalías en Mizpa-Ismael
hijo de Netanías, Johanán y Jonatán los hijos de Carea, Seraías hijo de
Tanhumet, los hijos de Efai el netofatita, y Jezanías\footnote{\textbf{40:8}
  También se escribe Jaazanías. Véase 2 Reyes 25:23.} hijo del maacateo.
\footnote{\textbf{40:8} Jer 41,1; Jer 41,11} \bibleverse{9} Geladalías
hijo de Ahicam, hijo de Safán, les hizo una promesa solemne, diciendo:
``No se preocupen por servir a los babilonios. Quédense aquí en el país
y sirvan al rey de Babilonia, y las cosas les irán bien. \bibleverse{10}
Yo mismo me quedaré aquí en Mizpa para representarlos ante los
babilonios cuando vengan a reunirse con nosotros. Ustedes mismos deben
ocuparse de cosechar uvas y frutos de verano y aceite de oliva,
almacenarlos en tinajas y vivir en las ciudades que han ocupado''.

\bibleverse{11} Los habitantes de Judea que vivían en Moab, Amón, Edom y
todos los demás países se enteraron de que el rey de Babilonia había
dejado a algunas personas en Judá y que había nombrado a Gedalías hijo
de Ahicam, hijo de Safán, como su gobernador. \bibleverse{12} Así que
todos regresaron de los diferentes lugares en los que se habían
dispersado y se dirigieron a Gedalías en Mizpa, en Judá. Pudieron
cosechar una gran cantidad de uvas y frutos de verano.

\hypertarget{gedaluxedas-asesinada-por-ismael}{%
\subsection{Gedalías asesinada por
Ismael}\label{gedaluxedas-asesinada-por-ismael}}

\bibleverse{13} Johanán hijo de Carea y todos los comandantes de los
hombres del campo se presentaron ante Gedalías en Mizpa \bibleverse{14}
y le dijeron: ``¿Sabes que Baalis, rey de los amonitas, ha enviado a
Ismael hijo de Netanías para matarte?'' Pero Gedalías no les creyó.

\bibleverse{15} Johanán fue a hablar en privado con Gedalías en Mizpa.
``Déjame ir a matar a Ismael hijo de Netanías'', le dijo. ``Nadie se
enterará. ¿Por qué se le permitiría matarlo? Todo el pueblo de Judá que
se ha unido a ti se dispersaría, de modo que incluso los que han
sobrevivido aquí serían asesinados''.

\bibleverse{16} Pero Gedalías le dijo a Johanán: ``¡No lo hagas! Lo que
dices de Ismael no es cierto''.

\hypertarget{section-40}{%
\section{41}\label{section-40}}

\bibleverse{1} En el séptimo mes del año, Ismael hijo de Netanías, hijo
de Elisama, miembro de la familia real y uno de los principales
funcionarios del rey, vino con diez de sus hombres a ver a Gedalías en
Mizpa. Mientras comían juntos, \bibleverse{2} de repente Ismael y sus
diez hombres se levantaron y atacaron a Gedalías, matándolo, el que
había sido nombrado por el rey de Babilonia como gobernador del país.
\footnote{\textbf{41:2} Jer 40,5} \bibleverse{3} Ismael también mató a
todos los demás judíos que estaban con Gedalías en Mizpa, junto con los
soldados babilónicos que estaban allí.

\hypertarget{ismael-asesina-a-los-peregrinos-del-templo-israelita-y-se-retira-de-mizpa-con-numerosos-prisioneros}{%
\subsection{Ismael asesina a los peregrinos del templo israelita y se
retira de Mizpa con numerosos
prisioneros}\label{ismael-asesina-a-los-peregrinos-del-templo-israelita-y-se-retira-de-mizpa-con-numerosos-prisioneros}}

\bibleverse{4} Al día siguiente del asesinato de Gedalías, y antes de
que nadie lo supiera, \bibleverse{5} llegó un grupo de ochenta hombres
de Siquem, Silo y Samaria. Se habían afeitado la barba, se habían
rasgado la ropa y se habían cortado.\footnote{\textbf{41:5} Todos los
  símbolos de la pena y el luto extremos.} Llevaban ofrendas de grano e
incienso para el Templo del Señor. \bibleverse{6} Ismael salió a su
encuentro desde Mizpa, llorando a su paso. Cuando Ismael se encontró con
los hombres, les dijo: ``¡Vengan a ver lo que le pasó a Gedalías hijo de
Ahicam!''. \bibleverse{7} Pero cuando llegaron a la ciudad, Ismael y sus
hombres mataron a la mayoría de\footnote{\textbf{41:7} ``La mayoría
  de'': se ha añadido para mayor claridad a la luz del siguiente
  versículo.} ellos y arrojaron sus cuerpos a una cisterna.
\bibleverse{8} Pero diez de ellos suplicaron a Ismael: ``¡No nos mates!
Tenemos cosas buenas escondidas en los campos: trigo, cebada, aceite de
oliva y miel''. Así que Ismael no los mató junto con los demás.

\bibleverse{9} (La cisterna donde Ismael había arrojado todos los
cuerpos de los hombres que había matado, incluido Gedalías, era una
cisterna grande que el rey Asa había cavado debido a la amenaza de
ataque de Basá, rey de Israel. Ismael la llenó de cadáveres).
\footnote{\textbf{41:9} 1Re 15,16; 1Re 15,22}

\bibleverse{10} Luego Ismael tomó prisionera a toda la gente que quedaba
en Mizpa, incluidas las hijas del rey, así como a todos los demás que
vivían allí. Esta era la gente que Nabuzaradán, el comandante de la
guardia, había puesto bajo el cuidado de Gedalías. Ismael los tomó
prisioneros y partió para ir a los amonitas.

\hypertarget{johanuxe1n-libera-a-los-juduxedos-capturados-por-ismael-en-gabauxf3n-y-se-embarca-para-emigrar-a-egipto}{%
\subsection{Johanán libera a los judíos capturados por Ismael en Gabaón
y se embarca para emigrar a
Egipto}\label{johanuxe1n-libera-a-los-juduxedos-capturados-por-ismael-en-gabauxf3n-y-se-embarca-para-emigrar-a-egipto}}

\bibleverse{11} Johanán y todos los comandantes de los ejércitos que
estaban con él se enteraron de todos los crímenes de Ismael.
\bibleverse{12} Así que reunieron a todos sus hombres y fueron a atacar
a Ismael. Lo alcanzaron cerca del gran estanque de Gabaón.
\bibleverse{13} Cuando los prisioneros de Ismael vieron a Johanán y a
todos los comandantes del ejército que estaban con él, se alegraron.
\bibleverse{14} Todos los que Ismael había hecho prisioneros en Mizpa se
volvieron y corrieron hacia Johanán. \bibleverse{15} Ismael y ocho de
sus hombres lograron escapar de Johanán y huir hacia los amonitas.

\bibleverse{16} Entonces Johanán y todos los comandantes del ejército
que estaban con él se hicieron cargo de los sobrevivientes de Mizpa que
había rescatado de Ismael en Gabaón: los soldados, las mujeres, los
niños y los funcionarios de la corte que Ismael había hecho prisioneros
después de haber matado a Gedalías. \bibleverse{17} Partieron hacia
Gerut Quimán, cerca de Belén, y se quedaron allí, antes de partir hacia
Egipto \bibleverse{18} para alejarse de los babilonios. Tenían miedo de
lo que hicieran los babilonios porque Ismael había asesinado a Gedalías,
el gobernador del país nombrado por el rey de Babilonia.

\hypertarget{jeremuxedas-preguntuxf3-a-dios-en-nombre-de-sus-conciudadanos-sobre-la-emigraciuxf3n}{%
\subsection{Jeremías preguntó a Dios en nombre de sus conciudadanos
sobre la
emigración}\label{jeremuxedas-preguntuxf3-a-dios-en-nombre-de-sus-conciudadanos-sobre-la-emigraciuxf3n}}

\hypertarget{section-41}{%
\section{42}\label{section-41}}

\bibleverse{1} Entonces todos los comandantes del ejército, junto con
Johanán hijo de Carea, Jezanías\footnote{\textbf{42:1} ``Jezanías'':
  dado como ``Azarías'' en 43:2.} hijo de Osaías, y todos, desde el más
pequeño hasta el más importante, vinieron a \bibleverse{2} Jeremías el
profeta y le dijeron: ``Por favor, escucha nuestra petición.
\bibleverse{3} Ruega al Señor tu Dios por todos nosotros. Como puedes
ver, sólo quedamos unos pocos en comparación con los que había antes. En
tu oración, por favor, pídele al Señor tu Dios que te diga a dónde ir y
qué hacer''.

\bibleverse{4} ``Haré lo que me pides'', respondió Jeremías.
``Definitivamente, rezaré al Señor tu Dios como me has pedido, y te diré
todo lo que él diga. No te ocultaré nada''.

\bibleverse{5} Entonces le dijeron a Jeremías: ``Que el Señor sea un
testigo fiel y verdadero contra nosotros si no hacemos todo lo que el
Señor tu Dios te dice que debemos hacer. \bibleverse{6} Sea bueno o
malo, obedeceremos lo que diga el Señor, nuestro Dios, al que le pedimos
que hable. Así todo nos irá bien, porque estaremos obedeciendo lo que
dice el Señor nuestro Dios''.

\hypertarget{jeremuxedas-advierte-contra-la-emigraciuxf3n-en-nombre-de-dios}{%
\subsection{Jeremías advierte contra la emigración en nombre de
Dios}\label{jeremuxedas-advierte-contra-la-emigraciuxf3n-en-nombre-de-dios}}

\bibleverse{7} Diez días después llegó a Jeremías un mensaje del Señor.
\bibleverse{8} Éste convocó a Johanán, a todos los comandantes del
ejército y a todos, desde el más pequeño hasta el más importante.
\bibleverse{9} Jeremías les dijo: Esto es lo que el Señor, el Dios de
Israel, dice a aquellos de los que me enviaste a presentar tu petición:
\bibleverse{10} Si se quedan aquí mismo en este país, entonces los
edificaré y no los derribaré; los plantaré y no los desarraigaré, porque
estoy muy triste por el desastre que he provocado en ustedes.
\bibleverse{11} Sé que temes al rey de Babilonia, pero no tienes por qué
temerle, declara el Señor. Yo estoy contigo para salvarte y rescatarte
de él. \bibleverse{12} Seré misericordioso con ustedes, para que él sea
misericordioso con ustedes y los deje quedarse en su país.

\bibleverse{13} Pero si decís: ``No nos quedaremos aquí en este país'',
y con ello desobedecen lo que dice el Señor, su Dios; \bibleverse{14} o
si, en cambio, ustedes dicen: ``No, nos vamos a Egipto a vivir allí,
donde no experimentaremos la guerra ni oiremos el sonido de las
trompetas ni pasaremos hambre''; \bibleverse{15} entonces escuchen lo
que dice el Señor, sobrevivientes de Judá. Esto es lo que dice el Señor
Todopoderoso, el Dios de Israel: Si están absolutamente decididos a ir a
Egipto y vivir allí, \bibleverse{16} entonces la guerra que tanto temen
los alcanzará allí, y el hambre que tanto temen los perseguirá hasta
Egipto, y morirán allí. \bibleverse{17} Todos los que decidan ir a
Egipto y vivir allí morirán por la guerra, el hambre y la enfermedad. Ni
uno solo sobrevivirá ni escapará al desastre que haré caer sobre ellos.
\footnote{\textbf{42:17} Jer 29,17-18}

\hypertarget{jeremuxedas-repite-la-amenaza-divina}{%
\subsection{Jeremías repite la amenaza
divina}\label{jeremuxedas-repite-la-amenaza-divina}}

\bibleverse{18} Esto es lo que dice el Señor Todopoderoso, el Dios de
Israel: Así como mi furia se derramó sobre el pueblo que vivía en
Jerusalén, así se derramará mi furia sobre ustedes si van a Egipto. La
gente se horrorizará de lo que te ocurra, y te convertirás en una
palabra de maldición, en un insulto, en una expresión de condena. No
volverás a ver tu tierra natal.

\bibleverse{19} ``El Señor les ha dicho, sobrevivientes de Judá, que no
vayan a Egipto'', concluyó Jeremías. ``¡Tengan muy clara esta
advertencia que les hago hoy! \bibleverse{20} Han cometido un gran error
que les costará la vida al enviarme al Señor, su Dios, pidiéndole:
`Ruega al Señor, nuestro Dios, por nosotros, y haznos saber todo lo que
el Señor, nuestro Dios, dice y lo haremos'. \bibleverse{21} ``Hoy les he
dicho lo que él ha dicho, pero no han obedecido todo lo que el Señor, su
Dios, me ha enviado a deciros. \bibleverse{22} Así que debes saber que
sin duda vas a morir por la guerra y el hambre y la enfermedad en
Egipto, donde quieres ir a vivir''.

\hypertarget{desobediencia-de-los-advertidos-jeremuxedas-y-baruc-son-llevados-contra-su-voluntad-a-egipto}{%
\subsection{Desobediencia de los advertidos; Jeremías y Baruc son
llevados contra su voluntad a
Egipto}\label{desobediencia-de-los-advertidos-jeremuxedas-y-baruc-son-llevados-contra-su-voluntad-a-egipto}}

\hypertarget{section-42}{%
\section{43}\label{section-42}}

\bibleverse{1} Después de que Jeremías terminó de decirles a todos todo
lo que el Señor, su Dios, le había enviado a decir, \bibleverse{2}
Azarías hijo de Oseas, Johanán hijo de Carea, y todos los hombres
orgullosos y rebeldes\footnote{\textbf{43:2} ``Orgullosos y rebeldes'':
  La palabra utilizada aquí tiene el significado básico de orgullo, pero
  siempre en un sentido negativo que incluye la arrogancia, la
  presunción y la rebelión.} le dijo a Jeremías: ``¡Mientes! El Señor,
nuestro Dios, no te ha enviado para decirnos: `No deben irse a vivir a
Egipto'. \bibleverse{3} ¡No, es Baruc hijo de Nerías quien te ha puesto
en contra de nosotros para entregarnos a los babilonios para que nos
maten o nos exilien a Babilonia!''

\bibleverse{4} Así que Johanán hijo de Carea y todos los comandantes del
ejército se negaron a obedecer la orden del Señor de permanecer en la
tierra de Judá. \bibleverse{5} En lugar de eso, Johanán hijo de Carea y
todos los comandantes del ejército se llevaron a todos los que quedaban
del pueblo de Judá, los que habían regresado al país desde todas las
naciones donde habían sido dispersados. \bibleverse{6} Entre ellos había
hombres, mujeres y niños, las hijas del rey y todos los que Nabuzaradán,
el comandante de la guardia, había permitido que se quedaran con
Gedalías, así como Jeremías y Baruc. \bibleverse{7} Fueron a Egipto
porque se negaron a obedecer el mandato del Señor. Fueron hasta Tafnes.
\footnote{\textbf{43:7} 2Re 25,26}

\hypertarget{en-thachpanches-jeremuxedas-anuncia-la-inminente-sumisiuxf3n-de-egipto-por-nabucodonosor}{%
\subsection{En Thachpanches, Jeremías anuncia la inminente sumisión de
Egipto por
Nabucodonosor}\label{en-thachpanches-jeremuxedas-anuncia-la-inminente-sumisiuxf3n-de-egipto-por-nabucodonosor}}

\bibleverse{8} Un mensaje del Señor llegó a Jeremías en Tafnes:
\bibleverse{9} Mientras el pueblo de Judá observa, consigue algunas
piedras grandes y ponlas en el cemento del pavimento de ladrillos en el
camino de entrada al palacio del faraón en Tafnes. \bibleverse{10} Diles
que esto es lo que dice el Señor Todopoderoso, el Dios de Israel: Voy a
enviar a buscar a mi siervo Nabucodonosor, rey de Babilonia, y lo traeré
aquí. Pondré su trono sobre estas piedras que he colocado en el
pavimento, y él extenderá su tienda real sobre ellas. \bibleverse{11}
Vendrá y atacará a Egipto, trayendo la muerte a los que están destinados
a morir, la prisión a los que están destinados a ser encarcelados y la
espada a los que están destinados a ser muertos por la espada.
\footnote{\textbf{43:11} Jer 15,2} \bibleverse{12} Prenderé fuego a los
templos de los dioses de Egipto. Nabucodonosor los quemará y saqueará
sus ídolos. Limpiará la tierra de Egipto como un pastor limpia su manto
de pulgas, y saldrá ileso. \footnote{\textbf{43:12} Jer 46,25}

\bibleverse{13} Derribará los pilares sagrados del templo del
sol\footnote{\textbf{43:13} ``Templo del sol'': el famoso templo de
  Heliópolis (``Heliópolis'' es la palabra griega que significa ``Ciudad
  del Sol'').} en Egipto, y quemará los templos de los dioses de Egipto.

\hypertarget{el-discurso-amenazador-de-jeremuxedas-contra-la-idolatruxeda-de-los-juduxedos}{%
\subsection{El discurso amenazador de Jeremías contra la idolatría de
los
judíos}\label{el-discurso-amenazador-de-jeremuxedas-contra-la-idolatruxeda-de-los-juduxedos}}

\hypertarget{section-43}{%
\section{44}\label{section-43}}

\bibleverse{1} Este es el mensaje que llegó a Jeremías con respecto a
todo el pueblo de Judá que vivía en Egipto -en Migdol, Tafnes y Menfis-
y en el Alto Egipto. \footnote{\textbf{44:1} Jer 43,7} \bibleverse{2}
Esto es lo que dice el Señor Todopoderoso, el Dios de Israel: Ustedes
vieron el completo desastre que hice caer sobre Jerusalén y todas las
ciudades de Judá. Puedes ver cómo hoy están arruinadas y abandonadas
\bibleverse{3} por el mal que hicieron. Me hicieron enojar quemando
incienso y sirviendo a otros dioses que no habían conocido, y que tú y
tus antepasados tampoco habían conocido. \bibleverse{4} Yo les envié a
todos mis siervos los profetas una y otra vez para advertirles: ``No
hagan estas cosas ofensivas que yo odio''. \bibleverse{5} Pero ustedes
no quisieron escuchar ni prestar atención. No dejaron de hacer sus
maldades ni de quemar incienso en adoración de otros dioses.
\bibleverse{6} Por eso mi furia se desbordó y prendió fuego a las
ciudades de Judá y ardió en las calles de Jerusalén, convirtiéndolas en
las ruinas abandonadas que todavía son.

\bibleverse{7} Esto es lo que dice el Señor Dios Todopoderoso, el Dios
de Israel: ¿Por qué se hacen tanto daño eliminando de Judá a todo
hombre, mujer, niño y bebé, a fin de que no quede nadie? \bibleverse{8}
¿Por qué me hacen enojar con lo que hacen, quemando incienso a otros
dioses en Egipto, donde han venido a vivir? Porque si esto sucede serás
destruido, y te convertirás en una palabra de maldición, en una
expresión de condena entre todas las naciones de la tierra.
\bibleverse{9} ¿Acaso has olvidado la maldad de tus antepasados y la
maldad de los reyes de Judá y la maldad de sus esposas, así como tu
propia maldad y la maldad de tus esposas, todo ello practicado en el
país de Judá y en las calles de Jerusalén? \bibleverse{10} Incluso hasta
ahora no has mostrado ningún remordimiento ni reverencia. No has seguido
mis normas y reglamentos que te di a ti y a tus antepasados.

\bibleverse{11} Así que esto es lo que dice el Señor Todopoderoso, el
Dios de Israel: Estoy decidido a traer el desastre y a eliminar a todos
los de Judá. \bibleverse{12} Voy a destruirlos a ustedes, los que
quedaron, los que decidieron ir a Egipto a vivir allí. Morirán allí,
serán asesinados por espada o por hambre. Seas quien seas, desde el más
pequeño hasta el más importante, morirá por espada o por hambre; y te
convertirás en una palabra de maldición, en algo horrible, en un
insulto, en una expresión de condena. \bibleverse{13} A ustedes que
viven en Egipto los voy a castigar como castigué a Jerusalén, con
guerra, hambre y enfermedad. \bibleverse{14} Nadie que quede de Judá que
haya ido a vivir a Egipto escapará o sobrevivirá para volver al país de
Judá. Ustedes anhelan volver y vivir allí, pero nadie regresará, salvo
unos pocos rezagados.

\hypertarget{la-comunidad-especialmente-las-mujeres-declaran-abiertamente-que-quieren-servir-a-la-reina-del-cielo}{%
\subsection{La comunidad, especialmente las mujeres, declaran
abiertamente que quieren servir a la Reina del
Cielo}\label{la-comunidad-especialmente-las-mujeres-declaran-abiertamente-que-quieren-servir-a-la-reina-del-cielo}}

\bibleverse{15} Todos los hombres que sabían que sus esposas estaban
quemando incienso a otros dioses, y todas las mujeres que estaban allí,
una gran multitud de gente, los que vivían en Egipto y en el Alto
Egipto, le dijeron a Jeremías: \footnote{\textbf{44:15} Is 11,11}
\bibleverse{16} ``Aunque digas que este mensaje es del Señor, no te
vamos a escuchar'' \bibleverse{17} De hecho, vamos a hacer todo lo que
dijimos que haríamos. Quemaremos incienso a la Reina del Cielo y
ofreceremos libaciones para adorarla como lo hicimos antes, al igual que
nuestros padres, nuestros reyes y nuestros funcionarios que hicieron lo
mismo en las ciudades de Judá y en las calles de Jerusalén. Entonces
teníamos mucha comida y estábamos bien y no nos pasaba nada malo.
\bibleverse{18} Pero desde que dejamos de quemar incienso a la Reina del
Cielo y de derramar ofrendas de bebida para adorarla, lo hemos perdido
todo y estamos muriendo a causa de la guerra y el hambre.

\bibleverse{19} ``Además -añadieron las mujeres-, cuando quemábamos
incienso a la Reina del Cielo y derramábamos libaciones para adorarla,
lo hacíamos sin que nuestros maridos lo supieran, que horneábamos
pasteles estampados con su imagen\footnote{\textbf{44:19} Véase 7:18.} y
derramaron libaciones para adorarla''.

\hypertarget{el-rechazo-de-jeremuxedas-a-sus-excusas-y-su-alejamiento-de-ellos}{%
\subsection{El rechazo de Jeremías a sus excusas y su alejamiento de
ellos}\label{el-rechazo-de-jeremuxedas-a-sus-excusas-y-su-alejamiento-de-ellos}}

\bibleverse{20} Jeremías respondió a todo el pueblo, hombres y mujeres,
que le respondían: \bibleverse{21} ``Sobre ese incienso que quemaron a
otros dioses\footnote{\textbf{44:21} ``A otros dioses'': añadido para
  mayor claridad.} en las ciudades de Judá y en las calles de Jerusalén,
así como a tus padres, a tus reyes, a tus funcionarios y a la gente
común; ¿no crees que el Señor no se acordaría ni pensaría en ello?
\bibleverse{22} El Señor no pudo soportarlo más -las cosas malas que
hiciste y tus actos repugnantes-, así que tu país se convirtió en un
páramo deshabitado, un lugar de horror y una palabra de maldición para
los demás, como lo sigue siendo hoy. \bibleverse{23} Como pueden ver
hoy, el desastre que han experimentado ocurrió porque quemaron incienso
a otros dioses y pecaron contra el Señor, negándose a escuchar al Señor
o a seguir sus instrucciones, sus reglas y sus reglamentos''.

\bibleverse{24} Entonces Jeremías les dijo a todos, incluyendo a todas
las mujeres: ``Escuchen el mensaje del Señor, todos ustedes, gente de
Judá que vive aquí en Egipto. \bibleverse{25} Esto es lo que dice el
Señor Todopoderoso, el Dios de Israel: Ustedes y sus esposas han dicho
lo que van a hacer, y han hecho lo que dijeron. Dijisteis: `Vamos a
cumplir nuestra promesa de quemar incienso a la Reina del Cielo y de
derramar libaciones para adorarla'. ¡Así que adelante! ¡Hagan lo que han
dicho! ¡Cumplan sus promesas! \footnote{\textbf{44:25} Jer 44,17}

\hypertarget{la-uxfaltima-profecuxeda-de-la-desgracia-de-jeremuxedas-sobre-la-comunidad-egipcia}{%
\subsection{La última profecía de la desgracia de Jeremías sobre la
comunidad
egipcia}\label{la-uxfaltima-profecuxeda-de-la-desgracia-de-jeremuxedas-sobre-la-comunidad-egipcia}}

\bibleverse{26} ``Pero aun así, escuchen lo que dice el Señor, todo el
pueblo de Judá que vive aquí en Egipto: Les garantizo por todo lo que
soy, dice el Señor, que ninguno de ustedes que vive en Egipto usará
jamás mi nombre ni jurará: `Vive el Señor Dios'. \bibleverse{27} ``Me
ocuparé de ellos en el sentido malo y no en el bueno. Todo el pueblo de
Judá que esté en Egipto morirá por la espada o por el hambre, hasta ser
aniquilado. \bibleverse{28} Los que logren evitar ser muertos por la
espada regresarán a Judá desde Egipto. Pero sólo serán unos pocos, y
entonces todos los que quedaron de Judá y se fueron a vivir a Egipto
sabrán quién dice la verdad: ¡ellos o yo!

\bibleverse{29} ``Esta es su señal para demostrar que los voy a castigar
aquí, declara el Señor, para que sepan con certeza que mis amenazas
contra ustedes realmente se cumplirán. \bibleverse{30} Esto es lo que
dice el Señor: Mira, voy a entregar al faraón Hofra, rey de Egipto, a
sus enemigos que intentan matarlo, de la misma manera que entregué a
Sedequías, rey de Judá, a Nabucodonosor, rey de Babilonia, su enemigo
que intentaba matarlo''.\footnote{\textbf{44:30} 2Cró 36,13; 2Cró 36,20}

\hypertarget{palabras-de-advertencia-y-consuelo-de-jeremuxedas-a-baruc}{%
\subsection{Palabras de advertencia y consuelo de Jeremías a
Baruc}\label{palabras-de-advertencia-y-consuelo-de-jeremuxedas-a-baruc}}

\hypertarget{section-44}{%
\section{45}\label{section-44}}

\bibleverse{1} Esto es lo que el profeta Jeremías le dijo a Baruc hijo
de Nerías cuando escribió en un pergamino estos mensajes que le dictaba
Jeremías. (Esto sucedió en el cuarto año de Joacim hijo de Josías, rey
de Judá). \bibleverse{2} Esto es lo que el Señor, el Dios de Israel, te
dice, Baruc: \bibleverse{3} Te has estado quejando, diciendo: ``¡Tengo
tantos problemas porque el Señor me ha dado tristeza para agravar mi
dolor! Me he agotado con mis gemidos. No consigo ningún alivio''.

\bibleverse{4} Esto es lo que se le dijo a Jeremías que le dijera a
Baruc: Esto es lo que dice el Señor: En todo el país voy a derribar lo
que he construido y a arrancar lo que he plantado. \bibleverse{5} En tu
caso, ¿crees que tendrás un trato especial? ¡Deja de buscar algo así!
Voy a hacer caer el desastre sobre todo ser viviente, declara el Señor.
Sin embargo, te prometo que tu recompensa será que seguirás viviendo,
vayas donde vayas.\footnote{\textbf{45:5} Jer 39,18; Jer 43,6}

\hypertarget{profecuxedas-contra-pueblos-extranjeros-paganos}{%
\subsection{Profecías contra pueblos extranjeros
(paganos)}\label{profecuxedas-contra-pueblos-extranjeros-paganos}}

\hypertarget{section-45}{%
\section{46}\label{section-45}}

\bibleverse{1} En el cuarto año del reinado de Joacim, hijo de Josías,
rey de Judá, le llegó al profeta Jeremías un mensaje del Señor sobre las
naciones extranjeras.

\hypertarget{la-marcha-orgullosa-de-los-egipcios-su-derrota-en-carchemisch}{%
\subsection{La marcha orgullosa de los egipcios; su derrota en
Carchemisch}\label{la-marcha-orgullosa-de-los-egipcios-su-derrota-en-carchemisch}}

\bibleverse{2} Se trata del faraón Neco, rey de Egipto, y del ejército
egipcio que fue derrotado en Carquemis, en el río Éufrates, por
Nabucodonosor, rey de Babilonia. \bibleverse{3} Recojan sus escudos
grandes y pequeños, y avancen listos para la batalla. \bibleverse{4}
¡Pongan los arneses a los caballos y suban a sus carros; tomen sus
posiciones con los cascos puestos! Afilen sus lanzas y pónganse la
armadura. \bibleverse{5} ¿Por qué veo sus líneas rotas y en retirada?
Sus soldados están derrotados. Huyen tan rápido que ni siquiera miran
hacia atrás porque están tan aterrorizados por lo que sucede a su
alrededor, declara el Señor. \bibleverse{6} Ni siquiera los más rápidos
pueden huir; los soldados no pueden escapar. Allí, en el norte, junto al
Éufrates, caen y mueren. \bibleverse{7} ¿Quién es ese que viene,
subiendo como el Nilo, como ríos arremolinados cuyas aguas se desbordan?
\bibleverse{8} Egipto está subiendo como el Nilo; sus aguas se
arremolinan como ríos que se desbordan, presumiendo: ``Me levantaré y
arrasaré la tierra; destruiré las ciudades y a sus habitantes''.
\bibleverse{9} ¡Caballos, a la carga! ¡Carros, avancen como locos! Hagan
avanzar a la infantería: soldados de Etiopía y de Put llevando sus
escudos, arqueros de Lidia con sus arcos. \bibleverse{10} Pero éste es
el día del Señor Dios Todopoderoso, un día de retribución en el que se
vengará de sus enemigos. La espada destruirá hasta que esté satisfecha,
hasta que se haya hartado de su sangre. El Señor Dios Todopoderoso está
celebrando un sacrificio en el país del norte, junto al Éufrates.
\footnote{\textbf{46:10} Deut 32,42; Is 34,5} \bibleverse{11} ¡Ve a
buscar ungüento curativo en Galaad, virgen hija de Egipto! Pero todo lo
que uses para ayudarte fracasará, porque no hay nada que te cure.
\footnote{\textbf{46:11} Jer 8,22} \bibleverse{12} Las demás naciones
han oído cómo fuiste humillada en la derrota. Todos pueden oír tus
gritos de dolor. Los soldados caen unos sobre otros y mueren juntos.

\hypertarget{grave-angustia-de-guerra-en-egipto-como-resultado-de-la-conquista-de-nabucodonosor-palabras-de-consuelo-a-israel}{%
\subsection{Grave angustia de guerra en Egipto como resultado de la
conquista de Nabucodonosor; Palabras de consuelo a
Israel}\label{grave-angustia-de-guerra-en-egipto-como-resultado-de-la-conquista-de-nabucodonosor-palabras-de-consuelo-a-israel}}

\bibleverse{13} Este es el mensaje que el Señor dio al profeta Jeremías
sobre el ataque de Nabucodonosor, rey de Babilonia, a Egipto:
\bibleverse{14} ¡Grita una advertencia en Egipto! Avisen a todos en
Migdol, y en Menfis y Tafnes: Prepárense para defenderse, porque la
guerra está destruyendo todo a su alrededor. \bibleverse{15} ¿Por qué
huyó Apis, tu dios toro?\footnote{\textbf{46:15} Esto sigue el
  significado de ``el fuerte'', un sinónimo del dios Apis, que es
  también la forma en que los traductores de la Septuaginta leen la
  palabra. De lo contrario, la traducción sería: ``¿Por qué han sido
  barridos tus hombres fuertes?'' Apis era el dios toro adorado en
  Egipto, particularmente en Menfis.} No pudo mantenerse en pie porque
el Señor lo derribó. \bibleverse{16} Muchos soldados\footnote{\textbf{46:16}
  Se trataría de tropas mercenarias empleadas por los egipcios para
  luchar por ellos, como se desprende del contexto.} se tropiezan y caen
unos sobre otros y dicen: ``¡Vamos! Volvamos a casa, a nuestro pueblo,
donde nacimos, si no nos van a matar''. \bibleverse{17} Cuando lleguen
allí, dirán del faraón, rey de Egipto: ``Sólo hace mucho ruido. Ha
desperdiciado su oportunidad''. \bibleverse{18} Vivo yo, declara el Rey
que tiene el nombre de ``el Señor Todopoderoso'', el rey de
Babilonia\footnote{\textbf{46:18} ``Rey de Babilonia'': se añade para
  mayor claridad, ya que se menciona por primera vez en el versículo 13.}
vendrá. Es como el monte Tabor, que sobresale de los demás montes, como
el monte Carmelo en lo alto del mar. \bibleverse{19} ¡Prepara tus
maletas para el exilio, hija que vives en Egipto! Menfis va a ser
destruida, un lugar vacío donde nadie vive. \bibleverse{20} Egipto es
una hermosa vaca joven, pero un insecto urticante del norte viene a
atacarla. \bibleverse{21} Los soldados que Egipto contrató son como
terneros engordados para el matadero. Ellos también se retirarán. No se
pondrán de pie y lucharán; todos huirán. Se acerca su día de
destrucción; el tiempo en que serán castigados. \bibleverse{22} Los
egipcios retrocederán con un crujido como el de una serpiente que se
desliza, porque el enemigo los atacará con hachas, acercándose a ellos
como cortadores de madera que cortan árboles. \bibleverse{23} Los
derribarán como un bosque espeso, declara el Señor, porque los invasores
son como una nube de langostas: son tantos que no se pueden contar.
\bibleverse{24} El pueblo de Egipto será humillado. Será entregado a los
pueblos del norte.

\bibleverse{25} El Señor Todopoderoso, el Dios de Israel, dice Vigilen,
porque castigaré a Amón, el dios de Tebas, y al faraón. Castigaré al
pueblo de Egipto con sus dioses y sus reyes, y a todos los que confían
en el Faraón. \footnote{\textbf{46:25} Jer 43,12} \bibleverse{26} Voy a
entregarlos a los que quieren matarlos, a Nabucodonosor, rey de
Babilonia, y a sus oficiales. Pero después de que todo esto ocurra, la
gente vivirá en Egipto como antes, declara el Señor.

\hypertarget{palabra-de-consuelo-para-israel}{%
\subsection{Palabra de consuelo para
Israel}\label{palabra-de-consuelo-para-israel}}

\bibleverse{27} Pero ustedes, descendientes de Jacob, mi siervo, no
tienen que tener miedo. Israelitas, no tienen que desanimarse. Prometo
rescatarlos desde sus lejanos lugares de exilio, a sus descendientes
desde los países donde están cautivos. Volverán a casa, a una vida
tranquila y cómoda, libre de cualquier amenaza. \bibleverse{28}
Descendientes de Jacob, ¡no tengan miedo! declara el Señor, porque yo
estaré con ustedes. Destruiré por completo todas las naciones en las que
los he dispersado, pero no los destruiré del todo. Sin embargo, los
disciplinaré como se lo merecen, y pueden estar seguros de que no los
dejaré impunes.\footnote{\textbf{46:28} Jer 30,11}

\hypertarget{diciendo-de-la-tierra-filistea}{%
\subsection{Diciendo de la tierra
filistea}\label{diciendo-de-la-tierra-filistea}}

\hypertarget{section-46}{%
\section{47}\label{section-46}}

\bibleverse{1} Este es el mensaje del Señor que llegó al profeta
Jeremías sobre los filisteos antes de que el faraón atacara Gaza.
\footnote{\textbf{47:1} Is 14,29-32; Ezeq 25,15-17}

\bibleverse{2} Esto es lo que dice el Señor: ¡Mira cómo suben las aguas
del norte! Se convertirán en un río desbordado que barrerá el país y
todo lo que hay en él, inundando las ciudades y las casas de todos. El
pueblo clamará por ayuda; todos los que viven en el país llorarán,
\bibleverse{3} al oír el ruido de los sementales que cargan, el
traqueteo de los carros y el estruendo de sus ruedas. Los padres no
volverán a socorrer a sus hijos: no tienen fuerzas porque están
aterrorizados. \bibleverse{4} Ha llegado el día en que todos los
filisteos serán destruidos, en que Tiro y Sidón no tendrán más aliados
que los ayuden. El Señor va a destruir a los filisteos, a los que quedan
de la isla de Creta. \bibleverse{5} Los habitantes de Gaza se afeitarán
la cabeza;\footnote{\textbf{47:5} Una señal de luto, al igual que la
  automutilación mencionada en el mismo verso.} la ciudad de Ascalón
está en ruinas. Tú, que quedas en la llanura costera, ¿hasta cuándo
seguirás cortándote? \footnote{\textbf{47:5} Am 1,6-8; Sof 2,4; Zac 9,5;
  Jer 41,5; Jer 48,37}

\bibleverse{6} Oh espada del Señor, ¿cuándo vas a dejar de matar? Vuelve
a tu vaina. ¡Deja de matar y quédate ahí! \bibleverse{7} ¿Pero cómo va a
dejar de matar la espada cuando el Señor le ha dado órdenes de atacar
Ascalón y sus costas?

\hypertarget{la-devastaciuxf3n-de-la-tierra-y-la-huida-general}{%
\subsection{La devastación de la tierra y la huida
general}\label{la-devastaciuxf3n-de-la-tierra-y-la-huida-general}}

\hypertarget{section-47}{%
\section{48}\label{section-47}}

\bibleverse{1} Esto es lo que el Señor Todopoderoso, el Dios de Israel,
dice sobre Moab: La ciudad de Nebo está a punto de sufrir un desastre,
porque será destruida. La ciudad de Quiriatáim será capturada y
humillada; la fortaleza será derribada y su pueblo avergonzado.
\bibleverse{2} Ya nadie alaba a Moab. La gente de Hesbón está tramando:
``Destruiremos a Moab como nación. Gente del pueblo de los locos,
también los silenciaremos: serán atacados con espadas y perseguidos''.
\bibleverse{3} Escuchen los gritos de Joronayin: ``¡Violencia y terrible
destrucción!'' \bibleverse{4} Moab será aplastado. Oigan a los pequeños
que claman por ayuda. \bibleverse{5} La gente llora al subir a Luhit, y
al bajar a Joronayin resuenan sus gritos tristes ante la destrucción.
\bibleverse{6} ¡Huyan! ¡Sálvense ustedes mismos! ¡Sed como un escaso
tamarisco en el desierto!

\hypertarget{la-culpa-y-el-castigo-de-moab}{%
\subsection{La culpa y el castigo de
Moab}\label{la-culpa-y-el-castigo-de-moab}}

\bibleverse{7} Por haber puesto su confianza en lo que hacen y en lo que
poseen, ustedes también serán capturados. Tu dios Quemos será llevado al
exilio junto con sus sacerdotes y líderes. \footnote{\textbf{48:7} 1Re
  11,7} \bibleverse{8} Los invasores atacarán todas las ciudades; ni una
sola escapará a la destrucción. El valle quedará arruinado y la llanura
será destruida, porque el Señor ha hablado. \bibleverse{9} Levanten
lápidas en Moab,\footnote{\textbf{48:9} Esta es una de las líneas más
  problemáticas de Jeremías. Otras traducciones propuestas son: ``Poner
  sal en la tierra de Moab'', ``Dar alas a Moab para que pueda volar'',
  ``Dar una flor a Moab'', etc. La Septuaginta traduce ``Poner marcas en
  Moab'' o ``Poner señales en Moab''.} porque el país se convertirá en
un páramo. Sus ciudades se convertirán en ruinas donde nadie vive.
\bibleverse{10} Una maldición para los que no hagan bien el trabajo del
Señor. Una maldición para los que no usan sus espadas para matar.

\hypertarget{el-largo-descanso-y-el-descuido-de-moab-son-seguidos-por-penurias-y-destrucciuxf3n}{%
\subsection{El largo descanso y el descuido de Moab son seguidos por
penurias y
destrucción}\label{el-largo-descanso-y-el-descuido-de-moab-son-seguidos-por-penurias-y-destrucciuxf3n}}

\bibleverse{11} El pueblo de Moab ha vivido cómodamente desde que se
fundó el país. Son como el vino que no se ha alterado, que no se ha
vertido de un recipiente a otro. Así que el sabor y la fragancia siguen
siendo los mismos. No han experimentado el exilio. \bibleverse{12} Pero
cuidado, se acerca el tiempo, declara el Señor, en que les enviaré
``bodegueros'' que los verterán como vino. Vaciarán a los moabitas y los
destrozarán como si fueran tinajas. \bibleverse{13} Entonces los
moabitas se sentirán defraudados por Quemos, así como el pueblo de
Israel se sintió defraudado cuando confió en el ídolo del becerro de oro
en Betel.\footnote{\textbf{48:13} Véase 1 Reyes 12.} \bibleverse{14}
¿Cómo es posible que ustedes, los moabitas, digan: ``Somos héroes,
hombres fuertes listos para pelear en la batalla''? \bibleverse{15} Moab
va a ser destruido y sus ciudades conquistadas. Sus mejores jóvenes
serán asesinados, declara el Rey, cuyo nombre es el Señor Todopoderoso.
\bibleverse{16} La perdición de Moab está a punto de ocurrir; la
destrucción se precipita sobre ellos.

\hypertarget{la-desdichada-miseria-del-pauxeds-y-todas-sus-ciudades}{%
\subsection{La desdichada miseria del país y todas sus
ciudades}\label{la-desdichada-miseria-del-pauxeds-y-todas-sus-ciudades}}

\bibleverse{17} ¡Lloren por ellos, todas las naciones de los
alrededores, todos los que los conocen! ¡Que otros sepan cómo ha sido
aplastado el gran cetro, la vara que antes gobernaba con orgullo!
\bibleverse{18} Bajen de su gloria y siéntense en el suelo polvoriento,
ustedes que viven en Dibón, porque el destructor de Moab vendrá y los
atacará, destruyendo sus fortalezas. \bibleverse{19} Ponte al borde del
camino y observa, tú que vives en Aroer. Pregunten a los hombres y
mujeres que huyen para escapar: ``¿Qué ha pasado?'' . \bibleverse{20}
Moab ha sido humillado porque ha sido derrotado. ¡Llora y grita! ¡Griten
junto al río Arnón que Moab ha sido destruido! \bibleverse{21} El
castigo ha llegado a las ciudades de la altiplanicie: a Holón, Jahzá y
Mefá, \bibleverse{22} a Dibón, Nebo y Bet-diblatáim, \bibleverse{23} a
Quiriatáim, Bet-gamul y Bet-meón, \bibleverse{24} a Queriot, Bozrá y a
todas las ciudades de Moab, ya sean lejanas o cercanas. \bibleverse{25}
La fuerza de Moab ha desaparecido; su poder ha sido quebrantado, declara
el Señor.

\hypertarget{nuevo-anuncio-de-devastaciuxf3n-como-castigo-por-el-riduxedculo-de-moab-hacia-israel-y-por-la-arrogancia-de-moab}{%
\subsection{Nuevo anuncio de devastación como castigo por el ridículo de
Moab hacia Israel y por la arrogancia de
Moab}\label{nuevo-anuncio-de-devastaciuxf3n-como-castigo-por-el-riduxedculo-de-moab-hacia-israel-y-por-la-arrogancia-de-moab}}

\bibleverse{26} Embriaga al pueblo de Moab, porque ha desafiado al
Señor. Entonces se revolcarán en su propio vómito, mientras la gente se
ríe de ellos. \footnote{\textbf{48:26} Jer 25,15} \bibleverse{27} ¿No
solían los moabitas ridiculizar a los israelitas? Pero nunca se
descubrió que fueran ladrones, ¿verdad? Sin embargo, cada vez que hablan
de ellos, mueven la cabeza con desprecio. \bibleverse{28} Ustedes,
habitantes de Moab, huyan de sus ciudades, vayan a vivir entre las
rocas. Sean como una paloma que anida en los acantilados a la entrada de
un barranco. \bibleverse{29} Ya sabemos lo pomposos que son los
moabitas, lo extremadamente orgullosos y engreídos que son, pensando
arrogantemente en sí mismos. \bibleverse{30} Estoy bien enterado de lo
irrespetuosos que son, declara el Señor, pero no importa. Se jactan de
ser vacíos, y lo que hacen es igual de vacío. \bibleverse{31} Por eso
lloraré por Moab; gritaré por todos los moabitas; me lamentaré por el
pueblo de Kir-heres. \bibleverse{32} Lloraré por ustedes, gente de la
ciudad de Sibma con sus viñas, más de lo que lloro por la ciudad de
Jazer. Sus viñas se han extendido hasta el mar, y hasta Jazer. Pero el
destructor ha robado tu cosecha de frutos y uvas de verano.
\bibleverse{33} Ya no hay fiesta ni alegría en los huertos y campos de
Moab. He detenido el jugo de uva que fluye de los lagares. Nadie grita
de alegría al pisar las uvas. No, sus gritos ahora no son de alegría.

\hypertarget{nueva-descripciuxf3n-de-la-angustia-de-la-guerra-de-moab-varios-aditivos}{%
\subsection{Nueva descripción de la angustia de la guerra de Moab;
varios
aditivos}\label{nueva-descripciuxf3n-de-la-angustia-de-la-guerra-de-moab-varios-aditivos}}

\bibleverse{34} Los gritos de auxilio llegan desde Hesbón hasta Elealeh
y hasta Yahaza. Gritan desde Zoar hasta Horonaim y Eglat-Selisiyá, pues
hasta el agua del arroyo Nimrín se ha secado. \bibleverse{35} Voy a
acabar con los que en Moab sacrifican en los lugares altos paganos y
queman incienso a sus dioses, declara el Señor. \bibleverse{36} Por eso
me lamento en mi interior como una flauta que toca una canción triste
por Moab; como una flauta que toca una melodía fúnebre por el pueblo de
Kir-heres, porque han perdido todo lo que tenían de valor y por lo que
han trabajado. \bibleverse{37} Como muestra de su luto,\footnote{\textbf{48:37}
  ``Como muestra de su luto'': añadido para mayor claridad.} toda cabeza
está afeitada, toda barba está recortada, toda mano tiene un corte, y
toda cintura está vestida de cilicio. \footnote{\textbf{48:37} Jer 47,5}
\bibleverse{38} Todo el mundo está de luto en todas las azoteas y calles
de Moab, porque he destrozado el país como una vasija que nadie quiere,
declara el Señor. \bibleverse{39} ¡Moab está completamente destrozado!
¡Escúchalos llorar! ¡Vean cómo los moabitas se apartan avergonzados!
Todas las naciones de alrededor se horrorizan de lo que le ha sucedido
al país, y se burlan de su gente.

\hypertarget{el-juicio-y-sus-devastadoras-consecuencias-reconfortante-referencia-a-la-restauraciuxf3n-de-moab}{%
\subsection{El juicio y sus devastadoras consecuencias; reconfortante
referencia a la restauración de
Moab}\label{el-juicio-y-sus-devastadoras-consecuencias-reconfortante-referencia-a-la-restauraciuxf3n-de-moab}}

\bibleverse{40} Esto es lo que dice el Señor: Mira como un enemigo como
un águila se abalanza, extendiendo sus alas mientras ataca a Moab.
\bibleverse{41} Quiriot ha sido conquistada, y las fortalezas
capturadas. En ese momento los guerreros de Moab estarán tan asustados
como una mujer de parto. \bibleverse{42} Moab dejará de existir como
nación por haber desafiado al Señor. \bibleverse{43} A ustedes,
habitantes de Moab, lo que les espera es el terror, las trampas y los
lazos, declara el Señor. \bibleverse{44} Ustedes huirán aterrorizados y
caerán en una trampa, y cuando salgan de la trampa, quedarán atrapados
en un lazo. Esto les haré a los moabitas en el momento en que sean
castigados, declara el Señor. \footnote{\textbf{48:44} Is 24,17-18}
\bibleverse{45} Los que huyan quedarán indefensos en Hesbón, adonde
fueron en busca de protección, porque de Hesbón sale fuego, un fuego de
donde reinó Sijón. Quema todo el país de Moab y su pueblo
desafiante.\footnote{\textbf{48:45} Literalmente, ``Consume las frentes
  de Moab y los cráneos de los hijos del tumulto''.} \footnote{\textbf{48:45}
  Núm 21,28-29}

\bibleverse{46} ¡Qué desastre os ha sobrevenido a los moabitas! El
pueblo de Quemos ha sido aniquilado. Sus hijos e hijas han sido hechos
prisioneros y han ido al exilio. \bibleverse{47} Pero aun así, más
adelante, haré regresar al pueblo de Moab del exilio, declara el Señor.
Este es el final de la descripción del juicio sobre Moab.

\hypertarget{hablando-de-los-amonitas}{%
\subsection{Hablando de los amonitas}\label{hablando-de-los-amonitas}}

\hypertarget{section-48}{%
\section{49}\label{section-48}}

\bibleverse{1} Esto es lo que dice el Señor sobre los amonitas: ¿Acaso
los israelitas no tienen hijos? ¿No tienen herederos que hereden sus
bienes? Entonces, ¿por qué Milcom\footnote{\textbf{49:1} ``Milcom'': el
  nombre del principal dios amonita, también identificado como
  ``Molec''.} se apoderó del territorio perteneciente a la tribu de Gad?
¿Por qué su gente vive en sus ciudades? \footnote{\textbf{49:1} Ezeq
  25,2-7; Am 1,13-15; Sof 2,8-11; 1Re 11,5} \bibleverse{2} ¡Cuidado! Se
acerca el momento, declara el Señor, en que señalaré el ataque a la
ciudad amonita de Rabá.\footnote{\textbf{49:2} ``Rabá'': la capital
  amonita, hoy conocida como Amán (Jordania).} Se convertirá en un
montón de ruinas, y sus pueblos serán incendiados. Entonces los
israelitas expulsarán a los pueblos que se apoderaron de su tierra, dice
el Señor. \bibleverse{3} Lloren, pueblos de Hesbón, porque la ciudad de
Hai ha sido destruida. Clamen por ayuda, pueblo de Rabá. Pónganse ropas
de cilicio y comiencen a llorar; corran de un lado a otro dentro de los
muros de su ciudad, porque su dios Milcom irá al exilio junto con sus
sacerdotes y líderes. \bibleverse{4} ¿Por qué te jactas de que tus
valles son tan productivos, pueblo infiel? Confían en sus riquezas,
diciendo: ``¿Quién se atreverá a atacarnos?'' . \bibleverse{5} ¡Cuidado!
Voy a traer a las naciones de alrededor para aterrorizarlos, declara el
Señor Dios Todopoderoso. Todos ustedes serán expulsados y dispersados, y
nadie podrá volver a reunir a los refugiados. \bibleverse{6} Sin
embargo, más adelante los haré volver del exilio a los amonitas, declara
el Señor. \footnote{\textbf{49:6} Jer 48,47}

\hypertarget{dichos-sobre-los-edomitas}{%
\subsection{Dichos sobre los edomitas}\label{dichos-sobre-los-edomitas}}

\bibleverse{7} Esto es lo que dice el Señor Todopoderoso sobre
Edom:\footnote{\textbf{49:7} Para una profecía paralela contra Edom,
  véase Abdías, que utiliza un lenguaje similar.} ¿No queda ninguna
persona sabia en Temán?\footnote{\textbf{49:7} Temán era un territorio
  en el país de Edom.} ¿No hay ningún buen consejo de los que tienen
visión? ¿Se ha podrido su sabiduría? \footnote{\textbf{49:7} Is 21,11;
  Is 34,5-15; Ezeq 25,12-14; Am 1,11; Am 1,1-12; Abd 1,-1}
\bibleverse{8} ¡Vuelvan y huyan! Busquen un lugar donde esconderse,
pueblo de Dedán, porque voy a hacer caer el desastre sobre ustedes,
descendientes de Esaú, cuando los castigue. \bibleverse{9} Si viniera
gente a cosechar uvas, dejaría algunas, ¿no es así? Si vinieran ladrones
durante la noche, sólo robarían lo que quisieran, ¿no es así?
\bibleverse{10} Pero yo voy a desnudar todo el país, dejando a su gente
sin ningún lugar donde esconderse. Todos los descendientes de Esaú serán
destruidos, junto con sus parientes y amigos; todos desaparecerán.
\bibleverse{11} Sin embargo, puedes dejarme a tus huérfanos porque yo
los protegeré. Haz que tus viudas pongan su confianza en mí.

\bibleverse{12} Esto es lo que dice el Señor: Si los que no tuvieron que
beber de la copa del juicio tuvieron que hacerlo, ¿cómo no van a ser
castigados ustedes? No se quedarán impunes, porque también tienen que
beberla. \footnote{\textbf{49:12} Jer 25,15; Jer 25,21} \bibleverse{13}
Me hice la solemne promesa, declara el Señor, de que la ciudad de Bosra
se convertirá en un lugar que horrorice a la gente, en una completa
humillación, en un montón de ruinas y en un nombre que se use como
palabra de maldición. Todos sus pueblos circundantes también quedarán en
ruinas para siempre. \footnote{\textbf{49:13} Jer 44,12} \bibleverse{14}
He recibido este mensaje del Señor. Ha enviado un mensajero a las
naciones: ¡Prepárense para atacar a Edom! ¡Prepárense para la batalla!
\bibleverse{15} Vean cómo los haré insignificantes en comparación con
otras naciones; todos los mirarán con desprecio. \bibleverse{16} El
miedo que una vez causaron en los demás, y el orgullo que llevan por
dentro los ha engañado, dándoles un exceso de confianza, ustedes que
viven en las cimas de las montañas rocosas. Aunque hagan sus casas en lo
alto, fuera de su propio alcance, como un nido de águilas, incluso de
allí los derribaré, declara el Señor. \bibleverse{17} La gente se
horrorizará de lo que le ha ocurrido a Edom. Todos los que pasen por
allí se escandalizarán y se burlarán de todo su daño. \footnote{\textbf{49:17}
  Jer 50,13} \bibleverse{18} Así como Sodoma y Gomorra fueron
destruidas, junto con sus ciudades vecinas, dice el Señor, nadie vivirá
allí; quedarán deshabitadas. \footnote{\textbf{49:18} Is 1,9}
\bibleverse{19} ¡Cuidado! Voy a salir como un león de la maleza junto al
Jordán para atacar a los animales que pastan\footnote{\textbf{49:19}
  ``Ataque a los animales que pastan'': suministrado para mayor
  claridad.} los verdes pastos. De hecho, voy a expulsar a los edomitas
de su tierra muy rápidamente. ¿A quién elegiré para conquistarlos?
¿Quién es como yo? ¿Quién puede desafiarme? ¿Qué líder\footnote{\textbf{49:19}
  ``Líder'': literalmente, ``pastor''.} ¿podría oponerse a mí?
\footnote{\textbf{49:19} Jer 50,44} \bibleverse{20} Así que escucha lo
que el Señor ha planeado hacer a Edom y al pueblo de Temán: Sus hijos
serán arrastrados como corderos del rebaño, y por su culpa sus pastos se
convertirán en un páramo. \bibleverse{21} Cuando caigan, el ruido que
hagan hará temblar la tierra; sus gritos se oirán hasta el Mar Rojo.
\bibleverse{22} Mira como un enemigo como un águila vuela alto, y luego
se abalanza, extendiendo sus alas mientras ataca a Bosra. En ese momento
los guerreros de Edom estarán tan asustados como una mujer de parto.

\hypertarget{diciendo-de-damasco}{%
\subsection{Diciendo de Damasco}\label{diciendo-de-damasco}}

\bibleverse{23} Una profecía sobre Damasco: Las ciudades de Hamat y
Arpad\footnote{\textbf{49:23} Dos ciudades a una distancia considerable
  al norte de Damasco.} están perturbados, porque han recibido malas
noticias. Están temerosos, inquietos como el mar. Nada puede calmar sus
preocupaciones. \footnote{\textbf{49:23} Is 17,1; Am 1,3-5}
\bibleverse{24} Los habitantes de Damasco están desmoralizados; se
vuelven y huyen despavoridos, invadidos por el dolor y la angustia como
una parturienta. \bibleverse{25} ¿Por qué la ciudad alabada no está
desierta, la ciudad que me hizo feliz?\footnote{\textbf{49:25} Algunos
  creen que este verso no fue pronunciado por el Señor, sino por uno de
  los habitantes de la ciudad.} \bibleverse{26} Porque ese día sus
jóvenes morirán en sus calles, todos sus defensores serán asesinados,
declara el Señor de los Ejércitos. \bibleverse{27} Voy a prender fuego a
las murallas de Damasco; eso quemará las fortalezas de Ben-Hadad.

\hypertarget{sobre-los-cedarienses-y-otras-tribus-uxe1rabes}{%
\subsection{Sobre los Cedarienses y otras tribus
árabes}\label{sobre-los-cedarienses-y-otras-tribus-uxe1rabes}}

\bibleverse{28} Profecía sobre la tierra de Cedar y los reinos de Hazor
que fueron atacados por Nabucodonosor, rey de Babilonia. Esto es lo que
dice el Señor: Ve y ataca a Cedar; ¡destruye a los pueblos del oriente!
\bibleverse{29} ¡Toma sus tiendas y sus rebaños! Llévense las cortinas
de sus tiendas y todas sus posesiones. Tomen sus camellos para ustedes.
Gritadles: ``¡El terror está en todas partes!'' \bibleverse{30}
Corran!\footnote{\textbf{49:30} Este verso se dirige a los que son
  atacados.} ¡Aléjense lo más que puedan! Busquen un lugar donde
esconderse, pueblo de Hazor, declara el Señor. Porque Nabucodonosor, rey
de Babilonia, ha hecho planes para atacarlos y destruirlos. \footnote{\textbf{49:30}
  Jer 49,8} \bibleverse{31} Vayan y ataquen\footnote{\textbf{49:31} El
  Señor le habla de nuevo a Nabucodonosor.} esa nación autocomplaciente
que se cree segura, declara el Señor. No tienen puertas cerradas y no
tienen aliados. \bibleverse{32} Sus camellos y grandes rebaños serán un
botín para ti. Los dispersaré por todas partes, a este pueblo del
desierto que se recorta el pelo a los lados de la cabeza. Haré descender
sobre ellos un desastre desde todas las direcciones, declara el Señor.
\bibleverse{33} Hazor se convertirá en un lugar donde viven chacales, un
lugar abandonado para siempre. Nadie vivirá allí; quedará deshabitado.
\footnote{\textbf{49:33} Jer 9,10}

\hypertarget{diciendo-sobre-elam}{%
\subsection{Diciendo sobre Elam}\label{diciendo-sobre-elam}}

\bibleverse{34} Este es el mensaje del Señor que llegó al profeta
Jeremías sobre Elam. Esto fue al comienzo del reinado de Sedequías, rey
de Judá. \footnote{\textbf{49:34} Jer 25,25} \bibleverse{35} Esto es lo
que dice el Señor Todopoderoso: Mira, voy a destrozar los arcos de los
elamitas, el arma en la que confían para su poder. \bibleverse{36}
Traeré vientos de todas las direcciones para atacar a Elam, y los
dispersaré en todas direcciones. No habrá nación que no tenga algunos
exiliados de Elam. \bibleverse{37} Aplastaré a los elamitas frente a sus
enemigos, ante los que quieren matarlos. En mi furioso enojo haré caer
el desastre sobre ellos, declara el Señor. Los perseguiré con la espada
hasta destruirlos. \bibleverse{38} En Elam instalaré mi trono y
destruiré a su rey y a sus funcionarios, declara el Señor.
\bibleverse{39} Sin embargo, más adelante haré volver a los elamitas del
exilio, declara el Señor.\footnote{\textbf{49:39} Jer 49,6}

\hypertarget{la-cauxedda-de-babilonia-y-su-importancia-para-el-sufrido-pueblo-juduxedo}{%
\subsection{La caída de Babilonia y su importancia para el sufrido
pueblo
judío}\label{la-cauxedda-de-babilonia-y-su-importancia-para-el-sufrido-pueblo-juduxedo}}

\hypertarget{section-49}{%
\section{50}\label{section-49}}

\bibleverse{1} Este es el mensaje del Señor que le dijo al profeta
Jeremías que diera sobre Babilonia y la nación de Babilonia.\footnote{\textbf{50:1}
  ``La nación de Babilonia'': literalmente, ``la tierra de los
  caldeos''.} \footnote{\textbf{50:1} Is 13,-1; Is 14,1-14}
\bibleverse{2} ¡Cuéntale a todo el mundo la noticia! ¡Levanten un cartel
y grítenlo, no se contengan! ¡Díganles que Babilonia ha caído!
Su\footnote{\textbf{50:2} Como aquí y en el Apocalipsis la ciudad de
  Babilonia se identifica con el pronombre femenino, la alusión se
  mantiene aquí.} dios Bel será humillado; el poder de su dios Marduc
será quebrantado; todos los ídolos de Babilonia serán humillados y su
poder será quebrantado. \footnote{\textbf{50:2} Is 46,1} \bibleverse{3}
Una nación del norte vendrá a atacarla y convertirá el país en un
páramo. Nadie vivirá allí; tanto las personas como los animales la
abandonarán. \bibleverse{4} Ese es el momento en que los pueblos de
Israel y de Judá se unirán, llorando al ir a adorar al Señor, su Dios,
declara el Señor. \bibleverse{5} Preguntarán por el camino de Sión y se
pondrán en marcha en esa dirección. Llegarán y se comprometerán con el
Señor en un acuerdo eterno que no se olvidará jamás. \bibleverse{6} Mi
pueblo es una oveja perdida, extraviada por sus pastores, que la hacen
vagar sin rumbo por los montes. Van de un lugar a otro en los montes y
colinas, olvidando dónde solían descansar. \bibleverse{7} Todos los que
se cruzan con ellos los atacan. Sus enemigos declararon: ``¡Nosotros no
tenemos la culpa! Ellos son los que pecaron contra el Señor, su
verdadero lugar de descanso; el Señor que fue la esperanza de sus
antepasados''. \bibleverse{8} ¡Huye de la ciudad de Babilonia; aléjate
del país de Babilonia! Dirige el camino como los machos cabríos que
guían el rebaño. \footnote{\textbf{50:8} Jer 51,6; Jer 51,45}
\bibleverse{9} ¡Mira! Estoy reuniendo una coalición de fuertes naciones
del norte que vendrán a atacar a Babilonia. Se alinearán en la batalla
contra ella; será conquistada desde el norte. Sus flechas serán como las
de los mejores guerreros: ¡no fallan! \bibleverse{10} Babilonia será
saqueada; todo el que la saquee tendrá mucho botín, declara el Señor.
\bibleverse{11} Aunque por ahora ustedes los babilonios celebran y
cantan triunfalmente mientras saquean a mi pueblo especial, aunque por
ahora saltan como una vaca joven y juguetona que pisa el grano, y
relinchan como sementales, \bibleverse{12} van a traer vergüenza a su
madre, van a deshonrar a la que los dio a luz. Mirad cómo se convierte
en la menos importante de todas las naciones, en un desierto, en una
tierra desértica y seca. \bibleverse{13} A causa del castigo airado del
Señor, quedará desierta, completamente desolada. Todos los que pasen por
allí se horrorizarán de lo que le ha sucedido a Babilonia, y se burlarán
de todas sus heridas. \footnote{\textbf{50:13} Jer 51,37; Jer 49,17}
\bibleverse{14} Todos ustedes, arqueros, alinéense para la batalla
alrededor de Babilonia. Disparen contra ella. No ahorren sus flechas,
porque ella ha pecado contra el Señor. \bibleverse{15} ¡Griten gritos de
guerra contra ella desde todos los lados! Ella levanta las manos en
señal de rendición. Sus torres se han derrumbado; sus muros han sido
demolidos. El Señor le está devolviendo el favor, así que tú también
puedes devolvérselo: hazle lo mismo que a los demás. \bibleverse{16}
Detengan al sembrador de sembrar en el país de Babilonia, y detengan al
cosechador de mover la hoz para cosechar el grano. Bajo la amenaza de
las espadas de los enemigos, todos huyen a su pueblo, vuelven al lugar
de donde vinieron.

\hypertarget{la-desgracia-anterior-de-israel-y-la-salvaciuxf3n-posterior}{%
\subsection{La desgracia anterior de Israel y la salvación
posterior}\label{la-desgracia-anterior-de-israel-y-la-salvaciuxf3n-posterior}}

\bibleverse{17} Los israelitas son un rebaño perseguido y dispersado por
los leones. El primero en atacar fue el rey de Asiria; después,
Nabucodonosor, rey de Babilonia, les aplastó los huesos.

\bibleverse{18} Así que esto es lo que dice el Señor Todopoderoso, el
Dios de Israel: Voy a castigar al rey de Babilonia y a su país como
castigué al rey de Asiria. \bibleverse{19} Llevaré a los israelitas de
vuelta a sus tierras de pastoreo, para que se alimenten en el Carmelo y
en Basán, para que satisfagan sus apetitos en las colinas de Efraín y
Galaad. \bibleverse{20} Será entonces cuando se busquen las culpas y los
pecados de Israel y de Judá, pero no se encontrará ninguno, porque
perdonaré a los que queden que yo cuide, declara el Señor. \footnote{\textbf{50:20}
  Jer 31,34; Jer 33,8}

\hypertarget{contra-el-pauxeds-doble-desafuxedo}{%
\subsection{Contra el país ``doble
desafío''}\label{contra-el-pauxeds-doble-desafuxedo}}

\bibleverse{21} Vayan y ataquen la tierra de Merataim y a la gente que
vive en Pecod.\footnote{\textbf{50:21} No son nombres de lugares reales.
  Significan ``doble rebellion'' y ``castigo''. Sin embargo, su sonido
  es similar al de dos localidades de Babilonia.} Mátenlos con espadas,
apártenlos para la destrucción,\footnote{\textbf{50:21} ``Apártenlos
  para la destrucción'': un término especial que describía una
  dedicación al Señor, indicando que lo prometido debía ser entregado al
  Señor y no retenido.} junto con todo lo que dejan atrás. Asegúrense de
hacer todo lo que les he ordenado, declara el Señor. \bibleverse{22} El
ruido de la batalla se oye en el país, el ruido de la terrible
destrucción. \bibleverse{23} ¡Mira cómo el martillo de toda la tierra
yace hecho pedazos en el suelo! Las naciones miran con horror en qué se
ha convertido Babilonia. \bibleverse{24} Babilonia, te tendí una trampa,
y fuiste atrapada antes de que te dieras cuenta. Fuiste perseguida y
capturada porque luchaste contra el Señor. \bibleverse{25} En su ira, el
Señor abrió su arsenal para sacar sus armas, porque esto es lo que el
Señor Dios Todopoderoso está haciendo en el país de Babilonia.
\bibleverse{26} ¡Vengan a atacarla por todos lados! Abre sus graneros;
recoge el botín que tomes de ella como montones de grano. Apártenla para
destruirla; no dejen ningún sobreviviente. \bibleverse{27} Maten a todos
sus novillos\footnote{\textbf{50:27} ``Toros'': refiriéndose a los
  jóvenes que servían como soldados para Babilonia.} con la espada; que
sean masacrados. Qué desastre para ellos, porque ha llegado su hora de
ser castigados. \bibleverse{28} (Escuchen a los refugiados y a los
sobrevivientes que han regresado de Babilonia, anunciando en Sión: ``El
Señor, nuestro Dios, les está pagando por lo que le pasó a su Templo'').
\bibleverse{29} ¡Llama a los arqueros para que ataquen a Babilonia, sí,
a todos! Rodéenla por completo; no dejen que nadie escape. Págale por lo
que ha hecho, porque en su orgullo desafió al Señor, el Santo de Israel.
\footnote{\textbf{50:29} Jer 50,15} \bibleverse{30} Como resultado, sus
jóvenes serán asesinados en las calles; todos sus soldados morirán ese
día, declara el Señor. \bibleverse{31} ¡Cuidado, porque estoy contra
ustedes, pueblo arrogante! declara el Señor Dios Todopoderoso. Ha
llegado el momento en que te castigaré. \bibleverse{32} Ustedes, los
arrogantes, tropezarán y caerán. No habrá nadie que os levante. Prenderé
fuego a sus ciudades y quemaré todo lo que los rodea.

\hypertarget{anunciaciuxf3n-a-israel}{%
\subsection{Anunciación a Israel}\label{anunciaciuxf3n-a-israel}}

\bibleverse{33} Esto es lo que dice el Señor Todopoderoso: El pueblo de
Israel y de Judá está siendo maltratado. Todos los que los capturaron se
aferran a ellos, negándose a dejarlos ir. \bibleverse{34} Pero el que
los rescata es poderoso; el Señor Todopoderoso es su nombre. Él los
defenderá a ellos y a su causa, para que traiga paz a la tierra, pero
problemas al pueblo de Babilonia.

\hypertarget{espada-diciendo}{%
\subsection{Espada diciendo}\label{espada-diciendo}}

\bibleverse{35} Una espada se levanta para atacar a los babilonios,
declara el Señor, lista para atacar a los que viven en Babilonia, a sus
funcionarios y a sus sabios. \bibleverse{36} Se ha levantado una espada
para atacar a sus falsos profetas, y ellos se convertirán en tontos. Se
ha levantado una espada para atacar a sus soldados, y quedarán
aterrorizados. \bibleverse{37} Se ha levantado una espada para atacar a
sus caballos y carros, junto con todos los soldados extranjeros que la
acompañan, y se convertirán en mujeres asustadas. Se ha levantado una
espada para atacar sus almacenes de tesoros, y serán saqueados.
\bibleverse{38} Una sequía ha golpeado sus ríos, y se secarán. Porque es
un país lleno de imágenes paganas. Estos horribles ídolos vuelven locos
a sus adoradores.

\hypertarget{varias-adiciones-y-repeticiones}{%
\subsection{Varias adiciones y
repeticiones}\label{varias-adiciones-y-repeticiones}}

\bibleverse{39} En consecuencia, vivirán allí animales del desierto y
hienas, y será un hogar para los búhos. Estará deshabitado para siempre:
no se habitará de una generación a otra. \bibleverse{40} De la misma
manera que Dios destruyó a Sodoma y Gomorra y a sus ciudades vecinas,
declara el Señor, nadie vivirá allí, nadie se quedará allí. \footnote{\textbf{50:40}
  Gén 19,24-25} \bibleverse{41} ¡Mira! Un ejército avanza desde el
norte. Una gran nación y muchos reyes vienen contra ti desde las tierras
lejanas. \footnote{\textbf{50:41} Jer 50,9} \bibleverse{42} Llevan arcos
y jabalinas. Son crueles y despiadados. Cuando gritan es como si el mar
rugiera. Montan a caballo y atacan en formación contra ustedes, pueblo
de Babilonia. \footnote{\textbf{50:42} Jer 6,23}

\bibleverse{43} El rey de Babilonia ha oído las noticias y está
aterrorizado. Está sobrecogido de miedo, con dolor como una mujer de
parto. \bibleverse{44} Tengan cuidado!\footnote{\textbf{50:44} El
  siguiente pasaje es paralelo al que se da contra Edom en 49:19-21.}
Voy a salir como un león de la maleza junto al Jordán para atacar a los
animales que pastan\footnote{\textbf{50:44} ``Ataque a los animales que
  pastan'': suministrado para mayor claridad.} los verdes pastos. De
hecho, voy a expulsar a los babilonios de su tierra muy rápidamente. ¿A
quién elegiré para conquistarlos? ¿Quién es como yo? ¿Quién puede
desafiarme? ¿Qué líder\footnote{\textbf{50:44} ``Líder'': literalmente,
  ``pastor''.} ¿podría oponerse a mí? \bibleverse{45} Así que escucha lo
que el Señor ha planeado hacer a Babilonia y al país de Babilonia: Sus
hijos serán arrastrados como corderos del rebaño, y por su culpa sus
pastos se convertirán en un páramo. \bibleverse{46} El sonido de la
captura de Babilonia hará temblar la tierra; sus gritos se escucharán en
todas las naciones.

\hypertarget{el-juicio-de-babilonia-ha-sido-decidido}{%
\subsection{El juicio de Babilonia ha sido
decidido}\label{el-juicio-de-babilonia-ha-sido-decidido}}

\hypertarget{section-50}{%
\section{51}\label{section-50}}

\bibleverse{1} Esto es lo que dice el Señor: Miren, voy a levantar un
viento destructor contra Babilonia y contra el pueblo de
Babilonia.\footnote{\textbf{51:1} Literalmente, ``Leb-kamai'', una
  palabra clave para Babilonia.} \bibleverse{2} Enviaré soldados
extranjeros a atacar a Babilonia para arrasar con\footnote{\textbf{51:2}
  El ``aventamiento'' es el proceso por el cual se lanza el grano al
  aire para que el viento pueda llevarse la paja.} ella y convertirán su
país en un desierto; la atacarán desde todas las direcciones cuando
llegue su momento de dificultad. \footnote{\textbf{51:2} Jer 15,7}
\bibleverse{3} El arquero no necesita usar su arco; el soldado de
infantería no necesita ponerse su armadura.\footnote{\textbf{51:3} El
  hebreo de esta línea se ha interpretado de diferentes maneras. Lo más
  probable es que diga que el ejército atacante pudo conquistar
  Babilonia sin muchos problemas.} No perdones a sus jóvenes soldados;
destina todo su ejército a la destrucción!\footnote{\textbf{51:3} Véase
  en la nota 50:21 el significado de ``apartar para la destrucción''.}
\bibleverse{4} Caerán heridos en sus calles, muertos en el país de
Babilonia. \bibleverse{5} Israel y Judá no han sido abandonados por su
Dios, el Señor Todopoderoso, aunque pecaron contra el Santo de Israel en
todo su país. \bibleverse{6} ¡Escapen de Babilonia! ¡Huyan por sus
vidas! No se dejen atrapar por su castigo para que no mueran, porque
este es el momento en que el Señor le pagará por sus pecados.
\footnote{\textbf{51:6} Jer 50,8; Apoc 18,4; Is 48,20}

\hypertarget{babilonia-el-vaso-de-dios-la-sentencia-de-muerte-para-babilonia}{%
\subsection{Babilonia el vaso de Dios: la sentencia de muerte para
Babilonia}\label{babilonia-el-vaso-de-dios-la-sentencia-de-muerte-para-babilonia}}

\bibleverse{7} En otro tiempo, Babilonia era una copa de oro que el
Señor tenía en su mano. Ella emborrachó a toda la tierra. Las naciones
bebieron su vino y por eso se volvieron locas. \footnote{\textbf{51:7}
  Jer 25,15; Apoc 17,4; Apoc 18,3} \bibleverse{8} Ahora, de repente,
Babilonia ha caído. Ha sido hecha pedazos. Lloren por ella; consigan
algún tratamiento para su dolor. Tal vez pueda ser curada. \footnote{\textbf{51:8}
  Apoc 18,2} \bibleverse{9} ``Tratamos de curarla, pero no se pudo. Así
que renuncien a ella. Todos debemos volver a casa, al lugar de donde
venimos. La noticia de su castigo ha llegado a todas partes, hasta el
cielo. \bibleverse{10} El Señor nos ha animado y apoyado.\footnote{\textbf{51:10}
  ``Nos ha animado y apoyado''. Esto se traduce a menudo como
  ``vindicación'', sin embargo, esto tiene frecuentemente el significado
  de ser ``probado correcto'', que no es el caso aquí para los
  israelitas que fueron al exilio porque no estaban bien con Dios.}
Vamos, digamos a la gente de Jerusalén lo que el Señor ha hecho por
nosotros''.

\hypertarget{la-ciudad-es-asaltada-por-la-decisiuxf3n-de-dios}{%
\subsection{La ciudad es asaltada por la decisión de
Dios}\label{la-ciudad-es-asaltada-por-la-decisiuxf3n-de-dios}}

\bibleverse{11} ¡Afilen las flechas! Recojan los escudos!\footnote{\textbf{51:11}
  O ``¡Llena las aljabas!''} El Señor ha animado a los reyes de los
medos, porque su plan está dirigido a la destrucción de Babilonia. El
Señor les está pagando por lo que le sucedió a su Templo.
\bibleverse{12} Levanten la bandera de señal para atacar las murallas de
Babilonia; refuercen la guardia; hagan que los centinelas ocupen sus
puestos; preparen la emboscada. El Señor planeó y cumplió sus amenazas
contra el pueblo de Babilonia. \bibleverse{13} Ustedes, que viven junto
a muchas aguas y tienen tantas riquezas, este es el momento de su fin:
su vida será cortada. \footnote{\textbf{51:13} Apoc 17,1}
\bibleverse{14} El Señor Todopoderoso juró con su propia vida, diciendo:
Me aseguraré de llenarte de tantos soldados enemigos que serán como
langostas. Gritarán al celebrar su victoria sobre ti.

\hypertarget{alabado-sea-el-seuxf1or-dios-de-israel}{%
\subsection{Alabado sea el Señor, Dios de
Israel}\label{alabado-sea-el-seuxf1or-dios-de-israel}}

\bibleverse{15} Fue Dios quien hizo la tierra con su poder. Creó el
mundo con su sabiduría y con su entendimiento puso los cielos.
\bibleverse{16} Las aguas de los cielos llueven con estruendo por orden
suya. Él hace que las nubes se eleven por toda la tierra. Hace que el
rayo acompañe a la lluvia, y envía el viento desde sus almacenes.
\bibleverse{17} Todos son estúpidos; no saben nada. Todos los
trabajadores del metal se avergüenzan de los ídolos que fabrican. Porque
sus imágenes hechas de metal fundido son fraudulentas: ¡no están vivas!
\bibleverse{18} Son inútiles, un objeto de risa. Serán destruidos en el
momento de su castigo. \bibleverse{19} El Dios de Jacob no es como esos
ídolos, pues es el creador de todo, incluso de su propio pueblo, que es
especial para él. El Señor Todopoderoso es su nombre.

\hypertarget{martillo-diciendo-el-juicio-de-babilonia-en-su-significado-histuxf3rico-mundial}{%
\subsection{Martillo diciendo; el juicio de Babilonia en su significado
histórico
mundial}\label{martillo-diciendo-el-juicio-de-babilonia-en-su-significado-histuxf3rico-mundial}}

\bibleverse{20} Tú\footnote{\textbf{51:20} ``Tú'': refiriéndose a
  Babilonia.} eres mi garrote de guerra, el arma que uso en la batalla.
Te uso para destruir naciones; te uso para destruir reinos. \footnote{\textbf{51:20}
  Jer 50,23; Is 10,5} \bibleverse{21} Te uso para destruir caballos y
sus jinetes; te uso para destruir carros y sus conductores.
\bibleverse{22} Te uso para destruir hombres y mujeres; te uso para
destruir ancianos y jóvenes; te uso para destruir jóvenes y niñas.
\bibleverse{23} Te uso para destruir a los pastores y sus rebaños; te
uso para destruir a los agricultores y su ganado; te uso para destruir a
los gobernantes y a los funcionarios del Estado.

\bibleverse{24} Delante de ti voy a pagar a Babilonia y a todos los que
viven en Babilonia todo el mal que le hicieron a Jerusalén, declara el
Señor. \bibleverse{25} Ten cuidado, porque estoy contra ti, monstruo
destructor que arrasa el mundo entero, declara el Señor. Llegaré a
atacarte; te haré rodar por los acantilados; te convertiré en una
montaña de ceniza. \bibleverse{26} Nadie podrá ni siquiera encontrar una
piedra angular o una piedra de cimentación entre tus ruinas, porque
serás destruido por completo, declara el Señor.

\hypertarget{descripciuxf3n-de-la-conquista-de-la-ciudad}{%
\subsection{Descripción de la conquista de la
ciudad}\label{descripciuxf3n-de-la-conquista-de-la-ciudad}}

\bibleverse{27} ¡Izad una bandera de señales en el país! ¡Toca la
trompeta de llamada a la guerra entre las naciones! Preparen a las
naciones para atacarla; convoquen a los reinos para marchar contra ella:
Ararat, Minni y Asquenaz. Elige a un comandante para que dirija los
ejércitos que la atacarán; envía a la caballería de batalla como una
nube de langostas. \footnote{\textbf{51:27} Is 13,3; Gén 10,3}
\bibleverse{28} Haz que los ejércitos de las naciones se preparen para
la batalla contra ella. Esto se aplica a los reyes de los medos, a sus
jefes y a todos sus oficiales, y a todos los países que gobiernan.
\bibleverse{29} La tierra se estremece y tiembla, porque el Señor está
decidido a cumplir lo que amenazó contra Babilonia: convertirla en un
páramo donde nadie viva. \bibleverse{30} Los defensores de Babilonia han
renunciado a luchar; se han quedado sentados en sus fortalezas. Están
agotados; se han vuelto como mujeres asustadas. Las casas de Babilonia
están en llamas; los barrotes que aseguran sus puertas han sido
destrozados. \bibleverse{31} Un corredor entrega su mensaje a otro para
que lo lleve; un mensajero sigue a otro mensajero, todos ellos alertan
al rey de Babilonia de la noticia de que su ciudad ha sido completamente
conquistada, \bibleverse{32} los cruces de los ríos han sido capturados,
los pantanos incendiados y sus soldados están aterrorizados.

\bibleverse{33} Esto es lo que dice el Señor Todopoderoso, el Dios de
Israel: El pueblo de Babilonia es como una era cuando el grano es
pisoteado. Su tiempo de cosecha llegará muy pronto.

\hypertarget{la-deuda-de-babilonia-con-jerusaluxe9n-y-la-venganza-de-dios}{%
\subsection{La deuda de Babilonia con Jerusalén y la venganza de
Dios}\label{la-deuda-de-babilonia-con-jerusaluxe9n-y-la-venganza-de-dios}}

\bibleverse{34} Nabucodonosor, rey de Babilonia, me masticó\footnote{\textbf{51:34}
  ``Me'': Refiriéndose a Jerusalén.} y me secó, dejándome tan vacía como
un frasco sin nada dentro. Me engulló como si fuera un monstruo,
llenándose de las partes más sabrosas de mí y tirando el resto.
\bibleverse{35} ``Babilonia debe cargar con la responsabilidad de los
violentos ataques contra nosotros'', dicen los habitantes de Sión. ``El
pueblo de Babilonia debe cargar con la responsabilidad de la sangre
derramada en mi ciudad'', dice Jerusalén. \bibleverse{36} Esto es lo que
dice el Señor: Mira cómo presento tu caso por ti y hago que tus enemigos
paguen por lo que te hicieron. Voy a secar su río y sus manantiales.
\bibleverse{37} Babilonia se convertirá en un montón de escombros, un
hogar para chacales, un lugar que horroriza a la gente, un lugar del que
se burlan, un lugar donde nadie vive. \bibleverse{38} Los babilonios
rugirán juntos como leones poderosos y gruñirán como cachorros de león.
\bibleverse{39} Pero mientras se despiertan sus pasiones, les serviré un
banquete y los embriagaré. Celebrarán tanto que se desmayarán y no
despertarán jamás, declara el Señor. \bibleverse{40} Los bajaré como
corderos para ser sacrificados, como carneros y cabras.

\hypertarget{lamento-por-la-cauxedda-de-la-ciudad-combinada-con-advertencias-a-israel}{%
\subsection{Lamento por la caída de la ciudad combinada con advertencias
a
Israel}\label{lamento-por-la-cauxedda-de-la-ciudad-combinada-con-advertencias-a-israel}}

\bibleverse{41} ¿Cómo puede ser? Babilonia\footnote{\textbf{51:41}
  Literalmente ``Sesac'': un nombre en clave para Babilonia.} ¡ha caído!
¡La ciudad más famosa del mundo ha sido conquistada! ¡En qué horrible
espectáculo se ha convertido Babilonia para todos los que la miran!
\bibleverse{42} Es como si el mar se hubiera desbordado sobre Babilonia,
cubriéndola de olas. \bibleverse{43} Las ciudades de Babilonia están en
ruinas, convertidas en un páramo seco y desértico donde nadie vive, ni
siquiera pasa por allí. \bibleverse{44} Yo castigaré a Bel\footnote{\textbf{51:44}
  ``Bel'': el dios babilónico principal.} en Babilonia. Le obligaré a
escupir lo que se ha tragado. La gente de otras naciones ya no correrá a
adorarle. Hasta la muralla de Babilonia ha caído. \footnote{\textbf{51:44}
  Jer 50,2} \bibleverse{45} ¡Pueblo mío, salid de ella! Cada uno de
ustedes, sálvense de la furiosa ira del Señor. \footnote{\textbf{51:45}
  Jer 51,6} \bibleverse{46} No pierdan el valor, y no tengan miedo
cuando oigan diferentes rumores que circulan por el país. Habrá un rumor
un año, y otro al siguiente, hablando de revolución violenta, de un
gobernante luchando contra otro. \bibleverse{47} Mira, se acerca el
momento en que castigaré a los ídolos de Babilonia. Todo el país será
humillado; estará lleno de los cadáveres de los asesinados.
\bibleverse{48} Entonces todos en el cielo y en la tierra celebrarán con
gritos de alegría lo que le ha sucedido a Babilonia, porque los
destructores del norte vendrán a atacarla, declara el Señor. \footnote{\textbf{51:48}
  Apoc 18,20} \bibleverse{49} Babilonia tiene que caer por culpa de los
israelitas y de la gente de otras naciones que ella mató.
\bibleverse{50} Aquellos de ustedes que han logrado escapar de ser
asesinados, ¡salgan ahora! ¡No se demoren! Recuerden al Señor en este
lugar lejano; piensen en Jerusalén. \bibleverse{51} ``Estamos
avergonzados porque nos han burlado, y nos agarramos la cabeza con
vergüenza porque los extranjeros entraron en los lugares santos del
Templo del Señor'',\footnote{\textbf{51:51} Estas palabras son
  expresadas por los israelitas.} \bibleverse{52} Por eso, manténganse
alerta, declara el Señor, porque se acerca el momento en que la
castigaré por adorar a los ídolos, y el sonido de los heridos gimiendo
se escuchará en todo el país. \bibleverse{53} Aunque Babilonia pudiera
subir al cielo para fortalecer sus altas fortalezas, los que yo envíe a
atacarla la destruirán, declara el Señor.

\hypertarget{conclusiuxf3n-y-revisiuxf3n}{%
\subsection{Conclusión y revisión}\label{conclusiuxf3n-y-revisiuxf3n}}

\bibleverse{54} Un grito viene de Babilonia; el ruido de una terrible
destrucción viene del país de Babilonia. \bibleverse{55} Porque el Señor
va a destruir a Babilonia; pondrá fin a su fanfarronería. Las olas del
ejército atacante rugirán como el mar que se estrella; el ruido de sus
gritos resonará por todas partes. \bibleverse{56} Un destructor viene a
atacar a Babilonia. Sus soldados serán tomados prisioneros y sus arcos
serán destrozados, porque el Señor es un Dios que castiga con justicia;
definitivamente les pagará. \footnote{\textbf{51:56} Deut 32,35}
\bibleverse{57} Embriagaré a sus dirigentes y sabios, así como a sus
comandantes, oficiales y soldados. Entonces se desmayarán y no
despertarán jamás, declara el Rey, cuyo nombre es el Señor Todopoderoso.
\footnote{\textbf{51:57} Jer 51,39}

\bibleverse{58} Esto es lo que dice el Señor Todopoderoso: Las enormes
murallas de Babilonia serán derribadas hasta los cimientos y sus altas
puertas serán quemadas. Todo lo que el pueblo trabajó no servirá para
nada; las otras naciones que vinieron a ayudar se agotarán, sólo para
ver que lo que han hecho arderá en llamas. \footnote{\textbf{51:58} Hab
  2,13}

\hypertarget{la-maldiciuxf3n-sobre-babilonia-es-hundida-en-el-uxe9ufrates-por-seraja-en-nombre-de-jeremuxedas}{%
\subsection{La maldición sobre Babilonia es hundida en el Éufrates por
Seraja en nombre de
Jeremías}\label{la-maldiciuxf3n-sobre-babilonia-es-hundida-en-el-uxe9ufrates-por-seraja-en-nombre-de-jeremuxedas}}

\bibleverse{59} Este es el mensaje que el profeta Jeremías dio a
Seraías, hijo de Nerías, hijo de Maseías, cuando acompañó al rey
Sedequías de Judá a Babilonia en el cuarto año del reinado de Sedequías.
Seraías era el asistente personal del rey. \footnote{\textbf{51:59} Jer
  36,4} \bibleverse{60} Jeremías había escrito en un pergamino una
descripción de todos los desastres que vendrían a Babilonia: todas estas
palabras escritas aquí sobre Babilonia. \bibleverse{61} Jeremías le dijo
a Seraías: ``Cuando llegues a Babilonia, asegúrate de leer en voz alta
todo lo que está escrito aquí, \bibleverse{62} y anuncia: `Señor, has
prometido destruir este lugar para que no quede nada, ni personas ni
animales. De hecho, quedará desierta para siempre'. \bibleverse{63}
``Cuando termines de leer este rollo en voz alta, ata una piedra a él y
arrójala al Éufrates. \footnote{\textbf{51:63} Apoc 18,21}

\bibleverse{64} ``Luego di: `Así es como Babilonia se hundirá y no
volverá a levantarse, por el desastre que estoy haciendo caer sobre
ella. Su pueblo se cansará'\,''.\footnote{\textbf{51:64} Esta frase
  final parece estar fuera de lugar.} Este es el final de los mensajes
de Jeremías.

\hypertarget{la-desolaciuxf3n-de-sedequuxedas-el-sitio-de-jerusaluxe9n-escape-y-captura-del-rey-juzgado-penal-de-ribla}{%
\subsection{La desolación de Sedequías; El sitio de Jerusalén; Escape y
captura del rey; Juzgado penal de
Ribla}\label{la-desolaciuxf3n-de-sedequuxedas-el-sitio-de-jerusaluxe9n-escape-y-captura-del-rey-juzgado-penal-de-ribla}}

\hypertarget{section-51}{%
\section{52}\label{section-51}}

\bibleverse{1} Sedequías tenía veintiún años cuando llegó a ser rey, y
reinó en Jerusalén durante once años. Su madre se llamaba Hamutal, hija
de Jeremías y era de Libna. \bibleverse{2} Hizo lo malo ante los ojos
del Señor, tal como lo había hecho Joacim. \bibleverse{3} Todo esto
sucedió en Jerusalén y en Judá, a causa de la ira del Señor, hasta que
finalmente los desterró de su presencia. Sedequías se rebeló contra el
rey de Babilonia.

\bibleverse{4} En el noveno año del reinado de Sedequías, el décimo día
del décimo mes, Nabucodonosor, rey de Babilonia, atacó Jerusalén con
todo su ejército. Acampó alrededor de la ciudad y construyó rampas de
asedio contra las murallas. \bibleverse{5} La ciudad permaneció sitiada
hasta el undécimo año del rey Sedequías.

\bibleverse{6} Para el noveno día del cuarto mes, la hambruna en la
ciudad era tan grave que la gente no tenía nada que comer.
\bibleverse{7} Entonces se rompió la muralla de la ciudad, y todos los
soldados huyeron, escapando de noche por la puerta entre las dos
murallas junto al jardín del rey, aunque los babilonios tenían la ciudad
rodeada. Se dirigieron en dirección al Arabá,\footnote{\textbf{52:7}
  ``Arabá'': el valle del Jordán.} \bibleverse{8} pero el ejército
babilónico persiguió al rey y lo alcanzó en las llanuras de Jericó. Todo
su ejército se había dispersado y lo había abandonado. \bibleverse{9}
Capturaron al rey y lo llevaron ante el rey de Babilonia en Ribla, donde
lo condenó. \bibleverse{10} El rey de Babilonia masacró a los hijos de
Sedequías mientras él miraba, y también mató a los funcionarios de Judá
allí en Riblá. \bibleverse{11} Luego le sacó los ojos a Sedequías y lo
ató con grilletes de bronce. El rey de Babilonia lo llevó a Babilonia y
lo encarceló allí hasta el día de su muerte. \footnote{\textbf{52:11}
  Jer 32,5}

\hypertarget{conquista-y-destrucciuxf3n-de-la-ciudad-saqueo-e-incendio-del-templo-traslado-de-habitantes-a-babilonia-ejecuciones-en-ribla}{%
\subsection{Conquista y destrucción de la ciudad; Saqueo e incendio del
templo; Traslado de habitantes a Babilonia; Ejecuciones en
Ribla}\label{conquista-y-destrucciuxf3n-de-la-ciudad-saqueo-e-incendio-del-templo-traslado-de-habitantes-a-babilonia-ejecuciones-en-ribla}}

\bibleverse{12} El día diez del mes quinto, en el año diecinueve de
Nabucodonosor, rey de Babilonia, entró en Jerusalén Nabuzaradán,
comandante de la guardia, oficial del rey de Babilonia. \bibleverse{13}
Quemó el Templo del Señor, el palacio real y todos los grandes edificios
de Jerusalén. \bibleverse{14} Todo el ejército babilónico bajo el mando
del comandante de la guardia derribó todos los muros alrededor de
Jerusalén. \bibleverse{15} Nabuzaradán, el comandante de la guardia,
deportó a algunos de los pobres y a los que quedaban en la ciudad,
incluso a los que se habían pasado al lado del rey de Babilonia, así
como al resto de los artesanos. \bibleverse{16} Pero Nabuzaradán
permitió que otros de los pobres que habían quedado en el campo se
quedaran cuidando las viñas y los campos.

\bibleverse{17} Los babilonios rompieron en pedazos las columnas de
bronce, los carros móviles y el mar de bronce que pertenecían al Templo
del Señor, y se llevaron todo el bronce a Babilonia. \bibleverse{18}
También se llevaron todas las ollas, las palas, los apagadores de
lámparas, las tazas de aspersión y todos los demás objetos de bronce que
se utilizaban en el servicio del Templo. \bibleverse{19} El comandante
de la guardia se llevó las palanganas, los incensarios, los aspersores,
las ollas, los candelabros, los platos y los tazones, todo lo que era de
oro puro o de plata.

\bibleverse{20} La cantidad de bronce que provenía de las dos columnas,
del Mar, de los doce toros de bronce que estaban debajo y de los carros
móviles que Salomón había hecho para el Templo del Señor, todo esto
pesaba más de lo que se podía medir. \footnote{\textbf{52:20} 1Re
  7,15-47} \bibleverse{21} Cada columna tenía dieciocho codos de altura
y doce codos de circunferencia. Eran huecas, con paredes de cuatro dedos
de espesor. \bibleverse{22} El capitel de bronce de una de las columnas
tenía una altura de cinco codos y una red de granadas de bronce a su
alrededor. La segunda columna era igual, y también tenía una red
decorativa. \bibleverse{23} Había noventa y seis granadas de bronce
alrededor de cada columna. Encima de la red había un total de cien
granadas.

\bibleverse{24} El comandante de la guardia tomó como prisioneros a
Seraías, el jefe de los sacerdotes, al sacerdote Sofonías, segundo en
rango, y a los tres porteros del Templo. \bibleverse{25} De los que
quedaron en la ciudad tomó al oficial a cargo de los soldados y a siete
de los consejeros del rey. También se llevó al secretario del comandante
del ejército, encargado de convocar al pueblo para el servicio militar,
y a otros sesenta hombres que estaban presentes en la ciudad.
\bibleverse{26} Nabuzaradán, el comandante de la guardia, los tomó y los
llevó ante el rey de Babilonia en Ribla. \bibleverse{27} El rey de
Babilonia los hizo ejecutar en Ribla, en la tierra de Hamat. Entonces el
pueblo de Judá tuvo que abandonar su tierra.

\hypertarget{los-nuxfameros-de-los-que-se-llevaron}{%
\subsection{Los números de los que se
llevaron}\label{los-nuxfameros-de-los-que-se-llevaron}}

\bibleverse{28} Este es un registro del número de personas que
Nabucodonosor llevó al exilio. En el séptimo año de su reinado se llevó
a 3. 023 judíos. \footnote{\textbf{52:28} 2Re 24,11-16} \bibleverse{29}
En su decimoctavo año, Nabucodonosor se llevó a otros 832 de Jerusalén.
\bibleverse{30} En el año veintitrés del reinado de Nabucodonosor,
Nabuzaradán, el comandante de la guardia, se llevó a otros 745 judíos,
haciendo un total de 4. 600.

\hypertarget{jojachuxedn-indultado-tras-treinta-y-siete-auxf1os-de-prisiuxf3n}{%
\subsection{Jojachín indultado tras treinta y siete años de
prisión}\label{jojachuxedn-indultado-tras-treinta-y-siete-auxf1os-de-prisiuxf3n}}

\bibleverse{31} En el año en que Evil-merodac se convirtió en rey de
Babilonia, liberó a Joaquín, rey de Judá, de la prisión. Esto sucedió el
día veinticinco del duodécimo mes del trigésimo séptimo año del
destierro de Joaquín, rey de Judá. \bibleverse{32} El rey de Babilonia
lo trató bien y le dio una posición de honor superior a la de los otros
reyes que estaban con él en Babilonia. \bibleverse{33} Así, Joaquín pudo
quitarse la ropa de la cárcel y comió con frecuencia en la mesa del rey
durante el resto de su vida. \bibleverse{34} El rey le dio a Joaquín una
pensión diaria por el resto de su vida hasta que murió.
