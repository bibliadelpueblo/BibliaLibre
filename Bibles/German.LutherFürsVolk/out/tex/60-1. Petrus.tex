\hypertarget{section}{%
\section{1}\label{section}}

\bibverse{1} Petrus, ein Apostel Jesu Christi, den erwählten Fremdlingen
hin und her in Pontus, Galatien, Kappadozien, Asien und Bithynien,
\bibverse{2} nach der Vorsehung Gottes, des Vaters, durch die Heiligung
des Geistes, zum Gehorsam und zur Besprengung mit dem Blut Jesu Christi:
Gott gebe euch viel Gnade und Frieden!

\bibverse{3} Gelobet sei Gott und der Vater unseres Herrn Jesu Christi,
der uns nach seiner großen Barmherzigkeit wiedergeboren hat zu einer
lebendigen Hoffnung durch die Auferstehung Jesu Christi von den Toten,
\footnote{\textbf{1:3} Kol 1,5} \bibverse{4} zu einem unvergänglichen
und unbefleckten und unverwelklichen Erbe, das behalten wird im Himmel
\footnote{\textbf{1:4} Kol 1,12} \bibverse{5} euch, die ihr aus Gottes
Macht durch den Glauben bewahrt werdet zur Seligkeit, welche bereitet
ist, dass sie offenbar werde zu der letzten Zeit. \footnote{\textbf{1:5}
  Joh 10,28} \bibverse{6} In derselben werdet ihr euch freuen, die ihr
jetzt eine kleine Zeit, wo es sein soll, traurig seid in mancherlei
Anfechtungen, \footnote{\textbf{1:6} 1Petr 5,10; 2Kor 4,17} \bibverse{7}
auf dass euer Glaube rechtschaffen und viel köstlicher erfunden werde
denn das vergängliche Gold, das durchs Feuer bewährt wird, zu Lob, Preis
und Ehre, wenn nun offenbart wird Jesus Christus, \footnote{\textbf{1:7}
  Spr 17,3; Mal 3,3} \bibverse{8} welchen ihr nicht gesehen und doch
liebhabt und nun an ihn glaubet, wiewohl ihr ihn nicht sehet, und werdet
euch freuen mit unaussprechlicher und herrlicher Freude \footnote{\textbf{1:8}
  Joh 20,29; 2Kor 5,7} \bibverse{9} und das Ende eures Glaubens
davonbringen, nämlich der Seelen Seligkeit.

\bibverse{10} Nach dieser Seligkeit haben gesucht und geforscht die
Propheten, die von der Gnade geweissagt haben, die auf euch kommen
sollte, \footnote{\textbf{1:10} Lk 10,24} \bibverse{11} und haben
geforscht, auf welche und welcherlei Zeit deutete der Geist Christi, der
in ihnen war und zuvor bezeugt hat die Leiden, die über Christum kommen
sollten, und die Herrlichkeit darnach; \footnote{\textbf{1:11} Jes
  53,-1; Ps 22,-1} \bibverse{12} welchen es offenbart ist. Denn sie
haben's nicht sich selbst, sondern uns dargetan, was euch nun verkündigt
ist durch die, die euch das Evangelium verkündigt haben durch den
heiligen Geist, der vom Himmel gesandt ist; was auch die Engel gelüstet
zu schauen. \footnote{\textbf{1:12} Eph 3,10}

\bibverse{13} Darum so begürtet die Lenden eures Gemütes, seid nüchtern
und setzet eure Hoffnung ganz auf die Gnade, die euch angeboten wird
durch die Offenbarung Jesu Christi, \footnote{\textbf{1:13} Lk 12,35-36}
\bibverse{14} als gehorsame Kinder, und stellet euch nicht gleichwie
vormals, da ihr in Unwissenheit nach den Lüsten lebtet; \footnote{\textbf{1:14}
  Röm 12,2} \bibverse{15} sondern nach dem, der euch berufen hat und
heilig ist, seid auch ihr heilig in allem eurem Wandel. \bibverse{16}
Denn es steht geschrieben: „Ihr sollt heilig sein, denn ich bin
heilig.``

\bibverse{17} Und sintemal ihr den zum Vater anruft, der ohne Ansehen
der Person richtet nach eines jeglichen Werk, so führet euren Wandel,
solange ihr hier wallet, mit Furcht \bibverse{18} und wisset, dass ihr
nicht mit vergänglichem Silber oder Gold erlöst seid von eurem eitlen
Wandel nach väterlicher Weise, \footnote{\textbf{1:18} 1Kor 6,20; 1Kor
  7,23; 1Petr 4,3} \bibverse{19} sondern mit dem teuren Blut Christi als
eines unschuldigen und unbefleckten Lammes, \footnote{\textbf{1:19} Joh
  1,29; Jes 53,7; Hebr 9,14} \bibverse{20} der zwar zuvor ersehen ist,
ehe der Welt Grund gelegt ward, aber offenbart zu den letzten Zeiten um
euretwillen, \footnote{\textbf{1:20} Röm 16,25-26} \bibverse{21} die ihr
durch ihn glaubet an Gott, der ihn auferweckt hat von den Toten und ihm
die Herrlichkeit gegeben, auf dass ihr Glauben und Hoffnung zu Gott
haben möchtet.

\bibverse{22} Und machet keusch eure Seelen im Gehorsam der Wahrheit
durch den Geist zu ungefärbter Bruderliebe und habt euch untereinander
inbrünstig lieb aus reinem Herzen, \bibverse{23} als die da
wiedergeboren sind, nicht aus vergänglichem, sondern aus unvergänglichem
Samen, nämlich aus dem lebendigen Wort Gottes, das da ewiglich bleibt.
\bibverse{24} Denn „alles Fleisch ist wie Gras und alle Herrlichkeit der
Menschen wie des Grases Blume. Das Gras ist verdorrt und die Blume
abgefallen; \footnote{\textbf{1:24} Jak 1,10; Jak 1,1-11} \bibverse{25}
aber des Herrn Wort bleibt in Ewigkeit.`` Das ist aber das Wort, welches
unter euch verkündigt ist. \# 2 \bibverse{1} So leget nun ab alle
Bosheit und allen Betrug und Heuchelei und Neid und alles Afterreden,
\bibverse{2} und seid begierig nach der vernünftigen, lauteren Milch als
die jetzt geborenen Kindlein, auf dass ihr durch dieselbe zunehmet,
\footnote{\textbf{2:2} Mt 18,3; Hebr 5,12-13} \bibverse{3} so ihr anders
geschmeckt habt, dass der Herr freundlich ist, \footnote{\textbf{2:3} Ps
  34,9} \bibverse{4} zu welchem ihr gekommen seid als zu dem lebendigen
Stein, der von Menschen verworfen ist, aber bei Gott ist er auserwählt
und köstlich. \footnote{\textbf{2:4} Ps 118,22; Mt 21,42} \bibverse{5}
Und auch ihr, als die lebendigen Steine, bauet euch zum geistlichen
Hause und zum heiligen Priestertum, zu opfern geistliche Opfer, die Gott
angenehm sind durch Jesum Christum. \footnote{\textbf{2:5} Eph 2,21-22;
  Hebr 3,6; Röm 12,1} \bibverse{6} Darum steht in der Schrift: „Siehe
da, ich lege einen auserwählten, köstlichen Eckstein in Zion; und wer an
ihn glaubt, der soll nicht zu Schanden werden.``

\bibverse{7} Euch nun, die ihr glaubet, ist er köstlich; den Ungläubigen
aber ist der Stein, den die Bauleute verworfen haben, der zum Eckstein
geworden ist,

\bibverse{8} ein Stein des Anstoßens und ein Fels des Ärgernisses; denn
sie stoßen sich an dem Wort und glauben nicht daran, wozu sie auch
gesetzt sind. \footnote{\textbf{2:8} Lk 2,34; Röm 9,33}

\bibverse{9} Ihr aber seid das auserwählte Geschlecht, das königliche
Priestertum, das heilige Volk, das Volk des Eigentums, dass ihr
verkündigen sollt die Tugenden des, der euch berufen hat von der
Finsternis zu seinem wunderbaren Licht; \footnote{\textbf{2:9} 2Mo 19,6;
  Offb 1,6; Eph 5,8} \bibverse{10} die ihr vordem nicht ein Volk waret,
nun aber Gottes Volk seid, und vordem nicht in Gnaden waret, nun aber in
Gnaden seid. \footnote{\textbf{2:10} Röm 9,24-26}

\bibverse{11} Liebe Brüder, ich ermahne euch als die Fremdlinge und
Pilgrime: enthaltet euch von fleischlichen Lüsten, welche wider die
Seele streiten, \footnote{\textbf{2:11} Ps 39,13} \bibverse{12} und
führet einen guten Wandel unter den Heiden, auf dass die, die von euch
afterreden als von Übeltätern, eure guten Werke sehen und Gott preisen,
wenn es nun an den Tag kommen wird. \footnote{\textbf{2:12} Mt 5,16}

\bibverse{13} Seid untertan aller menschlichen Ordnung um des Herrn
willen, es sei dem König, als dem Obersten, \footnote{\textbf{2:13} Röm
  13,1-7; Tit 3,1} \bibverse{14} oder den Hauptleuten, als die von ihm
gesandt sind zur Rache über die Übeltäter und zu Lobe den Frommen.
\bibverse{15} Denn das ist der Wille Gottes, dass ihr mit Wohltun
verstopfet die Unwissenheit der törichten Menschen, \footnote{\textbf{2:15}
  1Petr 3,16; Tit 2,8} \bibverse{16} als die Freien, und nicht, als
hättet ihr die Freiheit zum Deckel der Bosheit, sondern als die Knechte
Gottes. \footnote{\textbf{2:16} Gal 5,13; 2Petr 2,19}

\bibverse{17} Tut Ehre jedermann, habt die Brüder lieb; fürchtet Gott,
ehret den König! \footnote{\textbf{2:17} Röm 12,10; Spr 24,21}

\bibverse{18} Ihr Knechte, seid untertan mit aller Furcht den Herren,
nicht allein den gütigen und gelinden, sondern auch den wunderlichen.
\footnote{\textbf{2:18} Eph 6,5; Tit 2,9} \bibverse{19} Denn das ist
Gnade, wenn jemand um des Gewissens willen zu Gott das Übel verträgt und
leidet das Unrecht. \bibverse{20} Denn was ist das für ein Ruhm, so ihr
um Missetat willen Streiche leidet? Aber wenn ihr um Wohltat willen
leidet und erduldet, das ist Gnade bei Gott. \footnote{\textbf{2:20}
  1Petr 3,14; Mt 5,10} \bibverse{21} Denn dazu seid ihr berufen;
sintemal auch Christus gelitten hat für uns und uns ein Vorbild
gelassen, dass ihr sollt nachfolgen seinen Fußtapfen; \footnote{\textbf{2:21}
  1Petr 3,18; Mt 16,24} \bibverse{22} welcher keine Sünde getan hat, ist
auch kein Betrug in seinem Munde erfunden; \footnote{\textbf{2:22} Jes
  53,9; Joh 8,46} \bibverse{23} welcher nicht wiederschalt, da er
gescholten ward, nicht drohte, da er litt, er stellte es aber dem
anheim, der da recht richtet; \bibverse{24} welcher unsere Sünden selbst
hinaufgetragen hat an seinem Leibe auf das Holz, auf dass wir, der Sünde
abgestorben, der Gerechtigkeit leben; durch welches Wunden ihr seid heil
geworden. \bibverse{25} Denn ihr waret wie die irrenden Schafe; aber ihr
seid nun bekehrt zu dem Hirten und Bischof eurer Seelen. \footnote{\textbf{2:25}
  Jes 53,6; Joh 10,12}

\hypertarget{section-1}{%
\section{3}\label{section-1}}

\bibverse{1} Desgleichen sollen die Weiber ihren Männern untertan sein,
auf dass auch die, die nicht glauben an das Wort, durch der Weiber
Wandel ohne Wort gewonnen werden, \footnote{\textbf{3:1} Eph 5,22; 1Kor
  7,16} \bibverse{2} wenn sie ansehen euren keuschen Wandel in der
Furcht. \bibverse{3} Ihr Schmuck soll nicht auswendig sein mit
Haarflechten und Goldumhängen oder Kleideranlegen, \footnote{\textbf{3:3}
  Jes 3,18-24; 1Tim 2,9} \bibverse{4} sondern der verborgene Mensch des
Herzens unverrückt mit sanftem und stillem Geiste; das ist köstlich vor
Gott. \bibverse{5} Denn also haben sich auch vorzeiten die heiligen
Weiber geschmückt, die ihre Hoffnung auf Gott setzten und ihren Männern
untertan waren, \bibverse{6} wie die Sara Abraham gehorsam war und hieß
ihn Herr; deren Töchter ihr geworden seid, wenn ihr wohltut und euch
nicht lasset schüchtern machen.

\bibverse{7} Desgleichen, ihr Männer, wohnet bei ihnen mit Vernunft und
gebet dem weiblichen als dem schwächeren Werkzeuge seine Ehre, als die
auch Miterben sind der Gnade des Lebens, auf dass eure Gebete nicht
verhindert werden. \footnote{\textbf{3:7} Eph 5,25; 1Kor 7,5}

\bibverse{8} Endlich aber seid allesamt gleichgesinnt, mitleidig,
brüderlich, barmherzig, freundlich. \bibverse{9} Vergeltet nicht Böses
mit Bösem oder Scheltwort mit Scheltwort, sondern dagegen segnet, und
wisset, dass ihr dazu berufen seid, dass ihr den Segen erbet.
\bibverse{10} Denn wer leben will und gute Tage sehen, der schweige
seine Zunge, dass sie nichts Böses rede, und seine Lippen, dass sie
nicht trügen. \footnote{\textbf{3:10} Jak 1,26} \bibverse{11} Er wende
sich vom Bösen und tue Gutes; er suche Frieden und jage ihm nach.
\bibverse{12} Denn die Augen des Herrn merken auf die Gerechten und
seine Ohren auf ihr Gebet; das Angesicht aber des Herrn steht wider die,
die Böses tun.

\bibverse{13} Und wer ist, der euch schaden könnte, so ihr dem Gutem
nachkommt? \bibverse{14} Und ob ihr auch leidet um Gerechtigkeit willen,
so seid ihr doch selig. Fürchtet euch aber vor ihrem Trotzen nicht und
erschrecket nicht; \bibverse{15} heiliget aber Gott den Herrn in euren
Herzen. Seid allezeit bereit zur Verantwortung jedermann, der Grund
fordert der Hoffnung, die in euch ist, \bibverse{16} und das mit
Sanftmütigkeit und Furcht; und habt ein gutes Gewissen, auf dass die,
die von euch afterreden als von Übeltätern, zu Schanden werden, dass sie
geschmäht haben euren guten Wandel in Christo. \bibverse{17} Denn es ist
besser, wenn es Gottes Wille ist, dass ihr von Wohltat wegen leidet als
von Übeltat wegen. \bibverse{18} Sintemal auch Christus einmal für
unsere Sünden gelitten hat, der Gerechte für die Ungerechten, auf dass
er uns zu Gott führte, und ist getötet nach dem Fleisch, aber lebendig
gemacht nach dem Geist. \footnote{\textbf{3:18} 1Petr 2,21-24}
\bibverse{19} In demselben ist er auch hingegangen und hat gepredigt den
Geistern im Gefängnis, \footnote{\textbf{3:19} 1Petr 4,6} \bibverse{20}
die vorzeiten nicht glaubten, da Gott harrte und Geduld hatte zu den
Zeiten Noahs, da man die Arche zurüstete, in welcher wenige, das ist
acht Seelen, gerettet wurden durchs Wasser; \footnote{\textbf{3:20} 1Mo
  7,7; 1Mo 7,17; 2Petr 2,5} \bibverse{21} welches nun auch uns selig
macht in der Taufe, die durch jenes bedeutet ist, nicht das Abtun des
Unflats am Fleisch, sondern der Bund eines guten Gewissens mit Gott
durch die Auferstehung Jesu Christi, \footnote{\textbf{3:21} Eph 5,26;
  Hebr 10,22} \bibverse{22} welcher ist zur Rechten Gottes in den Himmel
gefahren, und sind ihm untertan die Engel und die Gewaltigen und die
Kräfte. \footnote{\textbf{3:22} Eph 1,20-21}

\hypertarget{section-2}{%
\section{4}\label{section-2}}

\bibverse{1} Weil nun Christus im Fleisch für uns gelitten hat, so
wappnet euch auch mit demselben Sinn; denn wer am Fleisch leidet, der
hört auf von Sünden, \bibverse{2} dass er hinfort die noch übrige Zeit
im Fleisch nicht der Menschen Lüsten, sondern dem Willen Gottes lebe.
\bibverse{3} Denn es ist genug, dass wir die vergangene Zeit des Lebens
zugebracht haben nach heidnischem Willen, da wir wandelten in Unzucht,
Lüsten, Trunkenheit, Fresserei, Sauferei und gräulichen Abgöttereien.
\bibverse{4} Das befremdet sie, dass ihr nicht mit ihnen laufet in
dasselbe wüste, unordentliche Wesen, und sie lästern; \bibverse{5} aber
sie werden Rechenschaft geben dem, der bereit ist, zu richten die
Lebendigen und die Toten. \footnote{\textbf{4:5} 2Tim 4,1} \bibverse{6}
Denn dazu ist auch den Toten das Evangelium verkündigt, auf dass sie
gerichtet werden nach dem Menschen am Fleisch, aber im Geist Gott leben.
\footnote{\textbf{4:6} 1Petr 3,19}

\bibverse{7} Es ist aber nahe gekommen das Ende aller Dinge. \footnote{\textbf{4:7}
  1Kor 10,11; 1Jo 2,18} \bibverse{8} So seid nun mäßig und nüchtern zum
Gebet. Vor allen Dingen aber habt untereinander eine inbrünstige Liebe;
denn die Liebe deckt auch der Sünden Menge. \footnote{\textbf{4:8} Jak
  5,20} \bibverse{9} Seid gastfrei untereinander ohne Murren.
\footnote{\textbf{4:9} Hebr 13,2} \bibverse{10} Und dienet einander, ein
jeglicher mit der Gabe, die er empfangen hat, als die guten Haushalter
der mancherlei Gnade Gottes: \bibverse{11} wenn jemand redet, dass er's
rede als Gottes Wort; wenn jemand ein Amt hat, dass er's tue als aus dem
Vermögen, das Gott darreicht, auf dass in allen Dingen Gott gepriesen
werde durch Jesum Christum, welchem sei Ehre und Gewalt von Ewigkeit zu
Ewigkeit! Amen.

\bibverse{12} Ihr Lieben, lasset euch die Hitze, die euch begegnet,
nicht befremden (die euch widerfährt, dass ihr versucht werdet), als
widerführe euch etwas Seltsames; \footnote{\textbf{4:12} 1Petr 1,6-7}
\bibverse{13} sondern freuet euch, dass ihr mit Christo leidet, auf dass
ihr auch zur Zeit der Offenbarung seiner Herrlichkeit Freude und Wonne
haben möget. \footnote{\textbf{4:13} Apg 5,41; Röm 8,17; Jak 1,2}
\bibverse{14} Selig seid ihr, wenn ihr geschmäht werdet über dem Namen
Christi; denn der Geist, der ein Geist der Herrlichkeit und Gottes ist,
ruht auf euch. Bei ihnen ist er verlästert, aber bei euch ist er
gepriesen. \footnote{\textbf{4:14} Mt 5,11; Eph 1,13} \bibverse{15}
Niemand aber unter euch leide als ein Mörder oder Dieb oder Übeltäter
oder der in ein fremdes Amt greift. \bibverse{16} Leidet er aber als ein
Christ, so schäme er sich nicht; er ehre aber Gott in solchem Fall.
\bibverse{17} Denn es ist Zeit, dass anfange das Gericht an dem Hause
Gottes. Wenn aber zuerst an uns, was will's für ein Ende werden mit
denen, die dem Evangelium Gottes nicht glauben? \footnote{\textbf{4:17}
  Jer 25,29; Hes 9,6} \bibverse{18} Und wenn der Gerechte kaum erhalten
wird, wo will der Gottlose und Sünder erscheinen? \footnote{\textbf{4:18}
  Spr 11,31} \bibverse{19} Darum, welche da leiden nach Gottes Willen,
die sollen ihm ihre Seelen befehlen als dem treuen Schöpfer in guten
Werken. \footnote{\textbf{4:19} Ps 31,6}

\hypertarget{section-3}{%
\section{5}\label{section-3}}

\bibverse{1} Die Ältesten, die unter euch sind, ermahne ich, der
Mitälteste und Zeuge der Leiden, die in Christo sind, und auch
teilhaftig der Herrlichkeit, die offenbart werden soll: \footnote{\textbf{5:1}
  Röm 8,17; 2Jo 1,-1} \bibverse{2} Weidet die Herde Christi, die euch
befohlen ist und sehet wohl zu, nicht gezwungen, sondern willig; nicht
um schändlichen Gewinns willen, sondern von Herzensgrund; \footnote{\textbf{5:2}
  Joh 21,16; Apg 20,28; 1Tim 3,2-7} \bibverse{3} nicht als die übers
Volk herrschen, sondern werdet Vorbilder der Herde. \footnote{\textbf{5:3}
  Hes 34,2-4; 2Kor 1,24; Tit 2,7} \bibverse{4} So werdet ihr, wenn
erscheinen wird der Erzhirte, die unverwelkliche Krone der Ehren
empfangen. \footnote{\textbf{5:4} 1Kor 9,25; 2Tim 4,8; Hebr 13,20}

\bibverse{5} Desgleichen, ihr Jüngeren, seid untertan den Ältesten.
Allesamt seid untereinander untertan und haltet fest an der Demut. Denn
Gott widersteht den Hoffärtigen, aber den Demütigen gibt er Gnade.
\footnote{\textbf{5:5} Spr 3,34; Mt 23,12; Eph 5,21; Jak 4,6}
\bibverse{6} So demütiget euch nun unter die gewaltige Hand Gottes, dass
er euch erhöhe zu seiner Zeit. \footnote{\textbf{5:6} Hi 22,29; Jak 4,10}
\bibverse{7} Alle eure Sorge werfet auf ihn; denn er sorgt für euch.
\footnote{\textbf{5:7} Ps 55,23; Mt 6,25; Phil 4,6}

\bibverse{8} Seid nüchtern und wachet; denn euer Widersacher, der
Teufel, geht umher wie ein brüllender Löwe und sucht, welchen er
verschlinge. \footnote{\textbf{5:8} 2Kor 2,11; 1Thes 5,6; Lk 22,31}
\bibverse{9} Dem widerstehet, fest im Glauben, und wisset, dass
ebendieselben Leiden über eure Brüder in der Welt gehen. \bibverse{10}
Der Gott aber aller Gnade, der uns berufen hat zu seiner ewigen
Herrlichkeit in Christo Jesu, der wird euch, die ihr eine kleine Zeit
leidet, vollbereiten, stärken, kräftigen, gründen. \bibverse{11} Ihm sei
Ehre und Macht von Ewigkeit zu Ewigkeit! Amen.

\bibverse{12} Durch euren treuen Bruder Silvanus (wie ich achte) habe
ich euch ein wenig geschrieben, zu ermahnen und zu bezeugen, dass das
die rechte Gnade Gottes ist, darin ihr stehet. \^{}\^{} \bibverse{13} Es
grüßen euch, die samt euch auserwählt sind zu Babylon, und mein Sohn
Markus. \bibverse{14} Grüßet euch untereinander mit dem Kuss der Liebe.
Friede sei mit allen, die in Christo Jesu sind! Amen.
