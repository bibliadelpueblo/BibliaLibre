\hypertarget{section}{%
\section{1}\label{section}}

\bibleverse{1} Paulus, ein Knecht Jesu Christi, berufen zum Apostel,
ausgesondert, zu predigen das Evangelium Gottes, \footnote{\textbf{1:1}
  Apg 9,15; Apg 13,2; Gal 1,15} \bibleverse{2} welches er zuvor
verheißen hat durch seine Propheten in der heiligen Schrift, \footnote{\textbf{1:2}
  Röm 16,25-26; Tit 1,2; Lk 1,70} \bibleverse{3} von seinem Sohn, der
geboren ist von dem Samen Davids nach dem Fleisch \footnote{\textbf{1:3}
  2Sam 7,12; Mt 22,42; Röm 9,5} \bibleverse{4} und kräftig erwiesen als
ein Sohn Gottes nach dem Geist, der da heiligt, seit der Zeit, da er
auferstanden ist von den Toten, Jesus Christus, unser Herr, \footnote{\textbf{1:4}
  Apg 13,33; Mt 28,18} \bibleverse{5} durch welchen wir haben empfangen
Gnade und Apostelamt, unter allen Heiden den Gehorsam des Glaubens
aufzurichten unter seinem Namen, \footnote{\textbf{1:5} Röm 15,18; Gal
  2,7; Gal 2,9; Apg 26,16-18} \bibleverse{6} unter welchen ihr auch
seid, die da berufen sind von Jesu Christo, -- \bibleverse{7} allen, die
zu Rom sind, den Liebsten Gottes und berufenen Heiligen: Gnade sei mit
euch und Friede von Gott, unserem Vater, und dem Herrn Jesus Christus!

\bibleverse{8} Aufs erste danke ich meinem Gott durch Jesum Christum
euer aller halben, dass man von eurem Glauben in aller Welt sagt.
\footnote{\textbf{1:8} Röm 16,19} \bibleverse{9} Denn Gott ist mein
Zeuge, welchem ich diene in meinem Geist am Evangelium von seinem Sohn,
dass ich ohne Unterlass euer gedenke \footnote{\textbf{1:9} Eph 1,16}
\bibleverse{10} und allezeit in meinem Gebet flehe, ob sich's einmal
zutragen wollte, dass ich zu euch käme durch Gottes Willen. \footnote{\textbf{1:10}
  Röm 15,23; Röm 15,32; Apg 19,21} \bibleverse{11} Denn mich verlangt,
euch zu sehen, auf dass ich euch mitteile etwas geistlicher Gabe, euch
zu stärken; \footnote{\textbf{1:11} Röm 15,29} \bibleverse{12} das ist,
dass ich samt euch getröstet würde durch euren und meinen Glauben, den
wir untereinander haben. \footnote{\textbf{1:12} 2Petr 1,1}

\bibleverse{13} Ich will euch aber nicht verhalten, liebe Brüder, dass
ich mir oft habe vorgesetzt, zu euch zu kommen (bin aber verhindert
bisher), dass ich auch unter euch Frucht schaffte gleichwie unter
anderen Heiden. \bibleverse{14} Ich bin ein Schuldner der Griechen und
der Ungriechen, der Weisen und der Unweisen. \bibleverse{15} Darum,
soviel an mir ist, bin ich geneigt, auch euch zu Rom das Evangelium zu
predigen.

\bibleverse{16} Denn ich schäme mich des Evangeliums von Christo nicht;
denn es ist eine Kraft Gottes, die da selig macht alle, die daran
glauben, die Juden vornehmlich und auch die Griechen. \bibleverse{17}
Sintemal darin offenbart wird die Gerechtigkeit, die vor Gott gilt,
welche kommt aus Glauben in Glauben; wie denn geschrieben steht: „Der
Gerechte wird seines Glaubens leben.`` \footnote{\textbf{1:17} Röm
  3,21-22}

\bibleverse{18} Denn Gottes Zorn vom Himmel wird offenbart über alles
gottlose Wesen und Ungerechtigkeit der Menschen, die die Wahrheit in
Ungerechtigkeit aufhalten. \bibleverse{19} Denn was man von Gott weiß,
ist ihnen offenbar; denn Gott hat es ihnen offenbart, \bibleverse{20}
damit dass Gottes unsichtbares Wesen, das ist seine ewige Kraft und
Gottheit, wird ersehen, so man des wahrnimmt, an den Werken, nämlich an
der Schöpfung der Welt; also dass sie keine Entschuldigung haben,
\footnote{\textbf{1:20} Ps 19,2; Hebr 11,3} \bibleverse{21} dieweil sie
wussten, dass ein Gott ist, und haben ihn nicht gepriesen als einen Gott
noch ihm gedankt, sondern sind in ihrem Dichten eitel geworden, und ihr
unverständiges Herz ist verfinstert. \footnote{\textbf{1:21} Eph 4,18}

\bibleverse{22} Da sie sich für weise hielten, sind sie zu Narren
geworden \footnote{\textbf{1:22} Jer 10,14; 1Kor 1,20} \bibleverse{23}
und haben verwandelt die Herrlichkeit des unvergänglichen Gottes in ein
Bild gleich dem vergänglichen Menschen und der Vögel und der vierfüßigen
und der kriechenden Tiere. \footnote{\textbf{1:23} 5Mo 4,15-19}
\bibleverse{24} Darum hat sie auch Gott dahingegeben in ihrer Herzen
Gelüste, in Unreinigkeit, zu schänden ihre eigenen Leiber an sich
selbst, \footnote{\textbf{1:24} Apg 14,16} \bibleverse{25} sie, die
Gottes Wahrheit haben verwandelt in die Lüge und haben geehrt und
gedient dem Geschöpfe mehr denn dem Schöpfer, der da gelobt ist in
Ewigkeit. Amen.

\bibleverse{26} Darum hat sie Gott auch dahingegeben in schändliche
Lüste: denn ihre Weiber haben verwandelt den natürlichen Brauch in den
unnatürlichen; \bibleverse{27} desgleichen auch die Männer haben
verlassen den natürlichen Brauch des Weibes und sind aneinander erhitzt
in ihren Lüsten und haben Mann mit Mann Schande getrieben und den Lohn
ihres Irrtums (wie es denn sein sollte) an sich selbst empfangen.
\bibleverse{28} Und gleichwie sie nicht geachtet haben, dass sie Gott
erkenneten, hat sie Gott auch dahingegeben in verkehrten Sinn, zu tun,
was nicht taugt, \bibleverse{29} voll alles Ungerechten, Hurerei,
Schalkheit, Geizes, Bosheit, voll Neides, Mordes, Haders, List, giftig,
Ohrenbläser, \bibleverse{30} Verleumder, Gottesverächter, Frevler,
hoffärtig, ruhmredig, Schädliche, den Eltern ungehorsam, \bibleverse{31}
Unvernünftige, Treulose, Lieblose, unversöhnlich, unbarmherzig.
\bibleverse{32} Sie wissen Gottes Gerechtigkeit, dass, die solches tun,
des Todes würdig sind, und tun es nicht allein, sondern haben auch
Gefallen an denen, die es tun. \# 2 \bibleverse{1} Darum, o Mensch,
kannst du dich nicht entschuldigen, wer du auch bist, der da richtet.
Denn worin du einen anderen richtest, verdammst du dich selbst; sintemal
du eben dasselbe tust, was du richtest. \footnote{\textbf{2:1} Mt 7,2;
  Joh 8,7; Jak 4,12} \bibleverse{2} Denn wir wissen, dass Gottes Urteil
ist recht über die, die solches tun. \bibleverse{3} Denkst du aber, o
Mensch, der du richtest die, die solches tun, und tust auch dasselbe,
dass du dem Urteil Gottes entrinnen werdest? \bibleverse{4} Oder
verachtest du den Reichtum seiner Güte, Geduld und Langmütigkeit? Weißt
du nicht, dass dich Gottes Güte zur Buße leitet? \bibleverse{5} Du aber
nach deinem verstockten und unbußfertigen Herzen häufest dir selbst den
Zorn auf den Tag des Zorns und der Offenbarung des gerechten Gerichtes
Gottes, \bibleverse{6} welcher geben wird einem jeglichen nach seinen
Werken: \footnote{\textbf{2:6} Mt 16,27; 2Kor 5,10} \bibleverse{7} Preis
und Ehre und unvergängliches Wesen denen, die mit Geduld in guten Werken
trachten nach dem ewigen Leben; \bibleverse{8} aber denen, die da
zänkisch sind und der Wahrheit nicht gehorchen, gehorchen aber der
Ungerechtigkeit, Ungnade, und Zorn; \bibleverse{9} Trübsal und Angst
über alle Seelen der Menschen, die da Böses tun, vornehmlich der Juden
und auch der Griechen;

\bibleverse{10} Preis aber und Ehre und Friede allen denen, die da Gutes
tun, vornehmlich den Juden und auch den Griechen. \bibleverse{11} Denn
es ist kein Ansehen der Person vor Gott. \footnote{\textbf{2:11} Apg
  10,34; 1Petr 1,17; Kol 3,25} \bibleverse{12} Welche ohne Gesetz
gesündigt haben, die werden auch ohne Gesetz verloren werden; und welche
unter dem Gesetz gesündigt haben, die werden durchs Gesetz verurteilt
werden \bibleverse{13} (sintemal vor Gott nicht, die das Gesetz hören,
gerecht sind, sondern die das Gesetz tun, werden gerecht sein.
\bibleverse{14} Denn so die Heiden, die das Gesetz nicht haben, doch von
Natur tun des Gesetzes Werk, sind dieselben, dieweil sie das Gesetz
nicht haben, sich selbst ein Gesetz, \footnote{\textbf{2:14} Apg 10,35}
\bibleverse{15} als die da beweisen, des Gesetzes Werk sei geschrieben
in ihrem Herzen, sintemal ihr Gewissen ihnen zeugt, dazu auch die
Gedanken, die sich untereinander verklagen oder entschuldigen),
\footnote{\textbf{2:15} Röm 1,32} \bibleverse{16} auf den Tag, da Gott
das Verborgene der Menschen durch Jesus Christus richten wird laut
meines Evangeliums. \footnote{\textbf{2:16} Lk 8,17}

\bibleverse{17} Siehe aber zu: du heißest ein Jude und verlässest dich
aufs Gesetz und rühmest dich Gottes \bibleverse{18} und weißt seinen
Willen; und weil du aus dem Gesetz unterrichtet bist, prüfest du, was
das Beste zu tun sei, \bibleverse{19} und vermissest dich, zu sein ein
Leiter der Blinden, ein Licht derer, die in Finsternis sind,
\bibleverse{20} ein Züchtiger der Törichten, ein Lehrer der Einfältigen,
hast die Form, was zu wissen und recht ist, im Gesetz. \bibleverse{21}
Nun lehrst du andere, und lehrst dich selber nicht; du predigst, man
solle nicht stehlen, und du stiehlst; \footnote{\textbf{2:21} Ps
  50,16-21; Mt 23,3-4} \bibleverse{22} du sprichst, man solle nicht
ehebrechen, und du brichst die Ehe; dir gräuelt vor den Götzen, und du
raubest Gott, was sein ist; \bibleverse{23} du rühmest dich des
Gesetzes, und schändest Gott durch Übertretung des Gesetzes;
\bibleverse{24} denn „eurethalben wird Gottes Name gelästert unter den
Heiden``, wie geschrieben steht. \bibleverse{25} Die Beschneidung ist
wohl nütz, wenn du das Gesetz hältst; hältst du das Gesetz aber nicht,
so bist du aus einem Beschnittenen schon ein Unbeschnittener geworden.
\bibleverse{26} So nun der Unbeschnittene das Recht im Gesetz hält,
meinst du nicht, dass da der Unbeschnittene werde für einen
Beschnittenen gerechnet? \footnote{\textbf{2:26} Gal 5,6}
\bibleverse{27} Und wird also, der von Natur unbeschnitten ist und das
Gesetz vollbringt, dich richten, der du unter dem Buchstaben und der
Beschneidung bist und das Gesetz übertrittst. \bibleverse{28} Denn das
ist nicht ein Jude, der auswendig ein Jude ist, auch ist das nicht eine
Beschneidung, die auswendig am Fleisch geschieht; \bibleverse{29}
sondern das ist ein Jude, der's inwendig verborgen ist, und die
Beschneidung des Herzens ist eine Beschneidung, die im Geist und nicht
im Buchstaben geschieht. Eines solchen Lob ist nicht aus Menschen,
sondern aus Gott. \# 3 \bibleverse{1} Was haben denn die Juden für
Vorteil, oder was nützt die Beschneidung? \bibleverse{2} Fürwahr sehr
viel. Zum ersten: ihnen ist vertraut, was Gott geredet hat. \footnote{\textbf{3:2}
  Röm 9,4; 5Mo 4,7-8; Ps 147,19-20} \bibleverse{3} Dass aber etliche
nicht daran glauben, was liegt daran? Sollte ihr Unglaube Gottes Glauben
aufheben? \footnote{\textbf{3:3} Röm 9,6; Röm 11,29; 2Tim 2,13}
\bibleverse{4} Das sei ferne! Es bleibe vielmehr also, dass Gott sei
wahrhaftig und alle Menschen Lügner; wie geschrieben steht: „Auf dass du
gerecht seist in deinen Worten und überwindest, wenn du gerichtet
wirst.`` \footnote{\textbf{3:4} Ps 116,11}

\bibleverse{5} Ist's aber also, dass unsere Ungerechtigkeit Gottes
Gerechtigkeit preist, was wollen wir sagen? Ist denn Gott auch
ungerecht, wenn er darüber zürnt? (Ich rede also auf Menschenweise.)
\bibleverse{6} Das sei ferne! Wie könnte sonst Gott die Welt richten?
\bibleverse{7} Denn so die Wahrheit Gottes durch meine Lüge herrlicher
wird zu seinem Preis, warum sollte ich denn noch als ein Sünder
gerichtet werden \bibleverse{8} und nicht vielmehr also tun, wie wir
gelästert werden und wie etliche sprechen, dass wir sagen: „Lasset uns
Übles tun, auf dass Gutes daraus komme``? welcher Verdammnis ist ganz
recht.

\bibleverse{9} Was sagen wir denn nun? Haben wir einen Vorteil? Gar
keinen. Denn wir haben droben bewiesen, dass beide, Juden und Griechen,
alle unter der Sünde sind, \footnote{\textbf{3:9} Röm 1,18-999}
\bibleverse{10} wie denn geschrieben steht: „Da ist nicht, der gerecht
sei, auch nicht einer. \footnote{\textbf{3:10} Ps 14,1-3; Ps 53,2-4}
\bibleverse{11} Da ist nicht, der verständig sei; da ist nicht, der nach
Gott frage. \bibleverse{12} Sie sind alle abgewichen und allesamt
untüchtig geworden. Da ist nicht, der Gutes tue, auch nicht einer.
\bibleverse{13} Ihr Schlund ist ein offenes Grab; mit ihren Zungen
handeln sie trüglich. Otterngift ist unter den Lippen; \footnote{\textbf{3:13}
  Ps 5,10; Ps 140,4} \bibleverse{14} ihr Mund ist voll Fluchens und
Bitterkeit. \footnote{\textbf{3:14} Ps 10,7} \bibleverse{15} Ihre Füße
sind eilend, Blut zu vergießen; \footnote{\textbf{3:15} Jes 59,7-8}
\bibleverse{16} auf ihren Wegen ist eitel Schaden und Herzeleid,
\bibleverse{17} und den Weg des Friedens wissen sie nicht.
\bibleverse{18} Es ist keine Furcht Gottes vor ihren Augen.``
\footnote{\textbf{3:18} Ps 36,2}

\bibleverse{19} Wir wissen aber, dass, was das Gesetz sagt, das sagt es
denen, die unter dem Gesetz sind, auf dass aller Mund verstopft werde
und alle Welt Gott schuldig sei; \footnote{\textbf{3:19} Röm 2,12; Gal
  3,22} \bibleverse{20} darum dass kein Fleisch durch des Gesetzes Werke
vor ihm gerecht sein kann; denn durch das Gesetz kommt Erkenntnis der
Sünde. \footnote{\textbf{3:20} Ps 143,2; Gal 2,16; Röm 7,7}

\bibleverse{21} Nun aber ist ohne Zutun des Gesetzes die Gerechtigkeit,
die vor Gott gilt, offenbart und bezeugt durch das Gesetz und die
Propheten. \footnote{\textbf{3:21} Röm 1,17; Apg 10,43} \bibleverse{22}
Ich sage aber von solcher Gerechtigkeit vor Gott, die da kommt durch den
Glauben an Jesum Christum zu allen und auf alle, die da glauben.
\footnote{\textbf{3:22} Phil 3,9} \bibleverse{23} Denn es ist hier kein
Unterschied: sie sind allzumal Sünder und mangeln des Ruhmes, den sie
bei Gott haben sollten, \footnote{\textbf{3:23} Röm 5,2; Joh 5,44; Ps
  84,12} \bibleverse{24} und werden ohne Verdienst gerecht aus seiner
Gnade durch die Erlösung, die durch Jesum Christum geschehen ist,
\footnote{\textbf{3:24} Röm 5,1; 2Kor 5,19; Eph 2,8} \bibleverse{25}
welchen Gott hat vorgestellt zu einem Gnadenstuhl durch den Glauben in
seinem Blut, damit er die Gerechtigkeit, die vor ihm gilt, darbiete in
dem, dass er Sünde vergibt, welche bisher geblieben war unter göttlicher
Geduld; \footnote{\textbf{3:25} 3Mo 16,12-15; Hebr 4,16} \bibleverse{26}
auf dass er zu diesen Zeiten darböte die Gerechtigkeit, die vor ihm
gilt; auf dass er allein gerecht sei und gerecht mache den, der da ist
des Glaubens an Jesum.

\bibleverse{27} Wo bleibt nun der Ruhm? Er ist ausgeschlossen. Durch
welches Gesetz? Durch der Werke Gesetz? Nicht also, sondern durch des
Glaubens Gesetz. \footnote{\textbf{3:27} 1Kor 1,29; 1Kor 1,31}
\bibleverse{28} So halten wir nun dafür, dass der Mensch gerecht werde
ohne des Gesetzes Werke, allein durch den Glauben. \footnote{\textbf{3:28}
  Gal 2,16} \bibleverse{29} Oder ist Gott allein der Juden Gott? Ist er
nicht auch der Heiden Gott? Ja freilich, auch der Heiden Gott.
\footnote{\textbf{3:29} Röm 10,12} \bibleverse{30} Sintemal es ist ein
einiger Gott, der da gerecht macht die Beschnittenen aus dem Glauben und
die Unbeschnittenen durch den Glauben. \footnote{\textbf{3:30} Röm
  4,11-12}

\bibleverse{31} Wie? Heben wir denn das Gesetz auf durch den Glauben?
Das sei ferne! sondern wir richten das Gesetz auf. \footnote{\textbf{3:31}
  Mt 5,17}

\hypertarget{section-1}{%
\section{4}\label{section-1}}

\bibleverse{1} Was sagen wir denn von unserem Vater Abraham, dass er
gefunden habe nach dem Fleisch? \bibleverse{2} Das sagen wir: Ist
Abraham durch die Werke gerecht, so hat er wohl Ruhm, aber nicht vor
Gott. \bibleverse{3} Was sagt denn die Schrift? „Abraham hat Gott
geglaubt, und das ist ihm zur Gerechtigkeit gerechnet.`` \bibleverse{4}
Dem aber, der mit Werken umgeht, wird der Lohn nicht aus Gnade
zugerechnet, sondern aus Pflicht. \footnote{\textbf{4:4} Röm 11,6}
\bibleverse{5} Dem aber, der nicht mit Werken umgeht, glaubt aber an
den, der die Gottlosen gerecht macht, dem wird sein Glaube gerechnet zur
Gerechtigkeit. \footnote{\textbf{4:5} Röm 3,26} \bibleverse{6} Nach
welcher Weise auch David sagt, dass die Seligkeit sei allein des
Menschen, welchem Gott zurechnet die Gerechtigkeit ohne Zutun der Werke,
da er spricht: \bibleverse{7} „Selig sind die, welchen ihre
Ungerechtigkeiten vergeben sind und welchen ihre Sünden bedeckt sind!
\bibleverse{8} Selig ist der Mann, welchem Gott die Sünde nicht
zurechnet!{}``

\bibleverse{9} Nun diese Seligkeit, geht sie über die Beschnittenen oder
auch über die Unbeschnittenen? Wir müssen ja sagen, dass Abraham sei
sein Glaube zur Gerechtigkeit gerechnet. \bibleverse{10} Wie ist er ihm
denn zugerechnet? Als er beschnitten oder als er unbeschnitten war?
Nicht, als er beschnitten, sondern als er unbeschnitten war.
\bibleverse{11} Das Zeichen aber der Beschneidung empfing er zum Siegel
der Gerechtigkeit des Glaubens, welchen er hatte, als er noch nicht
beschnitten war, auf dass er würde ein Vater aller, die da glauben und
nicht beschnitten sind, dass ihnen solches auch gerechnet werde zur
Gerechtigkeit; \footnote{\textbf{4:11} 1Mo 17,10-11} \bibleverse{12} und
würde auch ein Vater der Beschneidung, derer, die nicht allein
beschnitten sind, sondern auch wandeln in den Fußtapfen des Glaubens,
welcher war in unserem Vater Abraham, als er noch nicht beschnitten war.
\footnote{\textbf{4:12} Mt 3,9}

\bibleverse{13} Denn die Verheißung, dass er sollte sein der Welt Erbe,
ist nicht geschehen Abraham oder seinem Samen durchs Gesetz, sondern
durch die Gerechtigkeit des Glaubens. \footnote{\textbf{4:13} 1Mo
  22,17-18} \bibleverse{14} Denn wo die vom Gesetz Erben sind, so ist
der Glaube nichts, und die Verheißung ist abgetan. \bibleverse{15}
Sintemal das Gesetz nur Zorn anrichtet; denn wo das Gesetz nicht ist, da
ist auch keine Übertretung.

\bibleverse{16} Derhalben muss die Gerechtigkeit durch den Glauben
kommen, auf dass sie sei aus Gnaden und die Verheißung fest bleibe allem
Samen, nicht dem allein, der unter dem Gesetz ist, sondern auch dem, der
des Glaubens Abrahams ist, welcher ist unser aller Vater \bibleverse{17}
(wie geschrieben steht: „Ich habe dich gesetzt zum Vater vieler
Völker``) vor Gott, dem er geglaubt hat, der da lebendig macht die Toten
und ruft dem, was nicht ist, dass es sei. \footnote{\textbf{4:17} Hebr
  11,19; 2Kor 1,9} \bibleverse{18} Und er hat geglaubt auf Hoffnung, da
nichts zu hoffen war, auf dass er würde ein Vater vieler Völker, wie
denn zu ihm gesagt ist: „Also soll dein Same sein.`` \bibleverse{19} Und
er ward nicht schwach im Glauben, sah auch nicht an seinen eigenen Leib,
welcher schon erstorben war (weil er fast hundertjährig war), auch nicht
den erstorbenen Leib der Sara; \bibleverse{20} denn er zweifelte nicht
an der Verheißung Gottes durch Unglauben, sondern ward stark im Glauben
und gab Gott die Ehre \footnote{\textbf{4:20} Hebr 11,11}
\bibleverse{21} und wusste aufs allergewisseste, dass, was Gott
verheißt, das kann er auch tun. \bibleverse{22} Darum ist's ihm auch zur
Gerechtigkeit gerechnet. \bibleverse{23} Das ist aber nicht geschrieben
allein um seinetwillen, dass es ihm zugerechnet ist, \bibleverse{24}
sondern auch um unseretwillen, welchen es soll zugerechnet werden, wenn
wir glauben an den, der unseren Herrn Jesus auferweckt hat von den
Toten, \bibleverse{25} welcher ist um unserer Sünden willen dahingegeben
und um unserer Gerechtigkeit willen auferweckt. \# 5 \bibleverse{1} Nun
wir denn sind gerecht geworden durch den Glauben, so haben wir Frieden
mit Gott durch unseren Herrn Jesus Christus, \footnote{\textbf{5:1} Röm
  3,24; Röm 3,28; Jes 53,5} \bibleverse{2} durch welchen wir auch den
Zugang haben im Glauben zu dieser Gnade, darin wir stehen, und rühmen
uns der Hoffnung der zukünftigen Herrlichkeit, die Gott geben soll.
\footnote{\textbf{5:2} Joh 14,6; Eph 3,12} \bibleverse{3} Nicht allein
aber das, sondern wir rühmen uns auch der Trübsale, dieweil wir wissen,
dass Trübsal Geduld bringt; \footnote{\textbf{5:3} Jak 1,2; Jak 1,1-3}
\bibleverse{4} Geduld aber bringt Erfahrung; Erfahrung aber bringt
Hoffnung; \footnote{\textbf{5:4} Jak 1,12} \bibleverse{5} Hoffnung aber
lässt nicht zu Schanden werden. Denn die Liebe Gottes ist ausgegossen in
unser Herz durch den heiligen Geist, welcher uns gegeben ist.
\footnote{\textbf{5:5} Hebr 6,18-19; Ps 22,6; Ps 25,3; Ps 25,20}

\bibleverse{6} Denn auch Christus, da wir noch schwach waren nach der
Zeit, ist für uns Gottlose gestorben. \bibleverse{7} Nun stirbt kaum
jemand um eines Gerechten willen; um des Guten willen dürfte vielleicht
jemand sterben. \bibleverse{8} Darum preiset Gott seine Liebe gegen uns,
dass Christus für uns gestorben ist, da wir noch Sünder waren.

\bibleverse{9} So werden wir ja viel mehr durch ihn bewahret werden vor
dem Zorn, nachdem wir durch sein Blut gerecht geworden sind. \footnote{\textbf{5:9}
  Röm 1,18; Röm 2,5; Röm 2,8} \bibleverse{10} Denn so wir Gott versöhnt
sind durch den Tod seines Sohnes, als wir noch Feinde waren, viel mehr
werden wir selig werden durch sein Leben, so wir nun versöhnt sind.
\footnote{\textbf{5:10} Röm 8,7; Kol 1,21; 2Kor 5,18}

\bibleverse{11} Nicht allein aber das, sondern wir rühmen uns auch
Gottes durch unseren Herrn Jesus Christus, durch welchen wir nun die
Versöhnung empfangen haben. \bibleverse{12} Derhalben, wie durch einen
Menschen die Sünde ist gekommen in die Welt und der Tod durch die Sünde,
und ist also der Tod zu allen Menschen durchgedrungen, dieweil sie alle
gesündigt haben; -- \footnote{\textbf{5:12} 1Mo 2,17; 1Mo 3,19; Röm 6,23}
\bibleverse{13} denn die Sünde war wohl in der Welt bis auf das Gesetz;
aber wo kein Gesetz ist, da achtet man der Sünde nicht. \footnote{\textbf{5:13}
  Röm 4,15} \bibleverse{14} Doch herrschte der Tod von Adam an bis auf
Moses auch über die, die nicht gesündigt haben mit gleicher Übertretung
wie Adam, welcher ist ein Bild des, der zukünftig war.

\bibleverse{15} Aber nicht verhält sich's mit der Gabe wie mit der
Sünde. Denn so an eines Sünde viele gestorben sind, so ist viel mehr
Gottes Gnade und Gabe vielen reichlich widerfahren durch die Gnade des
einen Menschen Jesus Christus. \bibleverse{16} Und nicht ist die Gabe
allein über eine Sünde, wie durch des einen Sünders eine Sünde alles
Verderben. Denn das Urteil ist gekommen aus einer Sünde zur Verdammnis;
die Gabe aber hilft auch aus vielen Sünden zur Gerechtigkeit.
\bibleverse{17} Denn so um des einen Sünde willen der Tod geherrscht hat
durch den einen, viel mehr werden die, die da empfangen die Fülle der
Gnade und der Gabe zur Gerechtigkeit, herrschen im Leben durch einen,
Jesum Christum.

\bibleverse{18} Wie nun durch eines Sünde die Verdammnis über alle
Menschen gekommen ist, also ist auch durch eines Gerechtigkeit die
Rechtfertigung des Lebens über alle Menschen gekommen. \footnote{\textbf{5:18}
  1Kor 15,21-22} \bibleverse{19} Denn gleichwie durch eines Menschen
Ungehorsam viele Sünder geworden sind, also auch durch eines Gehorsam
werden viele Gerechte. \footnote{\textbf{5:19} Röm 3,26; Jes 53,11}
\bibleverse{20} Das Gesetz aber ist neben eingekommen, auf dass die
Sünde mächtiger würde. Wo aber die Sünde mächtig geworden ist, da ist
doch die Gnade viel mächtiger geworden, \footnote{\textbf{5:20} Röm 7,8;
  Röm 7,13; Gal 3,19} \bibleverse{21} auf dass, gleichwie die Sünde
geherrscht hat zum Tode, also auch herrsche die Gnade durch die
Gerechtigkeit zum ewigen Leben durch Jesum Christum, unseren Herrn.
\footnote{\textbf{5:21} Röm 6,23}

\hypertarget{section-2}{%
\section{6}\label{section-2}}

\bibleverse{1} Was wollen wir hierzu sagen? Sollen wir denn in der Sünde
beharren, auf dass die Gnade desto mächtiger werde? \footnote{\textbf{6:1}
  Röm 3,5-8} \bibleverse{2} Das sei ferne! Wie sollten wir in der Sünde
wollen leben, der wir abgestorben sind? \bibleverse{3} Wisset ihr nicht,
dass alle, die wir in Jesum Christum getauft sind, die sind in seinen
Tod getauft? \bibleverse{4} So sind wir ja mit ihm begraben durch die
Taufe in den Tod, auf dass, gleichwie Christus ist auferweckt von den
Toten durch die Herrlichkeit des Vaters, also sollen auch wir in einem
neuen Leben wandeln. \footnote{\textbf{6:4} Kol 2,12; 1Petr 3,21}

\bibleverse{5} So wir aber samt ihm gepflanzt werden zu gleichem Tode,
so werden wir auch seiner Auferstehung gleich sein, \bibleverse{6}
dieweil wir wissen, dass unser alter Mensch samt ihm gekreuzigt ist, auf
dass der sündliche Leib aufhöre, dass wir hinfort der Sünde nicht mehr
dienen. \bibleverse{7} Denn wer gestorben ist, der ist gerechtfertigt
von der Sünde. \bibleverse{8} Sind wir aber mit Christo gestorben, so
glauben wir, dass wir auch mit ihm leben werden, \bibleverse{9} und
wissen, dass Christus, von den Toten auferweckt, hinfort nicht stirbt;
der Tod wird hinfort über ihn nicht herrschen. \bibleverse{10} Denn was
er gestorben ist, das ist er der Sünde gestorben zu einem Mal; was er
aber lebt, das lebt er Gott. \footnote{\textbf{6:10} Hebr 9,26-28}
\bibleverse{11} Also auch ihr, haltet euch dafür, dass ihr der Sünde
gestorben seid und lebet Gott in Christo Jesu, unserem Herrn.
\footnote{\textbf{6:11} 2Kor 5,15; 1Petr 2,24}

\bibleverse{12} So lasset nun die Sünde nicht herrschen in eurem
sterblichen Leibe, ihr Gehorsam zu leisten in seinen Lüsten. \footnote{\textbf{6:12}
  1Mo 4,7} \bibleverse{13} Auch begebet nicht der Sünde eure Glieder zu
Waffen der Ungerechtigkeit, sondern begebet euch selbst Gott, als die da
aus den Toten lebendig sind, und eure Glieder Gott zu Waffen der
Gerechtigkeit. \footnote{\textbf{6:13} Röm 12,1} \bibleverse{14} Denn
die Sünde wird nicht herrschen können über euch, sintemal ihr nicht
unter dem Gesetze seid, sondern unter der Gnade. \footnote{\textbf{6:14}
  Röm 7,4-6; 1Jo 3,6}

\bibleverse{15} Wie nun? Sollen wir sündigen, dieweil wir nicht unter
dem Gesetz, sondern unter der Gnade sind? Das sei ferne! \footnote{\textbf{6:15}
  Röm 5,17; Röm 5,21} \bibleverse{16} Wisset ihr nicht: welchem ihr euch
begebet zu Knechten in Gehorsam, des Knechte seid ihr, dem ihr gehorsam
seid, es sei der Sünde zum Tode oder dem Gehorsam zur Gerechtigkeit?
\footnote{\textbf{6:16} Joh 8,34} \bibleverse{17} Gott sei aber gedankt,
dass ihr Knechte der Sünde gewesen seid, aber nun gehorsam geworden von
Herzen dem Vorbilde der Lehre, welchem ihr ergeben seid. \bibleverse{18}
Denn nun ihr frei geworden seid von der Sünde, seid ihr Knechte geworden
der Gerechtigkeit.

\bibleverse{19} Ich muss menschlich davon reden um der Schwachheit
willen eures Fleisches. Gleichwie ihr eure Glieder begeben habt zum
Dienst der Unreinigkeit und von einer Ungerechtigkeit zu der anderen,
also begebet auch nun eure Glieder zum Dienst der Gerechtigkeit, dass
sie heilig werden. \bibleverse{20} Denn da ihr der Sünde Knechte waret,
da waret ihr frei von der Gerechtigkeit. \bibleverse{21} Was hattet ihr
nun zu der Zeit für Frucht? Welcher ihr euch jetzt schämet; denn ihr
Ende ist der Tod. \footnote{\textbf{6:21} Röm 8,6; Röm 8,13}
\bibleverse{22} Nun ihr aber seid von der Sünde frei und Gottes Knechte
geworden, habt ihr eure Frucht, dass ihr heilig werdet, das Ende aber
das ewige Leben. \bibleverse{23} Denn der Tod ist der Sünde Sold; aber
die Gabe Gottes ist das ewige Leben in Christo Jesu, unserem Herrn. \# 7
\bibleverse{1} Wisset ihr nicht, liebe Brüder (denn ich rede mit
solchen, die das Gesetz wissen), dass das Gesetz herrscht über den
Menschen solange er lebt? \bibleverse{2} Denn ein Weib, das unter dem
Manne ist, ist an ihn gebunden durch das Gesetz, solange der Mann lebt;
wenn aber der Mann stirbt, so ist sie los vom Gesetz, das den Mann
betrifft. \footnote{\textbf{7:2} 1Kor 7,39} \bibleverse{3} Wenn sie nun
eines anderen Mannes wird, solange der Mann lebt, wird sie eine
Ehebrecherin geheißen; wenn aber der Mann stirbt, ist sie frei vom
Gesetz, dass sie nicht eine Ehebrecherin ist, wenn sie eines anderen
Mannes wird. \bibleverse{4} Also seid auch ihr, meine Brüder, getötet
dem Gesetz durch den Leib Christi, dass ihr eines anderen seid, nämlich
des, der von den Toten auferweckt ist, auf dass wir Gott Frucht bringen.
\bibleverse{5} Denn da wir im Fleisch waren, da waren die sündlichen
Lüste, welche durchs Gesetz sich erregten, kräftig in unseren Gliedern,
dem Tode Frucht zu bringen. \bibleverse{6} Nun aber sind wir vom Gesetz
los und ihm abgestorben, das uns gefangenhielt, also dass wir dienen
sollen im neuen Wesen des Geistes und nicht im alten Wesen des
Buchstabens. \footnote{\textbf{7:6} Röm 8,1-2; Röm 6,2; Röm 6,4}

\bibleverse{7} Was wollen wir denn nun sagen? Ist das Gesetz Sünde? Das
sei ferne! Aber die Sünde erkannte ich nicht, außer durchs Gesetz. Denn
ich wusste nichts von der Lust, wenn das Gesetz nicht hätte gesagt:
„Lass dich nicht gelüsten!{}`` \bibleverse{8} Da nahm aber die Sünde
Ursache am Gebot und erregte in mir allerlei Lust; denn ohne das Gesetz
war die Sünde tot. \bibleverse{9} Ich aber lebte vordem ohne Gesetz; da
aber das Gebot kam, ward die Sünde wieder lebendig, \bibleverse{10} ich
aber starb; und es fand sich, dass das Gebot mir zum Tode gereichte, das
mir doch zum Leben gegeben war. \footnote{\textbf{7:10} Jak 1,15; 3Mo
  18,5} \bibleverse{11} Denn die Sünde nahm Ursache am Gebot und betrog
mich und tötete mich durch dasselbe Gebot. \footnote{\textbf{7:11} Hebr
  3,13} \bibleverse{12} Das Gesetz ist ja heilig, und das Gebot ist
heilig, recht und gut. \footnote{\textbf{7:12} 1Tim 1,8}

\bibleverse{13} Ist denn, das da gut ist, mir zum Tod geworden? Das sei
ferne! Aber die Sünde, auf dass sie erscheine, wie sie Sünde ist, hat
sie mir durch das Gute den Tod gewirkt, auf dass die Sünde würde überaus
sündig durchs Gebot. \footnote{\textbf{7:13} Röm 5,20} \bibleverse{14}
Denn wir wissen, dass das Gesetz geistlich ist; ich bin aber
fleischlich, unter die Sünde verkauft. \footnote{\textbf{7:14} Joh 3,6}
\bibleverse{15} Denn ich weiß nicht, was ich tue. Denn ich tue nicht,
was ich will; sondern, was ich hasse, das tue ich. \bibleverse{16} So
ich aber das tue, was ich nicht will, so gebe ich zu, dass das Gesetz
gut sei. \bibleverse{17} So tue nun ich dasselbe nicht, sondern die
Sünde, die in mir wohnt. \bibleverse{18} Denn ich weiß, dass in mir, das
ist in meinem Fleische, wohnt nichts Gutes. Wollen habe ich wohl, aber
vollbringen das Gute finde ich nicht. \bibleverse{19} Denn das Gute, das
ich will, das tue ich nicht; sondern das Böse, das ich nicht will, das
tue ich. \bibleverse{20} So ich aber tue, was ich nicht will, so tue ich
dasselbe nicht; sondern die Sünde, die in mir wohnt. \bibleverse{21} So
finde ich mir nun ein Gesetz, der ich will das Gute tun, dass mir das
Böse anhangt. \bibleverse{22} Denn ich habe Lust an Gottes Gesetz nach
dem inwendigen Menschen. \bibleverse{23} Ich sehe aber ein anderes
Gesetz in meinen Gliedern, das da widerstreitet dem Gesetz in meinem
Gemüte und nimmt mich gefangen in der Sünde Gesetz, welches ist in
meinen Gliedern. \footnote{\textbf{7:23} Gal 5,17} \bibleverse{24} Ich
elender Mensch! wer wird mich erlösen von dem Leibe dieses Todes?
\bibleverse{25} Ich danke Gott durch Jesum Christum, unserem Herrn. So
diene ich nun mit dem Gemüte dem Gesetz Gottes, aber mit dem Fleische
dem Gesetz der Sünde. \# 8 \bibleverse{1} So ist nun nichts
Verdammliches an denen, die in Christo Jesu sind, die nicht nach dem
Fleisch wandeln, sondern nach dem Geist. \footnote{\textbf{8:1} Röm
  8,33-34} \bibleverse{2} Denn das Gesetz des Geistes, der da lebendig
macht in Christo Jesu, hat mich frei gemacht von dem Gesetz der Sünde
und des Todes. \bibleverse{3} Denn was dem Gesetz unmöglich war
(sintemal es durch das Fleisch geschwächt ward), das tat Gott und sandte
seinen Sohn in der Gestalt des sündlichen Fleisches und der Sünde halben
und verdammte die Sünde im Fleisch, \bibleverse{4} auf dass die
Gerechtigkeit, vom Gesetz erfordert, in uns erfüllt würde, die wir nun
nicht nach dem Fleische wandeln, sondern nach dem Geist. \footnote{\textbf{8:4}
  Gal 5,16; Gal 5,25} \bibleverse{5} Denn die da fleischlich sind, die
sind fleischlich gesinnt; die aber geistlich sind, die sind geistlich
gesinnt. \bibleverse{6} Aber fleischlich gesinnt sein ist der Tod, und
geistlich gesinnt sein ist Leben und Friede. \bibleverse{7} Denn
fleischlich gesinnt sein ist eine Feindschaft wider Gott, sintemal das
Fleisch dem Gesetz Gottes nicht untertan ist; denn es vermag's auch
nicht. \footnote{\textbf{8:7} Jak 4,4} \bibleverse{8} Die aber
fleischlich sind, können Gott nicht gefallen.

\bibleverse{9} Ihr aber seid nicht fleischlich, sondern geistlich, so
anders Gottes Geist in euch wohnt. Wer aber Christi Geist nicht hat, der
ist nicht sein. \bibleverse{10} So aber Christus in euch ist, so ist der
Leib zwar tot um der Sünde willen, der Geist aber ist Leben um der
Gerechtigkeit willen. \bibleverse{11} So nun der Geist des, der Jesum
von den Toten auferweckt hat, in euch wohnt, so wird auch derselbe, der
Christum von den Toten auferweckt hat, eure sterblichen Leiber lebendig
machen um deswillen, dass sein Geist in euch wohnt.

\bibleverse{12} So sind wir nun, liebe Brüder, Schuldner nicht dem
Fleisch, dass wir nach dem Fleisch leben. \footnote{\textbf{8:12} Röm
  6,7; Röm 6,18} \bibleverse{13} Denn wenn ihr nach dem Fleisch lebet,
so werdet ihr sterben müssen; wenn ihr aber durch den Geist des
Fleisches Geschäfte tötet, so werdet ihr leben. \footnote{\textbf{8:13}
  Röm 7,24; Gal 6,8; Eph 4,22-24} \bibleverse{14} Denn welche der Geist
Gottes treibt, die sind Gottes Kinder. \bibleverse{15} Denn ihr habt
nicht einen knechtischen Geist empfangen, dass ihr euch abermals
fürchten müsstet; sondern ihr habt einen kindlichen Geist empfangen,
durch welchen wir rufen: Abba, lieber Vater! \footnote{\textbf{8:15}
  2Tim 1,7; Gal 4,5-6}

\bibleverse{16} Derselbe Geist gibt Zeugnis unserem Geist, dass wir
Gottes Kinder sind. \footnote{\textbf{8:16} 2Kor 1,22} \bibleverse{17}
Sind wir denn Kinder, so sind wir auch Erben, nämlich Gottes Erben und
Miterben Christi, so wir anders mit leiden, auf dass wir auch mit zur
Herrlichkeit erhoben werden. \footnote{\textbf{8:17} Gal 4,7; Offb 21,7}

\bibleverse{18} Denn ich halte es dafür, dass dieser Zeit Leiden der
Herrlichkeit nicht wert sei, die an uns soll offenbart werden.
\footnote{\textbf{8:18} 2Kor 4,17} \bibleverse{19} Denn das ängstliche
Harren der Kreatur wartet auf die Offenbarung der Kinder Gottes.
\footnote{\textbf{8:19} Kol 3,4; 1Jo 3,2} \bibleverse{20} Sintemal die
Kreatur unterworfen ist der Eitelkeit ohne ihren Willen, sondern um
deswillen, der sie unterworfen hat, auf Hoffnung. \footnote{\textbf{8:20}
  1Mo 3,17; Pred 1,2} \bibleverse{21} Denn auch die Kreatur wird frei
werden vom Dienst des vergänglichen Wesens zu der herrlichen Freiheit
der Kinder Gottes. \footnote{\textbf{8:21} 2Petr 3,13} \bibleverse{22}
Denn wir wissen, dass alle Kreatur sehnt sich mit uns und ängstet sich
noch immerdar. \bibleverse{23} Nicht allein aber sie, sondern auch wir
selbst, die wir haben des Geistes Erstlinge, sehnen uns auch bei uns
selbst nach der Kindschaft und warten auf unseres Leibes Erlösung.
\bibleverse{24} Denn wir sind wohl selig, doch in der Hoffnung. Die
Hoffnung aber, die man sieht, ist nicht Hoffnung; denn wie kann man des
hoffen, das man sieht? \footnote{\textbf{8:24} 2Kor 5,7} \bibleverse{25}
So wir aber des hoffen, das wir nicht sehen, so warten wir sein durch
Geduld. \footnote{\textbf{8:25} Gal 5,5}

\bibleverse{26} Desgleichen auch der Geist hilft unserer Schwachheit
auf. Denn wir wissen nicht, was wir beten sollen, wie sich's gebührt;
sondern der Geist selbst vertritt uns aufs beste mit unaussprechlichem
Seufzen. \bibleverse{27} Der aber die Herzen erforscht, der weiß, was
des Geistes Sinn sei; denn er vertritt die Heiligen nach dem, das Gott
gefällt.

\bibleverse{28} Wir wissen aber, dass denen, die Gott lieben, alle Dinge
zum Besten dienen, denen, die nach dem Vorsatz berufen sind. \footnote{\textbf{8:28}
  Eph 1,11} \bibleverse{29} Denn welche er zuvor ersehen hat, die hat er
auch verordnet, dass sie gleich sein sollten dem Ebenbilde seines
Sohnes, auf dass derselbe der Erstgeborene sei unter vielen Brüdern.
\footnote{\textbf{8:29} Kol 1,18; Hebr 1,6} \bibleverse{30} Welche er
aber verordnet hat, die hat er auch berufen; welche er aber berufen hat,
die hat er auch gerecht gemacht, welche er aber hat gerecht gemacht, die
hat er auch herrlich gemacht. \footnote{\textbf{8:30} Röm 3,26; 2Thes
  2,13-14}

\bibleverse{31} Was wollen wir nun hierzu sagen? Ist Gott für uns, wer
mag wider uns sein? \footnote{\textbf{8:31} Ps 118,6} \bibleverse{32}
welcher auch seines eigenen Sohnes nicht hat verschont, sondern hat ihn
für uns alle dahingegeben; wie sollte er uns mit ihm nicht alles
schenken? \footnote{\textbf{8:32} Joh 3,16} \bibleverse{33} Wer will die
Auserwählten Gottes beschuldigen? Gott ist hier, der da gerecht macht.
\bibleverse{34} Wer will verdammen? Christus ist hier, der gestorben
ist, ja vielmehr, der auch auferweckt ist, welcher ist zur Rechten
Gottes und vertritt uns.

\bibleverse{35} Wer will uns scheiden von der Liebe Gottes? Trübsal oder
Angst oder Verfolgung oder Hunger oder Blöße oder Fährlichkeit oder
Schwert? \bibleverse{36} wie geschrieben steht: „Um deinetwillen werden
wir getötet den ganzen Tag; wir sind geachtet wie Schlachtschafe.``
\footnote{\textbf{8:36} 2Kor 4,11} \bibleverse{37} Aber in dem allem
überwinden wir weit um deswillen, der uns geliebt hat. \footnote{\textbf{8:37}
  1Jo 5,4} \bibleverse{38} Denn ich bin gewiss, dass weder Tod noch
Leben, weder Engel noch Fürstentümer noch Gewalten, weder Gegenwärtiges
noch Zukünftiges, \footnote{\textbf{8:38} Eph 6,12} \bibleverse{39}
weder Hohes noch Tiefes noch keine andere Kreatur mag uns scheiden von
der Liebe Gottes, die in Christo Jesu ist, unserem Herrn. \# 9
\bibleverse{1} Ich sage die Wahrheit in Christo und lüge nicht, wie mir
Zeugnis gibt mein Gewissen in dem Heiligen Geist, \bibleverse{2} dass
ich große Traurigkeit und Schmerzen ohne Unterlass in meinem Herzen
habe. \bibleverse{3} Ich habe gewünscht, verbannt zu sein von Christo
für meine Brüder, die meine Gefreundeten sind nach dem Fleisch;
\bibleverse{4} die da sind von Israel, welchen gehört die Kindschaft und
die Herrlichkeit und der Bund und das Gesetz und der Gottesdienst und
die Verheißungen; \footnote{\textbf{9:4} 2Mo 4,22; 5Mo 7,6; 1Mo 17,7;
  2Mo 20,-1; 2Mo 40,34} \bibleverse{5} welcher auch sind die Väter, und
aus welchen Christus herkommt nach dem Fleisch, der da ist Gott über
alles, gelobt in Ewigkeit. Amen. \footnote{\textbf{9:5} Mt 1,-1; Lk
  3,23-34; Joh 1,1; Röm 1,3}

\bibleverse{6} Aber nicht sage ich solches, als ob Gottes Wort darum aus
sei. Denn es sind nicht alle Israeliter, die von Israel sind;
\footnote{\textbf{9:6} 4Mo 23,19; Röm 2,28} \bibleverse{7} auch nicht
alle, die Abrahams Same sind, sind darum auch Kinder. Sondern „in Isaak
soll dir der Same genannt sein``. \bibleverse{8} Das ist: nicht sind das
Gottes Kinder, die nach dem Fleisch Kinder sind; sondern die Kinder der
Verheißung werden für Samen gerechnet. \bibleverse{9} Denn dies ist ein
Wort der Verheißung, da er spricht: „Um diese Zeit will ich kommen, und
Sara soll einen Sohn haben.`` \bibleverse{10} Nicht allein aber ist's
mit dem also, sondern auch, da Rebekka von dem einen, unserem Vater
Isaak, schwanger ward: \bibleverse{11} ehe die Kinder geboren waren und
weder Gutes noch Böses getan hatten -- auf dass der Vorsatz Gottes
bestünde nach der Wahl, \bibleverse{12} nicht aus Verdienst der Werke,
sondern aus Gnade des Berufers --, ward zu ihr gesagt: „Der Ältere soll
dienstbar werden dem Jüngeren``, \bibleverse{13} wie denn geschrieben
steht: „Jakob habe ich geliebt, aber Esau habe ich gehasst.``

\bibleverse{14} Was wollen wir denn hier sagen? Ist denn Gott ungerecht?
Das sei ferne! \bibleverse{15} Denn er spricht zu Mose: „Welchem ich
gnädig bin, dem bin ich gnädig; und welches ich mich erbarme, des
erbarme ich mich.`` \bibleverse{16} So liegt es nun nicht an jemandes
Wollen oder Laufen, sondern an Gottes Erbarmen. \footnote{\textbf{9:16}
  Eph 2,8} \bibleverse{17} Denn die Schrift sagt zum Pharao: „Ebendarum
habe ich dich erweckt, dass ich an dir meine Macht erzeige, auf dass
mein Name verkündigt werde in allen Landen.`` \bibleverse{18} So erbarmt
er sich nun, welches er will, und verstockt, welchen er will.

\bibleverse{19} So sagst du zu mir: Was beschuldigt er denn uns? Wer
kann seinem Willen widerstehen? \bibleverse{20} Ja, lieber Mensch, wer
bist du denn, dass du mit Gott rechten willst? Spricht auch ein Werk zu
seinem Meister: Warum machst du mich also? \footnote{\textbf{9:20} Jes
  45,9} \bibleverse{21} Hat nicht ein Töpfer Macht, aus einem Klumpen zu
machen ein Gefäß zu Ehren und das andere zu Unehren? \bibleverse{22}
Derhalben, da Gott wollte Zorn erzeigen und kundtun seine Macht, hat er
mit großer Geduld getragen die Gefäße des Zorns, die da zugerichtet sind
zur Verdammnis; \bibleverse{23} auf dass er kundtäte den Reichtum seiner
Herrlichkeit an den Gefäßen der Barmherzigkeit, die er bereitet hat zur
Herrlichkeit, \footnote{\textbf{9:23} Röm 8,29; Eph 1,3-12}
\bibleverse{24} welche er berufen hat, nämlich uns, nicht allein aus den
Juden, sondern auch aus den Heiden. \bibleverse{25} Wie er denn auch
durch Hosea spricht: „Ich will das mein Volk heißen, dass nicht mein
Volk war, und meine Liebe, die nicht die Liebe war.`` \bibleverse{26}
„Und soll geschehen: An dem Ort, da zu ihnen gesagt ward: „Ihr seid
nicht mein Volk``, sollen sie Kinder des lebendigen Gottes genannt
werden.``

\bibleverse{27} Jesaja aber schreit für Israel: „Wenn die Zahl der
Kinder Israel würde sein wie der Sand am Meer, so wird doch nur der
Überrest selig werden; \bibleverse{28} denn es wird ein Verderben und
Steuern geschehen zur Gerechtigkeit, und der Herr wird das Steuern tun
auf Erden.``

\bibleverse{29} Und wie Jesaja zuvor sagte: „Wenn uns nicht der Herr
Zebaoth hätte lassen Samen übrig bleiben, so wären wir wie Sodom
geworden und gleichwie Gomorra.``

\bibleverse{30} Was wollen wir nun hier sagen? Das wollen wir sagen: Die
Heiden, die nicht haben nach der Gerechtigkeit getrachtet, haben
Gerechtigkeit erlangt; ich sage aber von der Gerechtigkeit, die aus dem
Glauben kommt. \footnote{\textbf{9:30} Röm 10,20} \bibleverse{31} Israel
aber hat dem Gesetz der Gerechtigkeit nachgetrachtet, und hat das Gesetz
der Gerechtigkeit nicht erreicht. \footnote{\textbf{9:31} Röm 10,2-3}
\bibleverse{32} Warum das? Darum dass sie es nicht aus dem Glauben,
sondern aus den Werken des Gesetzes suchen. Denn sie haben sich gestoßen
an den Stein des Anlaufens, \bibleverse{33} wie geschrieben steht:
„Siehe da, ich lege in Zion einen Stein des Anlaufens und einen Fels des
Ärgernisses; und wer an ihn glaubt, der soll nicht zu Schanden werden.``
\footnote{\textbf{9:33} Mt 21,42; Mt 21,44; 1Petr 2,8}

\hypertarget{section-3}{%
\section{10}\label{section-3}}

\bibleverse{1} Liebe Brüder, meines Herzens Wunsch ist, und ich flehe
auch zu Gott für Israel, dass sie selig werden. \bibleverse{2} Denn ich
gebe ihnen das Zeugnis, dass sie eifern um Gott, aber mit Unverstand.
\bibleverse{3} Denn sie erkennen die Gerechtigkeit nicht, die vor Gott
gilt, und trachten, ihre eigene Gerechtigkeit aufzurichten, und sind
also der Gerechtigkeit, die vor Gott gilt, nicht untertan.
\bibleverse{4} Denn Christus ist des Gesetzes Ende; wer an den glaubt,
der ist gerecht.

\bibleverse{5} Mose schreibt wohl von der Gerechtigkeit, die aus dem
Gesetz kommt: „Welcher Mensch dies tut, der wird dadurch leben.``
\bibleverse{6} Aber die Gerechtigkeit aus dem Glauben spricht also:
„Sprich nicht in deinem Herzen: Wer will hinauf gen Himmel fahren?{}``
(Das ist nichts anderes denn Christum herabholen.) \bibleverse{7} Oder:
„Wer will hinab in die Tiefe fahren?{}`` (Das ist nichts anderes denn
Christum von den Toten holen.) \bibleverse{8} Aber was sagt sie? „Das
Wort ist dir nahe, in deinem Munde und in deinem Herzen.`` Dies ist das
Wort vom Glauben, das wir predigen. \bibleverse{9} Denn wenn du mit
deinem Munde bekennst Jesum, dass er der Herr sei, und glaubst in deinem
Herzen, dass ihn Gott von den Toten auferweckt hat, so wirst du selig.
\footnote{\textbf{10:9} Mt 10,32; 2Kor 4,5} \bibleverse{10} Denn wenn
man von Herzen glaubt, so wird man gerecht; und wenn man mit dem Munde
bekennt, so wird man selig. \bibleverse{11} Denn die Schrift spricht:
„Wer an ihn glaubt, wird nicht zu Schanden werden.``

\bibleverse{12} Es ist hier kein Unterschied unter Juden und Griechen;
es ist aller zumal ein Herr, reich über alle, die ihn anrufen.
\bibleverse{13} Denn „wer den Namen des Herrn wird anrufen, soll selig
werden.`` \bibleverse{14} Wie sollen sie aber den anrufen, an den sie
nicht glauben? Wie sollen sie aber an den glauben, von dem sie nichts
gehört haben? Wie sollen sie aber hören ohne Prediger? \bibleverse{15}
Wie sollen sie aber predigen, wenn sie nicht gesandt werden? Wie denn
geschrieben steht: „Wie lieblich sind die Füße derer, die den Frieden
verkündigen, die das Gute verkündigen!{}``

\bibleverse{16} Aber sie sind nicht alle dem Evangelium gehorsam. Denn
Jesaja spricht: „Herr, wer glaubt unserem Predigen?{}`` \bibleverse{17}
So kommt der Glaube aus der Predigt, das Predigen aber durch das Wort
Gottes. \footnote{\textbf{10:17} Joh 17,20} \bibleverse{18} Ich sage
aber: Haben sie es nicht gehört? Wohl, es ist ja in alle Lande
ausgegangen ihr Schall und in alle Welt ihre Worte. \footnote{\textbf{10:18}
  Röm 15,19}

\bibleverse{19} Ich sage aber: Hat es Israel nicht erkannt? Aufs erste
spricht Mose: „Ich will euch eifern machen über dem, das nicht ein Volk
ist; und über ein unverständiges Volk will ich euch erzürnen.``

\bibleverse{20} Jesaja aber darf wohl so sagen: „Ich bin gefunden von
denen, die mich nicht gesucht haben, und bin erschienen denen, die nicht
nach mir gefragt haben.`` \footnote{\textbf{10:20} Röm 9,30}

\bibleverse{21} Zu Israel aber spricht er: „Den ganzen Tag habe ich
meine Hände ausgestreckt zu dem Volk, das sich nicht sagen lässt und
widerspricht.`` \# 11 \bibleverse{1} So sage ich nun: Hat denn Gott sein
Volk verstoßen? Das sei ferne! Denn ich bin auch ein Israeliter von dem
Samen Abrahams, aus dem Geschlecht Benjamin. \bibleverse{2} Gott hat
sein Volk nicht verstoßen, welches er zuvor ersehen hat. Oder wisset ihr
nicht, was die Schrift sagt von Elia, wie er tritt vor Gott wider Israel
und spricht: \bibleverse{3} „Herr, sie haben deine Propheten getötet und
deine Altäre zerbrochen; und ich bin allein übriggeblieben, und sie
stehen mir nach meinem Leben``? \bibleverse{4} Aber was sagt ihm die
göttliche Antwort? „Ich habe mir lassen übrig bleiben siebentausend
Mann, die nicht haben ihre Knie gebeugt vor dem Baal.`` \bibleverse{5}
Also geht es auch jetzt zu dieser Zeit mit diesen, die übriggeblieben
sind nach der Wahl der Gnade. \footnote{\textbf{11:5} Röm 9,27}
\bibleverse{6} Ist's aber aus Gnaden, so ist's nicht aus Verdienst der
Werke; sonst würde Gnade nicht Gnade sein. Ist's aber aus Verdienst der
Werke, so ist die Gnade nichts; sonst wäre Verdienst nicht Verdienst.

\bibleverse{7} Wie denn nun? Was Israel sucht, das erlangte es nicht;
die Auserwählten aber erlangten es. Die anderen sind verstockt,
\bibleverse{8} wie geschrieben steht: „Gott hat ihnen gegeben eine Geist
des Schlafs, Augen, dass sie nicht sehen, und Ohren, dass sie nicht
hören, bis auf den heutigen Tag.`` \footnote{\textbf{11:8} 5Mo 29,3}

\bibleverse{9} Und David spricht: „Lass ihren Tisch zu einem Strick
werden und zu einer Berückung und zum Ärgernis und ihnen zur Vergeltung.
\bibleverse{10} Verblende ihre Augen, dass sie nicht sehen, und beuge
ihren Rücken allezeit.``

\bibleverse{11} So sage ich nun: Sind sie darum angelaufen, dass sie
fallen sollten? Das sei ferne! Sondern aus ihrem Fall ist den Heiden das
Heil widerfahren, auf dass sie denen nacheifern sollten. \bibleverse{12}
Denn so ihr Fall der Welt Reichtum ist, und ihr Schade ist der Heiden
Reichtum, wie viel mehr, wenn ihre Zahl voll würde?

\bibleverse{13} Mit euch Heiden rede ich; denn dieweil ich der Heiden
Apostel bin, will ich mein Amt preisen, \bibleverse{14} ob ich möchte
die, die mein Fleisch sind, zu eifern reizen und ihrer etliche selig
machen. \footnote{\textbf{11:14} 1Tim 4,16; 1Kor 9,20-22}
\bibleverse{15} Denn so ihre Verwerfung der Welt Versöhnung ist, was
wird ihre Annahme anderes sein als Leben von den Toten?

\bibleverse{16} Ist der Anbruch heilig, so ist auch der Teig heilig; und
wenn die Wurzel heilig ist, so sind auch die Zweige heilig.
\bibleverse{17} Ob aber nun etliche von den Zweigen ausgebrochen sind
und du, da du ein wilder Ölbaum warst, bist unter sie gepfropft und
teilhaftig geworden der Wurzel und des Safts im Ölbaum, \bibleverse{18}
so rühme dich nicht wider die Zweige. Rühmst du dich aber wider sie, so
sollst du wissen, dass du die Wurzel nicht trägst, sondern die Wurzel
trägt dich. \footnote{\textbf{11:18} Joh 4,22} \bibleverse{19} So
sprichst du: Die Zweige sind ausgebrochen, dass ich hineingepfropft
würde. \bibleverse{20} Ist wohl geredet! Sie sind ausgebrochen um ihres
Unglaubens willen; du stehst aber durch den Glauben. Sei nicht stolz,
sondern fürchte dich. \bibleverse{21} Hat Gott die natürlichen Zweige
nicht verschont, dass er vielleicht dich auch nicht verschone.
\bibleverse{22} Darum schau die Güte und den Ernst Gottes: den Ernst an
denen, die gefallen sind, die Güte aber an dir, sofern du an der Güte
bleibst; sonst wirst du auch abgehauen werden. \footnote{\textbf{11:22}
  Joh 15,2; Joh 15,4; Hebr 3,14} \bibleverse{23} Und jene, wenn sie
nicht bleiben in dem Unglauben, werden eingepfropft werden; Gott kann
sie wohl wieder einpfropfen. \bibleverse{24} Denn wenn du aus dem
Ölbaum, der von Natur wild war, bist abgehauen und wider die Natur in
den guten Ölbaum gepfropft, wie viel mehr werden die natürlichen
eingepfropft in ihren eigenen Ölbaum.

\bibleverse{25} Ich will euch nicht verhalten, liebe Brüder, dieses
Geheimnis (auf dass ihr nicht stolz seid): Blindheit ist Israel zum Teil
widerfahren, so lange, bis die Fülle der Heiden eingegangen sei
\bibleverse{26} und also das ganze Israel selig werde, wie geschrieben
steht: „Es wird kommen aus Zion, der da erlöse und abwende das gottlose
Wesen von Jakob. \footnote{\textbf{11:26} Mt 23,39; Ps 14,7}
\bibleverse{27} Und dies ist mein Testament mit ihnen, wenn ich ihre
Sünden werde wegnehmen.``

\bibleverse{28} Nach dem Evangelium sind sie zwar Feinde um euretwillen;
aber nach der Wahl sind sie Geliebte um der Väter willen.
\bibleverse{29} Gottes Gaben und Berufung können ihn nicht gereuen.
\bibleverse{30} Denn gleicherweise wie auch ihr vordem nicht habt
geglaubt an Gott, nun aber Barmherzigkeit überkommen habt durch ihren
Unglauben, \bibleverse{31} also haben auch jene jetzt nicht wollen
glauben an die Barmherzigkeit, die euch widerfahren ist, auf dass sie
auch Barmherzigkeit überkommen. \bibleverse{32} Denn Gott hat alle
beschlossen unter den Unglauben, auf dass er sich aller erbarme.
\footnote{\textbf{11:32} Gal 3,22; 1Tim 2,4}

\bibleverse{33} O welch eine Tiefe des Reichtums, beides, der Weisheit
und Erkenntnis Gottes! Wie gar unbegreiflich sind seine Gerichte und
unerforschlich seine Wege! \footnote{\textbf{11:33} Jes 45,15; Jes
  55,8-9} \bibleverse{34} Denn wer hat des Herrn Sinn erkannt, oder wer
ist sein Ratgeber gewesen? \footnote{\textbf{11:34} Jer 23,18; 1Kor 2,16}
\bibleverse{35} Oder wer hat ihm etwas zuvor gegeben, dass ihm werde
wiedervergolten?

\bibleverse{36} Denn von ihm und durch ihn und zu ihm sind alle Dinge.
Ihm sei Ehre in Ewigkeit! Amen. \# 12 \bibleverse{1} Ich ermahne euch
nun, liebe Brüder, durch die Barmherzigkeit Gottes, dass ihr eure Leiber
begebet zum Opfer, das da lebendig, heilig und Gott wohlgefällig sei,
welches sei euer vernünftiger Gottesdienst. \bibleverse{2} Und stellet
euch nicht dieser Welt gleich, sondern verändert euch durch Erneuerung
eures Sinnes, auf dass ihr prüfen möget, welches da sei der gute,
wohlgefällige und vollkommene Gotteswille. \footnote{\textbf{12:2} Eph
  4,23; Eph 5,10; Eph 5,17}

\bibleverse{3} Denn ich sage durch die Gnade, die mir gegeben ist,
jedermann unter euch, dass niemand weiter von sich halte, als sich's
gebührt zu halten, sondern dass er von sich mäßig halte, ein jeglicher,
nach dem Gott ausgeteilt hat das Maß des Glaubens. \footnote{\textbf{12:3}
  1Kor 4,6; 1Kor 12,11; Eph 4,7; Mt 20,26} \bibleverse{4} Denn
gleicherweise als wir in einem Leibe viele Glieder haben, aber alle
Glieder nicht einerlei Geschäft haben, \footnote{\textbf{12:4} 1Kor
  12,12} \bibleverse{5} also sind wir viele ein Leib in Christo, aber
untereinander ist einer des anderen Glied, \footnote{\textbf{12:5} 1Kor
  12,27; Eph 4,4; Eph 4,25} \bibleverse{6} und haben mancherlei Gaben
nach der Gnade, die uns gegeben ist. \footnote{\textbf{12:6} 1Kor 4,7;
  1Kor 12,4} \bibleverse{7} Hat jemand Weissagung, so sei sie dem
Glauben gemäß. Hat jemand ein Amt, so warte er des Amts. Lehrt jemand,
so warte er der Lehre. \footnote{\textbf{12:7} 1Petr 4,10-11}
\bibleverse{8} Ermahnt jemand, so warte er des Ermahnens. Gibt jemand,
so gebe er einfältig. Regiert jemand, so sei er sorgfältig. Übt jemand
Barmherzigkeit, so tue er's mit Lust. \footnote{\textbf{12:8} Mt 6,3;
  2Kor 8,2; 2Kor 9,7}

\bibleverse{9} Die Liebe sei nicht falsch. Hasset das Arge, hanget dem
Guten an. \footnote{\textbf{12:9} 1Tim 1,5; Am 5,15} \bibleverse{10} Die
brüderliche Liebe untereinander sei herzlich. Einer komme dem anderen
mit Ehrerbietung zuvor. \footnote{\textbf{12:10} Joh 13,4-15; Phil 2,3}
\bibleverse{11} Seid nicht träge in dem, was ihr tun sollt. Seid
brünstig im Geiste. Schicket euch in die Zeit. \footnote{\textbf{12:11}
  Offb 3,15; Apg 18,25; Kol 3,23} \bibleverse{12} Seid fröhlich in
Hoffnung, geduldig in Trübsal, haltet an am Gebet. \footnote{\textbf{12:12}
  1Thes 5,17; Lk 18,1-8; Kol 4,2} \bibleverse{13} Nehmet euch der
Notdurft der Heiligen an. Herberget gern. \footnote{\textbf{12:13} Hebr
  13,2; 3Jo 1,5-8}

\bibleverse{14} Segnet, die euch verfolgen; segnet und fluchet nicht.
\footnote{\textbf{12:14} Mt 5,44; 1Kor 4,12; Apg 7,59} \bibleverse{15}
Freuet euch mit den Fröhlichen und weinet mit den Weinenden. \footnote{\textbf{12:15}
  Ps 35,13-14; 2Kor 11,29} \bibleverse{16} Habt einerlei Sinn
untereinander. Trachtet nicht nach hohen Dingen, sondern haltet euch
herunter zu den Niedrigen. \footnote{\textbf{12:16} Röm 15,5; Phil 2,2}
\bibleverse{17} Haltet euch nicht selbst für klug. Vergeltet niemand
Böses mit Bösem. Fleißiget euch der Ehrbarkeit gegen jedermann.
\footnote{\textbf{12:17} Jes 5,21; 1Thes 5,15; Spr 20,22; 2Kor 8,21}
\bibleverse{18} Ist es möglich, soviel an euch ist, so habt mit allen
Menschen Frieden. \footnote{\textbf{12:18} Mk 9,50; Hebr 12,14}
\bibleverse{19} Rächet euch selber nicht, meine Liebsten, sondern gebet
Raum dem Zorn (Gottes); denn es steht geschrieben: „Die Rache ist mein;
ich will vergelten, spricht der Herr.`` \footnote{\textbf{12:19} 3Mo
  19,18; Mt 5,38-44} \bibleverse{20} So nun deinen Feind hungert, so
speise ihn; dürstet ihn, so tränke ihn. Wenn du das tust, so wirst du
feurige Kohlen auf sein Haupt sammeln. \footnote{\textbf{12:20} 2Kö 6,22}

\bibleverse{21} Lass dich nicht das Böse überwinden, sondern überwinde
das Böse mit Gutem. \# 13 \bibleverse{1} Jedermann sei untertan der
Obrigkeit, die Gewalt über ihn hat. Denn es ist keine Obrigkeit ohne von
Gott; wo aber Obrigkeit ist, die ist von Gott verordnet. \bibleverse{2}
Wer sich nun der Obrigkeit widersetzt, der widerstrebt Gottes Ordnung;
die aber widerstreben, werden über sich ein Urteil empfangen.
\bibleverse{3} Denn die Gewaltigen sind nicht den guten Werken, sondern
den bösen zu fürchten. Willst du dich aber nicht fürchten vor der
Obrigkeit, so tue Gutes, so wirst du Lob von ihr haben. \footnote{\textbf{13:3}
  1Petr 2,13-14} \bibleverse{4} Denn sie ist Gottes Dienerin dir zugut.
Tust du aber Böses, so fürchte dich; denn sie trägt das Schwert nicht
umsonst; sie ist Gottes Dienerin, eine Rächerin zur Strafe über den, der
Böses tut. \footnote{\textbf{13:4} 2Chr 19,6-7} \bibleverse{5} Darum
ist's not, untertan zu sein, nicht allein um der Strafe willen, sondern
auch um des Gewissens willen. \bibleverse{6} Derhalben müsst ihr auch
Schoß geben; denn sie sind Gottes Diener, die solchen Schutz sollen
handhaben. \bibleverse{7} So gebet nun jedermann, was ihr schuldig seid:
Schoß, dem der Schoß gebührt; Zoll, dem der Zoll gebührt; Furcht, dem
die Furcht gebührt; Ehre, dem die Ehre gebührt. \footnote{\textbf{13:7}
  Mt 22,21}

\bibleverse{8} Seid niemand nichts schuldig, als dass ihr euch
untereinander liebet; denn wer den anderen liebt, der hat das Gesetz
erfüllt. \footnote{\textbf{13:8} Gal 5,14; 1Tim 1,5} \bibleverse{9} Denn
was da gesagt ist: „Du sollst nicht ehebrechen; du sollst nicht töten;
du sollst nicht stehlen; du sollst nicht falsch Zeugnis geben; dich soll
nichts gelüsten``, und wenn ein anderes Gebot mehr ist, das wird in
diesem Wort zusammengefasst: „Du sollst deinen Nächsten lieben wie dich
selbst.`` \bibleverse{10} Die Liebe tut dem Nächsten nichts Böses. So
ist nun die Liebe des Gesetzes Erfüllung. \footnote{\textbf{13:10} 1Kor
  13,4; Mt 22,40}

\bibleverse{11} Und weil wir solches wissen, nämlich die Zeit, dass die
Stunde da ist, aufzustehen vom Schlaf (sintemal unser Heil jetzt näher
ist, denn da wir gläubig wurden; \footnote{\textbf{13:11} Eph 5,14;
  1Thes 5,6-8} \bibleverse{12} die Nacht ist vorgerückt, der Tag aber
nahe herbeigekommen): so lasset uns ablegen die Werke der Finsternis und
anlegen die Waffen des Lichtes. \footnote{\textbf{13:12} 1Jo 2,8; Eph
  5,11} \bibleverse{13} Lasset uns ehrbar wandeln als am Tage, nicht in
Fressen und Saufen, nicht in Kammern und Unzucht, nicht in Hader und
Neid; \footnote{\textbf{13:13} Lk 21,34; Eph 5,18} \bibleverse{14}
sondern ziehet an den Herrn Jesus Christus und wartet des Leibes, doch
also, dass er nicht geil werde. \footnote{\textbf{13:14} Gal 3,27; 1Kor
  9,27; Kol 2,23}

\hypertarget{section-4}{%
\section{14}\label{section-4}}

\bibleverse{1} Den Schwachen im Glauben nehmet auf und verwirret die
Gewissen nicht. \footnote{\textbf{14:1} Röm 15,1; 1Kor 8,9}
\bibleverse{2} Einer glaubt, er möge allerlei essen; welcher aber
schwach ist, der isst Kraut. \footnote{\textbf{14:2} 1Mo 1,29; 1Mo 9,3}
\bibleverse{3} Welcher isst, der verachte den nicht, der nicht isst; und
welcher nicht isst, der richte den nicht, der da isst; denn Gott hat ihn
aufgenommen. \footnote{\textbf{14:3} Kol 2,16} \bibleverse{4} Wer bist
du, dass du einen fremden Knecht richtest? Er steht oder fällt seinem
Herrn. Er mag aber wohl aufgerichtet werden; denn Gott kann ihn wohl
aufrichten. \footnote{\textbf{14:4} Mt 7,1; Jak 4,11; Jak 1,4-12}

\bibleverse{5} Einer hält einen Tag vor dem anderen; der andere aber
hält alle Tage gleich. Ein jeglicher sei in seiner Meinung gewiss.
\footnote{\textbf{14:5} Gal 4,10} \bibleverse{6} Welcher auf die Tage
hält, der tut's dem Herrn; und welcher nichts darauf hält, der tut's
auch dem Herrn. Welcher isst, der isst dem Herrn, denn er dankt Gott;
welcher nicht isst, der isst dem Herrn nicht und dankt Gott.
\bibleverse{7} Denn unser keiner lebt sich selber, und keiner stirbt
sich selber. \bibleverse{8} Leben wir, so leben wir dem Herrn; sterben
wir, so sterben wir dem Herrn. Darum, wir leben oder sterben, so sind
wir des Herrn. \footnote{\textbf{14:8} Gal 2,20; 2Kor 5,15}
\bibleverse{9} Denn dazu ist Christus auch gestorben und auferstanden
und wieder lebendig geworden, dass er über Tote und Lebendige Herr sei.

\bibleverse{10} Du aber, was richtest du deinen Bruder? Oder, du
anderer, was verachtest du deinen Bruder? Wir werden alle vor den
Richtstuhl Christi dargestellt werden; \bibleverse{11} denn es steht
geschrieben: „So wahr ich lebe, spricht der Herr, mir sollen alle Knie
gebeugt werden, und alle Zungen sollen Gott bekennen.`` \footnote{\textbf{14:11}
  Phil 2,10-11}

\bibleverse{12} So wird nun ein jeglicher für sich selbst Gott
Rechenschaft geben. \footnote{\textbf{14:12} Gal 6,5}

\bibleverse{13} Darum lasset uns nicht mehr einer den anderen richten;
sondern das richtet vielmehr, dass niemand seinem Bruder einen Anstoß
oder Ärgernis darstelle. \footnote{\textbf{14:13} 1Kor 10,33}
\bibleverse{14} Ich weiß und bin gewiss in dem Herrn Jesus, dass nichts
gemein ist an sich selbst; nur dem, der es rechnet für gemein, dem ist's
gemein. \footnote{\textbf{14:14} Mt 15,11; Apg 10,15; Tit 1,15}
\bibleverse{15} So aber dein Bruder um deiner Speise willen betrübt
wird, so wandelst du schon nicht nach der Liebe. Verderbe den nicht mit
deiner Speise, um welches willen Christus gestorben ist. \footnote{\textbf{14:15}
  1Kor 8,11-13} \bibleverse{16} Darum schaffet, dass euer Schatz nicht
verlästert werde. \bibleverse{17} Denn das Reich Gottes ist nicht Essen
und Trinken, sondern Gerechtigkeit und Friede und Freude in dem heiligen
Geiste. \bibleverse{18} Wer darin Christo dient, der ist Gott gefällig
und den Menschen wert. \bibleverse{19} Darum lasst uns dem nachstreben,
was zum Frieden dient und was zur Besserung untereinander dient.
\footnote{\textbf{14:19} Röm 12,18; Röm 15,2} \bibleverse{20} Verstöre
nicht um der Speise willen Gottes Werk. Es ist zwar alles rein; aber es
ist nicht gut dem, der es isst mit einem Anstoß seines Gewissens.
\bibleverse{21} Es ist besser, du essest kein Fleisch und trinkest
keinen Wein und tuest nichts, daran sich dein Bruder stößt oder ärgert
oder schwach wird.

\bibleverse{22} Hast du den Glauben, so habe ihn bei dir selbst vor
Gott. Selig ist, der sich selbst kein Gewissen macht in dem, was er
annimmt. \bibleverse{23} Wer aber darüber zweifelt, und isst doch, der
ist verdammt; denn es geht nicht aus dem Glauben. Was aber nicht aus dem
Glauben geht, das ist Sünde. \# 15 \bibleverse{1} Wir aber, die wir
stark sind, sollen der Schwachen Gebrechlichkeit tragen und nicht
Gefallen an uns selber haben. \footnote{\textbf{15:1} Röm 14,1}
\bibleverse{2} Es stelle sich ein jeglicher unter uns also, dass er
seinem Nächsten gefalle zum Guten, zur Besserung. \footnote{\textbf{15:2}
  1Kor 9,19; 1Kor 10,24; 1Kor 10,33} \bibleverse{3} Denn auch Christus
hatte nicht an sich selber Gefallen, sondern wie geschrieben steht: „Die
Schmähungen derer, die dich schmähen, sind auf mich gefallen.``
\bibleverse{4} Was aber zuvor geschrieben ist, das ist uns zur Lehre
geschrieben, auf dass wir durch Geduld und Trost der Schrift Hoffnung
haben. \footnote{\textbf{15:4} 1Kor 10,11} \bibleverse{5} Der Gott aber
der Geduld und des Trostes gebe euch, dass ihr einerlei gesinnt seid
untereinander nach Jesu Christo, \footnote{\textbf{15:5} Phil 2,2}
\bibleverse{6} auf dass ihr einmütig mit einem Munde lobet Gott und den
Vater unseres Herrn Jesu Christi.

\bibleverse{7} Darum nehmet euch untereinander auf, gleichwie euch
Christus hat aufgenommen zu Gottes Lobe. \bibleverse{8} Ich sage aber,
dass Jesus Christus sei ein Diener gewesen der Juden um der
Wahrhaftigkeit willen Gottes, zu bestätigen die Verheißungen, den Vätern
geschehen; \footnote{\textbf{15:8} Mt 15,24; Apg 3,25} \bibleverse{9}
dass die Heiden aber Gott loben um der Barmherzigkeit willen, wie
geschrieben steht: „Darum will ich dich loben unter den Heiden und
deinem Namen singen.``

\bibleverse{10} Und abermals spricht er: „Freut euch, ihr Heiden, mit
seinem Volk!{}``

\bibleverse{11} Und abermals: „Lobt den Herrn, alle Heiden, und preiset
ihn, alle Völker!{}``

\bibleverse{12} Und abermals spricht Jesaja: „Es wird sein die Wurzel
Jesses, und der auferstehen wird, zu herrschen über die Heiden; auf den
werden die Heiden hoffen.``

\bibleverse{13} Der Gott aber der Hoffnung erfülle euch mit aller Freude
und Frieden im Glauben, dass ihr völlige Hoffnung habet durch die Kraft
des heiligen Geistes.

\bibleverse{14} Ich weiß aber gar wohl von euch, liebe Brüder, dass ihr
selber voll Gütigkeit seid, erfüllt mit aller Erkenntnis, dass ihr euch
untereinander könnet ermahnen. \bibleverse{15} Ich habe es aber dennoch
gewagt und euch etwas wollen schreiben, liebe Brüder, euch zu erinnern,
um der Gnade willen, die mir von Gott gegeben ist, \footnote{\textbf{15:15}
  Röm 1,5; Röm 12,3} \bibleverse{16} dass ich soll sein ein Diener
Christi unter den Heiden, priesterlich zu warten des Evangeliums Gottes,
auf dass die Heiden ein Opfer werden, Gott angenehm, geheiligt durch den
heiligen Geist. \footnote{\textbf{15:16} Röm 11,13} \bibleverse{17}
Darum kann ich mich rühmen in Jesu Christo, dass ich Gott diene.
\bibleverse{18} Denn ich wollte nicht wagen, etwas zu reden, wo dasselbe
Christus nicht durch mich wirkte, die Heiden zum Gehorsam zu bringen
durch Wort und Werk, \footnote{\textbf{15:18} 2Kor 3,5; Röm 1,5}
\bibleverse{19} durch Kraft der Zeichen und Wunder und durch Kraft des
Geistes Gottes, also dass ich von Jerusalem an und umher bis Illyrien
alles mit dem Evangelium Christi erfüllt habe \footnote{\textbf{15:19}
  Mk 16,17; 2Kor 12,12} \bibleverse{20} und mich sonderlich geflissen,
das Evangelium zu predigen, wo Christi Name nicht bekannt war, auf dass
ich nicht auf einen fremden Grund baute, \footnote{\textbf{15:20} 2Kor
  10,15-16} \bibleverse{21} sondern wie geschrieben steht: „Welchen
nicht ist von ihm verkündigt, die sollen's sehen, und welche nicht
gehört haben, sollen's verstehen.``

\bibleverse{22} Das ist auch die Ursache, warum ich vielmal verhindert
worden, zu euch zu kommen. \bibleverse{23} Nun ich aber nicht mehr Raum
habe in diesen Ländern, habe aber Verlangen, zu euch zu kommen, von
vielen Jahren her, \footnote{\textbf{15:23} Röm 1,10-11} \bibleverse{24}
so will ich zu euch kommen, wenn ich reisen werde nach Spanien. Denn ich
hoffe, dass ich da durchreisen und euch sehen werde und von euch dorthin
geleitet werden möge, so doch, dass ich zuvor mich ein wenig an euch
ergötze. \bibleverse{25} Nun aber fahre ich hin gen Jerusalem den
Heiligen zu Dienst. \bibleverse{26} Denn die aus Mazedonien und Achaja
haben willig eine gemeinsame Steuer zusammengelegt den armen Heiligen zu
Jerusalem. \footnote{\textbf{15:26} 1Kor 16,1; 2Kor 8,1-4; 2Kor 8,9}
\bibleverse{27} Sie haben's willig getan, und sind auch ihre Schuldner.
Denn so die Heiden sind ihrer geistlichen Güter teilhaftig geworden,
ist's billig, dass sie ihnen auch in leiblichen Gütern Dienst beweisen.
\footnote{\textbf{15:27} 1Kor 9,11; Gal 6,6} \bibleverse{28} Wenn ich
nun solches ausgerichtet und ihnen diese Frucht versiegelt habe, will
ich durch euch nach Spanien ziehen. \bibleverse{29} Ich weiß aber, wenn
ich zu euch komme, dass ich mit vollem Segen des Evangeliums Christi
kommen werde.

\bibleverse{30} Ich ermahne euch aber, liebe Brüder, durch unseren Herrn
Jesus Christus und durch die Liebe des Geistes, dass ihr mir helfet
kämpfen mit Beten für mich zu Gott, \footnote{\textbf{15:30} 2Kor 1,11;
  2Thes 3,1} \bibleverse{31} auf dass ich errettet werde von den
Ungläubigen in Judäa und dass mein Dienst, den ich für Jerusalem tue,
angenehm werde den Heiligen, \footnote{\textbf{15:31} 1Thes 2,15}
\bibleverse{32} auf dass ich mit Freuden zu euch komme durch den Willen
Gottes und mich mit euch erquicke. \bibleverse{33} Der Gott aber des
Friedens sei mit euch allen! Amen. \# 16 \bibleverse{1} Ich befehle euch
aber unsere Schwester Phöbe, welche ist im Dienste der Gemeinde zu
Kenchreä, \bibleverse{2} dass ihr sie aufnehmet in dem Herrn, wie sich's
ziemt den Heiligen, und tut ihr Beistand in allem Geschäfte, darin sie
euer bedarf; denn sie hat auch vielen Beistand getan, auch mir selbst.

\bibleverse{3} Grüßet die Priscilla und den Aquila, meine Gehilfen in
Christo Jesu, \footnote{\textbf{16:3} Apg 18,2; Apg 18,18; Apg 18,26}
\bibleverse{4} welche haben für mein Leben ihren Hals dargegeben,
welchen nicht allein ich danke, sondern alle Gemeinden unter den Heiden.
\bibleverse{5} Auch grüßet die Gemeinde in ihrem Hause. Grüßet Epänetus,
meinen Lieben, welcher ist der Erstling unter denen aus Achaja in
Christo. \bibleverse{6} Grüßet Maria, welche viel Mühe und Arbeit mit
uns gehabt hat. \bibleverse{7} Grüßet den Andronikus und den Junias,
meine Gefreundeten und meine Mitgefangenen, welche sind berühmte Apostel
und vor mir gewesen in Christo. \bibleverse{8} Grüßet Amplias, meinen
Lieben in dem Herrn. \bibleverse{9} Grüßet Urban, unseren Gehilfen in
Christo, und Stachys, meinen Lieben. \bibleverse{10} Grüßet Apelles, den
Bewährten in Christo. Grüßet, die da sind von des Aristobulus Gesinde.
\bibleverse{11} Grüßet Herodion, meinen Gefreundeten. Grüßet, die da
sind von des Narzissus Gesinde in dem Herrn. \bibleverse{12} Grüßet die
Tryphäna und die Tryphosa, welche in dem Herrn gearbeitet haben. Grüßet
die Persis, meine Liebe, welch in dem Herrn viel gearbeitet hat.
\bibleverse{13} Grüßet Rufus, den Auserwählten in dem Herrn, und seine
und meine Mutter. \bibleverse{14} Grüßet Asynkritus, Phlegon, Hermas,
Patrobas, Hermes und die Brüder bei ihnen. \bibleverse{15} Grüßet
Philologus und die Julia, Nereus und seine Schwester und Olympas und
alle Heiligen bei ihnen. \bibleverse{16} Grüßet euch untereinander mit
dem heiligen Kuss. Es grüßen euch die Gemeinden Christi. \footnote{\textbf{16:16}
  1Kor 16,20}

\bibleverse{17} Ich ermahne euch aber, liebe Brüder, dass ihr achtet auf
die, die da Zertrennung und Ärgernis anrichten neben der Lehre, die ihr
gelernt habt, und weichet von ihnen. \footnote{\textbf{16:17} Mt 7,15;
  Tit 3,10; 2Thes 3,6} \bibleverse{18} Denn solche dienen nicht dem
Herrn Jesus Christus, sondern ihrem Bauche; und durch süße Worte und
prächtige Reden verführen sie die unschuldigen Herzen. \footnote{\textbf{16:18}
  Phil 3,19; Kol 2,4} \bibleverse{19} Denn euer Gehorsam ist bei
jedermann kund geworden. Derhalben freue ich mich über euch; ich will
aber, dass ihr weise seid zum Guten, aber einfältig zum Bösen.
\footnote{\textbf{16:19} Röm 1,8; 1Kor 14,20} \bibleverse{20} Aber der
Gott des Friedens zertrete den Satan unter eure Füße in kurzem. Die
Gnade unseres Herrn Jesu Christi sei mit euch!

\bibleverse{21} Es grüßen euch Timotheus, mein Gehilfe, und Luzius und
Jason und Sosipater, meine Gefreundeten. \footnote{\textbf{16:21} Apg
  16,1-3; Apg 17,6; Apg 19,22; Apg 20,4; Phil 2,19-22}

\bibleverse{22} Ich, Tertius, grüße euch, der ich diesen Brief
geschrieben habe, in dem Herrn. \bibleverse{23} Es grüßt euch Gajus,
mein und der ganzen Gemeinde Wirt. Es grüßt euch Erastus, der Stadt
Rentmeister, und Quartus, der Bruder. \bibleverse{24} Die Gnade unseres
Herrn Jesu Christi sei mit euch allen! Amen. \bibleverse{25} Dem aber,
der euch stärken kann laut meines Evangeliums und der Predigt von Jesu
Christo, durch welche das Geheimnis offenbart ist, das von der Welt her
verschwiegen gewesen ist, \^{}\^{} \bibleverse{26} nun aber offenbart,
auch kundgemacht durch der Propheten Schriften nach Befehl des ewigen
Gottes, den Gehorsam des Glaubens aufzurichten unter allen Heiden:
\bibleverse{27} demselben Gott, der allein weise ist, sei Ehre durch
Jesum Christum in Ewigkeit! Amen.
