\hypertarget{section}{%
\section{1}\label{section}}

\bibverse{1} Paulus, ein Knecht JEsu Christi, berufen zum Apostel,
ausgesondert, zu predigen das Evangelium GOttes, \bibverse{2} welches er
zuvor verheißen hat durch seine Propheten in der Heiligen Schrift,
\bibverse{3} von seinem Sohn, der geboren ist von dem Samen Davids nach
dem Fleisch \bibverse{4} und kräftiglich erweiset ein Sohn GOttes nach
dem Geist, der da heiliget, seit der Zeit er auferstanden ist von den
Toten, nämlich JEsus Christus, unser HErr \bibverse{5} (durch welchen
wir haben empfangen Gnade und Apostelamt, unter allen Heiden den
Gehorsam des Glaubens aufzurichten unter seinem Namen, \bibverse{6}
welcher ihr zum Teil auch seid, die da berufen sind von JEsu Christo):
\bibverse{7} Allen, die zu Rom sind, den Liebsten GOttes und berufenen
Heiligen: Gnade sei mit euch und Friede von GOtt, unserm Vater, und dem
HErrn JEsus Christus! \bibverse{8} Aufs erste danke ich meinem GOtt
durch JEsum Christum euer aller halben, daß man von eurem Glauben in
aller Welt saget. \bibverse{9} Denn GOtt ist mein Zeuge, welchem ich
diene in meinem Geist am Evangelium von seinem Sohn, daß ich ohne
Unterlaß euer gedenke \bibverse{10} und allezeit in meinem Gebet flehe,
ob sich's einmal zutragen wollte, daß ich zu euch käme durch GOttes
Willen. \bibverse{11} Denn mich verlanget, euch zu sehen, auf daß ich
euch mitteile etwas geistlicher Gabe, euch zu stärken, \bibverse{12} das
ist, daß ich samt euch getröstet würde durch euren und meinen Glauben,
den wir untereinander haben. \bibverse{13} Ich will euch aber nicht
verhalten, liebe Brüder, daß ich mir oft habe vorgesetzt, zu euch zu
kommen (bin aber verhindert bisher), daß ich auch unter euch Frucht
schaffete gleichwie unter andern Heiden. \bibverse{14} Ich bin ein
Schuldner beide, der Griechen und der Ungriechen, beide, der Weisen und
der Unweisen. \bibverse{15} Darum, soviel an mir ist bin ich geneigt,
auch euch zu Rom das Evangelium zu predigen. \bibverse{16} Denn ich
schäme mich des Evangeliums von Christo nicht; denn es ist eine Kraft
GOttes, die da selig machet alle, die daran glauben, die Juden
vornehmlich und auch die Griechen, \bibverse{17} sintemal darinnen
offenbaret wird die Gerechtigkeit, die vor GOtt gilt, welche kommt aus
Glauben in Glauben; wie denn geschrieben stehet: Der Gerechte wird
seines Glaubens leben. \bibverse{18} Denn GOttes Zorn vom Himmel wird
offenbart über alles gottlose Wesen und Ungerechtigkeit der Menschen,
die die Wahrheit in Ungerechtigkeit aufhalten. \bibverse{19} Denn daß
man weiß, daß GOtt sei, ist ihnen offenbar; denn GOtt hat es ihnen
offenbart \bibverse{20} damit, daß GOttes unsichtbares Wesen, das ist,
seine ewige Kraft und Gottheit, wird ersehen, so man des wahrnimmt an
den Werken, nämlich an der Schöpfung der Welt, also daß sie keine
Entschuldigung haben, \bibverse{21} dieweil sie wußten, daß ein GOtt
ist, und haben ihn nicht gepreiset als einen GOtt noch gedanket, sondern
sind in ihrem Dichten eitel worden, und ihr unverständiges Herz ist
verfinstert. \bibverse{22} Da sie sich für weise hielten, sind sie zu
Narren worden \bibverse{23} und haben verwandelt die Herrlichkeit des
unvergänglichen GOttes in ein Bild gleich dem vergänglichen Menschen und
der Vögel und der vierfüßigen und der kriechenden Tiere. \bibverse{24}
Darum hat sie auch GOtt dahingegeben in ihrer Herzen Gelüste, in
Unreinigkeit, zu schänden ihre eigenen Leiber an sich selbst.
\bibverse{25} Die GOttes Wahrheit haben verwandelt in die Lüge und haben
geehret und gedienet dem Geschöpfe mehr denn dem Schöpfer, der da
gelobet ist in Ewigkeit. Amen. \bibverse{26} Darum hat sie GOtt auch
dahingegeben in schändliche Lüste. Denn ihre Weiber haben verwandelt den
natürlichen Brauch in den unnatürlichen. \bibverse{27}
Desselbigengleichen auch die Männer haben verlassen den natürlichen
Brauch des Weibes und sind aneinander erhitzet in ihren Lüsten, und
haben Mann mit Mann Schande gewirket und den Lohn ihres Irrtums (wie es
denn sein sollte) an sich selbst empfangen. \bibverse{28} Und gleichwie
sie nicht geachtet haben, daß sie GOtt erkenneten, hat sie GOtt auch
dahingegeben in verkehrten Sinn, zu tun, was nicht taugt, \bibverse{29}
voll alles Ungerechten, Hurerei, Schalkheit, Geizes, Bosheit, voll
Hasses, Mordes, Haders, List, giftig, Ohrenbläser, \bibverse{30}
Verleumder, Gottesverächter, Frevler, hoffärtig, ruhmredig, Schädliche,
den Eltern ungehorsam, \bibverse{31} Unvernünftige, Treulose, störrig,
unversöhnlich, unbarmherzig \bibverse{32} die GOttes Gerechtigkeit
Wissen (daß, die solches tun, des Todes würdig sind), tun sie es nicht
allein, sondern haben auch Gefallen an denen, die es tun.

\hypertarget{section-1}{%
\section{2}\label{section-1}}

\bibverse{1} Darum, o Mensch, kannst du dich nicht entschuldigen, wer du
bist, der da richtet; denn worinnen du einen andern richtest, verdammst
du dich selbst, sintemal du eben dasselbige tust, was du richtest.
\bibverse{2} Denn wir wissen, daß GOttes Urteil ist recht über die, so
solches tun. \bibverse{3} Denkest du aber, o Mensch, der du richtest
die, so solches tun, und tust auch dasselbige, daß du dem Urteil GOttes
entrinnen werdest? \bibverse{4} Oder verachtest du den Reichtum seiner
Güte, Geduld und Langmütigkeit? Weißt du nicht, daß dich GOttes Güte zur
Buße leitet? \bibverse{5} Du aber nach deinem verstockten und
unbußfertigen Herzen häufest dir selbst den Zorn auf den Tag des Zorns
und der Offenbarung des gerechten Gerichtes GOttes \bibverse{6} welcher
geben wird einem jeglichen nach seinen Werken, \bibverse{7} nämlich
Preis und Ehre und, unvergängliches Wesen denen, die mit Geduld in guten
Werken trachten nach dem ewigen Leben, \bibverse{8} aber denen, die da
zänkisch sind und der Wahrheit nicht gehorchen, gehorchen aber dem
Ungerechten, Ungnade und Zorn; \bibverse{9} Trübsal und Angst über alle
Seelen der Menschen, die da Böses tun, vornehmlich der Juden und auch
der Griechen; \bibverse{10} Preis aber und Ehre und Friede allen denen,
die da Gutes tun, vornehmlich den Juden und auch den Griechen.
\bibverse{11} Denn es ist kein Ansehen der Person vor GOtt.
\bibverse{12} Welche ohne Gesetz gesündigt haben, die werden auch ohne
Gesetz verloren werden; und welche am Gesetz gesündiget haben, die
werden durchs Gesetz verurteilt werden, \bibverse{13} sintemal vor GOtt
nicht, die das Gesetz hören, gerecht sind, sondern die das Gesetz tun,
werden gerecht sein. \bibverse{14} Denn so die Heiden, die das Gesetz
nicht haben und doch von Natur tun des Gesetzes Werk, dieselbigen,
dieweil sie das Gesetz nicht haben, sind sie sich selbst ein Gesetz
\bibverse{15} damit, daß sie beweisen, des Gesetzes Werk sei beschrieben
in ihrem Herzen, sintemal ihr Gewissen sie bezeuget, dazu auch die
Gedanken, die sich untereinander verklagen oder entschuldigen,
\bibverse{16} auf den Tag, da GOtt das Verborgene der Menschen durch
JEsum Christum richten wird laut meines Evangeliums. \bibverse{17} Siehe
aber zu, du heißest ein Jude und verlässest dich aufs Gesetz und rühmest
dich GOttes \bibverse{18} und weißt seinen Willen, und weil du aus dem
Gesetz unterrichtet bist, prüfest du, was das Beste zu tun sei,
\bibverse{19} und vermissest dich, zu sein ein Leiter der Blinden ein
Licht derer, die in Finsternis sind, \bibverse{20} ein Züchtiger der
Törichten, ein Lehrer der Einfältigen, hast die Form, was zu wissen und
recht ist im Gesetz. \bibverse{21} Nun lehrest du andere und lehrest
dich selber nicht. Du predigest, man solle nicht stehlen, und du
stiehlst. \bibverse{22} Du sprichst, man solle nicht ehebrechen, und du
brichst die Ehe. Dir greuelt vor den Götzen und raubest GOtt, was sein
ist. \bibverse{23} Du rühmest dich des Gesetzes und schändest GOtt durch
Übertretung des Gesetzes. \bibverse{24} Denn eurethalben wird GOttes
Name gelästert unter den Heiden, als geschrieben stehet. \bibverse{25}
Die Beschneidung ist wohl nütz wenn du das Gesetz hältst; hältst du aber
das Gesetz nicht, so ist deine Beschneidung schon eine Vorhaut worden.
\bibverse{26} So nun die Vorhaut das Recht im Gesetz hält, meinest du
nicht, daß seine Vorhaut werde für eine Beschneidung gerechnet?
\bibverse{27} Und wird also, was von Natur eine Vorhaut ist und das
Gesetz vollbringet, dich richten, der du unter dem Buchstaben und
Beschneidung bist und das Gesetz übertrittst. \bibverse{28} Denn das ist
nicht ein Jude, der auswendig ein Jude ist, auch ist das nicht eine
Beschneidung, die auswendig im Fleisch geschieht, \bibverse{29} sondern
das ist ein Jude, der inwendig verborgen ist, und die Beschneidung des
Herzens ist eine Beschneidung, die im Geist und nicht im Buchstaben
geschieht, welches Lob ist nicht aus Menschen, sondern aus GOtt

\hypertarget{section-2}{%
\section{3}\label{section-2}}

\bibverse{1} Was haben denn die Juden Vorteils, oder was nützt die
Beschneidung? \bibverse{2} Zwar fast viel. Zum ersten, ihnen ist
vertrauet, was GOtt geredet hat. \bibverse{3} Daß aber etliche nicht
glauben an dasselbige, was liegt daran? Sollte ihr Unglaube GOttes
Glauben aufheben? \bibverse{4} Das sei ferne! Es bleibe vielmehr also,
daß GOtt sei wahrhaftig und alle Menschen falsch; wie geschrieben steht:
Auf daß du gerecht seiest in deinen Worten und überwindest, wenn du
gerichtet wirst. \bibverse{5} Ist's aber also, daß unsere
Ungerechtigkeit GOttes Gerechtigkeit preiset, was wollen wir sagen? Ist
denn GOtt auch ungerecht, daß er darüber zürnet? (Ich rede also auf
Menschenweise.) \bibverse{6} Das sei ferne! Wie könnte sonst GOtt die
Welt richten? \bibverse{7} Denn so die Wahrheit GOttes durch meine Lüge
herrlicher wird zu seinem Preis, warum sollte ich denn noch als ein
Sünder gerichtet werden \bibverse{8} und nicht vielmehr also tun, wie
wir gelästert werden, und wie etliche sprechen, daß wir sagen sollen:
Lasset uns Übel tun, auf daß Gutes daraus komme? Welcher Verdammnis ist
ganz recht. \bibverse{9} Was sagen wir denn nun? Haben wir einen
Vorteil? Gar keinen. Denn wir haben droben bewiesen daß beide, Juden und
Griechen, alle unter der Sünde sind, \bibverse{10} wie denn geschrieben
stehet: Da ist nicht, der gerecht sei, auch nicht einer; \bibverse{11}
da ist nicht, der verständig sei; da ist nicht, der nach GOtt frage.
\bibverse{12} Sie sind alle abgewichen und allesamt untüchtig worden; da
ist nicht, der Gutes tue, auch nicht einer. \bibverse{13} Ihr Schlund
ist ein offen Grab; mit ihren Zungen handeln sie trüglich; Otterngift
ist unter ihren Lippen; \bibverse{14} ihr Mund ist voll Fluchens und
Bitterkeit; \bibverse{15} ihre Füße sind eilend, Blut zu vergießen;
\bibverse{16} in ihren Wegen ist eitel Unfall und Herzeleid
\bibverse{17} und den Weg des Friedens wissen sie nicht. \bibverse{18}
Es ist keine Furcht GOttes vor ihren Augen. \bibverse{19} Wir wissen
aber, daß, was das Gesetz, sagt, das sagt es denen, die unter dem Gesetz
sind, auf daß aller Mund verstopfet werde, und alle Welt GOtt schuldig
sei \bibverse{20} darum, daß kein Fleisch durch des Gesetzes Werke vor
ihm gerecht sein mag; denn durch das Gesetz kommt Erkenntnis der Sünde.
\bibverse{21} Nun aber ist ohne Zutun des Gesetzes die Gerechtigkeit,
die vor GOtt gilt, offenbaret und bezeuget durch das Gesetz und die
Propheten. \bibverse{22} Ich sage aber von solcher Gerechtigkeit vor
GOtt, die da kommt durch den Glauben an JEsum Christum zu allen und auf
alle, die da glauben. \bibverse{23} Denn es ist hier kein Unterschied;
sie sind allzumal Sünder und mangeln des Ruhms, den sie an GOtt haben
sollten, \bibverse{24} und werden ohne Verdienst gerecht aus seiner
Gnade durch die Erlösung, so durch Christum JEsum geschehen ist,
\bibverse{25} welchen GOtt hat vorgestellt zu einem Gnadenstuhl durch
den Glauben in seinem Blut,damit er die Gerechtigkeit, die vor ihm gilt,
darbiete, in dem, daß er Sünde vergibt, welche bis anher geblieben war
unter göttlicher Geduld, \bibverse{26} auf daß er zu diesen Zeiten
darböte die Gerechtigkeit, die vor ihm gilt, auf daß er allein gerecht
sei und gerecht mache den, der da ist des Glaubens an JEsum.
\bibverse{27} Wo bleibt nun der Ruhm? Er ist aus. Durch welches Gesetz?
durch der Werke Gesetz? Nicht also, sondern durch des Glaubens Gesetz.
\bibverse{28} So halten wir es nun, daß der Mensch gerecht werde ohne
des Gesetzes Werke, allein durch den Glauben. \bibverse{29} Oder ist
GOtt allein der Juden GOtt? Ist er nicht auch der Heiden GOtt? Ja
freilich, auch der Heiden GOtt. \bibverse{30} Sintemal es ist ein
einiger GOtt, der da gerecht macht die Beschneidung aus dem Glauben und
die Vorhaut durch den Glauben. \bibverse{31} Wie? heben wir denn das
Gesetz auf durch den Glauben? Das sei ferne! sondern wir richten das
Gesetz auf.

\hypertarget{section-3}{%
\section{4}\label{section-3}}

\bibverse{1} Was sagen wir denn von unserm Vater Abraham, daß er
gefunden habe nach dem Fleisch? \bibverse{2} Das sagen wir: Ist Abraham
durch die Werke gerecht, so hat er wohl Ruhm, aber nicht vor GOtt.
\bibverse{3} Was sagt denn die Schrift? Abraham hat GOtt geglaubet, und
das ist ihm zur Gerechtigkeit gerechnet. \bibverse{4} Dem aber, der mit
Werken umgehet, wird der Lohn nicht aus Gnade zugerechnet, sondern aus
Pflicht. \bibverse{5} Dem aber, der nicht mit Werken umgehet, glaubet
aber an den, der die Gottlosen gerecht macht, dem wird sein Glaube
gerechnet zur Gerechtigkeit. \bibverse{6} Nach welcher Weise auch David
sagt, daß die Seligkeit sei allein des Menschen, welchem GOtt zurechnet
die Gerechtigkeit ohne Zutun der Werke, da er spricht: \bibverse{7}
Selig sind die, welchen ihre Ungerechtigkeiten vergeben sind, und
welchen ihre Sünden bedecket sind. \bibverse{8} Selig ist der Mann,
welchem GOtt keine Sünde zurechnet. \bibverse{9} Nun, diese Seligkeit,
gehet sie über die Beschneidung oder über die Vorhaut? Wir müssen je
sagen, daß Abraham sei sein Glaube zur Gerechtigkeit gerechnet.
\bibverse{10} Wie ist er ihm denn zugerechnet, in der Beschneidung oder
in der Vorhaut? Ohne Zweifel nicht in der Beschneidung, sondern in der
Vorhaut. \bibverse{11} Das Zeichen aber der Beschneidung empfing er zum
Siegel der Gerechtigkeit des Glaubens, welchen er noch in der Vorhaut
hatte, auf daß er würde ein Vater aller, die da glauben in der Vorhaut,
daß denselbigen solches auch gerechnet werde zur Gerechtigkeit
\bibverse{12} und würde auch ein Vater der Beschneidung, nicht allein
derer, die von der Beschneidung sind, sondern auch derer, die wandeln in
den Fußtapfen des Glaubens, welcher war in der Vorhaut unsers Vaters
Abraham. \bibverse{13} Denn die Verheißung, daß er sollte sein der Welt
Erbe, ist nicht geschehen Abraham oder seinem Samen durchs Gesetz,
sondern durch die Gerechtigkeit des Glaubens. \bibverse{14} Denn wo die
vom Gesetz Erben sind, so ist der Glaube nichts, und die Verheißung ist
ab. \bibverse{15} Sintemal das Gesetz richtet nur Zorn an; denn wo das
Gesetz nicht ist, da ist auch keine Übertretung. \bibverse{16} Derhalben
muß die Gerechtigkeit durch den Glauben kommen, auf daß sie sei aus
Gnaden, und die Verheißung fest bleibe allem Samen, nicht dem alleine,
der unter dem Gesetz ist, sondern auch dem, der des Glaubens Abrahams
ist, welcher ist unser aller Vater, \bibverse{17} wie geschrieben
stehet: Ich habe dich gesetzt zum Vater vieler Heiden vor GOtt, dem du
geglaubet hast, der da lebendig machet die Toten und rufet dem, das
nicht ist, daß es sei. \bibverse{18} Und er hat geglaubet auf Hoffnung,
da nichts zu hoffen war, auf daß er würde ein Vater vieler Heiden, wie
denn zu ihm gesagt ist: Also soll dein Same sein. \bibverse{19} Und er
ward nicht schwach im Glauben, sah auch nicht an seinen eigenen Leib,
welcher schon erstorben war, weil er fast hundertjährig war, auch nicht
den erstorbenen Leib der Sara. \bibverse{20} Denn er zweifelte nicht an
der Verheißung GOttes durch Unglauben, sondern ward stark im Glauben und
gab GOtt die Ehre \bibverse{21} und wußte aufs allergewisseste, daß, was
GOtt verheißet, das kann er auch tun. \bibverse{22} Darum ist's ihm auch
zur Gerechtigkeit gerechnet. \bibverse{23} Das ist aber nicht
geschrieben allein um seinetwillen, daß es ihm zugerechnet ist,
\bibverse{24} sondern auch um unsertwillen, welchen es soll zugerechnet
werden, so wir glauben an den, der unsern HErrn JEsum auferwecket hat
von den Toten. \bibverse{25} welcher ist um unserer Sünden willen
dahingegeben und um unserer Gerechtigkeit willen auferwecket.

\hypertarget{section-4}{%
\section{5}\label{section-4}}

\bibverse{1} Nun wir denn sind gerecht worden durch den Glauben, so
haben wir Frieden mit GOtt durch unsern HErrn JEsum Christum,
\bibverse{2} durch welchen wir auch einen Zugang haben im Glauben zu
dieser Gnade darinnen wir stehen, und rühmen uns der Hoffnung der
zukünftigen Herrlichkeit die GOtt geben soll. \bibverse{3} Nicht allein
aber das, sondern wir rühmen uns auch der Trübsale dieweil wir wissen,
daß Trübsal Geduld bringet. \bibverse{4} Geduld aber bringet Erfahrung,
Erfahrung aber bringet Hoffnung, \bibverse{5} Hoffnung aber läßt nicht
zuschanden werden. Denn die Liebe GOttes ist ausgegossen in unser Herz
durch den Heiligen Geist, welcher uns gegeben ist. \bibverse{6} Denn
auch Christus, da wir noch schwach waren nach der Zeit, ist für uns
Gottlose gestorben. \bibverse{7} Nun stirbt kaum jemand um des Rechtes
willen; um etwas Gutes willen dürfte vielleicht jemand sterben.
\bibverse{8} Darum preiset GOtt seine Liebe gegen uns, daß Christus für
uns gestorben ist, da wir noch Sünder waren. \bibverse{9} So werden wir
je viel mehr durch ihn behalten werden vor dem Zorn, nachdem wir durch
sein Blut gerecht worden sind. \bibverse{10} Denn so wir GOtt versöhnet
sind durch den Tod seines Sohns, da wir noch Feinde waren, viel mehr
werden wir selig werden durch sein Leben, so wir nun versöhnet sind.
\bibverse{11} Nicht allein aber das, sondern wir rühmen uns auch GOttes
durch unsern HErrn JEsum Christum, durch welchen wir nun die Versöhnung
empfangen haben. \bibverse{12} Derhalben, wie durch einen Menschen die
Sünde ist kommen in die Welt und der Tod durch die Sünde, und ist also
der Tod zu allen Menschen durchgedrungen, dieweil sie alle gesündiget
haben; \bibverse{13} (denn die Sünde war wohl in der Welt bis auf das
Gesetz; aber wo kein Gesetz ist, da achtet man der Sünde nicht,
\bibverse{14} sondern der Tod herrschte von Adam an bis auf Mose, auch
über die, die nicht gesündiget haben mit gleicher Übertretung wie Adam,
welcher ist ein Bild des, der zukünftig war. \bibverse{15} Aber nicht
hält sich's mit der Gabe wie mit der Sünde. Denn so an eines Sünde viele
gestorben sind, so ist viel mehr GOttes Gnade und Gabe vielen reichlich
widerfahren durch die Gnade des einigen Menschen, JEsu Christi.
\bibverse{16} Und nicht ist die Gabe allein über eine Sünde wie durch
des einigen Sünders einige Sünde alles Verderben. Denn das Urteil ist
kommen aus einer Sünde zur Verdammnis; die Gabe aber hilft auch aus
vielen Sünden zur Gerechtigkeit. \bibverse{17} Denn so um des einigen
Sünde willen der Tod geherrschet hat durch den einen, viel mehr werden
die, so da empfangen die Fülle der Gnade und der Gabe zur Gerechtigkeit,
herrschen im Leben durch einen, JEsum Christum): \bibverse{18} wie nun
durch eines Sünde die Verdammnis über alle Menschen kommen ist, also ist
auch durch eines Gerechtigkeit die Rechtfertigung des Lebens über alle
Menschen kommen. \bibverse{19} Denn gleichwie durch eines Menschen
Ungehorsam viel Sünder worden sind, also auch durch eines Gehorsam
werden viel Gerechte. \bibverse{20} Das Gesetz aber ist neben einkommen,
auf daß die Sünde mächtiger würde. Wo aber die Sünde mächtig worden ist,
da ist doch die Gnade viel mächtiger worden, \bibverse{21} auf daß,
gleichwie die Sünde geherrschet hat zu dem Tode, also auch herrsche die
Gnade durch die Gerechtigkeit zum ewigen Leben durch JEsum Christum,
unsern HErrn.

\hypertarget{section-5}{%
\section{6}\label{section-5}}

\bibverse{1} Was wollen wir hiezu sagen? Sollen wir denn in der Sünde
beharren, auf daß die Gnade desto mächtiger werde? \bibverse{2} Das sei
ferne! Wie sollten wir in der Sünde wollen leben, der wir abgestorben
sind? \bibverse{3} Wisset ihr nicht, daß alle, die wir in JEsum Christum
getauft sind, die sind in seinen Tod getauft? \bibverse{4} So sind wir
je mit ihm begraben durch die Taufe in den Tod, auf daß, gleichwie
Christus ist auferweckt von den Toten durch die Herrlichkeit des Vaters,
also sollen auch wir in einem neuen Leben wandeln. \bibverse{5} So wir
aber samt ihm gepflanzet werden zu gleichem Tode, so werden wir auch der
Auferstehung gleich sein, \bibverse{6} dieweil wir wissen, daß unser
alter Mensch samt ihm gekreuziget ist, auf daß der sündliche Leib
aufhöre, daß wir hinfort der Sünde nicht dienen. \bibverse{7} Denn wer
gestorben ist, der ist gerechtfertiget von der Sünde. \bibverse{8} Sind
wir aber mit Christo gestorben, so glauben wir, daß wir auch mit ihm
leben werden \bibverse{9} und wissen, daß Christus, von den Toten
erwecket, hinfort nicht stirbt; der Tod wird hinfort über ihn nicht
herrschen. \bibverse{10} Denn das er gestorben ist; das ist er der Sünde
gestorben zu einem Mal; das er aber lebet, das lebet er GOtt.
\bibverse{11} Also auch ihr, haltet euch dafür, daß ihr der Sünde
gestorben seid und lebet GOtt in Christo JEsu, unserm HErrn.
\bibverse{12} So lasset nun die Sünde nicht herrschen in eurem
sterblichen Leibe, ihm Gehorsam zu leisten in seinen Lüsten.
\bibverse{13} Auch begebet nicht der Sünde eure Glieder zu Waffen der
Ungerechtigkeit, sondern begebet euch selbst GOtt, als die da aus den
Toten lebendig sind, und eure Glieder GOtt zu Waffen der Gerechtigkeit,
\bibverse{14} Denn die Sünde wird nicht herrschen können über euch,
sintemal ihr nicht unter dem Gesetze seid, sondern unter der Gnade.
\bibverse{15} Wie nun? sollen wir sündigen, dieweil wir nicht unter dem
Gesetz, sondern unter der Gnade sind? Das sei ferne! \bibverse{16}
Wisset ihr nicht, welchem ihr euch begebet zu Knechten in Gehorsam, des
Knechte seid ihr, dem ihr gehorsam seid, es sei der Sünde zum Tode oder
dem Gehorsam zur Gerechtigkeit? \bibverse{17} GOtt sei aber gedanket,
daß ihr Knechte der Sünde gewesen seid, aber nun gehorsam worden von
Herzen dem Vorbilde der Lehre, welchem ihr ergeben seid. \bibverse{18}
Denn nun ihr frei worden seid von der Sünde, seid ihr Knechte worden der
Gerechtigkeit. \bibverse{19} Ich muß menschlich davon reden um der
Schwachheit willen eures Fleisches. Gleichwie ihr eure Glieder begeben
habt zu Dienste der Unreinigkeit und von einer Ungerechtigkeit zu der
andern, also begebet nun auch eure Glieder zu Dienste der Gerechtigkeit,
daß sie heilig werden. \bibverse{20} Denn da ihr der Sünde Knechte
waret, da waret ihr frei von der Gerechtigkeit. \bibverse{21} Was hattet
ihr nun zu der Zeit für Frucht? Welcher ihr euch jetzt schämet; denn das
Ende derselbigen ist der Tod. \bibverse{22} Nun ihr aber seid von der
Sünde frei und GOttes Knechte worden, habt ihr eure Frucht, daß ihr
heilig werdet, das Ende aber das ewige Leben. \bibverse{23} Denn der Tod
ist der Sünde Sold; aber die Gabe GOttes ist das ewige Leben in Christo
JEsu, unserm HErrn.

\hypertarget{section-6}{%
\section{7}\label{section-6}}

\bibverse{1} Wisset ihr nicht, liebe Brüder (denn ich rede mit denen,
die das Gesetz wissen), daß das Gesetz herrschet über den Menschen,
solange er lebet? \bibverse{2} Denn ein Weib, das unter dem Manne ist,
dieweil der Mann lebet, ist sie gebunden an das Gesetz; so aber der Mann
stirbt, so ist sie los vom Gesetz, das den Mann betrifft. \bibverse{3}
Wo sie nun bei einem andern Manne ist, weil der Mann lebet, wird sie
eine Ehebrecherin geheißen; so aber der Mann stirbt, ist sie frei vom
Gesetz, daß sie nicht eine Ehebrecherin ist, wo sie bei einem andern
Manne ist. \bibverse{4} Also auch, meine Brüder, ihr seid getötet dem
Gesetz durch den Leib Christi, daß ihr bei einem andern seid, nämlich
bei dem, der von den Toten auferwecket ist, auf daß wir GOtt Frucht
bringen. \bibverse{5} Denn da wir im Fleisch waren, da waren die
sündlichen Lüste, welche durchs Gesetz sich erregten, kräftig in unsern
Gliedern, dem Tode Frucht zu bringe. \bibverse{6} Nun aber sind wir vom
Gesetz los und ihm abgestorben, das uns gefangenhielt, also daß wir
dienen sollen im neuen Wesen des Geistes und nicht im alten Wesen des
Buchstabens. \bibverse{7} Was wollen wir denn nun sagen? Ist das Gesetz
Sünde? Das sei ferne! Aber die Sünde erkannte ich nicht ohne durchs
Gesetz. Denn ich wußte nichts von der Lust, wo das Gesetz nicht hätte
gesagt: Laß dich nicht gelüsten! \bibverse{8} Da nahm aber die Sünde
Ursache am Gebot und erregte in mir allerlei Lust. Denn ohne das Gesetz
war die Sünde tot. \bibverse{9} Ich aber lebte etwa ohne Gesetz. Da aber
das Gebot kam, ward die Sünde wieder lebendig. \bibverse{10} Ich aber
starb; und es befand sich, daß das Gebot mir zum Tode gereichte, das mir
doch zum Leben gegeben war. \bibverse{11} Denn die Sünde nahm Ursache am
Gebot und betrog mich und tötete mich durch dasselbige Gebot.
\bibverse{12} Das Gesetz ist je heilig, und das Gebot ist heilig, recht
und gut. \bibverse{13} Ist denn, was da gut ist, mir ein Tod worden? Das
sei ferne! Aber die Sünde, auf daß sie erscheine, wie sie Sünde ist, hat
sie mir durch das Gute den Tod gewirket, auf daß die Sünde würde überaus
sündig durchs Gebot. \bibverse{14} Denn wir wissen, daß das Gesetz
geistlich ist; ich aber bin fleischlich, unter die Sünde verkauft.
\bibverse{15} Denn ich weiß nicht, was, ich tue; denn ich tue nicht, was
ich will, sondern was ich hasse, das tue ich. \bibverse{16} So ich aber
das tue, was ich nicht will, so willige ich, daß das Gesetz gut sei.
\bibverse{17} So tue nun ich dasselbige nicht, sondern die Sünde, die in
mir wohnet. \bibverse{18} Denn ich weiß, daß in mir, das ist, in meinem
Fleische, wohnet nichts Gutes. Wollen habe ich wohl, aber vollbringen
das Gute finde ich nicht, \bibverse{19} Denn das Gute, das ich will, das
tue ich nicht, sondern das Böse, das ich nicht will, das tue ich.
\bibverse{20} So ich aber tue, was ich nicht will, so tue ich dasselbige
nicht, sondern die Sünde, die in mir wohnet. \bibverse{21} So finde ich
mir nun ein Gesetz, der ich will das Gute tun, daß mir das Böse
anhanget. \bibverse{22} Denn ich habe Lust an GOttes Gesetz nach dem
inwendigen Menschen. \bibverse{23} Ich sehe aber ein ander Gesetz in
meinen Gliedern, das da widerstreitet dem Gesetz in meinem Gemüte und
nimmt mich gefangen in der Sünde Gesetz, welches ist in meinen Gliedern.
\bibverse{24} Ich elender Mensch, wer wird mich erlösen von dem Leibe
dieses Todes? \bibverse{25} Ich danke GOtt durch JEsum Christum, unsern
HErrn. So diene ich nun mit dem Gemüte dem Gesetz GOttes, aber mit dem
Fleisch dem Gesetze der Sünde.

\hypertarget{section-7}{%
\section{8}\label{section-7}}

\bibverse{1} So ist nun nichts Verdammliches an denen, die in Christo
JEsu sind, die nicht nach dem Fleisch wandeln, sondern nach dem Geist.
\bibverse{2} Denn das Gesetz des Geistes der da lebendig macht in
Christo JEsu, hat mich freigemacht von dem Gesetz der Sünde und des
Todes. \bibverse{3} Denn was dem Gesetz unmöglich war (sintemal es durch
das Fleisch geschwächet ward), das tat GOtt und sandte seinen Sohn in
der Gestalt des sündlichen Fleisches und verdammte die Sünde im Fleisch
durch Sünde, \bibverse{4} auf daß die Gerechtigkeit, vom Gesetz
erfordert, in uns erfüllet würde, die wir nun nicht nach dem Fleische
wandeln sondern nach dem Geist. \bibverse{5} Denn die da fleischlich
sind, die sind fleischlich gesinnet; die aber geistlich sind, die sind
geistlich gesinnet. \bibverse{6} Aber fleischlich gesinnet sein ist der
Tod, und geistlich gesinnet sein ist Leben und Friede. \bibverse{7} Denn
fleischlich gesinnet sein ist eine Feindschaft wider GOtt, sintemal es
dem Gesetze GOttes nicht untertan ist; denn es vermag es auch nicht.
\bibverse{8} Die aber fleischlich sind, mögen GOtt nicht gefallen.
\bibverse{9} Ihr aber seid nicht fleischlich, sondern geistlich, so
anders GOttes Geist in euch wohnet. Wer aber Christi Geist nicht hat der
ist nicht sein. \bibverse{10} So aber Christus in euch ist so ist der
Leib zwar tot um der Sünde willen; der Geist aber ist das Leben um der
Gerechtigkeit willen. \bibverse{11} So nun der Geist des, der JEsum von
den Töten auferwecket hat, in euch wohnet, so wird auch derselbige, der
Christum von den Toten auferwecket hat, eure sterblichen Leiber lebendig
machen um deswillen, daß sein Geist in euch wohnet. \bibverse{12} So
sind wir nun, liebe Brüder, Schuldner nicht dem Fleisch, daß wir nach
dem Fleisch leben. \bibverse{13} Denn wo ihr nach dem Fleisch lebet, so
werdet ihr sterben müssen; wo ihr aber durch den Geist des Fleisches
Geschäfte tötet, so werdet ihr leben. \bibverse{14} Denn welche der
Geist GOttes treibet, die sind GOttes Kinder. \bibverse{15} Denn ihr
habt nicht einen knechtischen Geist empfangen, daß ihr euch abermal
fürchten müßtet, sondern ihr habt einen kindlichen Geist empfangen,
durch welchen wir rufen: Abba, lieber Vater! \bibverse{16} Derselbige
Geist gibt Zeugnis unserm Geist, daß wir GOttes Kinder sind.
\bibverse{17} Sind wir denn Kinder, so sind wir auch Erben, nämlich
GOttes Erben und Miterben Christi, so wir anders mit leiden, auf daß wir
auch mit zur Herrlichkeit erhoben werden. \bibverse{18} Denn ich halte
es dafür, daß dieser Zeit Leiden der Herrlichkeit nicht wert sei, die an
uns soll offenbaret werden. \bibverse{19} Denn das ängstliche Harren der
Kreatur wartet auf die Offenbarung der Kinder GOttes, \bibverse{20}
sintemal die Kreatur unterworfen ist der Eitelkeit ohne ihren Willen,
sondern um deswillen, der sie unterworfen hat auf Hoffnung.
\bibverse{21} Denn auch die Kreatur frei werden wird von dem Dienst des
vergänglichen Wesens zu der herrlichen Freiheit der Kinder GOttes.
\bibverse{22} Denn wir wissen, daß alle Kreatur sehnet sich mit uns und
ängstet sich noch immerdar. \bibverse{23} Nicht allein aber sie, sondern
auch wir selbst, die wir haben des Geistes Erstlinge, sehnen uns auch
bei uns selbst nach der Kindschaft und warten auf unsers Leibes
Erlösung. \bibverse{24} Denn wir sind wohl selig, doch in der Hoffnung.
Die Hoffnung aber, die man siehet, ist nicht Hoffnung; denn wie kann man
des, hoffen, das man siehet? \bibverse{25} So wir aber des hoffen, das
wir nicht sehen, so warten wir sein durch Geduld. \bibverse{26}
Desselbigengleichen auch der Geist hilft unserer Schwachheit auf. Denn
wir wissen nicht, was wir beten sollen, wie sich's gebühret, sondern der
Geist selbst vertritt uns aufs beste mit unaussprechlichem Seufzen.
\bibverse{27} Der aber die Herzen forschet, der weiß, was des Geistes
Sinn sei; denn er vertritt die Heiligen nach dem, was GOtt gefällt.
\bibverse{28} Wir wissen aber, daß denen, die GOtt lieben, alle Dinge
zum besten dienen, die nach dem Vorsatz berufen sind. \bibverse{29} Denn
welche er zuvor versehen hat, die hat er auch verordnet, daß sie gleich
sein sollten dem Ebenbilde seines Sohns, auf daß derselbige der
Erstgeborne sei unter vielen Brüdern. \bibverse{30} Welche er aber
verordnet hat, die hat er auch berufen; welche er aber berufen hat, die
hat er auch gerecht gemacht; welche er aber hat gerecht gemacht, die hat
er auch herrlich gemacht. \bibverse{31} Was wollen wir denn hiezu sagen?
Ist GOtt für uns, wer mag wider uns sein? \bibverse{32} Welcher auch
seines eigenen Sohnes nicht hat verschont, sondern hat ihn für uns alle
dahingegeben, wie sollte er uns mit ihm nicht alles schenken?
\bibverse{33} Wer will die Auserwählten GOttes beschuldigen? GOtt ist
hier, der da gerecht macht. \bibverse{34} Wer will verdammen? Christus
ist hier, der gestorben ist, ja vielmehr, der auch auferwecket ist,
welcher ist zur Rechten GOttes und vertritt uns. \bibverse{35} Wer will
uns scheiden von der Liebe GOttes? Trübsal oder Angst oder Verfolgung
oder Hunger oder Blöße oder Fährlichkeit oder Schwert? \bibverse{36} Wie
geschrieben stehet: Um deinetwillen werden wir getötet den ganzen Tag;
wir sind geachtet für Schlachtschafe. \bibverse{37} Aber in dem allem
überwinden wir weit um deswillen, der uns geliebet hat. \bibverse{38}
Denn ich bin gewiß, daß weder Tod noch Leben, weder Engel noch
Fürstentum noch Gewalt, weder Gegenwärtiges noch Zukünftiges,
\bibverse{39} weder Hohes noch Tiefes noch keine andere Kreatur mag uns
scheiden von der Liebe GOttes, die in Christo JEsu ist, unserm HErrn.

\hypertarget{section-8}{%
\section{9}\label{section-8}}

\bibverse{1} Ich sage die Wahrheit in Christo und lüge nicht, des mir
Zeugnis gibt mein Gewissen in dem Heiligen Geist, \bibverse{2} daß ich
große Traurigkeit und Schmerzen ohne Unterlaß in meinem Herzen habe.
\bibverse{3} Ich habe gewünschet, verbannet zu sein von Christo für
meine Brüder, die meine Gefreundeten sind nach dem Fleisch, \bibverse{4}
die da sind von Israel, welchen gehört die Kindschaft und die
Herrlichkeit und der Bund und das Gesetz und der Gottesdienst und die
Verheißung; \bibverse{5} welcher auch sind die Väter, aus welchen
Christus herkommt nach dem Fleische, der da ist GOtt über alles, gelobet
in Ewigkeit! Amen. \bibverse{6} Aber nicht sage ich solches, daß GOttes
Wort darum aus sei. Denn es sind nicht alle Israeliten, die von Israel
sind; \bibverse{7} auch nicht alle, die Abrahams Same sind, sind darum
auch Kinder, sondern: In Isaak soll dir der Same genannt sein.
\bibverse{8} Das ist, nicht sind das GOttes Kinder, die nach dem Fleisch
Kinder sind, sondern die Kinder der Verheißung werden für Samen
gerechnet. \bibverse{9} Denn dies ist ein Wort der Verheißung, da er
spricht: Um diese Zeit will ich kommen, und Sara soll einen Sohn haben.
\bibverse{10} Nicht allein aber ist's mit dem also, sondern auch, da
Rebecka von dem einigen Isaak, unserm Vater, schwanger ward;
\bibverse{11} ehe die Kinder geboren waren und weder Gutes noch Böses
getan hatten, auf daß der Vorsatz GOttes bestünde nach der Wahl, ward zu
ihr gesagt, \bibverse{12} nicht aus Verdienst der Werke, sondern aus
Gnaden des Berufes, also: Der Größere soll dienstbar werden dem
Kleinern, \bibverse{13} wie denn geschrieben stehet: Jakob habe ich
geliebet, aber Esau habe ich gehasset. \bibverse{14} Was wollen wir denn
hier sagen? Ist denn GOtt ungerecht? Das sei ferne! \bibverse{15} Denn
er spricht zu Mose: Welchem ich gnädig bin, dem bin ich gnädig, und
welches ich mich erbarme, des erbarme ich mich. \bibverse{16} So liegt
es nun nicht an jemandes Wollen oder Laufen, sondern an GOttes Erbarmen.
\bibverse{17} Denn die Schrift sagt zu Pharao: Eben darum hab' ich dich
erweckt, daß ich an dir meine Macht erzeige, auf daß mein Name
verkündiget werde in allen Landen. \bibverse{18} So erbarmet er sich
nun; welches er will, und verstocket, welchen er will. \bibverse{19} So
sagest du zu mir: Was schuldiget er denn uns? Wer kann seinem Willen
widerstehen? \bibverse{20} Ja, lieber Mensch, wer bist du denn, daß du
mit GOtt rechten willst? Spricht auch ein Werk zu seinem Meister: Warum
machst du mich also? \bibverse{21} Hat nicht ein Töpfer Macht, aus einem
Klumpen zu machen ein Faß zu Ehren und das andere zu Unehren?
\bibverse{22} Derhalben, da GOtt wollte Zorn erzeigen und kundtun seine
Macht, hat er mit großer Geduld getragen die Gefäße des Zorns, die da
zugerichtet sind zur Verdammnis, \bibverse{23} auf daß er kundtäte den
Reichtum seiner Herrlichkeit an den Gefäßen der Barmherzigkeit, die er
bereitet hat zur Herrlichkeit, \bibverse{24} welche er berufen hat,
nämlich uns, nicht allein aus den Juden, sondern auch aus den Heiden.
\bibverse{25} Wie er denn auch durch Hosea spricht: Ich will das mein
Volk heißen, das nicht mein Volk war, und meine Liebe, die nicht die
Liebe war. \bibverse{26} Und soll geschehen, an dem Ort, da zu ihnen
gesagt ward: Ihr seid nicht mein Volk, sollen sie Kinder des lebendigen
GOttes genannt werden. \bibverse{27} Jesaja aber schreiet für Israel:
Wenn die Zahl der Kinder von Israel würde sein wie der Sand am Meer, so
wird doch das Übrige selig werden; \bibverse{28} Denn es wird ein
Verderben und Steuern geschehen zur Gerechtigkeit, und der HErr wird
dasselbige Steuern tun auf Erden. \bibverse{29} Und wie Jesaja davor
sagt: Wenn uns nicht der HErr Zebaoth hätte lassen Samen überbleiben, so
wären wir wie Sodom worden und gleichwie Gomorra. \bibverse{30} Was
wollen wir nun hier sagen? Das wollen wir sagen: Die Heiden, die nicht
haben nach der Gerechtigkeit gestanden, haben die Gerechtigkeit
erlanget; ich sage aber von der Gerechtigkeit, die aus dem Glauben
kommt. \bibverse{31} Israel aber hat dem Gesetz der Gerechtigkeit
nachgestanden und hat das Gesetz der Gerechtigkeit nicht überkommen.
\bibverse{32} Warum das? Darum, daß sie es nicht aus dem Glauben,
sondern als aus den Werken des Gesetzes suchen. Denn sie haben sich
gestoßen an den Stein des Anlaufens, \bibverse{33} wie geschrieben
stehet: Siehe da, ich lege in Zion einen Stein des Anlaufens und einen
Fels des Ärgernisses; und wer an ihn glaubet, der soll nicht zuschanden
werden.

\hypertarget{section-9}{%
\section{10}\label{section-9}}

\bibverse{1} Liebe Brüder, meines Herzens Wunsch ist, und flehe auch zu
GOtt für Israel, daß sie selig werden. \bibverse{2} Denn ich gebe ihnen
das Zeugnis, daß sie eifern um GOtt, aber mit Unverstand. \bibverse{3}
Denn sie erkennen die Gerechtigkeit nicht, die vor GOtt gilt, und
trachten, ihre eigene Gerechtigkeit aufzurichten, und sind also der
Gerechtigkeit, die vor GOtt gilt, nicht untertan. \bibverse{4} Denn
Christus ist des Gesetzes Ende; wer an den glaubet, der ist gerecht.
\bibverse{5} Mose schreibt wohl von der Gerechtigkeit, die aus dem
Gesetz kommt: Welcher Mensch dies tut, der wird darinnen leben.
\bibverse{6} Aber die Gerechtigkeit aus dem Glauben spricht also: Sprich
nicht in deinem Herzen: Wer will hinauf gen Himmel fahren? (Das ist
nichts anderes, denn Christum herabholen.) \bibverse{7} Oder: Wer will
hinab in die Tiefe fahren? (Das ist nichts anderes, denn Christum von
den Toten holen.) \bibverse{8} Aber was sagt sie? Das Wort ist dir nahe,
nämlich in deinem Munde und in deinem Herzen. Dies ist das Wort vom
Glauben, das wir predigen. \bibverse{9} Denn so du mit deinem Munde
bekennest JEsum, daß er der HErr sei, und glaubest in deinem Herzen, daß
ihn GOtt von den Toten auferweckt hat, so wirst du selig. \bibverse{10}
Denn so man von Herzen glaubet, so wird man gerecht, und so man mit dem
Munde bekennet, so wird man selig. \bibverse{11} Denn die Schrift
spricht: Wer an ihn glaubet, wird nicht zuschanden werden. \bibverse{12}
Es ist hier kein Unterschied unter Juden und Griechen; es ist aller
zumal ein HErr, reich über alle, die ihn anrufen. \bibverse{13} Denn wer
den Namen des HErrn wird anrufen, soll selig werden. \bibverse{14} Wie
sollen sie aber anrufen, an den sie nicht glauben? Wie sollen sie aber
glauben, von dem sie nichts gehöret haben? Wie sollen sie aber hören
ohne Prediger? \bibverse{15} Wie sollen sie aber predigen, wo sie nicht
gesandt werden? wie denn geschrieben stehet: Wie lieblich sind die Füße
derer, die den Frieden verkündigen, die das Gute verkündigen!
\bibverse{16} Aber sie sind nicht alle dem Evangelium gehorsam. Denn
Jesaja spricht: HErr, wer glaubet unserm Predigen? \bibverse{17} So
kommt der Glaube aus der Predigt, das Predigen aber durch das Wort
GOttes. \bibverse{18} Ich sage aber: Haben sie es nicht gehörte? Zwar es
ist je in alle Lande ausgegangen ihr Schall und in alle Welt ihre Worte.
\bibverse{19} Ich sage aber: Hat es Israel nicht erkannt? Der erste Mose
spricht: Ich will euch eifern machen über dem, das nicht mein Volk ist,
und über einem unverständigen Volk will ich euch erzürnen. \bibverse{20}
Jesaja aber darf wohl so sagen: Ich bin erfunden von denen, die mich
nicht gesucht haben, und bin erschienen denen die nicht nach mir gefragt
haben. \bibverse{21} Zu Israel aber spricht er: Den ganzen Tag habe ich
meine Hände ausgestrecket zu dem Volk, das sich nicht sagen lässet und
widerspricht.

\hypertarget{section-10}{%
\section{11}\label{section-10}}

\bibverse{1} So sage ich nun: Hat denn GOtt sein Volk verstoßen? Das sei
ferne! Denn ich bin auch ein Israelit von dem Samen Abrahams, aus dem
Geschlecht Benjamin. \bibverse{2} GOtt hat sein Volk nicht verstoßen,
welches er zuvor versehen hat. Oder wisset ihr nicht, was die Schrift
sagt von Elia, wie er tritt vor GOtt wider Israel und spricht:
\bibverse{3} HErr, sie haben deine Propheten getötet und haben deine
Altäre ausgegraben; und ich bin allein überblieben, und sie stehen mir
nach meinem Leben? \bibverse{4} Aber was sagt ihm die göttliche Antwort?
Ich habe mir lassen überbleiben siebentausend Mann, die nicht haben ihre
Kniee gebeuget vor dem Baal. \bibverse{5} Also gehet's auch jetzt zu
dieser Zeit mit diesen Überbliebenen nach der Wahl der Gnaden.
\bibverse{6} Ist's aber aus Gnaden, so ist's nicht aus Verdienst der
Werke, sonst würde Gnade nicht Gnade sein. Ist's aber aus Verdienst der
Werke, so ist die Gnade nichts, sonst wäre Verdienst nicht Verdienst.
\bibverse{7} Wie denn nun? Was Israel sucht, das erlangt es nicht; die
Wahl aber erlanget es. Die andern sind verstockt, \bibverse{8} wie
geschrieben stehet: GOtt hat ihnen gegeben einen erbitterten Geist,
Augen, daß sie nicht sehen, und Ohren, daß sie nicht hören, bis auf den
heutigen Tag. \bibverse{9} Und David spricht: Laß ihren Tisch zu einem
Strick werden und zu einer Berückung und zum Ärgernis und ihnen zur
Vergeltung. \bibverse{10} Verblende ihre Augen, daß sie nicht sehen, und
beuge ihren Rücken allezeit. \bibverse{11} So sage ich nun: Sind sie
darum angelaufen, daß sie fallen sollten? Das sei ferne! Sondern aus
ihrem Fall ist den Heiden das Heil widerfahren, auf daß sie denen
nacheifern sollten. \bibverse{12} Denn so ihr Fall der Welt Reichtum
ist, und ihr Schade ist der Heiden Reichtum, wieviel mehr, wenn ihre
Zahl voll würde? \bibverse{13} Mit euch Heiden rede ich; denn dieweil
ich der Heiden Apostel bin, will ich mein Amt preisen, \bibverse{14} ob
ich möchte die, so mein Fleisch sind, zu eifern reizen und ihrer etliche
selig machen. \bibverse{15} Denn so ihr Verlust der Welt Versöhnung ist,
was wäre das anders, denn das Leben von den Toten nehmen? \bibverse{16}
Ist der Anbruch heilig, so ist auch der Teig heilig, und so die Wurzel
heilig ist, so sind auch die Zweige heilig. \bibverse{17} Ob aber nun
etliche von den Zweigen zerbrochen sind, und du, da du ein wilder Ölbaum
warest, bist unter sie gepfropfet und teilhaftig worden der Wurzel und
des Safts im Ölbaum, \bibverse{18} so rühme dich nicht wider die Zweige.
Rühmest du dich aber wider sie, so sollst du wissen, daß du die Wurzel
nicht trägest, sondern die Wurzel träget dich. \bibverse{19} So sprichst
du: Die Zweige sind zerbrochen, daß ich hineingepfropfet würde.
\bibverse{20} Ist wohl geredet. Sie sind zerbrochen um ihres Unglaubens
willen; du stehest aber durch den Glauben. Sei nicht stolz, sondern
fürchte dich. \bibverse{21} Hat GOtt der natürlichen Zweige nicht
verschonet, daß er vielleicht dein auch nicht verschone. \bibverse{22}
Darum schaue die Güte und den Ernst GOttes: den Ernst an denen, die
gefallen sind, die Güte aber an dir, soferne du an der Güte bleibest;
sonst wirst du auch abgehauen werden. \bibverse{23} Und jene, so sie
nicht bleiben in dem Unglauben, werden sie eingepfropfet werden; GOtt
kann sie wohl wieder ein pfropfen. \bibverse{24} Denn so du aus dem
Ölbaum, der von Natur wild war, bist ausgehauen und wider die Natur in
den guten Ölbaum gepfropfet, wieviel mehr werden die natürlichen
eingepfropfet in ihren eigenen Ölbaum! \bibverse{25} Ich will euch nicht
verhalten, liebe Brüder, dieses Geheimnis, auf daß ihr nicht stolz seid.
Blindheit ist Israel einesteils widerfahren, so lange, bis die Fülle der
Heiden eingegangen sei, \bibverse{26} und also das ganze Israel selig
werde, wie geschrieben stehet: Es wird kommen aus Zion, der da erlöse
und abwende das gottlose Wesen von Jakob. \bibverse{27} Und dies ist
mein Testament mit ihnen, wenn ich ihre Sünden werde weg nehmen.
\bibverse{28} Nach dem Evangelium halte ich sie für Feinde um
euretwillen; aber nach der Wahl habe ich sie lieb um der Väter willen.
\bibverse{29} GOttes Gaben und Berufung mögen ihn nicht gereuen.
\bibverse{30} Denn gleicherweise, wie auch ihr nicht habt geglaubet an
GOtt, nun aber habt ihr Barmherzigkeit überkommen über ihrem Unglauben,
\bibverse{31} also auch jene haben jetzt nicht wollen glauben an die
Barmherzigkeit, die euch widerfahren ist, auf daß sie auch
Barmherzigkeit überkommen. \bibverse{32} Denn GOtt hat alles beschlossen
unter den Unglauben, auf daß er sich aller erbarme. \bibverse{33} O
welch eine Tiefe des Reichtums, beide, der Weisheit und Erkenntnis
GOttes! Wie gar unbegreiflich sind seine Gerichte und unerforschlich
seine Wege! \bibverse{34} Denn wer hat des HErrn Sinn erkannt? Oder wer
ist sein Ratgeber gewesen? \bibverse{35} Oder wer hat ihm etwas zuvor
gegeben, das ihm werde wieder vergolten? \bibverse{36} Denn von ihm und
durch ihn und zu ihm sind alle Dinge. Ihm sei Ehre in Ewigkeit! Amen.

\hypertarget{section-11}{%
\section{12}\label{section-11}}

\bibverse{1} Ich ermahne euch, liebe Brüder, durch die Barmherzigkeit
GOttes, daß ihr eure Leiber begebet zum Opfer, das da lebendig, heilig
und GOtt wohlgefällig sei, welches sei euer vernünftiger Gottesdienst.
\bibverse{2} Und stellet euch nicht dieser Welt gleich, sondern
verändert euch durch Erneuerung eures Sinnes, auf daß ihr prüfen möget,
welches da sei der gute, der wohlgefällige und der vollkommene
Gotteswille. \bibverse{3} Denn ich sage durch die Gnade, die mir gegeben
ist, jedermann unter euch, daß niemand weiter von sich halte, denn
sich's gebührt zu halten, sondern daß von sich mäßiglich halte, ein
jeglicher nachdem GOtt ausgeteilet hat das Maß des Glaubens.
\bibverse{4} Denn gleicherweise, als wir in einem Leibe viel Glieder
haben, aber alle Glieder nicht einerlei Geschäft haben, \bibverse{5}
also sind wir viele ein Leib in Christo; aber untereinander ist einer
des andern Glied. \bibverse{6} Und haben mancherlei Gaben nach der
Gnade, die uns gegeben ist.Hat jemand Weissagung, so sei sie dem Glauben
ähnlich. \bibverse{7} Hat jemand ein Amt so warte er des Amts. Lehret
jemand, so warte er der Lehre. \bibverse{8} Ermahnet jemand, so warte er
des Ermahnens. Gibt jemand, so gebe er einfältiglich. Regieret jemand,
so sei er sorgfältig. Übet jemand Barmherzigkeit, so tu er's mit Lust.
\bibverse{9} Die Liebe sei nicht falsch. Hasset das Arge, hanget dem
Guten an. \bibverse{10} Die brüderliche Liebe untereinander sei
herzlich. Einer komme dem andern mit Ehrerbietung zuvor. \bibverse{11}
Seid nicht träge, was ihr tun sollt. Seid brünstig im Geiste. Schicket
euch in die Zeit. \bibverse{12} Seid fröhlich in Hoffnung; geduldig in
Trübsal, haltet an am Gebet. \bibverse{13} Nehmet euch der Heiligen
Notdurft an. Herberget gerne. \bibverse{14} Segnet, die euch verfolgen;
segnet, und fluchet nicht. \bibverse{15} Freuet euch mit den Fröhlichen
und weinet mit den Weinenden. \bibverse{16} Habt einerlei Sinn
untereinander. Trachtet nicht nach hohen Dingen, sondern haltet euch
herunter zu den Niedrigen. \bibverse{17} Haltet euch nicht selbst für
klug. Vergeltet niemand Böses mit Bösem. Fleißiget euch der Ehrbarkeit
gegen jedermann. \bibverse{18} Ist es möglich, soviel an euch ist, so
habt mit allen Menschen Frieden. \bibverse{19} Rächet euch selber nicht,
meine Liebsten, sondern gebet Raum dem Zorn; denn es stehet geschrieben:
Die Rache ist mein; ich will vergelten, spricht der HErr. \bibverse{20}
So nun deinen Feind hungert, so speise ihn; dürstet ihn, so tränke ihn.
Wenn du das tust, so wirst du feurige Kohlen auf sein Haupt sammeln.
\bibverse{21} Laß dich nicht das Böse überwinden, sondern überwinde das
Böse mit Gutem.

\hypertarget{section-12}{%
\section{13}\label{section-12}}

\bibverse{1} Jedermann sei untertan der Obrigkeit, die Gewalt über ihn
hat. Denn es ist keine Obrigkeit ohne von GOtt; wo aber Obrigkeit ist,
die ist von GOtt verordnet. \bibverse{2} Wer sich nun wider die
Obrigkeit setzet, der widerstrebet GOttes Ordnung; die aber
widerstreben, werden über sich ein Urteil empfangen. \bibverse{3} Denn
die Gewaltigen sind nicht den guten Werken, sondern den bösen zu
fürchten. Willst du dich aber nicht fürchten vor der Obrigkeit, so tue
Gutes, so wirst du Lob von derselbigen haben; \bibverse{4} denn sie ist
GOttes Dienerin dir zu gut. Tust du aber Böses, so fürchte dich; denn
sie trägt das Schwert nicht umsonst; sie ist GOttes Dienerin, eine
Rächerin zur Strafe über den, der Böses tut. \bibverse{5} So seid nun
aus Not untertan, nicht allein um der Strafe willen, sondern auch um des
Gewissens willen. \bibverse{6} Derhalben müsset ihr auch Schoß geben;
denn sie sind GOttes Diener, die solchen Schutz sollen handhaben.
\bibverse{7} So gebet nun jedermann, was ihr schuldig seid: Schoße dem
der Schoß gebührt; Zoll, dem der Zoll gebührt; Furcht dem die Furcht
gebührt; Ehre, dem die Ehre gebührt. \bibverse{8} Seid niemand nichts
schuldig, denn daß ihr euch untereinander liebet; denn wer den andern
liebet, der hat das Gesetz erfüllet. \bibverse{9} Denn das da gesagt
ist: Du sollst nicht ehebrechen; du sollst nicht töten; du sollst nicht
stehlen; du sollst nicht falsch Zeugnis geben; dich soll nichts
gelüsten, und so ein ander Gebot mehr ist, das wird in diesem Wort
verfasset: Du sollst deinen Nächsten lieben wie dich selbst:
\bibverse{10} Die Liebe tut dem Nächsten nichts Böses. So ist nun die
Liebe des Gesetzes Erfüllung. \bibverse{11} Und weil wir solches wissen,
nämlich die Zeit, daß die Stunde da ist, aufzustehen vom Schlaf,
sintemal unser Heil jetzt näher ist, denn da wir's glaubten,
\bibverse{12} die Nacht ist vergangen, der Tag aber herbeikommen: so
lasset uns ablegen die Werke der Finsternis und anlegen die Waffen des
Lichtes. \bibverse{13} Lasset uns ehrbarlich wandeln, als am Tage, nicht
in Fressen und Saufen, nicht in Kammern und Unzucht, nicht in Hader und
Neid. \bibverse{14} sondern ziehet an den HErrn JEsum Christum und
wartet des Leibes, doch also, daß er nicht geil werde.

\hypertarget{section-13}{%
\section{14}\label{section-13}}

\bibverse{1} Den Schwachen im Glauben nehmet auf und verwirret die
Gewissen nicht. \bibverse{2} Einer glaubt, er möge allerlei essen;
welcher aber schwach ist, der isset Kraut. \bibverse{3} Welcher isset,
der verachte den nicht, der da nicht isset; und welcher nicht isset, der
richte den nicht, der da isset; denn GOtt hat ihn aufgenommen.
\bibverse{4} Wer bist du, daß du einen fremden Knecht richtest? Er
stehet oder fället seinem Herrn. Er mag aber wohl aufgerichtet werden;
denn GOtt kann ihn Wohl aufrichten. \bibverse{5} Einer hält einen Tag
vor dem andern; der andere aber hält alle Tage gleich. Ein jeglicher sei
seiner Meinung gewiß. \bibverse{6} Welcher auf die Tage hält, der tut's
dem HErrn; und welcher nichts darauf hält, der tut's auch dem HErrn.
Welcher isset, der isset dem HErrn; denn er danket GOtt. Welcher nicht
isset, der isset dem HErrn nicht und danket GOtt. \bibverse{7} Denn
unser keiner lebt sich selber, und keiner stirbt sich selber.
\bibverse{8} Leben wir, so leben wir dem HErrn; sterben wir, so sterben
wir dem HErrn. Darum, wir leben oder sterben, so sind wir des HErrn.
\bibverse{9} Denn dazu ist Christus auch gestorben und auferstanden und
wieder lebendig worden, daß er über Tote und Lebendige HErr sei.
\bibverse{10} Du aber, was richtest du deinen Bruder? Oder du anderer,
was verachtest du deinen Bruder? Wir werden alle vor dem Richterstuhl
Christi dargestellt werden, \bibverse{11} nachdem geschrieben stehet: So
wahr als ich lebe, spricht der HErr, mir sollen alle Kniee gebeuget
werden, und alle Zungen sollen GOtt bekennen. \bibverse{12} So wird nun
ein jeglicher für sich selbst GOtt Rechenschaft geben. \bibverse{13}
Darum lasset uns nicht mehr einer den andern richten, sondern das
richtet vielmehr, daß niemand seinem Bruder einen Anstoß oder Ärgernis
darstelle. \bibverse{14} Ich weiß und bin's gewiß in dem HErrn JEsu, daß
nichts gemein ist an sich selbst; ohne der es rechnet für gemein,
demselbigen ist's gemein. \bibverse{15} So aber dein Bruder über deine
Speise betrübet wird, so wandelst du schon nicht nach der Liebe. Lieber,
verderbe den nicht mit deiner Speise, um welchen willen Christus
gestorben ist! \bibverse{16} Darum schaffet, daß euer Schatz nicht
verlästert werde! \bibverse{17} Denn das Reich GOttes ist nicht Essen
und Trinken, sondern Gerechtigkeit und Friede und Freude in dem Heiligen
Geiste. \bibverse{18} Wer darinnen Christo dienet, der ist GOtt gefällig
und den Menschen wert. \bibverse{19} Darum lasset uns dem nachstreben,
was zum Frieden dienet, und was zur Besserung untereinander dienet.
\bibverse{20} Lieber, verstöre nicht um der Speise willen GOttes Werk!
Es ist zwar alles rein, aber es ist nicht gut dem, der es isset mit
einem Anstoß seines Gewissens. \bibverse{21} Es ist besser, du essest
kein Fleisch und trinkest keinen Wein oder das, daran sich dein Bruder
stößet oder ärgert oder schwach wird. \bibverse{22} Hast du den Glauben,
so habe ihn bei dir selbst vor GOtt. Selig ist, der sich selbst kein
Gewissen macht in dem, was er annimmt. \bibverse{23} Wer aber darüber
zweifelt und isset doch, der ist verdammt; denn es gehet nicht aus dem
Glauben. Was aber nicht aus dem Glauben gehet, das ist Sünde.

\hypertarget{section-14}{%
\section{15}\label{section-14}}

\bibverse{1} Wir aber, die wir stark sind, sollen der Schwachen
Gebrechlichkeit tragen und nicht Gefallen an uns selber haben.
\bibverse{2} Es stelle sich aber ein jeglicher unter uns also, daß er
seinem Nächsten gefalle zum Guten, zur Besserung. \bibverse{3} Denn auch
Christus nicht an sich selber Gefallen hatte, sondern wie geschrieben
stehet: Die Schmähungen derer, die dich schmähen, sind über mich
gefallen. \bibverse{4} Was aber zuvor geschrieben ist, das ist uns zur
Lehre geschrieben, auf daß wir durch Geduld und Trost der Schrift
Hoffnung haben. \bibverse{5} GOtt aber der Geduld und des Trostes gebe
euch, daß ihr einerlei gesinnet seid untereinander nach JEsu Christo,
\bibverse{6} auf daß ihr einmütiglich mit einem Munde lobet GOtt und den
Vater unsers HErrn JEsu Christi. \bibverse{7} Darum nehmet euch
untereinander auf, gleichwie euch Christus hat aufgenommen zu GOttes
Lobe. \bibverse{8} Ich sage aber, daß JEsus Christus sei ein Diener
gewesen der Beschneidung um der Wahrheit willen GOttes, zu bestätigen
die Verheißung, den Vätern geschehen, \bibverse{9} daß die Heiden aber
GOtt loben um der Barmherzigkeit willen, wie geschrieben stehet: Darum
will ich dich loben unter den Heiden und deinem Namen singen.
\bibverse{10} Und abermal spricht er: Freuet euch, ihr Heiden, mit
seinem Volk! \bibverse{11} Und abermal: Lobet den HErrn, alle Heiden,
und preiset ihn, alle Völker! \bibverse{12} Und abermal spricht Jesaja:
Es wird sein die Wurzel Jesse, und der auferstehen wird, zu herrschen
über die Heiden; auf den werden die Heiden hoffen. \bibverse{13} GOtt
aber der Hoffnung erfülle euch mit aller Freude und Frieden im Glauben,
daß ihr völlige Hoffnung habet durch die Kraft des Heiligen Geistes.
\bibverse{14} Ich weiß aber fast wohl von euch, liebe Brüder, daß ihr
selbst voll Gütigkeit seid, erfüllet mit aller Erkenntnis, daß ihr euch
untereinander könnet ermahnen. \bibverse{15} Ich hab's aber dennoch
gewagt und euch etwas wollen schreiben, liebe Brüder, euch zu erinnern,
um der Gnade willen, die mir von GOtt gegeben ist, \bibverse{16} daß ich
soll sein ein Diener Christi unter den Heiden zu opfern das Evangelium
GOttes, auf daß die Heiden ein Opfer werden, GOtt angenehm, geheiliget
durch den Heiligen Geist. \bibverse{17} Darum kann ich mich rühmen in
JEsu Christo, daß ich GOtt diene. \bibverse{18} Denn ich dürfte nicht
etwas reden, wo dasselbige Christus nicht durch mich wirkte, die Heiden
zum Gehorsam zu bringen durch Wort und Werk, \bibverse{19} durch Kraft
der Zeichen und Wunder und durch Kraft des Geistes GOttes, also daß ich
von Jerusalem an und umher bis an Illyrikum alles mit dem Evangelium
Christi erfüllet habe, \bibverse{20} und mich sonderlich geflissen, das
Evangelium zu predigen, wo Christi Name nicht bekannt war, auf daß ich
nicht auf einen fremden Grund bauete \bibverse{21} sondern wie
geschrieben stehet: Welchen nicht ist von ihm verkündiget, die sollen's
sehen, und welche nicht gehöret haben, sollen's verstehen. \bibverse{22}
Das ist auch die Sache, darum ich vielmal verhindert bin, zu euch zu
kommen. \bibverse{23} Nun ich aber nicht mehr Raum habe in diesen
Ländern, habe aber Verlangen, zu euch zu kommen, von vielen Jahren her:
\bibverse{24} wenn ich reisen werde nach Spanien, will ich zu euch
kommen. Denn ich hoffe, daß ich da durchreisen und euch sehen werde und
von euch dorthin geleitet werden möge, so doch, daß ich zuvor mich ein
wenig mit euch ergötze. \bibverse{25} Nun aber fahre ich hin gen
Jerusalem den Heiligen zu Dienst. \bibverse{26} Denn die aus Mazedonien
und Achaja haben williglich eine gemeine Steuer zusammengelegt den armen
Heiligen zu Jerusalem. \bibverse{27} Sie haben's williglich getan und
sind auch ihre Schuldner. Denn so die Heiden sind ihrer geistlichen
Güter teilhaftig worden, ist's billig, daß sie ihnen auch in leiblichen
Gütern Dienst beweisen. \bibverse{28} Wenn ich nun solches ausgerichtet
und ihnen diese Frucht versiegelt habe, will ich durch euch nach Spanien
ziehen. \bibverse{29} Ich weiß aber, wenn ich zu euch komme, daß ich mit
vollem Segen des Evangeliums Christi kommen werde. \bibverse{30} Ich
ermahne euch aber, liebe Brüder, durch unsern HErrn JEsum Christum und
durch die Liebe des Geistes, daß ihr mir helfet kämpfen mit Beten für
mich zu GOtt, \bibverse{31} auf daß ich errettet werde von den
Ungläubigen in Judäa, und daß mein Dienst, den ich gen Jerusalem tue,
angenehm werde den Heiligen, \bibverse{32} auf daß ich mit Freuden zu
euch komme durch den Willen GOttes und mich mit euch erquicke.
\bibverse{33} Der GOtt aber des Friedens sei mit euch allen! Amen.

\hypertarget{section-15}{%
\section{16}\label{section-15}}

\bibverse{1} Ich befehle euch aber unsere Schwester Phöbe, welche ist am
Dienste der Gemeinde zu Kenchreä, \bibverse{2} daß ihr sie aufnehmet in
dem HErrn, wie sich's ziemet den Heiligen, und tut ihr Beistand in allem
Geschäfte, darinnen sie euer bedarf. Denn sie hat auch vielen Beistand
getan, auch mir selbst. \bibverse{3} Grüßet die Priscilla und den
Aquila, meine Gehilfen in Christo JEsu, \bibverse{4} welche haben für
mein Leben ihre Hälse dargegeben, welchen nicht allein ich danke,
sondern alle Gemeinden unter den Heiden. \bibverse{5} Auch grüßet die
Gemeinde in ihrem Hause. Grüßet Epänetum, meinen Liebsten, welcher ist
der Erstling unter denen aus Achaja in Christo. \bibverse{6} Grüßet
Maria, welche viel Mühe und Arbeit mit uns gehabt hat. \bibverse{7}
Grüßet den Andronikus und den Junias, meine Gefreundeten und meine
Mitgefangenen, welche sind berühmte Apostel und vor mir gewesen in
Christo. \bibverse{8} Grüßet Amplias, meinen Lieben in dem HErrn.
\bibverse{9} Grüßet Urban, unsern Gehilfen in Christo und Stachys,
meinen Lieben. \bibverse{10} Grüßet Apelles, den Bewährten in Christo.
Grüßet, die da sind von des Aristobulus Gesinde. \bibverse{11} Grüßet
Herodionus, meinen Gefreundeten. Grüßet, die da sind von des Narcissus
Gesinde in dem HErrn. \bibverse{12} Grüßet die Tryphäna und die
Tryphosa, welche in dem HErrn gearbeitet haben. Grüßet die Persida,
meine Liebe, welche in dem HErrn viel gearbeitet hat. \bibverse{13}
Grüßet Rufus, den Auserwählten in dem HErrn, und seine und meine Mutter.
\bibverse{14} Grüßet Asynkritus und Phlegon, Hermas, Patrobas, Hermes
und die Brüder bei ihnen. \bibverse{15} Grüßet Philologus und die Julia,
Nereus und seine Schwester und Olympas und alle Heiligen bei ihnen.
\bibverse{16} Grüßet euch untereinander mit dem heiligen Kuß. Es grüßen
euch die Gemeinden Christi. \bibverse{17} Ich ermahne aber euch, liebe
Brüder, daß ihr aufsehet auf die, die da Zertrennung und Ärgernis
anrichten neben der Lehre, die ihr gelernet habt, und weichet von
denselbigen! \bibverse{18} Denn solche dienen nicht dem HErr JEsu
Christo, sondern ihrem Bauche; und durch süße Worte und prächtige Rede
verführen sie die unschuldigen Herzen. \bibverse{19} Denn euer Gehorsam
ist unter jedermann auskommen. Derhalben freue ich mich über euch. Ich
will aber, daß ihr weise seid aufs Gute, aber einfältig aufs Böse.
\bibverse{20} Aber der GOtt des Friedens zertrete den Satan unter eure
Füße in kurzem! Die Gnade unsers HErrn JEsu Christi sei mit euch!
\bibverse{21} Es grüßen euch Timotheus, mein Gehilfe, und Lucius und
Jason und Sosipater, meine Gefreundeten. \bibverse{22} Ich, Tertius,
grüße euch, der ich diesen Brief geschrieben habe, in dem HErrn.
\bibverse{23} Es grüßet euch Gajus, mein und der ganzen Gemeinde Wirt.
Es grüßet euch Erastus, der Stadt Rentmeister, und Quartus, der Bruder.
\bibverse{24} Die Gnade unsers HErrn Jesu Christi sei mit euch allen!
Amen. \bibverse{25} Dem aber, der euch stärken kann laut meines
Evangeliums und Predigt von JEsu Christo, durch welche das Geheimnis
offenbaret ist, das von der Welt her verschwiegen gewesen ist,
\bibverse{26} nun aber offenbaret, auch kundgemacht durch der Propheten
Schriften aus Befehl des ewigen GOttes, den Gehorsam des Glaubens
aufzurichten unter allen Heiden: \bibverse{27} demselbigen GOtt, der
allein weise ist, sei Ehre durch Jesum Christum in Ewigkeit! Amen.
