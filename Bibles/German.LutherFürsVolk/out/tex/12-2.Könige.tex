\hypertarget{section}{%
\section{1}\label{section}}

\bibverse{1} Auch fielen die Moabiter ab von Israel, da Ahab tot war.
\bibverse{2} Und Ahasja fiel durchs Gitter in seinem Saal zu Samaria und
ward krank; und sandte Boten und sprach zu ihnen: Gehet hin und fraget
Baal-Sebub, den Gott zu Ekron, ob ich von dieser Krankheit genesen
werde. \bibverse{3} Aber der Engel des HErrn redete mit Elia, dem
Thisbiten: Auf! und begegne den Boten des Königs zu Samaria und sprich
zu ihnen: Ist denn nun kein GOtt in Israel, daß ihr hingehet zu fragen
Baal-Sebub, den Gott zu Ekron? \bibverse{4} Darum so spricht der HErr:
Du sollst nicht von dem Bette kommen, darauf du dich geleget hast,
sondern sollst des Todes sterben. Und Elia ging weg. \bibverse{5} Und da
die Boten wieder zu ihm kamen, sprach er zu ihnen: Warum kommt ihr
wieder? \bibverse{6} Sie sprachen zu ihm: Es kam uns ein Mann herauf
entgegen und sprach zu uns: Gehet wiederum hin zu dem Könige, der euch
gesandt hat, und sprechet zu ihm: So spricht der HErr: Ist denn kein
GOtt in Israel, daß du hinsendest, zu fragen Baal- Sebub, den Gott zu
Ekron? Darum sollst du nicht kommen von dem Bette, darauf du dich
geleget hast, sondern sollst des Todes sterben. \bibverse{7} Er sprach
zu ihnen: Wie war der Mann gestaltet, der euch begegnete und solches zu
euch sagte? \bibverse{8} Sie sprachen zu ihm: Er hatte eine rauche Haut
an und einen ledernen Gürtel um seine Lenden. Er aber sprach: Es ist
Elia, der Thisbiter. \bibverse{9} Und er sandte hin zu ihm einen
Hauptmann über fünfzig samt denselben Fünfzigen. Und da der zu ihm
hinaufkam, siehe, da saß er oben auf dem Berge. Er aber sprach zu ihm:
Du Mann GOttes, der König sagt: Du sollst herabkommen. \bibverse{10}
Elia antwortete dem Hauptmann über fünfzig und sprach zu ihm: Bin ich
ein Mann GOttes, so falle Feuer vom Himmel und fresse dich und deine
Fünfzig. Da fiel Feuer vom Himmel und fraß ihn und seine Fünfzig.
\bibverse{11} Und er sandte wiederum einen andern Hauptmann über fünfzig
zu ihm samt seinen Fünfzigen. Der antwortete und sprach zu ihm: Du Mann
GOttes, so spricht der König: Komm eilends herab! \bibverse{12} Elia
antwortete und sprach: Bin ich ein Mann GOttes, so falle Feuer vom
Himmel und fresse dich und deine Fünfzig. Da fiel das Feuer GOttes vom
Himmel und fraß ihn und seine Fünfzig. \bibverse{13} Da sandte er
wiederum den dritten Hauptmann über fünfzig samt seinen Fünfzigen. Da
der zu ihm hinaufkam, beugte er seine Kniee gegen Elia und flehete ihm
und sprach zu ihm: Du Mann GOttes, laß meine Seele und die Seele deiner
Knechte, dieser Fünfzig, vor dir etwas gelten! \bibverse{14} Siehe, das
Feuer ist vom Himmel gefallen und hat die ersten zween Hauptmänner über
fünfzig mit ihren Fünfzigen gefressen; nun aber laß meine Seele etwas
gelten vor dir! \bibverse{15} Da sprach der Engel des HErrn zu Elia:
Gehe mit ihm hinab und fürchte dich nicht vor ihm! Und er machte sich
auf und ging mit ihm hinab zum Könige. \bibverse{16} Und er sprach zu
ihm: So spricht der HErr: Darum, daß du hast Boten hingesandt und lassen
fragen Baal-Sebub, den Gott zu Ekron, als wäre kein GOtt in Israel, des
Wort man fragen möchte, so sollst du von dem Bette nicht kommen, darauf
du dich geleget hast, sondern sollst des Todes sterben. \bibverse{17}
Also starb er nach dem Wort des HErrn, das Elia geredet hatte. Und Joram
ward König an seiner Statt im andern Jahr Jorams, des Sohns Josaphats,
des Königs Judas; denn er hatte keinen Sohn. \bibverse{18} Was aber mehr
von Ahasja zu sagen ist, das er getan hat, siehe, das ist geschrieben in
der Chronik der Könige Israels.

\hypertarget{section-1}{%
\section{2}\label{section-1}}

\bibverse{1} Da aber der HErr wollte Elia im Wetter gen Himmel holen,
ging Elia und Elisa von Gilgal. \bibverse{2} Und Elia sprach zu Elisa:
Lieber, bleib hie; denn der HErr hat mich gen Bethel gesandt. Elisa aber
sprach: So wahr der HErr lebet und deine Seele, ich verlasse dich nicht.
Und da sie hinab gen Bethel kamen, \bibverse{3} gingen der Propheten
Kinder, die zu Bethel waren, heraus zu Elisa und sprachen zu ihm:
Weißest du auch, daß der HErr wird deinen HErrn heute von deinen Häupten
nehmen? Er aber sprach: Ich weiß es auch wohl; schweiget nur stille!
\bibverse{4} Und Elia sprach zu ihm: Elisa, Lieber, bleib hie; denn der
HErr hat mich gen Jericho gesandt. Er aber sprach: So wahr der HErr
lebet und deine Seele, ich verlasse dich nicht. Und da sie gen Jericho
kamen, \bibverse{5} traten der Propheten Kinder, die zu Jericho waren,
zu Elisa und sprachen zu ihm: Weißest du auch, daß der HErr wird deinen
Herrn heute von deinen Häupten nehmen? Er aber sprach: Ich weiß es auch
wohl; schweiget nur stille! \bibverse{6} Und Elia sprach zu ihm: Lieber,
bleib hie; denn der HErr hat mich gesandt an den Jordan. Er aber sprach:
So wahr der HErr lebet und deine Seele, ich verlasse dich nicht. Und
gingen die beiden miteinander. \bibverse{7} Aber fünfzig Männer unter
der Propheten Kindern gingen hin und traten gegenüber von ferne; aber
die beiden stunden am Jordan. \bibverse{8} Da nahm Elia seinen Mantel
und wickelte ihn zusammen und schlug ins Wasser; das teilete sich auf
beiden Seiten, daß die beiden trocken durchhin gingen. \bibverse{9} Und
da sie hinüberkamen, sprach Elia zu Elisa: Bitte, was ich dir tun soll,
ehe ich von dir genommen werde. Elisa sprach: Daß dein Geist bei mir sei
zwiefältig. \bibverse{10} Er sprach: Du hast ein Hartes gebeten; doch so
du mich sehen wirst, wenn ich von dir genommen werde, so wird's ja sein;
wo nicht, so wird's nicht sein. \bibverse{11} Und da sie miteinander
gingen, und er redete, siehe, da kam ein feuriger Wagen mit feurigen
Rossen, und schieden die beiden voneinander; und Elia fuhr also im
Wetter gen Himmel. \bibverse{12} Elisa aber sah es und schrie: Mein
Vater, mein Vater, Wagen Israels und seine Reiter! Und sah ihn nicht
mehr. Und er fassete seine Kleider und zerriß sie in zwei Stücke.
\bibverse{13} Und hub auf den Mantel Elias, der ihm entfallen war; und
kehrete um und trat an das Ufer des Jordans. \bibverse{14} Und nahm
denselben Mantel Elias, der ihm entfallen war, und schlug ins Wasser und
sprach: Wo ist nun der HErr, der GOtt Elias? Und schlug ins Wasser; da
teilete sich's auf beiden Seiten; und Elisa ging hindurch. \bibverse{15}
Und da ihn sahen der Propheten Kinder, die zu Jericho gegen ihm waren,
sprachen sie: Der Geist Elias ruhet auf Elisa; und gingen ihm entgegen
und beteten an zur Erde. \bibverse{16} Und sprachen zu ihm: Siehe, es
sind unter deinen Knechten fünfzig Männer, starke Leute, die laß gehen
und deinen Herrn suchen; vielleicht hat ihn der Geist des HErrn genommen
und irgend auf einen Berg, oder irgend in ein Tal geworfen. Er aber
sprach: Lasset nicht gehen! \bibverse{17} Aber sie nötigten ihn, bis daß
er sich ungebärdig stellete und sprach: Lasset hingehen! Und sie sandten
hin fünfzig Männer und suchten ihn drei Tage; aber sie fanden ihn nicht
\bibverse{18} und kamen wieder zu ihm. Und er blieb zu Jericho und
sprach zu ihnen: Sagte ich euch nicht, ihr solltet nicht hingehen?
\bibverse{19} Und die Männer der Stadt sprachen zu Elisa: Siehe, es ist
gut wohnen in dieser Stadt, wie mein Herr siehet; aber es ist böses
Wasser und das Land unfruchtbar. \bibverse{20} Er sprach: Bringet mir
her eine neue Schale und tut Salz drein. Und sie brachten es ihm.
\bibverse{21} Da ging er hinaus zu der Wasserquelle und warf das Salz
drein und sprach: So spricht der HErr: Ich habe dies Wasser gesund
gemacht; es soll hinfort kein Tod noch Unfruchtbarkeit daher kommen.
\bibverse{22} Also ward das Wasser gesund bis auf diesen Tag nach dem
Wort Elisas, das er redete. \bibverse{23} Und er ging hinauf gen Bethel.
Und als er auf dem Wege hinanging, kamen kleine Knaben zur Stadt heraus
und spotteten ihn und sprachen zu ihm: Kahlkopf, komm herauf! Kahlkopf,
komm herauf! \bibverse{24} Und er wandte sich um, und da er sie sah,
fluchte er ihnen im Namen des HErrn. Da kamen zween Bären aus dem Walde
und zerrissen der Kinder zweiundvierzig. \bibverse{25} Von dannen ging
er auf den Berg Karmel und kehrete um von dannen gen Samaria.

\hypertarget{section-2}{%
\section{3}\label{section-2}}

\bibverse{1} Joram, der Sohn Ahabs, ward König über Israel zu Samaria im
achtzehnten Jahr Josaphats, des Königs Judas; und regierte zwölf Jahre.
\bibverse{2} Und tat, das dem HErrn übel gefiel, doch nicht wie sein
Vater und seine Mutter. Denn er tat weg die Säule Baals, die sein Vater
machen ließ. \bibverse{3} Aber er blieb hangen an den Sünden Jerobeams,
des Sohns Nebats, der Israel sündigen machte, und ließ nicht davon.
\bibverse{4} Mesa aber, der Moabiter König, hatte viel Schafe und
zinsete dem Könige Israels Wolle von hunderttausend Lämmern und von
hunderttausend Widdern. \bibverse{5} Da aber Ahab tot war, fiel der
Moabiter König ab vom Könige Israels. \bibverse{6} Da zog zu derselben
Zeit aus der König Joram von Samaria und ordnete das ganze Israel.
\bibverse{7} Und sandte hin zu Josaphat, dem Könige Judas, und ließ ihm
sagen: Der Moabiter König ist von mir abgefallen; komm mit mir, zu
streiten wider die Moabiter! Er sprach: Ich will hinaufkommen; ich bin
wie du und mein Volk wie dein Volk und meine Rosse wie deine Rosse.
\bibverse{8} Und sprach: Durch welchen Weg wollen wir hinaufziehen? Er
sprach: Durch den Weg in der Wüste Edom. \bibverse{9} Also zog hin der
König Israels, der König Judas und der König Edoms. Und da sie sieben
Tagreisen zogen, hatte das Heer und das Vieh, das unter ihnen war, kein
Wasser. \bibverse{10} Da sprach der König Israels: O wehe! Der HErr hat
diese drei Könige geladen, daß er sie in der Moabiter Hände gäbe.
\bibverse{11} Josaphat aber sprach: Ist kein Prophet des HErrn hie, daß
wir den HErrn durch ihn ratfragten? Da antwortete einer unter den
Knechten des Königs Israels und sprach: Hie ist Elisa, der Sohn Saphats,
der Elia Wasser auf die Hände goß. \bibverse{12} Josaphat sprach: Des
HErrn Wort ist bei ihm. Also zogen zu ihm hinab der König Israels und
Josaphat und der König Edoms. \bibverse{13} Elisa aber sprach zum Könige
Israels: Was hast du mit mir zu schaffen? Gehe hin zu den Propheten
deines Vaters und zu den Propheten deiner Mutter! Der König Israels
sprach zu ihm: Nein; denn der HErr hat diese drei Könige geladen, daß er
sie in der Moabiter Hände gäbe. \bibverse{14} Elisa sprach: So wahr der
HErr Zebaoth lebet, vor dem ich stehe, wenn ich nicht Josaphat, den
König Judas, ansähe, ich wollte dich nicht ansehen noch achten.
\bibverse{15} So bringet mir nun einen Spielmann. Und da der Spielmann
auf der Saite spielte, kam die Hand des HErrn auf ihn. \bibverse{16} Und
er sprach: So spricht der HErr: Machet hie und da Graben an diesem Bach!
\bibverse{17} Denn so spricht der HErr: Ihr werdet keinen Wind noch
Regen sehen; dennoch soll der Bach voll Wassers werden, daß ihr und euer
Gesinde und euer Vieh trinket. \bibverse{18} Dazu ist das ein Geringes
vor dem HErrn, er wird auch die Moabiter in eure Hände geben,
\bibverse{19} daß ihr schlagen werdet alle festen Städte und alle
auserwählten Städte; und werdet fällen alle guten Bäume und werdet
verstopfen alle Wasserbrunnen und werdet allen guten Acker mit Steinen
verderben. \bibverse{20} Des Morgens aber, wenn man Speisopfer opfert,
siehe, da kam ein Gewässer des Weges von Edom und füllete das Land mit
Wasser. \bibverse{21} Da aber alle Moabiter höreten, daß die Könige
heraufzogen, wider sie zu streiten, beriefen sie alle, die zur Rüstung
alt genug und drüber waren, und traten an die Grenze. \bibverse{22} Und
da sie sich des Morgens frühe aufmachten, und die Sonne aufging auf das
Gewässer, deuchte die Moabiter das Gewässer gegen ihnen rot sein wie
Blut; \bibverse{23} und sprachen: Es ist Blut; die Könige haben sich mit
dem Schwert verderbet, und einer wird den andern geschlagen haben. Hui,
Moab, mache dich nun zur Ausbeute! \bibverse{24} Aber da sie zum Lager
Israels kamen, machte sich Israel auf und schlugen die Moabiter; und sie
flohen vor ihnen. Aber sie kamen hinein und schlugen Moab. \bibverse{25}
Die Städte zerbrachen sie, und ein jeglicher warf seine Steine auf alle
guten Äcker und machten sie voll; und verstopften alle Wasserbrunnen und
fälleten alle guten Bäume, bis daß nur die Steine an den Ziegelmauern
überblieben; und sie umgaben sie mit Schleudern und schlugen sie.
\bibverse{26} Da aber der Moabiter König sah, daß ihm der Streit zu
stark war, nahm er siebenhundert Mann zu sich, die das Schwert auszogen,
herauszureißen wider den König Edoms; aber sie konnten nicht.
\bibverse{27} Da nahm er seinen ersten Sohn, der an seiner Statt sollte
König werden, und opferte ihn zum Brandopfer auf der Mauer. Da ward
Israel sehr zornig, daß sie von ihm abzogen, und kehreten wieder zu
Lande.

\hypertarget{section-3}{%
\section{4}\label{section-3}}

\bibverse{1} Und es schrie ein Weib unter den Weibern der Kinder der
Propheten zu Elisa und sprach: Dein Knecht, mein Mann, ist gestorben; so
weißt du, daß er, dein Knecht, den HErrn fürchtete; nun kommt der
Schuldherr und will meine beiden Kinder nehmen zu eigenen Knechten.
\bibverse{2} Elisa sprach zu ihr: Was soll ich dir tun? Sage mir, was
hast du im Hause? Sie sprach: Deine Magd hat nichts im Hause denn einen
Ölkrug. \bibverse{3} Er sprach: Gehe hin und bitte draußen von allen
deinen Nachbarinnen leere Gefäße, und derselben nicht wenig.
\bibverse{4} Und gehe hinein und schleuß die Tür hinter dir zu mit
deinen Söhnen und geuß in alle Gefäße; und wenn du sie gefüllet hast, so
gib sie hin. \bibverse{5} Sie ging hin und schloß die Tür hinter ihr zu
samt ihren Söhnen; die brachten ihr die Gefäße zu, so goß sie ein.
\bibverse{6} Und da die Gefäße voll waren, sprach sie zu ihrem Sohn:
Lange mir noch ein Gefäß her! Er sprach zu ihr: Es ist kein Gefäß mehr
hie. Da stund das Öl. \bibverse{7} Und sie ging hin und sagte es dem
Mann GOttes an. Er sprach: Gehe hin, verkaufe das Öl und bezahle deinen
Schuldherrn; du aber und deine Söhne nähret euch von dem übrigen.
\bibverse{8} Und es begab sich zu der Zeit, daß Elisa ging gen Sunem.
Daselbst war eine reiche Frau; die hielt ihn, daß er bei ihr aß. Und als
er nun oft daselbst durchzog, ging er zu ihr ein und aß bei ihr.
\bibverse{9} Und sie sprach zu ihrem Manne: Siehe, ich merke, daß dieser
Mann GOttes heilig ist, der immerdar hie durchgehet. \bibverse{10} Laß
uns ihm eine kleine bretterne Kammer oben machen und ein Bett, Tisch,
Stuhl und Leuchter hineinsetzen, auf daß, wenn er zu uns kommt, dahin
sich tue. \bibverse{11} Und es begab sich zu der Zeit, daß er hineinkam
und legte sich oben in die Kammer und schlief drinnen. \bibverse{12} Und
sprach zu seinem Knaben Gehasi: Rufe der Sunamitin! Und da er ihr rief,
trat sie vor ihn. \bibverse{13} Er sprach zu ihm: Sage ihr: Siehe, du
hast uns all diesen Dienst getan; was soll ich dir tun? Hast du eine
Sache an den König oder an den Feldhauptmann? Sie sprach: Ich wohne
unter meinem Volk. \bibverse{14} Er sprach: Was ist ihr denn zu tun?
Gehasi sprach: Ach, sie hat keinen Sohn, und ihr Mann ist alt.
\bibverse{15} Er sprach: Rufe ihr! Und da er ihr rief, trat sie in die
Tür. \bibverse{16} Und er sprach: Um diese Zeit über ein Jahr sollst du
einen Sohn herzen. Sie sprach: Ach nicht, mein Herr, du Mann GOttes,
lüge deiner Magd nicht! \bibverse{17} Und die Frau ward schwanger und
gebar einen Sohn um dieselbe Zeit über ein Jahr, wie ihr Elisa geredet
hatte. \bibverse{18} Da aber das Kind groß ward, begab sich's, daß es
hinaus zu seinem Vater zu den Schnittern ging. \bibverse{19} Und sprach
zu seinem Vater: O mein Haupt, mein Haupt! Er sprach zu seinem Knaben:
Bringe ihn zu seiner Mutter! \bibverse{20} Und er nahm ihn und brachte
ihn hinein zu seiner Mutter; und sie setzte ihn auf ihren Schoß bis an
den Mittag; da starb er. \bibverse{21} Und sie ging hinauf und legte ihn
aufs Bett des Mannes GOttes, schloß zu und ging hinaus. \bibverse{22}
Und rief ihrem Mann und sprach: Sende mir der Knaben einen und eine
Eselin; ich will zu dem Mann GOttes und wiederkommen. \bibverse{23} Er
sprach: Warum willst du zu ihm? Ist doch heute nicht Neumond noch
Sabbat. Sie sprach: Es ist gut. \bibverse{24} Und sie sattelte die
Eselin und sprach zum Knaben: Treibe fort und säume mich nicht mit dem
Reiten, wie ich dir sage. \bibverse{25} Also zog sie hin und kam zu dem
Mann GOttes auf den Berg Karmel. Als aber der Mann GOttes sie gegen ihm
sah, sprach er zu seinem Knaben Gehasi: Siehe, die Sunamitin ist da.
\bibverse{26} So lauf ihr nun entgegen und frage sie, ob's ihr und ihrem
Mann und Sohn wohlgehe? Sie sprach: Wohl. \bibverse{27} Da sie aber zu
dem Mann GOttes auf den Berg kam, hielt sie ihn bei seinen Füßen; Gehasi
aber trat herzu, daß er sie abstieße. Aber der Mann GOttes sprach: Laß
sie, denn ihre Seele ist betrübt; und der HErr hat mir's verborgen und
nicht angezeiget. \bibverse{28} Sie sprach: Wann habe ich einen Sohn
gebeten von meinem Herrn? Sagte ich nicht, du solltest mich nicht
täuschen? \bibverse{29} Er sprach zu Gehasi: Gürte deine Lenden und nimm
meinen Stab in deine Hand und gehe hin (so dir jemand begegnet, so grüße
ihn nicht, und grüßet dich jemand, so danke ihm nicht) und lege meinen
Stab auf des Knaben Antlitz. \bibverse{30} Die Mutter aber des Knaben
sprach: So wahr der HErr lebet und deine Seele, ich lasse nicht von dir.
Da machte er sich auf und ging ihr nach. \bibverse{31} Gehasi aber ging
vor ihnen hin und legte den Stab dem Knaben aufs Antlitz; da war aber
keine Stimme noch Fühlen. Und er ging wiederum ihm entgegen und zeigte
ihm an und sprach: Der Knabe ist nicht aufgewacht. \bibverse{32} Und da
Elisa ins Haus kam, siehe, da lag der Knabe tot auf seinem Bette.
\bibverse{33} Und er ging hinein und schloß die Tür zu für sie beide und
betete zu dem HErrn. \bibverse{34} Und stieg hinauf und legte sich auf
das Kind und legte seinen Mund auf des Kindes Mund und seine Augen auf
seine Augen und seine Hände auf seine Hände; und breitete sich also über
ihn, daß des Kindes Leib warm ward. \bibverse{35} Er aber stund wieder
auf und ging im Hause einmal hieher und daher; und stieg hinauf und
breitete sich über ihn. Da schnaubte der Knabe siebenmal; danach tat der
Knabe seine Augen auf. \bibverse{36} Und er rief Gehasi und sprach: Rufe
der Sunamitin. Und da er ihr rief, kam sie hinein zu ihm. Er sprach: Da
nimm hin deinen Sohn! \bibverse{37} Da kam sie und fiel zu seinen Füßen
und betete an zur Erde; und nahm ihren Sohn und ging hinaus.
\bibverse{38} Da aber Elisa wieder gen Gilgal kam, ward Hungersnot im
Lande, und die Kinder der Propheten wohneten vor ihm. Und er sprach zu
seinem Knaben: Setze zu einen großen Topf und koche ein Gemüse für die
Kinder der Propheten. \bibverse{39} Da ging einer aufs Feld, daß er
Kraut läse, und fand wilde Ranken, und las davon Koloquinten, sein Kleid
voll; und da er kam, schnitt er's in den Topf zum Gemüse, denn sie
kannten es nicht. \bibverse{40} Und da sie es ausschütteten für die
Männer zu essen, und sie von dem Gemüse aßen, schrieen sie und sprachen:
O Mann GOttes, der Tod im Topf! Denn sie konnten's nicht essen.
\bibverse{41} Er aber sprach: Bringet Mehl her! Und er tat's in den Topf
und sprach: Schütte es dem Volk vor, daß sie essen. Da war nichts Böses
in dem Topf. \bibverse{42} Es kam aber ein Mann von Baal-Salisa und
brachte dem Mann GOttes Erstlingsbrot, nämlich zwanzig Brote, und neu
Getreide in seinem Kleid. Er aber sprach: Gib's dem Volk, daß sie essen!
\bibverse{43} Sein Diener sprach: Was soll ich hundert Mann an dem
geben? Er sprach: Gib dem Volk, daß sie essen! Denn so spricht der HErr:
Man wird essen, und wird überbleiben. \bibverse{44} Und er legte es
ihnen vor, daß sie aßen; und blieb noch über nach dem Wort des HErrn.

\hypertarget{section-4}{%
\section{5}\label{section-4}}

\bibverse{1} Naeman, der Feldhauptmann des Königs zu Syrien, war ein
trefflicher Mann vor seinem Herrn und hoch gehalten; denn durch ihn gab
der HErr Heil in Syrien. Und er war ein gewaltiger Mann, und aussätzig.
\bibverse{2} Die Kriegsleute aber in Syrien waren herausgefallen und
hatten eine kleine Dirne weggeführet aus dem Lande Israel; die war am
Dienst des Weibes Naemans. \bibverse{3} Die sprach zu ihrer Frau: Ach,
daß mein Herr wäre bei dem Propheten zu Samaria, der würde ihn von
seinem Aussatz losmachen. \bibverse{4} Da ging er hinein zu seinem Herrn
und sagte es ihm und sprach: So und so hat die Dirne aus dem Lande
Israel geredet. \bibverse{5} Der König zu Syrien sprach: So zeuch hin,
ich will dem König Israels einen Brief schreiben. Und er zog hin und
nahm mit sich zehn Zentner Silbers und sechstausend Gülden und zehn
Feierkleider. \bibverse{6} Und brachte den Brief dem Könige Israels, der
lautete also: Wenn dieser Brief zu dir kommt, siehe, so wisse, ich habe
meinen Knecht Naeman zu dir gesandt, daß du ihn von seinem Aussatz
losmachest. \bibverse{7} Und da der König Israels den Brief las, zerriß
er seine Kleider und sprach: Bin ich denn GOtt, daß ich töten und
lebendig machen könnte, daß er zu mir schicket, daß ich den Mann von
seinem Aussatz losmache? Merket und sehet, wie suchet er Ursache zu mir!
\bibverse{8} Da das Elisa, der Mann GOttes, hörete, daß der König
Israels seine Kleider zerrissen hatte, sandte er zu ihm und ließ ihm
sagen: Warum hast du deine Kleider zerrissen? Laß ihn zu mir kommen, daß
er inne werde, daß ein Prophet in Israel ist. \bibverse{9} Also kam
Naeman mit Rossen und Wagen und hielt vor der Tür am Hause Elisas.
\bibverse{10} Da sandte Elisa einen Boten zu ihm und ließ ihm sagen:
Gehe hin und wasche dich siebenmal im Jordan, so wird dir dein Fleisch
wiedererstattet und rein werden. \bibverse{11} Da erzürnete Naeman und
zog weg und sprach: Ich meinte, er sollte zu mir herauskommen und
hertreten und den Namen des HErrn, seines GOttes, anrufen und mit seiner
Hand über die Stätte fahren und den Aussatz also abtun. \bibverse{12}
Sind nicht die Wasser Amanas und Pharphars zu Damaskus besser denn alle
Wasser in Israel, daß ich mich drinnen wüsche und rein würde? Und wandte
sich und zog weg mit Zorn. \bibverse{13} Da machten sich seine Knechte
zu ihm, redeten mit ihm und sprachen: Lieber Vater, wenn dich der
Prophet etwas Großes hätte geheißen, solltest du es nicht tun? Wie viel
mehr, so er zu dir sagt: Wasche dich, so wirst du rein. \bibverse{14} Da
stieg er ab und taufte sich im Jordan siebenmal, wie der Mann GOttes
geredet hatte; und sein Fleisch ward wiedererstattet, wie ein Fleisch
eines jungen Knaben, und ward rein. \bibverse{15} Und er kehrete wieder
zu dem Mann GOttes samt seinem ganzen Heer. Und da er hineinkam, trat er
vor ihn und sprach: Siehe, ich weiß, daß kein GOtt ist in allen Landen
ohne in Israel; so nimm nun den Segen von deinem Knechte. \bibverse{16}
Er aber sprach: So war der HErr lebet, vor dem ich stehe, ich nehme es
nicht. Und er nötigte ihn, daß er's nähme; aber er wollte nicht.
\bibverse{17} Da sprach Naeman: Möchte denn deinem Knechte nicht gegeben
werden dieser Erde eine Last, so viel zwei Mäuler tragen? Denn dein
Knecht will nicht mehr andern Göttern opfern und Brandopfer tun, sondern
dem HErrn; \bibverse{18} daß der HErr deinem Knechte darinnen wolle
gnädig sein, wo ich anbete im Hause Rimons, wenn mein Herr ins Haus
Rimons gehet, daselbst anzubeten, und er sich an meine Hand lehnet.
\bibverse{19} Er sprach zu ihm: Zeuch hin mit Frieden! Und als er von
ihm weggezogen war, ein Feld Weges auf dem Lande, \bibverse{20} gedachte
Gehasi, der Knabe Elisas, des Mannes GOttes: Siehe, mein Herr hat diesen
Syrer Naeman verschonet, daß er nichts von ihm hat genommen, das er
gebracht hat. So wahr der HErr lebet, ich will ihm nachlaufen und etwas
von ihm nehmen. \bibverse{21} Also jagte Gehasi dem Naeman nach. Und da
Naeman sah, daß er ihm nachlief, stieg er vom Wagen ihm entgegen und
sprach: Gehet es recht zu? \bibverse{22} Er sprach: Ja. Aber mein Herr
hat mich gesandt und läßt dir sagen: Siehe, jetzt sind zu mir kommen vom
Gebirge Ephraim zween Knaben aus der Propheten Kindern; gib ihnen einen
Zentner Silbers und zwei Feierkleider. \bibverse{23} Naeman sprach:
Lieber, nimm zween Zentner: Und er nötigte ihn und band zween Zentner
Silbers in zween Beutel und zwei Feierkleider und gab's seinen zweien
Knaben, die trugen es vor ihm her. \bibverse{24} Und da er kam gen
Ophel, nahm er's von ihren Händen und legte es beiseit im Hause und ließ
die Männer gehen. \bibverse{25} Und da sie weg waren, trat er vor seinen
Herrn. Und Elisa sprach zu ihm: Woher, Gehasi? Er sprach: Dein Knecht
ist weder hieher noch daher gegangen. \bibverse{26} Er aber sprach zu
ihm: Wandelte nicht mein Herz, da der Mann umkehrete von seinem Wagen
dir entgegen? War das die Zeit, Silber und Kleider zu nehmen, Ölgärten,
Weinberge, Schafe, Rinder, Knechte und Mägde? \bibverse{27} Aber der
Aussatz Naemans wird dir anhangen und deinem Samen ewiglich. Da ging er
von ihm hinaus, aussätzig wie Schnee.

\hypertarget{section-5}{%
\section{6}\label{section-5}}

\bibverse{1} Die Kinder der Propheten sprachen zu Elisa: Siehe, der
Raum, da wir vor dir wohnen, ist uns zu enge. \bibverse{2} Laß uns an
den Jordan gehen und einen jeglichen daselbst Holz holen, daß wir uns
daselbst eine Stätte bauen, da wir wohnen. Er sprach: Gehet hin!
\bibverse{3} Und einer sprach: Lieber, gehe mit deinen Knechten! Er
sprach: Ich will mitgehen. \bibverse{4} Und er ging mit ihnen. Und da
sie an den Jordan kamen, hieben sie Holz ab. \bibverse{5} Und da einer
ein Holz fällete, fiel das Eisen ins Wasser. Und er schrie und sprach:
Awe, mein Herr! Dazu ist's entlehnet. \bibverse{6} Aber der Mann GOttes
sprach: Wo ist's entfallen? Und da er ihm den Ort zeigte, schnitt er ein
Holz ab und stieß daselbst hin. Da schwamm das Eisen. \bibverse{7} Und
er sprach: Hebe es auf! Da reckte er seine Hand aus und nahm's.
\bibverse{8} Und der König aus Syrien führete einen Krieg wider Israel
und beratschlagte sich mit seinen Knechten und sprach: Wir wollen uns
lagern da und da. \bibverse{9} Aber der Mann GOttes sandte zum Könige
Israels und ließ ihm sagen: Hüte dich, daß du nicht an den Ort ziehest;
denn die Syrer ruhen daselbst. \bibverse{10} So sandte denn der König
Israels hin an den Ort, den ihm der Mann GOttes sagte, verwahrete ihn
und hütete daselbst; und tat das nicht einmal oder zweimal allein.
\bibverse{11} Da ward das Herz des Königs zu Syrien Unmuts darüber und
rief seinen Knechten und sprach zu ihnen: Wollt ihr mir denn nicht
ansagen, wer ist aus den Unsern zu dem Könige Israels geflohen?
\bibverse{12} Da sprach seiner Knechte einer: Nicht also, mein Herr
König; sondern Elisa, der Prophet in Israel, sagt es alles dem Könige
Israels, was du in der Kammer redest, da dein Lager ist. \bibverse{13}
Er sprach: So gehet hin und sehet, wo er ist, daß ich hinsende und lasse
ihn holen. Und sie zeigten ihm an und sprachen: Siehe, er ist zu Dothan.
\bibverse{14} Da sandte er hin Rosse und Wagen und eine große Macht. Und
da sie bei der Nacht hinkamen, umgaben sie die Stadt. \bibverse{15} Und
der Diener des Mannes GOttes stund frühe auf, daß er sich aufmachte und
auszöge; und siehe, da lag eine Macht um die Stadt mit Rossen und Wagen.
Da sprach sein Knabe zu ihm: Awe, mein Herr! Wie wollen wir nun tun?
\bibverse{16} Er sprach: Fürchte dich nicht; denn derer ist mehr, die
bei uns sind, denn derer, die bei ihnen sind. \bibverse{17} Und Elisa
betete und sprach: HErr, öffne ihm die Augen, daß er sehe! Da öffnete
der HErr dem Knaben seine Augen, daß er sah; und siehe, da war der Berg
voll feuriger Rosse und Wagen um Elisa her. \bibverse{18} Und da sie zu
ihm hinabkamen, bat Elisa und sprach: HErr, schlage dies Volk mit
Blindheit! Und er schlug sie mit Blindheit nach dem Wort Elisas.
\bibverse{19} Und Elisa sprach zu ihnen: Dies ist nicht der Weg noch die
Stadt. Folget mir nach; ich will euch führen zu dem Mann, den ihr
suchet. Und führete sie gen Samaria. \bibverse{20} Und da sie gen
Samaria kamen, sprach Elisa: HErr, öffne diesen die Augen, daß sie
sehen! Und der HErr öffnete ihnen die Augen, daß sie sahen; und siehe,
da waren sie mitten in Samaria. \bibverse{21} Und der König Israels, da
er sie sah, sprach er zu Elisa: Mein Vater, soll ich sie schlagen?
\bibverse{22} Er sprach: Du sollst sie nicht schlagen. Welche du mit
deinem Schwert und Bogen fähest, die schlage. Setze ihnen Brot und
Wasser vor, daß sie essen und trinken; und laß sie zu ihrem Herrn
ziehen. \bibverse{23} Da ward ein groß Mahl zugerichtet. Und da sie
gegessen und getrunken hatten, ließ er sie gehen, daß sie zu ihrem Herrn
zogen. Seitdem kamen die Kriegsleute der Syrer nicht mehr ins Land
Israel. \bibverse{24} Nach diesem begab sich's, daß Benhadad, der König
zu Syrien, all sein Heer versammelte und zog herauf und belagerte
Samaria. \bibverse{25} Und es war eine große Hungersnot zu Samaria. Sie
aber belagerten die Stadt, bis daß ein Eselskopf achtzig Silberlinge und
ein Vierteil Kad Taubenmist fünf Silberlinge galt. \bibverse{26} Und da
der König Israels zur Mauer ging, schrie ihn ein Weib an und sprach:
Hilf mir, mein Herr König! \bibverse{27} Er sprach: Hilft dir der HErr
nicht, woher soll ich dir helfen? Von der Tenne oder von der Kelter?
\bibverse{28} Und der König sprach zu ihr: Was ist dir? Sie sprach: Dies
Weib sprach zu mir: Gib deinen Sohn her, daß wir heute essen; morgen
wollen wir meinen Sohn essen. \bibverse{29} So haben wir meinen Sohn
gekocht und gegessen. Und ich sprach zu ihr am andern Tage: Gib deinen
Sohn her und laß uns essen! Aber sie hat ihren Sohn versteckt.
\bibverse{30} Da der König die Worte des Weibes hörete, zerriß er seine
Kleider, indem er zur Mauer ging. Da sah alles Volk, daß er einen Sack
unten am Leibe anhatte. \bibverse{31} Und er sprach: GOtt tue mir dies
und das, wo das Haupt Elisas, des Sohns Saphats, heute auf ihm stehen
wird! \bibverse{32} (Elisa aber saß in seinem Hause, und die Ältesten
saßen bei ihm.) Und er sandte einen Mann vor ihm her. Aber ehe der Bote
zu ihm kam, sprach er zu den Ältesten: Habt ihr gesehen, wie dies
Mordkind hat hergesandt, daß er mein Haupt abreiße? Sehet zu, wenn der
Bote kommt, daß ihr die Tür zuschließet und stoßet ihn mit der Tür weg;
siehe, das Rauschen seines Herrn Füße folget ihm nach. \bibverse{33} Da
er noch also mit ihnen redete, siehe, da kam der Bote zu ihm herab und
sprach: Siehe, solch Übel kommt von dem HErrn; was soll ich mehr von dem
HErrn gewarten?

\hypertarget{section-6}{%
\section{7}\label{section-6}}

\bibverse{1} Elisa aber sprach: Höret des HErrn Wort! So spricht der
HErr: Morgen um diese Zeit wird ein Scheffel Semmelmehl einen Sekel
gelten und zween Scheffel Gerste einen Sekel unter dem Tor zu Samaria.
\bibverse{2} Da antwortete der Ritter, auf welches Hand sich der König
lehnte, dem Mann GOttes und sprach: Und wenn der HErr Fenster am Himmel
machte, wie könnte solches geschehen? Er sprach: Siehe da, mit deinen
Augen wirst du es sehen und nicht davon essen. \bibverse{3} Und es waren
vier aussätzige Männer an der Tür vor dem Tor; und einer sprach zum
andern: Was wollen wir hie bleiben, bis wir sterben? \bibverse{4} Wenn
wir gleich gedächten, in die Stadt zu kommen, so ist Hungersnot in der
Stadt, und müßten doch daselbst sterben; bleiben wir aber hie, so müssen
wir auch sterben. So laßt uns nun hingehen und zu dem Heer der Syrer
fallen. Lassen sie uns leben, so leben wir; töten sie uns, so sind wir
tot. \bibverse{5} Und machten sich in der Frühe auf, daß sie zum Heer
der Syrer kämen. Und da sie vorne an den Ort des Heers kamen, siehe, da
war niemand. \bibverse{6} Denn der HErr hatte die Syrer lassen hören ein
Geschrei von Rossen, Wagen und großer Heerkraft, daß sie untereinander
sprachen: Siehe, der König Israels hat wider uns gedinget die Könige der
Hethiter und die Könige der Ägypter, daß sie über uns kommen sollen.
\bibverse{7} Und machten sich auf und flohen in der Frühe; und ließen
ihre Hütten, Rosse und Esel im Lager, wie es stund, und flohen mit ihrem
Leben davon. \bibverse{8} Als nun die Aussätzigen an den Ort des Lagers
kamen, gingen sie in der Hütten eine, aßen und tranken und nahmen
Silber, Gold und Kleider und gingen hin und verbargen es; und kamen
wieder und gingen in eine andere Hütte und nahmen draus und gingen hin
und verbargen es. \bibverse{9} Aber einer sprach zum andern: Laßt uns
nicht also tun! Dieser Tag ist ein Tag guter Botschaft. Wo wir das
verschweigen und harren, bis daß licht Morgen wird, wird unsere Missetat
funden werden; so laßt uns nun hingehen, daß wir kommen und ansagen dem
Hause des Königs. \bibverse{10} Und da sie kamen, riefen sie am Tor der
Stadt und sagten es ihnen an und sprachen: Wir sind zum Lager der Syrer
kommen, und siehe, es ist niemand da, noch keine Menschenstimme, sondern
Rosse und Esel angebunden und die Hütten, wie sie stehen. \bibverse{11}
Da rief man den Torhütern, daß sie es drinnen ansagten im Hause des
Königs. \bibverse{12} Und der König stand auf in der Nacht und sprach zu
seinen Knechten: Laßt euch sagen, wie die Syrer mit uns umgehen. Sie
wissen, daß wir Hunger leiden, und sind aus dem Lager gegangen, daß sie
sich im Felde verkröchen, und denken: Wenn sie aus der Stadt gehen,
wollen wir sie lebendig greifen und in die Stadt kommen. \bibverse{13}
Da antwortete seiner Knechte einer und sprach: Man nehme die fünf
übrigen Rosse, die noch drinnen sind überblieben (siehe, die sind
drinnen überblieben von aller Menge in Israel, welche alle dahin ist),
die laßt uns senden und besehen. \bibverse{14} Da nahmen sie zween Wagen
mit Rossen; und der König sandte sie dem Lager der Syrer nach und
sprach: Ziehet hin und besehet! \bibverse{15} Und da sie ihnen nachzogen
bis an den Jordan, siehe, da lag der Weg voll Kleider und Geräte, welche
die Syrer von sich geworfen hatten, da sie eileten. Und da die Boten
wiederkamen und sagten es dem Könige an, \bibverse{16} ging das Volk
hinaus und beraubte das Lager der Syrer. Und es galt ein Scheffel
Semmelmehl einen Sekel und zween Scheffel Gerste auch einen Sekel nach
dem Wort des HErrn. \bibverse{17} Aber der König bestellete den Ritter,
auf des Hand er sich lehnte, unter das Tor. Und das Volk zertrat ihn im
Tor, daß er starb, wie der Mann GOttes geredet hatte, da der König zu
ihm hinabkam. \bibverse{18} Und geschah, wie der Mann GOttes dem Könige
sagte, da er sprach: Morgen um diese Zeit werden zween Scheffel Gerste
einen Sekel gelten und ein Scheffel Semmelmehl einen Sekel unter dem Tor
zu Samaria; \bibverse{19} und der Ritter antwortete dem Mann GOttes und
sprach: Siehe, wenn der HErr Fenster am Himmel machte, wie möchte
solches geschehen? Er aber sprach: Siehe, mit deinen Augen wirst du es
sehen und nicht davon essen. \bibverse{20} Und es ging ihm eben also;
denn das Volk zertrat ihn im Tor, daß er starb.

\hypertarget{section-7}{%
\section{8}\label{section-7}}

\bibverse{1} Elisa redete mit dem Weibe, des Sohn er hatte lebendig
gemacht, und sprach: Mache dich auf und gehe hin mit deinem Hause und
sei Fremdling, wo du kannst; denn der HErr wird eine Hungersnot rufen,
die wird ins Land kommen sieben Jahre lang. \bibverse{2} Das Weib machte
sich auf und tat, wie der Mann GOttes sagte, und zog hin mit ihrem Hause
und war Fremdling in der Philister Lande sieben Jahre. \bibverse{3} Da
aber die sieben Jahre um waren, kam das Weib wieder aus der Philister
Lande; und sie ging aus, den König anzuschreien um ihr Haus und Acker.
\bibverse{4} Der König aber redete mit Gehasi, dem Knaben des Mannes
GOttes, und sprach: Erzähle mir alle großen Taten, die Elisa getan hat.
\bibverse{5} Und indem er dem König erzählte, wie er hätte einen Toten
lebendig gemacht, siehe, da kam eben dazu das Weib, des Sohn er hatte
lebendig gemacht, und schrie den König an um ihr Haus und Acker. Da
sprach Gehasi: Mein Herr König, dies ist das Weib, und dies ist ihr
Sohn, den Elisa hat lebendig gemacht. \bibverse{6} Und der König fragte
das Weib; und sie erzählte es ihm. Da gab ihr der König einen Kämmerer
und sprach: Schaffe ihr wieder alles, das ihr ist; dazu alles Einkommen
des Ackers, seit der Zeit sie das Land verlassen hat, bis hieher.
\bibverse{7} Und Elisa kam gen Damaskus. Da lag Benhadad, der König zu
Syrien, krank; und man sagte es ihm an und sprach: Der Mann GOttes ist
herkommen. \bibverse{8} Da sprach der König zu Hasael: Nimm Geschenk mit
dir und gehe dem Mann GOttes entgegen; und frage den HErrn durch ihn und
sprich, ob ich von dieser Krankheit möge genesen. \bibverse{9} Hasael
ging ihm entgegen und nahm Geschenk mit sich und allerlei Güter zu
Damaskus, vierzig Kamelen Last. Und da er kam, trat er vor ihn und
sprach:. Dein Sohn Benhadad, der König zu Syrien, hat mich zu dir
gesandt und läßt dir sagen: Kann ich auch von dieser Krankheit genesen?
\bibverse{10} Elisa sprach zu ihm: Gehe hin und sage ihm: Du wirst
genesen; aber der HErr hat mir gezeiget, daß er des Todes sterben wird.
\bibverse{11} Und der Mann GOttes sah ernst und stellete sich ungebärdig
und weinete. \bibverse{12} Da sprach Hasael: Warum weinet mein Herr? Er
sprach: Ich weiß, was Übels du den Kindern Israel tun wirst. Du wirst
ihre festen Städte mit Feuer verbrennen und ihre junge Mannschaft mit
dem Schwert erwürgen und ihre jungen Kinder töten und ihre schwangeren
Weiber zerhauen. \bibverse{13} Hasael sprach: Was ist dein Knecht, der
Hund, daß er solch groß Ding tun sollte? Elisa sprach: Der HErr hat mir
gezeiget, daß du König zu Syrien sein wirst. \bibverse{14} Und er ging
weg von Elisa und kam zu seinem HErrn, der sprach zu ihm: Was sagte dir
Elisa? Er sprach: Er sagte mir: Du wirst genesen. \bibverse{15} Des
andern Tages aber nahm er den Kolter und tunkte ihn in Wasser und
breitete ihn über sich her; da starb er. Und Hasael ward König an seiner
Statt. \bibverse{16} Im fünften Jahr Jorams, des Sohns Ahabs, des Königs
Israels, ward Joram, der Sohn Josaphats, König in Juda. \bibverse{17}
Zweiunddreißig Jahre alt war er, da er König ward; und regierete acht
Jahre zu Jerusalem. \bibverse{18} Und wandelte auf dem Wege der Könige
Israels, wie das Haus Ahabs tat; denn Ahabs Tochter war sein Weib; und
er tat, das dem HErrn übel gefiel. \bibverse{19} Aber der HErr wollte
Juda nicht verderben um seines Knechts David willen; wie er ihm geredet
hatte, ihm zu geben eine Leuchte unter seinen Kindern immerdar.
\bibverse{20} Zu seiner Zeit fielen die Edomiter ab von Juda und machten
einen König über sich. \bibverse{21} Denn Joram war durch Zair gezogen
und alle Wagen mit ihm; und hatte sich des Nachts aufgemacht und die
Edomiter geschlagen, die um ihn her waren, dazu die Obersten über die
Wagen, daß das Volk floh in seine Hütten. \bibverse{22} Darum fielen die
Edomiter ab von Juda bis auf diesen Tag. Auch fiel zu derselben Zeit ab
Libna. \bibverse{23} Was aber mehr von Joram zu sagen ist, und alles,
was er getan hat, siehe, das ist geschrieben in der Chronik der Könige
Judas. \bibverse{24} Und Joram entschlief mit seinen Vätern und ward
begraben mit seinen Vätern in der Stadt Davids. Und Ahasja sein Sohn,
ward König an seiner Statt. \bibverse{25} Im zwölften Jahr Jorams, des
Sohns Ahabs, des Königs Israels, ward Ahasja, der Sohn Jorams, König in
Juda. \bibverse{26} Zweiundzwanzig Jahre alt war Ahasja, da er König
ward, und regierte ein Jahr zu Jerusalem. Seine Mutter hieß Athalja,
eine Tochter Amris, des Königs Israels. \bibverse{27} Und wandelte auf
dem Wege des Hauses Ahabs und tat, das dem HErrn übel gefiel, wie das
Haus Ahabs, denn er war Schwager im Hause Ahabs. \bibverse{28} Und er
zog mit Joram, dem Sohn Ahabs, in Streit wider Hasael, den König zu
Syrien, gen Ramoth in Gilead; aber die Syrer schlugen Joram.
\bibverse{29} Da kehrete Joram, der König, um, daß er sich heilen ließe
zu Jesreel von den Schlägen, die, ihm die Syrer geschlagen hatten zu
Rama, da er mit Hasael, dem Könige zu Syrien, stritt. Und Ahasja, der
Sohn Jorams, der König Judas, kam hinab, zu besehen Joram, den Sohn
Ahabs, zu Jesreel; denn er lag krank.

\hypertarget{section-8}{%
\section{9}\label{section-8}}

\bibverse{1} Elisa aber, der Prophet, rief der Propheten Kinder einem
und sprach zu ihm: Gürte deine Lenden und nimm diesen Ölkrug mit dir und
gehe hin gen Ramoth in Gilead. \bibverse{2} Und wenn du dahin kommst,
wirst du daselbst sehen Jehu, den Sohn Josaphats, des Sohns Nimsis. Und
gehe hinein und heiß ihn aufstehen unter seinen Brüdern und führe ihn in
die innerste Kammer. \bibverse{3} Und nimm den Ölkrug und schütte es auf
sein Haupt und sprich: So sagt der HErr: Ich habe dich zum Könige über
Israel gesalbet. Und sollst die Tür auftun und fliehen und nicht
verziehen. \bibverse{4} Und der Jüngling des Propheten, der Knabe, ging
hin gen Ramoth in Gilead. \bibverse{5} Und da er hineinkam, siehe, da
saßen die Hauptleute des Heers. Und er sprach: Ich habe dir, Hauptmann,
was zu sagen. Jehu sprach: Welchem unter uns allen? Er sprach: Dir,
Hauptmann. \bibverse{6} Da stund er auf und ging hinein. Er aber
schüttete das Öl auf sein Haupt und sprach zu ihm: So sagt der HErr, der
GOtt Israels: Ich habe dich zum Könige gesalbet über des HErrn Volk
Israel. \bibverse{7} Und du sollst das Haus Ahabs, deines Herrn,
schlagen, daß ich das Blut der Propheten, meiner Knechte, und das Blut
aller Knechte des HErrn räche von der Hand Isebels, \bibverse{8} daß das
ganze Haus Ahabs umkomme. Und ich will von Ahab ausrotten den, der an
die Wand pisset, und den Verschlossenen und Verlassenen in Israel.
\bibverse{9} Und will das Haus Ahabs machen wie das Haus Jerobeams, des
Sohns Nebats, und wie das Haus Baesas, des Sohns Ahias. \bibverse{10}
Und die Hunde sollen Isebel fressen auf dem Acker zu Jesreel, und soll
sie niemand begraben. Und er tat die Tür auf und floh. \bibverse{11} Und
da Jehu herausging zu den Knechten seines Herrn, sprach man zu ihm:
Stehet es wohl? Warum ist dieser Rasende zu dir kommen? Er sprach zu
ihnen: Ihr kennet doch den Mann wohl, und was er sagt. \bibverse{12} Sie
sprachen: Das ist nicht wahr; sage es uns aber an. Er sprach: So und so
hat er mit mir geredet und gesagt: So spricht der HErr: Ich habe dich
zum Könige über Israel gesalbet. \bibverse{13} Da eileten sie, und nahm
ein jeglicher sein Kleid und legte es unter ihn auf die hohen Stufen,
und bliesen mit der Posaune und sprachen: Jehu ist König worden!
\bibverse{14} Also machte Jehu, der Sohn Josaphats, des Sohns Nimsis,
einen Bund wider Joram. Joram aber lag vor Ramoth in Gilead mit dem
ganzen Israel wider Hasael, den König zu Syrien. \bibverse{15} Joram
aber, der König, war wiederkommen, daß er sich heilen ließe zu Jesreel
von den Schlägen, die ihm die Syrer geschlagen hatten, da er stritt mit
Hasael dem Könige zu Syrien. Und Jehu sprach: Ist's euer Gemüt, so soll
niemand entrinnen aus der Stadt, daß er hingehe und ansage zu Jesreel.
\bibverse{16} Und er ließ sich führen und zog gen Jesreel, denn Joram
lag daselbst; so war Ahasja, der König Judas, hinabgezogen, Joram zu
besehen. \bibverse{17} Der Wächter aber, der auf dem Turm zu Jesreel
stund, sah den Haufen Jehus kommen und sprach: Ich sehe einen Haufen. Da
sprach Joram: Nimm einen Reiter und sende ihnen entgegen und sprich:
Ist's Friede? \bibverse{18} Und der Reiter ritt hin ihm entgegen und
sprach: So sagt der König: Ist's Friede? Jehu sprach: Was gehet dich der
Friede an? Wende dich hinter mich! Der Wächter verkündigte und sprach:
Der Bote ist zu ihnen kommen und kommt nicht wieder. \bibverse{19} Da
sandte er einen andern Reiter. Da der zu ihm kam, sprach er: So spricht
der König: Ist's Friede? Jehu sprach: Was gehet dich der Friede an?
Wende dich hinter mich! \bibverse{20} Das verkündigte der Wächter und
sprach: Er ist zu ihnen kommen und kommt nicht wieder. Und es ist ein
Treiben, wie das Treiben Jehus, des Sohns Nimsis; denn er treibt, wie er
unsinnig wäre. \bibverse{21} Da sprach Joram: Spannet an! Und man
spannete seinen Wagen an; und sie zogen aus, Joram, der König Israels,
und Ahasja, der König Judas, ein jeglicher auf seinem Wagen, daß sie
Jehu entgegenkämen; und sie trafen ihn an auf dem Acker Naboths, des
Jesreeliten. \bibverse{22} Und da Joram Jehu sah, sprach er: Jehu, ist's
Friede? Er aber sprach: Was Friede? Deiner Mutter Isebel Hurerei und
Zauberei wird immer größer. \bibverse{23} Da wandte Joram seine Hand und
floh und sprach zu Ahasja: Es ist Verräterei, Ahasja! \bibverse{24} Aber
Jehu fassete den Bogen und schoß Joram zwischen die Arme, daß der Pfeil
durch sein Herz ausfuhr; und fiel in seinen Wagen. \bibverse{25} Und er
sprach zum Ritter Bidekar: Nimm und wirf ihn aufs Stück Acker Naboths,
des Jesreeliten. Denn ich gedenke, daß du mit mir auf einem Wagen seinem
Vater Ahab nachfuhrest, daß der HErr solche Last über ihn hub.
\bibverse{26} Was gilt's, sprach der HErr, ich will dir das Blut Naboths
und seiner Kinder, das ich gestern sah, vergelten auf diesem Acker. So
nimm nun und wirf ihn auf den Acker nach dem Wort des HErrn.
\bibverse{27} Da das Ahasja, der König Judas, sah, floh er des Weges zum
Hause des Gartens. Jehu aber jagte ihm nach und hieß ihn auch schlagen
auf dem Wagen gen Gur hinan, die bei Jeblaam liegt. Und er floh gen
Megiddo und starb daselbst. \bibverse{28} Und seine Knechte ließen ihn
führen gen Jerusalem und begruben ihn in seinem Grabe mit seinen Vätern
in der Stadt Davids. \bibverse{29} Ahasja aber regierete über Juda im
elften Jahr Jorams, des Sohns Ahabs. \bibverse{30} Und da Jehu gen
Jesreel kam, und Isebel das erfuhr, schminkte sie ihr Angesicht und
schmückte ihr Haupt und guckte zum Fenster aus. \bibverse{31} Und da
Jehu unter das Tor kam, sprach sie: Ist's Simri wohl gegangen, der
seinen Herrn erwürgete? \bibverse{32} Und er hub sein Angesicht auf zum
Fenster und sprach: Wer ist bei mir hie? Da wandten sich zween oder drei
Kämmerer zu ihm. \bibverse{33} Er sprach: Stürzet sie herab! Und sie
stürzten sie herab, daß die Wand und die Rosse mit ihrem Blut besprenget
wurden; und sie ward zertreten. \bibverse{34} Und da er hineinkam und
gegessen und getrunken hatte, sprach er: Besehet doch die Verfluchte und
begrabet sie; denn sie ist eines Königs Tochter. \bibverse{35} Da sie
aber hingingen, sie zu begraben, fanden sie nichts von ihr denn den
Schädel und Füße und ihre flachen Hände. \bibverse{36} Und kamen wieder
und sagten's ihm an. Er aber sprach: Es ist's, das der HErr geredet hat
durch seinen Knecht Elia, den Thisbiten, und gesagt: Auf dem Acker
Jesreels sollen die Hunde der Isebel Fleisch fressen. \bibverse{37} Also
war das Aas Isebels wie Kot auf dem Felde im Acker Jesreels, daß man
nicht sagen konnte: Das ist Isebel.

\hypertarget{section-9}{%
\section{10}\label{section-9}}

\bibverse{1} Ahab aber hatte siebenzig Söhne zu Samaria. Und Jehu
schrieb Briefe und sandte sie gen Samaria zu den Obersten der Stadt
Jesreel, zu den Ältesten und Vormündern Ahabs, die lauteten also:
\bibverse{2} Wenn dieser Brief zu euch kommt, bei denen eures Herrn
Söhne sind, Wagen, Rosse, feste Städte und Rüstung, \bibverse{3} so
sehet, welcher der beste und geschickteste sei unter den Söhnen eures
Herrn und setzet ihn auf seines Vaters Stuhl und streitet für eures
Herrn Haus. \bibverse{4} Sie aber fürchteten sich fast sehr und
sprachen: Siehe, zween Könige sind nicht gestanden vor ihm, wie wollen
wir denn stehen? \bibverse{5} Und die über das Haus und über die Stadt
waren, und die Ältesten und Vormünder sandten hin zu Jehu und ließen ihm
sagen: Wir sind deine Knechte; wir wollen alles tun, was du uns sagst;
wir wollen niemand zum Könige machen. Tue, was dir gefällt! \bibverse{6}
Da schrieb er den andern Brief zu ihnen, der lautete also: So ihr mein
seid und meiner Stimme gehorchet, so nehmet die Häupter von den Männern,
eures Herrn Söhnen, und bringet sie zu mir morgen um diese Zeit gen
Jesreel. (Der Söhne aber des Königs waren siebenzig Mann, und die
Größten der Stadt zogen sie auf.) \bibverse{7} Da nun der Brief zu ihnen
kam, nahmen sie des Königs Söhne und schlachteten siebenzig Mann; und
legten ihre Häupter in Körbe und schickten sie zu ihm gen Jesreel.
\bibverse{8} Und da der Bote kam und sagte es ihm an und sprach: Sie
haben die Häupter des Königs Kinder gebracht, sprach er: Legt sie auf
zween Haufen vor der Tür am Tor bis morgen. \bibverse{9} Und des
Morgens, da er ausging, trat er dahin und sprach zu allem Volk: Ihr
wollt ja recht haben. Siehe, habe ich wider meinen Herrn einen Bund
gemacht und ihn erwürget? Wer hat denn diese alle geschlagen?
\bibverse{10} So erkennet ihr ja, daß kein Wort des HErrn ist auf die
Erde gefallen, das der HErr geredet hat wider das Haus Ahabs; und der
HErr hat getan, wie er geredet hat durch seinen Knecht Elia.
\bibverse{11} Also schlug Jehu alle übrigen vom Hause Ahabs zu Jesreel,
alle seine Großen, seine Verwandten und seine Priester, bis daß ihm
nicht einer überblieb. \bibverse{12} Und machte sich auf, zog hin und
kam gen Samaria. Unterwegen aber war ein Hirtenhaus. \bibverse{13} Da
traf Jehu an die Brüder Ahasjas, des Königs Judas, und sprach: Wer seid
ihr? Sie sprachen: Wir sind Brüder Ahasjas und ziehen hinab, zu grüßen
des Königs Kinder und der Königin Kinder. \bibverse{14} Er aber sprach:
Greifet sie lebendig! Und sie griffen sie lebendig und schlachteten sie
bei dem Brunnen am Hirtenhaus, zweiundvierzig Mann, und ließ nicht einen
von ihnen übrig. \bibverse{15} Und da er von dannen zog, fand er
Jonadab, den Sohn Rechabs, der ihm begegnete, und grüßte ihn und sprach
zu ihm: Ist dein Herz richtig, wie mein Herz mit deinem Herzen? Jonadab
sprach: Ja. Ist's also, so gib mir deine Hand. Und er gab ihm seine
Hand. Und er ließ ihn zu ihm auf den Wagen sitzen \bibverse{16} und
sprach: Komm mit mir und sieh meinen Eifer um den HErrn. Und sie
führeten ihn mit ihm auf seinen Wagen. \bibverse{17} Und da er gen
Samaria kam, schlug er alles, was übrig war von Ahab, zu Samaria, bis
daß er ihn vertilgete, nach dem Wort des HErrn, das er zu Elia geredet
hatte. \bibverse{18} Und Jehu versammelte alles Volk und ließ zu ihnen
sagen: Ahab hat Baal wenig gedienet, Jehu will ihm baß dienen.
\bibverse{19} So laßt nun rufen alle Propheten Baals, alle seine Knechte
und alle seine Priester zu mir, daß man niemands vermisse; denn ich habe
ein groß Opfer dem Baal zu tun. Wes man vermissen wird, der soll nicht
leben. Aber Jehu tat solches zu untertreten, daß er die Diener Baals
umbrächte. \bibverse{20} Und Jehu sprach: Heiliget dem Baal das Fest und
laßt es ausrufen! \bibverse{21} Auch sandte Jehu in ganz Israel und ließ
alle Diener Baals kommen, daß niemand übrig war, der nicht käme. Und sie
kamen in das Haus Baals, daß das Haus Baals voll ward an allen Enden.
\bibverse{22} Da sprach er zu denen, die über das Kleiderhaus waren:
Bringet allen Dienern Baals Kleider heraus! Und sie brachten die Kleider
heraus. \bibverse{23} Und Jehu ging in die Kirche Baals mit Jonadab, dem
Sohn Rechabs, und sprach zu den Dienern Baals: Forschet und sehet zu,
daß nicht hie unter euch sei des HErrn Diener jemand, sondern Baals
Diener alleine. \bibverse{24} Und da sie hineinkamen, Opfer und
Brandopfer zu tun, bestellete ihm Jehu außen achtzig Mann und sprach:
Wenn der Männer jemand entrinnet, die ich unter eure Hände gebe, so soll
für seine Seele desselben Seele sein. \bibverse{25} Da er nun die
Brandopfer vollendet hatte, sprach Jehu zu den Trabanten und Rittern:
Gehet hinein und schlaget jedermann; lasset niemand herausgehen! Und sie
schlugen sie mit der Schärfe des Schwerts. Und die Trabanten und Ritter
warfen sie weg und gingen zur Stadt der Kirche Baals. \bibverse{26} Und
brachten heraus die Säulen in der Kirche Baals und verbrannten sie.
\bibverse{27} Und zerbrachen die Säule Baals samt der Kirche Baals und
machten ein heimlich Gemach draus bis auf diesen Tag. \bibverse{28} Also
vertilgete Jehu den Baal aus Israel. \bibverse{29} Aber von den Sünden
Jerobeams, des Sohns Nebats, der Israel sündigen machte, ließ Jehu
nicht, von den güldenen Kälbern zu Bethel und zu Dan. \bibverse{30} Und
der HErr sprach zu Jehu: Darum, daß du willig gewesen bist zu tun, was
mir gefallen hat, und hast am Hause Ahabs getan alles, was in meinem
Herzen war, sollen dir. auf deinem Stuhl Israels sitzen deine Kinder ins
vierte Glied. \bibverse{31} Aber doch hielt Jehu nicht, daß er im Gesetz
des HErrn, des GOttes Israels, wandelte von ganzem Herzen; denn er ließ
nicht von den Sünden Jerobeams, der Israel hatte sündigen gemacht.
\bibverse{32} Zur selbigen Zeit fing der HErr an, überdrüssig zu werden
über Israel; denn Hasael schlug sie in allen Grenzen Israels,
\bibverse{33} vom Jordan gegen der Sonnen Aufgang und das ganze Land
Gilead der Gaditer, Rubeniter und Manassiter, von Aroer an, die am Bach
bei Arnon liegt, und Gilead und Basan. \bibverse{34} Was aber mehr von
Jehu zu sagen ist, und alles, was er getan hat, und alle seine Macht,
siehe, das ist geschrieben in der Chronik der Könige Israels.
\bibverse{35} Und Jehu entschlief mit seinen Vätern, und sie begruben
ihn zu Samaria. Und Joahas, sein Sohn, ward König an seiner Statt.
\bibverse{36} Die Zeit aber, die Jehu über Israel regieret hat zu
Samaria, sind achtundzwanzig Jahre.

\hypertarget{section-10}{%
\section{11}\label{section-10}}

\bibverse{1} Athalja aber, Ahasjas Mutter, da sie sah, daß ihr Sohn tot
war, machte sie sich auf und brachte um allen königlichen Samen.
\bibverse{2} Aber Joseba, die Tochter des Königs Joram, Ahasjas
Schwester, nahm Joas, den Sohn Ahasjas, und stahl ihn aus des Königs
Kindern, die getötet wurden, mit seiner Amme in der Schlafkammer; und
sie verbargen ihn vor Athalja, daß er nicht getötet ward. \bibverse{3}
Und er war mit ihr versteckt im Hause des HErrn sechs Jahre. Athalja
aber war Königin im Lande. \bibverse{4} Im siebenten Jahr aber sandte
hin Jojada und nahm die Obersten über hundert mit den Hauptleuten und
die Trabanten und ließ sie zu sich ins Haus des HErrn kommen; und machte
einen Bund mit ihnen und nahm einen Eid von ihnen im Hause des HErrn;
und zeigte ihnen des Königs Sohn. \bibverse{5} Und gebot ihnen und
sprach: Das ist's, das ihr tun sollt: Euer ein dritter Teil, die ihr des
Sabbats angehet, sollen der Hut warten im Hause des Königs; \bibverse{6}
und ein dritter Teil soll sein am Tor Sur und ein dritter Teil am Tor,
das hinter den Trabanten ist; und sollt der Hut warten am Hause Massa.
\bibverse{7} Aber zwei Teile euer aller, die ihr des Sabbats abgehet,
sollen der Hut warten im Hause des HErrn um den König. \bibverse{8} Und
sollt rings um den König euch machen, und ein jeglicher mit seiner Wehr
in der Hand; und wer herein zwischen die Wand kommt, der sterbe, daß ihr
bei dem Könige seid, wenn er aus und ein gehet. \bibverse{9} Und die
Obersten über hundert taten alles, wie ihnen Jojada, der Priester,
geboten hatte, und nahmen zu sich ihre Männer, die des Sabbats angingen,
mit denen, die des Sabbats abgingen, und kamen zu dem Priester Jojada.
\bibverse{10} Und der Priester gab den Hauptleuten Spieße und Schilde,
die des Königs Davids gewesen waren und in dem Hause des HErrn waren.
\bibverse{11} Und die Trabanten stunden um den König her, ein jeglicher
mit seiner Wehr in der Hand, von dem Winkel des Hauses zur Rechten bis
zum Winkel zur Linken, zum Altar zu und zum Hause. \bibverse{12} Und er
ließ des Königs Sohn hervorkommen und setzte ihm eine Krone auf und gab
ihm das Zeugnis; und machten ihn zum Könige und salbeten ihn und
schlugen die Hände zusammen und sprachen: Glück zu dem Könige!
\bibverse{13} Und da Athalja hörete das Geschrei des Volks, das zulief,
kam sie zum Volk in das Haus des HErrn \bibverse{14} und sah, siehe, da
stund der König an der Säule, wie es Gewohnheit war, und die Sänger und
Trommeten bei dem Könige; und alles Volk des Landes war fröhlich und
bliesen mit Trommeten. Athalja aber zerriß ihre Kleider und sprach:
Aufruhr, Aufruhr! \bibverse{15} Aber der Priester Jojada gebot den
Obersten über hundert, die über das Heer gesetzt waren, und sprach zu
ihnen: Führet sie zum Hause hinaus in den Hof; und wer ihr folget, der
sterbe des Schwerts! Denn der Priester hatte gesagt, sie sollte nicht im
Hause des HErrn sterben. \bibverse{16} Und sie legten die Hände an sie;
und sie ging hinein des Weges, da die Rosse zum Hause des Königs gehen,
und ward daselbst getötet. \bibverse{17} Da machte Jojada einen Bund
zwischen dem HErrn und dem Könige und dem Volk, daß sie des HErrn Volk
sein sollten; also auch zwischen dem Könige und dem Volk. \bibverse{18}
Da ging alles Volk des Landes in die Kirche Baals und brachen seine
Altäre ab und zerbrachen seine Bildnisse recht wohl; und Mathan, den
Priester Baals, erwürgeten sie vor den Altären. Der Priester aber
bestellete die Ämter im Hause des HErrn. \bibverse{19} Und nahm die
Obersten über hundert und die Hauptleute und die Trabanten und alles
Volk des Landes und führeten den König hinab vom Hause des HErrn; und
kamen auf dem Wege von dem Tor der Trabanten zum Königshause; und er
setzte sich auf der Könige Stuhl. \bibverse{20} Und alles Volk im Lande
war fröhlich, und die Stadt war stille. Athalja aber töteten sie mit dem
Schwert in des Königs Hause. \bibverse{21} Und Joas war sieben Jahre
alt, da er König ward.

\hypertarget{section-11}{%
\section{12}\label{section-11}}

\bibverse{1} Im siebenten Jahr Jehus ward Joas König und regierete
vierzig Jahre zu Jerusalem. Seine Mutter hieß Zibea von Bersaba.
\bibverse{2} Und Joas tat, was recht war und dem HErrn wohlgefiel,
solange ihn der Priester Jojada lehrete, \bibverse{3} ohne daß sie die
Höhen nicht abtaten; denn das Volk opferte und räucherte noch auf den
Höhen. \bibverse{4} Und Joas sprach zu den Priestern: Alles Geld, das
geheiliget wird, daß es in das Haus des HErrn gebracht werde, das gang
und gäbe ist, das Geld, so jedermann gibt in der Schätzung seiner Seele,
und alles Geld, das jedermann von freiem Herzen opfert, daß es in des
HErrn Haus gebracht werde, \bibverse{5} das laßt die Priester zu sich
nehmen, einen jeglichen von seinem Bekannten. Davon sollen sie bessern,
was baufällig ist am Hause des HErrn, wo sie finden, das baufällig ist.
\bibverse{6} Da aber die Priester bis ins dreiundzwanzigste Jahr des
Königs Joas nicht besserten, was baufällig war am Hause, \bibverse{7}
rief der König Joas dem Priester Jojada samt den Priestern und sprach zu
ihnen: Warum bessert ihr nicht, was baufällig ist am Hause? So sollt ihr
nun nicht zu euch nehmen das Geld, ein jeglicher von seinen Bekannten,
sondern sollt es geben zu dem, das baufällig ist am Hause. \bibverse{8}
Und die Priester bewilligten vom Volk nicht Geld zu nehmen und das
Baufällige am Hause zu bessern. \bibverse{9} Da nahm der Priester Jojada
eine Lade und bohrte oben ein Loch drein und setzte sie zur rechten Hand
neben den Altar, da man in das Haus des HErrn gehet. Und die Priester,
die an der Schwelle hüteten, taten drein alles Geld, das zu des HErrn
Haus gebracht ward. \bibverse{10} Wenn sie dann sahen, daß viel Geld in
der Lade war, so kam des Königs Schreiber herauf mit dem Hohenpriester
und banden das Geld zusammen und zählten es, was für des HErrn Haus
funden ward. \bibverse{11} Und man gab das Geld bar über denen, die da
arbeiteten und bestellet waren zum Hause des HErrn; und sie gaben's
heraus den Zimmerleuten, die da baueten und arbeiteten am Hause des
HErrn, \bibverse{12} nämlich den Maurern und Steinmetzen, und die da
Holz und gehauene Steine kauften, daß das Baufällige am Hause des HErrn
gebessert wurde, und alles, was sie fanden am Hause zu bessern not sein.
\bibverse{13} Doch ließ man nicht machen silberne Schalen, Psalter,
Becken, Trommeten noch irgend ein gülden oder silbern Gerät im Hause des
HErrn von solchem Gelde, das zu des HErrn Hause gebracht ward,
\bibverse{14} sondern man gab's den Arbeitern, daß sie damit das
Baufällige am Hause des HErrn besserten. \bibverse{15} Auch durften die
Männer nicht berechnen, denen man das Geld tat, daß sie es den Arbeitern
gäben, sondern sie handelten auf Glauben. \bibverse{16} Aber das Geld
von Schuldopfern und Sündopfern ward nicht zum Hause des HErrn gebracht;
denn es war der Priester. \bibverse{17} Zu der Zeit zog Hasael, der
König zu Syrien, herauf und stritt wider Gath und gewann sie. Und da
Hasael sein Angesicht stellete, zu Jerusalem hinaufzuziehen,
\bibverse{18} nahm Joas, der König Judas, all das Geheiligte, das seine
Väter, Josaphat, Joram und Ahasja, die Könige Judas, geheiliget hatten,
und was er geheiliget hatte, dazu alles Gold, das man fand im Schatz in
des HErrn Hause und in des Königs Hause, und schickte es Hasael, dem
Könige zu Syrien. Da zog er ab von Jerusalem. \bibverse{19} Was aber
mehr von Joas zu sagen ist, und alles, was er getan hat, das ist
geschrieben in der Chronik der Könige Judas. \bibverse{20} Und seine
Knechte empörten sich und machten einen Bund und schlugen ihn im Hause
Millo, da man hinabgehet zu Silla. \bibverse{21} Denn Josabar, der Sohn
Simeaths, und Josabad, der Sohn Somers, seine Knechte, schlugen ihn tot.
Und man begrub ihn mit seinen Vätern in der Stadt Davids. Und Amazia,
sein Sohn, ward König an seiner Statt.

\hypertarget{section-12}{%
\section{13}\label{section-12}}

\bibverse{1} Im dreiundzwanzigsten Jahr Joas des Sohns Ahasjas, des
Königs Judas, ward Joahas, der Sohn Jehus, König über Israel zu Samaria
siebenzehn Jahre. \bibverse{2} Und tat, das dem HErrn übel gefiel, und
wandelte den Sünden nach Jerobeams, des Sohns Nebats, der Israel
sündigen machte, und ließ nicht davon. \bibverse{3} Und des HErrn Zorn
ergrimmete über Israel und gab sie unter die Hand Hasaels, des Königs zu
Syrien, und Benhadads, des Sohns Hasaels, ihr Leben lang. \bibverse{4}
Aber Joahas bat des HErrn Angesicht. Und der HErr erhörete ihn; denn er
sah den Jammer Israels an, wie sie der König zu Syrien drängete.
\bibverse{5} Und der HErr gab Israel einen Heiland, der sie aus der
Gewalt der Syrer führete, daß die Kinder Israel in ihren Hütten
wohneten, wie vorhin. \bibverse{6} Doch ließen sie nicht von der Sünde
des Hauses Jerobeams, der Israel sündigen machte, sondern wandelten
drinnen. Auch blieb stehen der Hain zu Samaria. \bibverse{7} Denn es war
des Volks Joahas nicht mehr überblieben denn fünfzig Reiter, zehn Wagen
und zehntausend Fußvolks. Denn der König zu Syrien hatte sie umgebracht
und hatte sie gemacht wie Drescherstaub. \bibverse{8} Was aber mehr von
Joahas zu sagen ist, und alles, was er getan hat, und seine Macht,
siehe, das ist geschrieben in der Chronik der Könige Israels.
\bibverse{9} Und Joahas entschlief mit seinen Vätern, und man begrub ihn
zu Samaria. Und sein Sohn Joas ward König an seiner Statt. \bibverse{10}
Im siebenunddreißigsten Jahr Joas des Königs Judas, ward Joas, der Sohn
Joahas König über Israel zu Samaria sechzehn Jahre. \bibverse{11} Und
tat, das dem HErrn übel gefiel, und ließ nicht von allen Sünden
Jerobeams, des Sohns Nebats, der Israel sündigen machte, sondern
wandelte drinnen. \bibverse{12} Was aber mehr von Joas zu sagen ist, und
was er getan hat, und seine Macht, wie er mit Amazia, dem Könige Judas,
gestritten hat, siehe, das ist geschrieben in der Chronik der Könige
Israels. \bibverse{13} Und Joas entschlief mit seinen Vätern, und
Jerobeam saß auf seinem Stuhl. Joas aber ward begraben zu Samaria bei
den Königen Israels. \bibverse{14} Elisa aber ward krank, daran er auch
starb. Und Joas, der König Israels, kam zu ihm hinab und weinete vor ihm
und sprach: Mein Vater, mein Vater, Wagen Israels, und seine Reiter!
\bibverse{15} Elisa aber sprach zu ihm: Nimm den Bogen und Pfeile! Und
da er den Bogen und die Pfeile nahm, \bibverse{16} sprach er zum Könige
Israels: Spanne mit deiner Hand den Bogen! Und er spannete mit seiner
Hand. Und Elisa legte seine Hand auf des Königs Hand \bibverse{17} und
sprach: Tue das Fenster auf gegen Morgen! Und er tat es auf. Und Elisa
sprach: Schieße! Und er schoß. Er aber sprach: Ein Pfeil des Heils vom
HErrn, ein Pfeil des Heils wider die Syrer; und du wirst die Syrer
schlagen zu Aphek, bis sie aufgerieben sind. \bibverse{18} Und er
sprach: Nimm die Pfeile! Und da er sie nahm, sprach er zum Könige
Israels: Schlage die Erde! Und er schlug dreimal und stund stille.
\bibverse{19} Da ward der Mann GOttes zornig auf ihn und sprach: Hättest
du fünf- oder sechsmal geschlagen, so würdest du die Syrer geschlagen
haben, bis sie aufgerieben wären; nun aber wirst du sie dreimal
schlagen. \bibverse{20} Da aber Elisa gestorben war, und man ihn
begraben hatte, fielen die Kriegsleute der Moabiter ins Land desselben
Jahres. \bibverse{21} Und es begab sich, daß sie einen Mann begruben; da
sie aber die Kriegsleute sahen, warfen sie den Mann in Elisas Grab. Und
da er hin kam und die Gebeine Elisas anrührete, ward er lebendig und
trat auf seine Füße. \bibverse{22} Also zwang nun Hasael, der König zu
Syrien, Israel, solange Joahas lebte. \bibverse{23} Aber der HErr tat
ihnen Gnade und erbarmete sich ihrer und wandte sich zu ihnen um seines
Bundes willen mit Abraham, Isaak und Jakob; und wollte sie nicht
verderben, verwarf sie auch nicht von seinem Angesicht bis auf diese
Stunde. \bibverse{24} Und Hasael, der König zu Syrien, starb, und sein
Sohn Benhadad ward König an seiner Statt. \bibverse{25} Joas aber
kehrete um und nahm die Städte aus der Hand Benhadads, des Sohns
Hasaels, die er aus der Hand seines Vaters Joahas genommen hatte mit
Streit. Dreimal schlug ihn Joas und brachte die Städte Israels wieder.

\hypertarget{section-13}{%
\section{14}\label{section-13}}

\bibverse{1} Im andern Jahr Joas, des Sohns Joahas, des Königs Israels,
ward Ahazia König, der Sohn Joas, des Königs Judas. \bibverse{2}
Fünfundzwanzig Jahre alt war er, da er König ward, und regierete
neunundzwanzig Jahre zu Jerusalem. Seine Mutter hieß Joadan von
Jerusalem. \bibverse{3} Und er tat, was dem HErrn wohlgefiel, doch nicht
wie sein Vater David, sondern wie sein Vater Joas tat er auch.
\bibverse{4} Denn die Höhen wurden nicht abgetan, sondern das Volk
opferte und räucherte noch auf den Höhen. \bibverse{5} Da er nun des
Königsreichs mächtig ward, schlug er seine Knechte, die seinen Vater,
den König, geschlagen hatten. \bibverse{6} Aber die Kinder der
Totschläger tötete er nicht; wie es denn geschrieben stehet im
Gesetzbuch Moses, da der HErr geboten hat und gesagt: Die Väter sollen
nicht um der Kinder willen sterben, und die Kinder sollen nicht um der
Väter willen sterben, sondern ein jeglicher soll um seiner Sünde willen
sterben. \bibverse{7} Er schlug auch der Edomiter im Salztal zehntausend
und gewann die Stadt Sela mit Streit; und hieß sie Jaktheel bis auf
diesen Tag. \bibverse{8} Da sandte Amazia Boten zu Joas, dem Sohn Joahas
des Sohns Jehus, dem Könige Israels, und ließ ihm sagen: Komm her, laß
uns miteinander besehen! \bibverse{9} Aber Joas, der König Israels,
sandte zu Amazia, dem Könige Judas, und ließ ihm sagen: Der Dornstrauch,
der im Libanon ist, sandte zur Zeder im Libanon und ließ ihr sagen: Gib
deine Tochter meinem Sohn zum Weibe! Aber das Wild auf dem Felde im
Libanon lief über den Dornstrauch und zertrat ihn. \bibverse{10} Du hast
die Edomiter geschlagen, des überhebt sich dein Herz. Habe den Ruhm und
bleibe daheim; warum ringest du nach Unglück, daß du fallest und Juda
mit dir? \bibverse{11} Aber Amazia gehorchte nicht. Da zog Joas, der
König Israels, herauf; und sie besahen sich miteinander, er und Amazia,
der König Judas, zu Beth-Semes, die in Juda liegt. \bibverse{12} Aber
Juda ward geschlagen vor Israel, daß ein jeglicher floh in seine Hütte.
\bibverse{13} Und Joas, der König Israels, griff Amazia, den König
Judas, den Sohn Joas des Sohns Ahasjas, zu Beth-Semes; und kam gen
Jerusalem und zerriß die Mauern Jerusalems von dem Tor Ephraim an bis an
das Ecktor, vierhundert Ellen lang. \bibverse{14} Und nahm alles Gold
und Silber und Geräte, das funden ward im Hause des HErrn und im Schatz
des Königshauses, dazu die Kinder zu Pfande; und zog wieder gen Samaria.
\bibverse{15} Was aber mehr von Joas zu sagen ist, das er getan hat, und
seine Macht, und wie er mit Amazia, dem Könige Judas, gestritten hat,
siehe, das ist geschrieben in der Chronik der Könige Israels.
\bibverse{16} Und Joas entschlief mit seinen Vätern und ward begraben zu
Samaria unter den Königen Israels. Und sein Sohn Jerobeam ward König an
seiner Statt. \bibverse{17} Amazia aber, der Sohn Joas des Königs Judas,
lebte nach dem Tode Joas des Sohns Joahas des Königs Israels, fünfzehn
Jahre. \bibverse{18} Was aber mehr von Amazia zu sagen ist, das ist
geschrieben in der Chronik der Könige Judas. \bibverse{19} Und sie
machten einen Bund wider ihn zu Jerusalem; er aber floh gen Lachis. Und
sie sandten hin ihm nach gen Lachis und töteten ihn daselbst.
\bibverse{20} Und sie brachten ihn auf Rossen; und er ward begraben zu
Jerusalem bei seinen Vätern in der Stadt Davids. \bibverse{21} Und das
ganze Volk Judas nahm Asarja in seinem sechzehnten Jahr und machten ihn
zum Könige anstatt seines Vaters Amazia. \bibverse{22} Er bauete Elath
und brachte sie wieder zu Juda, nachdem der König mit seinen Vätern
entschlafen war. \bibverse{23} Im fünfzehnten Jahr Amazias, des Sohns
Joas des Königs Judas, ward Jerobeam, der Sohn Joas König über Israel zu
Samaria einundvierzig Jahre. \bibverse{24} Und tat, das dem HErrn übel
gefiel, und ließ nicht ab von allen Sünden Jerobeams, des Sohns Nebats,
der Israel sündigen machte. \bibverse{25} Er aber brachte wieder herzu
die Grenze Israels von Hemath an bis ans Meer, das im blachen Felde
liegt, nach dem Wort des HErrn, des GOttes Israels, das er geredet hatte
durch seinen Knecht Jona, den Sohn Amithais, den Propheten, der von
Gath-Hepher war. \bibverse{26} Denn der HErr sah an den elenden Jammer
Israels, daß auch die Verschlossenen und Verlassenen dahin waren, und
kein Helfer war in Israel. \bibverse{27} Und der HErr hatte nicht
geredet, daß er wollte den Namen Israels austilgen unter dem Himmel, und
half ihnen durch Jerobeam, den Sohn Joas. \bibverse{28} Was aber mehr
von Jerobeam zu sagen ist, und alles, was er getan hat, und seine Macht,
wie er gestritten hat, und wie er Damaskus und Hemath wiedergebracht an
Juda in Israel, siehe, das ist geschrieben in der Chronik der Könige
Israels. \bibverse{29} Und Jerobeam entschlief mit seinen Vätern, mit
den Königen Israels. Und sein Sohn Sacharja ward König an seiner Statt.

\hypertarget{section-14}{%
\section{15}\label{section-14}}

\bibverse{1} Im siebenundzwanzigsten Jahr Jerobeams, des Königs Israels,
ward König Asarja, der Sohn Amazias, des Königs Judas. \bibverse{2} Und
war sechzehn Jahre alt, da er König ward, und regierete zweiundfünfzig
Jahre zu Jerusalem. Seine Mutter hieß Jechalja von Jerusalem.
\bibverse{3} Und tat, das dem HErrn wohlgefiel, allerdinge wie sein
Vater Amazia, \bibverse{4} ohne daß sie die Höhen nicht abtaten; denn
das Volk opferte und räucherte noch auf den Höhen. \bibverse{5} Der HErr
plagte aber den König, daß er aussätzig war bis an seinen Tod, und
wohnete in einem sondern Hause. Jotham aber, des Königs Sohn, regierete
das Haus und richtete das Volk im Lande. \bibverse{6} Was aber mehr von
Asarja zu sagen ist, und alles, was er getan hat, siehe, das ist
geschrieben in der Chronik der Könige Judas. \bibverse{7} Und Asarja
entschlief mit seinen Vätern; und man begrub ihn bei seinen Vätern in
der Stadt Davids. Und sein Sohn Jotham ward König an seiner Statt.
\bibverse{8} Im achtunddreißigsten Jahr Asarjas, des Königs Judas, ward
König Sacharja, der Sohn Jerobeams, über Israel zu Samaria sechs Monden.
\bibverse{9} Und tat, das dem HErrn übel gefiel, wie seine Väter getan
hatten. Er ließ nicht ab von den Sünden Jerobeams, des Sohns Nebats, der
Israel sündigen machte. \bibverse{10} Und Sallum, der Sohn Jabes, machte
einen Bund wider ihn und schlug ihn vor dem Volk und tötete ihn; und
ward König an seiner Statt. \bibverse{11} Was aber mehr von Sacharja zu
sagen ist, siehe, das ist geschrieben in der Chronik der Könige Israels.
\bibverse{12} Und das ist's, das der HErr Jehu geredet hatte: Dir sollen
Kinder ins vierte Glied sitzen auf dem Stuhl Israels; und ist also
geschehen. \bibverse{13} Sallum aber, der Sohn Jabes ward König im
neununddreißigsten Jahr Asarjas, des Königs Judas, und regierete einen
Monden zu Samaria. \bibverse{14} Denn Menahem, der Sohn Gadis, zog
herauf von Thirza und kam gen Samaria und schlug Sallum, den Sohn Jabes
zu Samaria und tötete ihn; und ward König an seiner Statt. \bibverse{15}
Was aber mehr von Sallum zu sagen ist und seinem Bund, den er
anrichtete, siehe, das ist geschrieben in der Chronik der Könige
Israels. \bibverse{16} Dazumal schlug Menahem Tiphsah und alle, die
drinnen waren, und ihre Grenze von Thirza, darum daß sie ihn nicht
wollten einlassen; und schlug alle ihre Schwangeren und zerriß sie.
\bibverse{17} Im neununddreißigsten Jahr Asarjas, des Königs Judas, ward
König Menahem, der Sohn Gadis, über Israel zehn Jahre zu Samaria.
\bibverse{18} Und tat, das dem HErrn übel gefiel. Er ließ sein Leben
lang nicht von den Sünden Jerobeams, des Sohns Nebats, der Israel
sündigen machte. \bibverse{19} Und es kam Phul, der König von Assyrien,
ins Land. Und Menahem gab dem Phul tausend Zentner Silbers, daß er's mit
ihm hielte, und bekräftigte ihm das Königreich. \bibverse{20} Und
Menahem setzte ein Geld in Israel auf die Reichsten, fünfzig Sekel
Silbers auf einen jeglichen Mann, das er dem Könige von Assyrien gäbe.
Also zog der König von Assyrien wieder heim und blieb nicht im Lande.
\bibverse{21} Was aber mehr von Menahem zu sagen ist, und alles, was er
getan hat, siehe, das ist geschrieben in der Chronik der Könige Israels.
\bibverse{22} Und Menahem entschlief mit seinen Vätern; und Pekahja,
sein Sohn, ward König an seiner Statt. \bibverse{23} Im fünfzigsten Jahr
Asarjas, des Königs Judas, ward König Pekahja, der Sohn Menahems, über
Israel zu Samaria zwei Jahre. \bibverse{24} Und tat, das dem HErrn übel
gefiel, denn er ließ nicht von der Sünde Jerobeams, des Sohns Nebats,
der Israel sündigen machte. \bibverse{25} Und es machte Pekah, der Sohn
Remaljas, seines Ritters, einen Bund wider ihn und schlug ihn zu Samaria
im Palast des Königshauses, mit Argob und Arne und fünfzig Mann mit ihm
von den Kindern Gileads, und tötete ihn; und ward König an seiner Statt.
\bibverse{26} Was aber mehr von Pekahja zu sagen ist, und alles, was er
getan hat, siehe, das ist geschrieben in der Chronik der Könige Israels.
\bibverse{27} Im zweiundfünfzigsten Jahr Asarjas, des Königs Judas, ward
König Pekah, der Sohn Remaljas, über Israel zu Samaria zwanzig Jahre.
\bibverse{28} Und tat, das dem HErrn übel gefiel; denn er ließ nicht von
der Sünde Jerobeams, des Sohns Nebats, der Israel sündigen machte.
\bibverse{29} Zu den Zeiten Pekahs, des Königs Israels, kam
Thiglath-Pilesser, der König zu Assyrien, und nahm Ijon,
Abel-Beth-Maecha, Janoah, Kedes, Hazor, Gilead, Galiläa und das ganze
Land Naphthali und führete sie weg nach Assyrien. \bibverse{30} Und
Hosea, der Sohn Elas, machte einen Bund wider Pekah, den Sohn Remaljas,
und schlug ihn tot; und ward König an seiner Statt im zwanzigsten Jahr
Jothams, des Sohns Usias. \bibverse{31} Was aber mehr von Pekah zu sagen
ist, und alles, was er getan hat, siehe, das ist geschrieben in der
Chronik der Könige Israels. \bibverse{32} Im andern Jahr Pekahs, des
Sohns Remaljas, des Königs Israels, ward König Jotham, der Sohn Usias,
des Königs Judas. \bibverse{33} Und war fünfundzwanzig Jahre alt, da er
König ward, und regierete sechzehn Jahre zu Jerusalem. Seine Mutter hieß
Jerusa, eine Tochter Zadoks. \bibverse{34} Und tat, das dem HErrn
wohlgefiel, allerdinge wie sein Vater Usia getan hatte, \bibverse{35}
ohne daß sie die Höhen nicht abtaten; denn das Volk opferte und
räucherte noch auf den Höhen. Er bauete das hohe Tor am Hause des HErrn.
\bibverse{36} Was aber mehr von Jotham zu sagen ist, und alles, was er
getan hat, siehe, das ist geschrieben in der Chronik der Könige Judas.
\bibverse{37} Zu der Zeit hub der HErr an zu senden in Juda Rezin, den
König zu Syrien, und Pekah, den Sohn Remaljas. \bibverse{38} Und Jotham
entschlief mit seinen Vätern und ward begraben bei seinen Vätern in der
Stadt Davids, seines Vaters; und Ahas, sein Sohn, ward König an seiner
Statt.

\hypertarget{section-15}{%
\section{16}\label{section-15}}

\bibverse{1} Im siebenzehnten Jahr Pekahs, des Sohns Remaljas, ward
König Ahas, der Sohn Jothams, des Königs Judas. \bibverse{2} Zwanzig
Jahre war Ahas alt, da er König ward, und regierete sechzehn Jahre zu
Jerusalem; und tat nicht, was dem HErrn, seinem GOtt wohlgefiel, wie
sein Vater David. \bibverse{3} Denn er wandelte auf dem Wege der Könige
Israels. Dazu ließ er seinen Sohn durchs Feuer gehen nach den Greueln
der Heiden, die der HErr vor den Kindern Israel vertrieben hatte.
\bibverse{4} Und tat Opfer und räucherte auf den Höhen und auf den
Hügeln und unter allen grünen Bäumen. \bibverse{5} Dazumal zog Rezin,
der König zu Syrien, und Pekah, der Sohn Remaljas, König in Israel,
hinauf gen Jerusalem, zu streiten, und belagerten Ahas; aber sie konnten
sie nicht gewinnen. \bibverse{6} Zur selbigen Zeit brachte Rezin, König
in Syrien, Elath wieder an Syrien und stieß die Juden aus Elath; aber
die Syrer kamen und wohneten drinnen bis auf diesen Tag. \bibverse{7}
Aber Ahas sandte Boten zu Thiglath-Pilesser, dem Könige zu Assyrien, und
ließ ihm sagen: Ich bin dein Knecht und dein Sohn; komm herauf und hilf
mir aus der Hand des Königs zu Syrien und des Königs Israels, die sich
wider mich haben aufgemacht. \bibverse{8} Und Ahas nahm das Silber und
Gold, das in dem Hause des HErrn und in den Schätzen des Königshauses
funden ward, und sandte dem Könige zu Assyrien Geschenke. \bibverse{9}
Und der König zu Assyrien gehorchte ihm und zog herauf gen Damaskus und
gewann sie; und führete sie weg gen Kir und tötete Rezin. \bibverse{10}
Und der König Ahas zog entgegen Thiglath-Pilesser, dem Könige zu
Assyrien, gen Damaskus. Und da er einen Altar sah, der zu Damaskus war,
sandte der König Ahas desselben Altars Ebenbild und Gleichnis zum
Priester Uria, wie derselbe gemacht war. \bibverse{11} Und Uria, der
Priester, bauete einen Altar und machte ihn, wie der König Ahas zu ihm
gesandt hatte von Damaskus, bis der König Ahas von Damaskus kam.
\bibverse{12} Und da der König von Damaskus kam und den Altar sah,
opferte er drauf. \bibverse{13} Und zündete drauf an sein Brandopfer,
Speisopfer und goß drauf seine Trankopfer und ließ das Blut der
Dankopfer, die er opferte, auf den Altar sprengen. \bibverse{14} Aber
den ehernen Altar, der vor dem HErrn stund, tat er weg, daß er nicht
stünde zwischen dem Altar und dem Hause des HErrn, sondern setzte ihn an
die Ecke des Altars gegen Mitternacht. \bibverse{15} Und der König Ahas
gebot Uria, dem Priester, und sprach: Auf dem großen Altar sollst du
anzünden die Brandopfer des Morgens und die Speisopfer des Abends und
die Brandopfer des Königs und sein Speisopfer und die Brandopfer alles
Volks im Lande samt ihrem Speisopfer und Trankopfer und alles Blut der
Brandopfer, und das Blut aller andern Opfer sollst du drauf sprengen;
aber mit dem ehernen Altar will ich denken, was ich mache. \bibverse{16}
Uria, der Priester, tat alles, was ihn der König Ahas hieß.
\bibverse{17} Und der König Ahas brach ab die Seiten an den Gestühlen
und tat die Kessel oben davon; und das Meer tat er von den ehernen
Ochsen, die drunter waren, und setzte es auf das steinerne Pflaster.
\bibverse{18} Dazu die Decke des Sabbats, die sie am Hause gebauet
hatten, und den Gang des Königs außen wandte er zum Hause des HErrn, dem
Könige zu Assyrien zu Dienst. \bibverse{19} Was aber mehr von Ahas zu
sagen ist, das er getan hat, siehe, das ist geschrieben in der Chronik
der Könige Judas. \bibverse{20} Und Ahas entschlief mit seinen Vätern
und ward begraben bei seinen Vätern in der Stadt Davids. Und Hiskia,
sein Sohn, ward König an seiner Statt.

\hypertarget{section-16}{%
\section{17}\label{section-16}}

\bibverse{1} Im zwölften Jahr Ahas, des Königs Judas, ward König über
Israel zu Samaria Hosea, der Sohn Elas, neun Jahre. \bibverse{2} Und
tat, das dem HErrn übel gefiel, doch nicht wie die Könige Israels, die
vor ihm waren. \bibverse{3} Wider denselben zog herauf Salmanasser, der
König zu Assyrien. Und Hosea ward ihm untertan, daß er ihm Geschenke
gab. \bibverse{4} Da aber der König zu Assyrien inne ward, daß Hosea
einen Bund anrichtete und Boten hatte zu So, dem Könige in Ägypten,
gesandt, und nicht darreichte Geschenke dem Könige zu Assyrien alle
Jahre, belagerte er ihn und legte ihn ins Gefängnis. \bibverse{5} Und
der König zu Assyrien zog aufs ganze Land und gen Samaria und belagerte
sie drei Jahre. \bibverse{6} Und im neunten Jahr Hoseas gewann der König
zu Assyrien Samaria und führete Israel weg nach Assyrien und setzte sie
zu Halah und zu Habor, am Wasser Gosan, und in den Städten der Meder.
\bibverse{7} Denn da die Kinder Israel wider den HErrn, ihren GOtt,
sündigten (der sie aus Ägyptenland geführet hatte, aus der Hand Pharaos,
des Königs in Ägypten) und andere Götter fürchteten \bibverse{8} und
wandelten nach der Heiden Weise, die der HErr vor den Kindern Israel
vertrieben hatte, und wie die Könige Israels taten; \bibverse{9} und die
Kinder Israel schmückten ihre Sachen wider den HErrn, ihren GOtt, die
doch nicht gut waren, nämlich daß sie ihnen Höhen baueten in allen
Städten, beide in Schlössern und festen Städten, \bibverse{10} und
richteten Säulen auf und Haine auf allen hohen Hügeln und unter allen
grünen Bäumen; \bibverse{11} und räucherten daselbst auf allen Höhen,
wie die Heiden, die der HErr vor ihnen weggetrieben hatte, und trieben
böse Stücke, damit sie den HErrn erzürneten; \bibverse{12} und dieneten
den Götzen, davon der HErr zu ihnen gesagt hatte: Ihr sollt solches
nicht tun; \bibverse{13} und wenn der HErr bezeugte in Israel und Juda
durch alle Propheten und Schauer und ließ ihnen sagen: Kehret um von
euren bösen Wegen und haltet meine Gebote und Rechte nach allem Gesetz,
das ich euren Vätern geboten habe und das ich zu euch gesandt habe durch
meine Knechte, die Propheten, \bibverse{14} so gehorchten sie nicht,
sondern härteten ihren Nacken, wie der Nacken ihrer Väter, die nicht
glaubeten an den HErrn ihren GOtt; \bibverse{15} dazu verachteten sie
seine Gebote und seinen Bund, den er mit ihren Vätern gemacht hatte, und
seine Zeugnisse, die er unter ihnen tat, sondern wandelten ihrer
Eitelkeit nach und wurden eitel den Heiden nach, die um sie her
wohneten, von welchen ihnen der HErr geboten hatte, sie sollten nicht
wie sie tun; \bibverse{16} aber sie verließen alle Gebote des HErrn,
ihres GOttes, und machten ihnen zwei gegossene Kälber und Haine; und
beteten an alle Heere des Himmels und dieneten Baal; \bibverse{17} und
ließen ihre Söhne und Töchter durchs Feuer gehen und gingen mit
Weissagen und Zaubern um; und übergaben sich zu tun, das dem HErrn übel
gefiel, ihn zu erzürnen; \bibverse{18} da ward der HErr sehr zornig über
Israel und tat sie von seinem Angesicht, daß nichts überblieb denn der
Stamm Juda alleine. \bibverse{19} Dazu hielt auch Juda nicht die Gebote
des HErrn, ihres GOttes, und wandelten nach den Sitten Israels, die sie
getan hatten. \bibverse{20} Darum verwarf der HErr allen Samen Israels
und drängete sie und gab sie in die Hände der Räuber, bis daß er sie
verwarf von seinem Angesicht. \bibverse{21} Denn Israel ward gerissen
vom Hause Davids; und sie machten zum Könige Jerobeam, den Sohn Nebats.
Derselbe wandte Israel hinten ab vom HErrn und machte, daß sie
schwerlich sündigten. \bibverse{22} Also wandelten die Kinder Israel in
allen Sünden Jerobeams, die er angerichtet hatte, und ließen nicht
davon, \bibverse{23} bis der HErr Israel von seinem Angesicht tat, wie
er geredet hatte durch alle seine Knechte, die Propheten. Also ward
Israel aus seinem Lande weggeführet nach Assyrien bis auf diesen Tag.
\bibverse{24} Der König aber zu Assyrien ließ kommen von Babel, von
Kutha, von Ava, von Hemath und Sepharvaim und besetzte die Städte in
Samaria anstatt der Kinder Israel. Und sie nahmen Samaria ein und
wohneten in derselben Städten. \bibverse{25} Da sie aber anhuben,
daselbst zu wohnen, und den HErrn nicht fürchteten, sandte der HErr
Löwen unter sie, die erwürgeten sie. \bibverse{26} Und sie ließen dem
Könige zu Assyrien sagen: Die Heiden, die du hast hergebracht und die
Städte Samarias damit besetzt, wissen nichts von der Weise des GOttes im
Lande; darum hat er Löwen unter sie gesandt, und siehe, dieselben töten
sie, weil sie nicht wissen um die Weise des GOttes im Lande.
\bibverse{27} Der König zu Assyrien gebot und sprach: Bringet dahin der
Priester einen, die von dannen sind weggeführet, und ziehet hin und
wohnet daselbst; und er lehre sie die Weise des GOttes im Lande.
\bibverse{28} Da kam der Priester einer, die von Samaria weggeführet
waren, und setzte sich zu Bethel und lehrete sie, wie sie den HErrn
fürchten sollten. \bibverse{29} Aber ein jeglich Volk machte seinen Gott
und taten sie in die Häuser auf den Höhen, die die Samariter machten,
ein jeglich Volk in ihren Städten, darinnen sie wohneten. \bibverse{30}
Die von Babel machten Suchoth-Benoth. Die von Chuth machten Nergel. Die
von Hemath machten Asima. \bibverse{31} Die von Ava machten Nibehas und
Tharthak. Die von Sepharvaim verbrannten ihre Söhne dem Adramelech und
Anamelech, den Göttern derer von Sepharvaim. \bibverse{32} Und weil sie
den HErrn auch fürchteten, machten sie ihnen Priester auf den Höhen aus
den Untersten unter ihnen und taten sie in die Häuser auf den Höhen.
\bibverse{33} Also fürchteten sie den HErrn und dieneten auch den
Göttern nach eines jeglichen Volks Weise, von dannen sie hergebracht
waren. \bibverse{34} Und bis auf diesen Tag tun sie nach der alten
Weise, daß sie weder den HErrn fürchten noch ihre Sitten und Rechte tun
nach dem Gesetz und Gebot, das der HErr geboten hat den Kindern Jakobs,
welchem er den Namen Israel gab, \bibverse{35} und machte einen Bund mit
ihnen und gebot ihnen und sprach: Fürchtet keine andern Götter und betet
sie nicht an und dienet ihnen nicht und opfert ihnen nicht,
\bibverse{36} sondern den HErrn, der euch aus Ägyptenland geführet hat
mit großer Kraft und ausgerecktem Arm, den fürchtet, den betet an und
dem opfert; \bibverse{37} und die Sitten, Rechte, Gesetze und Gebote,
die er euch hat beschreiben lassen, die haltet, daß ihr danach tut
allewege und nicht andere Götter fürchtet; \bibverse{38} und des Bundes,
den er mit euch gemacht hat, vergesset nicht, daß ihr nicht andere
Götter fürchtet, \bibverse{39} sondern fürchtet den HErrn, euren GOtt,
der wird euch erretten von allen euren Feinden. \bibverse{40} Aber diese
gehorchten nicht, sondern taten nach ihrer vorigen Weise. \bibverse{41}
Also fürchteten diese Heiden den HErrn und dieneten auch ihren Götzen.
Also taten auch ihre Kinder und Kindeskinder, wie ihre Väter getan
haben, bis auf diesen Tag.

\hypertarget{section-17}{%
\section{18}\label{section-17}}

\bibverse{1} Im dritten Jahr Hoseas, des Sohns Elas, des Königs Israels,
ward König Hiskia, der Sohn Ahas des Königs Judas. \bibverse{2} Und war
fünfundzwanzig Jahre alt, da er König ward, und regierete neunundzwanzig
Jahre zu Jerusalem. Seine Mutter hieß Abi, eine Tochter Sacharjas.
\bibverse{3} Und tat, was dem HErrn wohlgefiel, wie sein Vater David.
\bibverse{4} Er tat ab die Höhen und zerbrach die Säulen und rottete die
Haine aus und zerstieß die eherne Schlange, die Mose gemacht hatte; denn
bis zu der Zeit hatten ihr die Kinder Israel geräuchert, und man hieß
sie Nehusthan. \bibverse{5} Er vertrauete dem HErrn, dem GOtt Israels,
daß nach ihm seinesgleichen nicht war unter allen Königen Judas, noch
vor ihm gewesen. \bibverse{6} Er hing dem HErrn an und wich nicht hinten
von ihm ab und hielt seine Gebote, die der HErr Mose geboten hatte.
\bibverse{7} Und der HErr war mit ihm; und wo er auszog, handelte er
klüglich. Dazu ward er abtrünnig vom Könige zu Assyrien und war ihm
nicht untertan. \bibverse{8} Er schlug auch die Philister bis gen Gasa
und ihre Grenze, von den Schlössern an bis an die festen Städte.
\bibverse{9} Im vierten Jahr Hiskias, des Königs Judas (das war das
siebente Jahr Hoseas, des Sohns Elas, des Königs Israels), da zog
Salmanesser, der König zu Assyrien, herauf wider Samaria und belagerte
sie; \bibverse{10} und gewann sie nach dreien Jahren, im sechsten Jahr
Hiskias; das ist, im neunten Jahr Hoseas, des Königs Israels, da ward
Samaria gewonnen. \bibverse{11} Und der König zu Assyrien führete Israel
weg gen Assyrien und setzte sie zu Halah und Habor, am Wasser Gosan, und
in die Städte der Meder, \bibverse{12} darum daß sie nicht gehorchet
hatten der Stimme des HErrn, ihres GOttes, und übergangen hatten seinen
Bund und alles, was Mose, der Knecht des HErrn, geboten hatte; der
hatten sie keinem gehorchet noch getan. \bibverse{13} Im vierzehnten
Jahr aber des Königs Hiskia zog herauf Sanherib, der König zu Assyrien,
wider alle festen Städte Judas und nahm sie ein. \bibverse{14} Da sandte
Hiskia, der König Judas, zum Könige von Assyrien gen Lachis und ließ ihm
sagen: Ich habe mich versündiget, kehre um von mir; was du mir
auflegest, will ich tragen. Da legte der König von Assyrien auf Hiskia,
dem Könige Judas, dreihundert Zentner Silbers und dreißig Zentner
Goldes. \bibverse{15} Also gab Hiskia all das Silber, das im Hause des
HErrn und in den Schätzen des Königshauses funden ward. \bibverse{16}
Zur selbigen Zeit zerbrach Hiskia, der König Judas, die Türen am Tempel
des HErrn und die Bleche, die er selbst hatte überziehen lassen, und gab
sie dem Könige von Assyrien. \bibverse{17} Und der König von Assyrien
sandte Tharthan und den Erzkämmerer und den Rabsake von Lachis zum
Könige Hiskia mit großer Macht gen Jerusalem; und sie zogen herauf. Und
da sie hinkamen, hielten sie an der Wassergrube bei dem obern Teich, der
da liegt an der Straße auf dem Acker des Walkmüllers. \bibverse{18} Und
rief dem Könige. Da kam heraus zu ihnen Eliakim, der Sohn Hilkias, der
Hofmeister, und Sebena, der Schreiber, und Joah, der Sohn Assaphs, der
Kanzler. \bibverse{19} Und der Erzschenke sprach zu ihnen: Lieber, sagt
dem Könige Hiskia: So spricht der große König, der König von Assyrien:
Was ist das für ein Trotz, darauf du dich verlässest? \bibverse{20}
Meinest du, es sei noch Rat und Macht zu streiten? Worauf verlässest du
denn nun dich, daß du abtrünnig von mir bist worden? \bibverse{21}
Siehe, verlässest du dich auf diesen zerstoßenen Rohrstab, auf Ägypten?
welcher, so sich jemand drauf lehnet, wird er ihm in die Hand gehen und
sie durchbohren. Also ist Pharao, der König in Ägypten, allen, die sich
auf ihn verlassen. \bibverse{22} Ob ihr aber wolltet zu mir sagen: Wir
verlassen uns auf den HErrn, unsern GOtt, ist's denn nicht der, des
Höhen und Altäre Hiskia hat abgetan und gesagt zu Juda und Jerusalem:
Vor diesem Altar, der zu Jerusalem ist, sollt ihr anbeten? \bibverse{23}
Nun gelobe meinem Herrn, dem Könige von Assyrien; ich will dir
zweitausend Rosse geben, daß du mögest Reiter dazu geben. \bibverse{24}
Wie willst du denn bleiben vor dem geringsten Herrn, einem meines Herrn
Untertanen, und verlässest dich auf Ägypten um der Wagen und Reiter
willen? \bibverse{25} Meinest du aber, ich sei ohne den HErrn
heraufgezogen, daß ich diese Stätte verderbete? Der HErr hat mich's
geheißen: Zeuch hinauf in dies Land und verderbe es! \bibverse{26} Da
sprach Eliakim, der Sohn Hilkias, und Sebena und Joah zum Erzschenken:
Rede mit deinen Knechten auf syrisch, denn wir verstehen es; und rede
nicht mit uns auf jüdisch vor den Ohren des Volks, das auf der Mauer
ist. \bibverse{27} Aber der Erzschenke sprach zu ihnen: Hat mich denn
mein Herr zu deinem Herrn oder zu dir gesandt, daß ich solche Worte
rede? Ja zu den Männern, die auf der Mauer sitzen, daß sie mit euch
ihren eigenen Mist fressen und ihren Harn saufen. \bibverse{28} Also
stund der Erzschenke und rief mit lauter Stimme auf jüdisch; und redete
und sprach: Höret das Wort des großen Königs, des Königs von Assyrien!
\bibverse{29} So spricht der König: Laßt euch Hiskia nicht aufsetzen;
denn er vermag euch nicht zu erretten von meiner Hand. \bibverse{30} Und
laßt euch Hiskia nicht vertrösten auf den HErrn, daß er saget Der HErr
wird uns erretten, und diese Stadt wird nicht in die Hände des Königs
von Assyrien gegeben werden! \bibverse{31} Gehorchet Hiskia nicht! Denn
so spricht der König von Assyrien: Nehmet an meine Gnade und kommet zu
mir heraus, so soll jedermann seines Weinstocks und seines Feigenbaums
essen und seines Brunnens trinken, \bibverse{32} bis ich komme und hole
euch in ein Land, das eurem Lande gleich ist, da Korn, Most, Brot,
Weinberge, Ölbäume, Öl und Honig innen ist; so werdet ihr leben bleiben
und nicht sterben. Gehorchet Hiskia nicht; denn er verführet euch, daß
er spricht: Der HErr wird uns erretten. \bibverse{33} Haben auch die
Götter der Heiden ein jeglicher sein Land errettet von der Hand des
Königs von Assyrien? \bibverse{34} Wo sind die Götter zu Hemath und
Arphad? Wo sind die Götter zu Sepharvaim, Hena und Iwa? Haben sie auch
Samaria errettet von meiner Hand? \bibverse{35} Wo ist ein Gott unter
aller Lande Göttern, die ihr Land haben von meiner Hand errettet, daß
der HErr sollte Jerusalem von meiner Hand erretten? \bibverse{36} Das
Volk aber schwieg stille und antwortete ihm nichts; denn der König hatte
geboten und gesagt: Antwortet ihm nichts! \bibverse{37} Da kam Eliakim,
der Sohn Hilkias, der Hofmeister, und Sebena, der Schreiber, und Joah,
der Sohn Assaphs, der Kanzler, zu Hiskia mit zerrissenen Kleidern und
sagten ihm an die Worte des Erzschenken.

\hypertarget{section-18}{%
\section{19}\label{section-18}}

\bibverse{1} Da der König Hiskia das hörete, zerriß er seine Kleider und
legte einen Sack an und ging in das Haus des HErrn. \bibverse{2} Und
sandte Eliakim, den Hofmeister, und Sebena, den Schreiber, samt den
ältesten Priestern, mit Säcken angetan, zu dem Propheten Jesaja, dem
Sohn Amoz. \bibverse{3} Und sie sprachen zu ihm: So sagt Hiskia: Das ist
ein Tag der Not und Scheltens und Lästerns; die Kinder sind kommen an
die Geburt, und ist keine Kraft da zu gebären. \bibverse{4} Ob
vielleicht der HErr, dein GOtt, hören wollte alle Worte des Erzschenken,
den sein Herr, der König von Assyrien, gesandt hat, Hohn zu sprechen dem
lebendigen GOtt und zu schelten mit Worten, die der HErr, dein GOtt,
gehöret hat. So hebe dein Gebet auf für die übrigen, die noch vorhanden
sind. \bibverse{5} Und da die Knechte des Königs Hiskia zu Jesaja kamen,
\bibverse{6} sprach Jesaja zu ihnen: So saget eurem Herrn: So spricht
der HErr: Fürchte dich nicht vor den Worten, die du gehöret hast, damit
mich die Knaben des Königs von Assyrien gelästert haben. \bibverse{7}
Siehe, ich will ihm einen Geist geben, daß er ein Gerücht hören wird und
wieder in sein Land ziehen; und will ihn durchs Schwert fällen in seinem
Lande. \bibverse{8} Und da der Erzschenke wiederkam, fand er den König
von Assyrien streiten wider Libna; denn er hatte gehöret, daß er von
Lachis gezogen war. \bibverse{9} Und da er hörete von Thirhaka, dem
Könige der Mohren: Siehe, er ist ausgezogen, mit dir zu streiten, wandte
er um und sandte Boten zu Hiskia und ließ ihm sagen: \bibverse{10} So
saget Hiskia, dem Könige Judas: Laß dich deinen GOtt nicht aufsetzen,
auf den du dich verlässest, und sprichst: Jerusalem wird nicht in die
Hände des Königs von Assyrien gegeben werden. \bibverse{11} Siehe, du
hast gehöret, was die Könige von Assyrien getan haben allen Landen und
sie verbannet; und du solltest errettet werden? \bibverse{12} Haben der
Heiden Götter auch sie errettet, welche meine Väter haben verderbet:
Gosan, Haran, Rezeph und die Kinder Edens, die zu Thelassar waren?
\bibverse{13} Wo ist der König zu Hemath, der König zu Arphad und der
König der Stadt Sepharvaim, Hena und Iwa? \bibverse{14} Und da Hiskia
die Briefe von den Boten empfangen und gelesen hatte, ging er hinauf zum
Hause des HErrn und breitete sie aus vor dem HErrn. \bibverse{15} Und
betete vor dem HErrn und sprach: HErr, GOtt Israels, der du über
Cherubim sitzest, du bist allein GOtt unter allen Königreichen auf
Erden, du hast Himmel und Erde gemacht. \bibverse{16} HErr, neige deine
Ohren und höre, tu deine Augen auf und siehe, und höre die Worte
Sanheribs, der hergesandt hat, Hohn zu sprechen dem lebendigen GOtt.
\bibverse{17} Es ist wahr, HErr, die Könige von Assyrien haben die
Heiden mit dem Schwert umgebracht und ihr Land \bibverse{18} und haben
ihre Götter ins Feuer geworfen. Denn es waren nicht Götter, sondern
Menschenhände Werk, Holz und Steine; darum haben sie sie umgebracht.
\bibverse{19} Nun aber, HErr, unser GOtt, hilf uns aus seiner Hand, auf
daß alle Königreiche auf Erden erkennen, daß du, HErr, allein GOtt bist.
\bibverse{20} Da sandte Jesaja, der Sohn Amoz, zu Hiskia und ließ ihm
sagen: So spricht der HErr, der GOtt Israels: Was du zu mir gebetet hast
um Sanherib, den König von Assyrien, das habe ich gehöret. \bibverse{21}
Das ist's, das der HErr wider ihn geredet hat: Die Jungfrau, die Tochter
Zion, verachtet dich und spottet dein; die Tochter Jerusalem schüttelt
ihr Haupt dir nach. \bibverse{22} Wen hast du gehöhnet und gelästert?
Über wen hast du deine Stimme erhoben? Du hast deine Augen erhoben wider
den Heiligen in Israel. \bibverse{23} Du hast den HErrn durch deine
Boten gehöhnet und gesagt: Ich bin durch die Menge meiner Wagen auf die
Höhe der Berge gestiegen, auf den Seiten des Libanon; ich habe seine
hohen Zedern und auserlesenen Tannen abgehauen und bin kommen an die
äußerste Herberge des Waldes seines Karmels; \bibverse{24} ich habe
gegraben und ausgetrunken die fremden Wasser und habe vertrocknet mit
meinen Fußsohlen die Seen. \bibverse{25} Hast du aber nicht gehöret, daß
ich solches lange zuvor getan habe, und von Anfang habe ich's bereitet?
Nun, jetzt aber habe ich's kommen lassen, daß feste Städte würden fallen
in einen wüsten Steinhaufen, \bibverse{26} und die drinnen wohnen, matt
werden und sich fürchten und schämen müßten und werden wie das Gras auf
dem Felde und wie das grüne Kraut zum Heu auf den Dächern, das
verdorret, ehe denn es reif wird. \bibverse{27} Ich weiß dein Wohnen,
dein Aus- und Einziehen, und daß du tobest wider mich. \bibverse{28}
Weil du denn wider mich tobest, und dein Übermut vor meine Ohren
heraufkommen ist, so will ich dir einen Ring an deine Nase legen und ein
Gebiß in dein Maul und will dich den Weg wiederum führen, da du
herkommen bist. \bibverse{29} Und sei dir ein Zeichen: In diesem Jahr
iß, was zertreten ist; im andern Jahr, was selber wächst; im dritten
Jahr säet und erntet und pflanzet Weinberge und esset ihre Früchte.
\bibverse{30} Und die Tochter Juda, die errettet und überblieben ist,
wird fürder unter sich wurzeln und über sich Frucht tragen.
\bibverse{31} Denn von Jerusalem werden ausgehen, die überblieben sind,
und die Erretteten vom Berge Zion. Der Eifer des HErrn Zebaoth wird
solches tun. \bibverse{32} Darum spricht der HErr vom Könige zu Assyrien
also: Er soll nicht in diese Stadt kommen und keinen Pfeil drein
schießen und kein Schild davor kommen und soll keinen Wall drum
schütten, \bibverse{33} sondern er soll den Weg wiederum ziehen, den er
kommen ist, und soll in diese Stadt nicht kommen; der HErr sagt es.
\bibverse{34} Und ich will diese Stadt beschirmen, daß ich ihr helfe um
meinetwillen und um Davids, meines Knechts, willen. \bibverse{35} Und in
derselben Nacht fuhr aus der Engel des HErrn und schlug im Lager von
Assyrien hundertundfünfundachtzigtausend Mann. Und da sie sich des
Morgens frühe aufmachten, siehe, da lag es alles eitel tote Leichname.
\bibverse{36} Also brach Sanherib, der König von Assyrien, auf und zog
weg und kehrete um; und blieb zu Ninive. \bibverse{37} Und da er
anbetete im Hause Nisrochs, seines Gottes, schlugen ihn mit dem Schwert
Adramelech und Sarezer, seine Söhne; und sie entrannen ins Land Ararat.
Und sein Sohn Assar-Haddon ward König an seiner Statt.

\hypertarget{section-19}{%
\section{20}\label{section-19}}

\bibverse{1} Zu der Zeit ward Hiskia todkrank. Und der Prophet Jesaja,
der Sohn Amoz, kam zu ihm und sprach zu ihm: So spricht der HErr:
Beschicke dein Haus; denn du wirst sterben und nicht leben bleiben.
\bibverse{2} Er aber wandte sein Antlitz zur Wand und betete zum HErrn
und sprach: \bibverse{3} Ach, HErr, gedenke doch, daß ich vor dir
treulich gewandelt habe und mit rechtschaffenem Herzen und habe getan,
das dir wohlgefällt. Und Hiskia weinete sehr. \bibverse{4} Da aber
Jesaja noch nicht zur Stadt halb hinausgegangen war, kam des HErrn Wort
zu ihm und sprach: \bibverse{5} Kehre um und sage Hiskia, dem Fürsten
meines Volks: So spricht der HErr, der GOtt deines Vaters David: Ich
habe dein Gebet gehöret und deine Tränen gesehen. Siehe, ich. will dich
gesund machen; am dritten Tage wirst du hinauf in das Haus des HErrn
gehen. \bibverse{6} Und will fünfzehn Jahre zu deinem Leben tun und dich
und diese Stadt erretten von dem Könige zu Assyrien und diese Stadt
beschirmen um meinetwillen und um meines Knechts David willen.
\bibverse{7} Und Jesaja sprach: Bringet her ein Stück Feigen! Und da sie
die brachten, legten sie sie auf die Drüse; und er ward gesund.
\bibverse{8} Hiskia aber sprach zu Jesaja: Welches ist das Zeichen, daß
mich der HErr wird gesund machen, und ich in des HErrn Haus hinaufgehen
werde am dritten Tage? \bibverse{9} Jesaja sprach: Das Zeichen wirst du
haben vom HErrn, daß der HErr tun wird, was er geredet hat. Soll der
Schatten zehn Stufen fürder gehen, oder zehn Stufen zurückgehen?
\bibverse{10} Hiskia sprach: Es ist leicht, daß der Schatten zehn Stufen
niederwärts gehe; das will ich nicht, sondern daß er zehn Stufen hinter
sich zurückgehe. \bibverse{11} Da rief der Prophet Jesaja den HErrn an;
und der Schatten ging hinter sich zurück zehn Stufen am Zeiger Ahas die
er war niederwärts gegangen. \bibverse{12} Zu der Zeit sandte Brodach,
der Sohn Baledans, des Sohns Baledans, Königs zu Babel, Briefe und
Geschenke zu Hiskia; denn er hatte gehöret, daß Hiskia krank war
gewesen. \bibverse{13} Hiskia aber war fröhlich mit ihnen und zeigte
ihnen das ganze Schatzhaus, Silber, Gold, Spezerei und das beste Öl und
die Harnischkammer und alles, was in seinen Schätzen vorhanden war. Es
war nichts in seinem Hause und in seiner ganzen Herrschaft, das ihnen
Hiskia nicht zeigete. \bibverse{14} Da kam Jesaja, der Prophet, zu dem
Könige Hiskia und sprach zu ihm: Was haben diese Leute gesagt und woher
sind sie zu dir kommen? Hiskia sprach: Sie sind aus fernen Landen zu mir
kommen, von Babel. \bibverse{15} Er sprach: Was haben sie gesehen in
deinem Hause? Hiskia sprach: Sie haben alles gesehen, was in meinem
Hause ist; und ist nichts in meinen Schätzen, das ich nicht ihnen
gezeiget hätte. \bibverse{16} Da sprach Jesaja zu Hiskia: Höre des HErrn
Wort! \bibverse{17} Siehe, es kommt die Zeit, daß alles wird gen Babel
weggeführet werden aus deinem Hause, und was deine Väter gesammelt haben
bis auf diesen Tag; und wird nichts übergelassen werden, spricht der
HErr. \bibverse{18} Dazu der Kinder, die von dir kommen, die du zeugen
wirst, werden genommen werden, daß sie Kämmerer seien im Palast des
Königs zu Babel. \bibverse{19} Hiskia aber sprach zu Jesaja: Das ist
gut, das der HErr geredet hat. Und sprach weiter: Es wird doch Friede
und Treue sein zu meinen Zeiten. \bibverse{20} Was mehr von Hiskia zu
sagen ist, und alle seine Macht, und was er getan hat, und der Teich und
die Wasserröhren, damit er Wasser in die Stadt geleitet hat, siehe, das
ist geschrieben in der Chronik der Könige Judas. \bibverse{21} Und
Hiskia entschlief mit seinen Vätern. Und Manasse, sein Sohn, ward König
an seiner Statt.

\hypertarget{section-20}{%
\section{21}\label{section-20}}

\bibverse{1} Manasse war zwölf Jahre alt, da er König ward, und
regierete fünfundfünfzig Jahre zu Jerusalem. Seine Mutter hieß
Hephzi-Bah. \bibverse{2} Und er tat, das dem HErrn übel gefiel, nach den
Greueln der Heiden, die der HErr vor den Kindern Israel vertrieben
hatte. \bibverse{3} Und verkehrete sich und bauete die Höhen, die sein
Vater Hiskia hatte abgebracht, und richtete Baal Altäre auf und machte
Haine, wie Ahab, der König Israels, getan hatte, und betete an allerlei
Heer am Himmel und dienete ihnen. \bibverse{4} Und bauete Altäre im
Hause des HErrn, davon der HErr gesagt hatte: Ich will meinen Namen zu
Jerusalem setzen. \bibverse{5} Und er bauete allen Heeren am Himmel
Altäre, in beiden Höfen am Hause des HErrn. \bibverse{6} Und ließ seinen
Sohn durchs Feuer gehen und achtete auf Vogelgeschrei und Zeichen und
hielt Wahrsager und Zeichendeuter; und tat des viel, das dem HErrn übel
gefiel, damit er ihn erzürnete. \bibverse{7} Er setzte auch einen
Haingötzen, den er gemacht hatte, in das Haus, von welchem der HErr zu
David und zu Salomo, seinem Sohn, gesagt hatte: In diesem Hause und zu
Jerusalem, die ich erwählet habe aus allen Stämmen Israels, will ich
meinen Namen setzen ewiglich, \bibverse{8} und will den Fuß Israels
nicht mehr bewegen lassen vom Lande, das ich ihren Vätern gegeben habe,
so doch, so sie halten und tun nach allem, das ich geboten habe, und
nach allem Gesetz, das mein Knecht Mose ihnen geboten hat. \bibverse{9}
Aber sie gehorchten nicht, sondern Manasse verführete sie, daß sie ärger
taten denn die Heiden, die der HErr vor den Kindern Israel vertilget
hatte. \bibverse{10} Da redete der HErr durch seine Knechte, die
Propheten und sprach: \bibverse{11} Darum daß Manasse, der König Judas,
hat diese Greuel getan, die ärger sind denn alle Greuel, so die Amoriter
getan haben, die vor ihm gewesen sind, und hat auch Juda sündigen
gemacht mit seinen Götzen, \bibverse{12} darum spricht der HErr, der
GOtt Israels, also: Siehe, ich will Unglück über Jerusalem und Juda
bringen, daß, wer es hören wird, dem sollen seine beiden Ohren gellen.
\bibverse{13} Und will über Jerusalem die Meßschnur Samarias ziehen und
das Gewicht des Hauses Ahab; und will Jerusalem ausschütten, wie man
Schüsseln ausschüttet, und will sie umstürzen. \bibverse{14} Und ich
will etliche meines Erbteils überbleiben lassen und sie geben in die
Hände ihrer Feinde, daß sie ein Raub und Reißen werden aller ihrer
Feinde, \bibverse{15} darum daß sie getan haben, das mir übel gefällt,
und haben mich erzürnet von dem Tage an, da ihre Väter aus Ägypten
gezogen sind, bis auf diesen Tag. \bibverse{16} Auch vergoß Manasse sehr
viel unschuldig Blut, bis daß Jerusalem hie und da voll ward; ohne die
Sünde, damit er Juda sündigen machte, daß sie taten, das dem HErrn übel
gefiel. \bibverse{17} Was aber mehr von Manasse zu sagen ist, und alles,
was er getan hat, und seine Sünde, die er tat, siehe, das ist
geschrieben in der Chronik der Könige Judas. \bibverse{18} Und Manasse
entschlief mit seinen Vätern und ward begraben im Garten an seinem
Hause, nämlich im Garten Usas. Und sein Sohn Amon ward König an seiner
Statt. \bibverse{19} Zweiundzwanzig Jahre alt war Amon, da er König
ward, und regierete zwei Jahre zu Jerusalem. Seine Mutter hieß
Mesulemeth, eine Tochter Haruz von Jatba. \bibverse{20} Und tat, das dem
HErrn übel gefiel, wie sein Vater Manasse getan hatte, \bibverse{21} und
wandelte in allem Wege, den sein Vater gewandelt hatte, und dienete den
Götzen, welchen sein Vater gedienet hatte, und betete sie an.
\bibverse{22} Und verließ den HErrn, seiner Väter GOtt, und wandelte
nicht im Wege des HErrn. \bibverse{23} Und seine Knechte machten einen
Bund wider Amon und töteten den König in seinem Hause. \bibverse{24}
Aber das Volk im Lande schlug alle, die den Bund gemacht hatten wider
den König Amon. Und das Volk im Lande machte Josia, seinen Sohn, zum
Könige an seiner Statt. \bibverse{25} Was aber Amon mehr getan hat,
siehe, das ist geschrieben in der Chronik der Könige Judas.
\bibverse{26} Und man begrub ihn in seinem Grabe, im Garten Usas. Und
sein Sohn Josia ward König an seiner Statt.

\hypertarget{section-21}{%
\section{22}\label{section-21}}

\bibverse{1} Josia war acht Jahre alt, da er König ward, und regierete
einundddreißig Jahre zu Jerusalem. Seine Mutter hieß Jedida, eine
Tochter Adajas von Bazkath. \bibverse{2} Und tat, das dem HErrn
wohlgefiel; und wandelte in allem Wege seines Vaters David und wich
nicht weder zur Rechten noch zur Linken. \bibverse{3} Und im achtzehnten
Jahr des Königs Josia sandte der König hin Saphan, den Sohn Azaljas, des
Sohns Mesullams, den Schreiber, in das Haus des HErrn und sprach:
\bibverse{4} Gehe hinauf zu dem Hohenpriester Hilkia, daß man ihnen gebe
das Geld, das zum Hause des HErrn gebracht ist, das die Hüter an der
Schwelle gesammelt haben vom Volk, \bibverse{5} daß sie es geben den
Arbeitern, die bestellet sind im Hause des HErrn, und geben es den
Arbeitern am Hause des HErrn, daß sie bessern, was baufällig ist am
Hause, \bibverse{6} nämlich den Zimmerleuten und Bauleuten und Maurern,
und die da Holz und gehauene Steine kaufen sollen, das Haus zu bessern;
\bibverse{7} doch daß man keine Rechnung von ihnen nehme vom Gelde, das
unter ihre Hand getan wird, sondern daß sie es auf Glauben handeln.
\bibverse{8} Und der Hohepriester Hilkia sprach zu dem Schreiber Saphan:
Ich habe das Gesetzbuch gefunden im Hause des HErrn. Und Hilkia gab das
Buch Saphan, daß er's läse. \bibverse{9} Und Saphan, der Schreiber,
brachte es dem Könige und sagte es ihm wieder und sprach: Deine Knechte
haben das Geld zusammengestoppelt, das im Hause gefunden ist, und haben
es den Arbeitern gegeben, die bestellet sind am Hause des HErrn.
\bibverse{10} Auch sagte Saphan, der Schreiber, dem Könige und sprach:
Hilkia, der Priester, gab mir ein Buch. Und Saphan las es vor dem
Könige. \bibverse{11} Da aber der König hörete die Worte im Gesetzbuch,
zerriß er seine Kleider. \bibverse{12} Und der König gebot Hilkia, dem
Priester, und Ahikam, dem Sohn Saphans, und Achbor, dem Sohn Michajas,
und Saphan, dem Schreiber, und Asaja, dem Knechte des Königs, und
sprach: \bibverse{13} Gehet hin und fraget den HErrn für mich, für das
Volk und für ganz Juda um die Worte dieses Buchs, das gefunden ist; denn
es ist ein großer Grimm des HErrn, der über uns entbrannt ist, darum daß
unsere Väter nicht gehorchet haben den Worten dieses Buchs, daß sie
täten alles, was drinnen geschrieben ist. \bibverse{14} Da ging hin
Hilkia, der Priester, Ahikam, Achbor, Saphan und Asaja zu der Prophetin
Hulda, dem Weibe Sallums, des Sohns Thikwas, des Sohns Harhams, des
Hüters der Kleider, und sie wohnete zu Jerusalem im andern Teil; und sie
redeten mit ihr. \bibverse{15} Sie aber sprach zu ihnen: So spricht der
HErr, der GOtt Israels: Saget dem Mann, der euch zu mir gesandt hat:
\bibverse{16} So spricht der HErr: Siehe, ich will Unglück über diese
Stätte und ihre Einwohner bringen, alle Worte des Gesetzes, die der
König Juda hat lassen lesen, \bibverse{17} darum daß sie mich verlassen
und andern Göttern geräuchert haben, daß sie mich erzürneten mit allen
Werken ihrer Hände; darum wird mein Grimm sich wider diese Stätte
anzünden und nicht ausgelöschet werden. \bibverse{18} Aber dem Könige
Judas, der euch gesandt hat, den HErrn zu fragen, sollt ihr so sagen: So
spricht der HErr, der GOtt Israels: \bibverse{19} Darum daß dein Herz
erweichet ist über den Worten, die du gehöret hast, und hast dich
gedemütiget vor dem HErrn, da du höretest was ich geredet habe wider
diese Stätte und ihre Einwohner, daß sie sollen eine Verwüstung und
Fluch sein, und hast deine Kleider zerrissen und hast geweinet vor mir,
so habe ich's auch erhöret, spricht der HErr. \bibverse{20} Darum will
ich dich zu deinen Vätern sammeln, daß du mit Frieden in dein Grab
versammelt werdest, und deine Augen nicht sehen all das Unglück, das ich
über diese Stätte bringen will. Und sie sagten es dem Könige wieder.

\hypertarget{section-22}{%
\section{23}\label{section-22}}

\bibverse{1} Und der König sandte hin, und es versammelten sich zu ihm
alle Ältesten in Juda und Jerusalem. \bibverse{2} Und der König ging
hinauf ins Haus des HErrn, und alle Männer von Juda und alle Einwohner
zu Jerusalem mit ihm, Priester und Propheten und alles Volk, beide klein
und groß; und man las vor ihren Ohren alle Worte des Buchs vom Bunde,
das im Hause des HErrn gefunden war. \bibverse{3} Und der König trat an
eine Säule und machte einen Bund vor dem HErrn, daß sie sollten wandeln
dem HErrn nach und halten seine Gebote, Zeugnisse und Rechte von ganzem
Herzen und von ganzer Seele, daß sie aufrichteten die Worte dieses
Bundes, die geschrieben stunden in diesem Buch. Und alles Volk trat in
den Bund. \bibverse{4} Und der König gebot dem Hohenpriester Hilkia und
den Priestern der andern Ordnung und den Hütern an der Schwelle, daß sie
sollten aus dem Tempel des HErrn tun alles Gezeug, das dem Baal und dem
Hain und allem Heer des Himmels gemacht war. Und verbrannten sie außen
vor Jerusalem im Tal Kidron; und ihr Staub ward getragen gen Bethel.
\bibverse{5} Und er tat ab die Kamarim, welche die Könige Judas hatten
gestiftet, zu räuchern auf den Höhen in den Städten Judas und um
Jerusalem her, auch die Räucherer des Baal und der Sonne und des Mondes
und der Planeten und alles Heers am Himmel. \bibverse{6} Und ließ den
Hain aus dem Hause des HErrn führen hinaus vor Jerusalem in den Bach
Kidron, und verbrannten ihn im Bach Kidron und machte ihn zu Staub; und
warf den Staub auf die Gräber der gemeinen Leute. \bibverse{7} Und er
brach ab die Häuser der Hurer, die an dem Hause des HErrn waren,
darinnen die Weiber wirkten Häuser zum Hain. \bibverse{8} Und er ließ
kommen alle Priester aus den Städten Judas und verunreinigte die Höhen,
da die Priester räucherten, von Geba an bis gen Berseba; und brach ab
die Höhen in den Toren, die in der Tür des Tors waren Josuas, des
Stadtvogts, welches war zur Linken, wenn man zum Tor der Stadt gehet.
\bibverse{9} Doch hatten die Priester der Höhen nie geopfert auf dem
Altar des HErrn zu Jerusalem, sondern aßen des ungesäuerten Brots unter
ihren Brüdern. \bibverse{10} Er verunreinigte auch das Thopheth im Tal
der Kinder Hinnom, daß niemand seinen Sohn oder seine Tochter dem Molech
durchs Feuer ließe gehen. \bibverse{11} Und tat ab die Rosse, welche die
Könige Judas hatten der Sonne gesetzt im Eingange des HErrn Hauses an
der Kammer Nethan-Melechs, des Kämmerers, der zu Parwarim war; und die
Wagen der Sonne verbrannte er mit Feuer. \bibverse{12} Und die Altäre
auf dem Dache im Saal Ahas, die die Könige Judas gemacht hatten, und die
Altäre, die Manasse gemacht hatte in den zweien Höfen des HErrn Hauses,
brach der König ab; und lief von dannen und warf ihren Staub in den Bach
Kidron. \bibverse{13} Auch die Höhen, die vor Jerusalem waren, zur
Rechten am Berge Mashith, die Salomo, der König Israels, gebauet hatte
Asthoreth, dem Greuel von Zidon, und Kamos, dem Greuel von Moab, und
Milkom, dem Greuel der Kinder Ammon, verunreinigte der König.
\bibverse{14} Und zerbrach die Säulen und rottete aus die Haine und
füllete ihre Stätte mit Menschenknochen. \bibverse{15} Auch den Altar zu
Bethel, die Höhe, die Jerobeam gemacht hatte, der Sohn Nebats, der
Israel sündigen machte, denselben Altar brach er ab und die Höhe; und
verbrannte die Höhe und machte sie zu Staub und verbrannte den Hain.
\bibverse{16} Und Josia wandte sich und sah die Gräber, die da waren auf
dem Berge; und sandte hin und ließ die Knochen aus den Gräbern holen und
verbrannte sie auf dem Altar und verunreinigte ihn nach dem Wort des
HErrn, das der Mann GOttes ausgerufen hatte, der solches ausrief.
\bibverse{17} Und er sprach: Was ist das für ein Grabmal, das ich sehe?
Und die Leute in der Stadt sprachen zu ihm: Es ist das Grab des Mannes
GOttes, der von Juda kam und rief solches aus, das du getan hast, wider
den Altar zu Bethel. \bibverse{18} Und er sprach: Laßt ihn liegen;
niemand bewege seine Gebeine! Also wurden seine Gebeine errettet mit den
Gebeinen des Propheten, der von Samaria kommen war. \bibverse{19} Er tat
auch weg alle Häuser der Höhen in den Städten Samarias, welche die
Könige Israels gemacht hatten zu erzürnen; und tat mit ihnen allerdinge,
wie er zu Bethel getan hatte. \bibverse{20} Und er opferte alle Priester
der Höhen, die daselbst waren, auf den Altären; und verbrannte also
Menschenbeine drauf. Und kam wieder gen Jerusalem. \bibverse{21} Und der
König gebot dem Volk und sprach: Haltet dem HErrn, eurem GOtt, Passah,
wie geschrieben stehet im Buch dieses Bundes. \bibverse{22} Denn es war
kein Passah so gehalten, als dieses, von der Richter Zeit an, die Israel
gerichtet haben, und in allen Zeiten der Könige Israels und der Könige
Judas, \bibverse{23} sondern im achtzehnten Jahr des Königs Josia ward
dies Passah gehalten dem HErrn zu Jerusalem. \bibverse{24} Auch fegte
Josia aus alle Wahrsager, Zeichendeuter, Bilder und Götzen und alle
Greuel, die im Lande Juda und zu Jerusalem ersehen wurden, auf daß er
aufrichtete die Worte des Gesetzes, die geschrieben stunden im Buch, das
Hilkia, der Priester, fand im Hause des HErrn. \bibverse{25} Sein
gleichen war vor ihm kein König gewesen, der so von ganzem Herzen, von
ganzer Seele, von allen Kräften sich zum HErrn bekehrete nach allem
Gesetz Moses; und nach ihm kam sein gleichen nicht auf. \bibverse{26}
Doch kehrete sich der HErr nicht von dem Grimm seines großen Zorns,
damit er über Juda erzürnet war um aller der Reizungen willen, damit ihn
Manasse gereizet hatte. \bibverse{27} Und der HErr sprach: Ich will Juda
auch von meinem Angesicht tun, wie ich Israel weggetan habe; und will
diese Stadt verwerfen, die ich erwählet hatte, nämlich Jerusalem und das
Haus, davon ich gesagt habe: Mein Name soll daselbst sein. \bibverse{28}
Was aber mehr von Josia zu sagen ist, und alles, was er getan hat,
siehe, das ist geschrieben in der Chronik der Könige Judas.
\bibverse{29} Zu seiner Zeit zog Pharao-Necho, der König in Ägypten,
herauf wider den König von Assyrien an das Wasser Phrath. Aber der König
Josia zog ihm entgegen und starb zu Megiddo, da er ihn gesehen hatte.
\bibverse{30} Und seine Knechte führeten ihn tot von Megiddo und
brachten ihn gen Jerusalem und begruben ihn in seinem Grabe. Und das
Volk im Lande nahm Joahas, den Sohn Josias, und salbeten ihn und machten
ihn zum Könige an seines Vaters Statt. \bibverse{31} Dreiundzwanzig
Jahre war Joahas alt, da er König ward, und regierete drei Monden zu
Jerusalem. Seine Mutter hieß Hamutal, eine Tochter Jeremias von Libna.
\bibverse{32} Und tat, das dem HErrn übel gefiel, wie seine Väter getan
hatten. \bibverse{33} Aber Pharao-Necho fing ihn zu Riblath im Lande
Hemath, daß er nicht regieren sollte zu Jerusalem; und legte eine
Schätzung aufs Land, hundert Zentner Silbers und einen Zentner Goldes.
\bibverse{34} Und Pharao-Necho machte zum Könige Eliakim, den Sohn
Josias, anstatt seines Vaters Josia und wandte seinen Namen Jojakim.
Aber Joahas nahm er und brachte ihn nach Ägypten; daselbst starb er.
\bibverse{35} Und Jojakim gab das Silber und Gold Pharao; doch schätzte
er das Land, daß er solch Silber gäbe nach Befehl Pharaos; einen
jeglichen nach seinem Vermögen schätzte er am Silber und Gold unter dem
Volk im Lande, daß er dem Pharao- Necho gäbe. \bibverse{36}
Fünfundzwanzig Jahre alt war Jojakim, da er König ward, und regierete
elf Jahre zu Jerusalem. Seine Mutter hieß Sebuda, eine Tochter Pedajas
von Ruma. \bibverse{37} Und tat, das dem HErrn übel gefiel, wie seine
Väter getan hatten.

\hypertarget{section-23}{%
\section{24}\label{section-23}}

\bibverse{1} Zu seiner Zeit zog herauf Nebukadnezar der König zu Babel;
und Jojakim ward ihm untertänig drei Jahre. Und er wandte sich und ward
abtrünnig von ihm. \bibverse{2} Und der HErr ließ auf ihn Kriegsknechte
kommen aus Chaldäa, aus Syrien, aus Moab, aus den Kindern Ammon und ließ
sie in Juda kommen, daß sie ihn umbrächten, nach dem Wort des HErrn, das
er geredet hatte durch seine Knechte, die Propheten. \bibverse{3} Es
geschah aber Juda also nach dem Wort des HErrn, daß er sie von seinem
Angesicht täte, um der Sünde willen Manasses, die er getan hatte;
\bibverse{4} auch um des unschuldigen Bluts willen, das er vergoß, und
machte Jerusalem voll mit unschuldigem Blut, wollte der HErr nicht
vergeben. \bibverse{5} Was mehr zu sagen ist von Jojakim, und alles, was
er getan hat, siehe, das ist geschrieben in der Chronik der Könige
Judas. \bibverse{6} Und Jojakim entschlief mit seinen Vätern; und sein
Sohn Jojachin ward König an seiner Statt. \bibverse{7} Und der König in
Ägypten zog nicht mehr aus seinem Lande; denn der König zu Babel hatte
ihm genommen alles, was des Königs in Ägypten war, vom Bach Ägyptens an
bis an das Wasser Phrath. \bibverse{8} Achtzehn Jahre alt war Jojachin,
da er König ward, und regierete drei Monden zu Jerusalem. Seine Mutter
hieß Nehustha, eine Tochter Elnathans von Jerusalem. \bibverse{9} Und
tat, das dem HErrn übel gefiel, wie sein Vater getan hatte.
\bibverse{10} Zu der Zeit zogen herauf die Knechte Nebukadnezars, des
Königs zu Babel, gen Jerusalem und kamen an die Stadt mit Bollwerk.
\bibverse{11} Und da Nebukadnezar zur Stadt kam und seine Knechte,
belagerte er sie. \bibverse{12} Aber Jojachin, der König Judas, ging
heraus zum Könige von Babel mit seiner Mutter, mit seinen Knechten, mit
seinen Obersten und Kämmerern; und der König von Babel nahm ihn auf im
achten Jahr seines Königreichs. \bibverse{13} Und nahm von dannen heraus
alle Schätze im Hause des HErrn und im Hause des Königs und zerschlug
alle güldenen Gefäße, die Salomo, der König Israels, gemacht hatte im
Tempel des HErrn wie denn der HErr geredet hatte; \bibverse{14} und
führete weg das ganze Jerusalem, alle Obersten, alle Gewaltigen,
zehntausend Gefangene und alle Zimmerleute und alle Schmiede; und ließ
nichts übrig, denn gering Volk des Landes. \bibverse{15} Und führete weg
Jojachin gen Babel, die Mutter des Königs, die Weiber des Königs und
seine Kämmerer; dazu die Mächtigen im Lande führete er auch gefangen von
Jerusalem gen Babel, \bibverse{16} und was der besten Leute waren,
siebentausend, und die Zimmerleute und Schmiede, tausend, alle starken
Kriegsmänner; und der König von Babel brachte sie gen Babel.
\bibverse{17} Und der König von Babel machte Mathanja, seinen Vetter,
zum Könige an seiner Statt und wandelte seinen Namen Zidekia.
\bibverse{18} Einundzwanzig Jahre alt war Zidekia, da er König ward, und
regierete elf Jahre zu Jerusalem. Seine Mutter hieß Hamital, eine
Tochter Jeremias von Libna. \bibverse{19} Und er tat, das dem HErrn übel
gefiel, wie Jojakim getan hatte. \bibverse{20} Denn es geschah also mit
Jerusalem und Juda aus dem Zorn des HErrn, bis daß er sie von seinem
Angesicht würfe. Und Zidekia ward abtrünnig vom Könige zu Babel.

\hypertarget{section-24}{%
\section{25}\label{section-24}}

\bibverse{1} Und es begab sich im neunten Jahr seines Königreichs, am
zehnten Tage des zehnten Monden, kam Nebukadnezar, der König zu Babel,
mit aller seiner Macht wider Jerusalem; und sie lagerten sich wider sie
und baueten einen Schutt um sie her. \bibverse{2} Also ward die Stadt
belagert bis ins elfte Jahr des Königs Zidekia. \bibverse{3} Aber im
neunten des Monden ward der Hunger stark in der Stadt, daß das Volk des
Landes nichts zu essen hatte. \bibverse{4} Da brach man in die Stadt;
und alle Kriegsmänner flohen bei der Nacht des Weges von dem Tor
zwischen den zwo Mauern, der zu des Königs Garten gehet. Aber die
Chaldäer lagen um die Stadt. Und er floh des Weges zum blachen Felde.
\bibverse{5} Aber die Macht der Chaldäer jagten dem Könige nach und
ergriffen ihn im blachen Felde zu Jericho; und alle Kriegsleute, die bei
ihm waren, wurden von ihm zerstreuet. \bibverse{6} Sie aber griffen den
König und führeten ihn hinauf zum Könige von Babel gen Riblath; und sie
sprachen ein Urteil über ihn. \bibverse{7} Und sie schlachteten die
Kinder Zidekias vor seinen Augen und blendeten Zidekia seine Augen und
banden ihn mit Ketten und führeten ihn gen Babel. \bibverse{8} Am
siebenten Tage des fünften Monden, das ist das neunzehnte Jahr
Nebukadnezars, des Königs zu Babel, kam Nebusar-Adan, der Hofmeister,
des Königs zu Babel Knecht, gen Jerusalem \bibverse{9} und verbrannte
das Haus des HErrn und das Haus des Königs und alle Häuser zu Jerusalem
und alle großen Häuser verbrannte er mit Feuer. \bibverse{10} Und die
ganze Macht der Chaldäer, die mit dem Hofmeister war, zerbrach die
Mauern um Jerusalem her. \bibverse{11} Das andere Volk aber, das übrig
war in der Stadt und die zum Könige von Babel fielen, und den andern
Pöbel führete Nebusar-Adan, der Hofmeister, weg. \bibverse{12} Und von
den Geringsten im Lande ließ der Hofmeister Weingärtner und Ackerleute.
\bibverse{13} Aber die ehernen Säulen am Hause des HErrn und die
Gestühle und das eherne Meer, das am Hause des HErrn war, zerbrachen die
Chaldäer und führeten das Erz gen Babel. \bibverse{14} Und die Töpfe,
Schaufeln, Messer, Löffel und alle ehernen Gefäße, damit man dienete,
nahmen sie weg. \bibverse{15} Dazu nahm der Hofmeister die Pfannen und
Becken, und was gülden und silbern war, \bibverse{16} zwo Säulen, ein
Meer und die Gestühle, die Salomo gemacht hatte zum Hause des HErrn. Es
war nicht zu wägen das Erz aller dieser Gefäße. \bibverse{17} Achtzehn
Ellen hoch war eine Säule, und ihr Knauf drauf war auch ehern und drei
Ellen hoch, und die Reife und Granatäpfel an dem Knauf umher war alles
ehern. Auf die Weise war auch die andere Säule mit den Reifen.
\bibverse{18} Und der Hofmeister nahm den Priester Seraja der ersten
Ordnung und den Priester Zephanja der andern Ordnung und drei Türhüter
\bibverse{19} und einen Kämmerer aus der Stadt, der gesetzt war über die
Kriegsmänner, und fünf Männer, die stets vor dem Könige waren, die in
der Stadt funden wurden, und Sopher, den Feldhauptmann, der das Volk im
Lande kriegen lehrete, und sechzig Mann vom Volk auf dem Lande, die in
der Stadt funden wurden: \bibverse{20} diese nahm Nebusar-Adan, der
Hofmeister, und brachte sie zum Könige von Babel gen Riblath.
\bibverse{21} Und der König von Babel schlug sie tot zu Riblath im Lande
Hemath. Also ward Juda weggeführet aus seinem Lande. \bibverse{22} Aber
über das übrige Volk im Lande Juda, das Nebukadnezar, der König von
Babel, überließ, setzte er Gedalja, den Sohn Ahikams, des Sohns Saphans.
\bibverse{23} Da nun all das Kriegsvolk, Hauptleute und die Männer
höreten, daß der König von Babel Gedalja gesetzt hatte, kamen sie zu
Gedalja gen Mizpa, nämlich Ismael, der Sohn Nethanjas, und Johanan, der
Sohn Kareahs, und Seraja, der Sohn Thanhumeths, der Netophathiter, und
Jaesanja, der Sohn Maechathis, samt ihren Männern. \bibverse{24} Und
Gedalja schwur ihnen und ihren Männern und sprach zu ihnen: Fürchtet
euch nicht, untertan zu sein den Chaldäern; bleibet im Lande und seid
untertänig dem Könige von Babel, so wird's euch wohlgehen. \bibverse{25}
Aber im siebenten Monden kam Ismael, der Sohn Nethanjas, des Sohns
Elisamas, von königlichem Geschlecht, und zehn Männer mit ihm und
schlugen Gedalja tot, dazu die Juden und Chaldäer, die bei ihm waren zu
Mizpa. \bibverse{26} Da machten sich auf alles Volk, beide klein und
groß, und die Obersten des Krieges, und kamen nach Ägypten; denn sie
fürchteten sich vor den Chaldäern. \bibverse{27} Aber im
siebenunddreißigsten Jahr, nachdem Jojachin, der König Judas,
weggeführet war, am siebenundzwanzigsten Tage des zwölften Monden, hub
Evil- Merodach, der König zu Babel, im ersten Jahr seines Königreichs,
das Haupt Jojachins, des Königs Judas, aus dem Kerker hervor.
\bibverse{28} Und redete freundlich mit ihm und setzte seinen Stuhl über
die Stühle der Könige, die bei ihm waren zu Babel; \bibverse{29} und
wandelte die Kleider seines Gefängnisses; und er aß allewege vor ihm
sein Leben lang; \bibverse{30} und bestimmte ihm sein Teil, das man ihm
allewege gab vom Könige, auf einen jeglichen Tag sein ganz Leben lang.
