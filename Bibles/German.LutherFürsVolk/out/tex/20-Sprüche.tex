\hypertarget{section}{%
\section{1}\label{section}}

\bibverse{1} Dies sind die Sprüche Salomos, des Königs in Israel, des
Sohnes Davids, \footnote{\textbf{1:1} 1Kö 5,9-12} \bibverse{2} zu lernen
Weisheit und Zucht, Verstand, \bibverse{3} Klugheit, Gerechtigkeit,
Recht und Schlecht; \bibverse{4} dass die Unverständigen klug und die
Jünglinge vernünftig und vorsichtig werden. \bibverse{5} Wer weise ist,
der hört zu und bessert sich; und wer verständig ist, der lässt sich
raten, \bibverse{6} dass er verstehe die Sprüche und ihre Deutung, die
Lehre der Weisen und ihre Beispiele. \bibverse{7} Des HErrn Furcht ist
Anfang der Erkenntnis. Die Ruchlosen verachten Weisheit und Zucht.
\footnote{\textbf{1:7} Spr 9,10; Ps 111,10; Hi 28,28} \bibverse{8} Mein
Kind, gehorche der Zucht deines Vaters und verlass nicht das Gebot
deiner Mutter. \footnote{\textbf{1:8} Spr 6,20} \bibverse{9} Denn
solches ist ein schöner Schmuck deinem Haupt und eine Kette an deinem
Hals. \footnote{\textbf{1:9} Spr 4,9} \bibverse{10} Mein Kind, wenn dich
die bösen Buben locken, so folge nicht. \bibverse{11} Wenn sie sagen:
„Gehe mit uns! wir wollen auf Blut lauern und den Unschuldigen ohne
Ursache nachstellen; \bibverse{12} wir wollen sie lebendig verschlingen
wie die Hölle und die Frommen wie die, die hinunter in die Grube fahren;
\bibverse{13} wir wollen großes Gut finden; wir wollen unsere Häuser mit
Raub füllen; \bibverse{14} wage es mit uns! es soll unser aller ein
Beutel sein``: \bibverse{15} mein Kind, wandle den Weg nicht mit ihnen;
wehre deinem Fuß vor ihrem Pfad. \bibverse{16} Denn ihre Füße laufen zum
Bösen und eilen, Blut zu vergießen. \bibverse{17} Denn es ist
vergeblich, das Netz auswerfen vor den Augen der Vögel. \bibverse{18}
Sie aber lauern auf ihr eigen Blut und stellen sich selbst nach dem
Leben. \bibverse{19} Also geht es allen, die nach Gewinn geizen, dass
ihr Geiz ihnen das Leben nimmt. \bibverse{20} Die Weisheit klagt draußen
und lässt sich hören auf den Gassen; \bibverse{21} sie ruft in dem
Eingang des Tores, vorn unter dem Volk; sie redet ihre Worte in der
Stadt: \bibverse{22} Wie lange wollt ihr Unverständigen unverständig
sein und die Spötter Lust zu Spötterei haben und die Ruchlosen die Lehre
hassen? \bibverse{23} Kehret euch zu meiner Strafe. Siehe, ich will euch
heraussagen meinen Geist und euch meine Worte kundtun. \bibverse{24}
Weil ich denn rufe, und ihr weigert euch, ich recke meine Hand aus, und
niemand achtet darauf, \footnote{\textbf{1:24} Jes 65,2; Jes 65,12}
\bibverse{25} und lasst fahren allen meinen Rat und wollt meine Strafe
nicht: \bibverse{26} so will ich auch lachen in eurem Unglück und eurer
spotten, wenn da kommt, was ihr fürchtet, \bibverse{27} wenn über euch
kommt wie ein Sturm, was ihr fürchtet, und euer Unglück als ein Wetter,
wenn über euch Angst und Not kommt. \bibverse{28} Dann werden sie nach
mir rufen, aber ich werde nicht antworten; sie werden mich suchen, und
nicht finden. \footnote{\textbf{1:28} Jes 59,2; Mi 3,4} \bibverse{29}
Darum, dass sie hassten die Lehre und wollten des HErrn Furcht nicht
haben, \bibverse{30} wollten meinen Rat nicht und lästerten alle meine
Strafe: \bibverse{31} so sollen sie essen von den Früchten ihres Wesens
und ihres Rats satt werden. \bibverse{32} Was die Unverständigen
gelüstet, tötet sie, und der Ruchlosen Glück bringt sie um. \footnote{\textbf{1:32}
  Spr 8,36} \bibverse{33} Wer aber mir gehorcht, wird sicher bleiben und
genug haben und kein Unglück fürchten. \footnote{\textbf{1:33} Spr 8,34}

\hypertarget{section-1}{%
\section{2}\label{section-1}}

\bibverse{1} Mein Kind, wenn du willst meine Rede annehmen und meine
Gebote bei dir behalten, \bibverse{2} dass dein Ohr auf Weisheit achthat
und du dein Herz mit Fleiß dazu neigest; \bibverse{3} ja, wenn du mit
Fleiß darnach rufest und darum betest; \footnote{\textbf{2:3} Jak 1,5}
\bibverse{4} wenn du sie suchest wie Silber und nach ihr forschest wie
nach Schätzen: \bibverse{5} alsdann wirst du die Furcht des HErrn
verstehen und Gottes Erkenntnis finden. \bibverse{6} Denn der HErr gibt
Weisheit, und aus seinem Munde kommt Erkenntnis und Verstand.
\bibverse{7} Er lässt's den Aufrichtigen gelingen und beschirmt die
Frommen \bibverse{8} und behütet die, die recht tun, und bewahrt den Weg
seiner Heiligen. \bibverse{9} Alsdann wirst du verstehen Gerechtigkeit
und Recht und Frömmigkeit und allen guten Weg. \bibverse{10} Denn
Weisheit wird in dein Herz eingehen, dass du gerne lernest;
\bibverse{11} guter Rat wird dich bewahren, und Verstand wird dich
behüten, \bibverse{12} dass du nicht geratest auf den Weg der Bösen noch
unter die verkehrten Schwätzer, \bibverse{13} die da verlassen die
rechte Bahn und gehen finstere Wege, \bibverse{14} die sich freuen,
Böses zu tun, und sind fröhlich in ihrem bösen, verkehrten Wesen,
\bibverse{15} welche ihren Weg verkehren und folgen ihrem Abwege;
\bibverse{16} dass du nicht geratest an eines anderen Weib, an eine
Fremde, die glatte Worte gibt \footnote{\textbf{2:16} Spr 6,24; Spr 7,5;
  Spr 5,3} \bibverse{17} und verlässt den Freund ihrer Jugend und
vergisst den Bund ihres Gottes \bibverse{18} (denn ihr Haus neigt sich
zum Tod und ihre Gänge zu den Verlorenen; \bibverse{19} alle, die zu ihr
eingehen, kommen nicht wieder und ergreifen den Weg des Lebens nicht);
\bibverse{20} auf dass du wandelst auf gutem Wege und bleibest auf der
rechten Bahn. \bibverse{21} Denn die Gerechten werden im Lande wohnen,
und die Frommen werden darin bleiben; \footnote{\textbf{2:21} Ps 37,9;
  Ps 37,29; Mt 5,5} \bibverse{22} aber die Gottlosen werden aus dem
Lande ausgerottet, und die Verächter werden daraus vertilgt. \footnote{\textbf{2:22}
  Ps 37,10; Ps 37,22}

\hypertarget{section-2}{%
\section{3}\label{section-2}}

\bibverse{1} Mein Kind, vergiss meines Gesetzes nicht, und dein Herz
behalte meine Gebote. \bibverse{2} Denn sie werden dir langes Leben und
gute Jahre und Frieden bringen; \footnote{\textbf{3:2} Spr 4,10; 3Mo
  18,5} \bibverse{3} Gnade und Treue werden dich nicht lassen. Hänge sie
an deinen Hals und schreibe sie auf die Tafel deines Herzens,
\footnote{\textbf{3:3} Spr 6,21; 5Mo 6,8; Jer 31,33} \bibverse{4} so
wirst du Gunst und Klugheit finden, die Gott und Menschen gefällt.
\footnote{\textbf{3:4} Lk 2,52} \bibverse{5} Verlass dich auf den HErrn
von ganzem Herzen und verlass dich nicht auf deinen Verstand;
\bibverse{6} sondern gedenke an ihn in allen deinen Wegen, so wird er
dich recht führen. \bibverse{7} Dünke dich nicht, weise zu sein, sondern
fürchte den HErrn und weiche vom Bösen. \bibverse{8} Das wird deinem
Leibe gesund sein und deine Gebeine erquicken. \footnote{\textbf{3:8}
  Spr 4,22} \bibverse{9} Ehre den HErrn von deinem Gut und von den
Erstlingen all deines Einkommens, \footnote{\textbf{3:9} 2Mo 23,19}
\bibverse{10} so werden deine Scheunen voll werden und deine Kelter mit
Most übergehen. \bibverse{11} Mein Kind, verwirf die Zucht des HErrn
nicht und sei nicht ungeduldig über seine Strafe. \footnote{\textbf{3:11}
  Hi 5,17-19; Hebr 12,5-6} \bibverse{12} Denn welchen der HErr liebt,
den straft er, und hat doch Wohlgefallen an ihm wie ein Vater am Sohn.
\footnote{\textbf{3:12} Offb 3,19} \bibverse{13} Wohl dem Menschen, der
Weisheit findet, und dem Menschen, der Verstand bekommt! \footnote{\textbf{3:13}
  Mt 13,44} \bibverse{14} Denn es ist besser, sie zu erwerben, als
Silber; und ihr Ertrag ist besser als Gold. \footnote{\textbf{3:14} Spr
  8,10; Spr 8,19} \bibverse{15} Sie ist edler denn Perlen; und alles,
was du wünschen magst, ist ihr nicht zu vergleichen. \footnote{\textbf{3:15}
  Mt 13,45-46} \bibverse{16} Langes Leben ist zu ihrer rechten Hand; zu
ihrer Linken ist Reichtum und Ehre. \footnote{\textbf{3:16} Spr 3,2}
\bibverse{17} Ihre Wege sind liebliche Wege, und alle ihre Steige sind
Friede. \bibverse{18} Sie ist ein Baum des Lebens allen, die sie
ergreifen; und selig sind, die sie halten. \bibverse{19} Denn der HErr
hat die Erde durch Weisheit gegründet und durch seinen Rat die Himmel
bereitet. \footnote{\textbf{3:19} Spr 8,24-30} \bibverse{20} Durch seine
Weisheit sind die Tiefen zerteilt und die Wolken mit Tau triefend
gemacht. \bibverse{21} Mein Kind, lass sie nicht von deinen Augen
weichen, so wirst du glückselig und klug werden. \bibverse{22} Das wird
deiner Seele Leben sein und ein Schmuck deinem Halse. \bibverse{23} Dann
wirst du sicher wandeln auf deinem Wege, dass dein Fuß sich nicht stoßen
wird. \bibverse{24} Legst du dich, so wirst du dich nicht fürchten,
sondern süß schlafen, \footnote{\textbf{3:24} Ps 3,6; Ps 4,9}
\bibverse{25} dass du dich nicht fürchten darfst vor plötzlichem
Schrecken noch vor dem Sturm der Gottlosen, wenn er kommt. \bibverse{26}
Denn der HErr ist dein Trotz; der behütet deinen Fuß, dass er nicht
gefangen werde. \bibverse{27} Weigere dich nicht, dem Dürftigen Gutes zu
tun, so deine Hand von Gott hat, solches zu tun. \bibverse{28} Sprich
nicht zu deinem Nächsten: „Geh hin und komm wieder; morgen will ich dir
geben``, wenn du es doch wohl hast. \bibverse{29} Trachte nicht Böses
wider deinen Nächsten, der auf Treue bei dir wohnt. \bibverse{30} Hadere
nicht mit jemand ohne Ursache, wenn er dir kein Leid getan hat.
\bibverse{31} Eifere nicht einem Frevler nach und erwähle seiner Wege
keinen; \bibverse{32} denn der HErr hat Gräuel an dem Abtrünnigen, und
sein Geheimnis ist bei den Frommen. \footnote{\textbf{3:32} Ps 25,14}
\bibverse{33} Im Hause des Gottlosen ist der Fluch des HErrn; aber das
Haus der Gerechten wird gesegnet. \bibverse{34} Er wird der Spötter
spotten; aber den Elenden wird er Gnade geben. \bibverse{35} Die Weisen
werden Ehre erben; aber wenn die Narren hochkommen, werden sie doch zu
Schanden. \# 4 \bibverse{1} Höret, meine Kinder, die Zucht eures Vaters;
merkt auf, dass ihr lernet und klug werdet! \bibverse{2} Denn ich gebe
euch eine gute Lehre; verlasset mein Gesetz nicht. \bibverse{3} Denn ich
war meines Vaters Sohn, ein zarter und ein einziger vor meiner Mutter.
\bibverse{4} Und er lehrte mich und sprach: Lass dein Herz meine Worte
aufnehmen; halte meine Gebote, so wirst du leben. \footnote{\textbf{4:4}
  3Mo 18,5} \bibverse{5} Nimm an Weisheit, nimm an Verstand; vergiss
nicht und weiche nicht von der Rede meines Mundes. \footnote{\textbf{4:5}
  Spr 3,1} \bibverse{6} Verlass sie nicht, so wird sie dich bewahren;
liebe sie, so wird sie dich behüten. \bibverse{7} Denn der Weisheit
Anfang ist, wenn man sie gerne hört und die Klugheit lieber hat als alle
Güter. \bibverse{8} Achte sie hoch, so wird sie dich erhöhen, und wird
dich zu Ehren bringen, wo du sie herzest. \bibverse{9} Sie wird dein
Haupt schön schmücken und wird dich zieren mit einer prächtigen Krone.
\footnote{\textbf{4:9} Spr 1,9} \bibverse{10} So höre, mein Kind, und
nimm an meine Rede, so werden deiner Jahre viel werden. \footnote{\textbf{4:10}
  Spr 3,2} \bibverse{11} Ich will dich den Weg der Weisheit führen; ich
will dich auf rechter Bahn leiten, \footnote{\textbf{4:11} Ps 32,8; Ps
  27,11} \bibverse{12} dass, wenn du gehst, dein Gang dir nicht sauer
werde, und wenn du läufst, dass du nicht anstoßest. \bibverse{13} Fasse
die Zucht, lass nicht davon; bewahre sie, denn sie ist dein Leben.
\bibverse{14} Komm nicht auf der Gottlosen Pfad und tritt nicht auf den
Weg der Bösen. \bibverse{15} Lass ihn fahren und gehe nicht darin;
weiche von ihm und gehe vorüber. \bibverse{16} Denn sie schlafen nicht,
sie haben denn Übel getan; und ruhen nicht, sie haben den Schaden getan.
\bibverse{17} Denn sie nähren sich von gottlosem Brot und trinken vom
Wein des Frevels. \bibverse{18} Aber der Gerechten Pfad glänzt wie das
Licht, das immer heller leuchtet bis auf den vollen Tag. \bibverse{19}
Der Gottlosen Weg aber ist wie Dunkel; sie wissen nicht, wo sie fallen
werden. \footnote{\textbf{4:19} Spr 13,9; Spr 24,20} \bibverse{20} Mein
Sohn, merke auf meine Worte und neige dein Ohr zu meiner Rede.
\bibverse{21} Lass sie nicht von deinen Augen fahren, behalte sie in
deinem Herzen. \bibverse{22} Denn sie sind das Leben denen, die sie
finden, und gesund ihrem ganzen Leibe. \bibverse{23} Behüte dein Herz
mit allem Fleiß; denn daraus geht das Leben. \bibverse{24} Tue von dir
den verkehrten Mund und lass das Lästermaul ferne von dir sein.
\bibverse{25} Lass deine Augen stracks vor sich sehen und deine
Augenlider richtig vor dir hin blicken. \bibverse{26} Lass deinen Fuß
gleich vor sich gehen, so gehst du gewiss. \footnote{\textbf{4:26} Hebr
  12,13} \bibverse{27} Wanke weder zur Rechten noch zur Linken; wende
deinen Fuß vom Bösen. \footnote{\textbf{4:27} 5Mo 5,29}

\hypertarget{section-3}{%
\section{5}\label{section-3}}

\bibverse{1} Mein Kind, merke auf meine Weisheit; neige dein Ohr zu
meiner Lehre, \bibverse{2} dass du bewahrest guten Rat und dein Mund
wisse Unterschied zu halten. \bibverse{3} Denn die Lippen der Hure sind
süß wie Honigseim, und ihre Kehle ist glätter als Öl, \footnote{\textbf{5:3}
  Spr 2,16-19} \bibverse{4} aber hernach bitter wie Wermut und scharf
wie ein zweischneidiges Schwert. \bibverse{5} Ihre Füße laufen zum Tod
hinunter; ihre Gänge führen ins Grab. \bibverse{6} Sie geht nicht
stracks auf dem Wege des Lebens; unstet sind ihre Tritte, dass sie nicht
weiß, wo sie geht. \bibverse{7} So gehorchet mir nun, meine Kinder, und
weichet nicht von der Rede meines Mundes. \bibverse{8} Lass deine Wege
ferne von ihr sein, und nahe nicht zur Tür ihres Hauses, \bibverse{9}
dass du nicht den Fremden gebest deine Ehre und deine Jahre dem
Grausamen; \bibverse{10} dass sich nicht Fremde von deinem Vermögen
sättigen und deine Arbeit nicht sei in eines anderen Haus, \bibverse{11}
und müssest hernach seufzen, wenn du Leib und Gut verzehrt hast,
\bibverse{12} und sprechen: „Ach, wie habe ich die Zucht gehasst und wie
hat mein Herz die Strafe verschmäht! \bibverse{13} wie habe ich nicht
gehorcht der Stimme meiner Lehrer und mein Ohr nicht geneigt zu denen,
die mich lehrten! \bibverse{14} Ich bin schier in alles Unglück gekommen
vor allen Leuten und allem Volk.`` \bibverse{15} Trink Wasser aus deiner
Grube und Flüsse aus deinem Brunnen. \bibverse{16} Lass deine Brunnen
herausfließen und die Wasserbäche auf die Gassen. \bibverse{17} Habe du
aber sie allein, und kein Fremder mit dir. \bibverse{18} Dein Born sei
gesegnet, und freue dich des Weibes deiner Jugend. \bibverse{19} Sie ist
lieblich wie eine Hinde und holdselig wie ein Reh. Lass dich ihre Liebe
allezeit sättigen und ergötze dich allewege in ihrer Liebe.
\bibverse{20} Mein Kind, warum willst du dich an der Fremden ergötzen
und herzest eine andere? \bibverse{21} Denn jedermanns Wege sind offen
vor dem HErrn, und er misst alle ihre Gänge. \bibverse{22} Die Missetat
des Gottlosen wird ihn fangen, und er wird mit dem Strick seiner Sünde
gehalten werden. \bibverse{23} Er wird sterben, darum dass er sich nicht
will ziehen lassen; und um seiner großen Torheit willen wird's ihm nicht
wohl gehen. \# 6 \bibverse{1} Mein Kind, wirst du Bürge für deinen
Nächsten und hast deine Hand bei einem Fremden verhaftet, \bibverse{2}
so bist du verknüpft durch die Rede deines Mundes und gefangen mit den
Reden deines Mundes. \bibverse{3} So tue doch, mein Kind, also und
errette dich -- denn du bist deinem Nächsten in die Hände gekommen --:
eile, dränge und treibe deinen Nächsten. \bibverse{4} Lass deine Augen
nicht schlafen, noch deine Augenlider schlummern. \bibverse{5} Errette
dich wie ein Reh von der Hand und wie eine Vogel aus der Hand des
Voglers. \bibverse{6} Gehe hin zur Ameise, du Fauler; siehe ihre Weise
an und lerne! \footnote{\textbf{6:6} Spr 10,4; Spr 20,4} \bibverse{7} Ob
sie wohl keinen Fürsten noch Hauptmann noch Herrn hat, \bibverse{8}
bereitet sie doch ihr Brot im Sommer und sammelt ihre Speise in der
Ernte. \bibverse{9} Wie lange liegst du, Fauler? Wann willst du
aufstehen von deinem Schlaf? \bibverse{10} Ja, schlafe noch ein wenig,
schlummere ein wenig, schlage die Hände ineinander ein wenig, dass du
schlafest, \bibverse{11} so wird dich die Armut übereilen wie ein
Fußgänger und der Mangel wie ein gewappneter Mann. \bibverse{12} Ein
heilloser Mensch, ein schädlicher Mann geht mit verstelltem Munde,
\footnote{\textbf{6:12} Spr 10,31-32} \bibverse{13} winkt mit Augen,
deutet mit Füßen, zeigt mit Fingern, \footnote{\textbf{6:13} Spr 10,10}
\bibverse{14} trachtet allezeit Böses und Verkehrtes in seinem Herzen
und richtet Hader an. \bibverse{15} Darum wird ihm plötzlich sein
Verderben kommen, und er wird schnell zerbrochen werden, da keine Hilfe
dasein wird. \bibverse{16} Diese sechs Stücke hasst der HErr, und am
siebenten hat er einen Gräuel: \bibverse{17} hohe Augen, falsche Zunge,
Hände, die unschuldig Blut vergießen, \bibverse{18} Herz, das mit böser
Tücke umgeht, Füße, die behende sind, Schaden zu tun, \bibverse{19}
falscher Zeuge, der frech Lügen redet, und wer Hader zwischen Brüdern
anrichtet. \bibverse{20} Mein Kind, bewahre die Gebote deines Vaters und
lass nicht fahren das Gesetz deiner Mutter. \footnote{\textbf{6:20} Spr
  1,8} \bibverse{21} Binde sie zusammen auf dein Herz allewege und hänge
sie an deinen Hals, \footnote{\textbf{6:21} Spr 3,3} \bibverse{22} wenn
du gehst, dass sie dich geleiten; wenn du dich legst, dass sie dich
bewahren; wenn du aufwachst, dass sie zu dir sprechen. \footnote{\textbf{6:22}
  Ps 119,172} \bibverse{23} Denn das Gebot ist eine Leuchte und das
Gesetz ein Licht, und die Strafe der Zucht ist ein Weg des Lebens,
\bibverse{24} auf dass du bewahrt werdest vor dem bösen Weibe, vor der
glatten Zunge der Fremden. \bibverse{25} Lass dich ihre Schöne nicht
gelüsten in deinem Herzen und verfange dich nicht an ihren Augenlidern.
\bibverse{26} Denn eine Hure bringt einen ums Brot; aber eines anderen
Weib fängt das edle Leben. \bibverse{27} Kann auch jemand ein Feuer im
Busen behalten, dass seine Kleider nicht brennen? \bibverse{28} Wie
sollte jemand auf Kohlen gehen, dass seine Füße nicht verbrannt würden?
\bibverse{29} Also gehet's dem, der zu seines Nächsten Weib geht; es
bleibt keiner ungestraft, der sie berührt. \footnote{\textbf{6:29} Spr
  5,10-14} \bibverse{30} Es ist einem Diebe nicht so große Schmach, ob
er stiehlt, seine Seele zu sättigen, weil ihn hungert; \bibverse{31} und
ob er ergriffen wird, gibt er's siebenfältig wieder und legt dar alles
Gut in seinem Hause. \bibverse{32} Aber wer mit einem Weibe die Ehe
bricht, der ist ein Narr; der bringt sein Leben in das Verderben.
\bibverse{33} Dazu trifft ihn Plage und Schande, und seine Schande wird
nicht ausgetilgt. \bibverse{34} Denn der Grimm des Mannes eifert, und
schont nicht zur Zeit der Rache \bibverse{35} und sieht keine Person an,
die da versöhne, und nimmt's nicht an, ob du viel schenken wolltest. \#
7 \bibverse{1} Mein Kind, behalte meine Rede und verbirg meine Gebote
bei dir. \bibverse{2} Behalte meine Gebote, so wirst du leben, und mein
Gesetz wie deinen Augapfel. \bibverse{3} Binde sie an deine Finger;
schreibe sie auf die Tafel deines Herzens. \footnote{\textbf{7:3} Spr
  3,3} \bibverse{4} Sprich zur Weisheit: „Du bist meine Schwester``, und
nenne die Klugheit deine Freundin, \bibverse{5} dass du behütet werdest
vor dem fremden Weibe, vor einer anderen, die glatte Worte gibt.
\bibverse{6} Denn am Fenster meines Hauses guckte ich durchs Gitter
\bibverse{7} und sah unter den Unverständigen und ward gewahr unter den
Kindern eines törichten Jünglings, \bibverse{8} der ging auf der Gasse
an einer Ecke und trat daher auf dem Wege bei ihrem Hause, \bibverse{9}
in der Dämmerung, am Abend des Tages, da es Nacht ward und dunkel war.
\bibverse{10} Und siehe, da begegnete ihm ein Weib im Hurenschmuck,
listig, \bibverse{11} wild und unbändig, dass ihre Füße in ihrem Hause
nicht bleiben können. \bibverse{12} Jetzt ist sie draußen, jetzt auf der
Gasse, und lauert an allen Ecken. \bibverse{13} Und erwischte ihn und
küsste ihn unverschämt und sprach zu ihm: \bibverse{14} Ich habe
Dankopfer für mich heute bezahlt für meine Gelübde. \footnote{\textbf{7:14}
  3Mo 3,3-4} \bibverse{15} Darum bin herausgegangen, dir zu begegnen,
dein Angesicht zu suchen, und habe dich gefunden. \bibverse{16} Ich habe
mein Bett schön geschmückt mit bunten Teppichen aus Ägypten.
\bibverse{17} Ich habe mein Lager mit Myrrhe, Aloe und Zimt besprengt.
\bibverse{18} Komm, lass uns genug buhlen bis an den Morgen und lass uns
der Liebe pflegen. \bibverse{19} Denn der Mann ist nicht daheim; er ist
einen fernen Weg gezogen. \bibverse{20} Er hat den Geldsack mit sich
genommen; er wird erst aufs Fest wieder heimkommen. \bibverse{21} Sie
überredete ihn mit vielen Worten und gewann ihn mit ihrem glatten Munde.
\bibverse{22} Er folgt ihr alsbald nach, wie ein Ochse zur Fleischbank
geführt wird, und wie zur Fessel, womit man die Narren züchtigt,
\bibverse{23} bis sie ihm mit dem Pfeil die Leber spaltet; wie ein Vogel
zum Strick eilt und weiß nicht, dass es ihm das Leben gilt.
\bibverse{24} So gehorchet mir nun, meine Kinder, und merket auf die
Rede meines Mundes. \bibverse{25} Lass dein Herz nicht weichen auf ihren
Weg und lass dich nicht verführen auf ihrer Bahn. \bibverse{26} Denn sie
hat viele verwundet und gefällt, und sind allerlei Mächtige von ihr
erwürgt. \bibverse{27} Ihr Haus sind Wege zum Grab, da man hinunterfährt
in des Todes Kammern. \# 8 \bibverse{1} Ruft nicht die Weisheit, und die
Klugheit lässt sich hören? \footnote{\textbf{8:1} Spr 1,20-33}
\bibverse{2} Öffentlich am Wege und an der Straße steht sie.
\bibverse{3} An den Toren bei der Stadt, da man zur Tür eingeht, schreit
sie: \bibverse{4} O ihr Männer, ich schreie zu euch und rufe den Leuten.
\bibverse{5} Merkt, ihr Unverständigen, auf Klugheit und, ihr Toren,
nehmt es zu Herzen! \bibverse{6} Höret, denn ich will reden, was
fürstlich ist, und lehren, was recht ist. \bibverse{7} Denn mein Mund
soll die Wahrheit reden, und meine Lippen sollen hassen, was gottlos
ist. \bibverse{8} Alle Reden meines Mundes sind gerecht; es ist nichts
Verkehrtes noch Falsches darin. \bibverse{9} Sie sind alle gerade denen,
die sie verstehen, und richtig denen, die es annehmen wollen.
\bibverse{10} Nehmet an meine Zucht lieber denn Silber, und die Lehre
achtet höher denn köstliches Gold. \bibverse{11} Denn Weisheit ist
besser als Perlen; und alles, was man wünschen mag, kann ihr nicht
gleichen. \bibverse{12} Ich, Weisheit, wohne bei der Klugheit, und ich
weiß guten Rat zu geben. \bibverse{13} Die Furcht des HErrn hasst das
Arge, die Hoffart, den Hochmut und bösen Weg; und ich bin feind dem
verkehrten Mund. \footnote{\textbf{8:13} Spr 6,12-19} \bibverse{14} Mein
ist beides, Rat und Tat; ich habe Verstand und Macht. \bibverse{15}
Durch mich regieren die Könige und setzen die Ratsherren das Recht.
\bibverse{16} Durch mich herrschen die Fürsten und alle Regenten auf
Erden. \bibverse{17} Ich liebe, die mich lieben; und die mich frühe
suchen, finden mich. \bibverse{18} Reichtum und Ehre ist bei mir,
währendes Gut und Gerechtigkeit. \bibverse{19} Meine Frucht ist besser
denn Gold und feines Gold und mein Ertrag besser denn auserlesenes
Silber. \bibverse{20} Ich wandle auf dem rechten Wege, auf der Straße
des Rechts, \bibverse{21} dass ich wohl versorge, die mich lieben, und
ihre Schätze vollmache. \bibverse{22} Der HErr hat mich gehabt im Anfang
seiner Wege; ehe er etwas schuf, war ich da. \footnote{\textbf{8:22} Hi
  28,27} \bibverse{23} Ich bin eingesetzt von Ewigkeit, von Anfang, vor
der Erde. \bibverse{24} Da die Tiefen noch nicht waren, da war ich schon
geboren, da die Brunnen noch nicht mit Wasser quollen. \bibverse{25} Ehe
denn die Berge eingesenkt waren, vor den Hügeln war ich geboren,
\bibverse{26} da er die Erde noch nicht gemacht hatte und was darauf
ist, noch die Berge des Erdbodens. \bibverse{27} Da er die Himmel
bereitete, war ich daselbst, da er die Tiefe mit seinem Ziel fasste.
\bibverse{28} Da er die Wolken droben festete, da er festigte die
Brunnen der Tiefe, \bibverse{29} da er dem Meer das Ziel setzte und den
Wassern, dass sie nicht überschreiten seinen Befehl, da er den Grund der
Erde legte: \footnote{\textbf{8:29} Hi 38,10-11; Ps 104,9} \bibverse{30}
da war ich der Werkmeister bei ihm und hatte meine Lust täglich und
spielte vor ihm allezeit \bibverse{31} und spielte auf seinem Erdboden,
und meine Lust ist bei den Menschenkindern. \bibverse{32} So gehorchet
mir nun, meine Kinder. Wohl denen, die meine Wege halten! \bibverse{33}
Höret die Zucht und werdet weise und lasset sie nicht fahren.
\bibverse{34} Wohl dem Menschen, der mir gehorcht, dass er wache an
meiner Tür täglich, dass er warte an den Pfosten meiner Tür.
\bibverse{35} Wer mich findet, der findet das Leben und wird
Wohlgefallen vom HErrn erlangen. \footnote{\textbf{8:35} Spr 3,2}
\bibverse{36} Wer aber an mir sündigt, der verletzt seine Seele. Alle,
die mich hassen, lieben den Tod. \# 9 \bibverse{1} Die Weisheit baute
ihr Haus und hieb sieben Säulen, \bibverse{2} schlachtete ihr Vieh und
trug ihren Wein auf und bereitete ihren Tisch \bibverse{3} und sandte
ihre Dirnen aus, zu rufen oben auf den Höhen der Stadt: \bibverse{4}
„Wer verständig ist, der mache sich hierher!{}``, und zum Narren sprach
sie: \bibverse{5} „Kommet, zehret von meinem Brot und trinket den Wein,
den ich schenke; \bibverse{6} verlasset das unverständige Wesen, so
werdet ihr leben, und gehet auf dem Wege der Klugheit.`` \footnote{\textbf{9:6}
  Spr 1,22} \bibverse{7} Wer den Spötter züchtigt, der muss Schande auf
sich nehmen; und wer den Gottlosen straft, der muss gehöhnt werden.
\bibverse{8} Strafe den Spötter nicht, er hasst dich; strafe den Weisen,
der wird dich lieben. \bibverse{9} Gib dem Weisen, so wird er noch
weiser werden; lehre den Gerechten, so wird er in der Lehre zunehmen.
\bibverse{10} Der Weisheit Anfang ist des HErrn Furcht, und den Heiligen
erkennen ist Verstand. \footnote{\textbf{9:10} Spr 1,7} \bibverse{11}
Denn durch mich werden deiner Tage viel werden und werden dir der Jahre
des Lebens mehr werden. \footnote{\textbf{9:11} Spr 3,2; Spr 3,16}
\bibverse{12} Bist du weise, so bist du dir weise; bist du ein Spötter,
so wirst du es allein tragen. \bibverse{13} Es ist aber ein törichtes,
wildes Weib, voll Schwätzens, und weiß nichts; \bibverse{14} die sitzt
in der Tür ihres Hauses auf dem Stuhl, oben in der Stadt, \bibverse{15}
zu laden alle, die vorübergehen und richtig auf ihrem Wege wandeln:
\bibverse{16} „Wer unverständig ist, der mache sich hierher!{}``, und
zum Narren spricht sie: \bibverse{17} „Die gestohlenen Wasser sind süß,
und das verborgene Brot schmeckt wohl.`` \footnote{\textbf{9:17} Spr
  20,17} \bibverse{18} Er weiß aber nicht, dass daselbst Tote sind und
ihre Gäste in der tiefen Grube. \# 10 \bibverse{1} Dies sind die Sprüche
Salomos. Ein weiser Sohn ist seines Vaters Freude; aber ein törichter
Sohn ist seiner Mutter Grämen. \bibverse{2} Unrecht Gut hilft nicht;
aber Gerechtigkeit errettet vor dem Tode. \bibverse{3} Der HErr lässt
die Seele des Gerechten nicht Hunger leiden; er stößt aber weg der
Gottlosen Begierde. \footnote{\textbf{10:3} Ps 37,19; Ps 37,25}
\bibverse{4} Lässige Hand macht arm; aber der Fleißigen Hand macht
reich. \footnote{\textbf{10:4} Spr 6,6-11; Spr 12,24; Spr 12,27; Spr
  19,15; Spr 28,19} \bibverse{5} Wer im Sommer sammelt, der ist klug;
wer aber in der Ernte schläft, wird zu Schanden. \bibverse{6} Den Segen
hat das Haupt des Gerechten; aber den Mund der Gottlosen wird ihr Frevel
überfallen. \bibverse{7} Das Gedächtnis der Gerechten bleibt im Segen;
aber der Gottlosen Name wird verwesen. \footnote{\textbf{10:7} Hi 18,17;
  Ps 9,6} \bibverse{8} Wer weise von Herzen ist, nimmt die Gebote an;
wer aber ein Narrenmaul hat, wird geschlagen. \bibverse{9} Wer
unschuldig lebt, der lebt sicher; wer aber verkehrt ist auf seinen
Wegen, wird offenbar werden. \bibverse{10} Wer mit Augen winkt, wird
Mühsal anrichten; und der ein Narrenmaul hat, wird geschlagen.
\bibverse{11} Des Gerechten Mund ist ein Brunnen des Lebens; aber den
Mund der Gottlosen wird ihr Frevel überfallen. \footnote{\textbf{10:11}
  Spr 10,31; Spr 13,14} \bibverse{12} Hass erregt Hader; aber Liebe
deckt zu alle Übertretungen. \footnote{\textbf{10:12} 1Petr 4,8}
\bibverse{13} In den Lippen des Verständigen findet man Weisheit; aber
auf den Rücken der Narren gehört eine Rute. \bibverse{14} Die Weisen
bewahren die Lehre; aber der Narren Mund ist nahe dem Schrecken.
\bibverse{15} Das Gut des Reichen ist seine feste Stadt; aber die Armen
macht die Armut blöde. \footnote{\textbf{10:15} Spr 18,11} \bibverse{16}
Der Gerechte braucht sein Gut zum Leben; aber der Gottlose braucht sein
Einkommen zur Sünde. \footnote{\textbf{10:16} Lk 16,19} \bibverse{17}
Die Zucht halten ist der Weg zum Leben; wer aber der Zurechtweisung
nicht achtet, der bleibt in der Irre. \bibverse{18} Falsche Mäuler
bergen Hass; und wer verleumdet, der ist ein Narr. \bibverse{19} Wo viel
Worte sind, da geht's ohne Sünde nicht ab; wer aber seine Lippen hält,
ist klug. \bibverse{20} Des Gerechten Zunge ist köstliches Silber; aber
der Gottlosen Herz ist wie nichts. \bibverse{21} Des Gerechten Lippen
weiden viele; aber die Narren werden an ihrer Torheit sterben.
\bibverse{22} Der Segen des HErrn macht reich ohne Mühe. \footnote{\textbf{10:22}
  Ps 127,2} \bibverse{23} Ein Narr treibt Mutwillen und hat noch dazu
seinen Spott; aber der Mann ist weise, der aufmerkt. \bibverse{24} Was
der Gottlose fürchtet, das wird ihm begegnen; und was die Gerechten
begehren, wird ihnen gegeben. \bibverse{25} Der Gottlose ist wie ein
Wetter, das vorübergeht und nicht mehr ist; der Gerechte aber besteht
ewiglich. \bibverse{26} Wie der Essig den Zähnen und der Rauch den Augen
tut, so tut der Faule denen, die ihn senden. \bibverse{27} Die Furcht
des HErrn mehrt die Tage; aber die Jahre der Gottlosen werden verkürzt.
\footnote{\textbf{10:27} Spr 9,11; Spr 14,27} \bibverse{28} Das Warten
der Gerechten wird Freude werden; aber der Gottlosen Hoffnung wird
verloren sein. \footnote{\textbf{10:28} Ps 9,19; Hi 8,13} \bibverse{29}
Der Weg des HErrn ist des Frommen Trotz; aber die Übeltäter sind blöde.
\footnote{\textbf{10:29} Spr 3,26} \bibverse{30} Der Gerechte wird
nimmermehr umgestoßen; aber die Gottlosen werden nicht im Lande bleiben.
\footnote{\textbf{10:30} Ps 112,6; Spr 2,22} \bibverse{31} Der Mund des
Gerechten bringt Weisheit; aber die Zunge der Verkehrten wird
ausgerottet. \footnote{\textbf{10:31} Spr 10,11; Ps 37,30} \bibverse{32}
Die Lippen der Gerechten lehren heilsame Dinge; aber der Gottlosen Mund
ist verkehrt. \# 11 \bibverse{1} Falsche Waage ist dem HErrn ein Gräuel;
aber völliges Gewicht ist sein Wohlgefallen. \bibverse{2} Wo Stolz ist,
da ist auch Schmach; aber Weisheit ist bei den Demütigen. \footnote{\textbf{11:2}
  Spr 16,18; Spr 18,12} \bibverse{3} Unschuld wird die Frommen leiten;
aber die Bosheit wird die Verächter verstören. \footnote{\textbf{11:3}
  Ps 52,7} \bibverse{4} Gut hilft nicht am Tage des Zorns; aber
Gerechtigkeit errettet vom Tod. \footnote{\textbf{11:4} Spr 10,2}
\bibverse{5} Die Gerechtigkeit des Frommen macht seinen Weg eben; aber
der Gottlose wird fallen durch sein gottlos Wesen. \bibverse{6} Die
Gerechtigkeit der Frommen wird sie erretten; aber die Verächter werden
gefangen in ihrer Bosheit. \bibverse{7} Wenn der gottlose Mensch stirbt,
ist seine Hoffnung verloren und das Harren des Ungerechten wird
zunichte. \bibverse{8} Der Gerechte wird aus der Not erlöst, und der
Gottlose kommt an seine Statt. \footnote{\textbf{11:8} Spr 21,18; Jes
  43,3} \bibverse{9} Durch den Mund des Heuchlers wird sein Nächster
verderbt; aber die Gerechten merken's und werden erlöst. \bibverse{10}
Eine Stadt freut sich, wenn's den Gerechten wohl geht; und wenn die
Gottlosen umkommen, wird man froh. \bibverse{11} Durch den Segen der
Frommen wird eine Stadt erhoben; aber durch den Mund der Gottlosen wird
sie zerbrochen. \bibverse{12} Wer seinen Nächsten schändet, ist ein
Narr; aber ein verständiger Mann schweigt still. \bibverse{13} Ein
Verleumder verrät, was er heimlich weiß; aber wer eines getreuen Herzens
ist, verbirgt es. \bibverse{14} Wo nicht Rat ist, da geht das Volk
unter; wo aber viel Ratgeber sind, da geht es wohl zu. \bibverse{15} Wer
für einen anderen Bürge wird, der wird Schaden haben; wer aber sich vor
Geloben hütet, ist sicher. \footnote{\textbf{11:15} Spr 6,1-2}
\bibverse{16} Ein holdselig Weib erlangt Ehre; aber die Tyrannen
erlangen Reichtum. \bibverse{17} Ein barmherziger Mann tut sich selber
Gutes; aber ein unbarmherziger betrübt auch sein eigen Fleisch.
\bibverse{18} Der Gottlosen Arbeit wird fehlschlagen; aber wer
Gerechtigkeit sät, das ist gewisses Gut. \bibverse{19} Gerechtigkeit
fördert zum Leben; aber dem Übel nachjagen fördert zum Tod.
\bibverse{20} Der HErr hat Gräuel an den verkehrten Herzen, und
Wohlgefallen an den Frommen. \bibverse{21} Den Bösen hilft nichts, wenn
sie auch alle Hände zusammentäten; aber der Gerechten Same wird errettet
werden. \bibverse{22} Ein schönes Weib ohne Zucht ist wie eine Sau mit
einem goldenen Haarband. \footnote{\textbf{11:22} Spr 31,30}
\bibverse{23} Der Gerechten Wunsch muss doch wohl geraten, und der
Gottlosen Hoffen wird Unglück. \footnote{\textbf{11:23} Spr 11,7}
\bibverse{24} Einer teilt aus und hat immer mehr; ein anderer kargt, da
er nicht soll, und wird doch ärmer. \bibverse{25} Die Seele, die da
reichlich segnet, wird gelabt; und wer reichlich tränkt, der wird auch
getränkt werden. \footnote{\textbf{11:25} Spr 19,17} \bibverse{26} Wer
Korn innehält, dem fluchen die Leute; aber Segen kommt über den, der es
verkauft. \bibverse{27} Wer da Gutes sucht, dem widerfährt Gutes; wer
aber nach Unglück ringt, dem wird's begegnen. \bibverse{28} Wer sich auf
seinen Reichtum verlässt, der wird untergehen; aber die Gerechten werden
grünen wie ein Blatt. \bibverse{29} Wer sein eigen Haus betrübt, der
wird Wind zum Erbteil haben; und ein Narr muss ein Knecht des Weisen
sein. \bibverse{30} Die Frucht des Gerechten ist ein Baum des Lebens,
und ein Weiser gewinnt die Herzen. \footnote{\textbf{11:30} Spr 3,18;
  Spr 15,4} \bibverse{31} So der Gerechte auf Erden leiden muss, wie
viel mehr der Gottlose und der Sünder! \footnote{\textbf{11:31} 1Petr
  4,17-18}

\hypertarget{section-4}{%
\section{12}\label{section-4}}

\bibverse{1} Wer sich gern lässt strafen, der wird klug werden; wer aber
ungestraft sein will, der bleibt ein Narr. \footnote{\textbf{12:1} Spr
  13,1; Spr 13,18} \bibverse{2} Wer fromm ist, der bekommt Trost vom
HErrn; aber ein Ruchloser verdammt sich selbst. \bibverse{3} Ein gottlos
Wesen fördert den Menschen nicht; aber die Wurzel der Gerechten wird
bleiben. \bibverse{4} Ein tugendsam Weib ist eine Krone ihres Mannes;
aber eine böse ist wie Eiter in seinem Gebein. \footnote{\textbf{12:4}
  Spr 31,10-31} \bibverse{5} Die Gedanken der Gerechten sind redlich;
aber die Anschläge der Gottlosen sind Trügerei. \footnote{\textbf{12:5}
  1Kö 12,6-19} \bibverse{6} Der Gottlosen Reden richten Blutvergießen
an; aber der Frommen Mund errettet. \bibverse{7} Die Gottlosen werden
umgestürzt und nicht mehr sein; aber das Haus der Gerechten bleibt
stehen. \footnote{\textbf{12:7} Spr 10,25; Hi 8,13-19} \bibverse{8}
Eines weisen Mannes Rat wird gelobt; aber die da tückisch sind, werden
zu Schanden. \bibverse{9} Wer gering ist und wartet des Seinen, der ist
besser, denn der groß sein will, und des Brots mangelt. \bibverse{10}
Der Gerechte erbarmt sich seines Viehs; aber das Herz der Gottlosen ist
unbarmherzig. \bibverse{11} Wer seinen Acker baut, der wird Brot die
Fülle haben; wer aber unnötigen Sachen nachgeht, der ist ein Narr.
\footnote{\textbf{12:11} Spr 28,19} \bibverse{12} Des Gottlosen Lust
ist, Schaden zu tun; aber die Wurzel der Gerechten wird Frucht bringen.
\footnote{\textbf{12:12} Spr 12,3} \bibverse{13} Der Böse wird gefangen
in seinen eigenen falschen Worten; aber der Gerechte entgeht der Angst.
\bibverse{14} Viel Gutes kommt dem Mann durch die Frucht des Mundes; und
dem Menschen wird vergolten, nach dem seine Hände verdient haben.
\footnote{\textbf{12:14} Röm 2,6} \bibverse{15} Dem Narren gefällt seine
Weise wohl; aber wer auf Rat hört, der ist weise. \bibverse{16} Ein Narr
zeigt seinen Zorn alsbald; aber wer die Schmach birgt, ist klug.
\bibverse{17} Wer wahrhaftig ist, der sagt frei, was recht ist; aber ein
falscher Zeuge betrügt. \bibverse{18} Wer unvorsichtig herausfährt,
sticht wie ein Schwert; aber die Zunge der Weisen ist heilsam.
\bibverse{19} Wahrhaftiger Mund besteht ewiglich; aber die falsche Zunge
besteht nicht lange. \bibverse{20} Die, so Böses raten, betrügen; aber
die zum Frieden raten, schaffen Freude. \bibverse{21} Es wird dem
Gerechten kein Leid geschehen; aber die Gottlosen werden voll Unglück
sein. \bibverse{22} Falsche Mäuler sind dem HErrn ein Gräuel; die aber
treulich handeln, gefallen ihm wohl. \footnote{\textbf{12:22} Spr 6,17}
\bibverse{23} Ein verständiger Mann trägt nicht Klugheit zur Schau; aber
das Herz der Narren ruft seine Narrheit aus. \footnote{\textbf{12:23}
  Spr 29,11} \bibverse{24} Fleißige Hand wird herrschen; die aber lässig
ist, wird müssen zinsen. \footnote{\textbf{12:24} Spr 10,4}
\bibverse{25} Sorge im Herzen kränkt, aber ein freundliches Wort
erfreut. \footnote{\textbf{12:25} Spr 16,24} \bibverse{26} Der Gerechte
hat's besser denn sein Nächster; aber der Gottlosen Weg verführt sie.
\bibverse{27} Einem Lässigen gerät sein Handel nicht; aber ein fleißiger
Mensch wird reich. \footnote{\textbf{12:27} Spr 12,24} \bibverse{28} Auf
dem Wege der Gerechtigkeit ist Leben, und auf ihrem gebahnten Pfad ist
kein Tod. \# 13 \bibverse{1} Ein weiser Sohn lässt sich vom Vater
züchtigen; aber ein Spötter gehorcht der Strafe nicht. \bibverse{2} Die
Frucht des Mundes genießt man; aber die Verächter denken nur zu freveln.
\footnote{\textbf{13:2} Spr 12,14} \bibverse{3} Wer seinen Mund bewahrt,
der bewahrt sein Leben; wer aber mit seinem Maul herausfährt, der kommt
in Schrecken. \footnote{\textbf{13:3} Spr 12,18; Spr 21,23} \bibverse{4}
Der Faule begehrt und kriegt's doch nicht; aber die Fleißigen kriegen
genug. \footnote{\textbf{13:4} Spr 10,4} \bibverse{5} Der Gerechte ist
der Lüge feind; aber der Gottlose schändet und schmäht sich selbst.
\bibverse{6} Die Gerechtigkeit behütet den Unschuldigen; aber das
gottlose Wesen bringt zu Fall den Sünder. \bibverse{7} Mancher ist arm
bei großem Gut, und mancher ist reich bei seiner Armut. \bibverse{8} Mit
Reichtum kann einer sein Leben erretten; aber ein Armer hört kein
Schelten. \bibverse{9} Das Licht der Gerechten brennt fröhlich; aber die
Leuchte der Gottlosen wird auslöschen. \footnote{\textbf{13:9} Spr
  24,20; Hi 5,14; Hi 18,5-6; Hi 18,18} \bibverse{10} Unter den Stolzen
ist immer Hader; aber Weisheit ist bei denen, die sich raten lassen.
\footnote{\textbf{13:10} Spr 28,25; Spr 1,5} \bibverse{11} Reichtum wird
wenig, wo man's vergeudet; was man aber zusammenhält, das wird groß.
\bibverse{12} Die Hoffnung, die sich verzieht, ängstet das Herz; wenn's
aber kommt, was man begehrt, das ist ein Baum des Lebens. \bibverse{13}
Wer das Wort verachtet, der verderbt sich selbst; wer aber das Gebot
fürchtet, dem wird's vergolten. \bibverse{14} Die Lehre des Weisen ist
eine Quelle des Lebens, zu meiden die Stricke des Todes. \footnote{\textbf{13:14}
  Spr 10,11; Spr 14,27} \bibverse{15} Feine Klugheit schafft Gunst; aber
der Verächter Weg bringt Wehe. \bibverse{16} Ein Kluger tut alles mit
Vernunft; ein Narr aber breitet Narrheit aus. \bibverse{17} Ein
gottloser Bote bringt Unglück; aber ein treuer Bote ist heilsam.
\bibverse{18} Wer Zucht lässt fahren, der hat Armut und Schande; wer
sich gerne strafen lässt, wird zu ehren kommen. \footnote{\textbf{13:18}
  Spr 12,1} \bibverse{19} Wenn's kommt, was man begehrt, das tut dem
Herzen wohl; aber das Böse meiden ist den Toren ein Gräuel.
\bibverse{20} Wer mit den Weisen umgeht, der wird weise; wer aber der
Narren Geselle ist, der wird Unglück haben. \bibverse{21} Unglück
verfolgt die Sünder; aber den Gerechten wird Gutes vergolten.
\bibverse{22} Der Gute wird vererben auf Kindeskind; aber des Sünders
Gut wird für den Gerechten gespart. \bibverse{23} Es ist viel Speise in
den Furchen der Armen; aber die Unrecht tun, verderben. \bibverse{24}
Wer seine Rute schont, der hasst seinen Sohn; wer ihn aber liebhat, der
züchtigt ihn bald. \footnote{\textbf{13:24} Spr 22,15} \bibverse{25} Der
Gerechte isst, dass sein Seele satt wird; der Gottlosen Bauch aber hat
nimmer genug. \footnote{\textbf{13:25} Ps 34,11}

\hypertarget{section-5}{%
\section{14}\label{section-5}}

\bibverse{1} Durch weise Weiber wird das Haus erbaut; eine Närrin aber
zerbricht's mit ihrem Tun. \bibverse{2} Wer den HErrn fürchtet, der
wandelt auf rechter Bahn; wer ihn aber verachtet, der geht auf Abwegen.
\bibverse{3} Narren reden tyrannisch; aber die Weisen bewahren ihren
Mund. \bibverse{4} Wo nicht Ochsen sind, da ist die Krippe rein; aber wo
der Ochse geschäftig ist, da ist viel Einkommen. \bibverse{5} Ein treuer
Zeuge lügt nicht; aber ein falscher Zeuge redet frech Lügen.
\bibverse{6} Der Spötter sucht Weisheit, und findet sie nicht; aber dem
Verständigen ist die Erkenntnis leicht. \bibverse{7} Gehe von dem
Narren; denn du lernst nichts von ihm. \bibverse{8} Das ist des Klugen
Weisheit, dass er auf seinen Weg merkt; aber der Narren Torheit ist
eitel Trug. \bibverse{9} Die Narren treiben das Gespött mit der Sünde;
aber die Frommen haben Lust an den Frommen. \bibverse{10} Das Herz kennt
sein eigen Leid, und in seine Freude kann sich kein Fremder mengen.
\bibverse{11} Das Haus der Gottlosen wird vertilgt; aber die Hütte der
Frommen wird grünen. \footnote{\textbf{14:11} Hi 18,14; Spr 12,7}
\bibverse{12} Es gefällt manchem ein Weg wohl; aber endlich bringt er
ihn zum Tode. \bibverse{13} Auch beim Lachen kann das Herz trauern, und
nach der Freude kommt Leid. \bibverse{14} Einem losen Menschen wird's
gehen wie er handelt; aber ein Frommer wird über ihn sein. \bibverse{15}
Ein Unverständiger glaubt alles; aber ein Kluger merkt auf seinen Gang.
\bibverse{16} Ein Weiser fürchtet sich und meidet das Arge; ein Narr
aber fährt trotzig hindurch. \bibverse{17} Ein Ungeduldiger handelt
töricht; aber ein Bedächtiger hasst es. \bibverse{18} Die Unverständigen
erben Narrheit; aber es ist der Klugen Krone, vorsichtig handeln.
\bibverse{19} Die Bösen müssen sich bücken vor den Guten und die
Gottlosen in den Toren des Gerechten. \bibverse{20} Einen Armen hassen
auch seine Nächsten; aber die Reichen haben viel Freunde. \footnote{\textbf{14:20}
  Spr 19,4; Spr 19,7} \bibverse{21} Der Sünder verachtet seinen
Nächsten; aber wohl dem, der sich der Elenden erbarmt! \footnote{\textbf{14:21}
  Ps 41,2} \bibverse{22} Die mit bösen Ränken umgehen, werden fehlgehen;
die aber Gutes denken, denen wird Treue und Güte widerfahren.
\bibverse{23} Wo man arbeitet, da ist genug; wo man aber mit Worten
umgeht, da ist Mangel. \footnote{\textbf{14:23} Spr 10,4} \bibverse{24}
Den Weisen ist ihr Reichtum eine Krone; aber die Torheit der Narren
bleibt Torheit. \bibverse{25} Ein treuer Zeuge errettet das Leben; aber
ein falscher Zeuge betrügt. \bibverse{26} Wer den HErrn fürchtet, der
hat eine sichere Festung, und seine Kinder werden auch beschirmt.
\footnote{\textbf{14:26} Spr 18,10} \bibverse{27} Die Furcht des HErrn
ist eine Quelle des Lebens, dass man meide die Stricke des Todes.
\footnote{\textbf{14:27} Spr 13,14} \bibverse{28} Wo ein König viel
Volks hat, das ist seine Herrlichkeit; wo aber wenig Volks ist, das
macht einen Herrn blöde. \bibverse{29} Wer geduldig ist, der ist weise;
wer aber ungeduldig ist, der offenbart seine Torheit. \footnote{\textbf{14:29}
  Spr 16,32; Spr 19,11} \bibverse{30} Ein gütiges Herz ist des Leibes
Leben; aber Neid ist Eiter in den Gebeinen. \footnote{\textbf{14:30} Spr
  12,4} \bibverse{31} Wer dem Geringen Gewalt tut, der lästert desselben
Schöpfer; aber wer sich des Armen erbarmt, der ehrt Gott. \footnote{\textbf{14:31}
  Spr 17,5; Spr 19,17} \bibverse{32} Der Gottlose besteht nicht in
seinem Unglück; aber der Gerechte ist auch in seinem Tod getrost.
\bibverse{33} Im Herzen des Verständigen ruht Weisheit, und wird
offenbar unter den Narren. \bibverse{34} Gerechtigkeit erhöhet ein Volk;
aber die Sünde ist der Leute Verderben. \bibverse{35} Ein kluger Knecht
gefällt dem König wohl; aber einem schändlichen Knecht ist er feind. \#
15 \bibverse{1} Eine linde Antwort stillt den Zorn; aber ein hartes Wort
richtet Grimm an. \footnote{\textbf{15:1} Spr 15,18; 1Kö 12,13; 1Kö
  12,16} \bibverse{2} Der Weisen Zunge macht die Lehre lieblich; der
Narren Mund speit eitel Narrheit. \footnote{\textbf{15:2} Spr 12,23}
\bibverse{3} Die Augen des HErrn schauen an allen Orten beide, die Bösen
und Frommen. \bibverse{4} Ein heilsame Zunge ist ein Baum des Lebens;
aber eine lügenhafte macht Herzeleid. \bibverse{5} Der Narr lästert die
Zucht seines Vaters; wer aber Strafe annimmt, der wird klug werden.
\footnote{\textbf{15:5} Spr 15,32; Spr 13,1} \bibverse{6} In des
Gerechten Haus ist Guts genug; aber in dem Einkommen des Gottlosen ist
Verderben. \bibverse{7} Der Weisen Mund streut guten Rat; aber der
Narren Herz ist nicht richtig. \bibverse{8} Der Gottlosen Opfer ist dem
HErrn ein Gräuel; aber das Gebet der Frommen ist ihm angenehm.
\bibverse{9} Der Gottlosen Weg ist dem HErrn ein Gräuel; wer aber der
Gerechtigkeit nachjagt, den liebt er. \footnote{\textbf{15:9} Spr 11,20}
\bibverse{10} Den Weg verlassen bringt böse Züchtigung, und wer die
Strafe hasst, der muss sterben. \footnote{\textbf{15:10} Spr 10,17; Spr
  29,1} \bibverse{11} Hölle und Abgrund ist vor dem HErrn; wie viel mehr
der Menschen Herzen! \footnote{\textbf{15:11} Hi 26,6; Ps 139,8; Jer
  17,10} \bibverse{12} Der Spötter liebt den nicht, der ihn straft, und
geht nicht zu den Weisen. \footnote{\textbf{15:12} Spr 9,8; Spr 13,1}
\bibverse{13} Ein fröhlich Herz macht ein fröhlich Angesicht; aber wenn
das Herz bekümmert ist, so fällt auch der Mut. \footnote{\textbf{15:13}
  Spr 15,15} \bibverse{14} Ein kluges Herz handelt bedächtig; aber der
Narren Mund geht mit Torheit um. \bibverse{15} Ein Betrübter hat nimmer
einen guten Tag; aber ein guter Mut ist ein täglich Wohlleben.
\bibverse{16} Es ist besser ein wenig mit der Furcht des HErrn denn
großer Schatz, darin Unruhe ist. \footnote{\textbf{15:16} Spr 16,8; Spr
  17,1; Ps 37,16} \bibverse{17} Es ist besser ein Gericht Kraut mit
Liebe, denn ein gemästeter Ochse mit Hass. \bibverse{18} Ein zorniger
Mann richtet Hader an; ein Geduldiger aber stillt den Zank.
\bibverse{19} Der Weg des Faulen ist dornig; aber der Weg der Frommen
ist wohl gebahnt. \footnote{\textbf{15:19} Spr 24,30-31} \bibverse{20}
Ein weiser Sohn erfreut den Vater, und ein törichter Mensch ist seiner
Mutter Schande. \footnote{\textbf{15:20} Spr 10,1} \bibverse{21} Dem
Toren ist die Torheit eine Freude; aber ein verständiger Mann bleibt auf
dem rechten Wege. \bibverse{22} Die Anschläge werden zunichte, wo nicht
Rat ist; wo aber viel Ratgeber sind, bestehen sie. \footnote{\textbf{15:22}
  Spr 11,14} \bibverse{23} Es ist einem Mann eine Freude, wenn er
richtig antwortet; und ein Wort zu seiner Zeit ist sehr lieblich.
\bibverse{24} Der Weg des Lebens geht überwärts für den Klugen, auf dass
er meide die Hölle unterwärts. \bibverse{25} Der HErr wird das Haus des
Hoffärtigen zerbrechen und die Grenze der Witwe bestätigen.
\bibverse{26} Die Anschläge des Argen sind dem HErrn ein Gräuel; aber
freundlich reden die Reinen. \bibverse{27} Der Geizige verstört sein
eigen Haus; wer aber Geschenke hasst, der wird leben. \bibverse{28} Das
Herz des Gerechten ersinnt, was zu antworten ist; aber der Mund der
Gottlosen schäumt Böses. \bibverse{29} Der HErr ist fern von den
Gottlosen; aber der Gerechten Gebet erhört er. \footnote{\textbf{15:29}
  Spr 15,8; Joh 9,31} \bibverse{30} Freundlicher Anblick erfreut das
Herz; eine gute Botschaft labt das Gebein. \footnote{\textbf{15:30} Spr
  25,25} \bibverse{31} Das Ohr, das da hört die Strafe des Lebens, wird
unter den Weisen wohnen. \bibverse{32} Wer sich nicht ziehen lässt, der
macht sich selbst zunichte; wer aber auf Strafe hört, der wird klug.
\footnote{\textbf{15:32} Spr 15,5} \bibverse{33} Die Furcht des HErrn
ist Zucht zur Weisheit; und ehe man zu Ehren kommt, muss man zuvor
leiden. \footnote{\textbf{15:33} Spr 1,7; Spr 18,12}

\hypertarget{section-6}{%
\section{16}\label{section-6}}

\bibverse{1} Der Mensch setzt sich's wohl vor im Herzen; aber vom HErrn
kommt, was die Zunge reden soll. \bibverse{2} Einen jeglichen dünken
seine Wege rein; aber der HErr wägt die Geister. \footnote{\textbf{16:2}
  Spr 21,2} \bibverse{3} Befiehl dem HErrn deine Werke, so werden deine
Anschläge fortgehen. \footnote{\textbf{16:3} Ps 37,5} \bibverse{4} Der
HErr macht alles zu bestimmtem Ziel, auch den Gottlosen für den bösen
Tag. \bibverse{5} Ein stolzes Herz ist dem HErrn ein Gräuel und wird
nicht ungestraft bleiben, wenn sie sich gleich alle aneinander hängen.
\footnote{\textbf{16:5} Spr 11,21} \bibverse{6} Durch Güte und Treue
wird Missetat versöhnt, und durch die Furcht des HErrn meidet man das
Böse. \bibverse{7} Wenn jemands Wege dem HErrn wohl gefallen, so macht
er auch seine Feinde mit ihm zufrieden. \bibverse{8} Es ist besser wenig
mit Gerechtigkeit denn viel Einkommen mit Unrecht. \footnote{\textbf{16:8}
  Spr 15,16} \bibverse{9} Des Menschen Herz erdenkt sich seinen Weg;
aber der HErr allein gibt, dass er fortgehe. \footnote{\textbf{16:9} Spr
  19,21} \bibverse{10} Weissagung ist in dem Munde des Königs; sein Mund
fehlt nicht im Gericht. \bibverse{11} Rechte Waage und Gewicht ist vom
HErrn; und alle Pfunde im Sack sind seine Werke. \footnote{\textbf{16:11}
  Spr 11,1} \bibverse{12} Den Königen ist Unrecht tun ein Gräuel; denn
durch Gerechtigkeit wird der Thron befestigt. \footnote{\textbf{16:12}
  Spr 20,28; Spr 25,5; Spr 29,14} \bibverse{13} Recht raten gefällt den
Königen; und wer aufrichtig redet, wird geliebt. \bibverse{14} Des
Königs Grimm ist ein Bote des Todes; aber ein weiser Mann wird ihn
versöhnen. \footnote{\textbf{16:14} Spr 20,2} \bibverse{15} Wenn des
Königs Angesicht freundlich ist, das ist Leben, und seine Gnade ist wie
ein Spätregen. \footnote{\textbf{16:15} Spr 19,12} \bibverse{16} Nimm an
die Weisheit, denn sie ist besser als Gold; und Verstand haben ist edler
als Silber. \footnote{\textbf{16:16} Spr 3,14; Spr 8,10-11; Spr 8,19}
\bibverse{17} Der Frommen Weg meidet das Arge; und wer seinen Weg
bewahrt, der erhält sein Leben. \bibverse{18} Wer zu Grunde gehen soll,
der wird zuvor stolz; und Hochmut kommt vor dem Fall. \bibverse{19} Es
ist besser, niedriges Gemüts sein mit den Elenden, denn Raub austeilen
mit den Hoffärtigen. \bibverse{20} Wer eine Sache klüglich führt, der
findet Glück; und wohl dem, der sich auf den HErrn verlässt!
\bibverse{21} Ein Verständiger wird gerühmt für einen weisen Mann, und
liebliche Reden lehren wohl. \bibverse{22} Klugheit ist wie ein Brunnen
des Lebens dem, der sie hat; aber die Zucht der Narren ist Narrheit.
\footnote{\textbf{16:22} Spr 13,14; Spr 14,27} \bibverse{23} Ein weises
Herz redet klug und lehrt wohl. \bibverse{24} Die Reden des Freundlichen
sind Honigseim, trösten die Seele und erfrischen die Gebeine.
\bibverse{25} Manchem gefällt ein Weg wohl; aber zuletzt bringt er ihn
zum Tode. \footnote{\textbf{16:25} Spr 14,12} \bibverse{26} Mancher
kommt zu großem Unglück durch sein eigen Maul. \footnote{\textbf{16:26}
  Spr 18,7} \bibverse{27} Ein loser Mensch gräbt nach Unglück, und in
seinem Maul brennt Feuer. \bibverse{28} Ein verkehrter Mensch richtet
Hader an, und ein Verleumder macht Freunde uneins. \footnote{\textbf{16:28}
  Spr 6,14; Spr 6,19} \bibverse{29} Ein Frevler lockt seinen Nächsten
und führt ihn auf keinen guten Weg. \footnote{\textbf{16:29} Spr 1,10-14}
\bibverse{30} Wer mit den Augen winkt, denkt nichts Gutes; und wer mit
den Lippen andeutet, vollbringt Böses. \footnote{\textbf{16:30} Spr 6,13}
\bibverse{31} Graue Haare sind eine Krone der Ehren, die auf dem Wege
der Gerechtigkeit gefunden wird. \footnote{\textbf{16:31} Spr 20,29}
\bibverse{32} Ein Geduldiger ist besser denn ein Starker, und der seines
Mutes Herr ist, denn der Städte gewinnt. \footnote{\textbf{16:32} Spr
  14,29} \bibverse{33} Das Los wird geworfen in den Schoß; aber es
fällt, wie der HErr will. \# 17 \bibverse{1} Es ist ein trockener
Bissen, daran man sich genügen lässt, besser denn ein Haus voll
Geschlachtetes mit Hader. \bibverse{2} Ein kluger Knecht wird herrschen
über unfleißige Erben und wird unter den Brüdern das Erbe austeilen.
\bibverse{3} Wie das Feuer Silber und der Ofen Gold, also prüft der HErr
die Herzen. \footnote{\textbf{17:3} Ps 66,10} \bibverse{4} Ein Böser
achtet auf böse Mäuler, und ein Falscher gehorcht gern schädlichen
Zungen. \bibverse{5} Wer des Dürftigen spottet, der höhnt desselben
Schöpfer; und wer sich über eines anderen Unglück freut, wird nicht
ungestraft bleiben. \bibverse{6} Der Alten Krone sind Kindeskinder, und
der Kinder Ehre sind ihre Väter. \footnote{\textbf{17:6} Ps 128,6}
\bibverse{7} Es steht einem Narren nicht wohl an, von hohen Dingen
reden, viel weniger einem Fürsten, dass er gern lügt. \bibverse{8} Wer
zu schenken hat, dem ist's ein Edelstein; wo er sich hin kehrt, ist er
klug geachtet. \bibverse{9} Wer Sünde zudeckt, der macht Freundschaft;
wer aber die Sache aufrührt, der macht Freunde uneins. \bibverse{10}
Schelten bringt mehr ein an dem Verständigen denn hundert Schläge an dem
Narren. \bibverse{11} Ein bitterer Mensch trachtet, eitel Schaden zu
tun; aber es wird ein grimmiger Engel über ihn kommen. \bibverse{12} Es
ist besser, einem Bären begegnen, dem die Jungen geraubt sind, denn
einem Narren in seiner Narrheit. \bibverse{13} Wer Gutes mit Bösem
vergilt, von dessen Haus wird Böses nicht lassen. \bibverse{14} Wer
Hader anfängt, ist gleich dem, der dem Wasser den Damm aufreißt. Lass du
vom Hader, ehe du drein gemengt wirst. \bibverse{15} Wer den Gottlosen
gerechtspricht und den Gerechten verdammt, die sind beide dem HErrn ein
Gräuel. \footnote{\textbf{17:15} Jes 5,23} \bibverse{16} Was soll dem
Narren Geld in der Hand, Weisheit zu kaufen, wenn er doch ein Narr ist?
\bibverse{17} Ein Freund liebt allezeit, und als ein Bruder wird er in
Not erfunden. \bibverse{18} Es ist ein Narr, der in die Hand gelobt und
Bürge wird für seinen Nächsten. \footnote{\textbf{17:18} Spr 6,1}
\bibverse{19} Wer Zank liebt, der liebt Sünde; und wer seine Tür hoch
macht, ringt nach Einsturz. \bibverse{20} Ein verkehrtes Herz findet
nichts Gutes; und der verkehrter Zunge ist, wird in Unglück fallen.
\bibverse{21} Wer einen Narren zeugt, der hat Grämen; und eines Narren
Vater hat keine Freude. \bibverse{22} Ein fröhlich Herz macht das Leben
lustig; aber ein betrübter Mut vertrocknet das Gebein. \footnote{\textbf{17:22}
  Spr 15,13; Spr 15,15} \bibverse{23} Der Gottlose nimmt heimlich gern
Geschenke, zu beugen den Weg des Rechts. \bibverse{24} Ein Verständiger
gebärdet sich weise; ein Narr wirft die Augen hin und her. \bibverse{25}
Ein törichter Sohn ist seines Vaters Trauern und Betrübnis der Mutter,
die ihn geboren hat. \footnote{\textbf{17:25} Spr 17,21} \bibverse{26}
Es ist nicht gut, dass man den Gerechten schindet, noch den Edlen zu
schlagen, der recht handelt. \bibverse{27} Ein Vernünftiger mäßigt seine
Rede; und ein verständiger Mann ist kaltes Muts. \bibverse{28} Ein Narr,
wenn er schwiege, würde auch für weise gerechnet, und verständig, wenn
er das Maul hielte. \footnote{\textbf{17:28} Hi 13,5}

\hypertarget{section-7}{%
\section{18}\label{section-7}}

\bibverse{1} Wer sich absondert, der sucht, was ihn gelüstet, und setzt
sich wider alles, was gut ist. \bibverse{2} Ein Narr hat nicht Lust am
Verstand, sondern kundzutun, was in seinem Herzen steckt. \bibverse{3}
Wo der Gottlose hin kommt, da kommt Verachtung und Schmach mit Hohn.
\bibverse{4} Die Worte in eines Mannes Munde sind wie tiefe Wasser, und
die Quelle der Weisheit ist ein voller Strom. \bibverse{5} Es ist nicht
gut, die Person des Gottlosen achten, zu beugen den Gerechten im
Gericht. \bibverse{6} Die Lippen des Narren bringen Zank, und sein Mund
ringt nach Schlägen. \bibverse{7} Der Mund des Narren schadet ihm
selbst, und seine Lippen fangen seine eigene Seele. \footnote{\textbf{18:7}
  Spr 13,3; Spr 16,26} \bibverse{8} Die Worte des Verleumders sind
Schläge und gehen einem durchs Herz. \footnote{\textbf{18:8} Spr 26,22}
\bibverse{9} Wer lässig ist in seiner Arbeit, der ist ein Bruder des,
der das Seine umbringt. \footnote{\textbf{18:9} Spr 10,4} \bibverse{10}
Der Name des HErrn ist ein festes Schloss; der Gerechte läuft dahin und
wird beschirmt. \footnote{\textbf{18:10} Spr 14,26; Ps 20,2}
\bibverse{11} Das Gut des Reichen ist ihm eine feste Stadt und wie eine
hohe Mauer in seinem Dünkel. \footnote{\textbf{18:11} Spr 10,15}
\bibverse{12} Wenn einer zu Grunde gehen soll, wird sein Herz zuvor
stolz; und ehe man zu Ehren kommt, muss man zuvor leiden. \footnote{\textbf{18:12}
  Spr 16,18; Spr 15,33} \bibverse{13} Wer antwortet ehe er hört, dem
ist's Narrheit und Schande. \bibverse{14} Wer ein fröhlich Herz hat, der
weiß sich in seinem Leiden zu halten; wenn aber der Mut liegt, wer
kann's tragen? \footnote{\textbf{18:14} Spr 15,13; Spr 15,15}
\bibverse{15} Ein verständiges Herz weiß sich vernünftig zu halten; und
die Weisen hören gern, wie man vernünftig handelt. \bibverse{16} Das
Geschenk des Menschen macht ihm Raum und bringt ihn vor die großen
Herren. \bibverse{17} Ein jeglicher ist zuerst in seiner Sache gerecht;
kommt aber sein Nächster hinzu, so findet sich's. \bibverse{18} Das Los
stillt den Hader und scheidet zwischen den Mächtigen. \footnote{\textbf{18:18}
  Spr 16,33} \bibverse{19} Ein verletzter Bruder hält härter denn eine
feste Stadt, und Zank hält härter denn Riegel am Palast. \bibverse{20}
Einem Mann wird vergolten, darnach sein Mund geredet hat, und er wird
gesättigt von der Frucht seiner Lippen. \bibverse{21} Tod und Leben
steht in der Zunge Gewalt; wer sie liebt, der wird von ihrer Frucht
essen. \footnote{\textbf{18:21} Spr 13,3} \bibverse{22} Wer eine Ehefrau
findet, der findet etwas Gutes und kann guter Dinge sein im HErrn.
\footnote{\textbf{18:22} Spr 19,14; Spr 31,10} \bibverse{23} Ein Armer
redet mit Flehen, ein Reicher antwortet stolz. \bibverse{24} Ein treuer
Freund liebt mehr und steht fester bei denn ein Bruder. \# 19
\bibverse{1} Ein Armer, der in seiner Frömmigkeit wandelt, ist besser
denn ein Verkehrter mit seinen Lippen, der doch ein Narr ist.
\footnote{\textbf{19:1} Spr 28,6} \bibverse{2} Wo man nicht mit Vernunft
handelt, da geht's nicht wohl zu; und wer schnell ist mit Füßen, der tut
sich Schaden. \bibverse{3} Die Torheit eines Menschen verleitet seinen
Weg, und doch tobt sein Herz wider den HErrn. \bibverse{4} Gut macht
viel Freunde; aber der Arme wird von seinen Freunden verlassen.
\footnote{\textbf{19:4} Spr 14,20} \bibverse{5} Ein falscher Zeuge
bleibt nicht ungestraft; und wer Lügen frech redet, wird nicht
entrinnen. \footnote{\textbf{19:5} Spr 19,9; Spr 21,28; 5Mo 19,18-21}
\bibverse{6} Viele schmeicheln der Person des Fürsten; und alle sind
Freunde des, der Geschenke gibt. \bibverse{7} Den Armen hassen alle
seine Brüder; wie viel mehr halten sich seine Freunde von ihm fern! Und
wer sich auf Worte verlässt, dem wird nichts. \footnote{\textbf{19:7}
  Spr 19,4} \bibverse{8} Wer klug wird, liebt sein Leben; und der
Verständige findet Gutes. \bibverse{9} Ein falscher Zeuge bleibt nicht
ungestraft; und wer frech Lügen redet, wird umkommen. \bibverse{10} Dem
Narren steht nicht wohl an, gute Tage haben, viel weniger einem Knecht,
zu herrschen über Fürsten. \bibverse{11} Wer geduldig ist, der ist ein
kluger Mensch, und ist ihm eine Ehre, dass er Untugend überhören kann.
\bibverse{12} Die Ungnade des Königs ist wie das Brüllen eines jungen
Löwen; aber seine Gnade ist wie Tau auf dem Grase. \footnote{\textbf{19:12}
  Spr 20,2; Spr 16,14-15} \bibverse{13} Ein törichter Sohn ist seines
Vaters Herzeleid, und ein zänkisches Weib ein stetiges Triefen.
\footnote{\textbf{19:13} Spr 10,1} \bibverse{14} Haus und Güter vererben
die Eltern; aber ein vernünftiges Weib kommt vom HErrn. \footnote{\textbf{19:14}
  Spr 18,22} \bibverse{15} Faulheit bringt Schlafen, und eine lässige
Seele wird Hunger leiden. \footnote{\textbf{19:15} Spr 10,4; Spr 23,21}
\bibverse{16} Wer das Gebot bewahrt, der bewahrt sein Leben; wer aber
seines Weges nicht achtet, wird sterben. \footnote{\textbf{19:16} Spr
  16,17} \bibverse{17} Wer sich des Armen erbarmt, der leihet dem HErrn;
der wird ihm wieder Gutes vergelten. \footnote{\textbf{19:17} Spr 14,31;
  Ps 41,2-4; Mt 25,40} \bibverse{18} Züchtige deinen Sohn, solange
Hoffnung da ist; aber lass deine Seele nicht bewegt werden, ihn zu
töten. \footnote{\textbf{19:18} Eph 6,4} \bibverse{19} Großer Grimm muss
Schaden leiden; denn willst du ihm steuern, so wird er noch größer.
\bibverse{20} Gehorche dem Rat und nimm Zucht an, dass du hernach weise
seist. \bibverse{21} Es sind viele Anschläge in eines Mannes Herzen;
aber der Rat des HErrn besteht. \bibverse{22} Ein Mensch hat Lust an
seiner Wohltat; und ein Armer ist besser denn ein Lügner. \bibverse{23}
Die Furcht des HErrn fördert zum Leben, und wird satt bleiben, dass kein
Übel sie heimsuchen wird. \footnote{\textbf{19:23} Spr 14,27}
\bibverse{24} Der Faule verbirgt seine Hand im Topfe und bringt sie
nicht wieder zum Munde. \footnote{\textbf{19:24} Spr 26,15}
\bibverse{25} Schlägt man den Spötter, so wird der Unverständige klug;
straft man einen Verständigen, so wird er vernünftig. \footnote{\textbf{19:25}
  Spr 21,11} \bibverse{26} Wer Vater verstört und Mutter verjagt, der
ist ein schändliches und verfluchtes Kind. \bibverse{27} Lass ab, mein
Sohn, zu hören die Zucht, und doch abzuirren von vernünftiger Lehre.
\bibverse{28} Ein loser Zeuge spottet des Rechts, und der Gottlosen Mund
verschlingt das Unrecht. \bibverse{29} Den Spöttern sind Strafen
bereitet, und Schläge auf der Narren Rücken. \footnote{\textbf{19:29}
  Spr 26,3}

\hypertarget{section-8}{%
\section{20}\label{section-8}}

\bibverse{1} Der Wein macht lose Leute, und starkes Getränk macht wild;
wer dazu Lust hat, wird nimmer weise. \footnote{\textbf{20:1} Spr
  23,29-35; Spr 31,5} \bibverse{2} Das Schrecken des Königs ist wie das
Brüllen eines jungen Löwen; wer ihn erzürnt, der sündigt wider sein
Leben. \footnote{\textbf{20:2} Spr 16,14; Spr 19,12} \bibverse{3} Es ist
dem Mann eine Ehre, vom Hader bleiben; aber die gern hadern, sind
allzumal Narren. \bibverse{4} Um der Kälte willen will der Faule nicht
pflügen; so muss er in der Ernte betteln und nichts kriegen.
\bibverse{5} Der Rat im Herzen eines Mannes ist wie tiefe Wasser; aber
ein Verständiger kann's merken, was er meint. \footnote{\textbf{20:5}
  Spr 18,4} \bibverse{6} Viele Menschen werden fromm gerühmt; aber wer
will finden einen, der rechtschaffen fromm sei? \bibverse{7} Ein
Gerechter, der in seiner Frömmigkeit wandelt, des Kindern wird's wohl
gehen nach ihm. \bibverse{8} Ein König, der auf seinem Stuhl sitzt, zu
richten, zerstreut alles Arge mit seinen Augen. \footnote{\textbf{20:8}
  Ps 101,3-8} \bibverse{9} Wer kann sagen: Ich bin rein in meinem Herzen
und lauter von meiner Sünde? \footnote{\textbf{20:9} Spr 28,13; Spr
  30,12} \bibverse{10} Mancherlei Gewicht und Maß ist beides Gräuel dem
HErrn. \footnote{\textbf{20:10} Spr 20,23; Spr 11,1} \bibverse{11} Auch
einen Knaben kennt man an seinem Wesen, ob er fromm und redlich werden
will. \footnote{\textbf{20:11} Spr 22,6} \bibverse{12} Ein hörend Ohr
und sehend Auge, die macht beide der HErr. \bibverse{13} Liebe den
Schlaf nicht, dass du nicht arm werdest; lass deine Augen wacker sein,
so wirst du Brot genug haben. \footnote{\textbf{20:13} Spr 6,10}
\bibverse{14} „Böse, böse!{}`` spricht man, wenn man's hat; aber wenn's
weg ist, so rühmt man es dann. \bibverse{15} Es gibt Gold und viele
Perlen; aber ein vernünftiger Mund ist ein edles Kleinod. \bibverse{16}
Nimm dem sein Kleid, der für einen anderen Bürge wird, und pfände ihn um
des Fremden willen. \bibverse{17} Das gestohlene Brot schmeckt dem Manne
wohl; aber hernach wird ihm der Mund voll Kieselsteine werden.
\footnote{\textbf{20:17} Spr 9,17} \bibverse{18} Anschläge bestehen,
wenn man sie mit Rat führt; und Krieg soll man mit Vernunft führen.
\footnote{\textbf{20:18} Spr 24,6} \bibverse{19} Sei unverworren mit
dem, der Heimlichkeit offenbart, und mit dem Verleumder und mit dem
falschen Maul. \bibverse{20} Wer seinem Vater und seiner Mutter flucht,
des Leuchte wird verlöschen mitten in der Finsternis. \footnote{\textbf{20:20}
  2Mo 21,17} \bibverse{21} Das Erbe, darnach man zuerst sehr eilt, wird
zuletzt nicht gesegnet sein. \bibverse{22} Sprich nicht: Ich will Böses
vergelten! Harre des HErrn, der wird dir helfen. \bibverse{23}
Mancherlei Gewicht ist ein Gräuel dem HErrn, und eine falsche Waage ist
nicht gut. \footnote{\textbf{20:23} Spr 20,10} \bibverse{24} Jedermanns
Gänge kommen vom HErrn. Welcher Mensch versteht seinen Weg?
\bibverse{25} Es ist dem Menschen ein Strick, sich mit Heiligem
übereilen und erst nach dem Geloben überlegen. \bibverse{26} Ein weiser
König zerstreut die Gottlosen und bringt das Rad über sie. \bibverse{27}
Eine Leuchte des HErrn ist des Menschen Geist; die geht durch alle
Kammern des Leibes. \footnote{\textbf{20:27} 1Kor 2,11} \bibverse{28}
Fromm und wahrhaftig sein behütet den König, und sein Thron besteht
durch Frömmigkeit. \footnote{\textbf{20:28} Spr 16,12} \bibverse{29} Der
Jünglinge Stärke ist ihr Preis; und graues Haar ist der Alten Schmuck.
\footnote{\textbf{20:29} Spr 16,31} \bibverse{30} Man muss dem Bösen
wehren mit harter Strafe und mit ernsten Schlägen, die man fühlt. \# 21
\bibverse{1} Des Königs Herz ist in der Hand des HErrn wie Wasserbäche,
und er neigt es, wohin er will. \bibverse{2} Einen jeglichen dünkt sein
Weg recht; aber der HErr wägt die Herzen. \footnote{\textbf{21:2} Spr
  16,2; Spr 24,12} \bibverse{3} Wohl und recht tun ist dem HErrn lieber
denn Opfer. \footnote{\textbf{21:3} 1Sam 15,22; Jes 1,11-18; Hos 6,6}
\bibverse{4} Hoffärtige Augen und stolzer Mut, die Leuchte der
Gottlosen, ist Sünde. \footnote{\textbf{21:4} Jes 2,11} \bibverse{5} Die
Anschläge eines Emsigen bringen Überfluss; wer aber allzu jach ist, dem
wird's mangeln. \bibverse{6} Wer Schätze sammelt mit Lügen, der wird
fehlgehen und ist unter denen, die den Tod suchen. \bibverse{7} Der
Gottlosen Rauben wird sie erschrecken; denn sie wollten nicht tun, was
recht war. \bibverse{8} Wer mit Schuld beladen ist, geht krumme Wege;
wer aber rein ist, des Werk ist recht. \bibverse{9} Es ist besser,
wohnen im Winkel auf dem Dach, denn bei einem zänkischen Weibe in einem
Hause beisammen. \footnote{\textbf{21:9} Spr 21,19; Spr 25,24}
\bibverse{10} Die Seele des Gottlosen wünscht Arges und gönnt seinem
Nächsten nichts. \bibverse{11} Wenn der Spötter gestraft wird, so werden
die Unverständigen weise; und wenn man einen Weisen unterrichtet, so
wird er vernünftig. \bibverse{12} Der Gerechte hält sich weislich gegen
des Gottlosen Haus; aber die Gottlosen denken nur, Schaden zu tun.
\bibverse{13} Wer seine Ohren verstopft vor dem Schreien des Armen, der
wird auch rufen, und nicht erhört werden. \bibverse{14} Eine heimliche
Gabe stillt den Zorn, und ein Geschenk im Schoß den heftigen Grimm.
\footnote{\textbf{21:14} 1Sam 25,18} \bibverse{15} Es ist dem Gerechten
eine Freude, zu tun, was recht ist, aber eine Furcht den Übeltätern.
\bibverse{16} Ein Mensch, der vom Wege der Klugheit irrt, der wird
bleiben in der Toten Gemeinde. \bibverse{17} Wer gern in Freuden lebt,
dem wird's mangeln; und wer Wein und Öl liebt, wird nicht reich.
\bibverse{18} Der Gottlose muss für den Gerechten gegeben werden und der
Verächter für die Frommen. \footnote{\textbf{21:18} Spr 11,8}
\bibverse{19} Es ist besser, wohnen im wüsten Lande denn bei einem
zänkischen und zornigen Weibe. \footnote{\textbf{21:19} Spr 21,9}
\bibverse{20} Im Hause des Weisen ist ein lieblicher Schatz und Öl; aber
ein Narr verschlemmt es. \bibverse{21} Wer der Gerechtigkeit und Güte
nachjagt, der findet Leben, Gerechtigkeit und Ehre. \bibverse{22} Ein
Weiser gewinnt die Stadt der Starken und stürzt ihre Macht, darauf sie
sich verlässt. \footnote{\textbf{21:22} Spr 24,5} \bibverse{23} Wer
seinen Mund und seine Zunge bewahrt, der bewahrt seine Seele vor Angst.
\footnote{\textbf{21:23} Spr 13,3} \bibverse{24} Der stolz und vermessen
ist, heißt ein Spötter, der im Zorn Stolz beweist. \bibverse{25} Der
Faule stirbt über seinem Wünschen; denn seine Hände wollen nichts tun.
\footnote{\textbf{21:25} Spr 13,3} \bibverse{26} Er wünscht den ganzen
Tag; aber der Gerechte gibt, und versagt nicht. \bibverse{27} Der
Gottlosen Opfer ist ein Gräuel; denn es wird in Sünden geopfert.
\bibverse{28} Ein lügenhafter Zeuge wird umkommen; aber wer sich sagen
lässt, den lässt man auch allezeit wiederum reden. \footnote{\textbf{21:28}
  Spr 19,5; Spr 19,9} \bibverse{29} Der Gottlose fährt mit dem Kopf
hindurch; aber wer fromm ist, des Weg wird bestehen. \bibverse{30} Es
hilft keine Weisheit, kein Verstand, kein Rat wider den HErrn.
\bibverse{31} Rosse werden zum Streittage bereitet; aber der Sieg kommt
vom HErrn. \footnote{\textbf{21:31} Ps 33,17; Jes 31,1; Jes 31,3}

\hypertarget{section-9}{%
\section{22}\label{section-9}}

\bibverse{1} Ein guter Ruf ist köstlicher denn großer Reichtum, und
Gunst besser denn Silber und Gold. \footnote{\textbf{22:1} Pred 7,1}
\bibverse{2} Reiche und Arme müssen untereinander sein; der HErr hat sie
alle gemacht. \bibverse{3} Der Kluge sieht das Unglück und verbirgt
sich; die Unverständigen gehen hindurch und werden beschädigt.
\footnote{\textbf{22:3} Spr 27,12} \bibverse{4} Wo man leidet in des
HErrn Furcht, da ist Reichtum, Ehre und Leben. \bibverse{5} Stachel und
Stricke sind auf dem Wege des Verkehrten; wer aber sich davon fernhält,
bewahret sein Leben. \bibverse{6} Wie man einen Knaben gewöhnt, so lässt
er nicht davon, wenn er alt wird. \bibverse{7} Der Reiche herrscht über
die Armen; und wer borgt, ist des Leihers Knecht. \bibverse{8} Wer
Unrecht sät, der wird Mühsal ernten und wird durch die Rute seiner
Bosheit umkommen. \footnote{\textbf{22:8} Hi 4,8} \bibverse{9} Ein
gütiges Auge wird gesegnet; denn er gibt von seinem Brot den Armen.
\footnote{\textbf{22:9} Spr 19,17} \bibverse{10} Treibe den Spötter aus,
so geht der Zank weg, so hört auf Hader und Schmähung. \footnote{\textbf{22:10}
  Spr 26,20; 1Mo 21,9-10} \bibverse{11} Wer ein treues Herz und
liebliche Rede hat, des Freund ist der König. \footnote{\textbf{22:11}
  Ps 101,6} \bibverse{12} Die Augen des HErrn behüten guten Rat; aber
die Worte des Verächters verkehrt er. \bibverse{13} Der Faule spricht:
Es ist ein Löwe draußen, ich möchte erwürgt werden auf der Gasse.
\footnote{\textbf{22:13} Spr 26,13} \bibverse{14} Der Huren Mund ist
eine tiefe Grube; wem der HErr ungnädig ist, der fällt hinein.
\footnote{\textbf{22:14} Spr 5,3-4; Spr 23,27} \bibverse{15} Torheit
steckt dem Knaben im Herzen; aber die Rute der Zucht wird sie fern von
ihm treiben. \footnote{\textbf{22:15} Spr 23,14; Spr 29,17}
\bibverse{16} Wer dem Armen Unrecht tut, dass seines Guts viel werde,
der wird auch einem Reichen geben, und Mangel haben. \bibverse{17} Neige
deine Ohren und höre die Worte der Weisen und nimm zu Herzen meine
Lehre. \bibverse{18} Denn es wird dir sanft tun, wo du sie wirst im
Sinne behalten, und sie werden miteinander durch deinen Mund wohl
geraten. \bibverse{19} Dass deine Hoffnung sei auf den HErrn, erinnere
ich dich an solches heute dir zugut. \bibverse{20} Habe ich dir's nicht
mannigfaltig vorgeschrieben mit Raten und Lehren, \bibverse{21} dass ich
dir zeigte einen gewissen Grund der Wahrheit, dass du recht antworten
könntest denen, die dich senden? \bibverse{22} Beraube den Armen nicht,
ob er wohl arm ist, und unterdrücke den Elenden nicht im Tor.
\bibverse{23} Denn der HErr wird ihre Sache führen und wird ihre
Untertreter untertreten. \bibverse{24} Geselle dich nicht zum Zornigen
und halte dich nicht zu einem grimmigen Mann; \footnote{\textbf{22:24}
  Spr 29,22} \bibverse{25} du möchtest seinen Weg lernen und an deiner
Seele Schaden nehmen. \bibverse{26} Sei nicht bei denen, die ihre Hand
verhaften und für Schuld Bürge werden; \bibverse{27} denn wo du es nicht
hast, zu bezahlen, so wird man dir dein Bett unter dir wegnehmen.
\bibverse{28} Verrücke nicht die vorigen Grenzen, die deine Väter
gemacht haben. \footnote{\textbf{22:28} Spr 23,10; 5Mo 27,17}
\bibverse{29} Siehst du einen Mann behend in seinem Geschäft, der wird
vor den Königen stehen und wird nicht stehen vor den Unedlen.
\footnote{\textbf{22:29} Spr 21,5}

\hypertarget{section-10}{%
\section{23}\label{section-10}}

\bibverse{1} Wenn du sitzest und issest mit einem Herrn, so merke, wen
du vor dir hast, \bibverse{2} und setze ein Messer an deine Kehle, wenn
du gierig bist. \bibverse{3} Wünsche dir nichts von seinen feinen
Speisen; denn es ist falsches Brot. \bibverse{4} Bemühe dich nicht,
reich zu werden, und lass ab von deinen Fündlein. \footnote{\textbf{23:4}
  Spr 28,22; Pred 9,11} \bibverse{5} Lass deine Augen nicht fliegen nach
dem, was du nicht haben kannst; denn dasselbe macht sich Flügel wie ein
Adler und fliegt gen Himmel. \bibverse{6} Iss nicht Brot bei einem
Neidischen und wünsche dir von seinen feinen Speisen nichts.
\bibverse{7} Denn wie ein Gespenst ist er inwendig; er spricht: Iss und
trink! und sein Herz ist doch nicht mit dir. \bibverse{8} Deine Bissen
die du gegessen hattest, musst du ausspeien, und musst deine
freundlichen Worte verloren haben. \bibverse{9} Rede nicht vor des
Narren Ohren; denn er verachtet die Klugheit deiner Rede. \bibverse{10}
Verrücke nicht die vorigen Grenzen und gehe nicht auf der Waisen Acker.
\footnote{\textbf{23:10} Spr 22,28} \bibverse{11} Denn ihr Erlöser ist
mächtig; der wird ihre Sache wider dich ausführen. \bibverse{12} Gib
dein Herz zur Zucht und deine Ohren zu vernünftiger Rede. \bibverse{13}
Lass nicht ab, den Knaben zu züchtigen; denn wenn du ihn mit der Rute
haust, so wird man ihn nicht töten. \bibverse{14} Du haust ihn mit der
Rute; aber du errettest seine Seele vom Tod. \bibverse{15} Mein Sohn,
wenn dein Herz weise ist, so freut sich auch mein Herz; \bibverse{16}
und meine Nieren sind froh, wenn deine Lippen reden, was recht ist.
\bibverse{17} Dein Herz folge nicht den Sündern, sondern sei täglich in
der Furcht des HErrn. \bibverse{18} Denn es wird dir hernach gut sein,
und dein Warten wird nicht trügen. \bibverse{19} Höre, mein Sohn, und
sei weise und richte dein Herz in den Weg. \bibverse{20} Sei nicht unter
den Säufern und Schlemmern; \footnote{\textbf{23:20} Lk 21,34}
\bibverse{21} denn die Säufer und Schlemmer verarmen, und ein Schläfer
muss zerrissene Kleider tragen. \bibverse{22} Gehorche deinem Vater, der
dich gezeugt hat, und verachte deine Mutter nicht, wenn sie alt wird.
\bibverse{23} Kaufe Wahrheit, und verkaufe sie nicht, Weisheit, Zucht
und Verstand. \bibverse{24} Der Vater eines Gerechten freut sich; und
wer einen Weisen gezeugt hat, ist fröhlich darüber. \bibverse{25} Lass
sich deinen Vater und deine Mutter freuen, und fröhlich sein, die dich
geboren hat. \bibverse{26} Gib mir, mein Sohn, dein Herz, und lass
deinen Augen meine Wege wohl gefallen. \bibverse{27} Denn eine Hure ist
eine tiefe Grube, und eine Ehebrecherin ist ein enger Brunnen.
\footnote{\textbf{23:27} Spr 22,14} \bibverse{28} Auch lauert sie wie
ein Räuber, und die Frechen unter den Menschen sammelt sie zu sich.
\footnote{\textbf{23:28} Spr 7,12} \bibverse{29} Wo ist Weh? wo ist
Leid? wo ist Zank? wo ist Klagen? wo sind Wunden ohne Ursache? wo sind
trübe Augen? \footnote{\textbf{23:29} Spr 21,17; Spr 20,13}
\bibverse{30} Wo man beim Wein liegt und kommt, auszusaufen, was
eingeschenkt ist. \footnote{\textbf{23:30} Spr 20,1; Jes 5,11; Jes 5,22}
\bibverse{31} Sieh den Wein nicht an, dass er so rot ist und im Glase so
schön steht. Er geht glatt ein; \bibverse{32} aber darnach beißt er wie
eine Schlange und sticht wie eine Otter. \bibverse{33} So werden deine
Augen nach anderen Weibern sehen, und dein Herz wird verkehrte Dinge
reden, \bibverse{34} und wirst sein wie einer, der mitten im Meer
schläft, und wie einer schläft oben auf dem Mastbaum. \bibverse{35} „Sie
schlagen mich, aber es tut mir nicht weh; sie klopfen mich, aber ich
fühle es nicht. Wann will ich aufwachen, dass ich's mehr treibe?{}``
\footnote{\textbf{23:35} Jes 56,12}

\hypertarget{section-11}{%
\section{24}\label{section-11}}

\bibverse{1} Folge nicht bösen Leuten und wünsche nicht, bei ihnen zu
sein; \bibverse{2} denn ihr Herz trachtet nach Schaden, und ihre Lippen
raten zu Unglück. \bibverse{3} Durch Weisheit wird ein Haus gebaut und
durch Verstand erhalten. \bibverse{4} Durch ordentliches Haushalten
werden die Kammern voll aller köstlichen, lieblichen Reichtümer.
\bibverse{5} Ein weiser Mann ist stark, und ein vernünftiger Mann ist
mächtig von Kräften. \bibverse{6} Denn mit Rat muss man Krieg führen;
und wo viele Ratgeber sind, da ist der Sieg. \footnote{\textbf{24:6} Spr
  20,18; Spr 11,14} \bibverse{7} Weisheit ist dem Narren zu hoch; er
darf seinen Mund im Tor nicht auftun. \bibverse{8} Wer sich vornimmt,
Böses zu tun, den heißt man billig einen Erzbösewicht. \bibverse{9} Des
Narren Tücke ist Sünde, und der Spötter ist ein Gräuel vor den Leuten.
\bibverse{10} Der ist nicht stark, der in der Not nicht fest ist.
\bibverse{11} Errette die, die man töten will; und entzieh dich nicht
von denen, die man würgen will. \bibverse{12} Sprichst du: „Siehe, wir
verstehen's nicht!{}``, meinst du nicht, der die Herzen wägt, merkt es,
und der auf deine Seele achthat, kennt es und vergilt dem Menschen nach
seinem Werk? \footnote{\textbf{24:12} Spr 16,2; 1Sam 16,7; Röm 2,6}
\bibverse{13} Iss, mein Sohn, Honig, denn er ist gut, und Honigseim ist
süß in deinem Halse. \bibverse{14} Also lerne die Weisheit für deine
Seele. Wo du sie findest, so wird's hernach wohl gehen, und deine
Hoffnung wird nicht umsonst sein. \bibverse{15} Laure nicht als ein
Gottloser auf das Haus des Gerechten; verstöre seine Ruhe nicht.
\bibverse{16} Denn ein Gerechter fällt siebenmal und steht wieder auf;
aber die Gottlosen versinken im Unglück. \footnote{\textbf{24:16} Hi
  5,19; Ps 37,24} \bibverse{17} Freue dich des Falles deines Feindes
nicht, und dein Herz sei nicht froh über seinem Unglück; \footnote{\textbf{24:17}
  Hi 31,29} \bibverse{18} der HErr möchte es sehen, und es möchte ihm
übel gefallen und er seinen Zorn von ihm wenden. \bibverse{19} Erzürne
dich nicht über die Bösen und eifere nicht über die Gottlosen.
\footnote{\textbf{24:19} Spr 3,31; Ps 37,1; Ps 73,3} \bibverse{20} Denn
der Böse hat nichts zu hoffen, und die Leuchte der Gottlosen wird
verlöschen. \footnote{\textbf{24:20} Spr 13,9} \bibverse{21} Mein Kind,
fürchte den HErrn und den König und menge dich nicht unter die
Aufrührer. \footnote{\textbf{24:21} 1Petr 2,17} \bibverse{22} Denn ihr
Verderben wird plötzlich entstehen; und wer weiß, wann beider Unglück
kommt? \footnote{\textbf{24:22} Röm 13,2} \bibverse{23} Dies sind auch
Worte von Weisen. Die Person ansehen im Gericht ist nicht gut.
\footnote{\textbf{24:23} 3Mo 19,15} \bibverse{24} Wer zum Gottlosen
spricht: „Du bist fromm``, dem fluchen die Leute, und das Volk hasst
ihn. \bibverse{25} Welche aber strafen, die gefallen wohl, und kommt ein
reicher Segen auf sie. \bibverse{26} Eine richtige Antwort ist wie ein
lieblicher Kuss. \bibverse{27} Richte draußen dein Geschäft aus und
bearbeite deinen Acker; darnach baue dein Haus. \bibverse{28} Sei nicht
Zeuge ohne Ursache wider deinen Nächsten und betrüge nicht mit deinem
Munde. \footnote{\textbf{24:28} Spr 19,5} \bibverse{29} Sprich nicht:
„Wie man mir tut, so will ich wieder tun und einem jeglichen sein Werk
vergelten.`` \footnote{\textbf{24:29} Spr 20,22} \bibverse{30} Ich ging
am Acker des Faulen vorüber und am Weinberg des Narren; \bibverse{31}
und siehe, da waren eitel Nesseln darauf, und er stand voll Disteln, und
die Mauer war eingefallen. \bibverse{32} Da ich das sah, nahm ich's zu
Herzen und schaute und lernte daran. \bibverse{33} Du willst ein wenig
schlafen und ein wenig schlummern und ein wenig die Hände zusammentun,
dass du ruhest: \footnote{\textbf{24:33} Spr 6,9-11} \bibverse{34} aber
es wird dir deine Armut kommen wie ein Wanderer und dein Mangel wie ein
gewappneter Mann. \footnote{\textbf{24:34} Spr 10,4}

\hypertarget{section-12}{%
\section{25}\label{section-12}}

\bibverse{1} Dies sind auch Sprüche Salomos, die hinzugesetzt haben die
Männer Hiskias, des Königs in Juda. \bibverse{2} Es ist Gottes Ehre,
eine Sache verbergen; aber der Könige Ehre ist's, eine Sache erforschen.
\bibverse{3} Der Himmel ist hoch und die Erde tief; aber der Könige Herz
ist unerforschlich. \bibverse{4} Man tue den Schaum vom Silber, so wird
ein reines Gefäß daraus. \bibverse{5} Man tue den Gottlosen hinweg vor
dem König, so wird sein Thron mit Gerechtigkeit befestigt. \footnote{\textbf{25:5}
  Spr 16,12} \bibverse{6} Prange nicht vor dem König und tritt nicht an
den Ort der Großen. \bibverse{7} Denn es ist dir besser, dass man zu dir
sage: Tritt hier herauf! als dass du vor dem Fürsten erniedrigt wirst,
dass es deine Augen sehen müssen. \bibverse{8} Fahre nicht bald heraus,
zu zanken; denn was willst du hernach machen, wenn dich dein Nächster
beschämt hat? \bibverse{9} Führe deine Sache mit deinem Nächsten, und
offenbare nicht eines anderen Heimlichkeit, \footnote{\textbf{25:9} Spr
  20,19} \bibverse{10} auf dass nicht übel von dir spreche, der es hört,
und dein böses Gerücht nimmer ablasse. \bibverse{11} Ein Wort, geredet
zu seiner Zeit, ist wie goldene Äpfel auf silbernen Schalen.
\bibverse{12} Wer einem Weisen gehorcht, der ihn straft, das ist wie ein
goldenes Stirnband und goldenes Halsband. \bibverse{13} Wie die Kühle
des Schnees zur Zeit der Ernte, so ist ein getreuer Bote dem, der ihn
gesandt hat, und erquickt seines Herrn Seele. \bibverse{14} Wer viel
verspricht, und hält nicht, der ist wie Wolken und Wind ohne Regen.
\footnote{\textbf{25:14} 2Petr 2,17} \bibverse{15} Durch Geduld wird ein
Fürst versöhnt, und eine linde Zunge bricht die Härtigkeit. \footnote{\textbf{25:15}
  Spr 15,1} \bibverse{16} Findest du Honig, so iss davon, so viel dir
genug ist, dass du nicht zu satt werdest und speiest ihn aus.
\bibverse{17} Entzieh deinen Fuß vom Hause deines Nächsten; er möchte
dein überdrüssig und dir gram werden. \bibverse{18} Wer wider seinen
Nächsten falsch Zeugnis redet, der ist ein Spieß, Schwert und scharfer
Pfeil. \footnote{\textbf{25:18} Spr 19,5} \bibverse{19} Die Hoffnung auf
einen Treulosen zur Zeit der Not ist wie ein fauler Zahn und gleitender
Fuß. \bibverse{20} Wer einem betrübten Herzen Lieder singt, das ist, wie
wenn einer das Kleid ablegt am kalten Tage, und wie Essig auf der
Kreide. \bibverse{21} Hungert deinen Feind, so speise ihn mit Brot;
dürstet ihn, so tränke ihn mit Wasser. \bibverse{22} Denn du wirst
feurige Kohlen auf sein Haupt häufen, und der HErr wird dir's vergelten.
\bibverse{23} Der Nordwind bringt Ungewitter, und die heimliche Zunge
macht saures Angesicht. \bibverse{24} Es ist besser, im Winkel auf dem
Dache sitzen, denn bei einem zänkischen Weibe in einem Hause beisammen.
\footnote{\textbf{25:24} Spr 21,9; Spr 21,19} \bibverse{25} Eine gute
Botschaft aus fernen Landen ist wie kalt Wasser einer durstigen Seele.
\bibverse{26} Ein Gerechter, der vor einem Gottlosen fällt, ist wie ein
getrübter Brunnen und eine verderbte Quelle. \bibverse{27} Wer zuviel
Honig isst, das ist nicht gut; und wer schwere Dinge erforscht, dem
wird's zu schwer. \bibverse{28} Ein Mann, der seinen Geist nicht halten
kann, ist wie eine offene Stadt ohne Mauern. \# 26 \bibverse{1} Wie der
Schnee im Sommer und Regen in der Ernte, also reimt sich dem Narren Ehre
nicht. \footnote{\textbf{26:1} Spr 26,8} \bibverse{2} Wie ein Vogel
dahinfährt und eine Schwalbe fliegt, also ein unverdienter Fluch trifft
nicht. \bibverse{3} Dem Ross eine Geißel und dem Esel einen Zaum und dem
Narren eine Rute auf den Rücken! \bibverse{4} Antworte dem Narren nicht
nach seiner Narrheit, dass du ihm nicht auch gleich werdest.
\bibverse{5} Antworte aber dem Narren nach seiner Narrheit, dass er sich
nicht weise lasse dünken. \bibverse{6} Wer eine Sache durch einen
törichten Boten ausrichtet, der ist wie ein Lahmer an den Füßen und
nimmt Schaden. \bibverse{7} Wie einem Krüppel das Tanzen, also steht den
Narren an, von Weisheit zu reden. \bibverse{8} Wer einem Narren Ehre
antut, das ist, als wenn einer einen edlen Stein auf den Rabenstein
würfe. \footnote{\textbf{26:8} Spr 26,1} \bibverse{9} Ein Spruch in
eines Narren Mund ist wie ein Dornzweig, der in eines Trunkenen Hand
sticht. \bibverse{10} Ein guter Meister macht ein Ding recht; aber wer
einen Stümper dingt, dem wird's verderbt. \bibverse{11} Wie ein Hund
sein Gespeites wieder frisst, also ist der Narr, der seine Narrheit
wieder treibt. \bibverse{12} Wenn du einen siehst, der sich weise dünkt,
da ist an einem Narren mehr Hoffnung denn an ihm. \footnote{\textbf{26:12}
  Spr 3,7} \bibverse{13} Der Faule spricht: Es ist ein junger Löwe auf
dem Wege und ein Löwe auf den Gassen. \footnote{\textbf{26:13} Spr 22,13}
\bibverse{14} Ein Fauler wendet sich im Bette wie die Tür in der Angel.
\footnote{\textbf{26:14} Spr 6,9-11} \bibverse{15} Der Faule verbirgt
seine Hand in dem Topf, und wird ihm sauer, dass er sie zum Munde
bringe. \footnote{\textbf{26:15} Spr 19,24} \bibverse{16} Ein Fauler
dünket sich weiser denn sieben, die da Sitten lehren. \bibverse{17} Wer
vorgeht und sich mengt in fremden Hader, der ist wie einer, der den Hund
bei den Ohren zwackt. \bibverse{18} Wie ein Unsinniger mit Geschoss und
Pfeilen schießt und tötet, \bibverse{19} also tut ein falscher Mensch
mit seinem Nächsten und spricht darnach: Ich habe gescherzt.
\bibverse{20} Wenn nimmer Holz da ist, so verlischt das Feuer; und wenn
der Verleumder weg ist, so hört der Hader auf. \footnote{\textbf{26:20}
  Spr 22,10} \bibverse{21} Wie die Kohlen eine Glut und Holz ein Feuer,
also facht ein zänkischer Mann Hader an. \footnote{\textbf{26:21} Spr
  15,18} \bibverse{22} Die Worte des Verleumders sind wie Schläge, und
sie gehen durchs Herz. \footnote{\textbf{26:22} Spr 18,8} \bibverse{23}
Brünstige Lippen und böses Herz ist wie eine Scherbe, mit Silberschaum
überzogen. \bibverse{24} Der Feind verstellt sich mit seiner Rede, und
im Herzen ist er falsch. \bibverse{25} Wenn er seine Stimme holdselig
macht, so glaube ihm nicht; denn es sind sieben Gräuel in seinem Herzen.
\bibverse{26} Wer den Hass heimlich hält, Schaden zu tun, des Bosheit
wird vor der Gemeinde offenbar werden. \bibverse{27} Wer eine Grube
macht, der wird hineinfallen; und wer einen Stein wälzt, auf den wird er
zurückkommen. \bibverse{28} Eine falsche Zunge hasst den, der sie
straft; und ein Heuchelmaul richtet Verderben an. \# 27 \bibverse{1}
Rühme dich nicht des morgenden Tages; denn du weißt nicht, was heute
sich begeben mag. \footnote{\textbf{27:1} Jak 4,13; Jak 1,4-14}
\bibverse{2} Lass dich einen anderen loben, und nicht deinen Mund, --
einen Fremden, und nicht deine eigenen Lippen. \footnote{\textbf{27:2}
  2Kor 10,12} \bibverse{3} Stein ist schwer und Sand ist Last; aber des
Narren Zorn ist schwerer denn die beiden. \bibverse{4} Zorn ist ein
wütig Ding, und Grimm ist ungestüm; aber wer kann vor dem Neid bestehen?
\bibverse{5} Offene Strafe ist besser denn heimliche Liebe. \bibverse{6}
Die Schläge des Liebhabers meinen's recht gut; aber die Küsse des
Hassers sind gar zu reichlich. \footnote{\textbf{27:6} Ps 141,5}
\bibverse{7} Eine satte Seele zertritt wohl Honigseim; aber einer
hungrigen Seele ist alles Bittere süß. \bibverse{8} Wie ein Vogel, der
aus seinem Nest weicht, also ist, wer von seiner Stätte weicht.
\bibverse{9} Das Herz freut sich an Salbe und Räuchwerk; aber ein Freund
ist lieblich um Rats willen der Seele. \bibverse{10} Deinen Freund und
deines Vaters Freund verlass nicht, und gehe nicht ins Haus deines
Bruders, wenn dir's übel geht; denn dein Nachbar in der Nähe ist besser
als ein Bruder in der Ferne. \bibverse{11} Sei weise, mein Sohn, so
freut sich mein Herz, so will ich antworten dem, der mich schmäht.
\bibverse{12} Ein Kluger sieht das Unglück und verbirgt sich; aber die
Unverständigen gehen hindurch und leiden Schaden. \footnote{\textbf{27:12}
  Spr 21,29; Spr 22,3} \bibverse{13} Nimm dem sein Kleid, der für einen
anderen Bürge wird, und pfände ihn um der Fremden willen. \footnote{\textbf{27:13}
  Spr 20,16} \bibverse{14} Wenn einer seinen Nächsten des Morgens früh
mit lauter Stimme segnet, das wird ihm für einen Fluch gerechnet.
\bibverse{15} Ein zänkisches Weib und stetiges Triefen, wenn's sehr
regnet, werden wohl miteinander verglichen. \footnote{\textbf{27:15} Spr
  19,13; Spr 25,24} \bibverse{16} Wer sie aufhält, der hält den Wind und
will das Öl mit der Hand fassen. \bibverse{17} Ein Messer wetzt das
andere und ein Mann den anderen. \bibverse{18} Wer seinen Feigenbaum
bewahrt, der isst Früchte davon; und wer seinen Herrn bewahrt, wird
geehrt. \bibverse{19} Wie das Spiegelbild im Wasser ist gegenüber dem
Angesicht, also ist eines Menschen Herz gegenüber dem anderen.
\bibverse{20} Hölle und Abgrund werden nimmer voll, und der Menschen
Augen sind auch unersättlich. \bibverse{21} Ein Mann wird durch den Mund
des, der ihn lobt, bewährt wie Silber im Tiegel und das Gold im Ofen.
\bibverse{22} Wenn du den Narren im Mörser zerstießest mit dem Stämpfel
wie Grütze, so ließe doch seine Narrheit nicht von ihm. \bibverse{23}
Auf deine Schafe habe Acht und nimm dich deiner Herden an. \bibverse{24}
Denn Gut währt nicht ewiglich, und die Krone währt nicht für und für.
\footnote{\textbf{27:24} 1Tim 6,7} \bibverse{25} Das Heu ist weggeführt,
und wiederum ist Gras da und wird Kraut auf den Bergen gesammelt.
\bibverse{26} Die Lämmer kleiden dich, und die Böcke geben dir das Geld,
einen Acker zu kaufen. \bibverse{27} Du hast Ziegenmilch genug zu deiner
Speise, zur Speise deines Hauses und zur Nahrung deiner Dirnen. \# 28
\bibverse{1} Der Gottlose flieht, und niemand jagt ihn; der Gerechte
aber ist getrost wie ein junger Löwe. \bibverse{2} Um des Landes Sünde
willen werden viel Änderungen der Fürstentümer; aber um der Leute
willen, die verständig und vernünftig sind, bleiben sie lange.
\bibverse{3} Ein armer Mann, der die Geringen bedrückt, ist wie ein
Mehltau, der die Frucht verdirbt. \bibverse{4} Die das Gesetz verlassen,
loben den Gottlosen; die es aber bewahren, sind unwillig auf sie.
\bibverse{5} Böse Leute merken nicht aufs Recht; die aber nach dem HErrn
fragen, merken auf alles. \bibverse{6} Es ist besser ein Armer, der in
seiner Frömmigkeit geht, denn ein Reicher, der in verkehrten Wegen geht.
\footnote{\textbf{28:6} Spr 19,1} \bibverse{7} Wer das Gesetz bewahrt,
ist ein verständiges Kind; wer aber der Schlemmer Geselle ist, schändet
seinen Vater. \bibverse{8} Wer sein Gut mehrt mit Wucher und Zins, der
sammelt es für den, der sich der Armen erbarmt. \bibverse{9} Wer sein
Ohr abwendet, das Gesetz zu hören, des Gebet ist ein Gräuel. \footnote{\textbf{28:9}
  Spr 21,27} \bibverse{10} Wer die Frommen verführt auf bösem Wege, der
wird in seine Grube fallen; aber die Frommen werden Gutes ererben.
\bibverse{11} Ein Reicher dünkt sich, weise zu sein; aber ein
verständiger Armer durchschaut ihn. \bibverse{12} Wenn die Gerechten
Oberhand haben, so geht's sehr fein zu; wenn aber Gottlose aufkommen,
wendet sich's unter den Leuten. \bibverse{13} Wer seine Missetat
leugnet, dem wird es nicht gelingen; wer sie aber bekennt und lässt, der
wird Barmherzigkeit erlangen. \footnote{\textbf{28:13} Ps 32,3-5; 1Jo
  1,8; 1Jo 1,1-9} \bibverse{14} Wohl dem, der sich allewege fürchtet;
wer aber sein Herz verhärtet, wird in Unglück fallen. \bibverse{15} Ein
Gottloser, der über ein armes Volk regiert, das ist ein brüllender Löwe
und gieriger Bär. \bibverse{16} Wenn ein Fürst ohne Verstand ist, so
geschieht viel Unrecht; wer aber den Geiz hasst, der wird lange leben.
\bibverse{17} Ein Mensch, der am Blut einer Seele schuldig ist, der wird
flüchtig sein bis zur Grube, und niemand halte ihn auf. \bibverse{18}
Wer fromm einhergeht, dem wird geholfen; wer aber verkehrtes Weges ist,
wird auf einmal fallen. \bibverse{19} Wer seinen Acker baut, wird Brot
genug haben; wer aber dem Müßiggang nachgeht, wird Armut genug haben.
\footnote{\textbf{28:19} Spr 6,6-11; Spr 10,4; Spr 12,11} \bibverse{20}
Ein treuer Mann wird viel gesegnet; wer aber eilt, reich zu werden, wird
nicht unschuldig bleiben. \footnote{\textbf{28:20} Spr 28,22; Spr 20,21}
\bibverse{21} Person ansehen ist nicht gut; und mancher tut übel auch
wohl um ein Stück Brot. \bibverse{22} Wer eilt zum Reichtum und ist
neidisch, der weiß nicht, dass Mangel ihm begegnen wird. \footnote{\textbf{28:22}
  Spr 28,20; Spr 23,4; 1Tim 6,9} \bibverse{23} Wer einen Menschen
straft, wird hernach Gunst finden, mehr denn der da heuchelt.
\bibverse{24} Wer seinem Vater oder seiner Mutter etwas nimmt und
spricht, es sei nicht Sünde, der ist des Verderbers Geselle.
\bibverse{25} Ein Stolzer erweckt Zank; wer aber auf den HErrn sich
verlässt, wird gelabt. \bibverse{26} Wer sich auf sein Herz verlässt,
ist ein Narr; wer aber mit Weisheit geht, wird entrinnen. \footnote{\textbf{28:26}
  Spr 3,5} \bibverse{27} Wer dem Armen gibt, dem wird nichts mangeln;
wer aber seine Augen abwendet, der wird viel verflucht. \footnote{\textbf{28:27}
  2Kor 9,6; 2Kor 9,9} \bibverse{28} Wenn die Gottlosen aufkommen, so
verbergen sich die Leute; wenn sie aber umkommen, werden der Gerechten
viel. \footnote{\textbf{28:28} Spr 29,2}

\hypertarget{section-13}{%
\section{29}\label{section-13}}

\bibverse{1} Wer wider die Strafe halsstarrig ist, der wird plötzlich
verderben ohne alle Hilfe. \footnote{\textbf{29:1} Spr 15,10}
\bibverse{2} Wenn der Gerechten viel sind, freut sich das Volk; wenn
aber der Gottlose herrscht, seufzt das Volk. \footnote{\textbf{29:2} Spr
  11,10} \bibverse{3} Wer Weisheit liebt, erfreut seinen Vater; wer aber
mit Huren umgeht, kommt um sein Gut. \footnote{\textbf{29:3} Lk 15,13}
\bibverse{4} Ein König richtet das Land auf durchs Recht; ein geiziger
aber verderbt es. \footnote{\textbf{29:4} Jes 32,7} \bibverse{5} Wer mit
seinem Nächsten heuchelt, der breitet ein Netz aus für seine Tritte.
\bibverse{6} Wenn ein Böser sündigt, verstrickt er sich selbst; aber ein
Gerechter freut sich und hat Wonne. \bibverse{7} Der Gerechte erkennt
die Sache der Armen; der Gottlose achtet keine Vernunft. \bibverse{8}
Die Spötter bringen frech eine Stadt in Aufruhr; aber die Weisen stillen
den Zorn. \bibverse{9} Wenn ein Weiser mit einem Narren zu rechten
kommt, er zürne oder lache, so hat er nicht Ruhe. \bibverse{10} Die
Blutgierigen hassen den Frommen; aber die Gerechten suchen sein Heil.
\bibverse{11} Ein Narr schüttet seinen Geist ganz aus; aber ein Weiser
hält an sich. \footnote{\textbf{29:11} Spr 25,28; Spr 12,23}
\bibverse{12} Ein Herr, der zu Lügen Lust hat, des Diener sind alle
gottlos. \bibverse{13} Arme und Reiche begegnen einander; beider Augen
erleuchtet der HErr. \bibverse{14} Ein König, der die Armen treulich
richtet, des Thron wird ewig bestehen. \bibverse{15} Rute und Strafe
gibt Weisheit; aber ein Knabe, sich selbst überlassen, macht seiner
Mutter Schande. \footnote{\textbf{29:15} Spr 29,17; Spr 22,15}
\bibverse{16} Wo viele Gottlose sind, da sind viel Sünden; aber die
Gerechten werden ihren Fall erleben. \footnote{\textbf{29:16} Ps 37,36}
\bibverse{17} Züchtige deinen Sohn, so wird er dich ergötzen und wird
deiner Seele sanft tun. \footnote{\textbf{29:17} Spr 23,13}
\bibverse{18} Wo keine Weissagung ist, wird das Volk wild und wüst; wohl
aber dem, der das Gesetz handhabt! \bibverse{19} Ein Knecht lässt sich
mit Worten nicht züchtigen; denn ob er's gleich versteht, nimmt er
sich's doch nicht an. \bibverse{20} Siehst du einen, der schnell ist, zu
reden, da ist am Narren mehr Hoffnung denn an ihm. \bibverse{21} Wenn
ein Knecht von Jugend auf zärtlich gehalten wird, so will er darnach ein
Junker sein. \bibverse{22} Ein zorniger Mann richtet Hader an, und ein
Grimmiger tut viel Sünde. \footnote{\textbf{29:22} Spr 15,18; Spr 26,21}
\bibverse{23} Die Hoffart des Menschen wird ihn stürzen; aber der
Demütige wird Ehre empfangen. \footnote{\textbf{29:23} Mt 23,12; 1Petr
  5,5} \bibverse{24} Wer mit Dieben teilhat, den Fluch aussprechen hört,
und sagt's nicht an, der hasst sein Leben. \footnote{\textbf{29:24} 3Mo
  5,1} \bibverse{25} Vor Menschen sich scheuen bringt zu Fall; wer sich
aber auf den HErrn verlässt, wird beschützt. \bibverse{26} Viele suchen
das Angesicht eines Fürsten; aber eines jeglichen Gericht kommt vom
HErrn. \bibverse{27} Ein ungerechter Mann ist dem Gerechten ein Gräuel;
und wer rechtes Weges ist, der ist des Gottlosen Gräuel. \# 30
\bibverse{1} Dies sind die Worte Agurs, des Sohnes Jakes. Lehre und Rede
des Mannes: „Ich habe mich gemüht, o Gott; ich habe mich gemüht, o Gott,
und ablassen müssen. \bibverse{2} Denn ich bin der allernärrischste, und
Menschenverstand ist nicht bei mir; \bibverse{3} ich habe Weisheit nicht
gelernt, dass ich den Heiligen erkennete. \bibverse{4} Wer fährt hinauf
gen Himmel und herab? Wer fasst den Wind in seine Hände? Wer bindet die
Wasser in ein Kleid? Wer hat alle Enden der Welt gestellt? Wie heißt er?
Und wie heißt sein Sohn? Weißt du das?{}`` \bibverse{5} Alle Worte
Gottes sind durchläutert; er ist ein Schild denen, die auf ihn trauen.
\footnote{\textbf{30:5} Ps 12,7; Ps 18,31} \bibverse{6} Tue nichts zu
seinen Worten, dass er dich nicht strafe und werdest lügenhaft erfunden.
\footnote{\textbf{30:6} 5Mo 4,2} \bibverse{7} Zweierlei bitte ich von
dir; das wollest du mir nicht weigern, ehe denn ich sterbe: \bibverse{8}
Abgötterei und Lüge lass ferne von mir sein; Armut und Reichtum gib mir
nicht, lass mich aber mein beschieden Teil Speise dahinnehmen.
\footnote{\textbf{30:8} 1Tim 6,6-8; Mt 6,11} \bibverse{9} Ich möchte
sonst, wo ich zu satt würde, verleugnen und sagen: Wer ist der HErr?
Oder wo ich zu arm würde, möchte ich stehlen und mich an dem Namen
meines Gottes vergreifen. \bibverse{10} Verleumde den Knecht nicht bei
seinem Herrn, dass er dir nicht fluche und du die Schuld tragen müssest.
\bibverse{11} Es ist eine Art, die ihrem Vater flucht und ihre Mutter
nicht segnet; \bibverse{12} eine Art, die sich rein dünkt, und ist doch
von ihrem Kot nicht gewaschen; \bibverse{13} eine Art, die ihre Augen
hoch trägt und ihre Augenlider emporhält; \footnote{\textbf{30:13} Spr
  21,4} \bibverse{14} eine Art, die Schwerter für Zähne hat und Messer
für Backenzähne und verzehrt die Elenden im Lande und die Armen unter
den Leuten. \bibverse{15} Blutegel hat zwei Töchter: Bring her, bring
her! Drei Dinge sind nicht zu sättigen, und das vierte spricht nicht: Es
ist genug: \bibverse{16} die Hölle, der Frauen verschlossene Mutter, die
Erde wird nicht Wassers satt, und das Feuer spricht nicht: Es ist genug.
\bibverse{17} Ein Auge, das den Vater verspottet, und verachtet der
Mutter zu gehorchen, das müssen die Raben am Bach aushacken und die
jungen Adler fressen. \bibverse{18} Drei sind mir zu wunderbar, und das
vierte verstehe ich nicht: \footnote{\textbf{30:18} Spr 6,16}
\bibverse{19} des Adlers Weg am Himmel, der Schlange Weg auf einem
Felsen, des Schiffes Weg mitten im Meer und eines Mannes Weg an einer
Jungfrau. \bibverse{20} Also ist auch der Weg der Ehebrecherin; die
verschlingt und wischt ihr Maul und spricht: Ich habe kein Böses getan.
\bibverse{21} Ein Land wird durch dreierlei unruhig, und das vierte kann
es nicht ertragen: \bibverse{22} ein Knecht, wenn er König wird; ein
Narr, wenn er zu satt ist; \bibverse{23} eine Verschmähte, wenn sie
geehelicht wird; und eine Magd, wenn sie ihrer Frau Erbin wird.
\bibverse{24} Vier sind klein auf Erden und klüger denn die Weisen:
\bibverse{25} die Ameisen -- ein schwaches Volk; dennoch schaffen sie im
Sommer ihre Speise --, \footnote{\textbf{30:25} Spr 6,6-8; Spr 10,5}
\bibverse{26} Kaninchen -- ein schwaches Volk; dennoch legt es sein Haus
in den Felsen --, \bibverse{27} Heuschrecken -- haben keinen König;
dennoch ziehen sie aus ganz in Haufen --, \bibverse{28} die Spinne --
wirkt mit ihren Händen und ist in der Könige Schlössern. \bibverse{29}
Dreierlei haben einen feinen Gang, und das vierte geht wohl:
\bibverse{30} der Löwe, mächtig unter den Tieren und kehrt nicht um vor
jemand; \bibverse{31} ein Windhund von guten Lenden, und ein Widder, und
ein König, wider den sich niemand darf legen. \bibverse{32} Bist du ein
Narr gewesen und zu hoch gefahren und hast Böses vorgehabt, so lege die
Hand aufs Maul. \bibverse{33} Wenn man Milch stößt, so macht man Butter
daraus; und wer die Nase hart schneuzt, zwingt Blut heraus; und wer den
Zorn reizt, zwingt Hader heraus. \# 31 \bibverse{1} Dies sind die Worte
des Königs Lamuel, die Lehre, die ihn seine Mutter lehrte. \bibverse{2}
Ach mein Auserwählter, ach du Sohn meines Leibes, ach mein gewünschter
Sohn, \bibverse{3} lass nicht den Weibern deine Kraft und gehe die Wege
nicht, darin sich die Könige verderben! \bibverse{4} O, nicht den
Königen, Lamuel, nicht den Königen ziemt es, Wein zu trinken, noch den
Fürsten starkes Getränk! \^{}\^{} \bibverse{5} Sie möchten trinken und
der Rechte vergessen und verändern die Sache aller elenden Leute.
\bibverse{6} Gebt starkes Getränk denen, die am Umkommen sind, und den
Wein den betrübten Seelen, \bibverse{7} dass sie trinken und ihres
Elends vergessen und ihres Unglücks nicht mehr gedenken. \bibverse{8}
Tue deinen Mund auf für die Stummen und für die Sache aller, die
verlassen sind. \bibverse{9} Tue deinen Mund auf und richte recht und
räche den Elenden und Armen. \bibverse{10} Wem ein tugendsam Weib
beschert ist, die ist viel edler denn die köstlichsten Perlen. \^{}\^{}
\bibverse{11} Ihres Mannes Herz darf sich auf sie verlassen, und Nahrung
wird ihm nicht mangeln. \bibverse{12} Sie tut ihm Liebes und kein Leides
ihr Leben lang. \bibverse{13} Sie geht mit Wolle und Flachs um und
arbeitet gern mit ihren Händen. \bibverse{14} Sie ist wie ein
Kaufmannsschiff, das seine Nahrung von ferne bringt. \bibverse{15} Sie
steht vor Tage auf und gibt Speise ihrem Hause und Essen ihren Dirnen.
\bibverse{16} Sie denkt nach einem Acker und kauft ihn und pflanzt einen
Weinberg von den Früchten ihrer Hände. \bibverse{17} Sie gürtet ihre
Lenden mit Kraft und stärkt ihre Arme. \bibverse{18} Sie merkt, wie ihr
Handel Frommen bringt; ihre Leuchte verlischt des Nachts nicht.
\bibverse{19} Sie streckt ihre Hand nach dem Rocken, und ihre Finger
fassen die Spindel. \bibverse{20} Sie breitet ihre Hände aus zu dem
Armen und reicht ihre Hand dem Dürftigen. \bibverse{21} Sie fürchtet für
ihr Haus nicht den Schnee; denn ihr ganzes Haus hat zwiefache Kleider.
\bibverse{22} Sie macht sich selbst Decken; feine Leinwand und Purpur
ist ihr Kleid. \bibverse{23} Ihr Mann ist bekannt in den Toren, wenn er
sitzt bei den Ältesten des Landes. \bibverse{24} Sie macht einen Rock
und verkauft ihn; einen Gürtel gibt sie dem Krämer. \bibverse{25} Kraft
und Schöne sind ihr Gewand, und sie lacht des kommenden Tages.
\bibverse{26} Sie tut ihren Mund auf mit Weisheit, und auf ihrer Zunge
ist holdselige Lehre. \bibverse{27} Sie schaut, wie es in ihrem Hause
zugeht, und isst ihr Brot nicht mit Faulheit. \bibverse{28} Ihre Söhne
stehen auf und preisen sie selig; ihr Mann lobt sie: \bibverse{29}
„Viele Töchter halten sich tugendsam; du aber übertriffst sie alle.``
\bibverse{30} Lieblich und schön sein ist nichts; ein Weib, das den
HErrn fürchtet, soll man loben. \bibverse{31} Sie wird gerühmt werden
von den Früchten ihrer Hände, und ihre Werke werden sie loben in den
Toren.
