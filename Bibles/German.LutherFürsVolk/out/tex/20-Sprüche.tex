\hypertarget{section}{%
\section{1}\label{section}}

\bibverse{1} Dies sind die Sprüche Salomos, des Königs Israels, Davids
Sohns, \bibverse{2} zu lernen Weisheit und Zucht, Verstand, \bibverse{3}
Klugheit, Gerechtigkeit, Recht und Schlecht, \bibverse{4} daß die
Albernen witzig und die Jünglinge vernünftig und vorsichtig werden.
\bibverse{5} Wer weise ist, der höret zu und bessert sich; und wer
verständig ist, der läßt ihm raten, \bibverse{6} daß er vernehme die
Sprüche und ihre Deutung, die Lehre der Weisen und ihre Beispiele.
\bibverse{7} Des HErrn Furcht ist Anfang zu lernen. Die Ruchlosen
verachten Weisheit und Zucht. \bibverse{8} Mein Kind gehorche der Zucht
deines Vaters und verlaß nicht das Gebot deiner Mutter! \bibverse{9}
Denn solches ist ein schöner Schmuck deinem Haupt und eine Kette an
deinem Halse. \bibverse{10} Mein Kind, wenn dich die bösen Buben locken,
so folge nicht! \bibverse{11} Wenn sie sagen: Gehe mit uns, wir wollen
auf Blut lauern und den Unschuldigen ohne Ursache nachstellen;
\bibverse{12} wir wollen sie lebendig verschlingen wie die Hölle, und
die Frommen, als die hinunter in die Grube fahren; \bibverse{13} wir
wollen groß Gut finden; wir wollen unsere Häuser mit Raube füllen;
\bibverse{14} wage es mit uns; es soll unser aller ein Beutel sein:
\bibverse{15} mein Kind, wandle den Weg nicht mit ihnen; wehre deinem
Fuß von ihrem Pfad! \bibverse{16} Denn ihre Füße laufen zum Bösen und
eilen, Blut zu vergießen. \bibverse{17} Denn es ist vergeblich, das Netz
auswerfen vor den Augen der Vögel. \bibverse{18} Auch lauern sie selbst
untereinander auf ihr Blut, und stellet einer dem andern nach dem Leben.
\bibverse{19} Also tun alle Geizigen, daß einer dem andern das Leben
nimmt. \bibverse{20} Die Weisheit klagt draußen und läßt sich hören auf
den Gassen. \bibverse{21} Sie ruft in der Tür am Tor vorne unter dem
Volk; sie redet ihre Worte in der Stadt: \bibverse{22} Wie lange wollt
ihr Albernen albern sein und die Spötter Lust zu Spötterei haben und die
Ruchlosen die Lehre hassen? \bibverse{23} Kehret euch zu meiner Strafe!
Siehe, ich will euch heraussagen meinen Geist und euch meine Worte
kundtun. \bibverse{24} Weil ich denn rufe, und ihr weigert euch; ich
recke meine Hand aus, und niemand achtet drauf, \bibverse{25} und laßt
fahren allen meinen Rat und wollt meiner Strafe nicht: \bibverse{26} so
will ich auch lachen in eurem Unfall und euer spotten, wenn da kommt,
das ihr fürchtet, \bibverse{27} wenn über euch kommt wie ein Sturm, das
ihr fürchtet, und euer Unfall als ein Wetter, wenn über euch Angst und
Not kommt. \bibverse{28} Dann werden sie mir rufen, aber ich werde nicht
antworten; sie werden mich frühe suchen und nicht finden. \bibverse{29}
Darum daß sie hasseten die Lehre und wollten des HErrn Furcht nicht
haben, \bibverse{30} wollten meines Rats nicht und lästerten alle meine
Strafe, \bibverse{31} so sollen sie essen von den Früchten ihres Wesens
und ihres Rats satt werden. \bibverse{32} Das die Albernen gelüstet,
tötet sie, und der Ruchlosen Glück bringt sie um. \bibverse{33} Wer aber
mir gehorchet, wird sicher bleiben und genug haben und kein Unglück
fürchten.

\hypertarget{section-1}{%
\section{2}\label{section-1}}

\bibverse{1} Mein Kind, willst du meine Rede annehmen und meine Gebote
bei dir behalten, \bibverse{2} so laß dein Ohr auf Weisheit achthaben
und neige dein Herz mit Fleiß dazu. \bibverse{3} Denn so du mit Fleiß
danach rufest und darum betest, \bibverse{4} so du sie suchest wie
Silber und forschest sie wie die Schätze, \bibverse{5} alsdann wirst du
die Furcht des HErrn vernehmen und GOttes Erkenntnis finden.
\bibverse{6} Denn der HErr gibt Weisheit, und aus seinem Munde kommt
Erkenntnis und Verstand. \bibverse{7} Er läßt's den Aufrichtigen
gelingen und beschirmet die Frommen \bibverse{8} und behütet die, so
recht tun, und bewahret den Weg seiner Heiligen. \bibverse{9} Dann wirst
du verstehen Gerechtigkeit und Recht und Frömmigkeit und allen guten
Weg. \bibverse{10} Wo die Weisheit dir zu Herzen gehet, daß du gerne
lernest, \bibverse{11} so wird dich guter Rat bewahren und Verstand wird
dich behüten, \bibverse{12} daß du nicht geratest auf den Weg der Bösen
noch unter die verkehrten Schwätzer, \bibverse{13} die da verlassen die
rechte Bahn und gehen finstere Wege, \bibverse{14} die sich freuen,
Böses zu tun, und sind fröhlich in ihrem bösen, verkehrten Wesen,
\bibverse{15} welche ihren Weg verkehren und folgen ihrem Abwege;
\bibverse{16} daß du nicht geratest an eines andern Weib, und die nicht
dein ist, die glatte Worte gibt \bibverse{17} und verläßt den Herrn
ihrer Jugend und vergisset den Bund ihres GOttes; \bibverse{18} denn ihr
Haus neiget sich zum Tode und ihre Gänge zu den Verlornen; \bibverse{19}
alle, die zu ihr eingehen, kommen nicht wieder und ergreifen den Weg des
Lebens nicht: \bibverse{20} auf daß du wandelst auf gutem Wege und
bleibest auf der rechten Bahn. \bibverse{21} Denn die Gerechten werden
im Lande wohnen, und die Frommen werden drinnen bleiben; \bibverse{22}
aber die Gottlosen werden aus dem Lande gerottet, und die Verächter
werden draus vertilget.

\hypertarget{section-2}{%
\section{3}\label{section-2}}

\bibverse{1} Mein Kind, vergiß meines Gesetzes nicht, und dein Herz
behalte meine Gebote. \bibverse{2} Denn sie werden dir langes Leben und
gute Jahre und Frieden bringen; \bibverse{3} Gnade und Treue werden dich
nicht lassen. Hänge sie an deinen Hals und schreibe sie in die Tafel
deines Herzens, \bibverse{4} so wirst du Gunst und Klugheit finden, die
GOtt und Menschen gefällt. \bibverse{5} Verlaß dich auf den HErrn von
ganzem Herzen und verlaß dich nicht auf deinen Verstand; \bibverse{6}
sondern gedenke an ihn in allen deinen Wegen, so wird er dich recht
führen. \bibverse{7} Dünke dich nicht weise zu sein, sondern fürchte den
HErrn und weiche vom Bösen. \bibverse{8} Das wird deinem Nabel gesund
sein und deine Gebeine erquicken. \bibverse{9} Ehre den HErrn von deinem
Gut und von den Erstlingen all deines Einkommens, \bibverse{10} so
werden deine Scheunen voll werden und deine Kelter mit Most übergehen.
\bibverse{11} Mein Kind, verwirf die Zucht des HErrn nicht und sei nicht
ungeduldig über seiner Strafe! \bibverse{12} Denn welchen der HErr
liebet, den straft er, und hat Wohlgefallen an ihm wie ein Vater am
Sohn. \bibverse{13} Wohl dem Menschen, der Weisheit findet, und dem
Menschen, der Verstand bekommt! \bibverse{14} Denn es ist besser, um sie
hantieren, weder um Silber, und ihr Einkommen ist besser denn Gold.
\bibverse{15} Sie ist edler denn Perlen; und alles, was du wünschen
magst, ist ihr nicht zu gleichen. \bibverse{16} Langes Leben ist zu
ihrer rechten Hand; zu ihrer Linken ist Reichtum und Ehre. \bibverse{17}
Ihre Wege sind liebliche Wege, und alle ihre Steige sind Friede.
\bibverse{18} Sie ist ein Baum des Lebens allen, die sie ergreifen; und
selig sind, die sie halten. \bibverse{19} Denn der HErr hat die Erde
durch Weisheit gegründet und durch seinen Rat die Himmel bereitet.
\bibverse{20} Durch seine Weisheit sind die Tiefen zerteilet und die
Wolken mit Tau triefend gemacht. \bibverse{21} Mein Kind, laß sie nicht
von deinen Augen weichen, so wirst du glückselig und klug werden.
\bibverse{22} Das wird deiner Seele Leben sein, und dein Mund wird
holdselig sein. \bibverse{23} Dann wirst du sicher wandeln auf deinem
Wege, daß dein Fuß sich nicht stoßen wird. \bibverse{24} Legest du dich,
so wirst du dich nicht fürchten, sondern süß schlafen, \bibverse{25} daß
du dich nicht fürchten darfst vor plötzlichem Schrecken noch vor dem
Sturm der Gottlosen, wenn er kommt. \bibverse{26} Denn der HErr ist dein
Trotz; der behütet deinen Fuß, daß er nicht gefangen werde.
\bibverse{27} Weigere dich nicht, dem Dürftigen Gutes zu tun, so deine
Hand von GOtt hat, solches zu tun. \bibverse{28} Sprich nicht zu deinem
Freunde: Gehe hin und komm wieder, morgen will ich dir geben, so du es
doch wohl hast. \bibverse{29} Trachte nicht Böses wider deinen Freund,
der auf Treue bei dir wohnet. \bibverse{30} Hadre nicht mit jemand ohne
Ursache, so er dir kein Leid getan hat. \bibverse{31} Eifre nicht einem
Freveln nach und erwähle seiner Wege keinen. \bibverse{32} Denn der HErr
hat Greuel an dem Abtrünnigen und sein Geheimnis ist bei den Frommen.
\bibverse{33} Im Hause des Gottlosen ist der Fluch des HErrn; aber das
Haus der Gerechten wird gesegnet. \bibverse{34} Er wird der Spötter
spotten; aber den Elenden wird er Gnade geben. \bibverse{35} Die Weisen
werden Ehre erben; aber wenn die Narren hoch kommen, werden sie doch
zuschanden.

\hypertarget{section-3}{%
\section{4}\label{section-3}}

\bibverse{1} Höret, meine Kinder, die Zucht eures Vaters; merkt auf, daß
ihr lernet und klug werdet! \bibverse{2} Denn ich gebe euch eine gute
Lehre; verlasset mein Gesetz nicht! \bibverse{3} Denn ich war meines
Vaters Sohn, ein zarter und ein einiger vor meiner Mutter, \bibverse{4}
und er lehrete mich und sprach: Laß dein Herz meine Worte aufnehmen;
halte meine Gebote, so wirst du leben. \bibverse{5} Nimm an Weisheit,
nimm an Verstand; vergiß nicht und weiche nicht von der Rede meines
Mundes! \bibverse{6} Verlaß sie nicht, so wird sie dich behalten; liebe
sie, so wird sie dich behüten. \bibverse{7} Denn der Weisheit Anfang
ist, wenn man sie gerne höret und die Klugheit lieber hat denn alle
Güter. \bibverse{8} Achte sie hoch, so wird sie dich erhöhen und wird
dich zu Ehren machen, wo du sie herzest. \bibverse{9} Sie wird dein
Haupt schön schmücken und wird dich zieren mit einer hübschen Krone.
\bibverse{10} So höre, mein Kind, und nimm an meine Rede, so werden
deiner Jahre viel werden. \bibverse{11} Ich will dich den Weg der
Weisheit führen, ich will dich auf rechter Bahn leiten, \bibverse{12}
daß, wenn du gehest, dein Gang dir nicht sauer werde, und wenn du
läufst, daß du dich nicht anstoßest. \bibverse{13} Fasse die Zucht, laß
nicht davon; bewahre sie, denn sie ist dein Leben. \bibverse{14} Komm
nicht auf der Gottlosen Pfad und tritt nicht auf den Weg der Bösen.
\bibverse{15} Laß ihn fahren und gehe nicht drinnen; weiche von ihm und
gehe vorüber! \bibverse{16} Denn sie schlafen nicht, sie haben denn übel
getan; und sie ruhen nicht, sie haben denn Schaden getan. \bibverse{17}
Denn sie nähren sich von gottlosem Brot und trinken vom Wein des
Frevels. \bibverse{18} Aber der Gerechten Pfad glänzet wie ein Licht,
das da fortgeht, und leuchtet bis auf den vollen Tag. \bibverse{19} Der
Gottlosen Weg aber ist wie Dunkel und wissen nicht, wo sie fallen
werden. \bibverse{20} Mein Sohn, merke auf mein Wort und neige dein Ohr
zu meiner Rede! \bibverse{21} Laß sie nicht von deinen Augen fahren;
behalte sie in deinem Herzen! \bibverse{22} Denn sie sind das Leben
denen, die sie finden, und gesund ihrem ganzen Leibe. \bibverse{23}
Behüte dein Herz mit allem Fleiß; denn daraus gehet das Leben.
\bibverse{24} Tu von dir den verkehrten Mund und laß das Lästermaul
ferne von dir sein. \bibverse{25} Laß deine Augen stracks vor sich sehen
und deine Augenlider richtig vor dir hinsehen. \bibverse{26} Laß deinen
Fuß gleich vor sich gehen, so gehest du gewiß. \bibverse{27} Wanke weder
zur Rechten noch zur Linken; wende deinen Fuß vom Bösen!

\hypertarget{section-4}{%
\section{5}\label{section-4}}

\bibverse{1} Mein Kind, merke auf meine Weisheit; neige dein Ohr zu
meiner Lehre, \bibverse{2} daß du behaltest guten Rat und dein Mund
wisse Unterschied zu haben. \bibverse{3} Denn die Lippen der Hure sind
süß wie Honigseim, und ihre Kehle ist glätter denn Öl, \bibverse{4} aber
hernach bitter wie Wermut und scharf wie ein zweischneidig Schwert.
\bibverse{5} Ihre Füße laufen zum Tod hinunter, ihre Gänge erlangen die
Hölle. \bibverse{6} Sie gehet nicht stracks auf dem Wege des Lebens;
unstet sind ihre Tritte, daß sie nicht weiß, wo sie gehet. \bibverse{7}
So gehorchet mir nun, meine Kinder, und weichet nicht von der Rede
meines Mundes! \bibverse{8} Laß deine Wege ferne von ihr sein und nahe
nicht zur Tür ihres Hauses, \bibverse{9} daß du nicht den Fremden gebest
deine Ehre und deine Jahre dem Grausamen, \bibverse{10} daß sich nicht
Fremde von deinem Vermögen sättigen, und deine Arbeit nicht sei in eines
andern Haus, \bibverse{11} und müssest hernach seufzen, wenn du dein
Leib und Gut verzehret hast, \bibverse{12} und sprechen: Ach, wie habe
ich die Zucht gehasset, und mein Herz die Strafe verschmähet,
\bibverse{13} und habe nicht gehorchet der Stimme meiner Lehrer und mein
Ohr nicht geneigt zu denen, die mich lehreten! \bibverse{14} Ich bin
schier in all Unglück kommen vor allen Leuten und allem Volk.
\bibverse{15} Trinke Wasser aus deiner Grube und Flüsse aus deinem
Brunnen. \bibverse{16} Laß deine Brunnen herausfließen und die
Wasserbäche auf die Gassen. \bibverse{17} Habe du aber sie alleine und
kein Fremder mit dir. \bibverse{18} Dein Born sei gesegnet, und freue
dich des Weibes deiner Jugend! \bibverse{19} Sie ist lieblich wie eine
Hindin und holdselig wie ein Reh. Laß dich ihre Liebe allezeit sättigen,
und ergötze dich allewege in ihrer Liebe. \bibverse{20} Mein Kind, warum
willst du dich an der Fremden ergötzen und herzest dich mit einer
andern? \bibverse{21} Denn jedermanns Wege sind stracks vor dem HErrn,
und er misset gleich alle ihre Gänge. \bibverse{22} Die Missetat des
Gottlosen wird ihn fahen, und er wird mit dem Strick seiner Sünde
gehalten werden. \bibverse{23} Er wird sterben, daß er sich nicht will
ziehen lassen, und um seiner großen Torheit willen wird's ihm nicht
wohlgehen.

\hypertarget{section-5}{%
\section{6}\label{section-5}}

\bibverse{1} Mein Kind, wirst du Bürge für deinen Nächsten und hast
deine Hand bei einem Fremden verhaftet, \bibverse{2} so bist du
verknüpft mit der Rede deines Mundes und gefangen mit den Reden deines
Mundes. \bibverse{3} So tu doch, mein Kind, also und errette dich; denn
du bist deinem Nächsten in die Hände kommen; eile, dränge und treibe
deinen Nächsten! \bibverse{4} Laß deine Augen nicht schlafen noch deine
Augenlider schlummern! \bibverse{5} Errette dich wie ein Reh von der
Hand und wie ein Vogel aus der Hand des Voglers. \bibverse{6} Gehe hin
zur Ameise, du Fauler, siehe ihre Weise an und lerne! \bibverse{7} Ob
sie wohl keinen Fürsten noch Hauptmann noch Herrn hat, \bibverse{8}
bereitet sie doch ihr Brot im Sommer und sammelt ihre Speise in der
Ernte. \bibverse{9} Wie lange, liegst du, Fauler? Wann willst du
aufstehen von deinem Schlaf? \bibverse{10} Ja, schlaf noch ein wenig,
schlummere ein wenig, schlage die Hände ineinander ein wenig, daß du
schlafest, \bibverse{11} so wird dich die Armut übereilen wie ein
Fußgänger und der Mangel wie ein gewappneter Mann. \bibverse{12} Ein
loser Mensch, ein schädlicher Mann, gehet mit verkehrtem Munde,
\bibverse{13} winket mit Augen, deutet mit Füßen, zeiget mit Fingern,
\bibverse{14} trachtet allezeit Böses und Verkehrtes in seinem Herzen
und richtet Hader an. \bibverse{15} Darum wird ihm plötzlich sein Unfall
kommen und wird schnell zerbrochen werden, daß keine Hilfe da sein wird.
\bibverse{16} Diese sechs Stücke hasset der HErr, und am siebenten hat
er einen Greuel: \bibverse{17} hohe Augen, falsche Zungen, Hände, die
unschuldig Blut vergießen; \bibverse{18} Herz, das mit bösen Tücken
umgehet; Füße, die behende sind, Schaden zu tun; \bibverse{19} falscher
Zeuge, der frech Lügen redet, und der Hader zwischen Brüdern anrichtet.
\bibverse{20} Mein Kind, bewahre die Gebote deines Vaters und laß nicht
fahren das Gesetz deiner Mutter! \bibverse{21} Binde sie zusammen auf
dein Herz allewege und hänge sie an deinen Hals: \bibverse{22} wenn du
gehest, daß sie dich geleiten; wenn du dich legest, daß sie dich
bewahren; wenn du aufwachest, daß sie dein Gespräch seien. \bibverse{23}
Denn das Gebot ist eine Leuchte und das Gesetz ein Licht; und die Strafe
der Zucht ist ein Weg des Lebens, \bibverse{24} auf daß du bewahret
werdest vor dem bösen Weibe, vor der glatten Zunge der Fremden.
\bibverse{25} Laß dich ihre Schöne nicht gelüsten in deinem Herzen und
verfahe dich nicht an ihren Augenlidern. \bibverse{26} Denn eine Hure
bringet einen ums Brot; aber ein Eheweib fähet das edle Leben.
\bibverse{27} Kann auch jemand ein Feuer im Busen behalten, daß seine
Kleider nicht brennen? \bibverse{28} Wie sollte jemand auf Kohlen gehen,
daß seine Füße nicht verbrannt würden? \bibverse{29} Also gehet es, wer
zu seines Nächsten Weib gehet; es bleibt keiner ungestraft, der sie
berühret. \bibverse{30} Es ist einem Diebe nicht so große Schmach, ob er
stiehlt, seine Seele zu sättigen, weil ihn hungert. \bibverse{31} Und ob
er begriffen wird, gibt er's siebenfältig wieder und legt dar alles Gut
in seinem Hause. \bibverse{32} Aber der mit einem Weibe die Ehe bricht,
der ist ein Narr, der bringet sein Leben ins Verderben. \bibverse{33}
Dazu trifft ihn Plage und Schande, und seine Schande wird nicht
ausgetilget. \bibverse{34} Denn der Grimm des Mannes eifert und schonet
nicht zur Zeit der Rache \bibverse{35} und siehet keine Person an, die
da versöhne, und nimmt's nicht an, ob du viel schenken wolltest.

\hypertarget{section-6}{%
\section{7}\label{section-6}}

\bibverse{1} Mein Kind, behalte meine Rede und verbirg meine Gebote bei
dir! \bibverse{2} Behalte meine Gebote, so wirst du leben, und mein
Gesetz wie deinen Augapfel. \bibverse{3} Binde sie an deine Finger,
schreibe sie auf die Tafel deines Herzens! \bibverse{4} Sprich zur
Weisheit: Du bist meine Schwester, und nenne die Klugheit deine
Freundin, \bibverse{5} daß du behütet werdest vor dem fremden Weibe, vor
einer andern, die glatte Worte gibt. \bibverse{6} Denn am Fenster meines
Hauses guckte ich durchs Gitter und sah unter den Albernen \bibverse{7}
und ward gewahr unter den Kindern eines närrischen Jünglings,
\bibverse{8} der ging auf der Gasse an einer Ecke und trat daher auf dem
Wege an ihrem Hause, \bibverse{9} in der Dämmerung, am Abend des Tages,
da es Nacht ward und dunkel war. \bibverse{10} Und siehe, da begegnete
ihm ein Weib im Hurenschmuck, listig, \bibverse{11} wild und unbändig,
daß ihre Füße in ihrem Hause nicht bleiben können. \bibverse{12} Jetzt
ist sie draußen, jetzt auf der Gasse und lauert an allen Ecken.
\bibverse{13} Und erwischte ihn und küssete ihn unverschämt und sprach
zu ihm: \bibverse{14} Ich habe Dankopfer für mich heute bezahlet, für
meine Gelübde. \bibverse{15} Darum bin ich herausgegangen, dir zu
begegnen, dein Angesicht frühe zu suchen, und habe dich funden.
\bibverse{16} Ich habe mein Bett schön geschmückt mit bunten Teppichen
aus Ägypten. \bibverse{17} Ich habe mein Lager mit Myrrhen, Aloes und
Zinnamen besprengt. \bibverse{18} Komm, laß uns genug buhlen bis an den
Morgen und laß uns der Liebe pflegen; \bibverse{19} denn der Mann ist
nicht daheim, er ist einen fernen Weg gezogen; \bibverse{20} er hat den
Geldsack mit sich genommen; er wird erst aufs Fest wieder heimkommen.
\bibverse{21} Sie überredete ihn mit vielen Worten und gewann ihn ein
mit ihrem glatten Munde. \bibverse{22} Er folgte ihr bald nach; wie ein
Ochs zur Fleischbank geführt wird, und wie zur Fessel, da man die Narren
züchtiget, \bibverse{23} bis sie ihm mit dem Pfeil die Leber spaltete,
wie ein Vogel zum Strick eilet und weiß nicht, daß ihm das Leben gilt.
\bibverse{24} So gehorchet mir nun, meine Kinder, und merket auf die
Rede meines Mundes. \bibverse{25} Laß dein Herz nicht weichen auf ihren
Weg und laß dich nicht verführen auf ihre Bahn! \bibverse{26} Denn sie
hat viele verwundet und gefället, und sind allerlei Mächtige von ihr
erwürget. \bibverse{27} Ihr Haus sind Wege zur Hölle, da man
hinunterfährt in des Todes Kammer.

\hypertarget{section-7}{%
\section{8}\label{section-7}}

\bibverse{1} Rufet nicht die Weisheit und die Klugheit läßt sich hören?
\bibverse{2} Öffentlich am Wege und an der Straße stehet sie.
\bibverse{3} An den Toren bei der Stadt, da man zur Tür eingehet,
schreiet sie: \bibverse{4} O ihr Männer, ich schreie zu euch und rufe
den Leuten! \bibverse{5} Merket, ihr Albernen, den Witz; und ihr Toren,
nehmet es zu Herzen! \bibverse{6} Höret, denn ich will reden, was
fürstlich ist, und lehren, was recht ist. \bibverse{7} Denn mein Mund
soll die Wahrheit reden, und meine Lippen sollen hassen, das gottlos
ist. \bibverse{8} Alle Reden meines Mundes sind gerecht; es ist nichts
Verkehrtes noch Falsches drinnen. \bibverse{9} Sie sind alle gleichaus
denen, die sie vernehmen, und richtig denen, die es annehmen wollen.
\bibverse{10} Nehmet an meine Zucht lieber denn Silber und die Lehre
achtet höher denn köstlich Gold. \bibverse{11} Denn Weisheit ist besser
denn Perlen, und alles, was man wünschen mag, kann ihr nicht gleichen.
\bibverse{12} Ich, Weisheit, wohne bei dem Witz und ich weiß guten Rat
zu geben. \bibverse{13} Die Furcht des HErrn hasset das Arge, die
Hoffart, den Hochmut und bösen Weg, und bin feind dem verkehrten Munde.
\bibverse{14} Mein ist beides, Rat und Tat; ich habe Verstand und Macht.
\bibverse{15} Durch mich regieren die Könige und die Ratsherren setzen
das Recht. \bibverse{16} Durch mich herrschen die Fürsten und alle
Regenten auf Erden. \bibverse{17} Ich liebe, die mich lieben; und die
mich frühe suchen, finden mich. \bibverse{18} Reichtum und Ehre ist bei
mir, wahrhaftig Gut und Gerechtigkeit. \bibverse{19} Meine Frucht ist
besser denn Gold und fein Gold und mein Einkommen besser denn auserlesen
Silber. \bibverse{20} Ich wandle auf dem rechten Wege, auf der Straße
des Rechts, \bibverse{21} daß ich wohl berate, die mich lieben und ihre
Schätze voll mache. \bibverse{22} Der HErr hat mich gehabt im Anfang
seiner Wege; ehe er was machte, war ich da. \bibverse{23} Ich bin
eingesetzt von Ewigkeit, von Anfang vor der Erde. \bibverse{24} Da die
Tiefen noch nicht waren, da war ich schon bereitet, da die Brunnen noch
nicht mit Wasser quollen, \bibverse{25} Ehe denn die Berge eingesenkt
waren, vor den Hügeln war ich bereitet. \bibverse{26} Er hatte die Erde
noch nicht gemacht, und was dran ist, noch die Berge des Erdbodens.
\bibverse{27} Da er die Himmel bereitete, war ich daselbst; da er die
Tiefe mit seinem Ziel verfassete, \bibverse{28} da er die Wolken droben
festete, da er festigte die Brunnen der Tiefe, \bibverse{29} da er dem
Meer das Ziel setzte und den Wassern, daß sie nicht übergehen seinen
Befehl, da er den Grund der Erde legte: \bibverse{30} da war ich der
Werkmeister bei ihm und hatte meine Lust täglich und spielte vor ihm
allezeit \bibverse{31} und spielte auf seinem Erdboden; und meine Lust
ist bei den Menschenkindern. \bibverse{32} So gehorchet mir nun, meine
Kinder! Wohl denen, die meine Wege behalten! \bibverse{33} Höret die
Zucht und werdet weise, und laßt sie nicht fahren! \bibverse{34} Wohl
dem Menschen, der mir gehorchet, daß er wache an meiner Tür täglich, daß
er warte an den Pfosten meiner Tür. \bibverse{35} Wer mich findet, der
findet das Leben und wird Wohlgefallen vom HErrn bekommen. \bibverse{36}
Wer aber an mir sündiget, der verletzt seine Seele. Alle, die mich
hassen, lieben den Tod.

\hypertarget{section-8}{%
\section{9}\label{section-8}}

\bibverse{1} Die Weisheit bauete ihr Haus und hieb sieben Säulen,
\bibverse{2} schlachtete ihr Vieh und trug ihren Wein auf und bereitete
ihren Tisch \bibverse{3} und sandte ihre Dirnen aus, zu laden oben auf
die Paläste der Stadt: \bibverse{4} Wer albern ist, der mache sich
hieher! Und zum Narren sprach sie: \bibverse{5} Kommt, zehret von meinem
Brot und trinket des Weins, den ich schenke! \bibverse{6} Verlasset das
alberne Wesen, so werdet ihr leben; und gehet auf dem Wege des
Verstandes. \bibverse{7} Wer den Spötter züchtiget, der muß Schande auf
sich nehmen; und wer den Gottlosen straft, der muß gehöhnet werden.
\bibverse{8} Strafe den Spötter nicht, er hasset dich; strafe den
Weisen, der wird dich lieben. \bibverse{9} Gib dem Weisen, so wird er
noch weiser werden; lehre den Gerechten, so wird er in der Lehre
zunehmen. \bibverse{10} Der Weisheit Anfang ist des HErrn Furcht; und
der Verstand lehret, was heilig ist. \bibverse{11} Denn durch mich wird
deiner Tage viel werden, und werden dir der Jahre des Lebens mehr
werden. \bibverse{12} Bist du weise, so bist du dir weise; bist du ein
Spötter, so wirst du es allein tragen. \bibverse{13} Es ist aber ein
töricht, wild Weib, voll Schwätzens und weiß nichts; \bibverse{14} die
sitzt in der Tür ihres Hauses auf dem Stuhl, oben in der Stadt,
\bibverse{15} zu laden alle, die vorübergehen und richtig auf ihrem Wege
wandeln. \bibverse{16} Wer ist albern, der mache sich hieher! Und zum
Narren spricht sie: \bibverse{17} Die verstohlenen Wasser sind süß und
das verborgene Brot ist niedlich. \bibverse{18} Er weiß aber nicht, daß
daselbst Tote sind und ihre Gäste in der tiefen Hölle.

\hypertarget{section-9}{%
\section{10}\label{section-9}}

\bibverse{1} Dies sind die Sprüche Salomos. Ein weiser Sohn ist seines
Vaters Freude; aber ein törichter Sohn ist seiner Mutter Grämen.
\bibverse{2} Unrecht Gut hilft nicht; aber Gerechtigkeit errettet vom
Tode. \bibverse{3} Der HErr läßt die Seele des Gerechten nicht Hunger
leiden; er stürzt aber der Gottlosen Schinderei. \bibverse{4} Lässige
Hand macht arm; aber der Fleißigen Hand macht reich. \bibverse{5} Wer im
Sommer sammelt, der ist klug; wer aber in der Ernte schläft, wird
zuschanden. \bibverse{6} Den Segen hat das Haupt des Gerechten; aber den
Mund der Gottlosen wird ihr Frevel überfallen. \bibverse{7} Das
Gedächtnis der Gerechten bleibt im Segen; aber der Gottlosen Name wird
verwesen. \bibverse{8} Wer weise von Herzen ist, nimmt die Gebote an;
der aber ein Narrenmaul hat, wird geschlagen. \bibverse{9} Wer
unschuldig lebet, der lebet sicher; wer aber verkehrt ist auf seinen
Wegen, wird offenbar werden. \bibverse{10} Wer mit Augen winket, wird
Mühe anrichten; und der ein Narrenmaul hat, wird geschlagen.
\bibverse{11} Des Gerechten Mund ist ein lebendiger Brunn; aber den Mund
der Gottlosen wird ihr Frevel überfallen. \bibverse{12} Haß erreget
Hader; aber Liebe deckt zu alle Übertretungen. \bibverse{13} In den
Lippen des Verständigen findet man Weisheit; aber auf den Rücken des
Narren gehört eine Rute. \bibverse{14} Die Weisen bewahren die Lehre;
aber der Narren Mund ist nahe dem Schrecken. \bibverse{15} Das Gut des
Reichen ist seine feste Stadt; aber die Armen macht die Armut blöde.
\bibverse{16} Der Gerechte braucht seines Guts zum Leben; aber der
Gottlose braucht seines Einkommens zur Sünde. \bibverse{17} Die Zucht
halten, ist der Weg zum Leben; wer aber die Strafe verläßt, der bleibt
irrig. \bibverse{18} Falsche Mäuler decken Haß; und wer verleumdet, der
ist ein Narr. \bibverse{19} Wo viel Worte sind, da geht es ohne Sünde
nicht ab; wer aber seine Lippen hält, ist klug. \bibverse{20} Des
Gerechten Zunge ist köstlich Silber; aber der Gottlosen Herz ist nichts.
\bibverse{21} Des Gerechten Lippen weiden viele; aber die Narren werden
in ihrer Torheit sterben. \bibverse{22} Der Segen des HErrn macht reich
ohne Mühe. \bibverse{23} Ein Narr treibt Mutwillen und hat's noch dazu
seinen Spott; aber der Mann ist weise, der drauf merkt. \bibverse{24}
Was der Gottlose fürchtet, das wird ihm begegnen; und was die Gerechten
begehren, wird ihnen gegeben. \bibverse{25} Der Gottlose ist wie ein
Wetter, das überhin geht und nicht mehr ist; der Gerechte aber bestehet
ewiglich. \bibverse{26} Wie der Essig den Zähnen und der Rauch den Augen
tut, so tut der Faule denen, die ihn senden. \bibverse{27} Die Furcht
des HErrn mehret die Tage; aber die Jahre der Gottlosen werden verkürzt.
\bibverse{28} Das Warten der Gerechten wird Freude werden; aber der
Gottlosen Hoffnung wird verloren sein. \bibverse{29} Der Weg des HErrn
ist des Frommen Trotz; aber die Übeltäter sind blöde. \bibverse{30} Der
Gerechte wird nimmermehr umgestoßen; aber die Gottlosen werden nicht im
Lande bleiben. \bibverse{31} Der Mund des Gerechten bringt Weisheit;
aber das Maul der Verkehrten wird ausgerottet. \bibverse{32} Die Lippen
der Gerechten lehren heilsam Ding; aber der Gottlosen Mund ist verkehrt.

\hypertarget{section-10}{%
\section{11}\label{section-10}}

\bibverse{1} Falsche Waage ist dem HErrn ein Greuel; aber ein völlig
Gewicht ist sein Wohlgefallen. \bibverse{2} Wo Stolz ist, da ist auch
Schmach; aber Weisheit ist bei den Demütigen. \bibverse{3} Unschuld wird
die Frommen leiten; aber die Bosheit wird die Verächter verstören.
\bibverse{4} Gut hilft nicht am Tage des Zorns; aber Gerechtigkeit
errettet vom Tode. \bibverse{5} Die Gerechtigkeit des Frommen macht
seinen Weg eben; aber der Gottlose wird fallen durch sein gottlos Wesen.
\bibverse{6} Die Gerechtigkeit der Frommen wird sie erretten; aber die
Verächter werden gefangen in ihrer Bosheit. \bibverse{7} Wenn der
gottlose Mensch stirbt, ist Hoffnung verloren; und das Harren der
Ungerechten wird zunichte. \bibverse{8} Der Gerechte wird aus der Not
erlöset und der Gottlose kommt an seine Statt. \bibverse{9} Durch den
Mund des Heuchlers wird sein Nächster verderbet; aber die Gerechten
merken's und werden erlöset. \bibverse{10} Eine Stadt freuet sich,
wenn's den Gerechten wohlgehet; und wenn die Gottlosen umkommen, wird
man froh. \bibverse{11} Durch den Segen der Frommen wird eine Stadt
erhaben; aber durch den Mund der Gottlosen wird sie zerbrochen.
\bibverse{12} Wer seinen Nächsten schändet, ist ein Narr; aber ein
verständiger Mann stillet es. \bibverse{13} Ein Verleumder verrät, was
er heimlich weiß; aber wer eines getreuen Herzens ist, verbirgt
dasselbe. \bibverse{14} Wo nicht Rat ist, da gehet das Volk unter; wo
aber viel Ratgeber sind, da gehet es wohl zu. \bibverse{15} Wer für
einen andern Bürge wird, der wird Schaden haben; wer sich aber vor
Geloben hütet, ist sicher. \bibverse{16} Ein holdselig Weib erhält die
Ehre; aber die Tyrannen erhalten den Reichtum. \bibverse{17} Ein
barmherziger Mann tut seinem Leibe Gutes; aber ein unbarmherziger
betrübet auch sein Fleisch und Blut. \bibverse{18} Der Gottlosen Arbeit
wird fehlen; aber wer Gerechtigkeit säet, das ist gewiß Gut.
\bibverse{19} Denn Gerechtigkeit fördert zum Leben; aber dem Übel
nachjagen fördert zum Tode. \bibverse{20} Der HErr hat Greuel an den
verkehrten Herzen und Wohlgefallen an den Frommen. \bibverse{21} Den
Bösen hilft nichts, wenn sie auch alle Hände zusammentäten; aber der
Gerechten Same wird errettet werden. \bibverse{22} Ein schön Weib ohne
Zucht ist wie eine Sau mit einem güldenen Haarband. \bibverse{23} Der
Gerechten Wunsch muß doch wohl geraten; und der Gottlosen Hoffen wird
Unglück. \bibverse{24} Einer teilt aus und hat immer mehr; ein anderer
karget, da er nicht soll, und wird doch ärmer. \bibverse{25} Die Seele,
die da reichlich segnet, wird fett; und wer trunken macht, der wird auch
trunken werden. \bibverse{26} Wer Korn inhält, dem fluchen die Leute;
aber Segen kommt über den, so es verkauft. \bibverse{27} Wer da Gutes
sucht, dem widerfährt Gutes; wer aber nach Unglück ringet, dem wird's
begegnen. \bibverse{28} Wer sich auf seinen Reichtum verläßt, der wird
untergehen; aber die Gerechten werden grünen wie ein Blatt.
\bibverse{29} Wer sein eigen Haus betrübt, der wird Wind zu Erbteil
haben; und ein Narr muß ein Knecht des Weisen sein. \bibverse{30} Die
Frucht des Gerechten ist ein Baum des Lebens; und ein Weiser nimmt sich
der Leute herzlich an. \bibverse{31} So der Gerechte auf Erden leiden
muß, wie viel mehr der Gottlose und Sünder!

\hypertarget{section-11}{%
\section{12}\label{section-11}}

\bibverse{1} Wer sich gerne läßt strafen, der wird klug werden; wer aber
ungestraft sein will, der bleibt ein Narr. \bibverse{2} Wer fromm ist,
der bekommt Trost vom HErrn; aber ein Ruchloser verdammt sich selbst.
\bibverse{3} Ein gottlos Wesen fördert den Menschen nicht; aber die
Wurzel der Gerechten wird bleiben. \bibverse{4} Ein fleißig Weib ist
eine Krone ihres Mannes; aber eine Unfleißige ist ein Eiter in seinem
Gebeine. \bibverse{5} Die Gedanken der Gerechten sind redlich; aber die
Anschläge der Gottlosen sind Trügerei. \bibverse{6} Der Gottlosen
Predigt richtet Blutvergießen an; aber der Frommen Mund errettet.
\bibverse{7} Die Gottlosen werden umgestürzt und nicht mehr sein; aber
das Haus der Gerechten bleibt stehen. \bibverse{8} Eines weisen Mannes
Rat wird gelobt; aber die Tücken werden zuschanden. \bibverse{9} Wer
gering ist und wartet des Seinen, der ist besser, denn der groß sein
will, dem des Brots mangelt. \bibverse{10} Der Gerechte erbarmet sich
seines Viehes; aber das Herz der Gottlosen ist unbarmherzig.
\bibverse{11} Wer seinen Acker bauet, der wird Brots die Fülle haben;
wer aber unnötigen Sachen nachgehet, der ist ein Narr. \bibverse{12} Des
Gottlosen Lust ist, Schaden zu tun; aber die Wurzel der Gerechten wird
Frucht bringen. \bibverse{13} Der Böse wird gefangen in seinen eigenen
falschen Worten; aber der Gerechte entgehet der Angst. \bibverse{14}
Viel Gutes kommt einem durch die Frucht des Mundes; und dem Menschen
wird vergolten, nachdem seine Hände verdienet haben. \bibverse{15} Dem
Narren gefällt seine Weise wohl; aber wer Rat gehorcht; der ist weise.
\bibverse{16} Ein Narr zeigt seinen Zorn bald; aber wer die Schmach
birget, ist witzig. \bibverse{17} Wer wahrhaftig ist, der sagt frei, was
recht ist; aber ein falscher Zeuge betrügt. \bibverse{18} Wer
unvorsichtig herausfährt, sticht wie ein Schwert; aber die Zunge der
Weisen ist heilsam. \bibverse{19} Wahrhaftiger Mund bestehet ewiglich;
aber die falsche Zunge bestehet nicht lange. \bibverse{20} Die, so Böses
raten, betrügen; aber die zum Frieden raten, machen Freude.
\bibverse{21} Es wird dem Gerechten kein Leid geschehen; aber die
Gottlosen werden voll Unglücks sein. \bibverse{22} Falsche Mäuler sind
dem HErrn ein Greuel; die aber treulich handeln, gefallen ihm wohl.
\bibverse{23} Ein witziger Mann gibt nicht Klugheit vor; aber das Herz
der Narren ruft seine Narrheit aus. \bibverse{24} Fleißige Hand wird
herrschen; die aber lässig ist, wird müssen zinsen. \bibverse{25} Sorge
im Herzen kränket; aber ein freundlich Wort erfreuet. \bibverse{26} Der
Gerechte hat's besser denn sein Nächster; aber der Gottlosen Weg
verführet sie. \bibverse{27} Einem Lässigen gerät sein Handel nicht;
aber ein fleißiger Mensch wird reich. \bibverse{28} Auf dem rechten Wege
ist Leben, und auf dem gebahnten Pfad ist kein Tod.

\hypertarget{section-12}{%
\section{13}\label{section-12}}

\bibverse{1} Ein weiser Sohn läßt sich den Vater züchtigen; aber ein
Spötter gehorchet der Strafe nicht. \bibverse{2} Der Frucht des Mundes
geneußt man; aber die Verächter denken nur zu freveln. \bibverse{3} Wer
seinen Mund bewahret, der bewahret sein Leben; wer aber mit seinem Maul
herausfährt, der kommt in Schrecken. \bibverse{4} Der Faule begehrt und
kriegt's doch nicht; aber die Fleißigen kriegen genug. \bibverse{5} Der
Gerechte ist der Lüge feind; aber der Gottlose schändet und schmähet
sich selbst. \bibverse{6} Die Gerechtigkeit behütet den Unschuldigen;
aber das gottlose Wesen bringet einen zu der Sünde. \bibverse{7} Mancher
ist arm bei großem Gut; und mancher ist reich bei seiner Armut.
\bibverse{8} Mit Reichtum kann einer sein Leben erretten; aber ein Armer
höret das Schelten nicht. \bibverse{9} Das Licht der Gerechten macht
fröhlich; aber die Leuchte der Gottlosen wird auslöschen. \bibverse{10}
Unter den Stolzen ist immer Hader; aber Weisheit macht vernünftige
Leute. \bibverse{11} Reichtum wird wenig, wo man's vergeudet; was man
aber zusammenhält, das wird groß. \bibverse{12} Die Hoffnung, die sich
verzeucht, ängstet das Herz; wenn's aber kommt, das man begehret, das
ist ein Baum des Lebens. \bibverse{13} Wer das Wort verachtet, der
verderbet sich selbst; wer aber das Gebot fürchtet, dem wird's
vergolten. \bibverse{14} Die Lehre des Weisen ist eine lebendige Quelle,
zu meiden die Stricke des Todes. \bibverse{15} Ein guter Rat tut sanft;
aber der Verächter Weg bringt Wehe. \bibverse{16} Ein Kluger tut alles
mit Vernunft; ein Narr aber breitet Narrheit aus. \bibverse{17} Ein
gottloser Bote bringt Unglück; aber ein treuer Werber ist heilsam.
\bibverse{18} Wer Zucht läßt fahren, der hat Armut und Schande; wer sich
gerne strafen läßt, wird zu Ehren kommen. \bibverse{19} Wenn's kommt,
das man begehret, das tut dem Herzen wohl; aber der das Böse meidet, ist
den Toren ein Greuel. \bibverse{20} Wer mit den Weisen umgehet, der wird
weise; wer aber der Narren Geselle ist, der wird Unglück haben.
\bibverse{21} Unglück verfolget die Sünder; aber den Gerechten wird
Gutes vergolten. \bibverse{22} Der Gute wird erben auf Kindeskind aber
des Sünders Gut wird dem Gerechten vorgesparet. \bibverse{23} Es ist
viel Speise in den Furchen der Armen; aber die unrecht tun, verderben.
\bibverse{24} Wer seiner Rute schonet, der hasset seinen Sohn; wer ihn
aber liebhat, der züchtiget ihn bald. \bibverse{25} Der Gerechte isset,
daß seine Seele satt wird; der Gottlosen Bauch aber hat nimmer genug.

\hypertarget{section-13}{%
\section{14}\label{section-13}}

\bibverse{1} Durch weise Weiber wird das Haus erbauet; eine Närrin aber
zerbricht es mit ihrem Tun. \bibverse{2} Wer den HErrn fürchtet, der
gehet auf rechter Bahn; wer ihn aber verachtet, der weicht aus seinem
Wege. \bibverse{3} Narren reden tyrannisch; aber die Weisen bewahren
ihren Mund. \bibverse{4} Wo nicht Ochsen sind, da ist die Krippe rein;
aber wo der Ochse geschäftig ist, da ist viel Einkommens. \bibverse{5}
Ein treuer Zeuge lüget nicht; aber ein falscher Zeuge redet türstiglich
Lügen. \bibverse{6} Der Spötter suchet Weisheit und findet sie nicht;
aber dem Verständigen ist die Erkenntnis leicht. \bibverse{7} Gehe von
dem Narren; denn du lernest nichts von ihm. \bibverse{8} Das ist des
Klugen Weisheit, daß er auf seinen Weg merkt; aber das ist der Narren
Torheit, daß es eitel Trug mit ihnen ist. \bibverse{9} Die Narren
treiben das Gespött mit der Sünde; aber die Frommen haben Lust an den
Frommen. \bibverse{10} Wenn das Herz traurig ist, so hilft keine
äußerliche Freude. \bibverse{11} Das Haus der Gottlosen wird vertilget;
aber die Hütte der Frommen wird grünen. \bibverse{12} Es gefällt manchem
ein Weg wohl; aber endlich bringt er ihn zum Tode. \bibverse{13} Nach
dem Lachen kommt Trauern, und nach der Freude kommt Leid. \bibverse{14}
Einem losen Menschen wird's gehen, wie er handelt; aber ein Frommer wird
über ihn sein. \bibverse{15} Ein Alberner glaubt alles; aber ein
Witziger merkt auf seinen Gang. \bibverse{16} Ein Weiser fürchtet sich
und meidet das Arge; ein Narr aber fährt hindurch türstiglich.
\bibverse{17} Ein Ungeduldiger tut närrisch; aber ein Bedächtiger hasset
es. \bibverse{18} Die Albernen erben Narrheit; aber es ist der Witzigen
Krone, vorsichtiglich handeln. \bibverse{19} Die Bösen müssen sich
bücken vor den Guten und die Gottlosen in den Toren des Gerechten.
\bibverse{20} Einen Armen hassen auch seine Nächsten; aber die Reichen
haben viel Freunde. \bibverse{21} Der Sünder verachtet seinen Nächsten;
aber wohl dem, der sich der Elenden erbarmet! \bibverse{22} Die mit
bösen Ränken umgehen, werden fehlen; die aber Gutes denken, denen wird
Treue und Güte widerfahren. \bibverse{23} Wo man arbeitet, da ist genug;
wo man aber mit Worten umgeht, da ist Mangel. \bibverse{24} Den Weisen
ist ihr Reichtum eine Krone; aber die Torheit der Narren bleibt Torheit.
\bibverse{25} Ein treuer Zeuge errettet das Leben; aber ein falscher
Zeuge betrügt. \bibverse{26} Wer den HErrn fürchtet, der hat eine
sichere Festung, und seine Kinder werden auch beschirmet. \bibverse{27}
Die Furcht des HErrn ist eine Quelle des Lebens, daß man meide die
Stricke des Todes. \bibverse{28} Wo ein König viel Volks hat, das ist
seine Herrlichkeit; wo aber, wenig Volks ist, das macht einen Herrn
blöde. \bibverse{29} Wer geduldig ist, der ist weise; wer aber
ungeduldig ist, der offenbart seine Torheit. \bibverse{30} Ein gütiges
Herz ist des Leibes Leben; aber Neid ist Eiter in Beinen. \bibverse{31}
Wer dem Geringen Gewalt tut, der lästert desselben Schöpfer; aber wer
sich des Armen erbarmet, der ehret GOtt, \bibverse{32} Der Gottlose
bestehet nicht in seinem Unglück; aber der Gerechte ist auch in seinem
Tode getrost. \bibverse{33} Im Herzen des Verständigen ruhet Weisheit
und wird offenbar unter den Narren. \bibverse{34} Gerechtigkeit erhöhet
ein Volk; aber die Sünde ist der Leute Verderben. \bibverse{35} Ein
kluger Knecht gefällt dem Könige wohl; aber einem schändlichen Knechte
ist er feind.

\hypertarget{section-14}{%
\section{15}\label{section-14}}

\bibverse{1} Eine linde Antwort stillet den Zorn; aber ein hart Wort
richtet Grimm an. \bibverse{2} Der Weisen Zunge macht die Lehre
lieblich; der Narren Mund speiet eitel Narrheit. \bibverse{3} Die Augen
des HErrn schauen an allen Orten beide, die Bösen und Frommen.
\bibverse{4} Eine heilsame Zunge ist ein Baum des Lebens; aber eine
lügenhaftige macht Herzeleid. \bibverse{5} Der Narr lästert die Zucht
seines Vaters; wer aber Strafe annimmt, der wird klug werden.
\bibverse{6} In des Gerechten Hause ist Guts genug aber in dem Einkommen
des Gottlosen ist Verderben. \bibverse{7} Der Weisen Mund streuet guten
Rat; aber der Narren Herz ist nicht also. \bibverse{8} Der Gottlosen
Opfer ist dem HErrn ein Greuel; aber das Gebet der Frommen ist ihm
angenehm. \bibverse{9} Des Gottlosen Weg ist dem HErrn ein Greuel; wer
aber der Gerechtigkeit nachjagt, der wird geliebet. \bibverse{10} Das
ist eine böse Zucht, den Weg verlassen; und wer die Strafe hasset, der
muß sterben. \bibverse{11} Hölle und Verderbnis ist vor dem HErrn; wie
viel mehr der Menschen Herzen! \bibverse{12} Der Spötter liebt nicht,
der ihn straft, und gehet nicht zu den Weisen. \bibverse{13} Ein
fröhlich Herz macht ein fröhlich Angesicht; aber wenn das Herz bekümmert
ist, so fällt auch der Mut. \bibverse{14} Ein kluges Herz handelt
bedächtiglich; aber die kühnen Narren regieren närrisch. \bibverse{15}
Ein Betrübter hat nimmer keinen guten Tag; aber ein guter Mut ist ein
täglich Wohlleben. \bibverse{16} Es ist besser ein wenig mit der Furcht
des HErrn denn großer Schatz, darin Unruhe ist. \bibverse{17} Es ist
besser ein Gericht Kraut mit Liebe denn ein gemästeter Ochse mit Haß.
\bibverse{18} Ein zorniger Mann richtet Hader an; ein Geduldiger aber
stillet den Zank. \bibverse{19} Der Weg des Faulen ist dornig; aber der
Weg der Frommen ist wohl gebahnet. \bibverse{20} Ein weiser Sohn
erfreuet den Vater; und ein närrischer Mensch ist seiner Mutter Schande.
\bibverse{21} Dem Toren ist die Torheit eine Freude; aber ein
verständiger Mann bleibt auf dem rechten Wege. \bibverse{22} Die
Anschläge werden zunichte, wo nicht Rat ist; wo aber viel Ratgeber sind,
bestehen sie. \bibverse{23} Es ist einem eine Freude, wo man ihm richtig
antwortet; und ein Wort zu seiner Zeit ist sehr lieblich. \bibverse{24}
Der Weg des Lebens gehet überwärts klug zu machen, auf daß man meide die
Hölle unterwärts. \bibverse{25} Der HErr wird das Haus der Hoffärtigen
zerbrechen und die Grenze der Witwen bestätigen. \bibverse{26} Die
Anschläge des Argen sind dem HErrn ein Greuel; aber tröstlich reden die
Reinen. \bibverse{27} Der Geizige verstöret sein eigen Haus; wer aber
Geschenk hasset, der wird leben. \bibverse{28} Das Herz des Gerechten
dichtet, was zu antworten ist; aber der Mund der Gottlosen schäumet
Böses. \bibverse{29} Der HErr ist ferne von den Gottlosen; aber der
Gerechten Gebet erhöret er. \bibverse{30} Freundlicher Anblick erfreuet
das Herz; ein gut Gerücht macht das Gebeine fett. \bibverse{31} Das Ohr,
das da höret die Strafe des Lebens, wird unter den Weisen wohnen.
\bibverse{32} Wer sich nicht ziehen läßt, der macht sich selbst
zunichte; wer aber Strafe höret, der wird klug. \bibverse{33} Die Furcht
des HErrn ist Zucht zur Weisheit; und ehe man zu Ehren kommt, muß man
zuvor leiden.

\hypertarget{section-15}{%
\section{16}\label{section-15}}

\bibverse{1} Der Mensch setzt ihm wohl vor im Herzen; aber vom HErrn
kommt, was die Zunge reden soll. \bibverse{2} Einen jeglichen dünken
seine Wege rein sein; aber allein der HErr macht das Herz gewiß.
\bibverse{3} Befiehl dem HErrn deine Werke, so werden deine Anschläge
fortgehen. \bibverse{4} Der HErr macht alles um sein selbst willen, auch
den Gottlosen zum bösen Tage. \bibverse{5} Ein stolz Herz ist dem HErrn
ein Greuel und wird nicht ungestraft bleiben, wenn sie sich gleich alle
aneinander hängen. \bibverse{6} Durch Güte und Treue wird Missetat
versöhnet; und durch die Furcht des HErrn meidet man das Böse.
\bibverse{7} Wenn jemands Wege dem HErrn wohlgefallen, so macht er auch
seine Feinde mit ihm zufrieden. \bibverse{8} Es ist besser wenig mit
Gerechtigkeit denn viel Einkommens mit Unrecht. \bibverse{9} Des
Menschen Herz schlägt seinen Weg an, aber der HErr allein gibt, daß er
fortgehe. \bibverse{10} Weissagung ist in dem Munde des Königs; sein
Mund fehlet nicht im Gericht. \bibverse{11} Rechte Waage und Gewicht ist
vom HErrn; und alle Pfunde im Sack sind seine Werke. \bibverse{12} Vor
den Königen unrecht tun, ist ein Greuel; denn durch Gerechtigkeit wird
der Thron bestätiget. \bibverse{13} Recht raten gefällt den Königen; und
wer gleich zu rät, wird geliebet. \bibverse{14} Des Königs Grimm ist ein
Bote des Todes; aber ein weiser Mann wird ihn versöhnen. \bibverse{15}
Wenn des Königs Angesicht freundlich ist, das ist Leben; und seine Gnade
ist wie ein Abendregen. \bibverse{16} Nimm an die Weisheit, denn sie ist
besser weder Gold, und Verstand haben ist edler denn Silber.
\bibverse{17} Der Frommen Weg meidet das Arge; und wer seinen Weg
bewahret, der behält sein Leben. \bibverse{18} Wer zugrund gehen soll,
der wird zuvor stolz; und stolzer Mut kommt vor dem Fall. \bibverse{19}
Es ist besser niedriges Gemüts sein mit den Elenden, denn Raub austeilen
mit den Hoffärtigen. \bibverse{20} Wer eine Sache klüglich führet, der
findet Glück; und wohl dem, der sich auf den HErrn verläßt.
\bibverse{21} Ein Verständiger wird gerühmet für einen weisen Mann, und
liebliche Reden lehren wohl. \bibverse{22} Klugheit ist ein lebendiger
Brunn dem, der sie hat; aber die Zucht der Narren ist Narrheit.
\bibverse{23} Ein weises Herz redet klüglich und lehret wohl.
\bibverse{24} Die Reden des Freundlichen sind Honigseim, trösten die
Seele und erfrischen die Gebeine. \bibverse{25} Manchem gefällt ein Weg
wohl; aber sein Letztes reicht zum Tode. \bibverse{26} Mancher kommt zu
großem Unglück durch sein eigen Maul. \bibverse{27} Ein loser Mensch
gräbt nach Unglück, und in seinem Maul brennet Feuer. \bibverse{28} Ein
verkehrter Mensch richtet Hader an, und ein Verleumder macht Fürsten
uneins. \bibverse{29} Ein Frevler locket seinen Nächsten und führet ihn
auf keinen guten Weg. \bibverse{30} Wer mit den Augen winkt, denkt nicht
Gutes; und wer mit den Lippen deutet, vollbringet Böses. \bibverse{31}
Graue Haare sind eine Krone der Ehren, die auf dem Wege der
Gerechtigkeit funden werden. \bibverse{32} Ein Geduldiger ist besser
denn ein Starker, und der seines Muts Herr ist, denn der Städte
gewinnet. \bibverse{33} Los wird geworfen in den Schoß; aber es fället,
wie der HErr will.

\hypertarget{section-16}{%
\section{17}\label{section-16}}

\bibverse{1} Es ist ein trockner Bissen, daran man sich genügen läßt,
besser denn ein Haus voll Geschlachtetes mit Hader. \bibverse{2} Ein
kluger Knecht wird herrschen über unfleißige Erben und wird unter den
Brüdern das Erbe austeilen. \bibverse{3} Wie das Feuer Silber und der
Ofen Gold, also prüfet der HErr die Herzen. \bibverse{4} Ein Böser
achtet auf böse Mäuler, und ein Falscher gehorcht gerne schädlichen
Zungen. \bibverse{5} Wer des Dürftigen spottet, der höhnet desselben
Schöpfer; und wer sich seines Unfalls freuet, wird nicht ungestraft
bleiben. \bibverse{6} Der Alten Krone sind Kindeskinder, und der Kinder
Ehre sind ihre Väter. \bibverse{7} Es stehet einem Narren nicht wohl an,
von hohen Dingen reden, viel weniger einem Fürsten, daß er gerne lüget.
\bibverse{8} Wer zu schenken hat, dem ist's wie ein Edelstein; wo er
sich hinkehret, ist er klug geachtet. \bibverse{9} Wer Sünde zudeckt,
der macht Freundschaft; wer aber die Sache aufrührt, der macht Fürsten
uneins. \bibverse{10} Schelten schreckt mehr an dem Verständigen denn
hundert Schläge an dem Narren. \bibverse{11} Ein bitterer Mensch
trachtet Schaden zu tun; aber es wird ein grausamer Engel über ihn
kommen. \bibverse{12} Es ist besser, einem Bären begegnen, dem die
Jungen geraubet sind, denn einem Narren in seiner Narrheit.
\bibverse{13} Wer Gutes mit Bösem vergilt, von des Hause wird Böses
nicht lassen. \bibverse{14} Wer Hader anfähet, ist gleich, als der dem
Wasser den Damm aufreißt. Laß du vom Hader, ehe du drein gemenget wirst.
\bibverse{15} Wer den Gottlosen recht spricht und den Gerechten
verdammet, die sind beide dem HErrn ein Greuel. \bibverse{16} Was soll
dem Narren Geld in der Hand, Weisheit zu kaufen, so er doch ein Narr
ist? \bibverse{17} Ein Freund liebet allezeit, und ein Bruder wird in
der Not erfunden. \bibverse{18} Es ist ein Narr, der an die Hand gelobet
und Bürge wird für seinen Nächsten. \bibverse{19} Wer Zank liebt, der
liebt Sünde; und wer seine Tür hoch macht, ringet nach Unglück.
\bibverse{20} Ein verkehrt Herz findet nichts Gutes, und der verkehrter
Zunge ist, wird in Unglück fallen. \bibverse{21} Wer einen Narren
zeuget, der hat Grämen, und eines Narren Vater hat keine Freude.
\bibverse{22} Ein fröhlich Herz macht das Leben lustig; aber ein
betrübter Mut vertrocknet das Gebeine. \bibverse{23} Der Gottlose nimmt
heimlich gern Geschenke, zu beugen den Weg des Rechts. \bibverse{24} Ein
Verständiger gebärdet weislich; ein Narr wirft die Augen hin und her.
\bibverse{25} Ein närrischer Sohn ist seines Vaters Trauern und
Betrübnis seiner Mutter, die ihn geboren hat. \bibverse{26} Es ist nicht
gut, daß man den Gerechten schindet, den Fürsten zu schlagen, der recht
regieret. \bibverse{27} Ein Vernünftiger mäßiget seine Rede, und ein
verständiger Mann ist eine teure Seele. \bibverse{28} Ein Narr, wenn er
schwiege, würde auch weise gerechnet und verständig, wenn er das Maul
hielte.

\hypertarget{section-17}{%
\section{18}\label{section-17}}

\bibverse{1} Wer sich absondert, der sucht, was ihn gelüstet, und setzt
sich wider alles, was gut ist. \bibverse{2} Ein Narr hat nicht Lust am
Verstand, sondern was in seinem Herzen steckt. \bibverse{3} Wo der
Gottlose hinkommt, da kommt Verachtung und Schmach mit Hohn.
\bibverse{4} Die Worte in eines Munde sind wie tiefe Wasser, und die
Quelle der Weisheit ist ein voller Strom. \bibverse{5} Es ist nicht gut,
die Person des Gottlosen achten, zu beugen den Gerechten im Gericht.
\bibverse{6} Die Lippen des Narren bringen Zank, und sein Mund ringet
nach Schlägen. \bibverse{7} Der Mund des Narren schadet ihm selbst, und
seine Lippen fahen seine eigene Seele. \bibverse{8} Die Worte des
Verleumders sind Schläge und gehen einem durchs Herz. \bibverse{9} Wer
laß ist in seiner Arbeit, der ist ein Bruder des, der das Seine
umbringet. \bibverse{10} Der Name des HErrn ist ein festes Schloß; der
Gerechte läuft dahin und wird beschirmet. \bibverse{11} Das Gut des
Reichen ist ihm eine feste Stadt und wie eine hohe Mauer um ihn her.
\bibverse{12} Wenn einer zugrund gehen soll, wird sein Herz zuvor stolz;
und ehe man zu Ehren kommt, muß man zuvor leiden. \bibverse{13} Wer
antwortet, ehe er höret, dem ist's Narrheit und Schande. \bibverse{14}
Wer ein fröhlich Herz hat, der weiß sich in seinem Leiden zu halten;
wenn aber der Mut liegt, wer kann's tragen? \bibverse{15} Ein verständig
Herz weiß sich vernünftiglich zu halten, und die Weisen hören gern, daß
man vernünftiglich handelt. \bibverse{16} Das Geschenk des Menschen
macht ihm Raum und bringt ihn vor die großen Herren. \bibverse{17} Der
Gerechte ist seiner Sache zuvor gewiß; kommt sein Nächster, so findet er
ihn also. \bibverse{18} Das Los stillet den Hader und scheidet zwischen
den Mächtigen. \bibverse{19} Ein verletzter Bruder hält härter denn eine
feste Stadt; und Zank hält härter denn Riegel am Palast. \bibverse{20}
Einem Mann wird vergolten, danach sein Mund geredet hat, und wird
gesättiget von der Frucht seiner Lippen. \bibverse{21} Tod und Leben
steht in der Zunge Gewalt; wer sie liebet, der wird von ihrer Frucht
essen. \bibverse{22} Wer eine Ehefrau findet, der findet was Gutes und
bekommt Wohlgefallen vom HErrn. \bibverse{23} Ein Armer redet mit
Flehen; ein Reicher antwortet stolz. \bibverse{24} Ein treuer Freund
liebet mehr und steht fester bei denn ein Bruder.

\hypertarget{section-18}{%
\section{19}\label{section-18}}

\bibverse{1} Ein Armer, der in seiner Frömmigkeit wandelt, ist besser
denn ein Verkehrter mit seinen Lippen, der doch ein Narr ist.
\bibverse{2} Wo man nicht mit Vernunft handelt, da geht es nicht wohl
zu; und wer schnell ist mit Füßen, der tut Schaden. \bibverse{3} Die
Torheit eines Menschen verleitet seinen Weg; da sein Herz wider den
HErrn tobet. \bibverse{4} Gut macht viel Freunde; aber der Arme wird von
seinen Freunden verlassen. \bibverse{5} Ein falscher Zeuge bleibt nicht
ungestraft, und wer Lügen frech redet, wird nicht entrinnen.
\bibverse{6} Viele warten auf die Person des Fürsten und sind alle
Freunde des, der Geschenke gibt. \bibverse{7} Den Armen hassen alle
seine Brüder, ja auch seine Freunde fernen sich von ihm; und wer sich
auf Worte verläßt, dem wird nichts. \bibverse{8} Wer klug ist, liebet
sein Leben; und der Verständige findet Gutes. \bibverse{9} Ein falscher
Zeuge bleibt nicht ungestraft, und wer frech Lügen redet, wird umkommen.
\bibverse{10} Dem Narren stehet nicht wohl an, gute Tage haben, viel
weniger einem Knechte, zu herrschen über Fürsten. \bibverse{11} Wer
geduldig ist, der ist ein kluger Mensch, und ist ihm ehrlich, daß er
Untugend überhören kann. \bibverse{12} Die Ungnade des Königs ist wie
das Brüllen eines jungen Löwen; aber seine Gnade ist wie Tau auf dem
Grase. \bibverse{13} Ein närrischer Sohn ist seines Vaters Herzeleid und
ein zänkisch Weib ein stetiges Triefen. \bibverse{14} Haus und Güter
erben die Eltern; aber ein vernünftig Weib kommt vom HErrn.
\bibverse{15} Faulheit bringt Schlafen, und eine lässige Seele wird
Hunger leiden. \bibverse{16} Wer das Gebot bewahret, der bewahret sein
Leben; wer aber seinen Weg verachtet, wird sterben. \bibverse{17} Wer
sich des Armen erbarmet, der leihet dem HErrn; der wird ihm wieder Gutes
vergelten. \bibverse{18} Züchtige deinen Sohn, weil Hoffnung da ist;
aber laß deine Seele nicht bewegt werden, ihn zu töten. \bibverse{19}
Denn großer Grimm bringt Schaden; darum laß ihn los, so kannst du ihn
mehr züchtigen. \bibverse{20} Gehorche dem Rat und nimm Zucht an, daß du
hernach weise seiest. \bibverse{21} Es sind viel Anschläge in eines
Mannes Herzen; aber der Rat des HErrn bleibet stehen. \bibverse{22}
Einen Menschen lüstet seine Wohltat; und ein Armer ist besser denn ein
Lügner. \bibverse{23} Die Furcht des HErrn fördert zum Leben und wird
satt bleiben, daß kein Übel sie heimsuchen wird. \bibverse{24} Der Faule
verbirgt seine Hand im Topf und bringt sie nicht wieder zum Munde.
\bibverse{25} Schlägt man den Spötter, so wird der Alberne witzig;
straft man einen Verständigen, so wird er vernünftig. \bibverse{26} Wer
Vater verstöret und Mutter verjaget, der ist ein schändlich und
verflucht Kind. \bibverse{27} Laß ab, mein Sohn, zu hören die Zucht, die
da abführet von vernünftiger Lehre! \bibverse{28} Ein loser Zeuge
spottet des Rechts, und der Gottlosen Mund verschlinget das Unrecht.
\bibverse{29} Den Spöttern sind Strafen bereitet und Schläge auf der
Narren Rücken.

\hypertarget{section-19}{%
\section{20}\label{section-19}}

\bibverse{1} Der Wein macht lose Leute und stark Getränk macht wild; wer
dazu Lust hat, wird nimmer weise. \bibverse{2} Das Schrecken des Königs
ist wie das Brüllen eines jungen Löwen; wer ihn erzürnet, der sündiget
wider sein Leben. \bibverse{3} Es ist dem Mann eine Ehre, vom Hader
bleiben; aber die gerne hadern, sind allzumal Narren. \bibverse{4} Um
der Kälte willen will der Faule nicht pflügen; so muß er in der Ernte
betteln und nichts kriegen. \bibverse{5} Der Rat im Herzen eines Mannes
ist wie tiefe Wasser; aber ein Verständiger kann's merken, was er
meinet. \bibverse{6} Viele Menschen werden fromm gerühmet; aber wer will
finden einen, der rechtschaffen fromm sei? \bibverse{7} Ein Gerechter,
der in seiner Frömmigkeit wandelt, des Kindern wird's wohlgehen nach
ihm. \bibverse{8} Ein König, der auf dem Stuhl sitzt zu richten,
zerstreuet alles Arge mit seinen Augen. \bibverse{9} Wer kann sagen: Ich
bin rein in meinem Herzen und lauter von meiner Sünde? \bibverse{10}
Mancherlei Gewicht und Maß ist beides Greuel dem HErrn. \bibverse{11}
Auch kennet man einen Knaben an seinem Wesen, ob er fromm und redlich
werden will. \bibverse{12} Ein hörend Ohr und sehend Auge, die macht
beide der HErr. \bibverse{13} Liebe den Schlaf nicht, daß du nicht arm
werdest; laß deine Augen wacker sein, so wirst du Brots genug haben.
\bibverse{14} Böse, böse! spricht man, wenn man's hat; aber wenn's weg
ist, so rühmet man es denn. \bibverse{15} Es ist Gold und viel Perlen;
aber ein vernünftiger Mund ist ein edel Kleinod. \bibverse{16} Nimm dem
sein Kleid, der für einen anderen Bürge wird, und pfände ihn um des
Unbekannten willen. \bibverse{17} Das gestohlene Brot schmeckt jedermann
wohl; aber hernach wird ihm der Mund voll Kieseling werden.
\bibverse{18} Anschläge bestehen, wenn man sie mit Rat führet; und Krieg
soll man mit Vernunft führen. \bibverse{19} Sei unverworren mit dem, der
Heimlichkeit offenbart, und mit dem Verleumder und mit dem falschen
Maul. \bibverse{20} Wer seinem Vater und seiner Mutter flucht, des
Leuchte wird verlöschen mitten in Finsternis. \bibverse{21} Das Erbe,
danach man zuerst sehr eilet, wird zuletzt nicht gesegnet sein.
\bibverse{22} Sprich nicht: Ich will Böses vergelten. Harre des HErrn,
der wird dir helfen. \bibverse{23} Mancherlei Gewicht ist ein Greuel dem
HErrn, und eine falsche Waage ist nicht gut. \bibverse{24} Jedermanns
Gänge kommen vom HErrn. Welcher Mensch verstehet seinen Weg?
\bibverse{25} Es ist dem Menschen ein Strick, das Heilige lästern und
danach Gelübde suchen. \bibverse{26} Ein weiser König zerstreuet die
Gottlosen und bringet das Rad über sie. \bibverse{27} Die Leuchte des
HErrn ist des Menschen Odem; die gehet durchs ganze Herz. \bibverse{28}
Fromm und wahrhaftig sein behüten den König, und sein Thron bestehet
durch Frömmigkeit. \bibverse{29} Der Jünglinge Stärke ist ihr Preis; und
grau Haar ist der Alten Schmuck. \bibverse{30} Man muß dem Bösen wehren
mit harter Strafe und mit ernsten Schlägen, die man fühlet.

\hypertarget{section-20}{%
\section{21}\label{section-20}}

\bibverse{1} Des Königs Herz ist in der Hand des HErrn wie Wasserbäche,
und er neiget es, wohin er will. \bibverse{2} Einen jeglichen dünkt sein
Weg recht sein; aber allein der HErr macht die Herzen gewiß.
\bibverse{3} Wohl und recht tun ist dem HErrn lieber denn Opfer.
\bibverse{4} Hoffärtige Augen und stolzer Mut und die Leuchte der
Gottlosen ist Sünde. \bibverse{5} Die Anschläge eines Endelichen bringen
Überfluß; wer aber allzu jach ist, wird mangeln. \bibverse{6} Wer
Schätze sammelt mit Lügen, der wird fehlen und fallen unter die den Tod
suchen. \bibverse{7} Der Gottlosen Rauben wird sie schrecken; denn sie
wollten nicht tun, was recht war. \bibverse{8} Wer einen andern Weg
gehet, der ist verkehrt; wer aber in seinem Befehl gehet, des Werk ist
recht. \bibverse{9} Es ist besser wohnen im Winkel auf dem Dach, denn
bei einem zänkischen Weibe in einem Hause beisammen. \bibverse{10} Die
Seele des Gottlosen wünschet Arges und gönnet seinem Nächsten nichts.
\bibverse{11} Wenn der Spötter gestraft wird, so werden die Albernen
weise; und wenn man einen Weisen unterrichtet, so wird er vernünftig.
\bibverse{12} Der Gerechte hält sich weislich gegen des Gottlosen Haus;
aber die Gottlosen denken nur Schaden zu tun. \bibverse{13} Wer seine
Ohren verstopft vor dem Schreien des Armen, der wird auch rufen und
nicht erhöret werden. \bibverse{14} Eine heimliche Gabe stillet den Zorn
und ein Geschenk im Schoß den heftigen Grimm. \bibverse{15} Es ist dem
Gerechten eine Freude zu tun, was recht ist, aber eine Furcht den
Übeltätern. \bibverse{16} Ein Mensch, der vom Wege der Klugheit irret,
der wird bleiben in der Toten Gemeine. \bibverse{17} Wer gern in Wollust
lebt, wird mangeln; und wer Wein und Öl liebet, wird nicht reich.
\bibverse{18} Der Gottlose muß für den Gerechten gegeben werden und der
Verächter für die Frommen. \bibverse{19} Es ist besser wohnen im wüsten
Lande denn, bei einem zänkischen und zornigen Weibe. \bibverse{20} Im
Hause des Weisen ist ein lieblicher Schatz und Öl aber ein Narr
verschlemmt es. \bibverse{21} Wer der Barmherzigkeit und Güte nachjagt,
der findet das Leben, Barmherzigkeit und Ehre. \bibverse{22} Ein Weiser
gewinnet die Stadt der Starken und stürzet ihre Macht durch ihre
Sicherheit. \bibverse{23} Wer seinen Mund und Zunge bewahret, der
bewahret seine Seele vor Angst. \bibverse{24} Der stolz und vermessen
ist, heißt ein loser Mensch, der im Zorn Stolz beweiset. \bibverse{25}
Der Faule stirbt über seinem Wünschen; denn seine Hände wollen nichts
tun. \bibverse{26} Er wünscht täglich; aber der Gerechte gibt und
versagt nicht. \bibverse{27} Der Gottlosen Opfer ist ein Greuel; denn
sie werden in Sünden geopfert. \bibverse{28} Ein lügenhaftiger Zeuge
wird umkommen; aber wer gehorchet, den läßt man auch allezeit wiederum
reden. \bibverse{29} Der Gottlose fährt mit dem Kopf hindurch; aber wer
fromm ist, des Weg wird bestehen. \bibverse{30} Es hilft keine Weisheit,
kein Verstand, kein Rat wider den HErrn. \bibverse{31} Rosse werden zum
Streittage bereitet; aber der Sieg kommt vom HErrn.

\hypertarget{section-21}{%
\section{22}\label{section-21}}

\bibverse{1} Das Gerücht ist köstlicher denn großer Reichtum und Gunst
besser denn Silber und Gold. \bibverse{2} Reiche und Arme müssen
untereinander sein; der HErr hat sie alle gemacht. \bibverse{3} Der
Witzige siehet das Unglück und verbirgt sich; die Albernen gehen
durchhin und werden beschädigt. \bibverse{4} Wo man leidet in des HErrn
Furcht, da ist Reichtum, Ehre und Leben. \bibverse{5} Stacheln und
Stricke sind auf dem Wege des Verkehrten; wer aber sich davon fernet,
bewahret sein Leben. \bibverse{6} Wie man einen Knaben gewöhnt, so läßt
er nicht davon, wenn er alt wird. \bibverse{7} Der Reiche herrschet über
die Armen, und wer borget, ist des Lehners Knecht. \bibverse{8} Wer
Unrecht säet, der wird Mühe ernten und wird durch die Rute seiner
Bosheit umkommen. \bibverse{9} Ein gut Auge wird gesegnet; denn er gibt
seines Brots den Armen. \bibverse{10} Treibe den Spötter aus, so gehet
der Zank weg, so höret auf Hader und Schmach. \bibverse{11} Wer ein treu
Herz und liebliche Rede hat, des Freund ist der König. \bibverse{12} Die
Augen des HErrn behüten guten Rat; aber die Worte des Verächters
verkehret er. \bibverse{13} Der Faule spricht: Es ist ein Löwe draußen,
ich möchte erwürget werden auf der Gasse. \bibverse{14} Der Huren Mund
ist eine tiefe Grube; wem der HErr ungnädig ist, der fället drein.
\bibverse{15} Torheit steckt dem Knaben im Herzen; aber die Rute der
Zucht wird sie ferne von ihm treiben. \bibverse{16} Wer dem Armen
unrecht tut, daß seines Guts viel werde, der wird auch einem Reichen
geben und mangeln. \bibverse{17} Neige deine Ohren und höre die Worte
der Weisen und nimm zu Herzen meine Lehre. \bibverse{18} Denn es wird
dir sanft tun, wo du sie wirst bei dir behalten, und werden miteinander
durch deinen Mund wohl geraten, \bibverse{19} daß deine Hoffnung sei auf
den HErrn. Ich muß dich solches täglich erinnern dir zu gut.
\bibverse{20} Hab ich dir's nicht mannigfaltiglich vorgeschrieben mit
Raten und Lehren, \bibverse{21} daß ich dir zeigete einen gewissen Grund
der Wahrheit, daß du recht antworten könntest denen, die dich senden?
\bibverse{22} Beraube den Armen nicht, ob er wohl arm ist, und
unterdrücke den Elenden nicht im Tor; \bibverse{23} denn der HErr wird
ihre Sache handeln und wird ihre Untertreter untertreten. \bibverse{24}
Geselle dich nicht zum zornigen Mann und halte dich nicht zu einem
grimmigen Mann; \bibverse{25} du möchtest seinen Weg lernen und deiner
Seele Ärgernis empfahen. \bibverse{26} Sei nicht bei denen, die ihre
Hand verhaften und für Schuld Bürge werden; \bibverse{27} denn wo du es
nicht hast zu bezahlen, so wird man dir dein Bett unter dir wegnehmen.
\bibverse{28} Treibe nicht zurück die vorigen Grenzen, die deine Väter
gemacht haben! \bibverse{29} Siehest du einen Mann endelich in seinem
Geschäfte, der wird vor den Königen stehen und wird nicht vor den
Unedlen stehen.

\hypertarget{section-22}{%
\section{23}\label{section-22}}

\bibverse{1} Wenn du sitzest und issest mit einem Herrn, so merke, wen
du vor dir hast, \bibverse{2} und setze ein Messer an deine Kehle,
willst du das Leben behalten. \bibverse{3} Wünsche dir nicht seiner
Speise, denn es ist falsch Brot. \bibverse{4} Bemühe dich nicht, reich
zu werden, und laß ab von deinen Fündlein! \bibverse{5} Laß deine Augen
nicht fliegen dahin, das du nicht haben kannst; denn dasselbe macht ihm
Flügel wie ein Adler und fleugt gen Himmel. \bibverse{6} Iß nicht Brot
bei einem Neidischen und wünsche dir seiner Speise nicht. \bibverse{7}
Denn wie ein Gespenst ist er inwendig. Er spricht: Iß und trink! und
sein Herz ist doch nicht an dir. \bibverse{8} Deine Bissen, die du
gegessen hattest mußt du ausspeien und mußt deine freundlichen Worte
verloren haben. \bibverse{9} Rede nicht vor des Narren Ohren; denn er
verachtet die Klugheit deiner Rede. \bibverse{10} Treibe nicht zurück
die vorigen Grenzen und gehe nicht auf der Waisen Acker! \bibverse{11}
Denn ihr Erlöser ist mächtig; der wird ihre Sache wider dich ausführen.
\bibverse{12} Gib dein Herz zur Zucht und deine Ohren zu vernünftiger
Rede. \bibverse{13} Laß nicht ab, den Knaben zu züchtigen; denn wo du
ihn mit der Rute hauest, so darf man ihn nicht töten. \bibverse{14} Du
hauest ihn mit der Rute; aber du errettest seine Seele von der Hölle.
\bibverse{15} Mein Sohn, so du weise bist, so freuet sich auch mein
Herz; \bibverse{16} und meine Nieren sind froh, wenn deine Lippen reden,
was recht ist. \bibverse{17} Dein Herz folge nicht den Sündern, sondern
sei täglich in der Furcht des HErrn. \bibverse{18} Denn es wird dir
hernach gut sein und dein Warten wird nicht fehlen. \bibverse{19} Höre,
mein Sohn, und sei weise und richte dein Herz in den Weg. \bibverse{20}
Sei nicht unter den Säufern und Schlemmern; \bibverse{21} denn die
Säufer und Schlemmer verarmen, und ein Schläfer muß zerrissene Kleider
tragen. \bibverse{22} Gehorche deinem Vater, der dich gezeugt hat, und
verachte deine Mutter nicht, wenn sie alt wird! \bibverse{23} Kaufe
Wahrheit und verkaufe sie nicht, Weisheit, Zucht und Verstand.
\bibverse{24} Ein Vater des Gerechten freuet sich, und wer einen Weisen
gezeugt hat, ist fröhlich darüber. \bibverse{25} Laß sich deinen Vater
und deine Mutter freuen und fröhlich sein, die dich gezeuget hat.
\bibverse{26} Gib mir, mein Sohn, dein Herz und laß deinen Augen meine
Wege wohlgefallen. \bibverse{27} Denn eine Hure ist eine tiefe Grube,
und die Ehebrecherin ist eine enge Grube. \bibverse{28} Auch lauert sie
wie ein Räuber und die Frechen unter den Menschen sammelt sie zu sich.
\bibverse{29} Wo ist Weh? Wo ist Leid? Wo ist Zank? Wo ist Klagen? Wo
sind Wunden ohne Ursache? Wo sind rote Augen? \bibverse{30} Nämlich, wo
man beim Wein liegt und kommt auszusaufen, was eingeschenkt ist.
\bibverse{31} Siehe den Wein nicht an, daß er so rot ist und im Glase so
schön stehet. Er gehet glatt ein; \bibverse{32} aber danach beißt er wie
eine Schlange und sticht wie eine Otter. \bibverse{33} So werden deine
Augen nach andern Weibern sehen, und dein Herz wird verkehrte Dinge
reden, \bibverse{34} und wirst sein wie einer, der mitten im Meer
schläft, und wie einer schläft oben auf dem Mastbaum. \bibverse{35} Sie
schlagen mich, aber es tut mir nicht weh; sie klopfen mich, aber ich
fühle es nicht. Wann will ich aufwachen, daß ich's mehr treibe?

\hypertarget{section-23}{%
\section{24}\label{section-23}}

\bibverse{1} Folge nicht bösen Leuten und wünsche nicht, bei ihnen zu
sein. \bibverse{2} Denn ihr Herz trachtet nach Schaden, und ihre Lippen
raten zu Unglück. \bibverse{3} Durch Weisheit wird ein Haus gebauet und
durch Verstand erhalten. \bibverse{4} Durch ordentlich Haushalten werden
die Kammern voll aller köstlichen, lieblichen Reichtümer. \bibverse{5}
Ein weiser Mann ist stark und ein vernünftiger Mann ist mächtig von
Kräften. \bibverse{6} Denn mit Rat muß man Krieg führen; und wo viel
Ratgeber sind, da ist der Sieg. \bibverse{7} Weisheit ist dem Narren zu
hoch; er darf seinen Mund im Tor nicht auftun. \bibverse{8} Wer ihm
selbst Schaden tut, den heißt man billig einen Erzbösewicht.
\bibverse{9} Des Narren Tücke ist Sünde; und der Spötter ist ein Greuel
vor den Leuten. \bibverse{10} Der ist nicht stark, der in der Not nicht
fest ist. \bibverse{11} Errette die, so man töten will, und entzieh dich
nicht von denen, die man würgen will. \bibverse{12} Sprichst du: Siehe,
wir verstehen's nicht; meinest du nicht, der die Herzen weiß, merket es,
und der auf die Seele acht hat, kennet es und vergilt dem Menschen nach
seinem Werk? \bibverse{13} Iß, mein Sohn, Honig, denn es ist gut, und
Honigseim ist süß in deinem Halse. \bibverse{14} Also lerne die Weisheit
für deine Seele. Wenn du sie findest, so wird's hernach wohlgehen, und
deine Hoffnung wird nicht umsonst sein. \bibverse{15} Laure nicht, als
ein Gottloser, auf das Haus des Gerechten; verstöre seine Ruhe nicht!
\bibverse{16} Denn ein Gerechter fällt siebenmal und stehet wieder auf;
aber die Gottlosen versinken in Unglück. \bibverse{17} Freue dich des
Falles deines Feindes nicht, und dein Herz sei nicht froh über seinem
Unglück; \bibverse{18} es möchte der HErr sehen und ihm übel gefallen
und seinen Zorn von ihm wenden. \bibverse{19} Erzürne dich nicht über
den Bösen und eifre nicht über die Gottlosen; \bibverse{20} denn der
Böse hat nichts zu hoffen, und die Leuchte der Gottlosen wird
verlöschen. \bibverse{21} Mein Kind, fürchte den HErrn und den König und
menge dich nicht unter die Aufrührerischen! \bibverse{22} Denn ihr
Unfall wird plötzlich entstehen; und wer weiß, wann beider Unglück
kommt? \bibverse{23} Dies kommt auch von den Weisen: Die Person ansehen
im Gericht ist nicht gut. \bibverse{24} Wer zum Gottlosen spricht: Du
bist fromm, dem fluchen die Leute und hasset das Volk. \bibverse{25}
Welche aber strafen, die gefallen wohl, und kommt ein reicher Segen auf
sie. \bibverse{26} Eine richtige Antwort ist wie ein lieblicher Kuß.
\bibverse{27} Richte draußen dein Geschäft aus und arbeite deinen Acker;
danach baue dein Haus. \bibverse{28} Sei nicht Zeuge ohne Ursache wider
deinen Nächsten und betrüge nicht mit deinem Munde! \bibverse{29} Sprich
nicht: Wie man mir tut, so will ich wieder tun und einem jeglichen sein
Werk vergelten. \bibverse{30} Ich ging vor dem Acker des Faulen und vor
dem Weinberge des Narren, \bibverse{31} und siehe, da waren eitel
Nesseln drauf und stund voll Disteln, und die Mauer war eingefallen.
\bibverse{32} Da ich das sah, nahm ich's zu Herzen und schauete und
lernete dran. \bibverse{33} Du willst ein wenig schlafen und ein wenig
schlummern und ein wenig die Hände zusammentun, daß du ruhest;
\bibverse{34} aber es wird dir deine Armut kommen wie ein Wanderer und
dein Mangel wie ein gewappneter Mann.

\hypertarget{section-24}{%
\section{25}\label{section-24}}

\bibverse{1} Dies sind auch Sprüche Salomos, die hinzugesellt haben die
Männer Hiskias, des Königs Judas. \bibverse{2} Es ist GOttes Ehre, eine
Sache verbergen; aber der Könige Ehre ist's, eine Sache erforschen.
\bibverse{3} Der Himmel ist hoch und die Erde tief; aber der Könige Herz
ist unerforschlich. \bibverse{4} Man tue den Schaum vom Silber, so wird
ein rein Gefäß draus. \bibverse{5} Man tue gottlos Wesen vom Könige, so
wird sein Thron mit Gerechtigkeit bestätiget. \bibverse{6} Prange nicht
vor dem Könige und tritt nicht an den Ort der Großen. \bibverse{7} Denn
es ist dir besser, daß man zu dir sage: Tritt hie herauf! denn daß du
vor dem Fürsten geniedriget wirst, daß deine Augen sehen müssen.
\bibverse{8} Fahre nicht bald heraus zu zanken; denn was willst du
hernach machen, wenn du deinen Nächsten geschändet hast? \bibverse{9}
Handle deine Sache mit deinem Nächsten und offenbare nicht eines andern
Heimlichkeit, \bibverse{10} auf daß dir's nicht übel spreche, der es
höret, und dein böses Gerücht nimmer ablasse. \bibverse{11} Ein Wort,
geredet zu seiner Zeit, ist wie güldene Äpfel in silbernen Schalen.
\bibverse{12} Wer einen Weisen straft, der ihm gehorcht, das ist wie ein
gülden Stirnband und gülden Halsband. \bibverse{13} Wie die Kälte des
Schnees zur Zeit der Ernte, so ist ein getreuer Bote dem, der ihn
gesandt hat und erquickt seines Herrn Seele. \bibverse{14} Wer viel
geredet und hält nicht, der ist wie Wolken und Wind ohne Regen.
\bibverse{15} Durch Geduld wird ein Fürst versöhnet, und eine linde
Zunge bricht die Härtigkeit. \bibverse{16} Findest du Honig, so iß sein
genug, daß du nicht zu satt werdest und speiest ihn aus. \bibverse{17}
Entzeuch deinen Fuß vom Hause deines Nächsten, er möchte dein
überdrüssig und dir gram werden. \bibverse{18} Wer wider seinen Nächsten
falsch Zeugnis redet, der ist ein Spieß, Schwert und scharfer Pfeil.
\bibverse{19} Die Hoffnung des Verächters zur Zeit der Not ist wie ein
fauler Zahn und gleitender Fuß. \bibverse{20} Wer einem bösen Herzen
Lieder singet, das ist wie ein zerrissen Kleid im Winter und Essig auf
der Kreide. \bibverse{21} Hungert deinen Feind, so speise ihn mit Brot;
dürstet ihn, so tränke ihn mit Wasser. \bibverse{22} Denn du wirst
Kohlen auf sein Haupt häufen; und der HErr wird dir's vergelten.
\bibverse{23} Der Nordwind bringet Ungewitter, und die heimliche Zunge
macht sauer Angesicht. \bibverse{24} Es ist besser im Winkel auf dem
Dache sitzen denn bei einem zänkischen Weibe in einem Hause beisammen.
\bibverse{25} Ein gut Gerücht aus fernen Landen in wie kalt Wasser einer
durstigen Seele. \bibverse{26} Ein Gerechter, der vor einem Gottlosen
fällt, ist wie ein trüber Brunnen und verderbte Quelle. \bibverse{27}
Wer zu viel Honig isset, das ist nicht gut; und wer schwere Dinge
forschet, dem wird's zu schwer. \bibverse{28} Ein Mann, der seinen Geist
nicht halten kann, ist wie eine offene Stadt ohne Mauern.

\hypertarget{section-25}{%
\section{26}\label{section-25}}

\bibverse{1} Wie der Schnee im Sommer und Regen in der Ernte, also
reimet sich dem Narren Ehre nicht. \bibverse{2} Wie ein Vogel dahinfährt
und eine Schwalbe fleugt, also ein unverdienter Fluch trifft nicht.
\bibverse{3} Dem Roß eine Geißel und dem Esel ein Zaum; und dem Narren
eine Rute auf den Rücken. \bibverse{4} Antworte dem Narren nicht mich
seiner Narrheit, daß du ihm nicht auch gleich werdest. \bibverse{5}
Antworte aber dem Narren nach seiner Narrheit, daß er sich nicht weise
lasse dünken. \bibverse{6} Wer eine Sache durch einen törichten Boten
ausrichtet, der ist wie ein Lahmer an Füßen und nimmt Schaden.
\bibverse{7} Wie einem Krüppel das Tanzen, also stehet den Narren an,
von Weisheit reden. \bibverse{8} Wer einem Narren Ehre anlegt, das ist,
als wenn einer einen Edelstein auf den Rabenstein würfe. \bibverse{9}
Ein Spruch in eines Narren Mund ist wie ein Dornzweig, der in eines
Trunkenen Hand sticht. \bibverse{10} Ein guter Meister macht ein Ding
recht; aber wer einen Hümpler dinget, dem wird's verderbt. \bibverse{11}
Wie ein Hund sein Gespeietes wieder frißt, also ist der Narr, der seine
Narrheit wieder treibt. \bibverse{12} Wenn du einen siehest, der sich
weise dünket, da ist an einem Narren mehr Hoffnung denn an ihm.
\bibverse{13} Der Faule spricht: Es ist ein junger Löwe auf dem Wege und
ein Löwe auf den Gassen. \bibverse{14} Ein Fauler wendet sich im Bette
wie die Tür in der Angel. \bibverse{15} Der Faule verbirgt seine Hand in
dem Topf, und wird ihm sauer, daß er sie zum Munde bringe. \bibverse{16}
Ein Fauler dünkt sich weiser denn sieben, die da Sitten lehren.
\bibverse{17} Wer vorgehet und sich menget in fremden Hader, der ist wie
einer, der den Hund bei den Ohren zwacket. \bibverse{18} Wie einer
heimlich mit Geschoß und Pfeilen schießt und tötet, \bibverse{19} also
tut ein falscher Mensch mit seinem Nächsten und spricht danach: Ich habe
gescherzt. \bibverse{20} Wenn nimmer Holz da ist, so verlöscht das
Feuer, und wenn der Verleumder weg ist, so höret der Hader auf.
\bibverse{21} Wie die Kohlen eine Glut und Holz ein Feuer, also richtet
ein zänkischer Mann Hader an. \bibverse{22} Die Worte des Verleumders
sind wie Schläge und sie gehen durchs Herz. \bibverse{23} Giftiger Mund
und böses Herz ist wie ein Scherben mit Silberschaum überzogen.
\bibverse{24} Der Feind wird erkannt bei seiner Rede, wiewohl er im
Herzen falsch ist. \bibverse{25} Wenn er seine Stimme holdselig macht,
so glaube ihm nicht; denn es sind sieben Greuel in seinem Herzen.
\bibverse{26} Wer den Haß heimlich hält, Schaden zu tun, des Bosheit
wird vor der Gemeine offenbar werden. \bibverse{27} Wer eine Grube
macht, der wird dreinfallen; und wer einen Stein wälzet, auf den wird er
kommen. \bibverse{28} Eine falsche Zunge hasset, der ihn strafet; und
ein Heuchelmaul richtet Verderben an.

\hypertarget{section-26}{%
\section{27}\label{section-26}}

\bibverse{1} Rühme dich nicht des morgenden Tages; denn du weißest
nicht, was heute sich begeben mag. \bibverse{2} Laß dich einen andern
loben und nicht deinen Mund, einen Fremden und nicht deine eigenen
Lippen. \bibverse{3} Stein ist schwer und Sand ist Last; aber des Narren
Zorn ist schwerer denn die beiden. \bibverse{4} Zorn ist ein wütig Ding,
und Grimm ist ungestüm; und wer kann vor dem Neid bestehen? \bibverse{5}
Öffentliche Strafe ist besser denn heimliche Liebe. \bibverse{6} Die
Schläge des Liebhabers meinen's recht gut; aber das Küssen des Hassers
ist ein Gewäsch. \bibverse{7} Eine volle Seele zertritt wohl Honigseim;
aber einer hungrigen Seele ist alles Bittre süß. \bibverse{8} Wie ein
Vogel ist, der aus seinem Nest weicht, also ist, der von seiner Stätte
weicht. \bibverse{9} Das Herz freuet sich der Salbe und Räuchwerk; aber
ein Freund ist lieblich um Rats willen der Seele. \bibverse{10} Deinen
Freund und deines Vaters Freund verlaß nicht. Und gehe nicht ins Haus
deines Bruders, wenn dir's übel gehet; denn ein Nachbar ist besser in
der Nähe weder ein Bruder in der Ferne. \bibverse{11} Sei weise, mein
Sohn, so freuet sich mein Herz, so will ich antworten dem, der mich
schmähet. \bibverse{12} Ein Witziger siehet das Unglück und verbirgt
sich; aber die Albernen gehen durch und leiden Schaden. \bibverse{13}
Nimm dem sein Kleid der für einen andern Bürge wird, und pfände ihn um
der Fremden willen. \bibverse{14} Wer seinen Nächsten mit lauter Stimme
segnet und früh aufstehet, das wird ihm für einen Fluch geredet.
\bibverse{15} Ein zänkisch Weib und stetiges Triefen, wenn's sehr
regnet, werden wohl miteinander verglichen. \bibverse{16} Wer sie
aufhält, der hält den Wind und will das Öl mit der Hand fassen.
\bibverse{17} Ein Messer wetzt das andere und ein Mann den andern.
\bibverse{18} Wer seinen Feigenbaum bewahret, der isset Früchte davon;
und wer seinen Herrn bewahret, wird geehret. \bibverse{19} Wie der
Schemen im Wasser ist gegen das Angesicht, also ist eines Menschen Herz
gegen den andern. \bibverse{20} Hölle und Verderbnis werden nimmer voll,
und der Menschen Augen sind auch unsättig. \bibverse{21} Ein Mann wird
durch den Mund des Lobers bewähret wie das Silber im Tiegel und das Gold
im Ofen. \bibverse{22} Wenn du den Narren im Mörser zerstießest mit dem
Stämpfel wie Grütze, so ließe doch seine Narrheit nicht von ihm.
\bibverse{23} Auf deine Schafe hab acht und nimm dich deiner Herde an;
\bibverse{24} denn Gut währet nicht ewiglich, und die Krone währet nicht
für und für. \bibverse{25} Das Heu ist aufgegangen und ist da das Gras,
und wird Kraut auf den Bergen gesammelt. \bibverse{26} Die Lämmer
kleiden dich, und die Böcke geben dir das Ackergeld. \bibverse{27} Du
hast Ziegenmilch genug zur Speise deines Hauses und zur Nahrung deiner
Dirnen.

\hypertarget{section-27}{%
\section{28}\label{section-27}}

\bibverse{1} Der Gottlose fleucht, und niemand jagt ihn; der Gerechte
aber ist getrost wie ein junger Löwe. \bibverse{2} Um des Landes Sünde
willen werden viel Änderungen der Fürstentümer; aber um der Leute
willen, die verständig und vernünftig sind, bleiben sie lange.
\bibverse{3} Ein armer Mann, der die Geringen beleidigt, ist wie ein
Meltau, der die Frucht verderbt. \bibverse{4} Die das Gesetz verlassen,
loben den Gottlosen; die es aber bewahren, sind unwillig auf sie.
\bibverse{5} Böse Leute merken nicht aufs Recht; die aber nach dem HErrn
fragen, merken auf alles. \bibverse{6} Es ist besser ein Armer, der in
seiner Frömmigkeit gehet, denn ein Reicher, der in verkehrten Wegen
gehet. \bibverse{7} Wer das Gesetz bewahret, ist ein verständig Kind;
wer aber Schlemmer nähret, schändet seinen Vater. \bibverse{8} Wer sein
Gut mehret mit Wucher und Übersatz, der sammelt es zu Nutz der Armen.
\bibverse{9} Wer sein Ohr abwendet, zu hören das Gesetz, des Gebet ist
ein Greuel. \bibverse{10} Wer die Frommen verführet auf bösem Wege, der
wird in seine Grube fallen; aber die Frommen werden Gutes ererben.
\bibverse{11} Ein Reicher dünkt sich weise sein; aber ein armer
Verständiger merkt ihn. \bibverse{12} Wenn die Gerechten überhand haben,
so gehet es sehr fein zu; wenn aber Gottlose aufkommen, wendet sich's
unter den Leuten. \bibverse{13} Wer seine Missetat leugnet, dem wird's
nicht gelingen; wer sie aber bekennet und lässet, der wird
Barmherzigkeit erlangen. \bibverse{14} Wohl dem, der sich allewege
fürchtet! Wer aber halsstarrig ist, wird in Unglück fallen.
\bibverse{15} Ein Gottloser, der über ein arm Volk regieret, das ist ein
brüllender Löwe und gieriger Bär. \bibverse{16} Wenn ein Fürst ohne
Verstand ist, so geschiehet viel Unrechts; wer aber den Geiz hasset, der
wird lange leben. \bibverse{17} Ein Mensch, der am Blut einer Seele
unrecht tut, der wird nicht erhalten, ob er auch in die Hölle führe.
\bibverse{18} Wer fromm einhergehet, wird genesen; wer aber verkehrtes
Weges ist, wird auf einmal zerfallen. \bibverse{19} Wer seinen Acker
bauet, wird Brots genug haben; wer aber Müßiggang nachgehet, wird Armuts
genug haben. \bibverse{20} Ein treuer Mann wird viel gesegnet; wer aber
eilet, reich zu werden, wird nicht unschuldig bleiben. \bibverse{21}
Person ansehen ist nicht gut; denn er täte übel auch wohl um ein Stück
Brots. \bibverse{22} Wer eilet zum Reichtum und ist neidisch, der weiß
nicht, daß ihm Unfall begegnen wird. \bibverse{23} Wer einen Menschen
straft, wird hernach Gunst finden, denn der da heuchelt. \bibverse{24}
Wer seinem Vater oder Mutter nimmt und spricht, es sei nicht Sünde, der
ist des Verderbers Geselle. \bibverse{25} Ein Stolzer erweckt Zank; wer
aber auf den HErrn sich verläßt, wird fett. \bibverse{26} Wer sich auf
sein Herz verläßt, ist ein Narr; wer aber mit Weisheit gehet, wird
entrinnen. \bibverse{27} Wer dem Armen gibt, dem wird nicht mangeln; wer
aber seine Augen abwendet, der wird sehr verderben. \bibverse{28} Wenn
die Gottlosen aufkommen, so verbergen sich die Leute; wenn sie aber
umkommen, wird der Gerechten viel.

\hypertarget{section-28}{%
\section{29}\label{section-28}}

\bibverse{1} Wer wider die Strafe halsstarrig, ist, der wird plötzlich
verderben ohne alle Hilfe. \bibverse{2} Wenn der Gerechten viel ist,
freuet sich das Volk; wenn aber der Gottlose herrschet, seufzet das
Volk. \bibverse{3} Wer Weisheit liebet, erfreuet seinen Vater; wer aber
mit Huren sich nähret, kommt um sein Gut. \bibverse{4} Ein König richtet
das Land auf durchs Recht; ein Geiziger aber verderbet es. \bibverse{5}
Wer mit seinem Nächsten heuchelt, der breitet ein Netz zu seinen
Fußtapfen. \bibverse{6} Wenn ein Böser sündiget, verstrickt er sich
selbst; aber ein Gerechter freuet sich und hat Wonne. \bibverse{7} Der
Gerechte erkennet die Sache der Armen; der Gottlose achtet keine
Vernunft. \bibverse{8} Die Spötter bringen frechlich eine Stadt in
Unglück; aber die Weisen stillen den Zorn. \bibverse{9} Wenn ein Weiser
mit einem Narren zu handeln kommt, er zürne oder lache, so hat er nicht
Ruhe. \bibverse{10} Die Blutgierigen hassen den Frommen; aber die
Gerechten suchen seine Seele. \bibverse{11} Ein Narr schüttet seinen
Geist gar aus; aber ein Weiser hält an sich. \bibverse{12} Ein Herr, der
zu Lügen Lust hat, des Diener sind alle gottlos. \bibverse{13} Arme und
Reiche begegnen einander; aber beider Augen erleuchtet der HErr.
\bibverse{14} Ein König, der die Armen treulich richtet, des Thron wird
ewiglich bestehen. \bibverse{15} Rute und Strafe gibt Weisheit; aber ein
Knabe, ihm selbst gelassen, schändet seine Mutter. \bibverse{16} Wo viel
Gottlose sind, da sind viel Sünden; aber die Gerechten werden ihren Fall
erleben. \bibverse{17} Züchtige deinen Sohn, so wird er dich ergötzen
und wird deiner Seele sanft tun. \bibverse{18} Wenn die Weissagung aus
ist, wird das Volk wild und wüst; wohl aber dem, der das Gesetz
handhabet! \bibverse{19} Ein Knecht läßt sich mit Worten nicht
züchtigen; denn ob er's gleich verstehet, nimmt er sich's doch nicht an.
\bibverse{20} Siehest du einen schnell zu reden, da ist an einem Narren
mehr Hoffnung denn an ihm. \bibverse{21} Wenn ein Knecht von Jugend auf
zärtlich gehalten wird, so will er danach ein Junker sein. \bibverse{22}
Ein zorniger Mann richtet Hader an, und ein Grimmiger tut viel Sünde.
\bibverse{23} Die Hoffart des Menschen wird ihn stürzen; aber der
Demütige wird Ehre empfahen. \bibverse{24} Wer mit Dieben teil hat, hört
fluchen und sagt's nicht an, der hasset sein Leben. \bibverse{25} Vor
Menschen sich scheuen, bringet zu Fall; wer sich aber auf den HErrn
verlässet, wird beschützt. \bibverse{26} Viele suchen das Angesicht
eines Fürsten; aber eines jeglichen Gericht kommt vom HErrn.
\bibverse{27} Ein ungerechter Mann ist dem Gerechten ein Greuel, und wer
rechtes Weges ist, der ist des Gottlosen Greuel.

\hypertarget{section-29}{%
\section{30}\label{section-29}}

\bibverse{1} Dies sind die Worte Agurs, des Sohns Jakes, Lehre und Rede
des Mannes Leithiel, Leithiel und Uchal. \bibverse{2} Denn ich bin der
allernärrischste, und Menschenverstand ist nicht bei mir. \bibverse{3}
Ich habe Weisheit nicht gelernet, und was heilig sei, weiß ich nicht.
\bibverse{4} Wer fähret hinauf gen Himmel und herab? Wer fasset den Wind
in seine Hände? Wer bindet die Wasser in ein Kleid? Wer hat alle Enden
der Welt gestellet? Wie heißt er und wie heißt sein Sohn? Weißt du das?
\bibverse{5} Alle Worte GOttes sind durchläutert und sind ein Schild
denen, die auf ihn trauen. \bibverse{6} Tue nichts zu seinen Worten, daß
er dich nicht strafe, und werdest lügenhaftig erfunden. \bibverse{7}
Zweierlei bitte ich von dir, die wollest du mir nicht weigern, ehe denn
ich sterbe; \bibverse{8} Abgötterei und Lügen laß ferne von mir sein;
Armut und Reichtum gib mir nicht; laß mich aber mein bescheiden Teil
Speise dahinnehmen. \bibverse{9} Ich möchte sonst, wo ich zu satt würde,
verleugnen und sagen: Wer ist der HErr? Oder wo ich zu arm würde, möchte
ich stehlen und mich an dem Namen meines GOttes vergreifen.
\bibverse{10} Verrate den Knecht nicht gegen seinen HErrn; er möchte dir
fluchen und du die Schuld tragen müssest. \bibverse{11} Es ist eine Art,
die ihrem Vater flucht und ihre Mutter nicht segnet; \bibverse{12} eine
Art, die sich rein dünkt und ist doch von ihrem Kot nicht gewaschen;
\bibverse{13} eine Art, die ihre Augen hoch trägt und ihre Augenlider
emporhält; \bibverse{14} eine Art, die Schwerter für Zähne hat, die mit
ihren Backenzähnen frißt und verzehret die Elenden im Lande und die
Armen unter den Leuten. \bibverse{15} Die Igel hat zwo Töchter: Bring
her, bring her! Drei Dinge sind nicht zu sättigen, und das vierte
spricht nicht: Es ist genug: \bibverse{16} die Hölle, der Frauen
verschlossene Mutter, die Erde wird nicht Wassers satt, und das Feuer
spricht nicht: Es ist genug. \bibverse{17} Ein Auge das den Vater
verspottet und verachtet, der Mutter zu gehorchen, das müssen die Raben
am Bach aushacken und die jungen Adler fressen. \bibverse{18} Drei Dinge
sind mir zu wunderlich, und das vierte weiß ich nicht: \bibverse{19} des
Adlers Weg im Himmel, der Schlangen Weg auf einem Felsen, des Schiffs
Weg mitten im Meer und eines Mannes Weg an einer Magd. \bibverse{20}
Also ist auch der Weg der Ehebrecherin; die verschlinget und wischet ihr
Maul und spricht: Ich habe kein Übels getan. \bibverse{21} Ein Land wird
durch dreierlei unruhig, und das vierte mag es nicht ertragen:
\bibverse{22} ein Knecht, wenn er König wird; ein Narr, wenn er zu satt
ist; \bibverse{23} eine Feindselige, wenn sie geehelicht wird, und eine
Magd, wenn sie ihrer Frauen Erbe wird. \bibverse{24} Vier sind klein auf
Erden und klüger denn die Weisen: \bibverse{25} die Ameisen, ein schwach
Volk, dennoch schaffen sie im Sommer ihre Speise; \bibverse{26}
Kaninchen, ein schwach Volk, dennoch legt es sein Haus in den Felsen;
\bibverse{27} Heuschrecken haben, keinen König, dennoch ziehen sie aus
ganz mit Haufen; \bibverse{28} die Spinne wirkt mit ihren Händen und ist
in der Könige Schlössern. \bibverse{29} Dreierlei haben einen feinen
Gang, und das vierte gehet wohl: \bibverse{30} Der Löwe, mächtig unter
den Tieren, und kehrt nicht um vor jemand; \bibverse{31} ein Wind von
guten Lenden; und ein Widder; und der König, wider den sich niemand darf
legen. \bibverse{32} Hast du genarret und zu hoch gefahren und Böses
vorgehabt, so lege die Hand aufs Maul. \bibverse{33} Wenn man Milch
stößt, so macht man Butter draus; und wer die Nase hart schneuzet,
zwingt Blut heraus; und wer den Zorn reizet, zwingt Hader heraus.

\hypertarget{section-30}{%
\section{31}\label{section-30}}

\bibverse{1} Dies sind die Worte des Königs Lamuel, die Lehre, die ihn
seine Mutter lehrete: \bibverse{2} Ach, mein Auserwählter, ach, du Sohn
meines Leibes, ach, mein gewünschter Sohn, \bibverse{3} laß nicht den
Weibern dein Vermögen und gehe die Wege nicht, darin sich die Könige
verderben! \bibverse{4} O, nicht den Königen, Lamuel, gib den Königen
nicht Wein zu trinken noch den Fürsten stark Getränke. \bibverse{5} Sie
möchten trinken und der Rechte vergessen und verändern die Sache irgend
der elenden Leute. \bibverse{6} Gebet stark Getränke denen, die umkommen
sollen, und den Wein den betrübten Seelen, \bibverse{7} daß sie trinken
und ihres Elendes vergessen und ihres Unglücks nicht mehr gedenken.
\bibverse{8} Tu deinen Mund auf für die Stummen und für die Sache aller,
die verlassen sind. \bibverse{9} Tu deinen Mund auf und richte recht und
räche den Elenden und Armen. \bibverse{10} Wem ein tugendsam Weib
bescheret ist, die ist viel edler denn die köstlichsten Perlen.
\bibverse{11} Ihres Mannes Herz darf sich auf sie verlassen, und Nahrung
wird ihm nicht mangeln. \bibverse{12} Sie tut ihm Liebes und kein Leides
sein Leben lang. \bibverse{13} Sie gehet mit Wolle und Flachs um und
arbeitet gerne mit ihren Händen. \bibverse{14} Sie ist wie ein
Kaufmannsschiff, das seine Nahrung von ferne bringt. \bibverse{15} Sie
stehet des Nachts auf und gibt Futter ihrem Hause und Essen ihren
Dirnen. \bibverse{16} Sie denkt nach einem Acker und kauft ihn und
pflanzt einen Weinberg von den Früchten ihrer Hände. \bibverse{17} Sie
gürtet ihre Lenden fest und stärkt ihre Arme. \bibverse{18} Sie merkt,
wie ihr Handel Frommen bringt; ihre Leuchte verlöscht des Nachts nicht.
\bibverse{19} Sie streckt ihre Hand nach dem Rocken, und ihre Finger
fassen die Spindel. \bibverse{20} Sie breitet ihre Hände aus zu dem
Armen und reichet ihre Hand dem Dürftigen. \bibverse{21} Sie fürchtet
ihres Hauses nicht vor dem Schnee, denn ihr ganzes Haus hat zwiefache
Kleider. \bibverse{22} Sie macht ihr selbst Decken; weiße Seide und
Purpur ist ihr Kleid. \bibverse{23} Ihr Mann ist berühmt in den Toren,
wenn er sitzt bei den Ältesten des Landes. \bibverse{24} Sie macht einen
Rock und verkauft ihn; einen Gürtel gibt sie dem Krämer. \bibverse{25}
Ihr Schmuck ist, daß sie reinlich und fleißig ist; und wird hernach
lachen. \bibverse{26} Sie tut ihren Mund auf mit Weisheit, und auf ihrer
Zunge ist holdselige Lehre. \bibverse{27} Sie schauet, wie es in ihrem
Hause zugehet, und isset ihr Brot nicht mit Faulheit. \bibverse{28} Ihre
Söhne kommen auf und preisen sie selig; ihr Mann lobt sie. \bibverse{29}
Viele Töchter bringen Reichtum; du aber übertriffst sie alle.
\bibverse{30} Lieblich und schön sein ist nichts; ein Weib, das den
HErrn fürchtet, soll man loben. \bibverse{31} Sie wird gerühmt werden
von den Früchten ihrer Hände; und ihre Werke werden sie loben in den
Toren.
