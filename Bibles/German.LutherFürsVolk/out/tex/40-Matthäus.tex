\hypertarget{section}{%
\section{1}\label{section}}

\bibverse{1} Dies ist das Buch von der Geburt Jesu Christi, der da ist
ein Sohn Davids, des Sohnes Abrahams. \footnote{\textbf{1:1} 1Chr 17,11;
  1Mo 22,18}

\bibverse{2} Abraham zeugte Isaak. Isaak zeugte Jakob. Jakob zeugte Juda
und seine Brüder. \footnote{\textbf{1:2} 1Mo 21,3; 1Mo 21,12; 1Mo 25,26;
  1Mo 29,35; 1Mo 49,10} \bibverse{3} Juda zeugte Perez und Serah von der
Thamar. Perez zeugte Hezron. Hezron zeugte Ram. \footnote{\textbf{1:3}
  1Mo 38,29-30; Rt 4,18-22} \bibverse{4} Ram zeugte Amminadab. Amminadab
zeugte Nahesson. Nahesson zeugte Salma. \bibverse{5} Salma zeugte Boas
von der Rahab. Boas zeugte Obed von der Ruth. Obed zeugte Jesse.
\bibverse{6} Jesse zeugte den König David. Der König David zeugte Salomo
von dem Weib des Uria. \footnote{\textbf{1:6} 2Sam 12,24} \bibverse{7}
Salomo zeugte Rehabeam. Rehabeam zeugte Abia. Abia zeugte Asa.
\footnote{\textbf{1:7} 1Chr 3,10-16} \bibverse{8} Asa zeugte Josaphat.
Josaphat zeugte Joram. Joram zeugte Usia. \bibverse{9} Usia zeugte
Jotham. Jotham zeugte Ahas. Ahas zeugte Hiskia. \bibverse{10} Hiskia
zeugte Manasse. Manasse zeugte Amon. Amon zeugte Josia. \bibverse{11}
Josia zeugte Jechonja und seine Brüder um die Zeit der babylonischen
Gefangenschaft. \footnote{\textbf{1:11} 2Kö 25,-1}

\bibverse{12} Nach der babylonischen Gefangenschaft zeugte Jechonja
Sealthiel. Sealthiel zeugte Serubabel. \footnote{\textbf{1:12} 1Chr
  3,17; Esr 3,2} \bibverse{13} Serubabel zeugte Abiud. Abiud zeugte
Eliakim. Eliakim zeugte Asor. \bibverse{14} Asor zeugte Zadok. Zadok
zeugte Achim. Achim zeugte Eliud. \bibverse{15} Eliud zeugte Eleasar.
Eleasar zeugte Matthan. Matthan zeugte Jakob. \bibverse{16} Jakob zeugte
Joseph, den Mann Marias, von welcher ist geboren Jesus, der da heißt
Christus. \footnote{\textbf{1:16} Lk 1,27}

\bibverse{17} Alle Glieder von Abraham bis auf David sind vierzehn
Glieder. Von David bis auf die babylonische Gefangenschaft sind vierzehn
Glieder. Von der babylonischen Gefangenschaft bis auf Christus sind
vierzehn Glieder.

\bibverse{18} Die Geburt Christi war aber also getan. Als Maria, seine
Mutter, dem Joseph vertrauet war, fand sich's, ehe er sie heimholte,
dass sie schwanger war von dem heiligen Geist. \bibverse{19} Joseph
aber, ihr Mann, war fromm und wollte sie nicht in Schande bringen,
gedachte aber, sie heimlich zu verlassen. \bibverse{20} Indem er aber
also gedachte, siehe, da erschien ihm ein Engel des Herrn im Traum und
sprach: Joseph, du Sohn Davids, fürchte dich nicht, Maria, dein Gemahl,
zu dir zu nehmen; denn das in ihr geboren ist, das ist von dem heiligen
Geist. \bibverse{21} Und sie wird einen Sohn gebären, des Namen sollst
du Jesus heißen; denn er wird sein Volk selig machen von ihren Sünden.
\footnote{\textbf{1:21} Ps 130,8; Lk 2,21; Apg 4,12}

\bibverse{22} Das ist aber alles geschehen, auf dass erfüllet würde, was
der Herr durch den Propheten gesagt hat, der da spricht: \bibverse{23}
„Siehe, eine Jungfrau wird schwanger sein und einen Sohn gebären, und
sie werden seinen Namen Immanuel heißen``, das ist verdolmetscht: Gott
mit uns.

\bibverse{24} Da nun Joseph vom Schlaf erwachte, tat er, wie ihm des
Herrn Engel befohlen hatte, und nahm sein Gemahl zu sich. \bibverse{25}
Und er erkannte sie nicht, bis sie ihren ersten Sohn gebar; und hieß
seinen Namen Jesus. \# 2 \bibverse{1} Da Jesus geboren war zu Bethlehem
im jüdischen Lande, zur Zeit des Königs Herodes, siehe, da kamen die
Weisen vom Morgenland gen Jerusalem und sprachen: \footnote{\textbf{2:1}
  Lk 2,1-7} \bibverse{2} Wo ist der neugeborene König der Juden? Wir
haben seinen Stern gesehen im Morgenland und sind gekommen, ihn
anzubeten. \footnote{\textbf{2:2} 4Mo 24,17} \bibverse{3} Da das der
König Herodes hörte, erschrak er und mit ihm das ganze Jerusalem.
\bibverse{4} Und ließ versammeln alle Hohenpriester und Schriftgelehrten
unter dem Volk und erforschte von ihnen, wo Christus sollte geboren
werden. \bibverse{5} Und sie sagten ihm: Zu Bethlehem im jüdischen
Lande; denn also steht geschrieben durch den Propheten: \footnote{\textbf{2:5}
  Joh 7,42} \bibverse{6} „Und du, Bethlehem im jüdischen Lande, bist
mitnichten die kleinste unter den Fürsten Judas denn aus dir soll mir
kommen der Herzog, der über mein Volk Israel ein Herr sei.``

\bibverse{7} Da berief Herodes die Weisen heimlich und erlernte mit
Fleiß von ihnen, wann der Stern erschienen wäre, \bibverse{8} und wies
sie gen Bethlehem und sprach: Ziehet hin und forschet fleißig nach dem
Kindlein; wenn ihr's findet, so sagt mir's wieder, dass ich auch komme
und es anbete.

\bibverse{9} Als sie nun den König gehört hatten, zogen sie hin. Und
siehe, der Stern, den sie im Morgenland gesehen hatten, ging vor ihnen
hin, bis dass er kam und stand oben über, da das Kindlein war.
\bibverse{10} Da sie den Stern sahen, wurden sie hoch erfreut
\bibverse{11} und gingen in das Haus und fanden das Kindlein mit Maria,
seiner Mutter, und fielen nieder und beteten es an und taten ihre
Schätze auf und schenkten ihm Gold, Weihrauch und Myrrhe. \bibverse{12}
Und Gott befahl ihnen im Traum, dass sie sich nicht sollten wieder zu
Herodes lenken; und sie zogen durch einen anderen Weg wieder in ihr
Land.

\bibverse{13} Da sie aber hinweggezogen waren, siehe, da erschien der
Engel des Herrn dem Joseph im Traum und sprach: Stehe auf und nimm das
Kindlein und seine Mutter zu dir und flieh nach Ägyptenland und bleib
allda, bis ich dir sage; denn es ist vorhanden, dass Herodes das
Kindlein suche, dasselbe umzubringen.

\bibverse{14} Und er stand auf und nahm das Kindlein und seine Mutter zu
sich bei der Nacht und entwich nach Ägyptenland. \bibverse{15} Und blieb
allda bis nach dem Tod des Herodes, auf dass erfüllet würde, was der
Herr durch den Propheten gesagt hat, der da spricht: „Aus Ägypten habe
ich meinen Sohn gerufen.``

\bibverse{16} Da Herodes nun sah, dass er von den Weisen betrogen war,
ward er sehr zornig und schickte aus und ließ alle Kinder zu Bethlehem
töten und an seinen ganzen Grenzen, die da zweijährig und darunter
waren, nach der Zeit, die er mit Fleiß von den Weisen erlernt hatte.
\bibverse{17} Da ist erfüllt, was gesagt ist von dem Propheten Jeremia,
der da spricht: \bibverse{18} „Auf dem Gebirge hat man ein Geschrei
gehört, viel Klagens, Weinens und Heulens; Rahel beweinte ihre Kinder
und wollte sich nicht trösten lassen, denn es war aus mit ihnen.``
\footnote{\textbf{2:18} 1Mo 35,19}

\bibverse{19} Da aber Herodes gestorben war, siehe, da erschien der
Engel des Herrn dem Joseph im Traum in Ägyptenland \bibverse{20} und
sprach: Stehe auf und nimm das Kindlein und seine Mutter zu dir und zieh
hin in das Land Israel; sie sind gestorben, die dem Kinde nach dem Leben
standen.

\bibverse{21} Und er stand auf und nahm das Kindlein und seine Mutter zu
sich und kam in das Land Israel. \bibverse{22} Da er aber hörte, dass
Archelaus im jüdischen Lande König war anstatt seines Vaters Herodes,
fürchtete er sich, dahin zu kommen. Und im Traum empfing er Befehl von
Gott und zog in die Örter des galiläischen Landes. \bibverse{23} und kam
und wohnte in der Stadt die da heißt Nazareth; auf dass erfüllet würde,
was da gesagt ist durch die Propheten: Er soll Nazarenus heißen.
\footnote{\textbf{2:23} Lk 2,39; Joh 1,46}

\hypertarget{section-1}{%
\section{3}\label{section-1}}

\bibverse{1} Zu der Zeit (aber) kam Johannes der Täufer und predigte in
der Wüste des jüdischen Landes \footnote{\textbf{3:1} Lk 1,13}
\bibverse{2} und sprach: Tut Buße, das Himmelreich ist nahe
herbeigekommen! \footnote{\textbf{3:2} Mt 4,17; Röm 12,2} \bibverse{3}
Und er ist der, von dem der Prophet Jesaja gesagt hat und gesprochen:
„Es ist eine Stimme eines Predigers in der Wüste: Bereitet dem Herrn den
Weg und machet richtig seine Steige!{}`` \footnote{\textbf{3:3} Joh 1,23}

\bibverse{4} Er aber, Johannes, hatte ein Kleid von Kamelhaaren und
einen ledernen Gürtel um seine Lenden; seine Speise aber war
Heuschrecken und wilder Honig. \footnote{\textbf{3:4} 2Kö 1,8}
\bibverse{5} Da ging zu ihm hinaus die Stadt Jerusalem und das ganze
jüdische Land und alle Länder an dem Jordan \bibverse{6} und ließen sich
taufen von ihm im Jordan und bekannten ihre Sünden.

\bibverse{7} Als er nun viele Pharisäer und Sadduzäer sah zu seiner
Taufe kommen, sprach er zu ihnen: Ihr Otterngezüchte, wer hat denn euch
gewiesen, dass ihr dem künftigen Zorn entrinnen werdet? \bibverse{8}
Sehet zu, tut rechtschaffene Frucht der Buße! \bibverse{9} Denket nur
nicht, dass ihr bei euch wollt sagen: Wir haben Abraham zum Vater. Ich
sage euch: Gott vermag dem Abraham aus diesen Steinen Kinder zu
erwecken. \footnote{\textbf{3:9} Joh 8,33; Joh 8,39; Röm 2,28-29; Röm
  4,12} \bibverse{10} Es ist schon die Axt den Bäumen an die Wurzel
gelegt. Darum, welcher Baum nicht gute Frucht bringt, wird abgehauen und
ins Feuer geworfen. \footnote{\textbf{3:10} Lk 13,6-9}

\bibverse{11} Ich taufe euch mit Wasser zur Buße; der aber nach mir
kommt, ist stärker als ich, dem ich auch nicht genugsam bin, seine
Schuhe zu tragen; der wird euch mit dem Heiligen Geist und mit Feuer
taufen. \footnote{\textbf{3:11} Joh 1,26-27; Joh 1,33; Apg 1,5; Apg
  2,3-4} \bibverse{12} Und er hat seine Wurfschaufel in der Hand: er
wird seine Tenne fegen und den Weizen in seine Scheune sammeln; aber die
Spreu wird er verbrennen mit ewigem Feuer. \footnote{\textbf{3:12} Mt
  13,30}

\bibverse{13} Zu der Zeit kam Jesus aus Galiläa an den Jordan zu
Johannes, dass er sich von ihm taufen ließe. \bibverse{14} Aber Johannes
wehrte ihm und sprach: Ich bedarf wohl, dass ich von dir getauft werde,
und du kommst zu mir? \footnote{\textbf{3:14} Joh 13,6}

\bibverse{15} Jesus aber antwortete und sprach zu ihm: Lass es jetzt
also sein! also gebührt es uns, alle Gerechtigkeit zu erfüllen. Da ließ
er's ihm zu.

\bibverse{16} Und da Jesus getauft war, stieg er alsbald herauf aus dem
Wasser; und siehe, da tat sich der Himmel auf über ihm. Und er sah den
Geist Gottes gleich als eine Taube herabfahren und über ihn kommen.
\bibverse{17} Und siehe, eine Stimme vom Himmel herab sprach: Dies ist
mein lieber Sohn, an welchem ich Wohlgefallen habe. \footnote{\textbf{3:17}
  Mt 17,5; Jes 42,1}

\hypertarget{section-2}{%
\section{4}\label{section-2}}

\bibverse{1} Da ward Jesus vom Geist in die Wüste geführt, auf dass er
von dem Teufel versucht würde. \footnote{\textbf{4:1} Hebr 4,15}
\bibverse{2} Und da er vierzig Tage und vierzig Nächte gefastet hatte,
hungerte ihn. \footnote{\textbf{4:2} 2Mo 34,28; 1Kö 19,8} \bibverse{3}
Und der Versucher trat zu ihm und sprach: Bist du Gottes Sohn, so
sprich, dass diese Steine Brot werden. \footnote{\textbf{4:3} 1Mo 3,1-7}

\bibverse{4} Und er antwortete und sprach: Es steht geschrieben: „Der
Mensch lebt nicht vom Brot allein, sondern von einem jeglichen Wort, das
durch den Mund Gottes geht.``

\bibverse{5} Da führte ihn der Teufel mit sich in die heilige Stadt und
stellte ihn auf die Zinne des Tempels \bibverse{6} und sprach zu ihm:
Bist du Gottes Sohn, so lass dich hinab; denn es steht geschrieben: „Er
wird seinen Engeln über dir Befehl tun, und sie werden dich auf Händen
tragen, auf dass du deinen Fuß nicht an einen Stein stoßest.``

\bibverse{7} Da sprach Jesus zu ihm: Wiederum steht auch geschrieben:
„Du sollst Gott, deinen Herrn, nicht versuchen.``

\bibverse{8} Wiederum führte ihn der Teufel mit sich auf einen sehr
hohen Berg und zeigte ihm alle Reiche der Welt und ihre Herrlichkeit
\bibverse{9} und sprach zu ihm: Das alles will ich dir geben, wenn du
niederfällst und mich anbetest. \footnote{\textbf{4:9} Mt 16,26}

\bibverse{10} Da sprach Jesus zu ihm: Hebe dich weg von mir, Satan! denn
es steht geschrieben: „Du sollst anbeten Gott, deinen Herrn, und ihm
allein dienen.``

\bibverse{11} Da verließ ihn der Teufel; und siehe, da traten die Engel
zu ihm und dienten ihm.

\bibverse{12} Da nun Jesus hörte, dass Johannes überantwortet war, zog
er in das galiläische Land. \footnote{\textbf{4:12} Mt 14,3}
\bibverse{13} Und verließ die Stadt Nazareth, kam und wohnte zu
Kapernaum, das da liegt am Meer, im Lande Sebulon und Naphthali,
\bibverse{14} auf dass erfüllet würde, was da gesagt ist durch den
Propheten Jesaja, der da spricht: \bibverse{15} „Das Land Sebulon und
das Land Naphthali, am Wege des Meeres, jenseits des Jordans, und das
heidnische Galiläa, \bibverse{16} das Volk, das in Finsternis saß, hat
ein großes Licht gesehen; und die da saßen am Ort und Schatten des
Todes, denen ist ein Licht aufgegangen.``

\bibverse{17} Von der Zeit an fing Jesus an, zu predigen und zu sagen:
Tut Buße, das Himmelreich ist nahe herbeigekommen! \footnote{\textbf{4:17}
  Mt 3,2}

\bibverse{18} Als nun Jesus an dem Galiläischen Meer ging, sah er zwei
Brüder, Simon, der da heißt Petrus, und Andreas, seinen Bruder, die
warfen ihre Netze ins Meer; denn sie waren Fischer. \bibverse{19} Und er
sprach zu ihnen: Folget mir nach; ich will euch zu Menschenfischern
machen!

\bibverse{20} Alsbald verließen sie ihre Netze und folgten ihm nach.
\footnote{\textbf{4:20} Mt 19,27} \bibverse{21} Und da er von da
weiterging, sah er zwei andere Brüder, Jakobus, den Sohn des Zebedäus,
und Johannes, seinen Bruder, im Schiff mit ihrem Vater Zebedäus, dass
sie ihre Netze flickten; und er rief sie. \bibverse{22} Alsbald
verließen sie das Schiff und ihren Vater und folgten ihm nach.

\bibverse{23} Und Jesus ging umher im ganzen galiläischen Lande, lehrte
in ihren Schulen und predigte das Evangelium von dem Reich und heilte
allerlei Seuche und Krankheit im Volk. \bibverse{24} Und sein Gerücht
erscholl in das ganze Syrienland. Und sie brachten zu ihm allerlei
Kranke, mit mancherlei Seuchen und Qual behaftet, die Besessenen, die
Mondsüchtigen und die Gichtbrüchigen; und er machte sie alle gesund.
\footnote{\textbf{4:24} Mk 6,55} \bibverse{25} Und es folgte ihm nach
viel Volks aus Galiläa, aus den Zehn-Städten, von Jerusalem, aus dem
jüdischen Lande und von jenseits des Jordans. \footnote{\textbf{4:25} Mk
  3,7-8; Lk 6,17-19}

\hypertarget{section-3}{%
\section{5}\label{section-3}}

\bibverse{1} Da er aber das Volk sah, ging er auf einen Berg und setzte
sich; und seine Jünger traten zu ihm, \bibverse{2} Und er tat seinen
Mund auf, lehrte sie und sprach: \bibverse{3} Selig sind, die da
geistlich arm sind; denn das Himmelreich ist ihrer. \footnote{\textbf{5:3}
  Ps 51,19; Jes 57,15} \bibverse{4} Selig sind, die da Leid tragen; denn
sie sollen getröstet werden. \footnote{\textbf{5:4} Ps 126,5; Offb 7,17}
\bibverse{5} Selig sind die Sanftmütigen; denn sie werden das Erdreich
besitzen. \footnote{\textbf{5:5} Mt 11,29; Ps 37,11} \bibverse{6} Selig
sind, die da hungert und dürstet nach der Gerechtigkeit; denn sie sollen
satt werden. \footnote{\textbf{5:6} Lk 18,9-14; Joh 6,35} \bibverse{7}
Selig sind die Barmherzigen; denn sie werden Barmherzigkeit erlangen.
\footnote{\textbf{5:7} Mt 25,35-46; Jak 2,13} \bibverse{8} Selig sind,
die reines Herzens sind; denn sie werden Gott schauen. \footnote{\textbf{5:8}
  Ps 24,3-5; Ps 51,12; 1Jo 3,2; 1Jo 1,3} \bibverse{9} Selig sind die
Friedfertigen; denn sie werden Gottes Kinder heißen. \footnote{\textbf{5:9}
  Hebr 12,14} \bibverse{10} Selig sind, die um Gerechtigkeit willen
verfolgt werden; denn das Himmelreich ist ihrer. \footnote{\textbf{5:10}
  1Petr 3,14}

\bibverse{11} Selig seid ihr, wenn euch die Menschen um meinetwillen
schmähen und verfolgen und reden allerlei Übles wider euch, so sie daran
lügen. \footnote{\textbf{5:11} Mt 10,22; Apg 5,41; 1Petr 4,14}
\bibverse{12} Seid fröhlich und getrost; es wird euch im Himmel wohl
belohnt werden. Denn also haben sie verfolgt die Propheten, die vor euch
gewesen sind. \footnote{\textbf{5:12} Jak 5,10; Hebr 11,33-38}

\bibverse{13} Ihr seid das Salz der Erde. Wo nun das Salz dumm wird,
womit soll man's salzen? Es ist hinfort zu nichts nütze, denn das man es
hinausschütte und lasse es die Leute zertreten. \footnote{\textbf{5:13}
  Mk 9,50; Lk 14,34-35}

\bibverse{14} Ihr seid das Licht der Welt. Es kann die Stadt, die auf
einem Berge liegt, nicht verborgen sein. \footnote{\textbf{5:14} Joh
  8,12} \bibverse{15} Man zündet auch nicht ein Licht an und setzt es
unter einen Scheffel, sondern auf einen Leuchter; so leuchtet es denn
allen, die im Hause sind. \footnote{\textbf{5:15} Mk 4,21; Lk 8,16}
\bibverse{16} Also lasset euer Licht leuchten vor den Leuten, dass sie
eure guten Werke sehen und euren Vater im Himmel preisen. \footnote{\textbf{5:16}
  Joh 15,8; Eph 5,8-9; Phil 2,14-15}

\bibverse{17} Ihr sollt nicht wähnen, dass ich gekommen bin, das Gesetz
oder die Propheten aufzulösen; ich bin nicht gekommen, aufzulösen,
sondern zu erfüllen. \footnote{\textbf{5:17} Mt 3,15; Röm 3,31; 1Jo 2,7}
\bibverse{18} Denn ich sage euch wahrlich: Bis dass Himmel und Erde
zergehe, wird nicht zergehen der kleinste Buchstabe noch ein Tüttel vom
Gesetz, bis dass es alles geschehe. \footnote{\textbf{5:18} Lk 16,17}
\bibverse{19} Wer nun eins von diesen kleinsten Geboten auflöst und
lehrt die Leute also, der wird der Kleinste heißen im Himmelreich; wer
es aber tut und lehrt, der wird groß heißen im Himmelreich. \footnote{\textbf{5:19}
  Jak 2,10} \bibverse{20} Denn ich sage euch: Es sei denn eure
Gerechtigkeit besser als der Schriftgelehrten und Pharisäer, so werdet
ihr nicht in das Himmelreich kommen. \footnote{\textbf{5:20} Mt 23,2-33}

\bibverse{21} Ihr habt gehört, dass zu den Alten gesagt ist: „Du sollst
nicht töten; wer aber tötet, der soll des Gerichts schuldig sein.``
\bibverse{22} Ich aber sage euch: Wer mit seinem Bruder zürnet, der ist
des Gerichts schuldig; wer aber zu seinem Bruder sagt: Racha! der ist
des Rats schuldig; wer aber sagt: Du Narr! der ist des höllischen Feuers
schuldig. \footnote{\textbf{5:22} 1Jo 3,15}

\bibverse{23} Darum, wenn du deine Gabe auf dem Altar opferst und wirst
allda eingedenk, dass dein Bruder etwas wider dich habe, \bibverse{24}
so lass allda vor dem Altar deine Gabe und gehe zuvor hin und versöhne
dich mit deinem Bruder, und alsdann komm und opfere deine Gabe.
\bibverse{25} Sei willfährig deinem Widersacher bald, dieweil du noch
bei ihm auf dem Wege bist, auf dass dich der Widersacher nicht
dermaleinst überantworte dem Richter, und der Richter überantworte dich
dem Diener, und werdest in den Kerker geworfen. \footnote{\textbf{5:25}
  Mt 18,23-35; Lk 12,58-59} \bibverse{26} Ich sage dir wahrlich: Du
wirst nicht von dannen herauskommen, bis du auch den letzten Heller
bezahlest.

\bibverse{27} Ihr habt gehört, dass zu den Alten gesagt ist: „Du sollst
nicht ehebrechen.`` \bibverse{28} Ich aber sage euch: Wer ein Weib
ansieht, ihrer zu begehren, der hat schon mit ihr die Ehe gebrochen in
seinem Herzen. \bibverse{29} Ärgert dich aber dein rechtes Auge, so reiß
es aus und wirf's von dir. Es ist dir besser, dass eins deiner Glieder
verderbe, und nicht der ganze Leib in die Hölle geworfen werde.
\footnote{\textbf{5:29} Mt 18,8-9; Mk 9,43; Mk 9,47; Kol 3,5}
\bibverse{30} Ärgert dich deine rechte Hand, so haue sie ab und wirf sie
von dir. Es ist dir besser, dass eins deiner Glieder verderbe, und nicht
der ganze Leib in die Hölle geworfen werde.

\bibverse{31} Es ist auch gesagt: „Wer sich von seinem Weibe scheidet,
der soll ihr geben einen Scheidebrief.`` \bibverse{32} Ich aber sage
euch: Wer sich von seinem Weibe scheidet (es sei denn um Ehebruch), der
macht, dass sie die Ehe bricht; und wer eine Abgeschiedene freit, der
bricht die Ehe. \footnote{\textbf{5:32} Lk 16,18; 1Kor 7,10-11}

\bibverse{33} Ihr habt weiter gehört, dass zu den Alten gesagt ist: „Du
sollst keinen falschen Eid tun und sollst Gott deinen Eid halten.``
\bibverse{34} Ich aber sage euch, dass ihr überhaupt nicht schwören
sollt, weder bei dem Himmel, denn er ist Gottes Stuhl, \bibverse{35}
noch bei der Erde, denn sie ist seiner Füße Schemel, noch bei Jerusalem,
denn sie ist des großen Königs Stadt. \footnote{\textbf{5:35} Ps 48,3}
\bibverse{36} Auch sollst du nicht bei deinem Haupt schwören, denn du
vermagst nicht, ein einziges Haar weiß oder schwarz zu machen.
\bibverse{37} Eure Rede aber sei: Ja, ja; nein, nein. Was darüber ist,
das ist vom Übel.

\bibverse{38} Ihr habt gehört, dass da gesagt ist: „Auge um Auge, Zahn
um Zahn.`` \bibverse{39} Ich aber sage euch, dass ihr nicht widerstreben
sollt dem Übel; sondern, so dir jemand einen Streich gibt auf deinen
rechten Backen, dem biete den anderen auch dar. \footnote{\textbf{5:39}
  Kla 3,27-32; Joh 18,22-23; Röm 12,19; Röm 12,21; 1Petr 2,20-23}
\bibverse{40} Und wenn jemand mit dir rechten will und deinen Rock
nehmen, dem lass auch den Mantel. \footnote{\textbf{5:40} 1Kor 6,7; Hebr
  10,34} \bibverse{41} Und so dich jemand nötigt eine Meile, so gehe mit
ihm zwei. \bibverse{42} Gib dem, der dich bittet, und wende dich nicht
von dem, der dir abborgen will.

\bibverse{43} Ihr habt gehört, dass gesagt ist: „Du sollst deinen
Nächsten lieben und deinen Feind hassen.`` \bibverse{44} Ich aber sage
euch: Liebet eure Feinde; segnet, die euch fluchen; tut wohl denen, die
euch hassen; bittet für die, die euch beleidigen und verfolgen,
\footnote{\textbf{5:44} 2Mo 23,4-5; Lk 6,27-28; Lk 23,34; Röm 12,14; Röm
  12,20; Apg 7,59} \bibverse{45} auf dass ihr Kinder seid eures Vaters
im Himmel; denn er lässt seine Sonne aufgehen über die Bösen und über
die Guten und lässt regnen über Gerechte und Ungerechte. \footnote{\textbf{5:45}
  Eph 5,1} \bibverse{46} Denn so ihr liebet, die euch lieben, was werdet
ihr für Lohn haben? Tun nicht dasselbe auch die Zöllner? \bibverse{47}
Und so ihr euch nur zu euren Brüdern freundlich tut, was tut ihr
Sonderliches? Tun nicht die Zöllner auch also? \bibverse{48} Darum sollt
ihr vollkommen sein, gleichwie euer Vater im Himmel vollkommen ist.
\footnote{\textbf{5:48} 3Mo 19,2}

\hypertarget{section-4}{%
\section{6}\label{section-4}}

\bibverse{1} Habt Acht auf eure Almosen, dass ihr die nicht gebet vor
den Leuten, dass ihr von ihnen gesehen werdet; ihr habt anders keinen
Lohn bei eurem Vater im Himmel. \bibverse{2} Wenn du nun Almosen gibst,
sollst du nicht lassen vor dir posaunen, wie die Heuchler tun in den
Schulen und auf den Gassen, auf dass sie von den Leuten gepriesen
werden. Wahrlich ich sage euch: Sie haben ihren Lohn dahin. \bibverse{3}
Wenn du aber Almosen gibst, so lass deine linke Hand nicht wissen, was
die rechte tut, \footnote{\textbf{6:3} Mt 25,37-40; Röm 12,8}
\bibverse{4} auf dass dein Almosen verborgen sei; und dein Vater, der in
das Verborgene sieht, wird dir's vergelten öffentlich.

\bibverse{5} Und wenn du betest, sollst du nicht sein wie die Heuchler,
die da gern stehen und beten in den Schulen und an den Ecken auf den
Gassen, auf dass sie von den Leuten gesehen werden. Wahrlich ich sage
euch: Sie haben ihren Lohn dahin. \bibverse{6} Wenn aber du betest, so
gehe in dein Kämmerlein und schließ die Tür zu und bete zu deinem Vater
im Verborgenen; und dein Vater, der in das Verborgene sieht, wird dir's
vergelten öffentlich. \bibverse{7} Und wenn ihr betet, sollt ihr nicht
viel plappern wie die Heiden; denn sie meinen, sie werden erhört, wenn
sie viel Worte machen. \bibverse{8} Darum sollt ihr euch ihnen nicht
gleichstellen. Euer Vater weiß, was ihr bedürfet, ehe denn ihr ihn
bittet. \bibverse{9} Darum sollt ihr also beten: Unser Vater in dem
Himmel! Dein Name werde geheiligt. \footnote{\textbf{6:9} Hes 36,23; Lk
  11,2-4} \bibverse{10} Dein Reich komme. Dein Wille geschehe auf Erden
wie im Himmel. \footnote{\textbf{6:10} Lk 22,42} \bibverse{11} Unser
täglich Brot gib uns heute. \bibverse{12} Und vergib uns unsere Schuld,
wie wir unseren Schuldigern vergeben. \footnote{\textbf{6:12} Mt
  18,21-35} \bibverse{13} Und führe uns nicht in Versuchung, sondern
erlöse uns von dem Übel. Denn dein ist das Reich und die Kraft und die
Herrlichkeit in Ewigkeit. Amen. \footnote{\textbf{6:13} 1Chr 29,11-13;
  Joh 17,15}

\bibverse{14} Denn so ihr den Menschen ihre Fehler vergebet, so wird
euch euer himmlischer Vater auch vergeben, \bibverse{15} Wo ihr aber den
Menschen ihre Fehler nicht vergebet, so wird euch euer Vater eure Fehler
auch nicht vergeben. \footnote{\textbf{6:15} Mk 11,25-26}

\bibverse{16} Wenn ihr fastet, sollt ihr nicht sauer sehen wie die
Heuchler; denn sie verstellen ihr Angesicht, auf dass sie vor den Leuten
scheinen mit ihrem Fasten. Wahrlich ich sage euch: Sie haben ihren Lohn
dahin. \footnote{\textbf{6:16} Jes 58,5-9} \bibverse{17} Wenn du aber
fastest, so salbe dein Haupt und wasche dein Angesicht, \bibverse{18}
auf dass du nicht scheinest vor den Leuten mit deinem Fasten, sondern
vor deinem Vater, welcher verborgen ist; und dein Vater, der in das
Verborgene sieht, wird dir's vergelten öffentlich.

\bibverse{19} Ihr sollt euch nicht Schätze sammeln auf Erden, da sie die
Motten und der Rost fressen und da die Diebe nachgraben und stehlen.
\bibverse{20} Sammelt euch aber Schätze im Himmel, da sie weder Motten
noch Rost fressen und da die Diebe nicht nachgraben noch stehlen.
\footnote{\textbf{6:20} Mt 19,21; Lk 12,33-34; Kol 3,1-2} \bibverse{21}
Denn wo euer Schatz ist, da ist auch euer Herz.

\bibverse{22} Das Auge ist des Leibes Licht. Wenn dein Auge einfältig
ist, so wird dein ganzer Leib licht sein; \bibverse{23} ist aber dein
Auge ein Schalk, so wird dein ganzer Leib finster sein. Wenn nun das
Licht, das in dir ist, Finsternis ist, wie groß wird dann die Finsternis
sein! \footnote{\textbf{6:23} Joh 11,10}

\bibverse{24} Niemand kann zwei Herren dienen: entweder er wird den
einen hassen und den anderen lieben, oder er wird dem einen anhangen und
den anderen verachten. Ihr könnt nicht Gott dienen und dem Mammon.
\footnote{\textbf{6:24} Lk 16,9; Lk 16,13; Jak 4,4} \bibverse{25} Darum
sage ich euch: Sorget nicht für euer Leben, was ihr essen und trinken
werdet, auch nicht für euren Leib, was ihr anziehen werdet. Ist nicht
das Leben mehr denn die Speise? und der Leib mehr denn die Kleidung?
\footnote{\textbf{6:25} Phil 4,6; 1Petr 5,7; Lk 12,22-31} \bibverse{26}
Sehet die Vögel unter dem Himmel an: sie säen nicht, sie ernten nicht,
sie sammeln nicht in die Scheunen; und euer himmlischer Vater nährt sie
doch. Seid ihr denn nicht viel mehr denn sie? \footnote{\textbf{6:26} Mt
  10,29-31; Lk 12,6-7}

\bibverse{27} Wer ist aber unter euch, der seiner Länge eine Elle
zusetzen möge, ob er gleich darum sorget? \bibverse{28} Und warum sorget
ihr für die Kleidung? Schauet die Lilien auf dem Felde, wie sie wachsen:
sie arbeiten nicht, auch spinnen sie nicht. \bibverse{29} Ich sage euch,
dass auch Salomo in aller seiner Herrlichkeit nicht bekleidet gewesen
ist wie derselben eins. \footnote{\textbf{6:29} 1Kö 10,-1} \bibverse{30}
So denn Gott das Gras auf dem Felde also kleidet, das doch heute steht
und morgen in den Ofen geworfen wird: sollte er das nicht viel mehr euch
tun, o ihr Kleingläubigen?

\bibverse{31} Darum sollt ihr nicht sorgen und sagen: Was werden wir
essen, was werden wir trinken, womit werden wir uns kleiden?
\bibverse{32} Nach solchem allem trachten die Heiden. Denn euer
himmlischer Vater weiß, dass ihr des alles bedürfet. \bibverse{33}
Trachtet am ersten nach dem Reich Gottes und nach seiner Gerechtigkeit,
so wird euch solches alles zufallen. \bibverse{34} Darum sorgt nicht für
den anderen Morgen; denn der morgende Tag wird für das Seine sorgen. Es
ist genug, dass ein jeglicher Tag seine eigene Plage habe. \footnote{\textbf{6:34}
  2Mo 16,19}

\hypertarget{section-5}{%
\section{7}\label{section-5}}

\bibverse{1} Richtet nicht, auf dass ihr nicht gerichtet werdet.
\footnote{\textbf{7:1} Röm 2,1; 1Kor 4,5} \bibverse{2} Denn mit
welcherlei Gericht ihr richtet, werdet ihr gerichtet werden; und mit
welcherlei Maß ihr messet, wird euch gemessen werden. \footnote{\textbf{7:2}
  Mk 4,24} \bibverse{3} Was siehst du aber den Splitter in deines
Bruders Auge, und wirst nicht gewahr des Balkens in deinem Auge?
\bibverse{4} Oder wie darfst du sagen zu deinem Bruder: Halt, ich will
dir den Splitter aus deinem Auge ziehen, -- und siehe, ein Balken ist in
deinem Auge? \bibverse{5} Du Heuchler, zieh am ersten den Balken aus
deinem Auge; darnach siehe zu, wie du den Splitter aus deines Bruders
Auge ziehest!

\bibverse{6} Ihr sollt das Heiligtum nicht den Hunden geben, und eure
Perlen sollt ihr nicht vor die Säue werfen, auf dass sie dieselben nicht
zertreten mit ihren Füßen und sich wenden und euch zerreißen.

\bibverse{7} Bittet, so wird euch gegeben; suchet, so werdet ihr finden;
klopfet an, so wird euch aufgetan. \footnote{\textbf{7:7} Jer 29,13-14;
  Mk 11,24; Lk 11,5-13; Joh 14,13} \bibverse{8} Denn wer da bittet, der
empfängt; und wer da sucht, der findet; und wer da anklopft, dem wird
aufgetan. \bibverse{9} Welcher ist unter euch Menschen, wenn ihn sein
Sohn bittet ums Brot, der ihm einen Stein biete? \bibverse{10} oder,
wenn er ihn bittet um einen Fisch, der ihm eine Schlange biete?
\bibverse{11} So denn ihr, die ihr doch arg seid, könnt dennoch euren
Kindern gute Gaben geben, wie viel mehr wird euer Vater im Himmel Gutes
geben denen, die ihn bitten! \bibverse{12} Alles nun, was ihr wollt,
dass euch die Leute tun sollen, das tut ihr ihnen auch. Das ist das
Gesetz und die Propheten. \footnote{\textbf{7:12} Mt 22,36-40; Röm
  13,8-10; Gal 5,14}

\bibverse{13} Gehet ein durch die enge Pforte. Denn die Pforte ist weit,
und der Weg ist breit, der zur Verdammnis abführet; und ihrer sind
viele, die darauf wandeln. \footnote{\textbf{7:13} Lk 13,24}
\bibverse{14} Und die Pforte ist eng, und der Weg ist schmal, der zum
Leben führet; und wenige sind ihrer, die ihn finden. \footnote{\textbf{7:14}
  Mt 19,24; Apg 14,22}

\bibverse{15} Seht euch vor vor den falschen Propheten, die in
Schafskleidern zu euch kommen, inwendig aber sind sie reißende Wölfe.
\footnote{\textbf{7:15} Mt 24,4-5; Mt 24,24; 2Kor 11,13-15}
\bibverse{16} An ihren Früchten sollt ihr sie erkennen. Kann man auch
Trauben lesen von den Dornen oder Feigen von den Disteln? \footnote{\textbf{7:16}
  Gal 5,19-22; Jak 3,12} \bibverse{17} Also ein jeglicher guter Baum
bringt gute Früchte; aber ein fauler Baum bringt arge Früchte.
\footnote{\textbf{7:17} Mt 12,33} \bibverse{18} Ein guter Baum kann
nicht arge Früchte bringen, und ein fauler Baum kann nicht gute Früchte
bringen. \bibverse{19} Ein jeglicher Baum, der nicht gute Früchte
bringt, wird abgehauen und ins Feuer geworfen. \footnote{\textbf{7:19}
  Mt 3,10; Joh 15,2; Joh 15,6} \bibverse{20} Darum an ihren Früchten
sollt ihr sie erkennen.

\bibverse{21} Es werden nicht alle, die zu mir sagen: Herr, Herr! in das
Himmelreich kommen, sondern die den Willen tun meines Vaters im Himmel.
\bibverse{22} Es werden viele zu mir sagen an jenem Tage: Herr, Herr!
haben wir nicht in deinem Namen geweissagt, haben wir nicht in deinem
Namen Teufel ausgetrieben, und haben wir nicht in deinem Namen viele
Taten getan? \footnote{\textbf{7:22} Jer 27,13; Lk 13,25-27}
\bibverse{23} Dann werde ich ihnen bekennen: Ich habe euch noch nie
erkannt; weichet alle von mir, ihr Übeltäter! \footnote{\textbf{7:23} Mt
  25,12; 2Tim 2,19}

\bibverse{24} Darum, wer diese meine Rede hört und tut sie, den
vergleiche ich einem klugen Mann, der sein Haus auf einen Felsen baute.
\bibverse{25} Da nun ein Platzregen fiel und ein Gewässer kam und wehten
die Winde und stießen an das Haus, fiel es doch nicht; denn es war auf
einen Felsen gegründet. \bibverse{26} Und wer diese meine Rede hört und
tut sie nicht, der ist einem törichten Manne gleich, der sein Haus auf
den Sand baute. \bibverse{27} Da nun ein Platzregen fiel und kam ein
Gewässer und wehten die Winde und stießen an das Haus, da fiel es und
tat einen großen Fall. \footnote{\textbf{7:27} Hes 13,10-11}

\bibverse{28} Und es begab sich, da Jesus diese Rede vollendet hatte,
entsetzte sich das Volk über seine Lehre; \footnote{\textbf{7:28} Apg
  2,12} \bibverse{29} denn er predigte gewaltig und nicht wie die
Schriftgelehrten. \footnote{\textbf{7:29} Joh 7,16; Joh 7,46}

\hypertarget{section-6}{%
\section{8}\label{section-6}}

\bibverse{1} Da er aber vom Berge herabging, folgte ihm viel Volks nach.
\bibverse{2} Und siehe, ein Aussätziger kam und betete ihn an und
sprach: Herr, wenn du willst, kannst du mich wohl reinigen.

\bibverse{3} Und Jesus streckte seine Hand aus, rührte ihn an und
sprach: Ich will's tun; sei gereinigt! Und alsbald ward er von seinem
Aussatz rein. \bibverse{4} Und Jesus sprach zu ihm: Siehe zu, sage es
niemand; sondern gehe hin und zeige dich dem Priester und opfere die
Gabe, die Mose befohlen hat, zu einem Zeugnis über sie.

\bibverse{5} Da aber Jesus einging zu Kapernaum, trat ein Hauptmann zu
ihm, der bat ihn \bibverse{6} und sprach: Herr, mein Knecht liegt zu
Hause und ist gichtbrüchig und hat große Qual.

\bibverse{7} Jesus sprach zu ihm: Ich will kommen und ihn gesund machen.

\bibverse{8} Der Hauptmann antwortete und sprach: Herr, ich bin nicht
wert, dass du unter mein Dach gehest; sondern sprich nur ein Wort, so
wird mein Knecht gesund. \bibverse{9} Denn ich bin ein Mensch, der
Obrigkeit untertan, und habe unter mir Kriegsknechte; und wenn ich sage
zu einem: Gehe hin! so geht er; und zum anderen: Komm her! so kommt er;
und zu meinem Knecht: Tu das! so tut er's.

\bibverse{10} Da das Jesus hörte, verwunderte er sich und sprach zu
denen, die ihm nachfolgten: Wahrlich ich sage euch: Solchen Glauben habe
ich in Israel nicht gefunden! \footnote{\textbf{8:10} Mk 6,6; Lk 18,8}
\bibverse{11} Aber ich sage euch: Viele werden kommen vom Morgen und vom
Abend und mit Abraham und Isaak und Jakob im Himmelreich sitzen;
\footnote{\textbf{8:11} Lk 13,28-29} \bibverse{12} aber die Kinder des
Reichs werden ausgestoßen in die Finsternis hinaus; da wird sein Heulen
und Zähneklappen. \bibverse{13} Und Jesus sprach zu dem Hauptmann: Gehe
hin; dir geschehe, wie du geglaubt hast. Und sein Knecht ward gesund zu
derselben Stunde. \footnote{\textbf{8:13} Mt 9,29; Mt 15,28}

\bibverse{14} Und Jesus kam in des Petrus Haus und sah, dass seine
Schwiegermutter lag und hatte das Fieber. \footnote{\textbf{8:14} 1Kor
  9,5} \bibverse{15} Da griff er ihre Hand an, und das Fieber verließ
sie. Und sie stand auf und diente ihnen. \bibverse{16} Am Abend aber
brachten sie viele Besessene zu ihm; und er trieb die Geister aus mit
Worten und machte allerlei Kranke gesund, \bibverse{17} auf dass
erfüllet würde, was gesagt ist durch den Propheten Jesaja, der da
spricht: „Er hat unsere Schwachheiten auf sich genommen, und unsere
Seuchen hat er getragen.``

\bibverse{18} Und da Jesus viel Volks um sich sah, hieß er hinüber
jenseits des Meeres fahren.

\bibverse{19} Und es trat zu ihm ein Schriftgelehrter, der sprach zu
ihm: Meister, ich will dir folgen, wo du hin gehst.

\bibverse{20} Jesus sagt zu ihm: Die Füchse haben Gruben, und die Vögel
unter dem Himmel haben Nester; aber des Menschen Sohn hat nicht, da er
sein Haupt hin lege. \footnote{\textbf{8:20} 2Kor 8,9}

\bibverse{21} Und ein anderer unter seinen Jüngern sprach zu ihm: Herr,
erlaube mir, dass hingehe und zuvor meinen Vater begrabe. \footnote{\textbf{8:21}
  Mt 10,37}

\bibverse{22} Aber Jesus sprach zu ihm: Folge du mir und lass die Toten
ihre Toten begraben!

\bibverse{23} Und er trat in das Schiff, und seine Jünger folgten ihm.
\bibverse{24} Und siehe, da erhob sich ein großes Ungestüm im Meer, also
dass auch das Schifflein mit Wellen bedeckt ward; und er schlief.
\bibverse{25} Und die Jünger traten zu ihm und weckten ihn auf und
sprachen: Herr, hilf uns, wir verderben!

\bibverse{26} Da sagte er zu ihnen: Ihr Kleingläubigen, warum seid ihr
so furchtsam? Und stand auf und bedrohte den Wind und das Meer; da ward
es ganz stille. \footnote{\textbf{8:26} Ps 89,10; Apg 27,22; Apg 27,34}

\bibverse{27} Die Menschen aber verwunderten sich und sprachen: Was ist
das für ein Mann, dass ihm Wind und Meer gehorsam ist?

\bibverse{28} Und er kam jenseits des Meeres in die Gegend der
Gergesener. Da liefen ihm entgegen zwei Besessene, die kamen aus den
Totengräbern und waren sehr grimmig, also dass niemand diese Straße
wandeln konnte. \bibverse{29} Und siehe, sie schrien und sprachen: Ach
Jesu, du Sohn Gottes, was haben wir mit dir zu tun? Bist du hergekommen,
uns zu quälen, ehe denn es Zeit ist? \bibverse{30} Es war aber ferne von
ihnen eine große Herde Säue auf der Weide. \bibverse{31} Da baten ihn
die Teufel und sprachen: Willst du uns austreiben, so erlaube uns, in
die Herde Säue zu fahren.

\bibverse{32} Und er sprach: Fahret hin! Da fuhren sie aus und fuhren in
die Herde Säue. Und siehe, die ganze Herde Säue stürzte sich von dem
Abhang ins Meer und ersoffen im Wasser.

\bibverse{33} Und die Hirten flohen und gingen hin in die Stadt und
sagten das alles und wie es mit den Besessenen ergangen war.
\bibverse{34} Und siehe, da ging die ganze Stadt heraus Jesu entgegen.
Und da sie ihn sahen, baten sie ihn, dass er aus ihrer Gegend weichen
wollte. \# 9 \bibverse{1} Da trat er in das Schiff und fuhr wieder
herüber und kam in seine Stadt. \footnote{\textbf{9:1} Mt 4,13}
\bibverse{2} Und siehe, da brachten sie zu ihm einen Gichtbrüchigen, der
lag auf einem Bette. Da nun Jesus ihren Glauben sah, sprach er zu dem
Gichtbrüchigen: Sei getrost, mein Sohn; deine Sünden sind dir vergeben.
\footnote{\textbf{9:2} 2Mo 34,6-7; Ps 103,3}

\bibverse{3} Und siehe, etliche unter den Schriftgelehrten sprachen bei
sich selbst: Dieser lästert Gott. \footnote{\textbf{9:3} Mt 26,65}

\bibverse{4} Da aber Jesus ihre Gedanken sah, sprach er: Warum denkt ihr
so Arges in euren Herzen? \footnote{\textbf{9:4} Joh 2,25} \bibverse{5}
Welches ist leichter: zu sagen: Dir sind deine Sünden vergeben, oder zu
sagen: Stehe auf und wandle? \bibverse{6} Auf das ihr aber wisset, dass
des Menschen Sohn Macht habe, auf Erden die Sünden zu vergeben (sprach
er zu dem Gichtbrüchigen): Stehe auf, hebe dein Bett auf und gehe heim!
\footnote{\textbf{9:6} Joh 17,2}

\bibverse{7} Und er stand auf und ging heim. \bibverse{8} Da das Volk
das sah, verwunderte es sich und pries Gott, der solche Macht den
Menschen gegeben hat.

\bibverse{9} Und da Jesus von dannen ging, sah er einen Menschen am Zoll
sitzen, der hieß Matthäus und er sprach zu ihm: Folge mir! Und er stand
auf und folgte ihm. \bibverse{10} Und es begab sich, da er zu Tische saß
im Hause, siehe, da kamen viele Zöllner und Sünder und saßen zu Tische
mit Jesu und seinen Jüngern. \bibverse{11} Da das die Pharisäer sahen,
sprachen sie zu seinen Jüngern: Warum isset euer Meister mit den
Zöllnern und Sündern?

\bibverse{12} Da das Jesus hörte, sprach er zu ihnen: Die Starken
bedürfen des Arztes nicht, sondern die Kranken. \footnote{\textbf{9:12}
  Hes 34,16} \bibverse{13} Gehet aber hin und lernet, was das sei: „Ich
habe Wohlgefallen an Barmherzigkeit und nicht am Opfer.`` Ich bin
gekommen die Sünder zur Buße zu rufen, und nicht die Gerechten.
\footnote{\textbf{9:13} 1Sam 15,22; Mt 18,11}

\bibverse{14} Indes kamen die Jünger des Johannes zu ihm und sprachen:
Warum fasten wir und die Pharisäer so viel, und deine Jünger fasten
nicht? \footnote{\textbf{9:14} Lk 18,12}

\bibverse{15} Jesus sprach zu ihnen: Wie können die Hochzeitsleute Leid
tragen, solange der Bräutigam bei ihnen ist? Es wird aber die Zeit
kommen, dass der Bräutigam von ihnen genommen wird; alsdann werden sie
fasten. \footnote{\textbf{9:15} Joh 3,29} \bibverse{16} Niemand flickt
ein altes Kleid mit einem Lappen von neuem Tuch; denn der Lappen reißt
doch wieder vom Kleid, und der Riss wird ärger. \footnote{\textbf{9:16}
  Röm 7,6} \bibverse{17} Man fasst auch nicht Most in alte Schläuche;
sonst zerreißen die Schläuche und der Most wird verschüttet, und die
Schläuche kommen um. Sondern man fasst Most in neue Schläuche, so werden
sie beide miteinander erhalten.

\bibverse{18} Da er solches mit ihnen redete, siehe, da kam der Obersten
einer und fiel vor ihm nieder und sprach: Herr, meine Tochter ist jetzt
gestorben; aber komm und lege deine Hand auf sie, so wird sie lebendig.

\bibverse{19} Und Jesus stand auf und folgte ihm nach und seine Jünger.
\bibverse{20} Und siehe, ein Weib, das zwölf Jahre den Blutgang gehabt,
trat von hinten zu ihm und rührte seines Kleides Saum an. \bibverse{21}
Denn sie sprach bei sich selbst: Möchte ich nur sein Kleid anrühren, so
würde ich gesund.

\bibverse{22} Da wandte sich Jesus um und sah sie und sprach: Sei
getrost, meine Tochter; dein Glaube hat dir geholfen. Und das Weib ward
gesund zu derselben Stunde.

\bibverse{23} Und als er in des Obersten Haus kam und sah die Pfeifer
und das Getümmel des Volks, \bibverse{24} sprach er zu ihnen: Weichet!
denn das Mägdlein ist nicht tot, sondern es schläft. Und sie verlachten
ihn. \footnote{\textbf{9:24} Joh 11,11; Joh 11,14; Joh 11,25}

\bibverse{25} Als aber das Volk hinausgetrieben war, ging er hinein und
ergriff es bei der Hand; da stand das Mädglein auf. \bibverse{26} Und
dies Gerücht erscholl in dasselbe ganze Land.

\bibverse{27} Und da Jesus von da weiterging, folgten ihm zwei Blinde
nach, die schrien und sprachen: Ach, du Sohn Davids, erbarme dich unser!
\bibverse{28} Und da er heimkam, traten die Blinden zu ihm. Und Jesus
sprach zu ihnen: Glaubt ihr, dass ich euch solches tun kann? Da sprachen
sie zu ihm: Herr, ja. \footnote{\textbf{9:28} Apg 14,9}

\bibverse{29} Da rührte er ihre Augen an und sprach: Euch geschehe nach
eurem Glauben. \footnote{\textbf{9:29} Mt 8,13}

\bibverse{30} Und ihre Augen wurden geöffnet. Und Jesus bedrohte sie und
sprach: Sehet zu, dass es niemand erfahre! \footnote{\textbf{9:30} Mt
  8,4} \bibverse{31} Aber sie gingen aus und machten ihn ruchbar im
selben ganzen Lande.

\bibverse{32} Da nun diese waren hinausgekommen, siehe, da brachten sie
zu ihm einen Menschen, der war stumm und besessen. \bibverse{33} Und da
der Teufel war ausgetrieben, redete der Stumme. Und das Volk verwunderte
sich und sprach: Solches ist noch nie in Israel gesehen worden.

\bibverse{34} Aber die Pharisäer sprachen: Er treibt die Teufel aus
durch der Teufel Obersten.

\bibverse{35} Und Jesus ging umher in alle Städte und Märkte, lehrte in
ihren Schulen und predigte das Evangelium von dem Reich und heilte
allerlei Seuche und allerlei Krankheit im Volke. \bibverse{36} Und da er
das Volk sah, jammerte ihn desselben; denn sie waren verschmachtet und
zerstreut wie die Schafe, die keinen Hirten haben. \footnote{\textbf{9:36}
  Mk 6,34; Hes 34,5} \bibverse{37} Da sprach er zu seinen Jüngern: Die
Ernte ist groß, aber wenige sind der Arbeiter. \footnote{\textbf{9:37}
  Lk 10,2} \bibverse{38} Darum bittet den Herrn der Ernte, dass er
Arbeiter in seine Ernte sende. \# 10 \bibverse{1} Und er rief seine
zwölf Jünger zu sich und gab ihnen Macht über die unsauberen Geister,
dass sie die austrieben und heilten allerlei Seuche und allerlei
Krankheit. \bibverse{2} Die Namen aber der zwölf Apostel sind diese: der
erste Simon, genannt Petrus, und Andreas, sein Bruder; Jakobus, des
Zebedäus Sohn, und Johannes, sein Bruder; \footnote{\textbf{10:2} Mk
  3,16-19; Lk 6,14-16; Apg 1,13} \bibverse{3} Philippus und
Bartholomäus; Thomas und Matthäus, der Zöllner; Jakobus, des Alphäus
Sohn, Lebbäus, mit dem Zunamen Thaddäus; \bibverse{4} Simon von Kana und
Judas Ischariot, welcher ihn verriet.

\bibverse{5} Diese zwölf sandte Jesus, gebot ihnen und sprach: Gehet
nicht auf der Heiden Straße und ziehet nicht in der Samariter Städte,
\bibverse{6} sondern gehet hin zu den verlorenen Schafen aus dem Hause
Israel. \bibverse{7} Geht aber und predigt und sprecht: Das Himmelreich
ist nahe herbeigekommen. \footnote{\textbf{10:7} Mt 4,17; Lk 10,9}
\bibverse{8} Macht die Kranken gesund, reinigt die Aussätzigen, weckt
die Toten auf, treibt die Teufel aus. Umsonst habt ihr's empfangen,
umsonst gebt es auch. \footnote{\textbf{10:8} Mk 16,17; Apg 20,33}
\bibverse{9} Ihr sollt nicht Gold noch Silber noch Erz in euren Gürteln
haben, \bibverse{10} auch keine Tasche zur Weg-Fahrt, auch nicht zwei
Röcke, keine Schuhe, auch keinen Stecken. Denn ein Arbeiter ist seiner
Speise wert. \footnote{\textbf{10:10} Lk 10,4; 1Kor 9,14; 1Tim 5,18; 4Mo
  18,31} \bibverse{11} Wo ihr aber in eine Stadt oder einen Markt geht,
da erkundigt euch, ob jemand darin sei, der es wert ist; und bei
demselben bleibet, bis ihr von dannen zieht. \bibverse{12} Wenn ihr aber
in ein Haus geht, so grüßet es; \bibverse{13} und wenn es das Haus wert
ist, wird euer Friede auf sie kommen. Ist es aber nicht wert, so wird
sich euer Friede wieder zu euch wenden. \bibverse{14} Und wenn euch
jemand nicht annehmen wird noch eure Rede hören, so geht heraus von
demselben Hause oder der Stadt und schüttelt den Staub von euren Füßen.
\footnote{\textbf{10:14} Lk 10,10-12; Apg 13,51} \bibverse{15} Wahrlich
ich sage euch: Dem Lande der Sodomer und Gomorrer wird es erträglicher
gehen am Jüngsten Gericht denn solcher Stadt. \footnote{\textbf{10:15}
  1Mo 19,1-29}

\bibverse{16} Siehe, ich sende euch wie Schafe mitten unter die Wölfe;
darum seid klug wie die Schlangen und ohne Falsch wie die Tauben.
\footnote{\textbf{10:16} Röm 16,19; Eph 5,15} \bibverse{17} Hütet euch
aber vor den Menschen; denn sie werden euch überantworten vor ihre
Rathäuser und werden euch geißeln in ihren Schulen. \footnote{\textbf{10:17}
  Apg 5,40; 2Kor 11,24} \bibverse{18} Und man wird euch vor Fürsten und
Könige führen um meinetwillen, zum Zeugnis über sie und über die Heiden.
\footnote{\textbf{10:18} Apg 25,23; Apg 27,24} \bibverse{19} Wenn sie
euch nun überantworten werden, so sorget nicht, wie oder was ihr reden
sollt; denn es soll euch zu der Stunde gegeben werden, was ihr reden
sollt. \footnote{\textbf{10:19} Lk 12,11-12; Apg 4,8} \bibverse{20} Denn
ihr seid es nicht, die da reden, sondern eures Vaters Geist ist es, der
durch euch redet. \footnote{\textbf{10:20} 1Kor 2,4}

\bibverse{21} Es wird aber ein Bruder den anderen zum Tod überantworten
und der Vater den Sohn, und die Kinder werden sich empören wider die
Eltern und ihnen zum Tode helfen. \footnote{\textbf{10:21} Mi 7,6}
\bibverse{22} Und ihr müsset gehasst werden von jedermann um meines
Namens willen. Wer aber bis an das Ende beharrt, der wird selig.
\footnote{\textbf{10:22} Mt 24,9-13; 2Tim 2,12} \bibverse{23} Wenn sie
euch aber in einer Stadt verfolgen, so fliehet in eine andere. Wahrlich
ich sage euch: Ihr werdet mit den Städten Israels nicht zu Ende kommen,
bis des Menschen Sohn kommt. \footnote{\textbf{10:23} Mt 16,28; Apg 8,1}

\bibverse{24} Der Jünger ist nicht über seinen Meister noch der Knecht
über den Herrn. \footnote{\textbf{10:24} Lk 6,40; Joh 13,16; Joh 15,20}
\bibverse{25} Es ist dem Jünger genug, dass er sei wie sein Meister und
der Knecht wie sein Herr. Haben sie den Hausvater Beelzebub geheißen,
wie viel mehr werden sie seine Hausgenossen also heißen! \footnote{\textbf{10:25}
  Mt 12,24} \bibverse{26} So fürchtet euch denn nicht vor ihnen. Es ist
nichts verborgen, das nicht offenbar werde, und ist nichts heimlich, das
man nicht wissen werde. \footnote{\textbf{10:26} Mk 4,22; Lk 8,17}
\bibverse{27} Was ich euch sage in der Finsternis, das redet im Licht;
und was ihr hört in das Ohr, das predigt auf den Dächern. \bibverse{28}
Und fürchtet euch nicht vor denen, die den Leib töten, und die Seele
nicht können töten; fürchtet euch aber vielmehr vor dem, der Leib und
Seele verderben kann in die Hölle.

\bibverse{29} Kauft man nicht zwei Sperlinge um einen Pfennig? Dennoch
fällt deren keiner auf die Erde ohne euren Vater. \bibverse{30} Nun aber
sind auch eure Haare auf dem Haupt alle gezählt. \footnote{\textbf{10:30}
  Apg 27,34} \bibverse{31} So fürchtet euch denn nicht; ihr seid besser
als viele Sperlinge. \footnote{\textbf{10:31} Mt 6,26} \bibverse{32} Wer
nun mich bekennet vor den Menschen, den will ich bekennen vor meinem
himmlischen Vater. \footnote{\textbf{10:32} Offb 3,5} \bibverse{33} Wer
mich aber verleugnet vor den Menschen, den will ich auch verleugnen vor
meinem himmlischen Vater. \footnote{\textbf{10:33} Mk 8,38; Lk 9,26;
  2Tim 2,12}

\bibverse{34} Ihr sollt nicht wähnen, dass ich gekommen sei, Frieden zu
senden auf die Erde. Ich bin nicht gekommen, Frieden zu senden, sondern
das Schwert. \footnote{\textbf{10:34} Lk 12,51-53} \bibverse{35} Denn
ich bin gekommen, den Menschen zu erregen wider seinen Vater und die
Tochter wider ihre Mutter und die Schwiegertochter wider ihre
Schwiegermutter. \bibverse{36} Und des Menschen Feinde werden seine
eigenen Hausgenossen sein. \bibverse{37} Wer Vater oder Mutter mehr
liebt denn mich, der ist mein nicht wert; und wer Sohn oder Tochter mehr
liebt denn mich, der ist mein nicht wert. \footnote{\textbf{10:37} 5Mo
  13,7-12; 5Mo 33,9; Lk 14,26-27} \bibverse{38} Und wer nicht sein Kreuz
auf sich nimmt und folgt mir nach, der ist mein nicht wert. \footnote{\textbf{10:38}
  Mt 16,24-25} \bibverse{39} Wer sein Leben findet, der wird's
verlieren; und wer sein Leben verliert um meinetwillen, der wird's
finden. \footnote{\textbf{10:39} Lk 9,24; Joh 12,25}

\bibverse{40} Wer euch aufnimmt, der nimmt mich auf; und wer mich
aufnimmt, der nimmt den auf, der mich gesandt hat. \footnote{\textbf{10:40}
  Lk 9,48; Joh 13,20; Gal 4,14} \bibverse{41} Wer einen Propheten
aufnimmt in eines Propheten Namen, der wird eines Propheten Lohn
empfangen. Wer einen Gerechten aufnimmt in eines Gerechten Namen, der
wird eines Gerechten Lohn empfangen. \footnote{\textbf{10:41} 1Kö
  17,9-24} \bibverse{42} Und wer dieser Geringsten einen nur mit einem
Becher kalten Wassers tränkt in eines Jüngers Namen, wahrlich ich sage
euch: es wird ihm nicht unbelohnt bleiben. \footnote{\textbf{10:42} Mt
  25,40; Mk 9,41}

\hypertarget{section-7}{%
\section{11}\label{section-7}}

\bibverse{1} Und es begab sich, da Jesus solch Gebot an seine zwölf
Jünger vollendet hatte, ging er von da weiter, zu lehren und zu predigen
in ihren Städten.

\bibverse{2} Da aber Johannes im Gefängnis die Werke Christi hörte,
sandte er seiner Jünger zwei \footnote{\textbf{11:2} Mt 14,3}
\bibverse{3} und ließ ihm sagen: Bist du, der da kommen soll, oder
sollen wir eines anderen warten? \footnote{\textbf{11:3} Mt 3,11; Mal
  3,1}

\bibverse{4} Jesus antwortete und sprach zu ihnen: Gehet hin und saget
Johannes wieder, was ihr sehet und höret: \bibverse{5} die Blinden sehen
und die Lahmen gehen, die Aussätzigen werden rein und die Tauben hören,
die Toten stehen auf und den Armen wird das Evangelium gepredigt;
\footnote{\textbf{11:5} Jes 35,5-6; Jes 61,1} \bibverse{6} und selig
ist, der sich nicht an mir ärgert. \footnote{\textbf{11:6} Mt 13,57; Mt
  26,31}

\bibverse{7} Da die hingingen, fing Jesus an, zu reden zu dem Volk von
Johannes: Was seid ihr hinausgegangen in die Wüste zu sehen? Wolltet ihr
ein Rohr sehen, das der Wind hin und her bewegt? \footnote{\textbf{11:7}
  Mt 3,1; Mt 3,5} \bibverse{8} Oder was seid ihr hinausgegangen zu
sehen? Wolltet ihr einen Menschen in weichen Kleidern sehen? Siehe, die
da weiche Kleider tragen, sind in der Könige Häusern. \bibverse{9} Oder
was seid ihr hinausgegangen zu sehen? Wolltet ihr einen Propheten sehen?
Ja, ich sage euch, der auch mehr ist denn ein Prophet. \bibverse{10}
Denn dieser ist's, von dem geschrieben steht: „Siehe, ich sende meinen
Engel vor dir her, der deinen Weg vor dir bereiten soll.`` \bibverse{11}
Wahrlich ich sage euch: Unter allen, die von Weibern geboren sind, ist
nicht aufgekommen, der größer sei denn Johannes der Täufer; der aber der
Kleinste ist im Himmelreich, ist größer denn er. \footnote{\textbf{11:11}
  Mt 13,17} \bibverse{12} Aber von den Tagen Johannes des Täufers bis
hierher leidet das Himmelreich Gewalt, und die Gewalt tun, die reißen es
an sich. \footnote{\textbf{11:12} Lk 16,16} \bibverse{13} Denn alle
Propheten und das Gesetz haben geweissagt bis auf Johannes.
\bibverse{14} Und (so ihr's wollt annehmen) er ist Elia, der da soll
zukünftig sein. \footnote{\textbf{11:14} Mal 3,23; Mt 17,10-13}
\bibverse{15} Wer Ohren hat, zu hören, der höre!

\bibverse{16} Wem soll ich aber dies Geschlecht vergleichen? Es ist den
Kindlein gleich, die an dem Markt sitzen und rufen gegen ihre Gesellen
\bibverse{17} und sprechen: Wir haben euch gepfiffen, und ihr wolltet
nicht tanzen; wir haben euch geklagt, und ihr wolltet nicht weinen.
\bibverse{18} Johannes ist gekommen, aß nicht und trank nicht; so sagen
sie: Er hat den Teufel. \footnote{\textbf{11:18} Mt 3,4; Joh 10,20}
\bibverse{19} Des Menschen Sohn ist gekommen, isst und trinkt; so sagen
sie: Siehe, wie ist der Mensch ein Fresser und ein Weinsäufer, der
Zöllner und der Sünder Geselle! Und die Weisheit muss sich rechtfertigen
lassen von ihren Kindern. \footnote{\textbf{11:19} Mt 9,10-15; Joh 2,2;
  1Kor 1,24-30}

\bibverse{20} Da fing er an, die Städte zu schelten, in welchen am
meisten seiner Taten geschehen waren, und hatten sich doch nicht
gebessert: \bibverse{21} Wehe dir Chorazin! Weh dir, Bethsaida! Wären
solche Taten zu Tyrus und Sidon geschehen, wie bei euch geschehen sind,
sie hätten vorzeiten im Sack und in der Asche Buße getan. \footnote{\textbf{11:21}
  Jon 3,6} \bibverse{22} Doch ich sage euch: Es wird Tyrus und Sidon
erträglicher gehen am Jüngsten Gericht als euch. \bibverse{23} Und du,
Kapernaum, die du bist erhoben bis an den Himmel, du wirst bis in die
Hölle hinuntergestoßen werden. Denn so zu Sodom die Taten geschehen
wären, die bei dir geschehen sind, sie stünde noch heutigestages.
\bibverse{24} Doch ich sage euch: Es wird dem Sodomer Lande erträglicher
gehen am Jüngsten Gericht als dir. \footnote{\textbf{11:24} Mt 10,15}

\bibverse{25} Zu der Zeit antwortete Jesus und sprach: Ich preise dich,
Vater und Herr Himmels und der Erde, dass du solches den Weisen und
Klugen verborgen hast und hast es den Unmündigen offenbart. \footnote{\textbf{11:25}
  1Kor 1,19-29; Jes 29,14; Lk 10,21-22; Joh 17,25} \bibverse{26} Ja,
Vater; denn es ist also wohlgefällig gewesen vor dir. \bibverse{27} Alle
Dinge sind mir übergeben von meinem Vater. Und niemand kennet den Sohn
denn nur der Vater; und niemand kennet den Vater denn nur der Sohn und
wem es der Sohn will offenbaren. \footnote{\textbf{11:27} Mt 28,18; Joh
  3,35; Joh 17,2; Phil 2,9}

\bibverse{28} Kommet her zu mir alle, die ihr mühselig und beladen seid;
ich will euch erquicken. \footnote{\textbf{11:28} Mt 12,20; Mt 23,4; Jer
  31,25} \bibverse{29} Nehmet auf euch mein Joch und lernet von mir;
denn ich bin sanftmütig und von Herzen demütig; so werdet ihr Ruhe
finden für eure Seelen. \footnote{\textbf{11:29} Jes 28,12; Jer 6,16;
  Sach 9,9} \bibverse{30} Denn mein Joch ist sanft, und meine Last ist
leicht. \footnote{\textbf{11:30} Lk 11,46; 1Jo 5,3}

\hypertarget{section-8}{%
\section{12}\label{section-8}}

\bibverse{1} Zu der Zeit ging Jesus durch die Saat am Sabbat; und seine
Jünger waren hungrig, fingen an, Ähren auszuraufen, und aßen.
\footnote{\textbf{12:1} 5Mo 23,26} \bibverse{2} Da das die Pharisäer
sahen, sprachen sie zu ihm: Siehe, deine Jünger tun, was sich nicht
ziemt am Sabbat zu tun. \footnote{\textbf{12:2} 2Mo 20,10}

\bibverse{3} Er aber sprach zu ihnen: Habt ihr nicht gelesen, was David
tat, da ihn und die mit ihm waren, hungerte? \footnote{\textbf{12:3}
  1Sam 21,7} \bibverse{4} wie er in das Gotteshaus ging und aß die
Schaubrote, die ihm doch nicht ziemte zu essen noch denen, die mit ihm
waren, sondern allein den Priestern? \footnote{\textbf{12:4} 3Mo 24,9}
\bibverse{5} Oder habt ihr nicht gelesen im Gesetz, wie die Priester am
Sabbat im Tempel den Sabbat brechen und sind doch ohne Schuld?
\footnote{\textbf{12:5} 4Mo 28,9} \bibverse{6} Ich sage aber euch, dass
hier der ist, der auch größer ist denn der Tempel. \bibverse{7} Wenn ihr
aber wüsstet, was das sei: „Ich habe Wohlgefallen an der Barmherzigkeit
und nicht am Opfer``, -- hättet ihr die Unschuldigen nicht verdammt.
\bibverse{8} Des Menschen Sohn ist ein Herr auch über den Sabbat.

\bibverse{9} Und er ging von da weiter und kam in ihre Schule.
\bibverse{10} Und siehe, da war ein Mensch, der hatte eine verdorrte
Hand. Und sie fragten ihn und sprachen: Ist's auch recht, am Sabbat
heilen? auf dass sie eine Sache wider ihn hätten.

\bibverse{11} Aber er sprach zu ihnen: Wer ist unter euch, wenn er ein
Schaf hat, das ihm am Sabbat in eine Grube fällt, der es nicht ergreife
und aufhebe? \footnote{\textbf{12:11} Lk 14,3-5} \bibverse{12} Wie viel
besser ist nun ein Mensch denn ein Schaf! Darum mag man wohl am Sabbat
Gutes tun. \bibverse{13} Da sprach er zu dem Menschen: Strecke deine
Hand aus! Und er streckte sie aus; und sie ward ihm wieder gesund
gleichwie die andere. \bibverse{14} Da gingen die Pharisäer hinaus und
hielten einen Rat über ihn, wie sie ihn umbrächten.

\bibverse{15} Aber da Jesus das erfuhr, wich er von dannen. Und ihm
folgte viel Volks nach, und er heilte sie alle \bibverse{16} und
bedrohte sie, dass sie ihn nicht meldeten, \footnote{\textbf{12:16} Mt
  8,4} \bibverse{17} auf dass erfüllet würde, was gesagt ist durch den
Propheten Jesaja, der da spricht: \bibverse{18} „Siehe, das ist mein
Knecht, den ich erwählt habe, und mein Liebster, an dem meine Seele
Wohlgefallen hat; ich will meinen Geist auf ihn legen, und er soll den
Heiden das Gericht verkünden. \bibverse{19} Er wird nicht zanken noch
schreien, und man wird sein Geschrei nicht hören auf den Gassen;
\bibverse{20} das zerstoßene Rohr wird er nicht zerbrechen, und den
glimmenden Docht wird er nicht auslöschen, bis dass er ausführe das
Gericht zum Sieg; \bibverse{21} und die Heiden werden auf seinen Namen
hoffen.``

\bibverse{22} Da ward ein Besessener zu ihm gebracht, der war blind und
stumm; und er heilte ihn, also dass der Blinde und Stumme redete und
sah. \bibverse{23} Und alles Volk entsetzte sich und sprach: Ist dieser
nicht Davids Sohn? \bibverse{24} Aber die Pharisäer, da sie es hörten,
sprachen sie: Er treibt die Teufel nicht anders aus denn durch
Beelzebub, der Teufel Obersten. \footnote{\textbf{12:24} Mt 9,34}

\bibverse{25} Jesus kannte aber ihre Gedanken und sprach zu ihnen: Ein
jegliches Reich, so es mit sich selbst uneins wird, das wird wüst; und
eine jegliche Stadt oder Haus, so es mit sich selbst uneins wird, kann's
nicht bestehen. \bibverse{26} So denn ein Satan den anderen austreibt,
so muss er mit sich selbst uneins sein; wie kann denn sein Reich
bestehen? \bibverse{27} So ich aber die Teufel durch Beelzebub
austreibe, durch wen treiben sie eure Kinder aus? Darum werden sie eure
Richter sein. \bibverse{28} So ich aber die Teufel durch den Geist
Gottes austreibe, so ist ja das Reich Gottes zu euch gekommen.
\bibverse{29} Oder wie kann jemand in eines Starken Haus gehen und ihm
seinen Hausrat rauben, es sei denn, dass er zuvor den Starken binde und
alsdann ihm sein Haus beraube? \footnote{\textbf{12:29} Mt 4,1-11; Jes
  49,24}

\bibverse{30} Wer nicht mit mir ist, der ist wider mich; und wer nicht
mit mir sammelt, der zerstreut. \footnote{\textbf{12:30} Mk 9,40}
\bibverse{31} Darum sage ich euch: Alle Sünde und Lästerung wird den
Menschen vergeben; aber die Lästerung wider den Geist wird den Menschen
nicht vergeben. \footnote{\textbf{12:31} Hebr 6,4-6; Hebr 10,26; 1Jo
  5,16} \bibverse{32} Und wer etwas redet wider des Menschen Sohn, dem
wird es vergeben; aber wer etwas redet wider den Heiligen Geist, dem
wird's nicht vergeben, weder in dieser noch in jener Welt. \footnote{\textbf{12:32}
  1Tim 1,13}

\bibverse{33} Setzt entweder einen guten Baum, so wird die Frucht gut;
oder setzt einen faulen Baum, so wird die Frucht faul. Denn an der
Frucht erkennt man den Baum. \footnote{\textbf{12:33} Mt 7,17}
\bibverse{34} Ihr Otterngezüchte, wie könnt ihr Gutes reden, dieweil ihr
böse seid? Wes das Herz voll ist, des geht der Mund über. \footnote{\textbf{12:34}
  Mt 3,7} \bibverse{35} Ein guter Mensch bringt Gutes hervor aus seinem
guten Schatz des Herzens; und ein böser Mensch bringt Böses hervor aus
seinem bösen Schatz. \bibverse{36} Ich sage euch aber, dass die Menschen
müssen Rechenschaft geben am Jüngsten Gericht von einem jeglichen
unnützen Wort, das sie geredet haben. \footnote{\textbf{12:36} Jak 3,6;
  Jud 1,15} \bibverse{37} Aus deinen Worten wirst du gerechtfertigt
werden, und aus deinen Worten wirst du verdammt werden.

\bibverse{38} Da antworteten etliche unter den Schriftgelehrten und
Pharisäern und sprachen: Meister, wir wollten gern ein Zeichen von dir
sehen.

\bibverse{39} Und er antwortete und sprach zu ihnen: Die böse und
ehebrecherische Art sucht ein Zeichen; und es wird ihr kein Zeichen
gegeben werden denn das Zeichen des Propheten Jona. \bibverse{40} Denn
gleichwie Jona war drei Tage und drei Nächte in des Walfisches Bauch,
also wird des Menschen Sohn drei Tage und drei Nächte mitten in der Erde
sein. \bibverse{41} Die Leute von Ninive werden auftreten am Jüngsten
Gericht mit diesem Geschlecht und werden es verdammen; denn sie taten
Buße nach der Predigt des Jona. Und siehe, hier ist mehr als Jona.
\footnote{\textbf{12:41} Jon 3,5} \bibverse{42} Die Königin von Mittag
wird auftreten am Jüngsten Gericht mit diesem Geschlecht und wird es
verdammen; denn sie kam vom Ende der Erde, Salomos Weisheit zu hören.
Und siehe, hier ist mehr als Salomo. \footnote{\textbf{12:42} 1Kö
  10,1-10}

\bibverse{43} Wenn der unsaubere Geist von dem Menschen ausgefahren ist,
so durchwandelt er dürre Stätten, sucht Ruhe, und findet sie nicht.
\bibverse{44} Da spricht er denn: Ich will wieder umkehren in mein Haus,
daraus ich gegangen bin. Und wenn er kommt, so findet er's leer, gekehrt
und geschmückt. \bibverse{45} So geht er hin und nimmt zu sich sieben
andere Geister, die ärger sind denn er selbst; und wenn sie
hineinkommen, wohnen sie allda; und es wird mit demselben Menschen
hernach ärger, denn es zuvor war. Also wird's auch diesem argen
Geschlecht gehen. \footnote{\textbf{12:45} 2Petr 2,20}

\bibverse{46} Da er noch also zu dem Volk redete, siehe, da standen
seine Mutter und seine Brüder draußen, die wollten mit ihm reden.
\footnote{\textbf{12:46} Mt 13,55} \bibverse{47} Da sprach einer zu ihm:
Siehe, deine Mutter und deine Brüder stehen draußen und wollen mit dir
reden.

\bibverse{48} Er antwortete aber und sprach zu dem, der es ihm ansagte:
Wer ist meine Mutter, und wer sind meine Brüder? \footnote{\textbf{12:48}
  Mt 10,37; Lk 2,49} \bibverse{49} Und er reckte die Hand aus über seine
Jünger und sprach: Siehe da, das ist meine Mutter und meine Brüder!
\footnote{\textbf{12:49} Hebr 2,11} \bibverse{50} Denn wer den Willen
tut meines Vaters im Himmel, der ist mein Bruder, Schwester und Mutter.
\footnote{\textbf{12:50} Röm 8,29}

\hypertarget{section-9}{%
\section{13}\label{section-9}}

\bibverse{1} An demselben Tage ging Jesus aus dem Hause und setzte sich
an das Meer. \bibverse{2} Und es versammelte sich viel Volks zu ihm,
also dass er in das Schiff trat und saß, und alles Volk stand am Ufer.
\bibverse{3} Und er redete zu ihnen mancherlei durch Gleichnisse und
sprach: Siehe, es ging ein Sämann aus, zu säen. \bibverse{4} Und indem
er säte, fiel etliches an den Weg; da kamen die Vögel und fraßen's auf.
\bibverse{5} Etliches fiel in das Steinige, wo es nicht viel Erde hatte;
und ging bald auf, darum dass es nicht tiefe Erde hatte. \bibverse{6}
Als aber die Sonne aufging, verwelkte es, und dieweil es nicht Wurzel
hatte, ward es dürre. \bibverse{7} Etliches fiel unter die Dornen; und
die Dornen wuchsen auf und erstickten's. \bibverse{8} Etliches fiel auf
ein gutes Land und trug Frucht, etliches hundertfältig, etliches
sechzigfältig, etliches dreißigfältig. \bibverse{9} Wer Ohren hat zu
hören, der höre!

\bibverse{10} Und die Jünger traten zu ihm und sprachen: Warum redest du
zu ihnen durch Gleichnisse?

\bibverse{11} Er antwortete und sprach: Euch ist's gegeben, dass ihr das
Geheimnis des Himmelreichs verstehet; diesen aber ist's nicht gegeben.
\bibverse{12} Denn wer da hat, dem wird gegeben, dass er die Fülle habe;
wer aber nicht hat, von dem wird auch genommen, was er hat. \footnote{\textbf{13:12}
  Mt 25,28-29; Mk 4,25; Lk 8,18} \bibverse{13} Darum rede ich zu ihnen
durch Gleichnisse. Denn mit sehenden Augen sehen sie nicht, und mit
hörenden Ohren hören sie nicht; denn sie verstehen es nicht. \footnote{\textbf{13:13}
  5Mo 29,3; Joh 16,25} \bibverse{14} Und über ihnen wird die Weissagung
Jesaja's erfüllt, die da sagt: „Mit den Ohren werdet ihr hören, und
werdet es nicht verstehen; und mit sehenden Augen werdet ihr sehen, und
werdet es nicht vernehmen. \bibverse{15} Denn dieses Volkes Herz ist
verstockt, und ihre Ohren hören übel, und ihre Augen schlummern, auf
dass sie nicht dermaleinst mit den Augen sehen und mit den Ohren hören
und mit dem Herzen verstehen und sich bekehren, dass ich ihnen hülfe.``
\footnote{\textbf{13:15} Joh 9,39}

\bibverse{16} Aber selig sind eure Augen, dass sie sehen, und eure
Ohren, dass sie hören. \footnote{\textbf{13:16} Lk 10,23-24}
\bibverse{17} Wahrlich ich sage euch: Viele Propheten und Gerechte haben
begehrt zu sehen, was ihr sehet, und haben's nicht gesehen, und zu
hören, was ihr höret, und haben's nicht gehört. \footnote{\textbf{13:17}
  1Petr 1,10}

\bibverse{18} So höret nun ihr dieses Gleichnis von dem Sämann:
\bibverse{19} Wenn jemand das Wort von dem Reich hört und nicht
versteht, so kommt der Arge und reißt hinweg, was da gesät ist in sein
Herz; und das ist der, bei welchem an dem Wege gesät ist. \bibverse{20}
Das aber auf das Steinige gesät ist, das ist, wenn jemand das Wort hört
und es alsbald aufnimmt mit Freuden; \bibverse{21} aber er hat nicht
Wurzel in sich, sondern er ist wetterwendisch; wenn sich Trübsal und
Verfolgung erhebt um des Wortes willen, so ärgert er sich alsbald.
\bibverse{22} Das aber unter die Dornen gesät ist, das ist, wenn jemand
das Wort hört, und die Sorge dieser Welt und der Betrug des Reichtums
erstickt das Wort, und er bringt nicht Frucht. \bibverse{23} Das aber in
das gute Land gesät ist, das ist, wenn jemand das Wort hört und versteht
es und dann auch Frucht bringt; und etlicher trägt hundertfältig,
etlicher aber sechzigfältig, etlicher dreißigfältig.

\bibverse{24} Er legte ihnen ein anderes Gleichnis vor und sprach: Das
Himmelreich ist gleich einem Menschen, der guten Samen auf seinen Acker
säte. \bibverse{25} Da aber die Leute schliefen, kam sein Feind und säte
Unkraut zwischen den Weizen und ging davon. \bibverse{26} Da nun das
Kraut wuchs und Frucht brachte, da fand sich auch das Unkraut.
\bibverse{27} Da traten die Knechte zu dem Hausvater und sprachen: Herr,
hast du nicht guten Samen auf deinen Acker gesät? Woher hat er denn das
Unkraut?

\bibverse{28} Er sprach zu ihnen: Das hat der Feind getan. Da sprachen
die Knechte: Willst du denn, dass wir hingehen und es ausjäten?

\bibverse{29} Er sprach: Nein! auf dass ihr nicht zugleich den Weizen
mit ausraufet, so ihr das Unkraut ausjätet.

\bibverse{30} Lasset beides miteinander wachsen bis zu der Ernte; und um
der Ernte Zeit will ich zu den Schnittern sagen: Sammelt zuvor das
Unkraut und bindet es in Bündlein, dass man es verbrenne; aber den
Weizen sammelt mir in meine Scheuer. \footnote{\textbf{13:30} Mt 3,12;
  Mt 15,13; Offb 14,15}

\bibverse{31} Ein anderes Gleichnis legte er ihnen vor und sprach: Das
Himmelreich ist gleich einem Senfkorn, das ein Mensch nahm und säte es
auf seinen Acker; \bibverse{32} welches das kleinste ist unter allem
Samen; wenn es aber erwächst, so ist es das größte unter dem Kohl und
wird ein Baum, dass die Vögel unter dem Himmel kommen und wohnen unter
seinen Zweigen.

\bibverse{33} Ein anderes Gleichnis redete er zu ihnen: Das Himmelreich
ist einem Sauerteig gleich, den ein Weib nahm und vermengte ihn unter
drei Scheffel Mehl, bis dass es ganz durchsäuert ward. \footnote{\textbf{13:33}
  Lk 13,20-21}

\bibverse{34} Solches alles redete Jesus durch Gleichnisse zu dem Volk,
und ohne Gleichnis redete er nicht zu ihnen, \footnote{\textbf{13:34} Mk
  4,33-34} \bibverse{35} auf dass erfüllet würde, was gesagt ist durch
den Propheten, der da spricht: Ich will meinen Mund auftun in
Gleichnissen und will aussprechen die Heimlichkeiten von Anfang der
Welt.

\bibverse{36} Da ließ Jesus das Volk von sich und kam heim. Und seine
Jünger traten zu ihm und sprachen: Deute uns das Gleichnis vom Unkraut
auf dem Acker.

\bibverse{37} Er antwortete und sprach zu ihnen: Des Menschen Sohn
ist's, der da guten Samen sät. \bibverse{38} Der Acker ist die Welt. Der
gute Same sind die Kinder des Reichs. Das Unkraut sind die Kinder der
Bosheit. \footnote{\textbf{13:38} Joh 8,44; 1Kor 3,9} \bibverse{39} Der
Feind, der sie sät, ist der Teufel. Die Ernte ist das Ende der Welt. Die
Schnitter sind die Engel. \bibverse{40} Gleichwie man nun das Unkraut
ausjätet und mit Feuer verbrennt, so wird's auch am Ende dieser Welt
gehen: \bibverse{41} des Menschen Sohn wird seine Engel senden; und sie
werden sammeln aus seinem Reich alle Ärgernisse und die da Unrecht tun,
\bibverse{42} und werden sie in den Feuerofen werfen; da wird sein
Heulen und Zähneklappen. \bibverse{43} Dann werden die Gerechten
leuchten wie die Sonne in ihres Vaters Reich. Wer Ohren hat zu hören,
der höre! \footnote{\textbf{13:43} Dan 12,3}

\bibverse{44} Abermals ist gleich das Himmelreich einem verborgenen
Schatz im Acker, welchen ein Mensch fand und verbarg ihn und ging hin
vor Freuden über denselben und verkaufte alles, was er hatte, und kaufte
den Acker. \footnote{\textbf{13:44} Mt 19,29; Lk 14,33; Phil 3,7}

\bibverse{45} Abermals ist gleich das Himmelreich einem Kaufmann, der
gute Perlen suchte. \bibverse{46} Und da er eine köstliche Perle fand,
ging er hin und verkaufte alles, was er hatte, und kaufte sie.

\bibverse{47} Abermals ist gleich das Himmelreich einem Netze, das ins
Meer geworfen ist, womit man allerlei Gattung fängt. \footnote{\textbf{13:47}
  Mt 22,9-10} \bibverse{48} Wenn es aber voll ist, so ziehen sie es
heraus an das Ufer, sitzen und lesen die guten in ein Gefäß zusammen;
aber die faulen werfen sie weg. \bibverse{49} Also wird es auch am Ende
der Welt gehen: die Engel werden ausgehen und die Bösen von den
Gerechten scheiden \bibverse{50} und werden sie in den Feuerofen werfen;
da wird Heulen und Zähneklappen sein. \bibverse{51} Und Jesus sprach zu
ihnen: Habt ihr das alles verstanden? Sie sprachen: Ja, Herr.

\bibverse{52} Da sprach er: Darum ein jeglicher Schriftgelehrter, zum
Himmelreich gelehrt, ist gleich einem Hausvater, der aus seinem Schatz
Neues und Altes hervorträgt.

\bibverse{53} Und es begab sich, da Jesus diese Gleichnisse vollendet
hatte, ging er von dannen

\bibverse{54} und kam in seine Vaterstadt und lehrte sie in ihrer
Schule, also auch, dass sie sich entsetzten und sprachen: Woher kommt
diesem solche Weisheit und Taten? \bibverse{55} Ist er nicht eines
Zimmermanns Sohn? Heißt nicht seine Mutter Maria? und seine Brüder Jakob
und Joses und Simon und Judas? \bibverse{56} Und seine Schwestern, sind
sie nicht alle bei uns? Woher kommt ihm denn das alles? \footnote{\textbf{13:56}
  Joh 6,42; Joh 7,15; Joh 7,52} \bibverse{57} Und sie ärgerten sich an
ihm. Jesus aber sprach zu ihnen: Ein Prophet gilt nirgend weniger denn
in seinem Vaterland und in seinem Hause. \footnote{\textbf{13:57} Joh
  4,44}

\bibverse{58} Und er tat daselbst nicht viel Zeichen um ihres Unglaubens
willen. \# 14 \bibverse{1} Zu der Zeit kam das Gerücht von Jesu vor den
Vierfürsten Herodes. \bibverse{2} Und er sprach zu seinen Knechten:
Dieser ist Johannes der Täufer; er ist von den Toten auferstanden, darum
tut er solche Taten. \bibverse{3} Denn Herodes hatte Johannes gegriffen,
gebunden und in das Gefängnis gelegt wegen der Herodias, seines Bruders
Philippus Weib. \footnote{\textbf{14:3} Mt 11,2} \bibverse{4} Denn
Johannes hatte zu ihm gesagt: Es ist nicht recht, dass du sie habest.
\footnote{\textbf{14:4} Mt 19,9; 3Mo 18,16} \bibverse{5} Und er hätte
ihn gern getötet, fürchtete sich aber vor dem Volk; denn sie hielten ihn
für einen Propheten. \footnote{\textbf{14:5} Mt 21,26} \bibverse{6} Da
aber Herodes seinen Jahrestag beging, da tanzte die Tochter der Herodias
vor ihnen. Das gefiel Herodes wohl. \bibverse{7} Darum verhieß er ihr
mit einem Eide, er wollte ihr geben, was sie fordern würde. \bibverse{8}
Und wie sie zuvor von ihrer Mutter angestiftet war, sprach sie: Gib mir
her auf einer Schüssel das Haupt Johannes des Täufers!

\bibverse{9} Und der König ward traurig; doch um des Eides willen und
derer, die mit ihm zu Tisch saßen, befahl er's ihr zu geben.
\bibverse{10} Und schickte hin und enthauptete Johannes im Gefängnis.
\bibverse{11} Und sein Haupt ward hergetragen in einer Schüssel und dem
Mägdlein gegeben; und sie brachte es ihrer Mutter. \bibverse{12} Da
kamen seine Jünger und nahmen seinen Leib und begruben ihn; und kamen
und verkündigten das Jesus. \bibverse{13} Da das Jesus hörte, wich er
von dannen auf einem Schiff in eine Wüste allein. Und da das Volk das
hörte, folgte es ihm nach zu Fuß aus den Städten.

\bibverse{14} Und Jesus ging hervor und sah das große Volk; und es
jammerte ihn derselben, und er heilte ihre Kranken. \bibverse{15} Am
Abend aber traten seine Jünger zu ihm und sprachen: Dies ist eine Wüste,
und die Nacht fällt herein; lass das Volk von dir, dass sie hin in die
Märkte gehen und sich Speise kaufen.

\bibverse{16} Aber Jesus sprach zu ihnen: Es ist nicht not, dass sie
hingehen; gebt ihr ihnen zu essen.

\bibverse{17} Sie sprachen: Wir haben hier nichts denn fünf Brote und
zwei Fische.

\bibverse{18} Und er sprach: Bringet sie mir her. \bibverse{19} Und er
hieß das Volk sich lagern auf das Gras und nahm die fünf Brote und die
zwei Fische, sah auf gen Himmel und dankte und brach's und gab die Brote
den Jüngern, und die Jünger gaben sie dem Volk. \bibverse{20} Und sie
aßen alle und wurden satt und hoben auf, was übrigblieb von Brocken,
zwölf Körbe voll. \bibverse{21} Die aber gegessen hatten waren, waren
bei fünftausend Mann, ohne Weiber und Kinder.

\bibverse{22} Und alsbald trieb Jesus seine Jünger, dass sie in das
Schiff traten und vor ihm herüberfuhren, bis er das Volk von sich ließe.
\bibverse{23} Und da er das Volk von sich gelassen hatte, stieg er auf
einen Berg allein, dass er betete. Und am Abend war er allein daselbst.
\footnote{\textbf{14:23} Lk 6,12; Lk 9,18} \bibverse{24} Und das Schiff
war schon mitten auf dem Meer und litt Not von den Wellen; denn der Wind
war ihnen zuwider. \bibverse{25} Aber in der vierten Nachtwache kam
Jesus zu ihnen und ging auf dem Meer. \bibverse{26} Und da ihn die
Jünger sahen auf dem Meer gehen, erschraken sie und sprachen: Es ist ein
Gespenst! und schrien vor Furcht. \bibverse{27} Aber alsbald redete
Jesus mit ihnen und sprach: Seid getrost, ich bin's; fürchtet euch
nicht!

\bibverse{28} Petrus aber antwortete ihm und sprach: Herr, bist du es,
so heiß mich zu dir kommen auf dem Wasser.

\bibverse{29} Und er sprach: Komm her! Und Petrus trat aus dem Schiff
und ging auf dem Wasser, dass er zu Jesu käme.

\bibverse{30} Er sah aber einen starken Wind; da erschrak er und hob an
zu sinken, schrie und sprach: Herr, hilf mir!

\bibverse{31} Jesus aber reckte alsbald die Hand aus und ergriff ihn und
sprach zu ihm: O du Kleingläubiger, warum zweifeltest du? \bibverse{32}
Und sie traten in das Schiff, und der Wind legte sich. \bibverse{33} Die
aber im Schiff waren, kamen und fielen vor ihm nieder und sprachen: Du
bist wahrlich Gottes Sohn! \footnote{\textbf{14:33} Mt 16,16; Joh 1,49}

\bibverse{34} Und sie schifften hinüber und kamen in das Land
Genezareth. \bibverse{35} Und da die Leute am selbigen Ort sein gewahr
wurden, schickten sie aus in das ganze Land umher und brachten allerlei
Ungesunde zu ihm \bibverse{36} und baten ihn, dass sie nur seines
Kleides Saum anrührten. Und alle, die da anrührten, wurden gesund. \# 15
\bibverse{1} Da kamen zu ihm die Schriftgelehrten und Pharisäer von
Jerusalem und sprachen: \bibverse{2} Warum übertreten deine Jünger der
Ältesten Aufsätze? Sie waschen ihre Hände nicht, wenn sie Brot essen.
\footnote{\textbf{15:2} Lk 11,38}

\bibverse{3} Er antwortete und sprach zu ihnen: Warum übertretet denn
ihr Gottes Gebot um eurer Aufsätze willen? \bibverse{4} Gott hat
geboten: „Du sollst Vater und Mutter ehren; wer aber Vater und Mutter
flucht, der soll des Todes sterben.`` \bibverse{5} Aber ihr lehret: Wer
zum Vater oder zur Mutter spricht: „Es ist Gott gegeben, was dir sollte
von mir zu Nutz kommen``, -- der tut wohl. \bibverse{6} Damit geschieht
es, dass niemand hinfort seinen Vater oder seine Mutter ehrt, und also
habt ihr Gottes Gebot aufgehoben um eurer Aufsätze willen. \footnote{\textbf{15:6}
  1Tim 5,8} \bibverse{7} Ihr Heuchler, wohl fein hat Jesaja von euch
geweissagt und gesprochen: \bibverse{8} „Dieses Volk naht sich zu mir
mit seinem Munde und ehrt mich mit seinen Lippen, aber ihr Herz ist fern
von mir; \bibverse{9} aber vergeblich dienen sie mir, dieweil sie lehren
solche Lehren, die nichts denn Menschengebote sind.``

\bibverse{10} Und er rief das Volk zu sich und sprach zu ihm: Höret zu
und fasset es! \bibverse{11} Was zum Munde eingeht, das verunreinigt den
Menschen nicht; sondern was zum Munde ausgeht, das verunreinigt den
Menschen.

\bibverse{12} Da traten seine Jünger zu ihm und sprachen: Weißt du auch,
dass sich die Pharisäer ärgerten, da sie das Wort hörten?

\bibverse{13} Aber er antwortete und sprach: Alle Pflanzen, die mein
himmlischer Vater nicht pflanzte, die werden ausgereutet. \footnote{\textbf{15:13}
  Apg 5,38} \bibverse{14} Lasset sie fahren! Sie sind blinde
Blindenleiter. Wenn aber ein Blinder den anderen leitet, so fallen sie
beide in die Grube. \footnote{\textbf{15:14} Mt 23,24; Lk 6,39; Röm 2,19}

\bibverse{15} Da antwortete Petrus und sprach zu ihm: Deute uns dieses
Gleichnis.

\bibverse{16} Und Jesus sprach zu ihnen: Seid ihr denn auch noch
unverständig? \bibverse{17} Merket ihr noch nicht, dass alles, was zum
Munde eingeht, das geht in den Bauch und wird durch den natürlichen Gang
ausgeworfen? \bibverse{18} Was aber zum Munde herausgeht, das kommt aus
dem Herzen, und das verunreinigt den Menschen. \bibverse{19} Denn aus
dem Herzen kommen arge Gedanken: Mord, Ehebruch, Hurerei, Dieberei,
falsch Zeugnis, Lästerung. \footnote{\textbf{15:19} 1Mo 8,21}
\bibverse{20} Das sind die Stücke, die den Menschen verunreinigen. Aber
mit ungewaschenen Händen essen verunreinigt den Menschen nicht.

\bibverse{21} Und Jesus ging aus von dannen und entwich in die Gegend
von Tyrus und Sidon. \bibverse{22} Und siehe, ein kanaanäisches Weib kam
aus derselben Gegend und schrie ihm nach und sprach: Ach Herr, du Sohn
Davids, erbarme dich mein! Meine Tochter wird vom Teufel übel geplagt.

\bibverse{23} Und er antwortete ihr kein Wort. Da traten zu ihm seine
Jünger, baten ihn und sprachen: Lass sie doch von dir, denn sie schreit
uns nach.

\bibverse{24} Er antwortete aber und sprach: Ich bin nicht gesandt denn
nur zu den verlorenen Schafen von dem Hause Israel.

\bibverse{25} Sie kam aber und fiel vor ihm nieder und sprach: Herr,
hilf mir!

\bibverse{26} Aber er antwortete und sprach: Es ist nicht fein, dass man
den Kindern ihr Brot nehme und werfe es vor die Hunde.

\bibverse{27} Sie sprach: Ja, Herr; aber doch essen die Hündlein von den
Brosamen, die von ihrer Herren Tisch fallen.

\bibverse{28} Da antwortete Jesus und sprach zu ihr: O Weib, dein Glaube
ist groß! Dir geschehe, wie du willst. Und ihre Tochter ward gesund zu
derselben Stunde. \footnote{\textbf{15:28} Mt 8,10; Mt 8,13}

\bibverse{29} Und Jesus ging von da weiter und kam an das Galiläische
Meer und ging auf einen Berg und setzte sich allda.

\bibverse{30} Und es kam zu ihm viel Volks, die hatten mit sich Lahme,
Blinde, Stumme, Krüppel und viele andere und warfen sie Jesu vor die
Füße, und er heilte sie, \bibverse{31} dass sich das Volk verwunderte,
da sie sahen, dass die Stummen redeten, die Krüppel gesund waren, die
Lahmen gingen, die Blinden sahen; und sie priesen den Gott Israels.

\bibverse{32} Und Jesus rief seine Jünger zu sich und sprach: Es jammert
mich des Volks; denn sie beharren nun wohl drei Tage bei mir und haben
nichts zu essen; und ich will sie nicht ungegessen von mir lassen, auf
dass sie nicht verschmachten auf dem Wege. \footnote{\textbf{15:32} Mt
  14,13-21}

\bibverse{33} Da sprachen zu ihm seine Jünger: Woher mögen wir so viel
Brot nehmen in der Wüste, dass wir so viel Volks sättigen?

\bibverse{34} Und Jesus sprach zu ihnen: Wie viel Brote habt ihr? Sie
sprachen: Sieben und ein wenig Fischlein.

\bibverse{35} Und er hieß das Volk sich lagern auf die Erde

\bibverse{36} und nahm die sieben Brote und die Fische, dankte, brach
sie und gab sie seinen Jüngern; und die Jünger gaben sie dem Volk.
\bibverse{37} Und sie aßen alle und wurden satt; und hoben auf, was
übrig blieb von Brocken, sieben Körbe voll. \bibverse{38} Und die da
gegessen hatten, derer waren viertausend Mann, ausgenommen Weiber und
Kinder. \bibverse{39} Und da er das Volk hatte von sich gelassen, trat
er in ein Schiff und kam in das Gebiet Magdalas. \# 16 \bibverse{1} Da
traten die Pharisäer und Sadduzäer zu ihm; die versuchten ihn und
forderten, dass er sie ein Zeichen vom Himmel sehen ließe. \bibverse{2}
Aber er antwortete und sprach: Des Abends sprecht ihr: Es wird ein
schöner Tag werden, denn der Himmel ist rot; \bibverse{3} und des
Morgens sprecht ihr: Es wird heute Ungewitter sein, denn der Himmel ist
rot und trübe. Ihr Heuchler! über des Himmels Gestalt könnt ihr
urteilen; könnt ihr denn nicht auch über die Zeichen dieser Zeit
urteilen? \footnote{\textbf{16:3} Mt 11,4} \bibverse{4} Diese böse und
ehebrecherische Art sucht ein Zeichen; und soll ihr kein Zeichen gegeben
werden denn das Zeichen des Propheten Jona. Und er ließ sie und ging
davon. \footnote{\textbf{16:4} Mt 12,39-40}

\bibverse{5} Und da seine Jünger waren hinübergefahren, hatten sie
vergessen, Brot mit sich zu nehmen. \bibverse{6} Jesus aber sprach zu
ihnen: Sehet zu und hütet euch vor dem Sauerteig der Pharisäer und
Sadduzäer! \footnote{\textbf{16:6} Lk 12,1}

\bibverse{7} Da dachten sie bei sich selbst und sprachen: Das wird's
sein, dass wir nicht haben Brot mit uns genommen.

\bibverse{8} Da das Jesus merkte, sprach er zu ihnen: Ihr
Kleingläubigen, was bekümmert ihr euch doch, dass ihr nicht habt Brot
mit euch genommen? \bibverse{9} Vernehmet ihr noch nichts? Gedenket ihr
nicht an die fünf Brote unter die fünftausend und wie viel Körbe ihr da
aufhobt? \bibverse{10} auch nicht an die sieben Brote unter die
viertausend und wie viel Körbe ihr da aufhobt? \footnote{\textbf{16:10}
  Mt 15,34-38} \bibverse{11} Wie, verstehet ihr denn nicht, dass ich
euch nicht sage vom Brot, wenn ich sage: Hütet euch vor dem Sauerteig
der Pharisäer und Sadduzäer!

\bibverse{12} Da verstanden sie, dass er nicht gesagt hatte, dass sie
sich hüten sollten vor dem Sauerteig des Brots, sondern vor der Lehre
der Pharisäer und Sadduzäer.

\bibverse{13} Da kam Jesus in die Gegend der Stadt Cäsarea Philippi und
fragte seine Jünger und sprach: Wer sagen die Leute, dass des Menschen
Sohn sei?

\bibverse{14} Sie sprachen: Etliche sagen, du seist Johannes der Täufer;
die anderen, du seist Elia; etliche du seist Jeremia oder der Propheten
einer.

\bibverse{15} Er sprach zu ihnen: Wer sagt denn ihr, dass ich sei?

\bibverse{16} Da antwortete Simon Petrus und sprach: Du bist Christus,
des lebendigen Gottes Sohn! \footnote{\textbf{16:16} Joh 6,69}

\bibverse{17} Und Jesus antwortete und sprach zu ihm: Selig bist du,
Simon, Jonas Sohn; denn Fleisch und Blut hat dir das nicht offenbart,
sondern mein Vater im Himmel. \footnote{\textbf{16:17} Mt 11,27; Gal
  1,15-16} \bibverse{18} Und ich sage dir auch: Du bist Petrus, und auf
diesen Felsen will ich bauen meine Gemeinde, und die Pforten der Hölle
sollen sie nicht überwältigen. \footnote{\textbf{16:18} Joh 1,42; Eph
  2,20} \bibverse{19} Und ich will dir des Himmelsreichs Schlüssel
geben: alles, was du auf Erden binden wirst, soll auch im Himmel
gebunden sein, und alles, was du auf Erden lösen wirst, soll auch im
Himmel los sein. \footnote{\textbf{16:19} Mt 18,18} \bibverse{20} Da
verbot er seinen Jüngern, dass sie niemand sagen sollten, dass er,
Jesus, der Christus wäre. \footnote{\textbf{16:20} Mt 17,9}

\bibverse{21} Von der Zeit an fing Jesus an und zeigte seinen Jüngern,
wie er müsste hin gen Jerusalem gehen und viel leiden von den Ältesten
und Hohenpriestern und Schriftgelehrten und getötet werden und am
dritten Tage auferstehen. \footnote{\textbf{16:21} Mt 12,40; Joh 2,19}

\bibverse{22} Und Petrus nahm ihn zu sich, fuhr ihn an und sprach: Herr,
schone dein selbst; das widerfahre dir nur nicht!

\bibverse{23} Aber er wandte sich um und sprach zu Petrus: Hebe dich,
Satan, von mir! du bist mir ärgerlich; denn du meinst nicht was
göttlich, sondern was menschlich ist.

\bibverse{24} Da sprach Jesus zu seinen Jüngern: Will mir jemand
nachfolgen, der verleugne sich selbst und nehme sein Kreuz auf sich und
folge mir. \footnote{\textbf{16:24} Mt 10,38-39; 1Petr 2,21}
\bibverse{25} Denn wer sein Leben erhalten will, der wird's verlieren;
wer aber sein Leben verliert um meinetwillen, der wird's finden.
\footnote{\textbf{16:25} Offb 12,11} \bibverse{26} Was hülfe es dem
Menschen, wenn er die ganze Welt gewönne und nähme doch Schaden an
seiner Seele? Oder was kann der Mensch geben, damit er seine Seele
wieder löse? \footnote{\textbf{16:26} Lk 12,20} \bibverse{27} Denn es
wird geschehen, dass des Menschen Sohn komme in der Herrlichkeit seines
Vaters mit seinen Engeln; und alsdann wird er einem jeglichen vergelten
nach seinen Werken. \footnote{\textbf{16:27} Röm 2,6} \bibverse{28}
Wahrlich ich sage euch: Es stehen etliche hier, die nicht schmecken
werden den Tod, bis dass sie des Menschen Sohn kommen sehen in seinem
Reich. \footnote{\textbf{16:28} Mt 10,23}

\hypertarget{section-10}{%
\section{17}\label{section-10}}

\bibverse{1} Und nach sechs Tagen nahm Jesus zu sich Petrus und Jakobus
und Johannes, seinen Bruder, und führte sie beiseite auf einen hohen
Berg. \footnote{\textbf{17:1} Mt 26,37; Mk 5,37; Mk 13,3; Mk 14,33; Lk
  8,51} \bibverse{2} Und er ward verklärt vor ihnen, und sein Angesicht
leuchtete wie die Sonne, und seine Kleider wurden weiß wie ein Licht.
\footnote{\textbf{17:2} 2Petr 1,16-18; Offb 1,16} \bibverse{3} Und
siehe, da erschienen ihnen Mose und Elia; die redeten mit ihm.

\bibverse{4} Petrus aber antwortete und sprach zu Jesu: Herr, hier ist
gut sein! Willst du, so wollen wir hier drei Hütten machen: dir eine,
Mose eine und Elia eine.

\bibverse{5} Da er noch also redete, siehe, da überschattete sie eine
lichte Wolke. Und siehe, eine Stimme aus der Wolke sprach: Dies ist mein
lieber Sohn, an welchem ich Wohlgefallen habe, den sollt ihr hören!

\bibverse{6} Da das die Jünger hörten, fielen sie auf ihr Angesicht und
erschraken sehr. \bibverse{7} Jesus aber trat zu ihnen, rührte sie an
und sprach: Stehet auf und fürchtet euch nicht! \bibverse{8} Da sie aber
ihre Augen aufhoben, sahen sie niemand denn Jesum allein.

\bibverse{9} Und da sie vom Berge herabgingen, gebot ihnen Jesus und
sprach: Ihr sollt dies Gesicht niemand sagen, bis des Menschen Sohn von
den Toten auferstanden ist. \footnote{\textbf{17:9} Mt 16,20}

\bibverse{10} Und seine Jünger fragten ihn und sprachen: Was sagen denn
die Schriftgelehrten, Elia müsse zuvor kommen? \footnote{\textbf{17:10}
  Mt 11,14}

\bibverse{11} Jesus antwortete und sprach zu ihnen: Elia soll ja zuvor
kommen und alles zurechtbringen. \bibverse{12} Doch ich sage euch: Es
ist Elia schon gekommen, und sie haben ihn nicht erkannt, sondern haben
an ihm getan, was sie wollten. Also wird auch des Menschen Sohn leiden
müssen von ihnen. \footnote{\textbf{17:12} Mt 14,9-10} \bibverse{13} Da
verstanden die Jünger, dass er von Johannes dem Täufer zu ihnen geredet
hatte. \footnote{\textbf{17:13} Lk 1,17}

\bibverse{14} Und da sie zu dem Volk kamen, trat zu ihm ein Mensch und
fiel ihm zu Füßen \bibverse{15} und sprach: Herr, erbarme dich über
meinen Sohn! denn er ist mondsüchtig und hat ein schweres Leiden: er
fällt oft ins Feuer und oft ins Wasser; \bibverse{16} und ich habe ihn
zu deinen Jüngern gebracht, und sie konnten ihm nicht helfen.

\bibverse{17} Jesus aber antwortete und sprach: O du ungläubige und
verkehrte Art, wie lange soll ich bei euch sein? wie lange soll ich euch
dulden? Bringt mir ihn hierher! \bibverse{18} Und Jesus bedrohte ihn;
und der Teufel fuhr aus von ihm, und der Knabe ward gesund zu derselben
Stunde.

\bibverse{19} Da traten zu ihm seine Jünger besonders und sprachen:
Warum konnten wir ihn nicht austreiben? \footnote{\textbf{17:19} Mt 10,1}

\bibverse{20} Jesus aber antwortete und sprach zu ihnen: Um eures
Unglaubens willen. Denn wahrlich ich sage euch: So ihr Glauben habt wie
ein Senfkorn, so mögt ihr sagen zu diesem Berge: Hebe dich von hinnen
dorthin! so wird er sich heben; und euch wird nichts unmöglich sein.
\footnote{\textbf{17:20} Mt 21,21; Lk 17,6; 1Kor 13,2} \bibverse{21}
Aber diese Art fährt nicht aus denn durch Beten und Fasten. \footnote{\textbf{17:21}
  Mk 9,29}

\bibverse{22} Da sie aber ihr Wesen hatten in Galiläa, sprach Jesus zu
ihnen: Es wird geschehen, dass des Menschen Sohn überantwortet wird in
der Menschen Hände; \footnote{\textbf{17:22} Mt 16,21; Mt 20,18-19}
\bibverse{23} und sie werden ihn töten, und am dritten Tage wird er
auferstehen. Und sie wurden sehr betrübt.

\bibverse{24} Da sie nun gen Kapernaum kamen, gingen zu Petrus, die den
Zinsgroschen einnahmen, und sprachen: Pflegt euer Meister nicht den
Zinsgroschen zu geben? \footnote{\textbf{17:24} 2Mo 30,13}

\bibverse{25} Er sprach: Ja. Und als er heimkam, kam ihm Jesus zuvor und
sprach: Was dünkt dich, Simon? Von wem nehmen die Könige auf Erden den
Zoll oder Zins? Von ihren Kindern oder von den Fremden?

\bibverse{26} Da sprach zu ihm Petrus: Von den Fremden. Jesus sprach zu
ihm: So sind die Kinder frei.

\bibverse{27} Auf dass aber wir sie nicht ärgern, so gehe hin an das
Meer und wirf die Angel, und den ersten Fisch, der herauffährt, den
nimm; und wenn du seinen Mund auftust, wirst du einen Stater finden; den
nimm und gib ihnen für mich und dich. \# 18 \bibverse{1} Zu derselben
Stunde traten die Jünger zu Jesu und sprachen: Wer ist doch der Größte
im Himmelreich?

\bibverse{2} Jesus rief ein Kind zu sich und stellte das mitten unter
sie \bibverse{3} und sprach: Wahrlich ich sage euch: Es sei denn, dass
ihr euch umkehret und werdet wie die Kinder, so werdet ihr nicht ins
Himmelreich kommen. \bibverse{4} Wer nun sich selbst erniedrigt wie dies
Kind, der ist der Größte im Himmelreich. \bibverse{5} Und wer ein
solches Kind aufnimmt in meinem Namen, der nimmt mich auf. \footnote{\textbf{18:5}
  Mt 10,40} \bibverse{6} Wer aber ärgert dieser Geringsten einen, die an
mich glauben, dem wäre besser, dass ein Mühlstein an seinen Hals gehängt
und er ersäuft würde im Meer, da es am tiefsten ist. \footnote{\textbf{18:6}
  Lk 17,1-2}

\bibverse{7} Weh der Welt der Ärgernisse halben! Es muss ja Ärgernis
kommen; doch weh dem Menschen, durch welchen Ärgernis kommt!
\bibverse{8} So aber deine Hand oder dein Fuß dich ärgert, so haue ihn
ab und wirf ihn von dir. Es ist dir besser, dass du zum Leben lahm oder
als Krüppel eingehst, denn dass du zwei Hände oder zwei Füße habest und
werdest in das ewige Feuer geworfen. \footnote{\textbf{18:8} Mt 5,29-30}
\bibverse{9} Und so dich dein Auge ärgert, reiß es aus und wirf's von
dir. Es ist dir besser, dass du einäugig zum Leben eingehest, denn dass
du zwei Augen habest und werdest in das höllische Feuer geworfen.
\bibverse{10} Sehet zu, dass ihr nicht jemand von diesen Kleinen
verachtet. Denn ich sage euch: Ihre Engel im Himmel sehen allezeit das
Angesicht meines Vaters im Himmel. \bibverse{11} Denn des Menschen Sohn
ist gekommen, selig zu machen, das verloren ist. \footnote{\textbf{18:11}
  Mt 9,13; Lk 19,10}

\bibverse{12} Was dünkt euch? Wenn irgendein Mensch hundert Schafe hätte
und eins unter ihnen sich verirrte: lässt er nicht die neunundneunzig
auf den Bergen, geht hin und sucht das verirrte? \bibverse{13} Und so
sich's begibt, dass er's findet, wahrlich ich sage euch, er freut sich
darüber mehr denn über die neunundneunzig, die nicht verirrt sind.
\bibverse{14} Also auch ist's vor eurem Vater im Himmel nicht der Wille,
dass jemand von diesen Kleinen verloren werde.

\bibverse{15} Sündigt aber dein Bruder an dir, so gehe hin und strafe
ihn zwischen dir und ihm allein. Hört er dich, so hast du deinen Bruder
gewonnen. \bibverse{16} Hört er dich nicht, so nimm noch einen oder zwei
zu dir, auf dass alle Sache bestehe auf zweier oder dreier Zeugen Mund.
\footnote{\textbf{18:16} 5Mo 19,15} \bibverse{17} Hört er die nicht, so
sage es der Gemeinde. Hört er die Gemeinde nicht, so halt ihn als einen
Heiden und Zöllner. \footnote{\textbf{18:17} 1Kor 5,13; 2Thes 3,6; Tit
  3,10} \bibverse{18} Wahrlich ich sage euch: Was ihr auf Erden binden
werdet, soll auch im Himmel gebunden sein, und was ihr auf Erden lösen
werdet, soll auch im Himmel los sein. \footnote{\textbf{18:18} Mt 16,19;
  Joh 20,23} \bibverse{19} Weiter sage ich euch: Wo zwei unter euch eins
werden auf Erden, warum es ist, dass sie bitten wollen, das soll ihnen
widerfahren von meinem Vater im Himmel. \footnote{\textbf{18:19} Mk
  11,24} \bibverse{20} Denn wo zwei oder drei versammelt sind in meinem
Namen, da bin ich mitten unter ihnen. \footnote{\textbf{18:20} Mt 28,20}

\bibverse{21} Da trat Petrus zu ihm und sprach: Herr, wie oft muss ich
denn meinem Bruder, der an mir sündigt, vergeben? Ist's genug siebenmal?

\bibverse{22} Jesus sprach zu ihm: Ich sage dir: Nicht siebenmal,
sondern siebzigmal siebenmal. \bibverse{23} Darum ist das Himmelreich
gleich einem König, der mit seinen Knechten rechnen wollte.
\bibverse{24} Und als er anfing zu rechnen, kam ihm einer vor, der war
ihm zehntausend Pfund schuldig. \bibverse{25} Da er's nun nicht hatte,
zu bezahlen, hieß der Herr verkaufen ihn und sein Weib und seine Kinder
und alles, was er hatte, und bezahlen. \bibverse{26} Da fiel der Knecht
nieder und betete ihn an und sprach: Herr, habe Geduld mit mir, ich will
dir's alles bezahlen. \bibverse{27} Da jammerte den Herrn des Knechtes,
und er ließ ihn los, und die Schuld erließ er ihm auch.

\bibverse{28} Da ging derselbe Knecht hinaus und fand einen seiner
Mitknechte, der war ihm hundert Groschen schuldig; und er griff ihn an
und würgte ihn und sprach: Bezahle mir, was du mir schuldig bist!

\bibverse{29} Da fiel sein Mitknecht nieder und bat ihn und sprach: Habe
Geduld mit mir; ich will dir's alles bezahlen. \bibverse{30} Er wollte
aber nicht, sondern ging hin und warf ihn ins Gefängnis, bis dass er
bezahlte, was er schuldig war. \bibverse{31} Da aber seine Mitknechte
solches sahen, wurden sie sehr betrübt und kamen und brachten vor ihren
Herrn alles, was sich begeben hatte. \bibverse{32} Da forderte ihn sein
Herr vor sich und sprach zu ihm: Du Schalksknecht, alle diese Schuld
habe ich dir erlassen, dieweil du mich batest; \footnote{\textbf{18:32}
  Lk 6,36} \bibverse{33} solltest du denn dich nicht auch erbarmen über
deinen Mitknecht, wie ich mich über dich erbarmt habe? \footnote{\textbf{18:33}
  1Jo 4,11} \bibverse{34} Und sein Herr ward zornig und überantwortete
ihn den Peinigern, bis dass er bezahlte alles, was er ihm schuldig war.
\footnote{\textbf{18:34} Mt 5,26} \bibverse{35} Also wird euch mein
himmlischer Vater auch tun, so ihr nicht vergebet von eurem Herzen, ein
jeglicher seinem Bruder seine Fehler. \footnote{\textbf{18:35} Mt
  6,14-15; Jak 2,13}

\hypertarget{section-11}{%
\section{19}\label{section-11}}

\bibverse{1} Und es begab sich, da Jesus diese Reden vollendet hatte,
erhob er sich aus Galiläa und kam in das Gebiet des jüdischen Landes
jenseits des Jordans; \bibverse{2} und es folgte ihm viel Volks nach,
und er heilte sie daselbst.

\bibverse{3} Da traten zu ihm die Pharisäer, versuchten ihn und sprachen
zu ihm: Ist's auch recht, dass sich ein Mann scheide von seinem Weibe um
irgendeine Ursache? \footnote{\textbf{19:3} Mt 5,31-32}

\bibverse{4} Er antwortete aber und sprach zu ihnen: Habt ihr nicht
gelesen, dass, der im Anfang den Menschen gemacht hat, der machte, dass
ein Mann und ein Weib sein sollte, \footnote{\textbf{19:4} 1Mo 1,27}
\bibverse{5} und sprach: „Darum wird ein Mensch Vater und Mutter
verlassen und an seinem Weibe hangen, und werden die zwei ein Fleisch
sein``? \bibverse{6} So sind sie nun nicht zwei, sondern ein Fleisch.
Was nun Gott zusammengefügt hat, das soll der Mensch nicht scheiden.
\footnote{\textbf{19:6} 1Kor 7,10-11}

\bibverse{7} Da sprachen sie: Warum hat denn Mose geboten, einen
Scheidebrief zu geben und sich von ihr zu scheiden? \footnote{\textbf{19:7}
  5Mo 24,1}

\bibverse{8} Er sprach zu ihnen: Mose hat euch erlaubt zu scheiden von
euren Weibern wegen eures Herzens Härtigkeit; von Anbeginn aber ist's
nicht also gewesen. \bibverse{9} Ich sage aber euch: Wer sich von seinem
Weibe scheidet (es sei denn um der Hurerei willen) und freit eine
andere, der bricht die Ehe; und wer die Abgeschiedene freit, der bricht
auch die Ehe. \footnote{\textbf{19:9} Lk 16,18}

\bibverse{10} Da sprachen die Jünger zu ihm: Steht die Sache eines
Mannes mit seinem Weibe also, so ist's nicht gut, ehelich werden.

\bibverse{11} Er sprach aber zu ihnen: Das Wort fasst nicht jedermann,
sondern denen es gegeben ist. \bibverse{12} Denn es sind etliche
verschnitten, die sind aus Mutterleibe also geboren; und sind etliche
verschnitten, die von Menschen verschnitten sind; und sind etliche
verschnitten, die sich selbst verschnitten haben um des Himmelreiches
willen. Wer es fassen kann, der fasse es!

\bibverse{13} Da wurden Kindlein zu ihm gebracht, dass er die Hände auf
sie legte und betete. Die Jünger aber fuhren sie an. \bibverse{14} Aber
Jesus sprach: Lasset die Kindlein und wehret ihnen nicht, zu mir zu
kommen; denn solcher ist das Himmelreich. \footnote{\textbf{19:14} Mt
  18,2-3} \bibverse{15} Und legte die Hände auf sie und zog von dannen.

\bibverse{16} Und siehe, einer trat zu ihm und sprach: Guter Meister,
was soll ich Gutes tun, dass ich das ewige Leben möge haben?

\bibverse{17} Er aber sprach zu ihm: Was heißest du mich gut? Niemand
ist gut denn der einige Gott. Willst du aber zum Leben eingehen, so
halte die Gebote.

\bibverse{18} Da sprach er zu ihm: Welche? Jesus aber sprach: „Du sollst
nicht töten; du sollst nicht ehebrechen; du sollst nicht stehlen; du
sollst nicht falsch Zeugnis geben;

\bibverse{19} ehre Vater und Mutter;`` und: „du sollst deinen Nächsten
lieben wie dich selbst.`` \footnote{\textbf{19:19} 3Mo 19,18}

\bibverse{20} Da sprach der Jüngling zu ihm: Das habe ich alles gehalten
von meiner Jugend auf; was fehlt mir noch?

\bibverse{21} Jesus sprach zu ihm: Willst du vollkommen sein, so gehe
hin, verkaufe, was du hast, und gib's den Armen, so wirst du einen
Schatz im Himmel haben; und komm und folge mir nach! \bibverse{22} Da
der Jüngling das Wort hörte, ging er betrübt von ihm, denn er hatte
viele Güter. \footnote{\textbf{19:22} Ps 62,11}

\bibverse{23} Jesus aber sprach zu seinen Jüngern: Wahrlich ich sage
euch: Ein Reicher wird schwer ins Himmelreich kommen. \bibverse{24} Und
weiter sage ich euch: Es ist leichter, dass ein Kamel durch ein Nadelöhr
gehe, denn dass ein Reicher ins Reich Gottes komme.

\bibverse{25} Da das seine Jünger hörten, entsetzten sie sich sehr und
sprachen: Ja, wer kann denn selig werden?

\bibverse{26} Jesus aber sah sie an und sprach zu ihnen: Bei den
Menschen ist's unmöglich; aber bei Gott sind alle Dinge möglich.
\footnote{\textbf{19:26} Hi 42,2}

\bibverse{27} Da antwortete Petrus und sprach zu ihm: Siehe, wir haben
alles verlassen und sind dir nachgefolgt; was wird uns dafür?
\footnote{\textbf{19:27} Mt 4,20; Mt 4,22}

\bibverse{28} Jesus aber sprach zu ihnen: Wahrlich ich sage euch: Ihr,
die ihr mir seid nachgefolgt, werdet in der Wiedergeburt, da des
Menschen Sohn wird sitzen auf dem Stuhl seiner Herrlichkeit, auch sitzen
auf zwölf Stühlen und richten die zwölf Geschlechter Israels.
\footnote{\textbf{19:28} Lk 22,30; 1Kor 6,2; Offb 3,21} \bibverse{29}
Und wer verlässt Häuser oder Brüder oder Schwestern oder Vater oder
Mutter oder Weib oder Kinder oder Äcker um meines Namens willen, der
wird's hundertfältig nehmen und das ewige Leben ererben. \bibverse{30}
Aber viele, die da sind die Ersten, werden die Letzten, und die Letzten
werden die Ersten sein. \# 20 \bibverse{1} Das Himmelreich ist gleich
einem Hausvater, der am Morgen ausging, Arbeiter zu mieten in seinen
Weinberg. \bibverse{2} Und da er mit den Arbeitern eins ward um einen
Groschen zum Tagelohn, sandte er sie in seinen Weinberg. \bibverse{3}
Und ging aus um die dritte Stunde und sah andere an dem Markte müßig
stehen \bibverse{4} und sprach zu ihnen: Gehet ihr auch hin in den
Weinberg; ich will euch geben, was recht ist. \bibverse{5} Und sie
gingen hin. Abermals ging er aus um die sechste und die neunte Stunde
und tat gleichalso. \bibverse{6} Um die elfte Stunde aber ging er aus
und fand andere müßig stehen und sprach zu ihnen: Was stehet ihr hier
den ganzen Tag müßig?

\bibverse{7} Sie sprachen zu ihm: Es hat uns niemand gedingt. Er sprach
zu ihnen: Gehet ihr auch hin in den Weinberg, und was recht sein wird,
soll euch werden.

\bibverse{8} Da es nun Abend ward, sprach der Herr des Weinbergs zu
seinem Schaffner: Rufe die Arbeiter und gib ihnen den Lohn und heb an an
den Letzten bis zu den Ersten.

\bibverse{9} Da kamen, die um die elfte Stunde gedingt waren, und
empfing ein jeglicher seinen Groschen. \bibverse{10} Da aber die Ersten
kamen, meinten sie, sie würden mehr empfangen; und sie empfingen auch
ein jeglicher seinen Groschen. \bibverse{11} Und da sie den empfingen,
murrten sie wider den Hausvater \bibverse{12} und sprachen: Diese
Letzten haben nur eine Stunde gearbeitet, und du hast sie uns gleich
gemacht, die wir des Tages Last und die Hitze getragen haben.

\bibverse{13} Er antwortete aber und sagte zu einem unter ihnen: Mein
Freund, ich tue dir nicht Unrecht. Bist du nicht mit mir eins geworden
um einen Groschen? \bibverse{14} Nimm, was dein ist, und gehe hin! Ich
will aber diesem Letzten geben gleich wie dir. \bibverse{15} Oder habe
ich nicht Macht, zu tun, was ich will, mit dem Meinen? Siehst du darum
scheel, dass ich so gütig bin? \bibverse{16} Also werden die Letzten die
Ersten und die Ersten die Letzten sein. Denn viele sind berufen, aber
wenige sind auserwählt.

\bibverse{17} Und er zog hinauf gen Jerusalem und nahm zu sich die zwölf
Jünger besonders auf dem Wege und sprach zu ihnen: \bibverse{18} Siehe,
wir ziehen hinauf gen Jerusalem, und des Menschen Sohn wird den
Hohenpriestern und Schriftgelehrten überantwortet werden; und sie werden
ihn verdammen zum Tode \footnote{\textbf{20:18} Mt 16,21; Mt 17,22-23;
  Joh 2,13} \bibverse{19} und werden ihn überantworten den Heiden, zu
verspotten und zu geißeln und zu kreuzigen; und am dritten Tage wird er
wieder auferstehen.

\bibverse{20} Da trat zu ihm die Mutter der Kinder des Zebedäus mit
ihren Söhnen, fiel vor ihm nieder und bat etwas von ihm. \bibverse{21}
Und er sprach zu ihr: Was willst du? Sie sprach zu ihm: Lass diese meine
zwei Söhne sitzen in deinem Reich, einen zu deiner Rechten und den
anderen zu deiner Linken. \footnote{\textbf{20:21} Mt 19,28}

\bibverse{22} Aber Jesus antwortete und sprach: Ihr wisset nicht, was
ihr bittet. Könnt ihr den Kelch trinken, den ich trinken werde, und euch
taufen lassen mit der Taufe, mit der ich getauft werde? Sie sprachen zu
ihm: Jawohl. \footnote{\textbf{20:22} Mt 26,39; Lk 12,50}

\bibverse{23} Und er sprach zu ihnen: Meinen Kelch sollt ihr zwar
trinken, und mit der Taufe, mit der ich getauft werde, sollt ihr getauft
werden; aber das Sitzen zu meiner Rechten und Linken zu geben steht mir
nicht zu, sondern denen es bereitet ist von meinem Vater. \footnote{\textbf{20:23}
  Apg 12,2; Offb 1,9}

\bibverse{24} Da das die zehn hörten, wurden sie unwillig über die zwei
Brüder. \footnote{\textbf{20:24} Lk 22,24-26}

\bibverse{25} Aber Jesus rief sie zu sich und sprach: Ihr wisset, dass
die weltlichen Fürsten herrschen und die Oberherren haben Gewalt.

\bibverse{26} So soll es nicht sein unter euch. Sondern, wenn jemand
will unter euch gewaltig sein, der sei euer Diener; \footnote{\textbf{20:26}
  Mt 23,11; 1Kor 9,19}

\bibverse{27} und wer da will der Vornehmste sein, der sei euer Knecht,
-- \footnote{\textbf{20:27} Mk 9,35} \bibverse{28} gleichwie des
Menschen Sohn ist nicht gekommen, dass er sich dienen lasse, sondern
dass er diene und gebe sein Leben zu einer Erlösung für viele.
\footnote{\textbf{20:28} Lk 22,27; Phil 2,7; 1Petr 1,18-19}

\bibverse{29} Und da sie von Jericho auszogen, folgte ihm viel Volks
nach. \bibverse{30} Und siehe, zwei Blinde saßen am Wege; und da sie
hörten, dass Jesus vorüberging, schrien sie und sprachen: Ach Herr, du
Sohn Davids, erbarme dich unser! \bibverse{31} Aber das Volk bedrohte
sie, dass sie schweigen sollten. Aber sie schrien viel mehr und
sprachen: Ach Herr, du Sohn Davids, erbarme dich unser!

\bibverse{32} Jesus aber stand still und rief sie und sprach: Was wollt
ihr, dass ich euch tun soll?

\bibverse{33} Sie sprachen zu ihm: Herr, dass unsere Augen aufgetan
werden.

\bibverse{34} Und es jammerte Jesum, und er rührte ihre Augen an; und
alsbald wurden ihre Augen wieder sehend, und sie folgten ihm nach. \# 21
\bibverse{1} Da sie nun nahe an Jerusalem kamen, gen Bethphage an den
Ölberg, sandte Jesus seiner Jünger zwei \bibverse{2} und sprach zu
ihnen: Gehet hin in den Flecken, der vor euch liegt, und alsbald werdet
ihr eine Eselin finden angebunden und ihr Füllen bei ihr; löset sie auf
und führet sie zu mir! \bibverse{3} Und so euch jemand etwas wird sagen,
so sprechet: Der Herr bedarf ihrer; sobald wird er sie euch lassen.

\bibverse{4} Das geschah aber alles, auf dass erfüllet würde, was gesagt
ist durch den Propheten, der da spricht: \bibverse{5} „Saget der Tochter
Zion: Siehe, dein König kommt zu dir sanftmütig und reitet auf einem
Esel und auf einem Füllen der lastbaren Eselin.``

\bibverse{6} Die Jünger gingen hin und taten, wie ihnen Jesus befohlen
hatte, \bibverse{7} und brachten die Eselin und das Füllen und legten
ihre Kleider darauf und setzten ihn darauf. \bibverse{8} Aber viel Volks
breitete die Kleider auf den Weg; die anderen hieben Zweige von den
Bäumen und streuten sie auf den Weg. \footnote{\textbf{21:8} 2Kö 9,13}
\bibverse{9} Das Volk aber, das vorging und nachfolgte, schrie und
sprach: Hosianna dem Sohn Davids! Gelobt sei, der da kommt in dem Namen
des Herrn! Hosianna in der Höhe! \footnote{\textbf{21:9} Ps 118,25-26}

\bibverse{10} Und als er zu Jerusalem einzog, erregte sich die ganze
Stadt und sprach: Wer ist der?

\bibverse{11} Das Volk aber sprach: Das ist der Jesus, der Prophet von
Nazareth aus Galiläa.

\bibverse{12} Und Jesus ging zum Tempel Gottes hinein und trieb heraus
alle Verkäufer und Käufer im Tempel und stieß um der Wechsler Tische und
die Stühle der Taubenkrämer \bibverse{13} und sprach zu ihnen: Es steht
geschrieben: „Mein Haus soll ein Bethaus heißen``; ihr aber habt eine
Mördergrube daraus gemacht. \footnote{\textbf{21:13} Jer 7,11}

\bibverse{14} Und es gingen zu ihm Blinde und Lahme im Tempel, und er
heilte sie. \bibverse{15} Da aber die Hohenpriester und Schriftgelehrten
sahen die Wunder, die er tat, und die Kinder, die im Tempel schrien und
sagten: Hosianna dem Sohn Davids! wurden sie entrüstet \bibverse{16} und
sprachen zu ihm: Hörst du auch, was diese sagen? Jesus sprach zu ihnen:
Ja! Habt ihr nie gelesen: „Aus dem Munde der Unmündigen und Säuglinge
hast du Lob zugerichtet``?

\bibverse{17} Und er ließ sie da und ging zur Stadt hinaus gen Bethanien
und blieb daselbst.

\bibverse{18} Als er aber des Morgens wieder in die Stadt ging, hungerte
ihn;

\bibverse{19} und er sah einen Feigenbaum am Wege und ging hinzu und
fand nichts daran denn allein Blätter und sprach zu ihm: Nun wachse auf
dir hinfort nimmermehr eine Frucht! Und der Feigenbaum verdorrte
alsbald.

\bibverse{20} Und da das die Jünger sahen, verwunderten sie sich und
sprachen: Wie ist der Feigenbaum so bald verdorrt?

\bibverse{21} Jesus aber antwortete und sprach zu ihnen: Wahrlich ich
sage euch: So ihr Glauben habt und nicht zweifelt, so werdet ihr nicht
allein solches mit dem Feigenbaum tun, sondern, so ihr werdet sagen zu
diesem Berge: Hebe dich auf und wirf dich ins Meer! so wird's geschehen.
\footnote{\textbf{21:21} Mt 17,20}

\bibverse{22} Und alles, was ihr bittet im Gebet, so ihr glaubet, werdet
ihr's empfangen.

\bibverse{23} Und als er in den Tempel kam, traten zu ihm, als er
lehrte, die Hohenpriester und die Ältesten im Volk und sprachen: Aus was
für Macht tust du das? und wer hat dir die Macht gegeben?

\bibverse{24} Jesus aber antwortete und sprach zu ihnen: Ich will euch
auch ein Wort fragen; so ihr mir das saget, will ich euch auch sagen,
aus was für Macht ich das tue: \bibverse{25} Woher war die Taufe des
Johannes? War sie vom Himmel oder von den Menschen? Da gedachten sie bei
sich selbst und sprachen: Sagen wir, sie sei vom Himmel gewesen, so wird
er zu uns sagen: Warum glaubtet ihr ihm denn nicht?

\bibverse{26} Sagen wir aber, sie sei von Menschen gewesen, so müssen
wir uns vor dem Volk fürchten; denn sie halten alle Johannes für einen
Propheten. \footnote{\textbf{21:26} Mt 14,5} \bibverse{27} Und sie
antworteten Jesu und sprachen: Wir wissen's nicht. Da sprach er zu
ihnen: So sage ich euch auch nicht, aus was für Macht ich das tue.

\bibverse{28} Was dünkt euch aber? Es hatte ein Mann zwei Söhne und ging
zu dem ersten und sprach: Mein Sohn, gehe hin und arbeite heute in
meinem Weinberge. \bibverse{29} Er antwortete aber und sprach: Ich
will's nicht tun. Darnach reute es ihn und er ging hin. \bibverse{30}
Und er ging zum anderen und sprach gleichalso. Er antwortete aber und
sprach: Herr, ja! -- und ging nicht hin. \bibverse{31} Welcher unter den
zweien hat des Vaters Willen getan? Sie sprachen zu ihm: Der erste.
Jesus sprach zu ihnen: Wahrlich ich sage euch: Die Zöllner und Huren
mögen wohl eher ins Himmelreich kommen denn ihr. \footnote{\textbf{21:31}
  Lk 18,14}

\bibverse{32} Johannes kam zu euch und lehrte euch den rechten Weg, und
ihr glaubtet ihm nicht; aber die Zöllner und Huren glaubten ihm. Und ob
ihr's wohl sahet, tatet ihr dennoch nicht Buße, dass ihr ihm darnach
auch geglaubt hättet. \footnote{\textbf{21:32} Lk 7,29}

\bibverse{33} Höret ein anderes Gleichnis: Es war ein Hausvater, der
pflanzte einen Weinberg und führte einen Zaun darum und grub eine Kelter
darin und baute einen Turm und tat ihn den Weingärtnern aus und zog über
Land. \footnote{\textbf{21:33} Jes 5,1-2}

\bibverse{34} Da nun herbeikam die Zeit der Früchte, sandte er seine
Knechte zu den Weingärtnern, dass sie seine Früchte empfingen.
\bibverse{35} Da nahmen die Weingärtner seine Knechte; einen stäupten
sie, den anderen töteten sie, den dritten steinigten sie. \bibverse{36}
Abermals sandte er andere Knechte, mehr denn der ersten waren; und sie
taten ihnen gleichalso. \bibverse{37} Darnach sandte er seinen Sohn zu
ihnen und sprach: Sie werden sich vor meinem Sohn scheuen. \bibverse{38}
Da aber die Weingärtner den Sohn sahen, sprachen sie untereinander: Das
ist der Erbe; kommt, lasst uns ihn töten und sein Erbgut an uns bringen!
\bibverse{39} Und sie nahmen ihn und stießen ihn zum Weinberge hinaus
und töteten ihn. \bibverse{40} Wenn nun der Herr des Weinberges kommen
wird, was wird er diesen Weingärtnern tun?

\bibverse{41} Sie sprachen zu ihm: Er wird die Bösewichte übel umbringen
und seinen Weinberg anderen Weingärtnern austun, die ihm die Früchte zu
rechter Zeit geben.

\bibverse{42} Jesus sprach zu ihnen: Habt ihr nie gelesen in der
Schrift: „Der Stein, den die Bauleute verworfen haben, der ist zum
Eckstein geworden. Von dem Herrn ist das geschehen, und es ist wunderbar
vor unseren Augen``? \footnote{\textbf{21:42} Apg 4,11; 1Petr 2,4-8}

\bibverse{43} Darum sage ich euch: Das Reich Gottes wird von euch
genommen und einem Volke gegeben werden, das seine Früchte bringt.
\bibverse{44} Und wer auf diesen Stein fällt, der wird zerschellen; auf
wen aber er fällt, den wird er zermalmen.

\bibverse{45} Und da die Hohenpriester und Pharisäer seine Gleichnisse
hörten, verstanden sie, dass er von ihnen redete. \bibverse{46} Und sie
trachteten darnach, wie sie ihn griffen; aber sie fürchteten sich vor
dem Volk, denn es hielt ihn für einen Propheten. \# 22 \bibverse{1} Und
Jesus antwortete und redete abermals durch Gleichnisse zu ihnen und
sprach: \bibverse{2} Das Himmelreich ist gleich einem Könige, der seinem
Sohn Hochzeit machte. \footnote{\textbf{22:2} Joh 3,29} \bibverse{3} Und
sandte seine Knechte aus, dass sie die Gäste zur Hochzeit riefen; und
sie wollten nicht kommen. \bibverse{4} Abermals sandte er andere Knechte
aus und sprach: Saget den Gästen: Siehe, meine Mahlzeit habe ich
bereitet, meine Ochsen und mein Mastvieh ist geschlachtet und alles
bereit; kommt zur Hochzeit! \bibverse{5} Aber sie verachteten das und
gingen hin, einer auf seinen Acker, der andere zu seiner Hantierung;
\bibverse{6} etliche aber griffen seine Knechte, höhnten und töteten
sie. \bibverse{7} Da das der König hörte, ward er zornig und schickte
seine Heere aus und brachte diese Mörder um und zündete ihre Stadt an.
\footnote{\textbf{22:7} Mt 24,2}

\bibverse{8} Da sprach er zu seinen Knechten: Die Hochzeit ist zwar
bereit, aber die Gäste waren's nicht wert. \bibverse{9} Darum gehet hin
auf die Straßen und ladet zur Hochzeit, wen ihr findet. \bibverse{10}
Und die Knechte gingen aus auf die Straßen und brachten zusammen, wen
sie fanden, Böse und Gute; und die Tische wurden alle voll.

\bibverse{11} Da ging der König hinein, die Gäste zu besehen, und sah
allda einen Menschen, der hatte kein hochzeitlich Kleid an; \footnote{\textbf{22:11}
  Offb 19,8} \bibverse{12} und er sprach zu ihm: Freund, wie bist du
hereingekommen und hast doch kein hochzeitlich Kleid an? Er aber
verstummte. \bibverse{13} Da sprach der König zu seinen Dienern: Bindet
ihm Hände und Füße und werfet ihn in die Finsternis hinaus! da wird sein
Heulen und Zähneklappen. \bibverse{14} Denn viele sind berufen, aber
wenige sind auserwählt.

\bibverse{15} Da gingen die Pharisäer hin und hielten einen Rat, wie sie
ihn fingen in seiner Rede. \bibverse{16} Und sandten zu ihm ihre Jünger
samt des Herodes Dienern. Und sie sprachen: Meister, wir wissen, dass du
wahrhaftig bist und lehrst den Weg Gottes recht und du fragst nach
niemand; denn du achtest nicht das Ansehen der Menschen. \bibverse{17}
Darum sage uns, was dünkt dich: Ist's recht, dass man dem Kaiser Zins
gebe, oder nicht?

\bibverse{18} Da nun Jesus merkte ihre Schalkheit, sprach er: Ihr
Heuchler, was versuchet ihr mich? \bibverse{19} Weiset mir die
Zinsmünze! Und sie reichten ihm einen Groschen dar.

\bibverse{20} Und er sprach zu ihnen: Wes ist das Bild und die
Überschrift?

\bibverse{21} Sie sprachen zu ihm: Des Kaisers. Da sprach er zu ihnen:
So gebet dem Kaiser, was des Kaisers ist, und Gott, was Gottes ist!
\footnote{\textbf{22:21} Lk 23,2; Röm 13,7}

\bibverse{22} Da sie das hörten, verwunderten sie sich und ließen ihn
und gingen davon.

\bibverse{23} An dem Tage traten zu ihm die Sadduzäer, die da halten, es
sei kein Auferstehen, und fragten ihn

\bibverse{24} und sprachen: Meister, Mose hat gesagt: So einer stirbt
und hat nicht Kinder, so soll sein Bruder sein Weib freien und seinem
Bruder Samen erwecken.

\bibverse{25} Nun sind bei uns gewesen sieben Brüder. Der erste freite
und starb; und dieweil er nicht Samen hatte, ließ er sein Weib seinem
Bruder; \bibverse{26} desgleichen der andere und der dritte bis an den
siebenten. \bibverse{27} Zuletzt nach allen starb auch das Weib.
\bibverse{28} Nun in der Auferstehung, wes Weib wird sie sein unter den
sieben? Sie haben sie ja alle gehabt.

\bibverse{29} Jesus aber antwortete und sprach zu ihnen: Ihr irret und
wisset die Schrift nicht, noch die Kraft Gottes. \bibverse{30} In der
Auferstehung werden sie weder freien noch sich freien lassen, sondern
sie sind gleichwie die Engel Gottes im Himmel. \bibverse{31} Habt ihr
aber nicht gelesen von der Toten Auferstehung, was euch gesagt ist von
Gott, der da spricht: \bibverse{32} „Ich bin der Gott Abrahams und der
Gott Isaaks und der Gott Jakobs``? Gott aber ist nicht ein Gott der
Toten, sondern der Lebendigen.

\bibverse{33} Und da solches das Volk hörte, entsetzten sie sich über
seine Lehre.

\bibverse{34} Da aber die Pharisäer hörten, dass er den Sadduzäern das
Maul gestopft hatte, versammelten sie sich. \bibverse{35} Und einer
unter ihnen, ein Schriftgelehrter, versuchte ihn und sprach:
\bibverse{36} Meister, welches ist das vornehmste Gebot im Gesetz?

\bibverse{37} Jesus aber sprach zu ihm: „Du sollst lieben Gott, deinen
Herrn, von ganzem Herzen, von ganzer Seele und von ganzem Gemüte.``
\bibverse{38} Dies ist das vornehmste und größte Gebot. \bibverse{39}
Das andere aber ist ihm gleich: „Du sollst deinen Nächsten lieben wie
dich selbst.`` \bibverse{40} In diesen zwei Geboten hängt das ganze
Gesetz und die Propheten. \footnote{\textbf{22:40} Röm 13,9-10}

\bibverse{41} Da nun die Pharisäer beieinander waren, fragte sie Jesus
\bibverse{42} und sprach: Wie dünkt euch um Christus? wes Sohn ist er?
Sie sprachen: Davids.

\bibverse{43} Er sprach zu ihnen: Wie nennt ihn denn David im Geist
einen Herrn, da er sagt:

\bibverse{44} „Der Herr hat gesagt zu meinem Herrn: Setze dich zu meiner
Rechten, bis dass ich lege deine Feinde zum Schemel deiner Füße``?
\footnote{\textbf{22:44} Mt 26,64}

\bibverse{45} So nun David ihn einen Herrn nennt, wie ist er denn sein
Sohn?

\bibverse{46} Und niemand konnte ihm ein Wort antworten, und wagte auch
niemand von dem Tage an hinfort, ihn zu fragen. \# 23 \bibverse{1} Da
redete Jesus zu dem Volk und zu seinen Jüngern \bibverse{2} und sprach:
Auf Moses Stuhl sitzen die Schriftgelehrten und Pharisäer. \bibverse{3}
Alles nun, was sie euch sagen, dass ihr halten sollet, das haltet und
tut's; aber nach ihren Werken sollt ihr nicht tun: sie sagen's wohl, und
tun's nicht. \bibverse{4} Sie binden aber schwere und unerträgliche
Bürden und legen sie den Menschen auf den Hals; aber sie selbst wollen
dieselben nicht mit einem Finger regen. \footnote{\textbf{23:4} Mt
  11,28-30; Apg 15,10; Apg 15,28} \bibverse{5} Alle ihre Werke aber tun
sie, dass sie von den Leuten gesehen werden. Sie machen ihre Denkzettel
breit und die Säume an ihren Kleidern groß. \footnote{\textbf{23:5} Mt
  6,1; 2Mo 13,9; 4Mo 15,38-39} \bibverse{6} Sie sitzen gern obenan über
Tisch und in den Schulen \footnote{\textbf{23:6} Lk 14,7} \bibverse{7}
und haben's gern, dass sie gegrüßt werden auf dem Markt und von den
Menschen Rabbi genannt werden. \bibverse{8} Aber ihr sollt euch nicht
Rabbi nennen lassen; denn einer ist euer Meister, Christus; ihr aber
seid alle Brüder. \bibverse{9} Und sollt niemand Vater heißen auf Erden,
denn einer ist euer Vater, der im Himmel ist. \bibverse{10} Und ihr
sollt euch nicht lassen Meister nennen; denn einer ist euer Meister,
Christus. \bibverse{11} Der Größte unter euch soll euer Diener sein.
\bibverse{12} Denn wer sich selbst erhöht, der wird erniedrigt; und wer
sich selbst erniedrigt, der wird erhöht. \footnote{\textbf{23:12} Spr
  29,23; Hi 22,29; Hes 21,31; Lk 18,14; 1Petr 5,5}

\bibverse{13} Weh euch, Schriftgelehrte und Pharisäer, ihr Heuchler, die
ihr das Himmelreich zuschließet vor den Menschen! Ihr kommt nicht
hinein, und die hinein wollen, lasst ihr nicht hineingehen.

\bibverse{14} Weh euch, Schriftgelehrte und Pharisäer, ihr Heuchler, die
ihr der Witwen Häuser fresset und wendet lange Gebete vor! Darum werdet
ihr desto mehr Verdammnis empfangen. \bibverse{15} Weh euch,
Schriftgelehrte und Pharisäer, ihr Heuchler, die ihr Land und Wasser
umziehet, dass ihr einen Judengenossen machet; und wenn er's geworden
ist, macht ihr aus ihm ein Kind der Hölle, zwiefältig mehr denn ihr
seid!

\bibverse{16} Weh euch, verblendete Leiter, die ihr sagt: „Wer da
schwört bei dem Tempel, das ist nichts; wer aber schwört bei dem Gold am
Tempel, der ist's schuldig.`` \footnote{\textbf{23:16} Mt 15,4; Mt
  5,34-37} \bibverse{17} Ihr Narren und Blinden! Was ist größer: das
Gold oder der Tempel, der das Gold heiligt? \bibverse{18} „Wer da
schwört bei dem Altar, das ist nichts; wer aber schwört bei dem Opfer,
das darauf ist, der ist's schuldig.`` \bibverse{19} Ihr Narren und
Blinden! Was ist größer: das Opfer oder der Altar, der das Opfer
heiligt? \bibverse{20} Darum, wer da schwört bei dem Altar, der schwört
bei demselben und bei allem, was darauf ist. \bibverse{21} Und wer da
schwört bei dem Tempel, der schwört bei demselben und bei dem, der darin
wohnt. \bibverse{22} Und wer da schwört bei dem Himmel, der schwört bei
dem Stuhl Gottes und bei dem, der darauf sitzt.

\bibverse{23} Weh euch, Schriftgelehrte und Pharisäer, ihr Heuchler, die
ihr verzehntet die Minze, Dill und Kümmel, und lasset dahinten das
Schwerste im Gesetz, nämlich das Gericht, die Barmherzigkeit und den
Glauben! Dies sollte man tun und jenes nicht lassen. \footnote{\textbf{23:23}
  3Mo 27,30; Mi 6,8; Lk 18,12} \bibverse{24} Ihr verblendeten Leiter,
die ihr Mücken seihet und Kamele verschluckt!

\bibverse{25} Weh euch, Schriftgelehrte und Pharisäer, ihr Heuchler, die
ihr die Becher und Schüsseln auswendig reinlich haltet, inwendig aber
ist's voll Raubes und Fraßes! \bibverse{26} Du blinder Pharisäer,
reinige zum ersten das Inwendige an Becher und Schüssel, auf dass auch
das Auswendige rein werde! \footnote{\textbf{23:26} Joh 9,40; Tit 1,15}

\bibverse{27} Weh euch, Schriftgelehrte und Pharisäer, ihr Heuchler, die
ihr gleich seid wie die übertünchten Gräber, welche auswendig hübsch
scheinen, aber inwendig sind sie voller Totengebeine und alles Unflats!
\bibverse{28} Also auch ihr: von außen scheinet ihr vor den Menschen
fromm, aber inwendig seid ihr voller Heuchelei und Untugend.

\bibverse{29} Weh euch, Schriftgelehrte und Pharisäer, ihr Heuchler, die
ihr der Propheten Gräber bauet und schmücket der Gerechten Gräber
\bibverse{30} und sprecht: Wären wir zu unserer Väter Zeiten gewesen, so
wollten wir nicht teilhaftig sein mit ihnen an der Propheten Blut!
\bibverse{31} So gebt ihr über euch selbst Zeugnis, dass ihr Kinder seid
derer, die die Propheten getötet haben. \bibverse{32} Wohlan, erfüllet
auch ihr das Maß eurer Väter! \bibverse{33} Ihr Schlangen, ihr
Otterngezüchte! wie wollt ihr der höllischen Verdammnis entrinnen?
\bibverse{34} Darum siehe, ich sende zu euch Propheten und Weise und
Schriftgelehrte; und deren werdet ihr etliche töten und kreuzigen, und
etliche werdet ihr geißeln in euren Schulen und werdet sie verfolgen von
einer Stadt zu der anderen; \bibverse{35} auf dass über euch komme all
das gerechte Blut, das vergossen ist auf Erden, von dem Blut des
gerechten Abel an bis auf das Blut des Zacharias, des Sohnes Berechjas,
welchen ihr getötet habt zwischen dem Tempel und dem Altar. \footnote{\textbf{23:35}
  1Mo 4,8; 2Chr 24,20-21} \bibverse{36} Wahrlich ich sage euch, dass
solches alles wird über dies Geschlecht kommen.

\bibverse{37} Jerusalem, Jerusalem, die du tötest die Propheten und
steinigst, die zu dir gesandt sind! wie oft habe ich deine Kinder
versammeln wollen, wie eine Henne versammelt ihre Küchlein unter ihre
Flügel; und ihr habt nicht gewollt! \bibverse{38} Siehe, euer Haus soll
euch wüst gelassen werden. \bibverse{39} Denn ich sage euch: Ihr werdet
mich von jetzt an nicht sehen, bis ihr sprecht: Gelobt sei, der da kommt
im Namen des Herrn! \footnote{\textbf{23:39} Mt 21,9; Mt 26,64}

\hypertarget{section-12}{%
\section{24}\label{section-12}}

\bibverse{1} Und Jesus ging hinweg von dem Tempel, und seine Jünger
traten zu ihm, dass sie ihm zeigten des Tempels Gebäude. \bibverse{2}
Jesus aber sprach zu ihnen: Sehet ihr nicht das alles? Wahrlich, ich
sage euch: Es wird hier nicht ein Stein auf dem anderen bleiben, der
nicht zerbrochen werde.

\bibverse{3} Und als er auf dem Ölberge saß, traten zu ihm seine Jünger
besonders und sprachen: Sage uns, wann wird das geschehen? Und welches
wird das Zeichen sein deiner Zukunft und des Endes der Welt? \footnote{\textbf{24:3}
  Apg 1,6-8}

\bibverse{4} Jesus aber antwortete und sprach zu ihnen: Sehet zu, dass
euch nicht jemand verführe. \bibverse{5} Denn es werden viele kommen
unter meinem Namen, und sagen: „Ich bin Christus`` und werden viele
verführen. \bibverse{6} Ihr werdet hören Kriege und Geschrei von
Kriegen; sehet zu und erschrecket nicht. Das muss zum ersten alles
geschehen; aber es ist noch nicht das Ende da. \bibverse{7} Denn es wird
sich empören ein Volk wider das andere und ein Königreich wider das
andere, und werden sein Pestilenz und teure Zeit und Erdbeben hin und
wieder. \bibverse{8} Da wird sich allererst die Not anheben.

\bibverse{9} Alsdann werden sie euch überantworten in Trübsal und werden
euch töten. Und ihr müsset gehasst werden um meines Namens willen von
allen Völkern. \footnote{\textbf{24:9} Joh 16,2; Mt 10,17-22}
\bibverse{10} Dann werden sich viele ärgern und werden sich
untereinander verraten und werden sich untereinander hassen.
\bibverse{11} Und es werden sich viel falsche Propheten erheben und
werden viele verführen. \bibverse{12} Und dieweil die Ungerechtigkeit
wird überhandnehmen, wird die Liebe in vielen erkalten. \footnote{\textbf{24:12}
  2Tim 3,1-5} \bibverse{13} Wer aber beharret bis ans Ende, der wird
selig. \footnote{\textbf{24:13} Offb 13,10} \bibverse{14} Und es wird
gepredigt werden das Evangelium vom Reich in der ganzen Welt zu einem
Zeugnis über alle Völker, und dann wird das Ende kommen. \footnote{\textbf{24:14}
  Mt 28,19}

\bibverse{15} Wenn ihr nun sehen werdet den Gräuel der Verwüstung (davon
gesagt ist durch den Propheten Daniel), dass er steht an der heiligen
Stätte (wer das liest, der merke darauf!), \bibverse{16} alsdann fliehe
auf die Berge, wer im jüdischen Lande ist; \bibverse{17} und wer auf dem
Dach ist, der steige nicht hernieder, etwas aus seinem Hause zu holen;
\bibverse{18} und wer auf dem Felde ist, der kehre nicht um, seine
Kleider zu holen. \bibverse{19} Weh aber den Schwangeren und Säugerinnen
zu der Zeit! \footnote{\textbf{24:19} Lk 23,29} \bibverse{20} Bittet
aber, dass eure Flucht nicht geschehe im Winter oder am Sabbat.
\bibverse{21} Denn es wird alsdann eine große Trübsal sein, wie nicht
gewesen ist von Anfang der Welt bisher und wie auch nicht werden wird.
\bibverse{22} Und wo diese Tage nicht würden verkürzt, so würde kein
Mensch selig; aber um der Auserwählten willen werden die Tage verkürzt.

\bibverse{23} So alsdann jemand zu euch wird sagen: Siehe, hier ist
Christus! oder: da! so sollt ihr's nicht glauben. \bibverse{24} Denn es
werden falsche Christi und falsche Propheten aufstehen und große Zeichen
und Wunder tun, dass verführt werden in den Irrtum (wo es möglich wäre)
auch die Auserwählten. \footnote{\textbf{24:24} 5Mo 13,2-4; 2Thes 2,8-9;
  Offb 13,13}

\bibverse{25} Siehe, ich habe es euch zuvor gesagt.

\bibverse{26} Darum, wenn sie zu euch sagen werden: Siehe, er ist in der
Wüste! so gehet nicht hinaus, -- siehe, er ist in der Kammer! so glaubt
nicht. \bibverse{27} Denn gleichwie der Blitz ausgeht vom Aufgang und
scheint bis zum Niedergang, also wird auch sein die Zukunft des
Menschensohnes. \bibverse{28} Wo aber ein Aas ist, da sammeln sich die
Adler. \footnote{\textbf{24:28} Hi 39,30; Lk 17,37; Offb 19,17-18}

\bibverse{29} Bald aber nach der Trübsal derselben Zeit werden Sonne und
Mond den Schein verlieren, und die Sterne werden vom Himmel fallen, und
die Kräfte der Himmel werden sich bewegen. \footnote{\textbf{24:29} Jes
  13,10; 2Petr 3,10; Offb 6,12-13} \bibverse{30} Und alsdann wird
erscheinen das Zeichen des Menschensohns am Himmel. Und alsdann werden
heulen alle Geschlechter auf Erden und werden sehen kommen des Menschen
Sohn in den Wolken des Himmels mit großer Kraft und Herrlichkeit.
\footnote{\textbf{24:30} Mt 26,64; Dan 7,13-14; Offb 1,7; Offb 19,11-13}
\bibverse{31} Und er wird senden seine Engel mit hellen Posaunen, und
sie werden sammeln seine Auserwählten von den vier Winden, von einem
Ende des Himmels zu dem anderen. \footnote{\textbf{24:31} 1Kor 15,52;
  Offb 8,1-2}

\bibverse{32} An dem Feigenbaum lernet ein Gleichnis: wenn sein Zweig
jetzt saftig wird und Blätter gewinnt, so wisst ihr, dass der Sommer
nahe ist. \bibverse{33} Also auch wenn ihr das alles sehet, so wisset,
dass es nahe vor der Tür ist. \bibverse{34} Wahrlich ich sage euch: Dies
Geschlecht wird nicht vergehen, bis dass dieses alles geschehe.
\bibverse{35} Himmel und Erde werden vergehen; aber meine Worte werden
nicht vergehen. \footnote{\textbf{24:35} Mt 5,18}

\bibverse{36} Von dem Tage aber und von der Stunde weiß niemand, auch
die Engel nicht im Himmel, sondern allein mein Vater. \footnote{\textbf{24:36}
  Apg 1,7} \bibverse{37} Aber gleichwie es zu der Zeit Noahs war, also
wird auch sein die Zukunft des Menschensohnes. \footnote{\textbf{24:37}
  1Mo 6,11-13; Lk 17,26-27} \bibverse{38} Denn gleichwie sie waren in
den Tagen vor der Sintflut -- sie aßen, sie tranken, sie freiten und
ließen sich freien, bis an den Tag, da Noah zu der Arche einging;
\bibverse{39} und sie achteten's nicht, bis die Sintflut kam und nahm
sie alle dahin --, also wird auch sein die Zukunft des Menschensohnes.
\bibverse{40} Dann werden zwei auf dem Felde sein; einer wird
angenommen, und der andere wird verlassen werden. \bibverse{41} Zwei
werden mahlen auf der Mühle; eine wird angenommen, und die andere wird
verlassen werden. \bibverse{42} Darum wachet, denn ihr wisset nicht,
welche Stunde euer Herr kommen wird. \footnote{\textbf{24:42} Mt 25,13}
\bibverse{43} Das sollt ihr aber wissen: Wenn ein Hausvater wüsste,
welche Stunde der Dieb kommen wollte, so würde er ja wachen und nicht in
sein Haus brechen lassen. \bibverse{44} Darum seid ihr auch bereit; denn
des Menschen Sohn wird kommen zu einer Stunde, da ihr's nicht meinet.

\bibverse{45} Welcher ist aber nun ein treuer und kluger Knecht, den der
Herr gesetzt hat über sein Gesinde, dass er ihnen zu rechter Zeit Speise
gebe? \bibverse{46} Selig ist der Knecht, wenn sein Herr kommt und
findet ihn also tun. \bibverse{47} Wahrlich ich sage euch: Er wird ihn
über alle seine Güter setzen. \footnote{\textbf{24:47} Mt 25,21; Mt
  25,23} \bibverse{48} So aber jener, der böse Knecht, wird in seinem
Herzen sagen: Mein Herr kommt noch lange nicht, -- \footnote{\textbf{24:48}
  2Petr 3,4} \bibverse{49} und fängt an zu schlagen seine Mitknechte,
isst und trinkt mit den Trunkenen: \bibverse{50} so wird der Herr des
Knechtes kommen an dem Tage, des er sich nicht versieht, und zu der
Stunde, die er nicht meint, \bibverse{51} und wird ihn zerscheitern und
wird ihm seinen Lohn geben mit den Heuchlern; da wird sein Heulen und
Zähneklappen. \# 25 \bibverse{1} Dann wird das Himmelreich gleich sein
zehn Jungfrauen, die ihre Lampen nahmen und gingen aus, dem Bräutigam
entgegen. \footnote{\textbf{25:1} Lk 12,35-36; Offb 19,7} \bibverse{2}
Aber fünf unter ihnen waren töricht, und fünf waren klug. \bibverse{3}
Die törichten nahmen ihre Lampen; aber sie nahmen nicht Öl mit sich.
\bibverse{4} Die klugen aber nahmen Öl in ihren Gefäßen samt ihren
Lampen. \bibverse{5} Da nun der Bräutigam verzog, wurden sie alle
schläfrig und schliefen ein. \bibverse{6} Zur Mitternacht aber ward ein
Geschrei: Siehe, der Bräutigam kommt; gehet aus, ihm entgegen!
\bibverse{7} Da standen diese Jungfrauen alle auf und schmückten ihre
Lampen. \bibverse{8} Die törichten aber sprachen zu den klugen: Gebt uns
von eurem Öl, denn unsere Lampen verlöschen. \bibverse{9} Da antworteten
die klugen und sprachen: Nicht also, auf dass nicht uns und euch
gebreche; gehet aber hin zu den Krämern und kaufet für euch selbst.
\bibverse{10} Und da sie hingingen, zu kaufen, kam der Bräutigam; und
die bereit waren, gingen mit ihm hinein zur Hochzeit, und die Tür ward
verschlossen. \bibverse{11} Zuletzt kamen auch die anderen Jungfrauen
und sprachen: Herr, Herr, tu uns auf! \bibverse{12} Er antwortete aber
und sprach: Wahrlich ich sage euch: Ich kenne euch nicht. \footnote{\textbf{25:12}
  Mt 7,23} \bibverse{13} Darum wachet; denn ihr wisset weder Tag noch
Stunde, in welcher des Menschen Sohn kommen wird. \footnote{\textbf{25:13}
  Mt 24,42}

\bibverse{14} Gleichwie ein Mensch, der über Land zog, rief seine
Knechte und tat ihnen seine Güter aus; \bibverse{15} und einem gab er
fünf Zentner, dem anderen zwei, dem dritten einen, einem jeden nach
seinem Vermögen, und zog bald hinweg. \footnote{\textbf{25:15} Röm 12,6}
\bibverse{16} Da ging der hin, der fünf Zentner empfangen hatte, und
handelte mit ihnen und gewann andere fünf Zentner. \bibverse{17}
Desgleichen, der zwei Zentner empfangen hatte, gewann auch zwei andere.
\bibverse{18} Der aber einen empfangen hatte, ging hin und machte eine
Grube in die Erde und verbarg seines Herrn Geld.

\bibverse{19} Über eine lange Zeit kam der Herr dieser Knechte und hielt
Rechenschaft mit ihnen. \bibverse{20} Da trat herzu, der fünf Zentner
empfangen hatte, und legte andere fünf Zentner dar und sprach: Herr, du
hast mir fünf Zentner ausgetan; siehe da, ich habe damit andere fünf
Zentner gewonnen.

\bibverse{21} Da sprach sein Herr zu ihm: Ei, du frommer und getreuer
Knecht, du bist über wenigem getreu gewesen; ich will dich über viel
setzen; gehe ein zu deines Herrn Freude!

\bibverse{22} Da trat auch herzu, der zwei Zentner empfangen hatte, und
sprach: Herr, du hast mir zwei Zentner ausgetan; siehe da, ich habe mit
ihnen zwei andere gewonnen.

\bibverse{23} Sein Herr sprach zu ihm: Ei, du frommer und getreuer
Knecht, du bist über wenigem getreu gewesen, ich will dich über viel
setzen; gehe ein zu deines Herrn Freude!

\bibverse{24} Da trat auch herzu, der einen Zentner empfangen hatte, und
sprach: Herr, ich wusste, dass du ein harter Mann bist: du schneidest,
wo du nicht gesät hast, und sammelst, wo du nicht gestreut hast;
\bibverse{25} und fürchtete mich, ging hin und verbarg deinen Zentner in
die Erde. Siehe, da hast du das Deine.

\bibverse{26} Sein Herr aber antwortete und sprach zu ihm: Du Schalk und
fauler Knecht! wusstest du, dass ich schneide, wo ich nicht gesät habe,
und sammle, wo ich nicht gestreut habe? \bibverse{27} so solltest du
mein Geld zu den Wechslern getan haben, und wenn ich gekommen wäre,
hätte ich das Meine zu mir genommen mit Zinsen. \bibverse{28} Darum
nehmet von ihm den Zentner und gebt es dem, der zehn Zentner hat.
\bibverse{29} Denn wer da hat, dem wird gegeben werden, und er wird die
Fülle haben; wer aber nicht hat, dem wird auch, was er hat, genommen
werden. \footnote{\textbf{25:29} Mt 13,12} \bibverse{30} Und den
unnützen Knecht werft in die Finsternis hinaus; da wird sein Heulen und
Zähneklappen.

\bibverse{31} Wenn aber des Menschen Sohn kommen wird in seiner
Herrlichkeit und alle heiligen Engel mit ihm, dann wird er sitzen auf
dem Stuhl seiner Herrlichkeit, \bibverse{32} und werden vor ihm alle
Völker versammelt werden. Und er wird sie voneinander scheiden, gleich
als ein Hirte die Schafe von den Böcken scheidet, \footnote{\textbf{25:32}
  Mt 13,49; Röm 14,10} \bibverse{33} und wird die Schafe zu seiner
Rechten stellen und die Böcke zur Linken. \footnote{\textbf{25:33} Hes
  34,17} \bibverse{34} Da wird dann der König sagen zu denen zu seiner
Rechten: Kommt her, ihr Gesegneten meines Vaters, ererbet das Reich, das
euch bereitet ist von Anbeginn der Welt! \bibverse{35} Denn ich bin
hungrig gewesen, und ihr habt mich gespeist. Ich bin durstig gewesen,
und ihr habt mich getränkt. Ich bin ein Gast gewesen, und ihr habt mich
beherbergt. \footnote{\textbf{25:35} Jes 58,7} \bibverse{36} Ich bin
nackt gewesen und ihr habt mich bekleidet. Ich bin krank gewesen, und
ihr habt mich besucht. Ich bin gefangen gewesen, und ihr seid zu mir
gekommen.

\bibverse{37} Dann werden ihm die Gerechten antworten und sagen: Herr,
wann haben wir dich hungrig gesehen und haben dich gespeist? oder
durstig und haben dich getränkt? \bibverse{38} Wann haben wir dich als
einen Gast gesehen und beherbergt? oder nackt und dich bekleidet?
\bibverse{39} Wann haben wir dich krank oder gefangen gesehen und sind
zu dir gekommen?

\bibverse{40} Und der König wird antworten und sagen zu ihnen: Wahrlich
ich sage euch: Was ihr getan habt einem unter diesen meinen geringsten
Brüdern, das habt ihr mir getan. \footnote{\textbf{25:40} Mt 10,42; Spr
  19,17; Hebr 2,11} \bibverse{41} Dann wird er auch sagen zu denen zur
Linken: Gehet hin von mir, ihr Verfluchten, in das ewige Feuer, das
bereitet ist dem Teufel und seinen Engeln! \footnote{\textbf{25:41} Offb
  20,10; Offb 20,15} \bibverse{42} Ich bin hungrig gewesen, und ihr habt
mich nicht gespeist. Ich bin durstig gewesen, und ihr habt mich nicht
getränkt. \bibverse{43} Ich bin ein Gast gewesen, und ihr habt mich
nicht beherbergt. Ich bin nackt gewesen, und ihr habt mich nicht
bekleidet. Ich bin krank und gefangen gewesen, und ihr habt mich nicht
besucht.

\bibverse{44} Da werden sie ihm auch antworten und sagen: Herr, wann
haben wir dich gesehen hungrig oder durstig oder als einen Gast oder
nackt oder krank oder gefangen und haben dir nicht gedient?

\bibverse{45} Dann wird er ihnen antworten und sagen: Wahrlich ich sage
euch: Was ihr nicht getan habt einem unter diesen Geringsten, das habt
ihr mir auch nicht getan. \bibverse{46} Und sie werden in die ewige Pein
gehen, aber die Gerechten in das ewige Leben. \footnote{\textbf{25:46}
  Joh 5,29; Jak 2,13}

\hypertarget{section-13}{%
\section{26}\label{section-13}}

\bibverse{1} Und es begab sich, da Jesus alle diese Reden vollendet
hatte, sprach er zu seinen Jüngern: \bibverse{2} Ihr wisset, dass nach
zwei Tagen Ostern wird; und des Menschen Sohn wird überantwortet werden,
dass er gekreuzigt werde.

\bibverse{3} Da versammelten sich die Hohenpriester und Schriftgelehrten
und die Ältesten im Volk in den Palast des Hohenpriesters, der da hieß
Kaiphas, \footnote{\textbf{26:3} Lk 3,1-2} \bibverse{4} und hielten Rat,
wie sie Jesum mit List griffen und töteten. \bibverse{5} Sie sprachen
aber: Ja nicht auf das Fest, auf dass nicht ein Aufruhr werde im Volk!

\bibverse{6} Da nun Jesus war zu Bethanien im Hause Simons, des
Aussätzigen, \bibverse{7} da trat zu ihm ein Weib, das hatte ein Glas
mit köstlichem Wasser und goss es auf sein Haupt, da er zu Tische saß.
\bibverse{8} Da das seine Jünger sahen, wurden sie unwillig und
sprachen: Wozu dient diese Vergeudung? \bibverse{9} Dieses Wasser hätte
mögen teuer verkauft und den Armen gegeben werden.

\bibverse{10} Da das Jesus merkte, sprach er zu ihnen: Was bekümmert ihr
das Weib? Sie hat ein gutes Werk an mir getan. \bibverse{11} Ihr habt
allezeit Arme bei euch; mich aber habt ihr nicht allezeit. \bibverse{12}
Dass sie dies Wasser hat auf meinen Leib gegossen, hat sie getan, dass
sie mich zum Grabe bereite. \bibverse{13} Wahrlich ich sage euch: Wo
dies Evangelium gepredigt wird in der ganzen Welt, da wird man auch
sagen zu ihrem Gedächtnis, was sie getan hat.

\bibverse{14} Da ging hin der Zwölf einer, mit Namen Judas Ischariot, zu
den Hohenpriestern \bibverse{15} und sprach: Was wollt ihr mir geben?
Ich will ihn euch verraten. Und sie boten ihm dreißig Silberlinge.
\footnote{\textbf{26:15} Joh 11,57; Sach 11,12} \bibverse{16} Und von
dem an suchte er Gelegenheit, dass er ihn verriete.

\bibverse{17} Aber am ersten Tage der süßen Brote traten die Jünger zu
Jesu und sprachen zu ihm: Wo willst du, dass wir dir bereiten das
Osterlamm zu essen?

\bibverse{18} Er sprach: Gehet hin in die Stadt zu einem und sprecht zu
ihm: Der Meister lässt dir sagen: Meine Zeit ist nahe; ich will bei dir
Ostern halten mit meinen Jüngern. \footnote{\textbf{26:18} Mt 21,3}

\bibverse{19} Und die Jünger taten, wie ihnen Jesus befohlen hatte, und
bereiteten das Osterlamm.

\bibverse{20} Und am Abend setzte er sich zu Tische mit den Zwölfen.
\bibverse{21} Und da sie aßen, sprach er: Wahrlich ich sage euch: Einer
unter euch wird mich verraten.

\bibverse{22} Und sie wurden sehr betrübt und hoben an, ein jeglicher
unter ihnen, und sagten zu ihm: Herr, bin ich's?

\bibverse{23} Er antwortete und sprach: Der mit der Hand mit mir in die
Schüssel tauchte, der wird mich verraten. \bibverse{24} Des Menschen
Sohn geht zwar dahin, wie von ihm geschrieben steht; doch weh dem
Menschen, durch welchen des Menschen Sohn verraten wird! Es wäre ihm
besser, dass er nie geboren wäre.

\bibverse{25} Da antwortete Judas, der ihn verriet, und sprach: Bin
ich's, Rabbi? Er sprach zu ihm: Du sagst es.

\bibverse{26} Da sie aber aßen, nahm Jesus das Brot, dankte und brach's
und gab's den Jüngern und sprach: Nehmet, esset; das ist mein Leib.
\footnote{\textbf{26:26} 1Kor 10,16; 1Kor 11,23-25}

\bibverse{27} Und er nahm den Kelch und dankte, gab ihnen den und
sprach: Trinket alle daraus; \bibverse{28} das ist mein Blut des neuen
Testaments, welches vergossen wird für viele zur Vergebung der Sünden.
\bibverse{29} Ich sage euch: Ich werde von nun an nicht mehr von diesem
Gewächs des Weinstocks trinken bis an den Tag, da ich's neu trinken
werde mit euch in meines Vaters Reich.

\bibverse{30} Und da sie den Lobgesang gesprochen hatten, gingen sie
hinaus an den Ölberg. \footnote{\textbf{26:30} Ps 113,1-118}

\bibverse{31} Da sprach Jesus zu ihnen: In dieser Nacht werdet ihr euch
alle ärgern an mir. Denn es steht geschrieben: „Ich werde den Hirten
schlagen, und die Schafe der Herde werden sich zerstreuen.`` \footnote{\textbf{26:31}
  Joh 16,32} \bibverse{32} Wenn ich aber auferstehe, will ich vor euch
hingehen nach Galiläa. \footnote{\textbf{26:32} Mt 28,7}

\bibverse{33} Petrus aber antwortete und sprach zu ihm: Wenn sie auch
alle sich an dir ärgerten, so will ich doch mich nimmermehr ärgern.

\bibverse{34} Jesus sprach zu ihm: Wahrlich ich sage dir: In dieser
Nacht, ehe der Hahn kräht, wirst du mich dreimal verleugnen.

\bibverse{35} Petrus sprach zu ihm: Und wenn ich mit dir sterben müsste,
so will ich dich nicht verleugnen. Desgleichen sagten auch alle Jünger.

\bibverse{36} Da kam Jesus mit ihnen zu einem Hofe, der hieß Gethsemane,
und sprach zu seinen Jüngern: Setzet euch hier, bis dass ich dorthin
gehe und bete. \bibverse{37} Und nahm zu sich Petrus und die zwei Söhne
des Zebedäus und fing an zu trauern und zu zagen. \footnote{\textbf{26:37}
  Mt 17,1; Hebr 5,7} \bibverse{38} Da sprach Jesus zu ihnen: Meine Seele
ist betrübt bis an den Tod; bleibet hier und wachet mit mir! \footnote{\textbf{26:38}
  Joh 12,27}

\bibverse{39} Und ging hin ein wenig, fiel nieder auf sein Angesicht und
betete und sprach: Mein Vater, ist's möglich, so gehe dieser Kelch von
mir; doch nicht, wie ich will, sondern wie du willst! \footnote{\textbf{26:39}
  Joh 6,38; Joh 18,11; Hebr 5,8}

\bibverse{40} Und er kam zu seinen Jüngern und fand sie schlafend und
sprach zu Petrus: Könnet ihr denn nicht eine Stunde mit mir wachen?
\bibverse{41} Wachet und betet, dass ihr nicht in Anfechtung fallet! Der
Geist ist willig; aber das Fleisch ist schwach.

\bibverse{42} Zum andernmal ging er wieder hin, betete und sprach: Mein
Vater, ist's nicht möglich, dass dieser Kelch von mir gehe, ich trinke
ihn denn, so geschehe dein Wille!

\bibverse{43} Und er kam und fand sie abermals schlafend, und ihre Augen
waren voll Schlafs. \bibverse{44} Und er ließ sie und ging abermals hin
und betete zum drittenmal und redete dieselben Worte. \footnote{\textbf{26:44}
  2Kor 12,8} \bibverse{45} Da kam er zu seinen Jüngern und sprach zu
ihnen: Ach wollt ihr nun schlafen und ruhen? Siehe, die Stunde ist hier,
dass des Menschen Sohn in der Sünder Hände überantwortet wird.
\bibverse{46} Stehet auf, lasst uns gehen! Siehe, er ist da, der mich
verrät!

\bibverse{47} Und als er noch redete, siehe, da kam Judas, der Zwölf
einer, und mit ihm eine große Schar, mit Schwertern und mit Stangen, von
den Hohenpriestern und Ältesten des Volks. \bibverse{48} Und der
Verräter hatte ihnen ein Zeichen gegeben und gesagt: Welchen ich küssen
werde, der ist's; den greifet. \bibverse{49} Und alsbald trat er zu Jesu
und sprach: Gegrüßet seist du, Rabbi! und küsste ihn.

\bibverse{50} Jesus aber sprach zu ihm: Mein Freund, warum bist du
gekommen? Da traten sie hinzu und legten die Hände an Jesum und griffen
ihn.

\bibverse{51} Und siehe, einer aus denen, die mit Jesu waren, reckte die
Hand aus und zog sein Schwert aus und schlug des Hohenpriesters Knecht
und hieb ihm ein Ohr ab.

\bibverse{52} Da sprach Jesus zu ihm: Stecke dein Schwert an seinen Ort!
denn wer das Schwert nimmt, der soll durchs Schwert umkommen.
\bibverse{53} Oder meinst du, dass ich nicht könnte meinen Vater bitten,
dass er mir zuschickte mehr denn zwölf Legionen Engel? \footnote{\textbf{26:53}
  Mt 4,11} \bibverse{54} Wie würde aber die Schrift erfüllet? Es muss
also gehen.

\bibverse{55} Zu der Stunde sprach Jesus zu den Scharen: Ihr seid
ausgegangen wie zu einem Mörder, mit Schwertern und Stangen, mich zu
fangen. Bin ich doch täglich gesessen bei euch und habe gelehrt im
Tempel, und ihr habt mich nicht gegriffen. \bibverse{56} Aber das ist
alles geschehen, dass erfüllet würden die Schriften der Propheten. Da
verließen ihn alle Jünger und flohen.

\bibverse{57} Die aber Jesum gegriffen hatten, führten ihn zu dem
Hohenpriester Kaiphas, dahin die Schriftgelehrten und Ältesten sich
versammelt hatten.

\bibverse{58} Petrus aber folgte ihm nach von ferne bis in den Palast
des Hohenpriesters und ging hinein und setzte sich zu den Knechten, auf
dass er sähe, wo es hinaus wollte.

\bibverse{59} Die Hohenpriester aber und Ältesten und der ganze Rat
suchten falsch Zeugnis wider Jesum, auf dass sie ihn töteten,
\bibverse{60} und fanden keins. Und wiewohl viel falsche Zeugen
herzutraten, fanden sie doch keins. Zuletzt traten herzu zwei falsche
Zeugen \bibverse{61} und sprachen: Er hat gesagt: Ich kann den Tempel
Gottes abbrechen und in drei Tagen ihn bauen.

\bibverse{62} Und der Hohepriester stand auf und sprach zu ihm:
Antwortest du nichts zu dem, was diese wider dich zeugen? \bibverse{63}
Aber Jesus schwieg still. Und der Hohepriester antwortete und sprach zu
ihm: Ich beschwöre dich bei dem lebendigen Gott, dass du uns sagest, ob
du seist Christus, der Sohn Gottes. \footnote{\textbf{26:63} Mt 27,12;
  Joh 10,24}

\bibverse{64} Jesus sprach zu ihm: Du sagst es. Doch ich sage euch: Von
nun an wird's geschehen, dass ihr sehen werdet des Menschen Sohn sitzen
zur Rechten der Kraft und kommen in den Wolken des Himmels. \footnote{\textbf{26:64}
  Ps 110,1; Mt 16,27; Mt 24,30; 2Kor 13,4}

\bibverse{65} Da zerriss der Hohepriester seine Kleider und sprach: Er
hat Gott gelästert! Was bedürfen wir weiteres Zeugnis? Siehe, jetzt habt
ihr seine Gotteslästerung gehört. \footnote{\textbf{26:65} Joh 10,33}
\bibverse{66} Was dünkt euch? Sie antworteten und sprachen: Er ist des
Todes schuldig! \footnote{\textbf{26:66} Joh 19,7; 3Mo 24,16}

\bibverse{67} Da spien sie aus in sein Angesicht und schlugen ihn mit
Fäusten. Etliche aber schlugen ihn ins Angesicht \footnote{\textbf{26:67}
  Jes 50,6} \bibverse{68} und sprachen: Weissage uns, Christe, wer
ist's, der dich schlug?

\bibverse{69} Petrus aber saß draußen im Hof; und es trat zu ihm eine
Magd und sprach: Und du warst auch mit dem Jesus aus Galiläa.

\bibverse{70} Er leugnete aber vor ihnen allen und sprach: Ich weiß
nicht, was du sagst.

\bibverse{71} Als er aber zur Tür hinausging, sah ihn eine andere und
sprach zu denen, die da waren: Dieser war auch mit dem Jesus von
Nazareth.

\bibverse{72} Und er leugnete abermals und schwur dazu: Ich kenne den
Menschen nicht.

\bibverse{73} Und über eine kleine Weile traten hinzu, die dastanden,
und sprachen zu Petrus: Wahrlich du bist auch einer von denen; denn
deine Sprache verrät dich.

\bibverse{74} Da hob er an sich zu verfluchen und zu schwören: Ich kenne
den Menschen nicht. Uns alsbald krähte der Hahn.

\bibverse{75} Da dachte Petrus an die Worte Jesu, da er zu ihm sagte:
„Ehe der Hahn krähen wird, wirst du mich dreimal verleugnen``, und ging
hinaus und weinte bitterlich. \# 27 \bibverse{1} Des Morgens aber
hielten alle Hohenpriester und die Ältesten des Volks einen Rat über
Jesum, dass sie ihn töteten. \bibverse{2} Und banden ihn, führten ihn
hin und überantworteten ihn dem Landpfleger Pontius Pilatus.

\bibverse{3} Da das sah Judas, der ihn verraten hatte, dass er verdammt
war zum Tode, gereute es ihn, und brachte wieder die dreißig Silberlinge
den Hohenpriestern und den Ältesten \bibverse{4} und sprach: Ich habe
übel getan, dass ich unschuldig Blut verraten habe.

\bibverse{5} Sie sprachen: Was geht uns das an? Da siehe du zu! Und er
warf die Silberlinge in den Tempel, hob sich davon, ging hin und
erhängte sich selbst. \footnote{\textbf{27:5} Apg 1,18-19}

\bibverse{6} Aber die Hohenpriester nahmen die Silberlinge und sprachen:
Es taugt nicht, dass wir sie in den Gotteskasten legen; denn es ist
Blutgeld. \footnote{\textbf{27:6} 5Mo 23,19}

\bibverse{7} Sie hielten aber einen Rat und kauften den Töpfersacker
darum zum Begräbnis der Pilger. \bibverse{8} Daher ist dieser Acker
genannt der Blutacker bis auf den heutigen Tag. \bibverse{9} Da ist
erfüllet, was gesagt ist durch den Propheten Jeremia, da er spricht:
„Sie haben genommen dreißig Silberlinge, damit bezahlt war der
Verkaufte, welchen sie kauften von den Kindern Israel, \bibverse{10} und
haben sie gegeben um den Töpfersacker, wie mir der Herr befohlen hat.``

\bibverse{11} Jesus aber stand vor dem Landpfleger; und der Landpfleger
fragte ihn und sprach: Bist du der Juden König? Jesus aber sprach zu
ihm: Du sagst es.

\bibverse{12} Und da er verklagt ward von den Hohenpriestern und
Ältesten, antwortete er nichts. \footnote{\textbf{27:12} Mt 26,63; Jes
  53,7}

\bibverse{13} Da sprach Pilatus zu ihm: Hörst du nicht, wie hart sie
dich verklagen?

\bibverse{14} Und er antwortete ihm nicht auf ein Wort, also dass sich
auch der Landpfleger sehr verwunderte.

\bibverse{15} Auf das Fest aber hatte der Landpfleger die Gewohnheit,
dem Volk einen Gefangenen loszugeben, welchen sie wollten. \bibverse{16}
Er hatte aber zu der Zeit einen Gefangenen, einen sonderlichen vor
anderen, der hieß Barabbas. \bibverse{17} Und da sie versammelt waren,
sprach Pilatus zu ihnen: Welchen wollt ihr, dass ich euch losgebe?
Barabbas oder Jesus, von dem gesagt wird, er sei Christus? \bibverse{18}
Denn er wusste wohl, dass sie ihn aus Neid überantwortet hatten.
\footnote{\textbf{27:18} Joh 12,19}

\bibverse{19} Und da er auf dem Richtstuhl saß, schickte sein Weib zu
ihm und ließ ihm sagen: Habe du nichts zu schaffen mit diesem Gerechten;
ich habe heute viel erlitten im Traum seinetwegen.

\bibverse{20} Aber die Hohenpriester und die Ältesten überredeten das
Volk, dass sie um Barabbas bitten sollten und Jesum umbrächten.
\bibverse{21} Da antwortete nun der Landpfleger und sprach zu ihnen:
Welchen wollt ihr unter diesen zweien, den ich euch soll losgeben? Sie
sprachen: Barabbas.

\bibverse{22} Pilatus sprach zu ihnen: Was soll ich denn machen mit
Jesus, von dem gesagt wird, er sei Christus? Sie sprachen alle: Lass ihn
kreuzigen!

\bibverse{23} Der Landpfleger sagte: Was hat er denn Übles getan? Sie
schrien aber noch mehr und sprachen: Lass ihn kreuzigen!

\bibverse{24} Da aber Pilatus sah, dass er nichts schaffte, sondern dass
ein viel größer Getümmel ward, nahm er Wasser und wusch die Hände vor
dem Volk und sprach: Ich bin unschuldig an dem Blut dieses Gerechten;
sehet ihr zu!

\bibverse{25} Da antwortete das ganze Volk und sprach: Sein Blut komme
über uns und unsere Kinder. \footnote{\textbf{27:25} Apg 5,28}

\bibverse{26} Da gab er ihnen Barabbas los; aber Jesum ließ er geißeln
und überantwortete ihn, dass er gekreuzigt würde.

\bibverse{27} Da nahmen die Kriegsknechte des Landpflegers Jesum zu sich
in das Richthaus und sammelten über ihn die ganze Schar

\bibverse{28} und zogen ihn aus und legten ihm einen Purpurmantel an

\bibverse{29} und flochten eine Dornenkrone und setzten sie auf sein
Haupt und ein Rohr in seine rechte Hand und beugten die Knie vor ihm und
verspotteten ihn und sprachen: Gegrüßet seist du, der Juden König!

\bibverse{30} und spien ihn an und nahmen das Rohr und schlugen damit
sein Haupt. \bibverse{31} Und da sie ihn verspottet hatten, zogen sie
ihm den Mantel aus und zogen ihm seine Kleider an und führten ihn hin,
dass sie ihn kreuzigten.

\bibverse{32} Und indem sie hinausgingen, fanden sie einen Menschen von
Kyrene mit Namen Simon; den zwangen sie, dass er ihm sein Kreuz trug.
\bibverse{33} Und da sie an die Stätte kamen mit Namen Golgatha, das ist
verdeutscht: Schädelstätte, \bibverse{34} gaben sie ihm Essig zu trinken
mit Galle vermischt; und da er's schmeckte, wollte er nicht trinken.
\footnote{\textbf{27:34} Ps 69,22} \bibverse{35} Da sie ihn aber
gekreuzigt hatten, teilten sie seine Kleider und warfen das Los darum,
auf dass erfüllet würde, was gesagt ist durch den Propheten: „Sie haben
meine Kleider unter sich geteilt, und über mein Gewand haben sie das Los
geworfen.`` \footnote{\textbf{27:35} Joh 19,24} \bibverse{36} Und sie
saßen allda und hüteten sein. \bibverse{37} Und oben zu seinen Häupten
setzten sie die Ursache seines Todes, und war geschrieben: Dies ist
Jesus, der Juden König.

\bibverse{38} Und da wurden zwei Mörder mit ihm gekreuzigt, einer zur
Rechten und einer zur Linken. \footnote{\textbf{27:38} Jes 53,12}

\bibverse{39} Die aber vorübergingen, lästerten ihn und schüttelten ihre
Köpfe \footnote{\textbf{27:39} Ps 22,8} \bibverse{40} und sprachen: Der
du den Tempel Gottes zerbrichst und baust ihn in drei Tagen, hilf dir
selber! Bist du Gottes Sohn, so steig herab vom Kreuz! \footnote{\textbf{27:40}
  Mt 26,61; Joh 2,19}

\bibverse{41} Desgleichen auch die Hohenpriester spotteten sein samt den
Schriftgelehrten und Ältesten und sprachen: \bibverse{42} Anderen hat er
geholfen, und kann sich selber nicht helfen. Ist er der König Israels,
so steige er nun vom Kreuz, so wollen wir ihm glauben. \bibverse{43} Er
hat Gott vertraut; der erlöse ihn nun, hat er Lust zu ihm; denn er hat
gesagt: Ich bin Gottes Sohn. \bibverse{44} Desgleichen schmähten ihn
auch die Mörder, die mit ihm gekreuzigt waren.

\bibverse{45} Und von der sechsten Stunde an ward eine Finsternis über
das ganze Land bis zu der neunten Stunde. \bibverse{46} Und um die
neunte Stunde schrie Jesus laut und sprach: Eli, Eli, lama asabthani?
das heißt: Mein Gott, mein Gott, warum hast du mich verlassen?
\footnote{\textbf{27:46} Ps 22,2}

\bibverse{47} Etliche aber, die dastanden, da sie das hörten, sprachen
sie: Der ruft den Elia.

\bibverse{48} Und alsbald lief einer unter ihnen, nahm einen Schwamm und
füllte ihn mit Essig und steckte ihn auf ein Rohr und tränkte ihn.
\bibverse{49} Die anderen aber sprachen: Halt, lass sehen, ob Elia komme
und ihm helfe.

\bibverse{50} Aber Jesus schrie abermals laut und verschied.

\bibverse{51} Und siehe da, der Vorhang im Tempel zerriss in zwei Stücke
von obenan bis untenaus. \footnote{\textbf{27:51} 2Mo 26,31}
\bibverse{52} Und die Erde erbebte, und die Felsen zerrissen, und die
Gräber taten sich auf, und standen auf viele Leiber der Heiligen, die da
schliefen, \bibverse{53} und gingen aus den Gräbern nach seiner
Auferstehung und kamen in die heilige Stadt und erschienen vielen.

\bibverse{54} Aber der Hauptmann und die bei ihm waren und bewahrten
Jesum, da sie sahen das Erdbeben und was da geschah, erschraken sie sehr
und sprachen: Wahrlich, dieser ist Gottes Sohn gewesen!

\bibverse{55} Und es waren viele Weiber da, die von ferne zusahen, die
da Jesu waren nachgefolgt aus Galiläa und hatten ihm gedient;
\bibverse{56} unter welchen war Maria Magdalena und Maria, die Mutter
des Jakobus und Joses, und die Mutter der Kinder des Zebedäus.

\bibverse{57} Am Abend aber kam ein reicher Mann von Arimathia, der hieß
Joseph, welcher auch ein Jünger Jesu war. \footnote{\textbf{27:57} 5Mo
  21,22-23} \bibverse{58} Der ging zu Pilatus und bat ihn um den Leib
Jesu. Da befahl Pilatus man sollte ihm ihn geben. \bibverse{59} Und
Joseph nahm den Leib und wickelte ihn in eine reine Leinwand
\bibverse{60} und legte ihn in sein eigenes neues Grab, welches er hatte
lassen in einen Fels hauen, und wälzte einen großen Stein vor die Tür
des Grabes und ging davon. \bibverse{61} Es war aber allda Maria
Magdalena und die andere Maria, die setzten sich gegen das Grab.

\bibverse{62} Des anderen Tages, der da folgt nach dem Rüsttage, kamen
die Hohenpriester und Pharisäer sämtlich zu Pilatus \bibverse{63} und
sprachen: Herr, wir haben gedacht, dass dieser Verführer sprach, da er
noch lebte: Ich will nach drei Tagen auferstehen. \footnote{\textbf{27:63}
  Mt 20,19; 2Kor 6,8} \bibverse{64} Darum befiehl, dass man das Grab
verwahre bis an den dritten Tag, auf dass nicht seine Jünger kommen und
stehlen ihn und sagen zum Volk: Er ist auferstanden von den Toten, --
und werde der letzte Betrug ärger denn der erste.

\bibverse{65} Pilatus sprach zu ihnen: Da habt ihr die Hüter; gehet hin
und verwahret, wie ihr wisset. \bibverse{66} Sie gingen hin und
verwahrten das Grab mit Hütern und versiegelten den Stein. \# 28
\bibverse{1} Als aber der Sabbat um war und der erste Tag der Woche
anbrach, kam Maria Magdalena und die andere Maria, das Grab zu besehen.
\bibverse{2} Und siehe, es geschah ein großes Erdbeben. Denn der Engel
des Herrn kam vom Himmel herab, trat hinzu und wälzte den Stein von der
Tür und setzte sich darauf. \bibverse{3} Und seine Gestalt war wie der
Blitz und sein Kleid weiß wie Schnee. \footnote{\textbf{28:3} Mt 17,2;
  Apg 1,10} \bibverse{4} Die Hüter aber erschraken vor Furcht und
wurden, als wären sie tot. \bibverse{5} Aber der Engel antwortete und
sprach zu den Weibern: Fürchtet euch nicht! Ich weiß, dass ihr Jesum,
den Gekreuzigten, suchet. \bibverse{6} Er ist nicht hier; er ist
auferstanden, wie er gesagt hat. Kommet her und sehet die Stätte, da der
Herr gelegen hat. \bibverse{7} Und gehet eilend hin und saget es seinen
Jüngern, dass er auferstanden sei von den Toten. Und siehe, er wird vor
euch hingehen nach Galiläa; da werdet ihr ihn sehen. Siehe, ich habe es
euch gesagt. \footnote{\textbf{28:7} Mt 26,32}

\bibverse{8} Und sie gingen eilend zum Grabe hinaus mit Furcht und
großer Freude und liefen, dass sie es seinen Jüngern verkündigten. Und
da sie gingen seinen Jüngern zu verkündigen, \bibverse{9} siehe, da
begegnete ihnen Jesus und sprach: Seid gegrüßet! Und sie traten zu ihm
und griffen an seine Füße und fielen vor ihm nieder.

\bibverse{10} Da sprach Jesus zu ihnen: Fürchtet euch nicht! Gehet hin
und verkündigt es meinen Brüdern, dass sie gehen nach Galiläa; daselbst
werden sie mich sehen.

\bibverse{11} Da sie aber hingingen, siehe, da kamen etliche von den
Hütern in die Stadt und verkündigten den Hohenpriestern alles, was
geschehen war.

\bibverse{12} Und sie kamen zusammen mit den Ältesten und hielten einen
Rat und gaben den Kriegsknechten Gelds genug \bibverse{13} und sprachen:
Saget: Seine Jünger kamen des Nachts und stahlen ihn, dieweil wir
schliefen. \footnote{\textbf{28:13} Mt 27,64} \bibverse{14} Und wo es
würde auskommen bei dem Landpfleger, wollen wir ihn stillen und
schaffen, dass ihr sicher seid. \bibverse{15} Und sie nahmen das Geld
und taten, wie sie gelehrt waren. Solches ist eine gemeine Rede geworden
bei den Juden bis auf den heutigen Tag.

\bibverse{16} Aber die elf Jünger gingen nach Galiläa auf einen Berg,
dahin Jesus sie beschieden hatte. \bibverse{17} Und da sie ihn sahen,
fielen sie vor ihm nieder; etliche aber zweifelten. \bibverse{18} Und
Jesus trat zu ihnen, redete mit ihnen und sprach: Mir ist gegeben alle
Gewalt im Himmel und auf Erden. \bibverse{19} Darum gehet hin und lehret
alle Völker und taufet sie im Namen des Vaters und des Sohnes und des
heiligen Geistes, \^{}\^{} \bibverse{20} und lehret sie halten alles,
was ich euch befohlen habe. Und siehe, ich bin bei euch alle Tage bis an
der Welt Ende.
