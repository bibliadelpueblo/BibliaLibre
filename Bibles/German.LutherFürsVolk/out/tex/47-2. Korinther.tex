\hypertarget{section}{%
\section{1}\label{section}}

\bibleverse{1} Paulus, ein Apostel Jesu Christi durch den Willen Gottes,
und Bruder Timotheus der Gemeinde Gottes zu Korinth samt allen Heiligen
in ganz Achaja: \footnote{\textbf{1:1} 1Kor 1,1} \bibleverse{2} Gnade
sei mit euch und Friede von Gott, unserem Vater, und dem Herrn Jesus
Christus!

\bibleverse{3} Gelobet sei Gott und der Vater unseres Herrn Jesu
Christi, der Vater der Barmherzigkeit und Gott alles Trostes,
\bibleverse{4} der uns tröstet in aller unserer Trübsal, dass wir auch
trösten können, die da sind in allerlei Trübsal, mit dem Trost, damit
wir getröstet werden von Gott. \bibleverse{5} Denn gleichwie wir des
Leidens Christi viel haben, also werden wir auch reichlich getröstet
durch Christum. \footnote{\textbf{1:5} Ps 34,20; Ps 94,19; Röm 8,17}
\bibleverse{6} Wir haben aber Trübsal oder Trost, so geschieht es euch
zugute. Ist's Trübsal, so geschieht es euch zu Trost und Heil; welches
Heil sich beweist, wenn ihr leidet mit Geduld, dermaßen, wie wir leiden.
Ist's Trost, so geschieht auch das euch zu Trost und Heil; \footnote{\textbf{1:6}
  2Kor 4,8-11; 2Kor 4,15} \bibleverse{7} und unsere Hoffnung steht fest
für euch, dieweil wir wissen, dass, wie ihr des Leidens teilhaftig seid,
so werdet ihr auch des Trostes teilhaftig sein.

\bibleverse{8} Denn wir wollen euch nicht verhalten, liebe Brüder,
unsere Trübsal, die uns in Asien widerfahren ist, da wir über die Maßen
beschwert waren und über Macht, also dass wir auch am Leben verzagten
\footnote{\textbf{1:8} Apg 19,23-40; 1Kor 15,32} \bibleverse{9} und bei
uns beschlossen hatten, wir müssten sterben. Das geschah aber darum,
damit wir unser Vertrauen nicht auf uns selbst sollen stellen, sondern
auf Gott, der die Toten auferweckt, \bibleverse{10} welcher uns von
solchem Tode erlöst hat und noch täglich erlöst; und wir hoffen auf ihn,
er werde uns auch hinfort erlösen, \bibleverse{11} durch Hilfe auch
eurer Fürbitte für uns, auf dass über uns für die Gabe, die uns gegeben
ist, durch viel Personen viel Dank geschehe.

\bibleverse{12} Denn unser Ruhm ist dieser: das Zeugnis unseres
Gewissens, dass wir in Einfalt und göttlicher Lauterkeit, nicht in
fleischlicher Weisheit, sondern in der Gnade Gottes auf der Welt
gewandelt haben, allermeist aber bei euch. \footnote{\textbf{1:12} 2Kor
  2,17; Hebr 13,18; 1Kor 1,17} \bibleverse{13} Denn wir schreiben euch
nichts anderes, als was ihr leset und auch befindet. Ich hoffe aber, ihr
werdet uns auch bis ans Ende also befinden, gleichwie ihr uns zum Teil
befunden habt. \bibleverse{14} Denn wir sind euer Ruhm, gleichwie auch
ihr unser Ruhm seid auf des Herrn Jesu Tag.

\bibleverse{15} Und auf solch Vertrauen gedachte ich jenes Mal zu euch
zu kommen, auf dass ihr abermals eine Wohltat empfinget, \bibleverse{16}
und ich durch euch nach Mazedonien reiste und wiederum aus Mazedonien zu
euch käme und von euch geleitet würde nach Judäa. \footnote{\textbf{1:16}
  1Kor 16,5-6} \bibleverse{17} Bin ich aber leichtfertig gewesen, da ich
solches gedachte? Oder sind meine Anschläge fleischlich? Nicht also;
sondern bei mir ist Ja Ja, und Nein ist Nein. \bibleverse{18} Aber, o
ein treuer Gott, dass unser Wort an euch nicht Ja und Nein gewesen ist.
\bibleverse{19} Denn der Sohn Gottes, Jesus Christus, der unter euch
durch uns gepredigt ist, durch mich und Silvanus und Timotheus, der war
nicht Ja und Nein, sondern es war Ja in ihm. \bibleverse{20} Denn alle
Gottesverheißungen sind Ja in ihm und sind Amen in ihm, Gott zu Lobe
durch uns. \footnote{\textbf{1:20} Offb 3,14}

\bibleverse{21} Gott ist's aber, der uns befestigt samt euch in Christum
und uns gesalbt \footnote{\textbf{1:21} 1Jo 2,27} \bibleverse{22} und
versiegelt und in unsere Herzen das Pfand, den Geist, gegeben hat.
\footnote{\textbf{1:22} 2Kor 5,5; Röm 8,16; Eph 1,13}

\bibleverse{23} Ich rufe aber Gott an zum Zeugen auf meine Seele, dass
ich euch verschont habe in dem, dass ich nicht wieder gen Korinth
gekommen bin. \bibleverse{24} Nicht dass wir Herren seien über euren
Glauben, sondern wir sind Gehilfen eurer Freude; denn ihr stehet im
Glauben. \# 2 \bibleverse{1} Ich dachte aber solches bei mir, dass ich
nicht abermals in Traurigkeit zu euch käme. \footnote{\textbf{2:1} 1Kor
  4,21; 2Kor 12,21} \bibleverse{2} Denn, wenn ich euch traurig mache,
wer ist, der mich fröhlich mache, wenn nicht, der da von mir betrübt
wird? \bibleverse{3} Und dasselbe habe ich euch geschrieben, dass ich
nicht, wenn ich käme, über die traurig sein müsste, über welche ich mich
billig soll freuen; sintemal ich mich des zu euch allen versehe, dass
meine Freude euer aller Freude sei. \bibleverse{4} Denn ich schrieb euch
in großer Trübsal und Angst des Herzens mit viel Tränen; nicht, dass ihr
solltet betrübt werden, sondern auf dass ihr die Liebe erkenntet, welche
ich habe sonderlich zu euch.

\bibleverse{5} So aber jemand eine Betrübnis hat angerichtet, der hat
nicht mich betrübt, sondern zum Teil -- auf dass ich nicht zu viel sage
-- euch alle. \bibleverse{6} Es ist aber genug, dass derselbe von vielen
also gestraft ist, \bibleverse{7} dass ihr nun hinfort ihm desto mehr
vergebet und ihn tröstet, auf dass er nicht in allzu große Traurigkeit
versinke. \bibleverse{8} Darum ermahne ich euch, dass ihr die Liebe an
ihm beweiset. \bibleverse{9} Denn darum habe ich euch auch geschrieben,
dass ich erkennte, ob ihr rechtschaffen seid, gehorsam zu sein in allen
Stücken. \bibleverse{10} Welchem aber ihr etwas vergebet, dem vergebe
ich auch. Denn auch ich, wenn ich etwas vergebe jemand, das vergebe ich
um euretwillen, an Christi Statt, \bibleverse{11} auf dass wir nicht
übervorteilt werden vom Satan; denn uns ist nicht unbewusst, was er im
Sinn hat. \footnote{\textbf{2:11} Lk 22,31; 1Petr 5,8}

\bibleverse{12} Da ich aber gen Troas kam, zu predigen das Evangelium
Christi, und mir eine Tür aufgetan war in dem Herrn, \footnote{\textbf{2:12}
  Apg 14,27; 1Kor 16,9} \bibleverse{13} hatte ich keine Ruhe in meinem
Geist, da ich Titus, meinen Bruder, nicht fand; sondern ich machte
meinen Abschied mit ihnen und fuhr aus nach Mazedonien. \footnote{\textbf{2:13}
  Apg 20,1; 2Kor 7,6}

\bibleverse{14} Aber Gott sei gedankt, der uns allezeit Sieg gibt in
Christo und offenbart den Geruch seiner Erkenntnis durch uns an allen
Orten! \bibleverse{15} Denn wir sind Gott ein guter Geruch Christi unter
denen, die selig werden, und unter denen, die verloren werden:
\bibleverse{16} diesen ein Geruch des Todes zum Tode, jenen aber ein
Geruch des Lebens zum Leben. Und wer ist hierzu tüchtig? \footnote{\textbf{2:16}
  Lk 2,34; 2Kor 3,5} \bibleverse{17} Denn wir sind nicht, wie die
vielen, die das Wort Gottes verfälschen; sondern als aus Lauterkeit und
als aus Gott reden wir vor Gott in Christo. \footnote{\textbf{2:17} 2Kor
  1,12; 2Kor 4,2; 1Petr 4,11}

\hypertarget{section-1}{%
\section{3}\label{section-1}}

\bibleverse{1} Heben wir denn abermals an, uns selbst zu preisen? Oder
bedürfen wir, wie etliche, der Lobebriefe an euch oder Lobebriefe von
euch? \footnote{\textbf{3:1} 2Kor 5,12} \bibleverse{2} Ihr seid unser
Brief, in unser Herz geschrieben, der erkannt und gelesen wird von allen
Menschen; \footnote{\textbf{3:2} 1Kor 9,2} \bibleverse{3} die ihr
offenbar geworden seid, dass ihr ein Brief Christi seid, durch unseren
Dienst zubereitet, und geschrieben nicht mit Tinte, sondern mit dem
Geist des lebendigen Gottes, nicht in steinerne Tafeln, sondern in
fleischerne Tafeln des Herzens. \footnote{\textbf{3:3} 2Mo 24,12}

\bibleverse{4} Ein solch Vertrauen aber haben wir durch Christum zu
Gott. \bibleverse{5} Nicht, dass wir tüchtig sind von uns selber, etwas
zu denken als von uns selber; sondern dass wir tüchtig sind, ist von
Gott, \bibleverse{6} welcher auch uns tüchtig gemacht hat, das Amt zu
führen des Neuen Testaments, nicht des Buchstabens, sondern des Geistes.
Denn der Buchstabe tötet, aber der Geist macht lebendig. \footnote{\textbf{3:6}
  Jer 31,31; 1Kor 11,25; Röm 7,6; Joh 6,63}

\bibleverse{7} Wenn aber das Amt, das durch die Buchstaben tötet und in
die Steine gebildet war, Klarheit hatte, also dass die Kinder Israel
nicht konnten ansehen das Angesicht Moses um der Klarheit willen seines
Angesichtes, die doch aufhört, \footnote{\textbf{3:7} 2Mo 34,29-35}
\bibleverse{8} wie sollte nicht viel mehr das Amt, das den Geist gibt,
Klarheit haben! \footnote{\textbf{3:8} Gal 3,2; Gal 3,5} \bibleverse{9}
Denn wenn das Amt, das die Verdammnis predigt, Klarheit hat, wie viel
mehr hat das Amt, das die Gerechtigkeit predigt, überschwengliche
Klarheit. \footnote{\textbf{3:9} 5Mo 27,26; Röm 1,17; Röm 3,21}
\bibleverse{10} Denn auch jenes Teil, das verklärt war, ist nicht für
Klarheit zu achten gegen diese überschwengliche Klarheit.
\bibleverse{11} Denn wenn das Klarheit hatte, das da aufhört, wie viel
mehr wird das Klarheit haben, das da bleibt.

\bibleverse{12} Dieweil wir nun solche Hoffnung haben, sind wir voll
großer Freudigkeit \bibleverse{13} und tun nicht wie Mose, der die Decke
vor sein Angesicht hing, damit die Kinder Israel nicht ansehen konnten
das Ende des, das aufhört; \footnote{\textbf{3:13} 2Mo 34,33; 2Mo 34,35}
\bibleverse{14} sondern ihre Sinne sind verstockt. Denn bis auf den
heutigen Tag bleibt diese Decke unaufgedeckt über dem Alten Testament,
wenn sie es lesen, welche in Christo aufhört; \footnote{\textbf{3:14}
  Röm 11,25; Apg 28,27} \bibleverse{15} aber bis auf den heutigen Tag,
wenn Mose gelesen wird, hängt die Decke vor ihrem Herzen.
\bibleverse{16} Wenn es aber sich bekehrte zu dem Herrn, so würde die
Decke abgetan. \footnote{\textbf{3:16} Röm 11,23; Röm 11,26; 2Mo 34,34}
\bibleverse{17} Denn der Herr ist der Geist; wo aber der Geist des Herrn
ist, da ist Freiheit. \bibleverse{18} Nun aber spiegelt sich in uns
allen des Herrn Klarheit mit aufgedecktem Angesicht, und wir werden
verklärt in dasselbe Bild von einer Klarheit zu der anderen, als vom
Herrn, der der Geist ist. \# 4 \bibleverse{1} Darum, dieweil wir ein
solch Amt haben, wie uns denn Barmherzigkeit widerfahren ist, so werden
wir nicht müde, \footnote{\textbf{4:1} 2Kor 3,6; 1Kor 7,25}
\bibleverse{2} sondern meiden auch heimliche Schande und gehen nicht mit
Schalkheit um, fälschen auch nicht Gottes Wort; sondern mit Offenbarung
der Wahrheit beweisen wir uns wohl an aller Menschen Gewissen vor Gott.
\footnote{\textbf{4:2} 2Kor 2,17; 1Thes 2,5} \bibleverse{3} Ist nun
unser Evangelium verdeckt, so ist's in denen, die verloren werden,
verdeckt; \footnote{\textbf{4:3} 1Kor 1,18} \bibleverse{4} bei welchen
der Gott dieser Welt der Ungläubigen Sinn verblendet hat, dass sie nicht
sehen das helle Licht des Evangeliums von der Klarheit Christi, welcher
ist das Ebenbild Gottes. \footnote{\textbf{4:4} Hebr 1,3} \bibleverse{5}
Denn wir predigen nicht uns selbst, sondern Jesum Christum, dass er sei
der Herr, wir aber eure Knechte um Jesu willen. \footnote{\textbf{4:5}
  2Kor 1,24} \bibleverse{6} Denn Gott, der da hieß das Licht aus der
Finsternis hervorleuchten, der hat einen hellen Schein in unsere Herzen
gegeben, dass durch uns entstünde die Erleuchtung von der Erkenntnis der
Klarheit Gottes in dem Angesichte Jesu Christi. \footnote{\textbf{4:6}
  1Mo 1,3; 2Kor 3,18}

\bibleverse{7} Wir haben aber solchen Schatz in irdenen Gefäßen, auf
dass die überschwengliche Kraft sei Gottes und nicht von uns.
\footnote{\textbf{4:7} 1Kor 4,11-13; 2Kor 11,23-27} \bibleverse{8} Wir
haben allenthalben Trübsal, aber wir ängsten uns nicht; uns ist bange,
aber wir verzagen nicht; \bibleverse{9} wir leiden Verfolgung, aber wir
werden nicht verlassen; wir werden unterdrückt, aber wir kommen nicht
um; \bibleverse{10} und tragen allezeit das Sterben des Herrn Jesu an
unserem Leibe, auf dass auch das Leben des Herrn Jesu an unserem Leibe
offenbar werde. \bibleverse{11} Denn wir, die wir leben, werden immerdar
in den Tod gegeben um Jesu willen, auf dass auch das Leben Jesu offenbar
werde an unserem sterblichen Fleische. \footnote{\textbf{4:11} Röm 8,36}
\bibleverse{12} Darum ist nun der Tod mächtig in uns, aber das Leben in
euch.

\bibleverse{13} Dieweil wir aber denselben Geist des Glaubens haben,
nach dem, das geschrieben steht: „Ich glaube, darum rede ich``, so
glauben wir auch, darum so reden wir auch \bibleverse{14} und wissen,
dass der, der den Herrn Jesus hat auferweckt, wird uns auch auferwecken
durch Jesum und wird uns darstellen samt euch. \bibleverse{15} Denn es
geschieht alles um euretwillen, auf dass die überschwengliche Gnade
durch vieler Danksagen Gott reichlich preise. \footnote{\textbf{4:15}
  2Kor 1,6; 2Kor 1,11}

\bibleverse{16} Darum werden wir nicht müde; sondern, ob unser
äußerlicher Mensch verdirbt, so wird doch der innerliche von Tag zu Tag
erneuert. \footnote{\textbf{4:16} Eph 3,16} \bibleverse{17} Denn unsere
Trübsal, die zeitlich und leicht ist, schafft eine ewige und über alle
Maßen wichtige Herrlichkeit \footnote{\textbf{4:17} Röm 8,17-18; 1Petr
  1,6} \bibleverse{18} uns, die wir nicht sehen auf das Sichtbare,
sondern auf das Unsichtbare. Denn was sichtbar ist, das ist zeitlich;
was aber unsichtbar ist, das ist ewig. \footnote{\textbf{4:18} Hebr 11,1}

\hypertarget{section-2}{%
\section{5}\label{section-2}}

\bibleverse{1} Wir wissen aber, wenn unser irdisch Haus dieser Hütte
zerbrochen wird, dass wir einen Bau haben, von Gott erbauet, ein Haus,
nicht mit Händen gemacht, das ewig ist, im Himmel. \footnote{\textbf{5:1}
  Hi 4,19; 2Petr 1,14} \bibleverse{2} Und darüber sehnen wir uns auch
nach unserer Behausung, die vom Himmel ist, und uns verlangt, dass wir
damit überkleidet werden; \bibleverse{3} so doch, wo wir bekleidet und
nicht bloß erfunden werden. \bibleverse{4} Denn dieweil wir in der Hütte
sind, sehnen wir uns und sind beschwert; sintemal wir wollten lieber
nicht entkleidet, sondern überkleidet werden, auf dass das Sterbliche
würde verschlungen von dem Leben. \bibleverse{5} Der uns aber dazu
bereitet, das ist Gott, der uns das Pfand, den Geist, gegeben hat.
\footnote{\textbf{5:5} 2Kor 1,22; Röm 8,16; Röm 8,23; Eph 1,13-14}

\bibleverse{6} So sind wir denn getrost allezeit und wissen, dass,
dieweil wir im Leibe wohnen, so wallen wir ferne vom Herrn; \footnote{\textbf{5:6}
  Hebr 11,13} \bibleverse{7} denn wir wandeln im Glauben, und nicht im
Schauen. \footnote{\textbf{5:7} Röm 8,24; 1Petr 1,8} \bibleverse{8} Wir
sind aber getrost und haben vielmehr Lust, außer dem Leibe zu wallen und
daheim zu sein bei dem Herrn. \footnote{\textbf{5:8} Phil 1,23}
\bibleverse{9} Darum fleißigen wir uns auch, wir sind daheim oder
wallen, dass wir ihm wohl gefallen. \footnote{\textbf{5:9} Ps 39,13}
\bibleverse{10} Denn wir müssen alle offenbar werden vor dem Richtstuhl
Christi, auf dass ein jeglicher empfange, nach dem er gehandelt hat bei
Leibesleben, es sei gut oder böse. \footnote{\textbf{5:10} Joh 5,29; Apg
  17,31; Röm 2,16; Röm 14,10; Eph 6,8}

\bibleverse{11} Dieweil wir denn wissen, dass der Herr zu fürchten ist,
fahren wir schön mit den Leuten; aber Gott sind wir offenbar. Ich hoffe
aber, dass wir auch in eurem Gewissen offenbar sind. \bibleverse{12} Wir
loben uns nicht abermals bei euch, sondern geben euch eine Ursache, zu
rühmen von uns, auf dass ihr habt zu rühmen wider die, die sich nach dem
Ansehen rühmen, und nicht nach dem Herzen. \footnote{\textbf{5:12} Röm
  14,7-8} \bibleverse{13} Denn tun wir zu viel, so tun wir's Gott; sind
wir mäßig, so sind wir euch mäßig. \bibleverse{14} Denn die Liebe
Christi dringt in uns also, sintemal wir halten, dass, wenn einer für
alle gestorben ist, so sind sie alle gestorben; \bibleverse{15} und er
ist darum für alle gestorben, auf dass die, die da leben, hinfort nicht
sich selbst leben, sondern dem, der für sie gestorben und auferstanden
ist.

\bibleverse{16} Darum kennen wir von nun an niemand nach dem Fleisch;
und ob wir auch Christum gekannt haben nach dem Fleisch, so kennen wir
ihn doch jetzt nicht mehr. \bibleverse{17} Darum, ist jemand in Christo,
so ist er eine neue Kreatur; das Alte ist vergangen, siehe, es ist alles
neu geworden! \bibleverse{18} Aber das alles von Gott, der uns mit ihm
selber versöhnt hat durch Jesum Christum und das Amt gegeben, das die
Versöhnung predigt. \footnote{\textbf{5:18} Röm 5,10} \bibleverse{19}
Denn Gott war in Christo und versöhnte die Welt mit ihm selber und
rechnete ihnen ihre Sünden nicht zu und hat unter uns aufgerichtet das
Wort von der Versöhnung. \footnote{\textbf{5:19} Röm 3,24-25; Kol
  1,19-20}

\bibleverse{20} So sind wir nun Botschafter an Christi Statt, denn Gott
vermahnt durch uns; so bitten wir nun an Christi Statt: Lasset euch
versöhnen mit Gott. \footnote{\textbf{5:20} Lk 10,16} \bibleverse{21}
Denn er hat den, der von keiner Sünde wusste, für uns zur Sünde gemacht,
auf dass wir würden in ihm die Gerechtigkeit, die vor Gott gilt.
\footnote{\textbf{5:21} Jes 53,6; Joh 8,46; Röm 1,17}

\hypertarget{section-3}{%
\section{6}\label{section-3}}

\bibleverse{1} Wir ermahnen aber euch als Mithelfer, dass ihr nicht
vergeblich die Gnade Gottes empfanget. \footnote{\textbf{6:1} 2Kor 1,24}
\bibleverse{2} Denn er spricht: „Ich habe dich in der angenehmen Zeit
erhört und habe dir am Tage des Heils geholfen.`` Sehet, jetzt ist die
angenehme Zeit, jetzt ist der Tag des Heils! -- \footnote{\textbf{6:2}
  Lk 4,19; Lk 4,21}

\bibleverse{3} Und wir geben niemand irgendein Ärgernis, auf dass unser
Amt nicht verlästert werde; \bibleverse{4} sondern in allen Dingen
beweisen wir uns als die Diener Gottes: in großer Geduld, in Trübsalen,
in Nöten, in Ängsten, \bibleverse{5} in Schlägen, in Gefängnissen, in
Aufruhren, in Arbeit, in Wachen, in Fasten, \footnote{\textbf{6:5} 1Kor
  4,11-13; 2Kor 11,23-27} \bibleverse{6} in Keuschheit, in Erkenntnis,
in Langmut, in Freundlichkeit, in dem heiligen Geist, in ungefärbter
Liebe, \footnote{\textbf{6:6} 1Tim 4,12} \bibleverse{7} in dem Wort der
Wahrheit, in der Kraft Gottes, durch Waffen der Gerechtigkeit zur
Rechten und zur Linken, \footnote{\textbf{6:7} 2Kor 4,2; 1Kor 2,4; Eph
  6,14-17} \bibleverse{8} durch Ehre und Schande, durch böse Gerüchte
und gute Gerüchte: als die Verführer, und doch wahrhaftig;
\bibleverse{9} als die Unbekannten, und doch bekannt; als die
Sterbenden, und siehe, wir leben; als die Gezüchtigten, und doch nicht
ertötet; \bibleverse{10} als die Traurigen, aber allezeit fröhlich; als
die Armen, aber die doch viele reich machen; als die nichts innehaben,
und doch alles haben. \footnote{\textbf{6:10} Phil 4,12-13}

\bibleverse{11} O ihr Korinther! unser Mund hat sich zu euch aufgetan,
unser Herz ist weit. \bibleverse{12} Ihr habt nicht engen Raum in uns;
aber eng ist's in euren Herzen. \bibleverse{13} Ich rede mit euch als
mit meinen Kindern, dass ihr euch auch also gegen mich stellet und
werdet auch weit.

\bibleverse{14} Ziehet nicht am fremden Joch mit den Ungläubigen. Denn
was hat die Gerechtigkeit zu schaffen mit der Ungerechtigkeit? Was hat
das Licht für Gemeinschaft mit der Finsternis? \footnote{\textbf{6:14}
  Eph 5,11} \bibleverse{15} Wie stimmt Christus mit Belial? Oder was für
ein Teil hat der Gläubige mit dem Ungläubigen? \bibleverse{16} Was hat
der Tempel Gottes für Gleichheit mit den Götzen? Ihr aber seid der
Tempel des lebendigen Gottes; wie denn Gott spricht: „Ich will unter
ihnen wohnen und unter ihnen wandeln und will ihr Gott sein, und sie
sollen mein Volk sein. \bibleverse{17} Darum gehet aus von ihnen und
sondert euch ab, spricht der Herr, und rühret kein Unreines an, so will
ich euch annehmen \footnote{\textbf{6:17} Offb 18,14} \bibleverse{18}
und euer Vater sein, und ihr sollt meine Söhne und Töchter sein, spricht
der allmächtige Herr.`` \# 7 \bibleverse{1} Dieweil wir nun solche
Verheißungen haben, meine Liebsten, so lasset uns von aller Befleckung
des Fleisches und des Geistes uns reinigen und fortfahren mit der
Heiligung in der Furcht Gottes.

\bibleverse{2} Fasset uns: Wir haben niemand Leid getan, wir haben
niemand verletzt, wir haben niemand übervorteilt. \bibleverse{3} Nicht
sage ich solches, euch zu verdammen; denn ich habe droben zuvor gesagt,
dass ihr in unseren Herzen seid, mitzusterben und mitzuleben.
\footnote{\textbf{7:3} 2Kor 6,11-13; Röm 6,8} \bibleverse{4} Ich rede
mit großer Freudigkeit zu euch; ich rühme viel von euch; ich bin erfüllt
mit Trost; ich bin überschwenglich in Freuden und in aller unserer
Trübsal.

\bibleverse{5} Denn da wir nach Mazedonien kamen, hatte unser Fleisch
keine Ruhe; sondern allenthalben waren wir in Trübsal: auswendig Streit,
inwendig Furcht. \bibleverse{6} Aber Gott, der die Geringen tröstet, der
tröstete uns durch die Ankunft des Titus \footnote{\textbf{7:6} 2Kor
  2,13; 2Kor 4,8} \bibleverse{7} nicht allein aber durch seine Ankunft,
sondern auch durch den Trost, mit dem er getröstet war an euch, da er
uns verkündigte euer Verlangen, euer Weinen, euren Eifer um mich, also
dass ich mich noch mehr freute.

\bibleverse{8} Denn dass ich euch durch den Brief habe traurig gemacht,
reut mich nicht. Und ob's mich reute, dieweil ich sehe, dass der Brief
vielleicht eine Weile euch betrübt hat, \bibleverse{9} so freue ich mich
doch nun, nicht darüber, dass ihr seid betrübt worden, sondern dass ihr
betrübt seid worden zur Reue. Denn ihr seid göttlich betrübt worden,
dass ihr von uns ja keinen Schaden irgendworin nehmet. \bibleverse{10}
Denn die göttliche Traurigkeit wirkt zur Seligkeit einen Reue, die
niemand gereut; die Traurigkeit aber der Welt wirkt den Tod. \footnote{\textbf{7:10}
  Mt 26,75; Mt 27,3-5; Lk 15,17-24} \bibleverse{11} Siehe, dass ihr
göttlich seid betrübt worden, welchen Fleiß hat das in euch gewirkt,
dazu Verantwortung, Zorn, Furcht, Verlangen, Eifer, Rache! Ihr habt euch
bewiesen in allen Stücken, dass ihr rein seid in der Sache.
\bibleverse{12} Darum, ob ich euch geschrieben habe, so ist's doch nicht
geschehen um des willen, der beleidigt hat, auch nicht um des willen,
der beleidigt ist, sondern um deswillen, dass euer Fleiß gegen uns
offenbar würde bei euch vor Gott. \bibleverse{13} Derhalben sind wir
getröstet worden, dass ihr getröstet seid. Überschwenglicher aber haben
wir uns noch gefreut über die Freude des Titus denn sein Geist ist
erquickt an euch allen. \bibleverse{14} Denn was ich vor ihm von euch
gerühmt habe, darin bin ich nicht zu Schanden geworden; sondern,
gleichwie alles wahr ist, was ich mit euch geredet habe, also ist auch
unser Rühmen vor Titus wahr geworden. \bibleverse{15} Und er ist überaus
herzlich wohl gegen euch gesinnt, wenn er gedenkt an euer aller
Gehorsam, wie ihr ihn mit Furcht und Zittern habt aufgenommen.
\bibleverse{16} Ich freue mich, dass ich mich zu euch alles (Guten)
versehen darf. \# 8 \bibleverse{1} Ich tue euch kund, liebe Brüder, die
Gnade Gottes, die in den Gemeinden in Mazedonien gegeben ist.
\bibleverse{2} Denn ihre Freude war überschwenglich, da sie durch viel
Trübsal bewährt wurden; und wiewohl sie sehr arm sind, haben sie doch
reichlich gegeben in aller Einfalt. \bibleverse{3} Denn nach allem
Vermögen (das bezeuge ich) und über Vermögen waren sie willig
\bibleverse{4} und baten uns mit vielem Zureden, dass wir aufnähmen die
Wohltat und Gemeinschaft der Handreichung, die da geschieht den
Heiligen; \footnote{\textbf{8:4} Apg 11,29} \bibleverse{5} und nicht,
wie wir hofften, sondern sie ergaben sich selbst, zuerst dem Herrn und
darnach uns, durch den Willen Gottes, \bibleverse{6} dass wir mussten
Titus ermahnen, auf dass er, wie er zuvor angefangen hatte, also auch
unter euch solche Wohltat ausrichtete. \bibleverse{7} Aber gleichwie ihr
in allen Stücken reich seid, im Glauben und im Wort und in der
Erkenntnis und in allerlei Fleiß und in eurer Liebe zu uns, also
schaffet, dass ihr auch in dieser Wohltat reich seid.

\bibleverse{8} Nicht sage ich, dass ich etwas gebiete; sondern, dieweil
andere so fleißig sind, versuche ich auch eure Liebe, ob sie rechter Art
sei. \bibleverse{9} Denn ihr wisset die Gnade unseres Herrn Jesu
Christi, dass, ob er wohl reich ist, ward er doch arm um euretwillen,
auf dass ihr durch seine Armut reich würdet. \footnote{\textbf{8:9} Mt
  8,20; 2Kor 2,7} \bibleverse{10} Und meine Meinung hierin gebe ich;
denn solches ist euch nützlich, die ihr angefangen habt vom vorigen
Jahre her nicht allein das Tun, sondern auch das Wollen; \bibleverse{11}
nun aber vollbringet auch das Tun, auf dass, gleichwie da ist ein
geneigtes Gemüt, zu wollen, so sei auch da ein geneigtes Gemüt, zu tun
von dem, was ihr habt. \bibleverse{12} Denn so einer willig ist, so ist
er angenehm, nach dem er hat, nicht nach dem er nicht hat.
\bibleverse{13} Nicht geschieht das in der Meinung, dass die anderen
Ruhe haben, und ihr Trübsal, sondern dass es gleich sei. \bibleverse{14}
So diene euer Überfluss ihrem Mangel diese (teure) Zeit lang, auf dass
auch ihr Überfluss hernach diene eurem Mangel und ein Ausgleich
geschehe; \bibleverse{15} wie geschrieben steht: „Der viel sammelte,
hatte nicht Überfluss, der wenig sammelte, hatte nicht Mangel.``

\bibleverse{16} Gott aber sei Dank, der solchen Eifer für euch gegeben
hat in das Herz des Titus. \bibleverse{17} Denn er nahm zwar die
Ermahnung an; aber dieweil er fleißig war, ist er von selber zu euch
gereist. \footnote{\textbf{8:17} 2Kor 8,6; 2Kor 7,7; 2Kor 7,15}
\bibleverse{18} Wir haben aber einen Bruder mit ihm gesandt, der das Lob
hat am Evangelium durch alle Gemeinden. \footnote{\textbf{8:18} 2Kor
  12,18} \bibleverse{19} Nicht allein aber das, sondern er ist auch
verordnet von den Gemeinden zum Gefährten unserer Fahrt in dieser
Wohltat, welche durch uns ausgerichtet wird dem Herrn zu Ehren und zum
Preis eures guten Willens. \footnote{\textbf{8:19} Gal 2,10}
\bibleverse{20} Also verhüten wir, dass uns nicht jemand übel nachreden
möge solcher reichen Steuer halben, die durch uns ausgerichtet wird;
\bibleverse{21} und sehen darauf, dass es redlich zugehe, nicht allein
vor dem Herrn, sondern auch vor den Menschen. \bibleverse{22} Auch haben
wir mit ihnen gesandt unseren Bruder, den wir oft erfunden haben in
vielen Stücken, dass er fleißig sei, nun aber viel fleißiger.
\bibleverse{23} Und wir sind großer Zuversicht zu euch, es sei des Titus
halben, welcher mein Geselle und Gehilfe unter euch ist, oder unserer
Brüder halben, welche Boten sind der Gemeinden und eine Ehre Christi.
\bibleverse{24} Erzeiget nun die Beweisung eurer Liebe und unseres
Rühmens von euch an diesen auch öffentlich vor den Gemeinden!
\footnote{\textbf{8:24} 2Kor 7,14}

\hypertarget{section-4}{%
\section{9}\label{section-4}}

\bibleverse{1} Denn von solcher Steuer, die den Heiligen geschieht, ist
mir nicht not, euch zu schreiben. \bibleverse{2} Denn ich weiß euren
guten Willen, davon ich rühme bei denen aus Mazedonien und sage: Achaja
ist schon voriges Jahr bereit gewesen; und euer Beispiel hat viele
gereizt. \bibleverse{3} Ich habe aber diese Brüder darum gesandt, dass
nicht unser Rühmen von euch zunichte würde in dem Stücke, und dass ihr
bereit seid, gleichwie ich von euch gesagt habe; \bibleverse{4} auf dass
nicht, so die aus Mazedonien mit mir kämen und euch unbereit fänden, wir
(will nicht sagen: ihr) zu Schanden würden mit solchem Rühmen.
\bibleverse{5} So habe ich es nun für nötig angesehen, die Brüder zu
ermahnen, dass sie voranzögen zu euch, fertigzumachen diesen zuvor
verheißenen Segen, dass er bereit sei, also dass es sei ein Segen und
nicht ein Geiz.

\bibleverse{6} Ich meine aber das: Wer da kärglich sät, der wird auch
kärglich ernten; und wer da sät im Segen, der wird auch ernten im Segen.
\footnote{\textbf{9:6} Spr 11,24; Spr 19,17} \bibleverse{7} Ein
jeglicher nach seiner Willkür, nicht mit Unwillen oder aus Zwang; denn
einen fröhlichen Geber hat Gott lieb. \footnote{\textbf{9:7} Röm 12,8}
\bibleverse{8} Gott aber kann machen, dass allerlei Gnade unter euch
reichlich sei, dass ihr in allen Dingen volle Genüge habt und reich seid
zu allerlei guten Werken; \bibleverse{9} wie geschrieben steht: „Er hat
ausgestreut und gegeben den Armen; seine Gerechtigkeit bleibt in
Ewigkeit.``

\bibleverse{10} Der aber Samen reicht dem Sämann, der wird auch das Brot
reichen zur Speise und wird vermehren euren Samen und wachsen lassen das
Gewächs eurer Gerechtigkeit, \footnote{\textbf{9:10} Jes 55,10; Hos
  10,12} \bibleverse{11} dass ihr reich seid in allen Dingen mit aller
Einfalt, welche wirkt durch uns Danksagung Gott. \bibleverse{12} Denn
die Handreichung dieser Steuer erfüllt nicht allein den Mangel der
Heiligen, sondern ist auch überschwenglich darin, dass viele Gott danken
für diesen unseren treuen Dienst \bibleverse{13} und preisen Gott über
euer untertäniges Bekenntnis des Evangeliums Christi und über eure
einfältige Steuer an sie und an alle, \bibleverse{14} indem auch sie
nach euch verlangt im Gebet für euch um der überschwenglichen Gnade
Gottes willen in euch. \bibleverse{15} Gott aber sei Dank für seine
unaussprechliche Gabe! \# 10 \bibleverse{1} Ich aber, Paulus, ermahne
euch durch die Sanftmütigkeit und Lindigkeit Christi, der ich
gegenwärtig unter euch gering bin, abwesend aber dreist gegen euch.
\bibleverse{2} Ich bitte aber, dass mir nicht not sei, gegenwärtig
dreist zu handeln und der Kühnheit zu brauchen, die man mir zumisst,
gegen etliche, die uns schätzen, als wandelten wir fleischlicherweise.
\bibleverse{3} Denn ob wir wohl im Fleisch wandeln, so streiten wir doch
nicht fleischlicherweise. \bibleverse{4} Denn die Waffen unserer
Ritterschaft sind nicht fleischlich, sondern mächtig vor Gott, zu
zerstören Befestigungen; \footnote{\textbf{10:4} Eph 6,13-17}
\bibleverse{5} wir zerstören damit die Anschläge und alle Höhe, die sich
erhebt wider die Erkenntnis Gottes, und nehmen gefangen alle Vernunft
unter den Gehorsam Christi \bibleverse{6} und sind bereit, zu rächen
allen Ungehorsam, wenn euer Gehorsam erfüllt ist.

\bibleverse{7} Richtet ihr nach dem Ansehen? Verlässt sich jemand
darauf, dass er Christo angehöre, der denke solches auch wiederum bei
sich, dass, gleichwie er Christo angehöre, also auch wir Christo
angehören. \bibleverse{8} Und so ich auch etwas weiter mich rühmte von
unserer Gewalt, welche uns der Herr gegeben hat, euch zu bessern, und
nicht zu verderben, wollte ich nicht zu Schanden werden. \bibleverse{9}
Das sage ich aber, dass ihr nicht euch dünken lasset, als hätte ich euch
wollen schrecken mit Briefen. \bibleverse{10} Denn die Briefe, sprechen
sie, sind schwer und stark; aber die Gegenwart des Leibes ist schwach
und die Rede verächtlich. \bibleverse{11} Wer ein solcher ist, der
denke, dass, wie wir sind mit Worten in den Briefen abwesend, so werden
wir auch wohl sein mit der Tat gegenwärtig. \footnote{\textbf{10:11}
  2Kor 13,2; 2Kor 13,10}

\bibleverse{12} Denn wir wagen uns nicht unter die zu rechnen oder zu
zählen, die sich selbst loben, aber dieweil sie sich an sich selbst
messen und halten allein von sich selbst, verstehen sie nichts.
\footnote{\textbf{10:12} 2Kor 3,1; 2Kor 5,12} \bibleverse{13} Wir aber
rühmen uns nicht über das Ziel hinaus, sondern nur nach dem Ziel der
Regel, mit der uns Gott abgemessen hat das Ziel, zu gelangen auch bis zu
euch. \footnote{\textbf{10:13} Röm 12,3; Röm 15,20; Gal 2,7}
\bibleverse{14} Denn wir fahren nicht zu weit, als wären wir nicht
gelangt bis zu euch; denn wir sind ja auch bis zu euch gekommen mit dem
Evangelium Christi; \bibleverse{15} und rühmen uns nicht übers Ziel
hinaus in fremder Arbeit und haben Hoffnung, wenn nun euer Glaube in
euch wächst, dass wir unserer Regel nach wollen weiterkommen
\bibleverse{16} und das Evangelium auch predigen denen, die jenseits von
euch wohnen, und uns nicht rühmen in dem, was mit fremder Regel bereitet
ist. \bibleverse{17} Wer sich aber rühmt, der rühme sich des Herrn.
\bibleverse{18} Denn darum ist einer nicht tüchtig, dass er sich selbst
lobt, sondern dass ihn der Herr lobt. \footnote{\textbf{10:18} 1Kor 4,5}

\hypertarget{section-5}{%
\section{11}\label{section-5}}

\bibleverse{1} Wollte Gott, ihr hieltet mir ein wenig Torheit zugut!
Doch ihr haltet mir's wohl zugut. \bibleverse{2} Denn ich eifere um euch
mit göttlichem Eifer; denn ich habe euch vertraut einem Manne, dass ich
eine reine Jungfrau Christo zubrächte. \bibleverse{3} Ich fürchte aber,
dass, wie die Schlange Eva verführte mit ihrer Schalkheit, also auch
eure Sinne verrückt werden von der Einfalt in Christo. \footnote{\textbf{11:3}
  1Mo 3,4; 1Mo 3,13} \bibleverse{4} Denn wenn, der da zu euch kommt,
einen anderen Jesus predigte, den wir nicht gepredigt haben, oder ihr
einen anderen Geist empfinget, den ihr nicht empfangen habt, oder ein
anderes Evangelium, das ihr nicht angenommen habt, so vertrüget ihr's
billig. \footnote{\textbf{11:4} Gal 1,8-9} \bibleverse{5} Denn ich
achte, ich sei nicht weniger, als die „hohen`` Apostel sind. \footnote{\textbf{11:5}
  2Kor 12,11; 1Kor 15,10; Gal 2,6; Gal 2,9} \bibleverse{6} Und ob ich
nicht kundig bin der Rede, so bin ich doch nicht unkundig in der
Erkenntnis. Doch ich bin bei euch allenthalben wohl bekannt. \footnote{\textbf{11:6}
  1Kor 2,1-2; 1Kor 2,13; Eph 3,4}

\bibleverse{7} Oder habe ich gesündigt, dass ich mich erniedrigt habe,
auf dass ihr erhöht würdet? Denn ich habe euch das Evangelium Gottes
umsonst verkündigt \footnote{\textbf{11:7} 2Kor 12,13; 1Kor 9,12-18; Mt
  10,8} \bibleverse{8} und habe andere Gemeinden beraubt und Sold von
ihnen genommen, dass ich euch predigte. \footnote{\textbf{11:8} Phil
  4,10; Phil 4,15} \bibleverse{9} Und da ich bei euch war gegenwärtig
und Mangel hatte, war ich niemand beschwerlich. Denn meinen Mangel
erstatteten die Brüder, die aus Mazedonien kamen. So habe ich mich in
allen Stücken euch unbeschwerlich gehalten und will auch noch mich also
halten. \bibleverse{10} So gewiss die Wahrheit Christi in mir ist, so
soll mir dieser Ruhm in den Ländern Achajas nicht verstopft werden.
\bibleverse{11} Warum das? Dass ich euch nicht sollte liebhaben? Gott
weiß es.

\bibleverse{12} Was ich aber tue und tun will, das tue ich darum, dass
ich die Ursache abschneide denen, die Ursache suchen, dass sie rühmen
möchten, sie seien wie wir. \bibleverse{13} Denn solche falsche Apostel
und trügliche Arbeiter verstellen sich zu Christi Aposteln. \footnote{\textbf{11:13}
  2Kor 2,17} \bibleverse{14} Und das ist auch kein Wunder; denn er
selbst, der Satan, verstellt sich zum Engel des Lichtes. \bibleverse{15}
Darum ist es auch nicht ein Großes, wenn sich seine Diener verstellen
als Prediger der Gerechtigkeit; welcher Ende sein wird nach ihren
Werken.

\bibleverse{16} Ich sage abermals, dass nicht jemand wähne, ich sei
töricht; wo aber nicht, so nehmet mich an als einen Törichten, dass ich
mich auch ein wenig rühme. \bibleverse{17} Was ich jetzt rede, das rede
ich nicht als im Herrn, sondern als in der Torheit, dieweil wir in das
Rühmen gekommen sind. \bibleverse{18} Sintemal viele sich rühmen nach
dem Fleisch, will ich mich auch rühmen. \bibleverse{19} Denn ihr
vertraget gern die Narren, dieweil ihr klug seid. \footnote{\textbf{11:19}
  1Kor 4,10} \bibleverse{20} Ihr vertraget, wenn euch jemand zu Knechten
macht, wenn euch jemand schindet, wenn euch jemand gefangennimmt, wenn
jemand euch trotzt, wenn euch jemand in das Angesicht streicht.
\bibleverse{21} Das sage ich nach der Unehre, als wären wir schwach
geworden. Worauf aber jemand kühn ist (ich rede in Torheit!), darauf bin
ich auch kühn. \bibleverse{22} Sie sind Hebräer? -- Ich auch! Sie sind
Israeliter? -- Ich auch! Sie sind Abrahams Same? -- Ich auch!
\bibleverse{23} Sie sind Diener Christi? -- Ich rede töricht: Ich bin's
wohl mehr: Ich habe mehr gearbeitet, ich habe mehr Schläge erlitten, bin
öfter gefangen, oft in Todesnöten gewesen; \footnote{\textbf{11:23} 2Kor
  6,4-5; 1Kor 15,10} \bibleverse{24} von den Juden habe ich fünfmal
empfangen vierzig Streiche weniger eins; \footnote{\textbf{11:24} 5Mo
  25,3} \bibleverse{25} ich bin dreimal gestäupt, einmal gesteinigt,
dreimal habe ich Schiffbruch erlitten, Tag und Nacht habe ich zugebracht
in der Tiefe des Meers; \footnote{\textbf{11:25} Apg 16,22; Apg 14,19}
\bibleverse{26} ich bin oft gereist, ich bin in Gefahr gewesen durch die
Flüsse, in Gefahr durch die Mörder, in Gefahr unter den Juden, in Gefahr
unter den Heiden, in Gefahr in den Städten, in Gefahr in der Wüste, in
Gefahr auf dem Meer, in Gefahr unter den falschen Brüdern;
\bibleverse{27} in Mühe und Arbeit, in viel Wachen, in Hunger und Durst,
in viel Fasten, in Frost und Blöße;

\bibleverse{28} außer was sich sonst zuträgt, nämlich, dass ich täglich
werde angelaufen und trage Sorge für alle Gemeinden. \footnote{\textbf{11:28}
  Apg 20,18-21; Apg 20,31} \bibleverse{29} Wer ist schwach, und ich
werde nicht schwach? Wer wird geärgert, und ich brenne nicht?

\bibleverse{30} So ich mich ja rühmen soll, will ich mich meiner
Schwachheit rühmen. \bibleverse{31} Gott und der Vater unseres Herrn
Jesu Christi, welcher sei gelobt in Ewigkeit, weiß, dass ich nicht lüge.
\bibleverse{32} Zu Damaskus verwahrte der Landpfleger des Königs Aretas
die Stadt der Damasker und wollte mich greifen, \bibleverse{33} und ich
ward in einem Korbe zum Fenster hinaus durch die Mauer niedergelassen
und entrann aus seinen Händen. \footnote{\textbf{11:33} Apg 9,24-25}

\hypertarget{section-6}{%
\section{12}\label{section-6}}

\bibleverse{1} Es ist mir ja das Rühmen nichts nütze; doch will ich
kommen auf die Gesichte und Offenbarungen des Herrn. \bibleverse{2} Ich
kenne einen Menschen in Christo; vor vierzehn Jahren (ist er in dem
Leibe gewesen, so weiß ich's nicht; oder ist er außer dem Leibe gewesen,
so weiß ich's auch nicht; Gott weiß es) ward derselbe entzückt bis in
den dritten Himmel. \bibleverse{3} Und ich kenne denselben Menschen (ob
er im Leibe oder außer dem Leibe gewesen ist, weiß ich nicht; Gott weiß
es); \bibleverse{4} der ward entzückt in das Paradies und hörte
unaussprechliche Worte, welche kein Mensch sagen kann. \bibleverse{5}
Für denselben will ich mich rühmen; für mich selbst aber will ich mich
nichts rühmen, nur meiner Schwachheit. \bibleverse{6} Und wenn ich mich
rühmen wollte, täte ich daran nicht töricht; denn ich wollte die
Wahrheit sagen. Ich enthalte mich aber dessen, auf dass nicht jemand
mich höher achte, als er an mir sieht oder von mir hört. \footnote{\textbf{12:6}
  2Kor 10,8} \bibleverse{7} Und auf dass ich mich nicht der hohen
Offenbarung überhebe, ist mir gegeben ein Pfahl ins Fleisch, nämlich des
Satans Engel, der mich mit Fäusten schlage, auf dass ich mich nicht
überhebe. \bibleverse{8} Dafür ich dreimal zum Herrn gefleht habe, dass
er von mir wiche. \bibleverse{9} Und er hat zu mir gesagt: Lass dir an
meiner Gnade genügen; denn meine Kraft ist in den Schwachen mächtig.
Darum will ich mich am allerliebsten rühmen meiner Schwachheit, auf dass
die Kraft Christi bei mir wohne.

\bibleverse{10} Darum bin ich gutes Muts in Schwachheiten, in
Misshandlungen, in Nöten, in Verfolgungen, in Ängsten, um Christi
willen; denn, wenn ich schwach bin, so bin ich stark. \bibleverse{11}
Ich bin ein Narr geworden über dem Rühmen; dazu habt ihr mich gezwungen.
Denn ich sollte von euch gelobt werden, sintemal ich nichts weniger bin,
als die „hohen`` Apostel sind, wiewohl ich nichts bin. \footnote{\textbf{12:11}
  2Kor 11,5} \bibleverse{12} Denn es sind ja eines Apostels Zeichen
unter euch geschehen mit aller Geduld, mit Zeichen und mit Wundern und
mit Taten. \footnote{\textbf{12:12} Röm 15,19; Hebr 2,4} \bibleverse{13}
Was ist's, darin ihr geringer seid denn die anderen Gemeinden, außer
dass ich selbst euch nicht habe beschwert? Vergebet mir diese Sünde!
\footnote{\textbf{12:13} 2Kor 11,7-9}

\bibleverse{14} Siehe, ich bin bereit, zum drittenmal zu euch zu kommen,
und will euch nicht beschweren; denn ich suche nicht das Eure, sondern
euch. Denn es sollen nicht die Kinder den Eltern Schätze sammeln,
sondern die Eltern den Kindern. \bibleverse{15} Ich aber will sehr gern
hingeben und hingegeben werden für eure Seelen; wiewohl ich euch gar
sehr liebe, und doch weniger geliebt werde. \bibleverse{16} Aber lass es
also sein, dass ich euch nicht habe beschwert; sondern, dieweil ich
tückisch bin, habe ich euch mit Hinterlist gefangen. \bibleverse{17}
Habe ich aber etwa jemand übervorteilt durch derer einen, die ich zu
euch gesandt habe? \bibleverse{18} Ich habe Titus ermahnt und mit ihm
gesandt einen Bruder. Hat euch etwa Titus übervorteilt? Haben wir nicht
in einem Geist gewandelt? Sind wir nicht in einerlei Fußtapfen gegangen?
\footnote{\textbf{12:18} 2Kor 8,6-18}

\bibleverse{19} Lasset ihr euch abermals dünken, wir verantworten uns
vor euch? Wir reden in Christo vor Gott; aber das alles geschieht, meine
Liebsten, euch zur Besserung. \bibleverse{20} Denn ich fürchte, wenn ich
komme, dass ich euch nicht finde, wie ich will, und ihr mich auch nicht
findet, wie ihr wollt; dass Hader, Neid, Zorn, Zank, Afterreden,
Ohrenblasen, Aufblähen, Aufruhr dasei; \bibleverse{21} dass mich, wenn
ich abermals komme, mein Gott demütige bei euch und ich müsse Leid
tragen über viele, die zuvor gesündigt und nicht Buße getan haben für
die Unreinigkeit und Hurerei und Unzucht, die sie getrieben haben.
\footnote{\textbf{12:21} 2Kor 2,1; 2Kor 13,2}

\hypertarget{section-7}{%
\section{13}\label{section-7}}

\bibleverse{1} Komme ich zum drittenmal zu euch, so soll in zweier oder
dreier Zeugen Mund bestehen allerlei Sache. \footnote{\textbf{13:1} 2Kor
  10,2; Mt 18,16} \bibleverse{2} Ich habe es euch zuvor gesagt und sage
es euch zuvor, wie, als ich zum andernmal gegenwärtig war, so auch nun
abwesend schreibe ich es denen, die zuvor gesündigt haben, und den
anderen allen: Wenn ich abermals komme, so will ich nicht schonen;
\bibleverse{3} sintemal ihr suchet, dass ihr einmal gewahr werdet des,
der in mir redet, nämlich Christi, welcher unter euch nicht schwach ist,
sondern ist mächtig unter euch. \bibleverse{4} Und ob er wohl gekreuzigt
ist in der Schwachheit, so lebt er doch in der Kraft Gottes. Und ob wir
auch schwach sind in ihm, so leben wir doch mit ihm in der Kraft Gottes
unter euch.

\bibleverse{5} Versuchet euch selbst, ob ihr im Glauben seid; prüfet
euch selbst! Oder erkennet ihr euch selbst nicht, dass Jesus Christus in
euch ist? Es sei denn, dass ihr untüchtig seid. \bibleverse{6} Ich hoffe
aber, ihr erkennet, dass wir nicht untüchtig sind.

\bibleverse{7} Ich bitte aber Gott, dass ihr nichts Übles tut; nicht,
auf dass wir als tüchtig angesehen werden, sondern auf dass ihr das Gute
tut und wir wie die Untüchtigen seien. \bibleverse{8} Denn wir können
nichts wider die Wahrheit, sondern für die Wahrheit. \bibleverse{9} Wir
freuen uns aber, wenn wir schwach sind, und ihr mächtig seid. Und
dasselbe wünschen wir auch, nämlich eure Vollkommenheit. \bibleverse{10}
Derhalben schreibe ich auch solches abwesend, auf dass ich nicht, wenn
ich gegenwärtig bin, Schärfe brauchen müsse nach der Macht, welche mir
der Herr, zu bessern und nicht zu verderben, gegeben hat. \footnote{\textbf{13:10}
  2Kor 10,8; 2Kor 10,11}

\bibleverse{11} Zuletzt, liebe Brüder, freuet euch, seid vollkommen,
tröstet euch, habt einerlei Sinn, seid friedsam! so wird der Gott der
Liebe und des Friedens mit euch sein. \footnote{\textbf{13:11} Röm
  15,33; Phil 4,4} \bibleverse{12} Grüßet euch untereinander mit dem
heiligen Kuss. Es grüßen euch alle Heiligen.

\bibleverse{13} Die Gnade unseres Herrn Jesu Christi und die Liebe
Gottes und die Gemeinschaft des heiligen Geistes sei mit euch allen!
Amen.
