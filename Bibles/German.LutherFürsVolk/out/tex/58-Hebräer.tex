\hypertarget{section}{%
\section{1}\label{section}}

\bibleverse{1} Nachdem vorzeiten Gott manchmal und mancherleiweise
geredet hat zu den Vätern durch die Propheten, \bibleverse{2} hat er am
letzten in diesen Tagen zu uns geredet durch den Sohn, welchen er
gesetzt hat zum Erben über alles, durch welchen er auch die Welt gemacht
hat; \bibleverse{3} welcher, sintemal er ist der Glanz seiner
Herrlichkeit und das Ebenbild seines Wesens und trägt alle Dinge mit
seinem kräftigen Wort und hat gemacht die Reinigung unserer Sünden durch
sich selbst, hat er sich gesetzt zu der Rechten der Majestät in der Höhe
\footnote{\textbf{1:3} 2Kor 4,4; Kol 1,15; Hebr 9,14; Hebr 9,26; Mk
  16,19} \bibleverse{4} und ist so viel besser geworden denn die Engel,
so viel höher der Name ist, den er vor ihnen ererbt hat. \footnote{\textbf{1:4}
  Phil 2,9; 1Petr 3,22} \bibleverse{5} Denn zu welchem Engel hat er
jemals gesagt: „Du bist mein lieber Sohn, heute habe ich dich gezeugt``?
und abermals: „Ich werde sein Vater sein, und er wird mein Sohn sein``?

\bibleverse{6} Und abermals, da er einführt den Erstgeborenen in die
Welt, spricht er: „Und es sollen ihn alle Engel Gottes anbeten.``
\footnote{\textbf{1:6} Röm 8,29} \bibleverse{7} Von den Engeln spricht
er zwar: „Er macht seine Engel zu Winden und seine Diener zu
Feuerflammen``,

\bibleverse{8} aber von dem Sohn: „Gott, dein Stuhl währt von Ewigkeit
zu Ewigkeit; das Zepter deines Reichs ist ein richtiges Zepter.
\bibleverse{9} Du hast geliebt die Gerechtigkeit und gehasst die
Ungerechtigkeit; darum hat dich, o Gott, gesalbt dein Gott mit dem Öl
der Freuden über deine Genossen.``

\bibleverse{10} Und: „Du, Herr, hast von Anfang die Erde gegründet, und
die Himmel sind deiner Hände Werk. \bibleverse{11} Sie werden vergehen,
du aber wirst bleiben. Und sie werden alle veralten wie ein Kleid;
\bibleverse{12} und wie ein Gewand wirst du sie wandeln, und sie werden
sich verwandeln. Du aber bist derselbe, und deine Jahre werden nicht
aufhören.``

\bibleverse{13} Zu welchem Engel aber hat er jemals gesagt: „Setze dich
zu meiner Rechten, bis ich lege deine Feinde zum Schemel deiner Füße``?

\bibleverse{14} Sind sie nicht allzumal dienstbare Geister, ausgesandt
zum Dienst um derer willen, die ererben sollen die Seligkeit? \# 2
\bibleverse{1} Darum sollen wir desto mehr wahrnehmen des Worts, das wir
hören, damit wir nicht dahinfahren. \bibleverse{2} Denn so das Wort fest
geworden ist, das durch die Engel geredet ist, und eine jegliche
Übertretung und jeder Ungehorsam seinen rechten Lohn empfangen hat,
\footnote{\textbf{2:2} Apg 7,53; Gal 3,19} \bibleverse{3} wie wollen wir
entfliehen, so wir eine solche Seligkeit nicht achten? welche, nachdem
sie zuerst gepredigt ist durch den Herrn, auf uns gekommen ist durch
die, die es gehört haben; \footnote{\textbf{2:3} Hebr 10,29; Hebr 12,25}
\bibleverse{4} und Gott hat ihr Zeugnis gegeben mit Zeichen, Wundern und
mancherlei Kräften und mit Austeilung des heiligen Geistes nach seinem
Willen. \footnote{\textbf{2:4} Mk 16,20; 1Kor 12,4-11; 2Kor 12,12; Apg
  1,2-13; Apg 10,44-45}

\bibleverse{5} Denn er hat nicht den Engeln untergetan die zukünftige
Welt, davon wir reden. \bibleverse{6} Es bezeugt aber einer an einem Ort
und spricht: „Was ist der Mensch, dass du sein gedenkest, und des
Menschen Sohn, dass du auf ihn achtest? \bibleverse{7} Du hast ihn eine
kleine Zeit niedriger sein lassen denn die Engel; mit Preis und Ehre
hast du ihn gekrönt und hast ihn gesetzt über die Werke deiner Hände;
\bibleverse{8} alles hast du unter seine Füße getan.`` In dem, dass er
ihm alles hat untergetan, hat er nichts gelassen, das ihm nicht untertan
sei; jetzt aber sehen wir noch nicht, dass ihm alles untertan sei.

\bibleverse{9} Den aber, der eine kleine Zeit niedriger gewesen ist als
die Engel, Jesum, sehen wir durchs Leiden des Todes gekrönt mit Preis
und Ehre, auf dass er von Gottes Gnaden für alle den Tod schmeckte.

\bibleverse{10} Denn es ziemte dem, um deswillen alle Dinge sind und
durch den alle Dinge sind, der da viel Kinder hat zur Herrlichkeit
geführt, dass er den Herzog ihrer Seligkeit durch Leiden vollkommen
machte. \footnote{\textbf{2:10} Hebr 12,2} \bibleverse{11} Sintemal sie
alle von einem kommen, beide, der da heiligt und die da geheiligt
werden. Darum schämt er sich auch nicht, sie Brüder zu heißen,
\footnote{\textbf{2:11} Mk 3,34-35; Joh 17,19; Joh 20,17}
\bibleverse{12} und spricht: „Ich will verkündigen deinen Namen meinen
Brüdern und mitten in der Gemeinde dir lobsingen.``

\bibleverse{13} Und abermals: „Ich will mein Vertrauen auf ihn setzen.``
Und abermals: „Siehe da, ich und die Kinder, welche mir Gott gegeben
hat.`` \bibleverse{14} Nachdem nun die Kinder Fleisch und Blut haben,
ist er dessen gleichermaßen teilhaftig geworden, auf dass er durch den
Tod die Macht nehme dem, der des Todes Gewalt hatte, das ist dem Teufel,
\footnote{\textbf{2:14} 2Tim 1,10; 1Jo 3,8} \bibleverse{15} und erlösete
die, die durch Furcht des Todes im ganzen Leben Knechte sein mussten.
\bibleverse{16} Denn er nimmt sich ja nicht der Engel an, sondern des
Samens Abrahams nimmt er sich an. \bibleverse{17} Daher musste er in
allen Dingen seinen Brüdern gleich werden, auf dass er barmherzig würde
und ein treuer Hoherpriester vor Gott, zu versöhnen die Sünden des
Volks. \bibleverse{18} Denn worin er gelitten hat und versucht ist, kann
er helfen denen, die versucht werden. \footnote{\textbf{2:18} Hebr 4,15}

\hypertarget{section-1}{%
\section{3}\label{section-1}}

\bibleverse{1} Derhalben, ihr heiligen Brüder, die ihr mit berufen seid
durch die himmlische Berufung, nehmet wahr des Apostels und
Hohenpriesters, den wir bekennen, Christus Jesus, \footnote{\textbf{3:1}
  Hebr 4,14} \bibleverse{2} der da treu ist dem, der ihn gemacht hat,
wie auch Mose in seinem ganzen Hause. \footnote{\textbf{3:2} 4Mo 12,7}
\bibleverse{3} Dieser aber ist größerer Ehre wert denn Mose, soviel
größere Ehre denn das Haus der hat, der es bereitete. \bibleverse{4}
Denn ein jeglich Haus wird von jemand bereitet; der aber alles bereitet
hat, das ist Gott. \bibleverse{5} Und Mose war zwar treu in seinem
ganzen Hause als ein Knecht, zum Zeugnis des, das gesagt sollte werden,
\bibleverse{6} Christus aber als ein Sohn über sein Haus; des Haus sind
wir, so wir anders das Vertrauen und den Ruhm der Hoffnung bis ans Ende
fest behalten. \bibleverse{7} Darum, wie der heilige Geist spricht:
„Heute, so ihr hören werdet seine Stimme, \footnote{\textbf{3:7} Hebr
  4,7} \bibleverse{8} so verstocket eure Herzen nicht, wie geschah in
der Verbitterung am Tage der Versuchung in der Wüste, \footnote{\textbf{3:8}
  2Mo 17,7; 4Mo 20,2-5} \bibleverse{9} da mich eure Väter versuchten;
sie prüften mich und sahen meine Werke vierzig Jahre lang.
\bibleverse{10} Darum ward ich entrüstet über dies Geschlecht und
sprach: Immerdar irren sie mit dem Herzen! Aber sie erkannten meine Wege
nicht, \bibleverse{11} dass ich auch schwur in meinem Zorn, sie sollten
zu meiner Ruhe nicht kommen.`` \footnote{\textbf{3:11} Hebr 4,3; 4Mo
  14,21-23}

\bibleverse{12} Sehet zu, liebe Brüder, dass nicht jemand unter euch ein
arges, ungläubiges Herz habe, das da abtrete von dem lebendigen Gott;
\bibleverse{13} sondern ermahnet euch selbst alle Tage, solange es
„heute`` heißt, dass nicht jemand unter euch verstockt werde durch
Betrug der Sünde. \bibleverse{14} Denn wir sind Christi teilhaftig
geworden, so wir anders das angefangene Wesen bis ans Ende fest
behalten. \footnote{\textbf{3:14} Hebr 6,11} \bibleverse{15} Indem
gesagt wird: „Heute, so ihr seine Stimme hören werdet, so verstocket
eure Herzen nicht, wie in der Verbitterung geschah``, --

\bibleverse{16} welche denn hörten sie und richteten eine Verbitterung
an? Waren's nicht alle, die von Ägypten ausgingen durch Mose?
\bibleverse{17} Über welche aber ward er entrüstet vierzig Jahre lang?
Ist's nicht über die, die da sündigten, deren Leiber in der Wüste
verfielen? \bibleverse{18} Welchen schwur er aber, dass sie nicht zu
seiner Ruhe kommen sollten, wenn nicht den Ungläubigen? \bibleverse{19}
Und wir sehen, dass sie nicht haben können hineinkommen um des
Unglaubens willen. \# 4 \bibleverse{1} So lasset uns nun fürchten, dass
wir die Verheißung, einzukommen zu seiner Ruhe, nicht versäumen und
unser keiner dahintenbleibe. \bibleverse{2} Denn es ist uns auch
verkündigt gleichwie jenen; aber das Wort der Predigt half jenen nichts,
da nicht glaubten die, die es hörten. \bibleverse{3} Denn wir, die wir
glauben, gehen in die Ruhe, wie er spricht: „Dass ich schwur in meinem
Zorn, sie sollten zu meiner Ruhe nicht kommen.`` Und zwar, da die Werke
von Anbeginn der Welt gemacht waren, \bibleverse{4} sprach er an einem
Ort von dem siebenten Tag also: „Und Gott ruhte am siebenten Tage von
allen seinen Werken;`` \bibleverse{5} und hier an diesem Ort abermals:
„Sie sollen nicht kommen zu meiner Ruhe.``

\bibleverse{6} Nachdem es nun noch vorhanden ist, dass etliche sollen zu
ihr kommen, und die, denen es zuerst verkündigt ist, sind nicht dazu
gekommen um des Unglaubens willen, \bibleverse{7} bestimmt er abermals
einen Tag nach solcher langen Zeit und sagt durch David: „Heute,`` wie
gesagt ist, „heute, wenn ihr seine Stimme hören werdet, so verstocket
eure Herzen nicht.`` \footnote{\textbf{4:7} Hebr 3,7}

\bibleverse{8} Denn wenn Josua sie hätte zur Ruhe gebracht, würde er
nicht hernach von einem anderen Tage gesagt haben. \footnote{\textbf{4:8}
  5Mo 31,7; Jos 22,4} \bibleverse{9} Darum ist noch eine Ruhe vorhanden
dem Volke Gottes. \bibleverse{10} Denn wer zu seiner Ruhe gekommen ist,
der ruht auch von seinen Werken gleichwie Gott von seinen. \footnote{\textbf{4:10}
  Offb 14,13} \bibleverse{11} So lasset uns nun Fleiß tun, einzukommen
zu dieser Ruhe, auf dass nicht jemand falle in dasselbe Beispiel des
Unglaubens. \footnote{\textbf{4:11} Hebr 3,16-19} \bibleverse{12} Denn
das Wort Gottes ist lebendig und kräftig und schärfer denn kein
zweischneidig Schwert, und dringt durch, bis dass es scheidet Seele und
Geist, auch Mark und Bein, und ist ein Richter der Gedanken und Sinne
des Herzens. \footnote{\textbf{4:12} Jer 23,29; Offb 2,12}
\bibleverse{13} Und keine Kreatur ist vor ihm unsichtbar; es ist aber
alles bloß und entdeckt vor seinen Augen. Von dem reden wir.

\bibleverse{14} Dieweil wir denn einen großen Hohenpriester haben,
Jesum, den Sohn Gottes, der gen Himmel gefahren ist, so lasset uns
halten an dem Bekenntnis. \bibleverse{15} Denn wir haben nicht einen
Hohenpriester, der nicht könnte Mitleiden haben mit unseren
Schwachheiten, sondern der versucht ist allenthalben gleichwie wir, doch
ohne Sünde. \footnote{\textbf{4:15} Hebr 2,18; Joh 8,46} \bibleverse{16}
Darum lasset uns hinzutreten mit Freudigkeit zu dem Gnadenstuhl, auf
dass wir Barmherzigkeit empfangen und Gnade finden auf die Zeit, wenn
uns Hilfe not sein wird. \footnote{\textbf{4:16} Röm 3,25; Röm 5,2}

\hypertarget{section-2}{%
\section{5}\label{section-2}}

\bibleverse{1} Denn ein jeglicher Hoherpriester, der aus den Menschen
genommen wird, der wird gesetzt für die Menschen gegen Gott, auf dass er
opfere Gaben und Opfer für die Sünden; \bibleverse{2} der da könnte
mitfühlen mit denen, die da unwissend sind und irren, dieweil er auch
selbst umgeben ist mit Schwachheit. \bibleverse{3} Darum muss er auch,
gleichwie für das Volk, also auch für sich selbst opfern für die Sünden.
\footnote{\textbf{5:3} 3Mo 9,7} \bibleverse{4} Und niemand nimmt sich
selbst die Ehre, sondern er wird berufen von Gott gleichwie Aaron.
\footnote{\textbf{5:4} 2Mo 28,1} \bibleverse{5} Also auch Christus hat
sich nicht selbst in die Ehre gesetzt, dass er Hoherpriester würde,
sondern der zu ihm gesagt hat: „Du bist mein lieber Sohn, heute habe ich
dich gezeuget.``

\bibleverse{6} Wie er auch am anderen Ort spricht: „Du bist ein Priester
in Ewigkeit nach der Ordnung Melchisedeks.`` \footnote{\textbf{5:6} Hebr
  6,20}

\bibleverse{7} Und er hat in den Tagen seines Fleisches Gebet und Flehen
mit starkem Geschrei und Tränen geopfert zu dem, der ihm von dem Tode
konnte aushelfen; und ist auch erhört, darum dass er Gott in Ehren
hatte. \footnote{\textbf{5:7} Mt 26,39-46} \bibleverse{8} Und wiewohl er
Gottes Sohn war, hat er doch an dem, was er litt, Gehorsam gelernt.
\footnote{\textbf{5:8} Phil 2,8} \bibleverse{9} Und da er vollendet war,
ist er geworden allen, die ihm gehorsam sind, eine Ursache zur ewigen
Seligkeit, \bibleverse{10} genannt von Gott ein Hoherpriester nach der
Ordnung Melchisedeks.

\bibleverse{11} Davon hätten wir wohl viel zu reden; aber es ist schwer,
weil ihr so unverständig seid. \bibleverse{12} Und die ihr solltet
längst Meister sein, bedürfet wiederum, dass man euch die ersten
Buchstaben der göttlichen Worte lehre und dass man euch Milch gebe und
nicht starke Speise. \footnote{\textbf{5:12} 1Kor 3,1-3; 1Petr 2,2}
\bibleverse{13} Denn wem man noch Milch geben muss, der ist unerfahren
in dem Wort der Gerechtigkeit; denn er ist ein junges Kind. \footnote{\textbf{5:13}
  Eph 4,14} \bibleverse{14} Den Vollkommenen aber gehört starke Speise,
die durch Gewohnheit haben geübte Sinne, zu unterscheiden Gutes und
Böses. \# 6 \bibleverse{1} Darum wollen wir die Lehre vom Anfang
christlichen Lebens jetzt lassen und zur Vollkommenheit fahren, nicht
abermals Grund legen von Buße der toten Werke, vom Glauben an Gott,
\bibleverse{2} von der Taufe, von der Lehre, vom Händeauflegen, von der
Toten Auferstehung und vom ewigen Gericht. \bibleverse{3} Und das wollen
wir tun, so es Gott anders zulässt. \bibleverse{4} Denn es ist
unmöglich, die, die einmal erleuchtet sind und geschmeckt haben die
himmlische Gabe und teilhaftig geworden sind des heiligen Geistes
\footnote{\textbf{6:4} Hebr 10,26-29; 2Petr 2,20} \bibleverse{5} und
geschmeckt haben das gütige Wort Gottes und die Kräfte der zukünftigen
Welt, -- \bibleverse{6} wenn sie abfallen, wiederum zu erneuern zur
Buße, als die sich selbst den Sohn Gottes wiederum kreuzigen und für
Spott halten. \bibleverse{7} Denn die Erde, die den Regen trinkt, der
oft über sie kommt, und nützliches Kraut trägt denen, die sie bauen,
empfängt Segen von Gott. \bibleverse{8} Welche aber Dornen und Disteln
trägt, die ist untüchtig und dem Fluch nahe, dass man sie zuletzt
verbrennt.

\bibleverse{9} Wir versehen uns aber, ihr Liebsten, eines Besseren zu
euch und dass die Seligkeit näher sei, ob wir wohl also reden.
\bibleverse{10} Denn Gott ist nicht ungerecht, dass er vergesse eures
Werks und der Arbeit der Liebe, die ihr erzeigt habt an seinem Namen, da
ihr den Heiligen dientet und noch dienet. \bibleverse{11} Wir begehren
aber, dass euer jeglicher denselben Fleiß beweise, die Hoffnung
festzuhalten bis ans Ende, \footnote{\textbf{6:11} Hebr 3,14; Phil 1,6}
\bibleverse{12} dass ihr nicht träge werdet, sondern Nachfolger derer,
die durch Glauben und Geduld ererben die Verheißungen.

\bibleverse{13} Denn als Gott Abraham verhieß, da er bei keinem Größeren
zu schwören hatte, schwur er bei sich selbst \bibleverse{14} und sprach:
„Wahrlich, ich will dich segnen und vermehren.`` \bibleverse{15} Und
also trug er Geduld und erlangte die Verheißung. \bibleverse{16} Die
Menschen schwören ja bei einem Größeren, denn sie sind; und der Eid
macht ein Ende alles Haders, dabei es fest bleibt unter ihnen.
\bibleverse{17} So hat Gott, da er wollte den Erben der Verheißung
überschwenglich beweisen, dass sein Rat nicht wankte, einen Eid dazu
getan, \bibleverse{18} auf dass wir durch zwei Stücke, die nicht wanken
(denn es ist unmöglich, dass Gott lüge), einen starken Trost hätten, die
wir Zuflucht haben und halten an der angebotenen Hoffnung,
\bibleverse{19} welche wir haben als einen sicheren und festen Anker
unserer Seele, der auch hineingeht in das Inwendige des Vorhangs,
\footnote{\textbf{6:19} 3Mo 16,2; 3Mo 16,12} \bibleverse{20} dahin der
Vorläufer für uns eingegangen, Jesus, ein Hoherpriester geworden in
Ewigkeit nach der Ordnung Melchisedeks. \footnote{\textbf{6:20} Hebr 5,6}

\hypertarget{section-3}{%
\section{7}\label{section-3}}

\bibleverse{1} Dieser Melchisedek aber war ein König von Salem, ein
Priester Gottes, des Allerhöchsten, der Abraham entgegenging, da er von
der Könige Schlacht wiederkam, und segnete ihn; \footnote{\textbf{7:1}
  1Mo 14,18-20} \bibleverse{2} welchem auch Abraham gab den Zehnten
aller Güter. Aufs erste wird er verdolmetscht: ein König der
Gerechtigkeit; darnach aber ist er auch ein König Salems, das ist: ein
König des Friedens; \bibleverse{3} ohne Vater, ohne Mutter, ohne
Geschlecht und hat weder Anfang der Tage noch Ende des Lebens: -- er ist
aber verglichen dem Sohn Gottes und bleibt Priester in Ewigkeit.

\bibleverse{4} Schauet aber, wie groß ist der, dem auch Abraham, der
Patriarch, den Zehnten gibt von der eroberten Beute! \bibleverse{5} Zwar
die Kinder Levi, die das Priestertum empfangen, haben ein Gebot, den
Zehnten vom Volk, das ist von ihren Brüdern, zu nehmen nach dem Gesetz,
wiewohl auch diese aus den Lenden Abrahams gekommen sind. \footnote{\textbf{7:5}
  4Mo 18,21} \bibleverse{6} Aber der, des Geschlecht nicht genannt wird
unter ihnen, der nahm den Zehnten von Abraham und segnete den, der die
Verheißungen hatte. \bibleverse{7} Nun ist's ohne alles Widersprechen
also, dass das Geringere von dem Besseren gesegnet wird; \bibleverse{8}
und hier nehmen die Zehnten die sterbenden Menschen, aber dort einer,
dem bezeugt wird, dass er lebe. \bibleverse{9} Und, dass ich also sage,
es ist auch Levi, der den Zehnten nimmt, verzehntet durch Abraham,
\bibleverse{10} denn er war ja noch in den Lenden des Vaters, da ihm
Melchisedek entgegenging.

\bibleverse{11} Ist nun die Vollkommenheit durch das levitische
Priestertum geschehen (denn unter demselben hat das Volk das Gesetz
empfangen), was ist denn weiter not zu sagen, dass ein anderer Priester
aufkommen solle nach der Ordnung Melchisedeks und nicht nach der Ordnung
Aarons? \bibleverse{12} Denn wo das Priestertum verändert wird, da muss
auch das Gesetz verändert werden. \bibleverse{13} Denn von dem solches
gesagt ist, der ist von einem anderen Geschlecht, aus welchem nie einer
des Altars gewartet hat. \bibleverse{14} Denn es ist ja offenbar, dass
von Juda aufgegangen ist unser Herr, zu welchem Geschlecht Mose nichts
geredet hat vom Priestertum. \bibleverse{15} Und es ist noch viel
klarer, so nach der Weise Melchisedeks ein anderer Priester aufkommt,
\bibleverse{16} welcher nicht nach dem Gesetz des fleischlichen Gebots
gemacht ist, sondern nach der Kraft des unendlichen Lebens.
\bibleverse{17} Denn er bezeugt: „Du bist ein Priester ewiglich nach der
Ordnung Melchisedeks.`` \footnote{\textbf{7:17} Hebr 5,6}

\bibleverse{18} Denn damit wird das vorige Gebot aufgehoben, darum dass
es zu schwach und nicht nütze war \bibleverse{19} (denn das Gesetz
konnte nichts vollkommen machen); und wird eingeführt eine bessere
Hoffnung, durch welche wir zu Gott nahen; \bibleverse{20} und dazu, was
viel ist, nicht ohne Eid. Denn jene sind ohne Eid Priester geworden,
\bibleverse{21} dieser aber mit dem Eid, durch den, der zu ihm spricht:
„Der Herr hat geschworen, und es wird ihn nicht gereuen: Du bist ein
Priester in Ewigkeit nach der Ordnung Melchisedeks.``

\bibleverse{22} Also eines so viel besseren Testaments Ausrichter ist
Jesus geworden.

\bibleverse{23} Und jener sind viele, die Priester wurden, darum dass
sie der Tod nicht bleiben ließ; \bibleverse{24} dieser aber hat darum,
dass er ewiglich bleibt, ein unvergängliches Priestertum.
\bibleverse{25} Daher kann er auch selig machen immerdar, die durch ihn
zu Gott kommen, und lebt immerdar und bittet für sie. \footnote{\textbf{7:25}
  Röm 8,34; 1Jo 2,1}

\bibleverse{26} Denn einen solchen Hohenpriester sollten wir haben, der
da wäre heilig, unschuldig, unbefleckt, von den Sündern abgesondert und
höher, denn der Himmel ist; \bibleverse{27} dem nicht täglich not wäre,
wie jenen Hohenpriestern, zuerst für eigene Sünden Opfer zu tun, darnach
für des Volks Sünden; denn das hat er getan einmal, da er sich selbst
opferte. \bibleverse{28} Denn das Gesetz macht Menschen zu
Hohenpriestern, die da Schwachheit haben; dieses Wort aber des Eides,
das nach dem Gesetz gesagt ward, setzt den Sohn ein, der ewig und
vollkommen ist. \# 8 \bibleverse{1} Das ist nun die Hauptsache, davon
wir reden: Wir haben einen solchen Hohenpriester, der da sitzt zu der
Rechten auf dem Stuhl der Majestät im Himmel \footnote{\textbf{8:1} Hebr
  4,14} \bibleverse{2} und ist ein Pfleger des Heiligen und der
wahrhaftigen Hütte, welche Gott aufgerichtet hat und kein Mensch.
\bibleverse{3} Denn ein jeglicher Hoherpriester wird eingesetzt, zu
opfern Gaben und Opfer. Darum muss auch dieser etwas haben, das er
opfere. \bibleverse{4} Wenn er nun auf Erden wäre, so wäre er nicht
Priester, dieweil da Priester sind, die nach dem Gesetz die Gaben
opfern, \bibleverse{5} welche dienen dem Vorbilde und dem Schatten des
Himmlischen; wie die göttliche Antwort zu Mose sprach, da er sollte die
Hütte vollenden: „Schaue zu,`` sprach er, „dass du machest alles nach
dem Bilde, das dir auf dem Berge gezeigt ist.`` \bibleverse{6} Nun aber
hat er ein besseres Amt erlangt, als der eines besseren Testaments
Mittler ist, welches auch auf besseren Verheißungen steht. \footnote{\textbf{8:6}
  Hebr 7,22}

\bibleverse{7} Denn so jenes, das erste, untadelig gewesen wäre, würde
nicht Raum zu einem anderen gesucht. \bibleverse{8} Denn er tadelt sie
und sagt: „Siehe, es kommen die Tage, spricht der Herr, dass ich über
das Haus Israel und über das Haus Juda ein neues Testament machen will;
\bibleverse{9} nicht nach dem Testament, das ich gemacht habe mit ihren
Vätern an dem Tage, da ich ihre Hand ergriff, sie auszuführen aus
Ägyptenland. Denn sie sind nicht geblieben in meinem Testament, so habe
ich ihrer auch nicht wollen achten, spricht der Herr. \footnote{\textbf{8:9}
  2Mo 19,5-6} \bibleverse{10} Denn das ist das Testament, das ich machen
will dem Hause Israel nach diesen Tagen, spricht der Herr: Ich will
geben mein Gesetz in ihren Sinn, und in ihr Herz will ich es schreiben,
und will ihr Gott sein, und sie sollen mein Volk sein. \bibleverse{11}
Und soll nicht lehren jemand seinen Nächsten noch jemand seinen Bruder
und sagen: Erkenne den Herrn! denn sie sollen mich alle kennen von dem
Kleinsten an bis zu dem Größten. \bibleverse{12} Denn ich will gnädig
sein ihrer Untugend und ihren Sünden, und ihrer Ungerechtigkeit will ich
nicht mehr gedenken.``

\bibleverse{13} Indem er sagt: „Ein neues``, macht er das erste alt. Was
aber alt und überjahrt ist, das ist nahe bei seinem Ende. \# 9
\bibleverse{1} Es hatte zwar auch das erste seine Rechte des
Gottesdienstes und das äußerliche Heiligtum. \bibleverse{2} Denn es war
da aufgerichtet das Vorderteil der Hütte, darin der Leuchter war und der
Tisch und die Schaubrote; und dies heißt das Heilige. \footnote{\textbf{9:2}
  2Mo 25,23; 2Mo 25,30-31} \bibleverse{3} Hinter dem anderen Vorhang
aber war die Hütte, die da heißt das Allerheiligste; \footnote{\textbf{9:3}
  2Mo 26,33} \bibleverse{4} die hatte das goldene Räuchfass und die Lade
des Testaments allenthalben mit Gold überzogen, in welcher war der
goldene Krug mit dem Himmelsbrot und die Rute Aarons, die gegrünt hatte,
und die Tafeln des Testaments; \footnote{\textbf{9:4} 2Mo 16,33; 2Mo
  25,10-22; 4Mo 17,23-25} \bibleverse{5} oben darüber aber waren die
Cherubim der Herrlichkeit, die überschatteten den Gnadenstuhl; von
welchen Dingen jetzt nicht zu sagen ist insonderheit.

\bibleverse{6} Da nun solches also zugerichtet war, gingen die Priester
allezeit in die vordere Hütte und richteten aus den Gottesdienst.
\bibleverse{7} In die andere aber ging nur einmal im Jahr allein der
Hohepriester, nicht ohne Blut, das er opferte für seine und des Volkes
Versehen. \bibleverse{8} Damit deutete der heilige Geist, dass noch
nicht offenbart wäre der Weg zum Heiligen, solange die vordere Hütte
stünde, \footnote{\textbf{9:8} Hebr 10,20} \bibleverse{9} welche ist ein
Gleichnis auf die gegenwärtige Zeit, nach welchem Gaben und Opfer
geopfert werden, die nicht können vollkommen machen nach dem Gewissen
den, der da Gottesdienst tut \footnote{\textbf{9:9} Hebr 7,19; Hebr
  10,1-2} \bibleverse{10} allein mit Speise und Trank und mancherlei
Taufen und äußerlicher Heiligkeit, die bis auf die Zeit der Besserung
sind aufgelegt. \footnote{\textbf{9:10} 3Mo 11,-1; 4Mo 19,-1}

\bibleverse{11} Christus aber ist gekommen, dass er sei ein
Hoherpriester der zukünftigen Güter, und ist durch eine größere und
vollkommenere Hütte, die nicht mit der Hand gemacht, das ist, die nicht
von dieser Schöpfung ist, \bibleverse{12} auch nicht durch der Böcke
oder Kälber Blut, sondern durch sein eigen Blut einmal in das Heilige
eingegangen und hat eine ewige Erlösung erfunden. \bibleverse{13} Denn
wenn der Ochsen und der Böcke Blut und die Asche von der Kuh, gesprengt,
heiligt die Unreinen zu der leiblichen Reinigkeit, \bibleverse{14}
wieviel mehr wird das Blut Christi, der sich selbst ohne allen Fehl
durch den ewigen Geist Gott geopfert hat, unser Gewissen reinigen von
den toten Werken, zu dienen dem lebendigen Gott! \footnote{\textbf{9:14}
  Hebr 1,3; 1Petr 1,18-19; 1Jo 1,7; Offb 1,5} \bibleverse{15} Und darum
ist er auch ein Mittler des neuen Testaments, auf dass durch den Tod,
der geschehen ist zur Erlösung von den Übertretungen, die unter dem
ersten Testament waren, die, die berufen sind, das verheißene ewige Erbe
empfangen. \footnote{\textbf{9:15} Hebr 12,24; 1Tim 2,5} \bibleverse{16}
Denn wo ein Testament ist, da muss der Tod geschehen des, der das
Testament machte. \bibleverse{17} Denn ein Testament wird fest durch den
Tod; es hat noch nicht Kraft, wenn der noch lebt, der es gemacht hat.
\bibleverse{18} Daher auch das erste nicht ohne Blut gestiftet ward.
\bibleverse{19} Denn als Mose ausgeredet hatte von allen Geboten nach
dem Gesetz zu allem Volk, nahm er Kälber- und Bocksblut mit Wasser und
Scharlachwolle und Isop und besprengte das Buch und alles Volk
\bibleverse{20} und sprach: „Das ist das Blut des Testaments, das Gott
euch geboten hat.`` \footnote{\textbf{9:20} 4Mo 19,6}

\bibleverse{21} Und die Hütte und alles Geräte des Gottesdienstes
besprengte er gleicherweise mit Blut. \footnote{\textbf{9:21} 3Mo 8,15;
  3Mo 8,19} \bibleverse{22} Und es wird fast alles mit Blut gereinigt
nach dem Gesetz; und ohne Blutvergießen geschieht keine Vergebung.
\footnote{\textbf{9:22} 3Mo 17,11}

\bibleverse{23} So mussten nun der himmlischen Dinge Vorbilder mit
solchem gereinigt werden; aber sie selbst, die himmlischen, müssen
bessere Opfer haben, denn jene waren. \bibleverse{24} Denn Christus ist
nicht eingegangen in das Heilige, das mit Händen gemacht ist (welches
ist ein Gegenbild des wahrhaftigen), sondern in den Himmel selbst, nun
zu erscheinen vor dem Angesicht Gottes für uns; \bibleverse{25} auch
nicht, dass er sich oftmals opfere, gleichwie der Hohepriester geht alle
Jahre in das Heilige mit fremdem Blut; \bibleverse{26} sonst hätte er
oft müssen leiden von Anfang der Welt her. Nun aber, am Ende der Welt,
ist er einmal erschienen, durch sein eigen Opfer die Sünde aufzuheben.
\footnote{\textbf{9:26} Hebr 1,3; 1Kor 10,11; Gal 4,4} \bibleverse{27}
Und wie den Menschen gesetzt ist, einmal zu sterben, darnach aber das
Gericht: \footnote{\textbf{9:27} 1Mo 3,19} \bibleverse{28} also ist auch
Christus einmal geopfert, wegzunehmen vieler Sünden; zum andernmal wird
er ohne Sünde erscheinen denen, die auf ihn warten, zur Seligkeit.
\footnote{\textbf{9:28} Hebr 10,10; Hebr 10,12; Hebr 10,14}

\hypertarget{section-4}{%
\section{10}\label{section-4}}

\bibleverse{1} Denn das Gesetz hat den Schatten von den zukünftigen
Gütern, nicht das Wesen der Güter selbst; alle Jahre muss man opfern
immer einerlei Opfer, und es kann nicht, die da opfern, vollkommen
machen; \footnote{\textbf{10:1} Hebr 8,5} \bibleverse{2} sonst hätte das
Opfern aufgehört, wenn die, die am Gottesdienst sind, kein Gewissen mehr
hätten von den Sünden, wenn sie einmal gereinigt wären; \bibleverse{3}
sondern es geschieht dadurch nur ein Gedächtnis der Sünden alle Jahre.
\footnote{\textbf{10:3} 3Mo 16,34} \bibleverse{4} Denn es ist unmöglich,
durch Ochsen- und Bocksblut Sünden wegzunehmen. \bibleverse{5} Darum, da
er in die Welt kommt, spricht er: „Opfer und Gaben hast du nicht
gewollt; den Leib aber hast du mir bereitet. \bibleverse{6} Brandopfer
und Sündopfer gefallen dir nicht. \bibleverse{7} Da sprach ich: Siehe,
ich komme (im Buch steht von mir geschrieben), dass ich tue, Gott,
deinen Willen.``

\bibleverse{8} Nachdem er weiter oben gesagt hatte: „Opfer und Gaben,
Brandopfer und Sündopfer hast du nicht gewollt, sie gefallen dir auch
nicht`` (welche nach dem Gesetz geopfert werden), \bibleverse{9} da
sprach er: „Siehe, ich komme, zu tun, Gott, deinen Willen.`` Da hebt er
das erste auf, dass er das andere einsetze. \bibleverse{10} In diesem
Willen sind wir geheiligt auf einmal durch das Opfer des Leibes Jesu
Christi.

\bibleverse{11} Und ein jeglicher Priester ist eingesetzt, dass er alle
Tage Gottesdienst pflege und oftmals einerlei Opfer tue, welche
nimmermehr können die Sünden abnehmen. \footnote{\textbf{10:11} 2Mo
  29,38} \bibleverse{12} Dieser aber, da er hat ein Opfer für die Sünden
geopfert, das ewiglich gilt, sitzt er nun zur Rechten Gottes
\bibleverse{13} und wartet hinfort, bis dass seine Feinde zum Schemel
seiner Füße gelegt werden. \bibleverse{14} Denn mit einem Opfer hat er
in Ewigkeit vollendet, die geheiligt werden. \bibleverse{15} Es bezeugt
uns aber das auch der heilige Geist. Denn nachdem er zuvor gesagt hatte:
\bibleverse{16} „Das ist das Testament, das ich ihnen machen will nach
diesen Tagen``, spricht der Herr: „Ich will mein Gesetz in ihr Herz
geben, und in ihren Sinn will ich es schreiben, \footnote{\textbf{10:16}
  Hebr 8,10}

\bibleverse{17} und ihrer Sünden und ihrer Ungerechtigkeit will ich
nicht mehr gedenken.`` \footnote{\textbf{10:17} Hebr 8,12}

\bibleverse{18} Wo aber derselben Vergebung ist, da ist nicht mehr Opfer
für die Sünde.

\bibleverse{19} So wir denn nun haben, liebe Brüder, die Freudigkeit zum
Eingang in das Heilige durch das Blut Jesu, \footnote{\textbf{10:19} Mt
  27,51; Röm 5,2} \bibleverse{20} welchen er uns bereitet hat zum neuen
und lebendigen Wege durch den Vorhang, das ist durch sein Fleisch,
\footnote{\textbf{10:20} Hebr 9,8} \bibleverse{21} und haben einen
Hohenpriester über das Haus Gottes: \bibleverse{22} so lasset uns
hinzugehen mit wahrhaftigem Herzen in völligem Glauben, besprengt in
unseren Herzen und los von dem bösen Gewissen und gewaschen am Leibe mit
reinem Wasser; \footnote{\textbf{10:22} Hebr 4,16; Eph 5,26; 1Petr 3,21}
\bibleverse{23} und lasset uns halten an dem Bekenntnis der Hoffnung und
nicht wanken; denn er ist treu, der sie verheißen hat; \footnote{\textbf{10:23}
  Hebr 4,14}

\bibleverse{24} und lasset uns untereinander unser selbst wahrnehmen mit
Reizen zur Liebe und guten Werken \bibleverse{25} und nicht verlassen
unsere Versammlung, wie etliche pflegen, sondern einander ermahnen; und
das so viel mehr, soviel ihr sehet, dass sich der Tag naht. \footnote{\textbf{10:25}
  Hebr 3,13; Röm 13,11-12}

\bibleverse{26} Denn wenn wir mutwillig sündigen, nachdem wir die
Erkenntnis der Wahrheit empfangen haben, haben wir fürder kein anderes
Opfer mehr für die Sünden, \footnote{\textbf{10:26} Hebr 6,4-8}
\bibleverse{27} sondern ein schreckliches Warten des Gerichts und des
Feuereifers, der die Widersacher verzehren wird. \bibleverse{28} Wenn
jemand das Gesetz Moses bricht, der muss sterben ohne Barmherzigkeit
durch zwei oder drei Zeugen. \footnote{\textbf{10:28} 4Mo 15,30; 5Mo
  17,6} \bibleverse{29} Wie viel, meinet ihr, ärgere Strafe wird der
verdienen, der den Sohn Gottes mit Füßen tritt und das Blut des
Testaments unrein achtet, durch welches er geheiligt ist, und den Geist
der Gnade schmäht? \footnote{\textbf{10:29} Hebr 2,3; Hebr 12,25}
\bibleverse{30} Denn wir kennen den, der da sagte: „Die Rache ist mein;
ich will vergelten``, und abermals: „Der Herr wird sein Volk richten.``
\bibleverse{31} Schrecklich ist's, in die Hände des lebendigen Gottes zu
fallen. \footnote{\textbf{10:31} Hebr 12,29}

\bibleverse{32} Gedenket aber an die vorigen Tage, in welchen ihr,
nachdem ihr erleuchtet waret, erduldet habt einen großen Kampf des
Leidens \footnote{\textbf{10:32} Hebr 6,4} \bibleverse{33} und zum Teil
selbst durch Schmach und Trübsal ein Schauspiel wurdet, zum Teil
Gemeinschaft hattet mit denen, welchen es also geht. \footnote{\textbf{10:33}
  1Kor 4,9} \bibleverse{34} Denn ihr habt mit den Gebundenen Mitleiden
gehabt und den Raub eurer Güter mit Freuden erduldet, als die ihr
wisset, dass ihr bei euch selbst eine bessere und bleibende Habe im
Himmel habt. \footnote{\textbf{10:34} Mt 6,20; Mt 19,21; Mt 19,29}
\bibleverse{35} Werfet euer Vertrauen nicht weg, welches eine große
Belohnung hat. \bibleverse{36} Geduld aber ist euch not, auf dass ihr
den Willen Gottes tut und die Verheißung empfanget. \footnote{\textbf{10:36}
  Lk 21,19; Jak 5,7} \bibleverse{37} Denn „noch über eine kleine Weile,
so wird kommen, der da kommen soll, und nicht verziehen. \bibleverse{38}
Der Gerechte aber wird des Glaubens leben. Wer aber weichen wird, an dem
wird meine Seele keinen Gefallen haben.``

\bibleverse{39} Wir aber sind nicht von denen, die da weichen und
verdammt werden, sondern von denen, die da glauben und die Seele
erretten. \footnote{\textbf{10:39} 1Thes 3,3}

\hypertarget{section-5}{%
\section{11}\label{section-5}}

\bibleverse{1} Es ist aber der Glaube eine gewisse Zuversicht des, das
man hofft, und ein Nichtzweifeln an dem, das man nicht sieht.
\footnote{\textbf{11:1} 2Kor 5,7} \bibleverse{2} Durch den haben die
Alten Zeugnis überkommen. \bibleverse{3} Durch den Glauben merken wir,
dass die Welt durch Gottes Wort fertig ist; dass alles, was man sieht,
aus nichts geworden ist. \footnote{\textbf{11:3} 1Mo 1,-1}

\bibleverse{4} Durch den Glauben hat Abel Gott ein größeres Opfer getan
denn Kain; durch welchen er Zeugnis überkommen hat, dass er gerecht sei,
da Gott zeugte von seiner Gabe; und durch denselben redet er noch,
wiewohl er gestorben ist. \footnote{\textbf{11:4} 1Mo 4,4}

\bibleverse{5} Durch den Glauben ward Henoch weggenommen, dass er den
Tod nicht sähe, und ward nicht gefunden, darum dass ihn Gott wegnahm;
denn vor seinem Wegnehmen hat er Zeugnis gehabt, dass er Gott gefallen
habe. \footnote{\textbf{11:5} 1Mo 5,24} \bibleverse{6} Aber ohne Glauben
ist's unmöglich, Gott zu gefallen; denn wer zu Gott kommen will, der
muss glauben, dass er sei und denen, die ihn suchen, ein Vergelter sein
werde.

\bibleverse{7} Durch den Glauben hat Noah Gott geehrt und die Arche
zubereitet zum Heil seines Hauses, da er ein göttliches Wort empfing
über das, was man noch nicht sah; und verdammte durch denselben die Welt
und hat ererbt die Gerechtigkeit, die durch den Glauben kommt.

\bibleverse{8} Durch den Glauben ward gehorsam Abraham, da er berufen
ward, auszugehen in das Land, das er ererben sollte; und ging aus und
wusste nicht, wo er hinkäme. \footnote{\textbf{11:8} 1Mo 12,1-21}
\bibleverse{9} Durch den Glauben ist er ein Fremdling gewesen in dem
verheißenen Lande als in einem fremden und wohnte in Hütten mit Isaak
und Jakob, den Miterben derselben Verheißung; \bibleverse{10} denn er
wartete auf eine Stadt, die einen Grund hat, deren Baumeister und
Schöpfer Gott ist.

\bibleverse{11} Durch den Glauben empfing auch Sara Kraft, dass sie
schwanger ward und gebar über die Zeit ihres Alters; denn sie achtete
ihn treu, der es verheißen hatte. \bibleverse{12} Darum sind auch von
einem, wiewohl erstorbenen Leibes, viele geboren wie die Sterne am
Himmel und wie der Sand am Rande des Meeres, der unzählig ist.

\bibleverse{13} Diese alle sind gestorben im Glauben und haben die
Verheißungen nicht empfangen, sondern sie von ferne gesehen und sich
ihrer getröstet und wohl genügen lassen und bekannt, dass sie Gäste und
Fremdlinge auf Erden wären. \bibleverse{14} Denn die solches sagen, die
geben zu verstehen, dass sie ein Vaterland suchen. \bibleverse{15} Und
zwar, wenn sie das gemeint hätten, von welchem sie waren ausgezogen,
hatten sie ja Zeit, wieder umzukehren. \bibleverse{16} Nun aber begehren
sie eines besseren, nämlich eines himmlischen. Darum schämt sich Gott
ihrer nicht, zu heißen ihr Gott; denn er hat ihnen eine Stadt
zubereitet. \footnote{\textbf{11:16} 2Mo 3,6}

\bibleverse{17} Durch den Glauben opferte Abraham den Isaak, da er
versucht ward, und gab dahin den Eingeborenen, da er schon die
Verheißungen empfangen hatte, \footnote{\textbf{11:17} 1Mo 22,-1; Jak
  2,21} \bibleverse{18} von welchem gesagt war: „In Isaak wird dir dein
Same genannt werden``; \bibleverse{19} und dachte, Gott kann auch wohl
von den Toten erwecken; daher er auch ihn zum Vorbilde wiederbekam.

\bibleverse{20} Durch den Glauben segnete Isaak von den zukünftigen
Dingen den Jakob und Esau. \footnote{\textbf{11:20} 1Mo 27,-1; 1Mo
  48,1-48; 1Mo 50,1-50}

\bibleverse{21} Durch den Glauben segnete Jakob, da er starb, beide
Söhne Josephs und neigte sich gegen seines Stabes Spitze.

\bibleverse{22} Durch den Glauben redete Joseph vom Auszug der Kinder
Israel, da er starb, und tat Befehl von seinen Gebeinen.

\bibleverse{23} Durch den Glauben ward Mose, da er geboren war, drei
Monate verborgen von seinen Eltern, darum dass sie sahen, wie er ein
schönes Kind war, und fürchteten sich nicht vor des Königs Gebot.

\bibleverse{24} Durch den Glauben wollte Mose, da er groß ward, nicht
mehr ein Sohn heißen der Tochter Pharaos, \bibleverse{25} und erwählte
viel lieber, mit dem Volk Gottes Ungemach zu leiden, denn die zeitliche
Ergötzung der Sünde zu haben, \bibleverse{26} und achtete die Schmach
Christi für größeren Reichtum denn die Schätze Ägyptens; denn er sah an
die Belohnung. \bibleverse{27} Durch den Glauben verließ er Ägypten und
fürchtete nicht des Königs Grimm; denn er hielt sich an den, den er
nicht sah, als sähe er ihn. \bibleverse{28} Durch den Glauben hielt er
Ostern und das Blutgießen, auf dass, der die Erstgeburten erwürgte, sie
nicht träfe.

\bibleverse{29} Durch den Glauben gingen sie durchs Rote Meer wie durch
trockenes Land; was die Ägypter auch versuchten, und ersoffen.

\bibleverse{30} Durch den Glauben fielen die Mauern Jerichos, da sie
sieben Tage um sie herumgegangen waren.

\bibleverse{31} Durch den Glauben ward die Hure Rahab nicht verloren mit
den Ungläubigen, da sie die Kundschafter freundlich aufnahm.

\bibleverse{32} Und was soll ich mehr sagen? Die Zeit würde mir zu kurz,
wenn ich sollte erzählen von Gideon und Barak und Simson und Jephthah
und David und Samuel und den Propheten, \bibleverse{33} welche haben
durch den Glauben Königreiche bezwungen, Gerechtigkeit gewirkt,
Verheißungen erlangt, der Löwen Rachen verstopft, \bibleverse{34} des
Feuers Kraft ausgelöscht, sind des Schwertes Schärfe entronnen, sind
kräftig geworden aus der Schwachheit, sind stark geworden im Streit,
haben der Fremden Heere darniedergelegt. \bibleverse{35} Weiber haben
ihre Toten durch Auferstehung wiederbekommen. Andere aber sind
zerschlagen und haben keine Erlösung angenommen, auf dass sie die
Auferstehung, die besser ist, erlangten. \bibleverse{36} Etliche haben
Spott und Geißeln erlitten, dazu Bande und Gefängnis; \bibleverse{37}
sie wurden gesteinigt, zerhackt, zerstochen, durchs Schwert getötet; sie
sind umhergegangen in Schafpelzen und Ziegenfellen, mit Mangel, mit
Trübsal, mit Ungemach \bibleverse{38} (deren die Welt nicht wert war),
und sind im Elend umhergeirrt in den Wüsten, auf den Bergen und in den
Klüften und Löchern der Erde.

\bibleverse{39} Diese alle haben durch den Glauben Zeugnis überkommen
und nicht empfangen die Verheißung, \bibleverse{40} darum dass Gott
etwas Besseres für uns zuvor ersehen hat, dass sie nicht ohne uns
vollendet würden. \# 12 \bibleverse{1} Darum auch wir, dieweil wir eine
solche Wolke von Zeugen um uns haben, lasset uns ablegen die Sünde, die
uns immer anklebt und träge macht, und lasset uns laufen durch Geduld in
dem Kampf, der uns verordnet ist, \footnote{\textbf{12:1} 1Kor 9,24}
\bibleverse{2} und aufsehen auf Jesum, den Anfänger und Vollender des
Glaubens; welcher, da er wohl hätte mögen Freude haben, erduldete das
Kreuz und achtete der Schande nicht und hat sich gesetzt zur Rechten auf
den Stuhl Gottes. \footnote{\textbf{12:2} Hebr 5,8-9; Phil 2,8; Phil
  2,10}

\bibleverse{3} Gedenket an den, der ein solches Widersprechen von den
Sündern wider sich erduldet hat, dass ihr nicht in eurem Mut matt werdet
und ablasset. \footnote{\textbf{12:3} Lk 2,34; Mt 26,67} \bibleverse{4}
Denn ihr habt noch nicht bis aufs Blut widerstanden in den Kämpfen wider
die Sünde \bibleverse{5} und habt bereits vergessen des Trostes, der zu
euch redet als zu den Kindern: „Mein Sohn, achte nicht gering die
Züchtigung des Herrn und verzage nicht, wenn du von ihm gestraft wirst.
\bibleverse{6} Denn welchen der Herr liebhat, den züchtigt er; und er
stäupt einen jeglichen Sohn, den er aufnimmt.``

\bibleverse{7} Wenn ihr die Züchtigung erduldet, so erbietet sich euch
Gott als Kindern; denn wo ist ein Sohn, den der Vater nicht züchtigt?
\bibleverse{8} Seid ihr aber ohne Züchtigung, welcher sie alle sind
teilhaftig geworden, so seid ihr Bastarde und nicht Kinder.
\bibleverse{9} Und wenn wir haben unsere leiblichen Väter zu Züchtigern
gehabt und sie gescheut, sollten wir denn nicht viel mehr untertan sein
dem Vater der Geister, dass wir leben? \bibleverse{10} Denn jene haben
uns gezüchtigt wenig Tage nach ihrem Dünken, dieser aber zu Nutz, auf
dass wir seine Heiligung erlangen. \bibleverse{11} Alle Züchtigung aber,
wenn sie da ist, dünkt uns nicht Freude, sondern Traurigkeit zu sein;
aber darnach wird sie geben eine friedsame Frucht der Gerechtigkeit
denen, die dadurch geübt sind. \footnote{\textbf{12:11} 2Kor 4,17-18}
\bibleverse{12} Darum richtet wieder auf die lässigen Hände und die
müden Knie \footnote{\textbf{12:12} Jes 35,3} \bibleverse{13} und tut
gewisse Tritte mit euren Füßen, dass nicht jemand strauchle wie ein
Lahmer, sondern vielmehr gesund werde. \footnote{\textbf{12:13} Spr
  4,26-27}

\bibleverse{14} Jaget nach -- dem Frieden gegen jedermann und der
Heiligung, ohne welche wird niemand den Herrn sehen, \footnote{\textbf{12:14}
  Röm 12,18; 2Tim 2,22} \bibleverse{15} und sehet darauf, dass nicht
jemand Gottes Gnade versäume; dass nicht etwa eine bittere Wurzel
aufwachse und Unfrieden anrichte und viele durch dieselbe verunreinigt
werden; \footnote{\textbf{12:15} 5Mo 29,17} \bibleverse{16} dass nicht
jemand sei ein Hurer oder ein Gottloser wie Esau, der um einer Speise
willen seine Erstgeburt verkaufte. \footnote{\textbf{12:16} 1Mo 25,33-34}
\bibleverse{17} Wisset aber, dass er hernach, da er den Segen ererben
wollte, verworfen ward; denn er fand keinen Raum zur Buße, wiewohl er
sie mit Tränen suchte. \footnote{\textbf{12:17} 1Mo 27,30-40}

\bibleverse{18} Denn ihr seid nicht gekommen zu dem Berge, den man
anrühren konnte und der mit Feuer brannte, noch zu dem Dunkel und
Finsternis und Ungewitter \footnote{\textbf{12:18} 2Mo 19,12; 2Mo 19,16;
  2Mo 19,18; 5Mo 4,11} \bibleverse{19} noch zu dem Hall der Posaune und
zur Stimme der Worte, da sich weigerten, die sie hörten, dass ihnen das
Wort ja nicht gesagt würde; \footnote{\textbf{12:19} 2Mo 20,19}
\bibleverse{20} denn sie mochten's nicht ertragen, was da gesagt ward:
„Und wenn ein Tier den Berg anrührt, soll es gesteinigt oder mit einem
Geschoss erschossen werden``; \bibleverse{21} und also erschrecklich war
das Gesicht, dass Mose sprach: Ich bin erschrocken und zittere.

\bibleverse{22} Sondern ihr seid gekommen zu dem Berge Zion und zu der
Stadt des lebendigen Gottes, dem himmlischen Jerusalem, und zu der Menge
vieler tausend Engel \bibleverse{23} und zu der Gemeinde der
Erstgeborenen, die im Himmel angeschrieben sind, und zu Gott, dem
Richter über alle, und zu den Geistern der vollendeten Gerechten
\footnote{\textbf{12:23} Lk 10,20} \bibleverse{24} und zu dem Mittler
des neuen Testaments, Jesus, und zu dem Blut der Besprengung, das da
besser redet denn das Abels. \footnote{\textbf{12:24} Hebr 9,15; 1Mo
  4,10}

\bibleverse{25} Sehet zu, dass ihr den nicht abweiset, der da redet.
Denn wenn jene nicht entflohen sind, die ihn abwiesen, da er auf Erden
redete, viel weniger wir, wenn wir den abweisen, der vom Himmel redet;
\footnote{\textbf{12:25} Hebr 2,2; Hebr 10,28-29} \bibleverse{26} dessen
Stimme zu der Zeit die Erde bewegte, nun aber verheißt er und spricht:
„Noch einmal will ich bewegen nicht allein die Erde, sondern auch den
Himmel.`` \bibleverse{27} Aber solches „Noch einmal`` zeigt an, dass das
Bewegliche soll verwandelt werden, als das gemacht ist, auf dass da
bleibe das Unbewegliche. \bibleverse{28} Darum, dieweil wir empfangen
ein unbeweglich Reich, haben wir Gnade, durch welche wir sollen Gott
dienen, ihm zu gefallen, mit Zucht und Furcht; \bibleverse{29} denn
unser Gott ist ein verzehrend Feuer. \# 13 \bibleverse{1} Bleibet fest
in der brüderlichen Liebe. \footnote{\textbf{13:1} Joh 13,34; 2Petr 1,7}
\bibleverse{2} Gastfrei zu sein vergesset nicht; denn dadurch haben
etliche ohne ihr Wissen Engel beherbergt. \footnote{\textbf{13:2} 1Mo
  18,3; 1Mo 19,2-3; Röm 12,13; 1Petr 4,9; 3Jo 1,5-8} \bibleverse{3}
Gedenket der Gebundenen als die Mitgebundenen und derer, die Trübsal
leiden, als die ihr auch noch im Leibe lebet. \footnote{\textbf{13:3} Mt
  25,36} \bibleverse{4} Die Ehe soll ehrlich gehalten werden bei allen
und das Ehebett unbefleckt; die Hurer aber und die Ehebrecher wird Gott
richten.

\bibleverse{5} Der Wandel sei ohne Geiz; und lasset euch genügen an dem,
was da ist. Denn er hat gesagt: „Ich will dich nicht verlassen noch
versäumen``; \bibleverse{6} also dass wir dürfen sagen: „Der Herr ist
mein Helfer, ich will mich nicht fürchten; was sollte mir ein Mensch
tun?{}``

\bibleverse{7} Gedenket an eure Lehrer, die euch das Wort Gottes gesagt
haben; ihr Ende schauet an und folget ihrem Glauben nach. \bibleverse{8}
Jesus Christus gestern und heute und derselbe auch in Ewigkeit.
\footnote{\textbf{13:8} Jes 41,4; Offb 1,17-18; Offb 22,13; 1Kor 3,11}
\bibleverse{9} Lasset euch nicht mit mancherlei und fremden Lehren
umtreiben; denn es ist ein köstlich Ding, dass das Herz fest werde,
welches geschieht durch Gnade, nicht durch Speisen, davon keinen Nutzen
haben, die damit umgehen. \footnote{\textbf{13:9} 2Kor 1,21; 1Tim 4,8;
  Röm 14,17; Eph 4,14}

\bibleverse{10} Wir haben einen Altar, davon nicht Macht haben zu essen,
die der Hütte pflegen. \bibleverse{11} Denn welcher Tiere Blut getragen
wird durch den Hohenpriester in das Heilige für die Sünde, deren
Leichname werden verbrannt außerhalb des Lagers. \footnote{\textbf{13:11}
  3Mo 7,6; 3Mo 16,27} \bibleverse{12} Darum hat auch Jesus, auf dass er
heiligte das Volk durch sein eigen Blut, gelitten draußen vor dem Tor.
\footnote{\textbf{13:12} Joh 19,17; Mt 21,39} \bibleverse{13} So lasst
uns nun zu ihm hinausgehen aus dem Lager und seine Schmach tragen.
\footnote{\textbf{13:13} Hebr 11,26; Hebr 12,2} \bibleverse{14} Denn wir
haben hier keine bleibende Stadt, sondern die zukünftige suchen wir.
\footnote{\textbf{13:14} Hebr 11,10; Hebr 12,22} \bibleverse{15} So
lasset uns nun opfern durch ihn das Lobopfer Gott allezeit, das ist die
Frucht der Lippen, die seinen Namen bekennen. \footnote{\textbf{13:15}
  Hos 14,3; Ps 50,14; Ps 50,23} \bibleverse{16} Wohlzutun und
mitzuteilen vergesset nicht; denn solche Opfer gefallen Gott wohl.

\bibleverse{17} Gehorchet euren Lehrern und folget ihnen; denn sie
wachen über eure Seelen, als die da Rechenschaft dafür geben sollen; auf
dass sie das mit Freuden tun und nicht mit Seufzen; denn das ist euch
nicht gut.

\bibleverse{18} Betet für uns. Unser Trost ist der, dass wir ein gutes
Gewissen haben und fleißigen uns, guten Wandel zu führen bei allen.
\footnote{\textbf{13:18} Röm 15,30; 2Kor 1,11-12} \bibleverse{19} Ich
ermahne aber desto mehr, solches zu tun, auf dass ich umso schneller
wieder zu euch komme.

\bibleverse{20} Der Gott aber des Friedens, der von den Toten ausgeführt
hat den großen Hirten der Schafe durch das Blut des ewigen Testaments,
unseren Herrn Jesus, \bibleverse{21} der mache euch fertig in allem
guten Werk, zu tun seinen Willen, und schaffe in euch, was vor ihm
gefällig ist, durch Jesum Christum; welchem sei Ehre von Ewigkeit zu
Ewigkeit! Amen.

\bibleverse{22} Ich ermahne euch aber, liebe Brüder, haltet das Wort der
Ermahnung zugute; denn ich habe euch kurz geschrieben. \bibleverse{23}
Wisset, dass der Bruder Timotheus wieder frei ist; mit dem, so er bald
kommt, will ich euch sehen.

\bibleverse{24} Grüßet alle eure Lehrer und alle Heiligen. Es grüßen
euch die Brüder aus Italien.

\bibleverse{25} Die Gnade sei mit euch allen! Amen.
