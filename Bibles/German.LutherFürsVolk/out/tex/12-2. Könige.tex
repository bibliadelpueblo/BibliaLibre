\hypertarget{section}{%
\section{1}\label{section}}

\bibverse{1} Es fielen aber die Moabiter ab von Israel, da Ahab tot war.
\footnote{\textbf{1:1} 2Kö 3,5} \bibverse{2} Und Ahasja fiel durch das
Gitter in seinem Söller zu Samaria und ward krank; und sandte Boten und
sprach zu ihnen: Gehet hin und fragt Baal-Sebub, den Gott zu Ekron, ob
ich von dieser Krankheit genesen werde. \footnote{\textbf{1:2} 1Kö
  22,52; Jes 19,3} \bibverse{3} Aber der Engel des HErrn redete mit
Elia, dem Thisbiter: Auf! und begegne den Boten des Königs zu Samaria
und sprich zu ihnen: Ist denn nun kein Gott in Israel, dass ihr
hingehet, zu fragen Baal-Sebub, den Gott Ekrons? \footnote{\textbf{1:3}
  Jes 8,19} \bibverse{4} Darum so spricht der HErr: Du sollst nicht von
dem Bette kommen, darauf du dich gelegt hast, sondern sollst des Todes
sterben. Und Elia ging weg. \bibverse{5} Und da die Boten wieder zu ihm
kamen, sprach er zu ihnen: Warum kommt ihr wieder? \bibverse{6} Sie
sprachen zu ihm: Es kam ein Mann herauf uns entgegen und sprach zu uns:
Gehet wiederum hin zu dem König, der euch gesandt hat, und sprecht zu
ihm: So spricht der HErr: Ist denn kein Gott in Israel, dass du
hinsendest, zu fragen Baal-Sebub, den Gott Ekrons? Darum sollst du nicht
kommen von dem Bette, darauf du dich gelegt hast, sondern sollst des
Todes sterben. \bibverse{7} Er sprach zu ihnen: Wie war der Mann
gestaltet, der euch begegnete und solches zu euch sagte? \bibverse{8}
Sie sprachen zu ihm: Er hatte eine rauhe Haut an und einen ledernen
Gürtel um seine Lenden. Er aber sprach: Es ist Elia, der Thisbiter.
\footnote{\textbf{1:8} Sach 13,4; Mt 3,4} \bibverse{9} Und er sandte hin
zu ihm einen Hauptmann über 50 samt seinen fünfzigen. Und da der zu ihm
hinaufkam, siehe, da saß er oben auf dem Berge. Er aber sprach zu ihm:
Du Mann Gottes, der König sagt: Du sollst herabkommen! \bibverse{10}
Elia antwortete dem Hauptmann über fünfzig und sprach zu ihm: Bin ich
ein Mann Gottes, so falle Feuer vom Himmel und fresse dich und deine
fünfzig. Da fiel Feuer vom Himmel und fraß ihn und seine fünfzig.
\bibverse{11} Und er sandte wiederum einen anderen Hauptmann über
fünfzig zu ihm samt seinen fünfzigen. Der antwortete und sprach zu ihm:
Du Mann Gottes, so spricht der König: Komm eilends herab! \bibverse{12}
Elia antwortete und sprach: Bin ich ein Mann Gottes, so falle Feuer vom
Himmel und fresse dich und deine fünfzig. Da fiel das Feuer Gottes vom
Himmel und fraß ihn und seine fünfzig. \bibverse{13} Da sandte er
wiederum den dritten Hauptmann über fünfzig samt seinen fünfzigen. Da
der zu ihm hinaufkam, beugte er seine Knie gegen Elia und flehte ihn an
und sprach zu ihm: Du Mann Gottes, lass meine Seele und die Seele deiner
Knechte, dieser fünfzig, vor dir etwas gelten. \bibverse{14} Siehe, das
Feuer ist vom Himmel gefallen und hat die ersten zwei Hauptmänner über
fünfzig mit ihren fünfzigen gefressen; nun aber lass meine Seele etwas
gelten vor dir. \bibverse{15} Da sprach der Engel des HErrn zu Elia:
Gehe mit ihm hinab und fürchte dich nicht vor ihm! Und er machte sich
auf und ging mit ihm hinab zum König. \bibverse{16} Und er sprach zu
ihm: So spricht der HErr: Darum dass du hast Boten hingesandt und lassen
fragen Baal-Sebub, den Gott zu Ekron, als wäre kein Gott in Israel,
dessen Wort man fragen möchte, so sollst du von dem Bette nicht kommen,
darauf du dich gelegt hast, sondern sollst des Todes sterben.
\footnote{\textbf{1:16} 2Kö 1,3-4} \bibverse{17} Also starb er nach dem
Wort des HErrn, das Elia geredet hatte. Und Joram ward König an seiner
Statt im zweiten Jahr Jorams, des Sohnes Josaphats, des Königs Judas
denn er hatte keinen Sohn. \footnote{\textbf{1:17} 2Kö 3,1}
\bibverse{18} Was aber mehr von Ahasja zu sagen ist, das er getan hat,
siehe, das ist geschrieben in der Chronik der Könige Israels. \# 2
\bibverse{1} Da aber der HErr wollte Elia im Wetter gen Himmel holen,
gingen Elia und Elisa von Gilgal. \bibverse{2} Und Elia sprach zu Elisa:
Bleib doch hier; denn der HErr hat mich gen Beth-El gesandt. Elisa aber
sprach: So wahr der HErr lebt und deine Seele, ich verlasse dich nicht.
Und da sie hinab gen Beth-El kamen, \bibverse{3} gingen der Propheten
Kinder, die zu Beth-El waren, heraus zu Elisa und sprachen zu ihm: Weißt
du auch, dass der HErr wird deinen Herrn heute von deinen Häupten
nehmen? Er aber sprach: Ich weiß es auch wohl; schweigt nur still.
\bibverse{4} Und Elia sprach zu ihm: Elisa, bleib doch hier; denn der
HErr hat mich gen Jericho gesandt. Er aber sprach: So wahr der HErr lebt
und deine Seele, ich verlasse dich nicht. Und da sie gen Jericho kamen,
\bibverse{5} traten der Propheten Kinder, die zu Jericho waren, zu Elisa
und sprachen zu ihm: Weißt du auch, dass der HErr wird deinen Herrn
heute von deinen Häupten nehmen? Er aber sprach: Ich weiß es auch wohl;
schweigt nur still. \bibverse{6} Und Elia sprach zu ihm: Bleib doch
hier; denn der HErr hat mich gesandt an den Jordan. Er aber sprach: So
wahr der HErr lebt und deine Seele, ich verlasse dich nicht. Und gingen
die beiden miteinander. \bibverse{7} Aber 50 Männer unter der Propheten
Kindern gingen hin und traten gegenüber von ferne; aber die beiden
standen am Jordan. \bibverse{8} Da nahm Elia seinen Mantel und wickelte
ihn zusammen und schlug ins Wasser; das teilte sich auf beiden Seiten,
dass die beiden trocken hindurchgingen. \footnote{\textbf{2:8} 2Mo
  14,21-22; Jos 3,16} \bibverse{9} Und da sie hinüberkamen, sprach Elia
zu Elisa: Bitte, was ich dir tun soll, ehe ich von dir genommen werde.
Elisa sprach: dass mir werde ein zwiefältig Teil von deinem Geiste.
\footnote{\textbf{2:9} 5Mo 21,17} \bibverse{10} Er sprach: Du hast ein
Hartes gebeten. Doch, so du mich sehen wirst, wenn ich von dir genommen
werde, so wird's ja sein; wo nicht, so wird's nicht sein. \bibverse{11}
Und da sie miteinander gingen und redeten, siehe, da kam ein feuriger
Wagen mit feurigen Rossen, die schieden die beiden voneinander; und Elia
fuhr also im Wetter gen Himmel. \footnote{\textbf{2:11} 1Mo 5,24}
\bibverse{12} Elisa aber sah es und schrie: Mein Vater, mein Vater,
Wagen Israels und seine Reiter! und sah ihn nicht mehr. Und er fasste
sein Kleider und zerriss sie in zwei Stücke \footnote{\textbf{2:12} 2Kö
  13,14} \bibverse{13} und hob auf den Mantel Elias, der ihm entfallen
war, und kehrte um und trat an das Ufer des Jordans \footnote{\textbf{2:13}
  2Kö 2,8} \bibverse{14} und nahm den Mantel Elias, der ihm entfallen
war, und schlug ins Wasser und sprach: Wo ist nun der HErr, der Gott
Elias? und schlug ins Wasser; da teilte sich's auf beide Seiten, und
Elisa ging hindurch. \bibverse{15} Und da ihn sahen der Propheten
Kinder, die gegenüber zu Jericho waren, sprachen sie: Der Geist Elias
ruht auf Elisa; und gingen ihm entgegen und fielen vor ihm nieder zur
Erde \bibverse{16} und sprachen zu ihm: Siehe, es sind unter deinen
Knechten 50 Männer, starke Leute, die lass gehen und deinen Herrn
suchen; vielleicht hat ihn der Geist des HErrn genommen und irgend auf
einen Berg oder irgend in ein Tal geworfen. Er aber sprach: Lasst nicht
gehen! \bibverse{17} Aber sie nötigten ihn, bis dass er nachgab und
sprach: Lasst hingehen! Und sie sandten hin 50 Männer und suchten ihn
drei Tage; aber sie fanden ihn nicht. \bibverse{18} Und kamen wieder zu
ihm, da er noch zu Jericho war; und er sprach zu ihnen: Sagte ich euch
nicht, ihr solltet nicht hingehen? \bibverse{19} Und die Männer der
Stadt sprachen zu Elisa: Siehe, es ist gut wohnen in dieser Stadt, wie
mein Herr sieht; aber es ist böses Wasser und das Land unfruchtbar.
\bibverse{20} Er sprach: Bringet mir her eine neue Schale und tut Salz
darein! Und sie brachten's ihm. \bibverse{21} Da ging er hinaus zu der
Wasserquelle und warf das Salz hinein und sprach: So spricht der HErr:
Ich habe dies Wasser gesund gemacht; es soll hinfort kein Tod noch
Unfruchtbarkeit daher kommen. \bibverse{22} Also ward das Wasser gesund
bis auf diesen Tag nach dem Wort Elisas, das er redete. \bibverse{23}
Und er ging hinauf gen Beth-El. Und als er auf dem Wege hinanging, kamen
kleine Knaben zur Stadt heraus und spotteten sein und sprachen zu ihm:
Kahlkopf, komm herauf! Kahlkopf, komm herauf! \bibverse{24} Und er
wandte sich um; und da er sie sah, fluchte er ihnen im Namen des HErrn.
Da kamen zwei Bären aus dem Walde und zerrissen der Kinder 42.
\bibverse{25} Von da ging er auf den Berg Karmel und kehrte um von da
gen Samaria. \footnote{\textbf{2:25} 2Kö 4,25}

\hypertarget{section-1}{%
\section{3}\label{section-1}}

\bibverse{1} Joram, der Sohn Ahabs, ward König über Israel zu Samaria im
achtzehnten Jahr Josaphats, des Königs Judas, und regierte zwölf Jahre.
\footnote{\textbf{3:1} 2Kö 1,17} \bibverse{2} Und er tat, was dem HErrn
übel gefiel; doch nicht wie sein Vater und seine Mutter. Denn er tat weg
die Säule Baals, die sein Vater machen ließ. \footnote{\textbf{3:2} 1Kö
  16,32} \bibverse{3} Aber er blieb hangen an den Sünden Jerobeams, des
Sohnes Nebats, der Israel sündigen machte, und ließ nicht davon.
\footnote{\textbf{3:3} 1Kö 12,30} \bibverse{4} Mesa aber, der Moabiter
König, hatte viele Schafe und zinste dem König Israels Wolle von 100.000
Lämmern und von 100.000 Widdern. \bibverse{5} Da aber Ahab tot war, fiel
der Moabiter König ab vom König Israels. \bibverse{6} Da zog zur selben
Zeit aus der König Joram von Samaria und ordnete das ganze Israel
\bibverse{7} und sandte hin zu Josaphat, dem König Judas, und ließ ihm
sagen: Der Moabiter König ist von mir abgefallen; komm mit mir, zu
streiten wider die Moabiter! Er sprach: Ich will hinaufkommen; ich bin
wie du, und mein Volk wie dein Volk, und meine Rosse wie deine Rosse.
\footnote{\textbf{3:7} 1Kö 22,4} \bibverse{8} Und er sprach: Welchen Weg
wollen wir hinaufziehen? Er sprach: Den Weg durch die Wüste Edom.
\bibverse{9} Also zog hin der König Israels, der König Judas und der
König Edoms. Und da sie sieben Tagereisen zogen, hatte das Heer und das
Vieh, das unter ihnen war, kein Wasser. \bibverse{10} Da sprach der
König Israels: O wehe! der HErr hat diese drei Könige geladen, dass er
sie in der Moabiter Hände gebe. \bibverse{11} Josaphat aber sprach: Ist
kein Prophet des HErrn hier, dass wir den HErrn durch ihn ratfragen? Da
antwortete einer unter den Knechten des Königs Israels und sprach: Hier
ist Elisa, der Sohn Saphats, der Elia Wasser auf die Hände goss.
\bibverse{12} Josaphat sprach: Des HErrn Wort ist bei ihm. Also zogen zu
ihm hinab der König Israels und Josaphat und der König Edoms.
\bibverse{13} Elisa aber sprach zum König Israels: Was hast du mit mir
zu schaffen? Gehe hin zu den Propheten deines Vaters und zu den
Propheten deiner Mutter! Der König Israels sprach zu ihm: Nein! denn der
HErr hat diese drei Könige geladen, dass er sie in der Moabiter Hände
gebe. \bibverse{14} Elisa sprach: So wahr der HErr Zebaoth lebt, vor dem
ich stehe, wenn ich nicht Josaphat, den König Judas, ansähe, ich wollte
dich nicht ansehen noch achten. \footnote{\textbf{3:14} 1Kö 18,15; Ps
  15,4} \bibverse{15} So bringet mir nun einen Spielmann! Und da der
Spielmann auf den Saiten spielte, kam die Hand des HErrn auf ihn,
\bibverse{16} und er sprach: So spricht der HErr: Macht hier und da
Gräben an diesem Bach! \bibverse{17} Denn so spricht der HErr: Ihr
werdet keinen Wind noch Regen sehen; dennoch soll der Bach voll Wasser
werden, dass ihr und euer Gesinde und euer Vieh trinket. \bibverse{18}
Dazu ist das ein Geringes vor dem HErrn; er wird auch die Moabiter in
eure Hände geben, \bibverse{19} dass ihr schlagen werdet alle festen
Städte und alle auserwählten Städte und werdet fällen alle guten Bäume
und werdet verstopfen alle Wasserbrunnen und werdet allen guten Acker
mit Steinen verderben. \bibverse{20} Des Morgens aber, zur Zeit, da man
Speisopfer opfert, siehe, da kam ein Gewässer des Weges von Edom und
füllte das Land mit Wasser. \bibverse{21} Da aber alle Moabiter hörten,
dass die Könige heraufzogen, wider sie zu streiten, beriefen sie alle,
die zur Rüstung alt genug und darüber waren, und traten an die Grenze.
\bibverse{22} Und da sie sich des Morgens früh aufmachten und die Sonne
aufging über dem Gewässer, deuchte die Moabiter das Gewässer ihnen
gegenüber rot zu sein wie Blut; \bibverse{23} und sie sprachen: Es ist
Blut! Die Könige haben sich mit dem Schwert verderbt, und einer wird den
anderen geschlagen haben. Hui, Moab, mache dich nun auf zur Ausbeute!
\bibverse{24} Aber da sie zum Lager Israels kamen, machte sich Israel
auf und schlug die Moabiter; und sie flohen vor ihnen. Aber sie kamen
hinein und schlugen Moab. \bibverse{25} Die Städte zerbrachen sie, und
ein jeglicher warf seine Steine auf alle guten Äcker und machten sie
voll und verstopften alle Wasserbrunnen und fällten alle guten Bäume,
bis dass nur die Steine von Kir-Hareseth übrigblieben; und es umgaben
die Stadt die Schleuderer und warfen auf sie. \bibverse{26} Da aber der
Moabiter König sah, dass ihm der Streit zu stark war, nahm er 700 Mann
zu sich, die das Schwert auszogen, durchzubrechen wider den König Edoms;
aber sie konnten nicht. \bibverse{27} Da nahm er seinen ersten Sohn, der
an seiner Statt sollte König werden, und opferte ihn zum Brandopfer auf
der Mauer. Da kam ein großer Zorn über Israel, dass sie von ihm abzogen
und kehrten wieder in ihr Land. \# 4 \bibverse{1} Und es schrie ein Weib
unter den Weibern der Kinder der Propheten zu Elisa und sprach: Dein
Knecht, mein Mann, ist gestorben -- so weißt du, dass er, dein Knecht,
den HErrn fürchtete --; nun kommt der Schuldherr und will meine beiden
Kinder nehmen zu leibeigenen Knechten. \bibverse{2} Elisa sprach zu ihr:
Was soll ich dir tun? Sage mir, was hast du im Hause? Sie sprach: Deine
Magd hat nichts im Hause denn einen Ölkrug. \footnote{\textbf{4:2} 1Kö
  17,12} \bibverse{3} Er sprach: Gehe hin und bitte draußen von allen
deinen Nachbarinnen leere Gefäße, und derselben nicht wenig,
\bibverse{4} und gehe hinein und schließe die Tür zu hinter dir und
deinen Söhnen und gieße in alle Gefäße; und wenn du sie gefüllt hast, so
gib sie hin. \bibverse{5} Sie ging hin und schloss die Tür zu hinter
sich und ihren Söhnen; die brachten ihr die Gefäße zu, so goss sie ein.
\bibverse{6} Und da die Gefäße voll waren, sprach sie zu ihrem Sohn:
Lange mir noch ein Gefäß her! Er sprach zu ihr: Es ist kein Gefäß mehr
hier. Da stand das Öl. \bibverse{7} Und sie ging hin und sagte es dem
Mann Gottes an. Er sprach: Gehe hin, verkaufe das Öl und bezahle deinen
Schuldherrn; du aber und deine Söhne nähret euch von dem Übrigen.
\bibverse{8} Und es begab sich zu der Zeit, dass Elisa ging gen Sunem.
Daselbst war eine reiche Frau; die hielt ihn, dass er bei ihr aß. Und so
oft er daselbst durchzog, kehrte er zu ihr ein und aß bei ihr.
\bibverse{9} Und sie sprach zu ihrem Mann: Siehe, ich merke, dass dieser
Mann Gottes heilig ist, der immerdar hier durchgeht. \bibverse{10} Lass
uns ihm eine kleine bretterne Kammer oben machen und ein Bett, Tisch,
Stuhl und Leuchter hineinsetzen, auf dass er, wenn er zu uns kommt,
dahin sich tue. \bibverse{11} Und es begab sich zu der Zeit, dass er
hineinkam und legte sich oben in die Kammer und schlief darin
\bibverse{12} und sprach zu seinem Diener Gehasi: Rufe die Sunamitin!
Und da er sie rief, trat sie vor ihn. \bibverse{13} Er sprach zu ihm:
Sage ihr: Siehe, du hast uns allen diesen Dienst getan; was soll ich dir
tun? Hast du eine Sache an den König oder an den Feldhauptmann? Sie
sprach: Ich wohne unter meinem Volk. \bibverse{14} Er sprach: Was ist
ihr denn zu tun? Gehasi sprach: Ach, sie hat keinen Sohn, und ihr Mann
ist alt. \bibverse{15} Er sprach: Rufe sie! Und da er sie rief, trat sie
in die Tür. \bibverse{16} Und er sprach: Um diese Zeit über ein Jahr
sollst du einen Sohn herzen. Sie sprach: Ach nicht, mein Herr, du Mann
Gottes! lüge deiner Magd nicht! \footnote{\textbf{4:16} 1Mo 18,10; 1Mo
  18,14} \bibverse{17} Und die Frau ward schwanger und gebar einen Sohn
um dieselbe Zeit über ein Jahr, wie ihr Elisa geredet hatte.
\bibverse{18} Da aber das Kind groß ward, begab sich's, dass es hinaus
zu seinem Vater zu den Schnittern ging \bibverse{19} und sprach zu
seinem Vater: O mein Haupt, mein Haupt! Er sprach zu seinem Knecht:
Bringe ihn zu seiner Mutter! \bibverse{20} Und er nahm ihn und brachte
ihn hinein zu seiner Mutter, und sie setzte ihn auf ihren Schoß bis an
den Mittag; da starb er. \bibverse{21} Und sie ging hinauf und legte ihn
aufs Bett des Mannes Gottes, schloss zu und ging hinaus \bibverse{22}
und rief ihren Mann und sprach: Sende mir der Knechte einen und eine
Eselin; ich will zu dem Mann Gottes, und wiederkommen. \bibverse{23} Er
sprach: Warum willst du zu ihm? Ist doch heute nicht Neumond noch
Sabbat. Sie sprach: Es ist gut. \bibverse{24} Und sie sattelte die
Eselin und sprach zum Knecht: Treibe fort und säume mich nicht mit dem
Reiten, wie ich dir sage! \bibverse{25} Also zog sie hin und kam zu dem
Mann Gottes auf den Berg Karmel. Als aber der Mann Gottes sie kommen
sah, sprach er zu seinem Diener Gehasi: Siehe, die Sunamitin ist da!
\bibverse{26} So laufe ihr nun entgegen und frage sie, ob's ihr und
ihrem Mann und Sohn wohl gehe. Sie sprach: Wohl. \bibverse{27} Da sie
aber zu dem Mann Gottes auf den Berg kam, hielt sie ihn bei seinen
Füßen; Gehasi aber trat herzu, dass er sie abstieße. Aber der Mann
Gottes sprach: Lass sie! denn ihre Seele ist betrübt, und der HErr hat
mir's verborgen und nicht angezeigt. \bibverse{28} Sie sprach: Wann habe
ich einen Sohn gebeten von meinem Herrn? sagte ich nicht, du solltest
mich nicht täuschen? \footnote{\textbf{4:28} 2Kö 4,16} \bibverse{29} Er
sprach zu Gehasi: Gürte deine Lenden und nimm meinen Stab in deine Hand
und gehe hin (so dir jemand begegnet, so grüße ihn nicht, und grüßt dich
jemand, so danke ihm nicht), und lege meinen Stab auf des Knaben
Antlitz. \footnote{\textbf{4:29} Lk 10,4} \bibverse{30} Die Mutter aber
des Knaben sprach: So wahr der HErr lebt und deine Seele, ich lasse
nicht von dir! Da machte er sich auf und ging ihr nach. \bibverse{31}
Gehasi aber ging vor ihnen hin und legte den Stab dem Knaben aufs
Antlitz; da war aber keine Stimme noch Fühlen. Und er ging wiederum ihm
entgegen und zeigte ihm an und sprach: Der Knabe ist nicht aufgewacht.
\bibverse{32} Und da Elisa ins Haus kam, siehe, da lag der Knabe tot auf
seinem Bett. \bibverse{33} Und er ging hinein und schloss die Tür zu für
sie beide und betete zu dem HErrn \footnote{\textbf{4:33} Apg 9,40}
\bibverse{34} und stieg hinauf und legte sich auf das Kind und legte
seinen Mund auf des Kindes Mund und seine Augen auf seine Augen und
seine Hände auf seine Hände und breitete sich also über ihn, dass des
Kindes Leib warm ward. \footnote{\textbf{4:34} 1Kö 17,21} \bibverse{35}
Er aber stand wieder auf und ging im Haus einmal hierher und daher und
stieg hinauf und breitete sich über ihn. Da schnaubte der Knabe
siebenmal; darnach tat der Knabe seine Augen auf. \bibverse{36} Und er
rief Gehasi und sprach: Rufe die Sunamitin! Und da er sie rief, kam sie
hinein zu ihm. Er sprach: Da nimm hin deinen Sohn! \footnote{\textbf{4:36}
  Lk 7,15; Hebr 11,35} \bibverse{37} Da kam sie und fiel zu seinen Füßen
und beugte sich nieder zur Erde und nahm ihren Sohn und ging hinaus.
\bibverse{38} Da aber Elisa wieder gen Gilgal kam, ward Teuerung im
Lande, und die Kinder der Propheten wohnten vor ihm. Und er sprach zu
seinem Diener: Setze zu einen großen Topf und koche ein Gemüse für die
Kinder der Propheten! \bibverse{39} Da ging einer aufs Feld, dass er
Kraut läse, und fand wilde Ranken und las davon Koloquinten sein Kleid
voll; und da er kam, schnitt er's in den Topf zum Gemüse, denn sie
kannten's nicht. \bibverse{40} Und da sie es ausschütteten für die
Männer, zu essen, und sie von dem Gemüse aßen, schrien sie und sprachen:
O Mann Gottes, der Tod im Topf! denn sie konnten's nicht essen.
\bibverse{41} Er aber sprach: Bringet Mehl her! Und er tat's in den Topf
und sprach: Schütte es dem Volk vor, dass sie essen! Da war nichts Böses
in dem Topf. \bibverse{42} Es kam aber ein Mann von Baal-Salisa und
brachte dem Mann Gottes Erstlingsbrot, nämlich 20 Gerstenbrote, und
neues Getreide in seinem Kleid. Er aber sprach: Gib's dem Volk, dass sie
essen! \bibverse{43} Sein Diener sprach: Wie soll ich 100 Mann von dem
geben? Er sprach: Gib dem Volk, dass sie essen! Denn so spricht der
HErr: Man wird essen, und es wird übrigbleiben. \bibverse{44} Und er
legte es ihnen vor, dass sie aßen; und es blieb noch übrig nach dem Wort
des HErrn. \footnote{\textbf{4:44} Mt 16,9-10}

\hypertarget{section-2}{%
\section{5}\label{section-2}}

\bibverse{1} Naeman, der Feldhauptmann des Königs von Syrien, war ein
trefflicher Mann vor seinem Herrn und hoch gehalten; denn durch ihn gab
der HErr Heil in Syrien. Und er war ein gewaltiger Mann, und aussätzig.
\bibverse{2} Die Kriegsleute aber in Syrien waren herausgefallen und
hatten eine junge Dirne weggeführt aus dem Lande Israel; die war im
Dienst des Weibes Naemans. \bibverse{3} Die sprach zu ihrer Frau: Ach,
dass mein Herr wäre bei dem Propheten zu Samaria! der würde ihn von
seinem Aussatz losmachen. \bibverse{4} Da ging er hinein zu seinem Herrn
und sagte es ihm an und sprach: So und so hat die Dirne aus dem Lande
Israel geredet. \bibverse{5} Der König von Syrien sprach: So zieh hin,
ich will dem König Israels einen Brief schreiben. Und er zog hin und
nahm mit sich zehn Zentner Silber und 6000 Goldgulden und zehn
Feierkleider \bibverse{6} und brachte den Brief dem König Israels, der
lautete also: Wenn dieser Brief zu dir kommt, siehe, so wisse, ich habe
meinen Knecht Naeman zu dir gesandt, dass du ihn von seinem Aussatz
losmachest. \bibverse{7} Und da der König Israels den Brief las, zerriss
er seine Kleider und sprach: Bin ich denn Gott, dass ich töten und
lebendig machen könnte, dass er zu mir schickt, dass ich den Mann von
seinem Aussatz losmache? Merket und sehet, wie sucht er Ursache wider
mich! \bibverse{8} Da das Elisa, der Mann Gottes, hörte, dass der König
Israels seine Kleider zerrissen hatte, sandte er zu ihm und ließ ihm
sagen: Warum hast du deine Kleider zerrissen? Lass ihn zu mir kommen,
dass er innewerde, dass ein Prophet in Israel ist. \bibverse{9} Also kam
Naeman mit Rossen und Wagen und hielt vor der Tür am Hause Elisas.
\bibverse{10} Da sandte Elisa einen Boten zu ihm und ließ ihm sagen:
Gehe hin und wasche dich siebenmal im Jordan, so wird dir dein Fleisch
wieder erstattet und rein werden. \bibverse{11} Da erzürnte Naeman und
zog weg und sprach: Ich meinte, er sollte zu mir herauskommen und
hertreten und den Namen der HErrn, seines Gottes, anrufen und mit seiner
Hand über die Stätte fahren und den Aussatz also abtun. \bibverse{12}
Sind nicht die Wasser Amana und Pharphar zu Damaskus besser denn alle
Wasser in Israel, dass ich mich darin wüsche und rein würde? Und wandte
sich und zog weg mit Zorn. \bibverse{13} Da machten sich seine Knechte
zu ihm, redeten mit ihm und sprachen: Lieber Vater, wenn dich der
Prophet etwas Großes hätte geheißen, solltest du es nicht tun? Wie viel
mehr, wenn er zu dir sagt: Wasche dich, so wirst du rein! \bibverse{14}
Da stieg er ab und taufte sich im Jordan siebenmal, wie der Mann Gottes
geredet hatte; und sein Fleisch ward wieder erstattet wie das Fleisch
eines jungen Knaben, und er ward rein. \footnote{\textbf{5:14} Lk 4,27}
\bibverse{15} Und er kehrte wieder zu dem Mann Gottes samt seinem ganzen
Heer. Und da er hineinkam, trat er vor ihn und sprach: Siehe, ich weiß,
dass kein Gott ist in allen Landen, außer in Israel; so nimm nun den
Segen von deinem Knecht. \footnote{\textbf{5:15} 2Kö 5,5} \bibverse{16}
Er aber sprach: So wahr der HErr lebt, vor dem ich stehe, ich nehme es
nicht. Und er nötigte ihn, dass er's nähme; aber er wollte nicht.
\bibverse{17} Da sprach Naeman: Möchte denn deinem Knecht nicht gegeben
werden dieser Erde eine Last, soviel zwei Maultiere tragen? Denn dein
Knecht will nicht mehr anderen Göttern opfern und Brandopfer tun,
sondern dem HErrn. \bibverse{18} Nur darin wolle der HErr deinem Knecht
gnädig sein: wo ich anbete im Hause Rimmons, wenn mein Herr ins Haus
Rimmons geht, daselbst anzubeten, und er sich an meine Hand lehnt.
\footnote{\textbf{5:18} 2Kö 7,2} \bibverse{19} Er sprach zu ihm: Zieh
hin mit Frieden! Und als er von ihm weggezogen war ein Feld Wegs auf dem
Lande, \bibverse{20} gedachte Gehasi, der Diener Elisas, des Mannes
Gottes: Siehe, mein Herr hat diesen Syrer Naeman verschont, dass er
nichts von ihm hat genommen, das er gebracht hat. So wahr der HErr lebt,
ich will ihm nachlaufen und etwas von ihm nehmen. \bibverse{21} Also
jagte Gehasi dem Naeman nach. Und da Naeman sah, dass er ihm nachlief,
stieg er vom Wagen ihm entgegen und sprach: Steht es wohl? \bibverse{22}
Er sprach: Ja. Aber mein Herr hat mich gesandt und lässt dir sagen:
Siehe, jetzt sind zu mir gekommen vom Gebirge Ephraim zwei Jünglinge aus
der Propheten Kindern; gib ihnen einen Zentner Silber und zwei
Feierkleider! \bibverse{23} Naeman sprach: Nimm lieber zwei Zentner! Und
er nötigte ihn und band zwei Zentner Silber in zwei Beutel und zwei
Feierkleider und gab's zweien seiner Diener; die trugen's vor ihm her.
\bibverse{24} Und da er kam an den Hügel, nahm er's von ihren Händen und
legte es beiseit im Hause und ließ die Männer gehen. \bibverse{25} Und
da sie weg waren, trat er vor seinen Herrn. Und Elisa sprach zu ihm:
Woher, Gehasi? Er sprach: Dein Knecht ist weder hierher noch daher
gegangen. \bibverse{26} Er aber sprach zu ihm: Ist nicht mein Herz
mitgegangen, da der Mann umkehrte von seinem Wagen dir entgegen? War das
die Zeit, Silber und Kleider zu nehmen, Ölgärten, Weinberge, Schafe,
Rinder, Knechte und Mägde? \bibverse{27} Aber der Aussatz Naemans wird
dir anhangen und deinem Samen ewiglich. Da ging er von ihm hinaus
aussätzig wie Schnee. \# 6 \bibverse{1} Die Kinder der Propheten
sprachen zu Elisa: Siehe, der Raum, da wir vor dir wohnen, ist uns zu
enge. \bibverse{2} Lass uns an den Jordan gehen und einen jeglichen
daselbst Holz holen, dass wir uns daselbst eine Stätte bauen, da wir
wohnen. Er sprach: Gehet hin! \bibverse{3} Und einer sprach: Gehe lieber
mit deinen Knechten! Er sprach: Ich will mitgehen. \bibverse{4} Und er
ging mit ihnen. Und da sie an den Jordan kamen, hieben sie Holz ab.
\bibverse{5} Und da einer ein Holz fällte, fiel das Eisen ins Wasser.
Und er schrie und sprach: O weh, mein Herr! dazu ist's entlehnt.
\bibverse{6} Aber der Mann Gottes sprach: Wo ist's entfallen? Und da er
ihm den Ort zeigte, schnitt er ein Holz ab und stieß da hin. Da schwamm
das Eisen. \bibverse{7} Und er sprach: Heb's auf! Da reckte er seine
Hand aus und nahm's. \bibverse{8} Und der König von Syrien führte einen
Krieg wider Israel und beratschlagte sich mit seinen Knechten und
sprach: Wir wollen uns lagern da und da. \bibverse{9} Aber der Mann
Gottes sandte zum König Israels und ließ ihm sagen: Hüte dich, dass du
nicht an dem Ort vorüberziehst; denn die Syrer ruhen daselbst.
\bibverse{10} So sandte denn der König Israels hin an den Ort, den ihm
der Mann Gottes gesagt und vor dem er ihn gewarnt hatte, und war
daselbst auf der Hut; und tat das nicht einmal oder zweimal allein.
\bibverse{11} Da ward das Herz des Königs von Syrien voll Unmuts darüber
und er rief seine Knechte und sprach zu ihnen: Wollt ihr mir denn nicht
ansagen: Wer von den Unseren hält es mit dem König Israels?
\bibverse{12} Da sprach seiner Knechte einer: Nicht also, mein Herr
König; sondern Elisa, der Prophet in Israel, sagt alles dem König
Israels, was du in der Kammer redest, da dein Lager ist. \bibverse{13}
Er sprach: So gehet hin und sehet, wo er ist, dass ich hinsende und
lasse ihn holen. Und sie zeigten ihm an und sprachen: Siehe, er ist zu
Dothan. \bibverse{14} Da sandte er hin Rosse und Wagen und eine große
Macht. Und da sie bei der Nacht hinkamen, umgaben sie die Stadt.
\bibverse{15} Und der Diener des Mannes Gottes stand früh auf, dass er
sich aufmachte und auszöge; und siehe, da lag eine Macht um die Stadt
mit Rossen und Wagen. Da sprach sein Diener zu ihm: O weh, mein Herr!
wie wollen wir nun tun? \bibverse{16} Er sprach: Fürchte dich nicht!
denn derer ist mehr, die bei uns sind, als derer, die bei ihnen sind.
\bibverse{17} Und Elisa betete und sprach: HErr, öffne ihm die Augen,
dass er sehe! Da öffnete der HErr dem Diener die Augen, dass er sah; und
siehe, da war der Berg voll feuriger Rosse und Wagen um Elisa her.
\bibverse{18} Und da sie zu ihm hinabkamen, bat Elisa und sprach: HErr,
schlage dies Volk mit Blindheit! Und er schlug sie mit Blindheit nach
dem Wort Elisas. \footnote{\textbf{6:18} 1Mo 19,11} \bibverse{19} Und
Elisa sprach zu ihnen: Dies ist nicht der Weg noch die Stadt. Folget mir
nach! ich will euch führen zu dem Mann, den ihr sucht. Und führte sie
gen Samaria. \bibverse{20} Und da sie gen Samaria kamen, sprach Elisa:
HErr, öffne diesen die Augen, dass sie sehen! Und der HErr öffnete ihnen
die Augen, dass sie sahen; und siehe, da waren sie mitten in Samaria.
\bibverse{21} Und der König Israels, da er sie sah, sprach er zu Elisa:
Mein Vater, soll ich sie schlagen? \bibverse{22} Er sprach: Du sollst
sie nicht schlagen. Schlägst du denn die, welche du mit deinem Schwert
und Bogen gefangen hast? Setze ihnen Brot und Wasser vor, dass sie essen
und trinken, und lass sie zu ihrem Herrn ziehen! \bibverse{23} Da ward
ein großes Mahl zugerichtet. Und da sie gegessen und getrunken hatten,
ließ er sie gehen, dass sie zu ihrem Herrn zogen. Seit dem kamen
streifende Rotten der Syrer nicht mehr ins Land Israel. \bibverse{24}
Nach diesem begab sich's, dass Benhadad, der König von Syrien all sein
Heer versammelte und zog herauf und belagerte Samaria. \bibverse{25} Und
es war eine große Teuerung zu Samaria. Sie aber belagerten die Stadt,
bis dass ein Eselskopf 80 Silberlinge und ein viertel Kab Taubenmist
fünf Silberlinge galt. \bibverse{26} Und da der König Israels auf der
Mauer einherging, schrie ihn ein Weib an und sprach: Hilf mir, mein Herr
König! \bibverse{27} Er sprach: Hilft dir der HErr nicht, woher soll ich
dir helfen? von der Tenne oder von der Kelter? \bibverse{28} Und der
König sprach zu ihr: Was ist dir? Sie sprach: Dies Weib sprach zu mir:
Gib deinen Sohn her, dass wir heute essen; morgen wollen wir meinen Sohn
essen. \bibverse{29} So haben wir meinen Sohn gekocht und gegessen. Und
ich sprach zu ihr am anderen Tage: Gib deinen Sohn her und lass uns
essen! Aber sie hat ihren Sohn versteckt. \footnote{\textbf{6:29} 5Mo
  28,53} \bibverse{30} Da der König die Worte des Weibes hörte, zerriss
er seine Kleider, indem er auf der Mauer ging. Da sah alles Volk, dass
er darunter einen Sack am Leibe anhatte. \bibverse{31} Und er sprach:
Gott tue mir dies und das, wo das Haupt Elisas, des Sohnes Saphats,
heute auf ihm stehen wird! \bibverse{32} Elisa aber saß in seinem Hause,
und alle Ältesten saßen bei ihm. Und der König sandte einen Mann vor
sich her. Aber ehe der Bote zu ihm kam, sprach er zu den Ältesten: Habt
ihr gesehen, wie dies Mordkind hat hergesandt, dass er mein Haupt
abreiße? Sehet zu, wenn der Bote kommt, dass ihr die Tür zuschließt und
stoßet ihn mit der Tür weg! Siehe, das Rauschen der Füße seines Herrn
folgt ihm nach. \bibverse{33} Da er noch also mit ihnen redete, siehe,
da kam der Bote zu ihm hinab; und er sprach: Siehe, solches Übel kommt
von dem HErrn! Was soll ich mehr von dem HErrn erwarten? \# 7
\bibverse{1} Elisa aber sprach: Höret des HErrn Wort! So spricht der
HErr: Morgen um diese Zeit wird ein Scheffel Semmelmehl einen Silberling
gelten und zwei Scheffel Gerste einen Silberling unter dem Tor zu
Samaria. \footnote{\textbf{7:1} 2Kö 7,16} \bibverse{2} Da antwortete der
Ritter, auf dessen Hand sich der König lehnte, dem Mann Gottes und
sprach: Und wenn der HErr Fenster am Himmel machte, wie könnte solches
geschehen? Er sprach: Siehe da, mit deinen Augen wirst du es sehen, und
nicht davon essen! \footnote{\textbf{7:2} 2Kö 7,17; 2Kö 5,18}
\bibverse{3} Und es waren vier aussätzige Männer an der Tür vor dem Tor;
und einer sprach zum anderen: Was wollen wir hier bleiben, bis wir
sterben? \footnote{\textbf{7:3} 3Mo 13,46} \bibverse{4} Wenn wir gleich
gedächten, in die Stadt zu kommen, so ist Teuerung in der Stadt, und wir
müssten doch daselbst sterben; bleiben wir aber hier, so müssen wir auch
sterben. So lasst uns nun hingehen und zu dem Heer der Syrer fallen.
Lassen sie uns leben, so leben wir; töten sie uns, so sind wir tot.
\footnote{\textbf{7:4} Est 4,16} \bibverse{5} Und sie machten sich in
der Frühe auf, dass sie zum Heer der Syrer kämen. Und da sie vorn an den
Ort des Heeres kamen, siehe, da war niemand. \bibverse{6} Denn der HErr
hatte die Syrer lassen hören ein Geschrei von Rossen, Wagen und großer
Heereskraft, dass sie untereinander sprachen: Siehe, der König Israels
hat wider uns gedingt die Könige der Hethiter und die Könige der
Ägypter, dass sie über uns kommen sollen. \footnote{\textbf{7:6} 2Kö
  19,7} \bibverse{7} Und sie machten sich auf und flohen in der Frühe
und ließen ihre Hütten, Rosse und Esel im Lager, wie es stand, und
flohen mit ihrem Leben davon. \bibverse{8} Als nun die Aussätzigen an
den Ort des Lagers kamen, gingen sie in der Hütten eine, aßen und
tranken und nahmen Silber, Gold und Kleider und gingen hin und
verbargen's und kamen wieder und gingen in eine andere Hütte und nahmen
daraus und gingen hin und verbargen's. \bibverse{9} Aber einer sprach
zum anderen: Lasst uns nicht also tun; dieser Tag ist ein Tag guter
Botschaft. Wo wir das verschweigen und harren, bis dass es lichter
Morgen wird, wird unsere Missetat gefunden werden; so lasst uns nun
hingehen, dass wir kommen und es ansagen dem Hause des Königs.
\bibverse{10} Und da sie kamen, riefen sie am Tor der Stadt und sagten's
ihnen an und sprachen: Wir sind zum Lager der Syrer gekommen, und siehe,
es ist niemand da und keine Menschenstimme, sondern Rosse und Esel
angebunden und die Hütten, wie sie stehen. \bibverse{11} Da rief man den
Torhütern zu, dass sie es drinnen ansagten im Hause des Königs.
\bibverse{12} Und der König stand auf in der Nacht und sprach zu seinen
Knechten: Lasst euch sagen, wie die Syrer mit uns umgehen. Sie wissen,
dass wir Hunger leiden, und sind aus dem Lager gegangen, dass sie sich
im Felde verkröchen, und denken: Wenn sie aus der Stadt gehen, wollen
wir sie lebendig greifen und in die Stadt kommen. \bibverse{13} Da
antwortete seiner Knechte einer und sprach: Man nehme fünf Rosse von
denen, die noch drinnen sind übriggeblieben. Siehe, es wird ihnen gehen,
wie aller Menge Israels, so drinnen übriggeblieben oder schon dahin ist.
Die lasst uns senden und sehen. \bibverse{14} Da nahmen sie zwei Wagen
mit Rossen, und der König sandte sie dem Heere der Syrer nach und
sprach: Ziehet hin und sehet! \bibverse{15} Und da sie ihnen nachzogen
bis an den Jordan, siehe, da lag der Weg voll Kleider und Geräte, welche
die Syrer von sich geworfen hatten, da sie eilten. Und da die Boten
wiederkamen und sagten's dem König an, \bibverse{16} ging das Volk
hinaus und beraubte das Lager der Syrer. Und es galt ein Scheffel
Semmelmehl einen Silberling und zwei Scheffel Gerste auch einen
Silberling nach dem Wort des HErrn. \bibverse{17} Aber der König
bestellte den Ritter, auf dessen Hand er sich lehnte, unter das Tor. Und
das Volk zertrat ihn im Tor, dass er starb, wie der Mann Gottes geredet
hatte, da der König zu ihm hinabkam. \footnote{\textbf{7:17} 2Kö 7,2}
\bibverse{18} Und es geschah, wie der Mann Gottes dem König gesagt
hatte, da er sprach: Morgen um diese Zeit werden zwei Scheffel Gerste
einen Silberling gelten und ein Scheffel Semmelmehl einen Silberling
unter dem Tor zu Samaria, \bibverse{19} und der Ritter dem Mann Gottes
antwortete und sprach: Siehe, wenn der HErr Fenster am Himmel machte,
wie möchte solches geschehen? Er aber sprach: Siehe, mit deinen Augen
wirst du es sehen, und nicht davon essen! \bibverse{20} Und es ging ihm
eben also; denn das Volk zertrat ihn im Tor, dass er starb. \# 8
\bibverse{1} Elisa redete mit dem Weibe, dessen Sohn er hatte lebendig
gemacht, und sprach: Mache dich auf und gehe hin mit deinem Hause und
wohne in der Fremde, wo du kannst; denn der HErr wird eine Teuerung
rufen, die wird ins Land kommen sieben Jahre lang. \bibverse{2} Das Weib
machte sich auf und tat, wie der Mann Gottes sagte, und zog hin mit
ihrem Hause und wohnte in der Philister Land sieben Jahre. \bibverse{3}
Da aber die sieben Jahre um waren, kam das Weib wieder aus der Philister
Land; und sie ging aus, den König anzurufen um ihr Haus und ihren Acker.
\bibverse{4} Der König aber redete mit Gehasi, dem Diener des Mannes
Gottes, und sprach: Erzähle mir alle großen Taten, die Elisa getan hat!
\bibverse{5} Und indem er dem König erzählte, wie er hätte einen Toten
lebendig gemacht, siehe, da kam eben dazu das Weib, dessen Sohn er hatte
lebendig gemacht, und rief den König an um ihr Haus und ihren Acker. Da
sprach Gehasi: Mein Herr König, dies ist das Weib, und dies ist der
Sohn, den Elisa hat lebendig gemacht. \bibverse{6} Und der König fragte
das Weib; und sie erzählte es ihm. Da gab ihr der König einen Kämmerer
und sprach: Schaffe ihr wieder alles, was ihr gehört; dazu alles
Einkommen des Ackers, seit der Zeit, dass sie das Land verlassen hat,
bis hierher! \bibverse{7} Und Elisa kam gen Damaskus. Da lag Benhadad,
der König von Syrien, krank; und man sagte es ihm an und sprach: Der
Mann Gottes ist hergekommen. \bibverse{8} Da sprach der König zu Hasael:
Nimm Geschenke mir dir und gehe dem Mann Gottes entgegen und frage den
HErrn durch ihn und sprich, ob ich von dieser Krankheit möge genesen.
\bibverse{9} Hasael ging ihm entgegen und nahm Geschenke mit sich und
allerlei Güter zu Damaskus, eine Last für 40 Kamele. Und da er kam, trat
er vor ihn und sprach: Dein Sohn Benhadad, der König von Syrien, hat
mich zu dir gesandt und lässt dir sagen: Kann ich auch von dieser
Krankheit genesen? \bibverse{10} Elisa sprach zu ihm: Gehe hin und sage
ihm: Du wirst genesen! Aber der HErr hat mir gezeigt, dass er des Todes
sterben wird. \bibverse{11} Und der Mann Gottes schaute ihn starr und
lange an und weinte. \footnote{\textbf{8:11} Lk 19,41} \bibverse{12} Da
sprach Hasael: Warum weint mein Herr? Er sprach: Ich weiß, was für Übel
du den Kindern Israel tun wirst: du wirst ihre festen Städte mit Feuer
verbrennen und ihre junge Mannschaft mit dem Schwert erwürgen und ihre
jungen Kinder töten und ihre schwangeren Weiber zerhauen. \footnote{\textbf{8:12}
  2Kö 10,32} \bibverse{13} Hasael sprach: Was ist dein Knecht, der Hund,
dass er solch großes Ding tun sollte? Elisa sprach: Der HErr hat mir
gezeigt, dass du König von Syrien sein wirst. \footnote{\textbf{8:13}
  1Sam 24,15; 1Kö 19,15} \bibverse{14} Und er ging weg von Elisa und kam
zu seinem Herrn; der sprach zu ihm: Was sagte dir Elisa? Er sprach: Er
sagte mir: Du wirst genesen. \bibverse{15} Des anderen Tages aber nahm
er die Bettdecke und tauchte sie in Wasser und breitete sie über sein
Angesicht; da starb er. Und Hasael ward König an seiner Statt.
\bibverse{16} Im fünften Jahr Jorams, des Sohnes Ahabs, des Königs
Israels, ward Joram, der Sohn Josaphats, König in Juda. \bibverse{17}
Zweiunddreißig Jahre alt war er, da er König ward, und regierte acht
Jahre zu Jerusalem \bibverse{18} und wandelte auf dem Wege der Könige
Israels, wie das Haus Ahab tat; denn Ahabs Tochter war sein Weib. Und er
tat, was dem HErrn übel gefiel; \footnote{\textbf{8:18} 2Kö 8,26}
\bibverse{19} aber der HErr wollte Juda nicht verderben um seines
Knechtes David willen, wie er ihm verheißen hatte, ihm zu geben eine
Leuchte unter seinen Kindern immerdar. \footnote{\textbf{8:19} 2Sam
  7,11-16; 1Kö 11,36} \bibverse{20} Zu seiner Zeit fielen die Edomiter
ab von Juda und machten einen König über sich. \bibverse{21} Da zog
Joram gen Zair und alle Wagen mit ihm und machte sich des Nachts auf und
schlug die Edomiter, die um ihn her waren, dazu die Obersten über die
Wagen, dass das Volk floh in seine Hütten. \bibverse{22} Doch blieben
die Edomiter abtrünnig von Juda bis auf diesen Tag. Auch fiel zur selben
Zeit ab Libna. \bibverse{23} Was aber mehr von Joram zu sagen ist und
alles, was er getan hat, siehe, das ist geschrieben in der Chronik der
Könige Judas. \bibverse{24} Und Joram entschlief mit seinen Vätern und
ward begraben mit seinen Vätern in der Stadt Davids. Und Ahasja, sein
Sohn, ward König an seiner Statt. \bibverse{25} Im zwölften Jahr Jorams,
des Sohnes Ahabs, des Königs Israels, ward Ahasja, der Sohn Jorams,
König in Juda. \bibverse{26} Zweiundzwanzig Jahre alt war Ahasja, da er
König ward, und regierte ein Jahr zu Jerusalem. Seine Mutter hieß
Athalja, eine Tochter Omris, des Königs Israels. \footnote{\textbf{8:26}
  2Kö 8,18; 2Kö 11,1} \bibverse{27} Und er wandelte auf dem Wege des
Hauses Ahab und tat, was dem HErrn übel gefiel, wie das Haus Ahab; denn
er war Schwager im Hause Ahab. \bibverse{28} Und er zog mit Joram, dem
Sohn Ahabs, in den Streit wider Hasael, den König von Syrien, gen Ramoth
in Gilead; aber die Syrer schlugen Joram. \bibverse{29} Da kehrte Joram,
der König, um, dass er sich heilen ließe zu Jesreel von den Wunden, die
ihm die Syrer geschlagen hatten zu Rama, da er mit Hasael, dem König von
Syrien, stritt. Und Ahasja, der Sohn Jorams, der König Judas, kam hinab,
zu besuchen Joram, den Sohn Ahabs, zu Jesreel; denn er lag krank. \# 9
\bibverse{1} Elisa aber, der Prophet, rief der Propheten Kinder einen
und sprach zu ihm: Gürte deine Lenden und nimm diesen Ölkrug mit dir und
gehe hin gen Ramoth in Gilead. \bibverse{2} Und wenn du dahin kommst,
wirst du daselbst sehen Jehu, den Sohn Josaphats, des Sohnes Nimsis. Und
gehe hinein und heiße ihn aufstehen unter seinen Brüdern und führe ihn
in die innerste Kammer \bibverse{3} und nimm den Ölkrug und schütte es
auf sein Haupt und sprich: So sagt der HErr: Ich habe dich zum König
über Israel gesalbt. Und sollst die Tür auftun und fliehen und nicht
verziehen. \footnote{\textbf{9:3} 1Kö 19,16} \bibverse{4} Und der
Jüngling, der Diener des Propheten, ging hin gen Ramoth in Gilead.
\bibverse{5} Und da er hineinkam, siehe, da saßen die Hauptleute des
Heeres. Und er sprach: Ich habe dir, Hauptmann, was zu sagen. Jehu
sprach: Welchem unter uns allen? Er sprach: Dir, Hauptmann. \bibverse{6}
Da stand er auf und ging hinein. Er aber schüttete das Öl auf sein Haupt
und sprach zu ihm: So sagt der HErr, der Gott Israels: Ich habe dich zum
König gesalbt über des HErrn Volk Israel. \bibverse{7} Und du sollst das
Haus Ahabs, deines Herrn, schlagen, dass ich das Blut der Propheten,
meiner Knechte, und das Blut aller Knechte des HErrn räche, das die Hand
Isebels vergossen hat, \bibverse{8} dass das ganze Haus Ahab umkomme.
Und ich will von Ahab ausrotten, was männlich ist, den Verschlossenen
und Verlassenen in Israel, \footnote{\textbf{9:8} 1Kö 14,10}
\bibverse{9} und will das Haus Ahabs machen wie das Haus Jerobeams, des
Sohnes Nebats, und wie das Haus Baesas, des Sohnes Ahias. \footnote{\textbf{9:9}
  1Kö 15,29; 1Kö 16,3; 1Kö 16,11} \bibverse{10} Und die Hunde sollen
Isebel fressen auf dem Acker zu Jesreel, und soll sie niemand begraben.
Und er tat die Tür auf und floh. \bibverse{11} Und da Jehu herausging zu
den Knechten seines Herrn, sprach man zu ihm: Steht es wohl? Warum ist
dieser Rasende zu dir gekommen? Er sprach zu ihnen: Ihr kennt doch den
Mann wohl und was er sagt. \bibverse{12} Sie sprachen: Das ist nicht
wahr; sage es uns aber an! Er sprach: So und so hat er mit mir geredet
und gesagt: So spricht der HErr: Ich habe dich zum König über Israel
gesalbt. \bibverse{13} Da eilten sie und nahm ein jeglicher sein Kleid
und legte es unter ihn auf die hohen Stufen und bliesen mit der Posaune
und sprachen: Jehu ist König geworden! \footnote{\textbf{9:13} Mt 21,7}
\bibverse{14} Also machte Jehu, der Sohn Josaphats, des Sohnes Nimsis,
einen Bund wider Joram. Joram aber hatte mit ganz Israel vor Ramoth in
Gilead gelegen wider Hasael, den König von Syrien. \bibverse{15} Und
Joram, der König, war wiedergekommen, dass er sich heilen ließe zu
Jesreel von den Wunden, die ihm die Syrer geschlagen hatten, da er
stritt mit Hasael, dem König von Syrien. Und Jehu sprach: Ist's euer
Wille, so soll niemand entrinnen aus der Stadt, dass er hingehe und es
ansage zu Jesreel. \bibverse{16} Und er fuhr und zog gen Jesreel, denn
Joram lag daselbst; so war Ahasja, der König Judas, hinabgezogen, Joram
zu besuchen. \footnote{\textbf{9:16} 2Kö 8,29} \bibverse{17} Der Wächter
aber, der auf dem Turm zu Jesreel stand, sah den Haufen Jehus kommen und
sprach: Ich sehe einen Haufen. Da sprach Joram: Nimm einen Reiter und
sende ihnen entgegen und sprich: Ist's Friede? \bibverse{18} Und der
Reiter ritt hin ihm entgegen und sprach: So sagt der König: Ist's
Friede? Jehu sprach: Was geht dich der Friede an? Wende dich hinter
mich! Der Wächter verkündigte und sprach: Der Bote ist zu ihnen gekommen
und kommt nicht wieder. \bibverse{19} Da sandte er einen anderen Reiter.
Da der zu ihnen kam, sprach er: So spricht der König: Ist's Friede? Jehu
sprach: Was geht dich der Friede an? Wende dich hinter mich!
\bibverse{20} Das verkündigte der Wächter und sprach: Er ist zu ihnen
gekommen und kommt nicht wieder. Und es ist ein Treiben wie das Treiben
Jehus, des Sohnes Nimsis; denn er treibt, wie wenn er unsinnig wäre.
\bibverse{21} Da sprach Joram: Spannet an! Und man spannte seinen Wagen
an. Und sie zogen aus, Joram, der König Israels, und Ahasja, der König
Judas, ein jeglicher auf seinem Wagen, dass sie Jehu entgegenkämen; und
sie trafen ihn an auf dem Acker Naboths, des Jesreeliten. \bibverse{22}
Und da Joram Jehu sah, sprach er: Jehu, ist's Friede? Er aber sprach:
Was Friede? Deiner Mutter Isebel Abgötterei und Zauberei wird immer
größer. \bibverse{23} Da wandte Joram seine Hand und floh und sprach zu
Ahasja: Es ist Verräterei, Ahasja! \bibverse{24} Aber Jehu fasste den
Bogen und schoss Joram zwischen die Arme, dass der Pfeil durch sein Herz
ausfuhr, und er fiel in seinen Wagen. \bibverse{25} Und er sprach zu
seinem Ritter Bidekar: Nimm und wirf ihn auf den Acker Naboths, des
Jesreeliten! Denn ich gedenke, dass du mit mir auf einem Wagen seinem
Vater Ahab nachfuhrst, da der HErr solchen Spruch über ihn tat:
\footnote{\textbf{9:25} 1Kö 21,19} \bibverse{26} Was gilt's (sprach der
HErr), ich will dir das Blut Naboths und seiner Kinder, das ich gestern
sah, vergelten auf diesem Acker. So nimm nun und wirf ihn auf den Acker
nach dem Wort des HErrn. \bibverse{27} Da das Ahasja, der König Judas,
sah, floh er des Weges zum Hause des Gartens. Jehu aber jagte ihm nach
und hieß ihn auch schlagen in dem Wagen auf der Höhe Gur, die bei
Jibleam liegt. Und er floh gen Megiddo und starb daselbst. \bibverse{28}
Und seine Knechte ließen ihn führen gen Jerusalem und begruben ihn in
seinem Grabe mit seinen Vätern in der Stadt Davids. \footnote{\textbf{9:28}
  2Kö 14,2; 2Kö 23,30} \bibverse{29} Ahasja aber regierte über Juda im
elften Jahr Jorams, des Sohnes Ahabs. \bibverse{30} Und da Jehu gen
Jesreel kam und Isebel das erfuhr, schminkte sie ihr Angesicht und
schmückte ihr Haupt und guckte zum Fenster hinaus. \bibverse{31} Und da
Jehu unter das Tor kam, sprach sie: Ist's Simri wohl gegangen, der
seinen Herrn erwürgte? \bibverse{32} Und er hob sein Angesicht auf zum
Fenster und sprach: Wer hält's hier mit mir? Da sahen zwei oder drei
Kämmerer zu ihm heraus. \bibverse{33} Er sprach: Stürzet sie herab! Und
sie stürzten sie herab, dass die Wand und die Rosse mit ihrem Blut
besprengt wurden; und sie ward zertreten. \bibverse{34} Und da er
hineinkam und gegessen und getrunken hatte, sprach er: Sehet doch nach
der Verfluchten und begrabet sie; denn sie ist eines Königs Tochter!
\bibverse{35} Da sie aber hingingen, sie zu begraben, fanden sie nichts
von ihr denn den Schädel und die Füße und ihre flachen Hände.
\bibverse{36} Und sie kamen wieder und sagten's ihm an. Er aber sprach:
Es ist, was der HErr geredet hat durch seinen Knecht Elia, den
Thisbiter, und gesagt: Auf dem Acker Jesreel sollen die Hunde der Isebel
Fleisch fressen; \footnote{\textbf{9:36} 2Kö 9,10; 1Kö 21,23}
\bibverse{37} und das Aas Isebels soll wie Kot auf dem Felde sein im
Acker Jesreels, dass man nicht sagen könne: Das ist Isebel. \# 10
\bibverse{1} Ahab aber hatte 70 Söhne zu Samaria. Und Jehu schrieb
Briefe und sandte sie gen Samaria, zu den Obersten der Stadt Jesreel, zu
den Ältesten und Vormündern Ahabs, die lauteten also: \bibverse{2} Wenn
dieser Brief zu euch kommt, bei denen eures Herrn Söhne sind und Wagen,
Rosse, feste Städte und Rüstung, \bibverse{3} so sehet, welcher der
beste und geschickteste sei unter den Söhnen eures Herrn, und setzet ihn
auf seines Vaters Stuhl und streitet für eures Herrn Haus. \bibverse{4}
Sie aber fürchteten sich gar sehr und sprachen: Siehe, zwei Könige
konnten ihm nicht widerstehen; wie wollen wir denn stehen? \bibverse{5}
Und die über das Haus und über die Stadt waren und die Ältesten und
Vormünder sandten hin zu Jehu und ließen ihm sagen: Wir sind deine
Knechte: wir wollen alles tun, was du uns sagst; wir wollen niemand zum
König machen. Tue was dir gefällt. \bibverse{6} Da schrieb er den
anderen Brief an sie, der lautete also: So ihr mein seid und meiner
Stimme gehorcht, so nehmet die Häupter von den Männern, eures Herrn
Söhnen, und bringet sie zu mir morgen um diese Zeit gen Jesreel. (Der
Söhne aber des Königs waren siebzig Mann, und die Größten der Stadt
zogen sie auf.) \bibverse{7} Da nun der Brief zu ihnen kam, nahmen sie
des Königs Söhne und schlachteten die 70 Männer und legten ihre Häupter
in Körbe und schickten sie zu ihm gen Jesreel. \bibverse{8} Und da der
Bote kam und sagte es ihm an und sprach: Sie haben die Häupter der
Königskinder gebracht, sprach er: Legt sie auf zwei Haufen vor die Tür
am Tor bis morgen. \bibverse{9} Und des Morgens, da er ausging, trat er
dahin und sprach zu allem Volk: Ihr seid ja gerecht. Siehe, habe ich
wider meinen Herrn einen Bund gemacht und ihn erwürgt, wer hat denn
diese alle geschlagen? \bibverse{10} So erkennet ihr ja, dass kein Wort
des HErrn ist auf die Erde gefallen, das der HErr geredet hat wider das
Haus Ahab; und der HErr hat getan, wie er geredet hat durch seinen
Knecht Elia. \bibverse{11} Also schlug Jehu alle Übrigen vom Hause Ahab
zu Jesreel, alle seine Großen, seine Verwandten und seine Priester, bis
dass ihm nicht einer übrigblieb, \bibverse{12} und machte sich auf, zog
hin und kam gen Samaria. Unterwegs aber war ein Hirtenhaus.
\bibverse{13} Da traf Jehu an die Brüder Ahasjas, des Königs Judas, und
sprach: Wer seid ihr? Sie sprachen: Wir sind Brüder Ahasjas und ziehen
hinab, zu grüßen des Königs Kinder und der Königin Kinder. \footnote{\textbf{10:13}
  2Chr 22,8} \bibverse{14} Er aber sprach: Greifet sie lebendig! Und sie
griffen sie lebendig und schlachteten sie bei dem Brunnen am Hirtenhaus,
42 Mann, und ließen nicht einen von ihnen übrig. \bibverse{15} Und da er
von dannen zog, fand er Jonadab, den Sohn Rechabs, der ihm begegnete.
Und er grüßte ihn und sprach zu ihm: Ist dein Herz richtig wie mein Herz
mit deinem Herzen? Jonadab sprach: Ja. -- Ist's also, so gib mir deine
Hand! -- Und er gab ihm seine Hand! Und er ließ ihn zu sich auf den
Wagen sitzen \bibverse{16} und sprach: Komm mit mir und siehe meinen
Eifer um den HErrn! Und sie führten ihn mit ihm auf seinem Wagen.
\bibverse{17} Und da er gen Samaria kam, schlug er alles, was übrig war
von Ahab zu Samaria, bis dass er ihn vertilgte nach dem Wort des HErrn,
das er zu Elia geredet hatte. \footnote{\textbf{10:17} 1Kö 21,21-22}
\bibverse{18} Und Jehu versammelte alles Volk und ließ ihnen sagen: Ahab
hat Baal wenig gedient; Jehu will ihm besser dienen. \footnote{\textbf{10:18}
  1Kö 16,31-33} \bibverse{19} So lasst nun rufen alle Propheten Baals,
alle seine Knechte und alle seine Priester zu mir, dass man niemand
vermisse; denn ich habe ein großes Opfer dem Baal zu tun. Wen man
vermissen wird, der soll nicht leben. Aber Jehu tat solches mit
Hinterlist, dass er die Diener Baals umbrächte. \bibverse{20} Und Jehu
sprach: Heiliget dem Baal das Fest! Und sie ließen es ausrufen.
\bibverse{21} Auch sandte Jehu in ganz Israel und ließ alle Diener Baals
kommen, dass niemand übrig war, der nicht gekommen wäre. Und sie gingen
in das Haus Baals, dass das Haus Baals voll ward an allen Enden.
\bibverse{22} Da sprach er zu denen, die über das Kleiderhaus waren:
Bringet allen Dienern Baals Kleider heraus! Und sie brachten die Kleider
heraus. \bibverse{23} Und Jehu ging in das Haus Baal mit Jonadab, dem
Sohn Rechabs, und sprach zu den Dienern Baals: Forschet und sehet zu,
dass nicht hier unter euch sei jemand von des HErrn Dienern, sondern
Baals Diener allein! \footnote{\textbf{10:23} 2Kö 10,15} \bibverse{24}
Und da sie hineinkamen, Opfer und Brandopfer zu tun, bestellte sich Jehu
außen 80 Mann und sprach: Wenn der Männer jemand entrinnt, die ich unter
eure Hände gebe, so soll für seine Seele dessen Seele sein. \footnote{\textbf{10:24}
  1Kö 20,39} \bibverse{25} Da er nun die Brandopfer vollendet hatte,
sprach Jehu zu den Trabanten und Rittern: Gehet hinein und schlaget
jedermann; lasst niemand herausgehen! Und sie schlugen sie mit der
Schärfe des Schwerts. Und die Trabanten und Ritter warfen sie weg und
gingen zur Stadt des Hauses Baals \footnote{\textbf{10:25} 1Kö 18,40}
\bibverse{26} und brachten heraus die Säulen in dem Hause Baals und
verbrannten sie \footnote{\textbf{10:26} 2Kö 11,18} \bibverse{27} und
zerbrachen die Säule Baals samt dem Hause Baals und machten heimliche
Gemächer daraus bis auf diesen Tag. \footnote{\textbf{10:27} 2Kö 3,2}
\bibverse{28} Also vertilgte Jehu den Baal aus Israel; \bibverse{29}
aber von den Sünden Jerobeams, des Sohnes Nebats, der Israel sündigen
machte, ließ Jehu nicht, von den goldenen Kälbern zu Beth-El und zu Dan.
\bibverse{30} Und der HErr sprach zu Jehu: Darum, dass du willig gewesen
bist zu tun, was mir gefallen hat, und hast am Hause Ahab getan alles,
was in meinem Herzen war, sollen dir auf dem Stuhl Israels sitzen deine
Kinder ins vierte Glied. \footnote{\textbf{10:30} 2Kö 15,12}
\bibverse{31} Aber doch hielt Jehu nicht, dass er im Gesetz des HErrn,
des Gottes Israels, wandelte von ganzem Herzen; denn er ließ nicht von
den Sünden Jerobeams, der Israel hatte sündigen gemacht. \bibverse{32}
Zur selben Zeit fing der HErr an, Israel zu mindern; denn Hasael schlug
sie in allen Grenzen Israels, \bibverse{33} vom Jordan gegen der Sonne
Aufgang, das ganze Land Gilead der Gaditer, Rubeniter und Manassiter,
von Aroer an, das am Bach Arnon liegt, so Gilead wie Basan.
\bibverse{34} Was aber mehr von Jehu zu sagen ist und alles, was er
getan hat, und alle seine Macht, siehe, das ist geschrieben in der
Chronik der Könige Israels. \bibverse{35} Und Jehu entschlief mit seinen
Vätern, und sie begruben ihn zu Samaria. Und Joahas, sein Sohn, ward
König an seiner Statt. \footnote{\textbf{10:35} 2Kö 13,1} \bibverse{36}
Die Zeit aber, die Jehu über Israel regiert hat zu Samaria, sind 28
Jahre. \# 11 \bibverse{1} Athalja aber, Ahasjas Mutter, da sie sah, dass
ihr Sohn tot war, machte sie sich auf und brachte um alle aus dem
königlichen Geschlecht. \bibverse{2} Aber Joseba, die Tochter des Königs
Joram, Ahasjas Schwester, nahm Joas, den Sohn Ahasjas, und stahl ihn aus
des Königs Kindern, die getötet wurden, und tat ihn mit seiner Amme in
die Bettkammer; und sie verbargen ihn vor Athalja, dass er nicht getötet
ward. \bibverse{3} Und er war mit ihr versteckt im Hause des HErrn sechs
Jahre. Athalja aber war Königin im Lande. \bibverse{4} Im siebenten Jahr
aber sandte hin Jojada und nahm die Obersten über hundert von den
Leibwächtern und den Trabanten und ließ sie zu sich ins Haus des HErrn
kommen und machte einen Bund mit ihnen und nahm einen Eid von ihnen im
Hause des HErrn und zeigte ihnen des Königs Sohn \bibverse{5} und gebot
ihnen und sprach: Das ist's, was ihr tun sollt: Ein dritter Teil von
euch, die ihr des Sabbats antretet, sollen der Hut warten im Hause des
Königs, \bibverse{6} und ein dritter Teil soll sein am Tor Sur, und ein
dritter Teil am Tor, das hinter den Trabanten ist, und soll der Hut
warten am Hause Massah. \bibverse{7} Aber die zwei Teile euer aller, die
des Sabbats abtreten, sollen der Hut warten im Hause des HErrn um den
König, \bibverse{8} und sollt rings um den König euch machen, ein
jeglicher mit seiner Wehre in der Hand -- und wer herein zwischen die
Reihen kommt, der sterbe --, und sollt bei dem König sein, wenn er aus
und ein geht. \bibverse{9} Und die Obersten über hundert taten alles,
was ihnen Jojada, der Priester, geboten hatte, und nahmen zu sich ihre
Männer, die des Sabbats antraten, mit denen, die des Sabbats abtraten,
und kamen zu dem Priester Jojada. \bibverse{10} Und der Priester gab den
Hauptleuten Spieße und Schilde, die dem König David gehört hatten und in
dem Hause des HErrn waren. \footnote{\textbf{11:10} 2Sam 8,7}
\bibverse{11} Und die Trabanten standen um den König her, ein jeglicher
mit seiner Wehre in der Hand, von dem Winkel des Hauses zur Rechten bis
zum Winkel zur Linken, zum Altar zu und zum Hause. \bibverse{12} Und er
ließ des Königs Sohn hervorkommen und setzte ihm eine Krone auf und gab
ihm das Zeugnis, und sie machten ihn zum König und salbten ihn und
schlugen die Hände zusammen und sprachen: Glück zu dem König!
\bibverse{13} Und da Athalja hörte das Geschrei des Volkes, das zulief,
kam sie zum Volk in das Haus des HErrn \bibverse{14} und sah. Siehe, da
stand der König an der Säule, wie es Gewohnheit war, und die Obersten
und die Drommeter bei dem König; und alles Volk des Landes war fröhlich,
und man blies mit Drommeten. Athalja aber zerriss ihre Kleider und rief:
Aufruhr, Aufruhr! \bibverse{15} Aber der Priester Jojada gebot den
Obersten über hundert, die über das Heer gesetzt waren, und sprach zu
ihnen: Führet sie zwischen den Reihen hinaus; und wer ihr folgt, der
sterbe des Schwerts! Denn der Priester hatte gesagt, sie sollte nicht im
Hause des HErrn sterben. \bibverse{16} Und sie machten ihr Raum zu
beiden Seiten; und sie ging hinein des Weges, da die Rosse zum Hause des
Königs gehen, und ward daselbst getötet. \footnote{\textbf{11:16} Neh
  3,28} \bibverse{17} Da machte Jojada einen Bund zwischen dem HErrn und
dem König und dem Volk, dass sie des HErrn Volk sein sollten; also auch
zwischen dem König und dem Volk. \bibverse{18} Da ging alles Volk des
Landes in das Haus Baals und brachen seine Altäre ab und zerbrachen
seine Bildnisse gründlich, und Matthan, den Priester Baals, erwürgten
sie vor den Altären. Der Priester aber bestellte die Ämter im Hause des
HErrn \bibverse{19} und nahm die Obersten über hundert und die
Leibwächter und die Trabanten und alles Volk des Landes, und sie führten
den König hinab vom Hause des HErrn und kamen durchs Tor der Trabanten
zum Königshause; und er setzte sich auf der Könige Stuhl. \bibverse{20}
Und alles Volk im Lande war fröhlich, und die Stadt war still. Athalja
aber töteten sie mit dem Schwert in des Königs Hause. \# 12 \bibverse{1}
Und Joas war sieben Jahre alt, da er König ward. \bibverse{2} Im
siebenten Jahr Jehus ward Joas König, und regierte 40 Jahre zu
Jerusalem. Seine Mutter hieß Zibja von Beer-Seba. \bibverse{3} Und Joas
tat, was recht war und dem HErrn wohl gefiel, solange ihn der Priester
Jojada lehrte, \bibverse{4} nur, dass sie die Höhen nicht abtaten; denn
das Volk opferte und räucherte noch auf den Höhen. \footnote{\textbf{12:4}
  2Kö 14,4; 1Kö 22,44} \bibverse{5} Und Joas sprach zu den Priestern:
Alles Geld, das geheiligt wird, dass es in das Haus des HErrn gebracht
werde, das gang und gäbe ist, das Geld, das jedermann gibt in der
Schätzung seiner Seele, und alles Geld, das jedermann von freiem Herzen
opfert, dass es in des HErrn Haus gebracht werde, \bibverse{6} das lasst
die Priester zu sich nehmen, einen jeglichen von seinen Bekannten. Davon
sollen sie bessern, was baufällig ist am Hause, wo sie finden, dass es
baufällig ist. \bibverse{7} Da aber die Priester bis ins
dreiundzwanzigste Jahr des Königs Joas nicht besserten, was baufällig
war am Hause, \bibverse{8} rief der König Joas den Priester Jojada samt
den Priestern und sprach zu ihnen: Warum bessert ihr nicht, was
baufällig ist am Hause? So sollt ihr nun nicht zu euch nehmen das Geld,
ein jeglicher von seinen Bekannten, sondern sollt's geben zu dem, das
baufällig ist am Hause. \bibverse{9} Und die Priester willigten darein,
dass sie nicht vom Volk Geld nähmen und das Baufällige am Hause
besserten. \bibverse{10} Da nahm der Priester Jojada eine Lade und
bohrte oben ein Loch darein und setzte sie zur rechten Hand neben den
Altar, da man in das Haus des HErrn geht. Und die Priester, die an der
Schwelle hüteten, taten darein alles Geld, das zu des HErrn Haus
gebracht ward. \bibverse{11} Wenn sie dann sahen, dass viel Geld in der
Lade war, so kam des Königs Schreiber herauf mit dem Hohenpriester, und
banden das Geld zusammen und zählten es, was für des HErrn Haus gefunden
ward. \bibverse{12} Und man übergab das Geld bar den Werkmeistern, die
da bestellt waren zu dem Hause des HErrn; und sie gaben's heraus den
Zimmerleuten und Bauleuten, die da arbeiteten am Hause des HErrn,
\bibverse{13} nämlich den Maurern und Steinmetzen und denen, die da Holz
und gehauene Steine kaufen sollten, dass das Baufällige am Hause des
HErrn gebessert würde, und für alles, was not war, um am Hause zu
bessern. \bibverse{14} Doch ließ man nicht machen silberne Schalen,
Messer, Becken, Drommeten noch irgendein goldenes oder silbernes Gerät
im Hause des HErrn von solchem Geld, das zu des HErrn Hause gebracht
ward; \bibverse{15} sondern man gab's den Arbeitern, dass sie damit das
Baufällige am Hause des HErrn besserten. \bibverse{16} Auch brauchten
die Männer nicht Rechnung zu tun, denen man das Geld übergab, dass sie
es den Arbeitern gäben; sondern sie handelten auf Glauben. \bibverse{17}
Aber das Geld von Schuldopfern und Sündopfern ward nicht zum Hause des
HErrn gebracht; denn es gehörte den Priestern. \bibverse{18} Zu der Zeit
zog Hasael, der König von Syrien, herauf und stritt wider Gath und
gewann es. Und da Hasael sein Angesicht stellte, nach Jerusalem
hinaufzuziehen, \footnote{\textbf{12:18} 2Kö 10,32} \bibverse{19} nahm
Joas, der König Judas, all das Geheiligte, das seine Väter Josaphat,
Joram und Ahasja, die Könige Judas, geheiligt hatten und was er
geheiligt hatte, dazu alles Gold, das man fand im Schatz in des HErrn
Hause und in des Königs Hause, und schickte es Hasael, dem König von
Syrien. Da zog er ab von Jerusalem. \footnote{\textbf{12:19} 1Kö 15,18}
\bibverse{20} Was aber mehr von Joas zu sagen ist und alles, was er
getan hat, das ist geschrieben in der Chronik der Könige Judas.
\bibverse{21} Und seine Knechte empörten sich und machten einen Bund und
schlugen ihn im Haus Millo, da man hinabgeht zu Silla. \footnote{\textbf{12:21}
  2Kö 14,5} \bibverse{22} Denn Josachar, der Sohn Simeaths, und Josabad,
der Sohn Somers, seine Knechte, schlugen ihn tot. Und man begrub ihn mit
seinen Vätern in der Stadt Davids. Und Amazja, sein Sohn, ward König an
seiner Statt. \footnote{\textbf{12:22} 2Kö 14,1}

\hypertarget{section-3}{%
\section{13}\label{section-3}}

\bibverse{1} Im dreiundzwanzigsten Jahr des Joas, des Sohnes Ahasjas,
des Königs Judas, ward Joahas, der Sohn Jehus, König über Israel zu
Samaria 17 Jahre; \footnote{\textbf{13:1} 2Kö 10,35} \bibverse{2} und er
tat, was dem HErrn übel gefiel, und wandelte nach den Sünden Jerobeams,
des Sohnes Nebats, der Israel sündigen machte, und ließ nicht davon.
\footnote{\textbf{13:2} 1Kö 12,26-33} \bibverse{3} Und des HErrn Zorn
ergrimmte über Israel, und er gab sie unter die Hand Hasaels, des Königs
von Syrien, und Benhadads, des Sohnes Hasaels, die ganze Zeit.
\footnote{\textbf{13:3} 2Kö 10,32} \bibverse{4} Aber Joahas bat des
HErrn Angesicht. Und der HErr erhörte ihn; denn er sah den Jammer
Israels an, wie sie der König von Syrien drängte. \bibverse{5} Und der
HErr gab Israel einen Heiland, der sie aus der Gewalt der Syrer führte,
dass die Kinder Israel in ihren Hütten wohnten wie zuvor. \bibverse{6}
Doch sie ließen nicht von der Sünde des Hauses Jerobeams, der Israel
sündigen machte, sondern wandelten darin. Auch blieb stehen das
Ascherabild zu Samaria. \footnote{\textbf{13:6} 1Kö 16,33} \bibverse{7}
Denn es war des Volks des Joahas nicht mehr übriggeblieben als 50
Reiter, zehn Wagen und 10.000 Mann Fußvolk. Denn der König von Syrien
hatte sie umgebracht und hatte sie gemacht wie Staub beim Dreschen.
\bibverse{8} Was aber mehr von Joahas zu sagen ist und alles, was er
getan hat, und seine Macht, siehe, das ist geschrieben in der Chronik
der Könige Israels. \bibverse{9} Und Joahas entschlief mit seinen
Vätern, und man begrub ihn zu Samaria. Und sein Sohn Joas ward König an
seiner Statt. \bibverse{10} Im siebenunddreißigsten Jahr des Joas, des
Königs in Juda, ward Joas, der Sohn Joahas, König über Israel zu Samaria
16 Jahre; \bibverse{11} und er tat, was dem HErrn übel gefiel, und ließ
nicht von allen Sünden Jerobeams, des Sohnes Nebats, der Israel sündigen
machte, sondern wandelte darin. \bibverse{12} Was aber mehr von Joas zu
sagen ist und was er getan hat und seine Macht, wie er mit Amazja, dem
König Judas, gestritten hat, siehe, das ist geschrieben in der Chronik
der Könige Israels. \footnote{\textbf{13:12} 2Kö 14,8-16} \bibverse{13}
Und Joas entschlief mit seinen Vätern, und Jerobeam saß auf seinem
Stuhl. Joas aber ward begraben zu Samaria bei den Königen Israels.
\footnote{\textbf{13:13} 2Kö 14,23} \bibverse{14} Elisa aber war krank,
daran er auch starb. Und Joas, der König Israels, kam zu ihm hinab und
weinte vor ihm und sprach: Mein Vater, mein Vater! Wagen Israels und
seine Reiter! \footnote{\textbf{13:14} 2Kö 2,12} \bibverse{15} Elisa
aber sprach zu ihm: Nimm den Bogen und Pfeile! Und da er den Bogen und
die Pfeile nahm, \bibverse{16} sprach er zum König Israels: Spanne mit
deiner Hand den Bogen! Und er spannte mit seiner Hand. Und Elisa legte
seine Hand auf des Königs Hand \bibverse{17} und sprach: Tu das Fenster
auf gegen Morgen! Und er tat's auf. Und Elisa sprach: Schieß! Und er
schoss. Er aber sprach: Ein Pfeil des Heils vom HErrn, ein Pfeil des
Heils wider die Syrer; und du wirst die Syrer schlagen zu Aphek, bis sie
aufgerieben sind. \bibverse{18} Und er sprach: Nimm die Pfeile! Und da
er sie nahm, sprach er zum König Israels: Schlage die Erde! Und er
schlug dreimal und stand still. \bibverse{19} Da ward der Mann Gottes
zornig auf ihn und sprach: Hättest du fünf- oder sechsmal geschlagen, so
würdest du die Syrer geschlagen haben, bis sie aufgerieben wären; nun
aber wirst du sie dreimal schlagen. \bibverse{20} Da aber Elisa
gestorben war und man ihn begraben hatte, fielen die Kriegsleute der
Moabiter ins Land desselben Jahres. \bibverse{21} Und es begab sich,
dass man einen Mann begrub; da sie aber die Kriegsleute sahen, warfen
sie den Mann in Elisas Grab. Und da er hinabkam und die Gebeine Elisas
berührte, ward er lebendig und trat auf seine Füße. \bibverse{22} Also
zwang nun Hasael, der König von Syrien, Israel, solange Joahas lebte.
\bibverse{23} Aber der HErr tat ihnen Gnade und erbarmte sich ihrer und
wandte sich zu ihnen um seines Bundes willen mit Abraham, Isaak und
Jakob und wollte sie nicht verderben, verwarf sie auch nicht von seinem
Angesicht bis auf diese Stunde. \bibverse{24} Und Hasael, der König von
Syrien, starb, und sein Sohn Benhadad ward König an seiner Statt.
\bibverse{25} Joas aber nahm wieder die Städte aus der Hand Benhadads,
des Sohnes Hasaels, die er aus der Hand seines Vaters Joahas genommen
hatte mit Streit. Dreimal schlug ihn Joas und brachte die Städte Israels
wieder. \footnote{\textbf{13:25} 2Kö 13,19}

\hypertarget{section-4}{%
\section{14}\label{section-4}}

\bibverse{1} Im zweiten Jahr des Joas, des Sohnes des Joahas, des Königs
über Israel, ward Amazja König, der Sohn des Joas, des Königs in Juda.
\footnote{\textbf{14:1} 2Kö 12,22} \bibverse{2} Fünfundzwanzig Jahre alt
war er, da er König ward, und regierte 29 Jahre zu Jerusalem. Seine
Mutter hieß Joaddan von Jerusalem. \bibverse{3} Und er tat, was dem
HErrn wohl gefiel, doch nicht wie sein Vater David; sondern wie sein
Vater Joas tat er auch. \footnote{\textbf{14:3} 2Kö 12,3-4} \bibverse{4}
Denn die Höhen wurden nicht abgetan; sondern das Volk opferte und
räucherte noch auf den Höhen. \footnote{\textbf{14:4} 2Kö 15,4}
\bibverse{5} Da er nun des Königreichs mächtig ward, schlug er seine
Knechte, die seinen Vater, den König, geschlagen hatten. \footnote{\textbf{14:5}
  2Kö 12,21-22} \bibverse{6} Aber die Kinder der Totschläger tötete er
nicht, wie es denn geschrieben steht im Gesetzbuch Moses, da der HErr
geboten hat und gesagt: Die Väter sollen nicht um der Kinder willen
sterben, und die Kinder sollen nicht um der Väter willen sterben;
sondern ein jeglicher soll um seiner Sünde willen sterben. \footnote{\textbf{14:6}
  5Mo 24,16} \bibverse{7} Er schlug auch der Edomiter im Salztal 10.000
und gewann die Stadt Sela mit Streit und hieß sie Joktheel bis auf
diesen Tag. \bibverse{8} Da sandte Amazja Boten zu Joas, dem Sohn des
Joahas, des Sohnes Jehus, dem König über Israel, und ließ ihm sagen:
Komm her, wir wollen uns miteinander messen! \bibverse{9} Aber Joas, der
König Israels, sandte zu Amazja, dem Könige Judas, und ließ ihm sagen:
Der Dornstrauch, der im Libanon ist, sandte zur Zeder im Libanon und
ließ ihr sagen: Gib deine Tochter meinem Sohn zum Weib! Aber das Wild
auf dem Felde im Libanon lief über den Dornstrauch und zertrat ihn.
\footnote{\textbf{14:9} Ri 9,14} \bibverse{10} Du hast die Edomiter
geschlagen; des überhebt sich dein Herz. Habe den Ruhm und bleibe
daheim! Warum ringst du nach Unglück, dass du fällst und Juda mit dir?
\bibverse{11} Aber Amazja gehorchte nicht. Da zog Joas, der König
Israels, herauf; und sie maßen sich miteinander, er und Amazja, der
König Judas, zu Beth-Semes, das in Juda liegt. \bibverse{12} Aber Juda
ward geschlagen vor Israel, dass ein jeglicher floh in seine Hütte.
\bibverse{13} Und Joas, der König Israels, griff Amazja, den König in
Juda, den Sohn des Joas, des Sohnes des Ahasja, zu Beth-Semes und kam
gen Jerusalem und riss ein die Mauer Jerusalems von dem Tor Ephraim an
bis an das Ecktor, 400 Ellen lang, \bibverse{14} und nahm alles Gold und
Silber und Gerät, das gefunden ward im Hause des HErrn und im Schatz des
Königshauses, dazu die Geiseln, und zog wieder gen Samaria.
\bibverse{15} Was aber mehr von Joas zu sagen ist, was er getan hat, und
seine Macht, und wie er mit Amazja, dem König Judas, gestritten hat,
siehe, das ist geschrieben in der Chronik der Könige Israels.
\bibverse{16} Und Joas entschlief mit seinen Vätern und ward begraben zu
Samaria unter den Königen Israels. Und sein Sohn Jerobeam ward König an
seiner Statt. \bibverse{17} Amazja aber, der Sohn des Joas, des Königs
in Juda, lebte nach dem Tod des Joas, des Sohnes des Joahas, des Königs
über Israel, 15 Jahre. \bibverse{18} Was aber mehr von Amazja zu sagen
ist, das ist geschrieben in der Chronik der Könige Judas. \bibverse{19}
Und sie machten einen Bund wider ihn zu Jerusalem; er aber floh gen
Lachis. Und sie sandten hin, ihm nach, gen Lachis und töteten ihn
daselbst. \footnote{\textbf{14:19} 2Kö 12,21} \bibverse{20} Und sie
brachten ihn auf Rossen, und er ward begraben zu Jerusalem bei seinen
Vätern in der Stadt Davids. \footnote{\textbf{14:20} 2Kö 9,28}
\bibverse{21} Und das ganze Volk Judas nahm Asarja in seinem sechzehnten
Jahr und machten ihn zum König anstatt seines Vaters Amazja. \footnote{\textbf{14:21}
  2Kö 15,1-2} \bibverse{22} Er baute Elath und brachte es wieder zu
Juda, nachdem der König mit seinen Vätern entschlafen war. \footnote{\textbf{14:22}
  2Kö 16,6} \bibverse{23} Im fünfzehnten Jahr Amazjas, des Sohnes Joas,
des Königs in Juda, ward Jerobeam, der Sohn des Joas, König über Israel
zu Samaria 41 Jahre; \footnote{\textbf{14:23} 2Kö 14,16; Hos 1,1; Am 1,1}
\bibverse{24} Und er tat, was dem HErrn übel gefiel, und ließ nicht ab
von allen Sünden Jerobeams, des Sohnes Nebats, der Israel sündigen
machte. \footnote{\textbf{14:24} 1Kö 12,26-33} \bibverse{25} Er aber
brachte wieder herzu das Gebiet Israels von Hamath an bis ans Meer, das
im blachen Felde liegt, nach dem Wort des HErrn, des Gottes Israels, das
er geredet hatte durch seinen Knecht Jona, den Sohn Amitthais, den
Propheten, der von Gath-Hepher war. \footnote{\textbf{14:25} Jon 1,1}
\bibverse{26} Denn der HErr sah an den elenden Jammer Israels, dass auch
die Verschlossenen und Verlassenen dahin waren und kein Helfer war in
Israel. \footnote{\textbf{14:26} 5Mo 32,36} \bibverse{27} Und der HErr
hatte nicht geredet, dass er wollte den Namen Israels austilgen unter
dem Himmel, und half ihnen durch Jerobeam, den Sohn des Joas.
\footnote{\textbf{14:27} 2Kö 13,5} \bibverse{28} Was aber mehr von
Jerobeam zu sagen ist und alles, was er getan hat, und seine Macht, wie
er gestritten hat und wie er Damaskus und Hamath wiedergebracht an Juda
in Israel, siehe, das ist geschrieben in der Chronik der Könige Israels.
\bibverse{29} Und Jerobeam entschlief mit seinen Vätern, mit den Königen
Israels. Und sein Sohn Sacharja ward König an seiner Statt. \# 15
\bibverse{1} Im siebenundzwanzigsten Jahr Jerobeams, des Königs Israels,
ward König Asarja, der Sohn Amazjas, des Königs Judas \footnote{\textbf{15:1}
  2Kö 14,21} \bibverse{2} und er war sechzehn Jahre alt, da er König
ward, und regierte 52 Jahre zu Jerusalem. Seine Mutter hieß Jecholja von
Jerusalem. \bibverse{3} Und er tat, was dem HErrn wohl gefiel, ganz wie
sein Vater Amazja, \bibverse{4} nur, dass sie die Höhen nicht abtaten;
denn das Volk opferte und räucherte noch auf den Höhen. \bibverse{5} Der
HErr plagte aber den König, dass er aussätzig war bis an seinen Tod, und
wohnte in einem besonderen Hause. Jotham aber, des Königs Sohn, regierte
das Haus und richtete das Volk im Lande. \footnote{\textbf{15:5} 3Mo
  13,46} \bibverse{6} Was aber mehr von Asarja zu sagen ist und alles,
was er getan hat, siehe, das ist geschrieben in der Chronik der Könige
Judas. \bibverse{7} Und Asarja entschlief mit seinen Vätern, und man
begrub ihn bei seinen Vätern in der Stadt Davids. Und sein Sohn Jotham
ward König an seiner Statt. \bibverse{8} Im achtunddreißigsten Jahr
Asarjas, des Königs Judas, ward König Sacharja, der Sohn Jerobeams, über
Israel zu Samaria sechs Monate; \footnote{\textbf{15:8} 2Kö 14,29}
\bibverse{9} und er tat, was dem HErrn übel gefiel, wie seine Väter
getan hatten. Er ließ nicht ab von den Sünden Jerobeams, des Sohnes
Nebats, der Israel sündigen machte. \footnote{\textbf{15:9} 1Kö 12,26-33}
\bibverse{10} Und Sallum, der Sohn des Jabes, machte einen Bund wider
ihn und schlug ihn vor dem Volk und tötete ihn und ward König an seiner
Statt. \footnote{\textbf{15:10} 2Kö 15,14; Am 7,9} \bibverse{11} Was
aber mehr von Sacharja zu sagen ist, siehe, das ist geschrieben in der
Chronik der Könige Israels. \bibverse{12} Und das ist's, was der HErr zu
Jehu geredet hatte: Dir sollen Kinder ins vierte Glied sitzen auf dem
Stuhl Israels. Und ist also geschehen. \bibverse{13} Sallum aber, der
Sohn des Jabes, ward König im neununddreißigsten Jahr Usias, des Königs
in Juda, und regierte einen Monat zu Samaria. \bibverse{14} Denn
Menahem, der Sohn Gadis, zog herauf von Thirza und kam gen Samaria und
schlug Sallum, den Sohn des Jabes, zu Samaria und tötete ihn und ward
König an seiner Statt. \footnote{\textbf{15:14} 1Kö 16,17} \bibverse{15}
Was aber mehr von Sallum zu sagen ist und seinen Bund, den er
anrichtete, siehe, das ist geschrieben in der Chronik der Könige
Israels. \bibverse{16} Dazumal schlug Menahem Tiphsah und alle, die
darin waren, und ihr Gebiet von Thirza aus, darum dass sie ihn nicht
wollten einlassen, und schlug alle ihre Schwangeren und zerriss sie.
\bibverse{17} Im neununddreißigsten Jahr Asarjas, des Königs Judas, ward
König Menahem, der Sohn Gadis, über Israel zehn Jahre zu Samaria;
\bibverse{18} und er tat, was dem HErrn übel gefiel. Er ließ sein Leben
lang nicht von den Sünden Jerobeams, des Sohnes Nebats, der Israel
sündigen machte. \bibverse{19} Und es kam Phul, der König von Assyrien,
ins Land. Und Menahem gab dem Phul 1000 Zentner Silber, dass er's mit
ihm hielte und befestigte ihm das Königreich. \bibverse{20} Und Menahem
setzte ein Geld in Israel auf die Reichsten, 50 Silberlinge auf einen
jeglichen Mann, dass er's dem König von Assyrien gäbe. Also zog der
König von Assyrien wieder heim und blieb nicht im Lande. \footnote{\textbf{15:20}
  2Kö 23,35} \bibverse{21} Was aber mehr von Menahem zu sagen ist und
alles, was er getan hat, siehe, das ist geschrieben in der Chronik der
Könige Israels. \bibverse{22} Und Menahem entschlief mit seinen Vätern,
und Pekahja, sein Sohn, ward König an seiner Statt. \bibverse{23} Im
fünfzigsten Jahr Asarjas, des Königs in Juda, ward König Pekahja, der
Sohn Menahems, über Israel zu Samaria zwei Jahre; \bibverse{24} und er
tat, was dem HErrn übel gefiel; denn er ließ nicht von der Sünde
Jerobeams, des Sohnes Nebats, der Israel sündigen machte. \bibverse{25}
Und es machte Pekah, der Sohn Remaljas, sein Ritter, einen Bund wider
ihn und schlug ihn zu Samaria im Palast des Königshauses samt Argob und
Arje -- und mit ihm waren 50 Mann von den Kindern Gileads -- und tötete
ihn und ward König an seiner Statt. \footnote{\textbf{15:25} 2Kö 15,10;
  2Kö 15,14; 2Kö 15,30} \bibverse{26} Was aber mehr von Pekahja zu sagen
ist und alles, was er getan hat, siehe, das ist geschrieben in der
Chronik der Könige Israels. \bibverse{27} Im zweiundfünfzigsten Jahr
Asarjas, des Königs Judas, ward König Pekah, der Sohn Remaljas, über
Israel zu Samaria 20 Jahre; \bibverse{28} und er tat, was dem HErrn übel
gefiel; denn er ließ nicht von der Sünde Jerobeams, des Sohnes Nebats,
der Israel sündigen machte. \bibverse{29} Zu den Zeiten Pekahs, des
Königs Israels, kam Thiglath-Pileser, der König von Assyrien, und nahm
Ijon, Abel-Beth-Maacha, Janoah, Kedes, Hazor, Gilead und Galiläa, das
ganze Land Naphthali, und führte sie weg nach Assyrien. \footnote{\textbf{15:29}
  1Chr 5,26} \bibverse{30} Und Hosea, der Sohn Elas, machte einen Bund
wider Pekah, den Sohn Remaljas, und schlug ihn tot und ward König an
seiner Statt im zwanzigsten Jahr Jothams, des Sohnes Usias. \footnote{\textbf{15:30}
  2Kö 17,1; 2Kö 15,25} \bibverse{31} Was aber mehr von Pekah zu sagen
ist und alles, was er getan hat, siehe, das ist geschrieben in der
Chronik der Könige Israels. \bibverse{32} Im zweiten Jahr Pekahs, des
Sohnes Remaljas, des Königs über Israel, ward König Jotham, der Sohn
Usias, des Königs in Juda. \footnote{\textbf{15:32} 2Kö 15,5; 2Kö 15,7;
  2Chr 27,-1} \bibverse{33} Er war 25 Jahre alt, da er König ward, und
regierte 16 Jahre zu Jerusalem. Seine Mutter hieß Jerusa, eine Tochter
Zadoks. \bibverse{34} Und er tat, was dem HErrn wohl gefiel, ganz wie
sein Vater Usia getan hatte, \bibverse{35} nur, dass sie die Höhen nicht
abtaten; denn das Volk opferte und räucherte noch auf den Höhen. Er
baute das obere Tor am Hause des HErrn. \bibverse{36} Was aber mehr von
Jotham zu sagen ist und alles, was er getan hat, siehe, das ist
geschrieben in der Chronik der Könige Judas. \bibverse{37} Zu der Zeit
hob der HErr an, zu senden gegen Juda Rezin, den König von Syrien, und
Pekah, den Sohn Remaljas. \footnote{\textbf{15:37} 2Kö 16,5}
\bibverse{38} Und Jotham entschlief mit seinen Vätern, und ward begraben
bei seinen Vätern in der Stadt Davids, seines Vaters. \bibverse{39} Und
Ahas, sein Sohn, ward König an seiner Statt. \# 16 \bibverse{1} Im
siebzehnten Jahr Pekahs, des Sohnes Remaljas, ward König Ahas, der Sohn
Jothams, des Königs in Juda. \bibverse{2} Zwanzig Jahre war Ahas alt, da
er König ward, und regierte 16 Jahre zu Jerusalem; und er tat nicht, was
dem HErrn, seinem Gott, wohl gefiel, wie sein Vater David; \bibverse{3}
denn er wandelte auf dem Wege der Könige Israels. Dazu ließ er seinen
Sohn durchs Feuer gehen nach den Gräueln der Heiden, die der HErr vor
den Kindern Israel vertrieben hatte, \footnote{\textbf{16:3} 2Kö 21,6;
  3Mo 18,21} \bibverse{4} und tat Opfer und räucherte auf den Höhen und
auf den Hügeln und unter allen grünen Bäumen. \bibverse{5} Dazumal zogen
Rezin, der König von Syrien, und Pekah, der Sohn Remaljas, König in
Israel, hinauf gen Jerusalem, zu streiten, und belagerten Ahas; aber sie
konnten es nicht gewinnen. \bibverse{6} Zu derselben Zeit brachte Rezin,
König von Syrien, Elath wieder an Syrien und stieß die Juden aus Elath;
aber die Syrer kamen und wohnten darin bis auf diesen Tag. \footnote{\textbf{16:6}
  2Kö 14,22} \bibverse{7} Und Ahas sandte Boten zu Thiglath-Pileser, dem
König von Assyrien, und ließ ihm sagen: Ich bin dein Knecht und dein
Sohn; komm herauf und hilf mir aus der Hand des Königs von Syrien und
des Königs Israels, die sich wider mich haben aufgemacht! \footnote{\textbf{16:7}
  2Kö 15,29} \bibverse{8} Und Ahas nahm das Silber und Gold, das in dem
Hause des HErrn und in den Schätzen des Königshauses gefunden ward, und
sandte dem König von Assyrien Geschenke. \footnote{\textbf{16:8} 1Kö
  15,18} \bibverse{9} Und der König von Assyrien gehorchte ihm und zog
herauf gen Damaskus und gewann es und führte es weg gen Kir und tötete
Rezin. \bibverse{10} Und der König Ahas zog entgegen Thiglath-Pileser,
dem König zu Assyrien, gen Damaskus. Und da er einen Altar sah, der zu
Damaskus war, sandte der König Ahas desselben Altars Ebenbild und
Gleichnis zum Priester Uria, wie derselbe gemacht war. \bibverse{11} Und
Uria, der Priester, baute einen Altar und machte ihn, wie der König Ahas
zu ihm gesandt hatte von Damaskus, bis der König Ahas von Damaskus kam.
\bibverse{12} Und da der König von Damaskus kam und den Altar sah,
opferte er darauf \bibverse{13} und zündete darauf an sein Brandopfer
und Speisopfer und goss darauf sein Trankopfer und ließ das Blut der
Dankopfer, die er opferte, auf den Altar sprengen. \bibverse{14} Aber
den ehernen Altar, der vor dem HErrn stand, tat er weg, dass er nicht
stände zwischen dem Altar und dem Hause des HErrn, sondern setzte ihn an
die Seite des Altars gegen Mitternacht. \bibverse{15} Und der König Ahas
gebot Uria, dem Priester, und sprach: Auf dem großen Altar sollst du
anzünden die Brandopfer des Morgens und die Speisopfer des Abends und
die Brandopfer des Königs und sein Speisopfer und die Brandopfer alles
Volks im Lande samt ihrem Speisopfer und Trankopfer; und alles Blut der
Brandopfer und das Blut aller anderen Opfer sollst du darauf sprengen;
aber mit dem ehernen Altar will ich denken, was ich mache. \bibverse{16}
Uria, der Priester, tat alles, was ihn der König Ahas hieß.
\bibverse{17} Und der König Ahas brach ab die Seiten an den Gestühlen
und tat die Kessel oben davon; und das Meer tat er von den ehernen
Ochsen, die darunter waren, und setzte es auf ein steinernes Pflaster.
\bibverse{18} Dazu die bedeckte Sabbathalle, die sie am Hause gebaut
hatten, und den äußeren Eingang des Königs wandte er zum Hause des
HErrn, dem König von Assyrien zu Dienst. \bibverse{19} Was aber mehr von
Ahas zu sagen ist, was er getan hat, siehe, das ist geschrieben in der
Chronik der Könige Judas. \bibverse{20} Und Ahas entschlief mit seinen
Vätern und ward begraben bei seinen Vätern in der Stadt Davids. Und
Hiskia, sein Sohn, ward König an seiner Statt. \footnote{\textbf{16:20}
  2Kö 18,1}

\hypertarget{section-5}{%
\section{17}\label{section-5}}

\bibverse{1} Im zwölften Jahr des Ahas, des Königs in Juda, ward König
über Israel zu Samaria Hosea, der Sohn Elas, neun Jahre; \footnote{\textbf{17:1}
  2Kö 15,30} \bibverse{2} und er tat, was dem HErrn übel gefiel, doch
nicht wie die Könige Israels, die vor ihm waren. \bibverse{3} Wider
denselben zog herauf Salmanasser, der König von Assyrien. Und Hosea ward
ihm untertan, dass er ihm Geschenke gab. \footnote{\textbf{17:3} 2Kö
  18,9-12} \bibverse{4} Da aber der König von Assyrien inneward, dass
Hosea einen Bund anrichtete und hatte Boten zu So, dem König in Ägypten,
gesandt und nicht darreichte Geschenke dem König von Assyrien, wie alle
Jahre, griff er ihn und legte ihn ins Gefängnis. \footnote{\textbf{17:4}
  Hos 12,2} \bibverse{5} Nämlich der König von Assyrien zog über das
ganze Land und gen Samaria und belagerte es drei Jahre. \bibverse{6} Und
im neunten Jahr Hoseas gewann der König von Assyrien Samaria und führte
Israel weg nach Assyrien und setzte sie nach Halah und an den Habor, an
das Wasser Gosan und in die Städte der Meder. \bibverse{7} Denn die
Kinder Israel sündigten wider den HErrn, ihren Gott, der sie aus
Ägyptenland geführt hatte, aus der Hand Pharaos, des Königs von Ägypten,
und fürchteten andere Götter \bibverse{8} und wandelten nach der Heiden
Weise, die der HErr vor den Kindern Israel vertrieben hatte, und taten
wie die Könige Israels; \footnote{\textbf{17:8} 2Kö 16,3} \bibverse{9}
und die Kinder Israels schmückten ihre Sachen wider den HErrn, ihren
Gott, die doch nicht gut waren, also dass sie sich Höhen bauten in allen
Städten, von den Wachttürmen bis zu den festen Städten, \bibverse{10}
und richteten Säulen auf und Ascherabilder auf allen hohen Hügeln und
unter allen grünen Bäumen, \bibverse{11} und räucherten daselbst auf
allen Höhen wie die Heiden, die der HErr vor ihnen weggetrieben hatte,
und trieben böse Stücke, den HErrn zu erzürnen, \footnote{\textbf{17:11}
  2Kö 17,8} \bibverse{12} und dienten den Götzen, davon der HErr zu
ihnen gesagt hatte: Ihr sollt solches nicht tun; \footnote{\textbf{17:12}
  2Mo 20,2-3; 2Mo 23,13} \bibverse{13} und wenn der HErr bezeugte in
Israel und Juda durch alle Propheten und Schauer und ließ ihnen sagen:
Kehret um von euren bösen Wegen und haltet meine Gebote und Rechte nach
allem Gesetz, das ich euren Vätern geboten habe und das ich zu euch
gesandt habe durch meine Knechte, die Propheten: \bibverse{14} so
gehorchten sie nicht, sondern härteten ihren Nacken gleich dem Nacken
ihrer Väter, die nicht glaubten an den HErrn, ihren Gott; \bibverse{15}
dazu verachteten sie seine Gebote und seinen Bund, den er mit ihren
Vätern gemacht hatte, und seine Zeugnisse, die er unter ihnen tat, und
wandelten ihrer Eitelkeit nach und wurden eitel den Heiden nach, die um
sie her wohnten, von welchen ihnen der HErr geboten hatte, sie sollten
nicht wie sie tun; \footnote{\textbf{17:15} 2Mo 23,24} \bibverse{16}
aber sie verließen alle Gebote des HErrn, ihres Gottes, und machten sich
zwei gegossene Kälber und ein Ascherabild und beteten an alles Heer des
Himmels und dienten Baal \footnote{\textbf{17:16} 1Kö 12,28; 1Kö 16,33}
\bibverse{17} und ließen ihre Söhne und Töchter durchs Feuer gehen und
gingen mit Weissagen und Zaubern um und verkauften sich, zu tun, was dem
HErrn übel gefiel, ihn zu erzürnen: \footnote{\textbf{17:17} 2Kö 16,3}
\bibverse{18} da ward der HErr sehr zornig über Israel und tat sie von
seinem Angesicht, dass nichts übrigblieb denn der Stamm Juda allein.
\bibverse{19} (Dazu hielten auch die von Juda nicht die Gebote des
HErrn, ihres Gottes, und wandelten in den Sitten, darnach Israel getan
hatte.) \bibverse{20} Darum verwarf der HErr allen Samen Israels und
drängte sie und gab sie in die Hände der Räuber, bis dass er sie verwarf
von seinem Angesicht. \bibverse{21} Denn Israel ward gerissen vom Hause
Davids; und sie machten zum König Jerobeam, den Sohn Nebats. Derselbe
wandte Israel ab vom HErrn und machte, dass sie schwer sündigten.
\bibverse{22} Also wandelten die Kinder Israel in allen Sünden
Jerobeams, die er angerichtet hatte, und ließen nicht davon,
\bibverse{23} bis der HErr Israel von seinem Angesicht tat, wie er
geredet hatte durch alle seine Knechte, die Propheten. Also ward Israel
aus seinem Lande weggeführt nach Assyrien bis auf diesen Tag.
\footnote{\textbf{17:23} 5Mo 28,63-64} \bibverse{24} Der König aber von
Assyrien ließ kommen Leute von Babel, von Kutha, von Avva, von Hamath
und Sepharvaim und setzte sie in die Städte in Samaria anstatt der
Kinder Israel. Und sie nahmen Samaria ein und wohnten in derselben
Städten. \bibverse{25} Da sie aber anhoben daselbst zu wohnen und den
HErrn nicht fürchteten, sandte der HErr Löwen unter sie, die erwürgten
sie. \bibverse{26} Und sie ließen dem König von Assyrien sagen: Die
Heiden, die du hast hergebracht und die Städte Samarias damit besetzt,
wissen nichts von der Weise des Gottes im Lande; darum hat der HErr
Löwen unter sie gesandt und siehe, dieselben töten sie, weil sie nicht
wissen um die Weise des Gottes im Lande. \bibverse{27} Der König von
Assyrien gebot und sprach: Bringet dahin der Priester einen, die von
dort sind weggeführt, und ziehet hin und wohnet daselbst, und er lehre
sie die Weise des Gottes im Lande. \bibverse{28} Da kam der Priester
einer, die von Samaria weggeführt waren, und wohnte zu Beth-El und
lehrte sie, wie sie den HErrn fürchten sollten. \bibverse{29} Aber ein
jegliches Volk machte seinen Gott und taten sie in die Häuser auf den
Höhen, die die Samariter gemacht hatten, ein jegliches Volk in ihren
Städten, darin sie wohnten. \bibverse{30} Die von Babel machten
Sukkoth-Benoth; die von Chut machten Nergal; die von Hamath machten
Asima; \bibverse{31} die von Avva machten Nibehas und Tharthak; die von
Sepharvaim verbrannten ihre Söhne dem Adrammelech und Anammelech, den
Göttern derer von Sepharvaim. \bibverse{32} Und weil sie den HErrn auch
fürchteten, machten sie sich Priester auf den Höhen aus allem Volk unter
ihnen; die opferten für sie in den Häusern auf den Höhen. \bibverse{33}
Also fürchteten sie den HErrn und dienten auch den Göttern nach eines
jeglichen Volkes Weise, von wo sie hergebracht waren. \bibverse{34} Und
bis auf diesen Tag tun sie nach der alten Weise, dass sie weder den
HErrn fürchten noch ihre Sitten und Rechte tun nach dem Gesetz und
Gebot, das der HErr geboten hat den Kindern Jakobs, welchem er den Namen
Israel gab. \bibverse{35} Und er machte einen Bund mit ihnen und gebot
ihnen und sprach: Fürchtet keine anderen Götter und betet sie nicht an
und dienet ihnen nicht und opfert ihnen nicht; \footnote{\textbf{17:35}
  2Mo 23,24} \bibverse{36} sondern den HErrn, der euch aus Ägyptenland
geführt hat mit großer Kraft und ausgerecktem Arm, den fürchtet, den
betet an, und dem opfert; \bibverse{37} und die Sitten, Rechte, Gesetze
und Gebote, die er euch hat aufschreiben lassen, die haltet, dass ihr
darnach tut allewege und nicht andere Götter fürchtet; \bibverse{38} und
des Bundes, den er mit euch gemacht hat, vergesset nicht, dass ihr nicht
andere Götter fürchtet; \bibverse{39} sondern fürchtet den HErrn, euren
Gott, der wird euch erretten von allen euren Feinden. \bibverse{40} Aber
diese gehorchten nicht, sondern taten nach ihrer vorigen Weise.
\bibverse{41} Also fürchteten diese Heiden den HErrn und dienten auch
ihren Götzen. Also taten auch ihre Kinder und Kindeskinder, wie ihre
Väter getan haben, bis auf diesen Tag. \# 18 \bibverse{1} Im dritten
Jahr Hoseas, des Sohnes Elas, des Königs über Israel, ward König Hiskia,
der Sohn des Ahas, des Königs in Juda. \footnote{\textbf{18:1} 2Kö 16,20}
\bibverse{2} Er war 25 Jahre alt, da er König ward, und regierte 29
Jahre zu Jerusalem. Seine Mutter hieß Abi, eine Tochter Sacharjas.
\footnote{\textbf{18:2} 2Chr 29,1-2} \bibverse{3} Und er tat, was dem
HErrn wohl gefiel, wie sein Vater David. \footnote{\textbf{18:3} 2Kö
  20,3} \bibverse{4} Er tat ab die Höhen und zerbrach die Säulen und
rottete das Ascherabild aus und zerstieß die eherne Schlange, die Mose
gemacht hatte; denn bis zu der Zeit hatten ihr die Kinder Israel
geräuchert, und man hieß sie Nehusthan. \footnote{\textbf{18:4} 2Chr
  31,1; 2Kö 15,35; 4Mo 21,8-9} \bibverse{5} Er vertraute dem HErrn, dem
Gott Israels, dass nach ihm seinesgleichen nicht war unter allen Königen
Judas noch vor ihm gewesen. \footnote{\textbf{18:5} 2Kö 23,25}
\bibverse{6} Er hing dem HErrn an und wich nicht von ihm ab und hielt
seine Gebote, die der HErr dem Mose geboten hatte. \bibverse{7} Und der
HErr war mit ihm; und wo er auszog handelte er klüglich. Dazu ward er
abtrünnig vom König von Assyrien und war ihm nicht untertan.
\bibverse{8} Er schlug auch die Philister bis gen Gaza und ihr Gebiet
von den Wachttürmen an bis an die festen Städte. \bibverse{9} Im vierten
Jahr Hiskias, des Königs in Juda (das war das siebente Jahr Hoseas, des
Sohnes Elas, des Königs über Israel), da zog Salmanasser, der König von
Assyrien, herauf wider Samaria und belagerte es \bibverse{10} und gewann
es nach drei Jahren; im sechsten Jahr Hiskias, das ist im neunten Jahr
Hoseas, des Königs Israels, da ward Samaria gewonnen. \bibverse{11} Und
der König von Assyrien führte Israel weg gen Assyrien und setzte sie
nach Halah und an den Habor, an das Wasser Gosan und in die Städte der
Meder, \bibverse{12} darum dass sie nicht gehorcht hatten der Stimme des
HErrn, ihres Gottes, und übertreten hatten seinen Bund und alles, was
Mose, der Knecht des HErrn, geboten hatte; deren hatten sie keines
gehört noch getan. \bibverse{13} Im vierzehnten Jahr aber des Königs
Hiskia zog herauf Sanherib, der König von Assyrien, wider alle festen
Städte Judas und nahm sie ein. \bibverse{14} Da sandte Hiskia, der König
Judas, zum König von Assyrien gen Lachis und ließ ihm sagen: Ich habe
mich versündigt. Kehre um von mir; was du mir auflegst, will ich tragen.
Da legte der König von Assyrien Hiskia, dem Könige Judas, 300 Zentner
Silber auf und 30 Zentner Gold. \footnote{\textbf{18:14} 2Kö 18,7}
\bibverse{15} Also gab Hiskia all das Silber, das im Hause des HErrn und
in den Schätzen des Königshauses gefunden ward. \footnote{\textbf{18:15}
  2Kö 16,8} \bibverse{16} Zur selben Zeit zerbrach Hiskia, der König
Judas, die Türen am Tempel des HErrn und die Bleche, die er selbst hatte
darüberziehen lassen, und gab sie dem König von Assyrien. \bibverse{17}
Und der König von Assyrien sandte den Tharthan und den Erzkämmerer und
den Erzschenken von Lachis zum König Hiskia mit großer Macht gen
Jerusalem, und sie zogen herauf. Und da sie hinkamen, hielten sie an der
Wasserleitung des oberen Teichs, der da liegt an der Straße bei dem
Acker des Walkmüllers, \bibverse{18} und riefen nach dem König. Da kam
heraus zu ihnen Eljakim, der Sohn Hilkias, der Hofmeister, und Sebna,
der Schreiber, und Joah, der Sohn Asaphs, der Kanzler. \bibverse{19} Und
der Erzschenke sprach zu ihnen: Sagt doch dem König Hiskia: So spricht
der große König, der König von Assyrien: Was ist das für ein Trotz,
darauf du dich verlässest? \bibverse{20} Meinst du, es sei noch Rat und
Macht, zu streiten? Worauf verlässest du denn nun dich, dass du mir
abtrünnig geworden bist? \bibverse{21} Siehe, verlässest du dich auf
diesen zerstoßenen Rohrstab, auf Ägypten, welcher, so sich jemand darauf
lehnt, wird er ihm in die Hand gehen und sie durchbohren? Also ist
Pharao, der König in Ägypten, allen, die sich auf ihn verlassen.
\bibverse{22} Ob ihr aber wolltet zu mir sagen: Wir verlassen uns auf
den HErrn, unseren Gott! ist's denn nicht der, dessen Höhen und Altäre
Hiskia hat abgetan und gesagt zu Juda und zu Jerusalem: Vor diesem
Altar, der zu Jerusalem ist, sollt ihr anbeten? \footnote{\textbf{18:22}
  2Mo 20,24; 5Mo 12,14} \bibverse{23} Wohlan, nimm eine Wette an mit
meinem Herrn, dem König von Assyrien: ich will dir 2000 Rosse geben, ob
du könnest Reiter dazu geben. \bibverse{24} Wie willst du denn bleiben
vor der geringsten Hauptleute einem von meines Herrn Untertanen? Und du
verlässest dich auf Ägypten um der Wagen und Reiter willen.
\bibverse{25} Meinst du aber, ich sei ohne den HErrn heraufgezogen, dass
ich diese Stätte verderbe? Der HErr hat mich's geheißen: Ziehe hinauf in
dieses Land und verderbe es! \bibverse{26} Da sprach Eljakim, der Sohn
Hilkias, und Sebna und Joah zum Erzschenken: Rede mit deinen Knechten
auf syrisch, denn wir verstehen's; und rede nicht mit uns auf jüdisch
vor den Ohren des Volks, das auf der Mauer ist. \bibverse{27} Aber der
Erzschenke sprach zu ihnen: Hat mich denn mein Herr zu deinem Herrn oder
zu dir gesandt, dass ich solche Worte rede? und nicht vielmehr zu den
Männern, die auf der Mauer sitzen, dass sie mit euch ihren eigenen Mist
fressen und ihren Harn saufen? \bibverse{28} Also stand der Erzschenke
auf und redete mit lauter Stimme auf jüdisch und sprach: Höret das Wort
des großen Königs, des Königs von Assyrien! \bibverse{29} So spricht der
König: Lasst euch Hiskia nicht betrügen; denn er vermag euch nicht zu
erretten von meiner Hand. \bibverse{30} Und lasst euch Hiskia nicht
vertrösten auf den HErrn, dass er sagt: Der HErr wird uns erretten, und
diese Stadt wird nicht in die Hände des Königs von Assyrien gegeben
werden. \bibverse{31} Gehorchet Hiskia nicht! Denn so spricht der König
von Assyrien: Nehmet an meine Gnade und kommet zu mir heraus, so soll
jedermann von seinem Weinstock und seinem Feigenbaum essen und von
seinem Brunnen trinken, \bibverse{32} bis ich komme und hole euch in ein
Land, das eurem Lande gleich ist, darin Korn, Most, Brot, Weinberge,
Ölbäume und Honig sind; so werdet ihr leben bleiben und nicht sterben.
Gehorchet Hiskia nicht; denn er verführt euch, dass er spricht: Der HErr
wird uns erretten. \bibverse{33} Haben auch die Götter der Heiden ein
jeglicher sein Land errettet von der Hand des Königs von Assyrien?
\footnote{\textbf{18:33} Jes 10,10-11} \bibverse{34} Wo sind die Götter
zu Hamath und Arpad? Wo sind die Götter zu Sepharvaim, Hena und Iwwa?
Haben sie auch Samaria errettet von meiner Hand? \bibverse{35} Wo ist
ein Gott unter aller Lande Göttern, die ihr Land haben von meiner Hand
errettet, dass der HErr sollte Jerusalem von meiner Hand erretten?
\bibverse{36} Das Volk aber schwieg still und antwortete ihm nichts;
denn der König hatte geboten und gesagt: Antwortet ihm nichts.
\bibverse{37} Da kamen Eljakim, der Sohn Hilkias, der Hofmeister, und
Sebna, der Schreiber, und Joah, der Sohn Asaphs, der Kanzler, zu Hiskia
mit zerrissenen Kleidern und sagten ihm an die Worte des Erzschenken. \#
19 \bibverse{1} Da der König Hiskia das hörte, zerriss er seine Kleider
und legte einen Sack an und ging in das Haus des HErrn \bibverse{2} und
sandte Eljakim, den Hofmeister, und Sebna, den Schreiber, samt den
Ältesten der Priester, mit Säcken angetan, zu dem Propheten Jesaja, dem
Sohn des Amoz; \bibverse{3} und sie sprachen zu ihm: So sagt Hiskia: Das
ist ein Tag der Not, des Scheltens und Lästerns; die Kinder sind
gekommen an die Geburt und ist keine Kraft da, zu gebären. \bibverse{4}
Ob vielleicht der HErr, dein Gott, hören wollte alle Worte des
Erzschenken, den sein Herr, der König von Assyrien, gesandt hat, Hohn zu
sprechen dem lebendigen Gott und zu schelten mit Worten, die der HErr,
dein Gott, gehört hat: so erhebe dein Gebet für die Übrigen, die noch
vorhanden sind. \bibverse{5} Und da die Knechte des Königs Hiskia zu
Jesaja kamen, \bibverse{6} sprach Jesaja zu ihnen: So sagt eurem Herrn:
So spricht der HErr: Fürchte dich nicht vor den Worten, die du gehört
hast, womit mich die Knechte des Königs von Assyrien gelästert haben.
\bibverse{7} Siehe, ich will ihm einen Geist geben, dass er ein Gerücht
hören wird und wieder in sein Land ziehen, und will ihn durchs Schwert
fällen in seinem Lande. \footnote{\textbf{19:7} 2Kö 9,35-37}
\bibverse{8} Und da der Erzschenke wiederkam, fand er den König von
Assyrien streiten wider Libna; denn er hatte gehört, dass er von Lachis
gezogen war. \bibverse{9} Und da er hörte von Thirhaka, dem König der
Mohren: Siehe, er ist ausgezogen, mit dir zu streiten, sandte er
abermals Boten zu Hiskia und ließ ihm sagen: \bibverse{10} So sagt
Hiskia, dem König Judas: Lass dich deinen Gott nicht betrügen, auf den
du dich verlässest und sprichst: Jerusalem wird nicht in die Hand des
Königs von Assyrien gegeben werden. \bibverse{11} Siehe, du hast gehört,
was die Könige von Assyrien getan haben allen Landen und sie verbannt;
und du solltest errettet werden? \bibverse{12} Haben der Heiden Götter
auch sie errettet, welche meine Väter haben verderbt: Gosan, Haran,
Rezeph und die Kinder Edens, die zu Thelassar waren? \footnote{\textbf{19:12}
  2Kö 18,33-34} \bibverse{13} Wo ist der König von Hamath, der König zu
Arpad und der König der Stadt Sepharvaim, von Hena und Iwwa?
\bibverse{14} Und da Hiskia den Brief von den Boten empfangen und
gelesen hatte, ging er hinauf zum Hause des HErrn und breitete ihn aus
vor dem HErrn \bibverse{15} und betete vor dem HErrn und sprach: HErr,
Gott Israels, der du über den Cherubim sitzest, du bist allein Gott über
alle Königreiche auf Erden, du hast Himmel und Erde gemacht.
\bibverse{16} HErr, neige deine Ohren und höre; tue deine Augen auf und
siehe, und höre die Worte Sanheribs, der hergesandt hat, Hohn zu
sprechen dem lebendigen Gott. \footnote{\textbf{19:16} 2Kö 19,4; 1Sam
  17,10} \bibverse{17} Es ist wahr HErr, die Könige von Assyrien haben
die Heiden mit dem Schwert umgebracht und ihr Land \bibverse{18} und
haben ihre Götter ins Feuer geworfen. Denn es waren nicht Götter,
sondern Werke von Menschenhänden, Holz und Stein; darum haben sie sie
vertilgt. \bibverse{19} Nun aber, HErr, unser Gott, hilf uns aus seiner
Hand, auf dass alle Königreiche auf Erden erkennen, dass du, HErr,
allein Gott bist. \bibverse{20} Da sandte Jesaja, der Sohn des Amoz, zu
Hiskia und ließ ihm sagen: So spricht der HErr, der Gott Israels: Was du
zu mir gebetet hast um Sanherib, den König von Assyrien, das habe ich
gehört. \bibverse{21} Das ist's, was der HErr wider ihn geredet hat: Die
Jungfrau, die Tochter Zion, verachtet dich und spottet dein; die Tochter
Jerusalem schüttelt ihr Haupt dir nach. \bibverse{22} Wen hast du
gehöhnt und gelästert? Über wen hast du deine Stimme erhoben? Du hast
deine Augen erhoben wider den Heiligen in Israel. \bibverse{23} Du hast
den HErrn durch deine Boten gehöhnt und gesagt: „Ich bin durch die Menge
meiner Wagen auf die Höhen der Berge gestiegen, auf den innersten
Libanon; ich habe seine hohen Zedern und auserlesenen Tannen abgehauen
und bin gekommen an seine äußerste Herberge, an den Wald seines
Baumgartens. \bibverse{24} Ich habe gegraben und ausgetrunken die
fremden Wasser und werde austrocknen mit meinen Fußsohlen alle Flüsse
Ägyptens.`` \bibverse{25} Hast du aber nicht gehört, dass ich solches
lange zuvor getan habe, und von Anfang habe ich's bereitet? Nun aber
habe ich's kommen lassen, dass feste Städte werden fallen in einen
wüsten Steinhaufen, \bibverse{26} und die darin wohnen, matt werden und
sich fürchten und schämen müssen und werden wie das Gras auf dem Felde
und wie das grüne Kraut, wie Gras auf den Dächern, das verdorrt, ehe
denn es reif wird. \bibverse{27} Ich weiß dein Wohnen, dein Aus- und
Einziehen und dass du tobst wider mich. \bibverse{28} Weil du denn wider
mich tobst und dein Übermut vor meine Ohren heraufgekommen ist, so will
ich dir einen Ring an deine Nase legen und ein Gebiss in dein Maul und
will dich den Weg wieder zurückführen, da du hergekommen bist.
\bibverse{29} Und das sei dir ein Zeichen: In diesem Jahr iss, was von
selber wächst; im anderen Jahr, was noch aus den Wurzeln wächst: im
dritten Jahr säet und erntet, und pflanzet Weinberge und esset ihre
Früchte. \bibverse{30} Und was vom Hause Judas errettet und
übriggeblieben ist, wird fürder unter sich wurzeln und über sich Frucht
tragen. \bibverse{31} Denn von Jerusalem werden ausgehen, die
übriggeblieben sind, und die Erretteten vom Berge Zion. Der Eifer des
HErrn Zebaoth wird solches tun. \bibverse{32} Darum spricht der HErr vom
König von Assyrien also: Er soll nicht in diese Stadt kommen und keinen
Pfeil hineinschießen und mit keinem Schilde davorkommen und soll keinen
Wall darum schütten; \bibverse{33} sondern er soll den Weg wiederum
ziehen, den er gekommen ist, und soll in diese Stadt nicht kommen; der
HErr sagt's. \bibverse{34} Und ich will diese Stadt beschirmen, dass ich
ihr helfe um meinetwillen und um Davids, meines Knechtes, willen.
\footnote{\textbf{19:34} 2Kö 20,6} \bibverse{35} Und in derselben Nacht
fuhr aus der Engel des HErrn und schlug im Lager von Assyrien 185.000
Mann. Und da sie sich des Morgens früh aufmachten, siehe, da lag's alles
eitel tote Leichname. \bibverse{36} Also brach Sanherib, der König von
Assyrien, auf und zog weg und kehrte um und blieb zu Ninive.
\bibverse{37} Und da er anbetete im Hause Nisrochs, seines Gottes,
erschlugen ihn mit dem Schwert Adrammelech und Sarezer, seine Söhne, und
sie entrannen ins Land Ararat. Und sein Sohn Asar-Haddon ward König an
seiner Statt. \# 20 \bibverse{1} Zu der Zeit ward Hiskia todkrank. Und
der Prophet Jesaja, der Sohn des Amoz, kam zu ihm und sprach zu ihm: So
spricht der HErr: Beschicke dein Haus; denn du wirst sterben und nicht
leben bleiben! \bibverse{2} Er aber wandte sein Antlitz zur Wand und
betete zum HErrn und sprach: \bibverse{3} Ach, HErr, gedenke doch, dass
ich vor dir treulich gewandelt habe und mit rechtschaffenem Herzen und
habe getan, was dir wohl gefällt. Und Hiskia weinte sehr. \bibverse{4}
Da aber Jesaja noch nicht zur Stadt halb hinausgegangen war, kam des
HErrn Wort zu ihm und sprach: \bibverse{5} Kehre um und sage Hiskia, dem
Fürsten meines Volks: So spricht der HErr, der Gott deines Vaters David:
Ich habe dein Gebet gehört und deine Tränen gesehen. Siehe, ich will
dich gesund machen -- am dritten Tage wirst du hinauf in das Haus des
HErrn gehen -- \bibverse{6} und will 15 Jahre zu deinem Leben tun und
dich und diese Stadt erretten von dem König von Assyrien und diese Stadt
beschirmen um meinetwillen und um meines Knechtes David willen.
\footnote{\textbf{20:6} 2Kö 19,34} \bibverse{7} Und Jesaja sprach:
Bringet her ein Pflaster von Feigen! Und da sie das brachten, legten sie
es auf die Drüse; und er ward gesund. \bibverse{8} Hiskia aber sprach zu
Jesaja: Welches ist das Zeichen, dass mich der HErr wird gesund machen
und ich in des HErrn Haus hinaufgehen werde am dritten Tage?
\bibverse{9} Jesaja sprach: Das Zeichen wirst du haben vom HErrn, dass
der HErr tun wird, was er geredet hat: Soll der Schatten zehn Stufen
fürdergehen oder zehn Stufen zurückgehen? \bibverse{10} Hiskia sprach:
Es ist leicht, dass der Schatten zehn Stufen niederwärts gehe; das will
ich nicht, sondern dass er zehn Stufen hinter sich zurückgehe.
\bibverse{11} Da rief der Prophet Jesaja den HErrn an; und der Schatten
ging hinter sich zurück zehn Stufen am Zeiger Ahas, die er war
niederwärts gegangen. \bibverse{12} Zu der Zeit sandte Berodach-Baladan,
der Sohn Baladans, König zu Babel, Briefe und Geschenke zu Hiskia; denn
er hatte gehört, dass Hiskia krank gewesen war. \bibverse{13} Hiskia
aber war fröhlich mit ihnen und zeigte ihnen das ganze Schatzhaus,
Silber, Gold, Spezerei und das beste Öl, und das Zeughaus und alles, was
in seinen Schätzen vorhanden war. Es war nichts in seinem Hause und in
seiner ganzen Herrschaft, das ihnen Hiskia nicht zeigte. \bibverse{14}
Da kam Jesaja, der Prophet, zum König Hiskia und sprach zu ihm: Was
haben diese Leute gesagt? und woher sind sie zu dir gekommen? Hiskia
sprach: Sie sind aus fernen Landen zu mir gekommen, von Babel.
\bibverse{15} Er sprach: Was haben sie gesehen in deinem Hause? Hiskia
sprach: Sie haben alles gesehen, was in meinem Hause ist, und ist nichts
in meinen Schätzen, was ich ihnen nicht gezeigt hätte. \bibverse{16} Da
sprach Jesaja zu Hiskia: Höre des HErrn Wort: \bibverse{17} Siehe, es
kommt die Zeit, dass alles wird gen Babel weggeführt werden aus deinem
Hause und was deine Väter gesammelt haben bis auf diesen Tag; und wird
nichts übriggelassen werden, spricht der HErr. \bibverse{18} Dazu von
den Kindern, die von dir kommen, die du zeugen wirst, werden sie nehmen,
dass sie Kämmerer seien im Palast des Königs zu Babel. \footnote{\textbf{20:18}
  Dan 1,3-4} \bibverse{19} Hiskia aber sprach zu Jesaja: Das ist gut,
was der HErr geredet hat, -- und sprach weiter: Es wird doch Friede und
Treue sein zu meinen Zeiten. \footnote{\textbf{20:19} 1Sam 3,18}
\bibverse{20} Was mehr von Hiskia zu sagen ist und alle seine Macht und
was er getan hat und der Teich und die Wasserleitung, durch die er
Wasser in die Stadt geleitet hat, siehe, das ist geschrieben in der
Chronik der Könige Judas. \bibverse{21} Und Hiskia entschlief mit seinen
Vätern. Und Manasse, sein Sohn, ward König an seiner Statt. \# 21
\bibverse{1} Manasse war zwölf Jahre alt, da er König ward, und regierte
55 Jahre zu Jerusalem. Seine Mutter hieß Hephzibah. \bibverse{2} Und er
tat, was dem HErrn übel gefiel, nach den Gräueln der Heiden, die der
HErr vor den Kindern Israel vertrieben hatte, \bibverse{3} und baute
wieder die Höhen, die sein Vater Hiskia hatte zerstört, und richtete dem
Baal Altäre auf und machte ein Ascherabild, wie Ahab, der König Israels,
getan hatte, und betete an alles Heer des Himmels und diente ihnen.
\footnote{\textbf{21:3} 1Kö 16,33} \bibverse{4} Und baute Altäre im
Hause des HErrn, davon der HErr gesagt hatte: Ich will meinen Namen zu
Jerusalem setzen; \footnote{\textbf{21:4} 2Kö 21,7} \bibverse{5} und er
baute allem Heer des Himmels Altäre in beiden Höfen am Hause des HErrn.
\footnote{\textbf{21:5} 2Kö 23,12} \bibverse{6} Und ließ seinen Sohn
durchs Feuer gehen und achtete auf Vogelgeschrei und Zeichen und hielt
Wahrsager und Zeichendeuter und tat des viel, das dem HErrn übel gefiel,
ihn zu erzürnen. \footnote{\textbf{21:6} 2Kö 16,3} \bibverse{7} Er
setzte auch das Bild der Aschera, das er gemacht hatte, in das Haus, von
welchem der HErr zu David und zu Salomo, seinem Sohn, gesagt hatte: In
dieses Haus und nach Jerusalem, das ich erwählt habe aus allen Stämmen
Israels, will ich meinen Namen setzen ewiglich; \footnote{\textbf{21:7}
  1Kö 8,29; 1Kö 9,3} \bibverse{8} und will den Fuß Israels nicht mehr
bewegen lassen von dem Lande, das ich ihren Vätern gegeben habe, -- so
doch, dass sie halten und tun nach allem, was ich geboten habe und nach
allem Gesetz, das mein Knecht Mose ihnen geboten hat. \bibverse{9} Aber
sie gehorchten nicht; sondern Manasse verführte sie, dass sie ärger
taten denn die Heiden, die der HErr vor den Kindern Israel vertilgt
hatte. \bibverse{10} Da redete der HErr durch seine Knechte, die
Propheten, und sprach: \bibverse{11} Darum dass Manasse, der König
Judas, hat diese Gräuel getan, die ärger sind denn alle Gräuel, so die
Amoriter getan haben, die vor ihm gewesen sind, und hat auch Juda
sündigen gemacht mit seinen Götzen; \bibverse{12} darum spricht der
HErr, der Gott Israels, also: Siehe, ich will Unglück über Jerusalem und
Juda bringen, dass wer es hören wird, dem sollen seine beiden Ohren
gellen; \bibverse{13} und will über Jerusalem die Messschnur Samarias
ziehen und das Richtblei des Hauses Ahab; und will Jerusalem
ausschütten, wie man Schüsseln ausschüttet, und will sie umstürzen;
\bibverse{14} und ich will die Übrigen meines Erbteils verstoßen und sie
geben in die Hände ihrer Feinde, dass sie ein Raub und Reißen werden
aller ihrer Feinde, -- \bibverse{15} darum dass sie getan haben, was mir
übel gefällt, und haben mich erzürnt von dem Tage an, da ihre Väter aus
Ägypten gezogen sind, bis auf diesen Tag. \bibverse{16} Auch vergoss
Manasse sehr viel unschuldiges Blut, bis dass Jerusalem allerorten voll
ward, -- außer der Sünde, durch die er Juda sündigen machte, dass sie
taten, was dem HErrn übel gefiel. \footnote{\textbf{21:16} 2Kö 24,4}
\bibverse{17} Was aber mehr von Manasse zu sagen ist und alles, was er
getan hat, und seine Sünde, die er tat, siehe, das ist geschrieben in
der Chronik der Könige Judas. \bibverse{18} Und Manasse entschlief mit
seinen Vätern und ward begraben im Garten an seinem Hause, im Garten
Usas. Und sein Sohn Amon ward König an seiner Statt. \bibverse{19}
Zweiundzwanzig Jahre alt war Amon, da er König ward, und regierte zwei
Jahre zu Jerusalem. Seine Mutter hieß Mesullemeth, eine Tochter des
Haruz von Jotba. \bibverse{20} Und er tat, was dem HErrn übel gefiel,
wie sein Vater Manasse getan hatte, \bibverse{21} und wandelte in allem
Wege, den sein Vater gewandelt hatte, und diente den Götzen, welchen
sein Vater gedient hatte, und betete sie an \bibverse{22} und verließ
den HErrn, seiner Väter Gott, und wandelte nicht im Wege des HErrn.
\bibverse{23} Und seine Knechte machten einen Bund wider Amon und
töteten den König in seinem Hause. \footnote{\textbf{21:23} 2Kö 14,19}
\bibverse{24} Aber das Volk im Land schlug alle, die den Bund gemacht
hatten wider den König Amon. Und das Volk im Lande machte Josia, seinen
Sohn, zum König an seiner Statt. \bibverse{25} Was aber Amon mehr getan
hat, siehe, das ist geschrieben in der Chronik der Könige Judas.
\bibverse{26} Und man begrub ihn in seinem Grabe im Garten Usas. Und
sein Sohn Josia ward König an seiner Statt. \# 22 \bibverse{1} Josia war
acht Jahre alt, da er König ward, und regierte 31 Jahre zu Jerusalem.
Seine Mutter hieß Jedida, eine Tochter Adajas von Bozkath. \footnote{\textbf{22:1}
  2Chr 34,1-2; 2Chr 34,8-11} \bibverse{2} Und er tat was dem HErrn wohl
gefiel, und wandelte in allem Wege seines Vaters David und wich nicht,
weder zur Rechten noch zur Linken. \footnote{\textbf{22:2} 2Kö 18,3; 5Mo
  5,29} \bibverse{3} Und im achtzehnten Jahr des Königs Josia sandte der
König hin Saphan, den Sohn Azaljas, des Sohnes Mesullams, den Schreiber,
in das Haus des HErrn und sprach: \bibverse{4} Gehe hinauf zu dem
Hohenpriester Hilkia, dass er abgebe alles Geld, das zum Hause des HErrn
gebracht ist, das die Türhüter gesammelt haben vom Volk, \bibverse{5}
dass man es gebe den Werkmeistern, die bestellt sind im Hause des HErrn,
und sie es geben den Arbeitern am Hause des HErrn, dass sie bessern, was
baufällig ist am Hause, \bibverse{6} nämlich den Zimmerleuten und
Bauleuten und Maurern und denen, die da Holz und gehauene Steine kaufen
sollen, das Haus zu bessern; \bibverse{7} doch dass man keine Rechnung
von ihnen nehme von dem Geld, das unter ihre Hand getan wird, sondern
dass sie auf Glauben handeln. \footnote{\textbf{22:7} 2Kö 12,16}
\bibverse{8} Und der Hohepriester Hilkia sprach zu dem Schreiber Saphan:
Ich habe das Gesetzbuch gefunden im Hause des HErrn. Und Hilkia gab das
Buch Saphan, dass er's läse. \bibverse{9} Und Saphan, der Schreiber kam
zum König und gab ihm Bericht und sprach: Deine Knechte haben das Geld
ausgeschüttet, das im Hause gefunden ist und haben's den Werkmeistern
gegeben, die bestellt sind am Hause des HErrn. \bibverse{10} Auch sagte
Saphan, der Schreiber, dem König und sprach: Hilkia, der Priester, gab
mir ein Buch. Und Saphan las es vor dem König. \bibverse{11} Da aber der
König hörte die Worte im Gesetzbuch, zerriss er seine Kleider.
\bibverse{12} Und der König gebot Hilkia, dem Priester, und Ahikam, dem
Sohn Saphans, und Achbor, dem Sohn Michajas, und Saphan, dem Schreiber,
und Asaja, dem Knecht des Königs, und sprach: \bibverse{13} Gehet hin
und fraget den HErrn für mich, für das Volk und für ganz Juda um die
Worte dieses Buchs, das gefunden ist; denn es ist ein großer Grimm des
HErrn, der über uns entbrannt ist, darum dass unsere Väter nicht
gehorcht haben den Worten dieses Buchs, dass sie täten alles, was darin
geschrieben ist. \bibverse{14} Da gingen hin Hilkia, der Priester,
Ahikam, Achbor, Saphan und Asaja zu der Prophetin Hulda, dem Weibe
Sallums, des Sohnes Thikwas, des Sohnes Harhas, des Hüters der Kleider,
und sie wohnte zu Jerusalem im anderen Teil; und sie redeten mit ihr.
\bibverse{15} Sie aber sprach zu ihnen: So spricht der HErr, der Gott
Israels: Saget dem Mann, der euch zu mir gesandt hat: \bibverse{16} So
spricht der HErr: Siehe, ich will Unglück über diese Stätte und ihre
Einwohner bringen, alle Worte des Gesetzes, die der König Judas hat
lassen lesen. \bibverse{17} Darum dass sie mich verlassen und anderen
Göttern geräuchert haben, mich zu erzürnen mit allen Werken ihrer Hände,
darum wird mein Grimm sich wider diese Stätte entzünden und nicht
ausgelöscht werden. \bibverse{18} Aber dem König Judas, der euch gesandt
hat, den HErrn zu fragen, sollt ihr so sagen: So spricht der HErr, der
Gott Israels: \bibverse{19} Darum dass dein Herz erweicht ist über den
Worten, die du gehört hast, und hast dich gedemütigt vor dem HErrn, da
du hörtest, was ich geredet habe wider diese Stätte und ihre Einwohner,
dass sie sollen eine Verwüstung und ein Fluch sein, und hast deine
Kleider zerrissen und hast geweint vor mir, so habe ich's auch erhört,
spricht der HErr. \bibverse{20} Darum will ich dich zu deinen Vätern
sammeln, dass du mit Frieden in dein Grab versammelt werdest und deine
Augen nicht sehen all das Unglück, das ich über diese Stätte bringen
will. Und sie sagten es dem König wieder. \footnote{\textbf{22:20} Jes
  57,1-2}

\hypertarget{section-6}{%
\section{23}\label{section-6}}

\bibverse{1} Und der König sandte hin, und es versammelten sich zu ihm
alle Ältesten in Juda und Jerusalem. \bibverse{2} Und der König ging
hinauf ins Haus des HErrn und alle Männer von Juda und alle Einwohner zu
Jerusalem mit ihm, Priester und Propheten, und alles Volk, klein und
groß; und man las vor ihren Ohren alle Worte aus dem Buch des Bundes,
das im Hause des HErrn gefunden war. \bibverse{3} Und der König trat an
die Säule und machte einen Bund vor dem HErrn, dass sie sollten wandeln
dem HErrn nach und halten seine Gebote, Zeugnisse und Rechte von ganzem
Herzen und von ganzer Seele, dass sie aufrichteten die Worte dieses
Bundes, die geschrieben standen in diesem Buch. Und alles Volk trat in
den Bund. \bibverse{4} Und der König gebot dem Hohenpriester Hilkia und
den nächsten Priestern nach ihm und den Hütern an der Schwelle, dass sie
sollten aus dem Tempel des HErrn tun alle Geräte, die dem Baal und der
Aschera und allem Heer des Himmels gemacht waren. Und sie verbrannten
sie außen vor Jerusalem im Tal Kidron, und ihr Staub ward getragen gen
Beth-El. \footnote{\textbf{23:4} 2Kö 21,3} \bibverse{5} Und er tat ab
die Götzenpfaffen, welche die Könige Judas hatten eingesetzt, zu
räuchern auf den Höhen in den Städten Judas und um Jerusalem her, auch
die Räucherer des Baal und der Sonne und des Mondes und der Planeten und
alles Heeres am Himmel. \bibverse{6} Und ließ das Ascherabild aus dem
Hause des HErrn führen hinaus vor Jerusalem an den Bach Kidron und
verbrannte es am Bach Kidron und machte es zu Staub und warf den Staub
auf die Gräber der gemeinen Leute. \bibverse{7} Und er brach ab die
Häuser der Hurer, die an dem Hause des HErrn waren, darin die Weiber
wirkten Häuser für die Aschera. \bibverse{8} Und er ließ kommen alle
Priester aus den Städten Judas und verunreinigte die Höhen, da die
Priester räucherten, von Geba an bis gen Beer-Seba, und brach ab die
Höhen an den Toren, die an der Tür des Tors Josuas, des Stadtvogts,
waren und zur Linken, wenn man zum Tor der Stadt geht. \bibverse{9} Doch
durften die Priester der Höhen nicht opfern auf dem Altar des HErrn zu
Jerusalem, sondern aßen ungesäuertes Brot unter ihren Brüdern.
\bibverse{10} Er verunreinigte auch das Thopheth im Tal der Kinder
Hinnom, dass niemand seinen Sohn oder seine Tochter dem Moloch durchs
Feuer ließe gehen. \footnote{\textbf{23:10} 2Kö 17,17; 3Mo 18,21}
\bibverse{11} Und tat ab die Rosse, welche die Könige Judas hatten der
Sonne gesetzt am Eingang des Hauses des HErrn, an der Kammer
Nethan-Melechs, des Kämmerers, die im Parwarim war; und die Wagen der
Sonne verbrannte er mit Feuer. \bibverse{12} Und die Altäre auf dem
Dach, dem Söller des Ahas, die die Könige Judas gemacht hatten, und die
Altäre, die Manasse gemacht hatte in den zwei Höfen des Hauses des
HErrn, brach der König ab, und lief von dannen und warf ihren Staub in
den Bach Kidron. \bibverse{13} Auch die Höhen, die vor Jerusalem waren,
zur Rechten am Berge des Verderbens, die Salomo, der König Israels,
gebaut hatte der Asthoreth, dem Gräuel von Sidon, und Kamos, dem Gräuel
von Moab, und Milkom, dem Gräuel der Kinder Ammon, verunreinigte der
König, \footnote{\textbf{23:13} 1Kö 11,7} \bibverse{14} und zerbrach die
Säulen und rottete aus die Ascherabilder und füllte ihre Stätte mit
Menschenknochen. \bibverse{15} Auch den Altar zu Beth-El, die Höhe, die
Jerobeam gemacht hatte, der Sohn Nebats, der Israel sündigen machte,
denselben Altar brach er ab und die Höhe und verbrannte die Höhe und
machte sie zu Staub und verbrannte das Ascherabild. \bibverse{16} Und
Josia wandte sich und sah die Gräber, die da waren auf dem Berge, und
sandte hin und ließ die Knochen aus den Gräbern holen und verbrannte sie
auf dem Altar und verunreinigte ihn nach dem Wort des HErrn, das der
Mann Gottes ausgerufen hatte, der solches ausrief. \footnote{\textbf{23:16}
  1Kö 13,2} \bibverse{17} Und er sprach: Was ist das für ein Grabmal,
das ich sehe? Und die Leute in der Stadt sprachen zu ihm: Es ist das
Grab des Mannes Gottes, der von Juda kam und rief solches aus, das du
getan hast wider den Altar zu Beth-El. \footnote{\textbf{23:17} 1Kö
  13,30} \bibverse{18} Und er sprach: Lasst ihn liegen; niemand bewege
seine Gebeine! Also wurden seine Gebeine errettet mit den Gebeinen des
Propheten, der von Samaria gekommen war. \bibverse{19} Er tat auch weg
alle Häuser der Höhen in den Städten Samarias, welche die Könige Israels
gemacht hatten, (den HErrn) zu erzürnen, und tat mit ihnen ganz, wie er
zu Beth-El getan hatte. \bibverse{20} Und er opferte alle Priester der
Höhen, die daselbst waren, auf den Altären und verbrannte also
Menschengebeine darauf und kam wieder gen Jerusalem. \bibverse{21} Und
der König gebot dem Volk und sprach: Haltet dem HErrn, eurem Gott,
Passah, wie es geschrieben steht in diesem Buch des Bundes! \footnote{\textbf{23:21}
  2Mo 12,-1; 2Chr 35,1-19} \bibverse{22} Denn es war kein Passah so
gehalten wie dieses von der Richter Zeit an, die Israel gerichtet haben,
und in allen Zeiten der Könige Israels und der Könige Judas
\bibverse{23} sondern im achtzehnten Jahr des Königs Josia ward dieses
Passah gehalten dem HErrn zu Jerusalem. \bibverse{24} Auch fegte Josia
aus alle Wahrsager, Zeichendeuter, Bilder und Götzen und alle Gräuel,
die im Lande Juda und zu Jerusalem gesehen wurden, auf dass er
aufrichtete die Worte des Gesetzes, die geschrieben standen in dem Buch,
das Hilkia, der Priester, fand im Hause des HErrn. \bibverse{25}
Seinesgleichen war vor ihm kein König gewesen, der so von ganzem Herzen,
von ganzer Seele, von allen Kräften sich zum HErrn bekehrte nach allem
Gesetz Moses; und nach ihm kam seinesgleichen nicht auf. \footnote{\textbf{23:25}
  2Kö 18,5} \bibverse{26} Doch kehrte sich der HErr nicht von dem Grimm
seines großen Zorns, mit dem er über Juda erzürnt war um all der
Reizungen willen, durch die ihn Manasse gereizt hatte. \footnote{\textbf{23:26}
  2Kö 21,11-16} \bibverse{27} Und der HErr sprach: Ich will Juda auch
von meinem Angesicht tun, wie ich Israel weggetan habe, und will diese
Stadt verwerfen, die ich erwählt hatte, Jerusalem, und das Haus, davon
ich gesagt habe: Mein Name soll daselbst sein. \footnote{\textbf{23:27}
  2Kö 17,18; 1Kö 8,29} \bibverse{28} Was aber mehr von Josia zu sagen
ist und alles, was er getan hat, siehe, das ist geschrieben in der
Chronik der Könige Judas. \bibverse{29} Zu seiner Zeit zog Pharao Necho,
der König in Ägypten, herauf wider den König von Assyrien an das Wasser
Euphrat. Aber der König Josia zog ihm entgegen und starb zu Megiddo, da
er ihn gesehen hatte. \bibverse{30} Und seine Knechte führten ihn tot
von Megiddo und brachten ihn gen Jerusalem und begruben ihn in seinem
Grabe. Und das Volk im Lande nahm Joahas, den Sohn Josias, und salbten
ihn und machten ihn zum König an seines Vaters Statt. \bibverse{31}
Dreiundzwanzig Jahre war Joahas alt, da er König ward, und regierte drei
Monate zu Jerusalem. Seine Mutter hieß Hamutal, eine Tochter Jeremias
von Libna. \bibverse{32} Und er tat, was dem HErrn übel gefiel, wie
seine Väter getan hatten. \bibverse{33} Aber Pharao Necho legte ihn ins
Gefängnis zu Ribla im Lande Hamath, dass er nicht regieren sollte in
Jerusalem, und legte eine Schatzung aufs Land: hundert Zentner Silber
und einen Zentner Gold. \footnote{\textbf{23:33} Hes 19,4} \bibverse{34}
Und Pharao Necho machte zum König Eljakim, den Sohn Josias, anstatt
seines Vaters Josia und wandte seinen Namen in Jojakim. Aber Joahas nahm
er und brachte ihn nach Ägypten; daselbst starb er. \bibverse{35} Und
Jojakim gab das Silber und das Gold Pharao. Doch schätzte er das Land,
dass er solch Silber gäbe nach Befehl Pharaos; einen jeglichen nach
seinem Vermögen schätzte er am Silber und Gold unter dem Volk im Lande,
dass er es dem Pharao Necho gäbe. \bibverse{36} Fünfundzwanzig Jahre alt
war Jojakim, da er König ward, und regierte elf Jahre zu Jerusalem.
Seine Mutter hieß Sebuda, eine Tochter Pedajas von Ruma. \bibverse{37}
Und er tat, was dem HErrn übel gefiel, wie seine Väter getan hatten. \#
24 \bibverse{1} Zu seiner Zeit zog herauf Nebukadnezar, der König zu
Babel, und Jojakim war ihm untertänig drei Jahre; und er wandte sich und
ward abtrünnig von ihm. \bibverse{2} Und der HErr ließ auf ihn
Kriegsknechte kommen aus Chaldäa, aus Syrien, aus Moab und aus den
Kindern Ammon und ließ sie nach Juda kommen, dass sie es verderbten,
nach dem Wort des HErrn, das er geredet hatte durch seine Knechte, die
Propheten. \bibverse{3} Es geschah aber Juda also nach dem Wort des
HErrn, dass er sie von seinem Angesicht täte um der Sünden willen
Manasses, die er getan hatte; \footnote{\textbf{24:3} 2Kö 21,10-16; 2Kö
  23,26-27} \bibverse{4} auch um des unschuldigen Blutes willen, das er
vergoss und machte Jerusalem voll mit unschuldigem Blut, wollte der HErr
nicht vergeben. \bibverse{5} Was aber mehr zu sagen ist von Jojakim und
alles, was er getan hat, siehe, das ist geschrieben in der Chronik der
Könige Judas. \bibverse{6} Und Jojakim entschlief mit seinen Vätern. Und
sein Sohn Jojachin ward König an seiner Statt. \bibverse{7} Und der
König in Ägypten zog nicht mehr aus seinem Lande; denn der König zu
Babel hatte ihm genommen alles, was dem König in Ägypten gehörte vom
Bach Ägyptens an bis an das Wasser Euphrat. \bibverse{8} Achtzehn Jahre
alt war Jojachin, da er König ward, und regierte drei Monate zu
Jerusalem. Seine Mutter hieß Nehusta, eine Tochter Elnathans von
Jerusalem. \bibverse{9} Und er tat, was dem HErrn übel gefiel, wie sein
Vater getan hatte. \bibverse{10} Zu der Zeit zogen herauf die Knechte
Nebukadnezars, des Königs zu Babel, gen Jerusalem und kamen an die Stadt
mit Bollwerk. \bibverse{11} Und Nebukadnezar kam zur Stadt, da seine
Knechte sie belagerten. \bibverse{12} Aber Jojachin, der König Judas,
ging heraus zum König von Babel mit seiner Mutter, mit seinen Knechten,
mit seinen Obersten und Kämmerern; und der König von Babel nahm ihn
gefangen im achten Jahr seines Königreichs. \bibverse{13} Und er nahm
von dannen heraus alle Schätze im Hause des HErrn und im Hause des
Königs und zerschlug alle goldenen Gefäße, die Salomo, der König
Israels, gemacht hatte im Tempel des HErrn, wie denn der HErr geredet
hatte. \footnote{\textbf{24:13} 2Kö 20,17} \bibverse{14} Und führte weg
das ganze Jerusalem, alle Obersten, alle Gewaltigen, 10.000 Gefangene,
und alle Zimmerleute und alle Schmiede und ließ nichts übrig denn
geringes Volk des Landes. \bibverse{15} Und er führte weg Jojachin gen
Babel, die Mutter des Königs, die Weiber des Königs und seine Kämmerer;
dazu die Mächtigen im Lande führte er auch gefangen von Jerusalem gen
Babel, \bibverse{16} und was der besten Leute waren, 7000, und die
Zimmerleute und Schmiede, 1000, alles starke Kriegsmänner; und der König
von Babel brachte sie gen Babel. \bibverse{17} Und der König von Babel
machte Matthanja, Jojachins Oheim, zum König an seiner Statt und
wandelte seinen Namen in Zedekia. \bibverse{18} Einundzwanzig Jahre alt
war Zedekia, da er König ward, und regierte elf Jahre zu Jerusalem.
Seine Mutter hieß Hamutal, eine Tochter Jeremias von Libna. \footnote{\textbf{24:18}
  Jer 52,1-3} \bibverse{19} Und er tat, was dem HErrn übel gefiel, wie
Jojakim getan hatte. \bibverse{20} Denn es geschah also mit Jerusalem
und Juda aus dem Zorn des HErrn, bis dass er sie von seinem Angesicht
würfe. Und Zedekia ward abtrünnig vom König zu Babel. \# 25 \bibverse{1}
Und es begab sich im neunten Jahr seines Königreichs, am zehnten Tage
des zehnten Monats, kam Nebukadnezar, der König zu Babel, mit aller
seiner Macht wider Jerusalem; und sie lagerten sich dawider und bauten
Bollwerke darum her. \bibverse{2} Also ward die Stadt belagert bis ins
elfte Jahr des Königs Zedekia. \bibverse{3} Aber am neunten Tage des
(vierten) Monats ward der Hunger stark in der Stadt, dass das Volk des
Landes nichts zu essen hatte. \bibverse{4} Da brach man in die Stadt;
und alle Kriegsmänner flohen bei der Nacht auf dem Wege durch das Tor
zwischen den zwei Mauern, der zu des Königs Garten geht. Aber die
Chaldäer lagen um die Stadt. Und er floh des Weges zum blachen Felde.
\bibverse{5} Aber die Macht der Chaldäer jagte dem König nach, und sie
ergriffen ihn im blachen Felde zu Jericho, und alle Kriegsleute, die bei
ihm waren, wurden von ihm zerstreut. \bibverse{6} Sie aber griffen den
König und führten ihn hinauf zum König von Babel gen Ribla; und sie
sprachen ein Urteil über ihn. \bibverse{7} Und sie schlachteten die
Kinder Zedekias vor seinen Augen und blendeten Zedekia die Augen und
banden ihn mit Ketten und führten ihn gen Babel. \bibverse{8} Am
siebenten Tage des fünften Monats, das ist das neunzehnte Jahr
Nebukadnezars, des Königs zu Babel, kam Nebusaradan, der Hauptmann der
Trabanten, des Königs zu Babel Knecht, gen Jerusalem \bibverse{9} und
verbrannte das Haus des HErrn und das Haus des Königs und alle Häuser zu
Jerusalem; alle großen Häuser verbrannte er mit Feuer. \bibverse{10} Und
die ganze Macht der Chaldäer, die mit dem Hauptmann war, zerbrachen die
Mauern um Jerusalem her. \bibverse{11} Das andere Volk aber, das übrig
war in der Stadt, und die zum König von Babel fielen, und den anderen
Haufen führte Nebusaradan, der Hauptmann, weg. \bibverse{12} Und von den
Geringsten im Lande ließ der Hauptmann Weingärtner und Ackerleute.
\bibverse{13} Aber die ehernen Säulen am Hause des HErrn und die
Gestühle und das eherne Meer, das am Hause des HErrn war, zerbrachen die
Chaldäer, und führten das Erz gen Babel. \^{}\^{} \bibverse{14} Und die
Töpfe, Schaufeln, Messer, Löffel und alle ehernen Gefäße, womit man
diente, nahmen sie weg. \bibverse{15} Dazu nahm der Hauptmann die
Pfannen und Becken, was golden und silbern war, \bibverse{16} die zwei
Säulen, das Meer und das Gestühle, die Salomo gemacht hatte zum Hause
des HErrn. Es war nicht zu wägen das Erz aller dieser Gefäße.
\bibverse{17} Achtzehn Ellen hoch war eine Säule, und ihr Knauf darauf
war auch ehern und drei Ellen hoch, und das Gitterwerk und die
Granatäpfel an dem Knauf umher war alles ehern. Auf diese Weise war auch
die andere Säule mit dem Gitterwerk. \bibverse{18} Und der Hauptmann
nahm den Obersten Priester Seraja und den Priester Zephanja, den
nächsten nach ihm, und die drei Türhüter \bibverse{19} und einen
Kämmerer aus der Stadt, der gesetzt war über die Kriegsmänner, und fünf
Männer, die stets vor dem König waren, die in der Stadt gefunden wurden,
und den Schreiber des Feldhauptmanns, der das Volk im Lande zum Heere
aufbot, und 60 Mann vom Volk auf dem Lande, die in der Stadt gefunden
wurden; \bibverse{20} diese nahm Nebusaradan, der Hauptmann, und brachte
sie zum König von Babel gen Ribla. \bibverse{21} Und der König von Babel
schlug sie tot zu Ribla im Lande Hamath. Also ward Juda weggeführt aus
seinem Lande. \bibverse{22} Aber über das übrige Volk im Lande Juda, das
Nebukadnezar, der König von Babel, übrigließ, setzte er Gedalja, den
Sohn Ahikams, des Sohnes Saphans. \bibverse{23} Da nun alle Hauptleute
des Kriegsvolks und die Männer hörten, dass der König von Babel Gedalja
eingesetzt hatte, kamen sie zu Gedalja gen Mizpa, nämlich Ismael, der
Sohn Nethanjas, und Johanan, der Sohn Kareahs, und Seraja, der Sohn
Thanhumeths, der Netophathiter, und Jaasanja, der Sohn eines
Maachathiters, samt ihren Männern. \bibverse{24} Und Gedalja schwur
ihnen und ihren Männern und sprach zu ihnen: Fürchtet euch nicht,
untertan zu sein den Chaldäern; bleibet im Lande und seid untertänig dem
König von Babel, so wird's euch wohl gehen! \bibverse{25} Aber im
siebenten Monat kam Ismael, der Sohn Nethanjas, des Sohnes Elisamas, vom
königlichen Geschlecht, und zehn Männer mit ihm, und sie schlugen
Gedalja tot, dazu die Juden und Chaldäer, die bei ihm waren zu Mizpa.
\bibverse{26} Da machte sich auf alles Volk, klein und groß, und die
Obersten des Kriegsvolks und kamen nach Ägypten; denn sie fürchteten
sich vor den Chaldäern. \bibverse{27} Aber im siebenunddreißigsten Jahr,
nachdem Jojachin, der König Judas, weggeführt war, am
siebenundzwanzigsten Tage des zwölften Monats, hob Evil-Merodach, der
König zu Babel, im ersten Jahr seines Königreichs das Haupt Jojachins,
des Königs Judas, aus dem Kerker hervor \bibverse{28} und redete
freundlich mit ihm und setzte seinen Stuhl über die Stühle der Könige,
die bei ihm waren zu Babel, \bibverse{29} und wandelte die Kleider
seines Gefängnisses, und er aß allewege vor ihm sein Leben lang;
\bibverse{30} und es ward ihm sein Teil bestimmt, das man ihm allewege
gab vom König, auf einen jeglichen Tag sein ganzes Leben lang.
