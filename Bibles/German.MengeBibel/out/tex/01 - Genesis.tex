\hypertarget{das-erste-buch-mose}{%
\section{DAS ERSTE BUCH MOSE}\label{das-erste-buch-mose}}

\emph{(genannt Genesis, d.h. das Buch von der Schöpfung und dem Anfang)}

\hypertarget{i.-die-urgeschichte-kap.-1-11}{%
\subsection{I. Die Urgeschichte (Kap.
1-11)}\label{i.-die-urgeschichte-kap.-1-11}}

\hypertarget{die-schuxf6pfung-der-welt-in-sechs-tagewerken}{%
\subsubsection{1. Die Schöpfung der Welt in sechs
Tagewerken}\label{die-schuxf6pfung-der-welt-in-sechs-tagewerken}}

\hypertarget{a-erstes-tagewerk-die-urschuxf6pfung-und-die-erschaffung-des-lichts}{%
\paragraph{a) Erstes Tagewerk: Die Urschöpfung und die Erschaffung des
Lichts}\label{a-erstes-tagewerk-die-urschuxf6pfung-und-die-erschaffung-des-lichts}}

\hypertarget{section}{%
\section{1}\label{section}}

\bibleverse{1} Im Anfang schuf Gott den Himmel und die Erde;
\bibleverse{2} die Erde war aber eine Wüstenei und Öde, und Finsternis
lag über der weiten Flut\textless sup title=``=~dem
Urmeer''\textgreater✲, und der Geist Gottes schwebte (brütend) über der
Wasserfläche. \bibleverse{3} Da sprach Gott: »Es werde Licht!«, und es
ward Licht. \bibleverse{4} Und Gott sah, daß das Licht gut war; da
schied Gott das Licht von der Finsternis \bibleverse{5} und nannte das
Licht »Tag«, der Finsternis aber gab er den Namen »Nacht«. Und es wurde
Abend und wurde Morgen: erster Tag.

\hypertarget{b-zweites-tagewerk-erschaffung-des-himmelsgewuxf6lbes}{%
\paragraph{b) Zweites Tagewerk: Erschaffung des
Himmelsgewölbes}\label{b-zweites-tagewerk-erschaffung-des-himmelsgewuxf6lbes}}

\bibleverse{6} Dann sprach Gott: »Es entstehe ein festes Gewölbe
inmitten der Wasser und bilde eine Scheidewand zwischen den
beiderseitigen Wassern!« Und es geschah so. \bibleverse{7} So machte
Gott das feste Gewölbe und schied dadurch die Wasser unterhalb des
Gewölbes von den Wassern oberhalb des Gewölbes. \bibleverse{8} Und Gott
nannte das feste Gewölbe »Himmel«. Und es wurde Abend und wurde Morgen:
zweiter Tag.

\hypertarget{c-drittes-tagewerk-scheidung-von-land-und-meer-und-bekleidung-des-festlandes-mit-pflanzen}{%
\paragraph{c) Drittes Tagewerk: Scheidung von Land und Meer und
Bekleidung des Festlandes mit
Pflanzen}\label{c-drittes-tagewerk-scheidung-von-land-und-meer-und-bekleidung-des-festlandes-mit-pflanzen}}

\bibleverse{9} Dann sprach Gott: »Es sammle sich das Wasser unterhalb
des Himmels an einen besonderen Ort, damit das Trockene\textless sup
title=``=~das feste Land''\textgreater✲ sichtbar wird!« Und es geschah
so. \bibleverse{10} Und Gott nannte das Trockene »Erde«\textless sup
title=``oder: »Land«''\textgreater✲, dem Wasser aber, das sich gesammelt
hatte, gab er den Namen »Meer«\textless sup title=``d.h.
Weltmeer''\textgreater✲. Und Gott sah, daß es gut war.~--
\bibleverse{11} Dann sprach Gott: »Die Erde lasse junges Grün sprossen,
samentragende Pflanzen und Bäume, die je nach ihrer Art Früchte mit
Samen darin auf der Erde tragen!« Und es geschah so: \bibleverse{12} die
Erde ließ junges Grün hervorgehen, Kräuter, die je nach ihrer Art Samen
trugen, und Bäume, die Früchte mit Samen darin je nach ihrer Art trugen.
Und Gott sah, daß es gut war. \bibleverse{13} Und es wurde Abend und
wurde Morgen: dritter Tag.

\hypertarget{d-viertes-tagewerk-erschaffung-der-gestirne}{%
\paragraph{d) Viertes Tagewerk: Erschaffung der
Gestirne}\label{d-viertes-tagewerk-erschaffung-der-gestirne}}

\bibleverse{14} Dann sprach Gott: »Es sollen Lichter\textless sup
title=``oder: Leuchten''\textgreater✲ am Himmelsgewölbe entstehen, um
Tag und Nacht voneinander zu scheiden; die sollen Merkzeichen sein und
zur (Bestimmung von) Festzeiten sowie zur (Zählung von) Tagen und Jahren
dienen; \bibleverse{15} und sie sollen Lichter\textless sup
title=``oder: Leuchten''\textgreater✲ am Himmelsgewölbe sein, um Licht
über die Erde zu verbreiten!« Und es geschah so. \bibleverse{16} Da
machte Gott die beiden großen Lichter: das größere Licht zur Herrschaft
über den Tag und das kleinere Licht zur Herrschaft über die Nacht, dazu
auch die Sterne. \bibleverse{17} Gott setzte sie dann an das
Himmelsgewölbe, damit sie Licht über die Erde verbreiteten
\bibleverse{18} und am Tage und in der Nacht die Herrschaft führten und
das Licht von der Finsternis schieden. Und Gott sah, daß es gut war.
\bibleverse{19} Und es wurde Abend und wurde Morgen: vierter Tag.

\hypertarget{e-fuxfcnftes-tagewerk-erschaffung-der-wassertiere-und-der-vuxf6gel}{%
\paragraph{e) Fünftes Tagewerk: Erschaffung der Wassertiere und der
Vögel}\label{e-fuxfcnftes-tagewerk-erschaffung-der-wassertiere-und-der-vuxf6gel}}

\bibleverse{20} Dann sprach Gott: »Es wimmle das Wasser von einem
Gewimmel lebender Wesen, und Vögel sollen über der Erde am
Himmelsgewölbe hin fliegen!« \bibleverse{21} Da schuf Gott die großen
Seetiere und alle Arten der kleinen Lebewesen, die da sich regen, von
denen die Gewässer wimmeln, dazu alle Arten der beschwingten Vögel. Und
Gott sah, daß es gut war. \bibleverse{22} Da segnete Gott sie mit den
Worten: »Seid fruchtbar und mehret euch und erfüllet das Wasser in den
Meeren, und auch die Vögel sollen sich auf der Erde mehren!«
\bibleverse{23} Und es wurde Abend und wurde Morgen: fünfter Tag.

\hypertarget{f-sechstes-tagewerk-erschaffung-der-landtiere-und-des-menschen}{%
\paragraph{f) Sechstes Tagewerk: Erschaffung der Landtiere und des
Menschen}\label{f-sechstes-tagewerk-erschaffung-der-landtiere-und-des-menschen}}

\bibleverse{24} Dann sprach Gott: »Die Erde bringe alle Arten lebender
Wesen hervor, Vieh, Kriechgetier\textless sup title=``oder:
Gewürm''\textgreater✲ und wilde Landtiere, jedes nach seiner Art!« Und
es geschah so. \bibleverse{25} Da machte Gott alle Arten der wilden
Landtiere und alle Arten des Viehs und alles Getier, das auf dem
Erdboden kriecht, jedes nach seiner Art. Und Gott sah, daß es gut
war.~-- \bibleverse{26} Dann sprach Gott: »Laßt uns Menschen machen nach
unserm Bilde, uns ähnlich, die da herrschen sollen über die Fische im
Meer und über die Vögel des Himmels, über das (zahme) Vieh und über alle
(wilden) Landtiere und über alles Gewürm, das auf dem Erdboden kriecht!«
\bibleverse{27} Da schuf Gott den Menschen nach seinem Bilde: nach dem
Bilde Gottes schuf er ihn; als Mann und Weib schuf er sie.
\bibleverse{28} Gott segnete sie dann mit den Worten: »Seid fruchtbar
und mehrt euch, füllt die Erde an und macht sie euch untertan und
herrscht über die Fische im Meer und über die Vögel des Himmels und über
alle Lebewesen, die auf der Erde sich regen!« \bibleverse{29} Dann fuhr
Gott fort: »Hiermit übergebe ich euch alle samentragenden Pflanzen auf
der ganzen Erde und alle Bäume mit samentragenden Früchten: die sollen
euch zur Nahrung dienen! \bibleverse{30} Aber allen Tieren der Erde und
allen Vögeln des Himmels und allem, was auf der Erde kriecht, was
Lebensodem in sich hat, weise ich alles grüne Kraut der Pflanzen zur
Nahrung an.« Und es geschah so. \bibleverse{31} Und Gott sah alles an,
was er geschaffen hatte, und siehe: es war sehr gut. Und es wurde Abend
und wurde Morgen: der sechste Tag.

\hypertarget{g-der-siebte-tag-als-ruhetag-sabbat-von-gott-gesegnet-und-geheiligt-abschluuxdf}{%
\paragraph{g) Der siebte Tag als Ruhetag (Sabbat) von Gott gesegnet und
geheiligt;
Abschluß}\label{g-der-siebte-tag-als-ruhetag-sabbat-von-gott-gesegnet-und-geheiligt-abschluuxdf}}

\hypertarget{section-1}{%
\section{2}\label{section-1}}

\bibleverse{1} So waren der Himmel und die Erde mit ihrem ganzen Heer
vollendet. \bibleverse{2} Da brachte Gott am siebten Tage sein Werk, das
er geschaffen hatte, zur Vollendung und ruhte am siebten Tage von aller
seiner Arbeit, die er vollbracht hatte. \bibleverse{3} Und Gott segnete
den siebten Tag und heiligte ihn; denn an ihm hat Gott von seinem ganzen
Schöpfungswerk und seiner Arbeit geruht. \bibleverse{4} a Dies ist die
Entstehungsgeschichte des Himmels und der Erde, als sie geschaffen
wurden.

\hypertarget{anfangszustuxe4nde-auf-der-erde-und-erschaffung-des-mannes-pflanzung-des-gottesgartens-des-paradieses-in-eden-und-erschaffung-des-weibes}{%
\subsubsection{2. Anfangszustände auf der Erde und Erschaffung des
Mannes; Pflanzung des Gottesgartens (=~des Paradieses) in Eden und
Erschaffung des
Weibes}\label{anfangszustuxe4nde-auf-der-erde-und-erschaffung-des-mannes-pflanzung-des-gottesgartens-des-paradieses-in-eden-und-erschaffung-des-weibes}}

b Zur Zeit, als Gott der HERR Erde und Himmel schuf, \bibleverse{5} als
es auf der Erde noch keine Sträucher auf dem Felde gab und noch keine
Pflanzen auf den Fluren gewachsen waren, weil Gott der HERR noch keinen
Regen auf die Erde hatte fallen lassen und auch noch keine Menschen da
waren, um den Ackerboden zu bestellen~-- \bibleverse{6} es stieg aber
ein Wasserdunst von der Erde auf und tränkte die ganze Oberfläche des
Erdbodens --: \bibleverse{7} da bildete Gott der HERR den Menschen aus
Erde vom Ackerboden und blies ihm den Lebensodem in die Nase; so wurde
der Mensch zu einem lebenden Wesen.

\bibleverse{8} Hierauf pflanzte Gott der HERR einen Garten in Eden nach
Osten hin und versetzte dorthin den Menschen, den er gebildet hatte.
\bibleverse{9} Dann ließ Gott der HERR allerlei Bäume aus dem Erdboden
hervorwachsen, die lieblich anzusehen waren und wohlschmeckende Früchte
trugen, dazu auch den Baum des Lebens mitten im Garten und den Baum der
Erkenntnis des Guten und des Bösen\textless sup title=``oder: von Gut
und Böse''\textgreater✲.

\hypertarget{der-strom-im-paradies-und-seine-verzweigungen}{%
\paragraph{Der Strom im Paradies und seine
Verzweigungen}\label{der-strom-im-paradies-und-seine-verzweigungen}}

\bibleverse{10} Es entsprang aber ein Strom in Eden, um den Garten zu
bewässern, und teilte sich von dort aus, und zwar in vier Arme.
\bibleverse{11} Der erste heißt Pison: dieser ist es, der das ganze Land
Hawila umfließt, woselbst sich das Gold findet, \bibleverse{12} und das
Gold dieses Landes ist kostbar\textless sup title=``oder:
gediegen''\textgreater✲; dort kommt auch das Bedolachharz✲ vor und der
Edelstein Soham✲. \bibleverse{13} Der zweite Strom heißt Gihon: dieser
ist es, der das ganze Land Kusch✲ umfließt. \bibleverse{14} Der dritte
Strom heißt Hiddekel✲: dieser ist es, der östlich von Assyrien fließt;
und der vierte Strom ist der Euphrat.

\hypertarget{gottes-weisung-an-adam-besonders-bezuxfcglich-des-baumes-der-erkenntnis}{%
\paragraph{Gottes Weisung an Adam (besonders bezüglich des Baumes der
Erkenntnis)}\label{gottes-weisung-an-adam-besonders-bezuxfcglich-des-baumes-der-erkenntnis}}

\bibleverse{15} Als nun Gott der HERR den Menschen genommen und ihn in
den Garten Eden versetzt hatte, damit er ihn bestelle und behüte,
\bibleverse{16} gab Gott der HERR dem Menschen die Weisung: »Von allen
Bäumen des Gartens darfst du nach Belieben essen; \bibleverse{17} aber
vom Baum der Erkenntnis des Guten und des Bösen -- von dem darfst du
nicht essen; denn sobald du von diesem ißt, mußt du des Todes sterben.«

\hypertarget{erschaffung-des-weibes-und-stiftung-der-ehe}{%
\paragraph{Erschaffung des Weibes und Stiftung der
Ehe}\label{erschaffung-des-weibes-und-stiftung-der-ehe}}

\bibleverse{18} Hierauf sagte Gott der HERR: »Es ist nicht gut für den
Menschen, daß er allein ist: ich will ihm eine Hilfe schaffen, die zu
ihm paßt\textless sup title=``oder: ihm zur Seite stehe''\textgreater✲.«
\bibleverse{19} Da bildete Gott der HERR aus Erde alle Tiere des Feldes
und alle Vögel des Himmels und brachte sie zu dem Menschen, um zu sehen,
wie er sie benennen würde; und wie der Mensch sie alle\textless sup
title=``=~jedes einzelne''\textgreater✲ benennen würde, so sollten sie
heißen. \bibleverse{20} So legte denn der Mensch allem Vieh\textless sup
title=``=~allen zahmen Tieren''\textgreater✲ und den Vögeln des Himmels
und allen wilden Tieren Namen bei; aber für einen Menschen fand er keine
Hilfe\textless sup title=``oder: Gehilfin''\textgreater✲ darunter, die
zu ihm gepaßt hätte\textless sup title=``vgl. V.18''\textgreater✲.

\bibleverse{21} Da ließ Gott der HERR einen tiefen Schlaf auf den
Menschen fallen, so daß er einschlief; dann nahm er eine von seinen
Rippen heraus und verschloß deren Stelle wieder mit Fleisch;
\bibleverse{22} die Rippe aber, die Gott aus dem Menschen genommen
hatte, gestaltete er zu einem Weibe und führte dieses dem Menschen zu.
\bibleverse{23} Da rief der Mensch aus: »Diese endlich ist es: Gebein
von meinem Gebein und Fleisch von meinem Fleisch! Diese soll ›Männin‹
heißen; denn vom Manne ist diese genommen.« \bibleverse{24} Darum
verläßt ein Mann seinen Vater und seine Mutter und hängt seinem Weibe
an, und sie werden ein Fleisch sein. \bibleverse{25} Und sie waren beide
nackt, der Mensch\textless sup title=``oder: Mann''\textgreater✲ und
sein Weib, und doch schämten sie sich nicht (voreinander).

\hypertarget{der-suxfcndenfall-und-seine-folgen}{%
\subsubsection{3. Der Sündenfall und seine
Folgen}\label{der-suxfcndenfall-und-seine-folgen}}

\hypertarget{a-die-versuchung-und-die-suxfcnde}{%
\paragraph{a) Die Versuchung und die
Sünde}\label{a-die-versuchung-und-die-suxfcnde}}

\hypertarget{section-2}{%
\section{3}\label{section-2}}

\bibleverse{1} Nun war die Schlange listiger als alle Tiere des Feldes,
die Gott der HERR geschaffen hatte; die sagte zum Weibe: »Sollte Gott
wirklich gesagt haben: ›Ihr dürft von allen Bäumen des Gartens nicht
essen\textless sup title=``d.h. also: von gar keinem
Baum''\textgreater✲!‹« \bibleverse{2} Da antwortete das Weib der
Schlange: »Von den Früchten der Bäume im Garten dürfen wir essen;
\bibleverse{3} nur von den Früchten des Baumes, der mitten im Garten
steht, hat Gott gesagt: ›Ihr dürft von ihnen nicht essen, ja sie nicht
einmal anrühren, sonst müßt ihr sterben!‹« \bibleverse{4} Da erwiderte
die Schlange dem Weibe: »Ihr werdet sicherlich nicht sterben;
\bibleverse{5} sondern Gott weiß wohl, daß, sobald ihr davon eßt, euch
die Augen aufgehen werden und ihr wie Gott selbst sein werdet, indem ihr
erkennt, was gut und was böse ist.« \bibleverse{6} Da nun das Weib sah,
daß von dem Baume gut zu essen sei und daß er eine Lust für die Augen
und ein begehrenswerter Baum sei, weil man durch ihn klug werden könne,
so nahm sie eine von seinen Früchten und aß und gab auch ihrem Manne,
der bei ihr war, und der aß auch. \bibleverse{7} Da gingen ihnen beiden
die Augen auf, und sie nahmen wahr, daß sie nackt waren; darum hefteten
sie Blätter vom Feigenbaum zusammen und machten sich Schürze daraus.

\hypertarget{b-das-verhuxf6r-der-fluch-und-die-strafurteile}{%
\paragraph{b) Das Verhör, der Fluch und die
Strafurteile}\label{b-das-verhuxf6r-der-fluch-und-die-strafurteile}}

\bibleverse{8} Als sie dann aber die Stimme\textless sup title=``oder:
das Geräusch der Schritte''\textgreater✲ Gottes des HERRN hörten, der in
der Abendkühle im Garten sich erging, versteckten sie sich, der Mann
(Adam) und sein Weib, vor Gott dem HERRN unter den Bäumen des Gartens.
\bibleverse{9} Aber Gott der HERR rief nach dem Mann mit den Worten: »Wo
bist du?« \bibleverse{10} Da antwortete er: »Als ich deine Stimme im
Garten hörte, fürchtete ich mich, weil ich nackt bin; darum habe ich
mich versteckt.« \bibleverse{11} Da fragte Gott: »Wer hat dir gesagt,
daß du nackt bist? Du hast doch nicht etwa von dem Baume gegessen, von
dem zu essen ich dir verboten habe?« \bibleverse{12} Da antwortete Adam:
»Das Weib, das du mir beigesellt hast, die hat mir von dem Baume
gegeben, da habe ich gegessen.« \bibleverse{13} Da sagte Gott der HERR
zu dem Weibe: »Warum hast du das getan?« Das Weib antwortete: »Die
Schlange hat mich verführt; da habe ich gegessen.« \bibleverse{14} Da
sagte Gott der HERR zu der Schlange: »Weil du das getan hast, sollst du
verflucht sein vor\textless sup title=``oder: unter''\textgreater✲ allen
Tieren, zahmen und wilden! Auf dem Bauche sollst du kriechen und Staub
fressen dein Leben lang! \bibleverse{15} Und ich will Feindschaft
setzen\textless sup title=``=~herrschen lassen''\textgreater✲ zwischen
dir und dem Weibe und zwischen deinem Samen\textless sup title=``d.h.
Nachwuchs, Nachkommenschaft''\textgreater✲ und ihrem Samen: er wird dir
nach dem Kopfe treten\textless sup title=``oder: dir den Kopf
zertreten''\textgreater✲, und du wirst ihm nach der Ferse
schnappen\textless sup title=``oder: ihn in die Ferse
stechen''\textgreater✲.« \bibleverse{16} Zum Weibe aber sagte er: »Viele
Mühsal will ich dir bereiten, wenn du Mutter wirst: mit Schmerzen sollst
du Kinder gebären und doch nach deinem Manne Verlangen tragen; er aber
soll dein Herr sein!« \bibleverse{17} Zu dem Manne\textless sup
title=``oder: zu Adam''\textgreater✲ aber sagte er: »Weil du der
Aufforderung deines Weibes nachgekommen bist und von dem Baume gegessen
hast, von dem zu essen ich dir ausdrücklich verboten hatte, so soll der
Ackerboden verflucht sein um deinetwillen: mit Mühsal sollst du dich von
ihm nähren dein Leben lang! \bibleverse{18} Dornen und Gestrüpp soll er
dir wachsen lassen, und du sollst dich vom Gewächs des Feldes nähren!
\bibleverse{19} Im Schweiße deines Angesichts sollst du dein Brot essen,
bis du zum Erdboden zurückkehrst, von dem du genommen bist; denn
Staub\textless sup title=``oder: Erde''\textgreater✲ bist du, und zu
Staub\textless sup title=``oder: Erde''\textgreater✲ mußt du wieder
werden!«

\hypertarget{c-benennung-des-weibes-bekleidung-der-ersten-menschen-ihre-vertreibung-aus-dem-paradies}{%
\paragraph{c) Benennung des Weibes; Bekleidung der ersten Menschen; ihre
Vertreibung aus dem
Paradies}\label{c-benennung-des-weibes-bekleidung-der-ersten-menschen-ihre-vertreibung-aus-dem-paradies}}

\bibleverse{20} Adam gab dann seinem Weibe den Namen Eva✲; denn sie ist
die Stammutter aller Lebenden geworden. \bibleverse{21} Darauf machte
Gott der HERR dem Manne\textless sup title=``oder: Adam''\textgreater✲
und seinem Weibe Röcke von Fellen und bekleidete sie (damit).
\bibleverse{22} Und Gott der HERR sagte: »Der Mensch ist jetzt ja
geworden wie unsereiner, insofern er gut und böse zu unterscheiden weiß.
Nun aber -- daß er nur nicht seine Hand ausstreckt und auch (Früchte)
vom Baume des Lebens nimmt und (sie) ißt und unsterblich wird!«
\bibleverse{23} So stieß ihn denn Gott der HERR aus dem Garten Eden
hinaus, damit er den Erdboden bestelle, von dem er genommen war;
\bibleverse{24} und als er den Menschen hinausgetrieben hatte, ließ er
östlich vom Garten Eden die Cherube sich lagern und die Flamme des
kreisenden✲ Schwertes, damit sie den Zugang zum Baume des Lebens
bewachten.

\hypertarget{adams-suxf6hne-kain-und-abel-der-brudermord-kains-nachkommen-die-kainiten}{%
\subsubsection{4. Adams Söhne Kain und Abel; der Brudermord; Kains
Nachkommen (die
Kainiten)}\label{adams-suxf6hne-kain-und-abel-der-brudermord-kains-nachkommen-die-kainiten}}

\hypertarget{section-3}{%
\section{4}\label{section-3}}

\bibleverse{1} Eva gebar dann dem Adam, ihrem Gatten, einen Sohn
Kain\textless sup title=``d.h. Erwerb, Gewinn''\textgreater✲. Da sagte
sie: »Einen Mann\textless sup title=``=~männlichen Sproß''\textgreater✲
habe ich ins Dasein gerufen\textless sup title=``eig. erworben,
gewonnen''\textgreater✲ mit Hilfe des HERRN!« \bibleverse{2} Hierauf
gebar sie nochmals, nämlich seinen Bruder Abel\textless sup title=``d.h.
Hauch, Vergänglichkeit''\textgreater✲; und Abel wurde ein Hirt von
Kleinvieh, Kain aber ein Ackerbauer. \bibleverse{3} Nun begab es sich
nach Verlauf geraumer Zeit, daß Kain dem HERRN eine Opfergabe von den
Früchten des Ackers darbrachte; \bibleverse{4} und auch Abel opferte von
den Erstgeburten seiner Herde, und zwar von ihren Fettstücken. Da
schaute der HERR (mit Wohlgefallen) auf Abel und seine Opfergabe;
\bibleverse{5} aber Kain und seine Gabe sah er nicht an. Darüber geriet
Kain in heftige Erregung, so daß sein Angesicht sich finster
senkte\textless sup title=``oder: so daß er den Blick zu Boden
schlug''\textgreater✲. \bibleverse{6} Da sagte der HERR zu Kain: »Warum
bist du erregt geworden, und warum hat dein Angesicht sich finster
gesenkt\textless sup title=``oder: schlägst du den Blick zu
Boden''\textgreater✲? \bibleverse{7} Wird nicht, wenn du recht handelst,
dein Opfer angenommen? Lagert\textless sup title=``oder:
lauert''\textgreater✲ nicht, wenn du böse handelst, die Sünde vor der
Tür (als ein Feind, dessen) Verlangen auf dich gerichtet ist, den du
aber bezwingen sollst?« \bibleverse{8} Hierauf sagte Kain zu seinem
Bruder Abel: (»Laß uns aufs Feld gehen!«) Und als sie auf dem Felde
waren, fiel Kain über seinen Bruder Abel her und schlug ihn tot.

\hypertarget{die-verfluchung-und-bestrafung-des-brudermuxf6rders}{%
\paragraph{Die Verfluchung und Bestrafung des
Brudermörders}\label{die-verfluchung-und-bestrafung-des-brudermuxf6rders}}

\bibleverse{9} Da sagte der HERR zu Kain: »Wo ist dein Bruder Abel?« Er
antwortete: »Ich weiß es nicht; bin ich etwa meines Bruders Hüter?«
\bibleverse{10} Gott aber sagte: »Was hast du getan? Ich höre das Blut
deines Bruders zu mir aus dem Erdboden schreien! \bibleverse{11} Und nun
-- verflucht sollst du sein, (hinweggetrieben) vom Ackerboden, der
seinen Mund aufgetan hat, um das von deiner Hand vergossene Blut deines
Bruders in sich aufzunehmen! \bibleverse{12} Wenn du den Acker
bestellst, soll er dir hinfort keinen Ertrag mehr geben: unstet und
flüchtig sollst du auf der Erde sein!« \bibleverse{13} Da sagte Kain zum
HERRN: »Meine Strafe\textless sup title=``oder:
Sündenschuld''\textgreater✲ ist zu groß, als daß ich sie tragen könnte!
\bibleverse{14} Du treibst mich ja heute von dem Ackerland hinweg, und
ich muß mich vor deinen Augen verbergen und werde unstet und flüchtig
auf der Erde sein; so wird denn jeder, der mich antrifft, mich
totschlagen!« \bibleverse{15} Aber der HERR antwortete ihm: »Nicht also!
Jeder, der Kain totschlägt, soll siebenfältiger Rache verfallen!«
Hierauf brachte der HERR an Kain ein Wahrzeichen\textless sup
title=``vgl. Hes 9,4''\textgreater✲ an, damit ihn niemand erschlüge, der
mit ihm zusammenträfe. \bibleverse{16} So ging denn Kain vom Angesicht
des HERRN hinweg und ließ sich im Lande Nod östlich von Eden nieder.

\hypertarget{die-kainiten-und-ihre-weltliche-bildung-das-lamechlied}{%
\paragraph{Die Kainiten und ihre weltliche Bildung; das
Lamechlied}\label{die-kainiten-und-ihre-weltliche-bildung-das-lamechlied}}

\bibleverse{17} Dem Kain gebar hierauf sein Weib einen Sohn, Henoch; und
als Kain dann eine Stadt✲ erbaute, benannte er sie nach seines Sohnes
Namen Henoch. \bibleverse{18} Dem Henoch wurde dann Irad geboren; dieser
wurde der Vater Mehujaels, Mehujael wurde der Vater Methusaels und
Methusael der Vater Lamechs. \bibleverse{19} Lamech aber nahm sich zwei
Frauen, von denen die eine Ada, die andere Zilla hieß. \bibleverse{20}
Ada gebar dann den Jabal; dieser wurde der Stammvater der Zeltbewohner
und Herdenbesitzer. \bibleverse{21} Sein Bruder hieß Jubal; dieser wurde
der Stammvater aller Zither- und Flötenspieler. \bibleverse{22} Auch
Zilla gebar einen Sohn, nämlich Thubalkain, den Hämmerer\textless sup
title=``oder: Schmied''\textgreater✲ von allen schneidenden Geräten aus
Kupfer und Eisen. Die Schwester Thubalkains war Namma. \bibleverse{23}
Lamech aber sagte (einst) zu seinen Frauen: »Ada und Zilla, höret meine
Rede! Ihr Weiber Lamechs, vernehmet meinen Spruch! Einen Mann erschlage
ich, wenn er mich verwundet, und einen Jüngling, wenn er mir eine
Strieme beibringt! \bibleverse{24} Denn wenn Kain siebenfältig gerächt
werden soll, so Lamech siebenundsiebzigfach!«

\hypertarget{die-geburt-seths-die-bessere-menschheitslinie-der-vorflutlichen-zeit-stammbaum-der-sethiten-die-zehn-urvuxe4ter-von-adam-bis-noah}{%
\subsubsection{5. Die Geburt Seths; die bessere Menschheitslinie der
vorflutlichen Zeit: Stammbaum der Sethiten (die zehn Urväter von Adam
bis
Noah)}\label{die-geburt-seths-die-bessere-menschheitslinie-der-vorflutlichen-zeit-stammbaum-der-sethiten-die-zehn-urvuxe4ter-von-adam-bis-noah}}

\bibleverse{25} Dem Adam aber gebar sein Weib (Eva) nochmals einen Sohn,
dem sie den Namen Seth\textless sup title=``d.h. Setzling,
Ersatz''\textgreater✲ gab; »denn«, sagte sie, »Gott hat mir einen andern
Sproß✲ verliehen an Stelle Abels, weil Kain ihn erschlagen hat«.
\bibleverse{26} Auch dem Seth wurde ein Sohn geboren, den er
Enos\textless sup title=``=~Mensch, mit der Nebenbedeutung des
Schwächlichen?''\textgreater✲ nannte. Damals fing man an, den Namen des
HERRN\textless sup title=``vgl. 2,4''\textgreater✲ anzurufen.

\hypertarget{section-4}{%
\section{5}\label{section-4}}

\bibleverse{1} Dies ist die Geschlechtstafel\textless sup title=``=~das
Verzeichnis der Nachkommen''\textgreater✲ Adams: Am Tage, als Gott den
Adam\textless sup title=``=~den Menschen''\textgreater✲ schuf,
gestaltete er ihn nach Gottes Ebenbild; \bibleverse{2} als Mann und Weib
schuf er sie und segnete sie und gab ihnen den Namen »Mensch« damals,
als sie geschaffen wurden. \bibleverse{3} Adam aber war 130~Jahre alt,
als ihm ein Sohn geboren wurde, der ihm als sein Abbild glich und den er
Seth nannte. \bibleverse{4} Nach der Geburt Seths lebte Adam noch
800~Jahre und hatte Söhne und Töchter. \bibleverse{5} Demnach betrug die
ganze Lebenszeit Adams 930~Jahre; dann starb er.~-- \bibleverse{6} Als
Seth 105~Jahre alt war, wurde ihm Enos geboren. \bibleverse{7} Nach der
Geburt des Enos lebte Seth noch 807~Jahre und hatte Söhne und Töchter.
\bibleverse{8} Demnach betrug die ganze Lebenszeit Seths 912~Jahre; dann
starb er.~-- \bibleverse{9} Als Enos 90~Jahre alt war, wurde ihm Kenan
geboren. \bibleverse{10} Nach der Geburt Kenans lebte Enos noch
815~Jahre und hatte Söhne und Töchter. \bibleverse{11} Demnach betrug
die ganze Lebenszeit des Enos 905~Jahre; dann starb er.~--
\bibleverse{12} Als Kenan 70~Jahre alt war, wurde ihm Mahalalel geboren.
\bibleverse{13} Nach der Geburt Mahalalels lebte Kenan noch 840~Jahre
und hatte Söhne und Töchter. \bibleverse{14} Demnach betrug die ganze
Lebenszeit Kenans 910~Jahre; dann starb er.~-- \bibleverse{15} Als
Mahalalel 65~Jahre alt war, wurde ihm Jered geboren. \bibleverse{16}
Nach der Geburt Jereds lebte Mahalalel noch 830~Jahre und hatte Söhne
und Töchter. \bibleverse{17} Demnach betrug die ganze Lebenszeit
Mahalalels 895~Jahre; dann starb er.~-- \bibleverse{18} Als Jered
162~Jahre alt war, wurde ihm Henoch geboren. \bibleverse{19} Nach der
Geburt Henochs lebte Jered noch 800~Jahre und hatte Söhne und Töchter.
\bibleverse{20} Demnach betrug die ganze Lebenszeit Jereds 962~Jahre;
dann starb er.~-- \bibleverse{21} Als Henoch 65~Jahre alt war, wurde ihm
Methusalah geboren. \bibleverse{22} Henoch wandelte mit Gott; er lebte
nach der Geburt Methusalahs noch 300~Jahre und hatte Söhne und Töchter.
\bibleverse{23} Demnach betrug die ganze Lebenszeit Henochs 365~Jahre.
\bibleverse{24} Henoch wandelte mit Gott und war plötzlich nicht mehr
da, denn Gott hatte ihn hinweggenommen\textless sup title=``vgl. Hebr
11,5''\textgreater✲.~-- \bibleverse{25} Als Methusalah 187~Jahre alt
war, wurde ihm Lamech geboren. \bibleverse{26} Nach der Geburt Lamechs
lebte Methusalah noch 782~Jahre und hatte Söhne und Töchter.
\bibleverse{27} Demnach betrug die ganze Lebenszeit Methusalahs
969~Jahre; dann starb er.~-- \bibleverse{28} Als Lamech 182~Jahre alt
war, wurde ihm ein Sohn geboren, \bibleverse{29} den er
Noah\textless sup title=``d.h. Trost, Ruhe''\textgreater✲ nannte;
»denn«, sagte er, »dieser wird uns Trost verschaffen bei unserer Arbeit
und bei der Mühsal, die unsere Hände durch den Acker haben, den der HERR
verflucht hat«. \bibleverse{30} Nach der Geburt Noahs lebte Lamech noch
595~Jahre und hatte Söhne und Töchter. \bibleverse{31} Demnach betrug
die ganze Lebenszeit Lamechs 777~Jahre; dann starb er.~--
\bibleverse{32} Als Noah 500~Jahre alt war, wurden ihm seine Söhne Sem,
Ham und Japheth geboren.

\hypertarget{die-ehen-der-gottessuxf6hne-mit-den-tuxf6chtern-der-menschen}{%
\subsubsection{6. Die Ehen der Gottessöhne mit den Töchtern der
Menschen}\label{die-ehen-der-gottessuxf6hne-mit-den-tuxf6chtern-der-menschen}}

\hypertarget{section-5}{%
\section{6}\label{section-5}}

\bibleverse{1} Als nun die Menschen sich auf der Oberfläche des
Erdbodens zu vermehren begannen und ihnen auch Töchter geboren wurden
\bibleverse{2} und die Gottessöhne die Schönheit der Menschentöchter
sahen, nahmen sie sich von ihnen diejenigen zu Frauen, die ihnen
besonders gefielen. \bibleverse{3} Da sagte der HERR: »Mein Geist soll
nicht für immer im Menschen erniedrigt sein✲, weil er ja Fleisch ist; so
sollen denn seine Tage (fortan) nur noch hundertundzwanzig Jahre
betragen!« \bibleverse{4} Zu jener Zeit waren die Riesen auf der Erde
und auch später noch, solange die Gottessöhne mit den Menschentöchtern
verkehrten und diese ihnen (Kinder) gebaren. Das sind die
Helden\textless sup title=``oder: Recken''\textgreater✲, die in der
Urzeit lebten, die hochberühmten Männer.

\hypertarget{die-sintflut-d.h.-grouxdfe-flut}{%
\subsubsection{7. Die Sintflut (d.h. große
Flut)}\label{die-sintflut-d.h.-grouxdfe-flut}}

\hypertarget{a-zunehmende-verderbnis-der-menschen-ankuxfcndigung-der-sintflut-zur-vernichtung-der-menschheit}{%
\paragraph{a) Zunehmende Verderbnis der Menschen; Ankündigung der
Sintflut zur Vernichtung der
Menschheit}\label{a-zunehmende-verderbnis-der-menschen-ankuxfcndigung-der-sintflut-zur-vernichtung-der-menschheit}}

\bibleverse{5} Als nun der HERR sah, daß die Bosheit der Menschen groß
war auf der Erde und alles Sinnen und Trachten ihres Herzens immerfort
nur böse war, \bibleverse{6} da gereute es ihn, die Menschen auf der
Erde geschaffen zu haben, und er wurde in seinem Herzen tief betrübt.
\bibleverse{7} Darum sagte der HERR: »Ich will die Menschen, die ich
geschaffen habe, vom ganzen Erdboden weg vertilgen, die Menschen wie das
Vieh, das Gewürm wie die Vögel des Himmels; denn ich bereue es, sie
geschaffen zu haben.« \bibleverse{8} Noah aber hatte Gnade beim HERRN
gefunden.

\hypertarget{b-noahs-berufung-und-sein-bau-der-arche-genau-nach-den-weisungen-gottes}{%
\paragraph{b) Noahs Berufung und sein Bau der Arche genau nach den
Weisungen
Gottes}\label{b-noahs-berufung-und-sein-bau-der-arche-genau-nach-den-weisungen-gottes}}

\bibleverse{9} Dies ist die Geschichte Noahs: Noah war ein frommer,
unsträflicher\textless sup title=``vgl. 17,1''\textgreater✲ Mann unter
seinen Zeitgenossen: mit Gott wandelte Noah. \bibleverse{10} Er hatte
drei Söhne: Sem, Ham und Japheth. \bibleverse{11} Die Erde wurde aber
immer verderbter vor Gott und war voll von Gewalttaten. \bibleverse{12}
Als nun Gott die Erde ansah und die völlige Verderbtheit wahrnahm --
denn alles Fleisch\textless sup title=``=~die gesamte Menschen- und
Tierwelt''\textgreater✲ hatte sich in ihrem ganzen Tun auf Erden zum
Bösen gewandt --, \bibleverse{13} da sagte Gott zu Noah: »Das Ende aller
lebenden Geschöpfe ist bei mir beschlossen; denn die Erde ist durch ihre
Schuld voll von Gewalttaten; darum will ich sie mitsamt der Erde
verderben\textless sup title=``oder: vernichten''\textgreater✲.
\bibleverse{14} Baue dir eine Arche aus Tannenholz; mit lauter
Zellen\textless sup title=``oder: Kammern''\textgreater✲ sollst du die
Arche versehen und sie von innen und von außen mit Erdharz verpichen.
\bibleverse{15} Und so sollst du sie bauen: dreihundert Ellen soll die
Länge der Arche betragen, fünfzig Ellen ihre Breite und dreißig Ellen
ihre Höhe. \bibleverse{16} Eine Lichtöffnung sollst du an der Arche
anbringen, und zwar eine Elle hoch sollst du sie\textless sup
title=``d.h. die Lichtöffnung''\textgreater✲ ganz herum hoch oben
herstellen, und den Eingang zur Arche an ihrer Seite anbringen und ein
unteres, ein mittleres und ein oberes Stockwerk in ihr anlegen.
\bibleverse{17} Denn wisse wohl: ich will die große Flut über die Erde
kommen lassen, um alle Geschöpfe, die Lebensodem in sich haben, unter
dem ganzen Himmel zu vertilgen: alles, was auf der Erde lebt, soll
umkommen! \bibleverse{18} Mit dir aber will ich einen Bund schließen: du
sollst in die Arche gehen, du und mit dir deine Söhne und dein Weib und
die Weiber deiner Söhne\textless sup title=``=~deine
Schwiegertöchter''\textgreater✲. \bibleverse{19} Und von allen lebenden
Wesen, von allen Tieren, sollst du je ein Paar in die Arche mit
hineinnehmen, um sie mit dir am Leben zu erhalten: je ein Männliches und
ein Weibliches sollen es sein. \bibleverse{20} Von jeder Art der Vögel
und von jeder Art der Vierfüßler, von jeder Art der Kriechtiere des
Erdbodens -- von diesen allen soll immer ein Paar zu dir in die Arche
hineinkommen, damit sie am Leben erhalten bleiben. \bibleverse{21} Du
selbst aber nimm dir alle Arten von Nahrungsmitteln, die als Speise
genossen werden, und sammle bei dir Vorräte davon, damit sie dir und
ihnen zur Nahrung dienen.« \bibleverse{22} Und Noah tat es; er machte
alles genau so, wie Gott es ihm geboten hatte.

\hypertarget{c-die-sintflut}{%
\paragraph{c) Die Sintflut}\label{c-die-sintflut}}

\hypertarget{aa-auf-befehl-gottes-geht-noah-mit-den-seinen-und-den-tierpaaren-in-die-arche}{%
\subparagraph{aa) Auf Befehl Gottes geht Noah mit den Seinen und den
Tierpaaren in die
Arche}\label{aa-auf-befehl-gottes-geht-noah-mit-den-seinen-und-den-tierpaaren-in-die-arche}}

\hypertarget{section-6}{%
\section{7}\label{section-6}}

\bibleverse{1} Dann sagte der HERR zu Noah: »Gehe du mit deiner ganzen
Familie in die Arche, denn dich habe ich als gerecht vor mir erfunden
unter diesem Geschlecht. \bibleverse{2} Von allen reinen Tieren nimm je
sieben Paare zu dir, immer ein Männchen und sein Weibchen, aber von den
unreinen Tieren nur je zwei Stück, ein Männchen und sein Weibchen;
\bibleverse{3} auch von den Vögeln des Himmels je sieben Paare, Männchen
und Weibchen, damit Nachkommenschaft auf der ganzen Erde am Leben
erhalten bleibt; \bibleverse{4} denn es sind nur noch sieben Tage, dann
will ich es vierzig Tage und vierzig Nächte hindurch auf die Erde regnen
lassen und will den ganzen Bestand an Lebewesen, die ich geschaffen
habe, vom ganzen Erdboden vertilgen.« \bibleverse{5} Da tat Noah alles
genau so, wie der HERR es ihm geboten hatte. \bibleverse{6} Noah war
aber sechshundert Jahre alt, als die Sintflut über die Erde kam.

\bibleverse{7} Da ging Noah und mit ihm seine Söhne, sein Weib und seine
Schwiegertöchter in die Arche hinein vor den Gewässern der Sintflut.
\bibleverse{8} Von den reinen und von den unreinen Vierfüßlern sowie von
den Vögeln und von allem, was auf dem Erdboden kriecht, \bibleverse{9}
kamen immer zwei, ein Männchen und ein Weibchen, zu Noah in die Arche
hinein, wie Gott ihm geboten hatte. \bibleverse{10} Und nach Ablauf der
sieben Tage, da kamen die Gewässer der Sintflut über die Erde.

\hypertarget{bb-der-eintritt-das-steigen-der-huxf6chststand-und-die-vernichtende-wirkung-der-flut}{%
\subparagraph{bb) Der Eintritt, das Steigen, der Höchststand und die
vernichtende Wirkung der
Flut}\label{bb-der-eintritt-das-steigen-der-huxf6chststand-und-die-vernichtende-wirkung-der-flut}}

\bibleverse{11} Es war im sechshundertsten Lebensjahre Noahs, am
siebzehnten Tage des zweiten Monats: an diesem Tage brachen alle
Quellen\textless sup title=``oder: Brunnen''\textgreater✲ der großen
Tiefe✲ auf, und die Fenster des Himmels öffneten sich, \bibleverse{12}
und der Regen strömte vierzig Tage und vierzig Nächte hindurch auf die
Erde. \bibleverse{13} An eben diesem Tage ging Noah mit seinen Söhnen
Sem, Ham und Japheth und mit seinem Weibe und seinen drei
Schwiegertöchtern in die Arche hinein, \bibleverse{14} sie und alle
Arten der wilden Tiere und alle Arten des Viehs\textless sup
title=``=~der Haustiere''\textgreater✲ und alle Arten des Gewürms, das
auf der Erde kriecht, auch alle Arten der Vögel, alles, was Flügel hatte
und beschwingt war\textless sup title=``=~alles
Federvieh''\textgreater✲; \bibleverse{15} die kamen zu Noah in die Arche
hinein, je ein Paar von allen Geschöpfen, die Lebensodem in sich hatten;
\bibleverse{16} und die da hineinkamen, waren immer ein Männchen und ein
Weibchen von allem Fleische\textless sup title=``=~allen
Geschöpfen''\textgreater✲, wie Gott ihm geboten hatte. Hierauf schloß
der HERR hinter ihm zu.

\bibleverse{17} Da kam die Sintflut vierzig Tage lang über die Erde, und
das Wasser stieg und hob die Arche empor, so daß sie hoch über der Erde
schwamm. \bibleverse{18} Und das Wasser nahm gewaltig zu und stieg hoch
über der Erde, so daß die Arche auf der weiten Flut dahinfuhr.
\bibleverse{19} Und das Wasser stieg immer noch höher über der Erde, so
daß alle höchsten Berge, die unter dem ganzen Himmel sind, überflutet
wurden. \bibleverse{20} Fünfzehn Ellen hoch ging das Wasser über sie
hin, so daß die Berge überflutet wurden. \bibleverse{21} Damals kamen
alle Geschöpfe um, die auf der Erde sich regten: was an Vögeln, an Vieh
und an wilden Tieren da war, sowie alles Gewürm, von dem die Erde
wimmelte, und auch alle Menschen: \bibleverse{22} alles, in dessen Nase
ein Hauch von Lebensodem war, das starb, alles, soweit es auf dem
Trockenen lebte. \bibleverse{23} So vertilgte Gott alle Geschöpfe, die
auf dem ganzen Erdboden waren, vom Menschen bis zum Vieh, bis zum Gewürm
und bis zu den Vögeln des Himmels: sie wurden alle von der Erde
vertilgt; nur Noah blieb übrig und was sich bei ihm in der Arche befand.
\bibleverse{24} Das Wasser aber stieg unaufhörlich über der Erde
hundertundfünfzig Tage lang.

\hypertarget{d-das-verrinnen-und-ende-der-sintflut-noahs-auszug-aus-der-arche-und-sein-opfer-gottes-verheiuxdfung}{%
\paragraph{d) Das Verrinnen und Ende der Sintflut; Noahs Auszug aus der
Arche und sein Opfer; Gottes
Verheißung}\label{d-das-verrinnen-und-ende-der-sintflut-noahs-auszug-aus-der-arche-und-sein-opfer-gottes-verheiuxdfung}}

\hypertarget{section-7}{%
\section{8}\label{section-7}}

\bibleverse{1} Da dachte Gott an Noah und an alle wilden Tiere und an
all das Vieh, das bei ihm in der Arche war; und Gott ließ einen Wind
über die Erde wehen, so daß die Wasser sanken; \bibleverse{2} die
Quellen der Tiefe\textless sup title=``=~die Brunnen der Urflut; vgl.
7,11''\textgreater✲ und die Fenster des Himmels schlossen sich, und dem
Regen vom Himmel her wurde Einhalt getan. \bibleverse{3} Da verlief sich
das Wasser allmählich von der Erde und begann nach Ablauf der
hundertundfünfzig Tage zu fallen; \bibleverse{4} und am siebzehnten Tage
des siebten Monats saß die Arche auf einem der Berge von Ararat✲ fest.
\bibleverse{5} Das Wasser nahm dann immerfort ab bis zum zehnten Monat:
am ersten Tage des zehnten Monats kamen die Gipfel der Berge zum
Vorschein. \bibleverse{6} Nach Verlauf von vierzig Tagen aber öffnete
Noah das Fenster der Arche, das er angebracht hatte, \bibleverse{7} und
ließ den Raben ausfliegen; der flog hin und her, bis das Wasser auf der
Erde abgetrocknet war. \bibleverse{8} Hierauf ließ er die Taube
ausfliegen, um zu erfahren, ob das Wasser sich auf der Erdoberfläche
verlaufen habe. \bibleverse{9} Da die Taube aber keinen Ort fand, wo
ihre Füße hätten ruhen\textless sup title=``oder: sich
niederlassen''\textgreater✲ können, kehrte sie zu ihm zu der Arche
zurück; denn das Wasser bedeckte noch die Oberfläche der ganzen Erde. Da
streckte er seine Hand hinaus, ergriff sie und nahm sie wieder zu sich
in die Arche. \bibleverse{10} Hierauf wartete er noch weitere sieben
Tage und ließ dann die Taube zum zweitenmal aus der Arche fliegen.
\bibleverse{11} Da kam die Taube um die Abendzeit zu ihm zurück, und
siehe da: sie hatte ein frisches Ölbaumblatt im Schnabel! Daran erkannte
Noah, daß das Wasser auf der Erde sich verlaufen hatte. \bibleverse{12}
Nun wartete er nochmals weitere sieben Tage und ließ die Taube wieder
ausfliegen; doch diesmal kehrte sie nicht wieder zu ihm zurück.
\bibleverse{13} Und im sechshundertundersten Lebensjahre Noahs, am
ersten Tage des ersten Monats, da war das Wasser von der Erde
weggetrocknet. Als jetzt Noah das Dach von der Arche abnahm und Ausschau
hielt, da war der Erdboden abgetrocknet; \bibleverse{14} und am
siebenundzwanzigsten Tage des zweiten Monats war die Erde ganz trocken
geworden.

\bibleverse{15} Da gebot Gott dem Noah: \bibleverse{16} »Verlaß (jetzt)
die Arche, du und mit dir dein Weib und deine Söhne und deine
Schwiegertöchter! \bibleverse{17} Sämtliche Tiere von allen Arten, die
bei dir sind, Vögel, Vieh und alles Gewürm, das auf der Erde kriecht,
laß mit dir hinausgehen, damit sie sich auf der Erde frei bewegen und
fruchtbar seien und sich mehren auf der Erde.« \bibleverse{18} Da ging
Noah mit seinen Söhnen, seinem Weibe und seinen Schwiegertöchtern
hinaus; \bibleverse{19} auch alle vierfüßigen Tiere, alles Gewürm, alle
Vögel, alles, was sich auf der Erde regt, gingen nach ihren Arten aus
der Arche hinaus.

\hypertarget{noahs-dankopfer-und-gottes-verheiuxdfung}{%
\paragraph{Noahs Dankopfer und Gottes
Verheißung}\label{noahs-dankopfer-und-gottes-verheiuxdfung}}

\bibleverse{20} Noah baute dann dem HERRN einen Altar, nahm von allen
reinen Tieren und von allen reinen Vögeln (je ein Stück) und brachte
Brandopfer auf dem Altar dar. \bibleverse{21} Als nun der HERR den
lieblichen Duft roch, sagte er bei sich selbst: »Ich will hinfort den
Erdboden nicht noch einmal um der Menschen willen verfluchen; denn das
Sinnen und Trachten\textless sup title=``vgl. 6,5''\textgreater✲ des
Menschenherzens ist böse von Jugend auf; auch will ich hinfort nicht
noch einmal alles Lebende sterben lassen, wie ich es getan habe.
\bibleverse{22} Hinfort, solange die Erde steht, sollen Säen und Ernten,
Frost und Hitze, Sommer und Winter, Tag und Nacht nicht mehr aufhören!«

\hypertarget{gottes-bund-mit-noah}{%
\subsubsection{8. Gottes Bund mit Noah}\label{gottes-bund-mit-noah}}

\hypertarget{a-erneuerung-des-schuxf6pfungssegens-verbot-des-blutgenusses-und-des-vergieuxdfens-von-menschenblut}{%
\paragraph{a) Erneuerung des Schöpfungssegens; Verbot des Blutgenusses
und des Vergießens von
Menschenblut}\label{a-erneuerung-des-schuxf6pfungssegens-verbot-des-blutgenusses-und-des-vergieuxdfens-von-menschenblut}}

\hypertarget{section-8}{%
\section{9}\label{section-8}}

\bibleverse{1} Dann segnete Gott Noah und seine Söhne mit folgenden
Worten: »Seid fruchtbar und mehret euch und füllet die Erde.
\bibleverse{2} Die Furcht und der Schrecken vor euch soll auf allem
Getier der Erde liegen und auf allen Vögeln des Himmels! Alles, was sich
auf dem Erdboden\textless sup title=``oder: Lande''\textgreater✲ regt,
auch alle Fische des Meeres: in eure Gewalt sind sie gegeben.
\bibleverse{3} Alles, was sich regt und was da lebt, soll euch zur
Nahrung dienen: wie (einstmals) die grünenden Pflanzen, so weise ich
euch (jetzt) alles zu. \bibleverse{4} Nur Fleisch, das noch seine
Seele\textless sup title=``oder: Lebenskraft''\textgreater✲, nämlich
sein Blut, in sich hat, dürft ihr nicht essen. \bibleverse{5} Jedoch
euer eigenes Blut, um wessen Leben es sich auch bei euch handle, will
ich rächen; an jedem Tiere will ich es rächen; und auch an jedem
Menschen, an euch untereinander, will ich das Leben jedes Menschen
rächen: \bibleverse{6} Wer Menschenblut vergießt, dessen Blut soll
wieder durch Menschen vergossen werden; denn nach seinem Bilde hat Gott
den Menschen geschaffen. \bibleverse{7} Ihr aber -- seid fruchtbar und
mehret euch, wimmelt\textless sup title=``oder: seid
regsam''\textgreater✲ auf der Erde und werdet zahlreich auf ihr!«

\hypertarget{b-gottes-bundschlieuxdfung-mit-noah-und-der-gesamten-lebenden-schuxf6pfung-festsetzung-des-bundeszeichens}{%
\paragraph{b) Gottes Bundschließung mit Noah und der gesamten lebenden
Schöpfung; Festsetzung des
Bundeszeichens}\label{b-gottes-bundschlieuxdfung-mit-noah-und-der-gesamten-lebenden-schuxf6pfung-festsetzung-des-bundeszeichens}}

\bibleverse{8} Weiter sagte Gott zu Noah und seinen Söhnen, die bei ihm
waren, folgendes: \bibleverse{9} »Ich will (jetzt) einen Bund mit euch
aufrichten\textless sup title=``oder: schließen''\textgreater✲ und mit
eurer Nachkommenschaft, die nach euch sein wird, \bibleverse{10} auch
mit allen lebenden Wesen, die bei euch sind, mit den Vögeln, den zahmen
und allen wilden Tieren, die bei euch sind, nämlich mit allen denen, die
aus der Arche herausgegangen sind, mit allem Getier der Erde.
\bibleverse{11} Ich schließe also meinen Bund mit euch dahin, daß
hinfort niemals wieder alle lebenden Geschöpfe durch das Wasser einer
Sintflut vertilgt werden sollen und daß niemals wieder eine Sintflut
eintreten soll, um die Erde zu verheeren!« \bibleverse{12} Dann fuhr
Gott fort: »Dies soll das Zeichen des Bundes sein, den ich zwischen mir
und euch und allen lebenden Wesen, die bei euch sind, auf ewige Zeiten
festsetze: \bibleverse{13} meinen Bogen stelle ich in die Wolken; der
soll das Zeichen des Bundes zwischen mir und der Erde sein!
\bibleverse{14} Wenn ich hinfort Gewölk über der Erde sammle und der
Bogen in den Wolken sichtbar wird, \bibleverse{15} dann will ich meines
Bundes gedenken, der zwischen mir und euch und allen lebenden Wesen
jeglicher Fleischesart besteht; und das Wasser soll niemals wieder zu
einer Sintflut werden, um alle lebenden Geschöpfe zu vertilgen.
\bibleverse{16} Nein, wenn der Bogen in den Wolken steht, so will ich
ihn anschauen, um des ewigen Bundes zwischen Gott und allen lebenden
Wesen von jeglicher Fleischesart, die auf der Erde ist, zu gedenken.«
\bibleverse{17} Und Gott schloß mit den Worten an Noah: »Dies ist das
Zeichen des Bundes, den ich zwischen mir und allen lebenden Wesen auf
der Erde aufgerichtet\textless sup title=``oder:
geschlossen''\textgreater✲ habe.«

\hypertarget{noahs-weinbau-und-trunkenheit-das-verhalten-seiner-drei-suxf6hne-besonders-der-frevel-hams-noahs-fluch-und-segen-sein-tod}{%
\subsubsection{9. Noahs Weinbau und Trunkenheit; das Verhalten seiner
drei Söhne (besonders der Frevel Hams); Noahs Fluch und Segen; sein
Tod}\label{noahs-weinbau-und-trunkenheit-das-verhalten-seiner-drei-suxf6hne-besonders-der-frevel-hams-noahs-fluch-und-segen-sein-tod}}

\bibleverse{18} Die Söhne Noahs, die aus der Arche gingen, waren Sem,
Ham und Japheth; Ham aber ist der Vater Kanaans. \bibleverse{19} Diese
drei waren die Söhne Noahs, und von diesen aus ist die ganze Erde
bevölkert worden\textless sup title=``=~stammt die ganze Erdbevölkerung
ab''\textgreater✲.

\bibleverse{20} Noah aber wurde nun ein Landmann und legte auch einen
Weinberg an. \bibleverse{21} Als er dann aber von dem Weine trank, wurde
er trunken und lag entblößt in seinem Zelt. \bibleverse{22} Als nun Ham,
der Vater Kanaans, seinen Vater entblößt hatte daliegen sehen, erzählte
er es seinen beiden Brüdern draußen. \bibleverse{23} Da nahmen Sem und
Japheth das Obergewand (ihres Vaters), legten es beide gemeinsam auf
ihre Schultern, traten rückwärts hinzu und bedeckten ihren entblößten
Vater damit; ihr Gesicht aber war dabei abgewandt, so daß sie die Blöße
ihres Vaters nicht sahen. \bibleverse{24} Als nun Noah von seinem Rausch
erwachte und erfuhr, wie sein jüngster Sohn sich gegen ihn benommen
hatte, \bibleverse{25} rief er aus: »Verflucht sei Kanaan! Der
niedrigste Knecht soll er seinen Brüdern sein!« \bibleverse{26} Dann
fuhr er fort: »Gepriesen sei der HERR, der Gott Sems! Kanaan aber soll
sein Knecht sein! \bibleverse{27} Weiten Raum schaffe Gott dem Japheth,
und er wohne in den Zelten Sems! Kanaan aber soll sein Knecht sein!«

\bibleverse{28} Nach der Sintflut lebte Noah noch dreihundertfünfzig
Jahre; \bibleverse{29} demnach betrug die ganze Lebenszeit Noahs
neunhundertundfünfzig Jahre; dann starb er.

\hypertarget{die-vuxf6lkertafel}{%
\subsubsection{10. Die Völkertafel}\label{die-vuxf6lkertafel}}

\hypertarget{section-9}{%
\section{10}\label{section-9}}

\bibleverse{1} Dies ist der Stammbaum der Noahsöhne, Sem, Ham und
Japheth; Söhne wurden ihnen erst nach der Sintflut geboren.
\bibleverse{2} Die Söhne Japheths\textless sup title=``=~die
Japhethiten''\textgreater✲ waren: Gomer, Magog, Madai, Jawan, Thubal,
Mesech und Thiras. \bibleverse{3} Die Söhne Gomers waren: Askenas,
Riphath und Thogarma. \bibleverse{4} Und die Söhne Jawans: Elisa und
Tharsis, die Kitthiter und die Dodaniter. \bibleverse{5} Von diesen aus
haben sich die Bewohner der Meeresländer der (heidnischen) Völker
abgezweigt. Dies sind die Söhne✲ Japheths nach ihren Ländern, jeder nach
seiner Sprache, nach ihren Geschlechtern\textless sup title=``oder:
Stämmen''\textgreater✲, nach ihren Völkerschaften. \bibleverse{6} Die
Söhne Hams\textless sup title=``=~die Hamiten''\textgreater✲ waren:
Kusch, Mizraim, Put und Kanaan. \bibleverse{7} Und die Söhne Kuschs:
Seba, Hawila, Sabtha, Ragma und Sabthecha; und die Söhne Ragmas: Seban
und Dedan.~-- \bibleverse{8} Kusch war der Vater Nimrods; dieser wurde
der erste Gewalthaber auf der Erde. \bibleverse{9} Er war ein gewaltiger
Jäger vor dem HERRN; darum pflegt man zu sagen: »Ein gewaltiger Jäger
vor dem HERRN wie Nimrod.« \bibleverse{10} Den Anfang seines
Königtums\textless sup title=``oder: Reiches''\textgreater✲ bildeten
Babel, Erech, Akkad und Kalne im Lande Sinear\textless sup title=``d.h.
Babylonien''\textgreater✲. \bibleverse{11} Von diesem Lande zog er nach
Assur✲ und erbaute Ninive, Rehoboth-Ir und Kalah, \bibleverse{12} dazu
Resen zwischen Ninive und Kalah, das ist die große Stadt.~--
\bibleverse{13} Von Mizraim sodann stammen die Luditer, Anamiter,
Lehabiter, Naphthuchiter, \bibleverse{14} Pathrusiter, Kasluchiter und
Kaphthoriter, von denen die Philister ausgegangen sind.~--
\bibleverse{15} Kanaan aber hatte zu Söhnen Sidon, seinen Erstgeborenen,
und Heth, \bibleverse{16} ferner die Jebusiter, Amoriter, Girgasiter,
\bibleverse{17} Hewiter, Arkiter, Siniter, \bibleverse{18} Arwaditer,
Zemariter und Hamathiter. Später haben sich dann die
Geschlechter\textless sup title=``oder: Stämme''\textgreater✲ der
Kanaaniter zerstreut, \bibleverse{19} so daß das Gebiet der Kanaaniter
von Sidon in der Richtung auf Gerar bis Gaza, dann in der Richtung auf
Sodom und Gomorrha, Adma und Zeboim bis Lesa reichte. \bibleverse{20}
Dies sind die Söhne✲ Hams nach ihren Stämmen, ihren Sprachen, ihren
Ländern, ihren Völkerschaften.

\bibleverse{21} Aber auch dem Sem, dem Stammvater aller Söhne Ebers, dem
älteren Bruder Japheths, wurden Söhne geboren. \bibleverse{22} Die Söhne
Sems\textless sup title=``=~die Semiten''\textgreater✲ waren: Elam,
Assur, Arpachsad, Lud und Aram. \bibleverse{23} Und die Söhne Arams
waren: Uz, Hul, Gether und Mas. \bibleverse{24} Arpachsad aber war der
Vater Selahs und Selah der Vater Ebers. \bibleverse{25} Dem Eber aber
wurden zwei Söhne geboren; der eine hieß Peleg\textless sup title=``d.h.
Teilung''\textgreater✲, weil sich die Erde\textless sup title=``oder:
Erdbevölkerung''\textgreater✲ zu seiner Zeit teilte; und sein Bruder
hieß Joktan. \bibleverse{26} Joktan hatte zu Söhnen Almodad, Seleph,
Hazarmaweth, Jerah, \bibleverse{27} Hadoram, Usal, Dikla,
\bibleverse{28} Obal, Abimael, Seba, \bibleverse{29} Ophir, Hawila und
Jobab; diese alle waren Söhne Joktans, \bibleverse{30} und ihre
Wohnsitze erstreckten sich von Mesa in der Richtung auf Sephar bis zum
Ostgebirge. \bibleverse{31} Dies sind die Söhne✲ Sems nach ihren
Geschlechtern\textless sup title=``oder: Stämmen''\textgreater✲, nach
ihren Sprachen, ihren Ländern, ihren Völkerschaften.

\bibleverse{32} Dies sind die Geschlechter\textless sup title=``oder:
Stämme''\textgreater✲ der Söhne✲ Noahs nach ihrer Abstammung, nach ihren
Völkerschaften; und von ihnen aus haben sich die Völker auf der Erde
nach der Sintflut abgezweigt.

\hypertarget{der-turmbau-zu-babel-und-die-sprachverwirrung}{%
\subsubsection{11. Der Turmbau zu Babel und die
Sprachverwirrung}\label{der-turmbau-zu-babel-und-die-sprachverwirrung}}

\hypertarget{section-10}{%
\section{11}\label{section-10}}

\bibleverse{1} Es hatte aber die ganze Erdbevölkerung eine einzige
Sprache und einerlei Worte. \bibleverse{2} Als sie nun nach Osten hin
zogen, fanden sie eine Tiefebene im Lande Sinear✲ und blieben dort
wohnen. \bibleverse{3} Da sagten sie zueinander: »Auf! Wir wollen
Ziegel\textless sup title=``oder: Backsteine''\textgreater✲ streichen
und sie im Feuer hart brennen!« So dienten ihnen denn die Ziegel als
Bausteine, und das Erdharz\textless sup title=``oder: der
Asphalt''\textgreater✲ diente ihnen als Mörtel. \bibleverse{4} Dann
sagten sie: »Auf! Wir wollen uns eine Stadt und einen Turm bauen, dessen
Spitze bis in den Himmel reichen soll, und wollen uns einen
Namen\textless sup title=``oder: ein Denkmal''\textgreater✲ schaffen,
damit wir uns nicht über die ganze Erde hin zerstreuen!« \bibleverse{5}
Da fuhr der HERR herab, um sich die Stadt und den Turm anzusehen, welche
die Menschen erbauten\textless sup title=``oder: erbaut
hatten''\textgreater✲. \bibleverse{6} Da sagte der HERR: »Fürwahr, sie
sind ein einziges Volk und haben alle dieselbe Sprache, und dies ist
erst der Anfang ihres Unternehmens: hinfort wird ihnen nichts mehr
unausführbar sein, was sie sich vornehmen. \bibleverse{7} Auf! wir
wollen hinabfahren und ihre Sprache dort verwirren, so daß keiner mehr
die Sprache des andern versteht!« \bibleverse{8} So zerstreute sie denn
der HERR von dort über die ganze Erde, so daß sie den Bau der Stadt
aufgeben mußten. \bibleverse{9} Daher gab man der Stadt den Namen
Babel\textless sup title=``d.h. Verwirrung''\textgreater✲; denn dort hat
der HERR die Sprache der ganzen Erdbevölkerung verwirrt und sie von dort
über die ganze Erde zerstreut.

\hypertarget{der-stammbaum-sems-bis-zu-tharahs-suxf6hnen-uxfcberleitung-zur-geschichte-der-erzvuxe4ter}{%
\subsubsection{12. Der Stammbaum Sems bis zu Tharahs Söhnen; Überleitung
zur Geschichte der
Erzväter}\label{der-stammbaum-sems-bis-zu-tharahs-suxf6hnen-uxfcberleitung-zur-geschichte-der-erzvuxe4ter}}

\bibleverse{10} Dies ist der Stammbaum\textless sup title=``=~die
Nachkommenschaft oder: Familiengeschichte''\textgreater✲ Sems: Als Sem
100~Jahre alt war, wurde ihm Arpachsad geboren, zwei Jahre nach der
Sintflut. \bibleverse{11} Nach der Geburt Arpachsads aber lebte Sem noch
500~Jahre und hatte Söhne und Töchter.~-- \bibleverse{12} Als Arpachsad
35~Jahre alt war, wurde er der Vater Selahs. \bibleverse{13} Nach der
Geburt Selahs lebte Arpachsad noch 403~Jahre und hatte Söhne und
Töchter.~-- \bibleverse{14} Als Selah 30~Jahre alt war, wurde ihm sein
Sohn Eber geboren. \bibleverse{15} Nach der Geburt Ebers lebte Selah
noch 403~Jahre und hatte Söhne und Töchter.~-- \bibleverse{16} Als Eber
34~Jahre alt war, wurde ihm sein Sohn Peleg geboren. \bibleverse{17}
Nach der Geburt Pelegs lebte Eber noch 430~Jahre und hatte Söhne und
Töchter.~-- \bibleverse{18} Als Peleg 30~Jahre alt war, wurde ihm sein
Sohn Rehu geboren. \bibleverse{19} Nach der Geburt Rehus lebte Peleg
noch 209~Jahre und hatte Söhne und Töchter.~-- \bibleverse{20} Als Rehu
32~Jahre alt war, wurde ihm sein Sohn Serug geboren. \bibleverse{21}
Nach der Geburt Serugs lebte Rehu noch 207~Jahre und hatte Söhne und
Töchter.~-- \bibleverse{22} Als Serug 30~Jahre alt war, wurde er der
Vater Nahors. \bibleverse{23} Nach der Geburt Nahors lebte Serug noch
200~Jahre und hatte Söhne und Töchter.~-- \bibleverse{24} Als Nahor
29~Jahre alt war, wurde er der Vater Tharahs. \bibleverse{25} Nach der
Geburt Tharahs lebte Nahor noch 119~Jahre und hatte Söhne und
Töchter.~-- \bibleverse{26} Als Tharah 70~Jahre alt war, wurden ihm
seine Söhne Abram, Nahor und Haran geboren.

\hypertarget{tharahs-stammbaum-seine-auswanderung-aus-ur-in-chalduxe4a-nach-haran-im-nordwestlichen-mesopotamien}{%
\paragraph{Tharahs Stammbaum; seine Auswanderung aus Ur in Chaldäa nach
Haran im nordwestlichen
Mesopotamien}\label{tharahs-stammbaum-seine-auswanderung-aus-ur-in-chalduxe4a-nach-haran-im-nordwestlichen-mesopotamien}}

\bibleverse{27} Und dies ist der Stammbaum\textless sup title=``=~die
Nachkommenschaft oder: Familiengeschichte''\textgreater✲ Tharahs: Tharah
hatte drei Söhne: Abram, Nahor und Haran; Haran aber war der Vater Lots.
\bibleverse{28} Haran starb dann noch bei Lebzeiten seines Vaters Tharah
in seinem Geburtslande, zu Ur in Chaldäa. \bibleverse{29} Abram und
Nahor aber nahmen sich ebenfalls Frauen: Abrams Frau hieß Sarai, und
Nahors Frau hieß Milka; diese war eine Tochter Harans, des Vaters der
Milka und der Jiska. \bibleverse{30} Sarai aber war unfruchtbar: sie
hatte keine Kinder. \bibleverse{31} Da nahm Tharah seinen Sohn Abram und
seinen Enkel Lot, den Sohn Harans, und seine Schwiegertochter Sarai, die
Frau seines Sohnes Abram, und zog mit ihnen aus Ur in Chaldäa weg, um
sich ins Land Kanaan zu begeben; als sie aber bis Haran gekommen waren,
blieben sie daselbst wohnen. \bibleverse{32} Tharah brachte hierauf sein
Leben auf zweihundertfünf Jahre; dann starb er in Haran.

\hypertarget{ii.-die-geschichte-der-drei-erzvuxe4ter-kap.-12-50}{%
\subsection{II. Die Geschichte der drei Erzväter (Kap.
12-50)}\label{ii.-die-geschichte-der-drei-erzvuxe4ter-kap.-12-50}}

\hypertarget{die-geschichte-abrahams-bzw.-abrams-121-2518}{%
\subsubsection{1. Die Geschichte Abrahams (bzw. Abrams)
(12,1-25,18)}\label{die-geschichte-abrahams-bzw.-abrams-121-2518}}

\hypertarget{a-abrams-berufung-und-seine-einwanderung-in-kanaan-sein-zug-nach-uxe4gypten}{%
\paragraph{a) Abrams Berufung und seine Einwanderung in Kanaan; sein Zug
nach
Ägypten}\label{a-abrams-berufung-und-seine-einwanderung-in-kanaan-sein-zug-nach-uxe4gypten}}

\hypertarget{section-11}{%
\section{12}\label{section-11}}

\bibleverse{1} Der HERR sprach zu Abram: »Verlaß dein Land und deine
Verwandtschaft und deines Vaters Haus (und ziehe) in das Land, das ich
dir zeigen werde; \bibleverse{2} denn ich will ich zu einem großen Volke
machen und will dich segnen und deinen Namen groß✲ machen, und du sollst
ein Segen werden. \bibleverse{3} Ich will die segnen, die dich segnen,
und wer dich verflucht, den will ich verfluchen; und in dir sollen alle
Geschlechter der Erde gesegnet werden.«

\hypertarget{abrams-wegwanderung-aus-haran-und-einwanderung-in-kanaan}{%
\paragraph{Abrams Wegwanderung aus Haran und Einwanderung in
Kanaan}\label{abrams-wegwanderung-aus-haran-und-einwanderung-in-kanaan}}

\bibleverse{4} Da machte sich Abram auf den Weg, wie der HERR ihm
geboten hatte, auch Lot zog mit ihm; Abram aber war fünfundsiebzig Jahre
alt, als er aus Haran aufbrach. \bibleverse{5} Abram nahm also seine
Frau Sarai und Lot, den Sohn seines Bruders (Haran), und alle Habe, die
sie besaßen, und alles Gesinde, das sie in Haran erworben hatten, und so
zogen sie aus, um nach dem Lande Kanaan zu wandern. Als sie nun in
diesem Lande angekommen waren, \bibleverse{6} zog Abram im Lande umher
bis zu der heiligen Stätte von Sichem, bis zur Orakel-Terebinthe; die
Kanaanäer wohnten damals im Lande. \bibleverse{7} Da erschien der HERR
dem Abram und sagte zu ihm: »Deinen Nachkommen will ich dieses Land
geben!« Da baute er dort dem HERRN, der ihm erschienen war, einen Altar.
\bibleverse{8} Hierauf zog er von dort weiter nach dem Berglande östlich
von Bethel und schlug sein Zelt zwischen Bethel im Westen und Ai im
Osten auf; dort baute er dem HERRN einen Altar und rief den Namen des
HERRN an. \bibleverse{9} Dann brach er wieder auf und zog immer weiter
nach dem Südgau zu.

\hypertarget{abrams-und-sarais-erlebnisse-in-uxe4gypten}{%
\paragraph{Abrams und Sarais Erlebnisse in
Ägypten}\label{abrams-und-sarais-erlebnisse-in-uxe4gypten}}

\bibleverse{10} Als dann eine Hungersnot im Lande ausbrach, zog Abram
nach Ägypten hinab, um dort als Fremdling eine Zeitlang zu verbleiben;
denn die Hungersnot lag schwer auf dem Lande. \bibleverse{11} Als er nun
auf seinem Zuge von Ägypten nicht mehr weit entfernt war, sagte er zu
seiner Frau Sarai: »Ich weiß sehr wohl, daß du eine Frau von großer
Schönheit bist. \bibleverse{12} Wenn dich nun die Ägypter sehen und
denken: ›Das ist seine Frau‹, dann werden sie mich erschlagen, während
sie dich am Leben lassen. \bibleverse{13} Sage doch, du seiest meine
Schwester, damit es mir um deinetwillen gut ergehe und ich, soweit es
sich um dich handelt, am Leben bleibe.«

\bibleverse{14} Als nun Abram in Ägypten ankam, sahen die Ägypter, daß
die Frau überaus schön war; \bibleverse{15} und als die Hofleute des
Pharaos sie zu Gesicht bekommen hatten, rühmten sie die Frau dem Pharao
gegenüber; da wurde sie in dessen Palast geholt. \bibleverse{16} Dem
Abram aber bewies sich der Pharao um ihretwillen wohlwollend, so daß er
Kleinvieh und Rinder, Esel, Knechte und Mägde, Eselinnen und Kamele
geschenkt erhielt. \bibleverse{17} Aber der HERR suchte den Pharao und
sein Haus mit schweren Plagen\textless sup title=``oder:
Krankheiten''\textgreater✲ heim wegen Sarais, der Frau Abrams.
\bibleverse{18} Da ließ der Pharao Abram rufen und sagte zu ihm: »Was
hast du mir da angetan! Warum hast du mir nicht mitgeteilt, daß sie
deine Frau ist? \bibleverse{19} Warum hast du sie für deine Schwester
ausgegeben, so daß ich sie mir zur Frau genommen habe? Doch nun -- hier
hast du deine Frau: nimm sie und gehe!« \bibleverse{20} Hierauf entbot
der Pharao seinethalben Leute, die ihn samt seiner Frau und seiner
ganzen Habe (aus dem Lande) geleiten mußten.

\hypertarget{b-abrams-ruxfcckkehr-nach-suxfcdpaluxe4stina-lot-trennt-sich-von-ihm}{%
\paragraph{b) Abrams Rückkehr nach Südpalästina; Lot trennt sich von
ihm}\label{b-abrams-ruxfcckkehr-nach-suxfcdpaluxe4stina-lot-trennt-sich-von-ihm}}

\hypertarget{section-12}{%
\section{13}\label{section-12}}

\bibleverse{1} So zog denn Abram mit seiner Frau und mit all seinem Hab
und Gut aus Ägypten wieder hinauf nach dem Südgau; auch Lot war bei ihm.
\bibleverse{2} Abram war aber sehr reich an Herden, an Silber und Gold;
\bibleverse{3} und er zog weiter von einem Lagerplatz zum andern aus dem
Südgau bis nach Bethel, bis an die Stätte, wo sein Zelt anfangs
gestanden hatte, zwischen Bethel und Ai, \bibleverse{4} zu der Stätte,
wo der Altar stand, den er dort zuvor gebaut hatte; und Abram rief dort
den Namen des HERRN an.

\hypertarget{abrams-friedensliebe-gegenuxfcber-lot-seine-auseinandersetzung-mit-lot}{%
\paragraph{Abrams Friedensliebe gegenüber Lot; seine Auseinandersetzung
mit
Lot}\label{abrams-friedensliebe-gegenuxfcber-lot-seine-auseinandersetzung-mit-lot}}

\bibleverse{5} Aber auch Lot, der mit Abram zog, besaß Kleinvieh, Rinder
und Zelte. \bibleverse{6} So reichte denn das Land nicht aus, daß beide
hätten beisammen bleiben können; denn ihr Hab und Gut war groß geworden;
daher konnten sie nicht beieinander bleiben. \bibleverse{7} So entstand
denn Streit zwischen den Hirten von Abrams Herden und den Hirten von
Lots Vieh; es waren nämlich die Kanaanäer und Pherissiter damals im
Lande ansässig. \bibleverse{8} Da sagte Abram zu Lot: »Laß doch keine
Streitigkeiten zwischen mir und dir und zwischen meinen und deinen
Hirten herrschen: wir sind ja Brüder\textless sup title=``=~nahe
Verwandte''\textgreater✲. \bibleverse{9} Steht dir nicht das ganze Land
zur freien Verfügung? Trenne dich lieber von mir! Willst du nach der
linken Seite, so gehe ich nach rechts, und willst du nach der rechten
Seite, so gehe ich nach links.«

\hypertarget{lots-wegzug-nach-dem-jordantal}{%
\paragraph{Lots Wegzug nach dem
Jordantal}\label{lots-wegzug-nach-dem-jordantal}}

\bibleverse{10} Da hob Lot seine Augen auf und sah, daß die ganze
Gegend\textless sup title=``oder: Aue''\textgreater✲ am Jordan überall
wohlbewässertes Land war -- bevor nämlich der HERR Sodom und Gomorrha
zerstört hatte --, wie der Garten Gottes, wie das Land Ägypten, bis nach
Zoar hin. \bibleverse{11} Da wählte Lot für sich die ganze Gegend am
Jordan\textless sup title=``=~die Jordanaue''\textgreater✲ und zog
ostwärts. So trennten sich beide voneinander: \bibleverse{12} Abram
blieb im Lande Kanaan wohnen, während Lot sich in den Ortschaften der
Jordanaue niederließ und mit seinen Zelten bis nach Sodom zog.
\bibleverse{13} Die Einwohner von Sodom aber waren böse Leute und arge
Sünder vor dem HERRN.

\hypertarget{dem-selbstlosen-abram-wird-abermals-der-besitz-kanaans-verheiuxdfen}{%
\paragraph{Dem selbstlosen Abram wird abermals der Besitz Kanaans
verheißen}\label{dem-selbstlosen-abram-wird-abermals-der-besitz-kanaans-verheiuxdfen}}

\bibleverse{14} Der HERR aber sagte zu Abram, nachdem Lot sich von ihm
getrennt hatte: »Hebe deine Augen auf und schaue von der Stelle, auf der
du stehst, nach Norden und Süden, nach Osten und Westen: \bibleverse{15}
denn das ganze Land, das du siehst, will ich dir und deinen Nachkommen
auf ewige Zeiten geben \bibleverse{16} und will deine Nachkommenschaft
so zahlreich werden lassen wie den Staub der Erde, so daß, wenn jemand
den Staub der Erde zu zählen vermöchte, auch deine Nachkommenschaft
zählbar sein sollte. \bibleverse{17} Wohlan, durchziehe das Land nach
seiner Länge und Breite, denn dir will ich es geben!« \bibleverse{18} Da
zog Abram mit seinen Zelten weiter und nahm (endlich) seinen Wohnsitz
unter\textless sup title=``oder: bei''\textgreater✲ den Terebinthen
Mamres\textless sup title=``vgl. 14,13''\textgreater✲, die bei Hebron
stehen; dort baute er dem HERRN einen Altar.

\hypertarget{c-abrams-sieg-uxfcber-die-vier-kuxf6nige-des-ostens-lots-rettung-der-priesterfuxfcrst-melchisedek}{%
\paragraph{c) Abrams Sieg über die vier Könige des Ostens; Lots Rettung;
der Priesterfürst
Melchisedek}\label{c-abrams-sieg-uxfcber-die-vier-kuxf6nige-des-ostens-lots-rettung-der-priesterfuxfcrst-melchisedek}}

\hypertarget{aa-kedorlaomers-rachezug-nach-dem-jordantal-sein-sieg-im-tal-siddim}{%
\subparagraph{aa) Kedorlaomers Rachezug nach dem Jordantal; sein Sieg im
Tal
Siddim}\label{aa-kedorlaomers-rachezug-nach-dem-jordantal-sein-sieg-im-tal-siddim}}

\hypertarget{section-13}{%
\section{14}\label{section-13}}

\bibleverse{1} Es begab sich dann zur Zeit Amraphels, des Königs von
Sinear, Ariochs, des Königs von Ellasar, Kedorlaomers, des Königs von
Elam, und Thideals, des Königs von Gojim\textless sup title=``d.h. des
Königs der Völker oder: der Heiden?''\textgreater✲: \bibleverse{2} die
fingen Krieg an mit Bera, dem König von Sodom, und mit Birsa, dem König
von Gomorrha, mit Sineab, dem König von Adma, mit Semheber, dem König
von Zebojim, und mit dem König von Bela, das ist Zoar. \bibleverse{3}
Alle diese kamen als Verbündete im Tale von Siddim zusammen, wo jetzt
das Salzmeer liegt. \bibleverse{4} Zwölf Jahre lang waren sie dem
Kedorlaomer untertan✲ gewesen, aber im dreizehnten Jahre waren sie von
ihm abgefallen. \bibleverse{5} Im vierzehnten Jahre kamen dann
Kedorlaomer und die mit ihm verbündeten Könige und schlugen die
Rephaiter bei Astheroth-Karnajim und die Susiter bei Ham und die Emiter
in der Ebene von Kirjathajim \bibleverse{6} und die Horiter auf ihrem
Gebirge Seir bis nach El-Paran, das am Rand der Wüste\textless sup
title=``oder: Steppe''\textgreater✲ liegt. \bibleverse{7} Darauf kehrten
sie um und kamen nach En-Mispat\textless sup title=``d.h.
Gerichtsquelle''\textgreater✲, das ist Kades, und verwüsteten das ganze
Gefilde der Amalekiter sowie auch das Gebiet der Amoriter, die in
Hazazon-Thamar wohnten. \bibleverse{8} Da zogen der König von Sodom und
die Könige von Gomorrha, von Adma, von Zebojim und von Bela -- das ist
Zoar -- aus und stellten sich gegen sie zur Schlacht auf im Siddimtal,
\bibleverse{9} nämlich gegen Kedorlaomer, den König von Elam, und
Thideal, den König von Gojim, und Amraphel, den König von Sinear, und
Arioch, den König von Ellasar: vier Könige gegen die fünf.
\bibleverse{10} Das Siddimtal war aber voll von Gruben mit Erdharz✲. Als
nun der König von Sodom und der von Gomorrha in die Flucht geschlagen
waren, gerieten sie da hinein, die Überlebenden\textless sup
title=``oder: die Übriggebliebenen''\textgreater✲ aber flohen ins
Gebirge (Juda). \bibleverse{11} Da plünderten jene Sodom und Gomorrha
ganz aus, raubten alle ihre Lebensmittel und zogen damit ab;
\bibleverse{12} sie nahmen auch Lot, Abrams Brudersohn, der damals in
Sodom wohnte, samt seinem Hab und Gut mit sich und zogen ab.

\hypertarget{bb-abrams-tatkruxe4ftiges-und-erfolgreiches-eingreifen}{%
\subparagraph{bb) Abrams tatkräftiges und erfolgreiches
Eingreifen}\label{bb-abrams-tatkruxe4ftiges-und-erfolgreiches-eingreifen}}

\bibleverse{13} Da kam ein Flüchtling und meldete es Abram, dem
Hebräer\textless sup title=``d.h. dem von jenseits des Euphrat
Stammenden''\textgreater✲; dieser wohnte damals unter\textless sup
title=``oder: bei''\textgreater✲ den Terebinthen des Amoriters Mamre,
der ein Bruder Eskols und Aners, der Bundesgenossen Abrams, war.
\bibleverse{14} Als nun Abram die Kunde erhielt, daß sein Brudersohn
(Lot) gefangen weggeführt worden war, da bot er seine waffengeübten
Leute, dreihundertundachtzehn Mann, die in seinem Hause geboren waren,
zum Kampfe auf und eilte jenen nach bis Dan. \bibleverse{15} Hier teilte
er seine Leute in mehrere Haufen, überfiel die Feinde zur Nachtzeit mit
seinen Knechten, schlug sie und verfolgte sie bis Hoba, das nördlich von
Damaskus liegt. \bibleverse{16} So brachte er die gesamte Habe zurück;
auch seinen Brudersohn Lot und dessen Hab und Gut brachte er zurück,
ebenso die Frauen und das Volk\textless sup title=``d.h. sämtliche
gefangenen Leute''\textgreater✲.

\hypertarget{cc-abrams-zusammentreffen-mit-dem-priesterfuxfcrsten-melchisedek-von-salem}{%
\subparagraph{cc) Abrams Zusammentreffen mit dem Priesterfürsten
Melchisedek von
Salem}\label{cc-abrams-zusammentreffen-mit-dem-priesterfuxfcrsten-melchisedek-von-salem}}

\bibleverse{17} Als Abram nun von seinem Siege über Kedorlaomer und die
mit ihm verbündeten Könige zurückkehrte, ging ihm der König von Sodom
entgegen in das Tal Sawe, das ist das Königstal. \bibleverse{18}
Melchisedek aber, der König von Salem, brachte Brot und Wein aus der
Stadt heraus; er war aber ein Priester des höchsten Gottes.
\bibleverse{19} Er segnete ihn dann mit den Worten: »Gesegnet seist du,
Abram, vom höchsten Gott, dem Schöpfer des Himmels und der Erde,
\bibleverse{20} und gepriesen sei der höchste Gott, der dir deine Feinde
in die Hand geliefert hat!« Ihm gab (Abram) alsdann den Zehnten von
allem\textless sup title=``Hebr 7,2''\textgreater✲.

\hypertarget{dd-abrams-edelmut-gegen-den-kuxf6nig-von-sodom}{%
\subparagraph{dd) Abrams Edelmut gegen den König von
Sodom}\label{dd-abrams-edelmut-gegen-den-kuxf6nig-von-sodom}}

\bibleverse{21} Da sagte der König von Sodom zu Abram: »Gib mir die
(gefangenen) Leute und behalte die Habe für dich!« \bibleverse{22} Aber
Abram antwortete dem König von Sodom: »Ich hebe meine Hand zum HERRN
auf, zum höchsten Gott, dem Schöpfer des Himmels und der Erde (und
schwöre): \bibleverse{23} Keinen Faden und keinen Schuhriemen, überhaupt
nichts von deinem ganzen Eigentum will ich behalten! Du sollst nicht
sagen können, du habest Abram reich gemacht. \bibleverse{24} Ich will
nichts davon! Nur was die Knechte✲ verzehrt haben und den Beuteanteil,
der den mit mir verbündeten Männern Aner, Eskol und Mamre zukommt: die
sollen ihren Anteil nehmen!«

\hypertarget{d-abrams-bewuxe4hrter-glaube-und-gottes-erneuerte-verheiuxdfung-an-ihn}{%
\paragraph{d) Abrams bewährter Glaube und Gottes erneuerte Verheißung an
ihn}\label{d-abrams-bewuxe4hrter-glaube-und-gottes-erneuerte-verheiuxdfung-an-ihn}}

\hypertarget{aa-gottes-verheiuxdfung-eines-leibeserben-an-abram-und-abrams-glaubensfestigkeit}{%
\subparagraph{aa) Gottes Verheißung eines Leibeserben an Abram und
Abrams
Glaubensfestigkeit}\label{aa-gottes-verheiuxdfung-eines-leibeserben-an-abram-und-abrams-glaubensfestigkeit}}

\hypertarget{section-14}{%
\section{15}\label{section-14}}

\bibleverse{1} Nach diesen Begebenheiten erging das Wort des HERRN an
Abram in einem Gesicht also: »Fürchte dich nicht, Abram! Ich bin ja dein
Schild; dein Lohn soll sehr groß sein.« \bibleverse{2} Abram aber
antwortete: »Ach HERR, mein Gott, was könntest du mir geben? Ich gehe ja
als kinderloser Mann dahin\textless sup title=``=~von
hinnen''\textgreater✲, und der Besitzer meines Vermögens wird (mein
hausgeborener Knecht) Elieser von Damaskus sein.« \bibleverse{3} Dann
fuhr Abram fort: »Ach, du hast mir ja keine Kinder gegeben {[}darum wird
einer von den Knechten meines Hauses mein Erbe sein{]}!« \bibleverse{4}
Aber da erging das Wort des HERRN an ihn also: »Nicht dieser soll dein
Erbe sein; sondern ein leiblicher Sproß✲ soll es sein, der dich beerbt.«
\bibleverse{5} Darauf ließ er ihn ins Freie hinaustreten und sagte:
»Blicke zum Himmel empor und zähle die Sterne, wenn du sie zählen
kannst!« Dann fuhr er fort: »So (unzählbar) soll deine Nachkommenschaft
sein!« \bibleverse{6} Da glaubte Abram dem HERRN, und das rechnete
dieser ihm als Gerechtigkeit an\textless sup title=``Röm
4,3''\textgreater✲.

\hypertarget{bb-gott-bestuxe4tigt-seine-verheiuxdfung-durch-einen-feierlichen-bundesschluuxdf-unter-vornahme-einer-ernsten-opferhandlung}{%
\subparagraph{bb) Gott bestätigt seine Verheißung durch einen
feierlichen Bundesschluß unter Vornahme einer ernsten
Opferhandlung}\label{bb-gott-bestuxe4tigt-seine-verheiuxdfung-durch-einen-feierlichen-bundesschluuxdf-unter-vornahme-einer-ernsten-opferhandlung}}

\bibleverse{7} Dann sagte Gott zu ihm: »Ich bin der HERR, der dich aus
Ur in Chaldäa hat auswandern lassen, um dir dieses Land zum Besitz zu
geben.« \bibleverse{8} Abram erwiderte: »HERR, mein Gott! Woran soll ich
erkennen, daß ich es besitzen werde?« \bibleverse{9} Da antwortete er
ihm: »Hole mir eine dreijährige Kuh, eine dreijährige Ziege und einen
dreijährigen Widder, dazu eine Turteltaube und eine junge Taube!«
\bibleverse{10} Da holte er ihm alle diese Tiere, schnitt sie in der
Mitte durch und legte die Hälften eines jeden Tieres einander gegenüber;
die Vögel aber schnitt er nicht entzwei. \bibleverse{11} Da stießen die
Raubvögel auf die Fleischstücke herab, aber Abram verscheuchte sie.
\bibleverse{12} Als nun die Sonne sich zum Untergang neigte, fiel ein
tiefer Schlaf auf Abram, und zugleich stellte sich eine Beängstigung,
tiefe Finsternis, bei ihm ein. \bibleverse{13} Da sprach er\textless sup
title=``d.h. Gott''\textgreater✲ zu Abram: »Sicher wissen sollst du, daß
deine Nachkommen als Fremdlinge in einem Lande weilen werden, das ihnen
nicht gehört; dort werden sie als Knechte✲ dienen müssen, und man wird
sie bedrücken vierhundert Jahre lang. \bibleverse{14} Aber auch das
Volk, dem sie dienen müssen, will ich zur Rechenschaft ziehen; und
darnach werden sie mit reicher Habe ausziehen. \bibleverse{15} Du aber
sollst in Frieden zu deinen Vätern eingehen und in gutem✲ Alter begraben
werden. \bibleverse{16} Aber erst das vierte Geschlecht von ihnen wird
hierher zurückkehren; denn das Maß der Sündenschuld der Amoriter ist bis
jetzt noch nicht voll.«

\bibleverse{17} Als dann die Sonne untergegangen und tiefe Dunkelheit
eingetreten war, da war es wie ein rauchender Backofen und eine
Feuerfackel, was zwischen jenen Fleischstücken hindurchfuhr\textless sup
title=``oder: hindurchschritt''\textgreater✲. \bibleverse{18} An jenem
Tage schloß der HERR einen Bund mit Abram und erklärte: »Deiner
Nachkommenschaft will ich dieses Land geben vom Bach Ägyptens bis an den
großen Strom, den Euphratstrom: \bibleverse{19} die Keniter, Kenissiter,
Kadmoniter, \bibleverse{20} Hethiter, Pherissiter, Rephaiter,
\bibleverse{21} Amoriter, Kanaaniter, Girgasiter und Jebusiter.«

\hypertarget{e-sarai-gibt-ihre-leibmagd-hagar-dem-abram-zum-weibe-hagars-flucht-und-ruxfcckkehr-ismaels-geburt}{%
\paragraph{e) Sarai gibt ihre Leibmagd Hagar dem Abram zum Weibe; Hagars
Flucht und Rückkehr; Ismaels
Geburt}\label{e-sarai-gibt-ihre-leibmagd-hagar-dem-abram-zum-weibe-hagars-flucht-und-ruxfcckkehr-ismaels-geburt}}

\hypertarget{section-15}{%
\section{16}\label{section-15}}

\bibleverse{1} Sarai, Abrams Frau, hatte ihm keine Kinder geboren; sie
hatte aber eine ägyptische Leibmagd namens Hagar. \bibleverse{2} Da
sagte Sarai zu Abram: »Du siehst, daß der HERR mir Kindersegen versagt
hat. So gehe doch ein zu meiner Leibmagd: vielleicht komme ich durch sie
zu Kindern.« Als Abram auf diesen Vorschlag seiner Frau einging,
\bibleverse{3} nahm Sarai, Abrams Frau, ihre ägyptische Leibmagd Hagar
und gab sie ihrem Manne Abram zum Weibe\textless sup title=``=~zur
Nebenfrau''\textgreater✲. -- Abram hatte damals aber zehn Jahre lang im
Lande Kanaan gewohnt.~-- \bibleverse{4} Abram ging dann zu Hagar ein,
und sie wurde guter Hoffnung; als sie aber merkte, daß sie Mutter werden
würde, sah sie ihre Herrin geringschätzig an. \bibleverse{5} Da sagte
Sarai zu Abram: »Die Kränkung, die mir zugefügt wird, ist deine Schuld!
Ich selbst habe dir meine Leibmagd in die Arme gegeben; jetzt aber, da
sie fühlt, daß sie Mutter werden wird, sieht sie mich geringschätzig an:
der HERR sei Richter zwischen mir und dir!« \bibleverse{6} Da sagte
Abram zu Sarai: »Deine Leibmagd steht ja doch unter deiner Gewalt:
verfahre mit ihr, wie es dich gut dünkt!« Als nun Sarai sie hart
behandelte, entfloh sie ihr.

\hypertarget{gott-offenbart-sich-der-hagar-an-der-quelle-in-der-wuxfcste-ismaels-geburt}{%
\paragraph{Gott offenbart sich der Hagar an der Quelle in der Wüste;
Ismaels
Geburt}\label{gott-offenbart-sich-der-hagar-an-der-quelle-in-der-wuxfcste-ismaels-geburt}}

\bibleverse{7} Da fand der Engel des HERRN sie an einer Wasserquelle in
der Wüste, an der Quelle auf dem Wege nach Sur, \bibleverse{8} und
fragte sie: »Hagar, Leibmagd der Sarai, woher kommst du, und wohin
willst du?« Sie antwortete: »Ich bin auf der Flucht vor meiner Herrin
Sarai.« \bibleverse{9} Da sagte der Engel des HERRN zu ihr: »Kehre zu
deiner Herrin zurück und unterwirf dich ihrer Gewalt.« \bibleverse{10}
Dann fuhr der Engel des HERRN fort: »Ich will deine Nachkommenschaft
überaus zahlreich werden lassen, so daß man sie vor Menge nicht soll
zählen können.« \bibleverse{11} Weiter sagte der Engel des HERRN zu ihr:
»Du bist jetzt guter Hoffnung und wirst Mutter eines Sohnes werden, den
du Ismael\textless sup title=``d.h. Gott hört''\textgreater✲ nennen
sollst; denn der HERR hat auf deinen Notschrei gehört. \bibleverse{12}
Der wird ein Mensch wie ein Wildesel sein: seine Hand gegen alle und die
Hand aller gegen ihn, und allen seinen Brüdern wird er trotzig
gegenüberstehen.« \bibleverse{13} Da nannte sie den Namen des HERRN, der
zu ihr geredet hatte: »Du bist der Gott des Schauens; denn«, sagte sie,
»ich habe wirklich hier den geschaut, der nach mir geschaut hat.«
\bibleverse{14} Darum hat man den Brunnen ›Brunnen des Lebendigen, der
nach mir schaut‹ genannt; er liegt bekanntlich zwischen Kades und Bered.

\bibleverse{15} Hagar gebar dann dem Abram einen Sohn, und Abram gab
seinem Sohne, den Hagar ihm geboren hatte, den Namen Ismael.
\bibleverse{16} Abram war aber sechsundachtzig Jahre alt, als Hagar ihm
den Ismael gebar.

\hypertarget{f-wiederholter-bundesschluuxdf-gottes-mit-abram-einsetzung-der-beschneidung-ankuxfcndigung-der-geburt-isaaks}{%
\paragraph{f) Wiederholter Bundesschluß Gottes mit Abram; Einsetzung der
Beschneidung; Ankündigung der Geburt
Isaaks}\label{f-wiederholter-bundesschluuxdf-gottes-mit-abram-einsetzung-der-beschneidung-ankuxfcndigung-der-geburt-isaaks}}

\hypertarget{aa-mitteilung-der-guxf6ttlichen-bundesverheiuxdfungen-abram-erhuxe4lt-den-namen-abraham}{%
\subparagraph{aa) Mitteilung der göttlichen Bundesverheißungen; Abram
erhält den Namen
Abraham}\label{aa-mitteilung-der-guxf6ttlichen-bundesverheiuxdfungen-abram-erhuxe4lt-den-namen-abraham}}

\hypertarget{section-16}{%
\section{17}\label{section-16}}

\bibleverse{1} Als nun Abram neunundneunzig Jahre alt war, erschien ihm
der HERR und sagte zu ihm: »Ich bin der allmächtige Gott: wandle vor mir
und sei fromm! \bibleverse{2} Ich will einen Bund zwischen mir und dir
stiften und dich überaus zahlreich werden lassen.« \bibleverse{3} Da
warf sich Abram auf sein Angesicht nieder; Gott aber redete weiter mit
ihm so: \bibleverse{4} »Wisse wohl: mein Bund mit dir geht dahin, daß du
der Stammvater einer Menge von Völkern werden sollst. \bibleverse{5}
Darum sollst du hinfort nicht mehr Abram\textless sup title=``d.h.
erhabener Vater''\textgreater✲ heißen, sondern dein Name soll jetzt
Abraham\textless sup title=``d.h. Vater einer Menge''\textgreater✲
lauten; denn zum Stammvater einer Menge von Völkern habe ich dich
bestimmt. \bibleverse{6} Ich will dich also überaus zahlreich werden
lassen und dich zu (ganzen) Völkern machen; auch Könige sollen von dir
abstammen. \bibleverse{7} Und ich will meinen Bund
errichten\textless sup title=``oder: aufrechterhalten''\textgreater✲
zwischen mir und dir und deinen Nachkommen nach dir, Geschlecht für
Geschlecht, als einen ewigen Bund, um dein Gott zu sein und (der Gott)
deiner Nachkommen nach dir. \bibleverse{8} Und ich will dir und deinen
Nachkommen nach dir das Land, in dem du (jetzt) als Fremdling weilst,
nämlich das ganze Land Kanaan, zum ewigen Besitz geben und will ihr Gott
sein.«

\hypertarget{bb-das-gebot-der-beschneidung-als-des-uxe4uuxdferen-bundeszeichens}{%
\subparagraph{bb) Das Gebot der Beschneidung als des äußeren
Bundeszeichens}\label{bb-das-gebot-der-beschneidung-als-des-uxe4uuxdferen-bundeszeichens}}

\bibleverse{9} Weiter sagte Gott zu Abraham: »Was dich aber betrifft, so
sollst du den Bund mit mir halten, du samt deinen Nachkommen nach dir,
Geschlecht für Geschlecht! \bibleverse{10} Dies aber ist mein Bund, den
ihr halten sollt und der zwischen mir und euch und deinen Nachkommen
nach dir besteht: Alles Männliche soll bei euch beschnitten werden!
\bibleverse{11} Und zwar sollt ihr am Fleisch eurer Vorhaut beschnitten
werden: das soll das Zeichen des Bundes zwischen mir und euch sein!
\bibleverse{12} Jedes Knäblein soll im Alter von acht Tagen bei euch die
Beschneidung empfangen, Geschlecht für Geschlecht, auch der im Hause
geborene, sowie der für Geld von irgendeinem Fremden gekaufte Knecht,
mag er auch nicht zu deiner Nachkommenschaft gehören. \bibleverse{13}
Ja, beschnitten soll werden sowohl der in deinem Hause geborene als auch
der für Geld von dir gekaufte Knecht: darin soll mein Bundeszeichen an
eurem Leibe bestehen als ein ewiges Bundeszeichen! \bibleverse{14} Ein
unbeschnittener Männlicher aber, der am Fleisch seiner Vorhaut nicht
beschnitten worden ist -- ein solcher Mensch soll aus seinen
Volksgenossen ausgerottet werden: meinen Bund hat er gebrochen!«

\hypertarget{cc-sarai-erhuxe4lt-den-namen-sara-verheiuxdfung-der-geburt-isaaks-als-des-wahren-bundeserben}{%
\subparagraph{cc) Sarai erhält den Namen Sara; Verheißung der Geburt
Isaaks als des wahren
Bundeserben}\label{cc-sarai-erhuxe4lt-den-namen-sara-verheiuxdfung-der-geburt-isaaks-als-des-wahren-bundeserben}}

\bibleverse{15} Weiter sprach Gott zu Abraham: »Deine Frau Sarai sollst
du nicht mehr Sarai nennen, sondern Sara\textless sup title=``d.h.
Fürstin''\textgreater✲ soll ihr Name sein. \bibleverse{16} Denn ich will
sie segnen und dir auch von ihr einen Sohn geben; ja ich will sie
segnen, daß sie zu (ganzen) Völkern werden soll; sogar Könige von
Völkerschaften sollen von ihr abstammen.« \bibleverse{17} Da warf sich
Abraham auf sein Angesicht nieder und lachte; denn er dachte bei sich:
»Einem Hundertjährigen soll noch (ein Sohn) geboren werden? Und die
neunzigjährige Sara soll noch Mutter werden?« \bibleverse{18} So sagte
denn Abraham zu Gott: »Ach möchte nur Ismael vor dir am Leben bleiben!«
\bibleverse{19} Doch Gott antwortete: »Ganz gewiß wird deine Frau Sara
dir einen Sohn gebären, den du Isaak\textless sup title=``d.h.
Lacher''\textgreater✲ nennen sollst; und ich will meinen Bund mit ihm
aufrichten\textless sup title=``oder: aufrechterhalten''\textgreater✲
als einen ewigen Bund für seine Nachkommen nach ihm. \bibleverse{20}
Aber auch in betreff Ismaels habe ich dich erhört; wisse wohl: ich will
ihn segnen und fruchtbar werden lassen und ihm eine überaus zahlreiche
Nachkommenschaft verleihen: zwölf Fürsten soll er zu Nachkommen haben,
und zu einem großen Volke will ich ihn machen. \bibleverse{21} Jedoch
meinen Bund will ich mit Isaak aufrichten\textless sup title=``oder:
aufrechterhalten''\textgreater✲, der dir von Sara übers Jahr um diese
Zeit geboren werden soll.« \bibleverse{22} Als Gott nun seine
Unterredung mit Abraham beendet hatte, fuhr er (zum Himmel) empor von
Abraham weg.

\hypertarget{dd-vollziehung-der-beschneidung-im-gesamten-hause-abrahams}{%
\subparagraph{dd) Vollziehung der Beschneidung im gesamten Hause
Abrahams}\label{dd-vollziehung-der-beschneidung-im-gesamten-hause-abrahams}}

\bibleverse{23} Darauf nahm Abraham seinen Sohn Ismael und alle in
seinem Hause geborenen Knechte sowie alle für Geld von ihm gekauften
Knechte, alle männlichen Personen unter den Leuten in seinem Hause, und
vollzog die Beschneidung an ihnen noch an eben diesem Tage, wie Gott es
ihm geboten hatte. \bibleverse{24} Abraham war aber neunundneunzig Jahre
alt, als er vorschriftsgemäß beschnitten wurde, \bibleverse{25} und sein
Sohn Ismael war dreizehn Jahre alt, als man ihn vorschriftsgemäß
beschnitt. \bibleverse{26} An einem und demselben Tage wurden Abraham
und sein Sohn Ismael beschnitten; \bibleverse{27} und mit ihm wurden
alle männlichen Personen in seinem hause beschnitten, sowohl die im
Hause geborenen als auch die für Geld von Fremden gekauften Knechte.

\hypertarget{g-besuch-der-drei-himmlischen-bei-abraham-im-hain-mamres-zu-hebron-abermalige-gottesverheiuxdfung}{%
\paragraph{g) Besuch der drei Himmlischen bei Abraham (im Hain Mamres)
zu Hebron; abermalige
Gottesverheißung}\label{g-besuch-der-drei-himmlischen-bei-abraham-im-hain-mamres-zu-hebron-abermalige-gottesverheiuxdfung}}

\hypertarget{section-17}{%
\section{18}\label{section-17}}

\bibleverse{1} Dann erschien ihm der HERR bei den Terebinthen Mamres,
während er gerade um die Zeit der Mittagshitze am\textless sup
title=``oder: im''\textgreater✲ Eingang seines Zeltes saß.
\bibleverse{2} Als er nämlich aufblickte und hinsah, standen plötzlich
drei Männer vor ihm. Kaum hatte er sie erblickt, da eilte er ihnen vom
Eingang seines Zeltes aus entgegen, verneigte sich vor ihnen bis auf den
Boden \bibleverse{3} und sagte: »O Herr, wenn ich irgend Gnade in deinen
Augen gefunden habe, so gehe doch nicht an deinem Knechte vorüber!
\bibleverse{4} Man soll euch etwas Wasser bringen, damit ihr euch die
Füße waschen könnt; dann ruht euch unter dem Baume aus, \bibleverse{5}
und ich will euch etwas zu essen holen, damit ihr euch erquickt: danach
mögt ihr weiterziehen; ihr seid doch nun einmal bei eurem Knecht
vorübergekommen.« Sie antworteten: »Tu so, wie du gesagt hast!«

\bibleverse{6} Da eilte Abraham zu Sara ins Zelt und sagte: »Nimm
schnell drei Maß Mehl, feines Mehl, knete es und backe Kuchen!«
\bibleverse{7} Dann eilte Abraham zu den Rindern, nahm ein zartes,
gutes✲ Kalb und übergab es dem Knechte; der mußte es schnell zubereiten.
\bibleverse{8} Dann holte er Sauermilch und süße Milch sowie das Kalb,
das er hatte zubereiten lassen, und setzte es ihnen vor; er selbst aber
bediente sie unter dem Baume, während sie aßen. \bibleverse{9} Da
fragten sie ihn: »Wo ist deine Frau Sara?« Er antwortete: »Drinnen im
Zelt.« \bibleverse{10} Da sagte jener: »Übers Jahr um diese Zeit will
ich wieder zu dir kommen: dann wird deine Frau Sara einen Sohn haben.«
Sara horchte aber am Zelteingang, der hinter ihm war. \bibleverse{11}
Abraham und Sara waren aber alt und hochbetagt, so daß Sara nach ihrer
leiblichen Beschaffenheit keine Kinder mehr erwarten konnte.
\bibleverse{12} Darum lachte Sara in sich hinein und dachte: »Jetzt,
nachdem ich verwelkt bin, sollte ich noch an Liebeslust denken? Und mein
Eheherr ist ja auch ein Greis.« \bibleverse{13} Da sagte der HERR zu
Abraham: »Warum hat denn Sara gelacht und denkt: ›Sollte ich alte Frau
wirklich noch Mutter werden können?‹ \bibleverse{14} Ist etwa für den
HERRN irgend etwas unmöglich? Zu der genannten Zeit, übers Jahr, komme
ich wieder zu dir: dann wird Sara einen Sohn haben.« \bibleverse{15} Da
leugnete Sara und sagte: »Ich habe nicht gelacht!«, denn sie fürchtete
sich. Er aber entgegnete: »Doch, du hast gelacht!«

\hypertarget{h-aufbruch-der-muxe4nner-ankuxfcndigung-des-gerichts-uxfcber-sodom-abrahams-fuxfcrbitte-fuxfcr-die-gerechten-in-sodom}{%
\paragraph{h) Aufbruch der Männer; Ankündigung des Gerichts über Sodom;
Abrahams Fürbitte für (die Gerechten in)
Sodom}\label{h-aufbruch-der-muxe4nner-ankuxfcndigung-des-gerichts-uxfcber-sodom-abrahams-fuxfcrbitte-fuxfcr-die-gerechten-in-sodom}}

\bibleverse{16} Nunmehr brachen die drei Männer von dort auf und
schauten aus nach Sodom hinab, während Abraham mit ihnen ging, um ihnen
das Geleit zu geben. \bibleverse{17} Da dachte der HERR: »Soll ich vor
Abraham geheimhalten, was ich zu tun vorhabe? \bibleverse{18} Abraham
soll ja doch zu einem großen und mächtigen Volk werden, und in ihm
sollen alle Völker der Erde gesegnet werden; \bibleverse{19} denn ich
habe ihn dazu ausersehen, daß er seinen Söhnen und seinem ganzen Hause
nach ihm ans Herz lege, den Weg des HERRN innezuhalten, indem sie
Gerechtigkeit und Recht üben, damit der HERR für Abraham alles in
Erfüllung gehen lasse, was er in bezug auf ihn verheißen hat.«
\bibleverse{20} So sagte denn der HERR: »Das Geschrei\textless sup
title=``oder: der Klageruf''\textgreater✲ über Sodom und Gomorrha ist
gar groß geworden, und ihre Sünde ist wahrlich sehr schwer.
\bibleverse{21} Darum will ich hinabgehen und zusehen, ob sie wirklich
ganz so gehandelt haben, wie die lauten Klagen, die zu mir gedrungen
sind, von ihnen melden, oder ob es sich nicht so verhält: ich will es
erkunden.« \bibleverse{22} Hierauf wandten sich die (anderen beiden)
Männer von dort weg und gingen auf Sodom zu, während Abraham noch vor
dem HERRN stehenblieb.

\bibleverse{23} Da trat Abraham näher heran und sagte: »Willst du
wirklich die Gerechten\textless sup title=``oder:
Schuldlosen''\textgreater✲ zugleich mit den Gottlosen wegraffen?
\bibleverse{24} Vielleicht gibt es fünfzig Gerechte innerhalb der Stadt:
willst du die wirklich umkommen lassen und nicht lieber dem Orte
vergeben um der fünfzig Gerechten willen, die in ihm sind?
\bibleverse{25} Fern sei es von dir, so zu handeln, die Gerechten
zusammen mit den Gottlosen ums Leben zu bringen, so daß es den Gerechten
ebenso ergeht wie den Gottlosen: das sei fern von dir! Der Richter der
ganzen Erde muß doch Gerechtigkeit üben!« \bibleverse{26} Da antwortete
der HERR: »Wenn ich in Sodom fünfzig Gerechte innerhalb der Stadt finden
sollte, so will ich dem ganzen Ort um ihretwillen vergeben.«
\bibleverse{27} Darauf nahm Abraham wieder das Wort und sagte: »Ach
siehe, ich habe es gewagt, zu dem Allherrn zu reden, obgleich ich nur
Staub und Asche bin. \bibleverse{28} Vielleicht fehlen an den fünfzig
Gerechten noch fünf: willst du da wegen dieser fünf die ganze Stadt
vernichten?« Er antwortete: »Nein, ich will sie nicht vernichten, wenn
ich dort fünfundvierzig finde.« \bibleverse{29} Darauf fuhr Abraham
fort, ihn nochmals anzureden, und sagte: »Vielleicht finden sich deren
dort nur vierzig.« Jener erwiderte: »Ich will ihnen um der vierzig
willen nichts tun.« \bibleverse{30} Abraham sagte: »Möge doch der
Allherr nicht zürnen, wenn ich nochmals rede: vielleicht finden sich
dort nur dreißig.« Er antwortete: »Ich will ihnen nichts tun, wenn ich
dort dreißig finde.« \bibleverse{31} Er sagte weiter: »Siehe doch, ich
habe es gewagt, zu dem Allherrn zu reden: vielleicht finden sich dort
nur zwanzig.« Er antwortete: »Ich will sie schon um der zwanzig willen
nicht vernichten.« \bibleverse{32} Da sagte er: »Möge doch der Allherr
nicht zürnen, wenn ich noch dies eine Mal rede: vielleicht finden sich
dort nur zehn.« Er erwiderte: »Ich will sie schon um der zehn willen
nicht vernichten.« \bibleverse{33} Hierauf ging der HERR weg, nachdem er
das Gespräch mit Abraham beendet hatte; Abraham aber kehrte nach Hause
zurück.

\hypertarget{i-einkehr-der-beiden-himmlischen-muxe4nner-bei-lot-sodoms-sittenlosigkeit-und-untergang-lots-rettung}{%
\paragraph{i) Einkehr der beiden himmlischen Männer bei Lot; Sodoms
Sittenlosigkeit und Untergang; Lots
Rettung}\label{i-einkehr-der-beiden-himmlischen-muxe4nner-bei-lot-sodoms-sittenlosigkeit-und-untergang-lots-rettung}}

\hypertarget{section-18}{%
\section{19}\label{section-18}}

\bibleverse{1} Als nun die beiden Engel am Abend nach Sodom kamen, saß
Lot gerade am\textless sup title=``oder: im''\textgreater✲ Tor von
Sodom. Sobald Lot sie erblickte, erhob er sich vor ihnen, verneigte sich
mit dem Angesicht bis zur Erde \bibleverse{2} und sagte: »Bitte, meine
Herren! Kehrt doch im Hause eures Knechtes ein, um dort zu übernachten,
und wascht euch die Füße; morgen früh mögt ihr euch dann wieder
aufmachen und eures Weges ziehen.« Sie aber antworteten: »Nein, wir
wollen hier im Freien\textless sup title=``oder: auf der
Straße''\textgreater✲ übernachten.« \bibleverse{3} Da nötigte er sie
dringend, bis sie bei ihm einkehrten und in sein Haus eintraten. Dann
bereitete er ihnen ein Mahl und ließ ungesäuerte Kuchen backen, die sie
aßen. \bibleverse{4} Noch hatten sie sich aber nicht schlafen gelegt,
als die Männer der Stadt, die Bürger von Sodom, das Haus umzingelten,
jung und alt, die ganze Bevölkerung bis auf den letzten Mann.
\bibleverse{5} Die riefen nach Lot und sagten zu ihm: »Wo sind die
Männer, die heute abend zu dir gekommen sind? Bringe sie zu uns heraus,
damit wir uns an sie machen!« \bibleverse{6} Da trat Lot zu ihnen hinaus
an den Eingang des Hauses, schloß aber die Tür hinter sich zu
\bibleverse{7} und sagte: »Meine Brüder, vergeht euch doch nicht so arg!
\bibleverse{8} Hört: ich habe zwei Töchter, die noch mit keinem Manne zu
tun gehabt haben; die will ich zu euch herausbringen: macht dann mit
ihnen, was euch beliebt. Nur diesen Männern tut nichts zuleide, nachdem
sie einmal unter den Schatten meines Daches getreten sind!«
\bibleverse{9} Doch sie antworteten: »Zurück da!«, und weiter sagten
sie: »Der ist der einzige Fremde, der gekommen ist, um hier zu wohnen,
und will nun den Herrn spielen! Warte nur, wir wollen es mit dir noch
schlimmer machen als mit jenen!« So drangen sie denn auf den Mann, auf
Lot, mit Gewalt ein und gingen daran, die Tür zu erbrechen;
\bibleverse{10} doch die Männer griffen mit ihren Händen hinaus, zogen
Lot zu sich ins Haus herein und verschlossen die Tür; \bibleverse{11}
dann schlugen sie die Männer vor dem Eingang des Hauses mit Blindheit,
klein und groß, so daß sie sich vergebens bemühten, den Eingang zu
finden.

\hypertarget{lots-errettung-versteinerung-seiner-frau-zerstuxf6rung-sodoms-und-gomorrhas}{%
\paragraph{Lots Errettung; Versteinerung seiner Frau; Zerstörung Sodoms
und
Gomorrhas}\label{lots-errettung-versteinerung-seiner-frau-zerstuxf6rung-sodoms-und-gomorrhas}}

\bibleverse{12} Darauf sagten die Männer zu Lot: »Wen du sonst noch hier
hast -- einen Schwiegersohn sowie deine Söhne und Töchter und wer dir
sonst noch in der Stadt angehört --, die laß aus diesem Orte weggehen;
\bibleverse{13} denn wir wollen diesen Ort zerstören, weil schlimme
Klagen über ihn vor dem HERRN laut geworden sind; daher hat der HERR uns
gesandt, die Stadt zu zerstören.« \bibleverse{14} Da ging Lot aus dem
Hause hinaus und sagte zu seinen Schwiegersöhnen, die seine Töchter
geheiratet hatten\textless sup title=``oder: heiraten
wollten''\textgreater✲: »Macht euch auf und verlaßt diesen Ort! Denn der
HERR will die Stadt zerstören.« Aber er kam seinen Schwiegersöhnen vor
wie einer, der Scherz (mit ihnen) trieb. \bibleverse{15} Als dann die
Morgenröte aufstieg, drängten die Engel Lot zur Eile mit den Worten:
»Auf! Nimm deine Frau und deine beiden Töchter, die hier bei dir
anwesend sind, damit du nicht auch wegen der Sündhaftigkeit der Stadt
ums Leben kommst.« \bibleverse{16} Als er aber immer noch zögerte,
faßten die Männer ihn und seine Frau und seine beiden Töchter bei der
Hand, weil der HERR ihn verschonen wollte; sie führten ihn hinaus und
ließen ihn erst draußen vor der Stadt wieder los. \bibleverse{17} Als
sie nun mit ihnen draußen im Freien waren, sagte der eine: »Rette dich:
es gilt dein Leben! Sieh dich nicht um und bleibe nirgends in der
Jordan-Ebene stehen! Rette dich in das Gebirge, damit du nicht auch ums
Leben kommst!« \bibleverse{18} Da antwortete ihnen Lot: »Ach nein, mein
Herr! \bibleverse{19} Bedenke doch: dein Knecht hat (nun einmal) Gnade
in deinen Augen gefunden, und du hast mir die große Barmherzigkeit
erwiesen, mich am Leben zu erhalten; aber ich vermag mich nicht in das
Gebirge zu retten: das Verderben würde mich ereilen, so daß ich sterben
müßte! \bibleverse{20} Siehe, dort ist eine Ortschaft in der Nähe, so
daß ich dahin fliehen könnte, und sie ist ja ganz klein: dorthin möchte
ich mich retten; sie ist ja doch ganz klein; dann könnte ich am Leben
bleiben!« \bibleverse{21} Da antwortete er ihm: »Nun gut, ich will dir
auch in diesem Stück zu Willen sein, indem ich den Ort, von dem du
sprichst, nicht mit zerstöre. \bibleverse{22} Flüchte dich eilends
dorthin! denn ich kann nichts tun, bis du dorthin gekommen bist.« Daher
hat der Ort den Namen Zoar\textless sup title=``d.h.
Kleinheit''\textgreater✲ erhalten.

\bibleverse{23} Als dann die Sonne über der Erde aufgegangen und Lot in
Zoar angekommen war, \bibleverse{24} ließ der HERR Schwefel und Feuer
vom Himmel herab auf Sodom und Gomorrha regnen \bibleverse{25} und
vernichtete diese Städte und die ganze Jordan-Ebene samt allen Bewohnern
der Ortschaften und allem, was auf den Fluren gewachsen war.
\bibleverse{26} Lots Frau aber hatte sich hinter ihm umgeschaut; da
wurde sie zu einer Salzsäule.

\bibleverse{27} Als Abraham sich nun am folgenden Morgen in der Frühe an
den Ort begab, wo er vor dem HERRN gestanden hatte, \bibleverse{28} und
nach Sodom und Gomorrha hinabschaute und die ganze Fläche der
Jordan-Ebene überblickte, da sah er, wie der Rauch vom Lande aufstieg
gleich dem Rauch von einem Schmelzofen. \bibleverse{29} Gott aber hatte,
als er die Städte in der Jordan-Ebene zerstörte, an Abraham gedacht und
Lot mitten aus der Zerstörung hinausgeführt, als er die Städte
zerstörte, in denen Lot gewohnt hatte.

\hypertarget{k-das-suxfcndhafte-verhalten-der-zwei-tuxf6chter-lots-geburt-der-stammvuxe4ter-der-moabiter-und-der-ammoniter}{%
\paragraph{k) Das sündhafte Verhalten der zwei Töchter Lots; Geburt der
Stammväter der Moabiter und der
Ammoniter}\label{k-das-suxfcndhafte-verhalten-der-zwei-tuxf6chter-lots-geburt-der-stammvuxe4ter-der-moabiter-und-der-ammoniter}}

\bibleverse{30} Lot aber zog aus Zoar weiter aufwärts und nahm seinen
Wohnsitz zusammen mit seinen beiden Töchtern im Gebirge, denn er
fürchtete sich, in Zoar zu bleiben; er ließ sich vielmehr mit seinen
beiden Töchtern in einer Höhle nieder. \bibleverse{31} Da sagte die
ältere zu der jüngeren: »Unser Vater ist alt, und kein Mann ist sonst im
Lande, der Umgang mit uns haben könnte, wie es in aller Welt Brauch ist.
\bibleverse{32} Komm, wir wollen unserm Vater Wein zu trinken geben und
uns zu ihm legen, damit wir von unserm Vater Nachkommenschaft ins Leben
rufen.« \bibleverse{33} So gaben sie denn ihrem Vater an jenem Abend
Wein zu trinken, und die ältere ging dann hinein und legte sich zu ihrem
Vater; er aber merkte nichts davon, weder wie sie sich hinlegte, noch
als sie aufstand. \bibleverse{34} Am andern Morgen sagte dann die ältere
zu der jüngeren: »Siehst du, ich habe in der vorigen Nacht bei meinem
Vater gelegen. Wir wollen ihm nun auch heute abend Wein zu trinken
geben; dann gehst du hinein und legst dich zu ihm, damit wir von unserm
Vater Nachkommenschaft ins Leben rufen.« \bibleverse{35} So gaben sie
denn ihrem Vater auch an diesem Abend Wein zu trinken, und die jüngere
stand auf und legte sich zu ihm; er aber merkte nichts davon, weder als
sie sich hinlegte, noch als sie aufstand. \bibleverse{36} So wurden denn
die beiden Töchter Lots von ihrem Vater schwanger. \bibleverse{37} Und
die ältere gebar einen Sohn und nannte ihn »Moab«\textless sup
title=``d.h. vom Vater''\textgreater✲; der ist der Stammvater der
heutigen Moabiter. \bibleverse{38} Die jüngere gebar auch einen Sohn und
gab ihm den Namen »Ben-Ammi«\textless sup title=``d.h. Sohn meines
Volkes; oder: meines nächsten Verwandten''\textgreater✲; der ist der
Stammvater der heutigen Ammoniter.

\hypertarget{l-abraham-bei-abimelech-in-gerar-gefuxe4hrdung-und-bewahrung-der-ehre-saras}{%
\paragraph{l) Abraham bei Abimelech in Gerar; Gefährdung und Bewahrung
der Ehre
Saras}\label{l-abraham-bei-abimelech-in-gerar-gefuxe4hrdung-und-bewahrung-der-ehre-saras}}

\hypertarget{section-19}{%
\section{20}\label{section-19}}

\bibleverse{1} Abraham brach dann von dort\textless sup title=``d.h. von
Hebron''\textgreater✲ auf (und zog) in den Südgau, wo er seinen Wohnsitz
zwischen Kades und Sur nahm. Er hielt sich aber zeitweise als Fremdling
auch in Gerar auf \bibleverse{2} und gab dort seine Frau Sara für seine
Schwester aus. Da sandte Abimelech, der König von Gerar, hin und ließ
Sara zu sich holen. \bibleverse{3} Aber in der Nacht kam Gott zu
Abimelech im Traum und sagte zu ihm: »Jetzt bist du des Todes wegen der
Frau, die du dir hast holen lassen: sie ist ja eines Mannes Ehefrau!«
\bibleverse{4} Abimelech war ihr aber noch nicht nahegekommen; darum
antwortete er: »O Herr, du wirst doch nicht ein schuldloses Volk
umbringen? \bibleverse{5} Hat er nicht selbst zu mir gesagt, daß sie
seine Schwester sei? Und auch sie selbst hat erklärt, er sei ihr Bruder.
In der Unschuld meines Herzens und mit reinen Händen habe ich dies
getan!« \bibleverse{6} Da sagte Gott weiter im Traum zu ihm: »Auch ich
weiß wohl, daß du in der Unschuld deines Herzens so gehandelt hast, und
ich selbst habe dich davor behütet, daß du dich gegen mich versündigt
hast; darum habe ich auch nicht zugelassen, daß du sie berührtest.
\bibleverse{7} So gibt also jetzt dem Manne seine Frau zurück, denn er
ist ein Prophet; dann soll er Fürbitte für dich einlegen, so daß du am
Leben bleibst. Gibst du sie aber nicht zurück, so wisse, daß du mit
allen deinen Angehörigen sterben mußt!«

\bibleverse{8} Am andern Morgen in der Frühe berief Abimelech eiligst
alle seine Diener und teilte ihnen den ganzen Vorfall mit; da gerieten
die Männer in große Bestürzung. \bibleverse{9} Abimelech ließ dann
Abraham rufen und sagte zu ihm: »Was hast du uns da angetan? Worin habe
ich mich dir gegenüber verfehlt, daß du eine so große Verschuldung über
mich und mein Reich gebracht hast? Du hast an mir in einer Weise
gehandelt, wie es nicht recht ist!« \bibleverse{10} Weiter sagte
Abimelech zu Abraham: »Was hast du dir denn dabei gedacht, daß du so
gehandelt hast?« \bibleverse{11} Da antwortete Abraham: »Ja, ich dachte,
es sei sicherlich keine Gottesfurcht an diesem Orte zu finden und man
werde mich um meiner Frau willen ums Leben bringen. \bibleverse{12}
Übrigens ist sie wirklich meine Schwester, die Tochter meines Vaters,
nur nicht die Tochter meiner Mutter, und so hat sie meine Frau werden
können. \bibleverse{13} Als mich nun Gott einst aus meines Vaters
Hause\textless sup title=``oder: Familie''\textgreater✲ ins Ungewisse
wegziehen hieß, da habe ich zu ihr gesagt: ›Erweise mir die Liebe, daß
du überall, wohin wir kommen werden, von mir sagst, ich sei dein
Bruder.‹«

\bibleverse{14} Da nahm Abimelech Kleinvieh und Rinder, Knechte und
Mägde und schenkte sie dem Abraham, auch seine Frau Sara gab er ihm
zurück. \bibleverse{15} Dann fügte er hinzu: »Mein Land steht dir
nunmehr offen\textless sup title=``oder: zur Verfügung''\textgreater✲:
nimm deinen Wohnsitz, wo es dir gefällt!« \bibleverse{16} Zu Sara aber
sagte er: »Hier gebe ich deinem Bruder tausend Silberstücke: das soll
für dich ein Sühnegeld\textless sup title=``=~eine Entschädigung oder:
Ehrenrettung''\textgreater✲ in den Augen aller sein, die bei dir sind,
so daß du nun in allem gerechtfertigt dastehst!« \bibleverse{17} Darauf
legte Abraham Fürbitte bei Gott ein, und Gott ließ Abimelech, seine Frau
und seine Mägde wieder gesund werden, so daß sie wieder Kinder bekommen
konnten; \bibleverse{18} denn der HERR hatte den Mutterschoß aller
Frauen im Hause Abimelechs verschlossen um Saras, der Frau Abrahams,
willen.

\hypertarget{m-isaaks-geburt-beschneidung-entwuxf6hnungsfeier-und-erste-jugend-verstouxdfung-hagars-und-ismaels}{%
\paragraph{m) Isaaks Geburt, Beschneidung, Entwöhnungsfeier und erste
Jugend; Verstoßung Hagars und
Ismaels}\label{m-isaaks-geburt-beschneidung-entwuxf6hnungsfeier-und-erste-jugend-verstouxdfung-hagars-und-ismaels}}

\hypertarget{section-20}{%
\section{21}\label{section-20}}

\bibleverse{1} Der HERR suchte dann Sara gnädig heim, wie er verheißen
hatte, und tat an ihr, wie er zugesagt hatte: \bibleverse{2} Sara wurde
guter Hoffnung und gebar dem Abraham in seinem Greisenalter einen Sohn
zu der Zeit, die Gott ihm im voraus angegeben hatte. \bibleverse{3}
Abraham gab dann seinem Sohne, der ihm geboren worden war, den Sara ihm
geboren hatte, den Namen Isaak\textless sup title=``vgl.
17,19''\textgreater✲ \bibleverse{4} und beschnitt seinen Sohn, als er
acht Tage alt war, wie Gott ihm geboten hatte. \bibleverse{5} Hundert
Jahre war Abraham alt, als ihm sein Sohn Isaak geboren wurde.
\bibleverse{6} Da sagte Sara: »Ein Lachen hat mir Gott bereitet: jeder,
der von der Sache hört, wird über mich lachen.« \bibleverse{7} Weiter
sagte sie: »Wer hätte wohl je dem Abraham gesagt, daß Sara noch Kinder
an der Brust nähren würde? Und nun habe ich ihm doch noch einen Sohn in
seinem Greisenalter geboren!« \bibleverse{8} Und der Knabe wuchs heran
und wurde entwöhnt; da veranstaltete Abraham am Tage der Entwöhnung
Isaaks ein großes Festmahl.

\hypertarget{ismaels-verstouxdfung-und-errettung}{%
\paragraph{Ismaels Verstoßung und
Errettung}\label{ismaels-verstouxdfung-und-errettung}}

\bibleverse{9} Als nun Sara den Sohn der Ägypterin Hagar, den diese dem
Abraham geboren hatte, mit ihrem Sohne Isaak spielen sah,
\bibleverse{10} sagte sie zu Abraham: »Verstoße die Magd da und ihren
Sohn! Denn der Sohn dieser Magd soll nicht mit meinem Sohn, mit Isaak,
erben!« \bibleverse{11} Dieses Wort betrübte Abraham sehr mit Rücksicht
auf seinen Sohn; \bibleverse{12} aber Gott sagte zu Abraham: »Laß es dir
um den Knaben und um deine Magd nicht leid sein: gehorche der Sara in
allem, was sie von dir verlangt; denn nur nach Isaak soll dir
Nachkommenschaft genannt werden. \bibleverse{13} Doch auch den Sohn der
Magd will ich zu einem Volke werden lassen, weil er dein Sohn ist.«

\bibleverse{14} So stand denn Abraham am andern Morgen früh auf, nahm
Brot und einen Schlauch mit Wasser und gab dies der Hagar; den Knaben
aber setzte er ihr auf die Schulter und entließ so beide. Da ging sie
weg und irrte in der Wüste von Beerseba umher. \bibleverse{15} Als dann
das Wasser im Schlauch zu Ende gegangen war, warf sie den Knaben unter
einen der Sträucher, \bibleverse{16} ging weg und setzte sich abseits
ihm gegenüber, wohl einen Bogenschuß weit entfernt; »denn«, sagte sie,
»ich kann das Sterben des Knaben nicht ansehen!« Sie setzte sich also
ihm gegenüber; er aber fing an, laut zu weinen. \bibleverse{17} Da hörte
Gott das Schreien des Knaben, und der Engel Gottes rief der Hagar vom
Himmel her die Worte zu: »Was ist dir, Hagar? Fürchte dich nicht! Denn
Gott hat das Schreien des Knaben gehört, ebendort wo er liegt.
\bibleverse{18} Stehe auf, nimm den Knaben und halte ihn fest an der
Hand, denn ich will ihn zu einem großen Volke werden lassen.«
\bibleverse{19} Dann tat Gott ihr die Augen auf, so daß sie eine Quelle
mit Wasser erblickte; da ging sie hin, füllte den Schlauch mit Wasser
und gab dem Knaben zu trinken. \bibleverse{20} Und Gott war mit dem
Knaben, so daß er heranwuchs; er nahm seinen Aufenthalt in der Wüste und
wurde ein gewaltiger Bogenschütze; \bibleverse{21} und zwar nahm er
seinen Aufenthalt in der Wüste Paran, und seine Mutter nahm ihm eine
Ägypterin zur Frau.

\hypertarget{n-abrahams-vertrag-mit-abimelech-in-beerseba}{%
\paragraph{n) Abrahams Vertrag mit Abimelech in
Beerseba}\label{n-abrahams-vertrag-mit-abimelech-in-beerseba}}

\bibleverse{22} Zu derselben Zeit hatte Abimelech nebst seinem
Heerführer Pichol eine Unterredung mit Abraham und sagte: »Gott ist mit
dir in allem, was du unternimmst. \bibleverse{23} Darum schwöre mir
jetzt hier bei Gott, daß du weder gegen mich noch gegen meine Kinder und
Kindeskinder jemals treulos handeln, sondern dieselbe Freundschaft, die
ich dir erwiesen habe, auch mir und dem Lande erweisen willst, in
welchem du als Fremdling dich aufhältst!« \bibleverse{24} Da antwortete
Abraham: »Ja, ich will den Schwur leisten.« \bibleverse{25} Abraham
machte aber dem Abimelech Vorhalt wegen des Wasserbrunnens, den die
Knechte Abimelechs sich mit Gewalt angeeignet hatten. \bibleverse{26}
Abimelech erwiderte: »Ich weiß nicht, wer das getan hat: weder hast du
mir bisher etwas davon mitgeteilt, noch habe ich bis heute etwas davon
gehört.« \bibleverse{27} Hierauf nahm Abraham Kleinvieh und Rinder und
gab sie dem Abimelech, und sie schlossen beide einen Vertrag
miteinander. \bibleverse{28} Als nun Abraham noch sieben Schaflämmer
abgesondert stellte, \bibleverse{29} fragte Abimelech den Abraham: »Was
sollen die sieben Lämmer hier bedeuten, die du besonders gestellt hast?«
\bibleverse{30} Er antwortete: »Die sieben Lämmer mußt du von mir
annehmen, damit dies mir zum Zeugnis diene, daß ich diesen Brunnen
gegraben habe.« \bibleverse{31} Darum nennt man jenen Ort
»Beerseba«\textless sup title=``d.h. Siebenbrunnen oder:
Eidesbrunnen''\textgreater✲, weil sie beide dort einander geschworen
haben. \bibleverse{32} Nachdem sie so einen Vertrag\textless sup
title=``oder: Bund''\textgreater✲ zu Beerseba geschlossen hatten, brach
Abimelech mit seinem Heerführer Pichol auf, und sie kehrten ins
Philisterland zurück. \bibleverse{33} (Abraham) aber pflanzte eine
Tamariske in Beerseba und rief dort den Namen des HERRN, des ewigen
Gottes, an. \bibleverse{34} Abraham hielt sich dann noch geraume Zeit
als Fremdling im Philisterlande auf.

\hypertarget{o-abrahams-opfergang-nach-der-landschaft-morija}{%
\paragraph{o) Abrahams Opfergang nach der Landschaft
Morija}\label{o-abrahams-opfergang-nach-der-landschaft-morija}}

\hypertarget{aa-gottes-befehl-zur-opferung-isaaks}{%
\subparagraph{aa) Gottes Befehl zur Opferung
Isaaks}\label{aa-gottes-befehl-zur-opferung-isaaks}}

\hypertarget{section-21}{%
\section{22}\label{section-21}}

\bibleverse{1} Nach diesen Begebenheiten wollte Gott den Abraham auf die
Probe stellen✲ und sagte zu ihm: »Abraham!« Dieser antwortete: »Hier bin
ich!« \bibleverse{2} Da sagte Gott: »Nimm Isaak, deinen Sohn, deinen
einzigen, den du liebhast, und begib dich (mit ihm) in die Landschaft
Morija und bringe ihn dort als Brandopfer dar auf einem der Berge, den
ich dir angeben werde!«

\hypertarget{bb-abrahams-bereitwilligkeit-die-wanderung-zur-heiligen-stuxe4tte}{%
\subparagraph{bb) Abrahams Bereitwilligkeit; die Wanderung zur heiligen
Stätte}\label{bb-abrahams-bereitwilligkeit-die-wanderung-zur-heiligen-stuxe4tte}}

\bibleverse{3} Da sattelte\textless sup title=``oder:
bepackte''\textgreater✲ Abraham am andern Morgen früh seinen Esel und
nahm zwei von seinen Knechten und seinen Sohn Isaak mit sich; er
spaltete Holzscheite für das Brandopfer und machte sich dann auf den Weg
nach dem Orte, den Gott ihm angegeben hatte. \bibleverse{4} Als er am
dritten Tage die Augen aufschlug, sah er den Ort in der Ferne liegen.
\bibleverse{5} Da sagte Abraham zu seinen Knechten: »Bleibt ihr für euch
hier mit dem Esel; ich aber und der Knabe wollen dorthin gehen und
anbeten; dann kommen wir wieder zu euch zurück.« \bibleverse{6} Hierauf
nahm Abraham das Holz für das Brandopfer und belud seinen Sohn Isaak
damit; er selbst aber nahm das Feuer und das Schlachtmesser in die Hand,
und so gingen die beiden zusammen weiter. \bibleverse{7} Da sagte Isaak
zu seinem Vater Abraham: »Mein Vater!« Abraham antwortete: »Was willst
du, mein Sohn?« Da sagte er: »Wir haben hier wohl Feuer und Holz; aber
wo ist das Schaf für das Brandopfer?« \bibleverse{8} Abraham erwiderte:
»Gott wird schon für ein Schaf zum Brandopfer sorgen, mein Sohn.« So
gingen die beiden zusammen weiter.

\hypertarget{cc-die-vorbereitungen-zum-opfer-gottes-eingreifen}{%
\subparagraph{cc) Die Vorbereitungen zum Opfer; Gottes
Eingreifen}\label{cc-die-vorbereitungen-zum-opfer-gottes-eingreifen}}

\bibleverse{9} Als sie nun an den Ort gekommen waren, den Gott ihm
angegeben hatte, errichtete Abraham daselbst einen Altar und legte die
Holzscheite auf ihm zurecht; dann band er seinen Sohn Isaak und legte
ihn auf den Altar oben über die Scheite; \bibleverse{10} darauf streckte
er seine Hand aus und nahm das Messer, um seinen Sohn zu schlachten.
\bibleverse{11} Da rief ihm der Engel des HERRN vom Himmel her die Worte
zu: »Abraham, Abraham!« Er antwortete: »Hier bin ich!« \bibleverse{12}
Jener rief: »Lege deine Hand nicht an den Knaben und tu ihm nichts
zuleide! Denn jetzt weiß ich, daß du gottesfürchtig bist, weil du mir
deinen einzigen Sohn nicht vorenthalten hast.« \bibleverse{13} Als
Abraham dann um sich blickte, sah er hinter sich einen Widder, der sich
mit seinen Hörnern im Dickicht verfangen hatte. Da ging Abraham hin,
holte den Widder und brachte ihn statt seines Sohnes als Brandopfer dar.
\bibleverse{14} Abraham nannte dann jenen Ort: »Der HERR sieht«✲;
deshalb sagt man noch heutigentags: »Auf dem Berge, wo der HERR gesehen
wird«✲.

\hypertarget{dd-gottes-anerkennung-und-verheiuxdfungen-fuxfcr-abraham-schluuxdf}{%
\subparagraph{dd) Gottes Anerkennung und Verheißungen für Abraham;
Schluß}\label{dd-gottes-anerkennung-und-verheiuxdfungen-fuxfcr-abraham-schluuxdf}}

\bibleverse{15} Hierauf rief der Engel des HERRN dem Abraham zum
zweitenmal vom Himmel her die Worte zu: \bibleverse{16} »Ich schwöre bei
mir selbst« -- so lautet der Ausspruch des HERRN --: »darum, daß du so
gehandelt und mir deinen einzigen Sohn nicht vorenthalten hast,
\bibleverse{17} will ich dich reichlich segnen und deine
Nachkommenschaft überaus zahlreich machen wie die Sterne am Himmel und
wie den Sand am Gestade des Meeres; und deine Nachkommen sollen die Tore
ihrer Feinde besitzen, \bibleverse{18} und in deiner\textless sup
title=``oder: durch deine''\textgreater✲ Nachkommenschaft sollen alle
Völker der Erde gesegnet werden zum Lohn dafür, daß du meiner
Aufforderung nachgekommen bist!« \bibleverse{19} Darauf kehrte Abraham
zu seinen Knechten zurück; und sie machten sich auf den Weg und begaben
sich miteinander nach Beerseba; dort nahm Abraham seinen dauernden
Wohnsitz.

\hypertarget{p-die-nachkommen-nahors-des-in-haran-seuxdfhaften-bruders-von-abraham}{%
\paragraph{p) Die Nachkommen Nahors, des in Haran seßhaften Bruders von
Abraham}\label{p-die-nachkommen-nahors-des-in-haran-seuxdfhaften-bruders-von-abraham}}

\bibleverse{20} Nach diesen Begebenheiten wurde dem Abraham gemeldet:
»Auch Milka hat deinem Bruder Nahor Söhne geboren, \bibleverse{21}
nämlich seinen Erstgeborenen Uz und dessen Bruder Bus und Kemuel, den
Vater von Aram, \bibleverse{22} und Kesed sowie Haso, Pildas, Jidlaph
und Bethuel.« \bibleverse{23} - Bethuel aber war der Vater der Rebekka.
-- Diese acht Söhne gebar Milka dem Nahor, dem Bruder Abrahams.
\bibleverse{24} Auch sein Nebenweib namens Rehuma hatte Söhne geboren,
nämlich Tebah und Gaham, Thahas und Maacha.

\hypertarget{q-saras-tod-und-betrauerung-abrahams-verhandlung-mit-den-hethitern-zur-erwerbung-eines-erbbegruxe4bnisses-bei-hebron-saras-bestattung}{%
\paragraph{q) Saras Tod und Betrauerung; Abrahams Verhandlung mit den
Hethitern zur Erwerbung eines Erbbegräbnisses bei Hebron; Saras
Bestattung}\label{q-saras-tod-und-betrauerung-abrahams-verhandlung-mit-den-hethitern-zur-erwerbung-eines-erbbegruxe4bnisses-bei-hebron-saras-bestattung}}

\hypertarget{section-22}{%
\section{23}\label{section-22}}

\bibleverse{1} Als Sara nun ihr Leben auf hundertundsiebenundzwanzig
Jahre gebracht hatte, \bibleverse{2} starb sie in Kirjath-Arba, das ist
Hebron, im Lande Kanaan. Da ging Abraham hinein, um Sara zu beklagen und
zu beweinen. \bibleverse{3} Hierauf stand Abraham von der Seite seiner
Verstorbenen auf und verhandelte mit den Hethitern so: \bibleverse{4}
»Ich bin (nur) ein Fremdling und Beisasse\textless sup title=``d.h. Gast
ohne Grundbesitz''\textgreater✲ hier bei euch: überlaßt mir doch ein
Erbbegräbnis bei euch, damit ich meine Tote, die in meinem Hause liegt,
begraben kann!« \bibleverse{5} Die Hethiter gaben dem Abraham folgende
Antwort: \bibleverse{6} »Höre uns an, Herr! Du lebst hier als ein
Gottesfürst unter uns: begrabe deine Tote in dem besten von unsern
Gräbern: keiner von uns wird dir seine Grabstätte zur Bestattung deiner
Toten versagen.« \bibleverse{7} Da erhob sich Abraham, verneigte sich
tief vor den Bewohnern des Landes, den Hethitern, \bibleverse{8} und
sagte weiter zu ihnen: »Wenn ihr damit einverstanden seid, daß ich meine
Tote, die in meinem Hause liegt, hier begrabe, so erweist mir die Liebe
und legt ein gutes Wort für mich bei Ephron, dem Sohne Zohars, ein,
\bibleverse{9} daß er mir die Höhle in der Machpela überläßt, die ihm
gehört und am Ende seines Feldes liegt; für den vollen Wert möge er sie
mir zu einem Erbbegräbnis hier in eurer Mitte überlassen!«
\bibleverse{10} Nun saß Ephron mitten unter den Hethitern und gab dem
Abraham vor den versammelten Hethitern im Beisein aller, die ins Tor
seiner Stadt gekommen waren, folgende Antwort: \bibleverse{11} »Nicht
doch, Herr! Höre mich an! Das Feld\textless sup title=``oder:
Grundstück''\textgreater✲ schenke ich dir; auch die Höhle darauf schenke
ich dir; im Beisein meiner Volksgenossen schenke ich sie dir: begrabe
nur deine Tote!« \bibleverse{12} Da verneigte sich Abraham tief vor den
Bewohnern des Landes \bibleverse{13} und sagte dann zu Ephron, im
Beisein aller Bewohner des Landes: »O doch! Wenn du mich nur anhören
wolltest! Ich zahle dir den Preis für das Grundstück: nimm ihn von mir
an, so will ich meine Tote dort begraben!« \bibleverse{14} Darauf
antwortete Ephron dem Abraham: \bibleverse{15} »Höre mich doch an, Herr!
Ein Stück Land im Wert von vierhundert Schekel Silber -- was will das
zwischen mir und dir besagen?! Begrabe nur deine Tote!« \bibleverse{16}
Abraham nahm die Forderung Ephrons an und wog ihm den Kaufpreis dar, den
Betrag, welchen jener im Beisein der Hethiter gefordert hatte, nämlich
vierhundert Schekel Silber, nach der beim Kauf und Verkauf üblichen
Währung. \bibleverse{17} So wurde das Grundstück Ephrons, das in der
Machpela östlich von Mamre lag, das Feld samt der Höhle darauf nebst
allen Bäumen, die auf dem Grundstück in seinem ganzen Umfang ringsum
standen, \bibleverse{18} dem Abraham rechtskräftig als Eigentum
abgetreten im Beisein der Hethiter, so viele ihrer ins Tor seiner Stadt
gekommen waren. \bibleverse{19} Hierauf begrub Abraham seine Frau Sara
in der Höhle auf dem Grundstück in der Machpela östlich von Mamre, das
ist Hebron, im Lande Kanaan. \bibleverse{20} So wurde das Grundstück
samt der Höhle darauf dem Abraham als Erbbegräbnis von den Hethitern
rechtskräftig überlassen.

\hypertarget{r-elieser-wirbt-in-haran-fuxfcr-isaak-um-rebekka}{%
\paragraph{r) Elieser wirbt in Haran für Isaak um
Rebekka}\label{r-elieser-wirbt-in-haran-fuxfcr-isaak-um-rebekka}}

\hypertarget{aa-abrahams-auftrag-an-seinen-knecht-zur-brautfahrt-nach-haran}{%
\subparagraph{aa) Abrahams Auftrag an seinen Knecht zur Brautfahrt nach
Haran}\label{aa-abrahams-auftrag-an-seinen-knecht-zur-brautfahrt-nach-haran}}

\hypertarget{section-23}{%
\section{24}\label{section-23}}

\bibleverse{1} Als nun Abraham alt und hochbetagt geworden war und der
HERR ihn in allem gesegnet hatte, \bibleverse{2} sagte Abraham zu dem
ältesten Knechte seines Hauses, der seinen gesamten Besitz zu verwalten
hatte: »Lege deine Hand unter meine Hüfte: \bibleverse{3} ich will dir
beim HERRN, dem Gott des Himmels und dem Gott der Erde, einen Eid
abnehmen, daß du für meinen Sohn keine Frau aus den Töchtern der
Kanaanäer nehmen willst, unter denen ich hier wohne; \bibleverse{4}
nein, du sollst in mein Vaterland und zu meiner Verwandtschaft gehen und
dort um eine Frau für meinen Sohn Isaak werben!« \bibleverse{5} Da
antwortete ihm der Knecht: »Vielleicht wird das Weib mir in dieses Land
nicht folgen wollen; soll ich alsdann deinen Sohn wieder in das Land
zurückführen, aus dem du ausgewandert bist?« \bibleverse{6} Abraham
antwortete ihm: »Hüte dich wohl, meinen Sohn dorthin zurückzuführen!
\bibleverse{7} Der HERR, der Gott des Himmels, der mich aus meines
Vaters Hause und aus meinem Heimatlande weggeführt und der mir zugesagt
und mir zugeschworen hat: ›Deinen Nachkommen will ich dieses Land geben‹
-- der wird seinen Engel vor dir her senden, so daß du von dort eine
Frau für meinen Sohn gewinnst. \bibleverse{8} Wenn das Weib dir aber
nicht folgen will, so sollst du von diesem mir geleisteten Eide
entbunden sein; nur darfst du meinen Sohn nicht dorthin zurückführen!«
\bibleverse{9} Da legte der Knecht seine Hand seinem Herrn Abraham unter
die Hüfte und leistete ihm in dieser Sache den verlangten Eid.

\hypertarget{bb-die-brautfahrt-und-die-brautschau}{%
\subparagraph{bb) Die Brautfahrt und die
Brautschau}\label{bb-die-brautfahrt-und-die-brautschau}}

\bibleverse{10} Hierauf nahm der Knecht zehn Kamele von den Kamelen
seines Herrn und allerlei Kostbarkeiten seines Herrn zu sich, machte
sich auf den Weg und zog nach Mesopotamien nach der Stadt Nahors
(Haran). \bibleverse{11} Dort ließ er die Kamele draußen vor der Stadt
bei dem Wasserbrunnen sich lagern zur Abendzeit, zu der Zeit, wo die
Frauen herauszukommen pflegen, um Wasser zu holen. \bibleverse{12} Dann
betete er: »O HERR, du Gott meines Herrn Abraham! Laß es mir doch heute
glücken und erweise meinem Herrn Abraham Gnade! \bibleverse{13} Siehe,
ich stehe jetzt hier bei der Quelle, und die Töchter der Stadtbewohner
werden herauskommen, um Wasser zu holen. \bibleverse{14} Wenn ich nun zu
einem Mädchen sage: ›Neige, bitte, deinen Krug, damit ich trinke!‹ und
sie mir dann antwortet: ›Trinke! Und auch deinen Kamelen will ich zu
trinken geben!‹, so möge diese es sein, die du für deinen Knecht Isaak
bestimmt hast; und daran will ich erkennen, daß du meinem Herrn Gnade
erwiesen hast.«

\bibleverse{15} Er hatte noch nicht zu Ende geredet, da kam schon
Rebekka\textless sup title=``d.h. die Fesselnde,
Anziehende''\textgreater✲ heraus, die Tochter Bethuels, der ein Sohn der
Milka, der Frau Nahors, des Bruders Abrahams, war; sie trug ihren Krug
auf der Schulter. \bibleverse{16} Das Mädchen war von großer Schönheit
und noch unverheiratet, eine Jungfrau; sie stieg zur Quelle hinab,
füllte ihren Krug und kam wieder herauf. \bibleverse{17} Da eilte der
Knecht auf sie zu und sagte: »Laß mich doch ein wenig Wasser aus deinem
Kruge trinken!« \bibleverse{18} Sie antwortete: »Trinke, Herr!« und ließ
sogleich ihren Krug (von der Schulter) auf ihre Hand herab und ließ ihn
trinken. \bibleverse{19} Als sie aber seinen Durst gestillt hatte, sagte
sie: »Auch für deine Kamele will ich Wasser schöpfen, bis sie sich satt
getrunken haben.« \bibleverse{20} Mit diesen Worten goß sie ihren Krug
eilends in die Tränkrinne aus, lief dann nochmals zum Brunnen, um Wasser
zu schöpfen, und schöpfte so für alle seine Kamele, \bibleverse{21}
während jener ihr verwundert zusah, ohne jedoch ein Wort zu sagen, um zu
erkennen, ob der HERR Glück zu seiner Reise gegeben habe oder nicht.

\hypertarget{cc-die-einkehr-in-das-haus-der-brauteltern}{%
\subparagraph{cc) Die Einkehr in das Haus der
Brauteltern}\label{cc-die-einkehr-in-das-haus-der-brauteltern}}

\bibleverse{22} Als nun die Kamele sich satt getrunken hatten, nahm der
Mann einen goldenen Nasenring, einen halben Schekel schwer, und zwei
Spangen für ihre Arme, zehn Schekel Goldes schwer, \bibleverse{23} und
fragte sie: »Wessen Tochter bist du? Teile es mir doch mit! Ist wohl im
Hause deines Vaters Platz für uns zum Übernachten?« \bibleverse{24} Sie
antwortete ihm: »Ich bin die Tochter Bethuels, des Sohnes der Milka, den
sie dem Nahor geboren hat.«\textless sup title=``vgl.
22,23''\textgreater✲ \bibleverse{25} Dann fuhr sie fort: »Sowohl Stroh
als auch Futter haben wir in Menge und auch Platz zum Übernachten.«
\bibleverse{26} Da verneigte sich der Mann, warf sich vor dem HERRN
nieder \bibleverse{27} und rief aus: »Gepriesen sei der HERR, der Gott
meines Herrn Abraham, der seine Güte und Treue meinem Herrn nicht
entzogen hat! Gerades Weges zum Hause des Verwandten\textless sup
title=``=~des Bruders''\textgreater✲ meines Herrn hat mich der Ewige
geführt!«

\bibleverse{28} Das Mädchen aber war unterdessen hingelaufen und hatte
im Hause ihrer Mutter alles erzählt, was sich zugetragen hatte.
\bibleverse{29} Nun hatte Rebekka einen Bruder namens Laban; dieser
eilte zu dem Manne hinaus an die Quelle. \bibleverse{30} Sobald er
nämlich den Nasenring und die Spangen an den Armen seiner Schwester
erblickt und seine Schwester Rebekka hatte erzählen hören, was der Mann
zu ihr gesagt habe, ging er zu dem Manne hinaus, der immer noch bei den
Kamelen an der Quelle stand. \bibleverse{31} Er sagte nun zu ihm: »Komm
in mein Haus, du Gesegneter des HERRN! Warum stehst du hier draußen? Ich
habe das Haus schon aufräumen lassen und Platz für die Kamele
geschafft!« \bibleverse{32} So kam denn der Mann in das Haus; dort
zäumte Laban die Kamele ab, gab ihnen Stroh und Futter und brachte ihm
sowie den Leuten, die bei ihm waren, Wasser zum Waschen der Füße.
\bibleverse{33} Als man ihm aber zu essen vorsetzte, sagte er: »Ich
werde nicht eher essen, als bis ich meine Sache\textless sup
title=``oder: mein Anliegen''\textgreater✲ vorgetragen habe.« Jener
erwiderte: »So rede!«

\hypertarget{dd-die-brautwerbung}{%
\subparagraph{dd) Die Brautwerbung}\label{dd-die-brautwerbung}}

\bibleverse{34} Da berichtete er: »Ich bin ein Knecht Abrahams.
\bibleverse{35} Gott der HERR hat meinen Herrn außerordentlich gesegnet,
so daß er reich geworden ist; denn er hat ihm Kleinvieh und Rinder,
Silber und Gold, Knechte und Mägde, Kamele und Esel gegeben.
\bibleverse{36} Dazu hat Sara, die Frau meines Herrn, ihm noch in ihrem
Alter einen Sohn geboren; dem hat er alles übergeben, was er besitzt.
\bibleverse{37} Nun hat mein Herr mir folgenden Eid abgenommen: ›Du
darfst meinem Sohne keine Frau aus den Töchtern der Kanaanäer nehmen, in
deren Lande ich wohne, \bibleverse{38} sondern sollst zu meines Vaters
Hause und zu meiner Verwandtschaft ziehen, um für meinen Sohn dort eine
Frau zu nehmen.‹ \bibleverse{39} Ich entgegnete meinem Herrn:
›Vielleicht wird das Weib mir nicht folgen wollen.‹ \bibleverse{40} Da
erwiderte er mir: ›Gott der HERR, vor dessen Angesicht ich gewandelt
bin, wird seinen Engel mit dir senden und dir Glück zu deiner Reise
geben, damit du für meinen Sohn eine Frau aus meiner Verwandtschaft, und
zwar aus dem Hause meines Vaters, gewinnst. \bibleverse{41} Dann sollst
du von dem mir geleisteten Eid entbunden sein, wenn du zu meiner
Verwandtschaft kommst und man sie dir dort nicht geben will -- dann bist
du von dem mir geleisteten Eid entbunden.‹ \bibleverse{42} Nun bin ich
heute zu der Quelle gekommen und habe gebetet: ›O HERR, du Gott meines
Herrn Abraham! wenn du doch Glück zu der Reise geben möchtest, auf der
ich mich jetzt befinde! \bibleverse{43} Siehe, ich stehe jetzt hier an
der Quelle. Laß es doch geschehen, daß das Mädchen, das herauskommt, um
Wasser zu holen, und zu der ich sage: Gib mir, bitte, ein wenig Wasser
aus deinem Kruge zu trinken!, \bibleverse{44} daß die mir dann
antwortet: Trinke du selbst, und auch für deine Kamele will ich Wasser
schöpfen! -- so möge diese es sein, die Gott der HERR dem Sohne meines
Herrn zur Frau bestimmt hat!‹ \bibleverse{45} Ich hatte dieses bei mir
noch nicht zu Ende geredet, da kam auch schon Rebekka aus dem Orte
heraus mit ihrem Kruge auf ihrer Schulter; sie stieg zur Quelle hinab
und schöpfte Wasser. Da bat ich sie: ›Gib mir, bitte, zu trinken!‹
\bibleverse{46} Sogleich ließ sie ihren Krug (von der Schulter) herab
und sagte: ›Trinke, und auch deinen Kamelen will ich zu trinken geben!‹
Da trank ich, und sie tränkte dann auch die Kamele. \bibleverse{47}
Hierauf fragte ich sie, wessen Tochter sie sei, und sie antwortete: ›Die
Tochter Bethuels, des Sohnes Nahors, den Milka ihm geboren hat.‹ Da
legte ich ihr den Ring an die Nase und die Spangen an ihre Arme;
\bibleverse{48} dann verneigte ich mich vor Gott dem HERRN, warf mich
vor ihm nieder und pries den HERRN, den Gott meines Herrn Abraham, der
mich den rechten Weg geführt hatte, um die Tochter des Verwandten✲
meines Herrn für seinen Sohn zu gewinnen. \bibleverse{49} Und nun, wenn
ihr meinem Herrn Liebe und Treue erweisen wollt, so sagt es mir! Wo
nicht, so sagt es mir auch, damit ich mich zur Rechten oder zur Linken
wende!«

\hypertarget{ee-die-brautverlobung-und-die-brautentlassung}{%
\subparagraph{ee) Die Brautverlobung und die
Brautentlassung}\label{ee-die-brautverlobung-und-die-brautentlassung}}

\bibleverse{50} Da antworteten Laban und Bethuel: »Von Gott dem HERRN
ist dies ausgegangen\textless sup title=``=~so gefügt''\textgreater✲:
wir können dir nichts dazu sagen, weder ja noch nein. \bibleverse{51}
Rebekka steht dir zur Verfügung: nimm sie und ziehe hin, damit sie die
Frau des Sohnes deines Herrn wird, wie Gott der HERR es bestimmt hat!«
\bibleverse{52} Sobald der Knecht Abrahams diese ihre Worte gehört
hatte, verbeugte er sich vor Gott dem HERRN bis auf die Erde;
\bibleverse{53} dann holte er silberne und goldene Geräte\textless sup
title=``oder: Geschmeide''\textgreater✲ und Gewänder hervor und schenkte
sie der Rebekka; auch ihrem Bruder und ihrer Mutter schenkte er
Kostbarkeiten. \bibleverse{54} Dann aßen und tranken sie, er und die
Leute, die bei ihm waren, und blieben über Nacht da. Am andern Morgen
aber, als sie aufgestanden waren, sagte er: »Laßt mich nun zu meinem
Herrn ziehen!« \bibleverse{55} Da erwiderten ihm ihr Bruder und ihre
Mutter: »Laß doch das Mädchen noch einige Zeit oder (wenigstens) zehn
Tage bei uns bleiben, dann magst du aufbrechen.« \bibleverse{56} Doch er
entgegnete ihnen: »Haltet mich nicht auf! Da Gott der HERR Glück zu
meiner Reise gegeben hat, so laßt mich nun ziehen, damit ich zu meinem
Herrn zurückkehre.« \bibleverse{57} Da sagten sie: »Wir wollen das
Mädchen rufen und sie selbst entscheiden lassen.« \bibleverse{58} So
riefen sie denn Rebekka und fragten sie: »Willst du mit diesem Manne
ziehen?« Sie antwortete: »Ja, ich will mit ihm ziehen.« \bibleverse{59}
Da ließen sie ihre Schwester Rebekka samt ihrer Amme und ebenso den
Knecht Abrahams samt seinen Leuten ziehen \bibleverse{60} und segneten
Rebekka mit den Worten: »Du, unsere Schwester, werde die Mutter von
tausendmal Tausenden, und deine Nachkommen mögen die Tore ihrer Feinde
besetzen\textless sup title=``oder: in Besitz nehmen''\textgreater✲!«
\bibleverse{61} So machte sich denn Rebekka mit ihren Dienerinnen auf
den Weg; sie setzten sich auf die Kamele und zogen hinter dem Manne her:
der Knecht hatte Rebekka übernommen und zog von dannen.

\hypertarget{ff-die-ankunft-der-braut-bei-dem-bruxe4utigam}{%
\subparagraph{ff) Die Ankunft der Braut bei dem
Bräutigam}\label{ff-die-ankunft-der-braut-bei-dem-bruxe4utigam}}

\bibleverse{62} Isaak aber war gerade auf der Heimkehr von einem Gang
nach dem ›Brunnen des Lebendigen, der mich sieht‹✲; er wohnte nämlich im
Südgau \bibleverse{63} und war gegen Abend aufs Feld hinausgegangen, um
mit seinen Gedanken allein zu sein. Als er nun aufblickte, sah er auf
einmal Kamele daherkommen. \bibleverse{64} Als nun auch Rebekka ihre
Augen aufschlug und den Isaak erblickte, ließ\textless sup title=``oder:
beugte''\textgreater✲ sie sich rasch vom Kamele herab \bibleverse{65}
und fragte den Knecht: »Wer ist der Mann dort, der uns auf dem Felde
entgegenkommt?« Der Knecht antwortete: »Das ist mein Herr!« Da nahm sie
den Schleier und verhüllte sich. \bibleverse{66} Der Knecht erzählte
dann dem Isaak alles, wie es ihm ergangen war. \bibleverse{67} Isaak
aber führte Rebekka in das Zelt seiner (verstorbenen) Mutter Sara und
nahm sie auf; sie wurde seine Frau, und er gewann sie lieb. So tröstete
sich Isaak nach dem Hingang\textless sup title=``oder: über den
Verlust''\textgreater✲ seiner Mutter.

\hypertarget{s-abrahams-zweite-ehe-und-letzte-taten-sein-tod-und-begruxe4bnis}{%
\paragraph{s) Abrahams zweite Ehe und letzte Taten; sein Tod und
Begräbnis}\label{s-abrahams-zweite-ehe-und-letzte-taten-sein-tod-und-begruxe4bnis}}

\hypertarget{section-24}{%
\section{25}\label{section-24}}

\bibleverse{1} Abraham aber nahm nochmals eine Frau namens Ketura;
\bibleverse{2} die gebar ihm Simran und Joksan, Medan und Midian, Jisbak
und Suah. \bibleverse{3} Joksan wurde dann der Vater Sebas und Dedans;
und die Söhne Dedans waren die Assuriter und Letusiter und die
Leummiter. \bibleverse{4} Die Söhne Midians waren Epha und Epher,
Henoch, Abida und Eldaba. Alle diese sind Nachkommen der Ketura.
\bibleverse{5} Abraham aber übergab seinen gesamten Besitz dem Isaak;
\bibleverse{6} dagegen den Söhnen, die er von den Nebenweibern hatte,
gab er nur Geschenke und ließ sie noch bei seinen Lebzeiten von seinem
Sohn Isaak hinweg ostwärts in das Ostland ziehen. \bibleverse{7} Dies
aber ist die Zeit✲ der Lebensjahre, die Abraham gelebt hat: 175~Jahre;
\bibleverse{8} da verschied und starb er in gesegnetem Alter, hochbetagt
und lebenssatt, und wurde zu seinen Stammesgenossen versammelt.
\bibleverse{9} Seine Söhne Isaak und Ismael begruben ihn in der Höhle
der Machpela auf dem Felde des Hethiters Ephron, des Sohnes Zohars, das
östlich von Mamre lag, \bibleverse{10} auf dem Felde, das Abraham von
den Hethitern käuflich erworben hatte; dort sind Abraham und seine Frau
Sara begraben worden. \bibleverse{11} Nach Abrahams Tode aber segnete
Gott dessen Sohn Isaak; dieser wohnte bei dem ›Brunnen des Lebendigen,
der mich sieht‹✲.

\hypertarget{t-die-nachkommen-ismaels}{%
\paragraph{t) Die Nachkommen Ismaels}\label{t-die-nachkommen-ismaels}}

\bibleverse{12} Dies ist der Stammbaum\textless sup title=``=~die
Nachkommenschaft''\textgreater✲ Ismaels, des Sohnes Abrahams, den die
Ägypterin Hagar, die Leibmagd Saras, dem Abraham geboren hat;
\bibleverse{13} dies sind die Namen der Söhne Ismaels nach ihrer
Geburtsfolge: Der erstgeborene Sohn Ismaels war Nebajoth, sodann Kedar,
Abdeel und Mibsam, \bibleverse{14} Misma, Duma und Massa,
\bibleverse{15} Hadad und Thema, Jetur, Naphis und Kedma.
\bibleverse{16} Dies waren die Söhne Ismaels und dies ihre Namen nach
ihren Niederlassungen und ihren Zeltlagern; zwölf Fürsten entsprechend
ihren Völkerschaften. \bibleverse{17} Und dies war die Lebensdauer
Ismaels: 137~Jahre; da verschied er und starb und wurde zu seinen
Stammesgenossen versammelt. \bibleverse{18} Sie hatten aber ihre
Wohnsitze von Hawila an bis nach Sur, das östlich von Ägypten liegt, in
der Richtung nach Assyrien hin: er hatte sich ostwärts von allen seinen
Brüdern niedergelassen.

\hypertarget{die-geschichte-isaaks-2519-2746}{%
\subsubsection{2. Die Geschichte Isaaks
(25,19-27,46)}\label{die-geschichte-isaaks-2519-2746}}

\hypertarget{a-die-geburt-und-jugend-esaus-und-jakobs-der-zwillingssuxf6hne-isaaks-esau-verkauft-sein-erstgeburtsrecht}{%
\paragraph{a) Die Geburt und Jugend Esaus und Jakobs, der Zwillingssöhne
Isaaks; Esau verkauft sein
Erstgeburtsrecht}\label{a-die-geburt-und-jugend-esaus-und-jakobs-der-zwillingssuxf6hne-isaaks-esau-verkauft-sein-erstgeburtsrecht}}

\bibleverse{19} Dies ist der Stammbaum\textless sup title=``=~die
Familiengeschichte''\textgreater✲ Isaaks, des Sohnes Abrahams: Abraham
war der Vater Isaaks; \bibleverse{20} und Isaak war vierzig Jahre alt,
als er Rebekka, die Tochter des Aramäers✲ Bethuel aus Nord-Mesopotamien,
die Schwester des Aramäers Laban, zur Frau nahm. \bibleverse{21} Und
Isaak betete zum HERRN für seine Frau, denn sie hatte keine Kinder; da
ließ der HERR sich von ihm\textless sup title=``oder: für
ihn''\textgreater✲ erbitten, so daß seine Frau Rebekka guter Hoffnung
wurde. \bibleverse{22} Als aber die beiden Kinder sich in ihrem Schoße
stießen, sagte sie: »Wenn es so steht, wozu bin ich da in diesen Zustand
gekommen?« Und sie ging hin, um den HERRN zu befragen. \bibleverse{23}
Da antwortete ihr der HERR: »Zwei Völker sind in deinem Mutterschoße,
und zwei Volksstämme werden sich von deinem Leibe ausscheiden; der eine
Stamm wird stärker sein als der andere, und der Ältere wird dem Jüngeren
dienen.« \bibleverse{24} Als nun die Zeit ihrer Niederkunft da war,
stellte es sich wirklich heraus, daß Zwillinge in ihrem Leibe waren.
\bibleverse{25} Der erste, der zum Vorschein kam, war rotbraun, rauh am
ganzen Leibe wie ein haariger Mantel; darum nannte man ihn
Esau\textless sup title=``d.h. behaart, der Rauhe''\textgreater✲.
\bibleverse{26} Hierauf kam sein Bruder zum Vorschein, der mit seiner
Hand die Ferse Esaus gefaßt hielt; darum nannte man ihn
Jakob\textless sup title=``d.h. Fersenhalter, Überlister''\textgreater✲.
Isaak aber war bei ihrer Geburt sechzig Jahre alt.

\bibleverse{27} Als nun die Knaben heranwuchsen, wurde Esau ein
tüchtiger Jäger, ein Mann des freien Feldes\textless sup title=``d.h.
der sich auf dem Felde umhertrieb''\textgreater✲; Jakob dagegen war ein
stiller Mann, der in den Zelten blieb. \bibleverse{28} Isaak hatte den
Esau lieber, weil er gern Wildbret aß; Rebekka aber hatte Jakob lieber.

\hypertarget{jakob-kauft-von-esau-das-erstgeburtsrecht}{%
\paragraph{Jakob kauft von Esau das
Erstgeburtsrecht}\label{jakob-kauft-von-esau-das-erstgeburtsrecht}}

\bibleverse{29} Nun hatte Jakob eines Tages ein Gericht gekocht, als
Esau ganz erschöpft vom Felde heimkam. \bibleverse{30} Da sagte Esau zu
Jakob: »Laß mich doch schnell essen von dem Roten, dem roten Gericht da,
denn ich bin ganz erschöpft!« Darum gab man ihm den Namen
Edom\textless sup title=``d.h. der Rote''\textgreater✲. \bibleverse{31}
Aber Jakob antwortete: »Verkaufe mir zuvor\textless sup title=``oder:
heute''\textgreater✲ dein Erstgeburtsrecht!« \bibleverse{32} Da
erwiderte Esau: »Ach, ich muß ja doch (bald) sterben: wozu nützt mir da
das Erstgeburtsrecht?« \bibleverse{33} Jakob aber sagte: »Schwöre mir
zuvor\textless sup title=``oder: heute''\textgreater✲!« Da schwur er ihm
und verkaufte so dem Jakob sein Erstgeburtsrecht. \bibleverse{34}
Hierauf gab Jakob dem Esau Brot und von dem Linsengericht. Als er dann
gegessen und getrunken hatte, stand er auf und ging seines Weges. So gab
Esau sein Erstgeburtsrecht geringschätzig preis\textless sup
title=``Hebr 12,16''\textgreater✲.

\hypertarget{b-isaaks-freuden-und-leiden-in-gerar-und-im-suxfcdlichen-paluxe4stina}{%
\paragraph{b) Isaaks Freuden und Leiden in Gerar und im südlichen
Palästina}\label{b-isaaks-freuden-und-leiden-in-gerar-und-im-suxfcdlichen-paluxe4stina}}

\hypertarget{aa-isaak-zieht-bei-einer-hungersnot-unter-guxf6ttlicher-segensverheiuxdfung-nach-gerar}{%
\subparagraph{aa) Isaak zieht bei einer Hungersnot unter göttlicher
Segensverheißung nach
Gerar}\label{aa-isaak-zieht-bei-einer-hungersnot-unter-guxf6ttlicher-segensverheiuxdfung-nach-gerar}}

\hypertarget{section-25}{%
\section{26}\label{section-25}}

\bibleverse{1} Es kam aber eine Hungersnot über das Land, wie schon
früher einmal eine zur Zeit Abrahams geherrscht hatte; darum begab sich
Isaak nach Gerar zu dem Philisterkönig Abimelech\textless sup
title=``vgl. 20,2''\textgreater✲. \bibleverse{2} Denn der HERR war ihm
erschienen und hatte zu ihm gesagt: »Ziehe nicht nach Ägypten hinab,
sondern nimm deinen Wohnsitz in dem Lande, das ich dir angeben werde!
\bibleverse{3} Bleibe als Fremdling in diesem Lande wohnen; ich will mit
dir sein und dich segnen; denn dir und deinen Nachkommen will ich alle
diese Länder geben und so den Eid erfüllen, den ich deinem Vater Abraham
geschworen habe: \bibleverse{4} ich will deine Nachkommen so zahlreich
werden lassen wie die Sterne am Himmel und will deinen Nachkommen alle
diese Länder geben; und in deiner Nachkommenschaft sollen alle Völker
der Erde gesegnet werden, \bibleverse{5} zum Lohn dafür, daß Abraham
meinen Weisungen gehorsam gewesen ist und meine Anordnungen beobachtet
hat, meine Gebote, meine Satzungen und meine Gesetze.« \bibleverse{6} So
blieb denn Isaak in Gerar wohnen.

\hypertarget{bb-rebekka-wird-vor-der-entehrung-durch-den-philisterkuxf6nig-abimelech-bewahrt}{%
\subparagraph{bb) Rebekka wird vor der Entehrung durch den
Philisterkönig Abimelech
bewahrt}\label{bb-rebekka-wird-vor-der-entehrung-durch-den-philisterkuxf6nig-abimelech-bewahrt}}

\bibleverse{7} Als nun die Bewohner des Ortes sich nach seiner Frau
erkundigten, sagte er: »Sie ist meine Schwester«; er scheute sich
nämlich zu sagen: »Sie ist meine Frau«; »denn«, dachte er, »es könnten
sonst die Leute des Ortes mich um Rebekkas willen ums Leben bringen; sie
ist ja von großer Schönheit«. \bibleverse{8} Als er sich nun längere
Zeit dort aufgehalten hatte, schaute der Philisterkönig Abimelech einmal
zum Fenster hinaus und sah, wie Isaak seine Frau Rebekka herzte.
\bibleverse{9} Da ließ Abimelech den Isaak rufen und sagte: »Sie ist ja
doch deine Frau! Wie hast du sie da für deine Schwester ausgeben
können?« Isaak antwortete ihm: »Ja, ich dachte, ich müßte sonst
ihretwegen sterben.« \bibleverse{10} Da erwiderte Abimelech: »Was hast
du uns da angetan! Wie leicht hätte es geschehen können, daß einer aus
dem Volke hier deiner Frau Gewalt angetan hätte! Dann würdest du eine
Verschuldung über uns gebracht haben.« \bibleverse{11} Hierauf gebot
Abimelech dem ganzen Volke: »Wer sich an diesem Manne oder an seiner
Frau vergreift, soll unfehlbar mit dem Tode bestraft werden!«

\hypertarget{cc-isaaks-wachsender-reichtum-sein-wegzug-aus-gerar-brunnenstreitigkeiten-gottes-ermutigung}{%
\subparagraph{cc) Isaaks wachsender Reichtum; sein Wegzug aus Gerar;
Brunnenstreitigkeiten; Gottes
Ermutigung}\label{cc-isaaks-wachsender-reichtum-sein-wegzug-aus-gerar-brunnenstreitigkeiten-gottes-ermutigung}}

\bibleverse{12} Isaak säte dann in jenem Lande und erntete in jenem
Jahre das Hundertfache; denn der HERR segnete ihn. \bibleverse{13} So
wurde er denn ein reicher Mann und wurde immer reicher, bis er über die
Maßen reich war; \bibleverse{14} denn er besaß Herden von Kleinvieh und
Herden von Rindern und ein zahlreiches Gesinde, so daß die Philister
neidisch auf ihn wurden. \bibleverse{15} Daher verschütteten die
Philister alle Brunnen, welche die Knechte seines Vaters einst bei
Lebzeiten seines Vaters Abraham gegraben hatten, und füllten sie mit
Schutt an. \bibleverse{16} Da sagte Abimelech zu Isaak: »Verlaß unser
Land, denn du bist uns zu stark geworden.« \bibleverse{17} Da zog Isaak
von dort weg, schlug sein Lager im Tale von Gerar auf und nahm dort
seinen Wohnsitz.

\bibleverse{18} Hierauf ließ Isaak die Wasserbrunnen, welche man bei
Lebzeiten seines Vaters Abraham gegraben und die die Philister nach dem
Tode Abrahams verschüttet hatten, wieder aufgraben und legte ihnen
dieselben Namen bei, die sein Vater ihnen gegeben hatte. \bibleverse{19}
Auch gruben die Leute Isaaks im Talgrunde nach und fanden dort einen
Brunnen mit Quellwasser. \bibleverse{20} Aber die Hirten von Gerar
fingen mit den Hirten Isaaks Streit an, indem sie behaupteten, das
Wasser gehöre ihnen. Da nannte er den Brunnen ›Zankbrunnen‹, weil sie
sich dort mit ihm gezankt hatten. \bibleverse{21} Dann gruben sie einen
andern Brunnen, gerieten aber auch über diesen in Streit; daher nannte
er ihn ›Anfeindung‹. \bibleverse{22} Darauf zog er von dort weiter und
grub wieder einen Brunnen, über den dann kein Streit mehr entstand;
daher nannte er ihn ›Freier Raum‹, indem er sagte: »Jetzt hat der HERR
uns freien Raum geschafft, so daß wir uns im Lande ausbreiten können.«
\bibleverse{23} Von dort zog er dann nach Beerseba hinauf.

\bibleverse{24} Da erschien ihm der HERR in jener Nacht und sprach: »Ich
bin der Gott deines Vaters Abraham. Fürchte dich nicht, denn ich bin mit
dir; ich will dich segnen und deine Nachkommenschaft zahlreich werden
lassen um meines Knechtes Abraham willen.« \bibleverse{25} Da baute er
dort einen Altar, rief den Namen des HERRN an und schlug dort sein Zelt
auf; hierauf gruben die Knechte Isaaks dort nach einem Brunnen.

\hypertarget{dd-isaaks-vertrag-mit-abimelech-in-beerseba-auffindung-des-schwurbrunnens}{%
\subparagraph{dd) Isaaks Vertrag mit Abimelech in Beerseba; Auffindung
des
Schwurbrunnens}\label{dd-isaaks-vertrag-mit-abimelech-in-beerseba-auffindung-des-schwurbrunnens}}

\bibleverse{26} Da kam Abimelech mit seinem Freunde Ahussath und seinem
Heerführer Pichol✲ aus Gerar zu ihm. \bibleverse{27} Isaak fragte sie:
»Warum kommt ihr zu mir, da ihr doch feindlich gegen mich gesinnt seid
und mich aus eurem Lande vertrieben habt?« \bibleverse{28} Da
antworteten sie: »Wir haben klar erkannt, daß der HERR mit dir ist;
darum haben wir gedacht, es solle doch ein eidliches Abkommen zwischen
uns beiden, zwischen uns und dir, zustande kommen, und wir wollen einen
Vertrag mit dir schließen, \bibleverse{29} daß du uns nichts zuleide tun
willst, wie auch wir dir keinen Schaden zugefügt, sondern dir nur Gutes
erwiesen und dich in Frieden haben ziehen lassen: du bist nun einmal der
Gesegnete des HERRN!« \bibleverse{30} Da richtete er ihnen ein Gastmahl
aus, und sie aßen und tranken. \bibleverse{31} Am andern Morgen in der
Frühe aber leisteten sie einander den Schwur; dann ließ Isaak sie
ziehen, und sie schieden als Freunde von ihm.

\bibleverse{32} An demselben Tage kamen dann die Knechte Isaaks und
berichteten ihm von dem Brunnen, den sie gegraben hatten, mit den
Worten: »Wir haben Wasser gefunden!« \bibleverse{33} Da nannte er ihn
›Sibea‹ (Sebua =~Schwur, Eidvertrag). Daher heißt die Stadt dort
›Beerseba‹\textless sup title=``d.h. Schwurbrunnen; vgl.
21,31''\textgreater✲ bis auf den heutigen Tag.

\hypertarget{ee-esau-heiratet-zwei-hethiterinnen-gegen-den-willen-seiner-eltern}{%
\subparagraph{ee) Esau heiratet zwei Hethiterinnen gegen den Willen
seiner
Eltern}\label{ee-esau-heiratet-zwei-hethiterinnen-gegen-den-willen-seiner-eltern}}

\bibleverse{34} Als nun Esau vierzig Jahre alt war, heiratete er Judith,
die Tochter des Hethiters Beeri, und Basmath, die Tochter des Hethiters
Elon: \bibleverse{35} die waren ein Herzenskummer für Isaak und Rebekka.

\hypertarget{c-jakob-erlangt-durch-betrug-den-erstgeburtssegen}{%
\paragraph{c) Jakob erlangt durch Betrug den
Erstgeburtssegen}\label{c-jakob-erlangt-durch-betrug-den-erstgeburtssegen}}

\hypertarget{aa-isaak-bereitet-sich-zum-segnen-esaus-vor}{%
\subparagraph{aa) Isaak bereitet sich zum Segnen Esaus
vor}\label{aa-isaak-bereitet-sich-zum-segnen-esaus-vor}}

\hypertarget{section-26}{%
\section{27}\label{section-26}}

\bibleverse{1} Als aber Isaak alt geworden und sein Augenlicht erloschen
war, so daß er nicht mehr sehen konnte, berief er seinen älteren Sohn
Esau und sagte zu ihm: »Mein Sohn!« Er antwortete ihm: »Hier bin ich!«
\bibleverse{2} Jener fuhr fort: »Du siehst, ich bin alt geworden und
weiß nicht, wie bald ich sterben werde. \bibleverse{3} So nimm nun doch
deine Jagdgeräte, deinen Köcher und Bogen, und gehe aufs Feld hinaus und
erjage ein Stück Wild für mich; \bibleverse{4} dann bereite mir ein
schmackhaftes Gericht, wie ich es liebe, und bringe es mir herein, damit
ich esse und dich dann segne, bevor ich sterbe.«

\hypertarget{bb-rebekkas-betruxfcgliches-eingreifen}{%
\subparagraph{bb) Rebekkas betrügliches
Eingreifen}\label{bb-rebekkas-betruxfcgliches-eingreifen}}

\bibleverse{5} Rebekka hatte aber zugehört, als Isaak so zu seinem Sohne
Esau redete. Während nun Esau aufs Feld hinausging, um ein Stück Wild zu
erjagen und heimzubringen, \bibleverse{6} sagte Rebekka zu ihrem Sohne
Jakob: »Ich habe soeben gehört, wie dein Vater mit deinem Bruder Esau
geredet hat und zu ihm sagte: \bibleverse{7} ›Bringe mir doch ein Stück
Wild und bereite mir ein schmackhaftes Gericht, damit ich esse und dich
dann vor dem Angesicht des HERRN segne, bevor ich sterbe.‹
\bibleverse{8} So höre nun, mein Sohn, auf den Rat, den ich dir jetzt
gebe! \bibleverse{9} Gehe hin zur Herde und hole mir von dort zwei gute
Ziegenböckchen; die will ich dann für deinen Vater zu einem
schmackhaften Gericht zubereiten, wie er es liebt; \bibleverse{10} das
bringst du dann deinem Vater hinein, damit er es ißt und dich dann noch
vor seinem Tode segnet.« \bibleverse{11} Da erwiderte Jakob seiner
Mutter Rebekka: »Ja, aber mein Bruder Esau ist stark behaart, während
ich eine glatte Haut habe. \bibleverse{12} Vielleicht wird mein Vater
mich betasten: dann würde ich als Betrüger vor ihm dastehen und einen
Fluch statt des Segens über mich bringen.« \bibleverse{13} Aber seine
Mutter antwortete ihm: »Den Fluch, der dich treffen könnte, nehme ich
auf mich, mein Sohn! Folge du nur meinem Rat: geh hin und hole mir die
Böckchen!«

\bibleverse{14} Da ging er hin, holte die Böckchen und brachte sie
seiner Mutter; und diese bereitete davon ein schmackhaftes Gericht, wie
sein Vater es liebte. \bibleverse{15} Hierauf holte Rebekka die
Festtagskleider ihres älteren Sohnes Esau, die sich bei ihr in der
Wohnung befanden, und gab sie ihrem jüngeren Sohne Jakob zum Anziehen;
\bibleverse{16} die Felle der Ziegenböckchen aber legte sie ihm um die
Arme und um die glatten Stellen seines Halses; \bibleverse{17} dann gab
sie das schmackhafte Essen nebst dem Brot, das sie gebacken hatte, ihrem
Sohne Jakob in die Hand.

\hypertarget{cc-jakob-bringt-den-tuxe4uschungsplan-seiner-mutter-zur-ausfuxfchrung-und-erlangt-den-erstgeburtssegen}{%
\subparagraph{cc) Jakob bringt den Täuschungsplan seiner Mutter zur
Ausführung und erlangt den
Erstgeburtssegen}\label{cc-jakob-bringt-den-tuxe4uschungsplan-seiner-mutter-zur-ausfuxfchrung-und-erlangt-den-erstgeburtssegen}}

\bibleverse{18} So ging er denn zu seinem Vater hinein und sagte: »Mein
Vater!« Dieser antwortete: »Hier bin ich! Wer bist du, mein Sohn?«
\bibleverse{19} Jakob erwiderte seinem Vater: »Ich bin Esau, dein
erstgeborener Sohn; ich habe getan, wie du mir aufgetragen hast. Richte
dich nun auf, setze dich und iß von meinem Wildbret, damit du mich dann
segnest.« \bibleverse{20} Da fragte Isaak seinen Sohn: »Wie hast du denn
so schnell etwas gefunden, mein Sohn?« Er antwortete: »Ja, der HERR,
dein Gott, hat es mir entgegenlaufen lassen.« \bibleverse{21} Da sagte
Isaak zu Jakob: »Tritt doch näher heran, mein Sohn, damit ich dich
betaste, ob du wirklich mein Sohn Esau bist oder nicht!« \bibleverse{22}
Da trat Jakob nahe an seinen Vater Isaak heran, und als dieser ihn
betastet hatte, sagte er: »Die Stimme ist Jakobs Stimme, aber die Arme
sind Esaus Arme«; \bibleverse{23} und er erkannte ihn nicht, weil seine
Arme behaart waren wie die Arme seines Bruders Esau; so segnete er ihn
denn. \bibleverse{24} Er fragte nämlich: »Du bist doch wirklich mein
Sohn Esau?« Jener antwortete: »Ja, ich bin's.« \bibleverse{25} Da fuhr
er fort: »So reiche es mir her, damit ich von dem Wildbret meines Sohnes
esse und ich dich dann segne.« Da reichte er es ihm hin, und er aß; er
brachte ihm auch Wein, den er trank. \bibleverse{26} Hierauf sagte sein
Vater Isaak zu ihm: »Tritt nun nahe heran, mein Sohn, und küsse mich!«
\bibleverse{27} Da trat er heran und küßte ihn; dabei roch jener den
Geruch seiner Kleider und segnete ihn mit den Worten: »Ja, der Geruch
meines Sohnes ist wie der Geruch\textless sup title=``oder:
Duft''\textgreater✲ eines Feldes, das der HERR gesegnet hat.
\bibleverse{28} So gebe Gott dir denn vom Tau des Himmels und von den
Fruchtgefilden der Erde Überfluß sowohl an Korn als auch an Wein!
\bibleverse{29} Völker sollen dir dienen und Völkerschaften sich vor dir
beugen! Sei ein Herr über deine Brüder, und bücken sollen sich vor dir
die Söhne deiner Mutter! Wer dir flucht, der sei verflucht, und wer dich
segnet, der soll gesegnet sein!«

\hypertarget{dd-esaus-ruxfcckkehr-seine-klage-und-der-ihm-vom-vater-erteilte-segen}{%
\subparagraph{dd) Esaus Rückkehr, seine Klage und der ihm vom Vater
erteilte
Segen}\label{dd-esaus-ruxfcckkehr-seine-klage-und-der-ihm-vom-vater-erteilte-segen}}

\bibleverse{30} Als nun Isaak mit der Segnung Jakobs eben zu Ende war
und Jakob kaum von seinem Vater Isaak hinausgegangen war, da kam sein
Bruder Esau von seiner Jagd zurück. \bibleverse{31} Er bereitete
gleichfalls ein schmackhaftes Gericht, brachte es seinem Vater hinein
und sagte zu ihm: »Richte dich auf, mein Vater, und iß vom Wildbret
deines Sohnes, damit du mich dann segnest!« \bibleverse{32} Da fragte
ihn sein Vater Isaak: »Wer bist du?« Er antwortete: »Ich bin dein
erstgeborener Sohn Esau.« \bibleverse{33} Da erbebte✲ Isaak über alle
Maßen und sagte: »Wer ist denn der gewesen, der ein Stück Wild erjagt
und es mir gebracht hat? Ich habe von allem gegessen, ehe du kamst, und
habe ihn gesegnet; so wird er nun auch gesegnet bleiben.«
\bibleverse{34} Sobald Esau diese Worte seines Vaters vernahm, erhob er
ein überaus lautes und klägliches Geschrei und bat seinen Vater: »Segne
auch mich, mein Vater!« \bibleverse{35} Isaak aber antwortete: »Dein
Bruder ist mit List gekommen und hat den dir gebührenden Segen
vorweggenommen.« \bibleverse{36} Da sagte Esau: »Ja, er heißt mit Recht
Jakob\textless sup title=``d.h. Überlister; vgl. 25,26''\textgreater✲;
denn er hat mich nun schon zweimal überlistet: mein Erstgeburtsrecht hat
er mir genommen, und jetzt hat er mich auch um meinen Segen gebracht!«
Dann fragte er: »Hast du denn für mich keinen Segen zurückbehalten?«
\bibleverse{37} Da antwortete Isaak dem Esau mit den Worten: »Ich habe
ihn nun einmal zum Herrn über dich gesetzt und alle seine Brüder ihm zu
Knechten gegeben; mit Korn und Wein habe ich ihn versorgt! Was könnte
ich also nun noch für dich tun, mein Sohn?« \bibleverse{38} Da sagte
Esau zu seinem Vater: »Hast du denn nur den einen Segen, mein Vater?
Segne auch mich, mein Vater!« Und Esau begann laut zu weinen.
\bibleverse{39} Da antwortete ihm sein Vater Isaak mit den Worten: »Ach,
ohne fetten Erdboden wird dein Wohnsitz sein und ohne Tau vom Himmel
droben! \bibleverse{40} Mittels deines Schwertes mußt du leben, und
deinem Bruder sollst du dienstbar sein. Wenn du aber
rüttelst\textless sup title=``oder: dich anstrengst''\textgreater✲,
wirst du sein Joch dir vom Nacken abschütteln.«

\hypertarget{ee-esau-trachtet-seinem-bruder-nach-dem-leben-rebekkas-notbeschluuxdf}{%
\subparagraph{ee) Esau trachtet seinem Bruder nach dem Leben; Rebekkas
Notbeschluß}\label{ee-esau-trachtet-seinem-bruder-nach-dem-leben-rebekkas-notbeschluuxdf}}

\bibleverse{41} So wurde denn Esau dem Jakob feind wegen des Segens, den
sein Vater ihm erteilt hatte; und Esau dachte bei sich: »Bald werden die
Tage der Trauer um meinen Vater kommen, dann will ich meinen Bruder
Jakob totschlagen!« \bibleverse{42} Als nun der Rebekka diese Äußerungen
ihres älteren Sohnes Esau hinterbracht wurden, ließ sie ihren jüngeren
Sohn Jakob rufen und sagte zu ihm: »Wisse: dein Bruder Esau sinnt auf
Rache gegen dich und will dich totschlagen! \bibleverse{43} Darum höre
nun, was ich dir rate, mein Sohn! Mache dich auf, fliehe zu meinem
Bruder Laban nach Haran \bibleverse{44} und bleibe einige Zeit bei ihm,
bis der Groll deines Bruders sich gelegt hat! \bibleverse{45} Wenn dann
sein Zorn gegen dich geschwunden ist und er vergessen hat, was du ihm
angetan hast, dann will ich hinsenden und dich von dort zurückholen
lassen. Warum soll ich euch beide an einem Tage verlieren?«
\bibleverse{46} Hierauf sagte Rebekka zu Isaak: »Das Leben wird mir
verleidet durch diese Hethiterinnen! Wenn auch Jakob sich solch eine
Hethiterin zur Frau nähme, eine von den Töchtern des Landes, was hätte
ich da noch vom Leben?«

\hypertarget{die-geschichte-jakobs-und-seiner-suxf6hne-kap.-28-50}{%
\subsubsection{3. Die Geschichte Jakobs und seiner Söhne (Kap.
28-50)}\label{die-geschichte-jakobs-und-seiner-suxf6hne-kap.-28-50}}

\hypertarget{a-jakobs-flucht-zu-den-verwandten-im-fernen-haran-sein-traum-und-sein-geluxfcbde-in-bethel}{%
\paragraph{a) Jakobs Flucht zu den Verwandten im fernen Haran; sein
Traum und sein Gelübde in
Bethel}\label{a-jakobs-flucht-zu-den-verwandten-im-fernen-haran-sein-traum-und-sein-geluxfcbde-in-bethel}}

\hypertarget{aa-jakob-begibt-sich-von-seinem-vater-gesegnet-auf-die-wanderung}{%
\subparagraph{aa) Jakob begibt sich, von seinem Vater gesegnet, auf die
Wanderung}\label{aa-jakob-begibt-sich-von-seinem-vater-gesegnet-auf-die-wanderung}}

\hypertarget{section-27}{%
\section{28}\label{section-27}}

\bibleverse{1} Da ließ Isaak den Jakob rufen, segnete ihn und gebot ihm:
»Du darfst dir keine Frau aus den Töchtern der Kanaanäer nehmen.
\bibleverse{2} Mache dich auf, gehe nach Nord-Mesopotamien zum Hause
Bethuels, des Vaters deiner Mutter, und hole dir von dort eine Frau,
eine von den Töchtern Labans, des Bruders deiner Mutter! \bibleverse{3}
Der allmächtige Gott aber segne dich, er mache dich fruchtbar und lasse
dich zahlreich werden, so daß du zu einem Haufen\textless sup
title=``=~einer Menge''\textgreater✲ von Völkern wirst! \bibleverse{4}
Und er gewähre dir den Segen Abrahams, dir und deinen Nachkommen mit
dir, damit du das Land, in dem du bis jetzt als Fremdling gewohnt hast
und das Gott dem Abraham verliehen hat, in Besitz nimmst.«
\bibleverse{5} So ließ Isaak den Jakob ziehen, und dieser machte sich
auf den Weg nach Nord-Mesopotamien zu Laban, dem Sohne des Aramäers
Bethuel, dem Bruder Rebekkas, der Mutter Jakobs und Esaus.

\hypertarget{bb-esaus-neue-heirat-mit-einer-tochter-ismaels}{%
\subparagraph{bb) Esaus neue Heirat mit einer Tochter
Ismaels}\label{bb-esaus-neue-heirat-mit-einer-tochter-ismaels}}

\bibleverse{6} Als nun Esau sah\textless sup title=``oder:
erfuhr''\textgreater✲, daß Isaak den Jakob gesegnet und ihn nach
Nord-Mesopotamien hatte ziehen lassen, damit er sich von dort eine Frau
hole, und daß er ihn gesegnet und ihm die Weisung gegeben hatte, keine
Frau von den Töchtern der Kanaanäer zu nehmen, \bibleverse{7} und daß
Jakob seinem Vater und seiner Mutter gehorsam gewesen und nach
Nord-Mesopotamien gezogen war: \bibleverse{8} da merkte Esau, daß die
Töchter der Kanaanäer seinem Vater mißfielen. \bibleverse{9} Darum begab
er sich zu Ismael und nahm zu seinen Frauen noch eine andere Frau hinzu,
nämlich Mahalath, die Tochter Ismaels, des Sohnes Abrahams, die
Schwester Nebajoths.

\hypertarget{cc-jakobs-traum-in-bethel-von-der-himmelsleiter}{%
\subparagraph{cc) Jakobs Traum in Bethel von der
Himmelsleiter}\label{cc-jakobs-traum-in-bethel-von-der-himmelsleiter}}

\bibleverse{10} Als Jakob aber von Beerseba aufgebrochen war und sich
auf die Wanderschaft nach Haran begeben hatte, \bibleverse{11} gelangte
er an die (heilige) Stätte und blieb daselbst über Nacht; denn die Sonne
war schon untergegangen. Er nahm also einen von den Steinen, die dort
lagen, machte ihn zu seinem Kopflager und legte sich daselbst schlafen.
\bibleverse{12} Da hatte er einen Traum: er sah eine Leiter, die auf der
Erde stand und mit ihrer Spitze bis an den Himmel reichte, und die Engel
Gottes stiegen auf ihr hinauf und herab. \bibleverse{13} Plötzlich stand
dann der HERR auf ihr\textless sup title=``oder: vor ihm''\textgreater✲
und sagte: »Ich bin der HERR, der Gott deines Vaters✲ Abraham und der
Gott Isaaks; das Land, auf dem du liegst, will ich dir und deinen
Nachkommen geben; \bibleverse{14} und deine Nachkommen sollen so
zahlreich werden wie der Staub der Erde; und du sollst dich nach Westen
und Osten, nach Norden und Süden hin ausbreiten, und in dir und in
deinen Nachkommen sollen alle Geschlechter der Erde gesegnet werden.
\bibleverse{15} Und siehe, ich will mit dir sein und dich überall
behüten, wohin du gehst, und will dich auch in dieses Land
zurückbringen; denn ich will dich nicht verlassen, bis ich das
ausgeführt habe, was ich dir verheißen habe.«

\hypertarget{dd-jakob-weiht-einen-denkstein-als-anfang-zu-einem-gotteshause-in-bethel}{%
\subparagraph{dd) Jakob weiht einen Denkstein als Anfang zu einem
Gotteshause in
Bethel}\label{dd-jakob-weiht-einen-denkstein-als-anfang-zu-einem-gotteshause-in-bethel}}

\bibleverse{16} Da erwachte Jakob aus seinem Schlaf und sagte:
»Wahrlich, der HERR ist an dieser Stätte gegenwärtig, ohne daß ich es
wußte!« \bibleverse{17} Da fürchtete er sich und rief aus: »Wie schaurig
ist diese Stätte! Ja, hier ist das Haus\textless sup title=``oder: die
Wohnung''\textgreater✲ Gottes und hier die Pforte des Himmels!«
\bibleverse{18} Am Morgen aber in aller Frühe stand Jakob auf, nahm den
Stein, den er sich zum Lager für sein Haupt gemacht hatte, richtete ihn
als Denkstein auf und goß Öl oben darauf. \bibleverse{19} Er gab dann
jener Stätte den Namen ›Bethel‹\textless sup title=``d.h. Haus
Gottes''\textgreater✲ -- vordem hatte die Ortschaft ›Lus‹ geheißen~--
\bibleverse{20} und sprach hierauf folgendes Gelübde aus: »Wenn Gott mit
mir ist und mich auf dem Wege, den ich jetzt gehen muß, behütet und mir
Brot zur Nahrung und Kleidung zum Anziehen gibt \bibleverse{21} und ich
glücklich in mein Vaterhaus zurückkehre, so soll der HERR mein Gott
sein, \bibleverse{22} und dieser Stein, den ich als Denkstein
aufgerichtet habe, soll zu einem Gotteshause werden, und von allem, was
du mir geben wirst, will ich dir getreulich den Zehnten entrichten!«

\hypertarget{b-jakobs-ankunft-und-erste-dienstzeit-bei-laban-in-haran-seine-verheiratung-mit-lea-und-rahel}{%
\paragraph{b) Jakobs Ankunft und erste Dienstzeit bei Laban in Haran;
seine Verheiratung mit Lea und
Rahel}\label{b-jakobs-ankunft-und-erste-dienstzeit-bei-laban-in-haran-seine-verheiratung-mit-lea-und-rahel}}

\hypertarget{aa-jakob-am-brunnen-zu-haran-sein-gespruxe4ch-mit-den-hirten}{%
\subparagraph{aa) Jakob am Brunnen zu Haran; sein Gespräch mit den
Hirten}\label{aa-jakob-am-brunnen-zu-haran-sein-gespruxe4ch-mit-den-hirten}}

\hypertarget{section-28}{%
\section{29}\label{section-28}}

\bibleverse{1} Hierauf setzte Jakob seine Wanderung fort und gelangte in
das Land, das gegen Osten lag. \bibleverse{2} Als er sich dort umsah,
gewahrte er auf dem Felde einen Brunnen, an dem gerade drei Herden
Kleinvieh lagerten; denn aus diesem Brunnen pflegte man die Herden zu
tränken; über der Öffnung des Brunnens aber lag ein großer Stein.
\bibleverse{3} Diesen wälzte man erst dann, wenn alle Herden dort
zusammengetrieben waren, von der Brunnenöffnung ab und tränkte das
Kleinvieh; darauf legte man den Stein wieder zurück an seinen Platz über
der Öffnung des Brunnens. \bibleverse{4} Da sagte Jakob zu den Leuten:
»Meine Brüder, woher seid ihr?« Sie antworteten: »Wir sind aus Haran.«
\bibleverse{5} Hierauf fragte er sie: »Kennt ihr Laban, den Sohn
Nahors?« Sie antworteten: »Ja, den kennen wir.« \bibleverse{6} Da fragte
er sie: »Geht es ihm gut?« Sie erwiderten: »Ja; und da kommt gerade
seine Tochter Rahel mit dem Kleinvieh!« \bibleverse{7} Da sagte er: »Es
ist ja noch hoch am Tage, und noch ist's nicht die Zeit, das Vieh
zusammenzutreiben; tränkt doch das Kleinvieh und laßt es dann wieder
weiden!« \bibleverse{8} Sie antworteten: »Das können wir nicht, bis alle
Herden beisammen sind; dann erst wälzt man den Stein von der Öffnung des
Brunnens ab, und wir tränken das Kleinvieh.«

\hypertarget{bb-jakobs-begruxfcuxdfung-mit-rahel-und-seine-aufnahme-bei-laban}{%
\subparagraph{bb) Jakobs Begrüßung mit Rahel und seine Aufnahme bei
Laban}\label{bb-jakobs-begruxfcuxdfung-mit-rahel-und-seine-aufnahme-bei-laban}}

\bibleverse{9} Während er noch mit ihnen redete, war Rahel mit dem
Kleinvieh ihres Vaters herangekommen; denn sie war eine Hirtin.
\bibleverse{10} Sobald nun Jakob Rahel, die Tochter seines Oheims Laban,
und das Kleinvieh seines Oheims Laban erblickt hatte, trat er hinzu,
wälzte den Stein von der Brunnenöffnung ab und tränkte das Kleinvieh
seines Oheims Laban. \bibleverse{11} Dann küßte er Rahel, weinte laut
\bibleverse{12} und teilte ihr mit, daß er ein Neffe ihres Vaters, und
zwar ein Sohn Rebekkas, sei; da eilte sie weg und berichtete es ihrem
Vater. \bibleverse{13} Als nun Laban die Nachricht über Jakob, den Sohn
seiner Schwester, vernahm, lief er ihm entgegen, umarmte und küßte ihn
und führte ihn in sein Haus; da erzählte er dem Laban seine ganze
Lebensgeschichte. \bibleverse{14} Laban aber sagte zu ihm: »Fürwahr, du
bist von meinem Fleisch und Bein.«

\hypertarget{cc-jakob-tritt-bei-laban-in-dienst-bei-der-werbung-um-rahel-durch-die-unterschiebung-leas-von-laban-getuxe4uscht-dient-er-nochmals-sieben-jahre-um-rahel}{%
\subparagraph{cc) Jakob tritt bei Laban in Dienst; bei der Werbung um
Rahel durch die Unterschiebung Leas von Laban getäuscht, dient er
nochmals sieben Jahre um
Rahel}\label{cc-jakob-tritt-bei-laban-in-dienst-bei-der-werbung-um-rahel-durch-die-unterschiebung-leas-von-laban-getuxe4uscht-dient-er-nochmals-sieben-jahre-um-rahel}}

Als Jakob nun einen Monat lang bei Laban geblieben war, \bibleverse{15}
sagte dieser zu ihm: »Du bist doch mein Verwandter✲: solltest du da
umsonst für mich arbeiten? Laß mich wissen, was dein Lohn sein soll!«
\bibleverse{16} Nun hatte Laban zwei Töchter: die ältere hieß Lea, die
jüngere Rahel; \bibleverse{17} Lea hatte matte Augen, während Rahel
schön von Gestalt und schön von Angesicht war. \bibleverse{18} Daher
hatte Jakob die Rahel liebgewonnen und sagte: »Ich will dir sieben Jahre
lang um deine jüngere Tochter Rahel dienen.« \bibleverse{19} Laban
antwortete: »Es ist besser, ich gebe sie dir als einem fremden Manne:
bleibe also bei mir!« \bibleverse{20} So diente denn Jakob um Rahel
sieben Jahre, und diese kamen ihm wie wenige Tage vor: so lieb hatte er
Rahel.

\bibleverse{21} Hierauf sagte Jakob zu Laban: »Meine Zeit ist
abgelaufen: gib mir nun meine Frau, damit ich mich mit ihr verheirate.«
\bibleverse{22} Da lud Laban alle Einwohner des Ortes ein und
veranstaltete ein Festmahl; \bibleverse{23} am Abend aber nahm er seine
Tochter Lea und brachte sie zu ihm hinein, und er wohnte ihr bei;
\bibleverse{24} und Laban gab seiner Tochter Lea seine Magd Silpa zur
Leibmagd. Am andern Morgen aber stellte es sich heraus, daß es Lea war.
\bibleverse{25} Als er nun zu Laban sagte: »Was hast du mir da angetan!
Habe ich nicht um Rahel bei dir gedient? Warum hast du mich betrogen?«,
\bibleverse{26} antwortete Laban: »Hierzulande ist es nicht Sitte, die
jüngere Tochter vor der älteren wegzugeben. \bibleverse{27} Bringe die
Brautwoche mit dieser zu Ende, dann soll dir auch die andere gegeben
werden für den Dienst, den du mir noch weitere sieben Jahre leisten
mußt.« \bibleverse{28} Jakob willigte ein und hielt die Brautwoche mit
Lea aus; dann gab Laban ihm auch seine Tochter Rahel zur Frau;
\bibleverse{29} und Laban gab seiner Tochter Rahel seine Magd Bilha zur
Leibmagd. \bibleverse{30} Jakob ging nun auch zu Rahel ein, hatte aber
Rahel lieber als Lea; er blieb dann noch weitere sieben Jahre bei Laban
im Dienst.

\hypertarget{c-jakobs-kindersegen-sein-wachsender-reichtum-und-seine-letzten-dienstjahre-bei-laban}{%
\paragraph{c) Jakobs Kindersegen; sein wachsender Reichtum und seine
letzten Dienstjahre bei
Laban}\label{c-jakobs-kindersegen-sein-wachsender-reichtum-und-seine-letzten-dienstjahre-bei-laban}}

\hypertarget{aa-leas-vier-erste-suxf6hne}{%
\subparagraph{aa) Leas vier erste
Söhne}\label{aa-leas-vier-erste-suxf6hne}}

\bibleverse{31} Als nun der HERR sah, daß Lea ungeliebt war, machte er
sie fruchtbar, während Rahel kinderlos blieb. \bibleverse{32} Lea wurde
also guter Hoffnung und gebar einen Sohn, den sie Ruben\textless sup
title=``d.h. Sehet ein Sohn!''\textgreater✲ nannte; »denn«, sagte sie,
»der HERR hat mein Elend angesehen; ja, nun wird mein Mann mich
liebgewinnen«. \bibleverse{33} Hierauf wurde sie wieder guter Hoffnung,
und als sie einen Sohn geboren hatte, sagte sie: »Weil der HERR gehört
hat, daß ich ungeliebt bin, hat er mir auch diesen Sohn gegeben«; darum
nannte sie ihn Simeon\textless sup title=``d.h. Erhörung''\textgreater✲.
\bibleverse{34} Als sie dann wieder guter Hoffnung geworden war und
einen Sohn gebar, sagte sie: »Nun endlich wird mein Mann mir
anhangen\textless sup title=``oder: zugetan sein''\textgreater✲, denn
ich habe ihm drei Söhne geboren«; darum nannte sie ihn Levi\textless sup
title=``d.h. Anschließung, Anhänglichkeit''\textgreater✲.
\bibleverse{35} Hierauf wurde sie nochmals guter Hoffnung und gebar
einen Sohn; da sagte sie: »Diesmal will ich den HERRN preisen!« Darum
nannte sie ihn Juda\textless sup title=``d.h. Gegenstand des
Preises''\textgreater✲. Danach bekam sie kein Kind mehr.

\hypertarget{bb-die-zwei-suxf6hne-der-bilha-der-leibmagd-rahels}{%
\subparagraph{bb) Die zwei Söhne der Bilha, der Leibmagd
Rahels}\label{bb-die-zwei-suxf6hne-der-bilha-der-leibmagd-rahels}}

\hypertarget{section-29}{%
\section{30}\label{section-29}}

\bibleverse{1} Als nun Rahel sah, daß sie dem Jakob keine Kinder gebar,
wurde sie auf ihre Schwester neidisch\textless sup title=``oder:
eifersüchtig''\textgreater✲ und sagte zu Jakob: »Schaffe mir Kinder,
oder ich sterbe!« \bibleverse{2} Da geriet Jakob in Zorn gegen Rahel und
sagte: »Stehe ich etwa an Gottes Statt, der dir Kindersegen versagt
hat?« \bibleverse{3} Da erwiderte sie: »Hier hast du meine Leibmagd
Bilha; gehe zu ihr ein, damit sie auf meinen Knien\textless sup
title=``oder: auf meinem Schoß''\textgreater✲ gebiert und auch ich durch
sie zu Kindern komme!« \bibleverse{4} So gab sie ihm ihre Leibmagd Bilha
zum (Neben-) Weib, und Jakob ging zu ihr ein; \bibleverse{5} da wurde
Bilha guter Hoffnung und gebar dem Jakob einen Sohn. \bibleverse{6}
Rahel aber sagte: »Gott hat mich mein Recht finden lassen und auch meine
Bitte erhört und mir einen Sohn geschenkt!« Darum gab sie ihm den Namen
Dan\textless sup title=``d.h. Richter oder: einer, der Recht
schafft''\textgreater✲. \bibleverse{7} Hierauf wurde Bilha, die Leibmagd
Rahels, wieder guter Hoffnung und gebar dem Jakob einen zweiten Sohn.
\bibleverse{8} Da sagte Rahel: »Gotteskämpfe habe ich mit meiner
Schwester gekämpft und habe auch gesiegt!« Darum nannte sie ihn
Naphthali\textless sup title=``d.h. der Erkämpfte''\textgreater✲.

\hypertarget{cc-die-zwei-suxf6hne-der-silpa-der-leibmagd-leas}{%
\subparagraph{cc) Die zwei Söhne der Silpa, der Leibmagd
Leas}\label{cc-die-zwei-suxf6hne-der-silpa-der-leibmagd-leas}}

\bibleverse{9} Als nun Lea sah, daß sie nicht mehr Mutter wurde, nahm
sie ihre Leibmagd Silpa und gab sie dem Jakob zum (Neben-) Weibe.
\bibleverse{10} So gebar denn Silpa, die Leibmagd Leas, dem Jakob einen
Sohn. \bibleverse{11} Da sagte Lea: »Glückauf!« und gab ihm den Namen
Gad\textless sup title=``d.h. Glück''\textgreater✲. \bibleverse{12}
Hierauf gebar Silpa, die Leibmagd Leas, dem Jakob noch einen zweiten
Sohn. \bibleverse{13} Da sagte Lea: »Ich Glückliche! Ja, glücklich
werden mich die Töchter des Volkes\textless sup title=``=~die
Frauen''\textgreater✲ preisen!« Darum nannte sie ihn Asser\textless sup
title=``d.h. beglückt oder: Glückbringer''\textgreater✲.

\hypertarget{dd-die-wirkung-der-liebesuxe4pfel-die-letzten-kinder-der-lea}{%
\subparagraph{dd) Die Wirkung der Liebesäpfel; die letzten Kinder der
Lea}\label{dd-die-wirkung-der-liebesuxe4pfel-die-letzten-kinder-der-lea}}

\bibleverse{14} Als nun Ruben einmal in den Tagen der Weizenernte
ausging, fand er Liebesäpfel auf dem Felde und brachte sie seiner Mutter
Lea. Da sagte Rahel zu Lea: »Gib mir doch einige von den Liebesäpfeln
deines Sohnes!« \bibleverse{15} Aber sie antwortete ihr: »Ist es nicht
genug, daß du mir meinen Mann genommen hast? Willst\textless sup
title=``oder: mußt''\textgreater✲ du mir nun auch noch die Liebesäpfel
meines Sohnes nehmen?« Darauf antwortete Rahel: »So mag Jakob denn diese
Nacht bei dir verbringen zum Entgelt für die Liebesäpfel deines Sohnes!«
\bibleverse{16} Als Jakob nun am Abend vom Felde heimkam, ging Lea
hinaus ihm entgegen und sagte: »Zu mir mußt du eingehen; denn ich habe
dich um vollen Preis mit den Liebesäpfeln meines Sohnes erkauft!« So
verbrachte er denn jene Nacht bei ihr; \bibleverse{17} und Gott erhörte
die Bitte der Lea, so daß sie guter Hoffnung wurde und dem Jakob einen
fünften Sohn gebar. \bibleverse{18} Da sagte Lea: »Gott hat mir meinen
Lohn dafür gegeben, daß ich meine Leibmagd meinem Manne überlassen
habe«; darum gab sie ihm den Namen Issaschar\textless sup title=``d.h.
Lohnempfang oder: mein Lohn''\textgreater✲. \bibleverse{19} Hierauf
wurde Lea noch einmal guter Hoffnung und gebar dem Jakob einen sechsten
Sohn. \bibleverse{20} Da sagte Lea: »Gott hat mich mit einem schönen
Geschenk bedacht: nun endlich wird mein Mann bei mir wohnen; denn ich
habe ihm sechs Söhne geboren!« Darum nannte sie ihn Sebulon\textless sup
title=``d.h. Wohner''\textgreater✲. \bibleverse{21} Später gebar sie
noch eine Tochter, die sie Dina\textless sup title=``vgl. Kap.
34''\textgreater✲ nannte.

\hypertarget{ee-rahel-wird-die-mutter-josephs}{%
\subparagraph{ee) Rahel wird die Mutter
Josephs}\label{ee-rahel-wird-die-mutter-josephs}}

\bibleverse{22} Nun dachte Gott auch an Rahel: Gott erhörte sie und
vergönnte ihr Mutterfreuden; \bibleverse{23} sie wurde guter Hoffnung
und gebar einen Sohn. Da sagte sie: »Gott hat meine Schmach
hinweggenommen!« \bibleverse{24} Darum gab sie ihm den Namen
Joseph\textless sup title=``d.h. er nahm weg; oder: er fügte
hinzu!''\textgreater✲, indem sie sagte: »Der HERR möge mir noch einen
Sohn hinzufügen!«

\hypertarget{ff-jakobs-neuer-dienstvertrag-mit-laban}{%
\subparagraph{ff) Jakobs neuer Dienstvertrag mit
Laban}\label{ff-jakobs-neuer-dienstvertrag-mit-laban}}

\bibleverse{25} Als nun Rahel den Joseph geboren hatte, sagte Jakob zu
Laban: »Laß mich ziehen! Ich möchte in meine Heimat und in mein
Vaterland zurückkehren. \bibleverse{26} Gib mir meine Frauen und meine
Kinder, um die ich dir gedient habe, damit ich hinziehen kann; du weißt
ja selbst, welche Dienste ich dir geleistet habe.« \bibleverse{27} Da
antwortete ihm Laban: »Erweise mir doch eine Liebe! Es ist mir
klargeworden, daß der HERR mich um deinetwillen gesegnet hat.«
\bibleverse{28} Dann fuhr er fort: »Bestimme nur den Lohn, den du von
mir verlangst, so will ich ihn dir geben.« \bibleverse{29} Da antwortete
er ihm: »Du weißt selbst, wie ich dir gedient habe und was aus deinem
Viehbesitz unter meiner Hut geworden ist. \bibleverse{30} Du besaßest ja
vor meiner Ankunft nur wenig; aber nun hat er sich gewaltig vermehrt,
und der HERR hat dich bei allem, was ich unternommen habe, gesegnet. Nun
aber -- wann soll auch ich für meine Familie sorgen?« \bibleverse{31} Da
fragte jener: »Was soll ich dir geben?« Jakob antwortete: »Du brauchst
mir gar nichts zu geben! Wenn du nur auf folgenden Vorschlag von mir
eingehst, so will ich dein Kleinvieh von neuem weiden und hüten:
\bibleverse{32} ich will heute durch dein sämtliches Kleinvieh
hindurchgehen, indem ich daraus alle gesprenkelten und gefleckten Tiere
und überdies jedes dunkelfarbige Stück unter den Schaflämmern und alle
gefleckten und gesprenkelten Ziegen absondere; und nur solche Tiere
sollen mein Lohn sein. \bibleverse{33} Und darin soll an irgendeinem
künftigen Tage meine Ehrlichkeit sich klar erweisen: Wenn du kommst, um
dir meinen Lohn anzusehen, so soll jedes Stück, das unter den Ziegen
nicht gesprenkelt oder gefleckt und unter den Schafen nicht dunkelfarbig
ist, als von mir gestohlen gelten.« \bibleverse{34} Da sagte Laban:
»Gut! Dein Vorschlag soll gelten!«

\hypertarget{gg-jakob-gelangt-durch-list-zu-grouxdfem-viehbesitz}{%
\subparagraph{gg) Jakob gelangt durch List zu großem
Viehbesitz}\label{gg-jakob-gelangt-durch-list-zu-grouxdfem-viehbesitz}}

\bibleverse{35} Er sonderte dann noch an demselben Tage die gestreiften
und gefleckten Ziegenböcke und alle gesprenkelten und gefleckten Ziegen
ab, jedes Stück, woran nur etwas Weißes war, und alles, was unter den
Schafen dunkelfarbig war, und übergab diese Tiere der Hut seiner Söhne.
\bibleverse{36} Sodann setzte er einen Zwischenraum von drei Tagereisen
zwischen sich und Jakob fest; Jakob aber blieb als Hirt bei dem übrigen
Kleinvieh Labans.

\bibleverse{37} Nun holte sich Jakob frische Stäbe\textless sup
title=``oder: Schosse''\textgreater✲ von Weißpappeln sowie von
Mandelbäumen und Platanen und schälte an ihnen weiße Streifen heraus,
indem er das Weiße an den Stäben bloßlegte; \bibleverse{38} dann stellte
er die Stäbe, die er geschält hatte, in die Wassertröge, in die
Wassertränkrinnen, zu denen das Kleinvieh zum Trinken zu kommen pflegte,
gerade vor die Tiere hin. Wenn dann die Tiere, die zur Tränke kamen,
brünstig wurden \bibleverse{39} und sich vor den Stäben begatteten, so
brachten sie gestreifte, gesprenkelte und gefleckte Junge zur Welt.
\bibleverse{40} Jakob sonderte dann die Lämmer ab und richtete die
Blicke der Tiere auf das Gestreifte und alles Dunkelfarbige unter dem
Kleinvieh Labans und legte (so) für sich besondere Herden an, die er
nicht zu dem Kleinvieh Labans tat. \bibleverse{41} Und sooft fortan das
kräftige Kleinvieh brünstig wurde, stellte Jakob die Stäbe den Tieren
vor die Augen in die Wassertröge, damit sie sich vor den Stäben
begatteten; \bibleverse{42} wenn dagegen die Tiere schwächlich waren,
stellte er sie nicht hin; so kam es, daß die schwächlichen Tiere dem
Laban, die kräftigen aber dem Jakob zuteil wurden. \bibleverse{43} So
wurde er ein außerordentlich reicher Mann, und er erwarb sich große
Herden, auch Mägde und Knechte, Kamele und Esel.

\hypertarget{d-jakob-trennt-sich-von-laban-um-nach-kanaan-zuruxfcckzukehren}{%
\paragraph{d) Jakob trennt sich von Laban, um nach Kanaan
zurückzukehren}\label{d-jakob-trennt-sich-von-laban-um-nach-kanaan-zuruxfcckzukehren}}

\hypertarget{aa-die-anluxe4sse-zu-jakobs-flucht}{%
\subparagraph{aa) Die Anlässe zu Jakobs
Flucht}\label{aa-die-anluxe4sse-zu-jakobs-flucht}}

\hypertarget{section-30}{%
\section{31}\label{section-30}}

\bibleverse{1} Da erhielt Jakob Kunde von den Äußerungen der Söhne
Labans, die da sagten: »Jakob hat das ganze Hab und Gut unsers Vaters an
sich gebracht und seinen ganzen jetzigen Reichtum aus dem Besitz unsers
Vaters gewonnen.« \bibleverse{2} Zugleich merkte Jakob an Labans
Gesichtsausdruck wohl, daß er gegen ihn nicht mehr so gesinnt war wie
früher. \bibleverse{3} Da sagte der HERR zu Jakob: »Kehre in das Land
deiner Väter und zu deiner Verwandtschaft zurück; ich will mit dir
sein!«

\hypertarget{bb-jakobs-beratung-mit-seinen-frauen}{%
\subparagraph{bb) Jakobs Beratung mit seinen
Frauen}\label{bb-jakobs-beratung-mit-seinen-frauen}}

\bibleverse{4} Da sandte Jakob hin, ließ Rahel und Lea auf das Feld zu
seiner Herde rufen \bibleverse{5} und sagte zu ihnen: »Ich sehe es dem
Gesicht eures Vaters an, daß er gegen mich nicht mehr so gesinnt ist wie
früher, obgleich doch der Gott meines Vaters mit mir gewesen ist.
\bibleverse{6} Ihr selbst wißt ja, daß ich eurem Vater mit meiner ganzen
Kraft gedient habe; \bibleverse{7} doch euer Vater hat mich betrogen und
mir den Lohn schon zehnmal abgeändert, Gott aber hat ihm nicht
gestattet, mir Schaden zuzufügen. \bibleverse{8} Sooft er nämlich sagte:
›Die gesprenkelten Tiere sollen dein Lohn sein!‹, warf die ganze Herde
gesprenkelte Lämmer, und sooft er sagte: ›Die gestreiften Tiere sollen
dein Lohn sein!‹, warf die ganze Herde gestreifte Lämmer. \bibleverse{9}
So hat Gott eurem Vater den Viehbesitz genommen und ihn mir gegeben.
\bibleverse{10} In der Brunstzeit des Kleinviehs nämlich hob ich meine
Augen auf und sah im Traume, wie die Böcke, welche das Kleinvieh
belegten, gestreift, gesprenkelt und getüpfelt waren. \bibleverse{11}
Der Engel Gottes aber sagte im Traume zu mir: ›Jakob!‹ Ich antwortete:
›Hier bin ich!‹ \bibleverse{12} Da sagte er: ›Hebe doch deine Augen auf
und sieh: alle Böcke, die das Kleinvieh belegen, sind gestreift,
gesprenkelt und getüpfelt! Denn ich habe alles gesehen, was Laban dir
angetan hat. \bibleverse{13} Ich bin der Gott von Bethel, wo du einen
Denkstein gesalbt und wo du mir ein Gelübde getan hast. Mache dich jetzt
auf, verlaß dieses Land und kehre in dein Heimatland zurück!‹«
\bibleverse{14} Da antworteten ihm Rahel und Lea mit den Worten: »Haben
wir etwa noch ein Teil und Erbe im Hause unsers Vaters? \bibleverse{15}
Haben wir ihm nicht als Fremde gegolten? Er hat uns ja verhandelt und
den Erlös für uns längst vollständig verbraucht. \bibleverse{16} Ja, der
ganze Reichtum, den Gott unserm Vater entzogen hat, gehört uns und
unsern Söhnen\textless sup title=``oder: Kindern''\textgreater✲. Tu also
nun alles, was Gott dir geboten hat!«

\hypertarget{cc-jakobs-flucht-und-labans-verfolgung}{%
\subparagraph{cc) Jakobs Flucht und Labans
Verfolgung}\label{cc-jakobs-flucht-und-labans-verfolgung}}

\bibleverse{17} Da machte Jakob sich auf, setzte seine Kinder und seine
Frauen auf die Kamele \bibleverse{18} und nahm sein sämtliches Vieh und
all sein Hab und Gut mit, das er erworben hatte, das Vieh, das ihm
gehörte, das er in Nord-Mesopotamien erworben hatte, um sich zu seinem
Vater Isaak nach dem Lande Kanaan zu begeben. \bibleverse{19} Während
aber Laban hingegangen war, um seine Schafe zu scheren, entwandte Rahel
das Bild des Hausgottes ihres Vaters; \bibleverse{20} und auch Jakob
täuschte den Aramäer Laban, insofern er ihm nichts davon mitteilte, daß
er sich heimlich entfernen wollte. \bibleverse{21} Er entfloh also mit
allem, was ihm gehörte, und machte sich auf den Weg; er setzte über den
Euphratstrom und schlug den Weg\textless sup title=``oder: die
Richtung''\textgreater✲ nach dem Gebirge Gilead ein. \bibleverse{22}
Erst am dritten Tage erfuhr Laban, daß Jakob entflohen war.
\bibleverse{23} Da nahm er seine Stammesgenossen mit sich, verfolgte ihn
sieben Tagereisen weit und holte ihn am Gebirge Gilead ein.
\bibleverse{24} Aber Gott erschien dem Aramäer Laban nachts im Traum und
sagte zu ihm: »Hüte dich wohl, mit Jakob anders als freundlich zu
reden!«

\hypertarget{dd-labans-strafrede-und-haussuchung}{%
\subparagraph{dd) Labans Strafrede und
Haussuchung}\label{dd-labans-strafrede-und-haussuchung}}

\bibleverse{25} Als nun Laban den Jakob eingeholt hatte -- Jakob hatte
aber sein Zelt auf dem Berge\textless sup title=``oder: im
Gebirge''\textgreater✲ aufgeschlagen, und auch Laban lagerte sich mit
seinen Stammesgenossen im Gebirge Gilead --, \bibleverse{26} da sagte
Laban zu Jakob: »Warum hast du es unternommen, mich zu täuschen, und
hast meine Töchter wie Kriegsgefangene entführt? \bibleverse{27} Warum
bist du heimlich entflohen und hast mich hintergangen\textless sup
title=``oder: verstohlen gegen mich gehandelt''\textgreater✲ und mir
nichts davon mitgeteilt -- ich hätte dir sonst mit Sang und Klang, mit
Paukenschall und Saitenspiel das Geleit gegeben~-- \bibleverse{28} und
hast mir nicht einmal verstattet, meine Enkel und Töchter (zum Abschied)
zu küssen? Ja, du hast töricht gehandelt! \bibleverse{29} Es stände nun
wohl in meiner Macht, dir übel mitzuspielen; aber der Gott deines Vaters
hat gestern nacht zu mir gesagt: ›Hüte dich ja davor, mit Jakob anders
als freundlich zu reden!‹ \bibleverse{30} Nun gut, du bist von mir
weggegangen, weil du so starke Sehnsucht nach dem Hause deines Vaters
hattest; aber warum hast du mir meinen (Haus-) Gott gestohlen?«
\bibleverse{31} Da antwortete Jakob dem Laban: »(Ich bin geflohen) weil
ich mich fürchtete; denn ich dachte, du würdest mir deine Töchter
entreißen. \bibleverse{32} Bei wem du aber dein Götterbild findest, der
soll nicht am Leben bleiben! Durchsuche im Beisein unserer
Stammesgenossen alles, was ich bei mir habe, und nimm das an dich, was
dir gehört!« Jakob wußte nämlich nicht, daß Rahel (das Götterbild)
entwandt hatte. \bibleverse{33} Da ging Laban in Jakobs Zelt und in Leas
Zelt und in das Zelt der beiden Mägde, fand aber nichts. Aus dem Zelt
der Lea ging er dann in das Zelt der Rahel. \bibleverse{34} Diese hatte
aber das Götterbild genommen und es in den Sattelkorb des
Kamels\textless sup title=``oder: in die Kamelsänfte''\textgreater✲
gelegt und sich daraufgesetzt. Laban durchsuchte nun das ganze Zelt,
fand aber nichts. \bibleverse{35} Sie hatte nämlich zu ihrem Vater
gesagt: »O Herr, sei nicht ungehalten darüber, daß ich vor dir nicht
aufstehen kann! Ich bin eben unwohl nach der Frauen Weise.« So hatte er
denn trotz seines Suchens das Götterbild nicht gefunden.

\hypertarget{ee-jakobs-anklagerede}{%
\subparagraph{ee) Jakobs Anklagerede}\label{ee-jakobs-anklagerede}}

\bibleverse{36} Nunmehr geriet Jakob in Zorn und machte Laban laute
Vorwürfe mit den Worten: »Was habe ich nun verbrochen, was verschuldet,
daß du mich so hitzig verfolgt hast? \bibleverse{37} Du hast nun all
meinen Hausrat durchstöbert: was hast du denn von deinem gesamten
Hausrat gefunden? Lege es hierher vor meine und deine Stammesgenossen:
sie sollen entscheiden, wer von uns beiden im Recht ist! \bibleverse{38}
Zwanzig Jahre bin ich jetzt bei dir gewesen: deine Mutterschafe und
deine Ziegen haben nie fehlgeworfen, und von den Böcken deines
Kleinviehs habe ich keinen gegessen. \bibleverse{39} Wenn ein Stück Vieh
(von wilden Tieren) zerrissen war, habe ich es nicht zu dir bringen
dürfen, nein, ich habe es ersetzen müssen: von mir hast du es gefordert,
mochte es bei Tage oder in der Nacht geraubt sein. \bibleverse{40} So
ging es mir: bei Tage kam ich vor Hitze um und nachts vor Frost, und
kein Schlaf kam in meine Augen. \bibleverse{41} Jetzt sind es zwanzig
Jahre, daß ich dir in deinem Hause gedient habe: vierzehn Jahre um deine
beiden Töchter und sechs Jahre bei deinem Kleinvieh; und zehnmal hast du
mir den Lohn abgeändert. \bibleverse{42} Wenn nicht der Gott meines
Vaters✲, der Gott Abrahams, den auch Isaak verehrt, auf meiner Seite
gestanden hätte, ja, dann hättest du mich jetzt mit leeren Händen ziehen
lassen! Aber Gott hat mein Elend und die mühselige Arbeit meiner Hände
gesehen und gestern nacht sein Urteil abgegeben!«

\hypertarget{ff-labans-antwort-der-friedensvertrag-zwischen-ihm-und-jakob}{%
\subparagraph{ff) Labans Antwort; der Friedensvertrag zwischen ihm und
Jakob}\label{ff-labans-antwort-der-friedensvertrag-zwischen-ihm-und-jakob}}

\bibleverse{43} Da gab Laban dem Jakob zur Antwort: »Die Töchter sind
meine Töchter, und die Kinder sind meine Kinder, das Vieh ist mein Vieh,
und alles, was du hier siehst, gehört mir! Aber was könnte ich heute
noch für diese meine Töchter tun oder für ihre Kinder, die sie geboren
haben? \bibleverse{44} So komm denn, laß uns beide einen Vertrag
miteinander schließen, der soll als Zeuge zwischen mir und dir dienen!«
\bibleverse{45} Hierauf nahm Jakob einen Stein und richtete ihn als
Denkstein auf; \bibleverse{46} dann sagte er zu seinen Stammesgenossen:
»Lest Steine zusammen!« Da holten sie Steine und machten einen Haufen
davon; dann hielten sie dort auf dem Steinhaufen ein Mahl.
\bibleverse{47} Und Laban nannte ihn ›Jegar-Sahadutha‹\textless sup
title=``d.h. aramäisch: Haufe des Zeugnisses''\textgreater✲, Jakob aber
nannte ihn ›Galed‹\textless sup title=``d.h. hebräisch: ein als Zeuge
dienender Haufe''\textgreater✲. \bibleverse{48} Darauf sagte Laban:
»Dieser Steinhaufe ist heute ein Zeuge zwischen mir und dir!« Darum
nannte er ihn ›Galed‹; \bibleverse{49} und den Denkstein, den er
aufgerichtet hatte, nannte er ›Mizpa‹\textless sup title=``d.h. Warte
oder: Wacht''\textgreater✲, indem er sagte: »Der HERR sei Wächter
zwischen mir und dir, wenn wir einander aus den Augen gekommen sind!
\bibleverse{50} Solltest du je meine Töchter schlecht behandeln oder
noch andere Frauen zu meinen Töchtern hinzunehmen, wenn dann auch kein
Mensch bei uns sein sollte -- bedenke wohl: Gott ist Zeuge zwischen mir
und dir!« \bibleverse{51} Weiter sagte Laban zu Jakob: »Siehe, der
Steinhaufe hier und der Denkstein hier, den ich zwischen mir und dir
aufgerichtet habe: \bibleverse{52} dieser Steinhaufe soll ein Zeuge und
der Denkstein hier ein Zeugnis sein, daß weder ich über diesen
Steinhaufen zu dir hinausgehen darf, noch du über diesen Steinhaufen und
diesen Denkstein zu mir in böser Absicht hinausgehen darfst.
\bibleverse{53} Der Gott Abrahams und der Gott Nahors sollen Richter
zwischen uns sein, der Gott je ihres Stammvaters!« Als dann Jakob bei
dem Gott, den sein Vater Isaak verehrte\textless sup title=``vgl.
V.42''\textgreater✲, geschworen hatte, \bibleverse{54} brachte er ein
Schlachtopfer auf dem Berge dar und lud seine Stammesgenossen zur
Teilnahme am Mahl ein. So hielten sie denn das Mahl und übernachteten
auf dem Berge.

\hypertarget{gg-labans-abschiednahme}{%
\subparagraph{gg) Labans Abschiednahme}\label{gg-labans-abschiednahme}}

\hypertarget{section-31}{%
\section{32}\label{section-31}}

\bibleverse{1} Am andern Morgen aber in der Frühe küßte Laban seine
Enkel und seine Töchter und nahm Abschied von ihnen; dann brach er auf
und kehrte an seinen Wohnort zurück.

\hypertarget{e-jakobs-botschaft-an-esau-seine-verzagtheit-sein-gebet-und-gottes-nuxe4chtliches-ringen-mit-ihm}{%
\paragraph{e) Jakobs Botschaft an Esau; seine Verzagtheit, sein Gebet
und Gottes nächtliches Ringen mit
ihm}\label{e-jakobs-botschaft-an-esau-seine-verzagtheit-sein-gebet-und-gottes-nuxe4chtliches-ringen-mit-ihm}}

\hypertarget{aa-dem-jakob-erscheinen-engel-bei-seinem-eintritt-in-das-verheiuxdfungsland}{%
\subparagraph{aa) Dem Jakob erscheinen Engel bei seinem Eintritt in das
Verheißungsland}\label{aa-dem-jakob-erscheinen-engel-bei-seinem-eintritt-in-das-verheiuxdfungsland}}

\bibleverse{2} Als nun auch Jakob seines Weges zog, begegneten ihm Engel
Gottes. \bibleverse{3} Sobald Jakob sie erblickte, sagte er: »Hier ist
Gottes Heerlager!« Darum nannte er jenen Ort ›Mahanaim‹\textless sup
title=``d.h. zwei Lager, Doppellager''\textgreater✲.

\hypertarget{bb-jakob-sendet-boten-an-esau-mit-demuxfctiger-bitte-und-teilt-seinen-zug-aus-angst-vor-esau-in-zwei-teile}{%
\subparagraph{bb) Jakob sendet Boten an Esau mit demütiger Bitte und
teilt seinen Zug aus Angst vor Esau in zwei
Teile}\label{bb-jakob-sendet-boten-an-esau-mit-demuxfctiger-bitte-und-teilt-seinen-zug-aus-angst-vor-esau-in-zwei-teile}}

\bibleverse{4} Hierauf sandte Jakob Boten voraus an seinen Bruder Esau
nach der Landschaft Seir, ins Gebiet der Edomiter, \bibleverse{5} und
gab ihnen folgenden Auftrag: »So sollt ihr zu meinem Herrn, zu Esau,
sagen: ›Dein Knecht Jakob läßt dir folgendes melden: Ich habe bei Laban
in der Fremde gelebt und mich bis jetzt dort aufgehalten; \bibleverse{6}
ich habe mir dort Rinder und Esel, Kleinvieh, Knechte und Mägde erworben
und sende nun Boten, um es meinem Herrn mitzuteilen, damit ich Gnade in
deinen Augen\textless sup title=``=~bei dir''\textgreater✲ finde.‹«
\bibleverse{7} Die Boten kehrten dann zu Jakob zurück mit der Meldung:
»Wir sind zu deinem Bruder Esau gekommen, und er zieht dir auch schon in
Begleitung von vierhundert Mann entgegen.« \bibleverse{8} Da geriet
Jakob in große Angst, und es wurde ihm bange; er teilte daher die Leute,
die er bei sich hatte, und ebenso das Kleinvieh sowie die Rinder und
Kamele in zwei Heere; \bibleverse{9} denn er dachte: »Wenn Esau den
einen Zug überfällt und niederschlägt, so wird doch der andere Zug
entrinnen können.«

\hypertarget{cc-jakobs-gebet-um-hilfe}{%
\subparagraph{cc) Jakobs Gebet um
Hilfe}\label{cc-jakobs-gebet-um-hilfe}}

\bibleverse{10} Dann betete Jakob: »Gott meines Vaters✲ Abraham und
meines Vaters Isaak, HERR, der du mir geboten hast: ›Kehre in dein
Vaterland und zu deiner Verwandtschaft zurück, ich will dir Gutes
tun‹~-- \bibleverse{11} ich bin zu gering für all die Gnadenerweise und
all die Treue, die du deinem Knecht erwiesen hast! Denn nur mit meinem
Wanderstabe bin ich (einst) über den Jordan dort gezogen und bin jetzt
zu zwei Heeren geworden. \bibleverse{12} Ach, errette mich nun aus der
Hand meines Bruders, aus der Hand Esaus! Denn ich bin in Angst vor ihm,
daß er kommt und uns erschlägt, die Mütter samt den Kindern!
\bibleverse{13} Du hast mir doch verheißen: ›Gewiß, ich will dir Gutes
tun und deine Nachkommen so zahlreich werden lassen, daß sie sind wie
der Sand am Meer, den man vor Menge nicht zählen kann.‹«

\hypertarget{dd-abordnung-neuer-boten-an-esau-mit-geschenken}{%
\subparagraph{dd) Abordnung neuer Boten an Esau mit
Geschenken}\label{dd-abordnung-neuer-boten-an-esau-mit-geschenken}}

\bibleverse{14} Er blieb dann in jener Nacht dort und wählte aus dem
Vieh, das ihm gerade zur Hand war, ein Geschenk für seinen Bruder Esau
aus, \bibleverse{15} nämlich zweihundert Ziegen und zwanzig Böcke,
zweihundert Mutterschafe und zwanzig Widder, \bibleverse{16} dreißig
säugende Kamele nebst ihren Füllen, vierzig junge Kühe und zehn junge
Stiere, zwanzig Eselinnen und zehn Eselfüllen. \bibleverse{17} Er
übergab diese seinen Knechten, jede Herde besonders, und befahl seinen
Knechten: »Zieht vor mir her und laßt einen Abstand zwischen den
einzelnen Herden!« \bibleverse{18} Dann gab er dem ersten✲ folgende
Weisung: »Wenn mein Bruder Esau dir begegnet und dich fragt: ›Wem
gehörst du, und wohin willst du, und wem gehören die Tiere, die du da
treibst?‹, \bibleverse{19} so antworte: ›Sie gehören deinem Knecht
Jakob; es ist das ein Geschenk, das er meinem Herrn Esau sendet; er
selbst kommt gleich hinter uns her.‹« \bibleverse{20} Dieselbe Weisung
gab er auch dem zweiten und dem dritten und allen anderen, welche die
Herden trieben, nämlich: »Ganz ebenso sollt ihr zu Esau sagen, wenn ihr
ihm begegnet, \bibleverse{21} und sollt weiter sagen: ›Dein Knecht Jakob
kommt selbst gleich hinter uns her.‹« Er dachte nämlich: »Ich will ihn
durch das Geschenk versöhnen, das mir vorauszieht; erst dann will ich
ihm selber vor die Augen treten; vielleicht nimmt er mich dann
freundlich an.« \bibleverse{22} So zog also das Geschenk vor ihm her,
während er selbst jene Nacht im Lager zubrachte.

\hypertarget{ee-jakob-luxe4uxdft-die-seinen-den-jabbok-uxfcberschreiten-gottes-nuxe4chtlicher-ringkampf-mit-ihm-sein-neuer-name-israel}{%
\subparagraph{ee) Jakob läßt die Seinen den Jabbok überschreiten; Gottes
nächtlicher Ringkampf mit ihm; sein neuer Name
Israel}\label{ee-jakob-luxe4uxdft-die-seinen-den-jabbok-uxfcberschreiten-gottes-nuxe4chtlicher-ringkampf-mit-ihm-sein-neuer-name-israel}}

\bibleverse{23} Er machte sich aber noch in derselben Nacht auf, nahm
seine beiden Frauen und seine beiden Leibmägde samt seinen elf Söhnen
und setzte über die Furt des Jabbok. \bibleverse{24} Er nahm sie also
und ließ sie über den Fluß fahren, und als er dann auch alle seine Habe
hinübergebracht hatte, \bibleverse{25} blieb er allein zurück. Da rang
ein Mann mit ihm bis zum Aufgang der Morgenröte. \bibleverse{26} Als
dieser nun sah, daß er ihn nicht bezwingen konnte, gab er ihm einen
Schlag auf das Hüftgelenk; dadurch wurde Jakobs Hüftgelenk während
seines Ringens mit ihm verrenkt\textless sup title=``oder:
ausgerenkt''\textgreater✲. \bibleverse{27} Da sagte jener: »Laß mich
los, denn die Morgenröte ist schon heraufgezogen!« Jakob aber
antwortete: »Ich lasse dich nicht los, bevor du mich gesegnet hast.«
\bibleverse{28} Da fragte jener ihn: »Wie heißt du?« Er antwortete:
»Jakob.« \bibleverse{29} Da sagte er: »Du sollst hinfort nicht mehr
Jakob heißen, sondern ›Israel‹\textless sup title=``d.h. Streiter
Gottes, Gotteskämpfer''\textgreater✲; denn du hast mit Gott und mit
Menschen gekämpft und bist Sieger geblieben.« \bibleverse{30} Da
richtete Jakob die Bitte an ihn: »Teile mir doch deinen Namen mit!« Er
aber erwiderte: »Warum willst du meinen Namen wissen?« Hierauf segnete
er ihn dort. \bibleverse{31} Jakob nannte dann jenen Ort
›Pniel‹\textless sup title=``d.h. Angesicht Gottes''\textgreater✲;
»denn«, sagte er, »ich habe Gott von Angesicht zu Angesicht gesehen und
bin doch am Leben geblieben«. \bibleverse{32} Als er dann an
Pniel\textless sup title=``oder: Pnuel''\textgreater✲ vorübergezogen
war, ging ihm die Sonne auf; er hinkte aber an seiner Hüfte.
\bibleverse{33} Darum essen die Israeliten bis auf den heutigen Tag den
Muskel nicht, der über der Hüftpfanne liegt, weil er dem Jakob einen
Schlag auf die Hüftpfanne, den Hüftmuskel, versetzt hatte.

\hypertarget{f-jakobs-aussuxf6hnung-mit-esau-seine-niederlassung-bei-sichem}{%
\paragraph{f) Jakobs Aussöhnung mit Esau; seine Niederlassung bei
Sichem}\label{f-jakobs-aussuxf6hnung-mit-esau-seine-niederlassung-bei-sichem}}

\hypertarget{aa-jakobs-demut-und-esaus-herzlichkeit}{%
\subparagraph{aa) Jakobs Demut und Esaus
Herzlichkeit}\label{aa-jakobs-demut-und-esaus-herzlichkeit}}

\hypertarget{section-32}{%
\section{33}\label{section-32}}

\bibleverse{1} Als nun Jakob aufblickte und seinen Bruder Esau mit
vierhundert Mann herankommen sah, verteilte er die Kinder auf Lea, auf
Rahel und auf die beiden Leibmägde, \bibleverse{2} und zwar stellte er
die Leibmägde mit ihren Kindern vornan, dann Lea mit ihren Kindern
hinter sie und Rahel mit Joseph zuletzt. \bibleverse{3} Er selbst aber
ging vor ihnen her und verneigte sich siebenmal bis zur Erde, bis er
nahe an seinen Bruder herangekommen war. \bibleverse{4} Esau aber eilte
ihm entgegen und umarmte ihn, fiel ihm um den Hals und küßte ihn, und
sie weinten beide. \bibleverse{5} Als Esau dann aufblickte und die
Frauen mit den Kindern gewahrte, fragte er: »Wer sind diese da bei dir?«
Er antwortete: »Es sind die Kinder, mit denen Gott deinen Knecht
gesegnet hat.« \bibleverse{6} Da traten die beiden Leibmägde mit ihren
Kindern herzu und verneigten sich; \bibleverse{7} dann trat auch Lea mit
ihren Kindern herzu, indem sie sich verneigten; zuletzt traten Joseph
und Rahel herzu und verneigten sich. \bibleverse{8} Hierauf fragte Esau
weiter: »Was hat denn dieser ganze Zug (des Viehs) zu bedeuten, auf den
ich gestoßen bin?« Jakob antwortete: »Ich wollte dadurch die Gunst
meines Herrn gewinnen.« \bibleverse{9} Da sagte Esau: »Ich habe Besitz
genug, lieber Bruder: behalte, was dir gehört!« \bibleverse{10} Aber
Jakob erwiderte: »Ach nein! Wenn du mir eine Liebe erweisen willst, so
nimm mein Geschenk von mir an! Denn als ich dein Angesicht sah, war es
mir, als hätte ich Gottes Angesicht gesehen: so freundlich hast du mich
angesehen\textless sup title=``oder: aufgenommen''\textgreater✲.
\bibleverse{11} Nimm doch mein Bewillkommnungsgeschenk an, das dir
überbracht worden ist! Gott hat mich ja reich gesegnet, und ich habe
alles vollauf.« So nötigte er ihn mit Bitten, bis er es annahm.

\hypertarget{bb-jakob-lehnt-das-geleit-esaus-ab-dieser-kehrt-nach-seir-zuruxfcck}{%
\subparagraph{bb) Jakob lehnt das Geleit Esaus ab; dieser kehrt nach
Seir
zurück}\label{bb-jakob-lehnt-das-geleit-esaus-ab-dieser-kehrt-nach-seir-zuruxfcck}}

\bibleverse{12} Hierauf sagte Esau: »Laß uns nun aufbrechen und
weiterziehen! Ich will vor\textless sup title=``oder:
neben''\textgreater✲ dir herziehen.« \bibleverse{13} Aber Jakob
antwortete ihm: »Mein Herr sieht selbst, daß die Kinder noch zart sind
und daß ich noch Bedacht auf die säugenden Schafe und Kühe nehmen muß;
wenn man diese auch nur einen Tag übertriebe\textless sup title=``=~mit
Gewalt triebe''\textgreater✲, so würde die ganze Herde zugrunde gehen.
\bibleverse{14} Mein Herr wolle doch seinem Knecht vorausziehen; ich
aber will ganz langsam weiterziehen, wie eben das Vieh, das ich zu
treiben habe, und die Kinder fortkommen können, bis ich zu meinem Herrn
nach Seir gelange.« \bibleverse{15} Da sagte Esau: »So will ich
wenigstens einen Teil meiner Leute bei dir zurücklassen.« Doch er
antwortete: »Wozu das? Möchte ich nur Gnade in den Augen meines Herrn
finden!« \bibleverse{16} So kehrte denn Esau an jenem Tage um und zog
seines Weges nach Seir zurück.

\hypertarget{cc-jakob-zieht-nach-sukkoth-weiter-und-luxe4uxdft-sich-bei-sichem-nieder}{%
\subparagraph{cc) Jakob zieht nach Sukkoth weiter und läßt sich bei
Sichem
nieder}\label{cc-jakob-zieht-nach-sukkoth-weiter-und-luxe4uxdft-sich-bei-sichem-nieder}}

\bibleverse{17} Jakob aber brach nach Sukkoth auf, wo er sich ein Haus
baute und für sein Vieh Ställe errichtete; daher erhielt der Ort den
Namen Sukkoth\textless sup title=``d.h. Hütten, Ställe''\textgreater✲.
\bibleverse{18} Darauf kam Jakob bei seiner Rückkehr aus
Nord-Mesopotamien wohlbehalten nach der Stadt Sichem, die im Lande
Kanaan liegt, und schlug dort östlich von der Stadt sein Lager auf.
\bibleverse{19} Das Stück Land aber, auf dem er sein Zelt aufgeschlagen
hatte, kaufte er von den Söhnen Hemors, des Vaters Sichems, für hundert
Silberstücke; \bibleverse{20} und er baute dort einen Altar, den er
›Allgott ist der Gott Israels‹ nannte.

\hypertarget{g-dinas-entehrung-durch-den-hewiter-sichem-das-blutbad-in-sichem-als-rache-der-bruxfcder-dinas}{%
\paragraph{g) Dinas Entehrung durch den Hewiter Sichem; das Blutbad in
Sichem als Rache der Brüder
Dinas}\label{g-dinas-entehrung-durch-den-hewiter-sichem-das-blutbad-in-sichem-als-rache-der-bruxfcder-dinas}}

\hypertarget{aa-sichems-vergehen-an-dina-und-sein-darauffolgendes-ehrenwertes-verhalten}{%
\subparagraph{aa) Sichems Vergehen an Dina und sein darauffolgendes
ehrenwertes
Verhalten}\label{aa-sichems-vergehen-an-dina-und-sein-darauffolgendes-ehrenwertes-verhalten}}

\hypertarget{section-33}{%
\section{34}\label{section-33}}

\bibleverse{1} Als Dina, die Tochter Jakobs, welche Lea ihm geboren
hatte, einst ausging, um sich unter den Mädchen des Landes umzusehen,
\bibleverse{2} da sah Sichem sie, der Sohn des Hewiters Hemor, des
Landesfürsten; der ergriff sie und tat ihr Gewalt an. \bibleverse{3}
Sein Herz hing aber an Dina, der Tochter Jakobs; er gewann das Mädchen
lieb und redete ihr freundlich zu. \bibleverse{4} Er sagte daher zu
seinem Vater Hemor: »Nimm✲ mir dieses Mädchen zur Frau!« \bibleverse{5}
Nun hatte Jakob zwar von der Entehrung seiner Tochter Dina Kunde
erhalten; weil aber seine Söhne gerade bei seinem Vieh auf dem Felde
waren, verhielt er sich ruhig, bis sie heimkamen.

\hypertarget{bb-hemor-wirbt-um-dina-bei-den-suxf6hnen-jakobs}{%
\subparagraph{bb) Hemor wirbt um Dina bei den Söhnen
Jakobs}\label{bb-hemor-wirbt-um-dina-bei-den-suxf6hnen-jakobs}}

\bibleverse{6} Da begab sich Hemor, der Vater Sichems, zu Jakob hinaus,
um mit ihm zu reden. \bibleverse{7} Als nun die Söhne Jakobs vom Felde
heimgekommen waren und von dem Vorfall hörten, fühlten die Männer sich
schwer gekränkt und gerieten in lodernden Zorn; denn eine Schandtat
hatte Sichem an Israel durch die Entehrung der Tochter Jakobs verübt:
derartiges hätte nicht geschehen dürfen! \bibleverse{8} Da sprach sich
Hemor folgendermaßen gegen sie aus: »Mein Sohn Sichem hat sein Herz an
eure Tochter gehängt; gebt sie ihm doch zur Frau \bibleverse{9} und
verschwägert euch (überhaupt) mit uns: gebt uns eure Töchter und nehmt
euch die unsrigen \bibleverse{10} und bleibt bei uns wohnen! Das Land
soll euch zur Verfügung stehen: bleibt darin wohnen und durchzieht es
und macht euch darin ansässig!« \bibleverse{11} Sichem aber sagte zu
ihrem\textless sup title=``d.h. Dinas''\textgreater✲ Vater und zu ihren
Brüdern: »Gewährt mir doch meine Bitte! Alles, was ihr von mir verlangt,
will ich geben! \bibleverse{12} Mögt ihr auch noch so viel als
Heiratsgabe und Brautgeschenk von mir fordern: ich will euch geben,
soviel ihr von mir verlangt; nur gebt mir das Mädchen zur Frau!«

\hypertarget{cc-die-forderung-der-suxf6hne-jakobs-wird-von-den-sichemiten-angenommen-und-ausgefuxfchrt}{%
\subparagraph{cc) Die Forderung der Söhne Jakobs wird von den Sichemiten
angenommen und
ausgeführt}\label{cc-die-forderung-der-suxf6hne-jakobs-wird-von-den-sichemiten-angenommen-und-ausgefuxfchrt}}

\bibleverse{13} Da gaben die Söhne Jakobs dem Sichem und seinem Vater
Hemor eine arglistige Antwort -- es redeten nämlich die, deren
Vollschwester Dina er entehrt hatte --; \bibleverse{14} sie sagten
nämlich zu ihnen: »Darauf können wir uns nicht einlassen, unsere
Schwester einem unbeschnittenen Manne zu geben; denn das wäre eine
Schande für uns. \bibleverse{15} Nur unter der Bedingung wollen wir euch
zu Willen sein, wenn ihr so werden wollt, wie wir sind, nämlich wenn
alles, was männlichen Geschlechts bei euch ist, sich beschneiden läßt.
\bibleverse{16} In diesem Falle wollen wir euch unsere Töchter geben und
auch eure Töchter uns zu Frauen nehmen und wollen bei euch wohnen
bleiben und zu einem Volk mit euch werden. \bibleverse{17} Wollt ihr
aber auf unsern Vorschlag, euch beschneiden zu lassen, nicht eingehen,
so nehmen wir unsere Tochter und ziehen von hier weg.« \bibleverse{18}
Ihr Vorschlag gefiel dem Hemor und seinem Sohne Sichem, \bibleverse{19}
und der junge Mann zögerte nicht, darauf einzugehen; denn er war in die
Tochter Jakobs verliebt; auch war er der Angesehenste in der ganzen
Familie seines Vaters. \bibleverse{20} So gingen denn Hemor und sein
Sohn Sichem auf den Marktplatz ihrer Stadt, besprachen sich mit ihren
Mitbürgern und sagten: \bibleverse{21} »Diese Männer sind friedlich
gegen uns gesinnt; so mögen sie bei uns im Lande wohnen bleiben und
darin umherziehen; das Land ist ja groß genug für sie nach allen Seiten
hin. Wir wollen uns ihre Töchter zu Frauen nehmen und ihnen unsere
Töchter geben. \bibleverse{22} Jedoch nur unter der Bedingung sind die
Männer gewillt, bei uns zu bleiben und ein Volk mit uns zu bilden, wenn
alles, was männlichen Geschlechts bei uns ist, beschnitten wird, wie sie
selbst beschnitten sind; \bibleverse{23} ihre Herden, ihr Hab und Gut
und all ihr Vieh würden alsdann uns gehören. Ja, wir wollen ihre
Forderung annehmen, damit sie bei uns wohnen bleiben.« \bibleverse{24}
Der Vorschlag Hemors und seines Sohnes Sichem fand die Zustimmung aller,
die zum Tor seiner Stadt hinauszugehen\textless sup title=``oder: aus-
und einzugehen''\textgreater✲ pflegten: alle männlichen Personen, alle,
die zum Tor seiner Stadt hinauszugehen pflegten, ließen sich
beschneiden.

\hypertarget{dd-die-hinterlistige-rache-der-suxf6hne-jakobs-besonders-simeons-und-levis}{%
\subparagraph{dd) Die hinterlistige Rache der Söhne Jakobs (besonders
Simeons und
Levis)}\label{dd-die-hinterlistige-rache-der-suxf6hne-jakobs-besonders-simeons-und-levis}}

\bibleverse{25} Am dritten Tage aber, als sie im Wundfieber lagen, da
nahmen die beiden Söhne Jakobs, Simeon und Levi, die Vollbrüder der
Dina, jeder sein Schwert, drangen in die Stadt ein, die nichts Böses
ahnte, und erschlugen alles Männliche; \bibleverse{26} auch Hemor und
seinen Sohn Sichem erschlugen sie mit der Schärfe des Schwertes, holten
Dina aus dem Hause Sichems und zogen davon. \bibleverse{27} Die
(übrigen) Söhne Jakobs fielen dann über die Erschlagenen her und
plünderten die Stadt, weil man ihre Schwester entehrt hatte.
\bibleverse{28} Ihr Kleinvieh, ihre Rinder und Esel, sowohl was in der
Stadt als auch was draußen auf dem Felde war, nahmen sie weg,
\bibleverse{29} überhaupt ihren gesamten Besitz, auch alle ihre Kinder
und Frauen führten sie als Gefangene und als Beute weg und raubten
alles, was in den Häusern war.

\hypertarget{ee-jakobs-unwille-uxfcber-die-verwerfliche-tat-seiner-suxf6hne}{%
\subparagraph{ee) Jakobs Unwille über die verwerfliche Tat seiner
Söhne}\label{ee-jakobs-unwille-uxfcber-die-verwerfliche-tat-seiner-suxf6hne}}

\bibleverse{30} Da sagte Jakob zu Simeon und Levi: »Ihr habt mich ins
Unglück gestürzt, indem ihr mich bei den Bewohnern des Landes, den
Kanaanäern und Pherissitern, tödlich verhaßt gemacht habt, und ich bilde
doch nur ein leicht zählbares Häuflein. Wenn sie sich jetzt gegen mich
zusammentun, so werden sie mich erschlagen, und ich gehe samt meinem
Hause zugrunde!« \bibleverse{31} Sie aber antworteten: »Durfte
er\textless sup title=``oder: man''\textgreater✲ denn mit unserer
Schwester wie mit einer Dirne verfahren?«

\hypertarget{h-jakobs-wanderung-von-sichem-uxfcber-bethel-nach-hebron-benjamins-geburt-rahels-und-isaaks-tod}{%
\paragraph{h) Jakobs Wanderung von Sichem über Bethel nach Hebron;
Benjamins Geburt, Rahels und Isaaks
Tod}\label{h-jakobs-wanderung-von-sichem-uxfcber-bethel-nach-hebron-benjamins-geburt-rahels-und-isaaks-tod}}

\hypertarget{aa-auf-gottes-mahnung-bricht-jakob-von-sichem-auf}{%
\subparagraph{aa) Auf Gottes Mahnung bricht Jakob von Sichem
auf}\label{aa-auf-gottes-mahnung-bricht-jakob-von-sichem-auf}}

\hypertarget{section-34}{%
\section{35}\label{section-34}}

\bibleverse{1} Da gebot Gott dem Jakob: »Mache dich auf, ziehe nach
Bethel hinauf, nimm dort deinen Wohnsitz und errichte dort einen Altar
für den Gott, der dir erschienen ist, als du vor deinem Bruder Esau
flohst!« \bibleverse{2} Da befahl Jakob seiner Familie und allen
anderen, die bei ihm waren: »Schafft die fremden Götter weg, die ihr bei
euch habt, reinigt euch und legt andere Kleider an! \bibleverse{3} Wir
wollen aufbrechen und nach Bethel hinaufziehen: dort will ich einen
Altar errichten dem Gott, der mich zur Zeit meiner Not erhört hat und
auf dem Wege, den ich gezogen bin, mit mir gewesen ist.« \bibleverse{4}
Darauf übergaben sie dem Jakob alle fremden Götter, die in ihrem Besitz
waren, und ebenso ihre Ohrringe, und Jakob vergrub sie unter der
Terebinthe, die bei Sichem steht. \bibleverse{5} Als sie dann
aufbrachen, fiel ein Schrecken Gottes auf die Ortschaften der ganzen
Umgegend, so daß sie die Söhne Jakobs nicht verfolgten.

\hypertarget{bb-jakobs-ankunft-und-altarbau-in-bethel-deboras-tod}{%
\subparagraph{bb) Jakobs Ankunft und Altarbau in Bethel; Deboras
Tod}\label{bb-jakobs-ankunft-und-altarbau-in-bethel-deboras-tod}}

\bibleverse{6} So kam denn Jakob nach Lus, das im Lande Kanaan liegt --
das ist Bethel --, er samt allen Leuten, die sich bei ihm befanden.
\bibleverse{7} Er baute dort einen Altar und nannte die Stätte ›der Gott
von Bethel‹, weil Gott sich ihm dort geoffenbart hatte, als er vor
seinem Bruder floh. \bibleverse{8} Damals starb Debora, die Amme der
Rebekka, und wurde unterhalb Bethels unter der Eiche begraben, die
seitdem die ›Klageeiche‹ heißt.

\hypertarget{cc-jakob-von-gott-gesegnet-und-nochmals-vgl.-3229-durch-den-namen-israel-geehrt-weihung-eines-steindenkmals-in-bethel}{%
\subparagraph{cc) Jakob von Gott gesegnet und nochmals (vgl. 32,29)
durch den Namen Israel geehrt; Weihung eines Steindenkmals in
Bethel}\label{cc-jakob-von-gott-gesegnet-und-nochmals-vgl.-3229-durch-den-namen-israel-geehrt-weihung-eines-steindenkmals-in-bethel}}

\bibleverse{9} Da erschien Gott dem Jakob zum zweitenmal seit seiner
Rückkehr aus Nord-Mesopotamien und segnete ihn; \bibleverse{10} und Gott
sagte zu ihm: »Dein Name ist Jakob; aber künftig sollst du nicht mehr
Jakob heißen, sondern ›Israel‹ soll dein Name sein«; so gab er ihm den
Namen Israel✲. \bibleverse{11} Weiter sagte Gott zu ihm: »Ich bin der
allmächtige Gott; sei fruchtbar und mehre dich! Ein Volk, ja eine ganze
Menge\textless sup title=``oder: Schar, Gemeinde''\textgreater✲ von
Völkern soll aus dir werden\textless sup title=``oder: von dir
stammen''\textgreater✲, und Könige sollen unter deinen leiblichen
Nachkommen sein. \bibleverse{12} Und das Land, das ich Abraham und Isaak
gegeben habe, will ich dir geben und es auch deiner Nachkommenschaft
nach dir verleihen.« \bibleverse{13} Hierauf fuhr Gott von ihm in die
Höhe empor an der Stätte, wo er mit ihm geredet hatte. \bibleverse{14}
Da errichtete Jakob einen Denkstein an der Stätte, wo er mit ihm geredet
hatte, ein Denkmal von Stein, und goß ein Trankopfer auf dasselbe aus
und begoß es mit Öl. \bibleverse{15} Und Jakob nannte die Stätte, wo
Gott mit ihm geredet hatte, ›Bethel‹\textless sup title=``d.h. Haus
Gottes''\textgreater✲.

\hypertarget{dd-aufbruch-von-bethel-rahel-stirbt-bei-der-geburt-benjamins}{%
\subparagraph{dd) Aufbruch von Bethel; Rahel stirbt bei der Geburt
Benjamins}\label{dd-aufbruch-von-bethel-rahel-stirbt-bei-der-geburt-benjamins}}

\bibleverse{16} Hierauf zogen sie von Bethel weiter; und als sie nur
noch eine Strecke Weges bis nach Ephrath zu gehen hatten, wurde Rahel
von Geburtswehen befallen und hatte eine schwere Niederkunft.
\bibleverse{17} Als sie nun bei der Geburt schwer zu leiden hatte, sagte
die Wehmutter zu ihr: »Laß dir nicht bange sein! Denn du wirst auch
diesmal einen Sohn haben.« \bibleverse{18} Als ihr dann aber die
Seele\textless sup title=``=~das Leben''\textgreater✲ entfloh -- denn
sie mußte sterben --, nannte sie ihn ›Benoni‹\textless sup title=``d.h.
mein Schmerzenskind oder: Unglückssohn''\textgreater✲, sein Vater aber
nannte ihn ›Benjamin‹\textless sup title=``d.h.
Glückssohn''\textgreater✲. \bibleverse{19} So starb Rahel und wurde auf
dem Wege nach Ephrath, das jetzt Bethlehem heißt, begraben.
\bibleverse{20} Jakob errichtete dann einen Denkstein auf ihrem Grabe;
das ist der Denkstein, der auf dem Grabe Rahels noch heute steht.

\hypertarget{ee-rubens-schandtat-jakobs-zwuxf6lf-suxf6hne-seine-heimkehr-nach-hebron-isaaks-tod-und-begruxe4bnis}{%
\subparagraph{ee) Rubens Schandtat; Jakobs zwölf Söhne; seine Heimkehr
nach Hebron; Isaaks Tod und
Begräbnis}\label{ee-rubens-schandtat-jakobs-zwuxf6lf-suxf6hne-seine-heimkehr-nach-hebron-isaaks-tod-und-begruxe4bnis}}

\bibleverse{21} Hierauf zog Israel weiter und schlug sein Zelt jenseits
von Migdal-Eder\textless sup title=``d.h. Herdenturm''\textgreater✲ auf.
\bibleverse{22} Während nun Israel sich in jener Gegend aufhielt, ging
Ruben hin und verging sich mit Bilha, dem Nebenweibe seines Vaters; und
Israel erfuhr es.~--

Jakob hatte aber zwölf Söhne. \bibleverse{23} Die Söhne der Lea waren:
Ruben, der Erstgeborene Jakobs, ferner Simeon, Levi, Juda, Issaschar und
Sebulon. \bibleverse{24} Die Söhne der Rahel waren: Joseph und Benjamin.
\bibleverse{25} Die Söhne der Bilha, der Leibmagd Rahels, waren: Dan und
Naphthali, \bibleverse{26} und die Söhne der Silpa, der Leibmagd Leas,
waren: Gad und Asser. Dies sind die Söhne Jakobs, die ihm in
Nord-Mesopotamien geboren worden waren.

\bibleverse{27} Jakob kam dann zu seinem Vater Isaak nach Mamre, der
Stadt Arbas, das ist Hebron, woselbst Abraham und Isaak als Fremdlinge
gewohnt hatten. \bibleverse{28} Isaak brachte aber sein Leben auf
hundertundachtzig Jahre; \bibleverse{29} da verschied Isaak und starb
und wurde zu seinen Stammesgenossen versammelt, alt und lebenssatt; und
seine Söhne Esau und Jakob begruben ihn.

\hypertarget{i-esaus-familie-und-nachkommen-und-die-bewohner-von-seir}{%
\paragraph{i) Esaus Familie und Nachkommen und die Bewohner von
Seir}\label{i-esaus-familie-und-nachkommen-und-die-bewohner-von-seir}}

\hypertarget{aa-esaus-familie-und-wohnsitz}{%
\subparagraph{aa) Esaus Familie und
Wohnsitz}\label{aa-esaus-familie-und-wohnsitz}}

\hypertarget{section-35}{%
\section{36}\label{section-35}}

\bibleverse{1} Dies sind die Nachkommen Esaus, das ist
Edoms\textless sup title=``vgl. 25,30''\textgreater✲. \bibleverse{2}
Esau hatte seine Frauen aus den Kanaanäerinnen genommen, nämlich Ada,
die Tochter des Hethiters Elon, und Oholibama, die Tochter Anas, die
Enkelin des Hewiters Zibeon✲, \bibleverse{3} und Basmath, die Tochter
Ismaels, die Schwester Nebajoths✲. \bibleverse{4} Ada gebar dann dem
Esau den Eliphas, und Basmath gebar den Reguel; \bibleverse{5} und
Oholibama gebar Jehus und Jaglam und Korah. Dies sind die Söhne Esaus,
die ihm im Lande Kanaan geboren wurden. \bibleverse{6} Dann nahm Esau
seine Frauen, seine Söhne und Töchter, überhaupt alle Personen, die zu
seinem Hause gehörten, dazu auch seinen Besitz, sowohl all sein Vieh als
auch seine ganze Habe, die er im Lande Kanaan erworben hatte, und zog
von seinem Bruder Jakob weg in ein anderes Land (Seir); \bibleverse{7}
denn ihr Besitz war zu groß, als daß sie hätten beieinanderbleiben
können, und das Land, in dem sie als Fremdlinge wohnten, reichte für sie
wegen der Menge ihrer Herden nicht aus. \bibleverse{8} So ließ sich denn
Esau im Gebirge Seir nieder: Esau, das ist Edom.

\hypertarget{bb-esaus-suxf6hne-und-enkel-als-stammvuxe4ter}{%
\subparagraph{bb) Esaus Söhne und Enkel als
Stammväter}\label{bb-esaus-suxf6hne-und-enkel-als-stammvuxe4ter}}

\bibleverse{9} Dies sind die Nachkommen Esaus, des Stammvaters der
Edomiter im Gebirge Seir. \bibleverse{10} Dies sind die Namen der Söhne
Esaus: Eliphas, der Sohn der Ada, der Frau Esaus, Reguel, der Sohn der
Basmath, der Frau Esaus. \bibleverse{11} Die Söhne des Eliphas waren:
Theman, Omar, Zepho, Gaetham und Kenas. \bibleverse{12} Thimna aber war
ein Nebenweib des Eliphas, des Sohnes Esaus; die gebar dem Eliphas den
Amalek. Dies sind die Nachkommen der Ada, der Frau Esaus.
\bibleverse{13} Die Söhne Reguels aber sind diese: Nahath und Serah,
Samma und Missa. Dies waren die Nachkommen der Basmath, der Frau Esaus.
\bibleverse{14} Und die Söhne von Esaus Frau Oholibama, der Tochter
Anas, der Enkelin Zibeons, die sie dem Esau gebar, waren diese: Jehus,
Jaglam und Korah.

\hypertarget{cc-die-von-esau-abstammenden-gauhuxe4uptlinge}{%
\subparagraph{cc) Die von Esau abstammenden
Gauhäuptlinge}\label{cc-die-von-esau-abstammenden-gauhuxe4uptlinge}}

\bibleverse{15} Dies sind die Gaufürsten\textless sup title=``oder:
Häuptlinge''\textgreater✲ unter den Nachkommen Esaus: Die Söhne des
Eliphas, des erstgeborenen Sohnes Esaus, waren: der Häuptling Theman,
der Häuptling Omar, der Häuptling Zepho, der Häuptling Kenas,
\bibleverse{16} der Häuptling Korah, der Häuptling Gaetham, der
Häuptling Amalek. Dies sind die Häuptlinge, die von Eliphas im Lande
Edom abstammen, die Abkömmlinge der Ada. \bibleverse{17} Und dies sind
die Söhne Reguels, des Sohnes Esaus: der Häuptling Nahath, der Häuptling
Serah, der Häuptling Samma, der Häuptling Missa. Dies sind die
Häuptlinge, die von Reguel im Lande Edom abstammen, die Abkömmlinge der
Basmath, der Frau Esaus. \bibleverse{18} Und dies sind die Söhne der
Oholibama, der Frau Esaus: der Häuptling Jehus, der Häuptling Jaglam,
der Häuptling Korah. Dies sind die Häuptlinge, die von Esaus Frau
Oholibama, der Tochter Anas, abstammen. \bibleverse{19} Dies sind die
Söhne Esaus und dies ihre Häuptlinge: das ist Edom.

\hypertarget{dd-angaben-uxfcber-die-von-esau-unabhuxe4ngigen-horiter}{%
\subparagraph{dd) Angaben über die von Esau unabhängigen
Horiter}\label{dd-angaben-uxfcber-die-von-esau-unabhuxe4ngigen-horiter}}

\bibleverse{20} Dies sind die Söhne des Horiters Seir, die früher im
Lande wohnten: Lotan, Sobal, Zibeon, Ana, \bibleverse{21} Dison, Ezer
und Disan. Dies sind die Gaufürsten\textless sup title=``oder:
Häuptlinge''\textgreater✲ der Horiter, der Söhne Seirs, im Lande Edom.
\bibleverse{22} Die Söhne Lotans waren: Hori und Hemam, und Lotans
Schwester hieß Thimna. \bibleverse{23} Und dies sind die Söhne Sobals:
Alwan und Manahath und Ebal, Sepho und Onam. \bibleverse{24} Und dies
sind die Söhne Zibeons: Ajja und Ana. Das ist derselbe Ana, der die
heißen Quellen✲ in der Wüste entdeckte, als er seinem Vater Zibeon die
Esel hütete. \bibleverse{25} Die Söhne Anas aber sind diese: Dison; und
Oholibama war die Tochter Anas. \bibleverse{26} Und dies sind die Söhne
Disons: Hemdan, Esban, Jithran und Cheran. \bibleverse{27} Dies sind die
Söhne Ezers: Bilhan, Saawan und Akan. \bibleverse{28} Dies sind die
Söhne Disans: Uz und Aran. \bibleverse{29} Dies sind die Häuptlinge der
Horiter: der Häuptling Lotan, der Häuptling Sobal, der Häuptling Zibeon,
der Häuptling Ana, \bibleverse{30} der Häuptling Dison, der Häuptling
Ezer, der Häuptling Disan. Da sind die Häuptlinge der Horiter nach ihren
Stämmen\textless sup title=``oder: Gauen''\textgreater✲ im Lande Seir.

\hypertarget{ee-die-kuxf6nige-im-lande-edom-bis-auf-david}{%
\subparagraph{ee) Die Könige im Lande Edom bis auf
David}\label{ee-die-kuxf6nige-im-lande-edom-bis-auf-david}}

\bibleverse{31} Und dies sind die Könige, die im Lande Edom geherrscht
haben, ehe ein König der Israeliten geherrscht hat: \bibleverse{32}
Bela, der Sohn Beors, war König in Edom, und seine Stadt✲ hieß Dinhaba.
\bibleverse{33} Nach Belas Tode wurde König an seiner Statt Jobab, der
Sohn Serahs, aus Bozra. \bibleverse{34} Nach Jobabs Tode wurde König an
seiner Statt Husam aus der Landschaft der Themaniter. \bibleverse{35}
Nach Husams Tode wurde König an seiner Statt Hadad, der Sohn Bedads, der
die Midianiter auf der Hochebene der Moabiter besiegte; seine Stadt hieß
Awith. \bibleverse{36} Nach Hadads Tode wurde Samla aus Masreka König an
seiner Statt. \bibleverse{37} Nach Samlas Tode wurde Saul aus Rehoboth
am Euphratstrom König an seiner Statt. \bibleverse{38} Nach Sauls Tode
wurde König an seiner Statt Baal-Hanan, der Sohn Achbors.
\bibleverse{39} Als Baal-Hanan, der Sohn Achbors, starb, wurde König an
seiner Statt Hadar, dessen Hauptstadt Pagu hieß; seine Frau hieß
Mehetabeel und war die Tochter Matreds, die Enkelin Mesahabs.

\hypertarget{ff-die-gauhuxe4uptlinge-edoms-nach-ihren-wohnsitzen}{%
\subparagraph{ff) Die Gauhäuptlinge Edoms nach ihren
Wohnsitzen}\label{ff-die-gauhuxe4uptlinge-edoms-nach-ihren-wohnsitzen}}

\bibleverse{40} Dies sind die Namen der Gaufürsten\textless sup
title=``oder: Hauptleute''\textgreater✲ Esaus nach ihren Geschlechtern,
nach ihren Wohnplätzen und mit ihren Namen: der Häuptling von Thimna,
der Häuptling von Alwa, der Häuptling von Jetheth, \bibleverse{41} der
Häuptling von Oholibama, der Häuptling von Ela, der Häuptling von Pinon,
\bibleverse{42} der Häuptling von Kenas, der Häuptling von Theman, der
Häuptling von Mibzar, \bibleverse{43} der Häuptling von Magdiel, der
Häuptling von Iram. Das sind die Häuptlinge der Edomiter nach ihren
Wohnsitzen in dem Lande, das sie in Besitz genommen hatten. Das ist
Esau, der Stammvater der Edomiter.

\hypertarget{k-josephs-jugend-und-verkauf-nach-uxe4gypten}{%
\paragraph{k) Josephs Jugend und Verkauf nach
Ägypten}\label{k-josephs-jugend-und-verkauf-nach-uxe4gypten}}

\hypertarget{aa-die-anfuxe4nge-der-feindschaft-der-bruxfcder-gegen-joseph}{%
\subparagraph{aa) Die Anfänge der Feindschaft der Brüder gegen
Joseph}\label{aa-die-anfuxe4nge-der-feindschaft-der-bruxfcder-gegen-joseph}}

\hypertarget{section-36}{%
\section{37}\label{section-36}}

\bibleverse{1} Jakob aber blieb in dem Lande wohnen, in dem sich sein
Vater als Fremdling aufgehalten hatte, im Lande Kanaan. \bibleverse{2}
Dies ist die Geschichte Jakobs: Als Joseph siebzehn Jahre alt war,
hütete er das Kleinvieh mit seinen Brüdern, und zwar war er als junger
Bursche bei den Söhnen der Bilha und Silpa, der Frauen seines Vaters,
und was man diesen\textless sup title=``d.h. seinen
Brüdern''\textgreater✲ Übles nachsagte, hinterbrachte er ihrem Vater.
\bibleverse{3} Israel hatte aber Joseph lieber als alle seine anderen
Söhne, weil er ihm in seinem Alter geboren war; und so ließ er ihm ein
langes Ärmelkleid machen. \bibleverse{4} Als nun seine Brüder sahen, daß
ihr Vater ihn lieber hatte als alle seine Brüder, faßten sie einen Haß
gegen ihn und gewannen es nicht über sich, ein freundliches Wort mit ihm
zu reden.

\hypertarget{bb-josephs-truxe4ume}{%
\subparagraph{bb) Josephs Träume}\label{bb-josephs-truxe4ume}}

\bibleverse{5} Einst hatte Joseph einen Traum und teilte ihn seinen
Brüdern mit; seitdem haßten sie ihn noch mehr. \bibleverse{6} Er sagte
nämlich zu ihnen: »Hört einmal diesen Traum, den ich gehabt habe!
\bibleverse{7} Wir waren gerade damit beschäftigt, Garben draußen auf
dem Felde zu binden, und denkt nur: meine Garbe richtete sich empor und
blieb auch aufrecht stehen, eure Garben aber stellten sich rings im
Kreise um sie auf und verneigten sich vor meiner Garbe.« \bibleverse{8}
Da sagten seine Brüder zu ihm: »Du möchtest wohl gern König über uns
werden oder gar Herrscher über uns sein?« Seitdem haßten sie ihn noch
mehr wegen seiner Träume und wegen seiner Reden. \bibleverse{9} Ein
andermal hatte er wieder einen Traum, den er seinen Brüdern so erzählte:
»Hört, ich habe wieder einen Traum gehabt! Denkt nur: die Sonne, der
Mond und elf Sterne verneigten sich vor mir!« \bibleverse{10} Als er das
seinem Vater und seinen Brüdern erzählte, schalt ihn sein Vater und
sagte zu ihm: »Was ist das für ein Traum, den du da gehabt hast! Meinst
du, ich und deine Mutter und deine Brüder sollen kommen und uns vor dir
zur Erde verneigen?« \bibleverse{11} So wurden denn seine Brüder
eifersüchtig auf ihn, sein Vater aber behielt das Wort\textless sup
title=``oder: Vorkommnis''\textgreater✲ im Gedächtnis.

\hypertarget{cc-die-gelegenheit-zur-beseitigung-josephs}{%
\subparagraph{cc) Die Gelegenheit zur Beseitigung
Josephs}\label{cc-die-gelegenheit-zur-beseitigung-josephs}}

\bibleverse{12} Als nun seine Brüder einst hingegangen waren, um das
Kleinvieh ihres Vaters bei Sichem zu weiden, \bibleverse{13} sagte
Israel zu Joseph: »Du weißt, deine Brüder sind auf der Weide bei Sichem:
komm, ich will dich zu ihnen schicken.« Joseph antwortete ihm: »Ich bin
bereit!« \bibleverse{14} Da sagte er zu ihm: »Gehe doch hin und sieh zu,
wie es deinen Brüdern geht und wie es um das Vieh steht, und bringe mir
Bescheid!« So sandte er ihn aus dem Tal von Hebron, und Joseph kam nach
Sichem. \bibleverse{15} Während er nun dort auf dem Felde umherirrte,
traf ihn ein Mann; der fragte ihn: »Was suchst du?« \bibleverse{16} Er
antwortete: »Meine Brüder suche ich; sage mir doch, wo sie jetzt
weiden!« \bibleverse{17} Der Mann antwortete: »Sie sind von hier
weggezogen; denn ich habe sie sagen hören: ›Wir wollen nach Dothan
gehen.‹« Da ging Joseph hinter seinen Brüdern her und fand sie bei
Dothan. \bibleverse{18} Als sie ihn nun von weitem sahen, machten sie,
ehe er noch in ihre Nähe gekommen war, einen Anschlag auf sein Leben
\bibleverse{19} und sagten zueinander: »Da kommt ja der Träumer her!
\bibleverse{20} Nun wohlan! Wir wollen ihn totschlagen und in eine der
Gruben✲ werfen und dann sagen, ein wildes Tier habe ihn gefressen; dann
werden wir ja sehen, was aus seinen Träumen wird!«

\hypertarget{dd-ruben-und-juda-suchen-joseph-zu-retten}{%
\subparagraph{dd) Ruben und Juda suchen Joseph zu
retten}\label{dd-ruben-und-juda-suchen-joseph-zu-retten}}

\bibleverse{21} Als Ruben das hörte, suchte er ihn aus ihren Händen zu
retten, indem er sagte: »Wir wollen ihn nicht totschlagen!«
\bibleverse{22} Dann sagte Ruben weiter zu ihnen: »Vergießt kein Blut!
Werft ihn in die Grube✲ dort in der Steppe, aber legt nicht Hand an
ihn!« -- er wollte ihn nämlich aus ihrer Hand retten und ihn dann wieder
zu seinem Vater bringen. \bibleverse{23} Sobald nun Joseph bei seinen
Brüdern angekommen war, zogen sie ihm seinen Rock aus, das lange
Ärmelkleid, das er anhatte, \bibleverse{24} ergriffen ihn hierauf und
warfen ihn in die Grube; die Grube war aber leer, es befand sich kein
Wasser darin. \bibleverse{25} Als sie sich dann niedergesetzt hatten, um
zu essen, und in die Ferne schauten, sahen sie eine
Karawane\textless sup title=``d.h. einen Reisezug''\textgreater✲ von
Ismaelitern, die aus Gilead herkamen und deren Kamele mit Tragakanth,
Mastix und Ladanum beladen waren; sie wollten damit nach Ägypten
hinabziehen. \bibleverse{26} Da sagte Juda zu seinen Brüdern: »Welchen
Vorteil hätten wir davon, wenn wir unsern Bruder erschlügen und seine
Ermordung verheimlichten? \bibleverse{27} Kommt, wir wollen ihn an die
Ismaeliter verkaufen, aber nicht selbst Hand an ihn legen; er ist ja
doch unser Bruder, unser Fleisch und Blut!« Seine Brüder gingen auf den
Vorschlag ein.

\hypertarget{ee-ausfuxfchrung-der-missetat-und-versuch-sie-zu-verheimlichen}{%
\subparagraph{ee) Ausführung der Missetat und Versuch, sie zu
verheimlichen}\label{ee-ausfuxfchrung-der-missetat-und-versuch-sie-zu-verheimlichen}}

\bibleverse{28} Als nun die midianitischen Kaufleute vorüberkamen, zogen
sie\textless sup title=``d.h. die Brüder''\textgreater✲ Joseph aus der
Grube herauf und verkauften ihn für zwanzig Silberstücke an die
Ismaeliter; diese brachten Joseph dann nach Ägypten. \bibleverse{29} Als
Ruben nun zu der Grube zurückkehrte und Joseph sich nicht mehr in der
Grube befand, da zerriß er seine Kleider, \bibleverse{30} kehrte zu
seinen Brüdern zurück und rief aus: »Der Knabe ist nicht mehr da! Wohin
soll ich nun gehen?« \bibleverse{31} Hierauf nahmen sie Josephs Rock,
schlachteten einen Ziegenbock und tauchten den Rock in das Blut;
\bibleverse{32} dann ließen sie das lange Ärmelkleid durch einen Boten
ihrem Vater überbringen und ihm sagen: »Dieses haben wir gefunden: sieh
doch genau zu, ob es der Rock deines Sohnes ist oder nicht!«

\hypertarget{ff-jakobs-trauer-joseph-in-uxe4gypten-an-potiphar-verkauft}{%
\subparagraph{ff) Jakobs Trauer; Joseph in Ägypten an Potiphar
verkauft}\label{ff-jakobs-trauer-joseph-in-uxe4gypten-an-potiphar-verkauft}}

\bibleverse{33} Er sah es genau an und rief aus: »Es ist der Rock meines
Sohnes! Ein wildes Tier hat ihn gefressen! Ja, ja, Joseph ist
zerfleischt worden!« \bibleverse{34} Und Jakob zerriß seine Kleider,
legte ein härenes Gewand✲ um seine Hüften und trauerte um seinen Sohn
lange Zeit. \bibleverse{35} Alle seine Söhne und alle seine Töchter
bemühten sich zwar, ihn zu trösten, aber er wies jeden Trost zurück und
sagte: »Nein, im Trauerkleid will ich zu meinem Sohn in die Unterwelt
hinabfahren!« So beweinte ihn sein Vater. \bibleverse{36} Die Midianiter
aber verkauften Joseph nach Ägypten an Potiphar, einen Hofbeamten des
Pharaos, den Obersten der Leibwächter\textless sup title=``eig. der
Scharfrichter''\textgreater✲.

\hypertarget{l-juda-und-seine-schwiegertochter-thamar}{%
\paragraph{l) Juda und seine Schwiegertochter
Thamar}\label{l-juda-und-seine-schwiegertochter-thamar}}

\hypertarget{aa-judas-wegzug-von-seinen-bruxfcdern-und-seine-mehrfachen-versuxfcndigungen-sein-und-seiner-suxf6hne-verhalten-gegen-thamar}{%
\subparagraph{aa) Judas Wegzug von seinen Brüdern und seine mehrfachen
Versündigungen; sein und seiner Söhne Verhalten gegen
Thamar}\label{aa-judas-wegzug-von-seinen-bruxfcdern-und-seine-mehrfachen-versuxfcndigungen-sein-und-seiner-suxf6hne-verhalten-gegen-thamar}}

\hypertarget{section-37}{%
\section{38}\label{section-37}}

\bibleverse{1} Um diese Zeit begab es sich, daß Juda sich von seinen
Brüdern trennte und sich an einen Mann aus Adullam namens Hira anschloß.
\bibleverse{2} Dort sah Juda die Tochter eines Kanaanäers namens Sua;
die nahm er zur Frau und lebte mit ihr. \bibleverse{3} Sie wurde guter
Hoffnung und gebar einen Sohn, den er Ger nannte. \bibleverse{4} Hierauf
wurde sie wieder guter Hoffnung und gebar einen Sohn, den sie Onan
nannte. \bibleverse{5} Sodann wurde sie nochmals Mutter eines Sohnes,
dem sie den Namen Sela gab; sie befand sich aber in Chesib, als sie ihn
gebar.~-- \bibleverse{6} Juda nahm dann für seinen erstgeborenen Sohn
Ger eine Frau namens Thamar. \bibleverse{7} Aber Ger, der Erstgeborene
Judas, zog sich das Mißfallen des HERRN zu; daher ließ der HERR ihn
sterben. \bibleverse{8} Da sagte Juda zu Onan: »Gehe zu der Frau deines
Bruders ein und leiste ihr die Schwagerpflicht\textless sup
title=``=~vollziehe mit ihr die Schwagerehe; vgl. 5.Mose
25,5-10''\textgreater✲, damit du das Geschlecht deines Bruders
fortpflanzest!« \bibleverse{9} Da Onan aber wußte, daß die Kinder nicht
als seine eigenen gelten würden, ließ er, sooft er zu der Frau seines
Bruders einging, (den Samen) zur Erde fallen, um seinem Bruder keine
Nachkommen zu verschaffen. \bibleverse{10} Dieses sein Tun mißfiel aber
dem HERRN, und so ließ er auch ihn sterben. \bibleverse{11} Da sagte
Juda zu seiner Schwiegertochter Thamar: »Bleibe als Witwe im Hause
deines Vaters wohnen, bis mein Sohn Sela herangewachsen ist!« Er
fürchtete nämlich, daß auch dieser sterben würde wie seine Brüder. So
ging denn Thamar hin und wohnte im Hause ihres Vaters.

\hypertarget{bb-thamar-verschafft-sich-durch-list-nachkommenschaft-von-ihrem-schwiegervater-juda}{%
\subparagraph{bb) Thamar verschafft sich durch List Nachkommenschaft von
ihrem Schwiegervater
Juda}\label{bb-thamar-verschafft-sich-durch-list-nachkommenschaft-von-ihrem-schwiegervater-juda}}

\bibleverse{12} Als nun geraume Zeit vergangen war, starb die Tochter
Suas, die Frau Judas; und als die Trauerzeit vorüber war, ging Juda
(einmal) mit seinem Freunde Hira, dem Adullamiter, nach Thimna hinauf,
um seine Schafe zu scheren. \bibleverse{13} Als nun der Thamar berichtet
wurde, daß ihr Schwiegervater sich gerade zur Schafschur nach Thimna
hinauf begäbe, \bibleverse{14} legte sie ihre Witwenkleidung ab, hüllte
sich dicht in einen Schleier und setzte sich an den Eingang von
Enaim\textless sup title=``d.h. Zweibrunn''\textgreater✲, das am Wege
nach Thimna liegt; denn sie hatte gesehen, daß Sela erwachsen war, ohne
daß man sie ihm zur Frau gegeben hatte. \bibleverse{15} Als nun Juda sie
da sitzen sah, hielt er sie für eine Dirne; denn sie hatte ihr Gesicht
verhüllt. \bibleverse{16} Er bog also zu ihr ab zu der Stelle des Weges
hin, wo sie saß, und sagte: »Komm her, sei mir zu Willen!« Denn er wußte
nicht, daß sie seine Schwiegertochter war. Sie antwortete: »Was willst
du mir dafür geben, wenn ich dir zu Willen bin?« \bibleverse{17} Er
sagte: »Ich will dir ein Böckchen von der Herde herschicken.« Sie
erwiderte: »Ja, wenn du mir solange ein Pfand gibst, bis du es
herschickst.« \bibleverse{18} Da fragte er: »Was für ein Pfand ist es,
das ich dir geben soll?« Sie antwortete: »Deinen Siegelring, deine
Schnur✲ und den Stab, den du da in der Hand hast.« Da gab er es ihr und
wohnte ihr bei, und sie wurde schwanger von ihm. \bibleverse{19} Hierauf
stand sie auf, entfernte sich, legte ihren Schleier ab und zog ihre
Witwenkleidung wieder an. \bibleverse{20} Juda schickte nun das Böckchen
durch seinen Freund, den Adullamiter, um das Pfand von dem Weibe
zurückzuerhalten; aber der fand sie nicht; \bibleverse{21} und als er
bei den Leuten jenes Ortes nachfragte: »Wo ist die geweihte Buhlerin,
die hier bei Enaim am Wege gesessen hat?«, antworteten sie ihm: »Hier
ist keine geweihte Buhlerin gewesen.« \bibleverse{22} So kehrte er denn
zu Juda zurück und sagte: »Ich habe sie nicht gefunden; auch haben die
Leute des Ortes gesagt, es sei dort keine geweihte Buhlerin gewesen.«
\bibleverse{23} Da erwiderte Juda: »So mag sie es für sich behalten,
damit wir uns nicht den Spott der Leute zuziehen! Du weißt ja, daß ich
dies Böckchen geschickt habe; du hast sie aber nicht gefunden.«

\hypertarget{cc-judas-gerechter-urteilsspruch-uxfcber-sich-und-thamar}{%
\subparagraph{cc) Judas gerechter Urteilsspruch über sich und
Thamar}\label{cc-judas-gerechter-urteilsspruch-uxfcber-sich-und-thamar}}

\bibleverse{24} Ungefähr drei Monate später wurde dem Juda als sicher
berichtet: »Deine Schwiegertochter Thamar hat sich verführen lassen und
ist infolge ihrer Ausschweifung schwanger geworden.« Da gebot Juda:
»Führt sie hinaus, damit sie verbrannt wird!« \bibleverse{25} Als sie
nun hinausgeführt werden sollte, schickte sie zu ihrem Schwiegervater
und ließ ihm sagen: »Von dem Manne, dem diese Sachen hier gehören, bin
ich schwanger«; und weiter ließ sie ihm sagen: »Sieh doch genau zu, wem
dieser Siegelring, diese Schnur und dieser Stab gehören!«
\bibleverse{26} Als nun Juda die Sachen genau angesehen hatte, sagte er:
»Sie ist mir gegenüber im Recht: warum habe ich sie meinem Sohne Sela
nicht zur Frau gegeben!« Er vollzog aber hinfort keine Beiwohnung mehr
mit ihr.

\hypertarget{dd-thamar-gebiert-die-zwillinge-perez-und-serah}{%
\subparagraph{dd) Thamar gebiert die Zwillinge Perez und
Serah}\label{dd-thamar-gebiert-die-zwillinge-perez-und-serah}}

\bibleverse{27} Als nun die Zeit ihrer Niederkunft da war, ergab es
sich, daß Zwillinge in ihrem Mutterschoße waren; \bibleverse{28} und bei
der Geburt streckte das eine Kind die Hand vor; da griff die Wehmutter
zu, band ihm einen roten Faden um die Hand und sagte: »Dieser ist zuerst
zum Vorschein gekommen.« \bibleverse{29} Doch (das Kind) zog seine Hand
wieder zurück, und nun kam sein Bruder zum Vorschein. Da sagte sie: »Was
für einen Riß hast du dir da gerissen!« Daher nannte man ihn
›Perez‹\textless sup title=``d.h. Riß''\textgreater✲. \bibleverse{30}
Darauf kam sein Bruder zum Vorschein, an dessen Hand der rote Faden war;
daher nannte man ihn ›Serah‹\textless sup title=``d.h. Glanz,
rot''\textgreater✲.

\hypertarget{m-joseph-im-hause-potiphars-und-im-gefuxe4ngnis}{%
\paragraph{m) Joseph im Hause Potiphars und im
Gefängnis}\label{m-joseph-im-hause-potiphars-und-im-gefuxe4ngnis}}

\hypertarget{aa-joseph-steigt-bei-dem-vornehmen-uxe4gypter-zu-hoher-dienststellung-empor}{%
\subparagraph{aa) Joseph steigt bei dem vornehmen Ägypter zu hoher
Dienststellung
empor}\label{aa-joseph-steigt-bei-dem-vornehmen-uxe4gypter-zu-hoher-dienststellung-empor}}

\hypertarget{section-38}{%
\section{39}\label{section-38}}

\bibleverse{1} Als aber Joseph nach Ägypten gebracht worden war, kaufte
ihn Potiphar, ein Ägypter, ein Hofbeamter des Pharaos, der Oberste der
Leibwächter\textless sup title=``vgl. 37,36''\textgreater✲, von den
Ismaelitern, die ihn dorthin gebracht hatten. \bibleverse{2} Gott der
HERR aber war mit Joseph, so daß ihm alles gelang, während er im Hause
seines Herrn, des Ägypters, war. \bibleverse{3} Weil nun sein Herr sah,
daß Gott mit ihm war und daß Gott alles, was er vornahm, ihm gelingen
ließ, \bibleverse{4} wandte er dem Joseph seine Gunst zu, so daß er sich
selbst von ihm bedienen ließ; dann machte er ihn zum Aufseher über sein
Hauswesen und vertraute ihm alles an, was er besaß. \bibleverse{5} Und
von der Zeit an, wo er ihn zum Aufseher über sein Haus und zum Verwalter
seines ganzen Besitzes gemacht hatte, segnete Gott das Haus des Ägypters
um Josephs willen, so daß der Segen Gottes auf allem ruhte, was er
besaß, im Hause und auf dem Felde. \bibleverse{6} Daher überließ er sein
ganzes Besitztum der Verwaltung Josephs: er selbst kümmerte sich neben
ihm um nichts mehr als um seine Mahlzeiten.

\hypertarget{bb-liebe-hauxdf-und-verleumderische-anklage-der-ehebrecherischen-gattin-potiphars}{%
\subparagraph{bb) Liebe, Haß und verleumderische Anklage der
ehebrecherischen Gattin
Potiphars}\label{bb-liebe-hauxdf-und-verleumderische-anklage-der-ehebrecherischen-gattin-potiphars}}

Joseph war aber schön von Gestalt und schön von Angesicht.
\bibleverse{7} So kam es schließlich dahin, daß die Gattin seines Herrn
ihre Augen auf Joseph richtete und ihn verführen wollte. \bibleverse{8}
Er weigerte sich aber und sagte zu der Gattin seines Herrn: »Bedenke
doch! Mein Herr kümmert sich neben mir um nichts im Hause und hat mir
alles anvertraut, was er besitzt. \bibleverse{9} Er selbst hat in diesem
Hause keine größere Geltung als ich, und nichts hat er mir vorenthalten
als dich allein, weil du ja sein Weib bist. Wie sollte ich da ein so
großes Unrecht begehen und mich gegen Gott versündigen!« \bibleverse{10}
Obgleich sie daher Tag für Tag auf Joseph einredete, hörte er doch nicht
auf sie, daß er sich zu ihr getan und sich mit ihr vergangen hätte.

\bibleverse{11} Nun begab es sich eines Tages, daß Joseph, wie
gewöhnlich, ins Haus kam, um seine Geschäfte zu besorgen, während gerade
keiner von den Hausangehörigen\textless sup title=``oder: von der
Dienerschaft''\textgreater✲ drinnen im Hause anwesend war.
\bibleverse{12} Da faßte sie ihn am Gewand mit den Worten: »Sei mir zu
Willen!« Er aber ließ sein Gewand in ihrer Hand, ergriff die Flucht und
eilte zum Hause hinaus. \bibleverse{13} Als sie nun sah, daß er sein
Gewand in ihrer Hand gelassen hatte und zum Hause hinausgeflohen war,
\bibleverse{14} rief sie die Leute ihres Hauses herbei und sagte zu
ihnen: »Seht doch! Er hat uns da einen Hebräer hereingebracht, daß er
seinen Mutwillen an uns auslasse! Der ist zu mir hereingekommen, um mich
zu verführen; ich habe aber laut geschrien, \bibleverse{15} und als er
hörte, daß ich ein lautes Geschrei erhob und um Hilfe rief, hat er sein
Gewand neben mir liegen lassen und ist zum Hause hinaus entflohen!«
\bibleverse{16} Dann ließ sie sein Gewand neben sich liegen, bis sein
Herr nach Hause kam; \bibleverse{17} und sie erzählte ihm den Vorfall
mit denselben Worten, nämlich: »Der hebräische Sklave, den du uns
hergebracht hast, ist zu mir hereingekommen, um seinen Mutwillen an mir
auszulassen; \bibleverse{18} als ich aber ein lautes Geschrei erhob und
um Hilfe rief, hat er sein Gewand neben mir liegen lassen und ist zum
Hause hinausgeflohen!«

\hypertarget{cc-ins-gefuxe4ngnis-geworfen-erwirbt-sich-joseph-die-gunst-des-oberaufsehers}{%
\subparagraph{cc) Ins Gefängnis geworfen, erwirbt sich Joseph die Gunst
des
Oberaufsehers}\label{cc-ins-gefuxe4ngnis-geworfen-erwirbt-sich-joseph-die-gunst-des-oberaufsehers}}

\bibleverse{19} Als nun sein Herr die Mitteilung seiner Frau hörte, die
ihm berichtete: »So und so hat dein Sklave sich gegen mich benommen!«,
da stieg der Zorn in ihm auf; \bibleverse{20} und der Herr Josephs ließ
ihn ergreifen und ins Gefängnis werfen, an den Ort, wo die Gefangenen
des Königs in Gewahrsam lagen; dort saß er nun im Gefängnis.
\bibleverse{21} Aber Gott der HERR war mit Joseph und ließ ihn die
Zuneigung aller gewinnen und wandte ihm auch die Gunst des obersten
Aufsehers des Gefängnisses zu. \bibleverse{22} Dieser übergab alle
Gefangenen, die sich im Gefängnis befanden, dem Joseph zur Aufsicht; und
alles, was es dort zu tun gab, hatte dieser zu besorgen. \bibleverse{23}
Der oberste Aufseher des Gefängnisses kümmerte sich um gar nichts bei
allem, was ihm\textless sup title=``d.h. dem Joseph''\textgreater✲
anvertraut war; denn Gott der HERR war mit ihm, und Gott ließ alles
gelingen, was er vornahm.

\hypertarget{n-joseph-deutet-zwei-vornehmen-gefangenen-ihre-truxe4ume}{%
\paragraph{n) Joseph deutet zwei vornehmen Gefangenen ihre
Träume}\label{n-joseph-deutet-zwei-vornehmen-gefangenen-ihre-truxe4ume}}

\hypertarget{aa-einkerkerung-des-kuxf6niglichen-mundschenken-und-des-hofbuxe4ckers-des-pharaos}{%
\subparagraph{aa) Einkerkerung des königlichen Mundschenken und des
Hofbäckers des
Pharaos}\label{aa-einkerkerung-des-kuxf6niglichen-mundschenken-und-des-hofbuxe4ckers-des-pharaos}}

\hypertarget{section-39}{%
\section{40}\label{section-39}}

\bibleverse{1} Nun begab es sich einige Zeit danach, daß der Mundschenk
und der Bäcker des Königs von Ägypten sich gegen ihren Herrn, den König
von Ägypten, verfehlten. \bibleverse{2} Da geriet der Pharao über seine
beiden Hofbeamten, den Obermundschenken und den Oberbäcker, in Zorn,
\bibleverse{3} und er ließ sie im Hause des Obersten der
Leibwächter\textless sup title=``vgl. 37,36''\textgreater✲ in Haft
legen, ins Gefängnis, an den Ort, wo auch Joseph gefangen lag.
\bibleverse{4} Der Oberste der Leibwächter aber betraute Joseph mit der
Fürsorge für sie, so daß er sie zu bedienen hatte; und so befanden sie
sich eine Zeitlang in Haft.

\hypertarget{bb-joseph-redet-den-beiden-niedergeschlagenen-hofbeamten-teilnehmend-zu}{%
\subparagraph{bb) Joseph redet den beiden niedergeschlagenen Hofbeamten
teilnehmend
zu}\label{bb-joseph-redet-den-beiden-niedergeschlagenen-hofbeamten-teilnehmend-zu}}

\bibleverse{5} Da träumten sie beide in einer und derselben Nacht einen
Traum, und zwar jeder einen eigenen Traum von besonderer Bedeutung, der
Mundschenk und der Bäcker des Königs von Ägypten, die im Kerker gefangen
saßen. \bibleverse{6} Als nun Joseph am Morgen zu ihnen hineinkam und
bemerkte, daß sie mißgestimmt waren, \bibleverse{7} fragte er sie, die
beiden Hofbeamten des Pharaos, die sich mit ihm im Hause seines Herrn in
Haft befanden: »Warum seht ihr denn heute so mißmutig aus?«
\bibleverse{8} Sie antworteten ihm: »Wir haben einen Traum gehabt, und
nun ist niemand da, der ihn uns deuten könnte.« Da sagte Joseph zu
ihnen: »Traumdeutungen sind Sache Gottes: erzählt mir doch eure Träume!«

\hypertarget{cc-der-traum-des-mundschenken-und-seine-deutung}{%
\subparagraph{cc) Der Traum des Mundschenken und seine
Deutung}\label{cc-der-traum-des-mundschenken-und-seine-deutung}}

\bibleverse{9} Da erzählte der Obermundschenk dem Joseph seinen Traum
folgendermaßen: »In meinem Traume war es mir, als ob ich einen Weinstock
vor mir stehen sähe; \bibleverse{10} an diesem Weinstock waren drei
Reben; und sowie er anfing zu treiben, brachen auch schon seine Blüten
hervor, und die Trauben brachten die Beeren zur Reife. \bibleverse{11}
Ich aber hielt den Becher des Pharaos in der Hand, nahm die Trauben,
preßte sie aus in den Becher des Pharaos und gab dann den Becher dem
Pharao in die Hand.« \bibleverse{12} Da sagte Joseph zu ihm: »Dies ist
die Deutung: Die drei Weinreben sind drei Tage; \bibleverse{13} in drei
Tagen von heute ab wird der Pharao dir das Haupt erheben, indem er dich
wieder in dein Amt einsetzt, so daß du ihm den Becher in die Hand gibst
ganz nach der früheren Weise, als du noch sein Mundschenk warst.
\bibleverse{14} Aber halte dann auch die Erinnerung an mich fest, wenn
es dir wieder gut geht, erweise mir dann die Liebe, den Pharao auf mich
aufmerksam zu machen, und bringe mich aus diesem Hause hinaus!
\bibleverse{15} Denn ich bin aus dem Lande der Hebräer heimlich
gestohlen\textless sup title=``oder: schmählich entführt''\textgreater✲
worden und habe auch hier gar nichts begangen, daß man mich in den
Kerker geworfen hat.«

\hypertarget{dd-der-traum-des-buxe4ckers-und-seine-deutung}{%
\subparagraph{dd) Der Traum des Bäckers und seine
Deutung}\label{dd-der-traum-des-buxe4ckers-und-seine-deutung}}

\bibleverse{16} Als nun der Oberbäcker sah, daß Joseph eine günstige
Deutung gegeben hatte, sagte er zu Joseph: »Auch in meinem Traume war es
mir, als trüge ich drei Körbe mit feinem Gebäck auf meinem Haupte;
\bibleverse{17} und in dem obersten Korb befanden sich allerlei Eßwaren
für den Pharao, wie sie der Bäcker herstellt; aber die Vögel fraßen sie
aus dem Korbe auf meinem Haupte weg.« \bibleverse{18} Da sagte Joseph:
»Dies ist die Deutung des Traumes: Die drei Körbe sind drei Tage;
\bibleverse{19} in drei Tagen von heute ab wird der Pharao dir das Haupt
erheben, nämlich dich an einen Baum\textless sup title=``oder:
Pfahl''\textgreater✲ hängen lassen; da werden dann die Vögel das Fleisch
von dir oben wegfressen.«

\hypertarget{ee-die-erfuxfcllung-der-beiden-truxe4ume}{%
\subparagraph{ee) Die Erfüllung der beiden
Träume}\label{ee-die-erfuxfcllung-der-beiden-truxe4ume}}

\bibleverse{20} Drei Tage später nun war der Geburtstag des Pharaos; da
veranstaltete er ein Festmahl für alle seine Diener und erhob seinem
Obermundschenken und seinem Oberbäcker das Haupt inmitten seiner Diener;
\bibleverse{21} den Obermundschenken setzte er wieder in sein
Schenkenamt ein, so daß er dem Pharao wieder den Becher zu reichen
hatte; \bibleverse{22} den Oberbäcker aber ließ er hängen, ganz so, wie
Joseph ihnen (die Träume) gedeutet hatte. \bibleverse{23} Aber der
Obermundschenk dachte nicht mehr an Joseph, sondern vergaß ihn.

\hypertarget{o-die-truxe4ume-des-pharaos-von-joseph-gedeutet-die-erhuxf6hung-und-amtsfuxfchrung-josephs}{%
\paragraph{o) Die Träume des Pharaos von Joseph gedeutet; die Erhöhung
und Amtsführung
Josephs}\label{o-die-truxe4ume-des-pharaos-von-joseph-gedeutet-die-erhuxf6hung-und-amtsfuxfchrung-josephs}}

\hypertarget{aa-die-beiden-truxe4ume-des-pharaos-sind-fuxfcr-die-uxe4gyptischen-traumdeuter-unluxf6sbar}{%
\subparagraph{aa) Die beiden Träume des Pharaos sind für die ägyptischen
Traumdeuter
unlösbar}\label{aa-die-beiden-truxe4ume-des-pharaos-sind-fuxfcr-die-uxe4gyptischen-traumdeuter-unluxf6sbar}}

\hypertarget{section-40}{%
\section{41}\label{section-40}}

\bibleverse{1} Nun begab es sich nach Verlauf von zwei vollen Jahren,
daß der Pharao einen Traum hatte: ihm war es, er stehe am Nil.
\bibleverse{2} Da sah er aus dem Strom sieben schöne, wohlgenährte Kühe
heraufsteigen und im Riedgras weiden. \bibleverse{3} Dann sah er nach
diesen sieben andere Kühe aus dem Strom heraufsteigen, die sahen häßlich
aus und waren mager am Fleisch und traten neben die anderen Kühe am Ufer
des Stromes; \bibleverse{4} hierauf fraßen die häßlichen und mageren
Kühe die sieben schönen und wohlgenährten Kühe auf. Da erwachte der
Pharao. \bibleverse{5} Als er dann wieder eingeschlafen war, hatte er
einen zweiten Traum; und zwar sah er sieben Ähren oben an einem Halme
wachsen, dicke und schöne; \bibleverse{6} nach diesen aber schossen
sieben dünne und vom Ostwind versengte Ähren hervor, \bibleverse{7} und
diese dünnen Ähren verschlangen die sieben dicken und vollen Ähren. Da
erwachte der Pharao und merkte, daß es ein (bedeutungsvoller) Traum war.
\bibleverse{8} Am Morgen fühlte er sich darüber innerlich beunruhigt, so
daß er alle Schriftkundigen✲ Ägyptens und alle Weisen des Landes rufen
ließ; er erzählte ihnen seine Träume, aber es war keiner da, der sie dem
Pharao zu deuten vermochte.

\hypertarget{bb-der-obermundschenk-veranlauxdft-das-herbeiholen-josephs}{%
\subparagraph{bb) Der Obermundschenk veranlaßt das Herbeiholen
Josephs}\label{bb-der-obermundschenk-veranlauxdft-das-herbeiholen-josephs}}

\bibleverse{9} Da nahm der Obermundschenk das Wort und sagte zum Pharao:
»Ich muß heute meine Verfehlungen in Erinnerung bringen. \bibleverse{10}
Als der Pharao (einst) über seine Diener in Zorn geraten war und mich im
Hause des Obersten der Leibwächter in Gewahrsam hatte legen lassen, mich
und den Oberbäcker, \bibleverse{11} da hatten wir beide in einer und
derselben Nacht einen Traum, ich und er, und zwar jeder einen Traum von
besonderer Bedeutung. \bibleverse{12} Nun befand sich dort ein
hebräischer junger Mann bei uns, ein Sklave des Obersten der
Leibwächter; dem erzählten wir's, und er deutete uns unsere Träume: er
gab dem Traum eines jeden die entsprechende Deutung, \bibleverse{13} und
ganz so, wie er uns die Deutung gegeben hatte, so ist es eingetroffen:
mich hat (der Pharao) wieder in mein Amt eingesetzt, und jenen hat er
hängen lassen.« \bibleverse{14} Da sandte der Pharao hin und ließ Joseph
rufen; man holte\textless sup title=``oder: entließ''\textgreater✲ ihn
in aller Eile aus dem Gefängnis; er mußte sich scheren lassen und andere
Kleider anziehen und trat dann vor den Pharao.

\hypertarget{cc-der-pharao-erzuxe4hlt-seine-truxe4ume-joseph-gibt-ihre-deutung}{%
\subparagraph{cc) Der Pharao erzählt seine Träume, Joseph gibt ihre
Deutung}\label{cc-der-pharao-erzuxe4hlt-seine-truxe4ume-joseph-gibt-ihre-deutung}}

\bibleverse{15} Da sagte der Pharao zu Joseph: »Ich habe einen Traum
gehabt, aber niemand weiß ihn zu deuten. Nun habe ich von dir sagen
hören, du brauchtest einen Traum nur zu hören, so könntest du ihn schon
deuten.« \bibleverse{16} Da antwortete Joseph dem Pharao: »O nein, nicht
ich! Aber Gott wird etwas kundtun, was dem Pharao Segen bringt.«
\bibleverse{17} Nun sagte der Pharao zu Joseph: »In meinem Traume war es
mir, ich stände am Ufer des Nils; \bibleverse{18} da sah ich sieben
wohlgenährte, schöne Kühe aus dem Strom heraufsteigen und im Riedgras
weiden. \bibleverse{19} Nach diesen sah ich sieben andere Kühe
heraufsteigen, die dürr und sehr häßlich und mager am Fleisch waren; ich
habe in ganz Ägypten nirgends so häßliche gesehen wie diese.
\bibleverse{20} Hierauf fraßen die mageren und häßlichen Kühe die sieben
ersten wohlgenährten Kühe auf; \bibleverse{21} aber auch als sie in
ihren Leib gekommen waren, merkte man ihnen nicht an, daß sie in ihren
Leib gekommen waren; nein, ihr Aussehen blieb so häßlich wie im Anfang.
Da wachte ich auf. \bibleverse{22} Dann sah ich in meinem Traume sieben
Ähren, die oben an einem Halme wuchsen, volle und schöne.
\bibleverse{23} Nach diesen aber schossen sieben dürre, dünne, vom
Ostwind versengte Ähren hervor; \bibleverse{24} und die dünnen Ähren
verschlangen die sieben schönen Ähren. Ich habe dies schon den
Schriftkundigen✲ mitgeteilt, aber keiner hat mir eine Erklärung geben
können.«

\bibleverse{25} Da sagte Joseph zum Pharao: »(Beides,) was der Pharao
geträumt hat, bedeutet ein und dasselbe: Gott hat dem Pharao
angekündigt, was er zu tun gedenkt. \bibleverse{26} Die sieben schönen
Kühe bedeuten sieben Jahre, und die sieben schönen Ähren bedeuten auch
sieben Jahre: es ist ein und derselbe Traum. \bibleverse{27} Auch die
sieben mageren und häßlichen Kühe, die nach ihnen (aus dem Strom)
heraufstiegen, sind sieben Jahre, und die sieben leeren, vom Ostwind
versengten Ähren bedeuten, daß sieben Hungerjahre kommen werden.
\bibleverse{28} Das meinte ich, als ich (vorhin) zum Pharao sagte: ›Gott
hat dem Pharao geoffenbart, was er zu tun gedenkt.‹ \bibleverse{29}
Wisse: es werden sieben Jahre mit großem Überfluß im ganzen Land Ägypten
kommen; \bibleverse{30} aber nach diesen werden sieben Hungerjahre
eintreten, so daß der ganze Überfluß im Lande Ägypten vergessen sein
wird; und die Hungersnot wird das Land so verzehren, \bibleverse{31} daß
man von dem früheren Überfluß im Lande Ägypten nichts mehr merken wird
infolge der späteren Hungersnot; denn diese wird überaus schwer sein.
\bibleverse{32} Daß aber der Traum sich dem Pharao zweimal wiederholt
hat, das bedeutet: die Sache ist bei Gott fest beschlossen, und Gott
wird sie ohne Verzug ausführen.«

\hypertarget{dd-josephs-ratschlag-bezuxfcglich-der-verwertung-seiner-traumdeutung}{%
\subparagraph{dd) Josephs Ratschlag bezüglich der Verwertung seiner
Traumdeutung}\label{dd-josephs-ratschlag-bezuxfcglich-der-verwertung-seiner-traumdeutung}}

\bibleverse{33} »Und nun möge der Pharao sich nach einem einsichtigen
und weisen Manne umsehen, den er über das Land Ägypten setze!
\bibleverse{34} und der Pharao wolle Vorsorge tragen, daß er Aufseher
über das Land bestelle, und erhebe den fünften Teil des Ertrages vom
Lande Ägypten während der sieben Jahre des Überflusses! \bibleverse{35}
Man sammle so den gesamten Ernteertrag jener guten Jahre, die nun kommen
werden, und speichere das Getreide unter der Obhut des Pharaos als
Vorrat in den Städten auf und verwahre es dort. \bibleverse{36} Dann
wird dieser Vorrat dem Lande einen Rückhalt für die sieben Hungerjahre
gewähren, die im Lande Ägypten eintreten werden, und das Land wird durch
die Hungersnot nicht zugrunde gerichtet werden.«

\hypertarget{ee-josephs-erhebung-zum-huxf6chsten-beamten-im-staat}{%
\subparagraph{ee) Josephs Erhebung zum höchsten Beamten im
Staat}\label{ee-josephs-erhebung-zum-huxf6chsten-beamten-im-staat}}

\bibleverse{37} Diese Darlegung fand den Beifall des Pharaos und aller
seiner Diener; \bibleverse{38} und der Pharao sagte zu seinen Dienern:
»Könnten wir wohl noch einen Mann finden, in dem der Geist Gottes wohnt
wie in diesem?« \bibleverse{39} Zu Joseph aber sagte der Pharao:
»Nachdem\textless sup title=``oder: weil''\textgreater✲ Gott dir dieses
alles geoffenbart hat, gibt es keinen, der so einsichtig und weise wäre
wie du. \bibleverse{40} Du selber sollst über mein Haus gesetzt sein,
und deinen Befehlen soll mein ganzes Volk sich fügen; nur den Besitz des
Thrones will ich vor dir voraushaben.« \bibleverse{41} Weiter sagte der
Pharao zu Joseph: »Hiermit setze ich dich über das ganze Land Ägypten!«
\bibleverse{42} Darauf zog der Pharao seinen Siegelring vom Finger und
steckte ihn dem Joseph an die Hand, ließ ihn in Gewänder von Byssus
kleiden und legte ihm die goldene Kette um den Hals. \bibleverse{43}
Außerdem ließ er ihn auf seinem zweiten Staatswagen fahren, und man rief
vor ihm her aus: »Abrek!« So setzte er ihn über das ganze Land Ägypten.
\bibleverse{44} Sodann sagte der Pharao zu Joseph: »Ich bin der Pharao;
aber ohne deine Einwilligung soll niemand die Hand oder den Fuß im
ganzen Lande Ägypten rühren.« \bibleverse{45} Außerdem verlieh der
Pharao dem Joseph den Titel ›Zaphenath-Paneah‹\textless sup title=``d.h.
der das Leben Ernährende''\textgreater✲ und gab ihm Asenath, die Tochter
Potipheras, des Priesters von On\textless sup title=``=~Heliopolis in
Unterägypten''\textgreater✲ zur Frau. So gebot denn Joseph über das
(ganze) Land Ägypten.

\hypertarget{ff-josephs-mauxdfnahmen-wuxe4hrend-der-sieben-fruchtbaren-jahre-die-geburt-seiner-beiden-suxf6hne}{%
\subparagraph{ff) Josephs Maßnahmen während der sieben fruchtbaren
Jahre; die Geburt seiner beiden
Söhne}\label{ff-josephs-mauxdfnahmen-wuxe4hrend-der-sieben-fruchtbaren-jahre-die-geburt-seiner-beiden-suxf6hne}}

\bibleverse{46} Dreißig Jahre war Joseph alt, als er in den Dienst des
Pharaos, des Königs von Ägypten, trat. Nachdem Joseph sich nun vom
Pharao wegbegeben hatte, durchzog er das ganze Land Ägypten.
\bibleverse{47} Das Land trug aber während der sieben Jahre des
Überflusses (Getreide) in Hülle und Fülle. \bibleverse{48} Da sammelte
er den ganzen Ernteertrag der sieben guten Jahre, die über Ägypten
kamen, und ließ das Getreide in die Städte schaffen: nämlich in jede
Stadt brachte er den Ertrag der umliegenden Felder hinein.
\bibleverse{49} So speicherte Joseph Getreide auf wie Sand am Meer,
unendlich viel, bis man aufhörte, es zu messen, denn es war nicht mehr
zu messen\textless sup title=``=~unermeßlich viel''\textgreater✲.

\bibleverse{50} Noch ehe das Hungerjahr kam, wurden dem Joseph zwei
Söhne geboren von Asenath, der Tochter Potipheras, des Priesters von On.
\bibleverse{51} Joseph nannte seinen erstgeborenen Sohn
Manasse\textless sup title=``d.h. der vergessen macht''\textgreater✲;
»denn«, sagte er, »Gott hat mich all mein Unglück und mein ganzes
Vaterhaus vergessen lassen«. \bibleverse{52} Den zweiten aber nannte er
Ephraim\textless sup title=``d.h. doppelte Fruchtbarkeit''\textgreater✲;
»denn«, sagte er, »Gott hat mich fruchtbar werden lassen im Land meines
Elends«.

\hypertarget{gg-die-sieben-unfruchtbaren-jahre-und-josephs-getreideverkauf-wuxe4hrend-der-hungersnot}{%
\subparagraph{gg) Die sieben unfruchtbaren Jahre und Josephs
Getreideverkauf während der
Hungersnot}\label{gg-die-sieben-unfruchtbaren-jahre-und-josephs-getreideverkauf-wuxe4hrend-der-hungersnot}}

\bibleverse{53} Als dann die sieben Jahre des Überflusses, der im Lande
Ägypten geherrscht hatte, zu Ende waren, \bibleverse{54} da brachen die
sieben Hungerjahre an, ganz so wie Joseph es vorausgesagt hatte; und es
entstand eine Hungersnot in allen Ländern; aber in ganz Ägypten hatte
man Brot. \bibleverse{55} Als dann aber auch das ganze Land Ägypten
Hunger litt und das Volk zum Pharao um Brot schrie, sagte der Pharao zu
allen Ägyptern: »Wendet euch an Joseph; was der euch sagen wird, das
tut!« \bibleverse{56} Die Hungersnot erstreckte sich aber über die ganze
Erde. Da ließ Joseph allenthalben die Kornhäuser öffnen und den Ägyptern
Getreide verkaufen; und doch wurde die Hungersnot immer drückender in
Ägypten. \bibleverse{57} Da kam die ganze Erdbevölkerung zu Joseph nach
Ägypten, um Getreide zu kaufen; denn auf der ganzen Erde herrschte
drückende Hungersnot.

\hypertarget{p-erste-reise-der-bruxfcder-josephs-nach-uxe4gypten}{%
\paragraph{p) Erste Reise der Brüder Josephs nach
Ägypten}\label{p-erste-reise-der-bruxfcder-josephs-nach-uxe4gypten}}

\hypertarget{aa-die-zehn-uxe4lteren-suxf6hne-jakobs-ziehen-nach-uxe4gypten-zum-getreideeinkauf}{%
\subparagraph{aa) Die zehn älteren Söhne Jakobs ziehen nach Ägypten zum
Getreideeinkauf}\label{aa-die-zehn-uxe4lteren-suxf6hne-jakobs-ziehen-nach-uxe4gypten-zum-getreideeinkauf}}

\hypertarget{section-41}{%
\section{42}\label{section-41}}

\bibleverse{1} Als nun Jakob erfuhr, daß in Ägypten Getreide zu haben
sei, sagte er zu seinen Söhnen: »Was seht ihr euch lange an?«
\bibleverse{2} Dann fuhr er fort: »Wisset wohl: ich habe gehört, daß in
Ägypten Getreide zu haben ist; zieht hinab und kauft uns dort Getreide,
damit wir zu leben haben und nicht verhungern!« \bibleverse{3} So
machten sich denn zehn von den Brüdern Josephs auf den Weg, um Getreide
in Ägypten zu kaufen; \bibleverse{4} Benjamin aber, den Vollbruder
Josephs, ließ Jakob nicht mit seinen Brüdern ziehen; denn er fürchtete,
es könne ihm ein Unfall zustoßen. \bibleverse{5} So kamen denn die Söhne
Israels, um Getreide zu kaufen wie andere Leute, die auch hinzogen; denn
es herrschte Hungersnot im Lande Kanaan.

\hypertarget{bb-das-erste-schroffe-gespruxe4ch-josephs-mit-seinen-bruxfcdern-josephs-verdachtsuxe4uuxdferungen}{%
\subparagraph{bb) Das erste schroffe Gespräch Josephs mit seinen
Brüdern; Josephs
Verdachtsäußerungen}\label{bb-das-erste-schroffe-gespruxe4ch-josephs-mit-seinen-bruxfcdern-josephs-verdachtsuxe4uuxdferungen}}

\bibleverse{6} Nun war Joseph der Gebieter im Lande; er war es, der
allem Volk im Lande\textless sup title=``oder: der ganzen Bevölkerung
der Erde''\textgreater✲ das Getreide verkaufte. Als nun die Brüder
Josephs zu ihm kamen, verneigten sie sich vor ihm mit dem Angesicht bis
zur Erde. \bibleverse{7} Sobald Joseph seine Brüder sah, erkannte er
sie, gab sich ihnen aber nicht zu erkennen, sondern redete sie hart an
und fragte sie: »Woher seid ihr gekommen?« Sie antworteten: »Aus dem
Lande Kanaan, um Lebensmittel zu kaufen.« \bibleverse{8} Wiewohl Joseph
nun seine Brüder erkannt hatte, erkannten sie ihn doch nicht.
\bibleverse{9} Da mußte Joseph an die Träume denken, die er einst in
bezug auf sie geträumt hatte, und er sagte zu ihnen: »Kundschafter seid
ihr! Ihr seid nur hergekommen, um zu erspähen, wo das Land offen steht!«
\bibleverse{10} Sie antworteten ihm: »O nein, Herr! Deine Knechte sind
gekommen, um Lebensmittel zu kaufen. \bibleverse{11} Wir alle sind Söhne
eines Mannes, ehrliche Leute sind wir, deine Knechte sind keine
Kundschafter!« \bibleverse{12} Doch er antwortete ihnen: »Nein, sondern
ihr seid hergekommen, um zu erkunden, wo das Land offen steht!«
\bibleverse{13} Sie erwiderten: »Wir, deine Knechte, sind zwölf Brüder,
die Söhne eines Mannes im Lande Kanaan; der jüngste ist allerdings
augenblicklich bei unserm Vater, und der eine ist nicht mehr da.«
\bibleverse{14} Aber Joseph entgegnete ihnen: »Es ist doch so, wie ich
euch gesagt habe: ihr seid Kundschafter! \bibleverse{15} Daran sollt ihr
geprüft werden: Beim Leben des Pharaos: ihr sollt von hier nicht
weggehen, wenn euer jüngster Bruder nicht hierher kommt. \bibleverse{16}
Schickt einen von euch hin, daß er euren Bruder hole! Ihr anderen aber
bleibt so lange gefangen, bis eure Aussagen geprüft sind, ob ihr mit der
Wahrheit umgeht oder nicht! So wahr der Pharao lebt: ihr seid
Kundschafter!« \bibleverse{17} Hierauf ließ er sie beisammen drei Tage
lang in Gewahrsam nehmen.

\hypertarget{cc-das-zweite-gespruxe4ch-simeon-als-geisel-zuruxfcckbehalten}{%
\subparagraph{cc) Das zweite Gespräch: Simeon als Geisel
zurückbehalten}\label{cc-das-zweite-gespruxe4ch-simeon-als-geisel-zuruxfcckbehalten}}

\bibleverse{18} Am dritten Tage aber sagte Joseph zu ihnen: »Wollt ihr
am Leben bleiben, so müßt ihr es folgendermaßen machen; denn ich bin ein
gottesfürchtiger Mann. \bibleverse{19} Wenn ihr ehrliche Leute seid, so
soll nur einer von euch Brüdern als Gefangener hier in eurem bisherigen
Gewahrsam zurückbleiben; ihr anderen aber mögt hinziehen und Getreide
für den Bedarf eurer Familien mitnehmen! \bibleverse{20} Aber euren
jüngsten Bruder müßt ihr zu mir herbringen; dann sollen eure Aussagen
als wahr gelten, und ihr braucht nicht zu sterben!« Sie gingen darauf
ein, \bibleverse{21} sagten aber einer zum andern: »Wahrlich, das haben
wir an unserm Bruder verschuldet! Denn wir sahen seine Seelenangst, als
er uns anflehte, aber wir hörten nicht auf ihn; deshalb ist jetzt dieses
Unglück über uns gekommen!« \bibleverse{22} Da antwortete ihnen Ruben:
»Habe ich euch damals nicht gesagt: ›Versündigt euch nicht an dem
Knaben!‹, aber ihr wolltet nicht hören; so wird denn jetzt auch sein
Blut von uns gefordert!« \bibleverse{23} Sie wußten aber nicht, daß
Joseph sie verstand; denn er verhandelte mit ihnen durch einen
Dolmetscher. \bibleverse{24} Da wandte er sich von ihnen ab und weinte.
Als er dann wieder zu ihnen zurückgekehrt war und mit ihnen gesprochen
hatte, ließ er Simeon aus ihrer Mitte ergreifen und vor ihren Augen in
Fesseln legen.

\hypertarget{dd-der-bruxfcder-entlassung-und-ruxfcckkehr-nach-kanaan-ihr-bericht-an-den-vater-jakobs-klage}{%
\subparagraph{dd) Der Brüder Entlassung und Rückkehr nach Kanaan; ihr
Bericht an den Vater; Jakobs
Klage}\label{dd-der-bruxfcder-entlassung-und-ruxfcckkehr-nach-kanaan-ihr-bericht-an-den-vater-jakobs-klage}}

\bibleverse{25} Dann gab Joseph Befehl, man solle ihre Säcke mit
Getreide füllen, ihr Geld aber einem jeden wieder in seinen Sack legen
und ihnen auch Zehrung für den Weg mitgeben. Als das geschehen war,
\bibleverse{26} luden sie ihr Getreide auf ihre Esel und zogen von
dannen. \bibleverse{27} Als aber einer von ihnen in der Herberge seinen
Sack öffnete, um seinem Esel Futter zu geben, bemerkte er sein Geld, das
in seinem Sack obenauf lag. \bibleverse{28} Da sagte er zu seinen
Brüdern: »Mein Geld ist wieder da! Denkt euch nur: es liegt hier in
meinem Sack!« Da entfiel ihnen der Mut; sie sahen einander erschrocken
an und riefen aus: »Was hat Gott uns da angetan!«

\bibleverse{29} Als sie hierauf zu ihrem Vater Jakob ins Land Kanaan
zurückkamen, erzählten sie ihm alles, was sie erlebt hatten, mit den
Worten: \bibleverse{30} »Der Mann, der im Lande Herr ist, hat uns hart
angelassen und uns wie Leute behandelt, die das Land auskundschaften
wollten; \bibleverse{31} und als wir zu ihm sagten: ›Wir sind ehrliche
Leute, wir sind nie Kundschafter gewesen; \bibleverse{32} zwölf Brüder
sind wir, die Söhne unsers Vaters; der eine ist nicht mehr da, und der
jüngste befindet sich zur Zeit bei unserm Vater im Lande Kanaan‹,~--
\bibleverse{33} da erwiderte uns der Mann, der im Lande Herr ist: ›Daran
will ich erkennen, ob ihr ehrliche Leute seid: laßt einen von euch
Brüdern bei mir zurück und nehmt den Bedarf für eure hungernden Familien
mit und kehrt heim; \bibleverse{34} bringt dann aber euren jüngsten
Bruder zu mir! Daran werde ich erkennen, daß ihr keine Kundschafter,
sondern ehrliche Leute seid; dann will ich euch auch euren Bruder
zurückgeben, und ihr könnt im Lande frei verkehren.‹« \bibleverse{35}
Als sie dann ihre Säcke leerten, fand jeder seinen Geldbeutel in seinem
Sack; und als sie samt ihrem Vater sahen, daß es ihre Beutel waren,
erschraken sie. \bibleverse{36} Da sagte ihr Vater Jakob zu ihnen: »Ihr
beraubt mich meiner Kinder! Joseph ist nicht mehr da, Simeon ist nicht
mehr da, und nun sollt ihr Benjamin holen\textless sup title=``oder:
wollt ihr auch wegnehmen''\textgreater✲! Über mich ist all dieses Leid
hereingebrochen!« \bibleverse{37} Da antwortete Ruben seinem Vater:
»Meine beiden Söhne magst du töten, wenn ich ihn nicht zu dir
zurückbringe! Vertraue ihn mir an: ich bringe ihn zu dir zurück!«
\bibleverse{38} Doch Jakob entgegnete: »Mein Sohn soll nicht mit euch
hinabziehen! Sein rechter Bruder ist tot, und er ist allein
übriggeblieben; wenn ihm ein Unfall auf dem Wege zustieße, den ihr
ziehen müßt, so würdet ihr mein graues Haar mit Herzeleid\textless sup
title=``oder: durch den Kummer''\textgreater✲ in die Unterwelt bringen!«

\hypertarget{q-zweite-reise-der-bruxfcder-josephs-nach-uxe4gypten-mit-benjamin}{%
\paragraph{q) Zweite Reise der Brüder Josephs nach Ägypten mit
Benjamin}\label{q-zweite-reise-der-bruxfcder-josephs-nach-uxe4gypten-mit-benjamin}}

\hypertarget{aa-die-reise-wird-von-jakob-erst-dann-gebilligt-nachdem-juda-sich-fuxfcr-benjamin-verbuxfcrgt-hat}{%
\subparagraph{aa) Die Reise wird von Jakob erst dann gebilligt, nachdem
Juda sich für Benjamin verbürgt
hat}\label{aa-die-reise-wird-von-jakob-erst-dann-gebilligt-nachdem-juda-sich-fuxfcr-benjamin-verbuxfcrgt-hat}}

\hypertarget{section-42}{%
\section{43}\label{section-42}}

\bibleverse{1} Die Hungersnot lag aber schwer auf dem Lande.
\bibleverse{2} Als nun das Getreide, das sie aus Ägypten geholt hatten,
vollständig aufgezehrt war, sagte ihr Vater zu ihnen: »Zieht noch einmal
hin und kauft uns etwas Getreide zur Nahrung!« \bibleverse{3} Da
antwortete ihm Juda: »Der Mann hat uns eindringlich gewarnt mit den
Worten: ›Ihr dürft mir nicht mehr vor die Augen treten, wenn euer Bruder
nicht bei euch ist!‹ \bibleverse{4} Willst du uns also unsern Bruder
mitgeben, dann wollen wir hinabziehen und Lebensmittel für dich kaufen;
\bibleverse{5} willst du ihn aber nicht mitgehen lassen, so ziehen wir
nicht hinab; denn der Mann hat uns gesagt: ›Ihr dürft mir nicht vor die
Augen treten, wenn euer Bruder nicht bei euch ist!‹« \bibleverse{6} Da
sagte Israel: »Warum habt ihr mir das zuleide getan, dem Manne
mitzuteilen, daß ihr noch einen Bruder habt?« \bibleverse{7} Sie
antworteten: »Der Mann erkundigte sich genau nach uns und unserer
Familie und fragte: ›Lebt euer Vater noch? Habt ihr noch einen Bruder?‹
Da haben wir ihm auf seine Fragen Auskunft gegeben. Konnten wir denn
wissen, daß er verlangen würde: ›Bringt euren Bruder her‹?«
\bibleverse{8} Juda aber sagte zu seinem Vater Israel: »Laß den Knaben
mit mir gehen; dann wollen wir uns auf den Weg machen und hinziehen,
damit wir am Leben bleiben und nicht verhungern, weder wir selbst noch
du und unsere Familien! \bibleverse{9} Ich will Bürge für ihn sein; von
meiner Hand sollst du ihn zurückfordern! Wenn ich ihn nicht wieder zu
dir bringe und ihn dir nicht vor die Augen stelle, so will ich Zeit
meines Lebens schuldbeladen vor dir dastehen! \bibleverse{10} Ja, hätten
wir nicht so lange gezögert, gewiß, wir wären jetzt schon zweimal wieder
zurückgekehrt.« \bibleverse{11} Da erwiderte ihnen ihr Vater Israel:
»Wenn es denn sein muß, so macht es folgendermaßen: Nehmt von den besten
Erzeugnissen des Landes etwas in euren Säcken mit und überbringt es dem
Manne als Geschenk: etwas Mastixbalsam und etwas Honig, Gewürzkräuter
und Ladanum, Pistaziennüsse und Mandeln! \bibleverse{12} Außerdem nehmt
an Geld den doppelten Betrag mit! Denn auch das Geld, das sich oben in
euren Säcken wiedergefunden hat, müßt ihr wieder mit euch nehmen:
vielleicht liegt ein Versehen vor. \bibleverse{13} Auch euren Bruder
nehmt mit und macht euch auf den Weg und zieht wieder hin zu dem Manne!
\bibleverse{14} Der allmächtige Gott aber lasse euch Erbarmen bei dem
Manne finden, daß er euren andern Bruder wieder mit euch ziehen läßt und
auch Benjamin! Ich aber -- wie ich einstmals kinderlos gewesen bin, so
habe ich auch jetzt wieder keine Kinder!« \bibleverse{15} So nahmen denn
die Männer das betreffende Geschenk und den doppelten Betrag an Geld mit
sich, dazu auch den Benjamin, machten sich auf den Weg, zogen nach
Ägypten hinab und traten vor Joseph.

\hypertarget{bb-freundlicher-empfang-und-die-erste-unterredung-der-bruxfcder-mit-josephs-hausverwalter}{%
\subparagraph{bb) Freundlicher Empfang und die erste Unterredung der
Brüder mit Josephs
Hausverwalter}\label{bb-freundlicher-empfang-und-die-erste-unterredung-der-bruxfcder-mit-josephs-hausverwalter}}

\bibleverse{16} Als nun Joseph den Benjamin bei ihnen sah, befahl er
seinem Hausverwalter: »Führe die Männer ins Haus hinein, laß ein
Schlachttier schlachten und richte zu, denn die Männer sollen zu Mittag
bei mir speisen.« \bibleverse{17} Der Mann tat, wie Joseph ihm befohlen
hatte, und führte die Männer in Josephs Haus. \bibleverse{18} Da
fürchteten sie sich, daß sie in das Haus Josephs geführt wurden, und
dachten: »Wegen des Geldes, das vorigesmal wieder in unsere Säcke
geraten ist, werden wir hineingeführt: man will sich auf uns stürzen,
uns überwältigen und uns zu Sklaven machen samt unsern Eseln.«
\bibleverse{19} Darum traten sie an den Hausverwalter Josephs heran und
redeten ihn noch am Eingang des Hauses so an: \bibleverse{20} »Bitte,
mein Herr! Wir sind schon einmal hergekommen, um Lebensmittel zu kaufen.
\bibleverse{21} Als wir dann aber in die Herberge gekommen waren und
unsere Säcke öffneten, da fand sich das Geld eines jeden oben in seinem
Sack, unser Geld in vollem Betrage. Darum haben wir es jetzt wieder
mitgebracht, \bibleverse{22} haben aber auch noch anderes Geld bei uns,
um Lebensmittel zu kaufen. Wir wissen nicht, wer uns damals unser Geld
in unsere Säcke gelegt hat.« \bibleverse{23} Da antwortete jener: »Seid
unbesorgt, ihr habt nichts zu fürchten! Euer Gott und eures Vaters Gott
hat euch da heimlich einen Schatz in eure Säcke gelegt; euer Geld ist
richtig an mich gekommen.« Dann führte er Simeon zu ihnen heraus.
\bibleverse{24} Hierauf ließ er die Männer in das Haus Josephs
eintreten, gab ihnen Wasser zum Füßewaschen und ließ ihren Eseln Futter
geben. \bibleverse{25} Sie legten aber das Geschenk zurecht und warteten
dann, bis Joseph zur Mittagszeit kommen würde; sie hatten nämlich
gehört, daß sie dort speisen sollten.

\hypertarget{cc-joseph-empfuxe4ngt-und-bewirtet-seine-bruxfcder-aufs-freundlichste}{%
\subparagraph{cc) Joseph empfängt und bewirtet seine Brüder aufs
freundlichste}\label{cc-joseph-empfuxe4ngt-und-bewirtet-seine-bruxfcder-aufs-freundlichste}}

\bibleverse{26} Als nun Joseph nach Hause gekommen war, brachten sie ihm
das Geschenk, das sie bei sich hatten, ins Zimmer hinein und verneigten
sich vor ihm bis zur Erde. \bibleverse{27} Er begrüßte sie freundlich
und fragte sie: »Geht es eurem alten Vater wohl, von dem ihr mir erzählt
habt? Ist er noch am Leben?« \bibleverse{28} Sie antworteten: »Deinem
Knecht, unserm Vater, geht es wohl; ja, er ist noch am Leben.« Dabei
verbeugten sie sich wiederholt tief. \bibleverse{29} Als er dann
hinblickte und Benjamin, seinen Vollbruder, den Sohn seiner eigenen
Mutter, sah, fragte er: »Ist dies euer jüngster Bruder, von dem ihr mir
erzählt habt?« Dann fügte er hinzu: »Gott sei dir gnädig, mein Sohn!«
\bibleverse{30} Hierauf aber brach Joseph schnell ab, denn sein Gefühl
überwältigte ihn beim Anblick seines Bruders, so daß er weinen mußte; er
ging deshalb ins Innengemach und weinte sich dort aus. \bibleverse{31}
Dann wusch er sich das Gesicht und kam wieder heraus, nahm sich zusammen
und befahl: »Tragt das Essen auf!« \bibleverse{32} Da trug man für ihn
besonders auf und für sie besonders und auch für die Ägypter, die bei
ihm speisten, besonders; denn die Ägypter dürfen nicht mit den Hebräern
zusammen speisen, weil das für die Ägypter eine Verunreinigung sein
würde. \bibleverse{33} Sie hatten aber ihre Plätze nach seiner
Anweisung, vom Ältesten bis zum Jüngsten, genau nach ihrem Alter;
darüber sahen sie sich einander verwundert an. \bibleverse{34} Hierauf
ließ er von den vor ihm stehenden Gerichten Anteile zu ihnen hintragen;
es war aber dessen, was man Benjamin vorlegte, fünfmal so viel, als was
man allen anderen vorlegte. Und sie tranken mit ihm und wurden guter
Dinge.

\hypertarget{r-joseph-pruxfcft-seine-bruxfcder-zum-letztenmal}{%
\paragraph{r) Joseph prüft seine Brüder zum
letztenmal}\label{r-joseph-pruxfcft-seine-bruxfcder-zum-letztenmal}}

\hypertarget{aa-die-bruxfcder-machen-sich-auf-den-heimweg-josephs-becher-wird-in-benjamins-getreidesack-gefunden}{%
\subparagraph{aa) Die Brüder machen sich auf den Heimweg; Josephs Becher
wird in Benjamins Getreidesack
gefunden}\label{aa-die-bruxfcder-machen-sich-auf-den-heimweg-josephs-becher-wird-in-benjamins-getreidesack-gefunden}}

\hypertarget{section-43}{%
\section{44}\label{section-43}}

\bibleverse{1} Hierauf befahl er (Joseph) seinem Hausverwalter: »Fülle
den Männern ihre Säcke mit Getreide, soviel sie fortschaffen können, und
lege einem jeden sein Geld (wieder) oben in seinen Sack! \bibleverse{2}
Meinen Becher aber, den silbernen Becher, sollst du dem Jüngsten oben in
seinen Sack legen samt dem Gelde für sein Getreide!« Jener tat nach dem
von Joseph ihm erteilten Befehl. \bibleverse{3} Am andern Morgen, als es
hell wurde, ließ man die Männer mit ihren Eseln ziehen. \bibleverse{4}
Als sie aber kaum zur Stadt hinausgezogen und noch nicht weit gekommen
waren, befahl Joseph seinem Hausverwalter: »Auf! Eile den Männern nach!
Und hast du sie eingeholt, so sage zu ihnen: ›Warum habt ihr Gutes mit
Bösem vergolten? \bibleverse{5} Warum habt ihr den silbernen Becher
gestohlen? Es ist gerade der Becher, aus dem mein Herr trinkt und
mittels dessen er auch wahrzusagen pflegt. Ihr habt da schlecht
gehandelt!‹« \bibleverse{6} Als er sie nun eingeholt hatte, sagte er
diese Worte zu ihnen. \bibleverse{7} Da antworteten sie ihm: »O Herr,
wie kannst du so etwas sagen? Es liegt deinen Knechten fern, so etwas zu
tun! \bibleverse{8} Wir haben dir ja doch auch das Geld, das wir oben in
unsern Säcken gefunden hatten, aus dem Lande Kanaan zurückgebracht: wie
sollten wir da jetzt aus dem Hause deines Herrn Silber oder Gold
gestohlen haben? \bibleverse{9} Bei wem von deinen Knechten (der Becher)
gefunden wird, der soll sterben, und auch wir anderen wollen Sklaven
meines Herrn werden!« \bibleverse{10} Er antwortete: »Gut! Es sei, wie
ihr gesagt habt: bei wem er gefunden wird, der soll mein Sklave sein;
ihr anderen aber sollt frei ausgehen!« \bibleverse{11} Da setzten sie
jeder seinen Sack schnell auf die Erde nieder, und jeder öffnete seinen
Sack. \bibleverse{12} Er aber fing an zu suchen; beim Ältesten fing er
an, und beim Jüngsten hörte er auf: da fand sich der Becher im Sack
Benjamins. \bibleverse{13} Da zerrissen sie ihre Kleider; ein jeder
belud seinen Esel wieder, und sie kehrten in die Stadt zurück.

\hypertarget{bb-die-bruxfcder-kehren-in-die-stadt-zuruxfcck-und-demuxfctigen-sich-vor-joseph-judas-rede-an-joseph}{%
\subparagraph{bb) Die Brüder kehren in die Stadt zurück und demütigen
sich vor Joseph; Judas Rede an
Joseph}\label{bb-die-bruxfcder-kehren-in-die-stadt-zuruxfcck-und-demuxfctigen-sich-vor-joseph-judas-rede-an-joseph}}

\bibleverse{14} Als nun Juda und seine Brüder in das Haus Josephs
gekommen waren -- dieser war aber dort noch anwesend --, warfen sie sich
vor ihm auf die Erde nieder. \bibleverse{15} Da sagte Joseph zu ihnen:
»Was für eine Tat habt ihr da begangen! Wußtet ihr nicht, daß ein Mann
wie ich sich auf Zeichendeutung versteht?« \bibleverse{16} Da antwortete
Juda: »Was sollen wir zu meinem\textless sup title=``oder:
unserm''\textgreater✲ Herrn sagen? Was sollen wir reden und wie uns
rechtfertigen? Gott hat die Schuld deiner Knechte ans Licht gebracht:
wir (alle) gehören jetzt meinem Herrn als Sklaven, wir ebensogut wie
der, in dessen Besitz der Becher gefunden worden ist.« \bibleverse{17}
Joseph aber antwortete: »Fern sei es von mir, so zu verfahren! Nur der,
in dessen Besitz der Becher sich gefunden hat, soll mein Sklave werden;
ihr anderen aber mögt ungehindert zu eurem Vater zurückkehren!«

\bibleverse{18} Da trat Juda an ihn heran und sagte: »Bitte, mein Herr!
Laß doch deinen Knecht ein Wort an dich richten, mein Herr, ohne daß
dein Zorn gegen deinen Knecht entbrennt, obgleich du so hoch stehst wie
der Pharao! \bibleverse{19} Mein Herr hat vordem seine Knechte gefragt:
›Habt ihr noch einen Vater oder einen Bruder?‹ \bibleverse{20} Da
antworteten wir meinem Herrn: ›Wir haben noch einen alten Vater und
einen noch jungen Bruder, der ihm im Alter geboren ist; dessen rechter
Bruder ist tot, und so ist er ihm allein von seiner Mutter
übriggeblieben und ist deshalb der Liebling seines Vaters.‹
\bibleverse{21} Da befahlst du deinen Knechten: ›Bringt ihn zu mir her,
ich will ihn mit eigenen Augen zu sehen bekommen!‹ \bibleverse{22} Da
antworteten wir meinem Herrn: ›Der Knabe kann seinen Vater nicht
verlassen; denn wenn er seinen Vater verließe, so würde dieser sterben.‹
\bibleverse{23} Aber du antwortetest deinen Knechten: ›Kommt euer
jüngster Bruder nicht mit euch her, so dürft ihr mir nicht wieder vor
die Augen treten!‹ \bibleverse{24} Als wir dann zu deinem Knecht, meinem
Vater, zurückgekehrt waren, teilten wir ihm die Worte meines Herrn mit;
\bibleverse{25} und als später unser Vater sagte: ›Zieht wieder hin und
kauft uns etwas Getreide!‹, \bibleverse{26} da antworteten wir: ›Wir
können nicht hinabziehen; nur wenn unser jüngster Bruder uns begleitet,
wollen wir hinabziehen; denn wir dürfen uns vor dem Manne nicht sehen
lassen, wenn unser jüngster Bruder nicht bei uns ist.‹ \bibleverse{27}
Da erwiderte uns dein Knecht, mein Vater: ›Ihr wißt selbst, daß meine
Frau mir nur zwei Söhne geboren hat; \bibleverse{28} der eine ist von
mir weggegangen, und ich mußte mir sagen: Sicherlich hat ihn ein Tier
zerrissen!, und ich habe ihn bis heute nicht wiedergesehen.
\bibleverse{29} Wenn ihr mir nun auch diesen noch wegnehmt und ihm ein
Unglück zustößt, so werdet ihr mein graues Haar mit Jammer in die
Unterwelt hinabbringen!‹ \bibleverse{30} Käme ich jetzt also zu deinem
Knecht, meinem Vater, heim und der Knabe wäre nicht bei uns, an dem doch
sein ganzes Herz hängt, \bibleverse{31} und müßte er sehen, daß der
Knabe nicht da ist, so würde er sterben, und deine Knechte hätten
wirklich das graue Haar deines Knechtes, unsers Vaters, mit Herzeleid✲
in die Unterwelt hinabgebracht! \bibleverse{32} Weil nun dein Knecht
sich bei meinem Vater für den Knaben verbürgt und gelobt hat: ›Wenn ich
ihn dir nicht zurückbringe, so will ich zeit meines Lebens vor meinem
Vater schuldbeladen dastehen!‹, \bibleverse{33} so laß doch jetzt deinen
Knecht anstatt des Knaben als Sklaven meines Herrn hierbleiben, den
Knaben aber laß mit seinen Brüdern heimziehen! \bibleverse{34} Denn wie
könnte ich zu meinem Vater zurückkehren, ohne daß der Knabe bei mir
wäre? Ich könnte den Jammer nicht mitansehen, der meinen Vater träfe!«

\hypertarget{s-joseph-gibt-sich-seinen-bruxfcdern-zu-erkennen-ruxfcckkehr-der-bruxfcder}{%
\paragraph{s) Joseph gibt sich seinen Brüdern zu erkennen; Rückkehr der
Brüder}\label{s-joseph-gibt-sich-seinen-bruxfcdern-zu-erkennen-ruxfcckkehr-der-bruxfcder}}

\hypertarget{section-44}{%
\section{45}\label{section-44}}

\bibleverse{1} Da vermochte Joseph nicht länger an sich zu halten vor
allen, die um ihn her standen, sondern er rief aus: »Laßt jedermann von
mir weg hinausgehen!« So war denn niemand zugegen, als Joseph sich
seinen Brüdern zu erkennen gab. \bibleverse{2} Er brach aber in ein so
lautes Weinen aus, daß die Ägypter es hörten und auch das Haus des
Pharaos Kunde davon erhielt. \bibleverse{3} Joseph sagte aber zu seinen
Brüdern: »Ich bin Joseph! Lebt mein Vater noch?« Seine Brüder vermochten
aber nicht, ihm zu antworten: so bestürzt standen sie vor ihm.
\bibleverse{4} Da sagte er zu seinen Brüdern: »Tretet doch nahe an mich
heran!« Als sie nun näher getreten waren, sagte er: »Ich bin euer Bruder
Joseph, den ihr nach Ägypten verkauft habt! \bibleverse{5} Nun
beunruhigt euch aber nicht und macht euch keine Vorwürfe darüber, daß
ihr mich hierher verkauft habt! Denn um uns alle am Leben zu erhalten,
hat Gott mich euch vorausgesandt.«

\hypertarget{aa-joseph-ist-von-gott-ausgesandt-auch-zu-israels-heil}{%
\subparagraph{aa) Joseph ist von Gott ausgesandt auch zu Israels
Heil}\label{aa-joseph-ist-von-gott-ausgesandt-auch-zu-israels-heil}}

\bibleverse{6} »Denn jetzt herrscht die Hungersnot erst zwei Jahre im
Lande, und fünf Jahre stehen noch bevor, in denen kein Pflügen und kein
Ernten stattfinden wird. \bibleverse{7} Darum hat Gott mich euch
vorausgesandt, um das Fortbestehen eures Geschlechts auf Erden zu
sichern und um euch, eine große Schar von Erretteten, am Leben zu
erhalten. \bibleverse{8} So habt also nicht ihr mich hierher gebracht,
sondern Gott; der hat mich dem Pharao zum Vater\textless sup
title=``=~vertrauten Berater''\textgreater✲ gemacht und zum Herrn über
sein ganzes Haus und zum Gebieter im ganzen Lande Ägypten.
\bibleverse{9} Zieht nun eilends zu meinem Vater hinauf und meldet ihm:
›So läßt dir dein Sohn Joseph sagen: Gott hat mich zum Gebieter von ganz
Ägypten gemacht: komm zu mir herab, säume nicht! \bibleverse{10} Du
sollst im Lande Gosen wohnen und in meiner Nähe sein, du, deine Kinder
und Kindeskinder samt deinem Kleinvieh und deinen Rindern und deinem
ganzen Hab und Gut. \bibleverse{11} Ich will dich daselbst versorgen,
denn noch fünf Jahre wird die Hungersnot dauern, damit du nicht
verarmst\textless sup title=``oder: im Elend verkommst''\textgreater✲,
du und dein Haus und alles, was du besitzest.‹ \bibleverse{12} Ihr seht
es ja mit eigenen Augen, und auch mein Bruder Benjamin sieht es mit
eigenen Augen, daß ich persönlich es bin, der zu euch redet.
\bibleverse{13} Berichtet also meinem Vater alle die hohen Ehren, die
ich in Ägypten habe, und alles, was ihr gesehen habt, und bringt meinen
Vater eilends hierher!« \bibleverse{14} Darauf fiel er seinem Bruder
Benjamin um den Hals und weinte, und auch Benjamin weinte an seinem
Halse; \bibleverse{15} dann küßte er alle seine Brüder und umarmte sie
unter Tränen; nun erst vermochten auch seine Brüder mit ihm zu reden.

\hypertarget{bb-des-pharaos-gnuxe4dige-einladung-an-jakob-nach-uxe4gypten-uxfcberzusiedeln}{%
\subparagraph{bb) Des Pharaos gnädige Einladung an Jakob, nach Ägypten
überzusiedeln}\label{bb-des-pharaos-gnuxe4dige-einladung-an-jakob-nach-uxe4gypten-uxfcberzusiedeln}}

\bibleverse{16} Als nun die Kunde von der Ankunft der Brüder Josephs in
den Palast des Pharaos drang, war sie dem Pharao und seinen Dienern
angenehm. \bibleverse{17} Daher sagte der Pharao zu Joseph: »Sage deinen
Brüdern: ›Tut also: beladet eure Lasttiere und zieht heim ins Land
Kanaan, \bibleverse{18} holt euren Vater und eure Familien und kommt zu
mir! Ich will euch den besten Teil des Landes Ägypten geben, damit ihr
das Fett\textless sup title=``=~die vorzüglichsten
Erzeugnisse''\textgreater✲ des Landes genießen könnt.‹ \bibleverse{19}
Du bist ermächtigt (ihnen zu sagen): ›Tut also: nehmt euch aus Ägypten
Wagen mit für eure Kinder und Frauen, laßt auch euren Vater aufsteigen
und kommt hierher! \bibleverse{20} Laßt es euch um euren Hausrat nicht
leid sein! Denn das Beste vom ganzen Land Ägypten soll euch zuteil
werden.‹«

\hypertarget{cc-joseph-beschenkt-seine-heimziehenden-bruxfcder-reichlich-und-ermahnt-sie-liebevoll}{%
\subparagraph{cc) Joseph beschenkt seine heimziehenden Brüder reichlich
und ermahnt sie
liebevoll}\label{cc-joseph-beschenkt-seine-heimziehenden-bruxfcder-reichlich-und-ermahnt-sie-liebevoll}}

\bibleverse{21} So taten denn die Söhne Israels also, und Joseph gab
ihnen Wagen nach dem Befehl des Pharaos und außerdem Zehrung für die
Reise. \bibleverse{22} Ihnen allen schenkte er, einem jeden, einen
Festtagsanzug; dem Benjamin aber schenkte er dreihundert Silberstücke
und fünf Festtagsanzüge. \bibleverse{23} Seinem Vater aber sandte er
dementsprechend: zehn Esel, die mit den besten Erzeugnissen Ägyptens
beladen waren, und zehn Eselinnen, die Getreide, Brot und Reisekost für
seinen Vater trugen. \bibleverse{24} Alsdann entließ er seine Brüder,
und sie zogen ab, nachdem er sie noch ermahnt hatte:
»Erzürnt\textless sup title=``oder: zankt''\textgreater✲ euch nicht
unterwegs!«

\hypertarget{dd-jakobs-staunen-und-freude-und-sein-entschluuxdf-zu-seinem-sohn-nach-uxe4gypten-zu-ziehen}{%
\subparagraph{dd) Jakobs Staunen und Freude und sein Entschluß, zu
seinem Sohn nach Ägypten zu
ziehen}\label{dd-jakobs-staunen-und-freude-und-sein-entschluuxdf-zu-seinem-sohn-nach-uxe4gypten-zu-ziehen}}

\bibleverse{25} So zogen sie denn aus Ägypten ab und kamen ins Land
Kanaan zu ihrem Vater Jakob, \bibleverse{26} dem sie berichteten:
»Joseph lebt noch und ist Gebieter über das ganze Land Ägypten!« Aber
sein Herz blieb kalt dabei, denn er glaubte ihnen nicht. \bibleverse{27}
Als sie ihm aber alles erzählten, was Joseph ihnen aufgetragen hatte,
und als er die Wagen sah, die Joseph geschickt hatte, um ihn zu holen,
da kam wieder Leben in den Geist ihres Vaters Jakob, \bibleverse{28} so
daß er ausrief: »Genug! Mein Sohn Joseph lebt noch! Ich will hinziehen
und ihn noch einmal sehen, ehe ich sterbe!«

\hypertarget{t-die-uxfcbersiedlung-jakobs-und-seiner-familie-nach-uxe4gypten-verzeichnis-der-nachkommen-jakobs-sein-empfang-durch-joseph}{%
\paragraph{t) Die Übersiedlung Jakobs und seiner Familie nach Ägypten;
Verzeichnis der Nachkommen Jakobs; sein Empfang durch
Joseph}\label{t-die-uxfcbersiedlung-jakobs-und-seiner-familie-nach-uxe4gypten-verzeichnis-der-nachkommen-jakobs-sein-empfang-durch-joseph}}

\hypertarget{aa-gott-billigt-jakobs-uxfcbersiedelung-in-einer-offenbarung-zu-beerseba}{%
\subparagraph{aa) Gott billigt Jakobs Übersiedelung in einer Offenbarung
zu
Beerseba}\label{aa-gott-billigt-jakobs-uxfcbersiedelung-in-einer-offenbarung-zu-beerseba}}

\hypertarget{section-45}{%
\section{46}\label{section-45}}

\bibleverse{1} So brach denn Israel mit allen seinen Angehörigen auf,
und als er nach Beerseba gekommen war, brachte er dort dem Gott seines
Vaters Isaak Schlachtopfer dar. \bibleverse{2} Da redete Gott mit Israel
nachts in einem Gesicht und sagte: »Jakob, Jakob!« Er antwortete: »Hier
bin ich!« \bibleverse{3} Darauf sagte Gott: »Ich bin Gott, der Gott
deines Vaters! Fürchte dich nicht, nach Ägypten hinabzuziehen; denn ich
will dich dort zu einem großen Volk machen. \bibleverse{4} Ich selbst
will mit dir nach Ägypten hinabziehen, und ich selbst will dich (einst)
auch wieder zurückführen, und Josephs Hand soll dir die Augen
zudrücken.« \bibleverse{5} Da brach Jakob von Beerseba auf, und seine
Söhne ließen ihren Vater Jakob nebst ihren Kindern und ihren Frauen auf
den Wagen fahren, die der Pharao geschickt hatte, um ihn zu holen.
\bibleverse{6} Sie nahmen auch ihr Vieh und ihre Habe mit, die sie im
Lande Kanaan erworben hatten, und kamen so nach Ägypten, Jakob mit
seiner gesamten Nachkommenschaft: \bibleverse{7} seine Söhne und Enkel,
seine Töchter und Enkelinnen, überhaupt seine ganze Nachkommenschaft
brachte er mit sich nach Ägypten.

\hypertarget{bb-der-damalige-bestand-von-jakobs-gesamtfamilie}{%
\subparagraph{bb) Der damalige Bestand von Jakobs
Gesamtfamilie}\label{bb-der-damalige-bestand-von-jakobs-gesamtfamilie}}

\bibleverse{8} Dies aber sind die Namen der Nachkommen Israels, die nach
Ägypten kamen: Jakob und seine Söhne. Der erstgeborene Sohn Jakobs war
Ruben; \bibleverse{9} und die Söhne Rubens waren Hanoch, Pallu, Hezron
und Karmi. \bibleverse{10} Die Söhne Simeons waren Jemuel, Jamin, Ohad,
Jachin, Zohar und Saul, der Sohn der Kanaanäerin. \bibleverse{11} Die
Söhne Levis waren Gerson, Kehath und Merari. \bibleverse{12} Die Söhne
Judas waren Ger, Onan, Sela, Perez und Serah; aber Ger und Onan waren
schon im Lande Kanaan gestorben, und die Söhne des Perez waren Hezron
und Hamul. \bibleverse{13} Die Söhne Issaschars waren Thola, Puwwa, Job
und Simron. \bibleverse{14} Die Söhne Sebulons waren Sered, Elon und
Jahleel. \bibleverse{15} Dies sind die Söhne der Lea, die sie dem Jakob
in Nord-Mesopotamien geboren hatte, dazu seine Tochter Dina, insgesamt
dreiunddreißig Söhne und Töchter.~-- \bibleverse{16} Die Söhne Gads aber
waren Ziphjon und Haggi, Suni und Ezbon, Heri, Arodi und Areli.
\bibleverse{17} Die Söhne Assers waren Jimna, Jiswa, Jiswi, Beria und
ihre Schwester Serah; und die Söhne Berias waren Heber und Malkiel.
\bibleverse{18} Dies sind die Söhne der Silpa, die Laban seiner Tochter
Lea (als Leibmagd) gegeben hatte; diese hatte sie dem Jakob geboren,
zusammen sechzehn Seelen.~-- \bibleverse{19} Die Söhne der Rahel, der
Frau Jakobs, waren Joseph und Benjamin. \bibleverse{20} Dem Joseph aber
waren in Ägypten Manasse und Ephraim geboren, die ihm Asenath, die
Tochter Potipheras, des Priesters von On, geboren hatte. \bibleverse{21}
Die Söhne Benjamins waren Bela, Becher und Asbel, Gera und Naaman, Ehi
und Ros, Muppim und Huppim und Ard. \bibleverse{22} Dies sind die Söhne
der Rahel, die dem Jakob geboren waren, insgesamt vierzehn Seelen.~--
\bibleverse{23} Die Söhne Dans aber waren: Husim; \bibleverse{24} und
die Söhne Naphthalis: Jahzeel, Guni, Jezer und Sillem. \bibleverse{25}
Dies sind die Söhne der Bilha, die Laban seiner Tochter Rahel (als
Leibmagd) gegeben hatte; diese hatte sie dem Jakob geboren, insgesamt
sieben Seelen.~-- \bibleverse{26} Die Gesamtzahl der Seelen✲, die mit
Jakob nach Ägypten kamen, seine leiblichen Nachkommen, ungerechnet die
Frauen der Söhne Jakobs, betrug sechsundsechzig. \bibleverse{27} Die
Söhne Josephs aber, die ihm in Ägypten geboren wurden, waren zwei
Seelen. Daher betrug die Gesamtzahl der Seelen des Hauses Jakobs, die
nach Ägypten kamen, siebzig.

\hypertarget{cc-joseph-begruxfcuxdft-seinen-vater-in-gosen-und-erteilt-seinen-bruxfcdern-anweisung-fuxfcr-ihr-benehmen-beim-pharao}{%
\subparagraph{cc) Joseph begrüßt seinen Vater in Gosen und erteilt
seinen Brüdern Anweisung für ihr Benehmen beim
Pharao}\label{cc-joseph-begruxfcuxdft-seinen-vater-in-gosen-und-erteilt-seinen-bruxfcdern-anweisung-fuxfcr-ihr-benehmen-beim-pharao}}

\bibleverse{28} Jakob hatte aber Juda zu Joseph vorausgesandt, damit
dieser ihm die Landschaft Gosen im voraus anweisen lasse. Als sie nun im
Lande Gosen angekommen waren, \bibleverse{29} ließ Joseph seinen Wagen
anspannen und fuhr seinem Vater Israel nach Gosen entgegen; und als er
vor ihm erschien, fiel er ihm um den Hals und weinte lange an seinem
Halse. \bibleverse{30} Israel aber sagte zu Joseph: »Nun will ich gern
sterben, nachdem ich dein Angesicht gesehen habe (und weiß), daß du noch
am Leben bist.« \bibleverse{31} Hierauf sagte Joseph zu seinen Brüdern
und zu den anderen Angehörigen seines Vaters: »Ich will jetzt hinfahren
und dem Pharao Bericht erstatten und ihm melden: ›Meine Brüder sowie die
Angehörigen meines Vaters, die bisher im Lande Kanaan gewohnt haben,
sind zu mir gekommen. \bibleverse{32} Diese Leute sind Hirten von
Kleinvieh -- sie sind von jeher Viehzüchter gewesen --, und sie haben
ihr Kleinvieh, ihre Rinder und ihren gesamten Besitz mitgebracht.‹
\bibleverse{33} Wenn euch dann der Pharao rufen läßt und euch fragt:
›Was ist euer Gewerbe?‹, \bibleverse{34} so antwortet: ›Deine Knechte
sind von Jugend auf bis jetzt Viehzüchter gewesen, wir wie auch schon
unsere Väter‹, -- damit ihr im Lande Gosen wohnen dürft; denn alle
Hirten von Kleinvieh sind den Ägyptern ein Greuel\textless sup
title=``d.h. ein Gegenstand des Abscheus''\textgreater✲.«

\hypertarget{u-jakob-vor-dem-pharao-und-in-gosen-folgen-der-hungersnot-fuxfcr-die-uxe4gypter}{%
\paragraph{u) Jakob vor dem Pharao und in Gosen; Folgen der Hungersnot
für die
Ägypter}\label{u-jakob-vor-dem-pharao-und-in-gosen-folgen-der-hungersnot-fuxfcr-die-uxe4gypter}}

\hypertarget{aa-der-pharao-sagt-fuxfcnf-suxf6hnen-jakobs-bei-einer-vorstellung-die-ansiedlung-in-gosen-zu}{%
\subparagraph{aa) Der Pharao sagt fünf Söhnen Jakobs bei einer
Vorstellung die Ansiedlung in Gosen
zu}\label{aa-der-pharao-sagt-fuxfcnf-suxf6hnen-jakobs-bei-einer-vorstellung-die-ansiedlung-in-gosen-zu}}

\hypertarget{section-46}{%
\section{47}\label{section-46}}

\bibleverse{1} So ging denn Joseph hin und erstattete dem Pharao Bericht
mit den Worten: »Mein Vater und meine Brüder sind mit ihrem Kleinvieh,
ihren Rindern und ihrem gesamten Besitz aus dem Lande Kanaan angekommen
und befinden sich jetzt im Lande Gosen.« \bibleverse{2} Er hatte aber
aus der Gesamtzahl seiner Brüder fünf Männer mitgebracht, die er dem
Pharao vorstellte. \bibleverse{3} Als nun der Pharao die Brüder Josephs
fragte: »Was ist euer Gewerbe?«, antworteten sie dem Pharao: »Deine
Knechte sind Hirten von Kleinvieh, wir wie auch schon unsere Väter.«
\bibleverse{4} Weiter sagten sie zum Pharao: »Wir sind hergekommen, um
eine Zeitlang als Fremdlinge hier im Lande zu wohnen, weil deine Knechte
keine Weide mehr für ihr Kleinvieh haben: so schwer liegt die Hungersnot
auf dem Lande Kanaan. Laß nun doch deine Knechte im Lande Gosen wohnen!«
\bibleverse{5} Da sagte der Pharao zu Joseph: »Dein Vater und deine
Brüder sind zu dir gekommen: \bibleverse{6} das Land Ägypten steht dir
zur Verfügung; laß deinen Vater und deine Brüder im besten Teil des
Landes wohnen! Sie dürfen im Lande Gosen wohnen; und wenn du siehst, daß
tüchtige Leute unter ihnen sind, so mache sie zu Oberhirten über meine
eigenen Herden!«

\hypertarget{bb-jakob-dem-pharao-vorgestellt-und-sodann-in-gosen-angesiedelt}{%
\subparagraph{bb) Jakob dem Pharao vorgestellt und sodann in Gosen
angesiedelt}\label{bb-jakob-dem-pharao-vorgestellt-und-sodann-in-gosen-angesiedelt}}

\bibleverse{7} Hierauf ließ Joseph auch seinen Vater Jakob eintreten und
stellte ihn dem Pharao vor; Jakob aber begrüßte den Pharao mit einem
Segenswunsch. \bibleverse{8} Da sagte der Pharao zu Jakob: »Wie groß ist
die Zahl deiner Lebensjahre?« \bibleverse{9} Jakob antwortete dem
Pharao: »Die Zahl der Jahre meiner Wanderschaft beträgt
hundertunddreißig Jahre; gering an Zahl und mühselig sind die Tage
meiner Lebensjahre gewesen und reichen nicht an die Tage der Lebensjahre
meiner Väter in der Zeit ihrer Wanderschaft heran.« \bibleverse{10}
Hierauf entfernte sich Jakob vom Pharao mit einem Segenswunsch für ihn.
\bibleverse{11} Joseph aber wies seinem Vater und seinen Brüdern
Wohnsitze an und verlieh ihnen eigenen Grundbesitz in Ägypten, im besten
Teile des Landes, nämlich in der Landschaft Ramses, wie der Pharao
befohlen hatte. \bibleverse{12} Und Joseph versorgte seinen Vater, seine
Brüder und alle Angehörigen seines Vaters mit Brotkorn, einen jeden nach
der Zahl seiner Familienglieder.

\hypertarget{cc-joseph-verschafft-dem-pharao-zunuxe4chst-alles-geld-und-alles-vieh-in-uxe4gypten-und-macht-sodann-alles-ackerland-dem-pharao-zinsbar}{%
\subparagraph{cc) Joseph verschafft dem Pharao zunächst alles Geld und
alles Vieh in Ägypten und macht sodann alles Ackerland dem Pharao
zinsbar}\label{cc-joseph-verschafft-dem-pharao-zunuxe4chst-alles-geld-und-alles-vieh-in-uxe4gypten-und-macht-sodann-alles-ackerland-dem-pharao-zinsbar}}

\bibleverse{13} Es gab aber kein Brotkorn im ganzen Lande; denn die
Hungersnot war überaus drückend, so daß Ägypten ebenso wie das Land
Kanaan infolge der Hungersnot am Verschmachten war. \bibleverse{14} So
brachte denn Joseph (schließlich) alles Geld, das sich im Lande Ägypten
und in Kanaan vorfand, in seiner Hand zusammen für das Brotkorn, das man
kaufen mußte; und Joseph lieferte das Geld an das Haus des Pharaos ab.
\bibleverse{15} Als dann im Lande Ägypten und in Kanaan kein Geld mehr
vorhanden war, kamen alle Ägypter zu Joseph und sagten: »Schaffe uns
Brot! Warum sollen wir vor deinen Augen sterben? Denn das Geld ist zu
Ende gegangen!« \bibleverse{16} Joseph antwortete: »Bringt euer Vieh
her, so will ich euch Brotkorn als Entgelt für euer Vieh geben, wenn ihr
kein Geld mehr habt.« \bibleverse{17} Da brachten sie ihr Vieh zu
Joseph, und dieser gab ihnen Brotkorn als Entgelt für die Pferde, für
die Schaf- und Rinderherden und für die Esel; so versorgte er sie in
jenem Jahr mit Brotkorn um den Preis ihres gesamten Viehs.

\bibleverse{18} Als nun dieses Jahr zu Ende war, kamen sie im nächsten
Jahre wieder zu ihm und sagten: »Wir können es unserm Herrn nicht
verhehlen, daß das Geld zu Ende ist und, weil auch unser Bestand an Vieh
schon an unsern Herrn übergegangen ist, nichts mehr zur Verfügung
unseres Herrn übriggeblieben ist als unser Leib und unsere Äcker.
\bibleverse{19} Warum sollen wir vor deinen Augen zugrunde gehen, wir
samt unserm Landbesitz? Kaufe uns und unsern Landbesitz um Brotkorn, so
wollen wir samt unserm Landbesitz dem Pharao leibeigen sein; aber gib
uns Saatkorn, damit wir am Leben bleiben und nicht verhungern und die
Felder nicht zur Wüste werden!« \bibleverse{20} So kaufte denn Joseph
alles Ackerland der Ägypter für den Pharao auf; denn die Ägypter
verkauften ein jeder seine Felder, weil die Hungersnot schwer auf ihnen
lastete\textless sup title=``oder: sie dazu zwang''\textgreater✲; und so
wurde das Land Eigentum des Pharaos. \bibleverse{21} Und was die
Bevölkerung betrifft, so machte er sie leibeigen von einem Ende des
ägyptischen Gebiets bis zum andern. \bibleverse{22} Nur die Ländereien
der Priester kaufte er nicht an; denn die Priester bezogen ein festes
Einkommen von seiten des Pharaos und lebten von ihrem festen Einkommen,
das der Pharao ihnen angewiesen hatte\textless sup title=``oder:
gewährte''\textgreater✲; deshalb brauchten sie ihre Ländereien nicht zu
verkaufen. \bibleverse{23} Joseph aber sagte zum Volk: »Ich habe nunmehr
euch und eure Äcker für den Pharao angekauft; hier habt ihr Saatkorn zum
Besäen der Äcker! \bibleverse{24} Aber von dem Ertrage müßt ihr ein
Fünftel an den Pharao abgeben; die übrigen vier Fünftel dagegen sollt
ihr behalten zur Aussaat für die Felder sowie zur Nahrung für euch und
euer Gesinde und zur Ernährung eurer Familien.« \bibleverse{25} Da
antworteten sie: »Du hast uns am Leben erhalten! Möchten wir nur Gnade
finden vor den Augen unsers Herrn, so wollen wir gern dem Pharao
leibeigen sein!« \bibleverse{26} So machte Joseph es zu einer
gesetzlichen Verpflichtung, die bis auf diesen Tag für den Grundbesitz
der Ägypter besteht, daß dem Pharao der fünfte Teil des Ernteertrages
gehört; nur die Ländereien der Priester allein kamen nicht in den Besitz
des Pharaos.

\hypertarget{dd-gluxfcckliche-lage-der-israeliten-in-uxe4gypten-jakobs-letzter-wunsch-bezuxfcglich-seiner-bestattung}{%
\subparagraph{dd) Glückliche Lage der Israeliten in Ägypten; Jakobs
letzter Wunsch bezüglich seiner
Bestattung}\label{dd-gluxfcckliche-lage-der-israeliten-in-uxe4gypten-jakobs-letzter-wunsch-bezuxfcglich-seiner-bestattung}}

\bibleverse{27} So siedelten sich denn die Israeliten in Ägypten, in der
Landschaft Gosen, an; sie setzten sich darin fest und mehrten sich so,
daß sie überaus zahlreich wurden. \bibleverse{28} Jakob aber lebte in
Ägypten noch siebzehn Jahre, so daß seine ganze Lebensdauer 147~Jahre
betrug. \bibleverse{29} Als dann die Zeit herankam, daß er sterben
sollte, ließ er seinen Sohn Joseph rufen und sagte zu ihm: »Wenn ich dir
etwas gelte, so lege deine Hand unter meine Hüfte und erweise mir die
Liebe und Treue, mich nicht in Ägypten zu begraben, \bibleverse{30}
sondern ich möchte bei meinen Vätern ruhen! Darum bringe mich aus
Ägypten weg und begrabe mich in ihrer Ruhestätte!« Da antwortete er:
»Ja, ich werde nach deinem Wunsche tun.« \bibleverse{31} Da sagte er:
»Schwöre es mir!«, und er schwur ihm. Israel aber beugte\textless sup
title=``oder: verneigte''\textgreater✲ sich anbetend auf\textless sup
title=``oder: über''\textgreater✲ das Kopfende des Bettes hin.

\hypertarget{v-jakob-nimmt-die-beiden-suxf6hne-josephs-an-kindes-statt-an}{%
\paragraph{v) Jakob nimmt die beiden Söhne Josephs an Kindes Statt
an}\label{v-jakob-nimmt-die-beiden-suxf6hne-josephs-an-kindes-statt-an}}

\hypertarget{section-47}{%
\section{48}\label{section-47}}

\bibleverse{1} Nach diesen Begebenheiten meldete man dem Joseph: »Wisse,
dein Vater ist erkrankt«; da nahm er seine beiden Söhne Manasse und
Ephraim mit sich. \bibleverse{2} Als man nun dem Jakob mitteilte: »Dein
Sohn Joseph kommt zu dir«, da machte Israel sich stark, setzte sich im
Bett aufrecht hin \bibleverse{3} und sagte dann zu Joseph: »Der
allmächtige Gott ist mir einst zu Lus\textless sup title=``=~Bethel;
vgl. 28,19''\textgreater✲ im Lande Kanaan erschienen und hat mich
gesegnet \bibleverse{4} mit den Worten\textless sup title=``vgl.
35,11-12''\textgreater✲: ›Ich will dich fruchtbar machen und mehren und
dich zu einer Menge\textless sup title=``oder: Gemeinde''\textgreater✲
von Stämmen werden lassen und will dieses Land deiner Nachkommenschaft
nach dir zu ewigem Besitz geben.‹ \bibleverse{5} Und nun sollen deine
beiden Söhne, die dir im Lande Ägypten geboren worden sind, ehe ich zu
dir nach Ägypten kam, mir gehören: Ephraim und Manasse sollen mir
gehören wie Ruben und Simeon. \bibleverse{6} Deine übrigen Kinder aber,
die dir nach ihnen geboren sind, sollen dir gehören: den Namen (eines)
ihrer (beiden) Brüder sollen sie in ihrem\textless sup title=``oder:
deren?''\textgreater✲ Erbteil führen. \bibleverse{7} Was mich aber
betrifft: als ich aus Mesopotamien heimkehrte, starb mir Rahel unterwegs
im Lande Kanaan, als nur noch eine Strecke Weges bis Ephrath zu gehen
war, und ich begrub sie dort am Wege nach Ephrath {[}das ist
Bethlehem{]}.«\textless sup title=``vgl. 35,16-20''\textgreater✲

\hypertarget{jakob-segnet-die-beiden-josephssuxf6hne-unter-bevorzugung-des-juxfcngeren-ephraim}{%
\paragraph{Jakob segnet die beiden Josephssöhne unter Bevorzugung des
jüngeren
Ephraim}\label{jakob-segnet-die-beiden-josephssuxf6hne-unter-bevorzugung-des-juxfcngeren-ephraim}}

\bibleverse{8} Als nun Israel die Söhne Josephs sah, fragte er: »Wer
sind diese?« \bibleverse{9} Joseph antwortete seinem Vater: »Es sind
meine Söhne, die Gott mir hier geschenkt hat.« Da sagte er: »Bringe sie
her zu mir, damit ich sie segne!« \bibleverse{10} Israels Augen waren
nämlich infolge des Alters schwach geworden, so daß er nicht (mehr)
sehen konnte. Als er sie nun dicht an ihn herangebracht hatte, küßte und
umarmte er sie. \bibleverse{11} Hierauf sagte Israel zu Joseph: »Ich
hatte nicht gehofft, dein Angesicht je wiederzusehen; und nun hat Gott
mich sogar noch Kinder von dir sehen lassen!« \bibleverse{12} Darauf zog
Joseph sie wieder von seinen\textless sup title=``d.h.
Jakobs''\textgreater✲ Knien weg und verneigte sich vor ihm bis zur Erde.
\bibleverse{13} Dann nahm Joseph sie beide, Ephraim mit seiner rechten
Hand auf der linken Seite Israels und Manasse mit seiner linken Hand auf
der rechten Seite Israels, und ließ sie so an ihn herantreten.
\bibleverse{14} Da streckte Israel seine rechte Hand aus und legte sie
auf das Haupt Ephraims, obgleich er der jüngere war, und seine linke
Hand auf das Haupt Manasses, indem er seine Arme übers Kreuz legte; denn
Manasse war der Erstgeborene. \bibleverse{15} Dann segnete er Joseph mit
den Worten: »Der Gott, vor dessen Angesicht meine Väter Abraham und
Isaak gewandelt sind, der Gott, der mein Hirt gewesen ist, seitdem ich
lebe, bis auf diesen Tag, \bibleverse{16} der Engel, der mich aus allem
Unglück errettet hat: er segne diese Knaben, daß durch sie mein Name und
der Name meiner Väter Abraham und Isaak fortlebt und sie sich zu einer
großen Menge auf Erden\textless sup title=``oder: inmitten des
Landes?''\textgreater✲ vermehren!« \bibleverse{17} Als nun Joseph sah,
daß sein Vater seine rechte Hand andauernd auf das Haupt Ephraims legte,
mißfiel ihm dies; er faßte daher die Hand seines Vaters, um sie vom
Haupt Ephraims auf das Haupt Manasses zu legen, \bibleverse{18} indem er
dabei zu seinem Vater sagte: »Nicht so, lieber Vater, denn dieser ist
der Erstgeborene; lege deine rechte Hand auf sein Haupt!«
\bibleverse{19} Aber sein Vater wollte nicht und sagte: »Ich weiß es
wohl, mein Sohn, ich weiß es! Auch dieser wird zu einem
Volk\textless sup title=``oder: Stamm''\textgreater✲ werden, und auch er
wird groß werden; jedoch sein jüngerer Bruder wird größer sein als er,
und seine Nachkommen werden zu einer Menge von Völkern werden.«
\bibleverse{20} So segnete er sie denn an jenem Tage mit den Worten:
»Mit deinem Namen werden die Israeliten einen Segenswunsch aussprechen,
indem sie sagen: ›Gott mache dich gleich Ephraim und gleich Manasse!‹«
Damit stellte er Ephraim vor✲ Manasse.

\hypertarget{jakob-verleiht-seinem-lieblingssohne-joseph-einen-von-ihm-eroberten-landstrich-sichem-in-kanaan}{%
\paragraph{Jakob verleiht seinem Lieblingssohne Joseph einen von ihm
eroberten Landstrich (Sichem?) in
Kanaan}\label{jakob-verleiht-seinem-lieblingssohne-joseph-einen-von-ihm-eroberten-landstrich-sichem-in-kanaan}}

\bibleverse{21} Hierauf sagte Israel zu Joseph: »Ich werde nun sterben;
aber Gott wird mit euch sein und euch in das Land eurer Väter
zurückkehren lassen. \bibleverse{22} Ich aber gebe dir einen Landstrich,
den du vor deinen Brüdern voraushaben sollst; ich habe ihn den Amoritern
einst mit meinem Schwert und meinem Bogen abgenommen.«

\hypertarget{w-jakobs-weissagungsspruxfcche-uxfcber-seine-suxf6hne}{%
\paragraph{w) Jakobs Weissagungssprüche über seine
Söhne}\label{w-jakobs-weissagungsspruxfcche-uxfcber-seine-suxf6hne}}

\hypertarget{section-48}{%
\section{49}\label{section-48}}

\bibleverse{1} Dann berief Jakob seine Söhne und sagte: »Versammelt
euch, damit ich euch das verkünde, was euch in künftigen Tagen
widerfahren wird! \bibleverse{2} Schart euch zusammen und hört zu, ihr
Söhne Jakobs, ja, hört euren Vater Israel an!

\bibleverse{3} Du, Ruben, bist mein erstgeborener Sohn, meine Kraft und
der Erstling meiner Stärke\textless sup title=``oder:
Mannheit''\textgreater✲, bevorzugt an Würde und bevorzugt an Macht!
\bibleverse{4} Doch überwallend wie Wasser, sollst du keinen Vorzug
genießen! Denn du hast deines Vaters Lager bestiegen\textless sup
title=``vgl. 35,22''\textgreater✲; damals hast du es entweiht: mein Bett
hat er bestiegen!

\bibleverse{5} Simeon und Levi sind Brüder; Werkzeuge der Gewalttat sind
ihre Schwerter: \bibleverse{6} ich will nichts zu schaffen haben mit
ihren Ratschlägen, will keine Gemeinschaft haben mit ihren Entschlüssen!
Denn in ihrem Zorn haben sie Männer erschlagen und in ihrem
Mutwillen\textless sup title=``oder: Übermut''\textgreater✲ Stiere
verstümmelt. \bibleverse{7} Verflucht sei ihr Zorn, daß er so
gewalttätig ist, und ihre Wut, daß sie sich so grausam zeigt! Ich will
sie zerteilen in Jakob und will sie zerstreuen in Israel!

\bibleverse{8} Juda\textless sup title=``d.h. der
Gepriesene''\textgreater✲, du bist's, den deine Brüder preisen werden!
Deine Hand wird deinen Feinden auf dem Nacken liegen; vor dir werden
sich verbeugen die Söhne deines Vaters. \bibleverse{9} Ein junger Löwe
ist Juda: vom Raub bist du emporgestiegen, mein Sohn. Er kauert sich
nieder, streckt sich hin wie ein Löwe und wie eine Löwin; wer darf ihn
aufstören? \bibleverse{10} Nicht wird das Zepter von Juda weichen noch
der Herrscherstab zwischen seinen Füßen hinweg, bis der kommt, dem
er\textless sup title=``d.h. der Herrscherstab''\textgreater✲ gebührt,
und die Völker werden ihm Gehorsam leisten. \bibleverse{11} Er bindet
sein Eselfüllen an den Weinstock und das Junge seiner Eselin an die
Edelrebe; er wäscht im Wein sein Gewand und im Blut der Trauben seinen
Mantel; \bibleverse{12} trübe sind ihm die Augen vom Wein, und die Zähne
schimmern von Milch.

\bibleverse{13} Sebulon wird bis hin zum Meeresstrand wohnen, und zwar
am Gestade der Schiffe, und mit dem Rücken wird er sich an Sidon lehnen.

\bibleverse{14} Issaschar\textless sup title=``d.h. Mann des
Lohnes''\textgreater✲ ist ein starkknochiger Esel, der zwischen den
Hürden lagert. \bibleverse{15} Als er sah, daß die Ruhe etwas Schönes
und sein Land gar lieblich sei, da beugte er seinen Nacken\textless sup
title=``oder: Rücken''\textgreater✲ zum Lasttragen und wurde zum
dienstbaren Fronknecht.

\bibleverse{16} Dan\textless sup title=``d.h. Richter''\textgreater✲
wird seinem Volke Recht schaffen wie irgendeiner von den Stämmen
Israels; \bibleverse{17} Dan wird eine Schlange am Wege sein, eine
Hornotter am Pfad, die das Roß in die Fersen sticht, so daß sein Reiter
rücklings zu Boden stürzt.~-- \bibleverse{18} Auf dein Heil\textless sup
title=``oder: deine Hilfe''\textgreater✲ harre ich, HERR!

\bibleverse{19} Gad -- Kriegsscharen werden ihn bedrängen, er aber wird
ihnen nachdrängen auf der Ferse.

\bibleverse{20} Asser hat Brot im Überfluß; auch Königsleckerbissen wird
er liefern.

\bibleverse{21} Naphthali ist eine flüchtige\textless sup title=``oder:
dahinstürmende''\textgreater✲ Hirschkuh; er ist's, der schöne Lieder
vernehmen läßt.

\bibleverse{22} Joseph ist eine junge Fruchtrebe, eine junge Fruchtrebe
am Quell: ihre Schößlinge ranken über die Mauer empor. \bibleverse{23}
Wenn die Pfeilschützen ihm zusetzen und ihn beschießen und befehden,
\bibleverse{24} bleibt sein Bogen doch beständig gespannt, und gelenkig
sind seine Arme und Hände infolge der Hilfe des starken Gottes Jakobs,
von dorther, wo der Hirt, der Felsen Israels ist, \bibleverse{25} von
dem Gott deines Vaters: er helfe dir! --, und mit dem Beistand des
Allmächtigen: er segne dich mit Segensfülle vom Himmel droben, mit
Segensfülle aus der Urflut, die in der Tiefe lagert, mit Segensfülle aus
Brüsten und Mutterschoß! \bibleverse{26} Die Segnungen deines Vaters
überragen die Segensfülle der uralten Berge, die köstlichen Gaben der
ewigen Höhen; mögen sie zuteil werden dem Haupte Josephs und dem
Scheitel des Geweihten\textless sup title=``oder: Fürsten''\textgreater✲
unter seinen Brüdern!

\bibleverse{27} Benjamin ist ein räuberischer Wolf: am Morgen noch wird
er den Raub verzehren und abends Beute verteilen.«

\hypertarget{jakobs-nochmalige-bitte-vgl.-4729-31-um-seine-beisetzung-im-erbbegruxe4bnis-zu-hebron-sein-tod}{%
\paragraph{Jakobs nochmalige Bitte (vgl. 47,29-31) um seine Beisetzung
im Erbbegräbnis zu Hebron; sein
Tod}\label{jakobs-nochmalige-bitte-vgl.-4729-31-um-seine-beisetzung-im-erbbegruxe4bnis-zu-hebron-sein-tod}}

\bibleverse{28} Dies sind die zwölf Stämme Israels insgesamt, und dies
ist es, was ihr Vater zu ihnen geredet und womit er sie gesegnet hat,
einen jeden mit einem besonderen Segen. \bibleverse{29} Dann erteilte er
ihnen folgenden Auftrag: »Wenn ich jetzt zu meinen Stammesgenossen
versammelt\textless sup title=``oder: eingegangen''\textgreater✲ bin, so
begrabt mich bei meinen Vätern in der Höhle auf dem Felde des Hethiters
Ephron, \bibleverse{30} in der Höhle auf dem Felde Machpela östlich von
Mamre im Lande Kanaan, welche (Höhle) Abraham samt dem Felde von dem
Hethiter Ephron zum Erbbegräbnis gekauft hat. \bibleverse{31} Dort ist
Abraham und seine Frau Sara begraben, dort ist Isaak und seine Frau
Rebekka begraben, und dort habe ich Lea begraben. \bibleverse{32}
Abgekauft ist das Feld mit der Höhle darauf den Hethitern.«

\bibleverse{33} Als nun Jakob mit der Mitteilung seines letzten Willens
an seine Söhne zu Ende war, zog er seine Füße auf das Bett zurück und
verschied und wurde zu seinen Stammesgenossen versammelt.

\hypertarget{x-jakobs-begruxe4bnis-josephs-versuxf6hnlichkeit-und-tod}{%
\paragraph{x) Jakobs Begräbnis; Josephs Versöhnlichkeit und
Tod}\label{x-jakobs-begruxe4bnis-josephs-versuxf6hnlichkeit-und-tod}}

\hypertarget{aa-jakobs-einbalsamierung-und-feierliche-uxfcberfuxfchrung-nach-dem-erbbegruxe4bnis-in-hebron}{%
\subparagraph{aa) Jakobs Einbalsamierung und feierliche Überführung nach
dem Erbbegräbnis in
Hebron}\label{aa-jakobs-einbalsamierung-und-feierliche-uxfcberfuxfchrung-nach-dem-erbbegruxe4bnis-in-hebron}}

\hypertarget{section-49}{%
\section{50}\label{section-49}}

\bibleverse{1} Da warf sich Joseph über seinen Vater hin, weinte über
ihn gebeugt und küßte ihn. \bibleverse{2} Hierauf befahl Joseph den
Ärzten, die in seinem Dienst standen, seinen Vater einzubalsamieren; da
balsamierten die Ärzte Israel ein. \bibleverse{3} Darüber vergingen
vierzig Tage; denn so lange Zeit ist zum Einbalsamieren erforderlich;
und die Ägypter trauerten siebzig Tage lang um ihn. \bibleverse{4} Als
dann die Zeit der Trauer um ihn vorüber war, sagte Joseph zu den
Hofbeamten des Pharaos: »Wenn ich Gnade bei euch gefunden
habe\textless sup title=``=~wenn ihr mir eine Liebe erweisen
wollt''\textgreater✲, so bringt doch folgendes zur Kenntnis des Pharaos:
\bibleverse{5} Mein Vater hat mir einen Eid abgenommen und mir geboten:
›Wenn ich tot bin, sollst du mich in meinem Grabe beisetzen, das ich mir
im Lande Kanaan angelegt habe!‹ Deshalb möchte ich nun hinaufziehen, um
meinen Vater zu begraben; dann kehre ich wieder zurück.« \bibleverse{6}
Da ließ ihm der Pharao sagen: »Ziehe hinauf und begrabe deinen Vater,
wie du ihm hast schwören müssen!« \bibleverse{7} So zog denn Joseph hin,
um seinen Vater zu begraben, und mit ihm zogen alle Diener des Pharaos,
die höchsten Beamten seines Hofes und alle Würdenträger des Landes
Ägypten, \bibleverse{8} dazu die ganze Familie Josephs, seine Brüder und
überhaupt die Angehörigen seines Vaters; nur ihre kleinen Kinder sowie
ihr Kleinvieh und ihre Rinder ließen sie im Lande Gosen zurück.
\bibleverse{9} Ebenso zogen sowohl Wagen als auch Reiter mit ihm, so daß
es ein gewaltiger Zug war. \bibleverse{10} Als sie nun nach
Goren-Haatad, das jenseits des Jordans liegt, gekommen waren,
veranstalteten sie dort eine große\textless sup title=``oder:
würdevolle''\textgreater✲ und sehr feierliche Totenklage, und (Joseph)
stellte um seinen Vater eine siebentägige Leichenfeier an.
\bibleverse{11} Als nun die dort wohnenden Kanaanäer die Trauerfeier bei
Goren-Haatad sahen, sagten sie: »Da findet eine ernste Trauerfeier der
Ägypter statt«; daher gab man dem Ort den Namen ›Trauerfeier der
Ägypter‹; er liegt jenseits des Jordans. \bibleverse{12} Seine Söhne
hielten es dann mit ihm so, wie er ihnen befohlen hatte: \bibleverse{13}
sie brachten ihn nämlich in das Land Kanaan und bestatteten ihn in der
Höhle auf dem Felde Machpela, welche (Höhle) Abraham samt dem Felde zum
Erbbegräbnis von dem Hethiter Ephron gekauft hatte, östlich von Mamre.
\bibleverse{14} Hierauf kehrte Joseph, nachdem er seinen Vater bestattet
hatte, nach Ägypten zurück, er und seine Brüder und alle, die mit ihm
hinaufgezogen waren, um seinen Vater zu bestatten.

\hypertarget{bb-josephs-edelmut-gegen-seine-bruxfcder}{%
\subparagraph{bb) Josephs Edelmut gegen seine
Brüder}\label{bb-josephs-edelmut-gegen-seine-bruxfcder}}

\bibleverse{15} Als nun die Brüder Josephs sahen, daß ihr Vater tot war,
sagten sie: »Wie nun, wenn Joseph feindselig gegen uns aufträte und uns
all das Böse vergälte, das wir ihm zugefügt haben?« \bibleverse{16}
Darum sandten sie zu Joseph und ließen ihm sagen: »Dein Vater hat uns
vor seinem Tode folgende Weisung gegeben: \bibleverse{17} ›So sollt ihr
zu Joseph sagen: Ach bitte, vergib doch deinen Brüdern ihr Vergehen und
ihre Verschuldung, daß sie so übel an dir gehandelt haben!‹ So vergib
uns nun doch unser Vergehen; wir sind ja doch auch Knechte✲ des Gottes
deines Vaters!« Da weinte Joseph, als sie ihm das sagen ließen.
\bibleverse{18} Hierauf gingen seine Brüder selbst hin, warfen sich vor
ihm nieder und sagten: »Hier sind wir als deine Knechte!«
\bibleverse{19} Joseph aber antwortete ihnen: »Seid ohne Furcht! Denn
stehe ich etwa an Gottes Statt? \bibleverse{20} Ihr freilich hattet
Böses gegen mich im Sinn, aber Gott gedachte es zum Guten zu wenden, um
das auszuführen, was jetzt klar zutage liegt, nämlich um ein zahlreiches
Volk\textless sup title=``oder: viele Menschen''\textgreater✲ am Leben
zu erhalten. \bibleverse{21} Fürchtet euch also nicht! Ich selbst werde
euch und eure Kinder versorgen.« So tröstete er sie und redete ihnen
freundlich zu.

\hypertarget{cc-josephs-alter-und-tod-sein-letzter-wunsch}{%
\subparagraph{cc) Josephs Alter und Tod; sein letzter
Wunsch}\label{cc-josephs-alter-und-tod-sein-letzter-wunsch}}

\bibleverse{22} So wohnte denn Joseph in Ägypten samt der ganzen Familie
seines Vaters, und Joseph wurde 110~Jahre alt. \bibleverse{23} Von
Ephraim sah er Urenkel, und auch die Söhne Machirs, des Sohnes Manasses,
wurden noch bei Lebzeiten Josephs geboren. \bibleverse{24} Da sagte
Joseph zu seinen Brüdern: »Ich stehe nun nahe vor dem Tode; Gott aber
wird sich euer sicherlich gnädig annehmen und euch aus diesem Lande in
das Land zurückführen, das er Abraham, Isaak und Jakob zugeschworen
hat.« \bibleverse{25} Hierauf ließ Joseph die Söhne Israels folgendes
beschwören: »Wenn Gott sich (dereinst) euer gnädig annehmen wird, dann
sollt ihr meine Gebeine von hier mitnehmen.« \bibleverse{26} Dann starb
Joseph im Alter von 110~Jahren; man balsamierte ihn ein und legte ihn in
Ägypten in einen Sarg\textless sup title=``eig. Schrein,
Lade''\textgreater✲.
