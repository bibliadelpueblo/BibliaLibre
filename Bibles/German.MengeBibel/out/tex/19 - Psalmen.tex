\hypertarget{die-psalmen}{%
\section{DIE PSALMEN}\label{die-psalmen}}

\hypertarget{erstes-buch-psalm-1-41}{%
\subsection{Erstes Buch (Psalm 1-41)}\label{erstes-buch-psalm-1-41}}

\hypertarget{die-zwei-lebenswege}{%
\subsubsection{Die zwei Lebenswege}\label{die-zwei-lebenswege}}

\hypertarget{section}{%
\section{1}\label{section}}

\bibleverse{1}Wohl dem, der nicht wandelt

im Rat\textless sup title=``=~nach den Lehren''\textgreater✲ der
Gottlosen und nicht tritt auf den Weg der Sünder, noch sitzt im Kreise
der Spötter,

\bibleverse{2}vielmehr Gefallen hat am Gesetz des HERRN

und sinnt über sein Gesetz bei Tag und bei Nacht!

\bibleverse{3}Der gleicht einem Baum, gepflanzt an Wasserbächen,

der seine Früchte bringt zu rechter Zeit und dessen Laub nicht welkt;
und alles, was er beginnt, das gelingt.

\bibleverse{4}Nicht also die Gottlosen: nein,

sie gleichen der Spreu, die der Wind verweht.

\bibleverse{5}Darum werden die Gottlosen nicht im Gericht bestehn

und die Sünder nicht in der Gemeinde der Gerechten.

\bibleverse{6}Denn es kennt der HERR den Weg der Gerechten;

doch der Gottlosen Weg führt ins Verderben.{A}

\hypertarget{der-sieg-gottes-und-des-von-ihm-gesalbten-kuxf6nigs-uxfcber-die-tobende-vuxf6lkerwelt}{%
\subsubsection{Der Sieg Gottes und des von ihm gesalbten Königs über die
tobende
Völkerwelt}\label{der-sieg-gottes-und-des-von-ihm-gesalbten-kuxf6nigs-uxfcber-die-tobende-vuxf6lkerwelt}}

\hypertarget{section-1}{%
\section{2}\label{section-1}}

\bibleverse{1}Was soll das Toben der Völker

und das eitle Sinnen der Völkerschaften?\textless sup title=``Apg
4,25''\textgreater✲

\bibleverse{2}Die Könige der Erde rotten sich zusammen,

und die Fürsten halten Rat miteinander gegen den HERRN und den von ihm
Gesalbten:

\bibleverse{3}»Laßt uns zerreißen ihre Bande

und von uns werfen ihre Fesseln!«

\bibleverse{4}Der im Himmel thront, der lacht,

der Allherr spottet ihrer.

\bibleverse{5}Dann aber wird er zu ihnen reden in seinem Zorn

und sie schrecken in seinem Ingrimm:

\bibleverse{6}»Habe ich doch meinen König eingesetzt

auf dem Zion, meinem heiligen Berge!«~--

\bibleverse{7}Laßt mich kundtun den Ratschluß des HERRN!

Er hat zu mir gesagt: »Mein Sohn bist du; ich selbst habe heute dich
gezeugt\textless sup title=``Apg 13,33; Hebr 1,5; 5,5''\textgreater✲.

\bibleverse{8}Fordre von mir, so gebe ich dir die Völker zum Erbe

und dir zum Besitz die Enden der Erde.

\bibleverse{9}Du sollst sie mit eiserner Keule zerschmettern,

wie Töpfergeschirr sie zerschlagen!«~--

\bibleverse{10}So nehmt denn Klugheit an, ihr Könige,

laßt euch warnen, ihr Richter✲ der Erde!

\bibleverse{11}Dienet dem HERRN mit Furcht

und jubelt ihm zu mit Zittern!

\bibleverse{12}Küsset den Sohn, auf daß er nicht zürne

und ihr zugrunde geht auf eurem Wege! denn leicht entbrennt sein Zorn.
Wohl allen, die bei ihm sich bergen\textless sup title=``=~Zuflucht
suchen''\textgreater✲!

\hypertarget{morgenlied-eines-frommen-in-buxf6ser-zeit}{%
\subsubsection{Morgenlied eines Frommen in böser
Zeit}\label{morgenlied-eines-frommen-in-buxf6ser-zeit}}

\hypertarget{section-2}{%
\section{3}\label{section-2}}

\bibleverse{1}Ein Psalm Davids, als er vor seinem Sohne Absalom floh.
\bibleverse{2}Ach HERR, wie sind doch meine Bedränger so zahlreich,

wie viele erheben sich gegen mich!

\bibleverse{3}Gar viele sagen von mir:

»Es gibt keine Rettung\textless sup title=``oder: Hilfe''\textgreater✲
für ihn bei Gott!« SELA.

\bibleverse{4}Doch du, o HERR, bist ein Schild um mich her,

meine Ehre und der mir das Haupt erhebt.

\bibleverse{5}Laut ruf' ich zum HERRN,

und er erhört mich{A} von seinem heiligen Berge. SELA.

\bibleverse{6}Ich legte mich nieder, schlief ruhig ein:

erwacht bin ich wieder, denn der HERR stützt mich\textless sup
title=``=~hält mich aufrecht''\textgreater✲.

\bibleverse{7}Ich fürchte mich nicht vor vielen Tausenden Kriegsvolks,

die rings um mich her sich gelagert haben\textless sup title=``oder:
Aufstellung nehmen''\textgreater✲.

\bibleverse{8}Steh auf, o HERR! Hilf mir, mein Gott!

Du hast ja all meinen Feinden Backenstreiche versetzt, den Gottlosen die
Zähne zerschmettert.

\bibleverse{9}Beim HERRN steht die Hilfe\textless sup title=``oder:
Rettung''\textgreater✲:

über deinem Volke walte dein Segen! SELA.

\hypertarget{abendlied-eines-frommen-in-drangsalszeit}{%
\subsubsection{Abendlied eines Frommen in
Drangsalszeit}\label{abendlied-eines-frommen-in-drangsalszeit}}

\hypertarget{section-3}{%
\section{4}\label{section-3}}

\bibleverse{1}Dem Musikmeister, mit Saitenspiel; ein Psalm von David.
\bibleverse{2}Wenn ich rufe, erhöre mich,

du Gott meiner Gerechtigkeit\textless sup title=``oder: meines
Rechts''\textgreater✲! In Bedrängnis hast du mir (immer) Raum geschafft:
sei mir gnädig und höre mein Gebet!

\bibleverse{3}Ihr Herrensöhne, wie lange noch

soll meine Ehre geschändet werden? Wie lange noch wollt ihr an Eitlem
hangen, auf Lügen ausgehn? SELA.

\bibleverse{4}Erkennt doch, daß der HERR

den ihm Getreuen sich auserkoren: der HERR vernimmt's, wenn ich zu ihm
rufe.

\bibleverse{5}Seid zornerregt, doch versündigt euch nicht!\textless sup
title=``Eph 4,26''\textgreater✲

Denkt nach im stillen auf eurem Lager und schweigt! SELA.

\bibleverse{6}Bringt Opfer der Gerechtigkeit dar

und vertraut auf den HERRN!

\bibleverse{7}Es sagen gar viele:

»Wer läßt Gutes uns schauen\textless sup title=``=~Glück uns
erleben''\textgreater✲?« Erhebe\textless sup title=``=~laß
leuchten''\textgreater✲ über uns, o HERR, das Licht deines Angesichts!

\bibleverse{8}Du hast mir größere Freude ins Herz gegeben

als ihnen zur Zeit, wo sie Korn und Wein in Fülle haben.

\bibleverse{9}In Frieden will ich beides,

mich niederlegen und schlafen; denn du allein, HERR, läßt mich in
Sicherheit wohnen.

\hypertarget{morgengebet-im-tempel-gegen-gottlose-feinde}{%
\subsubsection{Morgengebet im Tempel gegen gottlose
Feinde}\label{morgengebet-im-tempel-gegen-gottlose-feinde}}

\hypertarget{section-4}{%
\section{5}\label{section-4}}

\bibleverse{1}Dem Musikmeister, nach (der Singweise =~Melodie) »die
Erbschaften«; ein Psalm von David. \bibleverse{2}Vernimm meine
Worte\textless sup title=``=~mein Gebet''\textgreater✲, o HERR,

merke auf mein Seufzen!

\bibleverse{3}Ach, hör' auf mein lautes Flehen,

mein König und mein Gott; denn zu dir geht mein Gebet!

\bibleverse{4}O HERR, in der Frühe schon hörst du mein Rufen,

in der Frühe schon richte ich dir (ein Opfer) zu und spähe aus (nach
dir).

\bibleverse{5}Du bist ja nicht ein Gott, dem gottlos Wesen gefällt:

kein Böser darf als Gast bei dir weilen;

\bibleverse{6}Ruhmredige dürfen dir nicht vor die Augen treten:

du hassest alle Übeltäter.

\bibleverse{7}Du läßt die Lügner zugrunde gehn;

wer mit Blutvergießen und Trug sich befaßt, den verabscheut der HERR.

\bibleverse{8}Ich aber darf nach deiner großen Gnade dein Haus betreten,

ich darf vor deinem heiligen Tempel\textless sup title=``oder: zu d.h.
T. gewandt''\textgreater✲ in Ehrfurcht vor dir mich niederwerfen✲.

\bibleverse{9}HERR, leite mich in deiner Gerechtigkeit

um meiner Feinde willen, ebne vor mir deinen Weg!

\bibleverse{10}Denn in ihrem Mund ist keine Aufrichtigkeit,

ihr Inneres\textless sup title=``oder: Sinnen''\textgreater✲ ist
Bosheit\textless sup title=``oder: sinnt Unheil''\textgreater✲; ein
offnes Grab ist ihre Kehle, mit ihrer Zunge reden sie glatte
Worte\textless sup title=``vgl. Röm 3,13''\textgreater✲.

\bibleverse{11}Laß sie büßen, o Gott, daß zu Fall sie kommen durch ihre
Anschläge!

Stoße sie weg von dir ob der Menge ihrer Frevel, denn sie haben dir
Trotz geboten!

\bibleverse{12}Dann werden alle sich freun, die auf dich vertrauen:

allzeit werden sie jubeln, daß du sie beschirmst; und frohlocken werden
alle über dich, die deinen Namen lieben.

\bibleverse{13}Denn du, HERR, segnest den Gerechten,

schirmst ihn mit (deiner) Gnade wie mit einem Schilde{A}.

\hypertarget{hilferuf-eines-an-leib-und-seele-schwerkranken-erster-buuxdfpsalm}{%
\subsubsection{Hilferuf eines an Leib und Seele Schwerkranken (Erster
Bußpsalm)}\label{hilferuf-eines-an-leib-und-seele-schwerkranken-erster-buuxdfpsalm}}

\hypertarget{section-5}{%
\section{6}\label{section-5}}

\bibleverse{1}\emph{Dem Musikmeister, mit Saitenspiel}, im Basston; ein
Psalm von David. \bibleverse{2}HERR, nicht in deinem Zorne strafe mich

und nicht in deinem Ingrimm züchtige mich!\textless sup title=``vgl.
38,2''\textgreater✲

\bibleverse{3}Sei mir gnädig, o HERR, denn ich bin am Verschmachten!

Heile mich, HERR, denn meine Gebeine sind erschrocken,{A}

\bibleverse{4}und meine Seele ist voller Angst!

Du aber, o HERR, -- wie lange noch (willst du fern sein)?

\bibleverse{5}Kehre doch wieder, o HERR, errette meine
Seele\textless sup title=``oder: mein Leben''\textgreater✲!

Hilf mir um deiner Gnade willen!

\bibleverse{6}Denn im Tode gedenkt man deiner nicht:

im Totenreich -- wer singt da dein Lob?

\bibleverse{7}Erschöpft bin ich von all meinem Seufzen;

in jeder Nacht netz' ich mein Bett (mit Zähren), mache mein Lager zu
einer Tränenflut.

\bibleverse{8}Geschwunden ist mein Augenlicht vor Gram,

gealtert (vom Weinen) ob all meinen Feinden.

\bibleverse{9}Hinweg von mir, ihr Übeltäter alle!

Denn der HERR hat mein lautes Weinen gehört;

\bibleverse{10}gehört hat der HERR mein Flehen:

der HERR nimmt mein Gebet an.

\bibleverse{11}Alle meine Feinde werden zuschanden werden

und ganz bestürzt dastehn: mit Schanden müssen sie abziehn
augenblicklich!

\hypertarget{der-herr-als-gerechter-richter-und-als-retter-der-bedruxe4ngten}{%
\subsubsection{Der Herr als gerechter Richter und als Retter der
Bedrängten}\label{der-herr-als-gerechter-richter-und-als-retter-der-bedruxe4ngten}}

\hypertarget{section-6}{%
\section{7}\label{section-6}}

\bibleverse{1}Ein Bittgebet\textless sup title=``oder: feierliches
Lied''\textgreater✲ Davids, das er dem Herrn wegen der Worte des
Benjaminiten Kusch sang\textless sup title=``oder:
dichtete''\textgreater✲. \bibleverse{2}HERR, mein Gott, bei dir such'
ich Zuflucht:

hilf mir von allen meinen Verfolgern und rette mich,

\bibleverse{3}daß der Feind mich nicht wie ein Löwe zerreiße

und zerfleische, weil kein Retter da ist!

\bibleverse{4}O HERR mein Gott! Hab' ich solches verübt,

klebt Unrecht an meinen Händen,

\bibleverse{5}hab' ich dem, der in Frieden mit mir lebte, Böses getan~--

ach nein, ich rettete ja, die mich grundlos bedrängten --:

\bibleverse{6}so möge der Feind mich verfolgen und einholen,

möge mein Leben zu Boden niedertreten und strecke meine Ehre in den
Staub!{A} SELA.

\bibleverse{7}Steh auf, o HERR, in deinem Zorn!

Erhebe dich gegen die Wut meiner Dränger! Werde wach mir zum Heil, du
hast ja Gericht verordnet!

\bibleverse{8}Laß die ganze Versammlung der Völker dich umringen,

und über ihr{A} kehre zurück zur Höhe!

\bibleverse{9}Der HERR ist Richter über die Völker:

schaffe mir Recht, o HERR, nach meiner Gerechtigkeit und nach meines
Herzens Unschuld!

\bibleverse{10}Mache der Gottlosen Bosheit ein Ende

und hilf dem Gerechten zu festem Stand, du Prüfer der Herzen und Nieren,
gerechter Gott!

\bibleverse{11}Meinen Schild hält Gott,

der Helfer der in ihrem Herzen Redlichen.

\bibleverse{12}Gott ist ein gerechter Richter

und ein Gott, der täglich droht\textless sup title=``oder:
zürnt''\textgreater✲.

\bibleverse{13}Wahrlich, wiederum schärft er sein Schwert,

hält seinen Bogen gespannt und zielt

\bibleverse{14}und richtet Todesgeschosse auf ihn,

seine Pfeile, die er zu Brandpfeilen macht.

\bibleverse{15}Seht: da brütet er\textless sup title=``d.h. der
Frevler''\textgreater✲ über Trug,

geht schwanger mit Unheil und gebiert Lüge\textless sup title=``oder:
Täuschung''\textgreater✲;

\bibleverse{16}eine Grube hat er gegraben und ausgescharrt,

stürzt selbst aber in die Grube, die er angelegt.

\bibleverse{17}Das Unheil, das er geplant, fällt ihm aufs eigne Haupt,

sein Frevel fährt auf seinen eignen Scheitel nieder.

\bibleverse{18}Preisen will ich den HERRN nach seiner Gerechtigkeit

und lobsingen dem Namen des HERRN, des Höchsten.

\hypertarget{des-menschen-niedrigkeit-und-hoheit-in-der-schuxf6pfung}{%
\subsubsection{Des Menschen Niedrigkeit und Hoheit in der
Schöpfung}\label{des-menschen-niedrigkeit-und-hoheit-in-der-schuxf6pfung}}

\hypertarget{section-7}{%
\section{8}\label{section-7}}

\bibleverse{1}Dem Musikmeister, nach der Keltertreterweise; ein Psalm
von David. \bibleverse{2}HERR, unser Herrscher, wie herrlich ist

dein Name auf der ganzen Erde, du, dessen Hoheit✲ am Himmel sich
kundtut!{A}

\bibleverse{3}Aus der Kinder und Säuglinge Mund

hast du ein Bollwerk dir zugerichtet deinen Gegnern zum Trotz, um Feinde
und Widersacher verstummen zu machen\textless sup title=``vgl. Mt
21,6''\textgreater✲.

\bibleverse{4}Wenn ich anschau' deinen Himmel, das Werk deiner Finger,

den Mond und die Sterne, die du hergerichtet:

\bibleverse{5}was ist der Mensch, daß du seiner gedenkst,

und der Menschensohn, daß du ihn beachtest?!\textless sup title=``Hebr
2,6-9''\textgreater✲

\bibleverse{6}Und doch hast du ihn nur wenig hinter die Gottheit
gestellt,

mit Herrlichkeit und Hoheit ihn gekrönt;

\bibleverse{7}du hast ihm die Herrschaft verliehn über deiner Hände
Werke,

ja alles ihm unter die Füße gelegt:

\bibleverse{8}Kleinvieh und Rinder allzumal,

dazu auch die wilden Tiere des Feldes,

\bibleverse{9}die Vögel des Himmels, die Fische im Meer,

alles, was die Pfade der Meere durchzieht.

\bibleverse{10}HERR, unser Herrscher, wie herrlich ist

dein Name auf der ganzen Erde!

\hypertarget{danklied-fuxfcr-gottes-gericht-uxfcber-heidnische-feinde-und-bitte-um-neue-hilfe}{%
\subsubsection{Danklied für Gottes Gericht über heidnische Feinde und
Bitte um neue
Hilfe}\label{danklied-fuxfcr-gottes-gericht-uxfcber-heidnische-feinde-und-bitte-um-neue-hilfe}}

\hypertarget{section-8}{%
\section{9}\label{section-8}}

\bibleverse{1}Dem Musikmeister, nach (der Singweise =~Melodie) »stirb
für den Sohn!«; ein Psalm von David. \bibleverse{2}Preisen will ich den
HERRN von ganzem Herzen,

verkünden all deine Wundertaten,

\bibleverse{3}ich will deiner mich freun und frohlocken,

will lobsingen deinem Namen, du Höchster,

\bibleverse{4}weil meine Feinde haben rückwärts weichen müssen:

sie sind gestrauchelt und umgekommen vor dir\textless sup title=``oder:
durch dich''\textgreater✲.

\bibleverse{5}Denn du hast mein Recht und meine Sache geführt,

hast auf dem Throne gesessen als gerechter Richter;

\bibleverse{6}du hast die Heiden bedroht, die Frevler vernichtet,

ihren Namen ausgelöscht für immer und ewig:

\bibleverse{7}Der Feind ist dahin, zertrümmert für immer;

auch Städte hast du zerstört, ihr Gedächtnis ist untergegangen.

\bibleverse{8}Der HERR aber thront in Ewigkeit;

zum Gericht hat er aufgestellt seinen Stuhl\textless sup title=``oder:
Thron''\textgreater✲;

\bibleverse{9}und er, er richtet den Erdkreis mit
Gerechtigkeit\textless sup title=``Apg 17,31''\textgreater✲,

spricht das Urteil den Völkern nach Gebühr.

\bibleverse{10}So ist denn der HERR eine Burg den Bedrückten,

eine Burg für die Zeiten der Drangsal.

\bibleverse{11}Drum vertrauen auf dich, die deinen Namen kennen;

denn du läßt nicht von denen, die dich, HERR, suchen.

\bibleverse{12}Lobsinget dem HERRN, der auf Zion thront,

verkündet unter den Völkern seine Taten!

\bibleverse{13}Denn als Rächer der Blutschuld hat er ihrer gedacht,

hat das Schreien der Elenden nicht vergessen.

\bibleverse{14}Sei mir gnädig, o HERR, sieh an, was ich leide durch
meine Feinde!

Du bist's, der den Pforten des Todes mich entreißt,

\bibleverse{15}auf daß ich verkünde alle deine Ruhmestaten,

in den Toren der Tochter Zion ob deiner Hilfe juble!

\bibleverse{16}Versunken sind die Heiden in die Grube, die sie gegraben,

im Netz, das sie heimlich gestellt, hat ihr eigner Fuß sich verstrickt.

\bibleverse{17}Kundgetan hat sich der HERR, hat Gericht gehalten:

durch das Eingreifen seiner Hände ist der Frevler gefangen. SAITENSPIEL.
SELA.

\bibleverse{18}Die Frevler fahren zur Unterwelt hinab,

alle Heidenvölker, die Gottes vergessen;

\bibleverse{19}denn nicht auf ewig bleibt der Arme vergessen,

und der Elenden Hoffnung geht nicht für immer verloren.

\bibleverse{20}Steh auf, o HERR! Laß Menschen nicht trotzig schalten,

laß die Heiden gerichtet werden vor dir!

\bibleverse{21}Lege doch, HERR, einen Schrecken auf sie!

Laß die Heiden erkennen, daß Menschen{A} sie sind! SELA.

\hypertarget{hilferuf-gegen-gottlose-gewaltmenschen}{%
\subsubsection{Hilferuf gegen gottlose
Gewaltmenschen}\label{hilferuf-gegen-gottlose-gewaltmenschen}}

\hypertarget{section-9}{%
\section{10}\label{section-9}}

\bibleverse{1}Warum, o HERR, stehst du so fern,

verhüllst dir (das Auge) in Zeiten der Not?

\bibleverse{2}Beim Hochmut der Gottlosen wird dem Bedrückten bange:

möchten sie selbst sich fangen in den Anschlägen, die sie ersinnen!

\bibleverse{3}Denn der Frevler rühmt sich jubelnd seiner frechen
Gelüste,

und der Wucherer gibt dem HERRN den Abschied, lästert ihn.

\bibleverse{4}Der Frevler wähnt in seinem Stolz: »Gott fragt nicht
danach!«

»Es gibt keinen Gott!« -- dahin geht all sein Denken.

\bibleverse{5}Allezeit hat er ja Glück in seinem Tun,

deine Strafgerichte bleiben himmelweit fern von ihm, alle seine Gegner
-- er bietet ihnen Hohn.{A}

\bibleverse{6}Er denkt im Herzen: »Nie komm' ich zu Fall;

nun und nimmer wird Unglück mich treffen!«

\bibleverse{7}Sein Mund ist voll Fluchens, voll Täuschung und Gewalttat;

unter seiner Zunge birgt sich Unheil{A} und Frevel.

\bibleverse{8}In (abgelegnen) Gehöften liegt er im Hinterhalt,

ermordet den Schuldlosen insgeheim\textless sup title=``oder: im
Versteck''\textgreater✲, nach dem Hilflosen spähen seine Augen.

\bibleverse{9}Er lauert im Versteck wie der Löwe in seinem Dickicht,

er lauert, den Elenden zu haschen; er hascht den Elenden, indem er ihn
in sein Netz zieht;

\bibleverse{10}er duckt sich, kauert nieder,

und die Hilflosen\textless sup title=``oder:
Unglückseligen''\textgreater✲ fallen ihm in die Klauen.

\bibleverse{11}Er denkt in seinem Herzen: »Gott hat's vergessen,

hat sein Antlitz verhüllt: er sieht es nimmer!«

\bibleverse{12}Steh auf, o HERR, erhebe, o Gott, deinen Arm,

vergiß die Elenden nicht!

\bibleverse{13}Warum darf der Frevler Gott lästern✲,

darf denken in seinem Herzen: »Du fragst nicht danach«?

\bibleverse{14}Du hast es wohl gesehn, denn auf Unheil und Herzeleid

achtest du wohl, in deine Hand es zu nehmen;{A} du bist's, dem der
Schwache es anheimstellt, der Waise bist du ein Helfer.

\bibleverse{15}Zerschmettre den Arm des Frevlers

und suche des Bösewichts gottloses Wesen heim, bis nichts mehr von ihm
zu finden!

\bibleverse{16}Der HERR ist König auf immer und ewig:

verschwinden müssen die Heiden aus seinem Lande!

\bibleverse{17}Das Verlangen der Elenden hörst du, o HERR;

du stärkst ihren Mut, leihst ihnen dein Ohr,

\bibleverse{18}um den Waisen und Bedrückten Recht zu schaffen:

nicht soll ein Mensch, der zur Erde gehört,{A} noch ferner
schrecken\textless sup title=``oder: trotzen''\textgreater✲.

\hypertarget{der-herr-ist-treu-und-gerecht}{%
\subsubsection{Der Herr ist treu und
gerecht}\label{der-herr-ist-treu-und-gerecht}}

\hypertarget{section-10}{%
\section{11}\label{section-10}}

\bibleverse{1}Dem Musikmeister, von David. Der HERR ist meine Zuflucht;

wie dürft ihr zu mir sagen: »Fliehet in euer Gebirge wie Vögel!

\bibleverse{2}Denn seht, die Gottlosen spannen den Bogen,

legen ihren Pfeil auf die Sehne, um im Dunkel zu schießen auf schuldlose
Herzen.

\bibleverse{3}Wenn die Grundpfeiler niedergerissen werden,~--

was kann da der Gerechte noch leisten?«

\bibleverse{4}Der HERR ist in seinem heiligen Palast,

der HERR, dessen Thron im Himmel steht; seine Augen halten Ausschau,
seine Blicke prüfen die Menschenkinder.

\bibleverse{5}Es prüft der HERR den Gerechten und den Gottlosen,

und wer Gewalttat liebt, den haßt seine Seele.

\bibleverse{6}Er läßt auf die Gottlosen Schlingen✲ regnen;

Feuer und Schwefel und Glutwind sind ihres Bechers Teil (das ihnen
zukommende Teil oder Los).

\bibleverse{7}Denn gerecht ist der HERR, ein Freund gerechten Tuns:

die Redlichen werden sein Angesicht schauen.

\hypertarget{trost-der-frommen-gegenuxfcber-der-gewalt-der-luxfcge}{%
\subsubsection{Trost der Frommen gegenüber der Gewalt der
Lüge}\label{trost-der-frommen-gegenuxfcber-der-gewalt-der-luxfcge}}

\hypertarget{section-11}{%
\section{12}\label{section-11}}

\bibleverse{1}Dem Musikmeister, im Basston; ein Psalm von David.
\bibleverse{2}Hilf doch, o HERR! Denn dahin sind die
Frommen\textless sup title=``d.h. Gesetzestreuen''\textgreater✲

und die Treuen ausgestorben inmitten der Menschenwelt!{A}

\bibleverse{3}Falschheit reden sie jeder mit dem andern,

mit glatten Lippen, mit doppeltem Herzen reden sie.~--

\bibleverse{4}O daß doch der HERR vertilgte alle glatten Lippen,

die Zunge, die vermessen redet,

\bibleverse{5}(die Leute) die da sagen: »Durch unsre Zunge sind wir
starke Helden,

unser Mund steht uns zur Verfügung: wer will uns meistern?«{A}

\bibleverse{6}»Ob der Knechtung der Niedrigen, ob dem Seufzen der Armen

will jetzt ich mich erheben«, spricht der HERR, »will Rettung schaffen
dem, der danach verlangt!«~--

\bibleverse{7}Die Worte des HERRN sind lautere Worte,

sind Silber, im Schmelzofen siebenfältig geläutert.

\bibleverse{8}Du, HERR, wirst treulich sie halten, wirst uns schirmen

vor diesem Geschlecht✲ zu jeder Zeit,

\bibleverse{9}vor den Gottlosen, die ringsum stolzieren,

weil Gemeinheit sich bläht inmitten der Menschheit.

\hypertarget{wie-lange-noch}{%
\subsubsection{Wie lange noch?}\label{wie-lange-noch}}

\hypertarget{section-12}{%
\section{13}\label{section-12}}

\bibleverse{1}Dem Musikmeister; ein Psalm Davids. \bibleverse{2}Wie
lange noch, HERR, willst du mich ganz vergessen,

wie lange dein Antlitz vor mir verhüllen?

\bibleverse{3}Wie lange noch soll ich Sorgen hegen in meiner Seele,

Kummer im Herzen tragen Tag für Tag? Wie lange noch soll mein Feind sich
gegen mich erheben?

\bibleverse{4}Blick her, erhöre mich, HERR, mein Gott,

laß die Augen mir wieder leuchten, daß zum Tode ich nicht entschlafe!

\bibleverse{5}Sonst rühmt sich mein Feind: »Ich habe ihn überwältigt!«,

und meine Gegner jubeln, wenn ich wanke.

\bibleverse{6}Doch nein, ich vertraue deiner Gnade:

jauchzen soll mein Herz ob deiner Hilfe! Singen will ich dem HERRN, daß
er Gutes an mir getan!

\hypertarget{gedanken-bei-der-allgemeinen-verderbtheit-der-welt-und-bitte-um-erluxf6sung}{%
\subsubsection{Gedanken bei der allgemeinen Verderbtheit der Welt und
Bitte um
Erlösung}\label{gedanken-bei-der-allgemeinen-verderbtheit-der-welt-und-bitte-um-erluxf6sung}}

\hypertarget{section-13}{%
\section{14}\label{section-13}}

\bibleverse{1}Dem Musikmeister, von David. Die Toren sprechen✲ in ihrem
Herzen:

»Es gibt keinen Gott«; verderbt, abscheulich ist ihr Tun: da ist keiner,
des Gutes täte.

\bibleverse{2}Der HERR schaut hernieder vom Himmel aus

nach den Menschenkindern, um zu sehn, ob da sei ein Verständiger, einer
der nach Gott fragt.

\bibleverse{3}Doch alle sind sie abgefallen,

insgesamt entartet; da ist keiner, der Gutes tut, auch nicht
einer\textless sup title=``Röm 3,10-12''\textgreater✲.

\bibleverse{4}Haben denn keinen Verstand die Übeltäter alle,

die mein Volk verzehren~-- die das Brot des HERRN wohl essen, doch ohne
ihn anzurufen?

\bibleverse{5}Damals gerieten sie in Angst und Schrecken,

denn Gott war mit dem gerechten Geschlecht.

\bibleverse{6}Beim Anschlag gegen den Elenden werdet zuschanden ihr
werden,

denn der HERR ist seine Zuflucht.

\bibleverse{7}O daß doch aus Zion die Rettung Israels käme!

Wenn der HERR einst wendet das Schicksal seines Volkes, wird Jakob
jubeln, Israel sich freuen.

\hypertarget{wer-darf-gast-des-herrn-sein}{%
\subsubsection{Wer darf Gast des Herrn
sein?}\label{wer-darf-gast-des-herrn-sein}}

\hypertarget{section-14}{%
\section{15}\label{section-14}}

\bibleverse{1}Ein Psalm von David. HERR, wer darf Gast sein in deinem
Zelte,

wer wohnen auf deinem heiligen Berge?

\bibleverse{2}Wer unsträflich wandelt und Gerechtigkeit übt

und die Wahrheit redet, wie's ihm ums Herz ist;

\bibleverse{3}wer keine Verleumdung mit seiner Zunge umherträgt,

seinem Nächsten kein Unrecht zufügt und keine Schmähung ausspricht gegen
Verwandte;

\bibleverse{4}wer Verworfne als wirklich verächtlich ansieht,

aber Gottesfürchtigen Ehre erweist; wer sich selbst zum Schaden schwört
und den Eid doch hält;

\bibleverse{5}wer sein Geld nicht ausleiht gegen Zins\textless sup
title=``oder: auf Wucher''\textgreater✲

und Bestechung nicht annimmt gegen Schuldlose: wer solches tut, wird
ewiglich nicht wanken.

\hypertarget{gott-das-huxf6chste-ja-einzige-gut-der-seinen}{%
\subsubsection{Gott das höchste, ja einzige Gut der
Seinen}\label{gott-das-huxf6chste-ja-einzige-gut-der-seinen}}

\hypertarget{section-15}{%
\section{16}\label{section-15}}

\bibleverse{1}Ein Lied von David. Behüte mich, Gott, denn bei dir such'
ich Zuflucht! \bibleverse{2}Ich sage zu Gott: »Mein Allherr bist du,

es gibt nichts Gutes\textless sup title=``oder: kein
Glück''\textgreater✲ für mich außer dir«;

\bibleverse{3}und von den Heiligen✲ im Lande (sag ich):

»Dies sind die Edlen, an denen mein ganzes Herz hängt.«

\bibleverse{4}Vielfaches Leid erwächst den Verehrern anderer (Götter):

ich mag ihre Bluttrankopfer nicht spenden und ihre Namen nicht auf meine
Lippen nehmen.

\bibleverse{5}Der HERR ist mein Erbgut und Becherteil\textless sup
title=``vgl. 11,6''\textgreater✲;

du bist's, der mein Los\textless sup title=``oder: Erbe''\textgreater✲
mir sichert.

\bibleverse{6}Die Meßschnur ist mir gefallen aufs
lieblichste\textless sup title=``oder: in lieblicher
Gegend''\textgreater✲

ja, mein Erbteil gefällt mir gar wohl.

\bibleverse{7}Ich preise den HERRN, der mich freundlich beraten;

auch nächtens mahnt mich mein Herz{A} dazu.

\bibleverse{8}Ich habe den HERRN mir beständig vor Augen gestellt:

steht er mir zur Rechten, so wanke ich nicht\textless sup title=``vgl.
Apg 2,25-28''\textgreater✲.

\bibleverse{9}Drum freut sich mein Herz, und meine Seele frohlockt:

auch mein Leib wird sicher wohnen\textless sup title=``=~bewahrt
sein''\textgreater✲.

\bibleverse{10}Denn du gibst meine Seele\textless sup title=``=~mein
Leben''\textgreater✲ dem Totenreich nicht preis,

du läßt deinen Frommen nicht schaun die Vernichtung{A}.

\bibleverse{11}Du weisest mir den Weg des Lebens\textless sup
title=``oder: zum Leben''\textgreater✲:

vor deinem Angesicht\textless sup title=``=~bei dir''\textgreater✲ sind
Freuden in Fülle und Segensgaben{A} in deiner Rechten ewiglich.

\hypertarget{hilferuf-eines-bedruxe4ngten-wider-ruchlose-feinde}{%
\subsubsection{Hilferuf eines Bedrängten wider ruchlose
Feinde}\label{hilferuf-eines-bedruxe4ngten-wider-ruchlose-feinde}}

\hypertarget{section-16}{%
\section{17}\label{section-16}}

\bibleverse{1}Ein Gebet Davids. Höre, o HERR, die gerechte Sache, merk'
auf mein lautes Rufen,

vernimm mein Gebet von Lippen ohne Trug!

\bibleverse{2}Von dir soll das Urteil über mich ergehn:

deine Augen sehen untrüglich.{A}

\bibleverse{3}Prüfst du mein Herz, siehst du nach mir bei Nacht,

durchforschest du mich: du findest nichts Böses; mein Mund macht sich
keines Vergehens schuldig.{A}

\bibleverse{4}Beim Treiben der Menschen hab' ich nach deiner Lippen Wort

gemieden die Pfade der Gewalttätigen.

\bibleverse{5}Meine Schritte haben sich fest an deine Bahnen gehalten,

meine Tritte haben nicht gewankt.

\bibleverse{6}Ich rufe zu dir, denn du erhörst mich, o Gott:

neige dein Ohr mir zu, vernimm meine Rede\textless sup title=``=~mein
Gebet''\textgreater✲!

\bibleverse{7}Erweise mir deine Wundergnade, du Retter derer,

die vor Widersachern Zuflucht suchen bei deiner Rechten!

\bibleverse{8}Behüte mich wie den Stern im Auge,

birg mich im Schatten deiner Flügel

\bibleverse{9}vor den Frevlern, die mir Gewalt antun,

vor meinen Feinden, die voll Gier mich umringen!

\bibleverse{10}Ihr gefühlloses Herz halten sie verschlossen,

ihr Mund stößt vermessene Reden aus.

\bibleverse{11}Auf Schritt und Tritt lauern sie jetzt uns auf,

richten ihr Trachten darauf, uns zu Boden zu werfen;

\bibleverse{12}sie gleichen dem Löwen, der gierig ist zu
rauben\textless sup title=``oder: zu zerreißen''\textgreater✲,

und dem jungen Leu, der da lauert im Versteck.

\bibleverse{13}Erhebe dich, HERR, tritt ihm entgegen, strecke ihn
nieder!

Errette mein Leben mit deinem Schwert vor dem Frevler,

\bibleverse{14}mit deiner Hand, o HERR, vor den Menschen,

vor den Leuten dieser Welt, deren Teil im Leben ist! Mit deinem
Aufgesparten\textless sup title=``=~deiner Vergeltung''\textgreater✲
fülle ihren Bauch! Mögen ihre Söhne satt daran werden und ihren Überrest
wieder ihren Kindern hinterlassen!{A}

\bibleverse{15}Doch ich in Gerechtigkeit\textless sup title=``=~ein
Gerechter''\textgreater✲ darf dein Angesicht schauen,

darf satt mich sehn beim Erwachen an deinem Bilde\textless sup
title=``oder: Anblick''\textgreater✲.

\hypertarget{davids-dank--und-siegeslied-nach-niederwerfung-seiner-feinde}{%
\subsubsection{Davids Dank- und Siegeslied nach Niederwerfung seiner
Feinde}\label{davids-dank--und-siegeslied-nach-niederwerfung-seiner-feinde}}

\hypertarget{section-17}{%
\section{18}\label{section-17}}

\bibleverse{1}Dem Musikmeister; vom Knecht des Herrn, von David, der
dieses Lied an den Herrn richtete zu der Zeit, als der Herr ihn aus der
Hand aller seiner Feinde, auch aus der Gewalt Sauls errettet hatte. Er
betete (damals) so: \bibleverse{2}Ich liebe dich, HERR, meine Stärke!
\bibleverse{3}Der HERR ist mein Fels, meine Burg und mein Erretter,

mein Gott ist mein Hort, bei dem ich Zuflucht suche, mein Schild und das
Horn meines Heils, meine Feste.

\bibleverse{4}Den Preiswürdigen rufe ich an, den HERRN:

so werd' ich von meinen Feinden errettet.

\bibleverse{5}Die Wogen des Todes hatten mich umringt,

und die Ströme des Unheils schreckten mich;

\bibleverse{6}die Netze des Totenreichs umfingen mich schon,

die Schlingen des Todes fielen über mich\textless sup title=``oder:
starrten mir entgegen''\textgreater✲.

\bibleverse{7}In meiner Angst rief ich zum HERRN

und schrie (um Hilfe) zu meinem Gott; da vernahm er in seinem Palast
mein Rufen, und mein Notschrei drang ihm zu Ohren.

\bibleverse{8}Da wankte und schwankte die Erde,

und der Berge Grundfesten bebten, sie wankten hin und her, denn er war
zornentbrannt.

\bibleverse{9}Rauch stieg auf von seiner Nase,

und fressendes Feuer drang aus seinem Munde, glühende Kohlen sprühten
von ihm aus.

\bibleverse{10}Er neigte den Himmel und fuhr herab,

Wolkennacht lag unter seinen Füßen;

\bibleverse{11}er fuhr auf dem Cherub und flog daher

und schoß herab auf den Fittichen des Sturms;

\bibleverse{12}Finsternis machte er zu seiner Hülle,

rings um sich her zu seinem Gezelt Regendunkel, dichtes Gewölk;

\bibleverse{13}aus dem Glanz vor ihm her brachen durch seine Wolken

Hagel und feurige Kohlen\textless sup title=``oder:
Feuerflammen''\textgreater✲.

\bibleverse{14}Dann donnerte der HERR im Himmel,

der Höchste ließ seine Stimme erschallen;

\bibleverse{15}er schoß seine Pfeile ab und zerstreute sie\textless sup
title=``d.h. die Feinde''\textgreater✲,

schleuderte Blitze und schreckte sie\textless sup title=``d.h. die
Feinde''\textgreater✲.

\bibleverse{16}Da wurden sichtbar die Tiefen des Meeres

und aufgedeckt die Grundfesten der Erde vor deinem Schelten, o HERR, vor
dem Zornesschnauben deiner Nase.

\bibleverse{17}Er streckte die Hand herab aus der Höhe, erfaßte mich,

zog mich heraus aus den großen Fluten,

\bibleverse{18}entriß mich meinem starken Feinde

und meinen Widersachern, die zu stark mir waren.

\bibleverse{19}Sie hatten mich überfallen an meinem Unglückstage;

doch der HERR ward mir zur Stütze;

\bibleverse{20}er führte mich heraus auf weiten Raum,

riß mich heraus, weil er Wohlgefallen an mir hatte.

\bibleverse{21}Der HERR hat mir gelohnt nach meiner Gerechtigkeit,

nach der Reinheit meiner Hände mir vergolten;

\bibleverse{22}denn ich habe eingehalten die Wege des HERRN

und bin von meinem Gott nicht treulos abgefallen;

\bibleverse{23}nein, alle seine Rechte haben mir vor Augen gestanden,

und seine Gebote hab' ich nicht von mir gewiesen.

\bibleverse{24}So bin ich unsträflich vor ihm gewandelt

und hab' mich vor jeder Verschuldung gehütet;

\bibleverse{25}drum hat mir der HERR vergolten nach meiner
Gerechtigkeit,

nach der Reinheit meiner Hände, die seinen Augen sichtbar war.

\bibleverse{26}Gegen den Guten erweist du dich gütig,

gegen den Redlichen zeigst du dich redlich,

\bibleverse{27}gegen den Reinen erweist du dich rein,

doch gegen den Falschen zeigst du dich enttäuschend;{A}

\bibleverse{28}denn du schaffst demütigen Leuten Hilfe,

aber stolze Augen erniedrigst du.

\bibleverse{29}Ja, du läßt meine Leuchte hell scheinen;

der HERR, mein Gott, macht meine Finsternis licht.

\bibleverse{30}Denn mit dir überrenne ich Feindesscharen,

und mit meinem Gott überspringe ich Mauern.

\bibleverse{31}Dieser Gott -- sein Walten ist vollkommen;

die Worte des HERRN sind lauter, ein Schild ist er allen, die zu ihm
sich flüchten.

\bibleverse{32}Denn wer ist Gott außer dem HERRN

und wer ein Fels als nur unser Gott?,

\bibleverse{33}dieser Gott, der mit Kraft mich gegürtet

und meinen Weg ohn' Anstoß gemacht;

\bibleverse{34}der mir Füße verliehen den Hirschen gleich

und mich sicher auf Bergeshöhen gestellt;

\bibleverse{35}der meine Hände streiten gelehrt,

daß meine Arme den ehernen Bogen spannten.

\bibleverse{36}Du reichtest mir deinen schützenden Schild,

deine Rechte stützte mich, und deine Gnade machte mich groß.

\bibleverse{37}Du schafftest weiten Raum meinen Schritten unter mir,

und meine Knöchel wankten nicht.

\bibleverse{38}Ich verfolgte meine Feinde, holte sie ein

und kehrte nicht um, bis ich sie vernichtet;

\bibleverse{39}ich zerschmetterte sie, daß sie nicht wieder aufstehn
konnten:

sie sanken unter meine Füße nieder.

\bibleverse{40}Und du gürtetest mich mit Kraft zum Streit,

beugtest unter mich alle, die sich gegen mich erhoben;

\bibleverse{41}du triebst meine Feinde vor mir in die Flucht,

und alle, die mich haßten, vernichtete ich:

\bibleverse{42}sie schrien um Hilfe -- doch da war kein Helfer~--

zum HERRN -- doch er hörte sie nicht;

\bibleverse{43}ich zermalmte sie wie Staub vor dem Winde,

wie Kot auf den Gassen schüttete ich sie hin.

\bibleverse{44}Du hast mich aus den Kämpfen für (mein) Volk errettet,

mich zum Oberhaupt von Völkern\textless sup title=``oder: der
Heiden''\textgreater✲ eingesetzt: Völker, die ich nicht kannte, dienen
mir;

\bibleverse{45}aufs bloße Wort gehorchen sie mir,

die Söhne des Auslands huldigen mir;

\bibleverse{46}die Söhne des Auslands sinken mutlos hin

und kommen zitternd hervor aus ihren Schlössern.

\bibleverse{47}Der HERR lebt: gepriesen sei mein Hort!

und erhaben ist der Gott meines Heils,

\bibleverse{48}der Gott, der mir Rache verliehen

und die Völker unter meine Herrschaft gezwungen,

\bibleverse{49}der von meinen grimmen Feinden mich gerettet

und über meine Widersacher mich erhöht, von dem Mann der Gewalttat mich
befreit hat!

\bibleverse{50}Drum will ich dich preisen, HERR, unter den Völkern

und deinem Namen lobsingen\textless sup title=``vgl. Röm
15,9''\textgreater✲,

\bibleverse{51}dir, der seinem Könige großes Heil verleiht

und Gnade an seinem Gesalbten übt, an David und seinem Hause ewiglich!

\hypertarget{lobpreis-gottes-des-schuxf6pfers-und-seines-gesetzes-bitte-um-suxfcndenvergebung-und-um-heiligung}{%
\subsubsection{Lobpreis Gottes des Schöpfers und seines Gesetzes; Bitte
um Sündenvergebung und um
Heiligung}\label{lobpreis-gottes-des-schuxf6pfers-und-seines-gesetzes-bitte-um-suxfcndenvergebung-und-um-heiligung}}

\hypertarget{section-18}{%
\section{19}\label{section-18}}

\bibleverse{1}Dem Musikmeister; ein Psalm von David. \bibleverse{2}Die
Himmel verkünden Gottes Herrlichkeit\textless sup title=``oder:
Ehre''\textgreater✲,

und vom Werk seiner Hände erzählt die Feste{A}.

\bibleverse{3}Ein Tag ruft dem andern die Botschaft zu,

und eine Nacht vermeldet der andern die Kunde.

\bibleverse{4}Da ist keine Sprache, da sind keine Worte,

unhörbar ist ihre Stimme;

\bibleverse{5}und doch: durch alle Lande dringt ihr Schall

und ihre Rede bis ans Ende des Erdkreises\textless sup title=``vgl. Röm
10,18''\textgreater✲; der Sonne hat er dort ein Zelt gesetzt.

\bibleverse{6}Und sie -- wie ein Bräutigam tritt sie hervor aus ihrem
Gemach,

sie freut sich wie ein Held, zu durchlaufen die Bahn.

\bibleverse{7}Vom Ende des Himmels geht sie aus,

und ihr Umlauf reicht wieder bis zu dessen Ende, und nichts bleibt
verborgen vor ihrer Lichtglut.~--

\bibleverse{8}Das Gesetz des HERRN ist vollkommen\textless sup
title=``oder: ohne Fehl''\textgreater✲:

erquickt die Seele; das Zeugnis des HERRN ist zuverlässig: macht die
Törichten weise;

\bibleverse{9}die Befehle des HERRN sind richtig:

erfreuen das Herz; das Gebot des HERRN ist lauter:
erleuchtet\textless sup title=``oder: läßt leuchten''\textgreater✲ die
Augen;

\bibleverse{10}die Furcht✲ vor dem HERRN ist rein:

bleibt ewig bestehn; die Gerichtsurteile des HERRN sind Wahrheit: sind
allzumal gerecht;

\bibleverse{11}sie sind köstlicher als Gold

und als Feingold in Menge, sind süßer als Honig und Wabenseim.

\bibleverse{12}Auch dein Knecht läßt durch sie sich warnen:

in ihrer Befolgung liegt ein reicher Lohn.

\bibleverse{13}Verfehlungen -- ach, wer nimmt sie wahr?

Von den unbewußten Fehlern sprich mich los!

\bibleverse{14}Auch vor Hochmutssünden behüte deinen Knecht:

laß sie nicht Macht über mich gewinnen! Dann steh' ich unsträflich da
und bleibe rein von schwerer Verschuldung.

\bibleverse{15}Laß wohlgefällig dir sein die Worte meines Mundes

und das Sinnen meines Herzens, o HERR, mein Fels und mein Erlöser!

\hypertarget{fuxfcrbitte-des-volkes-fuxfcr-den-kuxf6nig-beim-auszug-des-heeres}{%
\subsubsection{Fürbitte des Volkes für den König beim Auszug des
Heeres}\label{fuxfcrbitte-des-volkes-fuxfcr-den-kuxf6nig-beim-auszug-des-heeres}}

\hypertarget{section-19}{%
\section{20}\label{section-19}}

\bibleverse{1}Dem Musikmeister; ein Psalm von David. \bibleverse{2}Der
HERR erhöre dich am Tage der Drangsal,

es schütze dich der Name des Gottes Jakobs!

\bibleverse{3}Er sende dir Hilfe vom Heiligtum her

und leiste dir Beistand von Zion aus!

\bibleverse{4}Er gedenke aller deiner Speisopfer\textless sup
title=``oder: Gaben''\textgreater✲

und sehe dein Brandopfer wohlgefällig an! SELA.

\bibleverse{5}Er gewähre dir, was dein Herz begehrt,

und lasse all deine Pläne gelingen!

\bibleverse{6}Dann wollen wir jubeln ob deinem Heil

und im Namen unsers Gottes die Fahnen entfalten\textless sup
title=``oder: schwingen''\textgreater✲: der HERR erfülle dir all deine
Wünsche!~--

\bibleverse{7}Jetzt weiß ich, der HERR hilft seinem Gesalbten:

er erhört ihn aus seinem heiligen Himmel durch die hilfreichen Taten
seiner Rechten.

\bibleverse{8}Diese sind stark durch Wagen und jene durch Rosse,

doch wir sind stark durch den Namen des HERRN, unsers Gottes.{A}

\bibleverse{9}Sie stürzen nieder und fallen,

doch wir stehn fest und halten uns aufrecht.

\bibleverse{10}O HERR, hilf\textless sup title=``=~verleihe den
Sieg''\textgreater✲ dem König!

Erhör' uns, sooft wir (dich) anrufen!{A}

\hypertarget{dankgebet-fuxfcr-die-dem-kuxf6nig-von-gott-erwiesenen-wohltaten-besonders-fuxfcr-den-ihm-verliehenen-sieg-und-hoffnung-neuer-segnungen}{%
\subsubsection{Dankgebet für die dem König von Gott erwiesenen Wohltaten
(besonders für den ihm verliehenen Sieg) und Hoffnung neuer
Segnungen}\label{dankgebet-fuxfcr-die-dem-kuxf6nig-von-gott-erwiesenen-wohltaten-besonders-fuxfcr-den-ihm-verliehenen-sieg-und-hoffnung-neuer-segnungen}}

\hypertarget{section-20}{%
\section{21}\label{section-20}}

\bibleverse{1}Dem Musikmeister; ein Psalm von David. \bibleverse{2}O
HERR, ob deiner Kraft freut sich der König,

und ob deiner Hilfe{A} -- wie jauchzt er so laut!

\bibleverse{3}Seines Herzens Verlangen hast du ihm erfüllt

und den Wunsch seiner Lippen ihm nicht versagt; SELA.

\bibleverse{4}denn mit Glück und Segen bist du ihm begegnet,

hast aufs Haupt ihm gesetzt eine Krone von Feingold.

\bibleverse{5}Leben erbat er von dir: du hast's ihm gewährt,

der Jahre Fülle auf endlose Zeit.

\bibleverse{6}Groß ist sein Ruhm durch deine Hilfe,

mit Glanz und Hoheit hast du ihn geschmückt;

\bibleverse{7}für die Dauer hast du ihn zum Segen gemacht,

ihn beglückt mit Freude vor deinem Angesicht.

\bibleverse{8}Denn der König vertraut auf den HERRN

und wird durch des Höchsten Gnade nicht wanken.

\bibleverse{9}Deine Hand wird treffen alle deine Feinde,

deine Rechte alle erreichen, die dich hassen.

\bibleverse{10}Du wirst sie wie einen Feuerofen machen,

sobald du erscheinst; der HERR wird sie verschlingen in seinem Zorn, und
Feuer wird sie verzehren.

\bibleverse{11}Ihren Nachwuchs\textless sup title=``=~ihre
Kinder''\textgreater✲ wirst du vom Erdboden tilgen

und ihr Geschlecht aus der Menschenwelt.

\bibleverse{12}Wenn Böses sie gegen dich planen, auf Arglist sinnen:

sie werden nichts vermögen;

\bibleverse{13}denn du wirst sie zwingen, die Flucht zu ergreifen,

mit deinem Bogen auf ihr Antlitz zielen.

\bibleverse{14}Erhebe dich, HERR, in deiner Kraft:

wir wollen dein Heldentum besingen und preisen.

\hypertarget{klage-und-hoffnung-eines-von-gott-verlassenen-des-heilands-leidenspsalm}{%
\subsubsection{Klage und Hoffnung eines von Gott Verlassenen (Des
Heilands
Leidenspsalm)}\label{klage-und-hoffnung-eines-von-gott-verlassenen-des-heilands-leidenspsalm}}

\hypertarget{section-21}{%
\section{22}\label{section-21}}

\bibleverse{1}Dem Musikmeister, nach (der Singweise =~Melodie)
»Hirschkuh der Morgenröte«; ein Psalm von David. \bibleverse{2}Mein
Gott, mein Gott, warum hast du mich verlassen?\textless sup title=``Mt
27,46; Mk 15,34''\textgreater✲

Ach, fern von meiner Rettung bleiben die Worte meiner Klage!

\bibleverse{3}Mein Gott! Ich rufe bei Tage, doch du antwortest nicht,

und bei Nacht, doch Ruhe wird mir nicht zuteil!

\bibleverse{4}Und doch bist du der Heilige,

der da thront über Israels Lobgesängen.

\bibleverse{5}Auf dich haben unsre Väter vertraut,

sie haben vertraut, und du hast ihnen ausgeholfen;

\bibleverse{6}zu dir haben sie geschrien und Rettung gefunden,

auf dich haben sie vertraut und sind nicht enttäuscht worden.

\bibleverse{7}Doch ich bin ein Wurm und kein Mensch mehr,

bin der Leute Hohn und verachtet vom Volk;

\bibleverse{8}alle, die mich sehen, spotten mein\textless sup
title=``vgl. Mt 27,39-43''\textgreater✲,

reißen den Mund auf, schütteln den Kopf:

\bibleverse{9}»Er werf's auf den HERRN: der möge ihn befreien,

der möge ihn retten: er hat ja Wohlgefallen an ihm!«

\bibleverse{10}Ja du bist's, der mich der Mutter gelegt in den Schoß,

mich sicher geborgen an meiner Mutter Brust;

\bibleverse{11}von Geburt an bin ich auf dich geworfen✲,

vom Schoß meiner Mutter her bist du mein Gott.

\bibleverse{12}O bleibe nicht fern von mir, denn die Drangsal ist nahe,

und sonst ist kein Helfer zu sehen!

\bibleverse{13}Mich umzingeln mächtige Stiere,

Basans Riesenfarren halten mich umringt;

\bibleverse{14}den Rachen sperren sie gegen mich auf~--

ein reißender, brüllender Löwe!

\bibleverse{15}Wie Wasser bin ich ausgegossen,

alle meine Glieder sind ausgerenkt\textless sup title=``oder:
zerschlagen''\textgreater✲; das Herz ist mir geworden wie Wachs,
zerschmolzen in meinem Innern.

\bibleverse{16}Vertrocknet wie eine Scherbe ist meine Kraft,

und die Zunge klebt mir am Gaumen: in den Staub des Todes hast du mich
gelegt.

\bibleverse{17}Ach, Hunde umgeben mich rings,

eine Rotte von Übeltätern umkreist mich; sie haben mir Hände und Füße
durchbohrt.

\bibleverse{18}Alle meine Gebeine kann ich zählen:

sie aber blicken mich an und weiden sich an dem Anblick.

\bibleverse{19}Sie teilen meine Kleider unter sich

und werfen das Los um mein Gewand\textless sup title=``Mt 27,35; Joh
19,24''\textgreater✲.

\bibleverse{20}Doch du, HERR, bleibe nicht fern von mir,

du, meine Stärke, eile mir zu Hilfe!

\bibleverse{21}Errette vor dem Schwert mein Leben,

mein einziges Gut aus der Hunde Gewalt!{A}

\bibleverse{22}Hilf mir aus dem Rachen des Löwen

und bewahre mich vor den Hörnern der Büffel!

\bibleverse{23}Dann will ich deinen Namen meinen Brüdern kundtun,

inmitten der Gemeinde dich rühmen✲:

\bibleverse{24}»Die den HERRN ihr fürchtet, preiset ihn!

Ihr alle vom Hause Jakobs, ehret ihn und scheut euch vor ihm, ihr alle
von Israels Stamm!

\bibleverse{25}Denn er hat nicht übersehen und nicht verabscheut das
Elend des Dulders

und sein Antlitz vor ihm nicht verborgen, nein, als er zu ihm schrie,
auf ihn gehört.«

\bibleverse{26}Dir soll mein Loblied gelten in großer Gemeinde;

meine Gelübde will ich erfüllen vor denen, die ihn fürchten.

\bibleverse{27}Die Elenden sollen essen, daß sie satt werden,

und die da suchen den HERRN, sollen ihn preisen: aufleben soll euer Herz
für immer!

\bibleverse{28}Daran werden gedenken und zum HERRN sich bekehren

alle Enden der Erde, und vor dir werden sich niederwerfen alle
Geschlechter der Heiden;

\bibleverse{29}denn dem HERRN gehört die Herrschaft\textless sup
title=``oder: das Königtum''\textgreater✲,

und er ist der Völkergebieter.

\bibleverse{30}Vor ihm werden niederfallen alle Großen der Erde,

vor ihm die Knie beugen alle, die in den Erdstaub sinken und wer seine
Seele nicht am Leben erhalten kann.

\bibleverse{31}Die Nachwelt wird ihm dienen;

vom Allherrn wird man erzählen dem künft'gen Geschlecht.

\bibleverse{32}Sie werden kommen und seine Gerechtigkeit kundtun

dem nachgeborenen Volk, daß Er es vollführt hat.{A}

\hypertarget{der-herr-als-der-gute-hirt-und-der-freundliche-spender-von-trost-und-sicherheit}{%
\subsubsection{Der Herr als der gute Hirt und der freundliche Spender
von Trost und
Sicherheit}\label{der-herr-als-der-gute-hirt-und-der-freundliche-spender-von-trost-und-sicherheit}}

\hypertarget{section-22}{%
\section{23}\label{section-22}}

\bibleverse{1}Ein Psalm von David. Der HERR ist mein Hirt: mir mangelt
nichts. \bibleverse{2}Auf grünen Auen läßt er mich lagern,

zum Lagerplatz am Bache führt er mich.{A}

\bibleverse{3}Er erquickt meine Seele;

er leitet mich auf rechten Pfaden um seines Namens willen.

\bibleverse{4}Müßt' ich auch wandern in finsterm Tal:

ich fürchte kein Unglück, denn du bist bei mir: dein Hirtenstab und dein
Stecken, die sind mein Trost.

\bibleverse{5}Du deckst mir reichlich den Tisch

vor den Augen meiner Feinde; du salbst mir das Haupt mit Öl und schenkst
mir den Becher voll ein.

\bibleverse{6}Nur Gutes und Liebes\textless sup title=``oder: Glück und
Gnade''\textgreater✲ werden mich begleiten

mein ganzes Leben hindurch, und heimkehren werd' ich zum Hause des HERRN
für eine lange Reihe von Tagen.{B}

\hypertarget{festlied-beim-einzug-des-volkes-und-des-kuxf6nigs-der-ehren-in-das-heiligtum}{%
\subsubsection{Festlied beim Einzug (des Volkes und des Königs der
Ehren) in das
Heiligtum}\label{festlied-beim-einzug-des-volkes-und-des-kuxf6nigs-der-ehren-in-das-heiligtum}}

\hypertarget{section-23}{%
\section{24}\label{section-23}}

\bibleverse{1}Von David, ein Psalm. Dem HERRN gehört die Erde und ihre
Fülle,

der Erdkreis und seine Bewohner;

\bibleverse{2}denn er hat auf Meeren\textless sup title=``=~dem
Weltmeer''\textgreater✲ sie gegründet

und über Strömen sie festgestellt.

\bibleverse{3}Wer darf hinaufgehn zum Berge des HERRN,

wer stehen an seiner heiligen Stätte?

\bibleverse{4}Wer schuldlos ist an Händen und reinen Herzens,

wer nie den Sinn auf Täuschung richtet, und wer nicht betrügerisch
schwört:

\bibleverse{5}der wird Segen empfangen vom HERRN

und Gerechtigkeit\textless sup title=``oder:
Gerechtsprechung''\textgreater✲ vom Gott seines Heils.

\bibleverse{6}Dies ist das Geschlecht, das nach ihm verlangt,

die dein Angesicht suchen, Gott Jakobs.{A} SELA.

\bibleverse{7}Hebt hoch, ihr Tore, eure Häupter

und öffnet euch weit,{A} ihr uralten Pforten, daß der König der
Herrlichkeit\textless sup title=``oder: Ehren''\textgreater✲ einziehe!

\bibleverse{8}»Wer ist denn der König der Herrlichkeit\textless sup
title=``oder: Ehren''\textgreater✲?«

Der HERR, gar stark und ein Held, der HERR, ein Held in der Schlacht!

\bibleverse{9}Hebt hoch, ihr Tore, eure Häupter

und öffnet euch weit,{A} ihr uralten Pforten, daß der König der
Herrlichkeit\textless sup title=``oder: Ehren''\textgreater✲ einziehe!

\bibleverse{10}»Wer ist denn der König der Herrlichkeit\textless sup
title=``oder: Ehren''\textgreater✲?«

Der HERR der Heerscharen, der ist der König der
Herrlichkeit\textless sup title=``oder: Ehren''\textgreater✲! SELA.

\hypertarget{gebet-um-gottes-schutz-um-gnuxe4dige-leitung-und-vergebung-der-suxfcnden}{%
\subsubsection{Gebet um Gottes Schutz, um gnädige Leitung und Vergebung
der
Sünden}\label{gebet-um-gottes-schutz-um-gnuxe4dige-leitung-und-vergebung-der-suxfcnden}}

\hypertarget{section-24}{%
\section{25}\label{section-24}}

\bibleverse{1}Von David. Zu dir, o HERR, erheb' ich meine Seele,
\bibleverse{2}mein Gott, auf dich vertraue ich:

laß mich nicht enttäuscht werden, laß meine Feinde nicht über mich
frohlocken!

\bibleverse{3}Nein, keiner, der auf dich harrt, wird enttäuscht;

enttäuscht wird nur, wer dich treulos verläßt.~--

\bibleverse{4}Tu mir kund, o HERR, deine Wege,

deine Pfade lehre mich!

\bibleverse{5}Laß mich wandeln in deiner Wahrheit\textless sup
title=``vgl. 26,3''\textgreater✲ und lehre mich,

denn du bist der Gott meines Heils: deiner harre ich allezeit.~--

\bibleverse{6}Gedenke der Erweise deines Erbarmens, o HERR,

und daß deine Gnadenverheiße aus der Urzeit stammen;

\bibleverse{7}gedenke nicht der Sünden meiner Jugend und meiner
Vergehen:

nein, nach deiner Gnade gedenke meiner um deiner Güte willen!

\bibleverse{8}Gütig und aufrichtig ist der HERR;

darum weist er den Sündern den rechten Weg,

\bibleverse{9}läßt Bedrückte wandeln in richtiger Weise

und lehrt die Dulder seinen Weg.

\bibleverse{10}Alle Pfade des HERRN sind Gnade und Treue

denen, die seinen Bund und seine Gebote halten.

\bibleverse{11}Um deines Namens willen, o HERR,

vergib mir meine Schuld, denn sie ist groß!~--

\bibleverse{12}Wie steht's mit dem Mann, der den HERRN fürchtet?

Dem zeigt er den Weg, den er wählen soll.

\bibleverse{13}Er selbst wird wohnen im Glück,

und seine Kinder werden das Land besitzen.

\bibleverse{14}Freundschaft hält der HERR mit denen, die ihn fürchten,

und sein Bund will zur Erkenntnis sie führen.~--

\bibleverse{15}Meine Augen sind stets auf den HERRN gerichtet,

denn er wird meine Füße aus dem Netze ziehn.

\bibleverse{16}Wende dich mir zu und sei mir gnädig!

Denn einsam bin ich und elend.

\bibleverse{17}Die Ängste meines Herzens sind schwer geworden:

o führ' mich heraus aus meinen Nöten!

\bibleverse{18}Sieh mein Elend an und mein Ungemach

und vergib mir alle meine Sünden!~--

\bibleverse{19}Sieh meine Feinde an, wie viele ihrer sind

und wie sie mich hassen mit frevlem Haß.

\bibleverse{20}Behüte meine Seele und rette mich,

nicht enttäuscht laß mich werden: ich traue auf dich!

\bibleverse{21}Unschuld und Redlichkeit mögen mich behüten,

denn ich harre deiner, o HERR!~--

\bibleverse{22}O Gott, erlöse Israel aus allen seinen Nöten!

\hypertarget{hilferuf-eines-seiner-unschuld-sich-bewuuxdften-frommen}{%
\subsubsection{Hilferuf eines seiner Unschuld sich bewußten
Frommen}\label{hilferuf-eines-seiner-unschuld-sich-bewuuxdften-frommen}}

\hypertarget{section-25}{%
\section{26}\label{section-25}}

\bibleverse{1}Von David. Schaffe mir Recht, o HERR,

denn ich bin gewandelt in meiner Unschuld und habe vertraut auf den
HERRN ohne Wanken!

\bibleverse{2}Prüfe mich, HERR, und erprobe mich:

meine Nieren und mein Herz sind geläutert!{A}

\bibleverse{3}Denn deine Gnade steht mir vor Augen,

und ich wandle in deiner Wahrheit\textless sup title=``oder: in der
Treue gegen dich''\textgreater✲.

\bibleverse{4}Ich sitze nicht bei falschen Menschen

und verkehre nicht mit hinterlistigen Leuten;

\bibleverse{5}ich meide die Versammlung der Missetäter

und halte mich nicht zu den Gottlosen;

\bibleverse{6}ich wasche in Unschuld meine Hände

und schreite so um deinen Altar, o HERR,

\bibleverse{7}daß ich laut ein Danklied erschallen lasse

und alle deine Wundertaten verkünde.

\bibleverse{8}O HERR, ich habe lieb die Stätte deines Hauses

und den Ort, wo deine Herrlichkeit wohnt.

\bibleverse{9}Raffe nicht weg meine Seele mit den (Seelen der) Sünder,

noch mein Leben mit dem der Mordgesellen,

\bibleverse{10}an deren Händen Verbrechen kleben

und deren Rechte gefüllt ist mit Bestechung!

\bibleverse{11}Ich aber wandle in meiner Unschuld:

erlöse mich, HERR, und sei mir gnädig!

\bibleverse{12}Mein Fuß steht fest auf ebenem Plan\textless sup
title=``oder: auf ebener Bahn''\textgreater✲:

in Versammlungen will ich preisen den HERRN.

\hypertarget{freudige-zuversicht-auf-den-herrn-und-bitte-um-seinen-weiteren-schutz}{%
\subsubsection{Freudige Zuversicht auf den Herrn und Bitte um seinen
weiteren
Schutz}\label{freudige-zuversicht-auf-den-herrn-und-bitte-um-seinen-weiteren-schutz}}

\hypertarget{section-26}{%
\section{27}\label{section-26}}

\bibleverse{1}Von David. Der HERR ist mein Licht und mein Heil:

vor wem sollt' ich mich fürchten? Der HERR ist meines Lebens Schutzwehr:
vor wem sollte mir bangen?

\bibleverse{2}Wenn Übeltäter gegen mich anstürmen,

mich zu zerfleischen\textless sup title=``oder: zu
verschlingen''\textgreater✲, meine Widersacher und Feinde: sie
straucheln und fallen.

\bibleverse{3}Mag ein Heer sich gegen mich lagern:

mein Herz ist ohne Furcht; mag Krieg sich gegen mich erheben: trotzdem
bleib' ich getrost.

\bibleverse{4}Nur eines erbitt' ich vom HERRN,

danach trag' ich Verlangen: daß ich weilen möge im Hause des HERRN mein
ganzes Leben hindurch, um anzuschauen die Huld des HERRN und der Andacht
mich hinzugeben in seinem Tempel.

\bibleverse{5}Denn er birgt mich in seiner Hütte

am Tage des Unheils, beschirmt mich im Schirm seines Zeltes, hebt hoch
mich auf einen Felsen empor.

\bibleverse{6}So wird sich denn mein Haupt erheben

über meine Feinde rings um mich her; und opfern will ich in seinem Zelte
Schlachtopfer mit Jubelschall, will singen und spielen dem HERRN!

\bibleverse{7}Höre mich, HERR, laut ruf' ich zu dir!

Ach sei mir gnädig, erhöre mich!

\bibleverse{8}Mein Herz hält dein Gebot dir vor:

»Ihr sollt mein Angesicht suchen!« Darum suche ich, o HERR, dein
Angesicht.

\bibleverse{9}Verbirg dein Angesicht nicht vor mir,

weise nicht ab deinen Knecht im Zorn! Du bist meine Hilfe gewesen:
verwirf mich nicht und verlaß mich nicht, du Gott meines Heils!

\bibleverse{10}Wenn Vater und Mutter mich verlassen,

so nimmt doch der HERR mich auf.

\bibleverse{11}Lehre mich, HERR, deinen Weg

und führe mich auf ebener Bahn um meiner Feinde willen!

\bibleverse{12}Gib mich nicht preis der Gier meiner Dränger!

Denn Lügenzeugen sind gegen mich aufgetreten und schnauben Gewalttat
(gegen mich).

\bibleverse{13}Gott Lob! Ich bin gewiß, die Güte des HERRN

zu schauen im Lande der Lebenden.

\bibleverse{14}Harre des HERRN, sei getrost,

und dein Herz sei unverzagt! Ja, harre des HERRN!

\hypertarget{gebet-und-hilfe-gegen-gottlose-feinde-und-dank-fuxfcr-die-erhuxf6rung}{%
\subsubsection{Gebet und Hilfe gegen gottlose Feinde und Dank für die
Erhörung}\label{gebet-und-hilfe-gegen-gottlose-feinde-und-dank-fuxfcr-die-erhuxf6rung}}

\hypertarget{section-27}{%
\section{28}\label{section-27}}

\bibleverse{1}Von David. Zu dir, HERR, rufe ich:

mein Fels, o wende dich nicht schweigend von mir ab, auf daß nicht, wenn
du mir stumm bleibst, ich den ins Grab\textless sup title=``oder: in die
Unterwelt''\textgreater✲ Gesunknen gleiche.

\bibleverse{2}Höre mein lautes Flehen, wenn ich zu dir schreie,

wenn ich meine Hände erhebe nach deinem Allerheiligsten!

\bibleverse{3}Raffe mich nicht weg mit den Frevlern

und den Übeltätern, die freundlich reden mit ihren Nächsten und dabei
Arges im Herzen sinnen!

\bibleverse{4}Vergilt du ihnen nach ihrem Tun,

nach der Bosheit ihrer Handlungen, vergilt ihnen nach dem Werk ihrer
Hände, zahl' ihnen ihr Verhalten heim, wie sie's verdienen!

\bibleverse{5}Denn sie achten nicht auf das Walten des HERRN und das
Werk seiner Hände;

drum wird er sie niederreißen und nicht wieder aufbaun.

\bibleverse{6}Gepriesen sei der HERR, denn er hat gehört

meinen lauten Hilferuf!

7Der HERR ist meine Stärke und mein Schild;

auf ihn hat mein Herz vertraut, da ist mir Hilfe geworden. So frohlockt
denn mein Herz, und mit meinem Liede will ich ihm danken.

8Der HERR ist seines Volkes Stärke

und seines Gesalbten rettende Zuflucht.

9O hilf deinem Volk und segne dein Erbe,

weide sie und trage sie ewiglich!

\hypertarget{gottes-herrlichkeit-im-gewitter}{%
\subsubsection{Gottes Herrlichkeit im
Gewitter}\label{gottes-herrlichkeit-im-gewitter}}

\hypertarget{section-28}{%
\section{29}\label{section-28}}

1Ein Psalm von David. Bringt dar dem HERRN, ihr Gottessöhne\textless sup
title=``d.h. Engel''\textgreater✲,

bringt dar dem HERRN Ehre und Preis!{B}

2Bringt dar dem HERRN die Ehre seines Namens,

werft vor dem HERRN euch nieder in heiligem Schmuck!

3Der Donner des HERRN rollt über dem Meer;

der Gott der Herrlichkeit donnert, der HERR über weiter Meeresflut!

4Der Donner des HERRN erschallt mit Macht,

der Donner des HERRN in seiner Pracht!

5Der Donner des HERRN zerschmettert die Zedern,

ja der HERR zersplittert die Zedern des Libanons

6und läßt sie hüpfen wie Kälbchen,

den Libanon und Sirjon\textless sup title=``5.Mose 3,9''\textgreater✲
wie junge Büffel.

7Der Donner des HERRN läßt Feuerflammen sprühn; 8der Donner des HERRN
macht die Wüste erbeben,

der HERR macht erbeben die Wüste Kades\textless sup title=``5.Mose
1,19''\textgreater✲.

9Der Donner des HERRN macht Hirschkühe kreißen,

entästet die Wälder, und alles ruft in seinem Palast: »Ehre!«

10Der HERR hat über der Sintflut (einst) gethront,

und als König thront der HERR in Ewigkeit.

11Der HERR verleihe Kraft seinem Volk,

der HERR wolle segnen sein Volk mit Frieden\textless sup title=``oder:
Heil''\textgreater✲!

\hypertarget{danklied-eines-aus-todesnot-geretteten}{%
\subsubsection{Danklied eines aus Todesnot
Geretteten}\label{danklied-eines-aus-todesnot-geretteten}}

\hypertarget{section-29}{%
\section{30}\label{section-29}}

1\emph{Ein Psalm, ein Lied zur Tempelweihe}, von David. 2Ich will dich
erheben, o HERR, denn du hast aus der Tiefe mich gezogen

und meinen Feinden die Freude über mich vereitelt.

3O HERR, mein Gott, ich schrie zu dir (um Hilfe),

da hast du mir Heilung geschafft.

4O HERR, du hast meine Seele aus dem Totenreich heraufgeführt,

hast mich am Leben erhalten, so daß ich nicht ins Grab bin gesunken.{A}

5Lobsinget dem HERRN, ihr seine Frommen,

und preist seinen heiligen Namen!

6Denn sein Zorn währt nur einen Augenblick,

doch lebenslang seine Gnade: am Abend kehrt Weinen als Gast ein, doch am
Morgen herrscht Jubel.

7Ich aber dachte in meiner Sicherheit:

»Ich werde nimmermehr wanken!«

8O HERR, nach deiner Gnade

hattest du fest meinen Berg gegründet; dann aber verbargst du dein
Antlitz, und ich erschrak.

9Da rief ich zu dir, o HERR,

und flehte zu meinem Gott:

10»Was hast du für Gewinn von meinem Blut,

wenn zur Gruft\textless sup title=``oder: zur Unterwelt''\textgreater✲
ich fahre? Kann der Staub dich preisen und deine Treue verkünden?

11O höre mich, HERR, und erbarme dich mein,

sei du, o HERR, ein Helfer!«

12Du hast mir meine Klage in Reigentanz verwandelt,

das Trauerkleid mir gelöst und mit Freude mich gegürtet,

13auf daß dir meine Seele lobsinge und nicht schweige:

o HERR, mein Gott, in Ewigkeit will ich dir danken\textless sup
title=``oder: dich preisen''\textgreater✲!

\hypertarget{zuversichtliches-gebet-in-schwerer-not}{%
\subsubsection{Zuversichtliches Gebet in schwerer
Not}\label{zuversichtliches-gebet-in-schwerer-not}}

\hypertarget{section-30}{%
\section{31}\label{section-30}}

1Dem Musikmeister; ein Psalm von David. 2Bei dir, HERR, suche ich
Zuflucht:

laß mich nimmer enttäuscht werden! Nach deiner Gerechtigkeit errette
mich!

3Neige dein Ohr mir zu,

eile zu meiner Rettung herbei, sei mir ein schützender Fels, eine feste
Burg, mir zu helfen!

4Du bist ja doch mein Fels und meine Burg,

um deines Namens willen wirst du mich führen und leiten,

5mich befrein aus dem Netz, das man heimlich mir gestellt;

denn du bist meine Schutzwehr.

6In deine Hand befehl' ich meinen Geist\textless sup title=``Lk
23,46''\textgreater✲:

du wirst mich erlösen, o HERR, du treuer Gott.

7Du hassest, die sich an nichtige Götzen halten,

doch ich vertraue auf den HERRN.

8Ich will jubeln und fröhlich sein ob deiner Gnade,

daß du mein Elend hast angeschaut, auf die Angst meiner Seele geachtet

9und mich der Gewalt des Feindes nicht preisgegeben,

nein, meine Füße gestellt hast auf weiten Raum.

10Sei mir gnädig, o HERR, denn ich bin in Bedrängnis\textless sup
title=``oder: Angst''\textgreater✲;

getrübt vor Gram ist mir mein Auge, meine Seele und mein Leib;

11denn in Kummer verzehrt sich mein Leben

und meine Jahre in Seufzen; erschöpft durch mein Verschulden ist meine
Kraft, und verfallen sind meine Gebeine.

12Für alle meine Feinde bin ich zum Hohn geworden,

von meinen Nachbarn gemieden und ein Schrecken für meine Bekannten: wer
mich sieht auf der Straße, flieht scheu vor mir.

13Entschwunden bin ich wie ein Toter dem Gedenken,

bin geworden wie ein zerbrochnes Gefäß.

14Ich habe ja viele zischeln gehört: »Grauen ringsum!«

Wenn sie vereint sich gegen mich beraten, sinnen sie darauf, mir das
Leben zu rauben.

15Doch ich vertraue auf dich, o HERR;

ich sage: »Nur du bist mein Gott.«

16In deiner Hand steht meine Zeit\textless sup title=``=~mein
Geschick''\textgreater✲:

rette mich aus der Hand meiner Feinde und meiner Verfolger!

17Laß leuchten dein Angesicht über deinem Knecht,

hilf mir durch deine Gnade!

18HERR, laß mich nicht enttäuscht werden, denn ich rufe dich an!

Laß die Frevler enttäuscht werden, laß sie, zum Schweigen gebracht, in
die Unterwelt fahren!

19Verstummen müssen die Lügenlippen,

die Freches reden gegen den Gerechten in Hochmut und Verachtung!

20Wie groß ist deine Güte,

die du vorbehältst denen, die dich fürchten, die du denen erzeigst, die
ihre Zuflucht offen vor aller Welt zu dir nehmen!

21Du schirmst sie mit deines Angesichts Schirm

vor den Bosheitsplänen der Menschen, birgst sie in einer Hütte vor der
Anfeindung der Zungen.

22Gepriesen sei der HERR, daß er mir seine Gnade

wunderbar hat erwiesen in einer festen Stadt!

23Ich zwar hatte gedacht in meiner Verzagtheit,

ich sei verstoßen fern von deinen Augen; doch du hast mein lautes Flehen
gehört, als ich zu dir rief.

24Liebet den HERRN, ihr seine Frommen alle!

Die Treuen behütet der HERR, vergilt aber reichlich dem, der Hochmut
übt.

25Seid stark, und euer Herz sei unverzagt,

ihr alle, die ihr harret des HERRN!

\hypertarget{segen-der-buuxdfe-und-seligkeit-der-suxfcndenvergebung-zweiter-buuxdfpsalm}{%
\subsubsection{Segen der Buße und Seligkeit der Sündenvergebung (Zweiter
Bußpsalm)}\label{segen-der-buuxdfe-und-seligkeit-der-suxfcndenvergebung-zweiter-buuxdfpsalm}}

\hypertarget{section-31}{%
\section{32}\label{section-31}}

1Von David; ein Lehrgedicht\textless sup title=``oder: eine
Unterweisung, eine Betrachtung''\textgreater✲. Wohl dem\textless sup
title=``vgl. 1,1''\textgreater✲, dessen Missetat vergeben

und dessen Sünde zugedeckt✲ ist!

2Wohl dem Menschen, dem der HERR die Schuld nicht zurechnet

und in dessen Geist kein Trug\textless sup title=``oder:
Falsch''\textgreater✲{A} wohnt!

3Solange ich Schweigen übte, verzehrte sich mein Leib,

weil es unaufhörlich in mir schrie;

4denn bei Tag und bei Nacht lag schwer auf mir deine Hand:

mein Lebenssaft verdorrte wie durch Sommergluten. SELA.

5Da bekannte ich dir meine Sünde

und verhehlte meine Verschuldung nicht; ich sagte: »Bekennen will ich
dem HERRN meine Missetaten!« Da hast du mir meine Sündenschuld vergeben.
SELA.

6Darum möge jeder Fromme zu dir beten,

solange du dich finden läßt;{A} wenn dann gewaltige Fluten
daherstürzen~-- ihn werden sie nicht erreichen.

7Du bist mir ein Schirm, bewahrst mich vor Unheil:

mit Rettungsjubel du wirst mich umgeben. SELA.

8»Ich will dich unterweisen und dich lehren

den Weg, den du wandeln sollst; ich will dich beraten, mein Auge auf
dich richten.

9Seid nicht dem Roß, dem Maultier gleich,

die keinen Verstand besitzen; mit Zaum und Gebiß mußt du brechen ihren
Trotz, sonst kommen sie nicht zu dir.«

10Zahlreich sind die Leiden des Gottlosen;

doch wer auf den HERRN vertraut, den wird er mit Gnade umgeben.

11Freuet euch des HERRN und frohlockt, ihr Gerechten,

und jubelt, ihr redlich Gesinnten alle!

\hypertarget{aufforderung-zum-lobe-von-gottes-allmacht-und-gnade}{%
\subsubsection{Aufforderung zum Lobe von Gottes Allmacht und
Gnade}\label{aufforderung-zum-lobe-von-gottes-allmacht-und-gnade}}

\hypertarget{section-32}{%
\section{33}\label{section-32}}

1Jubelt, ihr Gerechten, über den HERRN!

Den Aufrichtigen ziemet Lobgesang.

2Preiset den HERRN mit der Zither,

spielt ihm auf zehnsaitiger Harfe!

3Singt ihm ein neues Lied,

laßt laut die Saiten erklingen mit Jubelschall!

4Denn das Wort des HERRN ist wahrhaftig,

und in all seinem Tun ist er treu;

5er liebt Gerechtigkeit und Recht;

von der Gnade\textless sup title=``oder: Güte''\textgreater✲ des HERRN
ist die Erde voll.

6Durch das Wort des HERRN sind die Himmel geschaffen,

und ihr ganzes Heer durch den Hauch seines Mundes.

7Er türmt die Wasser des Meeres auf wie einen Wall\textless sup
title=``oder: Garbenhaufen''\textgreater✲

und legt die Fluten in Vorratskammern.

8Es fürchte den HERRN die ganze Erde,

vor ihm müssen beben alle Erdenbewohner;

9denn er sprach: da geschah's;

er gebot: da stand es da.

10Der HERR hat den Ratschluß der Heiden zerschlagen,

die Gedanken der Völker vereitelt.

11Der Ratschluß des HERRN bleibt ewig bestehn,

seines Herzens Gedanken von Geschlecht zu Geschlecht.

12Wohl dem Volk, dessen Gott der HERR ist,

dem Volk, das zum Eigentum✲ er sich erwählt hat!

13Vom Himmel blickt der HERR herab,

sieht alle Menschenkinder;

14von der Stätte, wo er wohnt✲, überschaut er

alle Bewohner der Erde,

15er, der allen ihr Herz gestaltet,

der acht hat auf all ihr Tun.

16Ein König ist nicht geschützt\textless sup title=``oder:
siegreich''\textgreater✲ durch große Heeresmacht,

ein Kriegsheld rettet sich nicht durch große Kraft;

17betrogen ist, wer von Rossen die Rettung\textless sup title=``oder:
den Sieg''\textgreater✲ erhofft,

denn trotz all ihrer Stärke vermögen sie nicht zu retten.

18Bedenke: das Auge des HERRN ruht auf denen, die ihn fürchten,

auf denen, die seiner Gnade harren,

19auf daß er ihre Seele vom Tode errette

und sie am Leben erhalte in Hungersnot.

20Unsre Seele harret des HERRN:

unsre Hilfe und unser Schild ist er.

21Ja, seiner freut sich unser Herz,

denn auf seinen heiligen Namen vertrauen wir.

22Deine Gnade\textless sup title=``oder: Güte''\textgreater✲ walte über
uns, o HERR,

gleichwie wir auf dich geharrt haben\textless sup title=``oder: deiner
harren''\textgreater✲!

\hypertarget{gott-hilft-den-seinen-aus-aller-not}{%
\subsubsection{Gott hilft den Seinen aus aller
Not}\label{gott-hilft-den-seinen-aus-aller-not}}

\hypertarget{section-33}{%
\section{34}\label{section-33}}

1Von David, als er sich vor Abimelech✲ irrsinnig stellte\textless sup
title=``1.Sam 21,11-16''\textgreater✲ und dieser ihn von sich trieb, so
daß er von dannen ging. 2Ich will den HERRN allzeit preisen,

immerdar soll sein Lob in meinem Munde sein.

3Des HERRN soll meine Seele sich rühmen,

die Demütigen\textless sup title=``oder: Gebeugten''\textgreater✲ sollen
es hören und sich freuen.

4Verherrlicht mit mir den HERRN

und laßt uns gemeinsam seinen Namen erheben!

5Sooft den HERRN ich suchte, hat er mich erhört

und aus allen meinen Ängsten mich befreit.

6Wer auf ihn blickt, wird heiteren Sinnes,

und sein Antlitz braucht nicht beschämt zu erröten.

7Hier ist ein (solcher) Dulder, der rief: da hörte der HERR

und half ihm aus all seinen Nöten.

8Der Engel des HERRN lagert sich rings

um die Gottesfürchtigen und rettet sie.

9Schmecket und sehet, wie freundlich der HERR ist:

wohl dem Manne, der auf ihn vertraut!

10Fürchtet den HERRN, ihr seine Heiligen✲!

denn die ihn fürchten, leiden keinen Mangel.

11Junge Löwen müssen darben und leiden Hunger;

doch wer den HERRN sucht, entbehrt nichts Gutes.

12Kommt her, ihr Kinder, hört mir zu:

die Furcht des HERRN{A} will ich euch lehren!

13Wer ist der Mann, der langes Leben begehrt,

der viele Tage sich wünscht, um Glück zu genießen?

14Hüte deine Zunge vor Bösem

und deine Lippen vor Worten des Trugs!

15Halte dich fern vom Bösen und tu das Gute,

suche den Frieden und jage ihm nach!

16Die Augen des HERRN sind auf die Gerechten gerichtet

und seine Ohren auf ihr Hilfsgeschrei.

17Das Antlitz des HERRN steht gegen die Frevler,

um ihr Gedächtnis auszutilgen von der Erde.

18Wenn sie\textless sup title=``d.h. die Gerechten''\textgreater✲
schreien, so hört es der HERR

und rettet sie aus all ihren Nöten.{A}

19Der HERR ist nahe den zerbrochenen Herzen,

hilft denen, die zerschlagenen Geistes sind.

20Zahlreich sind die Leiden des Gerechten,

doch aus allen rettet ihn der HERR.

21Er behütet alle seine Gebeine,

daß nicht eins von ihnen zerbrochen wird.

22Den Gottlosen wird das Unglück töten,

und wer den Gerechten haßt, muß es büßen.

23Der HERR erlöst die Seele seiner Knechte,

und alle, die zu ihm sich flüchten, brauchen nicht zu büßen.

\hypertarget{hilferuf-eines-dulders-gegen-treulose-und-undankbare-feinde}{%
\subsubsection{Hilferuf eines Dulders gegen treulose und undankbare
Feinde}\label{hilferuf-eines-dulders-gegen-treulose-und-undankbare-feinde}}

\hypertarget{section-34}{%
\section{35}\label{section-34}}

1Von David. Streite, HERR, mit denen, die mich bestreiten,

kämpfe mit denen, die mich bekämpfen!

2Ergreife Schild und Tartsche\textless sup title=``=~den kleinen und den
großen Schild''\textgreater✲

und stehe auf zur Hilfe für mich!

3Zücke die Lanze und sperre meinen Verfolgern den Weg,

sprich zu meiner Seele: »Deine Hilfe bin ich!«

4Laß in Schmach und Schande geraten,

die mir nach dem Leben trachten; zurückweichen müssen und schamrot
werden, die auf Unheil gegen mich sinnen!

5Laß sie werden wie Spreu vor dem Winde,

während der Engel des HERRN sie zurückstößt!

6Ihr Weg müsse finster und schlüpfrig sein,

während der Engel des HERRN sie verfolgt!

7Denn ohn' Ursach haben sie heimlich ihr Netz mir gestellt,

meinem Leben ohn' Ursach eine Grube gegraben.

8Möge Verderben ihn unversehens treffen,

und sein Netz, das er heimlich gestellt, das möge ihn fangen: zum
Verderben gerate er selbst hinein!

9Dann wird mein Herz frohlocken über den HERRN

und sich freuen ob seiner Hilfe;

10alle Glieder meines Leibes werden bekennen:

»HERR, wer ist dir gleich? Du bist's, der den Elenden rettet vor dem
Überstarken und den Elenden und Armen vor dem Räuber.«

11Es treten Lügenzeugen (gegen mich) auf,

befragen mich über Dinge, von denen ich nichts weiß;

12sie vergelten mir Böses für Gutes,

bringen Vereinsamung über mich.

13Ich aber -- als krank sie lagen, war ein Sack mein Gewand;

ich kasteite mich mit Fasten\textless sup title=``vgl. 3.Mose
16,29''\textgreater✲, und mein Gebet kehrte sich gegen mich selbst;{A}

14als wär's mein Freund, mein Bruder, so ging ich einher;

wie einer, der Leid um die Mutter trägt, so senkte ich trauernd das
Haupt.

15Doch jetzt ob meinem Sturze frohlocken sie und tun sich zusammen,

sie treten zu kränkendem Spott zusammen gegen mich, und Leute, die ich
nicht kenne, lästern mich unaufhörlich,

16die heuchlerischen Kuchenbettler\textless sup title=``d.h. ruchlosen
Schmarotzer''\textgreater✲,

die doch mit den Zähnen gegen mich knirschen.

17O Allherr, wie lange noch willst du's ansehn?

Entreiß meine Seele ihren Lügenreden (oder Verwüstungen), mein Leben{A}
den jungen Löwen!

18Dann will ich dir danken in großer Versammlung,

vor zahlreichem Volke dich preisen.

19Laß sich nicht freun über mich, die ohn' Ursach mir feind sind,

laß nicht mit den Augen blinzeln, die ohne Grund mich hassen!

20Sie reden ja nicht, was zum Frieden dient,

nein, gegen die Stillen im Lande ersinnen sie Worte des Truges;

21sie reißen den Mund weit auf gegen mich,

sie rufen: »Haha, wir haben's mit unsern eigenen Augen gesehn!«

22Du hast's gesehn, HERR: bleibe nicht stumm,

o Allherr, bleibe nicht fern von mir,

23Erhebe dich doch, wache auf, mir Recht zu schaffen,

mein Gott und Allherr, meine Sache zu führen!

24Schaffe mir Recht nach deiner Gerechtigkeit, HERR mein Gott,

laß sie sich über mich nicht freuen!

25Laß sie in ihrem Herzen nicht sagen: »Haha! So wollten wir's!«

Laß sie nicht sagen: »Wir haben ihn verschlungen!«

26Laß sie alle enttäuscht und schamrot werden,

die meines Unglücks sich freuen, laß in Schmach und Schande sich
kleiden, die gegen mich großtun!

27Laß jubeln und fröhlich sein, die mein Recht mir wünschen,

und laß sie immer bekennen: »Groß ist der HERR, dem das Heil seines
Knechtes am Herzen liegt!«

28Dann soll meine Zunge verkünden deine Gerechtigkeit

(und) deinen Ruhm den ganzen Tag.

\hypertarget{das-heillose-treiben-der-gottlosen-und-die-segensfuxfclle-der-gottesgemeinschaft}{%
\subsubsection{Das heillose Treiben der Gottlosen und die Segensfülle
der
Gottesgemeinschaft}\label{das-heillose-treiben-der-gottlosen-und-die-segensfuxfclle-der-gottesgemeinschaft}}

\hypertarget{section-35}{%
\section{36}\label{section-35}}

1Dem Musikmeister; vom Knechte des Herrn, von David. 2Eingebung der
Sünde beherrscht den Frevler,

so läßt es im Innern meines Herzens sich hören:{A} kein Zagen vor Gott
steht ihm vor Augen\textless sup title=``vgl. Röm 3,18''\textgreater✲;

3denn sie\textless sup title=``d.h. die Sünde''\textgreater✲ verblendet
ihn mit Schmeichelreden,

daß er in Verschuldung gerät, indem er Haß ausübt✲.

4Was er ausspricht, ist Unheil und Trug;

aufgehört hat er, verständig zu sein, um gut zu handeln.

5Unheil sinnt er auf seinem Lager,

tritt hin auf den Weg der Bosheit, das Schlechte verabscheut er nicht.

6O HERR, bis zum Himmel reicht deine Gnade\textless sup title=``oder:
Güte''\textgreater✲,

deine Treue bis hin an die Wolken;

7deine Gerechtigkeit steht fest wie die Gottesberge,

deine Gerichte gleichen dem weiten\textless sup title=``oder:
tiefen''\textgreater✲ Weltmeer; Menschen und Tieren hilfst du, o HERR.

8Wie köstlich ist deine Gnade\textless sup title=``oder:
Güte''\textgreater✲, o Gott,

daß Menschenkinder sich bergen im Schatten deiner Flügel!

9Sie laben sich an den reichen Gütern deines Hauses,

und du tränkst sie mit dem Strom deiner Wonnen;

10denn bei dir ist der Brunnquell des Lebens,

und in deinem Lichte schauen wir Licht.

11Erhalte deine Gnade\textless sup title=``oder: Güte''\textgreater✲
denen, die dich kennen,

und deine Gerechtigkeit den redlich Gesinnten!

12Laß den Fuß des Hochmuts mich nicht treten

und die Hand der Frevler mich nicht vertreiben!

13Einst werden die Übeltäter gefallen sein,

niedergestürzt und können nicht wieder aufstehn.

\hypertarget{das-scheingluxfcck-der-frevler-gottlosen-darf-die-gerechten}{%
\subsubsection{Das Scheinglück der Frevler (=~Gottlosen) darf die
Gerechten✲}\label{das-scheingluxfcck-der-frevler-gottlosen-darf-die-gerechten}}

\hypertarget{section-36}{%
\section{37}\label{section-36}}

1Von David. Entrüste dich nicht über die Bösen

und ereifre dich nicht über die Übeltäter!

2denn schnell wie das Gras verwelken sie

und verdorren wie grünender Rasen.

3Vertrau auf den HERRN und tu das Gute,

bleib wohnen im Lande und übe Redlichkeit

4und habe deine Lust am HERRN:

so wird er dir geben, was dein Herz begehrt.

5Befiehl dem HERRN deine Wege

und vertraue auf ihn: er wird's wohl machen\textless sup
title=``=~heilsam lenken''\textgreater✲

6und deine Gerechtigkeit strahlen lassen wie das Licht

und dein Recht wie den hellen Mittag.

7Sei stille dem HERRN und harre auf ihn,

entrüste dich nicht über den, der Glück hat bei seinem Tun, über den
Mann, der Ränke übt!

8Steh ab vom Zorn und entsage dem Grimm,

entrüste dich nicht: es führt nur zum Bösestun!

9Denn die Übeltäter werden ausgerottet,

doch die da harren des HERRN, die werden das Land besitzen.

10Nur noch ein Weilchen, so wird der Frevler nicht mehr sein,

und siehst du dich um nach seiner Stätte, so ist er nicht mehr da;

11die stillen Dulder aber werden das Land besitzen

und sich freun an der Fülle des Friedens\textless sup title=``=~des
Wohlergehens''\textgreater✲.

12Böses sinnt der Frevler gegen den Gerechten

und knirscht mit den Zähnen gegen ihn;

13der Allherr aber lacht über ihn,

denn er sieht, daß sein Tag kommt.

14Die Frevler zücken das Schwert und spannen den Bogen,

um den Dulder und Armen niederzustrecken und die redlich Wandelnden
hinzumorden;

15doch ihr Schwert dringt ihnen ins eigne Herz,

und ihre Bogen werden zerbrochen.

16Das geringe Gut des Gerechten ist besser

als der Überfluß vieler Gottlosen;

17denn die Arme der Gottlosen werden zerbrochen,

die Gerechten aber stützt der HERR.

18Der HERR kennt wohl die Tage der Frommen,

und ihr Besitz ist für immer gesichert;

19sie werden nicht zuschanden in böser Zeit,

nein, in den Tagen des Hungers werden sie satt.

20Dagegen die Gottlosen gehen zugrunde,

und die Feinde des HERRN sind wie die Pracht der Auen: sie vergehen wie
Rauch, sie vergehen!

21Der Gottlose muß borgen und kann nicht zahlen,

der Gerechte aber schenkt und gibt;

22denn die vom HERRN Gesegneten erben das Land,

aber die von ihm Verfluchten werden vernichtet.

23Vom HERRN her werden die Schritte des Mannes gefestigt,

und zwar wenn Gefallen er hat an seinem Wandel;

24wenn er strauchelt, stürzt er nicht völlig nieder,

denn der HERR stützt ihm die Hand.

25Ich bin jung gewesen und alt geworden,

doch hab' ich nie den Gerechten verlassen gesehn, noch seine Kinder
betteln um Brot.

26Allzeit kann er schenken und darleihn,

und auch noch seine Kinder sind zum Segen.

27Halte dich fern vom Bösen und tu das Gute,

so wirst du für immer wohnen bleiben;

28Denn der HERR hat das Recht lieb

und verläßt seine Frommen nicht: ewiglich werden sie behütet, doch der
Gottlosen Nachwuchs wird ausgerottet.

29Die Gerechten werden das Land besitzen

und bleiben in ihm wohnen für immer.

30Des Gerechten Mund läßt Weisheit hören,

und seine Zunge redet Recht;

31das Gesetz seines Gottes wohnt ihm im Herzen,

und seine Schritte wanken nicht.

32Der Gottlose lauert dem Gerechten auf

und sucht ihn ums Leben zu bringen;

33doch der HERR läßt ihn nicht fallen in seine Hand

und läßt ihn nicht verdammen vor Gericht.

34Harre des HERRN und halte dich an seinen Weg,

so wird er dich erhöhn zum Besitz des Landes; an der Gottlosen
Vernichtung wirst du deine Freude sehn.

35Ich hab' einen Frevler gesehen, der trat gar trotzig auf

und spreizte sich stolz wie ein grünender, ragender Baum;{A}

36doch als ich (wieder) vorüberging, da war er verschwunden,

und als ich ihn suchte, war er nicht mehr zu finden.

37Bleibe (also) fromm und halte dich recht,

denn solchen wird es zuletzt wohl ergehn;{A}

38die Frevler aber werden allesamt vertilgt,

und der Gottlosen Nachwuchs wird ausgerottet.

39Die Hilfe der Gerechten kommt vom HERRN:

er ist ihre Schutzwehr\textless sup title=``oder:
Zuflucht''\textgreater✲ zur Zeit der Not;

40denn der HERR steht ihnen bei und rettet sie;

er rettet sie von den Frevlern und bringt ihnen Hilfe, weil auf ihn sie
ihr Vertrauen setzen.

\hypertarget{buuxdfgebet-und-hilferuf-in-schwerer-krankheit-und-seelennot-dritter-buuxdfpsalm}{%
\subsubsection{Bußgebet und Hilferuf in schwerer Krankheit und Seelennot
(Dritter
Bußpsalm)}\label{buuxdfgebet-und-hilferuf-in-schwerer-krankheit-und-seelennot-dritter-buuxdfpsalm}}

\hypertarget{section-37}{%
\section{38}\label{section-37}}

1Ein Psalm von David, bei Darbringung des Duftopfers. 2HERR, nicht in
deinem Zorne strafe mich,

und nicht in deinem Ingrimm züchtige mich!\textless sup title=``vgl.
6,2''\textgreater✲

3Denn deine Pfeile sind in mich eingedrungen,

und deine Hand liegt schwer auf mir:

4nichts ist gesund an meinem Leib ob deinem Zürnen,

nichts heil an meinen Gliedern ob meiner Sünde.

5Denn meine Missetaten schlagen mir über dem Haupt zusammen;

wie eine schwere Last sind sie mir zu schwer geworden\textless sup
title=``=~erdrücken sie mich''\textgreater✲.

6Es faulen, es eitern meine Wunden

infolge meiner Torheit✲.

7Ich bin gekrümmt, tief niedergebeugt;

den ganzen Tag geh' ich trauernd\textless sup title=``=~im
Trauergewand''\textgreater✲ einher;

8denn meine Lenden sind voll von Entzündung,

und nichts ist unversehrt an meinem Leibe.

9Erschöpft bin ich und ganz zerschlagen,

ich schreie auf infolge des Stöhnens meines Herzens.

10O Allherr, all mein Verlangen ist dir bekannt,

und meine Seufzer sind dir nicht verborgen.

11Mein Herz pocht stürmisch, meine Kraft hat mich verlassen,

und das Licht meiner Augen, auch das ist dahin!

12Meine Freunde und Genossen stehn abseits von meinem Elend,

und meine nächsten Verwandten halten sich fern.

13Die nach dem Leben mir trachten, legen mir Schlingen,

und die mein Unglück suchen, verabreden Unheil und sinnen auf Trug den
ganzen Tag.

14Doch ich bin wie ein Tauber, höre es nicht,

und bin wie ein Stummer, der den Mund nicht auftut;

15ja, ich bin wie einer, der nicht hören kann

und in dessen Mund keine Widerrede ist;

16denn auf dich, HERR, warte ich:

du wirst antworten\textless sup title=``oder: mich
erhören''\textgreater✲, o Allherr, mein Gott;

17denn ich sage: »Daß sie nur nicht über mich frohlocken,

nur nicht beim Wanken meines Fußes gegen mich großtun!«

18Denn nahe bin ich am Zusammenbrechen,

und mein Schmerz ist mir allezeit gegenwärtig.

19Ach! Ich bekenne meine Schuld,

bin bekümmert ob meiner Sünde!

20Dagegen die ohne Grund mich befeinden, sind stark,

und zahlreich sind, die ohn' Ursach' mich hassen,

21und solche, die mir Gutes mit Bösem vergelten,

sind meine Widersacher, weil fest am Guten ich halte.{A}

22Verlaß mich nicht, o HERR,

mein Gott, sei nicht ferne von mir!

23Eile zu meinem Schutz herbei,

o Allherr, meine Rettung!

\hypertarget{klage-und-bitte-eines-schwer-angefochtenen}{%
\subsubsection{Klage und Bitte eines schwer
Angefochtenen}\label{klage-und-bitte-eines-schwer-angefochtenen}}

\hypertarget{section-38}{%
\section{39}\label{section-38}}

1Dem Musikmeister Jeduthun; ein Psalm von David. 2Ich dachte: »Achten
will ich auf meine Wege\textless sup title=``=~mein
Verhalten''\textgreater✲,

daß ich nicht sünd'ge mit meiner Zunge; ich will meinem Mund einen Zaum
anlegen, solange noch der Frevler\textless sup title=``oder: ein
Gottloser''\textgreater✲ vor mir steht.«

3So ward ich denn stumm, ganz stumm, mit Gewalt schweigsam✲;

doch es wühlte mein Schmerz noch wilder.

4Das Herz ward mir heiß in der Brust,

ob meinem Grübeln brannte ein Feuer in mir; da ließ ich meiner Zunge
freien Lauf:

5»HERR, laß mein Ende mich wissen

und welches\textless sup title=``=~wie klein''\textgreater✲ das Maß
meiner Tage ist! Laß mich erkennen, wie vergänglich ich bin!

6Ach, spannenlang hast du mir die Tage gemacht,

und meines Lebens Dauer ist wie nichts vor dir: ja, nur als ein Hauch
steht jeglicher Mensch da!« SELA.

7Fürwahr nur als Schattenbild wandelt der Mensch einher,

nur um ein Nichts wird so viel Lärm gemacht; man häuft auf und weiß
nicht, wer es einheimst.

8Und nun, o Allherr, wes soll ich harren?

Meine Hoffnung geht auf dich (allein).

9Errette mich von allen meinen Sünden,

zum Spott der Toren laß mich nicht werden!

10Ich schweige, tu meinen Mund nicht auf,

denn du hast's so gefügt.

11Nimm deine Plage weg von mir:

unter dem Druck deiner Hand erlieg' ich.

12Züchtigst du einen Menschen

mit Strafen um der Sünde willen, so läßt du seine Schönheit vergehn wie
die Motte\textless sup title=``=~wie Mottenfraß''\textgreater✲: ach, nur
ein Hauch ist jeglicher Mensch! SELA.

13Höre, o HERR, mein Gebet und vernimm mein Schreien,

bleib' nicht stumm bei\textless sup title=``oder: zu''\textgreater✲
meinen Tränen! Denn ein Gast (nur) bin ich bei dir, ein Beisaß✲ wie all
meine Väter\textless sup title=``vgl. Hebr 11,13''\textgreater✲.

14Blick weg von mir, daß mein Antlitz sich wieder erheitert,

bevor ich dahinfahre und nicht mehr bin!

\hypertarget{dank--und-bittgebet}{%
\subsubsection{Dank- und Bittgebet}\label{dank--und-bittgebet}}

\hypertarget{section-39}{%
\section{40}\label{section-39}}

1Dem Musikmeister; von David ein Psalm. 2Geduldig hatte ich des HERRN
geharrt:

da neigte er sich zu mir und hörte mein Schreien;

3er zog mich herauf aus der Grube des Unheils,

aus dem schlammigen Sumpf, und stellte meine Füße auf Felsengrund,
verlieh meinen Schritten Festigkeit;

4er legte ein neues Lied mir in den Mund,

einen Lobgesang auf unsern Gott. Das werden viele sehen\textless sup
title=``oder: erfahren''\textgreater✲ und Ehrfurcht fühlen und Vertrauen
fassen zum HERRN.

5Glückselig der Mann\textless sup title=``vgl. 1,1''\textgreater✲, der
sein Vertrauen setzt auf den HERRN,

der's nicht mit den Stolzen hält und nicht mit den treulosen
Lügenfreunden!

6Zahlreich sind die Wunder, die du getan hast,

und deine Heilsgedanken mit uns, o HERR, mein Gott; dir ist nichts zu
vergleichen; wollt' ich von ihnen reden und sie verkünden -- sie
übersteigen jede Zahl.

7An Schlacht- und Speisopfern hast du kein Gefallen,

doch offne Ohren hast du mir gegeben; nach Brand- und Sündopfern trägst
du kein Verlangen.

8Da hab' ich gesagt: »Siehe, hier bin ich!

In der Rolle des Buches, da steht für mich geschrieben:

9Deinen Willen zu tun, mein Gott, ist meine Lust,

und dein Gesetz ist tief mir ins Herz geschrieben.«\textless sup
title=``vgl. Hebr 10,5-7''\textgreater✲

10Von (deiner) Gerechtigkeit hab' ich in großer Versammlung gesprochen,

siehe, meinen Lippen hab' ich nicht Einhalt getan: du selbst, HERR,
weißt es!

11Deine Gerechtigkeit habe ich nicht verborgen in meinem Herzen,

von deiner Treue und Hilfe laut geredet; ich habe deine Gnade und
Wahrheit\textless sup title=``oder: Treue''\textgreater✲ nicht
verschwiegen, vor der großen Versammlung.

12So wirst du, HERR, mir dein Erbarmen nicht versagen;

deine Gnade und Wahrheit werden stets mich behüten.

13Denn Leiden ohne Zahl umringen mich,

meine Sünden haben mich ereilt, unübersehbar; zahlreicher sind sie als
die Haare meines Hauptes, und der Mut ist mir entschwunden.

14Laß dir's wohlgefallen, o HERR, mich zu retten,

eile, o HERR, zu meiner Hilfe herbei!

15Laß sie allesamt beschämt und schamrot werden,

die nach dem Leben mir stehn, um es wegzuraffen! Laß mit Schande beladen
abziehn, die mein Unglück wünschen!

16Erschaudern müssen ob ihrer Schmach,

die über mich rufen: »Haha, haha!«

17Laß jubeln und deiner sich freuen

alle, die dich suchen; laß alle, die nach deinem Heil verlangen,
immerdar bekennen: »Groß ist der HERR.«

18Bin ich auch elend und arm -- der Allherr wird für mich sorgen.

Meine Hilfe und mein Retter bist du: mein Gott, säume nicht!

\hypertarget{klage-eines-kranken-uxfcber-boshafte-feinde-und-treulose-freunde}{%
\subsubsection{Klage eines Kranken über boshafte Feinde und treulose
Freunde}\label{klage-eines-kranken-uxfcber-boshafte-feinde-und-treulose-freunde}}

\hypertarget{section-40}{%
\section{41}\label{section-40}}

1Dem Musikmeister; ein Psalm von David. 2Wohl dem, der des
Schwachen\textless sup title=``oder: Geringen''\textgreater✲ sich
annimmt:

am Tage des Unglücks wird der HERR ihn erretten!

3Der HERR wird ihn behüten und am Leben erhalten,

daß er glücklich gepriesen wird im Lande; und du gibst ihn nicht preis
der Gier seiner Feinde.

4Der HERR wird ihn auf dem Siechbett erquicken:

sein ganzes Krankenlager machst du ihm leicht.{A}

5Ich sage: »O HERR, sei mir gnädig,

ach, heile meine Seele, denn an dir hab' ich gesündigt!«

6Meine Feinde reden Böses\textless sup title=``oder:
Schlimmes''\textgreater✲ von mir:

»Wann wird er sterben, daß sein Name verschwindet?«

7Kommt jemand, mich zu besuchen, so redet er Falschheit;

sein Herz sammelt Bosheit an; dann geht er hinaus, um draußen davon zu
reden.

8Alle, die mich hassen, zischeln vereint über mich,

Unheil sinnen sie gegen mich:

9»Ein heilloses Übel haftet ihm an!

Wer so sich gelegt hat, kommt nicht wieder hoch!«

10Sogar mein bester Freund, dem ich fest vertraute,

der mein Brot aß,{A} hat die Ferse gegen mich erhoben.

11Du aber, HERR, sei mir gnädig und hilf mir wieder auf,

so will ich's ihnen vergelten!

12Daran will ich erkennen, daß du Gefallen an mir hast,

wenn mein Feind nicht über mich jubeln wird,

13doch du mich ob meiner Unschuld aufrecht hältst

und mich vor deinem Angesicht stehn läßt immerdar. * * *

14Gepriesen sei der HERR, der Gott Israels,

von Ewigkeit zu Ewigkeit! Amen, ja Amen!

\hypertarget{zweites-buch-psalm-42-72}{%
\subsection{Zweites Buch (Psalm 42-72)}\label{zweites-buch-psalm-42-72}}

\hypertarget{sehnsucht-nach-gott-und-seinem-heiligtum-auf-zion}{%
\subsubsection{Sehnsucht nach Gott und seinem Heiligtum auf
Zion}\label{sehnsucht-nach-gott-und-seinem-heiligtum-auf-zion}}

\hypertarget{section-41}{%
\section{42}\label{section-41}}

1Dem Musikmeister; ein Lehrgedicht\textless sup title=``vgl.
32,1''\textgreater✲ der Korahiten\textless sup title=``vgl. 4.Mose
26,11''\textgreater✲. 2Wie der Hirsch lechzt nach Wasserbächen,

so lechzt meine Seele nach dir, o Gott!

3Meine Seele dürstet nach Gott, dem lebendigen Gott:

wann werde ich dahin kommen, daß ich erscheine vor Gottes Angesicht?

4Meine Tränen sind meine Speise geworden

bei Tag und bei Nacht, weil man den ganzen Tag zu mir sagt: »Wo ist nun
dein Gott?«

5Daran will ich gedenken -- und meinem Schmerz

freien Lauf in mir lassen --, wie einst ich dahinschritt in dichter
Schar, mit ihnen wallte zum Hause Gottes, umbraust von lautem Jubel und
Lobgesang inmitten der feiernden Menge.

6Was betrübst du dich, meine Seele,

und stürmst so ruhlos in mir? Harre auf Gott! Denn ich werde ihm noch
danken, ihm, meines Angesichts Hilfe und meinem Gott.

7Gebeugt ist meine Seele in mir: drum denk' ich an dich

im Lande des Jordans und der Hermongipfel, am Berge Mizar:

8Flut ruft der Flut zu beim Tosen deiner Wasserstürze;

alle, alle deine Wogen und Wellen sind über mich hingegangen!

9Bei Tag seufz' ich: »Es entbiete der HERR seine Gnade!«,

und nachts ist sein Lied in meinem Munde, ein Gebet zum Gott meines
Lebens.

10Ich sage zu Gott, meinem Felsen:

»Warum hast du mich vergessen? Warum muß ich trauernd einhergehn unter
dem Druck des Feindes?«

11Wie Zermalmung liegt mir in meinen Gebeinen

der Hohn meiner Dränger, weil sie den ganzen Tag zu mir sagen: »Wo ist
nun dein Gott?«

12Was betrübst du dich, meine Seele,

und stürmst so ruhlos in mir? Harre auf Gott! Denn ich werde ihm noch
danken, ihm, meines Angesichts Hilfe und meinem Gott.

\hypertarget{section-42}{%
\section{43}\label{section-42}}

1Schaffe mir Recht, o Gott, und führe meinen Rechtsstreit

gegen ein liebloses\textless sup title=``oder: unfrommes''\textgreater✲
Volk! Von Menschen des Trugs und der Bosheit errette mich, HERR!

2Du bist ja der Gott, der mich schützt:

warum hast du mich verstoßen? Warum muß ich trauernd einhergehn unter
dem Druck des Feindes?

3Sende dein Licht und deine Treue!

Die sollen mich leiten, mich bringen zu deinem heiligen Berge und deiner
Wohnstatt,

4damit ich zum Altar Gottes komme,

zu dem Gott meines Freudenjubels, und unter Zitherklang dich
preise\textless sup title=``oder: dir danke''\textgreater✲, o Gott, mein
Gott!

5Was betrübst du dich, meine Seele,

und stürmst so ruhlos in mir? Harre auf Gott! Denn ich werde ihm noch
danken, ihm, meines Angesichts Hilfe und meinem Gott.

\hypertarget{klagelied-und-hilferuf-des-gesetzestreuen-aber-von-seinen-feinden-besiegten-und-miuxdfhandelten-volkes}{%
\subsubsection{Klagelied und Hilferuf des gesetzestreuen, aber von
seinen Feinden besiegten und mißhandelten
Volkes}\label{klagelied-und-hilferuf-des-gesetzestreuen-aber-von-seinen-feinden-besiegten-und-miuxdfhandelten-volkes}}

\hypertarget{section-43}{%
\section{44}\label{section-43}}

1Dem Musikmeister; von den Korahiten✲ ein Lehrgedicht\textless sup
title=``vgl. 32,1''\textgreater✲. 2O Gott, mit eignen Ohren haben wir's
gehört,

unsre Väter haben's uns erzählt, was du Großes in ihren Tagen vollführt
hast, in den Tagen der Vorzeit.

3Du hast Heidenvölker mit deiner Hand vertrieben

und sie{A} an deren Stelle eingepflanzt; Völker hast du vernichtet,
sie{B} aber ausgebreitet.

4Denn nicht mit ihrem Schwerte haben sie das Land gewonnen,

und nicht ihr Arm hat ihnen den Sieg verschafft, nein, deine Rechte und
dein Arm und deines Angesichts Licht, denn du hattest Gefallen an ihnen.

5Nur du bist mein König, o Gott:

entbiete Hilfe\textless sup title=``oder: Heil, Siege''\textgreater✲ für
Jakob!

6Mit dir stoßen wir unsre Bedränger nieder,

mit deinem Namen zertreten wir unsre Gegner.

7Denn nicht auf meinen Bogen verlasse ich mich,

und nicht mein Schwert verschafft mir den Sieg;

8nein, du gewährst uns Hilfe gegen unsre Bedränger

und machst zuschanden, die uns hassen:

9Gottes rühmen wir uns allezeit

und preisen deinen Namen ewiglich. SELA.

10Und doch hast du uns verstoßen und Schmach uns angetan

und ziehst nicht mehr aus mit unsern Heeren;

11du hast vor dem Feinde uns weichen lassen,

und die uns hassen, haben sich Beute geholt;

12du hast uns hingegeben wie Schafe zur Schlachtung

und unter die Heiden uns zerstreut;

13du hast dein Volk verkauft um ein Spottgeld

und den Preis für sie gar niedrig angesetzt;{A}

14du hast uns unsern Nachbarn zum Hohn gemacht,

zum Spott und Gelächter rings umher;

15hast gemacht, daß den Heiden zum Sprichwort\textless sup title=``oder:
Spottlied''\textgreater✲ wir dienen,

daß den Kopf die Völker über uns schütteln.

16Allzeit steht meine Schmach mir vor Augen,

und die (Röte der) Scham bedeckt mir das Antlitz,

17weil ich höre den lauten Hohn und die Lästerreden,

weil den Feind und seine Rachgier ich sehn muß.

18Dies alles hat uns getroffen,

und wir hatten dich doch nicht vergessen und dem Bunde mit dir die Treue
nicht gebrochen.

19Unser Herz ist nicht von dir abgefallen

und unser Schritt nicht abgewichen von deinem Pfade,

20daß du zermalmt uns hast an der Stätte der Schakale

und mit Todesnacht uns umlagert hältst.

21Hätten wir unsres Gottes Namen vergessen

und unsre Hände erhoben zu einem fremden Gott:

22würde Gott das nicht entdecken?

Er kennt ja des Herzens geheimste Gedanken.

23Nein, um deinetwillen werden wir täglich gemordet

und werden dem Schlachtvieh gleich geachtet.

24Wach auf! Warum schläfst du, o Allherr?

Erwache! Verwirf nicht für immer!

25Warum verbirgst du dein Angesicht,

denkst nicht an unser Elend und unsre Bedrängnis?

26Ach, bis in den Staub ist unsre Seele gebeugt,

unser Leib liegt da, am Erdboden klebend!

27Steh auf, komm uns zu Hilfe

und erlöse uns um deiner Gnade willen!

\hypertarget{festlied-zur-vermuxe4hlung-des-kuxf6nigs}{%
\subsubsection{Festlied zur Vermählung des
Königs}\label{festlied-zur-vermuxe4hlung-des-kuxf6nigs}}

\hypertarget{section-44}{%
\section{45}\label{section-44}}

1Dem Musikmeister, nach (der Singweise =~Melodie) »Lilien«; von den
Korahiten✲ ein Lehrgedicht\textless sup title=``vgl.
32,1''\textgreater✲, ein Liebeslied\textless sup title=``oder: ein Lied
von lieblichen Dingen?''\textgreater✲. 2Das Herz wallt mir auf von
lieblichen Worten:

dem Könige weihe ich meine Lieder; meine Zunge gleicht dem Griffel eines
geübten Schreibers.

3Du bist so schön wie sonst kein Mensch auf Erden:

Anmut ist ausgegossen auf deine Lippen; darum hat Gott dich gesegnet für
immer.

4Gürte dein Schwert dir an die Seite, du Held,

dazu deine herrlich schimmernde Wehr!

5Glück auf! Fahre siegreich einher

für die Sache der Wahrheit, zum Schutz des Rechts, und furchtbare Taten
lasse dein Arm dich schauen!

6Deine Pfeile sind scharf -- Völker sinken unter dir hin --:

sie dringen den Feinden des Königs ins Herz.

7Dein Thron, ein Gottesthron, steht immer und ewig

ein gerechtes Zepter ist dein Herrscherstab.

8Du liebst Gerechtigkeit und hassest den Frevel;

darum hat dich Gott, dein Gott, gesalbt mit Freudenöl wie keinen
deinesgleichen.

9Von Myrrhe und Aloe duften, von Kassia alle deine Kleider;

aus Elfenbeinpalästen erfreut dich Saitenspiel.

10Königstöchter befinden sich unter deinen Geliebten✲;

die Gattin\textless sup title=``oder: Braut''\textgreater✲ steht dir zur
Rechten im Goldschmuck von Ophir.

11Höre, Tochter, blick her und neige dein Ohr:

Vergiß dein Volk und deines Vaters Haus;

12und trägt der König nach deiner Schönheit Verlangen

er ist ja dein Herr --: so huldige ihm!

13Die Bürgerschaft von Tyrus wird mit Gaben dir nahen,

um deine Gunst mühen sich die Reichsten des Volkes.

14Eitel Pracht ist die Königstochter drinnen,

aus gewirktem Gold besteht ihr Gewand;

15in buntgestickten Kleidern wird sie zum König geführt;

Jungfraun, ihr Gefolge, ihre Gespielinnen\textless sup title=``oder:
Freundinnen''\textgreater✲, werden zu dir geleitet;

16unter Freudenrufen und Jubel werden sie hingeführt,

ziehen ein in den Palast des Königs.

17An deiner Väter Stelle werden deine Söhne treten;

du wirst sie zu Fürsten bestellen im ganzen Land.

18Ich will ein Gedächtnis stiften deinem Namen

bei allen kommenden Geschlechtern; darum werden die Völker dich preisen
immer und ewig.

\hypertarget{ein-feste-burg-ist-unser-gott}{%
\subsubsection{Ein' feste Burg ist unser
Gott}\label{ein-feste-burg-ist-unser-gott}}

\hypertarget{section-45}{%
\section{46}\label{section-45}}

1Dem Musikmeister, von den Korahiten✲, im Bass-Stimmsatz, ein Lied.
2Gott ist uns Zuflucht und Stärke,

als Hilfe in Nöten wohlbewährt befunden.

3Darum bangen wir nicht, wenngleich die Erde vergeht,

wenn Berge mitten\textless sup title=``oder: tief''\textgreater✲ im Meer
versinken;

4mögen tosen, mögen schäumen seine Wogen,

mögen beben die Berge von seinem Ungestüm: der HERR der Heerscharen ist
mit uns, ein' feste Burg ist uns der Gott Jakobs! SELA.

5Ein Strom ist da: seine Bäche erfreuen die Gottesstadt,

das Heiligtum, die Wohnung des Höchsten.

6Gott ist in ihrer Mitte: sie wird nicht wanken,

Gott schützt sie, schon wenn der Morgen tagt.

7Völker tobten, Königreiche wankten:

er ließ seinen Donner dröhnen, da zerfloß\textless sup title=``oder:
erschrak''\textgreater✲ die Erde.

8Der HERR der Heerscharen ist mit uns,

ein' feste Burg ist uns der Gott Jakobs! SELA.

9Kommt her und schauet die Taten des HERRN,

der Wunderbares\textless sup title=``oder: Entsetzen''\textgreater✲
wirket auf Erden,

10der den Kriegen ein Ziel setzt bis ans Ende der Erde,

der Bogen zerbricht und Speere zerschlägt, Kriegswagen mit Feuer
verbrennt!

11»Laßt ab und erkennt, daß ich (nur) Gott bin,

erhaben unter den Völkern, erhaben auf Erden!«

12Der HERR der Heerscharen ist mit uns,

ein' feste Burg ist uns der Gott Jakobs! SELA.

\hypertarget{israels-gott-als-kuxf6nig-aller-vuxf6lker}{%
\subsubsection{Israels Gott als König aller
Völker}\label{israels-gott-als-kuxf6nig-aller-vuxf6lker}}

\hypertarget{section-46}{%
\section{47}\label{section-46}}

1Dem Musikmeister, von den Korahiten✲, ein Psalm. 2Ihr Völker alle,
klatscht in die Hände,

jauchzet Gott mit Jubelrufen zu!

3Denn der HERR, der Höchste, ist furchtbar✲,

ein mächtiger König über die ganze Erde.

4Er hat Völker unter unsre Herrschaft gebeugt

und Völkerschaften unter unsre Füße;

5er hat uns unser Erbteil auserwählt,

den Stolz Jakobs, den er liebt. SELA.

6Aufgefahren ist Gott unter Jauchzen,

der HERR beim Schall der Posaunen.

7Lobsinget Gott, lobsinget,

lobsingt unserm König, lobsinget!

8Denn König der ganzen Erde ist Gott:

so singt ihm denn ein kunstvolles Lied!

9Gott ist König geworden\textless sup title=``vgl. 96,10''\textgreater✲
über die Völker,

Gott hat sich gesetzt auf seinen heiligen Thron.

10Die Edlen\textless sup title=``oder: Fürsten''\textgreater✲ der Völker
haben sich versammelt

als das Volk des Gottes Abrahams; denn Gott sind untertan die
Schilde\textless sup title=``d.h. Beherrscher''\textgreater✲ der Erde:
hoch erhaben steht er da.

\hypertarget{der-festpilger-lobpreis-auf-zion-die-unbesiegte-gottesstadt}{%
\subsubsection{Der Festpilger Lobpreis auf Zion, die unbesiegte
Gottesstadt}\label{der-festpilger-lobpreis-auf-zion-die-unbesiegte-gottesstadt}}

\hypertarget{section-47}{%
\section{48}\label{section-47}}

1Ein Lied, ein Psalm von den Korahiten✲. 2Groß ist der HERR und hoch zu
preisen

in unsers Gottes Stadt, auf seinem heiligen Berge.

3Herrlich ragt empor, die Wonne der ganzen Erde\textless sup
title=``oder: des ganzen Landes''\textgreater✲,

der Zionsberg, der wahre Götterberg,{A} die Stadt eines\textless sup
title=``oder: des''\textgreater✲ großen Königs.

4Gott hat in ihren Palästen

sich kundgetan als eine feste Burg.

5Denn siehe, die Könige hatten sich versammelt\textless sup
title=``oder: verabredet''\textgreater✲,

waren vereint herangezogen;

6doch als sie's sahen, waren sie betroffen✲,

erschraken, flohen bestürzt davon;

7Zittern erfaßte sie dort,

Angst wie ein Weib in Wehen.

8Durch einen Oststurm zertrümmertest du

die stolzen Tharsisschiffe\textless sup title=``Jes 2,16''\textgreater✲.

9Wie wir's gehört, so haben wir's nun gesehen

in der Stadt des HERRN der Heerscharen, unsres Gottes Stadt: Gott läßt
sie auf ewig feststehn. SELA.

10Wir gedenken, o Gott, deiner Gnade

inmitten deines Tempels.

11Wie dein Name, o Gott, so reicht auch dein Ruhm

bis an die Enden der Erde; mit Gerechtigkeit ist deine Rechte gefüllt.

12Des freue sich der Zionsberg,

jubeln mögen die Töchter Judas um deiner Gerichte willen!

13Umkreist den Zion, umwandelt ihn rings

und zählt seine Türme;

14betrachtet genau seine Bollwerke,

mustert\textless sup title=``oder: durchschreitet''\textgreater✲ seine
Paläste, damit ihr dem künftgen Geschlecht erzählet,

15daß dies ist Gott, unser Gott:

immer und ewig wird er uns führen {[}bis zum Tode{]}.

\hypertarget{verguxe4nglichkeit-des-uxe4uuxdferen-gluxfccks-der-gottlosen}{%
\subsubsection{Vergänglichkeit des äußeren Glücks der
Gottlosen}\label{verguxe4nglichkeit-des-uxe4uuxdferen-gluxfccks-der-gottlosen}}

\hypertarget{section-48}{%
\section{49}\label{section-48}}

1Dem Musikmeister, von den Korahiten✲, ein Psalm. 2Höret dies, ihr
Völker alle,

merkt auf, ihr Bewohner der ganzen Welt,

3sowohl ihr Söhne des Volks als ihr Herrensöhne,

beide, so reich wie arm!

4Mein Mund soll volle Weisheit reden,

und meines Herzens Sinnen soll höchste Einsicht sein:

5ich will mein Ohr einer Gleichnisrede\textless sup title=``=~einem
Gottesspruch?''\textgreater✲ leihen,

will mein Rätsel eröffnen bei Saitenklang.

6Warum sollt' ich mich fürchten in bösen Tagen,

wenn die Bosheit meiner Verfolger mich umgibt,

7die auf ihr Vermögen vertrauen

und mit ihrem großen Reichtum prahlen?

8Den Bruder loszukaufen vermag ja doch kein Mensch,

noch an Gott das Lösegeld für ihn zu zahlen

9- denn unerschwinglich hoch ist der Kaufpreis für ihr Leben:

er muß davon Abstand nehmen für immer --,

10damit er dauernd weiterlebe

und die Grube nicht zu sehen bekomme.

11Nein, er bekommt es zu sehen, daß sterben die Weisen,

und Toren und Dumme gleicherweise umkommen und müssen andern ihr Gut
hinterlassen:

12Gräber sind ihre Behausung für immer,

ihre Wohnungen von Geschlecht zu Geschlecht, ob sie auch Länder mit
ihren Namen benannten.{A}

13Ja, der Mensch -- in Herrlichkeit lebt er nicht fort:

er gleicht den Tieren, die abgetan werden.

14Dies ist das Schicksal derer, die voll Zuversicht sind,

und der Ausgang derer, die ihren Reden Beifall schenken. SELA.

15Wie Schafe werden sie ins Totenreich versetzt;

der Tod weidet sie, und über sie herrschen die Frommen am Morgen✲; dem
Totenreich zur Vernichtung fällt ihre Gestalt anheim, so daß ihr keine
Wohnung bleibt.

16Aber Gott wird meine Seele erlösen

aus des Totenreichs Gewalt, denn er wird mich annehmen\textless sup
title=``oder: entrücken''\textgreater✲. SELA.

17Drum rege dich nicht auf, wenn jemand reich wird,

wenn seines Hauses Herrlichkeit sich mehrt;

18denn im Tode nimmt er das alles nicht mit:

seine Herrlichkeit fährt nicht mit ihm hinab.

19Mag er sich auch im Leben glücklich preisen

und mag man ihn rühmen, daß es ihm wohlergehe:{A}

20er wird doch eingehn zum Geschlecht seiner Väter,

die das Tageslicht nimmermehr sehen.

21Der Mensch, in Herrlichkeit lebend, doch ohne Einsicht,

gleicht den Tieren, die abgetan werden.

\hypertarget{der-rechte-gottesdienst}{%
\subsubsection{Der rechte Gottesdienst}\label{der-rechte-gottesdienst}}

\hypertarget{section-49}{%
\section{50}\label{section-49}}

1Ein Psalm von Asaph. Der Gott der Götter, der HERR, redet und ruft der
Erde zu

vom Aufgang der Sonne bis zu ihrem Niedergang;

2aus Zion, der Krone der Schönheit✲,

strahlt Gott in lichtem Glanz hervor:

3unser Gott kommt und kann nicht schweigen,

verzehrendes Feuer geht vor ihm her, und rings um ihn her stürmt es
gewaltig.

4Er ruft dem Himmel droben zu

und der Erde, um sein Volk zu richten:

5»Versammelt mir meine Gesetzestreuen✲,

die den Bund mit mir geschlossen beim Opfer!«

6Da taten die Himmel seine Gerechtigkeit kund;

denn Gott selbst ist's, der da Gericht hält. SELA.

7»Höre, mein Volk, und laß mich reden,

Israel, daß ich dich ernstlich warne: Gott, dein Gott bin ich!

8Nicht deiner Opfer wegen rüge ich dich,

sind doch deine Brandopfer stets mir vor Augen.

9Doch ich mag nicht Stiere nehmen aus deinem Hause\textless sup
title=``oder: Stall''\textgreater✲,

nicht Böcke aus deinen Hürden;

10denn mein ist alles Wild des Waldes,

das Getier auf meinen Bergen zu Tausenden.

11Ich kenne jeden Vogel auf den Bergen,

und was auf dem Felde sich regt, steht mir zur Verfügung.

12Hätte ich Hunger: ich brauchte es dir nicht zu sagen,

denn mein ist der Erdkreis und all seine Fülle.

13Esse ich etwa das Fleisch von Stieren,

und soll ich das Blut von Böcken trinken?

14Bringe Dank dem HERRN als Opfer dar

und bezahle so dem Höchsten deine Gelübde,

15und rufe mich an am Tage der Not,

so will ich dich retten, und du sollst mich preisen!«

16Zum Gottlosen aber spricht der Allherr:

»Was hast du meine Satzungen aufzuzählen und meinen Bund\textless sup
title=``=~mein Gesetz''\textgreater✲ im Munde zu führen,

17da du selbst doch die Zucht mißachtest

und meinen Worten den Rücken kehrst?{A}

18Siehst du einen Dieb, so befreundest du dich mit ihm,

und mit Ehebrechern hast du Gemeinschaft;

19deinem Munde läßt du freien Lauf zur Bosheit,

und deine Zunge zettelt Betrug an;

20du sitzest da und redest (Böses) gegen deinen Bruder

und bringst den Sohn deiner Mutter in Verruf!

21Das (alles) hast du getan, und ich habe geschwiegen;

da hast du gedacht, ich sei so wie du. Das mache ich dir zum Vorwurf und
gebe dir's zu bedenken.

22O beherzigt das wohl, ihr Gottvergeßnen:

sonst raffe ich euch hinweg ohne Rettung!

23Wer Dank als Opfer darbringt, erweist mir Ehre,

und wer unsträflich wandelt, den lasse ich schauen Gottes Heil.«

\hypertarget{davids-buuxdfgebet-vierter-buuxdfpsalm}{%
\subsubsection{Davids Bußgebet (Vierter
Bußpsalm)}\label{davids-buuxdfgebet-vierter-buuxdfpsalm}}

\hypertarget{section-50}{%
\section{51}\label{section-50}}

1Dem Musikmeister; ein Psalm von David, 2als der Prophet Nathan zu ihm
kam, nachdem er sich mit Bathseba vergangen hatte\textless sup
title=``2.Sam 12''\textgreater✲. 3Sei mir gnädig, o Gott, nach deiner
Güte!

Nach deinem großen Erbarmen tilge meine Vergehen!

4Wasche völlig mir ab meine Schuld

und mache mich rein von meiner Missetat!

5Ach, ich erkenne meine Vergehen wohl,

und meine Missetat steht mir immerdar vor Augen!

6Gegen dich allein hab' ich gesündigt

und habe getan, was böse ist in deinen Augen, auf daß du recht behältst
mit deinen Urteilssprüchen und rein dastehst mit deinem Richten.

7Ach, in Schuld bin ich geboren\textless sup title=``oder:
gezeugt''\textgreater✲,

und in Sünde hat meine Mutter mich empfangen.

8Du hast Gefallen an Wahrheit✲ im innersten Herzen,

und im Verborg'nen läßt du mich Weisheit erkennen.{A}

9Entsündige mich mit Ysop\textless sup title=``vgl. 2.Mose
12,22''\textgreater✲, daß ich rein werde,

wasche mich, daß ich weißer werde als Schnee.

10Laß mich (wieder) Freude und Wonne empfinden,

daß die Glieder frohlocken, die du zerschlagen.

11Verhülle dein Antlitz vor meinen Sünden

und tilge alle meine Missetaten!

12Schaffe mir, Gott, ein reines Herz

und stell' einen neuen, festen Geist in meinem Innern her!

13Verwirf mich nicht von deinem Angesicht

und nimm deinen heiligen Geist nicht weg von mir!

14Gib, daß ich deiner Hilfe\textless sup title=``oder: deines
Heils''\textgreater✲ mich wieder freue,

und rüste mich aus mit einem willigen Geist!

15Dann will ich die Übertreter deine Wege lehren,

und die Missetäter sollen zu dir sich bekehren.

16Errette mich von Blutschuld, o Gott, du Gott meines Heils,

damit meine Zunge deine Gerechtigkeit jubelnd preise!

17O Allherr, tu mir die Lippen auf,

damit mein Mund deinen Ruhm verkünde!

18Denn an Schlachtopfern hast du kein Gefallen,

und brächte ich Brandopfer dar: du möchtest sie nicht.

19Opfer, die Gott gefallen, sind ein zerbrochner Geist;

ein zerbrochnes und zerschlagnes Herz wirst du, o Gott, nicht
verschmähen.~--

20Tu doch Gutes an Zion nach deiner Gnade:

baue Jerusalems Mauern wieder auf!

21Dann wirst du auch Wohlgefallen haben an richtigen Opfern,

an Brand- und Ganzopfern; dann wird man Farren opfern auf deinem
Altar.{A}

\hypertarget{klage-uxfcber-einen-gewalttuxe4tigen-feind-und-frohlocken-uxfcber-die-guxf6ttliche-hilfe}{%
\subsubsection{Klage über einen gewalttätigen Feind und Frohlocken über
die göttliche
Hilfe}\label{klage-uxfcber-einen-gewalttuxe4tigen-feind-und-frohlocken-uxfcber-die-guxf6ttliche-hilfe}}

\hypertarget{section-51}{%
\section{52}\label{section-51}}

1Dem Musikmeister; ein Lehrgedicht✲ Davids, 2als der Edomiter Doeg kam
und dem Saul die Meldung brachte, David sei in das Haus Ahimelechs
gekommen\textless sup title=``1.Sam 21-22''\textgreater✲. 3Was rühmst du
dich der Bosheit, du Gewaltmensch?

Gottes Gnade währet alle Zeit.

4Auf Unheil sinnt deine Zunge

wie ein scharfes Schermesser, du Ränkeschmied!

5Du liebst das Böse mehr als das Gute,

sprichst lieber Lügen als Gerechtigkeit✲. SELA.

6Du liebst nur unheilvolle Reden,

du trügerische Zunge!

7So wird denn Gott dich auch vernichten für immer,

dich wegraffen und herausreißen aus dem Zelt, dich entwurzeln aus der
Lebenden Land! SELA.

8Die Gerechten werden es sehn und sich fürchten,

über ihn aber spotten:

9»Seht, das ist der Mann, der nicht Gott

zu seiner Schutzwehr machte, vielmehr sich verließ auf seinen großen
Reichtum und stark sich dünkte durch seine Bosheit!«

10Ich aber bin wie ein grünender Ölbaum im Hause Gottes,

ich vertraue auf Gottes Gnade immer und ewig.

11Preisen will ich dich immer, denn du hast's vollführt,

will rühmen deinen Namen, daß er so herrlich ist, vor deinen Frommen.

\hypertarget{gedanken-bei-der-allgemeinen-verderbtheit-der-welt-und-bitte-um-erluxf6sung-1}{%
\subsubsection{Gedanken bei der allgemeinen Verderbtheit der Welt und
Bitte um
Erlösung}\label{gedanken-bei-der-allgemeinen-verderbtheit-der-welt-und-bitte-um-erluxf6sung-1}}

\hypertarget{section-52}{%
\section{53}\label{section-52}}

1Dem Musikmeister, nach (der Singweise =~Melodie) »die Krankheit«; ein
Lehrgedicht\textless sup title=``vgl. 32,1''\textgreater✲ von David.
2Die Toren sprechen✲ in ihrem Herzen:

»Es gibt keinen Gott«; verderbt ist ihr Tun, abscheulich ihr Freveln: da
ist keiner, der Gutes täte.

3Gott schaut hernieder vom Himmel aus

nach den Menschenkindern, um zu sehn, ob da sei ein Verständiger, einer,
der nach Gott fragt.

4Doch alle sind sie abgefallen, insgesamt entartet;

da ist keiner, der Gutes tut, auch nicht einer.

5Haben denn keinen Verstand die Übeltäter,

die mein Volk verzehren? Die Gottes Brot wohl essen, doch ohne ihn
anzurufen?

6Damals\textless sup title=``vgl. 14,5''\textgreater✲ gerieten sie in
Angst und Schrecken,

ohne daß Grund zum Erschrecken war; denn Gott zerstreute die Gebeine
deiner Belagerer; du machtest sie zuschanden, denn Gott hatte sie
verworfen.{A}

7Ach, daß doch aus Zion die Rettung Israels käme!

Wenn Gott einst wendet das Schicksal seines Volkes, wird Jakob jubeln,
Israel sich freuen!

\hypertarget{hilferuf-gegen-gottlose-feinde}{%
\subsubsection{Hilferuf gegen gottlose
Feinde}\label{hilferuf-gegen-gottlose-feinde}}

\hypertarget{section-53}{%
\section{54}\label{section-53}}

1Dem Musikmeister, mit Saitenspiel; ein Lehrgedicht\textless sup
title=``vgl. 32,1''\textgreater✲ Davids, 2als die Siphiter kamen und zu
Saul sagten: »Weißt du nicht, daß David sich bei uns verborgen
hält?«\textless sup title=``1.Sam 23,19.26''\textgreater✲. 3Hilf mir, o
Gott, durch deinen Namen

und schaffe mir Recht durch deine Kraft!

4Höre, o Gott, mein Gebet,

vernimm die Worte meines Mundes!

5Denn Freche haben sich gegen mich erhoben,

und Gewalttätige trachten mir nach dem Leben: sie haben Gott sich nicht
vor Augen gestellt. SELA.

6Ich weiß: Gott ist mir ein Helfer,

der Allherr ist meiner Seele Stütze\textless sup title=``oder: meines
Lebens Halt''\textgreater✲.

7Das Böse wird auf meine Feinde zurückfallen:

nach deiner Treue vernichte sie!

8Dann will ich mit Freuden dir Opfer bringen,

will preisen deinen Namen, o HERR, daß er gütig ist;

9denn er hat mich aus aller Bedrängnis errettet,

und an meinen Feinden hat mein Auge sich geweidet.

\hypertarget{gebet-gegen-gottlose-feinde-und-klage-uxfcber-einen-treulosen-freund}{%
\subsubsection{Gebet gegen gottlose Feinde und Klage über einen
treulosen
Freund}\label{gebet-gegen-gottlose-feinde-und-klage-uxfcber-einen-treulosen-freund}}

\hypertarget{section-54}{%
\section{55}\label{section-54}}

1Dem Musikmeister, mit Saitenspiel, ein Lehrgedicht\textless sup
title=``vgl. 32,1''\textgreater✲ von David. 2Vernimm, o Gott, mein Gebet

und verbirg dich nicht vor meinem Flehen!

3Merke auf mich und erhöre mich!

Ich schwanke in meinem Kummer hin und her und stöhne

4ob dem Lärmen der Feinde,

ob dem Toben der Frevler; denn sie wälzen Unheil auf mich und befehden
mich wütend.

5Das Herz ängstigt sich mir in der Brust,

und die Schrecken des Todes haben mich befallen;

6Furcht und Zittern kommt mich an,

und ein Schauder überläuft mich.

7So ruf ich denn aus: »O hätt' ich doch Flügel wie die Taube!

Ich wollte fliegen, bis ich irgendwo Ruhe fände.«

8Ja weithin wollt ich entfliehen,

in der Wüste einen Rastort suchen; SELA.

9nach einem Zufluchtsorte für mich wollt' ich eilen

schneller als reißender Wind, als Sturm!{A}

10Vernichte\textless sup title=``oder: verwirre''\textgreater✲, Allherr,
entzweie ihre Zungen!

Denn ich sehe Gewalttat und Hader in der Stadt.

11Man macht bei Tag und bei Nacht die Runde um sie auf ihren Mauern,

Unheil und Elend herrschen in ihrer Mitte.

12Heilloses Treiben besteht in ihrem Innern,

und von ihrem Marktplatz weicht nicht Bedrückung und Trug.

13Denn\textless sup title=``oder: ach!''\textgreater✲ nicht ein Feind
ist's, der mich schmäht~--

das wollt' ich ertragen; nicht einer, der mich haßt, tut groß gegen
mich~-- ich würde vor ihm mich verbergen;

14nein, du bist's, ein Mann meinesgleichen,

mein Freund und trauter Bekannter,

15die wir innigen Verkehr miteinander pflegten,

zum Hause Gottes schritten im Festgetümmel.

16Möge der Tod sie ereilen,

mögen sie lebend zur Unterwelt fahren! Denn Bosheit herrscht in ihrer
Wohnstatt, in ihrem Herzen.

17Ich aber rufe zu Gott,

und der HERR wird mir helfen.

18Abends und morgens und mittags

will ich klagen und seufzen, so wird er mein Flehen vernehmen.

19Er wird meine Seele erlösen zum Frieden,

so daß sie nicht an mich können; denn ihrer sind viele gegen mich.

20Gott wird mich hören, wird sie demütigen (ihnen Antwort geben),

er, der von alters her auf dem Throne sitzt; SELA. sie wollen sich ja
nicht ändern und Gott nicht fürchten.

21Er\textless sup title=``d.h. der falsche Freund''\textgreater✲ hat die
Hand an seine Freunde gelegt,

hat seinen Bund entweiht✲.

22Glatt sind die Schmeichlerworte seines Mundes,

aber Krieg ist sein Sinnen; linder sind seine Reden als Öl, und sind
doch gezückte Schwerter.

23Wirf auf den HERRN deine Bürde:

er wird dich aufrecht erhalten; er läßt den Gerechten nicht ewig wanken.

24Ja du, Gott, wirst sie stürzen

in die Tiefe des Grabes;{A} die Männer des Blutvergießens und des Truges
werden ihre Tage nicht auf die Hälfte bringen. Ich aber vertraue auf
dich!

\hypertarget{getroster-mut-in-schwerer-bedruxe4ngnis}{%
\subsubsection{Getroster Mut in schwerer
Bedrängnis}\label{getroster-mut-in-schwerer-bedruxe4ngnis}}

\hypertarget{section-55}{%
\section{56}\label{section-55}}

1Dem Musikmeister, nach (der Singweise =~Melodie) »Die stumme Taube der
Ferne«; ein Lied\textless sup title=``vgl. 16,1''\textgreater✲ von
David, als die Philister ihn in Gath festgenommen hatten\textless sup
title=``1.Sam 21,11-16''\textgreater✲. 2Sei mir gnädig, o Gott, denn
Menschen stellen mir nach!

Immerfort bedrängen mich Krieger.

3Meine Feinde stellen mir immerfort nach,

ja viele sind's, die in Hochmut mich befehden.

4In Zeiten, da mir angst ist, vertrau ich auf dich! 5

Mit Gottes Hilfe werde sein Wort\textless sup title=``=~seine
Verheißung''\textgreater✲ ich rühmen. Auf Gott vertrau' ich, fürchte
mich nicht; was können Menschen mir antun?

6Allzeit suchen sie meiner Sache zu schaden;

gegen mich ist all ihr Sinnen gerichtet auf Böses.

7Sie rotten sich zusammen, lauern auf meine Schritte,

dieweil sie nach dem Leben mir trachten.

8Ob der Bosheit zahle ihnen heim,

im Zorn laß die Völker niedersinken, o Gott!

9Meines Elends Tage hast du gezählt,

meine Tränen in deinem Krüglein\textless sup title=``oder:
Schlauche''\textgreater✲ gesammelt; ja gewiß, sie stehen in deinem Buche
verzeichnet.

10So werden denn meine Feinde weichen, sobald (zu Gott) ich rufe;

dessen bin ich gewiß, daß Gott mir beisteht.

11Mit Gottes Hilfe werde sein Wort\textless sup title=``=~seine
Verheißung''\textgreater✲ ich rühmen,

mit Hilfe des HERRN werde sein Wort\textless sup title=``=~seine
Verheißung''\textgreater✲ ich rühmen.

12Auf Gott vertrau' ich, fürchte mich nicht:

was können Menschen mir antun?

13Mir obliegt es, dir, Gott, zu erfüllen meine Gelübde:

Dankopfer ich will dir entrichten;

14denn du hast meine Seele vom Tode errettet,

ja, meine Füße vom Straucheln, daß ich wandeln soll vor Gottes Angesicht
im Lichte der Lebenden\textless sup title=``oder: des
Lebens''\textgreater✲.

\hypertarget{zuversicht-zu-gott-inmitten-von-feinden}{%
\subsubsection{Zuversicht zu Gott inmitten von
Feinden}\label{zuversicht-zu-gott-inmitten-von-feinden}}

\hypertarget{section-56}{%
\section{57}\label{section-56}}

1Dem Musikmeister, nach (der Singweise =~Melodie) »Vertilge nicht«; ein
Lied\textless sup title=``vgl. 16,1''\textgreater✲ Davids, als er vor
Saul in die Höhle floh\textless sup title=``1.Sam 22,1-2;
24''\textgreater✲. 2Erbarme dich meiner, o Gott, erbarme dich meiner!

Denn bei dir sucht meine Seele Zuflucht, und im Schatten deiner Flügel
will ich mich bergen, bis das Verderben vorübergezogen.

3Ich rufe zu Gott, dem Höchsten,

zum Allherrn, der meine Sache hinausführt.

4Er sendet vom Himmel und hilft mir,

da der gierige Verfolger mich geschmäht hat! SELA. Es sendet Gott seine
Gnade und Treue!

5Mit meinem Leben liege ich mitten unter Löwen,

inmitten haßerfüllter Feinde, unter Menschen, deren Zähne Speere und
Pfeile und deren Zunge ein scharfes Schwert ist.

6Erhebe dich über den Himmel hinaus, o Gott,

über die ganze Erde (verbreite sich) deine Herrlichkeit!

7Sie haben meinen Füßen ein Netz gestellt:

meine Seele\textless sup title=``=~mein Mut''\textgreater✲ ist gebeugt;
eine Grube haben sie vor mir gegraben: sie selbst sind mitten
hineingestürzt. SELA.

8Mein Herz ist getrost, o Gott, mein Herz ist getrost;

singen will ich und spielen!

9Wach auf, meine Seele, wacht auf, Harfe und Zither:

ich will das Morgenrot wecken!

10Ich will dich preisen unter den Völkern, Allherr,

ich will dir lobsingen unter den Völkerschaften!

11Denn groß bis zum Himmel ist deine Gnade,

und bis an die Wolken geht deine Treue.

12Erhebe dich über den Himmel hinaus, o Gott,

über die ganze Erde (verbreite sich) deine Herrlichkeit!\textless sup
title=``vgl. 108,2-6''\textgreater✲

\hypertarget{gegen-ungerechte-richter-oder-herrscher}{%
\subsubsection{Gegen ungerechte Richter (oder
Herrscher)}\label{gegen-ungerechte-richter-oder-herrscher}}

\hypertarget{section-57}{%
\section{58}\label{section-57}}

1Dem Musikmeister, nach (der Singweise =~Melodie) »Vertilge nicht«; von
David ein Lied\textless sup title=``vgl. 16,1''\textgreater✲. 2Sprecht
in Wahrheit ihr Recht, ihr Götter{ (=~ihr Gewaltigen, ihr Machthaber auf
Erden; vgl. 82,1)}?

Richtet ihr die Menschen gerecht\textless sup title=``oder: in
gebührender Weise''\textgreater✲?

3Ach nein, im Herzen schmiedet ihr Frevel,

im Lande wägen eure Hände Gewalttat dar.

4Abtrünnig sind die Gottlosen schon von Geburt an,

schon vom Mutterleib an gehn die Lügenredner irre.

5Gift haben sie in sich wie Schlangengift,

sie gleichen der tauben Otter, die ihr Ohr verstopft,

6die nicht hört auf die Stimme der Beschwörer,

(auf die Stimme) des kundigen Bannspruchredners.

7Zerschmettre ihnen, Gott, die Zähne im Munde,

den jungen Löwen brich aus das Gebiß, o HERR!

8Laß sie vergehen wie Wasser, das sich verläuft!

Schießt er seine Pfeile ab: sie seien wie ohne Spitze!

9Wie die Schnecke beim Kriechen zerfließt, so muß er zergehn,

wie die Fehlgeburt eines Weibes, die das Licht nicht geschaut!

10Bevor noch eure Töpfe den (brennenden) Stechdorn spüren,

wird ihn, noch unverbrannt, die Zornglut hinwegstürmen.{A}

11Der Gerechte wird sich freun, daß er Rache erlebt,

seine Füße wird er baden im Blute des Frevlers,

12und die Menschen werden bekennen:

»Fürwahr, der Gerechte erntet noch Lohn! Fürwahr, noch gibt's einen
Gott, der auf Erden richtet!«

\hypertarget{hilferuf-eines-von-gewalttuxe4tigen-feinden-bedruxe4ngten}{%
\subsubsection{Hilferuf eines von gewalttätigen Feinden
Bedrängten}\label{hilferuf-eines-von-gewalttuxe4tigen-feinden-bedruxe4ngten}}

\hypertarget{section-58}{%
\section{59}\label{section-58}}

1Dem Musikmeister, nach (der Singweise =~Melodie) »Vertilge nicht«; ein
Lied\textless sup title=``vgl. 16,1''\textgreater✲ von David, als Saul
das Haus bewachen ließ, um ihn zu töten\textless sup title=``1.Sam
19,11-17''\textgreater✲. 2Rette mich von meinen Feinden, mein Gott!

Sei mir ein Schutz vor meinen Widersachern!

3Rette mich von den Übeltätern

und hilf mir gegen die Blutmenschen!

4Denn siehe, sie trachten mir nach dem Leben;

Starke rotten sich gegen mich zusammen ohne mein Verschulden, o HERR,
und ohne daß ich gefehlt.

5Gegen einen Schuldlosen stürmen sie an und stellen sich auf:

erwache, komm mir zu Hilfe und sieh darein!

6Ja, du, o HERR, Gott der Heerscharen, Israels Gott,

wache auf, um alle Heiden zu strafen! Verschone keinen der treulosen
Frevler! SELA.

7Jeden Abend kommen sie wieder, heulen wie Hunde

und streifen umher in der Stadt.

8Siehe, sie geifern mit ihrem Munde,

Schwerter stecken in ihren Lippen, denn (sie denken): »Wer hört es?«

9Doch du, o HERR, du lachest ihrer,

spottest aller Heiden.

10Meine Stärke, deiner will ich harren,

denn Gott ist meine feste Burg.

11Mein Gott kommt mir entgegen mit seiner Gnade;

Gott läßt meine Lust mich sehn an meinen Feinden.

12Töte sie nicht, daß mein Volk sie nicht vergesse!

Treibe sie in die Irre durch deine Macht und stürze sie nieder, du,
unser Schild, o Allherr!

13Sündhaft ist ihr Mund, das Wort ihrer Lippen;

drum laß sie sich fangen in ihrem Hochmut wegen der Flüche und Lügen,
die sie reden!

14Vertilge sie im Zorn, vertilge sie, daß sie nicht mehr sind!

Laß sie inne werden, daß Gott in Jakob herrscht, bis an die Enden der
Erde! SELA.

15Jeden Abend kommen sie wieder, heulen wie Hunde

und streifen umher in der Stadt;

16sie schweifen umher nach Fraß

und knurren, sind sie nicht satt geworden.

17Ich aber will deine Stärke\textless sup title=``oder:
Macht''\textgreater✲ besingen

und am Morgen ob deiner Gnade jubeln; denn du bist eine feste Burg für
mich gewesen, eine Zuflucht zur Zeit meiner Drangsal.

18Meine Stärke, dir will ich lobsingen!

Denn Gott ist meine feste Burg, der Gott, der mir Gnade erweist.

\hypertarget{gebet-nach-schwerer-niederlage-im-kriege}{%
\subsubsection{Gebet nach schwerer Niederlage im
Kriege}\label{gebet-nach-schwerer-niederlage-im-kriege}}

\hypertarget{section-59}{%
\section{60}\label{section-59}}

1Dem Musikmeister, nach (der Singweise =~Melodie) »Lilie des
Zeugnisses«; ein Lied\textless sup title=``vgl. 16,1''\textgreater✲
Davids zum Lehren, 2als er mit den Syrern von Mesopotamien und mit den
Syrern von Zoba Krieg führte und Joab zurückkehrte und die Edomiter im
Salztal schlug, zwölftausend Mann.\textless sup title=``2.Sam 8,3-14;
1.Chr 18,3-13''\textgreater✲. 3Gott, du hast uns verworfen, uns
zersprengt,

du hast (uns) gezürnt: stelle uns wieder her!{A}

4Du hast das Land\textless sup title=``oder: die Erde''\textgreater✲
erschüttert, hast es zerrissen:

o heile seine Risse, denn es wankt!

5Dein Volk hast du Hartes erleben lassen,

hast Taumelwein uns zu trinken gegeben;

6doch deinen Getreuen hast eine Flagge\textless sup title=``oder: ein
Panier''\textgreater✲ du wehen lassen,

damit sie sich flüchten konnten vor dem Bogen (des Feindes). SELA.

7Daß deine Geliebten\textless sup title=``oder: Freunde''\textgreater✲
gerettet werden,

hilf uns mit deiner Rechten, erhöre uns!

8Gott hat in\textless sup title=``oder: bei''\textgreater✲ seiner
Heiligkeit gesprochen✲:

»(Als Sieger) will ich frohlocken, will Sichem verteilen und das Tal von
Sukkoth (als Beutestück) vermessen.

9Mein ist Gilead, mein auch Manasse,

und Ephraim ist meines Hauptes Schutzwehr, Juda mein Herrscherstab.

10Moab (dagegen) ist mein Waschbecken,

auf Edom werfe ich meinen Schuh; jauchze mir zu, Philisterland!«

11Wer führt mich hin zur festen Stadt,

wer geleitet mich bis Edom?

12Hast nicht du uns, o Gott, verworfen

und ziehst nicht aus, o Gott, mit unsern Heeren?

13O schaff uns Hilfe gegen den Feind!

denn nichtig ist Menschenhilfe.

14Mit Gott werden wir Taten vollführen,

und er wird unsre Bedränger zertreten.

\hypertarget{fuxfcrbitte-fuxfcr-den-kuxf6nig-aus-der-ferne}{%
\subsubsection{Fürbitte für den König aus der
Ferne}\label{fuxfcrbitte-fuxfcr-den-kuxf6nig-aus-der-ferne}}

\hypertarget{section-60}{%
\section{61}\label{section-60}}

1Dem Musikmeister über das Saitenspiel; von David. 2Höre, o Gott, mein
lautes Rufen,

achte auf mein Gebet!

3Vom Ende der Erde\textless sup title=``oder: des Landes''\textgreater✲
ruf' ich zu dir,

da mein Herz verschmachtet\textless sup title=``=~vor Angst
vergeht''\textgreater✲. Auf einen Felsen, der mir zu hoch ist, wollest
du mich führen!

4Denn du bist mir stets eine Zuflucht gewesen,

ein starker Turm vor dem Feinde.

5Könnt' ich doch allzeit weilen in deinem Zelt,

im Schutze deiner Flügel mich bergen! SELA.

6Denn du, Gott, hörst auf meine Gelübde,

hast Besitz (mir) gewährt, wie die ihn erhalten, die deinen Namen
fürchten.{A}

7Füge neue Tage den Tagen des Königs hinzu,

laß seine Jahre dauern für und für!

8Ewig\textless sup title=``=~noch lange''\textgreater✲ möge er thronen
vor Gottes Angesicht!

Entbiete Gnade und Treue, daß sie ihn behüten!

9Dafür will ich ewig deinem Namen lobsingen,

auf daß\textless sup title=``oder: indem''\textgreater✲ ich meine
Gelübde bezahle✲ Tag für Tag.

\hypertarget{stille-in-gott-nichtigkeit-der-menschen}{%
\subsubsection{Stille in Gott! Nichtigkeit der
Menschen}\label{stille-in-gott-nichtigkeit-der-menschen}}

\hypertarget{section-61}{%
\section{62}\label{section-61}}

1Dem Musikmeister über die Jeduthuniden; ein Psalm Davids. 2Nur (im
Aufblick) zu Gott ist meine Seele still:

von ihm kommt meine Hilfe\textless sup title=``oder:
Rettung''\textgreater✲;

3nur er ist mein Fels und meine Hilfe,

meine Burg: ich werde nicht allzusehr wanken.

4Wie lange noch lauft ihr Sturm gegen einen Mann,

wollt ihn niederschlagen allesamt wie eine sinkende Wand, eine dem
Einsturz nahe Mauer?

5Ja, von seiner Höhe planen sie ihn zu stoßen; Lügen sind ihre Lust;

mit dem Munde segnen sie, doch im Herzen fluchen sie. SELA.

6Nur (im Aufblick) zu Gott sei still, meine Seele!

Denn von ihm kommt meine Hoffnung;

7nur er ist mein Fels und meine Hilfe\textless sup title=``oder:
Rettung''\textgreater✲,

meine Burg: ich werde nicht wanken.

8Auf Gott beruht mein Heil und meine Ehre;

mein starker Fels, meine Zuflucht liegt in Gott.

9Vertraut auf ihn zu aller Zeit, ihr Volksgenossen,

schüttet vor ihm euer Herz aus: Gott ist unsere Zuflucht. SELA.

10Nur ein Hauch sind Menschensöhne, ein Trug sind Herrensöhne;

auf der Waage schnellen sie empor, sind allesamt leichter als ein Hauch.

11Verlaßt euch nicht auf Erpressung\textless sup title=``=~unrecht
Gut''\textgreater✲

und setzt nicht eitle Hoffnung auf Raub; wenn der Reichtum sich mehrt,
so hängt das Herz nicht daran!

12Eins ist's, was Gott gesprochen, und zweierlei ist's, was ich
vernommen,

daß die Macht bei Gott steht.

13Und bei dir, o Allherr, steht auch die Gnade:

ja, du vergiltst einem jeden nach seinem Tun.

\hypertarget{sehnsucht-nach-gott-dem-labsal-der-seele-und-huxf6chsten-gut}{%
\subsubsection{Sehnsucht nach Gott, dem Labsal der Seele und höchsten
Gut}\label{sehnsucht-nach-gott-dem-labsal-der-seele-und-huxf6chsten-gut}}

\hypertarget{section-62}{%
\section{63}\label{section-62}}

1Ein Psalm Davids, als er in der Wüste Juda war\textless sup
title=``2.Sam 15-17''\textgreater✲. 2O Gott, du bist mein Gott: dich
suche ich,

es dürstet nach dir meine Seele; es lechzt nach dir mein Leib wie
dürres, schmachtendes, wasserloses Land.{A}

3So hab' ich nach dir im Heiligtum ausgeschaut,

um deine Macht und Herrlichkeit zu erblicken;

4denn deine Gnade ist besser als das Leben:

meine Lippen sollen dich rühmen.

5So will ich dich preisen mein Leben lang,

in deinem Namen meine Hände erheben.

6Wie an Mark und Fett ersättigt sich meine Seele,

und mit jubelnden Lippen lobpreist mein Mund,

7so oft ich deiner gedenke auf meinem Lager,

in den Stunden der Nacht{A} über dich sinne;

8denn du bist mir ein Helfer gewesen,

und im Schatten deiner Flügel darf ich jubeln.

9Meine Seele klammert sich an dich,

aufrecht hält mich deine rechte Hand.

10Doch sie, die nach dem Leben mir trachten, mich zu verderben,

sie werden in der Erde unterste Tiefen fahren.

11Man wird sie der Schärfe des Schwerts überliefern;

die Beute der Schakale werden sie sein.

12Der König dagegen wird Gottes sich freuen:

Ruhm wird ernten ein jeder, der bei ihm✲ schwört; den Lügnern dagegen
wird der Mund gestopft werden.

\hypertarget{bitte-um-schutz-gegen-buxf6swillige-feinde}{%
\subsubsection{Bitte um Schutz gegen böswillige
Feinde}\label{bitte-um-schutz-gegen-buxf6swillige-feinde}}

\hypertarget{section-63}{%
\section{64}\label{section-63}}

1Dem Musikmeister, ein Psalm Davids. 2Höre, o Gott, meine Stimme, wenn
ich klage,

vor dem Feinde, der mich schreckt, behüte mein Leben!

3Schirme mich vor den Plänen der bösen Buben,

vor der lärmenden Rotte der Übeltäter,

4die ihre Zunge schärfen wie ein Schwert,

giftige Worte als ihre Pfeile auf den Bogen legen,

5um im Versteck auf den Frommen zu schießen:

unversehens schießen sie auf ihn ohne Scheu.

6Sie ermutigen sich zu bösem Anschlag,

verabreden, heimlich Schlingen zu legen; sie denken: »Wer wird sie
sehen?«

7Sie sinnen auf Freveltaten:

»Wir sind fertig, der Plan ist fein erdacht!« Und das Innere jedes
Menschen und das Herz sind unergründlich.

8Da trifft sie Gott mit dem Pfeil,

urplötzlich fühlen sie sich verwundet:

9ihre eigene Zunge hat sie zu Fall gebracht;

alle, die sie sehen, schütteln das Haupt.

10Da fürchten sich alle Menschen

und bekennen: »Das hat Gott getan!« und erwägen sein Walten.

11Der Gerechte freut sich des HERRN

und nimmt seine Zuflucht zu ihm, und alle redlichen Herzen preisen sich
glücklich.

\hypertarget{danklied-fuxfcr-geistliche-wohltaten-gottes-und-fuxfcr-erntesegen}{%
\subsubsection{Danklied für geistliche Wohltaten Gottes und für
Erntesegen}\label{danklied-fuxfcr-geistliche-wohltaten-gottes-und-fuxfcr-erntesegen}}

\hypertarget{section-64}{%
\section{65}\label{section-64}}

1Dem Musikmeister; ein Psalm Davids, ein Lied. 2Dir gebührt Lobpreis, o
Gott, in Zion,

und dir muß man Gelübde bezahlen✲,

3der du Gebete erhörst: zu dir kommt alles Fleisch 4

um der Verschuldungen willen. Wenn uns unsere Sünden zu drückend werden,
du, HERR, vergibst sie.

5Wohl dem, den du erwählst und zu dir nahen läßt,

daß er in deinen Vorhöfen weilen darf! Wir wollen reichlich uns laben am
Segen deines Hauses, deines heiligen Tempels!

6Durch Wundertaten erhörst du uns in Gerechtigkeit,

du Gott unsers Heils, du Zuversicht aller Enden der Erde und der
fernsten Meere\textless sup title=``oder: Gestade''\textgreater✲,

7der da feststellt die Berge durch seine Kraft,

umgürtet mit Stärke,

8der da stillt das Brausen der Meere,

das Brausen ihrer Wogen und das Toben der Völker,

9so daß die Bewohner der Enden (des Erdrunds) sich fürchten

vor deinen Zeichen; die fernsten Länder des Ostens und Westens bringst
du zu lautem Jauchzen.~--

10Du hast das Land gesegnet, es strömt schier über;

du hast es gar reich gemacht~-- der Gottesbach hat Wasser in Fülle
gehabt --; du hast ihre Feldfrucht wohl geraten lassen, denn also hast
du das Land instand gesetzt;

11du hast seine Furchen getränkt, seine Schollen geebnet\textless sup
title=``oder: gelockert''\textgreater✲,

durch Regen es weich gemacht, sein Gewächs gesegnet.

12Du hast das Jahr gekrönt mit deiner Güte,

und deine Spuren✲ triefen von Fett.

13Es triefen die Anger der Steppe,

und mit Jubel umgürten sich die Hügel;

14die Auen bekleiden sich mit Herden,

und die Täler hüllen sich in Korn: man jauchzt einander zu und singt.

\hypertarget{danklied-des-volkes-fuxfcr-wunderbare-fuxfchrung-und-errettung}{%
\subsubsection{Danklied des Volkes für wunderbare Führung und
Errettung}\label{danklied-des-volkes-fuxfcr-wunderbare-fuxfchrung-und-errettung}}

\hypertarget{section-65}{%
\section{66}\label{section-65}}

1Dem Musikmeister; ein Lied, ein Psalm. Jauchzet Gott, ihr Lande✲ alle!
2Lobsinget der Ehre seines Namens,

macht seinen Lobpreis herrlich!

3Sprechet zu Gott: »Wie wunderbar ist dein Walten!

Ob der Fülle deiner Macht huldigen dir sogar deine Feinde.

4Alle Lande müssen vor dir sich niederwerfen und dir lobsingen,

lobsingen deinem Namen!« SELA.

5Kommt und schauet die Großtaten Gottes,

der wunderbar ist im Walten über den Menschenkindern!

6Er wandelte das Meer in trocknes Land,

so daß man den Strom zu Fuß durchzog; drum wollen wir uns freun!

7Ewig herrscht er in seiner Macht;

seine Augen haben acht auf die Völker: die Widerspenstigen dürfen sich
nicht stolz erheben. SELA.

8Preiset, ihr Völker, unsern Gott,

laßt laut seinen Ruhm erschallen,

9ihn, der unsre Seele am Leben erhalten

und unsern Fuß nicht hat wanken lassen.

10Wohl hast du uns geprüft, o Gott,

uns geläutert, wie man Silber läutert;{A}

11du hast uns ins Netz geraten lassen,

hast drückende Last auf unsern Rücken gelegt;

12Menschen hast du hinfahren lassen über unser Haupt,

durch Feuer und Wasser haben wir ziehen müssen: doch endlich hast du uns
ins Freie hinausgeführt.

13Ich komme mit Brandopfern in dein Haus,

entrichte dir meine Gelübde,

14zu denen meine Lippen sich verpflichtet haben,

und die mein Mund verheißen in meiner Not.

15Brandopfer von Mastvieh will ich dir bringen

samt dem Opferduft von Widdern; Rinder samt Böcken will ich zubereiten.
SELA.

16Kommt her und höret, ihr Gottesfürchtigen alle:

ich will erzählen, was er an meiner Seele getan!

17Zu ihm hab' ich laut mit meinem Munde gerufen,

während Lobpreis schon auf meiner Zunge lag.

18Wäre mein Sinn auf Böses gerichtet gewesen,

so hätte der Allherr mich nicht erhört.

19Aber Gott hat mich erhört,

hat geachtet auf mein lautes Flehen.

20Gepriesen sei Gott, der mein Flehen nicht verworfen

und seine Gnade mir nicht versagt hat!

\hypertarget{gott-segne-israel-erntedanklied}{%
\subsubsection{Gott segne Israel!
(Erntedanklied)}\label{gott-segne-israel-erntedanklied}}

\hypertarget{section-66}{%
\section{67}\label{section-66}}

1Dem Musikmeister, mit Saitenspiel; ein Psalm, ein Lied. 2Gott sei uns
gnädig und segne uns!

Er lasse sein Angesicht bei uns leuchten, SELA,

3daß man auf Erden dein Walten erkenne,

unter allen Heidenvölkern dein Heil!{A}

4Preisen müssen dich, Gott, die Völker,

preisen die Völker allesamt;

5sich freuen müssen die Völkerschaften und jubeln,

weil du die Völker gerecht richtest\textless sup title=``oder:
regierst''\textgreater✲ und leitest die Völkerschaften auf Erden. SELA.

6Preisen müssen dich, Gott, die Völker,

preisen die Völker allesamt!

7Das Land hat seinen Ertrag gespendet:

gesegnet hat uns Gott, unser Gott.

8Es segne uns Gott,

und fürchten müssen ihn alle Enden der Erde!

\hypertarget{der-sieg-des-gottes-israels-uxfcber-seine-feinde}{%
\subsubsection{Der Sieg des Gottes Israels über seine
Feinde}\label{der-sieg-des-gottes-israels-uxfcber-seine-feinde}}

\hypertarget{section-67}{%
\section{68}\label{section-67}}

1Dem Musikmeister, von David ein Psalm, ein Lied. 2Gott steht auf: da
zerstieben seine Feinde,

und die ihn hassen, fliehen vor seinem Angesicht\textless sup
title=``4.Mose 10,35''\textgreater✲.

3Wie Rauch verweht, so verwehst✲ du sie;

wie Wachs zerschmilzt vor dem Feuer, so kommen die Gottlosen um vor
Gottes Angesicht;

4die Gerechten aber freuen sich, jubeln vor Gottes Angesicht

und frohlocken voller Freude.

5Singet Gott, lobsingt seinem Namen,

machet Bahn ihm, der durch Wüsten einherfährt

›HERR‹ ist sein Name --, und jauchzet vor seinem Angesicht!

6Ein Vater der Waisen, ein Richter der Witwen

ist Gott in seiner heiligen Wohnstatt.

7Gott verhilft den Vereinsamten zum Hausstand,

führt Gefangne heraus zum Wohlergehn; doch Widerstrebende müssen wohnen
in dürrem Land.

8Als du auszogst, Gott, an der Spitze deines Volkes,

einherschrittest durch die Wüste: (SELA)

9da bebte die Erde, da troffen die Himmel vor Gottes Angesicht,

der Sinai dort vor dem Angesicht Gottes, des Gottes Israels.

10Regen in Fülle ließest du strömen, o Gott;

dein Eigentumsvolk, sooft ermattet es war: du machtest es wieder stark.

11Deine Herde\textless sup title=``oder: Schar''\textgreater✲ fand
Wohnung darin\textless sup title=``=~im Lande''\textgreater✲,

durch deine Güte stelltest du, Gott, die Schwachen wieder her.

12Der Allherr ließ Siegesruf erschallen

der Siegesbotinnen war eine große Schar --:

13»Die Könige der Heere fliehen, sie fliehn,

und die Hausfrau teilt die Beute aus.«

14»Wollt ihr zwischen den Hürden liegen bleiben?«~--

»Die Flügel der Tauben, mit Silber überzogen, und ihr Gefieder gelblich
schimmernd von Gold.«~--

15»Als Könige dort der Allmächtge zerstreute,

da fand ein Schneegestöber statt auf dem Zalmon.«{A}

16Du Gottesberg, Basansberg,

du gipfelreicher Berg, Basansberg:

17warum blickt ihr neidisch, ihr gipfelreichen Berge,

auf den Berg, den Gott zum Wohnsitz erkoren? Ja, ewig wird der HERR dort
wohnen\textless sup title=``oder: thronen''\textgreater✲.

18Der Kriegswagen Gottes sind vieltausendmal tausend;

der Allherr ist unter ihnen, ein Sinai an Heiligkeit.{A}

19Du bist zur Höhe aufgefahren, hast Gefangene weggeführt,

hast Gaben unter den Menschen angenommen\textless sup title=``oder:
empfangen''\textgreater✲; ja auch die Widerstrebenden müssen wohnen bei
Gott dem HERRN.

20Gepriesen sei der Allherr! Tag für Tag!

Uns trägt der Gott, der unsre Hilfe ist. SELA.

21Dieser Gott ist uns ein rettender Gott,

und Gott der HERR weiß Rat auch gegen den Tod.{A}

22Ja, Gott zerschmettert das Haupt seiner Feinde,

den Haarscheitel dessen, der in seinen Sünden einhergeht.

23Der Allherr hat verheißen: »Aus Basan bring' ich (sie) heim,

ja bringe (sie) heim aus den Tiefen des Meeres,

24auf daß du in Blut deine Füße badest

und die Zunge deiner Hunde an den Feinden sich letze.«~--

25Man hat, o Gott, deinen Festzug gesehn,

den Festzug meines Gottes, meines Königs, im Heiligtum:

26An der Spitze zogen Sänger, dahinter Saitenspieler

inmitten paukenschlagender Jungfrauen:

27»In Versammlungen\textless sup title=``oder: vollen
Chören''\textgreater✲ preiset Gott,

den Allherrn, ihr aus Israels Born\textless sup title=``oder:
Lebensquell''\textgreater✲!«

28Dort schritt Benjamin hin, der Jüngste, der sie doch beherrscht hat,

die Fürsten Judas nach ihrer großen Menge, Sebulons Fürsten, die Fürsten
von Naphthali.

29Entbiete, o Gott, deine Macht,

erhalte in Kraft, o Gott, was du uns erwirkt hast!

30Um deines Tempels willen müssen Könige dir

Geschenke hinauf nach Jerusalem bringen.{A}

31Bedrohe das Tier des Schilfrohrs,

die Rotte der Stiere\textless sup title=``d.h. der
Großmächte''\textgreater✲ samt den Völkerkälbern, die mit Silberbarren
sich unterwerfen; zerstreue die Völker, die Freude an Kriegen haben!{B}

32Kommen werden die Edlen aus Ägypten,

Äthiopien eilt mit vollen Händen Gott entgegen.

33Ihr Königreiche der Erde, singet Gott,

lobsinget dem Allherrn, SELA,

34ihm, der einherfährt im höchsten\textless sup title=``oder:
innersten''\textgreater✲ Himmel der Urzeit!

Horch! Er läßt seine Stimme erschallen, den rollenden Donner!

35Gebt Gott die Macht\textless sup title=``=~die Ehre''\textgreater✲!
Über Israel waltet seine Hoheit

und seine Macht in den Wolken.

36Furchtbar bist du, Gott, von deinem Heiligtum aus!

Israels Gott, er ist's, der Macht verleiht und Stärke seinem Volk:
gepriesen sei Gott!

\hypertarget{gebet-eines-frommen-um-errettung-aus-schmach-und-schwerer-bedruxe4ngnis}{%
\subsubsection{Gebet eines Frommen um Errettung aus Schmach und schwerer
Bedrängnis}\label{gebet-eines-frommen-um-errettung-aus-schmach-und-schwerer-bedruxe4ngnis}}

\hypertarget{section-68}{%
\section{69}\label{section-68}}

1Dem Musikmeister, nach (der Singweise =~Melodie) »Lilien«, von David.
2Hilf mir, o Gott,

denn die Wasser gehen mir bis ans Leben!

3Ich versinke im tiefen Schlamm, wo kein Grund ist;

in Wassertiefen bin ich geraten, und die Flut überströmt mich.

4Müde bin ich von (allem) Schreien, meine Kehle ist heiser;

erloschen sind mir die Augen, während ich harre auf meinen Gott.

5Größer als die Zahl der Haare auf meinem Haupt

ist die Zahl derer, die ohne Ursach' mich hassen; mächtig sind meine
Gegner, die ohne Grund mich befeinden: wo ich gar nicht geraubt, da soll
Ersatz ich leisten!

6Du, o Gott, du weißt um meine Torheit\textless sup title=``oder:
Verfehlung''\textgreater✲,

und meine Vergehen sind dir nicht verborgen.

7Laß nicht enttäuscht an mir\textless sup title=``oder: durch
mich''\textgreater✲ werden, die auf dich hoffen,

o Gott, o HERR der Heerscharen! Laß nicht beschämt an mir\textless sup
title=``oder: durch mich''\textgreater✲ werden, die dich, Gott Israels,
suchen!

8Denn um deinetwillen trage ich Schmach,

(für dich) bedeckt Beschämung mein Antlitz;

9ein Fremdling bin ich meinen Brüdern geworden

und unbekannt den Söhnen meiner Mutter.

10Denn der Eifer um dein Haus hat mich verzehrt,

und die Schmähungen derer, die dich schmähen, haben mich getroffen.

11Ich weinte und kasteite✲ mich durch Fasten,

doch es brachte mir nur Beschimpfung ein;

12als ein Trauergewand zu meinem Kleid ich machte,

da wurde ich ihnen zum Spottlied;

13es schwatzten von mir die Leute auf dem Markt,

und Schmachlieder sangen von mir die Zecher beim Wein.

14Ich aber richte mein Gebet an dich, o HERR,

zur Zeit, da dir es wohlgefällig ist; o Gott, nach deiner großen Gnade
erhöre mich, nach deiner heilspendenden Treue!

15Zieh mich heraus aus dem Schlamm, daß ich nicht versinke,

laß Rettung mich finden von meinen Hassern und aus den Wassertiefen!

16Laß die Wasserflut mich nicht überströmen

und die Tiefe\textless sup title=``oder: den Strudel''\textgreater✲ mich
nicht verschlingen und den Abgrund seinen Schlund nicht über mir
schließen!

17Erhöre mich, HERR, denn deine Güte ist köstlich!

Nach deinem großen Erbarmen wende dich mir zu

18und verbirg dein Angesicht nicht vor deinem Knecht,

denn ich bin in Not: erhöre mich eiligst!

19Nahe dich meiner Seele, erlöse sie,

um meiner Feinde willen mach mich frei!

20Du weißt um meine Schmach,

um meine Schande und Beschimpfung; meine Feinde sind alle dir
wohlbekannt.

21Die Schmach hat mir das Herz gebrochen,

so daß ich verzweifle; ich hoffte auf Mitleid, aber vergebens, und auf
Tröster, doch ich habe keine gefunden;

22nein, sie haben mir Gift in die Speise getan

und Essig mich trinken lassen für meinen Durst.

23Möge ihr Tisch vor ihnen zum Fangnetz werden

und ihnen, den Sichren,{A} zum Fallstrick!

24Laß ihre Augen dunkel werden, daß sie nicht sehen,

und ihre Hüften laß immerdar wanken!

25Gieße über sie deinen Grimm aus,

und deines Zornes Glut erreiche sie!

26Ihre Behausung müsse zur Öde werden,

in ihren Zelten kein Bewohner sein!

27Denn den du selbst geschlagen hast, verfolgen sie,

und vom Weh der durch dich Verwundeten schwatzen sie.

28Füge noch Schuld zu ihrer Verschuldung hinzu

und laß sie nicht kommen zur Gerechtigkeit vor dir!

29Sie müssen ausgelöscht werden aus dem Buche des Lebens

und nicht eingeschrieben werden mit den Gerechten!

30Doch ich bin elend und schmerzbeladen:

deine Hilfe, Gott, möge mich sicherstellen!

31Ich will den Namen Gottes preisen in Liedern,

will hoch ihn rühmen mit Danksagung;

32das wird dem HERRN willkommner sein als Rinder,

als Farren mit Hörnern und gespaltnen Hufen.

33Wenn die Bedrückten es sehen, so werden sie sich freuen:

ihr, die ihr Gott sucht: euer Herz lebe auf!

34Denn der HERR erhört die Armen,

und seine Gefangenen läßt er nicht unbeachtet.

35Es mögen ihn preisen Himmel und Erde,

die Meere und alles, was in ihnen sich regt!

36Denn Gott wird Zion retten

und Judas Städte wieder erbaun, daß man daselbst wohne und das Land
besitze;

37und der Nachwuchs seiner Knechte wird es erben,

und die seinen Namen lieben, werden darin wohnen.

\hypertarget{bitte-um-hilfe-in-verfolgung}{%
\subsubsection{Bitte um Hilfe in
Verfolgung}\label{bitte-um-hilfe-in-verfolgung}}

\hypertarget{section-69}{%
\section{70}\label{section-69}}

1Dem Musikmeister, von David; bei Darbringung des
Duftopfers\textless sup title=``vgl. 38,1''\textgreater✲. 2Gott, eile zu
meiner Rettung,

HERR, eile zu meiner Hilfe herbei!

3Laß alle beschämt und schamrot werden,

die nach dem Leben mir stehn (um es auszutilgen)! Laß mit Schande
beladen abziehn, die mein Unglück wünschen!

4Laß zurück sich wenden ob ihrer Schmach,

die über mich rufen: »Haha, haha!«

5Laß jubeln und deiner sich freuen

alle, die dich suchen! Laß alle, die nach deinem Heil verlangen,
immerdar bekennen: »Groß ist Gott!«

6Doch ich bin elend und arm:

o Gott, eile zu mir! Meine Hilfe und mein Retter bist du: o HERR, säume
nicht!

\hypertarget{verlauxdf-mich-nicht-im-alter}{%
\subsubsection{Verlaß mich nicht im
Alter!}\label{verlauxdf-mich-nicht-im-alter}}

\hypertarget{section-70}{%
\section{71}\label{section-70}}

1Bei dir, HERR, suche ich Zuflucht:

laß mich nimmermehr enttäuscht werden!

2Nach deiner Gerechtigkeit rette und befreie mich,

neige dein Ohr mir zu und hilf mir!

3Sei mir ein schützender Fels, zu dem ich allzeit fliehen kann;

du hast ja geboten, mich zu retten, denn mein Fels und meine Burg bist
du.

4Mein Gott, errette mich aus des Gottlosen Hand,

aus der Faust des Frevlers und Gewaltmenschen!

5Denn du bist meine Hoffnung, HERR, mein Gott,

du meine Zuversicht von Jugend an.

6Auf dich hab' ich mich gestützt seit meiner Geburt;

aus dem Mutterschoß hast du mich ans Licht gezogen: dir hat mein
Lobpreis immer gegolten.

7Wie ein Wunder\textless sup title=``oder: Schreckzeichen''\textgreater✲
komme ich vielen vor,

doch du bist meine starke Zuflucht.

8Mein Mund ist deines Ruhmes voll,

allzeit voll von deiner Verherrlichung.

9Verwirf mich nicht in den Tagen des Alters,

beim Schwinden meiner Kraft verlaß mich nicht!

10Denn schon verhandeln meine Feinde über mich,

und die den Tod mir wünschen, beraten sich zusammen

11und sagen: »Gott hat ihn verlassen:

verfolgt und ergreift ihn, denn er hat keinen Retter!«

12O Gott, bleib du nicht fern von mir,

mein Gott, eil' mir zu Hilfe!

13Es müssen enttäuscht und vernichtet werden, die mich befeinden!

Laß alle in Schmach und Schande sich hüllen, die mein Unglück suchen!

14Ich aber will immerdar harren

und all deinen Ruhm noch mehren.

15Mein Mund soll deine Gerechtigkeit künden, allzeit deine Heilserweise,

denn ich vermag sie nicht zu zählen.

16Kommen will ich mit den Machttaten Gottes des HERRN,

will preisen deine Gerechtigkeit, dich allein.

17Du hast mich, o Gott, von Jugend auf gelehrt,

und bis hierher habe ich deine Wunder verkündet;

18doch auch bis zum Greisenalter und grauen Haar

verlaß mich nicht, o Gott, auf daß ich deinen Arm\textless sup
title=``=~deine Taten''\textgreater✲ verkünde den Zeitgenossen und
allen, die noch kommen werden, deine Macht\textless sup title=``oder:
Kraft''\textgreater✲.

19Gott, deine Gerechtigkeit reicht bis hoch an den Himmel;

der du große Dinge getan, o Gott, wer ist dir gleich?

20Du hast viel Not und Leid uns fühlen lassen:

du wirst uns auch wieder beleben und aus den Tiefen der Erde empor uns
führen.

21Du wirst mich um so höher erheben

und mit Trost dich wieder zu mir wenden.

22So will denn auch ich dich preisen mit Saitenspiel,

für deine Treue dir danken, mein Gott; ich will auf der Zither dir
spielen, du Heiliger Israels.

23Jubeln sollen meine Lippen, wenn ich dir spiele,

und zugleich meine Seele, die du erlöst hast;

24auch meine Zunge soll allezeit von deiner Gerechtigkeit reden,

denn enttäuscht, denn schamrot sind geworden, die mein Unglück suchten.

\hypertarget{segenswuxfcnsche-fuxfcr-den-kuxf6nig}{%
\subsubsection{Segenswünsche für den
König}\label{segenswuxfcnsche-fuxfcr-den-kuxf6nig}}

\hypertarget{section-71}{%
\section{72}\label{section-71}}

1Von Salomo. Gott, dein richterlich Walten verleihe dem König

und deine Gerechtigkeit dem Königssohn,

2daß er dein Volk mit Gerechtigkeit richte

und deine Elenden\textless sup title=``oder: Bedrückten''\textgreater✲
nach dem Recht!

3Laß die Berge dem Volke Frieden\textless sup title=``oder:
Heil''\textgreater✲ tragen

und die Hügel sich kleiden in Gerechtigkeit!

4Er schaffe Recht den Elenden\textless sup title=``oder:
Bedrückten''\textgreater✲ im Volk,

er helfe den armen Leuten und zertrete den Bedrücker.{A}

5Möge er leben, solange die Sonne scheint

und der Mond (uns leuchtet), von Geschlecht zu Geschlecht!

6Er möge sein wie Regen für frischgemähte Wiesen,

wie Regenschauer, die das Land besprengen!

7In seinen Tagen möge der Gerechte blühen

und Friede\textless sup title=``oder: Heil''\textgreater✲ in Fülle
bestehn, bis kein Mond mehr scheint.

8Er herrsche von Meer zu Meer

und vom Euphratstrom bis hin an die Enden der Erde!

9Vor ihm müssen die Steppenvölker die Knie beugen

und seine Feinde den Staub lecken\textless sup title=``=~den Boden
küssen''\textgreater✲;

10die Könige von Tharsis und den Meeresländern

müssen Geschenke ihm bringen, die Herrscher von Saba und Seba{B}
Abgaben✲ entrichten;

11ja huldigen müssen ihm alle Könige,

die Völker alle ihm dienen!

12Denn er wird den Armen retten, der um Hilfe schreit,

den Leidenden und den, der keinen Helfer hat.

13Er wird sich erbarmen des Schwachen und Armen

und Hilfe gewähren den Seelen der Armen;

14aus Bedrückung und Gewalttat wird er ihre Seelen erlösen,

und ihr Blut wird kostbar sein in seinen Augen.

15So lebe er denn, und man gebe ihm vom Golde aus Saba,

man bete immerdar für ihn und segne ihn allezeit!

16Fülle von Korn möge sein im Lande

bis auf die Gipfel der Berge, es rausche seine Frucht wie der Libanon!
Und aus den Städten blühe das Volk hervor so zahlreich wie das Gras der
Erde!

17Sein Name möge ewig bestehn:

solange die Sonne scheint, lebe sein Name fort, so daß man in ihm sich
Segen wünscht und alle Völker ihn glücklich preisen! * * *

18Gepriesen sei Gott der HERR, der Gott Israels,

der Wunder vollbringt, er allein!

19Und gepriesen sei sein herrlicher Name in Ewigkeit,

und die ganze Erde sei seiner Herrlichkeit\textless sup title=``oder:
Ehre''\textgreater✲ voll! Amen, ja Amen!~--

20Zu Ende sind die Gebete Davids, des Sohnes Isais.

\hypertarget{drittes-buch-psalm-73-89}{%
\subsection{Drittes Buch (Psalm 73-89)}\label{drittes-buch-psalm-73-89}}

\hypertarget{das-gottgewirkte-dennoch}{%
\subsubsection{Das gottgewirkte
»Dennoch«}\label{das-gottgewirkte-dennoch}}

\hypertarget{section-72}{%
\section{73}\label{section-72}}

1Ein Psalm von Asaph\textless sup title=``vgl. Ps 50''\textgreater✲.
Dennoch ist Gott voll Güte gegen den Frommen,

der Herr gegen alle, die reinen Herzens sind.

2Doch ich -- fast wär' ich gestrauchelt mit meinen Füßen,

nichts fehlte, so wären meine Schritte ausgeglitten;

3denn ich ereiferte mich über die Großsprecher,

wenn ich sehen mußte der Gottlosen Wohlergehn.

4Denn bis zu ihrem Tode leiden sie keine Schmerzen,

und wohlgenährt ist ihr Leib;

5Unglück trifft sie nicht wie andere Sterbliche,

und sie werden nicht geplagt wie sonst die Menschen.

6Drum ist auch Hochmut ihr Halsgeschmeide,

und Gewalttat ist das Kleid, das sie umhüllt.

7Aus strotzendem Antlitz tritt ihr Auge hervor,

die Gebilde ihres Herzens wallen über.

8Sie höhnen und reden in Bosheit (nur) von Gewalttat,

führen Reden von oben herab;

9gegen den Himmel richten sie ihren Mund,

und ihre Zunge ergeht sich frei auf Erden.

10Darum wendet das Volk sich ihnen zu

und schlürft das Wasser (ihrer Lehren) in vollen Zügen;

11sie sagen: »Wie sollte Gott es wissen,

und wie sollte der Höchste Kenntnis davon haben?«

12Seht, so treiben's die Gottlosen,

und, immer in Sicherheit lebend, häufen sie Reichtum an.

13Ach, ganz umsonst hab' ich rein mein Herz erhalten

und in Unschuld meine Hände gewaschen;

14ich ward ja doch vom Unglück allzeit geplagt,

und alle Morgen war meine Züchtigung da.

15Doch hätt' ich gesagt\textless sup title=``oder:
gedacht''\textgreater✲: »Ich will auch so reden!«,

so hätt' ich treulos verleugnet deiner Söhne\textless sup title=``oder:
Kinder''\textgreater✲ Geschlecht.

16So sann ich denn nach, um dies zu begreifen,

doch es war zu schwer für mein Verständnis,

17bis ich eindrang in die Heiligtümer Gottes

und achtgab auf der Gottlosen Endgeschick.

18Fürwahr, auf schlüpfrigen Boden stellst du sie,

läßt sie fallen, daß sie in Trümmer zergehn.

19Wie werden sie doch im Nu vernichtet,

weggerafft, und nehmen ein Ende mit Schrecken!

20Wie ein Traumbild gleich nach dem Erwachen verfliegt,

so läßt du, o Allherr, beim Wachwerden ihr Bild verschwinden.

21Wenn mein Herz sich nun noch verbitterte

und ich in meinem Innern empört mich fühlte,

22so wär' ich ein ganzer Tor und bar der Einsicht,

benähme mich wie ein vernunftloses Tier gegen dich.

23Doch nein, ich bleibe stets mit dir verbunden,

du hältst mich fest bei meiner rechten Hand;

24du leitest mich nach deinem Ratschluß

und nimmst mich endlich auf in die Herrlichkeit.{A}

25Wen hätt' ich sonst noch im Himmel?

Und außer dir erfreut mich nichts auf Erden.

26Mag Leib und Seele mir verschmachten,

bleibt Gott doch allzeit meines Herzens Fels und mein Teil.

27Denn gewiß: wer von dir sich lossagt, der kommt um;

du vernichtest alle, die treulos von dir abfallen.

28Mir aber ist Gottes Nähe beglückend:

ich setze mein Vertrauen auf Gott den HERRN, um alle deine
Werke\textless sup title=``oder: Taten''\textgreater✲ zu verkünden.

\hypertarget{klage-der-gemeinde-uxfcber-die-verwuxfcstung-des-tempels-und-bitte-um-hilfe}{%
\subsubsection{Klage der Gemeinde über die Verwüstung des Tempels und
Bitte um
Hilfe}\label{klage-der-gemeinde-uxfcber-die-verwuxfcstung-des-tempels-und-bitte-um-hilfe}}

\hypertarget{section-73}{%
\section{74}\label{section-73}}

1Ein Lehrgedicht\textless sup title=``vgl. 32,1''\textgreater✲ von
Asaph\textless sup title=``vgl. Ps 50''\textgreater✲. Warum hast du uns,
o Gott, für immer verworfen,

warum raucht dein Zorn gegen die Herde, die du weidest?

2Gedenke deiner Gemeinde, die vor alters du erworben,

die zum Eigentumsvolk du dir erlöst hast! (Gedenke) des Berges Zion, auf
dem du Wohnung genommen!

3Lenk deine Schritte hinauf zu den ewigen Trümmern:

ach, alles hat der Feind im Heiligtum zerstört!

4Wild brüllen deine Feinde im Innern deiner Versammlungsstätte;

haben dort ihre Fahnen als Siegeszeichen aufgestellt.

5Es sieht sich an, als ob man die Äxte hoch

geschwungen hätte im Dickicht des Waldes.

6Und jetzt zerschlagen sie auch sein Schnitzwerk

allzumal mit Beilen und Hämmern.

7Sie haben dein Heiligtum in Brand gesteckt,

bis zum Boden entweiht die Wohnung deines Namens.

8Sie haben sich vorgenommen: »Wir rotten sie allesamt aus!«

und haben alle Gottesstätten\textless sup title=``d.h.
Synagogen''\textgreater✲ im Lande verbrannt.

9Unsre (heiligen) Zeichen sehn wir nicht mehr, kein Prophet ist mehr da,

und niemand weiß bei uns, wie lange das dauern soll.

10Wie lange, o Gott, soll der Widersacher noch schmähen,

der Feind deinen Namen immerfort lästern?

11Warum doch ziehst du deine Hand zurück?

O zieh deine Rechte hervor aus dem Busen, mach ein Ende!

12Gott ist ja doch mein König von alters her,

Rettungstaten vollführt er inmitten des Landes\textless sup
title=``oder: auf der ganzen Erde''\textgreater✲.

13Du hast das Meer durch deine Kraft gespalten,

die Häupter der Drachen{A} auf den Fluten zerschellt.

14Du hast Leviathans Köpfe zermalmt,

zum Fraß ihn hingegeben dem Volke der Wüstentiere.

15Du hast Quellen und Bäche hervorbrechen lassen,

du hast nieversiegende Ströme trocken gelegt.

16Dein ist der Tag, dein auch die Nacht,

du hast den Mond{A} und die Sonne hingestellt.

17Du hast der Erde rings die Grenzen festgesetzt,

Sommer und Winter, du hast sie gebildet.

18Denke daran: der Feind hat dich, o HERR, gehöhnt,

und ein gottloses Volk deinen Namen gelästert!

19Gib nicht den Raubtieren preis die Seele deiner Taube,

vergiß nicht für immer das Leben deiner Dulder!

20Blick hin auf den Bund! Denn angefüllt sind

die Verstecke des Landes mit Stätten der Gewalttat.

21Laß den Bedrängten nicht enttäuscht davongehn,

der Arme und Bedrückte müsse deinen Namen rühmen!

22Steh auf, Gott, verficht deine Sache!

Gedenke der Schmach, die dich trifft von den Ruchlosen Tag für Tag!

23Vergiß nicht das Geschrei\textless sup title=``oder: laute
Schmähen''\textgreater✲ deiner Feinde,

das Toben deiner Gegner, das allzeit aufsteigt!

\hypertarget{gott-der-gerechte-weltrichter}{%
\subsubsection{Gott der gerechte
Weltrichter}\label{gott-der-gerechte-weltrichter}}

\hypertarget{section-74}{%
\section{75}\label{section-74}}

1Dem Musikmeister, nach (der Singweise =~Melodie) »Vertilge nicht«; ein
Psalm von Asaph\textless sup title=``vgl. Ps 50''\textgreater✲, ein
Lied. 2Wir preisen dich, Gott, wir preisen!

Denn nahe ist uns dein Name: deine Wundertaten verkünden ihn.{B}

3»Wenn ich die Zeit gekommen erachte,

dann halte ich gerechtes Gericht.

4Mag wanken die Erde mit allen ihren Bewohnern:

ich bin's, der ihre Säulen festgestellt. SELA.

5Ich rufe den Stolzen zu: ›Seid nicht stolz!‹

und den Frevlern: ›Hebt den Kopf{A} nicht hoch!

6Hebt euren Kopf nicht gar so hoch,

redet nicht vermessen mit gerecktem Hals!‹«~--

7Denn nicht vom Aufgang (der Sonne) noch vom Niedergang

und nicht von der Wüste her kommt die Erhöhung\textless sup
title=``=~Fähigkeit zum Aufstieg''\textgreater✲;

8nein, Gott ist's, der da richtet:

diesen erniedrigt und jenen erhöht er.

9Denn ein Becher ist in der Hand des HERRN

mit schäumendem Wein, voll von berauschender Mischung; und er schenkt
daraus ein: sogar die Hefen davon müssen schlürfen und trinken alle
Frevler der Erde.

10Ich aber will das ewig verkünden,

will lobsingen dem Gotte Jakobs;

11und alle Hörner der Frevler will ich abhaun,

doch die Hörner der Gerechten sollen erhöht sein\textless sup
title=``=~hoch ragen''\textgreater✲.

\hypertarget{israels-siegeslied-zum-lobpreis-gottes}{%
\subsubsection{Israels Siegeslied zum Lobpreis
Gottes}\label{israels-siegeslied-zum-lobpreis-gottes}}

\hypertarget{section-75}{%
\section{76}\label{section-75}}

1Dem Musikmeister, mit Saitenspiel; ein Psalm von Asaph\textless sup
title=``vgl. Ps 50''\textgreater✲, ein Lied. 2Allbekannt ist Gott in
Juda,

in Israel ist groß sein Name;

3in Salem\textless sup title=``=~Jerusalem; 1.Mose 14,18''\textgreater✲
erstand seine Hütte\textless sup title=``oder: sein Zelt''\textgreater✲

und seine Wohnstatt in Zion.

4Allda hat er zerbrochen des Bogens Blitze,

Schild und Schwert und jegliche Kriegswehr. SELA.

5Ruhmvoll bist du, herrlich

von den ewigen Bergen her.{A}

6Ausgeplündert\textless sup title=``oder: entwaffnet''\textgreater✲
wurden die tapferen Streiter,

sanken hin in ihren Todesschlaf, und all den Helden versagte der
Arm\textless sup title=``=~die Kraft''\textgreater✲:

7vor deinem Drohruf, du Gott Jakobs,

sanken in Betäubung so Wagen wie Rosse.

8Ja du bist furchtbar, und wer kann bestehn

vor dir, sobald dein Zorn entbrannt ist?

9Vom Himmel her kündigtest du das Gericht an:

da erschrak die Erde und wurde still,

10als Gott sich erhob zum Gerichtsvollzug,

um allen Bedrückten auf Erden\textless sup title=``oder: des
Landes''\textgreater✲ zu helfen. SELA.

11Denn der Menschen Grimm wird dir zum Lobpreis,

wenn zuletzt du dich gürtest mit Zornesflammen.{A}

12Bringt Gelübde dar und erfüllt sie dem HERRN, eurem Gott:

alle, die ihn rings umgeben, müssen Geschenke dem
Schrecklichen\textless sup title=``=~Ehrfurcht
Gebietenden''\textgreater✲ bringen,

13ihm, der den Hochmut der Fürsten dämpft

und furchtbar ist den Königen der Erde.

\hypertarget{erinnerungen-in-leidvoller-zeit-an-gottes-fruxfchere-fuxfchrungen-und-klage-uxfcber-die-uxe4nderung-des-guxf6ttlichen-verhaltens-gegen-sein-volk}{%
\subsubsection{Erinnerungen in leidvoller Zeit an Gottes frühere
Führungen und Klage über die Änderung des göttlichen Verhaltens gegen
sein
Volk}\label{erinnerungen-in-leidvoller-zeit-an-gottes-fruxfchere-fuxfchrungen-und-klage-uxfcber-die-uxe4nderung-des-guxf6ttlichen-verhaltens-gegen-sein-volk}}

\hypertarget{section-76}{%
\section{77}\label{section-76}}

1Dem Musikmeister über die Jeduthuniden\textless sup title=``vgl.
62,1''\textgreater✲; von Asaph ein Psalm. 2Laut ruf' ich zu Gott, ja ich
will schreien,

laut ruf' ich zu Gott: »Ach, höre mein Flehen!«

3Wenn Drangsalszeiten über mich kommen, such' ich den Allherrn;

meine Hand ist nachts ohn' Ermatten ausgestreckt, meine Seele will sich
nicht trösten lassen.

4Denk' ich an Gott, so muß ich seufzen;

sinne ich nach,{A} so verzagt mein Geist. SELA.

5Du hältst mir die Augenlider offen,

ich bin voll Unruhe und kann nicht reden.

6Ich überdenke die Tage der Vorzeit,

die längst entschwundenen Jahre;

7ich denke bei Nacht an mein Saitenspiel,

ich sinne in meinem Herzen nach,{A} und es grübelt mein Geist und fragt:

8»Wird der Allherr auf ewig verstoßen

und niemals wieder Gnade üben?

9Ist seine Güte für immer erschöpft?

sind seine Verheißungen abgetan für alle Zukunft?

10Hat Gott vergessen, gnädig zu sein,

oder im Zorn sein Erbarmen verschlossen?« SELA.

11Da sagte ich mir: »Das bekümmert mich schmerzlich,

daß das Walten\textless sup title=``oder: Verhalten''\textgreater✲ des
Höchsten sich hat geändert.«

12Ich will gedenken der Taten des HERRN,

will gedenken deiner Wunder von der Vorzeit her,

13will sinnen über all dein Tun

und deine großen Taten erwägen.

14O Gott, erhaben ist dein Weg✲:

wo ist eine Gottheit so groß wie Gott?

15Du bist der Gott, der Wunder tut,

du hast deine Macht an den Völkern bewiesen,

16hast dein Volk erlöst mit starkem Arm,

die Kinder Jakobs und Josephs. SELA.

17Als die Wasser dich sahen, o Gott,

als die Wasser dich sahen, erbebten sie, auch die Tiefen\textless sup
title=``oder: Fluten''\textgreater✲ des Weltmeers zitterten;

18die Wolken ergossen sich in strömenden Regen,

das Gewölk ließ Donner erkrachen, und deine Pfeile fuhren einher;

19deine Donnerstimme dröhnte am Himmelsgewölbe,

Blitze erhellten den Erdkreis, es bebte und schwankte die Erde.

20Durchs Meer ging dein Weg dahin

und dein Pfad durch gewaltige Fluten; doch deine Spuren waren nicht zu
erkennen.

21Du hast dein Volk geführt wie eine Herde

unter Leitung von Mose und Aaron.

\hypertarget{warnender-ruxfcckblick-auf-israels-wiederholten-ungehorsam}{%
\subsubsection{Warnender Rückblick auf Israels wiederholten
Ungehorsam}\label{warnender-ruxfcckblick-auf-israels-wiederholten-ungehorsam}}

\hypertarget{section-77}{%
\section{78}\label{section-77}}

1Ein Lehrgedicht\textless sup title=``vgl. 32,1''\textgreater✲ von
Asaph\textless sup title=``vgl. 50,1''\textgreater✲. Gib acht, mein
Volk, auf meine Belehrung,

leiht euer Ohr den Worten meines Mundes!

2Ich will auftun meinen Mund zur Rede in Sprüchen,

will Rätsel verkünden von der Vorzeit her.

3Was wir gehört und erfahren

und unsere Väter uns erzählt haben,

4das wollen wir ihren Kindern nicht verschweigen,

sondern dem künftgen Geschlecht verkünden die Ruhmestaten des HERRN und
seine Stärke und die Wunder, die er getan hat.

5Denn er hat ein Zeugnis aufgerichtet in Jakob

und festgestellt in Israel ein Gesetz, von dem er unsern Vätern gebot,
es ihren Kindern kundzutun,

6auf daß die Nachwelt Kenntnis davon erhielte:

die Kinder, die geboren würden, sollten aufstehn und ihren Kindern davon
erzählen,

7daß sie auf Gott ihr Vertrauen setzten

und die Taten Gottes nicht vergäßen und seine Gebote befolgten,

8daß sie nicht wie ihre Väter würden,

ein trotziges und widerspenstiges Geschlecht, ein Geschlecht mit
wankelmütigem Herzen, dessen Geist sich nicht zuverlässig zu Gott hielt.

9Ephraims Söhne, bogengerüstete Schützen,

haben den Rücken gewandt am Tage des Kampfes.

10Sie hielten den gottgestifteten Bund nicht

und wollten nicht wandeln in seinem Gesetz;

11nein, sie vergaßen seine Taten

und seine Wunder, die er sie hatte sehen lassen.

12Vor ihren Vätern hatte er Wunder getan

im Lande Ägypten, im Gefilde von Zoan{A}.

13Er spaltete das Meer und ließ sie hindurchziehn

und türmte die Wasser auf wie einen Wall;

14er leitete sie bei Tag durch die Wolke

und während der ganzen Nacht durch Feuerschein;

15er spaltete Felsen in der Wüste

und tränkte sie reichlich wie mit Fluten;

16Bäche ließ er aus dem Felsen hervorgehn

und Wasser gleich Strömen niederfließen.

17Dennoch fuhren sie fort, gegen ihn zu sündigen,

und widerstrebten dem Höchsten in der Wüste;

18ja, sie versuchten Gott in ihren Herzen,

indem sie Speise verlangten für ihr Gelüst,

19und redeten gegen Gott mit den Worten:

»Kann Gott wohl einen Tisch in der Wüste uns decken?

20Wohl hat er den Felsen geschlagen, daß Wasser

flossen heraus und Bäche sich ergossen; doch wird er auch vermögen Brot
zu geben oder Fleisch seinem Volke zu schaffen?«

21Drum, als der HERR das hörte, ergrimmte er:

Feuer entbrannte gegen Jakob, und Zorn stieg auf gegen Israel,

22weil sie an Gott nicht glaubten

und auf seine Hilfe nicht vertrauten.

23Und doch gebot er den Wolken droben

und tat die Türen des Himmels auf,

24ließ Manna auf sie regnen zum Essen

und gab ihnen himmlisches Brotkorn:

25Engelspeise aßen sie allesamt,

Reisekost sandte er ihnen zur Sättigung.

26Hinfahren ließ er den Ostwind am Himmel

und führte durch seine Kraft den Südwind herbei;

27Fleisch ließ er auf sie regnen wie Staub

und beschwingte Vögel wie Meeressand;

28mitten in ihr Lager ließ er sie fallen,

rings um ihre Wohnungen her.

29Da aßen sie und wurden reichlich satt,

und was sie gewünscht, gewährte er ihnen.

30Noch hatten sie ihres Gelüsts sich nicht entschlagen,

noch hatten sie ihre Speise in ihrem Munde,

31da stieg der Ingrimm Gottes gegen sie auf

und erwürgte die kräftigen Männer unter ihnen und streckte Israels junge
Mannschaft zu Boden.

32Trotz alledem sündigten sie weiter

und glaubten nicht an seine Wunder✲.

33Drum ließ er ihre Tage vergehn wie einen Hauch

und ihre Jahre in angstvoller Hast.

34Wenn er sie sterben ließ, dann fragten sie nach ihm

und kehrten um und suchten Gott eifrig

35und dachten daran, daß Gott ihr Fels sei

und Gott, der Höchste, ihr Erlöser.

36Doch sie heuchelten ihm mit ihrem Munde

und belogen ihn mit ihrer Zunge;

37denn ihr Herz hing nicht fest an ihm,

und sie hielten nicht treu an seinem Bunde.

38Doch er war barmherzig, vergab die Schuld

und vertilgte sie nicht, nein, immer wieder hielt er seinen Zorn zurück
und ließ nicht seinen ganzen Grimm erwachen;

39denn er dachte daran, daß Fleisch sie waren,

ein Windhauch, der hinfährt und nicht wiederkehrt.

40Wie oft widerstrebten sie ihm in der Wüste,

kränkten sie ihn in der Öde!

41Und immer aufs neue versuchten sie Gott

und betrübten den Heiligen Israels.

42Sie dachten nicht mehr an seine starke Hand,

an den Tag, wo er sie vom Bedränger erlöste,

43als er seine Zeichen in Ägypten tat,

seine Wunder im Gefilde von Zoan✲.

44Er verwandelte dort in Blut ihre Ströme✲,

so daß man ihr fließendes Wasser nicht trinken konnte;

45er sandte unter sie Ungeziefer, das sie fraß,

und Frösche, die ihnen Verderben brachten;

46er gab ihre Ernte den Freßgrillen preis

und die Frucht ihrer Arbeit den Heuschrecken;

47er zerschlug ihre Reben mit Hagel,

ihre Maulbeerfeigenbäume mit Schloßen;

48er gab ihr Vieh dem Hagel preis

und ihren Besitz den Blitzen;

49er sandte gegen sie seines Zornes Glut,

Wut und Grimm und Drangsal: eine Schar{A} von Unglücksengeln;

50er ließ seinem Ingrimm freien Lauf,

entzog ihre Seele nicht dem Tode, überließ vielmehr ihr Leben der Pest;

51er ließ alle Erstgeburt in Ägypten sterben,

der Manneskraft Erstlinge in den Zelten Hams.

52Dann ließ er sein Volk ausziehn wie Schafe

und leitete sie in der Wüste wie eine Herde

53und führte sie sicher, so daß sie nicht bangten;

ihre Feinde aber bedeckte das Meer.

54So brachte er sie nach seinem heiligen Gebiet,

in das Bergland, das er mit seiner Rechten erworben,

55und vertrieb vor ihnen her die Völker,

verloste ihr Gebiet als erblichen Besitz und ließ in ihren Zelten die
Stämme Israels wohnen.

56Doch sie versuchten und reizten Gott, den Höchsten,

und hielten sich nicht an seine Gebote,

57sondern fielen ab und handelten treulos, ihren Vätern gleich;

sie versagten wie ein trüglicher\textless sup title=``oder:
schlaffer''\textgreater✲ Bogen

58und erbitterten ihn durch ihren Höhendienst

und reizten ihn zum Eifer durch ihre Götzenbilder.

59Als Gott es vernahm, ergrimmte er

und verwarf Israel ganz und gar:

60er gab seine Wohnung in Silo auf,

das Zelt, das er aufgeschlagen unter den Menschen;

61er ließ seine Macht in Gefangenschaft fallen

und seine Zier in die Hand des Feindes;

62er gab sein Volk dem Schwerte preis

und war entrüstet über sein Erbteil✲;

63seine jungen Männer fraß das Feuer,

und seine Jungfraun blieben ohne Brautlied;

64seine Priester fielen durchs Schwert,

und seine Witwen konnten keine Totenklage halten.{A}

65Da erwachte der Allherr wie ein Schlafender,

wie ein vom Wein übermannter Kriegsheld;

66er schlug seine Feinde von hinten

und gab sie ewiger Schande preis.

67Auch verwarf er das Zelt Josephs

und erwählte nicht den Stamm Ephraim,

68sondern erwählte den Stamm Juda,

den Berg Zion, den er liebgewonnen;

69und er baute den ragenden Bergen\textless sup title=``oder:
Palästen''\textgreater✲ gleich sein Heiligtum,

fest wie die Erde, die er auf ewig gegründet.

70Dann erwählte er David, seinen Knecht,

den er wegnahm von den Hürden des Kleinviehs;

71von den Mutterschafen holte er ihn,

daß er Jakob weide, sein Volk, und Israel, seinen Erbbesitz.

72Der weidete sie mit redlichem Herzen

und leitete sie mit kundiger Hand.

\hypertarget{klagelied-des-gottesvolkes-uxfcber-die-verwuxfcstung-jerusalems}{%
\subsubsection{Klagelied des Gottesvolkes über die Verwüstung
Jerusalems}\label{klagelied-des-gottesvolkes-uxfcber-die-verwuxfcstung-jerusalems}}

\hypertarget{section-78}{%
\section{79}\label{section-78}}

1Ein Psalm von Asaph\textless sup title=``vgl. 50,1''\textgreater✲. O
Gott, in dein Eigentum sind Heiden eingedrungen,

haben deinen heiligen Tempel entweiht, Jerusalem zu Trümmerhaufen
gemacht!

2Sie haben die Leichen deiner Knechte

den Vögeln des Himmels zum Fraß gegeben, den wilden Tieren des Landes
die Leiber deiner Frommen!

3Sie haben deren Blut vergossen wie Wasser

rings um Jerusalem her, und niemand hat sie begraben!

4Wir sind unsern Nachbarn zur Schmähung geworden,

ein Spott und Hohn den Völkern um uns her!

5Wie lange, o HERR, willst du unversöhnlich zürnen?

Bis wann soll lodern dein Eifer\textless sup title=``=~deine
Leidenschaft''\textgreater✲ wie Feuer?

6Gieß deine Zornglut über die Heiden aus, die dich nicht kennen,

auf die Reiche, die deinen Namen nicht anrufen!

7Denn sie haben Jakob gefressen

und seine Wohnstatt verwüstet.

8Rechne uns nicht die Schuld der Väter an,

laß eilends dein Erbarmen uns angedeihn! Denn gar schwach sind wir
geworden.

9Hilf uns, du Gott unsers Heils,

um der Ehre deines Namens willen! Errette uns und vergib uns unsere
Sünden um deines Namens willen!

10Warum sollen die Heiden sagen: »Wo ist ihr Gott?«

Laß kundwerden an den Heiden vor unsern Augen die Rache für das
vergossne Blut deiner Knechte!

11Laß vor dich kommen das Seufzen der Gefangenen;

kraft deines starken Armes erhalte am Leben die dem Tode Geweihten!{A}

12Und zahle unsern Nachbarn siebenfach heim in ihren Busen

den Hohn, mit dem sie dich, o Allherr, gehöhnt!

13Wir aber, dein Volk und die Herde, die du weidest,

wir wollen dir ewiglich danken, von Geschlecht zu Geschlecht verkünden
deinen Ruhm!

\hypertarget{gebet-um-wiederherstellung-israels-des-von-gott-gepflanzten-weinstocks}{%
\subsubsection{Gebet um Wiederherstellung Israels, des von Gott
gepflanzten
Weinstocks}\label{gebet-um-wiederherstellung-israels-des-von-gott-gepflanzten-weinstocks}}

\hypertarget{section-79}{%
\section{80}\label{section-79}}

1Dem Musikmeister, nach (der Singweise =~Melodie) »Lilien(rein) ist das
Zeugnis«; von Asaph\textless sup title=``vgl. 50,1''\textgreater✲ ein
Psalm. 2O Hirte Israels, merk auf,

der du Joseph leitest wie eine Herde! Der du thronst über den
Cheruben,{A} erscheine!

3Als Anführer Ephraims und Benjamins und Manasses

biete deine Heldenkraft auf und komm uns zu Hilfe!

4O Gott (der Heerscharen), stelle uns wieder her

und laß dein Angesicht leuchten, damit uns Rettung widerfährt!

5O HERR, Gott der Heerscharen, wie lange noch

raucht dein Zorn trotz der Gebete deines Volkes?

6Du hast uns Tränenbrot essen lassen

und uns eimerweise✲ getränkt mit Tränen;

7du hast uns gemacht zum Zankapfel unsern Nachbarn,

und unsere Feinde spotten über uns.

8O Gott der Heerscharen, stelle uns wieder her

und laß dein Angesicht leuchten, damit uns Rettung widerfährt!

9Einen Weinstock hast aus\textless sup title=``oder: in''\textgreater✲
Ägypten du ausgehoben,

hast Heidenvölker vertrieben, ihn eingepflanzt,

10hast weiten Raum vor ihm her geschafft,

daß er Wurzeln schlug und das Land erfüllte;

11die Berge wurden von seinem Schatten bedeckt

und von seinen Reben die Zedern Gottes;{A}

12er streckte seine Ranken aus bis ans Meer

und seine Schößlinge bis zum Euphratstrom.

13Warum hast du sein Gehege eingerissen,

so daß alle ihn zerpflücken, die des Weges ziehn?

14Es zerwühlt ihn der Eber aus dem Walde,

und die Tiere des Feldes fressen ihn kahl.

15O Gott der Heerscharen, kehre doch zurück,

schaue vom Himmel nieder und blicke her und nimm dich dieses Weinstocks
an,

16des Setzlings, den deine Rechte gepflanzt,

und des Schößlings, den du dir großgezogen!

17Er ist mit Feuer verbrannt, ist abgehauen:

vor dem Zornblick deines Angesichts kommen sie um.

18Halte schirmend die Hand über den Mann deiner Rechten,

den Menschensohn, den du dir großgezogen:

19so wollen wir nimmer von dir weichen!

Schenke uns neues Leben,{A} so wollen wir deinen Namen preisen!

20O HERR, Gott der Heerscharen, stelle uns wieder her,

laß dein Angesicht leuchten, damit uns Rettung widerfährt!

\hypertarget{festlied-am-passhafest-mit-buuxdfrede}{%
\subsubsection{Festlied (am Passhafest?) mit
Bußrede}\label{festlied-am-passhafest-mit-buuxdfrede}}

\hypertarget{section-80}{%
\section{81}\label{section-80}}

1Dem Musikmeister, nach der Keltertreterweise; von Asaph\textless sup
title=``vgl. 50,1''\textgreater✲. 2Singt jubelnd dem Gott, der unsre
Stärke ist,

jauchzet dem Gott Jakobs!

3Stimmt Lobgesang an und laßt die Pauken erschallen,

die liebliche Zither mitsamt der Harfe!

4Stoßt am Neumond in die Posaune,

beim Vollmond zur Feier unsres Festes!

5Denn so ist es Satzung für Israel,

ein Gebot des Gottes Jakobs;

6als Gesetz hat er's für Joseph verordnet,

als er auszog gegen Ägyptenland.~-- Eine Sprache, die ich bisher nicht
gekannt, vernehme ich:

7»Ich hab' seine Schulter\textless sup title=``oder: seinen
Rücken''\textgreater✲ der Last entzogen,

seine Hände sind des Tragkorbs ledig geworden.

8Als du riefst in der Drangsal, erlöste ich dich,

erhörte dich in der Hülle der Donnerwolke, prüfte dich am
Haderwasser\textless sup title=``4.Mose 20,13''\textgreater✲. SELA.

9›Höre, mein Volk, ich will dich warnen!

o Israel, möchtest du mir doch gehorchen!

10Kein fremder Gott soll unter dir sein,

vor keinem Gott des Auslands darfst du dich niederwerfen!

11Ich, der HERR, bin dein Gott,

der dich heraufgeführt aus Ägyptenland: tu deinen Mund weit auf, so will
ich ihn füllen!‹

12Doch mein Volk hat nicht gehört auf meine Stimme,

und Israel ist mir nicht zu Willen gewesen.

13Da hab ich sie preisgegeben dem Starrsinn ihres Herzens:

sie sollten nach ihren eignen Gedanken wandeln.

14O wollte mein Volk doch mir gehorchen,

Israel doch wandeln auf meinen Wegen!

15Wie bald würde ich ihre Feinde beugen

und gegen ihre Dränger kehren meine Hand!

16Die da hassen den HERRN, die müßten ihm schmeicheln,

und ihre Gerichtszeit sollte ewig währen.

17Doch ihn wollt' ich nähren mit dem Mark des Weizens,

dich sättigen aus dem Felsen mit Honig.«\textless sup title=``vgl.
5.Mose 32,13-14''\textgreater✲

\hypertarget{gottes-gericht-uxfcber-ungerechte-richter-d.h.-pflichtvergessene-gewalthaber}{%
\subsubsection{Gottes Gericht über ungerechte Richter (d.h.
pflichtvergessene
Gewalthaber)}\label{gottes-gericht-uxfcber-ungerechte-richter-d.h.-pflichtvergessene-gewalthaber}}

\hypertarget{section-81}{%
\section{82}\label{section-81}}

1Ein Psalm Asaphs\textless sup title=``vgl. 50,1''\textgreater✲. Gott
steht da in der Gottesversammlung,

hält inmitten der Götter{A} Gericht:

2»Wie lange noch wollt ihr ungerecht richten

und Partei für die Gottlosen nehmen? SELA.

3Schafft Recht dem Geringen und Verwaisten,

dem Bedrückten und Dürftigen verhelft zum Recht!

4Rettet den Geringen und Armen,

entreißt ihn der Hand der Gottlosen!«

5»Doch sie sind ohne Einsicht und ohne Erkenntnis;

in Finsternis gehn sie einher, mögen der Erde\textless sup title=``oder:
des Landes''\textgreater✲ Pfeiler auch alle wanken.

6Wohl hab' ich selber gesagt, daß ihr Götter\textless sup title=``vgl.
V.1''\textgreater✲ seid

und Söhne des Höchsten allesamt;

7dennoch wie (gewöhnliche) Menschen sollt ihr sterben.

und fallen wie irgendeiner der Fürsten.«

8Erhebe dich, Gott, richte die Erde!

Denn du bist der Erbherr über alle Völker.{A}

\hypertarget{feinde-ringsum-gebet-des-volkes-in-kriegsnot}{%
\subsubsection{Feinde ringsum! (Gebet des Volkes in
Kriegsnot)}\label{feinde-ringsum-gebet-des-volkes-in-kriegsnot}}

\hypertarget{section-82}{%
\section{83}\label{section-82}}

1Ein Lied, ein Psalm Asaphs\textless sup title=``vgl.
50,1''\textgreater✲. 2O Gott, halte dich nicht zurück,

verharre nicht im Schweigen und bleibe nicht ruhig, o Gott!

3Denn siehe, deine Feinde toben,

und die dich hassen, tragen das Haupt hoch!

4Gegen dein Volk ersinnen sie einen Anschlag

und beraten sich gegen deine Schutzbefohlnen;

5sie sagen: »Kommt, wir wollen sie vertilgen als Volk:

des Namens Israel soll man fürder nicht gedenken!«

6Ja, sie haben einmütigen Sinns sich beraten,

ein Bündnis gegen dich geschlossen:

7die Zelte Edoms und der Ismaeliter,

Moab und die Hagriter,

8Gebal und Ammon und Amalek,

das Philisterland samt den Bewohnern von Tyrus.

9Auch Assur hat sich zu ihnen gesellt,

es leiht den Nachkommen Lots{A} seinen Arm. SELA.

10Verfahre mit ihnen wie einst mit Midian\textless sup title=``Ri 7-8;
Jes 9,3''\textgreater✲,

wie mit Sisera, wie mit Jabin am Bache Kison\textless sup title=``Ri
4''\textgreater✲,

11die bei Endor\textless sup title=``Jos 17,11''\textgreater✲ den
Untergang fanden,

mit ihren Leibern das Erdreich düngten!

12Mache sie, ihre Edlen, wie Oreb und Seeb,

und wie Sebah und Zalmunna alle ihre Fürsten\textless sup title=``Ri
7-8''\textgreater✲,

13die gesprochen hatten: »Wir wollen für uns erobern

die Fluren\textless sup title=``oder: Gefilde''\textgreater✲
Gottes!«\textless sup title=``Ps 74,8''\textgreater✲

14Mein Gott, mache sie gleich dem verwehten Laub,

wie Spreu vor dem Winde!

15Wie Feuer, das den Wald verzehrt,

wie Flammen, welche die Berge versengen:

16so verfolge sie mit deinem Sturm

und schrecke sie mit deiner Windsbraut!

17Laß Beschämung ihr Antlitz bedecken,

auf daß sie nach deinem Namen fragen, o HERR!

18Laß sie beschämt und erschreckt sein für immer,

in Schande geraten und vergehn!

19Sie müssen erkennen, daß du, dessen Name »HERR« ist,

du allein der Höchste bist über die ganze Erde.

\hypertarget{sehnsucht-nach-dem-hause-gottes-ein-pilgerlied}{%
\subsubsection{Sehnsucht nach dem Hause Gottes (ein
Pilgerlied?)}\label{sehnsucht-nach-dem-hause-gottes-ein-pilgerlied}}

\hypertarget{section-83}{%
\section{84}\label{section-83}}

1Dem Musikmeister, nach der Keltertreterweise; von den Korahiten ein
Psalm. 2Wie lieblich ist deine Wohnstatt✲,

HERR der Heerscharen!

3Meine Seele hat sich gesehnt, ja geschmachtet

nach den Vorhöfen des HERRN; nun jubeln mein Herz und mein Leib dem
lebendigen Gott entgegen!

4Hat doch auch der Sperling ein Haus gefunden

und die Schwalbe ein Nest für sich, woselbst sie ihre Jungen birgt:
deine Altäre, o HERR der Heerscharen, mein König und mein Gott.

5Wohl denen, die da wohnen in deinem Haus,

dich allzeit preisen! SELA.

6Wohl allen, die in dir ihre Stärke finden,

wenn auf Pilgerfahrten sie sinnen!{A}

7Wenn sie wandern durchs Bakatal,

machen sie's zum Quellengrund, den auch der Frühregen kleidet in reichen
Segen.

8Sie wandern dahin mit stets erneuter Kraft,

bis vor Gott sie erscheinen in Zion.

9O HERR, Gott der Heerscharen, höre mein Gebet,

vernimm es, Gott Jakobs! SELA.

10Du unser Schild, blick her, o Gott,

und schau auf das Antlitz deines Gesalbten\textless sup title=``d.h. des
Königs''\textgreater✲!

11Denn ein einziger Tag in deinen Vorhöfen

ist besser als tausend andere; lieber will ich stehn an der Schwelle im
Hause meines Gottes, als wohnen in den Zelten der Frevler\textless sup
title=``=~der Gottlosen''\textgreater✲.

12Denn Sonne\textless sup title=``oder: Mauerzinne''\textgreater✲ und
Schild ist Gott der HERR;

Gnade und Ehre verleiht der HERR, nichts Gutes versagt er denen, die
unsträflich wandeln.

13O HERR der Heerscharen,

wohl dem Menschen, der dir vertraut!

\hypertarget{israels-gebet-um-neue-gnade-und-die-segensverheiuxdfung-gottes}{%
\subsubsection{Israels Gebet um neue Gnade und die Segensverheißung
Gottes}\label{israels-gebet-um-neue-gnade-und-die-segensverheiuxdfung-gottes}}

\hypertarget{section-84}{%
\section{85}\label{section-84}}

1Dem Musikmeister; von den Korahiten ein Psalm. 2Du hast zwar, HERR,
deinem Lande Gnade gewährt,

hast Jakobs Mißgeschick gewendet,{A}

3hast deinem Volke die Schuld vergeben

und all seine Sünde zugedeckt, SELA;

4hast deinem ganzen Groll entsagt,

von der Glut deines Zorns dich abgewandt:

5stell uns nun aber auch wieder her, du Gott unsers Heils,

und laß deinen Unmut gegen uns schwinden!

6Willst du denn unversöhnlich gegen uns zürnen

und deinen Zorn fortdauern lassen für und für?

7Willst du uns nicht wieder neu beleben,

daß dein Volk sich deiner mag freuen?

8Laß uns schauen, o HERR, deine Gnade

und gewähre uns dein Heil!

9Ich will doch hören✲, was Gott der HERR verkündet!~--

Fürwahr, er kündet Segen an seinem Volke und seinen Frommen; nur daß sie
nicht wieder sich wenden zur Torheit!

10Wahrlich, sein Heil\textless sup title=``oder: seine
Hilfe''\textgreater✲ ist denen nah, die ihn fürchten,

daß Herrlichkeit in unserm Lande wohne,

11daß Gnade und Treue einander begegnen\textless sup title=``oder:
begrüßen''\textgreater✲,

Gerechtigkeit und Friede sich küssen.

12Die Treue wird aus der Erde sprossen

und Gerechtigkeit vom Himmel niederschauen.

13Dann wird uns der HERR auch Segen spenden,

daß unser Land uns seinen Ertrag gewährt;

14Gerechtigkeit wird vor ihm hergehn

und achten auf den Weg seiner Schritte.{A}

\hypertarget{gebet-eines-frommen-in-feindlicher-bedruxe4ngnis}{%
\subsubsection{Gebet eines Frommen in feindlicher
Bedrängnis}\label{gebet-eines-frommen-in-feindlicher-bedruxe4ngnis}}

\hypertarget{section-85}{%
\section{86}\label{section-85}}

1Ein Gebet Davids. Neige, o HERR, dein Ohr, erhöre mich,

denn elend bin ich und arm!

2Bewahre meine Seele, denn ich bin fromm;

hilf du, mein Gott, deinem Knecht, der auf dich vertraut!

3Sei mir gnädig, o Allherr,

denn zu dir rufe ich allezeit.

4Erfreue das Herz deines Knechtes,

denn zu dir, o Allherr, erheb' ich meine Seele.

5Denn du, o Allherr, bist gütig und bereit zum Verzeihen,

bist reich an Gnade für alle, die dich anrufen.

6Vernimm, o HERR, mein Gebet

und merke auf mein lautes Flehen!

7Bin ich in Not, so ruf' ich zu dir,

denn du erhörst mich.

8Keiner kommt dir gleich unter den Göttern, o Allherr,

und nichts ist deinen Werken vergleichbar.

9Alle Völker, die du geschaffen,

werden kommen und vor dir anbeten, o Allherr, und deinen Namen ehren;

10denn du bist groß, und Wunder tust du:

ja du, nur du bist Gott.

11Lehre mich, HERR, deinen Weg,

daß ich ihn wandle in deiner Wahrheit\textless sup title=``oder: in
Treue gegen dich''\textgreater✲; richte mein Herz auf das Eine, daß es
deinen Namen fürchte!

12Preisen will ich dich, Allherr, mein Gott, von ganzem Herzen

und deinen Namen ewiglich ehren;

13denn deine Gnade ist groß gegen mich gewesen:

du hast meine Seele\textless sup title=``oder: mein Leben''\textgreater✲
errettet aus der Tiefe des Totenreichs.

14O Gott! Vermessene haben sich gegen mich erhoben,

eine Rotte von Schreckensmännern steht mir nach dem Leben; sie haben
dich nicht vor Augen.

15Doch du, o Allherr, bist ein Gott voll Erbarmen und Gnade,

langmütig und reich an Gnade und Treue.

16Wende dich zu mir und sei mir gnädig;

verleih deine Kraft deinem Knecht und hilf dem Sohne deiner Magd!

17Tu ein Zeichen an mir zum Guten,

daß meine Feinde es sehn und sich schämen müssen, weil du, o HERR, mein
Helfer und Tröster gewesen!

\hypertarget{zion-die-gottgeliebte-mutterstadt-der-vuxf6lker}{%
\subsubsection{Zion die gottgeliebte Mutterstadt der
Völker}\label{zion-die-gottgeliebte-mutterstadt-der-vuxf6lker}}

\hypertarget{section-86}{%
\section{87}\label{section-86}}

1Von den Korahiten ein Psalm, ein Lied.

Seine{A} Gründung liegt auf heiligen Bergen:

2lieb hat der HERR die Tore Zions

mehr als alle (anderen) Wohnstätten Jakobs.

3Herrliches ist von dir berichtet, du Gottesstadt. SELA. 4»Ich nenne
Ägypten und Babel als meine Bekenner,

hier das Philisterland und Tyrus samt Äthiopien~-- nämlich wer dort
seine Heimat hat.«

5Doch von Zion heißt es: »Mann für Mann hat dort seine Heimat,

und er selbst, der Höchste, macht es stark\textless sup title=``oder:
hat es gegründet''\textgreater✲.«

6Der HERR zählt, wenn er die Völker aufschreibt\textless sup
title=``oder: verzeichnet''\textgreater✲:

»Dieser hat dort seine Heimat.« SELA.

7Sie aber tanzen den Reigen und singen:

»Alle meine Quellen sind in dir (o Zion)!«

\hypertarget{hoffnungslose-klage-eines-schwerkranken-dulders}{%
\subsubsection{Hoffnungslose Klage eines schwerkranken
Dulders}\label{hoffnungslose-klage-eines-schwerkranken-dulders}}

\hypertarget{section-87}{%
\section{88}\label{section-87}}

1Ein Lied, ein Psalm von den Korahiten; dem Musikmeister, nach (der
Singweise =~Melodie) »die Krankheit«; ein Lehrgedicht\textless sup
title=``vgl. 32,1''\textgreater✲ von Heman, dem Esrahiten. 2O HERR, du
Gott meines Heils,

ich rufe bei Tage und schreie nachts vor dir:

3o laß mein Gebet vor dich kommen,

neige dein Ohr meinem Flehen zu!

4Denn meine Seele ist mit Leiden gesättigt,

und mein Leben naht sich dem Totenreich.

5Schon zählt man mich zu den ins Grab Gesunknen,

ich bin wie ein Mann ohne Lebenskraft.

6Unter den Toten hab' ich mein Lager

gleichwie Erschlagne, die im Grabe liegen, deren du nicht mehr gedenkst:
sie sind ja deiner Hand entrückt.

7Du hast mich in die Grube der Unterwelt versetzt,

in finstre Nacht, in die Tiefe;

8auf mir lastet schwer dein Grimm,

und mit all deinen Wogen drückst du mich nieder.{A} SELA.

9Meine Bekannten hast du mir entfremdet,

hast mich ihnen zum Abscheu gemacht; eingeschlossen bin ich und kann
nicht hinaus:

10

mein Auge erlischt vor Elend. Ich rufe zu dir, o HERR, jeden Tag, ich
breite zu dir meine Hände aus:

11»Kannst an den Toten du Wunder tun,

oder werden Schatten aufstehn, um dich zu preisen? SELA.

12Wird man im Grabe von deiner Gnade erzählen,

von deiner Treue im Abgrund?{A}

13Verkündet man dein Wunderwalten in der Finsternis

und deine Gerechtigkeit im Lande des Vergessens?«

14Ich dagegen rufe laut zu dir, o HERR,

schon am Morgen tritt mein Gebet vor dich:

15»Warum, o HERR, verwirfst du mich,

verbirgst du dein Antlitz vor mir?«

16Elend bin ich und siech von Jugend auf,

ich trage deine Schrecken und verzweifle.

17Deine Zornesgluten sind über mich hingegangen,

deine Schrecknisse haben mich vernichtet;

18sie umgeben mich immerdar wie Wasserfluten,

umringen mich allzumal.

19Freunde und Gefährten hast du mir entfremdet:

nur die Finsternis ist mir vertraut (geblieben).

\hypertarget{wo-bleiben-die-dem-davidhause-gegebenen-gnadenverheiuxdfungen-gottes}{%
\subsubsection{Wo bleiben die dem Davidhause gegebenen
Gnadenverheißungen
Gottes?}\label{wo-bleiben-die-dem-davidhause-gegebenen-gnadenverheiuxdfungen-gottes}}

\hypertarget{section-88}{%
\section{89}\label{section-88}}

1Ein Lehrgedicht\textless sup title=``vgl. 32,1''\textgreater✲ von
Ethan, dem Esrahiten. 2Die Gnadenerweise des HERRN will ich allzeit
besingen,

bis zum fernsten Geschlecht deine Treue laut verkünden.

3Denn du, Herr, hast verheißen:

»Auf ewig soll der Gnadenbund aufgebaut sein«

fest wie den Himmel hast du deine Treue gegründet{A} --:

4»Ich habe einen Bund geschlossen mit meinem Erwählten,

habe David, meinem Knecht, geschworen:

5›Deinem Geschlecht will ich ewige Dauer verleihen

und aufbaun deinen Thron für alle Zeiten.‹« SELA.

6Da priesen die Himmel deine Wundertat, o HERR,

dazu deine Treue in der Versammlung der Heiligen\textless sup
title=``d.h. Engel''\textgreater✲.

7Denn wer in der Wolkenhöhe kommt dem HERRN gleich,

ist dem HERRN vergleichbar unter den Gottessöhnen\textless sup
title=``d.h. Engeln''\textgreater✲,

8dem Gott, der gefürchtet ist im Kreise der Heiligen

und furchtbar über alle um ihn her?

9HERR, du Gott der Heerscharen, wer ist dir gleich?

Stark bist du, HERR, und deine Treue ist rings um dich her.

10Du herrschest über das Ungestüm des Meeres:

erheben sich seine Wogen -- du besänftigst sie.

11Du hast Rahab\textless sup title=``vgl. Hiob 9,13''\textgreater✲
zermalmt wie einen Durchbohrten,

deine Feinde mit deinem starken Arm zerstreut.

12Dein ist der Himmel, dein auch die Erde,

der Erdkreis und seine Fülle -- du hast sie gegründet;

13Norden und Süden -- du hast sie geschaffen,

der Thabor und Hermon bejubeln deinen Namen.

14Du hast einen Arm voll Heldenkraft:

stark ist deine Hand, deine Rechte hoch erhoben.

15Gerechtigkeit und Recht sind deines Thrones Stützen,

Gnade und Treue gehen vor dir her.

16Wohl dem Volk, das zu jubeln versteht,

das, o HERR, im Licht deines Angesichts wandelt:

17ob deinem Namen frohlocken sie allezeit,

ob deiner Gerechtigkeit sind sie hochgemut.

18Denn du bist ihr Ruhm und ihre Stärke,

und durch deine Gnade ragt hoch unser Horn\textless sup title=``vgl.
75,5''\textgreater✲;

19denn dem HERRN gehört unser Schild\textless sup title=``vgl.
47,10''\textgreater✲

und dem Heiligen Israels unser König.

20Damals✲ hast du in einem Gesicht zu deinem Frommen gesprochen:

»Ich habe die Hilfe einem Helden übertragen, einen Auserwählten über das
Volk erhöht:

21ich habe David als meinen Knecht gefunden,

mit meinem heiligen Öl ihn gesalbt,

22damit meine Hand beständig mit ihm sei

und mein Arm ihm Stärke verleihe.

23Kein Feind soll ihn überlisten

und kein Ruchloser ihn überwältigen;

24nein, seine Gegner will ich vor ihm zerschmettern,

und die ihn hassen, will ich niederschlagen.

25Doch mit ihm soll meine Treue und Gnade sein,

durch meinen Namen soll sein Horn hoch ragen;

26ich will das Meer unter seine Hand tun

und seine Rechte auf die Ströme legen.

27Er soll zu mir rufen: ›Mein Vater bist du,

mein Gott und der Fels meines Heils!‹

28So will auch ich ihn zum Erstgeborenen\textless sup title=``oder:
Erstling''\textgreater✲ machen,

zum höchsten unter den Königen der Erde.

29Für immer will ich ihm meine Gnade bewahren,

und mein Bund soll fest ihm bleiben;

30für immer will ich sein Geschlecht erhalten

und seinen Thron, solange der Himmel steht.

31Wenn seine Söhne mein Gesetz verlassen

und nicht in meinen Rechten wandeln,

32wenn sie meine Satzungen entweihen

und meine Gebote nicht beachten:

33so werde ich zwar mit der Rute ihren Abfall strafen

und ihre Übertretung mit Schlägen,

34doch meine Gnade will ich ihm nicht entziehen

und meine Treue nimmer verleugnen;

35ich werde meinen Bund nicht entweihen\textless sup title=``=~ungültig
machen''\textgreater✲

und den Ausspruch meiner Lippen nicht ändern.

36Ein für allemal hab' ich bei meiner Heiligkeit geschworen

niemals werde ich David belügen --:

37›Sein Geschlecht soll ewig bestehn,

sein Thron wie die Sonne vor mir,

38wie der Mond soll für immer er bleiben‹:

der Zeuge in Wolkenhöhen ist treu!« SELA.

39Und dennoch hast du verworfen und verstoßen,

hast Zorn gegen deinen Gesalbten betätigt;

40du hast den Bund mit deinem Knecht gebrochen,

seine Krone entweiht und zu Boden geschleudert;

41all seine Mauern hast du eingerissen,

seine festen Plätze in Trümmer gelegt.

42Es plündern ihn alle, die des Weges ziehen,

seinen Nachbarn ist er zum Spott geworden.

43Du hast den Arm seiner Dränger hoch erhoben

und all seine Feinde mit Freude erfüllt;

44auch hast du rückwärts gewandt sein scharfes Schwert

und im Krieg ihn nicht aufrecht gehalten (siegreich erhalten);

45du hast seinem Glanz ein Ende gemacht

und seinen Thron zu Boden gestürzt;

46du hast die Tage seiner Jugend verkürzt,

hast ihn mit Schande bedeckt. SELA.

47Bis wann, HERR, willst du dich ganz verbergen?

Bis wann soll lodern wie Feuer dein Zorn?

48Bedenke, wie kurz meine Lebenszeit ist,

wie vergänglich du alle Menschenkinder geschaffen!

49Wo ist ein Mensch, der leben bleibt und den Tod nicht sieht,

seine Seele errettet vor des Totenreichs Macht?

50Wo sind deine früheren Gnadenverheißungen, Allherr,

die du David zugeschworen in deiner Treue?

51Gedenke, Allherr, der Schmach deiner Knechte,

daß ich tragen muß in meinem Busen den Hohn von all den vielen Völkern,

52womit deine Feinde, o HERR, geschmäht uns haben,

womit{B} geschmäht sie haben die Fußstapfen✲ deines Gesalbten! * * *

53Gepriesen sei der HERR in Ewigkeit! Amen, ja Amen!

\hypertarget{viertes-buch-psalm-90-106}{%
\subsection{Viertes Buch (Psalm
90-106)}\label{viertes-buch-psalm-90-106}}

\hypertarget{der-ewige-gott-die-zuflucht-der-verguxe4nglichen-menschen}{%
\subsubsection{Der ewige Gott die Zuflucht der vergänglichen
Menschen}\label{der-ewige-gott-die-zuflucht-der-verguxe4nglichen-menschen}}

\hypertarget{section-89}{%
\section{90}\label{section-89}}

1Ein Gebet Moses, des Mannes Gottes. O Allherr, eine Zuflucht bist du
uns gewesen

von Geschlecht zu Geschlecht.

2Ehe die Berge geboren waren

und die Erde und die Welt von dir geschaffen wurden, ja von Ewigkeit zu
Ewigkeit bist du, o Gott.

3Du läßt die Menschen zum Staub zurückkehren

und sprichst: »Kommt wieder\textless sup title=``=~kehrt
zurück''\textgreater✲, ihr Menschenkinder!«

4Denn tausend Jahre sind in deinen Augen

wie der gestrige Tag, wenn er vergangen,{A} und wie eine Wache in der
Nacht.

5Du schwemmst sie hinweg; sie sind wie ein Schlaf am Morgen,

dem sprossenden Grase gleich:

6am Morgen grünt es und sprießt,

am Abend welkt es\textless sup title=``oder: man mäht es''\textgreater✲
ab, und es verdorrt.

7Denn wir vergehen durch deinen Zorn

und werden hinweggerafft durch deinen Grimm.

8Du hast unsre Sünden vor dich hingestellt,

unser geheimstes Denken ins Licht vor deinem Angesicht.

9Ach, alle unsre Tage fahren dahin durch deinen Grimm;

wir lassen unsre Jahre entschwinden wie einen Gedanken.

10Unsre Lebenszeit -- sie währt nur siebzig Jahre,

und, wenn's hoch kommt, sind's achtzig Jahre, und ihr Stolz ist Mühsal
und Nichtigkeit\textless sup title=``oder: Beschwer''\textgreater✲; denn
schnell ist sie enteilt, und wir fliegen davon.

11Doch wer bedenkt die Stärke deines Zorns

und deinen Grimm trotz deines furchtbaren Waltens?{A}

12Unsre Tage zählen, das lehre uns,

damit ein weises Herz wir gewinnen!

13Kehre dich wieder zu uns, o HERR! Wie lange noch (willst du zürnen)?

Erbarm dich deiner Knechte!

14Sättige früh uns am Morgen mit deiner Gnade\textless sup title=``oder:
Güte''\textgreater✲,

daß wir jubeln und uns freun unser Leben lang!

15Erfreue uns so viele Tage, wie du uns gebeugt hast,

so viele Jahre, wie Unglück wir erlebten!

16Laß deinen Knechten dein Walten sichtbar werden

und ihren Kindern deine Herrlichkeit!

17Und es ruhe auf uns die Huld des Allherrn, unsres Gottes,

und das Werk unsrer Hände segne bei uns! Ja, das Werk unsrer Hände
wollest du segnen!

\hypertarget{unter-der-obhut-des-huxf6chsten}{%
\subsubsection{Unter der Obhut des
Höchsten}\label{unter-der-obhut-des-huxf6chsten}}

\hypertarget{section-90}{%
\section{91}\label{section-90}}

1Wer da wohnt im Schirm des Höchsten

und im Schatten des Allmächtigen weilt,

2der spricht zum HERRN: »Meine Zuflucht und meine Burg,

mein Gott, auf den ich vertraue!«

3Denn er ist's, der dich rettet aus den Voglers Schlinge,

von der unheilvollen Pest.

4Mit seinen Fittichen deckt er dich,

und unter seinen Flügeln bist du geborgen, Schild und Panzer ist seine
Treue.

5Du brauchst dich nicht zu fürchten vor nächtlichem Schrecken,

vor dem Pfeil, der bei Tage daherfliegt,

6nicht vor der Pest, die im Finstern schleicht,

vor der Seuche, die mittags wütet.

7Ob tausend dir zur Seite fallen,

zehntausend zu deiner Rechten: an dich kommt's nicht heran;

8nein, lediglich mit eignen Augen wirst du's schauen

und zusehn, wie den Frevlern vergolten wird.

9Ja, du, o HERR, bist meine Zuflucht:

den Höchsten hast du{A} zum Schutz dir erwählt.

10Kein Übel wird dir begegnen,

kein Unheilsschlag deinem Zelte nahn;

11denn seine Engel wird er für dich entbieten,

daß sie dich behüten auf all deinen Wegen;

12auf den Armen werden sie dich tragen,

damit dein Fuß nicht stoße an einen Stein\textless sup title=``vgl. Mt
4,6''\textgreater✲;

13über Löwen und Ottern wirst du schreiten\textless sup title=``vgl. Lk
10,19''\textgreater✲,

wirst junge Löwen und Schlangen zertreten.

14»Weil er fest an mir hängt, so will ich ihn retten,

will ihn schützen, denn er kennt meinen Namen.

15Ruft er mich an, so will ich ihn erhören;

ich steh' ihm bei in der Not, will frei ihn machen und geehrt.{A}

16Mit langem Leben will ich ihn sättigen

und lasse ihn schauen mein Heil.«

\hypertarget{loblied-auf-das-gerechte-walten-gottes}{%
\subsubsection{Loblied auf das gerechte Walten
Gottes}\label{loblied-auf-das-gerechte-walten-gottes}}

\hypertarget{section-91}{%
\section{92}\label{section-91}}

1Ein Psalm; ein Lied für den Sabbattag. 2Köstlich ist's, dem HERRN zu
danken,

zu lobsingen deinem Namen, du Höchster,

3am Morgen deine Gnade zu künden

und deine Treue in den Nächten

4zum Klang zehnsaitigen Psalters und zur Harfe,

zum Saitenspiel auf der Zither.

5Denn du hast mich erfreut, o HERR, durch dein Tun,

ob den Werken deiner Hände juble ich.

6Wie groß sind deine Werke, o HERR,

gewaltig tief sind deine Gedanken!

7Nur ein unvernünft'ger Mensch\textless sup title=``oder: ein
Dummkopf''\textgreater✲ erkennt das nicht,

nur ein Tor sieht dies nicht ein.

8Wenn die Gottlosen sprossen wie Gras

und alle Übeltäter blühen, so ist's doch nur dazu, damit sie für immer
vertilgt werden.

9Du aber thronst auf ewig in der Höhe, HERR! 10Denn wahrlich deine
Feinde, o HERR,

ja wahrlich deine Feinde kommen um: alle Übeltäter werden zerstreut.

11Doch mein Horn erhöhst du wie das eines Wildstiers,

hast allzeit mich gesalbt mit frischem Öl;{A}

12mein Auge wird sich weiden an meinen Feinden;

vom Geschick der Bösen, die sich gegen mich erheben, wird mein Ohr mit
Freuden hören.

13Der Gerechte sproßt gleich dem Palmbaum,

er wächst wie auf dem Libanon die Zeder.

14Gepflanzt im Hause des HERRN,

sprossen sie reich in den Vorhöfen unsers Gottes,

15tragen Frucht noch im Greisenalter,

sind voller Saft und frischbelaubt,

16um zu verkünden, daß der HERR gerecht\textless sup title=``oder:
untadelig''\textgreater✲ ist,

mein Fels, an dem kein Unrecht haftet.

\hypertarget{die-herrlichkeit-gottes-des-ewigen-weltenkuxf6nigs}{%
\subsubsection{Die Herrlichkeit Gottes, des ewigen
Weltenkönigs}\label{die-herrlichkeit-gottes-des-ewigen-weltenkuxf6nigs}}

\hypertarget{section-92}{%
\section{93}\label{section-92}}

1Der HERR ist König\textless sup title=``vgl. 96,10''\textgreater✲! Er
hat sich gekleidet in Hoheit✲;

in Hoheit hat der HERR sich gekleidet, mit Kraft umgürtet, auch der
Erdkreis steht fest, so daß er nicht wankt.

2Fest steht dein Thron von Anbeginn,

von Ewigkeit her bist du.

3Fluten erhoben, o HERR, Fluten erhoben ihr Brausen,

Fluten werden (auch weiter) ihr Tosen erheben~--

4mächtiger als das Brausen gewaltiger Wasser,

mächtiger als die brandenden Meereswogen ist der HERR in der
Himmelshöhe!

5Was du verordnet\textless sup title=``oder: verheißen''\textgreater✲
hast, ist völlig zuverlässig,

deinem Hause gebührt Heiligkeit\textless sup title=``=~heilige
Scheu''\textgreater✲, o HERR, für die Dauer der Zeiten.

\hypertarget{bitte-um-rache-gegen-die-gottlosen-unterdruxfccker-des-volkes-gottes}{%
\subsubsection{Bitte um Rache gegen die gottlosen Unterdrücker des
Volkes
Gottes}\label{bitte-um-rache-gegen-die-gottlosen-unterdruxfccker-des-volkes-gottes}}

\hypertarget{section-93}{%
\section{94}\label{section-93}}

1Du Gott der Rache, o HERR,

du Gott der Rache, erscheine!

2Erhebe dich, Richter der Erde,

vergilt den Stolzen nach ihrem Tun!

3Wie lange noch sollen die Gottlosen, HERR,

wie lange noch sollen die Gottlosen jubeln,

4sollen sie geifern und trotzige Reden führen,

alle Übeltäter stolz sich brüsten?

5Dein Volk, o HERR, zertreten sie

und bedrücken dein Erbe✲;

6sie erwürgen Witwe und Fremdling

und morden die Waisen

7und sagen\textless sup title=``oder: denken''\textgreater✲ dabei:
»Nicht sieht es der HERR«

oder: »Nicht merkt es der Gott Jakobs.«

8Nehmt Verstand an, ihr Unvernünftigen im Volk,

und ihr Toren: wann wollt ihr Einsicht gewinnen?

9Der das Ohr gepflanzt, der sollte nicht hören?

Der das Auge gebildet, der sollte nicht sehn?

10Der die Völker erzieht, der sollte nicht strafen,

er, der die Menschen Erkenntnis lehrt?

11Der HERR kennt wohl die Gedanken der Menschen,

daß nur ein Hauch✲ sie sind.

12Wohl dem Manne, den du, HERR, in Zucht nimmst,

und den du aus deinem Gesetz belehrst,

13damit er sich Ruhe verschaffe vor Unglückstagen,

bis\textless sup title=``oder: während''\textgreater✲ dem Frevler die
Grube man gräbt!

14Denn der HERR wird sein Volk nicht verstoßen

und sein Erbe✲ nicht verlassen;

15denn Recht muß doch Recht bleiben,

und ihm{B} werden alle redlich Gesinnten sich anschließen.

16Wer leistet mir Beistand gegen die Bösen?

Wer tritt für mich ein gegen die Übeltäter?

17Wäre der HERR nicht mein Helfer gewesen,

so wohnte meine Seele wohl{A} schon im stillen Land.

18Sooft ich dachte: »Mein Fuß will wanken«,

hat deine Gnade, HERR, mich immer gestützt;

19bei der Menge meiner Sorgen in meiner Brust

haben deine Tröstungen mir das Herz erquickt.

20Sollte verbündet dir sein der Richterstuhl des Unheils,

der Verderben schafft durch Gesetzesverdrehung{A}?

21Sie tun sich ja zusammen gegen das Leben des Gerechten

und verurteilen unschuldig Blut.

22Doch der HERR ist mir zur festen Burg geworden,

mein Gott zu meinem Zufluchtsfelsen;

23er läßt ihren Frevel auf sie selber fallen

und wird sie ob ihrer Bosheit vertilgen: ja vertilgen wird sie der HERR,
unser Gott.

\hypertarget{lobpreis-gottes-beim-einzug-in-den-tempel-und-buuxdfrede}{%
\subsubsection{Lobpreis Gottes beim Einzug in den Tempel und
Bußrede}\label{lobpreis-gottes-beim-einzug-in-den-tempel-und-buuxdfrede}}

\hypertarget{section-94}{%
\section{95}\label{section-94}}

1Kommt, laßt uns dem HERRN zujubeln,

jauchzen dem Felsen unsers Heils!

2Laßt uns mit Dank vor sein Angesicht treten,

mit Liedern\textless sup title=``oder: Lobgesängen''\textgreater✲ ihm
jauchzen!

3Denn ein großer Gott ist der HERR

und ein großer König über alle Götter,

4er, in dessen Hand die Tiefen der Erde sind

und dem auch die Gipfel der Berge gehören;

5er, dem das Meer gehört: er hat's ja geschaffen,

und das Festland: seine Hände haben's gebildet.

6Kommt, laßt uns anbeten und niederfallen,

die Knie beugen vor dem HERRN, unserm Schöpfer!

7Denn er ist unser Gott,

und wir das Volk seiner Weide, die Herde seiner Hand\textless sup
title=``oder: Hut''\textgreater✲. Möchtet ihr heute doch hören auf seine
Stimme:

8»Verstockt nicht euer Herz wie bei Meriba,

wie am Tage von Massa in der Wüste\textless sup title=``2.Mose
17,1-7''\textgreater✲,

9woselbst eure Väter mich versuchten,

mich prüften, obwohl sie doch sahen mein Tun.

10Vierzig Jahre hegte ich Abscheu gegen dieses Geschlecht,

und sagte\textless sup title=``oder: dachte''\textgreater✲: ›Sie sind
ein Volk mit irrendem Herzen‹; sie aber wollten von meinen Wegen nichts
wissen.

11So schwur ich denn in meinem Zorn:

›Sie sollen nicht eingehn in meine Ruhstatt{A}!‹«\textless sup
title=``4.Mose 14,23''\textgreater✲

\hypertarget{lobpreis-gottes-als-des-weltherrschers-in-der-endzeit}{%
\subsubsection{Lobpreis Gottes als des Weltherrschers in der
Endzeit}\label{lobpreis-gottes-als-des-weltherrschers-in-der-endzeit}}

\hypertarget{section-95}{%
\section{96}\label{section-95}}

1Singet dem HERRN ein neues Lied,

singet dem HERRN, alle Lande!

2Singt dem HERRN, preist seinen Namen,

verkündet Tag für Tag sein Heil!

3Erzählt von seiner Herrlichkeit unter den Heiden,

unter allen Völkern seine Wundertaten!

4Denn groß ist der HERR und hoch zu preisen,

mehr zu fürchten als alle andern Götter;

5denn alle Götter der Heiden sind nichtige Götzen,

doch der HERR hat den Himmel geschaffen.

6Hoheit✲ und Pracht gehn vor ihm her,

Macht und Herrlichkeit füllen sein Heiligtum.

7Bringt dar dem HERRN, ihr Geschlechter der Völker,

bringt dar dem HERRN Ehre und Preis!

8Bringt dar dem HERRN die Ehre seines Namens,

bringt Opfergaben und kommt in seine Vorhöfe!

9Werft vor dem HERRN euch nieder in heiligem Schmuck,

erzittert vor ihm, alle Lande!

10Verkündet unter den Heiden: »Der HERR ist König!

Und feststehn wird der Erdkreis, daß er nicht wankt; richten wird er die
Völker nach Gebühr.«

11Des freue sich der Himmel, die Erde jauchze,

es brause das Meer und was darin wimmelt!

12Es jauchze die Flur und was auf ihr wächst!

Dann werden auch jubeln alle Bäume des Waldes

13vor dem HERRN, wenn er kommt,

wenn er kommt, zu richten die Erde. Richten wird er den Erdkreis mit
Gerechtigkeit und die Völker mit seiner Treue.

\hypertarget{gottes-regierungsantritt-und-kuxf6nigtum-in-der-endzeit}{%
\subsubsection{Gottes Regierungsantritt und Königtum in der
Endzeit}\label{gottes-regierungsantritt-und-kuxf6nigtum-in-der-endzeit}}

\hypertarget{section-96}{%
\section{97}\label{section-96}}

1Der HERR ist König\textless sup title=``vgl. 96,10''\textgreater✲! Des
juble die Erde,

die Menge der Meeresländer möge sich freuen!

2Gewölk und Dunkel umgibt ihn rings,

Gerechtigkeit und Recht sind seines Throns Stützen.

3Feuer geht vor ihm her

und rafft seine Feinde ringsum hinweg.

4Seine Blitze erleuchten den Erdkreis:

die Erde sieht's und erbebt in Angst.

5Die Berge zerschmelzen wie Wachs vor dem HERRN,

vor dem Herrscher der ganzen Erde.

6Die Himmel verkünden seine Gerechtigkeit

und alle Völker sehn seine Herrlichkeit.

7Zuschanden sollen werden alle Bilderverehrer,

die der nichtigen Götzen sich rühmen: alle Götter werfen vor ihm sich
nieder.

8Zion vernimmt es mit Freuden,

und die Töchter Judas jauchzen um deiner Gerichte willen, o HERR.

9Denn du, HERR, bist der Höchste über die ganze Erde,

hoch erhaben über alle Götter.

10Die den HERRN ihr lieb habt, hasset das Böse!

Er, der die Seelen seiner Frommen behütet, wird sie erretten aus der
Gottlosen Hand.

11Licht erstrahlt dem Gerechten

und Freude den redlich Gesinnten.

12Freut euch des HERRN, ihr Gerechten,

und preist seinen heiligen Namen!

\hypertarget{lobpreis-gottes-als-des-kuxf6nigs-und-gerechten-weltenrichters}{%
\subsubsection{Lobpreis Gottes als des Königs und gerechten
Weltenrichters}\label{lobpreis-gottes-als-des-kuxf6nigs-und-gerechten-weltenrichters}}

\hypertarget{section-97}{%
\section{98}\label{section-97}}

1Ein Psalm. Singet dem HERRN ein neues Lied!

Denn Wunderbares hat er vollbracht: den Sieg hat seine Rechte ihm
verschafft und sein heiliger\textless sup title=``oder:
furchtbarer''\textgreater✲ Arm.

2Der HERR hat kundgetan sein hilfreiches Tun,

vor den Augen der Völker seine Gerechtigkeit offenbart.

3Er hat gedacht seiner Gnade und Treue

gegenüber dem Hause Israel: alle Enden der Erde haben geschaut die
Heilstat{A} unsers Gottes.

4Jauchzet dem HERRN, alle Lande,

brecht in Jubel aus und spielt!

5Spielet zu Ehren des HERRN auf der Zither,

auf der Zither und mit lautem Gesang,

6mit Trompeten und Posaunenschall!

Jauchzt vor dem HERRN, dem König!

7Es tose das Meer und was darin wimmelt,

der Erdkreis und seine Bewohner!

8Die Ströme sollen in die Hände klatschen,

die Berge allesamt jubeln

9vor dem HERRN, wenn er kommt, zu richten die Erde.

Richten wird er den Erdkreis mit Gerechtigkeit und die Völker nach
Gebühr.

\hypertarget{preis-des-heiligen-gottes-des-allwaltenden-kuxf6nigs}{%
\subsubsection{Preis des heiligen Gottes, des allwaltenden
Königs}\label{preis-des-heiligen-gottes-des-allwaltenden-kuxf6nigs}}

\hypertarget{section-98}{%
\section{99}\label{section-98}}

1Der HERR ist König\textless sup title=``vgl. 96,10''\textgreater✲: es
zittern die Völker;

er thront über den Cheruben\textless sup title=``vgl.
80,2''\textgreater✲: es wankt die Erde.

2Groß ist der HERR in Zion

und hocherhaben über alle Völker.

3Preisen sollen sie\textless sup title=``oder: preisen soll
man''\textgreater✲ deinen Namen,

den großen und hehren -- heilig ist er --,

4und preisen die Stärke des Königs,

der da liebt das Recht. Du hast gerechte Ordnung fest gegründet, Recht
und Gerechtigkeit hast du in Jakob hergestellt.

5Erhebet den HERRN, unsern Gott,

und werft euch nieder vor dem Schemel seiner Füße: heilig ist er!

6Mose und Aaron waren unter seinen Priestern

und Samuel unter denen, die seinen Namen anriefen: sie riefen zum HERRN,
und er erhörte sie.

7In der Wolkensäule redete er zu ihnen;

sie wahrten seine Gebote, das Gesetz, das er ihnen gegeben.

8O HERR, unser Gott, du hast sie erhört,

ein verzeihender Gott bist du ihnen gewesen, doch auch ein strafender ob
ihrer Vergehen.

9Erhebet den HERRN, unsern Gott,

und werft euch nieder auf seinem heiligen Berge, denn heilig ist der
HERR, unser Gott!

\hypertarget{lobpreis-gottes-beim-einzug-in-den-tempel}{%
\subsubsection{Lobpreis Gottes beim Einzug in den
Tempel}\label{lobpreis-gottes-beim-einzug-in-den-tempel}}

\hypertarget{section-99}{%
\section{100}\label{section-99}}

1Ein Psalm als Dankbezeigung\textless sup title=``oder: bei Darbringung
eines Dankopfers''\textgreater✲. Jauchzet dem HERRN, alle Lande, 2dienet
dem HERRN mit Freuden,

kommt vor sein Angesicht mit Jubel!

3Erkennt, daß der HERR Gott ist!

Er hat uns geschaffen, und sein sind wir, sein Volk und die Herde, die
er weidet.{A}

4Zieht ein durch seine Tore mit Danken,

in seines Tempels Höfe mit Lobgesang, dankt ihm, preist seinen Namen!

5Denn freundlich ist der HERR, seine Gnade währt ewig

und seine Treue von Geschlecht zu Geschlecht.

\hypertarget{geluxfcbde-eines-herrschers-oder-fuxfcrstenspiegel}{%
\subsubsection{Gelübde eines Herrschers (oder:
Fürstenspiegel)}\label{geluxfcbde-eines-herrschers-oder-fuxfcrstenspiegel}}

\hypertarget{section-100}{%
\section{101}\label{section-100}}

1Von David, ein Psalm. Von Gnade und Recht will ich singen,

dir, o HERR, will ich spielen!

2Achten will ich auf fehllosen Wandel

wann wirst du zu mir kommen?{A} In Herzensreinheit will ich wandeln im
Innern meines Hauses.

3Ich will nicht mein Auge gerichtet halten

auf schandbare Dinge; das Tun der Abtrünnigen hasse ich: es soll mir
nicht anhaften.

4Ein falsches Herz soll fern von mir bleiben,

einen Bösen will ich nicht kennen.

5Wer seinen Nächsten heimlich verleumdet,

den will ich zum Schweigen bringen; wer stolze Augen hat und ein
hoffärtig Herz, den werde ich nicht ertragen.

6Meine Augen sollen blicken auf die Treuen im Lande:

die sollen bei mir wohnen; wer auf frommen Wege wandelt, der soll mir
dienen.

7Nicht darf inmitten meines Hauses weilen,

wer Trug verübt; wer Lügen redet, soll nicht bestehn vor meinen Augen.

8Jeden Morgen\textless sup title=``=~Tag für Tag''\textgreater✲ will ich
unschädlich machen

alle Frevler im Lande, um auszurotten aus der Stadt des HERRN alle
Übeltäter.

\hypertarget{buuxdffertiges-gebet-eines-leidenden-und-bitte-um-zions-wiederherstellung-fuxfcnfter-buuxdfpsalm}{%
\subsubsection{Bußfertiges Gebet eines Leidenden und Bitte um Zions
Wiederherstellung (Fünfter
Bußpsalm)}\label{buuxdffertiges-gebet-eines-leidenden-und-bitte-um-zions-wiederherstellung-fuxfcnfter-buuxdfpsalm}}

\hypertarget{section-101}{%
\section{102}\label{section-101}}

1Gebet eines Elenden, wenn er verzagt ist\textless sup title=``oder:
sich schwach fühlt''\textgreater✲ und seine Klage vor dem Herrn
ausschüttet. 2HERR, höre mein Gebet

und laß mein Schreien zu dir dringen!

3Verbirg dein Angesicht nicht vor mir

am Tage, wo mir angst ist! Neige dein Ohr mir zu am Tage, wo ich rufe;
erhöre mich eilends!

4Ach, meine Tage sind wie Rauch entschwunden

und meine Gebeine wie von Brand durchglüht;

5mein Herz ist versengt und verdorrt wie Gras,

so daß ich sogar vergesse, Speise zu genießen;

6infolge meines Ächzens und Stöhnens

klebt mein Gebein mir am Fleisch\textless sup title=``oder:
Leibe''\textgreater✲.

7Ich gleiche dem Wasservogel in der Wüste,

bin geworden wie ein Käuzlein in Trümmerstätten;

8ich finde keinen Schlaf und klage

wie ein einsamer Vogel auf dem Dache.

9Tagtäglich schmähen mich meine Feinde;

und die gegen mich toben, wünschen mir Unheil an.{A}

10Ach, Asche eß ich als Brot

und mische meinen Trank mit Tränen

11ob deinem Zorn und deinem Grimm;

denn du hast mich hochgehoben und niedergeschleudert.

12Meine Tage sind wie ein langgestreckter Schatten,

und ich selbst verdorre wie Gras!

13Du aber, HERR, thronst ewiglich,

und dein Gedächtnis bleibt von Geschlecht zu Geschlecht.

14Du wirst dich erheben, dich Zions erbarmen,

denn Zeit ist's, Gnade an ihm zu üben: die Stunde ist da

15- denn deine Knechte lieben Zions Steine,

und Weh erfaßt sie um seinen Schutt --,

16damit die Heiden fürchten lernen den Namen des HERRN

und alle Könige der Erde deine Herrlichkeit.

17Denn der HERR hat Zion wieder aufgebaut,

ist in seiner Herrlichkeit dort erschienen,

18hat dem Gebet der Verlass'nen sich zugewandt

und ihr Flehen nicht verachtet.

19Dies werde aufgeschrieben fürs kommende Geschlecht,

damit das neugeschaffne Volk den HERRN lobpreise,

20daß von seiner heiligen Höhe er herabgeschaut,

daß der HERR geblickt hat vom Himmel zur Erde,

21um das Seufzen der Gefangnen zu hören

und die dem Tode Geweihten{A} frei zu machen,

22damit man verkünde in Zion den Namen des HERRN

und seinen Ruhm in Jerusalem,

23wenn die Völker sich allzumal versammeln

und die Königreiche, um (Gott) dem HERRN zu dienen.

24Gelähmt hat er mir auf dem Wege die Kraft,

hat verkürzt meine Lebenstage.

25Nun fleh' ich: »Mein Gott, raffe mich nicht hinweg

in der Mitte meiner Tage, du, dessen Jahre währen für und für!«

26Vorzeiten hast du die Erde gegründet,

und die Himmel sind deiner Hände Werk:

27sie werden vergehen, du aber bleibst;

sie werden alle zerfallen wie ein Gewand, wie ein Kleid wirst du sie
verwandeln\textless sup title=``oder: wechseln''\textgreater✲, und so
werden sie sich wandeln✲.

28Du aber bleibst derselbe,

und deine Jahre nehmen kein Ende.

29Die Kinder deiner Knechte werden (sicher) wohnen,

und ihr Geschlecht wird fest bestehen vor dir.

\hypertarget{lobe-den-herrn-meine-seele}{%
\subsubsection{Lobe den Herrn, meine
Seele!}\label{lobe-den-herrn-meine-seele}}

\hypertarget{section-102}{%
\section{103}\label{section-102}}

1Von David. Lobe✲ den HERRN, meine Seele,

und all mein Inneres seinen heiligen Namen!

2Lobe den HERRN, meine Seele,

und vergiß nicht, was er dir Gutes getan!

3Der dir alle deine Schuld vergibt

und alle deine Gebrechen heilt;

4der dein Leben erlöst vom Verderben\textless sup title=``oder:
Tode''\textgreater✲,

der dich krönt mit Gnade und Erbarmen;

5der dein Alter mit guten Gaben sättigt,

daß, dem Adler gleich, sich erneut deine Jugend.

6Gerechtigkeit übt der HERR,

schafft allen Unterdrückten ihr Recht;

7er hat Mose seine Wege\textless sup title=``=~sein
Walten''\textgreater✲ kundgetan,

den Kindern Israel seine Großtaten.

8Barmherzig und gnädig ist der HERR,

voller Langmut und reich an Güte;

9er wird nicht ewig hadern

und den Zorn nicht immerdar festhalten;

10er handelt nicht mit uns\textless sup title=``oder: an
uns''\textgreater✲ nach unsern Sünden

und vergilt uns nicht nach unsern Missetaten;

11nein, so hoch der Himmel über der Erde ist,

so groß ist seine Gnade über denen, die ihn fürchten;

12so fern der Sonnenaufgang ist vom Niedergang,

läßt er unsre Verschuldungen fern von uns sein;

13wie ein Vater sich über die Kinder erbarmt,

so erbarmt der HERR sich derer, die ihn fürchten.

14Denn er weiß, welch ein Gebilde wir sind,

er denkt daran, daß wir Staub sind.

15Der Mensch -- dem Grase gleicht seine Lebenszeit,

wie die Blume des Feldes, so blüht er:

16wenn ein Windstoß über sie hinfährt, ist sie dahin,

und ihr Standort weiß nichts mehr von ihr.

17Doch die Gnade des HERRN erweist sich

von Ewigkeit zu Ewigkeit an denen, die ihn fürchten, und seine
Gerechtigkeit besteht für Kindeskinder

18bei denen, die seinen Bund bewahren

und seiner Gebote gedenken, um sie auszuführen.

19Der HERR hat seinen Thron im Himmel festgestellt,

und seine Königsmacht umschließt das All.

20Lobet✲ den HERRN, ihr seine Engel,

ihr starken Helden, die ihr sein Wort vollführt, gehorsam der Stimme
seines Gebots!

21Lobet den HERRN, alle seine Heerscharen,

ihr seine Diener, Vollstrecker seines Willens!

22Lobet den HERRN, alle seine Werke

an allen Orten seiner Herrschaft! Lobe den HERRN, meine Seele!

\hypertarget{die-herrlichkeit-gottes-in-der-natur}{%
\subsubsection{Die Herrlichkeit Gottes in der
Natur}\label{die-herrlichkeit-gottes-in-der-natur}}

\hypertarget{section-103}{%
\section{104}\label{section-103}}

1Lobe✲ den HERRN, meine Seele!

O HERR, mein Gott, wie bist du so groß!

In Erhabenheit\textless sup title=``oder: Majestät''\textgreater✲ und
Pracht bist du gekleidet,

2du, der in Licht sich hüllt wie in ein Gewand,

der den Himmel ausspannt wie ein Zeltdach,

3der die Balken seines Palastes im Wasser festlegt,

der Wolken macht zu seinem Wagen, einherfährt auf den Flügeln des
Windes;

4der Winde zu seinen Boten bestellt,

zu seinen Dienern lohendes Feuer\textless sup title=``=~flammende
Blitze''\textgreater✲.

5Er hat die Erde gegründet auf ihre Pfeiler\textless sup title=``oder:
Säulen''\textgreater✲,

so daß sie in alle Ewigkeit nicht wankt.

6Mit der Urflut gleich einem Kleide bedecktest du sie:

bis über die Berge standen die Wasser;

7doch vor deinem Schelten✲ flohen sie,

vor der Stimme deines Donners wichen sie angstvoll zurück.

8Da stiegen die Berge empor, und die Täler senkten sich

an den Ort, den du ihnen verordnet.

9Eine Grenze hast du gesetzt, die sie nicht überschreiten:

sie dürfen die Erde nicht nochmals bedecken.

10Quellen läßt er den Bächen zugehn:

zwischen den Bergen rieseln sie dahin;

11sie tränken alles Getier des Feldes,

die Wildesel löschen ihren Durst;

12an ihnen wohnen die Vögel des Himmels,

lassen ihr Lied aus den Zweigen erschallen.

13Er tränkt die Berge aus seinem Himmelspalast:

vom Segen deines Schaffens{A} wird die Erde satt.

14Gras läßt er sprossen für das Vieh

und Pflanzen für den Bedarf der Menschen, um Brotkorn aus der Erde
hervorgehn zu lassen und Wein, der des Menschen Herz erfreut;

15um jedes Antlitz erglänzen zu lassen vom Öl

und durch Brot das Herz des Menschen zu stärken.

16Es trinken sich satt die Bäume des HERRN,

die Zedern des Libanons, die er gepflanzt,

17woselbst die Vögel ihre Nester bauen,

der Storch, der Zypressen zur Wohnung wählt.

18Die hohen Berge gehören den Gemsen,

die Felsen sind der Klippdachse{A} Zuflucht.

19Er hat den Mond gemacht zur Bestimmung der Zeiten,

die Sonne, die ihren Niedergang kennt.

20Läßt du Finsternis entstehn, so wird es Nacht,

da regt sich alles Getier des Waldes:

21die jungen Löwen brüllen nach Raub,

indem sie von Gott ihre Nahrung fordern.

22Geht die Sonne auf, so ziehn sie sich zurück

und kauern\textless sup title=``oder: lagern sich''\textgreater✲ in
ihren Höhlen;

23dann geht der Mensch hinaus an seine Arbeit

und an sein Tagwerk bis zum Abend.

24Wie sind deiner Werke so viele, o HERR!

Du hast sie alle mit Weisheit geschaffen, voll ist die Erde von deinen
Geschöpfen\textless sup title=``oder: Gütern''\textgreater✲.

25Da ist das Meer, so groß und weit nach allen Seiten:

drin wimmelt es ohne Zahl von Tieren klein und groß.

26Dort fahren die Schiffe einher;

da ist der Walfisch, den du geschaffen, darin sich zu tummeln.{B}

27Sie alle schauen aus zu dir hin\textless sup title=``=~warten auf
dich''\textgreater✲,

daß du Speise ihnen gebest zu seiner Zeit;

28gibst du sie ihnen, so lesen sie auf;

tust deine Hand du auf, so werden sie satt des Guten;

29doch verbirgst du dein Angesicht, so befällt sie Schrecken;

nimmst du weg ihren Odem\textless sup title=``oder:
Geist''\textgreater✲, so sterben sie und kehren zurück zum Staub, woher
sie gekommen.

30Läßt du ausgehn deinen Odem\textless sup title=``oder:
Geist''\textgreater✲, so werden sie geschaffen,

und so erneust du das Antlitz der Erde.

31Ewig bleibe die Ehre des HERRN bestehn,

es freue der HERR sich seiner Werke!

32Blickt er die Erde an, so erbebt sie;

rührt er die Berge an, so stehn sie in Rauch.

33Singen will ich dem HERRN mein Leben lang,

will spielen\textless sup title=``oder: lobsingen''\textgreater✲ meinem
Gott, solange ich bin.

34Möge mein Sinnen ihm wohlgefällig sein:

ich will meine Freude haben am HERRN!

35Möchten die Sünder verschwinden vom Erdboden

und die Gottlosen nicht mehr sein! -- Lobe den HERRN, meine Seele!
Halleluja!

\hypertarget{gottes-heilstaten-an-israel-in-der-vorzeit}{%
\subsubsection{Gottes Heilstaten an Israel in der
Vorzeit}\label{gottes-heilstaten-an-israel-in-der-vorzeit}}

\hypertarget{section-104}{%
\section{105}\label{section-104}}

1Preiset den HERRN, ruft seinen Namen an,

macht seine Taten unter den Völkern bekannt!

2Singt ihm, spielet ihm,

redet von all seinen Wundern!

3Rühmt euch seines heiligen Namens!

Es mögen herzlich sich freun, die da suchen den HERRN!

4Fragt nach dem HERRN und seiner Stärke\textless sup title=``oder:
Macht''\textgreater✲,

suchet sein Angesicht allezeit!

5Gedenkt seiner Wunder, die er getan,

seiner Zeichen und der Urteilssprüche seines Mundes,

6ihr Kinder Abrahams, seines Knechtes,

ihr Söhne Jakobs, seine Erwählten!

7Er, der HERR, ist unser Gott,

über die ganze Erde ergehen seine Gerichte.

8Er gedenkt seines Bundes auf ewig,

des Wortes, das er geboten auf tausend Geschlechter,

9(des Bundes) den er mit Abraham geschlossen,

und des Eides, den er Isaak geschworen,

10den für Jakob er als Satzung bestätigt

und für Israel als ewigen Bund,

11da er sprach: »Dir will ich Kanaan geben,

das Land, das ich euch als Erbbesitztum zugeteilt!«\textless sup
title=``vgl. 1.Mose 15,18''\textgreater✲

12Damals waren sie noch ein kleines Häuflein,

gar wenige und nur Gäste im Lande;

13sie mußten wandern von Volk zu Volk,

von einem Reich zur andern Völkerschaft;

14doch keinem gestattete er, sie zu bedrücken,

ja Könige strafte er ihretwillen:

15»Tastet meine Gesalbten nicht an

und tut meinen Propheten nichts zuleide!«

16Dann, als er Hunger ins Land ließ kommen

und jegliche Stütze des Brotes zerbrach,

17da hatte er schon einen Mann vor ihnen her gesandt:

Joseph, der als Sklave verkauft war.

18Man hatte seine Füße gezwängt in den Stock,

in Eisen(-fesseln) war er gelegt,

19bis zu der Zeit, wo seine Weissagung eintraf

und der Ausspruch des HERRN ihn als echt erwies.

20Da sandte der König und ließ ihn entfesseln,

der Völkergebieter, und machte ihn frei;

21er bestellte ihn über sein Haus zum Herrn,

zum Gebieter über sein ganzes Besitztum;

22er sollte über seine Fürsten schalten nach Belieben

und seine höchsten Beamten Weisheit lehren.

23So kam denn Israel nach Ägypten,

und Jakob weilte als Gast im Lande Hams.

24Da machte Gott sein Volk gar fruchtbar

und ließ es stärker werden als seine Bedränger;

25er wandelte ihren Sinn, sein Volk zu hassen

und Arglist an seinen Knechten zu üben.

26Dann sandte er Mose, seinen Knecht,

und Aaron, den er erkoren;

27die richteten seine Zeichen unter ihnen aus

und die Wunder im Lande Hams:

28Er sandte Finsternis und ließ es dunkel werden;

doch sie achteten nicht auf seine Worte;

29er verwandelte ihre Gewässer in Blut

und ließ ihre Fische sterben;

30es wimmelte ihr Land von Fröschen

bis hinein in ihre Königsgemächer;

31er gebot, da kamen Bremsenschwärme,

Stechfliegen über ihr ganzes Gebiet;

32er gab ihnen Hagelschauer als Regen,

sandte flammendes Feuer in ihr Land;

33er schlug ihre Reben und Feigenbäume

und zerbrach die Bäume in ihrem Gebiet;

34er gebot, da kamen die Heuschrecken

und die Grillen in zahlloser Menge,

35die verzehrten alle Gewächse im Land

und fraßen die Früchte ihrer Felder.

36Dann schlug er alle Erstgeburt im Lande,

die Erstlinge all ihrer Manneskraft.

37Nun ließ er sie ausziehn mit Silber und Gold,

und kein Strauchelnder{A} war in seinen Stämmen;

38Ägypten war ihres Auszugs froh,

denn Angst vor ihnen hatte sie befallen.

39Er breitete Gewölk aus als Decke

und Feuer, um ihnen die Nacht zu erhellen;

40auf Moses Bitte ließ er Wachteln kommen

und sättigte sie mit Himmelsbrot;

41er spaltete einen Felsen: da rannen Wasser

und flossen durch die Steppen als Strom;

42denn er gedachte seines heiligen Wortes,

dachte an Abraham, seinen Knecht.

43So ließ er sein Volk in Freuden ausziehn,

unter Jubel seine Erwählten;

44dann gab er ihnen die Länder der Heiden,

und was die Völker erworben, das nahmen sie in Besitz,

45auf daß sie seine Gebote halten möchten

und seine Gesetze bewahrten. Halleluja!

\hypertarget{gottes-gnade-und-israels-undank}{%
\subsubsection{Gottes Gnade und Israels
Undank}\label{gottes-gnade-und-israels-undank}}

\hypertarget{section-105}{%
\section{106}\label{section-105}}

1Halleluja! Preiset den HERRN\textless sup title=``oder: danket dem
HERRN''\textgreater✲, denn er ist freundlich,

ja ewiglich währt seine Gnade!

2Wer kann des HERRN Machttaten gebührend preisen

und kundtun all seinen Ruhm?

3Wohl denen, die am Recht festhalten,

und dem, der Gerechtigkeit übt zu jeder Zeit!

4Gedenke meiner, o HERR, mit der Liebe zu deinem Volk,

nimm dich meiner an mit deiner Hilfe,

5daß ich schau' meine Lust am Glück deiner Erwählten,

an der Freude deines Volkes Anteil habe und glücklich mich preise mit
deinem Eigentumsvolke!

6Wir haben gesündigt gleich unsern Vätern,

wir haben gefehlt und gottlos gehandelt.

7Unsre Väter in Ägypten achteten nicht auf deine Wunder,

gedachten nicht der Fülle deiner Gnadenerweise, waren widerspenstig
gegen den Höchsten schon am Schilfmeer;

8dennoch half er ihnen um seines Namens willen,

um seine Heldenkraft zu erweisen.

9Er schalt\textless sup title=``oder: bedrohte''\textgreater✲ das
Schilfmeer: da ward es trocken,

und er ließ sie ziehn durch die Fluten wie über die Trift.

10So rettete er sie aus der Hand des Verfolgers

und erlöste sie aus der Gewalt des Feindes:

11die Fluten bedeckten ihre Bedränger,

nicht einer von ihnen blieb übrig.

12Da glaubten sie an seine Worte,

besangen seinen Ruhm.

13Doch schnell vergaßen sie seine Taten

und warteten seinen Ratschluß nicht ab;

14sie fröhnten ihrem Gelüst in der Wüste

und versuchten Gott in der Einöde:

15da gewährte er ihnen ihr Verlangen,

sandte aber die Seuche{A} gegen ihr Leben.

16Dann wurden sie eifersüchtig auf Mose im Lager,

auf Aaron, den Geweihten des HERRN:

17da tat die Erde sich auf und verschlang Dathan

und begrub die ganze Rotte Abirams,

18Feuer verbrannte ihre Rotte,

Flammen verzehrten die Frevler.

19Sie machten sich ein Kalb✲ am Horeb

und warfen vor einem Gußbild sich nieder

20und vertauschten so die Herrlichkeit ihres Gottes

mit dem Bildnis eines Stieres, der Gras frißt.

21Sie hatten Gott, ihren Retter, vergessen,

der große Dinge getan in Ägypten,

22Wunderzeichen im Lande Hams,

furchtbare Taten am Schilfmeer.

23Da gedachte er sie zu vertilgen,

wenn nicht Mose, sein Auserwählter, mit Fürbitte vor ihn hingetreten
wäre, um seinen Grimm vom Vernichten abzuwenden.

24Sodann verschmähten sie das herrliche Land

und schenkten seiner Verheißung keinen Glauben,

25sondern murrten in ihren Zelten,

gehorchten nicht der Weisung des HERRN.

26Da erhob er seine Hand gegen sie zum Schwur,

sie in der Wüste niederzuschlagen,

27ihre Nachkommen unter die Heiden niederzuwerfen

und sie rings zu zerstreuen in die Länder.

28Dann hängten sie sich an den Baal-Peor

und aßen Opferfleisch der toten (Götzen)

29und erbitterten ihn durch ihr ganzes Tun.

Als nun ein Sterben unter ihnen ausbrach,

30trat Pinehas auf und hielt Gericht\textless sup title=``oder: legte
sich ins Mittel''\textgreater✲:

da wurde dem Sterben Einhalt getan.

31Das wurde ihm angerechnet zur Gerechtigkeit

von Geschlecht zu Geschlecht in Ewigkeit.~--

32Dann erregten sie Gottes Zorn am Haderwasser,

und Mose erging es übel um ihretwillen;

33denn weil sie dem Geiste Gottes widerstrebten,

hatte er unbedacht mit seinen Lippen geredet.

34Sie vertilgten auch die Völker nicht,

von denen der HERR es ihnen geboten,

35sondern traten mit den Heiden in Verkehr

und gewöhnten sich an deren (böses) Tun

36und dienten ihren Götzen:

die wurden ihnen zum Fallstrick.

37Ja, sie opferten ihre Söhne

und ihre Töchter den bösen Geistern

38und vergossen unschuldig Blut {[}das Blut ihrer Söhne und Töchter, die
sie den Götzen Kanaans opferten{]}:

so wurde das Land durch Blutvergießen entweiht.

39Sie wurden unrein durch ihr Verhalten

und verübten Abfall durch ihr Tun.~--

40Da entbrannte der Zorn des HERRN gegen sein Volk,

und Abscheu fühlte er gegen sein Erbe✲;

41er ließ sie in die Hand der Heiden fallen,

so daß ihre Hasser über sie herrschten;

42ihre Feinde bedrängten sie hart,

so daß sie sich beugen mußten unter deren Hand.

43Oftmals zwar befreite er sie,

doch sie blieben widerspenstig gegen seinen Ratschluß und sanken immer
tiefer durch ihre Schuld.

44Er aber nahm sich ihrer Drangsal an,

sooft er ihr Wehgeschrei hörte,

45und gedachte seines Bundes ihnen zugut,

fühlte Mitleid nach seiner großen Güte

46und ließ sie Erbarmen finden

bei allen, die sie gefangen hielten.

47O hilf uns, HERR, unser Gott,

und bring uns wieder zusammen aus den Heiden, damit wir deinem heiligen
Namen danken, uns glücklich preisen, deinen Ruhm zu künden! * * *

48Gepriesen sei der HERR, der Gott Israels,

von Ewigkeit zu Ewigkeit! Und alles Volk sage »Amen!« Halleluja!

\hypertarget{fuxfcnftes-buch-ps-107-150}{%
\subsection{Fünftes Buch (Ps
107-150)}\label{fuxfcnftes-buch-ps-107-150}}

\hypertarget{lobpreis-gottes-des-retters-aus-aller-not}{%
\subsubsection{Lobpreis Gottes, des Retters aus aller
Not}\label{lobpreis-gottes-des-retters-aus-aller-not}}

\hypertarget{section-106}{%
\section{107}\label{section-106}}

1»Danket dem HERRN, denn er ist freundlich,

ja, ewiglich währt seine Gnade«:

2so sollen die vom HERRN Erlösten sprechen,

die er befreit hat aus Drangsal\textless sup title=``oder: aus
Feindeshand''\textgreater✲

3und die er gesammelt aus den Ländern

vom Aufgang her und vom Niedergang, vom Norden her und vom Meer✲.{A}

4Sie irrten umher in der Wüste, der Öde,

und fanden den Weg nicht zu einer Wohnstatt;

5gequält vom Hunger und vom Durst,

wollte ihre Seele in ihnen verschmachten✲.

6Da schrien sie zum HERRN in ihrer Not,

und er rettete sie aus ihren Ängsten

7und leitete sie auf richtigem Wege,

daß sie kamen zu einer bewohnten Ortschaft:~--

8die mögen danken dem HERRN für seine Güte

und für seine Wundertaten an den Menschenkindern,

9daß er die lechzende Seele gesättigt

und die hungernde Seele gefüllt hat mit Labung.

10Die da saßen in Finsternis und Todesnacht,

gefangen in Elend und Eisenbanden~--

11denn sie hatten Gottes Geboten getrotzt

und den Ratschluß\textless sup title=``oder: Willen''\textgreater✲ des
Höchsten verachtet,

12so daß er ihren Sinn durch Leiden beugte,

daß sie niedersanken und keinen Helfer hatten --;

13da schrien sie zum HERRN in ihrer Not,

und er rettete sie aus ihren Ängsten;

14er führte sie heraus aus Finsternis und Todesnacht

und zersprengte ihre Fesseln:~--

15die mögen danken dem HERRN für seine Güte

und für seine Wundertaten an den Menschenkindern,

16daß er eherne Türen zerbrochen

und eiserne Riegel zerschlagen.

17Die da krank waren infolge ihres Sündenlebens

und wegen ihrer Verfehlungen leiden mußten~--

18vor jeglicher Speise hatten sie Widerwillen,

so daß sie den Pforten des Todes nahe waren --;

19da schrien sie zum HERRN in ihrer Not,

und er rettete sie aus ihren Ängsten;

20er sandte sein Wort, sie gesund zu machen,

und ließ sie aus ihren Gruben\textless sup title=``oder:
Gräbern?''\textgreater✲ entrinnen:~--

21die mögen danken dem HERRN für seine Güte

und für seine Wundertaten an den Menschenkindern;

22sie mögen Opfer des Dankes bringen

und seine Taten mit Jubel verkünden!

23Die aufs Meer gefahren waren in Schiffen,

auf weiten Fluten Handelsgeschäfte trieben,

24die haben das Walten des HERRN geschaut

und seine Wundertaten auf hoher See.

25Denn er gebot und ließ einen Sturm entstehn,

der hoch die Wogen des Meeres türmte:

26sie stiegen empor zum Himmel und fuhren hinab in die Tiefen,

so daß ihr Herz vor Angst verzagte;

27sie wurden schwindlig und schwankten wie Trunkne,

und mit all ihrer Weisheit war's zu Ende:~--

28da schrien sie zum HERRN in ihrer Not,

und er befreite sie aus ihren Ängsten;

29er stillte das Ungewitter zum Säuseln,

und das Toben der Wogen verstummte;

30da wurden sie froh, daß es still geworden,

und er führte sie zum ersehnten Hafen:~--

31die mögen danken dem HERRN für seine Güte

und für seine Wundertaten an den Menschenkindern;

32sie mögen ihn erheben in der Volksgemeinde

und im Kreise der Alten ihn preisen!

33Er wandelte Ströme zur Wüste

und Wasserquellen zu dürrem Land,

34fruchtbares Erdreich zu salziger Steppe

wegen der Bosheit seiner Bewohner.

35Wiederum machte er wüstes Land zum Wasserteich

und dürres Gebiet zu Wasserquellen

36und ließ dort Hungrige seßhaft werden,

so daß sie eine Stadt zum Wohnsitz bauten

37und Felder besäten und Weinberge pflanzten,

die reichen Ertrag an Früchten brachten;

38und er segnete sie, daß sie stark sich mehrten,

und ließ ihres Viehs nicht wenig sein.

39Dann aber nahmen sie ab und wurden gebeugt

durch den Druck des Unglücks und Kummers;

40»über Edle goß er Verachtung aus

und ließ sie irren in pfadloser Öde«\textless sup title=``vgl. Hiob
12,21.24''\textgreater✲.

41Den Armen aber hob er empor aus dem Elend

und machte seine Geschlechter wie Kleinviehherden.

42»Die Gerechten sehen's und freuen sich,

alle Bosheit aber muß schließen ihren Mund«\textless sup title=``vgl.
Hiob 22,19; 5,16''\textgreater✲.

43Wer ist weise? Der beachte dies

und lerne die Gnadenerweise des HERRN verstehn!

\hypertarget{lob-der-gnade-gottes-und-bitte-um-hilfe}{%
\subsubsection{Lob der Gnade Gottes und Bitte um
Hilfe}\label{lob-der-gnade-gottes-und-bitte-um-hilfe}}

\hypertarget{section-107}{%
\section{108}\label{section-107}}

1Ein Lied, ein Psalm Davids. 2Mein Herz ist getrost, o Gott:

singen will ich und spielen! Wach auf, meine Seele!{A}

3Wacht auf, Harfe und Zither:

ich will das Morgenrot wecken!

4Ich will dich preisen unter den Völkern, o HERR,

und dir lobsingen unter den Völkerschaften!

5Denn groß bis über den Himmel hinaus ist deine Gnade,

und bis an die Wolken geht deine Treue.

6Erhebe dich über den Himmel hinaus, o Gott,

und über die ganze Erde (verbreite sich) deine Herrlichkeit!

7Daß deine Geliebten gerettet werden,

hilf uns mit deiner Rechten, erhör' uns!

8Gott hat in\textless sup title=``oder: bei''\textgreater✲ seiner
Heiligkeit gesprochen✲:

»(Als Sieger) will ich frohlocken, will Sichem verteilen und das Tal von
Sukkoth (als Beutestück) vermessen.

9Mein ist Gilead, mein auch Manasse,

und Ephraim ist meines Hauptes Schutzwehr, Juda mein Herrscherstab.

10Moab (dagegen) ist mein Waschbecken,

auf Edom werf' ich meinen Schuh; über das Philisterland will (als
Sieger) ich jauchzen.«

11Wer führt mich hin zur festen Stadt,

wer geleitet mich bis Edom?

12Hast nicht du uns, o Gott, verworfen

und ziehst nicht aus, o Gott, mit unsern Heeren?

13O schaffe uns Hilfe gegen den Feind!

Denn nichtig ist Menschenhilfe.

14Mit Gott werden wir Taten vollführen,

und er wird unsre Bedränger zertreten.

\hypertarget{verfluchung-gottloser-feinde}{%
\subsubsection{Verfluchung gottloser
Feinde}\label{verfluchung-gottloser-feinde}}

\hypertarget{section-108}{%
\section{109}\label{section-108}}

1Dem Musikmeister; von David ein Psalm. Du Gott, dem mein Lobpreis gilt,
bleibe nicht stumm! 2Denn Frevlermund und Lügenmaul

haben sich gegen mich aufgetan, mit trügerischer Zunge zu mir geredet;

3mit Worten des Hasses haben sie mich umschwirrt

und ohne Ursach' mich angegriffen;

4für meine Liebe befeinden sie mich,

während ich doch (stets für sie) bete;

5ja sie haben mir Böses für Gutes vergolten

und Haß für meine Liebe erwiesen.

6Bestell' einen Frevler zum Richter gegen ihn,

und ein Ankläger\textless sup title=``oder: Widersacher''\textgreater✲
steh' ihm zur Rechten!

7Als schuldig soll er hervorgehn aus dem Gericht

und sogar sein Gebet ihm als Sünde gelten!

8Seiner Lebenstage müssen nur wenige sein,

und sein Amt ein andrer empfangen!

9Seine Kinder müssen zu Waisen werden

und seine Frau eine Witwe!

10Seine Kinder müssen unstet umherziehn und betteln

und vertrieben werden aus ihres Vaterhauses Trümmern!

11Sein Gläubiger lege Beschlag auf alles, was er hat,

und Fremde✲ müssen seine Habe plündern!

12Er finde keinen, der ihm Schonung gewährt,

und niemand habe Erbarmen mit seinen Waisen!

13Sein Nachwuchs müsse der Ausrottung verfallen:

schon im zweiten Gliede müsse ihr Name erlöschen!

14Der Verschuldung seiner Väter werde beim HERRN gedacht,

und die Sünde seiner Mutter bleibe ungetilgt!

15Sie müssen beständig dem HERRN vor Augen stehn,

und er tilge ihr Gedächtnis aus von der Erde\textless sup title=``oder:
im Lande''\textgreater✲,

16dieweil er nicht daran dachte, Liebe zu üben,

vielmehr den Elenden und Armen verfolgte und den hoffnungslos Verzagten,
ihn vollends zu töten.

17Er liebte den Fluch: so treffe er ihn!

Er hatte am Segen keine Freude: so bleib' er ihm fern!

18Er zog den Fluch an wie sein Kleid:

so dringe er ihm in den Leib wie Wasser und wie Öl in seine Gebeine;

19er werde ihm wie der Mantel, in den er sich hüllt,

wie der Gürtel, den er sich ständig umlegt!

20Dies sei meiner Widersacher Lohn von seiten des HERRN

und derer, die Böses gegen mich reden!

21Du aber, HERR, mein Gott, tritt für mich ein um deines Namens willen!

Weil deine Gnade köstlich ist, errette mich!

22Denn elend bin ich und arm,

und mein Herz ist verwundet in meiner Brust.

23Wie ein Schatten, wenn er sich dehnt\textless sup title=``oder: neigt;
102,12''\textgreater✲, so schwinde ich hin,

bin vom Sturm verweht einer Heuschrecke gleich;

24meine Knie wanken vom Fasten,

mein Leib ist abgemagert, ohne Fett;

25und ich -- den Leuten bin ich zum Hohn geworden:

sehen sie mich, so schütteln sie höhnend den Kopf.

26Stehe mir bei, o HERR, mein Gott,

hilf mir nach deiner Gnade!

27Laß sie erkennen, daß dies deine Hand ist,

daß du, HERR, selbst es so gefügt hast!

28Sie mögen fluchen, du aber wollest segnen;

erheben sie sich, so laß sie zuschanden werden, dein Knecht aber müsse
sich freuen!

29Laß meine Widersacher in Schmach sich kleiden

und ihre Schande umtun wie einen Mantel!

30Laut soll mein Mund dem HERRN Dank sagen,

und inmitten vieler will ich ihn preisen;

31denn er steht dem Armen zur Rechten,

um ihn zu retten vor denen, die ihn schuldig sprechen.

\hypertarget{gottes-botschaft-an-den-priesterkuxf6nig}{%
\subsubsection{Gottes Botschaft an den
Priesterkönig}\label{gottes-botschaft-an-den-priesterkuxf6nig}}

\hypertarget{section-109}{%
\section{110}\label{section-109}}

1Von David, ein Psalm. So lautet der Ausspruch des HERRN an meinen
Herrn:

»Setze dich zu meiner Rechten, bis ich deine Feinde hinlege zum Schemel
für deine Füße!«

2Dein machtvolles Zepter wird der HERR von Zion hinausstrecken:

herrsche inmitten deiner Feinde!

3Dein Volk wird voller Willigkeit sein

am Tage deines Heereszuges; in heiligem Schmuck, wie aus des Frührots
Schoß der Tau, wird dir kommen deine junge Mannschaft.{B}

4Geschworen hat der HERR

und wird sich's nicht leid sein lassen: »Du sollst ein Priester in
Ewigkeit sein nach der Weise Melchisedeks.«{A}\textless sup
title=``1.Mose 14,18-20''\textgreater✲

5Der Allherr, der dir zur Rechten steht,

wird Könige zerschmettern am Tage seines Zorns;

6Gericht wird er unter den Völkern halten, füllt alles mit Leichen an,

zerschmettert ein Haupt\textless sup title=``oder:
Häupter''\textgreater✲ auf weitem Gefilde.

7Aus dem Bach am Wege wird er trinken;

darum wird er das Haupt hoch halten.

\hypertarget{lobpreis-der-leiblichen-und-geistlichen-segnungen-gottes}{%
\subsubsection{Lobpreis der leiblichen und geistlichen Segnungen
Gottes}\label{lobpreis-der-leiblichen-und-geistlichen-segnungen-gottes}}

\hypertarget{section-110}{%
\section{111}\label{section-110}}

1Halleluja! Preisen will ich den HERRN von ganzem Herzen

im Kreise der Frommen und in der Gemeinde.

2Groß sind die Werke des HERRN,

erforschenswert für alle, die Gefallen an ihnen haben.

3Ruhmvoll und herrlich ist sein Tun,

und seine Gerechtigkeit bleibt ewig bestehn.

4Er hat ein Gedächtnis seiner Wundertaten gestiftet;

gnädig und barmherzig ist der HERR.

5Speise hat er denen gegeben, die ihn fürchten;

er gedenkt seines Bundes ewiglich.

6Sein machtvolles Walten hat er kundgetan seinem Volk,

indem er ihnen das Erbe der Heiden gab.

7Die Werke seiner Hände sind Treue und Recht;

unwandelbar sind alle seine Gebote,

8festgestellt für immer, für ewig,

gegeben mit Treue und Redlichkeit.

9Erlösung hat er seinem Volk gesandt,

seinen Bund auf ewig verordnet; heilig und furchtgebietend ist sein
Name.

10Die Furcht des HERRN ist der Weisheit Anfang\textless sup title=``Spr
9,10''\textgreater✲,

eine treffliche Einsicht für alle, die sie üben: sein\textless sup
title=``d.h. Gottes''\textgreater✲ Ruhm besteht in Ewigkeit.

\hypertarget{der-segen-der-gottesfurcht-und-barmherzigkeit}{%
\subsubsection{Der Segen der Gottesfurcht und
Barmherzigkeit}\label{der-segen-der-gottesfurcht-und-barmherzigkeit}}

\hypertarget{section-111}{%
\section{112}\label{section-111}}

1Halleluja! Wohl dem Menschen, der den HERRN fürchtet,

an seinen Geboten herzliche Freude hat!

2Seine Nachkommen werden im Lande\textless sup title=``oder: auf
Erden''\textgreater✲ gewaltig sein,

als ein Geschlecht von Frommen wird man sie segnen.

3Wohlstand und Fülle herrscht in seinem Hause,

und seine Gerechtigkeit besteht für immer.

4Den Frommen geht er auf wie ein Licht in der Finsternis,

als gnädig, barmherzig und gerecht.

5Glücklich der Mann, der Barmherzigkeit übt und darleiht!

Er wird sein Recht behaupten vor Gericht;

6denn nimmermehr wird er wanken:

in ew'gem Gedächtnis bleibt der Gerechte.

7Vor bösem Leumund\textless sup title=``oder:
Unglücksbotschaft''\textgreater✲ braucht er sich nicht zu fürchten;

sein Herz ist fest, voll Vertraun auf den HERRN.

8Getrost ist sein Herz, er fürchtet sich nicht,

bis er sieht seine Lust an seinen Bedrängern.

9Reichlich teilt er aus und spendet den Armen;

seine Gerechtigkeit besteht fest für immer\textless sup title=``vgl.
2.Kor 9,9''\textgreater✲, sein Horn ragt hoch empor in Ehren.

10Der Gottlose sieht es und ärgert sich;

er knirscht mit den Zähnen und vergeht; der Gottlosen Wünsche bleiben
unerfüllt.

\hypertarget{lobpreis-des-erhabenen-und-gnuxe4digen-gottes}{%
\subsubsection{Lobpreis des erhabenen und gnädigen
Gottes}\label{lobpreis-des-erhabenen-und-gnuxe4digen-gottes}}

\hypertarget{section-112}{%
\section{113}\label{section-112}}

1Halleluja! Lobet, ihr Knechte✲ des HERRN,

lobet den Namen des HERRN!

2Gepriesen sei der Name des HERRN

von nun an bis in Ewigkeit!

3Vom Aufgang der Sonne bis zu ihrem Niedergang

sei gelobt der Name des HERRN!

4Erhaben über alle Völker ist der HERR,

den Himmel überragt seine Herrlichkeit!

5Wer ist dem HERRN gleich, unserm Gott,

der da thront in der Höhe,

6der niederschaut in die Tiefe,

im Himmel und auf Erden?

7Er hebt aus dem Staub den Geringen empor

und erhöht aus dem Schmutz den Armen,

8um ihn sitzen zu lassen neben Edlen,

neben den Edlen seines Volks.

9Er verleiht der kinderlosen Gattin Hausrecht,

macht sie zur fröhlichen Mutter von Kindern. Halleluja!

\hypertarget{gottes-wundermacht-beim-durchzug-der-israeliten-durch-das-rote-meer-und-durch-den-jordan}{%
\subsubsection{Gottes Wundermacht beim Durchzug der Israeliten durch das
Rote Meer und durch den
Jordan}\label{gottes-wundermacht-beim-durchzug-der-israeliten-durch-das-rote-meer-und-durch-den-jordan}}

\hypertarget{section-113}{%
\section{114}\label{section-113}}

1Halleluja! Als Israel aus Ägypten auszog,

Jakobs Haus aus dem Volk fremder Sprache,

2da ward Juda sein\textless sup title=``d.h. Gottes''\textgreater✲
Heiligtum,

Israel sein Herrschaftsgebiet.

3Das Meer sah es und floh\textless sup title=``vgl. 2.Mose
14,21''\textgreater✲,

der Jordan wandte sich rückwärts\textless sup title=``vgl. Jos
3,14-17''\textgreater✲,

4die Berge hüpften wie Widder,

die Hügel gleichwie Lämmer.

5Was war dir, o Meer, daß du flohest,

dir, Jordan, daß du dich rückwärts wandtest?

6(Was war euch) ihr Berge, daß ihr hüpftet wie Widder,

ihr Hügel gleichwie Lämmer?

7Vor dem Anblick des Herrn erbebe, du Erde,

vor dem Anblick des Gottes Jakobs,

8der Felsen wandelt zum Wasserteich,

Kieselgestein zum sprudelnden Quell!

\hypertarget{dem-lebendigen-gott-gebuxfchrt-allein-die-ehre}{%
\subsubsection{Dem lebendigen Gott gebührt allein die
Ehre}\label{dem-lebendigen-gott-gebuxfchrt-allein-die-ehre}}

\hypertarget{section-114}{%
\section{115}\label{section-114}}

1Nicht uns, o HERR, nicht uns,

nein, deinem Namen schaffe Ehre um deiner Gnade, um deiner Treue willen!

2Warum sollen die Heiden sagen:

»Wo ist denn ihr Gott?«

3Unser Gott ist ja im Himmel:

alles, was ihm gefällt, vollführt er.

4Ihre Götzen sind Silber und Gold,

Machwerk von Menschenhänden.

5Sie haben einen Mund und können nicht reden,

haben Augen und sehen nicht;

6sie haben Ohren und können nicht hören,

haben eine Nase und riechen nicht;

7mit ihren Händen können sie nicht greifen,

mit ihren Füßen nicht gehen; kein Laut dringt aus ihrer Kehle.

8Ihnen gleich sind ihre Verfertiger,

jeder, der auf sie vertraut.\textless sup title=``Vgl.
135,15-18''\textgreater✲

9Du, Israel, vertraue auf den HERRN!~--

Ihre Hilfe und ihr Schild ist er.

10Ihr vom Hause Aarons, vertraut auf den HERRN!~--

Ihre Hilfe und ihr Schild ist er.

11Ihr, die ihr fürchtet den HERRN, vertraut auf den HERRN!~--

Ihre Hilfe und ihr Schild ist er.

12Der HERR hat unser gedacht: er wird segnen,

segnen das Haus Israels, segnen das Haus Aarons;

13er wird segnen, die den HERRN fürchten,

die Kleinen samt den Großen\textless sup title=``=~die Jungen samt den
Alten''\textgreater✲.

14Der HERR wolle euch mehren,

euch selbst und eure Kinder!

15Gesegnet seid\textless sup title=``oder: seiet''\textgreater✲ ihr vom
HERRN,

der Himmel und Erde geschaffen!

16Der Himmel ist der Himmel des Allherrn,

die Erde aber hat er den Menschen gegeben.

17Nicht die Toten preisen den HERRN

und keiner, der ins stille Land gefahren.

18Doch wir, wir preisen den HERRN

von nun an bis in Ewigkeit. Halleluja!

\hypertarget{danklied-und-geluxfcbde-eines-aus-todesgefahr-geretteten-bei-darbringung-des-dankopfers}{%
\subsubsection{Danklied und Gelübde eines aus Todesgefahr Geretteten
(bei Darbringung des
Dankopfers)}\label{danklied-und-geluxfcbde-eines-aus-todesgefahr-geretteten-bei-darbringung-des-dankopfers}}

\hypertarget{section-115}{%
\section{116}\label{section-115}}

1Ich liebe den HERRN, denn er hat erhört

mein flehentlich Rufen;

2ja, er hat sein Ohr mir zugeneigt:

ich will zu ihm rufen mein Leben lang!

3Umschlungen hatten mich des Todes Netze

und die Ängste der Unterwelt mich befallen, in Drangsal und Kummer war
ich geraten.

4Da rief ich den Namen des HERRN an:

»Ach, HERR, errette meine Seele\textless sup title=``oder: mein
Leben''\textgreater✲!«

5Gnädig ist der HERR und gerecht,

und unser Gott ist voll Erbarmens;

6der HERR schützt den, der unbeirrt ihm traut:

ich war schwach geworden, aber er half mir.{A}

7Kehre zurück, meine Seele, zu deiner Ruhe,

denn der HERR hat Gutes an dir getan!

8Ja, du hast mein Leben vom Tode errettet,

meine Augen vom Weinen, meinen Fuß vom Anstoß\textless sup title=``oder:
Gleiten''\textgreater✲;

9ich werde noch wandeln vor dem HERRN

in den Landen des Lebens\textless sup title=``oder: der
Lebenden''\textgreater✲.

10Ich habe Glauben gehalten, wenn ich auch sagte:

»Ich bin gar tief gebeugt«;

11in meiner Verzagtheit hab' ich gesagt:

»Die Menschen sind Lügner allesamt.«

12Wie soll ich dem HERRN vergelten

alles, was er mir Gutes getan?

13Den Becher des Heils will ich erheben

und den Namen des HERRN anrufen;

14meine Gelübde will ich bezahlen✲ dem HERRN,

ja angesichts seines ganzen Volkes.

15Kostbar ist in den Augen des HERRN

der Tod seiner Frommen.{A}

16Ach, HERR, ich bin ja dein Knecht,

ich bin dein Knecht, der Sohn deiner Magd; meine Bande hast du gelöst:

17dir will ich Dankopfer bringen

und den Namen des HERRN anrufen;

18meine Gelübde will ich bezahlen✲ dem HERRN,

ja angesichts seines ganzen Volkes,

19in den Vorhöfen am Hause des HERRN,

in deiner Mitte, Jerusalem! Halleluja!

\hypertarget{aufforderung-an-die-heiden-zum-lobpreis-gottes}{%
\subsubsection{Aufforderung an die Heiden zum Lobpreis
Gottes}\label{aufforderung-an-die-heiden-zum-lobpreis-gottes}}

\hypertarget{section-116}{%
\section{117}\label{section-116}}

1Lobet den HERRN, ihr Heiden alle!

Preiset ihn, ihr Völker alle!\textless sup title=``vgl. Röm
15,11''\textgreater✲

2Denn machtvoll waltet über uns seine Gnade,

und die Treue des HERRN währt ewiglich. Halleluja!

\hypertarget{dankgebet-und-siegeslied-der-festgemeinde}{%
\subsubsection{Dankgebet und Siegeslied der
Festgemeinde}\label{dankgebet-und-siegeslied-der-festgemeinde}}

\hypertarget{section-117}{%
\section{118}\label{section-117}}

1Danket dem HERRN, denn er ist freundlich,

ja, ewiglich währt seine Gnade!

2So bekenne denn Israel:

»Ja, ewiglich währt seine Gnade!«

3So bekenne denn Aarons Haus:

»Ja, ewiglich währt seine Gnade!«

4So mögen denn alle Gottesfürcht'gen bekennen:

»Ja, ewiglich währt seine Gnade!«

5Aus meiner Bedrängnis rief ich zum HERRN:

da hat der HERR mich erhört, mir weiten Raum geschafft.

6Ist der HERR für mich, so fürchte ich nichts:

was können Menschen mir tun?

7Tritt der HERR für mich zu meiner Hilfe ein,

so werde ich siegreich jubeln über meine Feinde.

8Besser ist's auf den HERRN vertrauen

als auf Menschen sich verlassen;

9besser ist's auf den HERRN vertrauen

als auf Fürsten sich verlassen.

10Die Heidenvölker alle hatten mich umringt:

im Namen des HERRN, so vertilgte ich sie;

11sie hatten mich umringt, umzingelt:

im Namen des HERRN, so vertilgte ich sie;

12sie hatten mich umringt wie Bienenschwärme;

schnell wie ein Dornenfeuer sind sie erloschen: im Namen des HERRN, so
vertilgte ich sie.

13Man hat mich hart gestoßen, damit ich fallen sollte,

doch der HERR hat mir geholfen.

14Meine Stärke und mein Lobpreis ist der HERR,

und er ist mein Retter geworden.

15Jubel und Siegeslieder erschallen in den Zelten der Gerechten:

»Die Hand des HERRN schafft mächtige Taten,

16die Hand des HERRN erhöht\textless sup title=``oder: ist
erhaben''\textgreater✲,

die Hand des HERRN schafft mächtige Taten!«

17Ich werde nicht sterben, nein, ich werde leben

und die Taten des HERRN verkünden.

18Der HERR hat mich hart gezüchtigt,

doch dem Tode mich nicht preisgegeben.

19Öffnet mir die Tore der Gerechtigkeit:

ich will durch sie eingehn, dem HERRN zu danken.~--

20Dies ist das Tor des HERRN\textless sup title=``oder: zum
HERRN''\textgreater✲:

Gerechte dürfen hier eingehn.~--

21Ich danke dir, daß du mich erhört hast

und bist mir ein Retter geworden.

22Der Stein, den die Bauleute verworfen haben,

der ist zum Eckstein\textless sup title=``oder:
Schlußstein''\textgreater✲ geworden;

23vom HERRN ist dies geschehn,

in unsern Augen ein Wunder!

24Dies ist der Tag, den der HERR gemacht hat:

laßt uns jubeln und fröhlich an ihm sein!~--

25Ach hilf doch, HERR,

ach, HERR, laß wohl gelingen!~--

26Gesegnet sei, der da kommt im Namen des HERRN!

Wir segnen euch vom Hause des HERRN aus.

27Der HERR ist Gott, er hat uns Licht gegeben:

schlinget den Reigen, mit Zweigen (geschmückt), bis an die Hörner des
Altars!

28Du bist mein Gott, ich will dir danken;

mein Gott, ich will dich erheben!

29Danket dem HERRN, denn er ist freundlich,

ja, ewiglich währt seine Gnade.

\hypertarget{herrlichkeit-des-guxf6ttlichen-wortes-und-gesetzes-oder-das-goldene-alphabet}{%
\subsubsection{Herrlichkeit des göttlichen Wortes und Gesetzes (oder:
das goldene
Alphabet)}\label{herrlichkeit-des-guxf6ttlichen-wortes-und-gesetzes-oder-das-goldene-alphabet}}

\hypertarget{section-118}{%
\section{119}\label{section-118}}

1Wohl denen, deren Wandel unsträflich ist,

die einhergehn im\textless sup title=``oder: nach dem''\textgreater✲
Gesetz des HERRN!

2Wohl denen, die seine Zeugnisse beobachten,

die mit ganzem Herzen ihn suchen,

3die auch kein Unrecht verüben,

sondern auf seinen Wegen gehen!

4Du selbst hast deine Befehle erlassen,

daß man sie sorglich\textless sup title=``oder: genau''\textgreater✲
befolge.

5Ach möchte doch mein Wandel fest sein

in der Befolgung deiner Satzungen!

6Dann werde ich nicht beschämt sein,

wenn ich alle deine Gebote vor Augen habe.

7Ich will dir aufrichtigen Herzens danken,

indem ich die Rechte\textless sup title=``oder:
Verordnungen''\textgreater✲ deiner Gerechtigkeit lerne.

8Deine Satzungen will ich halten:

verlaß mich nicht ganz und gar!

9Wie wird ein Jüngling seinen Wandel rein gestalten?

Wenn er ihn führt\textless sup title=``oder: sich hält''\textgreater✲
nach deinem Wort.

10Mit ganzem Herzen suche ich dich:

laß mich von deinen Geboten nicht abirren!

11In meinem Herzen wahre ich dein Wort,

um mich nicht gegen dich zu verfehlen.

12Gepriesen seist du, o HERR:

lehre mich deine Satzungen!

13Mit meinen Lippen zähle ich her

alle Rechte\textless sup title=``oder: Verordnungen''\textgreater✲
deines Mundes.

14An dem Wege deiner Zeugnisse habe ich Freude

wie über irgendwelchen Reichtum.

15Über deine Befehle will ich sinnen

und achten auf deine Pfade.

16An deinen Satzungen habe ich meine Lust,

will deine Worte nicht vergessen.

17Tu Gutes an deinem Knecht, auf daß ich leben bleibe,

so will ich deine Worte befolgen.

18Öffne mir die Augen, daß ich klar erkenne

die Wunder in deinem Gesetz.

19Ich bin nur ein Gast auf Erden:

verbirg deine Gebote nicht vor mir.

20Meine Seele verzehrt sich vor Sehnsucht

nach deinen Rechten\textless sup title=``oder:
Verordnungen''\textgreater✲ allezeit.

21Gedroht hast du den Stolzen; verflucht sind,

die von deinen Geboten abweichen.

22Wälze Schmach und Verachtung von mir ab,

denn ich beobachte{A} deine Zeugnisse.

23Mögen auch Fürsten sitzen und wider mich beraten:

dein Knecht sinnt doch über deine Satzungen nach.

24Ja, deine Zeugnisse sind meine Freude,

meine Ratgeber sind sie.

25Mein Mut ist in den Staub gesunken:

belebe mich wieder nach deinem Wort\textless sup title=``=~deiner
Verheißung''\textgreater✲.

26Ich habe dir meine Lage geschildert, da hast du mich erhört:

lehre mich deine Satzungen.

27Laß mich den Weg verstehn, den deine Befehle gebieten,

so will ich sinnen über deine Wunder.

28Mein Herz zerfließt vor Kummer in Tränen;

richte mich auf nach deinen Worten.

29Den Weg der Lüge halte fern von mir,

doch begnade✲ mich mit deinem Gesetz!

30Den Weg der Treue habe ich erwählt,

deine Rechte\textless sup title=``oder: Verordnungen''\textgreater✲
unanstößig befunden.

31Ich halte fest an deinen Zeugnissen:

HERR, laß mich nicht zuschanden werden!

32Den Weg deiner Gebote will ich laufen,

denn du machst mir weit das Herz\textless sup title=``=~erfreust mir das
Herz''\textgreater✲.

33Lehre mich, HERR, den Weg deiner Satzungen,

so will ich ihn innehalten bis ans Ende.

34Verleihe mir Einsicht, damit ich deine Weisung beachte

und sie mit ganzem Herzen befolge.

35Laß mich wandeln auf dem Pfade deiner Gebote,

denn an diesem habe ich meine Freude.

36Neige mein Herz deinen Zeugnissen zu

und nicht zur Gewinnsucht\textless sup title=``oder: zu unrechtem
Gewinn''\textgreater✲.

37Wende meine Augen ab, daß sie nicht nach Eitlem schauen;

belebe mich auf deinen Wegen.

38Erfülle an deinem Knechte deine Verheißung,

die darauf abzielt, daß man dich fürchte.

39Wende ab meine Schmach, vor der mir graut;

denn deine Rechte\textless sup title=``oder: Verordnungen''\textgreater✲
sind heilsam.

40Fürwahr, ich sehne mich nach deinen Befehlen:

belebe mich durch deine Gerechtigkeit!

41Laß deine Gnadenerweise mir widerfahren, o HERR,

deine Hilfe nach deinem Wort\textless sup title=``oder: deiner
Verheißung''\textgreater✲,

42daß ich dem, der mich schmäht, zu antworten weiß;

denn ich verlasse mich auf dein Wort.

43Und entzieh meinem Munde nicht ganz das Wort der Wahrheit;

denn ich harre auf deine Rechte\textless sup title=``oder:
Verordnungen''\textgreater✲.

44Und befolgen will ich dein Gesetz beständig,

immer und ewiglich;

45so werde ich wandeln auf freier Bahn;

denn ich habe mich stets um deine Befehle gekümmert;

46und ich will von deinen Zeugnissen reden

vor Königen, ohne mich zu scheuen;

47denn ich habe meine Freude an deinen Geboten,

die mir lieb sind,

48und hebe meine Hände auf zu deinen Geboten, {[}die mir lieb sind,{]}

will über deine Satzungen sinnen.

49Halte deinem Knecht getreulich dein Wort,

auf das du mich hast hoffen lassen!

50Das ist mein Trost in meinem Elend,

daß dein Wort\textless sup title=``oder: deine Verheißung''\textgreater✲
mich neu belebt hat.

51Die Übermüt'gen verspotten mich maßlos,

doch ich bin von deinem Gesetz nicht abgewichen.

52Gedenke ich deiner Rechte\textless sup title=``oder:
Verordnungen''\textgreater✲ aus der Vorzeit,

so fühle ich mich, o HERR, getröstet.

53Heißer Zorn erfaßt mich wegen der Gottlosen,

die dein Gesetz verlassen haben.

54Deine Satzungen sind mir zu Lobgesängen geworden

im Hause meiner Pilgerschaft.

55In der Nacht sogar gedenke ich deines Namens, o HERR,

und befolge dein Gesetz.

56Das ist mir zuteil geworden,

daß ich deine Befehle befolgt habe.

57Meine Aufgabe ist, o HERR, ich bekenne es,

deine Worte zu befolgen.

58Von ganzem Herzen fleh' ich dich an:

»Sei mir gnädig nach deiner Verheißung!«

59Ich habe über meine Wege nachgedacht

und lenke (daher) meine Schritte zu deinen Zeugnissen zurück.

60Ich eile und säume nicht,

deine Gebote zu befolgen.

61Die Fallstricke der Gottlosen umringen mich;

dennoch vergesse ich dein Gesetz nicht.

62In der Mitte der Nacht stehe ich auf, um dir zu danken

für die Verordnungen deiner Gerechtigkeit.

63Befreundet bin ich mit allen, die dich fürchten,

und mit denen, die deine Befehle befolgen.

64Deiner Gnade\textless sup title=``oder: Güte''\textgreater✲, o HERR,
ist die Erde voll:

lehre mich deine Satzungen!

65Gutes hast du an deinem Knechte getan,

o HERR, nach deiner Verheißung.

66Rechte Einsicht und Erkenntnis lehre mich,

denn ich vertraue auf deine Gebote.

67Bevor ich gedemütigt wurde, ging ich irre;

jetzt aber beobachte{A} ich dein Wort.

68Du bist gütig und erweisest Gutes:

lehre mich deine Satzungen!

69Lügen haben die Stolzen gegen mich erdichtet,

ich aber befolge deine Befehle mit ganzem Herzen.

70Unempfindlich wie von Fett ist ihr Herz,

ich aber habe Freude an deinem Gesetz.

71Gut war's für mich, daß ich gedemütigt wurde,

damit ich deine Satzungen lernte.

72Die Weisung deines Mundes ist mir lieber

als Tausende von Gold- und Silberstücken.

73Deine Hände haben mich geschaffen und gebildet:

verleihe mir nun auch Einsicht, daß ich deine Gebote lerne!

74Die dich fürchten, werden mich sehen und sich freun;

denn ich habe auf dein Wort\textless sup title=``oder: deine
Verheißung''\textgreater✲ geharrt.

75Ich weiß, o HERR, daß deine Gerichte gerecht sind

und du mich in Treue gedemütigt hast.

76Laß doch deine Gnade mir Trost gewähren,

wie du deinem Knechte verheißen hast!

77Laß mir dein Erbarmen widerfahren, daß ich auflebe,

denn dein Gesetz ist meine Lust.

78Laß die Stolzen zuschanden werden,

weil sie ohne Grund mich niederdrücken;{A} ich aber sinne über deine
Befehle.

79Laß mir sich zuwenden, die dich fürchten

und die deine Zeugnisse anerkennen!

80Mein Herz halte treu an deinen Satzungen fest,

auf daß ich nicht zuschanden werde.

81Meine Seele schmachtet nach deiner Hilfe\textless sup title=``oder:
Rettung''\textgreater✲:

ich harre auf dein Wort.{A}

82Meine Augen schmachten nach deiner Verheißung,

indem ich frage: »Wann wirst du mich trösten?«

83Bin ich auch wie ein Schlauch im Rauch geworden,

hab' ich doch deine Satzungen nicht vergessen.

84Wie viele sind noch der Lebenstage deines Knechts?

Wann hältst du Gericht über meine Verfolger?

85Übermütige haben mir Gruben gegraben,

sie, die sich nicht nach deinem Gesetz verhalten.

86Alle deine Gebote sind Wahrheit;

mit Lüge\textless sup title=``oder: Unrecht''\textgreater✲ verfolgt man
mich: so hilf mir!

87Fast hätten sie mich im Lande umgebracht;

doch ich verlasse deine Befehle nicht.

88Nach deiner Gnade erhalte mich am Leben,

so will ich das Zeugnis deines Mundes befolgen.

89Auf ewige Zeit, o HERR,

steht fest dein Wort im Himmel.

90Von Geschlecht zu Geschlecht währt deine Treue;

du hast die Erde festgestellt, und sie steht;

91nach deinen Verordnungen stehn sie noch heute,

denn alle Dinge sind dir dienstbar✲.

92Wäre dein Gesetz nicht meine Freude gewesen,

so wär' ich in meinem Elend vergangen.

93Niemals will ich deine Befehle vergessen,

denn durch sie hast du mich neu belebt\textless sup title=``oder: am
Leben erhalten''\textgreater✲.

94Dein bin ich: hilf mir!

Denn ich beachte deine Verordnungen.

95Gottlose lauern mir auf, um mich umzubringen,

ich aber achte auf deine Zeugnisse.

96Von allem Vollkomm'nen habe ich eine Grenze gesehn;

doch dein Gebot ist völlig unbeschränkt.

97Wie habe ich dein Gesetz so lieb!

Den ganzen Tag ist es mein Sinnen.

98Weiser, als meine Feinde sind, machen mich deine Gebote,

denn mein sind sie für immer.

99Verständiger bin ich als alle meine Lehrer,

denn deine Zeugnisse sind mein Sinnen.

100Mehr Einsicht besitz' ich als die Greise;

denn ich beobachte{A} deine Befehle.

101Von jedem bösen Pfade halte ich meinen Fuß fern,

um dein Wort zu befolgen.

102Von deinen Rechten\textless sup title=``oder:
Verordnungen''\textgreater✲ weiche ich nicht ab,

denn du hast mich belehrt.

103Wie süß sind deine Worte\textless sup title=``oder:
Verheißungen''\textgreater✲ meinem Gaumen,

süßer als Honig meinem Munde!

104Aus deinen Befehlen gewinne ich Einsicht;

darum hasse ich jeglichen Lügenpfad.

105Dein Wort ist meines Fußes Leuchte

und ein Licht auf meinem Wege\textless sup title=``oder: für meinen
Wandel''\textgreater✲.

106Ich habe geschworen und den Vorsatz gefaßt,

den Verordnungen deiner Gerechtigkeit treu zu bleiben.

107Ich bin gar tief gebeugt:

o HERR, belebe mich wieder nach deiner Verheißung!

108Laß, HERR, dir gefallen die willigen Opfer meines Mundes

und lehre mich deine Rechte\textless sup title=``oder:
Verordnungen''\textgreater✲!

109Ich schwebe beständig in Todesgefahr,

doch dein Gesetz vergesse ich nicht.

110Die Gottlosen haben mir Schlingen gelegt,

aber von deinen Befehlen irre ich nicht ab.

111Deine Zeugnisse sind mein ewiger Erbbesitz,

denn sie sind die Wonne meines Herzens.

112Ich neige mein Herz dazu, deine Satzungen zu erfüllen

immerdar bis ans Ende.

113Die Doppelherzigen hasse ich,

aber dein Gesetz ist mir lieb.

114Mein Schirm und Schild bist du;

auf dein Wort\textless sup title=``=~die Erfüllung deiner
Verheißung''\textgreater✲ harre ich.

115Weicht von mir, ihr Übeltäter!

Ich will die Gebote meines Gottes halten.

116Stütze mich nach deiner Verheißung, daß ich lebe,

und laß mich nicht in meiner Hoffnung getäuscht werden!

117Stärke mich, auf daß ich Heil\textless sup title=``oder:
Rettung''\textgreater✲ erlange,

und laß mich stets auf deine Satzungen achten!

118Du verwirfst alle, die von deinen Satzungen abirren;

denn erfolglos ist ihre Täuschung.

119Wie Schlacken räumst du alle Gottlosen des Landes hinweg;

darum liebe ich deine Zeugnisse.

120Aus Furcht vor dir schaudert mein Leib,

und mir ist bange vor deinen Gerichten.

121Ich habe Recht und Gerechtigkeit geübt:

gib mich nicht meinen Bedrückern preis!

122Tritt für deinen Knecht zu seinem Heile ein,

laß die Stolzen mir nicht Gewalt antun!

123Meine Augen schmachten nach deiner Rettung

und nach der Bestätigung deiner Gerechtigkeit.

124Verfahre mit deinem Knecht nach deiner Gnade

und lehre mich deine Satzungen!

125Dein Knecht bin ich, verleihe mir Einsicht,

damit ich deine Zeugnisse verstehen lerne.

126Zeit ist's für den HERRN, zu handeln:

sie haben ja dein Gesetz gebrochen.

127Darum liebe ich deine Gebote

mehr als Gold und als Feingold.

128Darum schätze ich alle deine Befehle als richtig;

jeder Lügenpfad ist mir verhaßt.

129Wunderwerke sind deine Zeugnisse;

darum hält mein Herz an ihnen fest.

130Die Erschließung deiner Worte erleuchtet,

verleiht den Einfältigen Einsicht.

131Ich tue meinen Mund weit auf und lechze,

denn mich verlangt nach deinen Geboten.

132Wende dich zu mir und sei mir gnädig,

wie es recht ist bei denen, die deinen Namen lieben!

133Laß meine Schritte fest sein durch dein Wort

und laß nichts Trügerisches\textless sup title=``oder: kein
Unrecht''\textgreater✲ über mich herrschen.

134Erlöse mich von der Bedrückung der Menschen,

so will ich deine Befehle befolgen.

135Laß dein Angesicht leuchten gegen deinen Knecht

und lehre mich deine Satzungen.

136Tränenströme rinnen aus meinen Augen,

weil viele dein Gesetz nicht befolgen.

137Gerecht bist du, o HERR,

und richtig sind deine Rechte\textless sup title=``oder:
Verordnungen''\textgreater✲.

138In Gerechtigkeit hast du deine Zeugnisse verordnet

und in unerschütterlicher Treue.

139Mich verzehrt mein Eifer,

weil meine Gegner deine Worte vergessen.

140Dein Wort ist wohlgeläutert,

und dein Knecht hat es lieb.

141Gering bin ich und verachtet,

doch deine Befehle vergesse ich nicht.

142Deine Gerechtigkeit ist ewige Gerechtigkeit,

und dein Gesetz ist Wahrheit.

143Wenn Leiden und Not mich getroffen haben,

sind deine Gebote dennoch meine Freude.

144Deine Zeugnisse bleiben immerdar gerecht:

verleihe mir Verständnis, so werde ich leben.

145Ich rufe von ganzem Herzen: »Erhöre mich, HERR!«

Deine Satzungen will ich beobachten{A}.

146Ich rufe zu dir: »Hilf mir!

So will ich deine Zeugnisse beobachten{A}.«

147Früh bin ich auf vor Tagesanbruch und flehe laut;

auf dein Wort\textless sup title=``=~die Erfüllung deiner
Verheißung''\textgreater✲ harre ich.

148Meine Augen wachen die ganze Nacht hindurch,

um über dein Wort\textless sup title=``=~deine Verheißung''\textgreater✲
nachzusinnen.

149Höre meine Stimme nach deiner Gnade!

O HERR, nach deinen Rechten\textless sup title=``oder:
Verordnungen''\textgreater✲ laß mich aufleben!

150Mir haben sich Leute genaht, die der Arglist frönen:

von deinem Gesetz sind sie fern;

151doch du bist mir nahe, o HERR,

und alle deine Gebote sind Wahrheit.

152Längst weiß ich aus deinen Zeugnissen,

daß du sie\textless sup title=``d.h. deine Gebote''\textgreater✲ für
ewig festgestellt hast.

153Sieh mein Elend an und errette mich!

Denn dein Gesetz vergesse ich nicht.

154Führe meine Sache und erlöse mich,

schenke mir neues Leben nach deiner Verheißung!

155Den Gottlosen bleibt die Hilfe\textless sup title=``oder:
Rettung''\textgreater✲ fern,

denn sie kümmern sich nicht um deine Satzungen.

156Deine Barmherzigkeit ist groß, o HERR:

nach deinen Rechten\textless sup title=``oder:
Verordnungen''\textgreater✲ belebe mich wieder!

157Groß ist meiner Verfolger und Gegner Zahl,

doch von deinen Zeugnissen geh' ich nicht ab.

158Wenn ich Treulose sehe, so fühle ich Abscheu,

weil sie dein Wort\textless sup title=``oder: Gebot''\textgreater✲ nicht
befolgen.

159Sieh her, ich liebe deine Befehle:

HERR, schenke mir neues Leben nach deiner Gnade!

160Der ganze Inhalt deines Wortes ist Wahrheit,

und ewig gilt jede Verordnung deiner Gerechtigkeit.

161Fürsten haben mich ohne Ursach' verfolgt;

doch nur vor deinen Worten✲ erbebt mein Herz.

162Ich freue mich über dein Wort\textless sup title=``oder: deine
Verheißung''\textgreater✲

wie einer, der große Beute gewinnt.

163Lügen hasse und verabscheue ich,

aber dein Gesetz ist mir lieb.

164Siebenmal täglich preise ich dich

um der Verordnungen deiner Gerechtigkeit willen.

165Frieden\textless sup title=``oder: Heil''\textgreater✲ in Fülle
erlangen die Freunde deines Gesetzes,

denn es gibt für sie kein Straucheln.

166Ich hoffe auf deine Rettung, o HERR,

denn ich habe deine Gebote gehalten.

167Mein Herz befolgt deine Zeugnisse,

und ich habe sie aufrichtig lieb.

168Ich befolge deine Befehle und Zeugnisse;

denn alle meine Wege sind dir bekannt.

169Laß mein lautes Flehen zu dir dringen, o HERR;

verleih mir Verständnis für dein Wort!

170Laß mein Beten vor dich kommen:

errette mich nach deiner Verheißung!

171Meine Lippen sollen Lobpreis sprudeln lassen,

weil du mich deine Satzungen lehrst.

172Meine Zunge soll von deinem Worte\textless sup title=``oder: deiner
Verheißung''\textgreater✲ singen;

denn alle deine Gebote sind gerecht.

173Laß deine Hand bereit sein, mir zu helfen,

denn deine Befehle hab' ich (zu Führern) erwählt.

174Ich sehne mich nach deiner Hilfe, o HERR,

und dein Gesetz ist meine Freude.

175Laß meine Seele leben, daß sie dich preise,

und deine Rechte✲ mögen mir helfen!

176Geh ich irre wie ein verlorenes Schaf, so suche deinen Knecht!

Denn deine Gebote habe ich nicht vergessen.

\hypertarget{die-15-wallfahrtslieder-psalm-120-134}{%
\subsection{Die 15 Wallfahrtslieder (Psalm
120-134)}\label{die-15-wallfahrtslieder-psalm-120-134}}

\hypertarget{hilferuf-gegen-truxfcgerische-und-streitsuxfcchtige-widersacher}{%
\subsubsection{Hilferuf gegen trügerische und streitsüchtige
Widersacher}\label{hilferuf-gegen-truxfcgerische-und-streitsuxfcchtige-widersacher}}

\hypertarget{section-119}{%
\section{120}\label{section-119}}

1Ein Wallfahrtsliedoder Stufenlied?''\textgreater✲. Ich rief zum HERRN
in meiner Not:

da erhörte er mich.

2O HERR, errette mich von der Lügenlippe,

von der trügerischen Zunge!

3Was wird Er dir jetzt und in Zukunft bescheren,

du trügerische Zunge?

4Geschärfte Kriegerpfeile

samt Kohlen vom Ginsterstrauch!

5Wehe mir, daß ich als Fremdling in Mesech weile,

daß ich wohne bei den Zelten von Kedar{B}!

6Lange genug schon weile ich hier

bei Leuten, die den Frieden hassen.

7Ich bin ganz friedlich gestimmt, doch was ich auch rede:

sie gehen auf Krieg✲ aus.

\hypertarget{der-treue-huxfcter-der-menschen}{%
\subsubsection{Der treue Hüter der
Menschen}\label{der-treue-huxfcter-der-menschen}}

\hypertarget{section-120}{%
\section{121}\label{section-120}}

1Ein Lied für Wallfahrtenoder für die Stufen? vgl. Ps
120''\textgreater✲. Ich hebe meine Augen auf zu den Bergen:

von wo wird Hilfe mir kommen?

2Meine Hilfe kommt vom HERRN,

der Himmel und Erde geschaffen.

3Er wird deinen Fuß nicht wanken lassen;

nicht schlummert dein Hüter.

4Nein, nicht schlummert und nicht schläft

der Hüter Israels.

5Der HERR ist dein Hüter, der HERR dein Schatten

über deiner rechten Hand,{A}

6daß dich bei Tage die Sonne nicht sticht,

noch der Mond in der Nacht.

7Der HERR behütet dich vor allem Übel,

er behütet deine Seele\textless sup title=``oder: dein
Leben''\textgreater✲;

8der HERR behütet deinen Ausgang und Eingang

von nun an bis in Ewigkeit.

\hypertarget{segenswuxfcnsche-eines-pilgers-fuxfcr-jerusalem}{%
\subsubsection{Segenswünsche eines Pilgers für
Jerusalem}\label{segenswuxfcnsche-eines-pilgers-fuxfcr-jerusalem}}

\hypertarget{section-121}{%
\section{122}\label{section-121}}

1Ein Wallfahrtsliedoder Stufenlied? vgl. Ps 120''\textgreater✲ Davids.
Ich freute mich, als man mir sagte:

»Wir wollen pilgern zum Hause des HERRN!«

2So stehn denn nunmehr unsre Füße

in deinen Toren, Jerusalem!

3Jerusalem, du wiedererbaute

als eine Stadt, die fest in sich geschlossen,

4wohin die Stämme hinaufziehn, die Stämme des HERRN,

nach der für Israel gültigen Weisung, dort den Namen des HERRN zu
preisen;

5denn dort waren einst aufgestellt die Stühle zum Gericht,

die Stühle des Hauses Davids.

6Bringet Jerusalem dar den Friedensgruß:

»Heil denen, die dich lieben!

7Friede herrsche vor deinen Mauern,

sichere Ruhe in deinen Palästen!«

8Um meiner Brüder und Freunde willen

will ich dir Frieden\textless sup title=``oder: Heil''\textgreater✲
wünschen;

9um des Hauses des HERRN, unsres Gottes, willen

will ich Segen für dich erbitten.{A}

\hypertarget{gluxe4ubiger-aufblick-zu-gott-bei-schmach-und-spott}{%
\subsubsection{Gläubiger Aufblick zu Gott bei Schmach und
Spott}\label{gluxe4ubiger-aufblick-zu-gott-bei-schmach-und-spott}}

\hypertarget{section-122}{%
\section{123}\label{section-122}}

1Ein Wallfahrtsliedoder Stufenlied? vgl. Ps 120''\textgreater✲. Zu dir
erhebe ich meine Augen,

der du thronst im Himmel.

2Siehe, wie die Augen der Knechte

auf die Hand ihrer Herren, wie die Augen der Magd auf ihrer Gebieterin
Hand: so blicken unsre Augen hin auf den HERRN, unsern Gott, bis er sich
unser erbarmt.

3Erbarme dich, HERR, erbarme dich unser!

Denn gründlich sind wir satt der Verachtung;

4satt, ja übersatt ist uns die Seele

des Hohns der Leichtfertigen, der Verachtung der Stolzen.

\hypertarget{israels-retter-in-der-not}{%
\subsubsection{Israels Retter in der
Not}\label{israels-retter-in-der-not}}

\hypertarget{section-123}{%
\section{124}\label{section-123}}

1Ein Wallfahrtsliedoder Stufenlied? vgl. Ps 120''\textgreater✲ von
David. »Wäre der HERR nicht für uns gewesen«

so bekenne Israel! --,

2»wäre der HERR nicht für uns gewesen,

als Menschen sich gegen uns erhoben:

3dann hätten sie uns lebendig verschlungen,

als ihr Zorn gegen uns entbrannt war;

4dann hätten die Wasser uns überflutet,

ein Wildbach hätte sich über uns ergossen;

5dann wären über uns hingegangen

die wildwogenden\textless sup title=``oder: überwallenden''\textgreater✲
Fluten.«

6Gepriesen sei der HERR, der uns nicht ihren Zähnen

zum Raub hat preisgegeben!

7Unsre Seele ist entschlüpft wie ein Vogel dem Netz der Vogelsteller:

das Netz ist zerrissen, und wir sind frei geworden.

8Unsre Hilfe steht im Namen des HERRN,

der Himmel und Erde geschaffen.

\hypertarget{gott-schuxfctzt-sein-volk-israel-und-alle-frommen}{%
\subsubsection{Gott schützt sein Volk Israel und alle
Frommen}\label{gott-schuxfctzt-sein-volk-israel-und-alle-frommen}}

\hypertarget{section-124}{%
\section{125}\label{section-124}}

1Ein Wallfahrtsliedoder Stufenlied? vgl. Ps 120''\textgreater✲. Die auf
den HERRN vertrauen, die gleichen dem Berge Zion,

der nicht wankt, der in Ewigkeit feststeht.

2Wie Berge Jerusalem rings umgeben,

so umhegt der HERR sein Volk von nun an bis in Ewigkeit.

3Denn der Gottlosen Zepter wird nicht lasten bleiben

auf dem Erbteil der Gerechten, damit nicht auch die Gerechten ihre Hände
ausstrecken zum Frevel.

4Erweise deine Güte, HERR, den Guten

und denen, die redlichen Herzens sind!

5Doch die auf ihre krummen Wege abbiegen,

die lasse der HERR hinfahren mitsamt den Übeltätern! Heil über Israel!

\hypertarget{trost-in-truxe4nen}{%
\subsubsection{Trost in Tränen}\label{trost-in-truxe4nen}}

\hypertarget{section-125}{%
\section{126}\label{section-125}}

1Ein Wallfahrtsliedoder Stufenlied? vgl. Ps 120''\textgreater✲. Als der
HERR einst Zions Mißgeschick wandte,

da war's uns, als träumten wir.

2Damals war unser Mund voll Lachens

und unsre Zunge voll Jubels; damals sagte man unter den Heiden: »Der
HERR hat Großes an ihnen getan!«

3Ja, Großes hatte der HERR an uns getan:

wie waren wir fröhlich!

4Wende, o HERR, unser Mißgeschick

gleich den Bächen im Mittagsland!

5Die mit Tränen säen,

werden mit Jubel ernten.

6Wohl schreitet man weinend dahin,

wenn man trägt den Samen zur Aussaat; doch jubelnd kehrt man heim, mit
Garben beladen.

\hypertarget{an-gottes-segen-ist-alles-gelegen}{%
\subsubsection{An Gottes Segen ist alles
gelegen}\label{an-gottes-segen-ist-alles-gelegen}}

\hypertarget{section-126}{%
\section{127}\label{section-126}}

1Ein Wallfahrtsliedoder Stufenlied? vgl. Ps 120''\textgreater✲ Salomos.
Wenn der HERR das Haus nicht baut,

so arbeiten umsonst, die daran bauen; wenn der HERR nicht die Stadt
behütet, so wacht der Wächter umsonst.

2Vergebens ist's für euch, daß früh ihr aufsteht

und spät noch sitzt bei der Arbeit, um das Brot der Mühsal\textless sup
title=``=~mühsam erworbenes Brot''\textgreater✲ zu essen; ebenso
(reichlich) gibt er's seinen Freunden im Schlaf.~--

3Ja, Söhne sind ein Geschenk des HERRN,

und Kindersegen ist eine Belohnung.

4Wie Pfeile in der Hand eines Kriegers\textless sup title=``oder:
Helden''\textgreater✲,

so sind die Söhne der Jugendkraft:

5wohl dem Manne, der mit ihnen

seinen Köcher gefüllt hat! Die werden nicht zuschanden, wenn sie
verhandeln mit Widersachern im Stadttor.

\hypertarget{huxe4usliches-gluxfcck-als-segen-der-gottesfurcht}{%
\subsubsection{Häusliches Glück als Segen der
Gottesfurcht}\label{huxe4usliches-gluxfcck-als-segen-der-gottesfurcht}}

\hypertarget{section-127}{%
\section{128}\label{section-127}}

1Ein Wallfahrtsliedoder Stufenlied? vgl. Ps 120''\textgreater✲. Wohl
jedem, der den HERRN fürchtet

und auf seinen Wegen wandelt!

2Deiner Hände Erwerb -- du darfst ihn genießen:

wohl dir, du hast es gut!

3Dein Weib gleicht einem fruchtbaren Weinstock

im Innern deines Hauses; deine Kinder sind wie Ölbaumschosse rings um
deinen Tisch.

4Ja wahrlich, so wird der Mann gesegnet,

der da fürchtet den HERRN.

5Dich segne der HERR von Zion her,

daß du schauest deine Lust an Jerusalems Glück dein Leben lang

6und sehest Kinder von deinen Kindern!

Heil über Israel!

\hypertarget{israels-drangsale-und-errettung}{%
\subsubsection{Israels Drangsale und
Errettung}\label{israels-drangsale-und-errettung}}

\hypertarget{section-128}{%
\section{129}\label{section-128}}

1Ein Wallfahrtsliedoder Stufenlied? vgl. Ps 120''\textgreater✲. »Sie
haben mich hart bedrängt von meiner Jugend an«

so bekenne Israel --,

2»sie haben mich hart bedrängt von meiner Jugend an,

aber doch mich nicht überwältigt.

3Auf meinem Rücken haben die Pflüger gepflügt

und lange Furchen gezogen;

4doch der HERR ist gerecht: er hat zerhauen

der Gottlosen Stricke.«

5Zuschanden müssen werden und rückwärts weichen

alle, die Zion hassen!

6Sie müssen gleichen dem Gras auf den Dächern,

das dürr schon ist, bevor es in Halme schießt,{A}

7mit dem der Schnitter seine Hand nicht füllt,

noch der Garbenbinder seinen Gewandbausch✲,

8und bei dem, wer des Weges vorübergeht, nicht ruft:

»Gottes Segen sei über euch! Wir segnen euch im Namen des HERRN!«

\hypertarget{aus-tiefer-not-sechster-buuxdfpsalm}{%
\subsubsection{Aus tiefer Not (Sechster
Bußpsalm)}\label{aus-tiefer-not-sechster-buuxdfpsalm}}

\hypertarget{section-129}{%
\section{130}\label{section-129}}

1Ein Wallfahrtsliedoder Stufenlied? vgl. Ps 120''\textgreater✲. Aus der
Tiefe rufe ich, HERR, zu dir: 2»Allherr, höre auf meine Stimme,

laß deine Ohren merken auf mein lautes Flehen!«

3Wenn du, HERR, Sünden behalten\textless sup title=``oder:
anrechnen''\textgreater✲ willst,

o Allherr, wer kann bestehn!

4Doch bei dir ist die Vergebung,

auf daß man dich fürchte.

5Ich harre des HERRN, meine Seele harrt,

und ich warte auf sein Wort\textless sup title=``=~seine
Verheißung''\textgreater✲;

6meine Seele harrt auf den Allherrn

sehnsuchtsvoller als Wächter auf den Morgen.

7Sehnsuchtsvoller als Wächter auf den Morgen

harre, Israel, auf den HERRN! Denn beim HERRN ist die Gnade und Erlösung
bei ihm in Fülle,

8und er wird Israel erlösen

von allen seinen Sünden.{A}

\hypertarget{stilles-genuxfcgen-oder-ruhe-in-gott}{%
\subsubsection{Stilles Genügen (oder Ruhe in
Gott)}\label{stilles-genuxfcgen-oder-ruhe-in-gott}}

\hypertarget{section-130}{%
\section{131}\label{section-130}}

1Ein Wallfahrtsliedoder Stufenlied? vgl. Ps 120''\textgreater✲ Davids.
HERR, mein Herz ist nicht hochfahrend,

und meine Augen erheben sich nicht stolz; ich gehe nicht mit Dingen um,
die vermessen sind und mir zu hoch\textless sup title=``oder:
schwer''\textgreater✲.

2Nein, ich habe mein Herz beruhigt und gestillt;

wie ein entwöhntes Kind an der Mutter Brust, so ruht entwöhnt mein Herz
in mir.~--

3Israel, harre des HERRN

von nun an bis in Ewigkeit.

\hypertarget{gebet-fuxfcr-zion-im-hinblick-auf-gottes-verheiuxdfung-an-david}{%
\subsubsection{Gebet für Zion im Hinblick auf Gottes Verheißung an
David}\label{gebet-fuxfcr-zion-im-hinblick-auf-gottes-verheiuxdfung-an-david}}

\hypertarget{section-131}{%
\section{132}\label{section-131}}

1Ein Wallfahrtsliedoder Stufenlied? vgl. Ps 120''\textgreater✲. Gedenke,
HERR, dem David alle seine Mühsal, 2ihm, der dem HERRN einst zuschwor

und gelobte Jakobs mächtigem Gott\textless sup title=``vgl. 2.Sam 7;
1.Chr 17''\textgreater✲:

3»Wahrlich, ich will mein Wohnzelt nicht betreten,

nicht mein Ruhelager besteigen;

4ich will meinen Augen den Schlaf nicht gönnen,

nicht Schlummer meinen Augenlidern,

5bis eine Stätte dem HERRN ich gefunden,

eine Wohnung für Jakobs mächtigen Gott!«

6Ja, wir haben von ihr gehört in Ephrath,

sie gefunden im Gefilde von Jaar:

7»Laßt uns in seine Wohnung treten,

uns niederwerfen vor dem Schemel seiner Füße!

8Brich auf, o HERR, zu deiner Ruhstatt,

du und die Lade (das Sinnbild) deiner Macht!\textless sup title=``2.Chr
6,41''\textgreater✲

9Laß deine Priester sich kleiden in Gerechtigkeit\textless sup
title=``=~treue Dienstleistung''\textgreater✲,

und deine Frommen mögen jubeln!

10Um deines Knechtes David willen

weise das Antlitz\textless sup title=``=~die Bitte''\textgreater✲ deines
Gesalbten nicht ab!«

11Geschworen hat der HERR dem David einen Eid,

einen wahren Eid, von dem er nicht abgeht: »Von deinen leiblichen
Sprossen will einen ich setzen auf deinen Thron.

12Wenn deine Söhne meinen Bund beachten

und meine Zeugnisse\textless sup title=``oder: Gebote''\textgreater✲,
die ich sie lehren werde, so sollen auch ihre Söhne für und für sitzen
auf deinem Thron.«

13Denn der HERR hat Zion erwählt,

hat es zu seiner Wohnung begehrt:

14»Dies ist meine Ruhstatt für immer,

hier will ich wohnen, weil ich's so begehrt.

15Zions Nahrung will ich reichlich segnen,

seine Armen sättigen mit Brot;

16seine Priester werde in Heil ich kleiden,

seine Frommen sollen laut frohlocken.

17Dort will ich Davids Macht erblühen lassen;

eine Leuchte{B} hab' ich meinem Gesalbten bereitet.

18Seine Feinde will ich kleiden in Schmach,

doch ihm soll auf dem Haupt die Krone glänzen.«

\hypertarget{segen-der-bruxfcderlichen-eintracht}{%
\subsubsection{Segen der brüderlichen
Eintracht}\label{segen-der-bruxfcderlichen-eintracht}}

\hypertarget{section-132}{%
\section{133}\label{section-132}}

1Ein Wallfahrtsliedoder Stufenlied? vgl. Ps 120''\textgreater✲ Davids.
Seht, wie schön und wie lieblich ist's,

wenn Brüder auch (friedlich) beisammen wohnen!

2Das gleicht dem köstlichen Öl auf dem Haupt,

das herabtroff in den Bart, in Aarons Bart, der niederwallte auf den
Saum seiner Gewandung.

3Es gleicht dem Hermontau, der niederfällt

auf die Berge Zions; denn dorthin hat der HERR den Segen entboten, Leben
bis in Ewigkeit.

\hypertarget{lied-der-tempelwuxe4chter-beim-nachtgottesdienst}{%
\subsubsection{Lied der Tempelwächter beim
Nachtgottesdienst}\label{lied-der-tempelwuxe4chter-beim-nachtgottesdienst}}

\hypertarget{section-133}{%
\section{134}\label{section-133}}

1Ein Wallfahrtsliedoder Stufenlied? vgl. Ps 120''\textgreater✲. Wohlan,
preiset den HERRN, alle ihr Diener des HERRN,

die ihr steht in den Nächten im Hause des HERRN!

2Erhebt eure Hände zum Heiligtum hin

und preiset den HERRN!

3Dich segne der HERR von Zion her,

der Schöpfer von Himmel und Erde!

\hypertarget{lobpreis-des-allein-wahren-gottes}{%
\subsubsection{Lobpreis des allein wahren
Gottes}\label{lobpreis-des-allein-wahren-gottes}}

\hypertarget{section-134}{%
\section{135}\label{section-134}}

1Halleluja!

Preiset den Namen des HERRN, preist ihn, ihr Diener des HERRN,

2die ihr stehet im Hause des HERRN,

in den Höfen am Haus unsers Gottes!

3Preiset den HERRN, denn gütig ist der HERR;

lobsingt seinem Namen, denn lieblich ist er!

4Denn Jakob hat der HERR sich erwählt

und Israel sich zum Eigentum erkoren.

5Ja, ich weiß es: groß ist der HERR,

und unser Gott steht über allen Göttern;

6alles, was dem HERRN gefällt, das führt er aus

im Himmel und auf Erden, in den Meeren und allen Tiefen.

7Er ist's, der Wolken heraufführt vom Ende der Erde,

der Blitze bei Gewitterregen schafft, der den Wind aus seinen Speichern
herausläßt.

8Er war's, der Ägyptens Erstgeburten schlug

unter Menschen wie beim Vieh;

9der Zeichen und Wunder sandte in deine Mitte, Ägypten,

gegen den Pharao und all seine Knechte.

10Er war's, der viele\textless sup title=``oder: große''\textgreater✲
Völker schlug

und mächtige Könige tötete:

11Sihon, den König der Amoriter,

und Og, den König von Basan, und alle Königreiche Kanaans,

12und ihr Land als Erbbesitz hingab,

als Erbe seinem Volke Israel.

13O HERR, dein Name währt ewig,

dein Gedächtnis\textless sup title=``oder: Ruhm''\textgreater✲, o HERR,
von Geschlecht zu Geschlecht

14denn der HERR schafft Recht seinem Volk

und erbarmt sich über seine Knechte.

15Die Götzen der Heiden sind Silber und Gold,

das Machwerk von Menschenhänden;

16sie haben einen Mund und können nicht reden,

haben Augen und sehen nicht;

17sie haben Ohren und können nicht hören,

auch ist kein Odem in ihrem Munde.

18Ihnen gleich sind ihre Verfertiger,

jeder, der auf sie vertraut.\textless sup title=``vgl.
115,4-8''\textgreater✲

19Ihr vom Hause Israel, preiset den HERRN!

Ihr vom Hause Aaron, preiset den HERRN!

20Ihr vom Hause Levi, preiset den HERRN!

Ihr, die ihr fürchtet den HERRN, preiset den HERRN!

21Gepriesen sei der HERR von Zion aus,

er, der da wohnt in Jerusalem! Halleluja!

\hypertarget{danklied-fuxfcr-gottes-wohltaten-an-israel}{%
\subsubsection{Danklied für Gottes Wohltaten an
Israel}\label{danklied-fuxfcr-gottes-wohltaten-an-israel}}

\hypertarget{section-135}{%
\section{136}\label{section-135}}

1Danket dem HERRN, denn er ist freundlich,

ja, ewiglich währt seine Gnade!

2Danket dem Gott der Götter~--

ja, ewiglich währt seine Gnade!

3Danket dem Herrn der Herren~--

ja, ewiglich währt seine Gnade!

4Ihm, der große Wunder tut, er allein:~--

ja, ewiglich währt seine Gnade!

5der den Himmel mit Weisheit geschaffen:~--

ja, ewiglich währt seine Gnade!

6der die Erde über den Wassern ausgebreitet:~--

ja, ewiglich währt seine Gnade!

7der die großen Lichter\textless sup title=``oder:
Leuchten''\textgreater✲ geschaffen:~--

ja, ewiglich währt seine Gnade!

8die Sonne zur Herrschaft am Tage:~--

ja, ewiglich währt seine Gnade!

9den Mond und die Sterne zur Herrschaft bei Nacht:~--

ja, ewiglich währt seine Gnade!

10Ihm, der Ägypten schlug an seinen Erstgeburten:~--

ja, ewiglich währt seine Gnade!

11und Israel aus ihrer Mitte führte:~--

ja, ewiglich währt seine Gnade!

12mit starker Hand und hocherhobnem Arm:~--

ja, ewiglich währt seine Gnade!

13der das Schilfmeer in zwei Teile zerschnitt:~--

ja, ewiglich währt seine Gnade!

14und Israel mitten hindurchziehen ließ:~--

ja, ewiglich währt seine Gnade!

15und den Pharao und sein Heer ins Schilfmeer stürzte:~--

ja, ewiglich währt seine Gnade!

16Ihm, der sein Volk durch die Wüste führte:~--

ja, ewiglich währt seine Gnade!

17der große Könige schlug:~--

ja, ewiglich währt seine Gnade!

18und mächtige Könige tötete:~--

ja, ewiglich währt seine Gnade!

19Sihon, den König der Amoriter:~--

ja, ewiglich währt seine Gnade!

20und Og, den König von Basan:~--

ja, ewiglich währt seine Gnade!

21und ihr Land als Erbbesitz hingab:~--

ja, ewiglich währt seine Gnade!

22als Erbbesitz seinem Knechte Israel:~--

ja, ewiglich währt seine Gnade!

23ihm, der in unsrer Erniedrigung unser gedachte:~--

ja, ewiglich währt seine Gnade!

24und uns von unsern Drängern befreite:~--

ja, ewiglich währt seine Gnade!

25der Nahrung allen Geschöpfen gibt:~--

ja, ewiglich währt seine Gnade!

26Danket dem Gott des Himmels:

ja, ewiglich währt seine Gnade!

\hypertarget{klage-der-gefangenen-juduxe4er-an-babels-struxf6men}{%
\subsubsection{Klage der gefangenen Judäer an Babels
Strömen}\label{klage-der-gefangenen-juduxe4er-an-babels-struxf6men}}

\hypertarget{section-136}{%
\section{137}\label{section-136}}

1An Babels Strömen, da saßen wir und weinten,

wenn Zions wir gedachten;

2an die Weiden, die dort stehen,

hängten wir unsre Harfen;

3denn Lieder verlangten von uns dort unsre Zwingherrn,

und unsre Peiniger hießen uns fröhlich sein: »Singt uns eins von euren
Zionsliedern!«

4Wie sollten wir singen die Lieder des HERRN

auf fremdem Boden?

5Vergesse ich dich, Jerusalem,

so verdorre mir die rechte Hand!{A}

6Die Zunge bleibe mir am Gaumen kleben,

wenn ich deiner nicht eingedenk bleibe, wenn ich Jerusalem nicht stelle
über alles, was mir Freude macht!{A}

7Gedenke, HERR, den Söhnen Edoms

den Unglückstag Jerusalems, wie sie riefen: »Reißt nieder, reißt nieder
bis auf den Grund in ihm!«

8Bewohnerschaft Babels, Verwüsterin!

Heil dem, der dir vergilt dasselbe, was du an uns verübt!

9Heil dem, der deine Kindlein packt

und am Felsen sie zerschmettert!

\hypertarget{danklied-fuxfcr-gottes-guxfcte-und-treue}{%
\subsubsection{Danklied für Gottes Güte und
Treue}\label{danklied-fuxfcr-gottes-guxfcte-und-treue}}

\hypertarget{section-137}{%
\section{138}\label{section-137}}

1Von David. Danken will ich dir (HERR) von ganzem Herzen,

vor den Göttern{A} will ich dir lobsingen;

2vor deinem heiligen Tempel will ich anbeten

und deinen Namen preisen ob deiner Gnade und Treue; denn über deinen
ganzen Namen hinaus hast dein Wort\textless sup title=``=~deine
Verheißung''\textgreater✲ du groß gemacht.

3Als ich rief zu dir, da hast du mich erhört,

hast mir Mut verliehn: in mein Herz kam Kraft.

4Danken werden dir, HERR, alle Könige der Erde,

wenn sie hören die Worte deines Mundes,

5und werden singen vom Walten des HERRN,

denn groß ist die Herrlichkeit des HERRN.

6Denn der HERR ist erhaben und sieht doch den Niedrigen,

den Stolzen aber erkennt er schon von ferne.

7Wenn ich auch mitten in Drangsal wandle,

erhältst du mir dennoch das Leben; du streckst deine Hand aus gegen die
Wut meiner Feinde, und deine Rechte hilft mir.

8Der HERR wird's mir zum Heil vollführen;

o HERR, deine Gnade walte für immer: laß die Werke deiner Hände nicht
fahren\textless sup title=``=~nicht im Stich''\textgreater✲!

\hypertarget{gott-der-allwissende-und-allgegenwuxe4rtige}{%
\subsubsection{Gott der Allwissende und
Allgegenwärtige}\label{gott-der-allwissende-und-allgegenwuxe4rtige}}

\hypertarget{section-138}{%
\section{139}\label{section-138}}

1Dem Musikmeister, von David ein Psalm. HERR, du erforschest mich und
kennst mich; 2du weißt es, ob ich sitze oder aufstehe,

du verstehst, was ich denke, von ferne;

3ob ich wandre oder ruhe, du prüfst es

und bist mit all meinen Wegen vertraut;

4denn ehe ein Wort auf meiner Zunge liegt,

kennst du, o HERR, es schon genau.

5Du hältst mich von hinten und von vorne umschlossen

und hast deine Hand auf mich gelegt.

6Zu wunderbar ist solches Wissen für mich,

zu hoch: ich vermag's nicht zu begreifen!

7Wohin soll ich gehn vor deinem Geist

und wohin fliehn vor deinem Angesicht?

8Führe ich auf zum Himmel, so wärst du da,

und lagert' ich mich in der Unterwelt, so wärst du dort;

9nähme ich Schwingen des Morgenrots zum Flug

und ließe mich nieder am äußersten Westmeer,

10so würde auch dort deine Hand mich führen

und deine Rechte mich fassen;

11und spräch' ich: »Lauter Finsternis soll mich umhüllen

und Nacht sei das Licht um mich her!«~--

12auch die Finsternis würde für dich nicht finster sein,

vielmehr die Nacht dir leuchten wie der Tag: Finsternis wäre für dich
wie das Licht.

13Denn du bist's, der meine Nieren\textless sup title=``d.h. mein
Innerstes''\textgreater✲ gebildet,

mich gewoben im Schoß meiner Mutter.

14Ich danke dir, daß ich so überaus wunderbar bereitet bin:

wunderbar sind deine Werke, und meine Seele erkennt das wohl.

15Meine Wesensgestaltung war dir nicht verborgen,

als im Dunkeln ich gebildet ward, kunstvoll gewirkt in den Tiefen der
Erde.{A}

16Deine Augen sahen mich schon als formlosen Keim,

und in deinem Buch standen eingeschrieben alle Tage, die vorbedacht
waren, als noch keiner von ihnen da war.

17Für mich nun -- wie kostbar sind deine Gedanken, o Gott,

wie gewaltig sind ihre Summen!

18Wollt' ich sie zählen: ihrer sind mehr als des Sandes;

wenn ich erwache, bin ich noch immer bei dir.

19Möchtest du doch die Frevler töten, o Gott!

Und ihr Männer✲ der Blutschuld, weichet von mir!

20Sie, die von dir mit Arglist\textless sup title=``oder:
Hintergedanken''\textgreater✲ reden,

mit Falschheit reden als deine Widersacher.

21Sollt' ich nicht hassen, die dich, HERR, hassen,

nicht verabscheun, die sich erheben gegen dich?

22Ja, ich hasse sie mit tödlichem Haß:

als Feinde gelten sie mir.

23Erforsche mich, Gott, und erkenne mein Herz,

prüfe mich und erkenne meine Gedanken!

24Und sieh, ob ich wandle auf trüglichem Wege,

und leite mich auf dem ewigen Wege!

\hypertarget{gebet-um-errettung-von-hinterlistigen-feinden}{%
\subsubsection{Gebet um Errettung von hinterlistigen
Feinden}\label{gebet-um-errettung-von-hinterlistigen-feinden}}

\hypertarget{section-139}{%
\section{140}\label{section-139}}

1Dem Musikmeister, ein Psalm von David. 2Rette mich, HERR, von den bösen
Menschen!

Vor den Freunden der Gewalttat schütze mich,

3die auf Böses im Herzen sinnen

und allezeit Streit erregen!

4Sie spitzen\textless sup title=``oder: schärfen''\textgreater✲ ihre
Zungen der Schlange gleich,

Otterngift ist hinter ihren Lippen. SELA.

5Behüte mich, HERR, vor den Händen der Frevler!

Vor den Freunden der Gewalttat schütze mich, die darauf sinnen, zu Fall
mich zu bringen!{A}

6Die Frechen legen mir heimlich Schlingen und Fallstricke,

spannen Netze aus zur Seite des Wegs und stellen mir Fallen. SELA.

7Ich sage zum HERRN: »Du bist mein Gott,

vernimm, o HERR, mein lautes Flehen!«

8O HERR, mein Gott, meine starke Hilfe,

du hast mein Haupt beschirmt am Tage des Kampfes:

9gewähre nicht, HERR, die Gelüste✲ der Frevler,

laß ihr böses Trachten nicht gelingen! SELA.

10Erheben sie das Haupt rings um mich her,

so falle das Unheil ihrer Lippen auf sie selbst!

11Er lasse glühende Kohlen auf sie regnen,

ins Feuer stürze er sie, in Wasserfluten, daß sie nicht aufstehn können!

12Der Verleumder wird keinen Halt im Lande gewinnen;

der Mann der Gewalttat jage das Unglück Stoß auf Stoß{A}!

13Ich weiß, der HERR wird führen des Elenden Sache,

den Rechtsstreit der Armen.

14Ja, die Gerechten werden deinen Namen preisen,

die Redlichen bleiben wohnen vor deinem Angesicht.

\hypertarget{gebet-um-bewahrung-vor-buxf6sem-und-vor-verfolgern}{%
\subsubsection{Gebet um Bewahrung vor Bösem und vor
Verfolgern}\label{gebet-um-bewahrung-vor-buxf6sem-und-vor-verfolgern}}

\hypertarget{section-140}{%
\section{141}\label{section-140}}

1Ein Psalm Davids. HERR, ich rufe dich, eile mir zu Hilfe!

Vernimm meine Stimme, wenn ich zu dir rufe!

2Laß mein Gebet dir als Räucherwerk gelten,

das Aufheben meiner Hände als Abendopfer!

3Stelle, o HERR, eine Wache vor meinen Mund,

behüte das Tor meiner Lippen!

4Laß mein Herz sich nicht neigen zu bösem Tun,

daß ich gottlose Taten verübe im Verein mit Männern, die Übeltäter sind:
ich mag nicht essen von ihren Leckerbissen!

5Schlägt mich ein Gerechter: das ist Liebe,

und weist er mich zurecht: das ist Salbe fürs Haupt; nicht soll mein
Haupt dagegen sich sträuben; denn noch ist's der Fall, daß für ihre
Bosheit\textless sup title=``oder: Nöte?''\textgreater✲ mein Gebet
erfolgt.{A}

6Sind ihre Richter eine Felswand hinabgestürzt worden,

so wird man hören, daß meine Worte lieblich\textless sup title=``oder:
liebreich''\textgreater✲ sind.

7Wie einer das Erdreich furcht und aufreißt,

so sind unsere Gebeine hingestreut für den Rachen der Unterwelt.

8Denn auf dich, o Allherr, sind meine Augen gerichtet,

bei dir such' ich Zuflucht: gib mein Leben nicht hin in den Tod!{A}

9Behüte mich vor der Schlinge, die sie mir gelegt,

und vor den Fallstricken der Übeltäter!

10Laß die Frevler fallen in ihre eigenen Netze,

während ich zugleich daran vorübergehe!

\hypertarget{hilferuf-in-schwerer-bedruxe4ngnis}{%
\subsubsection{Hilferuf in schwerer
Bedrängnis}\label{hilferuf-in-schwerer-bedruxe4ngnis}}

\hypertarget{section-141}{%
\section{142}\label{section-141}}

1Ein Lehrgedicht\textless sup title=``vgl. 32,1''\textgreater✲ Davids,
als er sich in der Höhle befand\textless sup title=``vgl.
57,1''\textgreater✲, ein Gebet. 2Laut schrei' ich zum HERRN,

laut fleh' ich zum HERRN,

3ich schütte meine Klage vor ihm aus,

tue kund vor ihm meine Not.

4Wenn mein Geist in mir verschmachtet✲,

du kennst doch meinen Lebenspfad. Auf dem Wege, den ich wandeln
will\textless sup title=``oder: gehen muß''\textgreater✲, hat man mir
heimlich ein Fangnetz ausgespannt.

5Blick' ich nach rechts und halte Umschau:

ach, da ist keiner, der mich versteht\textless sup title=``oder:
kennt''\textgreater✲! Verschlossen ist mir jede Zuflucht: niemand fragt
nach mir!

6Ich schreie, HERR, zu dir,

ich sage: »Du bist meine Zuflucht, mein Anteil im Lande der Lebenden!«

7Ach, merk' auf mein Flehn, denn ich bin gar schwach geworden!

Rette mich vor meinen Verfolgern, denn sie sind mir zu stark!

8Führe mich aus der Umkreisung hinaus,

damit ich deinen Namen preise! Die Gerechten werden bei mir erwarten,
daß du mir wohltust.{A}

\hypertarget{hilferuf-in-uxe4uuxdferer-und-innerer-not-siebenter-buuxdfpsalm}{%
\subsubsection{Hilferuf in äußerer und innerer Not (Siebenter
Bußpsalm)}\label{hilferuf-in-uxe4uuxdferer-und-innerer-not-siebenter-buuxdfpsalm}}

\hypertarget{section-142}{%
\section{143}\label{section-142}}

1Ein Psalm Davids. HERR, höre mein Gebet,

vernimm mein Flehen um deiner Treue willen! Erhöre mich nach deiner
Gerechtigkeit

2und geh nicht ins Gericht mit deinem Knecht!

Denn vor dir ist kein Lebender gerecht.

3Ach, der Feind verfolgt meine Seele\textless sup title=``=~trachtet mir
nach dem Leben''\textgreater✲,

hat mein Leben zu Boden geschlagen, versetzt mich in Nacht wie die ewig
Toten\textless sup title=``oder: längst Gestorbnen''\textgreater✲.

4Nun will mein Geist in mir verzagen\textless sup title=``vgl.
142,4''\textgreater✲,

mein Herz erstarrt mir in der Brust.

5Ich gedenke der früheren Tage\textless sup title=``oder:
Zeiten''\textgreater✲,

rufe all deine Taten mir ins Gedächtnis, denke über dein ganzes Walten
nach;

6ich breite meine Hände aus nach dir:

meine Seele dürstet nach dir wie lechzendes Land. SELA.

7Eile, mich zu erhören, o HERR: mein Geist verzagt!

Verhülle dein Angesicht nicht vor mir, sonst werde ich denen gleich, die
ins Totenreich gefahren.

8Laß schon früh am Morgen mich deine Gnade erfahren,

denn auf dich vertraue ich! Tu mir kund den Weg, den ich gehn soll, denn
zu dir erhebe ich meine Seele!

9Rette mich, HERR, von meinen Feinden:

zu dir nehme ich meine Zuflucht!

10Lehre mich das dir Wohlgefällige tun,

denn du bist mein Gott: dein guter Geist führe mich auf ebener Bahn!

11Um deines Namens willen, HERR, erhalt' mich am Leben,

nach deiner Gerechtigkeit hilf mir aus der Not,

12und nach deiner Gnade vertilge meine Feinde

und vernichte alle, die meine Seele✲ bedrängen; ich bin ja dein Knecht!

\hypertarget{lob--und-bittgebet-israels-segensfuxfclle}{%
\subsubsection{Lob- und Bittgebet; Israels
Segensfülle}\label{lob--und-bittgebet-israels-segensfuxfclle}}

\hypertarget{section-143}{%
\section{144}\label{section-143}}

1Von David. Gepriesen sei der HERR, mein Fels,

der meine Hände✲ tüchtig gemacht zum Kampf, meine Finger geschickt zum
Kriege,

2mein Wohltäter und meine Burg, meine Feste und mein Retter,

mein Schild und der, auf den ich vertraue; Völker hat er mir
unterworfen!

3HERR, was ist der Mensch, daß du ihn beachtest,

des Menschen Sohn, daß du seiner gedenkst?

4Der Mensch gleicht einem Hauch,

seine Tage sind wie ein Schatten, der vorüberfliegt.

5HERR, neige deinen Himmel und fahre herab,

rühre die Berge an, daß sie rauchen!

6Schleudre Blitze und zerstreue sie\textless sup title=``d.h. meine
Feinde''\textgreater✲,

schieß deine Pfeile ab und laß sie zerstieben!

7Strecke deine Hände aus der Höhe herab, reiß mich heraus

und rette mich aus gewaltigen Fluten, aus der Hand der Söhne der Fremde,

8deren Lippen Lügen reden

und deren Rechte mit Täuschung umgeht.{A}

9Gott, ein neues Lied will (alsdann) ich dir singen,

auf zehnsaitiger Harfe dir spielen:

10dir, der den Königen Sieg verleiht,

der David, seinen Knecht, entrissen dem mörderischen Schwert.

11Reiß mich heraus und rette mich aus der Hand der Söhne der Fremde,

deren Lippen Lügen reden und deren Rechte mit Täuschung
umgeht\textless sup title=``vgl. V.8''\textgreater✲!~--

12O gib, daß unsere Söhne in ihrer Jugendkraft

hochgewachsenen Setzlingen gleichen! Daß unsre Töchter seien wie
schöngemeißelte Ecksäulen an prächtig gebauten Palästen!

13Daß unsre Speicher, wohlgefüllt,

spenden einen Vorrat nach dem andern! Daß unser Kleinvieh sich
tausendfach mehre, zehntausendfach auf unsern Triften!

14Daß unsre Rinder trächtig seien ohne Mißgeschick und ohne Fehlgeburt,

keine Spaltung im Volk und kein Wehgeschrei auf unsern Straßen!

15Glückselig das Volk, dem es so ergeht!

Glückselig das Volk, dessen Gott der HERR ist!

\hypertarget{loblied-auf-die-gruxf6uxdfe-und-guxfcte-gottes-des-erhalters-und-beherrschers-der-welt}{%
\subsubsection{Loblied auf die Größe und Güte Gottes, des Erhalters und
Beherrschers der
Welt}\label{loblied-auf-die-gruxf6uxdfe-und-guxfcte-gottes-des-erhalters-und-beherrschers-der-welt}}

\hypertarget{section-144}{%
\section{145}\label{section-144}}

1Ein Loblied von David. Ich will dich erheben, mein Gott, du König,

und deinen Namen preisen immer und ewig!

2An jedem Tage will ich dich preisen

und deinen Namen rühmen immer und ewig!

3Groß ist der HERR und hoch zu rühmen,

und seine Größe ist unausforschlich.

4Ein Geschlecht wird dem andern rühmen deine Werke

und kundtun deine gewaltigen Taten.

5Von der herrlichen Pracht deiner Hoheit✲ will ich reden,

und von deinen Wundertaten (will ich singen).

6Von der Macht deines furchtbaren Waltens wird man reden,

und deine Größe\textless sup title=``oder: Großtaten''\textgreater✲ --
davon will ich erzählen!

7Den Ruhm deiner reichen Güte wird man verkünden

und jubelnd preisen deine Gerechtigkeit.

8Gnädig und barmherzig ist der HERR,

langmütig und reich an Güte.

9Der HERR ist gütig gegen alle,

und sein Erbarmen umfaßt alle seine Werke.

10Alle deine Werke werden dich loben, HERR,

und deine Frommen dich preisen;

11die Herrlichkeit deines Königtums werden sie rühmen

und reden von deiner Macht,

12um den Menschenkindern kundzutun seine mächtigen Taten

und die herrliche Pracht seines Königtums.

13Dein Reich ist ein Reich für alle Ewigkeiten,

und deine Herrschaft besteht durch alle Geschlechter. Getreu ist der
HERR in seinen Worten und heilig in all seinem Tun.

14Der HERR stützt alle Fallenden

und richtet alle Gebeugten auf.

15Aller Augen warten auf dich,

und du gibst ihnen ihre Speise zu seiner\textless sup title=``d.h.
rechten''\textgreater✲ Zeit;

16du tust deine Hand auf

und sättigst alles, was lebt, mit Wohlgefallen{A}.

17Gerecht ist der HERR in all seinem Walten

und liebreich in all seinem Tun.

18Der HERR ist nahe allen, die ihn anrufen,

allen, die ihn in Treue{A} anrufen;

19er erfüllt das Begehren derer, die ihn fürchten,

er hört ihr Schreien und hilft ihnen.

20Der HERR behütet alle, die ihn lieben,

doch alle Frevler rottet er aus.

21Mein Mund soll verkünden den Lobpreis des HERRN,

und alles Fleisch\textless sup title=``=~alle Welt''\textgreater✲ soll
preisen seinen heiligen Namen immer und ewig!

\hypertarget{wohl-dem-der-auf-den-herr-vertraut}{%
\subsubsection{Wohl dem, der auf den Herr
vertraut!}\label{wohl-dem-der-auf-den-herr-vertraut}}

\hypertarget{section-145}{%
\section{146}\label{section-145}}

1Halleluja! Lobe den HERRN, meine Seele! 2Loben will ich den HERRN,
solange ich lebe,

will meinem Gott lobsingen, solange ich bin!

3Verlaßt euch nicht auf Fürsten,

nicht auf Menschen, die ja nicht helfen können!

4Geht der Odem\textless sup title=``oder: Geist''\textgreater✲ ihnen
aus, so kehren sie zurück zum Staube;

am gleichen Tage ist's aus mit ihren Plänen.

5Wohl dem, dessen Hilfe der Gott Jakobs ist,

dessen Hoffnung ruht auf dem HERRN, seinem Gott,

6auf ihm, der Himmel und Erde geschaffen,

das Meer mit allem, was in ihnen ist, der Treue ewiglich hält;

7der Recht den Unterdrückten schafft

und Brot den Hungrigen gibt. Der HERR macht die Gefangenen frei;

8der HERR gibt Blinden das Augenlicht,

der HERR richtet die Gebeugten auf, der HERR hat lieb die Gerechten;

9der HERR behütet den Fremdling;

Waisen und Witwen hält er aufrecht; doch den Weg der Gottlosen macht er
zum Irrweg.

10Der HERR wird König in Ewigkeit sein,

dein Gott, o Zion, für und für! Halleluja!

\hypertarget{lobpreis-der-allmacht-guxfcte-und-weisheit-gottes}{%
\subsubsection{Lobpreis der Allmacht, Güte und Weisheit
Gottes}\label{lobpreis-der-allmacht-guxfcte-und-weisheit-gottes}}

\hypertarget{section-146}{%
\section{147}\label{section-146}}

1Preiset den HERRN! Denn schön\textless sup title=``oder:
löblich''\textgreater✲ ist's, unserm Gott zu lobsingen,

ja lieblich und wohlgeziemend ist Lobgesang.

2Der HERR baut Jerusalem wieder auf,

er sammelt Israels zerstreute Söhne;

3er heilt, die zerbrochnen Herzens sind,

und verbindet ihre Wunden;

4er bestimmt den Sternen ihre Zahl

und ruft\textless sup title=``oder: benennt''\textgreater✲ sie alle mit
Namen\textless sup title=``Jes 40,26''\textgreater✲.

5Groß ist unser Herr und allgewaltig,

für seine Weisheit gibt's kein Maß.

6Der HERR hilft den Gebeugten auf,

doch die Gottlosen stürzt er nieder zu Boden.

7Stimmt für den HERRN ein Danklied an,

spielt unserm Gott auf der Zither~--

8ihm, der den Himmel mit Wolken bedeckt

und Regen schafft für die Erde, der Gras auf den Bergen sprießen läßt,

9der den Tieren ihr Futter gibt,

den jungen Raben, die zu ihm schreien!

10Er hat nicht Lust an der Stärke des Rosses,

nicht Gefallen an den Schenkeln✲ des Mannes;

11Gefallen hat der HERR an denen, die ihn fürchten,

an denen, die auf seine Gnade harren.

12Preise den HERRN, Jerusalem,

lobsinge, Zion, deinem Gott!

13Denn er hat die Riegel deiner Tore stark gemacht,

gesegnet deine Kinder in deiner Mitte;

14er schafft deinen Grenzen Sicherheit,

sättigt dich mit dem Mark des Weizens.{A}

15Er läßt sein Machtwort nieder zur Erde gehn:

gar eilig läuft sein Gebot dahin;

16er sendet Schnee wie Wollflocken

und streut den Reif wie Asche aus;

17er wirft seinen Hagel wie Brocken herab:

wer kann bestehn vor seiner Kälte?

18Doch läßt er sein Gebot ergehn, so macht er sie schmelzen;

läßt er wehn seinen Tauwind, so rieseln die Wasser.

19Er hat Jakob sein Wort verkündet,

Israel sein Gesetz und seine Rechte.

20Mit keinem (anderen) Volk ist so er verfahren,

drum kennen sie seine Rechte nicht. Halleluja!

\hypertarget{alle-welt-die-ganze-schuxf6pfung-lobe-den-herrn}{%
\subsubsection{Alle Welt (=~die ganze Schöpfung) lobe den
Herrn!}\label{alle-welt-die-ganze-schuxf6pfung-lobe-den-herrn}}

\hypertarget{section-147}{%
\section{148}\label{section-147}}

1Halleluja! Lobet den HERRN vom Himmel her,

lobet ihn in den Himmelshöhen!

2Lobet ihn, alle seine Engel,

lobet ihn, alle seine Heerscharen!

3Lobet ihn, Sonne und Mond,

lobet ihn, alle ihr leuchtenden Sterne!

4Lobet ihn, ihr Himmel der Himmel\textless sup title=``vgl. 5.Mose
10,14''\textgreater✲,

und ihr Wasser oberhalb des Himmels!

5Loben sollen sie den Namen des HERRN,

denn er gebot, da waren sie geschaffen,

6und er hat sie hingestellt für immer und ewig

und ihnen ein Gesetz gegeben, das übertreten sie nicht.

7Lobet den HERRN von der Erde her,

ihr Seeungeheuer und alle Meeresfluten,

8du Feuer und Hagel, du Schnee und Nebel,

du Sturmwind, der sein Gebot vollzieht;

9ihr Berge und Hügel allesamt,

ihr Fruchtbäume und Zedern allzumal,

10ihr Tiere alle, wilde und zahme,

du Gewürm und ihr beschwingte Vögel,

11ihr Könige der Erde und alle Völkerschaften,

ihr Fürsten und alle Richter auf Erden,

12ihr Jünglinge mitsamt den Jungfrauen,

ihr Greise samt den Jungen!

13Sie alle sollen loben den Namen des HERRN,

denn sein Name allein ist erhaben; seine Hoheit✲ überragt die Erde und
den Himmel.

14Er hat sein Volk aufs neue zu Ehren gebracht:

ein Ruhm ist das für alle seine Frommen, für Israels Söhne\textless sup
title=``oder: Kinder''\textgreater✲, das Volk, das am nächsten ihm
steht. Halleluja!

\hypertarget{israels-sieges--und-rachelied}{%
\subsubsection{Israels Sieges- und
Rachelied}\label{israels-sieges--und-rachelied}}

\hypertarget{section-148}{%
\section{149}\label{section-148}}

1Halleluja!

Singet dem HERRN ein neues Lied, seinen Lobpreis in der Versammlung der
Frommen{A}!

2Es freue sich Israel seines Schöpfers,

Zions Söhne sollen jubeln ob ihrem König!

3Sie sollen seinen Namen preisen im Reigentanz,

mit Pauken und Zithern ihm spielen!

4Denn der HERR hat Wohlgefallen an seinem Volk;

er schmückt\textless sup title=``oder: krönt''\textgreater✲ die
Gebeugten mit Sieg.{A}

5Frohlocken sollen die Frommen mit Stolz,

sollen jauchzen auf ihren Lagern,

6Lobeserhebungen Gottes im Mund

und ein doppelschneidiges Schwert in der Hand:

7um Rache zu vollziehn an den Heiden,

Vergeltung an den Völkern,

8um ihre Könige mit Ketten zu binden

und ihre Edlen mit eisernen Fesseln,

9um das längst geschriebene Urteil an ihnen zu vollstrecken:

eine Ehre ist dies für alle seine Frommen! Halleluja!

\hypertarget{alles-was-odem-hat-lobe-den-herrn}{%
\subsubsection{Alles, was Odem hat, lobe den
Herrn!}\label{alles-was-odem-hat-lobe-den-herrn}}

\hypertarget{section-149}{%
\section{150}\label{section-149}}

1Halleluja! Lobt Gott in seinem (himmlischen) Heiligtum,

lobt ihn in\textless sup title=``oder: an''\textgreater✲ seiner starken
Feste{A}!

2Lobt ihn ob seinen Wundertaten,

lobt ihn nach seiner gewaltigen Größe!

3Lobt ihn mit Posaunenschall,

lobt ihn mit Harfe und Zither!

4Lobt ihn mit Pauke und Reigentanz,

lobt ihn mit Saitenspiel und Flöte!

5Lobt ihn mit hellklingenden Zimbeln,

lobt ihn mit lautschallenden Zimbeln!

6Alles, was Odem hat, lobe den HERRN! Halleluja!
