\hypertarget{die-heilsbotschaft-nach-matthuxe4us}{%
\section{DIE HEILSBOTSCHAFT NACH
MATTHÄUS}\label{die-heilsbotschaft-nach-matthuxe4us}}

\hypertarget{i.-kindheitsgeschichte-jesu-kap.-1-2}{%
\subsection{I. Kindheitsgeschichte Jesu (Kap.
1-2)}\label{i.-kindheitsgeschichte-jesu-kap.-1-2}}

\hypertarget{stammbaum-jesu-als-des-nachkommen-abrahams-und-davids}{%
\subsubsection{1. Stammbaum Jesu als des Nachkommen Abrahams und
Davids}\label{stammbaum-jesu-als-des-nachkommen-abrahams-und-davids}}

\hypertarget{section}{%
\section{1}\label{section}}

\bibleverse{1} Stammbaum Jesu Christi, des Sohnes✲ Davids, des Sohnes✲
Abrahams: \bibleverse{2} Abraham war der Vater Isaaks; Isaak der Vater
Jakobs; Jakob der Vater Judas und seiner Brüder; \bibleverse{3} Juda war
der Vater des Phares und des Zara, deren Mutter Thamar war; Phares war
der Vater Esroms; Esrom der Vater Arams; \bibleverse{4} Aram der Vater
Aminadabs; Aminadab der Vater Naassons; Naasson der Vater Salmons;
\bibleverse{5} Salmon der Vater des Boas, dessen Mutter Rahab war; Boas
der Vater Obeds, dessen Mutter Ruth war; Obed war der Vater
Isais\textless sup title=``oder: Jesses''\textgreater✲; \bibleverse{6}
Isai war der Vater des Königs David.

David war der Vater Salomos, dessen Mutter (Bathseba) die Frau Urias
gewesen war; \bibleverse{7} Salomo war der Vater Rehabeams; Rehabeam der
Vater Abias; Abia der Vater Asas; \bibleverse{8} Asa der Vater
Josaphats; Josaphat der Vater Jorams; Joram der Vater Ussias;
\bibleverse{9} Ussia der Vater Jothams; Jotham der Vater des Ahas; Ahas
der Vater Hiskias; \bibleverse{10} Hiskia der Vater Manasses; Manasse
der Vater des Amon; Amon der Vater Josias; \bibleverse{11} Josia der
Vater Jechonjas und seiner Brüder zur Zeit der Wegführung nach
Babylon\textless sup title=``oder: der babylonischen
Gefangenschaft''\textgreater✲. \bibleverse{12} Nach der babylonischen
Gefangenschaft war Jechonja der Vater Salathiels; Salathiel der Vater
Serubabels; \bibleverse{13} Serubabel der Vater Abihuds; Abihud der
Vater Eljakims; Eljakim der Vater Azors; \bibleverse{14} Azor der Vater
Sadoks; Sadok der Vater Achims; Achim der Vater Elihuds; \bibleverse{15}
Elihud der Vater Eleasars; Eleasar der Vater Matthans; Matthan der Vater
Jakobs; \bibleverse{16} Jakob der Vater Josephs, des Ehemannes der
Maria, von welcher Jesus geboren ward, der da Christus\textless sup
title=``oder: der Messias, d.h. der Gesalbte''\textgreater✲ genannt
wird.

\bibleverse{17} Man sieht: von Abraham bis David sind es im ganzen
vierzehn Geschlechter, von David bis zur babylonischen Gefangenschaft
ebenfalls vierzehn Geschlechter, endlich von der babylonischen
Gefangenschaft bis auf Christus nochmals vierzehn Geschlechter.

\hypertarget{geburt-und-name-jesu}{%
\subsubsection{2. Geburt und Name Jesu}\label{geburt-und-name-jesu}}

\bibleverse{18} Mit der Geburt Jesu Christi aber verhielt es sich so:
Als seine Mutter Maria mit Joseph verlobt war, stellte es sich heraus,
noch ehe sie zusammengekommen✲ waren, daß sie vom heiligen Geist guter
Hoffnung war. \bibleverse{19} Da faßte Joseph, ihr Verlobter, der ein
rechtschaffener Mann war und sie nicht in üblen Ruf bringen wollte, den
Entschluß, sich ohne Aufsehen zu erregen von ihr loszusagen.
\bibleverse{20} Doch als er sich mit solchen Gedanken trug, siehe, da
erschien ihm ein Engel des Herrn im Traum und sagte zu ihm: »Joseph,
Sohn✲ Davids, trage keinerlei Bedenken, Maria, deine Verlobte, als
Ehefrau zu dir zu nehmen! Denn das von ihr zu erwartende Kind stammt vom
heiligen Geist. \bibleverse{21} Sie wird Mutter eines Sohnes werden, dem
du den Namen Jesus geben sollst; denn er ist es, der sein Volk von ihren
Sünden erretten wird.«\textless sup title=``Ps 130,8''\textgreater✲
\bibleverse{22} Dies alles ist aber geschehen, damit das Wort erfüllt
würde, das der Herr durch den Propheten gesprochen hat, der da
sagt\textless sup title=``Jes 7,14''\textgreater✲: \bibleverse{23}
»Siehe, die Jungfrau wird guter Hoffnung und Mutter eines Sohnes werden,
dem man den Namen Immanuel geben wird«, das heißt übersetzt: ›Mit uns
ist Gott.‹~-- \bibleverse{24} Als Joseph dann aus dem Schlaf erwacht
war, tat er, wie der Engel des Herrn ihm geboten hatte: er nahm seine
Verlobte (als Gattin) zu sich, \bibleverse{25} verkehrte aber nicht
ehelich mit ihr, bis sie einen Sohn geboren hatte; dem gab er den Namen
Jesus.

\hypertarget{weise-magier-aus-dem-morgenlande-kommen-zum-jesuskinde-und-huldigen-ihm}{%
\subsubsection{3. Weise (Magier) aus dem Morgenlande kommen zum
Jesuskinde und huldigen
ihm}\label{weise-magier-aus-dem-morgenlande-kommen-zum-jesuskinde-und-huldigen-ihm}}

\hypertarget{section-1}{%
\section{2}\label{section-1}}

\bibleverse{1} Als nun Jesus zu Bethlehem in Judäa in den
Tagen\textless sup title=``=~unter der Regierung''\textgreater✲ des
Königs Herodes geboren war, da kamen Weise aus dem Osten\textless sup
title=``oder: Morgenlande''\textgreater✲ nach Jerusalem \bibleverse{2}
und fragten: »Wo ist der neugeborene König der Juden? Wir haben nämlich
seinen Stern im Aufgehen\textless sup title=``oder: im
Osten''\textgreater✲ gesehen und sind hergekommen, um ihm unsere
Huldigung darzubringen.« \bibleverse{3} Als der König Herodes das
vernahm, erschrak er sehr und ganz Jerusalem mit ihm; \bibleverse{4} und
er ließ alle Hohenpriester und Schriftgelehrten des Volks zusammenkommen
und erkundigte sich bei ihnen, wo Christus\textless sup title=``=~der
Messias; vgl. 1,16''\textgreater✲ geboren werden sollte. \bibleverse{5}
Sie antworteten ihm: »Zu Bethlehem in Judäa; denn so steht bei dem
Propheten geschrieben\textless sup title=``Mi 5,1''\textgreater✲:
\bibleverse{6} ›Du, Bethlehem im Lande Judas, du bist durchaus nicht die
unbedeutendste unter den Fürstenstädten Judas; denn aus dir wird ein
Führer\textless sup title=``oder: Fürst''\textgreater✲ hervorgehen, der
mein Volk Israel weiden\textless sup title=``=~als Hirte
leiten''\textgreater✲ wird.‹« \bibleverse{7} Daraufhin berief Herodes
die Weisen heimlich zu sich und ließ sich von ihnen genau die Zeit
angeben, wann der Stern erschienen wäre; \bibleverse{8} dann wies er sie
nach Bethlehem und sagte: »Zieht hin und stellt genaue Nachforschungen
nach dem Kindlein an; und wenn ihr es gefunden habt, so teilt es mir
mit, damit auch ich hingehe und ihm meine Huldigung darbringe.«
\bibleverse{9} Als sie das vom Könige gehört hatten, machten sie sich
auf den Weg; und siehe da, der Stern, den sie im Osten\textless sup
title=``oder: bei seinem Aufgang''\textgreater✲ gesehen hatten, ging vor
ihnen her, bis er endlich über dem Ort stehen blieb, wo das Kindlein
sich befand. \bibleverse{10} Als sie den Stern erblickten, wurden sie
hoch erfreut. \bibleverse{11} Sie traten in das Haus ein und sahen das
Kindlein bei seiner Mutter Maria, warfen sich vor ihm nieder und
huldigten ihm; alsdann taten sie ihre Schatzbeutel auf und brachten ihm
Geschenke dar: Gold, Weihrauch und Myrrhe. \bibleverse{12} Weil sie
hierauf im Traume die göttliche Weisung erhielten, nicht wieder zu
Herodes zurückzukehren, zogen sie auf einem anderen Wege in ihr
Heimatland zurück.

\hypertarget{verfolgung-und-rettung-des-jesuskindes}{%
\subsubsection{4. Verfolgung und Rettung des
Jesuskindes}\label{verfolgung-und-rettung-des-jesuskindes}}

\hypertarget{a-josephs-flucht-nach-uxe4gypten}{%
\paragraph{a) Josephs Flucht nach
Ägypten}\label{a-josephs-flucht-nach-uxe4gypten}}

\bibleverse{13} Als sie nun weggezogen waren, da erschien ein Engel des
Herrn dem Joseph im Traume und gebot ihm: »Steh auf, nimm das Kindlein
und seine Mutter mit dir und fliehe nach Ägypten und bleibe so lange
dort, bis ich's dir sage! Denn Herodes geht damit um, nach dem Kindlein
suchen zu lassen, um es umzubringen.« \bibleverse{14} Da stand Joseph
auf, nahm in der Nacht das Kindlein und seine Mutter mit sich und
entwich nach Ägypten; \bibleverse{15} dort blieb er bis zum Tode des
Herodes. So sollte sich das Wort erfüllen, das der Herr durch den
Propheten gesprochen hat, der da sagt\textless sup title=``Hos
11,1''\textgreater✲: »Aus Ägypten habe ich meinen Sohn gerufen.«

\hypertarget{b-der-kindermord-des-herodes-in-bethlehem}{%
\paragraph{b) Der Kindermord des Herodes in
Bethlehem}\label{b-der-kindermord-des-herodes-in-bethlehem}}

\bibleverse{16} Als Herodes sich nun von den Weisen hintergangen sah,
geriet er in heftigen Zorn; er sandte (Diener) hin und ließ in Bethlehem
und dem ganzen Umkreis des Ortes sämtliche Knaben im Alter von zwei und
weniger Jahren töten, entsprechend der Zeit, die er sich von den Weisen
genau hatte angeben lassen. \bibleverse{17} Damals erfüllte sich, was
durch den Propheten Jeremia\textless sup title=``Jer
31,15''\textgreater✲ gesagt ist, der spricht: \bibleverse{18} »Ein
Geschrei hat man in Rama vernommen, lautes Weinen und viel Wehklagen:
Rahel weint um ihre Kinder und will sich nicht trösten lassen, daß sie
nicht mehr da sind.«

\hypertarget{c-josephs-ruxfcckkehr-aus-uxe4gypten-und-seine-niederlassung-in-nazareth}{%
\paragraph{c) Josephs Rückkehr aus Ägypten und seine Niederlassung in
Nazareth}\label{c-josephs-ruxfcckkehr-aus-uxe4gypten-und-seine-niederlassung-in-nazareth}}

\bibleverse{19} Als Herodes aber gestorben war, da erschien ein Engel
des Herrn dem Joseph in Ägypten im Traum \bibleverse{20} und gebot ihm:
»Steh auf, nimm das Kindlein und seine Mutter mit dir und ziehe heim ins
Land Israel; denn die sind gestorben, die dem Kindlein nach dem Leben
getrachtet haben.«\textless sup title=``2.Mose 4,19''\textgreater✲
\bibleverse{21} Da stand Joseph auf, nahm das Kindlein und seine Mutter
mit sich und kehrte in das Land Israel zurück. \bibleverse{22} Als er
aber vernahm, daß Archelaus an Stelle seines Vaters Herodes König über
Judäa sei, trug er Bedenken, dorthin zu gehen. Vielmehr begab er sich
infolge einer göttlichen Weisung, die er im Traum erhalten hatte, in die
Landschaft Galiläa \bibleverse{23} und ließ sich dort in einer Stadt
namens Nazareth nieder. So ging das Prophetenwort in Erfüllung, daß er
den Namen ›Nazarener‹ führen werde.

\hypertarget{ii.-johannes-der-vorluxe4ufer-jesu.-die-weihe-des-messias-31-411}{%
\subsection{II. Johannes, der Vorläufer Jesu. Die Weihe des Messias
(3,1-4,11)}\label{ii.-johannes-der-vorluxe4ufer-jesu.-die-weihe-des-messias-31-411}}

\hypertarget{auftreten-und-buuxdfpredigt-johannes-des-tuxe4ufers}{%
\subsubsection{1. Auftreten und Bußpredigt Johannes des
Täufers}\label{auftreten-und-buuxdfpredigt-johannes-des-tuxe4ufers}}

\hypertarget{section-2}{%
\section{3}\label{section-2}}

\bibleverse{1} In jenen Tagen trat aber Johannes der Täufer öffentlich
auf und predigte in der Wüste von Judäa: \bibleverse{2} »Tut Buße, denn
das Himmelreich ist nahe herbeigekommen!« \bibleverse{3} Dieser
(Johannes) ist nämlich der Mann, auf den sich das Wort des Propheten
Jesaja bezieht, der da sagt\textless sup title=``Jes
40,4''\textgreater✲: »Eine Stimme ruft laut in der Wüste: ›Bereitet den
Weg des Herrn! Macht gerade\textless sup title=``oder: ebnet
ihm''\textgreater✲ seine Pfade!‹« \bibleverse{4} Johannes selbst aber
trug ein Gewand von Kamelhaaren und einen Ledergurt um seine Hüften;
seine Nahrung bestand in Heuschrecken und wildem Honig. \bibleverse{5}
Da zog denn Jerusalem und ganz Judäa und die ganze Gegend am Jordan zu
ihm hinaus \bibleverse{6} und ließen sich im Jordanfluß von ihm taufen,
indem sie ihre Sünden offen bekannten. \bibleverse{7} Als er aber einmal
viele Pharisäer und Sadduzäer zu seiner Taufe kommen sah, sagte er zu
ihnen: »Ihr Schlangenbrut! Wer hat euch auf den Gedanken gebracht, dem
drohenden Zorngericht zu entfliehen? \bibleverse{8} So schafft denn
Früchte, die der Buße würdig sind\textless sup title=``oder:
entsprechen''\textgreater✲, \bibleverse{9} und laßt euch nicht in den
Sinn kommen, bei euch zu sagen\textless sup title=``oder: zu
denken''\textgreater✲: ›Wir haben ja Abraham zum Vater.‹ Denn ich sage
euch: Gott vermag dem Abraham aus den Steinen hier Kinder zu erwecken.
\bibleverse{10} Schon ist aber den Bäumen die Axt an die Wurzel gelegt,
und jeder Baum, der nicht gute Früchte bringt, wird abgehauen und ins
Feuer geworfen. \bibleverse{11} Ich taufe euch nur mit Wasser zur
Buße\textless sup title=``vgl. V.8''\textgreater✲; der aber nach mir
kommt, ist stärker als ich, und ich bin nicht gut genug, ihm seine
Schuhe abzunehmen\textless sup title=``oder:
nachzutragen''\textgreater✲: der wird euch mit heiligem Geist und mit
Feuer taufen. \bibleverse{12} Er hat die Worfschaufel in seiner Hand und
wird seine Tenne gründlich reinigen; seinen Weizen wird er in die
Scheuer sammeln, die Spreu aber mit unlöschbarem Feuer verbrennen.«

\hypertarget{die-taufe-und-messiasweihe-jesu}{%
\subsubsection{2. Die Taufe und Messiasweihe
Jesu}\label{die-taufe-und-messiasweihe-jesu}}

\bibleverse{13} Damals kam Jesus von Galiläa her an den Jordan zu
Johannes, um sich (auch) von ihm taufen zu lassen. \bibleverse{14} Der
wollte ihm aber nicht zu Willen sein und sagte: »Ich müßte von dir
getauft werden, und du kommst zu mir?« \bibleverse{15} Doch Jesus gab
ihm zur Antwort: »Laß es für diesmal geschehen\textless sup
title=``oder: so sein''\textgreater✲, denn so gebührt es uns, alle
Gerechtigkeit zu erfüllen.« Da gab Johannes ihm nach. \bibleverse{16}
Als Jesus aber getauft und soeben aus dem Wasser gestiegen war, siehe,
da taten sich ihm die Himmel auf, und er (Johannes oder Jesus) sah den
Geist Gottes wie eine Taube herabschweben und auf ihn\textless sup
title=``oder: sich''\textgreater✲ kommen. \bibleverse{17} Und siehe,
eine Stimme erscholl aus den Himmeln: »Dieser ist mein geliebter Sohn,
an dem ich Wohlgefallen gefunden habe!«

\hypertarget{die-versuchung-jesu-als-seine-messiasprobe}{%
\subsubsection{3. Die Versuchung Jesu als seine
Messiasprobe}\label{die-versuchung-jesu-als-seine-messiasprobe}}

\hypertarget{section-3}{%
\section{4}\label{section-3}}

\bibleverse{1} Hierauf wurde Jesus vom Geist✲ in die Wüste
hinaufgeführt, um vom Teufel versucht zu werden; \bibleverse{2} und als
er vierzig Tage und vierzig Nächte gefastet hatte, hungerte ihn zuletzt.
\bibleverse{3} Da trat der Versucher an ihn heran und sagte zu ihm:
»Bist du Gottes Sohn, so gebiete, daß diese Steine zu Broten werden.«
\bibleverse{4} Er aber gab ihm zur Antwort: »Es steht
geschrieben\textless sup title=``5.Mose 8,3''\textgreater✲: ›Nicht vom
Brot allein soll der Mensch leben, sondern von jedem Wort, das durch den
Mund Gottes ergeht.‹« \bibleverse{5} Hierauf nahm ihn der Teufel mit
sich in die heilige Stadt, stellte ihn dort auf die Zinne des Tempels
\bibleverse{6} und sagte zu ihm: »Bist du Gottes Sohn, so stürze dich
hier hinab! Denn es steht geschrieben\textless sup title=``Ps
91,11-12''\textgreater✲: ›Er wird seine Engel für dich entbieten, und
sie werden dich auf den Armen tragen, damit du mit deinem Fuß an keinen
Stein stoßest.‹« \bibleverse{7} Jesus antwortete ihm: »Es steht aber
auch geschrieben\textless sup title=``5.Mose 6,16''\textgreater✲: ›Du
sollst den Herrn, deinen Gott, nicht versuchen!‹« \bibleverse{8}
Nochmals nahm ihn der Teufel mit sich auf einen sehr hohen Berg, zeigte
ihm alle Königreiche der Welt samt ihrer Herrlichkeit \bibleverse{9} und
sagte zu ihm: »Dies alles will ich dir geben, wenn du dich niederwirfst
und mich anbetest.« \bibleverse{10} Da antwortete ihm Jesus: »Weg mit
dir, Satan! Denn es steht geschrieben\textless sup title=``5.Mose
6,13''\textgreater✲: ›Den Herrn, deinen Gott, sollst du anbeten und ihm
allein dienen!‹« \bibleverse{11} Nun ließ der Teufel von ihm ab, und
siehe, Engel traten zu ihm und leisteten ihm Dienste.

\hypertarget{iii.-das-wirken-jesu-in-galiluxe4a-412-111}{%
\subsection{III. Das Wirken Jesu in Galiläa
(4,12-11,1)}\label{iii.-das-wirken-jesu-in-galiluxe4a-412-111}}

\hypertarget{erstes-auftreten-jesu-in-galiluxe4a-und-berufung-der-ersten-juxfcnger}{%
\subsubsection{1. Erstes Auftreten Jesu in Galiläa und Berufung der
ersten
Jünger}\label{erstes-auftreten-jesu-in-galiluxe4a-und-berufung-der-ersten-juxfcnger}}

\hypertarget{a-jesus-tritt-sein-lehramt-in-kapernaum-an}{%
\paragraph{a) Jesus tritt sein Lehramt in Kapernaum
an}\label{a-jesus-tritt-sein-lehramt-in-kapernaum-an}}

\bibleverse{12} Als Jesus aber von der Gefangennahme des Johannes Kunde
erhielt, zog er sich nach Galiläa zurück; \bibleverse{13} er verließ
jedoch Nazareth und verlegte seinen Wohnsitz nach Kapernaum, das am See
(Genezaret) liegt im Gebiet\textless sup title=``d.h.
Grenzgebiet''\textgreater✲ von Sebulon und Naphthali, \bibleverse{14}
damit das Wort des Propheten Jesaja erfüllt werde\textless sup
title=``Jes 8,23; 9,1-2''\textgreater✲, das da lautet: \bibleverse{15}
»Das Land Sebulon und das Land Naphthali, das nach dem See zu liegt, das
Land jenseits des Jordans, das Galiläa der Heiden, \bibleverse{16} das
Volk, das im Finstern saß, hat ein großes Licht gesehen, und denen, die
im Lande und Schatten des Todes saßen, ist ein Licht aufgegangen.«
\bibleverse{17} Von dieser Zeit an begann Jesus die Heilsbotschaft mit
den Worten zu verkündigen: »Tut Buße\textless sup title=``vgl.
3,2''\textgreater✲, denn das Himmelreich ist nahe herbeigekommen!«

\hypertarget{b-jesus-beruft-die-beiden-ersten-juxfcngerpaare}{%
\paragraph{b) Jesus beruft die beiden ersten
Jüngerpaare}\label{b-jesus-beruft-die-beiden-ersten-juxfcngerpaare}}

\bibleverse{18} Als er nun (eines Tages) am Ufer des Galiläischen Sees
hinging, sah er zwei Brüder, Simon, der auch den Namen
Petrus\textless sup title=``d.h. Fels, Felsenmann''\textgreater✲ führt,
und seinen Bruder Andreas, die ein Netz in den See auswarfen; sie waren
nämlich Fischer. \bibleverse{19} Er sagte zu ihnen: »Kommt, folgt mir
nach, so will ich euch zu Menschenfischern machen!« \bibleverse{20} Da
ließen sie sogleich ihre Netze liegen und folgten ihm nach.
\bibleverse{21} Als er dann von dort weiterging, sah er ein anderes
Brüderpaar, nämlich Jakobus, den Sohn des Zebedäus, und seinen Bruder
Johannes, die im Boot mit ihrem Vater Zebedäus ihre Netze instand
setzten; und er berief auch sie. \bibleverse{22} Da verließen sie
sogleich das Boot und ihren Vater und folgten ihm nach.

\hypertarget{c-schilderung-der-lehr--und-heilwirksamkeit-jesu-und-ihres-erfolgs}{%
\paragraph{c) Schilderung der Lehr- und Heilwirksamkeit Jesu und ihres
Erfolgs}\label{c-schilderung-der-lehr--und-heilwirksamkeit-jesu-und-ihres-erfolgs}}

\bibleverse{23} Jesus zog dann in ganz Galiläa umher, indem er in
ihren\textless sup title=``=~den dortigen''\textgreater✲ Synagogen
lehrte, die Heilsbotschaft vom Reiche (Gottes) verkündigte und alle
Krankheiten und alle Gebrechen im Volke heilte; \bibleverse{24} und der
Ruf von ihm verbreitete sich durch ganz Syrien, und man brachte alle,
die an den verschiedenartigsten Krankheiten litten und mit schmerzhaften
Übeln behaftet waren, Besessene, Fallsüchtige und Gelähmte, und er
heilte sie. \bibleverse{25} So begleiteten ihn denn große Volksscharen
aus Galiläa und aus dem Gebiet der Zehn-Städte sowie aus Jerusalem und
Judäa und aus dem Ostjordanland.

\hypertarget{die-bergpredigt-jesus-leitet-seine-juxfcnger-zu-einer-gerechtigkeit-und-einem-glaubenswege-an-dadurch-sie-als-salz-der-erde-und-licht-der-welt-wirken-sollen-kap.-5-7}{%
\subsubsection{2. Die Bergpredigt: Jesus leitet seine Jünger zu einer
Gerechtigkeit und einem Glaubenswege an, dadurch sie als Salz der Erde
und Licht der Welt wirken sollen (Kap.
5-7)}\label{die-bergpredigt-jesus-leitet-seine-juxfcnger-zu-einer-gerechtigkeit-und-einem-glaubenswege-an-dadurch-sie-als-salz-der-erde-und-licht-der-welt-wirken-sollen-kap.-5-7}}

\hypertarget{a-einleitung}{%
\paragraph{a) Einleitung}\label{a-einleitung}}

\hypertarget{section-4}{%
\section{5}\label{section-4}}

\bibleverse{1} Als Jesus nun die Volksscharen sah, ging er ins
Gebirge\textless sup title=``oder: auf den Berg''\textgreater✲ hinauf,
und nachdem er sich dort gesetzt hatte, traten seine Jünger\textless sup
title=``d.h. Schüler, Zuhörer''\textgreater✲ zu ihm. \bibleverse{2} Da
tat er seinen Mund auf und lehrte sie mit den Worten:

\hypertarget{b-beschreibung-der-geistlichen-entwicklung-der-juxfcnger-bis-zum-vollen-besitz-der-gerechtigkeit-acht-seligpreisungen}{%
\paragraph{b) Beschreibung der geistlichen Entwicklung der Jünger bis
zum vollen Besitz der Gerechtigkeit (acht
Seligpreisungen)}\label{b-beschreibung-der-geistlichen-entwicklung-der-juxfcnger-bis-zum-vollen-besitz-der-gerechtigkeit-acht-seligpreisungen}}

\bibleverse{3} »Selig sind die geistlich Armen, denn ihnen wird das
Himmelreich zuteil! \bibleverse{4} Selig sind die Bekümmerten, denn sie
werden getröstet werden!~-- \bibleverse{5} Selig sind die Sanftmütigen,
denn sie werden das Land ererben\textless sup title=``oder: die Erde
besitzen''\textgreater✲! \bibleverse{6} Selig sind, die nach der
Gerechtigkeit hungern und dürsten, denn sie werden gesättigt werden!~--
\bibleverse{7} Selig sind die Barmherzigen, denn sie werden
Barmherzigkeit erlangen! \bibleverse{8} Selig sind, die reinen Herzens
sind, denn sie werden Gott schauen! \bibleverse{9} Selig sind die
Friedfertigen, denn sie werden Söhne Gottes\textless sup title=``vgl.
5,45''\textgreater✲ heißen!~-- \bibleverse{10} Selig sind, die um der
Gerechtigkeit willen Verfolgung erleiden, denn ihnen wird das
Himmelreich zuteil! \bibleverse{11} Selig seid ihr, wenn man euch um
meinetwillen schmäht und verfolgt und euch lügnerisch alles Böse
nachredet! \bibleverse{12} Freuet euch darüber und jubelt, denn euer
Lohn ist groß im Himmel! Ebenso hat man ja auch die Propheten vor euch
verfolgt.«

\hypertarget{c-der-grundgedanke-dieser-ganzen-unterweisung-die-juxfcnger-sollen-das-salz-der-erde-und-das-licht-der-welt-sein}{%
\paragraph{c) Der Grundgedanke dieser ganzen Unterweisung: Die Jünger
sollen das Salz der Erde und das Licht der Welt
sein}\label{c-der-grundgedanke-dieser-ganzen-unterweisung-die-juxfcnger-sollen-das-salz-der-erde-und-das-licht-der-welt-sein}}

\bibleverse{13} »Ihr seid das Salz der Erde\textless sup title=``=~für
die Erde''\textgreater✲! Wenn aber das Salz fade✲ geworden ist, womit
soll es wieder gesalzen werden\textless sup title=``d.h. seine Salzkraft
zurückerhalten''\textgreater✲? Es taugt zu nichts mehr, als aus dem
Hause geworfen und von den Leuten zertreten zu werden\textless sup
title=``Mk 9,50; Lk 14,34-35''\textgreater✲.~-- \bibleverse{14} Ihr seid
das Licht der Welt! Eine Stadt, die oben auf einem Berge liegt, kann
nicht verborgen bleiben. \bibleverse{15} Man zündet auch nicht ein Licht
an und stellt es unter den Scheffel, sondern auf den
Leuchter\textless sup title=``d.h. Lichtständer''\textgreater✲: dann
leuchtet es allen, die im Hause sind\textless sup title=``Mk 4,21; Lk
8,16; 11,33''\textgreater✲. \bibleverse{16} Ebenso soll auch euer Licht
vor den Menschen leuchten, damit sie eure guten Werke sehen und euren
Vater, der im Himmel ist, preisen.«

\hypertarget{d-die-hier-gelehrte-gerechtigkeit-bedeutet-im-vergleich-zu-den-forderungen-des-alten-bundes-vollkommenheit}{%
\paragraph{d) Die hier gelehrte Gerechtigkeit bedeutet im Vergleich zu
den Forderungen des Alten Bundes
Vollkommenheit}\label{d-die-hier-gelehrte-gerechtigkeit-bedeutet-im-vergleich-zu-den-forderungen-des-alten-bundes-vollkommenheit}}

\hypertarget{aa-um-erfuxfcllung-der-gebote-handelt-es-sich}{%
\subparagraph{aa) Um Erfüllung der Gebote handelt es
sich}\label{aa-um-erfuxfcllung-der-gebote-handelt-es-sich}}

\bibleverse{17} »Denkt nicht, daß ich gekommen sei, das Gesetz oder die
Propheten aufzulösen✲! Ich bin nicht gekommen aufzulösen, sondern zu
erfüllen\textless sup title=``d.h. zur Erfüllung zu
bringen''\textgreater✲. \bibleverse{18} Denn wahrlich ich sage euch: Bis
Himmel und Erde vergehen, wird vom Gesetz nicht ein einziges
Jota\textless sup title=``d.h. der kleinste Buchstabe''\textgreater✲ und
kein Strichlein vergehen\textless sup title=``=~aufgehoben
werden''\textgreater✲, bis alles in Erfüllung gegangen ist.
\bibleverse{19} Wer also ein einziges von diesen Geboten -- und wäre es
das geringste -- auflöst✲ und die Menschen demgemäß lehrt, der wird der
Geringste\textless sup title=``oder: Kleinste''\textgreater✲ im
Himmelreich heißen; wer sie aber tut und (die Menschen) so lehrt, der
wird groß im Himmelreich heißen. \bibleverse{20} Denn ich sage euch:
Wenn es mit eurer Gerechtigkeit nicht weit besser bestellt ist als bei
den Schriftgelehrten und Pharisäern, so werdet ihr nimmermehr ins
Himmelreich eingehen!«

\hypertarget{bb-das-wird-an-einigen-mosaischen-geboten-gezeigt}{%
\subparagraph{bb) Das wird an einigen mosaischen Geboten
gezeigt}\label{bb-das-wird-an-einigen-mosaischen-geboten-gezeigt}}

\bibleverse{21} »Ihr habt gehört, daß den Alten✲ geboten worden
ist\textless sup title=``2.Mose 20,13; 21,12''\textgreater✲: ›Du sollst
nicht töten‹, wer aber tötet, soll dem Gericht verfallen sein.
\bibleverse{22} Ich dagegen sage euch: Wer seinem Bruder auch nur zürnt,
der soll dem Gericht verfallen sein; und wer zu seinem Bruder ›Dummkopf‹
sagt, soll dem Hohen Rat verfallen sein; und wer ›du Narr‹✲ zu ihm sagt,
soll der Feuerhölle✲ verfallen sein. \bibleverse{23} Wenn du also deine
Opfergabe zum Altar bringst und dich dort erinnerst, daß dein Bruder
etwas gegen dich hat, \bibleverse{24} so laß deine Gabe dort vor dem
Altar und gehe zunächst hin und versöhne dich mit deinem Bruder; alsdann
geh hin und opfere deine Gabe! \bibleverse{25} Sei zum Vergleich mit
deinem Widersacher✲ ohne Säumen bereit, solange du mit ihm noch auf dem
Wege (zum Richter) bist, damit dein Widersacher dich nicht dem Richter
übergibt und der Richter dich dem Gerichtsdiener (überantwortet) und du
ins Gefängnis gesetzt wirst. \bibleverse{26} Wahrlich ich sage dir: Du
wirst von dort sicherlich nicht herauskommen, bis du den letzten Pfennig
bezahlt hast\textless sup title=``vgl. Lk 12,58-59''\textgreater✲.

\bibleverse{27} Ihr habt gehört, daß (den Alten) geboten worden
ist\textless sup title=``2.Mose 20,14''\textgreater✲: ›Du sollst nicht
ehebrechen!‹ \bibleverse{28} Ich dagegen sage euch: Wer eine Ehefrau
auch nur mit Begehrlichkeit anblickt, hat damit schon in seinem Herzen
Ehebruch an ihr begangen. \bibleverse{29} Wenn dich also dein rechtes
Auge ärgert\textless sup title=``oder: zum Bösen verführen
will''\textgreater✲, so reiß es aus und wirf es weg von dir; denn es ist
besser für dich, daß eines deiner Glieder (dir) verloren geht, als daß
dein ganzer Leib in die Hölle geworfen wird. \bibleverse{30} Und wenn
deine rechte Hand dich ärgert\textless sup title=``oder: zum Bösen
verführen will''\textgreater✲, so haue sie ab und wirf sie weg von dir;
denn es ist besser für dich, daß eines deiner Glieder (dir) verloren
geht, als daß dein ganzer Leib in die Hölle geworfen wird.~--
\bibleverse{31} Ferner ist (zu den Alten) gesagt worden\textless sup
title=``5.Mose 24,1''\textgreater✲: ›Wer seine Ehefrau
entläßt\textless sup title=``oder: sich von seiner Frau scheiden
will''\textgreater✲, der soll ihr einen Scheidebrief geben!‹
\bibleverse{32} Ich dagegen sage euch: Wer sich von seiner Frau scheidet
-- außer auf Grund von Unzucht --, der verschuldet es, daß dann Ehebruch
mit ihr verübt wird; und wer eine entlassene\textless sup title=``oder:
geschiedene''\textgreater✲ Frau heiratet, der begeht Ehebruch.

\bibleverse{33} Ihr habt weiter gehört, daß den Alten geboten worden
ist\textless sup title=``3.Mose 19,12; 4.Mose 30,3-4''\textgreater✲: ›Du
sollst nicht falsch schwören‹, ›sollst aber dem Herrn deine Eide
erfüllen!‹ \bibleverse{34} Ich dagegen sage euch: Ihr sollt überhaupt
nicht schwören, weder beim Himmel, denn er ist Gottes Thron,
\bibleverse{35} noch bei der Erde, denn sie ist der Schemel seiner Füße,
noch bei Jerusalem, denn es ist die Stadt des großen Königs\textless sup
title=``d.h. Gottes''\textgreater✲. \bibleverse{36} Auch bei deinem
Haupte sollst du nicht schwören, denn du vermagst kein einziges Haar
weiß oder schwarz zu machen. \bibleverse{37} Eure Rede sei vielmehr ›ja
ja -- nein nein‹; jeder weitere Zusatz ist vom Übel\textless sup
title=``oder: stammt vom Bösen''\textgreater✲.

\bibleverse{38} Ihr habt gehört, daß (den Alten) geboten worden
ist\textless sup title=``2.Mose 21,24; 3.Mose 24,19-20''\textgreater✲:
›Auge um✲ Auge und Zahn um✲ Zahn!‹ \bibleverse{39} Ich dagegen sage
euch: Ihr sollt dem Bösen\textless sup title=``=~der
Bosheit''\textgreater✲ keinen Widerstand leisten; sondern wer dich auf
die rechte Wange schlägt, dem halte auch die andere hin, \bibleverse{40}
und wer mit dir einen Rechtsstreit anfangen und dir den Rock nehmen✲
will, dem überlaß auch noch den Mantel, \bibleverse{41} und wer dich zu
einer Meile Weges nötigt✲, mit dem gehe zwei. \bibleverse{42} Wer dich
(um etwas) bittet, dem gib, und wer (Geld) von dir borgen will, den
weise nicht ab!

\bibleverse{43} Ihr habt gehört, daß (den Alten) geboten worden
ist\textless sup title=``3.Mose 19,18''\textgreater✲; ›Du sollst deinen
Nächsten lieben, und deinen Feind hassen!‹ \bibleverse{44} Ich dagegen
sage euch: Liebet eure Feinde und betet für eure Verfolger,
\bibleverse{45} damit ihr euch als Söhne\textless sup title=``bzw.
Kinder''\textgreater✲ eures himmlischen Vaters erweist. Denn er läßt
seine Sonne über Böse und Gute aufgehen und läßt regnen auf Gerechte und
Ungerechte. \bibleverse{46} Denn wenn ihr (nur) die liebt, die euch
lieben, welches Verdienst habt ihr da\textless sup title=``oder: welchen
Lohn habt ihr dafür zu erwarten''\textgreater✲? Tun das nicht auch die
Zöllner? \bibleverse{47} Und wenn ihr nur eure Freunde grüßt, was tut
ihr da Besonderes? Tun das nicht auch die Heiden? \bibleverse{48} Darum
sollt ihr vollkommen sein, wie euer himmlischer Vater vollkommen ist.«

\hypertarget{e-licht-und-salz-sein-heiuxdft-nicht-nach-dem-beifall-der-menschen-trachten}{%
\paragraph{e) Licht und Salz sein heißt nicht: nach dem Beifall der
Menschen
trachten}\label{e-licht-und-salz-sein-heiuxdft-nicht-nach-dem-beifall-der-menschen-trachten}}

\hypertarget{aa-gebt-acht-beim-almosengeben}{%
\subparagraph{aa) Gebt acht beim
Almosengeben}\label{aa-gebt-acht-beim-almosengeben}}

\hypertarget{section-5}{%
\section{6}\label{section-5}}

\bibleverse{1} »Gebt acht darauf, daß ihr eure
Gerechtigkeit\textless sup title=``=~Wohltätigkeit, das Spenden von
Almosen''\textgreater✲ nicht vor den Leuten ausübt, um von ihnen gesehen
zu werden: sonst habt ihr keinen Lohn (zu erwarten) bei eurem Vater im
Himmel! \bibleverse{2} Wenn du also Almosen spenden willst, so laß nicht
vor dir her posaunen, wie es die Heuchler\textless sup title=``oder:
Scheinheiligen''\textgreater✲ in den Synagogen und auf den Straßen tun,
um von den Leuten gerühmt zu werden. Wahrlich ich sage euch: Sie haben
ihren Lohn dahin\textless sup title=``=~damit schon
empfangen''\textgreater✲. \bibleverse{3} Nein, wenn du Almosen gibst, so
laß deine linke Hand nicht wissen, was deine rechte tut, \bibleverse{4}
damit deine Wohltätigkeit im Verborgenen geschehe\textless sup
title=``oder: bleibe''\textgreater✲; dein Vater aber, der auch ins
Verborgene hineinsieht, wird es dir alsdann vergelten.«

\hypertarget{bb-gebt-acht-beim-beten-das-vaterunser-mit-angeschlossener-mahnung}{%
\subparagraph{bb) Gebt acht beim Beten (das Vaterunser mit
angeschlossener
Mahnung)}\label{bb-gebt-acht-beim-beten-das-vaterunser-mit-angeschlossener-mahnung}}

\bibleverse{5} »Auch wenn ihr betet, sollt ihr es nicht wie die Heuchler
machen; denn sie stellen sich gern in den Synagogen und an den
Straßenecken auf und beten dort, um den Leuten in die Augen zu fallen;
wahrlich ich sage euch: Sie haben ihren Lohn dahin. \bibleverse{6} Du
aber, wenn du beten willst, so geh in deine Kammer, schließe deine Tür
zu und bete zu deinem Vater, der im Verborgenen ist; dein Vater aber,
der auch ins Verborgene hineinsieht, wird es dir alsdann vergelten.
\bibleverse{7} Und wenn ihr betet, sollt ihr nicht plappern wie die
Heiden; denn sie meinen, Erhörung zu finden, wenn sie viele Worte
machen. \bibleverse{8} Darum macht es nicht wie sie; euer Vater weiß ja,
was ihr bedürft, ehe ihr ihn bittet. \bibleverse{9} Darum sollt ihr so
beten:

›Unser Vater, der du bist im Himmel: Geheiligt werde dein Name!
\bibleverse{10} Dein Reich komme! Dein Wille geschehe wie im Himmel, so
auch auf der Erde! \bibleverse{11} Unser auskömmliches Brot gib uns
heute! \bibleverse{12} Und vergib uns unsere Schulden✲, wie auch wir sie
unsern Schuldnern vergeben haben! \bibleverse{13} Und führe uns nicht in
Versuchung, sondern erlöse uns von dem Bösen!‹ \bibleverse{14} Denn wenn
ihr den Menschen ihre Verfehlungen vergebt, so wird euer himmlischer
Vater sie auch euch vergeben; \bibleverse{15} wenn ihr sie aber den
Menschen nicht vergebt, so wird euer Vater euch eure Verfehlungen auch
nicht vergeben.«

\hypertarget{cc-gebt-acht-beim-fasten}{%
\subparagraph{cc) Gebt acht beim
Fasten}\label{cc-gebt-acht-beim-fasten}}

\bibleverse{16} »Weiter: Wenn ihr fastet, sollt ihr kein finsteres
Gesicht machen wie die Heuchler; denn sie geben sich ein trübseliges
Aussehen, um sich den Leuten mit ihrem Fasten zur Schau zu stellen.
Wahrlich ich sage euch: Sie haben ihren Lohn dahin. \bibleverse{17} Du
aber, wenn du fastest, salbe dir das Haupt und wasche dir das Gesicht,
\bibleverse{18} um dich nicht mit deinem Fasten den Leuten zu zeigen,
sondern deinem Vater, der im Verborgenen ist; dein Vater aber, der auch
ins Verborgene hineinsieht, wird es dir alsdann vergelten.«

\hypertarget{f-das-ziel-der-juxfcnger-jesu-bleibt-die-gerechtigkeit-vor-gott}{%
\paragraph{f) Das Ziel der Jünger Jesu bleibt die Gerechtigkeit vor
Gott}\label{f-das-ziel-der-juxfcnger-jesu-bleibt-die-gerechtigkeit-vor-gott}}

\hypertarget{aa-diese-gerechtigkeit-bildet-den-unverguxe4nglichen-reichtum-der-juxfcnger}{%
\subparagraph{aa) Diese Gerechtigkeit bildet den unvergänglichen
Reichtum der
Jünger}\label{aa-diese-gerechtigkeit-bildet-den-unverguxe4nglichen-reichtum-der-juxfcnger}}

\bibleverse{19} »Sammelt euch nicht Schätze hier auf der Erde, wo Motten
und Rost\textless sup title=``oder: Wurmfraß''\textgreater✲ sie
vernichten und wo Diebe einbrechen und stehlen! \bibleverse{20} Sammelt
euch vielmehr Schätze im Himmel, wo weder Motten noch Rost\textless sup
title=``oder: Wurmfraß''\textgreater✲ sie vernichten und wo keine Diebe
einbrechen und stehlen! \bibleverse{21} Denn wo dein Schatz ist, da wird
auch dein Herz sein.~-- \bibleverse{22} Die Leuchte des Leibes ist das
Auge. Wenn nun dein Auge richtig\textless sup title=``oder:
gesund''\textgreater✲ ist, so wird dein ganzer Leib voll Licht
sein\textless sup title=``oder: helles Licht haben''\textgreater✲;
\bibleverse{23} wenn aber dein Auge nichts taugt, so wird dein ganzer
Leib finster\textless sup title=``oder: in Dunkelheit''\textgreater✲
sein. Wenn also das in dir befindliche Licht Dunkelheit ist, wie groß
muß dann die Dunkelheit sein!~-- \bibleverse{24} Niemand kann
(gleichzeitig) zwei (sich widerstreitenden) Herren dienen; denn entweder
wird er den einen hassen und den andern lieben, oder er wird dem einen
ergeben sein und den andern mißachten: ihr könnt nicht (gleichzeitig)
Gott und dem Mammon dienen.«

\hypertarget{bb-das-trachten-nach-dieser-gerechtigkeit-uxfcberhebt-die-juxfcnger-jesu-der-irdischen-sorgen}{%
\subparagraph{bb) Das Trachten nach dieser Gerechtigkeit überhebt die
Jünger Jesu der irdischen
Sorgen}\label{bb-das-trachten-nach-dieser-gerechtigkeit-uxfcberhebt-die-juxfcnger-jesu-der-irdischen-sorgen}}

\bibleverse{25} »Deswegen sage ich euch: Macht euch keine Sorgen um euer
Leben, was ihr essen und was ihr trinken sollt, auch nicht um euren
Leib, was ihr anziehen sollt. Ist nicht das Leben wertvoller als die
Nahrung und der Leib wertvoller als die Kleidung? \bibleverse{26} Sehet
die Vögel des Himmels an: sie säen nicht und ernten nicht und sammeln
nichts in Scheuern, und euer himmlischer Vater ernährt sie doch. Seid
ihr denn nicht viel mehr wert als sie? \bibleverse{27} Wer von euch
vermöchte aber mit all seinem Sorgen der Länge seiner Lebenszeit auch
nur eine einzige Spanne zuzusetzen? \bibleverse{28} Und was macht ihr
euch Sorge um die Kleidung? Betrachtet die Lilien auf dem Felde, wie sie
wachsen! Sie arbeiten nicht und spinnen nicht; \bibleverse{29} und doch
sage ich euch: Auch Salomo in aller seiner Pracht ist nicht so herrlich
gekleidet gewesen wie eine von ihnen. \bibleverse{30} Wenn nun Gott
schon das Gras des Feldes, das heute steht und morgen in den Ofen
geworfen wird, so kleidet: wird er das nicht viel mehr euch tun, ihr
Kleingläubigen? \bibleverse{31} Darum sollt ihr nicht sorgen und sagen:
›Was sollen wir essen, was trinken, womit sollen wir uns kleiden?‹
\bibleverse{32} Denn auf alles derartige sind die Heiden bedacht. Euer
himmlischer Vater weiß ja, daß ihr dies alles bedürft. \bibleverse{33}
Nein, trachtet zuerst nach dem Reiche Gottes und nach seiner
Gerechtigkeit, dann wird euch all das andere obendrein gegeben werden.
\bibleverse{34} Macht euch also keine Sorgen um den morgenden Tag! Denn
der morgende Tag wird seine eigenen Sorgen haben; jeder Tag hat an
seiner eigenen Mühsal genug.«

\hypertarget{g-wie-das-trachten-nach-der-gerechtigkeit-vor-gott-auf-das-verhalten-der-juxfcnger-nach-allen-seiten-zuruxfcckwirkt}{%
\paragraph{g) Wie das Trachten nach der Gerechtigkeit vor Gott auf das
Verhalten der Jünger nach allen Seiten
zurückwirkt}\label{g-wie-das-trachten-nach-der-gerechtigkeit-vor-gott-auf-das-verhalten-der-juxfcnger-nach-allen-seiten-zuruxfcckwirkt}}

\hypertarget{aa-es-macht-sie-zuruxfcckhaltend-den-fehlern-der-bruxfcder-gegenuxfcber-und-vorsichtig-in-ihren-darbietungen-an-die-feinde-ihres-kostbaren-schatzes}{%
\subparagraph{aa) Es macht sie zurückhaltend den Fehlern der Brüder
gegenüber und vorsichtig in ihren Darbietungen an die Feinde ihres
kostbaren
Schatzes}\label{aa-es-macht-sie-zuruxfcckhaltend-den-fehlern-der-bruxfcder-gegenuxfcber-und-vorsichtig-in-ihren-darbietungen-an-die-feinde-ihres-kostbaren-schatzes}}

\hypertarget{section-6}{%
\section{7}\label{section-6}}

\bibleverse{1} »Richtet nicht, damit ihr nicht gerichtet werdet!
\bibleverse{2} Denn mit demselben Gericht\textless sup title=``oder:
Urteil''\textgreater✲, mit dem ihr richtet, werdet ihr wieder gerichtet
werden, und mit demselben Maße, mit dem ihr meßt, wird euch wieder
gemessen werden\textless sup title=``Mk 4,24''\textgreater✲.
\bibleverse{3} Was siehst du aber den Splitter im Auge deines Bruders,
während du den Balken in deinem eigenen Auge nicht wahrnimmst?
\bibleverse{4} Oder wie darfst du zu deinem Bruder sagen: ›Laß mich den
Splitter aus deinem Auge ziehen‹? Und dabei steckt der Balken in deinem
Auge! \bibleverse{5} Du Heuchler, ziehe zuerst den Balken aus deinem
Auge, dann magst du zusehen, wie du den Splitter aus deines Bruders Auge
ziehst.~-- \bibleverse{6} Gebt das Heilige nicht den Hunden preis und
werft eure Perlen\textless sup title=``vgl. 13,45-46''\textgreater✲
nicht den Schweinen vor, damit diese sie nicht mit ihren Füßen zertreten
und sich umwenden und euch zerreißen.«

\hypertarget{bb-es-macht-sie-fleiuxdfig-im-gebet-zu-gott}{%
\subparagraph{bb) Es macht sie fleißig im Gebet zu
Gott}\label{bb-es-macht-sie-fleiuxdfig-im-gebet-zu-gott}}

\bibleverse{7} »Bittet, so wird euch gegeben werden; suchet, so werdet
ihr finden; klopfet an, so wird euch aufgetan werden! \bibleverse{8}
Denn wer da bittet, der empfängt, und wer da sucht, der findet, und wer
anklopft, dem wird aufgetan werden. \bibleverse{9} Oder wo wäre jemand
unter euch, der seinem Sohne, wenn er ihn um Brot bittet, einen Stein
reichte? \bibleverse{10} Oder der, wenn er ihn um einen Fisch bittet,
ihm eine Schlange gäbe? \bibleverse{11} Wenn nun ihr, die ihr doch böse
seid, euren Kindern gute Gaben zu geben versteht; wieviel mehr wird euer
Vater im Himmel denen Gutes geben, die ihn bitten!«

\hypertarget{h-schluuxdf-der-bergpredigt}{%
\paragraph{h) Schluß der
Bergpredigt}\label{h-schluuxdf-der-bergpredigt}}

\hypertarget{aa-die-goldene-regel-fuxfcr-die-uxfcbung-der-nuxe4chstenliebe.-der-schmale-und-der-breite-weg}{%
\subparagraph{aa) Die »goldene Regel« für die Übung der Nächstenliebe.
Der schmale und der breite
Weg}\label{aa-die-goldene-regel-fuxfcr-die-uxfcbung-der-nuxe4chstenliebe.-der-schmale-und-der-breite-weg}}

\bibleverse{12} »Alles nun, was ihr von den Menschen\textless sup
title=``=~von anderen''\textgreater✲ erwartet, das erweist auch ihr
ihnen ebenso; denn darin besteht (die Erfüllung) des Gesetzes und der
Propheten.~-- \bibleverse{13} Gehet (in das Reich Gottes) durch die enge
Pforte ein; denn weit ist die Pforte und breit der Weg, der ins
Verderben führt, und es sind ihrer viele, die auf ihm hineingehen.
\bibleverse{14} Eng ist dagegen die Pforte und schmal der Weg, der ins
Leben führt, und nur wenige sind es, die ihn finden.«\textless sup
title=``Lk 13,24''\textgreater✲

\hypertarget{bb-warnung-vor-den-scheinpropheten-d.h.-falschen-lehrern-die-an-den-fruxfcchten-ihres-lebens-erkannt-werden}{%
\subparagraph{bb) Warnung vor den Scheinpropheten (d.h. falschen
Lehrern), die an den Früchten ihres Lebens erkannt
werden}\label{bb-warnung-vor-den-scheinpropheten-d.h.-falschen-lehrern-die-an-den-fruxfcchten-ihres-lebens-erkannt-werden}}

\bibleverse{15} »Hütet euch vor den falschen Propheten, die in
Schafskleidern zu euch kommen, im Inneren aber räuberische Wölfe sind.
\bibleverse{16} An ihren Früchten werdet ihr sie erkennen. Kann man etwa
Trauben lesen von Dornbüschen oder Feigen von Disteln? \bibleverse{17}
So bringt jeder gute✲ Baum gute Früchte, ein fauler\textless sup
title=``=~kernfauler, mit verdorbenen Säften''\textgreater✲ Baum aber
bringt schlechte Früchte; \bibleverse{18} ein guter Baum kann keine
schlechten Früchte bringen, und ein fauler Baum kann keine guten Früchte
bringen. \bibleverse{19} Jeder Baum, der nicht gute Früchte bringt, wird
abgehauen und ins Feuer geworfen. \bibleverse{20} Also: an ihren
Früchten werdet ihr sie erkennen.«

\hypertarget{cc-nicht-die-nur-mit-dem-mund-sich-zum-heiland-bekennenden-bestehen-im-endgericht-die-tuxe4ter-des-wortes-haben-auf-felsgrund-gebaut}{%
\subparagraph{cc) Nicht die nur mit dem Mund sich zum Heiland
Bekennenden bestehen im Endgericht -- die Täter des Wortes haben auf
Felsgrund
gebaut}\label{cc-nicht-die-nur-mit-dem-mund-sich-zum-heiland-bekennenden-bestehen-im-endgericht-die-tuxe4ter-des-wortes-haben-auf-felsgrund-gebaut}}

\bibleverse{21} »Nicht alle, die ›Herr, Herr‹ zu mir sagen, werden
(darum schon) ins Himmelreich eingehen, sondern nur, wer den Willen
meines himmlischen Vaters tut. \bibleverse{22} Viele werden an jenem
Tage\textless sup title=``d.h. am Tage des Gerichts''\textgreater✲ zu
mir sagen: ›Herr, Herr, haben wir nicht kraft deines Namens prophetisch
geredet und kraft deines Namens böse Geister ausgetrieben und kraft
deines Namens viele Wundertaten vollführt?‹ \bibleverse{23} Aber dann
werde ich ihnen erklären: ›Niemals habe ich euch gekannt; hinweg von
mir, ihr Täter der Gesetzlosigkeit!‹\textless sup title=``Ps
6,9''\textgreater✲

\bibleverse{24} Darum wird jeder, der diese meine Worte hört und nach
ihnen tut, einem klugen Manne gleichen, der sein Haus auf Felsengrund
gebaut hat. \bibleverse{25} Da strömte der Platzregen herab, es kamen
die Wasserströme, es wehten die Winde und stießen an✲ jenes Haus; doch
es stürzte nicht ein, denn es war auf den Felsen gegründet.
\bibleverse{26} Wer jedoch diese meine Worte hört und nicht nach ihnen
tut, der gleicht einem törichten Manne, der sein Haus auf den Sand
gebaut hat. \bibleverse{27} Da strömte der Platzregen herab, es kamen
die Wasserströme, es wehten die Winde und stürmten gegen jenes Haus: da
stürzte es ein, und sein Zusammensturz✲ war gewaltig.«

\hypertarget{i-die-wirkung-dieser-unterweisung-auf-das-volk-d.h.-auf-die-gesamte-zuhuxf6rerschaft}{%
\paragraph{i) Die Wirkung dieser Unterweisung auf das Volk (d.h. auf die
gesamte
Zuhörerschaft)}\label{i-die-wirkung-dieser-unterweisung-auf-das-volk-d.h.-auf-die-gesamte-zuhuxf6rerschaft}}

\bibleverse{28} Als Jesus diese Rede beendet hatte, waren die
Volksscharen über seine Lehre ganz betroffen; \bibleverse{29} denn er
lehrte sie wie einer, der (göttliche) Vollmacht hat, ganz anders als
ihre Schriftgelehrten.

\hypertarget{wunder--und-heiltuxe4tigkeit-jesu-in-kapernaum-und-auf-seinen-wanderungen}{%
\subsubsection{3. Wunder- und Heiltätigkeit Jesu in Kapernaum und auf
seinen
Wanderungen}\label{wunder--und-heiltuxe4tigkeit-jesu-in-kapernaum-und-auf-seinen-wanderungen}}

\hypertarget{a-heilung-eines-aussuxe4tzigen}{%
\paragraph{a) Heilung eines
Aussätzigen}\label{a-heilung-eines-aussuxe4tzigen}}

\hypertarget{section-7}{%
\section{8}\label{section-7}}

\bibleverse{1} Als er dann vom Berge herabgestiegen war, folgten ihm
große Volksscharen nach. \bibleverse{2} Da trat ein Aussätziger herzu,
warf sich vor ihm nieder und sagte: »Herr, wenn du willst, kannst du
mich reinigen.« \bibleverse{3} Jesus streckte seine Hand aus, faßte ihn
an und sagte: »Ich will's, werde rein!« Da wurde er sogleich von seinem
Aussatz rein. \bibleverse{4} Darauf sagte Jesus zu ihm: »Hüte dich,
jemandem etwas davon zu sagen! Gehe vielmehr hin, zeige dich dem
Priester und bringe die Opfergabe dar, die Mose\textless sup
title=``3.Mose 13,49; 14,2-32''\textgreater✲ geboten hat, zum Zeugnis✲
für sie!«

\hypertarget{b-heilung-des-dieners-des-heidnischen-hauptmanns-von-kapernaum}{%
\paragraph{b) Heilung des Dieners des (heidnischen) Hauptmanns von
Kapernaum}\label{b-heilung-des-dieners-des-heidnischen-hauptmanns-von-kapernaum}}

\bibleverse{5} Als er hierauf nach Kapernaum hineinkam, trat ein
Hauptmann zu ihm und bat ihn \bibleverse{6} mit den Worten: »Herr, mein
Diener✲ liegt gelähmt bei mir zu Hause darnieder und leidet schreckliche
Schmerzen.« \bibleverse{7} Jesus antwortete ihm: »Ich will kommen und
ihn heilen.« \bibleverse{8} Der Hauptmann aber entgegnete: »Herr, ich
bin nicht wert, daß du unter mein Dach trittst; nein, gebiete nur mit
einem Wort, dann wird mein Diener gesund werden. \bibleverse{9} Ich bin
ja auch ein Mann, der unter höherem Befehl steht, und habe Mannschaften
unter mir, und wenn ich zu dem einen sage: ›Gehe!‹, so geht er, und zu
dem andern: ›Komm!‹, so kommt er, und zu meinem Diener: ›Tu das!‹, so
tut er's.« \bibleverse{10} Als Jesus das hörte, verwunderte er sich und
sagte zu seinen Begleitern: »Wahrlich ich sage euch: In Israel habe ich
bei niemand solchen Glauben gefunden. \bibleverse{11} Ich sage euch
aber: Viele werden von Osten und Westen kommen und sich mit Abraham,
Isaak und Jakob im Himmelreich zum Mahl niederlassen\textless sup
title=``Lk 13,28-29''\textgreater✲; \bibleverse{12} aber die Söhne des
Reiches werden in die Finsternis draußen hinausgestoßen werden; dort
wird lautes Weinen und Zähneknirschen sein.« \bibleverse{13} Zu dem
Hauptmann aber sagte Jesus: »Geh hin✲! Wie du geglaubt hast, so geschehe
dir!« Und sein Diener wurde zur selben Stunde gesund.

\hypertarget{c-heilung-der-schwiegermutter-des-petrus-und-vieler-anderer-kranken-zu-kapernaum}{%
\paragraph{c) Heilung der Schwiegermutter des Petrus und vieler anderer
Kranken zu
Kapernaum}\label{c-heilung-der-schwiegermutter-des-petrus-und-vieler-anderer-kranken-zu-kapernaum}}

\bibleverse{14} Als Jesus dann in das Haus des Petrus gekommen war, sah
er dessen Schwiegermutter fieberkrank zu Bett liegen. \bibleverse{15} Er
faßte sie bei der Hand, da wich das Fieber von ihr: sie stand auf und
bediente ihn (bei der Mahlzeit).

\bibleverse{16} Als es dann Abend geworden war, brachte man viele
Besessene zu ihm, und er trieb die bösen Geister durchs Wort aus und
heilte alle, die ein Leiden hatten. \bibleverse{17} So sollte sich das
Wort des Propheten Jesaja erfüllen, der da sagt\textless sup title=``Jes
53,4''\textgreater✲: »Er hat unsere Gebrechen hinweggenommen und unsere
Krankheiten getragen\textless sup title=``oder: sich
aufgeladen''\textgreater✲.«

\hypertarget{d-jesus-entweicht-an-das-jenseitige-ufer-des-sees-spruxfcche-uxfcber-die-nachfolge-jesu}{%
\paragraph{d) Jesus entweicht an das jenseitige Ufer des Sees; Sprüche
über die Nachfolge
Jesu}\label{d-jesus-entweicht-an-das-jenseitige-ufer-des-sees-spruxfcche-uxfcber-die-nachfolge-jesu}}

\bibleverse{18} Als Jesus sich dann wieder von großen Volksscharen
umgeben sah, befahl er, an das jenseitige Ufer des Sees hinüberzufahren.
\bibleverse{19} Da trat ein Schriftgelehrter an ihn heran mit den
Worten: »Meister, ich will dir folgen, wohin du auch gehst!«
\bibleverse{20} Jesus antwortete ihm: »Die Füchse haben Gruben und die
Vögel des Himmels Nester; der Menschensohn aber hat keine Stätte, wo er
sein Haupt hinlegen kann.«~-- \bibleverse{21} Ein anderer von seinen
Jüngern\textless sup title=``=~Schülern, Anhängern''\textgreater✲ sagte
zu ihm: »Herr, erlaube mir, zuerst noch hinzugehen und meinen Vater zu
begraben!« \bibleverse{22} Jesus aber antwortete ihm: »Folge du mir
nach, und überlaß es den Toten\textless sup title=``d.h. den geistlich
Toten''\textgreater✲, ihre Toten zu begraben!«

\hypertarget{e-jesus-beschwichtigt-den-seesturm}{%
\paragraph{e) Jesus beschwichtigt den
Seesturm}\label{e-jesus-beschwichtigt-den-seesturm}}

\bibleverse{23} Jesus stieg dann ins Boot, und seine Jünger folgten ihm.
\bibleverse{24} Da erhob sich (plötzlich) ein heftiger Sturm auf dem
See, so daß das Boot von den Wellen bedeckt✲ wurde; er selbst aber
schlief. \bibleverse{25} Da traten sie an ihn heran und weckten ihn mit
den Worten: »Herr, hilf uns: wir gehen unter!« \bibleverse{26} Er aber
antwortete ihnen: »Was seid ihr so furchtsam, ihr Kleingläubigen!« Dann
stand er auf und bedrohte die Winde und den See; da trat völlige
Windstille ein. \bibleverse{27} Die Leute aber verwunderten sich und
sagten: »Was ist das für ein Mann, daß sogar die Winde und der See ihm
gehorsam sind!«

\hypertarget{f-heilung-zweier-besessener-im-land-der-gadarener}{%
\paragraph{f) Heilung zweier Besessener im Land der
Gadarener}\label{f-heilung-zweier-besessener-im-land-der-gadarener}}

\bibleverse{28} Als er hierauf an das jenseitige Ufer in das Gebiet der
Gadarener gekommen war, traten ihm zwei von bösen Geistern besessene
Männer entgegen, die aus den Gräbern\textless sup title=``vgl. Lk
8,27''\textgreater✲ hervorkamen und so gemeingefährliche Menschen waren,
daß niemand auf der Straße dort an ihnen vorbeigehen konnte.
\bibleverse{29} Kaum hatten sie ihn erblickt, da schrien sie laut: »Was
hast du mit uns vor, du Sohn Gottes? Bist du hergekommen, um uns vor der
Zeit zu quälen?« \bibleverse{30} Es befand sich aber in weiter
Entfernung von ihnen eine große Herde Schweine auf der Weide.
\bibleverse{31} Da baten ihn die bösen Geister: »Wenn du uns austreiben
willst, so laß uns doch in die Schweineherde fahren!« \bibleverse{32} Er
antwortete ihnen: »Hinweg mit euch!« Da fuhren sie aus und fuhren in die
Schweine hinein, und die ganze Herde stürmte infolgedessen den Abhang
hinab in den See und ertrank in den Fluten. \bibleverse{33} Die Hirten
aber ergriffen die Flucht und berichteten nach ihrer Ankunft in der
Stadt den ganzen Vorfall, auch das, was mit den beiden Besessenen
vorgegangen war. \bibleverse{34} Da zog die Einwohnerschaft der ganzen
Stadt hinaus, Jesus entgegen, und als sie bei ihm eingetroffen waren,
baten sie ihn, er möchte ihr Gebiet verlassen.

\hypertarget{g-heilung-eines-geluxe4hmten-in-kapernaum-jesus-vergibt-suxfcnden}{%
\paragraph{g) Heilung eines Gelähmten in Kapernaum; Jesus vergibt
Sünden}\label{g-heilung-eines-geluxe4hmten-in-kapernaum-jesus-vergibt-suxfcnden}}

\hypertarget{section-8}{%
\section{9}\label{section-8}}

\bibleverse{1} Er stieg nun in ein Boot, fuhr über den See zurück und
kam wieder in seine Stadt (Kapernaum). \bibleverse{2} Dort brachte man
ihm einen Gelähmten, der auf einem Tragbett lag. Weil nun Jesus ihren
Glauben sah, sagte er zu dem Gelähmten: »Sei getrost, mein Sohn: deine
Sünden sind (dir) vergeben!« \bibleverse{3} Da dachten einige von den
Schriftgelehrten bei sich: »Dieser lästert Gott!« \bibleverse{4} Weil
nun Jesus ihre Gedanken durchschaute, sagte er: »Warum denkt ihr Böses
in euren Herzen? \bibleverse{5} Was ist denn leichter, zu sagen: ›Deine
Sünden sind (dir) vergeben‹ oder zu sagen: ›Stehe auf und gehe umher!‹?
\bibleverse{6} Damit ihr aber wißt, daß der Menschensohn die Vollmacht
besitzt, Sünden auf der Erde zu vergeben« -- hierauf sagte er zu dem
Gelähmten: »Stehe auf, nimm dein Bett und gehe heim in dein Haus!«
\bibleverse{7} Da stand er auf und ging heim in sein Haus.
\bibleverse{8} Als die Volksmenge das sah, gerieten sie in Furcht und
priesen Gott, daß er den Menschen solche Macht gegeben habe.

\hypertarget{h-berufung-des-zuxf6llners-matthuxe4us-jesus-als-tischgenosse-der-zuxf6llner-und-suxfcnder}{%
\paragraph{h) Berufung des Zöllners Matthäus; Jesus als Tischgenosse der
Zöllner und
Sünder}\label{h-berufung-des-zuxf6llners-matthuxe4us-jesus-als-tischgenosse-der-zuxf6llner-und-suxfcnder}}

\bibleverse{9} Als Jesus dann von dort weiterging, sah er einen Mann
namens Matthäus✲ an der Zollstätte sitzen und sagte zu ihm: »Folge mir
nach!« Da stand er auf und folgte ihm. \bibleverse{10} Als Jesus dann im
Hause (des Matthäus) zu Tische saß, kamen viele Zöllner und Sünder und
nahmen mit Jesus und seinen Jüngern am Mahle teil. \bibleverse{11} Als
die Pharisäer das sahen, sagten sie zu seinen Jüngern: »Warum ißt euer
Meister mit den Zöllnern und Sündern?« \bibleverse{12} Als Jesus es
hörte, sagte er: »Die Gesunden haben keinen Arzt nötig, wohl aber die
Kranken. \bibleverse{13} Geht aber hin und lernt das Wort
verstehen\textless sup title=``Hos 6,6''\textgreater✲: ›An
Barmherzigkeit habe ich Wohlgefallen, nicht an
Schlachtopfern‹\textless sup title=``vgl. 12,7''\textgreater✲; denn ich
bin nicht gekommen, um Gerechte zu berufen, sondern Sünder.«

\hypertarget{i-die-fastenfrage-der-johannesjuxfcnger}{%
\paragraph{i) Die Fastenfrage der
Johannesjünger}\label{i-die-fastenfrage-der-johannesjuxfcnger}}

\bibleverse{14} Damals traten die Jünger des Johannes an ihn heran mit
der Frage: »Warum fasten wir und die Pharisäer (zum Zeichen der
Frömmigkeit), während deine Jünger es nicht tun?« \bibleverse{15} Jesus
antwortete ihnen: »Können etwa die Hochzeitsgäste trauern, solange der
Bräutigam noch in ihrer Mitte weilt? Es werden aber Tage kommen, wo der
Bräutigam ihnen genommen ist: dann werden sie fasten. \bibleverse{16}
Niemand setzt aber ein Stück ungewalkten Tuches\textless sup
title=``=~neuen Stoff''\textgreater✲ auf ein altes Kleid; denn der
eingesetzte Fleck\textless sup title=``=~das Flickstück''\textgreater✲
reißt doch von dem Kleide wieder ab, und es entsteht ein noch
schlimmerer Riß. \bibleverse{17} Auch füllt man neuen✲ Wein nicht in
alte Schläuche; sonst werden die Schläuche gesprengt, und der Wein läuft
aus, und auch die Schläuche gehen verloren; nein, man füllt neuen Wein
in neue Schläuche: dann bleiben beide erhalten.«

\hypertarget{k-auferweckung-des-tuxf6chterleins-des-jairus-und-heilung-der-blutfluxfcssigen-frau}{%
\paragraph{k) Auferweckung des Töchterleins des Jairus und Heilung der
blutflüssigen
Frau}\label{k-auferweckung-des-tuxf6chterleins-des-jairus-und-heilung-der-blutfluxfcssigen-frau}}

\bibleverse{18} Während Jesus noch so zu ihnen redete, trat ein
Vorsteher (der Synagoge) herzu, warf sich vor ihm nieder und sagte:
»Meine Tochter ist soeben gestorben; aber komm und lege ihr deine Hand
auf, dann wird sie wieder zum Leben erwachen.« \bibleverse{19} Da stand
Jesus auf und folgte ihm samt seinen Jüngern.

\bibleverse{20} Und siehe, eine Frau, die seit zwölf Jahren am Blutfluß
litt, trat von hinten an ihn heran und faßte die Quaste\textless sup
title=``vgl. 4.Mose 15,38-41''\textgreater✲ seines Rockes\textless sup
title=``oder: Mantels''\textgreater✲ an; \bibleverse{21} sie dachte
nämlich bei sich: »Wenn ich nur seinen Rock\textless sup title=``oder:
Mantel''\textgreater✲ anfasse, so wird mir geholfen sein.«
\bibleverse{22} Jesus aber wandte sich um, und als er sie sah, sagte er:
»Sei getrost, meine Tochter, dein Glaube hat dir geholfen!« Und die Frau
war von dieser Stunde an gesund.

\bibleverse{23} Als Jesus dann in das Haus des Vorstehers kam und die
Flötenbläser und das Getümmel der Volksmenge sah, \bibleverse{24} sagte
er: »Entfernt euch! Das Mädchen ist nicht tot, sondern schläft nur.« Da
verlachten sie ihn. \bibleverse{25} Als man aber die Volksmenge aus dem
Hause entfernt hatte, ging er (zu der Toten) hinein und faßte sie bei
der Hand: da erwachte das Mädchen\textless sup title=``oder: es stand
auf''\textgreater✲. \bibleverse{26} Die Kunde hiervon verbreitete sich
in der ganzen dortigen Gegend.

\hypertarget{l-heilung-zweier-blinden-und-eines-stummen-besessenen-abschluuxdf}{%
\paragraph{l) Heilung zweier Blinden und eines stummen Besessenen;
Abschluß}\label{l-heilung-zweier-blinden-und-eines-stummen-besessenen-abschluuxdf}}

\bibleverse{27} Als Jesus hierauf von dort weiterging, folgten ihm zwei
Blinde, die laut riefen: »Sohn Davids, erbarme dich unser!«
\bibleverse{28} Als er dann in das Haus\textless sup title=``=~nach
Hause''\textgreater✲ gekommen war, traten die Blinden zu ihm heran, und
Jesus fragte sie: »Glaubt ihr, daß ich (euch) dies zu tun vermag?« Sie
antworteten ihm: »Ja, Herr!« \bibleverse{29} Da rührte er ihre Augen an
und sagte: »Nach eurem Glauben geschehe euch!« \bibleverse{30} Da taten
sich ihre Augen auf; Jesus aber gab ihnen die strenge Weisung: »Hütet
euch! Niemand darf etwas davon erfahren!« \bibleverse{31} Sobald sie
aber hinausgegangen waren, verbreiteten sie die Kunde von ihm in jener
ganzen Gegend.

\bibleverse{32} Während diese hinausgingen, brachte man schon wieder
einen stummen Besessenen zu ihm; \bibleverse{33} und als der böse Geist
ausgetrieben war, konnte der Stumme reden. Da geriet die Volksmenge in
Staunen und sagte: »Noch niemals hat man etwas Derartiges in Israel
gesehen!« \bibleverse{34} Die Pharisäer aber erklärten: »Im Bunde mit
dem Obersten\textless sup title=``oder: in der Kraft des
Beherrschers''\textgreater✲ der bösen Geister treibt er die Geister
aus.«

\bibleverse{35} So durchwanderte Jesus alle Städte und Dörfer, indem er
in ihren\textless sup title=``=~den dortigen''\textgreater✲ Synagogen
lehrte, die Heilsbotschaft vom Reiche (Gottes) verkündigte und alle
Krankheiten und alle Gebrechen heilte✲.

\hypertarget{aussendung-der-zwuxf6lf-juxfcnger}{%
\subsubsection{4. Aussendung der zwölf
Jünger}\label{aussendung-der-zwuxf6lf-juxfcnger}}

\hypertarget{a-einleitung-jesu-mitleid-beim-anblick-des-volkes-das-wort-von-der-ernte}{%
\paragraph{a) Einleitung: Jesu Mitleid beim Anblick des Volkes; das Wort
von der
Ernte}\label{a-einleitung-jesu-mitleid-beim-anblick-des-volkes-das-wort-von-der-ernte}}

\bibleverse{36} Beim Anblick der Volksscharen aber erfaßte ihn tiefes
Mitleid mit ihnen, denn sie waren abgehetzt und verwahrlost wie Schafe,
die keinen Hirten haben\textless sup title=``4.Mose 27,17; Hes
34,1-6''\textgreater✲. \bibleverse{37} Da sagte er zu seinen Jüngern:
»Die Ernte ist groß, aber die Zahl der Arbeiter ist klein;
\bibleverse{38} bittet daher den Herrn der Ernte, daß er Arbeiter auf
sein Erntefeld sende!«

\hypertarget{b-berufung-und-namen-der-zwuxf6lf-juxfcnger}{%
\paragraph{b) Berufung und Namen der zwölf
Jünger}\label{b-berufung-und-namen-der-zwuxf6lf-juxfcnger}}

\hypertarget{section-9}{%
\section{10}\label{section-9}}

\bibleverse{1} Er rief dann seine zwölf Jünger herbei und verlieh ihnen
Macht über die unreinen Geister, so daß sie diese auszutreiben und alle
Krankheiten und jedes Gebrechen zu heilen vermochten. \bibleverse{2} Die
Namen der zwölf Apostel✲ aber sind folgende: Zuerst Simon, der auch
Petrus✲ heißt, und sein Bruder Andreas; sodann Jakobus, der Sohn des
Zebedäus, und sein Bruder Johannes; \bibleverse{3} Philippus und
Bartholomäus; Thomas und der Zöllner Matthäus; Jakobus, der Sohn des
Alphäus, und Lebbäus (mit dem Beinamen Thaddäus); \bibleverse{4} Simon,
der Kananäer\textless sup title=``=~der Eiferer; Lk 6,15''\textgreater✲
und Judas, der Iskariote\textless sup title=``d.h. Mann aus
Kariot''\textgreater✲, derselbe, der ihn verraten hat.

\hypertarget{c-die-aussendungsrede-an-die-zwuxf6lf-juxfcnger}{%
\paragraph{c) Die Aussendungsrede an die zwölf
Jünger}\label{c-die-aussendungsrede-an-die-zwuxf6lf-juxfcnger}}

\hypertarget{aa-die-weisungen}{%
\subparagraph{aa) Die Weisungen}\label{aa-die-weisungen}}

\bibleverse{5} Diese Zwölf sandte Jesus aus, nachdem er ihnen folgende
Weisungen gegeben hatte: »Den Weg zu den Heidenvölkern schlagt nicht ein
und tretet auch in keine Samariterstadt ein, \bibleverse{6} geht
vielmehr (nur) zu den verlorenen Schafen des Hauses Israel.
\bibleverse{7} Auf eurer Wanderung predigt: ›Das Himmelreich ist nahe
herbeigekommen!‹ \bibleverse{8} Heilt Kranke, weckt Tote auf, macht
Aussätzige rein, treibt böse Geister aus: umsonst habt ihr's empfangen,
umsonst sollt ihr's auch weitergeben! \bibleverse{9} Sucht euch kein
Gold, kein Silber, kein Kupfergeld in eure Gürtel zu verschaffen,
\bibleverse{10} nehmt keinen Ranzen\textless sup title=``oder: keine
Reisetasche''\textgreater✲ mit auf den Weg, auch nicht zwei
Röcke\textless sup title=``oder: Unterkleider''\textgreater✲, keine
Schuhe und keinen Stock, denn der Arbeiter ist seines
Unterhalts\textless sup title=``=~der Ernährung''\textgreater✲ wert.
\bibleverse{11} Wo ihr in eine Stadt oder ein Dorf eintretet, da
erkundigt euch, wer dort würdig sei (euch zu beherbergen), und bei dem
bleibt, bis ihr weiterzieht. \bibleverse{12} Beim Eintritt in das Haus
entbietet ihm den Friedensgruß, \bibleverse{13} und wenn das Haus es
verdient, soll der Friede, den ihr ihm gewünscht habt, ihm auch zuteil
werden; ist es dessen aber nicht würdig, so soll euer ihm gewünschter
Friede zu euch zurückkehren. \bibleverse{14} Wo man euch nicht aufnimmt
und euren Worten kein Gehör schenkt, da geht aus dem betreffenden Hause
oder Orte hinaus und schüttelt den Staub von euren Füßen ab!
\bibleverse{15} Wahrlich ich sage euch: Dem Lande Sodom und Gomorrha
wird es am Tage des Gerichts erträglicher ergehen als einer solchen
Stadt!

\bibleverse{16} Bedenket wohl: ich sende euch wie Schafe mitten unter
Wölfe; darum seid klug wie die Schlangen und ohne Falsch wie die
Tauben!«

\hypertarget{bb-ankuxfcndigung-der-den-juxfcngern-bevorstehenden-leiden}{%
\subparagraph{bb) Ankündigung der den Jüngern bevorstehenden
Leiden}\label{bb-ankuxfcndigung-der-den-juxfcngern-bevorstehenden-leiden}}

\bibleverse{17} »Nehmt euch aber vor den Menschen in acht! Denn sie
werden euch vor die Gerichtshöfe stellen und in ihren Synagogen euch
geißeln; \bibleverse{18} auch vor Statthalter und Könige werdet ihr um
meinetwillen geführt werden, um Zeugnis vor ihnen und den Heidenvölkern
abzulegen. \bibleverse{19} Wenn man euch nun (den Gerichten)
überliefert, so macht euch keine Sorge darüber, wie oder was ihr reden
sollt; denn es wird euch in jener Stunde eingegeben werden, was ihr
reden sollt; \bibleverse{20} nicht ihr seid es ja, die dann reden,
sondern der Geist eures Vaters ist es, der in euch redet.
\bibleverse{21} Es wird aber ein Bruder den Bruder zum Tode überliefern
und ein Vater den Sohn, und Kinder werden gegen ihre Eltern auftreten
und sie zum Tode bringen\textless sup title=``Mi 7,6''\textgreater✲,
\bibleverse{22} und ihr werdet allen um meines Namens willen verhaßt
sein; wer aber ausharrt bis ans Ende, der wird errettet werden.
\bibleverse{23} Wenn man euch aber in der einen Stadt verfolgt, so
flieht in eine andere; denn wahrlich ich sage euch: Ihr werdet mit den
Städten Israels noch nicht zu Ende sein, bis der
Menschensohn\textless sup title=``Dan 7,13-14''\textgreater✲ kommt.~--
\bibleverse{24} Ein Jünger✲ steht nicht über seinem Meister✲ und ein
Knecht nicht über seinem Herrn; \bibleverse{25} ein Jünger muß zufrieden
sein, wenn es ihm ergeht wie seinem Meister, und ein Knecht, (wenn es
ihm ergeht) wie seinem Herrn. Haben sie den Hausherrn
Beelzebul\textless sup title=``=~Satan; vgl. 2.Kön 1,2''\textgreater✲
genannt, wieviel mehr werden sie das bei seinen Hausgenossen tun!«

\hypertarget{cc-ermunterung-zu-treuem-ausharren-und-trost-fuxfcr-die-zeiten-der-truxfcbsal}{%
\subparagraph{cc) Ermunterung zu treuem Ausharren und Trost für die
Zeiten der
Trübsal}\label{cc-ermunterung-zu-treuem-ausharren-und-trost-fuxfcr-die-zeiten-der-truxfcbsal}}

\bibleverse{26} »Fürchtet euch nicht vor ihnen! Denn nichts ist
verhüllt, das nicht enthüllt werden wird, und nichts verborgen, das
nicht bekannt werden wird. \bibleverse{27} Was ich euch im Dunkel✲ sage,
das sprecht im Licht✲ aus, und was ihr (von mir) ins Ohr geflüstert
hört, das ruft auf den Dächern aus! \bibleverse{28} Fürchtet euch dabei
nicht vor denen, die wohl den Leib töten, aber die Seele nicht zu töten
vermögen; fürchtet euch vielmehr vor dem, der die Macht hat, sowohl die
Seele als den Leib in der Hölle zu verderben!~-- \bibleverse{29} Kosten
nicht zwei Sperlinge beim Einkauf nur ein paar Pfennige? Und doch fällt
keiner von ihnen auf die Erde ohne den Willen eures Vaters.
\bibleverse{30} Bei euch aber sind auch die Haare auf dem Haupte alle
gezählt. \bibleverse{31} Darum fürchtet euch nicht! Ihr seid mehr wert
als viele Sperlinge.~-- \bibleverse{32} Jeder nun, der sich vor den
Menschen zu mir bekennt, zu dem werde auch ich mich vor meinem
himmlischen Vater bekennen; \bibleverse{33} wer mich aber vor den
Menschen verleugnet, den werde auch ich vor meinem himmlischen Vater
verleugnen.«

\hypertarget{uxfcber-den-zweck-der-aussendung-der-juxfcnger-friede-und-schwert-verlust-und-gewinn}{%
\paragraph{Über den Zweck der Aussendung der Jünger; Friede und Schwert,
Verlust und
Gewinn}\label{uxfcber-den-zweck-der-aussendung-der-juxfcnger-friede-und-schwert-verlust-und-gewinn}}

\bibleverse{34} »Denkt nicht, ich sei gekommen, um Frieden auf die Erde
zu bringen! Nein, ich bin nicht gekommen, um Frieden zu bringen, sondern
das Schwert✲. \bibleverse{35} Denn ich bin gekommen, ›um den Sohn mit
seinem Vater, die Tochter mit ihrer Mutter und die Schwiegertochter mit
ihrer Schwiegermutter zu entzweien, \bibleverse{36} und die eigenen
Hausgenossen werden einander feindselig gegenüberstehen‹\textless sup
title=``Mi 7,6''\textgreater✲. \bibleverse{37} Wer Vater oder Mutter
mehr liebt als mich, ist meiner nicht wert, und wer Sohn oder Tochter
mehr liebt als mich, ist meiner nicht wert; \bibleverse{38} und wer
nicht sein Kreuz auf sich nimmt und mir nachfolgt, ist meiner nicht
wert.~-- \bibleverse{39} Wer sein Leben findet, wird es verlieren, und
wer sein Leben um meinetwillen verliert, der wird es finden.«

\hypertarget{dd-schluuxdf-der-rede-besonders-verheiuxdfungen}{%
\subparagraph{dd) Schluß der Rede (besonders
Verheißungen)}\label{dd-schluuxdf-der-rede-besonders-verheiuxdfungen}}

\bibleverse{40} »Wer euch aufnimmt, nimmt mich auf, und wer mich
aufnimmt, nimmt den auf, der mich gesandt hat. \bibleverse{41} Wer einen
Propheten aufnimmt, eben weil er ein Prophet heißt, der wird dafür den
Lohn eines Propheten empfangen; und wer einen Gerechten aufnimmt, eben
weil er ein Gerechter heißt, der wird dafür den Lohn eines Gerechten
empfangen; \bibleverse{42} und wer einem von diesen geringen Leuten
seines Namens wegen, weil er ein Jünger heißt, auch nur einen Becher
frischen Wassers zu trinken gibt -- wahrlich ich sage euch: Es soll ihm
nicht unbelohnt bleiben!«

\hypertarget{section-10}{%
\section{11}\label{section-10}}

\bibleverse{1} Als Jesus nun mit der Unterweisung seiner zwölf Jünger zu
Ende gekommen war, zog er von dort weiter, um in ihren\textless sup
title=``=~den dortigen''\textgreater✲ Städten zu lehren und zu predigen.

\hypertarget{iv.-unglaube-und-feindschaft-der-juden-gegen-jesus-112-1358}{%
\subsection{IV. Unglaube und Feindschaft der Juden gegen Jesus
(11,2-13,58)}\label{iv.-unglaube-und-feindschaft-der-juden-gegen-jesus-112-1358}}

\hypertarget{gesandtschaft-johannes-des-tuxe4ufers-aus-dem-gefuxe4ngnis-jesu-antwort-und-sein-zeugnis-uxfcber-johannes}{%
\subsubsection{1. Gesandtschaft Johannes des Täufers aus dem Gefängnis;
Jesu Antwort und sein Zeugnis über
Johannes}\label{gesandtschaft-johannes-des-tuxe4ufers-aus-dem-gefuxe4ngnis-jesu-antwort-und-sein-zeugnis-uxfcber-johannes}}

\bibleverse{2} Als aber Johannes im Gefängnis von dem Wirken Christi
hörte, sandte er durch seine Jünger Botschaft an ihn \bibleverse{3} und
ließ ihn fragen: »Bist du es, der da kommen soll\textless sup
title=``d.h. der verheißene Messias''\textgreater✲, oder sollen wir auf
einen andern warten?« \bibleverse{4} Jesus gab ihnen zur Antwort: »Geht
hin und berichtet dem Johannes, was ihr hört und seht: \bibleverse{5}
Blinde werden sehend und Lahme gehen, Aussätzige werden rein und Taube
hören, Tote werden auferweckt, und Armen wird die Heilsbotschaft
verkündigt\textless sup title=``Jes 35,5-6; 61,1''\textgreater✲,
\bibleverse{6} und selig ist, wer an mir keinen Anstoß
nimmt\textless sup title=``oder: nicht irre wird''\textgreater✲!«

\bibleverse{7} Als diese nun den Rückweg antraten, begann Jesus zu den
Volksscharen über Johannes zu reden: »Wozu seid ihr damals\textless sup
title=``oder: jüngst''\textgreater✲ in die Wüste hinausgezogen? Wolltet
ihr euch ein Schilfrohr ansehen, das vom Winde hin und her bewegt wird?
\bibleverse{8} Nein; aber wozu seid ihr hinausgezogen? Wolltet ihr einen
Mann in weichen Gewändern sehen? Nein; die Leute, welche weiche Gewänder
tragen, sind in den Königsschlössern zu finden. \bibleverse{9} Aber wozu
seid ihr denn hinausgezogen? Wolltet ihr einen Propheten sehen? Ja, ich
sage euch: einen Mann, der noch mehr ist als ein Prophet!
\bibleverse{10} Denn dieser ist es, auf den sich das Schriftwort
bezieht\textless sup title=``Mal 3,1''\textgreater✲: ›Siehe, ich sende
meinen Boten✲ vor dir her, der dir den Weg vor dir her bereiten soll.‹
\bibleverse{11} Wahrlich ich sage euch: Unter den von Frauen Geborenen
ist keiner aufgetreten, der größer wäre als Johannes der Täufer; doch
der Kleinste im Himmelreich ist größer als er. \bibleverse{12} Aber seit
den Tagen\textless sup title=``=~dem Auftreten''\textgreater✲ Johannes
des Täufers bis jetzt bricht das Himmelreich sich mit Gewalt Bahn, und
die, welche Gewalt anwenden, reißen es an sich. \bibleverse{13} Denn
alle Propheten und das Gesetz haben bis auf Johannes geweissagt,
\bibleverse{14} und wenn ihr es annehmen wollt: Er ist Elia, der da
kommen soll\textless sup title=``vgl. 17,12; Mal 3,23; Lk
16,16''\textgreater✲. \bibleverse{15} Wer Ohren hat, der höre!

\bibleverse{16} Mit wem soll ich aber das gegenwärtige Geschlecht
vergleichen? Kindern gleicht es, die auf den öffentlichen Plätzen sitzen
und ihren Gespielen zurufen: \bibleverse{17} ›Wir haben euch gepfiffen,
doch ihr habt nicht getanzt; wir haben ein Klagelied angestimmt, doch
ihr habt euch nicht an die Brust geschlagen\textless sup title=``=~nicht
getrauert''\textgreater✲!‹ \bibleverse{18} Denn Johannes ist gekommen,
der nicht aß und nicht trank; da sagen sie: ›Er hat einen bösen
Geist\textless sup title=``oder: er ist von Sinnen''\textgreater✲.‹
\bibleverse{19} Nun ist der Menschensohn gekommen, welcher ißt und
trinkt; da sagen sie: ›Seht, der Schlemmer und Weintrinker, der Freund
der Zöllner und Sünder!‹ Und doch ist die Weisheit (Gottes)
gerechtfertigt worden durch ihre Werke.«

\hypertarget{ruxfcckblicke-jesu}{%
\subsubsection{2. Rückblicke Jesu}\label{ruxfcckblicke-jesu}}

\hypertarget{a-weherufe-jesu-uxfcber-die-unbuuxdffertigen-galiluxe4ischen-stuxe4dte}{%
\paragraph{a) Weherufe Jesu über die unbußfertigen galiläischen
Städte}\label{a-weherufe-jesu-uxfcber-die-unbuuxdffertigen-galiluxe4ischen-stuxe4dte}}

\bibleverse{20} Damals\textless sup title=``oder: darauf''\textgreater✲
begann er gegen die Städte, in denen seine meisten Wunder geschehen
waren, Drohworte zu richten, weil sie nicht Buße\textless sup
title=``vgl. 3,2''\textgreater✲ getan hatten: \bibleverse{21} »Wehe dir,
Chorazin! Wehe dir, Bethsaida! Denn wenn in Tyrus und Sidon die Wunder
geschehen wären, die in euch geschehen sind, so hätten sie längst in
Sack und Asche Buße getan. \bibleverse{22} Doch ich sage euch: Es wird
Tyrus und Sidon am Tage des Gerichts erträglicher ergehen als euch!
\bibleverse{23} Und du, Kapernaum, wirst doch nicht etwa bis zum Himmel
erhöht werden? Nein, bis zur Totenwelt wirst du hinabgestoßen
werden\textless sup title=``Jes 14,13.15''\textgreater✲. Denn wenn in
Sodom die Wunder geschehen wären, die in dir geschehen sind, so stände
es noch heutigen Tages. \bibleverse{24} Doch ich sage euch: Dem Lande
Sodom wird es am Tage des Gerichts erträglicher ergehen als dir!«

\hypertarget{b-jesu-jubelruf-und-lobpreis-des-vaters}{%
\paragraph{b) Jesu Jubelruf und Lobpreis des
Vaters}\label{b-jesu-jubelruf-und-lobpreis-des-vaters}}

\bibleverse{25} Zu jener Zeit hob Jesus an und sagte: »Ich preise
dich\textless sup title=``oder: danke dir''\textgreater✲, Vater, Herr
des Himmels und der Erde, daß du dies vor Weisen\textless sup
title=``oder: Gelehrten''\textgreater✲ und Klugen verborgen und es
Unmündigen geoffenbart hast; \bibleverse{26} ja, Vater, denn so ist es
dir wohlgefällig gewesen!~-- \bibleverse{27} Alles ist mir von meinem
Vater übergeben worden, und niemand erkennt den Sohn als nur der Vater,
und niemand erkennt den Vater als nur der Sohn und der, welchem der Sohn
ihn\textless sup title=``oder: es''\textgreater✲ offenbaren will.«

\hypertarget{c-der-heilandsruf-an-die-muxfchseligen-und-beladenen}{%
\paragraph{c) Der Heilandsruf an die Mühseligen und
Beladenen}\label{c-der-heilandsruf-an-die-muxfchseligen-und-beladenen}}

\bibleverse{28} »Kommt her zu mir alle, die ihr niedergedrückt und
belastet seid: ich will euch Ruhe schaffen! \bibleverse{29} Nehmt mein
Joch auf euch und lernt von mir; denn ich bin sanftmütig\textless sup
title=``oder: liebreich''\textgreater✲ und von Herzen demütig: so werdet
ihr Ruhe finden für eure Seelen\textless sup title=``Jer
6,16''\textgreater✲; \bibleverse{30} denn mein Joch ist sanft, und meine
Last ist leicht.«

\hypertarget{das-uxe4hrenraufen-der-juxfcnger-am-sabbat-der-erste-streit-jesu-mit-den-pharisuxe4ern-uxfcber-die-sabbatheiligung}{%
\subsubsection{3. Das Ährenraufen der Jünger am Sabbat; der erste Streit
Jesu mit den Pharisäern über die
Sabbatheiligung}\label{das-uxe4hrenraufen-der-juxfcnger-am-sabbat-der-erste-streit-jesu-mit-den-pharisuxe4ern-uxfcber-die-sabbatheiligung}}

\hypertarget{section-11}{%
\section{12}\label{section-11}}

\bibleverse{1} Zu jener Zeit wanderte Jesus an einem Sabbat durch die
Kornfelder; seine Jünger aber hatten Hunger und begannen daher, Ähren
abzupflücken und (die Körner) zu essen. \bibleverse{2} Als die Pharisäer
das wahrnahmen, sagten sie zu ihm: »Sieh doch! Deine Jünger tun da
etwas, was man am Sabbat nicht tun darf!« \bibleverse{3} Da antwortete
er ihnen: »Habt ihr nicht gelesen\textless sup title=``1.Sam
21,2-7''\textgreater✲, was David getan hat, als ihn samt seinen
Begleitern\textless sup title=``oder: Leuten''\textgreater✲ hungerte?
\bibleverse{4} Wie er da ins Gotteshaus hineinging und sie die
Schaubrote aßen, die doch er und seine Begleiter nicht essen durften,
sondern nur die Priester? \bibleverse{5} Oder habt ihr im Gesetz nicht
gelesen\textless sup title=``4.Mose 28,9''\textgreater✲, daß am Sabbat
die Priester im Tempel den Sabbat entheiligen und sich dadurch doch
nicht versündigen? \bibleverse{6} Ich sage euch aber: Hier steht
Größeres\textless sup title=``d.h. einer, der mehr ist''\textgreater✲
als der Tempel! \bibleverse{7} Wenn ihr aber erkannt hättet, was das
Wort besagt\textless sup title=``Hos 6,6''\textgreater✲: ›An
Barmherzigkeit habe ich Wohlgefallen und nicht an Schlachtopfern‹, so
hättet ihr die Unschuldigen nicht verurteilt; \bibleverse{8} denn der
Menschensohn ist Herr über den Sabbat.«

\hypertarget{heilung-des-mannes-mit-dem-geluxe4hmten-arm-am-sabbat-der-zweite-streit-uxfcber-die-sabbatheiligung}{%
\subsubsection{4. Heilung des Mannes mit dem gelähmten Arm am Sabbat;
der zweite Streit über die
Sabbatheiligung}\label{heilung-des-mannes-mit-dem-geluxe4hmten-arm-am-sabbat-der-zweite-streit-uxfcber-die-sabbatheiligung}}

\bibleverse{9} Er ging dann von dort weiter und kam in ihre\textless sup
title=``d.h. die dortige''\textgreater✲ Synagoge. \bibleverse{10} Da war
ein Mann, der einen gelähmten Arm hatte; und sie richteten die Frage an
ihn: »Darf man am Sabbat heilen?« -- sie wollten nämlich einen Grund zu
einer Anklage gegen ihn haben. \bibleverse{11} Er aber antwortete ihnen:
»Wo wäre jemand unter euch, der ein einziges Schaf besitzt und, wenn
dieses ihm am Sabbat in eine Grube fällt, es nicht ergriffe und
herauszöge? \bibleverse{12} Wieviel wertvoller ist nun aber ein Mensch
als ein Schaf! Also darf man am Sabbat Gutes tun.« \bibleverse{13}
Hierauf sagte er zu dem Manne: »Strecke deinen Arm aus!« Er streckte ihn
aus, und er wurde wiederhergestellt, gesund wie der andere.
\bibleverse{14} Da gingen die Pharisäer hinaus und faßten einen Beschluß
gegen ihn, um ihn umzubringen\textless sup title=``oder: unschädlich zu
machen''\textgreater✲.

\hypertarget{jesus-entzieht-sich-der-verfolgung-seine-gottwohlgefuxe4llige-heilstuxe4tigkeit}{%
\subsubsection{5. Jesus entzieht sich der Verfolgung; seine
gottwohlgefällige
Heilstätigkeit}\label{jesus-entzieht-sich-der-verfolgung-seine-gottwohlgefuxe4llige-heilstuxe4tigkeit}}

\bibleverse{15} Als Jesus das erfuhr, zog er sich von dort zurück; und
es zogen ihm viele nach, die er alle heilte, \bibleverse{16} denen er
aber die strenge Weisung gab, sie sollten Stillschweigen über ihn
bewahren. \bibleverse{17} So sollte das Wort des Propheten Jesaja seine
Erfüllung finden, der da sagt\textless sup title=``Jes
42,1-4''\textgreater✲: \bibleverse{18} »Siehe, mein Knecht, den ich
erwählt habe, mein Geliebter, an dem mein Herz Wohlgefallen gefunden
hat! Ich will meinen Geist auf ihn legen, und er soll den Heidenvölkern
das Gericht ankündigen\textless sup title=``oder: das Recht
verkündigen''\textgreater✲. \bibleverse{19} Er wird nicht zanken noch
schreien, und niemand wird seine Stimme auf den Straßen hören;
\bibleverse{20} ein geknicktes Rohr wird er nicht zerbrechen und einen
glimmenden Docht nicht auslöschen, bis er das Gericht\textless sup
title=``oder: das Recht''\textgreater✲ siegreich durchgeführt hat;
\bibleverse{21} und auf seinen Namen werden die Heidenvölker ihre
Hoffnung setzen.«

\hypertarget{jesus-verteidigt-sich-gegen-die-beelzebul-luxe4sterung-der-pharisuxe4er}{%
\subsubsection{6. Jesus verteidigt sich gegen die Beelzebul-Lästerung
der
Pharisäer}\label{jesus-verteidigt-sich-gegen-die-beelzebul-luxe4sterung-der-pharisuxe4er}}

\bibleverse{22} Damals brachte man einen Besessenen zu ihm, der blind
und stumm war, und er heilte ihn, so daß der Stumme redete und sehen
konnte. \bibleverse{23} Da geriet die ganze Volksmenge vor Staunen außer
sich und sagte: »Sollte dieser nicht doch der Sohn Davids sein?«
\bibleverse{24} Als die Pharisäer das hörten, erklärten sie: »Dieser
treibt die bösen Geister nur im Bunde mit Beelzebul✲, dem Obersten✲ der
bösen Geister, aus.« \bibleverse{25} Weil Jesus nun ihre Gedanken
kannte, sagte er zu ihnen: »Jedes Reich, das in sich selbst uneinig ist,
wird verwüstet, und keine Stadt, kein Haus\textless sup title=``oder:
keine Familie''\textgreater✲, die in sich selbst uneinig sind, können
Bestand haben. \bibleverse{26} Wenn nun der Satan den Satan austreibt,
so ist er mit sich selbst in Zwiespalt geraten: wie kann da seine
Herrschaft Bestand haben? \bibleverse{27} Und wenn ich die bösen Geister
im Bunde mit Beelzebul austreibe, mit wessen Hilfe treiben dann eure
Söhne\textless sup title=``=~eigenen Leute''\textgreater✲ sie aus? Darum
werden diese eure Richter sein! \bibleverse{28} Wenn ich aber die bösen
Geister durch den Geist Gottes austreibe, so ist ja das Reich Gottes zu
euch gekommen\textless sup title=``=~schon unter euch''\textgreater✲.
\bibleverse{29} Oder wie könnte jemand in das Haus des Starken
eindringen und ihm sein Rüstzeug\textless sup title=``oder: seinen
Hausrat''\textgreater✲ rauben, ohne zunächst den Starken gefesselt zu
haben? Erst dann kann er ihm das Haus ausplündern.~-- \bibleverse{30}
Wer nicht mit mir ist, der ist gegen mich, und wer nicht mit mir
sammelt, der zerstreut.«\textless sup title=``Mk 9,40; Lk
9,50''\textgreater✲

\hypertarget{warnung-vor-der-luxe4sterung-des-geistes-vom-baum-und-den-fruxfcchten}{%
\paragraph{Warnung vor der Lästerung des Geistes; vom Baum und den
Früchten}\label{warnung-vor-der-luxe4sterung-des-geistes-vom-baum-und-den-fruxfcchten}}

\bibleverse{31} »Deshalb sage ich euch: Jede Sünde und Lästerung wird
den Menschen vergeben werden, aber die Lästerung des
Geistes\textless sup title=``=~gegen den Geist''\textgreater✲ wird nicht
vergeben werden. \bibleverse{32} Auch wenn jemand ein Wort\textless sup
title=``=~eine Schmähung''\textgreater✲ gegen den Menschensohn
ausspricht, wird es ihm vergeben werden; wer aber gegen den heiligen
Geist spricht, dem wird es weder in dieser Weltzeit noch in der
künftigen vergeben werden. \bibleverse{33} Entweder macht den Baum
gut\textless sup title=``oder: zu einem guten''\textgreater✲, dann ist
auch seine Frucht gut; oder macht den Baum faul\textless sup
title=``vgl. 7,17''\textgreater✲, dann ist auch seine Frucht faul; denn
an der Frucht erkennt man den Baum. \bibleverse{34} Ihr Schlangenbrut!
Wie solltet ihr imstande sein, Gutes zu reden, da ihr doch böse seid?
Denn wovon das Herz voll ist, davon redet der Mund. \bibleverse{35} Ein
guter Mensch bringt aus der guten Schatzkammer (seines Herzens) Gutes
hervor, während ein böser Mensch aus seiner bösen Schatzkammer Böses
hervorbringt. \bibleverse{36} Ich sage euch aber: Von jedem
unnützen\textless sup title=``oder: nichtsnutzigen''\textgreater✲ Wort,
das die Menschen reden, davon werden sie Rechenschaft am Tage des
Gerichts zu geben haben; \bibleverse{37} denn nach deinen
Worten\textless sup title=``=~auf Grund deiner Worte''\textgreater✲
wirst du gerechtgesprochen werden, und nach deinen Worten wirst du
verurteilt werden.«

\hypertarget{jesu-abweisung-der-zeichenforderung-das-jonazeichen-das-gleichnis-vom-ruxfcckfall}{%
\subsubsection{7. Jesu Abweisung der Zeichenforderung; das Jonazeichen;
das Gleichnis vom
Rückfall}\label{jesu-abweisung-der-zeichenforderung-das-jonazeichen-das-gleichnis-vom-ruxfcckfall}}

\bibleverse{38} Daraufhin entgegneten ihm einige von den
Schriftgelehrten und Pharisäern: »Meister, wir möchten ein Wunderzeichen
von dir sehen!« \bibleverse{39} Er aber gab ihnen zur Antwort: »Ein
böses und ehebrecherisches\textless sup title=``=~von Gott
abtrünniges''\textgreater✲ Geschlecht verlangt ein Zeichen; doch es wird
ihm kein Zeichen gegeben werden als das Zeichen des Propheten Jona.
\bibleverse{40} Denn wie Jona drei Tage und drei Nächte im Leibe des
Riesenfisches gewesen ist\textless sup title=``Jona
2,1-2''\textgreater✲, so wird der Menschensohn drei Tage und drei Nächte
im Inneren\textless sup title=``oder: Schoß''\textgreater✲ der Erde
sein. \bibleverse{41} Die Männer von Ninive werden beim Gericht
mit\textless sup title=``oder: neben''\textgreater✲ diesem Geschlecht
(als Zeugen) auftreten und seine Verurteilung herbeiführen; denn sie
haben auf Jonas Predigt hin Buße getan\textless sup title=``Jona
3,5''\textgreater✲, und hier steht doch Größeres\textless sup
title=``d.h. einer, der mehr ist''\textgreater✲ als Jona!
\bibleverse{42} Die Königin aus dem Südland\textless sup title=``1.Kön
10,1-10''\textgreater✲ wird beim Gericht mit\textless sup title=``oder:
neben''\textgreater✲ diesem Geschlecht (als Zeugin) auftreten und seine
Verurteilung herbeiführen; denn sie kam von den Enden der Erde, um die
Weisheit Salomos zu hören, und hier steht doch Größeres\textless sup
title=``d.h. einer, der mehr ist''\textgreater✲ als Salomo!

\bibleverse{43} Wenn aber der unreine Geist von einem Menschen
ausgefahren ist, so durchirrt er wüste Gegenden und sucht dort eine
Ruhestätte, findet aber keine. \bibleverse{44} Da sagt\textless sup
title=``oder: denkt''\textgreater✲ er dann: ›Ich will in mein Haus
zurückkehren, das ich verlassen habe!‹ Wenn er dann hinkommt, findet er
es leer stehen, sauber gefegt und schön aufgeräumt. \bibleverse{45}
Hierauf geht er hin und nimmt noch sieben andere Geister mit sich, die
noch schlimmer sind als er selbst, und sie ziehen ein und nehmen dort
Wohnung, und das Ende wird bei einem solchen Menschen schlimmer, als
sein Anfang war. Ebenso wird es auch diesem bösen Geschlecht ergehen.«

\hypertarget{die-wahren-verwandten-jesu}{%
\subsubsection{8. Die wahren Verwandten
Jesu}\label{die-wahren-verwandten-jesu}}

\bibleverse{46} Während er noch zu den Volksscharen redete, siehe, da
standen seine Mutter und seine Brüder draußen und wünschten ihn zu
sprechen. \bibleverse{47} Da sagte jemand zu ihm: »Deine Mutter und
deine Brüder stehen draußen und wünschen dich zu sprechen.«
\bibleverse{48} Er aber gab dem, der es ihm meldete, zur Antwort: »Wer
ist meine Mutter, und wer sind meine Brüder?« \bibleverse{49} Dann
streckte er seine Hand aus zu seinen Jüngern hin und sagte: »Seht, diese
hier sind meine Mutter und meine Brüder; \bibleverse{50} denn wer den
Willen meines himmlischen Vaters tut, der ist mein Bruder und Schwester
und Mutter!«

\hypertarget{jesu-seepredigt-in-sieben-gleichnissen-vom-himmelreich}{%
\subsubsection{9. Jesu Seepredigt in sieben Gleichnissen vom
Himmelreich}\label{jesu-seepredigt-in-sieben-gleichnissen-vom-himmelreich}}

\hypertarget{a-einleitende-bemerkungen-das-gleichnis-vom-suxe4mann-und-vierfachen-ackerfeld-mit-zwischenrede-und-deutung}{%
\paragraph{a) Einleitende Bemerkungen; das Gleichnis vom Sämann und
vierfachen Ackerfeld (mit Zwischenrede und
Deutung)}\label{a-einleitende-bemerkungen-das-gleichnis-vom-suxe4mann-und-vierfachen-ackerfeld-mit-zwischenrede-und-deutung}}

\hypertarget{section-12}{%
\section{13}\label{section-12}}

\bibleverse{1} An jenem Tage ging Jesus von Hause weg und setzte sich am
See nieder; \bibleverse{2} und es versammelte sich eine große Volksmenge
bei ihm, so daß er in ein Boot stieg und sich darin niedersetzte,
während die ganze Volksmenge längs dem Ufer stand. \bibleverse{3} Da
redete er mancherlei zu ihnen in Gleichnissen mit den Worten: »Seht, der
Sämann ging aus, um zu säen; \bibleverse{4} und beim Säen fiel einiges
(von dem Saatkorn) auf den Weg längshin\textless sup title=``oder:
daneben''\textgreater✲; da kamen die Vögel und fraßen es auf.
\bibleverse{5} Anderes fiel auf die felsigen Stellen, wo es nicht viel
Erdreich hatte und bald aufschoß, weil es nicht tief in den Boden
dringen konnte; \bibleverse{6} als dann aber die Sonne aufgegangen war,
wurde es versengt, und weil es nicht Wurzel (geschlagen) hatte,
verdorrte es. \bibleverse{7} Wieder anderes fiel unter die Dornen, und
die Dornen wuchsen empor und erstickten es. \bibleverse{8} Anderes aber
fiel auf den guten Boden und brachte Frucht, das eine hundertfältig, das
andere sechzigfältig, das andere dreißigfältig. \bibleverse{9} Wer Ohren
hat, der höre!«

\hypertarget{jesu-erkluxe4rung-uxfcber-den-grund-und-zweck-seiner-gleichnisse}{%
\paragraph{Jesu Erklärung über den Grund und Zweck seiner
Gleichnisse}\label{jesu-erkluxe4rung-uxfcber-den-grund-und-zweck-seiner-gleichnisse}}

\bibleverse{10} Da traten die Jünger an Jesus heran und fragten ihn:
»Warum redest du in Gleichnissen✲ zu ihnen?« \bibleverse{11} Er
antwortete: »Euch ist es gegeben\textless sup title=``oder:
verliehen''\textgreater✲, die Geheimnisse des Himmelreichs zu erkennen,
jenen aber ist es nicht gegeben. \bibleverse{12} Denn wer da hat, dem
wird gegeben werden, so daß er Überfluß\textless sup title=``oder:
reichlich''\textgreater✲ hat; wer aber nicht\textless sup title=``=~so
gut wie nichts''\textgreater✲ hat, dem wird auch das genommen werden,
was er hat. \bibleverse{13} Deshalb rede ich in Gleichnissen zu ihnen,
weil sie mit sehenden Augen doch nicht sehen und mit hörenden Ohren doch
nicht hören und nicht verstehen. \bibleverse{14} So geht an ihnen die
Weissagung Jesajas in Erfüllung\textless sup title=``Jes
6,9-10''\textgreater✲, die da lautet: ›Ihr werdet immerfort hören und
doch nicht verstehen, und ihr werdet immerfort sehen und doch nicht
wahrnehmen\textless sup title=``oder: erkennen''\textgreater✲!
\bibleverse{15} Denn das Herz dieses Volkes ist stumpf✲ geworden: ihre
Ohren sind schwerhörig geworden, und ihre Augen haben sie geschlossen,
damit sie mit den Augen nicht sehen und mit den Ohren nicht hören und
mit dem Herzen nicht zum Verständnis gelangen, und sie sich (nicht)
bekehren, daß ich sie heilen könnte.‹ \bibleverse{16} Aber eure Augen
sind selig (zu preisen), weil sie sehen, und eure Ohren, weil sie hören!
\bibleverse{17} Denn wahrlich ich sage euch: Viele Propheten und
Gerechte haben sehnlichst gewünscht, das zu sehen, was ihr seht, und
haben es nicht gesehen, und hätten gerne das gehört, was ihr hört, und
haben es nicht zu hören bekommen.«

\hypertarget{deutung-des-gleichnisses-vom-suxe4mann}{%
\paragraph{Deutung des Gleichnisses vom
Sämann}\label{deutung-des-gleichnisses-vom-suxe4mann}}

\bibleverse{18} »Ihr sollt also die Deutung des Gleichnisses vom Sämann
zu hören bekommen. \bibleverse{19} Bei jedem, der das Wort vom Reich
(Gottes) hört und es nicht versteht, da kommt der Böse und reißt das
aus, was in sein Herz gesät ist; bei diesem ist der Same auf den Weg
längshin\textless sup title=``oder: daneben''\textgreater✲ gefallen.
\bibleverse{20} Wo aber auf die felsigen Stellen gesät worden ist, das
bedeutet einen solchen, der das Wort hört und es für den Augenblick mit
Freuden annimmt; \bibleverse{21} er hat aber keine feste Wurzel in sich,
sondern ist ein Kind des Augenblicks; wenn dann Bedrängnis oder
Verfolgung um des Wortes willen eintritt, wird er sogleich irre.
\bibleverse{22} Wo sodann unter die Dornen gesät worden ist, das
bedeutet einen Menschen, der das Wort wohl hört, bei dem aber die
weltlichen Sorgen und der Betrug des Reichtums das Wort ersticken, so
daß es ohne Frucht bleibt. \bibleverse{23} Wo aber auf den guten Boden
gesät worden ist, das bedeutet einen solchen, der das Wort hört und auch
versteht; dieser bringt dann auch Frucht, und der eine trägt
hundertfältig, der andere sechzigfältig, der andere dreißigfältig.«

\hypertarget{b-das-gleichnis-vom-unkraut-unter-dem-weizen}{%
\paragraph{b) Das Gleichnis vom Unkraut unter dem
Weizen}\label{b-das-gleichnis-vom-unkraut-unter-dem-weizen}}

\bibleverse{24} Ein anderes Gleichnis legte er ihnen so vor: »Mit dem
Himmelreich verhält es sich wie mit einem Manne, der guten Samen auf
seinem Acker ausgesät hatte. \bibleverse{25} Während aber die Leute
schliefen, kam sein Feind, säte Unkraut zwischen den Weizen und
entfernte sich dann wieder. \bibleverse{26} Als nun die Saat aufwuchs
und Frucht ansetzte, da kam auch das Unkraut zum Vorschein.
\bibleverse{27} Da traten die Knechte zu dem Hausherrn und sagten:
›Herr, hast du nicht guten Samen auf deinen Acker gesät? Woher hat er
denn nun das Unkraut?‹ \bibleverse{28} Er antwortete ihnen: ›Das hat ein
Feind getan.‹ Die Knechte fragten ihn weiter: ›Willst du nun, daß wir
hingehen und es zusammenlesen?‹ \bibleverse{29} Doch er antwortete:
›Nein, ihr würdet sonst beim Sammeln des Unkrauts zugleich auch den
Weizen ausreißen. \bibleverse{30} Laßt beides zusammen bis zur Ernte
wachsen; dann will ich zur Erntezeit den Schnittern sagen: Lest zuerst
das Unkraut zusammen und bindet es in Bündel, damit man es verbrenne;
den Weizen aber sammelt in meine Scheuer!‹«

\hypertarget{c-die-zwei-gleichnisse-vom-senfkorn-und-vom-sauerteig}{%
\paragraph{c) Die zwei Gleichnisse vom Senfkorn und vom
Sauerteig}\label{c-die-zwei-gleichnisse-vom-senfkorn-und-vom-sauerteig}}

\bibleverse{31} Ein anderes Gleichnis legte er ihnen so vor: »Das
Himmelreich ist einem Senfkorn vergleichbar, das ein Mann nahm und auf
seinen Acker säte. \bibleverse{32} Dies ist das kleinste unter allen
Samenarten; wenn es aber herangewachsen ist, dann ist es größer als die
anderen Gartengewächse und wird zu einem Baum, so daß die Vögel des
Himmels kommen und in seinen Zweigen nisten.«\textless sup title=``vgl.
Hes 17,23; 31,6''\textgreater✲

\bibleverse{33} Noch ein anderes Gleichnis teilte er ihnen so mit: »Das
Himmelreich gleicht dem Sauerteig, den eine Frau nahm und unter drei
Scheffel Mehl mengte, bis der ganze Teig durchsäuert war.«

\hypertarget{d-erster-schluuxdf-der-gleichnisreden-und-deutung-des-gleichnisses-vom-unkraut-unter-dem-weizen}{%
\paragraph{d) Erster Schluß der Gleichnisreden und Deutung des
Gleichnisses vom Unkraut unter dem
Weizen}\label{d-erster-schluuxdf-der-gleichnisreden-und-deutung-des-gleichnisses-vom-unkraut-unter-dem-weizen}}

\bibleverse{34} Dies alles redete Jesus in Gleichnissen zu den
Volksscharen, und ohne Gleichnisse redete er nichts zu ihnen.
\bibleverse{35} So sollte sich das Wort des Propheten erfüllen, der da
sagt\textless sup title=``Ps 78,2''\textgreater✲: »Ich will meinen Mund
zu Gleichnissen auftun, ich will aussprechen, was seit Grundlegung der
Welt verborgen gewesen ist.«

\bibleverse{36} Hierauf entließ er die Volksmenge und begab sich in
seine Wohnung. Da traten seine Jünger zu ihm und baten ihn: »Erkläre uns
das Gleichnis vom Unkraut auf dem Acker!« \bibleverse{37} Er antwortete:
»Der Mann, der den guten Samen sät, ist der Menschensohn;
\bibleverse{38} der Acker ist die Welt; die gute Saat, das sind die
Söhne✲ des Reiches; das Unkraut dagegen sind die Söhne✲ des Bösen;
\bibleverse{39} der Feind ferner, der das Unkraut gesät hat, ist der
Teufel; die Ernte ist das Ende dieser Weltzeit, und die Schnitter sind
Engel. \bibleverse{40} Wie nun das Unkraut gesammelt und im Feuer
verbrannt wird, so wird es auch am Ende der Weltzeit der Fall sein:
\bibleverse{41} Der Menschensohn wird seine Engel aussenden; die werden
aus seinem Reich alle Ärgernisse\textless sup title=``d.h.
Verführer''\textgreater✲ und alle die sammeln, welche die
Gesetzlosigkeit üben, \bibleverse{42} und werden sie in den Feuerofen
werfen: dort wird lautes Weinen und Zähneknirschen sein. \bibleverse{43}
Alsdann werden die Gerechten im Reich ihres Vaters wie die Sonne
leuchten\textless sup title=``vgl. Dan 12,3''\textgreater✲. Wer Ohren
hat, der höre!«

\hypertarget{e-die-drei-letzten-gleichnisse-schatz-im-acker-kostbare-perle-fischnetz-abschluuxdf-der-gleichnisrede}{%
\paragraph{e) Die drei letzten Gleichnisse (Schatz im Acker; kostbare
Perle; Fischnetz); Abschluß der
Gleichnisrede}\label{e-die-drei-letzten-gleichnisse-schatz-im-acker-kostbare-perle-fischnetz-abschluuxdf-der-gleichnisrede}}

\bibleverse{44} »Das Himmelreich ist einem im Acker vergrabenen Schatz
gleich; den fand ein Mann und vergrub ihn (wieder); alsdann ging er in
seiner Freude hin, verkaufte alles, was er besaß, und kaufte jenen
Acker.

\bibleverse{45} Wiederum gleicht das Himmelreich einem Kaufmann, der
wertvolle Perlen suchte; \bibleverse{46} und als er eine besonders
kostbare Perle gefunden hatte, ging er heim, verkaufte alles, was er
besaß, und kaufte sie.

\bibleverse{47} Weiter ist das Himmelreich einem Schleppnetz gleich, das
ins Meer ausgeworfen wurde und in welchem sich Fische jeder Art in Menge
fingen. \bibleverse{48} Als es ganz gefüllt war, zog man es an den
Strand, setzte sich nieder und sammelte das Gute\textless sup
title=``=~die guten Fische''\textgreater✲ in Gefäße, das
Faule\textless sup title=``=~die unbrauchbaren''\textgreater✲ aber warf
man weg. \bibleverse{49} So wird es auch am Ende der Weltzeit zugehen:
Die Engel werden ausgehen und die Bösen aus der Mitte der Gerechten
absondern \bibleverse{50} und sie in den Feuerofen werfen: dort wird
lautes Weinen und Zähneknirschen sein.«

\bibleverse{51} »Habt ihr dies alles verstanden?« Sie antworteten ihm:
»Ja.« \bibleverse{52} Da sagte er zu ihnen: »Deshalb ist jeder
Schriftgelehrte\textless sup title=``oder: Lehrer''\textgreater✲, der in
der Schule des Himmelreichs ausgebildet ist, einem Hausherrn gleich, der
aus seinem Schatze\textless sup title=``oder: reichen
Vorrat''\textgreater✲ Neues und Altes hervorholt\textless sup
title=``oder: austeilt''\textgreater✲.«

\hypertarget{verwerfung-und-miuxdferfolg-jesu-in-seiner-vaterstadt-nazareth}{%
\subsubsection{10. Verwerfung und Mißerfolg Jesu in seiner Vaterstadt
Nazareth}\label{verwerfung-und-miuxdferfolg-jesu-in-seiner-vaterstadt-nazareth}}

\bibleverse{53} Als Jesus nun diese Gleichnisse beendigt hatte, brach er
von dort auf; \bibleverse{54} und als er in seine Vaterstadt (Nazareth)
gekommen war, machte er in ihrer Synagoge durch seine Lehre solchen
Eindruck auf sie, daß sie in Erstaunen gerieten und fragten: »Woher hat
dieser solche Weisheit und die Machttaten\textless sup title=``oder:
Wunderkräfte''\textgreater✲? \bibleverse{55} Ist dieser nicht der Sohn
des Zimmermanns? Heißt seine Mutter nicht Maria, und sind nicht Jakobus
und Joseph, Simon und Judas seine Brüder? \bibleverse{56} Wohnen nicht
auch seine Schwestern alle hier bei uns? Woher hat dieser also dies
alles?« \bibleverse{57} So nahmen sie Anstoß\textless sup title=``oder:
wurden sie irre''\textgreater✲ an ihm. Jesus aber sagte zu ihnen: »Ein
Prophet gilt nirgends weniger als in seiner Vaterstadt und in seiner
Familie.« \bibleverse{58} So tat er denn dort infolge ihres Unglaubens
nicht viele Wunder.

\hypertarget{v.-weitere-geschichten-aus-jesu-wanderleben-innerhalb-und-auuxdferhalb-galiluxe4as-141-1612}{%
\subsection{V. Weitere Geschichten aus Jesu Wanderleben innerhalb und
außerhalb Galiläas
(14,1-16,12)}\label{v.-weitere-geschichten-aus-jesu-wanderleben-innerhalb-und-auuxdferhalb-galiluxe4as-141-1612}}

\hypertarget{jesus-und-herodes-das-ende-johannes-des-tuxe4ufers}{%
\subsubsection{1. Jesus und Herodes; das Ende Johannes des
Täufers}\label{jesus-und-herodes-das-ende-johannes-des-tuxe4ufers}}

\hypertarget{section-13}{%
\section{14}\label{section-13}}

\bibleverse{1} Zu jener Zeit erhielt der Vierfürst Herodes Kunde von
Jesus \bibleverse{2} und sagte zu seinen Dienern✲: »Das ist Johannes der
Täufer; der ist von den Toten auferweckt worden; darum sind die
Wunderkräfte in ihm wirksam.« \bibleverse{3} Herodes hatte nämlich den
Johannes festnehmen und in Fesseln und ins Gefängnis werfen lassen mit
Rücksicht auf Herodias, die Gattin seines Bruders Philippus;
\bibleverse{4} denn Johannes hatte ihm vorgehalten: »Du darfst sie nicht
(zur Frau) haben.«\textless sup title=``3.Mose 18,16''\textgreater✲
\bibleverse{5} Er hätte ihn nun am liebsten ums Leben bringen lassen,
fürchtete sich aber vor dem Volk, weil dieses ihn für einen Propheten
hielt. \bibleverse{6} Als aber der Geburtstag des Herodes gefeiert
wurde, tanzte die Tochter der Herodias vor der Festgesellschaft und
gefiel dem Herodes so sehr, \bibleverse{7} daß er ihr mit einem Eide
zusagte, er wolle ihr jede Bitte gewähren. \bibleverse{8} Da sagte sie,
schon vorher von ihrer Mutter dazu angestiftet: »Gib mir hier auf einer
Schüssel den Kopf Johannes des Täufers!« \bibleverse{9} Obgleich nun der
König mißmutig darüber war, gab er doch wegen seiner Eide und mit
Rücksicht auf seine Tischgäste den Befehl, man solle ihn\textless sup
title=``d.h. den Kopf''\textgreater✲ ihr geben; \bibleverse{10} er
schickte also (Diener) hin und ließ Johannes im Gefängnis enthaupten.
\bibleverse{11} Sein Kopf wurde dann auf einer Schüssel gebracht und dem
Mädchen gegeben; die brachte ihn ihrer Mutter. \bibleverse{12} Die
Jünger des Johannes kamen hierauf, holten den Leichnam und bestatteten
ihn; dann gingen sie hin und berichteten es Jesus.

\hypertarget{speisung-der-fuxfcnftausend}{%
\subsubsection{2. Speisung der
Fünftausend}\label{speisung-der-fuxfcnftausend}}

\bibleverse{13} Als Jesus dies hörte, entwich er von dort in einem Boote
an einen einsamen Ort, um für sich allein zu sein; doch als die
Volksmenge das erfuhr, folgte sie ihm zu Fuß aus den Städten nach.
\bibleverse{14} Als er dann (aus der Einsamkeit) wieder hervorkam und
eine große Volksmenge sah, ergriff ihn Mitleid mit ihnen, und er heilte
ihre Kranken. \bibleverse{15} Als es aber Abend geworden war, traten
seine Jünger zu ihm und sagten: »Die Gegend hier ist öde und die Zeit
schon vorgerückt; laß daher das Volk ziehen, damit sie in die
Ortschaften gehen und sich Lebensmittel kaufen!« \bibleverse{16} Jesus
aber erwiderte ihnen: »Sie brauchen nicht wegzugehen: gebt ihr ihnen zu
essen!« \bibleverse{17} Da antworteten sie ihm: »Wir haben hier nichts
weiter als fünf Brote und zwei Fische.« \bibleverse{18} Er aber sagte:
»Bringt sie mir hierher!« \bibleverse{19} Er ließ dann die Volksscharen
sich auf dem Rasen lagern, nahm die fünf Brote und die beiden Fische,
blickte zum Himmel empor, sprach den Lobpreis (Gottes) und brach die
Brote; hierauf gab er sie\textless sup title=``d.h. die
Brotstücke''\textgreater✲ den Jüngern, die Jünger aber teilten sie dem
Volke zu. \bibleverse{20} Und sie aßen alle und wurden satt; dann
sammelte man die Brocken, die übriggeblieben waren: zwölf Körbe voll.
\bibleverse{21} Die Zahl derer aber, die gegessen hatten, betrug etwa
fünftausend Männer, ungerechnet die Frauen und die Kinder.

\hypertarget{ruxfcckfahrt-der-juxfcnger-uxfcber-den-see-bei-nacht-das-wandeln-jesu-auf-dem-see-die-landung-in-gennesaret}{%
\subsubsection{3. Rückfahrt der Jünger über den See bei Nacht; das
Wandeln Jesu auf dem See; die Landung in
Gennesaret}\label{ruxfcckfahrt-der-juxfcnger-uxfcber-den-see-bei-nacht-das-wandeln-jesu-auf-dem-see-die-landung-in-gennesaret}}

\bibleverse{22} Und sogleich nötigte Jesus seine Jünger, ins Boot zu
steigen und vor ihm nach dem jenseitigen Ufer hinüberzufahren, damit er
inzwischen die Volksscharen entließe. \bibleverse{23} Als er das getan
hatte, stieg er für sich allein den Berg hinan, um zu beten; und als es
Abend geworden war, befand er sich dort allein; \bibleverse{24} das Boot
aber war schon mitten auf dem See und wurde von den Wellen hart
bedrängt, denn der Wind stand ihnen entgegen. \bibleverse{25} In der
vierten Nachtwache aber kam Jesus auf sie zu, indem er über den See
dahinging. \bibleverse{26} Als nun die Jünger ihn so auf dem See wandeln
sahen, gerieten sie in Bestürzung, weil sie dachten, es sei ein
Gespenst, und sie schrien vor Angst laut auf. \bibleverse{27} Doch Jesus
redete sie sogleich mit den Worten an: »Seid getrost: ich bin es;
fürchtet euch nicht!« \bibleverse{28} Da antwortete ihm Petrus: »Herr,
wenn du es bist, so laß mich über das Wasser zu dir kommen!«
\bibleverse{29} Er erwiderte: »So komm!« Da stieg Petrus aus dem Boot,
ging über das Wasser hin und kam auf Jesus zu; \bibleverse{30} doch als
er den Sturmwind wahrnahm, wurde ihm angst, und als er unterzusinken
begann, rief er laut: »Herr, hilf mir!« \bibleverse{31} Sogleich
streckte Jesus die Hand aus, faßte ihn und sagte zu ihm: »Du
Kleingläubiger! Warum hast du gezweifelt?« \bibleverse{32} Als sie dann
in das Boot gestiegen waren, legte sich der Wind. \bibleverse{33} Die
Männer im Boot aber warfen sich vor ihm nieder und sagten: »Du bist
wahrhaftig Gottes Sohn!«

\hypertarget{der-volkszusammenlauf-und-die-krankenheilungen-in-gennesaret}{%
\paragraph{Der Volkszusammenlauf und die Krankenheilungen in
Gennesaret}\label{der-volkszusammenlauf-und-die-krankenheilungen-in-gennesaret}}

\bibleverse{34} Nachdem sie dann (über den See) hinübergefahren waren,
kamen sie ans Land nach Gennesaret. \bibleverse{35} Sobald die Bewohner
dieses Ortes ihn erkannt hatten, schickten sie Boten in die ganze
dortige Umgegend, und man brachte alle Kranken zu ihm, \bibleverse{36}
und (diese) baten ihn, nur die Quaste seines Rockes\textless sup
title=``oder: Mantels; vgl. 4.Mose 15,38-39''\textgreater✲ anfassen zu
dürfen, und alle, die sie anfaßten, wurden völlig geheilt.

\hypertarget{jesu-streit-mit-den-gegnern-um-das-huxe4ndewaschen-seine-warnung-vor-menschensatzungen-und-kennzeichnung-der-wahren-unreinheit}{%
\subsubsection{4. Jesu Streit mit den Gegnern um das Händewaschen; seine
Warnung vor Menschensatzungen und Kennzeichnung der wahren
Unreinheit}\label{jesu-streit-mit-den-gegnern-um-das-huxe4ndewaschen-seine-warnung-vor-menschensatzungen-und-kennzeichnung-der-wahren-unreinheit}}

\hypertarget{section-14}{%
\section{15}\label{section-14}}

\bibleverse{1} Damals kamen Pharisäer und Schriftgelehrte aus Jerusalem
zu Jesus und fragten ihn: \bibleverse{2} »Warum übertreten deine Jünger
die Satzungen, welche uns die Alten\textless sup title=``=~unsere
Vorfahren''\textgreater✲ überliefert haben? Sie waschen sich ja die
Hände nicht, wenn sie Brot essen\textless sup title=``oder: eine
Mahlzeit einnehmen''\textgreater✲ wollen.« \bibleverse{3} Da antwortete
er ihnen mit den Worten: »Warum übertretet auch ihr\textless sup
title=``oder: ihr selber''\textgreater✲ das Gebot Gottes euren
überlieferten Satzungen zuliebe? \bibleverse{4} Gott hat doch
geboten\textless sup title=``2.Mose 20,12''\textgreater✲: ›Ehre deinen
Vater und deine Mutter‹ und\textless sup title=``2.Mose
21,17''\textgreater✲: ›Wer Vater oder Mutter flucht\textless sup
title=``oder: schmäht''\textgreater✲, soll des Todes sterben!‹
\bibleverse{5} Ihr aber sagt: ›Wer zum Vater oder zur Mutter sagt: Ich
will Gott als Opfergabe (für den Tempelschatz) das weihen, was du sonst
als Unterstützung von mir empfangen hättest,~-- \bibleverse{6} der
braucht seinen Vater oder seine Mutter nicht weiter zu ehren.‹ Damit
habt ihr das Wort Gottes euren überlieferten Satzungen zulieb außer
Kraft gesetzt! \bibleverse{7} Ihr Heuchler✲! Treffend hat Jesaja von
euch geweissagt mit den Worten\textless sup title=``Jes
29,13''\textgreater✲: \bibleverse{8} ›Dieses Volk ehrt mich nur mit den
Lippen, ihr Herz aber ist weit entfernt von mir; \bibleverse{9} doch
vergeblich verehren sie mich, weil sie Menschensatzungen als Lehren
vortragen.‹« \bibleverse{10} Nachdem er dann die Volksmenge
herbeigerufen hatte, sagte er zu ihnen: »Hört zu und sucht es zu
verstehen! \bibleverse{11} Nicht das, was in den Mund hineingeht,
verunreinigt den Menschen, sondern was aus dem Munde herauskommt, das
macht den Menschen unrein.«

\bibleverse{12} Hierauf traten die Jünger an ihn heran und sagten zu
ihm: »Weißt du, daß die Pharisäer an dem Wort, das sie von dir haben
hören müssen, Anstoß genommen haben?« \bibleverse{13} Er aber
antwortete: »Jede Pflanze, die nicht mein himmlischer Vater gepflanzt
hat, wird mit der Wurzel ausgerissen werden. \bibleverse{14} Laßt sie
nur: sie sind blinde Blindenführer! Wenn aber ein Blinder einem anderen
Blinden Wegführer ist, werden beide in die Grube fallen.«

\bibleverse{15} Da nahm Petrus das Wort und sagte zu ihm: »Erkläre uns
das Gleichnis (von vorhin)!« \bibleverse{16} Da antwortete er: »Seid
auch ihr immer noch ohne Verständnis? \bibleverse{17} Begreift ihr
nicht, daß alles, was in den Mund hineingeht, in den Leib✲ gelangt und
auf dem natürlichen Wege wieder ausgeschieden wird? \bibleverse{18} Was
dagegen aus dem Munde herauskommt, geht aus dem Herzen hervor, und das
ist es, was den Menschen verunreinigt. \bibleverse{19} Denn aus dem
Herzen kommen böse Gedanken hervor: Mordtaten, Ehebruch, Unzucht,
Diebstahl, Verleumdungen und Lästerungen. \bibleverse{20} Das sind die
Dinge, die den Menschen verunreinigen; dagegen das Essen mit
ungewaschenen Händen macht den Menschen nicht unrein.«

\hypertarget{jesus-und-die-kanaanuxe4ische-frau-im-gebiet-von-tyrus-und-sidon}{%
\subsubsection{5. Jesus und die kanaanäische Frau im Gebiet von Tyrus
und
Sidon}\label{jesus-und-die-kanaanuxe4ische-frau-im-gebiet-von-tyrus-und-sidon}}

\bibleverse{21} Jesus ging dann von dort weg und zog sich in die Gegend
von Tyrus und Sidon zurück. \bibleverse{22} Da kam eine kanaanäische
Frau aus jenem Gebiet her und rief ihn laut an: »Erbarme dich meiner,
Herr, du Sohn Davids! Meine Tochter wird von einem bösen Geist schlimm
geplagt!« \bibleverse{23} Er antwortete ihr aber kein Wort. Da traten
seine Jünger zu ihm und baten ihn: »Fertige sie doch ab! Sie schreit ja
hinter uns her!« \bibleverse{24} Er aber antwortete: »Ich bin nur zu den
verlorenen Schafen des Hauses Israel\textless sup title=``Mt
10,6''\textgreater✲ gesandt.« \bibleverse{25} Sie aber kam, warf sich
vor ihm nieder und bat: »Herr, hilf mir!« \bibleverse{26} Doch er
erwiderte: »Es ist nicht recht, den Kindern das Brot zu nehmen und es
den Hündlein hinzuwerfen.« \bibleverse{27} Darauf sagte sie: »O doch,
Herr! Die Hündlein bekommen ja auch von den Brocken zu essen, die vom
Tisch ihrer Herren fallen.« \bibleverse{28} Da antwortete ihr Jesus: »O
Frau, dein Glaube ist groß; dir geschehe, wie du es wünschest!« Und ihre
Tochter wurde von dieser Stunde an gesund.

\hypertarget{heiltuxe4tigkeit-jesu-in-galiluxe4a-am-ostufer-des-sees-speisung-der-viertausend}{%
\subsubsection{6. Heiltätigkeit Jesu in Galiläa am Ostufer des Sees;
Speisung der
Viertausend}\label{heiltuxe4tigkeit-jesu-in-galiluxe4a-am-ostufer-des-sees-speisung-der-viertausend}}

\bibleverse{29} Jesus ging dann von dort wieder weg und kam an den
Galiläischen See, und als er den Berg hinangestiegen war, setzte er sich
dort nieder. \bibleverse{30} Da kamen große Scharen Volks zu ihm; sie
brachten Lahme, Blinde, Krüppel, Stumme und viele andere Kranke mit
sich, die sie ihm vor die Füße legten; und er heilte sie,
\bibleverse{31} so daß die Volksmenge sich verwunderte, als sie sah, daß
Stumme redeten, Krüppel gesund wurden, Lahme einhergehen konnten und
Blinde sehend wurden; und sie priesen den Gott Israels.

\bibleverse{32} Jesus aber rief seine Jünger zu sich und sagte: »Mich
jammert des Volks, denn sie halten nun schon drei Tage bei mir aus, ohne
daß sie etwas zu essen haben, und ich mag sie nicht von mir lassen, ehe
sie gegessen haben: sie würden sonst unterwegs verschmachten.«
\bibleverse{33} Da erwiderten ihm die Jünger: »Woher sollen wir hier in
der Einöde so viele Brote nehmen, daß wir eine solche Volksmenge
sättigen könnten?« \bibleverse{34} Doch Jesus fragte sie: »Wie viele
Brote habt ihr?« Sie antworteten: »Sieben und ein paar kleine Fische.«
\bibleverse{35} Da gebot er dem Volke, sich auf dem Erdboden zu lagern,
\bibleverse{36} nahm dann die sieben Brote und die Fische, sprach den
Lobpreis (Gottes), brach die Brote und gab sie\textless sup title=``d.h.
die Brotstücke''\textgreater✲ seinen Jüngern, die Jünger aber teilten
sie an die Volksmenge aus. \bibleverse{37} Und sie aßen alle und wurden
satt; dann hob man die übriggebliebenen Brote (vom Boden) auf: sieben
Körbe voll; \bibleverse{38} die Zahl derer aber, die gegessen hatten,
betrug etwa viertausend Männer, ungerechnet die Frauen und Kinder.
\bibleverse{39} Er ließ dann die Volksmenge gehen, stieg ins Boot und
kam in die Gegend von Magadan.

\hypertarget{abweisung-der-zeichenforderung-der-gegner-und-warnung-vor-dem-sauerteig-der-pharisuxe4er}{%
\subsubsection{7. Abweisung der Zeichenforderung der Gegner und Warnung
vor dem Sauerteig der
Pharisäer}\label{abweisung-der-zeichenforderung-der-gegner-und-warnung-vor-dem-sauerteig-der-pharisuxe4er}}

\hypertarget{section-15}{%
\section{16}\label{section-15}}

\bibleverse{1} Da traten die Pharisäer und Sadduzäer zu ihm heran, um
ihn auf die Probe zu stellen, und sprachen den Wunsch gegen ihn aus, er
möchte sie ein Wunderzeichen vom Himmel her sehen lassen. \bibleverse{2}
Er aber antwortete ihnen: »Am Abend sagt ihr: ›Es gibt schönes Wetter,
denn der Himmel ist rot‹; \bibleverse{3} und frühmorgens: ›Heute gibt es
Regenwetter, denn der Himmel ist rot und trübe.‹ Das Aussehen des
Himmels versteht ihr zu beurteilen, die Wahrzeichen der Zeit aber nicht.
\bibleverse{4} Ein böses und ehebrecherisches\textless sup title=``=~von
Gott abtrünniges''\textgreater✲ Geschlecht verlangt ein Zeichen; doch es
wird ihm kein Zeichen gegeben werden als nur das Zeichen des (Propheten)
Jona.« Mit diesen Worten ließ er sie stehen und ging weg.

\bibleverse{5} Als die Jünger dann an das jenseitige Ufer des Sees
kamen, hatten sie vergessen, Brote mitzunehmen. \bibleverse{6} Da sagte
Jesus zu ihnen: »Gebt acht und hütet euch vor dem Sauerteig der
Pharisäer und Sadduzäer!« \bibleverse{7} Sie erwogen nun im Gespräch
untereinander: »(Das sagt er deshalb,) weil wir keine Brote mitgenommen
haben.« \bibleverse{8} Als Jesus das merkte, sagte er: »Ihr
Kleingläubigen! Was macht ihr euch Gedanken darüber, daß ihr keine Brote
(mitgenommen) habt? \bibleverse{9} Besitzt ihr immer noch kein
Verständnis, und denkt ihr nicht an die fünf Brote für die Fünftausend
und wie viele Körbe voll ihr noch gesammelt habt? \bibleverse{10} Auch
nicht an die sieben Brote für die Viertausend und wie viele Körbchen
voll ihr noch aufgelesen habt? \bibleverse{11} Wie könnt ihr nur nicht
begreifen, daß ich nicht von Broten zu euch geredet habe! Hütet euch
aber vor dem Sauerteig der Pharisäer und Sadduzäer!« \bibleverse{12} Nun
verstanden sie, daß er nicht hatte sagen wollen, sie sollten sich vor
dem bei Broten verwendeten Sauerteig hüten, sondern vor der Lehre der
Pharisäer und Sadduzäer.

\hypertarget{vi.-vorbereitung-der-juxfcnger-auf-jesu-leiden-aufbruch-nach-jerusalem-1613-2028}{%
\subsection{VI. Vorbereitung der Jünger auf Jesu Leiden; Aufbruch nach
Jerusalem
(16,13-20,28)}\label{vi.-vorbereitung-der-juxfcnger-auf-jesu-leiden-aufbruch-nach-jerusalem-1613-2028}}

\hypertarget{das-messiasbekenntnis-des-petrus-bei-cuxe4sarea-philippi-berufung-des-petrus-zum-gruxfcnder-und-leiter-der-gemeinde}{%
\subsubsection{1. Das Messiasbekenntnis des Petrus bei Cäsarea Philippi;
Berufung des Petrus zum Gründer und Leiter der
Gemeinde}\label{das-messiasbekenntnis-des-petrus-bei-cuxe4sarea-philippi-berufung-des-petrus-zum-gruxfcnder-und-leiter-der-gemeinde}}

\bibleverse{13} Als Jesus dann in die Gegend von Cäsarea Philippi
gekommen war, fragte er seine Jünger: »Für wen halten die Leute den
Menschensohn?« \bibleverse{14} Sie antworteten: »Die einen für Johannes
den Täufer, andere für Elia, noch andere für Jeremia oder sonst einen
von den Propheten.« \bibleverse{15} Da fragte er sie weiter: »Ihr aber
-- für wen haltet ihr mich?« \bibleverse{16} Simon Petrus gab ihm zur
Antwort: »Du bist Christus\textless sup title=``=~der Messias; vgl.
1,16''\textgreater✲, der Sohn des lebendigen Gottes!« \bibleverse{17} Da
gab Jesus ihm zur Antwort: »Selig bist du (zu preisen), Simon, Sohn des
Jona, denn nicht Fleisch und Blut haben dir das geoffenbart, sondern
mein Vater droben im Himmel. \bibleverse{18} Und nun sage auch ich dir:
Du bist Petrus (Fels, d.h. Felsenmann), und auf diesem Felsen will ich
meine Gemeinde✲ erbauen, und die Pforten des Totenreiches sollen sie
nicht überwältigen. \bibleverse{19} Ich will dir die Schlüssel des
Himmelreiches geben, und was du auf der Erde bindest, das soll auch im
Himmel gebunden sein, und was du auf der Erde lösest, das soll auch im
Himmel gelöst sein!« \bibleverse{20} Hierauf gab er den Jüngern die
strenge Weisung, sie sollten es niemand sagen, daß er
Christus\textless sup title=``=~der Messias''\textgreater✲ sei.

\hypertarget{erste-leidensankuxfcndigung}{%
\subsubsection{2. Erste
Leidensankündigung}\label{erste-leidensankuxfcndigung}}

\bibleverse{21} Von da an begann Jesus seine Jünger darauf hinzuweisen,
daß er nach Jerusalem gehen und von den Ältesten und Hohenpriestern und
Schriftgelehrten vieles leiden müsse, und daß er getötet und am dritten
Tage auferweckt werden müsse. \bibleverse{22} Da nahm Petrus ihn
beiseite und begann auf ihn einzureden mit den Worten: »Herr, das
verhüte Gott! Nimmermehr darf dir das widerfahren!« \bibleverse{23} Er
aber wandte sich um und sagte zu Petrus: »Mir aus den Augen, Satan!
(Tritt) hinter mich! Ein Fallstrick\textless sup title=``oder: Anstoß,
Ärgernis''\textgreater✲ bist du für mich, denn deine Gedanken sind nicht
auf Gott, sondern auf die Menschen gerichtet.«

\hypertarget{spruxfcche-uxfcber-die-leidensnachfolge-der-juxfcnger}{%
\subsubsection{3. Sprüche über die Leidensnachfolge der
Jünger}\label{spruxfcche-uxfcber-die-leidensnachfolge-der-juxfcnger}}

\bibleverse{24} Damals sagte Jesus zu seinen Jüngern: »Will jemand mein
Nachfolger sein, so verleugne er sich selbst und nehme sein Kreuz auf
sich: dann kann er mein Nachfolger sein. \bibleverse{25} Denn wer sein
Leben retten will, der wird es verlieren; wer aber sein Leben um
meinetwillen verliert, der wird es finden\textless sup title=``oder:
gewinnen''\textgreater✲. \bibleverse{26} Denn was könnte es einem
Menschen helfen, wenn er die ganze Welt gewönne, aber sein
Leben\textless sup title=``oder: seine Seele''\textgreater✲ einbüßte?
Oder was könnte ein Mensch als Gegenwert\textless sup
title=``=~Kaufpreis oder Lösegeld''\textgreater✲ für sein
Leben\textless sup title=``oder: seine Seele''\textgreater✲ geben?
\bibleverse{27} Denn der Menschensohn wird in der Herrlichkeit seines
Vaters mit seinen Engeln kommen und dann einem jeden nach seinem Tun
vergelten. \bibleverse{28} Wahrlich ich sage euch: Einige von denen, die
hier stehen, werden den Tod nicht schmecken, bis sie den Menschensohn in
seiner Königsherrschaft haben kommen sehen.«

\hypertarget{jesu-verkluxe4rung-auf-dem-berge-und-sein-gespruxe4ch-mit-den-juxfcngern-beim-abstieg-die-eliafrage}{%
\subsubsection{4. Jesu Verklärung auf dem Berge und sein Gespräch mit
den Jüngern beim Abstieg (die
Eliafrage)}\label{jesu-verkluxe4rung-auf-dem-berge-und-sein-gespruxe4ch-mit-den-juxfcngern-beim-abstieg-die-eliafrage}}

\hypertarget{section-16}{%
\section{17}\label{section-16}}

\bibleverse{1} Sechs Tage später nahm Jesus den Petrus, Jakobus und
dessen Bruder Johannes mit sich und führte sie abseits\textless sup
title=``oder: in die Einsamkeit''\textgreater✲ auf einen hohen Berg.
\bibleverse{2} Da wurde er vor ihren Augen verwandelt: sein Antlitz
leuchtete wie die Sonne, und seine Kleider wurden hellglänzend wie das
Licht. \bibleverse{3} Und siehe, es erschienen ihnen Mose und Elia und
besprachen sich mit ihm. \bibleverse{4} Da nahm Petrus das Wort und
sagte zu Jesus: »Herr, hier sind wir gut aufgehoben! Willst du, so werde
ich hier drei Hütten bauen, eine für dich, eine für Mose und eine für
Elia.« \bibleverse{5} Während er noch redete, überschattete sie
plötzlich eine lichte Wolke, und eine Stimme erscholl aus der Wolke, die
sprach: »Dies ist mein geliebter Sohn, an dem ich Wohlgefallen gefunden
habe\textless sup title=``vgl. 3,17''\textgreater✲: höret auf ihn!«
\bibleverse{6} Als die Jünger das vernahmen, warfen sie sich auf ihr
Angesicht nieder und gerieten in große Furcht; \bibleverse{7} doch Jesus
trat herzu, faßte sie an und sagte: »Steht auf und fürchtet euch nicht!«
\bibleverse{8} Als sie aber ihre Augen aufschlugen, sahen sie niemand
mehr als Jesus allein.

\bibleverse{9} Als sie dann von dem Berge hinabstiegen, gebot ihnen
Jesus: »Erzählt niemand etwas von der Erscheinung, die ihr gesehen habt,
bis der Menschensohn von den Toten auferweckt worden ist.«
\bibleverse{10} Da fragten ihn die Jünger: »Wie können denn die
Schriftgelehrten behaupten, Elia müsse zuerst kommen?« \bibleverse{11}
Er gab ihnen zur Antwort: »Elia kommt allerdings und wird alles wieder
in den rechten Stand bringen\textless sup title=``Mal
3,23''\textgreater✲. \bibleverse{12} Ich sage euch aber: Elia ist
bereits gekommen, doch sie haben ihn nicht erkannt, sondern sind mit ihm
verfahren, wie es ihnen beliebte. Ebenso wird auch der Menschensohn
durch sie zu leiden haben.« \bibleverse{13} Da verstanden die Jünger,
daß er von Johannes dem Täufer zu ihnen gesprochen hatte.

\hypertarget{heilung-eines-fallsuxfcchtigen-knaben-belehrung-uxfcber-den-miuxdferfolg-der-juxfcnger}{%
\subsubsection{5. Heilung eines fallsüchtigen Knaben; Belehrung über den
Mißerfolg der
Jünger}\label{heilung-eines-fallsuxfcchtigen-knaben-belehrung-uxfcber-den-miuxdferfolg-der-juxfcnger}}

\bibleverse{14} Als sie dann zu der Volksmenge zurückgekommen waren,
trat ein Mann an ihn heran, warf sich vor ihm auf die Knie nieder
\bibleverse{15} und sagte: »Herr, erbarme dich meines Sohnes! Er ist
fallsüchtig und hat schwer zu leiden; denn oft fällt er ins Feuer und
oft auch ins Wasser. \bibleverse{16} Ich habe ihn schon zu deinen
Jüngern gebracht, doch sie haben ihn nicht heilen können.«
\bibleverse{17} Da antwortete Jesus: »O ihr ungläubige und verkehrte Art
von Menschen! Wie lange soll ich noch bei euch sein, wie lange es noch
mit euch aushalten? Bringt ihn mir hierher!« \bibleverse{18} Jesus
bedrohte alsdann den bösen Geist: da fuhr er von dem Knaben aus, so daß
dieser von Stund an gesund war. \bibleverse{19} Hierauf traten die
Jünger zu Jesus, als sie mit ihm allein waren, und fragten: »Warum haben
wir den Geist nicht austreiben können?« \bibleverse{20} Er antwortete
ihnen: »Wegen eures Kleinglaubens! Denn wahrlich ich sage euch: Wenn ihr
Glauben wie ein Senfkorn habt und diesem Berge gebietet: ›Rücke von hier
weg dorthin!‹, so wird er hinwegrücken, und nichts wird euch unmöglich
sein. \bibleverse{21} {[}Diese Art (von bösen Geistern) aber läßt sich
nur durch Gebet und Fasten austreiben.{]}«

\hypertarget{zweite-leidensankuxfcndigung-in-galiluxe4a}{%
\subsubsection{6. Zweite Leidensankündigung in
Galiläa}\label{zweite-leidensankuxfcndigung-in-galiluxe4a}}

\bibleverse{22} Während sie dann in Galiläa umherwanderten, sagte Jesus
zu ihnen: »Der Menschensohn wird in die Hände der Menschen überliefert
werden; \bibleverse{23} sie werden ihn töten, und am dritten Tage wird
er auferweckt werden.« Da wurden sie tief betrübt.

\hypertarget{die-tempelsteuer-und-ihre-wunderbare-entrichtung-in-kapernaum}{%
\subsubsection{7. Die Tempelsteuer und ihre wunderbare Entrichtung in
Kapernaum}\label{die-tempelsteuer-und-ihre-wunderbare-entrichtung-in-kapernaum}}

\bibleverse{24} Als sie dann nach Kapernaum zurückgekehrt waren, traten
die Einsammler der Tempelsteuer an Petrus heran und fragten ihn:
»Entrichtet euer Meister die Doppeldrachme✲ nicht?« \bibleverse{25} Er
antwortete: »Doch!« Als er dann ins Haus trat, kam ihm Jesus mit der
Frage zuvor: »Was meinst du, Simon? Von wem lassen sich die Könige der
Erde Zölle oder Steuern zahlen: von ihren Söhnen✲ oder von Fremden?«
\bibleverse{26} Als jener nun antwortete: »Von den Fremden«, sagte Jesus
zu ihm: »So sind die Söhne steuerfrei. \bibleverse{27} Damit wir aber
keinen Anstoß bei ihnen erregen, so geh an den See, wirf die Angel aus,
und den ersten Fisch, den du heraufziehst, den nimm und öffne ihm das
Maul; dann wirst du ein Silberstück finden; das nimm und gib es ihnen
(als Abgabe) für mich und dich.«

\hypertarget{gespruxe4che-mit-den-juxfcngern-weisungen-fuxfcr-den-juxfcngerkreis}{%
\subsubsection{8. Gespräche mit den Jüngern; Weisungen für den
Jüngerkreis}\label{gespruxe4che-mit-den-juxfcngern-weisungen-fuxfcr-den-juxfcngerkreis}}

\hypertarget{a-rangstreit-der-juxfcnger-jesu-mahnung-zur-demut}{%
\paragraph{a) Rangstreit der Jünger; Jesu Mahnung zur
Demut}\label{a-rangstreit-der-juxfcnger-jesu-mahnung-zur-demut}}

\hypertarget{section-17}{%
\section{18}\label{section-17}}

\bibleverse{1} In jener Stunde traten die Jünger zu Jesus mit der Frage:
»Wer ist denn der Größte im Himmelreich?« \bibleverse{2} Da rief er ein
Kind herbei, stellte es mitten unter sie \bibleverse{3} und sagte:
»Wahrlich ich sage euch: Wenn ihr nicht umkehrt und wie die Kinder
werdet, so werdet ihr nimmermehr ins Himmelreich eingehen.
\bibleverse{4} Wer sich demnach so erniedrigt\textless sup
title=``=~demütig unter andere stellt''\textgreater✲ wie dieses Kind
hier, der ist der Größte im Himmelreich; \bibleverse{5} und wer ein
einziges solches Kind auf meinen Namen hin\textless sup title=``oder: um
meines Namens willen''\textgreater✲ aufnimmt, der nimmt mich auf.«

\hypertarget{b-jesu-sorge-fuxfcr-die-kleinen-und-schwachen-warnung-vor-verfuxfchrern-zum-buxf6sen}{%
\paragraph{b) Jesu Sorge für die Kleinen und Schwachen; Warnung vor
Verführern zum
Bösen}\label{b-jesu-sorge-fuxfcr-die-kleinen-und-schwachen-warnung-vor-verfuxfchrern-zum-buxf6sen}}

\bibleverse{6} »Wer aber einen von diesen Kleinen\textless sup
title=``oder: geringen Leuten''\textgreater✲, die an mich glauben,
ärgert\textless sup title=``oder: zum Bösen verführt''\textgreater✲, für
den wäre es das beste, daß ihm ein Mühlstein um den Hals gehängt und er
ins Meer versenkt würde, wo es am tiefsten ist. \bibleverse{7} Wehe der
Welt um der Ärgernisse\textless sup title=``oder:
Verführungen''\textgreater✲ willen! Wohl müssen die Verführungen kommen;
doch wehe dem Menschen, durch den das Ärgernis\textless sup
title=``oder: die Verführung''\textgreater✲ kommt! \bibleverse{8} Wenn
nun deine Hand oder dein Fuß dich ärgert\textless sup title=``oder: zum
Bösen verführen will''\textgreater✲, so haue sie ab und wirf sie von
dir! Es ist besser für dich, verstümmelt oder lahm ins Leben einzugehen,
als daß du beide Hände oder beide Füße hast und in das ewige Feuer
geworfen wirst. \bibleverse{9} Und wenn dein Auge dich
ärgert\textless sup title=``oder: zum Bösen verführen
will''\textgreater✲, so reiße es aus und wirf es von dir! Es ist besser
für dich, einäugig ins Leben einzugehen, als daß du beide Augen hast und
ins Feuer der Hölle geworfen wirst.

\bibleverse{10} Sehet zu, daß ihr keinen von diesen Kleinen
geringschätzt! Denn ich sage euch: Ihre Engel✲ im Himmel schauen
allezeit das Angesicht meines himmlischen Vaters. \bibleverse{11}
{[}Denn der Menschensohn ist gekommen, das Verlorene zu retten.{]}«

\hypertarget{c-das-gleichnis-vom-verlorenen-schaf}{%
\paragraph{c) Das Gleichnis vom verlorenen
Schaf}\label{c-das-gleichnis-vom-verlorenen-schaf}}

\bibleverse{12} »Was meint ihr wohl? Wenn jemand hundert Schafe besitzt
und eins von ihnen sich verirrt: wird er da nicht die neunundneunzig auf
den Bergen zurücklassen und hingehen, um das verirrte zu suchen?
\bibleverse{13} Und wenn es ihm gelingt, es zu finden, wahrlich ich sage
euch: Er freut sich über dieses (eine) mehr als über die neunundneunzig,
die sich nicht verirrt hatten. \bibleverse{14} Ebenso ist es auch der
Wille eures himmlischen Vaters, daß keiner von diesen Kleinen
verlorengehen soll.«

\hypertarget{d-vom-verhalten-gegen-den-suxfcndigenden-bruder-uxfcber-die-wirkung-des-urteils-und-des-gebets-der-gemeinde}{%
\paragraph{d) Vom Verhalten gegen den sündigenden Bruder; über die
Wirkung des Urteils und des Gebets der
Gemeinde}\label{d-vom-verhalten-gegen-den-suxfcndigenden-bruder-uxfcber-die-wirkung-des-urteils-und-des-gebets-der-gemeinde}}

\bibleverse{15} »Wenn dein Bruder sich verfehlt, so gehe hin und halte
es ihm unter vier Augen vor. Hört er auf dich, so hast du deinen Bruder
gewonnen; \bibleverse{16} hört er aber nicht, so nimm noch einen oder
zwei (Brüder) mit dir, damit jede Sache\textless sup title=``oder: der
ganze Sachverhalt''\textgreater✲ auf Grund der Aussagen von zwei oder
drei Zeugen festgestellt wird\textless sup title=``5.Mose
19,15''\textgreater✲. \bibleverse{17} Will er auf diese (Brüder) nicht
hören, so teile es der Gemeinde✲ mit; will er auch auf die Gemeinde
nicht hören, so gelte er dir wie ein Heide und ein Zöllner.~--
\bibleverse{18} Wahrlich ich sage euch: Alles, was ihr auf der Erde
bindet\textless sup title=``vgl. 16,19''\textgreater✲, wird auch im
Himmel gebunden sein; und was ihr auf der Erde löst, wird auch im Himmel
gelöst sein.~-- \bibleverse{19} Weiter sage ich euch: Wenn zwei von euch
auf Erden eins werden, um irgend etwas zu bitten, so wird es ihnen von
meinem himmlischen Vater zuteil werden; \bibleverse{20} denn wo zwei
oder drei auf meinen Namen hin\textless sup title=``oder: in meinem
Namen''\textgreater✲ versammelt sind, da bin ich mitten unter ihnen.«

\hypertarget{e-von-der-versuxf6hnlichkeit-und-der-vergebung-das-gleichnis-vom-schalksknecht}{%
\paragraph{e) Von der Versöhnlichkeit und der Vergebung; das Gleichnis
vom
Schalksknecht}\label{e-von-der-versuxf6hnlichkeit-und-der-vergebung-das-gleichnis-vom-schalksknecht}}

\bibleverse{21} Hierauf trat Petrus an ihn heran und fragte ihn: »Herr,
wie oft muß ich meinem Bruder vergeben, wenn er sich gegen mich vergeht?
Bis zu siebenmal?« \bibleverse{22} Da antwortete ihm Jesus: »Ich sage
dir: Nicht bis zu siebenmal, sondern bis siebenzigmal
siebenmal\textless sup title=``d.h. siebenundsiebzigmal; vgl. 1.Mose
4,24''\textgreater✲. \bibleverse{23} Darum ist das Himmelreich einem
Könige vergleichbar, der mit seinen Knechten\textless sup
title=``=~Dienern oder Beamten''\textgreater✲ abrechnen wollte.
\bibleverse{24} Als er nun mit der Abrechnung begann, wurde ihm einer
vorgeführt, der ihm zehntausend Talente schuldig war. \bibleverse{25}
Weil er nun diese Schuld nicht bezahlen konnte, befahl der Herr, man
solle ihn samt Weib und Kindern und seinem gesamten Besitz verkaufen und
so Ersatz\textless sup title=``oder: Bezahlung''\textgreater✲ schaffen.
\bibleverse{26} Da warf sich der Knecht vor ihm zur Erde nieder und bat
ihn mit den Worten: ›Habe Geduld mit mir: ich will dir alles bezahlen.‹
\bibleverse{27} Da hatte der Herr Erbarmen mit diesem Knecht; er gab ihn
frei, und die Schuld erließ er ihm auch. \bibleverse{28} Als aber dieser
Knecht (aus dem Hause des Herrn) hinausgegangen war, traf er einen
seiner Mitknechte, der ihm hundert Denare schuldig war; den ergriff er,
packte ihn an der Kehle und sagte zu ihm: ›Bezahle, wenn du etwas
schuldig bist!‹ \bibleverse{29} Da warf sich sein Mitknecht vor ihm
nieder und bat ihn mit den Worten: ›Habe Geduld mit mir: ich will dir's
bezahlen!‹ \bibleverse{30} Er wollte aber nicht, sondern ging hin und
ließ ihn ins Gefängnis werfen, bis er die Schuld bezahlt hätte.
\bibleverse{31} Als nun seine Mitknechte sahen, was da vorgegangen war,
wurden sie sehr ungehalten; sie gingen hin und berichteten ihrem Herrn
den ganzen Vorfall. \bibleverse{32} Da ließ sein Herr ihn vor sich rufen
und sagte zu ihm: ›Du böser✲ Knecht! Jene ganze Schuld habe ich dir
erlassen, weil du mich darum batest; \bibleverse{33} hättest du da nicht
auch Erbarmen mit deinem Mitknecht haben müssen, wie ich Erbarmen mit
dir gehabt habe?‹ \bibleverse{34} Und voller Zorn übergab sein Herr ihn
den Folterknechten, bis er ihm seine ganze Schuld bezahlt hätte.
\bibleverse{35} Ebenso wird auch mein himmlischer Vater mit euch
verfahren, wenn ihr nicht ein jeder seinem Bruder von Herzen vergebt.«

\hypertarget{aufbruch-nach-jerusalem-und-wanderung-durch-das-ostjordanland-gespruxe4che-uxfcber-die-ehe-uxfcber-ehescheidung-und-verzicht-auf-die-ehe}{%
\subsubsection{9. Aufbruch nach Jerusalem und Wanderung durch das
Ostjordanland; Gespräche über die Ehe, über Ehescheidung und Verzicht
auf die
Ehe}\label{aufbruch-nach-jerusalem-und-wanderung-durch-das-ostjordanland-gespruxe4che-uxfcber-die-ehe-uxfcber-ehescheidung-und-verzicht-auf-die-ehe}}

\hypertarget{section-18}{%
\section{19}\label{section-18}}

\bibleverse{1} Als Jesus nun diese Reden beendet hatte, brach er aus
Galiläa auf und kam in das Gebiet von Judäa auf der anderen Seite des
Jordans; \bibleverse{2} eine große Volksmenge begleitete ihn, und er
heilte sie dort.

\bibleverse{3} Da traten Pharisäer an ihn heran, die ihn auf die Probe
stellen wollten, und legten ihm die Frage vor: »Darf man seine Frau aus
jedem beliebigen Grunde entlassen\textless sup title=``oder: sich von
seiner Frau scheiden''\textgreater✲?« \bibleverse{4} Er gab ihnen zur
Antwort: »Habt ihr nicht gelesen\textless sup title=``1.Mose
1,27''\textgreater✲, daß der Schöpfer die Menschen von Anfang an als
Mann und Weib geschaffen \bibleverse{5} und gesagt hat\textless sup
title=``1.Mose 2,24''\textgreater✲: ›Darum wird ein Mann seinen Vater
und seine Mutter verlassen und an seinem Weibe hangen, und die beiden
werden ein Fleisch sein‹? \bibleverse{6} Also sind sie nicht mehr zwei,
sondern ein Fleisch. Was somit Gott zusammengefügt hat, das soll der
Mensch nicht scheiden.« \bibleverse{7} Sie entgegneten ihm: »Warum hat
denn Mose geboten\textless sup title=``5.Mose 24,1''\textgreater✲, der
Frau einen Scheidebrief auszustellen und sie dann zu entlassen?«
\bibleverse{8} Er antwortete ihnen: »Mose hat euch (nur) mit Rücksicht
auf eure Herzenshärte gestattet, eure Frauen zu entlassen\textless sup
title=``oder: euch von euren Frauen zu scheiden''\textgreater✲; aber von
Anfang an ist es nicht so gewesen. \bibleverse{9} Ich sage euch aber:
Wer sich von seiner Frau scheidet -- außer wegen Unzucht -- und eine
andere heiratet, der begeht Ehebruch {[}und wer eine
Entlassene\textless sup title=``oder: Geschiedene''\textgreater✲
heiratet, begeht auch Ehebruch{]}.« \bibleverse{10} Da sagten die Jünger
zu ihm: »Wenn es mit dem Rechtsverhältnis des Mannes gegenüber seiner
Frau so steht, dann ist es nicht geraten, sich zu verheiraten.«
\bibleverse{11} Er aber antwortete ihnen: »Nicht alle fassen dieses
Wort, sondern nur die, denen es\textless sup title=``=~das Verständnis
dafür''\textgreater✲ gegeben ist. \bibleverse{12} Es gibt nämlich zur
Ehe Untüchtige, die vom Mutterleibe her so geboren worden sind; und es
gibt zur Ehe Untüchtige, die von Menschenhand zur Ehe untüchtig gemacht
worden sind; und es gibt zur Ehe Untüchtige, die sich selbst um des
Himmelreichs willen untüchtig gemacht haben. Wer es zu fassen vermag,
der fasse es!«

\hypertarget{jesus-segnet-die-kinder}{%
\subsubsection{10. Jesus segnet die
Kinder}\label{jesus-segnet-die-kinder}}

\bibleverse{13} Hierauf brachte man kleine Kinder zu ihm, damit er ihnen
die Hände auflegen und (für sie) beten möchte; die Jünger aber verwiesen
es in barscher Weise (denen, die sie brachten). \bibleverse{14} Doch
Jesus sagte: »Laßt die Kinder (in Frieden) und hindert sie nicht, zu mir
zu kommen! Denn für ihresgleichen ist das Himmelreich
bestimmt.«\textless sup title=``vgl. Mk 10,14''\textgreater✲
\bibleverse{15} Dann legte er ihnen die Hände auf und wanderte von dort
weiter.

\hypertarget{jesu-gespruxe4ch-mit-dem-reichen-juxfcngling-die-gefahr-des-reichtums}{%
\subsubsection{11. Jesu Gespräch mit dem reichen Jüngling; die Gefahr
des
Reichtums}\label{jesu-gespruxe4ch-mit-dem-reichen-juxfcngling-die-gefahr-des-reichtums}}

\bibleverse{16} Da trat einer an ihn heran und fragte ihn: »Meister, was
muß ich Gutes tun, um ewiges Leben zu erlangen?« \bibleverse{17} Er
antwortete ihm: »Was fragst du mich über das Gute? (Nur) einer ist der
Gute. Willst du aber ins Leben eingehen, so halte die Gebote.«
\bibleverse{18} »Welche?« entgegnete er. Jesus antwortete: »Diese: ›Du
sollst nicht töten, nicht ehebrechen, nicht stehlen, nicht falsches
Zeugnis ablegen, \bibleverse{19} ehre deinen Vater und deine Mutter‹ und
›du sollst deinen Nächsten lieben wie dich selbst.‹« \bibleverse{20} Der
Jüngling erwiderte ihm: »Dies alles habe ich beobachtet: was fehlt mir
noch?« \bibleverse{21} Jesus antwortete ihm: »Willst du vollkommen sein,
so gehe hin, verkaufe dein Hab und Gut und gib (den Erlös) den Armen, so
wirst du einen Schatz im Himmel haben; dann komm und folge mir nach!«

\bibleverse{22} Als der Jüngling das Wort gehört hatte, ging er betrübt
weg; denn er besaß ein großes Vermögen. \bibleverse{23} Jesus aber sagte
zu seinen Jüngern: »Wahrlich ich sage euch: Für einen Reichen wird es
schwer sein, ins Himmelreich einzugehen. \bibleverse{24} Nochmals sage
ich euch: Es ist leichter, daß ein Kamel durch ein Nadelöhr
hindurchgeht, als daß ein Reicher in das Reich Gottes eingeht.«
\bibleverse{25} Als die Jünger das hörten, wurden sie ganz bestürzt und
sagten: »Ja, wer kann dann gerettet werden?« \bibleverse{26} Jesus aber
blickte sie an und sagte zu ihnen: »Bei den Menschen ist dies unmöglich,
aber bei Gott ist alles möglich.«\textless sup title=``1.Mose
18,14''\textgreater✲

\hypertarget{vom-lohn-der-nachfolge-jesu-und-der-entsagung}{%
\subsubsection{12. Vom Lohn der Nachfolge Jesu und der
Entsagung}\label{vom-lohn-der-nachfolge-jesu-und-der-entsagung}}

\bibleverse{27} Hierauf nahm Petrus das Wort und sagte zu ihm: »Siehe,
wir haben alles verlassen und sind dir nachgefolgt: welcher Lohn wird
uns also dafür zuteil werden?« \bibleverse{28} Jesus antwortete ihnen:
»Wahrlich ich sage euch: Ihr, die ihr mir nachgefolgt seid, werdet bei
der Wiedergeburt\textless sup title=``=~bei der Neugestaltung aller
Dinge''\textgreater✲, wenn der Menschensohn auf dem Thron seiner
Herrlichkeit sitzt, gleichfalls auf zwölf Thronen sitzen und die zwölf
Stämme Israels richten✲. \bibleverse{29} Und jeder, der um meines Namens
willen Brüder oder Schwestern, Vater oder Mutter, Weib oder Kinder,
Äcker oder Häuser verlassen hat, wird vielmal Wertvolleres empfangen und
ewiges Leben erben. \bibleverse{30} Viele Erste aber werden Letzte sein
und viele Letzte die Ersten.«\textless sup title=``Lk
13,30''\textgreater✲

\hypertarget{gleichnis-von-den-arbeitern-im-weinberge}{%
\subsubsection{13. Gleichnis von den Arbeitern im
Weinberge}\label{gleichnis-von-den-arbeitern-im-weinberge}}

\hypertarget{section-19}{%
\section{20}\label{section-19}}

\bibleverse{1} »Denn das Himmelreich ist einem menschlichen Hausherrn
gleich, der frühmorgens ausging, um Arbeiter für seinen Weinberg
einzustellen. \bibleverse{2} Nachdem er nun mit den Arbeitern einen
Tagelohn von einem Denar vereinbart hatte, schickte er sie in seinen
Weinberg. \bibleverse{3} Als er dann um die dritte Tagesstunde wieder
ausging, sah er andere auf dem Marktplatz unbeschäftigt stehen
\bibleverse{4} und sagte zu ihnen: ›Geht auch ihr in meinen Weinberg,
ich will euch geben, was recht ist‹; \bibleverse{5} und sie gingen hin.
Wiederum ging er um die sechste und um die neunte Stunde aus und machte
es ebenso; \bibleverse{6} und als er um die elfte Stunde wieder ausging,
fand er noch andere dastehen und sagte zu ihnen: ›Was steht ihr hier den
ganzen Tag müßig?‹ \bibleverse{7} Sie antworteten ihm: ›Niemand hat uns
in Arbeit genommen.‹ Da sagte er zu ihnen: ›Geht auch ihr noch in den
Weinberg!‹ \bibleverse{8} Als es dann Abend geworden war, sagte der Herr
des Weinbergs zu seinem Verwalter: ›Rufe die Arbeiter und zahle ihnen
den Lohn aus! Fange bei den letzten an (und weiter so) bis zu den
ersten!‹ \bibleverse{9} Als nun die um die elfte Stunde Eingestellten
kamen, erhielten sie jeder einen Denar. \bibleverse{10} Als dann die
Ersten (an die Reihe) kamen, dachten sie, sie würden mehr erhalten; doch
sie erhielten gleichfalls jeder nur einen Denar. \bibleverse{11} Als sie
ihn empfangen hatten, murrten sie gegen den Hausherrn \bibleverse{12}
und sagten: ›Diese Letzten haben nur eine einzige Stunde gearbeitet, und
du hast sie uns gleichgestellt, die wir des (ganzen) Tages Last und
Hitze getragen haben!‹ \bibleverse{13} Er aber entgegnete einem von
ihnen: ›Freund, ich tue dir nicht unrecht; bist du nicht um einen Denar
mit mir eins geworden? \bibleverse{14} Nimm dein Geld und gehe! Es
gefällt mir nun einmal, diesem Letzten ebensoviel zu geben wie dir.
\bibleverse{15} Habe ich etwa nicht das Recht, mit dem, was mein ist, zu
machen, was ich will? Oder siehst du neidisch dazu, daß ich wohlwollend
bin?‹ \bibleverse{16} Ebenso werden die Letzten Erste und die Ersten
Letzte sein. {[}Denn viele sind berufen, aber wenige auserwählt.{]}«

\hypertarget{aufbruch-nach-jerusalem-dritte-leidensankuxfcndigung-jesu}{%
\subsubsection{14. Aufbruch nach Jerusalem; dritte Leidensankündigung
Jesu}\label{aufbruch-nach-jerusalem-dritte-leidensankuxfcndigung-jesu}}

\bibleverse{17} Als nun Jesus vorhatte, nach Jerusalem hinaufzuziehen,
nahm er die zwölf Jünger (vom Volk) gesondert zu sich und sagte
unterwegs zu ihnen: \bibleverse{18} »Seht, wir ziehen jetzt nach
Jerusalem hinauf: dort wird der Menschensohn den Hohenpriestern und
Schriftgelehrten überantwortet werden; die werden ihn zum Tode
verurteilen \bibleverse{19} und ihn den Heiden zur Verspottung, zur
Geißelung und zur Kreuzigung überliefern; und am dritten Tage wird er
auferweckt werden.«

\hypertarget{ehrgeizige-bitte-der-salome-fuxfcr-ihre-suxf6hne-jakobus-und-johannes}{%
\subsubsection{15. Ehrgeizige Bitte der Salome für ihre Söhne Jakobus
und
Johannes}\label{ehrgeizige-bitte-der-salome-fuxfcr-ihre-suxf6hne-jakobus-und-johannes}}

\bibleverse{20} Damals trat die Mutter der Söhne des Zebedäus mit ihren
(beiden) Söhnen zu ihm, fiel vor ihm nieder und wollte ihn um etwas
bitten. \bibleverse{21} Er fragte sie: »Was wünschest du?« Sie
antwortete ihm: »Bestimme, daß diese meine beiden Söhne dereinst in
deinem Königreich einer zu deiner Rechten und einer zu deiner Linken
sitzen sollen.« \bibleverse{22} Da antwortete Jesus: »Ihr wißt nicht, um
was ihr bittet. Könnt ihr den Kelch trinken, den ich trinken werde?« Sie
antworteten ihm: »Ja, wir können es.« \bibleverse{23} Er erwiderte
ihnen: »Meinen Kelch werdet ihr zwar trinken (müssen), aber die Plätze
zu meiner Rechten und zu meiner Linken habe nicht ich zu verleihen,
sondern sie werden denen zuteil, für die sie von meinem Vater bestimmt
sind.«

\hypertarget{von-der-pflicht-zu-dienen-um-des-himmelreichs-willen}{%
\subsubsection{16. Von der Pflicht zu dienen um des Himmelreichs
willen}\label{von-der-pflicht-zu-dienen-um-des-himmelreichs-willen}}

\bibleverse{24} Als die (übrigen) zehn Jünger das hörten, wurden sie
über die beiden Brüder unwillig; \bibleverse{25} Jesus aber rief sie zu
sich und sagte: »Ihr wißt, daß die weltlichen Herrscher sich als Herren
gegen ihre Völker benehmen und daß ihre Großen sie vergewaltigen.
\bibleverse{26} Bei euch aber darf es nicht so sein; wer unter euch als
Großer dastehen möchte, der muß euer Diener sein, \bibleverse{27} und
wer bei euch der Erste sein möchte, der muß euer Knecht sein,
\bibleverse{28} wie ja auch der Menschensohn nicht gekommen ist, sich
bedienen zu lassen, sondern zu dienen und sein Leben als Lösegeld
hinzugeben für viele.«

\hypertarget{vii.-jesu-einzug-in-jerusalem-und-letztes-wirken-2029-2546}{%
\subsection{VII. Jesu Einzug in Jerusalem und letztes Wirken
(20,29-25,46)}\label{vii.-jesu-einzug-in-jerusalem-und-letztes-wirken-2029-2546}}

\hypertarget{heilung-zweier-blinder-bei-jericho}{%
\subsubsection{1. Heilung zweier Blinder bei
Jericho}\label{heilung-zweier-blinder-bei-jericho}}

\bibleverse{29} Als sie dann aus Jericho hinauszogen, folgte ihm eine
große Volksmenge nach. \bibleverse{30} Da saßen dort zwei Blinde am
Wege; als diese hörten, daß Jesus vorüberziehe, riefen sie laut: »Herr,
erbarme dich unser, Sohn Davids!« \bibleverse{31} Die Volksmenge rief
ihnen drohend zu, sie sollten still sein; sie aber schrien nur noch
lauter: »Herr, erbarme dich unser, Sohn Davids!« \bibleverse{32} Da
blieb Jesus stehen, rief sie herbei und fragte sie: »Was wünscht ihr von
mir?« \bibleverse{33} Sie antworteten ihm: »Herr, daß unsere Augen
aufgetan werden!« \bibleverse{34} Da fühlte Jesus Mitleid mit ihnen; er
berührte ihre Augen, und sogleich konnten sie sehen und schlossen sich
ihm an.

\hypertarget{einzug-jesu-in-jerusalem}{%
\subsubsection{2. Einzug Jesu in
Jerusalem}\label{einzug-jesu-in-jerusalem}}

\hypertarget{section-20}{%
\section{21}\label{section-20}}

\bibleverse{1} Als sie sich dann Jerusalem näherten und nach Bethphage
an den Ölberg gekommen waren, da sandte Jesus zwei von seinen Jüngern ab
\bibleverse{2} mit der Weisung: »Geht in das Dorf, das vor euch liegt!
Ihr werdet dort sogleich (am Eingang) eine Eselin angebunden finden und
ein Füllen bei ihr; bindet sie los und bringt sie mir her!
\bibleverse{3} Und wenn euch jemand etwas sagen sollte, so antwortet
ihm: ›Der Herr hat sie nötig, wird sie aber sofort zurückschicken.‹«
\bibleverse{4} Dies ist aber geschehen, damit das Wort des Propheten
erfüllt werde, das da lautet\textless sup title=``Jes 62,11; Sach
9,9''\textgreater✲: \bibleverse{5} »Sagt der Tochter Zion: Siehe, dein
König kommt zu dir sanftmütig und auf einem Esel reitend, und zwar auf
einem Füllen, dem Jungen des Lasttiers.«

\bibleverse{6} Als nun die Jünger hingegangen waren und den Auftrag Jesu
ausgerichtet hatten, \bibleverse{7} führten sie die Eselin mit dem
Füllen herbei, legten ihre Mäntel auf sie, und er setzte sich darauf.
\bibleverse{8} Die überaus zahlreiche Volksmenge aber breitete ihre
Mäntel auf den Weg aus, andere hieben Zweige von den Bäumen ab und
streuten sie auf den Weg; \bibleverse{9} und die Scharen, die im Zuge
vor ihm her gingen und die, welche ihm nachfolgten, riefen laut:
»Hosianna dem Sohne Davids! Gepriesen\textless sup title=``oder:
gesegnet''\textgreater✲ sei, der da kommt im Namen des Herrn! Hosianna
in den Himmelshöhen!« \bibleverse{10} Als er dann in Jerusalem
eingezogen war, geriet die ganze Stadt in Bewegung, und zwar fragte man:
»Wer ist dieser?« \bibleverse{11} Da sagte die Volksmenge: »Dies ist der
Prophet Jesus aus Nazareth in Galiläa!«

\hypertarget{jesus-im-tempel}{%
\subsubsection{3. Jesus im Tempel}\label{jesus-im-tempel}}

\hypertarget{a-die-tempelreinigung}{%
\paragraph{a) Die Tempelreinigung}\label{a-die-tempelreinigung}}

\bibleverse{12} Jesus ging dann in den Tempel Gottes, trieb alle hinaus,
die im Tempel verkauften und kauften, warf die Tische der Geldwechsler
und die Sitze der Taubenverkäufer um \bibleverse{13} und sagte zu ihnen:
»Es steht geschrieben\textless sup title=``Jes 56,7''\textgreater✲:
›Mein Haus soll ein Bethaus heißen!‹ Ihr aber macht es zu einer
›Räuberhöhle‹!«\textless sup title=``Jer 7,11''\textgreater✲

\hypertarget{b-heilungen-im-tempel-und-huldigung-der-kinder}{%
\paragraph{b) Heilungen im Tempel und Huldigung der
Kinder}\label{b-heilungen-im-tempel-und-huldigung-der-kinder}}

\bibleverse{14} Es kamen auch Blinde und Lahme im Tempel zu ihm, und er
heilte sie. \bibleverse{15} Als aber die Hohenpriester und
Schriftgelehrten die Wunder sahen, die er tat, und (hörten) wie die
Kinder im Tempel laut riefen: »Hosianna dem Sohne Davids!«, wurden sie
unwillig \bibleverse{16} und sagten zu ihm: »Hörst du nicht, was diese
hier rufen?« Da antwortete Jesus ihnen: »Jawohl! Habt ihr noch niemals
(das Schriftwort) gelesen\textless sup title=``Ps 8,3''\textgreater✲:
›Aus dem Munde von Unmündigen und Säuglingen hast du (dir) Lobpreis
bereitet‹?« \bibleverse{17} Mit diesen Worten ließ er sie stehen, ging
aus der Stadt hinaus nach Bethanien und blieb über Nacht dort.

\hypertarget{die-verfluchung-des-unfruchtbaren-feigenbaums}{%
\subsubsection{4. Die Verfluchung des unfruchtbaren
Feigenbaums}\label{die-verfluchung-des-unfruchtbaren-feigenbaums}}

\bibleverse{18} Als er dann frühmorgens in die Stadt zurückkehrte,
hungerte ihn, \bibleverse{19} und als er einen einzelnen Feigenbaum am
Wege stehen sah, ging er zu ihm hin, fand aber nichts anderes an ihm als
Blätter. Da sagte er zu ihm: »Nie mehr soll noch Frucht von dir kommen
in Ewigkeit!«, und der Feigenbaum verdorrte sofort. \bibleverse{20} Als
die Jünger das sahen, verwunderten sie sich und sagten: »Wie kommt es,
daß der Feigenbaum sofort verdorrt ist?« \bibleverse{21} Da antwortete
ihnen Jesus: »Wahrlich ich sage euch: Wenn ihr Glauben habt und keinen
Zweifel hegt, so werdet ihr nicht nur das, was hier mit dem Feigenbaume
geschehen ist, tun können, sondern auch, wenn ihr zu dem Berge hier
sagtet: ›Hebe dich empor und stürze dich ins Meer!‹, so würde es
geschehen; \bibleverse{22} und alles, um was ihr im Gebet bittet, werdet
ihr empfangen, wenn ihr Glauben habt.«

\hypertarget{die-frage-des-hohen-rates-nach-jesu-vollmacht}{%
\subsubsection{5. Die Frage des Hohen Rates nach Jesu
Vollmacht}\label{die-frage-des-hohen-rates-nach-jesu-vollmacht}}

\bibleverse{23} Als er dann in den Tempel gekommen war, traten die
Hohenpriester und die Ältesten des Volkes zu ihm, während er lehrte, und
fragten ihn: »Auf Grund welcher Vollmacht trittst du in dieser Weise
hier auf, und wer hat dir die Vollmacht dazu gegeben?« \bibleverse{24}
Jesus gab ihnen zur Antwort: »Auch ich will euch eine einzige Frage
vorlegen; wenn ihr sie mir beantwortet, so will auch ich euch sagen, auf
Grund welcher Vollmacht ich hier so auftrete: \bibleverse{25} Woher
stammte die Taufe des Johannes? Vom Himmel oder von den Menschen?« Da
überlegten sie bei sich folgendermaßen: \bibleverse{26} »Sagen wir: ›Vom
Himmel‹, so wird er uns vorhalten: ›Warum habt ihr ihm dann keinen
Glauben geschenkt?‹ Sagen wir aber: ›Von den Menschen‹, so haben wir das
Volk zu fürchten; denn alle halten Johannes für einen Propheten.«
\bibleverse{27} So gaben sie denn Jesus zur Antwort: »Wir wissen es
nicht.« Da antwortete auch er ihnen: »Dann sage auch ich euch nicht, auf
Grund welcher Vollmacht ich hier so auftrete.«

\hypertarget{jesus-redet-zu-den-fuxfchrern-des-volkes-in-gleichnissen}{%
\subsubsection{6. Jesus redet zu den Führern des Volkes in
Gleichnissen}\label{jesus-redet-zu-den-fuxfchrern-des-volkes-in-gleichnissen}}

\hypertarget{a-das-gleichnis-von-den-zwei-ungleichen-suxf6hnen}{%
\paragraph{a) Das Gleichnis von den zwei ungleichen
Söhnen}\label{a-das-gleichnis-von-den-zwei-ungleichen-suxf6hnen}}

\bibleverse{28} »Was meint ihr aber (über folgendes)? Ein Mann hatte
zwei Söhne. Er ging nun zu dem ersten und sagte: ›Mein Sohn, gehe hin
und arbeite heute im Weinberge.‹ \bibleverse{29} Der antwortete: ›Ja,
Herr‹, ging aber nicht hin. \bibleverse{30} Dann ging er zu dem zweiten
und sagte zu ihm das gleiche. Der gab zur Antwort: ›Ich will nicht!‹
Später aber besann er sich eines Besseren und ging hin. \bibleverse{31}
Wer von den beiden hat nun den Willen des Vaters getan?« Sie
antworteten: »Der zweite.« Da sagte Jesus zu ihnen: »Wahrlich ich sage
euch: Die Zöllner und die Dirnen kommen vor euch\textless sup
title=``=~eher als ihr''\textgreater✲ in das Reich Gottes.
\bibleverse{32} Denn Johannes ist mit der Lehre von der Gerechtigkeit zu
euch gekommen, und ihr habt ihm nicht geglaubt, während die Zöllner und
die Dirnen ihm Glauben geschenkt haben. Ihr dagegen seid, obgleich ihr
das sahet, auch hinterher nicht in euch gegangen, daß ihr ihm geglaubt
hättet.«

\hypertarget{b-das-gleichnis-von-den-treulosen-weinguxe4rtnern}{%
\paragraph{b) Das Gleichnis von den treulosen
Weingärtnern}\label{b-das-gleichnis-von-den-treulosen-weinguxe4rtnern}}

\bibleverse{33} »Vernehmt noch ein anderes Gleichnis: Es war ein
Hausherr, der legte einen Weinberg an, umgab ihn mit einem Zaun, grub in
ihm eine Kelter, baute einen Wachtturm, verpachtete ihn an Weingärtner
und ging dann außer Landes\textless sup title=``Jes
5,1-2''\textgreater✲. \bibleverse{34} Als dann die Zeit der
Früchte\textless sup title=``oder: der Obsternte''\textgreater✲ kam,
sandte er seine Knechte zu den Weingärtnern, damit sie die ihm
zukommenden Früchte in Empfang nähmen. \bibleverse{35} Da ergriffen die
Weingärtner seine Knechte: den einen mißhandelten sie, den andern
erschlugen sie, den dritten steinigten sie. \bibleverse{36} Wiederum
sandte er andere Knechte in noch größerer Zahl als die ersten, doch sie
machten es mit ihnen ebenso. \bibleverse{37} Zuletzt sandte er seinen
Sohn zu ihnen, weil er dachte: ›Sie werden sich doch vor meinem Sohne
scheuen!‹ \bibleverse{38} Als aber die Weingärtner den Sohn sahen,
sagten sie unter sich: ›Dieser ist der Erbe: kommt, wir wollen ihn
töten, dann können wir sein Erbgut in Besitz nehmen!‹ \bibleverse{39} So
ergriffen sie ihn denn, stießen ihn zum Weinberg hinaus und schlugen ihn
tot. \bibleverse{40} Wenn nun der Herr des Weinbergs kommt, was wird er
mit diesen Weingärtnern machen?«

\bibleverse{41} Sie antworteten ihm: »Er wird die Elenden elendiglich
umbringen und den Weinberg an andere Weingärtner verpachten, die ihm die
Früchte zu rechter Zeit abliefern werden.« \bibleverse{42} Jesus fuhr
fort: »Habt ihr noch niemals in den (heiligen) Schriften das Wort
gelesen\textless sup title=``Ps 118,22-23''\textgreater✲: ›Der Stein,
den die Bauleute verworfen\textless sup title=``=~für unbrauchbar
erklärt''\textgreater✲ hatten, der ist zum Eckstein geworden; durch den
Herrn ist er das geworden, und ein Wunder ist er in unsern Augen‹?
\bibleverse{43} Deshalb sage ich euch: Das Reich Gottes wird euch
genommen und einem Volke gegeben werden, das dessen Früchte bringt.
\bibleverse{44} {[}Und wer auf diesen Stein fällt, wird zerschmettert
werden; auf wen aber (der Stein) fällt, den wird er zermalmen.{]}«
\bibleverse{45} Als die Hohenpriester und Pharisäer seine Gleichnisse
hörten, merkten sie, daß er von ihnen redete; \bibleverse{46} darum
hätten sie ihn am liebsten festgenommen, fürchteten sich aber vor der
Volksmenge, weil die ihn für einen Propheten hielt.

\hypertarget{c-das-gleichnis-vom-kuxf6niglichen-hochzeitsmahl}{%
\paragraph{c) Das Gleichnis vom königlichen
Hochzeitsmahl}\label{c-das-gleichnis-vom-kuxf6niglichen-hochzeitsmahl}}

\hypertarget{section-21}{%
\section{22}\label{section-21}}

\bibleverse{1} Und Jesus hob an und redete noch einmal in Gleichnissen
zu ihnen folgendermaßen: \bibleverse{2} »Das Himmelreich ist einem König
vergleichbar, der seinem Sohne die Hochzeit ausrichten wollte.
\bibleverse{3} Er sandte also seine Knechte aus, um die geladenen Gäste
zum Hochzeitsmahl zu bitten; doch sie wollten nicht kommen.
\bibleverse{4} Nochmals sandte er andere Knechte aus, denen er die
Weisung gab: ›Sagt den Geladenen: Seht, mein Festmahl habe ich
zugerichtet; meine Ochsen und das Mastvieh sind geschlachtet, und alles
ist bereit: kommt zum Hochzeitsmahl!‹ \bibleverse{5} Die aber beachteten
es nicht und gingen hin, der eine auf seinen Acker, der andere an sein
Handelsgeschäft; \bibleverse{6} die übrigen ergriffen seine Knechte,
mißhandelten und töteten sie. \bibleverse{7} Da wurde der König zornig;
er entsandte seine Heere, ließ jene Mörder umbringen und ihre Stadt
verbrennen. \bibleverse{8} Hierauf sagte er zu seinen Knechten: ›Das
Hochzeitsmahl ist zwar bereitet, aber die Geladenen waren unwürdig
(daran teilzunehmen). \bibleverse{9} Geht darum an die Straßenecken
hinaus und ladet alle zum Hochzeitsmahl ein, soviele ihr antrefft!‹
\bibleverse{10} So gingen denn jene Knechte auf die Straßen hinaus und
brachten alle, die sie trafen, zusammen, Böse wie Gute, und der
Hochzeitssaal füllte sich mit Gästen.

\bibleverse{11} Als aber der König hineinging, um sich die Gäste
anzusehen, bemerkte er dort einen Mann, der kein Hochzeitsgewand
angelegt hatte. \bibleverse{12} Da sagte er zu ihm: ›Freund, wie hast du
hierher kommen können, ohne ein Hochzeitsgewand anzuhaben?‹ Jener
verstummte. \bibleverse{13} Hierauf befahl der König seinen Dienern:
›Faßt ihn an Händen und Füßen und werft ihn hinaus in die Finsternis
draußen! Dort wird lautes Weinen und Zähneknirschen sein.‹
\bibleverse{14} Denn viele sind berufen, aber wenige auserwählt.«

\hypertarget{streitgespruxe4che-mit-pharisuxe4ern-und-sadduzuxe4ern}{%
\subsubsection{7. Streitgespräche mit Pharisäern und
Sadduzäern}\label{streitgespruxe4che-mit-pharisuxe4ern-und-sadduzuxe4ern}}

\hypertarget{a-die-steuerfrage-der-pharisuxe4er-oder-das-gespruxe4ch-vom-zinsgroschen}{%
\paragraph{a) Die Steuerfrage der Pharisäer (oder: das Gespräch vom
›Zinsgroschen‹)}\label{a-die-steuerfrage-der-pharisuxe4er-oder-das-gespruxe4ch-vom-zinsgroschen}}

\bibleverse{15} Hierauf gingen die Pharisäer hin und stellten eine
Beratung an, wie sie ihn durch einen Ausspruch (wie in einer Schlinge)
fangen könnten. \bibleverse{16} Sie sandten also ihre
Jünger\textless sup title=``oder: Schüler''\textgreater✲ nebst den
Anhängern des Herodes zu ihm, die mußten sagen: »Meister, wir wissen,
daß du wahrhaftig\textless sup title=``oder: aufrichtig''\textgreater✲
bist und den Weg Gottes mit Wahrhaftigkeit lehrst; auch nimmst du auf
niemand Rücksicht, denn du siehst die Person\textless sup
title=``=~äußere Stellung''\textgreater✲ der Menschen nicht an.
\bibleverse{17} So sage uns denn deine Meinung: ›Ist es recht, daß man
dem Kaiser Steuer entrichtet, oder nicht?‹« \bibleverse{18} Da Jesus nun
ihre böse Absicht durchschaute, antwortete er: »Was versucht ihr mich,
ihr Heuchler? \bibleverse{19} Zeigt mir die Steuermünze!« Als sie ihm
nun einen Denar gereicht hatten, \bibleverse{20} fragte er sie: »Wessen
Bild und Aufschrift ist das hier?« \bibleverse{21} Sie antworteten: »Des
Kaisers.« Da sagte er zu ihnen: »So gebt dem Kaiser, was dem Kaiser
zusteht, und Gott, was Gott zusteht!« \bibleverse{22} Als sie das
hörten, verwunderten sie sich, ließen von ihm ab und entfernten sich.

\hypertarget{b-die-sadduzuxe4erfrage-uxfcber-die-auferstehung-der-toten}{%
\paragraph{b) Die Sadduzäerfrage (über die Auferstehung der
Toten)}\label{b-die-sadduzuxe4erfrage-uxfcber-die-auferstehung-der-toten}}

\bibleverse{23} An demselben Tage traten Sadduzäer an ihn heran, die da
behaupten, es gebe keine Auferstehung, und fragten ihn: \bibleverse{24}
»Meister, Mose hat geboten\textless sup title=``5.Mose
25,5''\textgreater✲: ›Wenn jemand kinderlos stirbt, so soll sein Bruder
(als Schwager) dessen Frau✲ heiraten und für seinen Bruder das
Geschlecht fortpflanzen.‹ \bibleverse{25} Nun lebten sieben Brüder bei
uns; der erste✲, der sich verheiratet hatte, starb und hinterließ, weil
er keine Kinder hatte, seine Frau seinem Bruder; \bibleverse{26} ebenso
auch der zweite und der dritte, schließlich alle sieben; \bibleverse{27}
zuletzt nach allen starb auch die Frau. \bibleverse{28} Wem von den
Sieben wird sie nun in\textless sup title=``oder: bei''\textgreater✲ der
Auferstehung als Frau angehören? Alle haben sie ja zur Frau gehabt.«
\bibleverse{29} Jesus antwortete ihnen: »Ihr seid im Irrtum, weil ihr
weder die (heiligen) Schriften noch die Kraft Gottes kennt.
\bibleverse{30} Denn in der Auferstehung heiraten sie weder, noch werden
sie verheiratet, sondern sie sind wie Engel im Himmel. \bibleverse{31}
Was aber die Auferstehung der Toten betrifft: habt ihr nicht gelesen,
was euch darüber von Gott gesagt worden ist, wenn er
spricht\textless sup title=``2.Mose 3,6''\textgreater✲: \bibleverse{32}
›Ich bin der Gott Abrahams, der Gott Isaaks und der Gott Jakobs‹? Gott
ist doch nicht ein Gott von Toten, sondern von Lebenden.«
\bibleverse{33} Als die Volksmenge das hörte, staunte sie über seine
Lehre.

\hypertarget{c-die-frage-eines-gesetzeskundigen-nach-dem-vornehmsten-gebot}{%
\paragraph{c) Die Frage eines Gesetzeskundigen nach dem vornehmsten
Gebot}\label{c-die-frage-eines-gesetzeskundigen-nach-dem-vornehmsten-gebot}}

\bibleverse{34} Als aber die Pharisäer vernahmen, daß er die Sadduzäer
zum Schweigen gebracht hatte, versammelten sie sich (um ihn);
\bibleverse{35} und einer von ihnen, ein Gesetzeslehrer, versuchte ihn
mit der Frage: \bibleverse{36} »Meister, was ist ein Hauptgebot im
Gesetz?« \bibleverse{37} Er antwortete ihm: »Du sollst den Herrn, deinen
Gott, lieben mit deinem ganzen Herzen, mit deiner ganzen Seele und mit
deinem ganzen Denken.\textless sup title=``5.Mose 6,5''\textgreater✲
\bibleverse{38} Dies ist das Hauptgebot, das obenan steht.
\bibleverse{39} Ein zweites aber steht ihm gleich: ›Du sollst deinen
Nächsten lieben wie dich selbst!‹\textless sup title=``3.Mose
19,18''\textgreater✲ \bibleverse{40} In\textless sup title=``oder:
an''\textgreater✲ diesen beiden Geboten hängt das ganze Gesetz und die
Propheten.«

\hypertarget{d-die-gegenfrage-jesu-nach-dem-messias-als-dem-sohne-davids}{%
\paragraph{d) Die Gegenfrage Jesu nach dem Messias als dem Sohne
Davids}\label{d-die-gegenfrage-jesu-nach-dem-messias-als-dem-sohne-davids}}

\bibleverse{41} Da aber die Pharisäer beisammen waren, legte Jesus ihnen
die Frage vor: \bibleverse{42} »Wie denkt ihr über Christus\textless sup
title=``=~den Messias''\textgreater✲? Wessen Sohn✲ ist er?« Sie
antworteten ihm: »Er ist Davids Sohn.« \bibleverse{43} Da erwiderte
Jesus ihnen: »Wie kann ihn dann aber David im Geist\textless sup
title=``oder: durch den Geist''\textgreater✲ ›Herr‹ nennen, indem er
sagt\textless sup title=``Ps 110,1''\textgreater✲: \bibleverse{44} ›Der
Herr hat zu meinem Herrn gesagt: Setze dich zu meiner Rechten, bis ich
deine Feinde hinlege zum Schemel deiner Füße‹? \bibleverse{45} Wenn
David ihn\textless sup title=``d.h. den Messias''\textgreater✲ also
›Herr‹ nennt, wie kann er da sein Sohn sein?« \bibleverse{46} Und
niemand konnte ihm hierauf eine Antwort geben; auch wagte von diesem
Tage an niemand mehr, ihm eine Frage vorzulegen.

\hypertarget{die-grouxdfe-strafrede-jesu-gegen-die-schriftgelehrten-und-pharisuxe4er}{%
\subsubsection{8. Die große Strafrede Jesu gegen die Schriftgelehrten
und
Pharisäer}\label{die-grouxdfe-strafrede-jesu-gegen-die-schriftgelehrten-und-pharisuxe4er}}

\hypertarget{section-22}{%
\section{23}\label{section-22}}

\bibleverse{1} Damals richtete Jesus an das Volk und an seine Jünger
folgende Worte:

\hypertarget{a-eingang-der-rede-ruxfcge-des-verwerflichen-verhaltens-der-geistlichen-fuxfchrer-des-volks-in-ihrer-hohen-stellung}{%
\paragraph{a) Eingang der Rede: Rüge des verwerflichen Verhaltens der
geistlichen Führer des Volks in ihrer hohen
Stellung}\label{a-eingang-der-rede-ruxfcge-des-verwerflichen-verhaltens-der-geistlichen-fuxfchrer-des-volks-in-ihrer-hohen-stellung}}

\bibleverse{2} »Auf den Lehrstuhl Moses haben sich die Schriftgelehrten
und die Pharisäer gesetzt. \bibleverse{3} Alles nun, was sie euch
sagen\textless sup title=``=~zu tun gebieten''\textgreater✲, das tut und
befolgt, aber nach ihren Werken\textless sup title=``=~ihrem
Tun''\textgreater✲ richtet euch nicht; denn sie sagen es nur, tun es
aber nicht. \bibleverse{4} Sie binden schwere Lasten zusammen und legen
sie den Menschen auf die Schultern, sie selbst aber wollen sie mit
keinem Finger anrühren. \bibleverse{5} Alle ihre Werke tun sie in der
Absicht, von den Leuten gesehen zu werden; denn sie machen ihre
Gebetsriemen breit und ihre Mantelquasten\textless sup title=``4.Mose
15,38-39''\textgreater✲ lang; \bibleverse{6} sie lieben den ersten Platz
bei den Gastmählern und die Ehrensitze in den Synagogen; \bibleverse{7}
sie wollen auf den Märkten\textless sup title=``oder: öffentlichen
Plätzen''\textgreater✲ gegrüßt sein und lassen sich von den Leuten gern
›Rabbi‹\textless sup title=``d.h. Meister, Lehrer''\textgreater✲ nennen.
\bibleverse{8} Ihr aber sollt euch nicht ›Meister‹ nennen lassen; denn
einer ist euer Meister, ihr alle aber seid Brüder. \bibleverse{9} Und
niemand auf Erden sollt ihr euren ›Vater‹ nennen; denn einer ist euer
Vater, der im Himmel. \bibleverse{10} Auch ›Lehrer\textless sup
title=``oder: Führer''\textgreater✲‹ sollt ihr euch nicht nennen lassen;
denn einer ist euer Lehrer\textless sup title=``oder:
Führer''\textgreater✲, nämlich Christus. \bibleverse{11} Der Größte
unter euch soll euer Diener sein. \bibleverse{12} Wer sich aber selbst
erhöht, wird erniedrigt werden, und wer sich selbst erniedrigt, wird
erhöht werden.«\textless sup title=``Lk 14,11; 18,14''\textgreater✲

\hypertarget{b-die-sieben-weherufe-uxfcber-die-schriftgelehrten-und-pharisuxe4er}{%
\paragraph{b) Die sieben Weherufe über die Schriftgelehrten und
Pharisäer}\label{b-die-sieben-weherufe-uxfcber-die-schriftgelehrten-und-pharisuxe4er}}

\bibleverse{13} »Wehe euch, Schriftgelehrte und Pharisäer, ihr
Heuchler✲! Denn ihr verschließt das Himmelreich vor den Menschen. Ihr
selbst geht ja nicht hinein, laßt aber auch die nicht hinein, welche
hineingehen wollen.

\bibleverse{14} {[}Wehe euch, Schriftgelehrte und Pharisäer, ihr
Heuchler! Denn ihr bringt die Häuser der Witwen gierig an euch und
verrichtet zum Schein lange Gebete. Darum werdet ihr ein um so
strengeres Gericht erfahren.{]} \bibleverse{15} Wehe euch,
Schriftgelehrte und Pharisäer, ihr Heuchler! Denn ihr durchreist Land
und Meer, um einen einzigen Glaubensgenossen zu gewinnen; und wenn er es
geworden ist, macht ihr aus ihm ein Kind der Hölle, das doppelt so
schlimm ist als ihr selbst. \bibleverse{16} Wehe euch, ihr blinden
Führer, die ihr sagt✲: ›Wenn einer beim Tempel schwört, so hat das
nichts zu bedeuten; wer aber beim Gold des Tempels\textless sup
title=``oder: am Tempel''\textgreater✲ schwört, der ist gebunden.‹
\bibleverse{17} Ihr Toren und Blinde! Was steht denn höher: das
Gold\textless sup title=``=~der Goldschmuck''\textgreater✲ oder der
Tempel, der das Gold erst heilig gemacht hat? \bibleverse{18} Ferner
(sagt ihr): ›Wenn einer beim Altar schwört, so hat das nichts zu
bedeuten; wer aber bei der Opfergabe schwört, die auf dem Altar liegt,
der ist gebunden.‹ \bibleverse{19} Ihr Blinden! Was steht denn höher,
die Opfergabe oder der Altar, der die Gabe erst heilig macht?
\bibleverse{20} Wer also beim Altar schwört, der schwört bei ihm und bei
allem, was auf ihm liegt; \bibleverse{21} und wer beim Tempel schwört,
der schwört bei ihm und bei dem, der in ihm wohnt; \bibleverse{22} und
wer beim Himmel schwört, der schwört bei Gottes Thron und bei dem, der
auf ihm sitzt.

\bibleverse{23} Wehe euch, Schriftgelehrte und Pharisäer, ihr Heuchler!
Ihr entrichtet den Zehnten von Minze, von Anis und Kümmel, laßt aber das
Schwierigere\textless sup title=``oder: Wichtigere''\textgreater✲ im
Gesetz außer acht, nämlich das Gericht\textless sup title=``oder: die
Rechtspflege''\textgreater✲, die Barmherzigkeit und die
Treue\textless sup title=``oder: den Glauben''\textgreater✲. Diese
sollte man üben und jenes nicht außer acht lassen. \bibleverse{24} Ihr
blinden Führer, die ihr die Mücke seihet\textless sup title=``d.h. durch
Seihen der Getränke entfernt''\textgreater✲, aber das Kamel
hinuntertrinkt!

\bibleverse{25} Wehe euch, Schriftgelehrte und Pharisäer, ihr Heuchler!
Denn ihr haltet die Außenseite des Bechers und der Schüssel rein,
inwendig aber sind sie gefüllt mit Raub und Unmäßigkeit\textless sup
title=``d.h. mit dem, was ihr durch Raub an euch gebracht habt und mit
Unmäßigkeit genießt''\textgreater✲. \bibleverse{26} Du blinder
Pharisäer! Mache zuerst das rein, was den Inhalt des Bechers bildet,
dann wird auch seine Außenseite rein werden (können).

\bibleverse{27} Wehe euch, Schriftgelehrte und Pharisäer, ihr Heuchler!
Denn ihr gleicht frischgetünchten Gräbern, die von außen schön aussehen,
im Innern aber voll von Totengebeinen und lauter Verwesung sind.
\bibleverse{28} Ebenso zeigt auch ihr euch den Menschen von außen
gerecht, inwendig aber seid ihr voll von Heuchelei und
Gesetzlosigkeit\textless sup title=``oder: Gesetzesbruch''\textgreater✲.

\bibleverse{29} Wehe euch, Schriftgelehrte und Pharisäer, ihr Heuchler!
Denn ihr baut die Grabstätten der Propheten aus und schmückt die
Grabdenkmäler der Gerechten \bibleverse{30} und sagt: ›Hätten wir zur
Zeit unserer Väter gelebt, wir hätten uns nicht mit ihnen am Blut der
Propheten schuldig gemacht!‹ \bibleverse{31} Damit stellt ihr euch
selbst das Zeugnis aus, daß ihr die Söhne✲ der Prophetenmörder seid.
\bibleverse{32} So macht denn ihr das Maß (der Schuld) eurer Väter voll!
\bibleverse{33} Ihr Schlangen, ihr Otternbrut! Wie wollt ihr dem
Strafgericht der Hölle entrinnen?!«

\hypertarget{c-schluuxdf-der-strafrede}{%
\paragraph{c) Schluß der Strafrede}\label{c-schluuxdf-der-strafrede}}

\hypertarget{aa-drohung-gegen-das-seinem-heil-widerstrebende-blutbefleckte-volk}{%
\subparagraph{aa) Drohung gegen das seinem Heil widerstrebende,
blutbefleckte
Volk}\label{aa-drohung-gegen-das-seinem-heil-widerstrebende-blutbefleckte-volk}}

\bibleverse{34} »Deshalb seht: ich sende zu euch Propheten und Weise und
Schriftgelehrte\textless sup title=``oder: Lehrer''\textgreater✲; von
diesen werdet ihr die einen töten und kreuzigen, die anderen in euren
Synagogen geißeln und von Stadt zu Stadt verfolgen, \bibleverse{35}
damit über euch alles gerechte✲ Blut komme, das auf der Erde vergossen
worden ist, vom Blut des gerechten Abel an\textless sup title=``1.Mose
4,8''\textgreater✲ bis zum Blut Sacharjas✲, des Sohnes Berechjas, den
ihr zwischen dem Tempelhause und dem Brandopferaltar ermordet
habt\textless sup title=``vgl. 2.Chr 24,19-22''\textgreater✲.
\bibleverse{36} Wahrlich ich sage euch: (Die Strafe für) dies alles wird
über dieses Geschlecht kommen!«

\hypertarget{bb-jesu-abkehr-von-der-stadt-jerusalem-und-ankuxfcndigung-seiner-wiederkehr}{%
\subparagraph{bb) Jesu Abkehr von der Stadt Jerusalem und Ankündigung
seiner
Wiederkehr}\label{bb-jesu-abkehr-von-der-stadt-jerusalem-und-ankuxfcndigung-seiner-wiederkehr}}

\bibleverse{37} »Jerusalem, Jerusalem, das du die Propheten tötest und
die zu dir Gesandten steinigst! Wie oft habe ich deine Kinder um mich
sammeln wollen, wie eine Henne ihre Küchlein unter ihre Flügel sammelt;
doch ihr habt nicht gewollt. \bibleverse{38} Nunmehr wird euer Haus euch
verödet überlassen\textless sup title=``Jer 22,5''\textgreater✲;
\bibleverse{39} denn ich sage euch: Ihr werdet mich von jetzt an nicht
(mehr) sehen, bis ihr (einst bei meiner Wiederkunft) ausruft: ›Gepriesen
sei, der da kommt im Namen des Herrn!‹«\textless sup title=``Ps
118,26''\textgreater✲

\hypertarget{die-uxf6lbergrede-jesu-an-seine-juxfcnger-von-der-zerstuxf6rung-des-tempels-vom-ende-dieser-weltzeit-von-seiner-wiederkunft-und-vom-gericht-uxfcber-die-vuxf6lker-kap.-24-25}{%
\subsubsection{9. Die Ölbergrede Jesu an seine Jünger von der Zerstörung
des Tempels, vom Ende dieser Weltzeit, von seiner Wiederkunft und vom
Gericht über die Völker (Kap.
24-25)}\label{die-uxf6lbergrede-jesu-an-seine-juxfcnger-von-der-zerstuxf6rung-des-tempels-vom-ende-dieser-weltzeit-von-seiner-wiederkunft-und-vom-gericht-uxfcber-die-vuxf6lker-kap.-24-25}}

\hypertarget{a-einleitung-anlauxdf-der-rede-mit-ankuxfcndigung-der-zerstuxf6rung-des-tempels}{%
\paragraph{a) Einleitung: Anlaß der Rede (mit Ankündigung der Zerstörung
des
Tempels)}\label{a-einleitung-anlauxdf-der-rede-mit-ankuxfcndigung-der-zerstuxf6rung-des-tempels}}

\hypertarget{section-23}{%
\section{24}\label{section-23}}

\bibleverse{1} Jesus verließ dann den Tempel und wollte weitergehen; da
traten seine Jünger zu ihm heran, um ihn auf den Prachtbau des Tempels
aufmerksam zu machen. \bibleverse{2} Er aber antwortete ihnen mit den
Worten: »Ja, jetzt seht ihr dies alles noch. Wahrlich ich sage euch: Es
wird hier kein Stein auf dem andern bleiben, der nicht niedergerissen
wird!« \bibleverse{3} Als er sich dann auf dem Ölberg niedergesetzt
hatte, traten die Jünger, als sie für sich allein waren, an ihn mit der
Bitte heran: »Sage uns doch: wann wird dies geschehen? Und welches ist
das Zeichen deiner Ankunft\textless sup title=``bzw.
Wiederkunft''\textgreater✲ und der Vollendung\textless sup title=``=~des
Endes''\textgreater✲ der Weltzeit?«

\hypertarget{b-das-ende-dieser-weltzeit}{%
\paragraph{b) Das Ende dieser
Weltzeit}\label{b-das-ende-dieser-weltzeit}}

\hypertarget{aa-die-ersten-vorzeichen}{%
\subparagraph{aa) Die ersten
Vorzeichen}\label{aa-die-ersten-vorzeichen}}

\bibleverse{4} Jesus antwortete ihnen: »Sehet euch vor, daß niemand euch
irreführe! \bibleverse{5} Denn viele werden unter meinem Namen kommen
und behaupten: ›Ich bin der (wiederkehrende) Christus‹, und werden viele
irreführen. \bibleverse{6} Ihr werdet ferner von Kriegen und
Kriegsgerüchten hören: gebt acht, laßt euch dadurch nicht erschrecken!
Denn das muß so kommen, ist aber noch nicht das Ende. \bibleverse{7}
Denn ein Volk wird sich gegen das andere erheben und ein Reich gegen das
andere\textless sup title=``Jes 19,2''\textgreater✲; auch Hungersnöte
werden eintreten und Erdbeben hier und da stattfinden; \bibleverse{8}
dies alles ist aber erst der Anfang der Wehen\textless sup title=``d.h.
der Nöte oder: der Leiden''\textgreater✲.«

\hypertarget{bb-die-juxfcngerverfolgungen}{%
\subparagraph{bb) Die
Jüngerverfolgungen}\label{bb-die-juxfcngerverfolgungen}}

\bibleverse{9} »Hierauf wird man schwere Drangsale über euch bringen und
euch töten, und ihr werdet allen Völkern um meines Namens willen verhaßt
sein. \bibleverse{10} Alsdann werden viele Anstoß nehmen\textless sup
title=``d.h. am wahren Glauben irre werden''\textgreater✲ und sich
einander ausliefern✲ und einander hassen. \bibleverse{11} Auch falsche
Propheten werden in großer Zahl auftreten und viele irreführen;
\bibleverse{12} und weil die Gesetzlosigkeit überhand nimmt, wird die
Liebe in den meisten erkalten; \bibleverse{13} wer jedoch bis ans Ende
ausharrt, der wird gerettet werden. \bibleverse{14} Und diese
Heilsbotschaft vom Reich wird auf dem ganzen Erdkreis allen Völkern zum
Zeugnis gepredigt werden, und dann wird das Ende kommen.«

\hypertarget{cc-der-huxf6hepunkt-der-drangsal-in-juduxe4a}{%
\subparagraph{cc) Der Höhepunkt der Drangsal in
Judäa}\label{cc-der-huxf6hepunkt-der-drangsal-in-juduxe4a}}

\bibleverse{15} »Wenn ihr nun den Greuel der Verwüstung✲, der vom
Propheten Daniel angesagt worden ist\textless sup title=``Dan 9,27;
11,31; 12,11''\textgreater✲, an heiliger Stätte stehen seht -- der Leser
merke auf! --, \bibleverse{16} dann sollen die (Gläubigen), die in Judäa
sind, ins Gebirge fliehen! \bibleverse{17} Wer sich alsdann auf dem
Dache befindet, steige nicht erst noch hinab (ins Haus), um seine
Habseligkeiten aus dem Hause zu holen; \bibleverse{18} und wer auf dem
Felde weilt, kehre nicht zurück, um sich noch seinen Mantel zu holen.
\bibleverse{19} Wehe aber den Frauen, die guter Hoffnung sind, und
denen, die ein Kind in jenen Tagen zu nähren haben! \bibleverse{20}
Betet nur, daß eure Flucht nicht in den Winter\textless sup title=``vgl.
Joh 10,22''\textgreater✲ oder auf den Sabbat falle! \bibleverse{21} Denn
es wird alsdann eine schlimme Drangsalszeit eintreten, wie noch keine
seit Anfang der Welt bis jetzt dagewesen ist und wie auch keine wieder
kommen wird\textless sup title=``Dan 12,1''\textgreater✲;
\bibleverse{22} und wenn jene Tage nicht verkürzt würden, so würde kein
Fleisch✲ gerettet werden; aber um der Auserwählten willen werden jene
Tage verkürzt werden.«

\hypertarget{dd-weissagung-der-falschen-propheten}{%
\subparagraph{dd) Weissagung der falschen
Propheten}\label{dd-weissagung-der-falschen-propheten}}

\bibleverse{23} »Wenn dann jemand zu euch sagt: ›Seht, hier ist
Christus\textless sup title=``=~der Messias; vgl. 1,16''\textgreater✲!‹
oder: ›Dort (ist er)!‹, so glaubt es nicht! \bibleverse{24} Denn es
werden falsche Christusse\textless sup title=``oder:
Messiasse''\textgreater✲ und falsche Propheten auftreten und werden
große Zeichen und Wunder verrichten, um womöglich auch die Auserwählten
irrezuführen. \bibleverse{25} Seht, ich habe es euch vorhergesagt. Wenn
man also zu euch sagt: \bibleverse{26} ›Seht, er\textless sup
title=``d.h. Christus''\textgreater✲ ist in der Wüste!‹, so geht nicht
hinaus; und (sagt man:) ›Seht, er ist in den Gemächern (dieses oder
jenes Hauses)!‹, so glaubt es nicht! \bibleverse{27} Denn wie der Blitz
vom Osten ausgeht und bis zum Westen leuchtet, so wird es auch mit der
Ankunft✲ des Menschensohnes sein; \bibleverse{28} denn wo das
Aas\textless sup title=``=~ein verendetes Tier''\textgreater✲ liegt, da
sammeln sich die Geier.«\textless sup title=``Lk 17,37; Hiob
39,30''\textgreater✲

\hypertarget{c-die-letzten-vorzeichen-und-die-wiederkunft-des-menschensohnes-mit-ihren-begleiterscheinungen}{%
\paragraph{c) Die letzten Vorzeichen und die Wiederkunft des
Menschensohnes mit ihren
Begleiterscheinungen}\label{c-die-letzten-vorzeichen-und-die-wiederkunft-des-menschensohnes-mit-ihren-begleiterscheinungen}}

\bibleverse{29} »Sogleich aber nach jener Drangsalszeit wird die Sonne
sich verfinstern und der Mond seinen Schein verlieren\textless sup
title=``Jes 13,10''\textgreater✲; die Sterne werden vom Himmel fallen
und die Kräfte des Himmels in Erschütterung geraten\textless sup
title=``Jes 34,4''\textgreater✲. \bibleverse{30} Und dann wird das
Zeichen des Menschensohnes am Himmel erscheinen, und dann werden alle
Geschlechter\textless sup title=``oder: Völker''\textgreater✲ der Erde
wehklagen und werden den Menschensohn auf den Wolken des Himmels mit
großer Macht und Herrlichkeit kommen sehen\textless sup title=``Sach
12,10-12; Dan 7,13-14''\textgreater✲. \bibleverse{31} Und er wird seine
Engel unter lautem Posaunenschall aussenden, und sie werden seine
Auserwählten von den vier Windrichtungen her versammeln, von dem einen
Himmelsende bis zum andern\textless sup title=``Sach 2,6''\textgreater✲.

\bibleverse{32} Vom Feigenbaum aber mögt ihr das Gleichnis lernen✲:
Sobald seine Zweige saftig werden und Blätter hervorwachsen, so erkennt
ihr daran, daß der Sommer nahe ist. \bibleverse{33} So auch ihr: wenn
ihr dies alles seht, so erkennet daran, daß es\textless sup
title=``oder: er, d.h. der Menschensohn''\textgreater✲ nahe vor der Tür
steht. \bibleverse{34} Wahrlich ich sage euch: Dieses Geschlecht wird
nicht vergehen, bis dies alles geschieht. \bibleverse{35} Himmel und
Erde werden vergehen, meine Worte aber werden nimmermehr vergehen.
\bibleverse{36} Von jenem Tage aber und von jener Stunde hat niemand
Kenntnis, auch die Engel im Himmel nicht, auch der Sohn nicht, sondern
ganz allein der Vater. \bibleverse{37} Denn wie es einst mit den Tagen
Noahs gewesen ist, so wird es auch mit der Wiederkunft des
Menschensohnes sein. \bibleverse{38} Denn wie sie es in den Tagen vor
der Sintflut gehalten haben: sie aßen und tranken, sie heirateten und
verheirateten (ihre Töchter) bis zu dem Tage, als Noah in die Arche
ging, \bibleverse{39} und wie sie nichts merkten, bis die Sintflut kam
und alle hinwegraffte, ebenso wird es auch mit der Zeit der Ankunft✲ des
Menschensohnes der Fall sein. \bibleverse{40} Da werden zwei (Männer
zusammen) auf dem Felde sein: der eine wird angenommen\textless sup
title=``oder: mitgenommen''\textgreater✲, der andere zurückgelassen;
\bibleverse{41} zwei (Frauen) werden (zusammen) an der Handmühle mahlen:
die eine wird angenommen\textless sup title=``oder:
mitgenommen''\textgreater✲, die andere zurückgelassen.«

\hypertarget{d-schluuxdfermahnung-an-die-juxfcnger-zur-wachsamkeit-und-zur-bereitschaft-auf-das-gericht}{%
\paragraph{d) Schlußermahnung an die Jünger zur Wachsamkeit und zur
Bereitschaft auf das
Gericht}\label{d-schluuxdfermahnung-an-die-juxfcnger-zur-wachsamkeit-und-zur-bereitschaft-auf-das-gericht}}

\hypertarget{aa-mahnung-zur-wachsamkeit-im-allgemeinen}{%
\subparagraph{aa) Mahnung zur Wachsamkeit im
allgemeinen}\label{aa-mahnung-zur-wachsamkeit-im-allgemeinen}}

\bibleverse{42} »Seid also wachsam, denn ihr wißt nicht, an welchem Tage
der Herr kommt. \bibleverse{43} Das aber seht ihr ein: Wenn der Hausherr
wüßte, in welcher Stunde der Nacht✲ der Dieb kommt, so würde er wach
bleiben und keinen Einbruch in sein Haus zulassen. \bibleverse{44}
Deshalb haltet auch ihr euch bereit; denn der Menschensohn kommt zu
einer Stunde, wo ihr es nicht vermutet.«

\hypertarget{bb-gleichnis-vom-treuen-und-vom-untreuen-knecht}{%
\subparagraph{bb) Gleichnis vom treuen und vom untreuen
Knecht}\label{bb-gleichnis-vom-treuen-und-vom-untreuen-knecht}}

\bibleverse{45} »Wer ist demnach der treue und kluge Knecht, den sein
Herr über seine Dienerschaft gesetzt hat, damit er ihnen die Speise✲ zu
rechter Zeit gebe? \bibleverse{46} Selig ist ein solcher Knecht (zu
preisen), den sein Herr bei seiner Rückkehr in solcher Tätigkeit
antrifft. \bibleverse{47} Wahrlich ich sage euch: Er wird ihn über seine
sämtlichen Güter setzen. \bibleverse{48} Wenn aber ein solcher Knecht
schlecht ist und in seinem Herzen denkt: ›Mein Herr kommt noch lange
nicht!‹, \bibleverse{49} und wenn er seine Mitknechte zu schlagen
beginnt und mit den Trunkenen ißt und trinkt, \bibleverse{50} so wird
der Herr eines solchen Knechts an einem Tage kommen, an dem er es nicht
erwartet, und zu einer Stunde, die er nicht kennt, \bibleverse{51} und
er wird ihn zerhauen lassen und ihm seinen Platz\textless sup
title=``oder: sein gebührendes Teil''\textgreater✲ bei den Heuchlern
anweisen: dort wird lautes Weinen und Zähneknirschen sein.«

\hypertarget{cc-das-gleichnis-von-den-zehn-klugen-und-tuxf6richten-jungfrauen}{%
\subparagraph{cc) Das Gleichnis von den zehn (klugen und törichten)
Jungfrauen}\label{cc-das-gleichnis-von-den-zehn-klugen-und-tuxf6richten-jungfrauen}}

\hypertarget{section-24}{%
\section{25}\label{section-24}}

\bibleverse{1} »Alsdann wird das Himmelreich zehn Jungfrauen gleichen,
die sich mit ihren Lampen in der Hand zur Einholung des Bräutigams
aufmachten. \bibleverse{2} Fünf von ihnen waren töricht und fünf klug;
\bibleverse{3} denn die törichten nahmen wohl ihre Lampen, nahmen aber
kein Öl mit; \bibleverse{4} die klugen dagegen nahmen außer ihren Lampen
auch noch Öl in den Gefäßen mit sich. \bibleverse{5} Als nun der
Bräutigam auf sich warten ließ, wurden sie alle müde und schliefen ein.
\bibleverse{6} Um Mitternacht aber erscholl ein Geschrei: ›Der Bräutigam
ist da! Macht euch auf, ihn zu empfangen!‹ \bibleverse{7} Da erhoben
sich jene Jungfrauen alle vom Schlaf und brachten ihre Lampen in
Ordnung; \bibleverse{8} die törichten aber sagten zu den klugen: ›Gebt
uns von eurem Öl, denn unsere Lampen wollen ausgehen!‹ \bibleverse{9} Da
antworteten die klugen: ›Nein, es würde für uns und euch nicht reichen;
geht lieber zu den Krämern und kauft euch welches!‹ \bibleverse{10}
Während sie nun hingingen, um Öl einzukaufen, kam der Bräutigam, und die
Jungfrauen, welche in Bereitschaft waren, gingen mit ihm zum
Hochzeitsmahl hinein, und die Tür wurde verschlossen. \bibleverse{11}
Später kamen dann auch noch die übrigen Jungfrauen und riefen: ›Herr,
Herr, öffne uns doch!‹ \bibleverse{12} Er aber gab ihnen zur Antwort:
›Wahrlich ich sage euch: Ich kenne euch nicht!‹ \bibleverse{13} Darum
seid wachsam, denn Tag und Stunde sind euch unbekannt.«

\hypertarget{dd-gleichnis-von-den-anvertrauten-geldern-talenten}{%
\subparagraph{dd) Gleichnis von den anvertrauten Geldern
(Talenten)}\label{dd-gleichnis-von-den-anvertrauten-geldern-talenten}}

\bibleverse{14} »Es wird so sein wie bei einem Manne, der vor Antritt
einer Reise ins Ausland seine Knechte rief und ihnen sein Vermögen (zur
Verwaltung) übergab; \bibleverse{15} dem einen gab er fünf Talente, dem
andern zwei, dem dritten eins, einem jeden nach seiner Tüchtigkeit; dann
reiste er ab. \bibleverse{16} Da ging der, welcher die fünf Talente
empfangen hatte, sogleich ans Werk, machte Geschäfte mit dem Geld und
gewann andere fünf Talente; \bibleverse{17} ebenso gewann der, welcher
die zwei Talente (empfangen hatte), zwei andere dazu. \bibleverse{18}
Der (Knecht) aber, welcher das eine Talent erhalten hatte, ging hin,
grub ein Loch in die Erde und verbarg darin das Geld seines Herrn.
\bibleverse{19} Nach längerer Zeit kam der Herr dieser Knechte zurück
und rechnete mit ihnen ab. \bibleverse{20} Da trat der herzu, welcher
die fünf Talente empfangen hatte, brachte noch fünf andere Talente mit
und sagte: ›Herr, fünf Talente hast du mir übergeben; hier sind noch
andere fünf Talente, die ich dazugewonnen habe.‹ \bibleverse{21} Da
sagte sein Herr zu ihm: ›Schön, du guter und treuer Knecht! Du bist über
Wenigem treu gewesen, ich will dich über Vieles setzen: gehe ein zum
Freudenmahl deines Herrn!‹ \bibleverse{22} Dann kam auch der (Knecht)
herbei, der die zwei Talente (empfangen hatte), und sagte: ›Herr, zwei
Talente hast du mir übergeben; hier sind noch zwei andere Talente, die
ich dazugewonnen habe.‹ \bibleverse{23} Da sagte sein Herr zu ihm:
›Schön, du guter und treuer Knecht! Du bist über Wenigem treu gewesen,
ich will dich über Vieles setzen: gehe ein zum Freudenmahl deines
Herrn!‹

\bibleverse{24} Da trat auch der herzu, welcher das eine Talent
empfangen hatte, und sagte: ›Herr, ich wußte von dir, daß du ein harter
Mann bist: du erntest, wo du nicht gesät hast, und sammelst ein, wo du
nicht ausgestreut\textless sup title=``oder: geworfelt''\textgreater✲
hast. \bibleverse{25} Da bin ich aus Furcht hingegangen und habe dein
Talent in der Erde verborgen: hier hast du dein Geld wieder!‹
\bibleverse{26} Da antwortete ihm sein Herr: ›Du böser✲ und träger
Knecht! Du wußtest, daß ich ernte, wo ich nicht gesät habe, und
einsammle, wo ich nicht ausgestreut\textless sup title=``oder:
geworfelt''\textgreater✲ habe? \bibleverse{27} Nun, so hättest du mein
Geld bei den Bankhaltern anlegen sollen; dann hätte ich bei meiner
Rückkehr mein Geld mit Zinsen zurückerhalten. \bibleverse{28} So nehmt
ihm nun das Talent ab und gebt es dem, der die zehn Talente hat.
\bibleverse{29} Denn jedem, der da hat, wird noch hinzugegeben werden,
so daß er Überfluß hat; wer aber nicht\textless sup title=``d.h. so gut
wie nichts''\textgreater✲ hat, dem wird auch noch das genommen werden,
was er hat. \bibleverse{30} Den unnützen Knecht jedoch werft hinaus in
die Finsternis draußen! Dort wird lautes Weinen und Zähneknirschen
sein.«

\hypertarget{e-jesu-gericht-uxfcber-die-vuxf6lker-und-uxfcber-die-einzelnen-menschen-die-scheidung-der-schafe-von-den-buxf6cken}{%
\paragraph{e) Jesu Gericht über die Völker und über die einzelnen
Menschen; die Scheidung der Schafe von den
Böcken}\label{e-jesu-gericht-uxfcber-die-vuxf6lker-und-uxfcber-die-einzelnen-menschen-die-scheidung-der-schafe-von-den-buxf6cken}}

\bibleverse{31} »Wenn aber der Menschensohn in seiner Herrlichkeit kommt
und alle Engel mit ihm, dann wird er sich auf den Thron seiner
Herrlichkeit setzen; \bibleverse{32} alle Völker werden alsdann vor ihm
versammelt werden, und er wird sie voneinander scheiden, wie der Hirt
die Schafe von den Böcken scheidet; \bibleverse{33} und er wird die
Schafe zu seiner Rechten, die Böcke aber zu seiner Linken stellen.
\bibleverse{34} Dann wird der König zu denen auf seiner rechten Seite
sagen: ›Kommt her, ihr von meinem Vater Gesegneten! Empfangt als euer
Erbe das Königtum, das für euch seit Grundlegung der Welt bereitgehalten
ist. \bibleverse{35} Denn ich bin hungrig gewesen, und ihr habt mir zu
essen gegeben; ich bin durstig gewesen, und ihr habt mir zu trinken
gereicht; ich bin ein Fremdling gewesen, und ihr habt mich beherbergt;
\bibleverse{36} ich bin ohne Kleidung gewesen, und ihr habt mich
gekleidet; ich bin krank gewesen, und ihr habt mich besucht; ich habe im
Gefängnis gelegen, und ihr seid zu mir gekommen.‹

\bibleverse{37} Dann werden ihm die Gerechten antworten: ›Herr, wann
haben wir dich hungrig gesehen und haben dich gespeist? Oder durstig und
haben dir zu trinken gereicht? \bibleverse{38} Wann haben wir dich als
Fremdling gesehen und haben dich beherbergt? Oder ohne Kleidung und
haben dich bekleidet? \bibleverse{39} Wann haben wir dich krank oder im
Gefängnis gesehen und sind zu dir gekommen?‹ \bibleverse{40} Dann wird
der König ihnen antworten: ›Wahrlich ich sage euch: Alles, was ihr einem
von diesen meinen geringsten Brüdern getan habt, das habt ihr mir
getan.‹

\bibleverse{41} Alsdann wird er auch zu denen auf seiner linken Seite
sagen: ›Hinweg von mir, ihr Verfluchten, in das ewige Feuer, das für den
Teufel und seine Engel bereitet ist! \bibleverse{42} Denn ich bin
hungrig gewesen, aber ihr habt mir nichts zu essen gegeben; ich bin
durstig gewesen, aber ihr habt mir nichts zu trinken gereicht;
\bibleverse{43} ich bin ein Fremdling gewesen, aber ihr habt mich nicht
beherbergt; ohne Kleidung, aber ihr habt mich nicht bekleidet; krank und
im Gefängnis (habe ich gelegen), aber ihr habt mich nicht besucht.‹
\bibleverse{44} Dann werden auch diese antworten: ›Herr, wann haben wir
dich hungrig oder durstig, als einen Fremdling oder ohne Kleidung, wann
krank oder im Gefängnis gesehen und haben dir nicht gedient?‹
\bibleverse{45} Dann wird er ihnen zur Antwort geben: ›Wahrlich ich sage
euch: Alles, was ihr einem von diesen Geringsten nicht getan habt, das
habt ihr auch mir nicht getan.‹ \bibleverse{46} Und diese werden in die
ewige Strafe gehen, die Gerechten aber in das ewige Leben.«\textless sup
title=``Dan 12,2''\textgreater✲

\hypertarget{viii.-die-leidensgeschichte-und-der-tod-jesu-kap.-26-27}{%
\subsection{VIII. Die Leidensgeschichte und der Tod Jesu (Kap.
26-27)}\label{viii.-die-leidensgeschichte-und-der-tod-jesu-kap.-26-27}}

\hypertarget{letzte-leidensankuxfcndigung-jesu-mordanschlag-der-fuxfchrer-des-volkes}{%
\subsubsection{1. Letzte Leidensankündigung Jesu; Mordanschlag der
Führer des
Volkes}\label{letzte-leidensankuxfcndigung-jesu-mordanschlag-der-fuxfchrer-des-volkes}}

\hypertarget{section-25}{%
\section{26}\label{section-25}}

\bibleverse{1} Als nun Jesus alle diese Reden beendet hatte, sagte er zu
seinen Jüngern: \bibleverse{2} »Ihr wißt, daß übermorgen das Passah
stattfindet; da wird der Menschensohn zur Kreuzigung überliefert.«

\bibleverse{3} Damals kamen die Hohenpriester und die Ältesten des
Volkes im Palaste des Hohenpriesters namens Kaiphas zusammen
\bibleverse{4} und berieten sich in der Absicht, Jesus mit List
festzunehmen und zu töten. \bibleverse{5} Dabei sagten sie aber: »Nur
nicht während des Festes, damit keine Unruhen unter dem Volk entstehen!«

\hypertarget{salbung-jesu-in-bethanien}{%
\subsubsection{2. Salbung Jesu in
Bethanien}\label{salbung-jesu-in-bethanien}}

\bibleverse{6} Als Jesus sich aber in Bethanien im Hause Simons des
(einstmals) Aussätzigen befand, \bibleverse{7} trat eine Frau mit einem
Alabastergefäß voll kostbaren Salböls an ihn heran und goß es ihm über
das Haupt, während er bei Tische saß\textless sup title=``bzw.
lag''\textgreater✲. \bibleverse{8} Als die Jünger das sahen, wurden sie
unwillig und sagten: »Wozu diese Verschwendung? \bibleverse{9} Dieses
(Salböl) hätte man doch teuer verkaufen und den Erlös den Armen geben
können.« \bibleverse{10} Als Jesus es merkte, sagte er zu ihnen: »Warum
macht ihr der Frau Vorwürfe? Sie hat ja doch ein gutes\textless sup
title=``oder: schönes''\textgreater✲ Werk an mir getan! \bibleverse{11}
Denn die Armen habt ihr allezeit bei euch, mich aber habt ihr nicht
allezeit. \bibleverse{12} Daß sie dieses Öl auf meinen Leib gegossen
hat, das hat sie für mein Begräbnis getan. \bibleverse{13} Wahrlich ich
sage euch: Wo immer diese Heilsbotschaft in der ganzen Welt verkündet
wird, da wird man auch von dem, was diese Frau getan hat, zum ehrenden
Gedächtnis für sie erzählen.«

\hypertarget{verrat-des-judas}{%
\subsubsection{3. Verrat des Judas}\label{verrat-des-judas}}

\bibleverse{14} Hierauf ging einer von den Zwölfen namens Judas Iskariot
zu den Hohenpriestern \bibleverse{15} und sagte: »Was wollt ihr mir
geben, daß ich ihn euch in die Hände liefere?« Da zahlten sie ihm
dreißig Silberstücke aus\textless sup title=``Sach 11,12''\textgreater✲.
\bibleverse{16} Von da an suchte er nach einer guten Gelegenheit, um ihn
zu überliefern\textless sup title=``=~zu verraten''\textgreater✲.

\hypertarget{vorbereitung-des-passahmahls}{%
\subsubsection{4. Vorbereitung des
Passahmahls}\label{vorbereitung-des-passahmahls}}

\bibleverse{17} Am ersten Tage der ungesäuerten Brote aber traten die
Jünger zu Jesus und fragten ihn: »Wo sollen wir dir alles vorbereiten,
damit du das Passahmahl halten kannst?« \bibleverse{18} Er antwortete:
»Geht in die Stadt zu dem und dem und sagt zu ihm: ›Der Meister läßt dir
sagen: Meine Zeit ist nahe; bei dir will ich das Passahmahl mit meinen
Jüngern halten.‹« \bibleverse{19} Die Jünger taten, wie Jesus ihnen
aufgetragen hatte, und richteten das Passahmahl zu.

\hypertarget{jesu-letztes-mahl-im-juxfcngerkreise-enthuxfcllung-des-verrats-des-judas-einsetzung-des-heiligen-abendmahls}{%
\subsubsection{5. Jesu letztes Mahl im Jüngerkreise; Enthüllung des
Verrats des Judas; Einsetzung des heiligen
Abendmahls}\label{jesu-letztes-mahl-im-juxfcngerkreise-enthuxfcllung-des-verrats-des-judas-einsetzung-des-heiligen-abendmahls}}

\bibleverse{20} Als es dann Abend geworden war, setzte er sich mit den
zwölf Jüngern zu Tisch; \bibleverse{21} und während des Essens sagte er:
»Wahrlich ich sage euch: Einer von euch wird mich ausliefern✲!«
\bibleverse{22} Da wurden sie tief betrübt und fragten ihn, einer nach
dem andern: »Ich bin es doch nicht etwa, Herr?« \bibleverse{23} Er
antwortete: »Der die Hand zusammen mit mir in die Schüssel getaucht hat,
der wird mich ausliefern✲. \bibleverse{24} Der Menschensohn geht zwar
dahin, wie über ihn in der Schrift steht; doch wehe dem Menschen, durch
den der Menschensohn verraten wird! Für diesen Menschen wäre es
besser\textless sup title=``oder: das Beste''\textgreater✲, er wäre
nicht geboren!« \bibleverse{25} Da nahm Judas, der ihn verraten wollte,
das Wort und fragte: »Ich bin es doch nicht etwa, Rabbi✲?« Er erwiderte
ihm: »Doch, du bist es.« \bibleverse{26} Während des Essens aber nahm
Jesus das\textless sup title=``oder: ein''\textgreater✲ Brot, sprach den
Lobpreis (Gottes), brach das Brot und gab es den Jüngern mit den Worten:
»Nehmt, esset! Dies ist mein Leib.« \bibleverse{27} Dann nahm er
einen\textless sup title=``oder: den''\textgreater✲ Becher, sprach das
Dankgebet und gab ihnen den mit den Worten: »Trinkt alle daraus!
\bibleverse{28} Denn dies ist mein Blut, das Blut des (neuen)
Bundes\textless sup title=``2.Mose 24,8; Sach 9,11''\textgreater✲, das
für viele vergossen wird zur Vergebung der Sünden. \bibleverse{29} Ich
sage euch aber: Ich werde von nun an von diesem Erzeugnis des Weinstocks
nicht mehr trinken bis zu jenem Tage, an dem ich es mit euch neu trinken
werde im Reiche meines Vaters.«

\hypertarget{jesus-in-gethsemane}{%
\subsubsection{6. Jesus in Gethsemane}\label{jesus-in-gethsemane}}

\hypertarget{a-gang-nach-gethsemane-vorhersagung-des-uxe4rgernisses-der-juxfcnger-und-der-verleugnung-des-petrus-treugeluxfcbde-der-juxfcnger}{%
\paragraph{a) Gang nach Gethsemane; Vorhersagung des Ärgernisses der
Jünger und der Verleugnung des Petrus; Treugelübde der
Jünger}\label{a-gang-nach-gethsemane-vorhersagung-des-uxe4rgernisses-der-juxfcnger-und-der-verleugnung-des-petrus-treugeluxfcbde-der-juxfcnger}}

\bibleverse{30} Nachdem sie dann den Lobpreis (Ps 115-118) gesungen
hatten, gingen sie (aus der Stadt) hinaus an den Ölberg. \bibleverse{31}
Dabei✲ sagte Jesus zu ihnen: »Ihr werdet alle in dieser Nacht an mir
Anstoß nehmen\textless sup title=``oder: irre werden''\textgreater✲;
denn es steht geschrieben\textless sup title=``Sach 13,7''\textgreater✲:
›Ich werde den Hirten niederschlagen, dann werden die Schafe der Herde
sich zerstreuen.‹ \bibleverse{32} Nach meiner Auferweckung aber werde
ich euch voraus nach Galiläa gehen.« \bibleverse{33} Da antwortete ihm
Petrus: »Mögen auch alle an dir Anstoß nehmen\textless sup title=``oder:
irre werden''\textgreater✲: ich werde niemals an dir Anstoß
nehmen\textless sup title=``oder: irre werden''\textgreater✲!«
\bibleverse{34} Jesus erwiderte ihm: »Wahrlich ich sage dir: Noch in
dieser Nacht, ehe der Hahn kräht, wirst du mich dreimal verleugnen.«
\bibleverse{35} Petrus antwortete ihm: »Wenn ich auch mit dir sterben
müßte, werde ich dich doch niemals verleugnen!« Das gleiche versicherten
auch die anderen Jünger alle.

\hypertarget{b-jesu-seelenkampf-und-gebet-in-gethsemane-schwuxe4che-der-juxfcnger}{%
\paragraph{b) Jesu Seelenkampf und Gebet in Gethsemane; Schwäche der
Jünger}\label{b-jesu-seelenkampf-und-gebet-in-gethsemane-schwuxe4che-der-juxfcnger}}

\bibleverse{36} Hierauf kam Jesus mit ihnen an einen Ort namens
Gethsemane\textless sup title=``d.h. Ölkelter''\textgreater✲ und sagte
zu den Jüngern: »Setzt euch hier nieder, während ich dorthin gehe und
bete!« \bibleverse{37} Dann nahm er Petrus und die beiden Söhne des
Zebedäus mit sich und fing an zu trauern und zu zagen\textless sup
title=``vgl. Mk 14,33''\textgreater✲. \bibleverse{38} Da sagte er zu
ihnen: »Tiefbetrübt ist meine Seele bis zum Tode; bleibt hier und haltet
euch wach mit mir!« \bibleverse{39} Nachdem er dann ein wenig
weitergegangen war, warf er sich auf sein Angesicht nieder und betete
mit den Worten: »Mein Vater, wenn es möglich ist, so laß diesen Kelch an
mir vorübergehen! Doch nicht wie ich will, sondern wie du willst!«
\bibleverse{40} Hierauf ging er zu den Jüngern zurück und fand sie
schlafend und sagte zu Petrus: »So wenig seid ihr imstande gewesen, eine
einzige Stunde mit mir zu wachen? \bibleverse{41} Wachet, und betet,
damit ihr nicht in Versuchung geratet! Der Geist ist willig, das Fleisch
aber ist schwach.« \bibleverse{42} Wiederum ging er zum zweitenmal weg
und betete mit den Worten: »Mein Vater, wenn dieser Kelch nicht (an mir)
vorübergehen kann, ohne daß ich ihn trinke, so geschehe dein Wille!«
\bibleverse{43} Als er dann zurückkam, fand er sie (wieder) schlafend,
denn die Augen fielen ihnen vor Müdigkeit zu. \bibleverse{44} Da verließ
er sie, ging wieder weg und betete zum drittenmal, wieder mit denselben
Worten. \bibleverse{45} Hierauf kehrte er zu den Jüngern zurück und
sagte zu ihnen: »Schlaft ein andermal und ruht euch aus! Doch jetzt ist
die Stunde gekommen, daß der Menschensohn Sündern in die Hände geliefert
wird! \bibleverse{46} Steht auf, wir wollen gehen! Seht, mein Verräter
ist nahe gekommen!«

\hypertarget{c-gefangennahme-jesu-flucht-der-juxfcnger}{%
\paragraph{c) Gefangennahme Jesu; Flucht der
Jünger}\label{c-gefangennahme-jesu-flucht-der-juxfcnger}}

\bibleverse{47} Während er noch redete, da kam plötzlich Judas, einer
von den Zwölfen, und mit ihm eine große Schar mit Schwertern und
Knütteln, von den Hohenpriestern und Ältesten des Volkes her✲.
\bibleverse{48} Sein Verräter hatte aber ein Zeichen mit ihnen
verabredet, nämlich: »Der, den ich küssen werde, der ist's; den nehmt
fest!« \bibleverse{49} Er trat also sogleich auf Jesus zu mit den
Worten: »Sei gegrüßt, Rabbi✲!« und küßte ihn. \bibleverse{50} Jesus aber
sagte zu ihm: »Freund, (tu das) wozu du hergekommen bist!« Hierauf
traten sie herzu, legten Hand an Jesus und nahmen ihn fest.
\bibleverse{51} Einer jedoch von den Begleitern Jesu streckte die Hand
aus, zog sein Schwert, schlug damit nach dem Knechte des Hohenpriesters
und hieb ihm das Ohr ab. \bibleverse{52} Da sagte Jesus zu ihm: »Stecke
dein Schwert wieder an seinen Ort\textless sup title=``=~in die
Scheide''\textgreater✲! Denn wer zum Schwerte greift, wird durchs
Schwert umkommen! \bibleverse{53} Oder meinst du, ich könnte meinen
Vater nicht bitten, und er würde mir nicht sogleich mehr als zwölf
Legionen\textless sup title=``=~Heerscharen; vgl. Mk 5,9''\textgreater✲
Engel zu Hilfe senden? \bibleverse{54} Wie sollten dann aber die
Aussprüche der Schrift erfüllt werden, daß es so geschehen muß?«
\bibleverse{55} In jener Stunde sagte Jesus zu den Haufen: »Wie gegen
einen Räuber seid ihr mit Schwertern und Knütteln ausgezogen, um mich
gefangen zu nehmen. Täglich habe ich im Tempel gesessen und gelehrt, und
ihr habt mich nicht festgenommen. \bibleverse{56} Dies alles ist aber
geschehen, damit die Schriften der Propheten erfüllt werden!« Hierauf
verließen ihn die Jünger alle und ergriffen die Flucht.

\hypertarget{jesu-verhuxf6r-und-verurteilung-vor-dem-hohenpriester-und-dem-hohen-rat}{%
\subsubsection{7. Jesu Verhör und Verurteilung vor dem Hohenpriester und
dem Hohen
Rat}\label{jesu-verhuxf6r-und-verurteilung-vor-dem-hohenpriester-und-dem-hohen-rat}}

\bibleverse{57} Die Männer aber, die Jesus festgenommen hatten, führten
ihn zu dem Hohenpriester Kaiphas ab, wo die Schriftgelehrten und die
Ältesten sich versammelten. \bibleverse{58} Petrus aber folgte ihm von
fern bis zum Palast des Hohenpriesters, ging hinein und setzte sich dort
unter den Dienern hin, um den Ausgang der Sache abzuwarten.

\bibleverse{59} Die Hohenpriester aber und der gesamte Hohe Rat suchten
nach einer falschen Zeugenaussage gegen Jesus, um ihn zum Tode
verurteilen zu können; \bibleverse{60} doch sie fanden keine, obgleich
viele falsche Zeugen herzutraten. Zuletzt aber traten zwei auf
\bibleverse{61} und sagten aus: »Dieser Mensch hat behauptet: ›Ich kann
den Tempel Gottes abbrechen und ihn in drei Tagen wieder aufbauen.‹«
\bibleverse{62} Da stand der Hohepriester auf und fragte ihn:
»Entgegnest du nichts auf das, was diese Zeugen gegen dich aussagen?«
Jesus aber schwieg. \bibleverse{63} Da sagte der Hohepriester zu ihm:
»Ich beschwöre dich bei dem lebendigen Gott: Sage uns, bist du
Christus\textless sup title=``=~der Messias''\textgreater✲, der Sohn
Gottes?« \bibleverse{64} Da gab Jesus ihm zur Antwort: »Ja, ich bin es!
Doch ich tue euch kund: Von jetzt an werdet ihr den Menschensohn sitzen
sehen zur Rechten der Macht\textless sup title=``=~des
Allmächtigen''\textgreater✲ und kommen auf den Wolken des
Himmels.«\textless sup title=``Dan 7,13; Ps 110,1''\textgreater✲
\bibleverse{65} Da zerriß der Hohepriester seine Kleider und sagte: »Er
hat Gott gelästert! Wozu brauchen wir noch Zeugen? Jetzt habt ihr selbst
die Gotteslästerung gehört! Was urteilt ihr?« \bibleverse{66} Sie gaben
die Erklärung ab: »Er ist des Todes schuldig!« \bibleverse{67} Hierauf
spien sie ihm ins Gesicht und schlugen ihn mit den Fäusten; andere gaben
ihm Backenstreiche \bibleverse{68} und sagten: »Weissage uns, Christus✲!
Wer ist es, der dich geschlagen hat?«

\hypertarget{verleugnung-und-reue-des-petrus}{%
\subsubsection{8. Verleugnung und Reue des
Petrus}\label{verleugnung-und-reue-des-petrus}}

\bibleverse{69} Petrus aber saß (unterdessen) draußen im Hof. Da trat
eine Magd auf ihn zu und sagte: »Du bist auch bei Jesus, dem Galiläer,
gewesen!« \bibleverse{70} Er aber leugnete vor allen und sagte: »Ich
verstehe nicht, was du da sagst!« \bibleverse{71} Als er dann in die
Torhalle hinausgegangen war, bemerkte ihn eine andere Magd und sagte zu
den Leuten dort: »Dieser ist auch mit Jesus, dem Nazoräer\textless sup
title=``vgl. 2,23''\textgreater✲, zusammen gewesen!« \bibleverse{72} Da
leugnete er wieder, (diesmal) mit einem Eid: »ich kenne den Menschen
nicht!« \bibleverse{73} Nach einer kleinen Weile aber traten die Leute,
die dort standen, hinzu und sagten zu Petrus: »Wahrhaftig, du gehörst
auch zu ihnen: schon deine Sprache✲ verrät dich!« \bibleverse{74} Da
fing er an, sich zu verfluchen und zu schwören: »Ich kenne den Menschen
nicht!«, und sogleich darauf krähte der Hahn. \bibleverse{75} Da dachte
Petrus an das Wort Jesu, der ihm gesagt hatte\textless sup title=``vgl.
V.34''\textgreater✲: »Noch ehe der Hahn kräht, wirst du mich dreimal
verleugnen.« Und er ging hinaus und weinte bitterlich.

\hypertarget{letzte-beratung-des-hohen-rates-auslieferung-des-verurteilten-an-den-ruxf6mischen-statthalter-pilatus-ende-des-judas}{%
\subsubsection{9. Letzte Beratung des Hohen Rates; Auslieferung des
Verurteilten an den römischen Statthalter Pilatus; Ende des
Judas}\label{letzte-beratung-des-hohen-rates-auslieferung-des-verurteilten-an-den-ruxf6mischen-statthalter-pilatus-ende-des-judas}}

\hypertarget{section-26}{%
\section{27}\label{section-26}}

\bibleverse{1} Als es hierauf Tag geworden war, faßten alle
Hohenpriester und die Ältesten des Volkes einen Beschluß gegen Jesus, um
seine Hinrichtung zu erreichen. \bibleverse{2} Sie ließen ihn dann
fesseln und abführen und übergaben ihn dem Statthalter Pontius Pilatus.

\bibleverse{3} Als jetzt Judas, sein Verräter, erkannte, daß er (Jesus)
verurteilt worden war, bereute er seine Tat. Und er brachte die dreißig
Silberstücke den Hohenpriestern und Ältesten zurück \bibleverse{4} mit
den Worten: »Ich habe unrecht getan, daß ich unschuldiges Blut
überantwortet✲ habe!« Sie aber erwiderten: »Was geht das uns an? Da sich
du selber zu!« \bibleverse{5} Da warf er das Geld in das Tempelhaus und
machte sich davon, ging hin und erhängte sich. \bibleverse{6} Die
Hohenpriester aber nahmen das Geld und sagten: »Es geht nicht an, daß
wir es in den Tempelschatz tun, denn es ist Blutgeld.«\textless sup
title=``vgl. 5.Mose 23,18-19''\textgreater✲ \bibleverse{7} Nachdem sie
dann einen Beschluß gefaßt hatten, kauften sie für das Geld den
›Töpferacker‹ zum Begräbnisplatz für die Fremden; \bibleverse{8} daher
führt jener Acker den Namen ›Blutacker‹ (hebräisch Hakeldama) bis auf
den heutigen Tag. \bibleverse{9} Damals erfüllte sich das Wort des
Propheten Jeremia\textless sup title=``Sach 11,12-13; Jer
32,6''\textgreater✲: »Sie nahmen die dreißig Silberstücke, den
Geldbetrag für den so Gewerteten, auf den man von seiten der Israeliten
einen solchen Preis ausgesetzt hatte, \bibleverse{10} und gaben sie für
den Töpferacker, wie der Herr es mir geboten hatte.«

\hypertarget{verhuxf6r-jesu-vor-pilatus-jesus-von-dem-volke-verworfen-seine-verurteilung-und-geiuxdfelung}{%
\subsubsection{10. Verhör Jesu vor Pilatus; Jesus von dem Volke
verworfen; seine Verurteilung und
Geißelung}\label{verhuxf6r-jesu-vor-pilatus-jesus-von-dem-volke-verworfen-seine-verurteilung-und-geiuxdfelung}}

\bibleverse{11} Jesus aber wurde dem Statthalter vorgeführt, und dieser
befragte ihn mit den Worten: »Bist du der König der Juden?« Jesus
antwortete: »Ja, ich bin es.«\textless sup title=``vgl. Mk
15,2''\textgreater✲ \bibleverse{12} Während er dann von den
Hohenpriestern und Ältesten angeklagt wurde, gab er keine Antwort.
\bibleverse{13} Da fragte ihn Pilatus: »Hörst du nicht, was sie alles
gegen dich aussagen?« \bibleverse{14} Doch er antwortete ihm auf keine
einzige Frage, so daß der Statthalter sich sehr verwunderte.

\hypertarget{jesus-und-barabbas}{%
\paragraph{Jesus und Barabbas}\label{jesus-und-barabbas}}

\bibleverse{15} An jedem (Passah-) Fest aber pflegte der Statthalter dem
Volke einen Gefangenen nach ihrer Wahl freizugeben. \bibleverse{16} Man
hatte aber damals einen berüchtigten Gefangenen namens Barabbas (in
Haft). \bibleverse{17} Als die Menge nun versammelt war, fragte Pilatus
sie: »Wen soll ich euch freigeben, Barabbas oder Jesus, den man
Christus\textless sup title=``=~den Messias''\textgreater✲ nennt?«
\bibleverse{18} Er wußte nämlich wohl, daß sie ihn aus Neid
überantwortet hatten. \bibleverse{19} Während er aber auf dem
Richterstuhl saß, schickte seine Frau zu ihm und ließ ihm sagen: »Habe
du mit diesem Gerechten nichts zu schaffen! Denn ich habe heute nacht im
Traum viel um seinetwillen ausgestanden.« \bibleverse{20} Die
Hohenpriester und Ältesten aber redeten auf das Volk ein, sie möchten
sich den Barabbas erbitten, Jesus dagegen hinrichten lassen.
\bibleverse{21} Da richtete der Statthalter (nochmals) die Frage an sie:
»Wen von den beiden soll ich euch freigeben?« Sie riefen: »Barabbas!«
\bibleverse{22} Pilatus fragte sie weiter: »Was soll ich denn mit Jesus
machen, den man Christus✲ nennt?« Sie riefen alle: »Ans Kreuz mit ihm!«
\bibleverse{23} Der Statthalter entgegnete ihnen: »Was hat er denn Böses
getan?« Sie schrien nur noch lauter: »Ans Kreuz mit ihm!«
\bibleverse{24} Als nun Pilatus einsah, daß er nichts erreichte, der
Lärm vielmehr immer größer wurde, ließ er sich Wasser reichen, wusch
sich vor dem Volk die Hände und sagte: »Ich bin am Blut dieses Gerechten
unschuldig; seht ihr zu!« \bibleverse{25} Da antwortete das gesamte Volk
mit dem Ruf: »Sein Blut (komme) über uns und über unsere Kinder!«
\bibleverse{26} Daraufhin gab er ihnen den Barabbas frei, Jesus aber
ließ er geißeln und überwies ihn dann (den Soldaten) zur Kreuzigung.

\hypertarget{jesu-verspottung-und-miuxdfhandlung-durch-die-ruxf6mischen-soldaten}{%
\subsubsection{11. Jesu Verspottung und Mißhandlung durch die römischen
Soldaten}\label{jesu-verspottung-und-miuxdfhandlung-durch-die-ruxf6mischen-soldaten}}

\bibleverse{27} Hierauf nahmen die Soldaten des Statthalters Jesus mit
sich in die Statthalterei\textless sup title=``vgl. Mk
15,16''\textgreater✲ und riefen dort die ganze Kohorte✲ gegen ihn
zusammen. \bibleverse{28} Dann entkleideten sie ihn und legten ihm einen
scharlachroten Mantel um, \bibleverse{29} flochten aus Dornen eine
Krone, die\textless sup title=``oder: einen Kranz, den''\textgreater✲
sie ihm aufs Haupt setzten, und (gaben) ihm ein Rohr in die rechte Hand;
darauf warfen sie sich vor ihm auf die Knie nieder und verhöhnten ihn
mit den Worten: »Sei gegrüßt, Judenkönig!« \bibleverse{30} Auch spien
sie ihn an, nahmen das Rohr und schlugen ihn damit aufs Haupt.
\bibleverse{31} Nachdem sie ihn so verspottet hatten, nahmen sie ihm den
Mantel wieder ab und legten ihm seine eigenen Kleider an; dann führten
sie ihn zur Kreuzigung ab.

\hypertarget{jesu-todesgang-nach-golgatha-seine-kreuzigung-und-sein-sterben}{%
\subsubsection{12. Jesu Todesgang nach Golgatha, seine Kreuzigung und
sein
Sterben}\label{jesu-todesgang-nach-golgatha-seine-kreuzigung-und-sein-sterben}}

\bibleverse{32} Während sie aber (zur Stadt) hinauszogen, trafen sie
einen Mann aus Cyrene namens Simon an; diesen zwangen sie, ihm das Kreuz
zu tragen. \bibleverse{33} Als sie dann auf einem Platz namens Golgatha,
das bedeutet Schädelstätte, angekommen waren, \bibleverse{34} gaben sie
ihm Wein zu trinken, der mit Galle vermischt war\textless sup title=``Ps
69,22''\textgreater✲; doch als er ihn gekostet hatte, wollte er ihn
nicht trinken. \bibleverse{35} Nachdem sie ihn dann gekreuzigt hatten,
verteilten sie seine Kleidungsstücke unter sich, indem sie das Los um
sie warfen\textless sup title=``Ps 22,19''\textgreater✲, \bibleverse{36}
setzten sich hierauf nieder und bewachten ihn dort. \bibleverse{37} Über
seinem Haupt hatten sie eine Inschrift angebracht, die seine Schuld
angeben sollte; sie lautete: »Dieser ist Jesus, der König der Juden.«
\bibleverse{38} Sodann wurden zwei Räuber mit ihm gekreuzigt, der eine
zu seiner Rechten, der andere zu seiner Linken. \bibleverse{39} Die
Vorübergehenden aber schmähten ihn, wobei sie den Kopf
schüttelten\textless sup title=``Ps 22,8; 109,25''\textgreater✲
\bibleverse{40} und ausriefen: »Du wolltest ja den Tempel abbrechen und
ihn in drei Tagen wieder aufbauen! Hilf dir nun selbst, wenn du Gottes
Sohn bist, und steige vom Kreuz herab!« \bibleverse{41} Ebenso
verhöhnten ihn auch die Hohenpriester samt den Schriftgelehrten und
Ältesten mit den Worten: \bibleverse{42} »Anderen hat er geholfen, sich
selber kann er nicht helfen! Er ist der König von Israel: so steige er
jetzt vom Kreuz herab, dann wollen wir an ihn glauben! \bibleverse{43}
Er hat auf Gott vertraut: der rette ihn jetzt, wenn er ihm
wohlwill\textless sup title=``oder: Wohlgefallen an ihm
hat''\textgreater✲! Er hat ja doch behauptet: ›Ich bin Gottes Sohn.‹«
\bibleverse{44} In der gleichen Weise schmähten ihn auch die Räuber, die
mit ihm gekreuzigt waren.

\hypertarget{jesu-sterben-die-wunderzeichen-bei-seinem-tode}{%
\paragraph{Jesu Sterben; die Wunderzeichen bei seinem
Tode}\label{jesu-sterben-die-wunderzeichen-bei-seinem-tode}}

\bibleverse{45} Aber von der sechsten Stunde an trat eine Finsternis
über das ganze Land ein bis zur neunten Stunde. \bibleverse{46} Um die
neunte Stunde aber rief Jesus mit lauter Stimme aus: »Eli, Eli, lema
sabachthani?«, das heißt: »Mein Gott, mein Gott, warum hast du mich
verlassen?«\textless sup title=``Ps 22,2''\textgreater✲. \bibleverse{47}
Als einige von den dort Stehenden dies hörten, sagten sie: »Der ruft den
Elia!« \bibleverse{48} Und sogleich lief einer von ihnen hin, nahm einen
Schwamm, tränkte ihn mit Essig, steckte ihn an ein Rohr und wollte Jesus
trinken lassen. \bibleverse{49} Die anderen aber sagten: »Laß das! Wir
wollen doch sehen, ob Elia wirklich kommt, um ihm zu helfen.«
\bibleverse{50} Jesus aber stieß noch einmal einen lauten Schrei aus und
gab dann seinen Geist auf. \bibleverse{51} Da zerriß der Vorhang im
Tempel von oben bis unten in zwei Stücke, die Erde erbebte und die
Felsen zersprangen, \bibleverse{52} die Gräber taten sich auf, und viele
Leiber der entschlafenen Heiligen wurden auferweckt, \bibleverse{53}
kamen nach seiner Auferstehung aus ihren Gräbern hervor, gingen in die
heilige Stadt hinein und erschienen vielen. \bibleverse{54} Als aber der
Hauptmann und seine Leute, die Jesus zu bewachen hatten, das Erdbeben
und was (sonst noch) geschah, sahen, gerieten sie in große Furcht und
sagten: »Dieser ist wirklich Gottes Sohn gewesen!«

\bibleverse{55} Es waren dort aber auch viele Frauen zugegen, die von
weitem zuschauten; sie waren Jesus aus Galiläa nachgefolgt und hatten
ihm Dienste geleistet. \bibleverse{56} Unter ihnen befanden sich Maria
von Magdala und Maria, die Mutter des Jakobus und des Joseph, und die
Mutter der Söhne des Zebedäus.

\hypertarget{grablegung-jesu-bestellung-der-grabeswuxe4chter}{%
\subsubsection{13. Grablegung Jesu; Bestellung der
Grabeswächter}\label{grablegung-jesu-bestellung-der-grabeswuxe4chter}}

\bibleverse{57} Als es dann Spätnachmittag geworden war, kam ein reicher
Mann von Arimathäa namens Joseph, der gleichfalls ein Jünger Jesu
geworden war; \bibleverse{58} dieser begab sich zu Pilatus und bat ihn
um den Leichnam Jesu. Da gab Pilatus den Befehl, man solle ihm den
Leichnam übergeben. \bibleverse{59} Joseph nahm nun den Leichnam,
wickelte ihn in reine Leinwand \bibleverse{60} und legte ihn in das neue
Grab, das er für sich selbst im Felsen hatte aushauen lassen; dann
wälzte er einen großen Stein vor den Eingang des Grabes und entfernte
sich. \bibleverse{61} Es waren aber dort Maria von Magdala und die
andere Maria zugegen; die saßen dem Grabe gegenüber.~--

\bibleverse{62} Am nächsten Tage aber, der auf den Rüsttag\textless sup
title=``d.h. den Freitag''\textgreater✲ folgte, versammelten sich die
Hohenpriester und Pharisäer bei Pilatus \bibleverse{63} und sagten:
»Herr, es ist uns eingefallen, daß jener Betrüger bei seinen Lebzeiten
angekündigt hat: ›Nach drei Tagen werde ich auferweckt.‹ \bibleverse{64}
Gib also Befehl, daß das Grab bis zum dritten Tag sicher bewacht wird;
sonst könnten seine Jünger kommen, könnten ihn stehlen und dann zum
Volke sagen: ›Er ist von den Toten auferweckt worden‹; dann würde der
letzte Betrug noch schlimmer sein als der erste.« \bibleverse{65}
Pilatus antwortete ihnen: »Ihr sollt eine Wachmannschaft haben; geht hin
und verwahrt (das Grab) sicher, so gut ihr könnt!« \bibleverse{66} Da
gingen sie hin und sicherten das Grab unter Hinzuziehung der
Wachmannschaft, nachdem sie den Stein versiegelt hatten.

\hypertarget{ix.-der-auferstehungsbericht-kap.-28}{%
\subsection{IX. Der Auferstehungsbericht (Kap.
28)}\label{ix.-der-auferstehungsbericht-kap.-28}}

\hypertarget{die-beiden-frauen-beim-leeren-grab-am-ostermorgen-jesu-erste-erscheinung-betrug-der-fuxfchrer-des-volkes}{%
\subsubsection{1. Die beiden Frauen beim leeren Grab am Ostermorgen;
Jesu erste Erscheinung; Betrug der Führer des
Volkes}\label{die-beiden-frauen-beim-leeren-grab-am-ostermorgen-jesu-erste-erscheinung-betrug-der-fuxfchrer-des-volkes}}

\hypertarget{section-27}{%
\section{28}\label{section-27}}

\bibleverse{1} Nach Ablauf des Sabbats aber, als der erste Tag nach dem
Sabbat\textless sup title=``=~der erste Wochentag''\textgreater✲
anbrechen wollte, gingen Maria von Magdala und die andere Maria hin, um
nach dem Grabe zu sehen. \bibleverse{2} Da entstand plötzlich ein
starkes Erdbeben; denn ein Engel des Herrn, der vom Himmel herabgekommen
und herangetreten war, wälzte den Stein weg und setzte sich oben darauf.
\bibleverse{3} Sein Aussehen war (leuchtend) wie der Blitz und sein
Gewand weiß wie der Schnee. \bibleverse{4} Aus Furcht vor ihm zitterten
die Wächter und wurden wie tot. \bibleverse{5} Der Engel aber wandte
sich an die Frauen mit den Worten: »Fürchtet ihr euch nicht! Denn ich
weiß, daß ihr Jesus, den Gekreuzigten, sucht. \bibleverse{6} Er ist
nicht (mehr) hier, denn er ist auferweckt worden, wie er es vorausgesagt
hat. Kommt her, seht euch die Stelle an, wo er gelegen hat.
\bibleverse{7} Geht nun eilends hin und sagt seinen Jüngern: »Er ist von
den Toten auferweckt worden und geht euch voran nach Galiläa; dort
werdet ihr ihn wiedersehen; beachtet wohl, was ich euch gesagt habe!«
\bibleverse{8} Da gingen sie eilends vom Grabe weg voller Furcht und
(zugleich) voll großer Freude und eilten davon, um seinen Jüngern die
Botschaft zu bringen. \bibleverse{9} Und siehe! Jesus kam ihnen entgegen
mit den Worten: »Seid gegrüßt!« Da gingen sie auf ihn zu, umfaßten seine
Füße und warfen sich anbetend vor ihm nieder. \bibleverse{10} Hierauf
sagte Jesus zu ihnen: »Fürchtet euch nicht! Geht hin und verkündigt
meinen Brüdern, daß sie nach Galiläa gehen sollen: dort werden sie mich
wiedersehen.«

\hypertarget{die-erlogene-behauptung-der-fuxfchrer-des-volkes-von-der-entwendung-des-leichnams-jesu}{%
\paragraph{Die erlogene Behauptung der Führer des Volkes von der
Entwendung des Leichnams
Jesu}\label{die-erlogene-behauptung-der-fuxfchrer-des-volkes-von-der-entwendung-des-leichnams-jesu}}

\bibleverse{11} Während sie nun hingingen, begaben sich einige von der
Wachmannschaft (des Grabes) in die Stadt und meldeten den Hohenpriestern
alles, was sich zugetragen hatte. \bibleverse{12} Nachdem diese sich mit
den Ältesten versammelt und sich beraten\textless sup title=``oder:
einen Beschluß gefaßt''\textgreater✲ hatten, gaben sie den Soldaten
reichlich Geld \bibleverse{13} und sagten: »Macht folgende Aussagen:
›Seine Jünger sind bei Nacht gekommen und haben ihn gestohlen, während
wir schliefen.‹ \bibleverse{14} Und wenn dies dem Statthalter zu Ohren
kommen sollte, wollen wir ihn schon beschwichtigen und dafür sorgen, daß
ihr keine Angst zu haben braucht.« \bibleverse{15} Da nahmen sie (die
Soldaten) das Geld und verfuhren nach der empfangenen Weisung; und so
ist dieses Gerede bei den Juden in Umlauf gekommen bis zum heutigen Tag.

\hypertarget{jesu-erscheinung-auf-dem-berge-in-galiluxe4a-sein-letzter-befehl-an-die-elf-juxfcnger}{%
\subsubsection{2. Jesu Erscheinung auf dem Berge in Galiläa; sein
letzter Befehl an die elf
Jünger}\label{jesu-erscheinung-auf-dem-berge-in-galiluxe4a-sein-letzter-befehl-an-die-elf-juxfcnger}}

\bibleverse{16} Die elf Jünger aber begaben sich nach Galiläa auf den
Berg, wohin Jesus sie beschieden hatte; \bibleverse{17} und als sie ihn
erblickten, warfen sie sich vor ihm nieder; einige aber hegten Zweifel.
\bibleverse{18} Da trat Jesus herzu und redete sie mit den Worten an:
»Mir ist alle Gewalt im Himmel und auf Erden verliehen. \bibleverse{19}
Darum gehet hin und macht alle Völker zu (meinen) Jüngern\textless sup
title=``oder: zu Schülern''\textgreater✲: tauft sie auf den Namen des
Vaters, des Sohnes und des heiligen Geistes \bibleverse{20} und lehrt
sie alles halten, was ich euch geboten✲ habe. Und wisset wohl: Ich bin
bei euch alle Tage bis ans Ende der Weltzeit!«
