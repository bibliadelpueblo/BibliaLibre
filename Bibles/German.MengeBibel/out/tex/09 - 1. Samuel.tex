\hypertarget{das-erste-buch-samuel}{%
\section{DAS ERSTE BUCH SAMUEL}\label{das-erste-buch-samuel}}

\hypertarget{i.-samuel-und-saul-kap.-1-15}{%
\subsection{I. Samuel und Saul (Kap.
1-15)}\label{i.-samuel-und-saul-kap.-1-15}}

\hypertarget{samuels-jugendgeschichte-kap.-1-3}{%
\subsubsection{1. Samuels Jugendgeschichte (Kap.
1-3)}\label{samuels-jugendgeschichte-kap.-1-3}}

\hypertarget{a-samuels-geburt-und-weihe-zum-diener-des-herrn-in-silo-lobgesang-der-hanna}{%
\paragraph{a) Samuels Geburt und Weihe zum Diener des Herrn in Silo;
Lobgesang der
Hanna}\label{a-samuels-geburt-und-weihe-zum-diener-des-herrn-in-silo-lobgesang-der-hanna}}

\hypertarget{section}{%
\section{1}\label{section}}

1Es war einst ein Mann aus (den Bürgern von) Ramath, ein Zuphit vom
Gebirge Ephraim, mit Namen Elkana, ein Sohn Jerohams, des Sohnes Elihus,
des Sohnes Thohus, des Sohnes Zuphs, ein Ephrathiter✲. 2Der hatte zwei
Frauen: die eine hieß Hanna, die andere Peninna; und Peninna hatte
Kinder, Hanna aber war kinderlos. 3Dieser Mann zog Jahr für Jahr aus
seinem Wohnort nach Silo hinauf, um Gott, den HERRN der Heerscharen,
dort anzubeten und ihm zu opfern; dort waren die beiden Söhne Elis,
Hophni und Pinehas, Priester des HERRN. 4Sooft nun der Tag da war, an
welchem Elkana opferte, gab er seiner Frau Peninna und all ihren Söhnen
und Töchtern Anteile\textless sup title=``d.h. je einen Anteil oder je
ein Stück vom Opfermahl''\textgreater✲; 5der Hanna aber gab er einen
doppelten Anteil, denn er hatte Hanna lieber, obgleich der HERR ihr
Kinder versagt hatte. 6Dann kränkte aber ihre Nebenfrau\textless sup
title=``oder: Nebenbuhlerin''\textgreater✲ sie mit lieblosen Reden, um
sie zu ärgern, weil der HERR ihr Kindersegen versagt hatte. 7So ging es
Jahr für Jahr: so oft sie zum Hause des HERRN hinaufzog, kränkte jene
sie so, daß sie weinte und nichts aß. 8Elkana, ihr Gatte, fragte sie
dann: »Hanna, warum weinst du, und warum issest du nicht, und warum bist
du so betrübt? Bin ich dir nicht mehr wert als zehn Söhne?«

\hypertarget{hannas-geluxfcbde-in-silo-und-ihre-unterredung-mit-eli}{%
\paragraph{Hannas Gelübde in Silo und ihre Unterredung mit
Eli}\label{hannas-geluxfcbde-in-silo-und-ihre-unterredung-mit-eli}}

9Als man nun einst wieder in Silo in der Halle gegessen und getrunken
hatte, stand Hanna auf und trat vor den HERRN, während der Priester Eli
gerade auf dem Stuhl an einem der Türpfosten des Tempels des HERRN saß.
10In tiefer Bekümmernis und unter vielen Tränen betete sie zum HERRN
11und sprach folgendes Gelübde aus: »Gott, HERR der Heerscharen! Wenn du
das Elend deiner Magd ansehen und meiner gedenken und deine Magd nicht
vergessen willst und deiner Magd einen männlichen Sproß schenkst, so
will ich ihn dir, HERR, für sein ganzes Leben weihen, und kein
Schermesser soll an sein Haupt kommen.«

12Als sie nun lange Zeit so vor dem HERRN betete, stieg in Eli, der
ihren Mund beobachtete~-- 13Hanna redete nämlich leise für sich, nur
ihre Lippen bewegten sich, während man ihre Stimme nicht hören konnte
--, der Gedanke auf, sie sei trunken; 14und er sagte zu ihr: »Wie lange
willst du dich noch wie eine Trunkene benehmen? Werde endlich wieder
nüchtern!« 15Da erwiderte ihm Hanna: »Ach nein, Herr, ich bin eine
unglückliche Frau! Wein und berauschende Getränke habe ich nicht
genossen, sondern mein Herz vor dem HERRN ausgeschüttet. 16Halte deine
Magd nicht für ein verworfenes Weib! Denn nur infolge meiner großen
Bekümmernis und Traurigkeit habe ich so lange gebetet.« 17Da entgegnete
ihr Eli: »Gehe hin in Frieden! Der Gott Israels möge dir die Bitte
gewähren, die du an ihn gerichtet hast!« 18Sie erwiderte: »Laß deine
Magd✲ deiner Huld empfohlen sein!« Damit ging die Frau ihres Weges und
aß, und ihr trauriges Aussehen war verschwunden.

\hypertarget{samuels-geburt-erste-kindheit-und-weihe-in-silo}{%
\paragraph{Samuels Geburt, erste Kindheit und Weihe in
Silo}\label{samuels-geburt-erste-kindheit-und-weihe-in-silo}}

19Am andern Morgen machten sie sich früh auf und verrichteten ihre
Andacht\textless sup title=``eig. Anbetung''\textgreater✲ vor dem HERRN,
und als sie dann in ihr Haus nach Rama zurückgekehrt waren und Elkana zu
seiner Frau einging, da gedachte der HERR der Hanna: 20sie wurde guter
Hoffnung, und als die Zeit um war, gebar sie einen Sohn, dem sie den
Namen Samuel\textless sup title=``d.h. von Gott erhört''\textgreater✲
gab; »denn«, sagte sie, »vom HERRN habe ich ihn erbeten.«

21Als dann ihr Mann Elkana mit seiner ganzen Familie wieder hinaufzog,
um dem HERRN das jährliche Schlachtopfer und, was er gelobt hatte,
darzubringen, 22ging Hanna nicht mit hinauf; denn sie hatte zu ihrem
Manne gesagt: »Der Knabe soll erst entwöhnt sein; dann will ich ihn
hinbringen, damit er vor dem HERRN erscheint und für immer dort bleibt.«
23Da hatte Elkana, ihr Mann, ihr erwidert: »Mache es so, wie du es für
gut hältst: bleibe daheim, bis du ihn entwöhnt hast! Nur möge der HERR
seine\textless sup title=``oder: deine''\textgreater✲ Verheißung in
Erfüllung gehen lassen!« So blieb denn die Frau daheim und nährte ihren
Sohn bis zu seiner Entwöhnung. 24Sobald sie ihn aber entwöhnt hatte,
nahm sie ihn mit sich hinauf nebst einem dreijährigen Stier, einem Epha
Mehl und einem Schlauch Wein: so brachte sie den Knaben, wiewohl er noch
sehr jung war, in das Gotteshaus nach Silo. 25Als sie dort den Stier
geopfert hatten und sie den Knaben zu Eli brachten, 26sagte die Mutter
(zu Eli): »Verzeihe, mein Herr! So wahr du lebst, mein Herr: ich bin die
Frau, die einst hier bei dir gestanden hat, um zum HERRN zu beten. 27Um
diesen Knaben habe ich damals gebetet; nun hat der HERR mir die Bitte
gewährt, die ich ihm vorgetragen habe. 28So habe denn auch ich ihn dem
HERRN geweiht: solange er lebt, soll er dem HERRN geweiht sein!« Hierauf
beteten sie dort den HERRN an.

\hypertarget{lobgesang-der-hanna-samuels-dienstbeginn-in-silo}{%
\paragraph{Lobgesang der Hanna; Samuels Dienstbeginn in
Silo}\label{lobgesang-der-hanna-samuels-dienstbeginn-in-silo}}

\hypertarget{section-1}{%
\section{2}\label{section-1}}

1Hanna aber betete so: »Mein Herz frohlockt in dem\textless sup
title=``oder: durch den''\textgreater✲ HERRN, hoch ragt mein Horn durch
den HERRN; mein Mund hat weit sich aufgetan gegen meine Feinde, denn ich
freue mich deiner Hilfe. 2Niemand ist heilig wie der HERR, denn keiner
ist da außer dir und keiner ein Fels wie unser Gott. 3Laßt euer ewiges
stolzes Reden: kein vermessenes Wort entfahre eurem Munde! Denn ein
allwissender Gott ist der HERR und ein Gott, von dem die Taten gewogen
werden.

4Der Bogen der Starken\textless sup title=``oder: Helden''\textgreater✲
wird zerbrochen, Strauchelnde✲ aber gürten sich mit Kraft. 5Die da satt
waren, müssen um Brot sich verdingen, aber die Hunger litten, hungern
nicht mehr; sogar die Unfruchtbare wird Mutter von sieben, aber die
Kinderreiche welkt dahin\textless sup title=``=~vergeht in
Trauer''\textgreater✲.

6Der HERR tötet und macht lebendig, er stößt ins Totenreich hinab und
führt herauf; 7der HERR macht arm und macht reich, er erniedrigt und
erhöht auch; 8er hebt den Geringen empor aus dem Staube, erhöht den
Armen aus dem Kehricht, um sie neben den Fürsten\textless sup
title=``oder: Edlen''\textgreater✲ sitzen zu lassen und den Ehrenstuhl
ihnen anzuweisen. Denn in der Hand des HERRN sind die Säulen der Erde,
und den Erdkreis hat er darauf gegründet.

9Die Schritte seiner Frommen behütet er, aber die Gottlosen
verstummen\textless sup title=``oder: kommen um''\textgreater✲ in der
Finsternis; denn nicht durch (eigene) Kraft gewinnt der Mensch den Sieg.
10Die Widersacher des HERRN werden zerschmettert, über ihnen donnert er
im Himmel; der HERR richtet die Enden der Erde. Stärke verleiht er
seinem König und erhöht das Horn seines Gesalbten.«

11Darauf kehrte Elkana nach Rama in sein Haus zurück; der Knabe aber
diente dem HERRN unter der Aufsicht des Priesters Eli.

\hypertarget{b-gottlosigkeit-der-suxf6hne-elis-ankuxfcndigung-des-guxf6ttlichen-gerichts}{%
\paragraph{b) Gottlosigkeit der Söhne Elis; Ankündigung des göttlichen
Gerichts}\label{b-gottlosigkeit-der-suxf6hne-elis-ankuxfcndigung-des-guxf6ttlichen-gerichts}}

12Die Söhne Elis aber waren nichtswürdige Buben, die sich weder um den
HERRN 13noch um das Recht der Priester gegenüber dem Volke kümmerten.
Sooft nämlich jemand ein Schlachtopfer darbrachte, kam, während man das
Fleisch noch kochte, der Diener des Priesters mit einer dreizinkigen
Gabel in der Hand 14und stieß damit in den Kessel oder den Topf, in die
Pfanne oder den Tiegel, und alles, was die Gabel dann heraufbrachte,
nahm der Priester für sich. So machten sie es in Silo bei allen
Israeliten, die dorthin kamen. 15Sogar noch ehe man das Fett in Rauch
hatte aufgehen lassen, kam der Diener des Priesters und sagte zu dem,
der das Opfer darbrachte: »Gib Fleisch her zum Braten für den Priester!
Denn er will kein gekochtes Fleisch von dir haben, sondern rohes.«
16Wenn dann der betreffende Mann ihm entgegnete: »Zuerst muß doch das
Fett verbrannt werden, dann magst du dir nehmen, wie es dir beliebt!«,
so antwortete er: »Nein, gleich jetzt sollst du es hergeben, sonst nehme
ich es mit Gewalt!« 17So war denn die Versündigung der beiden jungen
Männer sehr schwer vor dem HERRN, weil sie das Opfer des HERRN
geringschätzig behandelten.

\hypertarget{hanna-und-der-chorknabe-samuel}{%
\paragraph{Hanna und der Chorknabe
Samuel}\label{hanna-und-der-chorknabe-samuel}}

18Samuel aber versah den Dienst vor dem HERRN als Tempelknabe, mit einem
linnenen Schulterkleid umgürtet; 19dazu pflegte seine Mutter ihm ein
kleines Obergewand anzufertigen und brachte es ihm jedes Jahr mit, wenn
sie mit ihrem Manne hinaufzog, um das jährliche Schlachtopfer
darzubringen. 20Dann segnete Eli jedesmal den Elkana und seine Frau mit
den Worten: »Der HERR wolle dir Kinder von dieser Frau schenken als
Entgelt für das Darlehen, das sie dem HERRN geliehen hat!« Danach
kehrten sie nach Hause zurück. 21Der HERR aber segnete Hanna, so daß sie
wiederum guter Hoffnung wurde und noch drei Söhne und zwei Töchter
gebar. Der junge Samuel aber wuchs heran beim\textless sup title=``oder:
unter der Obhut des''\textgreater✲ HERRN.

\hypertarget{elis-milde-mahnungen-an-seine-entarteten-suxf6hne}{%
\paragraph{Elis milde Mahnungen an seine entarteten
Söhne}\label{elis-milde-mahnungen-an-seine-entarteten-suxf6hne}}

22Wenn nun Eli, der ein sehr alter Mann war, von allem hörte, was seine
Söhne an allen Israeliten verübten, auch daß sie sich mit den Weibern
vergingen, die draußen vor dem Offenbarungszelt sich
einfanden\textless sup title=``oder: den Dienst
verrichteten''\textgreater✲, 23dann sagte er zu ihnen: »Warum laßt ihr
euch dergleichen zuschulden kommen? Denn ich höre diese üblen Reden über
euch von allen Leuten hier. 24Nicht doch, meine Söhne! Denn das ist kein
gutes Gerücht, das vom Volk des HERRN, wie ich höre, über euch
verbreitet wird. 25Wenn sich ein Mensch gegen einen Menschen vergeht, so
entscheidet die Gottheit als Richter über ihn; wenn sich aber ein Mensch
gegen den HERRN vergeht, wer könnte da als Vermittler\textless sup
title=``oder: Verteidiger''\textgreater✲ für ihn eintreten?« Doch sie
hörten nicht auf die Mahnungen ihres Vaters, denn der HERR hatte ihren
Tod beschlossen.~-- 26Der junge Samuel aber wuchs immer mehr heran und
gewann an Liebe sowohl beim HERRN als auch bei den Menschen.

\hypertarget{prophetenspruch-ankuxfcndigung-des-untergangs-elis-und-seines-hauses}{%
\paragraph{Prophetenspruch: Ankündigung des Untergangs Elis und seines
Hauses}\label{prophetenspruch-ankuxfcndigung-des-untergangs-elis-und-seines-hauses}}

27Da kam ein Gottesmann zu Eli und sagte zu ihm: »So hat der HERR
gesprochen: ›Allerdings habe ich mich dem Hause deines Vaters (Aaron)
geoffenbart, als sie\textless sup title=``d.h. die
Israeliten''\textgreater✲ noch in Ägypten (Knechte) des Hauses des
Pharaos waren, 28und habe ihn mir aus allen Stämmen Israels zum Priester
erwählt, damit er zu meinem Altar hinaufsteige und Räucherwerk anzünde
und das Schulterkleid vor mir trage; und ich habe dem Hause deines
Vaters alle Feueropfer der Israeliten zum Unterhalt zugewiesen. 29Warum
tretet ihr nun mit Füßen meine Schlachtopfer und meine Speisopfer, die
ich für meine Wohnung angeordnet habe? Und warum ehrst du deine Söhne
mehr als mich, so daß ihr euch von den Erstlingen\textless sup
title=``=~besten Stücken''\textgreater✲ aller Opfergaben meines Volkes
Israel mästet?‹ 30Darum lautet der Ausspruch des HERRN, des Gottes
Israels, so: ›Ich habe allerdings gesagt: Dein Haus und deines Vaters
Haus sollen ewiglich vor mir einhergehen\textless sup title=``oder: den
Dienst verrichten''\textgreater✲‹; jetzt aber lautet der Ausspruch des
HERRN so: ›Fern sei das von mir! Nein, wer mich ehrt, den will ich
wieder ehren, aber wer mich verachtet, der wird erniedrigt werden.
31Wisse wohl: es kommt die Zeit, da werde ich deinen Arm und den Arm
deines ganzen Geschlechts abhauen, so daß es keinen Betagten mehr in
deinem Hause geben soll. 32Dann wirst du neidisch auf alles Glück
hinblicken, mit dem der HERR Israel segnen wird, während sich in deinem
Hause nie mehr ein Betagter finden wird. 33Und gesetzt, ich tilge dir
einmal einen nicht weg von meinem Altar, um deine Augen nicht
verschmachten und dein Herz sich nicht abhärmen zu lassen, so soll doch
aller Nachwuchs deines Hauses schon im Mannesalter sterben. 34Und das
Schicksal, das deine beiden Söhne Hophni und Pinehas treffen wird, soll
dir ein Zeichen sein: an einem Tage werden sie beide sterben! 35Mir aber
will ich einen treuen Priester erstehen lassen, der nach meinem Herzen
und nach meinem Sinn handelt; dem will ich ein Haus bauen, das Bestand
hat, und er soll allezeit vor meinem Gesalbten\textless sup title=``d.h.
dem König''\textgreater✲ einhergehen\textless sup title=``oder: seines
Amtes warten''\textgreater✲. 36Dann wird es dahin kommen, daß jeder, der
von deinem Hause noch übriggeblieben ist, herbeikommt und sich vor ihm
niederwirft, um ein Geldstück als Almosen oder einen Laib Brot zu
erbetteln, und die Bitte ausspricht: Bringe mich doch bei einer der
Priesterschaften unter, damit ich ein Stück Brot zu essen habe!‹«

\hypertarget{c-samuels-berufung-zum-propheten}{%
\paragraph{c) Samuels Berufung zum
Propheten}\label{c-samuels-berufung-zum-propheten}}

\hypertarget{aa-gott-offenbart-sich-dem-samuel-und-kuxfcndigt-den-untergang-des-hauses-elis-an}{%
\subparagraph{aa) Gott offenbart sich dem Samuel und kündigt den
Untergang des Hauses Elis
an}\label{aa-gott-offenbart-sich-dem-samuel-und-kuxfcndigt-den-untergang-des-hauses-elis-an}}

\hypertarget{section-2}{%
\section{3}\label{section-2}}

1Zu der Zeit, wo der junge Samuel den Dienst des HERRN unter Elis
Aufsicht versah, waren Offenbarungen des HERRN in Israel etwas Seltenes:
Gesichte kamen nicht häufig vor. 2Nun begab es sich zu jener Zeit,
während Eli an seiner gewöhnlichen Stelle schlief -- seine Augen waren
aber allmählich schwach geworden, so daß er nicht mehr sehen konnte;
3die Lampe Gottes war noch nicht erloschen, Samuel aber schlief im
Tempel des HERRN da, wo die Lade Gottes stand --, 4da rief der HERR:
»Samuel!« Dieser antwortete: »Hier bin ich!« 5und lief dann zu Eli und
sagte: »Hier bin ich! Du hast mich ja gerufen.« Jener aber erwiderte:
»Ich habe nicht gerufen; lege dich wieder schlafen!« Da ging er hin und
legte sich schlafen. 6Der HERR aber rief wiederum: »Samuel!« Dieser
erhob sich, ging zu Eli und sagte: »Hier bin ich! Du hast mich ja
gerufen.« Jener antwortete: »Ich habe nicht gerufen, mein Sohn; lege
dich wieder schlafen!« 7Samuel hatte nämlich den HERRN noch nicht
kennengelernt, und es war ihm noch keine Offenbarung des HERRN zuteil
geworden. 8Nun rief der HERR wiederum zum drittenmal: »Samuel!« Da stand
er auf, ging zu Eli und sagte: »Hier bin ich! Du hast mich ja gerufen.«
Jetzt erkannte Eli, daß der HERR es war, der den Jüngling gerufen hatte;
9daher sagte Eli zu Samuel: »Gehe hin, lege dich schlafen! Doch wenn du
wieder angerufen wirst, so antworte: ›Rede, HERR, denn dein Knecht
hört!‹« So ging denn Samuel hin und legte sich auf seinem Platze
schlafen.

10Da kam der HERR, trat vor ihn hin und rief wie die vorigen Male:
»Samuel! Samuel!« Dieser antwortete: »Rede! Denn dein Knecht hört.« 11Da
sagte der HERR zu Samuel: »Wisse wohl, ich will in Israel etwas
vollführen, daß jedem, der es hört, beide Ohren gellen sollen. 12An
jenem Tage will ich an Eli alles in Erfüllung gehen lassen, was ich
seinem Hause angedroht habe, von Anfang bis zu Ende! 13Denn ich habe ihm
ankündigen lassen, daß ich sein Haus für alle Zeit richten will wegen
seiner Verschuldung; denn er hat gewußt, daß seine Söhne Gott in
Verachtung bringen, und er ist ihnen dennoch nicht entgegengetreten.
14Darum habe ich dem Hause Elis geschworen: ›Wahrlich, die Verfehlung
des Hauses Elis soll in Ewigkeit nicht, weder durch Schlachtopfer noch
durch Speisopfer, gesühnt werden!‹«

\hypertarget{bb-samuel-teilt-eli-die-offenbarung-mit-und-beginnt-seine-wirksamkeit-als-prophet-fuxfcr-ganz-israel}{%
\subparagraph{bb) Samuel teilt Eli die Offenbarung mit und beginnt seine
Wirksamkeit als Prophet für ganz
Israel}\label{bb-samuel-teilt-eli-die-offenbarung-mit-und-beginnt-seine-wirksamkeit-als-prophet-fuxfcr-ganz-israel}}

15Samuel blieb hierauf bis zum Morgen liegen; dann öffnete er die Türen
des Hauses des HERRN, scheute sich jedoch, dem Eli etwas von dem Gesicht
mitzuteilen. 16Aber Eli rief ihn zu sich und sagte: »Samuel, mein Sohn!«
Dieser antwortete: »Hier bin ich!« 17Da fragte jener: »Was hat
er\textless sup title=``d.h. der HERR''\textgreater✲ zu dir gesagt?
Verheimliche mir ja nichts! Gott möge dir's jetzt und künftig vergelten,
wenn du mir irgend etwas von allem verheimlichst, was er zu dir gesagt
hat!« 18Da teilte ihm Samuel die ganze Offenbarung mit, ohne ihm etwas
zu verschweigen; der aber erwiderte: »Er ist der HERR: er tue, was ihm
wohlgefällt!«

19Samuel aber wuchs heran, und der HERR war mit ihm und ließ nichts
unerfüllt von allem, was er angekündigt hatte; 20und ganz Israel von Dan
bis Beerseba erkannte, daß Samuel zum Propheten des HERRN bestellt
worden war. 21Der HERR aber erschien auch fernerhin in Silo; denn der
HERR offenbarte sich dem Samuel {[}in Silo durch das Wort des HERRN{]},
und Samuel teilte dann die Offenbarungen dem gesamten Israel mit.

\hypertarget{untergang-des-hauses-elis-samuel-als-richter-kap.-4-7}{%
\subsubsection{2. Untergang des Hauses Elis; Samuel als Richter (Kap.
4-7)}\label{untergang-des-hauses-elis-samuel-als-richter-kap.-4-7}}

\hypertarget{a-israels-niederlage-im-philisterkriege-verlust-der-bundeslade-tod-elis-und-seiner-beiden-suxf6hne}{%
\paragraph{a) Israels Niederlage im Philisterkriege; Verlust der
Bundeslade; Tod Elis und seiner beiden
Söhne}\label{a-israels-niederlage-im-philisterkriege-verlust-der-bundeslade-tod-elis-und-seiner-beiden-suxf6hne}}

\hypertarget{aa-die-bundeslade-ins-lager-der-israeliten-geholt}{%
\subparagraph{aa) Die Bundeslade ins Lager der Israeliten
geholt}\label{aa-die-bundeslade-ins-lager-der-israeliten-geholt}}

\hypertarget{section-3}{%
\section{4}\label{section-3}}

1Und die Israeliten zogen den Philistern zum Kampf entgegen und lagerten
bei Eben-Eser\textless sup title=``vgl. 7,12''\textgreater✲, während die
Philister ihr Lager bei Aphek aufgeschlagen hatten. 2Als sich nun die
Philister den Israeliten gegenüber in Schlachtordnung aufgestellt hatten
und der Kampf sich weit ausbreitete, wurden die Israeliten von den
Philistern besiegt, und diese erschlugen in der Schlacht auf offenem
Felde gegen viertausend Mann. 3Als dann das Heer ins Lager zurückgekehrt
war, sagten die Ältesten der Israeliten: »Warum hat der HERR uns heute
den Philistern unterliegen lassen? Wir wollen die Bundeslade des HERRN
aus Silo zu uns holen, damit er in unsere Mitte kommt und uns aus der
Hand unserer Feinde errettet.« 4So sandte denn das Volk nach Silo, und
man holte von dort die Bundeslade Gottes, des HERRN der Heerscharen, der
über den Cheruben thront; und die beiden Söhne Elis, Hophni und Pinehas,
begleiteten die Bundeslade Gottes.

\hypertarget{bb-die-wirkung-dieses-ereignisses-auf-die-kriegfuxfchrenden-teile-niederlage-der-israeliten-und-verlust-der-lade}{%
\subparagraph{bb) Die Wirkung dieses Ereignisses auf die kriegführenden
Teile; Niederlage der Israeliten und Verlust der
Lade}\label{bb-die-wirkung-dieses-ereignisses-auf-die-kriegfuxfchrenden-teile-niederlage-der-israeliten-und-verlust-der-lade}}

5Als nun die Bundeslade des HERRN ins Lager kam, erhob ganz Israel ein
so gewaltiges Jubelgeschrei, daß die Erde davon erdröhnte. 6Als die
Philister den lauten Jubel hörten, fragten sie: »Was hat dies laute
Jubelgeschrei im Lager der Hebräer zu bedeuten?« Als sie dann erfuhren,
daß die Lade des HERRN ins Lager gekommen sei, 7gerieten die Philister
in Angst, denn sie dachten: »Gott ist zu ihnen ins Lager gekommen!«, und
sie riefen aus: »Wehe uns! Denn so etwas ist vordem nie geschehen! 8Wehe
uns! Wer wird uns aus\textless sup title=``oder: vor''\textgreater✲ der
Hand dieser mächtigen Gottheit erretten? Das ist ja dieselbe Gottheit,
welche die Ägypter in der Wüste mit allerlei Plagen geschlagen hat!
9Haltet euch tapfer und zeigt euch als Männer, ihr Philister, sonst müßt
ihr den Hebräern dienen, wie sie euch gedient haben; ja, seid Männer und
kämpft!« 10Da kämpften denn die Philister, und die Israeliten wurden
geschlagen, und sie flohen ein jeder zu seinen Zelten\textless sup
title=``=~in seinen Wohnort''\textgreater✲; und die Niederlage war sehr
schwer: es fielen von den Israeliten dreißigtausend Mann Fußvolk. 11Auch
die Lade Gottes wurde (von den Feinden) erbeutet, und die beiden Söhne
Elis, Hophni und Pinehas, fanden den Tod.

\hypertarget{cc-die-traurigen-wirkungen-der-botschaft-in-silo-der-tod-elis-und-seiner-schwiegertochter}{%
\subparagraph{cc) Die traurigen Wirkungen der Botschaft in Silo; der Tod
Elis und seiner
Schwiegertochter}\label{cc-die-traurigen-wirkungen-der-botschaft-in-silo-der-tod-elis-und-seiner-schwiegertochter}}

12Ein Benjaminit nun war vom Schlachtfelde weggeeilt und noch an
demselben Tage nach Silo gelangt mit zerrissenen Kleidern und mit Erde
auf dem Haupt. 13Als er dort ankam, saß Eli gerade auf einem Stuhle
(neben dem Tor) und spähte die Straße entlang; denn sein Herz bangte um
die Lade Gottes. Als nun der Mann ankam, um der Stadt die Nachricht zu
bringen, erhob die ganze Stadt ein Wehgeschrei. 14Als Eli das laute
Geschrei hörte, fragte er: »Was bedeutet dieses gewaltige Geschrei?« Da
kam der Mann herangeeilt und brachte dem Eli die Meldung. 15Eli war aber
achtundneunzig Jahre alt und hatte das Augenlicht verloren, so daß er
nicht mehr sehen konnte. 16Als nun der Mann zu Eli sagte: »Ich bin der
Bote, der vom Schlachtfeld gekommen ist, und zwar bin ich heute vom
Schlachtfeld geflohen«, da fragte Eli: »Wie ist es (denn dort) gegangen,
mein Sohn?« 17Der Bote antwortete: »Die Israeliten sind vor den
Philistern geflohen; dazu hat unser Heer eine schwere Niederlage
erlitten; auch deine beiden Söhne, Hophni und Pinehas, sind tot, und die
Lade Gottes ist (von den Feinden) erbeutet.« 18Als er nun die Lade
Gottes erwähnte, da fiel Eli rücklings vom Stuhl herab neben dem Tor,
brach das Genick und war tot; denn er war alt und ein schwerer Mann.
Vierzig Jahre lang war er Richter in Israel gewesen.

19Seine Schwiegertochter aber, die Frau des Pinehas, war hochschwanger;
als sie nun die Kunde vom Verlust der Lade Gottes und vom Tode ihres
Schwiegervaters und ihres Mannes vernahm, brach sie zusammen und gebar,
da die Wehen sie plötzlich überfielen. 20Als es nun mit ihr zum Sterben
ging und die Frauen, die um sie standen, zu ihr sagten: »Fürchte dich
nicht; du hast einen Sohn geboren!«, gab sie keine Antwort und war ganz
teilnahmslos; 21doch nannte sie den Knaben Ikabod\textless sup
title=``d.h. Ruhmlos''\textgreater✲, womit sie sagen wollte: »Dahin ist
die Herrlichkeit\textless sup title=``oder: der Ruhm''\textgreater✲
Israels!«, weil ja die Lade Gottes verlorengegangen war und wegen ihres
Schwiegervaters und um ihres Mannes willen. 22Sie rief also aus: »Dahin
ist die Herrlichkeit\textless sup title=``oder: der Ruhm''\textgreater✲
Israels, denn die Lade Gottes ist verlorengegangen!«

\hypertarget{b-die-bundeslade-richtet-im-lande-der-philister-unheil-in-mehreren-stuxe4dten-an}{%
\paragraph{b) Die Bundeslade richtet im Lande der Philister Unheil in
mehreren Städten
an}\label{b-die-bundeslade-richtet-im-lande-der-philister-unheil-in-mehreren-stuxe4dten-an}}

\hypertarget{section-4}{%
\section{5}\label{section-4}}

1Die Philister aber hatten die Lade Gottes erbeutet und brachten sie von
Eben-Eser nach Asdod; 2dort nahmen die Philister die Lade Gottes,
brachten sie in den Tempel Dagons und stellten sie neben Dagon hin. 3Als
aber die Einwohner von Asdod am nächsten Morgen früh in den Tempel
Dagons kamen, fanden sie Dagon vor der Lade des HERRN am Boden auf dem
Angesicht liegen. Da nahmen sie Dagon und stellten ihn wieder an seinen
Platz. 4Als sie aber am folgenden Tage frühmorgens kamen, fanden sie
Dagon wieder auf seinem Angesicht am Boden vor der Lade des HERRN
liegen, und zwar lagen der Kopf Dagons und seine beiden Hände abgetrennt
auf der Schwelle des Tempels; nur sein Fischrumpf war neben
ihr\textless sup title=``d.h. der Lade''\textgreater✲ noch
übriggeblieben. 5Daher treten in Asdod die Priester Dagons und alle, die
in den Dagontempel hineingehen, nicht auf die Schwelle Dagons bis auf
den heutigen Tag.

6Hierauf legte sich die Hand des HERRN schwer auf die Einwohner von
Asdod; er setzte sie in Schrecken und suchte die Stadt und ihr Gebiet
mit Pestbeulen heim. 7Als nun die Leute von Asdod diese ihre schlimme
Lage erkannten, sagten sie: »Die Lade des Gottes Israels darf nicht bei
uns bleiben! Denn seine Hand lastet schwer auf uns und unserm Gott
Dagon.« 8Da beriefen sie durch Boten sämtliche Fürsten der Philister in
ihre Stadt zusammen und fragten: »Was sollen wir mit der Lade des Gottes
Israels machen?« Da antworteten✲ jene: »Nach Gath soll die Lade des
Gottes Israels überführt werden!«, und sie ließen sie auch wirklich
dorthin bringen. 9Aber als sie dorthin geschafft worden war, kam die
Hand des HERRN über die Stadt mit einer ganz gewaltigen Bestürzung, und
er schlug auch hier die Leute der Stadt, junge und alte, so daß
Pestbeulen an ihnen zum Ausbruch kamen. 10Als man dann die Lade Gottes
nach Ekron gebracht hatte, schrien die Einwohner der Stadt nach Ankunft
der Lade Gottes in Ekron: »Man hat die Lade des Gottes Israels zu uns
hergebracht, um uns und unsere Einwohnerschaft umzubringen!« 11Da ließen
sie wiederum alle Fürsten der Philister durch Boten zusammenrufen und
baten: »Schickt die Lade des Gottes der Israeliten wieder zurück an den
Ort, wohin sie gehört, damit sie nicht uns und unser Volk umbringt!«
Denn es war ein tödlicher Schrecken über die ganze Stadt gekommen:
überaus schwer lastete die Hand Gottes auf ihr; 12denn die Leute, die
nicht starben, litten qualvoll an Pestbeulen, und das Wehgeschrei der
Stadt stieg zum Himmel empor.

\hypertarget{c-zuruxfccksendung-der-bundeslade-mit-suxfchnegeschenken}{%
\paragraph{c) Zurücksendung der Bundeslade mit
Sühnegeschenken}\label{c-zuruxfccksendung-der-bundeslade-mit-suxfchnegeschenken}}

\hypertarget{aa-beschluuxdffassung-der-philister-uxfcber-die-zuruxfcckgabe-der-lade}{%
\subparagraph{aa) Beschlußfassung der Philister über die Zurückgabe der
Lade}\label{aa-beschluuxdffassung-der-philister-uxfcber-die-zuruxfcckgabe-der-lade}}

\hypertarget{section-5}{%
\section{6}\label{section-5}}

1Nachdem sich nun die Lade des HERRN sieben Monate lang im Gebiet der
Philister befunden hatte, 2beriefen die Philister ihre Priester und
Wahrsager und fragten sie: »Was sollen wir mit der Lade des
HERRN\textless sup title=``=~des Gottes Israels''\textgreater✲ machen?
Laßt uns wissen, auf welche Weise wir sie an den Ort, wohin sie gehört,
zurückschicken sollen!« 3Jene antworteten: »Wollt ihr die Lade des
Gottes Israels zurückschicken, so dürft ihr sie nicht ohne Geschenke
ziehen lassen, sondern müßt ihr jedenfalls ein Sühnegeschenk zum Entgelt
mitgeben; alsdann wird es euch wieder gutgehen, und ihr werdet auch in
Erfahrung bringen, warum seine Hand nicht von euch abläßt.« 4Da fragten
sie: »Worin soll das Sühnegeschenk bestehen, das wir ihm als Entgelt
entrichten sollen?« Jene antworteten: »Entsprechend der Zahl der Fürsten
der Philister: fünf goldene (Nachbildungen eurer) Pestbeulen und fünf
goldene Mäuse; denn die gleiche Plage ist euch allen, auch euren Fürsten
widerfahren. 5Laßt also Nachbildungen eurer Pestbeulen und Nachbildungen
von den Mäusen anfertigen, die euch das Land verwüsten, und erweist so
dem Gott Israels eine Ehre; vielleicht läßt er dann seine Hand nicht
mehr so schwer auf euch und eurem Gott und eurem Lande lasten. 6Warum
wollt ihr euer Herz verhärten, wie die Ägypter und der Pharao einst ihr
Herz verhärtet haben? Nicht wahr: erst als er\textless sup title=``d.h.
Gott''\textgreater✲ ihnen übel mitgespielt hatte, da ließen sie sie
ziehen, und sie durften gehen. 7Laßt also jetzt einen neuen Wagen
fertigen und nehmt ein Paar säugende Kühe, auf die noch nie ein Joch
gekommen ist; spannt dann die Kühe an den Wagen, ihre Kälber aber treibt
von ihnen weg nach Hause zurück. 8Dann nehmt die Lade des
HERRN\textless sup title=``=~des Gottes Israels''\textgreater✲, stellt
sie auf den Wagen und tut die goldenen Kleinodien, die ihr ihm als
Sühnegeschenk entrichten wollt, in ein Kästchen neben sie und laßt sie
so ihres Weges ziehen. 9Gebt aber acht: wenn sie den Heimweg antritt
hinauf nach Beth-Semes hin, so ist er es gewesen, der uns dieses große
Unglück hat zustoßen lassen; andernfalls wissen wir, daß nicht seine
Hand uns heimgesucht hat, sondern daß nur ein Zufall uns begegnet ist.«

\hypertarget{bb-ausfuxfchrung-des-beschlusses-ankunft-und-empfang-der-lade-in-beth-semes}{%
\subparagraph{bb) Ausführung des Beschlusses; Ankunft und Empfang der
Lade in
Beth-Semes}\label{bb-ausfuxfchrung-des-beschlusses-ankunft-und-empfang-der-lade-in-beth-semes}}

10Da taten die Leute so: sie nahmen ein Paar säugende Kühe und spannten
sie an den Wagen, während sie ihre Kälber zu Hause zurückbehielten.
11Dann stellten sie die Lade des HERRN auf den Wagen samt dem Kästchen
mit den goldenen Mäusen und den Nachbildungen ihrer Pestbeulen. 12Da
gingen die Kühe geradeaus auf dem Wege nach Beth-Semes zu, verfolgten,
unaufhörlich brüllend, dieselbe Straße, ohne nach rechts oder nach links
abzubiegen; die Fürsten der Philister aber gingen hinter ihnen her bis
an die Feldmark von Beth-Semes.

13Die Einwohner von Beth-Semes aber waren gerade mit der Weizenernte in
der Niederung beschäftigt. Als sie nun ihre Augen erhoben und die Lade
erblickten, freuten sie sich, sie wiederzusehen. 14Als der Wagen dann
auf dem Felde des Bethsemesiters Josua angekommen war, stand er daselbst
still. Dort lag ein großer Stein. Sie spalteten nun das Holz des Wagens
und brachten die beiden Kühe dem HERRN als Brandopfer dar. 15Die Leviten
hatten nämlich die Lade des HERRN herabgenommen und ebenso das neben ihr
stehende Kästchen, in welchem sich die goldenen Kleinodien befanden, und
hatten sie auf den großen Stein gestellt; hierauf brachten die Einwohner
von Beth-Semes dem HERRN noch an demselben Tage Brand- und Schlachtopfer
dar. 16Nachdem dann die fünf Fürsten der Philister alles mit angesehen
hatten, kehrten sie noch an demselben Tage nach Ekron zurück.~--
17Folgendes sind aber die goldenen Pestbeulen, welche die Philister dem
HERRN als Sühnegeschenk erstattet haben: für Asdod eine, für Gaza eine,
für Askalon eine, für Gath eine, für Ekron eine; 18die goldenen Mäuse
aber entsprachen der Zahl aller Ortschaften der Philister unter den fünf
Fürsten, sowohl der befestigten Städte als auch der offenen
Bauerndörfer. (Zeuge) hierfür ist der große Stein, auf den sie die Lade
des HERRN niedergesetzt hatten, der bis auf den heutigen Tag auf dem
Felde Josuas, des Bethsemesiters, liegt.

\hypertarget{cc-aufstellung-der-lade-in-kirjath-jearim}{%
\subparagraph{cc) Aufstellung der Lade in
Kirjath-Jearim}\label{cc-aufstellung-der-lade-in-kirjath-jearim}}

19(Der Herr) aber ließ unter den Einwohnern von Beth-Semes ein Sterben
ausbrechen, weil sie in die Lade des HERRN hineingeschaut hatten, und
ließ unter der Einwohnerschaft siebzig Personen sterben. Da trauerte die
Einwohnerschaft darüber, daß der HERR ein so schreckliches Unglück über
das Volk hatte kommen lassen; 20und die Leute von Beth-Semes riefen aus:
»Wer vermag zu bestehen vor dem HERRN, diesem heiligen Gott?!« und: »Zu
wem soll er nun hingehen von uns weg?« 21Sie sandten dann Boten zu den
Einwohnern von Kirjath-Jearim und ließen ihnen sagen: »Die Philister
haben die Lade des HERRN zurückgebracht; kommt herab und holt sie zu
euch hinauf!«

\hypertarget{section-6}{%
\section{7}\label{section-6}}

1Da kamen die Männer von Kirjath-Jearim, holten die Lade des HERRN
hinauf und brachten sie in das Haus Abinadabs auf der Anhöhe; seinen
Sohn Eleasar aber weihten sie zum Hüter der Lade des HERRN.

\hypertarget{d-israels-buuxdfe-niederlage-der-philister-samuel-als-richter}{%
\paragraph{d) Israels Buße; Niederlage der Philister; Samuel als
Richter}\label{d-israels-buuxdfe-niederlage-der-philister-samuel-als-richter}}

\hypertarget{aa-die-israeliten-bekehren-sich-buuxdffertig-zu-gott}{%
\subparagraph{aa) Die Israeliten bekehren sich bußfertig zu
Gott}\label{aa-die-israeliten-bekehren-sich-buuxdffertig-zu-gott}}

2Seit dem Tage nun, an welchem die Lade des HERRN in Kirjath-Jearim
untergebracht war, verging eine lange Zeit, wohl zwanzig Jahre. Als sich
dann das ganze Haus Israel mit Wehklagen an den HERRN wandte, 3sagte
Samuel zum ganzen Hause Israel: »Wenn ihr mit eurem ganzen Herzen zum
HERRN umkehren wollt, so schafft die fremden Götter und besonders die
Astartebilder aus eurer Mitte weg und richtet euer Herz auf den HERRN
und dient ihm allein, dann wird er euch aus der Gewalt der Philister
erretten.« 4Da entfernten die Israeliten die Bilder Baals und der
Astarte und dienten dem HERRN allein.

\hypertarget{bb-samuels-fuxfcrbitte-und-opfer-fuxfcr-israel-in-mizpa-besiegung-der-philister-der-stein-eben-eser}{%
\subparagraph{bb) Samuels Fürbitte und Opfer für Israel in Mizpa;
Besiegung der Philister; der Stein
Eben-Eser}\label{bb-samuels-fuxfcrbitte-und-opfer-fuxfcr-israel-in-mizpa-besiegung-der-philister-der-stein-eben-eser}}

5Hierauf machte Samuel bekannt: »Laßt ganz Israel in Mizpa
zusammenkommen, dann will ich Fürbitte für euch beim HERRN einlegen!«
6Als sie sich nun in Mizpa versammelt hatten, schöpften sie Wasser und
gossen es vor dem HERRN aus; zugleich fasteten sie an jenem Tage und
legten dort das Bekenntnis ab: »Wir haben gegen den HERRN gesündigt!«
Sodann sprach Samuel den Israeliten in Mizpa Recht.

7Als aber die Philister erfuhren, daß die Israeliten sich in Mizpa
versammelt hatten, zogen die Fürsten der Philister gegen Israel hinauf.
Als die Israeliten Kunde davon erhielten, gerieten sie in Furcht vor den
Philistern 8und baten Samuel: »Laß nicht ab, für uns zum HERRN, unserm
Gott, laut zu flehen, daß er uns aus der Gewalt der Philister errette!«
9Da nahm Samuel ein Milchlamm und brachte es dem HERRN als Ganzopfer
dar; dabei flehte Samuel laut für Israel zum HERRN, und der HERR erhörte
ihn. 10Während nämlich Samuel an jenem Tage das Brandopfer darbrachte,
waren die Philister zum Angriff gegen Israel herangerückt; aber der HERR
ließ ein Gewitter mit furchtbarem Donner über den
Philistern\textless sup title=``oder: gegen die Philister''\textgreater✲
ausbrechen und versetzte sie dadurch in solche Bestürzung, daß sie von
den Israeliten geschlagen wurden. 11Da brachen die Israeliten aus Mizpa
hervor, verfolgten die Philister und richteten ein Blutbad unter ihnen
an bis unterhalb von Beth-Kar. 12Darauf nahm Samuel einen Stein und
stellte ihn zwischen Mizpa und Sen\textless sup title=``oder: dem
Felszahn''\textgreater✲ auf und gab ihm den Namen Eben-Eser\textless sup
title=``d.h. Stein der Hilfe''\textgreater✲ und sagte: »Bis hierher hat
der HERR uns geholfen!«

\hypertarget{cc-friedenszustand-des-landes-samuels-wirksamkeit-als-richter}{%
\subparagraph{cc) Friedenszustand des Landes; Samuels Wirksamkeit als
Richter}\label{cc-friedenszustand-des-landes-samuels-wirksamkeit-als-richter}}

13So waren denn die Philister gedemütigt und fielen fortan nicht mehr in
das Gebiet der Israeliten ein; denn solange Samuel lebte, lag die Hand
des HERRN auf den Philistern. 14Auch die Städte, welche die Philister
den Israeliten abgenommen hatten, kamen wieder in den Besitz der
Israeliten, von Ekron an bis Gath; auch das zu ihnen gehörige Gebiet
machte Israel von der Herrschaft der Philister frei. Ebenso bestand
Friede zwischen Israel und den Amoritern.

15Samuel waltete dann als Richter in Israel während seines ganzen
Lebens. 16Er zog nämlich Jahr für Jahr umher und machte die Runde über
Bethel, Gilgal und Mizpa; wenn er dann an allen diesen Orten Gericht in
Israel gehalten hatte, 17kehrte er nach Rama zurück; denn dort hatte er
seinen Wohnsitz, und dort sprach er den Israeliten Recht; er baute dort
dem HERRN auch einen Altar.

\hypertarget{sauls-wahl-zum-kuxf6nig-kap.-8-10}{%
\subsubsection{3. Sauls Wahl zum König (Kap.
8-10)}\label{sauls-wahl-zum-kuxf6nig-kap.-8-10}}

\hypertarget{a-israels-verlangen-nach-einem-kuxf6nig}{%
\paragraph{a) Israels Verlangen nach einem
König}\label{a-israels-verlangen-nach-einem-kuxf6nig}}

\hypertarget{aa-das-verlangen-des-volkes-erregt-samuels-miuxdfmut-findet-aber-gottes-zustimmung}{%
\subparagraph{aa) Das Verlangen des Volkes erregt Samuels Mißmut, findet
aber Gottes
Zustimmung}\label{aa-das-verlangen-des-volkes-erregt-samuels-miuxdfmut-findet-aber-gottes-zustimmung}}

\hypertarget{section-7}{%
\section{8}\label{section-7}}

1Als aber Samuel alt geworden war, bestellte er seine (beiden) Söhne zu
Richtern über Israel. 2Sein ältester Sohn hieß Joel, sein zweiter Abia;
sie sprachen in Beerseba Recht. 3Aber seine Söhne wandelten nicht in
seinen Wegen, sondern gingen auf Gewinn aus, nahmen Bestechungsgeschenke
an und beugten das Recht. 4Da versammelten sich alle Ältesten der
Israeliten, kamen zu Samuel nach Rama 5und sagten zu ihm: »Du bist nun
alt geworden, und deine Söhne wandeln nicht in deinen Wegen; so setze
nun einen König über uns ein, der uns richten✲ soll, wie es bei allen
anderen Völkern der Fall ist.« 6Samuel war zwar unzufrieden damit, daß
sie von ihm die Einsetzung eines Königs verlangten, der über sie
herrschen sollte; doch als er zum HERRN betete, 7gab der HERR ihm die
Antwort: »Komm der Forderung des Volkes in allem nach, was sie von dir
verlangen! Denn nicht dich haben sie verworfen, sondern mich haben sie
verworfen, daß ich nicht (länger) König über sie sein soll. 8Sie machen
es jetzt mit dir ebenso, wie sie es mit mir immer gemacht haben seit der
Zeit, wo ich sie aus Ägypten hergeführt habe, bis auf diesen Tag, indem
sie mich verlassen und anderen Göttern gedient haben. 9So komm also
ihrer Forderung nach; nur verwarne sie ernstlich und weise sie hin auf
die Rechte des Königs, der über sie herrschen wird.«

\hypertarget{bb-samuel-teilt-dem-volke-die-einem-kuxf6nig-zustehenden-rechte-mit}{%
\subparagraph{bb) Samuel teilt dem Volke die einem König zustehenden
Rechte
mit}\label{bb-samuel-teilt-dem-volke-die-einem-kuxf6nig-zustehenden-rechte-mit}}

10Hierauf teilte Samuel dem Volke, das einen König von ihm forderte,
alles mit, was der HERR zu ihm gesagt hatte, 11und fuhr dann fort:
»Folgende Rechte wird der König haben, der über euch herrschen wird:
Eure Söhne wird er nehmen, um sie für sich bei seinen Kriegswagen und
seinen Reitern zu verwenden; er wird sie auch vor seinem Wagen herlaufen
lassen 12und sie als Befehlshaber über Tausend und als Befehlshaber über
Fünfzig für sich anstellen; sie werden ferner seine Äcker pflügen müssen
und seine Ernte einbringen und ihm Kriegsgeräte und Wagengeschirr
anzufertigen haben. 13Eure Töchter aber wird er nehmen und sie zum
Salbenbereiten, zum Kochen und zum Backen verwenden. 14Von euren Äckern,
euren Weinbergen und Ölbaumgärten wird er die besten nehmen und sie
seinen Dienern\textless sup title=``oder: Beamten''\textgreater✲ geben;
15und von euren Saatfeldern und Weinbergen wird er den Zehnten erheben
und ihn seinen Hofleuten und Beamten geben. 16Eure Knechte und Mägde,
eure schönsten Rinder und Esel wird er nehmen und sie für seine
Wirtschaft\textless sup title=``oder: Hofhaltung''\textgreater✲
verwenden. 17Von eurem Kleinvieh wird er den Zehnten erheben, und ihr
selbst werdet ihm als Knechte dienen müssen. 18Und wenn ihr dann wegen
eures Königs, den ihr euch erwählt habt, zum HERRN schreit, so wird der
HERR euch alsdann nicht erhören.«

\hypertarget{cc-das-volk-beharrt-bei-seiner-forderung-gottes-zustimmung}{%
\subparagraph{cc) Das Volk beharrt bei seiner Forderung; Gottes
Zustimmung}\label{cc-das-volk-beharrt-bei-seiner-forderung-gottes-zustimmung}}

19Aber das Volk wollte auf Samuels Vorstellungen nicht hören, sondern
erklärte: »Nein, es soll dennoch ein König an unserer Spitze stehen!
20Wir wollen es ebenso haben wie alle anderen Völker: unser König soll
uns Recht sprechen, soll unser Anführer sein und unsere Kriege führen.«
21Als nun Samuel alle Worte✲ des Volkes angehört und sie dem HERRN
vorgetragen hatte, 22sagte der HERR zu Samuel: »Komm ihrer Forderung
nach und setze einen König über sie ein!« Darauf sagte Samuel zu den
Männern von Israel: »Geht heim, ein jeder in seinen Wohnort!«

\hypertarget{b-sauls-erste-begegnung-mit-samuel}{%
\paragraph{b) Sauls erste Begegnung mit
Samuel}\label{b-sauls-erste-begegnung-mit-samuel}}

\hypertarget{aa-saul-kommt-auf-der-suche-nach-den-eselinnen-seines-vaters-in-den-wohnort-samuels}{%
\subparagraph{aa) Saul kommt auf der Suche nach den Eselinnen seines
Vaters in den Wohnort
Samuels}\label{aa-saul-kommt-auf-der-suche-nach-den-eselinnen-seines-vaters-in-den-wohnort-samuels}}

\hypertarget{section-8}{%
\section{9}\label{section-8}}

1Es war aber ein Mann aus (Gibea im) Stamme Benjamin namens Kis, ein
Sohn Abiels, des Sohnes Zerors, des Sohnes Bechoraths, des Sohnes
Aphiahs, ein Benjaminit, ein wohlhabender Mann; 2der hatte einen Sohn
namens Saul, von ungewöhnlicher Schönheit, so daß es unter den
Israeliten keinen schöneren Mann gab; alle anderen im Volk überragte er
um eines Hauptes Länge. 3Als nun Kis, dem Vater Sauls, einst die
Eselinnen sich verlaufen hatten, sagte Kis zu seinem Sohne Saul: »Nimm
einen von den Knechten mit dir und mache dich auf den Weg, um die
Eselinnen zu suchen!« 4Da durchwanderte er das ephraimitische Bergland
und durchzog die Landschaft Salisa, ohne die Tiere zu finden; weiter
zogen sie durch die Landschaft Saalim, aber sie waren nicht da; dann
ging's durch das Gebiet von Benjamin, aber sie fanden sie auch hier
nicht. 5Als sie endlich in der Landschaft Zuph angelangt waren, sagte
Saul zu seinem Knecht, der bei ihm war: »Komm, wir wollen umkehren;
sonst könnte sich mein Vater um uns statt um die Eselinnen Sorge
machen.« 6Doch jener entgegnete ihm: »In der Stadt dort wohnt ja ein
Gottesmann, ein hochangesehener Mann; alles, was er sagt, trifft sicher
ein. Laß uns also dorthin gehen! Vielleicht gibt er uns Auskunft in
unserer Sache, deretwegen wir unterwegs sind.« 7Saul erwiderte seinem
Knechte: »Wenn wir wirklich hingingen, ja was wollten wir dem Manne
zukommen lassen? Das Brot in unseren Ranzen ist ausgegangen, und wir
haben kein Geschenk, das wir dem Gottesmanne bringen könnten; was hätten
wir noch bei uns?« 8Da antwortete der Knecht dem Saul noch einmal: »Ich
habe hier noch einen Viertelschekel Silber bei mir; den magst du dem
Gottesmanne schenken, damit er uns Auskunft in unserer Sache gibt.«
9{[}Ehedem gebrauchte man in Israel, wenn man zur Befragung Gottes ging,
die Redensart: »Kommt, laßt uns zum Seher gehen!«; denn einen Mann, den
man heutzutage Prophet heißt, nannte man ehedem Seher.{]} 10Da
antwortete Saul seinem Knecht: »Dein Vorschlag ist gut! Komm, wir wollen
(zum Seher) hingehen!«

\hypertarget{bb-sauls-freundlicher-empfang-und-ehrenvolle-behandlung-von-seiten-samuels}{%
\subparagraph{bb) Sauls freundlicher Empfang und ehrenvolle Behandlung
von seiten
Samuels}\label{bb-sauls-freundlicher-empfang-und-ehrenvolle-behandlung-von-seiten-samuels}}

Als sie nun nach der Stadt hingingen, wo der Gottesmann war, 11und
gerade den Aufgang zur Stadt hinanstiegen, begegneten ihnen junge
Mädchen, die herauskamen, um Wasser zu holen. Diese fragten sie: »Ist
der Seher hier anwesend?« 12Sie antworteten ihnen: »Jawohl, er ist da
gerade vor dir; eile nun hin, denn er ist eben heute in die Stadt
gekommen, weil das Volk heute ein Opferfest auf der Höhe hält. 13Wenn
ihr in die Stadt kommt, werdet ihr ihn gerade noch treffen, bevor er auf
die Höhe zum Essen hinaufgeht; denn die Festgesellschaft ißt nicht eher,
als bis er gekommen ist; er muß nämlich das Opfermahl segnen: erst dann
essen die Geladenen. Geht also hinauf, denn eben jetzt werdet ihr ihn
treffen!« 14So gingen sie denn zur Stadt hinauf; und als sie gerade in
das Innere der Stadt getreten waren, kam Samuel heraus, ihnen entgegen,
um zur Anhöhe hinaufzugehen. 15Nun hatte aber der HERR am Tage zuvor,
ehe Saul kam, dem Samuel folgende Offenbarung zukommen lassen: 16»Morgen
um diese Stunde werde ich einen Mann aus dem Stamme Benjamin zu dir
kommen lassen: den salbe zum Fürsten über mein Volk Israel; er soll mein
Volk aus der Gewalt der Philister erretten; denn ich habe das Elend
meines Volkes angesehen, weil sein Hilferuf zu mir gedrungen ist.«
17Sobald nun Samuel den Saul erblickte, tat der HERR ihm kund: »Dies ist
der Mann, von dem ich dir gesagt habe, daß er über mein Volk herrschen
solle!«

18Da trat Saul innerhalb des Torplatzes auf Samuel zu mit der Bitte:
»Sage mir doch, wo hier die Wohnung des Sehers ist.« 19Samuel antwortete
dem Saul: »Ich bin der Seher: steige vor mir her die Höhe hinauf! Denn
ihr müßt heute mit mir essen; morgen früh will ich dich dann ziehen
lassen und dir Auskunft über alles geben, was dir auf dem Herzen liegt.
20Um die Eselinnen aber, die dir heute vor drei Tagen abhanden gekommen
sind, brauchst du dir keine Gedanken zu machen; denn sie haben sich
wiedergefunden. Und wem gehört alles, was es Wertvolles in Israel gibt?
Doch wohl dir und deines Vaters ganzem Hause?« 21Da entgegnete Saul:
»Ich bin doch nur ein Benjaminit, aus einem der kleinsten Stämme
Israels, und mein Geschlecht ist das geringste unter allen Geschlechtern
des Stammes Benjamin: warum redest du da solche Worte zu mir?« 22Samuel
aber nahm Saul samt seinem Knecht mit sich, führte sie in die Halle und
wies ihnen einen Platz an zuoberst der Gäste, deren ungefähr dreißig
anwesend waren. 23Hierauf sagte Samuel zu dem Koch: »Bringe das Stück
her, das ich dir übergeben und von dem ich dir gesagt habe, du möchtest
es beiseite legen.« 24Da brachte der Koch die Keule mit dem Nierenstück
herbei, Samuel aber setzte sie dem Saul vor und sagte: »Siehe, dies
Ehrenstück ist für dich aufgehoben! Lege es dir vor und iß! Denn bis zur
bestimmten Zeit\textless sup title=``=~für diese
Gelegenheit''\textgreater✲ ist es für dich aufbewahrt worden, schon als
ich sagte, ich hätte die Gäste geladen.« So aß denn Saul mit Samuel an
jenem Tage. 25Als sie dann von der Anhöhe zur Stadt hinuntergegangen
waren, bereitete man dem Saul das Lager auf dem Dach; 26und er legte
sich schlafen.

\hypertarget{c-saul-von-samuel-zum-kuxf6nig-gesalbt-seine-heimkehr-nach-gibea}{%
\paragraph{c) Saul von Samuel zum König gesalbt; seine Heimkehr nach
Gibea}\label{c-saul-von-samuel-zum-kuxf6nig-gesalbt-seine-heimkehr-nach-gibea}}

Als dann die Morgenröte aufging, rief Samuel aufs Dach hinauf dem Saul
die Worte zu: »Stehe auf! Ich will dich geleiten.« Da stand Saul auf,
und sie gingen beide, er und Samuel, hinaus auf die Straße. 27Während
sie nun ans Ende der Stadt\textless sup title=``oder: nach der Grenze
des Stadtgebiets''\textgreater✲ hinuntergingen, sagte Samuel zu Saul:
»Laß den Knecht uns vorausgehen! Du selbst aber bleibe jetzt hier
stehen: ich will dir ein Gotteswort kundtun!«

\hypertarget{section-9}{%
\section{10}\label{section-9}}

1Hierauf nahm Samuel die Ölflasche und goß sie ihm aufs Haupt, küßte ihn
dann und sagte: »Hiermit hat der HERR dich zum Fürsten über sein
Eigentumsvolk Israel gesalbt!«

\hypertarget{aa-samuel-prophezeit-dem-saul-drei-zeichen-die-auf-dem-heimwege-bei-ihm-eintreffen-sollen-und-bestellt-ihn-nach-gilgal}{%
\subparagraph{aa) Samuel prophezeit dem Saul drei Zeichen, die auf dem
Heimwege bei ihm eintreffen sollen, und bestellt ihn nach
Gilgal}\label{aa-samuel-prophezeit-dem-saul-drei-zeichen-die-auf-dem-heimwege-bei-ihm-eintreffen-sollen-und-bestellt-ihn-nach-gilgal}}

2»Wenn du jetzt von mir weggehst, wirst du beim Grabe der Rahel an der
Grenze von Benjamin, in Zelzah, zwei Männer treffen, die zu dir sagen
werden: ›Die Eselinnen, zu deren Aufsuchung du ausgezogen bist, haben
sich wiedergefunden; dein Vater denkt jetzt nicht mehr an den Vorfall
mit den Eselinnen, macht sich aber um euch Sorge und sagt: Was soll ich
wegen meines Sohnes tun?‹ 3Wenn du dann von dort weiterwanderst und zum
Terebinthenhain von Thabor gekommen bist, so werden dir dort drei Männer
begegnen, die zum Heiligtum Gottes nach Bethel hinaufgehen; der eine
trägt drei Böckchen, der andere drei Laibe Brot und der dritte einen
Schlauch Wein. 4Sie werden dich begrüßen und dir zwei Brote anbieten,
die du von ihnen annehmen sollst. 5Hierauf wirst du nach
Gibea-Elohim\textless sup title=``d.h. dem Hügel Gottes''\textgreater✲
kommen, wo die Säule der Philister steht, und beim Eintritt in den Ort
wirst du dort einer Schar Propheten begegnen, die von der Höhe
herabkommen, und vor ihnen her erschallt Musik von Harfen, Pauken,
Flöten und Zithern, während sie selbst sich in prophetischer
Begeisterung befinden. 6Da wird dann der Geist des HERRN auch über dich
kommen, so daß du mit ihnen in Begeisterung gerätst und in einen anderen
Menschen verwandelt wirst. 7Wenn nun diese Zeichen bei dir eingetroffen
sind, so tu, wozu du dich gerade getrieben fühlst, denn Gott ist mit
dir! 8Gehe vor mir nach Gilgal hinab; ich werde alsdann zu dir dorthin
kommen, um Brandopfer darzubringen und Heilsopfer zu schlachten. Sieben
Tage sollst du dort warten, bis ich zu dir komme und dir mitteile, was
du zu tun hast.«

\hypertarget{bb-das-eintreffen-der-angekuxfcndigten-zeichen-saul-unter-den-propheten}{%
\subparagraph{bb) Das Eintreffen der angekündigten Zeichen; Saul unter
den
Propheten}\label{bb-das-eintreffen-der-angekuxfcndigten-zeichen-saul-unter-den-propheten}}

9Sobald nun Saul den Rücken gewandt hatte, um von Samuel wegzugehen, da
verwandelte ihm Gott sein Herz; und alle diese Zeichen trafen an jenem
Tage ein. 10Denn als sie dorthin, nach Gibea, kamen und ihm die
Prophetenschar begegnete, da kam plötzlich der Geist Gottes über ihn, so
daß er mitten unter ihnen in prophetische Begeisterung geriet. 11Als nun
alle, die ihn von früher her kannten, sahen, wie er sich gleich den
anderen als Prophet gebärdete, fragten die Leute sich untereinander:
»Was ist denn mit dem Sohn des Kis vorgegangen? Gehört denn Saul auch zu
den Propheten?« 12Da antwortete ein Mann von dort: »Aber wer ist denn
deren Vater?« Daher rührt das Sprichwort: »Gehört Saul auch zu den
Propheten?«

\hypertarget{cc-saul-wieder-zu-hause-seine-zuruxfcckhaltende-unterredung-mit-seinem-vetter}{%
\subparagraph{cc) Saul wieder zu Hause; seine zurückhaltende Unterredung
mit seinem
Vetter}\label{cc-saul-wieder-zu-hause-seine-zuruxfcckhaltende-unterredung-mit-seinem-vetter}}

13Als dann seine prophetische Begeisterung vorüber war und er nach Hause
gekommen war, 14fragte sein Vetter ihn und seinen Knecht: »Wohin seid
ihr gegangen?« Er antwortete: »Wir wollten die Eselinnen suchen, und als
wir sahen, daß sie nirgends zu finden waren, sind wir zu Samuel
gegangen.« 15Da bat ihn sein Vetter: »Teile mir doch mit, was Samuel
euch gesagt hat!« 16Da erwiderte Saul seinem Vetter: »Nun, er hat uns
mitgeteilt, daß die Eselinnen sich wiedergefunden hätten.« Was ihm aber
Samuel in betreff des Königtums gesagt hatte, davon verriet er ihm
nichts.

\hypertarget{d-saul-wird-in-mizpa-durch-das-heilige-los-zum-kuxf6nig-bestimmt}{%
\paragraph{d) Saul wird in Mizpa durch das heilige Los zum König
bestimmt}\label{d-saul-wird-in-mizpa-durch-das-heilige-los-zum-kuxf6nig-bestimmt}}

17Hierauf berief Samuel das Volk zum (Heiligtum des) HERRN nach Mizpa
18und sagte zu den Israeliten: »So hat der HERR, der Gott Israels,
gesprochen: ›Ich bin's, der Israel aus Ägypten heraufgeführt und euch
aus der Hand der Ägypter und aus der Gewalt aller Königreiche, die euch
bedrängten, errettet hat. 19Ihr aber habt jetzt euren Gott verworfen,
der euch aus allen euren Nöten und Bedrängnissen errettet hat, und habt
zu ihm gesagt: ›Nein, einen König sollst du über uns einsetzen!‹ Nun
denn, so stellt euch vor dem HERRN nach euren Stämmen und nach euren
Tausendschaften auf!« 20Als dann Samuel alle Stämme Israels hatte
antreten lassen, wurde der Stamm Benjamin durchs Los getroffen; 21und
als er den Stamm Benjamin nach seinen Geschlechtern antreten ließ, wurde
das Geschlecht der Matriten und aus diesem Saul, der Sohn des Kis,
getroffen; als man aber nach ihm suchte, war er nicht zu finden. 22Da
fragte man nochmals beim HERRN an: »Ist der Mann überhaupt hergekommen?«
Der HERR antwortete: »Jawohl, er hält sich beim Gepäck versteckt.« 23Da
eilte man hin und holte ihn von dort; und als er dann mitten unter das
Volk trat, überragte er alle anderen um eines Hauptes Länge\textless sup
title=``vgl. 9,2''\textgreater✲. 24Da sagte Samuel zu dem ganzen Volk:
»Seht ihr wohl, wen der HERR sich erwählt hat? Diesem kommt keiner im
ganzen Volke gleich!« Da erhob das ganze Volk ein Jubelgeschrei und
rief: »Es lebe der König!«

25Darauf brachte Samuel die Rechte des Königtums zur Kenntnis des Volkes
und trug sie in ein Buch ein, das er vor den HERRN niederlegte; danach
entließ Samuel das ganze Volk, einen jeden in seine Heimat. 26Auch Saul
kehrte in seinen Wohnort nach Gibea zurück, und eine Kriegerschar der
Tapferen, denen Gott das Herz gerührt hatte, gab ihm das Geleit.
27Einige nichtswürdige Leute aber sagten: »Was wird der uns helfen
können?« So verachteten sie ihn und ließen ihm kein Huldigungsgeschenk
zukommen; doch er tat, als merkte er es nicht.

\hypertarget{sauls-erste-taten-und-seine-verwerfung-kap.-11-15}{%
\subsubsection{4. Sauls erste Taten und seine Verwerfung (Kap.
11-15)}\label{sauls-erste-taten-und-seine-verwerfung-kap.-11-15}}

\hypertarget{a-sauls-sieg-uxfcber-die-ammoniter-bestuxe4tigung-seines-kuxf6nigtums-in-gilgal}{%
\paragraph{a) Sauls Sieg über die Ammoniter; Bestätigung seines
Königtums in
Gilgal}\label{a-sauls-sieg-uxfcber-die-ammoniter-bestuxe4tigung-seines-kuxf6nigtums-in-gilgal}}

\hypertarget{aa-die-von-dem-ammoniter-nahas-schwer-bedruxe4ngte-stadt-jabes-ruft-die-hilfe-der-israeliten-an}{%
\subparagraph{aa) Die von dem Ammoniter Nahas schwer bedrängte Stadt
Jabes ruft die Hilfe der Israeliten
an}\label{aa-die-von-dem-ammoniter-nahas-schwer-bedruxe4ngte-stadt-jabes-ruft-die-hilfe-der-israeliten-an}}

\hypertarget{section-10}{%
\section{11}\label{section-10}}

1(Nach ungefähr einem Monat aber) zog der Ammoniter Nahas heran und
belagerte Jabes in Gilead. Da ließ die gesamte Bürgerschaft von Jabes
dem Nahas sagen: »Schließe einen Vertrag mit uns, so wollen wir uns dir
unterwerfen.« 2Aber der Ammoniter Nahas antwortete ihnen: »Nur unter der
Bedingung will ich einen Vertrag mit euch schließen, daß ich jedem von
euch das rechte Auge aussteche und dadurch ganz Israel beschimpfe.« 3Da
ließen ihm die Ältesten von Jabes sagen: »Gewähre uns eine Frist von
sieben Tagen, damit wir Boten in alle Teile Israels senden! Wenn dann
niemand uns Hilfe leistet, so wollen wir uns dir ergeben.«

\hypertarget{bb-sauls-entschlossenes-auftreten-und-herrlicher-sieg}{%
\subparagraph{bb) Sauls entschlossenes Auftreten und herrlicher
Sieg}\label{bb-sauls-entschlossenes-auftreten-und-herrlicher-sieg}}

4Als nun die Boten nach Gibea kamen, wo Saul wohnte, und ihr Anliegen
dem Volke vortrugen, brachen alle Leute in lautes Weinen aus. 5Saul aber
kam gerade hinter den Rindern her vom Felde heim und fragte: »Was hat
das Volk, daß es weint?« Als man ihm dann mitteilte, was die Männer von
Jabes berichtet hatten, 6kam beim Vernehmen dieser Nachricht der Geist
Gottes über ihn, und er geriet in heftigen Zorn. 7Er nahm also ein Paar
Rinder, zerstückte sie, sandte die Stücke durch Boten in alle Teile
Israels und ließ bekanntmachen: »Wer nicht mit auszieht hinter Saul und
Samuel her, dessen Rindern soll es ebenso ergehen!« Da fiel ein heiliger
Schrecken auf das Volk, so daß sie auszogen wie ein Mann; 8und als er
sie bei Besek musterte, waren es 300000 Israeliten und 30000~Mann vom
Stamme Juda.

9Man sagte nun zu den Boten, die gekommen waren: »Meldet den Einwohnern
von Jabes in Gilead folgendes: ›Morgen, wenn die Sonne heiß scheint,
wird euch Hilfe werden!‹« Als nun die Boten bei ihrer Heimkehr dieses
den Bürgern von Jabes berichteten, freuten diese sich 10und ließen den
Ammonitern sagen: »Morgen wollen wir zu euch hinausziehen\textless sup
title=``=~uns euch ergeben''\textgreater✲; dann mögt ihr mit uns
verfahren, wie es euch beliebt!« 11Als Saul dann am andern Morgen seine
Mannschaft in drei Heerhaufen geteilt hatte und diese in das feindliche
Lager zur Zeit der Morgenwache eindrangen, richteten sie ein Blutbad
unter den Ammonitern an, bis der Tag am heißesten wurde; und die
Übriggebliebenen wurden so zersprengt, daß nicht zwei von ihnen
beisammen blieben.

\hypertarget{cc-sauls-grouxdfmut-gegen-seine-veruxe4chter-freudenfest-in-gilgal}{%
\subparagraph{cc) Sauls Großmut gegen seine Verächter; Freudenfest in
Gilgal}\label{cc-sauls-grouxdfmut-gegen-seine-veruxe4chter-freudenfest-in-gilgal}}

12Da sagte das Volk zu Samuel: »Wer sind die, welche gesagt haben: ›Saul
sollte König über uns sein?‹ Her mit diesen Männern, daß wir sie
totschlagen!« 13Saul aber entgegnete: »Niemand soll heute den Tod
erleiden, denn heute hat der HERR dem Volke Israel Rettung
geschafft\textless sup title=``oder: Heil verliehen''\textgreater✲!«
14Samuel aber forderte das Volk auf: »Kommt, laßt uns nach Gilgal ziehen
und dort das Königtum bestätigen!« 15Da zog das gesamte Volk nach Gilgal
und setzte dort Saul vor dem HERRN in Gilgal zum König ein. Man
schlachtete dort Heilsopfer vor dem HERRN, und Saul samt allen
Israeliten feierten dort ein großes Freudenfest.

\hypertarget{b-samuels-freiwilliger-ruxfccktritt-und-feierlicher-abschied-vom-volke}{%
\paragraph{b) Samuels freiwilliger Rücktritt und feierlicher Abschied
vom
Volke}\label{b-samuels-freiwilliger-ruxfccktritt-und-feierlicher-abschied-vom-volke}}

\hypertarget{aa-das-volk-bestuxe4tigt-dem-samuel-die-unstruxe4fliche-fuxfchrung-des-richteramtes}{%
\subparagraph{aa) Das Volk bestätigt dem Samuel die unsträfliche Führung
des
Richteramtes}\label{aa-das-volk-bestuxe4tigt-dem-samuel-die-unstruxe4fliche-fuxfchrung-des-richteramtes}}

\hypertarget{section-11}{%
\section{12}\label{section-11}}

1Darauf sagte Samuel zu ganz Israel: »Seht, ich bin in allem, was ihr
mir vorgetragen habt, eurem Wunsche nachgekommen und habe einen König
über euch eingesetzt. 2So wird denn nunmehr der König an eurer Spitze
einhergehen. Ich aber bin alt und grau geworden, so daß nun meine Söhne
unter euch sind. Ich habe aber meinen Wandel von meiner Jugend an bis
auf den heutigen Tag vor euren Augen geführt. 3Hier stehe ich: tretet
vor dem HERRN und seinem Gesalbten gegen mich auf! Wem habe ich seinen
Ochsen, wem seinen Esel weggenommen? Wen habe ich übervorteilt, wem
Gewalt angetan? Oder von wem habe ich ein Geschenk angenommen, daß ich
mir dadurch die Augen hätte blenden lassen? -- so will ich es euch
zurückerstatten!« 4Da antworteten sie: »Du hast uns nicht übervorteilt
und uns keine Gewalt angetan, hast auch von niemand irgend etwas
angenommen.« 5Darauf fuhr er fort: »Der HERR ist heute mein Zeuge euch
gegenüber, und ebenso ist auch sein Gesalbter Zeuge, daß ihr gar nichts
(von unrechtem Gut) in meinem Besitz gefunden habt.« Sie riefen: »Ja, er
ist Zeuge!«

\hypertarget{bb-samuel-erinnert-das-volk-an-die-vielen-wohltaten-gottes}{%
\subparagraph{bb) Samuel erinnert das Volk an die vielen Wohltaten
Gottes}\label{bb-samuel-erinnert-das-volk-an-die-vielen-wohltaten-gottes}}

6Hierauf sagte Samuel weiter zum Volke: »Ja, Zeuge ist der HERR, der
Mose und Aaron geschaffen und der eure Väter aus dem Lande Ägypten
hergeführt hat! 7Jetzt aber tretet her, damit ich euch vor dem HERRN ins
Gewissen rede und euch alle Wohltaten vorhalte, die der HERR euch und
euren Vätern erwiesen hat! 8Als Jakob nach Ägypten gekommen war und eure
Väter zum HERRN um Hilfe schrien, da sandte der HERR Mose und Aaron,
damit sie eure Väter aus Ägypten wegführten und ihnen Wohnsitze in
diesem Lande gäben. 9Als sie (eure Väter) aber den HERRN, ihren Gott,
vergaßen, ließ er sie in die Gewalt Siseras, des Heerführers (des Königs
Jabin) von Hazor, fallen und in die Gewalt der Philister und in die
Gewalt des Königs der Moabiter; die fingen Kriege mit ihnen an. 10Als
sie nun zum HERRN um Hilfe schrien und bekannten: ›Wir haben gesündigt,
daß wir den HERRN verlassen und den Baalen und Astarten gedient haben!
Jetzt aber errette uns aus der Gewalt unserer Feinde, so wollen wir dir
dienen!«~-- 11da sandte der HERR Jerubbaal und Barak, Jephthah und
Samuel und errettete euch aus der Gewalt eurer Feinde ringsum, so daß
ihr in Sicherheit wohnen konntet.«

\hypertarget{cc-samuel-beweist-dem-volke-durch-ein-wunderbares-gotteszeichen-dauxdf-es-sich-durch-die-wahl-eines-kuxf6nigs-versuxfcndigt-habe}{%
\subparagraph{cc) Samuel beweist dem Volke durch ein wunderbares
Gotteszeichen, daß es sich durch die Wahl eines Königs versündigt
habe}\label{cc-samuel-beweist-dem-volke-durch-ein-wunderbares-gotteszeichen-dauxdf-es-sich-durch-die-wahl-eines-kuxf6nigs-versuxfcndigt-habe}}

12»Als ihr aber saht, daß Nahas, der König der Ammoniter, gegen euch
heranrückte, da sagtet ihr zu mir: ›Nein, ein König soll über uns
herrschen!‹, während doch der HERR, euer Gott, euer König ist. 13Nun
denn -- da ist der König, den ihr erwählt, den ihr verlangt habt; ja,
der HERR hat nun einen König über euch eingesetzt. 14Werdet ihr nun den
HERRN fürchten und ihm dienen, seinen Weisungen gehorchen und euch gegen
die Befehle des HERRN nicht auflehnen, sondern ihr beide, sowohl ihr
selbst als auch der König, der über euch herrscht, dem HERRN, eurem
Gott, folgsam sein? 15Wenn ihr aber den Weisungen des HERRN nicht
gehorcht, sondern euch gegen die Befehle des HERRN auflehnt, so wird die
Hand des HERRN gegen euch sein, wie sie vormals gegen eure Väter gewesen
ist. 16Jetzt aber tretet her und achtet auf das große Ereignis, das der
HERR vor euren Augen wird geschehen lassen. 17Es ist jetzt doch die Zeit
der Weizenernte; ich will aber den HERRN anrufen, er möge Donnerschläge
und Regen kommen lassen; dann werdet ihr erkennen und einsehen, ein wie
großes Unrecht ihr vor den Augen des HERRN begangen habt, indem ihr
einen König für euch verlangtet!« 18Als dann Samuel den HERRN anrief,
ließ dieser an jenem Tage Donner und Regen\textless sup title=``d.h. ein
Gewitter''\textgreater✲ kommen, so daß das ganze Volk in große Furcht
vor dem HERRN und vor Samuel geriet.

\hypertarget{dd-samuel-ermutigt-das-volk-ermahnt-es-zur-gottesfurcht-und-befiehlt-es-dem-guxf6ttlichen-segen}{%
\subparagraph{dd) Samuel ermutigt das Volk, ermahnt es zur Gottesfurcht
und befiehlt es dem göttlichen
Segen}\label{dd-samuel-ermutigt-das-volk-ermahnt-es-zur-gottesfurcht-und-befiehlt-es-dem-guxf6ttlichen-segen}}

19Da richtete das ganze Volk die Aufforderung an Samuel: »Lege Fürbitte
für deine Knechte bei dem HERRN, deinem Gott, ein, daß wir nicht zu
sterben brauchen, weil wir zu allen unsern Sünden auch noch das Unrecht
hinzugefügt haben, einen König für uns zu verlangen!« 20Samuel aber
erwiderte dem Volke: »Fürchtet euch nicht! Ihr habt zwar all dieses
Unrecht begangen; aber weicht nur nicht vom Gehorsam gegen den HERRN ab,
sondern dient dem HERRN mit eurem ganzen Herzen 21und fallt nicht von
ihm ab, daß ihr den Götzen nachgeht, die nicht helfen und nicht erretten
können, weil sie ja nichts sind. 22Der HERR dagegen wird um seines
großen Namens willen sein Volk nicht verstoßen, weil es dem HERRN
gefallen hat, euch zu seinem Volke zu machen. 23Auch von mir sei es
fern, mich am HERRN zu versündigen, daß ich davon ablassen sollte,
Fürbitte für euch einzulegen! Nein, ich will euch den guten und rechten
Weg lehren. 24Nur fürchtet den HERRN und dient ihm aufrichtig mit eurem
ganzen Herzen; denn beachtet wohl, was er Großes an euch getan hat!
25Wenn ihr aber dennoch böse handelt, so werdet ihr samt eurem König
hinweggerafft werden\textless sup title=``=~verloren
sein''\textgreater✲.«

\hypertarget{c-ausbruch-des-philisterkrieges-sauls-erster-ungehorsam-durch-voreiliges-opfern}{%
\paragraph{c) Ausbruch des Philisterkrieges; Sauls erster Ungehorsam
durch voreiliges
Opfern}\label{c-ausbruch-des-philisterkrieges-sauls-erster-ungehorsam-durch-voreiliges-opfern}}

\hypertarget{aa-zertruxfcmmerung-der-suxe4ule-der-philister-aufbietung-des-israelitischen-heerbannes-heranzug-der-philister-verzagtheit-der-israeliten}{%
\subparagraph{aa) Zertrümmerung der Säule der Philister; Aufbietung des
israelitischen Heerbannes; Heranzug der Philister; Verzagtheit der
Israeliten}\label{aa-zertruxfcmmerung-der-suxe4ule-der-philister-aufbietung-des-israelitischen-heerbannes-heranzug-der-philister-verzagtheit-der-israeliten}}

\hypertarget{section-12}{%
\section{13}\label{section-12}}

1Saul war~\ldots{} Jahre alt, als er König wurde, und herrschte~\ldots{}
Jahre über Israel. 2Er wählte sich dreitausend Mann aus Israel aus, von
denen sich zweitausend bei ihm zu Michmas und im Gebirge von Bethel
befanden, während tausend Mann unter Jonathan zu Gibea im Stamme
Benjamin standen; das übrige Kriegsvolk hatte er entlassen, einen jeden
zu seinen Zelten\textless sup title=``=~in seinen
Wohnort''\textgreater✲. 3Da zertrümmerte Jonathan die Säule\textless sup
title=``vgl. 10,5''\textgreater✲ der Philister, die in Gibea stand. Das
kam zur Kenntnis der Philister; Saul aber ließ die Posaune im ganzen
Lande erschallen und sagte: »Die Hebräer sollen es hören!« 4Als nun ganz
Israel die Kunde vernahm, Saul habe die Säule der Philister zertrümmert
und Israel habe dadurch die Philister tödlich beleidigt, da wurde das
Volk aufgeboten, dem Saul nach Gilgal zu folgen. 5Die Philister aber
sammelten sich zum Kampf mit Israel: dreitausend Kriegswagen und
sechstausend Reiter und Fußvolk so zahlreich wie der Sand am Ufer des
Meeres; die zogen herauf und lagerten bei Michmas östlich von Beth-Awen.
6Als nun die Mannschaft der Israeliten sah, daß sie sich in einer
schlimmen Lage und in arger Bedrängnis befand, versteckten sich die
Leute in den Höhlen und Dickichten\textless sup title=``oder:
Erdlöchern''\textgreater✲, in den Felsspalten, Gräbern und Zisternen;
7manche begaben sich sogar (über die Jordanfurten) ins Land der Gaditen
und nach Gilead. Saul aber befand sich immer noch in Gilgal, und das
ganze Kriegsvolk folgte ihm zitternd.

\hypertarget{bb-sauls-voreiliges-und-eigenmuxe4chtiges-opfern-in-gilgal-bruch-zwischen-samuel-und-dem-kuxf6nige-sauls-verwerfung}{%
\subparagraph{bb) Sauls voreiliges und eigenmächtiges Opfern in Gilgal;
Bruch zwischen Samuel und dem Könige; Sauls
Verwerfung}\label{bb-sauls-voreiliges-und-eigenmuxe4chtiges-opfern-in-gilgal-bruch-zwischen-samuel-und-dem-kuxf6nige-sauls-verwerfung}}

8Er wartete nun sieben Tage bis zu der von Samuel bestimmten Zeit; als
Samuel aber nicht nach Gilgal kam und seine Leute ihn verließen und sich
zerstreuten, 9befahl Saul: »Bringt mir (die Tiere für) das Brandopfer
und die Heilsopfer her!« Und er vollzog das Brandopfer. 10Kaum war er
aber mit der Darbringung des Brandopfers fertig, als Samuel erschien.
Saul ging hinaus ihm entgegen, um ihn zu begrüßen; 11Samuel aber fragte:
»Was hast du getan?« Saul antwortete: »Weil ich sah, daß das Kriegsvolk
mich verließ und sich zerstreute und du zur bestimmten Zeit nicht kamst,
die Philister aber sich schon bei Michmas gesammelt haben, 12da dachte
ich: ›Jetzt werden die Philister gegen mich nach Gilgal herabziehen, ehe
ich noch die Huld des HERRN für mich gewonnen habe‹; da habe ich mir
denn ein Herz gefaßt und das Brandopfer dargebracht.« 13Da sagte Samuel
zu Saul: »Du hast töricht gehandelt, daß du das Gebot, das der HERR,
dein Gott, dir gegeben hat, nicht beobachtet hast, sonst hätte der HERR
jetzt dein Königtum über Israel für immer bestätigt; 14nun aber wird
dein Königtum keinen Bestand haben. Der HERR hat sich einen Mann nach
seinem Herzen gesucht, und der HERR hat ihn zum Fürsten über sein Volk
bestellt; denn du hast nicht befolgt, was der HERR dir geboten hatte.«

\hypertarget{cc-die-geringe-heeresmacht-sauls-die-pluxfcnderungen-der-philister-wehrlosigkeit-der-israeliten}{%
\subparagraph{cc) Die geringe Heeresmacht Sauls; die Plünderungen der
Philister; Wehrlosigkeit der
Israeliten}\label{cc-die-geringe-heeresmacht-sauls-die-pluxfcnderungen-der-philister-wehrlosigkeit-der-israeliten}}

15Hierauf machte Samuel sich auf, verließ Gilgal und ging seines Weges;
der Rest des Heeres aber folgte dem Saul von Gilgal nach Gibea im Stamme
Benjamin; dort musterte Saul das Kriegsvolk, das sich noch bei ihm
befand, etwa sechshundert Mann, 16und verblieb mit seinem Sohne Jonathan
und der ihm zur Verfügung stehenden Mannschaft zu Geba im Stamme
Benjamin, während die Philister sich bei Michmas gelagert hatten. 17Da
zog die plündernde Schar aus dem Lager der Philister in drei Abteilungen
aus: die eine Abteilung wandte sich in der Richtung gegen Ophra in die
Landschaft Sual, 18die zweite Abteilung zog in der Richtung gegen
Beth-Horon, und die dritte Abteilung schlug die Richtung nach dem Hügel
ein, der über die Hyänenschlucht emporragt, nach der Wüste hin. 19Es war
aber kein Schmied im ganzen Lande Israel zu finden; denn die Philister
hatten gedacht: »Die Hebräer sollen sich keine Schwerter und Spieße
anfertigen dürfen!« 20Daher mußten alle Israeliten zu den Philistern
hinabgehen, sooft jemand seine Pflugschar oder Hacke, seine Axt oder
seine Sichel\textless sup title=``oder: seinen
Ochsenstecken''\textgreater✲ zu schärfen hatte 21und sooft die Schneiden
an den Pflugscharen oder an den Spaten und Hacken und Äxten stumpf
geworden waren und um die Ochsenstecken gerade zu richten. 22So kam es
denn, daß beim Ausbruch des Krieges kein Schwert und kein Spieß in der
Hand des ganzen Kriegsvolks, das Saul und Jonathan bei sich hatten, zu
finden war; nur Saul und sein Sohn Jonathan besaßen Waffen.~-- 23Ein
Posten der Philister aber war nach dem Paß von Michmas vorgerückt.

\hypertarget{d-jonathans-heldentat-sauls-sieg-uxfcber-die-philister}{%
\paragraph{d) Jonathans Heldentat; Sauls Sieg über die
Philister}\label{d-jonathans-heldentat-sauls-sieg-uxfcber-die-philister}}

\hypertarget{aa-jonathans-kuxfchner-handstreich-schrecken-im-lager-der-philister}{%
\subparagraph{aa) Jonathans kühner Handstreich; Schrecken im Lager der
Philister}\label{aa-jonathans-kuxfchner-handstreich-schrecken-im-lager-der-philister}}

\hypertarget{section-13}{%
\section{14}\label{section-13}}

1Eines Tages nun sagte Jonathan, der Sohn Sauls, zu dem Burschen, der
sein Waffenträger war: »Komm, wir wollen auf den Vorposten der Philister
losgehen, der dort drüben steht!« Seinem Vater aber sagte er nichts
davon; 2denn Saul befand sich gerade an der Grenze von Gibea unter dem
Granatbaume, der bei Migron\textless sup title=``oder: auf dem
Tennenplatz''\textgreater✲ steht; und die Leute, die er bei sich hatte,
machten ungefähr 600~Mann aus, 3und Ahia, der Sohn Ahitubs, des Bruders
Ikabods, des Sohnes des Pinehas, des Sohnes Elis, des Priesters des
HERRN zu Silo, trug damals das priesterliche Schulterkleid; auch das
Kriegsvolk wußte nichts davon, daß Jonathan weggegangen war.

4Es lag aber an der Übergangsstelle, durch welche Jonathan an den Posten
der Philister heranzukommen suchte, eine Felsspitze diesseits und eine
Felsspitze jenseits; die eine hieß Bozez\textless sup title=``d.h.
Blinker''\textgreater✲, die andere Sene\textless sup title=``d.h.
Dorn''\textgreater✲. 5Die eine Felsspitze fiel steil nach Norden ab
gegen Michmas, die andere nach Süden zu gegen Geba. 6Jonathan sagte also
zu seinem Waffenträger: »Komm, wir wollen auf den Posten dieser Heiden
drüben losgehen; vielleicht läßt der HERR uns etwas ausrichten; denn für
den HERRN gibt es kein Hindernis, durch viele oder durch wenige (Leute)
zu retten\textless sup title=``oder: den Sieg zu
gewinnen''\textgreater✲.« 7Da antwortete ihm sein Waffenträger: »Mache
es ganz so, wie du es beabsichtigst; ich bin mit allem einverstanden und
zu allem bereit.« 8Jonathan fuhr fort: »Gut! Wir gehen hinüber auf die
Leute los und wollen uns ihnen zeigen; 9wenn sie uns dann zurufen:
›Steht still, bis wir zu euch hinkommen!‹, so wollen wir auf unserem
Platze stehenbleiben und nicht zu ihnen hinaufsteigen; 10wenn sie uns
aber so zurufen: ›Kommt nur zu uns herauf!‹, so wollen wir zu ihnen
hinaufsteigen; denn dann hat der HERR sie in unsere Hand gegeben: dies
soll uns als Zeichen dienen!«

11Als nun die beiden dem Posten der Philister sichtbar wurden, sagten
die Philister: »Seht, da kommen ja Hebräer aus den Löchern hervor, in
die sie sich verkrochen haben!« 12Hierauf riefen die Mannschaften, die
dort auf Posten standen, dem Jonathan und seinem Waffenträger zu: »Kommt
nur herauf zu uns! Wir wollen euch einen Denkzettel geben!« Da sagte
Jonathan zu seinem Waffenträger: »Steige mir nach, denn der HERR hat sie
in die Hand Israels gegeben!« 13So kletterte denn Jonathan auf Händen
und Füßen hinan und sein Waffenträger hinter ihm her. (Jene wollten sich
zur Flucht vor Jonathan wenden, aber er hieb sie nieder), und sein
Waffenträger tötete sie vollends hinter ihm her. 14So belief sich das
erste Blutbad, das Jonathan mit seinem Waffenträger anrichtete, auf
ungefähr zwanzig Mann, auf einer Strecke nicht größer als eine halbe
Hufe Ackers✲. 15Da entstand ein Schrecken im Lager auf dem Felde und
unter dem ganzen Kriegsvolk; auch die auf Posten Stehenden und die
Plünderschar gerieten in Bestürzung; dazu bebte die Erde, und das rief
einen Gottesschrecken hervor.

\hypertarget{bb-saul-greift-in-den-kampf-ein-und-truxe4gt-einen-gluxe4nzenden-sieg-davon}{%
\subparagraph{bb) Saul greift in den Kampf ein und trägt einen
glänzenden Sieg
davon}\label{bb-saul-greift-in-den-kampf-ein-und-truxe4gt-einen-gluxe4nzenden-sieg-davon}}

16Als nun die Späher Sauls, die sich zu Gibea im Stamm Benjamin
befanden, ausschauten, da sahen sie, wie die Menge\textless sup
title=``=~das Lager der Philister''\textgreater✲ hin und her wogte.
17Nun befahl Saul den Leuten, die bei ihm waren: »Nehmt eine Musterung
vor und seht zu, wer von uns weggegangen ist!« Als man nun die Musterung
vornahm, stellte es sich heraus, daß Jonathan und sein Waffenträger
fehlten. 18Da befahl Saul dem Ahia✲: »Bringe die Lade Gottes her!« Denn
die Lade Gottes befand sich damals bei den Israeliten. 19Während aber
Saul noch mit dem Priester redete, wurde das Getümmel im Lager der
Philister immer größer; daher befahl Saul dem Priester: »Laß es sein!«
20Darauf trat Saul mit der ganzen Mannschaft, die bei ihm war, zum Kampf
an; doch als sie an das feindliche Lager kamen✲, fanden sie das Schwert
des einen gegen den andern gekehrt, und es herrschte eine heillose
Verwirrung. 21Auch die Hebräer, die es seit längerer Zeit mit den
Philistern gehalten hatten und mit ihnen ins Feld gezogen waren, auch
diese fielen jetzt ab, um sich den Israeliten unter Saul und Jonathan
anzuschließen. 22Als ferner alle Israeliten, die sich im Gebirge Ephraim
versteckt hielten, von der Flucht der Philister hörten, setzten sie
ihnen gleichfalls nach, um sie zu bekämpfen. 23So verlieh der HERR den
Israeliten an jenem Tage den Sieg.

\hypertarget{e-sauls-unzeitiger-eifer-jonathan-mit-dem-tode-bedroht-sauls-kriege-und-seine-familie}{%
\paragraph{e) Sauls unzeitiger Eifer; Jonathan mit dem Tode bedroht;
Sauls Kriege und seine
Familie}\label{e-sauls-unzeitiger-eifer-jonathan-mit-dem-tode-bedroht-sauls-kriege-und-seine-familie}}

\hypertarget{aa-sauls-unbedachter-befehl-von-jonathan-unwissentlich-uxfcbertreten-das-heer-bei-der-abendmahlzeit-durch-sauls-eingreifen-vor-versuxfcndigung-bewahrt}{%
\subparagraph{aa) Sauls unbedachter Befehl von Jonathan unwissentlich
übertreten; das Heer bei der Abendmahlzeit durch Sauls Eingreifen vor
Versündigung
bewahrt}\label{aa-sauls-unbedachter-befehl-von-jonathan-unwissentlich-uxfcbertreten-das-heer-bei-der-abendmahlzeit-durch-sauls-eingreifen-vor-versuxfcndigung-bewahrt}}

Als aber der Kampf sich bis über Beth-Awen hin ausbreitete, 24wurde die
Mannschaft der Israeliten im Laufe jenes Tages sehr müde. Saul hatte
nämlich seine Leute durch folgenden Fluch gebunden: »Verflucht ist
jeder, der bis zum Abend etwas genießt, bis ich Rache an meinen Feinden
genommen habe!« So nahm denn auch keiner von den Leuten Nahrung zu sich.
25Nun hatte sich damals die ganze Gegend mit Bienenwirtschaft befaßt,
und die Bienenstöcke befanden sich auf freiem Felde. 26Als nun das
Kriegsvolk zu den Stöcken kam, da flossen sie von Honig über; aber
niemand führte seine Hand zum Munde, weil die Leute sich vor dem Fluch
scheuten. 27Da Jonathan es aber nicht gehört hatte, als sein Vater das
Kriegsvolk beschwor, streckte er seinen Stab aus, den er in der Hand
hatte, tauchte seine Spitze in den Honigseim und führte seine Hand zum
Munde: da wurden seine Augen leuchtend. 28Einer von den Mannschaften
aber teilte ihm mit: »Dein Vater hat das Heer durch folgenden
feierlichen Fluch gebunden: ›Verflucht ist jeder, der heute etwas
genießt!‹« Das Heer war aber todmüde, 29und Jonathan antwortete: »Mein
Vater stürzt das Land ins Unglück! Seht doch, wie leuchtend meine Augen
geworden sind, weil ich ein wenig von diesem Honig genossen habe! 30Was
wäre es erst gewesen, wenn die Leute von der feindlichen Beute, die sie
vorgefunden haben, gehörig hätten essen dürfen! So aber ist die
Niederlage unter den Philistern nicht groß geworden«. 31Sie hatten aber
an jenem Tage ein Blutbad unter den Philistern von Michmas bis nach
Ajjalon angerichtet, obgleich das Kriegsvolk sehr ermattet war.

32(Am Abend) aber fielen die Leute über die Beute her, nahmen Kleinvieh,
Rinder und Kälber und schlachteten sie zur Erde hin\textless sup
title=``d.h. so daß das Blut auf die Erde floß''\textgreater✲, und die
Leute aßen das Fleisch samt dem Blut. 33Als man nun dem Saul meldete:
»Die Leute versündigen sich ja gegen den HERRN, indem sie das Fleisch
samt dem Blut essen«, rief er aus: »Ihr handelt gottlos! Wälzt mir einen
großen Stein hierher!« 34Dann befahl Saul: »Zerstreut euch unter die
Leute und macht ihnen bekannt: ›Bringt ein jeder sein Rind und ein jeder
sein Stück Kleinvieh zu mir her und schlachtet die Tiere hier und eßt
dann erst! Sonst versündigt ihr euch gegen den HERRN, indem ihr das
Fleisch samt dem Blute genießt.‹« So brachte denn jeder von den Leuten
das Stück Vieh, das in seinem Besitz war, an jenem Abend herbei und
schlachtete es dort. 35Dann baute Saul dem HERRN einen Altar; dies war
der erste Altar, den er dem HERRN erbaute.

\hypertarget{bb-jonathan-durch-sauls-blinden-eifer-mit-dem-tode-bedroht-wird-durch-das-heer-gerettet}{%
\subparagraph{bb) Jonathan, durch Sauls blinden Eifer mit dem Tode
bedroht, wird durch das Heer
gerettet}\label{bb-jonathan-durch-sauls-blinden-eifer-mit-dem-tode-bedroht-wird-durch-das-heer-gerettet}}

36Hierauf sagte Saul: »Laßt uns noch in der Nacht hinabziehen hinter den
Philistern her, damit wir bis Tagesanbruch Beute unter ihnen machen und
keinen von ihnen übriglassen!« Sie antworteten: »Tu ganz, wie es dir gut
scheint!« Der Priester aber sagte: »Laßt uns zuerst hier vor Gott
treten!« 37Als nun Saul bei Gott anfragte: »Soll ich zur Verfolgung der
Philister hinabziehen? Wirst du sie in die Hand Israels geben?«,
erteilte ihm der HERR an jenem Tage keine Antwort. 38Da befahl Saul:
»Tretet hierher, ihr Anführer des Heeres alle, und untersucht
sorgfältig, durch wen\textless sup title=``oder: wodurch''\textgreater✲
diese Versündigung heute begangen worden ist! 39Denn so wahr der HERR
lebt, der Israel den Sieg verliehen hat: selbst wenn die Schuld sich bei
meinem Sohne Jonathan fände, so müßte er unfehlbar sterben!« Aber keiner
von allen Leuten gab ihm eine Antwort. 40Hierauf befahl er dem gesamten
Israel: »Ihr sollt auf der einen Seite stehen, ich aber und mein Sohn
Jonathan wollen die andere Seite bilden.« Das Heer antwortete ihm: »Tu,
was dir gut dünkt.« 41Dann betete Saul zum HERRN: »Gott Israels, laß die
Wahrheit zutage treten!« Da wurden Jonathan und Saul getroffen, das Heer
aber ging frei aus. 42Darauf befahl Saul: »Werft das Los zwischen mir
und meinem Sohne Jonathan!« Da wurde Jonathan getroffen. 43Nun sagte
Saul zu Jonathan: »Bekenne mir, was du getan hast!« 44Da bekannte ihm
Jonathan: »Ich habe nur ein wenig Honig mit der Spitze des Stabes
gekostet, den ich in meiner Hand hatte: dafür soll ich jetzt sterben?«
Saul erwiderte: »Gott tue mir jetzt und künftig an, was er will: ja,
Jonathan, du mußt unbedingt sterben!« 45Aber das Heer erklärte dem Saul:
»Jonathan soll sterben, der diesen großen Sieg in Israel errungen hat?
Das sei fern! So wahr der HERR lebt: kein Haar soll von seinem Haupt auf
die Erde fallen! Denn mit Gott im Bunde hat er den Sieg heute errungen!«
So machte das Heer den Jonathan frei, daß er nicht zu sterben brauchte.
46Hierauf stand Saul von der Verfolgung der Philister ab und zog heim,
während die Philister in ihr Land zurückkehrten.

\hypertarget{cc-sauls-sonstige-kriegstaten-und-seine-familie}{%
\subparagraph{cc) Sauls sonstige Kriegstaten und seine
Familie}\label{cc-sauls-sonstige-kriegstaten-und-seine-familie}}

47Nachdem Saul das Königtum über Israel übernommen hatte, führte er
Kriege gegen alle seine Feinde ringsum: gegen die Moabiter, die
Ammoniter und die Edomiter, gegen die Könige von Zoba und gegen die
Philister, und überall, wohin er sich wandte, war er siegreich. 48Er
bewies sich als tapferen Mann, besiegte die Amalekiter und befreite
Israel von denen, die es (bis dahin) ausgeplündert hatten.~--

49Die Söhne Sauls waren: Jonathan, Jiswi\textless sup title=``oder:
Isjo''\textgreater✲ und Malkisua; und von seinen zwei Töchtern hieß die
ältere Merab und die jüngere Michal. 50Sauls Gattin hieß Ahinoam, sie
war die Tochter des Ahimaaz; sein Heerführer hieß Abner und war der Sohn
Ners, des Oheims Sauls; 51denn Kis, der Vater Sauls, und Ner, der Vater
Abners, waren beide Söhne Abiels.~-- 52Mit den Philistern aber hatte
Saul schwere Kämpfe zu bestehen, solange er lebte; wenn Saul daher
irgendwo einen tapferen und kriegstüchtigen Mann sah, nahm er ihn in
seine Dienste.

\hypertarget{f-sauls-feldzug-gegen-die-amalekiter-sein-ungehorsam-gegen-gott-und-seine-verwerfung}{%
\paragraph{f) Sauls Feldzug gegen die Amalekiter; sein Ungehorsam gegen
Gott und seine
Verwerfung}\label{f-sauls-feldzug-gegen-die-amalekiter-sein-ungehorsam-gegen-gott-und-seine-verwerfung}}

\hypertarget{aa-saul-unternimmt-auf-guxf6ttlichen-befehl-den-rachekrieg-gegen-die-amalekiter-vollzieht-aber-den-bann-in-ungenuxfcgender-weise}{%
\subparagraph{aa) Saul unternimmt auf göttlichen Befehl den Rachekrieg
gegen die Amalekiter, vollzieht aber den Bann in ungenügender
Weise}\label{aa-saul-unternimmt-auf-guxf6ttlichen-befehl-den-rachekrieg-gegen-die-amalekiter-vollzieht-aber-den-bann-in-ungenuxfcgender-weise}}

\hypertarget{section-14}{%
\section{15}\label{section-14}}

1Samuel aber sagte zu Saul: »Mich hat der HERR (einst) gesandt, dich zum
König über sein Volk Israel zu salben; so gehorche nun dem bestimmten
Befehl des HERRN! 2So hat Gott, der HERR der Heerscharen, gesprochen:
›Ich will das Unrecht ahnden, das die Amalekiter einst den Israeliten
zugefügt haben, indem sie ihnen beim Auszug aus Ägypten den Weg
verlegten. 3Darum ziehe jetzt hin, schlage die Amalekiter und
vollstrecke den Bann an ihnen und an allem, was sie besitzen! Übe keine
Schonung an ihnen, sondern laß alles sterben, Männer wie Weiber, Kinder
wie Säuglinge, Rinder wie Kleinvieh, Kamele wie Esel!‹« 4Da bot Saul das
Volk auf und musterte es in Telaim: 200000~Mann Fußvolk und 10000~Mann
aus Juda. 5Nachdem Saul dann vor die Hauptstadt der Amalekiter gerückt
war, legte er einen Hinterhalt im Bachtal; 6den Kenitern\textless sup
title=``vgl. Ri 1,16''\textgreater✲ aber ließ er sagen: »Auf! Zieht euch
zurück und entfernt euch aus der Mitte\textless sup title=``oder: dem
Gebiet''\textgreater✲ der Amalekiter, damit ich euch nicht zugleich mit
ihnen vernichte! Denn ihr habt allen Israeliten bei ihrem Auszuge aus
Ägypten Freundschaft erwiesen.« Da zogen die Keniter aus dem Gebiet der
Amalekiter weg. 7Hierauf besiegte Saul die Amalekiter (und richtete ein
Blutbad unter ihnen an) von Hawila bis nach Sur, das östlich von Ägypten
liegt. 8Agag, den König der Amalekiter, nahm er lebendig gefangen und
vollstreckte den Bann an dem ganzen Volk mit der Schärfe des Schwertes;
9doch verschonte Saul und seine Leute den Agag und die besten Stücke des
Kleinviehs und der Rinder, die feisten Tiere und die Lämmer und
überhaupt alles Wertvolle, und sie wollten den Bann an ihnen nicht
vollstrecken; nur was vom Vieh gering (und wertlos) war, an dem
vollstreckten sie den Bann.

\hypertarget{bb-saul-wegen-seines-ungehorsams-von-gott-verworfen-samuels-strafrede-und-sauls-schuldbekenntnis}{%
\subparagraph{bb) Saul wegen seines Ungehorsams von Gott verworfen;
Samuels Strafrede und Sauls
Schuldbekenntnis}\label{bb-saul-wegen-seines-ungehorsams-von-gott-verworfen-samuels-strafrede-und-sauls-schuldbekenntnis}}

10Da erging das Wort des HERRN an Samuel also: 11»Es reut mich, daß ich
Saul zum König gemacht habe; denn er hat sich vom Gehorsam gegen mich
abgewandt und meine Befehle nicht ausgerichtet.« Darüber geriet Samuel
in schmerzliche Aufregung, so daß er die ganze Nacht hindurch zum HERRN
schrie\textless sup title=``=~laut flehte''\textgreater✲.

12Am folgenden Morgen früh aber machte Samuel sich auf, um Saul
entgegenzugehen; da wurde ihm die Botschaft gebracht, Saul habe sich
nach Karmel begeben und sich dort ein Denkmal errichtet, sei dann aber
umgekehrt und weiter nach Gilgal hinabgezogen. 13Als nun Samuel zu Saul
kam, sagte Saul zu ihm: »Gesegnet seist du vom HERRN! Ich habe den
Befehl des HERRN ausgeführt.« 14Da antwortete Samuel: »Was ist denn das
für ein Blöken von Kleinvieh, das an meine Ohren dringt, und ein Gebrüll
von Rindern, das ich höre?« 15Saul erwiderte: »Von den Amalekitern haben
unsere Leute sie mitgebracht, weil sie die besten Stücke von dem
Kleinvieh und den Rindern verschont haben, um sie dem HERRN, deinem
Gott, zu opfern; aber an dem Übrigen haben wir den Bann vollstreckt.«
16Da entgegnete Samuel dem Saul: »Halt ein! Ich will dir mitteilen, was
der HERR in dieser Nacht zu mir gesagt hat.« Jener erwiderte ihm:
»Rede!«

17Da sagte Samuel: »Nicht wahr? Obgleich du dir selbst klein vorkamst,
bist du doch das Haupt der Stämme Israels geworden; denn der HERR hat
dich zum König über Israel gesalbt. 18Nun hat der HERR dich zu einem
Kriegszuge ausgesandt und dir geboten: ›Ziehe hin, vollstrecke den Bann
an den Frevlern, den Amalekitern, und bekriege sie, bis du sie
vernichtet hast!‹ 19Warum bist du nun der Weisung des HERRN nicht
nachgekommen, sondern hast dich über die Beute hergemacht und das getan,
was dem HERRN mißfällt?« 20Saul antwortete dem Samuel: »Ich bin ja doch
der Weisung des HERRN nachgekommen und habe den Kriegszug, zu dem der
HERR mich ausgesandt hatte, ausgeführt und habe Agag, den König der
Amalekiter, mitgebracht und an den Amalekitern den Bann vollstreckt.
21Aber das Kriegsvolk hat von der Beute Kleinvieh und Rinder genommen,
das Beste von dem Banngut, um es dem HERRN, deinem Gott, in Gilgal zu
opfern.« 22Da antwortete Samuel: »Hat der HERR etwa an Brandopfern und
Schlachtopfern das gleiche Wohlgefallen wie am Gehorsam gegen seine
Befehle? Wisse wohl: Gehorsam ist besser als Schlachtopfer, Folgsamkeit
besser als das Fett von Widdern; 23denn Ungehorsam ist ebenso schlimm
wie die Sünde der Zauberei, und Eigenwille ist wie Abgötterei und
Götzendienst. Weil du den Befehl des HERRN verworfen hast, hat er dich
auch verworfen, daß du nicht mehr König sein sollst!«

24Da sagte Saul zu Samuel: »Ich habe gesündigt, weil ich den Befehl des
HERRN und deine Weisungen übertreten habe; denn ich habe mich vor dem
Kriegsvolk gefürchtet und deshalb seiner Forderung nachgegeben. 25Doch
nun vergib mir meine Versündigung und kehre mit mir um, damit ich den
HERRN anbete!« 26Aber Samuel erwiderte ihm: »Ich kehre nicht mit dir um;
weil du den Befehl des HERRN verworfen hast, so hat der HERR dich auch
verworfen, daß du nicht länger König über Israel sein sollst.« 27Als nun
Samuel sich umwandte, um wegzugehen, erfaßte Saul den Zipfel seines
Mantels, aber dieser riß ab. 28Da sagte Samuel zu ihm: »Der HERR hat
heute das Königtum über Israel von dir gerissen und gibt es einem
andern, der besser ist als du. 29Und niemals lügt der ruhmwürdige Gott
Israels und empfindet keine Reue; denn er ist kein Mensch, daß ihn etwas
gereuen müßte.« 30Saul antwortete: »Ich habe gesündigt; aber erweise mir
doch jetzt vor den Ältesten meines Volkes und vor Israel die Ehre, mit
mir umzukehren, damit ich den HERRN, deinen Gott, anbete!« 31Da kehrte
Samuel um, und zwar hinter Saul her, und Saul verrichtete seine Anbetung
vor dem HERRN.

\hypertarget{cc-samuel-vollzieht-den-bann-am-kuxf6nig-agag-und-trennt-sich-von-saul-auf-nimmerwiedersehen}{%
\subparagraph{cc) Samuel vollzieht den Bann am König Agag und trennt
sich von Saul auf
Nimmerwiedersehen}\label{cc-samuel-vollzieht-den-bann-am-kuxf6nig-agag-und-trennt-sich-von-saul-auf-nimmerwiedersehen}}

32Hierauf befahl Samuel: »Bringt Agag, den König der Amalekiter, zu mir
her!« Da trat Agag wohlgemut\textless sup title=``oder:
mutig''\textgreater✲ vor ihn und sagte: »Fürwahr, der Tod hat seine
Bitterkeit (für mich) verloren!« 33Samuel aber sagte: »Wie dein Schwert
Frauen ihrer Kinder beraubt hat, so soll auch deine Mutter ihrer Kinder
unter den Frauen beraubt sein!« Hierauf hieb Samuel den Agag in Stücke
vor dem HERRN in Gilgal.

34Samuel begab sich dann nach Rama, während Saul nach dem Gibea Sauls
heimkehrte. 35Samuel besuchte dann Saul bis zu seinem Todestage nicht
wieder; denn Samuel trauerte um Saul, weil der HERR es bereut hatte,
Saul zum König über Israel gemacht zu haben.

\hypertarget{ii.-saul-und-david-kap.-16-31}{%
\subsection{II. Saul und David (Kap.
16-31)}\label{ii.-saul-und-david-kap.-16-31}}

\hypertarget{davids-berufung-und-salbung-durch-samuel}{%
\subsubsection{1. Davids Berufung und Salbung durch
Samuel}\label{davids-berufung-und-salbung-durch-samuel}}

\hypertarget{a-samuel-begibt-sich-auf-guxf6ttlichen-befehl-zum-opfermahl-nach-bethlehem}{%
\paragraph{a) Samuel begibt sich auf göttlichen Befehl zum Opfermahl
nach
Bethlehem}\label{a-samuel-begibt-sich-auf-guxf6ttlichen-befehl-zum-opfermahl-nach-bethlehem}}

\hypertarget{section-15}{%
\section{16}\label{section-15}}

1Da sagte der HERR zu Samuel: »Wie lange willst du noch um Saul trauern,
da ich ihn doch für unwürdig erachtet habe, noch länger König über
Israel zu sein? Fülle dein Horn mit Öl und mache dich auf den Weg: ich
will dich nach Bethlehem zu Isai senden; denn unter seinen Söhnen habe
ich mir einen König ersehen.« 2Samuel antwortete: »Wie kann ich
hingehen? Wenn Saul davon hört, bringt er mich um!« Da antwortete der
HERR: »Nimm eine junge Kuh mit dir und sage: ›Ich bin hergekommen, um
dem HERRN ein Opfer zu bringen‹. 3Wenn du dann Isai zum Opfermahl
eingeladen hast, werde ich selbst dir kundtun, was du zu tun hast; denn
du sollst mir den salben, den ich dir bezeichnen werde.«

4Samuel tat hierauf, was der HERR ihm geboten hatte, und begab sich nach
Bethlehem. Da gingen ihm die Ältesten des Ortes ängstlich entgegen und
fragten ihn: »Bedeutet dein Kommen etwas Gutes?« 5Er antwortete: »Ja,
Gutes! Ich bin gekommen, um dem HERRN ein Opfer zu bringen; heiligt euch
also und kommt mit mir zum Opfermahl.« Hierauf ließ er auch Isai und
seine Söhne sich heiligen und lud sie zum Opfermahl ein.

\hypertarget{b-samuel-salbt-isais-juxfcngsten-sohn-david-zum-kuxf6nig}{%
\paragraph{b) Samuel salbt Isais jüngsten Sohn David zum
König}\label{b-samuel-salbt-isais-juxfcngsten-sohn-david-zum-kuxf6nig}}

6Als sie sich nun einfanden und er Eliab sah, dachte er: »Sicherlich
(steht hier) vor dem HERRN der, den er zu seinem Gesalbten machen will.«
7Aber der HERR sagte zu Samuel: »Sieh nicht auf seine äußere Gestalt und
seinen hohen Wuchs! Denn diesen habe ich nicht erkoren. Gott sieht ja
nicht das an, worauf Menschen sehen; denn die Menschen sehen nach den
Augen\textless sup title=``oder: in die Augen =~auf das
Äußere''\textgreater✲, der HERR aber sieht nach dem Herzen\textless sup
title=``oder: ins Herz''\textgreater✲.« 8Da rief Isai den Abinadab und
ließ ihn vor Samuel treten\textless sup title=``oder:
vorübergehen''\textgreater✲; doch der erklärte: »Auch diesen hat der
HERR nicht erwählt.« 9Da ließ Isai den Samma vortreten, aber Samuel
erklärte: »Auch diesen hat der HERR nicht erwählt.« 10So führte Isai dem
Samuel sieben seiner Söhne vor, aber Samuel erklärte dem Isai: »Von
diesen hat der HERR keinen erwählt.« 11Hierauf fragte Samuel den Isai:
»Sind das die jungen Leute alle?« Jener erwiderte: »Es ist noch der
jüngste übrig; der hütet eben das Kleinvieh.« Da sagte Samuel zu Isai:
»Sende hin und laß ihn holen; denn wir werden uns nicht eher (zum Mahl)
setzen, als bis er hergekommen ist.« 12Da sandte er hin und ließ ihn
holen. (David) war aber bräunlich, hatte schöne Augen und eine kräftige
Gestalt. Da sagte der HERR: »Auf! Salbe ihn, denn dieser ist es!« 13Da
nahm Samuel das Ölhorn und salbte ihn inmitten seiner Brüder; da kam der
Geist des HERRN über David von diesem Tage an und (blieb) auch späterhin
(auf ihm). Samuel aber machte sich auf und kehrte nach Rama zurück.

\hypertarget{david-wird-zum-harfenspielen-an-sauls-hof-berufen-und-tritt-in-kuxf6niglichen-dienst}{%
\subsubsection{2. David wird zum Harfenspielen an Sauls Hof berufen und
tritt in königlichen
Dienst}\label{david-wird-zum-harfenspielen-an-sauls-hof-berufen-und-tritt-in-kuxf6niglichen-dienst}}

14Als nun der Geist des HERRN von Saul gewichen war und ihn ein vom
HERRN gesandter böser Geist ängstigte, 15sagten Sauls Diener zu ihm: »Du
weißt, daß ein böser Geist Gottes dich ängstigt. 16Unser Herr braucht
nur zu gebieten: deine Knechte stehen dir zu Diensten und werden einen
Mann suchen, der sich auf das Saitenspiel versteht; wenn dann der böse
Geist von Gott über dich kommt, soll er die Saiten (vor dir) rühren,
dann wird dir wohler werden.« 17Da befahl Saul seinen Dienern: »So seht
euch für mich nach einem Manne um, der gut zu spielen versteht, und
bringt ihn zu mir!« 18Da nahm einer von den Dienern das Wort und sagte:
»Ja, ich habe einen Sohn Isais in Bethlehem kennengelernt, der sich auf
das Saitenspiel versteht; er ist auch ein tapferer Mann und kriegsgeübt,
dazu redegewandt und von schönem Äußeren, und der HERR ist mit ihm.«
19Da schickte Saul Boten zu Isai und ließ ihm sagen: »Schicke mir doch
deinen Sohn David, der bei dem Kleinvieh ist.« 20Da nahm Isai einen mit
Brot beladenen Esel, einen Schlauch Wein und ein Ziegenböckchen und
sandte es durch seinen Sohn David an Saul. 21So kam David zu Saul und
trat in seine Dienste; und (Saul) gewann ihn sehr lieb, so daß er sein
Waffenträger wurde 22und Saul dem Isai sagen ließ: »Laß doch David in
meinem Dienste bleiben, denn er gefällt mir wohl.« 23Sooft nun der böse
Geist Gottes über Saul kam, nahm David die Zither und spielte; dann fand
Saul Erleichterung, so daß er sich wohler fühlte und der böse Geist von
ihm wich.

\hypertarget{david-und-der-feindliche-vorkuxe4mpfer-goliath}{%
\subsubsection{3. David und der feindliche Vorkämpfer
Goliath}\label{david-und-der-feindliche-vorkuxe4mpfer-goliath}}

\hypertarget{a-ausbruch-des-philisterkriegs-schilderung-der-uxe4uuxdferen-erscheinung-und-des-hochmuxfctigen-auftretens-goliaths}{%
\paragraph{a) Ausbruch des Philisterkriegs; Schilderung der äußeren
Erscheinung und des hochmütigen Auftretens
Goliaths}\label{a-ausbruch-des-philisterkriegs-schilderung-der-uxe4uuxdferen-erscheinung-und-des-hochmuxfctigen-auftretens-goliaths}}

\hypertarget{section-16}{%
\section{17}\label{section-16}}

1Da boten die Philister ihre Heere zum Kriege auf, sammelten sich bei
Socho, das zu Juda gehört, und schlugen ein Lager zwischen Socho und
Aseka bei Ephes-Dammim auf. 2Saul aber und die Israeliten sammelten sich
und bezogen ein Lager im Terebinthental und rüsteten sich zum Kampf
gegen die Philister; 3die Philister standen am Berge jenseits, die
Israeliten am Berge diesseits, so daß das Tal zwischen ihnen lag.

4Da trat aus den Reihen der Philister der Vorkämpfer namens Goliath
hervor, ein Gathiter, der sechs Ellen und eine Spanne hoch war; 5er trug
einen ehernen Helm auf dem Kopfe und hatte einen Schuppenpanzer an,
dessen Gewicht fünftausend Schekel Erz betrug. 6An den Beinen trug er
eherne Schienen und zwischen den Schultern\textless sup title=``=~auf
dem Rücken''\textgreater✲ einen ehernen Wurfspieß; 7der Schaft seines
Speeres war wie ein Weberbaum, und die Spitze seines Speeres bestand aus
sechshundert Schekel Eisen; sein Schildträger ging vor ihm her. 8Er
stellte sich hin und rief den in Reihen stehenden Israeliten die Worte
zu: »Warum zieht ihr aus, euch in Schlachtordnung aufzustellen? Bin ich
nicht da, der Philister, und ihr, die Knechte Sauls? Wählt euch einen
Mann aus, der komme zu mir herab! 9Vermag er mich im Kampfe zu bestehen
und erschlägt er mich, so wollen wir euch untertan sein; bin aber ich
ihm überlegen und erschlage ich ihn, so sollt ihr uns untertan sein und
müßt uns dienen!« 10Dann fügte der Philister noch hinzu: »Heute habe ich
den in Reihen stehenden Israeliten Hohn geboten: stellt mir einen Mann,
daß wir miteinander kämpfen!« 11Als Saul und alle Israeliten diese Worte
des Philisters hörten, erschraken sie und fürchteten sich sehr.

\hypertarget{b-david-von-seinem-vater-ins-lager-zu-seinen-bruxfcdern-gesandt-entruxfcstet-sich-uxfcber-den-hochmut-goliaths-und-fuxfchlt-sich-zum-kampf-mit-ihm-berufen}{%
\paragraph{b) David, von seinem Vater ins Lager zu seinen Brüdern
gesandt, entrüstet sich über den Hochmut Goliaths und fühlt sich zum
Kampf mit ihm
berufen}\label{b-david-von-seinem-vater-ins-lager-zu-seinen-bruxfcdern-gesandt-entruxfcstet-sich-uxfcber-den-hochmut-goliaths-und-fuxfchlt-sich-zum-kampf-mit-ihm-berufen}}

12David aber war der Sohn jenes Ephrathiten aus Bethlehem in Juda, der
Isai hieß und acht Söhne hatte und zur Zeit Sauls schon ein älterer, in
den Jahren vorgerückter Mann war. 13Die drei ältesten Söhne Isais waren
unter Saul in den Krieg gezogen; von diesen seinen drei Söhnen, die ins
Feld gezogen waren, hieß der älteste Eliab, der zweite Abinadab, der
dritte Samma; 14David aber war der jüngste. Da die drei ältesten unter
Saul in den Krieg gezogen waren, 15ging David ab und zu von Sauls Hofe
heim, um in Bethlehem das Kleinvieh seines Vaters zu hüten. 16Der
Philister aber trat morgens und abends auf und stellte sich vierzig Tage
lang (vor die Israeliten) hin.~-- 17Da sagte Isai (eines Tages) zu
seinem Sohne David: »Nimm doch für deine Brüder ein Epha von diesem
gerösteten Getreide und diese zehn Brote und bringe sie schnell zu
deinen Brüdern ins Lager; 18diese zehn frischen Käse aber nimm für den
Hauptmann der Tausendschaft mit und erkundige dich nach dem Befinden
deiner Brüder und laß dir ein Pfand von ihnen mitgeben!« 19Saul und sie
und alle Israeliten befanden sich nämlich im Terebinthental im Kriege
mit den Philistern.

20Da machte sich David am andern Morgen früh reisefertig, überließ das
Kleinvieh einem Hüter, packte die Lebensmittel ein und begab sich auf
den Weg, wie Isai ihm befohlen hatte. Er kam zur Wagenburg, als das Heer
gerade in Schlachtordnung ausrückte und man das Kriegsgeschrei erhob;
21Israel und die Philister stellten sich zum Kampf auf, Schlachtreihe
gegen Schlachtreihe. 22Da übergab David das Gepäck, das er mitgebracht
hatte, dem Gepäckhüter, lief dann in die Schlachtreihe und erkundigte
sich, als er hinkam, bei seinen Brüdern nach ihrem Ergehen. 23Während er
sich noch mit ihnen besprach, trat der Vorkämpfer -- er hieß Goliath und
war ein Philister aus Gath -- aus den Reihen der Philister hervor und
führte dieselben Reden wie früher, so daß David es hörte; 24alle
Israeliten aber, die den Mann erblickten, flohen vor ihm und fürchteten
sich sehr. 25Da sagte einer von den Israeliten: »Habt ihr diesen Mann
gesehen, der da heraufkommt? Ja, um Israel zu verhöhnen, tritt er auf!
Und wer ihn erschlägt, den will der König mit großem Reichtum belohnen
und will ihm seine Tochter geben und seines Vaters Haus steuerfrei in
Israel machen!« 26Da fragte David die Männer, die bei ihm standen: »Wie
soll der Mann belohnt werden, der diesen Philister da erschlägt und
Israel von der Schande befreit? Wer ist denn dieser Philister, dieser
Heide, daß er die Schlachtreihen des lebendigen Gottes beschimpfen
darf?« 27Da wiederholten ihm die Leute die frühere Mitteilung: »So und
so wird man den Mann belohnen, der ihn erschlägt!« 28Als nun sein
ältester Bruder Eliab hörte, wie er sich mit den Männern unterhielt,
geriet er in Zorn über David und rief aus: »Wozu bist du eigentlich
hergekommen, und wem hast du die paar Schafe dort in der Steppe
überlassen? Ich kenne deinen vorwitzigen und boshaften Sinn wohl: du
bist nur hergekommen, um dir den Krieg anzusehen!« 29David entgegnete:
»Nun, was habe ich denn jetzt getan? Es war ja nur eine Frage!« 30Damit
wandte er sich von ihm ab, einem andern zu, und wiederholte seine vorige
Frage, und die Leute gaben ihm dieselbe Auskunft wie zuvor.

\hypertarget{c-david-erbietet-sich-zum-zweikampf-weist-aber-die-waffenruxfcstung-sauls-zuruxfcck-und-nimmt-nur-seine-schleuder-als-waffe}{%
\paragraph{c) David erbietet sich zum Zweikampf, weist aber die
Waffenrüstung Sauls zurück und nimmt nur seine Schleuder als
Waffe}\label{c-david-erbietet-sich-zum-zweikampf-weist-aber-die-waffenruxfcstung-sauls-zuruxfcck-und-nimmt-nur-seine-schleuder-als-waffe}}

31Als man nun hörte, wie David sich ausgesprochen hatte, hinterbrachte
man es dem Saul, und dieser ließ ihn zu sich kommen. 32Da sagte David zu
Saul: »Kein Mensch braucht um den da den Mut zu verlieren! Dein Knecht
will hingehen und mit diesem Philister kämpfen.« 33Saul aber antwortete
ihm: »Du kannst diesem Philister nicht entgegentreten, um mit ihm zu
kämpfen; denn du bist noch ein Jüngling, er aber ist ein Kriegsmann von
Jugend auf!« 34Da entgegnete David dem Saul: »Dein Knecht hat seinem
Vater das Kleinvieh gehütet; wenn da ein Löwe oder ein Bär kam und ein
Stück aus der Herde wegtrug, 35so lief ich ihm nach und erschlug ihn und
riß es ihm aus dem Rachen; leistete er mir aber Widerstand, so packte
ich ihn am Bart\textless sup title=``oder: bei der Mähne''\textgreater✲
und schlug ihn tot. 36Löwen so gut wie Bären hat dein Knecht erschlagen,
und diesem Philister, diesem Heiden, soll es ebenso ergehen wie jenen
allen; denn er hat die Schlachtreihen des lebendigen Gottes verhöhnt!«
37Dann fuhr David fort: »Der HERR, der mich aus den Krallen der Löwen
und aus den Klauen der Bären errettet hat, der wird mich auch aus der
Hand dieses Philisters erretten.« Da sagte Saul zu David: »So gehe hin!
Der HERR wird mit dir sein!« 38Hierauf legte Saul dem David seine
Rüstung an: er setzte ihm einen ehernen Helm aufs Haupt und zog ihm
einen Panzer an; 39weiter mußte David Sauls Schwert über seinen
Waffenrock gürten und bemühte sich dann zu gehen; denn er hatte es noch
nie versucht. Aber er sagte zu Saul: »Ich kann darin nicht gehen, denn
ich bin nicht daran gewöhnt.« So legte David denn alles wieder ab,
40nahm nur seinen Stecken in die Hand, suchte sich aus dem Bach fünf
glatte Kieselsteine aus und tat sie in die Hirtentasche, die ihm als
Schleudertasche diente; dann nahm er seine Schleuder in die Hand und
ging auf den Philister los.

\hypertarget{d-davids-siegreicher-kampf-mit-goliath}{%
\paragraph{d) Davids siegreicher Kampf mit
Goliath}\label{d-davids-siegreicher-kampf-mit-goliath}}

41Der Philister aber kam immer näher an David heran, während sein
Schildträger vor ihm herschritt. 42Als nun der Philister hinblickte und
David sah, verachtete er ihn, weil er noch so jung war, ein bräunlicher
Jüngling von schmuckem Aussehen. 43Daher rief der Philister dem David
zu: »Bin ich etwa ein Hund, daß du mit Stöcken zu mir kommst?« Hierauf
fluchte der Philister dem David bei seinem Gott 44und rief dem David zu:
»Komm nur her zu mir, damit ich dein Fleisch den Vögeln des Himmels und
den Tieren des Feldes gebe!« 45David aber erwiderte ihm: »Du trittst mir
mit Schwert und Lanze und Wurfspieß entgegen, ich aber trete dir
entgegen mit dem Namen des HERRN der Heerscharen, des Gottes der
Schlachtreihen Israels, die du verhöhnt hast. 46Am heutigen Tage wird
dich der HERR in meine Hand fallen lassen, daß ich dich erschlage und
dir den Kopf abhaue; und (deinen Leichnam und) die Leichen des
Philisterheeres werde ich noch heute den Vögeln des Himmels und den
wilden Tieren des Landes übergeben, damit alle Welt erkennt, daß Israel
einen Gott hat! 47und alle, die hier versammelt sind, sollen erkennen,
daß der HERR nicht Schwert und Spieß braucht, um den Sieg zu schaffen;
denn der HERR hat die Entscheidung im Kampf, und er wird euch in unsere
Hand geben!«

48Als sich nun der Philister in Bewegung setzte und auf David losging,
lief dieser eilends aus der Schlachtreihe, dem Philister entgegen;
49dabei griff er mit der Hand in die Tasche, nahm einen Stein heraus,
schleuderte ihn und traf den Philister an die Stirn, so daß ihm der
Stein in die Stirn eindrang und er vornüber zu Boden fiel. 50So
überwältigte David den Philister mit der Schleuder und dem Stein,
besiegte den Philister und tötete ihn, ohne ein Schwert in der Hand zu
haben. 51Er lief nämlich hin, trat an den Philister heran, nahm dessen
Schwert, zog es aus der Scheide und tötete ihn vollends, indem er ihm
den Kopf damit abhieb. Als nun die Philister sahen, daß ihr stärkster
Mann tot war, ergriffen sie die Flucht. 52Da machten sich die Männer von
Israel und Juda auf, erhoben das Kriegsgeschrei und verfolgten die
Philister bis nach Gath und bis an die Tore von Ekron, so daß die
Leichen der erschlagenen Philister auf dem Wege von Saaraim bis nach
Gath und Ekron lagen. 53Als dann die Israeliten von der Verfolgung der
Philister zurückkehrten, plünderten sie deren Lager. 54David aber nahm
den Kopf des Philisters und brachte ihn nach Jerusalem; seine Rüstung
dagegen legte er in seinem\textless sup title=``d.h. des
HERRN''\textgreater✲ Zelte nieder.

\hypertarget{e-saul-erkundigt-sich-nach-david}{%
\paragraph{e) Saul erkundigt sich nach
David}\label{e-saul-erkundigt-sich-nach-david}}

55Als aber Saul sah, wie David dem Philister entgegenging, fragte er
seinen Heerführer Abner: »Wessen Sohn ist denn der Jüngling, Abner?«
Dieser antwortete: »Bei deinem Leben, o König: ich weiß es nicht!« 56Da
gab ihm der König den Auftrag: »Erkundige dich doch, wessen Sohn der
junge Mann ist!« 57Sobald nun David von dem Sieg über den Philister
zurückkehrte, nahm ihn Abner mit sich und führte ihn vor Saul, während
er den Kopf des Philisters noch in der Hand hatte. 58Da fragte ihn Saul:
»Wessen Sohn bist du, junger Mann?«, und David antwortete: »Der Sohn
deines Knechtes Isai aus Bethlehem.«

\hypertarget{davids-freundschaft-mit-jonathan-sauls-eifersucht-auf-david}{%
\subsubsection{4. Davids Freundschaft mit Jonathan; Sauls Eifersucht auf
David}\label{davids-freundschaft-mit-jonathan-sauls-eifersucht-auf-david}}

\hypertarget{a-david-kommt-an-sauls-hof-sein-verhuxe4ltnis-zu-saul-und-jonathan}{%
\paragraph{a) David kommt an Sauls Hof; sein Verhältnis zu Saul und
Jonathan}\label{a-david-kommt-an-sauls-hof-sein-verhuxe4ltnis-zu-saul-und-jonathan}}

\hypertarget{section-17}{%
\section{18}\label{section-17}}

1Als nun David seine Unterredung mit Saul beendet hatte, da schloß
Jonathan den David in sein Herz und gewann ihn lieb wie sein eigenes
Leben. 2Saul aber nahm David an jenem Tage zu sich und ließ ihn nicht
wieder in das Haus seines Vaters zurückkehren. 3Da schloß Jonathan einen
Freundschaftsbund mit David, weil er ihn wie sich selbst liebte. 4Dabei
zog Jonathan den Mantel aus, den er anhatte, und gab ihn David, dazu
auch seinen Waffenrock samt seinem Schwerte, seinem Bogen und seinem
Gürtel. 5Sooft nun David Kriegszüge unternahm, hatte er überall Glück,
wohin Saul ihn sandte; daher übertrug Saul ihm die Stelle eines
Anführers in seinem Heere; und er war beim ganzen Volk und auch bei den
Hofleuten Sauls beliebt.

\hypertarget{b-festliche-heimkehr-der-krieger-david-vom-volk-als-sieger-gefeiert}{%
\paragraph{b) Festliche Heimkehr der Krieger; David vom Volk als Sieger
gefeiert}\label{b-festliche-heimkehr-der-krieger-david-vom-volk-als-sieger-gefeiert}}

6Es begab sich aber bei der Heimkehr Sauls und des Heeres, als David
nach der Erschlagung des Philisters\textless sup title=``oder: aus der
Philisterschlacht''\textgreater✲ zurückkehrte: da zogen die Frauen aus
allen Ortschaften Israels singend und tanzend, mit Handpauken,
Jubelgeschrei und Zimbeln dem König Saul entgegen; 7und die Frauen hoben
im Wechselgesang an: »Saul hat seine Tausende geschlagen, David aber
seine Zehntausende!« 8Da geriet Saul in heftigen Zorn, weil dieses Lied
ihm durchaus mißfiel, und er sagte: »Dem David weisen sie zehntausend
zu, mir aber nur tausend; nun fehlt ihm nur noch das Königtum!« 9So sah
denn Saul den David seit jenem Tage und weiterhin mit Neid an.

\hypertarget{c-david-von-saul-tuxf6dlich-gehauxdft-bewuxe4hrt-sich-als-kriegsheld}{%
\paragraph{c) David, von Saul tödlich gehaßt, bewährt sich als
Kriegsheld}\label{c-david-von-saul-tuxf6dlich-gehauxdft-bewuxe4hrt-sich-als-kriegsheld}}

10Am folgenden Tage nun kam ein böser Geist Gottes über Saul, so daß er
im Hause✲ drinnen raste; David aber spielte die Zither, wie er dies alle
Tage zu tun pflegte, während Saul den Speer in der Hand hatte. 11Da
zückte Saul den Speer, indem er dachte: »Ich will David an die Wand
spießen!«, aber David wich ihm zweimal aus. 12Da fürchtete sich Saul vor
David, weil der HERR mit ihm war, während er von Saul gewichen war.
13Darum entfernte ihn Saul aus seiner Nähe und machte ihn zum Hauptmann
über tausend Mann. Er unternahm nun Kriegszüge an der Spitze seiner
Leute 14und hatte bei allen seinen Unternehmungen Glück, weil der HERR
mit ihm war. 15Als nun Saul sah, daß er außerordentliches Glück hatte,
geriet er in Angst vor ihm; 16aber bei ganz Israel und Juda war David
beliebt, weil er bei seinen Kriegszügen an ihrer Spitze aus- und einzog.

\hypertarget{d-david-um-die-verheiratung-mit-der-uxe4ltesten-tochter-sauls-getuxe4uscht-erhuxe4lt-deren-juxfcngere-schwester-michal-zur-frau}{%
\paragraph{d) David, um die Verheiratung mit der ältesten Tochter Sauls
getäuscht, erhält deren jüngere Schwester Michal zur
Frau}\label{d-david-um-die-verheiratung-mit-der-uxe4ltesten-tochter-sauls-getuxe4uscht-erhuxe4lt-deren-juxfcngere-schwester-michal-zur-frau}}

17Da sagte Saul zu David: »Hier ist meine älteste Tochter Merab, die
will ich dir zur Frau geben; nur mußt du dich mir als Held erweisen und
die Kriege des HERRN führen.« Saul dachte nämlich: »Ich selbst will
nicht Hand an ihn legen, sondern die Philister sollen ihn ums Leben
bringen.« 18Da antwortete David dem Saul: »Wer bin ich, und was ist
meine Familie, das Geschlecht meines Vaters, in Israel, daß ich des
Königs Schwiegersohn werden sollte!« 19Als dann aber die Zeit kam, wo
Merab, die Tochter Sauls, dem David gegeben werden sollte, wurde sie mit
Adriel von Mehola verheiratet.

\hypertarget{davids-kriegsdienst-um-die-braut}{%
\paragraph{Davids Kriegsdienst um die
Braut}\label{davids-kriegsdienst-um-die-braut}}

20Aber Sauls Tochter Michal faßte Liebe zu David. Als Saul Kenntnis
davon erhielt, fand die Sache seinen Beifall; 21er dachte nämlich: »Ich
will sie ihm zur Frau geben, damit sie für ihn zur Schlinge\textless sup
title=``d.h. die Veranlassung zum Untergang''\textgreater✲ wird und er
den Philistern in die Hände fällt.« So sagte denn Saul zu David: »Mit
der zweiten sollst du jetzt mein Schwiegersohn werden.« 22Darauf gab er
seinen Dienern\textless sup title=``vgl. 16,15''\textgreater✲ die
Weisung: »Redet vertraulich mit David und sagt ihm: ›Der König hat
offenbar Wohlgefallen an dir, und alle seine Diener haben dich gern; so
werde also nun der Schwiegersohn des Königs!‹« 23Als nun die Diener
Sauls in dieser Weise dem David zuredeten, entgegnete David: »Dünkt es
euch etwas Leichtes, des Königs Schwiegersohn zu werden? Ich bin ja doch
nur ein armer und geringer Mann.« 24Als nun die Diener Sauls diesem
berichteten: »So und so hat David sich ausgesprochen«, 25antwortete
Saul: »Teilt dem David mit, der König begehre keine andere Heiratsgabe✲
als hundert Vorhäute von Philistern, um Rache an den Feinden des Königs
zu nehmen.« Saul gedachte nämlich, David durch die Hand der Philister
aus der Welt zu schaffen. 26Als nun Sauls Diener dem David diese
Äußerung hinterbrachten, war David damit einverstanden, des Königs
Schwiegersohn zu werden; und ehe noch die Zeit um war, 27machte David
sich mit seinen Leuten auf den Weg und erschlug unter den Philistern
zweihundert Mann. Er brachte dann ihre Vorhäute heim und lieferte sie
dem Könige vollzählig ab, um des Königs Schwiegersohn zu werden. Da gab
ihm Saul seine Tochter Michal zur Frau. 28Als aber Saul immer klarer
erkannte, daß der HERR mit David war und daß Michal, die Tochter Sauls,
ihn liebte, 29fürchtete Saul sich noch mehr vor David und wurde ihm für
immer feind. 30Sooft aber die Fürsten der Philister ins Feld zogen,
hatte David allemal größeren Erfolg als alle anderen Heerführer Sauls,
so daß sein Name in hohen Ehren stand.

\hypertarget{sauls-aussuxf6hnung-mit-david-infolge-der-fuxfcrsprache-jonathans-nach-wiederholten-mordanschluxe4gen-sauls-flieht-david-zu-samuel}{%
\subsubsection{5. Sauls Aussöhnung mit David infolge der Fürsprache
Jonathans; nach wiederholten Mordanschlägen Sauls flieht David zu
Samuel}\label{sauls-aussuxf6hnung-mit-david-infolge-der-fuxfcrsprache-jonathans-nach-wiederholten-mordanschluxe4gen-sauls-flieht-david-zu-samuel}}

\hypertarget{a-sauls-eid-zu-davids-gunsten}{%
\paragraph{a) Sauls Eid zu Davids
Gunsten}\label{a-sauls-eid-zu-davids-gunsten}}

\hypertarget{section-18}{%
\section{19}\label{section-18}}

1Als nun Saul zu seinem Sohne Jonathan und seiner ganzen Umgebung davon
redete, daß er David töten wolle, 2hinterbrachte Jonathan, der Sohn
Sauls, es dem David, den er sehr liebhatte, mit den Worten: »Mein Vater
Saul trachtet dir nach dem Leben; nimm dich daher morgen früh in acht
und halte dich sorgfältig verborgen! 3Ich selbst aber will hinausgehen
und auf dem Felde✲, wo du dich befinden mußt, neben meinen Vater treten;
ich will dann mit meinem Vater von dir reden und sehen, wie die Sache
steht, und es dir dann mitteilen.« 4Hierauf trat Jonathan bei der
Unterredung mit seinem Vater Saul für David ein, indem er zu ihm sagte:
»Der König versündige sich nicht an seinem Diener David; denn er hat
sich dir gegenüber nichts zuschulden kommen lassen, sondern sich große
Verdienste um dich erworben; 5er hat sogar sein Leben aufs Spiel gesetzt
und den Philister erschlagen, und so hat der HERR dem ganzen Volk Israel
einen herrlichen Sieg verschafft. Du hast es mit angesehen und dich
darüber gefreut: warum willst du dich also an unschuldigem Blut
versündigen, indem du David ohne Grund tötest?« 6Da schenkte Saul den
Vorstellungen Jonathans Gehör und schwur: »So wahr der HERR lebt, er
soll nicht sterben!« 7Nun rief Jonathan den David zu sich und teilte ihm
diese ganze Unterredung mit; dann führte er David zu Saul, und er war
wieder in der Umgebung des Königs wie früher.

\hypertarget{b-davids-neues-kriegsgluxfcck-sauls-abermaliger-mordversuch}{%
\paragraph{b) Davids neues Kriegsglück; Sauls abermaliger
Mordversuch}\label{b-davids-neues-kriegsgluxfcck-sauls-abermaliger-mordversuch}}

8Als dann aber der Krieg aufs neue ausbrach und David, der gegen die
Philister ins Feld gezogen war, ihnen eine schwere Niederlage
beigebracht und sie in die Flucht geschlagen hatte, 9kam der böse Geist
des HERRN wieder über Saul, als er in seinem Hause saß und seinen Speer
in der Hand hatte, während David die Zither spielte. 10Da gedachte Saul
den David mit dem Speer an die Wand zu spießen; aber David wich dem Saul
aus, so daß der Speer in die Wand fuhr und David sich durch die Flucht
retten konnte.

\hypertarget{c-davids-flucht-in-sein-haus-und-seine-errettung-durch-michals-list}{%
\paragraph{c) Davids Flucht in sein Haus und seine Errettung durch
Michals
List}\label{c-davids-flucht-in-sein-haus-und-seine-errettung-durch-michals-list}}

11In derselben Nacht nun sandte Saul Boten in Davids Haus, die ihn
bewachen sollten, damit er ihn am andern Morgen töten könnte. Aber
Davids Frau Michal verriet es ihm und sagte: »Wenn du dein Leben nicht
noch in dieser Nacht in Sicherheit bringst, so bist du morgen des
Todes!« 12Nachdem Michal ihn dann durchs Fenster hinabgelassen hatte,
ergriff er die Flucht und entkam glücklich. 13Hierauf nahm Michal den
Hausgott, legte ihn aufs Bett, legte dann ein Geflecht von Ziegenhaaren
zu seinen Häupten und deckte ihn mit der Decke zu. 14Als nun Saul Boten
sandte, um David festnehmen zu lassen, erklärte sie, er sei krank. 15Da
sandte Saul die Boten zurück, um nach David zu sehen, mit dem Befehl:
»Bringt ihn mitsamt dem Bett zu mir her, damit ich ihn töte.« 16Als nun
die Boten hinkamen, fanden sie den Hausgott im Bett liegen und das
Geflecht von Ziegenhaaren zu seinen Häupten. 17Da sagte Saul zu Michal:
»Warum hast du mich so betrogen und meinen Feind entrinnen lassen, so
daß er sich in Sicherheit gebracht hat?« Michal gab ihm zur Antwort: »Er
sagte zu mir: ›Laß mich gehen, sonst muß ich dich töten!‹«

\hypertarget{d-david-bei-samuel-in-rama-sauls-prophetische-verzuxfcckung-im-dortigen-prophetenhause}{%
\paragraph{d) David bei Samuel in Rama; Sauls prophetische Verzückung im
dortigen
Prophetenhause}\label{d-david-bei-samuel-in-rama-sauls-prophetische-verzuxfcckung-im-dortigen-prophetenhause}}

18David also hatte sich glücklich durch die Flucht gerettet und war nach
Rama zu Samuel gekommen, dem er alles mitteilte, was Saul ihm angetan
hatte. Er ging dann mit Samuel hin, und (beide) wohnten im
Prophetenhause bei Rama. 19Als nun Saul die Kunde erhielt, David halte
sich im Prophetenhause bei Rama auf, 20sandte er Boten ab, die David
festnehmen sollten. Als diese aber die Versammlung der Propheten sahen,
die sich in Begeisterung\textless sup title=``oder:
Verzückung''\textgreater✲ befanden, und Samuel an ihrer Spitze stehend
erblickten, kam der Geist Gottes über die Boten Sauls, so daß auch sie
in prophetische Begeisterung gerieten. 21Als man das dem Saul meldete,
schickte er andere Boten ab, aber auch diese gerieten in Verzückung, und
ebenso erging es den Boten, die Saul zum drittenmal sandte. 22Nun ging
er selbst nach Rama; und als er bei der großen Zisterne, die sich in
Sechu befindet, angekommen war, fragte er: »Wo sind hier Samuel und
David?« Man antwortete ihm: »Im Prophetenhause zu Rama.« 23Er ging also
von dort nach dem Prophetenhause bei Rama; als er aber noch unterwegs
war, kam auch über ihn der Geist Gottes, und er befand sich auf dem
ganzen Wege in prophetischer Begeisterung bis zu seiner Ankunft im
Prophetenhause bei Rama. 24Dort zog auch er seine Oberkleider aus und
war ebenfalls in Verzückung vor Samuel und lag (schließlich) in bloßen
Unterkleidern während jenes ganzen Tages und der ganzen Nacht da. Daher
pflegt man zu sagen: »Gehört auch Saul zu den Propheten?«

\hypertarget{jonathans-verabredung-mit-david-und-letzter-vergeblicher-aussuxf6hnungsversuch}{%
\subsubsection{6. Jonathans Verabredung mit David und letzter
vergeblicher
Aussöhnungsversuch}\label{jonathans-verabredung-mit-david-und-letzter-vergeblicher-aussuxf6hnungsversuch}}

\hypertarget{a-davids-zusammenkunft-und-besprechung-mit-jonathan-erneuerung-ihres-freundschaftsbundes}{%
\paragraph{a) Davids Zusammenkunft und Besprechung mit Jonathan;
Erneuerung ihres
Freundschaftsbundes}\label{a-davids-zusammenkunft-und-besprechung-mit-jonathan-erneuerung-ihres-freundschaftsbundes}}

\hypertarget{aa-david-beklagt-sich-bei-jonathan}{%
\subparagraph{aa) David beklagt sich bei
Jonathan}\label{aa-david-beklagt-sich-bei-jonathan}}

\hypertarget{section-19}{%
\section{20}\label{section-19}}

1David aber kam heimlich aus der Prophetenwohnung bei Rama zu Jonathan
und sagte zu ihm: »Was habe ich verbrochen? Worin besteht meine
Verschuldung und worin mein Vergehen gegen deinen Vater, daß er mir nach
dem Leben trachtet?« 2Er antwortete ihm: »Behüte Gott! Du wirst nicht
sterben! Du weißt, mein Vater tut nichts, es sei wichtig oder
unbedeutend, ohne mir Mitteilung davon zu machen: warum sollte mein
Vater also dies vor mir verheimlichen? Es ist nichts dran!« 3Da
entgegnete ihm David mit der bestimmten Versicherung: »Dein Vater weiß
ganz genau, daß du mich liebgewonnen hast; darum wird er sich gesagt
haben: ›Dies darf Jonathan nicht erfahren, damit er sich darüber nicht
aufregt‹. Aber so wahr der HERR lebt und so wahr du selbst lebst:
zwischen mir und dem Tode ist nur ein Schritt!« 4Da antwortete Jonathan
dem David: »Ich will dir jeden Wunsch erfüllen.«

\hypertarget{bb-davids-vorschlag}{%
\subparagraph{bb) Davids Vorschlag}\label{bb-davids-vorschlag}}

5Nun sagte David zu Jonathan: »Wie du weißt, ist morgen Neumondstag; da
müßte ich eigentlich mit dem König zu Tisch sitzen; aber laß mich gehen:
ich will mich auf dem Felde verbergen bis zum dritten Abend. 6Sollte
dein Vater mich etwa vermissen, so sage ihm: ›David hat mich dringend um
Urlaub gebeten, um nach seiner Vaterstadt Bethlehem zu eilen; denn dort
findet das jährliche Opferfest für die ganze Familie statt‹. 7Wenn er
dann sagt: ›Gut!‹, so droht deinem Knecht keine Gefahr; gerät er aber in
Zorn, so wisse, daß das Unheil seinerseits beschlossene Sache ist. 8Gib
also deinem Knecht einen Beweis deiner Liebe; du hast ja deinen Knecht
in einen heiligen Freundschaftsbund mit dir treten lassen. Sollte aber
eine Schuld bei mir liegen, so töte du mich; aber zu deinem Vater bringe
mich nicht zurück!« 9Da erwiderte Jonathan: »Behüte Gott! Nein, wenn ich
erkennen sollte, daß von meinem Vater die Ausführung einer bösen Absicht
gegen dich beschlossen ist, so würde ich es dir selbstverständlich
mitteilen.« 10David entgegnete dem Jonathan: »Wenn mir nur jemand eine
Mitteilung machen wollte, ob dein Vater dir eine abweisende Antwort
gegeben hat!« 11Da sagte Jonathan zu David: »Komm, laß uns aufs Feld
hinausgehen!«

\hypertarget{cc-der-eid-auf-gegenseitigkeit}{%
\subparagraph{cc) Der Eid auf
Gegenseitigkeit}\label{cc-der-eid-auf-gegenseitigkeit}}

Als nun beide aufs Feld hinausgegangen waren, 12sagte Jonathan zu David:
»Der HERR, der Gott Israels, ist Zeuge! Wenn ich morgen um diese Zeit
oder übermorgen in sichere Erfahrung gebracht habe, daß mein Vater es
gut mit David meint, und ich alsdann nicht zu dir sende und es dich
wissen lasse, 13so möge der HERR den Jonathan jetzt und später dafür
büßen lassen! Wenn aber mein Vater Böses gegen dich im Sinne hat, so
will ich dir auch das mitteilen und dich ziehen lassen, daß du dich in
Sicherheit bringen kannst. Der HERR möge dann mit dir sein, wie er mit
meinem Vater gewesen ist! 14Und nicht nur, während ich noch lebe, und
nicht nur an mir erweise du die Barmherzigkeit des HERRN, daß ich nicht
sterbe, 15sondern auch meinem Hause entziehe niemals deine Güte, auch
dann nicht, wenn der HERR die Feinde Davids allesamt vom Erdboden
vertilgen wird!« 16So schloß denn Jonathan einen Bund mit dem Hause
Davids und sagte: »Der HERR möge Rache üben an den Feinden Davids!«
17Dann schwur Jonathan dem David noch einmal bei seiner Liebe zu ihm;
denn er liebte ihn wie sein eigenes Leben.

\hypertarget{b-verabredung-des-zur-mitteilung-der-auskunft-zu-beobachtenden-verfahrens}{%
\paragraph{b) Verabredung des zur Mitteilung der Auskunft zu
beobachtenden
Verfahrens}\label{b-verabredung-des-zur-mitteilung-der-auskunft-zu-beobachtenden-verfahrens}}

18Hierauf sagte Jonathan zu ihm: »Morgen ist Neumondstag; da wird man
dich vermissen, wenn dein Platz bei Tisch leer bleibt. 19Übermorgen aber
wird man dich erst recht vermissen; da begib dich an den Ort, wo du dich
am Tage des (damaligen) Vorkommnisses verborgen hattest\textless sup
title=``vgl. 19,1-3''\textgreater✲, und verstecke dich neben dem
Steinhaufen dort. 20Ich will dann übermorgen in seiner Nähe drei Pfeile
abschießen, als ob ich nach einem Ziele schösse. 21Dann werde ich den
Burschen abschicken mit den Worten: ›Geh, suche die Pfeile!‹ Wenn ich
dann dem Burschen zurufe: ›Die Pfeile liegen von dir ab herwärts, hole
sie!‹, so komm; denn das bedeutet Gutes für dich, und du hast nichts zu
fürchten, so wahr der HERR lebt! 22Rufe ich aber dem Jungen so zu: ›Die
Pfeile liegen von dir ab hinwärts!‹, so gehe! Denn der HERR heißt dich
weggehen. 23Aber für das, was wir beide, ich und du, miteinander
besprochen haben, dafür ist der HERR (Zeuge) zwischen mir und dir in
Ewigkeit!« 24Hierauf versteckte sich David auf dem Felde.

\hypertarget{c-verlauf-der-beiden-mittagsmahlzeiten-im-hause-sauls-am-neumond-und-am-folgenden-tage}{%
\paragraph{c) Verlauf der beiden Mittagsmahlzeiten im Hause Sauls am
Neumond und am folgenden
Tage}\label{c-verlauf-der-beiden-mittagsmahlzeiten-im-hause-sauls-am-neumond-und-am-folgenden-tage}}

Als nun der Neumondstag da war, setzte sich der König zu Tisch, um zu
speisen, 25und zwar setzte sich der König auf seinen gewöhnlichen Platz,
nämlich auf den Platz an der Wand; Jonathan setzte sich ihm gegenüber
und Abner neben Saul, während Davids Platz leer blieb. 26Saul sagte an
diesem Tage nichts, weil er dachte: »Es ist ein Zufall; David wird nicht
rein sein, weil er sich noch nicht hat reinigen lassen.« 27Als aber am
anderen Tage, der auf den Neumond folgte, Davids Platz wieder leer
blieb, fragte Saul seinen Sohn Jonathan: »Warum ist der Sohn Isais weder
gestern noch heute zum Essen gekommen?« 28Jonathan antwortete dem Saul:
»David hat sich dringend Urlaub von mir nach Bethlehem erbeten; 29er hat
nämlich zu mir gesagt: ›Laß mich doch hingehen! Denn wir halten ein
Familienopfer in der Stadt; mein Bruder selbst hat mich dazu eingeladen.
Willst du mir also eine Liebe erweisen, so gib mir Urlaub, damit ich
meine Angehörigen besuchen kann!‹ Darum ist er nicht an der Tafel des
Königs erschienen.« 30Da geriet Saul in Zorn über Jonathan und sagte zu
ihm: »O du Sohn entarteter Widerspenstigkeit! Meinst du, ich wüßte
nicht, daß du dir den Sohn Isais zum Freunde erwählt hast zu deiner
Schande und zur Schande der Scham deiner Mutter? 31Denn solange der Sohn
Isais auf Erden lebt, wirst weder du noch dein Königtum festen Bestand
haben. So schicke denn jetzt hin und laß ihn mir holen; denn er ist ein
Kind des Todes!« 32Da gab Jonathan seinem Vater Saul zur Antwort: »Warum
soll er sterben? Was hat er verbrochen?« 33Da stieß Saul mit dem Speer
nach ihm, um ihn zu durchbohren. Als nun Jonathan erkannte, daß der Tod
Davids bei seinem Vater beschlossene Sache war, 34erhob er sich von der
Tafel in glühendem Zorn, ohne an diesem zweiten Neumondstage etwas
genossen zu haben; denn er trug um David Leid, weil sein Vater ihn
beschimpft hatte.

\hypertarget{d-jonathan-gibt-dem-david-kunde-von-dem-unguxfcnstigen-stande-der-sache-und-nimmt-abschied-von-ihm}{%
\paragraph{d) Jonathan gibt dem David Kunde von dem ungünstigen Stande
der Sache und nimmt Abschied von
ihm}\label{d-jonathan-gibt-dem-david-kunde-von-dem-unguxfcnstigen-stande-der-sache-und-nimmt-abschied-von-ihm}}

35Am folgenden Morgen aber ging Jonathan, wie er mit David verabredet
hatte, aufs Feld\textless sup title=``vgl. 19,3''\textgreater✲ hinaus,
und ein junger Bursche begleitete ihn. 36Diesem befahl er: »Lauf und
suche mir die Pfeile, die ich abschieße!« Während nun der Bursche
hinlief, schoß er den Pfeil über ihn hinaus; 37und als der Bursche an
die Stelle kam, wo der von Jonathan abgeschossene Pfeil lag, rief
Jonathan hinter dem Burschen her: »Der Pfeil liegt ja von dir hinwärts!«
38Weiter rief er hinter dem Burschen her: »Mache schnell! Halte dich
nicht auf!« Da las der Bursche Jonathans die Pfeile auf und kam zu
seinem Herrn zurück, 39ohne etwas gemerkt zu haben; nur Jonathan und
David wußten um die Sache. 40Dann übergab Jonathan sein Schießgerät
seinem Burschen mit der Weisung: »Geh, trage das in die Stadt!« 41Als
nun der Bursche sich entfernt hatte, kam David hinter dem Steinhaufen
hervor, warf sich mit dem Angesicht zur Erde nieder und verneigte sich
dreimal; darauf küßten sie einander und weinten zusammen, bis David
seine Fassung wiedergewann. 42Jonathan sagte dann zu David: »Gehe in
Frieden! Was wir beide uns im\textless sup title=``oder:
beim''\textgreater✲ Namen des HERRN zugeschworen haben, dafür wird der
HERR zwischen mir und dir, zwischen meinen und deinen Nachkommen Zeuge
sein in Ewigkeit!«

\hypertarget{section-20}{%
\section{21}\label{section-20}}

1Hierauf machte sich David auf und entfernte sich; Jonathan aber kehrte
in die Stadt zurück.

\hypertarget{david-als-fluxfcchtling-in-nob-und-in-gath-sauls-priestermord}{%
\subsubsection{7. David als Flüchtling in Nob und in Gath; Sauls
Priestermord}\label{david-als-fluxfcchtling-in-nob-und-in-gath-sauls-priestermord}}

\hypertarget{a-david-erhuxe4lt-vom-priester-ahimelech-in-nob-die-heiligen-schaubrote-und-goliaths-schwert}{%
\paragraph{a) David erhält vom Priester Ahimelech in Nob die heiligen
Schaubrote und Goliaths
Schwert}\label{a-david-erhuxe4lt-vom-priester-ahimelech-in-nob-die-heiligen-schaubrote-und-goliaths-schwert}}

2David kam dann nach Nob zum Priester Ahimelech; dieser kam ihm
ängstlich\textless sup title=``oder: unterwürfig''\textgreater✲ entgegen
und fragte ihn: »Warum kommst du allein, ohne einen Begleiter bei dir zu
haben?« 3David antwortete dem Priester Ahimelech: »Der König hat mir
einen Auftrag gegeben und zu mir gesagt: ›Niemand darf etwas von der
Sache erfahren, deretwegen ich dich absende und die ich dir aufgetragen
habe!‹ Darum habe ich auch meine Leute an einen bestimmten Ort bestellt.
4Und nun, wenn du einige Lebensmittel zur Verfügung hast, etwa fünf
Brote, so gib sie mir, oder was sonst vorhanden ist!« 5Der Priester
antwortete David: »Gewöhnliches Brot ist nicht in meinem Besitz, sondern
nur geweihtes Brot ist da; wenn sich die Leute nur von Weibern
ferngehalten haben!« 6Da antwortete David dem Priester: »Gewiß! Weiber
sind uns schon seit mehreren Tagen\textless sup title=``oder: in letzter
Zeit''\textgreater✲ versagt gewesen; als ich auszog, waren die Leiber
der Leute rein, obwohl es sich nur um ein gewöhnliches Unternehmen
handelte; wieviel mehr werden sie heute am Leibe rein sein!« 7Da gab ihm
der Priester geweihtes Brot, weil dort kein anderes Brot vorhanden war
als nur die Schaubrote, die man vor dem Angesicht des HERRN wegnimmt, um
neugebackene Brote am Tage ihrer Wegnahme dafür aufzulegen. 8Es war aber
an jenem Tage ein Mann dort anwesend, im Heiligtum vor dem HERRN
abgesondert\textless sup title=``oder: eingeschlossen''\textgreater✲,
einer von Sauls Dienern\textless sup title=``vgl. 16,15''\textgreater✲,
ein Edomiter namens Doeg, der Aufseher der Hirten Sauls. 9David fragte
dann Ahimelech: »Hast du hier nicht irgendeinen Speer oder ein Schwert
zur Hand? Ich habe nämlich weder mein Schwert noch meine anderen Waffen
mitgenommen, weil der Auftrag des Königs große Eile hatte.« 10Der
Priester antwortete: »Das Schwert des Philisters Goliath, den du im
Terebinthental erschlagen hast, das ist noch hier, eingewickelt in ein
Tuch, hinter dem Priesterkleide; wenn du es für dich nehmen willst, so
nimm es hin; denn ein anderes ist sonst nicht hier.« David erwiderte:
»Seinesgleichen gibt es nicht: gib es mir her!«

\hypertarget{b-david-stellt-sich-beim-kuxf6nig-achis-in-gath-wahnsinnig}{%
\paragraph{b) David stellt sich beim König Achis in Gath
wahnsinnig}\label{b-david-stellt-sich-beim-kuxf6nig-achis-in-gath-wahnsinnig}}

11Hierauf machte sich David auf den Weg und floh an jenem Tage vor Saul
und begab sich zu Achis, dem König von Gath. 12Da sagten die Diener des
Achis zu diesem: »Das ist ja David, der König des Landes; das ist ja
der, zu dessen Ehren sie bei den Reigentänzen singen: ›Saul hat seine
Tausende geschlagen, David aber seine Zehntausende.‹« 13Diese Worte
machten David überaus bedenklich, und er geriet in große Furcht vor
Achis, dem König von Gath. 14Daher stellte er sich wahnsinnig vor ihnen
und gebärdete sich wie ein Rasender unter ihren Händen, trommelte an die
Torflügel und ließ den Speichel in seinen Bart fließen. 15Da sagte Achis
zu seinen Dienern: »Ihr seht doch, daß der Mann verrückt ist: warum
bringt ihr ihn zu mir? 16Fehlt es mir hier etwa an verrückten Leuten,
daß ihr mir auch diesen noch hergebracht habt, damit er den Verrückten
bei mir spiele? Der sollte mir ins Haus kommen?«

\hypertarget{c-davids-weitere-flucht-nach-adullam-mizpe-in-moab-und-jaar-hereth-in-juda-seine-fuxfcrsorge-fuxfcr-seine-eltern}{%
\paragraph{c) Davids weitere Flucht nach Adullam, Mizpe in Moab und
Jaar-Hereth in Juda; seine Fürsorge für seine
Eltern}\label{c-davids-weitere-flucht-nach-adullam-mizpe-in-moab-und-jaar-hereth-in-juda-seine-fuxfcrsorge-fuxfcr-seine-eltern}}

\hypertarget{section-21}{%
\section{22}\label{section-21}}

1Da ging David von dort weg und flüchtete sich in die Höhle von Adullam.
Als nun seine Brüder und seine ganze Familie dies erfuhren, kamen sie
dorthin zu ihm hinab. 2Außerdem sammelten sich bei ihm allerlei Leute,
die sich in bedrängter Lage befanden, und alle, die Schulden hatten, und
allerlei mißvergnügte Leute, deren Anführer er wurde, so daß etwa
vierhundert Mann bei ihm waren. 3Von dort begab sich David nach Mizpe im
Moabiterlande und bat den König der Moabiter: »Laß doch meinen Vater und
meine Mutter bei euch wohnen, bis ich weiß, was Gott mit mir vorhat!«
4Da brachte er sie zu dem Moabiterkönig, und sie blieben bei diesem,
solange David auf der Bergfeste war. 5Als aber der Prophet Gad zu David
sagte: »Bleibe nicht auf der Bergfeste, sondern begib dich ins Land
Juda!«, da brach David auf und kam nach Jaar-Hereth\textless sup
title=``oder: in den Wald Hereth''\textgreater✲.

\hypertarget{d-sauls-klage-vor-seiner-umgebung-in-gibea-verrat-des-edomiters-doeg-sauls-blutige-rache-an-den-priestern-von-nob}{%
\paragraph{d) Sauls Klage vor seiner Umgebung in Gibea; Verrat des
Edomiters Doeg; Sauls blutige Rache an den Priestern von
Nob}\label{d-sauls-klage-vor-seiner-umgebung-in-gibea-verrat-des-edomiters-doeg-sauls-blutige-rache-an-den-priestern-von-nob}}

6Als nun Saul erfuhr, daß David mit seinen Leuten wieder zum Vorschein
gekommen war -- Saul saß aber gerade in Gibea unter der Tamariske auf
der Anhöhe, mit dem Speer in der Hand, während alle seine Diener um ihn
herum standen --, 7da sagte Saul zu seinen Dienern, die ihn umgaben:
»Hört doch, ihr Benjaminiten! Wird wohl der Sohn Isais euch allen auch
Äcker und Weinberge schenken und euch alle zu Hauptleuten über
Tausendschaften und zu Hauptleuten über hundert Mann machen, 8daß ihr
euch alle gegen mich verschworen habt und niemand mir eine Mitteilung
gemacht hat, als mein Sohn einen Freundschaftsbund mit dem Sohne Isais
schloß, und daß niemand von euch Mitgefühl mit mir gehabt und mir eine
Mitteilung gemacht hat, als mein Sohn meinen Diener (David) zur
Feindschaft gegen mich aufwiegelte, wie es jetzt klar zutage liegt?« 9Da
nahm der Edomiter Doeg, der neben den Dienern Sauls stand, das Wort und
sagte: »Ich habe gesehen, wie der Sohn Isais nach Nob zu Ahimelech, dem
Sohne Ahitubs, kam. 10Da hat dieser den HERRN für ihn befragt und ihm
auch Zehrung gegeben und ihm das Schwert des Philisters Goliath
übergeben.«

\hypertarget{das-blutgericht-zu-gibea}{%
\paragraph{Das Blutgericht zu Gibea}\label{das-blutgericht-zu-gibea}}

11Da ließ der König den Priester Ahimelech, den Sohn Ahitubs, und sein
ganzes Geschlecht, die Priesterschaft, die in Nob war, rufen; und als
sie sich alle beim König eingestellt hatten, 12sagte Saul: »Höre doch,
Sohn Ahitubs!« Dieser antwortete: »Hier bin ich, Herr!« 13Da fuhr Saul
fort: »Warum habt ihr euch gegen mich verschworen, du und der Sohn
Isais, daß du ihm Brot und ein Schwert gegeben und Gott für ihn befragt
hast, damit er sich gegen mich erhebe und auflehne, wie es jetzt klar
zutage liegt?« 14Ahimelech antwortete: »Aber wer unter allen deinen
Dienern ist so treu wie David? Dazu ist er des Königs Schwiegersohn und
Oberster über deine Leibwache und geehrt in deinem Hause. 15Ist es denn
jetzt das erste Mal gewesen, daß ich Gott für ihn befragt habe? Das sei
fern von mir! Der König möge doch seinem Knecht und meinem ganzen
Geschlecht so etwas nicht zur Last legen! Denn dein Knecht hat von allem
diesem nicht das Geringste gewußt!« 16Doch der König entgegnete:
»Ahimelech, du mußt ohne Gnade sterben, du selbst und dein ganzes
Geschlecht!« 17Hierauf gab der König seinen Leibwächtern\textless sup
title=``oder: Trabanten''\textgreater✲, die vor ihm standen, den Befehl:
»Tretet herzu und tötet die Priester des HERRN! Denn auch sie halten es
mit David, und obwohl sie wußten, daß er auf der Flucht war, haben sie
mir keine Mitteilung gemacht!« Aber die Diener des Königs weigerten
sich, Hand an die Priester des HERRN zu legen, um sie
niederzustoßen\textless sup title=``oder: niederzuhauen''\textgreater✲.
18Da befahl der König dem Doeg: »Tritt du herzu und stoße\textless sup
title=``oder: haue''\textgreater✲ die Priester nieder!« Da trat der
Edomiter Doeg heran, und er stieß die Priester nieder; er tötete an
jenem Tage fünfundachtzig Männer, die das leinene Priesterkleid trugen.
19Sodann ließ er in der Priesterstadt Nob alles Lebende mit dem Schwerte
niedermachen, Männer wie Frauen, Kinder wie Säuglinge, Rinder, Esel und
das Kleinvieh, alles ließ er mit dem Schwert niedermachen.

\hypertarget{e-der-fluxfcchtige-priester-abjathar-findet-bei-david-freundliche-aufnahme}{%
\paragraph{e) Der flüchtige Priester Abjathar findet bei David
freundliche
Aufnahme}\label{e-der-fluxfcchtige-priester-abjathar-findet-bei-david-freundliche-aufnahme}}

20Nur ein einziger Sohn Ahimelechs, des Sohnes Ahitubs, namens Abjathar,
rettete sich durch die Flucht zu Davids Gefolgschaft 21und berichtete
dem David, daß Saul die Priester des HERRN ermordet habe. 22Da sagte
David zu Abjathar: »Ich wußte schon damals, daß der Edomiter Doeg, der
dort anwesend war, es Saul sicherlich verraten würde. Ich selbst trage
die Schuld am Tode aller Angehörigen deines Geschlechts! 23Bleibe bei
mir und fürchte dich nicht! Denn nur wer mir nach dem Leben trachtet,
wird auch dir nach dem Leben trachten können: du bist bei mir in voller
Sicherheit!«

\hypertarget{david-in-der-wuxfcste-juda-in-kegila-und-maon-seine-letzte-begegnung-mit-jonathan-verrat-der-siphiter}{%
\subsubsection{8. David in der Wüste Juda (in Kegila und Maon); seine
letzte Begegnung mit Jonathan; Verrat der
Siphiter}\label{david-in-der-wuxfcste-juda-in-kegila-und-maon-seine-letzte-begegnung-mit-jonathan-verrat-der-siphiter}}

\hypertarget{a-david-rettet-kegila-vor-der-eroberung-durch-die-philister-muuxdf-aber-die-stadt-wieder-verlassen}{%
\paragraph{a) David rettet Kegila vor der Eroberung durch die Philister,
muß aber die Stadt wieder
verlassen}\label{a-david-rettet-kegila-vor-der-eroberung-durch-die-philister-muuxdf-aber-die-stadt-wieder-verlassen}}

\hypertarget{section-22}{%
\section{23}\label{section-22}}

1Als dann die Meldung bei David einlief, daß die Philister Kegila
belagerten und die Tennen plünderten, 2befragte David den HERRN: »Soll
ich hinziehen und die Philister dort schlagen?« Der HERR antwortete ihm:
»Ziehe hin, schlage die Philister und entsetze Kegila!« 3Aber Davids
Leute sagten zu ihm: »Wir müssen schon hier in Juda in ewiger Angst
leben; wie erst, wenn wir nach Kegila gegen die Schlachtreihen der
Philister ziehen!« 4Da befragte David den HERRN noch einmal und erhielt
vom HERRN die Antwort: »Mache dich auf und ziehe nach Kegila hinab! Denn
ich werde die Philister in deine Hand geben.« 5Da zog David mit seinen
Leuten nach Kegila, griff die Philister an, erbeutete ihre Herden und
brachte ihnen eine schwere Niederlage bei. So rettete David die
Einwohner von Kegila.~-- 6Als aber Abjathar, der Sohn Ahimelechs, zu
David {[}nach Kegila{]} floh, hatte er ein Priesterkleid mitgebracht.~--

7Als nun Saul erfuhr, daß David nach Kegila gezogen sei, rief er aus:
»Gott hat ihn mir in die Hände geliefert! Denn er hat sich selbst
eingeschlossen, indem er sich in eine Stadt mit Toren und Riegeln
begeben hat.« 8Saul bot also das ganze Volk zum Kriege auf, um nach
Kegila zur Belagerung Davids und seiner Leute hinabzuziehen. 9Als nun
David erkannte, daß Saul Böses gegen ihn im Schilde führte, befahl er
dem Priester Abjathar: »Bringe das Priesterkleid herbei!« 10Dann betete
David: »HERR, Gott Israels! Dein Knecht hat als gewiß gehört, daß Saul
nach Kegila zu ziehen gedenkt, um die Stadt um meinetwillen zu
vernichten. 11Werden die Bürger von Kegila mich ihm ausliefern? Wird
Saul wirklich herabkommen, wie dein Knecht vernommen hat? HERR, Gott
Israels, tu das doch deinem Knechte kund!« Da antwortete der HERR: »Ja,
er wird herabkommen.« 12Dann fragte David weiter: »Werden die Bürger von
Kegila mich und meine Leute an Saul ausliefern?« Der HERR antwortete:
»Ja, sie werden dich ausliefern.« 13Da machte sich David mit seinen
Leuten, etwa sechshundert Mann, auf und zog aus Kegila ab, und sie
streiften aufs Geratewohl umher. Als dann dem Saul gemeldet wurde, David
sei aus Kegila entronnen, stand er vom Zug dorthin ab.

\hypertarget{b-david-in-der-wuxfcste-siph-von-saul-verfolgt-seine-unterredung-mit-jonathan-in-horesa}{%
\paragraph{b) David in der Wüste Siph von Saul verfolgt; seine
Unterredung mit Jonathan in
Horesa}\label{b-david-in-der-wuxfcste-siph-von-saul-verfolgt-seine-unterredung-mit-jonathan-in-horesa}}

14David hielt sich dann in der Wüste auf den Berghöhen auf, und zwar
besonders im Gebirge in der Wüste Siph; und Saul suchte während der
ganzen Zeit nach ihm, aber Gott ließ ihn nicht in seine Hände fallen.
15Als nun David sah, daß Saul ausgezogen war, um ihn ums Leben zu
bringen -- er befand sich gerade in Horesa\textless sup title=``oder: in
der Heide''\textgreater✲ in der Wüste Siph --, 16da machte sich
Jonathan, der Sohn Sauls, auf und begab sich zu David nach
Horesa\textless sup title=``oder: nach der Heide''\textgreater✲ und
weckte neues Vertrauen auf Gott in ihm, 17indem er zu ihm sagte:
»Fürchte dich nicht! Denn die Hand meines Vaters Saul wird dich nicht
erreichen, sondern du wirst König über Israel werden, und ich werde der
Zweite nach dir sein; auch mein Vater Saul weiß das wohl.« 18Dann
schlossen beide einen Bund miteinander vor dem HERRN; und David blieb in
Horesa, während Jonathan wieder heimkehrte.

\hypertarget{c-david-von-den-siphitern-verraten-und-in-der-wuxfcste-maon-wunderbar-vor-saul-gerettet}{%
\paragraph{c) David von den Siphitern verraten und in der Wüste Maon
wunderbar vor Saul
gerettet}\label{c-david-von-den-siphitern-verraten-und-in-der-wuxfcste-maon-wunderbar-vor-saul-gerettet}}

19Hierauf gingen einige Siphiter zu Saul nach Gibea hinauf und meldeten:
»David hält sich bei uns auf den Berghöhen in Horesa\textless sup
title=``oder: in der Heide''\textgreater✲ verborgen, auf dem Höhenzuge
Hachila, der südlich von der Öde liegt. 20Nun denn, o König, komm zu uns
herab, sobald es dir beliebt; unsere Sache soll es alsdann sein, ihn in
die Gewalt des Königs zu bringen.« 21Da antwortete Saul: »Der HERR möge
euch dafür segnen, daß ihr Teilnahme für mich bewiesen habt! 22Geht nun
hin, vergewissert euch noch mehr und gebt sorgsam acht, um seinen
Aufenthaltsort zu erkunden und wer ihn daselbst gesehen hat; denn man
hat mir gesagt, er sei ein sehr listiger Mann. 23Wenn ihr alle
Schlupfwinkel, wo er sich versteckt hält, sicher erkundet habt, so kommt
wieder zu mir mit zuverlässiger Auskunft, dann will ich mit euch gehen;
und wenn er wirklich im Lande ist, so will ich ihn schon unter allen
Tausendschaften Judas aufspüren!« 24Da machten sich die Siphiter auf und
gingen Saul voraus nach Siph; David aber befand sich damals mit seinen
Leuten in der Wüste Maon, in der Steppe südlich von der Einöde.

25Als nun Saul mit seinen Leuten hinzog, um ihn aufzusuchen, und man es
David hinterbrachte, zog er nach dem Felsen hinab, der in der Wüste Maon
liegt. Auf die Kunde hiervon folgte Saul dem David eiligst in die Wüste
Maon nach. 26Da zog nun Saul auf der einen Seite des Bergzuges und David
mit seinen Leuten auf der anderen Seite. Als dann David sich beeilte,
dem Saul zu entkommen, war Saul mit seinen Leuten gerade im Begriff,
David und seine Leute zu umzingeln, um sie gefangenzunehmen. 27Da traf
plötzlich ein Bote bei Saul ein mit der Meldung: »Mache dich schleunigst
auf den Weg! Denn die Philister sind ins Land eingefallen.« 28Da mußte
Saul die Verfolgung Davids aufgeben und den Philistern entgegenziehen;
daher nennt man jenen Ort den ›Trennungsfelsen‹.

\hypertarget{davids-grouxdfmut-gegen-saul-in-der-huxf6hle-bei-engedi}{%
\subsubsection{9. Davids Großmut gegen Saul in der Höhle bei
Engedi}\label{davids-grouxdfmut-gegen-saul-in-der-huxf6hle-bei-engedi}}

\hypertarget{section-23}{%
\section{24}\label{section-23}}

1David zog dann von dort hinweg und setzte sich auf den Berghöhen von
Engedi fest. 2Als nun Saul von der Verfolgung der Philister
zurückgekehrt war, meldete man ihm, David sei jetzt in der Wüste von
Engedi. 3Da nahm Saul dreitausend Mann, auserlesene Leute aus ganz
Israel, und brach auf, um David und seine Leute auf der Ostseite der
Steinbockfelsen zu suchen. 4Als er nun zu den Schafhürden am Wege kam,
war dort eine Höhle, in die Saul hineinging, um seine Notdurft zu
verrichten; David aber saß mit seinen Leuten hinten in der Höhle. 5Da
sagten Davids Leute zu ihm: »Wahrlich, dies ist der Tag, von dem der
HERR zu dir gesagt hat: ›Fürwahr, ich will deinen Feind dir in die Hände
liefern, so daß du mit ihm verfahren kannst, wie es dir beliebt.‹«
Hierauf stand David auf und schnitt unbemerkt einen Zipfel von Sauls
Mantel ab. 6Hinterher aber schlug ihm doch das Gewissen, daß er den
Zipfel von Sauls Mantel abgeschnitten hatte, 7und er sagte zu seinen
Leuten: »Der HERR bewahre mich davor, so etwas zu tun und mich an meinem
Herrn, dem Gesalbten Gottes, zu vergreifen! Er ist ja der Gesalbte
Gottes!« 8Mit diesen Worten trat David seinen Leuten entgegen und
gestattete ihnen nicht, Saul ein Leid anzutun.

\hypertarget{die-zwischen-saul-und-david-gewechselten-reden-ihr-abschied}{%
\paragraph{Die zwischen Saul und David gewechselten Reden; ihr
Abschied}\label{die-zwischen-saul-und-david-gewechselten-reden-ihr-abschied}}

Als dann Saul die Höhle verlassen hatte und seines Weges weiterzog,
9machte sich auch David alsbald auf, trat aus der Höhle hinaus und rief
hinter Saul her: »Mein Herr und König!« Als Saul sich nun umwandte,
verneigte David sich mit dem Angesicht zur Erde nieder und brachte dem
Könige seine Huldigung dar; 10dann rief er dem Saul zu: »Warum hörst du
auf das Gerede der Leute, die da sagen, David sinne auf dein Verderben?
11Siehe, am heutigen Tage hast du mit eigenen Augen sehen können, daß
der HERR dich heute in der Höhle in meine Gewalt gegeben hatte; und
obgleich man mir zuredete, ich möchte dich umbringen, habe ich dich doch
verschont und habe gedacht: ›Ich will mich nicht an meinem Herrn
vergreifen, weil er der Gesalbte Gottes ist.‹ 12Und nun, mein Vater,
sieh her! Ja, sieh hier den Zipfel deines Mantels in meiner Hand! Daran,
daß ich den Zipfel von deinem Mantel abgeschnitten habe, ohne dich zu
töten, daran kannst du mit Sicherheit erkennen, daß ich an nichts Böses
und an keinen Verrat gegen dich gedacht und mich nicht an dir versündigt
habe, während du mir nachstellst, um mich des Lebens zu berauben. 13Der
HERR möge Richter zwischen mir und dir sein, und der HERR möge mich an
dir rächen! Aber meine Hand soll nicht gegen dich sein! 14Schon das alte
Sprichwort sagt: ›Von den Gottlosen mag Gottlosigkeit ausgehen!‹, aber
meine Hand soll nicht gegen dich sein! 15Hinter wem zieht denn der König
von Israel her? Wen verfolgst du? Einen toten Hund! Einen einzelnen
Floh! 16So sei denn der HERR Richter und entscheide zwischen mir und
dir! Er möge die Untersuchung in meiner Sache führen und mich vertreten
und mir Recht gegen dich schaffen!«

17Als David diese Worte an Saul gerichtet hatte, antwortete dieser: »Ist
das nicht deine Stimme, mein Sohn David?« Hierauf fing Saul an, laut zu
weinen, 18und rief dem David zu: »Du bist gerechter✲ als ich; denn du
hast mir Gutes erwiesen, während ich böse an dir gehandelt habe. 19Und
heute hast du mir deine Liebe in besonderem Maße dadurch bewiesen, daß
du mich nicht getötet hast, als der HERR mich in deine Hand gegeben
hatte. 20Denn wenn jemand seinen Feind antrifft, läßt er ihn da wohl
friedlich seines Weges ziehen? So möge denn der HERR dir mit Gutem
vergelten, was du heute an mir getan hast! 21Und nun, siehe, ich weiß
gewiß, daß du König werden wirst und daß in deiner Hand\textless sup
title=``oder: durch dich''\textgreater✲ das Königtum über Israel Bestand
haben wird. 22So schwöre mir denn jetzt beim HERRN, daß du meine
Nachkommen nach meinem Tode nicht ausrotten und meinen Namen aus meinem
Geschlecht nicht austilgen willst!« 23Da schwur David es dem Saul; dann
zog Saul heim, während David sich mit seinen Leuten auf die Bergfeste
begab.

\hypertarget{samuels-tod-nabals-torheit-david-und-abigail}{%
\subsubsection{10. Samuels Tod; Nabals Torheit; David und
Abigail}\label{samuels-tod-nabals-torheit-david-und-abigail}}

\hypertarget{section-24}{%
\section{25}\label{section-24}}

1Da starb Samuel, und ganz Israel versammelte sich und hielt um ihn die
Totenklage; man begrub ihn dann in\textless sup title=``oder:
bei''\textgreater✲ seinem Hause zu Rama.

\hypertarget{a-nabals-tuxf6richtes-verhalten-gegenuxfcber-davids-ersuchen}{%
\paragraph{a) Nabals törichtes Verhalten gegenüber Davids
Ersuchen}\label{a-nabals-tuxf6richtes-verhalten-gegenuxfcber-davids-ersuchen}}

David aber machte sich auf und zog in die Wüste Paran hinab. 2Nun lebte
da in Maon ein Mann, der sein Anwesen in Karmel hatte, ein sehr
begüterter Mann, der dreitausend Schafe und tausend Ziegen besaß; und er
war gerade mit der Schur seiner Schafe in Karmel beschäftigt. 3Der Mann
hieß Nabal und seine Frau Abigail; die Frau war klug und von großer
Schönheit, der Mann dagegen roh und bösartig in all seinem Tun, ein
echter Kalebiter. 4Als nun David in der Wüste hörte, daß Nabal eben
Schafschur hielt, 5schickte er zehn von seinen Leuten ab mit dem
Auftrage: »Geht nach Karmel hinauf, kehrt bei Nabal ein, grüßt ihn von
mir 6und sagt zu meinem Bruder: ›Heil dir und Heil deiner Familie und
Heil allem, was du besitzest! 7Ich habe jetzt eben vernommen, daß die
Schafschur bei dir stattfindet. Da nun deine Hirten sich bei uns hier
aufgehalten haben, ohne daß wir ihnen etwas zuleide getan und ohne daß
sie während der ganzen Zeit ihres Aufenthalts in Karmel das Geringste
vermißt haben~-- 8frage nur deine Leute, sie werden es dir bestätigen!
--, so erweise dich nun freundlich gegen die Leute, zumal da wir an
einem Festtage zu dir kommen. Gib also deinen Knechten und deinem Sohne
David, was dir gerade vor die Hand kommt!‹« 9Als nun Davids Leute
hinkamen, richteten sie den Auftrag im Namen Davids bei Nabal genau aus
und warteten dann schweigend. 10Nabal aber gab den Leuten Davids zur
Antwort: »Wer ist David, und wer ist der Sohn Isais? Heutzutage gibt es
Knechte genug, die ihren Herren entlaufen! 11Soll ich etwa mein Brot,
mein Wasser und mein Geschlachtetes, das ich für meine Scherer
geschlachtet habe, nehmen und es Leuten geben, von denen ich nicht
einmal weiß, woher sie sind?« 12Darauf wandten sich Davids Leute um und
zogen ihres Weges zurück und erstatteten David nach ihrer Rückkehr
genauen Bericht über den Vorfall.

\hypertarget{b-david-bricht-zum-rachezuge-auf-abigail-erhuxe4lt-kunde-von-der-unbesonnenheit-ihres-mannes}{%
\paragraph{b) David bricht zum Rachezuge auf; Abigail erhält Kunde von
der Unbesonnenheit ihres
Mannes}\label{b-david-bricht-zum-rachezuge-auf-abigail-erhuxe4lt-kunde-von-der-unbesonnenheit-ihres-mannes}}

13Da befahl David seinen Leuten: »Jeder gürte sein Schwert um!« Nachdem
nun alle dem Befehle nachgekommen waren und auch David sich sein Schwert
umgegürtet hatte, zogen sie unter Davids Führung hinauf, etwa
vierhundert Mann, während zweihundert beim Gepäck zurückblieben.

14Inzwischen hatte aber einer von den Knechten der Abigail, der Frau
Nabals, berichtet: »David hat soeben Boten aus der Wüste hergeschickt,
um unsern Herrn begrüßen zu lassen, der aber hat sie grob angefahren.
15Und doch sind die Männer sehr gut gegen uns gewesen; es ist uns von
ihnen nichts zuleide geschehen, und wir haben nicht das Geringste
vermißt während der ganzen Zeit, die wir bei ihnen auf dem Felde
umhergezogen sind; 16nein, sie sind eine Mauer um uns bei Tag und bei
Nacht gewesen, solange wir das Kleinvieh in ihrer Nähe gehütet haben.
17Überlege jetzt also und sieh zu, was du tun willst! Denn unserm Herrn
und seinem ganzen Hause steht sicherlich ein Unglück bevor; er selbst
aber ist ein zu bösartiger Mann, als daß man mit ihm reden könnte.«

\hypertarget{c-abigail-huxe4lt-david-durch-ihr-kluges-verfahren-vom-vollzug-der-rache-ab}{%
\paragraph{c) Abigail hält David durch ihr kluges Verfahren vom Vollzug
der Rache
ab}\label{c-abigail-huxe4lt-david-durch-ihr-kluges-verfahren-vom-vollzug-der-rache-ab}}

18Da nahm Abigail in aller Eile zweihundert Brote und zwei Schläuche
Wein, fünf zubereitete Schafe, fünf Scheffel geröstetes Getreide,
hundert Rosinentrauben\textless sup title=``oder:
Traubenkuchen''\textgreater✲ und zweihundert Feigenkuchen, lud alles auf
Esel 19und befahl ihren Knechten: »Zieht mir voraus, ich komme sogleich
hinter euch her!« Ihrem Manne Nabal aber sagte sie nichts davon.
20Während sie nun an einer durch den Berg verdeckten Stelle auf ihrem
Esel abwärts ritt, kam gerade auch David mit seinen Leuten von der
entgegengesetzten Seite herab, und sie traf mit ihnen zusammen. 21Nun
hatte David bei sich überlegt: »Rein umsonst habe ich diesem Menschen
seine gesamte Habe in der Wüste beschützt, so daß ihm nie das Geringste
von seinem gesamten Besitz verlorengegangen ist; er aber hat mir Gutes
mit Bösem vergolten. 22Gott möge es den Feinden Davids jetzt und künftig
gut ergehen lassen, wenn ich von allem, was ihm gehört, bis morgen früh
ein einziges Mannesbild übriglasse!«

23Als nun Abigail Davids ansichtig wurde, stieg sie schleunigst von
ihrem Esel herab, warf sich vor David auf ihr Angesicht nieder,
verneigte sich dann zur Erde 24und rief kniefällig aus: »Auf mir allein,
mein Herr, liegt die Schuld! Laß doch deine Magd vor dir reden und
schenke den Worten deiner Magd Gehör! 25Mein Herr gebe doch nichts auf
diesen nichtswürdigen Menschen, auf Nabal! Denn er ist wirklich so, wie
sein Name besagt: er heißt Nabal\textless sup title=``d.h.
Tor''\textgreater✲ und verübt nur Torheiten; ich aber, deine Magd, habe
die Leute, die du, mein Herr, gesandt hast, nicht zu Gesicht bekommen.
26Und nun, mein Herr, so wahr Gott lebt und so wahr du selbst lebst:
Gott hat dich davor behütet, in Blutschuld zu geraten und dich mit
eigener Hand zu rächen. So mögen nun deine Feinde und alle, die Böses
gegen meinen Herrn sinnen, dem Nabal gleich werden! 27Und nun, dieses
Geschenk hier, das deine Magd für meinen Herrn mitgebracht hat, ist für
die Leute bestimmt, die meinem Herrn auf seinen Zügen folgen. 28Vergib
deiner Magd ihr Vergehen! Denn sicherlich wird Gott meinem Herrn ein
Haus bauen, das Bestand hat, weil mein Herr im Dienste Gottes streitet
und kein Unrecht sich an dir finden wird, solange du lebst. 29Und wenn
ein Mensch sich erheben sollte, dich zu verfolgen und dir nach dem Leben
zu trachten, so möge die Seele\textless sup title=``oder: das
Leben''\textgreater✲ meines Herrn eingebunden\textless sup
title=``=~wohl verwahrt''\textgreater✲ sein im Bündel des
Lebens\textless sup title=``oder: der Lebenden''\textgreater✲ beim
HERRN, deinem Gott! die Seele\textless sup title=``oder: das
Leben''\textgreater✲ deiner Feinde aber möge er wegschleudern in der
Schleuderpfanne! 30Wenn Gott dann meinem Herrn all das Glück verleihen
wird, das er dir verheißen hat, und dich zum Fürsten über Israel
bestellt, 31so wirst du dich frei in deinem Inneren fühlen, und mein
Herr braucht sich keine Vorwürfe zu machen, daß du, mein Herr, Blut ohne
Ursache\textless sup title=``=~unschuldiges Blut''\textgreater✲
vergossen und dir mit eigener Hand Recht geschafft habest. Wenn aber
Gott meinem Herrn Glück verleihen wird, so gedenke deiner Magd!«

32Da antwortete David der Abigail: »Gepriesen sei der HERR, der Gott
Israels, der dich mir heute hat entgegenkommen lassen! 33Und gepriesen
sei deine Klugheit und gepriesen du selbst, daß du mich heute davon
abgehalten hast, in Blutschuld zu geraten und mir mit eigener Hand
Genugtuung zu verschaffen! 34Denn so wahr der HERR lebt, der Gott
Israels, der mich davor behütet hat, dir ein Leid anzutun: wärst du mir
nicht so schnell entgegengekommen, so wäre dem Nabal bis morgen früh
kein einziges Mannesbild übriggeblieben!« 35Darauf nahm David von ihr
an, was sie ihm mitgebracht hatte; zu ihr selbst aber sagte er: »Kehre
in Frieden in dein Haus zurück! Wisse wohl: ich habe dir Gehör geschenkt
und Rücksicht auf dich genommen!«

\hypertarget{c-nabals-juxe4her-tod-davids-verheiratung-mit-abigail-und-mit-ahinoam}{%
\paragraph{c) Nabals jäher Tod; Davids Verheiratung mit Abigail (und mit
Ahinoam)}\label{c-nabals-juxe4her-tod-davids-verheiratung-mit-abigail-und-mit-ahinoam}}

36Als dann Abigail zu Nabal zurückkehrte, hielt er gerade ein Gastmahl
in seinem Hause, ein geradezu königliches Festgelage, und er befand sich
in der fröhlichsten Stimmung. Da er schwer betrunken war, teilte sie ihm
nicht das Geringste mit, bis der Morgen anbrach. 37Als er aber am
folgenden Morgen seinen Rausch ausgeschlafen hatte, machte seine Frau
ihm Mitteilung von allem, was vorgegangen war. Da erlitt er einen
Schlaganfall und wurde wie ein Stein; 38und nach etwa zehn Tagen traf
ihn die Hand des HERRN, daß er starb. 39Als nun David die Nachricht vom
Tode Nabals erhielt, rief er aus: »Gepriesen sei der HERR, der die mir
von Nabal zugefügte Schmach gerächt und mich, seinen Knecht, vom
Bösestun zurückgehalten, die Bosheit Nabals aber auf ihn selbst hat
zurückfallen lassen!« Darauf sandte David hin und warb um Abigail, um
sie sich zum Weibe zu nehmen. 40Als nun Davids Boten nach Karmel zu
Abigail kamen und die Werbung anbrachten mit den Worten: »David hat uns
zu dir gesandt: er wünscht dich als sein Weib heimzuführen«, 41da erhob
sie sich, verneigte sich mit dem Antlitz bis zur Erde und sagte: »Ja,
deine Magd ist bereit, als Dienerin den Knechten meines Herrn die Füße
zu waschen!« 42Sodann machte Abigail sich schleunigst auf und setzte
sich auf ihren Esel; ebenso ihre fünf Dienerinnen, die ihre Begleitung
bildeten. So folgte sie den Boten Davids und wurde sein Weib.

43David hatte sich aber auch Ahinoam aus Jesreel (in Juda)
geholt\textless sup title=``oder: gewonnen''\textgreater✲; so wurden
beide zumal seine Frauen. 44Saul dagegen hatte seine Tochter Michal, die
mit David verheiratet war, Palti, dem Sohne des Lais aus Gallim, zur
Frau gegeben.

\hypertarget{davids-nochmalige-grouxdfmut-gegen-saul-in-der-wuxfcste-siph}{%
\subsubsection{11. Davids nochmalige Großmut gegen Saul in der Wüste
Siph}\label{davids-nochmalige-grouxdfmut-gegen-saul-in-der-wuxfcste-siph}}

\hypertarget{a-saul-von-neuem-auf-der-verfolgung-david-und-abisai-bei-nacht-im-lager-sauls}{%
\paragraph{a) Saul von neuem auf der Verfolgung; David und Abisai bei
Nacht im Lager
Sauls}\label{a-saul-von-neuem-auf-der-verfolgung-david-und-abisai-bei-nacht-im-lager-sauls}}

\hypertarget{section-25}{%
\section{26}\label{section-25}}

1Da kamen die Siphiter zu Saul nach Gibea und meldeten: »Wisse, David
hält sich auf dem Höhenzuge Hachila am Saume der Einöde\textless sup
title=``vgl. 23,19''\textgreater✲ verborgen.« 2Da machte sich Saul mit
dreitausend Mann auserlesener Israeliten auf den Weg und zog in die
Wüste Siph hinab, um David in der Wüste Siph zu suchen. 3Saul lagerte
sich dann auf dem Höhenzuge Hachila, der am Saum der Einöde an der
Straße liegt, während David sich in der Wüste aufhielt. Als dieser nun
erfuhr, daß Saul zu seiner Verfolgung in die Wüste gekommen war, 4sandte
er Kundschafter aus und erhielt so die sichere Nachricht von der Ankunft
Sauls. 5Er machte sich also auf und kam an den Ort, wo Saul sich
gelagert hatte. Als aber David den Platz sah, wo Saul mit seinem
Heerführer Abner, dem Sohne Ners, lag -- Saul lag nämlich innerhalb der
Wagenburg, während das Kriegsvolk rings um ihn her gelagert war --, 6da
wandte sich David an den Hethiter Ahimelech und an Abisai, den Sohn der
Zeruja, den Bruder Joabs, mit der Frage: »Wer geht mit mir zu Saul in
das Lager hinab?« Abisai antwortete: »Ich gehe mit dir hinab!« 7Als nun
David und Abisai nachts zu den Leuten gekommen waren, da fanden sie Saul
schlafend innerhalb der Wagenburg liegen, und sein Speer steckte zu
seinen Häupten im Boden; Abner aber und das Kriegsvolk lagen um ihn her.
8Da sagte Abisai zu David: »Heute hat Gott deinen Feind dir in die Hände
geliefert; und nun laß mich ihn doch mit seinem Speer an den Boden
spießen mit einem Stoß: es soll kein zweiter Stoß für ihn nötig sein!«
9Aber David erwiderte dem Abisai: »Tu ihm nichts zuleide! Denn wer
könnte Hand an den Gesalbten des HERRN legen und bliebe ungestraft?«
10Dann fuhr David fort: »So wahr der HERR lebt, nein! Sicherlich wird
die Hand des HERRN ihn treffen, sei es, daß der Tag eines natürlichen
Todes für ihn kommt oder daß er in den Krieg zieht und hinweggerafft
wird; 11aber der HERR behüte mich davor, die Hand an den Gesalbten des
HERRN zu legen! Nein, nimm jetzt den Speer, der zu seinen Häupten steht,
und den Wasserkrug; dann wollen wir gehen!« 12Hierauf nahm David den
Speer und den Wasserkrug, der zu Sauls Häupten stand, und sie entfernten
sich, ohne daß jemand sie gesehen oder etwas gemerkt hätte, und niemand
wachte auf; vielmehr schliefen sie allesamt, weil ein tiefer, vom HERRN
gesandter Schlaf auf sie gefallen war.

\hypertarget{b-davids-huxf6hnischer-zuruf-an-abner}{%
\paragraph{b) Davids höhnischer Zuruf an
Abner}\label{b-davids-huxf6hnischer-zuruf-an-abner}}

13Als dann David auf die gegenüberliegende Seite (des Tales) gekommen
und in einiger Entfernung auf die Spitze des Berges getreten war, so daß
ein großer Zwischenraum zwischen ihnen lag, 14rief David dem Kriegsvolk
und Abner, dem Sohne Ners, die Worte zu: »Gibst du keine Antwort,
Abner?« Da erwiderte Abner: »Wer bist du, daß du dem Könige so zurufst?«
15David antwortete dem Abner: »Du bist mir fürwahr ein rechter Mann, und
deinesgleichen gibt es nicht in ganz Israel! Warum hast du auf deinen
Herrn, den König, nicht achtgegeben? Es ist ja einer von den Leuten
eingedrungen, um deinen Herrn, den König, zu ermorden! 16Das bringt dir
keine Ehre, was du da getan hast. So wahr der HERR lebt: den Tod habt
ihr verdient, weil ihr auf euren Herrn, den Gesalbten Gottes, nicht
achtgegeben habt! Und nun sieh doch mal nach, wo der Speer des Königs
geblieben ist und wo der Wasserkrug, der zu seinen Häupten gestanden
hat!«

\hypertarget{c-die-zwischen-saul-und-david-gewechselten-reden-das-auseinandergehen-beider}{%
\paragraph{c) Die zwischen Saul und David gewechselten Reden; das
Auseinandergehen
beider}\label{c-die-zwischen-saul-und-david-gewechselten-reden-das-auseinandergehen-beider}}

17Da erkannte Saul die Stimme Davids und rief aus: »Ist das nicht deine
Stimme, mein Sohn David?« David erwiderte: »Jawohl, mein Herr und
König!« 18Dann fuhr er fort: »Warum verfolgt mein Herr seinen Knecht?
Was habe ich denn verbrochen, und was für Böses klebt an meiner Hand?
19Möchte doch mein Herr, der König, jetzt den Worten seines Knechtes
Gehör schenken! Wenn Gott es ist, der dich gegen mich erbittert hat, so
möge\textless sup title=``oder: soll''\textgreater✲ er den Duft einer
Opfergabe zu riechen bekommen; wenn es aber Menschen sind, so seien sie
verflucht vor Gott, weil sie mich jetzt von der Teilnahme am Erbbesitz
des HERRN ausgeschlossen haben, als wollten sie zu mir sagen: ›Hinweg
mit dir! Diene anderen Göttern!‹ 20So möge nun mein Blut nicht fern vom
Angesicht Gottes zur Erde fallen! Denn der König von Israel ist
ausgezogen, um mein Leben zu erjagen, wie der Habicht auf ein Rebhuhn in
den Bergen Jagd macht.«

21Da antwortete Saul: »Ich habe unrecht getan: kehre zurück, mein Sohn
David! Denn ich will dir fortan nichts mehr zuleide tun zum Lohn dafür,
daß du mein Leben heute verschont hast. Ja, ich habe töricht gehandelt
und mich schwer vergangen!« 22Darauf antwortete David: »Siehe, hier ist
der Speer des Königs: so komme denn einer von den Leuten herüber und
hole ihn! 23Der HERR aber vergilt einem jeden seine Gerechtigkeit und
seine Treue; denn der HERR hatte dich heute in meine Hand gegeben, ich
aber habe mich nicht an dem Gesalbten des HERRN vergreifen wollen. 24So
wertvoll aber dein Leben heute mir gewesen ist, ebenso wertvoll möge
mein Leben dem HERRN sein, daß er mich aus aller Bedrängnis errette!«
25Da rief Saul dem David zu: »Gesegnet seist du, mein Sohn David! Du
wirst deine Sache sicherlich durchführen und glücklich ans Ziel
gelangen!« Hierauf ging David seines Weges, Saul aber kehrte nach Hause
zurück.

\hypertarget{davids-uxfcbertritt-zu-den-philistern-sein-aufenthalt-beim-philisterfuxfcrsten-achis-in-gath-und-in-ziklag}{%
\subsubsection{12. Davids Übertritt zu den Philistern; sein Aufenthalt
beim Philisterfürsten Achis in Gath und in
Ziklag}\label{davids-uxfcbertritt-zu-den-philistern-sein-aufenthalt-beim-philisterfuxfcrsten-achis-in-gath-und-in-ziklag}}

\hypertarget{section-26}{%
\section{27}\label{section-26}}

1David aber sagte zu sich selbst: »Eines Tages werde ich doch noch durch
Sauls Hand weggerafft werden\textless sup title=``=~den Tod
finden''\textgreater✲; es bleibt mir nichts Besseres übrig, als daß ich
mich eilends im Lande der Philister in Sicherheit bringe; dann wird Saul
davon abstehen, auf mich noch länger in allen Teilen Israels zu fahnden,
und ich bin seinen Händen entronnen.« 2So machte David sich denn auf und
zog mit den sechshundert Mann, die er bei sich hatte, zu Achis, dem
Sohne Maochs, dem König von Gath, hinüber\textless sup title=``vgl.
21,11-16''\textgreater✲. 3Er nahm bei diesem seinen Wohnsitz in Gath mit
seinen Leuten, ein jeder mit seiner Familie, David selbst mit seinen
beiden Frauen, Ahinoam aus Jesreel und Abigail, der Witwe Nabals, aus
Karmel. 4Als nun Saul die Nachricht erhielt, daß David nach Gath
entwichen sei, gab er es auf, noch länger nach ihm zu suchen.

5David aber sagte zu Achis: »Wenn du mir eine Gnade erweisen willst, so
laß mir einen Platz in einer der Ortschaften des offenen Landes zum
Wohnsitz anweisen; denn warum soll dein Knecht bei dir in der
Königsstadt wohnen?« 6Da wies ihm Achis noch an demselben Tage Ziklag
an. Infolgedessen hat Ziklag den Königen von Juda bis auf den heutigen
Tag gehört. 7Die ganze Zeit aber, die David im Lande der Philister
zubrachte, betrug ein Jahr und vier Monate.

\hypertarget{davids-freibeuterleben-seine-tuxe4uschung-der-philister}{%
\paragraph{Davids Freibeuterleben; seine Täuschung der
Philister}\label{davids-freibeuterleben-seine-tuxe4uschung-der-philister}}

8David unternahm nun mit seinen Leuten Raubzüge und machte Einfälle in
das Gebiet der Gesuriter, der Girsiter und der Amalekiter; denn das sind
die Bewohner des Landes von alters her\textless sup title=``oder: von
Telam an''\textgreater✲ bis nach Sur hin und bis nach Ägypten. 9Sooft er
aber in ihr Land einfiel, ließ er weder Männer noch Frauen am Leben und
raubte Kleinvieh, Rinder, Esel, Kamele und Kleidungsstücke; darauf
kehrte er wieder zu Achis zurück. 10Wenn Achis ihn dann fragte: »Wo habt
ihr diesmal einen Einfall gemacht?«, so antwortete David: »Im Südland
von Juda« oder »im Südland der Jerahmeeliter« oder »im Südland der
Keniter«. 11Männer und Frauen aber ließ David deshalb nicht am Leben, um
sie nicht nach Gath mitnehmen zu müssen; denn er dachte: »Sie könnten
gegen uns aussagen und berichten: ›So und so ist David zu Werke
gegangen.‹« Dieses Verfahren beobachtete David während der ganzen Zeit,
die er im Lande der Philister zubrachte. 12Achis aber schenkte dem David
Vertrauen, weil er dachte: »Er hat sich bei seinem Volk, bei Israel,
tödlich verhaßt gemacht; darum wird er für immer mein Dienstmann
bleiben.«

\hypertarget{der-krieg-mit-den-philistern-saul-bei-der-totenbeschwuxf6rerin-zu-endor}{%
\subsubsection{13. Der Krieg mit den Philistern; Saul bei der
Totenbeschwörerin zu
Endor}\label{der-krieg-mit-den-philistern-saul-bei-der-totenbeschwuxf6rerin-zu-endor}}

\hypertarget{a-david-erkluxe4rt-sich-zur-teilnahme-am-kriege-gegen-sein-volk-bereit}{%
\paragraph{a) David erklärt sich zur Teilnahme am Kriege gegen sein Volk
bereit}\label{a-david-erkluxe4rt-sich-zur-teilnahme-am-kriege-gegen-sein-volk-bereit}}

\hypertarget{section-27}{%
\section{28}\label{section-27}}

1Als nun die Philister in jener Zeit ihre Heere zu einem Kriegszuge
gegen die Israeliten sammelten, sagte Achis zu David: »Du sollst
bestimmt wissen, daß du samt deinen Leuten mit mir im Heerbann ausziehen
mußt.« 2David antwortete ihm: »Gut! Du wirst nun sehen, was dein Knecht
zu leisten vermag.« Achis erwiderte dem David: »Gut! Ich ernenne dich
auf Lebenszeit zum Hüter meines Hauptes\textless sup title=``=~zu meinem
Leibwächter''\textgreater✲.«

\hypertarget{b-beginn-des-krieges-saul-beschlieuxdft-in-seiner-ratlosigkeit-die-befragung-eines-totenorakels}{%
\paragraph{b) Beginn des Krieges; Saul beschließt in seiner Ratlosigkeit
die Befragung eines
Totenorakels}\label{b-beginn-des-krieges-saul-beschlieuxdft-in-seiner-ratlosigkeit-die-befragung-eines-totenorakels}}

3Samuel aber war gestorben✲, und ganz Israel hatte die Totenklage um ihn
gehalten und ihn in seiner Vaterstadt Rama begraben. Saul aber hatte die
Totenbeschwörer und Wahrsager aus dem Lande vertrieben.~-- 4Nachdem sich
nun die Philister gesammelt hatten und ins Land eingefallen waren,
lagerten sie sich bei Sunem, während Saul, der ganz Israel aufgeboten
hatte, sein Lager auf dem (Gebirge) Gilboa aufschlug. 5Als nun Saul das
Heerlager der Philister erblickte, geriet er in Angst und erschrak im
innersten Herzen. 6Er befragte daher den HERRN, aber der HERR gab ihm
keine Antwort weder durch Träume, noch durch das
Priester-Orakel\textless sup title=``vgl. 2.Mose 28,30''\textgreater✲,
noch durch die Propheten. 7Da gab Saul seinen Dienern den Befehl: »Macht
mir eine Frau ausfindig, die sich auf Totenbeschwörung versteht: ich
will zu ihr gehen und sie befragen.« Seine Diener antworteten ihm: »In
Endor wohnt eine Frau, die Tote zu beschwören vermag.«

\hypertarget{c-saul-bei-der-totenbeschwuxf6rerin-in-endor-die-erscheinung-und-ungluxfccksprophezeiung-des-geistes-samuels}{%
\paragraph{c) Saul bei der Totenbeschwörerin in Endor; die Erscheinung
und Unglücksprophezeiung des Geistes
Samuels}\label{c-saul-bei-der-totenbeschwuxf6rerin-in-endor-die-erscheinung-und-ungluxfccksprophezeiung-des-geistes-samuels}}

8Da machte sich Saul unkenntlich, legte andere Kleider an und begab sich
mit zwei Begleitern auf den Weg. Als sie nachts bei der Frau angekommen
waren, sagte er zu ihr: »Wahrsage mir durch Totenbeschwörung und laß mir
aus der Unterwelt den erscheinen, den ich dir nennen werde.« 9Aber die
Frau antwortete ihm: »Du weißt doch selbst, was Saul getan hat: daß er
die Totenbeschwörer und Wahrsager im Lande ausgerottet hat; warum
stellst du mir also eine Falle, um mich ums Leben zu bringen?« 10Da
schwur ihr Saul beim HERRN: »So wahr der HERR lebt, es soll dich in
diesem Falle keine Schuld treffen!« 11Da fragte das Weib: »Wen soll ich
dir heraufbringen\textless sup title=``=~erscheinen
lassen''\textgreater✲?« Er antwortete: »Laß mir Samuel erscheinen!«
12Als nun die Frau Samuel anblickte, schrie sie laut auf und sagte zu
Saul: »Warum hast du mich betrogen? Du bist ja Saul!« 13Der König
erwiderte ihr: »Fürchte dich nicht! Sondern (sage): was siehst du?« Die
Frau antwortete ihm: »Ein Götterwesen\textless sup
title=``=~übermenschliches Wesen''\textgreater✲ sehe ich aus der Erde
aufsteigen.« 14Da fragte er sie: »Wie sieht es aus?« Sie antwortete:
»Ein alter Mann steigt herauf, in einen Mantel eingehüllt.« Da erkannte
Saul, daß es Samuel war; er neigte sich also mit dem Antlitz zur Erde
und bezeugte ihm seine Ehrfurcht. 15Samuel aber sprach zu Saul: »Warum
störst du mich in meiner Ruhe, daß du mich heraufkommen läßt?« Saul
erwiderte: »Ich befinde mich in großer Not; denn die Philister haben
Krieg mit mir angefangen, Gott aber hat mich verlassen und gibt mir
keine Antwort mehr weder durch die Propheten noch durch Träume; darum
habe ich dich rufen lassen, um von dir zu erfahren, was ich tun soll.«
16Samuel antwortete: »Was fragst du mich denn, da doch der HERR dich
verlassen hat und dein Feind geworden ist? 17Der HERR hat dir so getan,
wie er dir durch mich hat ankündigen lassen: der HERR hat dir das
Königtum entrissen und es einem andern, dem David, gegeben. 18Weil du
dem Befehl des HERRN nicht gehorcht und seinen lodernden Zorn an den
Amalekitern nicht vollzogen hast, darum hat der HERR dich jetzt in diese
Lage kommen lassen. 19Und der HERR wird auch die Israeliten zugleich mit
dir in die Gewalt der Philister fallen lassen: morgen wirst du mitsamt
deinen Söhnen bei mir sein, auch das Heer der Israeliten wird der HERR
in die Gewalt der Philister fallen lassen!«

\hypertarget{d-wirkung-der-prophezeiung-auf-saul}{%
\paragraph{d) Wirkung der Prophezeiung auf
Saul}\label{d-wirkung-der-prophezeiung-auf-saul}}

20Da fiel Saul voller Entsetzen seiner ganzen Länge nach zu Boden:
solchen Eindruck hatten die Worte Samuels auf ihn gemacht; auch hatte er
keine Kraft mehr in sich, weil er während des ganzen Tages und der
ganzen Nacht keine Nahrung zu sich genommen hatte. 21Als die Frau nun an
ihn herantrat und bemerkte, daß er ganz fassungslos war, sagte sie zu
ihm: »Siehe, deine Magd ist deiner Aufforderung nachgekommen, und ich
habe mein Leben aufs Spiel gesetzt und deinen Wunsch erfüllt, den du
gegen mich ausgesprochen hast. 22So schenke nun auch du deiner Magd
Gehör und laß mich dir ein wenig zu essen vorsetzen: iß nun, damit du
die nötigen Kräfte für den Rückweg besitzest.« 23Er schlug aber alles ab
und erklärte: »Ich mag nichts essen!« Als dann aber seine Begleiter und
auch die Frau in ihn drangen, gab er ihrer Aufforderung nach; er stand
vom Boden auf und setzte sich auf das Lager. 24Die Frau aber hatte ein
gemästetes Kalb im Hause, das sie in aller Eile schlachtete; weiter nahm
sie Mehl, knetete es und buk Brotkuchen daraus; 25das setzte sie Saul
und seinen Begleitern vor. Nachdem sie dann gegessen hatten, machten sie
sich noch in derselben Nacht auf den Rückweg.

\hypertarget{davids-auszug-mit-den-philistern-und-seine-ruxfcckkehr-nach-ziklag-seine-rache-an-den-amalekitern}{%
\subsubsection{14. Davids Auszug mit den Philistern und seine Rückkehr
nach Ziklag; seine Rache an den
Amalekitern}\label{davids-auszug-mit-den-philistern-und-seine-ruxfcckkehr-nach-ziklag-seine-rache-an-den-amalekitern}}

\hypertarget{a-die-heimsendung-davids-auf-druxe4ngen-der-miuxdftrauischen-philisterfuxfcrsten}{%
\paragraph{a) Die Heimsendung Davids auf Drängen der mißtrauischen
Philisterfürsten}\label{a-die-heimsendung-davids-auf-druxe4ngen-der-miuxdftrauischen-philisterfuxfcrsten}}

\hypertarget{section-28}{%
\section{29}\label{section-28}}

1Als nun die Philister ihre gesamte Heeresmacht bei Aphek gesammelt
hatten -- die Israeliten aber hatten sich an der Quelle bei Jesreel
gelagert~-- 2und die Fürsten der Philister nach ihren Hundertschaften
und Tausendschaften aufzogen und zuletzt auch David und seine Leute mit
Achis aufmarschierten, 3da riefen die Fürsten der Philister: »Was sollen
diese Hebräer da?« Achis erwiderte den Fürsten der Philister: »Das ist
ja David, der Diener des Königs Saul von Israel, der schon seit Jahr und
Tag bei mir gewesen ist, ohne daß ich an ihm, seitdem er zu mir
übergetreten ist, bis heute etwas Verdächtiges bemerkt hätte.« 4Aber die
Fürsten der Philister wurden ungehalten über ihn und sagten zu ihm:
»Schicke den Mann zurück! Er soll wieder an den Ort zurückkehren, den du
ihm angewiesen hast, und soll nicht mit uns in den Krieg ziehen, damit
er nicht in der Schlacht zum Verräter an uns wird. Denn womit könnte der
sich wohl besser in Gunst bei seinem Herrn setzen als auf Kosten der
Köpfe unserer Leute? 5Das ist ja derselbe David, dem zu Ehren sie beim
Reigentanz das Lied singen: ›Saul hat seine Tausende geschlagen, aber
David seine Zehntausende.‹«

6Da ließ Achis David rufen und sagte zu ihm: »So wahr Gott lebt! Du bist
ein Ehrenmann, und mir würde es lieb sein, wenn du den Feldzug mit mir
im Heere mitmachtest; denn ich habe an dir nichts zu tadeln gefunden,
seitdem du zu mir gekommen bist, bis auf den heutigen Tag. Aber den
übrigen Fürsten bist du nicht erwünscht. 7Daher kehre jetzt zurück und
ziehe in Frieden heim, damit du nichts tust, was den Fürsten der
Philister mißfällt!« 8David antwortete dem Achis: »Was habe ich denn
getan, und was hast du an deinem Knecht Verwerfliches gefunden, seitdem
ich in deinen Dienst getreten bin, bis zum heutigen Tage, daß ich nicht
mitziehen und kämpfen darf gegen die Feinde meines Herrn, des Königs?«
9Da antwortete Achis dem David: »Ich weiß, daß du mir so lieb bist wie
ein Engel Gottes; jedoch die (anderen) Fürsten der Philister haben
erklärt, du dürfest nicht neben ihnen in den Krieg ziehen. 10Daher mache
dich morgen früh auf den Weg, du und die Knechte deines Herrn, die mit
dir hergekommen sind; brecht also in aller Frühe auf, sobald es Tag
wird, und zieht ab!« 11So machte sich denn David mit seinen Leuten am
nächsten Morgen in aller Frühe auf den Weg, um ins Land der Philister
zurückzukehren; die Philister aber zogen hinauf nach Jesreel.

\hypertarget{b-david-ruxe4cht-die-gewalttat-der-amalekiter}{%
\paragraph{b) David rächt die Gewalttat der
Amalekiter}\label{b-david-ruxe4cht-die-gewalttat-der-amalekiter}}

\hypertarget{aa-david-findet-ziklag-von-den-amalekitern-verwuxfcstet-seine-bestuxfcrzung-und-ermutigung}{%
\subparagraph{aa) David findet Ziklag von den Amalekitern verwüstet;
seine Bestürzung und
Ermutigung}\label{aa-david-findet-ziklag-von-den-amalekitern-verwuxfcstet-seine-bestuxfcrzung-und-ermutigung}}

\hypertarget{section-29}{%
\section{30}\label{section-29}}

1Als nun David mit seinen Leuten am dritten Tage in Ziklag ankam, hatten
die Amalekiter einen Einfall in das Südland und in Ziklag gemacht und
hatten Ziklag geplündert und niedergebrannt. 2Die Frauen und alles, was
im Orte anwesend war, klein und groß, hatten sie gefangengenommen, ohne
jedoch jemand zu töten, hatten sie dann weggeführt und waren ihres Weges
gezogen. 3Als nun David und seine Leute zu der Stadt zurückkamen und sie
niedergebrannt und ihre Frauen, Söhne und Töchter in Gefangenschaft
weggeführt fanden, 4da erhoben David und seine Leute ein lautes
Wehgeschrei und weinten, bis sie keine Kraft mehr zum Weinen hatten.
5Auch die beiden Frauen Davids waren gefangen weggeführt worden, Ahinoam
aus Jesreel und Abigail, die Witwe Nabals, aus Karmel. 6David aber
geriet persönlich in große Gefahr, weil seine Leute schon daran dachten,
ihn zu steinigen; denn sie waren alle über den Verlust ihrer Söhne und
Töchter ganz verzweifelt.

David aber gewann neue Kraft durch sein Vertrauen auf den HERRN, seinen
Gott, 7und befahl dem Priester Abjathar, dem Sohne Ahimelechs: »Bringe
mir das Priesterkleid her!« Als nun Abjathar das Priesterkleid zu David
brachte, 8richtete David die Frage an den HERRN: »Soll ich dieser
Räuberschar nachsetzen? Werde ich sie einholen?« Da erhielt er die
Antwort: »Ja, verfolge sie! Du wirst sie sicher einholen und (die
Gefangenen) befreien.« 9Da machte sich David mit den sechshundert Mann,
die er bei sich hatte, auf den Weg, und sie kamen an den Bach Besor (wo
sie zweihundert Mann zurückließen). 10David aber setzte die Verfolgung
mit vierhundert Mann fort, während zweihundert Mann, die zu ermüdet
waren, um über den Bach Besor zu gehen, zurückbleiben mußten.

\hypertarget{bb-davids-verfolgung-und-vernichtung-der-amalekitischen-ruxe4uberschar}{%
\subparagraph{bb) Davids Verfolgung und Vernichtung der amalekitischen
Räuberschar}\label{bb-davids-verfolgung-und-vernichtung-der-amalekitischen-ruxe4uberschar}}

11Da fanden sie einen Ägypter auf freiem Felde, den brachten sie zu
David; und als sie ihm Brot zu essen und Wasser zu trinken gegeben 12und
ihm auch ein Stück Feigenkuchen und zwei Rosinentrauben zu essen gegeben
hatten, kam er wieder zu sich; denn er hatte seit drei Tagen und drei
Nächten nichts gegessen und nichts getrunken. 13David fragte ihn nun:
»Wem gehörst du, und woher bist du?« Er antwortete: »Ich bin ein
ägyptischer Bursche, der Sklave eines Amalekiters; mein Herr hat mich
hier liegen lassen, weil ich heute vor drei Tagen krank geworden war.
14Wir hatten einen Einfall gemacht ins Südland der Kreter und ins Gebiet
von Juda und ins Südland von Kaleb und haben Ziklag niedergebrannt.«
15Da fragte ihn David: »Willst du mich zu dieser Räuberschar
hinabführen?« Er erwiderte: »Schwöre mir bei Gott, daß du mich nicht
töten und mich nicht meinem Herrn ausliefern willst, so will ich dich zu
dieser Horde hinabführen.«

16Als er ihn nun hinabführte, hatten (die Amalekiter) sich weithin über
die ganze Gegend zerstreut, aßen und tranken und feierten ein
Freudenfest wegen all der großen Beute, die sie im Lande der Philister
und im Lande Juda gewonnen hatten. 17Da richtete David (am folgenden
Tage) ein Blutbad unter ihnen an vom frühen Morgen bis zum Abend, und
keiner von ihnen entkam außer vierhundert jungen Leuten, welche die
Kamele bestiegen hatten und entflohen. 18So fiel dem David alles in die
Hände, was die Amalekiter geraubt hatten; auch seine beiden Frauen
gewann er wieder, 19so daß von ihnen nicht das Geringste vermißt wurde,
weder Söhne noch Töchter, auch nichts von der Beute; überhaupt alles,
was sie mit sich genommen hatten, brachte David zurück. 20David nahm
dann alles Kleinvieh und die Rinder; die trieben sie vor der andern
Herde her und riefen: »Das ist Davids Beute!«

\hypertarget{c-david-veranlauxdft-seine-leute-zu-anstuxe4ndigem-verfahren-gegen-ihre-kameraden}{%
\paragraph{c) David veranlaßt seine Leute zu anständigem Verfahren gegen
ihre
Kameraden}\label{c-david-veranlauxdft-seine-leute-zu-anstuxe4ndigem-verfahren-gegen-ihre-kameraden}}

21Als David dann zu den zweihundert Mann zurückkam, die zu ermattet
gewesen waren, um mit David weiterzuziehen, und die man deshalb am Bache
Besor zurückgelassen hatte, kamen diese ihm und seinen Leuten
entgegengezogen; David ging auf die Leute zu und begrüßte sie
freundlich. 22Da ließen alle bösen und nichtswürdigen Leute unter der
Mannschaft, die mit David gezogen waren, sich dahin vernehmen: »Weil sie
nicht mit uns gezogen sind, wollen wir ihnen auch von der Beute, die wir
wiedergewonnen haben, nichts abgeben als nur einem jeden seine Frau und
seine Kinder; die mögen sie hinnehmen und dann ihres Weges ziehen!«
23Aber David sagte: »Verfahrt nicht so, meine Brüder, mit dem, was der
HERR uns hat zuteil werden lassen! Er hat uns ja beschützt und die
Räuberbande, die bei uns eingedrungen war, in unsere Hand fallen lassen:
24wer könnte da in dieser Sache eurer Ansicht beitreten? Nein, der
Anteil dessen, der beim Gepäck Wache gehalten hat, soll ebenso groß sein
wie der Anteil dessen, der in den Kampf gezogen ist: gleichen Anteil
sollen sie erhalten!« 25Und dabei ist es seit jenem Tage in der
Folgezeit geblieben; man hat das zu einem feststehenden Grundsatz in
Israel gemacht bis auf den heutigen Tag.

\hypertarget{d-david-sendet-geschenke-an-die-uxe4ltesten-zahlreicher-stuxe4dte-von-juda}{%
\paragraph{d) David sendet Geschenke an die Ältesten zahlreicher Städte
von
Juda}\label{d-david-sendet-geschenke-an-die-uxe4ltesten-zahlreicher-stuxe4dte-von-juda}}

26Als David dann nach Ziklag zurückkam, sandte er Teile der Beute an die
ihm befreundeten Ältesten von Juda und ließ ihnen dabei sagen: »Hier
habt ihr eine Begrüßungsgabe aus der Beute von den Feinden des HERRN!«
27Solche Geschenke sandte er an die Ältesten von Bethel und von Ramath
im Südland sowie an die von Jatthir, 28von Aroer, von Siphmoth, von
Esthemoa, 29von Rachal und von den Ortschaften der Jerahmeeliter und der
Keniter; 30ferner an die Ältesten von Horma, von Bor-Asan, von Athach,
31von Hebron und an alle Ortschaften, wo David mit seinen Leuten
umhergezogen war.

\hypertarget{israels-niederlage-und-der-ungluxfccksschlag-uxfcber-saul-und-sein-haus}{%
\subsubsection{15. Israels Niederlage und der Unglücksschlag über Saul
und sein
Haus}\label{israels-niederlage-und-der-ungluxfccksschlag-uxfcber-saul-und-sein-haus}}

\hypertarget{a-israels-besiegung-durch-die-philister-auf-dem-gebirge-gilboa-tod-sauls-und-seiner-drei-suxf6hne}{%
\paragraph{a) Israels Besiegung durch die Philister auf dem Gebirge
Gilboa; Tod Sauls und seiner drei
Söhne}\label{a-israels-besiegung-durch-die-philister-auf-dem-gebirge-gilboa-tod-sauls-und-seiner-drei-suxf6hne}}

\hypertarget{section-30}{%
\section{31}\label{section-30}}

1Als es aber zwischen den Philistern und Israeliten zur Schlacht kam,
wurden die Israeliten von den Philistern in die Flucht geschlagen, und
viele Erschlagene lagen auf dem Gebirge Gilboa umher. 2Die Philister
setzten Saul und seinen Söhnen hart zu und erschlugen Sauls Söhne
Jonathan, Abinadab und Malchisua. 3Als dann ein wilder Kampf um Saul her
entstand und die Bogenschützen ihn ausfindig gemacht hatten, da ward er
von den Bogenschützen in den Unterleib getroffen. 4Da befahl Saul seinem
Waffenträger: »Ziehe dein Schwert und durchbohre mich damit, auf daß
nicht diese Heiden kommen und ihren Mutwillen an mir auslassen!« Aber
sein Waffenträger weigerte sich, weil er sich zu sehr fürchtete. Da nahm
Saul das Schwert und stürzte sich hinein. 5Als nun sein Waffenträger
sah, daß Saul tot war, stürzte er sich gleichfalls in sein Schwert und
starb neben ihm. 6So fanden Saul, seine drei Söhne und sein Waffenträger
an jenem Tage zusammen ihren Tod. 7Als aber die Israeliten, die in den
Städten der Ebene (Jesreel) und in den Städten am Jordan wohnten, sahen,
daß die Israeliten geflohen und daß Saul samt seinen Söhnen gefallen
war, verließen sie ihre Städte und ergriffen die Flucht; da kamen die
Philister und setzten sich darin fest.

\hypertarget{b-das-schicksal-der-leichname-sauls-und-seiner-suxf6hne}{%
\paragraph{b) Das Schicksal der Leichname Sauls und seiner
Söhne}\label{b-das-schicksal-der-leichname-sauls-und-seiner-suxf6hne}}

8Als dann am nächsten Tage die Philister kamen, um die Gefallenen
auszuplündern, fanden sie die Leichen Sauls und seiner drei Söhne auf
dem Gebirge Gilboa liegen. 9Da hieben sie ihm den Kopf ab, zogen ihm
seine Rüstung aus und sandten (Boten) in allen Teilen des
Philisterlandes umher, um die Siegesbotschaft in ihren Götzentempeln und
unter dem Volke zu verkünden. 10Seine Waffen✲ legten sie im Tempel der
Astarte nieder, seine Leiche aber hängten sie an der Mauer von Beth-San
auf. 11Als aber die Einwohner von Jabes in Gilead erfuhren, was die
Philister an Saul verübt hatten, 12machten sich alle streitbaren Männer
auf, marschierten die ganze Nacht hindurch und nahmen den Leichnam Sauls
und die Leichen seiner Söhne von der Mauer von Beth-San herab, kehrten
mit ihnen nach Jabes zurück und verbrannten sie dort. 13Hierauf nahmen
sie ihre Gebeine, begruben sie unter der Tamariske in Jabes und hielten
sieben Tage lang ein Fasten.
