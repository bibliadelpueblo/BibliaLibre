\hypertarget{die-heilsbotschaft-nach-markus}{%
\section{DIE HEILSBOTSCHAFT NACH
MARKUS}\label{die-heilsbotschaft-nach-markus}}

\hypertarget{einleitung-die-vorbereitung}{%
\subsection{Einleitung: Die
Vorbereitung}\label{einleitung-die-vorbereitung}}

\hypertarget{auftreten-und-wirksamkeit-johannes-des-tuxe4ufers}{%
\subsubsection{1. Auftreten und Wirksamkeit Johannes des
Täufers}\label{auftreten-und-wirksamkeit-johannes-des-tuxe4ufers}}

\hypertarget{section}{%
\section{1}\label{section}}

\bibleverse{1} Die Heilsbotschaft von Jesus Christus, dem Sohne Gottes,
hat folgenden Anfang: \bibleverse{2} Wie beim Propheten Jesaja
geschrieben steht: »Siehe\textless sup title=``oder: Wisset
wohl''\textgreater✲, ich sende meinen Boten vor dir her, der dir den Weg
herrichten soll«\textless sup title=``Mal 3,1''\textgreater✲;
\bibleverse{3} »eine Stimme ruft in der Wüste: ›Bereitet den Weg des
Herrn, macht gerade\textless sup title=``oder: ebnet''\textgreater✲
seine Pfade!‹«\textless sup title=``Jes 40,3''\textgreater✲ --:
\bibleverse{4} so trat Johannes der Täufer in der Wüste auf, indem er
eine Taufe der Buße\textless sup title=``oder: Sinnesänderung; vgl. Mt
3,2''\textgreater✲ predigte zur Vergebung der Sünden. \bibleverse{5} Da
zog das ganze jüdische Land und auch alle Einwohner Jerusalems zu ihm
hinaus und ließen sich von ihm im Jordanfluß taufen, indem sie ihre
Sünden bekannten. \bibleverse{6} Johannes trug aber ein
Fell\textless sup title=``oder: Gewand''\textgreater✲ von Kamelhaaren
und einen ledernen Gurt um seine Hüften; er nährte sich von Heuschrecken
und wildem Honig, \bibleverse{7} und seine Predigt lautete: »Nach mir
kommt der, welcher stärker ist als ich, für den ich nicht gut genug bin,
ihm gebückt seine Schuhriemen aufzulösen. \bibleverse{8} Ich habe euch
(nur) mit Wasser getauft, er aber wird euch mit heiligem Geiste taufen.«

\hypertarget{jesu-taufe-und-versuchung}{%
\subsubsection{2. Jesu Taufe und
Versuchung}\label{jesu-taufe-und-versuchung}}

\bibleverse{9} In jenen Tagen begab es sich nun auch, daß Jesus aus
Nazareth in Galiläa kam und sich von Johannes im Jordan taufen ließ.
\bibleverse{10} Da, als er gerade aus dem Wasser heraufstieg, sah er
(Johannes oder Jesus) den Himmel sich spalten\textless sup
title=``=~sich auftun''\textgreater✲ und den Geist wie eine Taube auf
ihn\textless sup title=``oder: auf sich''\textgreater✲ herabschweben;
\bibleverse{11} und eine Stimme erscholl aus den Himmeln: »Du bist mein
geliebter Sohn; an dir habe ich Wohlgefallen gefunden!«~--
\bibleverse{12} Und sogleich trieb der Geist ihn in die Wüste hinaus;
\bibleverse{13} und er war vierzig Tage lang in der Wüste und wurde vom
Satan versucht; er weilte dort bei den wilden Tieren, und die Engel
leisteten ihm Dienste.

\hypertarget{i.-die-anfuxe4nge-der-wirksamkeit-jesu-in-galiluxe4a-114-45}{%
\subsection{I. Die Anfänge der Wirksamkeit Jesu in Galiläa
(1,14-45)}\label{i.-die-anfuxe4nge-der-wirksamkeit-jesu-in-galiluxe4a-114-45}}

\hypertarget{erstes-auftreten-jesu-in-galiluxe4a}{%
\subsubsection{1. Erstes Auftreten Jesu in
Galiläa}\label{erstes-auftreten-jesu-in-galiluxe4a}}

\bibleverse{14} Nachdem dann Johannes ins Gefängnis gesetzt war, begab
Jesus sich nach Galiläa und verkündete dort die Heilsbotschaft Gottes
\bibleverse{15} mit den Worten: »Die Zeit ist erfüllt und das Reich
Gottes nahe herbeigekommen; tut Buße\textless sup title=``vgl. Mt
3,2''\textgreater✲ und glaubt an die Heilsbotschaft!«

\hypertarget{berufung-der-ersten-vier-juxfcnger-der-beiden-fischerbruxfcderpaare}{%
\subsubsection{2. Berufung der ersten vier Jünger (der beiden
Fischerbrüderpaare)}\label{berufung-der-ersten-vier-juxfcnger-der-beiden-fischerbruxfcderpaare}}

\bibleverse{16} Als Jesus nun (eines Tages) am Ufer des Galiläischen
Sees hinging, sah er Simon und Andreas, den Bruder Simons, die Netze im
See auswerfen; sie waren nämlich Fischer. \bibleverse{17} Da sagte Jesus
zu ihnen: »Kommt, folgt mir nach, ich will euch zu Menschenfischern
machen!« \bibleverse{18} Sogleich ließen sie ihre Netze liegen und
folgten ihm nach. \bibleverse{19} Als er dann ein wenig weitergegangen
war, sah er Jakobus, den Sohn des Zebedäus, und seinen Bruder Johannes,
die, ebenfalls im Boot, ihre Netze instand setzten. \bibleverse{20}
Sogleich berief er sie; da ließen sie ihren Vater Zebedäus mit den
Lohnknechten\textless sup title=``oder: Tagelöhnern''\textgreater✲ im
Boot und folgten ihm nach.

\hypertarget{jesu-erste-predigt-und-heilung-eines-besessenen-in-der-synagoge-zu-kapernaum}{%
\subsubsection{3. Jesu erste Predigt und Heilung eines Besessenen in der
Synagoge zu
Kapernaum}\label{jesu-erste-predigt-und-heilung-eines-besessenen-in-der-synagoge-zu-kapernaum}}

\bibleverse{21} Sie begaben sich dann nach Kapernaum hinein; und
sogleich am (nächsten) Sabbat ging er in die Synagoge und lehrte.
\bibleverse{22} Da waren sie über seine Lehre betroffen; denn er lehrte
sie wie einer, der Vollmacht\textless sup title=``=~göttlichen
Beruf''\textgreater✲ hat, ganz anders als die
Schriftgelehrten\textless sup title=``Mt 7,29''\textgreater✲.
\bibleverse{23} Nun war da gerade in ihrer Synagoge ein Mann mit einem
unreinen Geist behaftet; der schrie auf \bibleverse{24} und rief: »Was
willst du von uns, Jesus von Nazareth? Du bist gekommen, um uns zu
verderben! Ich weiß von dir, wer du bist: der Heilige Gottes!«
\bibleverse{25} Jesus bedrohte ihn mit den Worten: »Verstumme und fahre
aus von ihm!« \bibleverse{26} Da riß der unreine Geist den Mann (in
Krämpfen) hin und her und fuhr dann mit einem lauten Schrei von ihm aus.
\bibleverse{27} Da gerieten sie allesamt in Staunen, so daß sie sich
miteinander besprachen und sich befragten: »Was ist dies? Eine neue
Lehre mit (göttlicher) Vollmacht! Auch den unreinen Geistern gebietet
er, und sie gehorchen ihm!« \bibleverse{28} Und der Ruf von ihm
verbreitete sich alsbald überall in der ganzen umliegenden Landschaft
Galiläa.

\hypertarget{in-simons-hause-heilung-der-schwiegermutter-simons-und-anderer-kranken-in-kapernaum}{%
\subsubsection{4. In Simons Hause: Heilung der Schwiegermutter Simons
und anderer Kranken in
Kapernaum}\label{in-simons-hause-heilung-der-schwiegermutter-simons-und-anderer-kranken-in-kapernaum}}

\bibleverse{29} Sobald sie dann die Synagoge verlassen hatten, begaben
sie sich in Begleitung des Jakobus und Johannes in das Haus des Simon
und Andreas. \bibleverse{30} Die Schwiegermutter Simons aber lag dort
fieberkrank zu Bett, was man ihm sogleich von ihr mitteilte.
\bibleverse{31} Er trat nun zu ihr, faßte sie bei der Hand und richtete
sie auf; da wich das Fieber sogleich von ihr, und sie wartete ihnen (bei
der Mahlzeit) auf.

\bibleverse{32} Als es dann Abend geworden und die Sonne untergegangen
war, brachte man alle Kranken und Besessenen zu ihm, \bibleverse{33} und
die ganze Stadt war an der Tür versammelt. \bibleverse{34} Und er heilte
viele, die an Krankheiten aller Art litten, und trieb viele böse Geister
aus, ließ dabei aber die Geister nicht reden, weil sie ihn
kannten\textless sup title=``oder: ließ sie nicht aussprechen, daß sie
ihn kannten''\textgreater✲.

\hypertarget{jesus-verluxe4uxdft-kapernaum-seine-wanderpredigt-und-heiltuxe4tigkeit-in-galiluxe4a}{%
\subsubsection{5. Jesus verläßt Kapernaum; seine Wanderpredigt und
Heiltätigkeit in
Galiläa}\label{jesus-verluxe4uxdft-kapernaum-seine-wanderpredigt-und-heiltuxe4tigkeit-in-galiluxe4a}}

\bibleverse{35} Frühmorgens aber, als es noch ganz dunkel war, stand er
auf, verließ das Haus und begab sich an einen einsamen Ort, wo er
betete. \bibleverse{36} Simon jedoch und seine Genossen eilten ihm nach,
\bibleverse{37} und als sie ihn gefunden hatten, sagten sie zu ihm:
»Alle suchen dich!« \bibleverse{38} Er aber antwortete ihnen: »Wir
wollen anderswohin in die benachbarten Ortschaften gehen, damit ich auch
dort die Botschaft ausrichte; denn dazu bin ich ausgezogen.«
\bibleverse{39} So wanderte er denn in ganz Galiläa umher, indem er in
ihren\textless sup title=``=~den dortigen''\textgreater✲ Synagogen
predigte und die bösen Geister austrieb.

\hypertarget{jesus-heilt-einen-aussuxe4tzigen-und-entweicht-in-die-einsamkeit}{%
\subsubsection{6. Jesus heilt einen Aussätzigen und entweicht in die
Einsamkeit}\label{jesus-heilt-einen-aussuxe4tzigen-und-entweicht-in-die-einsamkeit}}

\bibleverse{40} Da kam ein Aussätziger zu ihm, fiel vor ihm auf die Knie
nieder und bat ihn flehentlich mit den Worten: »Wenn du willst, kannst
du mich reinigen.« \bibleverse{41} Jesus hatte Mitleid mit ihm, streckte
seine Hand aus, faßte ihn an und sagte zu ihm: »Ich will's: werde rein!«
\bibleverse{42} Da verschwand der Aussatz sogleich von ihm, und er wurde
rein. \bibleverse{43} Jesus aber gab ihm strenge Weisung, hieß ihn auf
der Stelle weggehen \bibleverse{44} und sagte zu ihm: »Hüte dich,
jemandem etwas davon zu sagen! Gehe vielmehr hin, zeige dich dem
Priester und bringe für deine Reinigung das Opfer dar, das
Mose\textless sup title=``3.Mose 13,49; 14,10''\textgreater✲ geboten
hat, zum Zeugnis✲ für sie!« \bibleverse{45} Als jener aber weggegangen
war, fing er an, vielfach\textless sup title=``oder:
eifrig''\textgreater✲ davon zu erzählen und die Sache überall bekannt zu
machen, so daß Jesus nicht mehr offen in eine Stadt hineingehen konnte,
sondern sich draußen an einsamen Orten aufhalten mußte; und doch kamen
die Leute von allen Seiten her zu ihm.

\hypertarget{ii.-zusammenstuxf6uxdfe-mit-den-fuxfchrern-des-volkes-schriftgelehrten-und-pharisuxe4ern-21-36}{%
\subsection{II. Zusammenstöße mit den Führern des Volkes
(Schriftgelehrten und Pharisäern)
(2,1-3,6)}\label{ii.-zusammenstuxf6uxdfe-mit-den-fuxfchrern-des-volkes-schriftgelehrten-und-pharisuxe4ern-21-36}}

\hypertarget{heilung-eines-geluxe4hmten-in-kapernaum-jesus-vergibt-suxfcnden}{%
\subsubsection{1. Heilung eines Gelähmten in Kapernaum; Jesus vergibt
Sünden}\label{heilung-eines-geluxe4hmten-in-kapernaum-jesus-vergibt-suxfcnden}}

\hypertarget{section-1}{%
\section{2}\label{section-1}}

\bibleverse{1} Als er dann nach einiger Zeit wieder nach Kapernaum
heimgekommen war und die Kunde sich verbreitet hatte, daß er im Hause✲
sei, \bibleverse{2} da versammelten sich alsbald so viele Leute, daß
selbst der Platz vor der Tür für sie nicht mehr ausreichte; und er
verkündigte ihnen das Wort\textless sup title=``=~die
Heilsbotschaft''\textgreater✲. \bibleverse{3} Da kamen Leute zu ihm, die
einen Gelähmten brachten, der von vier Männern getragen wurde.
\bibleverse{4} Weil sie nun mit ihm\textless sup title=``d.h. dem
Kranken''\textgreater✲ wegen der Volksmenge nicht an ihn herankommen
konnten, deckten sie über der Stelle, wo Jesus sich befand, das Hausdach
ab und ließen das Tragbett, auf dem der Gelähmte lag, durch eine
Öffnung, die sie hindurchgebrochen hatten, hinab. \bibleverse{5} Als
Jesus nun ihren Glauben erkannte, sagte er zu dem Gelähmten: »Mein Sohn,
deine Sünden sind (dir) vergeben!« \bibleverse{6} Es saßen dort aber
einige Schriftgelehrte, die machten sich in ihrem Herzen Gedanken:
\bibleverse{7} »Wie kann dieser so reden? Er lästert ja Gott! Wer kann
Sünden vergeben außer Gott allein?« \bibleverse{8} Da nun Jesus in
seinem Geiste sogleich erkannte, daß sie so bei sich dachten, sagte er
zu ihnen: »Warum denkt ihr so in euren Herzen? \bibleverse{9} Was ist
leichter, zu dem Gelähmten zu sagen: ›Deine Sünden sind (dir) vergeben‹,
oder zu sagen: ›Stehe auf, nimm dein Tragbett und gehe umher‹?
\bibleverse{10} Damit ihr aber wißt\textless sup title=``=~erkennen
lernt''\textgreater✲, daß der Menschensohn Vollmacht hat, Sünden auf
Erden zu vergeben« -- hierauf sagte er zu dem Gelähmten: \bibleverse{11}
»Ich sage dir: Stehe auf, nimm dein Bett und gehe heim in dein Haus!«
\bibleverse{12} Da stand er auf, nahm sogleich das Tragbett und ging vor
aller Augen hinaus, so daß alle vor Staunen außer sich gerieten und Gott
priesen, indem sie erklärten: »So etwas haben wir noch nie gesehen!«

\hypertarget{berufung-des-zuxf6llners-levi-matthuxe4us-jesus-als-tischgenosse-der-zuxf6llner-und-suxfcnder}{%
\subsubsection{2. Berufung des Zöllners Levi (=~Matthäus); Jesus als
Tischgenosse der Zöllner und
Sünder}\label{berufung-des-zuxf6llners-levi-matthuxe4us-jesus-als-tischgenosse-der-zuxf6llner-und-suxfcnder}}

\bibleverse{13} Er ging hierauf wieder hinaus an den See, und die ganze
Volksmenge kam zu ihm, und er lehrte sie. \bibleverse{14} Im
Vorübergehen sah er dann Levi, den Sohn des Alphäus, an der Zollstätte
sitzen und sagte zu ihm: »Folge mir nach!« Da stand er auf und folgte
ihm nach. \bibleverse{15} Nun begab es sich, als Jesus in Levis Hause zu
Tische saß, daß viele Zöllner und Sünder mit Jesus und seinen Jüngern am
Mahl teilnahmen; denn es waren ihrer viele, die ihm (beständig)
nachfolgten. \bibleverse{16} Als nun die Schriftgelehrten, die zu den
Pharisäern gehörten, ihn mit den Zöllnern und Sündern zusammen essen
sahen, sagten sie zu seinen Jüngern: »Wie (ist's nur möglich), daß er
mit den Zöllnern und Sündern ißt und trinkt?« \bibleverse{17} Als Jesus
das hörte, sagte er zu ihnen: »Die Gesunden haben keinen Arzt nötig,
wohl aber die Kranken. Ich bin nicht gekommen, Gerechte zu berufen,
sondern Sünder.«

\hypertarget{die-fastenfrage-der-johannesjuxfcnger-und-pharisuxe4er}{%
\subsubsection{3. Die Fastenfrage der Johannesjünger und
Pharisäer}\label{die-fastenfrage-der-johannesjuxfcnger-und-pharisuxe4er}}

\bibleverse{18} Die Jünger des Johannes und die Pharisäer fasteten
gerade\textless sup title=``d.h. hatten ihre Fastenzeit''\textgreater✲.
Da kamen Leute zu Jesus mit der Frage: »Warum fasten die Jünger des
Johannes und die Schüler der Pharisäer, während deine Jünger es nicht
tun?« \bibleverse{19} Jesus antwortete ihnen: »Können etwa die
Hochzeitsgäste fasten, solange der Bräutigam noch bei ihnen weilt? Nein,
solange sie den Bräutigam noch bei sich haben, können sie nicht fasten.
\bibleverse{20} Es werden aber Tage kommen, wo der Bräutigam ihnen
genommen sein wird; dann, an jenem Tage, werden sie fasten.~--
\bibleverse{21} Niemand setzt ein Stück von ungewalktem
Tuch\textless sup title=``=~neuen Stoff''\textgreater✲ auf ein altes
Kleid; sonst reißt der eingesetzte neue Fleck\textless sup title=``=~das
neue Flickstück''\textgreater✲ von dem alten Kleide wieder ab, und es
entsteht ein noch schlimmerer Riß. \bibleverse{22} Auch füllt niemand
neuen✲ Wein in alte Schläuche; sonst sprengt der Wein die Schläuche, und
der Wein geht samt den Schläuchen verloren. Nein, neuer✲ Wein gehört in
neue Schläuche.«

\hypertarget{das-uxe4hrenraufen-der-juxfcnger-am-sabbat-der-erste-streit-jesu-mit-den-pharisuxe4ern-uxfcber-die-sabbatheiligung}{%
\subsubsection{4. Das Ährenraufen der Jünger am Sabbat; der erste Streit
Jesu mit den Pharisäern über die
Sabbatheiligung}\label{das-uxe4hrenraufen-der-juxfcnger-am-sabbat-der-erste-streit-jesu-mit-den-pharisuxe4ern-uxfcber-die-sabbatheiligung}}

\bibleverse{23} (Einst) begab es sich, daß Jesus am Sabbat durch die
Kornfelder wanderte, und seine Jünger begannen im Dahingehen Ähren
abzupflücken. \bibleverse{24} Da sagten die Pharisäer zu ihm: »Sieh, was
sie da am Sabbat Unerlaubtes tun!« \bibleverse{25} Er antwortete ihnen:
»Habt ihr noch niemals gelesen\textless sup title=``1.Sam
21,2-7''\textgreater✲, was David getan hat, als er Mangel litt und ihn
samt seinen Begleitern hungerte? \bibleverse{26} Wie er da ins
Gotteshaus ging zur Zeit des Hohenpriesters Abjathar und die Schaubrote
aß, die doch niemand außer den Priestern essen darf\textless sup
title=``3.Mose 24,5-9''\textgreater✲, und wie er auch seinen Begleitern
davon gab?« \bibleverse{27} Dann fuhr er fort: »Der Sabbat ist um des
Menschen willen da und nicht der Mensch um des Sabbats willen;
\bibleverse{28} somit ist der Menschensohn Herr auch über den Sabbat.«

\hypertarget{heilung-des-mannes-mit-dem-geluxe4hmten-arm-am-sabbat-der-zweite-streit-uxfcber-die-sabbatheiligung}{%
\subsubsection{5. Heilung des Mannes mit dem gelähmten Arm am Sabbat;
der zweite Streit über die
Sabbatheiligung}\label{heilung-des-mannes-mit-dem-geluxe4hmten-arm-am-sabbat-der-zweite-streit-uxfcber-die-sabbatheiligung}}

\hypertarget{section-2}{%
\section{3}\label{section-2}}

\bibleverse{1} Als er dann wieder einmal in eine Synagoge gegangen war,
befand sich dort ein Mann, der einen gelähmten\textless sup title=``eig.
verdorrten''\textgreater✲ Arm hatte; \bibleverse{2} und sie lauerten ihm
auf, ob er ihn am Sabbat heilen würde, um dann eine Anklage gegen ihn zu
erheben. \bibleverse{3} Da sagte er zu dem Manne, der den gelähmten Arm
hatte: »Stehe auf (und tritt vor) in die Mitte!« \bibleverse{4} Dann
fragte er sie: »Darf man am Sabbat Gutes tun, oder (soll man) Böses tun?
Darf man ein Leben\textless sup title=``eig. eine Seele''\textgreater✲
retten oder soll man es töten\textless sup title=``=~zugrunde gehen
lassen''\textgreater✲?« Sie aber schwiegen. \bibleverse{5} Da blickte er
sie ringsum voll Zorn an, betrübt über die Verstocktheit ihres Herzens,
und sagte zu dem Manne: »Strecke deinen Arm aus!« Er streckte ihn aus,
und sein Arm wurde wiederhergestellt. \bibleverse{6} Da gingen die
Pharisäer sogleich hinaus und berieten sich mit den Anhängern des
Herodes\textless sup title=``Mt 22,16''\textgreater✲ über ihn, wie sie
ihn umbringen\textless sup title=``oder: unschädlich
machen''\textgreater✲ könnten.

\hypertarget{iii.-die-grouxdfen-zeichen-und-worte-jesu-in-galiluxe4a-und-auuxdferhalb-galiluxe4as-37-826}{%
\subsection{III. Die großen Zeichen und Worte Jesu in Galiläa und
außerhalb Galiläas
(3,7-8,26)}\label{iii.-die-grouxdfen-zeichen-und-worte-jesu-in-galiluxe4a-und-auuxdferhalb-galiluxe4as-37-826}}

\hypertarget{zulauf-des-volkes-viele-heilungen-am-see}{%
\subsubsection{1. Zulauf des Volkes; viele Heilungen am
See}\label{zulauf-des-volkes-viele-heilungen-am-see}}

\bibleverse{7} Jesus zog sich dann mit seinen Jüngern an den See zurück,
und eine große Volksmenge aus Galiläa begleitete ihn; auch aus Judäa
\bibleverse{8} und Jerusalem, aus Idumäa und dem Ostjordanlande und aus
der Gegend von Tyrus und Sidon kamen die Leute auf die Kunde von allen
seinen Taten in großen Scharen zu ihm. \bibleverse{9} So befahl er denn
seinen Jüngern, ein Boot solle wegen der Volksmenge beständig für ihn
bereitgehalten werden, damit man ihn nicht zu arg dränge;
\bibleverse{10} denn weil er viele heilte, so suchten alle, die ein
Leiden hatten, mit Gewalt an ihn heranzukommen, um ihn anrühren zu
können; \bibleverse{11} und sooft die unreinen Geister ihn erblickten,
warfen sie sich vor ihm nieder und riefen laut: »Du bist der Sohn
Gottes!« \bibleverse{12} Er gab ihnen dann allemal die strenge Weisung,
sie sollten ihn nicht (als Messias) offenbar\textless sup
title=``=~öffentlich bekannt''\textgreater✲ machen.

\hypertarget{berufung-und-namen-der-zwuxf6lf-juxfcnger}{%
\subsubsection{2. Berufung und Namen der zwölf
Jünger}\label{berufung-und-namen-der-zwuxf6lf-juxfcnger}}

\bibleverse{13} Da stieg er auf den Berg hinauf und rief die zu sich,
die er selbst (bei sich zu haben) wünschte; und sie traten zu ihm heran.
\bibleverse{14} So bestellte er denn zwölf {[}die er auch
Apostel\textless sup title=``d.h. Sendboten''\textgreater✲ nannte{]};
diese sollten beständig bei ihm sein, und er wollte sie auch aussenden,
damit sie (die Heilsbotschaft) verkündigten; \bibleverse{15} sie sollten
auch Vollmacht zur Austreibung der bösen Geister haben. \bibleverse{16}
So setzte er die Zwölf ein und legte dem Simon den Namen
Petrus\textless sup title=``d.h. Fels, Felsenmann''\textgreater✲ bei;
\bibleverse{17} ferner Jakobus, den Sohn des Zebedäus, und Johannes, den
Bruder des Jakobus, denen er den Namen Boanerges, das heißt
›Donnersöhne‹, beilegte; \bibleverse{18} ferner Andreas, Philippus,
Bartholomäus, Matthäus, Thomas, Jakobus, den Sohn des Alphäus, Thaddäus,
Simon den Kananäer\textless sup title=``vgl. Mt 10,4''\textgreater✲
\bibleverse{19} und Judas Iskariot, denselben, der ihn (später)
überantwortet\textless sup title=``oder: verraten''\textgreater✲ hat.

\hypertarget{auseinandersetzungen}{%
\subsubsection{3. Auseinandersetzungen}\label{auseinandersetzungen}}

\hypertarget{a-das-anwachsen-der-bewegung}{%
\paragraph{a) Das Anwachsen der
Bewegung}\label{a-das-anwachsen-der-bewegung}}

\bibleverse{20} Er ging dann in ein Haus\textless sup title=``oder: er
kam nach Hause''\textgreater✲; da sammelte sich wieder eine solche
Volksmenge an, daß sie nicht einmal Zeit zum Essen hatten.
\bibleverse{21} Als seine Angehörigen Kunde davon erhielten, machten sie
sich auf den Weg, um sich seiner zu bemächtigen; denn sie
sagten\textless sup title=``oder: waren der Meinung''\textgreater✲, er
sei von Sinnen gekommen.

\hypertarget{b-jesus-verteidigt-sich-gegen-die-beelzebul-luxe4sterung-der-schriftgelehrten.-von-der-suxfcnde-gegen-den-heiligen-geist}{%
\paragraph{b) Jesus verteidigt sich gegen die Beelzebul-Lästerung der
Schriftgelehrten. Von der Sünde gegen den heiligen
Geist}\label{b-jesus-verteidigt-sich-gegen-die-beelzebul-luxe4sterung-der-schriftgelehrten.-von-der-suxfcnde-gegen-den-heiligen-geist}}

\bibleverse{22} Auch die Schriftgelehrten, die von Jerusalem
herabgekommen waren, sagten: »Er ist von Beelzebul\textless sup
title=``=~dem Satan; vgl. 2.Kön 1,2''\textgreater✲ besessen«, und: »Im
Bunde mit dem Obersten✲ der bösen Geister treibt er die Geister aus.«
\bibleverse{23} Da rief Jesus sie zu sich und redete in Gleichnissen zu
ihnen: »Wie kann der Satan den Satan austreiben? \bibleverse{24} Und
wenn ein Reich in sich selbst uneinig ist, so kann ein solches Reich
keinen Bestand haben; \bibleverse{25} und wenn ein Haus\textless sup
title=``=~eine Familie''\textgreater✲ in sich selbst uneinig ist, so
wird ein solches Haus keinen Bestand haben können; \bibleverse{26} und
wenn der Satan sich gegen sich selbst erhebt und mit sich selbst in
Zwiespalt gerät, so kann er nicht bestehen, sondern es ist zu Ende mit
ihm. \bibleverse{27} Niemand kann aber in das Haus des Starken
eintreten✲ und ihm sein Rüstzeug\textless sup title=``oder: seinen
Hausrat''\textgreater✲ rauben, ohne zuvor den Starken gefesselt zu
haben: erst dann kann er sein Haus ausplündern. \bibleverse{28} Wahrlich
ich sage euch: Alle Sünden werden den Menschenkindern vergeben werden,
auch die Lästerungen, so viele sie deren aussprechen mögen;
\bibleverse{29} wer sich aber gegen den heiligen Geist der Lästerung
schuldig macht, der erlangt in Ewigkeit keine Vergebung, sondern ist
einer ewigen Sünde schuldig«~-- \bibleverse{30} (das sagte Jesus) weil
sie behaupteten, er sei von einem unreinen Geist besessen.

\hypertarget{c-die-wahren-verwandten-jesu}{%
\paragraph{c) Die wahren Verwandten
Jesu}\label{c-die-wahren-verwandten-jesu}}

\bibleverse{31} Da kamen seine Mutter und seine Brüder; sie blieben
draußen stehen, schickten zu ihm und ließen ihn rufen, \bibleverse{32}
während gerade eine große Volksmenge um ihn herum saß. Als man ihm nun
meldete: »Deine Mutter und deine Brüder {[}und deine Schwestern{]} sind
draußen und fragen nach dir«, \bibleverse{33} gab er ihnen zur Antwort:
»Wer ist meine Mutter, und wer sind meine Brüder?« \bibleverse{34} Und
indem er auf die blickte, welche rings im Kreise um ihn saßen, sagte er:
»Seht, diese hier sind meine Mutter und meine Brüder! \bibleverse{35}
Jeder, der den Willen Gottes tut, der ist mir Bruder und Schwester und
Mutter.«

\hypertarget{jesu-seepredigt-in-gleichnissen-d.h.-bilderreden}{%
\subsubsection{4. Jesu Seepredigt in Gleichnissen (d.h.
Bilderreden)}\label{jesu-seepredigt-in-gleichnissen-d.h.-bilderreden}}

\hypertarget{a-einleitende-bemerkungen-gleichnis-vom-suxe4mann-und-viererlei-acker}{%
\paragraph{a) Einleitende Bemerkungen; Gleichnis vom Sämann und
viererlei
Acker}\label{a-einleitende-bemerkungen-gleichnis-vom-suxe4mann-und-viererlei-acker}}

\hypertarget{section-3}{%
\section{4}\label{section-3}}

\bibleverse{1} Und wieder einmal begann er am See zu lehren; und es
sammelte sich eine sehr große Volksmenge bei ihm, so daß er in ein Boot
stieg und sich darin auf dem See niedersetzte, während das gesamte Volk
sich auf dem Lande am Ufer des Sees befand. \bibleverse{2} Da trug er
ihnen vielerlei Lehren in Gleichnissen vor und sagte zu ihnen in seiner
Belehrung:

\bibleverse{3} »Hört zu! Seht, der Sämann ging aus, um zu säen;
\bibleverse{4} und beim Säen fiel einiges (vom Saatkorn) auf den Weg
längshin\textless sup title=``oder: daneben''\textgreater✲; da kamen die
Vögel und fraßen es auf. \bibleverse{5} Anderes fiel auf felsigen Boden,
wo es nicht viel Erdreich hatte und bald aufschoß, weil es nicht tief in
den Boden dringen konnte; \bibleverse{6} als dann die Sonne aufgegangen
war, wurde es versengt und verdorrte, weil es keine Wurzel (geschlagen)
hatte. \bibleverse{7} Wieder anderes fiel unter die Dornen; und die
Dornen wuchsen auf und erstickten es, und es brachte keine Frucht.
\bibleverse{8} Anderes aber fiel auf den guten Boden und brachte Frucht,
indem es aufging und wuchs; und das eine trug dreißigfältig, das andere
sechzigfältig, noch anderes hundertfältig.« \bibleverse{9} Er schloß mit
den Worten: »Wer Ohren hat zu hören, der höre!«

\hypertarget{b-gespruxe4ch-uxfcber-bedeutung-und-zweck-der-gleichnisse}{%
\paragraph{b) Gespräch über Bedeutung und Zweck der
Gleichnisse}\label{b-gespruxe4ch-uxfcber-bedeutung-und-zweck-der-gleichnisse}}

\bibleverse{10} Als er dann allein war, fragten ihn die, welche samt den
Zwölfen bei ihm waren, um das Gleichnis\textless sup title=``=~nach dem
Sinn des Gleichnisses''\textgreater✲. \bibleverse{11} Da antwortete er
ihnen: »Euch ist es gegeben, das Geheimnis des Reiches Gottes (zu
erkennen); den Außenstehenden aber wird alles nur in Gleichnissen
zuteil, \bibleverse{12} ›damit sie immerfort sehen und doch nicht
wahrnehmen, und immerfort hören und doch kein Verständnis haben, auf daß
sie sich nicht bekehren und ihnen nicht Vergebung zuteil
werde‹.«\textless sup title=``Jes 6,9-10''\textgreater✲ \bibleverse{13}
Dann fuhr er fort: »Ihr versteht dieses Gleichnis nicht? Ja, wie wollt
ihr da die Gleichnisse überhaupt verstehen?«

\hypertarget{c-deutung-des-gleichnisses-vom-suxe4mann}{%
\paragraph{c) Deutung des Gleichnisses vom
Sämann}\label{c-deutung-des-gleichnisses-vom-suxe4mann}}

\bibleverse{14} »Der Sämann sät das Wort. \bibleverse{15} Die aber, bei
denen der Same auf den Weg längshin\textless sup title=``oder:
daneben''\textgreater✲ fällt, sind solche: da wird das Wort (wohl)
gesät, doch wenn sie es gehört haben, kommt sogleich der Satan und nimmt
das Wort weg, das in sie gesät war. \bibleverse{16} Ebenso die, bei
denen der Same auf felsiges Land fällt, das sind solche: wenn sie das
Wort hören, nehmen sie es für den Augenblick mit Freuden an;
\bibleverse{17} doch sie haben keine Wurzel in sich, sondern sind
Kinder✲ des Augenblicks; wenn nachher Drangsal oder Verfolgung um des
Wortes willen kommt, werden sie sogleich (am Glauben) irre.
\bibleverse{18} Bei anderen fällt der Same unter die Dornen; das sind
solche, die das Wort wohl gehört haben, \bibleverse{19} doch die
weltlichen Sorgen und der Betrug des Reichtums und die sonstigen Gelüste
dringen in sie ein und ersticken das Wort: so bleibt es ohne Frucht.
\bibleverse{20} Wo aber auf den guten Boden gesät ist, das sind solche,
die das Wort hören und aufnehmen und Frucht bringen, dreißigfältig und
sechzigfältig und hundertfältig.«

\hypertarget{d-spruxfcche-uxfcber-die-juxfcngerpflicht-bezuxfcglich-der-kuxfcnftigen-klaren-und-reichlichen-verkuxfcndigung-der-heilsbotschaft}{%
\paragraph{d) Sprüche über die Jüngerpflicht (bezüglich der künftigen
klaren und reichlichen Verkündigung der
Heilsbotschaft)}\label{d-spruxfcche-uxfcber-die-juxfcngerpflicht-bezuxfcglich-der-kuxfcnftigen-klaren-und-reichlichen-verkuxfcndigung-der-heilsbotschaft}}

\bibleverse{21} Weiter sagte er zu ihnen: »Kommt etwa die Lampe (in das
Zimmer), damit man sie unter den Scheffel oder unter das Bett stelle?
Nein, damit sie auf den Leuchter✲ gestellt werde\textless sup title=``Mt
5,15; Lk 8,16; 11,33''\textgreater✲. \bibleverse{22} Denn es gibt nichts
Verborgenes, außer damit es offenbart werde, und nichts ist in Geheimnis
gehüllt worden, außer damit es ans Tageslicht komme\textless sup
title=``Mt 10,26; Lk 12,2''\textgreater✲. \bibleverse{23} Wer Ohren hat
zu hören, der höre!«~-- \bibleverse{24} Dann fuhr er fort: »Seid achtsam
auf das, was ihr hört! Mit demselben Maß, mit dem ihr meßt, wird euch
wieder gemessen werden, und es wird euch noch hinzugetan
werden\textless sup title=``Mt 7,2; Lk 6,38''\textgreater✲.
\bibleverse{25} Denn wer da hat, dem wird noch dazugegeben werden; und
wer nicht hat, dem wird auch das genommen werden, was er
hat.«\textless sup title=``Mt 13,12; 25,29; Lk 19,26''\textgreater✲

\hypertarget{e-gleichnisse-von-der-still-von-selbst-wachsenden-saat-und-vom-senfkorn}{%
\paragraph{e) Gleichnisse von der still von selbst wachsenden Saat und
vom
Senfkorn}\label{e-gleichnisse-von-der-still-von-selbst-wachsenden-saat-und-vom-senfkorn}}

\bibleverse{26} Er fuhr dann fort: »Mit dem Reiche Gottes verhält es
sich so, wie wenn jemand den Samen auf das Land wirft \bibleverse{27}
und dann schläft und aufsteht in der Nacht und bei Tag; und der Same
sproßt und wächst hoch, ohne daß er selbst etwas davon weiß.
\bibleverse{28} Von selbst bringt die Erde Frucht hervor, zuerst die
grünen Halme, dann die Ähren, dann den vollen✲ Weizen in den Ähren.
\bibleverse{29} Wenn aber die Frucht es zuläßt\textless sup
title=``=~ausgewachsen ist''\textgreater✲, legt er sofort die Sichel
an\textless sup title=``d.h. er schickt die Schnitter
hin''\textgreater✲; denn die Ernte ist da.«

\bibleverse{30} Weiter sagte er: »Wie sollen wir ein Bild vom Reiche
Gottes entwerfen oder in welchem Gleichnis es darstellen?
\bibleverse{31} Es gleicht einem Senfkorn, das, wenn man es in den
Erdboden sät, kleiner ist als alle anderen Samenarten auf der Erde;
\bibleverse{32} doch wenn es gesät ist, geht es auf und wird größer als
alle anderen Gartengewächse und treibt große Zweige, so daß unter seinem
Schatten die Vögel des Himmels nisten können.«

\bibleverse{33} In vielen derartigen Gleichnissen verkündete Jesus ihnen
das Wort\textless sup title=``=~die Heilsbotschaft''\textgreater✲, je
nach dem\textless sup title=``=~so, wie''\textgreater✲ sie es zu
verstehen vermochten; \bibleverse{34} aber ohne Gleichnis redete er
nicht zu ihnen; wenn er dann mit seinen Jüngern allein war, so gab er
ihnen die Auslegung von allem.

\hypertarget{drei-wunderberichte-jesu-macht-uxfcber-sturm-buxf6se-geister-und-tod}{%
\subsubsection{5. Drei Wunderberichte: Jesu Macht über Sturm, böse
Geister und
Tod}\label{drei-wunderberichte-jesu-macht-uxfcber-sturm-buxf6se-geister-und-tod}}

\hypertarget{a-jesus-beschwichtigt-den-seesturm}{%
\paragraph{a) Jesus beschwichtigt den
Seesturm}\label{a-jesus-beschwichtigt-den-seesturm}}

\bibleverse{35} Er sagte dann zu ihnen an jenem Tage, als es Abend
geworden war: »Wir wollen ans andere Ufer (des Sees) hinüberfahren!«
\bibleverse{36} So ließen sie denn die Volksmenge gehen und nahmen ihn,
wie er war, im Boote mit; doch auch noch andere Boote begleiteten ihn.
\bibleverse{37} Da erhob sich ein gewaltiger Sturmwind, und die Wellen
schlugen in das Boot, so daß das Boot sich schon mit Wasser zu füllen
begann; \bibleverse{38} er selbst aber lag am hinteren Teil des Bootes
und schlief auf dem Kissen. Sie weckten ihn nun und sagten zu ihm:
»Meister, liegt dir nichts daran, daß wir untergehen?« \bibleverse{39}
Da stand er auf, bedrohte den Wind und gebot dem See: »Schweige! Werde
still!« Da legte sich der Wind, und es trat völlige Windstille ein.
\bibleverse{40} Hierauf sagte er zu ihnen: »Was seid ihr so furchtsam?
Habt ihr immer noch keinen Glauben?« \bibleverse{41} Da gerieten sie in
große Furcht und sagten zueinander: »Wer ist denn dieser, daß auch der
Wind und der See ihm gehorsam sind?«

\hypertarget{b-jesus-heilt-den-besessenen-im-lande-der-gerasener}{%
\paragraph{b) Jesus heilt den Besessenen im Lande der
Gerasener}\label{b-jesus-heilt-den-besessenen-im-lande-der-gerasener}}

\hypertarget{section-4}{%
\section{5}\label{section-4}}

\bibleverse{1} Sie kamen dann an das jenseitige Ufer des Sees in das
Gebiet der Gerasener. \bibleverse{2} Als er dort aus dem Boot gestiegen
war, lief ihm sogleich von den Gräbern\textless sup title=``=~der
Gräberstätte''\textgreater✲ her ein Mann entgegen, der von einem
unreinen Geist besessen war. \bibleverse{3} Er hatte seinen Aufenthalt
in den Gräbern\textless sup title=``vgl. Lk 8,27''\textgreater✲, und
niemand vermochte ihn zu fesseln, auch nicht mit einer Kette;
\bibleverse{4} denn man hatte ihn schon oft mit Fußfesseln und Ketten
gebunden, aber er hatte die Ketten immer wieder zerrissen und die
Fußfesseln zerrieben, und niemand war stark genug, ihn zu überwältigen.
\bibleverse{5} Er hielt sich allezeit, bei Tag und bei Nacht, in den
Gräbern und auf den Bergen auf, schrie laut und zerschlug sich mit
Steinen. \bibleverse{6} Als er nun Jesus von weitem sah, kam er
herzugelaufen, warf sich vor ihm nieder \bibleverse{7} und stieß laut
schreiend die Worte aus: »Was willst du von mir, Jesus, du Sohn Gottes,
des Höchsten? Ich beschwöre dich bei Gott: quäle mich nicht!«
\bibleverse{8} Jesus war nämlich im Begriff, ihm zu gebieten: »Fahre
aus, du unreiner Geist, aus dem Manne!« \bibleverse{9} Da fragte Jesus
ihn: »Wie heißt du?« Er antwortete ihm: »Legion\textless sup
title=``oder: Heerschar''\textgreater✲ heiße ich, denn wir sind unser
viele.« \bibleverse{10} Dann bat er ihn inständig, er möchte sie nicht
aus der Gegend verweisen. \bibleverse{11} Nun befand sich dort am Berge
eine große Herde Schweine auf der Weide. \bibleverse{12} Da baten sie
ihn: »Schicke uns in die Schweine! Laß uns in sie fahren!«
\bibleverse{13} Das erlaubte Jesus ihnen auch, und so fuhren denn die
unreinen Geister aus und fuhren in die Schweine hinein; und die Herde
stürmte den Abhang hinab in den See hinein, etwa zweitausend Tiere, und
sie ertranken im See.

\bibleverse{14} Ihre Hirten aber ergriffen die Flucht und berichteten
den Vorfall in der Stadt und in den Gehöften; da kamen die Leute, um zu
sehen, was geschehen war. \bibleverse{15} Als sie nun zu Jesus gekommen
waren, sahen sie den (früher) Besessenen ruhig dasitzen, bekleidet und
ganz vernünftig, ihn, der die Legion (unreiner Geister) in sich gehabt
hatte, und sie gerieten darüber in Furcht. \bibleverse{16} Die
Augenzeugen erzählten ihnen nun, was mit dem Besessenen vorgegangen war,
und auch die Begebenheit mit den Schweinen. \bibleverse{17} Da verlegten
sie sich aufs Bitten, er möchte ihr Gebiet verlassen.

\bibleverse{18} Als er dann ins Boot steigen wollte, bat ihn der
(früher) Besessene, bei ihm bleiben zu dürfen; \bibleverse{19} doch
Jesus gestattete es ihm nicht, sondern sagte zu ihm: »Gehe heim in dein
Haus zu deinen Angehörigen und berichte ihnen, wie Großes der Herr an
dir getan und wie er sich deiner erbarmt hat!« \bibleverse{20} Da ging
er weg und begann in der Landschaft der Zehn-Städte zu verkündigen, wie
Großes Jesus an ihm getan hatte; und alle verwunderten sich darüber.

\hypertarget{c-jesus-heilt-in-kapernaum-die-blutfluxfcssige-frau-und-erweckt-das-tuxf6chterlein-des-jairus}{%
\paragraph{c) Jesus heilt (in Kapernaum) die blutflüssige Frau und
erweckt das Töchterlein des
Jairus}\label{c-jesus-heilt-in-kapernaum-die-blutfluxfcssige-frau-und-erweckt-das-tuxf6chterlein-des-jairus}}

\bibleverse{21} Als Jesus dann im Boot wieder an das jenseitige Ufer
hinübergefahren war, sammelte sich eine große Volksmenge bei ihm,
während er sich noch am See befand. \bibleverse{22} Da kam einer von den
Vorstehern der Synagoge namens Jairus, und als er Jesus erblickte, warf
er sich vor ihm nieder \bibleverse{23} und bat ihn inständig mit den
Worten: »Mein Töchterlein ist todkrank; komm doch und lege ihr die Hände
auf, damit sie gerettet wird und am Leben bleibt!« \bibleverse{24} Da
ging Jesus mit ihm; es folgte ihm aber eine große Volksmenge und
umdrängte ihn.

\bibleverse{25} Nun war da eine Frau, die schon zwölf Jahre lang am
Blutfluß gelitten \bibleverse{26} und mit vielen Ärzten viel
durchgemacht und ihr ganzes Vermögen dabei zugesetzt hatte, ohne Nutzen
davon gehabt zu haben -- es war vielmehr immer noch schlimmer mit ihr
geworden --; \bibleverse{27} die hatte von Jesus gehört und kam nun in
der Volksmenge von hinten herzu und faßte seinen Rock\textless sup
title=``oder: Mantel''\textgreater✲; \bibleverse{28} sie dachte nämlich:
»Wenn ich auch nur seine Kleider anfasse, so wird mir geholfen sein.«
\bibleverse{29} Und sogleich hörte ihr Blutfluß auf, und sie spürte in
ihrem Körper, daß sie von ihrem Leiden geheilt war. \bibleverse{30} Da
nun auch Jesus sogleich die Empfindung in sich hatte, daß die
Heilungskraft von ihm ausgegangen war, wandte er sich in der Volksmenge
um und fragte: »Wer hat meine Kleider angefaßt?« \bibleverse{31} Da
sagten seine Jünger zu ihm: »Du siehst doch, wie sehr die Volksmenge
dich umdrängt, und da fragst du: ›Wer hat mich angefaßt?‹«
\bibleverse{32} Doch er blickte rings um sich nach der, die es getan
hatte. \bibleverse{33} Da kam die Frau voller Angst und zitternd herbei,
weil sie wohl wußte, was mit ihr vorgegangen war, warf sich vor ihm
nieder und bekannte ihm die ganze Wahrheit. \bibleverse{34} Er aber
sagte zu ihr: »Meine Tochter, dein Glaube hat dich gerettet: gehe hin in
Frieden und sei✲ von deinem Leiden geheilt!«

\bibleverse{35} Während er noch redete, kamen Leute aus dem Hause des
Synagogenvorstehers mit der Meldung: »Deine Tochter ist gestorben: was
bemühst du den Meister noch?« \bibleverse{36} Jesus aber ließ die
Nachricht, die da gemeldet wurde, unbeachtet und sagte zu dem
Synagogenvorsteher: »Fürchte dich nicht, glaube nur!« \bibleverse{37}
Und er ließ niemand mit sich gehen außer Petrus, Jakobus und Johannes,
den Bruders des Jakobus. \bibleverse{38} So kamen sie zum Hause des
Synagogenvorstehers, wo er das Getümmel wahrnahm und wie sie weinten und
laut wehklagten. \bibleverse{39} Als er dann eingetreten war, sagte er
zu den Leuten: »Wozu lärmt und weint ihr? Das Kind ist nicht tot,
sondern schläft nur!« \bibleverse{40} Da verlachten sie ihn. Er aber
entfernte alle aus dem Hause, nahm nur den Vater des Kindes nebst der
Mutter und seine Jünger, die ihn begleiteten, mit sich und ging (in das
Zimmer) hinein, wo das Kind lag. \bibleverse{41} Dann faßte er das Kind
bei der Hand und sagte zu ihm: »Talitha kumi!«, was übersetzt heißt:
»Mädchen, ich sage dir: stehe auf!« \bibleverse{42} Da stand das Mädchen
sogleich auf und ging umher; denn sie war zwölf Jahre alt. Da gerieten
sie sofort vor Staunen ganz außer sich. \bibleverse{43} Er gebot ihnen
dann ernstlich, niemand solle etwas von dem Geschehenen erfahren, und
ordnete an, man möge ihr zu essen geben.

\hypertarget{verwerfung-und-miuxdferfolg-jesu-in-seiner-vaterstadt-nazareth}{%
\subsubsection{6. Verwerfung und Mißerfolg Jesu in seiner Vaterstadt
Nazareth}\label{verwerfung-und-miuxdferfolg-jesu-in-seiner-vaterstadt-nazareth}}

\hypertarget{section-5}{%
\section{6}\label{section-5}}

\bibleverse{1} Er zog dann von dort weiter und kam in seine Vaterstadt
(Nazareth), und seine Jünger begleiteten ihn. \bibleverse{2} Als nun der
Sabbat gekommen war, fing er an, in der Synagoge zu lehren; und die
vielen, die ihm zuhörten, gerieten in Staunen und sagten: »Woher hat er
das\textless sup title=``d.h. solche Gaben''\textgreater✲? Und was ist
das für eine Weisheit, die diesem verliehen ist? Und solche Wundertaten
geschehen durch seine Hände! \bibleverse{3} Ist dieser nicht der
Zimmermann, der Sohn der Maria und der Bruder des Jakobus, des Joses,
des Judas und des Simon? Und leben nicht auch seine Schwestern bei uns?«
So wurden sie irre an ihm. \bibleverse{4} Da sagte Jesus zu ihnen: »Ein
Prophet gilt nirgends weniger als in seiner Vaterstadt und bei seinen
Verwandten und in seiner Familie.« \bibleverse{5} Er konnte dort auch
kein Wunder vollbringen, außer daß er einige Kranke durch Handauflegen
heilte. \bibleverse{6} Und er verwunderte sich über ihren Unglauben.

\hypertarget{aussendung-und-anweisung-der-zwuxf6lf-juxfcnger}{%
\subsubsection{7. Aussendung und Anweisung der zwölf
Jünger}\label{aussendung-und-anweisung-der-zwuxf6lf-juxfcnger}}

Er zog dann in den umliegenden Ortschaften umher und lehrte dort.
\bibleverse{7} Darauf rief er die Zwölf zu sich und begann sie paarweise
auszusenden; dabei gab er ihnen Macht über die unreinen Geister
\bibleverse{8} und gebot ihnen, nichts auf den Weg mitzunehmen als nur
einen Stock, kein Brot, keinen Ranzen\textless sup title=``oder: keine
Reisetasche''\textgreater✲ und kein Geld im Gürtel; \bibleverse{9}
jedoch Sandalen sollten sie sich unterbinden, aber nicht zwei
Röcke\textless sup title=``oder: Unterkleider''\textgreater✲ anziehen.
\bibleverse{10} Weiter gab er ihnen die Weisung: »Wo ihr in ein Haus
eingetreten✲ seid, da bleibt, bis ihr von dort weiterzieht;
\bibleverse{11} und wenn ein Ort euch nicht aufnimmt und man euch nicht
hören will, so geht von dort weg und schüttelt den Staub von euren
Fußsohlen ab zum Zeugnis für sie!« \bibleverse{12} So machten sie sich
denn auf den Weg und predigten, man solle Buße tun\textless sup
title=``vgl. Mt 3,2''\textgreater✲; \bibleverse{13} sie trieben auch
viele böse Geister aus, salbten viele Kranke mit Öl und heilten sie.

\hypertarget{das-urteil-des-herodes-uxfcber-jesus-das-ende-johannes-des-tuxe4ufers}{%
\subsubsection{8. Das Urteil des Herodes über Jesus; das Ende Johannes
des
Täufers}\label{das-urteil-des-herodes-uxfcber-jesus-das-ende-johannes-des-tuxe4ufers}}

\bibleverse{14} Auch der König Herodes hörte davon\textless sup
title=``d.h. von Jesus''\textgreater✲; denn sein Name war bekannt
geworden, und man sagte: »Johannes der Täufer ist von den Toten
auferweckt worden, darum sind die Wunderkräfte in ihm wirksam.«
\bibleverse{15} Andere aber sagten, er sei Elia; noch andere
behaupteten, er sei ein Prophet wie einer der (alten) Propheten.
\bibleverse{16} Als aber Herodes davon hörte, sagte er: »Johannes, den
ich habe enthaupten lassen, der ist wieder auferweckt worden.«

\bibleverse{17} Eben dieser Herodes nämlich hatte (Diener) ausgesandt
und Johannes festnehmen und ihn gefesselt ins Gefängnis werfen lassen um
der Herodias willen, der Gattin seines Bruders Philippus, weil er sie
geheiratet hatte; \bibleverse{18} denn Johannes hatte dem Herodes
vorgehalten: »Du darfst die Frau deines Bruders nicht (zur Frau)
haben.«\textless sup title=``3.Mose 18,16''\textgreater✲ \bibleverse{19}
Das trug Herodias ihm nach und hätte ihn am liebsten ums Leben gebracht,
vermochte es aber nicht; \bibleverse{20} denn Herodes hatte Scheu vor
Johannes, weil er ihn als einen gerechten und heiligen Mann kannte, und
er nahm ihn in seinen Schutz; und oftmals, wenn er ihn gehört hatte, war
er schwer betroffen, hörte ihn aber dennoch gern. \bibleverse{21} Da kam
ein (für Herodias) gelegener Tag, als nämlich Herodes an seinem
Geburtstage seinen Würdenträgern\textless sup title=``oder:
Hofleuten''\textgreater✲ und Heeresobersten\textless sup
title=``=~höchsten Offizieren''\textgreater✲ sowie den vornehmsten
Männern von Galiläa ein Festmahl veranstaltete. \bibleverse{22} Als
dabei die Tochter eben jener Herodias (in den Saal) eintrat und einen
Tanz aufführte, gefiel sie dem Herodes und seinen Tischgästen wohl. Da
sagte der König zu dem Mädchen: »Erbitte dir von mir, was du willst: ich
will es dir geben!«, \bibleverse{23} und er schwur ihr: »Was du dir auch
von mir erbitten magst, das will ich dir geben bis zur Hälfte meines
Reiches!« \bibleverse{24} Da ging sie hinaus und fragte ihre Mutter:
»Was soll ich mir erbitten?« Die antwortete: »Den Kopf Johannes des
Täufers!« \bibleverse{25} Sogleich ging sie in Eile zum König hinein und
sprach die Bitte aus: »Ich möchte, du gäbest mir gleich jetzt auf einer
Schüssel den Kopf Johannes des Täufers!« \bibleverse{26} Obgleich nun
der König sehr betrübt darüber wurde, mochte er sie doch mit Rücksicht
auf seine Eide und auf seine Tischgäste keine Fehlbitte tun lassen.
\bibleverse{27} So schickte denn der König sogleich einen von seinen
Leibwächtern ab mit dem Befehl, den Kopf des Johannes zu bringen. Der
ging hin, enthauptete ihn im Gefängnis, \bibleverse{28} brachte seinen
Kopf auf einer Schüssel und gab ihn dem Mädchen, und das Mädchen gab ihn
seiner Mutter. \bibleverse{29} Als die Jünger des Johannes Kunde davon
erhielten, kamen sie, nahmen seinen Leichnam und bestatteten ihn in
einem Grabe.

\hypertarget{ruxfcckkehr-der-zwuxf6lf-apostel-jesus-entweicht-in-die-einsamkeit-speisung-der-fuxfcnftausend}{%
\subsubsection{9. Rückkehr der zwölf Apostel; Jesus entweicht in die
Einsamkeit; Speisung der
Fünftausend}\label{ruxfcckkehr-der-zwuxf6lf-apostel-jesus-entweicht-in-die-einsamkeit-speisung-der-fuxfcnftausend}}

\bibleverse{30} Die Apostel versammelten sich dann wieder bei Jesus und
berichteten ihm alles, was sie getan und was sie gelehrt hatten.
\bibleverse{31} Da sagte er zu ihnen: »Kommt ihr für euch allein (mit
mir) abseits an einen einsamen Ort und ruht dort ein wenig aus!« Denn
die Zahl der Leute, die da kamen und gingen, war groß, so daß
sie\textless sup title=``d.h. die Apostel''\textgreater✲ nicht einmal
Zeit zum Essen hatten. \bibleverse{32} So fuhren sie denn im Boot in
eine einsame Gegend, um für sich allein zu sein; \bibleverse{33} doch
man hatte sie abfahren sehen, und viele hatten ihre Absicht gemerkt; sie
eilten daher aus allen Ortschaften zu Fuß dort zusammen und kamen noch
vor ihnen an.

\bibleverse{34} Als Jesus nun (aus dem Boote) ausstieg und eine große
Menge Volks versammelt sah, ergriff ihn tiefes Mitleid mit
ihnen\textless sup title=``Mt 9,36''\textgreater✲, denn sie waren wie
Schafe, die keinen Hirten haben\textless sup title=``4.Mose
27,17''\textgreater✲; und er fing an, sie vieles zu lehren.
\bibleverse{35} Als dann die Zeit schon weit vorgerückt war, traten
seine Jünger zu ihm und sagten: »Die Gegend hier ist öde und die Zeit
schon weit vorgerückt; \bibleverse{36} laß die Leute ziehen, damit sie
in die umliegenden Gehöfte und in die Ortschaften gehen und sich dort
etwas zu essen kaufen können.« \bibleverse{37} Er aber antwortete ihnen:
»Gebt ihr ihnen zu essen!« Da sagten sie zu ihm: »Sollen wir hingehen
und für zweihundert Denar✲ Brot kaufen, um ihnen zu essen zu geben?«
\bibleverse{38} Er aber antwortete ihnen: »Wie viele Brote habt ihr?
Geht hin, seht nach!« Als sie nun nachgesehen hatten, meldeten sie ihm:
»Fünf (Brote) und zwei Fische.« \bibleverse{39} Da gab er ihnen die
Weisung, sie sollten alle sich zu einzelnen Tischgenossenschaften auf
dem grünen Rasen lagern; \bibleverse{40} so ließen sie sich denn
gruppenweise zu hundert und zu fünfzig nieder. \bibleverse{41} Hierauf
nahm er die fünf Brote und die beiden Fische, blickte zum Himmel auf,
sprach den Lobpreis (Gottes), brach die Brote und gab sie\textless sup
title=``d.h. die Stücke''\textgreater✲ seinen Jüngern, damit diese sie
dem Volk vorlegten; auch die beiden Fische teilte er für alle aus.
\bibleverse{42} Und sie aßen alle und wurden satt; \bibleverse{43} dann
hob man an Brocken noch zwölf Körbe voll (vom Boden) auf, dazu auch
Überbleibsel von den Fischen. \bibleverse{44} Und die Zahl derer, die
von den Broten gegessen hatten, betrug fünftausend Männer.

\hypertarget{ruxfcckfahrt-uxfcber-den-see-bei-nacht-das-wandeln-jesu-auf-dem-see-die-landung-in-gennesaret}{%
\subsubsection{10. Rückfahrt über den See bei Nacht; das Wandeln Jesu
auf dem See; die Landung in
Gennesaret}\label{ruxfcckfahrt-uxfcber-den-see-bei-nacht-das-wandeln-jesu-auf-dem-see-die-landung-in-gennesaret}}

\bibleverse{45} Und sogleich nötigte er seine Jünger, in das Boot zu
steigen und ihm an das jenseitige Ufer nach Bethsaida vorauszufahren,
während er selbst die Volksmenge entlassen wollte. \bibleverse{46}
Nachdem er sie dann verabschiedet hatte, ging er auf den Berg hinauf, um
zu beten. \bibleverse{47} Als es so Abend geworden war, befand sich das
Boot mitten auf dem See, während er selbst allein noch auf dem Lande
war. \bibleverse{48} Als er nun sah, wie sie sich (auf der Fahrt) beim
Rudern abmühten -- denn der Wind stand ihnen entgegen --, kam er um die
vierte Nachtwache\textless sup title=``vgl. Mt 14,25''\textgreater✲ auf
sie zu, indem er auf dem See dahinging, und wollte an ihnen
vorübergehen. \bibleverse{49} Als sie ihn aber so auf dem See wandeln
sahen, dachten sie, es sei ein Gespenst, und schrien auf;
\bibleverse{50} denn alle sahen ihn und waren in Angst geraten. Er aber
redete sie sogleich an und sagte zu ihnen: »Seid getrost, ich bin's:
fürchtet euch nicht!« \bibleverse{51} Er stieg darauf zu ihnen ins Boot:
da legte sich der Wind. Nun gerieten sie vollends vor Erstaunen ganz
außer sich; \bibleverse{52} denn bei der Brotspeisung war ihnen noch
kein Verständnis gekommen, sondern ihr Herz war verhärtet.

\bibleverse{53} Als sie dann ans Land hinübergefahren waren, kamen sie
nach Gennesaret und legten dort an. \bibleverse{54} Als sie aus dem Boot
gestiegen waren, erkannten die Leute dort ihn sogleich, \bibleverse{55}
liefen in jener ganzen Gegend umher und begannen die Kranken auf den
Bahren umherzutragen (und dahin zu bringen), wo er, dem Vernehmen nach,
sich gerade aufhielt. \bibleverse{56} Und wo er in Dörfern oder Städten
oder Gehöften einkehrte, legten sie die Kranken auf den freien Plätzen
nieder und baten ihn, daß sie auch nur die Quaste seines
Rockes\textless sup title=``oder: Mantels''\textgreater✲ anfassen
dürften; und alle, die ihn\textless sup title=``oder: sie''\textgreater✲
anfaßten, wurden gesund.

\hypertarget{streit-mit-den-gegnern-um-das-huxe4ndewaschen-warnung-vor-menschensatzungen-und-kennzeichnung-der-wahren-unreinheit}{%
\subsubsection{11. Streit mit den Gegnern um das Händewaschen; Warnung
vor Menschensatzungen und Kennzeichnung der wahren
Unreinheit}\label{streit-mit-den-gegnern-um-das-huxe4ndewaschen-warnung-vor-menschensatzungen-und-kennzeichnung-der-wahren-unreinheit}}

\hypertarget{section-6}{%
\section{7}\label{section-6}}

\bibleverse{1} Da versammelten sich bei ihm die Pharisäer und einige
Schriftgelehrte, die von Jerusalem gekommen waren; \bibleverse{2} und
als sie einige seiner Jünger die Brote\textless sup title=``oder:
Speisen''\textgreater✲ mit unreinen, das heißt ungewaschenen Händen zu
sich nehmen sahen~-- \bibleverse{3} die Pharisäer nämlich und die Juden
überhaupt essen nur, wenn sie sich die Hände mit der Faust✲ gewaschen
haben, weil sie an den von den Alten überlieferten Satzungen festhalten;
\bibleverse{4} und auch wenn sie vom Markt heimkommen, essen sie nicht,
ohne sich zunächst (die Hände) abgespült zu haben; und noch viele andere
Vorschriften gibt es, deren strenge Beobachtung sie überkommen haben,
z.B. das Eintauchen\textless sup title=``oder: Waschungen''\textgreater✲
von Bechern, Krügen und Kupfergeschirr --, \bibleverse{5} da fragten ihn
die Pharisäer und Schriftgelehrten: »Warum halten sich deine Jünger in
ihrer Lebensweise nicht an die Überlieferung der Alten, sondern nehmen
die Speisen\textless sup title=``oder: Mahlzeiten''\textgreater✲ mit
unreinen✲ Händen zu sich?« \bibleverse{6} Er antwortete ihnen: »Treffend
hat Jesaja von euch Heuchlern✲ geweissagt, wie geschrieben
steht\textless sup title=``Jes 29,13''\textgreater✲: ›Dieses Volk ehrt
mich (nur) mit den Lippen, ihr Herz aber ist weit entfernt von mir;
\bibleverse{7} doch vergeblich verehren sie mich, weil sie
Menschengebote zu ihren Lehren machen.‹ \bibleverse{8} Das Gebot Gottes
laßt ihr außer acht und haltet an den euch überlieferten Satzungen der
Menschen fest {[}ihr nehmt Abwaschungen von Krügen und Bechern vor und
tut Ähnliches derart noch vielfach{]}.« \bibleverse{9} Dann fuhr er
fort: »Trefflich versteht ihr es, das Gebot Gottes aufzuheben, um die
euch überlieferten Satzungen festzuhalten. \bibleverse{10} Mose hat z.B.
geboten\textless sup title=``2.Mose 20,12''\textgreater✲: ›Ehre deinen
Vater und deine Mutter‹ und\textless sup title=``2.Mose
21,12''\textgreater✲: ›Wer den Vater oder die Mutter schmäht, soll des
Todes sterben.‹ \bibleverse{11} Ihr aber sagt: ›Wenn jemand zu seinem
Vater oder zu seiner Mutter sagt: Korban, das heißt: eine Gabe für den
Tempelschatz soll das sein, was dir sonst als Unterstützung von mir
zugute gekommen wäre‹, \bibleverse{12} so laßt ihr ihn für seinen Vater
oder seine Mutter nichts mehr tun \bibleverse{13} und hebt damit das
Wort Gottes durch eure Überlieferung auf, die ihr
weitergegeben\textless sup title=``=~allmählich
herausgebildet''\textgreater✲ habt; und Ähnliches derart tut ihr
vielfach.«

\bibleverse{14} Nachdem er dann die Volksmenge wieder herbeigerufen
hatte, sagte er zu ihnen: »Hört mir alle zu und sucht es zu verstehen!
\bibleverse{15} Nichts geht von außen in den Menschen hinein, was ihn zu
verunreinigen vermag, sondern was aus dem Menschen herauskommt, das ist
es, was den Menschen verunreinigt. \bibleverse{16} {[}Wer Ohren hat zu
hören, der höre!{]}«

\bibleverse{17} Als er dann vom Volk weggegangen und ins
Haus\textless sup title=``oder: nach Hause''\textgreater✲ gekommen war,
befragten ihn seine Jünger über das Gleichnis\textless sup title=``=~den
dunklen Ausspruch V.15''\textgreater✲. \bibleverse{18} Da sagte er zu
ihnen: »So seid auch ihr immer noch ohne Verständnis? Begreift ihr
nicht, daß alles, was von außen her in den Menschen hineingeht, ihn
nicht zu verunreinigen vermag, \bibleverse{19} weil es ihm nicht ins
Herz hineingeht, sondern in den Leib✲ und auf dem natürlichen Wege, der
alle Speisen reinigt, wieder ausgeschieden wird?« \bibleverse{20} Dann
fuhr er fort: »Was dagegen aus dem Menschen herauskommt, das
verunreinigt den Menschen. \bibleverse{21} Denn von innen her, aus dem
Herzen der Menschen, kommen die bösen Gedanken hervor: Unzucht,
Diebstahl, Mordtaten, \bibleverse{22} Ehebruch, Habsucht, Bosheit,
Arglist, Ausschweifung, Scheelsucht, Lästerung, Hochmut, Unverstand.
\bibleverse{23} Alles Böse dieser Art kommt von innen heraus und
verunreinigt den Menschen.«

\hypertarget{jesus-und-die-syrophuxf6nizierin-im-gebiet-von-tyrus-und-sidon}{%
\subsubsection{12. Jesus und die Syrophönizierin im Gebiet von Tyrus und
Sidon}\label{jesus-und-die-syrophuxf6nizierin-im-gebiet-von-tyrus-und-sidon}}

\bibleverse{24} Er brach dann von dort auf und begab sich in das Gebiet
von Tyrus. Als er dort in einem Hause Aufnahme gefunden hatte, wünschte
er, daß niemand es erführe; doch er konnte nicht verborgen bleiben,
\bibleverse{25} sondern alsbald hörte eine Frau von ihm, deren
Töchterlein von einem unreinen Geist besessen war; sie kam also und warf
sich vor ihm nieder~-- \bibleverse{26} die Frau war aber eine
Griechin\textless sup title=``=~griechisch redende
Heidin''\textgreater✲, ihrer Herkunft nach eine Syrophönizierin -- und
bat ihn, er möchte den bösen Geist aus ihrer Tochter austreiben.
\bibleverse{27} Da entgegnete er ihr: »Laß zuerst die Kinder satt
werden; denn es ist nicht recht, das den Kindern zukommende Brot zu
nehmen und es den Hündlein hinzuwerfen.« \bibleverse{28} Sie aber gab
ihm zur Antwort: »O doch, Herr! Auch die Hündlein bekommen ja unter dem
Tisch von den Brocken der Kinder zu essen.« \bibleverse{29} Da sagte er
zu ihr: »Um dieses Wortes willen gehe heim: der böse Geist ist aus
deiner Tochter ausgefahren.« \bibleverse{30} Als sie nun in ihr Haus
zurückkam, traf sie ihr Kind an, wie es ruhig auf dem Bett lag, und der
böse Geist war ausgefahren.

\hypertarget{jesu-ruxfcckkehr-nach-galiluxe4a-an-das-ostufer-des-sees-heilung-eines-taubstummen}{%
\subsubsection{13. Jesu Rückkehr nach Galiläa an das Ostufer des Sees;
Heilung eines
Taubstummen}\label{jesu-ruxfcckkehr-nach-galiluxe4a-an-das-ostufer-des-sees-heilung-eines-taubstummen}}

\bibleverse{31} Nachdem er dann das Gebiet von Tyrus wieder verlassen
hatte, kam er über Sidon an den Galiläischen See (und zwar) mitten in
das Gebiet der Zehn-Städte. \bibleverse{32} Da brachten sie einen Tauben
zu ihm, der kaum lallen konnte, und baten ihn, er möchte ihm die Hand
auflegen. \bibleverse{33} So nahm er ihn denn von der Volksmenge weg
abseits, legte ihm, als er mit ihm allein war, seine Finger in die
Ohren, benetzte sie mit Speichel und berührte ihm die Zunge;
\bibleverse{34} nachdem er dann zum Himmel aufgeblickt hatte, seufzte er
und sagte zu ihm: »Effatha!«, das heißt (übersetzt) »Tu dich auf!«
\bibleverse{35} Da taten sich seine Ohren auf, die Gebundenheit seiner
Zunge löste sich, und er redete richtig. \bibleverse{36} Jesus gebot
ihnen dann ernstlich, daß sie niemand etwas davon sagen sollten; aber je
mehr er es ihnen gebot, um so mehr und um so eifriger verbreiteten sie
die Kunde; \bibleverse{37} und sie gerieten vor Staunen ganz außer sich
und sagten: »Er hat alles wohl gemacht, auch die Tauben macht er hören
und die Sprachlosen reden!«

\hypertarget{speisung-der-viertausend}{%
\subsubsection{14. Speisung der
Viertausend}\label{speisung-der-viertausend}}

\hypertarget{section-7}{%
\section{8}\label{section-7}}

\bibleverse{1} Als in jenen Tagen wieder einmal eine große Volksmenge
zugegen war und sie nichts zu essen hatten, rief Jesus seine Jünger
herbei und sagte zu ihnen: \bibleverse{2} »Mich jammert des Volkes; denn
sie halten nun schon drei Tage bei mir aus und haben nichts zu essen;
\bibleverse{3} und wenn ich sie nach Hause gehen lasse, ohne daß sie
gegessen haben, so werden sie unterwegs verschmachten; sie sind ja auch
zum Teil von weit her gekommen.« \bibleverse{4} Da erwiderten ihm seine
Jünger: »Woher\textless sup title=``oder: wie''\textgreater✲ sollte man
diese hier in einer so öden Gegend mit Brot sättigen können?«
\bibleverse{5} Er fragte sie: »Wie viele Brote habt ihr?« Sie
antworteten: »Sieben.« \bibleverse{6} Da gebot er der Volksmenge, sich
auf der Erde zu lagern; dann nahm er die sieben Brote, sprach den
Lobpreis (Gottes), brach die Brote und gab (die Stücke) seinen Jüngern,
damit diese sie austeilten; die legten sie dann der Volksmenge vor.
\bibleverse{7} Sie hatten auch noch ein paar kleine Fische; er sprach
den Segen über sie und ließ auch diese austeilen. \bibleverse{8} So aßen
sie denn und wurden satt; dann sammelten sie die übriggebliebenen
Stücke\textless sup title=``oder: Brocken''\textgreater✲, sieben Körbe
voll. \bibleverse{9} Es waren aber gegen viertausend Menschen, (die
gegessen hatten und) die er nun gehen ließ.

\hypertarget{jesu-abweisung-der-zeichenforderung-der-pharisuxe4er}{%
\subsubsection{15. Jesu Abweisung der Zeichenforderung der
Pharisäer}\label{jesu-abweisung-der-zeichenforderung-der-pharisuxe4er}}

\bibleverse{10} Er stieg hierauf sogleich mit seinen Jüngern in das Boot
ein und gelangte in die Gegend von Dalmanutha. \bibleverse{11} Da kamen
die Pharisäer zu ihm hinaus, begannen mit ihm zu verhandeln und
verlangten von ihm ein Wunderzeichen vom Himmel her, weil sie ihn auf
die Probe stellen wollten. \bibleverse{12} Da seufzte er in seinem
Geiste tief auf und sagte: »Wozu verlangt dieses Geschlecht ein Zeichen?
Wahrlich ich sage euch: Nimmermehr wird diesem Geschlecht ein Zeichen
gegeben werden!« \bibleverse{13} Mit diesen Worten ließ er sie stehen,
stieg wieder in das Boot ein und fuhr an das jenseitige Ufer hinüber.

\hypertarget{warnung-vor-dem-sauerteig-der-pharisuxe4er-und-vor-dem-des-herodes}{%
\subsubsection{16. Warnung vor dem Sauerteig der Pharisäer und vor dem
des
Herodes}\label{warnung-vor-dem-sauerteig-der-pharisuxe4er-und-vor-dem-des-herodes}}

\bibleverse{14} Sie\textless sup title=``d.h. die Jünger''\textgreater✲
hatten aber vergessen, Brote mitzunehmen, und hatten nur ein einziges
Brot bei sich im Boot. \bibleverse{15} Da sagte er warnend zu ihnen:
»Gebt acht, hütet euch vor dem Sauerteig der Pharisäer und vor dem
Sauerteig des Herodes!« \bibleverse{16} Da erwogen sie im Gespräch
miteinander: »(Das sagt er deshalb) weil wir keine Brote haben.«
\bibleverse{17} Als Jesus das merkte, sagte er zu ihnen: »Was macht ihr
euch Gedanken darüber, daß ihr keine Brote (mitgenommen) habt? Begreift
und versteht ihr denn immer noch nicht? Habt ihr auch jetzt noch ein
Herz, das verhärtet ist? \bibleverse{18} Ihr habt Augen und seht nicht,
habt Ohren und hört nicht?\textless sup title=``Jer 5,21; Hes
12,2''\textgreater✲ Denkt ihr denn nicht daran: \bibleverse{19} Als ich
die fünf Brote für die Fünftausend brach, wie viele Körbe voll Brocken
habt ihr da aufgelesen?« Sie antworteten ihm: »Zwölf.« \bibleverse{20}
»Und damals als ich die sieben (Brote) für die Viertausend (brach), wie
viele Körbchen habt ihr da mit den Brocken gefüllt, die ihr aufgelesen
hattet?« Sie antworteten: »Sieben.« \bibleverse{21} Da sagte er zu
ihnen: »Habt ihr immer noch kein Verständnis?«

\hypertarget{blindenheilung-in-bethsaida}{%
\subsubsection{17. Blindenheilung in
Bethsaida}\label{blindenheilung-in-bethsaida}}

\bibleverse{22} Sie kamen dann nach Bethsaida. Dort brachte man einen
Blinden zu ihm und bat ihn, er möchte ihn anrühren. \bibleverse{23} Er
faßte denn auch den Blinden bei der Hand und führte ihn vor das Dorf
hinaus; dann tat er ihm Speichel in die Augen, legte ihm die Hände auf
und fragte ihn, ob er etwas sähe. \bibleverse{24} Jener schlug die Augen
auf und antwortete: »Ich nehme die Menschen wahr: sie kommen mir bei
ihrem Umhergehen wie Bäume vor.« \bibleverse{25} Darauf legte er ihm die
Hände nochmals auf die Augen; da konnte er deutlich sehen und war
geheilt, so daß er auch in der Ferne alles scharf sah. \bibleverse{26}
Nun schickte Jesus ihn heim in sein Haus mit der Weisung: »Gehe auch
nicht (erst wieder) in das Dorf hinein!«

\hypertarget{iv.-jesus-bereitet-die-juxfcnger-auf-sein-leiden-und-sterben-vor-827-1045}{%
\subsection{IV. Jesus bereitet die Jünger auf sein Leiden und Sterben
vor
(8,27-10,45)}\label{iv.-jesus-bereitet-die-juxfcnger-auf-sein-leiden-und-sterben-vor-827-1045}}

\hypertarget{das-messiasbekenntnis-des-petrus-bei-cuxe4sarea-philippi}{%
\subsubsection{1. Das Messiasbekenntnis des Petrus bei Cäsarea
Philippi}\label{das-messiasbekenntnis-des-petrus-bei-cuxe4sarea-philippi}}

\bibleverse{27} Jesus zog dann mit seinen Jüngern weiter in die
Ortschaften bei Cäsarea Philippi. Unterwegs richtete er an seine Jünger
die Frage: »Für wen halten mich die Leute?« \bibleverse{28} Sie
antworteten ihm: »Für Johannes den Täufer, andere für Elia, noch andere
für sonst einen von den (alten) Propheten.« \bibleverse{29} Nun fragte
er sie weiter: »Ihr aber -- für wen haltet ihr mich?« Petrus gab ihm zur
Antwort: »Du bist Christus\textless sup title=``=~der Messias; vgl. Mt
1,16''\textgreater✲!« \bibleverse{30} Da gab er ihnen die strenge
Weisung, sie sollten das zu niemand von ihm sagen.

\hypertarget{jesu-erste-leidensankuxfcndigung}{%
\subsubsection{2. Jesu erste
Leidensankündigung}\label{jesu-erste-leidensankuxfcndigung}}

\bibleverse{31} Hierauf begann er sie darauf hinzuweisen, der
Menschensohn müsse vieles leiden und von den Ältesten, den
Hohenpriestern und den Schriftgelehrten verworfen und getötet werden und
nach drei Tagen auferstehen; \bibleverse{32} und er sprach das ganz
offen aus. Da nahm Petrus ihn beiseite und begann auf ihn einzureden;
\bibleverse{33} er aber wandte sich um, blickte seine Jünger an und wies
Petrus scharf ab mit den Worten: »Hinweg von mir, Satan! (Tritt) hinter
mich! Deine Gedanken sind nicht die Gedanken Gottes, sondern sind
Menschengedanken.«

\hypertarget{spruxfcche-uxfcber-die-leidensnachfolge-der-juxfcnger}{%
\subsubsection{3. Sprüche über die Leidensnachfolge der
Jünger}\label{spruxfcche-uxfcber-die-leidensnachfolge-der-juxfcnger}}

\bibleverse{34} Dann rief er die Volksmenge samt seinen Jüngern herbei
und sagte zu ihnen: »Will jemand mir nachfolgen, so verleugne er sich
selbst und nehme sein Kreuz auf sich, und so werde er mein Nachfolger.
\bibleverse{35} Denn wer sein Leben retten will, der wird es verlieren;
wer aber sein Leben um meinetwillen und um der Heilsbotschaft willen
verliert, der wird es retten. \bibleverse{36} Denn was hülfe es einem
Menschen, wenn er die ganze Welt gewönne und doch sein Leben einbüßte?
\bibleverse{37} Denn was könnte ein Mensch als Gegenwert\textless sup
title=``=~Entgelt oder: Lösegeld''\textgreater✲ für sein Leben geben?
\bibleverse{38} Denn wer sich meiner und meiner Worte unter diesem
ehebrecherischen\textless sup title=``=~von Gott
abtrünnigen''\textgreater✲ und sündigen Geschlecht schämt, dessen wird
sich auch der Menschensohn schämen, wenn er in der Herrlichkeit seines
Vaters mit den heiligen Engeln kommt.«

\hypertarget{section-8}{%
\section{9}\label{section-8}}

\bibleverse{1} Dann fuhr er fort: »Wahrlich ich sage euch: Einige von
denen, die hier stehen, werden den Tod nicht schmecken, bis sie das
Reich Gottes mit Macht haben kommen sehen.«

\hypertarget{jesu-verkluxe4rung-auf-dem-berge-und-sein-gespruxe4ch-mit-den-juxfcngern-beim-abstieg}{%
\subsubsection{4. Jesu Verklärung auf dem Berge und sein Gespräch mit
den Jüngern beim
Abstieg}\label{jesu-verkluxe4rung-auf-dem-berge-und-sein-gespruxe4ch-mit-den-juxfcngern-beim-abstieg}}

\bibleverse{2} Sechs Tage später nahm Jesus den Petrus, Jakobus und
Johannes mit sich und führte sie abseits auf einen hohen Berg hinauf, wo
sie allein waren. Da wurde er vor ihren Augen verwandelt, \bibleverse{3}
und seine Kleider wurden glänzend, ganz weiß, wie sie kein Walker auf
Erden so weiß machen kann. \bibleverse{4} Und es erschien ihnen Elia mit
Mose; die besprachen sich mit Jesus. \bibleverse{5} Da sagte Petrus zu
Jesus: »Rabbi\textless sup title=``oder: Meister''\textgreater✲, hier
sind wir gut aufgehoben!\textless sup title=``vgl. Mt
17,4''\textgreater✲ Wir wollen hier drei Hütten bauen, eine für dich,
eine für Mose und eine für Elia«~-- \bibleverse{6} er wußte nämlich
nicht, was er sagen sollte; in solche Furcht waren sie geraten.
\bibleverse{7} Dann kam eine Wolke, die sie überschattete, und eine
Stimme erscholl aus der Wolke: »Dieser ist mein geliebter Sohn: höret
auf ihn!«\textless sup title=``5.Mose 18,15''\textgreater✲
\bibleverse{8} Und mit einemmal, als sie um sich blickten, sahen sie
niemand mehr bei sich als Jesus allein.

\bibleverse{9} Als sie dann von dem Berge hinabstiegen, gebot er ihnen,
sie sollten von dem, was sie gesehen hätten, niemand etwas erzählen, ehe
nicht der Menschensohn von den Toten auferstanden wäre. \bibleverse{10}
Dies Gebot hielten sie dann auch fest, besprachen sich jedoch
untereinander darüber, was wohl mit der Auferstehung von den Toten
gemeint sei. \bibleverse{11} Hierauf befragten sie ihn: »Die
Schriftgelehrten behaupten ja, Elia müsse zuerst kommen.«
\bibleverse{12} Er antwortete ihnen: »Ja, Elia kommt allerdings zuerst
und bringt alles wieder in den rechten Stand\textless sup title=``Mal
3,23''\textgreater✲. Doch wie kann denn über den Menschensohn
geschrieben stehen, daß er vieles leiden und daß er verworfen werden
müsse?\textless sup title=``Jes 53,3''\textgreater✲ \bibleverse{13} Aber
ich sage euch: Elia ist wirklich gekommen; doch sie haben ihm angetan,
was ihnen beliebte, wie über ihn geschrieben steht.«\textless sup
title=``1.Kön 19,2.10''\textgreater✲.

\hypertarget{heilung-eines-fallsuxfcchtigen-knaben-die-unfuxe4higkeit-der-juxfcnger}{%
\subsubsection{5. Heilung eines fallsüchtigen Knaben; die Unfähigkeit
der
Jünger}\label{heilung-eines-fallsuxfcchtigen-knaben-die-unfuxe4higkeit-der-juxfcnger}}

\bibleverse{14} Als sie dann zu den (anderen) Jüngern zurückkamen, sahen
sie eine große Volksmenge um sie versammelt, auch Schriftgelehrte, die
sich mit ihnen besprachen. \bibleverse{15} Sobald nun die Menge ihn
erblickte, gerieten alle in freudige Erregung; sie eilten auf ihn zu und
begrüßten ihn. \bibleverse{16} Er fragte sie nun: »Was habt ihr mit
ihnen zu verhandeln?« \bibleverse{17} Da antwortete ihm einer aus der
Menge: »Meister, ich habe meinen Sohn zu dir gebracht, der von einem
sprachlosen Geist besessen ist; \bibleverse{18} sooft der ihn packt,
reißt er ihn hin und her; dann tritt ihm der Schaum vor den Mund, und er
knirscht mit den Zähnen und wird ganz kraftlos. Ich habe deine Jünger
gebeten, sie möchten ihn austreiben, doch sie haben es nicht gekonnt.«
\bibleverse{19} Jesus antwortete ihnen mit den Worten: »O ihr ungläubige
Art von Menschen! Wie lange soll ich noch bei euch sein? Wie lange soll
ich es noch mit euch aushalten? Bringt ihn her zu mir!« \bibleverse{20}
Da brachten sie ihn zu ihm. Als nun der Geist ihn\textless sup
title=``d.h. Jesus''\textgreater✲ erblickte, zog er den Knaben sogleich
krampfhaft zusammen, so daß er auf den Boden fiel und sich mit Schaum
vor dem Munde wälzte. \bibleverse{21} Da fragte Jesus den Vater des
Knaben: »Wie lange hat er dies Leiden schon?« Er antwortete: »Von
Kindheit an; \bibleverse{22} und oft hat der Geist ihn sogar ins Feuer
und ins Wasser gestürzt, um ihn umzubringen. Wenn du es jedoch irgend
vermagst, so hilf uns und habe Erbarmen mit uns!« \bibleverse{23} Jesus
antwortete ihm: »Was dein ›Wenn du es vermagst‹ betrifft, so wisse:
Alles ist dem möglich, der Glauben hat.« \bibleverse{24} Sogleich rief
der Vater des Knaben laut aus: »Ich glaube: hilf meinem Unglauben!«
\bibleverse{25} Als Jesus nun sah, daß immer mehr Leute zusammenliefen,
bedrohte er den unreinen Geist mit den Worten: »Du sprachloser und
tauber Geist, ich gebiete dir: Fahre von ihm aus und fahre nicht wieder
in ihn hinein!« \bibleverse{26} Da schrie er laut auf und fuhr unter
heftigen Krämpfen aus; und der Knabe lag wie tot da, so daß die meisten
sagten: »Er ist gestorben!« \bibleverse{27} Jesus aber faßte ihn bei der
Hand und richtete ihn in die Höhe: da stand er auf.~-- \bibleverse{28}
Als Jesus dann in ein Haus eingetreten\textless sup title=``oder: nach
Hause gekommen''\textgreater✲ war, fragten ihn seine Jünger, während sie
mit ihm allein waren: »Warum haben wir den Geist nicht austreiben
können?« \bibleverse{29} Er antwortete ihnen: »Diese Art (von bösen
Geistern) läßt sich nur durch Gebet austreiben.«

\hypertarget{zweite-leidensankuxfcndigung}{%
\subsubsection{6. Zweite
Leidensankündigung}\label{zweite-leidensankuxfcndigung}}

\bibleverse{30} Sie zogen dann von dort weiter und wanderten durch
Galiläa hin, und er wünschte, daß niemand es erfahren möchte;
\bibleverse{31} denn er erteilte seinen Jüngern Unterricht und sagte zu
ihnen: »Der Menschensohn wird in die Hände der Menschen überliefert; sie
werden ihn töten, und drei Tage nach seiner Tötung wird er auferstehen.«
\bibleverse{32} Sie aber verstanden den Ausspruch nicht, scheuten sich
jedoch, ihn deswegen zu befragen.

\hypertarget{gespruxe4che-mit-den-juxfcngern-in-kapernaum-weisungen-fuxfcr-den-juxfcngerkreis}{%
\subsubsection{7. Gespräche mit den Jüngern in Kapernaum; Weisungen für
den
Jüngerkreis}\label{gespruxe4che-mit-den-juxfcngern-in-kapernaum-weisungen-fuxfcr-den-juxfcngerkreis}}

\hypertarget{a-rangstreit-der-juxfcnger-jesu-mahnung-zur-demut}{%
\paragraph{a) Rangstreit der Jünger; Jesu Mahnung zur
Demut}\label{a-rangstreit-der-juxfcnger-jesu-mahnung-zur-demut}}

\bibleverse{33} So kamen sie nach Kapernaum; und als er zu Hause
(angelangt) war, fragte er sie: »Worüber habt ihr unterwegs gesprochen?«
\bibleverse{34} Sie aber schwiegen; denn sie hatten unterwegs
miteinander darüber gesprochen, wer (von ihnen) der Größte sei.
\bibleverse{35} Da setzte er sich, rief die Zwölf herbei und sagte zu
ihnen: »Wenn jemand der Erste sein will, muß er von allen der Letzte und
der Diener aller sein!« \bibleverse{36} Dann nahm er ein Kind, stellte
es mitten unter sie, schloß es in seine Arme und sagte zu ihnen:
\bibleverse{37} »Wer eins von solchen Kindern auf meinen Namen
hin\textless sup title=``=~um meines Namens willen''\textgreater✲
aufnimmt, der nimmt mich auf; und wer mich aufnimmt, der nimmt nicht
mich auf, sondern den, der mich gesandt hat.«\textless sup title=``Mt
10,40; Joh 13,20''\textgreater✲

\hypertarget{b-belehrung-uxfcber-die-duldsamkeit}{%
\paragraph{b) Belehrung über die
Duldsamkeit}\label{b-belehrung-uxfcber-die-duldsamkeit}}

\bibleverse{38} Da sagte Johannes zu ihm: »Meister, wir haben einen, der
nicht mit uns dir nachfolgt, unter Anwendung deines Namens böse Geister
austreiben sehen und haben es ihm untersagt, weil er uns nicht
nachfolgt.« \bibleverse{39} Jesus aber erwiderte ihm: »Untersagt es ihm
nicht; denn so leicht wird niemand, der ein Wunder unter Benutzung
meines Namens vollführt, dazu kommen, Böses von mir zu reden.
\bibleverse{40} Denn wer nicht gegen uns ist, der ist für
uns\textless sup title=``Mt 12,30; Lk 11,23''\textgreater✲.~--
\bibleverse{41} Denn wenn jemand euch im Hinblick darauf, daß ihr
Christus angehört, auch nur einen Becher Wasser zu trinken gibt, --
wahrlich ich sage euch: Es wird ihm nicht unbelohnt
bleiben!«\textless sup title=``Mt 10,42''\textgreater✲

\hypertarget{c-warnung-vor-der-verfuxfchrung-zum-unglauben-und-zur-suxfcnde-spruxfcche-vom-salz}{%
\paragraph{c) Warnung vor der Verführung (zum Unglauben und zur Sünde);
Sprüche vom
Salz}\label{c-warnung-vor-der-verfuxfchrung-zum-unglauben-und-zur-suxfcnde-spruxfcche-vom-salz}}

\bibleverse{42} »Und wer einen von diesen Kleinen\textless sup
title=``oder: geringen Leuten''\textgreater✲, die (an mich) glauben,
ärgert, für den wäre es das beste, wenn ihm ein Mühlstein um den Hals
gehängt und er ins Meer geworfen wäre. \bibleverse{43} Und wenn deine
Hand dich ärgert\textless sup title=``=~zum Bösen verführen
will''\textgreater✲, so haue sie ab! Es ist besser für dich, verstümmelt
in das Leben einzugehen, als daß du deine beiden Hände hast und in die
Hölle kommst, in das unauslöschliche Feuer. \bibleverse{44}
\bibleverse{45} Und wenn dein Fuß dich ärgert\textless sup title=``=~zum
Bösen verführen will''\textgreater✲, so haue ihn ab! Es ist besser für
dich, als Lahmer in das Leben einzugehen, als daß du deine beiden Füße
hast und in die Hölle geworfen wirst. \bibleverse{46} \bibleverse{47}
Und wenn dein Auge dich ärgert\textless sup title=``=~zum Bösen
verführen will''\textgreater✲, so reiße es aus! Es ist besser für dich,
einäugig in das Reich Gottes einzugehen, als daß du beide Augen hast und
in die Hölle geworfen wirst, \bibleverse{48} wo ihr Wurm nicht stirbt
und das Feuer nicht erlischt\textless sup title=``Jes
66,24''\textgreater✲.~-- \bibleverse{49} Denn jeder wird mit Feuer
gesalzen werden {[}wie jedes Schlachtopfer mit Salz gewürzt wird{]}.
\bibleverse{50} Das Salz ist etwas Gutes; wenn aber das Salz fade✲
geworden ist, wodurch wollt ihr ihm die Würzkraft
wiedergeben?\textless sup title=``Mt 5,13; Lk 14,34''\textgreater✲ Habt
Salz in euch und haltet Frieden untereinander.«

\hypertarget{jesus-in-juduxe4a-und-im-ostjordanlande-gespruxe4che-uxfcber-die-ehe-und-uxfcber-ehescheidung}{%
\subsubsection{8. Jesus in Judäa und im Ostjordanlande; Gespräche über
die Ehe und über
Ehescheidung}\label{jesus-in-juduxe4a-und-im-ostjordanlande-gespruxe4che-uxfcber-die-ehe-und-uxfcber-ehescheidung}}

\hypertarget{section-9}{%
\section{10}\label{section-9}}

\bibleverse{1} Jesus brach dann von dort auf und kam in das Gebiet von
Judäa, und zwar in das Ostjordanland; und wieder strömte das Volk in
Scharen bei ihm zusammen, und wieder lehrte er sie, wie es seine
Gewohnheit war. \bibleverse{2} Da traten Pharisäer an ihn heran und
fragten ihn, ob ein Ehemann seine Frau entlassen\textless sup
title=``oder: sich von seiner Frau scheiden lassen''\textgreater✲ dürfe;
sie wollten ihn nämlich versuchen. \bibleverse{3} Er aber gab ihnen zur
Antwort: »Was hat Mose euch geboten?« \bibleverse{4} Sie sagten: »Mose
hat gestattet, einen Scheidebrief auszustellen und dann (die Frau) zu
entlassen.«\textless sup title=``5.Mose 24,1''\textgreater✲
\bibleverse{5} Jesus aber sagte zu ihnen: »Mit Rücksicht auf eure
Herzenshärte hat er euch dieses Gebot vorgeschrieben; \bibleverse{6}
aber vom Anfang der Schöpfung an hat Gott die Menschen als Mann und Weib
geschaffen\textless sup title=``1.Mose 1,27''\textgreater✲.
\bibleverse{7} Darum wird ein Mann seinen Vater und seine Mutter
verlassen und seinem Weibe anhangen, \bibleverse{8} und die beiden
werden zu einem Leibe werden\textless sup title=``1.Mose
2,24''\textgreater✲, so daß sie nicht mehr zwei sind, sondern ein Leib.
\bibleverse{9} Was nun Gott zusammengefügt hat, das soll der Mensch
nicht scheiden.«

\bibleverse{10} Zu Hause befragten ihn dann seine Jünger nochmals
hierüber, \bibleverse{11} und er erklärte ihnen: »Wer seine Frau
entläßt\textless sup title=``oder: sich von seiner Frau
scheidet''\textgreater✲ und eine andere heiratet, begeht
ihr\textless sup title=``d.h. der ersten Frau''\textgreater✲ gegenüber
Ehebruch, \bibleverse{12} und (ebenso) wenn sie sich von ihrem Manne
scheidet und einen andern heiratet, so begeht sie
Ehebruch.«\textless sup title=``Mt 5,32''\textgreater✲

\hypertarget{jesus-segnet-die-kinder}{%
\subsubsection{9. Jesus segnet die
Kinder}\label{jesus-segnet-die-kinder}}

\bibleverse{13} Und man brachte Kinder zu ihm, damit er sie anrühre; die
Jünger aber verwiesen es ihnen\textless sup title=``d.h. denen, die sie
brachten''\textgreater✲ in barscher Weise. \bibleverse{14} Als Jesus das
sah, wurde er unwillig und sagte zu seinen Jüngern: »Laßt die Kinder zu
mir kommen, hindert sie nicht daran! Denn für ihresgleichen ist das
Reich Gottes bestimmt. \bibleverse{15} Wahrlich ich sage euch: Wer das
Reich Gottes nicht annimmt wie ein Kind, wird sicherlich nicht
hineinkommen!« \bibleverse{16} Dann schloß er sie in seine Arme und
segnete sie, indem er ihnen die Hände auflegte.

\hypertarget{jesu-gespruxe4ch-mit-dem-reichen-und-sein-hinweis-auf-die-gefahr-des-reichtums}{%
\subsubsection{10. Jesu Gespräch mit dem Reichen und sein Hinweis auf
die Gefahr des
Reichtums}\label{jesu-gespruxe4ch-mit-dem-reichen-und-sein-hinweis-auf-die-gefahr-des-reichtums}}

\bibleverse{17} Als er dann (wieder) aufbrach, um weiterzuwandern, lief
einer auf ihn zu, warf sich vor ihm auf die Knie nieder und fragte ihn:
»Guter Meister, was muß ich tun, um ewiges Leben zu erben\textless sup
title=``=~zu gewinnen''\textgreater✲?« \bibleverse{18} Jesus antwortete
ihm: »Was nennst du mich gut? Niemand ist gut als Gott allein.
\bibleverse{19} Du kennst die Gebote: ›Du sollst nicht töten, nicht
ehebrechen, nicht stehlen, nicht falsches Zeugnis ablegen, keinem das
ihm Zukommende vorenthalten, ehre deinen Vater und deine Mutter!‹«
\bibleverse{20} Jener erwiderte ihm: »Meister, dies alles habe ich von
meiner Jugend an gehalten.« \bibleverse{21} Jesus blickte ihn an, gewann
ihn lieb und sagte zu ihm: »Eins fehlt dir noch: gehe hin, verkaufe
alles, was du besitzest, und gib (den Erlös) den Armen: so wirst du
einen Schatz im Himmel haben; dann komm und folge mir nach!«
\bibleverse{22} Er aber wurde über dies Wort unmutig und ging betrübt
weg; denn er besaß ein großes Vermögen.

\bibleverse{23} Da blickte Jesus rings um sich und sagte zu seinen
Jüngern: »Wie schwer wird es doch für die Begüterten sein, in das Reich
Gottes einzugehen!« \bibleverse{24} Die Jünger waren über diese seine
Worte betroffen, Jesus aber wiederholte seinen Ausspruch nochmals mit
den Worten: »Kinder, wie schwer ist es doch {[}für Menschen, die sich
auf Geld und Gut verlassen{]}, in das Reich Gottes einzugehen!
\bibleverse{25} Es ist leichter, daß ein Kamel durch ein Nadelöhr
hindurchgeht, als daß ein Reicher in das Reich Gottes eingeht.«
\bibleverse{26} Da erschraken sie noch weit mehr und sagten zueinander:
»Ja, wer kann dann gerettet werden?« \bibleverse{27} Jesus blickte sie
an und sagte: »Bei den Menschen ist es unmöglich, nicht aber bei Gott;
denn bei Gott ist alles möglich.«\textless sup title=``1.Mose
18,14''\textgreater✲

\hypertarget{vom-lohn-der-nachfolge-jesu-und-der-entsagung}{%
\subsubsection{11. Vom Lohn der Nachfolge Jesu und der
Entsagung}\label{vom-lohn-der-nachfolge-jesu-und-der-entsagung}}

\bibleverse{28} Da nahm Petrus das Wort und sagte zu ihm: »Siehe, wir
haben alles verlassen und sind dir nachgefolgt.« \bibleverse{29} Jesus
erwiderte: »Wahrlich ich sage euch: Niemand hat Haus oder Brüder und
Schwestern oder Mutter, Vater und Kinder oder Äcker um meinetwillen und
um der Heilsbotschaft willen verlassen, \bibleverse{30} ohne daß er
hundertmal Wertvolleres (wieder) empfängt, nämlich schon jetzt in dieser
Zeitlichkeit Häuser, Brüder und Schwestern, Mütter, Kinder und Äcker
(wenn auch) inmitten von Verfolgungen und in der künftigen Weltzeit
ewiges Leben. \bibleverse{31} Viele Erste aber werden Letzte sein und
die Letzten Erste.«

\hypertarget{aufbruch-nach-jerusalem-dritte-leidensankuxfcndigung-jesu}{%
\subsubsection{12. Aufbruch nach Jerusalem; dritte Leidensankündigung
Jesu}\label{aufbruch-nach-jerusalem-dritte-leidensankuxfcndigung-jesu}}

\bibleverse{32} Sie waren aber auf der Wanderung begriffen, um nach
Jerusalem hinaufzuziehen; Jesus ging ihnen (dabei) voran, und sie waren
darüber erstaunt; die ihm Nachfolgenden aber waren voll Furcht. Da nahm
er die Zwölf nochmals (allein) zu sich und begann mit ihnen von dem
Geschick zu sprechen, das ihm bevorstände: \bibleverse{33} »Seht, wir
ziehen jetzt nach Jerusalem hinauf, und der Menschensohn wird den
Hohenpriestern und Schriftgelehrten ausgeliefert werden; sie werden ihn
zum Tode verurteilen und ihn den Heiden ausliefern; \bibleverse{34} die
werden ihn dann verspotten und anspeien, geißeln und töten; und nach
drei Tagen wird er auferstehen.«

\hypertarget{ehrgeizige-bitte-der-beiden-zebeduxe4ussuxf6hne}{%
\subsubsection{13. Ehrgeizige Bitte der beiden
Zebedäussöhne}\label{ehrgeizige-bitte-der-beiden-zebeduxe4ussuxf6hne}}

\bibleverse{35} Da traten Jakobus und Johannes, die Söhne des Zebedäus,
an ihn heran und sagten zu ihm: »Meister, wir möchten, daß du uns eine
Bitte erfüllst.« \bibleverse{36} Er fragte sie: »Was wünscht ihr von
mir?« \bibleverse{37} Sie antworteten ihm: »Gewähre uns, daß wir in
deiner Herrlichkeit einer zu deiner Rechten und einer zu deiner Linken
sitzen dürfen!« \bibleverse{38} Da sagte Jesus zu ihnen: »Ihr wißt
nicht, um was ihr da bittet. Könnt ihr den Kelch trinken, den ich zu
trinken habe, oder die Taufe erleiden, mit der ich getauft werde?«
\bibleverse{39} Sie antworteten ihm: »Ja, wir können es.« Da sagte Jesus
zu ihnen: »Den Kelch, den ich zu trinken habe, werdet (auch) ihr
trinken, und mit der Taufe, mit der ich getauft werde, werdet ihr auch
getauft werden; \bibleverse{40} aber den Sitz zu meiner Rechten oder zu
meiner Linken habe nicht ich zu verleihen, sondern er wird denen zuteil,
für die er bestimmt ist.«

\bibleverse{41} Als nun die zehn (übrigen Jünger) dies hörten, begann
sich der Unwille über Jakobus und Johannes in ihnen zu regen.
\bibleverse{42} Da rief Jesus sie zu sich und sagte zu ihnen: »Ihr wißt,
daß die, welche als Herrscher der Völker gelten, sich als Herren gegen
sie benehmen und daß ihre Großen sie vergewaltigen. \bibleverse{43} Bei
euch aber darf es nicht so sein, sondern wer unter euch groß werden
möchte, muß euer Diener sein, \bibleverse{44} und wer unter euch der
Erste sein möchte, muß der Knecht aller sein; \bibleverse{45} denn auch
der Menschensohn ist nicht (dazu) gekommen, um sich bedienen zu lassen,
sondern um selbst zu dienen und sein Leben als Lösegeld für viele
hinzugeben.«

\hypertarget{v.-jesu-einzug-in-jerusalem-und-letztes-wirken-1046-1337}{%
\subsection{V. Jesu Einzug in Jerusalem und letztes Wirken
(10,46-13,37)}\label{v.-jesu-einzug-in-jerusalem-und-letztes-wirken-1046-1337}}

\hypertarget{heilung-des-blinden-bartimuxe4us-bei-jericho}{%
\subsubsection{1. Heilung des blinden Bartimäus bei
Jericho}\label{heilung-des-blinden-bartimuxe4us-bei-jericho}}

\bibleverse{46} Sie kamen dann nach Jericho; und als er mit seinen
Jüngern und einer großen Volksmenge aus Jericho hinauszog, saß der Sohn
des Timäus, Bartimäus, ein blinder Bettler, am Wege. \bibleverse{47} Als
dieser hörte, es sei Jesus von Nazareth, begann er laut zu rufen: »Sohn
Davids, Jesus, erbarme dich meiner!« \bibleverse{48} Viele riefen ihm
drohend zu, er solle still sein; doch er rief nur noch lauter: »Sohn
Davids, erbarme dich meiner!« \bibleverse{49} Da blieb Jesus stehen und
sagte: »Ruft ihn her!« So riefen sie denn den Blinden und sagten zu ihm:
»Sei guten Mutes, stehe auf: er ruft dich!« \bibleverse{50} Da warf er
seinen Mantel ab, sprang auf und kam zu Jesus. \bibleverse{51} Dieser
redete ihn mit den Worten an: »Was wünschest du von mir?« Der Blinde
antwortete ihm: »Rabbuni\textless sup title=``d.h. verehrter oder lieber
Meister''\textgreater✲, ich möchte sehen können!« \bibleverse{52} Jesus
sagte zu ihm: »Gehe hin, dein Glaube hat dich gerettet\textless sup
title=``oder: dir Heilung verschafft''\textgreater✲.« Da konnte er
augenblicklich sehen und schloß sich an Jesus auf der Wanderung an.

\hypertarget{jesu-einzug-in-jerusalem}{%
\subsubsection{2. Jesu Einzug in
Jerusalem}\label{jesu-einzug-in-jerusalem}}

\hypertarget{section-10}{%
\section{11}\label{section-10}}

\bibleverse{1} Als sie dann in die Nähe von Jerusalem nach {[}Bethphage
und{]} Bethanien an den Ölberg gekommen waren, sandte er zwei von seinen
Jüngern ab \bibleverse{2} mit der Weisung: »Geht in das Dorf, das dort
vor euch liegt; und sogleich, wenn ihr hineinkommt, werdet ihr ein
Eselsfüllen angebunden finden, auf dem noch nie ein Mensch gesessen hat;
bindet es los und bringt es her! \bibleverse{3} Und wenn jemand euch
fragen sollte: ›Was macht ihr da?‹, so antwortet: ›Der Herr bedarf
seiner und schickt es sogleich wieder her.‹« \bibleverse{4} Da gingen
sie hin und fanden ein Eselsfüllen angebunden am Haustor draußen nach
der Dorfstraße zu und banden es los. \bibleverse{5} Und einige von den
Leuten, die dort standen, sagten zu ihnen: »Was macht ihr da, daß ihr
das Füllen losbindet?« \bibleverse{6} Sie antworteten ihnen, wie Jesus
ihnen geboten hatte, da ließ man sie gewähren. \bibleverse{7} Sie
brachten nun das Füllen zu Jesus und legten ihre Mäntel auf das Tier,
und er setzte sich darauf. \bibleverse{8} Viele breiteten sodann ihre
Mäntel auf den Weg, andere streuten Laubzweige aus, die sie auf den
Feldern abgeschnitten hatten. \bibleverse{9} Und die, welche vorn im
Zuge gingen, und die, welche nachfolgten, riefen laut: »Hosianna!
Gepriesen\textless sup title=``oder: gesegnet''\textgreater✲ sei, der da
kommt im Namen des Herrn!\textless sup title=``Ps
118,25-26''\textgreater✲ \bibleverse{10} Gepriesen\textless sup
title=``oder: gesegnet''\textgreater✲ sei das Königtum unsers Vaters
David, das da kommt! Hosianna in den Himmelshöhen!« \bibleverse{11} So
zog er in Jerusalem ein (und begab sich) in den Tempel; und nachdem er
sich dort alles ringsum angesehen hatte, ging er, da es schon spät am
Tage war, mit den Zwölfen nach Bethanien hinaus.

\hypertarget{verfluchung-eines-unfruchtbaren-feigenbaumes-und-die-tempelreinigung}{%
\subsubsection{3. Verfluchung eines unfruchtbaren Feigenbaumes und die
Tempelreinigung}\label{verfluchung-eines-unfruchtbaren-feigenbaumes-und-die-tempelreinigung}}

\hypertarget{a-die-verfluchung}{%
\paragraph{a) Die Verfluchung}\label{a-die-verfluchung}}

\bibleverse{12} Als sie dann am folgenden Morgen von Bethanien wieder
aufgebrochen waren, hungerte ihn. \bibleverse{13} Als er nun in der
Ferne einen Feigenbaum sah, der Blätter hatte, ging er hin, ob er nicht
einige Früchte an ihm fände, doch als er zu ihm hinkam, fand er nichts
als Blätter, denn es war noch nicht die Feigenzeit. \bibleverse{14} Da
rief er dem Baume die Worte zu: »Nie mehr in Ewigkeit soll jemand eine
Frucht von dir essen!« Und seine Jünger hörten es.

\hypertarget{b-die-tempelreinigung}{%
\paragraph{b) Die Tempelreinigung}\label{b-die-tempelreinigung}}

\bibleverse{15} Sie kamen dann nach Jerusalem, und als er dort in den
Tempel hineingegangen war, machte er sich daran, die, welche im Tempel
verkauften und kauften, hinauszutreiben, stieß die Tische der
Geldwechsler und die Sitze✲ der Taubenhändler um \bibleverse{16} und
duldete nicht, daß jemand ein Hausgerät über den Tempelplatz trug.
\bibleverse{17} Und er belehrte sie mit den Worten: »Steht nicht
geschrieben\textless sup title=``Jes 56,7''\textgreater✲: ›Mein Haus
soll ein Bethaus für alle Völker heißen‹? Ihr aber habt eine
›Räuberhöhle‹ aus ihm gemacht!«\textless sup title=``Jer
7,11''\textgreater✲ \bibleverse{18} Die Hohenpriester und die
Schriftgelehrten hörten davon und überlegten, wie sie ihn
umbringen\textless sup title=``oder: unschädlich machen''\textgreater✲
könnten; denn sie hatten Furcht vor ihm, weil seine Lehre auf das ganze
Volk einen tiefen Eindruck machte.~-- \bibleverse{19} Und sooft es Abend
geworden war, gingen sie\textless sup title=``d.h. Jesus und seine
Jünger''\textgreater✲ aus der Stadt hinaus.

\hypertarget{c-ruxfcckblick-auf-den-verdorrten-feigenbaum-mit-anschlieuxdfendem-hinweis-auf-die-macht-des-glaubens-und-des-gebets-mahnung}{%
\paragraph{c) Rückblick auf den verdorrten Feigenbaum mit anschließendem
Hinweis auf die Macht des Glaubens und des Gebets;
Mahnung}\label{c-ruxfcckblick-auf-den-verdorrten-feigenbaum-mit-anschlieuxdfendem-hinweis-auf-die-macht-des-glaubens-und-des-gebets-mahnung}}

\bibleverse{20} Als sie nun am folgenden Morgen vorübergingen, sahen sie
den Feigenbaum von den Wurzeln aus\textless sup title=``=~bis zu den
Wurzeln hin''\textgreater✲ verdorrt. \bibleverse{21} Da erinnerte sich
Petrus (des Vorfalls) und sagte zu ihm: »Rabbi✲, sieh doch: der
Feigenbaum, den du verflucht hast, ist verdorrt!« \bibleverse{22} Jesus
gab ihnen zur Antwort: »Habt Glauben an Gott! \bibleverse{23} Wahrlich
ich sage euch: Wer zu dem Berge dort sagt: ›Hebe dich empor und stürze
dich ins Meer!‹ und in seinem Herzen nicht zweifelt, sondern glaubt, daß
das, was er ausspricht, in Erfüllung geht, dem wird es auch erfüllt
werden. \bibleverse{24} Darum sage ich euch: Bei allem, was ihr im Gebet
erbittet -- glaubt nur, daß ihr es (tatsächlich) empfangen habt, so wird
es euch zuteil werden. \bibleverse{25} Und wenn ihr dasteht und beten
wollt, so vergebt (zunächst), wenn ihr etwas gegen jemand habt, damit
auch euer himmlischer Vater euch eure Übertretungen vergebe\textless sup
title=``Mt 6,14''\textgreater✲. \bibleverse{26} {[}Wenn aber ihr nicht
vergebt, so wird auch euer himmlischer Vater euch eure Übertretungen
nicht vergeben\textless sup title=``Mt 6,15''\textgreater✲.{]}«

\hypertarget{die-frage-des-hohen-rates-nach-jesu-vollmacht}{%
\subsubsection{4. Die Frage des Hohen Rates nach Jesu
Vollmacht}\label{die-frage-des-hohen-rates-nach-jesu-vollmacht}}

\bibleverse{27} Sie kamen dann wieder nach Jerusalem; und als er dort im
Tempel umherging, traten die Hohenpriester, die Schriftgelehrten und die
Ältesten an ihn heran \bibleverse{28} und fragten ihn: »Auf Grund
welcher Vollmacht trittst du hier in solcher Weise auf? Oder wer hat dir
die Vollmacht\textless sup title=``=~das Recht''\textgreater✲ dazu
gegeben, hier so aufzutreten?« \bibleverse{29} Da antwortete Jesus
ihnen: »Ich will euch eine einzige Frage vorlegen: beantwortet sie mir,
dann will ich euch sagen, auf Grund welcher Vollmacht ich hier so
auftrete. \bibleverse{30} Stammte die Taufe des Johannes vom Himmel oder
von Menschen? Gebt mir eine Antwort!« \bibleverse{31} Da überlegten sie
miteinander folgendermaßen: »Sagen wir: ›Vom Himmel‹, so wird er
einwenden: ›Warum habt ihr ihm dann keinen Glauben geschenkt?‹
\bibleverse{32} Sollen wir dagegen sagen: ›Von Menschen?‹« -- da
fürchteten sie sich vor dem Volk; denn alle glaubten von Johannes, daß
er wirklich ein Prophet gewesen sei. \bibleverse{33} So gaben sie denn
Jesus zur Antwort: »Wir wissen es nicht.« Da erwiderte Jesus ihnen:
»Dann sage auch ich euch nicht, auf Grund welcher Vollmacht ich hier so
auftrete.«

\hypertarget{gleichnis-von-den-treulosen-weinguxe4rtnern}{%
\subsubsection{5. Gleichnis von den treulosen
Weingärtnern}\label{gleichnis-von-den-treulosen-weinguxe4rtnern}}

\hypertarget{section-11}{%
\section{12}\label{section-11}}

\bibleverse{1} Er begann dann in Gleichnissen zu ihnen zu reden: »Ein
Mann legte einen Weinberg an, umgab ihn mit einem Zaun, grub eine Kelter
darin, baute einen Wachtturm, verpachtete ihn an Weingärtner und ging
außer Landes\textless sup title=``Jes 5,1-2''\textgreater✲.
\bibleverse{2} Zu rechter Zeit sandte er dann einen Knecht zu den
Weingärtnern, um seinen Teil der Früchte des Weinbergs von den
Weingärtnern in Empfang zu nehmen. \bibleverse{3} Die aber ergriffen den
Knecht, mißhandelten ihn und schickten ihn mit leeren Händen zurück.
\bibleverse{4} Da sandte er nochmals einen anderen Knecht zu ihnen; auch
diesem zerschlugen sie den Kopf und beschimpften ihn. \bibleverse{5} Er
sandte noch einen anderen, den sie töteten, und noch viele andere
(sandte er), von denen sie die einen mißhandelten, die anderen töteten.
\bibleverse{6} Nun hatte er noch einen einzigen, seinen geliebten Sohn;
den sandte er zuletzt auch noch zu ihnen, weil er dachte: ›Sie werden
sich doch vor meinem Sohne scheuen.‹ \bibleverse{7} Jene Weingärtner
aber sagten zueinander: ›Dieser ist der Erbe; kommt, wir wollen ihn
töten; dann wird das Erbgut uns gehören.‹ \bibleverse{8} So ergriffen
sie ihn denn, schlugen ihn tot und warfen ihn vor den Weinberg hinaus.
\bibleverse{9} Was wird nun der Herr des Weinbergs tun? Er wird kommen
und die Weingärtner umbringen und wird den Weinberg an andere
vergeben.~-- \bibleverse{10} Habt ihr nicht auch dieses Schriftwort
gelesen\textless sup title=``Ps 118,22-23''\textgreater✲: ›Der Stein,
den die Bauleute verworfen haben, der ist zum Eckstein geworden;
\bibleverse{11} durch den Herrn ist er das geworden, und ein Wunder ist
er in unsern Augen?‹« \bibleverse{12} Da hätten sie ihn am liebsten
festgenommen, fürchteten sich jedoch vor dem Volk; sie hatten nämlich
wohl gemerkt, daß er das Gleichnis gegen sie gerichtet hatte. So ließen
sie denn von ihm ab und entfernten sich.

\hypertarget{die-steuerfrage-der-pharisuxe4er-oder-das-gespruxe4ch-vom-zinsgroschen}{%
\subsubsection{6. Die Steuerfrage der Pharisäer (oder das Gespräch vom
›Zinsgroschen‹)}\label{die-steuerfrage-der-pharisuxe4er-oder-das-gespruxe4ch-vom-zinsgroschen}}

\bibleverse{13} Sie sandten darauf einige von den Pharisäern und den
Anhängern des Herodes\textless sup title=``vgl. dazu Mt
22,16''\textgreater✲ zu ihm, um ihn durch einen Ausspruch zu fangen.
\bibleverse{14} Jene kamen also und sagten zu ihm: »Meister, wir wissen,
daß du wahrhaftig bist und auf niemand Rücksicht nimmst; denn du siehst
die Person\textless sup title=``=~das Äußere der Menschen''\textgreater✲
nicht an, sondern lehrst den Weg Gottes mit Wahrhaftigkeit. Ist es
recht, daß man dem Kaiser Steuer entrichtet, oder nicht? Sollen wir sie
entrichten oder nicht?« \bibleverse{15} Da er nun ihre Heuchelei
durchschaute, antwortete er ihnen: »Warum versucht ihr mich? Reicht mir
einen Denar, damit ich ihn ansehe!« \bibleverse{16} Als sie ihm nun
einen (Denar) gereicht hatten, fragte er sie: »Wessen Bild und
Aufschrift ist das hier?« Sie antworteten ihm: »Des Kaisers.«
\bibleverse{17} Da sagte Jesus zu ihnen: »So gebt dem Kaiser, was dem
Kaiser zusteht, und Gott, was Gott zusteht!« Und sie gerieten in Staunen
über ihn.

\hypertarget{die-frage-der-sadduzuxe4er-uxfcber-die-auferstehung-der-toten}{%
\subsubsection{7. Die Frage der Sadduzäer (über die Auferstehung der
Toten)}\label{die-frage-der-sadduzuxe4er-uxfcber-die-auferstehung-der-toten}}

\bibleverse{18} Es traten dann Sadduzäer zu ihm, die da behaupten, es
gebe keine Auferstehung, und legten ihm folgende Frage vor:
\bibleverse{19} »Meister, Mose hat uns vorgeschrieben\textless sup
title=``5.Mose 25,5-10''\textgreater✲: ›Wenn einem sein Bruder stirbt
und eine Frau hinterläßt, aber kein Kind zurückläßt, so soll sein Bruder
die Frau heiraten und für seinen Bruder das Geschlecht fortpflanzen.‹
\bibleverse{20} Nun waren da sieben Brüder; der erste\textless sup
title=``=~der älteste''\textgreater✲ nahm eine Frau, ließ aber bei
seinem Tode keine Kinder zurück. \bibleverse{21} Da heiratete sie der
zweite, starb aber auch, ohne Kinder zu hinterlassen; der dritte ebenso,
\bibleverse{22} und alle sieben hinterließen keine Kinder; zuletzt nach
allen starb auch die Frau. \bibleverse{23} In der Auferstehung nun, wenn
sie auferstehen: wem von ihnen wird sie dann als Frau angehören? Alle
sieben haben sie ja zur Frau gehabt.« \bibleverse{24} Jesus antwortete
ihnen: »Befindet ihr euch nicht deshalb im Irrtum, weil ihr die
(heiligen) Schriften und die Kraft Gottes nicht kennt? \bibleverse{25}
Denn wenn sie von den Toten auferstehen, dann heiraten sie weder, noch
werden sie verheiratet, sondern sie sind wie Engel im Himmel.
\bibleverse{26} Was aber die Auferweckung der Toten betrifft: habt ihr
nicht im Buche Moses bei der Erzählung vom Dornbusch gelesen, wie Gott
zu Mose die Worte gesprochen hat\textless sup title=``2.Mose
3,6''\textgreater✲: ›Ich bin der Gott Abrahams, der Gott Isaaks und der
Gott Jakobs‹? \bibleverse{27} Gott ist doch nicht der Gott von Toten,
sondern von Lebenden. Ihr seid arg im Irrtum!«

\hypertarget{die-frage-eines-schriftgelehrten-nach-dem-vornehmsten-gebot}{%
\subsubsection{8. Die Frage eines Schriftgelehrten nach dem vornehmsten
Gebot}\label{die-frage-eines-schriftgelehrten-nach-dem-vornehmsten-gebot}}

\bibleverse{28} Da trat einer von den Schriftgelehrten hinzu, der ihnen
zugehört hatte, wie sie miteinander verhandelten\textless sup
title=``oder: stritten''\textgreater✲; und da er wußte, daß Jesus ihnen
treffend geantwortet hatte, fragte er ihn: »Welches Gebot ist das erste
von allen?« \bibleverse{29} Jesus antwortete: »Das erste ist: ›Höre,
Israel: der Herr, unser Gott, ist Herr allein, \bibleverse{30} und du
sollst den Herrn, deinen Gott, lieben mit deinem ganzen Herzen, mit
deiner ganzen Seele, mit deinem ganzen Denken und mit aller deiner
Kraft!‹\textless sup title=``5.Mose 6,4-5''\textgreater✲ \bibleverse{31}
An zweiter Stelle steht dieses (Gebot): ›Du sollst deinen Nächsten
lieben wie dich selbst!‹\textless sup title=``3.Mose
19,18''\textgreater✲ Kein anderes Gebot steht höher als diese beiden.«
\bibleverse{32} Da sagte der Schriftgelehrte zu ihm: »Meister, mit Recht
hast du der Wahrheit gemäß gesagt, daß Gott nur einer\textless sup
title=``oder: der einzige''\textgreater✲ ist und es keinen anderen außer
ihm gibt\textless sup title=``5.Mose 4,35; 6,4''\textgreater✲;
\bibleverse{33} und ihn mit ganzem Herzen und aus voller Überzeugung und
mit ganzer Kraft lieben und den Nächsten wie sich selbst lieben, das ist
weit mehr wert als alle Brandopfer und die Opfer
überhaupt.«\textless sup title=``1.Sam 15,22''\textgreater✲
\bibleverse{34} Als Jesus ihn so verständig antworten hörte, sagte er zu
ihm: »Du bist nicht weit vom Reiche Gottes entfernt.« Und niemand wagte
fortan noch, Fragen an ihn zu richten.

\hypertarget{die-gegenfrage-jesu-nach-dem-messias-als-dem-sohne-davids}{%
\subsubsection{9. Die Gegenfrage Jesu nach dem Messias als dem Sohne
Davids}\label{die-gegenfrage-jesu-nach-dem-messias-als-dem-sohne-davids}}

\bibleverse{35} Jesus warf dann, während er im Tempel lehrte, die Frage
auf: »Wie können die Schriftgelehrten behaupten, daß
Christus\textless sup title=``=~der Messias''\textgreater✲ Davids Sohn
sei? \bibleverse{36} David selbst hat doch im heiligen Geist
gesagt\textless sup title=``Ps 110,1''\textgreater✲: ›Der Herr hat zu
meinem Herrn gesagt: Setze dich zu meiner Rechten, bis ich deine Feinde
hinlege zum Schemel deiner Füße.‹ \bibleverse{37} David selbst nennt
ihn\textless sup title=``d.h. den Messias''\textgreater✲ ›Herr‹: wie
kann er da sein Sohn sein?« Und die große Volksmenge hörte ihm gern zu.

\hypertarget{jesu-warnung-vor-dem-ehrgeiz-und-der-habgier-der-schriftgelehrten}{%
\subsubsection{10. Jesu Warnung vor dem Ehrgeiz und der Habgier der
Schriftgelehrten}\label{jesu-warnung-vor-dem-ehrgeiz-und-der-habgier-der-schriftgelehrten}}

\bibleverse{38} Und in seiner Belehrung sagte er: »Hütet euch vor den
Schriftgelehrten, die es lieben, in langen Gewändern einherzugehen und
auf den Märkten\textless sup title=``oder: Straßen''\textgreater✲
gegrüßt zu werden; \bibleverse{39} die die Ehrensitze in den Synagogen
und die obersten Plätze bei den Gastmählern beanspruchen;
\bibleverse{40} die die Häuser der Witwen verschlingen\textless sup
title=``=~gierig an sich bringen''\textgreater✲ und zum Schein lange
Gebete verrichten. Sie werden ein um so strengeres Gericht erfahren.«

\hypertarget{jesu-lob-der-zwei-scherflein-der-armen-witwe}{%
\subsubsection{11. Jesu Lob der zwei Scherflein der armen
Witwe}\label{jesu-lob-der-zwei-scherflein-der-armen-witwe}}

\bibleverse{41} Als er sich dann dem Opferkasten gegenüber hingesetzt
hatte, sah er zu, wie das Volk Geld in den Kasten einwarf, und viele
Reiche taten viel hinein. \bibleverse{42} Da kam auch eine arme Witwe
und legte zwei Scherflein hinein, die einen Pfennig ausmachen.
\bibleverse{43} Da rief er seine Jünger herbei und sagte zu ihnen:
»Wahrlich ich sage euch: Diese arme Witwe hat mehr eingelegt als alle,
die etwas in den Opferkasten getan haben. \bibleverse{44} Denn jene
haben alle von ihrem Überfluß eingelegt, sie aber hat aus ihrer
Dürftigkeit heraus alles, was sie besaß, eingelegt, ihren ganzen
Lebensunterhalt.«

\hypertarget{die-uxf6lbergsrede-jesu-an-seine-juxfcnger-von-der-zerstuxf6rung-des-tempels-vom-ende-dieser-weltzeit-und-von-seiner-erscheinung-am-juxfcngsten-tage}{%
\subsubsection{12. Die Ölbergsrede Jesu an seine Jünger von der
Zerstörung des Tempels, vom Ende dieser Weltzeit und von seiner
Erscheinung am jüngsten
Tage}\label{die-uxf6lbergsrede-jesu-an-seine-juxfcnger-von-der-zerstuxf6rung-des-tempels-vom-ende-dieser-weltzeit-und-von-seiner-erscheinung-am-juxfcngsten-tage}}

\hypertarget{a-einleitung-anlauxdf-der-rede-mit-ankuxfcndigung-der-zerstuxf6rung-des-tempels}{%
\paragraph{a) Einleitung: Anlaß der Rede (mit Ankündigung der Zerstörung
des
Tempels)}\label{a-einleitung-anlauxdf-der-rede-mit-ankuxfcndigung-der-zerstuxf6rung-des-tempels}}

\hypertarget{section-12}{%
\section{13}\label{section-12}}

\bibleverse{1} Als Jesus dann den Tempel verließ, sagte einer von seinen
Jüngern zu ihm: »Meister, sieh einmal: was für Steine und was für ein
Prachtbau ist das!« \bibleverse{2} Da antwortete ihm Jesus: »Ja, jetzt
siehst du dieses gewaltige Bauwerk (noch stehen). Es wird hier (aber)
kein Stein auf dem andern bleiben, der nicht niedergerissen wird!«~--
\bibleverse{3} Als er sich dann am Ölberg dem Tempel gegenüber
niedergesetzt hatte, fragten ihn Petrus, Jakobus, Johannes und Andreas,
als sie für sich allein waren: \bibleverse{4} »Sage uns doch: wann wird
dies geschehen, und welches ist das Zeichen dafür, wann dies alles in
Erfüllung gehen wird?«

\hypertarget{b-das-ende-dieser-weltzeit}{%
\paragraph{b) Das Ende dieser
Weltzeit}\label{b-das-ende-dieser-weltzeit}}

\hypertarget{aa-die-ersten-vorzeichen-der-endzeit}{%
\subparagraph{aa) Die ersten Vorzeichen der
Endzeit}\label{aa-die-ersten-vorzeichen-der-endzeit}}

\bibleverse{5} Da begann Jesus, zu ihnen zu sagen: »Seht euch vor, daß
niemand euch irreführt! \bibleverse{6} Viele werden unter meinem Namen
kommen und sagen: ›Ich bin es‹\textless sup title=``d.h. Christus oder:
der Messias''\textgreater✲ und werden viele irreführen. \bibleverse{7}
Wenn ihr ferner von Kriegen und Kriegsgerüchten hört, so laßt euch
dadurch nicht ängstigen! Dies muß so kommen, bedeutet aber noch nicht
das Ende. \bibleverse{8} Denn ein Volk wird sich gegen das andere
erheben und ein Reich gegen das andere\textless sup title=``Jes
19,2''\textgreater✲; Erdbeben werden hier und da stattfinden,
Hungersnöte werden kommen. \bibleverse{9} a Dies ist (aber erst) der
Anfang der Wehen.«

\hypertarget{bb-die-juxfcngerverfolgung}{%
\subparagraph{bb) Die
Jüngerverfolgung}\label{bb-die-juxfcngerverfolgung}}

b »Gebt ihr jedoch acht auf euch selbst\textless sup title=``d.h. auf
das, was mit euch selbst geschehen wird''\textgreater✲! Man wird euch
vor die Gerichtshöfe stellen und euch in den Synagogen geißeln; auch vor
Statthalter und Könige werdet ihr um meinetwillen gestellt werden ihnen
zum Zeugnis\textless sup title=``d.h. um Zeugnis vor ihnen
abzulegen''\textgreater✲; \bibleverse{10} und unter allen Völkern muß
zuvor die Heilsbotschaft verkündigt werden. \bibleverse{11} Wenn man
euch nun abführt und vor Gericht stellt, so macht euch nicht im voraus
Sorge darüber, was ihr reden sollt, sondern was euch in jener Stunde
eingegeben wird, das redet; nicht ihr seid es ja, die da reden, sondern
der heilige Geist. \bibleverse{12} Es wird aber ein Bruder den anderen
zum Tode überliefern und der Vater seinen Sohn, und Kinder werden gegen
ihre Eltern auftreten und sie zum Tode bringen\textless sup title=``Mi
7,6''\textgreater✲, \bibleverse{13} und ihr werdet allen verhaßt sein um
meines Namens willen. Wer aber bis ans Ende ausharrt, der wird gerettet
werden.«

\hypertarget{cc-der-huxf6hepunkt-der-drangsal-in-juduxe4a}{%
\subparagraph{cc) Der Höhepunkt der Drangsal in
Judäa}\label{cc-der-huxf6hepunkt-der-drangsal-in-juduxe4a}}

\bibleverse{14} »Wenn ihr aber den ›Greuel der Verwüstung‹\textless sup
title=``=~der Entweihung''\textgreater✲ da stehen seht, wo er nicht
stehen darf\textless sup title=``Dan 9,27; 11,31; 12,11''\textgreater✲
-- der Leser merke auf! --, dann sollen die (Gläubigen), welche in Judäa
sind, in die Berge fliehen. \bibleverse{15} Wer sich alsdann auf dem
Dache befindet, steige nicht erst (ins Haus) hinab und gehe nicht
hinein, um noch etwas aus seinem Hause zu holen; \bibleverse{16} und wer
auf dem Felde ist, kehre nicht zurück, um noch seinen Mantel zu holen.
\bibleverse{17} Wehe aber den Frauen, die in jenen Tagen guter Hoffnung
sind, und denen, die ein Kind zu nähren haben! \bibleverse{18} Betet
aber auch, daß dies nicht zur Winterszeit\textless sup title=``vgl. Joh
10,22''\textgreater✲ eintrete! \bibleverse{19} Denn jene Tage werden
eine Drangsalszeit sein, wie eine solche seit dem Anfang, als Gott die
Welt schuf, bis jetzt noch nicht dagewesen ist und wie auch keine je
wieder kommen wird\textless sup title=``Dan 12,1''\textgreater✲.
\bibleverse{20} Und wenn der Herr diese Tage nicht verkürzt hätte, so
würde kein Fleisch✲ gerettet werden; aber um der Auserwählten willen,
die er erwählt hat, hat er diese Tage verkürzt.«

\hypertarget{dd-weissagung-bezuxfcglich-der-falschen-propheten}{%
\subparagraph{dd) Weissagung bezüglich der falschen
Propheten}\label{dd-weissagung-bezuxfcglich-der-falschen-propheten}}

\bibleverse{21} »Wenn alsdann jemand zu euch sagt: ›Seht, hier ist
Christus\textless sup title=``=~der Messias; vgl. Mt
1,16''\textgreater✲; seht, dort ist er!‹, so glaubt es nicht!
\bibleverse{22} Denn es werden falsche Christusse\textless sup
title=``oder: Messiasse''\textgreater✲ und falsche Propheten auftreten
und werden Zeichen und Wunder tun\textless sup title=``5.Mose
13,2''\textgreater✲, um womöglich die Erwählten irrezuführen.
\bibleverse{23} Seht ihr euch aber vor! Ich habe euch alles
vorhergesagt.«

\hypertarget{c-die-letzten-vorzeichen-und-die-erscheinung-des-menschensohnes-am-juxfcngsten-tage}{%
\paragraph{c) Die letzten Vorzeichen und die Erscheinung des
Menschensohnes am Jüngsten
Tage}\label{c-die-letzten-vorzeichen-und-die-erscheinung-des-menschensohnes-am-juxfcngsten-tage}}

\bibleverse{24} »In jenen Tagen aber, nach jener Drangsalszeit, wird die
Sonne sich verfinstern und der Mond seinen Schein verlieren\textless sup
title=``Jes 13,10; 34,4''\textgreater✲; \bibleverse{25} die Sterne
werden vom Himmel fallen und die Kräfte am Himmel in Erschütterung
geraten\textless sup title=``Jes 34,4''\textgreater✲. \bibleverse{26}
Und dann wird man den Menschensohn in Wolken kommen sehen mit großer
Macht und Herrlichkeit\textless sup title=``Dan 7,13''\textgreater✲;
\bibleverse{27} und dann wird er die Engel aussenden und seine Erwählten
von den vier Windrichtungen her versammeln vom Ende der Erde bis zum
Ende des Himmels\textless sup title=``Sach 2,6; 5.Mose
30,4''\textgreater✲.

\bibleverse{28} Vom Feigenbaum aber mögt ihr das Gleichnis lernen✲:
Sobald seine Zweige saftig werden und Blätter hervorsprossen, so erkennt
ihr daran, daß der Sommer nahe ist. \bibleverse{29} Ebenso auch ihr:
wenn ihr dies alles eintreten seht, so erkennet daran, daß
es\textless sup title=``oder: er, d.h. der Menschensohn''\textgreater✲
nahe vor der Tür steht. \bibleverse{30} Wahrlich ich sage euch: Dieses
Geschlecht wird nicht vergehen, bis dies alles geschieht.
\bibleverse{31} Der Himmel und die Erde werden vergehen, aber meine
Worte werden nimmermehr vergehen! \bibleverse{32} Von jenem Tage aber
und jener Stunde hat niemand Kenntnis, auch die Engel im Himmel nicht,
auch der Sohn nicht, niemand außer dem Vater.«

\hypertarget{d-schluuxdfermahnung-an-die-juxfcnger-zur-wachsamkeit}{%
\paragraph{d) Schlußermahnung an die Jünger zur
Wachsamkeit}\label{d-schluuxdfermahnung-an-die-juxfcnger-zur-wachsamkeit}}

\bibleverse{33} »Haltet die Augen offen, seid wachsam! Denn ihr wißt
nicht, wann der Zeitpunkt da ist. \bibleverse{34} Wie ein Mann, der auf
Reisen geht, beim Verlassen seines Hauses seinen Knechten die Vollmacht
übergibt\textless sup title=``=~die Verantwortung
überläßt''\textgreater✲ und einem jeden sein Geschäft (zuweist) und dem
Türhüter gebietet, wachsam zu sein,~-- \bibleverse{35} so wachet also!
Denn ihr wißt nicht, wann der Herr des Hauses kommt, ob spät am Abend
oder um Mitternacht oder beim Hahnenschrei oder erst frühmorgens:
\bibleverse{36} daß er nur nicht, wenn er unvermutet kommt, euch im
Schlaf findet! \bibleverse{37} Was ich aber euch sage, das sage ich
allen: wachet\textless sup title=``oder: seid wachsam''\textgreater✲!«

\hypertarget{vi.-jesu-leiden-und-sterben-kap.-14-15}{%
\subsection{VI. Jesu Leiden und Sterben (Kap.
14-15)}\label{vi.-jesu-leiden-und-sterben-kap.-14-15}}

\hypertarget{mordanschlag-der-fuxfchrer-des-volkes}{%
\subsubsection{1. Mordanschlag der Führer des
Volkes}\label{mordanschlag-der-fuxfchrer-des-volkes}}

\hypertarget{section-13}{%
\section{14}\label{section-13}}

\bibleverse{1} Es stand aber in zwei Tagen das Passahfest und die (Tage
der) ungesäuerten Brote bevor. Da überlegten die Hohenpriester und die
Schriftgelehrten, auf welche Weise sie Jesus mit List festnehmen und
töten könnten; \bibleverse{2} denn sie sagten: »Nur nicht während des
Festes (selbst), damit keine Unruhen im Volk entstehen!«

\hypertarget{salbung-jesu-in-bethanien}{%
\subsubsection{2. Salbung Jesu in
Bethanien}\label{salbung-jesu-in-bethanien}}

\bibleverse{3} Als nun Jesus in Bethanien im Hause Simons des
(einstmals) Aussätzigen war, kam, während er bei Tische saß, eine Frau,
die ein Alabasterfläschchen mit echtem, kostbarem Nardensalböl hatte;
sie zerbrach das Gefäß und goß es ihm über das Haupt. \bibleverse{4}
Darüber wurden einige (der Anwesenden) unwillig und sagten zueinander:
»Wozu hat diese Verschwendung des Salböls stattgefunden? \bibleverse{5}
Dieses Salböl hätte man ja für mehr als dreihundert Denare✲ verkaufen
und (den Erlös) den Armen geben können«; und sie machten der Frau laute
Vorwürfe. \bibleverse{6} Da sagte Jesus: »Laßt sie in Ruhe! Warum
bekümmert ihr sie? Sie hat ein gutes Werk an mir getan! \bibleverse{7}
Denn die Armen habt ihr allezeit bei euch und könnt ihnen Gutes tun,
sooft ihr wollt; mich aber habt ihr nicht allezeit. \bibleverse{8} Sie
hat getan, was in ihren Kräften stand: sie hat meinen Leib im voraus zur
Bestattung gesalbt. \bibleverse{9} Wahrlich ich sage euch: Überall, wo
die Heilsbotschaft in der ganzen Welt verkündet werden wird, da wird man
auch von dem sprechen, was diese Frau getan hat, zu ihrem Gedächtnis.«

\hypertarget{verrat-des-judas}{%
\subsubsection{3. Verrat des Judas}\label{verrat-des-judas}}

\bibleverse{10} Da ging Judas Iskariot, der eine von den Zwölfen, zu den
Hohenpriestern, um ihnen Jesus in die Hände zu liefern. \bibleverse{11}
Als sie das hörten, freuten sie sich und versprachen, ihm Geld zu geben.
Darauf suchte er nach einer guten Gelegenheit, um ihn zu
überantworten\textless sup title=``oder: zu verraten''\textgreater✲.

\hypertarget{zuruxfcstung-des-passahmahles}{%
\subsubsection{4. Zurüstung des
Passahmahles}\label{zuruxfcstung-des-passahmahles}}

\bibleverse{12} Am ersten Tag der ungesäuerten Brote aber, an dem man
das Passahlamm zu schlachten pflegte, fragten ihn seine Jünger: »Wohin
sollen wir gehen, um alles vorzubereiten, damit du das Passahlamm
essen\textless sup title=``=~das Passahmahl halten''\textgreater✲
kannst?« \bibleverse{13} Da sandte er zwei von seinen Jüngern ab und
trug ihnen auf: »Geht in die Stadt: da wird euch ein Mann begegnen, der
einen Krug mit Wasser trägt; folgt ihm nach, \bibleverse{14} und wo er
hineingeht, da sagt zu dem Hausherrn: ›Der Meister läßt fragen: Wo ist
das Eßzimmer\textless sup title=``=~der Speisesaal''\textgreater✲ für
mich, in dem ich das Passahlamm mit meinen Jüngern essen kann?‹
\bibleverse{15} Dann wird er euch ein geräumiges Obergemach zeigen, das
mit Tischpolstern ausgestattet ist und schon bereit steht; dort richtet
(alles) für uns zu!« \bibleverse{16} Da machten sich die Jünger auf den
Weg, und als sie in die Stadt gekommen waren, fanden sie es dort so, wie
er ihnen gesagt hatte, und richteten das Passahmahl zu.

\hypertarget{jesu-letztes-mahl-im-juxfcngerkreise-ankuxfcndigung-des-verrats-des-judas-einsetzung-des-heiligen-abendmahls}{%
\subsubsection{5. Jesu letztes Mahl im Jüngerkreise; Ankündigung des
Verrats des Judas; Einsetzung des heiligen
Abendmahls}\label{jesu-letztes-mahl-im-juxfcngerkreise-ankuxfcndigung-des-verrats-des-judas-einsetzung-des-heiligen-abendmahls}}

\bibleverse{17} Als es dann Abend geworden war, fand er sich dort mit
den Zwölfen ein; \bibleverse{18} und während sie bei Tische saßen und
das Mahl einnahmen, sagte Jesus: »Wahrlich ich sage euch: Einer von euch
wird mich überantworten\textless sup title=``oder:
verraten''\textgreater✲, einer, der hier mit mir ißt.«\textless sup
title=``Ps 41,10''\textgreater✲ \bibleverse{19} Da wurden sie betrübt
und fragten ihn, einer nach dem andern: »Ich bin es doch nicht?«
\bibleverse{20} Er antwortete ihnen: »Einer von euch Zwölfen, der mit
mir in die Schüssel eintaucht. \bibleverse{21} Denn der Menschensohn
geht zwar dahin, wie über ihn in der Schrift steht; doch wehe dem
Menschen, durch den der Menschensohn verraten wird! Für diesen Menschen
wäre es besser\textless sup title=``=~das Beste''\textgreater✲, er wäre
nicht geboren!«

\bibleverse{22} Und während des Essens nahm Jesus ein Brot, sprach den
Lobpreis (Gottes), brach (das Brot) und gab es ihnen mit den Worten:
»Nehmet! Dies ist mein Leib.« \bibleverse{23} Dann nahm er einen Becher,
sprach das Dankgebet und gab ihnen den, und sie tranken alle daraus;
\bibleverse{24} und er sagte zu ihnen: »Dies ist mein Blut, das
Bundesblut\textless sup title=``2.Mose 24,8; Sach 9,11''\textgreater✲,
das für viele vergossen wird. \bibleverse{25} Wahrlich ich sage euch:
Ich werde vom Erzeugnis des Weinstocks hinfort nicht mehr trinken bis zu
jenem Tage, an dem ich es neu trinke im Reiche Gottes.«

\hypertarget{jesus-in-gethsemane}{%
\subsubsection{6. Jesus in Gethsemane}\label{jesus-in-gethsemane}}

\hypertarget{a-gang-nach-gethsemane-vorhersagung-des-uxe4rgernisses-der-juxfcnger-und-der-verleugnung-des-petrus-treuegeluxfcbde-der-juxfcnger}{%
\paragraph{a) Gang nach Gethsemane; Vorhersagung des Ärgernisses der
Jünger und der Verleugnung des Petrus; Treuegelübde der
Jünger}\label{a-gang-nach-gethsemane-vorhersagung-des-uxe4rgernisses-der-juxfcnger-und-der-verleugnung-des-petrus-treuegeluxfcbde-der-juxfcnger}}

\bibleverse{26} Nachdem sie dann den Lobpreis (Ps 115-118) gesungen
hatten, gingen sie (aus der Stadt) hinaus an den Ölberg. \bibleverse{27}
Dabei sagte Jesus zu ihnen: »Ihr werdet alle Anstoß nehmen\textless sup
title=``=~an mir irre werden''\textgreater✲; denn es steht
geschrieben\textless sup title=``Sach 13,7''\textgreater✲: ›Ich werde
den Hirten niederschlagen, dann werden die Schafe sich zerstreuen.‹
\bibleverse{28} Aber nach meiner Auferweckung werde ich euch nach
Galiläa vorausgehen.« \bibleverse{29} Da antwortete Petrus: »Mögen auch
alle Anstoß nehmen\textless sup title=``=~an dir irre
werden''\textgreater✲, so doch ich sicherlich nicht!« \bibleverse{30}
Jesus erwiderte ihm: »Wahrlich ich sage dir: Du wirst mich noch heute in
dieser Nacht, ehe der Hahn zweimal kräht, dreimal verleugnen!«
\bibleverse{31} Er beteuerte aber nur um so eifriger: »Wenn ich auch mit
dir sterben müßte, werde ich dich doch nicht verleugnen!« Das gleiche
versicherten auch (die anderen) alle.

\hypertarget{b-jesu-seelenkampf-und-gebet-in-gethsemane-schwuxe4che-der-juxfcnger}{%
\paragraph{b) Jesu Seelenkampf und Gebet in Gethsemane; Schwäche der
Jünger}\label{b-jesu-seelenkampf-und-gebet-in-gethsemane-schwuxe4che-der-juxfcnger}}

\bibleverse{32} Sie kamen dann an einen Ort mit Namen
Gethsemane\textless sup title=``vgl. Mt 26,36''\textgreater✲; dort sagte
er zu seinen Jüngern: »Laßt euch hier nieder, bis ich gebetet habe!«
\bibleverse{33} Dann nahm er Petrus, Jakobus und Johannes mit sich und
fing an zu zittern und zu zagen \bibleverse{34} und sagte zu ihnen:
»Tiefbetrübt ist meine Seele bis zum Tode; bleibt hier und haltet euch
wach!« \bibleverse{35} Dann ging er noch ein wenig weiter, warf sich auf
die Erde nieder und betete, daß, wenn es möglich sei, die Stunde an ihm
vorübergehen möchte; \bibleverse{36} dabei sagte er: »Abba, Vater! Alles
ist dir möglich: laß diesen Kelch✲ an mir vorübergehen! Doch nicht, was
ich will, sondern was du willst!« \bibleverse{37} Dann ging er zurück
und fand sie schlafen und sagte zu Petrus: »Simon, du schläfst? Hattest
du nicht die Kraft, eine einzige Stunde wach zu bleiben? \bibleverse{38}
Wachet, und betet, damit ihr nicht in Versuchung geratet! Der Geist ist
willig, das Fleisch aber ist schwach.« \bibleverse{39} Darauf ging er
wieder weg und betete mit denselben Worten; \bibleverse{40} und als er
zurückkam, fand er sie wiederum schlafen; denn die Augen fielen ihnen
vor Müdigkeit zu, und sie wußten ihm nichts zu antworten.
\bibleverse{41} Und er kam zum drittenmal und sagte zu ihnen: »Schlaft
ein andermal und ruht euch aus! Es ist genug so: die Stunde ist
gekommen! Sehet, der Menschensohn wird den Sündern in die Hände
geliefert! \bibleverse{42} Steht auf, laßt uns gehen! Seht, der mich
überantwortet\textless sup title=``=~mein Verräter''\textgreater✲ ist
nahe gekommen!«

\hypertarget{c-gefangennahme-jesu-flucht-der-juxfcnger}{%
\paragraph{c) Gefangennahme Jesu; Flucht der
Jünger}\label{c-gefangennahme-jesu-flucht-der-juxfcnger}}

\bibleverse{43} Und sogleich darauf, während er noch redete, erschien
Judas, einer von den Zwölfen, und mit ihm eine Schar mit Schwertern und
Knütteln, von den Hohenpriestern, den Schriftgelehrten und Ältesten
(abgesandt). \bibleverse{44} Sein Verräter hatte aber ein Zeichen mit
ihnen verabredet, nämlich: »Der, den ich küssen werde, der ist's; den
nehmt fest und führt ihn sicher ab!« \bibleverse{45} Als er nun ankam,
trat er sogleich auf Jesus zu und sagte: »Rabbi\textless sup
title=``oder: Meister''\textgreater✲!« und küßte ihn; \bibleverse{46} da
legten sie Hand an Jesus und nahmen ihn fest. \bibleverse{47} Einer
jedoch von den Dabeistehenden zog das Schwert, schlug auf den Knecht des
Hohenpriesters ein und hieb ihm das Ohr ab. \bibleverse{48} Jesus aber
sagte zu ihnen: »Wie gegen einen Räuber seid ihr mit Schwertern und
Knütteln ausgezogen, um mich gefangen zu nehmen. \bibleverse{49} Täglich
bin ich bei euch im Tempel gewesen und habe dort gelehrt, und ihr habt
mich nicht festgenommen; doch die Aussprüche der Schrift müssen erfüllt
werden.«

\bibleverse{50} Da verließen ihn alle und ergriffen die Flucht.
\bibleverse{51} Aber ein junger Mann folgte ihm nach, der nur einen
linnenen Überwurf auf dem bloßen Leibe anhatte; den ergriffen sie;
\bibleverse{52} er aber ließ seinen Überwurf fahren und entfloh
unbekleidet.

\hypertarget{jesu-verhuxf6r-bekenntnis-und-verurteilung-vor-dem-hohenpriester-und-dem-hohen-rat}{%
\subsubsection{7. Jesu Verhör, Bekenntnis und Verurteilung vor dem
Hohenpriester und dem Hohen
Rat}\label{jesu-verhuxf6r-bekenntnis-und-verurteilung-vor-dem-hohenpriester-und-dem-hohen-rat}}

\bibleverse{53} Sie führten nun Jesus zu dem Hohenpriester ab, und alle
Hohenpriester sowie die Ältesten und Schriftgelehrten kamen (dort)
zusammen. \bibleverse{54} Petrus aber war ihm von ferne bis hinein in
den Palast\textless sup title=``oder: Hof''\textgreater✲ des
Hohenpriesters gefolgt; dort hatte er sich unter die Diener gesetzt und
wärmte sich am Feuer.

\bibleverse{55} Die Hohenpriester aber und der gesamte Hohe Rat suchten
nach einer Zeugenaussage gegen Jesus, um ihn zum Tode verurteilen zu
können, fanden jedoch keine; \bibleverse{56} denn viele legten wohl
falsches Zeugnis gegen ihn ab, doch ihre Aussagen stimmten nicht
überein. \bibleverse{57} Einige traten auch auf und brachten ein
falsches Zeugnis gegen ihn vor, indem sie aussagten: \bibleverse{58}
»Wir haben ihn sagen hören: ›Ich werde diesen Tempel, der von
Menschenhänden errichtet ist, abbrechen und in drei Tagen einen anderen
bauen, der nicht von Menschenhänden errichtet ist‹«; \bibleverse{59}
doch auch darin war ihr Zeugnis nicht übereinstimmend. \bibleverse{60}
Da stand der Hohepriester auf, trat in die Mitte und fragte Jesus mit
den Worten: »Entgegnest du nichts auf die Aussage dieser Zeugen?«
\bibleverse{61} Er aber schwieg und gab keine Antwort. Nochmals befragte
ihn der Hohepriester mit den Worten: »Bist du Christus\textless sup
title=``=~der Messias''\textgreater✲, der Sohn des Hochgelobten?«
\bibleverse{62} Jesus antwortete: »Ja, ich bin es, und ihr werdet den
Menschensohn sitzen sehen zur Rechten der Macht\textless sup
title=``=~des Allmächtigen''\textgreater✲ und kommen mit den Wolken des
Himmels!«\textless sup title=``Dan 7,13; Ps 110,1''\textgreater✲
\bibleverse{63} Da zerriß der Hohepriester seine Kleider und sagte:
»Wozu brauchen wir noch weiter Zeugen? \bibleverse{64} Ihr habt die
Gotteslästerung gehört: was urteilt ihr?« Da gaben sie alle das Urteil
über ihn ab, er sei des Todes schuldig. \bibleverse{65} Nun fingen
einige an, ihn anzuspeien, ihm das Gesicht zu verhüllen, ihn dann mit
der Faust zu schlagen und zu ihm zu sagen: »Weissage uns!« Auch die
Gerichtsdiener versetzten ihm bei der Übernahme Schläge ins Gesicht.

\hypertarget{verleugnung-und-reue-des-petrus}{%
\subsubsection{8. Verleugnung und Reue des
Petrus}\label{verleugnung-und-reue-des-petrus}}

\bibleverse{66} Während nun Petrus unten im Hofe (des Palastes) war, kam
eine von den Mägden des Hohenpriesters, \bibleverse{67} und als sie
Petrus (am Feuer) sich wärmen sah, blickte sie ihn scharf an und sagte:
»Du bist auch mit dem Nazarener, mit Jesus, zusammengewesen!«
\bibleverse{68} Er aber leugnete und sagte: »Ich weiß nicht und verstehe
nicht, was du da sagst!« Er ging dann in die Vorhalle des Hofes hinaus,
\bibleverse{69} und als die Magd ihn dort sah, fing sie wieder an und
sagte zu den Dabeistehenden: »Dieser ist auch einer von ihnen!«
\bibleverse{70} Er aber leugnete wiederum. Nach einer kleinen Weile
sagten die Dabeistehenden wieder zu Petrus: »Wahrhaftig, du gehörst zu
ihnen! Du bist ja auch ein Galiläer.« \bibleverse{71} Er aber fing an,
sich zu verfluchen und eidlich zu beteuern: »Ich kenne diesen Menschen
nicht, von dem ihr da redet!« \bibleverse{72} Und alsbald krähte der
Hahn zum zweitenmal. Da dachte Petrus an das Wort Jesu, wie er zu ihm
gesagt hatte: »Ehe der Hahn zweimal kräht, wirst du mich dreimal
verleugnen.« Als er daran dachte, brach er in Tränen aus.

\hypertarget{jesu-verhuxf6r-vor-dem-ruxf6mischen-statthalter-pontius-pilatus-seine-verurteilung-und-geiuxdfelung}{%
\subsubsection{9. Jesu Verhör vor dem römischen Statthalter Pontius
Pilatus; seine Verurteilung und
Geißelung}\label{jesu-verhuxf6r-vor-dem-ruxf6mischen-statthalter-pontius-pilatus-seine-verurteilung-und-geiuxdfelung}}

\hypertarget{section-14}{%
\section{15}\label{section-14}}

\bibleverse{1} Und sogleich frühmorgens fertigten die Hohenpriester mit
den Ältesten und Schriftgelehrten und (somit) der gesamte Hohe Rat den
(endgültigen) Beschluß aus, ließen Jesus fesseln und abführen und
übergaben ihn dem Pilatus. \bibleverse{2} Dieser befragte ihn: »Bist du
der König der Juden?« Er antwortete ihm mit den Worten: »Ja, ich bin
es.« \bibleverse{3} Die Hohenpriester brachten dann viele Anklagen gegen
ihn vor; \bibleverse{4} da fragte Pilatus ihn nochmals: »Entgegnest du
nichts? Höre nur, was sie alles gegen dich vorbringen!« \bibleverse{5}
Jesus aber gab keine Antwort mehr, so daß Pilatus sich verwunderte.

\bibleverse{6} An jedem Fest aber pflegte er ihnen einen Gefangenen
freizugeben, den sie sich erbaten\textless sup title=``=~erbitten
durften''\textgreater✲. \bibleverse{7} Nun saß damals ein unter dem
Namen Barabbas bekannter Mensch im Gefängnis mit den (anderen)
Aufrührern, die beim Aufruhr einen Mord begangen hatten. \bibleverse{8}
So zog denn die Volksmenge hinauf und begann um das zu bitten, was er
ihnen gewöhnlich gewährte. \bibleverse{9} Pilatus antwortete ihnen mit
der Frage: »Wollt ihr, daß ich euch den König der Juden freigebe?«
\bibleverse{10} Er war sich nämlich klar darüber geworden, daß die
Hohenpriester ihn aus Neid überantwortet hatten. \bibleverse{11} Die
Hohenpriester aber hetzten die Volksmenge zu der Forderung auf, er
möchte ihnen lieber den Barabbas freigeben. \bibleverse{12} Nun richtete
Pilatus nochmals die Frage an sie: »Was soll ich denn mit dem (Manne)
machen, den ihr den König der Juden nennt?« \bibleverse{13} Sie schrien
zurück: »Laß ihn kreuzigen!« \bibleverse{14} Pilatus entgegnete ihnen:
»Was hat er denn Böses getan?« Da schrien sie noch lauter: »Laß ihn
kreuzigen!« \bibleverse{15} Um nun dem Volke den Willen zu tun, gab
Pilatus ihnen den Barabbas frei, Jesus aber ließ er geißeln und übergab
ihn dann (den Soldaten) zur Kreuzigung.

\hypertarget{jesu-verspottung-und-miuxdfhandlung-durch-die-ruxf6mischen-soldaten}{%
\subsubsection{10. Jesu Verspottung und Mißhandlung durch die römischen
Soldaten}\label{jesu-verspottung-und-miuxdfhandlung-durch-die-ruxf6mischen-soldaten}}

\bibleverse{16} Nun führten ihn die Soldaten ab in das Innere des
Palastes -- das ist nämlich die Statthalterei -- und riefen die ganze
Abteilung\textless sup title=``eig. Kohorte''\textgreater✲ zusammen,
\bibleverse{17} dann legten sie ihm einen Purpur\textless sup
title=``d.h. scharlachroten Mantel''\textgreater✲ um, setzten ihm eine
Dornenkrone auf, die sie geflochten hatten, \bibleverse{18} und fingen
an, ihm als König zu huldigen mit dem Zuruf: »Sei gegrüßt, Judenkönig!«
\bibleverse{19} Dabei schlugen sie ihn mit einem Rohr aufs Haupt, spien
ihn an, warfen sich vor ihm auf die Knie nieder und brachten ihm
Huldigungen dar. \bibleverse{20} a Nachdem sie ihn so verspottet hatten,
nahmen sie ihm den Purpurmantel wieder ab und legten ihm seine eigenen
Kleider an.

\hypertarget{jesu-todesgang-nach-golgatha-seine-kreuzigung-und-sein-sterben}{%
\subsubsection{11. Jesu Todesgang nach Golgatha, seine Kreuzigung und
sein
Sterben}\label{jesu-todesgang-nach-golgatha-seine-kreuzigung-und-sein-sterben}}

b Dann führten sie ihn zur Kreuzigung (aus der Stadt) hinaus
\bibleverse{21} und zwangen einen Vorübergehenden, Simon aus Cyrene, der
vom Felde kam, den Vater Alexanders und des Rufus, ihm das Kreuz zu
tragen. \bibleverse{22} So brachten sie ihn nach dem Platz Golgatha, das
bedeutet übersetzt ›Schädel(stätte)‹, \bibleverse{23} und reichten ihm
mit Myrrhe gewürzten Wein, den er aber nicht nahm\textless sup
title=``Ps 69,22''\textgreater✲. \bibleverse{24} Dann kreuzigten sie ihn
und verteilten seine Kleider unter sich, indem sie das Los um sie
warfen, welches Stück jeder erhalten sollte\textless sup title=``Ps
22,19''\textgreater✲. \bibleverse{25} Es war aber die dritte
Tagesstunde, als sie ihn kreuzigten; \bibleverse{26} und die Inschrift
mit der Angabe seiner Schuld lautete so: »Der König der Juden.«
\bibleverse{27} Mit ihm kreuzigten sie auch zwei Räuber, den einen zu
seiner Rechten, den anderen zu seiner Linken. \bibleverse{28} {[}So
wurde das Schriftwort erfüllt, das da lautet\textless sup title=``Jes
53,12''\textgreater✲: »Er ist unter die Gesetzlosen✲ gerechnet
worden.«{]} \bibleverse{29} Und die Vorübergehenden schmähten ihn,
schüttelten die Köpfe\textless sup title=``Ps 22,8;
109,25''\textgreater✲ und riefen aus: »Ha du, der du den Tempel
abbrichst und ihn in drei Tagen wieder aufbaust: \bibleverse{30} hilf
dir selbst und steige vom Kreuz herab!« \bibleverse{31} Ebenso
verhöhnten ihn auch die Hohenpriester untereinander samt den
Schriftgelehrten mit den Worten: »Anderen hat er geholfen, sich selbst
kann er nicht helfen! \bibleverse{32} Der Gottgesalbte\textless sup
title=``=~Christus oder: der Messias''\textgreater✲, der König von
Israel, steige jetzt vom Kreuz herab, damit wir es sehen und gläubig
werden!« Auch die (beiden) mit ihm Gekreuzigten schmähten ihn.

\hypertarget{jesu-sterben-das-wunderzeichen-bei-seinem-tode}{%
\paragraph{Jesu Sterben; das Wunderzeichen bei seinem
Tode}\label{jesu-sterben-das-wunderzeichen-bei-seinem-tode}}

\bibleverse{33} Als dann aber die sechste Stunde gekommen war, trat eine
Finsternis über das ganze Land ein bis zur neunten Stunde;
\bibleverse{34} und in der neunten Stunde rief Jesus mit lauter Stimme:
»Eloi, Eloi, lema sabachthani?«\textless sup title=``vgl. Mt
27,46''\textgreater✲, das heißt übersetzt: »Mein Gott, mein Gott, warum
hast du mich verlassen?«\textless sup title=``Ps 22,2''\textgreater✲
\bibleverse{35} Als dies einige von den Dabeistehenden hörten, sagten
sie: »Hört, er ruft den Elia!« \bibleverse{36} Da lief einer hin,
tränkte einen Schwamm mit Essig, steckte ihn an ein Rohr und wollte ihm
zu trinken geben\textless sup title=``Ps 69,22''\textgreater✲, wobei er
sagte: »Laßt mich✲! Wir wollen doch sehen, ob Elia kommt, um ihn
herabzunehmen!« \bibleverse{37} Jesus aber stieß noch einen lauten
Schrei aus und verschied dann. \bibleverse{38} Da zerriß der Vorhang des
Tempels in zwei Stücke von oben bis unten. \bibleverse{39} Als aber der
Hauptmann, der ihm gegenüber in der Nähe stand, ihn so verscheiden sah,
erklärte er: »Dieser Mann ist wirklich Gottes Sohn gewesen.«

\bibleverse{40} Es waren aber auch Frauen da, die von weitem zuschauten,
unter ihnen Maria von Magdala\textless sup title=``vgl. Lk
8,2''\textgreater✲ und Maria, die Mutter Jakobus des
Kleinen\textless sup title=``=~des Jüngeren''\textgreater✲ und des
Joses, und Salome, \bibleverse{41} die ihm schon, als er noch in Galiläa
war, nachgefolgt waren und ihm Dienste geleistet hatten, und noch viele
andere (Frauen), die mit ihm nach Jerusalem hinaufgezogen waren.

\hypertarget{grablegung-jesu}{%
\subsubsection{12. Grablegung Jesu}\label{grablegung-jesu}}

\bibleverse{42} Als es nun bereits Spätnachmittag geworden war -- es war
nämlich Rüsttag✲, das ist der Tag vor dem Sabbat --, \bibleverse{43} kam
Joseph von Arimathäa, ein angesehener Ratsherr, der auch auf das Reich
Gottes wartete, begab sich mit kühnem Entschluß zu Pilatus hinein und
bat ihn um den Leichnam Jesu. \bibleverse{44} Pilatus wunderte sich, daß
er schon gestorben sein sollte; er ließ (deshalb) den Hauptmann zu sich
rufen und fragte ihn, ob er schon lange tot sei; \bibleverse{45} und als
er von dem Hauptmann das Nähere erfahren hatte, schenkte er den Leichnam
dem Joseph. \bibleverse{46} Der kaufte nun Leinwand, nahm ihn (vom
Kreuz) herab, wickelte ihn in die Leinwand und setzte ihn in einem Grabe
bei, das in einen Felsen gehauen war; dann wälzte er einen Stein vor den
Eingang des Grabes. \bibleverse{47} Maria von Magdala aber und Maria,
die Mutter des Joses, sahen sich genau den Ort an, wohin er gelegt
worden war.

\hypertarget{vii.-die-auferstehungsberichte-kap.-16}{%
\subsection{VII. Die Auferstehungsberichte (Kap.
16)}\label{vii.-die-auferstehungsberichte-kap.-16}}

\hypertarget{entdeckung-des-leeren-grabes-am-ostermorgen-die-offenbarung-des-engels-an-die-frauen}{%
\subsubsection{1. Entdeckung des leeren Grabes am Ostermorgen; die
Offenbarung des Engels an die
Frauen}\label{entdeckung-des-leeren-grabes-am-ostermorgen-die-offenbarung-des-engels-an-die-frauen}}

\hypertarget{section-15}{%
\section{16}\label{section-15}}

\bibleverse{1} Als dann der Sabbat vorüber war, kauften Maria von
Magdala und Maria, die Mutter des Jakobus, und Salome wohlriechende
Salben, um hinzugehen und ihn zu salben; \bibleverse{2} und ganz früh am
ersten Tage der Woche\textless sup title=``oder: nach dem Sabbat, d.h.
am Sonntag''\textgreater✲ kamen sie zum Grabe, als die Sonne (eben)
aufgegangen war; \bibleverse{3} und sie sagten zueinander: »Wer wird uns
den Stein vom Eingang des Grabes wegwälzen?«, er war nämlich sehr groß;
\bibleverse{4} doch als sie hinblickten, sahen sie, daß der Stein schon
weggewälzt war. \bibleverse{5} Als sie dann in das Grab hineingetreten
waren, sahen sie einen Jüngling auf der rechten Seite sitzen, der mit
einem langen, weißen Gewande bekleidet war, und sie erschraken sehr.
\bibleverse{6} Er aber sagte zu ihnen: »Erschreckt nicht! Ihr sucht
Jesus von Nazareth, den Gekreuzigten: er ist auferweckt worden, ist
nicht mehr hier; seht, da ist die Stelle, wohin man ihn gelegt hatte!
\bibleverse{7} Geht aber hin und sagt seinen Jüngern und (besonders) dem
Petrus, daß er euch nach Galiläa vorausgeht: dort werdet ihr ihn
wiedersehen, wie er euch gesagt hat.« \bibleverse{8} Da gingen sie
hinaus und flohen vom Grabe hinweg; denn Zittern und Entsetzen hatte sie
befallen; und sie sagten niemand etwas davon, denn sie fürchteten sich.

\hypertarget{zusammenfassender-ersatzschluuxdf}{%
\subsubsection{2. Zusammenfassender
Ersatzschluß}\label{zusammenfassender-ersatzschluuxdf}}

\hypertarget{a-jesus-erscheint-der-maria-von-magdala-und-den-zwei-juxfcngern-von-emmaus}{%
\paragraph{a) Jesus erscheint der Maria von Magdala und den zwei Jüngern
von
Emmaus}\label{a-jesus-erscheint-der-maria-von-magdala-und-den-zwei-juxfcngern-von-emmaus}}

\bibleverse{9} Nachdem Jesus aber am ersten Tage der Woche\textless sup
title=``vgl. V.2''\textgreater✲ frühmorgens auferstanden war, erschien
er zuerst der Maria von Magdala, aus der er sieben böse Geister
ausgetrieben hatte. \bibleverse{10} Diese ging hin und verkündete es
denen, die bei ihm\textless sup title=``oder: in seiner
Umgebung''\textgreater✲ gewesen waren und (jetzt) trauerten und weinten.
\bibleverse{11} Doch als diese hörten, daß er lebe und ihr erschienen
sei, wollten sie es nicht glauben.~-- \bibleverse{12} Darauf offenbarte
er sich in veränderter Gestalt zweien von ihnen, als sie auf einer
Wanderung über Land gingen. \bibleverse{13} Auch diese gingen hin und
verkündeten es den übrigen; doch auch ihnen glaubten sie nicht.

\hypertarget{b-jesu-erscheinung-vor-den-elf-aposteln-und-sein-missionsbefehl}{%
\paragraph{b) Jesu Erscheinung vor den elf Aposteln und sein
Missionsbefehl}\label{b-jesu-erscheinung-vor-den-elf-aposteln-und-sein-missionsbefehl}}

\bibleverse{14} Später aber offenbarte er sich den elf (Jüngern selbst),
als sie bei Tische saßen, und schalt✲ ihren Unglauben und ihre
Herzenshärte, weil sie denen, die ihn nach seiner Auferstehung gesehen
hatten, keinen Glauben geschenkt hatten. \bibleverse{15} Darauf sagte er
zu ihnen: »Geht hin in alle Welt und verkündigt die Heilsbotschaft der
ganzen Schöpfung! \bibleverse{16} Wer da gläubig geworden ist und sich
hat taufen lassen, wird gerettet werden; wer aber ungläubig geblieben
ist, wird verurteilt werden. \bibleverse{17} Denen aber, die zum Glauben
gekommen sind, werden diese Wunderzeichen folgen\textless sup
title=``=~dauernd zuteil werden''\textgreater✲: in meinem Namen werden
sie böse Geister austreiben, in✲ neuen Zungen reden, \bibleverse{18}
werden Schlangen aufheben und, wenn sie etwas Todbringendes\textless sup
title=``oder: Giftiges''\textgreater✲ trinken, wird es ihnen nicht
schaden; Kranken werden sie die Hände auflegen, und sie werden gesund
werden.«

\hypertarget{c-jesu-himmelfahrt}{%
\paragraph{c) Jesu Himmelfahrt}\label{c-jesu-himmelfahrt}}

\bibleverse{19} Nachdem nun der Herr Jesus zu ihnen geredet hatte, wurde
er in den Himmel emporgehoben und setzte sich zur Rechten Gottes.
\bibleverse{20} Sie aber zogen aus und predigten überall, wobei der Herr
mitwirkte und das Wort durch die Zeichen bestätigte, die dabei
geschahen.
