\hypertarget{der-prophet-hesekiel-ezechiel}{%
\section{DER PROPHET HESEKIEL
(EZECHIEL)}\label{der-prophet-hesekiel-ezechiel}}

\hypertarget{a.-erster-hauptteil-gerichtsbuch-oder-reden-vor-der-zerstuxf6rung-jerusalems-kap.-1-24}{%
\subsection{A. Erster Hauptteil: Gerichtsbuch oder Reden vor der
Zerstörung Jerusalems (Kap.
1-24)}\label{a.-erster-hauptteil-gerichtsbuch-oder-reden-vor-der-zerstuxf6rung-jerusalems-kap.-1-24}}

\hypertarget{i.-hesekiels-berufung-und-weihe-zum-propheten-11-321}{%
\subsection{I. Hesekiels Berufung und Weihe zum Propheten
(1,1-3,21)}\label{i.-hesekiels-berufung-und-weihe-zum-propheten-11-321}}

\hypertarget{einleitung-zeit-und-ort-der-berufung-die-erscheinung-der-herrlichkeit-gottes}{%
\subsubsection{1. Einleitung: Zeit und Ort der Berufung; die Erscheinung
der Herrlichkeit
Gottes}\label{einleitung-zeit-und-ort-der-berufung-die-erscheinung-der-herrlichkeit-gottes}}

\hypertarget{section}{%
\section{1}\label{section}}

\bibleverse{1}Und es begab sich im dreißigsten Jahre im vierten Monat,
am fünften Tage des Monats, als ich mich unter den in die
Verbannung\textless sup title=``oder: Gefangenschaft''\textgreater✲
Weggeführten am Flusse✲ Kebar befand: da tat sich der Himmel auf, und
ich sah göttliche Gesichte. \bibleverse{2}Am fünften Tage des Monats --
es war das fünfte Jahr seit der Wegführung✲ des Königs Jojachin --:
\bibleverse{3}da erging das Wort des HERRN an den Priester Hesekiel, den
Sohn Busis, im Lande der Chaldäer am Flusse Kebar; dort kam die Hand des
HERRN über ihn.

\hypertarget{die-erscheinung-der-herrlichkeit-gottes-v.4-28}{%
\paragraph{Die Erscheinung der Herrlichkeit Gottes
(V.4-28)}\label{die-erscheinung-der-herrlichkeit-gottes-v.4-28}}

\hypertarget{a-die-lichtwolke-und-der-licht--und-feuerwagen-mit-den-vier-cheruben}{%
\paragraph{a) Die Lichtwolke und der Licht- und Feuerwagen mit den vier
Cheruben}\label{a-die-lichtwolke-und-der-licht--und-feuerwagen-mit-den-vier-cheruben}}

\bibleverse{4}Als ich nämlich hinblickte, sah ich plötzlich einen
Sturmwind von Norden daherfahren und eine gewaltige Wolke und
zusammengeballtes✲ Feuer, von Lichtglanz rings umgeben, und mitten aus
ihm\textless sup title=``d.h. aus dem Feuer''\textgreater✲ blinkte etwas
hervor wie der Schimmer von Glanzerz {[}aus der Mitte des Feuers{]}.
\bibleverse{5}Mitten in ihm erschien dann etwas, das vier lebenden Wesen
glich, deren Aussehen folgendes war: sie hatten Menschengestalt,
\bibleverse{6}aber jedes hatte vier Gesichter und jedes von ihnen vier
Flügel. \bibleverse{7}Ihre Beine standen gerade, aber ihre Fußsohlen
waren (abgerundet) wie die Fußsohle eines Kalbes, und sie funkelten so
hell wie geglättetes Kupfer. \bibleverse{8}Unter ihren Flügeln befanden
sich\textless sup title=``oder: hatten sie''\textgreater✲ Menschenhände
an allen vier Seiten, und alle vier hatten Flügel, \bibleverse{9}von
denen immer einer den des nächsten berührte; ihre Gesichter wandten sich
nicht um, wenn sie gingen, sondern sie gingen ein jedes geradeaus vor
sich hin. \bibleverse{10}Ihre Gesichter sahen aber so aus: (vorn war)
ein Menschengesicht, rechts ein Löwengesicht bei allen vieren, links ein
Stiergesicht bei allen vieren, und nach innen\textless sup
title=``=~nach hinten''\textgreater✲ ein Adlergesicht bei allen vieren.
\bibleverse{11}Ihre Flügel waren nach oben hin ausgebreitet, bei jedem
zwei, die sich untereinander berührten, und zwei bedeckten ihre Leiber.
\bibleverse{12}Sie gingen ein jedes geradeaus vor sich hin: wohin der
Geist sie zu gehen trieb, dahin gingen sie, ohne beim Gehen eine Wendung
vorzunehmen. \bibleverse{13}Und mitten zwischen den lebenden Wesen war
etwas, das wie brennende Feuerkohlen aussah, wie Fackeln, deren Feuer
zwischen den Wesen beständig hin und her fuhr; und das Feuer hatte einen
strahlenden Glanz, und Blitze gingen aus dem Feuer hervor;
\bibleverse{14}und die lebenden Wesen liefen hin und her, so daß es
aussah wie Blitzstrahlen.

\hypertarget{b-die-vier-ruxe4der-des-wagens}{%
\paragraph{b) Die vier Räder des
Wagens}\label{b-die-vier-ruxe4der-des-wagens}}

\bibleverse{15}Als ich nun die lebenden Wesen näher betrachtete, sah ich
je ein Rad auf dem Erdboden neben jedem der vier Wesen.
\bibleverse{16}Das Aussehen der Räder war wie der Schimmer von
Chrysolith, und alle vier hatten die gleiche Gestalt, und sie waren so
hergestellt✲, als ob ein Rad innerhalb des andern Rades wäre.
\bibleverse{17}Nach allen vier Seiten hin liefen sie, wenn sie liefen,
ohne beim Laufen eine Wendung vorzunehmen. \bibleverse{18}Ihre Felgen
aber -- sie hatten eine gewaltige Höhe und Furchtbarkeit -- waren bei
allen vier Rädern ringsum voller Augen; \bibleverse{19}und wenn die
lebenden Wesen sich in Bewegung setzten, so liefen auch die Räder neben
ihnen; und wenn die lebenden Wesen sich vom Erdboden erhoben, dann
erhoben sich auch die Räder: \bibleverse{20}wohin der Geist jene zu
gehen trieb, dahin gingen die Räder ebenfalls und erhoben sich zugleich
mit ihnen; denn der Geist der lebenden Wesen war in den Rädern:
\bibleverse{21}wenn jene gingen, so gingen auch sie, und wenn jene
stehen blieben, so blieben auch sie stehen, und wenn jene sich von der
Erde erhoben, so erhoben sich auch die Räder zugleich mit ihnen; denn
der Geist der lebenden Wesen war in den Rädern.

\hypertarget{c-das-himmelsgewuxf6lbe-daruxfcber-und-der-auf-ihm-thronende-gott}{%
\paragraph{c) Das Himmelsgewölbe darüber und der auf ihm thronende
Gott}\label{c-das-himmelsgewuxf6lbe-daruxfcber-und-der-auf-ihm-thronende-gott}}

\bibleverse{22}Über den Häuptern der lebenden Wesen aber war etwas, das
sah aus wie ein Himmelsgewölbe, wie wundervoll glänzender Bergkristall;
oben über ihren Häuptern war es ausgebreitet. \bibleverse{23}Unterhalb
des Himmelsgewölbes aber waren ihre Flügel geradegerichtet\textless sup
title=``=~waagrecht ausgespannt''\textgreater✲, jeder nach dem andern
hin, von jedem zwei; mit den beiden anderen bedeckten sie ihre Leiber.
\bibleverse{24}Und ich hörte das Rauschen ihrer Flügel wie das Rauschen
gewaltiger Wasser\textless sup title=``oder: Fluten''\textgreater✲, wie
den Donner des Allmächtigen. Wenn sie gingen, glich das tosende Rauschen
dem Getöse eines Heerlagers; wenn sie aber stillstanden, ließen sie ihre
Flügel schlaff herabhängen. \bibleverse{25}{[}Und es kam eine Stimme von
oberhalb des Himmelsgewölbes, das über ihren Häuptern war; wenn sie
stillstanden, ließen sie ihre Flügel schlaff herabhängen.{]}

\bibleverse{26}Oben über dem Himmelsgewölbe aber, das sich über ihren
Häuptern befand, da war es anzusehen wie Saphirstein, etwas, das einem
Thron glich; und auf diesem Throngebilde war eine Gestalt zu sehen, die
wie eine Mann aussah, oben darauf. \bibleverse{27}Und ich sah etwas wie
den Schimmer von Glanzerz, wie das Aussehen von Feuer, das ringsum ein
Gehäuse hat; von dem Körperteile an, der wie seine Hüften aussah, nach
oben zu, und von dem Körperteile an, der wie seine Hüften aussah, nach
unten zu sah ich es -- wie Feuer anzuschauen; und strahlendes Licht war
rings um ihn her. \bibleverse{28}Wie der Bogen aussieht, der am
Regentage in den Wolken erscheint, so war das strahlende Licht ringsum
anzusehen.

So war das Aussehen der Erscheinung der Herrlichkeit des HERRN; und als
ich sie erblickte, warf ich mich auf mein Angesicht nieder und hörte die
Stimme eines, der da redete.

\hypertarget{berufung-und-weihe-hesekiels-zum-prophetenamt}{%
\subsubsection{2. Berufung und Weihe Hesekiels zum
Prophetenamt}\label{berufung-und-weihe-hesekiels-zum-prophetenamt}}

\hypertarget{a-hesekiels-sendung-an-das-abtruxfcnnige-israel}{%
\paragraph{a) Hesekiels Sendung an das abtrünnige
Israel}\label{a-hesekiels-sendung-an-das-abtruxfcnnige-israel}}

\hypertarget{section-1}{%
\section{2}\label{section-1}}

\bibleverse{1}Und er sagte zu mir: »Menschensohn, tritt auf deine Füße;
denn ich will mit dir reden.« \bibleverse{2}Als er so zu mir sprach, kam
eine Gotteskraft in mich, die mich auf meine Füße treten ließ; und ich
hörte den an, der mit mir redete. \bibleverse{3}Er sagte dann zu mir:
»Menschensohn! Ich sende dich zu denen vom Hause Israel, zu den
abtrünnigen Stämmen, die sich gegen mich aufgelehnt haben; sie und ihre
Väter sind von mir abgefallen bis auf den heutigen Tag.
\bibleverse{4}Und die Söhne\textless sup title=``oder:
Kinder''\textgreater✲ haben ein trotziges Gesicht und ein hartes✲ Herz;
zu ihnen sende ich dich, und du sollst zu ihnen sagen: ›So hat Gott, der
HERR gesprochen!‹ \bibleverse{5}Mögen sie dann darauf hören oder mögen
sie es lassen -- denn sie sind ein widerspenstiges Geschlecht --, so
sollen sie doch erkennen, daß ein Prophet unter ihnen aufgetreten ist.
\bibleverse{6}Du aber, Menschensohn, fürchte dich nicht vor ihnen und
laß dich durch ihre Reden nicht einschüchtern, ob auch Disteln und
Dornen um dich sind und du unter Skorpionen wohnst: fürchte dich nicht
vor ihren Reden und erschrick nicht vor ihren Mienen! Denn sie sind ein
widerspenstiges Geschlecht. \bibleverse{7}Vielmehr sollst du ihnen meine
Worte verkündigen, mögen sie darauf hören oder mögen sie es lassen; denn
sie sind widerspenstig!«

\hypertarget{b-die-guxf6ttliche-inspiration-d.h.-die-eingebung-des-weissagungsinhalts-durch-verzehren-einer-schriftrolle}{%
\paragraph{b) Die göttliche Inspiration (d.h. die Eingebung des
Weissagungsinhalts) durch Verzehren einer
Schriftrolle}\label{b-die-guxf6ttliche-inspiration-d.h.-die-eingebung-des-weissagungsinhalts-durch-verzehren-einer-schriftrolle}}

\bibleverse{8}»Du aber, Menschensohn, höre, was ich dir sage: Sei du
nicht widerspenstig wie das widerspenstige Geschlecht! Öffne deinen Mund
und iß, was ich dir jetzt gebe!« \bibleverse{9}Als ich nun hinblickte,
sah ich eine Hand, die sich mir entgegenstreckte, und in ihr befand sich
eine Schriftrolle. \bibleverse{10}Er breitete sie vor mir aus, und sie
war auf der Vorderseite und auf der Rückseite beschrieben; und zwar
standen Klagen, Seufzer und Wehe auf ihr geschrieben.

\hypertarget{section-2}{%
\section{3}\label{section-2}}

\bibleverse{1}Dann sagte er zu mir: »Menschensohn, iß, was du da vor dir
siehst! Iß diese Schriftrolle und gehe dann hin und rede zum Hause
Israel!« \bibleverse{2}Da öffnete ich meinen Mund, und er gab mir jene
Rolle zu essen; \bibleverse{3}dabei sagte er zu mir: »Menschensohn,
verschlucke diese Schriftrolle, die ich dir gebe, und fülle deinen
Leib\textless sup title=``oder: Magen''\textgreater✲ mit ihr!« Da aß ich
sie, und sie schmeckte mir im Munde süß wie Honig.

\hypertarget{c-einschuxe4rfung-des-auftrags-ausruxfcstung-des-propheten-fuxfcr-sein-amt}{%
\paragraph{c) Einschärfung des Auftrags; Ausrüstung des Propheten für
sein
Amt}\label{c-einschuxe4rfung-des-auftrags-ausruxfcstung-des-propheten-fuxfcr-sein-amt}}

\bibleverse{4}Darauf sagte er zu mir: »Menschensohn, auf! Begib dich zum
Hause Israel und rede in meinem Namen zu ihnen! \bibleverse{5}Denn nicht
zu einem Volk mit dunkler Sprache und unverständlicher Rede wirst du
gesandt, sondern zum Hause Israel; \bibleverse{6}auch nicht zu
zahlreichen Völkern mit fremder Sprache und unverständlicher Rede, deren
Worte du nicht verstehst -- freilich, wenn ich dich zu diesen sendete,
würden sie auf dich hören --; \bibleverse{7}aber das Haus Israel wird
nicht auf dich hören wollen: sie wollen ja auch auf mich nicht hören;
denn das ganze Haus Israel hat eine harte Stirn und ein verstocktes
Herz. \bibleverse{8}Doch wisse: ich will auch dein Angesicht hart machen
gleich dem ihrigen und deine Stirn hart gleich der ihrigen:
\bibleverse{9}wie Diamant, härter als Kieselstein, will ich deine Stirn
machen. Fürchte dich nicht vor ihnen und laß dich durch ihre Mienen
nicht einschüchtern! Denn sie sind ein widerspenstiges Geschlecht.«
\bibleverse{10}Dann fuhr er fort: »Menschensohn, alle meine Worte, die
ich zu dir reden werde, nimm dir zu Herzen und laß sie in deine Ohren
eindringen! \bibleverse{11}Und nun mache dich auf und begib dich zu den
in der Verbannung\textless sup title=``oder:
Gefangenschaft''\textgreater✲ Lebenden, zu deinen Volksgenossen; rede zu
ihnen und sage ihnen: ›So hat Gott der HERR gesprochen!‹ -- mögen sie
nun darauf hören, oder mögen sie es lassen!«

\hypertarget{die-entlassung-des-propheten-neue-guxf6ttliche-offenbarung-und-belehrung}{%
\subsubsection{3. Die Entlassung des Propheten; neue göttliche
Offenbarung und
Belehrung}\label{die-entlassung-des-propheten-neue-guxf6ttliche-offenbarung-und-belehrung}}

\hypertarget{a-das-entschwinden-der-gotteserscheinung-die-entruxfcckung-des-propheten-auf-das-arbeitsfeld}{%
\paragraph{a) Das Entschwinden der Gotteserscheinung; die Entrückung des
Propheten auf das
Arbeitsfeld}\label{a-das-entschwinden-der-gotteserscheinung-die-entruxfcckung-des-propheten-auf-das-arbeitsfeld}}

\bibleverse{12}Da hob die Gotteskraft mich empor, und ich vernahm hinter
mir ein lautes, gewaltiges Getöse, als die Herrlichkeit des HERRN sich
von ihrer Stelle erhob, \bibleverse{13}nämlich das Rauschen der sich
gegenseitig berührenden Flügel der vier lebenden Wesen und das Gerassel
der Räder neben\textless sup title=``oder: zugleich mit''\textgreater✲
ihnen, ein lautes gewaltiges Getöse. \bibleverse{14}Als mich nun die
Gotteskraft emporhob und mich entrückte, ging ich dahin, tief betrübt in
der Erregung meines Geistes, während die Hand des HERRN übermächtig auf
mir lastete. \bibleverse{15}So kam ich denn zu den in der
Verbannung\textless sup title=``oder: Gefangenschaft''\textgreater✲
Lebenden nach Thel-Abib, wo sie am Flusse Kebar wohnten, und weilte dort
sieben Tage lang unter ihnen, in dumpfes Schweigen
versunken\textless sup title=``=~in starrer Betäubung''\textgreater✲.

\hypertarget{b-hesekiels-einsetzung-in-das-verantwortungsvolle-geistliche-wuxe4chteramt-uxfcber-die-verbannten}{%
\paragraph{b) Hesekiels Einsetzung in das verantwortungsvolle geistliche
Wächteramt über die
Verbannten}\label{b-hesekiels-einsetzung-in-das-verantwortungsvolle-geistliche-wuxe4chteramt-uxfcber-die-verbannten}}

\bibleverse{16}Als aber die sieben Tage um waren, erging das Wort des
HERRN an mich also: \bibleverse{17}»Menschensohn, ich habe dich zum
Wächter für das Haus Israel bestellt: wenn du ein Wort aus meinem Munde
vernommen hast, sollst du sie in meinem Namen verwarnen!
\bibleverse{18}Wenn ich also zum Gottlosen sage: ›Du mußt des Todes
sterben!‹ und du verwarnst ihn nicht und sagst kein Wort, um den
Gottlosen vor seinem bösen Wandel zu warnen, um ihn am Leben zu
erhalten, so wird er als Gottloser um seiner Verschuldung willen
sterben, aber für den Verlust seines Lebens werde ich dich
verantwortlich machen. \bibleverse{19}Hast du aber den Gottlosen gewarnt
und hat er sich trotzdem von seiner Gottlosigkeit und seinem bösen
Wandel nicht abgewandt, so wird er zwar um seiner Verschuldung willen
sterben, du aber hast deine Seele\textless sup title=``oder: dein
Leben''\textgreater✲ gerettet. \bibleverse{20}Und wenn ein Gerechter
sich von seiner Gerechtigkeit abkehrt und Böses tut und ich ihm dann
entgegentrete, so daß er stirbt: wenn du ihn dann nicht gewarnt hast, so
wird er zwar infolge seiner Sünde sterben, und der gerechten Werke, die
er vollbracht hat, wird nicht mehr gedacht werden; aber für den Verlust
seines Lebens werde ich dich verantwortlich machen. \bibleverse{21}Wenn
du ihn aber, den Gerechten, gewarnt hast, daß er keine Sünde begehen
möge, und er dann auch keine Sünde begeht, so wird er am Leben bleiben,
weil er sich hat warnen lassen, und du hast deine Seele\textless sup
title=``oder: Leben''\textgreater✲ gerettet.«

\hypertarget{ii.-erste-reihe-von-drohweissagungen-gegen-juda-und-jerusalem-in-bild-und-wort-322-727}{%
\subsection{II. Erste Reihe von Drohweissagungen gegen Juda und
Jerusalem in Bild und Wort
(3,22-7,27)}\label{ii.-erste-reihe-von-drohweissagungen-gegen-juda-und-jerusalem-in-bild-und-wort-322-727}}

\hypertarget{zweite-erscheinung-der-herrlichkeit-gottes-die-dem-propheten-von-gott-auferlegte-zuruxfcckhaltung}{%
\subsubsection{1. Zweite Erscheinung der Herrlichkeit Gottes; die dem
Propheten von Gott auferlegte
Zurückhaltung}\label{zweite-erscheinung-der-herrlichkeit-gottes-die-dem-propheten-von-gott-auferlegte-zuruxfcckhaltung}}

\bibleverse{22}Da kam dort die Hand des HERRN über mich, und er gebot
mir: »Mache dich auf und gehe in die Tal-Ebene hinaus: dort will ich mit
dir reden!« \bibleverse{23}Da machte ich mich auf und ging in die
Tal-Ebene hinaus; und siehe, dort stand die Herrlichkeit des HERRN
gerade so, wie ich sie am Flusse Kebar\textless sup title=``vgl.
1,1''\textgreater✲ geschaut hatte; da warf ich mich auf mein Angesicht
nieder. \bibleverse{24}Aber die Gotteskraft kam in mich und ließ mich
auf meine Füße treten; und er\textless sup title=``d.h.
Gott''\textgreater✲ redete mich an mit den Worten: »Gehe heim und
schließe dich in deinem Hause ein! \bibleverse{25}Und du, Menschensohn,
wisse wohl: man wird dir Fesseln anlegen und dich mit ihnen binden, so
daß du nicht unter sie\textless sup title=``d.h. deine
Volksgenossen''\textgreater✲ wirst hinausgehen können.
\bibleverse{26}Und ich werde dir die Zunge am Gaumen kleben lassen:
stumm sollst du werden und kein Strafprediger mehr für sie sein; denn
sie sind ein widerspenstiges Geschlecht. \bibleverse{27}Wenn ich aber
mit dir rede, werde ich dir den Mund auftun, und du sollst zu ihnen
sagen: ›So hat Gott der HERR gesprochen!‹ Wer dann hören will, mag
hören, und wer es lassen will, der mag es lassen! Denn sie sind ein
widerspenstiges Geschlecht.«

\hypertarget{ankuxfcndigung-des-gerichts-uxfcber-jerusalem-durch-vier-sinnbildliche-zeichen}{%
\subsubsection{2. Ankündigung des Gerichts über Jerusalem durch vier
sinnbildliche
Zeichen}\label{ankuxfcndigung-des-gerichts-uxfcber-jerusalem-durch-vier-sinnbildliche-zeichen}}

\hypertarget{a-erstes-zeichen-die-belagerte-stadt-jerusalem}{%
\paragraph{a) Erstes Zeichen: die belagerte Stadt
Jerusalem}\label{a-erstes-zeichen-die-belagerte-stadt-jerusalem}}

\hypertarget{section-3}{%
\section{4}\label{section-3}}

\bibleverse{1}»Du aber, Menschensohn, nimm dir einen Ziegelstein, lege
ihn vor dich hin und ritze auf ihm eine Stadt ein, nämlich Jerusalem.
\bibleverse{2}Dann eröffne die Belagerung gegen sie: errichte
Belagerungswerke gegen sie, schütte gegen sie einen Wall auf, schlage
Heerlager gegen sie auf und stelle Sturmböcke\textless sup title=``oder:
Mauerbrecher''\textgreater✲ ringsum gegen sie auf. \bibleverse{3}Darauf
hole dir eine eiserne Platte\textless sup title=``oder:
Bratpfanne''\textgreater✲ und stelle sie als eiserne Mauer zwischen dich
und die Stadt hin; dann richte deine Blicke unverwandt gegen sie: so
soll sie sich im Belagerungszustand befinden, und du sollst sie
belagern: ein Wahrzeichen soll dies für das Haus Israel sein!«

\hypertarget{b-zweites-zeichen-die-tage-des-gebundenseins-fuxfcr-die-schuld-des-nord--und-des-suxfcdreiches}{%
\paragraph{b) Zweites Zeichen: die Tage des Gebundenseins für die Schuld
des Nord- und des
Südreiches}\label{b-zweites-zeichen-die-tage-des-gebundenseins-fuxfcr-die-schuld-des-nord--und-des-suxfcdreiches}}

\bibleverse{4}»Sodann lege du dich auf deine linke Seite und trage die
Verschuldung des Hauses Israel auf ihr; nach der Zahl der Tage, die du
auf ihr liegst, sollst du ihre Verschuldung tragen. \bibleverse{5}Ich
aber will dir die Jahre ihrer Verschuldung in einer gleichgroßen Anzahl
von Tagen auflegen, nämlich 390 Tage: so lange sollst du die
Verschuldung des Hauses Israel tragen. \bibleverse{6}Wenn du dann mit
diesen Tagen zu Ende bist, so lege dich zum zweitenmal auf deine rechte
Seite und trage die Verschuldung des Hauses Juda vierzig Tage lang: für
jedes Jahr lege ich dir einen Tag auf. \bibleverse{7}Dabei sollst du
deine Blicke und deinen entblößten Arm unverwandt auf die Belagerung
Jerusalems richten und gegen die Stadt weissagen. \bibleverse{8}Und
wisse wohl: ich will dir Fesseln anlegen, daß du dich nicht von einer
Seite auf die andere umwenden kannst, bis du mit den Tagen deiner
Belagerung\textless sup title=``oder: Bedrängnis''\textgreater✲ zu Ende
bist.«

\hypertarget{c-drittes-zeichen-die-ungenuxfcgende-und-unreine-kost-wuxe4hrend-der-belagerung-der-stadt-und-wuxe4hrend-der-verbannung}{%
\paragraph{c) Drittes Zeichen: die ungenügende und unreine Kost während
der Belagerung der Stadt (und während der
Verbannung?)}\label{c-drittes-zeichen-die-ungenuxfcgende-und-unreine-kost-wuxe4hrend-der-belagerung-der-stadt-und-wuxe4hrend-der-verbannung}}

\bibleverse{9}»Du aber nimm dir Weizen und Gerste, Bohnen und Linsen,
Hirse und Spelt, tu sie zusammen in ein einziges Gefäß und bereite dir
Brot daraus nach der Zahl der Tage, während derer du auf deiner Seite
liegen mußt: 390\textless sup title=``190? V.5''\textgreater✲ Tage lang
sollst du das essen; \bibleverse{10}und zwar soll die Speise, die du zu
dir nimmst, abgewogen täglich zwanzig Schekel betragen, und du sollst
sie auf mehrere Mahlzeiten verteilen. \bibleverse{11}Auch das Wasser
sollst du abgemessen trinken, jedesmal ein Sechstel Hin; auch dieses
sollst du in Zwischenräumen genießen. \bibleverse{12}Und zwar sollst du
(das Brot) wie Gerstenbrotkuchen zubereitet essen und diese vor ihren
Augen auf Ballen von Menschenkot backen. \bibleverse{13}(Und du sollst
sagen:) ›So hat der HERR gesprochen: Ebenso sollen die Kinder Israel ihr
Brot unrein essen unter den Heidenvölkern, unter die ich sie verstoßen
werde.‹«

\bibleverse{14}Da entgegnete ich: »Ach, HERR, mein Gott! Siehe, mein
Inneres✲ ist noch niemals durch Unreines befleckt worden, und von
verendeten oder zerrissenen Tieren habe ich seit meiner Jugend bis jetzt
niemals etwas genossen, und abscheuliches Fleisch ist noch nie in meinen
Mund gekommen.« \bibleverse{15}Da antwortete er mir: »Nun gut! Ich will
dir Rindermist statt des Menschenkotes gestatten: auf diesem magst du
das Brot für dich bereiten.« \bibleverse{16}Dann fuhr er fort:
»Menschensohn, wisse wohl: ich will den Stab\textless sup title=``oder:
die Stütze''\textgreater✲ des Brotes in Jerusalem zerbrechen, so daß sie
das Brot abgewogen und mit Angst essen müssen und das Wasser abgemessen
und mit Entsetzen trinken, \bibleverse{17}weil es ihnen an Brot und
Wasser mangeln soll und damit sie einer wie der andere verzweifeln und
infolge ihrer Verschuldung verschmachten.«

\hypertarget{d-viertes-zeichen-die-vuxf6llige-vernichtung-des-volkes-bis-auf-einen-geringen-rest-beim-ausgang-der-belagerung}{%
\paragraph{d) Viertes Zeichen: die völlige Vernichtung des Volkes bis
auf einen geringen Rest beim Ausgang der
Belagerung}\label{d-viertes-zeichen-die-vuxf6llige-vernichtung-des-volkes-bis-auf-einen-geringen-rest-beim-ausgang-der-belagerung}}

\hypertarget{section-4}{%
\section{5}\label{section-4}}

\bibleverse{1}»Du aber, Menschensohn, nimm dir ein scharfes Schwert; als
Schermesser sollst du es für dich\textless sup title=``oder: an
dir''\textgreater✲ benutzen und mit ihm über dein Haupt und deinen Bart
fahren. Sodann nimm dir eine Waage zum Abwägen und teile (die Haare) mit
ihr ab: \bibleverse{2}ein Drittel verbrenne im Feuer inmitten der Stadt,
wenn die Tage der Belagerung voll sind\textless sup title=``=~zu Ende
gehen''\textgreater✲; das zweite Drittel nimm und schlage es mit dem
Schwert rings um die Stadt her; und das letzte Drittel verstreue in den
Wind; ich will dann das Schwert hinter ihnen her zücken.
\bibleverse{3}Doch nimm von diesen eine kleine Anzahl und binde sie in
deinen Rockzipfel ein; \bibleverse{4}hierauf nimm auch von diesen
nochmals einige, wirf sie mitten ins Feuer und laß sie im Feuer
verbrennen: davon soll ein Feuer über das ganze Haus Israel ausgehen.«

\hypertarget{e-die-deutung-und-zugleich-begruxfcndung-der-vier-zeichen-des-gerichts}{%
\paragraph{e) Die Deutung und zugleich Begründung der vier Zeichen des
Gerichts}\label{e-die-deutung-und-zugleich-begruxfcndung-der-vier-zeichen-des-gerichts}}

\bibleverse{5}»Alsdann sage zum ganzen Hause Israel: ›So hat Gott der
HERR gesprochen: So steht's mit Jerusalem! Mitten unter die Heiden habe
ich die Stadt gestellt und (deren) Länder rings um sie her.
\bibleverse{6}Aber sie ist gegen meine Gebote in gottloser Weise
widerspenstig gewesen, ärger noch als die Heidenvölker, und gegen meine
Satzungen ärger als die Länder rings um sie her; denn meine Gebote haben
sie verachtet, und nach meinen Satzungen sind sie nicht gewandelt.
\bibleverse{7}Darum hat Gott der HERR so gesprochen: Weil ihr trotziger
gewesen seid als die Heidenvölker rings um euch her, weil ihr euch in
eurem Wandel nach meinen Satzungen nicht gerichtet und meine Gebote
nicht befolgt habt, vielmehr nach den Geboten der Heidenvölker rings um
euch her gehandelt habt, \bibleverse{8}darum spricht Gott der HERR so:
Fürwahr, auch ich will nun gegen dich vorgehen und will Gerichte in
deiner Mitte vollstrecken vor den Augen der Heidenvölker!
\bibleverse{9}Und ich will um all deiner Greuel willen an dir tun, was
ich noch nie getan habe und was ich in gleicher Weise auch nie wieder
tun werde. \bibleverse{10}Darum werden Väter in deiner Mitte ihre Kinder
aufessen, und Kinder werden ihre Väter verzehren; und ich will
Strafgerichte an dir vollstrecken und alles, was von dir noch
übriggeblieben ist, in alle Winde zerstreuen! \bibleverse{11}Darum, so
wahr ich lebe!‹ -- so lautet der Ausspruch Gottes des HERRN --:
›fürwahr, weil du mein Heiligtum durch all deine scheußlichen Götzen und
durch all deine Greuel verunreinigt hast, so will auch ich nun mein Auge
von dir abwenden ohne Mitleid und will keine Schonung mehr üben!
\bibleverse{12}Der dritte Teil von dir soll an der Pest sterben und
durch Hunger umkommen in deiner Mitte; das zweite Drittel soll durchs
Schwert fallen rings um dich her; und das letzte Drittel will ich in
alle Winde zerstreuen und das Schwert hinter ihnen her zücken.
\bibleverse{13}Wenn so mein Zorn sich voll ausgewirkt hat und ich meinen
Grimm an ihnen gestillt und Rache genommen habe, dann werden sie
erkennen, daß ich, der HERR, in meinem Eifer geredet habe, wenn ich
meinen Grimm an ihnen voll auswirke. \bibleverse{14}Ja, ich will dich
zur Einöde machen und zum Gegenstand des Hohnes unter den Heidenvölkern
rings um dich her, vor den Augen aller Vorüberziehenden.
\bibleverse{15}Und du sollst ein Gegenstand des Hohnes und der Schmach
sein, eine Warnung und ein Entsetzen für die Heidenvölker rings um dich
her, wenn ich Strafgerichte an dir vollziehe im Zorn und im Grimm und
mit meinen grimmigen Heimsuchungen -- ich, der HERR, habe es gesagt! --,
\bibleverse{16}wenn ich die schlimmen, verderblichen Pfeile des Hungers
gegen euch abschieße, die ich entsenden werde, um euch zu vernichten,
und wenn ich die Hungersnot immer schrecklicher bei euch wirken lasse
und ich euch den Stab\textless sup title=``oder: die
Stütze''\textgreater✲ des Brotes zerbreche. \bibleverse{17}Und außer der
Hungersnot will ich auch böse Tiere gegen euch loslassen, damit sie dich
deiner Kinder berauben; und Pest und Blutvergießen sollen bei dir
umgehen, und auch das Schwert will ich über dich kommen lassen -- ich,
der HERR, habe es gesagt!‹«

\hypertarget{straf--und-drohreden-gegen-land-und-volk-von-juda}{%
\subsubsection{3. Straf- und Drohreden gegen Land und Volk von
Juda}\label{straf--und-drohreden-gegen-land-und-volk-von-juda}}

\hypertarget{a-unheilspredigt-gegen-die-berge-israels-d.h.-des-ganzen-landes-paluxe4stina}{%
\paragraph{a) Unheilspredigt gegen die Berge Israels (d.h. des ganzen
Landes
Palästina)}\label{a-unheilspredigt-gegen-die-berge-israels-d.h.-des-ganzen-landes-paluxe4stina}}

\hypertarget{aa-ankuxfcndigung-vuxf6lliger-vernichtung-fuxfcr-alle-stuxe4tten-des-guxf6tzendienstes}{%
\subparagraph{aa) Ankündigung völliger Vernichtung für alle Stätten des
Götzendienstes}\label{aa-ankuxfcndigung-vuxf6lliger-vernichtung-fuxfcr-alle-stuxe4tten-des-guxf6tzendienstes}}

\hypertarget{section-5}{%
\section{6}\label{section-5}}

\bibleverse{1}Weiter erging das Wort des HERRN an mich so:
\bibleverse{2}»Menschensohn, richte deine Blicke gegen die Berge Israels
und sprich folgende Weissagungen gegen sie aus: \bibleverse{3}›Ihr Berge
Israels, hört das Wort Gottes, des HERRN! So spricht Gott der HERR zu
den Bergen und Hügeln, zu den Rinnsalen✲ und Tälern: Fürwahr ich lasse
das Schwert über euch kommen und mache eurem Höhendienst ein Ende!
\bibleverse{4}Eure Altäre sollen zerstört und eure Sonnensäulen
zertrümmert werden, und eure Erschlagenen will ich vor eure Götzen
hinwerfen\textless sup title=``oder: hinsinken lassen''\textgreater✲,
\bibleverse{5}ja ich will die Leichen der Söhne Israels vor ihre Götzen
hinwerfen und eure Gebeine rings um eure Altäre verstreuen!
\bibleverse{6}Überall, wo ihr wohnt, sollen die Ortschaften verwüstet
werden und die Opferhöhen verödet werden, damit eure Altäre verlassen
und zerstört dastehen und eure Götzen zertrümmert werden und
verschwinden, eure Sonnensäulen umgehauen und eure Machwerke vernichtet
werden; \bibleverse{7}und Durchbohrte sollen in eurer Mitte zu Boden
fallen, damit ihr erkennt, daß ich der HERR bin.‹«

\hypertarget{bb-heilsverkuxfcndigung-fuxfcr-einen-kleinen-teil-des-volkes}{%
\subparagraph{bb) Heilsverkündigung für einen kleinen Teil des
Volkes}\label{bb-heilsverkuxfcndigung-fuxfcr-einen-kleinen-teil-des-volkes}}

\bibleverse{8}»›Doch will ich einige von euch übriglassen, indem von
euch Schwertentronnene unter den Heidenvölkern leben werden, wenn ihr in
die Länder zerstreut seid. \bibleverse{9}Diese von euch Entronnenen
werden dann unter den Heidenvölkern, wohin sie als Gefangene weggeführt
worden sind, meiner gedenken, wenn ich ihr Herz, das treulos von mir
abgefallen war, und ihre Augen, die buhlerisch auf ihre Götzen gerichtet
waren, zerbrochen\textless sup title=``oder: zerschlagen''\textgreater✲
habe. Dann werden sie vor sich selbst Abscheu empfinden wegen der
Missetaten, die sie mit all ihren Greueln begangen haben,
\bibleverse{10}und sie werden erkennen, daß ich, der HERR, nicht umsonst
gedroht habe, ihnen solches Unheil widerfahren zu lassen.‹«

\hypertarget{cc-erneute-ankuxfcndigung-schwerer-heimsuchungen-wegen-der-heidnischen-greuel}{%
\subparagraph{cc) Erneute Ankündigung schwerer Heimsuchungen wegen der
heidnischen
Greuel}\label{cc-erneute-ankuxfcndigung-schwerer-heimsuchungen-wegen-der-heidnischen-greuel}}

\bibleverse{11}So hat Gott der HERR gesprochen: »Schlage die Hände
zusammen und stampfe mit dem Fuße und rufe ›Wehe!‹ über all die
schlimmen Greuel des Hauses Israel, deretwegen sie durch das Schwert,
durch Hunger und durch Pest umkommen sollen! \bibleverse{12}Wer in der
Ferne weilt, soll durch die Pest sterben, und wer in der Nähe ist, soll
durch das Schwert fallen, und wer dann noch übriggeblieben und mit dem
Leben davongekommen ist, soll Hungers sterben: so will ich meinen Grimm
an ihnen voll auswirken! \bibleverse{13}Alsdann werdet ihr erkennen, daß
ich der HERR bin, wenn ihre Erschlagenen inmitten ihrer Götzen rings um
ihre Altäre daliegen auf jedem hohen Hügel und auf allen Berggipfeln,
unter jedem grünenden Baume und unter jeder dichtbelaubten Terebinthe,
an den Stätten, wo sie all ihren Götzen lieblichen Opferduft gespendet
haben. \bibleverse{14}Ja, ich will meine Hand gegen sie ausstrecken und
das Land zur Wüste und Wildnis machen von der Steppe an bis nach Ribla
hin, überall, wo sie wohnen, damit sie erkennen, daß ich der HERR bin.«

\hypertarget{b-ankuxfcndigung-des-uxfcber-das-land-und-das-volk-kommenden-endgerichts-und-aller-seiner-schrecken}{%
\paragraph{b) Ankündigung des über das Land und das Volk kommenden
Endgerichts und aller seiner
Schrecken}\label{b-ankuxfcndigung-des-uxfcber-das-land-und-das-volk-kommenden-endgerichts-und-aller-seiner-schrecken}}

\hypertarget{section-6}{%
\section{7}\label{section-6}}

\bibleverse{1}Weiter erging das Wort des HERRN an mich folgendermaßen:
\bibleverse{2}»Du, Menschensohn, so spricht Gott der HERR zum Lande
Israel: ›Ein Ende kommt! Es kommt das Ende über alle vier
Enden\textless sup title=``oder: Seiten''\textgreater✲ des Landes!
\bibleverse{3}Jetzt kommt das Ende über dich; ich will meinen Zorn gegen
dich loslassen und dich nach deinem ganzen Tun richten und dich für alle
deine Greuel büßen lassen! \bibleverse{4}Mein Auge soll nicht mehr
mitleidig nach dir blicken, und ich werde keine Schonung mehr üben,
sondern ich will dich für dein ganzes Tun büßen lassen, und die Folgen
deiner Greuel sollen sich bei dir fühlbar machen, damit ihr erkennt, daß
ich der HERR bin!‹« \bibleverse{5}So hat Gott der HERR gesprochen:
»Unglück über Unglück! Siehe da, es kommt! \bibleverse{6}Ein Ende kommt!
Es kommt das Ende! Es erwacht gegen dich: siehe da, es kommt!
\bibleverse{7}Das Verhängnis kommt über dich, Bewohnerschaft des Landes!
Es kommt die Zeit, nahe ist der Tag, ein Tag der Bestürzung\textless sup
title=``oder: des Kriegsgetümmels''\textgreater✲ und nicht des Jauchzens
auf den Bergen! \bibleverse{8}Nunmehr will ich gar bald meinen Grimm
über dich ausgießen und meinen Zorn sich an dir erschöpfen lassen und
will dich nach deinem ganzen Tun richten und dich für all deine Greuel
büßen lassen. \bibleverse{9}Mein Auge soll nicht mehr mitleidig nach dir
blicken, und ich werde keine Schonung üben; nein, ich will dir nach
deinem ganzen Tun vergelten, und die Folgen deiner Greuel sollen sich
bei dir fühlbar machen✲, damit ihr erkennt, daß ich, der HERR, es bin,
der da schlägt.

\bibleverse{10}Siehe, da ist der Tag! Siehe, da kommt er! Das
Verhängnis\textless sup title=``oder: der Kranz =~die
Krone''\textgreater✲ ist aufgesproßt, die Rute\textless sup
title=``oder: der Stab, Stecken''\textgreater✲ aufgewachsen, der Übermut
steht in Blüte! \bibleverse{11}Die Gewalttätigkeit hat sich zur Rute des
Frevels erhoben\textless sup title=``oder: entwickelt''\textgreater✲;
nichts wird von ihnen übrigbleiben, nichts von ihrem
Gepränge\textless sup title=``oder: von ihrer Menge''\textgreater✲,
nichts von ihrem Reichtum, nichts von ihrer Herrlichkeit!
\bibleverse{12}Es kommt die Zeit, der Tag ist nahe! Der Käufer freue
sich nicht, und der Verkäufer gräme sich nicht! Denn Zornglut ergeht
über all ihr Gepränge\textless sup title=``oder: ihre
Menge''\textgreater✲. \bibleverse{13}Denn der Verkäufer wird nicht
wieder zu dem verkauften Gut gelangen, wenn er auch noch so lange unter
den Lebenden lebt; denn meine Zornglut gegen all ihr Gepränge kehrt
nicht um, und keiner wird infolge seiner Verschuldung sein Leben in
festem Besitz haben. \bibleverse{14}Stoßt immerhin in die Trompete und
rüstet alles zu: es ist doch keiner da, der zum Kampf auszieht! Denn
meine Zornglut ergeht über all ihr Gepränge\textless sup title=``vgl.
V.12''\textgreater✲: \bibleverse{15}das Schwert draußen, die Pest und
die Hungersnot drinnen! Wer auf dem Felde ist, wird durchs Schwert den
Tod finden; und wer in der Stadt ist, den werden Hunger und Pest ums
Leben bringen; \bibleverse{16}und wenn Füchtlinge von ihnen entrinnen,
so werden sie auf den Bergen sein wie die Tauben in den Schluchten,
allesamt girrend\textless sup title=``oder: klagend''\textgreater✲, ein
jeder um seiner Verschuldung willen. \bibleverse{17}Alle Arme werden
schlaff herabsinken und alle Knie wie Wasser zerfließen;
\bibleverse{18}sie werden sich mit Sackleinen umgürten\textless sup
title=``=~sich in Trauergewänder hüllen''\textgreater✲, und Schauder
wird ihre Glieder durchrieseln; auf allen Gesichtern wird Schamröte
liegen und auf all ihren Häuptern Kahlheit. \bibleverse{19}Ihr Silber
werden sie auf die Straßen werfen, und ihr Gold wird ihnen als Unrat
gelten; denn ihr Silber und ihr Gold vermag sie am Tage des göttlichen
Zorns nicht zu retten: ihren Hunger werden sie damit nicht zu stillen
und ihren Leib damit nicht zu füllen vermögen; denn es ist für sie der
Anstoß\textless sup title=``oder: Anlaß''\textgreater✲ zu ihrer
Verschuldung gewesen. \bibleverse{20}Der daraus verfertigte kostbare
Schmuck hat sie zur Überhebung verführt, und sie haben ihre greulichen
Götterbilder, ihre scheußlichen Götzen, daraus hergestellt; darum will
ich es ihnen zum Unrat machen \bibleverse{21}und will es den Fremden zum
Raube und den Gottlosesten der Erde zur Plünderung preisgeben, damit sie
es entweihen. \bibleverse{22}Und ich will mein Angesicht von ihnen
abwenden, damit sie mein Kleinod\textless sup title=``d.h. den Tempel
auf Zion''\textgreater✲ entweihen; Räuber sollen in dasselbe eindringen
und es entweihen. \bibleverse{23}Man verfertige die Kette! Denn das Land
ist voll von Blutschuld und die Stadt voll von Verbrechen.
\bibleverse{24}So will ich denn die schlimmsten der Heidenvölker
herbeiholen: die sollen ihre Häuser in Besitz nehmen; und ich will dem
Hochmut der Mächtigen ein Ende machen, und ihre Heiligtümer sollen
entweiht werden. \bibleverse{25}Angst kommt: da werden sie Rettung
suchen, aber keine ist zu finden. \bibleverse{26}Unglück über Unglück
kommt, und eine Schreckenskunde nach der andern trifft ein: da werden
sie (vergeblich) eine Weissagung von Propheten verlangen, und den
Priestern wird die Belehrung fehlen und den Ältesten der gute Rat.
\bibleverse{27}Der König wird trauern\textless sup title=``oder: sich
härmen''\textgreater✲ und der Fürst sich in Entsetzen kleiden, und dem
Volk im Lande werden die Arme vor Schrecken gelähmt sein. Nach ihrem
ganzen Tun will ich mit ihnen verfahren und auf Grund ihrer eigenen
Rechtsbestimmungen sie richten: dann werden sie erkennen, daß ich der
HERR bin.«

\hypertarget{iii.-das-gesicht-von-den-abguxf6ttischen-greueln-in-jerusalem-und-von-dem-guxf6ttlichen-gericht-uxfcber-die-stadt-kap.-8-11}{%
\subsection{III. Das Gesicht von den abgöttischen Greueln in Jerusalem
und von dem göttlichen Gericht über die Stadt (Kap.
8-11)}\label{iii.-das-gesicht-von-den-abguxf6ttischen-greueln-in-jerusalem-und-von-dem-guxf6ttlichen-gericht-uxfcber-die-stadt-kap.-8-11}}

\hypertarget{die-greuel-des-guxf6tzendienstes-verehrung-fremder-gottheiten-im-tempel-zu-jerusalem}{%
\subsubsection{1. Die Greuel des Götzendienstes (=~Verehrung fremder
Gottheiten) im Tempel zu
Jerusalem}\label{die-greuel-des-guxf6tzendienstes-verehrung-fremder-gottheiten-im-tempel-zu-jerusalem}}

\hypertarget{a-einleitung}{%
\paragraph{a) Einleitung}\label{a-einleitung}}

\hypertarget{section-7}{%
\section{8}\label{section-7}}

\bibleverse{1}Es begab sich sodann im sechsten Jahre, am fünften Tage
des sechsten Monats, während ich in meinem Hause saß und die Ältesten
von Juda vor mir saßen, da fiel dort die Hand Gottes des HERRN auf mich;
\bibleverse{2}ich hatte ein Gesicht: Ich sah plötzlich eine Gestalt, die
wie ein Mann aussah; von dem Körperteile an, der wie seine Hüften
aussah, nach unten zu war's wie Feuer, und von seinen Hüften aufwärts
sah es wie Lichtglanz aus, wie der Schimmer von Glanzerz.
\bibleverse{3}Da streckte er etwas aus, das wie eine Hand gebildet war,
und faßte mich bei den Locken meines Hauptes; dann hob die Gotteskraft
mich zwischen Erde und Himmel empor und brachte mich im Zustand der
Verzückung nach Jerusalem an den Eingang des Tores zum inneren Vorhof,
das nach Norden zu liegt, woselbst das die Eifersucht des HERRN
erregende Eiferbild seinen Standort hatte. \bibleverse{4}Und siehe, dort
stand die Herrlichkeit des Gottes Israels in derselben Erscheinung, wie
ich sie in der Tal-Ebene\textless sup title=``vgl. Kap. 1''\textgreater✲
gesehen hatte.

\hypertarget{b-das-eiferbild-der-astarte}{%
\paragraph{b) Das Eiferbild (der
Astarte)}\label{b-das-eiferbild-der-astarte}}

\bibleverse{5}Da sagte er zu mir: »Menschensohn, erhebe deine Augen in
der Richtung nach Norden!« Als ich nun meine Augen nach Norden hin
richtete, sah ich das betreffende Eiferbild nördlich vom Altartor am
Eingang stehen. \bibleverse{6}Da sagte er zu mir: »Menschensohn, siehst
du wohl, was sie da treiben? Arge Greuel sind es, die das Haus Israel
hier verübt, um mich von meinem Heiligtum weit wegzutreiben. Aber du
wirst weiterhin noch ärgere Greuel zu sehen bekommen.«

\hypertarget{c-die-bilderkammer-mit-dem-guxf6tzendienst-der-siebzig-uxe4ltesten}{%
\paragraph{c) Die Bilderkammer mit dem Götzendienst der siebzig
Ältesten}\label{c-die-bilderkammer-mit-dem-guxf6tzendienst-der-siebzig-uxe4ltesten}}

\bibleverse{7}Darauf brachte er mich an den Eingang des Vorhofes; und
als ich mich umsah, bemerkte ich ein Loch in der Wand. \bibleverse{8}Da
sagte er zu mir: »Menschensohn, zwänge dich durch die Wand hindurch!«
Als ich mich nun durch die Wand hindurchgezwängt hatte, kam dort eine
Tür zum Vorschein. \bibleverse{9}Darauf sagte er zu mir: »Gehe hinein
und sieh dir die schlimmen Greuel an, die sie hier verüben!«
\bibleverse{10}Als ich nun hineingegangen war und mich umsah, fanden
sich da allerlei Abbildungen von scheußlichen kriechenden und
vierfüßigen Tieren und allerlei Götzen des Hauses Israel, ringsherum auf
die Wand gezeichnet. \bibleverse{11}Vor diesen standen siebzig Männer
von den Ältesten des Hauses Israel, und mitten unter ihnen stand
Jaasanja, der Sohn Saphans; ein jeder von ihnen hatte seine
Räucherpfanne in der Hand, und der Duft der Weihrauchwolken stieg empor.
\bibleverse{12}Da sagte er zu mir: »Hast du wohl gesehen, Menschensohn,
was die Ältesten des Hauses Israel hier im Verborgenen treiben, ein
jeder in seiner Bilderkammer? Denn sie sagen: ›Der HERR sieht uns nicht,
der HERR hat ja das Land verlassen!‹« \bibleverse{13}Dann fuhr er fort:
»Du wirst weiterhin noch ärgere Greuel sehen, die sie verüben.«

\hypertarget{d-die-verehrung-des-thammus-adonis-durch-die-frauen}{%
\paragraph{d) Die Verehrung des Thammus (=~Adonis) durch die
Frauen}\label{d-die-verehrung-des-thammus-adonis-durch-die-frauen}}

\bibleverse{14}Hierauf brachte er mich an den Eingang des nördlichen
Tores am Tempel des HERRN; dort sah ich die Frauen sitzen, die den
Thammus beweinten. \bibleverse{15}Da sagte er zu mir: »Siehst du es
wohl, Menschensohn? Aber du wirst noch andere ärgere Greuel sehen als
diese.«

\hypertarget{e-verehrung-des-sonnengottes}{%
\paragraph{e) Verehrung des
Sonnengottes}\label{e-verehrung-des-sonnengottes}}

\bibleverse{16}Darauf führte er mich in den inneren Vorhof beim Tempel
des HERRN; dort sah ich am Eingang zum Tempelhause des HERRN, zwischen
der Vorhalle und dem Altar, ungefähr fünfundzwanzig Männer stehen; die
beteten, mit dem Rücken gegen den Tempel des HERRN und mit dem Gesicht
gegen Osten gewandt, die Sonne nach Osten hin an.

\hypertarget{f-guxf6tzendienst-im-ganzen-lande-juda-gottes-drohung}{%
\paragraph{f) Götzendienst im ganzen Lande Juda; Gottes
Drohung}\label{f-guxf6tzendienst-im-ganzen-lande-juda-gottes-drohung}}

\bibleverse{17}Da sagte er zu mir: »Hast du das gesehen, Menschensohn?
Genügt es dem Hause Juda nicht, die Greuel zu verüben, die sie hier
treiben, daß sie auch noch das Land mit Gewalttat erfüllen und mich
immer wieder zum Zorn reizen? Und nun siehe, da halten sie grüne Zweige
an die Nase! \bibleverse{18}So will denn auch ich im Grimm gegen sie
vorgehen: mein Auge soll nicht mehr mitleidig nach ihnen blicken, und
ich will keine Schonung mehr üben! Und wenn sie mir noch so laut in die
Ohren schreien, will ich doch nicht auf sie hören!«

\hypertarget{das-vernichtungsgericht-uxfcber-jerusalem}{%
\subsubsection{2. Das Vernichtungsgericht über
Jerusalem}\label{das-vernichtungsgericht-uxfcber-jerusalem}}

\hypertarget{a-die-bezeichnung-der-frommen-und-die-niedermetzelung-der-gottlosen-bewohner-jerusalems}{%
\paragraph{a) Die Bezeichnung der frommen und die Niedermetzelung der
gottlosen Bewohner
Jerusalems}\label{a-die-bezeichnung-der-frommen-und-die-niedermetzelung-der-gottlosen-bewohner-jerusalems}}

\hypertarget{section-8}{%
\section{9}\label{section-8}}

\bibleverse{1}Hierauf hörte ich ihn mit lauter Stimme so rufen: »Es naht
das Strafgericht über die Stadt! Ein jeder nehme jetzt also sein
Zerstörungswerkzeug in die Hand!« \bibleverse{2}Und siehe, da kamen
sechs Männer des Weges vom oberen Tore her, das nach Norden zu liegt;
ein jeder von ihnen hatte sein Werkzeug zum Zertrümmern in der Hand;
einer aber befand sich unter ihnen, der in ein linnenes Gewand gekleidet
war und ein Schreibzeug (am Gürtel) an seiner Hüfte\textless sup
title=``=~an der Seite''\textgreater✲ hatte; die kamen und traten neben
den ehernen Altar. \bibleverse{3}Die Herrlichkeit des Gottes Israels
aber hatte sich inzwischen von dem Cherubwagen, auf dem sie sich
befunden hatte, erhoben und war auf die Schwelle des Tempels getreten;
dort rief er dem in Linnen gekleideten Manne, der das Schreibzeug an
seiner Hüfte hatte, \bibleverse{4}die Worte zu: »Gehe mitten durch die
Stadt, mitten durch Jerusalem, und bringe ein Zeichen auf der Stirn der
Männer an, die da klagen und seufzen über all die Greuel, die innerhalb
der Stadt verübt werden!« \bibleverse{5}Zu den anderen aber sagte er so,
daß ich es hörte: »Geht hinter diesem her durch die Stadt und schlagt
darein! Eure Augen sollen kein Mitleid haben, und ihr dürft keine
Schonung üben! \bibleverse{6}Greise, Jünglinge und Jungfrauen, Kinder
und Frauen metzelt nieder, bis alles vernichtet ist! Aber alle, die das
Zeichen an sich tragen, laßt unberührt! Und bei meinem Heiligtum hier
macht den Anfang!« Da fingen sie bei jenen Ältesten an, die vor dem
Tempelhause standen. \bibleverse{7}Dann sagte er zu ihnen:
»Verunreinigt\textless sup title=``oder: entweiht''\textgreater✲ das
Tempelhaus und füllt die Vorhöfe mit Erschlagenen an! Zieht aus!« So
zogen sie denn aus und schlugen in der Stadt nieder.

\hypertarget{hesekiels-erfolglose-fuxfcrbitte}{%
\paragraph{Hesekiels erfolglose
Fürbitte}\label{hesekiels-erfolglose-fuxfcrbitte}}

\bibleverse{8}Als sie nun so mordeten und ich allein übriggeblieben war,
da warf ich mich auf mein Angesicht nieder, schrie laut auf und rief:
»Ach, HERR, mein Gott! Willst du denn alles umbringen, was von Israel
noch übrig ist, indem du deinen Zorn über Jerusalem ausschüttest?«
\bibleverse{9}Da antwortete er mir: »Die Schuld des Hauses Israel und
Juda ist über alle Maßen groß: das Land ist voll von Bluttaten und die
Stadt mit Verbrechen angefüllt! denn sie sagen: ›Der HERR hat das Land
verlassen‹ und ›der HERR sieht es nicht!‹ \bibleverse{10}So soll denn
auch mein Auge nicht mehr mitleidig blicken, und ich will keine Schonung
mehr üben; nein, ich will die Strafe für ihr ganzes Tun auf ihr Haupt
fallen lassen!« \bibleverse{11}Da erstattete der in Linnen gekleidete
Mann, an dessen Hüfte sich das Schreibzeug befand, Bericht mit den
Worten: »Ich habe getan, wie du mir geboten hast!«

\hypertarget{b-gottes-befehl-zur-einuxe4scherung-der-stadt-verbunden-mit-nochmaliger-beschreibung-des-guxf6ttlichen-cherubwagens-abzug-gottes-aus-dem-heiligtum}{%
\paragraph{b) Gottes Befehl zur Einäscherung der Stadt, verbunden mit
nochmaliger Beschreibung des göttlichen Cherubwagens; Abzug Gottes aus
dem
Heiligtum}\label{b-gottes-befehl-zur-einuxe4scherung-der-stadt-verbunden-mit-nochmaliger-beschreibung-des-guxf6ttlichen-cherubwagens-abzug-gottes-aus-dem-heiligtum}}

\hypertarget{section-9}{%
\section{10}\label{section-9}}

\bibleverse{1}Als ich nun hinschaute, sah ich auf dem
Himmelsgewölbe\textless sup title=``vgl. 1,22''\textgreater✲, das sich
über dem Haupt der Cherube befand, etwas, das wie Saphirstein aussah:
etwas wie ein Thron Gestaltetes wurde über ihnen sichtbar.
\bibleverse{2}Da gab er dem in Linnen gekleideten Manne den Befehl:
»Tritt in den Raum zwischen dem Räderwerk unten am Cherubwagen hinein,
fülle deine Hände mit glühenden Kohlen aus dem Raum zwischen den
Cheruben und streue sie über die Stadt hin!« Da ging er vor meinen Augen
hinein. \bibleverse{3}Der Cherubwagen stand aber an der Südseite des
Tempelhauses, als der Mann hineinging, während die Wolke den inneren
Vorhof erfüllte. \bibleverse{4}Da erhob sich die Herrlichkeit des HERRN
von dem Cherubwagen nach der Schwelle des Tempels hin, so daß der Tempel
mit der Wolke erfüllt wurde und der Vorhof voll vom Lichtglanz der
Herrlichkeit des HERRN war; \bibleverse{5}und das Rauschen der Flügel
der Cherube war bis in den äußeren Vorhof vernehmbar gleich der Stimme
Gottes, des Allmächtigen, wenn er redete. \bibleverse{6}Als er nun dem
in Linnen gekleideten Manne den Befehl erteilt hatte, Feuer aus dem Raum
zwischen dem Räderwerk, aus dem Raum zwischen den Cheruben, zu nehmen,
und dieser hineingegangen und neben das eine Rad getreten war,
\bibleverse{7}da streckte einer der Cherube seine Hand aus dem zwischen
den Cheruben befindlichen Raum hervor nach dem Feuer hin, das sich in
dem Raum zwischen den Cheruben befand, und hob einen Teil davon ab und
gab es dem in Linnen gekleideten Manne in die Hände; der nahm es und
ging weg. \bibleverse{8}{[}Es wurde aber an den Cheruben etwas, das wie
eine Menschenhand gebildet war, unter ihren Flügeln sichtbar.
\bibleverse{9}Als ich nämlich hinschaute, sah ich vier Räder, die neben
den Cheruben waren, immer ein Rad neben jedem Cherub; die Räder aber
waren wie glänzender Chrysolithstein anzusehen. \bibleverse{10}Was aber
ihr Aussehen betrifft, so hatten die vier alle dieselbe Gestalt, wie
wenn ein Rad innerhalb des andern wäre. \bibleverse{11}Wenn sie sich in
Bewegung setzten, konnten sie nach allen vier Seiten hin gehen: sie
brauchten sich nicht zu wenden, wenn sie gingen; sie gingen vielmehr
immer nach der Richtung, die der Vordere\textless sup title=``d.h. der
vorn befindliche Cherub''\textgreater✲ einschlug, hinter ihm her, ohne
eine Wendung zu machen, wenn sie gingen. \bibleverse{12}Der ganze Leib
der Cherube aber, auch ihr Rücken, ihre Arme und Flügel und ebenso auch
die Räder, waren bei allen vieren ringsum voll von Augen.
\bibleverse{13}Was aber die Räder betrifft, so führten sie, wie ich mit
eigenen Ohren hörte, den Namen ›Wirbelwind‹\textless sup title=``oder:
Räderwerk''\textgreater✲. \bibleverse{14}Ein jeder von den Cheruben
hatte vier Gesichter: das eine Gesicht war ein Stiergesicht, das zweite
ein Menschengesicht, das dritte ein Löwengesicht und das vierte ein
Adlergesicht.{]} \bibleverse{15}Und die Cherube erhoben sich {[}es waren
dies dieselben lebenden Wesen, die ich schon am Flusse Kebar gesehen
hatte. \bibleverse{16}Wenn nämlich die Cherube sich in Bewegung setzten,
so fingen auch die Räder neben ihnen an zu laufen; und wenn die Cherube
ihre Flügel erhoben, um vom Erdboden emporzusteigen, so entfernten sich
die Räder nicht von ihrer Seite: \bibleverse{17}wenn jene stehenblieben,
so standen auch sie still, und wenn jene sich erhoben, so erhoben auch
sie sich mit ihnen; denn der Geist der lebenden Wesen war in
ihnen{]}.~--

\hypertarget{der-auszug-der-herrlichkeit-gottes-aus-dem-heiligtum}{%
\paragraph{Der Auszug der Herrlichkeit Gottes aus dem
Heiligtum}\label{der-auszug-der-herrlichkeit-gottes-aus-dem-heiligtum}}

\bibleverse{18}Darauf verließ die Herrlichkeit des HERRN die Schwelle
des Tempelhauses und nahm ihren Stand wieder über den
Cheruben\textless sup title=``oder: auf dem Cherubwagen''\textgreater✲.
\bibleverse{19}Da schwangen die Cherube ihre Flügel und stiegen vor
meinen Augen vom Erdboden empor, indem sie sich hinwegbegaben, und die
Räder zugleich mit ihnen. Aber am Eingang des östlichen Tores des
Tempels des HERRN machten sie wieder halt, während die Herrlichkeit des
Gottes Israels sich oben über ihnen befand. \bibleverse{20}Es waren dies
dieselben lebenden Wesen, die ich unterhalb des Gottes Israels schon am
Flusse Kebar gesehen hatte; und ich erkannte\textless sup title=``oder:
wußte nun''\textgreater✲, daß es Cherube waren. \bibleverse{21}Ein jeder
hatte vier Gesichter und jeder vier Flügel; und etwas, das wie eine
Menschenhand gebildet war, befand sich unter ihren Flügeln.
\bibleverse{22}Was aber die äußere Erscheinung ihrer Gesichter betrifft,
so waren es dieselben Gesichter, die ich schon am Flusse Kebar gesehen
hatte; sie gingen ein jeder geradeaus vor sich hin\textless sup
title=``vgl. 1,12''\textgreater✲.

\hypertarget{c-das-gottesgericht-uxfcber-die-schlimmsten-volksverfuxfchrer-verbannung-umkehr-und-erneuerung-des-volkes-gott-verluxe4uxdft-die-stadt}{%
\paragraph{c) Das Gottesgericht über die schlimmsten Volksverführer;
Verbannung, Umkehr und Erneuerung des Volkes; Gott verläßt die
Stadt}\label{c-das-gottesgericht-uxfcber-die-schlimmsten-volksverfuxfchrer-verbannung-umkehr-und-erneuerung-des-volkes-gott-verluxe4uxdft-die-stadt}}

\hypertarget{section-10}{%
\section{11}\label{section-10}}

\bibleverse{1}Darauf hob die Gotteskraft mich empor und brachte mich an
das östliche Tor am Tempel des HERRN, das nach Osten zu liegt; dort sah
ich am Eingang des Tores fünfundzwanzig Männer stehen und erblickte in
ihrer Mitte die Häupter\textless sup title=``oder: Fürsten,
Obersten''\textgreater✲ des Volkes, Jaasanja, den Sohn Assurs, und
Pelatja, den Sohn Benajas. \bibleverse{2}Da sagte er\textless sup
title=``d.h. Gott''\textgreater✲ zu mir: »Menschensohn, das sind die
Männer, die in dieser Stadt auf Unheil sinnen und böse Ratschläge
erteilen, \bibleverse{3}die da sagen: ›Sind nicht erst kürzlich Häuser
erbaut worden? Die Stadt ist der Kessel, und wir sind das Fleisch.‹
\bibleverse{4}Darum weissage gegen sie, ja weissage, Menschensohn!«
\bibleverse{5}Da fiel der Geist des HERRN auf mich und sagte zu mir:
»Sprich: So hat Gott der HERR gesprochen: ›So redet ihr, Haus Israel,
und die Gedanken, die in eurem Innern aufsteigen, kenne ich wohl.
\bibleverse{6}Ihr habt die Zahl der von euch Erschlagenen\textless sup
title=``oder: Ermordeten''\textgreater✲ in dieser Stadt groß gemacht und
die Straßen hier mit Erschlagenen angefüllt.‹ \bibleverse{7}Darum hat
Gott der HERR so gesprochen: ›Die von euch Erschlagenen, die ihr im
Innern der Stadt hingestreckt habt, die sind das Fleisch, und sie ist
der Kessel; euch aber werde ich aus ihrer Mitte hinausführen.
\bibleverse{8}Vor dem Schwert fürchtet ihr euch, das Schwert will ich
über euch kommen lassen!‹ -- so lautet der Ausspruch Gottes des HERRN.
\bibleverse{9}›Und ich will euch mitten aus dieser Stadt wegführen und
euch in Feindeshand fallen lassen und Strafgerichte an euch vollziehen.
\bibleverse{10}Durch das Schwert sollt ihr fallen: an der Grenze Israels
will ich Gericht über euch halten, damit ihr erkennt, daß ich der HERR
bin! \bibleverse{11}Diese Stadt soll für euch nicht der Kessel sein, und
ihr sollt nicht das Fleisch in ihr sein; an der Grenze Israels will ich
Gericht über euch halten! \bibleverse{12}Dann werdet ihr erkennen, daß
ich der HERR bin, ich, nach dessen Satzungen ihr nicht gewandelt und
dessen Geboten ihr nicht nachgekommen seid; vielmehr habt ihr nach den
Bräuchen der Heidenvölker gelebt, die rings um euch her wohnen!‹«~--
\bibleverse{13}Da begab es sich, während ich so als Prophet redete, daß
Pelatja, der Sohn Benajas, tot niederstürzte. Da warf ich mich auf mein
Angesicht nieder, schrie laut und rief aus: »Ach, HERR, mein Gott!
willst du denn den letzten Rest Israels gänzlich vernichten?«

\hypertarget{die-verbannung-heimkehr-und-erneuerung-des-gottesvolkes}{%
\paragraph{Die Verbannung, Heimkehr und Erneuerung des
Gottesvolkes}\label{die-verbannung-heimkehr-und-erneuerung-des-gottesvolkes}}

\bibleverse{14}Hierauf erging das Wort des HERRN an mich folgendermaßen:
\bibleverse{15}»Menschensohn, deine Brüder, ja deine Brüder, deine
Mitverbannten und das ganze Haus Israel, sie alle sind es, von denen die
Bewohner Jerusalems sagen: ›Sie sind nun fern vom HERRN; uns ist das
Land zu Besitz gegeben!‹ \bibleverse{16}Darum sage zu ihnen: ›So hat
Gott der HERR gesprochen: Ich habe sie zwar in die Ferne unter die
Heidenvölker gebracht und sie in die Länder zerstreut und bin ihnen nur
wenig zum Heiligtum\textless sup title=``=~nur zu einem spärlichen
Ersatz des Heiligtums''\textgreater✲ geworden in den Ländern, in die sie
gekommen sind.‹ \bibleverse{17}Darum sage zu ihnen: ›So hat Gott der
HERR gesprochen: Aber ich will sie aus den Völkern sammeln und sie aus
den Ländern, in die sie zerstreut worden sind, wieder zusammenbringen
und ihnen das Land Israel zurückgeben.‹ \bibleverse{18}Wenn sie dann
dorthin zurückgekehrt sind und all seine scheußlichen Götzen und all
seine Greuel aus ihm weggeschafft haben, \bibleverse{19}will ich ihnen
ein anderes Herz verleihen und ihnen einen neuen Geist eingeben; ich
will das steinerne Herz aus ihrer Brust herausnehmen und ihnen ein Herz
von Fleisch einsetzen, \bibleverse{20}damit sie nach meinen Satzungen
wandeln und meine Gebote beobachten und nach ihnen tun: alsdann sollen
sie mein Volk sein, und ich will ihr Gott sein. \bibleverse{21}Die aber,
deren Herz ihren scheußlichen Götzen anhängt und ihren Greueln zugewandt
bleibt, die will ich für ihren Wandel\textless sup title=``oder: ihr
ganzes Tun''\textgreater✲ büßen lassen!« -- so lautet der Ausspruch
Gottes des HERRN.

\hypertarget{gott-verluxe4uxdft-das-stadtgebiet-hesekiel-erwacht-aus-der-verzuxfcckung}{%
\paragraph{Gott verläßt das Stadtgebiet; Hesekiel erwacht aus der
Verzückung}\label{gott-verluxe4uxdft-das-stadtgebiet-hesekiel-erwacht-aus-der-verzuxfcckung}}

\bibleverse{22}Darauf erhoben die Cherube ihre Flügel, und die Räder
setzten sich zugleich mit ihnen in Bewegung, während die Herrlichkeit
des Gottes Israels sich oben über ihnen befand\textless sup
title=``=~oben über ihnen thronte''\textgreater✲. \bibleverse{23}Da
stieg die Herrlichkeit des HERRN aus dem Bereich der Stadt empor und
machte auf dem Berge halt, der östlich von der Stadt liegt.
\bibleverse{24}Mich aber hob die Gotteskraft empor und brachte mich im
Zustand der Verzückung zurück ins Chaldäerland zu den in der
Verbannung\textless sup title=``oder: Gefangenschaft''\textgreater✲
Lebenden, und das Gesicht, das ich geschaut hatte, entschwand meinen
Blicken. \bibleverse{25}Hierauf teilte ich den in der Gefangenschaft
Lebenden alle Worte mit, die der HERR mich hatte schauen\textless sup
title=``=~bei dem Gesicht vernehmen''\textgreater✲ lassen.

\hypertarget{iv.-zweite-reihe-von-drohweissagungen-und-strafreden-gegen-jerusalem-und-juda-kap.-12-19}{%
\subsection{IV. Zweite Reihe von Drohweissagungen und Strafreden gegen
Jerusalem und Juda (Kap.
12-19)}\label{iv.-zweite-reihe-von-drohweissagungen-und-strafreden-gegen-jerusalem-und-juda-kap.-12-19}}

\hypertarget{zwei-neue-sinnbildliche-handlungen-des-propheten-zur-darstellung-der-dem-volke-und-der-stadt-bevorstehenden-not}{%
\subsubsection{1. Zwei neue sinnbildliche Handlungen des Propheten zur
Darstellung der dem Volke und der Stadt bevorstehenden
Not}\label{zwei-neue-sinnbildliche-handlungen-des-propheten-zur-darstellung-der-dem-volke-und-der-stadt-bevorstehenden-not}}

\hypertarget{a-die-auswanderung-des-propheten-als-veranschaulichung-der-wegfuxfchrung-von-fuxfcrst-und-volk-in-die-verbannung-oder-gefangenschaft}{%
\paragraph{a) Die Auswanderung des Propheten als Veranschaulichung der
Wegführung von Fürst und Volk in die Verbannung (oder:
Gefangenschaft)}\label{a-die-auswanderung-des-propheten-als-veranschaulichung-der-wegfuxfchrung-von-fuxfcrst-und-volk-in-die-verbannung-oder-gefangenschaft}}

\hypertarget{section-11}{%
\section{12}\label{section-11}}

\bibleverse{1}Hierauf erging das Wort des HERRN an mich folgendermaßen:
\bibleverse{2}»Menschensohn, du wohnst inmitten des widerspenstigen
Geschlechts, das Augen hat zum Sehen und doch nicht sieht, das Ohren hat
zum Hören und doch nicht hört; denn sie sind ein widerspenstiges
Geschlecht. \bibleverse{3}Darum du, Menschensohn: mache dir
Auswanderungsgeräte zurecht und ziehe bei Tage vor ihren Augen aus und
wandere vor ihren Augen von deiner Wohnstätte weg nach einem andern Ort:
vielleicht kommen sie dann zur Erkenntnis, daß sie ein widerspenstiges
Geschlecht sind. \bibleverse{4}Schaffe also deine Geräte wie
Auswanderungsgeräte bei Tage vor ihren Augen hinaus; du selbst aber
sollst am Abend vor ihren Augen ausziehen, wie jemand auszieht, der
auswandern will. \bibleverse{5}Vor ihren Augen brich dir ein Loch durch
die Wand und gehe durch dieses hinaus. \bibleverse{6}Vor ihren Augen
nimm (deine Geräte) auf die Schulter und in der Dunkelheit ziehe aus;
das Gesicht verhülle dir, so daß du das Land nicht sehen kannst; denn
ich habe dich zu einem Wahrzeichen für das Haus Israel bestimmt.«

\bibleverse{7}Da tat ich so, wie mir geboten worden war: meine Geräte
trug ich wie Auswanderungsgeräte hinaus, und zwar bei Tage, am Abend
aber brach ich mir ein Loch durch die Wand {[}mit der Hand{]}; in der
Dunkelheit ging ich hinaus und trug meine Sachen vor ihren Augen auf der
Schulter.

\bibleverse{8}Am nächsten Morgen aber erging das Wort des HERRN an mich
folgendermaßen: \bibleverse{9}»Menschensohn, haben nicht die vom Hause
Israel, das widerspenstige Geschlecht, dich gefragt: ›Was machst du da?‹
\bibleverse{10}Sage zu ihnen: ›So hat Gott der HERR gesprochen: Auf den
Fürsten in Jerusalem bezieht sich dieser Gottesspruch und auf das ganze
Haus Israel, das darin wohnt. \bibleverse{11}Sage zu ihnen: Ich bin ein
Wahrzeichen für euch; wie ich es hier gemacht habe, so wird es auch
ihnen ergehen: in die Verbannung werden sie als Gefangene ziehen müssen.
\bibleverse{12}Der Fürst aber, der sich mitten unter ihnen befindet,
wird seine Sachen\textless sup title=``oder: Geräte''\textgreater✲ auf
der Schulter tragen und sich im Finstern davonmachen; durch die Mauer
wird er sich ein Loch brechen, um da hinauszuziehen; das Gesicht wird er
sich verhüllen, damit er das Land mit seinen Augen nicht mehr sehe.
\bibleverse{13}Ich werde aber mein Fangnetz über ihn ausbreiten, damit
er in meinem Garn gefangen wird, und ich werde ihn nach Babylon bringen
in das Land der Chaldäer, aber auch das wird er nicht sehen; und dort
wird er sterben. \bibleverse{14}Und seine ganze Umgebung, seine Helfer
und seine Kriegsscharen insgesamt, will ich in alle Winde zerstreuen und
das Schwert hinter ihnen her zücken; \bibleverse{15}dann werden sie
erkennen, daß ich der HERR bin, wenn ich sie unter die Heidenvölker
zerstreue und sie in die Länder versprenge. \bibleverse{16}Nur einige
wenige von ihnen will ich vom Schwert, vom Hunger und von der Pest
verschont bleiben lassen, damit sie unter den Heidenvölkern, zu denen
sie kommen werden, all ihre Greuel erzählen, und auch diese erkennen,
daß ich der HERR bin.‹«

\hypertarget{b-das-essen-und-trinken-mit-beben-und-zittern-als-zeichen-der-belagerungsnuxf6te}{%
\paragraph{b) Das Essen und Trinken mit Beben und Zittern als Zeichen
der
Belagerungsnöte}\label{b-das-essen-und-trinken-mit-beben-und-zittern-als-zeichen-der-belagerungsnuxf6te}}

\bibleverse{17}Weiter erging das Wort des HERRN an mich folgendermaßen:
\bibleverse{18}»Menschensohn, dein Brot sollst du mit Beben essen und
dein Wasser mit Zittern und in Angst trinken \bibleverse{19}und sollst
zum Volk des Landes sagen: ›So hat Gott der HERR in betreff der Bewohner
Jerusalems im Lande Israel gesprochen: Ihr Brot sollen sie in Angst
essen und ihr Wasser mit Entsetzen trinken, weil ihr Land veröden soll,
seiner Fülle beraubt wegen der Gottlosigkeit aller seiner Bewohner.
\bibleverse{20}Ihre Städte, die jetzt noch bewohnt sind, sollen veröden,
und das Land soll zur Wüste werden, damit ihr erkennt, daß ich der HERR
bin.‹«

\hypertarget{uxfcber-wahre-und-falsche-weissagung}{%
\subsubsection{2. Über wahre und falsche
Weissagung}\label{uxfcber-wahre-und-falsche-weissagung}}

\hypertarget{a-zwei-drohworte-an-die-veruxe4chter-der-wahren-weissagung}{%
\paragraph{a) Zwei Drohworte an die Verächter der wahren
Weissagung}\label{a-zwei-drohworte-an-die-veruxe4chter-der-wahren-weissagung}}

\hypertarget{aa-an-die-welche-der-weissagung-uxfcberhaupt-allen-wert-absprachen}{%
\subparagraph{aa) An die, welche der Weissagung überhaupt allen Wert
absprachen}\label{aa-an-die-welche-der-weissagung-uxfcberhaupt-allen-wert-absprachen}}

\bibleverse{21}Hierauf erging das Wort des HERRN an mich folgendermaßen:
\bibleverse{22}»Menschensohn, was für eine Redensart ist da bei euch im
Lande Israel im Gebrauch, daß man sagt: ›Die Zeit zieht sich Tag für Tag
hin, und alle Weissagung wird hinfällig‹? \bibleverse{23}Darum sage zu
ihnen: ›So hat Gott der HERR gesprochen: Ich will dieser Redensart ein
Ende machen; man soll sie in Israel nicht länger im Munde führen.‹ Sage
ihnen vielmehr: ›Nahe herbeigekommen ist die Zeit und die Erfüllung
aller Weissagungen! \bibleverse{24}Denn es wird hinfort keine täuschende
Weissagung und keine trügerische Prophezeiung mehr im Hause Israel
geben; \bibleverse{25}sondern ich, der HERR, werde reden, und was ich
rede, das wird auch eintreffen, und zwar ohne längeren Verzug! Ja, noch
in euren Tagen, du widerspenstiges Geschlecht, werde ich einen Ausspruch
tun und ihn auch zur Ausführung bringen!‹« -- so lautet der Ausspruch
Gottes des HERRN.

\hypertarget{bb-an-die-welche-behaupteten-hesekiels-weissagung-beziehe-sich-auf-die-ferne-zukunft}{%
\subparagraph{bb) An die, welche behaupteten, Hesekiels Weissagung
beziehe sich auf die ferne
Zukunft}\label{bb-an-die-welche-behaupteten-hesekiels-weissagung-beziehe-sich-auf-die-ferne-zukunft}}

\bibleverse{26}Weiter erging das Wort des HERRN an mich folgendermaßen:
\bibleverse{27}»Menschensohn, wisse wohl: die vom Hause Israel sagen:
›Die Weissagungen, welche der da\textless sup title=``d.h.
Ezechiel''\textgreater✲ infolge von Offenbarungen ausspricht, gehen auf
lange Zeit hinaus, und er prophezeit auf weite Zeiträume\textless sup
title=``=~ferne Zukunft''\textgreater✲ hinaus.‹ \bibleverse{28}Darum
sage zu ihnen: ›So hat Gott der HERR gesprochen: Für keines meiner Worte
soll es noch längeren Verzug geben; jedes Wort, das ich ausspreche, soll
sich auch erfüllen!‹« -- so lautet der Ausspruch Gottes des HERRN.

\hypertarget{b-gegen-die-falschen-propheten-und-prophetinnen}{%
\paragraph{b) Gegen die falschen Propheten und
Prophetinnen}\label{b-gegen-die-falschen-propheten-und-prophetinnen}}

\hypertarget{section-12}{%
\section{13}\label{section-12}}

\bibleverse{1}Weiter erging das Wort des HERRN an mich folgendermaßen:
\bibleverse{2}»Menschensohn, weissage gegen die Propheten Israels!
Weissage und sprich zu denen, die nach eigener Eingebung weissagen:
›Hört das Wort des HERRN! \bibleverse{3}So hat Gott der HERR gesprochen:
Wehe über die gewissenlosen Propheten, die ihrem eigenen Geist nachgehen
und dem, was sie gar nicht gesehen haben! \bibleverse{4}Wie Füchse in
den Trümmerstätten sind deine Propheten geworden, Israel.
\bibleverse{5}Sie sind nicht in die Risse✲ eingetreten und haben keine
Mauer um das Haus Israel her aufgeführt, damit es feststehen möchte im
Kampf am Tage des HERRN. \bibleverse{6}Ihr Prophezeien ist Lüge gewesen
und ihr Wahrsagen Trug, sooft sie sagten: ›So lautet der Ausspruch des
HERRN!‹, obwohl der HERR sie nicht gesandt hatte, und dann darauf
warteten, daß er ihren Ausspruch in Erfüllung gehen ließe.
\bibleverse{7}Habt ihr nicht nur Truggesichte geschaut und
Lügenweissagungen ausgesprochen, sooft ihr sagtet: ›So lautet der
Ausspruch des HERRN!‹, wiewohl ich nicht geredet hatte?‹«

\bibleverse{8}Darum hat Gott der HERR so gesprochen: »Weil ihr Trug
geredet und Lügen prophezeit habt, darum will ich nunmehr an
euch\textless sup title=``=~gegen euch vorgehen''\textgreater✲« -- so
lautet der Ausspruch Gottes des HERRN. \bibleverse{9}»Ja, meine Hand
will ich ausstrecken gegen die Propheten, die Trug prophezeien und Lügen
wahrsagen! Sie sollen mit meinem Volk in keiner Gemeinschaft mehr stehen
und nicht in die Bürgerliste des Hauses Israel eingeschrieben werden,
auch nicht mehr in das Land Israel zurückkommen, damit ihr erkennt, daß
ich Gott der HERR bin. \bibleverse{10}Darum, ja darum, weil sie mein
Volk irreführen, indem sie von Heil reden, wiewohl kein Heil da ist, und
wenn (das Volk) sich eine Mauer baut, ihrerseits diese mit Kalk
übertünchen. \bibleverse{11}So sage denn zu diesen Tünchestreichern:
›(Die Mauer) soll einstürzen!‹ Wenn ein strömender\textless sup
title=``oder: überschwemmender''\textgreater✲ Regen kommt und ich
Hagelsteine fallen lasse und ein Sturmwind losbricht, \bibleverse{12}ja,
dann fällt die Mauer zusammen, und man wird euch höhnisch fragen: ›Wo
ist nun die Tünche geblieben, mit der ihr getüncht habt?‹«

\bibleverse{13}Darum hat Gott der HERR so gesprochen: »So will ich denn
einen Sturmwind losbrechen lassen in meinem Grimm, und ein
strömender\textless sup title=``oder: überschwemmender''\textgreater✲
Regen soll kommen infolge meines Zornes und Hagelsteine sollen fallen
infolge meines Grimms zur Vernichtung; \bibleverse{14}und ich will die
Mauer niederreißen, die ihr mit Kalk übertüncht habt, und will sie zu
Boden werfen, daß ihre Grundsteine bloßgelegt werden; und wenn sie
einstürzt, sollt ihr in ihrer Mitte\textless sup title=``d.h. innerhalb
ihrer Trümmer''\textgreater✲ den Untergang finden, damit ihr erkennt,
daß ich der HERR bin. \bibleverse{15}Wenn ich dann an der Mauer und an
denen, die sie mit Tünche bestrichen haben, meinen Grimm sich voll habe
auswirken lassen, wird man zu euch sagen: ›Verschwunden ist die Mauer,
und die Leute, die sie getüncht haben, sind auch nicht mehr da,
\bibleverse{16}die Propheten Israels, welche über Jerusalem weissagten
und für die Stadt Gesichte des Heils schauten, wiewohl kein Heil da
war‹« -- so lautet der Ausspruch Gottes des HERRN.

\hypertarget{gegen-die-falschen-prophetinnen-bzw.-wahrsagerinnen}{%
\paragraph{Gegen die falschen Prophetinnen (bzw.
Wahrsagerinnen)}\label{gegen-die-falschen-prophetinnen-bzw.-wahrsagerinnen}}

\bibleverse{17}»Du aber, Menschensohn, tritt gegen die Töchter deines
Volkes auf, die sich nach eigenem Gutdünken\textless sup title=``vgl.
V.2''\textgreater✲ als Prophetinnen gebärden; sprich dich gegen sie aus
\bibleverse{18}und sage: ›So hat Gott der HERR gesprochen: Wehe den
Weibern, die da Zauberbinden zusammennähen für alle Handgelenke und
Hüllen\textless sup title=``oder: Schleier''\textgreater✲ anfertigen für
Köpfe jedes Körperwuchses\textless sup title=``oder: jeder
Größe''\textgreater✲, um Seelen zu fangen! Seelen wollt ihr töten, die
zu meinem Volke gehören, und Seelen erhaltet ihr euch zugute am Leben?!
\bibleverse{19}Ihr entheiligt mich bei meinem Volk um ein paar Hände
voll Gerste und um einiger Bissen Brotes willen, um Seelen zu töten, die
nicht sterben sollen, und andere Seelen am Leben zu erhalten, die nicht
am Leben bleiben sollen, wobei ihr mein Volk belügt, das gern auf Lügen
hört.‹«

\bibleverse{20}Darum hat Gott der HERR so gesprochen: »Wisset wohl: ich
will nun gegen eure Zauberbinden vorgehen, mit denen ihr Seelen fangt,
und will sie euch von den Armen abreißen und die Seelen, die ihr
einfangt, frei fliegen lassen wie Vögel. \bibleverse{21}Auch eure
Kopfhüllen will ich zerreißen und mein Volk aus eurer Hand befreien: sie
sollen nicht länger als Beute in eurer Gewalt sein, und ihr sollt
erkennen, daß ich der HERR bin. \bibleverse{22}Weil ihr den Herzen der
Frommen durch Lügen wehegetan habt, denen ich keinen Schmerz zuzufügen
gedachte, dagegen die Gottlosen in ihrem Tun bestärkt habt, so daß sie
sich von ihrem bösen Wandel nicht abkehrten, um am Leben erhalten zu
bleiben: \bibleverse{23}darum sollt ihr nicht länger erlogene Gesichte
schauen und keine Wahrsagerei mehr treiben, sondern ich will mein Volk
euch aus den Händen reißen, dann werdet ihr erkennen, daß ich der HERR
bin.«

\hypertarget{c-ausschluuxdf-der-guxf6tzendiener-vom-gnadenmittel-der-befragung-gottes}{%
\paragraph{c) Ausschluß der Götzendiener vom Gnadenmittel der Befragung
Gottes}\label{c-ausschluuxdf-der-guxf6tzendiener-vom-gnadenmittel-der-befragung-gottes}}

\hypertarget{section-13}{%
\section{14}\label{section-13}}

\bibleverse{1}Es kamen aber einige von den Ältesten\textless sup
title=``oder: Vornehmsten''\textgreater✲ Israels zu mir und ließen sich
vor mir nieder. \bibleverse{2}Da erging das Wort des HERRN an mich
folgendermaßen: \bibleverse{3}»Menschensohn, diese Männer haben ihre
Götzen in ihr Herz geschlossen und sie als Anstoß zu ihrer Verschuldung
vor sich hingestellt: sollte ich mich da noch von ihnen befragen
lassen?«

\hypertarget{aa-das-gottesgericht-uxfcber-jeden-guxf6tzendiener-der-einen-propheten-befragt}{%
\subparagraph{aa) Das Gottesgericht über jeden Götzendiener, der einen
Propheten
befragt}\label{aa-das-gottesgericht-uxfcber-jeden-guxf6tzendiener-der-einen-propheten-befragt}}

\bibleverse{4}»Darum rede mit ihnen und sage zu ihnen: ›So hat Gott der
HERR gesprochen: Wenn irgend jemand aus dem Hause Israel seine Götzen in
sein Herz geschlossen hat und sie als Anstoß zu seiner Verschuldung vor
sich hinstellt und sich dennoch zu einem Propheten begibt: dem will ich,
der HERR, persönlich die Antwort erteilen trotz\textless sup
title=``oder: gemäß''\textgreater✲ der Menge seiner Götzen,
\bibleverse{5}um die vom Hause Israel, die sich um all ihrer Götzen
willen von mir abgewandt haben, am Herzen zu fassen.‹ \bibleverse{6}So
sage denn zum Hause Israel: ›So hat Gott der HERR gesprochen: Kehrt um!
Laßt euch von eurem Götzendienst bekehren und wendet eure Augen von all
euren Greueln ab! \bibleverse{7}Denn wenn irgend jemand vom Hause Israel
und von den Fremden, die in Israel wohnen, sich von mir lossagt und
seine Götzen in sein Herz schließt und sie als Anstoß zu seiner
Verschuldung vor sich hinstellt und sich trotzdem zu einem Propheten
begibt, um mich für sich\textless sup title=``=~in seiner
Angelegenheit''\textgreater✲ zu befragen, so will ich, der HERR, selbst
ihm die Antwort erteilen; \bibleverse{8}und zwar will ich gegen den
betreffenden Mann vorgehen und ihn zu einem warnenden Beispiel und zu
einem Sprichwort machen: ich will ihn aus der Mitte meines Volkes
ausrotten, damit ihr erkennet, daß ich der HERR bin.‹«

\hypertarget{bb-das-gottesgericht-uxfcber-den-einem-guxf6tzendiener-weissagenden-propheten}{%
\subparagraph{bb) Das Gottesgericht über den einem Götzendiener
weissagenden
Propheten}\label{bb-das-gottesgericht-uxfcber-den-einem-guxf6tzendiener-weissagenden-propheten}}

\bibleverse{9}»Wenn aber der Prophet sich dazu verleiten läßt, einen
Ausspruch zu tun, dann habe ich, der HERR, diesen Propheten verleitet,
und ich werde meine Hand gegen ihn ausstrecken und ihn aus der Mitte
meines Volkes Israel ausrotten. \bibleverse{10}So sollen sie beide für
ihre Verschuldung büßen: der Prophet soll ebenso schuldig sein wie der
Befragende, \bibleverse{11}damit die vom Hause Israel fortan nicht mehr
von mir abirren und sich nicht länger durch all ihre Abfallsünden
beflecken; alsdann sollen sie mein Volk sein, und ich will ihr Gott
sein« -- so lautet der Ausspruch Gottes des HERRN.

\hypertarget{warum-gott-bei-dem-unwiderruflichen-strafgericht-uxfcber-jerusalem-einen-teil-der-gottlosen-bevuxf6lkerung-uxfcbrigbleiben-luxe4uxdft}{%
\subsubsection{3. Warum Gott bei dem unwiderruflichen Strafgericht über
Jerusalem einen Teil der gottlosen Bevölkerung übrigbleiben
läßt}\label{warum-gott-bei-dem-unwiderruflichen-strafgericht-uxfcber-jerusalem-einen-teil-der-gottlosen-bevuxf6lkerung-uxfcbrigbleiben-luxe4uxdft}}

\bibleverse{12}Darauf erging das Wort des HERRN an mich folgendermaßen:
\bibleverse{13}»Menschensohn, wenn ein Land sich gegen mich versündigte,
indem es Treubruch beginge, und ich dann meine Hand gegen dieses Volk
ausstreckte und ihm den Stab\textless sup title=``oder: die
Stütze''\textgreater✲ des Brotes\textless sup title=``vgl.
4,16''\textgreater✲ zerbräche und Hungersnot ihm zusendete und Menschen
samt Vieh in ihm ausrottete, \bibleverse{14}und es befänden sich diese
drei Männer in seiner Mitte, Noah, Daniel und Hiob: so würden nur diese
drei ihr Leben infolge ihrer Gerechtigkeit retten« -- so lautet der
Ausspruch Gottes des HERRN. \bibleverse{15}»Oder wenn ich wilde Tiere
das Land durchstreifen ließe, damit sie es entvölkerten, auf daß es zu
einer Einöde würde, die niemand mehr aus Furcht vor den wilden Tieren
durchwanderte, \bibleverse{16}und jene drei Männer befänden sich in
seiner Mitte: so wahr ich lebe!« -- so lautet der Ausspruch Gottes des
HERRN --, »sie würden weder Söhne noch Töchter retten: nein, sie allein
würden gerettet, das Land aber würde zur Einöde werden.
\bibleverse{17}Oder wenn ich das Schwert über das betreffende Land
kommen ließe und dem Schwert geböte, durch das Land dahinzufahren, und
wenn ich dann Menschen samt Vieh darin ausrottete \bibleverse{18}und
jene drei Männer befänden sich in seiner Mitte: so wahr ich lebe!« -- so
lautet der Ausspruch Gottes des HERRN --, »sie würden weder Söhne noch
Töchter retten, sondern sie allein würden gerettet werden.
\bibleverse{19}Oder wenn ich die Pest in das betreffende Land schickte
und meinen Grimm als Blut(-regen) über das Land ausgösse, um Menschen
samt Vieh in ihm auszurotten, \bibleverse{20}und Noah, Daniel und Hiob
befänden sich in seiner Mitte: so wahr ich lebe!« -- so lautet der
Ausspruch Gottes des HERRN --, »sie würden weder Sohn noch Tochter
retten, sondern durch ihre Gerechtigkeit nur ihr eigenes Leben retten.«

\bibleverse{21}Und doch hat Gott der HERR so gesprochen: »Trotz alledem,
wenn ich meine vier schlimmen Strafgerichte: Schwert und Hunger, wilde
Tiere und Pest, über Jerusalem hereinbrechen lasse, um Menschen samt
Vieh darin auszurotten, \bibleverse{22}fürwahr, so soll doch eine Schar
von Geretteten in der Stadt übrigbleiben, welche Söhne und Töchter aus
ihr herausführen. Seht, wenn diese dann zu euch hierher kommen und ihr
ihren Wandel und ihr ganzes Tun seht, so werdet ihr euch über das
Unglück trösten, das ich über Jerusalem verhängt habe, über alle Leiden,
die ich über die Stadt habe kommen lassen. \bibleverse{23}Sie werden
euch dann einen Trost gewähren, wenn ihr ihren Wandel und ihr ganzes Tun
seht, und ihr werdet erkennen, daß ich alles, was ich ihr\textless sup
title=``d.h. der Stadt''\textgreater✲ habe widerfahren lassen, nicht
ohne Grund getan habe« -- so lautet der Ausspruch Gottes des HERRN.

\hypertarget{drei-gleichnisse-von-dem-unwert-jerusalems-und-judas}{%
\subsubsection{4. Drei Gleichnisse von dem Unwert Jerusalems und
Judas}\label{drei-gleichnisse-von-dem-unwert-jerusalems-und-judas}}

\hypertarget{a-jerusalem-das-unnuxfctze-holz-der-rebe}{%
\paragraph{a) Jerusalem das unnütze Holz der
Rebe}\label{a-jerusalem-das-unnuxfctze-holz-der-rebe}}

\hypertarget{section-14}{%
\section{15}\label{section-14}}

\bibleverse{1}Weiter erging das Wort des HERRN an mich folgendermaßen:
\bibleverse{2}»Menschensohn, was hat das Holz der Rebe vor allem andern
Holz voraus, die Weinranke, die sich unter\textless sup title=``oder:
an''\textgreater✲ den Bäumen des Waldes befindet? \bibleverse{3}Nimmt
man etwa Holz von ihr, um es zu einer Arbeit zu verwenden? Oder nimmt
man auch nur einen Pflock von ihr, um irgendeinen Gegenstand daran
aufzuhängen? \bibleverse{4}Nicht wahr, man wirft es ins Feuer, damit es
da verzehrt wird; hat das Feuer dann seine beiden Enden verzehrt und ist
sein Mittelstück angebrannt\textless sup title=``oder:
versengt''\textgreater✲, taugt es da noch zu irgend einer Arbeit?
\bibleverse{5}Nein, selbst wenn es noch unversehrt ist, kann man es zu
keiner Arbeit verwenden; wieviel weniger läßt es sich dann, wenn es vom
Feuer angefressen und versengt ist, zu irgend etwas verarbeiten!«

\hypertarget{deutung-des-gleichnisses}{%
\paragraph{Deutung des Gleichnisses}\label{deutung-des-gleichnisses}}

\bibleverse{6}»Darum sprich: ›So hat Gott der HERR gesprochen: Wie das
Holz der Rebe unter den Bäumen\textless sup title=``oder:
Hölzern''\textgreater✲ des Waldes ist, das ich dem Feuer zum Fraß
bestimmt habe, ebenso halte ich es mit den Bewohnern Jerusalems.
\bibleverse{7}Ich will gegen sie vorgehen: aus dem Feuer sind sie zwar
herausgekommen, doch das Feuer soll sie verzehren, damit ihr erkennt,
daß ich der HERR bin, wenn ich gegen sie vorgehe \bibleverse{8}und das
Land zur Wüste mache, weil sie Treubruch begangen haben‹« -- so lautet
der Ausspruch Gottes des HERRN.

\hypertarget{b-jerusalem-das-miuxdfratene-pflegekind-bzw.-das-untreue-eheweib}{%
\paragraph{b) Jerusalem das mißratene Pflegekind (bzw. das untreue
Eheweib)}\label{b-jerusalem-das-miuxdfratene-pflegekind-bzw.-das-untreue-eheweib}}

\hypertarget{aa-einleitung-mit-dem-hinweis-dauxdf-jerusalem-aus-dem-heidentum-stamme}{%
\subparagraph{aa) Einleitung mit dem Hinweis, daß Jerusalem aus dem
Heidentum
stamme}\label{aa-einleitung-mit-dem-hinweis-dauxdf-jerusalem-aus-dem-heidentum-stamme}}

\hypertarget{section-15}{%
\section{16}\label{section-15}}

\bibleverse{1}Weiter erging das Wort des HERRN an mich folgendermaßen:
\bibleverse{2}»Menschensohn, halte der Stadt Jerusalem ihre Greuel vor
\bibleverse{3}mit den Worten: ›So spricht Gott der HERR zu Jerusalem:
Deiner Herkunft und Geburt nach stammst du aus dem Lande der Kanaanäer:
dein Vater war ein Amoriter und deine Mutter eine Hethiterin.‹«

\hypertarget{bb-gottes-liebestaten-an-israel-in-den-anfangszeiten-bis-zur-einfuxfchrung-in-kanaan}{%
\subparagraph{bb) Gottes Liebestaten an Israel in den Anfangszeiten bis
zur Einführung in
Kanaan}\label{bb-gottes-liebestaten-an-israel-in-den-anfangszeiten-bis-zur-einfuxfchrung-in-kanaan}}

\bibleverse{4}»›Und was deine Geburt betrifft, so wurde dir an dem Tage,
als du zur Welt kamst, weder die Nabelschnur abgeschnitten, noch wurdest
du in einem Wasserbade rein gewaschen, noch mit Salz eingerieben und
nicht in Windeln gewickelt; \bibleverse{5}kein Auge blickte mitleidig
auf dich hin, um dir irgendeinen derartigen Liebesdienst zu erweisen und
sich deiner zu erbarmen; sondern du wurdest aufs freie Feld hingeworfen:
so wenig machte man sich aus deinem Leben am Tage deiner Geburt.
\bibleverse{6}Da kam ich an dir vorüber und sah dich in deinem Blut
zappeln und sagte zu dir, als du in deinem Blut dalagst: Du sollst
leben! Ja, ich sagte zu dir in deinem Blut: Bleibe leben
\bibleverse{7}und wachse heran wie die Grashalme auf der Flur! Da
wuchsest du heran und wurdest groß und gelangtest zu vollster
Jugendblüte: die Brüste wölbten sich dir, dein Haar sproßte kräftig;
doch du warst immer noch nackt und bloß. \bibleverse{8}Als ich nun
wieder an dir vorüberkam und dich sah, siehe, da war deine Zeit da, die
Zeit der Liebe! Da breitete ich meinen Mantelzipfel\textless sup
title=``vgl. Ruth 3,9''\textgreater✲ über dich aus und bedeckte deine
Blöße; ich schwur dir Treue und ging einen Bund mit dir ein‹ -- so
lautet der Ausspruch Gottes des HERRN --, ›und du wurdest mein.
\bibleverse{9}Dann wusch ich dich mit Wasser, spülte dein Blut von dir
ab und salbte dich mit Öl; \bibleverse{10}ich kleidete dich in bunte
Gewänder, ließ dich Schuhe von Seekuhfell anziehen und einen Kopfbund
von feiner Leinwand anlegen und hüllte dich in Seide; \bibleverse{11}ich
schmückte dich mit Geschmeide, ich legte dir Spangen an die Arme und
eine Kette um den Hals, \bibleverse{12}tat dir einen Ring an die Nase,
Gehänge an die Ohren und eine prächtige Krone aufs Haupt.
\bibleverse{13}So warst du geschmückt mit Gold und Silber, deine
Kleidung bestand aus feiner Leinwand, aus Seide und bunten Geweben; du
nährtest dich von Semmel, von Honig und Öl, wurdest immer schöner und
brachtest es bis zur Königswürde. \bibleverse{14}Dein Ruhm erscholl
unter den Völkern wegen deiner Schönheit; denn diese war vollkommen
infolge des herrlichen Schmuckes, den ich dir angelegt hatte‹« -- so
lautet der Ausspruch Gottes des HERRN.

\hypertarget{cc-israels-undank-verschuldungen-in-religiuxf6ser-und-politischer-beziehung}{%
\subparagraph{cc) Israels Undank (Verschuldungen in religiöser und
politischer
Beziehung)}\label{cc-israels-undank-verschuldungen-in-religiuxf6ser-und-politischer-beziehung}}

\bibleverse{15}»›Aber da verließest du dich auf deine Schönheit und
buhltest\textless sup title=``oder: triebst Unzucht''\textgreater✲ im
Vertrauen auf deine Berühmtheit und warfst dich mit deiner buhlerischen
Liebe an jeden Vorübergehenden weg, so daß du dich ihm preisgabst.
\bibleverse{16}Du nahmst von deinen Gewändern, machtest dir bunte
Opferhöhen und triebst dort deine Unzucht {[}wie sie nie vorgekommen ist
und nie wieder stattfinden wird{]}. \bibleverse{17}Dann nahmst du auch
deine prächtigen Geschmeide, die aus meinem Gold und meinem Silber, die
ich dir geschenkt hatte, angefertigt waren, und machtest dir Mannsbilder
daraus, mit denen du Buhlerei triebst; \bibleverse{18}auch nahmst du
deine bunten Gewänder und legtest sie ihnen an, und mein Öl und meinen
Weihrauch brachtest du vor ihren Augen dar; \bibleverse{19}und was ich
dir als Speise gegeben hatte, Semmel, Öl und Honig, die ich dich hatte
essen lassen, das setztest du ihnen als lieblich duftende Opfergabe vor.
Ja, das alles ist geschehen!‹ -- so lautet der Ausspruch Gottes des
HERRN. \bibleverse{20}›Auch nahmst du deine Söhne und Töchter, die du
mir geboren hattest, und schlachtetest sie ihnen zum Fraß. Genügte deine
Buhlerei noch nicht, \bibleverse{21}daß du auch noch meine Kinder
schlachten mußtest und sie hingabst, indem du sie ihnen als Opfer
verbranntest? \bibleverse{22}Und bei all deinen Greueln und deinen
Buhlereien dachtest du nicht an die Tage deiner Jugend zurück, wie du
damals nackt und bloß warst und zappelnd in deinem Blut dalagst!
\bibleverse{23}Und nach all deiner Bosheit -- wehe, wehe dir!‹ -- so
lautet der Ausspruch Gottes des HERRN -- ›kam es dahin,
\bibleverse{24}daß du dir erhöhte Opferplätze bautest und dir
Götzenstätten auf jedem freien Platz anlegtest; \bibleverse{25}an allen
Straßenecken bautest du dir deine erhöhten Götzenstätten und schändetest
deine Schönheit; denn für jeden Vorübergehenden spreiztest du deine
Beine und triebst es mit deiner Unzucht immer ärger. \bibleverse{26}Du
buhltest mit den Ägyptern, deinen Nachbarn, den starkgliedrigen, und
triebst es immer ärger mit deiner Unzucht, um mich zu erbittern.
\bibleverse{27}Da streckte ich denn meine Hand gegen dich aus und
minderte den dir bestimmten Lebensunterhalt und gab dich der Gier deiner
Feindinnen, der Töchter der Philister, preis, die sich ob deinem
unzüchtigen Treiben schämten. \bibleverse{28}Alsdann buhltest du auch
mit den Assyrern, weil du nie satt wurdest; du buhltest mit ihnen,
wurdest aber auch dadurch noch nicht gesättigt; \bibleverse{29}vielmehr
triebst du noch ärgere Buhlerei nach dem Krämerland Chaldäa hin; doch
auch davon wurdest du nicht satt. \bibleverse{30}Wie schmachtend war
doch dein Herz‹ -- so lautet der Ausspruch Gottes des HERRN --, ›daß du
dies alles verübtest, wie es eine zügellose Erzbuhlerin zu tun pflegt!
\bibleverse{31}Daß du dir an jeder Straßenecke einen erhöhten Opferplatz
anlegtest und dir Götzenstätten auf jedem freien Platze bautest! Und
dabei warst du nicht einmal wie eine gewöhnliche Buhlerin, daß du
Buhllohn eingesammelt hättest: \bibleverse{32}du ehebrecherisches Weib,
das statt ihres Ehemannes Fremde annahm! \bibleverse{33}Sonst gibt man
allen Dirnen Buhllohn; du aber gabst deinerseits allen deinen Liebhabern
Geschenke und erkauftest sie, damit sie von allen Seiten zu dir
eingingen, um Unzucht mit dir zu treiben. \bibleverse{34}So war es bei
dir in deiner Buhlerei umgekehrt wie sonst bei den Weibern: nicht dir
stellte man buhlerisch nach, sondern, indem du Buhllohn gabst, während
dir kein Lohn gegeben wurde, fand bei dir das Umgekehrte statt.‹«

\hypertarget{dd-gottes-strafurteil-uxfcber-die-ehebrecherin-und-kindesmuxf6rderin-gerade-die-vuxf6lker-deren-guxf6ttern-israel-gedient-hat-sollen-es-vernichten}{%
\subparagraph{dd) Gottes Strafurteil über die Ehebrecherin und
Kindesmörderin: gerade die Völker, deren Göttern Israel gedient hat,
sollen es
vernichten}\label{dd-gottes-strafurteil-uxfcber-die-ehebrecherin-und-kindesmuxf6rderin-gerade-die-vuxf6lker-deren-guxf6ttern-israel-gedient-hat-sollen-es-vernichten}}

\bibleverse{35}»›Darum, du Buhlerin, vernimmt das Wort des HERRN!
\bibleverse{36}So spricht Gott der HERR: Weil deine Verworfenheit sich
überallhin verbreitet hat, und du deine Blöße bei deinen Buhlereien vor
deinen Liebhabern aufgedeckt hast, und wegen all deiner greulichen
Götzen und um des Blutes deiner Kinder willen, die du ihnen hingegeben
hast: \bibleverse{37}darum will ich, wisse es wohl, alle deine
Liebhaber, zu denen du in Liebe entbrannt warst, zusammenholen, und zwar
alle, die du gern gehabt hast, samt allen denen, die dir zuwider
geworden sind: die will ich von allen Seiten ringsum gegen dich
zusammenholen und deine Blöße vor ihnen aufdecken, daß sie deine ganze
Scham zu sehen bekommen. \bibleverse{38}Sodann will ich dir nach den für
Ehebrecherinnen und Mörderinnen geltenden Rechtsbestimmungen das Urteil
sprechen und meinen Grimm und meine Eifersucht an dir stillen.
\bibleverse{39}Und ich will dich in ihre Gewalt geben, damit sie deine
erhöhten Opferplätze niederreißen und deine Götzenstätten zerstören; und
sie sollen dir deine Gewänder ausziehen und dir deine prächtigen
Geschmeide wegnehmen und dich nackt und bloß hinstellen.
\bibleverse{40}Dann werden sie eine Gemeindeversammlung gegen dich
berufen und dich steinigen und dich mit ihren Schwertern in Stücke
hauen; \bibleverse{41}auch werden sie deine Häuser mit Feuer verbrennen
und das Strafgericht an dir vollstrecken vor den Augen zahlreicher
Frauen. So will ich deiner Buhlerei ein Ende machen, und du sollst
fortan auch keinen Buhllohn mehr geben. \bibleverse{42}Wenn ich so
meinen Grimm an dir gestillt habe und meine Eifersucht gegen dich
geschwunden ist, so werde ich mich beruhigt fühlen und mich nicht mehr
zu entrüsten brauchen. \bibleverse{43}Weil du der Tage deiner Jugend
nicht gedacht und mich durch dies alles zum Zorn gereizt hast, so will
auch ich nunmehr die Strafe für dein Tun auf dein Haupt fallen lassen!‹
-- so lautet der Ausspruch Gottes, des HERRN. ›Hast du etwa nicht
Unzucht getrieben zu all deinen Greueln hinzu?‹«

\hypertarget{ee-jerusalems-schuld-ist-gruxf6uxdfer-als-die-schuld-samarias-und-sodoms-und-verlangt-die-huxe4rteste-strafe}{%
\subparagraph{ee) Jerusalems Schuld ist größer als die Schuld Samarias
und Sodoms und verlangt die härteste
Strafe}\label{ee-jerusalems-schuld-ist-gruxf6uxdfer-als-die-schuld-samarias-und-sodoms-und-verlangt-die-huxe4rteste-strafe}}

\bibleverse{44}»›Wahrlich, jeder, der sich der Sprichwörter\textless sup
title=``oder: Spottverse''\textgreater✲ bedient, wird auf dich den
Spruch anwenden: Wie die Mutter, so die Tochter! \bibleverse{45}Du bist
wirklich die Tochter deiner Mutter, die ihres Mannes und ihrer Kinder
überdrüssig geworden ist; und du bist wirklich die Schwester deiner
Schwestern, die ihre Männer und ihre Kinder von sich gestoßen haben.
Eure Mutter ist eine Hethiterin und euer Vater ein Amoriter gewesen.
\bibleverse{46}Deine größere Schwester ist Samaria mit ihren
Tochterstädten, die nördlich von dir wohnt; und deine kleinere
Schwester, die südlich von dir wohnt, ist Sodom mit ihren
Tochterstädten. \bibleverse{47}Aber nicht auf ihren Wegen bist du
gewandelt, und nicht Greuel wie sie hast du verübt; nein, nur ein
kleines Weilchen hat es gedauert, da hast du es ärger als sie getrieben
in all deinem Wandel\textless sup title=``oder: Tun''\textgreater✲.
\bibleverse{48}So wahr ich lebe!‹ -- so lautet der Ausspruch Gottes des
HERRN --, ›deine Schwester Sodom samt ihren Tochterstädten hat nicht so
übel gehandelt, wie du es getan hast samt deinen Tochterstädten.
\bibleverse{49}Siehe, das war die Verschuldung deiner Schwester Sodom:
Hoffart, Brot in Fülle und sorglose Ruhe\textless sup title=``oder:
Wohlleben''\textgreater✲ war ihr samt ihren Tochterstädten eigen; aber
den Armen und Notleidenden reichten sie niemals die Hand zur Hilfe,
\bibleverse{50}sondern sie wurden hochmütig und verübten Greuel vor
meinen Augen; darum schaffte ich sie weg, sobald ich das sah.
\bibleverse{51}Auch Samaria hat nicht halb so viel gesündigt wie du,
sondern du hast so viel mehr Greuel verübt als jene beiden, daß deine
Schwestern dir gegenüber als gerecht erscheinen infolge all deiner
Greuel, die du verübt hast. \bibleverse{52}So trage nun auch du deine
Schande, weil du für deine Schwestern ins Mittel getreten bist: infolge
deiner Sünden, durch die du ärgere Greuel verübt hast als sie, stehen
sie gerechter da als du. So schäme nun auch du dich und trage deine
Schmach dafür, daß du deine Schwestern als gerecht hast erscheinen
lassen.‹«

\hypertarget{ff-jerusalems-gnuxe4dige-wiederannahme-in-gemeinschaft-mit-samaria-und-sodom}{%
\subparagraph{ff) Jerusalems gnädige Wiederannahme in Gemeinschaft mit
Samaria und
Sodom}\label{ff-jerusalems-gnuxe4dige-wiederannahme-in-gemeinschaft-mit-samaria-und-sodom}}

\bibleverse{53}»›Ich werde aber ihr Schicksal wenden, das Schicksal
Sodoms und ihrer Tochterstädte und das Schicksal Samarias und ihrer
Tochterstädte, und dann auch dein eigenes Schicksal in ihrer
Mitte\textless sup title=``=~zugleich mit dem ihren''\textgreater✲,
\bibleverse{54}auf daß du deine Schmach trägst und dich alles dessen
schämst, was du verübt hast, indem du ihnen dadurch Trost verschafftest.
\bibleverse{55}Wenn dann deine Schwester Sodom nebst ihren
Tochterstädten und Samaria nebst ihren Tochterstädten wiederhergestellt
worden sind, wie sie einst gewesen, so sollst auch du mit deinen
Tochterstädten in den früheren Zustand zurückversetzt werden.
\bibleverse{56}War nicht der Name deiner Schwester Sodom als
abschreckendes Beispiel in deinem Munde zur Zeit deines Hochmuts,
\bibleverse{57}ehe deine Blöße\textless sup title=``oder: eigene
Verworfenheit''\textgreater✲ aufgedeckt wurde, wie (dies jetzt der Fall
ist) jetzt, da die Töchter Syriens dich schmähen und alle Töchter der
Philister dich ringsumher verachten? \bibleverse{58}Für deine Unzucht
und deine Greuel mußt du nun die Strafe büßen!‹ -- so lautet der
Ausspruch Gottes des HERRN.«

\hypertarget{gg-trostreicher-ausblick-in-die-zukunft}{%
\subparagraph{gg) Trostreicher Ausblick in die
Zukunft}\label{gg-trostreicher-ausblick-in-die-zukunft}}

\bibleverse{59}»Denn so spricht Gott der HERR: ›Ja, ich werde mit dir
verfahren, wie du verfahren bist: du hast ja den Schwur\textless sup
title=``oder: Fluch''\textgreater✲ mißachtet, indem du den Bund
brachest. \bibleverse{60}Doch ich will meines Bundes gedenken, den ich
mit dir in den Tagen deiner Jugend geschlossen habe, und will einen
ewigen Bund mit dir aufrichten\textless sup title=``oder:
eingehen''\textgreater✲. \bibleverse{61}Da wirst du dann an dein ganzes
Tun zurückdenken und dich seiner schämen, wenn ich deine Schwestern,
sowohl die größeren als auch die kleineren, nehme und sie dir zu
Töchtern gebe, allerdings nicht auf Grund des mit dir geschlossenen
Bundes\textless sup title=``oder: wegen deiner
Bundestreue''\textgreater✲. \bibleverse{62}Sondern ich will meinerseits
einen Bund mit dir schließen, und du sollst erkennen, daß ich der HERR
bin, \bibleverse{63}damit du daran gedenkst und dich schämst und infolge
deiner Schmach\textless sup title=``oder: vor lauter
Beschämung''\textgreater✲ den Mund nicht mehr auftust, wenn ich dir
alles vergeben habe, was du getan hast‹ -- so lautet der Ausspruch
Gottes des HERRN.«

\hypertarget{c-gleichnis-vom-grouxdfen-adler-und-treulosen-weinstock-nebst-anwendung-auf-zedekia-wiederherstellung-des-davidischen-kuxf6nigtums}{%
\paragraph{c) Gleichnis vom großen Adler und treulosen Weinstock nebst
Anwendung auf Zedekia; Wiederherstellung des davidischen
Königtums}\label{c-gleichnis-vom-grouxdfen-adler-und-treulosen-weinstock-nebst-anwendung-auf-zedekia-wiederherstellung-des-davidischen-kuxf6nigtums}}

\hypertarget{section-16}{%
\section{17}\label{section-16}}

\bibleverse{1}Weiter erging das Wort des HERRN an mich folgendermaßen:
\bibleverse{2}»Menschensohn, gib dem Hause Israel ein Rätsel auf und
trage ihm ein Gleichnis vor! \bibleverse{3}Sage zu ihnen: ›So hat Gott
der HERR gesprochen: Der große Adler mit großen Flügeln, mit langen
Schwungfedern und voll schillernden Gefieders kam zum Libanon und nahm
dort den Wipfel der Zeder weg: \bibleverse{4}den obersten ihrer Schosse
riß er ab und brachte ihn ins Krämerland, versetzte ihn in eine
Handelsstadt. \bibleverse{5}Dann nahm er einen von den Schößlingen des
Landes, pflanzte ihn in ein Saatfeld, wo reichlich Wasser war,
behandelte ihn wie ein Weidengewächs, \bibleverse{6}damit er dort wüchse
und sich zu einem üppigen Weinstock von niedrigem Wuchs entwickelte;
seine Ranken sollten sich zu ihm (dem Adler) hinkehren und seine Wurzeln
ihm untertan sein. Und es wurde wirklich ein Weinstock daraus, der
Ranken ansetzte und Zweige trieb. \bibleverse{7}Nun war da aber auch
noch ein anderer großer Adler mit großen Flügeln und starkem Gefieder;
und siehe da: jener Weinstock bog seine Wurzeln zu diesem hin und
streckte seine Ranken zu ihm aus, damit er ihn noch besser tränken
möchte, als das Beet es tat, auf das er gepflanzt war; \bibleverse{8}und
er war doch auf guten Boden, an reichliches Wasser gepflanzt, um Ranken
zu treiben und Früchte zu tragen und sich zu einem herrlichen Weinstock
zu entwickeln.‹ \bibleverse{9}Sage nun zu ihnen: ›So hat Gott der HERR
gesprochen: Wird das gut ablaufen? Wird jener (Adler) ihm nicht die
Wurzeln ausreißen und seine Trauben abschneiden, so daß er verdorrt? Ja,
alle seine frischsprossenden Triebe werden verdorren, und zwar wird kein
gewaltiger Arm und keine große Anzahl von Leuten dazu erforderlich sein,
um ihn aus seinem Wurzelboden herauszuheben. \bibleverse{10}Allerdings
gepflanzt ist er: aber -- wird es gut ablaufen? Wird er nicht, sobald
der Ostwind ihn trifft, gänzlich verdorren? Ja, auf dem Beet, auf dem er
herangewachsen ist, wird er verdorren.‹«

\hypertarget{anwendung-des-ruxe4tsels-bzw.-gleichnisses-auf-das-treulose-verhalten-des-kuxf6nigs-zedekia}{%
\paragraph{Anwendung des Rätsels (bzw. Gleichnisses) auf das treulose
Verhalten des Königs
Zedekia}\label{anwendung-des-ruxe4tsels-bzw.-gleichnisses-auf-das-treulose-verhalten-des-kuxf6nigs-zedekia}}

\bibleverse{11}Alsdann erging das Wort des HERRN an mich folgendermaßen:
\bibleverse{12}»Sage doch dem widerspenstigen Geschlecht: ›Versteht ihr
denn nicht, was dies bedeutet?‹ Sage ihnen: ›Seht, der König von Babylon
ist nach Jerusalem gekommen, hat den dortigen König und die dortigen
Oberen\textless sup title=``oder: Großen''\textgreater✲ mitgenommen und
sie zu sich nach Babylon gebracht. \bibleverse{13}Dann hat er einen
Sprößling des Königshauses genommen und einen Vertrag mit ihm
geschlossen und ihn durch einen Eid verpflichtet; die Vornehmsten des
Landes aber hat er mit sich genommen, \bibleverse{14}damit die
Königsmacht bescheiden bliebe und sich nicht wieder erhöbe und damit
jener den geschlossenen Vertrag hielte, damit er Bestand habe.
\bibleverse{15}Der aber fiel von ihm ab, indem er seine Gesandten nach
Ägypten schickte, damit man ihm Rosse und viel Kriegsvolk gäbe. Wird er
wohl Glück haben? Wird er, der so gehandelt hat, ungestraft davonkommen?
Er hat den Vertrag gebrochen: sollte er der Strafe entgehen?
\bibleverse{16}So wahr ich lebe!‹ -- so lautet der Ausspruch Gottes des
HERRN --, ›an dem Wohnsitz des Königs, der ihn zum Herrscher gemacht,
dem er den Eid nicht gehalten hat und gegen den er vertragsbrüchig
geworden ist, bei diesem soll er sterben mitten in Babylon!
\bibleverse{17}Der Pharao aber wird ihm nicht mit großer Heeresmacht und
zahlreichem Aufgebot im Kriege Beistand leisten, wenn man einen Wall
aufführt und Belagerungstürme baut, um viele Menschen ums Leben zu
bringen. \bibleverse{18}Denn weil er meineidig durch Vertragsbruch
geworden ist und trotz des Handschlags, den er gegeben, alles dies getan
hat, wird er nicht ungestraft davonkommen!‹« \bibleverse{19}Darum hat
Gott der HERR so gesprochen: »So wahr ich lebe! Für den Eid, den er bei
mir geschworen und mißachtet hat, und für den Vertrag, den er vor mir
geschlossen und gebrochen hat, will ich ihn die Strafe büßen lassen
\bibleverse{20}und mein Fangnetz über ihn ausbreiten, so daß er in
meinem Garn gefangen wird! Dann will ich ihn nach Babylon bringen und
dort Abrechnung mit ihm halten wegen des Treubruchs, dessen er sich
gegen mich schuldig gemacht hat. \bibleverse{21}Alle seine auserlesenen
Krieger aber in allen seinen Heerscharen sollen durchs Schwert fallen
und die Übriggebliebenen in alle Winde zerstreut werden, damit ihr
erkennt, daß ich, der HERR, es angesagt habe.«

\hypertarget{verheiuxdfung-eines-gottgesegneten-herrschers-aus-davids-hause}{%
\paragraph{Verheißung eines gottgesegneten Herrschers aus Davids
Hause}\label{verheiuxdfung-eines-gottgesegneten-herrschers-aus-davids-hause}}

\bibleverse{22}So hat Gott der HERR gesprochen: »Ich selbst aber werde
einen Zweig vom Wipfel der hohen Zeder nehmen und ihn einsetzen; von dem
obersten ihrer Sprossen werde ich ein zartes Reis abbrechen, und ich
selbst werde es auf einem hohen und ragenden Berge einpflanzen:
\bibleverse{23}auf der Bergeshöhe\textless sup title=``oder: dem
höchsten Berge''\textgreater✲ Israels will ich es einpflanzen, damit es
dort Zweige treibt und Früchte trägt und sich zu einer herrlichen Zeder
entwickelt, unter welcher allerlei Wild lagert, allerlei Vögel jeglichen
Gefieders wohnen und im Schatten ihrer Zweige nisten.
\bibleverse{24}Dann werden alle Bäume des Feldes erkennen, daß ich, der
HERR, den hohen Baum erniedrigt und den niedrigen Baum erhöht habe, daß
ich den saftreichen Baum habe verdorren lassen und den dürren Baum zur
Blüte gebracht habe. Ich, der HERR, habe es angesagt und es auch
vollführt.«

\hypertarget{die-fuxfcr-die-vergeltende-gerechtigkeit-gottes-guxfcltigen-grundsuxe4tze}{%
\subsubsection{5. Die für die vergeltende Gerechtigkeit Gottes gültigen
Grundsätze}\label{die-fuxfcr-die-vergeltende-gerechtigkeit-gottes-guxfcltigen-grundsuxe4tze}}

\hypertarget{a-einleitung-und-aufstellung-des-mauxdfgebenden-grundsatzes}{%
\paragraph{a) Einleitung und Aufstellung des maßgebenden
Grundsatzes}\label{a-einleitung-und-aufstellung-des-mauxdfgebenden-grundsatzes}}

\hypertarget{section-17}{%
\section{18}\label{section-17}}

\bibleverse{1}Weiter erging das Wort des HERRN an mich folgendermaßen:
\bibleverse{2}»Wie kommt ihr dazu, im Lande Israel diesen Spruch✲ im
Munde zu führen, daß ihr sagt: ›Die Väter haben saure Trauben gegessen,
und den Söhnen werden die Zähne stumpf davon‹? \bibleverse{3}So wahr ich
lebe!« -- so lautet der Ausspruch Gottes des HERRN --: »ihr sollt fortan
diesen Spruch in Israel nicht mehr im Munde führen!
\bibleverse{4}Bedenkt wohl: alle Seelen gehören mir, die Seele des
Vaters so gut wie die des Sohnes -- beide gehören mir: die Seele, die da
sündigt, die soll sterben!«\textless sup title=``vgl. Jer
31,29-30''\textgreater✲

\hypertarget{b-gott-richtet-jeden-auf-grund-seines-persuxf6nlichen-tuns}{%
\paragraph{b) Gott richtet jeden auf Grund seines persönlichen
Tuns}\label{b-gott-richtet-jeden-auf-grund-seines-persuxf6nlichen-tuns}}

\hypertarget{aa-der-gerechte-lebt}{%
\subparagraph{aa) Der Gerechte lebt}\label{aa-der-gerechte-lebt}}

\bibleverse{5}»Wenn also jemand gerecht ist und Recht und Gerechtigkeit
übt, \bibleverse{6}an den Opfermahlen auf den Bergen nicht teilnimmt und
seine Augen nicht zu den Götzen des Hauses Israel erhebt, das Weib
seines Nächsten nicht entehrt und einem Weibe zur Zeit ihrer Unreinheit
nicht naht, \bibleverse{7}niemanden übervorteilt, dem Schuldner sein
Pfand zurückgibt, sich keine Erpressung zuschulden kommen
läßt\textless sup title=``=~kein fremdes Gut an sich
bringt''\textgreater✲, dem Hungrigen von seinem Brot abgibt und den
Nackten mit Kleidung versieht, \bibleverse{8}kein Geld auf Wucher
ausleiht und keine Zinsen nimmt, seine Hand vom Unrecht fernhält, in
Streitsachen zwischen den Parteien der Wahrheit gemäß richtet,
\bibleverse{9}nach meinen Satzungen wandelt und meine Gebote beobachtet,
indem er sie getreulich erfüllt: der ist ein gerechter Mann, er soll
gewißlich am Leben bleiben!« -- so lautet der Ausspruch Gottes des
HERRN.

\hypertarget{bb-der-gottlose-sohn-eines-frommen-vaters-soll-sterben}{%
\subparagraph{bb) Der gottlose Sohn eines frommen Vaters soll
sterben}\label{bb-der-gottlose-sohn-eines-frommen-vaters-soll-sterben}}

\bibleverse{10}»Ist er nun aber Vater eines gewalttätigen Sohnes, der
Blut vergießt und eine von jenen Sünden begeht~-- \bibleverse{11}während
jener dies alles nicht getan hat --, wenn er sogar an den Opfermahlen
auf den Bergen teilnimmt und das Weib seines Nächsten entehrt,
\bibleverse{12}den Armen und Notleidenden übervorteilt, Erpressung
verübt\textless sup title=``vgl. V.7''\textgreater✲, Gepfändetes nicht
zurückgibt und seine Augen zu den Götzen erhebt, Abscheuliches✲ begeht,
\bibleverse{13}Geld auf Wucher ausleiht und Zinsen nimmt: sollte ein
solcher Mensch am Leben bleiben? Nein, er soll nicht am Leben bleiben!
Weil er alle diese Abscheulichkeiten verübt hat, soll er unfehlbar den
Tod erleiden: die Strafe für seine Blutschuld soll ihn treffen!«

\hypertarget{cc-der-fromme-sohn-eines-gottlosen-vaters-soll-leben}{%
\subparagraph{cc) Der fromme Sohn eines gottlosen Vaters soll
leben}\label{cc-der-fromme-sohn-eines-gottlosen-vaters-soll-leben}}

\bibleverse{14}»Wenn der nun aber Vater eines Sohnes ist, der alle
Sünden sieht, die sein Vater begeht, und, obgleich er sie sieht, dennoch
nicht ebenso handelt: \bibleverse{15}er nimmt nicht teil an den
Opfermahlen auf den Bergen und erhebt seine Augen nicht zu den Götzen
des Hauses Israel, er entehrt nicht das Weib seines Nächsten
\bibleverse{16}und übervorteilt niemand, er läßt sich kein Pfand geben
und verübt keine Erpressung, er gibt dem Hungrigen von seinem Brot ab
und versieht den Nackten mit Kleidung, \bibleverse{17}er hält seine Hand
vom Unrecht fern, leiht kein Geld auf Wucher aus und nimmt keine Zinsen,
er beobachtet meine Satzungen und wandelt nach meinen Geboten: der soll
wegen der Verschuldung seines Vaters nicht sterben, sondern sicherlich
am Leben bleiben. \bibleverse{18}Sein Vater aber, der Bedrückung verübt
und Erpressung am Bruder begangen und inmitten seiner Volksgenossen
nicht gut gehandelt hat: fürwahr, der soll wegen seiner Verschuldung
sterben!

\bibleverse{19}Wenn ihr aber fragt: ›Warum soll der Sohn die Schuld
seines Vaters nicht mittragen?‹, so bedenkt wohl: Der Sohn hat doch
Recht und Gerechtigkeit geübt, hat alle meine Satzungen beobachtet und
nach ihnen gehandelt: darum soll er am Leben bleiben! \bibleverse{20}Ein
jeder, der Sünde tut, der soll sterben; aber der Sohn soll die Schuld
seines Vaters nicht mittragen und der Vater nicht die Schuld seines
Sohnes; nein, dem Gerechten soll der Lohn für seine Gerechtigkeit zuteil
werden und ebenso dem Gottlosen die Strafe für seine Gottlosigkeit!«

\hypertarget{c-jeder-wird-nach-seinem-endguxfcltigen-nicht-nach-seinem-fruxfcheren-tun-gerichtet}{%
\paragraph{c) Jeder wird nach seinem endgültigen, nicht nach seinem
früheren Tun
gerichtet}\label{c-jeder-wird-nach-seinem-endguxfcltigen-nicht-nach-seinem-fruxfcheren-tun-gerichtet}}

\bibleverse{21}»Bekehrt sich jedoch der Gottlose von all seinen Sünden,
die er begangen hat, und beobachtet er alle meine Satzungen und übt er
Recht und Gerechtigkeit, so soll er gewißlich am Leben bleiben, soll
nicht sterben! \bibleverse{22}Keine von allen Sünden, die er begangen
hat, soll ihm noch angerechnet werden: um der Gerechtigkeit willen, die
er geübt hat, soll er am Leben bleiben. \bibleverse{23}Habe ich etwa
Wohlgefallen am Tode des Gottlosen?« -- so lautet der Ausspruch Gottes
des HERRN -- »und nicht vielmehr daran, daß er sich von seinem bösen
Wandel bekehrt und am Leben bleibt?

\bibleverse{24}Wenn aber ein Gerechter sich von seiner Gerechtigkeit
abwendet und Böses verübt, alle die Abscheulichkeiten begeht, die der
Gottlose zu verüben pflegt: sollte er da am Leben bleiben? Nein, keine
von all seinen gerechten Taten, die er vollbracht hat, soll ihm
angerechnet werden: um des Treubruchs willen, dessen er sich schuldig
gemacht, und wegen der Sünde, die er begangen hat, ihretwegen soll er
sterben!

\bibleverse{25}Wenn ihr nun sagt: ›Das Verfahren des Herrn ist nicht das
richtige!‹, so hört doch, ihr vom Hause Israel! Sollte wirklich mein
Verfahren nicht das richtige sein? Ist nicht vielmehr euer Verfahren
unrichtig? \bibleverse{26}Wenn der Gerechte sich von seiner
Gerechtigkeit abwendet und Böses verübt, so muß er deswegen sterben:
wegen des Bösen, das er begangen hat, deswegen muß er sterben.
\bibleverse{27}Wenn sich dagegen der Gottlose von der Gottlosigkeit, die
er begangen hat, abwendet und Recht und Gerechtigkeit übt, so wird ein
solcher Mensch seine Seele am Leben erhalten. \bibleverse{28}Wenn er zur
Einsicht kommt und von allen Übertretungen, deren er sich schuldig
gemacht hat, abläßt, so soll er gewißlich das Leben behalten und nicht
sterben! \bibleverse{29}Wenn also das Haus Israel sagt: ›Das Verfahren
des Herrn ist nicht das richtige‹ -- sollte wirklich mein Verfahren
nicht das richtige sein, Haus Israel? Ist nicht vielmehr euer Verfahren
unrichtig?«

\hypertarget{d-ermahnung-an-die-verbannten-sich-zu-bekehren}{%
\paragraph{d) Ermahnung an die Verbannten, sich zu
bekehren}\label{d-ermahnung-an-die-verbannten-sich-zu-bekehren}}

\bibleverse{30}»Darum werde ich einen jeden von euch, ihr vom Hause
Israel, nach seinem Wandel richten« -- so lautet der Ausspruch Gottes
des HERRN. »Kehrt um und wendet euch von all euren Übertretungen ab,
damit sie euch nicht weiter ein Anlaß zur Verschuldung werden!
\bibleverse{31}Werft alle eure Übertretungen, durch die ihr euch gegen
mich vergangen habt, von euch ab und schafft euch ein neues Herz und
einen neuen Geist! Denn warum wollt ihr sterben, Haus Israel?
\bibleverse{32}Ich habe ja kein Wohlgefallen am Tode dessen, der sterben
muß« -- so lautet der Ausspruch Gottes des HERRN --; »darum bekehrt
euch, so werdet ihr leben!«

\hypertarget{klagelied-uxfcber-die-ungluxfcckliche-kuxf6nigin-mutter-hamutal-und-ihre-beiden-suxf6hne-joahas-und-zedekia}{%
\subsubsection{6. Klagelied über die unglückliche Königin-Mutter
(Hamutal) und ihre beiden Söhne (Joahas und
Zedekia)}\label{klagelied-uxfcber-die-ungluxfcckliche-kuxf6nigin-mutter-hamutal-und-ihre-beiden-suxf6hne-joahas-und-zedekia}}

\hypertarget{a-das-bild-von-der-luxf6wenmutter-mit-ihren-beiden-jungen}{%
\paragraph{a) Das Bild von der Löwenmutter mit ihren beiden
Jungen}\label{a-das-bild-von-der-luxf6wenmutter-mit-ihren-beiden-jungen}}

\hypertarget{section-18}{%
\section{19}\label{section-18}}

\bibleverse{1}»Du aber stimme ein Klagelied an über die Fürsten Israels
\bibleverse{2}und sprich: ›Wie war doch deine Mutter eine Löwin unter
Löwen! Sie hatte ihr Lager inmitten von Jungleuen, zog ihre Jungen groß.
\bibleverse{3}Eins von ihren Jungen brachte sie hoch, zum Jungleu wurde
es; der lernte Raub erbeuten, Menschen fraß er. \bibleverse{4}Da
erließen die Völker einen Aufruf gegen ihn: in ihrer Grube wurde er
gefangen, und sie brachten ihn mit Ringen\textless sup title=``oder:
Haken an seinen Kinnbacken''\textgreater✲ nach dem Lande Ägypten.~--
\bibleverse{5}Als nun seine Mutter sah, daß sie getäuscht, ihre Hoffnung
vernichtet war, da nahm sie ein anderes von ihren Jungen und machte es
zu einem Jungleu. \bibleverse{6}Der schritt stolz unter den Löwen
einher, wurde ein Jungleu; er lernte Raub erbeuten, Menschen fraß er.
\bibleverse{7}Er machte ihre Witwen zahlreich und entvölkerte ihre
Städte, so daß das Land und alles, was darin war, sich vor seinem
dröhnenden Gebrüll entsetzte. \bibleverse{8}Da stellten sich die Völker
ringsum aus den Landschaften gegen ihn auf und breiteten ihr Netz über
ihn aus: in ihrer Grube wurde er gefangen. \bibleverse{9}Dann taten sie
ihn an Nasenringen in einen Käfig und brachten ihn zum König von
Babylon, brachten ihn in eine der Burgen, damit man sein Gebrüll auf den
Bergen Israels nicht mehr höre.‹«

\hypertarget{b-das-bild-von-der-prangenden-und-dann-vernichteten-weinrebe}{%
\paragraph{b) Das Bild von der prangenden und dann vernichteten
Weinrebe}\label{b-das-bild-von-der-prangenden-und-dann-vernichteten-weinrebe}}

\bibleverse{10}»›Deine Mutter war wie ein Weinstock, im Weingarten am
Wasser gepflanzt, reich an Früchten und voller Ranken infolge des
reichlichen Wassers; \bibleverse{11}an ihm wuchs ein starker Schoß zum
Herrscherstabe, und hoch ragte sein Wuchs empor zwischen dem dichten
Laubwerk, und er war weithin sichtbar durch seine Höhe, durch die Fülle
seiner Ranken. \bibleverse{12}Da wurde er\textless sup title=``d.h. der
Weinstock''\textgreater✲ im Grimm ausgerissen, auf die Erde geworfen,
und der Ostwind dörrte seine Ranken aus; sein starker Schoß wurde
abgerissen und verdorrte, Feuer hat ihn verzehrt. \bibleverse{13}Jetzt
ist er\textless sup title=``d.h. der Weinstock''\textgreater✲ in die
Wüste verpflanzt, in dürres, lechzendes Land; \bibleverse{14}und Feuer
ist von seinem Schoß ausgegangen, hat seine Ranken verzehrt; und es ist
an ihm kein starker Schoß mehr geblieben, kein Stab zum Herrschen.‹«

Ein Klagelied ist dies, und es ist zum Klagelied geworden.

\hypertarget{v.-dritte-reihe-von-drohweissagungen-und-strafreden-gegen-jerusalem-und-juda-kap.-20-24}{%
\subsection{V. Dritte Reihe von Drohweissagungen und Strafreden gegen
Jerusalem und Juda (Kap.
20-24)}\label{v.-dritte-reihe-von-drohweissagungen-und-strafreden-gegen-jerusalem-und-juda-kap.-20-24}}

\hypertarget{israels-fruxfchere-und-jetzige-verfehlungen-dereinstige-luxe4uterung-und-begnadigung}{%
\subsubsection{1. Israels (frühere und jetzige) Verfehlungen,
dereinstige Läuterung und
Begnadigung}\label{israels-fruxfchere-und-jetzige-verfehlungen-dereinstige-luxe4uterung-und-begnadigung}}

\hypertarget{a-hesekiel-huxe4lt-den-uxe4ltesten-die-suxfcnden-der-vuxe4ter-und-suxf6hne-vor}{%
\paragraph{a) Hesekiel hält den Ältesten die Sünden der Väter und Söhne
vor}\label{a-hesekiel-huxe4lt-den-uxe4ltesten-die-suxfcnden-der-vuxe4ter-und-suxf6hne-vor}}

\hypertarget{section-19}{%
\section{20}\label{section-19}}

\bibleverse{1}Im siebten Jahre, am zehnten Tage des fünften Monats,
kamen einige von den Ältesten\textless sup title=``oder:
Vornehmsten''\textgreater✲ Israels, um den HERRN zu befragen, und sie
ließen sich vor mir nieder. \bibleverse{2}Da erging das Wort des HERRN
an mich folgendermaßen: \bibleverse{3}»Menschensohn, sprich zu den
Ältesten Israels und sage zu ihnen: ›So hat Gott der HERR gesprochen:
Mich zu befragen seid ihr gekommen? So wahr ich lebe: ich lasse mich von
euch nicht befragen!‹ -- so lautet der Ausspruch Gottes des HERRN.
\bibleverse{4}Willst du ihnen nicht vielmehr das Urteil sprechen? Willst
du mit ihnen nicht ins Gericht gehen, Menschensohn? Halte ihnen die
Greueltaten ihrer Väter vor!«

\hypertarget{b-uxfcberblick-uxfcber-israels-suxfcndige-vergangenheit-und-gegenwart}{%
\paragraph{b) Überblick über Israels sündige Vergangenheit und
Gegenwart}\label{b-uxfcberblick-uxfcber-israels-suxfcndige-vergangenheit-und-gegenwart}}

\hypertarget{aa-guxf6tzendienst-in-uxe4gypten}{%
\subparagraph{aa) Götzendienst in
Ägypten}\label{aa-guxf6tzendienst-in-uxe4gypten}}

\bibleverse{5}»Sage zu ihnen: ›So hat Gott der HERR gesprochen: An dem
Tage, als ich Israel erwählte, erhob ich meine Hand für die Angehörigen
des Hauses Jakob zum Schwur und gab mich ihnen im Lande Ägypten zu
erkennen; ich erhob meine Hand für sie zum Schwur und sprach: Ich bin
der HERR, euer Gott! \bibleverse{6}An demselben Tage erhob ich meine
Hand und schwur ihnen, daß ich sie aus dem Lande Ägypten in ein Land
führen würde, das ich für sie ausersehen hätte, das von Milch und Honig
überfließe und die Krone\textless sup title=``oder: das Kleinod =~das
herrlichste''\textgreater✲ unter allen Ländern sei. \bibleverse{7}Dabei
gebot ich ihnen: Ein jeder von euch werfe die Scheusale weg, auf die er
bisher seine Augen gerichtet hat, und verunreinigt euch nicht an den
ägyptischen Götzen! Ich, der HERR, bin euer Gott! \bibleverse{8}Aber sie
waren widerspenstig gegen mich und wollten nicht auf mich hören; keiner
von ihnen warf die Scheusale weg, auf die er bisher seine Augen
gerichtet hatte, und die ägyptischen Götzen ließen sie nicht fahren. Da
gedachte ich meinen Grimm über sie auszugießen, meinen ganzen Zorn
mitten im Lande Ägypten an ihnen auszulassen; \bibleverse{9}aber ich
nahm Rücksicht auf meinen Namen, damit dieser nicht entehrt würde vor
den Augen der Heidenvölker, unter denen sie wohnten und vor deren Augen
ich mich ihnen geoffenbart hatte, um sie aus dem Lande Ägypten
wegzuführen.‹«

\hypertarget{bb-ungehorsam-nach-der-gesetzgebung-bei-der-wuxfcstenwanderung}{%
\subparagraph{bb) Ungehorsam nach der Gesetzgebung bei der
Wüstenwanderung}\label{bb-ungehorsam-nach-der-gesetzgebung-bei-der-wuxfcstenwanderung}}

\bibleverse{10}»›So führte ich sie denn aus dem Lande Ägypten weg und
brachte sie in die Wüste. \bibleverse{11}Da gab ich ihnen meine
Satzungen und lehrte sie meine Gebote kennen, durch deren Beobachtung
der Mensch das Leben hat\textless sup title=``oder: sein Leben
erhält''\textgreater✲. \bibleverse{12}Auch meine Sabbate gab ich ihnen,
die ein Zeichen (des Bundes) zwischen mir und ihnen sein sollten, damit
sie zur Erkenntnis kämen, daß ich, der HERR, es bin, der sie
heiligt\textless sup title=``2.Mose 31,13''\textgreater✲.
\bibleverse{13}Aber das Haus Israel war auch in der Wüste widerspenstig
gegen mich: sie wandelten nicht nach meinen Satzungen und mißachteten
meine Gebote, durch deren Beobachtung doch der Mensch das Leben
hat\textless sup title=``oder: sein Leben erhält''\textgreater✲; auch
meine Sabbate hielten sie durchaus nicht heilig. Da gedachte ich meinen
Grimm in der Wüste über sie auszugießen, um sie ganz auszurotten;
\bibleverse{14}aber ich nahm Rücksicht auf meinen Namen, damit er nicht
entehrt würde vor den Augen der Heidenvölker, vor deren Augen ich sie
weggeführt hatte. \bibleverse{15}Doch erhob ich meine Hand in der Wüste
und schwur ihnen, daß ich sie nicht in das verheißene Land bringen
würde, ein Land, das von Milch und Honig überfließe und die
Krone\textless sup title=``vgl. 20,6''\textgreater✲ unter allen Ländern
sei, \bibleverse{16}weil sie meine Gebote mißachtet und nicht nach
meinen Satzungen gewandelt und meine Sabbate entheiligt hätten; denn ihr
Herz war immer nur hinter ihren Götzen her. \bibleverse{17}Dennoch
blickte mein Auge mitleidsvoll auf sie, so daß ich sie nicht vernichtete
und sie in der Wüste nicht völlig ausrottete.‹«

\hypertarget{cc-ungehorsam-des-zweiten-geschlechts-in-der-wuxfcste}{%
\subparagraph{cc) Ungehorsam des zweiten Geschlechts in der
Wüste}\label{cc-ungehorsam-des-zweiten-geschlechts-in-der-wuxfcste}}

\bibleverse{18}»›Da gebot ich ihren Söhnen in der Wüste: Wandelt nicht
nach den Gewohnheiten eurer Väter und beobachtet nicht die von ihnen
geübten Bräuche und verunreinigt euch nicht an ihren Götzen!
\bibleverse{19}Ich, der HERR, bin euer Gott: wandelt nach meinen
Satzungen, beobachtet meine Gebote und handelt nach ihnen!
\bibleverse{20}Haltet auch meine Sabbate heilig, damit sie ein Zeichen
(des Bundes) zwischen mir und euch seien und damit man erkenne, daß ich
der HERR, euer Gott, bin. \bibleverse{21}Aber auch die Söhne waren
widerspenstig gegen mich: sie wandelten nicht nach meinen Satzungen und
befolgten meine Gebote nicht, daß sie nach ihnen gehandelt hätten,
obgleich doch der Mensch durch ihre Beobachtung das Leben hat; auch
meine Sabbate entheiligten sie. Da gedachte ich meinen Grimm über sie
auszugießen und meinen Zorn in der Wüste an ihnen völlig auszuwirken;
\bibleverse{22}aber ich zog meine Hand wieder zurück und nahm Rücksicht
auf meinen Namen, damit er nicht entehrt würde vor den Augen der
Heidenvölker, vor deren Augen ich sie ausgeführt hatte.
\bibleverse{23}Doch erhob ich meine Hand in der Wüste und schwur ihnen,
daß ich sie unter die Heidenvölker zerstreuen und sie in die Länder
versprengen würde, \bibleverse{24}weil sie nicht nach meinen Geboten
lebten, meine Satzungen mißachteten, meine Sabbate nicht heilig hielten
und weil ihre Augen auf die Götzen ihrer Väter gerichtet seien.
\bibleverse{25}So gab denn auch ich ihnen Satzungen, die nicht zum Guten
waren, und Gebote, durch die sie nicht das Leben haben konnten;
\bibleverse{26}ich machte sie durch ihre Opfergaben unrein, dadurch daß
sie jegliche Erstgeburt als Opfer verbrannten: ich wollte ihnen eben
Entsetzen\textless sup title=``oder: Grausen''\textgreater✲ einflößen,
damit sie erkennen möchten, daß ich der HERR bin.‹«

\hypertarget{dd-huxf6hendienst-im-lande-kanaan}{%
\subparagraph{dd) Höhendienst im Lande
Kanaan}\label{dd-huxf6hendienst-im-lande-kanaan}}

\bibleverse{27}»Darum, Menschensohn, rede zum Hause Israel und sage zu
ihnen: ›So hat Gott der HERR gesprochen: Auch noch dadurch haben eure
Väter mich beschimpft, daß sie mir die Treue gebrochen haben.
\bibleverse{28}Nachdem ich sie nämlich in das Land gebracht hatte,
dessen Verleihung ich ihnen zugeschworen hatte, da schlachteten sie, wo
immer sie einen hohen Hügel und einen dichtbelaubten Baum erblickt
hatten, daselbst ihre Opfertiere, brachten dort ihre Gaben zu meiner
Kränkung dar, ließen daselbst ihre lieblichen Weihrauchdüfte aufsteigen
und gossen dort ihre Trankopfer aus. \bibleverse{29}{[}Da fragte ich
sie: Was ist das für eine Höhe, auf die ihr euch da begebt? Daher ist
der Name ›Höhe‹ geblieben bis auf den heutigen Tag.{]}‹«

\hypertarget{ee-das-jetzige-den-vuxe4tern-gleiche-geschlecht-wird-von-gott-verworfen}{%
\subparagraph{ee) Das jetzige den Vätern gleiche Geschlecht wird von
Gott
verworfen}\label{ee-das-jetzige-den-vuxe4tern-gleiche-geschlecht-wird-von-gott-verworfen}}

\bibleverse{30}»Darum sage zum Hause Israel: ›So hat Gott der HERR
gesprochen: Wie? Nach der Weise eurer Väter wollt auch ihr euch
verunreinigen und mit ihren Scheusalen ebenfalls Buhlerei treiben?
\bibleverse{31}Ja, dadurch, daß ihr eure Gaben darbringt, daß ihr eure
Kinder als Opfer verbrennt, verunreinigt ihr euch an all euren Götzen
bis auf den heutigen Tag; und da sollte ich mich von euch befragen
lassen, Haus Israel? So wahr ich lebe!‹ -- so lautet der Ausspruch
Gottes des HERRN --: ›ich will mich von euch nicht befragen lassen!‹«

\hypertarget{c-gottes-werk-in-der-zukunft}{%
\paragraph{c) Gottes Werk in der
Zukunft}\label{c-gottes-werk-in-der-zukunft}}

\hypertarget{aa-die-einstige-luxe4uterung-der-verbannten}{%
\subparagraph{aa) Die einstige Läuterung der
Verbannten}\label{aa-die-einstige-luxe4uterung-der-verbannten}}

\bibleverse{32}»›Und das, was euch in den Sinn gekommen ist, darf
nimmermehr zur Ausführung gelangen, daß ihr sagt: Wir wollen es machen
wie die Heiden, wie die Völker in den anderen Ländern, indem wir Holz
und Stein anbeten! \bibleverse{33}So wahr ich lebe!‹ -- so lautet der
Ausspruch Gottes des HERRN --: ›Mit starker Hand und hocherhobenem Arm
und so, daß ich meinem Ingrimm freien Lauf lasse, will ich mich als
König über euch erweisen! \bibleverse{34}Ich will euch aus den
Heidenvölkern herausführen und euch aus den Ländern sammeln, in die ihr
zerstreut worden seid, mit starker Hand und hocherhobenem Arm und so,
daß ich meinem Ingrimm freien Lauf lasse, \bibleverse{35}und will euch
in die Wüste inmitten der Völker bringen und dort ins Gericht mit euch
gehen von Angesicht zu Angesicht! \bibleverse{36}Wie ich einst in der
Wüste des Landes Ägypten mit euren Vätern ins Gericht gegangen bin,
ebenso will ich auch über euch Gericht halten!‹ -- so lautet der
Ausspruch Gottes des HERRN. \bibleverse{37}›Da will ich euch unter
meinem Stabe an mir vorübergehen lassen und euch zur Erfüllung der
Bundespflichten zwingen, \bibleverse{38}und ich will die Ungehorsamen
und die von mir Abgefallenen aus euch aussondern: aus dem Lande, in dem
sie als Fremdlinge gelebt haben, will ich sie herausführen; aber auf
Israels Boden soll keiner von ihnen zurückkehren, damit ihr erkennt, daß
ich der HERR bin!‹«

\hypertarget{bb-der-wohlgefuxe4llige-gottesdienst-der-bekehrten-im-lande-israel}{%
\subparagraph{bb) Der wohlgefällige Gottesdienst der Bekehrten im Lande
Israel}\label{bb-der-wohlgefuxe4llige-gottesdienst-der-bekehrten-im-lande-israel}}

\bibleverse{39}»Was euch aber betrifft, ihr vom Hause Israel, so hat
Gott der HERR folgendermaßen gesprochen: ›Geht nur hin und dient ein
jeder seinen Götzen! Später aber werdet ihr sicherlich auf mich hören
und meinen heiligen Namen nicht länger durch eure Opfergaben und durch
eure Götzen entweihen, \bibleverse{40}sondern auf meinem heiligen Berge,
auf Israels Bergeshöhe\textless sup title=``=~dem höchsten Berge
Israels''\textgreater✲‹ -- so lautet der Ausspruch Gottes des HERRN --,
›dort wird mir das ganze Haus Israel dienen, alle, die sich im Lande
befinden; dort will ich euch gnädig annehmen und dort eure Hebeopfer und
eure Erstlingsgaben mir darbringen lassen, alles, was ihr an Weihegaben
opfert. \bibleverse{41}Beim lieblichen Opferduft\textless sup
title=``oder: als einen lieblichen Opferduft''\textgreater✲ will ich
euch gnädig annehmen, wenn ich euch aus den Heidenvölkern herausführe
und euch aus den Ländern sammle, in die ihr zerstreut worden seid, und
ich will mich an euch vor den Augen der Heidenvölker als den Heiligen
erweisen. \bibleverse{42}Da werdet ihr dann erkennen, daß ich der HERR
bin, wenn ich euch in das Land Israel zurückführe, in das Land, dessen
Verleihung ich euren Vätern einst durch einen Schwur zugesagt habe.
\bibleverse{43}Dort werdet ihr dann an euren Wandel und all eure Taten
zurückdenken, durch die ihr euch verunreinigt habt, und werdet einen
Abscheu vor euch selbst empfinden wegen all des Bösen, das ihr begangen
habt. \bibleverse{44}Dann werdet ihr auch erkennen, daß ich der HERR
bin, wenn ich so mit euch verfahre um meines Namens willen und nicht
nach✲ eurem bösen Wandel und nach euren verwerflichen Taten, Haus
Israel!‹ -- so lautet der Ausspruch Gottes des HERRN.«

\hypertarget{das-nahende-verhuxe4ngnis}{%
\subsubsection{2. Das nahende
Verhängnis}\label{das-nahende-verhuxe4ngnis}}

\hypertarget{a-das-gleichnis-vom-verheerenden-waldbrand-oder-der-kriegslohe}{%
\paragraph{a) Das Gleichnis vom verheerenden Waldbrand (oder der
Kriegslohe)}\label{a-das-gleichnis-vom-verheerenden-waldbrand-oder-der-kriegslohe}}

\hypertarget{section-20}{%
\section{21}\label{section-20}}

\bibleverse{1}Weiter erging das Wort des HERRN an mich folgendermaßen:
\bibleverse{2}»Menschensohn, richte deine Blicke nach Süden zu, predige
gegen Mittag hin und weissage gegen den Wald, der im Gefilde des
Südlandes liegt, \bibleverse{3}und sprich zu dem Walde im Südland: ›Höre
das Wort des HERRN! So hat Gott der HERR gesprochen: Siehe, ich will ein
Feuer in dir anzünden, das soll alle saftreichen und alle dürren Bäume
in dir verzehren; die lodernde Flamme soll nicht erlöschen, und alle
Gesichter vom Südland bis zum Norden sollen durch sie versengt werden!
\bibleverse{4}Dann wird alles Fleisch\textless sup title=``=~die gesamte
Bevölkerung''\textgreater✲ einsehen, daß ich, der HERR, sie
angezündet\textless sup title=``oder: entfacht''\textgreater✲ habe,
indem sie nicht erlischt.‹« \bibleverse{5}Da entgegnete ich: »Ach, HERR,
mein Gott! Die Leute sagen von mir: ›Trägt der nicht immer Rätselreden
vor?‹«

\hypertarget{b-das-muxf6rderische-racheschwert-gottes-gegen-jerusalem-und-die-ammoniter}{%
\paragraph{b) Das mörderische Racheschwert Gottes gegen Jerusalem und
die
Ammoniter}\label{b-das-muxf6rderische-racheschwert-gottes-gegen-jerusalem-und-die-ammoniter}}

\bibleverse{6}Da erging das Wort des HERRN an mich folgendermaßen:
\bibleverse{7}»Menschensohn, richte deine Blicke auf Jerusalem, predige
gegen ihr Heiligtum und weissage gegen das Land Israel!
\bibleverse{8}Sage zum Lande Israel: ›So hat der HERR gesprochen: Siehe,
ich will an dich\textless sup title=``=~ich will gegen dich
vorgehen''\textgreater✲! Ich will mein Schwert aus der Scheide ziehen
und Gerechte wie Gottlose in dir ausrotten! \bibleverse{9}Weil ich
beide, Gerechte wie Gottlose, in dir ausrotten will, darum soll mein
Schwert aus der Scheide fahren gegen alles Fleisch\textless sup
title=``d.h. die gesamte Bevölkerung''\textgreater✲ vom Südland bis zum
Norden. \bibleverse{10}Dann wird alle Welt erkennen, daß ich, der HERR,
mein Schwert aus der Scheide habe herausfahren lassen, und es nicht
wieder dahin zurückkehrt.‹«

\hypertarget{schmerzbezeugung-und-unheilverkuxfcndigung-des-propheten}{%
\paragraph{Schmerzbezeugung und Unheilverkündigung des
Propheten}\label{schmerzbezeugung-und-unheilverkuxfcndigung-des-propheten}}

\bibleverse{11}»Du aber, Menschensohn, seufze! Mit zusammenbrechenden
Hüften und in bitterem Schmerz seufze vor ihren Augen!
\bibleverse{12}Wenn sie dich dann fragen, worüber du seufzest, so
antworte ihnen: ݆ber eine Schreckenskunde! Bei ihrem Eintreffen werden
alle Herzen verzagen und alle Arme schlaff herabhängen; aller Mut wird
schwinden, und alle Knie werden wie Wasser zerfließen.‹✲ Fürwahr, es
kommt und geht in Erfüllung!« -- so lautet der Ausspruch Gottes des
HERRN.

\hypertarget{das-schwertlied}{%
\paragraph{Das Schwertlied}\label{das-schwertlied}}

\bibleverse{13}Hierauf erging das Wort des HERRN an mich folgendermaßen:
\bibleverse{14}»Menschensohn, verkünde folgende Weissagung: So hat der
HERR gesprochen: Sprich zu ihnen: ›Ein Schwert, ein Schwert ist
geschärft und auch gefegt\textless sup title=``d.h. blank
geglättet''\textgreater✲; \bibleverse{15}um ein Schlachten✲ anzurichten,
dazu ist es geschärft: damit es blitzt und blinkt, dazu ist es gefegt.
Oder sollen wir uns freuen? Die für meinen Sohn bestimmte Rute verachtet
alles Holz. \bibleverse{16}Er hat es zum Fegen hingegeben, um es in die
Hand zu nehmen; es ist geschärft worden, das Schwert, und gefegt, damit
man es dem Würger\textless sup title=``oder: Schlächter''\textgreater✲
in die Hand gebe.‹ \bibleverse{17}Schreie und wehklage, Menschensohn!
Denn dieses Schwert richtet sich gegen mein Volk, richtet sich gegen
alle Fürsten Israels: dem Schwert verfallen sind sie samt meinem Volk!
Darum schlage dich auf die Hüften, \bibleverse{18}denn die Probe ist
gemacht! Aber wie, wenn die Rute selber (das Dreinhauen) verschmäht? Es
soll nicht geschehen!« -- lautet der Ausspruch Gottes des HERRN.
\bibleverse{19}»Du aber, Menschensohn, weissage und schlage die Hände
zusammen! Denn das Schwert wird zweifach, ja dreifach kommen. Ein
Mordschwert ist es, ein großes Mordschwert, das sie umkreisen soll,
\bibleverse{20}damit ihre Herzen verzagen und viele tot hinstürzen an
all ihren Toren. Ich lasse blinken das Schwert. Wehe! Zum Blitzen ist es
gefertigt, zum Schlachten geschärft! \bibleverse{21}Setze in Schrecken,
fahre nach rechts, kehre dich nach links, wohin eben deine Schneide
gerichtet ist! \bibleverse{22}So will denn auch ich meine Hände
zusammenschlagen und meinem Zorn freien Lauf lassen: ich, der HERR, habe
es gesprochen!«

\hypertarget{c-der-kuxf6nig-nebukadnezar-am-scheidewege}{%
\paragraph{c) Der König Nebukadnezar am
Scheidewege}\label{c-der-kuxf6nig-nebukadnezar-am-scheidewege}}

\bibleverse{23}Darauf erging das Wort des HERRN an mich folgendermaßen:
\bibleverse{24}»Du aber, Menschensohn, stelle dir zwei Wege dar, auf
denen das Schwert des Königs von Babylon kommen soll\textless sup
title=``oder: kann''\textgreater✲: von einem Lande sollen sie beide
ausgehen. Dann stelle einen Wegweiser auf am Anfang des Weges nach der
einen wie nach der anderen Stadt, \bibleverse{25}damit das Schwert
sowohl über Rabbath im Ammoniterlande als auch über Juda und Jerusalem
in dessen Mitte kommen kann. \bibleverse{26}Denn der König von Babylon
steht am Scheidewege, am Anfang✲ der beiden Wege, um eine Offenbarung zu
erhalten: er schüttelt die Pfeile, befragt die Hausgötter\textless sup
title=``oder: seinen Hausgott; vgl. 1.Mose 31,19''\textgreater✲ und
beschaut die Leber. \bibleverse{27}In seiner Rechten ist das Los
›Jerusalem‹, daß er den Mund auftue zum Kriegsgeschrei, lauten
Schlachtruf erschallen lasse, Sturmböcke gegen die Tore aufstelle, einen
Wall aufschütte und Belagerungstürme baue. \bibleverse{28}Freilich
scheint das ihnen\textless sup title=``den Israeliten?''\textgreater✲
eine trügerische Wahrsagung zu sein, da sie ja die feierlichsten Eide
geschworen haben; doch er✲ bringt ihre Verschuldung (bei sich) in
Erinnerung, damit sie gefangen\textless sup title=``=~ergriffen oder:
gefaßt''\textgreater✲ werden.«

\hypertarget{gottes-drohung-gegen-den-hauptschuldigen-zedekia-und-gegen-die-stadt-der-messianische-kuxf6nig}{%
\paragraph{Gottes Drohung gegen den Hauptschuldigen, Zedekia, und gegen
die Stadt; der messianische
König}\label{gottes-drohung-gegen-den-hauptschuldigen-zedekia-und-gegen-die-stadt-der-messianische-kuxf6nig}}

\bibleverse{29}Darum hat Gott der HERR so gesprochen: »Weil ihr mich an
eure Verschuldung erinnert habt dadurch, daß eure Übertretungen
aufgedeckt worden sind, so daß eure Sünden in all eurem Tun klar zu Tage
liegen -- ja, weil ihr euch so in Erinnerung gebracht habt, sollt ihr um
ihretwillen ergriffen\textless sup title=``oder: gefaßt''\textgreater✲
werden. \bibleverse{30}Du aber, verruchter Frevler, Fürst Israels,
dessen Tag gekommen ist zur Zeit, wo seine Schuld endgültig gebüßt
wird~-- \bibleverse{31}so hat Gott der HERR gesprochen: ›Hinweg mit der
Königsbinde! Herunter mit der Krone! Das bleibt nicht so, wie es jetzt
ist! Das Niedrige soll erhöht werden, und das Hohe muß herunter!
\bibleverse{32}Zu Trümmern, Trümmern, Trümmern will ich (alles) machen!
Wehe ihm! So soll es bleiben, bis der kommt, der das Anrecht darauf hat:
dem will ich es übergeben.‹«

\hypertarget{d-das-racheschwert-trifft-die-ammoniter}{%
\paragraph{d) Das Racheschwert trifft die
Ammoniter}\label{d-das-racheschwert-trifft-die-ammoniter}}

\bibleverse{33}»Du aber, Menschensohn, verkünde folgende Weissagung: So
hat Gott der HERR in betreff der Ammoniter und in betreff ihrer
Hohnreden gesprochen! Verkünde: ›Ein Schwert, ein Schwert ist gezückt
zum Schlachten✲, ist gefegt zum Blinken, damit es blitze~--
\bibleverse{34}während man dich durch nichtige Gesichte getäuscht und
dir Lügen gewahrsagt hat --, um es\textless sup title=``d.h. das
Schwert''\textgreater✲ verruchten Frevlern an den Hals zu setzen, deren
Tag kommt\textless sup title=``oder: gekommen ist''\textgreater✲ zu der
Zeit, wo ihre Schuld endgültig gebüßt wird. \bibleverse{35}Stecke das
Schwert wieder in seine Scheide! An dem Orte, wo du geschaffen bist, im
Lande, aus dem du stammst, will ich dich richten \bibleverse{36}und will
meinen Zorn sich über dich ergießen lassen, das Feuer meines Ingrimms
gegen dich anfachen und dich der Gewalt tierischer\textless sup
title=``oder: gefühlloser''\textgreater✲ Menschen preisgeben, die
Verderben (für dich) schmieden. \bibleverse{37}Dem Feuer sollst du zum
Fraß dienen, dein Blut soll inmitten deines Landes
vergossen\textless sup title=``oder: tief von der Erde
bedeckt''\textgreater✲ liegen, deiner soll nicht mehr gedacht werden;
denn ich, der HERR, habe gesprochen!‹«

\hypertarget{jerusalems-schuld-und-strafe}{%
\subsubsection{3. Jerusalems Schuld und
Strafe}\label{jerusalems-schuld-und-strafe}}

\hypertarget{a-die-anklage-und-die-hauptklagepunkte-blutvergieuxdfen-und-guxf6tzendienst}{%
\paragraph{a) Die Anklage und die Hauptklagepunkte (Blutvergießen und
Götzendienst)}\label{a-die-anklage-und-die-hauptklagepunkte-blutvergieuxdfen-und-guxf6tzendienst}}

\hypertarget{section-21}{%
\section{22}\label{section-21}}

\bibleverse{1}Weiter erging das Wort des HERRN an mich folgendermaßen:
\bibleverse{2}»Du, Menschensohn, willst du nicht der blutbefleckten
Stadt das Urteil sprechen? Willst du sie nicht richten? Halte ihr alle
ihre Greuel vor \bibleverse{3}mit den Worten: ›So hat Gott der HERR
gesprochen: Wehe der Stadt, die Blut in ihrer Mitte vergossen hat, damit
ihre Zeit herbeikomme\textless sup title=``oder: um so schneller
komme''\textgreater✲, und die sich zu ihrem Unheil Götzen angefertigt
hat, um sich zu verunreinigen! \bibleverse{4}Durch dein Blut, das du
vergossen, hast du dich mit Schuld beladen, und durch deine Götzen, die
du dir angefertigt hast, bist du unrein geworden; du hast die Tage
deines Gerichts nahe herangebracht und bist zum Abschluß deiner Jahre
gekommen. Darum mache ich dich zum Hohn für die Völker und zum Spott für
alle Länder. \bibleverse{5}Mögen sie in deiner Nähe oder fern von dir
wohnen, sie werden dich verspotten, weil dein Ruf befleckt ist und
überall Verwirrung\textless sup title=``oder: Unzucht''\textgreater✲ in
dir herrscht.‹«

\hypertarget{b-nuxe4here-darlegung-der-suxfcnden}{%
\paragraph{b) Nähere Darlegung der
Sünden}\label{b-nuxe4here-darlegung-der-suxfcnden}}

\bibleverse{6}»›Siehe, die Fürsten Israels in deiner Mitte sind alle,
soviel ein jeder mit seiner Faust vermochte, beflissen gewesen, Blut zu
vergießen. \bibleverse{7}Vater und Mutter verachtet man in dir, den
Fremdling behandelt man gewalttätig in deiner Mitte, Waisen und Witwen
bedrückt man in dir. \bibleverse{8}Was mir heilig ist, mißachtest du,
und meine Sabbate entweihst du. \bibleverse{9}Verleumder weilen in dir,
die auf Blutvergießen ausgehen, und auf den Bergen hält man bei dir
Opfermahle, Unzucht\textless sup title=``oder:
Schandtaten''\textgreater✲ verübt man in deiner Mitte.
\bibleverse{10}Man treibt Unzucht in dir mit dem Weibe des
Vaters\textless sup title=``d.h. mit der Stiefmutter''\textgreater✲ und
mißbraucht in dir die vom Blutgang unreinen Frauen. \bibleverse{11}Ein
jeder treibt Ehebruch mit der Frau seines Nächsten; ein anderer lebt in
Blutschande mit seiner Schwiegertochter, der andere schändet in dir
seine Schwester\textless sup title=``d.h. die
Stiefschwester''\textgreater✲, die Tochter seines Vaters.
\bibleverse{12}Bestechungsgeschenke nimmt man in dir an, um Blut zu
vergießen; du treibst Wucher und läßt dir Zinsen zahlen und
übervorteilst deinen Nächsten durch Erpressung; mich aber hast du
vergessen!‹ -- so lautet der Ausspruch Gottes des HERRN.
\bibleverse{13}›Aber wisse wohl: ich schlage meine Hände zusammen über
den unredlichen Gewinn, den du gemacht hast, und über deine Bluttaten,
die in deiner Mitte begangen sind. \bibleverse{14}Wird wohl dein Herz✲
standhalten, oder werden deine Hände stark bleiben zu der Zeit, wo ich
mit dir ins Gericht gehen werde? Ich, der HERR, habe es angesagt und
werde es auch vollführen! \bibleverse{15}Denn ich werde dich unter die
Völker zerstreuen und dich in die Länder versprengen und deine
Unreinheit gänzlich aus dir wegschaffen, \bibleverse{16}damit du durch
eigene Schuld entehrt vor den Augen der Heidenvölker dastehst; dann
wirst du zu der Erkenntnis kommen, daß ich der HERR bin.‹«

\hypertarget{c-die-schmelzung-der-schlacken-im-schmelzofen-des-belagerten-jerusalem}{%
\paragraph{c) Die Schmelzung der Schlacken im Schmelzofen des belagerten
Jerusalem}\label{c-die-schmelzung-der-schlacken-im-schmelzofen-des-belagerten-jerusalem}}

\bibleverse{17}Weiter erging das Wort des HERRN an mich folgendermaßen:
\bibleverse{18}»Menschensohn, die vom Hause Israel sind für mich zu
Schlacken geworden; sie sind alle wie Kupfer und Zinn, Eisen und Blei:
Silberschlacken sind sie geworden!« \bibleverse{19}Darum hat Gott der
HERR so gesprochen: »Weil ihr alle zu Schlacken geworden seid, darum
will ich euch nunmehr inmitten Jerusalems zusammenbringen.
\bibleverse{20}Wie man Silber und Kupfer, Eisen, Blei und Zinn im
Schmelzofen zusammentut, um Feuer darunter\textless sup title=``oder:
dawider''\textgreater✲ anzufachen, damit es zum Schmelzen gebracht wird,
so will ich euch in meinem Zorn und Grimm zusammentun und euch
hineinlegen und zum Schmelzen bringen. \bibleverse{21}Versammeln will
ich euch und das Feuer meines Ingrimms gegen euch anfachen, daß ihr
darin\textless sup title=``d.h. in Jerusalem''\textgreater✲ geschmolzen
werden sollt. \bibleverse{22}Wie man Silber im Schmelzofen schmelzt, so
sollt ihr in der Stadt geschmolzen werden, damit ihr erkennt, daß ich,
der HERR, meinen Grimm über euch ausgegossen habe!«

\hypertarget{d-die-verderbtheit-erstreckt-sich-uxfcber-das-ganze-volk}{%
\paragraph{d) Die Verderbtheit erstreckt sich über das ganze
Volk}\label{d-die-verderbtheit-erstreckt-sich-uxfcber-das-ganze-volk}}

\bibleverse{23}Weiter erging das Wort des HERRN an mich folgendermaßen:
\bibleverse{24}»Menschensohn, sage zu ihm\textless sup title=``d.h. zu
Jerusalem und zum Lande Juda''\textgreater✲: ›Du bist ein Land, das
nicht benetzt, nicht beregnet worden ist in der Zeit des Grolls,
\bibleverse{25}dessen Fürsten in ihm wie ein brüllender und
beutegieriger Löwe gewesen sind: sie haben Menschenleben gefressen,
Reichtum und Kostbarkeiten an sich gebracht, die Zahl der Witwen in ihm
gemehrt. \bibleverse{26}Seine Priester haben meinem Gesetz Gewalt
angetan und das, was mir heilig ist, entweiht; zwischen Heiligem und
Unheiligem haben sie keinen Unterschied gemacht und das, was rein und
unrein ist, nicht zu unterscheiden gelehrt; vor meinen Sabbaten aber
haben sie ihre Augen geschlossen, so daß ich unter ihnen nicht mehr als
heilig gelte. \bibleverse{27}Ihre Fürsten\textless sup title=``oder:
Oberen''\textgreater✲ sind in ihrer Mitte wie beutegierige Wölfe: sie
gehen darauf aus, Blut zu vergießen und Menschenleben zu vernichten, um
Gewinn zu erraffen. \bibleverse{28}Ihre Propheten aber überstreichen
ihnen alles mit Tünche, indem sie erdichtete Gesichte schauen und ihnen
Lügen wahrsagen mit der Versicherung: So hat Gott der HERR gesprochen!,
während doch der HERR gar nicht geredet hat. \bibleverse{29}Das Volk im
Lande\textless sup title=``oder: das gewöhnliche Volk''\textgreater✲
verübt Gewalttätigkeit und begeht Raub, bedrückt die Armen und Elenden
und übervorteilt die Fremdlinge gegen alles Recht. \bibleverse{30}Ich
habe unter ihnen nach einem Manne gesucht, der eine Mauer aufführen
könnte und vor mir für das Land in den Riß\textless sup title=``=~in die
Bresche''\textgreater✲ treten möchte, damit ich es nicht zu Grunde
richtete, aber ich habe keinen gefunden. \bibleverse{31}Da habe ich denn
meinen Zorn sich über sie ergießen lassen, habe sie durch das Feuer
meines Grimms vernichtet und die Strafe für ihren Wandel auf ihr Haupt
fallen lassen!‹« -- so lautet der Ausspruch Gottes des HERRN.

\hypertarget{die-sittliche-verderbnis-der-beiden-schwesterreiche-israel-und-juda-bildlich-veranschaulicht}{%
\subsubsection{4. Die sittliche Verderbnis der beiden Schwesterreiche
Israel und Juda bildlich
veranschaulicht}\label{die-sittliche-verderbnis-der-beiden-schwesterreiche-israel-und-juda-bildlich-veranschaulicht}}

\hypertarget{a-einleitung-die-beiden-unzuxfcchtigen-schwestern-ohola-d.h.-samaria-und-oholibasup-titled.h.-jerusalem}{%
\paragraph{a) Einleitung: die beiden unzüchtigen Schwestern Ohola (d.h.
Samaria) und Oholiba\textless sup title=``d.h.
Jerusalem''\textgreater✲}\label{a-einleitung-die-beiden-unzuxfcchtigen-schwestern-ohola-d.h.-samaria-und-oholibasup-titled.h.-jerusalem}}

\hypertarget{section-22}{%
\section{23}\label{section-22}}

\bibleverse{1}Hierauf erging das Wort des HERRN an mich folgendermaßen:
\bibleverse{2}»Menschensohn, es waren zwei Frauen, Töchter derselben
Mutter, \bibleverse{3}die trieben Unzucht in Ägypten und buhlten schon
in ihrer Jugend; sie ließen dort ihre Brüste drücken, und dort betastete
man ihnen den jungfräulichen Busen: \bibleverse{4}die ältere hieß
Ohola\textless sup title=``d.h. ihr Zelt''\textgreater✲ und ihre
Schwester Oholiba\textless sup title=``d.h. mein Zelt in
ihr''\textgreater✲. Sie wurden beide mein\textless sup title=``d.h.
meine Gattinnen''\textgreater✲ und wurden Mütter von Söhnen und
Töchtern; und was ihre Namen betrifft: Ohola ist Samaria, und Oholiba
ist Jerusalem.«

\hypertarget{b-oholas-buhlerei-mit-den-assyrern-und-uxe4gyptern}{%
\paragraph{b) Oholas Buhlerei mit den Assyrern und
Ägyptern}\label{b-oholas-buhlerei-mit-den-assyrern-und-uxe4gyptern}}

\bibleverse{5}»Ohola aber trieb Buhlerei, obgleich sie mein Weib war,
und entbrannte in Liebe zu ihren Liebhabern, zu den Assyrern, die zu ihr
kamen, \bibleverse{6}gekleidet in blauen Purpur, Statthalter und
Befehlshaber, lauter schmucke Jünglinge, Reiter hoch zu Roß.
\bibleverse{7}An diese machte sie sich mit ihren Buhlkünsten, an alle
auserlesenen Assyrer, und befleckte sich mit allen, für die sie in Liebe
entbrannt war, mit deren gesamtem Götzendienst. \bibleverse{8}Dabei gab
sie aber auch ihr unzüchtiges Treiben mit den Ägyptern nicht auf; denn
die hatten ihr schon in ihrer Jugend beigewohnt und ihren jungfräulichen
Busen betastet und in Unzucht mit ihr gelebt. \bibleverse{9}Darum gab
ich sie der Gewalt ihrer Buhlen preis, der Gewalt der Assyrer, in die
sie leidenschaftlich verliebt war. \bibleverse{10}Die deckten ihre Blöße
auf, nahmen ihre Söhne und Töchter mit sich weg und brachten sie selbst
mit dem Schwerte um, so daß sie zum Gerede\textless sup title=``oder:
zum abschreckenden Beispiel''\textgreater✲ für die Frauen wurde, nachdem
man das Strafgericht an ihr vollstreckt hatte.«

\hypertarget{c-oholibas-buhlerei-mit-den-assyrern-chalduxe4ern-und-uxe4gyptern}{%
\paragraph{c) Oholibas Buhlerei mit den Assyrern, Chaldäern und
Ägyptern}\label{c-oholibas-buhlerei-mit-den-assyrern-chalduxe4ern-und-uxe4gyptern}}

\bibleverse{11}»Ihre Schwester Oholiba hatte das zwar gesehen, trieb es
aber trotzdem mit ihrer Verliebtheit noch ärger als jene und mit ihren
Buhlereien noch heilloser, als ihre unsittliche Schwester es getan
hatte. \bibleverse{12}Sie entbrannte in Liebe zu den Assyrern, zu
Statthaltern und Befehlshabern, die zu ihr kamen\textless sup
title=``vgl. V.5''\textgreater✲ und gar prächtig gekleidet waren, Reiter
hoch zu Roß, lauter schmucke Jünglinge. \bibleverse{13}Da sah ich, daß
auch sie sich befleckte: beide Schwestern trieben es in derselben Weise.
\bibleverse{14}Sie aber ging in ihrer Buhlerei noch weiter; und als sie
Männer auf die Wand mit Rötel gemalt sah, Bilder von Chaldäern,
\bibleverse{15}die um die Hüften einen Gürtel trugen und deren Haupt mit
überhängenden Kopfbünden✲ bedeckt war und die allesamt wie vornehme
Krieger aussahen, ähnlich den Babyloniern, deren Geburtsland Chaldäa
ist, \bibleverse{16}da entbrannte sie in Liebe zu ihnen, sobald sie
ihrer ansichtig wurde, und sandte Boten an sie ins Land der Chaldäer.
\bibleverse{17}Da kamen denn die Babylonier zu ihr, um der Liebe mit ihr
zu pflegen, und befleckten sie durch ihre Buhlerei; als sie sich aber an
ihnen verunreinigt hatte, wurde sie ihrer überdrüssig. \bibleverse{18}Da
nun ihre Unzucht offenkundig geworden war {[}und sie ihre Blöße
aufdeckte{]}, da wurde ich ihrer überdrüssig, wie ich ihrer Schwester
überdrüssig geworden war. \bibleverse{19}Sie aber trieb es mit ihrer
Buhlerei immer noch schlimmer, indem sie der Tage ihrer Jugendzeit
gedachte, als sie in Ägypten gebuhlt hatte; \bibleverse{20}und sie
entbrannte in Liebe zu den dortigen Wollüstlingen, die Glieder hatten
wie die Esel und Samenerguß wie die Hengste. \bibleverse{21}Ja, du
schautest dich um\textless sup title=``=~sehntest dich''\textgreater✲
nach der Unzucht deiner Jugend, als die Ägypter dir den Busen betastet
und deine jugendlichen Brüste gedrückt hatten.«

\hypertarget{d-ankuxfcndigung-des-durch-babylonien-an-oholiba-zu-vollstreckenden-guxf6ttlichen-gerichts}{%
\paragraph{d) Ankündigung des durch Babylonien an Oholiba zu
vollstreckenden göttlichen
Gerichts}\label{d-ankuxfcndigung-des-durch-babylonien-an-oholiba-zu-vollstreckenden-guxf6ttlichen-gerichts}}

\bibleverse{22}Darum, Oholiba, hat Gott der HERR so gesprochen: »Ich
will nunmehr deine Buhlen gegen dich aufreizen, eben die, deren du
überdrüssig geworden bist, und will sie von allen Seiten gegen dich
herankommen lassen: \bibleverse{23}die Babylonier und alle Chaldäer, die
von Pekod und Schoa und Koa, auch alle Assyrer mit ihnen, schmucke
Jünglinge, lauter Statthalter und Befehlshaber, vornehme Krieger und
edle Herren, alle hoch zu Roß. \bibleverse{24}Sie sollen (von Norden)
gegen dich heranziehen, ein Getümmel von Rossen und Wagen, und mit
Scharen von Völkern; mit großen und kleinen Schilden und Helmen werden
sie ringsum gegen dich anrücken; ihnen will ich den Rechtsstreit
vorlegen, damit sie dir nach ihren Rechtssatzungen das Urteil sprechen.
\bibleverse{25}Dann will ich dich meine Eifersucht fühlen lassen, damit
sie voll Ingrimms mit dir verfahren: Nase und Ohren werden sie dir
abschneiden, und was von dir noch übriggeblieben ist, wird durch das
Schwert fallen; deine Söhne und Töchter werden sie wegführen, und was
von dir noch übrigbleibt, wird vom Feuer verzehrt werden.
\bibleverse{26}Sie werden dir die Kleider ausziehen und deine kostbaren
Geschmeide mit fortnehmen. \bibleverse{27}So will ich deiner Buhlerei
und deiner Unzucht von Ägypten her ein Ende machen, so daß du deine
Augen nicht mehr zu ihnen erheben und an Ägypten nicht mehr zurückdenken
wirst.«

\bibleverse{28}Denn so hat Gott der HERR gesprochen: »Ich will dich
nunmehr in die Gewalt derer fallen lassen, die du hassest, in die Gewalt
derer, von denen dein Herz sich abgewandt hat. \bibleverse{29}Sie werden
dich ihren Haß fühlen lassen und dir alles wegnehmen, was du dir
erworben hast, und dich nackt und bloß liegen lassen, so daß deine
buhlerische Blöße aufgedeckt wird. Deine Unzucht und Ehebrecherei
\bibleverse{30}haben dir dieses eingetragen, weil du den Heidenvölkern
in deiner Gier nachgelaufen bist und zur Strafe dafür, daß du dich an
ihren Götzen verunreinigt hast. \bibleverse{31}Du bist denselben Weg
gegangen wie deine Schwester; darum gebe ich auch dir jetzt ihren Becher
zu trinken in die Hand.«

\bibleverse{32}So hat Gott der HERR gesprochen: »Den Becher deiner
Schwester sollst du trinken, den tiefen und weiten -- zum Gelächter und
Gespött sollst du werden --; er faßt reichlich viel, \bibleverse{33}so
daß du ganz trunken und des Jammers voll werden wirst: ein Becher des
Schauders und Entsetzens ist der Becher deiner Schwester Samaria.
\bibleverse{34}Du sollst ihn trinken und leeren\textless sup
title=``=~bis auf die Neige''\textgreater✲ und seine Scherben noch
ablecken und dir den Busen daran zerreißen; denn ich habe es gesagt!« --
so lautet der Ausspruch Gottes des HERRN.~-- \bibleverse{35}Darum hat
Gott der HERR so gesprochen: »Weil du mich vergessen und mich hinter
deinen Rücken geworfen hast, so trage auch du nun (die Strafe) für deine
Unzucht und deine Buhlerei!«

\hypertarget{e-abschlieuxdfende-zusammenfassung-von-schuld-besonders-von-religiuxf6sen-greueln-und-gerechter-strafe-beider-schwestern}{%
\paragraph{e) Abschließende Zusammenfassung von Schuld (besonders von
religiösen Greueln) und gerechter Strafe beider
Schwestern}\label{e-abschlieuxdfende-zusammenfassung-von-schuld-besonders-von-religiuxf6sen-greueln-und-gerechter-strafe-beider-schwestern}}

\bibleverse{36}Hierauf sagte der HERR zu mir: »Menschensohn, willst du
nicht mit Ohola und Oholiba ins Gericht gehen?\textless sup title=``vgl.
20,4''\textgreater✲ So halte ihnen denn ihre Greuel vor,
\bibleverse{37}daß sie Ehebruch getrieben haben und Blut an ihren Händen
klebt; und zwar haben sie mit ihren Götzen Ehebruch getrieben und ihnen
sogar ihre Kinder, die sie mir geboren hatten, als Opfer zum Fraß im
Feuer dargebracht. \bibleverse{38}Außerdem haben sie sich noch dadurch
an mir vergangen, daß sie an demselben Tage mein Heiligtum verunreinigt
und meine Sabbate entweiht haben. \bibleverse{39}Denn wenn sie ihre
Kinder ihren Götzen geschlachtet hatten, sind sie noch an demselben Tage
in mein Heiligtum gekommen, um es zu entweihen: siehe, so haben sie es
inmitten meines Hauses getrieben! \bibleverse{40}Ja, sie haben sogar zu
Männern gesandt, die von weither kommen sollten; und sobald ein Bote an
sie gesandt worden war, stellten sie sich auch ein. Für diese hast du
dich gebadet, hast dir die Augen geschminkt und dir Geschmeide angelegt;
\bibleverse{41}dann ließest du dich auf ein prächtiges Ruhebett nieder,
vor dem ein Tisch zugerichtet war, auf den du meinen Weihrauch und mein
Öl gestellt hattest. \bibleverse{42}Dabei erscholl dann der laute Gesang
einer frohgestimmten Menge; und zu Männern vom gemeinen Volk sandten
sie, und Säufer aus der Wüste wurden herbeigebracht; denen legten sie
Spangen an die Arme und setzten ihnen prachtvolle Kronen aufs Haupt.
\bibleverse{43}Da dachte ich von der durch Ehebruch Entkräfteten: ›Wird
sie, ja sie\textless sup title=``oder: da sie doch so
ist''\textgreater✲, jetzt noch ihre Unzucht treiben?‹
\bibleverse{44}Doch man kehrte bei ihr ein; wie man zu einer
öffentlichen Dirne eingeht: ebenso kehrte man bei Ohola und Oholiba, den
unzüchtigen Weibern, ein. \bibleverse{45}Aber gerechte Männer, die
sollen ihnen beiden das Urteil sprechen nach dem Recht, das für
Ehebrecherinnen und Mörderinnen gültig ist; denn Ehebrecherinnen sind
sie, und Blut klebt an ihren Händen.« \bibleverse{46}Denn so hat Gott
der HERR gesprochen: »Man berufe eine Gemeindeversammlung gegen sie und
gebe sie der Mißhandlung und Plünderung preis! \bibleverse{47}Die
Volksgemeinde soll sie dann steinigen und sie mit ihren Schwertern
zerhauen; ihre Söhne und Töchter wird man umbringen und ihre Häuser in
Flammen aufgehen lassen. \bibleverse{48}So will ich der Unzucht im Lande
ein Ende machen, damit alle Weiber sich warnen lassen und nicht Unzucht
treiben wie sie. \bibleverse{49}So wird man euch für eure Verworfenheit
büßen lassen, und ihr sollt die Strafe leiden für das, was ihr mit eurem
Götzendienst verschuldet habt, damit ihr erkennt, daß ich Gott der HERR
bin!«

\hypertarget{letzte-worte-uxfcber-die-zerstuxf6rung-jerusalems}{%
\subsubsection{5. Letzte Worte über die Zerstörung
Jerusalems}\label{letzte-worte-uxfcber-die-zerstuxf6rung-jerusalems}}

\hypertarget{a-gleichnis-vom-unheilbar-verrosteten-kochtopf-und-seinem-schicksal}{%
\paragraph{a) Gleichnis vom unheilbar verrosteten Kochtopf und seinem
Schicksal}\label{a-gleichnis-vom-unheilbar-verrosteten-kochtopf-und-seinem-schicksal}}

\hypertarget{section-23}{%
\section{24}\label{section-23}}

\bibleverse{1}Im neunten Jahre, am zehnten Tage des zehnten Monats,
erging das Wort des HERRN an mich folgendermaßen:
\bibleverse{2}»Menschensohn, schreibe dir den Namen\textless sup
title=``=~das Datum''\textgreater✲ des Tages auf, eben dieses heutigen
Tages! Gerade am heutigen Tage ist der König von Babylon vor Jerusalem
gerückt! \bibleverse{3}So trage denn dem widerspenstigen Geschlecht ein
Gleichnis vor mit folgenden Worten: ›So hat Gott der HERR gesprochen:
Setze den\textless sup title=``oder: einen''\textgreater✲ Kochtopf aufs
Feuer, setze ihn auf und gieße auch Wasser hinein; \bibleverse{4}tu die
Fleischstücke zusammen hinein, lauter gute Stücke, Lende und Schulter;
fülle ihn mit auserlesenen Knochen; \bibleverse{5}nimm eins von den
besten Schafen und schichte auch die Holzscheite darunter auf; laß es
tüchtig sieden, damit auch seine Knochen in ihm kochen.‹«

\bibleverse{6}Darum hat Gott der HERR so gesprochen: »Wehe über die
blutbefleckte Stadt, über den Kochtopf, an dem der Rost sitzt und von
dem sein Rost nicht abgeht! Hole die Fleischstücke einzeln aus ihm
heraus, ohne eine Auswahl bei ihnen zu treffen! \bibleverse{7}Denn das
von der Stadt vergossene Blut ist noch mitten in ihr: auf den nackten
Felsen hat sie es fließen lassen, hat es nicht auf den Boden gegossen,
daß man es mit Erde bedecken könnte. \bibleverse{8}Um den Zorn in mir
aufsteigen zu lassen, um Rache üben zu können, habe ich das von ihr
vergossene Blut auf den nackten Felsen fließen lassen, damit es nicht
bedeckt werde.«

\bibleverse{9}Darum hat Gott der HERR so gesprochen: »Wehe über die
blutbefleckte Stadt! Nun will ich selbst einen Holzstoß hoch
aufschichten. \bibleverse{10}Bringe viel Holz herbei, zünde das Feuer
an, koche das Fleisch gar, laß die Brühe einkochen und die Knochen
anbrennen! \bibleverse{11}Dann stelle den Topf leer auf seine
Kohlenglut, damit sein Erz glühend heiß wird und sein Schmutz in ihm
zerschmilzt und sein Rost verschwindet. \bibleverse{12}Aber alle Mühe
ist bei ihm verloren, denn der viele Rost geht doch nicht von ihm ab;
auch im Feuer bleibt der Rost an ihm sitzen. \bibleverse{13}Wegen deiner
schandbaren Unreinheit, weil ich dich habe reinigen wollen und du doch
von deinem Schmutz nicht rein geworden bist, sollst du auch ferner nicht
rein werden, bis ich meinen Grimm an dir gestillt habe!
\bibleverse{14}Ich, der HERR, habe es gesagt: es trifft ein, und ich
führe es aus, ohne nachzulassen und Schonung zu üben oder Mitleid zu
haben. Nach deinem Wandel und nach deinem ganzen Tun will ich dich
richten!« -- so lautet der Ausspruch Gottes des HERRN.

\hypertarget{b-unbeklagt-wie-die-verstorbene-gattin-des-propheten-wird-jerusalem-dahinsinken}{%
\paragraph{b) Unbeklagt wie die verstorbene Gattin des Propheten wird
Jerusalem
dahinsinken}\label{b-unbeklagt-wie-die-verstorbene-gattin-des-propheten-wird-jerusalem-dahinsinken}}

\hypertarget{aa-ankuxfcndigung-des-juxe4hen-todes-der-geliebten-gattin-und-verbot-der-totenklage-und-trauerbruxe4uche-um-sie}{%
\subparagraph{aa) Ankündigung des jähen Todes der geliebten Gattin und
Verbot der Totenklage und Trauerbräuche um
sie}\label{aa-ankuxfcndigung-des-juxe4hen-todes-der-geliebten-gattin-und-verbot-der-totenklage-und-trauerbruxe4uche-um-sie}}

\bibleverse{15}Weiter erging das Wort des HERRN an mich folgendermaßen:
\bibleverse{16}»Menschensohn, wisse wohl: ich will dir die Lust deiner
Augen durch einen plötzlichen Schlag✲ nehmen; aber du darfst dann weder
klagen noch weinen, und keine Träne soll dir kommen! \bibleverse{17}Du
magst im stillen seufzen, darfst aber keine Totentrauer anstellen; binde
dir deinen Kopfbund um\textless sup title=``oder: setze dir deinen
Turban auf''\textgreater✲ und ziehe deine Schuhe an deine Füße, lege dir
keine Hülle um den Bart\textless sup title=``vgl. Mi 3,7''\textgreater✲
und genieße die Speisen nicht, welche die Leute dir schicken.«

\hypertarget{bb-eintritt-des-todes-der-gattin-hesekiels-aufkluxe4rende-mitteilungen-an-die-gemeinde-der-verbannten}{%
\subparagraph{bb) Eintritt des Todes der Gattin; Hesekiels aufklärende
Mitteilungen an die Gemeinde der
Verbannten}\label{bb-eintritt-des-todes-der-gattin-hesekiels-aufkluxe4rende-mitteilungen-an-die-gemeinde-der-verbannten}}

\bibleverse{18}Nachdem ich hierauf am Morgen noch zum Volke geredet
hatte, starb meine Frau am Abend, und ich tat am folgenden Morgen so,
wie mir geboten war. \bibleverse{19}Als nun die Leute mich fragten:
»Willst du uns nicht mitteilen, was das für uns bedeuten soll, daß du so
verfährst?«, \bibleverse{20}antwortete ich ihnen: »Das Wort des HERRN
ist folgendermaßen an mich ergangen: \bibleverse{21}›Sage zum Hause
Israel: So hat Gott der HERR gesprochen: Wisset wohl: ich will mein
Heiligtum, den Gegenstand eures höchsten Stolzes, die Lust eurer Augen
und die Sehnsucht eures Herzens, entweihen, und eure Söhne und Töchter,
die ihr dort zurückgelassen habt, sollen durchs Schwert fallen.
\bibleverse{22}Da werdet ihr es dann so machen, wie ich jetzt getan
habe: um den Bart werdet ihr euch keine Hülle legen und die Speisen
nicht genießen, welche die Leute euch schicken; \bibleverse{23}euren
Kopfbund✲ werdet ihr auf dem Haupt und eure Schuhe an den Füßen
behalten, werdet nicht klagen noch weinen, wohl aber im Bewußtsein eurer
Verschuldung vergehen und einer gegen den andern seufzen\textless sup
title=``oder: den andern anstarren''\textgreater✲. \bibleverse{24}So
wird euch also Hesekiel als Wahrzeichen\textless sup title=``oder:
vorbildliches Zeichen''\textgreater✲ dienen, so daß ihr, wenn es
eintrifft, euch ganz so verhalten werdet, wie er sich verhalten hat;
dann werdet ihr auch erkennen, daß ich Gott der HERR bin.‹«

\hypertarget{c-ankuxfcndigung-neuer-wunderbarer-ereignisse-besonders-eines-bedeutsamen-wandels-in-hesekiels-prophetischer-tuxe4tigkeit}{%
\paragraph{c) Ankündigung neuer, wunderbarer Ereignisse (besonders eines
bedeutsamen Wandels in Hesekiels prophetischer
Tätigkeit)}\label{c-ankuxfcndigung-neuer-wunderbarer-ereignisse-besonders-eines-bedeutsamen-wandels-in-hesekiels-prophetischer-tuxe4tigkeit}}

\bibleverse{25}»Du aber, Menschensohn, wisse wohl: An dem Tage, wo ich
ihnen ihr Bollwerk✲ nehmen werde, ihren Stolz und ihre innige Freude,
die Lust ihrer Augen und die Sehnsucht ihres Herzens, dazu auch ihre
Söhne und Töchter: \bibleverse{26}an demselben Tage wird ein Flüchtling
zu dir kommen, um es dir persönlich zu melden. \bibleverse{27}An jenem
Tage wird dir der Mund zugleich mit dem Eintreffen des Flüchtlings
aufgetan werden: du wirst dann wieder reden können und nicht länger
stumm sein\textless sup title=``vgl. 3,26-27''\textgreater✲. So wirst du
ihnen zum Wahrzeichen\textless sup title=``oder: vorbildlichen
Zeichen''\textgreater✲ dienen, und sie werden erkennen, daß ich der HERR
bin.«

\hypertarget{b.-zweiter-hauptteil-drohweissagungen-gegen-die-heidnischen-nachbarvuxf6lker-kap.-25-32}{%
\subsection{B. Zweiter Hauptteil: Drohweissagungen gegen die heidnischen
Nachbarvölker (Kap.
25-32)}\label{b.-zweiter-hauptteil-drohweissagungen-gegen-die-heidnischen-nachbarvuxf6lker-kap.-25-32}}

\hypertarget{drohreden-gegen-ammon-moab-edom-und-die-philister}{%
\subsubsection{1. Drohreden gegen Ammon, Moab, Edom und die
Philister}\label{drohreden-gegen-ammon-moab-edom-und-die-philister}}

\hypertarget{section-24}{%
\section{25}\label{section-24}}

\bibleverse{1}Hierauf erging das Wort des HERRN an mich folgendermaßen:
\bibleverse{2}»Menschensohn, richte deine Blicke gegen die Ammoniter und
weissage gegen sie! \bibleverse{3}Sprich zu den Ammonitern: ›Hört das
Wort Gottes, des HERRN! So hat Gott der HERR gesprochen: Weil du Haha!
gerufen hast über mein Heiligtum, weil es entweiht wurde, und über das
Land Israel, weil es verwüstet wurde, und über das Haus Juda, weil sie
in die Verbannung\textless sup title=``oder:
Gefangenschaft''\textgreater✲ wandern mußten: \bibleverse{4}darum,
fürwahr, will ich dich den Söhnen des Ostens zum Besitztum geben, daß
sie ihre Zeltlager in dir aufschlagen und ihre Wohnungen in dich
verlegen; sie werden deine Früchte essen und sie deine Milch trinken.
\bibleverse{5}Und ich will Rabba zu einer Weide für Kamele machen und
die Ortschaften der Ammoniter zu einem Lagerplatz für Kleinvieh, damit
ihr erkennt, daß ich der HERR bin. \bibleverse{6}Denn so hat Gott der
HERR gesprochen: »Weil du in die Hände geklatscht und mit dem Fuß
gestampft und dich gefühllos von ganzem Herzen über das Land Israel
gefreut hast: \bibleverse{7}darum will ich nunmehr meine Hand gegen dich
ausstrecken und dich den Völkern zur Plünderung preisgeben, ich will
dich aus den Völkerschaften ausrotten und dich endgültig aus der Zahl
der Länder verschwinden lassen, damit du erkennst, daß ich der HERR
bin!«~--

\bibleverse{8}So hat Gott der HERR gesprochen: »Weil Moab und Seir
sagen\textless sup title=``oder: gesagt haben''\textgreater✲: ›Nunmehr
ergeht es dem Hause Juda wie allen anderen Völkern!‹,
\bibleverse{9}darum will ich jetzt die Abhänge Moabs
entblößen\textless sup title=``oder: zugänglich machen''\textgreater✲,
so daß es der Städte verlustig geht, seiner Städte verlustig ohne alle
Ausnahme, der Zierde des Landes: Beth-Jesimoth, Baal-Meon und
Kirjathaim. \bibleverse{10}Den Söhnen des Ostens will ich es samt dem
Lande der Ammoniter zum Eigentum geben, damit der Ammoniter nicht mehr
gedacht wird unter den Völkern. \bibleverse{11}An den Moabitern aber
will ich (so) das Strafgericht vollstrecken, damit sie erkennen, daß ich
der HERR bin!«~--

\bibleverse{12}So hat Gott der HERR gesprochen: »Weil Edom mit Rachgier
am Hause Juda gehandelt und sich durch Vollziehung der Rache an ihnen
schwer verschuldet hat, \bibleverse{13}darum hat Gott der HERR so
gesprochen: ›Ich will meine Hand gegen Edom ausstrecken und Menschen
samt Vieh in ihm ausrotten und will es zur Einöde machen; von Theman an,
bis nach Dedan hin sollen sie durchs Schwert fallen! \bibleverse{14}Ich
will aber die Vollstreckung meiner Rache an Edom in die Hand meines
Volkes Israel legen, daß sie mit den Edomitern so verfahren, wie es
meinem Zorn und meinem Grimm entspricht, und jene meine Rache fühlen!‹«
-- so lautet der Ausspruch Gottes des HERRN.~--

\bibleverse{15}So hat Gott der HERR gesprochen: »Weil die Philister mit
Rachgier gehandelt und mit gefühllosem Herzen in nie endender
Feindschaft Rache geübt haben, um Verderben anzurichten:
\bibleverse{16}darum hat Gott der HERR so gesprochen: ›Nunmehr will ich
meine Hand gegen die Philister ausstrecken und die Kreter ausrotten und,
was von ihnen an der Meeresküste noch übrig ist, vertilgen.
\bibleverse{17}Ja, ich will schwere Rachetaten an ihnen vollziehen durch
schonungslose Züchtigungen, damit sie erkennen, daß ich der HERR bin,
wenn ich sie meine Rache fühlen lasse!‹«

\hypertarget{drohreden-gegen-tyrus-und-sidon-kap.-26-28}{%
\subsubsection{2. Drohreden gegen Tyrus und Sidon (Kap.
26-28)}\label{drohreden-gegen-tyrus-und-sidon-kap.-26-28}}

\hypertarget{a-gottesspruch-gegen-tyrus}{%
\paragraph{a) Gottesspruch gegen
Tyrus}\label{a-gottesspruch-gegen-tyrus}}

\hypertarget{aa-die-schuld-der-stadt-und-gottes-drohung}{%
\subparagraph{aa) Die Schuld der Stadt und Gottes
Drohung}\label{aa-die-schuld-der-stadt-und-gottes-drohung}}

\hypertarget{section-25}{%
\section{26}\label{section-25}}

\bibleverse{1}Im elften Jahre, am ersten Tage des (elften) Monats, da
erging das Wort des HERRN an mich folgendermaßen:
\bibleverse{2}»Menschensohn, weil Tyrus über Jerusalem ausgerufen hat:
›Haha! Zertrümmert ist das Tor der Völker, mir hat es sich aufgetan: ich
werde nun alles vollauf haben, weil Jerusalem zerstört ist!‹ --,
\bibleverse{3}darum spricht Gott der HERR so: ›Ich will nunmehr an
dich\textless sup title=``=~gegen dich vorgehen''\textgreater✲, Tyrus,
und will Völker in Menge gegen dich heranführen, wie das Meer seine
Wogen heranfluten läßt! \bibleverse{4}Sie sollen die Mauern von Tyrus
zerstören und seine Türme niederreißen; und ich will das Erdreich von
ihm wegfegen und es zu einem nackten Felsen machen: \bibleverse{5}ein
Trockenplatz für Fischernetze soll es werden inmitten des Meeres; denn
ich habe es gesagt‹ -- so lautet der Ausspruch Gottes des HERRN --, ›und
es soll den Völkern zur Beute werden; \bibleverse{6}seine Tochterstädte
aber, die auf dem Festlande liegen, sollen durch das Schwert vernichtet
werden, damit sie erkennen, daß ich der HERR bin!‹«

\hypertarget{bb-die-stadt-wird-durch-nebukadnezar-belagert-und-vernichtet-werden}{%
\subparagraph{bb) Die Stadt wird durch Nebukadnezar belagert und
vernichtet
werden}\label{bb-die-stadt-wird-durch-nebukadnezar-belagert-und-vernichtet-werden}}

\bibleverse{7}Denn so hat Gott der HERR gesprochen: »Nunmehr will ich
Nebukadnezar, den König von Babylon, den König der Könige, von Norden
her gegen Tyrus heranziehen lassen mit Rossen, Kriegswagen und Reitern
und mit der Heeresmacht vieler Völker. \bibleverse{8}Deine Tochterstädte
auf dem Festland wird er mit dem Schwert vernichten; gegen dich aber
wird er Belagerungstürme errichten und einen Damm gegen dich aufführen
und Schilddächer gegen dich aufstellen; \bibleverse{9}den Stoß seiner
Sturmböcke\textless sup title=``oder: Mauerbrecher''\textgreater✲ wird
er gegen deine Mauern richten und deine Türme mit seinen Eisenhaken
niederreißen. \bibleverse{10}Infolge des Heranflutens\textless sup
title=``=~der heranflutenden Menge''\textgreater✲ seiner Rosse wird ihr
Staub dich bedecken; vom Getöse seiner Reiter und der Räder seiner
Streitwagen werden deine Mauern erbeben, wenn er in deine Tore einzieht,
wie man in eine eroberte Stadt einzieht. \bibleverse{11}Mit den Hufen
seiner Rosse wird er alle deine Straßen zerstampfen, wird dein Volk mit
dem Schwert umbringen, und die Bildsäulen, auf die du dein Vertrauen
setzst, wird er zu Boden stürzen. \bibleverse{12}Sie werden deine
Schätze rauben, deine Handelsgüter plündern, deine Mauern niederreißen,
deine Prachtgebäude zerstören und deine Steine, deine Balken und den
Schutt von dir ins Wasser werfen. \bibleverse{13}So will ich dem Getön
deiner Lieder ein Ende machen, und der Klang deiner Harfen soll nicht
mehr vernommen werden. \bibleverse{14}Ich will dich zu einem kahlen
Felsen machen: zu einem Trockenplatz für Fischernetze sollst du werden,
und nie sollst du wieder aufgebaut werden, denn ich, der HERR, habe es
gesagt!« -- so lautet der Ausspruch Gottes des HERRN.

\hypertarget{cc-schilderung-des-eindrucks-dieser-zerstuxf6rung-auf-die-vuxf6lkerwelt}{%
\subparagraph{cc) Schilderung des Eindrucks dieser Zerstörung auf die
Völkerwelt}\label{cc-schilderung-des-eindrucks-dieser-zerstuxf6rung-auf-die-vuxf6lkerwelt}}

\bibleverse{15}So hat Gott, der HERR, in bezug auf Tyrus gesprochen:
»Fürwahr, vom Dröhnen deines Falles, wenn die Erschlagenen\textless sup
title=``oder: tödlich Verwundeten''\textgreater✲ ächzen, wenn das
Schwert in deiner Mitte mordet, werden die Meeresländer erbeben;
\bibleverse{16}alle Fürsten des Meeres werden von ihren Thronen
herabsteigen, ihre Prachtgewänder\textless sup title=``oder:
Turbane''\textgreater✲ ablegen und ihre buntgestickten Kleider
ausziehen; in Trauer werden sie sich kleiden, auf den Erdboden werden
sie sich setzen und jeden Augenblick\textless sup title=``=~ohne
Unterlaß''\textgreater✲ erzittern und deinethalben schaudern.
\bibleverse{17}Da werden sie denn in Wehklagen über dich ausbrechen und
zu\textless sup title=``oder: von''\textgreater✲ dir sagen: ›Ach, wie
bist du untergegangen, vom Meer verschwunden, du hochberühmte Stadt, die
da mächtig auf dem Meere war, sie und ihre Bewohner, welche allen
Anwohnern des Meeres Schrecken vor sich einflößten! \bibleverse{18}Jetzt
erzittern die Meeresländer am Tage deines Falles, und die Inseln im Meer
sind entsetzt über deinen Ausgang!‹«

\hypertarget{dd-gott-hat-die-vollstuxe4ndige-vernichtung-der-stadt-beschlossen}{%
\subparagraph{dd) Gott hat die vollständige Vernichtung der Stadt
beschlossen}\label{dd-gott-hat-die-vollstuxe4ndige-vernichtung-der-stadt-beschlossen}}

\bibleverse{19}Denn so hat Gott der HERR gesprochen: »Wenn ich dich zu
einer verödeten Stadt mache wie andere Städte, die nicht mehr bewohnt
werden; wenn ich die Meereswogen über\textless sup title=``oder:
gegen''\textgreater✲ dich heranfluten lasse, daß die weiten Wasser dich
bedecken: \bibleverse{20}da will ich dich hinunterstoßen zu den in die
Grube\textless sup title=``oder: ins Grab''\textgreater✲
Hinabgefahrenen, zu dem Volk der Vorzeit, und dir deine Wohnung anweisen
in den tiefsten Tiefen der Erde, in den uralten Trümmerstätten, bei
denen, die in die Grube hinabgefahren sind, damit du nicht mehr bewohnt
wirst und nicht mehr zur Schau dastehst im Lande der Lebenden!
\bibleverse{21}Einem schreckensvollen Untergang will ich dich
preisgeben, auf daß du nicht mehr da bist: man wird dich suchen, aber
dich in Ewigkeit nicht mehr finden!« -- so lautet der Ausspruch Gottes
des HERRN.

\hypertarget{b-klagelied-uxfcber-den-untergang-der-stadt-tyrus}{%
\paragraph{b) Klagelied über den Untergang der Stadt
Tyrus}\label{b-klagelied-uxfcber-den-untergang-der-stadt-tyrus}}

\hypertarget{aa-tyrus-als-stolzes-prachtschiff-die-herrlichkeit-der-stadt-besonders-als-des-weltmarktes}{%
\subparagraph{aa) Tyrus als stolzes Prachtschiff; die Herrlichkeit der
Stadt, besonders als des
Weltmarktes}\label{aa-tyrus-als-stolzes-prachtschiff-die-herrlichkeit-der-stadt-besonders-als-des-weltmarktes}}

\hypertarget{section-26}{%
\section{27}\label{section-26}}

\bibleverse{1}Weiter erging das Wort des HERRN an mich folgendermaßen:
\bibleverse{2}»Du, Menschensohn, stimme ein Klagelied über Tyrus an
\bibleverse{3}und sprich zu Tyrus: ›(O Stadt), die du wohnst am Zugang
zum Meer und Handel mit den Völkern treibst nach den vielen
Meeresländern hin: so hat Gott der HERR gesprochen: Tyrus, du hast
gedacht: Ich bin ein Schiff, vollkommen an Schönheit! \bibleverse{4}Dein
Gebiet liegt im Herzen der Meere; deine Erbauer haben dich zu (einem
Schiff von) vollendeter Schönheit gemacht. \bibleverse{5}Aus Zypressen
vom Senir haben sie dir alles Plankenwerk gebaut und eine Zeder vom
Libanon genommen, um den Mastbaum auf dir\textless sup title=``oder: für
dich''\textgreater✲ daraus zu fertigen. \bibleverse{6}Aus Eichen von
Basan haben sie deine Ruder hergestellt, dein Verdeck aus Edeltannenholz
{[}mit Elfenbein ausgelegt{]} von den Eilanden der Kitthäer.
\bibleverse{7}Feine Leinwand mit Buntstickerei aus Ägypten war dein
Segel, um dir als Flagge zu dienen; blauer und roter Purpur von den
Gestaden Elisas war deine Überdachung\textless sup title=``=~dein
Kajütendach''\textgreater✲. \bibleverse{8}Die Einwohner\textless sup
title=``oder: Fürsten''\textgreater✲ von Sidon und Arwad dienten dir als
Ruderer; die erfahrensten Männer aus deiner eigenen Mitte waren deine
Steuerleute; \bibleverse{9}die Ältesten von Gebal und die dortigen
Meister waren in\textless sup title=``oder: auf''\textgreater✲ dir als
Ausbesserer deiner Lecke. Alle Seeschiffe samt ihrer Bemannung fanden
sich bei dir ein, um Tauschhandel mit dir zu treiben.
\bibleverse{10}Perser und Put und Lud\textless sup title=``=~Libyer und
Lyder''\textgreater✲ dienten dir in deinem Heer als deine Kriegsleute;
Schild und Helm hängten sie bei dir auf, die verliehen dir
Glanz\textless sup title=``oder: Ansehen''\textgreater✲.
\bibleverse{11}Männer aus Arwad und deine eigene Heeresmacht standen auf
deinen Mauern ringsum und Gammadäer auf deinen Türmen; ihre Schilde
hängten sie ringsum an deinen Mauern auf; sie machten deine Schönheit
vollkommen. \bibleverse{12}Tharsis trieb Handel mit dir wegen der Fülle
an allerlei Gütern: mit Silber, Eisen, Zinn und Blei bezahlten sie deine
Waren. \bibleverse{13}Die Jonier, Tibarener und Moscher machten
Handelsgeschäfte mit dir: Sklaven und eherne Geräte gaben sie dir als
Zahlung beim Tauschhandel. \bibleverse{14}Die vom Hause\textless sup
title=``oder: Stamm''\textgreater✲ Thogarma bezahlten deine Waren mit
Rossen, Reitpferden und Mauleseln. \bibleverse{15}Die Rhodier waren
deine Kaufleute; zahlreiche Meeresländer standen im Verkehr mit dir:
Elefantenzähne und Ebenholz lieferten sie dir als Zahlung.
\bibleverse{16}Die Syrer schlossen Geschäfte mit dir ab wegen der Menge
deiner Erzeugnisse: mit Karfunkel, rotem Purpur und bunten Geweben, mit
feiner Leinwand, Korallen\textless sup title=``oder:
Perlen''\textgreater✲ und Rubinen bezahlten sie deine Waren.
\bibleverse{17}Juda und das Land Israel waren deine Kaufleute: Weizen
von Minnith und Wachs, Honig, Öl und Balsam gaben sie dir als Zahlung
beim Tauschhandel. \bibleverse{18}Damaskus trieb Handel mit dir ob der
Menge deiner Erzeugnisse, wegen der Fülle an allerlei Gütern, mit Wein
von Helbon und Wolle von Zahar. \bibleverse{19}Wedan und Jawan brachten
aus Usal kunstvoll geschmiedetes Eisen auf deinen Markt; Kassia und
Kalmus hatten sie für dich als Tauschwaren. \bibleverse{20}Dedan trieb
Handel mit dir in Satteldecken zum Reiten. \bibleverse{21}Arabien und
alle Häuptlinge Kedars standen in Verkehr mit dir: mit Lämmern, Widdern
und Böcken trieben sie Handel mit dir. \bibleverse{22}Die Kaufleute von
Saba und Ragma handelten mit dir: mit den köstlichsten Gewürzen, mit
allerlei Edelsteinen und mit Gold bezahlten sie deine Waren.
\bibleverse{23}Haran, Kanne und Eden handelten mit dir; Assur und ganz
Medien waren deine Kunden; \bibleverse{24}sie handelten mit dir in
Prachtgewändern, in Mänteln von blauem Purpur und buntgewirkten Stoffen,
in farbenreichen Teppichen, in geflochtenen und fest gedrehten Tauen,
gegen deine Waren. \bibleverse{25}Die Schiffe von Tharsis vertrieben
deine Güter im Tauschhandel, und so wurdest du mit Reichtum angefüllt
und kamst zu hoher Macht inmitten der Meere.‹«

\hypertarget{bb-der-juxe4he-untergang-des-schiffes-eindruck-dieses-ereignisses-auf-die-vuxf6lkerwelt}{%
\subparagraph{bb) Der jähe Untergang des Schiffes; Eindruck dieses
Ereignisses auf die
Völkerwelt}\label{bb-der-juxe4he-untergang-des-schiffes-eindruck-dieses-ereignisses-auf-die-vuxf6lkerwelt}}

\bibleverse{26}»›Deine Ruderer haben dich auf die hohe See
hinausgeführt, doch der Ostwind bringt dich zum Scheitern inmitten des
Meeres. \bibleverse{27}Deine Reichtümer und Waren, deine Handelsgüter,
deine Matrosen und Steuerleute, deine Leckausbesserer, deine
Tauschhändler und alle deine Kriegsleute, die sich auf dir befinden,
mitsamt der ganzen Volksmenge auf dir werden in die Tiefe des Meeres
sinken am Tage deines Untergangs. \bibleverse{28}Vom lauten Geschrei
deiner Steuerleute wird die weite Meeresfläche erbeben.
\bibleverse{29}Da werden dann alle, die das Ruder führen, die Seeleute
und alle, die das Meer durchsteuern, aus ihren Schiffen steigen, werden
ans Land treten \bibleverse{30}und lauten Wehruf über dich erschallen
lassen und kläglich schreien; sie werden sich Staub aufs Haupt streuen
und sich in der Asche wälzen; \bibleverse{31}sie werden sich um
deinetwillen kahl scheren, mit Sackleinen✲ sich umgürten und mit
bekümmertem Herzen um dich weinen in bitterer Klage. \bibleverse{32}In
ihrem Schmerz werden sie ein Klagelied über dich anstimmen und über dich
wehklagen: Welcher Ort ist so totenstill wie Tyrus inmitten des Meeres!
\bibleverse{33}Solange deine Waren dem Meer entstiegen, hast du die
Bedürfnisse vieler Völker befriedigt und durch die Fülle deiner Güter
und Waren die Könige der Erde reich gemacht. \bibleverse{34}Jetzt aber,
da du zertrümmert, vom Meere verschwunden, in Wassertiefen begraben bist
und dein Handel und deine ganze Volksmenge mitten in dir versunken ist,
\bibleverse{35}entsetzen sich alle Bewohner der Meeresländer über dich,
ihre Könige sind von Schauder erfaßt, ihre Angesichter zucken
schmerzlich. \bibleverse{36}Die Kaufleute in der ganzen Welt
zischen\textless sup title=``=~pfeifen höhnisch''\textgreater✲ über
dich: ein Ende mit Schrecken hast du genommen; du bist dahin für
immer!‹«

\hypertarget{c-gottesspruch-gegen-den-fuxfcrsten-von-tyrus}{%
\paragraph{c) Gottesspruch gegen den Fürsten von
Tyrus}\label{c-gottesspruch-gegen-den-fuxfcrsten-von-tyrus}}

\hypertarget{aa-gottes-gericht-uxfcber-den-hochmut-des-fuxfcrsten}{%
\subparagraph{aa) Gottes Gericht über den Hochmut des
Fürsten}\label{aa-gottes-gericht-uxfcber-den-hochmut-des-fuxfcrsten}}

\hypertarget{section-27}{%
\section{28}\label{section-27}}

\bibleverse{1}Weiter erging das Wort des HERRN an mich folgendermaßen:
\bibleverse{2}»Menschensohn, sage zum Fürsten von Tyrus: ›So hat Gott
der HERR gesprochen: Weil dein Sinn hoch hinaus wollte und du gesagt
hast: Ein Gott bin ich, einen Göttersitz bewohne ich mitten im Meer! --
während du doch nur ein Mensch bist und kein Gott --, und weil du dich
in deinem Herzen dünktest wie ein Gott~-- \bibleverse{3}natürlich bist
du weiser als Daniel, nichts Verborgenes ist dunkel für dich!
\bibleverse{4}Durch deine Weisheit und Einsicht hast du dir ja Reichtum
erworben und Gold und Silber in deine Schatzkammern geschafft;
\bibleverse{5}durch deine große Weisheit hast du bei deinem
Handelsbetrieb deinen Reichtum gemehrt, und dein Sinn ging infolge
deines Reichtums hoch hinaus --: \bibleverse{6}darum hat Gott der HERR
so gesprochen: Weil dein Herz sich überhoben hat, als ob du ein Gott
wärst, \bibleverse{7}darum will ich nunmehr Fremde gegen dich
heranziehen lassen, die wildesten Völkerschaften; die werden deiner
schönen Weisheit mit dem Schwert zu Leibe gehen und deinen Glanz trüben.
\bibleverse{8}In die Grube werden sie dich hinabstoßen, und du wirst den
Tod eines Erschlagenen sterben mitten im Meer! \bibleverse{9}Wirst du
dann wohl angesichts deiner Mörder auch noch sagen: Ein Gott bin ich!,
während du doch nur ein Mensch bist und kein Gott, in der Hand derer,
die dich durchbohren? \bibleverse{10}Den Tod von Unbeschnittenen wirst
du erleiden durch die Hand von Fremden! Denn ich habe es gesagt!‹« -- so
lautet der Ausspruch Gottes des HERRN.

\hypertarget{bb-klagelied-uxfcber-den-tod-dieses-fuxfcrsten}{%
\subparagraph{bb) Klagelied über den Tod dieses
Fürsten}\label{bb-klagelied-uxfcber-den-tod-dieses-fuxfcrsten}}

\bibleverse{11}Weiter erging das Wort des HERRN an mich folgendermaßen:
\bibleverse{12}»Menschensohn, stimme ein Klagelied an über den König von
Tyrus und sage zu ihm: ›So hat Gott der HERR gesprochen: Der du das Bild
der Vollkommenheit warst✲, voll von Weisheit und von vollendeter
Schönheit: \bibleverse{13}in Eden, dem Garten Gottes, befandest du dich,
allerlei Edelsteine bedeckten deine Gewandung: Karneol, Topas und
Jaspis, Chrysolith, Beryll und Onyx, Saphir, Rubin und Smaragd, und aus
Gold waren deine Einfassungen und die Verzierungen an dir gearbeitet; am
Tage deiner Erschaffung wurden sie eingesetzt. \bibleverse{14}Du warst
ein gesalbter schirmender Cherub: ich hatte dich dazu bestellt; auf dem
heiligen Götterberge weiltest du, inmitten feuriger Steine wandeltest
du. \bibleverse{15}Unsträflich warst du in all deinem Tun vom Tage
deiner Erschaffung an, bis Verschuldung an dir gefunden wurde.
\bibleverse{16}Infolge deines ausgedehnten Handelsverkehrs füllte sich
dein Inneres mit Frevel, und als du dich versündigt hattest, trieb ich
dich vom Götterberge weg, und der schirmende Cherub verstieß dich aus
der Mitte der feurigen Steine. \bibleverse{17}Dein Sinn war hochfahrend
geworden infolge deiner Schönheit, und du hattest deine Weisheit außer
acht gelassen um deines Glanzes willen; darum schleuderte ich dich auf
die Erde hinab und gab dich vor Könige hin, damit sie eine Augenweide an
dir hätten. \bibleverse{18}Infolge der Menge deiner Verschuldungen,
durch die Unehrlichkeit deines Handelsbetriebes hattest du deine
Heiligtümer entweiht; darum habe ich ein Feuer aus deiner Mitte
hervorbrechen lassen, das dich verzehrt hat, und ich habe dich in Asche
auf die Erde hingelegt vor den Augen aller, die dich sahen.
\bibleverse{19}Alle, die dich unter den Völkern gekannt haben, sind über
dich\textless sup title=``=~über dein Geschick''\textgreater✲ entsetzt;
ein Ende mit Schrecken hast du genommen: du bist dahin für immer!‹«

\hypertarget{d-gottesspruch-gegen-sidon}{%
\paragraph{d) Gottesspruch gegen
Sidon}\label{d-gottesspruch-gegen-sidon}}

\bibleverse{20}Weiter erging das Wort des HERRN an mich folgendermaßen:
\bibleverse{21}»Menschensohn, richte deine Blicke gegen Sidon und
weissage gegen es \bibleverse{22}mit folgenden Worten: ›So hat Gott der
HERR gesprochen: Nunmehr will ich an dich\textless sup title=``=~gegen
dich vorgehen''\textgreater✲, Sidon, und will meine Macht in deiner
Mitte\textless sup title=``oder: an dir''\textgreater✲ erweisen, damit
sie erkennen, daß ich der HERR bin, wenn ich Strafgerichte an dir
vollziehe und mich als den Heiligen an dir erweise. \bibleverse{23}Ich
will die Pest in dich hineinsenden und Blutvergießen auf deine Straßen;
und vom Schwert Erschlagene sollen in deiner Mitte ringsum hinsinken,
damit sie erkennen, daß ich der HERR bin.‹«

\hypertarget{e-abschluuxdf-zweck-aller-dieser-gottesgerichte-ausblick-in-die-spuxe4tere-heilszeit-israels}{%
\paragraph{e) Abschluß: Zweck aller dieser Gottesgerichte; Ausblick in
die spätere Heilszeit
Israels}\label{e-abschluuxdf-zweck-aller-dieser-gottesgerichte-ausblick-in-die-spuxe4tere-heilszeit-israels}}

\bibleverse{24}»Für das Haus Israel aber wird es alsdann keinen
stechenden Dorn und keinen schmerzenden Stachel mehr geben von seiten
aller umwohnenden Völker, die sie verächtlich behandelt haben, und sie
werden erkennen, daß ich Gott, der HERR, bin.« \bibleverse{25}So hat
Gott der HERR gesprochen: »Wenn ich die vom Hause Israel aus den
Völkern, unter die sie zerstreut worden sind, wieder sammle, dann will
ich mich an ihnen vor den Augen der Heidenvölker als den Heiligen
erweisen, und sie sollen in ihrem Lande wohnen, das ich meinem Knecht
Jakob gegeben habe. \bibleverse{26}Und sie sollen in Sicherheit darin
wohnen und Häuser bauen und Weinberge anlegen; ja in Sicherheit sollen
sie wohnen, während ich Strafgerichte an allen umwohnenden Völkern
vollstrecke, die sie verächtlich behandelt haben; dann werden sie
erkennen, daß ich, der HERR, ihr Gott bin.«

\hypertarget{weissagungen-meist-strafreden-gegen-uxe4gypten-kap.-29-32}{%
\subsubsection{3. Weissagungen (meist Strafreden) gegen Ägypten (Kap.
29-32)}\label{weissagungen-meist-strafreden-gegen-uxe4gypten-kap.-29-32}}

\hypertarget{a-uxe4gyptens-untergang-und-spuxe4tere-wiederherstellung-entschuxe4digung-nebukadnezars}{%
\paragraph{a) Ägyptens Untergang und spätere Wiederherstellung;
Entschädigung
Nebukadnezars}\label{a-uxe4gyptens-untergang-und-spuxe4tere-wiederherstellung-entschuxe4digung-nebukadnezars}}

\hypertarget{section-28}{%
\section{29}\label{section-28}}

\bibleverse{1}Im zehnten Jahre, am zwölften Tage des zehnten Monats,
erging das Wort des HERRN an mich folgendermaßen:
\bibleverse{2}»Menschensohn, richte deine Blicke gegen den Pharao, den
König von Ägypten, und sprich gegen ihn und gegen ganz Ägypten folgende
Weissagungen aus.«

\hypertarget{aa-ankuxfcndigung-der-vernichtung-des-pharaos-des-grouxdfen-krokodils}{%
\subparagraph{aa) Ankündigung der Vernichtung des Pharaos, des großen
Krokodils}\label{aa-ankuxfcndigung-der-vernichtung-des-pharaos-des-grouxdfen-krokodils}}

\bibleverse{3}»So hat Gott der HERR gesprochen: ›Nunmehr will ich an
dich\textless sup title=``=~gegen dich vorgehen''\textgreater✲, Pharao,
König von Ägypten, du großes Krokodil, das inmitten seiner Ströme✲
lagert, das da spricht: Mir gehört mein Strom, und ich habe ihn mir
geschaffen! \bibleverse{4}So will ich dir nun Haken in die Kinnbacken
legen und die Fische deiner Ströme an deinen Schuppen ankleben lassen
und will dich mitten aus deinen Strömen heraufziehen samt allen Fischen
deiner Ströme, die fest an deinen Schuppen hängen. \bibleverse{5}Dann
will ich dich in die Wüste hinwerfen, dich und alle Fische deiner
Ströme; auf das freie Feld sollst du fallen, ohne aufgehoben und
bestattet zu werden: den Tieren des Feldes und den Vögeln des Himmels
will ich dich zum Fraß geben. \bibleverse{6}Da werden denn alle Bewohner
Ägyptens erkennen, daß ich der HERR bin, weil du für das Haus Israel nur
ein Rohrstab\textless sup title=``=~eine Stütze von
Schilfrohr''\textgreater✲ gewesen bist~-- \bibleverse{7}wenn sie dich in
die Hand nahmen, knicktest du ein und rissest ihnen die ganze Hand auf;
und wenn sie sich auf dich stützen wollten, zerbrachst du und machtest
ihnen die ganzen Hüften wanken.‹«

\hypertarget{bb-uxe4gyptens-verwuxfcstung}{%
\subparagraph{bb) Ägyptens
Verwüstung}\label{bb-uxe4gyptens-verwuxfcstung}}

\bibleverse{8}»Darum hat Gott der HERR so gesprochen: ›Nunmehr will ich
das Schwert über dich kommen lassen und Menschen samt Vieh in dir
ausrotten; \bibleverse{9}und Ägyptenland soll zur Wüste und Einöde
werden, damit man erkennt, daß ich der HERR bin. Weil du gesagt hast:
›Mir gehört der Nilstrom, und ich habe ihn geschaffen!‹,
\bibleverse{10}darum will ich nunmehr an dich und an deine Ströme✲ und
will das Land Ägypten zu Wüsteneien machen, zu wüsten Einöden von Migdol
bis nach Syene, bis an die Grenze von Äthiopien. \bibleverse{11}Keines
Menschen Fuß soll es durchwandern, und auch der Fuß keines Tieres soll
es durchschreiten, und es soll vierzig Jahre lang unbewohnt bleiben.
\bibleverse{12}Ja, ich will das Land Ägypten zu einer Wüste machen
inmitten verwüsteter Länder, und seine Städte sollen inmitten verödeter
Städte vierzig Jahre lang wüst daliegen; und die Ägypter werde ich unter
die Völker zerstreuen und in die Länder versprengen.‹«

\hypertarget{cc-uxe4gyptens-dereinstige-wiederherstellung}{%
\subparagraph{cc) Ägyptens dereinstige
Wiederherstellung}\label{cc-uxe4gyptens-dereinstige-wiederherstellung}}

\bibleverse{13}Doch so hat Gott der HERR gesprochen: »Nach Ablauf der
vierzig Jahre will ich die Ägypter aus den Völkern, unter die sie
versprengt waren, wieder sammeln \bibleverse{14}und das Schicksal der
Ägypter wenden und sie nach Oberägypten, in ihr Geburtsland,
zurückbringen; dort werden sie dann ein bescheidenes Königreich bilden.
\bibleverse{15}Dieses soll weniger mächtig sein als die anderen
Königreiche und sich fernerhin nicht mehr über die anderen Völker
erheben; und ich will sie wenig zahlreich werden lassen, so daß sie
nicht mehr über die anderen Völker herrschen sollen. \bibleverse{16}Dann
wird dieses Reich für das Haus Israel nicht mehr den Gegenstand des
Vertrauens bilden, was mich an ihre Verschuldung erinnern würde, wenn
sie sich nach ihnen hinwendeten\textless sup title=``=~sich an sie
anschlössen''\textgreater✲; und sie werden erkennen, daß ich Gott der
HERR bin.«

\hypertarget{dd-nachtrag-uxe4gypten-zum-ersatzlohn-fuxfcr-nebukadnezars-vergebliche-belagerung-von-tyrus-bestimmt}{%
\subparagraph{dd) Nachtrag: Ägypten zum Ersatzlohn für Nebukadnezars
vergebliche Belagerung von Tyrus
bestimmt}\label{dd-nachtrag-uxe4gypten-zum-ersatzlohn-fuxfcr-nebukadnezars-vergebliche-belagerung-von-tyrus-bestimmt}}

\bibleverse{17}Hierauf begab es sich im siebenundzwanzigsten Jahre, am
ersten Tage des ersten Monats, da erging das Wort des HERRN an mich
folgendermaßen: \bibleverse{18}»Menschensohn, Nebukadnezar, der König
von Babylon, hat sein Heer schwere Arbeit verrichten lassen gegen Tyrus,
so daß allen die Köpfe kahl geworden und die Schultern\textless sup
title=``oder: Rücken''\textgreater✲ allen wund gerieben sind, aber Lohn
ist weder ihm noch seinem Heere von Tyrus zuteil geworden für die
Arbeit, die er gegen die Stadt\textless sup title=``oder: um der Stadt
willen''\textgreater✲ geleistet hat.« \bibleverse{19}Darum hat Gott der
HERR so gesprochen: »Ich will nunmehr Nebukadnezar, dem Könige von
Babylon, das Land Ägypten geben, damit er sich dessen Reichtum aneignet
und es ausraubt und ausplündert: das soll seinem Heere als Lohn zuteil
werden. \bibleverse{20}Als seinen Sold, um den er sich abgemüht hat,
gebe ich ihm das Land Ägypten, weil sie für mich gearbeitet haben« -- so
lautet der Ausspruch Gottes des HERRN. \bibleverse{21}»An jenem Tage
will ich das Haus Israel zu neuer Macht erwachsen lassen und dir
gewähren, den Mund in ihrer Mitte wieder frei aufzutun, damit sie
erkennen, daß ich der HERR bin.«

\hypertarget{b-neue-drohworte-uxfcber-das-den-uxe4gyptern-bevorstehende-guxf6ttliche-strafgericht}{%
\paragraph{b) Neue Drohworte über das den Ägyptern bevorstehende
göttliche
Strafgericht}\label{b-neue-drohworte-uxfcber-das-den-uxe4gyptern-bevorstehende-guxf6ttliche-strafgericht}}

\hypertarget{section-29}{%
\section{30}\label{section-29}}

\bibleverse{1}Weiter erging das Wort des HERRN an mich folgendermaßen:
\bibleverse{2}»Menschensohn, verkünde folgende Weissagungen: ›So hat
Gott der HERR gesprochen: Wehklagt! O welch ein Tag! \bibleverse{3}Denn
nahe ist der Tag, ja, nahe ist der Tag des HERRN, ein dunkelbewölkter
Tag: die Endzeit\textless sup title=``oder: Gerichtszeit''\textgreater✲
für die Heidenvölker wird er sein! \bibleverse{4}Da wird ein Schwert
nach\textless sup title=``oder: an''\textgreater✲ Ägypten kommen und in
Äthiopien große Angst herrschen, wenn Durchbohrte in Ägypten hinsinken
und man seinen Reichtum wegschleppt und seine Grundfesten eingerissen
werden. \bibleverse{5}Die Äthiopier, Put und Lud\textless sup
title=``=~Libyer und Lyder''\textgreater✲ samt all dem Völkergemisch und
Kub samt den Bewohnern der verbündeten Länder werden mit ihnen durch das
Schwert fallen.‹«

\bibleverse{6}So hat der HERR gesprochen: »Da werden dann die Stützen
Ägyptens fallen und seine stolze Pracht dahinsinken; von Migdol bis nach
Syene✲ werden sie im Lande durch das Schwert fallen!« -- so lautet der
Ausspruch Gottes des HERRN. \bibleverse{7}»Ihr Land soll verwüstet
daliegen inmitten verwüsteter Länder und seine Städte zerstört sein
inmitten zerstörter Städte. \bibleverse{8}Da werden sie denn erkennen,
daß ich der HERR bin, wenn ich Feuer an Ägypten lege und alle, die ihnen
helfen, zerschmettert werden. \bibleverse{9}An jenem Tage werden Boten
von mir ausfahren auf Schiffen, um Äthiopien aus seiner Sicherheit
aufzuschrecken, und große Angst wird unter ihnen herrschen wegen des
Unglückstages Ägyptens; denn dieser kommt unfehlbar!«

\hypertarget{das-unheil-kommt-fuxfcr-uxe4gypten-durch-nebukadnezar}{%
\paragraph{Das Unheil kommt für Ägypten durch
Nebukadnezar}\label{das-unheil-kommt-fuxfcr-uxe4gypten-durch-nebukadnezar}}

\bibleverse{10}So hat Gott der HERR gesprochen: »So will ich denn dem
Gepränge\textless sup title=``oder: der Volksmenge''\textgreater✲
Ägyptens ein Ende machen durch die Hand Nebukadnezars, des Königs von
Babylon. \bibleverse{11}Er und sein Kriegsvolk mit ihm, die wildesten
der Heidenvölker, werden herbeigeholt werden, um das Land zu verheeren;
sie werden ihre Schwerter gegen Ägypten zücken und das Land mit
Erschlagenen füllen. \bibleverse{12}Und ich will die Ströme✲ trocken
legen und das Land der Gewalt von Bösewichtern preisgeben und das Land
samt allem, was darin ist, durch die Hand von Fremden verwüsten: ich,
der HERR, habe es gesagt!«

\hypertarget{das-gericht-uxfcber-die-wichtigsten-stuxe4dte-uxe4gyptens}{%
\paragraph{Das Gericht über die wichtigsten Städte
Ägyptens}\label{das-gericht-uxfcber-die-wichtigsten-stuxe4dte-uxe4gyptens}}

\bibleverse{13}So hat Gott der HERR gesprochen: »Ja, ich will die Götzen
vernichten und den falschen Göttern in Memphis ein Ende machen; es soll
künftig auch keine Fürsten mehr im Lande Ägypten geben, und ich will das
Land Ägypten in Furcht versetzen. \bibleverse{14}Ich will Oberägypten
verwüsten und Feuer an Zoan\textless sup title=``=~Tanis; Jes
19,11''\textgreater✲ legen und Strafgerichte an Theben vollstrecken;
\bibleverse{15}ich will meinen Zorn an Pelusium, dem Bollwerk Ägyptens,
auslassen und das Gepränge✲ von Theben vernichten; \bibleverse{16}und
ich will Feuer an Ägypten legen: Pelusium soll zittern und beben, Theben
wird erstürmt werden und Memphis Feinde am hellen Tage sehen.
\bibleverse{17}Die jungen Krieger von Heliopolis und Bubastis werden
durch das Schwert fallen, die übrigen Bewohner aber in die
Gefangenschaft wandern. \bibleverse{18}Und in Daphne\textless sup
title=``vgl. Jer 43,7.9''\textgreater✲ soll sich der Tag in Finsternis
verwandeln, wenn ich daselbst die Herrscherstäbe✲ Ägyptens zerbreche und
seiner stolzen Pracht dort ein Ende gemacht wird; es selbst -- Gewölk
wird es umhüllen, und seine Tochterstädte müssen in die
Verbannung\textless sup title=``oder: Gefangenschaft''\textgreater✲
wandern. \bibleverse{19}So werde ich Strafgerichte an Ägypten
vollziehen, damit sie erkennen, daß ich der HERR bin!«

\hypertarget{c-die-zerschmetterung-des-armes-des-pharaos}{%
\paragraph{c) Die Zerschmetterung des Armes des
Pharaos}\label{c-die-zerschmetterung-des-armes-des-pharaos}}

\bibleverse{20}Im elften Jahre, am siebten Tage des ersten Monats,
erging das Wort des HERRN an mich folgendermaßen:
\bibleverse{21}»Menschensohn, den (einen) Arm des Pharaos, des Königs
von Ägypten, habe ich zerbrochen; und siehe, er ist nicht verbunden
worden, daß man Heilmittel angewandt, daß man eine Binde als Verband
angelegt hätte, damit er wieder stark genug würde, das Schwert zu
führen.«

\hypertarget{drohung-gegen-den-pharao-und-gegen-uxe4gypten}{%
\paragraph{Drohung gegen den Pharao und gegen
Ägypten}\label{drohung-gegen-den-pharao-und-gegen-uxe4gypten}}

\bibleverse{22}Darum hat Gott der HERR so gesprochen: »Nunmehr will ich
an den Pharao\textless sup title=``=~gegen den Pharao
vorgehen''\textgreater✲, den König von Ägypten, und will ihm beide Arme
zerschmettern, den gesunden und den zerbrochenen, und ihm das Schwert
aus der Hand schlagen; \bibleverse{23}dann will ich die Ägypter unter
die Völker zerstreuen und sie in die Länder versprengen.
\bibleverse{24}Dagegen will ich dem König von Babylon die Arme stärken
und ihm mein Schwert in die Hand geben; aber die Arme des Pharaos will
ich zerbrechen, daß er vor ihm ächzen soll wie ein tödlich Verwundeter.
\bibleverse{25}Ja, die Arme des Königs von Babylon will ich stärken,
während die Arme des Pharaos herabsinken sollen, damit man erkennt, daß
ich der HERR bin, wenn ich dem König von Babylon mein Schwert in die
Hand gebe, damit er es gegen das Land Ägypten schwinge.
\bibleverse{26}Alsdann werde ich die Ägypter unter die Völker zerstreuen
und sie in die Länder versprengen, damit sie erkennen, daß ich der HERR
bin.«

\hypertarget{d-stolze-huxf6he-und-juxe4her-sturz-des-mit-einer-prachtzeder-verglichenen-pharaos}{%
\paragraph{d) Stolze Höhe und jäher Sturz des mit einer Prachtzeder
verglichenen
Pharaos}\label{d-stolze-huxf6he-und-juxe4her-sturz-des-mit-einer-prachtzeder-verglichenen-pharaos}}

\hypertarget{section-30}{%
\section{31}\label{section-30}}

\bibleverse{1}Im elften Jahre, am ersten Tage des dritten Monats, erging
das Wort des HERRN an mich folgendermaßen: \bibleverse{2}»Menschensohn,
richte an den Pharao, den König von Ägypten, und an sein
Gepränge\textless sup title=``vgl. 30,10''\textgreater✲ folgende Worte.«

\hypertarget{aa-die-unvergleichliche-pracht-der-zeder}{%
\subparagraph{aa) Die unvergleichliche Pracht der
Zeder}\label{aa-die-unvergleichliche-pracht-der-zeder}}

\bibleverse{3}»›Wem glichest du in deiner Größe? Ja, du warst einer
Edeltanne gleich, einer Zeder auf dem Libanon, die schön von Geäst und
mit beschattendem Laubwerk und hoch an Wuchs war, so daß ihr Wipfel bis
in die Wolken hineinragte. \bibleverse{4}Das Wasser hatte sie groß
wachsen lassen, die unterirdische Flut sie in die Höhe getrieben; denn
deren Strömung ging rings um ihren Standort herum, während sie sonst nur
Rinnsale an alle Bäume des Gefildes gelangen ließ. \bibleverse{5}Darum
ragte ihr Wuchs über alle Bäume des Gefildes empor, und ihre Zweige
wurden zahlreich und ihre Äste lang von der reichlichen Bewässerung,
indem sie sich ausbreitete. \bibleverse{6}In ihren Zweigen nisteten alle
Vögel des Himmels, unter ihrem Laubdach warfen alle wilden Tiere ihre
Jungen, und in ihrem Schatten wohnten all die vielen Völker.
\bibleverse{7}Schön war sie in ihrem hohen Wuchs, durch die Länge ihrer
Zweige; denn ihre Wurzeln lagen an reichlichem Wasser.
\bibleverse{8}Keine Zeder im Garten Gottes reichte an sie heran, die
Zypressen kamen ihr nicht gleich mit ihren Zweigen, und die Platanen
hatten nicht solche Äste wie sie: kein Baum im Garten Gottes konnte sich
an Schönheit mit ihr vergleichen. \bibleverse{9}Ich hatte sie durch die
Menge ihrer Zweige so schön gemacht, daß alle Bäume Edens im Garten
Gottes neidisch auf sie waren.‹«

\hypertarget{bb-der-juxe4he-sturz-der-zeder-und-gottes-absicht-dabei}{%
\subparagraph{bb) Der jähe Sturz der Zeder und Gottes Absicht
dabei}\label{bb-der-juxe4he-sturz-der-zeder-und-gottes-absicht-dabei}}

\bibleverse{10}Darum hat Gott der HERR so gesprochen: »Weil sie so
hochragend an Wuchs geworden war und ihren Wipfel bis in die Wolken
hatte hineinragen lassen und ihr Sinn infolge ihres hohen Wuches
hochfahrend geworden war, \bibleverse{11}so habe ich sie der Gewalt
eines Mächtigen unter den Völkern preisgegeben, der mit ihr nach ihrer
Bosheit verfahren soll: infolge ihrer Verfehlung habe ich sie
verstoßen.« \bibleverse{12}Da haben Fremde sie umgehauen, die wildesten
unter den Völkern, und haben sie hingeworfen; auf die Berge und in alle
Täler sind ihre Zweige gefallen: ihre Äste lagen zerbrochen in allen
Flußtälern des Landes, und alle Völker der Erde zogen aus ihrem Schatten
hinweg und ließen sie unbeachtet liegen. \bibleverse{13}Auf ihren
umgefallenen Stamm setzten sich alle Vögel des Himmels nieder, und an
ihre Zweige machten sich alle Tiere des Feldes, \bibleverse{14}damit
keine Bäume am Wasser sich forthin wegen ihres hohen Wuchses überheben
möchten und ihren Wipfel bis in die Wolken hineinragen ließen und die
Gewaltigen unter ihnen sich nicht stolz hinstellten in ihrer Hoheit,
alle, die vom Wasser getränkt werden; denn sie sind alle dem Tode
geweiht und müssen in das unterirdische Land hinab, mitten unter die
anderen Menschenkinder, zu denen hin, die in die Grube hinabgefahren
sind.

\hypertarget{cc-die-wirkung-und-bedeutung-dieses-sturzes}{%
\subparagraph{cc) Die Wirkung und Bedeutung dieses
Sturzes}\label{cc-die-wirkung-und-bedeutung-dieses-sturzes}}

\bibleverse{15}So hat Gott der HERR gesprochen: »An dem Tage, als sie in
das Totenreich hinabfuhr, ließ ich die unterirdische Flut sich in Trauer
um sie hüllen und hielt ihre Strömung zurück, so daß die reichlichen
Wasser gehemmt wurden; den Libanon hüllte ich ihretwegen in ein
Trauergewand, und alle Bäume des Gefildes mußten ihretwegen
verschmachten. \bibleverse{16}Durch das Gedröhn ihres Sturzes machte ich
die Völker erzittern, als ich sie in das Totenreich hinabsinken ließ zu
den in die Grube Hinabgefahrenen; da trösteten sich im unterirdischen
Lande alle Bäume Edens, die erlesensten und schönsten auf dem Libanon,
alle, die vom Wasser getränkt wurden. \bibleverse{17}Auch sie mußten mit
ihr in das Totenreich hinabfahren zu den vom Schwert Durchbohrten, die
vordem als ihre Helfer in ihrem Schatten gewohnt hatten inmitten der
Völker.

\bibleverse{18}Wem glichest du also an Herrlichkeit und Größe unter den
Bäumen Edens? Und doch wirst du mit den Bäumen Edens in das
unterirdische Land hinabgestoßen werden; inmitten Unbeschnittener wirst
du da bei den vom Schwert Erschlagenen liegen: das ist der Pharao und
all sein Gepränge!«\textless sup title=``vgl. 30,10''\textgreater✲ -- so
lautet der Ausspruch Gottes des HERRN.

\hypertarget{e-klagelied-auf-den-pharao}{%
\paragraph{e) Klagelied auf den
Pharao}\label{e-klagelied-auf-den-pharao}}

\hypertarget{section-31}{%
\section{32}\label{section-31}}

\bibleverse{1}Im zwölften\textless sup title=``oder:
elften''\textgreater✲ Jahre, am ersten Tage des zwölften Monats, erging
das Wort des HERRN an mich folgendermaßen: \bibleverse{2}»Menschensohn,
stimme ein Klagelied an über\textless sup title=``oder:
auf''\textgreater✲ den Pharao, den König von Ägypten, und sage zu ihm«:

\hypertarget{aa-bildlicher-teil-die-schmachvolle-vernichtung-des-grouxdfen-krokodils}{%
\subparagraph{aa) Bildlicher Teil: die schmachvolle Vernichtung des
großen
Krokodils}\label{aa-bildlicher-teil-die-schmachvolle-vernichtung-des-grouxdfen-krokodils}}

»›Einem jungen Löwen unter den Völkern wurdest du
verglichen\textless sup title=``oder: dünktest du dich
gleich''\textgreater✲, und warst doch nur wie ein Krokodil im Nilstrom,
spritztest mit deinen Nüstern, trübtest das Wasser mit deinen Füßen und
wühltest seine Fluten auf.‹ \bibleverse{3}So hat Gott der HERR
gesprochen: ›So will ich denn mein Netz über dich ausbreiten durch eine
Schar\textless sup title=``oder: Versammlung''\textgreater✲ vieler
Völker, die sollen dich in meinem Fanggarn emporziehen.
\bibleverse{4}Dann will ich dich auf die Erde hinwerfen, dich auf das
freie Feld schleudern und will machen, daß alle Vögel des Himmels sich
auf dich niedersetzen und die wilden Tiere der ganzen Erde sich an dir
sättigen. \bibleverse{5}Dann will ich dein Fleisch auf die Berge
verschleppen lassen und die Täler mit deinem Aas füllen
\bibleverse{6}und das Land mit deinem Ausfluß tränken von deinem Blut
bis an die Berge, und die Rinnsale✲ sollen voll von dir werden.
\bibleverse{7}Alsdann will ich, wenn ich dich erlöschen lasse, den
Himmel verschleiern und seine Sterne verdunkeln, will die Sonne in
Gewölk hüllen, und der Mond soll sein Licht nicht leuchten lassen.
\bibleverse{8}Alle leuchtenden Himmelslichter will ich deinethalben
verdunkeln und Finsternis über dein Land ausbreiten‹ -- so lautet der
Ausspruch Gottes des HERRN. \bibleverse{9}›Da werde ich denn das Herz
vieler Völker in Betrübnis versetzen, wenn ich die Kunde von deinem
Untergang unter die Nationen gelangen lasse, in Länder, die du nicht
gekannt hast; \bibleverse{10}und ich werde machen, daß viele Völker sich
über dich entsetzen und ihre Könige über dein Geschick schaudern, wenn
ich mein Schwert vor ihren Augen schwinge; und sie werden unaufhörlich
zittern, ein jeder für sein Leben, am Tage deines Sturzes.‹«

\hypertarget{bb-eigentliche-rede-die-vernichtung-uxe4gyptens-durch-den-kuxf6nig-von-babylon}{%
\subparagraph{bb) Eigentliche Rede: die Vernichtung Ägyptens durch den
König von
Babylon}\label{bb-eigentliche-rede-die-vernichtung-uxe4gyptens-durch-den-kuxf6nig-von-babylon}}

\bibleverse{11}»Denn so hat Gott der HERR gesprochen: ›Das Schwert des
Königs von Babylon soll über dich kommen! \bibleverse{12}Durch die
Schwerter tapferer Krieger will ich dein Gepränge✲ zu Fall bringen --
die wildesten unter den Völkern sind sie allesamt --: die sollen die
Herrlichkeit Ägyptens zerstören, daß all sein Gepränge✲ vernichtet wird;
\bibleverse{13}auch seine gesamte Tierwelt will ich zugrunde gehen
lassen von den vielen Gewässern hinweg, so daß keines Menschen Fuß sie
noch aufwühlen und keines Tieres Klaue sie noch trüben soll.
\bibleverse{14}Alsdann will ich ihre Gewässer klären und ihre Ströme wie
Öl dahingleiten lassen‹ -- so lautet der Ausspruch Gottes, des HERRN --,
\bibleverse{15}›wenn ich das Land Ägypten zu einer Wüste mache und das
Land öde wird, seiner Fülle beraubt, indem ich alle seine Bewohner
sterben lasse, damit sie erkennen, daß ich der HERR bin.‹

\bibleverse{16}Ein Klagelied ist dies, das man klagend singen soll; die
Töchter der Völker sollen es klagend singen; über Ägypten und all sein
Gepränge✲ sollen sie es klagend singen!« -- so lautet der Ausspruch
Gottes des HERRN.

\hypertarget{f-grabgesang-fuxfcr-den-pharao}{%
\paragraph{f) Grabgesang für den
Pharao}\label{f-grabgesang-fuxfcr-den-pharao}}

\bibleverse{17}Und im zwölften Jahre, am fünfzehnten Tage des (zwölften
oder: ersten) Monats, erging das Wort des HERRN an mich folgendermaßen:
\bibleverse{18}»Menschensohn, stimme eine Totenklage an über das
Gepränge\textless sup title=``vgl. 30,10''\textgreater✲ Ägyptens und
senke es hinab, du und die Töchter mächtiger Völker, in das
unterirdische Land zu den in die Grube Hinabgefahrenen!«

\hypertarget{aa-der-pharao-zur-unseligkeit-verurteilt-sein-empfang-von-seiten-der-bewohner-der-unterwelt}{%
\subparagraph{aa) Der Pharao zur Unseligkeit verurteilt; sein Empfang
von seiten der Bewohner der
Unterwelt}\label{aa-der-pharao-zur-unseligkeit-verurteilt-sein-empfang-von-seiten-der-bewohner-der-unterwelt}}

\bibleverse{19}»Vor wem hast du etwas voraus an
Lieblichkeit\textless sup title=``oder: Glück''\textgreater✲? Fahre
hinab und laß dich bei den Unbeschnittenen betten! \bibleverse{20}Mitten
unter den vom Schwert Erschlagenen sollen sie hinsinken! Das Schwert ist
schon dargereicht: schleppt Ägypten herbei und sein ganzes Gepränge!
\bibleverse{21}Da werden die Vornehmsten unter den tapferen Kriegern
mitten aus dem Totenreich heraus ihm samt seinen Helfern zurufen: Sie
sind herabgekommen, sie liegen da, die Unbeschnittenen, vom Schwert
erschlagen!‹«

\hypertarget{bb-der-pharao-in-der-unterwelt-inmitten-der-unseligen-der-unbeschnittenen-und-vom-schwert-erschlagenen}{%
\subparagraph{bb) Der Pharao in der Unterwelt inmitten der Unseligen
(der Unbeschnittenen und vom Schwert
Erschlagenen)}\label{bb-der-pharao-in-der-unterwelt-inmitten-der-unseligen-der-unbeschnittenen-und-vom-schwert-erschlagenen}}

\bibleverse{22}»›Dort ist Assur und sein ganzes Kriegsvolk; rings um ihn
her liegen ihre Gräber: alle sind sie erschlagen, durch das Schwert
gefallen. \bibleverse{23}Seine Gräber sind im tiefsten
Grunde\textless sup title=``oder: am äußersten Ende''\textgreater✲ der
Grube untergebracht, und sein Kriegsvolk liegt rings um sein Grab herum;
sie sind alle erschlagen, durch das Schwert gefallen, sie, die einstmals
Schrecken verbreitet haben im Lande der Lebenden.

\bibleverse{24}Dort ist Elam und alle seine Krieger rings um sein Grab
her, lauter Erschlagene, durch das Schwert Gefallene, die als
Unbeschnittene in das unterirdische Land hinabgefahren sind, sie, die
einstmals Schrecken vor sich her im Lande der Lebenden verbreitet haben
und jetzt ihre Schmach tragen müssen bei den in die Grube
Hinabgefahrenen. \bibleverse{25}Mitten unter Erschlagenen hat man
ihm\textless sup title=``d.h. Elam''\textgreater✲ ein Lager angewiesen
samt all seinen Kriegern, deren Gräber rings um ihn her liegen. Sie sind
allesamt unbeschnitten, vom Schwert erschlagen, weil einstmals der
Schrecken vor ihnen im Lande der Lebenden verbreitet war. So tragen sie
nun ihre Schmach bei den in die Grube Hinabgefahrenen: mitten unter
Erschlagene hat man sie hingelegt.

\bibleverse{26}Dort sind Mesech und Thubal und alle ihre Krieger, deren
Gräber rings um sie her liegen; sie sind allesamt unbeschnitten, vom
Schwert erschlagen, sie, die einstmals Schrecken vor sich her im Lande
der Lebenden verbreitet haben. \bibleverse{27}Sie liegen nicht bei den
in der Vorzeit gefallenen Helden, die in ihrer vollen Kriegswehr in das
Totenreich hinabgefahren sind, denen man ihre Schwerter unter das Haupt
und ihre Schilde auf die Gebeine gelegt hat, weil einstmals ein
Schrecken vor ihrer Heldenkraft im Lande der Lebenden geherrscht hat.
\bibleverse{28}So wirst auch du (Ägypten) inmitten Unbeschnittener
zerschmettert bei den vom Schwert Erschlagenen gebettet sein.

\bibleverse{29}Dort ist Edom mit seinen Königen und all seinen Fürsten,
die trotz ihres Heldentums den vom Schwert Erschlagenen beigesellt
worden sind: sie müssen bei Unbeschnittenen liegen, bei den in die Grube
Hinabgefahrenen.

\bibleverse{30}Dort sind die Herrscher des Nordens insgesamt und alle
Sidonier, die als Erschlagene hinabgefahren und trotz ihrer
Furchtbarkeit, trotz ihres Heldentums zuschanden geworden sind. Da
liegen sie nun als Unbeschnittene bei den vom Schwert Erschlagenen und
müssen ihre Schmach tragen mit den in die Grube Hinabgefahrenen.‹«

\hypertarget{cc-der-pharao-von-gott-in-die-unterwelt-verstouxdfen}{%
\subparagraph{cc) Der Pharao von Gott in die Unterwelt
verstoßen}\label{cc-der-pharao-von-gott-in-die-unterwelt-verstouxdfen}}

\bibleverse{31}»Wenn der Pharao diese alle erblickt, mag er sich über
all seine Gepränge\textless sup title=``vgl. 30,10''\textgreater✲
trösten; vom Schwert erschlagen ist der Pharao samt seiner ganzen
Heeresmacht« -- so lautet der Ausspruch Gottes des HERRN --,
\bibleverse{32}»denn er hat Schrecken vor sich her verbreitet im Lande
der Lebenden; darum wird er inmitten Unbeschnittener bei den vom Schwert
Erschlagenen gebettet liegen, der Pharao und all sein Gepränge!« -- so
lautet der Ausspruch Gottes des HERRN.

\hypertarget{c.-dritter-hauptteil-trostbuch-oder-reden-uxfcber-jerusalems-wiederherstellung-kap.-33-48}{%
\subsection{C. Dritter Hauptteil: Trostbuch oder Reden über Jerusalems
Wiederherstellung (Kap.
33-48)}\label{c.-dritter-hauptteil-trostbuch-oder-reden-uxfcber-jerusalems-wiederherstellung-kap.-33-48}}

\hypertarget{i.-die-vorbereitungen-auf-die-heilszeit-kap.-33-39}{%
\subsection{I. Die Vorbereitungen auf die Heilszeit (Kap.
33-39)}\label{i.-die-vorbereitungen-auf-die-heilszeit-kap.-33-39}}

\hypertarget{einleitung-zum-trostbuch}{%
\subsubsection{1. Einleitung zum
Trostbuch}\label{einleitung-zum-trostbuch}}

\hypertarget{a-die-wende-in-der-prophetischen-tuxe4tigkeit-hesekiels}{%
\paragraph{a) Die Wende in der prophetischen Tätigkeit
Hesekiels}\label{a-die-wende-in-der-prophetischen-tuxe4tigkeit-hesekiels}}

\hypertarget{aa-bedeutung-des-wuxe4chteramts-fuxfcr-den-wuxe4chter-und-fuxfcr-die-landesbewohner}{%
\subparagraph{aa) Bedeutung des Wächteramts für den Wächter und für die
Landesbewohner}\label{aa-bedeutung-des-wuxe4chteramts-fuxfcr-den-wuxe4chter-und-fuxfcr-die-landesbewohner}}

\hypertarget{section-32}{%
\section{33}\label{section-32}}

\bibleverse{1}Das Wort des HERRN erging an mich folgendermaßen:
\bibleverse{2}»Menschensohn, rede zu deinen Volksgenossen und sage zu
ihnen: Wenn ich das Schwert✲ über ein Land kommen lasse und das Volk des
Landes einen Mann aus seiner Gesamtheit wählt und ihn für sich zum
Wächter bestellt, \bibleverse{3}und der sieht das Schwert\textless sup
title=``=~den bewaffneten Feind''\textgreater✲ in das Land einbrechen
und stößt in die Trompete und warnt dadurch das Volk~--
\bibleverse{4}wenn dann einer zwar den Schall der Trompete hört, aber
sich nicht warnen läßt, so daß der bewaffnete Feind kommt und ihn ums
Leben bringt, so soll die Schuld an seinem Tode ihm selbst beigemessen
werden; \bibleverse{5}er hat ja den Schall der Trompete gehört, aber
sich nicht warnen lassen: er hat seinen Tod selbst verschuldet; denn
hätte er sich warnen lassen, so würde er sein Leben gerettet haben.
\bibleverse{6}Wenn aber der Wächter den bewaffneten Feind kommen sieht
und nicht in die Trompete stößt, so daß das Volk ungewarnt bleibt, und
der bewaffnete Feind kommt und bringt einen von ihnen ums Leben, so wird
der Betreffende zwar infolge seiner Sündenschuld weggerafft, aber für
den Verlust seines Lebens werde ich den Wächter verantwortlich machen.«

\hypertarget{bb-hesekiels-berufung-zum-wuxe4chteramt-seine-verantwortlichkeit}{%
\subparagraph{bb) Hesekiels Berufung zum Wächteramt; seine
Verantwortlichkeit}\label{bb-hesekiels-berufung-zum-wuxe4chteramt-seine-verantwortlichkeit}}

\bibleverse{7}»Du nun, Menschensohn -- dich habe ich zum Wächter für das
Haus Israel bestellt, damit du sie, wenn du ein Wort aus meinem Munde
vernommen hast, in meinem Namen warnst. \bibleverse{8}Wenn ich zu dem
Gottlosen sage: ›Gottloser, du mußt des Todes sterben!‹, du aber nichts
sagst, um den Gottlosen vor seinem bösen Wandel zu warnen, so wird er,
der Gottlose, zwar sein Leben um seiner Verschuldung willen verlieren,
aber für den Verlust seines Lebens werde ich dich verantwortlich machen.
\bibleverse{9}Wenn du aber deinerseits den Gottlosen vor seinem bösen
Wandel gewarnt hast, damit er von ihm umkehre, er sich aber von seinem
Wandel nicht abbringen läßt, so wird er zwar um seiner Verschuldung
willen sterben, du aber hast dein Leben gerettet.«

\hypertarget{cc-hesekiels-buuxdfpredigt-auf-gottes-befehl-und-seine-verkuxfcndigung-der-guxf6ttlichen-gnade-und-gerechtigkeit-fuxfcr-die-buuxdffertigen-suxfcnder}{%
\subparagraph{cc) Hesekiels Bußpredigt auf Gottes Befehl und seine
Verkündigung der göttlichen Gnade und Gerechtigkeit für die bußfertigen
Sünder}\label{cc-hesekiels-buuxdfpredigt-auf-gottes-befehl-und-seine-verkuxfcndigung-der-guxf6ttlichen-gnade-und-gerechtigkeit-fuxfcr-die-buuxdffertigen-suxfcnder}}

\bibleverse{10}»Und du nun, Menschensohn, sage zum Hause Israel:
›Folgendes Bekenntnis habt ihr abgelegt: Ja, unsere Übertretungen und
Sünden lasten auf uns, und durch sie vergehen wir ganz: wie könnten wir
denn am Leben bleiben?‹ \bibleverse{11}Sage zu ihnen: ›So wahr ich
lebe!‹ -- so lautet der Ausspruch Gottes des HERRN --: ›ich habe kein
Wohlgefallen am Tode des Gottlosen, sondern daran, daß der Gottlose sich
von seinem Wandel bekehrt und am Leben bleibt! Kehrt um, ja bekehrt euch
von eurem bösen Wandel! Denn warum wollt ihr sterben, Haus Israel?‹~--
\bibleverse{12}Du also, Menschensohn, sage zu deinen Volksgenossen: ›Den
Gerechten wird seine Gerechtigkeit nicht retten an dem Tage, wo er in
Sünde verfällt; und den Gottlosen wird seine Gottlosigkeit nicht zu Fall
bringen an dem Tage, wo er von seiner Gottlosigkeit umkehrt; aber auch
der Gerechte kann um seiner Gerechtigkeit willen nicht am Leben erhalten
bleiben an dem Tage, wo er in Sünde verfällt. \bibleverse{13}Wenn ich
dem Gerechten verheiße, er solle ganz gewiß das Leben behalten, und er
sich auf seine (bisherige) Gerechtigkeit verläßt und Böses tut, so wird
seines ganzen gerechten Tuns nicht mehr gedacht werden, sondern um des
Bösen willen, das er verübt hat, um deswillen muß er sterben.
\bibleverse{14}Und wenn ich dem Gottlosen androhe: ›Du mußt des Todes
sterben!‹ und er sich von seiner Sünde abkehrt und nunmehr Recht und
Gerechtigkeit übt, so daß er das ihm Verpfändete zurückgibt, Geraubtes
wiedererstattet und nach den Satzungen wandelt, deren Beobachtung zum
Leben führt, so daß er nichts Böses mehr tut, so soll er gewißlich das
Leben behalten und nicht sterben: \bibleverse{15} \bibleverse{16}keine
von allen Sünden, die er begangen hat, soll ihm noch angerechnet werden;
Recht und Gerechtigkeit hat er geübt: er soll gewißlich das Leben
behalten!‹

\bibleverse{17}Freilich sagen deine Volksgenossen: ›Das Verfahren des
Herrn ist nicht das richtige!‹, während doch ihr eigenes Verfahren nicht
das richtige ist. \bibleverse{18}Wenn ein Gerechter sich von seiner
Gerechtigkeit abwendet und Unrecht tut, so muß er auf Grund davon
sterben; \bibleverse{19}wenn dagegen ein Gottloser von seiner
Gottlosigkeit abläßt und Recht und Gerechtigkeit übt, so soll er
infolgedessen am Leben bleiben. \bibleverse{20}Und ob ihr auch
behauptet, das Verfahren des Herrn sei nicht das richtige, so werde ich
doch jeden von euch nach seinem Wandel richten, Haus Israel!«

\hypertarget{dd-eintritt-der-wende-fuxfcr-hesekiel-beim-eintreffen-der-kunde-von-der-eroberung-jerusalems-durch-luxf6sung-seiner-zunge}{%
\subparagraph{dd) Eintritt der Wende für Hesekiel (beim Eintreffen der
Kunde von der Eroberung Jerusalems) durch Lösung seiner
Zunge}\label{dd-eintritt-der-wende-fuxfcr-hesekiel-beim-eintreffen-der-kunde-von-der-eroberung-jerusalems-durch-luxf6sung-seiner-zunge}}

\bibleverse{21}Es begab sich aber im zwölften\textless sup title=``oder:
elften''\textgreater✲ Jahre unserer Verbannung\textless sup
title=``oder: Gefangenschaft''\textgreater✲, am fünften Tage des zehnten
Monats, da kam ein Flüchtling aus Jerusalem zu mir mit der Nachricht:
»Die Stadt ist erobert!« \bibleverse{22}Die Hand des HERRN war aber
schon am Abend vor der Ankunft des Flüchtlings über mich gekommen, und
er hatte mir den Mund aufgetan\textless sup title=``=~die Sprache
zurückgegeben''\textgreater✲, ehe jener am folgenden Morgen bei mir
eintraf. So war mir denn der Mund aufgetan worden, und ich bin seitdem
nie wieder stumm geworden\textless sup title=``vgl.
24,25-27''\textgreater✲.

\hypertarget{b-strafrede-auf-die-anspruchsvolle-uxfcberhebung-der-in-paluxe4stina-zuruxfcckgebliebenen-volksgenossen}{%
\paragraph{b) Strafrede auf die anspruchsvolle Überhebung der in
Palästina zurückgebliebenen
Volksgenossen}\label{b-strafrede-auf-die-anspruchsvolle-uxfcberhebung-der-in-paluxe4stina-zuruxfcckgebliebenen-volksgenossen}}

\hypertarget{aa-darlegung-des-sachverhalts}{%
\subparagraph{aa) Darlegung des
Sachverhalts}\label{aa-darlegung-des-sachverhalts}}

\bibleverse{23}Hierauf erging das Wort des HERRN an mich folgendermaßen:
\bibleverse{24}»Menschensohn, die Bewohner jener Trümmerstätten im Lande
Israel sagen immer wieder: ›Abraham war nur ein einzelner Mann und hat
doch das Land zum Besitz erhalten; unser aber sind viele: uns ist das
Land als Besitz zugewiesen!‹ \bibleverse{25}Darum sage zu ihnen: ›So hat
Gott der HERR gesprochen: Ihr genießt das Fleisch mitsamt dem Blut und
erhebt eure Augen zu euren Götzen und vergießt Blut, und da solltet ihr
das Land zum Besitz haben? \bibleverse{26}Ihr verlaßt euch fest auf euer
Schwert, verübt Greuel und entehrt ein jeder das Weib des andern, und da
solltet ihr das Land im Besitz haben?‹«

\hypertarget{bb-androhung-des-strafgerichts}{%
\subparagraph{bb) Androhung des
Strafgerichts}\label{bb-androhung-des-strafgerichts}}

\bibleverse{27}»Folgendermaßen sollst du zu ihnen sagen: ›So hat Gott
der HERR gesprochen: So wahr ich lebe: die in den Trümmerstätten
Wohnenden sollen durch das Schwert fallen, und wer sich auf freiem Felde
aufhält, den will ich den wilden Tieren zum Fraß hingeben, und wer sich
auf den Berghöhen und in den Höhlen befindet, soll an der Pest sterben!
\bibleverse{28}Und ich will das Land zur Wüste und Einöde machen: seine
stolze Pracht soll ein Ende haben, und das Bergland Israels soll wüst
daliegen, so daß niemand es mehr durchwandert! \bibleverse{29}Dann
werden sie erkennen, daß ich der HERR bin, wenn ich das Land zur Wüste
und Einöde mache wegen all ihrer Greuel, die sie verübt haben.‹«

\hypertarget{c-strafrede-auf-die-unbuuxdffertigkeit-der-zuhuxf6rer-des-propheten}{%
\paragraph{c) Strafrede auf die Unbußfertigkeit der Zuhörer des
Propheten}\label{c-strafrede-auf-die-unbuuxdffertigkeit-der-zuhuxf6rer-des-propheten}}

\bibleverse{30}»Du aber, Menschensohn -- deine Volksgenossen unterhalten
sich über dich an den Mauern und in den Toreingängen der Häuser und
sagen einer zum andern: ›Kommt doch und hört, was für ein Ausspruch es
ist, den der HERR ergehen läßt!‹ \bibleverse{31}Da kommen sie denn zu
dir wie bei einem Volksauflauf und setzen sich vor dich hin als mein
Volk und hören deine Worte an, handeln aber nicht danach, sondern sie
tun liebevoll mit ihrem Munde, während ihr Herz hinter ihrem Gewinn
herläuft. \bibleverse{32}Und wisse wohl: du bist ihnen wie ein
Liebeslied✲, wie einer, der eine schöne Stimme hat und die Leier gut zu
spielen versteht; und so hören sie denn deine Worte an, handeln aber
nicht danach. \bibleverse{33}Wenn es aber eintrifft -- und es trifft
unfehlbar ein! --, dann werden sie erkennen, daß ein Prophet unter ihnen
dagewesen ist.«

\hypertarget{die-schlechten-hirten-und-gott-als-der-gute-hirt}{%
\subsubsection{2. Die schlechten Hirten und Gott als der gute
Hirt}\label{die-schlechten-hirten-und-gott-als-der-gute-hirt}}

\hypertarget{a-die-bisherigen-pflichtvergessenen-und-verderblichen-hirten}{%
\paragraph{a) Die bisherigen pflichtvergessenen und verderblichen
Hirten}\label{a-die-bisherigen-pflichtvergessenen-und-verderblichen-hirten}}

\hypertarget{section-33}{%
\section{34}\label{section-33}}

\bibleverse{1}Hierauf erging das Wort des HERRN an mich folgendermaßen:
\bibleverse{2}»Menschensohn, richte deine Weissagungen gegen die Hirten
Israels und sage zu ihnen: ›Zu den Hirten spricht Gott der HERR also:
Wehe den Hirten Israels, die sich selbst geweidet haben! Ist's nicht die
Herde, welche die Hirten weiden sollen? \bibleverse{3}Die Milch habt ihr
genossen, mit der Wolle euch bekleidet und die fetten Tiere
geschlachtet, aber meine Herde nicht geweidet. \bibleverse{4}Die
schwachen Tiere habt ihr nicht gestärkt und die kranken nicht geheilt,
die verwundeten nicht verbunden, die versprengten nicht zurückgeholt und
die verirrten nicht aufgesucht, sondern mit Gewalt und Härte über sie
geschaltet. \bibleverse{5}So haben denn (meine Schafe) sich zerstreut,
weil sie keinen Hirten hatten, und sind in ihrer Zerstreuung eine Beute
aller wilden Tiere geworden. \bibleverse{6}Auf allen Bergen und auf
jedem hohen Hügel sind meine Schafe umhergeirrt, und über das ganze Land
hin haben meine Schafe sich zerstreut, ohne daß sich jemand um sie
gekümmert oder auf sie geachtet hätte. \bibleverse{7}Darum, ihr Hirten,
vernehmt das Wort des HERRN! \bibleverse{8}So wahr ich lebe!‹ -- so
lautet der Ausspruch Gottes des HERRN --: ›weil meine Schafe geraubt und
meine Schafe von allen wilden Tieren des Feldes gefressen worden sind,
ohne daß ein Hirt da war, und weil meine Hirten sich nicht um meine
Schafe gekümmert, sondern nur sich selbst, aber nicht meine Schafe
geweidet haben: \bibleverse{9}darum, ihr Hirten, vernehmt das Wort des
HERRN! \bibleverse{10}So spricht Gott der HERR: Nunmehr will ich an die
Hirten\textless sup title=``=~gegen die Hirten vorgehen''\textgreater✲
und will meine Schafe von ihnen zurückfordern und ihrem Hirtenamt ein
Ende machen, damit die Hirten nicht mehr sich selbst weiden! Nein, ich
will meine Schafe ihnen aus dem Rachen reißen, daß sie von ihnen nicht
mehr gefressen werden!‹«

\hypertarget{b-gott-als-der-gute-hirt-der-zukunft}{%
\paragraph{b) Gott als der gute Hirt der
Zukunft}\label{b-gott-als-der-gute-hirt-der-zukunft}}

\hypertarget{aa-er-sammelt-seine-schafe-und-nimmt-sich-ihrer-liebevoll-an}{%
\subparagraph{aa) Er sammelt seine Schafe und nimmt sich ihrer liebevoll
an}\label{aa-er-sammelt-seine-schafe-und-nimmt-sich-ihrer-liebevoll-an}}

\bibleverse{11}Denn so hat Gott der HERR gesprochen: »Wisset wohl, ich
selbst will jetzt nach meinen Schafen sehen und mich ihrer annehmen.
\bibleverse{12}Wie ein Hirt sich seiner Herde annimmt, sobald einige von
seinen Schafen sich abgesondert haben, so will auch ich mich meiner
Schafe annehmen und sie aus all den Orten zurückholen, wohin sie
zerstreut worden sind am Tage des Gewölks und des Wetterdunkels.
\bibleverse{13}Herausführen will ich sie aus den Völkern und sie sammeln
aus den Ländern und sie in ihr Heimatland zurückbringen; da will ich sie
weiden auf den Bergen Israels, in den Talgründen und in allen bewohnten
Gegenden des Landes. \bibleverse{14}Auf guter Weide will ich sie weiden,
und auf den Bergeshöhen Israels soll ihre Trift sein; dort sollen sie
sich auf guter Trift lagern und fette Weide haben auf den Bergen
Israels. \bibleverse{15}Ich selbst will der Hirt meiner Schafe sein und
ich selbst sie lagern lassen« -- so lautet der Ausspruch Gottes des
HERRN. \bibleverse{16}»Die verirrten will ich aufsuchen und die
versprengten zurückholen, die verwundeten Tiere verbinden und die
kranken gesund machen; die fetten und starken will ich behüten; ich
werde sie weiden, wie es recht ist.«

\hypertarget{bb-er-vollzieht-die-scheidung-innerhalb-der-herde-und-schuxfctzt-die-schwachen-tiere-gegen-die-gewalttuxe4tigen}{%
\subparagraph{bb) Er vollzieht die Scheidung innerhalb der Herde und
schützt die schwachen Tiere gegen die
gewalttätigen}\label{bb-er-vollzieht-die-scheidung-innerhalb-der-herde-und-schuxfctzt-die-schwachen-tiere-gegen-die-gewalttuxe4tigen}}

\bibleverse{17}»Ihr aber, meine Herde« -- so hat Gott der HERR
gesprochen --: »ich will nunmehr Gericht halten zwischen den Schafen
untereinander und gegenüber den Widdern und den Böcken.
\bibleverse{18}Genügt es euch nicht, die beste Weide abzuweiden? Müßt
ihr auch noch das übrige Weideland mit den Füßen zertreten? Ihr habt
klares Wasser zu trinken: müßt ihr da noch das übriggebliebene mit euren
Füßen aufwühlen, \bibleverse{19}so daß meine Schafe das abweiden, was
ihr mit euren Füßen zerstampft habt, und das trinken, was ihr mit euren
Füßen aufgewühlt habt?« \bibleverse{20}Darum spricht Gott der HERR so zu
ihnen: »Seht, nunmehr will ich selbst Gericht zwischen den fetten und
den mageren Schafen halten. \bibleverse{21}Weil ihr die schwachen Tiere
alle mit der Seite und Schulter weggedrängt und mit euren Hörnern
gestoßen habt, bis ihr sie hinausgetrieben hattet, \bibleverse{22}so
will ich nun meinen Schafen zu Hilfe kommen, damit sie euch nicht mehr
zur Beute werden, und ich will zwischen den einzelnen Schafen Gericht
halten.«

\hypertarget{cc-er-gibt-ihnen-seinen-knecht-david-d.h.-einen-nachkommen-davids-als-messias-zu-seinem-stellvertreter-und-schlieuxdft-einen-friedensbund-mit-ihnen}{%
\subparagraph{cc) Er gibt ihnen seinen Knecht David (d.h. einen
Nachkommen Davids als Messias) zu seinem Stellvertreter und schließt
einen Friedensbund mit
ihnen}\label{cc-er-gibt-ihnen-seinen-knecht-david-d.h.-einen-nachkommen-davids-als-messias-zu-seinem-stellvertreter-und-schlieuxdft-einen-friedensbund-mit-ihnen}}

\bibleverse{23}»Ich will aber einen einzigen Hirten über sie bestellen,
der sie weiden soll, meinen Knecht David: der soll sie weiden, und der
soll ihr Hirt sein! \bibleverse{24}Und ich, der HERR, will ihr Gott
sein, und mein Knecht David soll Fürst in ihrer Mitte sein: ich, der
HERR, bestimme es so! \bibleverse{25}Und ich will einen Friedensbund mit
ihnen schließen und die bösen\textless sup title=``=~reißenden,
schädlichen''\textgreater✲ Tiere aus dem Lande verschwinden lassen, so
daß sie sogar in der Steppe sicher wohnen und in den Wäldern schlafen
können. \bibleverse{26}Ich will ihnen und der ganzen Umgebung meines
Hügels Segen verleihen und den Regen zu rechter Zeit fallen lassen:
segenspendende Regengüsse sollen es sein. \bibleverse{27}Die Bäume des
Feldes sollen ihre Früchte bringen und das Ackerland seinen Ertrag
geben; und sie sollen auf ihrem Grund und Boden sicher wohnen und
erkennen, daß ich der HERR bin, wenn ich die Stäbe ihres Joches
zerbreche und sie aus der Gewalt derer errette, die sie knechten.
\bibleverse{28}Sie sollen alsdann nicht mehr eine Beute der Heidenvölker
sein, und die Raubtiere des Landes sollen sie nicht mehr fressen,
sondern sie sollen in Sicherheit wohnen, ohne daß jemand sie
aufschreckt. \bibleverse{29}Und ich will ihnen eine ruhmeswerte✲
Pflanzung aufsprießen\textless sup title=``oder: erstehen''\textgreater✲
lassen, so daß sie nicht mehr vom Hunger im Lande weggerafft werden und
den Hohn der Heidenvölker nicht mehr zu erdulden haben.
\bibleverse{30}Dann werden sie erkennen, daß ich, der HERR, ihr Gott,
mit ihnen bin und daß sie, das Haus Israel, mein Volk sind« -- so lautet
der Ausspruch Gottes des HERRN. \bibleverse{31}»Denn ihr seid meine
Schafe\textless sup title=``oder: Herde''\textgreater✲, die Herde meiner
Weide, und ich bin euer Gott!« -- so lautet der Ausspruch Gottes des
HERRN.

\hypertarget{die-vernichtung-edoms-und-die-wiederherstellung-von-israels-verwuxfcstetem-gebiet}{%
\subsubsection{3. Die Vernichtung Edoms und die Wiederherstellung von
Israels verwüstetem
Gebiet}\label{die-vernichtung-edoms-und-die-wiederherstellung-von-israels-verwuxfcstetem-gebiet}}

\hypertarget{a-vernichtung-des-arglistigen-und-ruxe4uberischen-erbfeindes-edom}{%
\paragraph{a) Vernichtung des arglistigen und räuberischen Erbfeindes
Edom}\label{a-vernichtung-des-arglistigen-und-ruxe4uberischen-erbfeindes-edom}}

\hypertarget{section-34}{%
\section{35}\label{section-34}}

\bibleverse{1}Weiter erging das Wort des HERRN an mich folgendermaßen:
\bibleverse{2}»Menschensohn, richte deine Blicke gegen das Gebirge Seir
und verkünde folgende Weissagungen gegen dasselbe: \bibleverse{3}›So hat
Gott der HERR gesprochen: Nunmehr will ich an dich\textless sup
title=``=~gegen dich vorgehen''\textgreater✲, Gebirge Seir, ich will
meine Hand gegen dich ausstrecken und dich zur Wüste und Einöde machen!
\bibleverse{4}Deine Städte will ich in Trümmer legen, und du selbst
sollst zur Wüste werden, damit du erkennst, daß ich der HERR bin!
\bibleverse{5}Weil du immerfort Feindschaft gehegt und die Israeliten
zur Zeit ihres Unglücks, zur Zeit, als sie ihre Schuld endgültig büßten,
dem Schwert ausgeliefert hast: \bibleverse{6}darum, so wahr ich lebe!‹
-- so lautet der Ausspruch Gottes des HERRN --: ›bluten will ich dich
machen, und Blut soll dich verfolgen! Weil du dich durch Blutvergießen
verschuldet hast, soll Blut dich verfolgen! \bibleverse{7}Ich will das
Gebirge Seir zur Wüste und Einöde machen und alle in ihm ausrotten, die
da hin- und herziehen. \bibleverse{8}Und ich will seine Berge überall
mit Erschlagenen bedecken: auf deinen Höhen, in deinen Tälern und in
allen deinen Schluchten sollen vom Schwert Erschlagene niedersinken.
\bibleverse{9}Zu ewigen Wüsteneien will ich dich machen, und deine
Städte sollen unbewohnt sein, damit ihr erkennt, daß ich der HERR bin.
\bibleverse{10}Weil du gesagt hast: Die beiden Völker und die beiden
Länder müssen mein werden, und ich will sie in Besitz nehmen! -- obwohl
doch der HERR dort wohnt: \bibleverse{11}darum, so wahr ich lebe!‹ -- so
lautet der Ausspruch Gottes des HERRN --, ›will ich entsprechend deinem
eigenen Zorn und Eifer mit dir verfahren, wie du infolge deines Hasses
gegen sie verfahren bist, und ich will mich dir zu erkennen geben, wenn
ich mit dir ins Gericht gehe. \bibleverse{12}Dann wirst du auch
erkennen, daß ich, der HERR, alle deine Lästerungen gehört habe, die du
gegen das Bergland Israel ausgestoßen hast, indem du sagtest: Wüst liegt
es da: uns ist es zum Verspeisen gegeben! \bibleverse{13}So habt ihr den
Mund voll gegen mich genommen und vermessene Reden gegen mich
ausgestoßen; ich habe es wohl gehört!‹

\bibleverse{14}So hat Gott der HERR gesprochen: ›Zur Freude der ganzen
Erde will ich Verwüstung über dich bringen! \bibleverse{15}Wie du dich
darüber gefreut hast, daß der Erbbesitz des Hauses Israel verwüstet
wurde, ebenso will ich es dir widerfahren lassen: zur Wüste sollst du
werden, Gebirge Seir, und du, Edom, insgesamt, damit du zur Erkenntnis
kommst, daß ich der HERR bin!‹«

\hypertarget{b-begnadigung-und-wiederherstellung-des-landes-israel}{%
\paragraph{b) Begnadigung und Wiederherstellung des Landes
Israel}\label{b-begnadigung-und-wiederherstellung-des-landes-israel}}

\hypertarget{aa-dem-von-den-feinden-verhuxf6hnten-berglande-israel-wird-beseitigung-seiner-unrechtmuxe4uxdfigen-besitzer-zugesagt}{%
\subparagraph{aa) Dem von den Feinden verhöhnten Berglande Israel wird
Beseitigung seiner unrechtmäßigen Besitzer
zugesagt}\label{aa-dem-von-den-feinden-verhuxf6hnten-berglande-israel-wird-beseitigung-seiner-unrechtmuxe4uxdfigen-besitzer-zugesagt}}

\hypertarget{section-35}{%
\section{36}\label{section-35}}

\bibleverse{1}»Du aber, Menschensohn, sprich über das Bergland Israel
folgende Weissagung aus: ›Ihr Berge Israels, vernehmt das Wort des
HERRN! \bibleverse{2}So hat Gott der HERR gesprochen: Weil der Feind
über euch ausgerufen hat: Haha! Die Höhen sind verwüstet auf ewig, als
Eigentum uns zugefallen! --, \bibleverse{3}darum sprich folgende
Weissagungen aus: So hat Gott der HERR gesprochen: Darum, ja eben darum,
weil man euch angeschnaubt\textless sup title=``=~höhnisch
bedroht''\textgreater✲ und von allen Seiten über euch gieriges Verlangen
nach euch getragen hat, so daß ihr in den Besitz der noch
übriggebliebenen Heidenvölker gekommen und ins Gerede der Zungen und in
die üble Nachrede der Leute geraten seid~-- \bibleverse{4}darum, ihr
Berge Israels, vernehmt das Wort Gottes des HERRN! So spricht Gott der
HERR zu den Bergen und Hügeln, zu den Tälern und Schluchten, zu den öden
Trümmerstätten und den verlassenen Städten, die den noch
übriggebliebenen Heidenvölkern ringsum zur Beute und zum Gespött
geworden sind~-- \bibleverse{5}darum spricht Gott der HERR also:
Wahrlich, in glühendem Eifer rede ich gegen die noch übriggebliebenen
Heidenvölker und gegen das gesamte Edom, weil sie mit schadenfrohen
Herzen und völliger Gefühllosigkeit sich in den Besitz meines Landes
gesetzt haben, um die Bewohner auszutreiben und auszuplündern.
\bibleverse{6}Darum weissage über das Land Israel und richte an die
Berge und Hügel, an die Täler und Schluchten die Worte: So hat Gott der
HERR gesprochen: Fürwahr, in meinem Eifer und in meinem Grimm rede ich,
weil ihr den Hohn der Heidenvölker habt tragen müssen!‹
\bibleverse{7}Darum spricht Gott der HERR also: Ich hebe meine Hand auf
zum Schwur, daß die Völkerschaften, die um euch her wohnen, ihre
Schmähung selbst tragen sollen!‹«

\hypertarget{bb-verheiuxdfung-neuen-segens-und-herrlicher-entwicklung-des-berglandes-israel}{%
\subparagraph{bb) Verheißung neuen Segens und herrlicher Entwicklung des
Berglandes
Israel}\label{bb-verheiuxdfung-neuen-segens-und-herrlicher-entwicklung-des-berglandes-israel}}

\bibleverse{8}»›Ihr aber, ihr Berge Israels, sollt eure Zweige sprossen
lassen und eure Früchte tragen für mein Volk Israel, denn gar bald
werden sie heimkehren! \bibleverse{9}Denn wisset wohl: ich werde zu euch
kommen und mich euch wieder zuwenden, und ihr sollt wieder besät und
bepflanzt werden. \bibleverse{10}Ich will die Menschen auf euch
zahlreich werden lassen, das ganze Haus Israel insgesamt, die Städte
sollen wieder bewohnt und die Trümmer neu aufgebaut werden.
\bibleverse{11}Und ich werde Menschen und Vieh auf euch zahlreich
machen: sie sollen sich mehren und fruchtbar sein; und ich will euch
wieder bewohnt sein lassen wie in euren früheren Zeiten und euch noch
mehr Gutes erweisen als je zuvor, damit ihr erkennt, daß ich der HERR
bin! \bibleverse{12}Menschen will ich wieder auf euch wandeln lassen,
nämlich mein Volk Israel: die sollen dich wieder in Besitz nehmen, und
du sollst ihnen als Erbbesitz gehören und sie hinfort nicht mehr ihrer
Kinder berauben!‹« \bibleverse{13}So hat Gott der HERR (zum Lande)
gesprochen: »Weil man dir vorgeworfen hat, du seiest eine
Menschenfresserin gewesen und eine Kindesmörderin für dein eigenes Volk,
\bibleverse{14}darum sollst du hinfort keine Menschen mehr fressen und
dein Volk nicht mehr kinderlos machen!« -- so lautet der Ausspruch
Gottes des HERRN. \bibleverse{15}»Und ich will dich hinfort nicht länger
die Schmähung der Heiden hören lassen, und den Hohn der Völker sollst du
nicht mehr zu tragen haben und sollst deine Bevölkerung nicht mehr ihrer
Kinder berauben!« -- so lautet der Ausspruch Gottes des HERRN.

\hypertarget{cc-die-gruxfcnde-durch-welche-gott-zu-solchem-verfahren-bestimmt-wird}{%
\subparagraph{cc) Die Gründe, durch welche Gott zu solchem Verfahren
bestimmt
wird}\label{cc-die-gruxfcnde-durch-welche-gott-zu-solchem-verfahren-bestimmt-wird}}

\bibleverse{16}Weiter erging das Wort des HERRN an mich folgendermaßen:
\bibleverse{17}»Menschensohn, solange die vom Hause Israel in ihrem
Lande wohnten, verunreinigten sie es durch ihren Wandel und ihr ganzes
Tun; ihr Wandel war vor meinen Augen wie die Unreinheit einer Frau, die
ihren Blutgang hat. \bibleverse{18}Da ließ ich denn meinem Zorn gegen
sie freien Lauf wegen des Blutes, das sie im Lande vergossen hatten, und
wegen ihres Götzendienstes, durch den sie es verunreinigt hatten.
\bibleverse{19}Ich zerstreute sie also unter die Heidenvölker, und sie
wurden in die Länder versprengt: nach ihrem Wandel und ihrem ganzen Tun
ging ich mit ihnen ins Gericht. \bibleverse{20}Als sie nun unter die
Heidenvölker kamen, da brachten sie, wohin sie kamen, meinen heiligen
Namen in Unehre, indem man von ihnen sagte: ›Das Volk des HERRN sind
diese, und doch haben sie aus seinem Lande wegziehen müssen!‹
\bibleverse{21}Da tat es mir leid um meinen heiligen Namen, weil das
Haus Israel ihn unter den Heidenvölkern überall, wohin sie kamen, in
Unehre brachte. \bibleverse{22}Darum sage zum Hause Israel: ›So hat Gott
der HERR gesprochen: Nicht um euretwillen, Haus Israel, greife ich
ein\textless sup title=``oder: verfahre ich so''\textgreater✲, sondern
um meines heiligen Namens willen, den ihr unter den Heidenvölkern
überall entehrt habt, wohin ihr gekommen seid. \bibleverse{23}So will
ich denn meinen großen Namen, der unter den Heiden entheiligt worden
ist, weil ihr ihn unter ihnen entheiligt habt, wieder zu Ehren bringen,
damit die Heiden erkennen, daß ich der HERR bin‹ -- so lautet der
Ausspruch Gottes des HERRN --, ›wenn ich mich vor ihren Augen an euch
als den Heiligen erweise.‹«

\hypertarget{dd-die-stufenweise-ausfuxfchrung-des-guxf6ttlichen-planes-und-zwar-aus-gnade}{%
\subparagraph{dd) Die stufenweise Ausführung des göttlichen Planes, und
zwar aus
Gnade}\label{dd-die-stufenweise-ausfuxfchrung-des-guxf6ttlichen-planes-und-zwar-aus-gnade}}

\bibleverse{24}»›Ich will euch also aus den Heidenvölkern herausholen
und euch aus allen Ländern sammeln und euch in euer Land zurückbringen.
\bibleverse{25}Dann will ich reines Wasser über euch sprengen, damit ihr
rein werdet: von all euren Befleckungen und von all eurem Götzendienst
will ich euch reinigen. \bibleverse{26}Und ich will euch ein neues Herz
verleihen und euch einen neuen Geist eingeben: das steinerne Herz will
ich aus eurer Brust herausnehmen und euch dafür ein Herz von Fleisch
verleihen. \bibleverse{27}Ich will meinen Geist in euer Inneres geben
und will solche Leute aus euch machen, die nach meinen Satzungen wandeln
und meine Weisungen beobachten und tatsächlich ausführen.
\bibleverse{28}Dann sollt ihr wohnen bleiben in dem Lande, das ich euren
Vätern gegeben habe; ihr sollt mein Volk sein, und ich will euer Gott
sein. \bibleverse{29}Wenn ich euch alsdann von all euren Befleckungen
befreit habe, will ich den Getreidesegen herbeirufen und ihn mehren und
keine Hungersnot mehr über euch verhängen. \bibleverse{30}Auch die
Früchte der Bäume und den Ertrag der Felder will ich mehren, damit ihr
euch nicht noch einmal unter den Heidenvölkern wegen einer Hungersnot
schmähen lassen müßt. \bibleverse{31}Wenn ihr alsdann an euren bösen
Wandel zurückdenkt und an euer ganzes verwerfliches Tun, so werdet ihr
vor euch selbst einen Abscheu empfinden wegen eurer Verschuldungen und
eurer Greuel. \bibleverse{32}Nicht um euretwillen greife ich
ein\textless sup title=``oder: verfahre ich so''\textgreater✲‹ -- so
lautet der Ausspruch Gottes des HERRN --: ›das sei euch kundgetan!
Schämt euch vielmehr und errötet über euren Wandel, ihr vom Hause
Israel!‹«

\hypertarget{ee-die-anerkennung-der-gruxf6uxdfe-gottes-auch-von-seiten-der-heiden}{%
\subparagraph{ee) Die Anerkennung der Größe Gottes auch von seiten der
Heiden}\label{ee-die-anerkennung-der-gruxf6uxdfe-gottes-auch-von-seiten-der-heiden}}

\bibleverse{33}So hat Gott der HERR gesprochen: »Zu derselben Zeit, wo
ich euch von all euren Verschuldungen reinige, will ich auch die Städte
neu bevölkern, und die Trümmer sollen wieder aufgebaut werden;
\bibleverse{34}das verödete Land soll aufs neue bestellt werden, während
es zuvor als Wüste vor den Augen aller Vorüberziehenden dagelegen hat.
\bibleverse{35}Dann wird man sagen: ›Dieses Land, das verödet dalag, ist
wie der Garten Eden geworden, und die Städte, die in Trümmern lagen und
verwüstet und zerstört waren, sind jetzt wohlbefestigt und volkreich.‹
\bibleverse{36}Da werden dann die Völkerschaften, die rings um euch her
übriggeblieben sind, zu der Erkenntnis kommen, daß ich, der HERR, es
bin, der das Zerstörte neu aufgebaut und das Verwüstete neu bepflanzt
hat. Ich, der HERR, habe es verheißen und werde es auch vollführen!«

\bibleverse{37}So hat Gott der HERR gesprochen: »Auch darin will ich
mich noch vom Hause Israel erbitten lassen, daß ich es ihnen gewähre:
ich will sie an Menschen so zahlreich werden lassen wie eine Herde von
Schafen. \bibleverse{38}Wie der Tempel von Opferschafen, wie Jerusalem
an seinen hohen Festen von Schafen, so sollen die jetzt verödeten Städte
voll von Menschenherden sein, damit man erkennt, daß ich der HERR bin.«

\hypertarget{die-wiederbelebung-neuschaffung-des-toten-volkes-israel}{%
\subsubsection{4. Die Wiederbelebung (=~Neuschaffung) des toten Volkes
Israel}\label{die-wiederbelebung-neuschaffung-des-toten-volkes-israel}}

\hypertarget{a-israels-wunderbare-auferweckung-vom-tode}{%
\paragraph{a) Israels wunderbare Auferweckung vom
Tode}\label{a-israels-wunderbare-auferweckung-vom-tode}}

\hypertarget{aa-das-dem-propheten-zuteil-gewordene-gesicht-von-der-belebung-der-totengebeine}{%
\subparagraph{aa) Das dem Propheten zuteil gewordene Gesicht von der
Belebung der
Totengebeine}\label{aa-das-dem-propheten-zuteil-gewordene-gesicht-von-der-belebung-der-totengebeine}}

\hypertarget{section-36}{%
\section{37}\label{section-36}}

\bibleverse{1}Die Hand des HERRN kam über mich: er führte mich im
Zustande der Verzückung\textless sup title=``vgl. 11,24''\textgreater✲
hinaus und ließ mich mitten in der Tal-Ebene nieder, die voll von
Totengebeinen war. \bibleverse{2}Er führte mich ringsherum✲ an diesen
vorüber; und siehe, es lagen ihrer sehr viele über die ganze Tal-Ebene
hin, aber alle waren ganz verdorrt. \bibleverse{3}Da fragte er mich:
»Menschensohn, können wohl diese Gebeine wieder lebendig werden?« Ich
antwortete: »HERR, mein Gott, du weißt es.« \bibleverse{4}Hierauf gebot
er mir: »Weissage über diese Gebeine und rufe ihnen zu: ›Ihr verdorrten
Gebeine, vernehmt das Wort des HERRN! \bibleverse{5}So hat Gott der HERR
zu\textless sup title=``oder: von''\textgreater✲ diesen Gebeinen gesagt:
Fürwahr, ich will Odem\textless sup title=``oder:
Lebensgeist''\textgreater✲ in euch kommen lassen, damit ihr wieder
lebendig werdet, \bibleverse{6}und will Sehnen an euch schaffen und
Fleisch über euch wachsen lassen, ich will euch mit Haut überziehen und
euch Odem\textless sup title=``oder: Lebensgeist''\textgreater✲
einflößen, damit ihr wieder lebendig werdet und erkennt, daß ich der
HERR bin.‹«

\bibleverse{7}Da weissagte ich, wie mir geboten war; und als ich
geweissagt hatte, entstand plötzlich ein Rascheln, und die Gebeine
fügten sich zusammen, eins an das andere. \bibleverse{8}Als ich nun
hinschaute, nahm ich wohl Sehnen an ihnen wahr, und Fleisch war über sie
gewachsen, und mit Haut waren sie oben überzogen, aber Odem\textless sup
title=``oder: Lebensgeist''\textgreater✲ war noch nicht in ihnen.
\bibleverse{9}Da sagte er zu mir: »Richte eine Weissagung an den
Odem\textless sup title=``oder: Lebensgeist''\textgreater✲, ja weissage,
Menschensohn, und sprich zu dem Lebensgeist: ›So hat Gott der HERR
gesprochen: O Geist, komm von den vier Winden herbei und hauche diese
Erschlagenen an, daß sie wieder lebendig werden!‹« \bibleverse{10}Als
ich nun so weissagte, wie er mir geboten hatte, da kam der Lebensgeist
in sie, so daß sie lebendig wurden und auf ihre Füße traten, eine
gewaltig große Heerschar.

\hypertarget{bb-deutung-des-merkwuxfcrdigen-vorganges-als-eines-bildes}{%
\subparagraph{bb) Deutung des merkwürdigen Vorganges als eines
Bildes}\label{bb-deutung-des-merkwuxfcrdigen-vorganges-als-eines-bildes}}

\bibleverse{11}Hierauf sagte er zu mir: »Menschensohn, diese Gebeine
hier sind das ganze Haus Israel. Siehe, sie sagen jetzt: ›Verdorrt sind
unsere Gebeine, und geschwunden ist unsere Hoffnung: es ist aus mit
uns!‹ \bibleverse{12}Darum weissage du und sage zu ihnen: ›So hat Gott
der HERR gesprochen: Wisset wohl: ich will eure Gräber öffnen und euch,
mein Volk, aus euren Gräbern hervorgehen lassen und euch in das Land
Israel zurückbringen: \bibleverse{13}dann werdet ihr erkennen, daß ich
der HERR bin, wenn ich eure Gräber öffne und euch, mein Volk, aus euren
Gräbern hervorgehen lasse. \bibleverse{14}Ich will also meinen Geist in
euch kommen lassen, daß ihr lebendig werdet, und will euch wieder in
euer Land versetzen, damit ihr erkennt, daß ich, der HERR, es verheißen
habe und es auch zur Ausführung bringe!‹ -- so lautet der Ausspruch des
HERRN.«

\hypertarget{b-die-wiederherstellung-des-davidischen-einheitsreiches-durch-die-vereinigung-der-beiden-reiche-juda-und-israel}{%
\paragraph{b) Die Wiederherstellung des davidischen Einheitsreiches
durch die Vereinigung der beiden Reiche Juda und
Israel}\label{b-die-wiederherstellung-des-davidischen-einheitsreiches-durch-die-vereinigung-der-beiden-reiche-juda-und-israel}}

\hypertarget{aa-die-sinnbildliche-vom-propheten-vorzunehmende-handlung}{%
\subparagraph{aa) Die sinnbildliche, vom Propheten vorzunehmende
Handlung}\label{aa-die-sinnbildliche-vom-propheten-vorzunehmende-handlung}}

\bibleverse{15}Weiter erging das Wort des HERRN an mich folgendermaßen:
\bibleverse{16}»Du, Menschensohn, nimm dir einen Holzstab und schreibe
darauf: ›Juda und die mit ihm vereinten Israeliten‹. Sodann nimm noch
einen anderen Holzstab und schreibe darauf: ›Joseph, der Stab Ephraims
und des ganzen mit ihm vereinten Hauses Israel‹. \bibleverse{17}Dann
füge✲ dir beide Holzstäbe zu einem einzigen Stabe zusammen, so daß sie
ein Ganzes in deiner Hand bilden!«

\hypertarget{bb-die-deutung-der-sinnbildlichen-handlung}{%
\subparagraph{bb) Die Deutung der sinnbildlichen
Handlung}\label{bb-die-deutung-der-sinnbildlichen-handlung}}

\bibleverse{18}»Wenn dann deine Volksgenossen zu dir sagen: ›Willst du
uns nicht erklären, was dies bedeuten soll?‹, \bibleverse{19}so antworte
ihnen: ›So hat Gott der HERR gesprochen: Seht, ich werde den Stab
Josephs und der mit ihm vereinten Stämme Israels, der in der Hand
Ephraims ist, nehmen und ihn zu dem Stabe Judas hinzutun und sie (beide)
zu einem einzigen Stabe machen, so daß sie ein Ganzes in meiner Hand
bilden.‹ \bibleverse{20}Wenn du dann die Stäbe, die du mit Inschriften
versehen hast, vor ihren Augen in deiner Hand hältst, \bibleverse{21}so
sage zu ihnen: ›So hat Gott der HERR gesprochen: Wisset wohl: ich will
die Kinder Israels aus den Heidenvölkern, unter die sie haben ziehen
müssen, herausholen und sie von allen Seiten her sammeln und sie in ihr
Land zurückbringen. \bibleverse{22}Ich will sie dann zu einem einzigen
Volk machen in dem Lande, auf den Bergen Israels, so daß ein einziger
König über sie alle herrscht; sie sollen alsdann nicht wieder zwei
Völker bilden und nicht wieder in zwei Reiche geteilt sein.‹«

\hypertarget{cc-die-herrliche-zukunft-des-geeinten-volkes}{%
\subparagraph{cc) Die herrliche Zukunft des geeinten
Volkes}\label{cc-die-herrliche-zukunft-des-geeinten-volkes}}

\bibleverse{23}»›Dann sollen sie sich nicht mehr an ihren Götzen und
abscheulichen Abgöttern und durch all ihre Abfallssünden verunreinigen;
nein, ich will sie frei machen von all ihren Treubrüchen, durch die sie
sich versündigt haben, und will sie reinigen; dann sollen sie mein Volk
werden, und ich will ihr Gott sein. \bibleverse{24}Mein Knecht David
aber soll König über sie sein, und sie sollen alle einen einzigen Hirten
haben; dann werden sie nach meinen Weisungen wandeln, meine Satzungen
beobachten und nach ihnen handeln. \bibleverse{25}Sie sollen dann wieder
in dem Lande wohnen, das ich meinem Knecht Jakob gegeben habe und in
welchem ihre Väter gewohnt haben; auch sie sollen darin wohnen samt
ihren Kindern und Kindeskindern bis in Ewigkeit; und mein Knecht David
soll ihr Herrscher sein für immer. \bibleverse{26}Dann will ich auch
einen Friedensbund mit ihnen schließen, ein ewiger Bund soll mit ihnen
bestehen; und ich will sie seßhaft machen und mehren und mein Heiligtum
in ihrer Mitte belassen ewiglich. \bibleverse{27}Meine Wohnung aber wird
über ihnen sein; ich will ihr Gott sein, und sie sollen mein Volk sein.
\bibleverse{28}Auch die Heidenvölker werden dann erkennen, daß ich der
HERR bin, der Israel heiligt, wenn mein Heiligtum sich in ihrer Mitte
befindet ewiglich!‹«

\hypertarget{letzter-ansturm-und-endguxfcltige-vernichtung-der-gottfeindlichen-nordischen-heidenmuxe4chte}{%
\subsubsection{5. Letzter Ansturm und endgültige Vernichtung der
gottfeindlichen nordischen
Heidenmächte}\label{letzter-ansturm-und-endguxfcltige-vernichtung-der-gottfeindlichen-nordischen-heidenmuxe4chte}}

\hypertarget{a-gott-selbst-fuxfchrt-die-rohen-horden-gogs-aus-dem-lande-magog-gegen-paluxe4stina-herbei}{%
\paragraph{a) Gott selbst führt die rohen Horden Gogs aus dem Lande
Magog gegen Palästina
herbei}\label{a-gott-selbst-fuxfchrt-die-rohen-horden-gogs-aus-dem-lande-magog-gegen-paluxe4stina-herbei}}

\hypertarget{section-37}{%
\section{38}\label{section-37}}

\bibleverse{1}Das Wort des HERRN erging weiter an mich folgendermaßen:
\bibleverse{2}»Menschensohn, richte deine Blicke auf\textless sup
title=``oder: gegen''\textgreater✲ Gog im Lande Magog, den Fürsten von
Ros, Mesech und Thubal, und sprich folgende Weissagungen über ihn aus:
\bibleverse{3}›So hat Gott der HERR gesprochen: Nunmehr will ich an
dich\textless sup title=``=~gegen dich vorgehen''\textgreater✲, Gog,
Fürst von Ros, Mesech und Thubal! \bibleverse{4}Ich will dich
herbeilocken\textless sup title=``oder: zurückführen''\textgreater✲ und
dir Haken in die Kinnbacken legen und dich ins Feld ziehen lassen mit
deiner ganzen Kriegsmacht, Rosse und Reiter, allesamt in voller
Ausrüstung, ein gewaltiges Heer mit Schilden und Tartschen\textless sup
title=``=~mit Langschilden und Kurzschilden''\textgreater✲, durchweg mit
Schwertern bewaffnet: \bibleverse{5}Perser, Äthiopier und Libyer
befinden sich unter ihnen, allesamt mit Schild und Helm;
\bibleverse{6}Kimmerier mit all ihren Scharen, das Haus Thogarma✲ aus
dem äußersten Norden mit all seinen Scharen: ja viele Völker sind mit
dir. \bibleverse{7}Rüste dich und halte dich bereit, du mit all deinen
Scharen, die sich bei dir gesammelt haben, und sei du ihr Anführer!
\bibleverse{8}Nach geraumer Zeit sollst du Befehl erhalten: am Ende der
Jahre sollst du über ein Land kommen, das sich vom Kriege\textless sup
title=``oder: von der Verwüstung''\textgreater✲ erholt hat, (zu einem
Volk) das aus vielen Völkern auf den Bergen Israels, die dauernd verödet
lagen, gesammelt worden ist; jetzt aber ist es aus den Völkern
zurückgeführt, und sie wohnen nun in Sicherheit allesamt.
\bibleverse{9}Da wirst du dann heranziehen, wie ein Ungewitter
daherkommen, wirst wie eine Wetterwolke sein, um das Land zu bedecken,
du und alle deine Scharen und die Völkermenge mit dir.‹«

\hypertarget{b-die-buxf6sen-absichten-des-feindes-und-der-grouxdfe-plan-gottes}{%
\paragraph{b) Die bösen Absichten des Feindes und der große Plan
Gottes}\label{b-die-buxf6sen-absichten-des-feindes-und-der-grouxdfe-plan-gottes}}

\bibleverse{10}So hat Gott der HERR gesprochen: »Zu jener Zeit werden
(böse) Gedanken in deinem Herzen aufsteigen, und du wirst einen
schlimmen Anschlag ersinnen; \bibleverse{11}du wirst nämlich denken:
›Ich will zu Felde ziehen gegen ein Land von Bauernhöfen\textless sup
title=``d.h. das offen daliegt''\textgreater✲, will über friedliche
Leute herfallen, die ruhig und sorglos leben; sie wohnen ja allesamt (in
Ortschaften) ohne Mauern und haben keine Riegel und Tore.‹
\bibleverse{12}(Gegen diese gedenkst du zu ziehen,) um schonungslos zu
rauben und Beute zu machen, um deine Hand an wiederbewohnte
Trümmerstätten zu legen und an ein Volk, das aus den Heidenländern
gesammelt worden ist, das sich Hab und Gut erworben hat und auf dem
Nabel✲ der Erde wohnt. \bibleverse{13}Seba und Dedan und ihre Kaufleute,
Tharsis und all seine raubgierigen Löwen✲ werden zu dir sagen: ›Bist du
gekommen, um Beute zu machen? Hast du deine Scharen aufgeboten, um zu
plündern, um Silber und Gold zu rauben, um Hab und Gut wegzunehmen, um
reiche Beute zu machen?‹ \bibleverse{14}Darum verkünde, Menschensohn,
dem Gog folgende Weissagungen: ›So hat Gott der HERR gesprochen: Jawohl,
zu jener Zeit, wo mein Volk Israel wieder in Sicherheit wohnt, wirst du
aufbrechen \bibleverse{15}und von deinem Wohnsitz, vom äußersten Norden
her, kommen, du und viele Völker mit dir, allesamt hoch zu Roß, eine
große Schar und ein gewaltiges Heer; \bibleverse{16}und du wirst gegen
mein Volk Israel heranziehen wie eine Wetterwolke, um das Land zu
bedecken. Am Ende der Tage wird es geschehen, daß ich dich gegen mein
Land zu Felde ziehen lasse, damit die Heidenvölker mich kennenlernen,
wenn ich mich vor ihren Augen an dir, Gog, als den Heiligen erweise.‹«

\hypertarget{c-gottes-furchtbares-gericht-uxfcber-gog}{%
\paragraph{c) Gottes furchtbares Gericht über
Gog}\label{c-gottes-furchtbares-gericht-uxfcber-gog}}

\bibleverse{17}So hat Gott der HERR (zu Gog) gesprochen: »Bist du es
nicht, auf den ich in früheren Tagen durch den Mund meiner Knechte, der
Propheten Israels, hingewiesen habe, die zu jener Zeit
jahrelang\textless sup title=``=~immer wieder''\textgreater✲ geweissagt
haben, daß ich dich gegen sie heranführen würde? \bibleverse{18}So wird
denn an demselben Tage, an dem Gog in das Land Israel einrückt« -- so
lautet der Ausspruch Gottes des HERRN --, »da wird die Zornesglut in mir
auflodern; \bibleverse{19}und in meinem Zorneseifer, im Feuer meines
Ingrimms spreche ich es aus: ›Wahrlich, an jenem Tage wird ein großes
Erdbeden im Lande Israel stattfinden! \bibleverse{20}Da sollen vor mir
erbeben die Fische im Meer und die Vögel unter dem Himmel, die Tiere auf
dem Felde und alles Gewürm, das auf dem Erdboden kriecht, und alle
Menschen, die auf der ganzen Erde wohnen; die Berge sollen einstürzen
und die Felswände umfallen und alle Mauern zu Boden stürzen.
\bibleverse{21}Dann werde ich in meinem ganzen Berglande das Schwert
gegen ihn aufbieten‹ -- so lautet der Ausspruch Gottes des HERRN --, ›so
daß das Schwert eines jeden sich gegen den andern kehrt.
\bibleverse{22}Und ich will das Strafgericht an ihm vollziehen durch
Pest und Blutvergießen, durch Wolkenbrüche\textless sup
title=``=~überschwemmende Regenfluten''\textgreater✲ und Hagelsteine;
Feuer und Schwefel will ich regnen lassen auf ihn und auf seine
Kriegsscharen und auf die vielen Völker, die bei ihm sind.
\bibleverse{23}So will ich meine Größe und meine Heiligkeit erweisen und
mich vor den Augen vieler Völker kundtun, damit sie erkennen, daß ich
der HERR bin!‹«

\hypertarget{d-nochmalige-ankuxfcndigung-des-untergangs-fuxfcr-gog}{%
\paragraph{d) Nochmalige Ankündigung des Untergangs für
Gog}\label{d-nochmalige-ankuxfcndigung-des-untergangs-fuxfcr-gog}}

\hypertarget{section-38}{%
\section{39}\label{section-38}}

\bibleverse{1}»Du also, Menschensohn, sprich gegen Gog folgende
Weissagungen aus: ›So hat Gott der HERR gesprochen: Wisse wohl: ich will
an dich\textless sup title=``=~gegen dich vorgehen''\textgreater✲, Gog,
Fürst von Ros, Mesech und Thubal! \bibleverse{2}Ich will dich
herbeilocken\textless sup title=``vgl. 38,4''\textgreater✲ und am
Gängelbande führen und dich vom äußersten Norden heranziehen lassen und
dich auf die Berge Israels kommen lassen. \bibleverse{3}Aber (dort) will
ich dir den Bogen aus der linken Hand schlagen und die Pfeile deiner
rechten Hand entfallen lassen. \bibleverse{4}Auf den Bergen Israels
sollst du fallen, du selbst und alle deine Scharen und die Völker, die
bei dir sind; den Raubvögeln, allem Getier, das Flügel hat, und den
Raubtieren des Feldes überlasse ich dich zum Fraß: \bibleverse{5}auf
freiem Felde sollst du fallen; denn ich habe es gesagt!‹ -- so lautet
der Ausspruch Gottes des HERRN. \bibleverse{6}›Da will ich an Magog und
an die in Sorglosigkeit lebenden Bewohner der Meeresländer Feuer legen,
damit sie erkennen, daß ich der HERR bin. \bibleverse{7}Aber inmitten
meines Volkes Israel will ich meinem heiligen Namen Anerkennung
verschaffen und werde meinen heiligen Namen nicht länger entweihen
lassen, damit die Heidenvölker erkennen, daß ich der HERR bin, der
Heilige in Israel. \bibleverse{8}Wisse wohl: es kommt und geht in
Erfüllung!‹ -- so lautet der Ausspruch Gottes des HERRN --; ›das ist der
Tag, auf den ich hingewiesen habe!‹«

\hypertarget{e-niederlage-und-bestattung-der-scharen-gogs}{%
\paragraph{e) Niederlage und Bestattung der Scharen
Gogs}\label{e-niederlage-und-bestattung-der-scharen-gogs}}

\bibleverse{9}»Da werden denn die Bewohner der Städte Israels
hinausziehen und Feuer anmachen und einheizen mit den Waffen, den
Kurzschilden und Langschilden, mit den Bogen und Pfeilen, mit den Keulen
und Lanzen, und werden sieben Jahre lang Feuer mit ihnen machen.
\bibleverse{10}Die brauchen dann kein Holz mehr vom Felde zu holen und
keins in den Wäldern zu hauen, sondern werden die Waffen als Brennholz
benutzen und Raub gewinnen von denen, welche sie beraubt hatten, und die
plündern, welche sie geplündert hatten« -- so lautet der Ausspruch
Gottes des HERRN. \bibleverse{11}»Und an jenem Tage werde ich dem Gog
eine Grabstätte in Israel anweisen, nämlich das Tal der Wanderer auf der
Ostseite des (Toten) Meeres: dies wird ihrem Wanderzuge ein Ende machen.
Dort wird man Gog und seine gesamte Heeresmacht begraben und es das ›Tal
der Heeresmacht Gogs‹ nennen. \bibleverse{12}Das Haus Israel wird dann
sieben Monate lang mit ihrem Begräbnis zu tun haben, um das Land zu
reinigen; \bibleverse{13}und die gesamte Bevölkerung des Landes wird
sich an dem Begräbnis beteiligen; und das wird ihnen zum Ruhm gereichen
an dem Tage, wo ich mich verherrlichen werde« -- so lautet der Ausspruch
Gottes des HERRN. \bibleverse{14}»Dann wird man Männer bestellen, die
das ständige Geschäft haben, im Lande umherzuziehen, um die von dem
Wandervolk im Lande noch liegengebliebenen Toten zu begraben und so (das
ganze Land) zu reinigen; nach Ablauf der sieben Monate sollen sie die
Durchsuchung vornehmen. \bibleverse{15}Wenn sie dann auf ihrer Wanderung
das Land durchziehen und einer von ihnen ein Menschengerippe erblickt,
so soll er ein Mal daneben errichten, bis die Totengräber es im Tal der
Heeresmacht Gogs begraben haben. \bibleverse{16}Auch wird es dort eine
Stadt namens Hamona\textless sup title=``d.h. Menge,
Getümmel''\textgreater✲ geben. So sollen sie das Land reinigen.«

\hypertarget{f-einladung-der-vuxf6gel-und-raubtiere-zu-einem-grouxdfen-opfermahl}{%
\paragraph{f) Einladung der Vögel und Raubtiere zu einem großen
Opfermahl}\label{f-einladung-der-vuxf6gel-und-raubtiere-zu-einem-grouxdfen-opfermahl}}

\bibleverse{17}»Du aber, Menschensohn« -- so hat Gott der HERR
gesprochen --, »sage zu den Vögeln, zu allem Getier, das Flügel hat, und
zu allen Raubtieren des Feldes: ›Versammelt euch und kommt herbei!
Schart euch von allen Seiten her zusammen zu meinem Opferschmaus, den
ich euch veranstalte, zu dem großen Opferschmaus auf den Bergen Israels!
Ihr sollt Fleisch fressen und Blut trinken! \bibleverse{18}Fleisch von
Heerführern sollt ihr fressen und das Blut von Fürsten der Erde trinken:
Widder und Lämmer, Böcke und Stiere, lauter Mastvieh aus Basan;
\bibleverse{19}in Fett sollt ihr euch satt fressen und Blut bis zur
Trunkenheit trinken von\textless sup title=``oder: bei''\textgreater✲
meinem Opferschmaus, den ich euch veranstalte. \bibleverse{20}An meiner
Tafel sollt ihr euch sättigen an Rossen und Reitern\textless sup
title=``oder: Reitpferden''\textgreater✲, an Heerführern und
Kriegsleuten aller Art!‹« -- so lautet der Ausspruch Gottes des HERRN.

\hypertarget{g-anerkennung-gottes-durch-die-heidenvuxf6lker}{%
\paragraph{g) Anerkennung Gottes durch die
Heidenvölker}\label{g-anerkennung-gottes-durch-die-heidenvuxf6lker}}

\bibleverse{21}»So will ich denn meine Herrlichkeit unter den
Heidenvölkern offenbar werden lassen, und alle Heidenvölker sollen mein
Strafgericht sehen, das ich vollzogen habe, und meine Hand, die ich sie
habe fühlen lassen; \bibleverse{22}das Haus Israel aber wird erkennen,
daß ich, der HERR, ihr Gott bin, von jenem Tage an und weiterhin;
\bibleverse{23}und die Heidenvölker werden erkennen, daß das Haus Israel
um seiner Verschuldung willen in die Verbannung\textless sup
title=``oder: Gefangenschaft''\textgreater✲ hat wandern müssen zur
Strafe dafür, daß sie treulos gegen mich geworden waren, und weil ich
mein Angesicht vor ihnen verborgen und sie der Gewalt ihrer Feinde
preisgegeben hatte, so daß sie allesamt durch das Schwert fallen mußten.
\bibleverse{24}Wie ihre Unreinheit und ihre Treubrüche es verdienten, so
bin ich mit ihnen verfahren und habe mein Angesicht vor ihnen
verborgen.«

\hypertarget{h-verheiuxdfung-der-begnadigung-israels-und-seiner-ruxfcckkehr-in-die-heimat}{%
\paragraph{h) Verheißung der Begnadigung Israels und seiner Rückkehr in
die
Heimat}\label{h-verheiuxdfung-der-begnadigung-israels-und-seiner-ruxfcckkehr-in-die-heimat}}

\bibleverse{25}Darum hat Gott der HERR so gesprochen: »Nunmehr will ich
das Geschick Jakobs wenden und mich des gesamten Hauses Israel erbarmen
und für meinen heiligen Namen eifern. \bibleverse{26}Dann sollen sie
ihre Schmach und alle ihre Treulosigkeit, die sie sich gegen mich haben
zuschulden kommen lassen, vergessen, wenn sie wieder sicher in ihrem
Lande wohnen und niemand sie mehr aufschreckt. \bibleverse{27}Wenn ich
sie aus den Völkern zurückgebracht und sie aus den Ländern ihrer Feinde
gesammelt und mich vor den Augen der Heidenvölker als den Heiligen an
ihnen erwiesen habe, \bibleverse{28}dann werden sie auch erkennen, daß
ich, der HERR, ihr Gott bin, der ich sie zwar unter die Heidenvölker in
die Gefangenschaft geführt habe, aber sie nun auch wieder in ihrem Lande
versammle und fortan keinen von ihnen dort zurücklasse.
\bibleverse{29}Und ich werde mein Angesicht nicht mehr vor ihnen
verbergen, weil ich (alsdann) meinen Geist auf\textless sup
title=``oder: über''\textgreater✲ das Haus Israel ausgegossen habe«, --
so lautet der Ausspruch Gottes des HERRN.

\hypertarget{ii.-prophetische-gesichte-von-den-ordnungen-des-neuen-messianischen-gottesreiches-kap.-40-48}{%
\subsection{II. Prophetische Gesichte von den Ordnungen des neuen
(messianischen) Gottesreiches (Kap.
40-48)}\label{ii.-prophetische-gesichte-von-den-ordnungen-des-neuen-messianischen-gottesreiches-kap.-40-48}}

\hypertarget{der-tempel-der-zukunft-und-seine-einweihung-kap.-40-43}{%
\subsubsection{1. Der Tempel der Zukunft und seine Einweihung (Kap.
40-43)}\label{der-tempel-der-zukunft-und-seine-einweihung-kap.-40-43}}

\hypertarget{a-einleitung-hesekiels-entruxfcckung-und-das-erscheinen-eines-himmlischen-fuxfchrers}{%
\paragraph{a) Einleitung: Hesekiels Entrückung und das Erscheinen eines
himmlischen
Führers}\label{a-einleitung-hesekiels-entruxfcckung-und-das-erscheinen-eines-himmlischen-fuxfchrers}}

\hypertarget{section-39}{%
\section{40}\label{section-39}}

\bibleverse{1}Im fünfundzwanzigsten Jahre unserer
Verbannung\textless sup title=``oder: Gefangenschaft''\textgreater✲, im
Anfang des Jahres, am zehnten Tage des Monats, im vierzehnten Jahre nach
der Eroberung der Stadt (Jerusalem) -- an eben diesem Tage kam die Hand
des HERRN über mich und brachte mich dorthin; \bibleverse{2}im Zustand
der Verzückung\textless sup title=``vgl. 8,3''\textgreater✲ führte er
mich ins Land Israel und ließ mich auf einem sehr hohen Berge nieder,
auf dessen Südseite sich ein Bauwerk nach Art einer Stadt befand.
\bibleverse{3}Als er mich dorthin gebracht hatte, da stand dort mit
einemmal ein Mann, der sah aus, als wäre er von Erz; er hatte eine
leinene Schnur und einen Meßstab in der Hand und stand im Tor.
\bibleverse{4}Dieser Mann redete mich so an: »Menschensohn, gib genau
acht mit deinen Augen und mit deinen Ohren und richte deine
Aufmerksamkeit auf alles, was ich dir zeigen werde; denn dazu bist du
hierher gebracht worden, daß man es dir zeige. Berichte dem Hause Israel
alles, was du hier zu sehen bekommst!«

\hypertarget{b-der-uxe4uuxdfere-vorhof-und-die-torbauten}{%
\paragraph{b) Der äußere Vorhof und die
Torbauten}\label{b-der-uxe4uuxdfere-vorhof-und-die-torbauten}}

\hypertarget{aa-die-uxe4uuxdfere-umfassungsmauer-und-das-uxe4uuxdfere-osttor}{%
\subparagraph{aa) Die äußere Umfassungsmauer und das äußere
Osttor}\label{aa-die-uxe4uuxdfere-umfassungsmauer-und-das-uxe4uuxdfere-osttor}}

\bibleverse{5}Da sah ich eine Mauer, die außen den Tempel(-bezirk) rings
umgab; der Meßstab aber, den der Mann in der Hand hielt, war sechs Ellen
lang, jede Elle zu einer gewöhnlichen Elle und einer Handbreite
gerechnet; mit diesem maß er die Breite\textless sup title=``oder:
Dicke''\textgreater✲ des Mauerbaues: sie betrug eine Rute und die Höhe
auch eine Rute. \bibleverse{6}Nun trat er in den Torbau, dessen
Vorderseite nach Osten zu lag; er stieg auf dessen (sieben) Stufen
hinauf und maß die Schwelle des Tores: eine Rute breit;
\bibleverse{7}sodann die (erste) Wachtstube\textless sup
title=``=~Nische für die Torwache''\textgreater✲: eine Rute lang und
eine Rute breit, und den Raum zwischen den Wachtstuben: fünf Ellen; und
die Schwelle des Tores neben der Vorhalle des Tores auf der Innenseite:
eine Rute. \bibleverse{8}Dann maß er die Vorhalle des Tores:
\bibleverse{9}acht Ellen, und ihre Wandpfeiler: zwei Ellen; die Vorhalle
des Tores lag aber nach innen zu. \bibleverse{10}Von den Wachtstuben des
Osttores lagen drei auf der einen und drei auf der andern Seite; alle
drei waren gleich groß; und ebenso hatten auch die Wandpfeiler auf
beiden Seiten einerlei Maß. \bibleverse{11}Dann maß er die Breite des
Toreingangs: zehn Ellen, und die Länge des Torweges: dreizehn Ellen.
\bibleverse{12}An der Vorderseite der Wachtstuben befand sich eine
Einfriedigung von je einer Elle Breite auf dieser wie auf jener Seite,
während die Wachtstube selbst sechs Ellen im Geviert✲ maß.
\bibleverse{13}Dann maß er das Torgebäude von der Hinterwand einer
Wachtstube bis zu der Hinterwand der gegenüberliegenden Wachtstube:
fünfundzwanzig Ellen Breite, Tür gegen Tür. \bibleverse{14}Hierauf
bestimmte er die Wandpfeiler zu sechzig Ellen✲, und an die Wandpfeiler
stieß der Vorhof rings um das Torgebäude✲, \bibleverse{15}und von der
Vorderseite des Eingangstores bis an die Vorderseite der inneren
Vorhalle des Tores waren es fünfzig Ellen. \bibleverse{16}Fenster, die
nach den Wachtstuben und nach ihren Wandpfeilern nach innen zu schräg
einfielen, befanden sich am Torgebäude ringsherum; und ebenso hatte auch
die Vorhalle Fenster ringsherum nach innen zu; an den Wandpfeilern aber
waren Palmenverzierungen (auf beiden Seiten) angebracht.

\hypertarget{bb-der-uxe4uuxdfere-vorhof}{%
\subparagraph{bb) Der äußere Vorhof}\label{bb-der-uxe4uuxdfere-vorhof}}

\bibleverse{17}Sodann führte er mich in den äußeren Vorhof, wo sich
zunächst Zellen befanden, und ein Steinpflaster war im Vorhof ringsum
hergestellt: dreißig Zellen lagen an dem Steinpflaster hin.
\bibleverse{18}Dieses Steinpflaster befand sich aber an der Seitenwand
der Tore, entsprechend der Länge der Tore, nämlich das untere
Steinpflaster. \bibleverse{19}Dann maß er die Breite des Vorhofs von der
inneren Vorderseite des unteren Tores bis zur Vorderseite des inneren
Vorhofs\textless sup title=``oder: Tores?''\textgreater✲ nach außen zu:
hundert Ellen {[}an der Ostseite und an der Nordseite{]}.

\hypertarget{cc-das-uxe4uuxdfere-nordtor-und-das-uxe4uuxdfere-suxfcdtor}{%
\subparagraph{cc) Das äußere Nordtor und das äußere
Südtor}\label{cc-das-uxe4uuxdfere-nordtor-und-das-uxe4uuxdfere-suxfcdtor}}

\bibleverse{20}(Sodann führte er mich in der Richtung nach Norden; dort
war) ein Tor, dessen Vorderseite nach Norden zu lag, am äußeren Vorhof;
auch dessen Länge und Breite maß er. \bibleverse{21}Von seinen
Wachtstuben\textless sup title=``vgl. V.7''\textgreater✲ lagen drei auf
der einen und drei auf der andern Seite; und seine Wandpfeiler und seine
Vorhalle hatten dieselben Maße wie das erste Tor: fünfzig Ellen betrug
seine Länge und fünfundzwanzig Ellen seine Breite. \bibleverse{22}Auch
seine Fenster und die Fenster seiner Vorhalle sowie seine
Palmenverzierungen waren ebenso wie beim Osttor. Auf sieben Stufen stieg
man zu ihm hinauf, und seine Vorhalle lag nach innen zu.
\bibleverse{23}Und ein Tor, das in den inneren Vorhof führte, und dem
äußeren Nordtor gegenüber lag, entsprach dem Osttor; und er maß von Tor
zu Tor hundert Ellen.

\bibleverse{24}Hierauf führte er mich in der Richtung nach Süden; und da
sah ich in der Richtung nach Süden ein Tor; er maß seine Wachtstuben,
seine Wandpfeiler und die Vorhalle und fand bei diesen dieselben Maße.
\bibleverse{25}Es waren auch Fenster darin vorhanden, ebenso wie in
seiner Vorhalle ringsum, deren Maße den schon erwähnten Fenstern gleich
waren; die Länge betrug fünfzig, die Breite fünfundzwanzig Ellen.
\bibleverse{26}Eine Treppe von sieben Stufen führte zu ihm hinauf; und
seine Vorhalle lag nach der Innenseite zu und hatte Palmenverzierungen
an ihren Wandpfeilern sowohl auf dieser wie auf jener Seite.
\bibleverse{27}Auch befand sich ein Tor zum inneren Vorhof in der
Richtung nach Süden; und er maß von dem einen Tor zum andern in der
Richtung nach Süden: hundert Ellen.

\hypertarget{c-der-innere-vorhof}{%
\paragraph{c) Der innere Vorhof}\label{c-der-innere-vorhof}}

\hypertarget{aa-die-drei-tore-zum-inneren-vorhof}{%
\subparagraph{aa) Die drei Tore zum inneren
Vorhof}\label{aa-die-drei-tore-zum-inneren-vorhof}}

\bibleverse{28}Darauf führte er mich durch das Südtor in den inneren
Vorhof und maß das Südtor aus, das die gleichen Maße hatte wie die
vorerwähnten; \bibleverse{29}auch seine Wachtstuben, seine Wandpfeiler
und seine Vorhalle wiesen die gleichen Maße auf; es besaß auch Fenster
ebenso wie seine Vorhalle ringsherum; die Länge betrug fünfzig, die
Breite fünfundzwanzig Ellen \bibleverse{30}{[}und Vorhallen lagen
ringsherum, fünfundzwanzig Ellen lang und fünf Ellen breit{]}.
\bibleverse{31}Seine Vorhalle aber lag nach dem äußeren Vorhof zu, und
Palmenverzierungen waren an seinen Wandpfeilern angebracht; eine Treppe
von acht Stufen führte zu ihm hinauf.~-- \bibleverse{32}Sodann führte er
mich {[}in den inneren Vorhof{]} zu dem Tor, das gegen Osten lag, und
maß das Tor aus: es hatte dieselben Maße wie die anderen;
\bibleverse{33}auch seine Wachtstuben, seine Wandpfeiler und seine
Vorhalle wiesen die gleichen Maße auf; es besaß auch Fenster ebenso wie
seine Vorhalle ringsherum; die Länge betrug fünfzig, die Breite
fünfundzwanzig Ellen. \bibleverse{34}Seine Vorhalle aber lag nach dem
äußeren Vorhof zu; und Palmenverzierungen waren an seinen Wandpfeilern
auf dieser wie auf jener Seite angebracht; eine Treppe von acht Stufen
führte zu ihm hinauf.~-- \bibleverse{35}Sodann führte er mich zu dem
Nordtor und maß es aus: es hatte die gleichen Maße wie die vorerwähnten;
\bibleverse{36}auch seine Wachtstuben, seine Wandpfeiler und seine
Vorhalle wiesen die gleichen Maße auf; es besaß auch Fenster ringsherum
(ebenso wie seine Vorhalle); die Länge betrug fünfzig, die Breite
fünfundzwanzig Ellen. \bibleverse{37}Seine Vorhalle lag nach dem äußeren
Vorhof zu; und Palmenverzierungen waren an seinen Wandpfeilern auf
dieser wie auf jener Seite angebracht; eine Treppe von acht Stufen
führte zu ihm hinauf.

\hypertarget{bb-die-opfervorrichtungen-im-inneren-osttor}{%
\subparagraph{bb) Die Opfervorrichtungen im inneren
Osttor}\label{bb-die-opfervorrichtungen-im-inneren-osttor}}

\bibleverse{38}Auch eine Zelle war da, deren Eingang sich an den
Pfeilern\textless sup title=``d.h. in der Vorhalle''\textgreater✲ des
Tores befand; dort hatte man das Brandopfer abzuspülen.
\bibleverse{39}In der Vorhalle des Tores aber waren zwei Tische auf der
einen Seite und zwei Tische auf der andern Seite aufgestellt, um auf
ihnen das Brandopfer, das Sünd- und Schuldopfer zu schlachten.
\bibleverse{40}Auch an der äußeren Seitenwand, an der Nordseite, wenn
man zum Toreingang hinaufstieg, standen zwei Tische, und an der anderen
Seitenwand der Vorhalle des Tores gleichfalls zwei Tische,
\bibleverse{41}also vier Tische auf der einen Seite und vier Tische auf
der andern Seite an der Seitenwand des Tores, zusammen acht Tische, auf
denen man die Schlachtopfer schlachten sollte\textless sup title=``oder:
auf die man das Fleisch der geschlachteten Opfertiere
legte''\textgreater✲. \bibleverse{42}Weiter waren noch vier Tische für
Brandopfer aus behauenen Steinen aufgestellt, anderthalb Ellen lang,
anderthalb Ellen breit und eine Elle hoch; auf diese hatte man die
Geräte zu legen, mit denen man die Brand- und Schlachtopfertiere
schlachtete. \bibleverse{43}Ringsherum an diesen Tischen, nach innen
(geneigt), waren feste Leisten angebracht, eine Handbreite hoch; auf die
Tische aber kam das Opferfleisch zu liegen.

\hypertarget{cc-die-zwei-priesterzellen-am-inneren-nord--und-suxfcdtor}{%
\subparagraph{cc) Die zwei Priesterzellen am inneren Nord- und
Südtor}\label{cc-die-zwei-priesterzellen-am-inneren-nord--und-suxfcdtor}}

\bibleverse{44}Hierauf führte er mich aus dem Tor hinaus in den inneren
Vorhof, woselbst ich zwei Zellen sah, von denen die eine an der
Seitenwand des Nordtores lag und mit ihrer Vorderseite gegen Süden
gerichtet war, während die andere an der Seitenwand des Südtores mit
ihrer Vorderseite gegen Norden gerichtet lag. \bibleverse{45}Da sagte er
zu mir: »Diese Zelle da, deren Vorderseite gegen Süden liegt, ist für
die Priester bestimmt, die den Dienst im Tempelhause verrichten;
\bibleverse{46}dagegen die Zelle, deren Vorderseite gegen Norden liegt,
ist für die Priester bestimmt, die den Dienst am Altar zu verrichten
haben« -- das sind die Nachkommen Zadoks, die (allein) von den
Nachkommen Levis dem HERRN nahen dürfen, um den Dienst vor ihm zu
verrichten.

\hypertarget{dd-gruxf6uxdfe-und-gestalt-des-inneren-vorhofs}{%
\subparagraph{dd) Größe und Gestalt des inneren
Vorhofs}\label{dd-gruxf6uxdfe-und-gestalt-des-inneren-vorhofs}}

\bibleverse{47}Dann maß er den (inneren) Vorhof aus; dieser bildete ein
Viereck von hundert Ellen Länge und hundert Ellen Breite; der Altar aber
stand vor der Vorderseite des Tempelhauses.

\hypertarget{d-das-tempelgebuxe4ude-und-seine-umgebung}{%
\paragraph{d) Das Tempelgebäude und seine
Umgebung}\label{d-das-tempelgebuxe4ude-und-seine-umgebung}}

\hypertarget{aa-beschreibung-des-tempelhauses}{%
\subparagraph{aa) Beschreibung des
Tempelhauses}\label{aa-beschreibung-des-tempelhauses}}

\bibleverse{48}Hierauf führte er mich zur Vorhalle des Tempelhauses und
maß die (beiden) Wandpfeiler der Vorhalle, von denen jeder hüben wie
drüben fünf Ellen breit war; die Breite des Tores aber betrug vierzehn
Ellen und die der Seitenwände des Tores drei Ellen auf beiden Seiten.
\bibleverse{49}Die Länge der Vorhalle betrug zwanzig Ellen und die Tiefe
zwölf Ellen; und auf zehn Stufen stieg man zu ihr hinauf. An den
Pfeilern aber standen Säulen, eine auf dieser und eine auf jener Seite.

\hypertarget{bb-die-tempelhalle-und-das-allerheiligste}{%
\subparagraph{bb) Die Tempelhalle und das
Allerheiligste}\label{bb-die-tempelhalle-und-das-allerheiligste}}

\hypertarget{section-40}{%
\section{41}\label{section-40}}

\bibleverse{1}Darauf führte er mich in die Tempelhalle hinein und maß
die Pfeiler: sechs Ellen Breite auf der einen wie auf der andern Seite.
\bibleverse{2}Die Breite des Eingangs betrug zehn Ellen und die der
Seitenwände des Eingangs beiderseits fünf Ellen. Sodann maß er die Länge
der Halle: vierzig Ellen, und die Breite: zwanzig Ellen.
\bibleverse{3}Hierauf trat er in den Innenraum hinein und maß den
Pfeiler des Eingangs: zwei Ellen Breite; und die Breite des Eingangs:
sechs Ellen; und die Seitenwände des Eingangs: sieben Ellen auf der
einen wie auf der andern Seite. \bibleverse{4}Dann maß er die Länge des
Raumes: zwanzig Ellen, und die Breite: zwanzig Ellen, entsprechend der
Breite der Tempelhalle; und er sagte dann zu mir: »Dies ist das
Allerheiligste.«

\hypertarget{cc-der-dreistuxf6ckige-seitenbau-und-das-hintergebuxe4ude}{%
\subparagraph{cc) Der dreistöckige Seitenbau und das
Hintergebäude}\label{cc-der-dreistuxf6ckige-seitenbau-und-das-hintergebuxe4ude}}

\bibleverse{5}Hierauf maß er die (Dicke der) Mauer des Tempelgebäudes:
sechs Ellen, und die Dicke (der Mauer) des seitlichen Anbaues rings um
den Tempel herum: vier Ellen. \bibleverse{6}Die Seitengemächer aber
waren, Gemach an Gemach, dreißig, und zwar in (allen) drei Stockwerken;
und es waren Absätze an der Mauer des Tempelhauses für die
Seitengemächer ringsum, um als Träger zu dienen, ohne daß sie jedoch in
die Tempelmauer selbst eingelassen waren. \bibleverse{7}Und immer
breiter wurden die Seitengemächer nach oben hin mehr und mehr,
entsprechend der der Tempelmauer abgewonnenen Erweiterung nach oben hin
rings um das Tempelhaus herum. Daher nahm die Breite des Anbaus nach
oben hin zu. Man stieg aber vom unteren Stockwerk zum obersten durch das
mittlere hinauf. \bibleverse{8}Und ich sah rings um das Tempelhaus ein
erhöhtes Pflaster herumlaufen. Der Unterbau der Seitengemächer hatte die
Höhe einer vollen Rute, sechs Ellen nach der Verbindung hin.
\bibleverse{9}Die Dicke der Außenmauer des Anbaues betrug fünf Ellen;
und was freigelassen war zwischen den Seitengemächern, die sich am Hause
befanden, \bibleverse{10}und zwischen den Zellen, hatte eine Breite von
zwanzig Ellen rings um das ganze Tempelhaus herum. \bibleverse{11}Von
den zwei Türen des seitlichen Anbaues, die nach dem freigelassenen
Raum\textless sup title=``oder: Platz''\textgreater✲ hinausführten, ging
die eine nach Norden, die andere nach Süden; die Breite des
freigelassenen Platzes aber betrug fünf Ellen ringsherum.~--
\bibleverse{12}Das Hintergebäude aber, das längs der Einfriedigung an
der Westseite lag, hatte eine Breite von siebzig Ellen; und die Wand des
Gebäudes maß fünf Ellen an Dicke ringsherum, und seine Länge betrug
neunzig Ellen.

\hypertarget{dd-die-gesamtmauxdfe-des-tempelhauses}{%
\subparagraph{dd) Die Gesamtmaße des
Tempelhauses}\label{dd-die-gesamtmauxdfe-des-tempelhauses}}

\bibleverse{13}Hierauf maß er das Tempelhaus, dessen Länge hundert Ellen
betrug; sodann den eingefriedigten Platz mit dem Hintergebäude und
seinen Mauern: eine Länge von hundert Ellen; \bibleverse{14}dann die
Breite der Vorderseite des Tempelhauses nebst dem eingefriedigten Platz
nach Osten zu: hundert Ellen.~-- \bibleverse{15}Weiter maß er die Länge
des Hintergebäudes längs der Einfriedigung, die sich auf seiner
Hinterseite befand, sowie seine Galerien auf dieser wie auf jener Seite:
hundert Ellen.

\hypertarget{ee-die-innere-ausstattung-des-tempelhauses}{%
\subparagraph{ee) Die (innere) Ausstattung des
Tempelhauses}\label{ee-die-innere-ausstattung-des-tempelhauses}}

Die Tempelhalle aber sowie das Allerheiligste und die äußere Vorhalle
\bibleverse{16}waren getäfelt, und schräg einfallende Fenster ließen
Licht in die drei Räume ringsum eindringen; und die Wände drinnen waren
ringsum mit Holz getäfelt vom Boden an bis zu den Fenstern und von den
Seitenwänden. \bibleverse{17}Oberhalb der Tür des Tempelraumes inwendig
und auswendig und auf allen Wänden ringsherum {[}innen und außen{]}
waren Bildwerke angebracht, \bibleverse{18}nämlich Cherube und Palmen,
und zwar je eine Palme zwischen zwei Cheruben. Jeder Cherub aber hatte
zwei Gesichter, \bibleverse{19}nämlich auf der einen Seite ein
Menschengesicht, das einer Palme zugekehrt war, und auf der andern Seite
ein Löwengesicht, das gleichfalls einer Palme zugekehrt war: so war es
im ganzen Tempelgebäude ringsherum gemacht: \bibleverse{20}vom Fußboden
bis über die Tür hinauf waren die Cherube und die Palmen an den
Tempelwänden angebracht. \bibleverse{21}Die Tempelhalle hatte
vierkantige Türpfosten; und an der Vorderseite des Allerheiligsten stand
ein Gegenstand, der aussah \bibleverse{22}wie ein Altar von Holz, drei
Ellen hoch, zwei Ellen lang und zwei Ellen breit; er hatte auch
Hörner\textless sup title=``=~vorspringende Ecken''\textgreater✲, und
sein Fußgestell sowie seine Wände waren von Holz. Da sagte er zu mir:
»Dies ist der Tisch, der vor dem HERRN steht.« \bibleverse{23}Die
Tempelhalle wie auch das Allerheiligste hatten zwei Flügeltüren,
\bibleverse{24}von denen jede aus zwei Türblättern bestand, die beide
drehbar waren, und zwar an beiden Türflügeln. \bibleverse{25}Auch an
ihnen waren Cherube und Palmen angebracht, geradeso wie es an den Wänden
der Fall war; und ein hölzernes Schutzdach befand sich draußen an der
Vorderseite der Vorhalle. \bibleverse{26}Schräg einfallende Fenster und
Palmenverzierungen befanden sich an beiden Seitenwänden der Vorhalle und
an den Seitenzimmern\textless sup title=``oder:
Seitenflügeln?''\textgreater✲ des Tempelhauses und an den
Schutzdächern✲.

\hypertarget{e-der-den-tempel-umgebende-bezirk-der-vorhuxf6fe-mit-seinen-baulichkeiten}{%
\paragraph{e) Der den Tempel umgebende Bezirk (der Vorhöfe) mit seinen
Baulichkeiten}\label{e-der-den-tempel-umgebende-bezirk-der-vorhuxf6fe-mit-seinen-baulichkeiten}}

\hypertarget{aa-die-nuxf6rdlichen-priesterzellen}{%
\subparagraph{aa) Die nördlichen
Priesterzellen}\label{aa-die-nuxf6rdlichen-priesterzellen}}

\hypertarget{section-41}{%
\section{42}\label{section-41}}

\bibleverse{1}Hierauf führte er mich hinaus, und zwar in den inneren
Vorhof (und dann) in der Richtung nach Norden, und brachte mich zu den
(beiden) Zellengebäuden, die dem eingefriedigten Platz gegenüber und dem
Hintergebäude nach Norden gegenüber lagen, das eine auf dieser,
\bibleverse{2}das andere auf der andern Seite. Die Länge (der
Zellengebäude) betrug hundert Ellen und die Breite fünfzig Ellen.
\bibleverse{3}Gegenüber den zwanzig Ellen des inneren Vorhofs und
gegenüber dem Steinpflaster des äußeren Vorhofs lief eine Galerie
vor\textless sup title=``oder: neben''\textgreater✲ der andern hin in
drei Stockwerken. \bibleverse{4}Vor den Zellen aber befand sich ein Gang
von zehn Ellen Breite und hundert Ellen Länge; und ihre
Türen\textless sup title=``oder: Eingänge''\textgreater✲ lagen nach
Norden zu. \bibleverse{5}Die obersten Zellen waren aber, weil die
Galerien ihnen einen Teil des Raumes wegnahmen, schmaler als die
untersten und die mittleren des Gebäudes; \bibleverse{6}denn sie waren
dreistöckig, hatten aber keine Säulen wie die Säulen auf dem äußeren
Vorhof; darum war von den untersten und mittleren Zellen Raum
weggenommen. \bibleverse{7}Eine Mauer aber war da, die an der
Außenseite, den Zellen gleichlaufend, in der Richtung nach dem äußeren
Vorhof hin lief; soweit sie den Zellen entlang lief, betrug ihre Länge
fünfzig Ellen; \bibleverse{8}denn die Länge der Zellen, die nach dem
äußeren Vorhof zu lagen, betrug fünfzig Ellen, während sie dem Tempel
gegenüber hundert Ellen betrug. \bibleverse{9}Unterhalb dieser Zellen
aber befand sich der Eingang von Osten her, wenn man zu ihnen vom
äußeren Vorhof her kam, \bibleverse{10}am Anfang der Mauer des Vorhofs.

\hypertarget{bb-die-suxfcdlichen-priesterzellen}{%
\subparagraph{bb) Die südlichen
Priesterzellen}\label{bb-die-suxfcdlichen-priesterzellen}}

(Als er mich dann) nach Süden zu (geführt hatte, sah ich, daß dort) vor
dem eingefriedigten Platz und vor dem Hintergebäude ebenfalls Zellen
lagen, \bibleverse{11}vor denen sich ein Gang befand und die dasselbe
Aussehen hatten wie die Zellen auf der Nordseite, ebenso lang und ebenso
breit wie jene; auch bezüglich ihrer Ausgänge und Einrichtungen waren
sie völlig gleichartig; und wie ihre Eingänge, \bibleverse{12}so waren
auch die Eingänge der Zellen, die nach Süden zu lagen: ein Eingang am
Anfang des Weges, der nach dem äußeren Vorhof führte, gegen Osten, wo
man hineinkam.

\hypertarget{cc-die-bestimmung-der-priesterzellen}{%
\subparagraph{cc) Die Bestimmung der
Priesterzellen}\label{cc-die-bestimmung-der-priesterzellen}}

\bibleverse{13}Da sagte er zu mir: »Die Zellen an der Nordseite und die
Zellen an der Südseite, die vor dem eingefriedigten Platz liegen, das
sind die heiligen Zellen, in denen die Priester, die dem HERRN nahen
dürfen, die hochheiligen Gaben essen sollen; dort sollen sie das
Hochheilige niederlegen, sowohl das Speisopfer als auch das Sünd- und
das Schuldopfer, denn es ist eine heilige Stätte. \bibleverse{14}Wenn
die Priester eintreten -- sie dürfen aber nicht aus dem Heiligtum
(unmittelbar) in den äußeren Vorhof hinaustreten --, so sollen sie dort
ihre Gewänder niederlegen, in denen sie den Dienst verrichtet haben;
denn diese sind heilig: sie sollen andere Kleider anziehen und dann erst
sich nach dem für das Volk bestimmten Raume begeben.«

\hypertarget{dd-abschluuxdf-der-vermessung-des-ganzen-heiligen-bezirks}{%
\subparagraph{dd) Abschluß der Vermessung des ganzen heiligen
Bezirks}\label{dd-abschluuxdf-der-vermessung-des-ganzen-heiligen-bezirks}}

\bibleverse{15}Als er nun mit der Ausmessung der inneren Tempelbauten
fertig war, führte er mich hinaus in der Richtung nach dem Tor zu,
dessen Vorderseite gegen Osten lag, und maß den ganzen Umfang aus.
\bibleverse{16}Er maß die Ostseite mit der Meßrute aus: fünfhundert
Ruten Länge im ganzen; \bibleverse{17}dann wandte er sich nach Norden
und maß die Nordseite: fünfhundert Ruten Länge im ganzen; dann ging er
herum \bibleverse{18}und maß die Südseite: fünfhundert Ruten Länge im
ganzen. \bibleverse{19}Dann ging er nach der Westseite herum und maß:
fünfhundert Ruten nach der Meßrute; \bibleverse{20}auf allen vier
Windrichtungen nahm er die Messung vor. Der ganze Tempelbezirk hatte
ringsum eine Mauer von fünfhundert Ruten Länge und fünfhundert Ruten
Breite, um den heiligen Raum von dem unheiligen zu scheiden.

\hypertarget{f-die-weihe-des-tempels-und-des-brandopferaltars}{%
\paragraph{f) Die Weihe des Tempels und des
Brandopferaltars}\label{f-die-weihe-des-tempels-und-des-brandopferaltars}}

\hypertarget{aa-gottes-einzug-in-den-neuerbauten-tempel-und-seine-heiligsprechung}{%
\subparagraph{aa) Gottes Einzug in den neuerbauten Tempel und seine
Heiligsprechung}\label{aa-gottes-einzug-in-den-neuerbauten-tempel-und-seine-heiligsprechung}}

\hypertarget{section-42}{%
\section{43}\label{section-42}}

\bibleverse{1}Als er mich hierauf zum Tor, zu dem nach Osten gerichteten
Tore, zurückgeführt hatte, \bibleverse{2}erschien plötzlich die
Herrlichkeit des Gottes Israels von Osten her; ihr Rauschen schallte wie
das Rauschen gewaltiger Wasserfluten, und die Erde\textless sup
title=``oder: das Land''\textgreater✲ leuchtete von seiner Herrlichkeit.
\bibleverse{3}Die Erscheinung, die sich mir darbot, glich der
Erscheinung, die ich geschaut hatte, als er gekommen war, um die Stadt
zu vernichten, und die Erscheinung des Wagens, den ich erblickte, war
dieselbe Erscheinung, die ich schon am Flusse Kebar gesehen hatte; und
ich warf mich auf mein Angesicht nieder. \bibleverse{4}Als dann die
Herrlichkeit des HERRN durch das nach Osten gerichtete Tor in den
Tempelbezirk eingezogen war, \bibleverse{5}hob die Gotteskraft mich
empor und brachte mich in den inneren Vorhof; und siehe da: der Tempel
war von der Herrlichkeit des HERRN erfüllt. \bibleverse{6}Da hörte ich,
wie jemand vom Tempel her mich anredete, während der Mann noch neben mir
stand. \bibleverse{7}Und jener sagte zu mir: »Menschensohn, dies ist die
Stätte meines Thrones und (dies) die Stätte meiner Fußsohlen, wo ich für
immer inmitten der Kinder Israel wohnen will. Das Haus Israel wird aber
hinfort meinen heiligen Namen nicht mehr entweihen, weder sie noch ihre
Könige, durch ihre Abgötterei und durch die Leichen ihrer Könige bei
deren Tode, \bibleverse{8}dadurch daß sie ihre\textless sup title=``d.h.
der Könige''\textgreater✲ Schwelle an meiner Schwelle und ihre
Türpfosten neben den meinigen anbrachten, so daß sich nur die Wand
zwischen mir und ihnen befand und sie so meinen heiligen Namen durch
ihre Greuel entweihten, die sie verübten, so daß ich sie in meinem Zorn
vernichtete. \bibleverse{9}Wenn sie nunmehr aber ihre Abgötterei und die
Leichen ihrer Könige fern von mir halten, dann will ich für immer unter
ihnen wohnen. \bibleverse{10}Du aber, Menschensohn, erstatte dem Hause
Israel Bericht über diesen Tempel, damit sie sich ihrer Verschuldungen
schämen. \bibleverse{11}Wenn sie sich dann alles dessen schämen, was sie
begangen haben, dann laß sie den Bauplan des Tempels und seine
Einrichtung, seine Aus- und Eingänge, seine ganze Gestalt und alle für
ihn gültigen Satzungen\textless sup title=``oder:
Anordnungen''\textgreater✲ und Bestimmungen wissen und schreibe es vor
ihren Augen auf, damit sie auf seine ganze Gestaltung und alle für ihn
gültigen Anordnungen achtgeben und sie zur Ausführung bringen.
\bibleverse{12}Dies ist die für den Tempel gültige Ordnung\textless sup
title=``oder: Bestimmung''\textgreater✲: ›Auf dem Gipfel des Berges soll
er liegen und sein ganzer Bezirk ringsum hochheilig sein!‹ Siehe, dies
ist die für den Tempel gültige Ordnung!«

\hypertarget{bb-der-brandopferaltar-und-seine-weihe}{%
\subparagraph{bb) Der Brandopferaltar und seine
Weihe}\label{bb-der-brandopferaltar-und-seine-weihe}}

\bibleverse{13}Dies nun waren die Maße des Altars nach Ellen, die Elle
zu einer gewöhnlichen Elle und einer Handbreite gerechnet: seine
Grundeinfassung war eine Elle hoch und eine Elle breit, und die
Randleiste an ihrem Rande ringsum eine Spanne hoch. Und dies war die
Höhe des Altars: \bibleverse{14}von der Grundeinfassung am Boden bis zu
der unteren Umfriedigung\textless sup title=``oder:
Einfassung''\textgreater✲ zwei Ellen Höhe und eine Elle Breite; dann von
der kleinen Umfriedigung bis zu der großen Umfriedigung vier Ellen Höhe
und eine Elle Breite; \bibleverse{15}dann der Opferherd\textless sup
title=``eig. Gottesherd''\textgreater✲ vier Ellen Höhe; und vom
Opferherd ragten die vier Hörner (eine Elle) empor. \bibleverse{16}Der
Opferherd war zwölf Ellen lang bei zwölf Ellen Breite, quadratförmig an
seinen vier Seiten. \bibleverse{17}Die große Umfriedigung\textless sup
title=``oder: Einfassung''\textgreater✲ hatte vierzehn Ellen Länge bei
vierzehn Ellen Breite, so daß ihre vier Seiten ein Quadrat bildeten; und
die Randleiste rings um sie herum war eine halbe Elle breit und die
Grundeinfassung an ihr eine Elle (breit) ringsum. Die Stufen des Altars
aber befanden sich auf der Ostseite.

\hypertarget{cc-die-einweihung-des-altars}{%
\subparagraph{cc) Die Einweihung des
Altars}\label{cc-die-einweihung-des-altars}}

\bibleverse{18}Hierauf sagte er zu mir: »Menschensohn, so spricht Gott
der HERR: ›Dies sind die Satzungen\textless sup title=``oder:
Verordnungen''\textgreater✲, die für den Altar gültig sind an dem Tage,
wo er hergestellt sein wird, daß man Brandopfer auf ihm darbringe und
Blut auf\textless sup title=``oder: an''\textgreater✲ ihn sprenge.
\bibleverse{19}Da sollst du den levitischen Priestern, die zu den
Nachkommen Zadoks gehören und die mir nahen dürfen‹ -- so lautet der
Ausspruch Gottes des HERRN --, ›um meinen Dienst zu versehen, einen
jungen Stier zum Sündopfer geben, \bibleverse{20}und sie sollen etwas
von seinem Blute nehmen und es an die vier Hörner des Altars tun und an
die vier Ecken der Umfriedigung\textless sup title=``oder:
Einfassung''\textgreater✲ und an die Randleiste ringsum, um ihn so zu
entsündigen und die Sühne für ihn zu vollziehen. \bibleverse{21}Alsdann
sollen sie den zum Sündopfer dienenden Stier nehmen, damit man ihn auf
dem dazu bestimmten Platze des Tempelbezirks außerhalb des Heiligtums
verbrenne. \bibleverse{22}Am zweiten Tage aber sollen sie einen
fehllosen Ziegenbock als Sündopfer darbringen und mit ihm den Altar
entsündigen, wie sie ihn mit dem Stier entsündigt haben.
\bibleverse{23}Wenn sie dann mit der Entsündigung fertig sind, sollen
sie einen fehllosen jungen Stier von den Rindern und einen fehllosen
Widder vom Kleinvieh opfern; \bibleverse{24}die sollen sie vor den HERRN
bringen und {[}die Priester sollen{]} Salz auf sie streuen und sie dem
HERRN als Brandopfer darbringen. \bibleverse{25}Sieben Tage lang sollen
sie so täglich einen Bock als Sündopfer darbringen; auch einen jungen
Stier von den Rindern und einen Widder vom Kleinvieh, fehllose Tiere,
sollen sie opfern. \bibleverse{26}Sieben Tage lang sollen sie so die
Entsündigung des Altars vornehmen und ihn reinigen und ihn so einweihen.
\bibleverse{27}Wenn man dann mit diesen Tagen zu Ende ist, sollen die
Priester am achten Tage und weiterhin eure Brand- und Heilsopfer auf dem
Altar darbringen, und ich werde euch gnädig annehmen!« -- so lautet der
Ausspruch Gottes des HERRN.

\hypertarget{die-religiuxf6sen-und-gottesdienstlichen-ordnungen-kap.-44-46}{%
\subsubsection{2. Die religiösen und gottesdienstlichen Ordnungen (Kap.
44-46)}\label{die-religiuxf6sen-und-gottesdienstlichen-ordnungen-kap.-44-46}}

\hypertarget{a-die-diener-des-heiligtums}{%
\paragraph{a) Die Diener des
Heiligtums}\label{a-die-diener-des-heiligtums}}

\hypertarget{aa-das-uxe4uuxdfere-osttor-des-tempelbezirks-und-seine-bestimmung-fuxfcr-den-fuxfcrsten}{%
\subparagraph{aa) Das äußere Osttor des Tempelbezirks und seine
Bestimmung für den
Fürsten}\label{aa-das-uxe4uuxdfere-osttor-des-tempelbezirks-und-seine-bestimmung-fuxfcr-den-fuxfcrsten}}

\hypertarget{section-43}{%
\section{44}\label{section-43}}

\bibleverse{1}Hierauf führte er mich zurück in der Richtung nach dem
äußeren gegen Osten gerichteten Tor des Heiligtums, das aber
verschlossen war. \bibleverse{2}Da sagte der HERR zu mir: »Dieses Tor
soll verschlossen bleiben, es darf nicht geöffnet werden, und niemand
darf durch dasselbe eingehen; weil der HERR, der Gott Israels, hier
eingezogen ist: darum soll es verschlossen bleiben! \bibleverse{3}Nur
der Fürst darf sich, eben weil er Fürst ist, darin niedersetzen, um das
Opfermahl vor dem HERRN zu halten; auf dem Wege durch die Vorhalle des
Tores soll er eintreten und auf demselben Wege wieder hinausgehen!«

\hypertarget{bb-ausscheidung-der-heiden-aus-dem-tempeldienst-und-ausschluuxdf-der-leviten-vom-priesteramt}{%
\subparagraph{bb) Ausscheidung der Heiden aus dem Tempeldienst und
Ausschluß der Leviten vom
Priesteramt}\label{bb-ausscheidung-der-heiden-aus-dem-tempeldienst-und-ausschluuxdf-der-leviten-vom-priesteramt}}

\bibleverse{4}Darauf führte er mich in der Richtung nach dem Nordtor an
die Vorderseite des Tempelhauses; und als ich hinschaute, sah ich die
Herrlichkeit des HERRN, die das Tempelhaus erfüllte. Als ich mich nun
auf mein Angesicht niedergeworfen hatte, \bibleverse{5}sagte der HERR zu
mir: »Menschensohn, gib genau acht und richte deine Augen und Ohren
aufmerksam auf alles, was ich jetzt mit dir reden werde in bezug auf
alle Verordnungen und Satzungen, die für den Tempel des HERRN Geltung
haben sollen. Richte (zunächst) deine Aufmerksamkeit darauf, wie man in
den Tempel an allen Ausgängen des Heiligtums eintreten soll.
\bibleverse{6}Sage also zum Hause Israel, zu diesem widerspenstigen
Geschlecht: ›So hat Gott der HERR gesprochen: Laßt es jetzt genug sein
mit all euren Greueln, ihr vom Hause Israel! \bibleverse{7}Ihr habt
Fremdlinge\textless sup title=``oder: Ausländer''\textgreater✲, an Herz
und Leib unbeschnittene Leute, eintreten lassen, so daß sie in meinem
Heiligtum waren und mein Haus durch ihre Anwesenheit entweihten, wenn
ihr mir meine Speise, Fett und Blut, darbrachtet, und habt dadurch
meinen Bund gebrochen zu all euren übrigen Greueln hinzu;
\bibleverse{8}und statt selbst meines heiligen Dienstes zu warten, habt
ihr sie\textless sup title=``d.h. diese Fremden''\textgreater✲ an eurer
Stelle zur Besorgung meines Dienstes in meinem Heiligtum bestellt.‹

\bibleverse{9}Darum hat Gott der HERR so gesprochen: ›Kein Ausländer,
der an Herz und Leib unbeschnitten ist, darf in mein Heiligtum kommen,
keiner von allen Ausländern, die unter den Israeliten leben.
\bibleverse{10}Vielmehr die Leviten, die mir untreu geworden sind, als
Israel abgeirrt war und nach seinem Abfall von mir hinter seinen Götzen
herlief, die sollen ihre Schuld büßen. \bibleverse{11}Sie sollen Dienste
in meinem Heiligtum leisten als Wächter an den Tempeltoren und als
Diener des Tempels; sie sollen die Brand- und Schlachtopfer für das Volk
schlachten, und sie sollen ihnen (den Volksgenossen) zur Verfügung
stehen, um ihnen Dienste zu leisten. \bibleverse{12}Weil sie ihnen vor
ihren Götzen Dienste geleistet haben und dadurch dem Hause Israel ein
Anlaß zur Verschuldung geworden sind, darum habe ich meine Hand zum
Schwur gegen sie erhoben‹ -- so lautet der Ausspruch Gottes des HERRN
--, ›daß sie ihre Schuld büßen sollen. \bibleverse{13}Sie dürfen mir
nicht nahen, um Priesterdienste vor mir zu verrichten und um an alles,
was mir heilig ist, an die hochheiligen Gegenstände, heranzutreten;
sondern sie sollen ihre Schmach tragen und für die Greuel büßen, die sie
verübt haben. \bibleverse{14}So will ich sie denn zu Wärtern für den
Tempeldienst machen, zur Besorgung aller Obliegenheiten und alles
dessen, was es in ihm zu tun gibt.‹«

\hypertarget{cc-weisungen-fuxfcr-die-rechtmuxe4uxdfigen-priester-d.h.-die-zadoksuxf6hne}{%
\subparagraph{cc) Weisungen für die rechtmäßigen Priester (d.h. die
Zadoksöhne)}\label{cc-weisungen-fuxfcr-die-rechtmuxe4uxdfigen-priester-d.h.-die-zadoksuxf6hne}}

\bibleverse{15}»›Aber die levitischen Priester, die Nachkommen Zadoks,
die des Dienstes an meinem Heiligtum treu gewartet haben, als die
Israeliten nach ihrem Abfall von mir irregingen, die sollen mir nahen,
um mir zu dienen, und sollen vor mich hintreten, um mir Fett und Blut
als Opfer darzubringen‹ -- so lautet der Ausspruch Gottes des HERRN.
\bibleverse{16}›Sie sollen in mein Heiligtum hineingehen, und sie sollen
meinem Tische nahen, um mir zu dienen, und sollen meinen Dienst
besorgen. \bibleverse{17}Wenn sie aber in die Tore des inneren Vorhofs
eintreten, dann müssen sie leinene Gewänder anlegen; von Wolle dürfen
sie nichts an sich haben, solange sie innerhalb der Tore des inneren
Vorhofs und im Tempel Dienst tun. \bibleverse{18}Kopfbunde von Leinen
sollen auf ihrem Haupt sein und Beinkleider von Leinen an ihren
Schenkeln; in schweißerregende Stoffe dürfen sie sich nicht kleiden.
\bibleverse{19}Wenn sie jedoch in den äußeren Vorhof zum Volk
hinausgehen, sollen sie ihre Gewänder, in denen sie den Dienst
verrichtet haben, ausziehen und sie in den heiligen Zellen niederlegen
und sollen andere Kleider anziehen, um nicht dem Volk durch ihre Kleider
eine Weihe mitzuteilen. \bibleverse{20}Ihr Haupt sollen sie nicht kahl
scheren, aber auch das Haar nicht lang herabhängen\textless sup
title=``oder: frei wachsen''\textgreater✲ lassen, sondern sie sollen ihr
Haupthaar geschnitten tragen. \bibleverse{21}Wein darf kein Priester
trinken, wenn sie in den inneren Vorhof hineingehen wollen.
\bibleverse{22}Mit einer Witwe oder einer entlassenen✲ Frau dürfen sie
sich nicht verheiraten, sondern nur mit Jungfrauen von israelitischer
Herkunft; eine Witwe jedoch, die ein Priester als Witwe hinterlassen
hat, dürfen sie heiraten. \bibleverse{23}Sie sollen mein Volk zwischen
Heiligem und Unheiligem unterscheiden lehren und ihm den Unterschied von
Unrein und Rein klarmachen. \bibleverse{24}Bei Rechtshändeln sollen sie
zu Gericht sitzen: auf Grund meiner Rechtsbestimmungen sollen sie
entscheiden und meine Weisungen und Satzungen an allen meinen Festen
beobachten und meine Sabbate heilig halten.‹«

\hypertarget{dd-verhalten-der-priester-bei-todesfuxe4llen}{%
\subparagraph{dd) Verhalten der Priester bei
Todesfällen}\label{dd-verhalten-der-priester-bei-todesfuxe4llen}}

\bibleverse{25}»An die Leiche eines Menschen dürfen sie nicht
herantreten, weil sie sich dadurch verunreinigen würden; nur an der
Leiche von Vater und Mutter, von Sohn und Tochter, von Bruder und
Schwester, sofern diese unverheiratet geblieben ist, dürfen sie sich
verunreinigen. \bibleverse{26}Wenn der Betreffende dann wieder rein
geworden ist, soll er sich noch sieben Tage abzählen; \bibleverse{27}und
an dem Tage, an dem er das Heiligtum, nämlich den inneren Vorhof, wieder
betritt, um im Heiligtum Dienst zu tun, soll er ein Sündopfer für sich
darbringen‹« -- so lautet der Ausspruch Gottes des HERRN.

\hypertarget{ee-erbbesitz-eigentum-und-einkuxfcnfte-der-priester}{%
\subparagraph{ee) Erbbesitz, Eigentum und Einkünfte der
Priester}\label{ee-erbbesitz-eigentum-und-einkuxfcnfte-der-priester}}

\bibleverse{28}»›Erbbesitz sollen sie nicht haben: ich bin ihr
Erbbesitz; und ein Besitztum sollt ihr ihnen in Israel nicht geben: ich
bin ihr Besitztum! \bibleverse{29}Sie sind es, die das Speisopfer, das
Sündopfer und das Schuldopfer verzehren sollen, und alles, was in Israel
dem Bann verfällt, soll ihnen gehören. \bibleverse{30}Auch das Beste von
allen Erstlingen jedweder Art und alle eure Hebopfer✲ jedweder Art ohne
Ausnahme sollen den Priestern zufallen; und auch das Beste von eurem
Schrotmehl sollt ihr dem Priester geben, damit Segen auf euren Häusern
ruht. \bibleverse{31}Fleisch von verendeten oder zerrissenen Tieren,
seien es Vögel oder Vierfüßler, dürfen die Priester nicht essen.‹«

\hypertarget{b-gemischte-weisungen-besonders-opferverordnungen}{%
\paragraph{b) Gemischte Weisungen, besonders
Opferverordnungen}\label{b-gemischte-weisungen-besonders-opferverordnungen}}

\hypertarget{aa-uxfcberweisung-eines-heiligen-bezirks-an-die-priester-und-die-leviten-sowie-eines-landanteils-an-die-stadt-und-den-fuxfcrsten}{%
\subparagraph{aa) Überweisung eines heiligen Bezirks an die Priester und
die Leviten sowie eines Landanteils an die Stadt und den
Fürsten}\label{aa-uxfcberweisung-eines-heiligen-bezirks-an-die-priester-und-die-leviten-sowie-eines-landanteils-an-die-stadt-und-den-fuxfcrsten}}

\hypertarget{section-44}{%
\section{45}\label{section-44}}

\bibleverse{1}»Wenn ihr dann das Land zum Erbbesitz verlost, sollt ihr
für den HERRN ein Hebopfer\textless sup title=``d.h. einen Anteil am
Opfer''\textgreater✲ vorwegnehmen, eine Weihegabe, nämlich einen
Landstrich von 25000 Ellen Länge und 20000 Ellen Breite: das soll in
seinem ganzen Umfange ein heiliger Bezirk sein. \bibleverse{2}Von diesem
Gebiet soll auf den Tempel entfallen ein Geviert✲ von 500 Ellen
Seitenlänge; und rings um dieses soll ein freier Raum von fünfzig Ellen
gelassen werden. \bibleverse{3}Weiter sollst du von jenem abgemessenen
Bezirk ein Stück von 25000 Ellen Länge und 10000 Ellen Breite abmessen,
auf welches das Heiligtum als Hochheiliges zu stehen kommt.
\bibleverse{4}Dies soll der heilige Teil vom Lande sein: den Priestern,
die den Dienst im Heiligtum zu verrichten haben und die dem HERRN nahen,
um ihm zu dienen, soll er gehören und als Platz für Häuser von ihnen
verwandt werden und als heiliger Raum zum Heiligtum gehören.~--
\bibleverse{5}Ferner soll ein Bezirk von 25000 Ellen Länge und 10000
Ellen Breite den Leviten, den Tempeldienern\textless sup title=``vgl.
44,10-14''\textgreater✲, als Eigentum überwiesen werden, für Ortschaften
zum Bewohnen.~-- \bibleverse{6}Ferner sollt ihr der Stadt ein Gebiet von
5000 Ellen Breite und 25000 Ellen Länge als Eigentum überweisen; es soll
sich zur Seite des heiligen Weihebezirks erstrecken und dem ganzen Hause
Israel gehören.~-- \bibleverse{7}Dem Fürsten endlich sollt ihr
Landbesitz zuweisen auf beiden Seiten des heiligen Weihebezirks und des
Eigentums der Stadt, längs dieser beiden Bezirke sowohl auf der
Westseite westwärts als auch auf der Ostseite ostwärts und in der Länge
entsprechend jedem einzelnen\textless sup title=``=~genauso wie
jeder''\textgreater✲ der (Stammes-) Anteile von der Westgrenze bis zur
Ostgrenze \bibleverse{8}des Landes. Das soll ihm als Eigentum in Israel
gehören, damit meine Fürsten hinfort mein Volk nicht mehr bedrücken,
sondern das (übrige) Land dem Hause Israel nach seinen Stämmen
überlassen.«

\hypertarget{bb-die-pflicht-des-fuxfcrsten-gerechtigkeit-zu-uxfcben-und-fuxfcr-gerechtigkeit-zu-sorgen}{%
\subparagraph{bb) Die Pflicht des Fürsten, Gerechtigkeit zu üben und für
Gerechtigkeit zu
sorgen}\label{bb-die-pflicht-des-fuxfcrsten-gerechtigkeit-zu-uxfcben-und-fuxfcr-gerechtigkeit-zu-sorgen}}

\bibleverse{9}So hat Gott der HERR gesprochen: »Laßt es nun genug sein,
ihr Fürsten Israels! Steht ab von Gewalttätigkeit und Bedrückung, übt
vielmehr Recht und Gerechtigkeit und laßt mein Volk nicht mehr unter
euren Besitzräubereien leiden!« -- so lautet der Ausspruch Gottes des
HERRN. \bibleverse{10}»Richtige Waage, richtiges Epha und richtiges Bath
sollt ihr führen\textless sup title=``vgl. 3.Mose 19,36; 5.Mose
25,15''\textgreater✲. \bibleverse{11}Das Epha und das Bath sollen
gleichen Gehalt\textless sup title=``=~einerlei Maß''\textgreater✲
haben, so daß das Bath den zehnten Teil eines Chomers
beträgt\textless sup title=``oder: faßt''\textgreater✲ und das Epha
ebenfalls den zehnten Teil eines Chomers: nach dem Chomer soll ihre
Maßbestimmung erfolgen. \bibleverse{12}Weiter soll der Schekel zwanzig
Gera betragen; fünf Schekel sollen fünf Schekel sein, und zehn Schekel
sollen zehn Schekel sein, und fünfzig Schekel soll bei euch die Mine
gelten.«

\hypertarget{cc-die-an-den-fuxfcrsten-zu-entrichtenden-abgaben-und-die-von-ihm-darzubringenden-opfer}{%
\subparagraph{cc) Die an den Fürsten zu entrichtenden Abgaben und die
von ihm darzubringenden
Opfer}\label{cc-die-an-den-fuxfcrsten-zu-entrichtenden-abgaben-und-die-von-ihm-darzubringenden-opfer}}

\bibleverse{13}»Folgendes ist die Hebe\textless sup title=``d.h.
feststehende Abgabe''\textgreater✲, die ihr entrichten sollt: ein
Sechstel Epha von jedem Chomer Weizen und ein Sechstel Epha von jedem
Chomer Gerste. \bibleverse{14}Sodann soll die Gebühr beim Öl ein Zehntel
Bath von jedem Kor betragen {[}denn zehn Bath machen ein Kor aus{]}.
\bibleverse{15}Ferner: ein Stück Kleinvieh aus einer Herde von
zweihundert Stück, als Hebe von allen Geschlechtern Israels zum Speis-
und zum Brandopfer und zum Heilsopfer, um euch Sühne zu erwirken« -- so
lautet der Ausspruch Gottes des HERRN. \bibleverse{16}»Das ganze Volk
des Landes soll zu dieser Abgabe an den Fürsten in Israel verpflichtet
sein. \bibleverse{17}Dem Fürsten dagegen sollen das Brandopfer, das
Speis- und das Trankopfer an den Festen sowie an den Neumonden und
Sabbaten (und) bei allen Festversammlungen des Hauses Israel obliegen.
Er hat das Sündopfer, das Speis- und Brandopfer und die Heilsopfer
auszurichten, um dem Hause Israel Sühne zu erwirken.«

\hypertarget{dd-die-feste-und-opfer}{%
\subparagraph{dd) Die Feste und Opfer}\label{dd-die-feste-und-opfer}}

\bibleverse{18}So hat Gott der HERR gesprochen: »Im ersten Monat, am
ersten Tage des Monats, sollt ihr einen fehllosen jungen Stier zur
Entsündigung des Heiligtums nehmen. \bibleverse{19}Der Priester soll
dann etwas von dem Blut des Sündopfers nehmen und es an die Türpfosten
des Tempelhauses und an die vier Ecken der Umfriedigung\textless sup
title=``oder: Einfassung''\textgreater✲ des Altars und an die Pfosten
des Tores zum inneren Vorhof tun; \bibleverse{20}und ebenso sollt ihr es
am ersten Tage des siebten Monats machen mit Rücksicht auf die, welche
sich unabsichtlich oder unwissentlich versündigt haben, und so sollt ihr
Sühne für den Tempel erwirken.

\bibleverse{21}Am vierzehnten Tage des ersten Monats sollt ihr das
Passah feiern, das siebentägige Fest, während dessen Dauer ungesäuertes
Brot gegessen werden soll. \bibleverse{22}Der Fürst aber hat an diesem
Tage für sich selbst und für das gesamte Volk des Landes einen Stier als
Sündopfer darzubringen \bibleverse{23}und soll an den sieben Tagen des
Festes dem HERRN sieben Stiere und sieben Widder, fehllose Tiere,
täglich während der sieben Tage als Brandopfer darbringen, und als
Sündopfer täglich einen Ziegenbock. \bibleverse{24}Als Speisopfer aber
soll er je ein Epha Feinmehl zu jedem Stier und je ein Epha zu jedem
Widder opfern und an Öl je ein Hin auf jedes Epha.

\bibleverse{25}Am fünfzehnten Tage des siebten Monats, am
Feste\textless sup title=``d.h. am Laubhüttenfest''\textgreater✲, soll
er die sieben Tage hindurch die gleichen Opfergaben herrichten, sowohl
Sündopfer als auch Brand- und Speisopfer und Öl.«

\hypertarget{ee-vorschriften-fuxfcr-die-sabbat--und-neumondfeier}{%
\subparagraph{ee) Vorschriften für die Sabbat- und
Neumondfeier}\label{ee-vorschriften-fuxfcr-die-sabbat--und-neumondfeier}}

\hypertarget{section-45}{%
\section{46}\label{section-45}}

\bibleverse{1}So hat Gott der HERR gesprochen: »Das Tor des inneren
Vorhofs, das nach Osten zu liegt, soll während der sechs Werktage
geschlossen bleiben; aber am Sabbattage und ebenso am Neumondtage soll
es geöffnet werden. \bibleverse{2}Wenn dann der Fürst durch die Vorhalle
des Tores von außen her eingetreten und am Pfosten\textless sup
title=``oder: an der Schwelle''\textgreater✲ des Tores stehen geblieben
ist, dann sollen die Priester sein Brandopfer und sein Heilsopfer
darbringen; nachdem er dann auf der Schwelle des Tores die Anbetung
verrichtet hat und wieder hinausgegangen ist, soll das Tor bis zum Abend
unverschlossen bleiben; \bibleverse{3}und auch das Volk des Landes soll
am Eingang dieses Tores an den Sabbaten und Neumonden vor dem HERRN
anbeten.~-- \bibleverse{4}Das Brandopfer aber, das der Fürst dem HERRN
am Sabbattage darzubringen hat, soll aus sechs Lämmern ohne Fehl und
einem Widder ohne Fehl bestehen; \bibleverse{5}dazu kommt als Speisopfer
ein Epha Feinmehl zu jedem Widder, und zu den Lämmern eine beliebig
große Gabe von Mehl; außerdem ein Hin Öl zu jedem Epha.
\bibleverse{6}Ferner soll er am Neumondstage einen fehllosen jungen
Stier von den Rindern und sechs Lämmer sowie einen Widder opfern, lauter
fehllose Tiere; \bibleverse{7}dazu als Speisopfer zu dem Stier und zu
dem Widder je ein Epha Feinmehl; und zu den Lämmern eine beliebig große
Gabe; an Öl aber ein Hin zu jedem Epha.«

\hypertarget{ff-vorschriften-uxfcber-das-aus--und-eingehen-des-fuxfcrsten-und-des-volkes-in-den-tempelbezirk}{%
\subparagraph{ff) Vorschriften über das Aus- und Eingehen des Fürsten
und des Volkes in den
Tempelbezirk}\label{ff-vorschriften-uxfcber-das-aus--und-eingehen-des-fuxfcrsten-und-des-volkes-in-den-tempelbezirk}}

\bibleverse{8}»Wenn aber der Fürst sich einfindet, so soll er durch die
Vorhalle des Tores eintreten und auf demselben Wege sich wieder
entfernen. \bibleverse{9}Wenn dagegen das Volk des Landes✲ an den Festen
vor dem HERRN erscheint, so soll, wer durch das Nordtor eingetreten ist,
um anzubeten, durch das Südtor wieder hinausgehen, und wer durch das
Südtor eingetreten ist, soll durch das Nordtor wieder hinausgehen:
niemand soll durch dasselbe Tor zurückkehren, durch das er
hereingekommen ist, sondern er soll durch das gegenüberliegende
hinausgehen; \bibleverse{10}der Fürst aber soll, wenn sie eintreten, in
ihrer Mitte eintreten; und wenn sie hinausgehen, sollen sie zusammen
hinausgehen.«

\hypertarget{gg-vereinzelte-opfervorschriften}{%
\subparagraph{gg) Vereinzelte
Opfervorschriften}\label{gg-vereinzelte-opfervorschriften}}

\bibleverse{11}»An den Festen aber und an den Feiertagen soll das
Speisopfer ein Epha Feinmehl sowohl auf jeden Stier als auch auf jeden
Widder betragen, und für die Lämmer eine beliebig große Gabe; an Öl aber
ein Hin auf jedes Epha.~-- \bibleverse{12}Wenn ferner der Fürst dem
HERRN ein Brandopfer oder Heilsopfer als freiwillige Gabe darbringen
will, so soll man ihm das gegen Osten liegende Tor öffnen, und er bringe
dann sein Brandopfer und sein Heilsopfer in derselben Weise dar, wie er
es am Sabbattage zu tun pflegt; und wenn er dann hinausgegangen ist,
soll man das Tor nach seinem Weggang wieder schließen.~--
\bibleverse{13}Ferner soll er dem HERRN täglich ein fehlloses
einjähriges Lamm als Brandopfer darbringen; jeden Morgen soll er es
herrichten, \bibleverse{14}und dazu als Speisopfer alle Morgen ein
Sechstel Epha Feinmehl und an Öl ein Drittel Hin zur
Befeuchtung\textless sup title=``oder: Besprengung''\textgreater✲ des
Feinmehls als Speisopfer dem HERRN darbringen: das ist eine ständige
Satzung für alle Zukunft. \bibleverse{15}So habt ihr also das Lamm nebst
dem Speisopfer und dem Öl alle Morgen als regelmäßiges Brandopfer
darzubringen.«

\hypertarget{hh-nachtruxe4gliche-verordnung-uxfcber-den-grundbesitz-des-fuxfcrsten}{%
\subparagraph{hh) Nachträgliche Verordnung über den Grundbesitz des
Fürsten}\label{hh-nachtruxe4gliche-verordnung-uxfcber-den-grundbesitz-des-fuxfcrsten}}

\bibleverse{16}So hat Gott der HERR gesprochen: »Wenn der Fürst einem
seiner Söhne ein Geschenk von seinem Erbbesitz macht, so soll dies
seinen Söhnen als vererbliches Eigentum gehören. \bibleverse{17}Macht er
aber einem seiner Diener ein Geschenk von seinem Erbbesitz, so soll es
diesem nur bis zum Freijahr\textless sup title=``=~Jahr der
Freilassung''\textgreater✲ gehören, dann aber wieder an den Fürsten
zurückfallen; nur seinen Söhnen soll es als Erbbesitz dauernd
verbleiben. \bibleverse{18}Von dem Grundbesitz des Volkes aber darf der
Fürst nichts wegnehmen, so daß er sie gewaltsam aus ihrem Eigentum
verdrängt. Von seinem eigenen Besitz mag er seine Söhne mit vererblichem
Grundbesitz ausstatten, damit niemand von den zu meinem Volk Gehörigen
aus seinem Eigentum verdrängt wird.«

\hypertarget{ii-die-opferkuxfcchen-der-priester-und-des-volkes-im-tempelbezirk}{%
\subparagraph{ii) Die Opferküchen der Priester und des Volkes im
Tempelbezirk}\label{ii-die-opferkuxfcchen-der-priester-und-des-volkes-im-tempelbezirk}}

\bibleverse{19}Darauf führte er mich durch den Eingang, der an der
Seitenwand des Tores lag, zu den heiligen, für die Priester bestimmten
Zellen, die nach Norden zu gelegen waren; dort sah ich einen Raum ganz
hinten in dem Winkel nach Westen zu. \bibleverse{20}Da sagte er zu mir:
»Dies ist der Raum, in welchem die Priester das Schuld- und das
Sündopfer kochen und wo sie das Speisopfer backen sollen; sonst müßten
sie es in den äußeren Vorhof hinaustragen, wodurch sie dem Volk eine
Weihe mitteilen würden.«

\bibleverse{21}Hierauf führte er mich in den äußeren Vorhof hinaus und
ließ mich ihn nach seinen vier Ecken hin durchqueren; dabei bemerkte ich
einen kleinen Hof in jeder Ecke des Vorhofs. \bibleverse{22}In allen
vier Ecken des Vorhofs waren abgesonderte\textless sup title=``oder:
kleine''\textgreater✲ Höfe von vierzig Ellen Länge und dreißig Ellen
Breite; alle vier Eckräume hatten die gleiche Größe; \bibleverse{23}und
es lief eine gemauerte Steinwand rings um alle vier; und unten an den
Steinwänden waren ringsum Kochherde angebracht. \bibleverse{24}Da sagte
er zu mir: »Dies sind die Küchen, wo die Tempeldiener\textless sup
title=``vgl. 44,10-14''\textgreater✲ die Schlachtopfer des Volkes kochen
müssen.«

\hypertarget{die-wunderbare-tempelquelle-als-segensstrom}{%
\subsubsection{3. Die wunderbare Tempelquelle als
Segensstrom}\label{die-wunderbare-tempelquelle-als-segensstrom}}

\hypertarget{section-46}{%
\section{47}\label{section-46}}

\bibleverse{1}Als er mich hierauf an den Eingang des Tempelhauses
zurückgeführt hatte, sah ich Wasser unter der Schwelle des Tempels
hervorfließen nach Osten hin -- die Vorderseite des Tempels lag ja nach
Osten zu --; und das Wasser floß unterhalb der südlichen Seitenwand des
Tempelhauses hinab, südlich vom Altar. \bibleverse{2}Als er mich dann
durch das Nordtor hinausgeführt und mich auf dem Wege draußen zu dem
äußeren, nach Osten gerichteten Tor hatte herumgehen lassen, sah ich
dort Wasser von der südlichen Seitenwand herrieseln. \bibleverse{3}Indem
dann der Mann mit einer Meßschnur in der Hand nach Osten zu weiterging
und nach Abmessung von tausend Ellen mich durch das Wasser gehen ließ,
ging mir das Wasser bis an die Knöchel; \bibleverse{4}als er dann
nochmals tausend Ellen abgemessen hatte und mich durch das Wasser gehen
hieß, ging mir das Wasser bis an die Knie; als er hierauf nochmals
tausend Ellen abgemessen hatte und mich hindurchgehen hieß, ging mir das
Wasser bis an die Hüften; \bibleverse{5}und nach nochmaliger Abmessung
von tausend Ellen war es ein Fluß geworden, den man nicht mehr
durchschreiten konnte; denn das Wasser war so tief geworden, daß man es
hätte durchschwimmen müssen, ein Fluß, der sich nicht mehr
durchschreiten ließ. \bibleverse{6}Da fragte er mich: »Hast du das wohl
gesehen, Menschensohn?« Dann ließ er mich am Ufer des Flusses wieder
zurückwandern. \bibleverse{7}Auf dem Rückwege sah ich nun am Ufer des
Flusses auf beiden Seiten sehr viele Bäume stehen. \bibleverse{8}Da
sagte er zu mir: »Dieses Gewässer fließt in den östlichen Bezirk hinaus,
strömt dann in die Jordan-Ebene hinab und mündet in das (Tote) Meer; und
wo es sich dort hinein ergießt, da wird das Salzwasser des (Toten)
Meeres gesund. \bibleverse{9}Und alle lebenden Wesen, alles, was dort
wimmelt, wird, wohin immer (der Fluß) kommt, Leben gewinnen; und der
Fischreichtum wird überaus groß sein; denn wenn dieses Gewässer dorthin
kommt, so wird das Wasser (des Toten Meeres) gesund werden, und alles,
wohin der Fluß kommt, wird Leben gewinnen. \bibleverse{10}Auch Fischer
werden an ihm stehen: von En-Gedi bis En-Eglaim wird es Plätze zum
Auswerfen der Netze geben, und sein Fischreichtum wird wie der des
großen Meeres überaus groß sein. \bibleverse{11}Aber seine Lachen und
Tümpel werden nicht gesund werden: sie sind zur Salzgewinnung bestimmt.
\bibleverse{12}An dem Flusse aber werden an seinem Ufer auf beiden
Seiten allerlei Bäume mit eßbaren Früchten wachsen, Bäume, deren Laub
nicht verwelkt und deren Früchte nicht ausgehen. Alle Monate werden sie
reife\textless sup title=``oder: frische''\textgreater✲ Früchte tragen;
denn das Wasser, an dem sie stehen, fließt aus dem Heiligtum hervor;
daher werden ihre Früchte zur Nahrung dienen und ihre Blätter zu
Heilzwecken.«

\hypertarget{das-heilige-land-und-die-heilige-stadt}{%
\subsubsection{4. Das heilige Land und die heilige
Stadt}\label{das-heilige-land-und-die-heilige-stadt}}

\hypertarget{a-die-grenzen-des-landes}{%
\paragraph{a) Die Grenzen des Landes}\label{a-die-grenzen-des-landes}}

\bibleverse{13}So hat Gott der HERR gesprochen: »Dies ist die Grenze,
innerhalb derer ihr das Land nach den zwölf Stämmen Israels als
Erbbesitz zugeteilt erhalten sollt {[}-- für Josef zwei Anteile~--{]}.
\bibleverse{14}Und zwar sollt ihr es zu gleichen Teilen als Erbbesitz
erhalten, der eine wie der andere, weil ich einst meine Hand zum Schwur
erhoben habe, es euren Vätern zu verleihen; daher soll dies Land euch
als Erbbesitz zufallen. \bibleverse{15}Dies soll aber die Grenze des
Landes sein: Auf der Nordseite vom großen Meer an in der Richtung auf
Hethlon zu bis wo es nach Hamath hineingeht, nach Zedad hin,
\bibleverse{16}Berotha, Sibraim, das zwischen dem Gebiet von Damaskus
und dem Gebiet von Hamath liegt, nach Hazar-Enon, das an der Grenze von
Hauran liegt. \bibleverse{17}Die Grenze soll also vom Meer an bis
Hazar-Enon laufen, so daß das Gebiet von Damaskus nördlich davon liegen
bleibt und ebenso das Gebiet von Hamath: dies ist die Nordseite.
\bibleverse{18}Was sodann die Ostseite betrifft, so läuft die Grenze von
Hazar-Enon an, das zwischen Hauran und Damaskus liegt, und wird zwischen
Gilead und dem Lande Israel durch den Jordan gebildet bis zum östlichen
(Toten) Meer, bis nach Thamar hin: dies ist die Ostseite.
\bibleverse{19}Sodann die Südseite gegen Mittag geht von Thamar bis zum
Haderwasser bei Kades nach dem Bach (Ägyptens) hin (und diesen entlang)
bis an das große Meer: dies ist die Südseite gegen Mittag.
\bibleverse{20}Auf der Westseite aber bildet das große Meer die Grenze
bis gerade gegenüber der Stelle, wo es nach Hamath hineingeht: dies ist
die Westseite.«

\hypertarget{b-die-verteilung-des-landes-an-die-zwuxf6lf-stuxe4mme}{%
\paragraph{b) Die Verteilung des Landes an die zwölf
Stämme}\label{b-die-verteilung-des-landes-an-die-zwuxf6lf-stuxe4mme}}

\hypertarget{aa-behandlung-der-fremdlinge-bei-der-verteilung}{%
\subparagraph{aa) Behandlung der Fremdlinge bei der
Verteilung}\label{aa-behandlung-der-fremdlinge-bei-der-verteilung}}

\bibleverse{21}»Dieses Land also sollt ihr unter euch verteilen nach den
Stämmen Israels; \bibleverse{22}und zwar sollt ihr es als vererbliches
Eigentum unter euch und die Fremdlinge, die unter euch wohnen und
Familien unter euch gegründet haben, verlosen: sie sollen euch wie
eingeborene Israeliten gelten: mit euch sollen sie um Erbbesitz inmitten
der Stämme Israels losen; \bibleverse{23}und zwar sollt ihr jedem
Fremdling seinen Erbbesitz in dem Stamme zuweisen, bei dem er wohnt« --
so lautet der Ausspruch Gottes des HERRN.

\hypertarget{bb-die-sieben-stuxe4mme-nuxf6rdlich-vom-heiligen-weihebezirk}{%
\subparagraph{bb) Die sieben Stämme nördlich vom heiligen
Weihebezirk}\label{bb-die-sieben-stuxe4mme-nuxf6rdlich-vom-heiligen-weihebezirk}}

\hypertarget{section-47}{%
\section{48}\label{section-47}}

\bibleverse{1}»Dies sind nun die Namen der Stämme: Im äußersten Norden,
vom Meere an in der Richtung nach Hethlon bis dahin, wo es nach Hamath
hineingeht, und bis hin nach Hazar-Enon -- das Gebiet von Damaskus aber
bleibt nordwärts liegen, seitwärts von Hamath --, von der Ostseite bis
zur Westseite, erhält Dan ein Stammgebiet. \bibleverse{2}Neben dem
Gebiet Dans, von der Ostseite bis zur Westseite, erhält Asser ein
Stammgebiet. \bibleverse{3}Neben dem Gebiet Assers, von der Ostseite bis
zur Westseite, erhält Naphthali ein Stammgebiet. \bibleverse{4}Neben dem
Gebiete Naphthalis, von der Ostseite bis zur Westseite, erhält Manasse
ein Stammgebiet. \bibleverse{5}Neben dem Gebiet Manasses, von der
Ostseite bis zur Westseite, erhält Ephraim ein Stammgebiet.
\bibleverse{6}Neben dem Gebiet Ephraims, von der Ostseite bis zur
Westseite, erhält Ruben ein Stammgebiet. \bibleverse{7}Neben dem Gebiet
Rubens, von der Ostseite bis zur Westseite, erhält Juda ein
Stammgebiet.«

\hypertarget{cc-der-heilige-weihebezirk}{%
\subparagraph{cc) Der heilige
Weihebezirk}\label{cc-der-heilige-weihebezirk}}

\bibleverse{8}»Neben dem Gebiete Judas aber, von der Ostseite bis zur
Westseite, soll der Weiheteil liegen, den ihr abzugeben habt. 25000
Ellen an Breite und so lang wie jeder Stammesanteil von der Ostseite bis
zur Westseite; und das Heiligtum soll mitten darin liegen.
\bibleverse{9}Der Weiheteil, den ihr für den HERRN abzugeben habt, soll
25000 Ellen lang und 20000 Ellen breit sein; \bibleverse{10}und
folgenden Besitzern soll der heilige Weihebezirk gehören: den Priestern
ein Stück\textless sup title=``oder: Gebiet''\textgreater✲ nach Norden
25000 (Ellen) an Länge, nach Westen 10000 Ellen an Breite, nach Osten
10000 Ellen an Breite und nach Süden 25000 Ellen an Länge; und das
Heiligtum des HERRN soll mitten darin liegen. \bibleverse{11}Den
geweihten Priestern, den Nachkommen Zadoks, die meinen Dienst verrichtet
haben und die nicht irregegangen sind, als die Israeliten zusammen mit
den Leviten von mir abfielen, \bibleverse{12}ihnen soll es als ein
geweihtes Stück von dem Weiheteil des Landes gehören, als Hochheiliges,
neben dem Gebiet der Leviten. \bibleverse{13}Die Leviten aber sollen
neben\textless sup title=``oder: entsprechend''\textgreater✲ dem Gebiet
der Priester ein Gebiet von 25000 Ellen Länge und 10000 Ellen Breite
erhalten; im ganzen soll also die Länge 25000 und die Breite 20000 Ellen
betragen. \bibleverse{14}Davon dürfen sie jedoch nichts verkaufen und
nichts vertauschen und den wertvollsten Teil des Landes nicht in fremden
Besitz übergehen lassen, denn er ist dem HERRN heilig.
\bibleverse{15}Die 5000 Ellen aber, die an der Breitseite längs der
25000 Ellen noch übrig sind, sollen nichtheiliger Gemeinbesitz der Stadt
sein und zum Bewohnen✲ und als Gemeindetrift dienen; und die Stadt soll
mitten darin liegen. \bibleverse{16}Deren Maße sollen folgende sein: die
Nordseite 4500 Ellen, die Südseite 4500, die Ostseite 4500 und die
Westseite 4500 Ellen; \bibleverse{17}und die Gemeindetrift der Stadt
soll im Norden 250 Ellen, im Süden 250, im Osten 250 und im Westen 250
Ellen betragen. \bibleverse{18}Was dann von der Länge noch übrig ist
längs dem heiligen Weihebezirk, nämlich 10000 Ellen nach Osten und 10000
nach Westen, dessen Ertrag soll den Arbeitern der Stadt zur Ernährung
dienen; \bibleverse{19}die Arbeiter der Stadt aber sollen Leute aus
allen Stämmen Israels sein. \bibleverse{20}Insgesamt sollt ihr (also)
als Weihegabe 25000 Ellen ins Geviert\textless sup title=``=~in Form
eines Quadrates''\textgreater✲ abgeben, nämlich den heiligen Weihebezirk
nebst dem der Stadt gehörigen Grundbesitz. \bibleverse{21}Was dann noch
übrig ist, soll dem Fürsten gehören, (nämlich das Gebiet) auf beiden
Seiten des heiligen Weihebezirks und des der Stadt gehörigen
Grundbesitzes (ostwärts) längs der 25000 Ellen bis zur Ostgrenze und
westwärts längs der 25000 Ellen bis zur Westgrenze, entsprechend den
Stammesanteilen: dem Fürsten soll es gehören, und der heilige
Weihebezirk mit dem Tempelheiligtum soll mitten darin liegen.
\bibleverse{22}

\hypertarget{dd-die-fuxfcnf-stuxe4mme-suxfcdlich-vom-heiligen-weihebezirk}{%
\subparagraph{dd) Die fünf Stämme südlich vom heiligen
Weihebezirk}\label{dd-die-fuxfcnf-stuxe4mme-suxfcdlich-vom-heiligen-weihebezirk}}

\bibleverse{23}»Was sodann die übrigen Stämme betrifft, so erhält von
der Ostseite bis zur Westseite Benjamin ein Stammgebiet.
\bibleverse{24}Neben dem Gebiet Benjamins, von der Ostseite bis zur
Westseite, erhält Simeon ein Stammgebiet. \bibleverse{25}Neben dem
Gebiet Simeons, von der Ostseite bis zur Westseite, erhält Issaschar ein
Stammgebiet. \bibleverse{26}Neben dem Gebiet Issaschars, von der
Ostseite bis zur Westseite, erhält Sebulon ein Stammgebiet.
\bibleverse{27}Neben dem Gebiet Sebulons, von der Ostseite bis zur
Westseite, erhält Gad ein Stammgebiet. \bibleverse{28}Neben dem Gebiet
Gads aber, auf der Südseite, nach Mittag zu, da soll die Grenze von
Thamar an bis zum Haderwasser bei Kades nach dem Bach Ägyptens hin (und
diesem entlang) bis an das große Meer gehen.

\bibleverse{29}Dies ist das Land, das ihr als Erbbesitz unter die Stämme
Israels verlosen sollt, und dies sollen ihre Anteile sein« -- so lautet
der Ausspruch Gottes, des HERRN.

\hypertarget{c-die-heilige-stadt-ihre-zwuxf6lf-tore-ihr-umfang-und-ihr-name}{%
\paragraph{c) Die heilige Stadt, ihre zwölf Tore, ihr Umfang und ihr
Name}\label{c-die-heilige-stadt-ihre-zwuxf6lf-tore-ihr-umfang-und-ihr-name}}

\bibleverse{30}»Dies sollen aber die Ausgänge der Stadt sein -- und zwar
sollen die Tore der Stadt nach den Stämmen Israels benannt sein --:
\bibleverse{31}Auf der Nordseite, die eine Länge von 4500 Ellen hat,
sollen drei Tore liegen: das Rubentor, das Judator, das Levitor;
\bibleverse{32}auf der Ostseite, die eine Länge von 4500 Ellen hat, drei
Tore: das Josephtor, das Benjamintor, das Dantor; \bibleverse{33}auf der
Südseite, die eine Länge von 4500 Ellen hat, drei Tore: das Simeontor,
das Issaschartor, das Sebulontor; \bibleverse{34}auf der Westseite, die
4500 Ellen lang ist, drei Tore: das Gadtor, das Assertor, das
Naphthalitor. \bibleverse{35}Der ganze Umfang beträgt 18000 Ellen, und
der Name der Stadt soll fortan lauten ›Gottesheim‹.«
