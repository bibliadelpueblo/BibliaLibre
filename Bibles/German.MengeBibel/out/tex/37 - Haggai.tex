\hypertarget{der-prophet-haggai}{%
\section{DER PROPHET HAGGAI}\label{der-prophet-haggai}}

\hypertarget{das-erste-gotteswort-die-aufforderung-zum-tempelbau-nebst-angabe-ihres-erfolges}{%
\subsubsection{1. Das erste Gotteswort: Die Aufforderung zum Tempelbau
nebst Angabe ihres
Erfolges}\label{das-erste-gotteswort-die-aufforderung-zum-tempelbau-nebst-angabe-ihres-erfolges}}

\hypertarget{section}{%
\section{1}\label{section}}

\bibleverse{1} Im zweiten Regierungsjahre des Königs Darius, am ersten
Tage des sechsten Monats, erging das Wort des HERRN durch den Propheten
Haggai\textless sup title=``vgl. Esr 5''\textgreater✲ an Serubbabel, den
Sohn Sealthiels, den Statthalter von Juda, und an den Hohenpriester
Josua, den Sohn Jozadaks, folgendermaßen: \bibleverse{2} »So hat der
HERR der Heerscharen gesprochen: Dieses Volk da sagt: ›Die Zeit, den
Tempel des HERRN wieder aufzubauen, ist jetzt noch nicht gekommen!‹«
\bibleverse{3} Daher erging das Wort des HERRN durch den Propheten
Haggai folgendermaßen: \bibleverse{4} »Ist es etwa für euch selbst an
der Zeit, in euren getäfelten Häusern zu wohnen, während dieses Haus in
Trümmern daliegt?« \bibleverse{5} »Und nun« -- so spricht der HERR der
Heerscharen -- »achtet wohl darauf, wie es euch bisher ergangen ist!
\bibleverse{6} Ihr habt reichlich ausgesät, aber kärglich eingebracht;
ihr habt wohl zu essen, aber es reicht nicht zum Sattwerden; ihr trinkt
und stillt doch den Durst nicht; ihr habt wohl etwas zum Anziehen, aber
keiner wird recht warm davon; und wer um Lohn arbeitet, der sammelt den
Lohn in einen löcherigen Beutel.«

\bibleverse{7} So spricht der HERR der Heerscharen: »Achtet wohl darauf,
wie es euch bisher ergangen ist! \bibleverse{8} Steigt ins Gebirge
hinauf, schafft Holz herbei und bauet den Tempel wieder auf, damit ich
meine Freude daran habe und mich in meiner Herrlichkeit zeige! -- so
spricht der HERR. \bibleverse{9} Ihr hattet auf viel gerechnet, aber es
wurde wenig daraus; und wenn ihr das eingebracht hattet, so blies ich es
weg. Warum das?« -- so lautet der Ausspruch des HERRN der Heerscharen.
»Um meines Hauses willen, das in Trümmern daliegt, während ein jeder von
euch an seinem eigenen Hause seine Freude hat\textless sup title=``oder:
für sein eigenes Haus eifrig sorgt''\textgreater✲. \bibleverse{10} Darum
hat der Himmel seinen Tau über euch zurückgehalten und die Erde euch
ihren Ertrag versagt; \bibleverse{11} und ich habe Dürre über das Land
kommen lassen und über die Berge, über das Getreide, den Most und das
Öl, kurz über alles, was der Erdboden hervorbringt, auch über die
Menschen und das Vieh und über allen Ertrag der Hände.«

\bibleverse{12} Da hörten Serubbabel, der Sohn Sealthiels, und der
Hohepriester Josua, der Sohn Jozadaks, und alle, die vom Volk noch übrig
waren, auf die Mahnung des HERRN, ihres Gottes, nämlich auf die Worte
des Propheten Haggai, der, wie sie erkannten, vom HERRN, ihrem Gott, zu
ihnen gesandt worden war; ja, das Volk geriet in Furcht vor dem HERRN.
\bibleverse{13} Da machte aber Haggai, der Bote des HERRN, kraft
göttlicher Botschaft dem Volk folgende Eröffnung: »›Ich bin mit euch!‹
-- so lautet der Ausspruch des HERRN.« \bibleverse{14} Hierauf erweckte
der HERR den Geist\textless sup title=``oder: Eifer''\textgreater✲
Serubbabels, des Sohnes Sealthiels, des Statthalters von Juda, und den
Eifer des Hohenpriesters Josua, des Sohnes Jozadaks, und den Eifer aller
vom Volk Übriggebliebenen, so daß sie kamen und die Arbeit am Tempel des
HERRN der Heerscharen, ihres Gottes, in Angriff nahmen \bibleverse{15}
am vierundzwanzigsten Tage des sechsten Monats.

\hypertarget{das-zweite-gotteswort-die-verheiuxdfung-der-kuxfcnftigen-herrlichkeit-des-neuen-tempels}{%
\subsubsection{2. Das zweite Gotteswort: Die Verheißung der künftigen
Herrlichkeit des neuen
Tempels}\label{das-zweite-gotteswort-die-verheiuxdfung-der-kuxfcnftigen-herrlichkeit-des-neuen-tempels}}

Im zweiten Regierungsjahr des Königs Darius,

\hypertarget{section-1}{%
\section{2}\label{section-1}}

\bibleverse{1} am einundzwanzigsten Tage des siebten Monats erging das
Wort des HERRN durch den\textless sup title=``=~durch Vermittlung
des''\textgreater✲ Propheten Haggai folgendermaßen: \bibleverse{2} »Sage
doch zu Serubbabel, dem Sohne Sealthiels, dem Statthalter von Juda, und
zu dem Hohenpriester Josua, dem Sohne Jozadaks, und zu allen vom Volk
Übriggebliebenen folgendes: \bibleverse{3} ›Wer ist unter euch noch am
Leben, der diesen Tempel in seiner früheren Herrlichkeit gesehen hat,
und wie seht ihr ihn heute? Nicht wahr? Wie nichts kommt er euch vor.
\bibleverse{4} Nun aber sei getrost, Serubbabel!‹ -- so lautet der
Ausspruch des HERRN -- ›und sei getrost, Josua, Sohn Jozadaks, du
Hoherpriester, und seid getrost ihr alle, die ihr das Volk des Landes
bildet‹ -- so lautet der Ausspruch des HERRN --, ›und arbeitet, denn ich
bin mit euch!‹ -- so lautet der Ausspruch des HERRN der Heerscharen.
\bibleverse{5} ›Die Verheißung, die ich euch bei eurem Auszug aus
Ägypten feierlich gegeben habe, bleibt bestehen, und mein Geist waltet
in eurer Mitte: fürchtet euch nicht!‹ \bibleverse{6} Denn so spricht der
HERR der Heerscharen: ›Nur noch {[}einmal{]} eine kurze Zeit währt es;
da werde ich den Himmel und die Erde, das Meer und das feste Land
erschüttern, \bibleverse{7} und ich werde alle Völker in Bewegung
setzen, daß die Kostbarkeiten aller Heidenvölker herbeigebracht werden;
und ich will dieses Haus mit Herrlichkeit erfüllen!‹ -- so spricht der
HERR der Heerscharen. \bibleverse{8} ›Mein ist das Silber und mein das
Gold‹ -- so lautet der Ausspruch des HERRN der Heerscharen.
\bibleverse{9} ›Größer wird die künftige Herrlichkeit dieses Tempels
sein, als die des ersten gewesen ist‹ -- so spricht der HERR der
Heerscharen --, ›und an dieser Stätte will ich Frieden\textless sup
title=``oder: Segen''\textgreater✲ spenden‹ -- so lautet der Ausspruch
des HERRN der Heerscharen.«

\hypertarget{das-dritte-gotteswort-eine-zurechtweisung-und-aufmunterung}{%
\subsubsection{3. Das dritte Gotteswort: Eine Zurechtweisung und
Aufmunterung}\label{das-dritte-gotteswort-eine-zurechtweisung-und-aufmunterung}}

\hypertarget{a-das-unreine-volk-und-die-unreinheit-der-opfer}{%
\paragraph{a) Das unreine Volk und die Unreinheit der
Opfer}\label{a-das-unreine-volk-und-die-unreinheit-der-opfer}}

\bibleverse{10} Am vierundzwanzigsten Tage des neunten Monats, im
zweiten Regierungsjahre des Darius, erging das Wort des HERRN durch den
Propheten Haggai folgendermaßen: \bibleverse{11} »So spricht der HERR
der Heerscharen: ›Erbitte dir doch von den Priestern Belehrung über
folgende Frage: \bibleverse{12} Wenn jemand heiliges Fleisch✲ im Zipfel
seines Gewandes trägt und mit seinem Zipfel Brot oder Gekochtes, Wein,
Öl oder sonst irgend etwas Genießbares berührt: wird dieses dadurch
heilig?‹« Da gaben die Priester zur Antwort: »Nein!« \bibleverse{13} Nun
fragte Haggai weiter: »Wenn aber ein durch eine Leiche Verunreinigter
irgendeins von derartigen Dingen berührt, wird es dadurch unrein?« Da
gaben die Priester zur Antwort: »Ja, es wird unrein!« \bibleverse{14} Da
erklärte Haggai folgendes: »›Ebenso steht es um diese Leute, und ebenso
ist dieses Volk da in meinen Augen beschaffen‹ -- so lautet der
Ausspruch des HERRN --, ›und ebenso steht es mit allem Tun ihrer Hände
und mit dem, was sie mir dort als Opfer darbringen: es ist unrein!‹«

\hypertarget{b-hinweis-auf-den-mit-dem-tempelbau-sicher-eintretenden-segen}{%
\paragraph{b) Hinweis auf den mit dem Tempelbau sicher eintretenden
Segen}\label{b-hinweis-auf-den-mit-dem-tempelbau-sicher-eintretenden-segen}}

\bibleverse{15} »Nun aber richtet doch eure Aufmerksamkeit vom heutigen
Tage an (auf die Zeit) rückwärts! Bevor man Stein auf Stein am Tempel
des HERRN gelegt hat: \bibleverse{16} wie ist es euch da ergangen? Kam
man damals zu einem Garbenhaufen von (vermutlich) zwanzig Scheffeln, so
wurden es nur zehn; und kam einer zur Kelter in der Erwartung, fünfzig
Eimer aus ihr zu schöpfen, so wurden es nur zwanzig. \bibleverse{17}
›Ich habe euch mit Getreidebrand und Vergilben geschlagen und mit Hagel
den Ertrag aller eurer Feldarbeit, und doch seid ihr nicht zu mir
umgekehrt!‹ -- so lautet der Ausspruch des HERRN\textless sup
title=``vgl. Am 4,9''\textgreater✲. \bibleverse{18} Nun aber gebt acht
auf das, was in der Folgezeit vom heutigen Tage an geschehen wird,
nämlich vom vierundzwanzigsten Tage des neunten Monats an, von dem Tage
an, wo der Grundstein zum Tempel des HERRN gelegt worden ist. Gebt acht
darauf, \bibleverse{19} ob die Aussaat noch weiterhin im Speicher liegen
bleibt und ob der Weinstock und Feigenbaum, die Granate und der Ölbaum
auch fürderhin nicht tragen! Von diesem Tage an werde ich segnen.«

\hypertarget{das-vierte-gotteswort-der-untergang-der-heidnischen-reiche-und-die-verheiuxdfung-von-serubbabels-erhuxf6hung}{%
\subsubsection{4. Das vierte Gotteswort: Der Untergang der heidnischen
Reiche und die Verheißung von Serubbabels
Erhöhung}\label{das-vierte-gotteswort-der-untergang-der-heidnischen-reiche-und-die-verheiuxdfung-von-serubbabels-erhuxf6hung}}

\bibleverse{20} Hierauf erging das Wort des HERRN zum zweitenmal an
Haggai am vierundzwanzigsten Tage des (gleichen) Monats folgendermaßen:
\bibleverse{21} »Sage zu Serubbabel, dem Statthalter von Juda,
folgendes: ›Ich werde den Himmel und die Erde erschüttern
\bibleverse{22} und die Throne der Könige umstürzen und die Macht der
heidnischen Königreiche vernichten, und zwar werde ich die Streitwagen
umstürzen samt denen, die darauf fahren, und es sollen die Rosse samt
ihren Reitern zu Boden stürzen, einer durch das Schwert des andern
(fallen). \bibleverse{23} An jenem Tage‹ -- so lautet der Ausspruch des
HERRN der Heerscharen -- ›will ich dich, Serubbabel, Sohn Sealthiels,
nehmen und dich zu meinem Knecht✲ machen‹ -- so lautet der Ausspruch des
HERRN -- ›und dich wie einen Siegelring halten; denn dich habe ich
erwählt!‹ -- so lautet der Ausspruch des HERRN der Heerscharen.«
