\hypertarget{das-buch-hiob}{%
\section{DAS BUCH HIOB}\label{das-buch-hiob}}

\hypertarget{i.-der-eingang-des-buches-kap.-1-3}{%
\subsection{I. Der Eingang des Buches (Kap.
1-3)}\label{i.-der-eingang-des-buches-kap.-1-3}}

\hypertarget{hiobs-fruxf6mmigkeit-und-uxe4uuxdferer-gluxfccksstand-seine-sorge-fuxfcr-die-gottesfurcht-seiner-kinder}{%
\subsubsection{1. Hiobs Frömmigkeit und äußerer Glücksstand; seine Sorge
für die Gottesfurcht seiner
Kinder}\label{hiobs-fruxf6mmigkeit-und-uxe4uuxdferer-gluxfccksstand-seine-sorge-fuxfcr-die-gottesfurcht-seiner-kinder}}

\hypertarget{section}{%
\section{1}\label{section}}

1Es lebte einst ein Mann im Lande Uz, Hiob mit Namen, und dieser Mann
war fromm und rechtschaffen, fürchtete Gott und mied das Böse. 2Sieben
Söhne und drei Töchter wurden ihm geboren; 3dazu besaß er siebentausend
Stück Kleinvieh und dreitausend Kamele, fünfhundert Joch✲ Rinder,
fünfhundert Eselinnen und ein sehr zahlreiches Gesinde, so daß dieser
Mann unter allen Bewohnern des Ostlandes der angesehenste war.

4Nun pflegten seine Söhne im Hause eines jeden von ihnen an seinem Tage✲
ein festliches Mahl zu veranstalten und luden dann allemal auch ihre
drei Schwestern ein, mit ihnen zu essen und zu trinken. 5Wenn aber die
Tage des betreffenden Gastmahls um waren, ließ Hiob ihnen sagen, sie
möchten sich heiligen; er stand dann am andern Morgen früh auf und
brachte für jeden von ihnen ein Brandopfer dar; denn Hiob dachte:
»Vielleicht haben meine Kinder sich versündigt und in ihrem Herzen Gott
verwünscht\textless sup title=``d.h. sich von Gott
losgesagt''\textgreater✲.« So machte es Hiob jedesmal.

\hypertarget{hiobs-fruxf6mmigkeit-bewuxe4hrt-sich-in-der-ersten-pruxfcfung}{%
\subsubsection{2. Hiobs Frömmigkeit bewährt sich in der ersten
Prüfung}\label{hiobs-fruxf6mmigkeit-bewuxe4hrt-sich-in-der-ersten-pruxfcfung}}

\hypertarget{a-gespruxe4ch-und-abmachung-gottes-mit-dem-satan-in-der-ersten-versammlung-der-gottessuxf6hne-d.h.-der-himmlischen-der-engel}{%
\paragraph{a) Gespräch und Abmachung Gottes mit dem Satan in der ersten
Versammlung der Gottessöhne (d.h. der Himmlischen, der
Engel)}\label{a-gespruxe4ch-und-abmachung-gottes-mit-dem-satan-in-der-ersten-versammlung-der-gottessuxf6hne-d.h.-der-himmlischen-der-engel}}

6Nun begab es sich eines Tages, daß die Gottessöhne kamen, um sich vor
Gott, den HERRN, zu stellen; und unter ihnen erschien auch der Satan.
7Da fragte der HERR den Satan: »Woher kommst du?« Der Satan gab dem
HERRN zur Antwort: »Ich bin auf der Erde umhergestreift und habe eine
Wanderung auf ihr vorgenommen.« 8Da sagte der HERR zum Satan: »Hast du
wohl auf meinen Knecht Hiob achtgegeben? Denn so wie er ist kein Mensch
auf der Erde, so fromm und rechtschaffen, so gottesfürchtig und dem
Bösen feind.« 9Der Satan erwiderte dem HERRN: »Ist Hiob etwa umsonst so
gottesfürchtig? 10Hast du nicht selbst ihn und sein Haus und seinen
ganzen Besitz rings umhegt? Was seine Hände angreifen, das segnest du,
so daß sein Herdenbesitz sich immer weiter im Lande ausgebreitet hat.
11Aber strecke doch einmal deine Hand aus und lege sie an alles, was er
besitzt: dann wird er sich schon offen von dir lossagen\textless sup
title=``oder: dir fluchen''\textgreater✲.« 12Da antwortete der HERR dem
Satan: »Gut! alles, was ihm gehört, soll in deine Gewalt gegeben sein!
Nur an ihn selbst darfst du die Hand nicht legen!« Da ging der Satan vom
Angesicht des HERRN hinweg.

\hypertarget{b-vernichtung-des-uxe4uuxdferen-gluxfccksstandes-hiobs}{%
\paragraph{b) Vernichtung des äußeren Glücksstandes
Hiobs}\label{b-vernichtung-des-uxe4uuxdferen-gluxfccksstandes-hiobs}}

13Während nun eines Tages Hiobs Söhne und Töchter im Hause ihres
ältesten Bruders schmausten und Wein tranken, 14kam plötzlich ein Bote
zu Hiob und meldete: »Die Rinder pflügten gerade, und die Eselinnen
befanden sich neben ihnen auf der Weide, 15da machten die Sabäer einen
Überfall und trieben sie weg und erschlugen die Knechte mit dem Schwert;
ich bin der einzige, der entronnen ist, um es dir zu melden!« 16Während
dieser noch redete, kam schon ein anderer und berichtete: »Feuer
Gottes\textless sup title=``d.h. der Blitz''\textgreater✲ ist vom Himmel
gefallen und hat das Kleinvieh und die Knechte vollständig verbrannt;
ich bin der einzige, der entronnen ist, um es dir zu melden!« 17Während
dieser noch redete, kam schon wieder ein anderer und berichtete: »Die
Chaldäer sind in drei Heerhaufen, die sie aufgestellt hatten, über die
Kamele hergefallen und haben sie weggetrieben; sie haben auch die
Knechte mit dem Schwert niedergemacht; ich bin der einzige, der
entronnen ist, um es dir zu melden!« 18Dieser hatte noch nicht
ausgeredet, da kam wieder ein anderer und berichtete: »Deine Söhne und
Töchter waren beim Essen und Weintrinken im Hause ihres ältesten
Bruders, 19da kam plötzlich ein gewaltiger Sturmwind über die Steppe
herüber und faßte das Haus an seinen vier Ecken, so daß es auf die
jungen Leute stürzte und sie ums Leben kamen; ich bin der einzige, der
entronnen ist, um es dir zu melden!«

\hypertarget{c-hiobs-demuxfctige-ergebung-in-gottes-willen}{%
\paragraph{c) Hiobs demütige Ergebung in Gottes
Willen}\label{c-hiobs-demuxfctige-ergebung-in-gottes-willen}}

20Da stand Hiob auf, zerriß sein Gewand und schor sich das Haupt; dann
warf er sich auf die Erde nieder, berührte den Boden mit der Stirn,
21und sagte: »Nacht bin ich aus meiner Mutter Schoß gekommen, und nackt
werde ich dorthin zurückkehren; der HERR hat's gegeben, der HERR hat's
genommen: der Name des HERRN sei gepriesen!«

22Bei allen diesen Heimsuchungen versündigte sich Hiob nicht und tat
nichts Ungebührliches vor Gott.

\hypertarget{hiobs-fruxf6mmigkeit-bewuxe4hrt-sich-auch-in-der-zweiten-pruxfcfung}{%
\subsubsection{3. Hiobs Frömmigkeit bewährt sich auch in der zweiten
Prüfung}\label{hiobs-fruxf6mmigkeit-bewuxe4hrt-sich-auch-in-der-zweiten-pruxfcfung}}

\hypertarget{a-neue-abmachungen-gottes-mit-dem-satan}{%
\paragraph{a) Neue Abmachungen Gottes mit dem
Satan}\label{a-neue-abmachungen-gottes-mit-dem-satan}}

\hypertarget{section-1}{%
\section{2}\label{section-1}}

1Da begab es sich eines Tages, daß die Gottessöhne✲ wiederum kamen, um
sich vor Gott den HERRN zu stellen; und unter ihnen erschien auch der
Satan, um sich vor den HERRN zu stellen. 2Da fragte der HERR den Satan:
»Woher kommst du?« Der Satan gab dem HERRN zur Antwort: »Ich bin auf der
Erde umhergestreift und habe eine Wanderung auf ihr vorgenommen.« 3Da
sagte der HERR zum Satan: »Hast du auch auf meinen Knecht Hiob
achtgegeben? Denn so wie er ist kein Mensch auf der Erde, so fromm und
rechtschaffen, so gottesfürchtig und dem Bösen feind; noch immer hält er
an seiner Frömmigkeit fest, wiewohl du mich gegen ihn gereizt hast, ihn
ohne Grund unglücklich zu machen.« 4Der Satan aber erwiderte dem HERRN:
»Haut um Haut! Ja alles, was ein Mensch hat, gibt er für sein Leben hin.
5Aber strecke nur einmal deine Hand aus und lege sie an sein Gebein und
sein Fleisch, so wird er sich sicherlich offen von dir
lossagen!«\textless sup title=``vgl. 1,5''\textgreater✲ 6Da sagte der
HERR zum Satan: »Gut! er soll in deine Gewalt gegeben sein: nur sein
Leben sollst du schonen!«

\hypertarget{b-hiob-bleibt-auch-vom-aussatz-befallen-trotz-der-versuchung-durch-seine-frau-unbeirrt-fromm}{%
\paragraph{b) Hiob bleibt, auch vom Aussatz befallen, trotz der
Versuchung durch seine Frau unbeirrt
fromm}\label{b-hiob-bleibt-auch-vom-aussatz-befallen-trotz-der-versuchung-durch-seine-frau-unbeirrt-fromm}}

7Da ging der Satan vom HERRN hinweg und schlug Hiob mit bösartigen
Geschwüren von der Fußsohle bis zum Scheitel, 8so daß er sich eine
Scherbe nahm, um sich mit ihr zu schaben, während er mitten in der Asche
saß. 9Da sagte seine Frau zu ihm: »Hältst du denn immer noch an deiner
Frömmigkeit fest? Sage dich los von Gott\textless sup
title=``=~verfluche doch Gott''\textgreater✲ und stirb!« 10Er aber
antwortete ihr: »Du redest, wie die erste beste Törin reden würde! Das
Gute haben wir von Gott hingenommen und sollten das Schlimme nicht auch
hinnehmen?« Bei allen diesen Heimsuchungen versündigte sich Hiob nicht
mit seinen Lippen.

\hypertarget{die-drei-freunde-hiobs-bleiben-bei-ihrem-besuch-vor-entsetzen-stumm-hiobs-schmerzensausbruch}{%
\subsubsection{4. Die drei Freunde Hiobs bleiben bei ihrem Besuch vor
Entsetzen stumm; Hiobs
Schmerzensausbruch}\label{die-drei-freunde-hiobs-bleiben-bei-ihrem-besuch-vor-entsetzen-stumm-hiobs-schmerzensausbruch}}

\hypertarget{a-die-drei-freunde-hiobs}{%
\paragraph{a) Die drei Freunde Hiobs}\label{a-die-drei-freunde-hiobs}}

11Als nun die drei Freunde Hiobs von all diesem Unglück hörten, das ihn
betroffen hatte, machten sie sich, ein jeder aus seinem Wohnort, auf den
Weg, nämlich Eliphas aus Theman, Bildad aus Suah und Zophar aus Naama,
und zwar verabredeten sie sich, miteinander hinzugehen, um ihm ihr
Beileid auszudrücken und ihn zu trösten. 12Als sie nun von ferne ihre
Augen aufschlugen, erkannten sie ihn nicht mehr; da fingen sie an, laut
zu weinen, zerrissen ein jeder sein Gewand und warfen Staub in die Luft
auf ihre Häupter herab. 13Dann saßen sie bei ihm auf dem Erdboden sieben
Tage und sieben Nächte lang, ohne daß einer ein Wort zu ihm redete; denn
sie sahen, daß sein Schmerz überaus groß war.

\hypertarget{b-hiobs-verzweiflungsvolle-klage}{%
\paragraph{b) Hiobs verzweiflungsvolle
Klage}\label{b-hiobs-verzweiflungsvolle-klage}}

\hypertarget{section-2}{%
\section{3}\label{section-2}}

1Endlich öffnete Hiob den Mund und verfluchte den Tag seiner Geburt,
2indem er ausrief:

3»Vernichtet sei der Tag, an dem ich geboren wurde, und die Nacht, die
da verkündete: ›Ein Mann✲ ist empfangen worden!‹ 4Jener Tag möge zu
Finsternis werden! Nicht kümmere sich um ihn Gott in der Höhe, und kein
Tageslicht möge über ihm erglänzen! 5Nein, Finsternis und Todesschatten
mögen ihn als ihr Eigentum zurückfordern, Wolkendunkel sich über ihm
lagern, Verdüsterung des Tageslichts ihn schreckensvoll machen! 6Jene
Nacht -- sie sei ein Raub des Dunkels! sie werde den Tagen des Jahres
nicht beigesellt, in die Zahl der Monate nicht eingereiht! 7Nein, jene
Nacht bleibe unfruchtbar, kein Jubelruf\textless sup title=``d.h.
Hochzeitsjubel''\textgreater✲ sei ihr je beschieden! 8Verwünschen mögen
sie die Tagbeschwörer, die es verstehen, den Leviathan✲ in Wut zu
versetzen! 9Finster müssen die Sterne ihrer Dämmerung bleiben: sie warte
auf Licht, doch es bleibe aus, und niemals erblicke sie die Wimpern des
Morgenrots! 10Denn sie hat mir die Pforte des Mutterschoßes nicht
verschlossen und das Unheil vor meinen Augen nicht verborgen.

11Warum bin ich nicht gleich vom Mutterleibe weg\textless sup
title=``=~gleich bei der Geburt''\textgreater✲ gestorben, nicht dem Tode
verfallen, als ich aus dem Mutterschoß hervorgekommen war? 12Weshalb
haben sich mir Knie liebreich dargeboten und wozu Brüste, daß ich an
ihnen trinken konnte? 13Denn ich würde jetzt im Grabesfrieden liegen,
würde schlafen: da hätte ich Ruhe 14mit Königen und Volksberatern der
Erde, die sich Grabpaläste erbaut haben, 15oder mit Fürsten, die reich
an Gold waren und ihre Häuser mit Silber gefüllt hatten; 16oder, einer
verscharrten Fehlgeburt gleich, wäre ich nicht ins Dasein getreten, den
Kindlein gleich, die das Licht nicht erblickt haben. 17Dort haben die
Frevler abgelassen vom Wüten, und dort ruhen die aus, deren Kraft
erschöpft ist; 18dort leben die Gefangenen allesamt in Frieden, hören
nicht mehr die Stimme eines Treibers\textless sup title=``oder:
Fronvogts''\textgreater✲. 19Niedrige und Hohe gelten dort gleich, und
frei ist der Knecht✲ von seinem Herrn.

20Warum gibt er\textless sup title=``d.h. Gott''\textgreater✲ dem
Mühseligen das Licht, und das Leben denen, die verzweifelten Herzens
sind? 21Die sich nach dem Tode sehnen, ohne daß er kommt, und die nach
ihm eifriger graben als nach Schätzen? 22Die sich bis zum Jubel freuen,
ja aufjauchzen würden, wenn sie das Grab fänden? 23(Warum gibt er's
nicht) dem Manne, dem sein Weg✲ in Nacht verborgen ist und dem Gott
jeden Ausweg versperrt hat? 24Denn Seufzen ist für mich das tägliche
Brot, und gleich dem Wasser ergießt sich meine laute Klage. 25Denn bebe
ich vor etwas Furchtbarem, so trifft es bei mir ein, und wovor mir
graut, das bricht über mich herein: 26ich darf nicht aufatmen noch
rasten noch ruhen, so stellt sich schon wieder eine Qual ein.«

\hypertarget{ii.-erster-gespruxe4chsgang-kap.-4-14}{%
\subsection{II. Erster Gesprächsgang (Kap.
4-14)}\label{ii.-erster-gespruxe4chsgang-kap.-4-14}}

\hypertarget{erste-rede-des-eliphas}{%
\subsubsection{1. Erste Rede des Eliphas}\label{erste-rede-des-eliphas}}

\hypertarget{a-eliphas-entschuldigt-seinen-versuch-einer-zurechtweisung-hiobs-mit-dem-hinweis-auf-hiobs-eigenes-fruxfcheres-verhalten-vielen-leidenden-gegenuxfcber}{%
\paragraph{a) Eliphas entschuldigt seinen Versuch einer Zurechtweisung
Hiobs mit dem Hinweis auf Hiobs eigenes früheres Verhalten vielen
Leidenden
gegenüber}\label{a-eliphas-entschuldigt-seinen-versuch-einer-zurechtweisung-hiobs-mit-dem-hinweis-auf-hiobs-eigenes-fruxfcheres-verhalten-vielen-leidenden-gegenuxfcber}}

\hypertarget{section-3}{%
\section{4}\label{section-3}}

1Da hob Eliphas von Theman an und sagte:

2»Wird es dich verdrießen, wenn man ein Wort an dich zu richten wagt?
Doch wer vermöchte die Worte zurückzuhalten? 3Hast du doch selbst vielen
(Leidenden) Mut zugesprochen und erschlaffte Hände gestärkt; 4manchen
Wankenden haben deine Worte aufrecht gehalten, und niedersinkenden Knien
hast du neue Kraft verliehen. 5Nun aber, da die Reihe an dich gekommen,
bist du verzagt; nun es dich selbst trifft, verlierst du den Halt!«

\hypertarget{b-hiob-soll-bedenken-dauxdf-niemand-schuldlos-leide-und-dauxdf-nur-frevler-untergehen}{%
\paragraph{b) Hiob soll bedenken, daß niemand schuldlos leide und daß
nur Frevler
untergehen}\label{b-hiob-soll-bedenken-dauxdf-niemand-schuldlos-leide-und-dauxdf-nur-frevler-untergehen}}

6»Ist deine Gottesfurcht nicht deine Zuversicht und dein unsträflicher
Wandel deine Hoffnung? 7Bedenke doch: Wo ist je ein Unschuldiger
zugrunde gegangen, und wo sind Rechtschaffene vernichtet worden? 8Soweit
meine Erfahrung reicht: die Unheil gepflügt und Frevel gesät hatten, die
haben es auch geerntet. 9Durch Gottes Odem kommen sie um, und durch den
Hauch\textless sup title=``oder: das Schnauben''\textgreater✲ seines
Zornes vergehen sie. 10Des Löwen Gebrüll und die Stimme des Leuen (sind
verstummt), und den jungen Löwen sind die Zähne ausgebrochen; 11da kommt
auch ein Löwe um aus Mangel an Raub, und die Jungen der Löwin müssen
sich zerstreuen.«

\hypertarget{c-eliphas-weiuxdf-durch-eine-ihm-gewordene-nachterscheinung-dauxdf-vor-gott-niemand-ohne-schuld-ist}{%
\paragraph{c) Eliphas weiß durch eine ihm gewordene Nachterscheinung,
daß vor Gott niemand ohne Schuld
ist}\label{c-eliphas-weiuxdf-durch-eine-ihm-gewordene-nachterscheinung-dauxdf-vor-gott-niemand-ohne-schuld-ist}}

12»Zu mir ist aber ein Wort verstohlen gedrungen, und mein Ohr hat einen
flüsternden Laut davon\textless sup title=``oder: von
daher''\textgreater✲ vernommen 13beim Spiel der durch Traumbilder
erregten Gedanken, in der Zeit, wo tiefer Schlaf sich auf die Menschen
senkt: 14ein Grauen überfiel mich und ein Zittern, durch alle meine
Gebeine ging ein Schauder; 15ein Lufthauch\textless sup title=``oder:
ein Geist''\textgreater✲ strich leise an meinem Antlitz vorüber; es
sträubte sich mir das Haar am Leibe empor! 16Da stand -- ihr Aussehen
konnte ich nicht erkennen -- eine Gestalt vor meinen Augen, und eine
Stimme hörte ich flüstern: 17›Kann wohl ein Mensch gerecht vor Gott sein
oder ein Sterblicher rein vor seinem Schöpfer? 18Bedenke: seinen Dienern
kann er nicht trauen, und seinen Engeln legt er Mängel\textless sup
title=``oder: Irrtümer''\textgreater✲ zur Last: 19wieviel mehr denen,
die Lehmhütten bewohnen, deren Grundbau im Staube liegt! Sie werden
zerdrückt, als wären sie Motten; 20vom Morgen bis zum Abend werden sie
zerschmettert; unbeachtet vergehen sie auf ewig. 21Nicht wahr, so ist
es: wird das Haltseil ihres Zeltes bei ihnen ausgerissen, so sterben sie
und wissen nicht wie.‹«

\hypertarget{d-selbstverschuldetes-leiden-findet-keinen-fuxfcrsprecher-und-wird-durch-unmut-nur-vergruxf6uxdfert}{%
\paragraph{d) Selbstverschuldetes Leiden findet keinen Fürsprecher und
wird durch Unmut nur
vergrößert}\label{d-selbstverschuldetes-leiden-findet-keinen-fuxfcrsprecher-und-wird-durch-unmut-nur-vergruxf6uxdfert}}

\hypertarget{section-4}{%
\section{5}\label{section-4}}

1»Ja, rufe nur! Ist jemand da, der dir Antwort gibt? Und an wen von den
heiligen (Engeln) willst du dich wenden? 2Vielmehr den Toren bringt sein
Unmut um, und den Einfältigen tötet sein Eifern\textless sup
title=``oder: Hadern''\textgreater✲. 3Ich selbst habe einen Toren zwar
Wurzel schlagen sehen, doch gar schnell hatte ich seine Wohnstätte zu
verwünschen. 4Seinen Kindern blieb die Hilfe\textless sup title=``oder:
das Wohlergehen''\textgreater✲ fern, und sie wurden im Tor\textless sup
title=``=~vor Gericht''\textgreater✲ zertreten, ohne daß ein Retter da
war. 5Seine Ernte verzehrte ein anderer, der danach hungerte und sie
sogar hinter dem Dorngehege wegholte; und Durstige schnappten nach
seinem Vermögen. 6Denn nicht aus dem Erdenstaube\textless sup
title=``oder: Erdboden''\textgreater✲ erwächst das Unheil, und das Leid
sproßt nicht aus der Ackererde hervor, 7sondern der Mensch erzeugt das
Leid, wie die Kinder der Flamme\textless sup title=``d.h. die
Feuerfunken''\textgreater✲ einen hohen Flug zu nehmen pflegen.«

\hypertarget{e-rettung-kann-hiob-nur-durch-demut-und-durch-anrufung-der-guxfcte-gottes-erlangen}{%
\paragraph{e) Rettung kann Hiob nur durch Demut und durch Anrufung der
Güte Gottes
erlangen}\label{e-rettung-kann-hiob-nur-durch-demut-und-durch-anrufung-der-guxfcte-gottes-erlangen}}

8»Doch ich, an den Höchsten würde ich mich wenden und meine Sache Gott
anheimstellen, 9ihm, der große und unerforschliche Dinge tut,
Wunderbares ohne Maß und Zahl~-- 10ihm, der Regen über die Erde hin
sendet und des Himmels Naß auf die Fluren fallen läßt --, 11insofern er
Niedrige emporhebt und Trauernde sich des höchsten Glücks erfreuen läßt;
12ihm, der die Pläne der Listigen vereitelt, so daß ihre Hände nichts
Erfolgreiches schaffen; 13ihm, der die Klugen trotz ihrer Schlauheit
fängt, so daß die Verschlagenen sich in ihren Anschlägen überstürzen:
14am hellen Tage stoßen sie auf Finsternis, und am Mittag tappen sie im
Dunkel wie bei Nacht. 15So rettet er den Wehrlosen vor dem Schwert aus
ihrem Rachen, und aus des Starken Faust den Geringen. 16So erblüht dem
Schwachen neue Hoffnung, die Bosheit aber muß ihren Mund schließen.«

\hypertarget{f-wenn-hiob-sich-unter-gottes-zuxfcchtigung-beuge-so-werde-die-heimsuchung-ihm-segensreich-fuxfcr-sein-ganzes-kuxfcnftiges-leben-werden}{%
\paragraph{f) Wenn Hiob sich unter Gottes Züchtigung beuge, so werde die
Heimsuchung ihm segensreich für sein ganzes künftiges Leben
werden}\label{f-wenn-hiob-sich-unter-gottes-zuxfcchtigung-beuge-so-werde-die-heimsuchung-ihm-segensreich-fuxfcr-sein-ganzes-kuxfcnftiges-leben-werden}}

17»O wohl dem Menschen, den Gott in Zucht nimmt! Darum verschmähe die
Züchtigung des Allmächtigen nicht! 18Denn er verwundet wohl, doch er
verbindet auch; wenn er zerschlägt, so heilen seine Hände auch wieder.
19In sechs Drangsalen errettet er dich, und in sieben wird kein Unheil
dich treffen. 20In Hungersnot bewahrt er dich vor dem Tode und im Kriege
vor der Gewalt des Schwertes. 21Vor den Geißelhieben der Zunge wirst du
geborgen sein und brauchst nicht vor der Verheerung zu bangen, daß sie
dich erreicht. 22Der Verwüstung und der Hungersnot darfst du lachen und
hast von den wilden Tieren des Landes nichts zu befürchten; 23denn mit
den Steinen des Feldes stehst du im Bunde, und das Getier des Feldes
lebt mit dir in Frieden. 24So wirst du es denn erfahren, daß dein Zelt
in Sicherheit ist, und überblickst du dein Gehöft\textless sup
title=``oder: deine Wohnstätte''\textgreater✲, so wirst du nichts
vermissen 25und wirst es erleben, daß deine Nachkommenschaft zahlreich
ist und dein Nachwuchs gleich dem Gras der Flur. 26In vollreifem Alter
wirst du in die Gruft eingehen, wie der Garbenhaufen eingebracht wird
zur rechten Zeit. 27Siehe, dies ist es, was wir erforscht haben, so ist
es: vernimm es und beherzige es zu deinem Heil!«

\hypertarget{hiobs-antwort-erste-gegenrede}{%
\subsubsection{2. Hiobs Antwort (erste
Gegenrede)}\label{hiobs-antwort-erste-gegenrede}}

\hypertarget{a-hiob-entschuldigt-die-bitterkeit-der-von-ihm-ausgestouxdfenen-klage-mit-der-furchtbaren-schwere-seines-leidens}{%
\paragraph{a) Hiob entschuldigt die Bitterkeit der von ihm ausgestoßenen
Klage mit der furchtbaren Schwere seines
Leidens}\label{a-hiob-entschuldigt-die-bitterkeit-der-von-ihm-ausgestouxdfenen-klage-mit-der-furchtbaren-schwere-seines-leidens}}

\hypertarget{section-5}{%
\section{6}\label{section-5}}

1Da antwortete Hiob folgendermaßen:

2»Ach, würde doch mein Unmut genau gewogen und legte man mein Unglück
zugleich✲ auf die Waage! 3Denn dann würde es schwerer erfunden werden
als der Sand am Meere; darum ist meine Rede irre gegangen. 4Denn die
Pfeile des Allmächtigen stecken in mir, deren brennendes Gift mein Geist
in sich einsaugt: Gottes Schrecknisse stellen sich in Schlachtordnung
gegen mich auf. 5Schreit etwa ein Wildesel auf grasiger Weide? Oder
brüllt ein Rind bei seinem Futterkorn? 6Genießt man fade Speisen ohne
Salz? Oder ist Wohlgeschmack im Schleim des Eidotters\textless sup
title=``=~im Eiweiß''\textgreater✲? 7Meine Seele sträubt sich dagegen,
solche Sachen anzurühren, und ihnen gleicht die Ekelhaftigkeit meiner
Speise.«

\hypertarget{b-hiob-wuxfcnscht-durch-schnellen-tod-von-seinen-leiden-und-seiner-guxe4nzlichen-hilflosigkeit-erluxf6st-zu-werden}{%
\paragraph{b) Hiob wünscht durch schnellen Tod von seinen Leiden und
seiner gänzlichen Hilflosigkeit erlöst zu
werden}\label{b-hiob-wuxfcnscht-durch-schnellen-tod-von-seinen-leiden-und-seiner-guxe4nzlichen-hilflosigkeit-erluxf6st-zu-werden}}

8»O daß doch meine Bitte erfüllt würde und Gott mir meine Hoffnung
gewährte! 9Gefiele es doch Gott, mich zu zermalmen! Streckte er doch
seine Hand aus und schnitte meinen Lebensfaden ab! 10So würde doch das
noch ein Trost für mich sein -- ja aufhüpfen wollte ich trotz des
schonungslosen Schmerzes --, daß ich die Gebote des Heiligen nie
verleugnet habe. 11Wie groß ist denn meine Kraft noch, daß ich ausharren
könnte? Und welcher Ausgang wartet meiner, daß ich mich noch gedulden
sollte? 12Ist meine Kraft etwa hart wie die Kraft der Steine oder mein
Leib aus Erz gegossen? 13Ach, bin ich nicht ganz und gar hilflos? Und
ist mir nicht alles entrissen, worauf ich mich stützen könnte?«

\hypertarget{c-hiobs-klage-uxfcber-die-ihm-durch-die-freunde-bereitete-beschimpfung-und-enttuxe4uschung}{%
\paragraph{c) Hiobs Klage über die ihm durch die Freunde bereitete
Beschimpfung und
Enttäuschung}\label{c-hiobs-klage-uxfcber-die-ihm-durch-die-freunde-bereitete-beschimpfung-und-enttuxe4uschung}}

14»Dem Verzweifelnden gebührt Liebe von seinem Nächsten, selbst wenn er
die Furcht vor dem Allmächtigen preisgibt. 15Meine Freunde aber haben
sich treulos bewiesen wie ein Wildbach, wie die Rinnsale von Wildbächen,
die (in der Regenzeit) überströmen, 16die trübe vom Eiswasser
dahinfließen, wenn der (geschmolzene) Schnee sich in ihnen birgt; 17doch
zur Zeit, wo die Sonnenglut sie trifft, versiegen sie: wenn es heiß
wird, sind sie spurlos verschwunden. 18Da schlängeln sich die Pfade
ihres Laufes, verdunsten in die leere Luft und verlieren sich. 19Die
Handelszüge✲ von Thema\textless sup title=``Jes 21,14''\textgreater✲
schauen nach ihnen aus, die Wanderzüge der Sabäer✲ setzen ihre Hoffnung
auf sie, 20werden jedoch in ihrem Vertrauen betrogen: sie kommen hin und
sehen sich getäuscht. 21So seid auch ihr jetzt ein Nichts für mich
geworden: ihr seht das Schreckliche und seid fassungslos! 22Habe ich
etwa gebeten: ›Gebt mir etwas und macht mir ein Geschenk von eurem
Vermögen; 23rettet mich aus der Hand meines Bedrängers und kauft mich
los aus der Gewalt unbarmherziger Gläubiger‹?«

\hypertarget{d-hiob-verlangt-nur-bei-bestimmter-angabe-boshafter-verschuldung-als-suxfcnder-bezeichnet-zu-werden}{%
\paragraph{d) Hiob verlangt, nur bei bestimmter Angabe boshafter
Verschuldung als Sünder bezeichnet zu
werden}\label{d-hiob-verlangt-nur-bei-bestimmter-angabe-boshafter-verschuldung-als-suxfcnder-bezeichnet-zu-werden}}

24»Belehrt mich, so will ich schweigen, und macht mir klar, worin ich
mich verfehlt habe! 25Wie eindringlich sind Worte der Wahrheit! Aber was
beweist der Tadel, den ihr aussprecht? 26Beabsichtigt ihr, Worte von mir
richtigzustellen? Für den Wind sind ja doch die Worte eines
Verzweifelnden! 27Sogar über ein Waisenkind würdet ihr das Los werfen
und euren eigenen Freund verschachern! 28Nun aber -- versteht euch doch
dazu, mich anzublicken: ich werde euch doch wahrlich nicht ins Angesicht
belügen! 29O kehrt euch her zu mir: tut mir nicht unrecht! Nein, kehrt
euch her zu mir; noch steht das Recht in dieser Sache auf meiner Seite!
30Entsteht denn durch meine Zunge Unrecht? Oder fehlt mir das Vermögen,
Unglücksschläge zu unterscheiden?«

\hypertarget{e-hiob-beklagt-die-muxfchsal-und-kuxfcrze-des-menschenlebens-uxfcberhaupt-und-seine-eigene-verzweiflungsvolle-lage-im-besonderen}{%
\paragraph{e) Hiob beklagt die Mühsal und Kürze des Menschenlebens
überhaupt und seine eigene verzweiflungsvolle Lage im
besonderen}\label{e-hiob-beklagt-die-muxfchsal-und-kuxfcrze-des-menschenlebens-uxfcberhaupt-und-seine-eigene-verzweiflungsvolle-lage-im-besonderen}}

\hypertarget{section-6}{%
\section{7}\label{section-6}}

1»Hat der Mensch nicht harten Kriegsdienst✲ auf Erden zu leisten, und
gleichen seine Lebenstage nicht den Tagen eines Tagelöhners? 2Gleich
einem Sklaven, der nach Schatten lechzt, und wie ein Tagelöhner, der auf
seinen Lohn harrt, 3so habe auch ich Monate des Elends als Erbteil
zugewiesen erhalten, und qualvolle Nächte sind mir zugeteilt worden.
4Sobald ich mich niedergelegt habe, denke ich: ›Wann werde ich wieder
aufstehen?‹ Dann dehnt sich die Nacht endlos aus, und ich werde des Hin-
und Herwerfens (über)satt bis zum Morgengrauen. 5Mein Leib hat sich mit
Gewürm und erdiger Kruste umkleidet; meine Haut ist zusammengeschrumpft,
um eiternd wieder aufzubrechen. 6Meine Tage fliegen schneller dahin als
ein Weberschiffchen und entschwinden hoffnungslos. 7Bedenke, daß mein
Leben nur ein Hauch ist! Mein Auge wird das Glück nie wieder zu sehen
bekommen! 8Das Auge dessen, der mich jetzt noch erblickt, wird mich bald
nicht mehr schauen: suchen deine Augen nach mir, so bin ich nicht mehr
da. 9Wie eine Wolke sich auflöst und zergeht, so kommt auch, wer ins
Totenreich hinabgefahren ist, nicht wieder herauf: 10nie kehrt er wieder
in sein Haus zurück, und seine Wohnstätte weiß nichts mehr von ihm!«

\hypertarget{f-hiob-erkluxe4rt-seine-klage-fuxfcr-berechtigt-und-gottes-verhalten-gegen-ihn-der-dem-tode-nahe-sei-fuxfcr-erbarmungslos-seine-bitte-an-gott-um-schonung}{%
\paragraph{f) Hiob erklärt seine Klage für berechtigt und Gottes
Verhalten gegen ihn, der dem Tode nahe sei, für erbarmungslos; seine
Bitte an Gott um
Schonung}\label{f-hiob-erkluxe4rt-seine-klage-fuxfcr-berechtigt-und-gottes-verhalten-gegen-ihn-der-dem-tode-nahe-sei-fuxfcr-erbarmungslos-seine-bitte-an-gott-um-schonung}}

11»So will nun auch ich meinem Munde nicht wehren, will in der Angst
meines Herzens reden, in der Verzweiflung meiner Seele klagen. 12Bin ich
etwa ein Meer oder ein Seeungeheuer, daß du eine Wache gegen mich
aufstellst? 13Wenn ich denke: ›Trösten wird mich mein Lager, mein Bett
wird mir meinen Jammer tragen helfen‹, 14so ängstigst du mich durch
Träume und schreckst mich durch Nachtgesichte auf, 15so daß ich lieber
erwürgt sein möchte, lieber den Tod sähe als dies mein Gerippe. 16Nun
habe ich's satt, ich mag nicht ewig so leben: laß ab von mir, denn nur
noch ein Hauch sind meine Tage. 17Was ist der Mensch, daß du ihn so groß
achtest und überhaupt dein Augenmerk auf ihn richtest? 18Daß du alle
Morgen nach ihm ausschaust und ihn alle Augenblicke prüfst? 19Wann wirst
du endlich deine Blicke von mir wegwenden und mir Ruhe gönnen, während
ich nur meinen Speichel verschlucke? 20Habe ich gesündigt: was habe ich
dir damit geschadet, du Menschenbeobachter? Warum hast du mich zur
Zielscheibe deiner Angriffe hingestellt, so daß ich mir selbst zur Last
bin? 21Und warum vergibst du mir meine Sünde nicht und schenkst meiner
Schuld nicht Verzeihung? Denn jetzt werde ich mich in den Staub legen,
und suchst du dann nach mir, so bin ich nicht mehr da.«

\hypertarget{erste-rede-bildads}{%
\subsubsection{3. Erste Rede Bildads}\label{erste-rede-bildads}}

\hypertarget{a-scharfe-hervorhebung-der-gerechtigkeit-gottes-die-sich-an-hiobs-kindern-als-strafgericht-erwiesen-habe-und-sich-an-hiob-als-guxfcte-erweisen-werde-wenn-er-gott-ernstlich-suche}{%
\paragraph{a) Scharfe Hervorhebung der Gerechtigkeit Gottes, die sich an
Hiobs Kindern als Strafgericht erwiesen habe und sich an Hiob als Güte
erweisen werde, wenn er Gott ernstlich
suche}\label{a-scharfe-hervorhebung-der-gerechtigkeit-gottes-die-sich-an-hiobs-kindern-als-strafgericht-erwiesen-habe-und-sich-an-hiob-als-guxfcte-erweisen-werde-wenn-er-gott-ernstlich-suche}}

\hypertarget{section-7}{%
\section{8}\label{section-7}}

1Da nahm Bildad von Suah das Wort und sagte:

2»Wie lange noch willst du solche Reden führen, und wie lange noch
sollen die Worte deines Mundes als Sturmwind daherfahren? 3Beugt Gott
etwa das Recht, oder verdreht der Allmächtige die Gerechtigkeit? 4Nur
wenn\textless sup title=``oder: weil''\textgreater✲ deine Kinder gegen
ihn gesündigt hatten, hat er sie die Folge ihrer Übertretung tragen
lassen. 5Wenn du aber Gott ernstlich suchst und zum Allmächtigen flehst,
6wenn du dabei unsträflich und rechtschaffen bist: ja, dann wird er zu
deinem Heil erwachen und deine Wohnung als eine Stätte der
Gerechtigkeit\textless sup title=``oder: als eine dir gebührende
Stätte''\textgreater✲ wiederherstellen. 7Da wird dann dein vormaliger
Glücksstand klein erscheinen gegenüber der Größe deiner nachmaligen
Lage.«

\hypertarget{b-die-erfahrung-und-die-uxfcberlieferung-der-vuxe4ter-bezeugen-den-sicheren-untergang-der-gottlosen}{%
\paragraph{b) Die Erfahrung und die Überlieferung der Väter bezeugen den
sicheren Untergang der
Gottlosen}\label{b-die-erfahrung-und-die-uxfcberlieferung-der-vuxe4ter-bezeugen-den-sicheren-untergang-der-gottlosen}}

8»Denn befrage nur das frühere Geschlecht und achte auf das, was ihre
Väter erforscht haben! 9Denn wir sind nur von gestern her und wissen
nichts, weil unsere Tage nur ein Schatten auf Erden sind; 10sie aber
werden dich sicherlich belehren, werden dir's sagen und aus der Tiefe
ihrer Einsicht die Worte hervorgehen lassen: 11›Schießt Schilfrohr auf,
wo kein Sumpf ist? Wächst Riedgras ohne Wasser auf? 12Noch steht es in
frischem Triebe, ist noch nicht reif zum Schnitt, da verdorrt es schon
vor allem andern Gras. 13So ergeht es auch allen, die Gott vergessen,
und so wird die Hoffnung des Ruchlosen zunichte; 14denn seine Zuversicht
setzt er auf Sommerfäden, und das, worauf er vertraut, ist ein
Spinngewebe. 15Er lehnt sich an sein Haus, doch es hält nicht stand; er
klammert sich fest daran, doch es bleibt nicht stehen. 16Er strotzt von
Saft auch in der Sonnenglut, und seine Schößlinge breiten sich über
seinen Garten aus; 17(sogar) um Steingeröll schlingen sich seine
Wurzeln, und in Steingemäuer bohren sie sich hinein; 18wenn aber
er\textless sup title=``d.h. Gott''\textgreater✲ ihn von seiner Stätte
wegreißt, so verleugnet diese ihn: Ich habe dich nie gesehen! 19Siehe,
das ist die Freude, die er von seinem Lebenswege hat, und aus dem Boden
sprossen wieder andere auf.‹«

\hypertarget{c-truxf6stlicher-ausblick-hiob-wird-wenn-er-sich-vom-gottlosen-wesen-abwendet-von-gott-wieder-gesegnet-werden}{%
\paragraph{c) Tröstlicher Ausblick: Hiob wird, wenn er sich vom
gottlosen Wesen abwendet, von Gott wieder gesegnet
werden}\label{c-truxf6stlicher-ausblick-hiob-wird-wenn-er-sich-vom-gottlosen-wesen-abwendet-von-gott-wieder-gesegnet-werden}}

20»Nein, Gott verwirft den Frommen nicht und reicht keinem Frevler die
Hand. 21Während er dir den Mund wieder mit Lachen füllen wird und deine
Lippen mit lautem Jubel, 22werden deine Widersacher mit Schande bedeckt
dastehen, und das Zelt der Frevler wird verschwunden sein.«

\hypertarget{hiobs-antwort-zweite-gegenrede}{%
\subsubsection{4. Hiobs Antwort (zweite
Gegenrede)}\label{hiobs-antwort-zweite-gegenrede}}

\hypertarget{a-ja-gott-hat-immer-recht-weil-ihm-dem-allmuxe4chtigen-niemand-standhalten-kann}{%
\paragraph{a) Ja, Gott hat immer recht, weil ihm, dem Allmächtigen,
niemand standhalten
kann}\label{a-ja-gott-hat-immer-recht-weil-ihm-dem-allmuxe4chtigen-niemand-standhalten-kann}}

\hypertarget{section-8}{%
\section{9}\label{section-8}}

1Darauf antwortete Hiob folgendermaßen:

2»Gewiß, ich weiß, daß es sich so verhält, und wie könnte ein Mensch
Gott gegenüber recht behalten? 3Wenn es ihn gelüstete, sich mit Gott in
einen Rechtsstreit einzulassen, so könnte er ihm auf tausend Fragen
keine einzige Antwort geben. 4Ist einer auch reich an Klugheit und stark
an Kraft: wer hat ihm (Gott) je getrotzt und ist heil davongekommen? 5Er
ist es ja, der Berge versetzt, ohne daß sie es merken, der sie in seinem
Zorn umkehrt; 6er macht die Erde aufbeben von ihrer Stätte, daß ihre
Säulen ins Wanken geraten; 7er gebietet der Sonne, so geht sie nicht
auf, und legt die Sterne unter Siegel; 8er spannt das Himmelszelt aus,
er allein, und schreitet hoch auf den Meereswogen einher; 9er hat das
Bärengestirn und den Orion geschaffen, das Siebengestirn und die
Kammern\textless sup title=``d.h. die Sternbilder''\textgreater✲ des
Südens; 10er vollführt große Dinge, daß sie nicht zu erforschen sind,
und Wunderwerke, daß man sie nicht zählen kann. 11Siehe, er geht an mir
vorüber, doch ich sehe ihn nicht; er schwebt dahin, doch ich nehme ihn
nicht wahr. 12Wenn er hinwegrafft -- wer will's ihm wehren? Wer darf zu
ihm sagen: ›Was machst du da?‹«

\hypertarget{b-hiob-wuxfcrde-selbst-wenn-er-im-recht-wuxe4re-bei-einem-rechtsstreit-mit-gott-als-schuldig-dastehen}{%
\paragraph{b) Hiob würde, selbst wenn er im Recht wäre, bei einem
Rechtsstreit mit Gott als schuldig
dastehen}\label{b-hiob-wuxfcrde-selbst-wenn-er-im-recht-wuxe4re-bei-einem-rechtsstreit-mit-gott-als-schuldig-dastehen}}

13»Gott läßt von seinem Zorn nicht ab -- unter ihn haben sich sogar die
Helfer Rahabs beugen müssen --, 14geschweige denn, daß ich ihm Rede
stehen könnte und ihm gegenüber die rechten Worte zu wählen wüßte.
15Wenn ich auch im Recht wäre, könnte ich ihm doch nicht antworten,
sondern müßte ihn als meinen Richter noch anflehen! 16Selbst wenn ich
ihn vor Gericht zöge und er mir Rede stünde, würde ich doch nicht
glauben, daß er meinen Aussagen Gehör schenkte; 17nein, er würde im
Sturmesbrausen mich zermalmen und meine Wunden ohne Ursache zahlreich
machen; 18er würde mich nicht zu Atem kommen lassen, sondern mich mit
bitteren Leiden sättigen. 19Kommt es auf die Kraft des Starken an, so
würde er sagen: ›Hier bin ich!‹, und handelt es sich um ein
Rechtsverfahren: ›Wer will mich vorladen?‹ 20Wäre ich auch im Recht, so
müßte doch mein eigener Mund mich verdammen, und wäre ich schuldlos, so
würde er mich doch als schuldig erscheinen lassen.«

\hypertarget{c-um-das-qualvolle-leben-mit-dem-tode-zu-vertauschen-spricht-hiob-bewuuxdft-die-luxe4sterung-aus-gott-verfahre-willkuxfcrlich-gegen-fromme-wie-gegen-suxfcnder}{%
\paragraph{c) Um das qualvolle Leben mit dem Tode zu vertauschen,
spricht Hiob bewußt die Lästerung aus, Gott verfahre willkürlich gegen
Fromme wie gegen
Sünder}\label{c-um-das-qualvolle-leben-mit-dem-tode-zu-vertauschen-spricht-hiob-bewuuxdft-die-luxe4sterung-aus-gott-verfahre-willkuxfcrlich-gegen-fromme-wie-gegen-suxfcnder}}

21»Schuldlos bin ich! Mir liegt nichts an meinem Leben; ich achte mein
Dasein für nichts! 22Es kommt auf eins heraus, darum spreche ich es frei
aus: Den Unschuldigen vernichtet er wie den Bösewicht. 23Wenn die Geißel
(schwerer Volksplagen) jähen Tod bringt, so lacht er über die
Verzweiflung der Unschuldigen. 24Ist ein Land in die Hand eines Frevlers
gegeben, so verhüllt er die Augen seiner Richter; wenn er es nicht tut
-- wer denn sonst? 25Und meine Tage eilen schneller dahin als ein
Läufer, sind entschwunden, ohne das Glück gesehen zu haben; 26sie sind
dahingeschossen wie Rohrkähne, wie ein Adler, der auf seine Beute stößt.
27Wenn ich mir vornehme: ›Ich will meinen Jammer vergessen, will mein
finsteres Aussehen abtun und heiter blicken!‹, 28so faßt mich doch immer
wieder ein Schauder vor allen meinen Schmerzen; ich weiß ja, daß du (o
Gott) mich nicht für schuldlos erklären wirst.«

\hypertarget{d-gott-will-nun-einmal-hiobs-recht-nicht-gelten-lassen-sonst-wuxfcrde-hiob-ihm-gern-rede-stehen}{%
\paragraph{d) Gott will nun einmal Hiobs Recht nicht gelten lassen,
sonst würde Hiob ihm gern Rede
stehen}\label{d-gott-will-nun-einmal-hiobs-recht-nicht-gelten-lassen-sonst-wuxfcrde-hiob-ihm-gern-rede-stehen}}

29»Ich muß nun einmal als schuldig gelten: wozu soll ich mich da noch
vergebens mühen? 30Wenn ich mich auch mit Schnee wüsche und meine Hände
mit Lauge reinigte, 31so würdest du mich doch in die schlammgefüllte
Grube eintauchen, so daß meine eigenen Kleider sich vor mir ekelten.
32Denn Gott ist nicht ein Mann wie ich, daß ich ihm Rede stünde, daß wir
zusammen vor Gericht treten könnten; 33es gibt zwischen uns keinen
Schiedsmann, der seine Hand auf uns beide legen könnte. 34Er nehme seine
Rute von mir weg und lasse seinen Schrecken mich nicht mehr ängstigen:
35so will ich reden, ohne mich vor ihm zu fürchten; denn nicht
also\textless sup title=``=~solcher Dinge''\textgreater✲ bin ich's mir
bewußt (daß ich ihn fürchten müßte).«

\hypertarget{e-wie-kann-nur-gott-bei-seiner-allwissenheit-und-vollkommenheit-ein-verfolger-hiobs-sein}{%
\paragraph{e) Wie kann nur Gott bei seiner Allwissenheit und
Vollkommenheit ein Verfolger Hiobs
sein?}\label{e-wie-kann-nur-gott-bei-seiner-allwissenheit-und-vollkommenheit-ein-verfolger-hiobs-sein}}

\hypertarget{section-9}{%
\section{10}\label{section-9}}

1»Mir ekelt vor meinem Leben: so will ich denn meiner Klage über
ihn\textless sup title=``d.h. Gott''\textgreater✲ freien Lauf lassen,
will reden in der Verzweiflung meiner Seele! 2Ich will zu Gott sagen:
›Behandle mich nicht als einen Frevler! Laß mich wissen, warum du gegen
mich im Streite liegst! 3Ist es wohlgetan von dir, daß du gewaltsam
verfährst, daß du das Gebilde deiner Hände verwirfst, während du zu den
Anschlägen der Frevler dein Licht leuchten läßt? 4Sind deine Augen von
Fleisch\textless sup title=``=~wie die eines Sterblichen''\textgreater✲,
oder siehst du die Dinge so an, wie Menschen sie sehen? 5Gleichen deine
Tage denen eines Sterblichen, oder sind deine Jahre wie die Lebenstage
eines Mannes, 6daß du nach einer Verschuldung bei mir suchst und nach
einer Missetat bei mir forschest, 7obgleich du weißt, daß es für mich
keine Rettung gibt, und daß niemand da ist, der mich aus deiner Hand
erretten kann?«

\hypertarget{f-gott-hat-hiob-zwar-kunstvoll-bereitet-und-ihm-fruxfcher-liebe-und-guxfcte-erwiesen-aber-er-hat-es-doch-von-anfang-an-feindlich-mit-ihm-gemeint}{%
\paragraph{f) Gott hat Hiob zwar kunstvoll bereitet und ihm früher Liebe
und Güte erwiesen, aber er hat es doch von Anfang an feindlich mit ihm
gemeint}\label{f-gott-hat-hiob-zwar-kunstvoll-bereitet-und-ihm-fruxfcher-liebe-und-guxfcte-erwiesen-aber-er-hat-es-doch-von-anfang-an-feindlich-mit-ihm-gemeint}}

8»Deine Hände haben mich kunstvoll gebildet und sorgsam gestaltet,
danach aber hast du dich dazu gewandt, mich zu vernichten. 9Denke doch
daran, daß du mich wie Ton geformt hast; und nun willst du mich wieder
zu Staub machen? 10Hast du mich nicht einstmals wie Milch hingegossen
und wie Molken\textless sup title=``oder: Käse''\textgreater✲ mich
gerinnen lassen? 11Mit Haut und Fleisch hast du mich umkleidet und mit
Knochen und Sehnen mich durchflochten; 12Leben und Huld\textless sup
title=``oder: Wohltaten''\textgreater✲ hast du mir gewährt, und deine
Obhut hat meinen Odem bewahrt. 13Doch du hast dabei im geheimen den
Gedanken gehegt -- ich weiß, daß dies bei dir fest beschlossen gewesen
ist --: 14Sobald ich sündigte, wolltest du es mir gedenken und mich von
meiner Verfehlung nicht freisprechen. 15Würde ich mich verschulden, dann
wehe mir! Aber auch wenn ich schuldlos bliebe, sollte ich doch mein
Haupt nicht erheben, sondern mit Schande gesättigt und mit Elend vollauf
getränkt werden; 16würde mein Haupt sich aber emporrichten: wie ein Löwe
wolltest du mich jagen und immer wieder deine Wundermacht an mir
erweisen; 17wolltest immer neue Zeugen gegen mich auftreten lassen und
deinen Zorn gegen mich noch steigern, ein immer neues Heer von Leiden
gegen mich aufbieten.«

\hypertarget{g-muxf6chte-gott-ihn-doch-nie-ins-leben-gerufen-haben-oder-ihm-jetzt-vor-dem-tode-ein-wenig-ruhe-schenken}{%
\paragraph{g) Möchte Gott ihn doch nie ins Leben gerufen haben oder ihm
jetzt vor dem Tode ein wenig Ruhe
schenken!}\label{g-muxf6chte-gott-ihn-doch-nie-ins-leben-gerufen-haben-oder-ihm-jetzt-vor-dem-tode-ein-wenig-ruhe-schenken}}

18»Aber warum hast du mich aus dem Mutterschoß hervorgehen lassen? Ich
hätte verscheiden sollen, noch ehe ein Auge mich sah, 19hätte werden
sollen, als wäre ich nie gewesen, vom Mutterschoß weg sogleich zum Grabe
getragen! 20Sind nicht meine Lebenstage nur noch wenige? So höre doch
auf und laß ab von mir, damit ich noch ein wenig heiter blicken✲ kann,
21bevor ich, ohne zurückzukehren, dahinfahre in das Land der Finsternis
und des Todesschattens, 22in das Land, das düster ist wie tiefe Nacht,
in das Land des Todesschattens und des Wustes, wo das Aufleuchten (des
Tages) so hell ist wie Finsternis.«

\hypertarget{erste-rede-zophars}{%
\subsubsection{5. Erste Rede Zophars}\label{erste-rede-zophars}}

\hypertarget{a-hiobs-gerede-verlangt-zuruxfcckweisung-gott-hat-mit-seinem-scharfblick-hiobs-schuld-klar-durchschaut-und-ihn-noch-mit-nachsicht-bestraft}{%
\paragraph{a) Hiobs Gerede verlangt Zurückweisung; Gott hat mit seinem
Scharfblick Hiobs Schuld klar durchschaut und ihn noch mit Nachsicht
bestraft}\label{a-hiobs-gerede-verlangt-zuruxfcckweisung-gott-hat-mit-seinem-scharfblick-hiobs-schuld-klar-durchschaut-und-ihn-noch-mit-nachsicht-bestraft}}

\hypertarget{section-10}{%
\section{11}\label{section-10}}

1Da nahm Zophar von Naama das Wort und sagte:

2»Soll (dieser) Wortschwall ohne Antwort bleiben und dieser Zungenheld
recht behalten? 3Dein Gerede sollte Männer zum Schweigen bringen, und du
solltest höhnen dürfen, ohne von jemand widerlegt zu werden?!« 4Du hast
ja doch behauptet: ›Meine Darlegung ist richtig‹, und: ›Ich stehe
unsträflich in deinen Augen da!‹ 5Ach, möchte Gott doch reden und seine
Lippen gegen dich auftun 6und dir die verborgenen Tiefen der Weisheit
offenbaren, daß sie allseitig an wahrem Wissen sind! Dann würdest du
erkennen, daß Gott dir einen Teil deiner Sündenschuld noch zugute hält.
7Kannst du den Urgrund der Gottheit erreichen oder bis zur
Vollkommenheit des Allmächtigen vordringen? 8Himmelhoch ist sie -- was
kannst du denn erreichen? Tiefer als das Totenreich ist sie -- wie weit
reicht denn dein Wissen? 9Länger als die Erde ist ihr Maß und breiter
als das Meer. 10Wenn er daherfährt und in Verhaft nimmt und zur
Gerichtsverhandlung ruft -- wer will ihm da wehren? 11Denn er kennt die
nichtswürdigen Leute und nimmt das Unrecht wahr, ohne besonderer
Aufmerksamkeit zu bedürfen.«

\hypertarget{b-hiob-muuxdf-seine-verblendung-ablegen-durch-ernstliche-buuxdfe-kann-er-noch-das-heil-erlangen-wuxe4hrend-der-frevler-verloren-ist}{%
\paragraph{b) Hiob muß seine Verblendung ablegen; durch ernstliche Buße
kann er noch das Heil erlangen, während der Frevler verloren
ist}\label{b-hiob-muuxdf-seine-verblendung-ablegen-durch-ernstliche-buuxdfe-kann-er-noch-das-heil-erlangen-wuxe4hrend-der-frevler-verloren-ist}}

12»Da muß selbst ein Hohlkopf zu Verstand\textless sup title=``oder: zur
Besinnung''\textgreater✲ kommen und ein Wildeselfüllen zum Menschen
umgeboren werden. 13Wenn du nun dein Herz in die rechte Verfassung
setzen und deine Hände zu ihm\textless sup title=``d.h. zu
Gott''\textgreater✲ ausbreiten wolltest~-- 14klebt eine Schuld an deiner
Hand, so entferne sie und laß in deinen Zelten kein Unrecht wohnen! --:
15ja, dann könntest du dein Angesicht vorwurfsfrei erheben und würdest
wie aus Erz gegossen✲ dastehen, frei von aller Furcht; 16ja, dann
würdest du dein Leiden vergessen, würdest daran zurückdenken wie an
Wasser, das sich verlaufen hat. 17Heller als der Mittag würde das Leben
dir aufgehen; mag auch einmal Dunkel dich umgeben, wie lichter Morgen
würde es werden. 18Du würdest dich dessen getrösten, daß noch Hoffnung
vorhanden sei, und wenn du Umschau hieltest, getrost dich zum Schlafen
niederlegen; 19du würdest dich lagern, ohne von jemand aufgeschreckt zu
werden, und viele würden sich um deine Gunst bemühen. 20Dagegen die
Augen der Frevler erlöschen: für sie ist jede Möglichkeit zum Entfliehen
verloren, und ihre (einzige) Hoffnung ist -- die Seele\textless sup
title=``=~das Leben''\textgreater✲ auszuhauchen!«

\hypertarget{hiobs-antwort-dritte-gegenrede}{%
\subsubsection{6. Hiobs Antwort (dritte
Gegenrede)}\label{hiobs-antwort-dritte-gegenrede}}

\hypertarget{a-hiobs-klage-uxfcber-die-eingebildete-weisheit-und-die-unbarmherzige-lieblosigkeit-der-freunde}{%
\paragraph{a) Hiobs Klage über die eingebildete Weisheit und die
unbarmherzige Lieblosigkeit der
Freunde}\label{a-hiobs-klage-uxfcber-die-eingebildete-weisheit-und-die-unbarmherzige-lieblosigkeit-der-freunde}}

\hypertarget{section-11}{%
\section{12}\label{section-11}}

1Da antwortete Hiob folgendermaßen:

2»Wahrhaftig, ihr seid das Volk\textless sup title=``=~vertretet die
ganze Menschheit''\textgreater✲, und mit euch wird die Weisheit
aussterben! 3Ich besitze auch Verstand ebensogut wie ihr: ich stehe
hinter euch nicht zurück; wem sollten auch derartige Dinge unbekannt
sein? 4Dem eigenen Freunde muß ich zum Spott dienen, ich, der ich vordem
Gott angerufen und auch Erhörung gefunden habe! Zum Spott muß der
Gerechte, der Fromme dienen! 5Dem Unglück gebührt Verachtung nach der
Ansicht des sich sicher Fühlenden: ein Stoß noch denen, deren Fuß
bereits wankt! 6In Ruhe liegen die Zelte von Gewalttätigen da, und in
Sicherheit leben die, welche Gott Trotz bieten, ein jeder, der seinen
Gott in seiner Faust führt.«

\hypertarget{b-gottes-allmacht-und-weisheit-wird-von-allen-seinen-geschuxf6pfen-bezeugt-ihre-erkenntnis-ist-kein-vorrecht-der-greise}{%
\paragraph{b) Gottes Allmacht und Weisheit wird von allen seinen
Geschöpfen bezeugt; ihre Erkenntnis ist kein Vorrecht der
Greise}\label{b-gottes-allmacht-und-weisheit-wird-von-allen-seinen-geschuxf6pfen-bezeugt-ihre-erkenntnis-ist-kein-vorrecht-der-greise}}

7»Aber frage doch das Vieh, das wird dich's lehren, und die Vögel des
Himmels, die werden dir's kundtun; 8oder betrachte (den Wurm auf der)
Erde, er wird dich's lehren, und die Fische des Meeres werden dir's
bezeugen: 9wer von diesen allen wüßte nicht, daß die Hand des HERRN
diese Welt geschaffen hat, 10er, in dessen Hand die Seele aller lebenden
Geschöpfe liegt und der Odem eines jeden Menschenwesens? 11Soll nicht
das Ohr die Worte prüfen, gleichwie der Gaumen sich die Speisen kostend
auswählt? 12›Bei den Greisen soll die Weisheit wohnen, und langes Leben
Einsicht verleihen?‹ 13Nein, bei ihm\textless sup title=``d.h. bei
Gott''\textgreater✲ wohnt Weisheit und Stärke, sein ist der Rat und die
Einsicht!«

\hypertarget{c-dem-menschen-stellt-sich-gottes-verhalten-als-sinnloses-furchtbares-und-willkuxfcrliches-walten-seiner-allmacht-dar}{%
\paragraph{c) Dem Menschen stellt sich Gottes Verhalten als sinnloses,
furchtbares und willkürliches Walten seiner Allmacht
dar}\label{c-dem-menschen-stellt-sich-gottes-verhalten-als-sinnloses-furchtbares-und-willkuxfcrliches-walten-seiner-allmacht-dar}}

14»Siehe, wenn er niederreißt, so wird nicht wieder aufgebaut; wen er
einkerkert, dem wird nicht wieder aufgetan. 15Siehe, wenn er die Wasser
hemmt, so versiegen sie, und wenn er sie entfesselt, so wühlen sie die
Erde um. 16Bei ihm ist Kraft und vollkommenes Wissen: ihm fällt der
Irrende wie der Irreführende in die Hände. 17Er läßt Ratsherren als
Barfüßige\textless sup title=``=~ihres Amtsschmucks
entkleidet''\textgreater✲ hinwegziehen und erweist Richter als Toren;
18die Zwingherrschaft von Königen löst er auf und schlingt ihnen einen
Strick um die eigenen Hüften; 19Priester führt er als
Barfüßige\textless sup title=``=~ihres Amtsschmucks
entkleidet''\textgreater✲ hinweg und bringt die im Amt Ergrauten zu
Fall; 20erprobten Wortführern entzieht er die Rede und benimmt den
Greisen\textless sup title=``=~alten Ratsherren''\textgreater✲ das
gesunde Urteil; 21über Edle gießt er Schande aus und löst den
Schwertgurt von Gewalthabern; 22Tiefverborgenes enthüllt er aus dem
Dunkel heraus und zieht finstere Nacht ans Licht hervor; 23er läßt
Völker groß aufwachsen und vernichtet sie wieder; er breitet Völker weit
aus und läßt sie dann verschleppen; 24er raubt den Volkshäuptern des
Landes den Verstand und läßt sie umherirren in pfadloser Einöde, 25daß
sie in lichtloser Finsternis tappen, und er läßt sie
umherirren\textless sup title=``oder: taumeln''\textgreater✲ wie
Trunkene.«

\hypertarget{d-hiob-stellt-sein-wissen-dem-der-freunde-gleich-und-beruft-sich-auf-das-wissen-gottes-der-in-den-freunden-nur-strafwuxfcrdige-luxfcgenanwuxe4lte-seiner-gerechtigkeit-erblicken-kuxf6nne}{%
\paragraph{d) Hiob stellt sein Wissen dem der Freunde gleich und beruft
sich auf das Wissen Gottes, der in den Freunden nur strafwürdige
Lügenanwälte seiner Gerechtigkeit erblicken
könne}\label{d-hiob-stellt-sein-wissen-dem-der-freunde-gleich-und-beruft-sich-auf-das-wissen-gottes-der-in-den-freunden-nur-strafwuxfcrdige-luxfcgenanwuxe4lte-seiner-gerechtigkeit-erblicken-kuxf6nne}}

\hypertarget{section-12}{%
\section{13}\label{section-12}}

1»Seht, dies alles hat mein Auge gesehen, hat mein Ohr gehört und es
sich gemerkt. 2Soviel ihr wißt, weiß ich auch: ich stehe hinter euch
nicht zurück. 3Doch ich will zum Allmächtigen reden und trage Verlangen,
mich mit Gott auseinanderzusetzen. 4Ihr dagegen seid nur Lügenschmiede,
Pfuscherärzte allesamt. 5O wolltet ihr doch ganz stille schweigen: das
würde euch als Weisheit angerechnet werden. 6Hört doch meine
Rechtfertigung an und achtet auf die Entgegnungen meiner Lippen! 7Wollt
ihr Gott zur Ehre Lügen reden und ihm zuliebe Trug vorbringen? 8Wollt
ihr Parteilichkeit zu seinen Gunsten üben oder Gottes Sachwalter✲
spielen? 9Würde es gut für euch ablaufen, wenn er euch ins Verhör nimmt,
oder könnt ihr ihn narren, wie man Menschen narrt? 10Mit aller Strenge
wird er euch strafen, wenn ihr im geheimen✲ Partei (für ihn) ergreift.
11Wird nicht sein bloßes Sich-Erheben euch fassungslos machen und
Schrecken vor ihm euch befallen? 12Eure Denksprüche sind Sprüche so lose
wie Asche, eure Schanzen erweisen sich als Schanzen von Lehm!«

\hypertarget{e-hiob-fordert-gott-zum-rechtsstreit-heraus}{%
\paragraph{e) Hiob fordert Gott zum Rechtsstreit
heraus}\label{e-hiob-fordert-gott-zum-rechtsstreit-heraus}}

\hypertarget{aa-hiob-tritt-in-diesen-rechtsstreit-zuversichtlich-ein-vorausgesetzt-dauxdf-gott-ihm-die-erforderliche-ruxfccksicht-durch-fernhalten-seines-schreckens-gewuxe4hren-wollte}{%
\subparagraph{aa) Hiob tritt in diesen Rechtsstreit zuversichtlich ein,
vorausgesetzt daß Gott ihm die erforderliche Rücksicht durch Fernhalten
seines Schreckens gewähren
wollte}\label{aa-hiob-tritt-in-diesen-rechtsstreit-zuversichtlich-ein-vorausgesetzt-dauxdf-gott-ihm-die-erforderliche-ruxfccksicht-durch-fernhalten-seines-schreckens-gewuxe4hren-wollte}}

13»So schweigt denn vor mir still: ich will reden, es mag über mich
hereinfahren, was da will! 14Warum sollte ich mein Fleisch in meinen
Zähnen forttragen und meine Seele\textless sup title=``=~mein
Leben''\textgreater✲ in meine offene Hand legen? 15Er wird mich ja doch
töten, ich habe auf nichts mehr zu hoffen; nur meinen bisherigen Wandel
will ich offen vor ihm darlegen. 16Schon das muß mir zugute kommen, denn
kein Heuchler darf ihm vor die Augen treten. 17So hört denn meine Rede
aufmerksam an und laßt meine Darlegung in euer Ohr dringen! 18Seht doch:
ich bin zum Rechtsstreit gerüstet! Ich weiß, daß ich, ja ich, recht
behalten werde. 19Wer ist es, der mit mir rechten dürfte? Denn in diesem
Fall wollte ich lieber verstummen und den Tod erleiden! 20Nur zweierlei
tu mir dabei nicht an (o Gott), dann will ich mich vor deinem Angesicht
nicht verbergen: 21ziehe deine Hand von mir zurück und laß deine
schreckliche Erscheinung mich nicht ängstigen! 22Dann rufe
mich\textless sup title=``oder: lade mich vor''\textgreater✲, so will
ich mich verantworten; oder ich will reden, und du entgegne mir!«

\hypertarget{bb-in-der-hoffnung-darauf-legt-er-gott-schon-jetzt-die-frage-nach-seiner-schuld-vor}{%
\subparagraph{bb) In der Hoffnung darauf legt er Gott schon jetzt die
Frage nach seiner Schuld
vor}\label{bb-in-der-hoffnung-darauf-legt-er-gott-schon-jetzt-die-frage-nach-seiner-schuld-vor}}

23»Wie viele Übertretungen und Missetaten habe ich (begangen)? Meine
Übertretung und meine Sünde laß mich wissen! 24Warum verbirgst du dein
Angesicht vor mir und siehst in mir deinen Feind? 25Willst du ein
verwehtes Blatt noch aufschrecken und einem dürren Strohhalm noch
nachjagen, 26daß du mir so bittere Arzneien verschreibst und mich sogar
die Verfehlungen meiner Jugend büßen läßt? 27Daß du meine Füße in den
Block legst und alle meine Pfade überwachst, meinen Füßen jede freie
Bewegung entziehst, 28mir, einem Manne, der wie ein vom Wurm
zerfressenes Gerät zerfällt, wie ein Kleid, das die Motten zernagt
haben?«

\hypertarget{cc-das-menschliche-leben-ist-kurz-und-dabei-voller-muxfchsal-warum-luxe4uxdft-gott-es-nicht-in-ruhe-verlaufen}{%
\subparagraph{cc) Das menschliche Leben ist kurz und dabei voller
Mühsal; warum läßt Gott es nicht in Ruhe
verlaufen?}\label{cc-das-menschliche-leben-ist-kurz-und-dabei-voller-muxfchsal-warum-luxe4uxdft-gott-es-nicht-in-ruhe-verlaufen}}

\hypertarget{section-13}{%
\section{14}\label{section-13}}

1»Der Mensch, vom Weibe geboren, ist arm an Lebenszeit, aber überreich
an Unruhe: 2wie eine Blume sprießt er auf und verwelkt, er flieht wie
ein Schatten dahin und hat keinen Bestand. 3Dennoch hältst du über einem
solchen (Wesen) deine Augen offen und ziehst ihn vor deinen
Richterstuhl! 4Wie könnte wohl ein Reiner von Unreinen herkommen? nein,
nicht ein einziger. 5Wenn denn seine Tage genau bemessen sind, wenn die
Zahl seiner Monde bei dir feststeht und du ihm eine Grenze gesetzt hast,
die er nicht überschreiten darf, 6so wende doch deine Blicke von ihm
weg, damit er Ruhe habe, bis er wie ein Tagelöhner mit Befriedigung auf
seinen Tag hinblicken kann!«

\hypertarget{dd-fuxfcr-den-menschen-gibt-es-nach-dem-tode-keine-hoffnung-keine-zukunft-mehr}{%
\subparagraph{dd) Für den Menschen gibt es nach dem Tode keine Hoffnung,
keine Zukunft
mehr}\label{dd-fuxfcr-den-menschen-gibt-es-nach-dem-tode-keine-hoffnung-keine-zukunft-mehr}}

7»Denn für einen Baum bleibt eine Hoffnung bestehen: wird er abgehauen,
so schlägt er von neuem aus, und seine Schößlinge hören nicht auf. 8Wenn
auch seine Wurzel in der Erde altert und sein Stumpf im Boden abstirbt,
9so treibt er doch vom Duft✲ des Wassers neue Sprossen und bringt Zweige
hervor wie ein junges Reis. 10Wenn aber ein Mann stirbt, so liegt er
hingestreckt da, und wenn ein Mensch verscheidet, wo ist er dann? 11Wie
das Wasser aus einem Teich verdunstet und ein Strom versiegt und
austrocknet, 12so legt der Mensch sich nieder und steht nicht wieder
auf: bis der Himmel nicht mehr ist, erwachen sie nicht wieder und werden
aus ihrem Schlaf nicht aufgerüttelt.«

\hypertarget{ee-hiob-kann-wegen-des-zustandes-der-verstorbenen-im-totenreich-keine-hoffnung-auf-auferstehung-auf-rechtfertigung-und-gluxfcck-haben-denn-mit-dem-tode-ist-alles-erfreuliche-zu-ende}{%
\subparagraph{ee) Hiob kann wegen des Zustandes der Verstorbenen im
Totenreich keine Hoffnung auf Auferstehung, auf Rechtfertigung und Glück
haben, denn mit dem Tode ist alles Erfreuliche zu
Ende}\label{ee-hiob-kann-wegen-des-zustandes-der-verstorbenen-im-totenreich-keine-hoffnung-auf-auferstehung-auf-rechtfertigung-und-gluxfcck-haben-denn-mit-dem-tode-ist-alles-erfreuliche-zu-ende}}

13»O wenn du mich doch im Totenreiche verwahrtest, mich dort verbergen
wolltest, bis dein Zorn sich gelegt hätte, mir eine Frist bestimmtest
und dann meiner gedächtest! 14Doch wenn der Mensch gestorben ist -- kann
er wohl wieder aufleben? Dann wollte ich alle Tage meines
Frondienstes\textless sup title=``oder: Leidenskampfes''\textgreater✲
harren, bis die Ablösung für mich käme: 15dann würdest du rufen und ich
gäbe dir Antwort; nach dem Werk deiner Hände würdest du Verlangen
tragen; 16ja, dann würdest du meine Schritte sorglich zählen, über einen
Fehltritt von mir kein strenger Wächter sein; 17nein, versiegelt würde
meine Übertretung in einem Bündel\textless sup title=``oder: im
Beutel''\textgreater✲ liegen, und meine Schuld hättest du
verklebt\textless sup title=``=~würdest du unbeachtet
lassen''\textgreater✲. 18Doch nein -- Berge stürzen in sich zusammen,
und Felsen werden von ihrer Stelle weggerückt, 19Steine höhlt das Wasser
aus, und seine Güsse schwemmen das Erdreich weg: so machst du auch die
Hoffnung des Menschen zunichte. 20Du überwältigst ihn auf ewig, und er
muß davon; sein Antlitz entstellend, läßt du ihn dahinfahren. 21Gelangen
seine Kinder zu Ehren -- er weiß nichts davon; und sinken sie in Schande
hinab -- er achtet nicht auf sie. 22Nur seines eigenen Leibes Schmerzen
fühlt er, und nur um sich selbst empfindet seine Seele Trauer.«

\hypertarget{iii.-zweiter-gespruxe4chsgang-kap.-15-21}{%
\subsection{III. Zweiter Gesprächsgang (Kap.
15-21)}\label{iii.-zweiter-gespruxe4chsgang-kap.-15-21}}

\hypertarget{zweite-rede-des-eliphas}{%
\subsubsection{1. Zweite Rede des
Eliphas}\label{zweite-rede-des-eliphas}}

\hypertarget{a-eliphas-ruxfcgt-hiobs-uxe4uuxdferungen-als-nichtiges-gottloses-und-duxfcnkelhaftes-gerede-gegen-gott}{%
\paragraph{a) Eliphas rügt Hiobs Äußerungen als nichtiges, gottloses und
dünkelhaftes Gerede gegen
Gott}\label{a-eliphas-ruxfcgt-hiobs-uxe4uuxdferungen-als-nichtiges-gottloses-und-duxfcnkelhaftes-gerede-gegen-gott}}

\hypertarget{section-14}{%
\section{15}\label{section-14}}

1Da nahm Eliphas von Theman das Wort und sagte:

2»Wird wohl ein Weiser windiges Wissen als Antwort vortragen und seine
Lunge mit (bloßem) Ostwind blähen, 3um sich mit Reden zu verantworten,
die nichts taugen, und mit Worten, durch die er nichts nützt? 4Dazu
vernichtest du die fromme Scheu und tust der Andachtsstille Abbruch, die
Gott gebührt; 5denn dein Schuldbewußtsein macht deinen Mund beredt, und
du wählst die Sprache der Verschmitzten. 6Dein eigener Mund verurteilt
dich, nicht ich, und deine eigenen Lippen zeugen gegen dich. 7Bist du
etwa als erster der Menschen geboren und noch vor den Bergen auf die
Welt gekommen? 8Hast du im Rate\textless sup title=``=~in der
Ratssitzung''\textgreater✲ Gottes als Zuhörer gelauscht und dort die
Weisheit an dich gerissen? 9Was weißt du denn, das wir nicht auch
wüßten? was verstehst du, das uns nicht auch bekannt wäre? 10Auch unter
uns sind Ergraute, sind Weißköpfe, reicher noch als dein Vater an
Lebenstagen.

11Sind dir die Tröstungen Gottes minderwertig, und gilt ein Wort der
Sanftmut nichts bei dir? 12Was reißt deine Leidenschaft dich fort, und
was rollen\textless sup title=``oder: zwinkern''\textgreater✲ deine
Augen, 13daß du gegen Gott deine Wut richtest und (solche) Reden deinem
Munde entfahren läßt? 14Was ist der Mensch, daß er rein sein könnte, und
der vom Weibe Geborene, daß er als gerecht dastände? 15Bedenke doch:
selbst seinen heiligen (Engeln) traut er nicht, und nicht einmal der
Himmel ist rein in seinen Augen: 16geschweige denn der Abscheuliche und
Entartete, der Mensch, dem Unrechttun wie Wassertrinken ist!«

\hypertarget{b-darlegung-und-begruxfcndung-der-von-den-vuxe4tern-uxfcberlieferten-lehre-vom-miuxdfgeschick-und-untergang-der-frevler}{%
\paragraph{b) Darlegung und Begründung der von den Vätern überlieferten
Lehre vom Mißgeschick und Untergang der
Frevler}\label{b-darlegung-und-begruxfcndung-der-von-den-vuxe4tern-uxfcberlieferten-lehre-vom-miuxdfgeschick-und-untergang-der-frevler}}

17»Ich will dich unterweisen: höre mir zu; und was ich gesehen habe,
will ich berichten, 18was die Weisen von ihren Vätern überkommen und
ohne Hehl verkündigt haben~-- 19ihnen war noch allein das Land
übergeben, und noch kein Fremder war unter ihnen umhergezogen --:
20›Sein ganzes Leben lang muß der Frevler sich ängstigen, und zwar alle
die Jahre hindurch, die dem Gewalttätigen beschieden sind.
21Schreckensrufe dringen ihm laut ins Ohr; mitten im ruhigen Glück
überfällt ihn der Verderber; 22er hegt keine Zuversicht, aus der
Finsternis wieder herauszukommen, und ist (in seiner Angst) für das
Schwert ausersehen. 23Er irrt nach Brot umher -- wo findet er's? Er
weiß, daß durch ihn\textless sup title=``d.h. Gott''\textgreater✲ der
Tag des Verderbens festgesetzt ist. 24Angst und Bangigkeit schrecken
ihn: sie überwältigen ihn wie ein König, der zum Sturm gerüstet ist.
25Weil er seine Hand gegen Gott erhoben und dem Allmächtigen Trotz
geboten hat~-- 26er stürmte gegen ihn an mit emporgerecktem Halse, mit
den dichten Buckeln seiner Schilde~-- 27weil er sein Gesicht von Fett
hatte strotzen lassen und Schmer an seinen Lenden angesetzt 28und sich
in gebannten Städten angesiedelt hatte, in Häusern, die unbewohnt
bleiben sollten, die zu Trümmerhaufen bestimmt waren: 29so bringt er's
nicht zu Reichtum, und sein Wohlstand hat keinen Bestand, und seine
Sichel\textless sup title=``oder: Ähre =~sein Besitz''\textgreater✲
neigt sich nicht zur Erde. 30Er kommt nicht aus der Finsternis heraus;
seine Schößlinge versengt die Gluthitze, und er selbst vergeht durch den
Zornhauch des Mundes Gottes. 31Er verlasse sich nicht auf Trug: er
täuscht sich nur; denn Trug wird auch das sein, was er durch seinen
eigenen (Trug) erzielt: 32ehe noch seine Zeit da ist, erfüllt sich sein
Geschick, während sein Wipfel noch nicht gegrünt hat. 33Wie der
Weinstock stößt er seine Beeren unreif ab und läßt wie der Ölbaum seine
Blüten abfallen. 34Denn die Rotte des Frevlers bleibt ohne Frucht, und
Feuer verzehrt die Zelte der Bestechung\textless sup title=``=~der
Bestechlichen''\textgreater✲. 35Mit Unheil gehen sie schwanger und
gebären Frevel, und ihr Inneres\textless sup title=``oder:
Schoß''\textgreater✲ bringt nur Selbsttäuschung zutage.‹«

\hypertarget{hiobs-antwort-vierte-gegenrede}{%
\subsubsection{2. Hiobs Antwort (vierte
Gegenrede)}\label{hiobs-antwort-vierte-gegenrede}}

\hypertarget{a-hiob-weist-die-truxf6stungen-der-freunde-als-windige-reden-und-als-hohn-zuruxfcck}{%
\paragraph{a) Hiob weist die Tröstungen der Freunde als windige Reden
und als Hohn
zurück}\label{a-hiob-weist-die-truxf6stungen-der-freunde-als-windige-reden-und-als-hohn-zuruxfcck}}

\hypertarget{section-15}{%
\section{16}\label{section-15}}

1Darauf antwortete Hiob folgendermaßen:

2»Dergleichen habe ich nun schon vieles gehört: leidige\textless sup
title=``oder: elende''\textgreater✲ Tröster seid ihr allesamt! 3Haben
die windigen Reden nun ein Ende? Oder was drängt dich dazu, mir noch
weiter zu erwidern? 4Auch ich könnte reden wie ihr -- o wärt ihr nur an
meiner Stelle! --, ich würde (aber) freundliche Worte gegen euch
aufbringen und beifällig mit dem Kopfe euch zunicken; 5ich wollte euch
mit meinem Munde Mut zusprechen, und das Beileid meiner Lippen sollte
euch Trost bringen!«

\hypertarget{b-gott-selbst-stempelt-hiob-durch-sein-leiden-offenbar-zum-suxfcnder-obgleich-er-ihn-schuldlos-weiuxdf-und-gibt-ihn-erbarmungslos-den-angriffen-der-freunde-und-der-verkennung-der-menschen-preis}{%
\paragraph{b) Gott selbst stempelt Hiob durch sein Leiden offenbar zum
Sünder, obgleich er ihn schuldlos weiß, und gibt ihn erbarmungslos den
Angriffen der Freunde und der Verkennung der Menschen
preis}\label{b-gott-selbst-stempelt-hiob-durch-sein-leiden-offenbar-zum-suxfcnder-obgleich-er-ihn-schuldlos-weiuxdf-und-gibt-ihn-erbarmungslos-den-angriffen-der-freunde-und-der-verkennung-der-menschen-preis}}

6»Wenn ich rede, wird mein Schmerz nicht gelindert, und wenn ich's
unterlasse -- um was werde ich erleichtert? 7Doch nunmehr hat
er\textless sup title=``d.h. Gott''\textgreater✲ meine Kraft erschöpft!
Verwüstet hast du meinen ganzen Hausstand\textless sup title=``oder:
Freundeskreis''\textgreater✲ 8und hast mich gepackt; das muß als Zeugnis
gegen mich gelten, und mein Siechtum\textless sup title=``oder: meine
Verlassenheit''\textgreater✲ tritt gegen mich auf, klagt mich ins
Angesicht an. 9Sein Zorn hat mich zerfleischt und befeindet; er hat mit
den Zähnen gegen mich geknirscht; als mein Gegner wirft er mir
durchbohrende Blicke zu. 10Ihr Maul haben sie gegen mich aufgerissen,
unter Schmähung mir Faustschläge ins Gesicht versetzt; zusammen hat man
sich vollzählig gegen mich aufgestellt. 11Gott hat mich Bösewichten
preisgegeben und mich in die Hände von Frevlern fallen lassen. 12In
Frieden lebte ich, da schreckte er mich auf, faßte mich beim Genick und
schmetterte mich nieder und ließ mich nur wieder aufstehen, damit ich
ihm als Zielscheibe diente: 13seine Pfeile umschwirren mich, er
durchbohrt mir die Nieren erbarmungslos, läßt mein Herzblut zur Erde
fließen. 14Er schlägt mir Wunde auf Wunde, stürmt gegen mich an wie ein
wilder Krieger. 15Das Trauergewand habe ich mir um den krustigen Leib
geheftet und mein Horn tief in den Staub hineingebohrt. 16Mein Gesicht
ist vom Weinen hochgerötet, und auf meinen Augenlidern lagert
Todesschatten, 17obwohl keine Schuld an meinen Händen klebt und mein
Gebet aufrichtig ist.«

\hypertarget{c-dennoch-bleibt-gott-hiobs-zeuge-und-buxfcrge-fuxfcr-seine-unschuld-und-wird-wenn-auch-erst-nach-hiobs-tode-fuxfcr-ihn-eintreten}{%
\paragraph{c) Dennoch bleibt Gott Hiobs Zeuge und Bürge für seine
Unschuld und wird, wenn auch erst nach Hiobs Tode, für ihn
eintreten}\label{c-dennoch-bleibt-gott-hiobs-zeuge-und-buxfcrge-fuxfcr-seine-unschuld-und-wird-wenn-auch-erst-nach-hiobs-tode-fuxfcr-ihn-eintreten}}

18»O Erde, decke mein Blut nicht zu, und mein Wehgeschrei finde keine
Ruhestatt! 19Schon jetzt -- wisset es wohl! -- ist ein Zeuge für mich im
Himmel vorhanden und mein Bürge\textless sup title=``oder:
Eideshelfer''\textgreater✲ in der Höhe. 20Meine Freunde verhöhnen mich
-- zu Gott blickt mein Auge tränenvoll empor, 21daß er dem
Manne\textless sup title=``oder: Sterblichen''\textgreater✲ Recht
schaffe Gott gegenüber und zwischen dem Menschen und seinem Freunde
entscheide. 22Denn nur noch wenige Jahre werden kommen, dann werde ich
den Pfad wandeln, auf dem es keine Rückkehr für mich gibt.

\hypertarget{section-16}{%
\section{17}\label{section-16}}

1Meine Lebenskraft ist gebrochen, meine Tage sind erloschen; nur die
Gräberstätte\textless sup title=``=~der Friedhof''\textgreater✲ wartet
meiner noch!«

\hypertarget{d-hiob-legt-die-gruxfcnde-dar-die-gott-gegenuxfcber-der-torheit-und-gefuxfchllosigkeit-der-freunde-und-mit-ruxfccksicht-auf-die-teilnahme-der-frommen-zum-eintreten-fuxfcr-ihn-veranlassen-muxfcssen}{%
\paragraph{d) Hiob legt die Gründe dar, die Gott gegenüber der Torheit
und Gefühllosigkeit der Freunde und mit Rücksicht auf die Teilnahme der
Frommen zum Eintreten für ihn veranlassen
müssen}\label{d-hiob-legt-die-gruxfcnde-dar-die-gott-gegenuxfcber-der-torheit-und-gefuxfchllosigkeit-der-freunde-und-mit-ruxfccksicht-auf-die-teilnahme-der-frommen-zum-eintreten-fuxfcr-ihn-veranlassen-muxfcssen}}

2»Wahrlich, der Spott treibt sein Spiel mit mir, und mein Auge muß auf
ihren Beleidigungen weilen! 3O setze doch das Pfand ein, verbürge dich
doch für mich bei dir selbst! Wer sollte sonst als Bürge mir den
Handschlag leisten? 4Denn ihr Herz hast du der Einsicht verschlossen;
darum kannst du sie auch nicht obsiegen✲ lassen. 5Wenn jemand seine
Freunde verrät, um etwas von ihrem Besitz an sich zu bringen, so werden
die Augen seiner Kinder dafür verschmachten. 6Und mich hat
er\textless sup title=``d.h. Gott''\textgreater✲ für alle Welt zum
Gespött gemacht, und ich muß mir ins Angesicht speien lassen; 7da ist
mein Auge vor Gram erloschen, und alle meine Glieder sind nur noch wie
ein Schatten. 8Darüber entsetzen sich die Rechtschaffenen, und der
Unschuldige gerät in Empörung über den Ruchlosen. 9Doch der Gerechte
soll\textless sup title=``oder: wird''\textgreater✲ an seinem Wege
festhalten, und wer reine Hände hat, soll\textless sup title=``oder:
wird''\textgreater✲ an Kraft noch zunehmen.«

\hypertarget{e-hiob-weist-die-bekehrungs--und-trostreden-der-freunde-als-tuxf6richt-zuruxfcck-da-er-mit-dem-leben-abgeschlossen-habe}{%
\paragraph{e) Hiob weist die Bekehrungs- und Trostreden der Freunde als
töricht zurück, da er mit dem Leben abgeschlossen
habe}\label{e-hiob-weist-die-bekehrungs--und-trostreden-der-freunde-als-tuxf6richt-zuruxfcck-da-er-mit-dem-leben-abgeschlossen-habe}}

10»Ihr alle aber, kommt immerhin aufs neue heran: ich werde doch keinen
Weisen unter euch finden. 11Meine Tage sind abgelaufen, meine Pläne
vereitelt, die Bestrebungen meines Herzens! 12Die Nacht wollen sie zum
Tage machen: das Licht soll mir näher sein als die Finsternis! 13Wenn
ich schon das Totenreich als meine Behausung erwarte, in der Finsternis
mir mein Lager schon ausgebreitet habe, 14wenn ich dem Grabe bereits
zugerufen habe: ›Mein Vater bist du!‹ und dem Gewürm: ›Meine Mutter und
meine Schwester!‹~-- 15wo ist da noch eine Hoffnung für mich? Ja, eine
Hoffnung für mich -- wer mag sie erschauen? 16Zu den
Riegeln\textless sup title=``=~Toren, Pforten''\textgreater✲ des
Totenreichs fährt sie (die Hoffnung) hinab, wenn zugleich (für den Leib)
im Staube✲ Ruhe sein wird.«

\hypertarget{zweite-rede-bildads}{%
\subsubsection{3. Zweite Rede Bildads}\label{zweite-rede-bildads}}

\hypertarget{a-uxe4uuxdferung-des-unwillens-uxfcber-hiobs-anmauxdfende-und-selbstgerechte-reden}{%
\paragraph{a) Äußerung des Unwillens über Hiobs anmaßende und
selbstgerechte
Reden}\label{a-uxe4uuxdferung-des-unwillens-uxfcber-hiobs-anmauxdfende-und-selbstgerechte-reden}}

\hypertarget{section-17}{%
\section{18}\label{section-17}}

1Da nahm Bildad von Suah das Wort und sagte:

2»Wie lange wollt ihr noch Jagd auf (bloße) Worte machen? Nehmt Verstand
an: dann wollen wir reden! 3Warum werden wir den vernunftlosen Tieren
gleichgeachtet, von euch als vernagelt\textless sup title=``oder:
stockdumm''\textgreater✲ angesehen? 4Du, der in seinem Zorn sich selbst
zerfleischt -- soll um deinetwillen die Erde menschenleer werden und der
Fels von seiner Stelle wegrücken?«

\hypertarget{b-nochmalige-schilderung-des-unfehlbaren-und-schrecklichen-untergangs-den-gott-fuxfcr-den-frevler-und-seine-angehuxf6rigen-bereithuxe4lt}{%
\paragraph{b) Nochmalige Schilderung des unfehlbaren und schrecklichen
Untergangs, den Gott für den Frevler und seine Angehörigen
bereithält}\label{b-nochmalige-schilderung-des-unfehlbaren-und-schrecklichen-untergangs-den-gott-fuxfcr-den-frevler-und-seine-angehuxf6rigen-bereithuxe4lt}}

5»Jawohl, das Licht des Frevlers wird erlöschen und die Flamme seines
Herdfeuers nicht mehr leuchten; 6das Licht wird dunkel werden in seinem
Zelt, und seine Leuchte erlischt über ihm; 7seine sonst so rüstigen
Schritte werden kurz, und seine eigenen Anschläge bringen ihn zu Fall;
8denn er wird von seinen eigenen Füßen ins Netz getrieben, und auf
Fallgittern wandelt er dahin. 9Die Schlinge erfaßt seine Ferse, der
Fallstrick hält ihn fest; 10am Boden liegt das Fanggarn für ihn
verborgen, und die Falle wartet seiner auf dem Pfade. 11Ringsum
ängstigen ihn Schrecknisse und hetzen ihn auf Schritt und Tritt. 12Das
ihm bestimmte Unheil hungert nach ihm, und das Verderben steht zu seinem
Sturz bereit. 13Es frißt die Glieder seines Leibes, es frißt seine
Glieder der erstgeborene Sohn des Todes. 14Herausgerissen wird er aus
seinem Zelt, wo er sich sicher fühlte, und es treibt ihn hin zum König
der Schrecken. 15In seinem Zelt haust eine Bewohnerschaft, die nicht zu
ihm gehört; Schwefel wird auf seine Wohnstätte gestreut. 16Unten
verdorren seine Wurzeln, und oben verwelken seine Zweige. 17Das Andenken
an ihn verschwindet von der Erde\textless sup title=``oder: aus dem
Lande''\textgreater✲, und kein Name verbleibt ihm draußen weit und
breit; 18er\textless sup title=``d.h. Gott''\textgreater✲ stößt ihn aus
dem Licht in die Finsternis hinaus und verjagt ihn vom Erdenrund.
19Nicht Sproß noch Schoß\textless sup title=``=~kein Sohn und kein
Enkel''\textgreater✲ bleibt ihm in seinem Volk erhalten, und kein
Überlebender findet sich in seinen Wohnsitzen. 20Ob seinem
Gerichtstage\textless sup title=``d.h. Endgeschick''\textgreater✲
schaudern die im Westen Wohnenden, und die Leute im Osten erfaßt
Entsetzen. 21Ja, so ergeht es den Wohnungen\textless sup title=``=~dem
Heim''\textgreater✲ des Frevlers und so der Stätte des
Gottesverächters!«

\hypertarget{hiobs-antwort-fuxfcnfte-gegenrede}{%
\subsubsection{4. Hiobs Antwort (fünfte
Gegenrede)}\label{hiobs-antwort-fuxfcnfte-gegenrede}}

\hypertarget{a-hiobs-klage-uxfcber-seine-freunde-die-ihn-ohne-beweis-beschimpfen-statt-die-schuld-auf-die-grundlose-feindschaft-gottes-zu-schieben}{%
\paragraph{a) Hiobs Klage über seine Freunde, die ihn ohne Beweis
beschimpfen, statt die Schuld auf die grundlose Feindschaft Gottes zu
schieben}\label{a-hiobs-klage-uxfcber-seine-freunde-die-ihn-ohne-beweis-beschimpfen-statt-die-schuld-auf-die-grundlose-feindschaft-gottes-zu-schieben}}

\hypertarget{section-18}{%
\section{19}\label{section-18}}

1Da antwortete Hiob folgendermaßen:

2»Wie lange wollt ihr mein Herz noch betrüben und mich mit Reden
martern? 3Schon zehnmal habt ihr mich geschmäht; ihr schämt euch nicht,
mir wehzutun! 4Und hätte ich mich wirklich verfehlt, so wäre doch meine
Verfehlung meine eigene Sache. 5Wollt ihr wirklich gegen mich
großtun\textless sup title=``=~über mich triumphieren''\textgreater✲, so
erbringt mir den Beweis für das mich Beschämende! 6Erkennt doch, daß
Gott mir unrecht getan und mich mit seinem Fangnetz rings umgarnt hat!«

\hypertarget{b-hiobs-klage-uxfcber-das-schwere-von-gott-zu-unrecht-ihm-zugefuxfcgte-leid-und-uxfcber-das-veruxe4chtliche-auftreten-der-menschen-gegen-ihn}{%
\paragraph{b) Hiobs Klage über das schwere, von Gott zu Unrecht ihm
zugefügte Leid und über das verächtliche Auftreten der Menschen gegen
ihn}\label{b-hiobs-klage-uxfcber-das-schwere-von-gott-zu-unrecht-ihm-zugefuxfcgte-leid-und-uxfcber-das-veruxe4chtliche-auftreten-der-menschen-gegen-ihn}}

7»Seht: schreie ich über Gewalttat, so finde ich keine Erhörung; rufe
ich um Hilfe, so gibt es keinen Rechtsspruch. 8Den Weg hat er mir
vermauert, so daß ich nicht weiterschreiten kann, und über meine Pfade
hat er Finsternis ausgebreitet. 9Meiner Ehre hat er mich entkleidet und
die Krone mir vom Haupte weggenommen. 10Er hat mich niedergerissen um
und um, so daß es aus mit mir ist, und hat meine Hoffnung ausgerissen
wie einen Baum. 11Er hat seinen Zorn gegen mich lodern lassen und mich
seinen Feinden gleichgeachtet. 12Allzumal sind seine Kriegerscharen
herangerückt, haben sich einen Weg zum Angriff gegen mich aufgeschüttet
und sich rings um mein Zelt her gelagert. 13Meine Brüder haben sich fern
von mir gehalten, und meine Bekannten sind mir ganz entfremdet; 14meine
Verwandten bleiben weg, und meine vertrauten Freunde haben mich
vergessen; 15meine Hausgenossen und selbst meine Mägde sehen in mir
einen Fremden: ein Unbekannter bin ich in ihren Augen geworden. 16Rufe
ich meinen Knecht, so antwortet er mir nicht: ich muß ihn anflehen und
ihm gute Worte geben. 17Mein Atem ist meinem Weibe zuwider und mein
übler Geruch meinen leiblichen Brüdern. 18Selbst die Buben mißachten
mich: mache ich (vergebliche) Versuche zum Aufstehen, so verspotten sie
mich. 19Allen meinen Vertrauten ekelt vor mir, und die ich liebgehabt
habe, stehen mir feindlich gegenüber. 20An meiner Haut und meinem
Fleisch kleben meine Knochen, und von meinen Zähnen habe ich nur die
Haut übrigbehalten.«

\hypertarget{c-hiob-bittet-die-freunde-um-mitleid-und-spricht-die-feste-hoffnung-aus-dauxdf-gott-ihm-dereinst-recht-schaffen-aber-auch-die-gefuxfchllosigkeit-der-freunde-strafen-werde}{%
\paragraph{c) Hiob bittet die Freunde um Mitleid und spricht die feste
Hoffnung aus, daß Gott ihm dereinst Recht schaffen, aber auch die
Gefühllosigkeit der Freunde strafen
werde}\label{c-hiob-bittet-die-freunde-um-mitleid-und-spricht-die-feste-hoffnung-aus-dauxdf-gott-ihm-dereinst-recht-schaffen-aber-auch-die-gefuxfchllosigkeit-der-freunde-strafen-werde}}

21»Habt Mitleid, habt Mitleid mit mir, ihr meine Freunde! Denn Gottes
Hand hat mich schwer getroffen. 22Warum verfolgt ihr mich ebenso wie
Gott und werdet nicht satt, mich zu zerfleischen? 23O daß doch meine
Worte aufgeschrieben, o daß sie in ein Buch eingetragen würden, 24mit
eisernem Griffel in Blei eingegraben, auf ewig in den Felsen eingehauen
würden! 25Ich aber, ich weiß, daß mein Löser✲ lebt und als letzter auf
dem Staube\textless sup title=``d.h. hier auf der Erde''\textgreater✲
auftreten wird; 26und danach werde ich, mag jetzt auch meine Haut so
ganz zerfetzt und ich meines Fleisches ledig\textless sup title=``oder:
beraubt''\textgreater✲ sein, Gott schauen, 27den ich schauen werde mir
zum Heil und den meine Augen sehen werden, und zwar nicht mehr als einen
Entfremdeten✲, ihn, um den sich mir das Herz in der Brust abgehärmt hat.
28Wenn ihr aber sagt: ›Wie wollen wir ihn verfolgen!‹ und ›der letzte
Grund der Sache\textless sup title=``d.h. meiner Leiden''\textgreater✲
sei in mir selbst zu finden‹, 29so fürchtet euch vor dem Schwert -- denn
derartige Verschuldungen verdienen die Strafe des Schwertes --, damit
ihr erkennt, daß es noch ein Gericht gibt!«

\hypertarget{zweite-rede-zophars}{%
\subsubsection{5. Zweite Rede Zophars}\label{zweite-rede-zophars}}

\hypertarget{a-kurze-abweisung-der-beleidigenden-rede-hiobs}{%
\paragraph{a) Kurze Abweisung der beleidigenden Rede
Hiobs}\label{a-kurze-abweisung-der-beleidigenden-rede-hiobs}}

\hypertarget{section-19}{%
\section{20}\label{section-19}}

1Nun nahm Zophar von Naama das Wort und sagte:

2»Eben darum veranlassen meine Gedanken mich zu einer Antwort, und eben
deswegen bin ich innerlich erregt: 3eine mich beschimpfende
Zurechtweisung muß ich hören! Doch der Geist gibt mir eine Antwort aus
meiner Einsicht ein.«

\hypertarget{b-leidenschaftliche-schilderung-des-unfehlbaren-untergangs-des-frevlers-mit-liebloser-anspielung-auf-hiobs-vermutliche-freveltaten}{%
\paragraph{b) Leidenschaftliche Schilderung des unfehlbaren Untergangs
des Frevlers mit liebloser Anspielung auf Hiobs vermutliche
Freveltaten}\label{b-leidenschaftliche-schilderung-des-unfehlbaren-untergangs-des-frevlers-mit-liebloser-anspielung-auf-hiobs-vermutliche-freveltaten}}

4»Kennst du nicht die Wahrheit von alters her, seitdem der Mensch seinen
Wohnsitz auf der Erde hat, 5daß das Frohlocken der Frevler von kurzer
Dauer ist und die Freude der Ruchlosen nur einen Augenblick währt?
6Sollte auch sein Dünkel sich bis zum Himmel erheben und sein Haupt bis
an die Wolken reichen, 7so vergeht er doch wie sein Unrat für immer, und
die ihn gekannt haben, werden fragen: ›Wo ist er geblieben?‹ 8Wie ein
Traum verfliegt er, so daß man ihn nicht mehr findet, und er wird
hinweggescheucht wie ein Nachtgesicht: 9das Auge, das ihn gesehen,
erblickt ihn nimmer wieder, und seine Stätte gewahrt ihn nicht mehr.
10Seine Söhne müssen die (durch ihn) Verarmten mit Bitten beschwichtigen
und seine eigenen Hände\textless sup title=``oder: seine
Kinder''\textgreater✲ sein Vermögen wieder herausgeben. 11Mögen auch
seine Glieder von Jugendkraft strotzen: sie muß sich doch mit ihm in den
Staub legen. 12Mag das Böse auch seinem Munde süß schmecken, so daß er
es lange unter seiner Zunge birgt, 13daß er es schonend hegt und es
nicht fahren lassen will, sondern es an seinem Gaumen zurückhält, 14so
verwandelt sich doch seine Speise in seinen Eingeweiden: zu Otterngalle
wird sie in seinem Leibe. 15Den Reichtum, den er verschlungen hat, muß
er wieder ausspeien: aus seinem Bauche treibt Gott ihn wieder heraus.
16Otterngift hat er eingesogen: nun gibt ihm die Zunge der Viper den
Tod. 17Nicht darf er seine Lust mehr sehen an den Bächen, an den
wogenden Strömen von Honig und Sahne. 18Das Erraffte muß er wieder
herausgeben, ohne es verschlucken zu können; wieviel Gut er auch
erworben hat, er darf nicht frohlocken\textless sup title=``oder: er
findet kein Ergötzen daran''\textgreater✲. 19Denn er hat die Armen
niedergeschlagen und hilflos verkommen lassen, hat Häuser an sich
gerissen, wird sie aber nicht häuslich einrichten dürfen; 20denn er
kannte keine Befriedigung in seiner Gier: darum wird er auch von seinen
Kostbarkeiten nichts davonbringen. 21Nichts entging seinem
Fressen\textless sup title=``=~seiner unersättlichen
Gier''\textgreater✲: darum hat sein Wohlstand keine Dauer. 22In der
Fülle seines Überflusses wird ihm enge: die ganze Gewalt des Unheils
kommt über ihn. 23Da entfesselt Gott dann, um ihm den Bauch zu füllen,
seine Zornesglut gegen ihn und läßt sie als seine Speise auf ihn regnen.
24Flieht er vor der eisernen Rüstung, so durchbohrt ihn der eherne
Bogen; 25er zieht den Pfeil heraus, da fährt's aus seinem Rücken hervor:
ein Blutstrahl schießt aus seiner Galle\textless sup title=``=~seinem
Herzen''\textgreater✲, Todesschrecken brechen über ihn herein. 26Alles
Unheil ist seinen Schätzen aufgespart: ein Feuer, das nicht (von
Menschen) angefacht ist, frißt sie und verzehrt, was in seinem Zelt noch
übriggeblieben ist. 27Der Himmel deckt Sündenschuld auf, und die Erde
erhebt sich gegen ihn. 28Was in seinem Hause zusammengescharrt liegt,
wird weggeschleppt, zerrinnt (wie Wasser) am Tage des göttlichen
Zorngerichts. 29Das ist des ruchlosen Menschen Teil\textless sup
title=``oder: Schicksalslos''\textgreater✲ von seiten Gottes und das vom
Allherrn ihm zugesprochene Erbe.«

\hypertarget{hiobs-antwort-sechste-gegenrede}{%
\subsubsection{6. Hiobs Antwort (sechste
Gegenrede)}\label{hiobs-antwort-sechste-gegenrede}}

\hypertarget{a-hiobs-bitte-an-die-freunde-seine-folgende-bedeutsame-wenn-auch-schmerzlich-wirkende-darlegung-anzuhuxf6ren}{%
\paragraph{a) Hiobs Bitte an die Freunde, seine folgende bedeutsame,
wenn auch schmerzlich wirkende Darlegung
anzuhören}\label{a-hiobs-bitte-an-die-freunde-seine-folgende-bedeutsame-wenn-auch-schmerzlich-wirkende-darlegung-anzuhuxf6ren}}

\hypertarget{section-20}{%
\section{21}\label{section-20}}

1Darauf antwortete Hiob folgendermaßen:

2»Hört, o höret an, was ich zu sagen habe! Das soll mir eure Tröstungen
ersetzen! 3Erlaubt mir, daß ich rede, und nachdem ich gesprochen habe,
magst du es bespötteln! 4Richtet sich meine Klage etwa gegen Menschen?
Oder warum sollte ich nicht ungeduldig werden? 5Wendet euch her zu mir,
so werdet ihr euch entsetzen und euch die Hand auf den Mund legen! 6Wenn
ich bloß daran denke, gerate ich in Bestürzung, und ein Schauder
überläuft meinen Leib!«

\hypertarget{b-feststellung-der-tatsache-dauxdf-die-frevler-oft-im-leben-und-im-sterben-gluxfccklich-sind}{%
\paragraph{b) Feststellung der Tatsache daß die Frevler (oft) im Leben
und im Sterben glücklich
sind}\label{b-feststellung-der-tatsache-dauxdf-die-frevler-oft-im-leben-und-im-sterben-gluxfccklich-sind}}

7»Warum bleiben die Frevler am Leben, werden alt, nehmen sogar an Kraft
zu? 8Ihr Nachwuchs steht bei fester Gesundheit vor ihnen, ja neben
ihnen, und deren Sprößlinge vor ihren Augen. 9Ihre Häuser stehen
ungefährdet da, ohne Furcht vor Schrecknis, und Gottes Zuchtrute fährt
nicht auf sie nieder. 10Sein Stier belegt und befruchtet sicher, seine
Kuh kalbt leicht und tut keine Fehlgeburt. 11Ihre Buben lassen sie wie
eine Herde Lämmer ausziehen, und ihre kleineren Kinder hüpfen tanzend
umher; 12sie singen laut zur Pauke und Zither und sind vergnügt beim
Klang der Schalmei. 13Sie verbringen im Wohlergehen ihre Tage und fahren
in Ruhe zum Totenreich hinab\textless sup title=``=~erleiden einen
schmerzlosen Tod''\textgreater✲. 14Und doch haben sie zu Gott gesagt:
›Bleibe fern von uns; denn nach der Erkenntnis deiner Wege tragen wir
kein Verlangen. 15Was ist der Allmächtige, daß wir ihm dienen sollten?
Und könnte es uns nützen, daß wir ihn mit Bitten angehen?‹«

\hypertarget{c-gluxfcck-und-ungluxfcck-werden-von-gott-willkuxfcrlich-ausgeteilt}{%
\paragraph{c) Glück und Unglück werden von Gott willkürlich
ausgeteilt}\label{c-gluxfcck-und-ungluxfcck-werden-von-gott-willkuxfcrlich-ausgeteilt}}

16»Seht, ihr Wohlergehen liegt allerdings nicht in ihrer Hand -- die
Denkweise der Frevler steht mir fern! --, 17aber wie oft kommt es denn
vor, daß die Leuchte der Frevler erlischt und ihr Verderben über sie
hereinbricht? Daß Gott ihnen die Lose gemäß seinem Zorn zuteilt? 18Daß
es ihnen ergeht wie dem Strohhalm vor dem Wind und wie der Spreu, die
der Sturm entführt hat? 19›Gott spart‹, sagt ihr, ›sein Unheil für die
Kinder des Frevlers auf‹ -- doch ihm selber sollte er vergelten, daß er
es fühlte! 20Sehen müßten seine eigenen Augen das Verderben, und er
selbst sollte von der Zornglut des Allmächtigen trinken! 21Denn was wird
er sich noch um seine Familie nach seinem Tode kümmern, nachdem die Zahl
seiner Monde abgeschnitten\textless sup title=``=~zu Ende''\textgreater✲
ist? 22Doch -- darf man Gott Erkenntnis lehren, ihn, der die himmlischen
(Geister) richtet? 23Der eine stirbt im Vollbesitz des Glücks, ganz
sorgenfrei und in Ruhe: 24seine Kufen sind mit Milch gefüllt, und so ist
das Mark in seinen Knochen wohlversorgt; 25der andere aber stirbt in
bitterem Herzeleid, ohne je vom Glück etwas geschmeckt zu haben:
26gleicherweise liegen sie in der Erde, und Gewürm legt sich als Decke
über beide.«

\hypertarget{d-die-anwendung-der-lehre-von-gottes-vergeltung-auf-hiob-und-die-vertruxf6stung-hiobs-auf-spuxe4teres-gluxfcck-ist-verfehlt-oder-gar-boshaft}{%
\paragraph{d) Die Anwendung der Lehre von Gottes Vergeltung auf Hiob und
die Vertröstung Hiobs auf späteres Glück ist verfehlt oder gar
boshaft}\label{d-die-anwendung-der-lehre-von-gottes-vergeltung-auf-hiob-und-die-vertruxf6stung-hiobs-auf-spuxe4teres-gluxfcck-ist-verfehlt-oder-gar-boshaft}}

27»Seht, ich kenne eure Gedanken wohl und die Anschläge, mit denen ihr
mir Gewalt antut. 28Wenn ihr sagt: ›Wo ist das Haus des Gewaltmenschen
geblieben und wo das Zelt, in welchem die Frevler wohnten?‹~-- 29habt
ihr euch denn noch nie bei den weitgereisten\textless sup title=``oder:
des Wegs vorüberziehenden''\textgreater✲ Leuten erkundigt, deren
beweiskräftige Aussagen ihr doch nicht verwerfen könnt: 30daß am
Unglückstage der Böse verschont bleibt und am Tage des (göttlichen)
Zorngerichts heil davonkommt? 31Wer hält ihm auch nur seinen
Lebenswandel unverhohlen vor? Und hat er etwas verübt, wer vergilt es
ihm? 32Nein, man gibt ihm noch das feierliche Geleit zur Gräberstätte
und hält über seinem Grabhügel noch Wache. 33Sanft liegen auf ihm die
Schollen des Tales, und hinter ihm her zieht alle Welt, wie Unzählige
ihm vorangegangen sind. 34Wie mögt ihr mir da so nichtigen Trost bieten?
Und eure Entgegnungen -- von denen bleibt nur Treubruch übrig!«

\hypertarget{iv.-dritter-gespruxe4chsgang-kap.-22-26}{%
\subsection{IV. Dritter Gesprächsgang (Kap.
22-26)}\label{iv.-dritter-gespruxe4chsgang-kap.-22-26}}

\hypertarget{dritte-rede-des-eliphas}{%
\subsubsection{1. Dritte Rede des
Eliphas}\label{dritte-rede-des-eliphas}}

\hypertarget{a-eliphas-spricht-nun-ruxfcckhaltlos-aus-dauxdf-hiob-sein-ungluxfcck-verdient-habe}{%
\paragraph{a) Eliphas spricht nun rückhaltlos aus, daß Hiob sein Unglück
verdient
habe}\label{a-eliphas-spricht-nun-ruxfcckhaltlos-aus-dauxdf-hiob-sein-ungluxfcck-verdient-habe}}

\hypertarget{section-21}{%
\section{22}\label{section-21}}

1Da nahm Eliphas von Theman das Wort und sagte:

2»Kann wohl ein Mensch Gott Nutzen schaffen? Nein, nur sich selbst nützt
der Fromme\textless sup title=``oder: Verständige''\textgreater✲. 3Hat
der Allmächtige Vorteil davon, wenn du rechtschaffen bist? Oder bringt
es ihm Gewinn, wenn du unsträflich wandelst? 4Meinst du, wegen deiner
Gottesfurcht strafe er dich und gehe deshalb mit dir ins Gericht? 5Ist
nicht vielmehr deine Bosheit groß, und sind nicht deine Verschuldungen
ohne Ende?«

\hypertarget{b-hiob-hat-seine-strafe-durch-schwere-freveltaten-verdient}{%
\paragraph{b) Hiob hat seine Strafe durch schwere Freveltaten
verdient}\label{b-hiob-hat-seine-strafe-durch-schwere-freveltaten-verdient}}

6»Denn oftmals hast du deine Volksgenossen ohne Grund gepfändet und den
Halbnackten ihre Kleider ausziehen lassen; 7dem vor Durst Lechzenden
hast du keinen Trunk Wasser gereicht und dem Hungrigen ein Stück Brot
versagt. 8Dem Manne der Faust -- ihm gehörte das Land, und nur die
Hochangesehenen durften darin wohnen. 9Witwen ließest du mit leeren
Händen gehen, und alles, was den Waisen zu Gebote stand, wurde zugrunde
gerichtet. 10Darum bist du jetzt rings von Schlingen umgeben, und jäher
Schrecken versetzt dich in Angst; 11dein Licht ist Finsternis geworden,
so daß du nicht sehen kannst, und eine Wasserflut bedeckt dich.«

\hypertarget{c-hiob-hat-sich-durch-gottlose-gesinnung-und-durch-frevelhafte-reden-gegen-gott-schwer-versuxfcndigt}{%
\paragraph{c) Hiob hat sich durch gottlose Gesinnung und durch
frevelhafte Reden gegen Gott schwer
versündigt}\label{c-hiob-hat-sich-durch-gottlose-gesinnung-und-durch-frevelhafte-reden-gegen-gott-schwer-versuxfcndigt}}

12»Ist Gott nicht so hoch wie der Himmel? Und schaue den Gipfel der
Sterne an, wie hoch sie ragen! 13Und da sagst du: ›Was weiß denn Gott?
Kann er durch Wolkendunkel hindurch Gericht halten? 14Dichte Wolken sind
ihm eine Hülle, so daß er nichts sehen kann, und nur die Räume des
Himmelsgewölbes durchwandelt er.‹ 15Willst du die Bahn der Vorwelt
innehalten, auf der die Männer des Frevels einst gewandelt sind? 16Sie,
die vor der Zeit weggerafft wurden -- der feste Boden unter ihnen
zerfloß zu einem Strom --; 17die zu Gott sagten: ›Bleibe fern von uns!‹
und ›was der Allmächtige ihnen antun könne?‹ 18Und doch hatte er ihre
Häuser mit Segen gefüllt. Aber die Denkweise der Frevler bleibe fern von
mir! 19Die Gerechten sehen es und freuen sich, und der Schuldlose ruft
ihnen spottend zu: 20›Fürwahr, unsere Widersacher sind vernichtet, und
ihre Hinterlassenschaft\textless sup title=``oder: den letzten Rest von
ihnen''\textgreater✲ hat das Feuer verzehrt!‹«

\hypertarget{d-sicherlich-wird-hiob-im-falle-der-bekehrung-neues-heil-von-gott-erlangen}{%
\paragraph{d) Sicherlich wird Hiob im Falle der Bekehrung neues Heil von
Gott
erlangen}\label{d-sicherlich-wird-hiob-im-falle-der-bekehrung-neues-heil-von-gott-erlangen}}

21»Befreunde dich doch mit Gott und halte Frieden mit ihm! Dadurch wird
dein Geschick sich heilsam gestalten. 22Nimm doch Belehrung aus seinem
Munde an und laß seine Worte in deinem Herzen wohnen\textless sup
title=``oder: dir zu Herzen gehen''\textgreater✲! 23Wenn du dich zum
Allmächtigen bekehrst\textless sup title=``=~wieder
hinwendest''\textgreater✲, so wirst du wieder aufgebaut\textless sup
title=``=~in Wohlstand versetzt''\textgreater✲ werden; wenn du die Sünde
aus deinen Zelten entfernst~-- 24ja, wirf das Golderz von dir in den
Staub und Ophirs Gold unter die Kiesel der Bäche, 25daß der Allmächtige
dein Golderz ist\textless sup title=``oder: darstellt''\textgreater✲ und
Silber dir sein Gesetz --: 26ja, dann wirst du dich auf den Allmächtigen
getrost verlassen und zu Gott dein Angesicht vertrauensvoll erheben.
27Flehst du zu ihm, so wird er dich erhören, und deine Gelübde wirst du
bezahlen können; 28nimmst du dir etwas vor, so wird es dir gelingen, und
Licht wird über deinen Wegen strahlen. 29Wenn sie abwärts führen, so
rufst du: ›Empor!‹, und dem Niedergeschlagenen hilft er auf. 30Selbst
den Nichtschuldlosen wird er entkommen lassen, und zwar wird er durch
die Reinheit deiner Hände entkommen.«

\hypertarget{hiobs-antwort-siebte-gegenrede}{%
\subsubsection{2. Hiobs Antwort (siebte
Gegenrede)}\label{hiobs-antwort-siebte-gegenrede}}

\hypertarget{a-hiob-fuxfchlt-sich-weniger-durch-sein-schuldloses-leiden-als-durch-das-unbegreifliche-ihm-keine-gelegenheit-zur-rechtfertigung-bietende-verhalten-gottes-beunruhigt}{%
\paragraph{a) Hiob fühlt sich weniger durch sein schuldloses Leiden als
durch das unbegreifliche, ihm keine Gelegenheit zur Rechtfertigung
bietende Verhalten Gottes
beunruhigt}\label{a-hiob-fuxfchlt-sich-weniger-durch-sein-schuldloses-leiden-als-durch-das-unbegreifliche-ihm-keine-gelegenheit-zur-rechtfertigung-bietende-verhalten-gottes-beunruhigt}}

\hypertarget{section-22}{%
\section{23}\label{section-22}}

1Da antwortete Hiob folgendermaßen:

2»Auch jetzt noch gilt meine Klage euch als Trotz: schwer lastet seine
Hand auf meinem Seufzen. 3O daß ich ihn zu finden wüßte, daß ich
gelangen könnte bis zu seiner Wohnstätte\textless sup title=``oder: vor
seinen Richterstuhl''\textgreater✲! 4Ich wollte meine Sache vor ihm
darlegen und meinen Mund mit Beweisgründen füllen; 5ich erführe dann,
was er mir entgegnete, und würde vernehmen, was er mir zu sagen hat.
6Würde er dann wohl mit der ganzen Fülle seiner Macht mit mir streiten?
Nein, nur seine Aufmerksamkeit würde er mir zuwenden. 7Da würde sich
dann ein Rechtschaffener vor ihm verantworten, und für immer würde ich
von meinem Richter freikommen. 8Doch ach! Gehe ich nach Osten, so ist er
nicht da, und gehe ich nach Westen, so gewahre ich ihn nicht; 9wirkt er
im Norden, so erblicke ich ihn nicht, biegt er nach Süden ab, so sehe
ich ihn nicht. 10Er kennt ja doch den von mir eingehaltenen Weg✲, und
prüfte er mich -- wie Gold aus der Schmelze würde ich hervorgehen!
11Denn an seine Spur hat mein Fuß sich angeschlossen; den von ihm
gewiesenen Weg habe ich eingehalten, ohne davon abzuweichen; 12von dem
Gebot seiner Lippen bin ich nicht abgegangen: in meinem Busen habe ich
die Weisungen seines Mundes geborgen. 13Doch er bleibt sich immer gleich
-- wer kann ihm wehren? und was sein Sinn einmal will, das führt er auch
aus. 14So wird er denn auch vollführen, was er mir bestimmt hat, und
dergleichen hat er noch vieles im Sinn. 15Darum bebe ich vor seinem
Anblick: überdenke ich's, so graut mir vor ihm! 16Ja, Gott hat mein Herz
verzagt gemacht und der Allmächtige mich mit Angst erfüllt; 17denn nicht
wegen Finsternis\textless sup title=``=~äußerer Trübsal''\textgreater✲
fühle ich mich vernichtet und nicht wegen meiner Person, die er mit
Dunkel✲ umhüllt hat.«

\hypertarget{b-hiob-legt-das-unbegreifliche-walten-gottes-im-leidensgeschick-der-unschuldigen-und-im-gluxfcck-der-gottlosen-an-beispielen-dar}{%
\paragraph{b) Hiob legt das unbegreifliche Walten Gottes im
Leidensgeschick der Unschuldigen und im Glück der Gottlosen an
Beispielen
dar}\label{b-hiob-legt-das-unbegreifliche-walten-gottes-im-leidensgeschick-der-unschuldigen-und-im-gluxfcck-der-gottlosen-an-beispielen-dar}}

\hypertarget{section-23}{%
\section{24}\label{section-23}}

1»Warum sind vom Allmächtigen nicht Zeiten für Strafgerichte vorgesehen
worden, und warum bekommen seine Getreuen nicht seine Gerichtstage zu
sehen? 2Man verrückt die Grenzsteine, raubt Herden samt den
Hirten\textless sup title=``oder: und weidet sie als
eigene''\textgreater✲; 3den Esel der Verwaisten treibt man weg, nimmt
die Kuh der Witwe als Pfand; 4die Armen drängt man vom Wege ab; allesamt
müssen die Elenden des Landes sich verkriechen. 5Seht nur! Wie Wildesel
in der Wüste ziehen sie früh zu ihrem Tagewerk aus, nach Beute
ausspähend; die Steppe liefert ihnen Brot✲ für die Kinder; 6auf dem
Felde des Gottlosen müssen sie den Sauerampfer abernten und Nachlese in
seinem Weinberge halten; 7nackt bringen sie die Nacht zu, ohne Gewand,
und haben keine Decke in der Kälte. 8Von den Regengüssen der Berge
triefen sie und schmiegen sich obdachlos an die Felsen. 9Man reißt die
Waise von der Mutterbrust weg, und was der Elende an hat, nimmt man zum
Pfande. 10Nackt gehen sie einher, ohne Kleidung, und hungernd schleppen
sie Garben (im Dienst der Reichen); 11innerhalb der Mauern der Gottlosen
pressen sie Öl, treten die Keltern und leiden Durst dabei. 12Aus den
Städten heraus lassen Sterbende ihr Ächzen hören, und die Seele von
Erschlagenen schreit um Rache; aber Gott rechnet es nicht als Ungebühr
an!

13Andere (Gottlose) gehören zu den Feinden des Tageslichts: sie wollen
von Gottes Wegen nichts wissen und bleiben nicht auf seinen Pfaden.
14Ehe es hell wird, steht der Mörder auf, tötet den Elenden und Armen;
und in der Nacht treibt der Dieb sein Wesen. 15Das Auge des Ehebrechers
aber lauert auf die Abenddämmerung, indem er denkt: ›Kein Auge soll mich
erblicken!‹, und er legt sich eine Hülle\textless sup title=``oder:
Maske''\textgreater✲ vors Gesicht. 16In der Finsternis bricht man in die
Häuser ein, bei Tage halten sie sich eingeschlossen: sie wollen vom
Licht nichts wissen. 17Denn als Morgenlicht gilt ihnen allesamt tiefe
Nacht, weil sie mit den Schrecknissen der tiefen Nacht wohlvertraut
sind.

18Im Fluge fährt er\textless sup title=``d.h. der Frevler''\textgreater✲
über die Wasserfläche dahin; mit dem Fluch wird ihr Erbteil\textless sup
title=``oder: Grundbesitz''\textgreater✲ im Lande belegt; er schlägt
nicht mehr den Weg zu den Weinbergen ein. 19Wie Dürre und Sonnenglut die
Schneewasser wegraffen, ebenso das Totenreich die, welche gesündigt
haben. 20Selbst der Mutterschoß\textless sup title=``=~die
Mutter''\textgreater✲ vergißt ihn, das Gewürm labt sich an ihm; nicht
mehr wird seiner gedacht, und wie ein Baum wird der Frevler abgehauen,
21er, der die einsam dastehende, kinderlose Frau ausgeplündert und
keiner Witwe Gutes getan hat.

22Ebenso erhält Gott Gewalttätige lange Zeit durch seine Kraft: mancher
steht wieder auf, der schon am Leben verzweifelte. 23Er verleiht ihm
Sicherheit, so daß er gestützt dasteht, und seine Augen wachen über
ihren Wegen. 24Wenn sie hoch gestiegen sind -- ein Augenblick nur, so
sind sie nicht mehr da; sie sinken hin, werden hinweggerafft wie alle
anderen auch; wie eine Ährenspitze werden sie abgeschnitten.

25Ist's etwa nicht so? Wer will mich Lügen strafen und meine Rede als
nichtig erweisen?«

\hypertarget{dritte-rede-bildads}{%
\subsubsection{3. Dritte Rede Bildads}\label{dritte-rede-bildads}}

\hypertarget{hinweis-auf-das-unwiderstehliche-walten-gottes-in-der-huxf6he-und-auf-die-suxfcndige-art-und-unvollkommenheit-des-menschen}{%
\paragraph{Hinweis auf das unwiderstehliche Walten Gottes in der Höhe
und auf die sündige Art und Unvollkommenheit des
Menschen}\label{hinweis-auf-das-unwiderstehliche-walten-gottes-in-der-huxf6he-und-auf-die-suxfcndige-art-und-unvollkommenheit-des-menschen}}

\hypertarget{section-24}{%
\section{25}\label{section-24}}

1Da nahm Bildad von Suah das Wort und sagte:

2»Herrschergewalt✲ und Schrecken sind bei ihm, der da Frieden schafft in
seinen Höhen. 3Sind seine Heerscharen zu zählen? Und wo ist einer, über
den sein Licht\textless sup title=``d.h. sein allsehendes
Auge''\textgreater✲ sich nicht erhöbe? 4Wie könnte da ein Mensch recht
behalten\textless sup title=``oder: gerecht sein''\textgreater✲ Gott
gegenüber und wie ein vom Weibe Geborener neben ihm rein erscheinen?
5Bedenke nur: sogar der Mond ist nicht hell, und die Sterne sind nicht
rein in seinen Augen~-- 6wieviel weniger der Sterbliche, die Made, und
der Menschensohn, der Wurm!«

\hypertarget{hiobs-antwort-achte-gegenrede}{%
\subsubsection{4. Hiobs Antwort (achte
Gegenrede)}\label{hiobs-antwort-achte-gegenrede}}

\hypertarget{a-bittere-abfertigung-der-weder-trost-noch-weisen-rat-enthaltenden-rede-bildads}{%
\paragraph{a) Bittere Abfertigung der weder Trost noch weisen Rat
enthaltenden Rede
Bildads}\label{a-bittere-abfertigung-der-weder-trost-noch-weisen-rat-enthaltenden-rede-bildads}}

\hypertarget{section-25}{%
\section{26}\label{section-25}}

1Da antwortete Hiob folgendermaßen:

2»Wie hast du doch dem Schwachen beigestanden und den kraftlosen Arm
gestützt! 3Wie gut hast du doch den Unweisen beraten und tiefes Wissen
in Fülle kundgetan! 4Wem hast du einen Lehrvortrag gehalten, und wessen
Odem\textless sup title=``oder: Geist''\textgreater✲ ist dir
entströmt\textless sup title=``=~hat aus dir gesprochen''\textgreater✲?«

\hypertarget{b-hiob-erkennt-die-unermeuxdfliche-erhabenheit-gottes-in-einer-gluxe4nzenden-schilderung-an}{%
\paragraph{b) Hiob erkennt die unermeßliche Erhabenheit Gottes in einer
glänzenden Schilderung
an}\label{b-hiob-erkennt-die-unermeuxdfliche-erhabenheit-gottes-in-einer-gluxe4nzenden-schilderung-an}}

5»Die Schatten erzittern (vor Gott) tief unter den Wassern und deren
Bewohnern; 6nackt✲ liegt das Totenreich vor ihm da und unverhüllt der
Abgrund\textless sup title=``=~die Unterwelt''\textgreater✲. 7Er spannt
den Norden (der Erde) über der Leere aus, hängt die Erde an dem Nichts
auf. 8Er bindet die Wasser in seine Wolken ein, ohne daß das Gewölk
unter ihrer Last zerplatzt. 9Er verhüllt den Anblick seines Thrones,
indem er sein Gewölk über ihn ausbreitet. 10Eine Grenzlinie hat er über
den weiten Wassern abgezirkelt bis zur äußersten Grenze, wo das Licht
mit der Finsternis zusammentrifft. 11Die Säulen des Himmels geraten ins
Wanken und beben infolge seines Scheltens. 12Durch seine Kraft beruhigt
er das Meer, und durch seine Klugheit hat er Rahab\textless sup
title=``=~Ungetüme; vgl. 9,13''\textgreater✲ zerschmettert. 13Durch
seinen Hauch gewinnt der Himmel Heiterkeit; durchbohrt hat seine Hand
den flüchtigen Drachen\textless sup title=``vgl. 9,13''\textgreater✲.
14Siehe, das sind nur die Säume seines Waltens, und welch ein leises
Flüstern nur ist es, das wir von ihm vernehmen! Doch die Donnersprache
seiner Machterweise -- wer versteht diese?«

\hypertarget{v.-hiobs-schluuxdfrede-an-seine-freunde-kap.-27-28}{%
\subsection{V. Hiobs Schlußrede an seine Freunde (Kap.
27-28)}\label{v.-hiobs-schluuxdfrede-an-seine-freunde-kap.-27-28}}

\hypertarget{a-hiob-erkluxe4rt-an-der-uxfcberzeugung-von-seiner-eidlich-beteuerten-unschuld-festhalten-zu-muxfcssen-um-nicht-zum-luxfcgner-zu-werden}{%
\paragraph{a) Hiob erklärt, an der Überzeugung von seiner eidlich
beteuerten Unschuld festhalten zu müssen, um nicht zum Lügner zu
werden}\label{a-hiob-erkluxe4rt-an-der-uxfcberzeugung-von-seiner-eidlich-beteuerten-unschuld-festhalten-zu-muxfcssen-um-nicht-zum-luxfcgner-zu-werden}}

\hypertarget{section-26}{%
\section{27}\label{section-26}}

1Hierauf fuhr Hiob nochmals in seiner Rede so fort:

2»So wahr Gott lebt, der mir mein Recht entzogen, und der Allmächtige,
der mich in Verzweiflung gestürzt hat: 3Solange irgend noch mein
Lebensodem in mir ist und Gottes Hauch in meiner Nase~-- 4nie sollen
meine Lippen eine Unwahrheit reden und meine Zunge eine Täuschung
aussprechen! 5Fern sei es also von mir, euch recht zu geben, nein, bis
zum letzten Atemzuge verleugne ich meine Unschuld nicht! 6An meiner
Gerechtigkeit halte ich fest und lasse sie nicht fahren: mein Gewissen
straft mich wegen keines einzigen meiner Lebenstage!«

\hypertarget{b-das-schicksal-des-frevlers-d.h.-luxfcgners-ist-hiob-wohlbekannt}{%
\paragraph{b) Das Schicksal des Frevlers (d.h. Lügners) ist Hiob
wohlbekannt}\label{b-das-schicksal-des-frevlers-d.h.-luxfcgners-ist-hiob-wohlbekannt}}

7»Wie dem Frevler möge es meinem Feinde ergehen und meinem Widersacher
wie dem Bösewicht! 8Denn welche Hoffnung hat der Ruchlose noch, wenn
Gott seinen Lebensfaden abschneidet, wenn er ihm seine Seele abfordert?
9Wird Gott wohl sein Schreien hören, wenn Drangsal über ihn
hereinbricht? 10Oder darf er auf den Allmächtigen sich getrost
verlassen, Gott anrufen zu jeder Zeit?«

\hypertarget{c-schilderung-des-unfehlbaren-untergangs-der-frevler-trotz-aller-willkuxfcr-gottes}{%
\paragraph{c) Schilderung des unfehlbaren Untergangs der Frevler (trotz
aller Willkür
Gottes)}\label{c-schilderung-des-unfehlbaren-untergangs-der-frevler-trotz-aller-willkuxfcr-gottes}}

11»Ich will euch über Gottes Tun belehren und, wie der Allmächtige es
hält, euch nicht verhehlen. 12Seht doch, ihr alle habt euch selbst davon
überzeugt: warum seid ihr gleichwohl in so eitlem Wahn befangen? 13Dies
ist das Teil\textless sup title=``=~Schicksal, Los''\textgreater✲ des
frevelhaften Menschen bei Gott und das Erbe der Gewalttätigen, das sie
vom Allmächtigen empfangen: 14Wenn seine Kinder\textless sup
title=``oder: Söhne''\textgreater✲ groß werden, so ist's für das
Schwert, und seine Sprößlinge haben nicht satt zu essen. 15Wer ihm dann
von den Seinen noch übrigbleibt, wird durch die Pest ins Grab gebracht,
und ihre Witwen stellen nicht einmal eine Totenklage an. 16Wenn er Geld
aufhäuft wie Staub und Gewänder ansammelt wie Gassenschmutz: 17er
sammelt sie wohl, aber ein Gerechter bekleidet sich mit ihnen, und das
Geld wird ein Schuldloser in Besitz nehmen. 18Er hat sein Haus gebaut
wie ein Spinngewebe und wie eine Hütte, die ein Feldhüter sich
aufschlägt: 19als reicher Mann legt er sich schlafen, ohne daß
es\textless sup title=``d.h. das Geld''\textgreater✲ schon weggerafft
wäre -- schlägt er die Augen auf, so ist nichts mehr da; 20Schrecknisse
überfallen ihn bei Tage, bei Nacht rafft der Sturmwind ihn hinweg; 21der
Ostwind hebt ihn empor, so daß er dahinfährt, und stürmt ihn hinweg von
seiner Stätte. 22Gott schleudert seine Geschosse erbarmungslos auf ihn;
seiner Hand möchte er um jeden Preis entfliehen. 23Man klatscht über ihn
in die Hände, und Zischen folgt ihm nach von seiner Wohnstätte her.«

\hypertarget{d-schilderung-der-wahren-weisheit-gottes-und-des-menschen}{%
\paragraph{d) Schilderung der (wahren) Weisheit (Gottes und des
Menschen)}\label{d-schilderung-der-wahren-weisheit-gottes-und-des-menschen}}

\hypertarget{aa-alle-schuxe4tze-auch-die-in-den-tiefen-der-erde-verborgenen-weiuxdf-der-mensch-zu-finden-und-sich-zu-eigen-zu-machen}{%
\subparagraph{aa) Alle Schätze, auch die in den Tiefen der Erde
verborgenen, weiß der Mensch zu finden und sich zu eigen zu
machen}\label{aa-alle-schuxe4tze-auch-die-in-den-tiefen-der-erde-verborgenen-weiuxdf-der-mensch-zu-finden-und-sich-zu-eigen-zu-machen}}

\hypertarget{section-27}{%
\section{28}\label{section-27}}

1»Denn wohl gibt es für das Silber einen Fundort\textless sup
title=``oder: eine Herkunftsstelle''\textgreater✲ und eine Stätte für
das Golderz, wo man es auswäscht\textless sup title=``oder:
läutert''\textgreater✲. 2Eisen wird aus der Erde herausgeholt, und
Gestein schmelzt man zu Kupfer um. 3Der Finsternis hat (der Mensch) ein
Ziel gesetzt, und bis in die äußersten Tiefen durchforscht das in Nacht
und Grauen verborgene Gestein. 4Man bricht einen Stollen fern von den im
Licht Wohnenden; vergessen und fern vom Fuß der über ihnen
Hinschreitenden hangen sie da (an Seilen), fern von den Menschen
schweben sie. 5Die Erde, aus welcher Brotkorn hervorwächst, wird in der
Tiefe umgewühlt wie mit Feuer. 6Man findet Saphir im Gestein und Staub,
darin Gold ist. 7Den Pfad dorthin kennt der Adler nicht, und das Auge
des Falken hat ihn nicht erspäht; 8nicht betreten ihn die stolzen
Raubtiere, noch schreitet der Leu auf ihm einher. 9An das harte Gestein
legt (der Mensch) seine Hand, wühlt die Berge um von der Wurzel aus;
10in die Felsen bricht er Schächte, und allerlei Kostbares erblickt sein
Auge. 11Die Wasseradern verbaut er, daß sie nicht durchsickern, und
zieht so die verborgenen Schätze ans Licht hervor.«

\hypertarget{bb-aber-die-weisheit-das-kostbarste-gut-ist-in-der-ganzen-schuxf6pfung-nirgends-zu-finden}{%
\subparagraph{bb) Aber die Weisheit, das kostbarste Gut, ist in der
ganzen Schöpfung nirgends zu
finden}\label{bb-aber-die-weisheit-das-kostbarste-gut-ist-in-der-ganzen-schuxf6pfung-nirgends-zu-finden}}

12»Die Weisheit aber -- wo findet man diese? und wo ist die Fundstätte
der Erkenntnis? 13Kein Mensch kennt den Weg zu ihr, und im Lande der
Lebendigen ist sie nicht zu finden. 14Die Flut der Tiefe\textless sup
title=``d.h. das tiefe Weltmeer''\textgreater✲ sagt: ›In mir ist sie
nicht‹; und das Meer erklärt: ›Bei mir weilt sie nicht‹. 15Für
geläutertes Gold ist sie nicht feil, und Silber kann nicht als Kaufpreis
für sie dargewogen werden; 16sie läßt sich nicht aufwägen mit Feingold
von Ophir, mit kostbarem Onyx und Saphir. 17Gold und Prachtglas kann man
ihr nicht gleichstellen, noch sie eintauschen gegen Kunstwerke von
gediegenem Gold; 18Korallen und Kristall kommen (neben ihr) nicht in
Betracht, und der Besitz der Weisheit ist mehr wert als Perlen.
19Äthiopiens Topas reicht nicht an sie heran, mit reinstem Feingold wird
sie nicht aufgewogen. 20Die Weisheit also -- woher kommt sie, und wo ist
die Fundstätte der Erkenntnis? 21Verborgen ist sie vor den Augen aller
lebenden Wesen und verhüllt sogar vor den Vögeln des Himmels. 22Die
Unterwelt und das Totenreich sagen von ihr: ›Nur ein Gerücht von ihr ist
uns zu Ohren gedrungen.‹«

\hypertarget{cc-gott-allein-besitzt-die-weisheit-und-hat-sie-in-der-schuxf6pfung-der-welt-betuxe4tigt-der-mensch-kann-sie-nur-als-gottesfurcht-besitzen}{%
\subparagraph{cc) Gott allein besitzt die Weisheit und hat sie in der
Schöpfung der Welt betätigt; der Mensch kann sie nur als Gottesfurcht
besitzen}\label{cc-gott-allein-besitzt-die-weisheit-und-hat-sie-in-der-schuxf6pfung-der-welt-betuxe4tigt-der-mensch-kann-sie-nur-als-gottesfurcht-besitzen}}

23»Gott hat den Weg zu ihr (allein) erschaut, und er kennt ihre
Fundstätte; 24denn er blickt bis zu den Enden der Erde und sieht, was
unter dem ganzen Himmel ist. 25Als er dem Winde seine Wucht bestimmte
und die Wasser mit dem Maß abwog, 26als er dem Regen sein Gesetz
vorschrieb und dem Wetterstrahl die Bahn anwies: 27da sah er sie und
betätigte\textless sup title=``oder: entfaltete''\textgreater✲ sie,
setzte sie ein und erforschte sie auch. 28Zu dem Menschen aber sprach
er: ›Wisse wohl: die Furcht vor dem Allherrn -- das ist Weisheit, und
das Böse meiden -- das ist Verstand!‹«

\hypertarget{vi.-hiobs-selbstgespruxe4ch-kap.-29-31}{%
\subsection{VI. Hiobs Selbstgespräch (Kap.
29-31)}\label{vi.-hiobs-selbstgespruxe4ch-kap.-29-31}}

\hypertarget{hiobs-ruxfcckblick-auf-sein-fruxfcheres-gluxfcck}{%
\subsubsection{1. Hiobs Rückblick auf sein früheres
Glück}\label{hiobs-ruxfcckblick-auf-sein-fruxfcheres-gluxfcck}}

\hypertarget{a-schilderung-des-fruxfcheren-guxf6ttlichen-segens}{%
\paragraph{a) Schilderung des früheren göttlichen
Segens}\label{a-schilderung-des-fruxfcheren-guxf6ttlichen-segens}}

\hypertarget{section-28}{%
\section{29}\label{section-28}}

1Hierauf fuhr Hiob in seiner Rede so fort:

2»O daß es mit mir noch so stände wie in den früheren Monden, wie in den
Tagen, wo Gott mich behütete,~-- 3als seine Leuchte noch über meinem
Haupte strahlte und ich in seinem Licht durch das Dunkel wandelte, 4so,
wie es mit mir in den Tagen meines Herbstes\textless sup title=``d.h.
meiner Vollreife oder: Vollkraft''\textgreater✲ stand, als Gottes
Freundschaft über meinem Zelt waltete, 5als der Allmächtige noch auf
meiner Seite stand, meine Söhne\textless sup title=``oder:
Kinder''\textgreater✲ noch rings um mich her waren, 6als meiner Füße
Tritte sich in Milch badeten und jeder Fels neben mir Bäche von Öl
fließen ließ!«

\hypertarget{b-schilderung-seines-fruxfcheren-hohen-ansehens-seiner-gerechtigkeit-und-seines-erfolgreichen-wirkens}{%
\paragraph{b) Schilderung seines früheren hohen Ansehens, seiner
Gerechtigkeit und seines erfolgreichen
Wirkens}\label{b-schilderung-seines-fruxfcheren-hohen-ansehens-seiner-gerechtigkeit-und-seines-erfolgreichen-wirkens}}

7»Wenn ich (damals) hinaufging zum Tor der Stadt und meinen Stuhl auf
dem Marktplatz aufstellte, 8da traten die jungen Männer zurück, sobald
sie mich sahen, und die Greise erhoben sich und blieben stehen; 9die
Fürsten\textless sup title=``oder: Vornehmen''\textgreater✲ hielten an
sich mit ihrem Reden und legten die Hand auf ihren Mund; 10die Stimme
der Edlen verstummte, und die Zunge blieb ihnen am Gaumen kleben. 11Denn
wessen Ohr mich hörte, der pries mich glücklich, und jedes Auge, das
mich sah, legte Zeugnis für mich ab; 12denn ich rettete den Elenden, der
um Hilfe schrie, und die Waise, die sonst keinen Helfer hatte. 13Der
Segensspruch dessen, der verloren schien, erscholl über mich, und das
Herz der Witwe machte ich jubeln. 14In Gerechtigkeit kleidete ich mich,
und sie war mein Ehrenkleid: wie ein Prachtgewand und Kopfbund schmückte
mich mein Rechttun\textless sup title=``=~meine
Ehrenhaftigkeit''\textgreater✲. 15Für den Blinden war ich das Auge und
für den Lahmen der Fuß; 16ein Vater war ich für die Armen, und der
Rechtssache des mir Unbekannten nahm ich mich gewissenhaft an; 17dem
Frevler\textless sup title=``oder: Rechtsverdreher''\textgreater✲
zerschmetterte ich das Gebiß und riß ihm den Raub aus den Zähnen. 18So
dachte ich denn: ›Im Besitz meines Nestes werde ich sterben und mein
Leben werde ich lange wie der Phönix erhalten; 19meine Wurzel wird am
Wasser ausgebreitet liegen und der Tau auf meinen Zweigen nächtigen;
20mein Ansehen wird unverändert mir verbleiben und mein Bogen sich in
meiner Hand stets verjüngen.‹ 21Mir hörten sie zu und warteten auf mich
und lauschten schweigend auf meinen Rat. 22Wenn ich gesprochen hatte,
nahm keiner nochmals das Wort, sondern meine Rede träufelte auf sie
herab. 23Sie warteten auf meine Rede wie auf den Regen und sperrten den
Mund nach mir auf wie nach Frühlingsregen. 24Ich lächelte ihnen zu, wenn
sie mutlos waren, und das heitere Antlitz vermochten sie mir nicht zu
trüben. 25Sooft ich den Weg zu ihnen einschlug, saß ich als Haupt da und
thronte wie ein König in der Kriegerschar, wie einer, der Leidtragenden
Trost spendet.«

\hypertarget{hiobs-schilderung-seines-jetzigen-elends}{%
\subsubsection{2. Hiobs Schilderung seines jetzigen
Elends}\label{hiobs-schilderung-seines-jetzigen-elends}}

\hypertarget{a-hiob-erfuxe4hrt-verachtung-beleidigungen-und-angriffe-sogar-von-seiten-der-ehrlosesten-leute}{%
\paragraph{a) Hiob erfährt Verachtung, Beleidigungen und Angriffe sogar
von seiten der ehrlosesten
Leute}\label{a-hiob-erfuxe4hrt-verachtung-beleidigungen-und-angriffe-sogar-von-seiten-der-ehrlosesten-leute}}

\hypertarget{section-29}{%
\section{30}\label{section-29}}

1»Jetzt aber lachen über mich auch solche, die jünger an Jahren sind als
ich, deren Väter ich nicht gewürdigt habe, sie neben den Wachhunden
meines Kleinviehs anzustellen. 2Wozu hätte mir auch die Kraft ihrer
Hände nützen können? Bei ihnen war ja die volle Rüstigkeit
verlorengegangen. 3Durch Mangel und Hunger erschöpft, nagen sie das
dürre Land ab, die unfruchtbare und öde Steppe; 4sie pflücken sich
Melde\textless sup title=``oder: Salzkraut''\textgreater✲ am Buschwerk
ab, und die Ginsterwurzel ist ihr Brot. 5Aus der Gemeinde\textless sup
title=``oder: menschlichen Gesellschaft''\textgreater✲ werden sie
ausgestoßen: man schreit über sie wie über Diebe. 6In schauerlichen
Klüften müssen sie wohnen, in Erdlöchern und Felshöhlen; 7zwischen
Sträuchern brüllen sie, unter Dorngestrüpp halten sie Zusammenkünfte:
8verworfenes und ehrloses Gesindel, das man aus dem Lande
hinausgepeitscht hat.

9Und jetzt bin ich ihr Spottlied geworden und diene ihrem Gerede zur
Kurzweil\textless sup title=``oder: als Zielscheibe''\textgreater✲.
10Mit Abscheu halten sie sich fern von mir und scheuen sich nicht, vor
mir auszuspeien; 11weil Gott meine Bogensehne abgespannt und mich
niedergebeugt hat, lassen sie den Zügel vor mir schießen. 12Zu meiner
Rechten erhebt sich die Brut; sie stoßen meine Füße weg und schütten
ihre Unheilsstraßen gegen mich auf. 13Meinen Pfad haben sie aufgerissen,
auf meinen Sturz arbeiten sie hin, niemand tut ihnen Einhalt. 14Wie
durch einen breiten Mauerriß\textless sup title=``oder: eine
Bresche''\textgreater✲ kommen sie heran, durch die Trümmer\textless sup
title=``oder: mit wildem Lärm''\textgreater✲ wälzen sie sich daher:
15ein Schreckensheer hat sich gegen mich gekehrt; wie vom Sturmwind wird
meine Ehre weggerafft, und wie eine Wolke ist mein Glück
vorübergezogen!«

\hypertarget{b-schilderung-der-durch-gottes-allgewalt-und-offenkundig-feindliche-gesinnung-uxfcber-hiob-hereingebrochenen-leiden}{%
\paragraph{b) Schilderung der durch Gottes Allgewalt und offenkundig
feindliche Gesinnung über Hiob hereingebrochenen
Leiden}\label{b-schilderung-der-durch-gottes-allgewalt-und-offenkundig-feindliche-gesinnung-uxfcber-hiob-hereingebrochenen-leiden}}

16»So verblutet sich denn jetzt das Herz in mir: die Tage des Elends
halten mich in ihrer Gewalt. 17Die Nacht bohrt in meinen Gebeinen und
löst sie von mir ab, und die an mir nagenden Schmerzen schlafen nicht.
18Durch Allgewalt ist mein Gewand\textless sup title=``d.h. meine
Haut''\textgreater✲ entstellt: so eng wie mein Unterkleid\textless sup
title=``oder: Panzer''\textgreater✲ umschließt es mich. 19Gott hat mich
in den Kot geworfen, und ich bin (an Ansehen) dem Staub und der Asche
gleichgestellt. 20Schreie ich zu dir, so antwortest du mir nicht; trete
ich vor dich hin, so achtest du nicht auf mich: 21du hast dich mir in
einen erbarmungslosen Feind verwandelt; mit deiner starken Hand
bekämpfst du mich. 22Du hebst mich auf (die Fittiche) des Sturmwindes
empor, läßt mich dahinfahren und im Sturmestosen vergehen. 23Ja, ich
weiß es: in den Tod willst du mich heimführen und in das
Versammlungshaus aller Lebenden!«

\hypertarget{c-hiobs-versicherung-dauxdf-er-gerechten-grund-zur-klage-habe}{%
\paragraph{c) Hiobs Versicherung, daß er gerechten Grund zur Klage
habe}\label{c-hiobs-versicherung-dauxdf-er-gerechten-grund-zur-klage-habe}}

24»Doch streckt man nicht beim Ertrinken die Hand (nach Rettung) aus,
und erhebt man beim Versinken nicht darob einen Hilferuf? 25Habe ich
denn nicht um den geweint, der harte Tage durchzumachen hatte, und ist
mein Herz nicht um den Armen bekümmert gewesen? 26Ja, auf Glück habe ich
gewartet, aber Unheil kam; und ich harrte auf Licht, aber es kam
Finsternis. 27Mein Inneres ist in Aufruhr ohne Unterlaß, Leidenstage
haben mich überfallen. 28In Trauer gehe ich einher ohne
Sonne\textless sup title=``oder: ohne Trost''\textgreater✲; ich stehe in
der versammelten Gemeinde auf und schreie; 29den (heulenden) Schakalen
bin ich ein Bruder geworden und den (klagenden) Straußen ein Genosse.
30Meine Haut löst sich, schwarz geworden, von mir ab, und mein Gebein
ist von Fieberglut ausgedörrt. 31So ist denn mein Zitherspiel zum
Trauerlied\textless sup title=``=~zur Totenklage''\textgreater✲ geworden
und meine Schalmei zu Tönen der Klage!«

\hypertarget{hiobs-beteuerung-seines-unstruxe4flichen-wandels-vor-gott-und-menschen}{%
\subsubsection{3. Hiobs Beteuerung seines unsträflichen Wandels vor Gott
und
Menschen}\label{hiobs-beteuerung-seines-unstruxe4flichen-wandels-vor-gott-und-menschen}}

\hypertarget{a-der-grouxdfe-reinigungseid-hiobs-zur-feststellung-seiner-gerechtigkeit-unverletzten-gottesfurcht}{%
\paragraph{a) Der große Reinigungseid Hiobs zur Feststellung seiner
Gerechtigkeit (=~unverletzten
Gottesfurcht)}\label{a-der-grouxdfe-reinigungseid-hiobs-zur-feststellung-seiner-gerechtigkeit-unverletzten-gottesfurcht}}

\hypertarget{section-30}{%
\section{31}\label{section-30}}

1»Mit meinen Augen habe ich einen Bund abgeschlossen, daß ich ja nicht
lüstern nach einer Jungfrau blickte. 2Denn was wäre der Lohn Gottes von
oben gewesen und die Vergeltung des Allmächtigen aus Himmelshöhen?
3Trifft nicht Verderben den Frevler und Unglück die Überltäter? 4Sieht
er\textless sup title=``d.h. Gott''\textgreater✲ nicht meine Wege, und
zählt er nicht alle meine Schritte?

5Wenn ich mit Falschheit umgegangen bin und mein Fuß jemals der
Täuschung zugeeilt ist: 6Gott wäge mich auf gerechter✲ Waage, so wird er
meine Unschuld erkennen! 7Wenn mein Schritt jemals vom rechten Wege
abgewichen und mein Herz meinen Augen Folge geleistet hat und ein
Flecken an meinen Händen kleben geblieben ist, 8so will ich säen und ein
anderer möge es verzehren, und alles, was mir sproßt, möge ausgerissen
werden!

9Wenn mein Herz sich um eines Weibes willen hat betören lassen und ich
an der Tür meines Nächsten auf der Lauer gestanden habe, 10so soll mein
Weib für einen andern die Mühle drehen und andere mögen sich über sie
hinstrecken! 11Denn das wäre eine Schandtat gewesen und das ein Vergehen
für den Strafrichter; 12ja, ein Feuer wäre das gewesen, das bis zum
Abgrund\textless sup title=``=~zur Unterwelt; vgl. 26,6''\textgreater✲
gefressen und meinen gesamten Besitz bis auf die Wurzel hätte vernichten
müssen.

13Wenn ich das Recht meines Knechtes und meiner Magd mißachtet hätte,
sooft sie im Streit mit mir lagen: 14was hätte ich da tun sollen, wenn
Gott aufgestanden wäre? Und was hätte ich ihm bei seiner Untersuchung
erwidern können? 15Hat nicht mein Schöpfer auch ihn im Mutterleibe
geschaffen und ein und derselbe uns im Mutterschoße gebildet?

16Wenn ich den Geringen ihr Begehren versagt und die Augen der Witwe
habe schmachten lassen 17und meinen Bissen für mich allein verzehrt
habe, ohne daß der Verwaiste sein Teil davon genossen hat~-- 18nein, von
meiner Jugend an ist er mir ja wie einem Vater aufgewachsen, und von
meiner Mutter Leibe an bin ich ein Beschützer für jenen gewesen --;
19wenn ich jemand habe verkommen sehen aus Mangel an Kleidung und daß
ein Armer keine Schlafdecke hatte, 20und dann seine Hüften mich nicht
gesegnet haben und er sich nicht durch meiner Lämmer Wolle erwärmt hat;
21wenn ich meine Faust jemals gegen eine Waise geschwungen habe, weil
ich im Tor\textless sup title=``=~vor Gericht''\textgreater✲ auf
Beistand rechnen konnte: 22so möge meine Schulter von ihrem Nacken
fallen und mein Arm aus seiner Röhre ausgebrochen werden! 23Denn als ein
Schrecken wäre auf mich das Strafgericht Gottes eingedrungen, und vor
seiner Erhabenheit hätte ich nicht zu bestehen vermocht.

24Wenn ich je auf Gold mein Vertrauen gesetzt und zum Feingold gesagt
habe: ›Du bist meine Zuversicht!‹; 25wenn ich mich darüber gefreut habe,
daß mein Vermögen groß war und daß meine Hand Ansehnliches erworben
hatte; 26wenn ich die Sonne angeschaut habe, wie hell sie strahlt, und
den Mond, wie er in Pracht dahinwandelt, 27und mein Herz sich insgeheim
hat betören lassen, daß ich ihnen eine Kußhand zuwarf: 28auch das wäre
eine Verschuldung für den Strafrichter gewesen, denn damit hätte ich
Gott in der Höhe die Treue gebrochen.~-- 29Wenn ich mich je über das
Unglück meines Feindes gefreut und darüber gejubelt habe, daß ein
Mißgeschick ihm zugestoßen war~-- 30nein, nie habe ich meiner Zunge zu
sündigen gestattet, daß sie durch einen Fluch sein Leben gefordert
hätte~-- 31wenn meine Zeltgenossen nicht gesagt haben: ›Wo ist einer,
der vom Fleisch seines Schlachtviehs nicht satt geworden wäre?‹ --;
32nein, der Fremdling durfte nicht im Freien übernachten, und meine Tür
hielt ich dem Wanderer offen --; 33wenn ich meine Übertretungen, wie
Menschen tun, verheimlicht habe, indem ich mein Vergehen in meinem Busen
verbarg, 34weil ich mich vor der großen Menge scheute und die Mißachtung
der Geschlechter mich schreckte, so daß ich mich still verhielt, nicht
vor die Tür hinaustrat;

\hypertarget{b-hiobs-wunsch-und-bereitschaft-mit-gott-in-einen-rechtsstreit-einzutreten}{%
\paragraph{b) Hiobs Wunsch und Bereitschaft, mit Gott in einen
Rechtsstreit
einzutreten}\label{b-hiobs-wunsch-und-bereitschaft-mit-gott-in-einen-rechtsstreit-einzutreten}}

35»O hätte ich doch einen, der mich anhören wollte! Siehe, hier ist
meine Unterschrift! Der Allmächtige antworte mir! Und hätte ich doch die
von meinem Gegner ausgefertigte Klageschrift! 36Wahrlich, an meiner
Schulter wollte ich sie zur Schau tragen, als Ehrenkranz sie mir um die
Schläfe winden! 37Denn über die Zahl meiner Schritte wollte ich ihm Rede
stehen, wie zu einem Fürsten müßte er herannahen!«

{[}Die Reden Hiobs sind zu Ende.{]}

38wenn mein Acker je über mich geschrien und seine Furchen allesamt
geweint haben; 39wenn ich seinen Ertrag ohne Zahlung verzehrt und seinen
Besitzer ums Leben gebracht habe\textless sup title=``oder: habe seufzen
lassen''\textgreater✲: 40so sollen mir Disteln statt des Weizens
aufgehen und Unkraut statt der Gerste!«

\hypertarget{vii.-die-reden-elihus-kap.-32-37}{%
\subsection{VII. Die Reden Elihus (Kap.
32-37)}\label{vii.-die-reden-elihus-kap.-32-37}}

\hypertarget{eingang}{%
\subsubsection{1. Eingang}\label{eingang}}

\hypertarget{a-angaben-uxfcber-elihu-und-sein-bisheriges-verhalten}{%
\paragraph{a) Angaben über Elihu und sein bisheriges
Verhalten}\label{a-angaben-uxfcber-elihu-und-sein-bisheriges-verhalten}}

\hypertarget{section-31}{%
\section{32}\label{section-31}}

1Als nun jene drei Männer es aufgegeben hatten, dem Hiob (darauf) zu
antworten, daß er sich selbst für gerecht hielt, 2da entbrannte der Zorn
des Busiters Elihu, des Sohnes Barachels, aus dem Geschlechte
Ram\textless sup title=``vgl. Ruth 4,19''\textgreater✲. Gegen Hiob war
er in Zorn geraten, weil dieser Gott gegenüber im Recht zu sein
behauptete; 3und gegen dessen drei Freunde war er deshalb in Zorn
geraten, weil sie nicht die (rechte) Antwort gefunden hatten, um Hiob
als schuldig zu erweisen. 4Elihu hatte aber mit einer Entgegnung an Hiob
an sich gehalten, weil jene älter an Jahren waren als er. 5Als Elihu
aber sah, daß im Munde der drei Männer keine Widerlegung sich fand,
geriet er in Zorn. 6So nahm denn der Busiter Elihu, der Sohn Barachels,
das Wort und sagte:

\hypertarget{b-die-selbsteinfuxfchrung-elihus}{%
\paragraph{b) Die Selbsteinführung
Elihus}\label{b-die-selbsteinfuxfchrung-elihus}}

\hypertarget{aa-elihu-begruxfcndet-sein-bisheriges-schweigen}{%
\subparagraph{aa) Elihu begründet sein bisheriges
Schweigen}\label{aa-elihu-begruxfcndet-sein-bisheriges-schweigen}}

»Noch jung bin ich an Tagen, und ihr seid Greise; darum habe ich mich
gescheut und an mich gehalten, euch mein Wissen kundzutun. 7Ich dachte:
›Das Alter mag reden und die Menge der Jahre Weisheit an den Tag legen!‹
8Jedoch der Geist ist es in den Menschen und der Hauch\textless sup
title=``oder: Odem''\textgreater✲ des Allmächtigen, der ihnen Einsicht
verleiht. 9Nicht die Bejahrten sind die weisesten, und nicht die Greise
(an sich) verstehen sich auf das, was Recht ist. 10Darum sage ich: ›Hört
mir zu! Laßt auch mich mein Wissen euch kundtun.‹ 11Seht, ich habe auf
eure Reden geharrt, habe nach einsichtigen Darlegungen von euch
hingehorcht, bis ihr die rechten Worte ausfindig machen würdet, 12ja,
ich habe aufmerksam auf euch achtgegeben; doch seht: keiner hat Hiob
widerlegt, keiner von euch auf seine Reden die (rechte) Antwort gegeben.
13Wendet nur nicht ein: ›Wir sind (bei ihm) auf Weisheit gestoßen: nur
Gott kann ihn aus dem Felde schlagen, nicht ein Mensch!‹ 14Gegen mich
hat er ja noch keine Beweisgründe ins Treffen geführt, und nicht mit
euren Reden werde ich ihm entgegentreten.«

\hypertarget{bb-elihu-erkluxe4rt-dauxdf-sein-geist-sich-zur-unparteiischen-kundgebung-seiner-einsicht-getrieben-fuxfchle}{%
\subparagraph{bb) Elihu erklärt, daß sein Geist sich zur unparteiischen
Kundgebung seiner Einsicht getrieben
fühle}\label{bb-elihu-erkluxe4rt-dauxdf-sein-geist-sich-zur-unparteiischen-kundgebung-seiner-einsicht-getrieben-fuxfchle}}

15»Bestürzt stehen sie da, finden keine Antwort mehr; die Worte sind
ihnen ausgegangen! 16Und da sollte ich warten, weil sie nicht mehr
reden, weil sie dastehen, ohne zu antworten? 17Nein, auch ich will mein
Teil erwidern, auch ich will mein Wissen kundtun! 18Denn voll bin ich
von Worten; der Geist drängt und beengt mich in meinem Inneren, zu
reden. 19Seht, meiner Brust geht es wie dem Wein, dem man nicht Luft
schafft: sie droht zu bersten wie neugefüllte Schläuche. 20Reden will
ich, um mir Luft zu schaffen, will meine Lippen auftun und entgegnen!
21Ich will dabei für niemand Partei nehmen und keinem Menschen zu
Gefallen reden; 22denn ich verstehe mich nicht darauf, zu Gefallen zu
reden: gar bald würde mein Schöpfer mich sonst hinwegraffen.«

\hypertarget{cc-elihus-freundliche-anrede-und-aufforderung-an-hiob-zur-stellungnahme}{%
\subparagraph{cc) Elihus freundliche Anrede und Aufforderung an Hiob zur
Stellungnahme}\label{cc-elihus-freundliche-anrede-und-aufforderung-an-hiob-zur-stellungnahme}}

\hypertarget{section-32}{%
\section{33}\label{section-32}}

1»Nun aber höre, Hiob, meine Reden und leihe dein Ohr allen meinen
Worten! 2Wisse wohl: wenn ich meinen Mund jetzt auftue und meine Zunge
sich vernehmlich hören läßt, 3so sind meine Worte aufrichtig wie mein
Herz, und was meine Lippen wissen, sprechen sie unverfälscht aus. 4Der
Geist Gottes, der mich geschaffen hat, und der Hauch\textless sup
title=``oder: Odem''\textgreater✲ des Allmächtigen belebt mich. 5Wenn
du's vermagst, so widerlege mich: rüste dich mit Beweisgründen gegen
mich, tritt an zum Kampf! 6Siehe, ich stehe zu Gott ebenso wie du: aus
Ton✲ bin auch ich gebildet. 7Nein, Angst vor mir braucht dich nicht
einzuschüchtern, und meine Wucht soll dich nicht niederdrücken!«

\hypertarget{erste-rede-elihus}{%
\subsubsection{2. Erste Rede Elihus}\label{erste-rede-elihus}}

\hypertarget{a-kurze-darlegung-und-zuruxfcckweisung-der-klagen-hiobs-gegen-gott}{%
\paragraph{a) Kurze Darlegung und Zurückweisung der Klagen Hiobs gegen
Gott}\label{a-kurze-darlegung-und-zuruxfcckweisung-der-klagen-hiobs-gegen-gott}}

8»Nun aber hast du vor meinen Ohren ausgesprochen, und deutlich habe ich
deine Worte gehört: 9›Unschuldig bin ich, ohne Missetat, rein bin ich,
und kein Vergehen haftet mir an! 10Fürwahr, er (Gott) erfindet
Feindseligkeiten gegen mich\textless sup title=``oder: findet
Widerwärtigkeiten an mir''\textgreater✲, sieht in mir einen Feind; 11er
legt meine Füße in den Block, überwacht alle meine Pfade.‹ 12Sieh, darin
hast du unrecht, entgegne ich dir; denn Gott ist größer als ein Mensch.«

\hypertarget{b-zuruxfcckweisung-der-klage-hiobs-dauxdf-gott-ihm-nicht-antworte}{%
\paragraph{b) Zurückweisung der Klage Hiobs, daß Gott ihm nicht
antworte}\label{b-zuruxfcckweisung-der-klage-hiobs-dauxdf-gott-ihm-nicht-antworte}}

\hypertarget{aa-gott-belehrt-die-menschen-uxfcber-seine-absichten-und-uxfcber-ihre-suxfcnde-bald-durch-truxe4ume-bald-durch-leiden-besonders-durch-krankheit}{%
\subparagraph{aa) Gott belehrt die Menschen über seine Absichten und
über ihre Sünde bald durch Träume, bald durch Leiden, besonders durch
Krankheit}\label{aa-gott-belehrt-die-menschen-uxfcber-seine-absichten-und-uxfcber-ihre-suxfcnde-bald-durch-truxe4ume-bald-durch-leiden-besonders-durch-krankheit}}

13»Warum hast du den Vorwurf gegen ihn erhoben, daß er dir auf alle
deine Worte keine Antwort gebe? 14Vielmehr redet Gott einmal und
zweimal\textless sup title=``oder: auf die eine und andere
Weise''\textgreater✲, man achtet nur nicht darauf. 15Im Traum, im
Nachtgesicht, wenn tiefer Schlaf die Menschen befällt, im
Schlummerzustand auf dem Lager: 16da öffnet er den Menschen das Ohr und
schreckt sie durch Verwarnung, 17um den Menschen von seinem (bösen) Tun
abzubringen und den Mann vor Überhebung zu behüten, 18um seine Seele vor
der Grube\textless sup title=``oder: der Unterwelt''\textgreater✲ zu
bewahren und sein Leben vor dem Geschoß des Todes. 19Auch wird er durch
Schmerzen auf seinem Lager in Zucht genommen\textless sup title=``oder:
gemahnt''\textgreater✲ und durch andauernden Leidenskampf in seinen
Gliedern, 20so daß für seinen Lebenstrieb alle Nahrung zum Ekel wird und
für seine Eßlust sogar die Lieblingsspeise; 21sein Fleisch schwindet
dahin, daß es nicht mehr zu sehen ist, und seine vordem verborgenen
Knochen treten zu Tage, 22so daß seine Seele der Grube\textless sup
title=``oder: der Unterwelt''\textgreater✲ nahe kommt und sein Leben den
Todesmächten.«

\hypertarget{bb-wenn-die-menschen-sich-dann-unter-beistand-eines-engelmittlers-bekehren-so-erhalten-sie-verzeihung-von-gott-und-vuxf6llige-begnadigung}{%
\subparagraph{bb) Wenn die Menschen sich dann unter Beistand eines
Engelmittlers bekehren, so erhalten sie Verzeihung von Gott und völlige
Begnadigung}\label{bb-wenn-die-menschen-sich-dann-unter-beistand-eines-engelmittlers-bekehren-so-erhalten-sie-verzeihung-von-gott-und-vuxf6llige-begnadigung}}

23»Wenn dann ein Engel für ihn da ist, ein Fürsprecher\textless sup
title=``oder: Mittler''\textgreater✲, ein einziger aus den tausend, um
für den Menschen Zeugnis von seiner Gerechtigkeit abzulegen, 24und
dieser sich seiner erbarmt und (zu Gott) spricht: ›Laß ihn frei, daß er
nicht in die Grube\textless sup title=``oder: die
Unterwelt''\textgreater✲ hinabfährt! Ich habe eine Sühne\textless sup
title=``oder: das Lösegeld''\textgreater✲ gefunden‹, 25so strotzt sein
Leib wieder von Jugendkraft, so daß er in die Tage seines
Jünglingsalters zurückversetzt wird. 26Er betet zu Gott, und dieser
nimmt ihn gnädig an, läßt ihn sein Angesicht unter Jauchzen schauen und
gibt dem Menschen seine Gerechtigkeit zurück. 27Er singt vor dem Volke
und bekennt: ›Ich hatte gesündigt und das Recht verkehrt, aber es ist
mir nicht vergolten worden! 28Erlöst hat (Gott) meine Seele, daß sie
nicht in die Grube\textless sup title=``oder: Unterwelt''\textgreater✲
gefahren ist, und mein Leben erfreut sich am Anblick des Lichts!‹«

\hypertarget{c-aufforderung-an-hiob-sich-entweder-belehren-zu-lassen-oder-elihus-darlegung-zu-widerlegen}{%
\paragraph{c) Aufforderung an Hiob, sich entweder belehren zu lassen
oder Elihus Darlegung zu
widerlegen}\label{c-aufforderung-an-hiob-sich-entweder-belehren-zu-lassen-oder-elihus-darlegung-zu-widerlegen}}

29»Sieh, dies alles tut Gott zweimal, ja dreimal an dem Menschen, 30um
seine Seele von der Grube\textless sup title=``oder:
Unterwelt''\textgreater✲ fernzuhalten und damit er vom Licht des
Lebens\textless sup title=``oder: der Lebenden''\textgreater✲ umleuchtet
werde. 31Merke auf, Hiob, höre mir zu, schweige und laß mich reden!
32Hast du etwas einzuwenden, so widerlege mich; sprich, denn ich möchte
dich gern rechtfertigen\textless sup title=``oder: dir recht
geben''\textgreater✲. 33Hast du aber nichts, so höre mir zu; schweige,
damit ich dich Weisheit lehre!«

\hypertarget{zweite-rede-elihus}{%
\subsubsection{3. Zweite Rede Elihus}\label{zweite-rede-elihus}}

\hypertarget{a-feststellung-der-behauptung-hiobs-gott-sei-ungerecht}{%
\paragraph{a) Feststellung der Behauptung Hiobs, Gott sei
ungerecht}\label{a-feststellung-der-behauptung-hiobs-gott-sei-ungerecht}}

\hypertarget{section-33}{%
\section{34}\label{section-33}}

1Elihu hob dann wieder an und sagte: 2»Vernehmt, ihr Weisen, meine Worte
und, ihr Einsichtigen, schenkt mir Gehör! 3denn das Ohr prüft die Worte,
wie der Gaumen die Speisen kostet. 4Wir wollen doch prüfend das Recht
finden, wollen gemeinsam erforschen, was gut ist. 5Denn Hiob hat
behauptet: ›Ich bin gerecht\textless sup title=``oder: im
Recht''\textgreater✲, aber Gott hat mir mein Recht vorenthalten; 6trotz
meines Rechtes soll ich ein Lügner sein! Tödlich steckt sein Pfeil in
mir, ohne daß ich mich verschuldet habe!‹«

\hypertarget{b-durch-diese-luxe4sterung-gottes-macht-sich-hiob-zu-einem-frevler-denn-gott-kann-seinem-ganzen-wesen-nach-besonders-als-weltherrscher-nicht-ungerecht-sein}{%
\paragraph{b) Durch diese Lästerung Gottes macht sich Hiob zu einem
Frevler; denn Gott kann seinem ganzen Wesen nach (besonders als
Weltherrscher) nicht ungerecht
sein}\label{b-durch-diese-luxe4sterung-gottes-macht-sich-hiob-zu-einem-frevler-denn-gott-kann-seinem-ganzen-wesen-nach-besonders-als-weltherrscher-nicht-ungerecht-sein}}

7»Wo ist ein Mann wie Hiob, der Lästerrede trinkt wie Wasser 8und in
Gemeinschaft mit Übeltätern getreten ist und mit Frevlern Umgang pflegt?
9Denn er hat behauptet: ›Der Mensch hat keinen Nutzen davon, daß er mit
Gott die Freundschaft aufrecht hält.‹

10Darum hört mich an, ihr einsichtsvollen Männer! Fern bleibe der
Vorwurf von Gott, daß er Frevel verübe, und vom Allmächtigen, daß er
Unrecht tue! 11Nein, was der Mensch tut, das vergilt er ihm und läßt es
jedem nach seinem Lebenswandel ergehen. 12Ja wahrlich, Gott handelt
nicht frevelhaft, und der Allmächtige beugt das Recht nicht. 13Wer hat
die Erde seiner Obhut anvertraut und wer den ganzen Erdkreis
hergestellt? 14Wenn er nur an sich selbst dächte, seinen Geist und
seinen Odem in sich zurückzöge, 15so müßte alles Fleisch insgesamt
verscheiden und der Mensch wieder zu Staub werden.

16Wenn du also verständig bist, so höre dies und gib wohl acht, wie
meine Worte lauten! 17Kann auch, wer das Recht haßt, ein Gemeinwesen
leiten? Oder willst du den Allgerechten verdammen, 18ihn, der zum Könige
sagt: ›Du Nichtswürdiger!‹ und zu den Hochgestellten: ›Du Bösewicht!‹,
19ihn, der die Person der Fürsten\textless sup title=``oder:
Großen''\textgreater✲ nicht ansieht und den Vornehmen nicht vor dem
Geringen bevorzugt, weil sie ja alle das Werk seiner Hände sind. 20In
einem Augenblick sterben sie, und mitten in der Nacht wird ein Volk
erschüttert und muß dahinfahren, und Machthaber beseitigt er, ohne die
Hand zu rühren. 21Denn seine Augen sind auf die Wege\textless sup
title=``=~den Wandel''\textgreater✲ eines jeden Menschen gerichtet, und
er sieht alle seine Schritte: 22da gibt es kein Dunkel und keine noch so
dichte Finsternis, daß die Frevler sich darin verbergen könnten. 23Denn
er braucht einen Menschen nicht erst lange zu beobachten, damit er vor
Gott zum Gericht erscheine: 24nein, er zerschmettert Gewalthaber ohne
Untersuchung und läßt andere an ihre Stelle treten. 25Somit kennt er
ihre Taten wohl und stürzt sie über Nacht, so daß sie zermalmt werden.
26Als Frevler, die sie sind, geißelt er sie vor aller Augen 27zur Strafe
dafür, daß sie von ihm abgefallen sind und alle seine Wege\textless sup
title=``=~sein ganzes Walten''\textgreater✲ unbeachtet gelassen haben,
28so daß sie den Hilferuf des Armen zu ihm hinaufdringen ließen und er
den Notschrei der Bedrückten vernehmen mußte. 29Verhält er sich aber
ruhig, wer darf ihn verdammen? Und verhüllt er sein Angesicht, wer kann
ihn schauen? So waltet er sowohl über Völkern als auch über einzelnen
Menschen gleicherweise, 30damit nicht ruchlose Menschen die Herrschaft
führen, Leute, welche Fallstricke für das Volk sein würden.«

\hypertarget{c-hiobs-urteil-uxfcber-gott-ist-anmauxdfend-tuxf6richt-und-frevelhaft-und-verdient-die-schwerste-strafe}{%
\paragraph{c) Hiobs Urteil über Gott ist anmaßend, töricht und
frevelhaft und verdient die schwerste
Strafe}\label{c-hiobs-urteil-uxfcber-gott-ist-anmauxdfend-tuxf6richt-und-frevelhaft-und-verdient-die-schwerste-strafe}}

31»Denn soll etwa Gott zu dir sagen: ›Ich habe mich geirrt; will (aber)
nicht wieder verkehrt handeln? 32Über das, was ich nicht sehe, belehre
du mich; wenn ich unrecht gehandelt habe, will ich es nicht wieder tun.‹
33Soll er nach deinem Sinn Vergeltung üben, weil du unzufrieden bist,
und sagen: ›Du hast das Bessere zu bestimmen, nicht ich; was du also
weißt, das sprich aus!‹?

34Verständige Leute werden mir zugestehen und jeder weise Mann, der mir
zuhört: 35›Hiob redet ohne Einsicht, und seine Worte sind nicht
wohlbedacht.‹ 36O daß doch Hiob fort und fort geprüft würde wegen seiner
Widerreden nach Art der Frevler! 37Denn zu seiner Verfehlung fügt er
noch den Abfall (von Gott) hinzu: er höhnt laut in unserer Mitte und
macht viel Redens gegen Gott.«

\hypertarget{dritte-rede-elihus}{%
\subsubsection{4. Dritte Rede Elihus}\label{dritte-rede-elihus}}

\hypertarget{a-elihu-bekuxe4mpft-die-behauptung-hiobs-dauxdf-die-gottesfurcht-keinen-nutzen-bringe-mit-dem-hinweis-darauf-dauxdf-das-tun-der-menschen-nicht-fuxfcr-gott-sondern-nur-fuxfcr-die-menschen-von-bedeutung-sei}{%
\paragraph{a) Elihu bekämpft die Behauptung Hiobs, daß die Gottesfurcht
keinen Nutzen bringe, mit dem Hinweis darauf, daß das Tun der Menschen
nicht für Gott, sondern nur für die Menschen von Bedeutung
sei}\label{a-elihu-bekuxe4mpft-die-behauptung-hiobs-dauxdf-die-gottesfurcht-keinen-nutzen-bringe-mit-dem-hinweis-darauf-dauxdf-das-tun-der-menschen-nicht-fuxfcr-gott-sondern-nur-fuxfcr-die-menschen-von-bedeutung-sei}}

\hypertarget{section-34}{%
\section{35}\label{section-34}}

1Elihu hob dann wieder an und sagte: 2»Hältst du das für recht, nennst
du das ›meine Gerechtigkeit vor Gott‹, 3daß du fragst: ›Was nützt sie
mir?‹ und: ›Was habe ich mehr davon, als wenn ich sündigte?‹ 4Ich will
dir darauf die Antwort geben, dir und zugleich deinen Freunden neben
dir. 5Blicke zum Himmel empor und sieh ihn an und schaue zu den Wolken
hinauf, die hoch über dir sind: 6wenn du sündigst, was tust du ihm damit
zuleide? Und sind deine Übertretungen zahlreich, welchen Schaden fügst
du ihm damit zu? 7Und so auch: wenn du gerecht✲ bist, welches Geschenk
machst du ihm damit, oder was empfängt er aus deiner Hand? 8Nur den
Menschen, wie du einer bist, geht dein Freveln an, und nur dir, dem
Menschensohn, kommt dein Gerechtsein zugute.«

\hypertarget{b-die-huxe4ufigen-fuxe4lle-von-nichterhuxf6rung-schuldloser-menschen-die-uxfcber-gewalttuxe4tige-behandlung-zu-klagen-haben-erkluxe4ren-sich-aus-dem-mangel-an-gottvertrauen-oder-an-gottesfurcht-der-betreffenden}{%
\paragraph{b) Die häufigen Fälle von Nichterhörung schuldloser Menschen,
die über gewalttätige Behandlung zu klagen haben, erklären sich aus dem
Mangel an Gottvertrauen oder an Gottesfurcht der
Betreffenden}\label{b-die-huxe4ufigen-fuxe4lle-von-nichterhuxf6rung-schuldloser-menschen-die-uxfcber-gewalttuxe4tige-behandlung-zu-klagen-haben-erkluxe4ren-sich-aus-dem-mangel-an-gottvertrauen-oder-an-gottesfurcht-der-betreffenden}}

9»Man schreit wohl über die Menge der Bedrückungen, klagt laut über die
Gewalttätigkeit der Großen, 10doch keiner sagt: ›Wo ist Gott, mein
Schöpfer, der Lobgesänge schenkt in der Nacht, 11der uns Belehrung
verleiht wie keinem Tiere des Feldes und uns höhere Weisheit gewinnen
läßt als die Vögel des Himmels?‹ 12Da schreit man denn, ohne Erhörung
bei ihm zu finden, wegen des Übermuts der Bösen. 13Jawohl: auf eitles
Klagen hört Gott nicht, sondern der Allmächtige läßt es unbeachtet.
14Nun sagst du aber gar, du sehest ihn nicht; deine Sache liege ihm vor,
du wartest aber vergeblich auf seine Entscheidung! 15Und nun, da sein
Zorn noch nicht gestraft und er sich um Torheit nicht sonderlich
gekümmert hat, 16da reißt Hiob seinen Mund zu leerem Gerede auf und
ergeht sich ohne Einsicht in vermessenen Worten!«

\hypertarget{vierte-rede-elihus}{%
\subsubsection{5. Vierte Rede Elihus}\label{vierte-rede-elihus}}

\hypertarget{a-elihu-begruxfcndet-seine-folgende-unterweisung-mit-dem-hinweis-auf-sein-vollkommenes-wissen}{%
\paragraph{a) Elihu begründet seine folgende Unterweisung mit dem
Hinweis auf sein vollkommenes
Wissen}\label{a-elihu-begruxfcndet-seine-folgende-unterweisung-mit-dem-hinweis-auf-sein-vollkommenes-wissen}}

\hypertarget{section-35}{%
\section{36}\label{section-35}}

1Hierauf fuhr Elihu weiter fort zu reden: 2»Gedulde dich nur noch ein
wenig, daß ich dich unterweise! Denn ich habe für Gottes Sache noch mehr
zu sagen. 3Ich will mit meinem Wissen weit ausholen, um meinem Schöpfer
zu seinem Recht zu verhelfen; 4denn wahrlich, meine Worte sind kein
Trug: ein Mann mit vollkommener Erkenntnis verhandelt mit dir.«

\hypertarget{b-gott-will-die-menschen-durch-leiden-zu-ihrem-heil-erziehen-besonders-zur-selbsterkenntnis-und-zum-gehorsam-fuxfchren-was-ihm-aber-nur-bei-den-gottesfuxfcrchtigen-gelingt}{%
\paragraph{b) Gott will die Menschen durch Leiden zu ihrem Heil
erziehen, besonders zur Selbsterkenntnis und zum Gehorsam führen, was
ihm aber nur bei den Gottesfürchtigen
gelingt}\label{b-gott-will-die-menschen-durch-leiden-zu-ihrem-heil-erziehen-besonders-zur-selbsterkenntnis-und-zum-gehorsam-fuxfchren-was-ihm-aber-nur-bei-den-gottesfuxfcrchtigen-gelingt}}

5»Siehe, Gott ist gewaltig und doch nicht teilnahmslos, gewaltig an
Kraft des Herzens✲. 6Er erhält den Frevler nicht am Leben, läßt aber den
Elenden ihr Recht zukommen. 7Er wendet seine Augen von dem Gerechten
nicht ab, und Königen auf dem Thron verschafft er für immer einen festen
Sitz, damit sie erhöht sind. 8Wenn sie aber mit Ketten gefesselt sind
und in Unglücksbanden gefangen liegen, 9so hält er ihnen damit ihr Tun
vor, ihre Übertretungen, daß sie sich nämlich überhoben haben; 10da
öffnet er ihnen das Ohr für Warnungen und mahnt sie, sich vom Frevel
abzuwenden. 11Wenn sie nun darauf hören und sich unterwerfen, so beenden
sie ihre Tage im Glück und ihre Jahre in Wonne\textless sup
title=``oder: im Wohlergehen''\textgreater✲; 12wollen sie aber nicht
darauf hören, so fallen sie dem Todesgeschoß\textless sup
title=``=~einem plötzlichen Tode''\textgreater✲ anheim und verscheiden
in Unverstand\textless sup title=``=~ohne Erkenntnis''\textgreater✲.
13Dann geraten aber solche ruchlos Gesinnte in Zorn: sie schreien nicht
um Hilfe, obgleich er\textless sup title=``d.h. Gott''\textgreater✲ sie
in Fesseln geschlagen hat. 14So stirbt denn ihre Seele schon in der
Jugendkraft dahin, und ihr Leben endet wie das der Lustknaben. 15Die
Dulder dagegen errettet er (gerade) durch ihr Dulden und öffnet ihnen
durch die Leiden das Ohr.«

\hypertarget{c-daher-muxf6ge-jetzt-auch-hiob-durch-sein-leiden-sich-luxe4utern-lassen-um-des-guxf6ttlichen-segens-teilhaftig-zu-werden}{%
\paragraph{c) Daher möge jetzt auch Hiob durch sein Leiden sich läutern
lassen, um des göttlichen Segens teilhaftig zu
werden}\label{c-daher-muxf6ge-jetzt-auch-hiob-durch-sein-leiden-sich-luxe4utern-lassen-um-des-guxf6ttlichen-segens-teilhaftig-zu-werden}}

16»So sucht er auch dich aus dem Rachen der Not auf weiten Raum zu
führen, wo keine Enge mehr ist, und dein Tisch würde mit fettem Mahl
reich besetzt sein; 17du aber hast dich ganz dem frevelhaften Urteilen
hingegeben, darum werden Urteil und Gericht dich treffen. 18Laß die
Leidenschaft dich ja nicht zu Lästerungen verleiten und die Größe des
Lösegeldes\textless sup title=``oder: der auferlegten
Sühne''\textgreater✲ dich nicht beirren! 19Wird etwa dein Geschrei dich
aus der Bedrängnis herausbringen und alle noch so gewaltigen
Anstrengungen? 20Sehne die Nacht nicht herbei, wo Völker an ihrer Stätte
auffahren! 21Hüte dich, wende dich nicht dem Frevel zu; denn dazu bist
du eher geneigt als zum Leiden. 22Bedenke wohl: Gott vollbringt erhabene
Dinge durch seine Kraft: wer ist ein Lehrmeister wie er? 23Wer hat ihm
sein Walten vorgeschrieben? Und wer hat je zu ihm sagen dürfen: ›Du hast
unrecht gehandelt\textless sup title=``oder: Frevel
verübt''\textgreater✲‹? 24Sei darauf bedacht, sein Tun\textless sup
title=``oder: Walten''\textgreater✲ zu erheben, das die Menschen in
Liedern preisen! 25Alle Menschen schauen es bewundernd an, und doch
erblickt es der Sterbliche nur von ferne.«

\hypertarget{d-schilderung-von-gottes-gruxf6uxdfe-herrlichkeit-und-weisheit-die-sich-in-der-natur-offenbaren}{%
\paragraph{d) Schilderung von Gottes Größe, Herrlichkeit und Weisheit,
die sich in der Natur
offenbaren}\label{d-schilderung-von-gottes-gruxf6uxdfe-herrlichkeit-und-weisheit-die-sich-in-der-natur-offenbaren}}

26»Bedenke wohl: Gott ist zu erhaben für unsere Erkenntnis; die Zahl
seiner Jahre, sie ist unerforschlich. 27Denn er zieht Tropfen aus dem
Meer empor, daß sie von dem Dunst, den er bildet, als Regen
niederträufeln, 28von dem die Wolken triefen und den sie auf die
Menschenmenge rieseln lassen. 29Wie kann man vollends die Ausbreitungen
der Gewitterwolken verstehen, den Donnerschall seines Zeltes? 30Siehe,
er breitet sein Licht darüber\textless sup title=``oder: um sich
her''\textgreater✲ aus und bedeckt damit die tiefsten Tiefen des Meeres!
31denn dadurch richtet er die Völker, spendet zugleich aber auch Nahrung
in reicher Fülle. 32Beide Hände hüllt er in den leuchtenden Blitz und
entbietet ihn gegen den Angreifer. 33Sein Donnergetöse kündigt ihn an
als einen, der seinen Zorn gegen den Frevel eifern\textless sup
title=``=~in Eifer geraten''\textgreater✲ läßt.

\hypertarget{section-36}{%
\section{37}\label{section-36}}

1Ja, darüber erzittert mein Herz und fährt stürmisch empor von seiner
Stelle. 2Hört, o hört auf das Donnern seiner Stimme und auf das Tosen,
das seinem Munde entfährt! 3Er entfesselt es unter dem ganzen Himmel hin
und sein Blitzesleuchten bis an die Säume der Erde. 4Hinter (dem Blitz)
her brüllt der Donner; er dröhnt mit seiner hehren Stimme und hält (die
Blitze) nicht zurück, sobald sein Donner sich vernehmen läßt. 5Gott
donnert mit seiner Stimme wunderbar, er, der große Dinge tut, die wir
nicht begreifen. 6Denn dem Schnee gebietet er: ›Falle auf die Erde
nieder!‹ und ebenso dem Regenguß: ›Falle als Dauerregen nieder!‹ 7Dann
zwingt er die Hände aller Menschen zur Untätigkeit, damit alle Menschen
zur Erkenntnis seines Wirkens\textless sup title=``oder:
Waltens''\textgreater✲ kommen. 8Da zieht sich das Wild in sein Versteck
zurück und hält sich ruhig in seinen Schlupfwinkeln. 9Aus der
Kammer\textless sup title=``des Südens; vgl. 9,9''\textgreater✲ bricht
der Sturm hervor und von den Nordwinden die Kälte: 10durch den Hauch
Gottes entsteht das Eis, und die weite Wasserfläche liegt in enger Haft.
11Auch belastet er mit Wasserfülle\textless sup title=``oder:
Hagel''\textgreater✲ das Gewölk, läßt seine Blitzwolken überströmen;
12die wenden sich dann unter seiner Leitung hierhin und dorthin, um
alles, was er ihnen gebietet, auszurichten auf dem ganzen weiten
Erdkreise: 13bald als Rute✲, wenn sie seinem Lande not tut, bald als
Huldbeweis\textless sup title=``=~zum Segen''\textgreater✲ läßt er sie
sich entladen.«

\hypertarget{e-mahnung-an-hiob-diesem-gott-gegenuxfcber-nicht-hochmuxfctige-herausforderung-sondern-demuxfctige-beugung-zu-beobachten}{%
\paragraph{e) Mahnung an Hiob, diesem Gott gegenüber nicht hochmütige
Herausforderung, sondern demütige Beugung zu
beobachten}\label{e-mahnung-an-hiob-diesem-gott-gegenuxfcber-nicht-hochmuxfctige-herausforderung-sondern-demuxfctige-beugung-zu-beobachten}}

14»Vernimm dies, Hiob! Stehe still und erwäge die Wunderwerke Gottes!
15Begreifst du es, wie Gott ihnen Befehl erteilt und das
Licht\textless sup title=``=~den Blitzstrahl''\textgreater✲ seines
Gewölks aufleuchten läßt? 16Verstehst du dich auf das Schweben der
Wolken, auf die Wundertaten des an Weisheit Vollkommenen, 17du, dem die
Kleider zu heiß werden, wenn das Land beim Südwind in schwüler Hitze
daliegt? 18Kannst du gleich ihm das Himmelsgewölbe ausbreiten, das fest
ist wie ein gegossener Spiegel? 19Laß uns wissen, was wir ihm sagen
sollen! Wir können vor Finsternis nichts vorbringen. 20Soll ihm gemeldet
werden, daß ich reden wolle? Hat wohl je ein Mensch gefordert, er wolle
vernichtet sein? 21Und nun: in das Sonnenlicht kann man nicht blicken,
wenn es am Himmelsgewölbe strahlt, nachdem der Wind darüber hingefahren
ist und (den Himmel) geklärt hat. 22Von Norden her kommt das Nordlicht:
um Gott her liegt furchtbare Pracht✲. 23Den Allmächtigen, wir erreichen
ihn nicht, ihn, der an Kraft gewaltig ist; aber das Recht und die volle
Gerechtigkeit beugt er nicht. 24Darum sollen die Menschen ihn fürchten:
er sieht keinen an, der sich selbst weise dünkt!«

\hypertarget{viii.-gottes-offenbarung-und-hiobs-demuxfctigung-381-426}{%
\subsection{VIII. Gottes Offenbarung und Hiobs Demütigung
(38,1-42,6)}\label{viii.-gottes-offenbarung-und-hiobs-demuxfctigung-381-426}}

\hypertarget{erste-rede-gottes-und-die-antwort-hiobs}{%
\subsubsection{1. Erste Rede Gottes und die Antwort
Hiobs}\label{erste-rede-gottes-und-die-antwort-hiobs}}

\hypertarget{a-gottes-aufforderung-an-hiob-ihm-rede-zu-stehen}{%
\paragraph{a) Gottes Aufforderung an Hiob, ihm Rede zu
stehen}\label{a-gottes-aufforderung-an-hiob-ihm-rede-zu-stehen}}

\hypertarget{section-37}{%
\section{38}\label{section-37}}

1Da antwortete der HERR dem Hiob aus dem Wettersturme heraus
folgendermaßen: 2»Wer ist's, der da den Heilsplan Gottes verdunkelt mit
Worten ohne Einsicht? 3Auf! Gürte dir die Lenden wie ein Mann, so will
ich dich fragen, und du belehre mich\textless sup title=``oder: gib mir
Bescheid''\textgreater✲!«

\hypertarget{b-fragen-aus-dem-gebiet-der-weltschuxf6pfung-und-der-leblosen-natur-sowie-des-tierlebens-auf-die-hiob-die-antwort-schuldig-bleibt}{%
\paragraph{b) Fragen aus dem Gebiet der Weltschöpfung und der leblosen
Natur sowie des Tierlebens, auf die Hiob die Antwort schuldig
bleibt}\label{b-fragen-aus-dem-gebiet-der-weltschuxf6pfung-und-der-leblosen-natur-sowie-des-tierlebens-auf-die-hiob-die-antwort-schuldig-bleibt}}

4»Wo warst du, als ich die Erde baute? Sprich es aus, wenn du Einsicht
besitzest\textless sup title=``oder: Bescheid weißt''\textgreater✲! 5Wer
hat ihre Maße bestimmt\textless sup title=``oder: ihren Bauplan
entworfen''\textgreater✲ -- du weißt es ja! --, oder wer hat die
Meßschnur über sie ausgespannt? 6Worauf sind ihre Grundpfeiler
eingesenkt worden, oder wer hat ihren Eckstein✲ gelegt, 7während die
Morgensterne allesamt laut frohlockten und alle Gottessöhne\textless sup
title=``d.h. Engel''\textgreater✲ jauchzten?

8Und wer hat das Meer mit Toren verschlossen, als es hervorbrach, aus
dem Mutterschoß heraustrat? 9Als ich Gewölk zu seinem Kleide machte und
dunkle Nebel zu seinen Windeln? 10Als ich ihm das von mir bestimmte
Gebiet absteckte und ihm Riegel und Tore herstellte 11und sprach: ›Bis
hierher darfst du kommen, aber nicht weiter, und hier soll sich der
Stolz deiner Wellen brechen!‹

12Hast du jemals, seitdem du lebst, das Morgenlicht bestellt? Hast du
dem Frührot seine Stätte angewiesen, 13daß es die Säume der Erde erfasse
und die Frevler von ihr verscheucht werden? 14Sie (die Erde) verwandelt
sich alsdann wie Wachs unter dem Siegel, und alles stellt sich dar wie
ein Prachtgewand; 15den Frevlern aber wird ihr Licht entzogen, und der
zum Schlagen schon erhobene Arm zerbricht.

16Bist du bis zu den Quellen des Meeres gekommen, und hast du die
tiefsten Tiefen des Weltmeers durchwandelt? 17Haben sich vor dir die
Pforten des Todes aufgetan, und hast du die Pforten des Schattenreichs
gesehen? 18Hast du die weiten Flächen der Erde überschaut? Sage an, wenn
du dies alles weißt!

19Wo geht denn der Weg nach der Wohnung des Lichts, und die Finsternis,
wo hat sie ihre Heimstätte, 20daß du sie in ihr Gebiet hinbringen
könntest und daß die Pfade zu ihrem Hause dir bekannt wären? 21Du weißt
es ja, denn damals wurdest du ja geboren, und die Zahl deiner Lebenstage
ist groß!

22Bist du zu den Vorratskammern des Schnees gekommen, und hast du die
Speicher des Hagels gesehen, 23den ich aufgespart habe für die
Drangsalszeiten, für den Tag des Kampfes und des Krieges?

24Wo ist der Weg dahin, wo das Licht sich teilt und von wo der Ostwind
sich über die Erde verbreitet? 25Wer hat der Regenflut Kanäle gespalten
und einen Weg dem Donnerstrahl gebahnt, 26um regnen zu lassen auf
menschenleeres Land, auf die Steppe, wo niemand wohnt, 27um die Einöde
und Wildnis reichlich zu tränken und Pflanzengrün sprießen zu lassen?
28Hat der Regen einen Vater, oder wer erzeugt die Tropfen des Taues?
29Aus wessen Mutterschoße geht das Eis hervor, und wer läßt den Reif des
Himmels entstehen? 30Wie zu Stein verhärten sich die Wasser, und der
Spiegel der Fluten schließt sich zur festen Decke zusammen.

31Vermagst du die Bande des Siebengestirns zu knüpfen oder die
Fesseln\textless sup title=``oder: den Gürtel''\textgreater✲ des Orion
zu lösen? 32Läßt du die Bilder des Tierkreises zur rechten Zeit
hervortreten, und leitest du den Großen Bären samt seinen Jungen?
33Kennst du die für den Himmel gültigen Gesetze, oder bestimmst du seine
Herrschaft über die Erde? 34Kannst du deine Stimme hoch zu den Wolken
dringen lassen, daß strömender Regen dich bedecke? 35Entsendest du die
Blitze, daß sie hinfahren und zu dir sagen: ›Hier sind wir‹? 36Wer hat
Weisheit in die Wolkenschichten gelegt oder wer dem Luftgebilde Verstand
verliehen? 37Wer zählt die Federwolken mit Weisheit ab, und die
Schläuche des Himmels, wer läßt sie sich ergießen, 38wenn das Erdreich
sich zu Metallguß verhärtet hat und die Schollen sich fest
zusammenballen?

39Erjagst du für die Löwin die Beute, und stillst du die Gier der jungen
Leuen, 40wenn sie in ihren Höhlen kauern, im Dickicht auf der Lauer
liegen? 41Wer verschafft dem Raben sein Futter, wenn seine Jungen zu
Gott schreien und wegen Mangels an Nahrung umherirren?

\hypertarget{section-38}{%
\section{39}\label{section-38}}

1Kennst du die Zeit, wo die Felsgemsen\textless sup title=``oder:
Steinböcke''\textgreater✲ werfen, und überwachst du das Kreißen der
Hirschkühe? 2Zählst du die Monde, während derer sie trächtig sind, und
weißt du die Zeit, wann sie gebären? 3Sie kauern nieder, lassen ihre
Jungen zur Welt kommen, entledigen sich leicht ihrer Geburtsschmerzen.
4Ihre Jungen erstarken, werden im Freien groß; sie laufen davon und
kehren nicht wieder zu ihnen zurück.

5Wer hat den Wildesel frei laufen lassen und wer die Bande dieses
Wildfangs gelöst, 6dem ich die Steppe zur Heimat angewiesen habe und zur
Wohnung die Salzgegend? 7Er lacht des Gewühls der Stadt, den lauten
Zuruf des Treibers hört er nicht. 8Was er auf den Bergen erspäht, ist
seine Weide, und jedem grünen Halme spürt er nach.

9Wird der Büffel Lust haben, dir zu dienen oder nachts an deiner Krippe
zu lagern? 10Kannst du den Büffel mit seinem Leitseil an die Furche
binden, oder wird er über Talgründe die Egge hinter dir herziehen?
11Darfst du ihm trauen, weil er große Kraft besitzt, und ihm deinen
Ernteertrag\textless sup title=``oder: deine Feldarbeit''\textgreater✲
überlassen? 12Darfst du ihm zutrauen, daß er deine Saat einbringen und
sie auf deiner Tenne zusammenfahren werde?

13Die Straußenhenne schwingt fröhlich ihre Flügel: sind es aber des
(liebevollen) Storches Schwingen und Gefieder? 14Nein, sie vertraut ihre
Eier der Erde an und läßt sie auf dem Sande warm werden; 15denn sie
denkt nicht daran, daß ein Fuß sie\textless sup title=``oder:
eins''\textgreater✲ dort zerdrücken und ein wildes Tier sie\textless sup
title=``oder: eins''\textgreater✲ zertreten kann. 16Hart behandelt sie
ihre Jungen, als gehörten sie ihr nicht; ob ihre Mühe vergeblich ist,
das kümmert sie nicht; 17denn Gott hat ihr große Klugheit versagt und
ihr keinen Verstand zugeteilt. 18Doch sobald sie hoch auffährt zum
Laufen, verlacht sie das Roß und seinen Reiter.

19Gibst du dem Roß die gewaltige Stärke? Bekleidest du seinen Hals mit
der wallenden Mähne? 20Machst du es springen wie die Heuschrecke? Sein
stolzes Schnauben -- wie erschreckend! 21Es scharrt den Boden im
Blachfeld und freut sich seiner Kraft, zieht der gewappneten Schar
entgegen. 22Es lacht über Furcht und erschrickt nicht, macht nicht kehrt
vor dem Schwert; 23auf ihm klirrt ja der Köcher, blitzen der Speer und
der Kurzspieß. 24Mit Ungestüm und laut stampfend sprengt es im Fluge
dahin und läßt sich nicht halten, wenn die Posaune erschallt; 25bei
jedem Trompetenstoß ruft es ›Hui!‹ und wittert den Kampf von fern, den
Donnerruf✲ der Heerführer und das Schlachtgetöse.

26Hebt der Habicht dank deiner Einsicht die Schwingen, breitet seine
Flügel aus nach dem Süden zu? 27Oder schwebt der Adler auf dein Geheiß
empor und baut sein Nest in der Höhe? 28Auf Felsen wohnt er und horstet
auf Felszacken und Bergspitzen; 29von dort späht er nach Beute aus: in
weite Ferne blicken seine Augen; 30und seine Jungen schon verschlingen
gierig das Blut, und wo Erschlagene liegen, da ist auch er.«

\hypertarget{c-gottes-aufforderung-an-hiob-mit-ihm-in-den-rechtsstreit-einzutreten-hiob-gibt-seine-anklagen-gegen-gott-auf}{%
\paragraph{c) Gottes Aufforderung an Hiob, mit ihm in den Rechtsstreit
einzutreten; Hiob gibt seine Anklagen gegen Gott
auf}\label{c-gottes-aufforderung-an-hiob-mit-ihm-in-den-rechtsstreit-einzutreten-hiob-gibt-seine-anklagen-gegen-gott-auf}}

\hypertarget{section-39}{%
\section{40}\label{section-39}}

1Hierauf wandte sich der HERR weiter an Hiob mit der Frage: 2»Hadern
will der Tadler mit dem Allmächtigen? Der Ankläger Gottes gebe Antwort
darauf!«

3Da antwortete Hiob dem HERRN: 4»Ach, ich bin zu gering: was soll ich
dir entgegnen? Ich lege meine Hand auf den Mund! 5Einmal habe ich
geredet, werde aber nichts mehr entgegnen; und noch ein zweites Mal habe
ich es getan, doch niemals tue ich es wieder.«

\hypertarget{zweite-rede-gottes-und-hiobs-antwort}{%
\subsubsection{2. Zweite Rede Gottes und Hiobs
Antwort}\label{zweite-rede-gottes-und-hiobs-antwort}}

\hypertarget{a-gottes-frage-ob-hiob-die-gerechtigkeit-gottes-wirklich-in-zweifel-zu-ziehen-wage}{%
\paragraph{a) Gottes Frage, ob Hiob die Gerechtigkeit Gottes wirklich in
Zweifel zu ziehen
wage}\label{a-gottes-frage-ob-hiob-die-gerechtigkeit-gottes-wirklich-in-zweifel-zu-ziehen-wage}}

6Weiter antwortete der HERR dem Hiob aus dem Wettersturm heraus
folgendermaßen:

7»Auf! Gürte dir die Lenden wie ein Mann: ich will dich fragen, und du
belehre mich! 8Willst du wirklich mein Recht zunichte machen, mich
schuldig sprechen, damit du als gerecht dastehst\textless sup
title=``=~recht behältst''\textgreater✲?«

\hypertarget{b-gottes-aufforderung-an-hiob-in-guxf6ttlicher-majestuxe4t-aufzutreten-und-die-stolzen-und-frevler-zu-zermalmen}{%
\paragraph{b) Gottes Aufforderung an Hiob, in göttlicher Majestät
aufzutreten und die Stolzen und Frevler zu
zermalmen}\label{b-gottes-aufforderung-an-hiob-in-guxf6ttlicher-majestuxe4t-aufzutreten-und-die-stolzen-und-frevler-zu-zermalmen}}

9»Hast du etwa einen Arm wie Gott, und vermagst du den Donner so laut
rollen zu lassen wie er? 10So schmücke dich doch mit Erhabenheit und
Hoheit und kleide dich in Pracht und Herrlichkeit! 11Laß die Ausbrüche
deines Zorns sich ergießen! Und gewahrst du irgendeinen Hochmütigen, so
wirf ihn nieder! 12Ja, gewahrst du irgendeinen Hochmütigen, so demütige
ihn und stürze die Frevler nieder, wo sie stehen! 13Laß sie allesamt
tief in den Staub sinken, laß ihr Angesicht erstarren in Todesgrauen!
14Dann will auch ich dich lobend anerkennen, daß deine Rechte dir den
Sieg verliehen hat.«

\hypertarget{c-gottes-aufforderung-an-hiob-die-gewalt-uxfcber-das-nilpferd-oder-uxfcber-das-krokodil-zu-uxfcbernehmen}{%
\paragraph{c) Gottes Aufforderung an Hiob, die Gewalt über das Nilpferd
oder über das Krokodil zu
übernehmen}\label{c-gottes-aufforderung-an-hiob-die-gewalt-uxfcber-das-nilpferd-oder-uxfcber-das-krokodil-zu-uxfcbernehmen}}

15»Sieh doch das Nilpferd an, das ich geschaffen habe wie dich: von
Pflanzen nährt es sich wie das Rind! 16Sieh doch, welche Kraft bei ihm
in den Lenden wohnt und welche Stärke in den Muskeln seines Leibes! 17Es
macht seinen Schwanz so starr wie eine Zeder; die Sehnen seiner Schenkel
sind fest verflochten. 18Seine Knochen sind Röhren von Erz, seine
Gebeine\textless sup title=``oder: Schulterblätter''\textgreater✲ gleich
geschmiedeten Eisenstangen. 19Es ist der Erstling\textless sup
title=``=~das Meisterstück''\textgreater✲ der schöpferischen Tätigkeit
Gottes; sein Bildner hat ihm auch sein Schwert verliehen. 20Denn Futter
liefern ihm die Anhöhen, wo alle wilden Landtiere spielen\textless sup
title=``=~sich lustig tummeln''\textgreater✲. 21Unter Lotusbüschen
lagert es sich, im Versteck von Schilfrohr und Sumpf; 22Lotusbüsche
geben ihm Deckung mit ihrem Schattendach, und die Weiden des Baches
umgeben es. 23Selbst wenn der Strom mächtig anschwillt, gerät es nicht
in Unruhe: es bleibt wohlgemut, wenn auch ein Jordan\textless sup
title=``oder: Sturzbach''\textgreater✲ gegen seinen Rachen andringt.
24Wer will es von vorn packen, wer mit einem Fangseil ihm die Nase
durchbohren?

25Kannst du das Krokodil\textless sup title=``eig. der
Leviathan''\textgreater✲ am Angelhaken heranziehen und ihm die Zunge mit
der Schnur\textless sup title=``oder: dem Fangseil''\textgreater✲
niederdrücken? 26Kannst du ihm einen Binsenring durch die Nase ziehen
und einen Dorn✲ durch seinen Kinnbacken bohren? 27Meinst du, es werde
viele Bitten an dich richten oder dir gute Worte geben? 28Wird es einen
Vertrag mit dir schließen, wonach du es für immer in deine Dienste
nähmest? 29Wirst du mit ihm spielen wie mit einem Vöglein und es zur
Kurzweil\textless sup title=``=~als Spielzeug''\textgreater✲ für deine
Mägdlein anbinden? 30Treibt die Fischerzunft Handel mit ihm, daß sie es
stückweise an die Händler abgibt? 31Kannst du ihm die Haut mit Spießen
spicken und seinen Kopf mit Fischerhaken✲ durchbohren? 32Vergreife dich
nur einmal an ihm: mache dich auf Kampf gefaßt! Du wirst's gewiß nicht
wieder tun!

\hypertarget{section-40}{%
\section{41}\label{section-40}}

1Ja, eine solche Hoffnung erweist sich als Trug: schon bei seinem
Anblick bricht man zusammen. 2Niemand ist so tollkühn, daß er es
aufstört; und wer ist es, der ihm entgegengetreten und heil
davongekommen wäre? 3Wer unter dem ganzen Himmel ist es? 4Nicht
schweigen will ich von seinen Gliedmaßen, weder von seiner Kraftfülle
noch von der Schönheit seines Baues. 5Wer hat je sein Panzerkleid oben
aufgedeckt und wer sich in die Doppelreihe seines Gebisses hineingewagt?
6Wer hat je das Doppeltor seines Rachens geöffnet? Rings um seine Zähne
herum lagert Schrecken. 7Prachtvoll sind die Zeilen seiner
Schilder\textless sup title=``oder: die Rinnen seiner
Schuppenplatten''\textgreater✲, jede einzelne enganliegend wie durch ein
festes Siegel: 8eine schließt sich eng an die andere an, und kein
Lüftchen dringt zwischen ihnen ein: 9jede haftet fest an der andern, sie
greifen untrennbar ineinander. 10Sein Niesen läßt einen Lichtschein
erglänzen, und seine Augen gleichen den Wimpern des Morgenrots. 11Aus
seinem Rachen schießen Flammen, sprühen Feuerfunken hervor. 12Aus seinen
Nüstern strömt Rauch heraus wie aus einem siedenden Topf und wie aus
Binsenfeuer. 13Sein Atem setzt Kohlen in Brand, und Flammen entfahren
seinem Rachen. 14In seinem Nacken wohnt Kraft, und vor ihm her stürmt
bange Furcht dahin. 15Die Wampen seines Leibes haften fest zusammen,
sind wie angegossen an ihm, unbeweglich. 16Sein Herz ist hart wie ein
Stein und unbeweglich wie ein unterer Mühlstein. 17Wenn es auffährt,
schaudern selbst Helden\textless sup title=``oder:
Vorkämpfer''\textgreater✲, geraten vor Entsetzen außer sich. 18Trifft
man es mit dem Schwert -- das haftet ebensowenig wie Speer, Wurfspieß
und Pfeil. 19Eisen achtet es gleich Stroh, Erz gleich morschem Holz.
20Kein Pfeil des Bogens bringt es zum Fliehen; Schleudersteine
verwandeln sich ihm in Spreu. 21Wie ein Strohhalm kommt ihm die Keule
vor, und nur ein Lächeln hat es für den Anprall der Lanze. 22Seine
Unterseite bilden spitze Scherben; einen breiten Dreschschlitten drückt
es in den Schlamm ein. 23Es macht die tiefe Wasserflut wie einen
Kochtopf sieden, rührt das Meer\textless sup title=``d.h. den
Nil''\textgreater✲ auf wie einen Salbenkessel. 24Hinter ihm her leuchtet
sein Pfad: man könnte die Schaumflut für Silberhaar halten. 25Auf Erden
gibt es nicht seinesgleichen; es ist dazu geschaffen, sich nie zu
fürchten. 26Auf alles Hohe sieht es mit Verachtung hin: der König ist es
über alle stolzen Tiere.«

\hypertarget{d-hiobs-letzte-antwort-seine-anerkennung-der-gruxf6uxdfe-gottes-und-sein-buuxdffertiger-widerruf}{%
\paragraph{d) Hiobs letzte Antwort: seine Anerkennung der Größe Gottes
und sein bußfertiger
Widerruf}\label{d-hiobs-letzte-antwort-seine-anerkennung-der-gruxf6uxdfe-gottes-und-sein-buuxdffertiger-widerruf}}

\hypertarget{section-41}{%
\section{42}\label{section-41}}

1Da antwortete Hiob dem HERRN folgendermaßen:

2»Ich habe anerkannt, daß du alles vermagst und kein
Vorhaben\textless sup title=``oder: Plan''\textgreater✲ dir unausführbar
ist. 3{[}›Wer ist's, der da den Ratschluß Gottes verdunkelt ohne
Einsicht?‹\textless sup title=``vgl. 38,2''\textgreater✲{]} So habe ich
denn in Unverstand geurteilt über Dinge, die zu wunderbar für mich waren
und die ich nicht verstand. 4{[}›Höre doch und laß mich reden! Ich will
dich fragen, und du belehre mich!‹\textless sup title=``vgl.
40,7''\textgreater✲{]} 5Nur durch Hörensagen hatte ich von dir
vernommen, jetzt aber hat mein Auge dich geschaut. 6Darum bekenne ich
mich schuldig und bereue in Staub und Asche.«

\hypertarget{ix.-hiobs-rechtfertigung-durch-gott-wiederherstellung-seines-gluxfccksstandes-427-17}{%
\subsection{IX. Hiobs Rechtfertigung durch Gott; Wiederherstellung
seines Glücksstandes
(42,7-17)}\label{ix.-hiobs-rechtfertigung-durch-gott-wiederherstellung-seines-gluxfccksstandes-427-17}}

\hypertarget{a-gottes-verurteilung-der-drei-freunde-und-deren-begnadigung-nach-dargebrachtem-opfer-auf-hiobs-fuxfcrbitte}{%
\paragraph{a) Gottes Verurteilung der drei Freunde und deren Begnadigung
nach dargebrachtem Opfer auf Hiobs
Fürbitte}\label{a-gottes-verurteilung-der-drei-freunde-und-deren-begnadigung-nach-dargebrachtem-opfer-auf-hiobs-fuxfcrbitte}}

7Darauf, nachdem der HERR so zu Hiob gesprochen hatte, sagte der HERR zu
Eliphas von Theman: »Entbrannt ist mein Zorn gegen dich und gegen deine
beiden Freunde; denn ihr habt nicht richtig\textless sup title=``oder:
aufrichtig''\textgreater✲ von mir geredet wie mein Knecht Hiob. 8Darum
holt euch nun sieben junge Stiere und sieben Widder, begebt euch zu
meinem Knecht Hiob und bringt ein Brandopfer für euch dar! Mein Knecht
Hiob soll dann Fürbitte für euch einlegen; denn nur aus Rücksicht auf
ihn will ich euch eure Torheit nicht entgelten lassen, weil ihr nicht
richtig\textless sup title=``oder: aufrichtig''\textgreater✲ von mir
geredet habt wie mein Knecht Hiob.« 9Da gingen Eliphas von Theman,
Bildad von Suah und Zophar von Naama hin und taten, wie der HERR ihnen
geboten hatte; und der HERR nahm Rücksicht auf Hiob\textless sup
title=``d.h. nahm Hiobs Fürbitte an''\textgreater✲.

\hypertarget{b-wiederherstellung-und-mehrung-des-uxe4uuxdferen-gluxfccksstandes-hiobs}{%
\paragraph{b) Wiederherstellung und Mehrung des äußeren Glücksstandes
Hiobs}\label{b-wiederherstellung-und-mehrung-des-uxe4uuxdferen-gluxfccksstandes-hiobs}}

10Der HERR stellte dann Hiobs Glücksstand wieder her, als er Fürbitte
für seine Freunde eingelegt hatte; und der HERR vermehrte den ganzen
Besitz Hiobs so, daß er doppelt so groß war als früher. 11Da kamen alle
seine Brüder und Schwestern und alle seine früheren Bekannten zu ihm;
sie aßen mit ihm in seinem Hause, bezeigten ihm ihr Beileid und
trösteten ihn wegen all des Unglücks, mit dem der HERR ihn heimgesucht
hatte; auch schenkten sie ihm ein jeder ein wertvolles
Geldstück\textless sup title=``vgl. 1.Mose 33,19''\textgreater✲ und
jeder einen goldenen Ring. 12Der HERR aber segnete die nachfolgende
Lebenszeit Hiobs noch mehr als seine frühere, so daß er es auf 14000
Stück Kleinvieh, 6000 Kamele, 1000 Joch✲ Rinder und 1000 Eselinnen
brachte. 13Auch wurden ihm wieder sieben Söhne und drei Töchter geboren;
14die eine\textless sup title=``oder: die erste''\textgreater✲ nannte er
Jemima\textless sup title=``d.h. Täubchen''\textgreater✲, die andere
Kezia\textless sup title=``d.h. Kassia, Zimtduft''\textgreater✲, die
dritte Keren-Happuch\textless sup title=``d.h. Schminktöpfchen oder:
Augenweide''\textgreater✲; 15und man fand im ganzen Lande keine so
schönen Frauen wie die Töchter Hiobs; und ihr Vater gab ihnen ein
Erbteil unter ihren Brüdern.~-- 16Danach lebte Hiob noch
hundertundvierzig Jahre und sah seine Kinder und Kindeskinder, vier
Geschlechter; 17dann starb Hiob alt und lebenssatt.
