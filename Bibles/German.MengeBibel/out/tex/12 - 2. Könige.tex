\hypertarget{das-zweite-buch-der-kuxf6nige}{%
\section{DAS ZWEITE BUCH DER
KÖNIGE}\label{das-zweite-buch-der-kuxf6nige}}

\hypertarget{i.-fortsetzung-der-geschichte-beider-reiche-bis-zum-untergang-des-reiches-israel-kap.-1-17}{%
\subsection{I. Fortsetzung der Geschichte beider Reiche bis zum
Untergang des Reiches Israel (Kap.
1-17)}\label{i.-fortsetzung-der-geschichte-beider-reiche-bis-zum-untergang-des-reiches-israel-kap.-1-17}}

\hypertarget{ahasja-und-der-prophet-elia}{%
\subsubsection{1. Ahasja und der Prophet
Elia}\label{ahasja-und-der-prophet-elia}}

\hypertarget{a-dem-erkrankten-und-abguxf6ttischen-kuxf6nig-ahasja-kuxfcndigt-elia-den-tod-an}{%
\paragraph{a) Dem erkrankten und abgöttischen König Ahasja kündigt Elia
den Tod
an}\label{a-dem-erkrankten-und-abguxf6ttischen-kuxf6nig-ahasja-kuxfcndigt-elia-den-tod-an}}

\hypertarget{section}{%
\section{1}\label{section}}

1Nach dem Tode Ahabs fielen die Moabiter von Israel ab.~-- 2Als Ahasja
aber in Samaria durch das Gitterfenster in\textless sup title=``oder:
an''\textgreater✲ seinem Obergemach gefallen war und krank darniederlag,
sandte er Boten ab mit dem Auftrag: »Geht hin und befragt Baal-Sebub,
den Gott von Ekron, ob ich von dieser Krankheit genesen werde.« 3Der
Engel des HERRN aber gebot dem Elia aus Thisbe: »Mache dich auf, gehe
den Boten des Königs von Samaria entgegen und sage zu ihnen: ›Es gibt
wohl keinen Gott in Israel, daß ihr hingehen müßt, um bei Baal-Sebub,
dem Gott von Ekron, anzufragen?‹ 4Darum hat der HERR so gesprochen: ›Von
dem Lager, das du bestiegen hast, sollst du nicht wieder herunterkommen,
sondern sollst unfehlbar sterben!‹« Darauf ging Elia von dannen.

5Als nun die Boten zum König zurückkehrten, fragte er sie: »Warum kommt
ihr denn zurück?« 6Sie antworteten ihm: »Ein Mann ist uns
entgegengekommen, der zu uns sagte: ›Kehrt wieder zum König zurück, der
euch entsandt hat, und sagt zu ihm: So hat der HERR gesprochen: Es gibt
wohl keinen Gott in Israel, daß du hinsenden mußt, um bei Baal-Sebub,
dem Gott von Ekron, anzufragen? Darum sollst du von dem Lager, das du
bestiegen hast, nicht wieder herunterkommen, sondern sollst unfehlbar
sterben.‹« 7Da fragte er sie: »Wie sah der Mann aus, der euch
entgegengekommen ist und diese Worte an euch gerichtet hat?« 8Sie
antworteten ihm: »Der Mann trug ein haariges Fell als
Mantel\textless sup title=``oder: einen haarigen Mantel''\textgreater✲
und hatte sich einen ledernen Gürtel um die Hüften gebunden.« Da rief
Ahasja aus: »Das ist Elia aus Thisbe!«

\hypertarget{b-elia-und-die-drei-hauptleute}{%
\paragraph{b) Elia und die drei
Hauptleute}\label{b-elia-und-die-drei-hauptleute}}

9Hierauf schickte er einen Hauptmann über eine Fünfzigschaft mit seinen
fünfzig Leuten nach ihm aus. Als dieser zu ihm hinaufkam -- Elia saß
nämlich oben auf einem Berge --, sagte er zu ihm: »Mann Gottes, der
König befiehlt dir herabzukommen!« 10Aber Elia antwortete dem Hauptmann:
»Wenn ich denn ein Gottesmann bin, so falle Feuer vom Himmel herab und
verzehre dich samt deinen fünfzig Leuten!« Da fiel Feuer vom Himmel
herab und verzehrte ihn samt seinen fünfzig Leuten. 11Hierauf sandte der
König nochmals einen andern Hauptmann mit seinen fünfzig Leuten nach ihm
aus; der redete ihn mit den Worten an: »Mann Gottes, so hat der König
befohlen: ›Komm sofort herab!‹« 12Aber Elia antwortete ihnen: »Wenn ich
ein Gottesmann bin, so falle Feuer vom Himmel herab und verzehre dich
samt deinen fünfzig Leuten!« Da fiel Feuer Gottes vom Himmel herab und
verzehrte ihn samt seinen fünfzig Leuten. 13Hierauf sandte der König
nochmals einen dritten Hauptmann mit seinen fünfzig Leuten ab. Als
dieser dritte Hauptmann zu ihm hinaufkam, fiel er vor Elia auf die Knie
und flehte ihn an mit den Worten: »Mann Gottes, laß doch mein Leben und
das Leben deiner Knechte, dieser Fünfzig, etwas bei dir gelten! 14Du
weißt: Feuer ist vom Himmel herabgefallen und hat die beiden ersten
Hauptleute samt ihren fünfzig Leuten verzehrt; nun aber laß doch meinem
Leben Schonung widerfahren!« 15Da sagte der Engel des HERRN zu Elia:
»Gehe mit ihm hinab, fürchte dich nicht vor jenem!«

\hypertarget{c-elia-bei-ahasja-tod-des-kuxf6nigs}{%
\paragraph{c) Elia bei Ahasja; Tod des
Königs}\label{c-elia-bei-ahasja-tod-des-kuxf6nigs}}

Da stand Elia auf und ging mit ihm zum König hinab. 16Zu diesem sagte er
dann: »So hat der HERR gesprochen: ›Weil du Boten abgesandt hast, um
Baal-Sebub, den Gott von Ekron, zu befragen -- als ob es keinen Gott in
Israel gäbe, dessen Ausspruch man einholen könnte --, darum wirst du von
dem Lager, das du bestiegen hast, nicht wieder herunterkommen, sondern
wirst unfehlbar sterben!‹« 17So starb er denn, wie der HERR es durch den
Mund Elias hatte verkündigen lassen; und sein Bruder Joram folgte ihm in
der Regierung nach im zweiten Jahre der Regierung Jorams, des Sohnes
Josaphats, des Königs von Juda; denn er hatte keinen Sohn hinterlassen.
18Die übrige Geschichte Ahasjas aber und alles, was er unternommen hat,
das findet sich bekanntlich aufgezeichnet im Buch der
Denkwürdigkeiten\textless sup title=``oder: Chronik''\textgreater✲ der
Könige von Israel.

\hypertarget{die-himmelfahrt-elias}{%
\subsubsection{2. Die Himmelfahrt Elias}\label{die-himmelfahrt-elias}}

\hypertarget{a-elia-auf-der-wanderung-mit-seinem-treuen-diener-elisa}{%
\paragraph{a) Elia auf der Wanderung mit seinem treuen Diener
Elisa}\label{a-elia-auf-der-wanderung-mit-seinem-treuen-diener-elisa}}

\hypertarget{section-1}{%
\section{2}\label{section-1}}

1Als dann der HERR den Elia im Wettersturm zum Himmel auffahren lassen
wollte, ging Elia mit Elisa aus Gilgal weg. 2Da sagte Elia zu Elisa:
»Bleibe doch hier! Denn der HERR hat mich nach Bethel gesandt.« Doch
Elisa erwiderte: »So wahr der HERR lebt und so wahr du selbst lebst, ich
verlasse dich nicht!« Als sie nun nach Bethel hinabwanderten, 3kamen die
Prophetenjünger, die in Bethel wohnten, zu Elisa hinaus und sagten zu
ihm: »Weißt du wohl, daß Gott der HERR heute deinen Herrn über deinem
Haupt entführen wird?« Er antwortete: »Auch ich weiß es: schweigt
still!« 4Da sagte Elia zu ihm: »Elisa, bleibe doch hier! Denn der HERR
hat mich nach Jericho gesandt.« Doch er entgegnete: »So wahr der HERR
lebt und so wahr du selbst lebst, ich verlasse dich nicht!« Als sie nun
nach Jericho gekommen waren, 5traten die Prophetenjünger, die in Jericho
wohnten, an Elisa heran und sagten zu ihm: »Weißt du wohl, daß Gott der
HERR heute deinen Herrn über deinem Haupt entführen wird?« Er
antwortete: »Auch ich weiß es: schweigt still!« 6Darauf sagte Elia zu
ihm: »Bleibe doch hier! Denn der HERR hat mich an den Jordan gesandt.«
Doch er entgegnete: »So wahr der HERR lebt und so wahr du selbst lebst,
ich verlasse dich nicht!« So gingen sie denn beide miteinander weiter.
7Auch fünfzig Mann von den Prophetenjüngern gingen mit, blieben aber
abseits in einiger Entfernung stehen, während die beiden an den Jordan
traten. 8Da nahm Elia seinen Mantel\textless sup title=``vgl. 1.Kön
19,19''\textgreater✲, wickelte ihn zusammen und schlug damit auf das
Wasser; da zerteilte es sich nach beiden Seiten hin, so daß sie beide
trockenen Fußes hindurchgehen konnten.

\hypertarget{b-elias-abschied-von-elisa-seine-himmelfahrt}{%
\paragraph{b) Elias Abschied von Elisa; seine
Himmelfahrt}\label{b-elias-abschied-von-elisa-seine-himmelfahrt}}

9Als sie nun drüben angekommen waren, sagte Elia zu Elisa: »Erbitte dir
etwas, was ich dir tun soll, ehe ich von dir hinweggenommen werde.«
Elisa antwortete: »Möchte mir doch ein doppelter Anteil von deinem Geist
zufallen!«\textless sup title=``vgl. 5.Mose 21,17''\textgreater✲ 10Elia
entgegnete: »Du hast eine schwer zu erfüllende Bitte ausgesprochen. Wenn
du mit ansehen darfst, wie ich von dir entrückt werde, so wird deine
Bitte erfüllt werden, sonst nicht!« 11Während sie dann im Gespräch
miteinander immer weiter gingen, erschien plötzlich ein feuriger Wagen
mit feurigen Rossen und trennte beide voneinander; und Elia fuhr im
Wettersturm zum Himmel empor. 12Als Elisa das sah, rief er laut: »Mein
Vater, mein Vater! Du Wagen Israels und seine Reiter!«\textless sup
title=``vgl. 13,14''\textgreater✲ Als er ihn dann nicht mehr sah, faßte
er seine Kleider und zerriß sie in zwei Stücke.

\hypertarget{c-elisas-ruxfcckkehr-uxfcber-den-jordan-nach-jericho-elia-bleibt-verschwunden}{%
\paragraph{c) Elisas Rückkehr über den Jordan nach Jericho; Elia bleibt
verschwunden}\label{c-elisas-ruxfcckkehr-uxfcber-den-jordan-nach-jericho-elia-bleibt-verschwunden}}

13Hierauf hob er den Mantel auf, der dem Elia entfallen war, kehrte um
und trat an das Ufer des Jordans; 14hier nahm er den Mantel Elias, der
ihm entfallen war, schlug damit auf das Wasser und rief aus: »Wo ist der
HERR, der Gott Elias?« Als auch er so auf das Wasser geschlagen hatte,
zerteilte es sich nach beiden Seiten hin, so daß Elisa hindurchgehen
konnte. 15Als die Prophetenjünger, die aus Jericho waren, das von drüben
her sahen, riefen sie aus: »Der Geist Elias ruht auf Elisa!« Sie gingen
ihm dann entgegen, verneigten sich vor ihm bis zur Erde 16und sagten zu
ihm: »Siehe, es befinden sich hier bei deinen Knechten fünfzig rüstige
Männer; die sollen hingehen und deinen Herrn suchen, ob ihn nicht etwa
der Geist des HERRN entführt und ihn auf irgendeinen Berg oder in
irgendein Tal entrückt hat.« 17Aber er erwiderte: »Ihr braucht sie nicht
auszuschicken!« Als sie jedoch bis aufs äußerste in ihn drangen,
erklärte er schließlich: »Nun, so laßt sie abgehen!« Da schickten sie
die fünfzig Männer aus, die suchten drei Tage lang, fanden ihn aber
nicht. 18Als sie dann zu ihm zurückkehrten, während er noch in Jericho
weilte, sagte er zu ihnen: »Habe ich euch nicht gesagt, ihr möchtet
nicht hingehen?«

\hypertarget{der-prophet-elisa-219-815}{%
\subsubsection{3. Der Prophet Elisa
(2,19-8,15)}\label{der-prophet-elisa-219-815}}

\hypertarget{a-erstes-auftreten-elisas-das-wunder-am-ungesunden-wasser-zu-jericho}{%
\paragraph{a) Erstes Auftreten Elisas: Das Wunder am ungesunden Wasser
zu
Jericho}\label{a-erstes-auftreten-elisas-das-wunder-am-ungesunden-wasser-zu-jericho}}

19Die Einwohner der Stadt (Jericho) aber sagten zu Elisa: »In unserer
Stadt ist gewiß gut wohnen, wie du, Herr, selbst siehst; aber das Wasser
ist ungesund, und die Gegend verursacht Fehlgeburten.« 20Da antwortete
er: »Bringt mir eine neue Schüssel und tut Salz hinein.« Als man sie ihm
gebracht hatte, 21ging er an die Wasserquelle vor die Stadt hinaus, warf
das Salz hinein und sagte: »So hat der HERR gesprochen: ›Ich habe dieses
Wasser gesund gemacht; es soll hinfort weder Tod noch Fehlgeburt daher
kommen!‹« 22Da wurde das Wasser gesund bis auf den heutigen Tag infolge
des Wortes, das Elisa ausgesprochen hatte.

\hypertarget{elisa-und-die-buxf6sen-buben-von-bethel}{%
\paragraph{Elisa und die bösen Buben von
Bethel}\label{elisa-und-die-buxf6sen-buben-von-bethel}}

23Von dort ging er dann nach Bethel zurück; und als er so auf dem Wege
hinaufging, kamen kleine Jungen aus dem Ort heraus, die ihn
verspotteten, indem sie ihm zuriefen: »Komm herauf, Kahlkopf! Komm
herauf, Kahlkopf!« 24Da wandte er sich um, und als er sie sah,
verwünschte er sie im Namen des HERRN. Da kamen zwei Bären aus dem Walde
heraus und zerrissen zweiundvierzig von den Knaben. 25Von dort begab er
sich nach dem Berge Karmel und kehrte von da nach Samaria zurück.

\hypertarget{b-kuxf6nig-joram-von-israel}{%
\paragraph{b) König Joram von
Israel}\label{b-kuxf6nig-joram-von-israel}}

\hypertarget{aa-allgemeine-angaben-uxfcber-joram}{%
\subparagraph{aa) Allgemeine Angaben über
Joram}\label{aa-allgemeine-angaben-uxfcber-joram}}

\hypertarget{section-2}{%
\section{3}\label{section-2}}

1Joram, der Sohn Ahabs, wurde König über Israel zu Samaria im
achtzehnten Jahre der Regierung des Königs Josaphat von Juda und
regierte zwölf Jahre. 2Er tat, was dem HERRN mißfiel, jedoch nicht so
schlimm wie sein Vater und seine Mutter; denn er entfernte die
Götzensäule Baals, die sein Vater hatte aufstellen lassen\textless sup
title=``vgl. 1.Kön 16,32-33''\textgreater✲, 3aber an dem sündhaften
Stierdienst Jerobeams, des Sohnes Nebats, zu dem er Israel verführt
hatte, hielt er fest und ließ nicht davon ab.

\hypertarget{bb-ausbruch-des-kriegs-mit-den-moabitern-buxfcndnis-jorams-mit-josaphat-marsch-in-die-steppe-von-edom}{%
\subparagraph{bb) Ausbruch des Kriegs mit den Moabitern; Bündnis Jorams
mit Josaphat; Marsch in die Steppe von
Edom}\label{bb-ausbruch-des-kriegs-mit-den-moabitern-buxfcndnis-jorams-mit-josaphat-marsch-in-die-steppe-von-edom}}

4Da Mesa, der König der Moabiter, Schafzüchter war, hatte er dem König
von Israel hunderttausend Lämmer und die Wolle von hunderttausend
Widdern als regelmäßige Abgabe zu entrichten. 5Aber nach Ahabs Tode fiel
der König der Moabiter vom König von Israel ab. 6Da zog der König Joram
zu jener Zeit aus Samaria aus und bot ganz Israel zum Kriege auf;
7gleichzeitig schickte er eine Gesandtschaft an den König Josaphat von
Juda und ließ ihm sagen: »Der König der Moabiter ist von mir abgefallen:
willst du nicht mit mir gegen die Moabiter zu Felde ziehen?« Er
antwortete: »Ja, ich will mitziehen; ich will sein wie du: mein Volk wie
dein Volk, meine Rosse wie deine Rosse!«\textless sup title=``vgl. 1.Kön
22,4''\textgreater✲ 8Als er dann fragte: »Welchen Weg wollen wir
einschlagen?«, erwiderte jener: »Den Weg durch die Steppe von Edom.«

\hypertarget{cc-schlimme-lage-des-heeres-infolge-von-wassermangel-elisas-gluxfcckverkuxfcndende-weissagung}{%
\subparagraph{cc) Schlimme Lage des Heeres infolge von Wassermangel;
Elisas glückverkündende
Weissagung}\label{cc-schlimme-lage-des-heeres-infolge-von-wassermangel-elisas-gluxfcckverkuxfcndende-weissagung}}

9So zog denn der König von Israel mit dem König von Juda und dem König
der Edomiter aus. Als sie aber sieben Tagemärsche zur Umgehung
zurückgelegt hatten, fehlte es dem Heere und dem Vieh, das mit ihnen
zog, an Wasser. 10Da rief der König von Israel aus: »Wehe! So hat also
der HERR diese drei Könige zum Kriege aufgeboten, um sie in die Hand der
Moabiter fallen zu lassen!« 11Nun fragte Josaphat: »Ist denn hier kein
Prophet des HERRN, durch den wir den HERRN befragen können?« Da
antwortete einer von den Hofleuten des Königs von Israel: »Hier ist
Elisa, der Sohn Saphats, der als Diener bei Elia gelebt hat.« 12Josaphat
sagte: »Ja wirklich, bei dem ist das Wort des HERRN zu finden!«

Als sich nun der König von Israel und Josaphat und der König der
Edomiter zu ihm hinabbegeben hatten, 13sagte Elisa zum König von Israel:
»Was habe ich mit dir zu schaffen? Wende dich doch an die Propheten
deines Vaters und an die Propheten deiner Mutter!« Aber der König von
Israel entgegnete ihm: »Nicht doch! Hat etwa der HERR diese drei Könige
zum Krieg aufgeboten, um sie in die Hand der Moabiter fallen zu lassen?«
14Da sagte Elisa: »So wahr der HERR der Heerscharen lebt, in dessen
Dienst ich stehe! Wenn ich nicht auf den König Josaphat von Juda
Rücksicht nähme, so würde ich dich wahrlich nicht beachten und dich
keines Blickes würdigen! 15Nun aber schafft mir einen Saitenspieler
her!« Als dann der Saitenspieler die Saiten rührte, kam die Hand des
HERRN über ihn, 16und er sagte: »So hat der HERR gesprochen: ›Macht in
diesem Tal Grube an Grube!‹ 17Denn so hat der HERR gesprochen: ›Ihr
werdet keinen Wind wahrnehmen und keinen Regen sehen, und doch wird sich
dieses Tal mit Wasser füllen, so daß ihr samt eurem Heer und eurem Vieh
trinken könnt. 18Aber dies genügt dem HERRN noch nicht: er wird euch
auch noch die Moabiter in die Hände liefern, 19so daß ihr alle festen
Städte erobern, alle Fruchtbäume fällen, alle Wasserquellen verschütten
und alles gute Ackerland mit Steinen verderben werdet.‹« 20Und wirklich,
am folgenden Morgen zu der Zeit, wo man das Speisopfer darbringt, kam
plötzlich Wasser von Edom her geflossen, so daß die ganze Gegend
überschwemmt wurde.

\hypertarget{dd-sieg-der-israeliten-mesa-opfert-seinen-erstgeborenen-sohn-und-bewirkt-dadurch-den-abzug-der-israeliten}{%
\subparagraph{dd) Sieg der Israeliten; Mesa opfert seinen erstgeborenen
Sohn und bewirkt dadurch den Abzug der
Israeliten}\label{dd-sieg-der-israeliten-mesa-opfert-seinen-erstgeborenen-sohn-und-bewirkt-dadurch-den-abzug-der-israeliten}}

21Als nun das ganze Volk der Moabiter hörte, daß die Könige herangezogen
waren, um sie zu bekriegen, wurde alles, was die Waffen tragen konnte,
aufgeboten, ja sogar die noch nicht Waffenfähigen, und sie stellten sich
an der Grenze auf. 22Als aber am folgenden Morgen früh die Sonne beim
Aufgang über das Wasser hin strahlte, erschien den Moabitern das Wasser
drüben rot wie Blut, 23so daß sie ausriefen: »Das ist Blut! Gewiß sind
die Könige mit dem Schwert aneinandergeraten und haben ein Blutbad unter
sich angerichtet: jetzt an die Beute, Moabiter!« 24Als sie aber an das
israelitische Lager herankamen, machten die Israeliten einen Ausfall und
schlugen die Moabiter in die Flucht, drangen dann immer weiter ins Land
ein und schlugen die Moabiter aufs neue. 25Die Städte zerstörten sie,
auf alles gute Ackerland warfen sie ein jeder seinen Stein, so daß es
ganz damit bedeckt war, alle Wasserquellen verschütteten sie und hieben
alle Fruchtbäume um, bis nichts mehr übrig war als (die Stadt)
Kir-Hareseth mit ihrer festen Steinmauer. Als dann die Schleuderer die
Stadt umzingelten und beschossen 26und der König der Moabiter einsah,
daß er dem Angriff nicht gewachsen sei, nahm er siebenhundert mit
Schwertern bewaffnete Krieger mit sich, um sich zum König der Edomiter
durchzuschlagen; aber es gelang ihnen nicht. 27Da nahm er seinen
erstgeborenen Sohn, der ihm dereinst in der Regierung nachfolgen sollte,
und brachte ihn auf der Mauer als Brandopfer dar. Da kam ein gewaltiger
Zorn über Israel, so daß sie die Belagerung aufhoben und in ihr Land
zurückkehrten.

\hypertarget{c-wundertaten-elisas-41-623}{%
\paragraph{c) Wundertaten Elisas
(4,1-6,23)}\label{c-wundertaten-elisas-41-623}}

\hypertarget{aa-die-geschichte-vom-uxf6lkrug-der-witwe}{%
\subparagraph{aa) Die Geschichte vom Ölkrug der
Witwe}\label{aa-die-geschichte-vom-uxf6lkrug-der-witwe}}

\hypertarget{section-3}{%
\section{4}\label{section-3}}

1Eine Frau von den Ehefrauen der Prophetenjünger flehte einst Elisa laut
mit den Worten an: »Mein Mann, dein Knecht, ist gestorben, und du weißt
selbst, daß dein Knecht ein gottesfürchtiger Mann gewesen ist. Nun ist
der Gläubiger gekommen und will sich meine beiden Söhne zu Sklaven
nehmen!« 2Elisa antwortete ihr: »Was soll ich für dich tun? Sage mir,
was du im Hause hast!« Sie erwiderte: »Deine Magd hat gar nichts mehr im
Hause als nur einen Krug mit etwas Öl.« 3Da sagte er: »Gehe hin, borge
dir Gefäße von allen deinen Nachbarn draußen, leere Gefäße, aber nimm
nicht zu wenige; 4hierauf gehe heim, schließe die Tür hinter dir und
deinen beiden Söhnen zu und gieße in alle jene Gefäße ein; und wenn eins
voll ist, so setze es beiseite.«« 5Sie ging dann von ihm weg und schloß
die Tür hinter sich und ihren Söhnen zu; diese reichten ihr (die
Gefäße), und sie goß sie voll. 6Als nun die Gefäße gefüllt waren, sagte
sie zu ihrem Sohn: »Reiche mir noch ein Gefäß!«, aber er antwortete ihr:
»Es ist kein Gefäß mehr da«; da hörte das Öl auf zu fließen. 7Als sie
nun zu dem Gottesmann kam und es ihm berichtete, sagte er: »Gehe hin,
verkaufe das Öl und bezahle deine Schuld; von dem, was dir dann noch
übrigbleibt, kannst du mit deinen Söhnen leben.«

\hypertarget{bb-elisa-und-die-sunamitin-elisa-verheiuxdft-der-sunamitin-einen-sohn}{%
\subparagraph{bb) Elisa und die Sunamitin; Elisa verheißt der Sunamitin
einen
Sohn}\label{bb-elisa-und-die-sunamitin-elisa-verheiuxdft-der-sunamitin-einen-sohn}}

8Eines Tages ging Elisa nach Sunem\textless sup title=``1.Sam
28,4''\textgreater✲ hinüber; dort wohnte eine reiche Frau, die ihn
nötigte, bei ihr zu essen. Sooft er nun später an dem Ort vorüberkam,
kehrte er dort zum Essen ein. 9Da sagte sie zu ihrem Manne: »Sieh doch,
ich habe erkannt, daß dieser ein heiliger Gottesmann ist, der immer bei
uns einkehrt; 10wir wollen ihm doch ein kleines Zimmer oben im Hause
aufmauern lassen und ihm ein Bett, einen Tisch, einen Stuhl und einen
Leuchter hineinstellen; dann kann er dort ein Unterkommen finden, sooft
er zu uns kommt.« 11Als er nun eines Tages wieder hinkam, kehrte er in
dem Oberstübchen ein und schlief darin. 12Nachher befahl er seinem
Diener Gehasi: »Rufe mir unsere Sunamitin!« Als er sie nun gerufen hatte
und sie vor ihn getreten war, 13sagte Elisa zu dem Diener: »Sage ihr:
›Du hast dir unsertwegen alle diese Unruhe gemacht: was kann man für
dich tun? Brauchst du Fürsprache beim König oder Feldhauptmann?‹« Sie
antwortete: »Ich wohne hier ja sicher inmitten meines Volkes✲.« 14Als er
nun wieder fragte: »Was könnte man wohl für sie tun?«, antwortete
Gehasi: »Ach, sie ist kinderlos, und ihr Mann ist schon alt.« 15Darauf
sagte Elisa: »Rufe sie her!« Als er sie nun gerufen hatte und sie in die
Tür getreten war, 16sagte er: »Übers Jahr um diese Zeit wirst du einen
Sohn herzen!« Aber sie entgegnete: »Ach nein, mein Herr, du Mann Gottes:
täusche doch deine Magd nicht!« 17Die Frau aber wurde wirklich guter
Hoffnung und gebar um dieselbe Zeit im nächsten Jahre einen Sohn, wie
Elisa ihr verheißen hatte.

\hypertarget{der-tod-des-knaben-wanderung-der-mutter-zu-elisa}{%
\paragraph{Der Tod des Knaben; Wanderung der Mutter zu
Elisa}\label{der-tod-des-knaben-wanderung-der-mutter-zu-elisa}}

18Als nun der Knabe herangewachsen war, begab es sich eines Tages, daß
er zu seinem Vater zu den Schnittern hinausging. 19Da klagte er
(plötzlich) seinem Vater: »Mein Kopf, mein Kopf!« Jener befahl einem
Knecht: »Trage ihn heim zu seiner Mutter!« 20Als dieser ihn auf den Arm
genommen und zu seiner Mutter gebracht hatte, saß er bis zum Mittag auf
ihrem Schoß; dann starb er. 21Da stieg sie hinauf, legte ihn auf das
Bett des Gottesmannes, schloß hinter ihm zu und ging hinaus; 22dann ließ
sie ihren Mann rufen und sagte zu ihm: »Schicke mir doch einen von den
Knechten und eine Eselin; ich will zu dem Gottesmann eilen, komme aber
schnell wieder zurück.« 23Er entgegnete: »Warum willst du gerade heute
zu ihm gehen? Es ist doch weder Neumond noch Sabbat!« Doch sie
erwiderte: »Das schadet nichts.« 24Hierauf ließ sie die Eselin satteln
und befahl ihrem Knecht: »Treibe das Tier immerfort an und mache mir
keinen Aufenthalt beim Reiten, es sei denn, daß ich es dir sage!« 25So
machte sie sich auf den Weg und gelangte zu dem Gottesmann auf den Berg
Karmel.

\hypertarget{elisa-begibt-sich-in-das-haus-der-mutter}{%
\paragraph{Elisa begibt sich in das Haus der
Mutter}\label{elisa-begibt-sich-in-das-haus-der-mutter}}

Als nun der Gottesmann sie in einiger Entfernung erblickte, sagte er zu
seinem Diener Gehasi: »Da ist ja unsere Sunamitin\textless sup
title=``vgl. 1.Sam 28,4''\textgreater✲! 26Wohlan, laufe ihr entgegen und
frage sie, ob es ihr sowie ihrem Mann und dem Knaben gutgehe.« Sie
antwortete: »Ja.« 27Als sie aber zu dem Gottesmann auf den Berg gekommen
war, umfaßte sie seine Füße; da trat Gehasi hinzu, um sie wegzustoßen;
aber der Gottesmann sagte: »Laß sie! Denn sie ist tief betrübt, und Gott
hat es mir verborgen und mir's nicht geoffenbart.« 28Sie sagte dann:
»Bin ich es gewesen, die meinen Herrn um einen Sohn gebeten hat? Habe
ich nicht vielmehr gesagt, du möchtest mich nicht täuschen?« 29Da befahl
er Gehasi: »Gürte dir die Lenden, nimm meinen Stab in deine Hand und
gehe hin! Wenn du jemand triffst, so grüße ihn nicht, und wenn dich
jemand grüßt, so danke ihm nicht! Lege dann dem Knaben meinen Stab auf
das Gesicht!« 30Aber die Mutter des Knaben rief: »So wahr der HERR lebt
und so wahr du selbst lebst: ich lasse nicht von dir!« Da machte er sich
auf und folgte ihr.

31Gehasi war ihnen unterdessen vorausgeeilt und hatte dem Knaben den
Stab auf das Gesicht gelegt, aber kein Laut und kein Lebenszeichen war
erfolgt. Da kehrte er um, (seinem Herrn) entgegen, und berichtete ihm,
der Knabe sei nicht aufgewacht; 32und als Elisa dann in das Haus kam,
fand er den Knaben tot auf seinem eigenen Bette liegen.

\hypertarget{wiederbelebung-des-knaben}{%
\paragraph{Wiederbelebung des Knaben}\label{wiederbelebung-des-knaben}}

33Nun ging er hinein, schloß die Tür hinter sich zu und betete zum
HERRN; 34dann stieg er auf das Bett, streckte sich über den Knaben hin
und legte seinen Mund auf dessen Mund, seine Augen auf dessen Augen und
seine Hände auf die Hände jenes. Als er sich so über ihn hinstreckte,
erwärmte sich der Leib des Knaben. 35Dann stand er wieder auf und ging
im Zimmer hin und her, stieg dann wieder hinauf und streckte sich über
ihn hin. Da nieste der Knabe siebenmal und schlug die Augen hell auf.
36Nun rief er Gehasi und befahl ihm: »Rufe unsere Sunamitin!« Der rief
sie herbei, und als sie zu ihm hereinkam, sagte er: »Nimm da deinen
Sohn!« 37Da trat sie heran, fiel ihm zu Füßen, verneigte sich tief bis
zur Erde, nahm ihren Sohn auf den Arm und ging hinaus\textless sup
title=``vgl. noch 8,1-6''\textgreater✲.

\hypertarget{cc-der-tod-giftige-speise-im-topf-und-die-wunderbare-speisung-der-hundert}{%
\subparagraph{cc) Der Tod (=~giftige Speise) im Topf und die wunderbare
Speisung der
Hundert}\label{cc-der-tod-giftige-speise-im-topf-und-die-wunderbare-speisung-der-hundert}}

38Elisa kehrte dann nach Gilgal zurück, während eine Hungersnot im Lande
herrschte. Als nun die Prophetenjünger vor ihm saßen, gab er seinem
Diener den Auftrag, den größten Kochtopf aufs Feuer zu setzen und ein
Gericht für die Prophetenjünger zu kochen. 39Da ging einer von ihnen auf
das Feld hinaus, um Kräuter zu sammeln, und als er ein wildes
Schlinggewächs fand, pflückte er davon wilde Gurken ab, seinen ganzen
Mantel voll; dann kehrte er heim und zerschnitt sie in den Kochtopf;
denn er kannte sie nicht. 40Als man sie dann zum Essen für die Männer
ausgeschüttet hatte und diese von dem Gericht aßen, schrien sie laut auf
und riefen: »Der Tod ist im Topf, Mann Gottes!«, und sie konnten es
nicht essen. 41Da sagte er: »So bringt Mehl her!« Er warf es in den Topf
und sagte dann: »Fülle es jetzt für die Leute aus, damit sie es essen.«
Da war nichts Schädliches mehr im Topf.

42Hierauf kam ein Mann aus Baal-Salisa und brachte dem Gottesmann
Erstlingsbrote, nämlich zwanzig Gerstenbrote, dazu Schrotkorn in seinem
Sack. Da befahl er: »Gib es den Leuten zu essen!« 43Sein Diener aber
entgegnete: »Wie kann ich dies hundert Männern vorsetzen?« Doch er
befahl: »Gib es den Leuten zu essen! Denn so hat der HERR gesprochen:
›Man wird essen und noch übriglassen‹«. 44Als er es ihnen nun vorsetzte,
aßen sie und ließen noch davon übrig, wie der HERR es verheißen hatte.

\hypertarget{dd-heilung-des-syrers-naeman-vom-aussatz-bestrafung-der-untreue-gehasis}{%
\subparagraph{dd) Heilung des Syrers Naeman vom Aussatz; Bestrafung der
Untreue
Gehasis}\label{dd-heilung-des-syrers-naeman-vom-aussatz-bestrafung-der-untreue-gehasis}}

\hypertarget{naeman-sucht-heilung-in-samaria}{%
\paragraph{Naeman sucht Heilung in
Samaria}\label{naeman-sucht-heilung-in-samaria}}

\hypertarget{section-4}{%
\section{5}\label{section-4}}

1Naeman, der Feldhauptmann des Königs von Syrien, galt bei seinem Herrn
viel und stand in hohem Ansehen; denn durch ihn hatte Gott der HERR den
Syrern den Sieg verliehen; aber dieser Mann, ein großer Kriegsheld,
wurde aussätzig. 2Nun hatten die Syrer einst auf einem Streifzuge ein
junges Mädchen aus dem Lande Israel gefangen weggeführt; die war dann
bei Naemans Gattin Dienerin geworden 3und sagte (eines Tages) zu ihrer
Herrin: »Ach wenn mein Herr sich doch an den Propheten zu Samaria
wendete! Dann würde der ihn von seinem Aussatz befreien.« 4Da ging
Naeman zu seinem Herrn und teilte ihm mit: »So und so hat das Mädchen
berichtet, das aus dem Lande Israel stammt.« 5Darauf entgegnete der
König von Syrien: »Nun gut, ziehe hin! Ich will dir ein Schreiben an den
König von Israel mitgeben.« Da machte er sich auf den Weg, nahm zehn
Talente Silber, sechstausend Schekel Gold und zehn Festgewänder mit 6und
überreichte dem König von Israel das Schreiben, das so lautete: »Wenn
dieses Schreiben an dich gelangt, so wisse: ich habe meinen Diener
Naeman zu dir gesandt, damit du ihn von seinem Aussatz befreist.« 7Als
der König von Israel das Schreiben gelesen hatte, zerriß er seine
Kleider und rief aus: »Bin ich etwa ein Gott, daß ich töten und lebendig
machen kann?! Dieser verlangt ja von mir, daß ich einen Menschen von
seinem Aussatz befreie! Da seht ihr nun deutlich, daß er nur einen
Vorwand zum Streit mit mir sucht!«

\hypertarget{naemans-heilung-durch-elisa}{%
\paragraph{Naemans Heilung durch
Elisa}\label{naemans-heilung-durch-elisa}}

8Als nun der Gottesmann Elisa erfuhr, daß der König von Israel seine
Kleider zerrissen habe, sandte er zum König und ließ ihm sagen: »Warum
hast du deine Kleider zerrissen? Laß ihn doch zu mir kommen: er soll
erfahren, daß es wirklich noch einen Propheten in Israel gibt!« 9So kam
denn Naeman mit seinen Rossen und seinem Wagen und hielt bei Elisa vor
der Haustür an. 10Da ließ ihm Elisa durch einen Boten sagen: »Gehe hin
und bade dich siebenmal im Jordan, dann wird dir dein Leib wieder gesund
werden, und du wirst rein sein.« 11Darüber wurde Naeman unwillig und
fuhr auf seinem Wagen weg mit den Worten: »Ich hatte als sicher
angenommen, er würde selbst zu mir herauskommen und vor mich hintreten
und den Namen des HERRN, seines Gottes, anrufen und seine Hand nach der
heiligen Stätte hin\textless sup title=``oder: über die kranke
Stelle''\textgreater✲ schwingen und so den Aussatz wegschaffen. 12Sind
nicht der Amana und der Pharphar, die Flüsse von Damaskus, besser als
alle Wasser in Israel? Kann ich mich nicht in ihnen baden, um rein zu
werden?« Damit wandte er sich um und entfernte sich voller Zorn. 13Da
traten seine Diener an ihn heran und redeten ihm mit den Worten zu:
»Mein Vater, wenn der Prophet etwas Schwieriges von dir verlangt hätte,
so hättest du es sicherlich getan; wieviel mehr also jetzt, da er nur zu
dir gesagt hat: ›Bade dich, so wirst du rein sein!‹« 14Als er sich nun
an den Jordan hatte hinabfahren lassen und sich nach der Weisung des
Gottesmannes siebenmal darin untergetaucht hatte, wurde sein Leib wieder
so rein wie der Leib eines kleinen Kindes.

\hypertarget{naemans-danksagung-und-lobpreis-gottes}{%
\paragraph{Naemans Danksagung und Lobpreis
Gottes}\label{naemans-danksagung-und-lobpreis-gottes}}

15Er kehrte nun mit seinem ganzen Gefolge zu dem Gottesmann zurück, trat
nach seiner Ankunft vor ihn hin und sagte: »Wisse wohl: jetzt habe ich
erkannt, daß es auf der ganzen Erde keinen Gott gibt als nur in Israel.
Nimm nun doch ein Geschenk von deinem Diener an!« 16Doch Elisa
entgegnete: »So wahr der HERR lebt, in dessen Dienst ich stehe: ich
nehme nichts an!« Und wie er ihn auch zur Annahme drängte, er blieb doch
bei seiner Weigerung. 17Da sagte Naeman: »Wenn denn nicht, so möge doch
deinem Diener wenigstens eine Last Erde, soviel ein Paar Maultiere
tragen können, mitgegeben werden; denn dein Diener wird fortan keinem
andern Gott Brand- und Schlachtopfer darbringen als dem HERRN allein.
18Nur in diesem einen Stück wolle der HERR mit deinem Diener Nachsicht
haben: Wenn mein königlicher Herr in den Tempel Rimmons geht, um
daselbst anzubeten, und sich dabei auf meinen Arm stützt und sich im
Tempel Rimmons niederwirft und ich mich dann ebenfalls im Tempel Rimmons
niederwerfe, so möge Gott der HERR in diesem einen Fall deinem Diener
Verzeihung zuteil werden lassen!« 19Er erwiderte ihm: »Ziehe hin in
Frieden!«

\hypertarget{gehasis-habsucht-bestraft}{%
\paragraph{Gehasis Habsucht bestraft}\label{gehasis-habsucht-bestraft}}

20Als (Naeman) aber eine Strecke Weges von ihm weggezogen war, dachte
Gehasi, der Diener des Gottesmannes Elisa: »Da hat nun mein Herr
wahrhaftig diesen Syrer Naeman geschont, statt etwas von dem anzunehmen,
was jener mitgebracht hatte! So wahr der HERR lebt: ich laufe hinter ihm
her und lasse mir etwas von ihm geben!« 21So eilte denn Gehasi dem
Naeman nach. Als dieser nun sah, daß einer hinter ihm herlief, sprang er
vom Wagen herab, ging ihm entgegen und fragte: »Geht es dir wohl?« 22Er
antwortete: »Ja! Mein Herr schickt mich und läßt dir sagen: ›Jetzt eben
sind vom Gebirge Ephraim zwei junge Leute von den Prophetenjüngern zu
mir gekommen; gib mir doch für sie ein Talent Silber und zwei
Festkleider!‹« 23Naeman erwiderte: »Tu mir den Gefallen und nimm zwei
Talente!« Er bat ihn dann dringend und ließ zwei Talente Silber in zwei
Beutel schnüren, tat dazu zwei Festkleider und ließ sie durch zwei
seiner Diener vor ihm her tragen. 24Als er aber bei dem Hügel ankam,
nahm er sie ihnen ab, brachte sie im Hause unter und entließ dann die
Leute, die nun zurückkehrten. 25Als er aber hineingegangen und vor
seinen Herrn getreten war, fragte ihn Elisa: »Woher kommst du, Gehasi?«
Er antwortete: »Dein Knecht ist überhaupt nicht ausgegangen.« 26Da sagte
(Elisa) zu ihm: »Bin ich nicht im Geist mit dir gegangen, als sich
jemand von seinem Wagen aus nach dir umwandte? Ist es jetzt an der Zeit,
Geld und Kleider anzunehmen und Ölbaumgärten und Weinberge, Kleinvieh
und Rinder, Knechte und Mägde dafür (zu erwerben)? 27So soll denn der
Aussatz Naemans an dir und deinen Nachkommen ewig haften!« Da ging
(Gehasi) von ihm weg, vom Aussatz weiß wie Schnee.

\hypertarget{ee-das-schwimmende-eisen}{%
\subparagraph{ee) Das schwimmende
Eisen}\label{ee-das-schwimmende-eisen}}

\hypertarget{section-5}{%
\section{6}\label{section-5}}

1Die Prophetenjünger sagten einst zu Elisa: »Sieh doch, der Raum, in dem
wir hier beim Unterricht vor dir sitzen, ist zu eng für uns. 2Wir wollen
doch an den Jordan gehen und von dort ein jeder einen Balken holen,
damit wir uns hier einen Raum herrichten, wo wir wohnen\textless sup
title=``oder: sitzen''\textgreater✲ können.« Er antwortete: »Ja, geht
hin!« 3Da bat einer: »Sei doch so freundlich, deine Knechte zu
begleiten!« Er erwiderte: »Gut, ich will mitgehen.« 4So ging er denn mit
ihnen, und als sie an den Jordan gekommen waren, hieben sie dort Bäume
um. 5Da begab es sich, daß einem, der einen Stamm fällte, das eiserne
Beilblatt ins Wasser fiel; und er rief laut: »O weh, Herr! Und es ist
noch dazu entlehnt!« 6Der Mann Gottes aber fragte: »Wohin ist es
gefallen?« Als er ihm nun die Stelle gezeigt hatte, schnitt (Elisa) ein
Stück Holz zurecht, warf es dorthin und brachte dadurch das Eisen zum
Schwimmen. 7Dann forderte er ihn auf: »Hole es dir herauf!« Der faßte
mit der Hand zu und ergriff es.

\hypertarget{ff-ein-heer-der-syrer-infolge-von-elisas-gebet-mit-blindheit-geschlagen-und-irregefuxfchrt}{%
\subparagraph{ff) Ein Heer der Syrer infolge von Elisas Gebet mit
Blindheit geschlagen und
irregeführt}\label{ff-ein-heer-der-syrer-infolge-von-elisas-gebet-mit-blindheit-geschlagen-und-irregefuxfchrt}}

\hypertarget{der-mehrmals-verratene-hinterhalt}{%
\paragraph{Der mehrmals verratene
Hinterhalt}\label{der-mehrmals-verratene-hinterhalt}}

8Als einst der König von Syrien Krieg mit Israel führte, traf er mit
seinen Heerführern die Verabredung: »An dem und dem Ort soll mein Lager
stehen.« 9Da sandte der Gottesmann (Elisa) zum König von Israel und ließ
ihm sagen: »Hüte dich, an jenem Ort vorüberzuziehen; denn dort liegen
die Syrer im Hinterhalt!« 10Darauf sandte der König von Israel an den
Ort, den ihm der Gottesmann bezeichnet und vor dem er ihn gewarnt hatte,
und er nahm sich dort in acht; und das geschah mehr als einmal oder
zweimal. 11Da geriet der König von Syrien in Erregung über dieses
Vorkommnis, so daß er seine Heerführer berief und zu ihnen sagte: »Könnt
ihr mir nicht angeben, wer von den Unsrigen im Bunde mit dem König von
Israel steht?« 12Da antwortete einer von seinen Heerführern: »Nicht
doch, mein Herr und König! Sondern Elisa, der Prophet in Israel, teilt
dem König von Israel die Worte mit, die du in deinem Schlafgemach
redest.« 13Da befahl er: »Geht hin und bringt in Erfahrung, wo er sich
befindet: ich will dann hinsenden und ihn festnehmen lassen.«

\hypertarget{die-blendung-der-syrer}{%
\paragraph{Die Blendung der Syrer}\label{die-blendung-der-syrer}}

Als man ihm nun meldete, (Elisa) befinde sich in Dothan, 14sandte er
Reiter, Wagen und ein starkes Heer dorthin, die bei Nacht dort ankamen
und die Stadt umzingelten. 15Als nun der Diener des Gottesmannes am
Morgen früh aufstand und aus dem Hause hinaustrat, lag da ein Heer um
die Stadt herum mit Rossen und Wagen, so daß sein Bursche ihm zurief: »O
weh, Herr! Was sollen wir machen?« 16Er aber erwiderte: »Fürchte dich
nicht! Denn unsere Kriegsmacht ist stärker als die Macht jener.«
17Hierauf betete Elisa mit den Worten: »HERR, öffne ihm doch die Augen,
damit er sehe!« Da öffnete der HERR dem Diener die Augen, und als er
hinblickte, sah er, wie das Gebirge rings um Elisa her voll von feurigen
Rossen und Wagen war. 18Als nun (die Feinde) gegen ihn heranrückten,
betete Elisa zum HERRN mit den Worten: »Schlage doch diese Leute mit
Blindheit!« Da schlug er sie mit Blindheit, wie Elisa es gewünscht
hatte.

\hypertarget{das-syrische-heer-von-elisa-nach-samaria-gefuxfchrt-und-dort-freundlich-behandelt}{%
\paragraph{Das syrische Heer von Elisa nach Samaria geführt und dort
freundlich
behandelt}\label{das-syrische-heer-von-elisa-nach-samaria-gefuxfchrt-und-dort-freundlich-behandelt}}

19Elisa sagte dann zu ihnen: »Dies ist nicht der rechte Weg und dies
nicht die richtige Stadt; folgt mir, so will ich euch zu dem Manne
führen, den ihr sucht!« Darauf führte er sie nach Samaria. 20Sobald sie
aber in Samaria angekommen waren, betete Elisa: »HERR, öffne diesen
Leuten nun die Augen, damit sie sehen!« Da öffnete der HERR ihnen die
Augen, und sie sahen, daß sie sich mitten in Samaria befanden. 21Als nun
der König von Samaria sie erblickte, fragte er Elisa: »Mein Vater, soll
ich sie ohne Gnade niederhauen lassen?« 22Doch er antwortete: »Nein, das
darfst du nicht tun! Willst du denn Leute niederhauen lassen, die du
nicht mit deinem Schwert und deinem Bogen gefangengenommen hast? Setze
ihnen Speise und Trank vor; wenn sie dann gegessen und getrunken haben,
laß sie wieder zu ihrem Herrn ziehen.« 23Da ließ er ihnen ein großes
Mahl zurichten, und als sie gegessen und getrunken hatten, ließ er sie
zu ihrem Herrn heimziehen. Seitdem fielen keine Streifscharen der Syrer
mehr ins Land Israel ein.

\hypertarget{d-hungersnot-und-wohlfeile-zeit-in-samarien}{%
\paragraph{d) Hungersnot und wohlfeile Zeit in
Samarien}\label{d-hungersnot-und-wohlfeile-zeit-in-samarien}}

\hypertarget{aa-belagerung-samarias-und-hungersnot}{%
\subparagraph{aa) Belagerung Samarias und
Hungersnot}\label{aa-belagerung-samarias-und-hungersnot}}

24Später begab es sich, daß Benhadad\textless sup title=``vgl. 1.Kön
20,1''\textgreater✲, der König von Syrien, seine ganze Heeresmacht
zusammenzog, vor Samaria rückte und es belagerte. 25Da entstand eine
schreckliche Hungersnot in Samaria, und es kam während der Belagerung
dahin, daß ein Eselskopf achtzig Schekel Silber und ein halbes Liter
Taubenmist fünf Schekel Silber kostete. 26Als nun der König von Israel
einmal auf der Mauer einherging, rief ihm eine Frau laut die Worte zu:
»Hilf mir, mein Herr und König!« 27Aber er antwortete: »Wenn dir Gott
der HERR nicht hilft, wie sollte ich dir helfen? Etwa mit einer Gabe von
der Tenne oder von der Kelter?« 28Dann fuhr der König fort: »Was willst
du denn?« Da antwortete sie: »Diese Frau da hatte zu mir gesagt: ›Gib
deinen Sohn her, damit wir ihn heute essen; morgen wollen wir dann
meinen Sohn verzehren!‹ 29So haben wir denn meinen Sohn gekocht und
gegessen; als ich aber am folgenden Tage zu ihr sagte: ›Gib jetzt deinen
Sohn her, damit wir ihn verzehren!‹, da hatte sie ihren Sohn versteckt.«
30Als der König diese Worte der Frau hörte, zerriß er seine Kleider,
während er auf der Mauer einherging; und dabei nahm das Volk wahr, daß
er darunter ein härenes Trauergewand auf dem bloßen Leibe trug. 31Und er
rief aus: »Gott soll mich jetzt und künftig strafen, wenn heute der Kopf
Elisas, des Sohnes Saphats, auf seinen Schultern sitzen bleibt!«

\hypertarget{bb-elisas-gluxfccksverheiuxdfung-fuxfcr-die-stadt}{%
\subparagraph{bb) Elisas Glücksverheißung für die
Stadt}\label{bb-elisas-gluxfccksverheiuxdfung-fuxfcr-die-stadt}}

32Elisa aber saß unterdessen in seiner Wohnung, und die Ältesten waren
bei ihm versammelt. Da sandte (der König) einen Mann vor sich her; aber
ehe noch der Bote bei Elisa eintraf, hatte dieser zu den Ältesten
gesagt: »Wißt ihr wohl, daß dieser Mordgeselle hergesandt hat, um mir
den Kopf abschlagen zu lassen? Gebt wohl acht! Sobald der Bote kommt,
verschließt die Tür und stemmt euch mit der Tür gegen ihn! Ist nicht
schon der Schall der Schritte seines Herrn hinter ihm hörbar?« 33Während
er noch mit ihnen redete, trat auch schon der König bei ihm ein und
sagte: »Siehe, dieses Unglück ist von Gott verhängt: was soll ich da
noch ferner auf Gott hoffen?«

\hypertarget{section-6}{%
\section{7}\label{section-6}}

1Da sagte Elisa: »Hört das Wort des HERRN! So hat der HERR gesprochen:
›Morgen um diese Zeit wird ein Maß Feinmehl einen Schekel kosten und
zwei Maß Gerste auch einen Schekel im Tor\textless sup title=``=~auf dem
Markt''\textgreater✲ von Samaria!‹« 2Da antwortete der
Ritter\textless sup title=``=~Offizier oder: Adjutant''\textgreater✲,
auf dessen Arm der König sich stützte, dem Gottesmanne folgendermaßen:
»Selbst wenn Gott der HERR Fenster am Himmel aufmachte: wie könnte so
etwas möglich sein?« Elisa aber entgegnete: »Wisse wohl: du wirst es mit
eigenen Augen sehen, aber nicht davon essen.«

\hypertarget{cc-erlebnisse-der-vier-aussuxe4tzigen-im-syrischen-lager}{%
\subparagraph{cc) Erlebnisse der vier Aussätzigen im syrischen
Lager}\label{cc-erlebnisse-der-vier-aussuxe4tzigen-im-syrischen-lager}}

3Nun befanden sich vier aussätzige Männer außerhalb des Stadttores, die
sagten zueinander: »Wozu wollen wir hier bleiben, bis wir sterben? 4Wenn
wir uns vornehmen, in die Stadt zu gehen, so herrscht die Hungersnot in
der Stadt, und wir müssen dort sterben; bleiben wir aber hier, so müssen
wir auch sterben. Darum kommt, wir wollen auf das Lager der Syrer
losgehen! Lassen sie uns am Leben, so bleiben wir leben; töten sie uns
aber, nun, so sterben wir!« 5So machten sie sich denn in der
Abenddämmerung auf, um sich ins Lager der Syrer zu begeben. Als sie nun
an den Rand\textless sup title=``=~vorderen Eingang''\textgreater✲ des
syrischen Lagers kamen, war dort kein Mensch zu sehen. 6Gott der HERR
hatte nämlich das syrische Heer ein Getöse von Wagen und Rossen, das
Getöse einer großen Heeresmacht, hören lassen, so daß einer zum andern
sagte: »Gewiß hat der König von Israel die Könige der Hethiter und die
Könige von Ägypten gegen uns gedungen, daß sie uns überfallen sollen!«
7So hatten sie sich also noch in der Abenddämmerung aufgemacht und die
Flucht ergriffen, hatten ihre Zelte, ihre Pferde und Esel, kurz das
ganze Lager, wie es war, im Stich gelassen und waren davongelaufen, um
ihr Leben zu retten. 8Als nun jene Aussätzigen an den vorderen Eingang
des Lagers gekommen waren, gingen sie in ein Zelt, aßen und tranken,
nahmen Silber, Gold und Kleider daraus weg und vergruben es anderswo;
dann kehrten sie um und gingen in ein anderes Zelt, plünderten es aus
und vergruben den Raub. 9Darauf aber sagten sie zueinander: »Wir handeln
nicht recht! Der heutige Tag ist ein Tag guter Botschaft; schweigen wir
aber und warten wir, bis es morgen hell ist, so trifft uns eine
Verschuldung. Wir wollen also jetzt hingehen und es im königlichen
Palast melden!«

\hypertarget{meldung-der-aussuxe4tzigen-in-der-stadt-und-deren-wirkung}{%
\paragraph{Meldung der Aussätzigen in der Stadt und deren
Wirkung}\label{meldung-der-aussuxe4tzigen-in-der-stadt-und-deren-wirkung}}

10Sie machten sich also auf, riefen die Wache am Stadttor an und
meldeten dort: »Wir sind ins Lager der Syrer gekommen; aber da war kein
Mensch zu sehen und keine Menschenstimme zu hören, sondern nur die
Pferde und die Esel standen dort angebunden und die Zelte, wie sie
gewesen waren.« 11Da riefen die Torwächter es in die Stadt hinein, und
man ließ es drinnen im Palast des Königs melden. 12Da stand der König
noch in der Nacht auf und sagte zu seinen Dienern: »Ich will euch sagen,
was die Syrer gegen uns im Schilde führen! Weil sie wissen, daß wir
Hunger leiden, haben sie ihr Lager verlassen, um sich irgendwo in der
Gegend zu verstecken, indem sie denken: ›Wenn die aus der Stadt
herausgekommen sind, wollen wir sie lebendig gefangennehmen und dann in
die Stadt eindringen.‹« 13Da antwortete einer von seinen Dienern: »So
nehme man doch fünf\textless sup title=``=~einige wenige''\textgreater✲
von den übriggebliebenen Pferden, die hier noch übrig sind -- es wird
ihnen ja doch nur ergehen wie der ganzen Menge, die bereits dahin ist
--: die wollen wir ausschicken, um nachzusehen.« 14Da nahm man zwei
Gespanne Rosse\textless sup title=``oder: zwei
Berittene?''\textgreater✲, die schickte der König hinter dem syrischen
Heere her mit dem Befehl: »Geht hin und seht nach!« 15Als diese nun
hinter ihnen her bis an den Jordan zogen, stellte es sich heraus, daß
der ganze Weg mit Kleidern und Waffen bedeckt war, welche die Syrer auf
ihrer eiligen Flucht weggeworfen hatten.

\hypertarget{elisas-weissagung-geht-in-erfuxfcllung}{%
\paragraph{Elisas Weissagung geht in
Erfüllung}\label{elisas-weissagung-geht-in-erfuxfcllung}}

Als dann die Boten zurückgekehrt waren und dem König Bericht erstattet
hatten, 16zog das Volk aus der Stadt hinaus und plünderte das syrische
Lager; und nun kostete ein Maß Feinmehl einen Schekel und zwei Maß
Gerste auch einen Schekel, wie der HERR es angekündigt hatte. 17Der
König hatte aber dem Ritter\textless sup title=``vgl.
V.2''\textgreater✲, auf dessen Arm er sich stützte, die Aufsicht über
den Markt übertragen; dabei zertrat ihn das Volk auf dem Markt, so daß
er starb, wie der Gottesmann es vorausgesagt hatte, als der König zu ihm
ins Haus gekommen war. 18Als nämlich der Gottesmann zum Könige gesagt
hatte: »Zwei Maß Gerste werden morgen um diese Zeit auf dem Markt von
Samaria einen Schekel kosten und ein Maß Feinmehl auch einen Schekel«,
19da hatte der Ritter dem Gottesmann zur Antwort gegeben: »Selbst wenn
Gott, der HERR, Fenster am Himmel aufmachte: wie könnte so etwas möglich
sein?« (Elisa) aber hatte entgegnet: »Wisse wohl: du wirst es mit
eigenen Augen sehen, aber nicht davon essen.« 20Und so erging es ihm
jetzt wirklich: das Volk zertrat ihn auf dem Markt, so daß er den Tod
fand.

\hypertarget{e-elisa-und-die-sunamitin-fortsetzung-und-schluuxdf-der-erzuxe4hlung-von-48-37}{%
\paragraph{e) Elisa und die Sunamitin (Fortsetzung und Schluß der
Erzählung von
4,8-37)}\label{e-elisa-und-die-sunamitin-fortsetzung-und-schluuxdf-der-erzuxe4hlung-von-48-37}}

\hypertarget{section-7}{%
\section{8}\label{section-7}}

1Elisa hatte aber der Frau, deren Sohn er ins Leben zurückgerufen hatte,
den Rat gegeben: »Mache dich auf, wandere mit deiner Familie aus und
halte dich irgendwo in der Fremde auf; denn der HERR hat eine Hungersnot
verhängt, die sieben Jahre lang im Lande herrschen wird.« 2Da machte
sich die Frau auf und folgte der Aufforderung des Gottesmannes: sie
wanderte mit ihrer Familie aus und hielt sich sieben Jahre lang im Lande
der Philister auf. 3Als dann nach Ablauf der sieben Jahre die Frau aus
dem Philisterlande zurückgekehrt war, machte sie sich auf den Weg, um
den König wegen ihres Hauses und ihrer Felder um Hilfe anzurufen. 4Der
König aber besprach sich gerade mit Gehasi, dem Diener des Gottesmannes,
und forderte ihn auf, ihm alle die Wundertaten zu erzählen, die Elisa
verrichtet habe. 5Während er nun dem Könige eben erzählte, wie Elisa den
Toten lebendig gemacht hatte, da erschien die Frau, deren Sohn er ins
Leben zurückgerufen hatte, um den König wegen ihres Hauses und ihrer
Felder um Hilfe anzurufen. Da sagte Gehasi: »Mein Herr und König, dies
ist die Frau und dies ihr Sohn, den Elisa lebendig gemacht hat!« 6Da
erkundigte sich der König bei der Frau, und sie mußte ihm alles
erzählen. Darauf gab der König ihr einen Kammerherrn mit, dem er
auftrug: »Verschaffe ihr alles wieder, was ihr gehört, auch den gesamten
Ertrag der Felder von dem Tage ab, an dem sie das Land verlassen hat,
bis heute!«

\hypertarget{f-elisa-und-hasael-von-damaskus}{%
\paragraph{f) Elisa und Hasael von
Damaskus}\label{f-elisa-und-hasael-von-damaskus}}

\hypertarget{aa-elisa-in-damaskus-von-hasael-befragt-bezuxfcglich-des-erkrankten-kuxf6nigs-benhadad}{%
\subparagraph{aa) Elisa in Damaskus von Hasael befragt bezüglich des
erkrankten Königs
Benhadad}\label{aa-elisa-in-damaskus-von-hasael-befragt-bezuxfcglich-des-erkrankten-kuxf6nigs-benhadad}}

7Einst kam Elisa nach Damaskus, wo Benhadad, der König von Syrien, krank
lag. Als man diesem nun mitteilte, daß der Gottesmann dorthin komme,
8befahl der König dem Hasael: »Nimm Geschenke mit dir und gehe dem
Gottesmann entgegen und laß Gott den HERRN durch ihn befragen, ob ich
von dieser meiner Krankheit genesen werde.« 9Da ging Hasael ihm entgegen
und nahm Geschenke an sich, allerlei Kostbarkeiten von Damaskus, eine
Last für vierzig Kamele. Als er nun hingekommen und vor ihn getreten
war, sagte er: »Dein Sohn Benhadad, der König von Syrien, hat mich zu
dir gesandt und läßt fragen, ob er von dieser seiner Krankheit genesen
werde.«

\hypertarget{bb-elisas-eruxf6ffnung-an-hasael-benhadads-ermordung-hasaels-regierungsantritt}{%
\subparagraph{bb) Elisas Eröffnung an Hasael; Benhadads Ermordung;
Hasaels
Regierungsantritt}\label{bb-elisas-eruxf6ffnung-an-hasael-benhadads-ermordung-hasaels-regierungsantritt}}

10Da antwortete ihm Elisa: »Gehe hin und sage ihm, daß er gewißlich
wieder gesund werden würde; aber Gott der HERR hat mir geoffenbart, daß
er sterben muß.« 11Dabei starrte der Gottesmann unverwandt vor sich hin
und war aufs äußerste entsetzt und brach dann in Tränen aus. 12Als
Hasael ihn nun fragte: »Warum weint mein Herr?«, antwortete er: »Weil
ich weiß, wieviel Unheil du den Israeliten zufügen wirst: ihre festen
Städte wirst du in Brand stecken, ihre jungen Männer mit dem Schwert
umbringen, ihre kleinen Kinder zerschmettern und ihren schwangeren
Frauen den Leib aufschlitzen.« 13Da erwiderte Hasael: »Was ist denn dein
Knecht, der Hund, daß er solche großen Dinge tun sollte?« Elisa
entgegnete ihm: »Gott der HERR hat dich mir als König über Syrien
geoffenbart.« 14Darauf ging (Hasael) von Elisa weg, und als er zu seinem
Herrn kam und dieser ihn fragte: »Was hat Elisa dir gesagt?«, antwortete
er: »Er hat mir gesagt, du würdest gewißlich wieder gesund werden.« 15Am
folgenden Tage aber nahm er die Bettdecke✲, tauchte sie in Wasser und
breitete sie ihm über das Gesicht, so daß er starb. Hasael aber wurde
König an seiner Statt.

\hypertarget{von-joram-in-juda-und-jehu-in-israel-bis-amazja-in-juda-und-joas-in-israel-816-1422}{%
\subsubsection{4. Von Joram in Juda und Jehu in Israel bis Amazja in
Juda und Joas in Israel
(8,16-14,22)}\label{von-joram-in-juda-und-jehu-in-israel-bis-amazja-in-juda-und-joas-in-israel-816-1422}}

\hypertarget{a-joram-und-sein-sohn-ahasja-kuxf6nige-von-juda}{%
\paragraph{a) Joram und sein Sohn Ahasja, Könige von
Juda}\label{a-joram-und-sein-sohn-ahasja-kuxf6nige-von-juda}}

16Im fünften Jahre der Regierung Jorams, des Sohnes Ahabs, des Königs
von Israel, kam Joram, der Sohn des Königs Josaphat von Juda, zur
Regierung. 17Er war zweiunddreißig Jahre alt, als er König wurde, und
acht Jahre regierte er in Jerusalem. 18Er wandelte auf dem Wege der
Könige von Israel, wie es im Hause Ahabs durchweg der Fall war -- er
hatte sich nämlich mit einer Tochter Ahabs verheiratet --; so tat er,
was dem HERRN mißfiel. 19Aber der HERR wollte Juda nicht untergehen
lassen um seines Knechtes David willen, weil er ihm zugesagt hatte, daß
er ihm allezeit eine Leuchte\textless sup title=``vgl. 1.Kön
11,36''\textgreater✲ vor seinem Angesicht verleihen wolle.

\hypertarget{der-abfall-der-edomiter-und-der-tod-jorams}{%
\paragraph{Der Abfall der Edomiter und der Tod
Jorams}\label{der-abfall-der-edomiter-und-der-tod-jorams}}

20Unter seiner Regierung fielen die Edomiter von der Oberherrschaft
Judas ab und setzten einen eigenen König über sich ein. 21Da zog Joram
mit all seinen Kriegswagen hinüber nach Zair; doch als er nachts
aufgebrochen war, schlugen ihn✲ die Edomiter, die ihn und die
Befehlshaber der Wagen umzingelt hatten, das Kriegsvolk aber floh nach
Hause. 22So fielen die Edomiter von der Oberherrschaft Judas ab und sind
unabhängig geblieben bis auf den heutigen Tag. Damals fiel auch Libna
ab, zu derselben Zeit. 23Die übrige Geschichte Jorams aber und alles,
was er unternommen hat, das findet sich bekanntlich aufgezeichnet im
Buch der Denkwürdigkeiten\textless sup title=``oder:
Chronik''\textgreater✲ der Könige von Juda. 24Als Joram sich dann zu
seinen Vätern gelegt und man ihn bei seinen Vätern in der Davidsstadt
begraben hatte, folgte ihm sein Sohn Ahasja in der Regierung nach.

\hypertarget{ahasja-von-juda-krieg-mit-hasael}{%
\paragraph{Ahasja von Juda; Krieg mit
Hasael}\label{ahasja-von-juda-krieg-mit-hasael}}

25Im zwölften Jahre der Regierung Jorams, des Sohnes Ahabs, des Königs
von Israel, kam Ahasja, der Sohn des Königs Joram von Juda, zur
Regierung. 26Zweiundzwanzig Jahre war Ahasja alt, als er auf den Thron
kam, und ein Jahr hat er in Jerusalem regiert; seine Mutter hieß Athalja
und war die Enkelin des Königs Omri von Israel. 27Er wandelte auf dem
Wege des Hauses Ahabs und tat, was dem HERRN mißfiel, wie das Haus
Ahabs, weil er mit dem Hause Ahabs verschwägert war. 28Er zog mit Joram,
dem Sohne Ahabs, gegen Hasael, den König von Syrien, zu Felde und
kämpfte mit ihm bei Ramoth in Gilead. Als aber die Syrer dort den König
Joram verwundet hatten, 29kehrte der König Joram zurück, um sich in
Jesreel von den Wunden heilen zu lassen, die ihm die Syrer bei Rama
beigebracht hatten, als er gegen den König Hasael von Syrien Krieg
führte. Darauf kam Ahasja, der Sohn Jorams, der König von Juda, um
Joram, den Sohn Ahabs, in Jesreel zu besuchen, weil er dort krank lag.

\hypertarget{b-jehu-vollzieht-das-gericht-gottes-am-hause-ahabs-und-am-guxf6tzendienst-israels}{%
\paragraph{b) Jehu vollzieht das Gericht Gottes am Hause Ahabs und am
Götzendienst
Israels}\label{b-jehu-vollzieht-das-gericht-gottes-am-hause-ahabs-und-am-guxf6tzendienst-israels}}

\hypertarget{aa-jehu-auf-betreiben-elisas-zum-kuxf6nig-gesalbt}{%
\subparagraph{aa) Jehu auf Betreiben Elisas zum König
gesalbt}\label{aa-jehu-auf-betreiben-elisas-zum-kuxf6nig-gesalbt}}

\hypertarget{section-8}{%
\section{9}\label{section-8}}

1Der Prophet Elisa aber rief einen von den Prophetenjüngern zu sich und
befahl ihm: »Gürte dir die Lenden, nimm dieses Ölfläschchen mit dir und
begib dich nach Ramoth in Gilead. 2Wenn du dort angekommen bist, so sieh
dich daselbst nach Jehu um, dem Sohne Josaphats, des Sohnes Nimsis. Gehe
dann zu ihm ins Haus, fordere ihn auf, aus dem Kreise seiner Genossen
herauszutreten, und führe ihn in das innerste Gemach. 3Dann nimm das
Ölfläschchen und gieße es ihm aufs Haupt mit den Worten: ›So spricht der
HERR: Ich salbe dich hiermit zum König über Israel!‹ Dann öffne die Tür
und entfliehe unverzüglich!«

4Als nun der junge Mann, der Diener des Propheten, nach Ramoth in Gilead
gekommen 5und in das Haus eingetreten war, saßen da die Hauptleute des
Heeres gerade beisammen. Er sagte: »Ich habe einen Auftrag an dich,
Hauptmann.« Als Jehu nun fragte: »An wen von uns allen?«, antwortete er:
»An dich, Hauptmann.« 6Da stand Jehu auf und ging (mit ihm) ins Haus
hinein; jener aber goß ihm das Öl aufs Haupt und sagte zu ihm: »So
spricht der HERR, der Gott Israels: ›Ich habe dich hiermit zum König
über das Volk des HERRN, über Israel, gesalbt. 7Du sollst nun das Haus
Ahabs, deines Herrn, ausrotten, damit ich das Blut der Propheten, meiner
Knechte, und das Blut aller Knechte des HERRN an Isebel räche. 8Denn das
ganze Haus Ahabs soll umkommen, und ich will von den Angehörigen Ahabs
alles ausrotten, was männlichen Geschlechts ist, sowohl die Unmündigen
als auch die Mündigen\textless sup title=``d.h. alle ohne
Ausnahme''\textgreater✲ in Israel; 9und ich will mit dem Hause Ahabs
verfahren wie mit dem Hause Jerobeams, des Sohnes Nebats, und wie mit
dem Hause Baesas, des Sohnes Ahias. 10Isebel aber sollen die Hunde auf
der Feldmark von Jesreel fressen, und niemand soll sie begraben!‹«
Hierauf öffnete er die Tür und entfloh.

\hypertarget{bb-jehu-von-den-heerfuxfchrern-als-kuxf6nig-anerkannt}{%
\subparagraph{bb) Jehu von den Heerführern als König
anerkannt}\label{bb-jehu-von-den-heerfuxfchrern-als-kuxf6nig-anerkannt}}

11Als nun Jehu wieder zu den anderen Hauptleuten seines Herrn hinauskam
und sie ihn fragten: »Steht alles gut? Warum ist dieser Verrückte zu dir
gekommen?«, antwortete er ihnen: »Ihr kennt ja den Mann und sein
Geschwätz\textless sup title=``oder: und wißt, was er
wollte''\textgreater✲.« 12Aber sie riefen: »Das sind Ausflüchte! Teile
es uns nur mit!« Da sagte er: »So und so hat er zu mir gesagt, nämlich:
›So spricht der HERR: Ich habe dich zum König über Israel gesalbt.‹«
13Sofort nahmen sie alle ihre Mäntel, legten sie ihm zu Füßen auf die
bloßen Stufen, ließen die Posaune blasen und riefen: »Jehu ist König!«
14Auf diese Weise zettelte Jehu, der Sohn Josaphats, des Sohnes Nimsis,
eine Verschwörung gegen Joram an. -- Joram hatte nämlich mit ganz Israel
Ramoth in Gilead gegen den syrischen König Hasael verteidigt, 15war dann
aber zurückgekehrt, um sich in Jesreel von den Wunden heilen zu lassen,
die ihm die Syrer im Kampfe mit dem syrischen Könige Hasael beigebracht
hatten. -- Jehu aber sagte: »Wenn ihr einverstanden seid, so darf
niemand die Stadt verlassen, um hinzugehen und das Geschehene in Jesreel
zu melden.«

\hypertarget{cc-jehu-ermordet-joram-und-ahasja}{%
\subparagraph{cc) Jehu ermordet Joram und
Ahasja}\label{cc-jehu-ermordet-joram-und-ahasja}}

16Hierauf bestieg Jehu seinen Wagen und trat die Fahrt nach Jesreel an;
denn dort lag Joram krank darnieder, und Ahasja, der König von Juda, war
dorthin gekommen, um Joram zu besuchen. 17Als nun der Wächter, der auf
dem Turm zu Jesreel stand, die Kriegerschar Jehus herankommen sah, rief
er: »Ich sehe eine Kriegerschar!« Da befahl Joram: »Man nehme einen
Berittenen und schicke ihnen den entgegen, daß er frage, ob sie in
friedlicher Absicht kommen!« 18Der Reiter ritt ihm also entgegen und
sagte: »Der König läßt fragen, ob ihr in friedlicher Absicht kommt.«
Jehu antwortete: »Was geht dich der Friede an? Mache kehrt und reite
hinter mir her!« Da meldete der Wächter: »Der Bote ist zu ihnen
hingekommen, kehrt aber nicht zurück.« 19Da schickte er einen zweiten
Reiter ab; als der bei ihnen ankam und sagte: »Der König läßt fragen, ob
ihr in friedlicher Absicht kommt«, antwortete Jehu wieder: »Was geht
dich der Friede an? Mache kehrt und reite hinter mir her!« 20Da meldete
der Wächter: »Der ist auch zu ihnen hingekommen, kehrt aber nicht wieder
zurück. Doch die Art, wie jener fährt, sieht so aus, als ob es Jehu, der
Sohn✲ Nimsis, wäre; denn er fährt wie wahnsinnig.«

21Da befahl Joram anzuspannen; und als man seinen Wagen angespannt
hatte, fuhren Joram, der König von Israel, und Ahasja, der König von
Juda, hinaus, jeder auf seinem Wagen; sie fuhren Jehu entgegen und
trafen bei dem Acker Naboths, des Jesreeliters, mit ihm zusammen. 22Als
nun Joram den Jehu sah und ihn fragte: »Kommst du in friedlicher
Absicht, Jehu?«, antwortete dieser: »Was friedliche Absicht bei all dem
Götzendienst deiner Mutter Isebel und all ihren Zaubereien!« 23Da ließ
Joram seinen Wagen zur Flucht umwenden und rief dem Ahasja zu: »Verrat,
Ahasja!« 24Jehu aber hatte seinen Bogen schon gespannt und traf Joram
zwischen die Schulterblätter, so daß der Pfeil ihm durch das Herz fuhr
und er in seinem Wagen niedersank. 25Dann befahl er Bidkar, seinem
Ritter\textless sup title=``vgl. 7,2''\textgreater✲: »Nimm ihn und wirf
ihn auf den Acker Naboths, des Jesreeliters! Denke daran, wie wir beide
nebeneinander hinter seinem Vater Ahab herritten und Gott der HERR
dieses Drohwort gegen ihn aussprach: 26›So wahr ich gestern das Blut
Naboths und das Blut seiner Söhne gesehen habe‹ -- so lautet der
Ausspruch des HERRN --, ›so gewiß will ich es dir auf diesem Acker
vergelten!‹ -- so lautet der Ausspruch des HERRN. So nimm ihn nun und
wirf ihn auf den Acker nach dem Geheiß des HERRN!«

27Als Ahasja, der König von Juda, das sah, floh er in der Richtung auf
Beth-Haggan. Jehu aber jagte ihm nach und rief: »Schießt auch ihn
nieder!« Da schoß man nach ihm auf dem Wagen und verwundete ihn auf der
Anhöhe von Gur, die bei Jibleam liegt; er floh dann noch bis Megiddo und
starb dort. 28Seine Diener brachten ihn dann zu Wagen nach Jerusalem,
und man begrub ihn in seiner Grabstätte bei seinen Vätern in der
Davidsstadt.~-- 29Ahasja war aber König von Juda geworden im elften
Jahre der Regierung Jorams, des Sohnes Ahabs.

\hypertarget{dd-isebels-grausiges-ende}{%
\subparagraph{dd) Isebels grausiges
Ende}\label{dd-isebels-grausiges-ende}}

30Jehu aber war nach Jesreel gekommen; und sobald Isebel dies erfuhr,
schminkte sie sich die Augen, schmückte sich das Haupt und schaute zum
Fenster hinaus. 31Als nun Jehu ins Tor hereinkam, rief sie ihm zu: »Ist
es Simri, dem Mörder seines Herrn, gut ergangen?«\textless sup
title=``vgl. 1.Kön 16,9-19''\textgreater✲. 32Da blickte er nach dem
Fenster hinauf und rief: »Wer hält es mit mir? Wer?« Als nun zwei oder
drei Kammerherren zu ihm hinabschauten, 33rief er ihnen zu: »Stürzt sie
herab!« Da stürzten sie sie hinab, so daß die Wand und die Rosse mit
ihrem Blut bespritzt wurden und diese sie zerstampften. 34Als er dann in
das Schloß eingetreten war und gegessen und getrunken hatte, befahl er:
»Seht doch nach jenem verfluchten Weibe und begrabt sie! Denn sie ist
eine Königstochter.« 35Als man aber hinging, um sie zu begraben, fand
man von ihr nichts mehr als den Schädel, die Füße und die Hände. 36Als
sie nun zurückkamen und es dem Jehu meldeten, rief er aus: »So lautet
das Wort des HERRN, das er durch den Mund seines Knechtes Elia, des
Thisbiters, hat verkünden lassen\textless sup title=``1.Kön
21,23''\textgreater✲: ›Auf der Feldmark von Jesreel sollen die Hunde das
Fleisch Isebels fressen, 37und der Leichnam Isebels soll auf der
Feldmark von Jesreel wie Dünger auf dem Felde liegen, so daß man nicht
mehr wird sagen können: Das ist Isebel.‹«

\hypertarget{ee-jehu-luxe4uxdft-die-siebzig-kuxf6niglichen-prinzen-ermorden-und-rottet-alle-zum-hause-ahabs-gehuxf6rigen-aus}{%
\subparagraph{ee) Jehu läßt die siebzig königlichen Prinzen ermorden und
rottet alle zum Hause Ahabs Gehörigen
aus}\label{ee-jehu-luxe4uxdft-die-siebzig-kuxf6niglichen-prinzen-ermorden-und-rottet-alle-zum-hause-ahabs-gehuxf6rigen-aus}}

\hypertarget{section-9}{%
\section{10}\label{section-9}}

1Nun befanden sich siebzig Söhne\textless sup title=``oder:
Enkel''\textgreater✲ Ahabs in Samaria. Daher schrieb Jehu Briefe und
sandte sie nach Samaria an die Befehlshaber der Stadt sowie an die
Ältesten und an die Erzieher der königlichen Prinzen; die lauteten so:
2»Und nun, wenn dieses Schreiben an euch gelangt -- ihr seid ja über die
Söhne\textless sup title=``oder: Enkel''\textgreater✲ eures Herrn sowie
über die Wagen und Rosse, über die festen Plätze und Waffenvorräte
bestellt --: 3so wählt euch den besten und tüchtigsten unter den
Söhnen\textless sup title=``oder: Enkeln''\textgreater✲ eures Herrn aus
und setzt ihn auf den Thron seines Vaters\textless sup title=``oder:
Großvaters''\textgreater✲ und kämpft für das Haus eures Herrn!« 4Aber
sie fürchteten sich gar sehr und sagten: »Nachdem sich die beiden Könige
gegen ihn nicht haben behaupten können, wie sollten wir da bestehen?«
5So ließen denn der Hausminister\textless sup title=``=~Vorsteher des
königlichen Palastes''\textgreater✲ und der Stadtoberste sowie die
Ältesten und die Erzieher folgende Botschaft an Jehu gelangen: »Wir sind
deine Knechte und wollen allen deinen Befehlen nachkommen; wir wollen
niemand zum König machen: tu, was dir beliebt!« 6Da schrieb Jehu einen
zweiten Brief an sie, der so lautete: »Wenn ihr es mit mir haltet und
mir gehorsam sein wollt, so nehmt die Köpfe der Männer, der
Söhne\textless sup title=``oder: Enkel''\textgreater✲ eures Herrn, und
kommt damit morgen um diese Zeit zu mir nach Jesreel.« Die
Söhne\textless sup title=``oder: Enkel''\textgreater✲ des Königs
befanden sich nämlich, siebzig an der Zahl, bei den vornehmsten Männern
der Stadt, die sie zu erziehen hatten. 7Sobald nun das Schreiben an sie
gelangte, nahmen sie die Prinzen und ermordeten alle siebzig; ihre Köpfe
legten sie dann in Körbe beisammen und sandten sie an ihn nach Jesreel.
8Als dann der Bote hinkam und ihm meldete, man habe die Köpfe der
Söhne\textless sup title=``oder: Enkel''\textgreater✲ des Königs
gebracht, befahl er: »Schichtet sie in zwei Haufen am Eingang des Tores
bis morgen früh auf!« 9Am folgenden Morgen aber ging er hinaus und trat
vor das ganze Volk mit den Worten: »Ihr seid ohne Schuld! Ich bin es ja
gewesen, der sich gegen meinen Herrn verschworen und ihn ums Leben
gebracht hat; doch wer hat diese alle ermordet? 10So erkennt denn
hieraus, daß keine Drohung, die der HERR gegen das Haus Ahabs
ausgesprochen hat, unerfüllt bleibt; nein, der HERR hat alles
ausgeführt, was er durch den Mund seines Knechtes Elia angekündigt hat.«
11Hierauf ließ Jehu alle, die in Jesreel vom Hause Ahabs noch übrig
waren, umbringen, auch alle seine Großen, seine Vertrauten und seine
Priester, bis er ihm keinen einzigen mehr am Leben übriggelassen hatte.

\hypertarget{ff-jehu-ermordet-die-juduxe4ischen-prinzen}{%
\subparagraph{ff) Jehu ermordet die judäischen
Prinzen}\label{ff-jehu-ermordet-die-juduxe4ischen-prinzen}}

12Hierauf machte sich Jehu auf den Weg, um sich nach Samaria zu begeben.
Als er unterwegs bei Beth-Eked-Haroim\textless sup title=``d.h.
Versammlungshaus der Hirten''\textgreater✲ war, 13traf er die Brüder
Ahasjas, des Königs von Juda, an, und fragte sie, wer sie seien. Sie
antworteten: »Wir sind die Brüder Ahasjas und sind hergekommen, um die
Söhne des Königs und die Söhne der Königin-Mutter zu besuchen.« 14Da gab
er den Befehl: »Ergreift sie lebendig!« Da ergriff man sie lebendig,
ermordete sie (und warf sie) in die Zisterne von Beth-Eked,
zweiundvierzig Mann; keinen einzigen von ihnen ließ er am Leben.

\hypertarget{gg-jehu-nimmt-den-rechabiten-jonadab-in-seine-freundschaft-auf}{%
\subparagraph{gg) Jehu nimmt den Rechabiten Jonadab in seine
Freundschaft
auf}\label{gg-jehu-nimmt-den-rechabiten-jonadab-in-seine-freundschaft-auf}}

15Als er dann von dort weiterzog, traf er auf Jonadab, den Sohn Rechabs,
der ihm entgegenkam. Der begrüßte ihn, er aber fragte ihn: »Bist du
aufrichtig gegen mich gesinnt, wie ich gegen dich?« Als Jonadab mit »Ja«
antwortete, sagte Jehu: »Wenn es wirklich so ist, so gib mir deine
Hand!« Da reichte er ihm seine Hand, und Jehu ließ ihn zu sich in den
Wagen steigen 16und sagte: »Komm mit mir und sieh dir mit Freuden meinen
Eifer für den HERRN an!« So nahm er ihn denn in seinem Wagen mit 17und
ließ nach seiner Ankunft in Samaria alle umbringen, die in Samaria von
Angehörigen Ahabs noch übrig waren, bis er sie alle ausgerottet hatte,
wie der HERR es dem Elia zuvor angekündigt hatte\textless sup
title=``1.Kön 21,21-22''\textgreater✲.

\hypertarget{hh-jehu-rottet-die-baalsverehrer-in-samaria-aus}{%
\subparagraph{hh) Jehu rottet die Baalsverehrer in Samaria
aus}\label{hh-jehu-rottet-die-baalsverehrer-in-samaria-aus}}

18Hierauf ließ Jehu das ganze Volk zusammenkommen und sagte zu ihnen:
»Ahab hat dem Baal nur eine geringe Verehrung erwiesen, Jehu aber wird
ihm eifriger dienen. 19Daher laßt jetzt alle Propheten Baals, alle seine
Diener und alle seine Priester zu mir kommen, keiner darf fehlen! Denn
ich habe für Baal ein großes Opferfest im Sinn: jeder, der dabei fehlt,
ist des Todes!« Jehu ging aber mit Hinterlist so zu Werke, um die
Baalsdiener auszurotten. 20Dann befahl Jehu: »Kündigt eine
Festversammlung zu Ehren Baals an!« Als man sie öffentlich
bekanntgemacht hatte, 21sandte Jehu Boten in alle Teile Israels umher.
Da fanden sich alle Verehrer Baals ein, kein einziger blieb übrig, der
nicht erschienen wäre. Als sie sich dann in den Baalstempel begeben
hatten, so daß der Tempel von einem Ende bis zum andern mit Menschen
angefüllt war, 22gab er dem Aufseher über die Kleiderkammer den Befehl,
allen Baalsverehrern Gewänder zu verabreichen. Als der die Gewänder für
sie herausgegeben hatte, 23begab sich Jehu mit Jonadab, dem Sohne
Rechabs, in den Baalstempel und sagte zu den Baalsverehrern: »Seht genau
nach, daß sich hier unter euch ja kein Verehrer Gottes des HERRN
befinde, sondern ausschließlich Verehrer Baals!« 24Hierauf schickten sie
sich an, die Schlacht- und Brandopfer darzubringen. Jehu hatte aber
draußen achtzig Mann aufgestellt und zu ihnen gesagt: »Wer einen von den
Männern, die ich euch in die Hände liefere, entkommen läßt, soll mit
seinem eigenen Leben für ihn haften!« 25Als man dann mit der Darbringung
des Brandopfers fertig war, befahl Jehu den Leibwächtern und
Rittern\textless sup title=``vgl. 7,2''\textgreater✲: »Geht hinein, haut
sie nieder: keiner darf davonkommen!« Sie machten sie also mit dem
Schwert nieder, warfen den Altar um✲ und drangen in das Allerheiligste
des Baalstempels ein✲; 26dann schafften sie die Götzensäulen aus dem
Baalstempel hinaus und verbrannten sie, 27zertrümmerten das Standbild
Baals, rissen den Baalstempel nieder und machten Aborte daraus, die bis
auf den heutigen Tag geblieben sind.

\hypertarget{ii-gottes-verkuxfcndigung-an-jehu-miuxdferfolge-jehus-abschluuxdf-der-jehugeschichte}{%
\subparagraph{ii) Gottes Verkündigung an Jehu; Mißerfolge Jehus;
Abschluß der
Jehugeschichte}\label{ii-gottes-verkuxfcndigung-an-jehu-miuxdferfolge-jehus-abschluuxdf-der-jehugeschichte}}

28So rottete Jehu den Baalsdienst in Israel aus; 29jedoch von den
Sünden, zu denen Jerobeam, der Sohn Nebats, die Israeliten verführt
hatte, von diesen ließ auch Jehu nicht ab, nämlich von der Verehrung der
goldenen Stierbilder, die in Bethel und Dan aufgestellt waren. 30Der
HERR ließ zwar dem Jehu verkünden: »Weil du alles, was mir wohlgefällig
war, eifrig ausgeführt und am Hause Ahabs ganz nach meinem Sinn
gehandelt hast, so sollen Nachkommen von dir bis ins vierte Glied auf
dem Throne von Israel sitzen«; 31aber Jehu ließ es sich nicht angelegen
sein, mit ganzem Herzen nach der Weisung des HERRN, des Gottes Israels,
zu wandeln; er ließ nicht ab von den Sünden, zu denen Jerobeam Israel
verführt hatte.

32Zu jener Zeit begann der HERR Teile vom Gebiet Israels loszureißen;
denn Hasael schlug sie in allen Grenzgebieten Israels: 33östlich vom
Jordan die ganze Landschaft Gilead, die Stämme Gad, Ruben und Manasse,
von Aroer an, das am Arnonflusse liegt, sowohl Gilead als auch Basan.~--
34Die übrige Geschichte Jehus aber und alles, was er unternommen hat,
sowie alle seine tapferen Taten, das findet sich bekanntlich
aufgezeichnet im Buch der Denkwürdigkeiten\textless sup title=``oder:
Chronik''\textgreater✲ der Könige von Israel. 35Als Jehu sich dann zu
seinen Vätern gelegt und man ihn in Samaria begraben hatte, folgte ihm
sein Sohn Joahas in der Regierung nach. 36Die Zeit aber, die Jehu in
Samaria über Israel regiert hat, betrug achtundzwanzig Jahre.

\hypertarget{c-regierung-sturz-und-tod-der-athalja-in-juda}{%
\paragraph{c) Regierung, Sturz und Tod der Athalja in
Juda}\label{c-regierung-sturz-und-tod-der-athalja-in-juda}}

\hypertarget{aa-athaljas-thronraub-und-mordtaten-rettung-des-joas}{%
\subparagraph{aa) Athaljas Thronraub und Mordtaten; Rettung des
Joas}\label{aa-athaljas-thronraub-und-mordtaten-rettung-des-joas}}

\hypertarget{section-10}{%
\section{11}\label{section-10}}

1Als aber Athalja, die Mutter Ahasjas, erfuhr, daß ihr Sohn tot sei,
machte sie sich daran, alle, die zur königlichen Familie gehörten,
umzubringen. 2Aber Joseba, die Tochter des Königs Joram, Ahasjas
Schwester, nahm Joas, den Sohn Ahasjas, und schaffte ihn aus der Mitte
der Königssöhne, die ermordet werden sollten, heimlich beiseite, indem
sie ihn mit seiner Amme in die Bettzeugkammer brachte; sie verbarg ihn
dort vor Athalja, so daß er der Ermordung entging. 3Er blieb dann sechs
Jahre lang bei ihr im Hause des HERRN versteckt, während Athalja das
Land regierte.

\hypertarget{bb-die-verschwuxf6rung-jojadas}{%
\subparagraph{bb) Die Verschwörung
Jojadas}\label{bb-die-verschwuxf6rung-jojadas}}

4Im siebten Jahre aber ließ Jojada\textless sup title=``der Priester;
vgl. V.9''\textgreater✲ die Hauptleute der Karer und der Läufer holen
und zu sich in den Tempel des HERRN kommen; da schloß er ein feierliches
Abkommen mit ihnen und ließ sie im Tempel einen Eid ablegen; dann zeigte
er ihnen den Königssohn 5und gab ihnen die Weisung: »Folgendermaßen müßt
ihr zu Werke gehen: Das eine Drittel von euch, das am Sabbat (aus dem
Tempel) abzieht, um die Wache im königlichen Palast zu übernehmen, 6und
das (andere) Drittel am Tore Sur und das (letzte) Drittel am Tore hinter
den Leibwächtern, die ihr die Wache beim Palast gehalten habt~-- 7beide
anderen Abteilungen, die am Sabbat aufziehen, sie alle sollen im Tempel
des HERRN die Wache beim König übernehmen. 8Ihr müßt euch also rings um
den König scharen, ein jeder mit seinen Waffen in der Hand; und wer in
die Reihen eindringt, soll getötet werden; und ihr müßt dann beständig
um den König sein, wenn er (aus dem Tempel) auszieht und wenn er (in den
Palast) einzieht.«

9Die Hauptleute verfuhren dann genau nach der Anweisung des Priesters
Jojada: jeder nahm seine Mannschaft zu sich, sowohl die, welche am
Sabbat abzog, als auch die, welche am Sabbat aufzog, und so kamen sie
zum Priester Jojada. 10Dieser gab dann den Hauptleuten die Speere und
die Schilde, die dem König David gehört hatten und die sich im Tempel
des HERRN befanden. 11Nachdem sich hierauf die Leibwächter, ein jeder
mit seinen Waffen in der Hand, von der Südseite des Tempels bis an den
Altar und von da wieder bis an die Nordseite des Tempels aufgestellt
hatten, 12führte er den Königssohn heraus und legte ihm die Königsbinde
und die Armspangen\textless sup title=``2.Sam 1,10''\textgreater✲ an. So
machten sie ihn zum König und salbten ihn, klatschten in die Hände und
riefen: »Es lebe der König!«

\hypertarget{cc-athaljas-gefangennahme-und-ermordung}{%
\subparagraph{cc) Athaljas Gefangennahme und
Ermordung}\label{cc-athaljas-gefangennahme-und-ermordung}}

13Als nun Athalja das Geschrei {[}der Leibwächter und{]} des Volkes
vernahm, begab sie sich zum Volk in den Tempel des HERRN. 14Hier sah sie
dann den König an der Säule, wie es Brauch war, und die Hauptleute und
die Trompeter neben dem Könige stehen, während das gesamte Volk des
Landes voller Freude war und in die Trompeten stieß. Da zerriß Athalja
ihre Kleider und rief: »Verrat, Verrat!« 15Aber der Priester Jojada gab
den Hauptleuten {[}den Befehlshabern des Heeres{]} den Befehl: »Führt
sie hinaus zwischen den Reihen hindurch, und wer ihr folgt, den haut mit
dem Schwerte nieder!« Der Priester hatte nämlich befohlen, sie dürfe ja
nicht im Tempel des HERRN getötet werden. 16Da legte man Hand an sie,
und als sie an dem Wege, der zum Eingang für die Pferde bestimmt war,
beim königlichen Palast angelangt war, wurde sie dort getötet.

\hypertarget{dd-jojadas-mauxdfnahmen-zur-ehre-gottes-kruxf6nung-des-joas}{%
\subparagraph{dd) Jojadas Maßnahmen zur Ehre Gottes; Krönung des
Joas}\label{dd-jojadas-mauxdfnahmen-zur-ehre-gottes-kruxf6nung-des-joas}}

17Darauf schloß Jojada zwischen dem HERRN und dem König und dem Volk das
feierliche Abkommen, daß sie das Volk des HERRN sein
sollten\textless sup title=``oder: werden wollten''\textgreater✲;
{[}ebenso zwischen dem König und dem Volk{]}. 18Darauf zog das ganze
Volk des Landes nach dem Baalstempel und riß ihn nieder; seine Altäre
und Götterbilder zerschlugen sie vollständig und töteten den
Baalspriester Matthan vor den Altären. Nachdem der Priester dann
Wachtposten am Tempel des HERRN aufgestellt hatte, 19nahm er die
Hauptleute und die Karer und Trabanten sowie alles Volk des Landes mit
sich, und sie führten den König aus dem Tempel des HERRN hinab und zogen
durch das Leibwächtertor in das königliche Schloß ein, wo er sich auf
den königlichen Thron setzte. 20Da war die gesamte Bevölkerung voller
Freude, und die Stadt blieb ruhig; Athalja aber hatten sie im
königlichen Schloß mit dem Schwert getötet.

\hypertarget{d-joas-kuxf6nig-von-juda}{%
\paragraph{d) Joas König von Juda}\label{d-joas-kuxf6nig-von-juda}}

\hypertarget{aa-eingangswort}{%
\subparagraph{aa) Eingangswort}\label{aa-eingangswort}}

\hypertarget{section-11}{%
\section{12}\label{section-11}}

1Joas war beim Regierungsantritt sieben Jahre alt; 2im siebten Jahre
Jehus war er König geworden, vierzig Jahre regierte er in Jerusalem;
seine Mutter hieß Zibja und stammte aus Beerseba. 3Joas tat, was dem
HERRN wohlgefiel, sein ganzes Leben lang, weil der Priester Jojada ihn
beriet; 4nur der Höhendienst wurde nicht beseitigt, sondern das Volk
brachte immer noch Schlacht- und Rauchopfer auf den Höhen dar.

\hypertarget{bb-verordnung-des-kuxf6nigs-uxfcber-die-ausbesserung-des-tempels-und-uxfcber-die-verwaltung-und-verwendung-der-einkommenden-tempelgelder}{%
\subparagraph{bb) Verordnung des Königs über die Ausbesserung des
Tempels und über die Verwaltung und Verwendung der einkommenden
Tempelgelder}\label{bb-verordnung-des-kuxf6nigs-uxfcber-die-ausbesserung-des-tempels-und-uxfcber-die-verwaltung-und-verwendung-der-einkommenden-tempelgelder}}

5Joas ließ aber den Priestern die Weisung zugehen: »Alles Geld, das als
Weihegabe\textless sup title=``oder: behufs heiliger
Abgaben''\textgreater✲ in den Tempel des HERRN gebracht wird: das Geld,
das jemand durch Schätzung auferlegt wird, ferner alles Geld, das jemand
aus freien Stücken in den Tempel des HERRN bringt, 6das sollen die
Priester in Empfang nehmen, ein jeder von seinen Bekannten✲; aber sie
sollen hinwieder davon alles ausbessern lassen, was am Tempel baufällig
ist, wo immer ein Schaden daran sichtbar wird.« 7Da jedoch die Priester
im dreiundzwanzigsten Jahre der Regierung des Joas die Schäden am Tempel
noch nicht ausgebessert hatten, 8ließ der König Joas den Priester Jojada
und die übrigen Priester zu sich kommen und sagte zu ihnen: »Warum habt
ihr die Schäden am Tempel nicht ausgebessert? Fortan sollt ihr kein Geld
mehr von euren Bekannten✲ in Empfang nehmen, sondern sollt es für die
Ausbesserung der Schäden am Tempel abliefern.« 9Die Priester erklärten
sich damit einverstanden, daß sie kein Geld mehr vom Volk in Empfang
nehmen sollten, aber auch für die Ausbesserung der Schäden am Tempel
nicht mehr zu sorgen brauchten.

10Darauf nahm der Priester Jojada einen Kasten, ließ ein Loch in seinen
Deckel bohren und ihn neben dem Malstein rechts vom Eingang in den
Tempel des HERRN aufstellen; dahinein mußten die Priester, die an der
Schwelle Wache hielten, alles Geld tun, das im Tempel des HERRN
abgeliefert wurde. 11Sooft sie nun wahrnahmen, daß viel Geld in dem
Kasten war, kamen der Schreiber des Königs und der Hohepriester hinauf,
banden das Geld, das sich im Tempel des HERRN vorfand, in einen Beutel
zusammen und zählten es. 12Dann händigten sie das abgezählte Geld den
Werkführern ein, denen die Ausführung der Arbeiten am Tempel des HERRN
übertragen war, und diese zahlten es an die Zimmerleute und an die
übrigen Handwerker aus, die am Tempel des HERRN arbeiteten, 13sowie an
die Maurer und Steinmetzen und für den Ankauf von Hölzern und behauenen
Steinen, um die Schäden am Tempel des HERRN auszubessern und überhaupt
alle Kosten zu bestreiten, welche die Instandhaltung des Tempels
verursachte. 14Doch ließ man für den Tempel des HERRN keine silbernen
Becken, Messer, Sprengschalen, Trompeten, kurz keinerlei goldene und
silberne Geräte von dem Gelde anfertigen, das im Tempel des HERRN
einging, 15sondern man gab es den Werkleuten, damit sie dafür den Tempel
des HERRN ausbesserten. 16Dabei verlangte man von den Männern, denen man
das Geld einhändigte, damit sie es an die Arbeiter auszahlten, keine
Rechnungslegung, sondern sie handelten auf Treu und Glauben. 17Das Geld
von Schuld- und Sündopfern aber wurde nicht an den Tempel des HERRN
abgeliefert, sondern es gehörte den Priestern.

\hypertarget{cc-joas-bewahrt-jerusalem-vor-hasaels-angriff-durch-geldzahlung-seine-ermordung}{%
\subparagraph{cc) Joas bewahrt Jerusalem vor Hasaels Angriff durch
Geldzahlung; seine
Ermordung}\label{cc-joas-bewahrt-jerusalem-vor-hasaels-angriff-durch-geldzahlung-seine-ermordung}}

18Damals zog der syrische König Hasael heran, belagerte Gath und
eroberte es. Als Hasael sich dann anschickte, auch gegen Jerusalem
hinaufzuziehen, 19nahm Joas, der König von Juda, alle Weihgeschenke, die
von seinen Vorfahren, den judäischen Königen Josaphat, Joram und Ahasja
herrührten, sowie seine eigenen Weihgeschenke, ferner alles Gold, das
sich in den Schatzkammern des Tempels des HERRN und des königlichen
Palastes vorfand, und sandte es an Hasael, den König von Syrien. Da
stand dieser von dem Zuge gegen Jerusalem ab.

20Die übrige Geschichte des Joas aber und alles, was er unternommen hat,
das findet sich bekanntlich aufgezeichnet im Buch der
Denkwürdigkeiten\textless sup title=``oder: Chronik''\textgreater✲ der
Könige von Juda. 21Es taten sich aber seine Diener zu einer Verschwörung
gegen ihn zusammen und ermordeten Joas, als er in die Burg Millo
hinabging; 22und zwar waren es seine Diener Josachar, der Sohn Simeaths,
und Josabad, der Sohn Somers, die ihn ermordeten. Man begrub ihn dann
bei seinen Vätern in der Davidsstadt, und sein Sohn Amazja folgte ihm in
der Regierung nach.

\hypertarget{e-joahas-und-joas-kuxf6nige-von-israel-elisas-tod-kriege-mit-den-syrern}{%
\paragraph{e) Joahas und Joas Könige von Israel; Elisas Tod; Kriege mit
den
Syrern}\label{e-joahas-und-joas-kuxf6nige-von-israel-elisas-tod-kriege-mit-den-syrern}}

\hypertarget{aa-joahas-kuxf6nig-von-israel}{%
\subparagraph{aa) Joahas König von
Israel}\label{aa-joahas-kuxf6nig-von-israel}}

\hypertarget{section-12}{%
\section{13}\label{section-12}}

1Im dreiundzwanzigsten Regierungsjahre des Joas, des Sohnes des Königs
Ahasja von Juda, wurde Joahas, der Sohn Jehus, König über Israel und
regierte siebzehn Jahre in Samaria. 2Er tat, was dem HERRN mißfiel, und
wandelte in den Sünden Jerobeams, des Sohnes Nebats, der Israel zur
Sünde verführt hatte; er ließ nicht davon ab. 3Da entbrannte der Zorn
des HERRN gegen Israel, so daß er sie in die Gewalt Hasaels, des Königs
von Syrien, und in die Gewalt Benhadads, des Sohnes Hasaels, die ganze
Zeit hindurch fallen ließ. 4Als Joahas dann aber den HERRN mit Gebeten
anging, erhörte ihn der HERR; denn er sah die Bedrängnis der Israeliten,
weil der syrische König sie hart bedrückte. 5Daher ließ der HERR den
Israeliten einen Retter erstehen, so daß sie sich von der Herrschaft der
Syrer frei machten und die Israeliten wieder ruhig in ihren Zelten
wohnen konnten wie ehedem. 6Dennoch gaben sie die Sünde\textless sup
title=``=~den sündhaften Stierdienst''\textgreater✲ des Hauses Jerobeams
nicht auf, wozu dieser die Israeliten verführt hatte: sie hielten daran
fest; sogar die Bildsäule der Aschera\textless sup title=``vgl. 1.Kön
16,33''\textgreater✲ blieb in Samaria stehen. 7Er (der HERR) hatte dem
Joahas an Kriegsvolk nichts übriggelassen als fünfzig Reiter, zehn
Kriegswagen und zehntausend Mann Fußvolk; denn der syrische König hatte
sie vernichtet und sie dem Staub gleichgemacht, den man zertritt.

8Die übrige Geschichte des Joahas aber sowie alles, was er unternommen
hat, und seine tapferen Taten, das findet sich bekanntlich aufgezeichnet
im Buch der Denkwürdigkeiten\textless sup title=``oder:
Chronik''\textgreater✲ der Könige von Israel. 9Als Joahas sich dann zu
seinen Vätern gelegt und man ihn in Samaria begraben hatte, folgte ihm
sein Sohn Joas in der Regierung nach.

\hypertarget{bb-joas-kuxf6nig-von-israel}{%
\subparagraph{bb) Joas König von
Israel}\label{bb-joas-kuxf6nig-von-israel}}

10Im siebenunddreißigsten Jahre der Regierung des Joas, des Königs von
Juda, wurde Joas, der Sohn des Joahas, König über Israel und regierte
sechzehn Jahre in Samaria. 11Er tat, was dem HERRN mißfiel; er ließ in
keinem Stück von den Sünden Jerobeams, des Sohnes Nebats, ab, der Israel
zur Sünde verführt hatte; nein, er hielt daran fest.~-- 12Die übrige
Geschichte des Joas aber sowie alles, was er unternommen hat, seine
tapferen Taten (und) wie er mit Amazja, dem Könige von Juda, Krieg
geführt hat, das findet sich bekanntlich aufgezeichnet im Buch der
Denkwürdigkeiten\textless sup title=``oder: Chronik''\textgreater✲ der
Könige von Israel.~-- 13Als Joas sich dann zu seinen Vätern gelegt und
Jerobeam als sein Nachfolger den Thron bestiegen hatte, wurde Joas in
Samaria bei den Königen von Israel begraben.

\hypertarget{cc-joas-bei-dem-erkrankten-elisa-tod-elisas}{%
\subparagraph{cc) Joas bei dem erkrankten Elisa; Tod
Elisas}\label{cc-joas-bei-dem-erkrankten-elisa-tod-elisas}}

14Als Elisa aber an der Krankheit darniederlag, an der er sterben
sollte, kam Joas, der König von Israel, zu ihm hinab, weinte
über\textless sup title=``oder: vor''\textgreater✲ ihm und rief aus:
»Mein Vater, mein Vater! Du Wagen Israels und seine
Reiter!«\textless sup title=``vgl. 2,12''\textgreater✲ 15Da sagte Elisa
zu ihm: »Hole einen Bogen und Pfeile!«, und als er ihm einen Bogen und
Pfeile geholt hatte, 16sagte er zum König von Israel: »Lege deine Hand
auf den Bogen\textless sup title=``oder: nimm den Bogen zur
Hand''\textgreater✲!« Als er es getan hatte, legte Elisa seine Hände auf
die Hände des Königs 17und sagte: »Öffne das Fenster nach Osten zu!«
Nachdem er es geöffnet hatte, forderte Elisa ihn auf zu schießen. Da
schoß er, (Elisa) aber rief aus: »Ein Siegespfeil vom HERRN ist es, und
zwar ein Siegespfeil gegen die Syrer! So wirst du die Syrer bei Aphek
bis zur Vernichtung schlagen!« 18Dann fuhr er fort: »Nimm die Pfeile!«
Als er sie genommen hatte, sagte er zum König von Israel: »Schlage damit
auf die Erde!« Da schlug er dreimal und hielt dann inne. 19Da wurde der
Gottesmann unwillig über ihn und sagte: »Du hättest fünf- oder sechsmal
schlagen sollen, dann hättest du die Syrer bis zur Vernichtung
geschlagen; nun aber wirst du die Syrer nur dreimal schlagen!«

\hypertarget{dd-elisa-noch-im-grabe-wundertuxe4tig}{%
\subparagraph{dd) Elisa noch im Grabe
wundertätig}\label{dd-elisa-noch-im-grabe-wundertuxe4tig}}

20Als Elisa dann gestorben war, begrub man ihn. Es pflegten aber
moabitische Streifscharen Jahr für Jahr ins Land einzufallen. 21Nun
begab es sich, als man gerade einen Mann begraben wollte, daß man
plötzlich eine Streifschar herankommen sah; da warf man den Mann in das
Grab Elisas und ging weg. Sobald aber der Mann hineinkam und mit den
Gebeinen Elisas in Berührung kam, wurde er wieder lebendig und stellte
sich aufrecht auf seine Füße.

\hypertarget{ee-die-drei-siege-des-joas-uxfcber-die-syrer}{%
\subparagraph{ee) Die drei Siege des Joas über die
Syrer}\label{ee-die-drei-siege-des-joas-uxfcber-die-syrer}}

22Der König Hasael von Syrien aber hatte die Israeliten während der
ganzen Regierung des Joahas bedrängt; 23doch nun erwies der HERR ihnen
Gnade, erbarmte sich ihrer und wandte sich ihnen wieder zu wegen seines
Bundes mit Abraham, Isaak und Jakob; denn er wollte sie noch nicht
zugrunde gehen lassen, hatte sie auch bis jetzt noch nicht von seinem
Angesicht verworfen. 24Als daher Hasael, der König von Syrien, gestorben
und sein Sohn Benhadad ihm in der Regierung nachgefolgt war, 25entriß
Joas, der Sohn des Joahas, dem Benhadad, dem Sohne Hasaels, die Städte
wieder, die dieser seinem Vater Joahas im Kriege entrissen hatte.
Dreimal schlug ihn Joas und gewann so die israelitischen Städte zurück.

\hypertarget{f-amazja-kuxf6nig-von-juda}{%
\paragraph{f) Amazja König von Juda}\label{f-amazja-kuxf6nig-von-juda}}

\hypertarget{aa-guter-regierungsanfang}{%
\subparagraph{aa) Guter
Regierungsanfang}\label{aa-guter-regierungsanfang}}

\hypertarget{section-13}{%
\section{14}\label{section-13}}

1Im zweiten Regierungsjahre des Joas, des Sohnes des Königs Joahas von
Israel, wurde Amazja, der Sohn des Joas, König über Juda. 2Im Alter von
fünfundzwanzig Jahren kam er auf den Thron, und neunundzwanzig Jahre
regierte er in Jerusalem; seine Mutter hieß Joaddan und stammte aus
Jerusalem. 3Er tat, was dem HERRN wohlgefiel, doch nicht so wie sein
Ahnherr David, sondern ganz, wie sein Vater Joas getan hatte; 4jedoch
der Höhendienst wurde nicht abgeschafft, sondern das Volk brachte immer
noch Schlacht- und Rauchopfer auf den Höhen dar.~-- 5Sobald er nun die
Herrschaft fest in Händen hatte, ließ er von seinen Dienern diejenigen
hinrichten, die den König, seinen Vater, ermordet hatten. 6Aber die
Söhne der Mörder ließ er nicht hinrichten, sondern verfuhr so, wie im
Gesetzbuch Moses\textless sup title=``5.Mose 24,16''\textgreater✲
geschrieben steht, wo der HERR ausdrücklich geboten hat: »Väter sollen
nicht wegen einer Verschuldung ihrer Söhne\textless sup title=``oder:
Kinder''\textgreater✲ getötet werden, und Söhne\textless sup
title=``oder: Kinder''\textgreater✲ sollen nicht wegen einer
Verschuldung ihrer Väter getötet werden, sondern ein jeder soll nur
wegen seiner eigenen Sünde getötet werden.« 7Er war es auch, der die
Edomiter im Salztal schlug, zehntausend Mann, und Sela\textless sup
title=``d.h. den Fels''\textgreater✲ im Sturm eroberte und dem Ort den
Namen Joktheel\textless sup title=``d.h. Gottestrümmer''\textgreater✲
beilegte, den er bis auf den heutigen Tag führt.

\hypertarget{bb-amazjas-ungluxfccklicher-krieg-mit-joas-von-israel}{%
\subparagraph{bb) Amazjas unglücklicher Krieg mit Joas von
Israel}\label{bb-amazjas-ungluxfccklicher-krieg-mit-joas-von-israel}}

8Damals schickte Amazja Gesandte an den König Joas von Israel, den Sohn
des Joahas, des Sohnes Jehus, und ließ ihm sagen: »Komm, wir wollen
unsere Kräfte miteinander messen!« 9Da ließ Joas, der König von Israel,
dem König Amazja von Juda durch eine Gesandtschaft antworten: »Der
Dornstrauch auf dem Libanon sandte (einst) zu der Zeder auf dem Libanon
und ließ ihr sagen: ›Gib deine Tochter meinem Sohne zur Frau!‹, aber da
liefen die wilden Tiere auf dem Libanon über den Dornstrauch hin und
zertraten ihn. 10Weil du die Edomiter glücklich besiegt hast, ist dir
der Mut gewachsen. Begnüge dich mit dem Ruhme und bleibe zu Hause
sitzen: warum willst du das Unglück herausfordern, daß du zu Fall kommst
und Juda mit dir?« 11Da aber Amazja nicht hören wollte, zog Joas, der
König von Israel, heran, und beide maßen ihre Kräfte miteinander, er und
der König Amazja von Juda, bei Beth-Semes, das zu Juda gehört. 12Da
wurden die Judäer von den Israeliten geschlagen, so daß ein jeder in
seine Heimat floh. 13Den Amazja selber aber, den König von Juda, den
Sohn des Joas, des Sohnes Ahasjas, nahm Joas, der König von Israel, bei
Beth-Semes gefangen und ließ, als er nach Jerusalem gekommen war, ein
Stück der Mauer Jerusalems vom Ephraimstor bis zum Ecktor auf einer
Strecke von vierhundert Ellen niederreißen. 14Außerdem nahm er alles
Gold und Silber sowie alle Geräte, die sich im Tempel des HERRN und in
den Schatzkammern des königlichen Palastes vorfanden, dazu Geiseln, und
kehrte dann nach Samaria zurück.

\hypertarget{cc-schluuxdfwort-uxfcber-joas-von-israel}{%
\subparagraph{cc) Schlußwort über Joas von
Israel}\label{cc-schluuxdfwort-uxfcber-joas-von-israel}}

15Die übrige Geschichte des Joas aber, alles, was er unternommen hat,
und seine tapferen Taten und wie er mit dem König Amazja von Juda Krieg
geführt hat, das findet sich bekanntlich aufgezeichnet im Buch der
Denkwürdigkeiten\textless sup title=``oder: Chronik''\textgreater✲ der
Könige von Israel. 16Als Joas sich dann zu seinen Vätern gelegt und man
ihn in Samaria bei den Königen von Israel begraben hatte, folgte ihm
sein Sohn Jerobeam in der Regierung nach.

\hypertarget{dd-schluuxdfwort-uxfcber-amazja-von-juda-seine-ermordung}{%
\subparagraph{dd) Schlußwort über Amazja von Juda; seine
Ermordung}\label{dd-schluuxdfwort-uxfcber-amazja-von-juda-seine-ermordung}}

17Amazja aber, der Sohn des Joas, der König von Juda, überlebte den
König Joas von Israel, den Sohn des Joahas, noch fünfzehn Jahre. 18Die
übrige Geschichte Amazjas aber findet sich bekanntlich aufgezeichnet im
Buch der Denkwürdigkeiten\textless sup title=``oder:
Chronik''\textgreater✲ der Könige von Juda.~-- 19Als man aber in
Jerusalem eine Verschwörung gegen ihn anstiftete, floh er nach Lachis;
doch man sandte Leute nach Lachis hinter ihm her, die ihn dort
ermordeten. 20Dann lud man ihn auf Rosse, und er wurde in Jerusalem bei
seinen Vätern in der Davidstadt begraben.

\hypertarget{ee-regierungsantritt-asarjas}{%
\subparagraph{ee) Regierungsantritt
Asarjas}\label{ee-regierungsantritt-asarjas}}

21Hierauf nahm die ganze Bevölkerung von Juda den Asarja✲, der sechzehn
Jahre alt war, und machte ihn zum König als Nachfolger seines Vaters
Amazja. 22Er befestigte Elath, das er an Juda zurückgebracht hatte,
sogleich nachdem der König sich zu seinen Vätern gelegt hatte.

\hypertarget{von-jerobeam-ii.-in-israel-und-asarja-in-juda-bis-zum-untergang-des-reiches-israel-1423-1741}{%
\subsubsection{5. Von Jerobeam II. in Israel und Asarja in Juda bis zum
Untergang des Reiches Israel
(14,23-17,41)}\label{von-jerobeam-ii.-in-israel-und-asarja-in-juda-bis-zum-untergang-des-reiches-israel-1423-1741}}

\hypertarget{a-jerobeam-ii.-kuxf6nig-von-israel}{%
\paragraph{a) Jerobeam II. König von
Israel}\label{a-jerobeam-ii.-kuxf6nig-von-israel}}

23Im fünfzehnten Regierungsjahre des Königs Amazja von Juda, des Sohnes
des Joas, wurde Jerobeam, der Sohn des Königs Joas von Israel, König zu
Samaria und regierte einundvierzig Jahre. 24Er tat, was dem HERRN
mißfiel; er wich in keinem Stück von den Sünden Jerobeams, des Sohnes
Nebats, der Israel zur Sünde verführt hatte. 25Er stellte die Grenze
Israels wieder her von der Gegend um Hamath an bis an den großen
Steppensee (das Tote Meer), der Verheißung entsprechend, die der HERR,
der Gott Israels, durch seinen Knecht, den Propheten Jona, den Sohn
Amittais aus Gath-Hepher, gegeben hatte. 26Denn der HERR hatte das gar
bittere Elend Israels wahrgenommen und gesehen, daß Unmündige ebenso wie
Mündige dahin waren und daß kein Helfer für Israel da war. 27Auch hatte
der HERR noch nicht die Drohung ausgesprochen, daß er den Namen Israels
unter dem Himmel austilgen wolle; darum half er ihnen jetzt durch
Jerobeam, den Sohn des Joas.

28Die übrige Geschichte Jerobeams aber und alles, was er unternommen
hat, und seine tapferen Taten, wie er Krieg geführt und wie er Damaskus
und Hamath, die zu Juda gehört hatten, für Israel zurückerobert hat, das
findet sich bekanntlich aufgezeichnet im Buch der
Denkwürdigkeiten\textless sup title=``oder: Chronik''\textgreater✲ der
Könige von Israel.~-- 29Als Jerobeam sich dann zu seinen Vätern, den
Königen von Israel, gelegt hatte, folgte ihm sein Sohn Sacharja in der
Regierung nach.

\hypertarget{b-asarja-oder-ussia-kuxf6nig-von-juda}{%
\paragraph{b) Asarja (oder Ussia) König von
Juda}\label{b-asarja-oder-ussia-kuxf6nig-von-juda}}

\hypertarget{section-14}{%
\section{15}\label{section-14}}

1Im siebenundzwanzigsten\textless sup title=``oder:
siebzehnten?''\textgreater✲ Regierungsjahr Jerobeams, des Königs von
Israel, wurde Asarja✲ König, der Sohn des Königs Amazja von Juda. 2Im
Alter von sechzehn Jahren bestieg er den Thron, und zweiundfünfzig Jahre
regierte er in Jerusalem; seine Mutter hieß Jecholja und stammte aus
Jerusalem. 3Er tat, was dem HERRN wohlgefiel, ganz so wie sein Vater
Amazja getan hatte; 4jedoch der Höhendienst wurde nicht abgeschafft, das
Volk brachte immer noch Schlacht- und Rauchopfer auf den Höhen dar. 5Der
HERR aber suchte den König schwer heim, daß er bis zu seinem Todestage
am Aussatz litt und in einem Hause abgesondert für sich wohnte. Jotham
aber, der Sohn des Königs, waltete im königlichen Hause (oder Palaste)
als Familienhaupt und versah die Regierungsgeschäfte für das Land.~--
6Die übrige Geschichte Asarjas aber und alles, was er unternommen hat,
findet sich bekanntlich aufgezeichnet im Buch der
Denkwürdigkeiten\textless sup title=``oder: Chronik''\textgreater✲ der
Könige von Juda. 7Als Asarja sich dann zu seinen Vätern gelegt und man
ihn bei seinen Vätern in der Davidsstadt begraben hatte, folgte ihm sein
Sohn Jotham in der Regierung nach.

\hypertarget{c-sacharja-sallum-menahem-pekahja-und-pekah-kuxf6nige-von-israel}{%
\paragraph{c) Sacharja, Sallum, Menahem, Pekahja und Pekah Könige von
Israel}\label{c-sacharja-sallum-menahem-pekahja-und-pekah-kuxf6nige-von-israel}}

\hypertarget{aa-sacharja-kuxf6nig-von-israel}{%
\subparagraph{aa) Sacharja König von
Israel}\label{aa-sacharja-kuxf6nig-von-israel}}

8Im achtunddreißigsten Regierungsjahre des Königs Asarja von Juda wurde
Sacharja, der Sohn Jerobeams, König über Israel in Samaria und regierte
ein halbes Jahr. 9Er tat, was dem HERRN mißfiel, so wie seine Väter
getan hatten; er ließ nicht ab von den Sünden Jerobeams, des Sohnes
Nebats, der Israel zur Sünde verführt hatte. 10Da zettelte Sallum, der
Sohn des Jabes, eine Verschwörung gegen ihn an, ermordete ihn in Jibleam
und trat als König an seine Stelle. 11Die übrige Geschichte Sacharjas
aber findet sich bereits aufgezeichnet im Buch der
Denkwürdigkeiten\textless sup title=``oder: Chronik''\textgreater✲ der
Könige von Israel. 12So erfüllte sich die Verheißung, die der HERR dem
Jehu gegeben hatte mit den Worten\textless sup title=``vgl. 2.Kön
10,30''\textgreater✲: »Es sollen Nachkommen von dir bis ins vierte Glied
auf dem Thron von Israel sitzen.« So ist es auch geschehen.

\hypertarget{bb-sallum-kuxf6nig-von-israel}{%
\subparagraph{bb) Sallum König von
Israel}\label{bb-sallum-kuxf6nig-von-israel}}

13Sallum, der Sohn des Jabes, wurde König im neununddreißigsten
Regierungsjahre des Königs Ussia von Juda und regierte einen Monat lang
zu Samaria. 14Da zog Menahem, der Sohn Gadis, aus Thirza gegen ihn
heran, drang in Samaria ein, brachte Sallum, den Sohn des Jabes, in
Samaria ums Leben und trat als König an seine Stelle. 15Die übrige
Geschichte Sallums aber und die Verschwörung, die er angezettelt hatte,
das findet sich bereits aufgezeichnet im Buch der
Denkwürdigkeiten\textless sup title=``oder: Chronik''\textgreater✲ der
Könige von Israel. 16Damals verwüstete Menahem die Stadt Thappuah und
alles, was darin war, sowie ihr ganzes Gebiet von Thirza an; weil man
ihm die Tore nicht geöffnet hatte, verwüstete er es und ließ allen
Frauen dort, die guter Hoffnung waren, den Leib aufschlitzen.

\hypertarget{cc-menahem-kuxf6nig-von-israel}{%
\subparagraph{cc) Menahem König von
Israel}\label{cc-menahem-kuxf6nig-von-israel}}

17Im neununddreißigsten Regierungsjahre des Königs Asarja von Juda wurde
Menahem, der Sohn Gadis, König über Israel und regierte zehn Jahre in
Samaria. 18Er tat, was dem Herrn mißfiel; er ließ nicht ab von den
Sünden Jerobeams, des Sohnes Nebats, der Israel zur Sünde verführt
hatte. Unter seiner Regierung 19fiel der assyrische König Phul in das
Land ein, und Menahem gab dem Phul tausend Talente Silber, damit er ihm
Beistand gewähre, um ihn im Besitz des Königtums zu sichern. 20Das Geld
ließ Menahem dann von den Israeliten aufbringen, nämlich von allen
wohlhabenden Leuten, um dem König von Assyrien Zahlung leisten zu
können; fünfzig Schekel Silber kamen auf jeden Mann. Da zog der König
von Assyrien wieder ab und blieb nicht länger dort im Lande.~-- 21Die
übrige Geschichte Menahems aber und alles, was er unternommen hat,
findet sich bekanntlich aufgezeichnet im Buch der
Denkwürdigkeiten\textless sup title=``oder: Chronik''\textgreater✲ der
Könige von Israel. 22Als Menahem sich dann zu seinen Vätern gelegt
hatte, folgte ihm sein Sohn Pekahja in der Regierung nach.

\hypertarget{dd-pekahja-kuxf6nig-von-israel}{%
\subparagraph{dd) Pekahja König von
Israel}\label{dd-pekahja-kuxf6nig-von-israel}}

23Im fünfzigsten Regierungsjahre des Königs Asarja von Juda wurde
Pekahja, der Sohn Menahems, König über Israel in Samaria und regierte
zwei Jahre. 24Er tat, was dem HERRN mißfiel; er ließ nicht ab von den
Sünden Jerobeams, des Sohnes Nebats, der Israel zur Sünde verführt
hatte. 25Da zettelte sein Ritter\textless sup title=``vgl. 2.Kön
7,2''\textgreater✲ Pekah, der Sohn Remaljas, eine Verschwörung gegen ihn
an und ermordete ihn zu Samaria in der Burg\textless sup title=``oder:
im Turm''\textgreater✲ des königlichen Palastes {[}zugleich auch den
Argob und den Arje{]}, zur Seite standen ihm dabei fünfzig Mann von den
Gileaditern. Nachdem er ihn getötet hatte, folgte er ihm als König in
der Regierung nach.~-- 26Die übrige Geschichte Pekahjas aber und alles,
was er unternommen hat, das findet sich bereits aufgezeichnet im Buch
der Denkwürdigkeiten\textless sup title=``oder: Chronik''\textgreater✲
der Könige von Israel.

\hypertarget{ee-pekah-kuxf6nig-von-israel}{%
\subparagraph{ee) Pekah König von
Israel}\label{ee-pekah-kuxf6nig-von-israel}}

27Im zweiundfünfzigsten Regierungsjahre des Königs Asarja von Juda wurde
Pekah, der Sohn Remaljas, König über Israel in Samaria und regierte
zwanzig Jahre. 28Er tat, was dem HERRN mißfiel; er ließ nicht ab von den
Sünden Jerobeams, des Sohnes Nebats, der Israel zur Sünde verführt
hatte. 29Unter der Regierung Pekahs, des Königs von Israel, zog
Thiglath-Pileser, der König von Assyrien, heran und eroberte Ijjon,
Abel-Beth-Maacha, Janoah, Kedes und Hazor, Gilead und Galiläa, das ganze
Land Naphthali, und führte die Bewohner in die
Gefangenschaft\textless sup title=``oder: Verbannung''\textgreater✲ nach
Assyrien. 30Da zettelte Hosea, der Sohn Elas, eine Verschwörung gegen
Pekah, den Sohn Remaljas an, brachte ihn ums Leben und trat dann als
König an seine Stelle im zwanzigsten Regierungsjahre Jothams, des Sohnes
Ussias.~-- 31Die übrige Geschichte Pekahs aber und alles, was er
unternommen hat, das findet sich bereits aufgezeichnet im Buch der
Denkwürdigkeiten\textless sup title=``oder: Chronik''\textgreater✲ der
Könige von Israel.

\hypertarget{d-jotham-kuxf6nig-von-juda}{%
\paragraph{d) Jotham König von Juda}\label{d-jotham-kuxf6nig-von-juda}}

32Im zweiten Regierungsjahre Pekahs, des Sohnes des Königs Remalja von
Israel, wurde Jotham, der Sohn Ussias, König über Juda. 33Im Alter von
fünfundzwanzig Jahren wurde er König, und sechzehn Jahre regierte er in
Jerusalem; seine Mutter hieß Jerusa und war eine Tochter Zadoks. 34Er
tat, was dem HERRN wohlgefiel, ganz wie sein Vater Ussia getan hatte;
35jedoch der Höhendienst wurde nicht abgeschafft; das Volk brachte immer
noch Schlacht- und Rauchopfer auf den Höhen dar. Er baute das obere Tor
am Tempel des HERRN. 36Die übrige Geschichte Jothams aber und alles, was
er unternommen hat, das findet sich bekanntlich aufgezeichnet im Buch
der Denkwürdigkeiten\textless sup title=``oder: Chronik''\textgreater✲
der Könige von Juda. 37Zu jener Zeit begann der HERR den König Rezin von
Syrien und Pekah, den Sohn Remaljas, gegen Juda vorgehen zu lassen.
38Als Jotham sich dann zu seinen Vätern gelegt und man ihn bei seinen
Vätern in der Stadt seines Ahnherrn David begraben hatte, folgte ihm
sein Sohn Ahas als König in der Regierung nach.

\hypertarget{e-ahas-kuxf6nig-von-juda}{%
\paragraph{e) Ahas König von Juda}\label{e-ahas-kuxf6nig-von-juda}}

\hypertarget{aa-des-ahas-heidnische-greuel}{%
\subparagraph{aa) Des Ahas heidnische
Greuel}\label{aa-des-ahas-heidnische-greuel}}

\hypertarget{section-15}{%
\section{16}\label{section-15}}

1Im siebzehnten Regierungsjahre Pekahs, des Sohnes Remaljas, wurde Ahas
König, der Sohn des Königs Jotham von Juda. 2Im Alter von zwanzig Jahren
wurde Ahas König, und sechzehn Jahre regierte er in Jerusalem. Er tat
nicht, was dem HERRN, seinem Gott, wohlgefiel, wie sein Ahnherr David
getan hatte, 3sondern er wandelte auf dem Wege der Könige von Israel,
ja, er ließ sogar seinen Sohn als Opfer verbrennen nach der grauenhaften
Sitte der heidnischen Völker, die der HERR vor den Israeliten vertrieben
hatte. 4Er brachte auch Schlacht- und Rauchopfer dar auf den Höhen und
auf den Hügeln und unter jedem dichtbelaubten Baum.

\hypertarget{bb-sein-krieg-mit-syrien-und-israel-ahas-wird-den-assyrern-tributpflichtig}{%
\subparagraph{bb) Sein Krieg mit Syrien und Israel; Ahas wird den
Assyrern
tributpflichtig}\label{bb-sein-krieg-mit-syrien-und-israel-ahas-wird-den-assyrern-tributpflichtig}}

5Damals zogen Rezin, der König von Syrien, und Pekah, der Sohn Remaljas,
der König von Israel, zum Angriff gegen Jerusalem heran und belagerten
(die Stadt), waren aber nicht imstande, sie zu erstürmen. 6Zu jener Zeit
brachte der König von Edom Elath wieder an Edom und vertrieb die Judäer
aus Elath; da kamen die Edomiter nach Elath zurück und sind dort wohnen
geblieben bis auf den heutigen Tag. 7Ahas aber schickte eine
Gesandtschaft an Thiglath-Pileser, den König von Assyrien, und ließ ihm
sagen: »Ich bin dein Knecht und dein Sohn: komm mir zu Hilfe und rette
mich aus der Gewalt des Königs von Syrien und aus der Gewalt des Königs
von Israel, die gegen mich zu Felde gezogen sind!« 8Zugleich nahm Ahas
das Silber und Gold, das sich im Tempel des HERRN und in den
Schatzkammern des königlichen Palastes vorfand, und sandte es an den
König von Assyrien als Geschenk. 9Der König von Assyrien kam dann auch
seiner Aufforderung nach, rückte gegen Damaskus, eroberte es und führte
(die Einwohner) gefangen nach Kir; den König Rezin aber ließ er
hinrichten.

\hypertarget{cc-ahas-luxe4uxdft-einen-neuen-brandopferaltar-bauen-erluxe4uxdft-eine-neue-opferordnung-und-greift-in-den-tempelbesitz-ein}{%
\subparagraph{cc) Ahas läßt einen neuen Brandopferaltar bauen, erläßt
eine neue Opferordnung und greift in den Tempelbesitz
ein}\label{cc-ahas-luxe4uxdft-einen-neuen-brandopferaltar-bauen-erluxe4uxdft-eine-neue-opferordnung-und-greift-in-den-tempelbesitz-ein}}

10Als sich nun der König Ahas nach Damaskus begeben hatte, um dort mit
Thiglath-Pileser, dem König von Assyrien, zusammenzutreffen, und den
Altar sah, der in Damaskus stand, sandte der König Ahas dem Priester
Uria die Maße\textless sup title=``oder: eine Zeichnung''\textgreater✲
des Altars und eine bis in alle Einzelheiten ausgeführte Nachbildung.
11Da ließ der Priester Uria einen Altar genau nach dem Vorbild✲ bauen,
das ihm der König Ahas aus Damaskus hatte zugehen lassen; so ließ der
Priester Uria ihn herstellen, ehe noch der König Ahas aus Damaskus
zurückkehrte. 12Als dann der König nach seiner Rückkehr aus Damaskus den
Altar sah, trat der König an den Altar heran und stieg zu ihm hinauf,
13ließ dann selber sein Brand- und Speisopfer in Rauch aufgehen, goß
sein Trankopfer aus und sprengte das Blut seiner Heilsopfer an den
Altar. 14Den ehernen Altar aber, der vor dem (Tempel des) HERRN stand,
den ließ er von der Vorderseite des Tempels, von der Stelle zwischen dem
(neuen) Altar und dem Tempel des HERRN, entfernen und ihn auf die
Nordseite des (neuen) Altars versetzen. 15Sodann erteilte der König Ahas
dem Priester Uria folgenden Befehl: »Auf dem großen (neuen) Altar
verbrenne das Morgenbrandopfer und das Abendspeisopfer sowie das
Brandopfer des Königs nebst seinem Speisopfer, ebenso auch die
Brandopfer aller Privatleute des Landes samt ihren Speis- und
Trankopfern, und sprenge alles Blut der Brandopfer und alles Blut der
Schlachtopfer an diesen Altar; in betreff des ehernen Altars aber will
ich mich noch bedenken.« 16Da verfuhr der Priester Uria genau nach dem
Befehl des Königs Ahas.

17Sodann ließ der König Ahas die Beschläge✲ an den
Gestühlen\textless sup title=``vgl. 1.Kön 7,27-39''\textgreater✲
herausbrechen und die Kessel von ihnen herunternehmen; auch das große
Wasserbecken ließ er von den ehernen Rindern, auf denen es
ruhte\textless sup title=``vgl. 1.Kön 7,23-26''\textgreater✲,
herabnehmen und es auf eine Unterlage von Steinen setzen. 18Weiter ließ
er den überdeckten Sabbat-Gang, den man im Tempel erbaut hatte, und den
äußeren Königszugang am Tempel des HERRN beseitigen mit Rücksicht auf
den König von Assyrien.

\hypertarget{dd-schluuxdfwort}{%
\subparagraph{dd) Schlußwort}\label{dd-schluuxdfwort}}

19Die übrige Geschichte des Ahas aber und alles, was er unternommen hat,
findet sich bekanntlich aufgezeichnet im Buch der
Denkwürdigkeiten\textless sup title=``oder: Chronik''\textgreater✲ der
Könige von Juda. 20Als Ahas sich dann zu seinen Vätern gelegt und man
ihn bei seinen Vätern in der Davidstadt begraben hatte, trat sein Sohn
Hiskia als König an seine Stelle.

\hypertarget{f-das-ende-des-reiches-israel}{%
\paragraph{f) Das Ende des Reiches
Israel}\label{f-das-ende-des-reiches-israel}}

\hypertarget{aa-hosea-kuxf6nig-von-israel-untergang-des-reiches-assyrische-gefangenschaft-oder-verbannung}{%
\subparagraph{aa) Hosea König von Israel; Untergang des Reiches;
assyrische Gefangenschaft (oder
Verbannung)}\label{aa-hosea-kuxf6nig-von-israel-untergang-des-reiches-assyrische-gefangenschaft-oder-verbannung}}

\hypertarget{section-16}{%
\section{17}\label{section-16}}

1Im zwölften Jahre des Königs Ahas von Juda wurde Hosea, der Sohn Elas,
König über Israel in Samaria und regierte neun Jahre. 2Er tat, was dem
HERRN mißfiel, wenn auch nicht so arg wie die Könige von Israel, die vor
ihm regiert hatten. 3Gegen ihn zog Salmanassar, der König von Assyrien,
heran; Hosea unterwarf sich ihm und wurde ihm tributpflichtig. 4Als aber
der König von Assyrien eine Verschwörung Hoseas entdeckte -- er hatte
nämlich Gesandte an den König Sewe von Ägypten geschickt und dem Könige
von Assyrien nicht mehr wie sonst alljährlich den Tribut entrichtet --,
da ließ der König von Assyrien ihn festnehmen und gefesselt ins
Gefängnis werfen. 5Der König von Assyrien hatte nämlich das ganze Land
mit Krieg überzogen, war vor Samaria gerückt und hatte es drei Jahre
lang belagert. 6Als der König von Assyrien dann Samaria im neunten
Regierungsjahre Hoseas erobert hatte, führte er die Israeliten in die
Gefangenschaft\textless sup title=``oder: Verbannung''\textgreater✲ nach
Assyrien und wies ihnen Wohnsitze in Halah und am Habor, dem Flusse
Gosans, sowie in den Ortschaften Mediens an.

\hypertarget{bb-die-ursachen-welche-die-verwerfung-und-den-untergang-des-nordreiches-herbeigefuxfchrt-haben}{%
\subparagraph{bb) Die Ursachen, welche die Verwerfung und den Untergang
des Nordreiches herbeigeführt
haben}\label{bb-die-ursachen-welche-die-verwerfung-und-den-untergang-des-nordreiches-herbeigefuxfchrt-haben}}

7Das ist aber geschehen, weil die Israeliten sich am HERRN, ihrem Gott,
versündigt hatten, der sie aus Ägypten aus der Gewalt des Pharaos, des
ägyptischen Königs, weggeführt hatte, und weil sie andere Götter verehrt
hatten 8und nach den Satzungen der Heidenvölker gewandelt waren, die der
HERR vor den Israeliten vertrieben hatte, und nach den Bräuchen, welche
die Könige von Israel eingeführt hatten. 9So hatten denn die Israeliten
gegen den Willen des HERRN, ihres Gottes, Dinge getrieben, die nicht
recht waren; denn sie hatten sich Höhenheiligtümer in allen ihren
Ortschaften erbaut, von den Wachttürmen an bis zu den festen Städten,
10und hatten sich Malsteine und Götzensäulen auf jedem hohen Hügel und
unter jedem dichtbelaubten Baume errichtet 11und dort auf allen Höhen
geopfert wie die heidnischen Völkerschaften, die der HERR vor ihnen
vertrieben hatte. Sie hatten also böse Dinge verübt, um den HERRN zum
Zorn zu reizen, 12und den Götzen gedient, in betreff deren der HERR
ihnen geboten hatte: »Ihr dürft so etwas nicht tun!« 13Und der HERR
hatte doch Israel und Juda durch den Mund aller Propheten, aller Seher
warnen lassen, indem er ihnen vorhielt: »Kehrt von euren bösen Wegen um
und haltet meine Gebote und meine Verordnungen genau nach der Weisung,
die ich euren Vätern gegeben und die ich euch durch meine Knechte, die
Propheten, habe zukommen lassen!« 14Aber sie hatten nicht hören wollen,
sondern sich halsstarrig gezeigt, wie auch ihre Väter, die dem HERRN,
ihrem Gott, nicht vertraut hatten; 15sie mißachteten seine Satzungen und
seinen Bund, den er mit ihren Vätern geschlossen, und seine Warnungen,
die er an sie gerichtet hatte; sie liefen vielmehr hinter den nichtigen
Götzen her und wandten sich einem nichtigen Treiben zu nach dem Vorbild
der heidnischen Völkerschaften, die um sie her wohnten, bezüglich deren
der HERR ihnen geboten hatte, es nicht so zu machen wie jene. 16Sie
vernachlässigten alle Gebote des HERRN, ihres Gottes, und fertigten sich
zwei gegossene Stierbilder an, verfertigten sich Götzensäulen, beteten
das ganze Sternenheer des Himmels an und dienten dem Baal. 17Sie
verbrannten auch ihre Söhne und Töchter als Opfer, trieben Wahrsagerei
und Zauberei und gaben sich dazu her, das, was dem HERRN mißfiel, zu
verüben, um ihn zu erbittern. 18Da geriet der HERR in heftigen Zorn
gegen die Israeliten und verstieß sie von seinem Angesicht, so daß
nichts übrigblieb als der Stamm Juda allein. 19Aber auch die dem Stamm
Juda Angehörigen beobachteten die Gebote des HERRN, ihres Gottes, nicht,
sondern wandelten in den Bräuchen, welche die Israeliten in Aufnahme
gebracht hatten. 20Da verwarf denn der HERR die ganze Nachkommenschaft
Israels, demütigte sie und ließ sie in die Gewalt von Räubern fallen,
bis er sie ganz von seinem Angesicht verstoßen hatte.

\hypertarget{cc-jerobeams-verhuxe4ngnisvolle-schuld}{%
\subparagraph{cc) Jerobeams verhängnisvolle
Schuld}\label{cc-jerobeams-verhuxe4ngnisvolle-schuld}}

21Als nämlich Israel sich vom Hause Davids losgerissen und sie Jerobeam,
den Sohn Nebats, zum König gemacht hatten, da hatte Jerobeam die
Israeliten zum Abfall vom HERRN veranlaßt und sie zu schwerer Sünde
verführt. 22So waren denn die Israeliten in allen Sünden gewandelt, die
Jerobeam begangen hatte; sie waren nicht davon abgegangen, 23bis der
HERR die Israeliten von seinem Angesicht verstieß, wie er es durch den
Mund aller seiner Knechte, der Propheten, angedroht hatte. So wurden
also die Israeliten aus ihrem Lande nach Assyrien weggeführt, wo sie
sich bis auf den heutigen Tag befinden.

\hypertarget{g-neubesiedelung-des-landes-entstehung-der-samariter-und-ihrer-religion}{%
\paragraph{g) Neubesiedelung des Landes; Entstehung der Samariter und
ihrer
Religion}\label{g-neubesiedelung-des-landes-entstehung-der-samariter-und-ihrer-religion}}

24Der König von Assyrien aber ließ Leute aus Babylon, Kutha, Awwa,
Hamath und Sepharwaim kommen und siedelte sie an Stelle der Israeliten
in den Ortschaften Samarias an; die nahmen Besitz von Samaria und
wohnten in den dortigen Ortschaften. 25Weil sie aber in der ersten Zeit
ihrer Niederlassung daselbst den HERRN nicht verehrten, ließ der HERR
Löwen gegen sie los, welche Verheerungen unter ihnen anrichteten. 26Da
machte man dem König von Assyrien die Meldung: »Die Völkerschaften, die
du aus ihrer Heimat hast auswandern lassen und in den Ortschaften
Samarias angesiedelt hast, verstehen sich nicht auf die dem Landesgott
gebührende Verehrung; daher hat er Löwen gegen sie losgelassen, die nun
Verheerungen unter ihnen anrichten, weil sie sich nicht auf die dem
Landesgott gebührende Verehrung verstehen.« 27Da erließ der König von
Assyrien folgenden Befehl: »Laßt einen von den Priestern, die ihr von
dort weggeführt habt, dorthin zurückkehren; der soll sich dorthin
begeben und dort wohnen und sie in der dem Landesgott gebührenden
Verehrung unterweisen!« 28So kam denn einer von den Priestern, die man
aus Samaria weggeführt hatte, zurück, ließ sich in Bethel nieder und
unterwies sie, wie sie den HERRN zu verehren hätten. 29Dabei machten sie
sich aber, jede einzelne Völkerschaft, ihren besonderen Gott und
stellten diesen in den Höhentempeln auf, welche die
Samarier\textless sup title=``oder: Samaritaner''\textgreater✲ erbaut
hatten, eine jede Völkerschaft in ihren Ortschaften, in denen sie
angesiedelt waren. 30Die Leute von Babylon z.B. fertigten ein Bild
des\textless sup title=``oder: der?''\textgreater✲ Sukkoth-Benoth an,
die Leute aus Kuth\textless sup title=``vgl. V.24''\textgreater✲ ein
Bild des Nergal, die Leute aus Hamath ein Bild des\textless sup
title=``oder: der?''\textgreater✲ Asima; 31die Awwiter fertigten ein
Bild des Nibhas und Tharthak an, und die Sepharwiter verbrannten ihre
Kinder als Feueropfer zu Ehren Adrammelechs und Anammelechs, der Götter
von Sepharwaim. 32Daneben verehrten sie (jetzt) aber auch Gott den HERRN
und bestellten sich Leute aus ihrer Mitte zu Höhenpriestern, die für sie
Opfer in den Höhenheiligtümern darbrachten. 33So waren sie zwar Verehrer
des HERRN, blieben zugleich aber auch Diener ihrer Götter nach den
Bräuchen der Völkerschaften, aus denen man sie weggeführt hatte.

34Bis auf den heutigen Tag verfahren sie nach den ursprünglichen
Bräuchen: sie verehren den HERRN nicht in rechter Weise und verfahren
nicht nach seinen Satzungen und seinen Verordnungen, nicht nach dem
Gesetz und den Geboten, die der HERR den Nachkommen Jakobs geboten hat,
dem er den Namen ›Israel‹ beilegte. 35Und doch hatte der HERR einen Bund
mit ihnen geschlossen und ihnen ausdrücklich geboten: »Ihr dürft keine
anderen Götter verehren und dürft sie nicht anbeten und ihnen nicht
dienen noch opfern; 36sondern Gott den HERRN, der euch aus Ägypten mit
großer Kraft und hocherhobenem Arm hergeführt hat, den sollt ihr
verehren und den sollt ihr anbeten und ihm opfern! 37Die Satzungen und
Verordnungen aber, das Gesetz und die Gebote, die er euch vorgeschrieben
hat, müßt ihr beobachten, daß ihr allezeit danach tut, und dürft keine
anderen Götter verehren! 38Und des Bundes, den ich mit euch geschlossen
habe, dürft ihr nicht vergessen und dürft keine anderen Götter verehren;
39sondern den HERRN, euren Gott, sollt ihr verehren: er ist es, der euch
aus der Gewalt aller eurer Feinde erretten wird.« 40Aber sie wollten
nicht darauf hören, sondern taten nach ihrer früheren Weise.~-- 41So
waren also diese Völkerschaften zwar Verehrer des HERRN, zugleich aber
Diener ihrer Götzenbilder; auch ihre Kinder und Kindeskinder halten es
so, wie ihre Väter getan haben, bis auf den heutigen Tag.

\hypertarget{ii.-geschichte-des-reiches-juda-von-der-regierung-hiskias-bis-zum-untergang-des-reiches-kap.-18-25}{%
\subsection{II. Geschichte des Reiches Juda von der Regierung Hiskias
bis zum Untergang des Reiches (Kap.
18-25)}\label{ii.-geschichte-des-reiches-juda-von-der-regierung-hiskias-bis-zum-untergang-des-reiches-kap.-18-25}}

\hypertarget{der-kuxf6nig-hiskia-und-der-prophet-jesaja}{%
\subsubsection{1. Der König Hiskia und der Prophet
Jesaja}\label{der-kuxf6nig-hiskia-und-der-prophet-jesaja}}

\hypertarget{a-hiskias-regierungsantritt-seine-fruxf6mmigkeit-und-seine-verdienste-um-den-gottesdienst-und-das-staatswohl}{%
\paragraph{a) Hiskias Regierungsantritt, seine Frömmigkeit und seine
Verdienste um den Gottesdienst und das
Staatswohl}\label{a-hiskias-regierungsantritt-seine-fruxf6mmigkeit-und-seine-verdienste-um-den-gottesdienst-und-das-staatswohl}}

\hypertarget{section-17}{%
\section{18}\label{section-17}}

1Darauf, im dritten Regierungsjahre Hoseas, des Sohnes Elas, des Königs
von Israel, wurde Hiskia König, der Sohn des Königs Ahas von Juda. 2Im
Alter von fünfundzwanzig Jahren wurde er König, und neunundzwanzig Jahre
regierte er in Jerusalem; seine Mutter hieß Abi und war die Tochter
Sacharjas. 3Er tat, was dem HERRN wohlgefiel, ganz so wie sein Ahnherr
David getan hatte. 4Er war es, der den Höhendienst abschaffte, die
Malsteine zertrümmerte, die Götzenbäume umhieb und die eherne Schlange
zerschlug, die Mose angefertigt hatte\textless sup title=``vgl. 4.Mose
21,8-9''\textgreater✲; denn bis zu dieser Zeit hatten die Israeliten ihr
immerfort geopfert, und man nannte sie Nehustan\textless sup
title=``d.h. Erzbild''\textgreater✲. 5Er setzte sein Vertrauen auf den
HERRN, den Gott Israels, so daß unter allen Königen von Juda weder nach
ihm noch unter denen, die vor ihm gewesen waren, irgendeiner ihm
gleichgekommen ist. 6Er hielt am HERRN fest, ohne von ihm abzuweichen,
und beobachtete seine Gebote, die der HERR dem Mose gegeben hatte. 7So
war denn auch der HERR mit ihm, so daß er bei allen seinen
Unternehmungen Glück hatte. Er fiel auch vom König von Assyrien ab und
machte sich unabhängig von ihm. 8Er schlug auch die Philister bis nach
Gaza hin, und zwar bis an die Grenze dieser Stadt, vom Wächterturm an
bis zur befestigten Stadt.

\hypertarget{samarias-untergang}{%
\paragraph{Samarias Untergang}\label{samarias-untergang}}

9Im vierten Regierungsjahre des Königs Hiskia aber -- das war das siebte
Regierungsjahr des Königs Hosea von Israel, des Sohnes Elas -- zog der
assyrische König Salmanassar gegen Samaria heran, belagerte es 10und
nahm es nach Ablauf von drei Jahren ein; im sechsten Regierungsjahr
Hiskias -- das war das neunte Regierungsjahr des Königs Hosea von Israel
-- wurde Samaria erobert. 11Der König von Assyrien führte dann die
Israeliten in die Gefangenschaft\textless sup title=``oder:
Verbannung''\textgreater✲ nach Assyrien und verpflanzte sie nach Halah
und an den Habor, den Fluß Gosans, und in die Ortschaften der Meder,
12zur Strafe dafür, daß sie den Weisungen des HERRN, ihres Gottes, nicht
nachgekommen waren und seinen Bund übertreten hatten, alles was Mose,
der Knecht des HERRN, ihnen geboten hatte: sie hatten weder darauf
gehört noch danach getan.

\hypertarget{b-sanheribs-einfall-in-juda-jerusalems-wunderbare-rettung}{%
\paragraph{b) Sanheribs Einfall in Juda; Jerusalems wunderbare
Rettung}\label{b-sanheribs-einfall-in-juda-jerusalems-wunderbare-rettung}}

\hypertarget{aa-hiskia-schickt-erfolglos-den-von-sanherib-geforderten-tribut}{%
\subparagraph{aa) Hiskia schickt erfolglos den von Sanherib geforderten
Tribut}\label{aa-hiskia-schickt-erfolglos-den-von-sanherib-geforderten-tribut}}

13Im vierzehnten Regierungsjahr des Königs Hiskia aber zog der
assyrische König Sanherib gegen alle festen Städte Judas heran und
eroberte sie. 14Da schickte der König Hiskia von Juda eine Gesandtschaft
an den König von Assyrien nach Lachis und ließ ihm sagen: »Ich habe
unrecht getan; ziehe aus meinem Lande wieder ab! Was du mir auferlegst,
will ich tragen.« Da legte der König von Assyrien dem Könige Hiskia von
Juda die Zahlung von dreihundert Talenten Silber und dreißig Talenten
Gold auf; 15und Hiskia gab alles Silber hin, das sich im Tempel des
HERRN und in den Schatzkammern des königlichen Palastes vorfand. 16Zu
jener Zeit ließ Hiskia von den Türen im Tempel des HERRN und von den
Pfeilern, die er selbst mit Goldblech hatte überziehen lassen, das Gold
abnehmen und gab es dem König von Assyrien.

\hypertarget{bb-sanherib-luxe4uxdft-von-lachis-aus-die-stadt-jerusalem-durch-seinen-grouxdfwesir-oberfeldherrn-huxf6hnisch-zur-uxfcbergabe-auffordern}{%
\subparagraph{bb) Sanherib läßt von Lachis aus die Stadt Jerusalem durch
seinen Großwesir (=~Oberfeldherrn) höhnisch zur Übergabe
auffordern}\label{bb-sanherib-luxe4uxdft-von-lachis-aus-die-stadt-jerusalem-durch-seinen-grouxdfwesir-oberfeldherrn-huxf6hnisch-zur-uxfcbergabe-auffordern}}

17Aber der assyrische König sandte seinen Großwesir✲ und den
Oberkämmerer und den Obermundschenk mit einem starken Heere von Lachis
aus gegen den König Hiskia nach Jerusalem. Als diese hinaufgezogen und
vor Jerusalem angekommen waren, stellten sie sich bei der
Wasserleitung\textless sup title=``oder: am Wasserabfluß''\textgreater✲
des oberen Teiches auf, der an der Straße nach dem Walkerfeld liegt.
18Als sie nun den König zu sprechen verlangten, ging der Hausminister
Eljakim, der Sohn Hilkias, mit dem Staatsschreiber Sebna und dem Kanzler
Joah, dem Sohne Asaphs, zu ihnen hinaus. 19Da sagte der Obermundschenk
zu ihnen: »Berichtet dem Hiskia: So hat der Großkönig, der König von
Assyrien, gesprochen: ›Worauf beruht denn das feste Vertrauen, das du
hegst? 20Meinst du etwa, der Verlauf und Ausgang eines Krieges hänge
lediglich von Worten ab? Auf wen verläßt du dich eigentlich, daß du dich
gegen mich empört hast? 21Nun ja, du verläßt dich auf Ägypten, auf
diesen eingeknickten Rohrstab, der jedem, welcher sich darauf stützt, in
die Hand fährt und sie durchbohrt: so erweist sich nämlich der Pharao,
der König von Ägypten, allen denen, die sich auf ihn verlassen. 22Wenn
ihr mir aber entgegnen wollt: Auf den HERRN, unsern Gott, verlassen wir
uns! -- ist das nicht derselbe, dessen Höhen(- dienst) und Altäre Hiskia
beseitigt hat, als er in Juda und Jerusalem den Befehl erließ: Nur vor
dem Altar hier in Jerusalem dürft ihr euch niederwerfen? 23Und nun, gehe
doch mit meinem Herrn, dem König von Assyrien, eine Wette ein: ich will
dir zweitausend Pferde liefern, ob du wohl imstande bist, die Reiter für
sie aufzubringen! 24Wie willst du da den Kampf mit einem einzigen
Befehlshaber von den geringsten Dienern meines Herrn aufnehmen? Und doch
verläßt du dich auf Ägypten um der Wagen und Reiter willen! 25Zudem: bin
ich etwa ohne Zutun des HERRN, eures Gottes, gegen diesen Ort
herangezogen, um ihn zu verheeren? Der HERR selbst hat mich
aufgefordert, gegen dieses Land zu ziehen und es zu verheeren!‹«

\hypertarget{cc-sanheribs-und-seiner-gesandten-hochmut}{%
\subparagraph{cc) Sanheribs und seiner Gesandten
Hochmut}\label{cc-sanheribs-und-seiner-gesandten-hochmut}}

26Hierauf sagten Eljakim, der Sohn Hilkias, und Sebna und Joah zu dem
Obermundschenk: »Sprich doch aramäisch mit deinen Knechten, denn wir
verstehen es, und sprich nicht judäisch mit uns vor den Ohren des
Volkes, das auf der Mauer steht!« 27Aber der Obermundschenk erwiderte
ihnen: »Hat mich mein Herr etwa nur zu deinem Herrn und zu dir gesandt,
um diese Verhandlungen zu führen, und nicht auch zu den Männern, die
dort auf der Mauer sitzen, um schließlich mit euch zusammen ihren
eigenen Kot zu verzehren und ihren Harn zu trinken?« 28Hierauf trat der
Obermundschenk vor und rief mit gehobener Stimme auf judäisch die Worte
aus: »Vernehmt die Botschaft des Großkönigs, des Königs von Assyrien!
29So läßt euch der König sagen: ›Laßt euch von Hiskia nicht täuschen!
Denn er vermag euch nicht aus meiner Gewalt zu erretten; 30auch laßt
euch von Hiskia nicht auf Gott den HERRN vertrösten, wenn er sagt: Gott
der HERR wird uns sicherlich erretten, und unsere Stadt wird nicht in
die Hand des Königs von Assyrien fallen! 31Hört nicht auf Hiskia! Denn
so läßt euch der König von Assyrien sagen: Schließt Frieden mit mir und
ergebt euch mir! Dann sollt ihr ein jeder von seinem eigenen Weinstock
und seinem eigenen Feigenbaum essen und ein jeder das Wasser aus seiner
eigenen Zisterne trinken, 32bis ich komme, um euch in ein Land
mitzunehmen, das gleich dem eurigen ist, ein Land voll von Getreide und
Most, ein Land voll von Brot und Weinbergen, ein Land voll von Ölbäumen
und Honig, damit ihr am Leben bleibt und nicht zu sterben braucht. Aber
hört nicht auf Hiskia! Denn er will euch nur betören, wenn er sagt: Gott
der HERR wird uns erretten! 33Hat etwa von den Göttern der anderen
Völker irgendeiner sein Land aus der Gewalt des Königs von Assyrien
gerettet? 34Wo sind\textless sup title=``oder: waren''\textgreater✲ die
Götter von Hamath und Arpad? Wo die Götter von Sepharwaim, von Hena und
Iwwa? Und ebensowenig haben sie Samaria aus meiner Gewalt gerettet. 35Wo
ist unter allen Göttern der Länder ein einziger, der sein Land aus
meiner Gewalt gerettet hätte, daß jetzt Gott der HERR Jerusalem aus
meiner Gewalt erretten sollte?‹« 36Da schwieg das Volk still und
antwortete ihm kein Wort; denn es lag ein Befehl des Königs vor, der
geboten hatte: »Ihr sollt ihm nicht antworten!«~-- 37Hierauf kehrten der
Hausminister Eljakim, der Sohn Hilkias, und der Staatsschreiber Sebna
und der Kanzler Joah, der Sohn Asaphs, mit zerrissenen Kleidern zu
Hiskia zurück und berichteten ihm, was der Obermundschenk gesagt hatte.

\hypertarget{dd-ermutigung-hiskias-durch-jesaja}{%
\subparagraph{dd) Ermutigung Hiskias durch
Jesaja}\label{dd-ermutigung-hiskias-durch-jesaja}}

\hypertarget{section-18}{%
\section{19}\label{section-18}}

1Als nun der König Hiskia das vernommen hatte, zerriß er seine Kleider,
hüllte sich in Sackleinen\textless sup title=``=~ein grobes
Trauergewand''\textgreater✲ und begab sich in den Tempel des HERRN;
2seinen Hausminister Eljakim aber und den Staatsschreiber Sebna samt den
vornehmsten Priestern sandte er, ebenfalls in Trauerkleider gehüllt, zu
dem Propheten Jesaja, dem Sohne des Amoz. 3Diese sagten zu ihm: »So läßt
Hiskia dir sagen: ›Ein Tag der Bedrängnis, der Züchtigung und Schmach
ist der heutige Tag; denn die Kinder sind bis zum Muttermund gekommen,
aber es fehlt an Kraft zum Gebären. 4Vielleicht aber wird der HERR, dein
Gott, auf alle Worte des Großwesirs hören, den sein Herr, der König von
Assyrien, hergesandt hat, um den lebendigen Gott zu verhöhnen, und wird
ihn für die Worte strafen, die der HERR, dein Gott, gehört hat. So lege
denn Fürbitte ein für den Rest, der noch vorhanden ist!‹« 5Als nun die
Diener des Königs Hiskia zu Jesaja gekommen waren, 6sagte dieser zu
ihnen: »Bringt eurem Herrn folgenden Bescheid: ›So hat der HERR
gesprochen: Fürchte dich nicht vor den Reden, die du gehört hast, mit
denen die Diener des Königs von Assyrien mich geschmäht haben! 7Wisse
wohl: ich will ihm einen Geist\textless sup title=``oder: den
Entschluß''\textgreater✲ eingeben, daß er, wenn er eine Kunde erhält, in
sein Land zurückkehrt, und ich will ihn dann in seinem eigenen Lande
durch das Schwert umkommen lassen.‹«

\hypertarget{ee-zweite-aufforderung-sanheribs-durch-ein-drohschreiben-von-libna-aus}{%
\subparagraph{ee) Zweite Aufforderung Sanheribs durch ein Drohschreiben
von Libna
aus}\label{ee-zweite-aufforderung-sanheribs-durch-ein-drohschreiben-von-libna-aus}}

8Als hierauf der Großwesir zurückgekehrt war, fand er den König von
Assyrien mit der Belagerung von Libna beschäftigt; er hatte nämlich
erfahren, daß jener von Lachis abgezogen war. 9Als (Sanherib) sodann in
betreff des äthiopischen Königs Thirhaka die Nachricht erhielt, dieser
sei zum Kriege gegen ihn ausgezogen, sandte er nochmals Boten an Hiskia
und ließ ihm sagen: 10»Folgende Botschaft sollt ihr dem König Hiskia von
Juda überbringen: Laß dich nicht von deinem Gott täuschen, auf den du
dein Vertrauen setzt, indem du meinst: ›Jerusalem wird nicht in die
Gewalt des Königs von Assyrien fallen!‹ 11Du hast doch selbst gehört,
wie die Könige von Assyrien mit allen Ländern verfahren sind, indem sie
den Bann an ihnen vollstreckten, und da solltest du gerettet werden?
12Haben etwa die Götter der Völkerschaften, welche meine Väter
vernichtet haben, diese Völker errettet: Gosan, Haran und Rezeph und die
Bewohner von Eden in Thelassar? 13Wo ist der König von Hamath, der König
von Arpad und der König von Lair und Sepharwaim, von Hena und Iwwa?«

\hypertarget{ff-hiskias-bittgebet-im-tempel}{%
\subparagraph{ff) Hiskias Bittgebet im
Tempel}\label{ff-hiskias-bittgebet-im-tempel}}

14Als nun Hiskia das Schreiben aus der Hand der Gesandten in Empfang
genommen und es gelesen hatte, ging er in den Tempel des HERRN hinauf
und breitete es dort vor dem HERRN aus. 15Alsdann betete Hiskia vor dem
HERRN mit den Worten: »HERR, Gott Israels, der du über den Cheruben
thronst! Du allein bist der Gott über alle Reiche auf Erden; du bist's,
der Himmel und Erde geschaffen hat! 16Neige, HERR, dein Ohr und höre!
Öffne deine Augen, HERR, und blicke her! Ja, höre die Worte, die
Sanherib hier hat sagen lassen, um den lebendigen Gott zu verhöhnen!
17Es ist allerdings wahr, HERR, daß die Könige von Assyrien die
Völkerschaften und ihr Land verwüstet 18und deren Götter ins Feuer
geworfen haben; aber das waren auch keine Götter, sondern nur Machwerk
von Menschenhänden, Holz und Stein: die konnten sie vernichten. 19Nun
aber, HERR, unser Gott, rette uns doch aus seiner Hand, damit alle
Reiche auf Erden erkennen, daß du, HERR, allein Gott bist!«

\hypertarget{gg-jesaja-sendet-dem-kuxf6nig-hiskia-im-namen-gottes-bescheid-auf-sein-gebet}{%
\subparagraph{gg) Jesaja sendet dem König Hiskia im Namen Gottes
Bescheid auf sein
Gebet}\label{gg-jesaja-sendet-dem-kuxf6nig-hiskia-im-namen-gottes-bescheid-auf-sein-gebet}}

20Da sandte Jesaja, der Sohn des Amoz, zu Hiskia und ließ ihm sagen: »So
hat der HERR, der Gott Israels, gesprochen: ›Was das Gebet betrifft, das
du wegen Sanheribs, des Königs von Assyrien, an mich gerichtet hast, so
habe ich es gehört.‹ 21Folgenden Ausspruch hat der HERR über ihn getan:
›Es verachtet dich, es spottet deiner die jungfräuliche Tochter Zion;
hinter dir her schüttelt das Haupt die Tochter Jerusalem! 22Wen hast du
geschmäht und gelästert und gegen wen deine Stimme erhoben und deine
Augen hochmütig emporgerichtet? Gegen den Heiligen Israels! 23Durch den
Mund deiner Gesandten hast du den HERRN geschmäht und hast gesagt: Mit
meiner Kriegswagen Menge ersteige ich die Höhen der Berge, den obersten
Gipfel des Libanon; ich haue nieder den Hochwald seiner Zedern, die
Auslese seiner Zypressen, und dringe vor bis zu seinem obersten Gipfel,
in seinen dichtesten Baumgarten. 24Ich grabe den Erdboden auf und trinke
Wasser und mache andrerseits mit der Sohle meiner Füße versiegen alle
Nilarme Ägyptens!~-- 25Hast du es nicht gehört? Von lange her habe ich
es festgesetzt und seit den Tagen der Vorzeit es vorbereitet, nunmehr
aber es eintreten lassen, daß du feste Städte zu wüsten Steinhaufen
verheeren solltest, 26und ihre Bewohner, deren Arm zu kurz war, sollten
erschreckt dastehen und zuschanden werden, sollten wie Kraut des Feldes
und grünender Rasen sein, wie Gras auf den Dächern und wie Getreide, das
versengt ist, ehe es aufschießt✲. 27Mir ist dein Aufstehen und dein
Sitzen offenbar✲, dein Gehen und Kommen kenne ich wohl, auch dein Toben
wider mich. 28Weil du nun wider mich tobst und dein Großtun zu meinen
Ohren aufgestiegen ist, will ich dir meinen Ring in die Nase legen und
meinen Zaum an deine Lippen und will dich auf dem Wege zurückkehren
lassen, auf dem du hergekommen bist.‹ 29Folgendes aber möge dir, Hiskia,
als Wahrzeichen dienen: In diesem Jahr wird man den Brachwuchs essen und
im nächsten Jahre den Wurzelwuchs; im dritten Jahre aber sollt ihr säen
und ernten, sollt ihr Weinberge anlegen und ihren Ertrag genießen. 30Was
dann vom Hause Juda entronnen und als Rest übriggeblieben ist, wird aufs
neue unten Wurzel treiben und oben Früchte tragen; 31denn von Jerusalem
wird ein Überrest ausgehen und eine Schar Entronnener vom Berge Zion;
der Eifer des HERRN der Heerscharen wird dies vollführen!

32Darum hat der HERR in bezug auf den König von Assyrien so gesprochen:
›Er soll nicht in diese Stadt hineinkommen und keinen Pfeil
hineinschießen; er soll mit keinem Schild gegen sie anrücken und keinen
Wall gegen sie aufführen! 33Nein, auf dem Wege, auf dem er gekommen ist,
soll er zurückkehren und in diese Stadt nicht eindringen!‹ -- so lautet
der Ausspruch des HERRN. 34›Ja, ich will diese Stadt beschirmen, um sie
zu erretten, um meiner selbst willen und um meines Knechtes David
willen!‹«

\hypertarget{hh-die-erfuxfcllung-der-verheiuxdfung-sanheribs-abzug-und-ermordung}{%
\subparagraph{hh) Die Erfüllung der Verheißung: Sanheribs Abzug und
Ermordung}\label{hh-die-erfuxfcllung-der-verheiuxdfung-sanheribs-abzug-und-ermordung}}

35In derselben Nacht aber ging der Engel des HERRN aus und ließ im Lager
der Assyrer 185000~Mann sterben; und als man am Morgen früh aufstand,
fand man sie alle tot als Leichen vor. 36Da brach Sanherib, der König
von Assyrien, auf und zog ab; er kehrte nach Hause zurück und nahm
seinen Wohnsitz\textless sup title=``=~seine Residenz''\textgreater✲ in
Ninive. 37Als er aber dort (einmal) im Tempel seines Gottes Nisroch
anbetete, erschlugen ihn seine Söhne Adrammelech und Sarezer mit dem
Schwert; sie entflohen dann ins Land Ararat✲, und sein Sohn Esarhaddon
folgte ihm als König in der Regierung nach.

\hypertarget{ii-hiskias-krankheit-und-genesung-die-gesandtschaft-aus-babylon}{%
\subparagraph{ii) Hiskias Krankheit und Genesung; die Gesandtschaft aus
Babylon}\label{ii-hiskias-krankheit-und-genesung-die-gesandtschaft-aus-babylon}}

\hypertarget{section-19}{%
\section{20}\label{section-19}}

1Als Hiskia in jenen Tagen auf den Tod erkrankte, begab sich der Prophet
Jesaja, der Sohn des Amoz, zu ihm und sagte zu ihm: »So hat der HERR
gesprochen: ›Bestelle dein Haus, denn du mußt sterben und wirst nicht
wieder gesund werden!‹« 2Da kehrte er sein Gesicht gegen die Wand hin
und betete zum HERRN: 3»Ach, HERR! Denke doch daran, wie ich in Treue
und mit ungeteiltem Herzen vor deinem Angesicht gewandelt bin und getan
habe, was dir wohlgefällt!« Hierauf brach Hiskia in heftiges Weinen aus.
4Als nun Jesaja den inneren\textless sup title=``oder:
mittleren''\textgreater✲ Vorhof des Palastes noch nicht verlassen hatte,
da erging das Wort des HERRN an ihn folgendermaßen: 5»Kehre um und sage
zu Hiskia, dem Fürsten meines Volks: So hat der HERR, der Gott deines
Ahnherrn David, gesprochen: ›Ich habe dein Gebet gehört und deine Tränen
gesehen; so will ich dich denn wieder gesund werden lassen: schon
übermorgen sollst du zum Tempel des HERRN hinaufgehen! 6Ich will dann zu
deinen Lebenstagen noch fünfzehn Jahre hinzufügen; dazu will ich dich
und diese Stadt aus der Gewalt des Königs von Assyrien erretten und
diese Stadt beschirmen um meinetwillen und um meines Knechtes David
willen.‹« 7Darauf sagte Jesaja: »Bringt ein Feigenpflaster her!« Da
holten sie ein solches und legten es auf das Geschwür: da wurde er
gesund.

\hypertarget{jj-das-guxf6ttliche-wunderzeichen-an-der-sonnenuhr}{%
\subparagraph{jj) Das göttliche Wunderzeichen an der
Sonnenuhr}\label{jj-das-guxf6ttliche-wunderzeichen-an-der-sonnenuhr}}

8Als Hiskia aber Jesaja fragte: »Welches ist das Wahrzeichen dafür, daß
der HERR mich heilen wird und daß ich übermorgen zum Tempel des HERRN
hinaufgehen kann?« 9Da antwortete Jesaja: »Folgendes soll dir von seiten
des HERRN als Wahrzeichen dafür dienen, daß der HERR die Verheißung
erfüllen wird, die er gegeben hat: Soll der Schatten zehn Stufen
vorwärts oder zehn Stufen rückwärts gehen?« 10Hiskia antwortete: »Es
wäre für den Schatten ein leichtes, zehn Stufen hinabzusteigen; nein,
der Schatten soll zehn Stufen wieder rückwärts gehen!« 11Da rief der
Prophet Jesaja den HERRN an, und dieser ließ den Schatten an den Stufen,
welche (die Sonne) auf den Stufen des Sonnenzeigers\textless sup
title=``oder: der Sonnenuhr''\textgreater✲ des Ahas bereits
hinabgestiegen war, um zehn Stufen rückwärts gehen.

\hypertarget{kk-gesandtschaft-merodach-baladans-aus-babylon}{%
\subparagraph{kk) Gesandtschaft Merodach-Baladans aus
Babylon}\label{kk-gesandtschaft-merodach-baladans-aus-babylon}}

12Zu jener Zeit sandte Merodach-Baladan, der Sohn Baladans, der König
von Babylon, ein Schreiben und ein Geschenk an Hiskia; er hatte nämlich
gehört, daß Hiskia krank gewesen war. 13Hiskia hörte sie (die Gesandten)
gern an und zeigte ihnen sein ganzes Schatzhaus: das Silber und das
Gold, die Spezereien\textless sup title=``=~Wohlgerüche,
Gewürze''\textgreater✲ und das kostbare Öl, sein ganzes Zeughaus und
überhaupt alles, was sich in seinen Schatzhäusern vorfand; es gab nichts
in seinem Palast und im ganzen Bereich seiner Herrschaft, was Hiskia
ihnen nicht gezeigt hätte.

\hypertarget{ll-jesajas-strafrede-uxfcber-die-unvorsichtige-prunksucht-des-kuxf6nigs-und-seine-weissagung-uxfcber-die-babylonische-gefangenschaft}{%
\subparagraph{ll) Jesajas Strafrede über die unvorsichtige Prunksucht
des Königs und seine Weissagung über die babylonische
Gefangenschaft}\label{ll-jesajas-strafrede-uxfcber-die-unvorsichtige-prunksucht-des-kuxf6nigs-und-seine-weissagung-uxfcber-die-babylonische-gefangenschaft}}

14Da begab sich der Prophet Jesaja zum König Hiskia und fragte ihn: »Was
haben diese Männer gewollt, und woher sind sie gekommen?« Hiskia
antwortete: »Aus einem fernen Lande sind sie gekommen, aus Babylon.«
15Darauf fragte jener: »Was haben sie in deinem Palast zu sehen
bekommen?« Hiskia antwortete: »Alles, was in meinem Palast ist, haben
sie zu sehen bekommen; es gibt in meinen Schatzhäusern nichts, was ich
ihnen nicht gezeigt hätte.« 16Da sagte Jesaja zu Hiskia: »Vernimm das
Wort des HERRN: 17›Wisse wohl: es kommt die Zeit, da wird alles, was in
deinem Palast vorhanden ist und was an Schätzen deine Väter bis zum
heutigen Tage aufgehäuft haben, nach Babylon weggebracht werden; nichts
wird zurückbleiben!‹ -- so hat der HERR gesprochen --; 18›und von deinen
leiblichen Söhnen, die dir geboren werden, wird man einige nehmen, damit
sie im Palast des Königs von Babylon als Kämmerlinge\textless sup
title=``oder: Höflinge''\textgreater✲ dienen.‹«

\hypertarget{mm-hiskias-gottergebene-aber-reuelose-antwort}{%
\subparagraph{mm) Hiskias gottergebene, aber reuelose
Antwort}\label{mm-hiskias-gottergebene-aber-reuelose-antwort}}

19Da antwortete Hiskia dem Jesaja: »Gütig ist das Wort des HERRN, das du
mir mitgeteilt hast!« Er dachte nämlich: »Nun gut: es wird ja doch
Friede und Sicherheit herrschen, solange ich lebe.«

\hypertarget{nn-schluuxdfwort-zur-geschichte-hiskias}{%
\subparagraph{nn) Schlußwort zur Geschichte
Hiskias}\label{nn-schluuxdfwort-zur-geschichte-hiskias}}

20Die übrige Geschichte Hiskias aber und alle seine
Herrschermacht\textless sup title=``oder: Siege oder: tapferen
Taten''\textgreater✲ und wie er den Teich und die Wasserleitung angelegt
und das Wasser in die Stadt geleitet hat, das findet sich bekanntlich
aufgezeichnet im Buch der Denkwürdigkeiten\textless sup title=``oder:
Chronik''\textgreater✲ der Könige von Juda. 21Als Hiskia sich dann zu
seinen Vätern gelegt hatte, folgte ihm sein Sohn Manasse als König in
der Regierung nach.

\hypertarget{manasse-und-amon-kuxf6nige-von-juda}{%
\subsubsection{2. Manasse und Amon Könige von
Juda}\label{manasse-und-amon-kuxf6nige-von-juda}}

\hypertarget{a-manasses-abguxf6tterei}{%
\paragraph{a) Manasses Abgötterei}\label{a-manasses-abguxf6tterei}}

\hypertarget{section-20}{%
\section{21}\label{section-20}}

1Im Alter von zwölf Jahren wurde Manasse König und regierte
fünfundfünfzig Jahre in Jerusalem; seine Mutter hieß Hephziba. 2Er tat,
was dem HERRN mißfiel, im Anschluß an den greuelhaften Götzendienst der
heidnischen Völker, die der HERR vor den Israeliten vertrieben hatte.
3Er baute die Höhen wieder auf, die sein Vater Hiskia zerstört hatte,
errichtete dem Baal Altäre, ließ ein Standbild der Aschera\textless sup
title=``oder: Astarte''\textgreater✲ herstellen, wie es der König Ahab
von Israel getan hatte, betete das ganze Sternenheer des Himmels an und
erwies ihnen Verehrung. 4Er erbaute sogar Altäre im Tempel des HERRN,
von dem doch der HERR gesagt hatte: »In Jerusalem will ich meinen Namen
wohnen lassen«; 5und zwar erbaute er dem ganzen Sternenheer des Himmels
Altäre in den beiden Vorhöfen des Tempels des HERRN. 6Auch ließ er
seinen eigenen Sohn als Brandopfer verbrennen, trieb Zauberei und
Wahrsagerei und bestellte Totenbeschwörer und Zeichendeuter: er tat gar
vieles, was dem HERRN mißfiel und ihn zum Zorn reizen mußte. 7Das
geschnitzte Astartebild, das er hatte anfertigen lassen, stellte er
sogar im Tempel auf, von dem doch der HERR zu David und dessen Sohne
Salomo gesagt hatte\textless sup title=``2.Kön 8,29; 9,3''\textgreater✲:
»In diesem Hause und in Jerusalem, das ich aus allen Stämmen Israels
erwählt habe, will ich meinen Namen für ewige Zeiten wohnen lassen; 8und
ich will den Fuß Israels fortan nicht wieder wandern lassen aus dem
Lande, das ich ihren Vätern gegeben habe, wofern sie nur darauf bedacht
sind, alles zu tun, wie ich es ihnen geboten habe, nämlich ganz nach dem
Gesetz (zu leben), das mein Knecht Mose ihnen geboten hat.« 9Aber sie
gehorchten nicht; und Manasse verleitete sie dazu, es noch ärger zu
treiben als die heidnischen Völker, welche der HERR vor den Israeliten
vertilgt hatte.

\hypertarget{b-gottes-drohung-gegen-manasse-grausamkeit-manasses-und-schluuxdfwort-uxfcber-ihn}{%
\paragraph{b) Gottes Drohung gegen Manasse; Grausamkeit Manasses und
Schlußwort über
ihn}\label{b-gottes-drohung-gegen-manasse-grausamkeit-manasses-und-schluuxdfwort-uxfcber-ihn}}

10Da ließ sich der HERR durch den Mund seiner Knechte, der Propheten, so
vernehmen: 11»Weil Manasse, der König von Juda, diese Greuel verübt hat,
die ärger sind als alles, was die Amoriter vordem getan haben; und weil
er auch Juda durch seinen Götzendienst zur Sünde verführt hat, 12darum
hat der HERR, der Gott Israels, so gesprochen: ›Wisset wohl: ich will
Unglück über Jerusalem und Juda kommen lassen, daß allen, die davon
hören, beide Ohren gellen sollen! 13Und zwar will ich über Jerusalem die
Meßschnur ziehen wie (vordem) über Samaria und will die Setzwaage
stellen wie beim Hause Ahabs, und ich will Jerusalem ausscheuern, wie
man eine Schüssel ausscheuert und sie nach dem Ausscheuern auf ihre
Oberseite umkehrt. 14Und ich will den Überrest meines Eigentumsvolkes
verstoßen und sie in die Gewalt ihrer Feinde fallen lassen, daß sie
allen ihren Feinden zum Raub und zur Beute werden sollen, 15zur Strafe
dafür, daß sie getan haben, was mir mißfällt, und mich immerfort zum
Zorn gereizt haben seit der Zeit, wo ihre Väter aus Ägypten ausgezogen
sind, bis auf den heutigen Tag!‹«

16Auch sehr viel unschuldiges Blut vergoß Manasse, so daß er Jerusalem
damit bis oben an den Rand anfüllte, abgesehen von der Sünde, zu der er
Juda verführte, das zu tun, was dem HERRN mißfiel.

17Die übrige Geschichte Manasses aber und alles, was er unternommen, und
seine Versündigung, die er begangen hat, das findet sich bekanntlich
aufgezeichnet im Buch der Denkwürdigkeiten\textless sup title=``oder:
Chronik''\textgreater✲ der Könige von Juda.~-- 18Als Manasse sich dann
zu seinen Vätern gelegt und man ihn im Garten seines Palastes, im Garten
Ussas begraben hatte, folgte ihm sein Sohn Amon als König in der
Regierung nach.

\hypertarget{c-amon-von-juda}{%
\paragraph{c) Amon von Juda}\label{c-amon-von-juda}}

19Im Alter von zweiundzwanzig Jahren wurde Amon König, und zwei Jahre
regierte er in Jerusalem; seine Mutter hieß Mesullameth und war die
Tochter des Haruz von Jotba. 20Er tat, was dem HERRN mißfiel, wie sein
Vater Manasse getan hatte, 21und wandelte ganz auf dem Wege, den sein
Vater gewandelt war: er verehrte die Götzen, die sein Vater verehrt
hatte, und betete sie an; 22aber vom HERRN, dem Gott seiner Väter,
wandte er sich ab und wandelte nicht auf dem Wege des HERRN. 23Da
verschworen sich seine eigenen Diener gegen ihn und ermordeten den König
in seinem Palaste; 24die Landbevölkerung aber erschlug alle, die an der
Verschwörung gegen den König Amon teilgenommen hatten, und erhob dann
seinen Sohn Josia zu seinem Nachfolger auf dem Throne.

25Die übrige Geschichte Amons aber, alles was er unternommen hat, findet
sich bekanntlich aufgezeichnet im Buch der Denkwürdigkeiten\textless sup
title=``oder: Chronik''\textgreater✲ der Könige von Juda. 26Als man ihn
aber in seinem Begräbnis im Garten Ussas begraben hatte, folgte ihm sein
Sohn Josia als König in der Regierung nach.

\hypertarget{der-kuxf6nig-josia-auffindung-des-gesetzbuches-und-reinigung-des-gottesdienstes}{%
\subsubsection{3. Der König Josia; Auffindung des Gesetzbuches und
Reinigung des
Gottesdienstes}\label{der-kuxf6nig-josia-auffindung-des-gesetzbuches-und-reinigung-des-gottesdienstes}}

\hypertarget{a-eingangswort}{%
\paragraph{a) Eingangswort}\label{a-eingangswort}}

\hypertarget{section-21}{%
\section{22}\label{section-21}}

1Im Alter von acht Jahren wurde Josia König, und einunddreißig Jahre
regierte er in Jerusalem; seine Mutter hieß Jedida und war die Tochter
Adajas von Bozkath. 2Er tat, was dem HERRN wohlgefiel, und wandelte ganz
auf dem Wege seines Ahnherrn David, ohne nach rechts oder nach links
davon abzuweichen.

\hypertarget{b-josia-sorgt-fuxfcr-die-ausbesserung-des-tempels-bericht-uxfcber-die-auffindung-des-gesetzbuches-und-seine-erste-wirkung}{%
\paragraph{b) Josia sorgt für die Ausbesserung des Tempels; Bericht über
die Auffindung des Gesetzbuches und seine erste
Wirkung}\label{b-josia-sorgt-fuxfcr-die-ausbesserung-des-tempels-bericht-uxfcber-die-auffindung-des-gesetzbuches-und-seine-erste-wirkung}}

3In seinem achtzehnten Regierungsjahre aber sandte der König Josia den
Staatsschreiber Saphan, den Sohn Azaljas, des Sohnes Mesullams, in den
Tempel des HERRN mit der Weisung: 4»Gehe zum Hohenpriester Hilkia
hinauf, er soll das Geld ausschütten\textless sup title=``oder: den
Gesamtbetrag des Geldes feststellen''\textgreater✲, das in den Tempel
des HERRN gebracht worden ist und das die Schwellenhüter vom Volk
eingesammelt haben; 5man soll es dann den Werkführern einhändigen, die
am Tempel des HERRN zu Aufsehern bestellt sind, damit diese es den
Arbeitern auszahlen, die am Tempel des HERRN mit der Ausbesserung der
Schäden des Tempels beschäftigt sind, 6den Zimmerleuten, Bauleuten und
Maurern sowie für den Ankauf von Hölzern und behauenen Steinen zur
Instandsetzung des Tempels. 7Doch soll über das Geld, das man ihnen
einhändigt, keine Verrechnung mit ihnen stattfinden; denn sie handeln
auf Treu und Glauben.«

8Da sagte dann der Hohepriester Hilkia zum Staatsschreiber Saphan: »Ich
habe das Gesetzbuch im Tempel des HERRN gefunden«; damit übergab Hilkia
dem Saphan das Buch, und er las es. 9Als hierauf der Staatsschreiber
Saphan zum Könige kam und diesem Bericht erstattet hatte mit den Worten:
»Deine Knechte\textless sup title=``oder: Diener''\textgreater✲ haben
das Geld, das sich im Tempel vorfand, ausgeschüttet und es den
Werkführern eingehändigt, die am Tempel des HERRN zu Aufsehern bestellt
sind«, 10machte der Staatsschreiber Saphan dem Könige noch die
Mitteilung: »Der Priester Hilkia hat mir ein Buch übergeben«, und Saphan
las es dem Könige vor.

11Als nun der König den Inhalt des Gesetzbuches vernommen hatte, zerriß
er seine Kleider 12und gab sodann dem Priester Hilkia und Ahikam, dem
Sohne Saphans, und Achbor, dem Sohne Michajas, und dem Staatsschreiber
Saphan und Asaja, dem Leibdiener des Königs, folgenden Befehl: 13»Geht
hin und befragt den HERRN für mich und für das Volk und für ganz Juda in
betreff des Inhalts dieses Buches, das man aufgefunden hat! Denn groß
ist der Grimm des HERRN, der gegen uns entbrannt ist, weil unsere Väter
den Weisungen dieses Buches nicht gehorcht haben, um das genau zu
befolgen, was darin\textless sup title=``oder: für uns?''\textgreater✲
geschrieben steht.«

\hypertarget{c-befragung-und-antwort-der-prophetin-hulda}{%
\paragraph{c) Befragung und Antwort der Prophetin
Hulda}\label{c-befragung-und-antwort-der-prophetin-hulda}}

14Da begab sich der Priester Hilkia mit Ahikam, Achbor, Saphan und Asaja
zu der Prophetin Hulda, der Frau des Kleiderhüters Sallum, des Sohnes
Thikwas, des Sohnes Harhas; die wohnte zu Jerusalem im zweiten Bezirk.
Als sie sich mit ihr besprachen, 15sagte sie zu ihnen: »So hat der HERR,
der Gott Israels, gesprochen: ›Sagt dem Mann, der euch zu mir gesandt
hat: 16So hat der HERR gesprochen: ›Wisset wohl: ich will Unglück über
diesen Ort und seine Bewohner kommen lassen, nämlich alle Drohungen des
Buches, das der König von Juda gelesen hat. 17Zur Strafe dafür, daß sie
mich verlassen und anderen Göttern geopfert haben, um mich mit all dem
Machwerk ihrer Hände zum Zorn zu reizen, soll mein Grimm gegen diesen
Ort entbrennen und nicht wieder erlöschen!‹ 18Aber zum König von Juda,
der euch gesandt hat, um den HERRN zu befragen, zu dem sollt ihr so
sagen: ›So hat der HERR, der Gott Israels, gesprochen: Was die Drohungen
betrifft, die du vernommen hast: 19weil dein Herz weich geworden ist und
du dich vor dem HERRN gedemütigt hast, als du vernahmst, was ich diesem
Ort und seinen Bewohnern angedroht habe, daß sie nämlich zu einem
abschreckenden Beispiel und zu einem Fluch werden sollen, und weil du
deine Kleider zerrissen und vor mir geweint hast, so habe auch ich dir
Gehör geschenkt‹ -- so lautet der Ausspruch des HERRN. 20›Darum wisse
wohl: ich will dich zu deinen Vätern versammeln, daß du in Frieden in
deine Grabstätte gebracht wirst, und deine Augen sollen all das Unglück,
das ich über diesen Ort bringen werde, nicht zu sehen bekommen!‹«

\hypertarget{d-josia-schlieuxdft-den-neuen-gottesbund-im-verein-mit-den-uxe4ltesten-des-volkes-ab}{%
\paragraph{d) Josia schließt den neuen Gottesbund im Verein mit den
Ältesten des Volkes
ab}\label{d-josia-schlieuxdft-den-neuen-gottesbund-im-verein-mit-den-uxe4ltesten-des-volkes-ab}}

\hypertarget{section-22}{%
\section{23}\label{section-22}}

1Als sie nun dem König Bericht erstattet hatten, sandte der König Boten
aus, daß sie alle Ältesten von Juda und Jerusalem bei ihm versammelten.
2Hierauf ging der König zum Tempel des HERRN hinauf und mit ihm alle
Männer von Juda und alle Bewohner Jerusalems, auch die Priester und die
Propheten, überhaupt das ganze Volk, klein und groß; und er las ihnen
den ganzen Inhalt des Bundesbuches vor, das man im Tempel des HERRN
gefunden hatte. 3Hierauf trat der König an die Säule\textless sup
title=``oder: auf den Hochstand; vgl. 11,14''\textgreater✲ und schloß
den Bund vor dem HERRN, daß sie dem HERRN nachwandeln und seine Gebote,
seine Verordnungen und seine Satzungen mit ganzem Herzen und mit ganzer
Seele beobachten wollten, um so den Bestimmungen dieses Bundes, die in
diesem Buche geschrieben standen, Geltung zu verschaffen. Und das
gesamte Volk trat dem Bunde bei.

\hypertarget{e-josia-reinigt-den-tempel-und-den-gesamten-uxf6ffentlichen-gottesdienst}{%
\paragraph{e) Josia reinigt den Tempel und den gesamten öffentlichen
Gottesdienst}\label{e-josia-reinigt-den-tempel-und-den-gesamten-uxf6ffentlichen-gottesdienst}}

4Hierauf befahl der König dem Hohenpriester Hilkia und den Priestern
zweiter Ordnung und den Schwellenhütern, alle Geräte, die für Baal und
Astarte und für das ganze Sternenheer des Himmels angefertigt worden
waren, aus dem Tempel des HERRN hinauszuschaffen; er ließ sie dann
außerhalb Jerusalems auf den Feldern am Kidron verbrennen und ihre Asche
nach Bethel bringen. 5Sodann setzte er die Götzenpriester ab, welche die
Könige von Juda eingesetzt hatten und die auf den Höhen in den
Ortschaften Judas und in der Umgegend von Jerusalem geopfert, sowie die,
welche dem Baal, der Sonne und dem Monde, den Bildern des Tierkreises
und dem ganzen Sternenheer des Himmels Opfer dargebracht hatten. 6Ferner
ließ er das Standbild der Aschera\textless sup title=``oder:
Astarte''\textgreater✲ aus dem Tempel des HERRN hinaus vor die Tore
Jerusalems in das Kidrontal schaffen und es dort verbrennen und zu Staub
zerstampfen und den Staub davon auf die Gräber der gemeinen Leute
werfen. 7Sodann ließ er die Stuben der Heiligtumsbuhler\textless sup
title=``oder: der geweihten Buhler''\textgreater✲ niederreißen, die sich
im Tempel des HERRN befanden und in denen die Weiber Hüllen für Astarte
zu weben pflegten. 8Weiter ließ er alle Priester aus den Ortschaften
Judas kommen und die Höhen entweihen, auf denen die Priester geopfert
hatten, von Geba bis Beerseba; auch ließ er die Höhen der Bocksgestalten
niederreißen, die am Eingang des Tores des Stadthauptmannes Josua auf
der linken Seite standen, wenn man zum Stadttor hineinging. 9Doch
durften die Höhenpriester nicht den Opferdienst auf dem Altar des HERRN
in Jerusalem versehen, wohl aber aßen sie die ungesäuerten Brote
inmitten ihrer Amtsbrüder. 10Auch die Greuelstätte, die im Tal
Ben-Hinnom lag, ließ er entweihen, damit niemand mehr seinen Sohn oder
seine Tochter dem Moloch als Brandopfer darbrächte. 11Ferner ließ er die
Rosse beseitigen, welche die Könige von Juda zu Ehren des Sonnengottes
am Eingang zum Tempel des HERRN in der Richtung nach der Zelle des
Kämmerers Nethan-Melech, im Parwarim\textless sup title=``d.h. Anbau am
Tempel''\textgreater✲ aufgestellt hatten, und ließ den Sonnenwagen im
Feuer verbrennen. 12Auch die Altäre, die auf dem Dach, dem Söller des
Ahas, standen und von den Königen von Juda herrührten, sowie die Altäre,
welche Manasse in den beiden Vorhöfen des Tempels des HERRN errichtet
hatte, ließ der König niederreißen und zerschlagen und den Schutt von
ihnen in das Kidrontal werfen. 13Auch die Höhen, die östlich von
Jerusalem, südlich vom Unheilsberge✲ lagen, die der König Salomo von
Israel zu Ehren der Astarte, des greulichen Götzen der Sidonier, und für
Kamos, das Scheusal der Moabiter, und für Milkom, den greulichen Götzen
der Ammoniter, errichtet hatte, ließ der König entweihen; 14ebenso
zertrümmerte er die Malsteine und ließ die Götzenbäume umhauen und ihren
Platz mit Menschengebeinen anfüllen.

\hypertarget{josias-verfahren-in-bethel-und-gegen-den-huxf6hendienst-in-samaria}{%
\paragraph{Josias Verfahren in Bethel und gegen den Höhendienst in
Samaria}\label{josias-verfahren-in-bethel-und-gegen-den-huxf6hendienst-in-samaria}}

15Aber auch den Altar zu Bethel, die Höhe, welche Jerobeam, der Sohn
Nebats, der Israel zur Sünde verführte, hergestellt hatte, auch diesen
Altar samt der Höhe ließ er niederreißen; er verwüstete die Höhe mit
Feuer, zermalmte sie\textless sup title=``d.h. die Steine des
Altars''\textgreater✲ zu Staub und verbrannte das Standbild der Aschera.
16Als Josia dabei umherblickte und die Gräber dort am Bergabhang
wahrnahm, sandte er Leute hin, ließ die Gebeine aus den Gräbern
herausnehmen und auf dem Altar verbrennen und entweihte ihn auf diese
Weise gemäß der Drohung des HERRN, die der Gottesmann einst
ausgesprochen hatte, der diese Dinge voraussagte\textless sup
title=``1.Kön 13,2''\textgreater✲. 17Als er dann fragte: »Was ist das
für ein Grabmal, das ich dort sehe?«, antworteten ihm die Leute der
Stadt: »Das ist das Grab des Gottesmannes, der aus Juda gekommen war und
das angesagt hat, was du jetzt am Altar von Bethel getan hast.« 18Da
befahl er: »Laßt ihn liegen, niemand störe seine Gebeine in ihrer Ruhe!«
So ließ man denn seine Gebeine unversehrt samt den Gebeinen des
Propheten, der aus Samaria war.~-- 19Außerdem beseitigte Josia auch alle
Höhenheiligtümer, die sich in den Ortschaften Samarias befanden und die
von den israelitischen Königen angelegt waren, um (den HERRN) zum Zorn
zu reizen; er verfuhr mit ihnen gerade so, wie er zu Bethel verfahren
war. 20Alle Höhenpriester aber, die daselbst waren, ließ er auf den
Altären schlachten und Menschengebeine darauf verbrennen; alsdann kehrte
er nach Jerusalem zurück.

\hypertarget{f-gesetzesstrenge-feier-des-passahs}{%
\paragraph{f) Gesetzesstrenge Feier des
Passahs}\label{f-gesetzesstrenge-feier-des-passahs}}

21Hierauf ließ der König folgenden Befehl an das ganze Volk ergehen:
»Feiert das Passahfest zu Ehren des HERRN, eures Gottes, so wie es in
diesem Bundesbuche geschrieben steht!« 22Denn ein solches Passah wie
dieses war nicht gefeiert worden seit der Zeit der Richter, die in
Israel gewaltet hatten, und während der ganzen Zeit der Könige von
Israel und der Könige von Juda; 23vielmehr erst im achtzehnten
Regierungsjahre des Königs Josia wurde dieses Passah zu Ehren des HERRN
in Jerusalem begangen.

\hypertarget{g-vorgehen-gegen-die-abguxf6tterei-im-privatleben-fortdauer-des-guxf6ttlichen-zornes-gegen-juda}{%
\paragraph{g) Vorgehen gegen die Abgötterei im Privatleben; Fortdauer
des göttlichen Zornes gegen
Juda}\label{g-vorgehen-gegen-die-abguxf6tterei-im-privatleben-fortdauer-des-guxf6ttlichen-zornes-gegen-juda}}

24Außerdem rottete Josia auch die Totenbeschwörer und Zeichendeuter, die
Hausgötter\textless sup title=``1.Mose 31,19; 1.Sam 19,13''\textgreater✲
und die Götzen, überhaupt alle Abgötter aus, die im Lande Juda und in
Jerusalem zu sehen waren, um den Bestimmungen des Gesetzes nachzukommen,
die in dem Buch, das der Priester Hilkia im Tempel des HERRN gefunden
hatte, geschrieben standen. 25Und seinesgleichen hat es vor ihm keinen
König gegeben, der sich so von ganzem Herzen, von ganzer Seele und mit
aller seiner Kraft dem HERRN ganz nach dem mosaischen Gesetze zugewandt
hätte, und auch nach ihm ist keiner seinesgleichen erstanden.
26Gleichwohl ließ der HERR von der gewaltigen Glut seines Zornes nicht
ab, weil sein Zorn einmal gegen Juda entbrannt war wegen all der
Ärgernisse, durch die Manasse ihn erbittert hatte. 27Daher sprach der
HERR: »Auch Juda will ich mir aus den Augen schaffen, wie ich Israel
verstoßen habe, und will diese Stadt verwerfen, die ich einst erwählt
hatte, Jerusalem, und das Haus, von dem ich einst gesagt hatte, mein
Name solle daselbst wohnen.«

\hypertarget{h-schluuxdfwort-necho-von-uxe4gypten-und-josias-tod}{%
\paragraph{h) Schlußwort; Necho von Ägypten und Josias
Tod}\label{h-schluuxdfwort-necho-von-uxe4gypten-und-josias-tod}}

28Die übrige Geschichte Josias aber und alles, was er unternommen hat,
das findet sich bekanntlich aufgezeichnet im Buch der
Denkwürdigkeiten\textless sup title=``oder: Chronik''\textgreater✲ der
Könige von Juda.

29Während seiner Regierung zog der Pharao Necho, der König von Ägypten,
gegen den König von Assyrien zu Felde an den Euphratstrom. Der König
Josia zog ihm entgegen, aber (Necho) tötete ihn bei Megiddo, sobald er
seiner ansichtig geworden war. 30Da fuhren seine Diener ihn zu Wagen tot
von Megiddo hinweg, brachten ihn nach Jerusalem und setzten ihn in
seiner Grabstätte bei. Die Landbevölkerung nahm dann Joahas, den Sohn
Josias, salbte ihn und machte ihn zum König an seines Vaters Statt.

\hypertarget{josias-suxf6hne-und-sein-enkel-kuxf6nige-von-juda}{%
\subsubsection{4. Josias Söhne und sein Enkel Könige von
Juda}\label{josias-suxf6hne-und-sein-enkel-kuxf6nige-von-juda}}

\hypertarget{a-joahas}{%
\paragraph{a) Joahas}\label{a-joahas}}

31Im Alter von dreiundzwanzig Jahren wurde Joahas König und regierte
drei Monate in Jerusalem; seine Mutter hieß Hamutal und war die Tochter
Jeremias von Libna. 32Er tat, was dem HERRN mißfiel, ganz so wie seine
Väter getan hatten. 33Der Pharao Necho aber ließ ihn zu Ribla in der
Landschaft Hamath ins Gefängnis werfen, damit er nicht länger König in
Jerusalem wäre, und legte dem Lande eine Geldbuße von hundert Talenten
Silber und einem Talent Gold auf. 34Sodann machte der Pharao Necho
Eljakim, den Sohn Josias, zum König an Stelle seines Vaters Josia und
änderte seinen Namen in Jojakim ab; den Joahas aber nahm er mit sich; so
kam dieser nach Ägypten und starb daselbst. 35Das Silber und das Gold
lieferte Jojakim dem Pharao ab, mußte jedoch seinem Lande eine besondere
Steuer auferlegen, um das Geld nach der Forderung des Pharaos zahlen zu
können: je nachdem ein jeder abgeschätzt worden war, trieb er das Silber
und das Gold von der Bevölkerung des Landes ein, um es dann dem Pharao
Necho zu übergeben.

\hypertarget{b-jojakim-von-juda}{%
\paragraph{b) Jojakim von Juda}\label{b-jojakim-von-juda}}

36Im Alter von fünfundzwanzig Jahren wurde Jojakim König und regierte
elf Jahre in Jerusalem; seine Mutter hieß Sebudda und war die Tochter
Pedajas aus Ruma. 37Er tat, was dem HERRN mißfiel, ganz so wie seine
Väter getan hatten.

\hypertarget{section-23}{%
\section{24}\label{section-23}}

1Während seiner Regierung kam Nebukadnezar, der König von Babylon,
herangezogen, und Jojakim wurde ihm drei Jahre lang untertan, fiel dann
aber wieder von ihm ab. 2Da ließ Gott der HERR die Kriegsscharen der
Chaldäer und der Syrer sowie die Scharen der Moabiter und der Ammoniter
gegen ihn heranziehen; die ließ er in Juda einfallen, um es zugrunde zu
richten, gemäß der Drohung, die der HERR durch den Mund seiner Knechte,
der Propheten, hatte aussprechen lassen. 3Nur nach dem Ausspruch des
HERRN ist dies Unheil über Juda hereingebrochen, damit er es sich aus
den Augen schaffte wegen der Sünden Manasses, infolge alles dessen, was
er verübt hatte; 4besonders auch das unschuldige Blut, das er vergossen
hatte, so daß er Jerusalem mit unschuldigem Blut anfüllte, auch das
wollte der HERR nicht vergeben.

5Die übrige Geschichte Jojakims aber und alles, was er unternommen hat,
das findet sich bekanntlich aufgezeichnet im Buch der
Denkwürdigkeiten\textless sup title=``oder: Chronik''\textgreater✲ der
Könige von Juda. 6Als Jojakim sich dann zu seinen Vätern gelegt hatte,
folgte ihm sein Sohn Jojachin als König in der Regierung nach. 7Der
König von Ägypten aber unternahm fortan keinen Kriegszug mehr aus seinem
Lande; denn der König von Babylon hatte alles in Besitz genommen, was
dem König von Ägypten gehört hatte, vom Bach\textless sup title=``d.h.
Grenzbach''\textgreater✲ Ägyptens an bis zum Euphratstrom.

\hypertarget{c-jojachin-von-juda-die-erste-eroberung-jerusalems-und-die-erste-wegfuxfchrung-nach-babylon}{%
\paragraph{c) Jojachin von Juda; die erste Eroberung Jerusalems und die
erste Wegführung nach
Babylon}\label{c-jojachin-von-juda-die-erste-eroberung-jerusalems-und-die-erste-wegfuxfchrung-nach-babylon}}

8Im Alter von achtzehn Jahren wurde Jojachin König und regierte drei
Monate in Jerusalem; seine Mutter hieß Nehustha und war die Tochter
Elnathans aus Jerusalem. 9Er tat, was dem HERRN mißfiel, ganz so wie
sein Vater getan hatte. 10Zu jener Zeit zogen die Heerführer
Nebukadnezars, des Königs von Babylon, gegen Jerusalem heran, und die
Stadt wurde eingeschlossen. 11Als dann Nebukadnezar, der König von
Babylon, selbst vor der Stadt ankam, während seine Heerführer sie
belagerten, 12ging Jojachin, der König von Juda, zum König von Babylon
hinaus, er mit seiner Mutter, seinen Hofbeamten, seinen Heeresobersten
und seinen Kämmerlingen, und der König von Babylon nahm ihn im achten
Jahre seiner Regierung gefangen. 13Er ließ dann alle Schätze des Tempels
des HERRN und die Schätze des königlichen Palastes von dort wegbringen
und brach den Metallbeschlag von allen goldenen Geräten ab, die Salomo,
der König von Israel, für den Tempel des HERRN hatte anfertigen lassen:
wie der HERR es angekündigt hatte. 14Ganz Jerusalem aber führte er in
die Gefangenschaft: alle hohen Beamten und alle kriegstüchtigen Männer,
zehntausend Gefangene, dazu alle Schmiede und Schlosser: nichts blieb
zurück außer der niederen Bevölkerung des Landes. 15Auch Jojachin führte
er nach Babylon in die Gefangenschaft, ebenso die Mutter des Königs und
die königlichen Frauen; auch seine Kämmerlinge und die vornehmsten
Männer des Landes führte er als Gefangene von Jerusalem weg nach
Babylon; 16dazu alle kriegstüchtigen Männer, siebentausend an Zahl,
ferner die Schmiede und Schlosser, tausend an Zahl, lauter
kriegstüchtige, streitbare Männer; die brachte der König von Babylon als
Gefangene nach Babylon. 17Hierauf machte der König von Babylon
Matthanja, den Oheim Jojachins, zum König an dessen Stelle und änderte
seinen Namen in Zedekia ab.

\hypertarget{zedekia-kuxf6nig-von-juda-ende-des-reiches-juda}{%
\subsubsection{5. Zedekia König von Juda; Ende des Reiches
Juda}\label{zedekia-kuxf6nig-von-juda-ende-des-reiches-juda}}

\hypertarget{a-eingangswort-die-allgemeine-gottlosigkeit}{%
\paragraph{a) Eingangswort; die allgemeine
Gottlosigkeit}\label{a-eingangswort-die-allgemeine-gottlosigkeit}}

18Im Alter von einundzwanzig Jahren kam Zedekia auf den Thron und
regierte elf Jahre in Jerusalem; seine Mutter hieß Hamutal und war die
Tochter Jeremias aus Libna\textless sup title=``vgl.
23,31''\textgreater✲. 19Er tat, was dem HERRN mißfiel, ganz wie Jojakim
getan hatte. 20Denn infolge des Zornes des HERRN kam es mit Jerusalem
und Juda dahin, daß der HERR sie von seinem Angesicht verstieß.

\hypertarget{b-zedekias-abfall-belagerung-jerusalems-flucht-und-gefangennahme-des-kuxf6nigs-strafgericht-in-ribla}{%
\paragraph{b) Zedekias Abfall; Belagerung Jerusalems; Flucht und
Gefangennahme des Königs; Strafgericht in
Ribla}\label{b-zedekias-abfall-belagerung-jerusalems-flucht-und-gefangennahme-des-kuxf6nigs-strafgericht-in-ribla}}

\hypertarget{section-24}{%
\section{25}\label{section-24}}

1Als Zedekia aber vom König von Babylon abgefallen war, da -- es war im
neunten Jahre seiner Regierung, am zehnten Tage des zehnten Monats --
kam Nebukadnezar, der König von Babylon, in eigener Person mit seiner
ganzen Heeresmacht gegen Jerusalem herangezogen und belagerte es. Man
führte rings um die Stadt Belagerungswerke auf, 2und die Stadt blieb
dann eingeschlossen bis ins elfte Jahr der Regierung Zedekias. 3Am
neunten Tage des (vierten) Monats, als die Hungersnot in der Stadt
übermächtig geworden war und die Landbevölkerung kein Brot mehr hatte,
4da wurde die Stadtmauer durchbrochen, und alle Kriegsleute ergriffen
nachts die Flucht auf dem Wege durch das Tor zwischen den beiden Mauern,
das am Königsgarten liegt, während die Chaldäer noch rings um die Stadt
her lagen; und man wandte sich dann der Jordanebene zu. 5Aber das Heer
der Chaldäer setzte dem Könige nach und holte ihn in den Steppen von
Jericho ein, nachdem sein ganzes Heer sich zerstreut und ihn verlassen
hatte. 6So wurde denn der König gefangengenommen und zum König von
Babylon nach Ribla hinaufgeführt, wo man Gericht über ihn hielt. 7Die
Söhne Zedekias ließ er vor dessen Augen grausam hinrichten; Zedekia aber
ließ er blenden und in Ketten legen; dann brachte man ihn nach Babylon.

\hypertarget{c-eroberung-und-zerstuxf6rung-jerusalems-pluxfcnderung-und-verbrennung-des-tempels-wegfuxfchrung-von-einwohnern-nach-babylon-hinrichtungen-zu-ribla}{%
\paragraph{c) Eroberung und Zerstörung Jerusalems; Plünderung und
Verbrennung des Tempels; Wegführung von Einwohnern nach Babylon;
Hinrichtungen zu
Ribla}\label{c-eroberung-und-zerstuxf6rung-jerusalems-pluxfcnderung-und-verbrennung-des-tempels-wegfuxfchrung-von-einwohnern-nach-babylon-hinrichtungen-zu-ribla}}

8Am siebten Tage des fünften Monats aber -- das war das neunzehnte
Regierungsjahr des Königs Nebukadnezar, des Königs von Babylon -- kam
Nebusaradan, der Befehlshaber der Leibwache, der dem Könige von Babylon
besonders nahe stand, nach Jerusalem 9und verbrannte den Tempel des
HERRN sowie den königlichen Palast und alle Häuser in Jerusalem: alle
größeren Häuser ließ er in Flammen aufgehen. 10Sodann mußte das ganze
chaldäische Heer, welches der Befehlshaber der Leibwache bei sich hatte,
die Mauern rings um Jerusalem niederreißen. 11Den Rest des Volkes aber,
was an Einwohnern in der Stadt noch übriggeblieben war, und die
Überläufer, die zum König von Babylon übergegangen waren, {[}sowie den
Rest der Werkleute{]} ließ Nebusaradan, der Befehlshaber der Leibwache,
in die Gefangenschaft nach Babylon führen; 12von der niederen
Bevölkerung des Landes aber ließ der Befehlshaber der Leibwache einen
Teil als Weingärtner und Ackerleute zurück. 13Aber die ehernen Säulen,
die am Tempel des HERRN standen, sowie die Gestühle und das große eherne
Wasserbecken, die beim Tempel des HERRN waren, zerschlugen die Chaldäer
und nahmen das Erz davon mit sich nach Babylon. 14Auch die Kessel,
Schaufeln, Messer zum Lichtputzen, die Schalen und alle übrigen ehernen
Geräte, die man beim Gottesdienst gebraucht hatte, nahmen sie weg;
15ebenso die Kohlenpfannen und Sprengschalen, alles, was ganz aus Gold
oder ganz aus Silber bestand, nahm der Befehlshaber der Leibwache weg.
16Was die beiden Säulen sowie das eine große Wasserbecken und die
Gestühle betrifft, die Salomo für den Tempel des HERRN hatte anfertigen
lassen, so war es unmöglich, das Erz aller dieser Kunstwerke zu wägen.
17Achtzehn Ellen war die eine Säule hoch, und ein Knauf von Erz befand
sich oben darauf; die Höhe des Knaufes betrug drei Ellen, und ein
Flechtwerk und Granatäpfel waren ringsum an dem Knauf angebracht, alles
von Erz; ebenso war auch die andere Säule beschaffen nebst dem
Flechtwerk.

18Weiter ließ der Befehlshaber der Leibwache den Oberpriester Seraja,
den Unterpriester Zephanja und die drei Schwellenhüter verhaften;
19ferner nahm er aus der Stadt den einen Kämmerer fest, der den
Oberbefehl über das Kriegsvolk gehabt hatte, sowie fünf Männer von
denen, die zu der ständigen Umgebung des Königs gehört hatten und die in
der Stadt vorgefunden wurden, außerdem den Schreiber des Feldhauptmanns,
der das Landvolk zum Heeresdienst ausgehoben hatte, außerdem sechzig
Personen von der Landbevölkerung, die noch in der Stadt angetroffen
worden waren. 20Diese nahm Nebusaradan, der Befehlshaber der Leibwache,
und brachte sie zum König von Babylon nach Ribla; 21der König von
Babylon aber ließ sie zu Ribla in der Landschaft Hamath grausam
hinrichten. So wurde Juda von seinem heimischen Boden gefangen
weggeführt.

\hypertarget{d-gedalja-zum-statthalter-eingesetzt-sammelt-die-juden-zu-einer-kolonie-in-mizpa.-nach-seiner-ermordung-wandern-die-juden-nach-uxe4gypten-aus}{%
\paragraph{d) Gedalja, zum Statthalter eingesetzt, sammelt die Juden zu
einer Kolonie in Mizpa. Nach seiner Ermordung wandern die Juden nach
Ägypten
aus}\label{d-gedalja-zum-statthalter-eingesetzt-sammelt-die-juden-zu-einer-kolonie-in-mizpa.-nach-seiner-ermordung-wandern-die-juden-nach-uxe4gypten-aus}}

22Über das Volk aber, das im Lande Juda zurückgeblieben war, weil
Nebukadnezar, der König von Babylon, es übriggelassen hatte, über diese
setzte er Gedalja, den Sohn Ahikams, des Sohnes Saphans, als Statthalter
ein. 23Als nun alle Truppenführer samt ihren Leuten erfuhren, daß der
König von Babylon Gedalja zum Statthalter bestellt habe, da begaben sie
sich zu Gedalja nach Mizpa, nämlich Ismael, der Sohn Nethanjas, Johanan,
der Sohn Kareahs, Seraja, der Sohn Thanhumeths aus Netopha, und
Jaasanja, der Sohn eines Maachathiters, samt ihren Leuten. 24Da richtete
Gedalja an sie und ihre Leute unter feierlicher Anrufung Gottes folgende
Ansprache: »Fürchtet euch nicht vor den Beamten der Chaldäer! Bleibt im
Lande und seid dem König von Babylon untertan: ihr werdet euch gut dabei
stehen!« 25Aber im siebenten Monat kam Ismael, der Sohn Nethanjas, des
Sohnes Elisamas, ein Mann von königlicher Abkunft, und mit ihm zehn
Männer; die erschlugen Gedalja samt den Judäern und Chaldäern, die bei
ihm in Mizpa waren. 26Da machte sich das ganze Volk, klein und groß,
mitsamt den Truppenführern auf und zog nach Ägypten; denn sie fürchteten
sich vor den Chaldäern.

\hypertarget{e-jojachins-begnadigung-nach-siebenunddreiuxdfigjuxe4hriger-gefangenschaft}{%
\paragraph{e) Jojachins Begnadigung nach siebenunddreißigjähriger
Gefangenschaft}\label{e-jojachins-begnadigung-nach-siebenunddreiuxdfigjuxe4hriger-gefangenschaft}}

27Aber im siebenunddreißigsten Jahr der Wegführung\textless sup
title=``oder: der Gefangenschaft''\textgreater✲ Jojachins, des Königs
von Juda, am siebenundzwanzigsten Tage des zwölften Monats, begnadigte
Ewil-Merodach, der König von Babylon, -- im Jahre seines
Regierungsantritts -- den König Jojachin von Juda und entließ ihn aus
dem Gefängnis. 28Er redete freundlich mit ihm und wies ihm seinen Sitz
an über den Sitz der anderen Könige, die bei ihm in Babylon waren. 29Da
durfte er auch seine Gefangenenkleidung ablegen und speiste regelmäßig
an der königlichen Tafel, solange er noch lebte. 30Sein Unterhalt aber
wurde ihm als ständiger Unterhalt, soviel er täglich bedurfte, von
seiten des Königs bis zu seinem Todestage zugewiesen, solange er noch
lebte.
