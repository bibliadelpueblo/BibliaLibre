\hypertarget{erstes-buch-der-chronik}{%
\section{ERSTES BUCH DER CHRONIK}\label{erstes-buch-der-chronik}}

\emph{(d.h. der Zeitgeschichte oder: der Denkwürdigkeiten)}

\hypertarget{i.-die-geschlechtstafeln-kap.-1-9}{%
\subsection{I. Die Geschlechtstafeln (Kap.
1-9)}\label{i.-die-geschlechtstafeln-kap.-1-9}}

\hypertarget{stammbaum-der-erzvuxe4ter-von-adam-bis-auf-abraham-und-seine-familie}{%
\subsubsection{1. Stammbaum der Erzväter von Adam bis auf Abraham und
seine
Familie}\label{stammbaum-der-erzvuxe4ter-von-adam-bis-auf-abraham-und-seine-familie}}

\hypertarget{a-die-urvuxe4ter-bis-zur-sintflut}{%
\paragraph{a) Die Urväter bis zur
Sintflut}\label{a-die-urvuxe4ter-bis-zur-sintflut}}

\hypertarget{section}{%
\section{1}\label{section}}

1Adam, Seth, Enos, 2Kenan, Mahalaleel, Jared, 3Henoch, Methusalah,
Lamech, 4Noah, Sem, Ham und Japheth.

\hypertarget{b-die-nachkommen-noahs-bis-auf-abraham}{%
\paragraph{b) Die Nachkommen Noahs bis auf
Abraham}\label{b-die-nachkommen-noahs-bis-auf-abraham}}

\hypertarget{aa-die-japhethiten}{%
\subparagraph{aa) Die Japhethiten}\label{aa-die-japhethiten}}

5Die Söhne Japheths waren: Gomer, Magog, Madai, Jawan, Thubal, Mesech
und Thiras. 6Und die Söhne Gomers: Askenas, Diphath und Thogarma. 7Und
die Söhne Jawans: Elisa, Tharsis, die Kitthiter und die Rodaniter.

\hypertarget{bb-die-hamiten}{%
\subparagraph{bb) Die Hamiten}\label{bb-die-hamiten}}

8Die Söhne Hams waren: Kusch und Mizraim, Put und Kanaan. 9Und die Söhne
Kuschs: Seba, Hawila, Sabtha, Ragma und Sabthecha; und die Söhne Ragmas:
Seba und Dedan. 10Kusch war der Vater Nimrods; dieser war der erste
Gewaltherrscher auf der Erde. 11Und Mizraim war der Stammvater der
Luditer, der Anamiter, der Lehabiter, der Naphthuhiter, 12der
Pathrusiter, der Kasluhiter -- von denen die Philister ausgegangen sind
-- und der Kaphthoriter. 13Kanaan aber hatte zu Söhnen: Sidon, seinen
Erstgeborenen, und den Heth, 14ferner die Jebusiter, Amoriter,
Girgasiter, 15Hewiter, Arkiter, Siniter, 16Arwaditer, Zemariter und
Hamathiter.

\hypertarget{cc-die-semiten}{%
\subparagraph{cc) Die Semiten}\label{cc-die-semiten}}

17Die Söhne Sems waren: Elam, Assur, Arpachsad, Lud, Aram, Uz, Hul,
Gether und Mesech. 18Arpachsad aber war der Vater Selahs, und Selah war
der Vater Ebers. 19Und dem Eber wurden zwei Söhne geboren: der eine hieß
Peleg\textless sup title=``d.h. Teilung''\textgreater✲, weil zu seiner
Zeit die Bevölkerung der Erde sich teilte, und sein Bruder hieß Joktan.
20Und Joktan hatte zu Söhnen: Almodad, Seleph, Hazarmaweth, Jerah,
21Hadoram, Usal, Dikla, 22Ebal, Abimael, Seba, 23Ophir, Hawila und
Jobab; diese alle waren Söhne Joktans.

\hypertarget{dd-die-gerade-linie-von-sem-bis-abraham}{%
\subparagraph{dd) Die gerade Linie von Sem bis
Abraham}\label{dd-die-gerade-linie-von-sem-bis-abraham}}

24Sem, Arpachsad, Selah, 25Eber, Peleg, Regu, 26Serug, Nahor, Therah,
27Abram, das ist Abraham.

\hypertarget{c-die-heidnischen-nachkommen-abrahams}{%
\paragraph{c) Die heidnischen Nachkommen
Abrahams}\label{c-die-heidnischen-nachkommen-abrahams}}

\hypertarget{aa-die-ismaeliten}{%
\subparagraph{aa) Die Ismaeliten}\label{aa-die-ismaeliten}}

28Die Söhne Abrahams waren: Isaak und Ismael. 29Dies ist ihr Stammbaum:
der Erstgeborene Ismaels Nebajoth; sodann Kedar, Adbeel, Mibsam,
30Misma, Duma, Massa, Hadad, Thema, 31Jetur, Naphis und Kedma; das sind
die Söhne Ismaels.

\hypertarget{bb-die-nachkommen-der-ketura}{%
\subparagraph{bb) Die Nachkommen der
Ketura}\label{bb-die-nachkommen-der-ketura}}

32Und die Söhne der Ketura, der Nebenfrau Abrahams: sie gebar Simran,
Joksan, Medan, Midian, Jisbak und Suah. Und die Söhne Joksans waren:
Seba und Dedan. 33Und die Söhne Midians: Epha, Epher, Hanoch, Abida und
Eldaa. Diese alle waren Söhne oder Enkel der Ketura.

\hypertarget{cc-die-nachkommen-esaus-oder-die-edomiter}{%
\subparagraph{cc) Die Nachkommen Esaus (oder: die
Edomiter)}\label{cc-die-nachkommen-esaus-oder-die-edomiter}}

34Abraham aber war der Vater Isaaks; die Söhne Isaaks waren: Esau und
Israel✲. 35Die Söhne Esaus waren: Eliphas, Reguel, Jegus, Jaglam und
Korah. 36Die Söhne des Eliphas waren: Theman, Omar, Zephi, Gaetham,
Kenas, Thimna und Amalek. 37Die Söhne Reguels waren: Nahath, Serah,
Samma und Missa.~-- 38Und die Söhne Seirs: Lotan, Sobal, Zibeon, Ana,
Dison, Ezer und Disan. 39Und die Söhne Lotans: Hori und Homam; und die
Schwester Lotans war Thimna. 40Die Söhne Sobals waren: Aljan, Manahath,
Ebal, Sephi und Onam; und die Söhne Zibeons: Ajja und Ana. 41Die Söhne
Anas waren: Dison; und die Söhne Disons: Hamran, Esban, Jithran und
Cheran. 42Die Söhne Ezers waren: Bilhan, Saawan und Jaakan. Die Söhne
Disans waren: Uz und Aran.

\hypertarget{die-edomitischen-kuxf6nige-und-huxe4uptlinge}{%
\subsubsection{2. Die edomitischen Könige und
Häuptlinge}\label{die-edomitischen-kuxf6nige-und-huxe4uptlinge}}

43Und dies sind die Könige, die im Lande Edom geherrscht haben, ehe ein
König über die Israeliten herrschte: Bela, der Sohn Beors; seine
Stadt\textless sup title=``d.h. die Hauptstadt, Residenz''\textgreater✲
hieß Dinhaba. 44Nach dem Tode Belas wurde Jobab, der Sohn Serahs, aus
Bozra, König an seiner Statt. 45Nach dem Tode Jobabs wurde Husam aus dem
Lande der Themaniter König an seiner Statt. 46Nach dem Tode Husams wurde
Hadad, der Sohn Bedads, König an seiner Statt, derselbe, der die
Midianiter auf der Hochebene der Moabiter schlug; seine Stadt hieß
Awith. 47Nach dem Tode Hadads wurde Samla aus Masreka König an seiner
Statt. 48Nach dem Tode Samlas wurde Saul aus Rehoboth am Euphratstrom
König an seiner Statt. 49Nach dem Tode Sauls wurde Baal-Hanan, der Sohn
Achbors, König an seiner Statt. 50Nach dem Tode Baal-Hanans wurde Hadad
König an seiner Statt: seine Stadt hieß Pagi, und seine Frau hieß
Mehetabeel, eine Tochter Matreds, eine Enkelin Mesahabs.~-- 51Nach dem
Tode Hadads waren die Häuptlinge der Edomiter: der Häuptling Thimna, der
Häuptling Alja\textless sup title=``oder: Alwa''\textgreater✲, der
Häuptling Jetheth, 52der Häuptling Oholibama, der Häuptling Ela, der
Häuptling Pinon, 53der Häuptling Kenas, der Häuptling Theman, der
Häuptling Mibzar, 54der Häuptling Magdiel, der Häuptling Iram. Dies sind
die Häuptlinge der Edomiter.

\hypertarget{die-suxf6hne-jakob-israels-und-die-geschlechter-des-stammes-juda}{%
\subsubsection{3. Die Söhne Jakob-Israels und die Geschlechter des
Stammes
Juda}\label{die-suxf6hne-jakob-israels-und-die-geschlechter-des-stammes-juda}}

\hypertarget{section-1}{%
\section{2}\label{section-1}}

1Dies sind die Söhne Israels: Ruben, Simeon, Levi und Juda, Issaschar
und Sebulon, 2Dan, Joseph und Benjamin, Naphthali, Gad und Asser.

\hypertarget{a-von-juda-bis-hezron}{%
\paragraph{a) Von Juda bis Hezron}\label{a-von-juda-bis-hezron}}

3Die Söhne Judas waren: Ger, Onan und Sela; diese drei wurden ihm von
der Tochter der Kanaanitin Sua geboren. Ger aber, der erstgeborene Sohn
Judas, machte sich dem HERRN verhaßt; darum ließ er ihn sterben. 4Thamar
aber, seine Schwiegertochter, hatte ihm Perez und Serah geboren, so daß
die Gesamtzahl der Söhne Judas fünf war. 5Die Söhne des Perez waren:
Hezron und Hamul; 6und die Söhne Serahs: Simri, Ethan, Heman, Chalkol
und Dara, im ganzen fünf.~-- 7Und die Söhne Karmis: Achar, der Israel
ins Unglück stürzte, weil er sich an gebanntem Gut vergriff. 8Und die
Söhne Ethans: Asarja.

\hypertarget{b-von-hezron-bis-david-die-linie-ram}{%
\paragraph{b) Von Hezron bis David (die Linie
Ram)}\label{b-von-hezron-bis-david-die-linie-ram}}

9Und die Söhne Hezrons, die ihm geboren wurden: Jerahmeel, Ram und
Kelubai✲. 10Ram aber zeugte Amminadab; und Amminadab zeugte Nahasson,
den Fürsten der Judäer; 11und Nahasson zeugte Salma; Salma zeugte Boas,
12Boas zeugte Obed, Obed zeugte Isai; 13Isai zeugte Eliab, seinen
Erstgeborenen, und Abinadab als zweiten, Simea als dritten, 14Nethaneel
als vierten, Raddai als fünften, 15Ozem als sechsten, David als siebten.
16Und ihre Schwestern waren: Zeruja und Abigail; und die Söhne der
Zeruja waren: Abisai, Joab und Asahel, die drei. 17Abigail aber gebar
Amasa, und der Vater Amasas war der Ismaelit Jether.

\hypertarget{c-die-linie-kaleb}{%
\paragraph{c) Die Linie Kaleb}\label{c-die-linie-kaleb}}

18Kaleb aber, der Sohn Hezrons, hatte Kinder von seiner Frau Asuba und
von Jerigoth; und dies sind deren Söhne: Jeser, Sobab und Ardon. 19Nach
dem Tode der Asuba aber heiratete Kaleb die Ephrath; die gebar ihm den
Hur. 20Hur aber zeugte Uri, und Uri zeugte Bezaleel.~-- 21Darnach
verband sich Hezron mit der Tochter Machirs, des Vaters Gileads, und
heiratete sie, als er sechzig Jahre alt war; die gebar ihm den Segub.
22Segub zeugte dann Jair; der besaß dreiundzwanzig Städte im Lande
Gilead; 23aber die Gesuriter und Syrer nahmen ihnen die Jairdörfer weg,
Kenath mit den zugehörigen Ortschaften, sechzig Städte. Diese alle waren
Söhne Machirs, des Vaters Gileads. 24Und nach dem Tode Hezrons in Kaleb
Ephratha gebar Abia, Hezrons Frau, ihm Ashur, den Stammvater von Thekoa.

\hypertarget{d-die-linie-jerahmeel}{%
\paragraph{d) Die Linie Jerahmeel}\label{d-die-linie-jerahmeel}}

25Die Söhne Jerahmeels, des ältesten Sohnes Hezrons, waren: der
Erstgeborene Ram, sodann Buna, Oren und Ozem von Ahija\textless sup
title=``oder: seine Brüder?''\textgreater✲. 26Jerahmeel hatte aber noch
eine andere Frau namens Atara; diese war die Mutter Onams.~-- 27Die
Söhne Rams, des erstgeborenen Sohnes Jerahmeels, waren: Maaz, Jamin und
Eker.~-- 28Die Söhne Onams waren: Sammai und Jada; und die Söhne
Sammais: Nadab und Abisur. 29Die Frau Abisurs aber hieß Abihail; die
gebar ihm den Achban und Molid. 30Und die Söhne Nadabs waren: Seled und
Appaim; Seled aber starb kinderlos.~-- 31Und die Söhne Appaims waren:
Jisei; und die Söhne Jiseis: Sesan; und die Söhne Sesans: Achlai.~--
32Die Söhne Jadas, des Bruders Sammais, waren: Jether und Jonathan;
Jether aber starb kinderlos. 33Die Söhne Jonathans waren: Peleth und
Sasa. Dies waren die Söhne Jerahmeels.~-- 34Sesan hatte keine Söhne,
sondern nur Töchter; er hatte aber einen ägyptischen Knecht namens
Jarha. 35Diesem gab Sesan seine Tochter zur Frau; die gebar ihm Atthai.
36Atthai zeugte dann Nathan, und Nathan zeugte Sabad; 37Sabad zeugte
Ephlal, Ephlal zeugte Obed, 38Obed zeugte Jehu, Jehu zeugte Asarja,
39Asarja zeugte Helez, Helez zeugte Elasa, 40Elasa zeugte Sisemai,
Sisemai zeugte Sallum, 41Sallum zeugte Jekamja, Jekamja zeugte Elisama.

\hypertarget{e-die-linie-kaleb-als-nachtrag-zu-v.18-24}{%
\paragraph{e) Die Linie Kaleb (als Nachtrag zu
V.18-24)}\label{e-die-linie-kaleb-als-nachtrag-zu-v.18-24}}

42Die Söhne Kalebs, des Bruders Jerahmeels, waren: sein Erstgeborener
Mesa -- der ist der Vater Siphs -- und die Söhne Maresas, des Vaters von
Hebron. 43Die Söhne Hebrons waren: Korah, Thappuah, Rekem und Sema.
44Sema aber zeugte Raham, den Vater Jorkeams; und Rekem zeugte Sammai.
45Der Sohn Sammais war Maon, und Maon war der Vater Beth-Zurs.~-- 46Und
Epha, das Nebenweib Kalebs, gebar Haran, Moza und Gases; Haran aber
zeugte Gases.~-- 47Und die Söhne Jahdais waren: Regem, Jotham, Geesan,
Pelet, Epha und Saaph.~-- 48Maacha, das Nebenweib Kalebs, gebar Seber
und Thirhana; 49sie gebar auch Saaph, den Vater Madmannas, und Sewa, den
Vater Machbenas und den Vater Gibeas; und die Tochter Kalebs war Achsa.
50Dies waren die Söhne Kalebs.

Die Söhne Hurs, des Erstgeborenen von der Ephratha, waren: Sobal, der
Stammvater von Kirjath-Jearim, 51Salma, der Stammvater von Bethlehem,
Hareph, der Stammvater von Beth-Gader. 52Und Sobal, der Stammvater von
Kirjath-Jearim, hatte als Söhne: Haroe, halb Menuhoth✲; 53und die
Geschlechter von Kirjath-Jearim waren: die Jithriter, die Puthiter, die
Sumathiter und die Misraiter; von diesen sind die Zorathiter und die
Esthauliter ausgegangen.~-- 54Die Söhne Salmas waren: Bethlehem und die
Netophathiter, Ateroth, Beth-Joab und die Hälfte der Manahthiter, die
Zoreiter; 55und die Geschlechter der Schriftgelehrten, die in Jabez
wohnten: die Thireathiter, die Simeathiter und die Suchathiter. Das sind
die Kiniter, die von Hammath, dem Stammvater des Hauses Rechab,
abstammen.

\hypertarget{f-stammbaum-des-hauses-davids}{%
\paragraph{f) Stammbaum des Hauses
Davids}\label{f-stammbaum-des-hauses-davids}}

\hypertarget{aa-davids-suxf6hne}{%
\subparagraph{aa) Davids Söhne}\label{aa-davids-suxf6hne}}

\hypertarget{section-2}{%
\section{3}\label{section-2}}

1Dies waren die Söhne Davids, die ihm in Hebron geboren wurden: Der
Erstgeborene war Amnon, von der Jesreelitin Ahinoam; der zweite
Daniel\textless sup title=``vgl. 2.Sam 3,3''\textgreater✲, von der
Karmelitin Abigail; 2der dritte Absalom, der Sohn der Maacha, der
Tochter des Königs Thalmai von Gesur; der vierte Adonia, der Sohn der
Haggith; 3der fünfte Sephatja, von Abital; der sechste Jithream, von
seiner Frau Egla. 4Diese sechs wurden ihm in Hebron geboren; dort
regierte er sieben Jahre und sechs Monate; aber in Jerusalem regierte er
dreiunddreißig Jahre. 5Folgende aber wurden ihm in Jerusalem geboren:
Simea, Sobab, Nathan und Salomo, zusammen vier, von Bath-Sewa✲, der
Tochter Ammiels; 6ferner Jibhar, Elisua, Eliphelet, 7Nogah, Nepheg,
Japhia, 8Elisama, Eljada und Eliphelet, zusammen neun. 9Dies sind
allesamt Söhne Davids, außer den Söhnen von Nebenweibern; und Thamar war
ihre Schwester.

\hypertarget{bb-die-davidischen-kuxf6nige-von-salomo-bis-zur-zerstuxf6rung-jerusalems}{%
\subparagraph{bb) Die davidischen Könige von Salomo bis zur Zerstörung
Jerusalems}\label{bb-die-davidischen-kuxf6nige-von-salomo-bis-zur-zerstuxf6rung-jerusalems}}

10Salomos Sohn war Rehabeam; dessen Sohn war Abija, dessen Sohn Asa,
dessen Sohn Josaphat, 11dessen Sohn Joram, dessen Sohn Ahasja, dessen
Sohn Joas, 12dessen Sohn Amazja, dessen Sohn Asarja, dessen Sohn Jotham,
13dessen Sohn Ahas, dessen Sohn Hiskia, dessen Sohn Manasse, 14dessen
Sohn Amon, dessen Sohn Josia.~-- 15Und die Söhne Josias waren: der
Erstgeborene Johanan, der zweite Jojakim, der dritte Zedekia, der vierte
Sallum. 16Die Söhne Jojakims waren: sein Sohn Jechonja; dessen Sohn war
Zedekia\textless sup title=``? vgl. V.15''\textgreater✲.

\hypertarget{cc-die-weiteren-nachkommen-davids-von-jechonja-an}{%
\subparagraph{cc) Die weiteren Nachkommen Davids (von Jechonja
an)}\label{cc-die-weiteren-nachkommen-davids-von-jechonja-an}}

17Die Söhne Jechonjas, des Gefangenen, waren: sein Sohn Sealthiel,
18Malchiram, Pedaja, Sena'azzar, Jekamja, Hosama und Nebadja. 19Die
Söhne Pedajas waren: Serubbabel und Simei; und die Söhne Serubbabels:
Mesullam und Hananja, und deren Schwester war Selomith; 20und (die Söhne
Mesullams waren:) Hasuba, Ohel, Berechja, Hasadja, Jusab-Hesed, zusammen
fünf. 21Die Söhne Hananjas waren: Pelatja und Jesaja; die Söhne
Rephajas, die Söhne Arnans, die Söhne Obadjas, die Söhne Sechanjas.
22Die Söhne Sechanjas waren: Semaja; und die Söhne Semajas: Hattus,
Jigeal, Bariah, Nearja und Saphat, zusammen sechs. 23Die Söhne Nearjas
waren: Eljoenai, Hiskia und Asrikam, zusammen drei. 24Die Söhne
Eljoenais aber waren: Hodawja, Eljasib, Pelaja, Akkub, Johanan, Delaja
und Anani, zusammen sieben.

\hypertarget{g-weitere-angaben-uxfcber-die-geschlechter-des-stammes-juda}{%
\paragraph{g) Weitere Angaben über die Geschlechter des Stammes
Juda}\label{g-weitere-angaben-uxfcber-die-geschlechter-des-stammes-juda}}

\hypertarget{section-3}{%
\section{4}\label{section-3}}

1Die Söhne Judas waren: Perez, Hezron, Karmi, Hur und Sobal. 2Reaja
aber, der Sohn Sobals, zeugte Jahath; Jahath zeugte Ahumai und Lahad.
Dies sind die Geschlechter der Zoreathiter.~-- 3Und dies sind die Söhne
Hurs, des Vaters Etams: Jesreel, Jisma und Jidbas; und ihre Schwester
hieß Hazlelponi; 4sodann Pnuel, der Vater Gedors, und Eser, der Vater
Husas. Dies sind die Söhne Hurs, des erstgeborenen Sohnes der Ephratha,
des Stammvaters von Bethlehem.~-- 5Ashur aber, der Stammvater von
Thekoa, hatte zwei Frauen: Helea und Naara. 6Naara gebar ihm Ahussam,
Hepher, Themni und die Ahasthariter. Dies sind die Söhne der Naara. 7Und
die Söhne der Helea waren: Zereth, Zohar\textless sup title=``oder:
Jizhar''\textgreater✲ und Ethnan (und Koz).~-- 8Koz aber zeugte Anub,
Zobeba und die Geschlechter Aharhels, des Sohnes Harums. 9Jaebez aber
war angesehener als seine Brüder; seine Mutter hatte ihn Jaebez genannt,
indem sie sagte: »Ich habe ihn mit Schmerzen geboren.« 10Jaebez aber
rief den Gott Israels an mit den Worten: »Ach, daß du mich segnetest und
mein Gebiet erweitertest und deine Hand mit mir wäre und du mich vor
Unglück behütetest, so daß mich kein Schmerz trifft!« Da erfüllte ihm
Gott seine Bitte.~-- 11Kelub aber, der Bruder Suhas, zeugte Mehir; der
ist der Vater Esthons. 12Esthon aber zeugte Beth-Rapha, Paseah und
Thehinna, den Stammvater der Stadt Nahas; das sind die Männer von
Recha.~-- 13Die Söhne des Kenas waren: Othniel und Seraja; und die Söhne
Othniels: Hathath (und Meonothai). 14Meonothai aber zeugte Ophra; und
Seraja zeugte Joab, den Stammvater des Tals der Zimmerleute\textless sup
title=``oder: Schmiede''\textgreater✲; sie waren nämlich
Zimmerleute\textless sup title=``oder: Schmiede''\textgreater✲.~-- 15Die
Söhne Kalebs, des Sohnes Jephunnes, waren: Iru, Ela und Naam; und der
Sohn Elas: Kenas.~-- 16Die Söhne Jehallelels waren: Siph und Sipha,
Thirja und Asareel.~-- 17Die Söhne Esras waren: Jether, Mered, Epher und
Jalon. Und dies sind die Söhne der Bithja, der Tochter des Pharaos, die
Mered geheiratet hatte: sie gebar Mirjam, Sammai und Jisbah, den
Stammvater von Esthemoa. 18Seine Frau aber, die Judäerin, gebar Jered,
den Stammvater von Gedor, und Heber, den Stammvater von Socho, und
Jekuthiel, den Stammvater von Sanoah.~-- 19Die Söhne von Hodijas Frau,
der Schwester Nahams, des Stammvaters von Kegila, waren: der Garmiter
und Esthemoa, der Maachathiter.~-- 20Die Söhne Simons aber waren: Amnon
und Rinna, Ben-Hanan und Thilon; -- und die Söhne Jiseis waren: Soheth
und der Sohn Soheths.

21Die Söhne Selas, des Sohnes Judas, waren: Ger, der Stammvater von
Lecha, und Laheda, der Stammvater von Maresa, und die Geschlechter der
Byssusarbeiter von Beth-Asbea; 22ferner Jokim und die Männer von Koseba
und Joas und Saraph, die über Moab herrschten, und Jasubi-Lechem. Doch
dies sind alte Geschichten. 23Dies sind die Töpfer und die Bewohner von
Netaim und Gedera; sie hatten dort ihren Wohnsitz bei dem König, in
seinem Dienste.

\hypertarget{der-stamm-simeon}{%
\subsubsection{4. Der Stamm Simeon}\label{der-stamm-simeon}}

\hypertarget{a-angaben-bezuxfcglich-der-nachkommen-simeons}{%
\paragraph{a) Angaben bezüglich der Nachkommen
Simeons}\label{a-angaben-bezuxfcglich-der-nachkommen-simeons}}

24Die Söhne Simeons waren: Nemuel und Jamin, Jarib, Serah, Saul;
25dessen Sohn war Sallum, dessen Sohn Mibsam, dessen Sohn Misma. 26Die
Söhne Mismas waren: sein Sohn Hammuel, dessen Sohn Sakkur, dessen Sohn
Simei. 27Simei aber hatte sechzehn Söhne und sechs Töchter, während
seine Brüder nicht viele Kinder hatten, und ihr ganzes Geschlecht
vermehrte sich nicht so stark wie die Judäer.

\hypertarget{b-die-uxe4ltesten-wohnsitze-des-stammes}{%
\paragraph{b) Die ältesten Wohnsitze des
Stammes}\label{b-die-uxe4ltesten-wohnsitze-des-stammes}}

28Sie wohnten aber in Beerseba, Molada, Hazar-Sual, 29in Bilha, Ezem,
Tholad, 30in Bethuel, Horma, Ziklag, 31in Beth-Markaboth, Hazar-Susim,
Beth-Birei und Saaraim. Dies waren ihre Städte bis zur Regierung Davids.
32Ihre Dörfer aber waren Etam, Ain, Rimmon, Thochen und Asan, zusammen
fünf Ortschaften. 33Dazu alle ihre Dörfer, die rings um diese
Ortschaften her lagen bis nach Baal hin. Dies waren ihre Wohnsitze; und
sie hatten ihr eigenes Geschlechtsverzeichnis.

\hypertarget{c-angabe-anderer-simeonitischer-familienhuxe4upter-die-zwei-eroberungszuxfcge-der-simeoniten}{%
\paragraph{c) Angabe anderer simeonitischer Familienhäupter; die zwei
Eroberungszüge der
Simeoniten}\label{c-angabe-anderer-simeonitischer-familienhuxe4upter-die-zwei-eroberungszuxfcge-der-simeoniten}}

34Ferner: Mesobab, Jamlech, Josa, der Sohn Amazjas, 35Joel und Jehu, der
Sohn Josibjas, des Sohnes Serajas, des Sohnes Asiels; 36Eljoenai,
Jaakoba, Jesohaja, Asaja, Adiel, Jesimiel und Benaja 37und Sisa, der
Sohn Sipheis, des Sohnes Allons, des Sohnes Jedajas, des Sohnes Simris,
des Sohnes Semajas; 38diese hier mit Namen Angeführten waren Fürsten in
ihren Geschlechtern, nachdem ihre Familien sich stark ausgebreitet
hatten. 39So zogen sie denn bis in die Gegend von Gedor\textless sup
title=``oder: Gerar?''\textgreater✲ hin, bis an die Ostseite des Tales,
um Weideplätze für ihre Herden zu suchen; 40und sie fanden dort fette
und gute Weide, und das Land war nach allen Seiten hin geräumig, ruhig
und friedlich; denn die früheren Bewohner waren Hamiten gewesen. 41So
kamen denn jene oben mit Namen Verzeichneten während der Regierung
Hiskias, des Königs von Juda, überfielen deren Zelte und die Mehuniter,
die sich dort vorfanden, vollzogen den Blutbann an ihnen bis auf den
heutigen Tag und ließen sich an ihrer Statt nieder; denn es gab dort
Weideplätze für ihre Herden.~-- 42Ein Teil aber von ihnen, den
Simeoniten, zog in das Bergland Seir, fünfhundert Mann, an ihrer Spitze
Pelatja, Neaja, Rephaja, und Ussiel, die Söhne Jiseis; 43sie erschlugen
dann die letzten Überreste der Amalekiter und sind daselbst wohnen
geblieben bis auf den heutigen Tag.

\hypertarget{der-stamm-ruben}{%
\subsubsection{5. Der Stamm Ruben}\label{der-stamm-ruben}}

\hypertarget{a-angaben-uxfcber-ruben-und-seine-nachkommen}{%
\paragraph{a) Angaben über Ruben und seine
Nachkommen}\label{a-angaben-uxfcber-ruben-und-seine-nachkommen}}

\hypertarget{section-4}{%
\section{5}\label{section-4}}

1Die Söhne Rubens, des Erstgeborenen Israels -- er war nämlich der
Erstgeborene; weil er aber seines Vaters Lager entweiht hatte, wurde
sein Erstgeburtsrecht den Söhnen Josephs, des Sohnes Israels, verliehen,
nur daß dieser im Geschlechtsverzeichnis nicht als Erstgeborener
verzeichnet wurde; 2denn Juda hatte zwar die Obmacht\textless sup
title=``oder: Vorrang''\textgreater✲ unter seinen Brüdern, so daß einer
aus ihm zum Fürsten gewählt wurde, aber das Erstgeburtsrecht wurde
Joseph zuteil --, 3die Söhne Rubens also, des erstgeborenen Sohnes
Israels, waren: Hanoch und Pallu, Hezron und Karmi. 4Die Söhne Joels
waren: sein Sohn Semaja, dessen Sohn Gog, dessen Sohn Simei, 5dessen
Sohn Micha, dessen Sohn Reaja, dessen Sohn Baal, 6dessen Sohn Beera, den
Thilgath-Pilneser, der König von Assyrien, in die Gefangenschaft führte;
er war ein Fürst der Rubeniten. 7Seine Brüder aber nach ihren Familien,
so wie sie nach ihrer Abstammung in ihr Geschlechtsverzeichnis
eingetragen wurden, waren: der erste Jehiel, dann Sacharja, 8dann Bela,
der Sohn Asas, des Sohnes Semas, des Sohnes Joels.

\hypertarget{b-geschichtliche-angabe-uxfcber-bela}{%
\paragraph{b) Geschichtliche Angabe über
Bela}\label{b-geschichtliche-angabe-uxfcber-bela}}

Dieser wohnte in Aroer und bis Nebo und Baal-Meon; 9und nach Osten zu
wohnte er bis an die Steppe hin, die sich vom Euphratstrom her
erstreckt; denn ihre Herden waren zahlreich in der Landschaft Gilead.
10Zur Zeit Sauls aber führten sie Krieg mit den Hagaritern; und als
diese durch ihre Hand gefallen waren, wohnten sie in deren Zeltlagern
auf der ganzen Ostseite von Gilead.

\hypertarget{der-stamm-gad}{%
\subsubsection{6. Der Stamm Gad}\label{der-stamm-gad}}

\hypertarget{a-angaben-uxfcber-die-geschlechter-und-wohnpluxe4tze-sowie-uxfcber-die-abschuxe4tzung-der-gaditen}{%
\paragraph{a) Angaben über die Geschlechter und Wohnplätze sowie über
die Abschätzung der
Gaditen}\label{a-angaben-uxfcber-die-geschlechter-und-wohnpluxe4tze-sowie-uxfcber-die-abschuxe4tzung-der-gaditen}}

11Die Gaditen wohnten ihnen gegenüber\textless sup title=``=~neben
ihnen''\textgreater✲ in der Landschaft Basan bis Salcha; 12an der Spitze
stand Joel, ihm zunächst Sapham, dann Jahenai und Saphat in Basan.
13Ihre Brüder nach ihren Familien waren: Michael, Mesullam, Seba, Jorai,
Jahekan, Sia und Eber, zusammen sieben. 14Dies waren die Söhne Abihails,
des Sohnes Huris, des Sohnes Jaroahs, des Sohnes Gileads, des Sohnes
Michaels, des Sohnes Jesisais, des Sohnes Jahdos, des Sohnes des Bus.
15Ahi, der Sohn Abdiels, des Sohnes Gunis, war das Haupt ihrer Familien.
16Sie wohnten aber in Gilead, in Basan und den zugehörigen Ortschaften
und auf allen Weidetriften Sarons bis an ihre Ausgänge\textless sup
title=``oder: Grenzen''\textgreater✲. 17Diese alle sind während der
Regierung Jothams, des Königs von Juda, und während der Regierung
Jerobeams, des Königs von Israel, im Geschlechtsverzeichnis
aufgezeichnet worden.

\hypertarget{b-die-kuxe4mpfe-der-drei-ostjordanischen-stuxe4mme-mit-den-hagaritern}{%
\paragraph{b) Die Kämpfe der drei ostjordanischen Stämme mit den
Hagaritern}\label{b-die-kuxe4mpfe-der-drei-ostjordanischen-stuxe4mme-mit-den-hagaritern}}

18Die Rubeniten, die Gaditen und der halbe Stamm Manasse, soweit sie
tapfere Leute waren, Männer, die Schild und Schwert trugen und den Bogen
spannten und kampfgeübt waren -- 44760 kriegstüchtige Männer --, 19die
führten Krieg mit den Hagaritern und mit Jetur, Naphis und Nodab. 20Und
sie gewannen die Oberhand über sie, so daß die Hagariter samt allen, die
mit ihnen verbündet waren, in ihre Hand gegeben wurden; denn als sie
während des Kampfes zu Gott um Hilfe riefen, ließ er sich von ihnen
erbitten, weil sie ihr Vertrauen auf ihn gesetzt hatten. 21Sie führten
dann das Vieh jener als Beute weg: 50000 Kamele, 250000 Stück Kleinvieh
und 2000 Esel; dazu 100000 Menschen als Sklaven. 22Denn viele waren
gefallen, vom Schwert erschlagen, weil der Krieg von Gott verhängt war.
Sie wohnten dann an ihrer Statt bis zur Wegführung in die
Gefangenschaft.

\hypertarget{der-halbe-stamm-manasse-im-osten}{%
\subsubsection{7. Der halbe Stamm Manasse im
Osten}\label{der-halbe-stamm-manasse-im-osten}}

\hypertarget{a-die-wohnsitze-und-die-geschlechtsteilung-der-manassiten}{%
\paragraph{a) Die Wohnsitze und die Geschlechtsteilung der
Manassiten}\label{a-die-wohnsitze-und-die-geschlechtsteilung-der-manassiten}}

23Die zum halben Stamm Manasse Gehörigen wohnten im Lande von Basan bis
Baal-Hermon und bis zum Senir\textless sup title=``vgl. Hes
27,5''\textgreater✲ und zum Hermongebirge. Sie waren zahlreich; 24und
dies waren ihre Familienhäupter: Epher, Jisei, Eliel, Asriel, Jeremia,
Hodawja und Jahdiel, tapfere Krieger, berühmte Männer, Häupter in ihren
Familien.

\hypertarget{b-bestrafung-des-abfalls-der-drei-ostjordanischen-stuxe4mme-durch-die-assyrischen-kuxf6nige}{%
\paragraph{b) Bestrafung des Abfalls der drei ostjordanischen Stämme
durch die assyrischen
Könige}\label{b-bestrafung-des-abfalls-der-drei-ostjordanischen-stuxe4mme-durch-die-assyrischen-kuxf6nige}}

25Da sie aber dem Gott ihrer Väter untreu wurden und mit den Göttern der
im Lande wohnenden heidnischen Völker, die doch Gott vor ihnen her
vertilgt hatte, Götzendienst trieben, 26da erregte der Gott Israels den
assyrischen König Pul und den assyrischen König Thilgath-Pilneser zur
Wut, so daß er die Rubeniten, die Gaditen und den halben Stamm Manasse
in die Gefangenschaft schleppte und sie nach Halah sowie an den Habor
und nach Hara und an den Fluß Gosan bringen ließ, bis auf den heutigen
Tag.

\hypertarget{der-stamm-levi}{%
\subsubsection{8. Der Stamm Levi}\label{der-stamm-levi}}

\hypertarget{a-von-levi-bis-zu-den-suxf6hnen-aarons}{%
\paragraph{a) Von Levi bis zu den Söhnen
Aarons}\label{a-von-levi-bis-zu-den-suxf6hnen-aarons}}

27Die Söhne Levis waren: Gerson, Kehath und Merari; 28und die Söhne
Kehaths: Amram, Jizhar, Hebron und Ussiel. 29Und die Söhne Amrams: Aaron
und Mose, und (ihre Schwester) Mirjam. Und die Söhne Aarons: Nadab und
Abihu, Eleasar und Ithamar.

\hypertarget{b-die-hohepriesterliche-linie-von-eleasar-bis-zur-babylonischen-gefangenschaft}{%
\paragraph{b) Die hohepriesterliche Linie von Eleasar bis zur
babylonischen
Gefangenschaft}\label{b-die-hohepriesterliche-linie-von-eleasar-bis-zur-babylonischen-gefangenschaft}}

30Eleasar zeugte Pinehas, Pinehas zeugte Abisua, 31Abisua zeugte Bukki,
Bukki zeugte Ussi, 32Ussi zeugte Serahja, Serahja zeugte Merajoth,
33Merajoth zeugte Amarja, Amarja zeugte Ahitub, 34Ahitub zeugte Zadok,
Zadok zeugte Ahimaaz, 35Ahimaaz zeugte Asarja, Asarja zeugte Johanan,
36Johanan zeugte Asarja -- das ist derselbe, der den Priesterdienst
versah in dem Tempel, den Salomo in Jerusalem erbaut hatte --; 37Asarja
zeugte Amarja, Amarja zeugte Ahitub, 38Ahitub zeugte Zadok, Zadok zeugte
Sallum, 39Sallum zeugte Hilkia, Hilkia zeugte Asarja, 40Asarja zeugte
Seraja, Seraja zeugte Jozadak; 41Jozadak aber zog mit, als der HERR die
Bewohner von Juda und Jerusalem durch Nebukadnezar in die Gefangenschaft
führen ließ.

\hypertarget{c-die-nachkommenschaft-levis}{%
\paragraph{c) Die Nachkommenschaft
Levis}\label{c-die-nachkommenschaft-levis}}

\hypertarget{section-5}{%
\section{6}\label{section-5}}

1Die Söhne Levis waren: Gersom, Kehath und Merari. 2Dies aber sind die
Namen der Söhne Gersoms: Libni und Simei. 3Und die Söhne Kehaths waren:
Amram, Jizhar, Hebron und Ussiel. 4Die Söhne Meraris waren: Mahli und
Musi. -- Dies aber sind die Geschlechter der Leviten nach ihren
Familienvätern: 5Von Gersom stammten: sein Sohn Libni, dessen Sohn
Jahath, dessen Sohn Simma, 6dessen Sohn Joah, dessen Sohn Iddo, dessen
Sohn Serah, dessen Sohn Jeathrai.~-- 7Die Söhne Kehaths waren: sein Sohn
Amminadab, dessen Sohn Korah, dessen Sohn Assir, 8dessen Sohn Elkana,
dessen Sohn Ebjasaph, dessen Sohn Assir, 9dessen Sohn Thahath, dessen
Sohn Uriel, dessen Sohn Ussia, dessen Sohn Saul. 10Und die Söhne Elkanas
waren: Amasai und Ahimoth; 11dessen Sohn war Elkana, dessen Sohn Zophai,
dessen Sohn Nahath, 12dessen Sohn Eliab, dessen Sohn Jeroham, dessen
Sohn Elkana (dessen Sohn Samuel). 13Und die Söhne Samuels waren: der
Erstgeborene Joel und der zweite Abia.~-- 14Die Söhne Meraris waren:
Mahli, dessen Sohn Libni, dessen Sohn Simei, dessen Sohn Ussa, 15dessen
Sohn Simea, dessen Sohn Haggia, dessen Sohn Asaja.

\hypertarget{d-die-drei-levitischen-suxe4ngerfamilien-heman-asaph-und-ethan}{%
\paragraph{d) Die drei levitischen Sängerfamilien Heman, Asaph und
Ethan}\label{d-die-drei-levitischen-suxe4ngerfamilien-heman-asaph-und-ethan}}

16Die folgenden sind es, die David zur Leitung des Gesangs im Tempel des
HERRN bestellte, nachdem die Lade dort einen festen Platz gefunden
hatte; 17sie dienten aber als Sänger vor der Wohnung des
Offenbarungszeltes, bis Salomo den Tempel des HERRN in Jerusalem erbaut
hatte, und verrichteten ihr Amt nach den ihnen erteilten Weisungen.
18Die folgenden sind es, die das Amt versahen, und ihre Nachkommen: Von
den zu den Kehathiten Gehörigen: Heman, der Sänger, der Sohn Joels, des
Sohnes Samuels, 19des Sohnes Elkanas, des Sohnes Jerohams, des Sohnes
Eliels, des Sohnes Thoahs, 20des Sohnes Zuphs, des Sohnes Elkanas, des
Sohnes Mahaths, des Sohnes Amasais, 21des Sohnes Elkanas, des Sohnes
Joels, des Sohnes Asarjas, des Sohnes Zephanjas, 22des Sohnes Thahaths,
des Sohnes Assirs, des Sohnes Ebjasaphs, des Sohnes Korahs, 23des Sohnes
Jizhars, des Sohnes Kehaths, des Sohnes Levis, des Sohnes Israels.

24Und sein Geschlechtsgenosse war Asaph, der ihm zur Rechten stand:
Asaph, der Sohn Berechjas, des Sohnes Simeas, 25des Sohnes Michaels, des
Sohnes Baasejas, des Sohnes Malkias, 26des Sohnes Ethnis, des Sohnes
Serahs, des Sohnes Adajas, 27des Sohnes Ethans, des Sohnes Simmas, des
Sohnes Simeis, 28des Sohnes Jahaths, des Sohnes Gersoms, des Sohnes
Levis.

29Und von den Nachkommen Meraris, ihren Geschlechtsgenossen, stand zur
Linken: Ethan, der Sohn Kisis, des Sohnes Abdis, des Sohnes Malluchs,
30des Sohnes Hasabjas, des Sohnes Amazjas, des Sohnes Hilkias, 31des
Sohnes Amzis, des Sohnes Banis, des Sohnes Semers, 32des Sohnes Mahlis,
des Sohnes Musis, des Sohnes Meraris, des Sohnes Levis.

\hypertarget{e-die-leviten-und-aaroniten-im-tempeldienst}{%
\paragraph{e) Die Leviten und Aaroniten im
Tempeldienst}\label{e-die-leviten-und-aaroniten-im-tempeldienst}}

33Ihre Geschlechtsgenossen aber, die übrigen Leviten, waren für den
gesamten Dienst an der Wohnung des Tempels Gottes bestellt; 34und zwar
versahen Aaron und seine Nachkommen den Opferdienst auf dem
Brandopferaltar und auf dem Räucheraltar und den gesamten Dienst am
Allerheiligsten und was zur Versöhnung Israels diente, ganz so, wie
Mose, der Knecht Gottes, geboten hatte.

\hypertarget{zweite-hohepriesterliche-linie-von-aaron-bis-ahimaaz}{%
\paragraph{Zweite hohepriesterliche Linie von Aaron bis
Ahimaaz}\label{zweite-hohepriesterliche-linie-von-aaron-bis-ahimaaz}}

35Dies waren aber die Nachkommen Aarons: sein Sohn Eleasar, dessen Sohn
Pinehas, dessen Sohn Abisua, 36dessen Sohn Bukki, dessen Sohn Ussi,
dessen Sohn Serahja, 37dessen Sohn Merajoth, dessen Sohn Amarja, dessen
Sohn Ahitub, 38dessen Sohn Zadok, dessen Sohn Ahimaaz.

\hypertarget{f-die-levitenstuxe4dte}{%
\paragraph{f) Die Levitenstädte}\label{f-die-levitenstuxe4dte}}

39Folgendes aber sind ihre Wohnsitze nach ihren Niederlassungen in ihrem
Gebiet: Den Nachkommen Aarons von dem Geschlecht der Kehathiten -- denn
auf sie war das erste Los gefallen --, 40ihnen gab man Hebron im Lande
Juda samt den zugehörigen Weidetriften rings um sie her; 41die zur Stadt
gehörigen Felder aber und die zugehörigen Dörfer übergab man Kaleb, dem
Sohne Jephunnes. 42Weiter übergab man den Nachkommen Aarons die
Zufluchtstadt Hebron sowie Libna samt den zugehörigen Weidetriften,
ferner Jatthir und Esthemoa samt den zugehörigen Weidetriften,
43Holon\textless sup title=``vgl. Jos 21,15''\textgreater✲ samt den
zugehörigen Weidetriften, Debir samt den zugehörigen Weidetriften,
44Asan samt den zugehörigen Weidetriften und Beth-Semes samt den
zugehörigen Weidetriften. 45Weiter vom Stamme Benjamin: Geba und
Allemeth samt den zugehörigen Weidetriften und Anathoth samt den
zugehörigen Weidetriften. Die Gesamtzahl ihrer Städte betrug dreizehn
nach ihren Familien.~-- 46Die übrigen Nachkommen Kehaths aber erhielten
nach ihren Geschlechtern vom Stamme Ephraim und vom Stamme Dan und vom
halben Stamm Manasse durchs Los zehn Städte.

47Die Nachkommen Gersoms erhielten nach ihren Geschlechtern vom Stamme
Issaschar, vom Stamme Asser, vom Stamme Naphthali und vom halben Stamm
Manasse in Basan dreizehn Städte.~-- 48Die Nachkommen Meraris erhielten
nach ihren Geschlechtern vom Stamme Ruben, vom Stamme Gad und vom Stamme
Sebulon durchs Los zwölf Städte.

49So übergaben die Israeliten den Leviten die Städte samt den
zugehörigen Weidetriften; 50und zwar übergaben sie durchs Los jene
namentlich angeführten Städte aus den Stämmen Juda, Simeon und Benjamin.

51Was dann die übrigen Geschlechter der Nachkommen Kahaths anbetrifft,
so erhielten sie die Städte, die ihnen durchs Los zufielen, vom Stamme
Ephraim abgetreten; 52und zwar übergab man ihnen die Zufluchtstadt
Sichem samt den zugehörigen Weidetriften auf dem Gebirge Ephraim; ferner
Geser samt den zugehörigen Weidetriften, 53Jokmeam samt den zugehörigen
Weidetriften, Beth-Horon samt den zugehörigen Weidetriften, 54Ajjalon
samt den zugehörigen Weidetriften und Gath-Rimmon samt den zugehörigen
Weidetriften; 55dazu vom halben Stamm Manasse: Aner\textless sup
title=``oder: Taanach? vgl. Jos 21,25''\textgreater✲ samt den
zugehörigen Weidetriften und Jibleam samt den zugehörigen Weidetriften
-- diese übergab man den Familien der übrigen Nachkommen Kahaths.

56Die Nachkommen Gersoms erhielten nach ihren Geschlechtern vom halben
Stamm Manasse: Golan in Basan samt den zugehörigen Weidetriften und
Astharoth samt den zugehörigen Weidetriften; 57ferner vom Stamme
Issaschar: Kedes samt den zugehörigen Weidetriften, Daberath samt den
zugehörigen Weidetriften, 58Ramoth samt den zugehörigen Weidetriften und
Anem samt den zugehörigen Weidetriften; 59dazu vom Stamme Asser: Masal
samt den zugehörigen Weidetriften, Abdon samt den zugehörigen
Weidetriften, 60Hukok samt den zugehörigen Weidetriften und Rehob samt
den zugehörigen Weidetriften; 61ferner vom Stamme Naphthali: Kedes in
Galiläa samt den zugehörigen Weidetriften, Hammot samt den zugehörigen
Weidetriften und Kirjathaim samt den zugehörigen Weidetriften.

62Die übrigen Nachkommen Meraris erhielten vom Stamme Sebulon: Rimmon
samt den zugehörigen Weidetriften und Thabor samt den zugehörigen
Weidetriften; 63und jenseits des Jordans gegenüber von Jericho, östlich
vom Jordan, erhielten sie vom Stamme Ruben: Bezer in der Steppe samt den
zugehörigen Weidetriften, Jahza samt den zugehörigen Weidetriften,
64Kedemoth samt den zugehörigen Weidetriften und Mephaath samt den
zugehörigen Weidetriften; 65ferner vom Stamme Gad: Ramoth in Gilead samt
den zugehörigen Weidetriften, Mahanaim samt den zugehörigen
Weidetriften; 66Hesbon samt den zugehörigen Weidetriften und Jaser samt
den zugehörigen Weidetriften.

\hypertarget{der-stamm-issaschar}{%
\subsubsection{9. Der Stamm Issaschar}\label{der-stamm-issaschar}}

\hypertarget{section-6}{%
\section{7}\label{section-6}}

1Die Söhne Issaschars waren: Thola und Pua, Jasub und Simron, zusammen
vier. 2Die Söhne Tholas waren: Ussi, Rephaja, Jeriel, Jahmai, Jibsam und
Samuel, Häupter ihrer Familien, von Thola, kriegstüchtige Männer, nach
ihren Geschlechtern; ihre Zahl betrug zur Zeit Davids 22600. 3Die Söhne
Ussis waren: Jisrahja; und die Söhne Jisrahjas: Michael, Obadja, Joel
und Jissia, insgesamt fünf Familienhäupter. 4Zu ihnen gehörten nach
ihren Geschlechtern, nach ihren Familien, Kriegerscharen: 36000~Mann;
denn sie hatten viele Frauen und Kinder. 5Dazu ihre Stammesgenossen,
sämtliche Geschlechter Issaschars, waren kriegstüchtige Männer; 87000
ergab ihr Verzeichnis im ganzen.

\hypertarget{der-stamm-benjamin-und-dan-v.12}{%
\subsubsection{10. Der Stamm Benjamin (und Dan?
V.12)}\label{der-stamm-benjamin-und-dan-v.12}}

6Die Söhne Benjamins waren: Bela, Becher und Jediael, zusammen drei.
7Die Söhne Belas waren: Ezbon, Ussi, Ussiel, Jerimoth und Iri, zusammen
fünf, Familienhäupter, kriegstüchtige Männer; ihr Verzeichnis ergab
22034. 8Die Söhne Bechers waren: Semira, Joas, Elieser, Eljoenai, Omri,
Jeremoth, Abia, Anathoth und Alemeth; alle diese waren Söhne Bechers,
9und ihr Verzeichnis nach ihren Geschlechtern, nach ihren
Familienhäuptern ergab 20200 kriegstüchtige Männer. 10Die Söhne Jediaels
waren: Bilhan; und die Söhne Bilhans: Jehus, Benjamin, Ehud, Kenaana,
Sethan, Tharsis und Ahisahar; 11alle diese waren Söhne Jediaels,
Familienhäupter, kriegstüchtige Männer, 17200, die kampfbereit ins Feld
zogen.~-- 12Und Suppim und Huppim waren Söhne Irs; Husim die Söhne
Ahers.

\hypertarget{der-stamm-naphthali}{%
\subsubsection{11. Der Stamm Naphthali}\label{der-stamm-naphthali}}

13Die Söhne Naphthalis waren: Jahziel, Guni, Jezer und Sallum,
Nachkommen der Bilha.

\hypertarget{der-stamm-westmanasse}{%
\subsubsection{12. Der Stamm Westmanasse}\label{der-stamm-westmanasse}}

14Die Söhne Manasses waren: Asriel, den sein syrisches Nebenweib gebar;
sie gebar auch Machir, den Vater Gileads. 15Machir nahm dann {[}für
Huppim und Suppim{]} eine Frau namens Maacha, und seine Schwester hieß
Hammolecheth; sein Bruder hieß Zelophhad, und dieser hatte nur Töchter.
16Und Maacha, die Frau Machirs\textless sup title=``oder: Gileads; vgl.
V.15''\textgreater✲, gebar einen Sohn, den sie Peres nannte; sein Bruder
aber hieß Seres, und dessen Söhne waren Ulam und Rekem. 17Die Söhne
Ulams waren: Bedan. Dies sind die Söhne Gileads, des Sohnes Machirs, des
Sohnes Manasses.~-- 18Seine Schwester Hammolecheth aber gebar Ishod,
Abieser und Mahla.~-- 19Und die Söhne Semidas waren: Ahjan, Sichem,
Likhi und Aniam.

\hypertarget{der-stamm-ephraim}{%
\subsubsection{13. Der Stamm Ephraim}\label{der-stamm-ephraim}}

\hypertarget{a-die-geschlechter-des-stammes-mit-geschichtlichen-angaben}{%
\paragraph{a) Die Geschlechter des Stammes mit geschichtlichen
Angaben}\label{a-die-geschlechter-des-stammes-mit-geschichtlichen-angaben}}

20Die Söhne Ephraims waren: Suthelah; dessen Sohn war Bered, dessen Sohn
Thahath, dessen Sohn Elada, dessen Sohn Thahath, 21dessen Sohn Sabad,
dessen Sohn Suthelah und Eser und Elead. Die Bewohner von Gath aber, die
Eingeborenen des Landes, erschlugen sie, weil sie hinabgezogen waren, um
ihnen ihre Herden zu rauben. 22Da trauerte ihr Vater Ephraim lange Zeit,
und seine Brüder kamen, um ihn zu trösten. 23Da wohnte er seiner Frau
bei, und sie wurde guter Hoffnung und gebar einen Sohn, den er
Beria\textless sup title=``d.h. im Unglück''\textgreater✲ nannte, weil
sein Haus sich im Unglück befunden hatte. 24Seine Tochter aber war
Seera; die erbaute das untere und das obere Beth-Horon und Ussen-Seera.
25Dessen\textless sup title=``d.h. Berias''\textgreater✲ Sohn war
Rephah, dessen Sohn Reseph, dessen Sohn Thelah, dessen Sohn Thahan,
26dessen Sohn Laedan, dessen Sohn Ammihud, dessen Sohn Elisama, 27dessen
Sohn Nun, dessen Sohn Josua.

\hypertarget{b-wohnsitze-des-stammes}{%
\paragraph{b) Wohnsitze des Stammes}\label{b-wohnsitze-des-stammes}}

28Ihr Erbbesitz aber und ihre Wohnsitze waren: Bethel samt den
zugehörigen Ortschaften, und nach Osten hin Naaran und gegen Westen
Geser samt den zugehörigen Ortschaften; sodann Sichem samt den
zugehörigen Ortschaften bis nach Ajja\textless sup title=``oder:
Gaza?''\textgreater✲ samt den zugehörigen Ortschaften. 29Und im Besitz
der Manassiten waren: Beth-Sean samt den zugehörigen Ortschaften,
Thaanach samt den zugehörigen Ortschaften, Megiddo samt den zugehörigen
Ortschaften, Dor samt den zugehörigen Ortschaften. In diesen wohnten die
Nachkommen Josephs, des Sohnes Israels.

\hypertarget{der-stamm-asser}{%
\subsubsection{14. Der Stamm Asser}\label{der-stamm-asser}}

30Die Söhne Assers waren: Jimna, Jiswa, Jiswi und Beria; ihre Schwester
war Serah. 31Die Söhne Berias waren: Heber und Malkiel, das ist der
Stammvater von Birsajith. 32Heber zeugte Japhlet, Semer, Hotham und ihre
Schwester Sua. 33Die Söhne Japhlets waren: Pasach, Bimhal und Aswath;
dies waren die Söhne Japhlets; 34und die Söhne (seines Bruders) Semers:
Ahi, Rohga, Hubba und Aram.~-- 35Die Söhne seines Bruders Hotham waren:
Zophah, Jimna, Seles und Amal. 36Die Söhne Zophahs waren: Suah,
Harnepher, Sual, Beri, Jimra, 37Bezer, Hod, Samma, Silsa, Jithran und
Beera; 38und die Söhne Jethers: Jephunne, Pispa und Ara.~-- 39Die Söhne
Ullas waren: Arah, Hanniel und Rizja. 40Diese alle waren Nachkommen
Assers, Familienhäupter, auserlesene, kriegstüchtige Männer, Häupter
unter den Fürsten; und die Zahl der aus ihnen für den Kriegsdienst
Aufgezeichneten betrug 26000~Mann.

\hypertarget{genauere-angaben-uxfcber-den-stamm-benjamin}{%
\subsubsection{15. Genauere Angaben über den Stamm
Benjamin}\label{genauere-angaben-uxfcber-den-stamm-benjamin}}

\hypertarget{a-benjamins-suxf6hne-und-nachkommen-durch-bela}{%
\paragraph{a) Benjamins Söhne und Nachkommen durch
Bela}\label{a-benjamins-suxf6hne-und-nachkommen-durch-bela}}

\hypertarget{section-7}{%
\section{8}\label{section-7}}

1Benjamin zeugte Bela als seinen Erstgeborenen, Asbel als den zweiten,
Ahiram\textless sup title=``4.Mose 26,38''\textgreater✲ als den dritten,
2Noha als den vierten und Rapha als den fünften. 3Bela aber hatte
folgende Söhne: Ard\textless sup title=``4.Mose 26,40''\textgreater✲,
Gera, Abihud, 4Abisua, Naaman, Ahoah, 5Gera, Sephuphan und Huram.

\hypertarget{b-die-suxf6hne-ehuds}{%
\paragraph{b) Die Söhne Ehuds}\label{b-die-suxf6hne-ehuds}}

6Und dies waren die Söhne Ehuds: -- sie waren Familienhäupter unter den
Bewohnern von Geba, und man führte sie gefangen weg nach Manahath,
7nämlich Naaman, Ahia und Gera; dieser führte sie weg; -- er zeugte aber
Ussa und Ahihud.

\hypertarget{c-das-geschlecht-saharaims}{%
\paragraph{c) Das Geschlecht
Saharaims}\label{c-das-geschlecht-saharaims}}

8Saharaim aber zeugte (Söhne) im Gefilde der Moabiter, nachdem er seine
Frauen Husim und Baara verstoßen hatte; 9da zeugte er mit seiner Frau
Hodes: Jobab, Zibja, Mesa, Malkam, 10Jehuz, Sochja und Mirma; dies waren
seine Söhne, Familienhäupter. 11Mit Husim aber hatte er Abitub und
Elpaal gezeugt. 12Die Söhne Elpaals waren: Eber, Miseam und Semer;
dieser erbaute Ono und Lod samt den zugehörigen Ortschaften.

\hypertarget{d-fuxfcnf-benjaminitische-geschlechter-in-ajjalon-und-jerusalem}{%
\paragraph{d) Fünf benjaminitische Geschlechter in Ajjalon und
Jerusalem}\label{d-fuxfcnf-benjaminitische-geschlechter-in-ajjalon-und-jerusalem}}

13Weiter: Beria und Sema -- das waren die Familienhäupter unter den
Bewohnern von Ajjalon; sie hatten die Bewohner von Gath in die Flucht
geschlagen --; 14(und ihre Brüder waren Elpaal,) Sasak und Jeremoth.
15Sebadja aber und Arad, Eder, 16Michael, Jispa und Joha waren die Söhne
Berias; 17und Sebadja, Mesullam, Hiski, Heber, 18Jismerai, Jislia und
Jobab waren die Söhne Elpaals.~-- 19Jakim, Sichri, Sabdi, 20Eljoenai,
Zillethai, Eliel, 21Adaja, Beraja und Simrath waren die Söhne Simeis.~--
22Jispan, Eber, Eliel, 23Abdon, Sichri, Hanan, 24Hananja, Elam,
Anthothija, 25Jiphdeja und Pnuel waren die Söhne Sasaks.~-- 26Samserai,
Seharja, Athalja, 27Jaaresja, Elia und Sichri waren die Söhne Jerohams.
28Diese waren Familienhäupter in ihren Geschlechtern, Häupter; diese
wohnten in Jerusalem.

\hypertarget{e-das-geschlecht-des-kuxf6nigs-saul}{%
\paragraph{e) Das Geschlecht des Königs
Saul}\label{e-das-geschlecht-des-kuxf6nigs-saul}}

29In Gibeon aber wohnten: Jehuel\textless sup title=``vgl.
9,35''\textgreater✲, der Stammvater von Gibeon, dessen Frau Maacha hieß.
30Sein erstgeborener Sohn war Abdon; außerdem Zur, Kis, Baal, Ner,
Nadab, 31Gedor, Ahjo, Secher 32und Mikloth, der Simea zeugte. Auch diese
wohnten ihren Stammesgenossen gegenüber in Jerusalem bei ihren
Stammesgenossen.~-- 33Ner aber zeugte Abner, und Kis zeugte Saul, Saul
zeugte Jonathan, Malchisua, Abinadab und Esbaal. 34Der Sohn Jonathans
war Merib-Baal\textless sup title=``vgl. 2.Sam 4,4''\textgreater✲, und
dieser zeugte Micha. 35Die Söhne Michas waren: Pithon, Melech, Tharea
und Ahas. 36Ahas zeugte Jehoadda, Jehoadda zeugte Alemeth, Asmaweth und
Simri; Simri aber zeugte Moza, 37Moza zeugte Binea; dessen Sohn war
Rapha, dessen Sohn Eleasa, dessen Sohn Azel. 38Azel aber hatte sechs
Söhne, die hießen: Asrikam, Bochru, Ismael, Searja, Obadja und Hanan.
Alle diese waren Söhne Azels.~-- 39Die Söhne seines Bruders Esek waren:
Ulam, sein Erstgeborener, Jehus der zweite und Eliphelet der dritte.
40Die Söhne Ulams waren kriegstüchtige Männer, die den Bogen zu spannen
wußten, und sie hatten zahlreiche Söhne und Enkel, hundertundfünfzig an
der Zahl. Diese alle gehören zu den Nachkommen Benjamins.

\hypertarget{verzeichnis-von-hervorragenden-bewohnern-jerusalems-in-der-zeit-nach-der-gefangenschaft}{%
\subsubsection{16. Verzeichnis von hervorragenden Bewohnern Jerusalems
(in der Zeit nach der
Gefangenschaft)}\label{verzeichnis-von-hervorragenden-bewohnern-jerusalems-in-der-zeit-nach-der-gefangenschaft}}

\hypertarget{a-einleitende-geschichtliche-angaben}{%
\paragraph{a) Einleitende geschichtliche
Angaben}\label{a-einleitende-geschichtliche-angaben}}

\hypertarget{section-8}{%
\section{9}\label{section-8}}

1Alle Israeliten wurden nach den Geschlechtern in ein Verzeichnis
eingetragen; sie finden sich bekanntlich im Buche der Könige von Israel
aufgezeichnet. Die Judäer aber wurden wegen ihres treulosen Abfalls nach
Babylon (in die Gefangenschaft) geführt. 2Die früheren Bewohner nun, die
in ihrem Erbbesitz, in ihren Ortschaften lebten, waren: die (gemeinen)
Israeliten, die Priester, die Leviten und die Tempelhörigen\textless sup
title=``vgl. Neh 11,3''\textgreater✲.

\hypertarget{b-die-bewohnerschaft-jerusalems}{%
\paragraph{b) Die Bewohnerschaft
Jerusalems}\label{b-die-bewohnerschaft-jerusalems}}

3In Jerusalem aber wohnten von den Judäern und Benjaminiten sowie von
den Ephraimiten und Manassiten folgende:

4Von den Judäern: Uthai, der Sohn Ammihuds, des Sohnes Omris, des Sohnes
Imris, des Sohnes Banis, von den Nachkommen des Perez, des Sohnes Judas;
5und von den Siloniten: Asaja, der Erstgeborene, und seine Söhne; 6und
von den Nachkommen Serahs: Jehuel und seine Geschlechtsgenossen,
zusammen 690.

7Ferner von den Benjaminiten: Sallu, der Sohn Mesullams, des Sohnes
Hodawjas, des Sohnes Hassenuas; 8sodann Jibneja, der Sohn Jerohams, und
Ela, der Sohn Ussis, des Sohnes Michris; und Mesullam, der Sohn
Sephatjas, des Sohnes Reguels, des Sohnes Jibnejas; 9dazu ihre Genossen
nach ihren Geschlechtern, zusammen 956. Alle diese Männer waren
Familienhäupter in ihren Familien.

10Ferner von den Priestern: Jedaja, Jojarib, Jachin 11und Asarja, der
Sohn Hilkias, des Sohnes Mesullams, des Sohnes Zadoks, des Sohnes
Merajoths, des Sohnes Ahitubs, der Fürst✲ des Tempels Gottes; 12sodann
Adaja, der Sohn Jerohams, des Sohnes Pashurs, des Sohnes Malkias; und
Maesai, der Sohn Adiels, des Sohnes Jahseras, des Sohnes Mesullams, des
Sohnes Mesillemiths, des Sohnes Immers; 13dazu ihre Geschlechtsgenossen,
Häupter ihrer Familien, zusammen 1760, tüchtige Männer zur Verrichtung
des Dienstes im Tempel Gottes.

14Ferner von den Leviten: Semaja, der Sohn Hasubs, des Sohnes Asrikams,
des Sohnes Hasabjas, von den Nachkommen Meraris; 15und Bakbakkar, Heres,
Galal, Matthanja, der Sohn Michas, des Sohnes Sichris, des Sohnes
Asaphs; 16und Obadja, der Sohn Semajas, des Sohnes Gallals, des Sohnes
Jeduthuns; und Berechja, der Sohn Asas, des Sohnes Elkanas, der in den
Dörfern der Netophathiter wohnte.

\hypertarget{die-torhuxfcter-und-ihre-dienstleistungen}{%
\paragraph{Die Torhüter und ihre
Dienstleistungen}\label{die-torhuxfcter-und-ihre-dienstleistungen}}

17Ferner die Torhüter: Sallum, Akkub, Talmon und Ahiman mit ihren
Geschlechtsgenossen; Sallum war das Haupt 18und hat bis heute die Wache
am Königstor auf der Ostseite. Das sind die Torhüter im Lager der
Leviten. 19Sallum aber, der Sohn Kores, des Sohnes Ebjasaphs, des Sohnes
Korahs, und seine Geschlechtsgenossen aus seiner Familie, die Korahiten,
waren mit der Dienstleistung als Schwellenhüter am heiligen Zelt
betraut, wie auch schon ihre Vorfahren im Lager des HERRN als Hüter des
Eingangs gedient hatten: 20Pinehas, der Sohn Eleasars\textless sup
title=``vgl. 4.Mose 25,11''\textgreater✲, -- der HERR sei mit ihm! --
war vor Zeiten ihr Vorsteher gewesen. 21Sacharja aber, der Sohn
Meselemjas, war Torhüter am Eingang des Offenbarungszeltes. 22Die
Gesamtzahl derer, die zu Torhütern an den Schwellen ausersehen waren,
betrug 212; sie waren in ihren Dörfern in ein Verzeichnis eingetragen;
David und Samuel, der Seher, hatten sie in ihr Amt eingesetzt. 23Sie
selbst und ihre Nachkommen standen an den Toren des Hauses Gottes, der
Zeltwohnung, um Wache zu halten. 24Nach den vier Weltgegenden sollten
die Torhüter stehen, nach Osten, Westen, Norden und Süden. 25Ihre
Geschlechtsgenossen aber in ihren Dörfern hatten sich von sieben zu
sieben Tagen, von einem Zeitpunkte zum andern, zugleich mit jenen zum
Dienst einzustellen; 26denn sie, die vier Vorsteher der Torhüter,
standen dauernd in Ausübung ihres Amtes.

\hypertarget{angaben-uxfcber-die-dienstlichen-obliegenheiten-der-leviten}{%
\paragraph{Angaben über die dienstlichen Obliegenheiten der
Leviten}\label{angaben-uxfcber-die-dienstlichen-obliegenheiten-der-leviten}}

Das sind die Leviten. Sie waren aber auch über die Zellen und
Schatzkammern\textless sup title=``oder: Vorratskammern''\textgreater✲
im Tempel Gottes gesetzt 27und blieben auch über Nacht rings um das
Gotteshaus her, weil ihnen die Bewachung oblag und sie das Aufschließen
zu besorgen hatten, und zwar Morgen für Morgen. 28Einige von ihnen
hatten auch die Aufsicht über die gottesdienstlichen Geräte, die beim
Hinein- wie beim Herausbringen allemal gezählt wurden. 29Ferner waren
einige von ihnen für die Gerätschaften angestellt, und zwar für alle
Gerätschaften des Heiligtums, sowie für das Feinmehl, den Wein, das Öl,
den Weihrauch und die Spezereien✲. 30Einige von den Mitgliedern der
Priesterschaft hatten die wohlriechenden Salb-Öle aus den Spezereien
herzustellen; 31und Matthithja, einem der Leviten, dem Erstgeborenen des
Korahiten Sallum, war die Pfannenbäckerei anvertraut. 32Ferner waren
einige von den Kehathiten, ihren Geschlechtsgenossen, für die
Schaubrote\textless sup title=``oder: Schichtbrote''\textgreater✲
bestellt, die sie auf jeden Sabbat zuzurichten hatten.

\hypertarget{angabe-uxfcber-die-tempelsuxe4nger-schluuxdfwort}{%
\paragraph{Angabe über die Tempelsänger;
Schlußwort}\label{angabe-uxfcber-die-tempelsuxe4nger-schluuxdfwort}}

33Das sind nun die Sänger, levitische Familienhäupter, die in den
Zellen✲ wohnten, von anderen Dienstleistungen befreit; denn Tag und
Nacht waren sie in ihrem Amt beschäftigt. 34Das sind die levitischen
Familienhäupter nach ihren Geschlechtern, die Häupter; diese wohnten in
Jerusalem.

\hypertarget{anhang-die-bewohner-gibeons-und-ein-zweites-geschlechtsregister-des-hauses-sauls}{%
\paragraph{Anhang: Die Bewohner Gibeons und ein zweites
Geschlechtsregister des Hauses
Sauls}\label{anhang-die-bewohner-gibeons-und-ein-zweites-geschlechtsregister-des-hauses-sauls}}

35In Gibeon aber wohnten: der Stammvater von Gibeon, Jehuel, dessen Frau
Maacha hieß, 36und sein erstgeborener Sohn war Abdon; außerdem Zur, Kis,
Baal, Ner, Nadab, 37Gedor, Ahjo, Sacharja und Mikloth; 38Mikloth aber
zeugte Simeam. Auch diese wohnten ihren Stammesgenossen gegenüber in
Jerusalem bei ihren Stammesgenossen.~-- 39Ner aber zeugte Abner, und Kis
zeugte Saul, Saul zeugte Jonathan, Malchisua, Abinadab und Esbaal. 40Der
Sohn Jonathans war Merib-Baal, und dieser zeugte Micha. 41Die Söhne
Michas waren: Pithon, Melech, Thahrea und Ahas. 42Ahas zeugte Jara, Jara
zeugte Alemeth, Asmaweth und Simri; Simri zeugte Moza, 43und Moza zeugte
Binea; dessen Sohn war Rephaja, dessen Sohn Eleasa, dessen Sohn Azel.
44Azel aber hatte sechs Söhne, die hießen: Asrikam, Bochru, Ismael,
Searja, Obadja und Hanan. Dies waren die Söhne Azels.

\hypertarget{ii.-das-kuxf6nigtum-davids-kap.-10-29}{%
\subsection{II. Das Königtum Davids (Kap.
10-29)}\label{ii.-das-kuxf6nigtum-davids-kap.-10-29}}

\hypertarget{sauls-untergang}{%
\subsubsection{1. Sauls Untergang}\label{sauls-untergang}}

\hypertarget{a-israel-von-den-philistern-auf-dem-gebirge-gilboa-besiegt-tod-sauls-und-seiner-drei-suxf6hne}{%
\paragraph{a) Israel von den Philistern auf dem Gebirge Gilboa besiegt;
Tod Sauls und seiner drei
Söhne}\label{a-israel-von-den-philistern-auf-dem-gebirge-gilboa-besiegt-tod-sauls-und-seiner-drei-suxf6hne}}

\hypertarget{section-9}{%
\section{10}\label{section-9}}

1Als es aber zwischen den Philistern und Israeliten zur Schlacht kam,
wurde die Mannschaft der Israeliten von den Philistern in die Flucht
geschlagen, und (viele) Erschlagene lagen auf dem Gebirge Gilboa umher.
2Die Philister setzten dem Saul und seinen Söhnen hart nach und
erschlugen Sauls Söhne Jonathan, Abinadab und Malchisua. 3Als dann ein
wilder Kampf gegen Saul entstand und die Bogenschützen ihn entdeckt
hatten, wurde er von Angst vor den Schützen ergriffen. 4Da befahl Saul
seinem Waffenträger: »Ziehe dein Schwert und durchbohre mich, damit
nicht diese Heiden kommen und ihren Mutwillen an mir auslassen!« Aber
sein Waffenträger weigerte sich, weil er sich zu sehr fürchtete. Da nahm
Saul sein Schwert und stürzte sich hinein. 5Als nun sein Waffenträger
sah, daß Saul tot war, stürzte er sich gleichfalls ins Schwert und
starb. 6So fanden Saul und seine drei Söhne den Tod: sein ganzes Haus
kam gleichzeitig ums Leben. 7Als aber die gesamte israelitische
Bevölkerung, die in der Ebene wohnte, gewahrte, daß die Israeliten
geflohen waren und daß Saul mit seinen Söhnen gefallen war, verließen
sie ihre Ortschaften und ergriffen die Flucht; da kamen die Philister
und setzten sich darin fest.

\hypertarget{b-das-schicksal-der-leichname-sauls-und-seiner-suxf6hne}{%
\paragraph{b) Das Schicksal der Leichname Sauls und seiner
Söhne}\label{b-das-schicksal-der-leichname-sauls-und-seiner-suxf6hne}}

8Als dann am folgenden Tage die Philister kamen, um die Gefallenen
auszuplündern, fanden sie Saul und seine Söhne auf dem Gebirge Gilboa
liegen. 9Da raubten sie ihn aus, nahmen seinen Kopf und seine Rüstung
mit und sandten Boten in allen Teilen des Philisterlandes umher, um die
Siegesbotschaft ihren Götzen und dem Volk zu verkünden. 10Seine Waffen✲
legten sie im Tempel ihres Gottes nieder und nagelten seinen Schädel im
Tempel Dagons an. 11Als aber die gesamte Einwohnerschaft von Jabes in
Gilead alles erfuhr, was die Philister an Sauls Leichnam verübt hatten,
12machten sich alle streitbaren Männer auf, nahmen den Leichnam Sauls
und die Leichen seiner Söhne mit sich und brachten sie nach Jabes; dann
begruben sie ihre Gebeine unter der Terebinthe in Jabes und fasteten
sieben Tage lang.

\hypertarget{c-ruxfcckblick-auf-sauls-verschuldung-gegen-gott}{%
\paragraph{c) Rückblick auf Sauls Verschuldung gegen
Gott}\label{c-ruxfcckblick-auf-sauls-verschuldung-gegen-gott}}

13So fand Saul den Tod infolge seiner Treulosigkeit, deren er sich gegen
den HERRN schuldig gemacht hatte, weil er das Gebot des HERRN nicht
beobachtet und auch, weil er eine Totenbeschwörerin aufgesucht hatte, um
sie zu befragen, 14statt sich an den HERRN um eine Offenbarung zu
wenden. Darum ließ dieser ihn ums Leben kommen und wandte das Königtum
David, dem Sohne Isais, zu.

\hypertarget{davids-salbung-zu-hebron-und-die-eroberung-jerusalems}{%
\subsubsection{2. Davids Salbung zu Hebron und die Eroberung
Jerusalems}\label{davids-salbung-zu-hebron-und-die-eroberung-jerusalems}}

\hypertarget{section-10}{%
\section{11}\label{section-10}}

1Nun fanden sich alle Israeliten bei David in Hebron ein und sagten:
»Wir sind ja doch von deinem Fleisch und Bein! 2Schon früher, schon als
Saul noch König war, bist du es gewesen, der Israel ins Feld und wieder
heimgeführt hat; dazu hat der HERR, dein Gott, dir verheißen: ›Du sollst
mein Volk Israel weiden, und du sollst Fürst über mein Volk Israel
sein!‹« 3Als so alle Ältesten der Israeliten zum König nach Hebron
gekommen waren, schloß David einen Vertrag mit ihnen in Hebron vor dem
Angesicht des HERRN; dann salbten sie David zum König über Israel,
entsprechend dem Wort des HERRN, das durch Samuel ergangen
war\textless sup title=``1.Sam 16,1-13''\textgreater✲.

4Als David dann mit ganz Israel gegen Jerusalem zog -- das ist Jebus,
und dort waren die Jebusiter, die das Land (noch) bewohnten --, 5da
sagten die Bewohner von Jebus zu David: »Hier wirst du nicht
eindringen!« Aber David eroberte die Burg Zion, das ist die (jetzige)
Davidsstadt. 6Da sagte David: »Wer die Jebusiter zuerst schlägt, soll
oberster Heerführer werden!« Da stieg Joab, der Sohn der Zeruja, zuerst
hinauf und wurde dadurch Heerführer. 7David nahm dann seinen Wohnsitz
auf der Burg; darum nannte man sie ›Davidsstadt‹; 8auch befestigte er
die Stadt ringsum, von der Burg Millo\textless sup title=``2.Sam
5,9''\textgreater✲ an rund umher, während Joab die übrige Stadt
wiederherstellte. 9Davids Macht wuchs nun immer mehr, weil der HERR der
Heerscharen mit ihm war.

\hypertarget{verzeichnis-und-heldentaten-von-davids-kriegern}{%
\subsubsection{3. Verzeichnis und Heldentaten von Davids
Kriegern}\label{verzeichnis-und-heldentaten-von-davids-kriegern}}

10Folgendes sind die vornehmsten Helden Davids, die sich im Verein mit
ganz Israel bei seiner Erhebung zur Königswürde fest zu ihm hielten, um
ihn zum König zu machen, wie der HERR den Israeliten geboten hatte.
11Folgendes ist also das Verzeichnis der Helden Davids: Isbaal, der Sohn
Hachmonis, das Haupt der Ritter\textless sup title=``oder: der Drei;
vgl. 2.Sam 23,8''\textgreater✲, der seinen Speer über dreihundert
(Feinden) schwang, die er auf einmal erschlagen hatte. 12Nach ihm kam
Eleasar, der Sohn Dodos, der Ahohiter, der zu den drei (vornehmsten)
Helden gehörte. 13Er befand sich (einst) mit David in Pas-Dammim, als
die Philister sich dort zur Schlacht versammelt hatten. Nun war da ein
Ackerstück mit Gerste; und als das übrige Heer vor den Philistern floh,
14da trat er mitten auf das Feld, behauptete es und schlug die
Philister; so verlieh der HERR ihnen einen herrlichen Sieg.

\hypertarget{wagnis-dreier-helden}{%
\paragraph{Wagnis dreier Helden}\label{wagnis-dreier-helden}}

15Einst kamen die drei vornehmsten von den dreißig Rittern zu David nach
dem Felsennest hinab, in die Höhle Adullam, während das Heer der
Philister sich im Tale\textless sup title=``oder: in der
Ebene''\textgreater✲ Rephaim gelagert hatte. 16David befand sich aber
damals in der Bergfeste, während eine Besatzung der Philister damals in
Bethlehem lag. 17Da verspürte David ein Gelüst und rief aus: »Wer
verschafft mir Wasser zu trinken aus dem Brunnen, der in Bethlehem am
Stadttor liegt?« 18Da schlugen sich die drei (Helden) durch das Lager
der Philister hindurch, schöpften Wasser aus dem Brunnen am Tor von
Bethlehem und brachten es glücklich zu David hin. Aber dieser wollte es
nicht trinken, sondern goß es als Spende für den HERRN aus 19mit den
Worten: »Der HERR behüte mich davor, daß ich so etwas tun sollte! Ich
sollte das Blut dieser Männer trinken, die unter Lebensgefahr hingezogen
sind? Denn mit Daransetzung ihres Lebens haben sie es geholt!« Und er
wollte es nicht trinken. Das hatten die drei Helden ausgeführt.

\hypertarget{abisai-und-benaja}{%
\paragraph{Abisai und Benaja}\label{abisai-und-benaja}}

20Abisai aber, der Bruder Joabs, der war das Haupt der Dreißig; der
schwang seinen Speer über dreihundert Feinden, die er erschlagen hatte,
und besaß hohes Ansehen unter den Dreißig. 21Unter den Dreißig genoß er
die höchste Ehre, so daß er auch ihr Oberster wurde; aber an die
(ersten) drei Helden reichte er nicht heran.~-- 22Benaja, der Sohn
Jojadas, ein tapferer Mann, groß an Taten, stammte aus Kabzeel; er war
es, der die beiden Söhne Ariels aus Moab erschlug. Auch stieg er einmal
in eine Zisterne hinab und erschlug drunten in der Grube einen Löwen an
einem Tage, an dem Schnee gefallen war. 23Auch erschlug er den Ägypter,
einen Riesen von fünf Ellen Länge, der einen Speer in der Hand hatte so
dick wie ein Weberbaum; er aber ging mit einem Stecken auf ihn los, riß
dem Ägypter den Speer aus der Hand und tötete ihn mit seinem eigenen
Speer. 24Solche Taten vollführte Benaja, der Sohn Jojadas; er besaß
hohes Ansehen unter den dreißig Rittern, 25ja, er war der geehrteste
unter den Dreißig; aber an die (ersten) drei Helden reichte er nicht
heran. David stellte ihn an die Spitze seiner Leibwache.

\hypertarget{eine-liste-anderer-helden-davids}{%
\paragraph{Eine Liste anderer Helden
Davids}\label{eine-liste-anderer-helden-davids}}

26Die (übrigen) ausgezeichnetsten Kriegshelden waren: Asahel, der Bruder
Joabs; Elhanan aus Bethlehem, der Sohn Dodos; 27Sammoth aus Harod; Helez
aus Pelet; 28Ira aus Thekoa, der Sohn des Ikkes; Abieser aus Anathoth;
29Sibbechai aus Husa; Ilai aus Ahoh; 30Maharai aus Netopha; Heled aus
Netopha, der Sohn Baanas; 31Itthai, der Sohn Ribais, aus Gibea im Stamme
Benjamin; Benaja aus Pirathon; 32Hiddai aus Nahale-Gaas; Abiel aus
Araba; 33Asmaweth aus Bahurim; Eljahba aus Saalbon; 34Jasen aus Guni;
Jonathan, der Sohn Sages, aus Harar; 35Ahiam, der Sohn Sachars, aus
Harar; Elipheleth, der Sohn Urs; 36Hepher aus Mechera; Ahia aus Palon;
37Hezro aus Karmel; Naarai, der Sohn Esbais; 38Joel, der Bruder Nathans;
Mibhar, der Sohn Hagris; 39Zelek, der Ammoniter; Nahrai aus Beeroth, der
Waffenträger Joabs, des Sohnes der Zeruja; 40Ira aus Jattir; Gareb aus
Jattir; 41Uria, der Hethiter; Sabad, der Sohn Ahlais; 42Adina, der Sohn
Sisas, aus dem Stamme Ruben, ein Haupt der Rubeniten, und mit ihm
dreißig Mann; 43Hanan, der Sohn Maachas, und Josaphat aus Methen;
44Ussia aus Asthera; Sama und Jehiel, die Söhne Hothams, aus Aroer;
45Jediael, der Sohn Simris, und sein Bruder Joha, der Thiziter; 46Eliel
aus Mahanaim; Jeribai und Josawja, die Söhne Elnaams, und Jithma, der
Moabiter; 47Eliel und Obed und Jaasiel aus Zoba.

\hypertarget{davids-erste-anhuxe4nger-in-seiner-fluxfcchtlingszeit-und-bei-seiner-kuxf6nigswahl}{%
\subsubsection{4. Davids erste Anhänger in seiner Flüchtlingszeit und
bei seiner
Königswahl}\label{davids-erste-anhuxe4nger-in-seiner-fluxfcchtlingszeit-und-bei-seiner-kuxf6nigswahl}}

\hypertarget{a-davids-anhuxe4nger-in-ziklag-und-adullam-noch-bei-lebzeiten-sauls}{%
\paragraph{a) Davids Anhänger in Ziklag und Adullam noch bei Lebzeiten
Sauls}\label{a-davids-anhuxe4nger-in-ziklag-und-adullam-noch-bei-lebzeiten-sauls}}

\hypertarget{section-11}{%
\section{12}\label{section-11}}

1Und dies sind die Männer, die zu David nach Ziklag kamen, als er noch
von Saul, dem Sohne des Kis, verbannt war; auch sie gehörten zu den
Helden, die ihm im Kampf halfen, 2ausgerüstet mit Bogen und geübt, mit
der Rechten und mit der Linken Steine zu schleudern und Pfeile mit dem
Bogen zu schießen.

Von den Stammesgenossen Sauls aus (dem Stamme) Benjamin waren da: 3der
Oberste Ahieser und Joas, der Sohn Semaas aus Gibea; ferner Jesiel und
Pelet, die Söhne Asmaweths, Beracha und Jehu aus Anathoth, 4Jismaja aus
Gibeon, ein Held unter den Dreißig und der Anführer der Dreißig; 5ferner
Jeremia, Jahasiel und Johanan; Josabad aus Gedera; 6Elusai, Jerimoth,
Bealja und Semarja; Sephatja aus Hariph; 7Elkana, Jissija, Asarel,
Joeser und Jasobam, die Korhiter; 8Joela und Sebadja, die Söhne
Jerohams, aus Gedor.

9Von den Gaditen aber gingen tapfere Helden zu David nach der Bergfeste
in der Wüste über, kampfgeübte Krieger, die Schild und Lanze zu führen
wußten, die anzusehen waren wie Löwen und schnellfüßig wie Gazellen auf
den Bergen: 10der Oberste war Eser, der zweite Obadja, der dritte Eliab,
11der vierte Mismanna, der fünfte Jeremia, 12der sechste Atthai, der
siebte Eliel, 13der achte Johanan, der neunte Elsabad, 14der zehnte
Jeremia, der elfte Machbannai. 15Diese, von den Gaditen, waren
Heerführer, deren unbedeutendster es mit hundert, deren tüchtigster es
mit tausend Mann aufnehmen konnte. 16Diese waren es, die (einst) im
ersten Monat über den Jordan setzten, als er alle seine Ufer überflutet
hatte, und alle Bewohner der Niederungen, im Osten wie im Westen, in die
Flucht jagten.

17Als aber einmal auch von den Benjaminiten und Judäern Leute zu David
nach der Bergfeste kamen, 18trat David draußen vor sie hin und redete
sie mit folgenden Worten an: »Kommt ihr als Freunde zu mir, um mir zu
helfen, so will ich mich zu herzlicher Gemeinschaft mit euch vereinigen;
wollt ihr mich aber an meine Feinde verraten, wiewohl kein Unrecht an
meinen Händen klebt, so möge der Gott unserer Väter dareinsehen und es
strafen!« 19Da rief Abisai, der (spätere) Hauptmann der Dreißig, von
Begeisterung ergriffen, aus: »Dein sind wir, David, und mit dir halten
wir es, Sohn Isais! Heil, Heil dir und Heil denen, die es mit dir
halten! Denn dir hilft dein Gott!« Darauf nahm David sie an und machte
sie zu Anführern von Kriegerscharen.

20Auch von den Manassiten traten einige zu David über, als er mit den
Philistern gegen Saul zu Felde zog -- ohne ihnen jedoch Hilfe zu
leisten; denn die Fürsten der Philister hatten ihn aus ihrem Heere
entlassen, nachdem sie eine Beratung gehalten hatten, weil sie sich
sagten: »Um den Preis unserer Köpfe könnte er zu Saul, seinem Herrn,
übergehen« --; 21während er dann also (wieder) nach Ziklag zog, fielen
ihm folgende Manassiten zu: Adnah, Josabad, Jediael, Michael, Josabad,
Elihu und Zillethai, Häupter der Tausendschaften von Manasse. 22Diese
leisteten dem David Beistand gegen die Streifschar\textless sup
title=``oder: Räuberschar''\textgreater✲; denn sie waren sämtlich
tapfere Krieger und wurden Anführer im Heere. 23Denn Tag für Tag kamen
Leute zu David, um Kriegsdienste bei ihm zu tun, bis es ein großes Heer
geworden war wie ein Heer Gottes.

\hypertarget{b-zahl-der-krieger-bei-davids-kuxf6nigswahl-in-hebron}{%
\paragraph{b) Zahl der Krieger bei Davids Königswahl in
Hebron}\label{b-zahl-der-krieger-bei-davids-kuxf6nigswahl-in-hebron}}

24Folgendes sind nun die Zahlen der zum Heeresdienst Gerüsteten, die
sich bei David in Hebron einfanden, um ihm nach dem Befehl des HERRN das
Königtum Sauls zu übertragen: 25Von den Judäern, die Schild und Lanze
führten, waren es 6800 zum Kriegsdienst Gerüstete;~-- 26von den
Simeoniten 7100 tapfere Krieger;~-- 27von den Leviten 4600; 28dazu
Jojada, der Fürst über die zum Hause Aarons gehörigen Krieger, und mit
ihm 3700~Mann; 29sodann Zadok, ein tapferer junger Held, dessen Familie
22 Hauptleute stellte;~-- 30von den Benjaminiten, Sauls Stammesgenossen,
3000~Mann; denn bis dahin hielt der größte Teil von ihnen noch treu zum
Hause Sauls;~-- 31von den Ephraimiten 20800 tapfere, in ihren Familien
hochangesehene Männer;~-- 32vom halben Stamm Manasse 18000, die mit
Namen aufgeführt worden waren, daß sie hinziehen sollten, um David zum
König zu machen;~-- 33von den Issaschariten, die sich auf die
Zeitverhältnisse verstanden, um zu wissen, was Israel tun müsse: ihre
200 Häupter\textless sup title=``oder: Hauptleute''\textgreater✲ und
unter deren Befehl ihre sämtlichen Stammesgenossen;~-- 34von Sebulon
50000~Mann, die zum Heeresdienst auszogen, in voller Kriegsrüstung und
einmütig (zum Kampfe) sich ordnend;~-- 35von Naphthali 1000 Anführer und
mit ihnen 37000~Mann mit Schild und Speer;~-- 36von den Daniten 28600
kampfgerüstete Leute;~-- 37von Asser 40000~Mann, die zum Heeresdienst
auszogen, zum Kampfe bereit;~-- 38von jenseits des Jordans: von den
Rubeniten, Gaditen und dem halben Stamm Manasse: 120000~Mann in voller
feldmäßiger Kriegsrüstung.

39Alle diese Kriegsleute, zum Kampf in Schlachtreihen geordnet, kamen
einmütigen Sinnes nach Hebron, um David zum König über ganz Israel zu
machen; aber auch das ganze übrige Israel war einmütig in dem Entschluß,
David zum König zu machen. 40Drei Tage lang blieben sie dort bei David,
aßen und tranken; denn ihre Volksgenossen hatten für ihren Unterhalt
gesorgt; 41außerdem brachten die in ihrer Nähe bis nach Issaschar,
Sebulon und Naphthali hin Wohnenden Lebensmittel auf Eseln, Kamelen,
Maultieren und Rindern herbei: Mundvorrat von Mehl, Feigenkuchen und
Rosinenkuchen, Wein und Öl, auch Rinder und Kleinvieh in Menge; denn es
herrschte eine freudige Stimmung in Israel.

\hypertarget{miuxdflingen-der-einholung-der-bundeslade-aus-kirjath-jearim-nach-jerusalem}{%
\subsubsection{5. Mißlingen der Einholung der Bundeslade aus
Kirjath-Jearim nach
Jerusalem}\label{miuxdflingen-der-einholung-der-bundeslade-aus-kirjath-jearim-nach-jerusalem}}

\hypertarget{a-aufbietung-des-ganzen-volkes-zum-zweck-der-einholung}{%
\paragraph{a) Aufbietung des ganzen Volkes zum Zweck der
Einholung}\label{a-aufbietung-des-ganzen-volkes-zum-zweck-der-einholung}}

\hypertarget{section-12}{%
\section{13}\label{section-12}}

1Als David sich dann mit den Anführern der Tausendschaften und der
Hundertschaften, mit allen Fürsten beraten hatte, 2sagte er zu der
ganzen Volksgemeinde Israels: »Wenn es euch gut dünkt und es vom HERRN,
unserm Gott, gebilligt wird, so wollen wir schleunigst zu unseren
übrigen Volksgenossen, die in allen Gegenden Israels zurückgeblieben
sind, sowie zu den Priestern und den Leviten, die bei ihnen in den
Ortschaften ihrer Bezirke wohnen, Boten senden, damit sie insgesamt zu
uns herkommen; 3wir wollen dann die Lade unseres Gottes zu uns
herüberholen; denn während der Regierung Sauls haben wir uns nicht um
sie gekümmert.« 4Da erklärte sich die ganze Versammlung mit dem
Vorschlage einverstanden; denn er hatte den Beifall des ganzen Volkes
gefunden. 5So ließ denn David alle Israeliten vom ägyptischen Flusse
Sihor an bis in die Gegend von Hamath hin zusammenkommen, damit sie die
Lade Gottes aus Kirjath-Jearim herbeiholten.

\hypertarget{b-miuxdflingen-des-planes}{%
\paragraph{b) Mißlingen des Planes}\label{b-miuxdflingen-des-planes}}

6Darauf zog David mit ganz Israel hinauf nach Baala, das ist nach
Kirjath-Jearim, welches zu Juda gehört, um von dort die Lade Gottes
heraufzuholen, die nach dem Namen des HERRN benannt ist, der über den
Cheruben thront. 7Sie führten dann die Lade Gottes auf einem neuen Wagen
aus dem Hause Abinadabs weg, indem Ussa und Ahjo den Wagen leiteten;
8David aber und alle Israeliten tanzten vor Gott her mit Aufbietung
aller Kräfte: mit Gesängen und unter Begleitung von Zithern und Harfen,
Handpauken, Zimbeln und Trompeten. 9Als sie nun so bis zur Tenne Kidons
gekommen waren, streckte Ussa seine Hand aus, um die Lade festzuhalten,
weil die Rinder ausgeglitten\textless sup title=``oder: zu Fall
gekommen''\textgreater✲ waren. 10Da entbrannte der Zorn des HERRN gegen
Ussa, und er schlug ihn zur Strafe dafür, daß er mit der Hand nach der
Lade gegriffen hatte, so daß er dort vor den Augen Gottes starb. 11Da
wurde David tief betrübt darüber, daß der HERR einen solchen Schlag
gegen Ussa geführt hatte; daher nannte man jenen Ort
Perez-Ussa\textless sup title=``d.h. Ussas Schlag oder:
Riß''\textgreater✲ bis auf den heutigen Tag.

\hypertarget{c-die-lade-im-hause-obed-edoms-untergebracht}{%
\paragraph{c) Die Lade im Hause Obed-Edoms
untergebracht}\label{c-die-lade-im-hause-obed-edoms-untergebracht}}

12David aber geriet an jenem Tage in Furcht vor Gott, so daß er ausrief:
»Wie kann ich da die Lade Gottes zu mir bringen?« 13Weil David also die
Lade des HERRN nicht zu sich in die Davidsstadt bringen lassen wollte,
ließ er sie abseits in das Haus des Gathiters Obed-Edom setzen. 14So
blieb denn die Lade Gottes ein Vierteljahr lang bei der Familie
Obed-Edoms, in dessen Hause, stehen; der HERR aber segnete das Haus
Obed-Edoms und seinen gesamten Besitz\textless sup title=``vgl. Kap.
15-16''\textgreater✲.

\hypertarget{davids-palastbau-und-neue-heiraten-seine-siegreichen-kriege-mit-den-philistern}{%
\subsubsection{6. Davids Palastbau und neue Heiraten; seine siegreichen
Kriege mit den
Philistern}\label{davids-palastbau-und-neue-heiraten-seine-siegreichen-kriege-mit-den-philistern}}

\hypertarget{section-13}{%
\section{14}\label{section-13}}

1Nun schickte Hiram, der König von Tyrus, Gesandte an David mit
Zedernstämmen, dazu Steinmetzen und Zimmerleute, damit sie ihm einen
Palast bauten; 2daran erkannte David, daß der HERR ihn als König über
Israel bestätigt habe, weil sein Königtum zu hohem Ansehen erhoben
worden war um seines Volkes Israel willen.

\hypertarget{davids-in-jerusalem-geborene-suxf6hne}{%
\paragraph{Davids in Jerusalem geborene
Söhne}\label{davids-in-jerusalem-geborene-suxf6hne}}

3In Jerusalem nahm sich David dann noch mehr Frauen und wurde Vater von
noch mehr Söhnen und Töchtern. 4Dies aber sind die Namen der Söhne, die
ihm in Jerusalem geboren wurden: Sammua und Sobab, Nathan und Salomo,
5Jibhar, Elisua, Elpelet, 6Nogah, Nepheg, Japhia, 7Elisama, Beeljada und
Eliphelet.

\hypertarget{zwei-siegreiche-kuxe4mpfe-davids-mit-den-philistern}{%
\paragraph{Zwei siegreiche Kämpfe Davids mit den
Philistern}\label{zwei-siegreiche-kuxe4mpfe-davids-mit-den-philistern}}

8Als aber die Philister vernahmen, daß David zum König über ganz Israel
gesalbt worden war, zogen die Philister insgesamt heran, um seiner
habhaft zu werden; aber David erhielt Kunde davon und zog ihnen
entgegen. 9Als nun die Philister eingedrungen waren und sich in der
Ebene Rephaim ausbreiteten, 10richtete David die Anfrage an Gott: »Soll
ich gegen die Philister hinaufziehen, und wirst du sie in meine Hand
geben?« Da antwortete ihm der HERR: »Ziehe hin, ich will sie in deine
Hand geben!« 11Als sie nun nach Baal-Perazim hinaufzogen und David sie
dort geschlagen hatte, rief David aus: »Gott hat meine Feinde durch
meine Hand durchbrochen, wie das Wasser einen Damm durchbricht!« Darum
hat man jenem Ort den Namen Baal-Perazim\textless sup title=``d.h. Ort
der Durchbrüche''\textgreater✲ gegeben. 12Da (die Philister) ihre
Götterbilder dort zurückgelassen hatten, gab David den Befehl, daß man
sie verbrennen solle.

13Die Philister zogen dann nochmals heran und breiteten sich in der
Ebene (Rephaim) aus. 14Als David nun Gott wiederum befragte, antwortete
dieser ihm: »Du sollst nicht hinter ihnen her hinaufziehen, sondern
umgehe sie, damit du sie vom Baka-Gehölz her überfällst! 15Sobald du
dann in den Wipfeln des Baka-Gehölzes das Geräusch von Schritten
vernimmst, dann gehe zum Angriff über! Denn alsdann ist Gott vor dir her
ausgezogen, um das Heer der Philister zu schlagen.« 16Da tat David, wie
Gott ihm geboten hatte, und so schlugen sie das Heer der Philister von
Gibeon bis nach Geser hin. 17Hierauf verbreitete sich der Ruhm Davids in
alle Lande, und der HERR flößte allen Völkern Furcht vor ihm ein.

\hypertarget{uxfcberfuxfchrung-der-bundeslade-nach-dem-zion-in-jerusalem}{%
\subsubsection{7. Überführung der Bundeslade nach dem Zion in
Jerusalem}\label{uxfcberfuxfchrung-der-bundeslade-nach-dem-zion-in-jerusalem}}

\hypertarget{a-vorbereitungen-zur-uxfcberfuxfchrung-der-heiligen-lade-bezeichnung-und-anweisung-der-tragenden-leviten}{%
\paragraph{a) Vorbereitungen zur Überführung der heiligen Lade;
Bezeichnung und Anweisung der tragenden
Leviten}\label{a-vorbereitungen-zur-uxfcberfuxfchrung-der-heiligen-lade-bezeichnung-und-anweisung-der-tragenden-leviten}}

\hypertarget{section-14}{%
\section{15}\label{section-14}}

1Hierauf baute er sich Häuser in der Davidsstadt und richtete für die
Lade Gottes eine Stätte her, indem er ein Zelt für sie aufschlug.
2Damals ordnete David an: »Niemand darf die Lade Gottes tragen außer den
Leviten! Denn diese hat der HERR dazu erwählt, die Lade Gottes zu tragen
und ihm allezeit Dienste zu verrichten.« 3Dann entbot David das gesamte
Israel nach Jerusalem, um die Lade des HERRN an die Stätte
hinaufzuschaffen, die er für sie hatte herrichten lassen; 4und zwar ließ
David die Nachkommen Aarons sowie die Leviten zusammenkommen, 5nämlich
von den Nachkommen Kehaths: Uriel, den Obersten\textless sup
title=``oder: Fürsten''\textgreater✲ des Geschlechts, und seine
Geschlechtsgenossen, hundertundzwanzig an der Zahl; 6von den Nachkommen
Meraris: Asaja, den Obersten des Geschlechts, und seine
Geschlechtsgenossen, zweihundertundzwanzig; 7von den Nachkommen Gersoms:
Joel, den Obersten des Geschlechts, und seine Geschlechtsgenossen,
hundertunddreißig; 8von den Nachkommen Elizaphans: Semaja, den Obersten
des Geschlechts, und seine Geschlechtsgenossen, zweihundert; 9von den
Nachkommen Hebrons: Eliel, den Obersten des Geschlechts, und seine
Geschlechtsgenossen, achtzig; 10von den Nachkommen Ussiels: Amminadab,
den Obersten des Geschlechts, und seine Geschlechtsgenossen,
hundertundzwölf.~-- 11Darauf berief David die Priester Zadok und
Abjathar und die Leviten Uriel, Asaja und Joel, Semaja und Eliel und
Amminadab 12und richtete folgende Worte an sie: »Ihr seid die
Familienhäupter der Leviten: so heiligt euch nun, ihr samt euren
Geschlechtsgenossen! Denn ihr sollt die Lade des HERRN, des Gottes
Israels, an die Stätte hinaufbringen, die ich für sie habe herrichten
lassen. 13Weil ihr nämlich das vorige Mal nicht zugegen gewesen seid,
hat der HERR, unser Gott, Unheil unter uns angerichtet, zur Strafe
dafür, daß wir nicht in gebührender Weise Rücksicht auf ihn genommen
hatten.« 14Da heiligten sich die Priester und die Leviten, um die Lade
des HERRN, des Gottes Israels, hinaufzubringen; 15sodann hoben die
Leviten die Lade Gottes so, wie Mose es nach dem Befehl des HERRN
geboten hatte\textless sup title=``vgl. 2.Mose 25,14; 4.Mose
4,15''\textgreater✲, auf ihre Schultern, indem sie die Tragstangen auf
sich legten.

\hypertarget{b-bestellung-der-levitischen-suxe4nger-musiker-und-torhuxfcter}{%
\paragraph{b) Bestellung der levitischen Sänger, Musiker und
Torhüter}\label{b-bestellung-der-levitischen-suxe4nger-musiker-und-torhuxfcter}}

16Hierauf befahl David den Obersten\textless sup title=``oder:
Vorstehern''\textgreater✲ der Leviten, ihre Geschlechtsgenossen, die
Sänger, mit ihren Musikinstrumenten, den Harfen, Zithern und Zimbeln,
antreten zu lassen, damit sie Musik und Gesang mit lautem Jubelschall
ertönen ließen. 17Da stellten denn die Leviten Heman, den Sohn Joels,
auf und von dessen Geschlechtsgenossen Asaph, den Sohn Berechjas, und
von den Nachkommen Meraris, ihren Geschlechtsgenossen, Ethan, den Sohn
Kusajas, 18und mit ihnen ihre Amtsgenossen zweiten Ranges: Sacharja,
{[}den Sohn,{]} sowie Jaasiel, Semiramoth, Jehiel, Unni, Eliab, Benaja,
Maaseja, Matthithja, Eliphelehu, Mikneja, Obed-Edom und Jeiel, die
Torhüter; 19dazu die Sänger Heman, Asaph und Ethan mit kupfernen
Zimbeln, um die Musik zu verstärken, 20Sacharja dagegen nebst Ussiel,
Semiramoth, Jehiel, Unni, Eliab, Maaseja und Benaja mit Leiern, zu denen
sie in hoher Tonlage\textless sup title=``=~im Tenor''\textgreater✲
sangen; 21ferner Matthithja, Eliphelehu, Mikneja, Obed-Edom, Jehiel und
Asasja mit Harfen, zu denen sie mit Baßstimmen sangen und den Gesang
leiteten. 22Kenanja ferner, der Oberste der Leviten beim Tragen (der
heiligen Geräte), führte die Aufsicht beim Tragen, denn er verstand sich
darauf; 23und Berechja und Elkana waren Torhüter bei der Lade; 24Sebanja
aber sowie Josaphat, Nethaneel, Amasai, Sacharja, Benaja und Elieser,
die Priester, bliesen die Trompeten vor der Lade Gottes her, während
Obed-Edom und Jehia Torhüter bei der Lade waren.

\hypertarget{c-davids-persuxf6nliche-anteilnahme-an-der-uxfcberfuxfchrung-das-opfer--und-dankfest}{%
\paragraph{c) Davids persönliche Anteilnahme an der Überführung; das
Opfer- und
Dankfest}\label{c-davids-persuxf6nliche-anteilnahme-an-der-uxfcberfuxfchrung-das-opfer--und-dankfest}}

25So zogen denn David und die Ältesten\textless sup title=``oder:
Vornehmsten''\textgreater✲ der Israeliten und die Befehlshaber der
Tausendschaften hin, um die Lade mit dem Bundesgesetz des HERRN aus dem
Hause Obed-Edoms voller Freude hinaufzubringen; 26und da Gott sich den
Leviten, welche die Lade mit dem Bundesgesetz des HERRN trugen, gnädig
bewies, opferte man sieben Stiere und sieben Widder. 27Dabei war David
mit einem Überwurf von Byssus bekleidet, ebenso alle Leviten, welche die
Lade trugen, sowie die Sänger und Kenanja, der das Tragen\textless sup
title=``=~den Umzug''\textgreater✲ zu leiten hatte; David aber hatte ein
linnenes Schulterkleid (darüber) angelegt. 28So brachte denn ganz Israel
die Lade mit dem Bundesgesetz des HERRN unter lautem Jubel und
Posaunenschall, mit Trompeten- und Zimbelklang und unter Harfen- und
Zitherspiel hinauf. 29Da begab es sich, als die Lade mit dem
Bundesgesetz des HERRN in die Davidsstadt einzog, daß Sauls Tochter
Michal zum Fenster hinausschaute; als sie nun den König David so
springen und tanzen sah, empfand sie Verachtung für ihn in ihrem Herzen.

\hypertarget{section-15}{%
\section{16}\label{section-15}}

1Nachdem man dann die Lade Gottes hineingebracht und sie im Innern des
Zeltes, das David für sie hatte aufschlagen lassen, niedergesetzt hatte,
brachte man Brandopfer und Heilsopfer vor Gott dar; 2und als David mit
der Darbringung der Brand- und Heilsopfer fertig war, segnete er das
Volk im Namen des HERRN 3und ließ dann unter sämtliche Israeliten,
sowohl an Männer wie an Frauen, an einen jeden einen Brotkuchen, ein
Stück Fleisch und einen Rosinenkuchen austeilen\textless sup
title=``vgl. 2.Sam 6,19''\textgreater✲.

\hypertarget{d-ordnung-des-gesang--und-musikdienstes-bei-der-lade}{%
\paragraph{d) Ordnung des Gesang- und Musikdienstes bei der
Lade}\label{d-ordnung-des-gesang--und-musikdienstes-bei-der-lade}}

4Hierauf bestellte er mehrere Leviten zu Dienern vor der Lade des HERRN,
damit sie dem HERRN, dem Gott Israels, Preis, Dank und Lob darbrächten,
5nämlich Asaph als Vorsteher und Sacharja als zweiten im Range nach ihm,
ferner Jeghiel\textless sup title=``oder: Jaasiel? vgl.
15,18''\textgreater✲, Semiramoth, Jehiel, Matthithja, Eliab, Benaja,
Obed-Edom und Jehiel, mit Musikinstrumenten, Harfen und Zithern, während
Asaph die Zimbeln 6und die Priester Benaja und Jahasiel die Trompeten
ständig vor der Lade mit dem Bundesgesetz Gottes erschallen ließen.

\hypertarget{e-davids-dank--und-loblied}{%
\paragraph{e) Davids Dank- und
Loblied}\label{e-davids-dank--und-loblied}}

7Damals, an jenem Tage, beauftragte David zum erstenmal Asaph und seine
Amtsgenossen, das Danklied auf den HERRN zu singen:

8Preiset den HERRN\textless sup title=``oder: danket dem
HERRN''\textgreater✲, ruft seinen Namen an, macht seine Taten unter den
Völkern bekannt! 9Singet ihm, spielet ihm, redet von all seinen Wundern!
10Rühmt euch seines heiligen Namens! Es mögen sich herzlich freuen, die
da suchen den HERRN! 11Fragt nach dem HERRN und seiner
Stärke\textless sup title=``oder: Macht''\textgreater✲, suchet sein
Angesicht allezeit! 12Gedenkt seiner Wunder, die er getan, seiner
Zeichen und der Urteilssprüche seines Mundes, 13ihr Kinder Israels,
seine Knechte, ihr Söhne Jakobs, seine Erwählten!

14Er, der HERR, ist unser Gott, über die ganze Erde ergehen seine
Gerichte. 15Er gedenkt seines Bundes auf ewig, des Wortes, das er
geboten auf tausend Geschlechter, 16(des Bundes) den er mit Abraham
geschlossen, und des Eides, den er Isaak geschworen, 17den für Jakob er
als Satzung bestätigt und für Israel als ewigen Bund, 18da er sprach:
»Dir will ich Kanaan geben, das Land, das ich euch als Erbbesitztum
zugeteilt!«

19Damals wart ihr noch ein kleines Häuflein, gar wenige und nur Gäste im
Land; 20sie mußten wandern von Volk zu Volk und von einem Reich zur
anderen Völkerschaft; 21doch keinem gestattete er, sie zu bedrücken, ja,
Könige strafte er um ihretwillen: 22»Tastet meine Gesalbten nicht an und
tut meinen Propheten nichts zuleide!«

23Singet dem HERRN, alle Lande✲, verkündet Tag für Tag sein Heil!
24Erzählt von seiner Herrlichkeit unter den Heiden, unter allen Völkern
von seinen Wundertaten! 25Denn groß ist der HERR und hoch zu preisen,
und mehr zu fürchten als alle anderen Götter; 26denn alle Götter der
Heiden sind nichtige Götzen, doch der HERR hat den Himmel geschaffen.
27Hoheit und Pracht gehn vor ihm her, Macht und Wonne füllen seine
Wohnstatt. 28Bringt dar dem HERRN, ihr Geschlechter der Völker, bringt
dar dem HERRN Ehre und Preis! 29Bringt dar dem HERRN die Ehre seines
Namens! Bringt Opfergaben und kommt vor sein Angesicht! Werft vor dem
HERRN euch nieder in heiligem Schmuck! 30Erzittert vor ihm, alle Lande✲!
Auch den Erdkreis hat er gefestigt, daß er nicht wankt. 31Des freue sich
der Himmel, die Erde jauchze, und man sage unter den Heiden: »Der HERR
ist König!«\textless sup title=``Ps 96,10''\textgreater✲ 32Es brause das
Meer und was darin wimmelt! Es jauchze die Flur und was auf ihr wächst!
33Dann werden auch jubeln die Bäume des Waldes vor dem HERRN, wenn er
kommt, zu richten die Erde.

34Preiset den HERRN\textless sup title=``oder: danket dem
HERRN''\textgreater✲, denn er ist freundlich, ja, ewiglich währt seine
Güte. 35Und betet: »Hilf uns, du Gott unsrer Hilfe! Bringe uns wieder
zusammen und rette uns aus den Heiden, damit wir danken deinem heiligen
Namen, uns deines Lobpreises rühmen!«\textless sup title=``vgl. Ps
106,47''\textgreater✲ 36Gepriesen sei der HERR, der Gott Israels, von
Ewigkeit zu Ewigkeit!

Da rief alles Volk »Amen!« und »Preis sei dem HERRN!«

\hypertarget{f-einrichtung-des-torhuxfcter--priester--und-suxe4ngerdienstes-bei-der-lade-schluuxdf-des-festes}{%
\paragraph{f) Einrichtung des Torhüter-, Priester- und Sängerdienstes
bei der Lade; Schluß des
Festes}\label{f-einrichtung-des-torhuxfcter--priester--und-suxe4ngerdienstes-bei-der-lade-schluuxdf-des-festes}}

37(David) ließ dann Asaph und seine Geschlechtsgenossen dort vor der
Bundeslade des HERRN verbleiben, damit sie den Dienst vor der Lade
ständig so verrichteten, wie ein jeder Tag es erforderte; 38ebenso ließ
er Obed-Edom, den Sohn Jeduthuns, und Hosa samt ihren
Geschlechtsgenossen, achtundsechzig an der Zahl, als Torhüter dienen.
39Den Priester Zadok aber nebst seinen Geschlechtsgenossen, den
Priestern, ließ er vor der Wohnung des HERRN auf der Höhe zu Gibeon
weiter verbleiben, 40damit sie dem HERRN regelmäßig, morgens und abends,
Brandopfer auf dem Brandopferaltar darbrächten, und zwar genau so, wie
es im Gesetz geschrieben steht, das der HERR den Israeliten geboten hat.
41Bei ihnen befanden sich auch Heman und Jeduthun nebst den übrigen, die
auserwählt und mit Namen bezeichnet worden waren, um das »Danket dem
HERRN, denn seine Güte währet ewiglich!« zu singen. 42Dazu wurden ihnen
Trompeten und Zimbeln gegeben zur Verwendung bei der Musik und
Instrumente für die Gotteslieder; die Söhne Jeduthuns aber versahen den
Dienst als Torhüter.~--

43Darauf kehrte jedermann im Volk nach Hause zurück; David aber wandte
sich dazu, seine Familie zu begrüßen.

\hypertarget{davids-plan-eines-tempelbaues-von-gott-verworfen-die-grouxdfe-verheiuxdfung-fuxfcr-david-und-sein-haus}{%
\subsubsection{8. Davids Plan eines Tempelbaues von Gott verworfen; die
große Verheißung für David und sein
Haus}\label{davids-plan-eines-tempelbaues-von-gott-verworfen-die-grouxdfe-verheiuxdfung-fuxfcr-david-und-sein-haus}}

\hypertarget{a-nathan-billigt-davids-plan-des-tempelbaues}{%
\paragraph{a) Nathan billigt Davids Plan des
Tempelbaues}\label{a-nathan-billigt-davids-plan-des-tempelbaues}}

\hypertarget{section-16}{%
\section{17}\label{section-16}}

1Als nun David in seinem Hause✲ wohnte, sagte er (eines Tages) zu dem
Propheten Nathan: »Bedenke doch: ich wohne hier in einem Zedernpalast,
während die Lade mit dem Bundesgesetz des HERRN unter\textless sup
title=``oder: hinter''\textgreater✲ Zelttüchern steht.« 2Da antwortete
Nathan dem David: »Führe alles aus, was du im Sinn hast, denn Gott ist
mit dir!«

\hypertarget{b-gott-verwirft-den-plan-nathans-prophetische-rede-der-tempelbau-soll-von-davids-sohn-ausgefuxfchrt-werden}{%
\paragraph{b) Gott verwirft den Plan; Nathans prophetische Rede; der
Tempelbau soll von Davids Sohn ausgeführt
werden}\label{b-gott-verwirft-den-plan-nathans-prophetische-rede-der-tempelbau-soll-von-davids-sohn-ausgefuxfchrt-werden}}

3Aber noch in derselben Nacht erging das Wort Gottes an Nathan
folgendermaßen: 4»Gehe hin und sage meinem Knecht David: ›So hat der
HERR gesprochen: Nicht du sollst mir das Haus zur Wohnung bauen; 5ich
habe ja doch in keinem Hause gewohnt seit der Zeit, da ich die
Israeliten aus Ägypten hergeführt habe, bis auf den heutigen Tag,
sondern ich habe mich auf der Wanderung von einem Zelt zum andern und
von einer Wohnung zur andern befunden. 6Habe ich etwa, solange ich unter
allen Israeliten umherzog, zu einem von den Richtern Israels, die ich zu
Hirten meines Volkes bestellt hatte, jemals auch nur ein Wort derart
gesagt: ›Warum habt ihr mir kein Zedernhaus gebaut?‹ 7Darum sollst du
jetzt meinem Knecht David folgendes sagen: ›So hat der HERR der
Heerscharen gesprochen: Ich habe dich von der Weide hinter der Herde
weggeholt, damit du Fürst über mein Volk Israel würdest; 8und ich bin
bei allem, was du unternommen hast, mit dir gewesen und habe alle deine
Feinde vor dir her ausgerottet und habe dir einen Namen geschaffen, wie
ihn nur die Größten auf Erden haben. 9Und ich will meinem Volke Israel
eine Stätte anweisen und es daselbst einpflanzen, daß es an seiner
Stätte ruhig wohnen kann und sich nicht mehr zu ängstigen braucht und
daß gewalttätige Menschen es nicht mehr aufreiben wie früher, 10seit der
Zeit, wo ich Richter über mein Volk Israel bestellt habe; sondern ich
will alle deine Feinde demütigen und verkündige dir, daß der HERR dir
ein Haus bauen wird. 11Und wenn einst deine Tage voll sind, so daß du zu
deinen Vätern hingehst, dann will ich nach deinem Tode deine
Nachkommenschaft, und zwar einen von deinen Söhnen, zu deinem Nachfolger
erheben und ihm sein Königtum befestigen. 12Der soll mir dann ein Haus
bauen, und ich will seinen Thron feststellen für immer.‹«

\hypertarget{c-gottes-grouxdfe-heilsverkuxfcndigung-an-david-betreffs-der-ewigen-dauer-seines-hauses}{%
\paragraph{c) Gottes große Heilsverkündigung an David betreffs der
ewigen Dauer seines
Hauses}\label{c-gottes-grouxdfe-heilsverkuxfcndigung-an-david-betreffs-der-ewigen-dauer-seines-hauses}}

13»›Ich will ihm Vater sein, und er soll mir Sohn sein, und ich will ihm
meine Gnade nicht entziehen, wie ich sie deinem Vorgänger entzogen habe,
14sondern für immer will ich ihn über mein Haus und mein Königtum
einsetzen, und sein Thron soll feststehen für immer!‹«

\hypertarget{d-davids-dank--und-bittgebet}{%
\paragraph{d) Davids Dank- und
Bittgebet}\label{d-davids-dank--und-bittgebet}}

15Nachdem Nathan diesen Worten und dieser Offenbarung genau entsprechend
zu David geredet hatte, 16ging der König David (in das Gotteszelt)
hinein, setzte sich vor dem HERRN nieder und betete: »Wer bin ich, HERR,
mein Gott, und was ist mein Haus, daß du mich bis hierher gebracht hast!
17Und dies hast du für noch nicht genügend gehalten, o Gott, sondern
jetzt hast du auch in bezug auf das Haus deines Knechtes noch
Verheißungen für ferne Zeiten gegeben und hast mich schauen lassen
Geschlechter der Menschen, HERR, mein Gott. 18Was soll da David noch
weiter zu dir sagen {[}von der Ehre an deinem Knechte{]}? Du selbst
kennst ja deinen Knecht! 19HERR, um deines Knechtes willen und nach
deinem Wohlgefallen hast du all dieses Große getan, um alle diese
Großtaten kundwerden zu lassen. 20HERR, niemand ist dir gleich, und es
gibt keinen Gott außer dir nach allem, was wir mit eigenen Ohren
vernommen haben. 21Und wo ist ein anderes Volk, das deinem Volke Israel
gliche? Es ist das einzige Volk auf Erden, um deswillen Gott hingegangen
ist, es sich zum Eigentumsvolk zu erkaufen, um dir einen Namen zu
schaffen durch große und wunderbare Taten, indem du vor deinem Volke,
das du aus Ägypten erlöst hast, Heidenvölker vertriebst. 22So hast du
denn dein Volk Israel für alle Zeiten zu deinem Volk bestimmt, und du,
HERR, bist ihr Gott geworden. 23Und nun, HERR -- die Verheißung, die du
in betreff deines Knechtes und seines Hauses ausgesprochen hast, möge
für alle Zeiten gültig bleiben, und verfahre du so, wie du zugesagt
hast! 24Dann wird dein Name sich als treu erweisen und für immer geehrt
sein, wenn man sagt: ›Der HERR der Heerscharen, der Gott Israels, ist
der Gott für Israel‹; und das Haus deines Knechtes David wird Bestand
vor dir haben! 25Denn du selbst, mein Gott, hast deinem Knechte die
Offenbarung zuteil werden lassen, daß du ihm ein Haus bauen wollest;
darum hat dein Knecht den Mut gefunden, dieses Gebet an dich zu richten.
26Und nun, HERR, du bist Gott, und nachdem du deinem Knechte diese
herrliche Zusage gemacht hast~-- 27nun denn, so möge es dir auch
gefallen, das Haus deines Knechtes zu segnen, damit es für immer vor dir
bestehe! Denn was du, HERR, gesegnet hast, das ist gesegnet ewiglich!«

\hypertarget{davids-kriegstaten-und-oberste-beamte}{%
\subsubsection{9. Davids Kriegstaten und oberste
Beamte}\label{davids-kriegstaten-und-oberste-beamte}}

\hypertarget{a-davids-siege-uxfcber-die-philister-moabiter-syrer-und-edomiter}{%
\paragraph{a) Davids Siege über die Philister, Moabiter, Syrer und
Edomiter}\label{a-davids-siege-uxfcber-die-philister-moabiter-syrer-und-edomiter}}

\hypertarget{section-17}{%
\section{18}\label{section-17}}

1Nachmals besiegte David die Philister und unterwarf sie und entriß der
Gewalt der Philister Gath und die zugehörigen Ortschaften.~-- 2Er
besiegte auch die Moabiter, so daß sie ihm tributpflichtige Untertanen
wurden.

\hypertarget{davids-siege-uxfcber-die-syrer-die-verwendung-der-beute-gluxfcckwunsch-des-kuxf6nigs-thou}{%
\paragraph{Davids Siege über die Syrer; die Verwendung der Beute;
Glückwunsch des Königs
Thou}\label{davids-siege-uxfcber-die-syrer-die-verwendung-der-beute-gluxfcckwunsch-des-kuxf6nigs-thou}}

3Sodann besiegte David Hadareser, den König von Zoba, das nach Hamath zu
liegt, als er\textless sup title=``d.h. Hadareser''\textgreater✲
ausgezogen war, um seine Herrschaft am Euphratstrom fest zu begründen.
4Dabei nahm David ihm 1000 Wagen, 7000~Reiter und 20000~Mann Fußvolk weg
und ließ die Wagenpferde sämtlich lähmen; nur hundert Wagenpferde
behielt er davon für sich.

5Als dann die Syrer von Damaskus dem König Hadareser von Zoba zu Hilfe
kamen, erschlug David von den Syrern 22000~Mann 6und setzte dann Vögte
über das damascenische Syrien ein, so daß die dortigen Syrer zu
tributpflichtigen Untertanen Davids wurden. So verlieh der HERR dem
David den Sieg auf allen Zügen, die er unternahm. 7Auch erbeutete David
die goldenen Schilde, welche die Hofbeamten Hadaresers getragen hatten,
und ließ sie nach Jerusalem bringen; 8und in Tibhath und Kun, den
Städten Hadaresers, fiel dem König David sehr viel Kupfer in die Hände,
woraus Salomo später das große eherne Becken sowie die Säulen und die
kupfernen Gerätschaften anfertigen ließ.

9Als aber Thou, der König von Hamath, erfuhr, daß David die ganze
Heeresmacht Hadaresers, des Königs von Zoba, geschlagen hatte, 10sandte
er seinen Sohn Hadoram zum König David, um ihn zu begrüßen und ihm Glück
zu wünschen, daß er aus dem Kriege mit Hadareser als Sieger
hervorgegangen sei -- Thou hatte nämlich auf dem Kriegsfuß mit Hadareser
gestanden --, dazu allerlei goldene, silberne und kupferne
Kunstgegenstände, 11die der König David ebenfalls dem HERRN weihte, wie
er es auch mit dem Silber und Gold machte, das er von allen Völkern
erhoben hatte, von den Edomitern, Moabitern, Ammonitern, Philistern und
Amalekitern.

\hypertarget{besiegung-und-unterwerfung-der-edomiter}{%
\paragraph{Besiegung und Unterwerfung der
Edomiter}\label{besiegung-und-unterwerfung-der-edomiter}}

12Abisai aber, der Sohn der Zeruja, besiegte die Edomiter im Salztal und
erschlug von ihnen 18000~Mann, 13worauf er Vögte über Edom einsetzte, so
daß alle Edomiter Untertanen Davids wurden. So verlieh der HERR dem
David den Sieg überall, wohin er zog.

\hypertarget{b-davids-oberste-beamte}{%
\paragraph{b) Davids oberste Beamte}\label{b-davids-oberste-beamte}}

14So herrschte denn David über ganz Israel und ließ Recht und
Gerechtigkeit in seinem ganzen Volke walten.~-- 15Joab, der Sohn der
Zeruja, stand an der Spitze des Heeres; Josaphat, der Sohn Ahiluds, war
Kanzler; 16Zadok, der Sohn Ahitubs, und Ahimelech, der Sohn Abjathars,
waren Priester, Sawsa Staatsschreiber, 17Benaja, der Sohn Jojadas,
Befehlshaber (der Leibwache) der Krethi und Plethi\textless sup
title=``2.Sam 8,18''\textgreater✲; und die Söhne Davids waren die Ersten
zur Seite\textless sup title=``=~im Dienst''\textgreater✲ des Königs.

\hypertarget{besiegung-der-ammoniter-und-syrer-kuxe4mpfe-mit-den-philistern}{%
\subsubsection{10. Besiegung der Ammoniter und Syrer; Kämpfe mit den
Philistern}\label{besiegung-der-ammoniter-und-syrer-kuxe4mpfe-mit-den-philistern}}

\hypertarget{a-das-schmachvolle-vergehen-der-ammoniter-gegen-davids-gesandte}{%
\paragraph{a) Das schmachvolle Vergehen der Ammoniter gegen Davids
Gesandte}\label{a-das-schmachvolle-vergehen-der-ammoniter-gegen-davids-gesandte}}

\hypertarget{section-18}{%
\section{19}\label{section-18}}

1Danach begab es sich, daß der Ammoniterkönig Nahas starb und sein Sohn
(Hanun) ihm in der Regierung nachfolgte. 2Da dachte David: »Ich will
mich freundlich gegen Hanun, den Sohn des Nahas, beweisen, weil sein
Vater sich mir gegenüber freundlich bewiesen hat.« So schickte denn
David Gesandte hin, um ihm sein Beileid wegen (des Todes) seines Vaters
ausdrücken zu lassen. Als aber die Gesandten Davids im Lande der
Ammoniter bei Hanun angekommen waren, um ihm das Beileid zu bezeugen,
3sagten die Fürsten der Ammoniter zu Hanun: »Meinst du etwa, daß David
Beileidsgesandte an dich deshalb geschickt hat, um deinem Vater eine
Ehre zu erweisen? Nein, offenbar sind seine Gesandten nur deshalb zu dir
gekommen, um das Land gründlich auszukundschaften und um zu spionieren.«
4Da ließ Hanun die Gesandten Davids festnehmen, ließ sie scheren und
ihnen die Röcke\textless sup title=``oder: Gewänder''\textgreater✲ halb
abschneiden bis unter den Gürtel und entließ sie dann. 5So machten sie
sich denn auf den Weg. Als man nun dem David meldete, was den Männern
geschehen war, schickte er ihnen Boten entgegen -- denn die Männer waren
schwer beschimpft --; und der König ließ ihnen sagen: »Bleibt in
Jericho, bis euch der Bart wieder gewachsen ist: dann kehrt zurück!«

\hypertarget{b-ausbruch-des-krieges-erster-sieg-joabs}{%
\paragraph{b) Ausbruch des Krieges; erster Sieg
Joabs}\label{b-ausbruch-des-krieges-erster-sieg-joabs}}

6Als nun die Ammoniter einsahen, daß sie David tödlich beleidigt hatten,
schickten Hanun und die Ammoniter tausend Talente Silber hin, um von den
Syrern in Mesopotamien und von den Syrern in Maacha und in Zoba
Kriegswagen und Reiter in Sold zu nehmen. 7So nahmen sie denn 32000
Wagen in ihren Dienst, dazu den König von Maacha mit seinem Heere; die
kamen und lagerten sich vor Medeba; die Ammoniter aber sammelten sich
auch aus ihren Ortschaften und rückten zum Kampf ins Feld. 8Als David
das erfuhr, ließ er Joab mit dem ganzen Heer, auch den Rittern,
ausrücken. 9Die Ammoniter zogen dann aus der Stadt hinaus und stellten
sich vor dem Stadttor in Schlachtordnung auf, während die Könige, die
herbeigekommen waren, für sich im freien Felde standen. 10Als nun Joab
sah, daß ihm sowohl von vorn als auch im Rücken ein Angriff drohe, nahm
er aus allen auserlesenen israelitischen Kriegern eine Auswahl vor und
stellte sich mit ihnen den Syrern gegenüber auf; 11den Rest des Heeres
aber überwies er seinem Bruder Abisai, damit sie sich den Ammonitern
gegenüber aufstellten. 12Dann sagte er: »Wenn die Syrer mir zu stark
sind, so kommst du mir zu Hilfe; und wenn die Ammoniter dir überlegen
sind, so komme ich dir zu Hilfe. 13Nur Mut! Wir wollen tapfer kämpfen
für unser Volk und für die Städte unsers Gottes; der HERR aber möge tun,
was ihm wohlgefällt!« 14Darauf rückte Joab mit der Mannschaft, die unter
seinem Befehl stand, zum Angriff gegen die Syrer vor, und diese wandten
sich vor ihm zur Flucht; 15und als die Ammoniter die Flucht der Syrer
(vor Joab) gewahrten, flohen auch sie vor seinem Bruder Abisai und zogen
sich in die Stadt zurück. Joab aber kehrte nach Jerusalem zurück.

\hypertarget{c-david-persuxf6nlich-im-felde-sein-sieg-uxfcber-die-mit-den-ammonitern-verbuxfcndeten-syrer}{%
\paragraph{c) David persönlich im Felde; sein Sieg über die mit den
Ammonitern verbündeten
Syrer}\label{c-david-persuxf6nlich-im-felde-sein-sieg-uxfcber-die-mit-den-ammonitern-verbuxfcndeten-syrer}}

16Als nun die Syrer sich von den Israeliten geschlagen sahen, sandten
sie Boten hin und ließen die Syrer von jenseits des Euphrats ins Feld
rücken, und zwar unter der Führung Sophachs, des Feldherrn Hadaresers.
17Auf die Kunde hiervon bot David ganz Israel zum Kriege auf,
überschritt den Jordan und gelangte (nach Helam). Als David sich dann
den Syrern gegenüber in Schlachtordnung aufgestellt hatte und sie
handgemein mit ihnen geworden waren, 18wurden die Syrer von den
Israeliten in die Flucht geschlagen, und David erschlug von den Syrern
7000 Wagenkämpfer und 40000~Mann Fußvolk; auch den Feldherrn Sophach
tötete er. 19Als sich nun die Leute Hadaresers von den Israeliten
besiegt sahen, schlossen sie Frieden mit David und unterwarfen sich ihm.
Infolgedessen hatten auch die Syrer keine Lust mehr, den Ammonitern noch
Hilfe zu leisten.

\hypertarget{d-joab-erobert-rabba-davids-triumph-und-bestrafung-der-ammoniter}{%
\paragraph{d) Joab erobert Rabba; Davids Triumph und Bestrafung der
Ammoniter}\label{d-joab-erobert-rabba-davids-triumph-und-bestrafung-der-ammoniter}}

\hypertarget{section-19}{%
\section{20}\label{section-19}}

1Im folgenden Jahre aber führte Joab zu der Zeit, wo die Könige ins Feld
zu ziehen pflegen, das Heer ins Feld und verwüstete das Land der
Ammoniter; dann zog er hin und belagerte Rabba, während David in
Jerusalem geblieben war. Als Joab dann Rabba erobert und zerstört hatte,
2nahm David ihrem Götzen Milkom die Krone vom Haupt -- diese wog, wie er
feststellte, ein Talent Gold und war mit einem kostbaren Edelstein
besetzt; der kam (nun) auf das Haupt Davids --; und er führte aus der
Stadt eine überaus reiche Beute weg. 3Die Bevölkerung aber, die sich
dort vorfand, ließ er wegführen und stellte sie als Zwangsarbeiter an
die Sägen, an die eisernen Picken und die Äxte. Ebenso verfuhr er mit
allen übrigen Städten der Ammoniter. Dann kehrte David mit dem ganzen
Heere nach Jerusalem zurück.

\hypertarget{e-einige-heldentaten-der-krieger-davids-in-den-philisterkriegen}{%
\paragraph{e) Einige Heldentaten der Krieger Davids in den
Philisterkriegen}\label{e-einige-heldentaten-der-krieger-davids-in-den-philisterkriegen}}

4Später kam es dann nochmals zum Kampf mit den Philistern bei Geser.
Damals erschlug der Husathiter Sibbechai den Sippai, einen von den
Riesenkindern, und so wurden sie gedemütigt. 5Als dann nochmals ein
Kampf mit den Philistern stattfand, erschlug Elhanan, der Sohn Jairs,
Lahmi, den Bruder Goliaths aus Gath, dessen Speerschaft wie ein
Weberbaum war.~-- 6Als es dann wiederum zum Kampf bei Gath kam, war da
ein Mann von riesiger Größe, der je sechs Finger und sechs Zehen hatte,
im ganzen vierundzwanzig; auch dieser stammte von dem Riesengeschlecht.
7Er hatte die Israeliten verhöhnt; aber Jonathan, der Sohn Simeas, des
Bruders Davids, erschlug ihn. 8Diese drei stammten aus dem
Riesengeschlecht in Gath und fielen durch die Hand Davids und seiner
Krieger.

\hypertarget{davids-volkszuxe4hlung-und-ihre-bestrafung}{%
\subsubsection{11. Davids Volkszählung und ihre
Bestrafung}\label{davids-volkszuxe4hlung-und-ihre-bestrafung}}

\hypertarget{a-david-beschlieuxdft-auf-satans-anstiften-die-volkszuxe4hlung-trotz-joabs-warnung-ergebnis-der-zuxe4hlung}{%
\paragraph{a) David beschließt auf Satans Anstiften die Volkszählung
trotz Joabs Warnung; Ergebnis der
Zählung}\label{a-david-beschlieuxdft-auf-satans-anstiften-die-volkszuxe4hlung-trotz-joabs-warnung-ergebnis-der-zuxe4hlung}}

\hypertarget{section-20}{%
\section{21}\label{section-20}}

1Es trat aber (der) Satan gegen Israel auf und verführte David dazu,
eine Zählung der Israeliten vorzunehmen. 2So gebot denn David dem Joab
und den Obersten des Volkes: »Geht hin und nehmt eine Zählung in Israel
vor von Beerseba bis Dan und erstattet mir Bericht, damit ich die Zahl
des Volkes erfahre!« 3Joab entgegnete: »Der HERR möge sein Volk, so
zahlreich es auch jetzt schon ist, noch hundertmal zahlreicher werden
lassen! Sie sind ja doch alle, mein Herr und König, gehorsame Untertanen
meines Herrn. Warum hegt mein Herr solchen Wunsch? Warum soll Israel
eine Verschuldung auf sich laden?« 4Aber der Befehl des Königs blieb
trotz der Vorstellungen Joabs bestehen; und so machte sich denn dieser
auf den Weg und durchwanderte ganz Israel. Als er dann nach Jerusalem
zurückgekehrt war, 5teilte er David das Ergebnis der Volkszählung mit;
da belief sich die Zahl der schwertbewaffneten Männer in ganz Israel auf
1 110000 und in Juda auf 470000 schwertbewaffnete Männer. 6Die Stämme
Levi und Benjamin aber hatte er nicht mit in die Zählung einbegriffen;
denn der Befehl des Königs war für Joab ein Greuel.

\hypertarget{b-davids-reue-eingreifen-des-propheten-gad-david-wuxe4hlt-zur-suxfchnung-seiner-schuld-ein-volkssterben}{%
\paragraph{b) Davids Reue; Eingreifen des Propheten Gad; David wählt zur
Sühnung seiner Schuld ein
Volkssterben}\label{b-davids-reue-eingreifen-des-propheten-gad-david-wuxe4hlt-zur-suxfchnung-seiner-schuld-ein-volkssterben}}

7Weil aber diese ganze Sache das Mißfallen Gottes erregte, so daß er
Israel schwer heimsuchte, 8betete David zu Gott: »Ich habe mich schwer
dadurch versündigt, daß ich dies getan habe; nun aber, laß doch deinem
Knecht seine Verschuldung ungestraft hingehen! Denn ich habe in großer
Verblendung gehandelt.« 9Aber der HERR ließ Gad, dem Seher Davids,
folgende Weisung zukommen: 10»Gehe hin und sage zu David: ›So hat der
HERR gesprochen: Dreierlei lege ich dir vor! Wähle dir eines davon,
damit ich es an dir zur Ausführung bringe!‹« 11Da begab sich Gad zu
David und sagte zu ihm: »So hat der HERR gesprochen: ›Wähle dir
12entweder drei Jahre Hungersnot oder daß du drei Monate vor deinen
Widersachern fliehen mußt und das Schwert deiner Feinde dir zusetzt oder
daß das Schwert des HERRN und die Pest drei Tage lang im Lande wütet, so
daß der Engel des HERRN im ganzen Bereich Israels Verderben anrichtet.‹
Und nun überlege, welche Antwort ich dem bringen soll, der mich gesandt
hat!« 13Da sagte David zu Gad: »Mir ist sehr bange! Ich will lieber in
die Hand des HERRN fallen, denn seine Gnadenerweise sind sehr groß; aber
in die Hand der Menschen möchte ich nicht fallen!«

\hypertarget{c-das-guxf6ttliche-strafgericht-davids-buuxdf--und-bittgebet}{%
\paragraph{c) Das göttliche Strafgericht; Davids Buß- und
Bittgebet}\label{c-das-guxf6ttliche-strafgericht-davids-buuxdf--und-bittgebet}}

14So ließ denn der HERR eine Pest über Israel kommen, so daß
siebzigtausend Menschen in Israel den Tod fanden. 15Dann sandte der HERR
einen Engel nach Jerusalem, um es zu verheeren; doch als er Verheerungen
darin anrichtete, sah der HERR darein, und es gereute ihn das Unheil,
und er gebot dem Engel, der die Verheerung anzurichten hatte: »Es ist
genug so: laß jetzt deine Hand ruhen!« Der Engel des HERRN stand aber
gerade bei der Tenne des Jebusiters Ornan. 16Als nun David aufblickte
und den Engel des HERRN zwischen Erde und Himmel stehen sah, wie er das
gezückte Schwert, das über\textless sup title=``oder:
gegen''\textgreater✲ Jerusalem ausgestreckt war, in der Hand hielt, da
fielen David und die Ältesten\textless sup title=``oder:
Vornehmsten''\textgreater✲, in Trauerkleider gehüllt, auf ihr Angesicht
nieder, 17und David betete zu Gott: »Ach, ich bin's ja, der die
Volkszählung veranlaßt hat, und ich bin es, der gesündigt und ein großes
Unrecht begangen hat! Diese Herde aber, was hat sie verschuldet? HERR,
mein Gott, laß doch deine Hand mich und das Haus meines Vaters treffen,
aber nicht dein Volk zum Sterben!«

\hypertarget{d-david-erwirbt-die-tenne-ornans-und-weiht-sie-zur-opfer--und-tempelstuxe4tte-ein-ende-der-pest}{%
\paragraph{d) David erwirbt die Tenne Ornans und weiht sie zur Opfer-
und Tempelstätte ein; Ende der
Pest}\label{d-david-erwirbt-die-tenne-ornans-und-weiht-sie-zur-opfer--und-tempelstuxe4tte-ein-ende-der-pest}}

18Nun gebot der Engel des HERRN dem Gad, er möge David sagen, David
solle hinaufgehen und dem HERRN einen Altar auf der Tenne des Jebusiters
Ornan errichten. 19Da begab sich David nach der Aufforderung, die Gad im
Namen des HERRN an ihn gerichtet hatte, hinauf. 20{[}Als nun Ornan sich
umwandte, sah er den Engel, während seine vier Söhne sich bei ihm
versteckt hielten -- Ornan drosch nämlich gerade Weizen --.{]} 21Als nun
David zu Ornan kam und dieser aufblickte und David gewahrte, trat er aus
der Tenne hinaus und verneigte sich vor David mit dem Angesicht bis zur
Erde. 22Da sagte David zu Ornan: »Überlaß mir den Platz der Tenne hier,
damit ich dem HERRN einen Altar darauf erbaue; für den vollen Geldeswert
überlaß ihn mir, damit dem Sterben unter dem Volke Einhalt getan wird.«
23Da antwortete Ornan dem David: »Nimm ihn hin: mein Herr und König
wolle tun, was ihm beliebt! Siehe da, ich gebe dir die Rinder zu den
Brandopfern und die Dreschschlitten als Brennholz und den Weizen zum
Speisopfer: das alles gebe ich her!« 24Aber der König David erwiderte
dem Ornan: »Nein, käuflich will ich es von dir erwerben zum vollen Wert;
denn ich will dir nicht dein Eigentum für den HERRN wegnehmen und will
keine Brandopfer darbringen, die mir geschenkt sind!« 25So gab denn
David dem Ornan für den Platz Gold im Gewicht von sechshundert Schekeln.
26David erbaute alsdann dem HERRN dort einen Altar und brachte
Brandopfer und Heilsopfer dar; und als er den HERRN anrief, erhörte
dieser ihn, indem er Feuer vom Himmel auf den Brandopferaltar fallen
ließ. 27Der HERR gebot alsdann dem Engel, und dieser steckte sein
Schwert wieder in die Scheide.

28Weil David damals die Erfahrung gemacht hatte, daß der HERR ihn auf
der Tenne des Jebusiters Ornan erhört hatte, brachte er dort Opfer dar.
29Aber die Wohnung des HERRN, die Mose in der Wüste hergestellt hatte,
und der Brandopferaltar befanden sich zu jener Zeit auf der Höhe bei
Gibeon. 30Doch David gewann es nicht über sich, dort vor Gott zu treten,
um ihn zu verehren, weil er durch das Schwert des Engels des HERRN in
Schrecken versetzt worden war.

\hypertarget{section-21}{%
\section{22}\label{section-21}}

1Und David erklärte: »Dies hier muß das Haus\textless sup title=``=~die
Wohnstätte''\textgreater✲ Gottes, des HERRN, werden und dies der
Brandopferaltar für Israel!«

\hypertarget{davids-vorbereitungen-fuxfcr-den-tempelbau}{%
\subsubsection{12. Davids Vorbereitungen für den
Tempelbau}\label{davids-vorbereitungen-fuxfcr-den-tempelbau}}

\hypertarget{a-sammlung-von-baumaterialien}{%
\paragraph{a) Sammlung von
Baumaterialien}\label{a-sammlung-von-baumaterialien}}

2Hierauf befahl David, man solle die Fremdlinge✲, die sich im Lande
Israel befanden, zusammenbringen, und er beschäftigte sie als
Steinmetzen, um Quadersteine zum Bau des Tempels Gottes zuzuhauen. 3Auch
Eisen beschaffte David in Menge zu Nägeln für die Torflügel und zu
Klammern, ebenso Kupfer in solcher Menge, daß man es nicht wägen konnte;
4ferner zahllose Zedernstämme, denn die Sidonier und Tyrier brachten dem
David Zedernholz in Menge. 5David dachte nämlich: »Mein Sohn Salomo ist
noch jung und zart, der Tempel aber, der dem HERRN erbaut werden soll,
muß überaus großartig werden, damit er in der ganzen Welt berühmt und
bewundert dasteht; darum will ich die zum Bau erforderlichen
Vorbereitungen für ihn treffen.« So traf denn David vor seinem Tode
Zurüstungen in Menge.

\hypertarget{b-davids-anweisungen-an-seinen-sohn-salomo}{%
\paragraph{b) Davids Anweisungen an seinen Sohn
Salomo}\label{b-davids-anweisungen-an-seinen-sohn-salomo}}

6Sodann ließ er seinen Sohn Salomo zu sich kommen und machte es ihm zur
Pflicht, dem HERRN, dem Gott Israels, einen Tempel zu erbauen. 7Dabei
richtete David folgende Worte an Salomo: »Mein Sohn! Ich selbst hatte
die Absicht, dem Namen des HERRN, meines Gottes, einen Tempel zu
erbauen; 8aber da erging das Wort des HERRN an mich folgendermaßen: ›Du
hast Blut in Menge vergossen und schwere Kriege geführt; du darfst
meinem Namen kein Haus erbauen, weil du viel Blut vor meinen Augen zur
Erde hast fließen lassen. 9Doch wisse: ein Sohn wird dir geboren werden,
der wird ein Mann der Ruhe sein, und ich will ihm Ruhe vor all seinen
Feinden ringsum schaffen; denn Salomo\textless sup title=``d.h. der
Friedliche, Friedensmann''\textgreater✲ soll sein Name sein, und Frieden
und Ruhe will ich in Israel während seiner Regierung herrschen lassen.
10Der soll meinem Namen ein Haus erbauen; er soll mir als Sohn gelten
und ich ihm als Vater, und seinen Königsthron über Israel will ich auf
ewig feststellen.‹ 11Nun denn, mein Sohn: der HERR wolle mit dir sein,
damit es dir gelingt, den Tempel des HERRN, deines Gottes, zu erbauen,
wie er es von dir verheißen hat! 12Nur möge der HERR dir Klugheit und
Einsicht verleihen, wenn er dich zum Herrscher über Israel bestellt, und
(wolle geben,) daß du das Gesetz des HERRN, deines Gottes, beobachtest!
13Dann wirst du Glück\textless sup title=``oder: Erfolg''\textgreater✲
haben, wenn du die Satzungen und Verordnungen gewissenhaft befolgst, die
der HERR den Israeliten durch Mose geboten hat. Sei fest und stark!
Fürchte dich nicht und laß dir nicht bange sein! 14Siehe, trotz meines
mühseligen Lebens habe ich für den Tempel des HERRN hunderttausend
Talente Gold und eine Million Talente Silber beschafft, ferner an Kupfer
und Eisen so viel, daß es nicht zu wägen ist: in solcher Menge ist es
vorhanden; auch Holzstämme und Steine habe ich beschafft, und du wirst
noch mehr dazutun. 15Auch Werkleute stehen dir in großer Zahl zur
Verfügung: Steinmetzen, Maurer und Zimmerleute, kunstverständige Männer
jeder Art für alle Arbeiten 16in Gold und Silber, in Kupfer und Eisen
ohne Zahl. Wohlan, gehe ans Werk, und der HERR sei mit dir!«

\hypertarget{c-davids-mahnung-an-die-fuxfcrsten-israels}{%
\paragraph{c) Davids Mahnung an die Fürsten
Israels}\label{c-davids-mahnung-an-die-fuxfcrsten-israels}}

17Hierauf forderte David alle Obersten\textless sup title=``oder:
Fürsten''\textgreater✲ Israels auf, seinem Sohne Salomo hilfreich zur
Seite zu stehen, mit den Worten: 18»Ist nicht der HERR, euer Gott, mit
euch gewesen, und hat er euch nicht ringsum Ruhe verschafft? Er hat ja
die Bewohner des Landes in meine Gewalt gegeben, und das Land liegt
unterworfen vor dem HERRN und seinem Volke da. 19So richtet denn euer
Herz und euren Sinn darauf, den HERRN, euren Gott, zu suchen, und macht
euch daran, das Heiligtum Gottes, des HERRN, zu erbauen, damit ihr die
Bundeslade des HERRN und die heiligen Geräte Gottes in den Tempel
bringen könnt, der dem Namen des HERRN erbaut werden soll!«

\hypertarget{einteilung-und-amtsgeschuxe4fte-der-leviten-der-priester-und-der-sonstigen-beamten}{%
\subsubsection{13. Einteilung und Amtsgeschäfte der Leviten, der
Priester und der sonstigen
Beamten}\label{einteilung-und-amtsgeschuxe4fte-der-leviten-der-priester-und-der-sonstigen-beamten}}

\hypertarget{a-dienstordnung-der-leviten}{%
\paragraph{a) Dienstordnung der
Leviten}\label{a-dienstordnung-der-leviten}}

\hypertarget{aa-zuxe4hlung-und-verrichtungen-der-leviten}{%
\subparagraph{aa) Zählung und Verrichtungen der
Leviten}\label{aa-zuxe4hlung-und-verrichtungen-der-leviten}}

\hypertarget{section-22}{%
\section{23}\label{section-22}}

1Als nun David alt und lebenssatt war, machte er seinen Sohn Salomo zum
König über Israel. 2Er ließ dabei alle Obersten\textless sup
title=``oder: Fürsten''\textgreater✲ Israels sowie die Priester und
Leviten zusammenkommen; 3und als man die Leviten im Alter von dreißig
und mehr Jahren zählte, ergab sich bei ihnen eine Kopfzahl von 38000
Männern. 4»Von diesen«, gebot David, »sollen 24000 dem Dienst am Hause
des HERRN vorstehen, 6000 sollen Amtleute und Richter sein, 54000
Torhüter, und 4000 sollen zum Lobpreis des HERRN die Instrumente
spielen, die ich zu diesem Zweck habe anfertigen lassen.«

\hypertarget{bb-einteilung-der-leviten-nach-gerson-kehath-und-merari}{%
\subparagraph{bb) Einteilung der Leviten nach Gerson, Kehath und
Merari}\label{bb-einteilung-der-leviten-nach-gerson-kehath-und-merari}}

6David teilte sie dann in Abteilungen nach den Söhnen Levis, nach
Gerson, Kahath und Merari. 7Zu den Gersoniten gehörten: Laedan und
Simei. 8Die Söhne Laedans waren: Jehiel, das Oberhaupt, ferner Setham
und Joel, zusammen drei. 9Die Söhne Simeis waren: Selomith, Hasiel und
Haran, zusammen drei. Dies waren die Familienhäupter der Laedaniten.
10Die Söhne Simeis waren: Jahath, Sisa, Jehus und Beria; dies waren
Simeis Söhne, zusammen vier; 11und zwar war Jahath das Oberhaupt und
Sisa der zweite; Jehus und Beria aber hatten nicht viele
Söhne\textless sup title=``oder: Kinder''\textgreater✲; darum bildeten
sie nur eine Familie, nur eine Klasse.

12Die Söhne Kehaths waren: Amram, Jizhar, Hebron und Ussiel, zusammen
vier. 13Die Söhne Amrams waren: Aaron und Mose. Aaron aber wurde
ausgesondert, damit er den Dienst für das Allerheiligste versehe, er und
seine Nachkommen auf ewige Zeiten, um vor dem HERRN die Rauchopfer
darzubringen, ihm zu dienen und in seinem Namen den Segen zu sprechen
ewiglich. 14Was aber Mose, den Mann Gottes, betrifft, so wurden seine
Söhne einfach zum Stamme Levi gerechnet. 15Die Söhne Moses waren: Gersom
und Elieser; 16die Söhne Gersoms: Sebuel\textless sup title=``oder:
Subael; vgl. 24,20''\textgreater✲, das Oberhaupt. 17Und die Söhne
Eliesers waren: Rehabja, das Oberhaupt; andere Söhne hatte Elieser
nicht, während die Söhne Rehabjas überaus zahlreich waren.~-- 18Die
Söhne Jizhars waren: Selomith, das Oberhaupt; 19die Söhne Hebrons:
Jerija, das Oberhaupt, Amarja der zweite, Ussiel der dritte, Jekameam
der vierte. 20Die Söhne Ussiels waren: Micha, das Oberhaupt, und Jissia,
der zweite.

21Die Söhne Meraris waren: Mahli und Musi; die Söhne Mahlis: Eleasar und
Kis. 22Eleasar aber hinterließ bei seinem Tode keine Söhne, sondern nur
Töchter, die sich mit ihren Vettern, den Söhnen des Kis, verheirateten.
23Die Söhne Musis waren: Mahli, Eder und Jeremoth, zusammen drei.

\hypertarget{cc-dienstliche-anweisungen-fuxfcr-die-leviten}{%
\subparagraph{cc) Dienstliche Anweisungen für die
Leviten}\label{cc-dienstliche-anweisungen-fuxfcr-die-leviten}}

24Dies waren die Nachkommen Levis nach ihren Familien, die
Familienhäupter, soviele ihrer gemustert wurden, nach Köpfen namentlich
aufgezählt, die beim Dienst am Tempel des HERRN beschäftigt waren, von
zwanzig Jahren an und darüber. 25Denn David dachte: »Der HERR, der Gott
Israels, hat seinem Volke Ruhe verschafft und wohnt nun für immer in
Jerusalem; 26so brauchen nun auch die Leviten das Zelt und alle zu
seiner Bedienung erforderlichen Geräte nicht mehr zu tragen«; 27denn
nach den letzten Anordnungen Davids waren die Leviten im Alter von
zwanzig Jahren und darüber gezählt worden. 28Vielmehr dient nun ihre
Amtstätigkeit zur Unterstützung der Nachkommen Aarons beim Dienst am
Tempel des HERRN als Aufseher über die Vorhöfe und über die Zellen und
über die Reinigung\textless sup title=``oder: Reinhaltung''\textgreater✲
alles Heiligen, überhaupt für alles, was zu den Dienstleistungen am
Tempel des HERRN gehört, 29außerdem für die Besorgung der
aufgeschichteten Schaubrote und des Feinmehls zu den Speisopfern, für
die ungesäuerten Fladen, für das Pfannenbackwerk und das Eingerührte
sowie für alles (Messen mit) Hohl- und Längenmaßen. 30Auch müssen sie
alle Morgen antreten, um dem HERRN Lob- und Danklieder zu singen, ebenso
auch am Abend. 31Auch bei jeder Darbringung von Brandopfern für den
HERRN haben sie an den Sabbaten, Neumonden und Festen Dienst zu tun,
sooft Opfer nach der dafür feststehenden Ordnung regelmäßig dem HERRN
darzubringen sind. 32So haben sie also die Geschäfte am Offenbarungszelt
und die Geschäfte am Heiligtum, überhaupt die Geschäfte zur
Unterstützung der Nachkommen Aarons, ihrer Stammesgenossen, beim Dienst
am Tempel des HERRN zu besorgen.

\hypertarget{b-die-priester--und-levitenklassen}{%
\paragraph{b) Die Priester- und
Levitenklassen}\label{b-die-priester--und-levitenklassen}}

\hypertarget{aa-die-auslosung-der-24-priesterklassen}{%
\subparagraph{aa) Die Auslosung der 24
Priesterklassen}\label{aa-die-auslosung-der-24-priesterklassen}}

\hypertarget{section-23}{%
\section{24}\label{section-23}}

1Was sodann die Nachkommen Aarons betrifft, so waren ihre Abteilungen
folgende: Die Söhne Aarons waren: Nadab und Abihu, Eleasar und Ithamar.
2Nadab und Abihu starben jedoch vor ihrem Vater, ohne Söhne zu
hinterlassen; daher übten Eleasar und Ithamar den Priesterdienst allein
aus. 3David teilte sie nun, im Einvernehmen mit Zadok von den
Nachkommen\textless sup title=``=~aus dem Geschlecht''\textgreater✲
Eleasars und mit Ahimelech von den Nachkommen Ithamars, in Klassen ein
je nach ihrem Amt bei ihrer Dienstleistung. 4Dabei stellte es sich nun
heraus, daß die Nachkommen Eleasars an Familienhäuptern zahlreicher
waren als die Nachkommen Ithamars; daher teilte man sie so ab, daß auf
die Nachkommen Eleasars sechzehn, auf die Nachkommen Ithamars acht
Familienhäupter kamen. 5Man teilte sie aber, die einen wie die anderen,
durch Lose ab; denn sowohl unter Eleasars als auch unter Ithamars
Nachkommen gab es ›Fürsten\textless sup title=``oder:
Oberpriester''\textgreater✲ des Heiligtums‹ und ›Fürsten\textless sup
title=``oder: Oberpriester''\textgreater✲ Gottes‹; 6und Semaja, der Sohn
Nethaneels, der Schriftführer unter den Leviten, schrieb sie in
Gegenwart des Königs und der Fürsten sowie des Priesters Zadok und
Ahimelechs, des Sohnes Abjathars, und der Familienhäupter der Priester
und der Leviten auf: je eine Familie wurde für Ithamar ausgelost, und
dann wurde je zweimal eine für Eleasar ausgelost.

7Das erste Los fiel auf Jojarib, das zweite auf Jedaja, 8das dritte auf
Harim, das vierte auf Seorim, 9das fünfte auf Malchia, das sechste auf
Mijjamin, 10das siebte auf Hakkoz, das achte auf Abia, 11das neunte auf
Jesua, das zehnte auf Sechanja, 12das elfte auf Eljasib, das zwölfte auf
Jakim, 13das dreizehnte auf Huppa, das vierzehnte auf
Jesebab\textless sup title=``oder: Isbaal''\textgreater✲, 14das
fünfzehnte auf Bilga, das sechzehnte auf Immer, 15das siebzehnte auf
Hesir, das achtzehnte auf Happizzez, 16das neunzehnte auf Pethahja, das
zwanzigste auf Jeheskel, 17das einundzwanzigste auf Jachin, das
zweiundzwanzigste auf Gamul, 18das dreiundzwanzigste auf Delaja, das
vierundzwanzigste auf Maasja. 19Dies war ihre Klassenordnung für ihren
Dienst, damit sie entsprechend der durch ihren Ahnherrn Aaron für sie
bestimmten Verordnung in den Tempel des HERRN einträten, wie der HERR,
der Gott Israels, ihm geboten hatte.

\hypertarget{bb-die-levitenklassen-und-ihre-vorsteher}{%
\subparagraph{bb) Die Levitenklassen und ihre
Vorsteher}\label{bb-die-levitenklassen-und-ihre-vorsteher}}

20Was aber die übrigen Nachkommen Levis betrifft, so war von den
Nachkommen Amrams Subael da, von den Nachkommen Subaels Jehdeja; 21von
den Nachkommen Rehabjas war Jissia das Oberhaupt.~-- 22Von den
Jizhariten: Selomoth, von den Nachkommen Selomoths: Jahath.~-- 23Die
Nachkommen Hebrons waren: Jerija das Oberhaupt, Amarja der zweite,
Jahasiel\textless sup title=``oder: Ussiel''\textgreater✲ der dritte,
Jekameam der vierte.~-- 24Die Nachkommen Ussiels waren: Micha; von den
Nachkommen Michas: Samir. 25Michas Bruder war Jissia; von den Nachkommen
Jissias: Sacharja.~-- 26Die Nachkommen Meraris waren: Mahli und Musi und
die Nachkommen seines Sohnes Jaasia. 27Die Nachkommen Meraris von seinem
Sohne Ussia waren: Soham, Sakkur und Ibri; 28von Mahli: Eleasar, der
aber keine Söhne hatte, und Kis; 29von Kis: die Söhne des Kis:
Jerahmeel. 30Die Nachkommen Musis waren: Mahli, Eder und Jerimoth. Dies
waren die Nachkommen der Leviten nach ihren Familien.~-- 31Auch sie
wurden ausgelost ganz wie ihre Stammesgenossen, die Nachkommen Aarons,
in Gegenwart des Königs David und Zadoks und Ahimelechs sowie der
Familienhäupter der Priester und der Leviten, und zwar die
Familienhäupter ganz ebenso wie ihre jüngsten\textless sup title=``oder:
geringsten''\textgreater✲ Stammesgenossen.

\hypertarget{c-die-auslosung-der-24-abteilungen-der-heiligen-suxe4nger-und-musiker}{%
\paragraph{c) Die Auslosung der 24 Abteilungen der heiligen Sänger und
Musiker}\label{c-die-auslosung-der-24-abteilungen-der-heiligen-suxe4nger-und-musiker}}

\hypertarget{section-24}{%
\section{25}\label{section-24}}

1Weiter sonderte David mit den Heeresobersten von den
Söhnen\textless sup title=``oder: Nachkommen''\textgreater✲ Asaphs,
Hemans und Jeduthuns diejenigen aus, die auf Zithern, Harfen und mit
Zimbeln als Leiter der geistlichen Kunstmusik für den heiligen Dienst
tätig waren\textless sup title=``eig. weissagten''\textgreater✲. Die
Zahl der zu diesem Dienst bestellten Männer war folgende: 2Von den
Söhnen\textless sup title=``oder: Nachkommen''\textgreater✲ Asaphs:
Sakkur, Joseph, Nethanja und Asarela, die Söhne Asaphs, unter der
Leitung Asaphs, der nach Anweisung des Königs geistliche Kunstmusik
darbot. 3Von Jeduthun: Jeduthuns Söhne: Gedalja, Zeri\textless sup
title=``oder: Zizri''\textgreater✲, Jesaja, Hasabja, Matthithja und
Simei, zusammen sechs, unter der Leitung ihres Vaters Jeduthun, der zum
Lobpreis und zur Verherrlichung des HERRN geistliche Kunstmusik auf der
Harfe darbot. 4Von Heman: Hemans Söhne: Bukkia, Matthanja, Ussiel,
Subael, Jerimoth, Hananja, Hanani, Eliatha, Giddalthi, Romamthi-Eser,
Josbekasa, Mallothi, Hothir und Mahasioth. 5Diese alle waren Söhne
Hemans, des Sehers des Königs, nach der Verheißung Gottes, ihm das Horn
zu erhöhen; denn Gott hatte dem Heman vierzehn Söhne und drei Töchter
geschenkt. 6Diese alle waren unter der Leitung ihres Vaters Asaph beim
Gesang im Tempel des HERRN mit Zimbeln, Harfen und Zithern für den
Gottesdienst im Tempel nach der Anweisung des Königs, Asaphs, Jeduthuns
und Hemans tätig. 7Ihre Anzahl, inbegriffen ihre Amtsgenossen, die im
Gesang für den HERRN geübt waren, allesamt Künstler\textless sup
title=``oder: Meister''\textgreater✲, betrug 288.

8Als sie nun die Losung zur Feststellung der Reihenfolge ihres Dienstes
vornahmen, die jüngeren ganz wie die älteren, die Meister samt den
Schülern, 9fiel das erste Los für Asaph auf Joseph nebst seinen Söhnen
und Brüdern, zusammen zwölf; das zweite auf Gedalja nebst seinen Brüdern
und Söhnen, zusammen zwölf; 10das dritte auf Sakkur nebst seinen Söhnen
und Brüdern, zusammen zwölf; 11das vierte auf Jizri nebst seinen Söhnen
und Brüdern, zusammen zwölf; 12das fünfte auf Nethanja nebst seinen
Söhnen und Brüdern, zusammen zwölf; 13das sechste auf Bukkia nebst
seinen Söhnen und Brüdern, zusammen zwölf; 14das siebte auf Jesarela
nebst seinen Söhnen und Brüdern, zusammen zwölf; 15das achte auf Jesaja
nebst seinen Söhnen und Brüdern, zusammen zwölf; 16das neunte auf
Matthanja nebst seinen Söhnen und Brüdern, zusammen zwölf; 17das zehnte
auf Simei nebst seinen Söhnen und Brüdern, zusammen zwölf; 18das elfte
auf Ussiel nebst seinen Söhnen und Brüdern, zusammen zwölf; 19das
zwölfte auf Hasabja nebst seinen Söhnen und Brüdern, zusammen zwölf;
20das dreizehnte auf Subael nebst seinen Söhnen und Brüdern, zusammen
zwölf; 21das vierzehnte auf Matthitja nebst seinen Söhnen und Brüdern,
zusammen zwölf; 22das fünfzehnte auf Jeremoth nebst seinen Söhnen und
Brüdern, zusammen zwölf; 23das sechzehnte auf Hananja nebst seinen
Söhnen und Brüdern, zusammen zwölf; 24das siebzehnte auf Josbekasa nebst
seinen Söhnen und Brüdern, zusammen zwölf; 25das achtzehnte auf Hanani
nebst seinen Söhnen und Brüdern, zusammen zwölf; 26das neunzehnte auf
Mallothi nebst seinen Söhnen und Brüdern, zusammen zwölf; 27das
zwanzigste auf Eliatha nebst seinen Söhnen und Brüdern, zusammen zwölf;
28das einundzwanzigste auf Hothir nebst seinen Söhnen und Brüdern,
zusammen zwölf; 29das zweiundzwanzigste auf Giddalthi nebst seinen
Söhnen und Brüdern, zusammen zwölf; 30das dreiundzwanzigste auf
Mahasioth nebst seinen Söhnen und Brüdern, zusammen zwölf; 31das
vierundzwanzigste auf Romamthi-Eser nebst seinen Söhnen und Brüdern,
zusammen zwölf.

\hypertarget{d-die-abteilungen-der-levitischen-torhuxfcter}{%
\paragraph{d) Die Abteilungen der levitischen
Torhüter}\label{d-die-abteilungen-der-levitischen-torhuxfcter}}

\hypertarget{section-25}{%
\section{26}\label{section-25}}

1Was die Abteilungen der Torhüter betrifft, so gehörte von den Korahiten
Meselemja, der Sohn Kores, zu den Nachkommen Ebjasaphs\textless sup
title=``vgl. 9,19''\textgreater✲; 2Meselemjas Söhne aber waren: Sacharja
der erstgeborene, Jediael der zweite, Sebadja der dritte, Jathniel der
vierte, 3Elam der fünfte, Johanan der sechste, Eljehoenai der siebte.~--
4Die Söhne Obed-Edoms waren: Semaja der erstgeborene, Josabad der
zweite, Joah der dritte, Sachar der vierte, Nethaneel der fünfte,
5Ammiel der sechste, Issaschar der siebte, Pehullethai der achte; denn
Gott hatte ihn gesegnet.~-- 6Auch seinem Sohne Semaja waren Söhne
geboren, welche in ihrer Familie eine leitende Stellung einnahmen, denn
sie waren tüchtige Männer. 7Die Söhne Semajas waren: Othni, Rephael,
Obed, Elsabad und seine Brüder, Elihu und Semachja, wackere Männer.
8Diese alle gehörten zu den Nachkommen Obed-Edoms, sie, ihre Söhne und
ihre Brüder, tüchtige Männer, tauglich zu Dienstleistungen, zusammen
zweiundsechzig von Obed-Edom.~-- 9Auch Meselemja hatte Söhne und Brüder,
tüchtige Männer, zusammen achtzehn.~-- 10Die Söhne Hosas, der zu den
Nachkommen Meraris gehörte, waren: Simri als Oberhaupt -- obwohl er
nicht der Erstgeborene war, hatte ihn sein Vater doch zum Oberhaupt
eingesetzt --, 11Hilkia der zweite Sohn, Tebalja der dritte, Sacharja
der vierte; die Gesamtzahl aller Söhne und Brüder Hosas betrug dreizehn.

\hypertarget{die-verteilung-der-torhuxfcter-an-die-verschiedenen-orte}{%
\paragraph{Die Verteilung der Torhüter an die verschiedenen
Orte}\label{die-verteilung-der-torhuxfcter-an-die-verschiedenen-orte}}

12Diesen Abteilungen der Torhüter, und zwar den Familienhäuptern, fielen
gerade wie ihren Stammesgenossen Amtsgeschäfte im Dienst am Tempel des
HERRN zu. 13Als man nämlich die Verlosung der einzelnen Tore nach den
Familien, für die älteren wie für die jüngeren, vornahm, 14fiel für die
Ostseite das Los auf Selemja. Auch für seinen Sohn Sacharja, einen
klugen Ratgeber, warf man Lose, und das Los fiel für ihn auf die
Nordseite; 15für Obed-Edom auf die Südseite und für seine Söhne auf das
Vorratsgebäude; 16für Hosa auf die Westseite beim Tor Sallecheth, da wo
die Straße ansteigt, eine Wache neben der andern. 17Auf der Ostseite
waren täglich sechs Leviten auf Wache, nach Norden zu täglich vier, nach
Süden zu täglich vier, und am Vorratsgebäude je zwei; 18am
Parbar\textless sup title=``d.h. Anbau; vgl. 2.Kön 23,11''\textgreater✲
auf der Westseite: vier an der Straße und zwei am Parbar. 19Dies sind
die Abteilungen der Torhüter von den Nachkommen\textless sup
title=``=~aus dem Geschlecht''\textgreater✲ der Korahiten und von den
Nachkommen Meraris.

\hypertarget{e-die-levitischen-schatzmeister-und-die-beamten-der-verwaltung}{%
\paragraph{e) Die levitischen Schatzmeister und die Beamten der
Verwaltung}\label{e-die-levitischen-schatzmeister-und-die-beamten-der-verwaltung}}

20Ihre levitischen Stammesgenossen aber, welche die Aufsicht über die
Schatzkammern\textless sup title=``oder: Vorräte''\textgreater✲ des
Hauses Gottes und über die Schätze an geweihten Gegenständen hatten,
waren: 21die Nachkommen Ladans, die Nachkommen des Gersoniten Ladan, die
Familienhäupter vom Geschlecht des Gersoniten Ladan, die Jehieliten.
22Die Nachkommen der Jehieliten waren: Setham und sein Bruder Joel,
welche die Aufsicht über die Schätze\textless sup title=``oder:
Vorräte''\textgreater✲ des Gotteshauses hatten.~-- 23Was sodann die
Amramiten, Jizhariten, Hebroniten und Ussieliten betrifft, 24so war
Subael, ein Nachkomme Gersoms, des Sohnes Moses, Oberaufseher über die
Schätze\textless sup title=``oder: Vorräte''\textgreater✲.~-- 25Was
ferner seine Stammesgenossen von Elieser her betrifft, so war dessen
Sohn Rehabja, dessen Sohn Jesaja, dessen Sohn Joram, dessen Sohn Sichri,
dessen Sohn Selomith. 26Dieser Selomith und seine Brüder waren über alle
Schätze an geweihten Gegenständen gesetzt, welche der König David und
die Familienhäupter, ferner die Befehlshaber der Tausendschaften und der
Hundertschaften sowie die Heerführer geweiht hatten,~-- 27von der Beute
aus den Kriegen hatten sie sie geweiht, um das Haus des HERRN damit
auszustatten --; 28auch alles, was der Seher Samuel und Saul, der Sohn
des Kis, und Abner, der Sohn Ners, und Joab, der Sohn der Zeruja,
geweiht hatten, überhaupt alle Weihgeschenke standen unter der Aufsicht
Selomiths und seiner Geschlechtsgenossen.

29Aus (dem Geschlecht) der Jizhariten waren Kenanja und seine Söhne für
die auswärtigen Geschäfte bezüglich Israels als Amtleute und Richter
bestellt.~-- 30Aus (dem Geschlecht) der Hebroniten waren Hasabja und
seine Geschlechtsgenossen, tüchtige Männer, 1700 an der Zahl, über die
Verwaltung Israels westlich vom Jordan für alle Angelegenheiten des
HERRN und für den Dienst des Königs eingesetzt.~-- 31Zu den Hebroniten
gehörte Jerija, das Oberhaupt -- was die Hebroniten betrifft, so
forschte man nach ihren Geschlechtern und Familien im vierzigsten
Regierungsjahre Davids, und es fanden sich unter ihnen tüchtige Männer
zu Jaeser in Gilead --, 32dazu seine Stammesgenossen, tüchtige Männer,
zusammen 2700; denen übertrug der König David die Verwaltung der Stämme
Ruben, Gad und halb Manasse für alle Angelegenheiten Gottes und für die
Angelegenheiten des Königs.

\hypertarget{die-zwuxf6lf-heerfuxfchrer-die-stammesfuxfcrsten-und-die-sonstigen-huxf6heren-beamten-davids}{%
\subsubsection{14. Die zwölf Heerführer, die Stammesfürsten und die
sonstigen höheren Beamten
Davids}\label{die-zwuxf6lf-heerfuxfchrer-die-stammesfuxfcrsten-und-die-sonstigen-huxf6heren-beamten-davids}}

\hypertarget{a-die-zwuxf6lfteilung-des-heeres}{%
\paragraph{a) Die Zwölfteilung des
Heeres}\label{a-die-zwuxf6lfteilung-des-heeres}}

\hypertarget{section-26}{%
\section{27}\label{section-26}}

1Und dies waren die Israeliten nach ihrer Zahl, die Familienhäupter und
die Befehlshaber der Tausendschaften und der Hundertschaften sowie ihre
Amtleute, die im Dienst des Königs standen, in allen Angelegenheiten der
Abteilungen, welche Monat für Monat, alle Monate des Jahres hindurch,
antraten und abtraten; eine jede Abteilung war 24000~Mann stark.

2An der Spitze der ersten Abteilung für den ersten Monat stand Jasobeam,
der Sohn Sabdiels, und seine Abteilung belief sich auf 24000~Mann; 3er
gehörte zu den Nachkommen des Perez und war das Oberhaupt aller
Heerführer für den ersten Monat. 4An der Spitze der Abteilung für den
zweiten Monat stand Eleasar\textless sup title=``vgl.
11,12''\textgreater✲, der Sohn Dodais, der Ahohiter, und von seiner
Abteilung war Mikloth der Oberaufseher; seine Abteilung belief sich auf
24000. 5Der dritte Heerführer für den dritten Monat war Benaja, der Sohn
des Priesters Jojada, als Oberhaupt; seine Abteilung belief sich auf
24000~Mann. 6Dieser Benaja war ein Held unter den Dreißig und war über
die Dreißig gesetzt, und bei\textless sup title=``oder: an der
Spitze''\textgreater✲ seiner Abteilung stand sein Sohn Ammisabad. 7Der
vierte, für den vierten Monat, war Joabs Bruder Asahel und nach ihm sein
Sohn Sebadja; seine Abteilung belief sich auf 24000. 8Der fünfte, für
den fünften Monat, war der Fürst Samhuth, der Jisrahither; seine
Abteilung belief sich auf 24000. 9Der sechste, für den sechsten Monat,
war Ira, der Sohn des Ikkes, aus Thekoa; seine Abteilung belief sich auf
24000. 10Der siebte, für den siebten Monat, war Helez, der Pelonite, aus
dem Stamme Ephraim; seine Abteilung belief sich auf 24000. 11Der achte,
für den achten Monat, war Sibbechai, der Hussathiter, aus den Serahiten;
seine Abteilung belief sich auf 24000. 12Der neunte, für den neunten
Monat, war Abieser aus Anathoth, aus dem Stamme Benjamin; seine
Abteilung belief sich auf 24000. 13Der zehnte, für den zehnten Monat,
war Maharai, von Netopha, aus den Serahiten; seine Abteilung belief sich
auf 24000. 14Der elfte, für den elften Monat, war Benaja aus Pirathon,
aus dem Stamme Ephraim; seine Abteilung belief sich auf 24000. 15Der
zwölfte, für den zwölften Monat, war Heldai aus Netopha, aus dem
Geschlecht Othniels; seine Abteilung belief sich auf 24000.

\hypertarget{b-die-zwuxf6lf-stammesfuxfcrsten-israels}{%
\paragraph{b) Die zwölf Stammesfürsten
Israels}\label{b-die-zwuxf6lf-stammesfuxfcrsten-israels}}

16An der Spitze der Stämme Israels standen: von den Rubeniten als Fürst
Elieser, der Sohn Sichris; von den Simeoniten Sephatja, der Sohn
Maachas; 17von Levi Hasabja, der Sohn Kemuels; von Aaron Zadok; 18von
Juda Elihu, einer der Brüder Davids; von Issaschar Omri, der Sohn
Michaels; 19von Sebulon Jismaja, der Sohn Obadjas; von Naphthali
Jerimoth, der Sohn Asriels; 20von den Ephraimiten Hosea, der Sohn
Asasjas; vom halben Stamm Manasse Joel, der Sohn Pedajas; 21vom anderen
halben Stamm Manasse in Gilead Iddo, der Sohn Sacharjas; von Benjamin
Jaasiel, der Sohn Abners; 22von Dan Asareel, der Sohn Jerohams. Dies
waren die Fürsten der israelitischen Stämme.

\hypertarget{c-bemerkung-uxfcber-die-unvollstuxe4ndig-gebliebene-volkszuxe4hlung}{%
\paragraph{c) Bemerkung über die unvollständig gebliebene
Volkszählung}\label{c-bemerkung-uxfcber-die-unvollstuxe4ndig-gebliebene-volkszuxe4hlung}}

23David hatte aber die Zählung der Israeliten von zwanzig Jahren an und
darunter nicht vornehmen lassen, weil der HERR verheißen hatte, er wolle
Israel so zahlreich machen wie die Sterne am Himmel. 24Joab, der Sohn
der Zeruja, hatte zwar die Zählung begonnen, sie aber nicht zu Ende
geführt; es war nämlich um ihretwillen ein Zorngericht über Israel
ergangen; daher ist die Zahl nicht unter die Zahlen in der
Zeitgeschichte des Königs David aufgenommen worden.

\hypertarget{d-die-verwalter-des-kuxf6niglichen-besitzes-schatz--und-rentmeister}{%
\paragraph{d) Die Verwalter des königlichen Besitzes (Schatz- und
Rentmeister)}\label{d-die-verwalter-des-kuxf6niglichen-besitzes-schatz--und-rentmeister}}

25Und über die Schätze\textless sup title=``oder: Vorräte; d.h. als
Schatzmeister''\textgreater✲ des Königs war Osmaweth, der Sohn Adiels,
bestellt und als Rentmeister für die Besitzungen an Grundstücken in den
Städten, Dörfern und Burgen Jonathan, der Sohn Ussias. 26Über die
Feldarbeiter, die das Land zu bebauen hatten, war Esri, der Sohn
Chelubs, gesetzt; 27über die Weinberge Simei aus Rama; über die
Weinvorräte in den Weinkellern Sabdi, der Siphmite; 28über die
Olivengärten und die Maulbeerfeigenbäume in der Niederung Baal-Hanan aus
Gader; über die Ölvorräte Joas; 29über die Rinder, die in der
Saron-Ebene weideten, Sitrai, der Saronite; über die Rinder in den
Tälern Saphat, der Sohn Adlais; 30über die Kamele Obil, der Ismaelite;
über die Eselinnen Jehdeja, der Meronothite; 31über das Kleinvieh Jasis,
der Hagrite. Diese alle waren Verwalter des Vermögens, das der König
David besaß.

\hypertarget{e-die-huxf6chsten-reichsbeamten-ruxe4te-des-kuxf6nigs}{%
\paragraph{e) Die höchsten Reichsbeamten (Räte des
Königs)}\label{e-die-huxf6chsten-reichsbeamten-ruxe4te-des-kuxf6nigs}}

32Jonathan, Davids Oheim, war Rat, ein einsichtiger und gelehrter Mann;
Jehiel, der Sohn Hachmonis, war Erzieher der Söhne des Königs.
33Ahithophel war königlicher Rat, und Husai, der Arkiter, der Vertraute
des Königs. 34Nach Ahithophel war es Jojada, der Sohn Benajas, und
Abjathar; Joab aber war der Feldhauptmann des Königs.

\hypertarget{davids-letzter-reichstag-und-letzte-verfuxfcgungen-in-betreff-des-tempelbaues-und-der-thronfolge-sein-gebet-und-tod}{%
\subsubsection{15. Davids letzter Reichstag und letzte Verfügungen in
betreff des Tempelbaues und der Thronfolge; sein Gebet und
Tod}\label{davids-letzter-reichstag-und-letzte-verfuxfcgungen-in-betreff-des-tempelbaues-und-der-thronfolge-sein-gebet-und-tod}}

\hypertarget{a-davids-ansprache-an-die-huxe4upter-israels}{%
\paragraph{a) Davids Ansprache an die Häupter
Israels}\label{a-davids-ansprache-an-die-huxe4upter-israels}}

\hypertarget{section-27}{%
\section{28}\label{section-27}}

1David berief zu einer Versammlung nach Jerusalem alle höchststehenden
Männer der Israeliten, nämlich die Stammesfürsten, die Obersten der
Abteilungen, die im Dienst des Königs standen, die Befehlshaber der
Tausendschaften und der Hundertschaften, die Verwalter der liegenden und
beweglichen Güter des Königs und seiner Söhne, samt den Kämmerern und
den Rittern und allen übrigen hervorragenden Männern.

\hypertarget{david-stellt-den-oberen-des-volkes-salomo-als-seinen-nachfolger-vor}{%
\paragraph{David stellt den Oberen des Volkes Salomo als seinen
Nachfolger
vor}\label{david-stellt-den-oberen-des-volkes-salomo-als-seinen-nachfolger-vor}}

2Da erhob sich der König David (inmitten der Versammlung) von seinem
Sitz und hielt folgende Ansprache: »Hört mich an, meine Brüder und mein
Volk! Ich hatte selbst die Absicht, für die Lade mit dem Bundesgesetz
des HERRN und für den Schemel der Füße unsers Gottes eine
Unterkunftsstätte zu erbauen, und hatte schon Vorbereitungen für den Bau
getroffen. 3Aber da sprach Gott zu mir: ›Du sollst meinem Namen kein
Haus bauen; denn du bist ein Kriegsmann und hast Blut vergossen.‹ 4Nun
hat aber der HERR, der Gott Israels, mich aus dem ganzen Hause meines
Vaters erwählt, daß ich König für immer über Israel sein soll; denn Juda
hat er zum Fürsten ausersehen und im Stamme Juda das Haus meines Vaters;
und unter den Söhnen meines Vaters bin ich es, an dem er Wohlgefallen
gefunden, so daß er mich zum König über ganz Israel gemacht hat. 5Von
meinen sämtlichen Söhnen aber -- der HERR hat mir ja viele Söhne
geschenkt -- hat er meinen Sohn Salomo dazu ersehen, auf dem
Königsthrone des HERRN über Israel zu sitzen. 6Dabei hat er zu mir
gesagt: ›Dein Sohn Salomo, der soll mein Haus und meine Vorhöfe bauen;
denn ihn habe ich mir zum Sohn erwählt, und ich will ihm ein Vater sein;
7und ich will sein Königtum für immer fest gründen\textless sup
title=``oder: bestätigen''\textgreater✲, wenn er so, wie es heute der
Fall ist, daran festhält, meine Gebote und Weisungen zu beobachten.‹
8Und nun -- vor den Augen von ganz Israel, der Gemeinde des HERRN, und
vor den Ohren unseres Gottes ermahne ich euch: Beobachtet gewissenhaft
alle Gebote des HERRN, eures Gottes, damit ihr im Besitz dieses schönen
Landes bleibt und es später euren Nachkommen als Erbe auf ewig
hinterlaßt!«

\hypertarget{b-davids-anweisungen-und-beisteuer-an-salomo}{%
\paragraph{b) Davids Anweisungen und Beisteuer an
Salomo}\label{b-davids-anweisungen-und-beisteuer-an-salomo}}

9»Du aber, mein Sohn Salomo, erkenne den Gott deines Vaters und diene
ihm mit ungeteiltem Herzen und mit willigem Geiste; denn der HERR
erforscht alle Herzen und weiß um jeden geheimen Gedanken; wenn du ihn
suchst, wird er sich von dir finden lassen, wenn du ihn aber verläßt,
wird er dich für immer verwerfen. 10So sieh nun wohl zu! Denn der HERR
hat dich dazu ersehen, ihm ein Haus zum Heiligtum zu erbauen: mache dich
mutig ans Werk!«

\hypertarget{david-uxfcbergibt-dem-salomo-das-modell-des-tempelhauses-und-die-fuxfcr-den-bau-gesammelten-schuxe4tze}{%
\paragraph{David übergibt dem Salomo das Modell des Tempelhauses und die
für den Bau gesammelten
Schätze}\label{david-uxfcbergibt-dem-salomo-das-modell-des-tempelhauses-und-die-fuxfcr-den-bau-gesammelten-schuxe4tze}}

11Hierauf übergab David seinem Sohne Salomo das Modell\textless sup
title=``d.h. Vorbild, Muster''\textgreater✲ der Vorhalle und des
Tempelhauses, der Schatzkammern, Obergemächer und inneren Räume sowie
des Raumes für die Sühnung\textless sup title=``d.h. das
Allerheiligste''\textgreater✲; 12dazu den Plan von allem, was seinem
Geiste vorschwebte: hinsichtlich der Vorhöfe des Tempelhauses des HERRN
und der ringsum laufenden Zellen sowie hinsichtlich der Vorratskammern
des Gotteshauses und der Schatzkammern für die geweihten Gegenstände;
13ferner (die Verordnungen) in betreff der Abteilungen der Priester und
der Leviten sowie in betreff der ganzen Gestaltung der Dienstleistungen
im Tempel des HERRN und bezüglich aller für den Tempeldienst
erforderlichen Gerätschaften; 14hinsichtlich des Goldes -- nach dem
Gewicht des Goldes für jedes einzelne gottesdienstliche Gerät;
desgleichen hinsichtlich aller silbernen Geräte -- nach dem Gewicht für
jedes einzelne gottesdienstliche Gerät; 15ebenso auch das Gewicht für
die goldenen Leuchter und die zugehörigen goldenen Lampen -- nach dem
Gewicht eines jeden einzelnen Leuchters und der zugehörigen Lampen; auch
für die silbernen Leuchter -- nach dem Gewicht für jeden einzelnen
Leuchter und die zugehörigen Lampen, wie es die Bestimmung jedes
einzelnen Leuchters erforderte; 16ferner das Gold nach dem Gewicht für
die Schaubrottische, für jeden einzelnen Tisch, und das Silber für die
silbernen Tische; 17und für die Gabeln, Sprengschalen und Kannen --
gediegenes Gold; ferner für die goldenen Krüge\textless sup
title=``oder: Becher''\textgreater✲ -- nach dem Gewicht für jeden
einzelnen Krug\textless sup title=``oder: Becher''\textgreater✲ sowie
für die silbernen Krüge\textless sup title=``oder: Becher''\textgreater✲
-- nach dem Gewicht für jeden einzelnen Krug\textless sup title=``oder:
Becher''\textgreater✲; 18ferner für den Räucheraltar -- geläutertes Gold
nach dem bestimmten Gewicht; auch das Modell✲ des Wagens, nämlich der
goldenen Cherube, die mit ausgebreiteten Flügeln die Bundeslade des
HERRN überdecken sollten. 19»Über dies alles«, sagte David, »hat er mir
Anweisung durch eine von der Hand des HERRN stammende Schrift gegeben,
über alle zur Ausführung des Bauplanes erforderlichen Arbeiten.«

20Dann richtete David an seinen Sohn Salomo folgende Worte: »Sei stark
und mutig und gehe ans Werk! Fürchte dich nicht und sei unverzagt! Denn
Gott der HERR, mein Gott, wird mit dir sein: er wird dich nicht
versäumen und dich nicht verlassen, bis alle Arbeiten für den Dienst am
Hause des HERRN vollendet sind. 21Schon sind hier die Abteilungen der
Priester und der Leviten für den gesamten Dienst am Hause Gottes bereit,
und zur Verfügung stehen dir für die Ausführung des ganzen Werkes
Männer, die zu allen Verrichtungen willig und geschickt sind, dazu auch
die Fürsten und das ganze Volk für alle deine Anordnungen.«

\hypertarget{c-die-beisteuer-der-fuxfcrsten-zum-tempelbau-infolge-der-mahnung-davids}{%
\paragraph{c) Die Beisteuer der Fürsten zum Tempelbau infolge der
Mahnung
Davids}\label{c-die-beisteuer-der-fuxfcrsten-zum-tempelbau-infolge-der-mahnung-davids}}

\hypertarget{section-28}{%
\section{29}\label{section-28}}

1Hierauf redete der König David die ganze Versammlung folgendermaßen an:
»Mein Sohn Salomo, {[}der einzige,{]} den Gott erwählt hat, ist noch
jung und zart, das Werk aber gewaltig; denn nicht für einen Menschen ist
dieser Prachtbau bestimmt, sondern für Gott den HERRN. 2Daher habe ich
mit allen mir zu Gebote stehenden Kräften für das Haus meines Gottes
Gold zu den goldenen, Silber zu den silbernen, Kupfer zu den kupfernen,
Eisen zu den eisernen und Holz zu den hölzernen Geräten beschafft,
außerdem Onyxsteine und Edelsteine zu Einfassungen, Steine zu
Verzierungen, buntfarbige und kostbare Steine jeder Art und
Marmorgestein in Menge. 3Überdies will ich infolge meines Eifers für das
Haus meines Gottes das, was ich als eigenes Gut an Gold und Silber
besitze, für das Haus meines Gottes hingeben, zu allem dem noch hinzu,
was ich bereits für das heilige Haus beschafft habe, 4nämlich 3000
Talente Gold, Ophirgold, und 7000 Talente geläutertes Silber, um die
Wände der heiligen Räume damit zu überziehen 5und für alle sonstigen
Arbeiten, zu denen die Künstler Gold und Silber gebrauchen. Wer ist nun
bereit, heute gleichfalls für den HERRN eine Gabe von seinem Vermögen
beizusteuern?«

6Da bewiesen die Familienhäupter und die Fürsten der Stämme Israels
sowie die Befehlshaber der Tausendschaften und der Hundertschaften und
die obersten Beamten im Dienst des Königs ihre Bereitwilligkeit 7und
spendeten für den Bau des Gotteshauses 5000 Talente Gold und 10000
Goldstücke, 10000 Talente Silber, 18000 Talente Kupfer und 100000
Talente Eisen. 8Wer ferner Edelsteine besaß, gab sie für den Schatz des
Hauses des HERRN an den Gersoniten Jehiel ab. 9Da freute sich das Volk
über ihre Freigebigkeit; denn mit ungeteiltem Herzen hatten sie
freiwillig für den HERRN beigesteuert; auch der König David war hoch
erfreut.

\hypertarget{d-davids-schluuxdfgebet}{%
\paragraph{d) Davids Schlußgebet}\label{d-davids-schluuxdfgebet}}

10Da pries David den HERRN vor der ganzen Versammlung mit den Worten:
»Gepriesen seist du, HERR, du Gott unsers Vaters Israel, von Ewigkeit zu
Ewigkeit! 11Dein, o HERR, ist die Hoheit und die Macht, die
Herrlichkeit, der Ruhm und die Majestät; denn dein ist alles im Himmel
und auf Erden. Dein, o HERR, ist die Herrschaft, und du bist als Haupt
über alles erhaben. 12Reichtum und Ehre kommen von dir, und du bist
Herrscher über alles; in deiner Hand liegen Kraft und Stärke, und in
deiner Hand steht es, jedermann groß und stark zu machen. 13Nun denn,
unser Gott: wir danken dir und rühmen deinen herrlichen Namen; 14denn
wer bin ich, und was ist mein Volk, daß wir imstande sein sollten,
freiwillige Gaben in solcher Weise darzubringen? Nein, von dir kommt
dies alles, und aus deiner Hand haben wir dir gespendet. 15Wir sind ja
nur Gäste und Fremdlinge\textless sup title=``3.Mose
25,23''\textgreater✲ vor dir wie alle unsere Väter; wie ein Schatten
sind unsere Lebenstage auf Erden und ohne Hoffnung, (hienieden zu
bleiben). 16HERR, unser Gott, dieser ganze Reichtum, den wir
bereitgestellt haben, um dir ein Haus für deinen heiligen Namen zu
bauen, -- aus deiner Hand kommt er, und dein ist das alles! 17Ich weiß
aber, mein Gott, daß du das Herz prüfst und an Aufrichtigkeit
Wohlgefallen hast; -- nun, ich habe dies alles mit aufrichtigem Herzen
freiwillig gespendet und habe jetzt mit Freuden gesehen, daß auch dein
Volk, das hier versammelt ist, dir bereitwillig Gaben dargebracht hat.
18HERR, du Gott unserer Väter Abraham, Isaak und Israel, erhalte solche
Gesinnung und Denkweise immerdar im Herzen deines Volkes und richte ihr
Herz auf dich hin! 19Meinem Sohne Salomo aber verleihe ein
ungeteiltes\textless sup title=``oder: aufrichtiges''\textgreater✲ Herz,
daß er deine Gebote, deine Verordnungen und Weisungen beobachte und
alles tue, um den Prachtbau aufzuführen, den ich vorbereitet habe!«

\hypertarget{e-feierlicher-schluuxdf-der-versammlung-salomos-salbung-zum-kuxf6nig-ende-der-regierung-davids}{%
\paragraph{e) Feierlicher Schluß der Versammlung; Salomos Salbung zum
König; Ende der Regierung
Davids}\label{e-feierlicher-schluuxdf-der-versammlung-salomos-salbung-zum-kuxf6nig-ende-der-regierung-davids}}

20Als David hierauf die ganze Versammlung aufforderte, den HERRN, ihren
Gott, zu preisen, da priesen alle Versammelten den HERRN, den Gott ihrer
Väter, verneigten sich und warfen sich vor dem HERRN und vor dem Könige
nieder. 21Am folgenden Tage aber opferten sie dem HERRN Schlachtopfer
und brachten dem HERRN Brandopfer dar, nämlich tausend Stiere, tausend
Widder und tausend Lämmer nebst den zugehörigen Trankspenden, außerdem
Schlachtopfer in Menge für ganz Israel. 22Dann aßen und tranken sie an
jenem Tage vor dem HERRN voller Freude und machten Salomo, Davids Sohn,
zum zweitenmal zum König und salbten ihn vor dem HERRN zum Fürsten und
salbten Zadok zum Priester. 23So saß denn Salomo auf dem Throne des
HERRN als König an Stelle\textless sup title=``=~als
Nachfolger''\textgreater✲ seines Vaters David und hatte das Glück, daß
ganz Israel ihn anerkannte 24und alle Fürsten und die Ritter sowie alle
Söhne des Königs David sich dem König Salomo unterwarfen. 25Der HERR
aber ließ dann Salomo zu überhaus hohem Ansehen bei ganz Israel gelangen
und verlieh seinem Königtum einen Glanz, wie ihn vor ihm kein König über
Israel besessen hatte.

\hypertarget{davids-ende-und-die-quellen-seiner-geschichte}{%
\paragraph{Davids Ende und die Quellen seiner
Geschichte}\label{davids-ende-und-die-quellen-seiner-geschichte}}

26So hatte nun David, der Sohn Isais, über ganz Israel geherrscht, 27und
zwar betrug die Zeit seiner Regierung über Israel vierzig Jahre; in
Hebron hatte er sieben Jahre geherrscht und in Jerusalem dreiunddreißig.
28Er starb in hohem Alter, satt an Lebenstagen wie an Reichtum und
Ehren, und sein Sohn Salomo folgte ihm in der Regierung nach.

29Die Geschichte des Königs David aber findet sich bekanntlich von
Anfang bis zu Ende aufgezeichnet in der Geschichte des Sehers Samuel
sowie in der Geschichte des Propheten Nathan und in der Geschichte des
Sehers Gad, 30samt seiner ganzen machtvollen Regierung und seinen Siegen
und den Schicksalen, die ihn und Israel und alle Reiche der Länder
betroffen haben.
