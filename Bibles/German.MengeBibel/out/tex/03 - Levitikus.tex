\hypertarget{das-dritte-buch-mose}{%
\section{DAS DRITTE BUCH MOSE}\label{das-dritte-buch-mose}}

\emph{(genannt Levitikus, d.h. Priesterbuch)}

\hypertarget{die-opfergesetze-kap.-1-7}{%
\subsubsection{1. Die Opfergesetze (Kap.
1-7)}\label{die-opfergesetze-kap.-1-7}}

\hypertarget{a-vorschriften-in-betreff-der-brandopfer}{%
\paragraph{a) Vorschriften in betreff der
Brandopfer}\label{a-vorschriften-in-betreff-der-brandopfer}}

\hypertarget{section}{%
\section{1}\label{section}}

\bibleverse{1}Der HERR berief hierauf Mose und gebot ihm aus dem
Offenbarungszelte folgendes: \bibleverse{2}»Rede zu den Israeliten und
befiehl ihnen: Wenn jemand von euch dem HERRN eine Opfergabe darbringen
will, so sollt ihr eure Opfergabe vom Vieh, und zwar von den Rindern und
vom Kleinvieh, darbringen.«

\hypertarget{aa-brandopfer-vom-rindvieh}{%
\subparagraph{aa) Brandopfer vom
Rindvieh}\label{aa-brandopfer-vom-rindvieh}}

\bibleverse{3}»Wenn seine Opfergabe in einem Brandopfer bestehen soll,
und zwar von einem Rind, so muß er ein fehlerloses männliches Tier
opfern. Er bringe dieses an den Eingang des Offenbarungszeltes, damit es
ihm das Wohlgefallen des HERRN verschaffe. \bibleverse{4}Dann lege er
seine Hand fest auf den Kopf des Brandopfertieres, so wird es
wohlgefällig aufgenommen werden und ihm Sühne verschaffen.
\bibleverse{5}Hierauf schlachte man das junge Rind vor dem HERRN; und
die Priester, die Söhne Aarons, sollen das Blut herzubringen, und zwar
so, daß sie das Blut ringsum an den Altar sprengen, der am\textless sup
title=``oder: vor dem''\textgreater✲ Eingang des Offenbarungszeltes
steht. \bibleverse{6}Dann soll man dem Brandopfertier die Haut abziehen
und es in seine Stücke zerlegen; \bibleverse{7}die Söhne Aarons aber,
die Priester, sollen Feuer auf den Altar tun und Holzstücke über dem
Feuer aufschichten; \bibleverse{8}sodann sollen die Söhne Aarons, die
Priester, die Stücke, auch den Kopf und das Fett, auf dem Holz, das auf
dem Altar über dem Feuer liegt, gehörig zurechtlegen. \bibleverse{9}Die
Eingeweide und Beine des Tieres aber soll man mit Wasser waschen, und
der Priester soll dann das Ganze auf dem Altar in Rauch aufgehen lassen:
so ist es ein Brandopfer, ein Feueropfer zu\textless sup title=``oder:
von''\textgreater✲ lieblichem Geruch für den HERRN.«

\hypertarget{bb-brandopfer-vom-kleinvieh}{%
\subparagraph{bb) Brandopfer vom
Kleinvieh}\label{bb-brandopfer-vom-kleinvieh}}

\bibleverse{10}»Wenn aber seine Opfergabe, die ein Brandopfer sein soll,
in einem Stück Kleinvieh, einem Schaf oder einer Ziege, besteht, so muß
es ein männliches, fehlerloses Tier sein, das er zum Opfer bringt.
\bibleverse{11}Man schlachte dieses vor dem HERRN an der Nordseite des
Altars; die Söhne Aarons aber, die Priester, sollen das Blut des Tieres
ringsum an den Altar sprengen. \bibleverse{12}Dann soll man es in seine
Stücke zerlegen, und der Priester soll diese, und zwar auch den Kopf und
das Fett, auf dem Holz, das auf dem Altar über dem Feuer liegt, gehörig
zurechtlegen. \bibleverse{13}Die Eingeweide und Beine aber soll man mit
Wasser waschen, und der Priester soll dann das Ganze darbringen und es
auf dem Altar in Rauch aufgehen lassen: so ist es ein Brandopfer, ein
Feueropfer zu\textless sup title=``oder: von''\textgreater✲ lieblichem
Geruch für den HERRN.«

\hypertarget{cc-brandopfer-von-vuxf6geln}{%
\subparagraph{cc) Brandopfer von
Vögeln}\label{cc-brandopfer-von-vuxf6geln}}

\bibleverse{14}»Wenn aber seine Opfergabe für den HERRN, die ein
Brandopfer sein soll, in Geflügel besteht, so müssen es Turteltauben
oder junge Tauben sein, die er als Opfergabe darbringt.
\bibleverse{15}Der Priester bringe das Tier an den Altar, knicke ihm den
Kopf ab und lasse diesen auf dem Altar in Rauch aufgehen; das Blut aber
lasse er an die Wand des Altars auslaufen. \bibleverse{16}Darauf soll er
den Kropf mit seinem Unrat\textless sup title=``oder:
Gefieder?''\textgreater✲ entfernen und ihn neben den Altar gegen Osten
auf die Aschenstelle werfen. \bibleverse{17}Hierauf soll er dem Vogel
die Flügel einreißen, jedoch ohne sie ganz abzutrennen, und der Priester
soll ihn auf dem Altar, auf dem Holz, das über dem Feuer liegt, in Rauch
aufgehen lassen: so ist es ein Brandopfer, ein Feueropfer
zu\textless sup title=``oder: von''\textgreater✲ lieblichem Geruch für
den HERRN.«

\hypertarget{b-vorschriften-in-betreff-der-speisopfer}{%
\paragraph{b) Vorschriften in betreff der
Speisopfer}\label{b-vorschriften-in-betreff-der-speisopfer}}

\hypertarget{section-1}{%
\section{2}\label{section-1}}

\bibleverse{1}»Wenn aber jemand dem HERRN ein Speisopfer als Opfergabe
darbringen will, so muß seine Gabe aus Feinmehl bestehen, das er mit Öl
übergießen und zu dem er Weihrauch hinzufügen muß. \bibleverse{2}Wenn er
es dann den Söhnen Aarons, den Priestern, gebracht hat, soll der
Priester eine Handvoll davon nehmen, nämlich von dem dargebrachten
Feinmehl und Öl samt dem ganzen zugehörigen Weihrauch, und der Priester
soll den zum Duftopfer bestimmten Teil auf dem Altar in Rauch aufgehen
lassen: so ist es ein Feueropfer zu lieblichem Geruch für den HERRN.
\bibleverse{3}Was dann von dem Speisopfer noch übrig ist, soll Aaron und
seinen Söhnen gehören als etwas Hochheiliges von den Feueropfern des
HERRN.

\bibleverse{4}Willst du aber als Opfergabe eines Speisopfers etwas im
Ofen Gebackenes darbringen, so muß es aus Feinmehl bereitet sein:
ungesäuerte, mit Öl gemengte Kuchen oder ungesäuerte, mit Öl bestrichene
Fladen.~-- \bibleverse{5}Soll aber deine Opfergabe in einem Speisopfer
auf der Platte bestehen, so muß es aus ungesäuertem, mit Öl gemengtem
Feinmehl bereitet sein. \bibleverse{6}Wenn du es dann in Stücke
zerbrichst und Öl darübergießest, so ist es ein Speisopfer.~--
\bibleverse{7}Soll deine Opfergabe aber ein in der Pfanne\textless sup
title=``oder: im Topf''\textgreater✲ bereitetes Speisopfer sein, so muß
es aus Feinmehl mit Öl hergestellt sein.«

\hypertarget{aa-allgemeines-uxfcber-die-zubereitung-und-darbringung-der-speisopfer}{%
\subparagraph{aa) Allgemeines über die Zubereitung und Darbringung der
Speisopfer}\label{aa-allgemeines-uxfcber-die-zubereitung-und-darbringung-der-speisopfer}}

\bibleverse{8}»Du sollst dann das Speisopfer, das aus diesen Zutaten
bereitet ist, dem HERRN hinbringen, und zwar übergebe man es dem
Priester, damit der es an den Altar trage. \bibleverse{9}Der Priester
soll dann von dem Speisopfer den zum Duftopfer bestimmten Teil abheben
und diesen auf dem Altar in Rauch aufgehen lassen: so ist es ein
Feueropfer zu\textless sup title=``oder: von''\textgreater✲ lieblichem
Geruch für den HERRN. \bibleverse{10}Was dann von dem Speisopfer noch
übrig ist, soll Aaron und seinen Söhnen gehören als etwas Hochheiliges
von den Feueropfern des HERRN.~-- \bibleverse{11}Kein Speisopfer, das
ihr dem HERRN darbringt, darf aus Gesäuertem hergestellt sein; denn
aller Sauerteig und aller Honig -- davon dürft ihr dem HERRN kein
Feueropfer darbringen; \bibleverse{12}als Erstlingsgabe dürft ihr sie
dem HERRN darbringen, aber auf den Altar dürfen sie nicht zum lieblichen
Geruch kommen.~-- \bibleverse{13}Alle deine Gaben aber, die du als
Speisopfer darbringst, mußt du gehörig salzen und darfst niemals das
Salz des Bundes\textless sup title=``vgl. 4.Mose 18,19''\textgreater✲
deines Gottes bei deinen Speisopfern fehlen lassen: zu all deinen
Opfergaben mußt du auch Salz darbringen.«

\hypertarget{bb-speisopfer-von-getreide-erstlingen}{%
\subparagraph{bb) Speisopfer von
Getreide-Erstlingen}\label{bb-speisopfer-von-getreide-erstlingen}}

\bibleverse{14}»Willst du aber dem HERRN ein Speisopfer von den
Erstlingsfrüchten darbringen, so mußt du am Feuer geröstete Ähren (oder)
zerstoßene Körner von der frischen Frucht als Speisopfer von deinen
Erstlingsfrüchten darbringen. \bibleverse{15}Wenn du Öl daraufgießest
und Weihrauch hinzufügst, so ist es ein Speisopfer. \bibleverse{16}Der
Priester soll dann den zum Duftopfer bestimmten Teil von ihm -- von
seinen zerstoßenen Körnern und von seinem Öl -- samt dem ganzen
zugehörigen Weihrauch in Rauch aufgehen lassen: so ist es ein Feueropfer
für den HERRN.«

\hypertarget{c-vorschriften-in-betreff-der-heils-oder-dank--oder-friedens--opfer}{%
\paragraph{c) Vorschriften in betreff der Heils-(oder Dank- oder
Friedens-)
Opfer}\label{c-vorschriften-in-betreff-der-heils-oder-dank--oder-friedens--opfer}}

\hypertarget{aa-heilsopfer-vom-rindvieh}{%
\subparagraph{aa) Heilsopfer vom
Rindvieh}\label{aa-heilsopfer-vom-rindvieh}}

\hypertarget{section-2}{%
\section{3}\label{section-2}}

\bibleverse{1}»Soll aber seine Opfergabe ein Heilsopfer sein, so soll
er, wenn er sie von den Rindern darbringen will, mag es ein männliches
oder ein weibliches Tier sein, ein fehlerloses Tier vor den HERRN
bringen. \bibleverse{2}Dann lege er seine Hand fest auf den Kopf seines
Opfertieres, und man schlachte es am Eingang\textless sup title=``=~vor
der Tür''\textgreater✲ des Offenbarungszeltes; die Söhne Aarons aber,
die Priester, sollen das Blut ringsum an den Altar sprengen.
\bibleverse{3}Hierauf soll er von dem Heilsopfer dem HERRN ein
Feueropfer darbringen, nämlich das Fett, das die Eingeweide bedeckt, und
alles Fett, das an den Eingeweiden sitzt, \bibleverse{4}ferner die
beiden Nieren samt dem Fett, das an ihnen, an den Lendenmuskeln sitzt,
und den Lappen an der Leber -- bei den Nieren soll er\textless sup
title=``oder: man?''\textgreater✲ es ablösen. \bibleverse{5}Wenn die
Söhne Aarons es dann auf dem Altar über dem Brandopfer, das auf dem Holz
über dem Feuer liegt, in Rauch aufgehen lassen, so ist es ein Feueropfer
zu lieblichem Geruch für den HERRN.«

\hypertarget{bb-heilsopfer-vom-kleinvieh-schaf-oder-ziege}{%
\subparagraph{bb) Heilsopfer vom Kleinvieh (Schaf oder
Ziege)}\label{bb-heilsopfer-vom-kleinvieh-schaf-oder-ziege}}

\bibleverse{6}»Wenn aber seine Opfergabe, die er dem HERRN als
Heilsopfer darbringen will, dem Kleinvieh entnommen ist, so muß es ein
fehlerloses männliches oder weibliches Tier sein, das er opfert.
\bibleverse{7}Bringt er ein Schaf als seine Opfergabe dar, so bringe er
es vor den HERRN, \bibleverse{8}lege seine Hand fest auf den Kopf seines
Opfertieres, und man schlachte es dann vor dem Offenbarungszelt; die
Söhne Aarons aber sollen sein Blut ringsum an den Altar sprengen.
\bibleverse{9}Hierauf soll er dem HERRN von dem Heilsopfer ein
Feueropfer darbringen, nämlich das Fett des Tieres: den ganzen
Fettschwanz, den man dicht am Steißbein\textless sup title=``oder:
Schwanzwirbel''\textgreater✲ ablösen muß, ferner das Fett, das die
Eingeweide bedeckt, und alles Fett, das an den Eingeweiden sitzt,
\bibleverse{10}ferner die beiden Nieren samt dem Fett, das an ihnen, an
den Lendenmuskeln sitzt, und den Lappen an der Leber -- bei den Nieren
soll man es ablösen. \bibleverse{11}Wenn der Priester es dann auf dem
Altar in Rauch aufgehen läßt, so ist es eine Feueropferspeise für den
HERRN.

\bibleverse{12}Wenn aber seine Opfergabe in einer Ziege besteht, so soll
er sie vor den HERRN bringen, \bibleverse{13}dann seine Hand fest auf
ihren Kopf legen, und man schlachte sie dann vor dem Offenbarungszelt;
die Söhne Aarons aber sollen ihr Blut ringsum an den Altar sprengen.
\bibleverse{14}Hierauf soll er dem HERRN von ihr seine Gabe als
Feueropfer darbringen, nämlich das Fett, das die Eingeweide bedeckt, und
alles Fett, das an den Eingeweiden sitzt, \bibleverse{15}ferner die
beiden Nieren samt dem Fett, das an ihnen, an den Lendenmuskeln sitzt,
und den Lappen an der Leber -- bei den Nieren soll man es ablösen.
\bibleverse{16}Wenn der Priester es dann auf dem Altar in Rauch aufgehen
läßt, so ist es eine Feueropferspeise zu lieblichem Geruch für den
HERRN. Alles Fett gehört dem HERRN. \bibleverse{17}Das ist eine
ewiggültige Satzung für eure Geschlechter, wo ihr auch wohnen mögt:
keinerlei Fett und keinerlei Blut dürft ihr genießen!«

\hypertarget{d-vorschriften-in-betreff-der-suxfcndopfer}{%
\paragraph{d) Vorschriften in betreff der
Sündopfer}\label{d-vorschriften-in-betreff-der-suxfcndopfer}}

\hypertarget{section-3}{%
\section{4}\label{section-3}}

\bibleverse{1}Weiter gebot der HERR dem Mose folgendes:
\bibleverse{2}»Rede zu den Israeliten und befiehl ihnen: Wenn jemand
sich unvorsätzlich\textless sup title=``oder: aus
Versehen''\textgreater✲ gegen irgendein Verbot des HERRN (über Dinge)
vergeht, die nicht getan werden dürfen, und gegen irgendeins von ihnen
verstößt (so sollen folgende Bestimmungen gelten).«

\hypertarget{aa-opfer-bei-versuxfcndigung-des-hohenpriesters}{%
\subparagraph{aa) Opfer bei Versündigung des
Hohenpriesters}\label{aa-opfer-bei-versuxfcndigung-des-hohenpriesters}}

\bibleverse{3}»Wenn sich der gesalbte Priester versündigt, so daß er
eine Verschuldung über das Volk bringt, so soll er für sein Vergehen,
das er begangen hat, dem HERRN einen fehlerlosen jungen Stier als
Sündopfer darbringen. \bibleverse{4}Er hat also den Stier vor den HERRN
an den Eingang des Offenbarungszeltes zu bringen, dann seine Hand fest
auf den Kopf des Stieres zu legen und den Stier vor dem HERRN zu
schlachten. \bibleverse{5}Hierauf nehme der gesalbte Priester etwas von
dem Blut des Stieres und bringe es in das Offenbarungszelt hinein.
\bibleverse{6}Dort tauche der Priester seinen Finger in das Blut und
sprenge etwas von dem Blut siebenmal vor dem HERRN, nämlich vor den
Vorhang des Heiligtums. \bibleverse{7}Dann streiche der Priester etwas
von dem Blut an die Hörner des für das wohlriechende Räucherwerk
bestimmten Altars, der vor dem HERRN im Offenbarungszelt steht; alles
übrige Blut des jungen Stieres aber schütte er an den Fuß des
Brandopferaltars, der am Eingang des Offenbarungszeltes steht.
\bibleverse{8}Dann löse er alles Fett von dem Sündopferstier ab, nämlich
das Fett, das die Eingeweide bedeckt, und alles Fett, das an den
Eingeweiden sitzt, \bibleverse{9}ferner die beiden Nieren samt dem Fett,
das an ihnen, an den Lendenmuskeln sitzt, sowie den Lappen an der Leber
-- bei den Nieren soll er es ablösen --, \bibleverse{10}so wie es von
dem Stier bei einem Heilsopfer abgelöst wird; dann lasse der Priester
(alle Fettstücke) auf dem Brandopferaltar in Rauch aufgehen.
\bibleverse{11}Aber das Fell des Stieres und all sein Fleisch samt dem
Kopf und den Beinen sowie seine Eingeweide und seinen Gedärmeinhalt,
\bibleverse{12}also das ganze Tier soll man an einen reinen Ort
außerhalb des Lagers hinausschaffen, an den Platz, wohin man die
Fettasche schüttet, und ihn dort auf Holzscheiten im Feuer verbrennen:
an dem Ort, wohin man die Fettasche schüttet, soll er verbrannt werden.«

\hypertarget{bb-opfer-bei-versuxfcndigung-der-ganzen-gemeinde}{%
\subparagraph{bb) Opfer bei Versündigung der ganzen
Gemeinde}\label{bb-opfer-bei-versuxfcndigung-der-ganzen-gemeinde}}

\bibleverse{13}»Wenn aber die ganze Gemeinde Israel sich unvorsätzlich
verfehlt hat, ohne daß das Volk sich dessen bewußt gewesen ist, und sie
gegen irgendein Verbot des HERRN verstoßen haben und dadurch in
Verschuldung geraten sind, \bibleverse{14}so soll die Gemeinde, sobald
das Vergehen, dessen sie sich durch die Übertretung schuldig gemacht
haben, erkannt worden ist, einen jungen Stier als Sündopfer darbringen
und ihn vor das Offenbarungszelt führen. \bibleverse{15}Dort sollen die
Ältesten\textless sup title=``oder: Vornehmsten''\textgreater✲ der
Gemeinde ihre Hände vor dem HERRN fest auf den Kopf des Stieres legen,
und dann soll man den Stier vor dem HERRN schlachten.
\bibleverse{16}Hierauf bringe der gesalbte Priester etwas von dem Blut
des Stieres in das Offenbarungszelt hinein, \bibleverse{17}tauche dort
seinen Finger in das Blut und sprenge etwas von dem Blut siebenmal vor
dem HERRN, nämlich vor den Vorhang. \bibleverse{18}Dann streiche er
etwas von dem Blut an die Hörner des Altars, der vor dem HERRN im
Offenbarungszelte steht; alles übrige Blut aber schütte er an den Fuß
des Brandopferaltars, der am Eingang des Offenbarungszeltes steht.
\bibleverse{19}Dann soll er alles Fett von ihm ablösen und es auf dem
Altar in Rauch aufgehen lassen \bibleverse{20}und hierauf mit dem Stiere
so verfahren, wie er es mit dem eigenen Stier beim Sündopfer getan hat:
genau so soll er auch mit ihm verfahren. Wenn der Priester ihnen so
Sühne erwirkt hat, wird ihnen Vergebung zuteil werden.
\bibleverse{21}Den Stier aber soll man vor das Lager hinausschaffen und
ihn dort so verbrennen, wie man den zuerst erwähnten Stier verbrannt
hat. Dies ist das Sündopfer für die Gemeinde.«

\hypertarget{cc-opfer-bei-versuxfcndigung-des-stammfuxfcrsten}{%
\subparagraph{cc) Opfer bei Versündigung des
Stammfürsten}\label{cc-opfer-bei-versuxfcndigung-des-stammfuxfcrsten}}

\bibleverse{22}»Wenn ein Stammfürst sich versündigt und unvorsätzlich
irgend etwas tut, wovon der HERR, sein Gott, geboten hat, daß man es
nicht tun dürfe, und er dadurch in Verschuldung geraten ist,
\bibleverse{23}so soll er, sobald das Vergehen, dessen er sich schuldig
gemacht hat, ihm bekannt\textless sup title=``oder:
bewußt''\textgreater✲ geworden ist, einen fehlerlosen Ziegenbock als
seine Opfergabe darbringen. \bibleverse{24}Er lege dabei seine Hand fest
auf den Kopf des Bockes, und man schlachte ihn an dem Ort, wo man die
Brandopfer vor dem HERRN schlachtet: so ist es ein Sündopfer.
\bibleverse{25}Alsdann soll der Priester mit seinem Finger etwas von dem
Blut des Sündopfers nehmen und es an die Hörner des Brandopferaltars
streichen, sein übriges Blut aber an den Fuß des Brandopferaltars
schütten. \bibleverse{26}Hierauf soll er das gesamte Fett des Tieres auf
dem Altar in Rauch aufgehen lassen wie das Fett des Heilsopfers. Wenn
der Priester ihm so Sühne wegen seiner Versündigung erwirkt hat, wird
ihm Vergebung zuteil werden.«

\hypertarget{dd-opfer-bei-versuxfcndigung-eines-gewuxf6hnlichen-israeliten}{%
\subparagraph{dd) Opfer bei Versündigung eines gewöhnlichen
Israeliten}\label{dd-opfer-bei-versuxfcndigung-eines-gewuxf6hnlichen-israeliten}}

\bibleverse{27}»Wenn sich aber jemand aus dem gemeinen Volk
unvorsätzlich versündigt, indem er irgend etwas von dem tut, was nach
den Geboten des HERRN nicht getan werden darf, und dadurch in
Verschuldung geraten ist, \bibleverse{28}so soll er, sobald das
Vergehen, dessen er sich schuldig gemacht hat, ihm bekannt\textless sup
title=``oder: bewußt''\textgreater✲ geworden ist, eine fehlerlose Ziege
als Opfergabe für sein Vergehen bringen, das er sich hat zu Schulden
kommen lassen. \bibleverse{29}Er lege dann seine Hand fest auf den Kopf
des Sündopfers, und man schlachte das Sündopfer an dem für die
Brandopfer bestimmten Ort. \bibleverse{30}Darauf soll der Priester mit
seinem Finger etwas von dem Blut nehmen und es an die Hörner des
Brandopferaltars streichen, das gesamte übrige Blut des Tieres aber an
den Fuß des Altars schütten. \bibleverse{31}Hierauf soll man alles Fett
des Tieres ablösen, wie das Fett von einem Heilsopfer abgelöst wird, und
der Priester soll es auf dem Altar in Rauch aufgehen lassen zum
lieblichen Geruch für den HERRN. Wenn der Priester ihm so Sühne erwirkt
hat, wird ihm Vergebung zuteil werden.~-- \bibleverse{32}Wenn er aber
ein Schaf als seine Gabe zum Sündopfer bringen will, so muß es ein
fehlerloses weibliches Tier sein, das er bringt. \bibleverse{33}Er lege
dann seine Hand fest auf den Kopf des Sündopfers, und man schlachte es
als Sündopfer an dem Ort, wo man die Brandopfer schlachtet.
\bibleverse{34}Darauf soll der Priester mit seinem Finger etwas von dem
Blut des Sündopfers nehmen und es an die Hörner des Brandopferaltars
streichen, das gesamte übrige Blut aber an den Fuß des Altars schütten.
\bibleverse{35}Hierauf soll man alles Fett ablösen, wie das Fett des
Schafes von einem Heilsopfer abgelöst wird, und der Priester soll es auf
dem Altar, über\textless sup title=``oder: neben''\textgreater✲ den
Feueropfern des HERRN, in Rauch aufgehen lassen. Wenn der Priester ihm
so Sühne für das Vergehen, das er sich hat zu Schulden kommen lassen,
erwirkt hat, wird ihm Vergebung zuteil werden.«

\hypertarget{ee-uxfcber-einige-besondere-anluxe4sse-zu-suxfcndopfern}{%
\subparagraph{ee) Über einige besondere Anlässe zu
Sündopfern}\label{ee-uxfcber-einige-besondere-anluxe4sse-zu-suxfcndopfern}}

\hypertarget{section-4}{%
\section{5}\label{section-4}}

\bibleverse{1}»Wenn jemand sich dadurch vergeht, daß er nach Anhörung
der gerichtlichen Verfluchung (über einen Verbrecher), obgleich er als
Zeuge auftreten könnte, weil er entweder die Tat gesehen oder die Sache
sonst in Erfahrung gebracht hat, trotzdem keine Aussage\textless sup
title=``oder: Anzeige''\textgreater✲ macht und so in Verschuldung
gerät;~-- \bibleverse{2}oder wenn jemand irgend etwas Unreines berührt,
sei es das Aas eines unreinen wilden Tieres oder das Aas eines unreinen
Haustieres oder das Aas eines unreinen Kriechtieres, ohne sich dessen
(zunächst) bewußt zu sein, aber doch so, daß er unrein geworden ist und
sich dessen bewußt wird;~-- \bibleverse{3}oder wenn er mit der
Unreinigkeit eines Menschen in Berührung kommt, mit irgendeiner
Unreinigkeit, durch die man sich verunreinigen kann, ohne daß er es
(zunächst) weiß, nachher aber Kenntnis davon erhält und er sich schuldig
fühlt;~-- \bibleverse{4}oder wenn jemand unbesonnen schwört, indem der
Schwur seinen Lippen entfährt, daß er etwas Gutes oder Böses tun wolle,
wie ja jemandem ein Schwur unbesonnenerweise entfahren mag, ohne daß er
sich dessen (zunächst) bewußt ist, nachher aber zur Erkenntnis kommt und
so in bezug auf irgend etwas Derartiges sich schuldig fühlt:
\bibleverse{5}so soll er, wenn er durch irgend etwas Derartiges eine
Schuld auf sich geladen hat, das Vergehen, dessen er sich schuldig
gemacht hat, bekennen \bibleverse{6}und dann dem HERRN als Buße für das
Vergehen, das er sich hat zu Schulden kommen lassen, ein weibliches
Stück Kleinvieh, ein Schaf oder eine Ziege, als Sündopfer darbringen;
und der Priester soll ihm dadurch Sühne für sein Vergehen erwirken.«

\hypertarget{ff-ersatz-des-regelrechten-suxfcndopfers-fuxfcr-unbemittelte-und-fuxfcr-ganz-arme-fortsetzung-zu-435}{%
\subparagraph{ff) Ersatz des regelrechten Sündopfers für Unbemittelte
und für ganz Arme (Fortsetzung zu
4,35)}\label{ff-ersatz-des-regelrechten-suxfcndopfers-fuxfcr-unbemittelte-und-fuxfcr-ganz-arme-fortsetzung-zu-435}}

\bibleverse{7}»Wenn aber sein Vermögen zur Beschaffung eines Stückes
Kleinvieh nicht ausreicht, so bringe er als seine Buße für das, wodurch
er sich vergangen hat, dem HERRN zwei Turteltauben oder zwei junge
Tauben dar, die eine zum Sündopfer, die andere zum Brandopfer.
\bibleverse{8}Er bringe sie also zum Priester, und dieser soll die zum
Sündopfer bestimmte zuerst darbringen, und zwar so, daß er ihr den Kopf
dicht beim Genick abknickt, doch ohne ihn ganz abzutrennen.
\bibleverse{9}Dann soll er etwas von dem Blut des Sündopfers an die Wand
des Altars sprengen, das übrige Blut aber an den Fuß des Altars
ausdrücken\textless sup title=``=~auslaufen lassen''\textgreater✲: so
ist es ein Sündopfer. \bibleverse{10}Die andere Taube aber soll er in
der vorgeschriebenen\textless sup title=``oder: gehörigen''\textgreater✲
Weise zum Brandopfer herrichten. Wenn der Priester ihm so Sühne für das
Vergehen, das er sich hat zu Schulden kommen lassen, erwirkt hat, wird
ihm Vergebung zuteil werden.~-- \bibleverse{11}Wenn aber sein Vermögen
nicht einmal zur Beschaffung zweier Turteltauben oder zweier jungen
Tauben ausreicht, so bringe er als seine Opfergabe für das, wodurch er
sich vergangen hat, ein Zehntel Epha Feinmehl als Sündopfer dar, ohne
jedoch Öl daraufzugießen oder Weihrauch hinzuzufügen; denn es ist ein
Sündopfer. \bibleverse{12}Er bringe es also zum Priester, und der
Priester nehme eine Handvoll davon als den zum Duftopfer bestimmten Teil
und lasse es auf dem Altar über den Feueropfern des HERRN in Rauch
aufgehen: so ist es ein Sündopfer. \bibleverse{13}Wenn der Priester ihm
so für sein Vergehen, das er sich in irgendeinem der ebengenannten Fälle
hat zu Schulden kommen lassen, Sühne erwirkt hat, wird ihm Vergebung
zuteil werden. (Was dann von dem Sündopfer noch übrig ist) soll dem
Priester gehören wie das (gewöhnliche) Speisopfer.«

\hypertarget{e-vorschriften-in-betreff-der-schuldopfer}{%
\paragraph{e) Vorschriften in betreff der
Schuldopfer}\label{e-vorschriften-in-betreff-der-schuldopfer}}

\hypertarget{aa-bei-veruntreuung-heiliger-abgaben}{%
\subparagraph{aa) Bei Veruntreuung heiliger
Abgaben}\label{aa-bei-veruntreuung-heiliger-abgaben}}

\bibleverse{14}Weiter gebot der HERR dem Mose folgendes:
\bibleverse{15}»Wenn jemand eine Veruntreuung begeht, daß er sich
unvorsätzlich an Dingen vergreift, die dem HERRN geheiligt\textless sup
title=``oder: geweiht''\textgreater✲ sind, so soll er dem HERRN als sein
Schuldopfer\textless sup title=``=~seine Buße''\textgreater✲ einen
fehlerlosen Widder von seinem Kleinvieh, der nach deiner Schätzung
mindestens zwei Schekel Silber nach dem Gewicht des
Heiligtums\textless sup title=``vgl. 2.Mose 30,13''\textgreater✲ wert
ist, als Schuldopfer darbringen. \bibleverse{16}Außerdem soll er den
Betrag, um den er das Heiligtum freventlich geschädigt hat, erstatten
und noch ein Fünftel des Betrags dazulegen und es dem Priester
übergeben. Wenn der Priester ihm dann durch den als Schuldopfer
dargebrachten Widder Sühne erwirkt hat, wird ihm Vergebung zuteil
werden.«

\hypertarget{bb-bei-unbewuuxdfter-verschuldung}{%
\subparagraph{bb) Bei unbewußter
Verschuldung}\label{bb-bei-unbewuuxdfter-verschuldung}}

\bibleverse{17}»Wenn sich aber jemand vergeht, indem er unwissentlich
irgend etwas tut, was man nach den Geboten des HERRN nicht tun darf, und
er unbewußt in Schuld geraten ist und ein Unrecht auf sich geladen hat,
\bibleverse{18}so soll er einen fehlerlosen Widder von seinem Kleinvieh
nach deiner Schätzung als Schuldopfer zum Priester bringen. Wenn der
Priester ihm dann für sein Vergehen, das er unwissentlich begangen hat,
Sühne erwirkt hat, so wird ihm Vergebung zuteil werden.
\bibleverse{19}Es ist ein Schuldopfer; er hat sich ja doch gegen den
HERRN verschuldet.«

\hypertarget{cc-bei-schuxe4digung-des-eigentums-eines-mitmenschen}{%
\subparagraph{cc) Bei Schädigung des Eigentums eines
Mitmenschen}\label{cc-bei-schuxe4digung-des-eigentums-eines-mitmenschen}}

\bibleverse{20}Weiter gebot der HERR dem Mose folgendes:
\bibleverse{21}»Wenn jemand sich versündigt und sich eine Veruntreuung
gegen den HERRN zu Schulden kommen läßt, indem er seinem Volksgenossen
gegenüber etwas Anvertrautes oder Hinterlegtes oder Entwendetes
ableugnet oder seinen Volksgenossen um etwas übervorteilt
\bibleverse{22}oder etwas Verlorenes gefunden hat und es verhehlt, oder
wenn er falsch schwört in bezug auf irgendeine Handlung, durch die sich
jemand versündigen kann~-- \bibleverse{23}wenn er sich also auf solche
Weise vergangen hat und in Verschuldung geraten ist, so soll er das
Entwendete, das er an sich gebracht, oder das Erpreßte, das er sich mit
Unrecht angeeignet hat, oder das Anvertraute, das ihm in Verwahrung
gegeben worden ist, oder das Verlorene, das er gefunden hat, zurückgeben
\bibleverse{24}oder alles, in bezug worauf er falsch geschworen hat,
zurückerstatten, und zwar soll er es nach seinem vollen Wert erstatten
und noch ein Fünftel des Betrags dazulegen: wem es zukommt, dem soll er
es erstatten am Tage, an dem er sein Schuldopfer darbringt.
\bibleverse{25}Als seine Buße für den HERRN aber soll er einen
fehlerlosen Widder von seinem Kleinvieh nach deiner Schätzung als
Schuldopfer zum Priester bringen. \bibleverse{26}Wenn der Priester ihm
dann Sühne vor dem HERRN erwirkt hat, wird ihm Vergebung zuteil werden
für alle Handlungen, durch deren Begehung er sich eine Verschuldung
zugezogen hat.«

\hypertarget{f-weitere-opfervorschriften-besonders-uxfcber-pflichten-und-anteile-der-priester}{%
\paragraph{f) Weitere Opfervorschriften, besonders über Pflichten und
Anteile der
Priester}\label{f-weitere-opfervorschriften-besonders-uxfcber-pflichten-und-anteile-der-priester}}

\hypertarget{aa-vorschriften-fuxfcr-die-priester-in-betreff-des-tuxe4glichen-brandopfers}{%
\subparagraph{aa) Vorschriften für die Priester in betreff des täglichen
Brandopfers}\label{aa-vorschriften-fuxfcr-die-priester-in-betreff-des-tuxe4glichen-brandopfers}}

\hypertarget{section-5}{%
\section{6}\label{section-5}}

\bibleverse{1}Weiter gebot der HERR dem Mose folgendes:
\bibleverse{2}»Gib Aaron und seinen Söhnen folgende Weisungen: Diese
Vorschriften gelten für das Brandopfer: Dieses, das Brandopfer, soll auf
dem Altar da, wo es angezündet worden ist, die ganze Nacht hindurch bis
zum Morgen verbleiben, und das Altarfeuer soll dadurch brennend erhalten
werden. \bibleverse{3}Der Priester soll sein linnenes Gewand anziehen
und sich die linnenen Beinkleider an den Leib legen und die Fettasche,
in welche das Feuer das Brandopfer auf dem Altar verbrannt hat,
wegräumen und sie neben den Altar schütten. \bibleverse{4}Alsdann soll
er seine (linnenen) Kleider ausziehen und andere (gewöhnliche) Kleider
anlegen und die Asche an einen reinen Ort vor das Lager hinausschaffen.
\bibleverse{5}Das Feuer aber auf dem Altar soll dadurch\textless sup
title=``vgl. V.2''\textgreater✲ in Brand erhalten werden und darf nicht
erlöschen: jeden Morgen soll der Priester Holzscheite auf dem Altar
anzünden und das (Morgen-) Brandopfer auf ihm zurechtlegen und über ihm
die Fettstücke der Heilsopfer in Rauch aufgehen lassen.
\bibleverse{6}Ein beständiges Feuer soll auf dem Altar unterhalten
werden; es darf nie erlöschen!«

\hypertarget{bb-vorschriften-fuxfcr-die-priester-in-betreff-des-tuxe4glichen-speisopfers}{%
\subparagraph{bb) Vorschriften für die Priester in betreff des
(täglichen?)
Speisopfers}\label{bb-vorschriften-fuxfcr-die-priester-in-betreff-des-tuxe4glichen-speisopfers}}

\bibleverse{7}»Folgende Vorschriften aber gelten für das Speisopfer: Die
Söhne Aarons sollen es vor den HERRN an die Vorderseite des Altars
heranbringen. \bibleverse{8}Dann nehme er\textless sup title=``d.h.
einer von ihnen''\textgreater✲ eine Handvoll davon, nämlich von dem
Feinmehl des Speisopfers und von dessen Öl, dazu den gesamten Weihrauch,
der auf dem Speisopfer liegt, und lasse es auf dem Altar in Rauch
aufgehen als ein Feueropfer lieblichen Geruchs, als den Duftteil davon
für den HERRN. \bibleverse{9}Was dann (von dem Speisopfer) noch übrig
ist, sollen Aaron und seine Söhne essen; ungesäuert soll es an heiliger
Stätte verzehrt werden; im Vorhof des Offenbarungszeltes sollen sie es
essen. \bibleverse{10}Es darf nicht mit Sauerteig gebacken werden; es
ist ihr Anteil, den ich ihnen von meinen Feueropfern zugewiesen habe:
hochheilig ist es wie das Sündopfer und wie das Schuldopfer.
\bibleverse{11}Alle männlichen Personen unter den Nachkommen Aarons
dürfen es genießen; eine für ewige Zeiten festgesetzte Gebühr von den
Feueropfern des HERRN soll es von Geschlecht zu Geschlecht sein. Jeder
(Unbefugte), der diese Dinge berührt, soll dem Heiligtum verfallen
sein.«

\hypertarget{cc-vorschriften-in-betreff-des-speisopfers-des-hohenpriesters}{%
\subparagraph{cc) Vorschriften in betreff des Speisopfers des
Hohenpriesters}\label{cc-vorschriften-in-betreff-des-speisopfers-des-hohenpriesters}}

\bibleverse{12}Weiter gebot der HERR dem Mose folgendes:
\bibleverse{13}»Dies soll die Opfergabe Aarons und seiner Söhne sein,
die sie dem HERRN darzubringen haben an dem Tage, an welchem
er\textless sup title=``d.h. einer von ihnen''\textgreater✲ gesalbt
wird: ein Zehntel Epha Feinmehl als regelmäßiges Speisopfer, die eine
Hälfte davon am Morgen, die andere Hälfte am Abend. \bibleverse{14}Auf
einer Platte\textless sup title=``oder: in der Pfanne''\textgreater✲
soll es mit Öl zubereitet werden; zusammengerührt\textless sup
title=``oder: wohlvermengt''\textgreater✲ sollst du es darbringen;
zerbröckelt zu einem Brockenspeisopfer sollst du es darbringen als einen
lieblichen Geruch für den HERRN. \bibleverse{15}Auch der Priester, der
an Aarons Statt aus der Zahl seiner Söhne gesalbt ist\textless sup
title=``oder: wird''\textgreater✲, soll es als eine auf ewige Zeiten
festgesetzte Gebühr für den HERRN herrichten: als Ganzopfer soll es
verbrannt werden; \bibleverse{16}denn jedes Speisopfer eines Priesters
soll ein Ganzopfer sein: es darf nichts davon gegessen werden.«

\hypertarget{dd-vorschriften-besonders-fuxfcr-die-priester-in-betreff-des-suxfcndopfers}{%
\subparagraph{dd) Vorschriften besonders für die Priester in betreff des
Sündopfers}\label{dd-vorschriften-besonders-fuxfcr-die-priester-in-betreff-des-suxfcndopfers}}

\bibleverse{17}Weiter gebot der HERR dem Mose folgendes:
\bibleverse{18}»Rede mit Aaron und seinen Söhnen und gebiete ihnen:
Folgende Vorschriften gelten für das Sündopfer: An dem Orte, wo das
Brandopfer geschlachtet wird, soll auch das Sündopfer vor dem HERRN
geschlachtet werden: es ist hochheilig. \bibleverse{19}Der Priester, der
das Sündopfer darbringt, soll es verzehren; an heiliger Stätte soll es
gegessen werden, nämlich im Vorhof des Offenbarungszeltes.
\bibleverse{20}Jeder (Unbefugte), der sein Fleisch berührt, soll dem
Heiligtum verfallen sein; und wenn etwas von seinem Blut an ein Kleid
spritzt, so mußt du das Kleidungsstück, an das es gespritzt ist, an
heiliger Stätte waschen. \bibleverse{21}Ein irdenes Gefäß, in dem man es
gekocht hat, muß zerbrochen werden; ist es aber in einem kupfernen Gefäß
gekocht worden, so muß dieses gescheuert und mit Wasser ausgespült
werden. \bibleverse{22}Alle männlichen Personen der Priesterschaft
dürfen davon essen: es ist hochheilig. \bibleverse{23}Aber von allen
Sündopfern, von deren Blut ein Teil in das Offenbarungszelt gebracht
worden ist, (damit) im Heiligtum die Sühnung zu vollziehen, darf nichts
gegessen werden, sondern sie sind im Feuer zu verbrennen.«

\hypertarget{ee-vorschriften-in-betreff-des-schuldopfers}{%
\subparagraph{ee) Vorschriften in betreff des
Schuldopfers}\label{ee-vorschriften-in-betreff-des-schuldopfers}}

\hypertarget{section-6}{%
\section{7}\label{section-6}}

\bibleverse{1}»Folgende Vorschriften gelten für das Schuldopfer: es ist
hochheilig. \bibleverse{2}An dem Orte, wo man das Brandopfer schlachtet,
soll man auch das Schuldopfer schlachten, und sein Blut soll man ringsum
an den Altar sprengen; \bibleverse{3}das gesamte Fett des Tieres aber
soll man darbringen, nämlich den Fettschwanz und das Fett, das die
Eingeweide bedeckt, \bibleverse{4}ferner die beiden Nieren samt dem
Fett, das an ihnen, an den Lendenmuskeln sitzt, und den Lappen an der
Leber -- bei den Nieren soll man es ablösen. \bibleverse{5}Dies alles
soll der Priester dann auf dem Altar als ein Feueropfer für den HERRN in
Rauch aufgehen lassen: so ist es ein Schuldopfer. \bibleverse{6}Alle
männlichen Personen der Priesterschaft dürfen davon essen; an heiliger
Stätte muß es gegessen werden: es ist hochheilig. \bibleverse{7}Wie mit
dem Sündopfer, so soll auch mit dem Schuldopfer verfahren werden;
dieselbe Bestimmung soll für beide gelten: es soll dem Priester gehören,
der die Sühnung mit ihm vollzogen hat.

\hypertarget{ff-anhang-uxfcber-den-anteil-des-priesters-an-privat-brandopfern-und-privat-speisopfern}{%
\subparagraph{ff) Anhang über den Anteil des Priesters an
Privat-Brandopfern und
Privat-Speisopfern}\label{ff-anhang-uxfcber-den-anteil-des-priesters-an-privat-brandopfern-und-privat-speisopfern}}

\bibleverse{8}Ebenso soll dem Priester, der jemandes Brandopfer
darbringt, das Fell des Brandopfertieres gehören, das er dargebracht
hat. \bibleverse{9}Ebenso soll jedes Speisopfer, das im Ofen gebacken,
sowie alles, was in der Pfanne oder auf der Platte\textless sup
title=``vgl. 2,7''\textgreater✲ zubereitet worden ist, dem Priester
gehören, der es darbringt. \bibleverse{10}Aber jedes Speisopfer, das mit
Öl gemengt oder trocken ist, soll allen Söhnen Aarons gehören, dem einen
wie dem andern.«

\hypertarget{gg-vorschriften-fuxfcr-verschiedene-arten-von-heilsopfern}{%
\subparagraph{gg) Vorschriften für verschiedene Arten von
Heilsopfern}\label{gg-vorschriften-fuxfcr-verschiedene-arten-von-heilsopfern}}

\bibleverse{11}»Folgende Vorschriften gelten für das Heilsopfer, das man
dem HERRN darbringt: \bibleverse{12}Wenn jemand es zum Zweck der
Danksagung darbringt, so soll er außer dem Dank-Schlachtopfer noch
ungesäuerte, mit Öl gemengte Kuchen und ungesäuerte, mit Öl bestrichene
Fladen und mit Öl eingerührtes Feinmehl darbringen; \bibleverse{13}nebst
Kuchen von gesäuertem Brotteig soll er seine Opfergabe darbringen, außer
dem Opfertier, in welchem sein Dank-Heilsopfer besteht.
\bibleverse{14}Und zwar soll er davon je ein Stück von jeder Opfergabe
als Hebe für den HERRN darbringen; (was dann noch übrig ist) soll dem
Priester gehören, der das Blut des Heilsopfers an den Altar gesprengt
hat. \bibleverse{15}Das Fleisch des Tieres aber, das als Dank-Heilsopfer
geschlachtet wird, muß noch am Tage seiner Darbringung gegessen werden:
man darf nichts davon bis zum nächsten Morgen übriglassen.~--
\bibleverse{16}Wenn aber die Opfergabe eines Schlachttieres infolge
eines Gelübdes dargebracht wird oder eine freiwillige Leistung ist, so
soll zwar das Fleisch am Tage der Darbringung des Schlachtopfers
gegessen werden; jedoch darf das, was etwa davon übriggeblieben ist,
auch noch am folgenden Tage gegessen werden. \bibleverse{17}Was aber
dann noch vom Fleisch des Opfertieres am dritten Tage übrig ist, muß im
Feuer verbrannt werden; \bibleverse{18}denn wenn man vom Fleisch seines
Heilsopfers noch am dritten Tage äße, so würde dies nicht
gottwohlgefällig sein; es würde dem, der es dargebracht hat, nicht
zugerechnet werden, sondern als Greuel\textless sup
title=``=~verdorbenes Fleisch''\textgreater✲ gelten, und jeder, der
davon äße, würde eine Verschuldung auf sich laden.~--
\bibleverse{19}Auch solches Fleisch, das mit etwas Unreinem in Berührung
gekommen ist, darf nicht gegessen werden, sondern ist im Feuer zu
verbrennen. Was sonst aber das Opferfleisch betrifft, so darf jeder
Reine es genießen; \bibleverse{20}aber ein Mensch, der im Zustand der
Unreinheit Fleisch von einem Heilsopfer genießt, das dem HERRN gehört,
die Seele\textless sup title=``=~das Leben''\textgreater✲ eines solchen
Menschen soll aus seinen Volksgenossen ausgerottet werden.
\bibleverse{21}Wenn ferner jemand mit irgend etwas Unreinem in Berührung
gekommen ist, sei es mit einem unreinen Menschen oder mit einem unreinen
Haustier oder mit irgendeinem unreinen Gewürm, und trotzdem von dem
Fleisch eines Heilsopfers ißt, das dem HERRN gehört, ein solcher
Mensch\textless sup title=``vgl. V.20''\textgreater✲ soll aus seinen
Volksgenossen ausgerottet werden.«

\hypertarget{hh-verbot-des-genusses-von-fett-und-blut}{%
\subparagraph{hh) Verbot des Genusses von Fett und
Blut}\label{hh-verbot-des-genusses-von-fett-und-blut}}

\bibleverse{22}Weiter gebot der HERR dem Mose folgendes:
\bibleverse{23}»Gib den Israeliten folgende Weisung: Keinerlei Fett von
Rindern, Schafen und Ziegen dürft ihr genießen! \bibleverse{24}Das Fett
von verendeten oder zerrissenen Tieren darf zwar zu beliebigen Zwecken
verwendet werden, aber genießen dürft ihr es nimmermehr;
\bibleverse{25}denn jeder, der Fett von den Haustieren genießt, von
denen man dem HERRN Feueropfer darbringt, dessen Seele✲ soll, weil er es
genossen hat, aus seinen Volksgenossen ausgerottet werden.
\bibleverse{26}Ebenso dürft ihr, wo ihr auch wohnen mögt, kein Blut
genießen, weder von Vögeln noch von vierfüßigen Tieren.
\bibleverse{27}Ein jeder, der irgendwelches Blut genießt, dessen Seele
soll aus seinen Volksgenossen ausgerottet werden.«

\hypertarget{ii-bestimmungen-uxfcber-die-opferanteile-der-priester-bei-den-heilsopfern}{%
\subparagraph{ii) Bestimmungen über die Opferanteile der Priester bei
den
Heilsopfern}\label{ii-bestimmungen-uxfcber-die-opferanteile-der-priester-bei-den-heilsopfern}}

\bibleverse{28}Weiter gebot der HERR dem Mose: \bibleverse{29}»Gib den
Israeliten folgende Weisungen: Wer dem HERRN ein Heilsopfer darbringt,
soll dem HERRN von seinem Heilsopfer den gebührenden Anteil zukommen
lassen. \bibleverse{30}Mit eigenen Händen soll er die zum Feueropfer für
den HERRN bestimmten Stücke herbeibringen; nämlich das Fett samt der
Brust soll er herbeibringen, und zwar die Brust, damit sie als
Webeopfer✲ vor dem HERRN gewebt✲ werde. \bibleverse{31}Das Fett soll der
Priester dann auf dem Altar in Rauch aufgehen lassen, die Brust aber
soll Aaron und seinen Söhnen gehören. \bibleverse{32}Auch die rechte
Keule sollt ihr als Hebe dem Priester von euren Heilsopfern geben.
\bibleverse{33}Wer von den Söhnen Aarons das Blut und das Fett der
Heilsopfer darbringt, dem soll die rechte Keule als Anteil gehören.
\bibleverse{34}Denn die Webebrust und die Hebekeule habe ich von den
Israeliten als meinen Anteil an ihren Heilsopfern genommen und habe sie
dem Priester Aaron und seinen Söhnen als eine von seiten der Israeliten
ewig zu leistende Gebühr überwiesen.«

\hypertarget{jj-schluuxdfbemerkungen}{%
\subparagraph{jj) Schlußbemerkungen}\label{jj-schluuxdfbemerkungen}}

\bibleverse{35}Dies ist der Anteil Aarons und seiner Söhne an den
Feueropfern des HERRN, der ihnen überwiesen worden ist an dem Tage, als
der HERR sie zu sich herantreten ließ, damit sie ihm als Priester
dienten. \bibleverse{36}Diesen Anteil hat der HERR ihnen als eine von
seiten der Israeliten zu leistende Gebühr an dem Tage, als er sie
salbte, überwiesen; es ist eine ewige, für ihre Geschlechter
verbindliche Gebühr.

\bibleverse{37}Dies sind die Vorschriften in betreff des Brandopfers,
des Speisopfers, des Sündopfers, des Schuldopfers, des Einweihungsopfers
und des Heilsopfers, \bibleverse{38}wie sie der HERR dem Mose auf dem
Berge Sinai geboten hat an dem Tage, als er den Israeliten gebot, ihre
Opfergaben dem HERRN darzubringen, in der Wüste Sinai.

\hypertarget{einkleidung-weihe-und-amtsantritt-der-priester-kap.-8-10}{%
\subsubsection{2. Einkleidung, Weihe und Amtsantritt der Priester (Kap.
8-10)}\label{einkleidung-weihe-und-amtsantritt-der-priester-kap.-8-10}}

\hypertarget{a-weihe-aarons-und-seiner-vier-suxf6hne}{%
\paragraph{a) Weihe Aarons und seiner vier
Söhne}\label{a-weihe-aarons-und-seiner-vier-suxf6hne}}

\hypertarget{section-7}{%
\section{8}\label{section-7}}

\bibleverse{1}Hierauf gebot der HERR dem Mose folgendes:
\bibleverse{2}»Nimm Aaron und seine Söhne mit ihm, dazu die heiligen
Kleider und das Salböl, ferner den jungen Stier zum Sündopfer, die
beiden Widder sowie den Korb mit dem ungesäuerten Backwerk,
\bibleverse{3}und versammle die ganze Gemeinde am Eingang des
Offenbarungszeltes!« \bibleverse{4}Mose tat, wie der HERR ihm geboten
hatte; und als die Gemeinde sich am Eingang des Offenbarungszeltes
versammelt hatte, \bibleverse{5}sagte Mose zu der Gemeinde: »Dies ist
es, was der HERR zu tun geboten hat.«

\hypertarget{aa-waschung-einkleidung-und-salbung-der-priester}{%
\subparagraph{aa) Waschung, Einkleidung und Salbung der
Priester}\label{aa-waschung-einkleidung-und-salbung-der-priester}}

\bibleverse{6}Hierauf ließ Mose Aaron mit seinen Söhnen herantreten und
nahm eine Waschung mit Wasser an ihnen vor; \bibleverse{7}dann ließ er
ihn das Unterkleid anlegen, umgürtete ihn mit dem Gürtel, bekleidete ihn
mit dem Obergewand, legte ihm das Schulterkleid darüber an, umgürtete
ihn mit der Binde des Schulterkleides und legte es ihm vermittels dieser
fest an. \bibleverse{8}Dann befestigte er auf demselben das
Brustschild\textless sup title=``=~die Tasche''\textgreater✲ und tat die
heiligen Lose Urim und Thummim\textless sup title=``vgl. 2.Mose
28,30''\textgreater✲ in das Brustschild hinein. \bibleverse{9}Hierauf
setzte er ihm den Kopfbund auf das Haupt und befestigte an der
Vorderseite des Kopfbundes das goldene Stirnblatt, das heilige Diadem✲,
wie der HERR dem Mose geboten hatte. \bibleverse{10}Dann nahm Mose das
Salböl, salbte die heilige Wohnung und alles, was sich in ihr befand,
und heiligte sie so; \bibleverse{11}auch sprengte er etwas davon
siebenmal auf den Altar und salbte den Altar nebst allen seinen Geräten,
auch das Becken samt seinem Gestell, um sie dadurch zu weihen.
\bibleverse{12}Hierauf goß er dem Aaron etwas von dem Salböl auf das
Haupt und salbte ihn, um ihn dadurch zu weihen. \bibleverse{13}Dann ließ
Mose die Söhne Aarons herantreten, ließ sie die Unterkleider anziehen,
umgürtete sie mit den Gürteln und setzte ihnen die hohen Mützen auf, wie
der HERR dem Mose geboten hatte.

\hypertarget{bb-das-priestersuxfcndopfer}{%
\subparagraph{bb) Das
Priestersündopfer}\label{bb-das-priestersuxfcndopfer}}

\bibleverse{14}Dann ließ er den jungen Stier zum Sündopfer heranführen,
und Aaron und seine Söhne legten ihre Hände fest auf den Kopf des
Sündopferstieres. \bibleverse{15}Hierauf schlachtete man ihn, Mose nahm
das Blut, strich etwas davon mit seinem Finger an die Hörner des Altars
ringsum und entsündigte so den Altar; das (übrige) Blut aber goß er an
den Fuß des Altars und heiligte ihn so, indem er die Sühngebräuche an
ihm vollzog. \bibleverse{16}Dann nahm man das ganze Fett, das an den
Eingeweiden saß, sowie den Lappen an der Leber und die beiden Nieren
samt dem Fett daran, und Mose ließ es auf dem Altar in Rauch aufgehen;
\bibleverse{17}den Stier aber samt seinem Fell, seinem Fleisch und dem
Inhalt seiner Gedärme verbrannte man in einem Feuer außerhalb des
Lagers, wie der HERR dem Mose geboten hatte.

\hypertarget{cc-das-brandopfer}{%
\subparagraph{cc) Das Brandopfer}\label{cc-das-brandopfer}}

\bibleverse{18}Dann ließ er den Widder zum Brandopfer herbeibringen, und
Aaron und seine Söhne legten ihre Hände fest auf den Kopf des Widders.
\bibleverse{19}Dann schlachtete man ihn, und Mose sprengte das Blut
ringsum an den Altar; \bibleverse{20}den Widder aber zerlegte man in
seine Stücke, und Mose ließ den Kopf sowie die Fleischstücke und das
Fett in Rauch aufgehen. \bibleverse{21}Nachdem man dann die Eingeweide
und die Beine mit Wasser abgewaschen hatte, ließ Mose den ganzen Widder
auf dem Altar in Rauch aufgehen: so war es ein Brandopfer
zu\textless sup title=``oder: von''\textgreater✲ lieblichem Geruch, ein
Feueropfer für den HERRN, wie der HERR dem Mose geboten hatte.

\hypertarget{dd-einweihungsopfer-und-besprengung}{%
\subparagraph{dd) Einweihungsopfer und
Besprengung}\label{dd-einweihungsopfer-und-besprengung}}

\bibleverse{22}Hierauf ließ er den zweiten Widder, den
Einweihungswidder, herbeibringen, und Aaron und seine Söhne legten ihre
Hände fest auf den Kopf des Widders. \bibleverse{23}Dann schlachtete man
ihn, und Mose nahm etwas von seinem Blut und strich es an das rechte
Ohrläppchen Aarons sowie an den Daumen seiner rechten Hand und an die
große Zehe seines rechten Fußes. \bibleverse{24}Dann ließ Mose die Söhne
Aarons herantreten und strich etwas von dem Blut an ihr rechtes
Ohrläppchen sowie an ihren rechten Daumen und an die große Zehe ihres
rechten Fußes; das (übrige) Blut aber sprengte Mose ringsum an den
Altar. \bibleverse{25}Hierauf nahm er das Fett, nämlich den Fettschwanz
und alles Fett, das an den Eingeweiden saß, ferner den Lappen an der
Leber, sowie die beiden Nieren samt ihrem Fett und die rechte Keule;
\bibleverse{26}weiter nahm er aus dem Korbe mit dem ungesäuerten
Backwerk, der vor dem HERRN stand, einen ungesäuerten Kuchen sowie einen
mit Öl zubereiteten Brotkuchen und einen Fladen, legte dies auf die
Fettstücke und auf die rechte Keule, \bibleverse{27}gab dann dies alles
dem Aaron und seinen Söhnen in die Hände und ließ es als Webeopfer vor
dem HERRN weben✲. \bibleverse{28}Dann nahm Mose ihnen alles wieder aus
den Händen zurück und ließ es auf dem Altar über dem Brandopfer in Rauch
aufgehen; so war es ein Einweihungsopfer zu lieblichem Geruch, ein
Feueropfer für den HERRN. \bibleverse{29}Dann nahm Mose die Brust und
webte✲ sie als Webeopfer\textless sup title=``vgl. V.27''\textgreater✲
vor dem HERRN; von dem Einweihungswidder fiel sie dem Mose als Anteil
zu, wie der HERR dem Mose geboten hatte. \bibleverse{30}Hierauf nahm
Mose etwas von dem Salböl und von dem Blut, das sich auf dem Altar
befand, und besprengte damit Aaron und seine Kleider sowie dessen Söhne
und deren Kleider und weihte so Aaron und seine Kleider sowie dessen
Söhne mit ihm nebst deren Kleidern.

\hypertarget{ee-vorschriften-in-betreff-des-opfermahles-und-der-siebentuxe4gigen-absonderung}{%
\subparagraph{ee) Vorschriften in betreff des Opfermahles und der
siebentägigen
Absonderung}\label{ee-vorschriften-in-betreff-des-opfermahles-und-der-siebentuxe4gigen-absonderung}}

\bibleverse{31}Hierauf gebot Mose dem Aaron und dessen Söhnen: »Kocht
das Fleisch am Eingang zum Offenbarungszelt und eßt es dort mitsamt dem
Brot, das sich in dem zum Einweihungsopfer gehörigen Korb befindet, wie
es mir geboten worden ist mit den Worten: ›Aaron und seine Söhne sollen
es essen.‹ \bibleverse{32}Was dann aber von dem Fleisch und dem Brot
übrigbleibt, müßt ihr im Feuer verbrennen. \bibleverse{33}Auch dürft ihr
sieben Tage lang nicht vom Eingang des Offenbarungszeltes weggehen bis
zu dem Tage, an dem die für euer Einweihungsopfer festgesetzte Zeit
abgelaufen ist; denn sieben Tage lang soll eure Einweihung dauern.
\bibleverse{34}Wie man heute verfahren ist, so soll man nach des HERRN
Gebot (auch an den folgenden Tagen) tun, um Sühne für euch zu erwirken.
\bibleverse{35}Ihr müßt also sieben Tage lang Tag und Nacht am Eingang
des Offenbarungszeltes bleiben und die Verordnungen des HERRN
beobachten, damit ihr nicht sterbt; denn so ist mir geboten worden.«
\bibleverse{36}So taten denn Aaron und seine Söhne alles, was der HERR
durch Mose geboten hatte.

\hypertarget{b-das-erste-feierliche-opfer-aarons-und-seiner-suxf6hne}{%
\paragraph{b) Das erste feierliche Opfer Aarons und seiner
Söhne}\label{b-das-erste-feierliche-opfer-aarons-und-seiner-suxf6hne}}

\hypertarget{aa-die-vorbereitungen}{%
\subparagraph{aa) Die Vorbereitungen}\label{aa-die-vorbereitungen}}

\hypertarget{section-8}{%
\section{9}\label{section-8}}

\bibleverse{1}Am achten Tage aber berief Mose Aaron und dessen Söhne
sowie die Ältesten der Israeliten \bibleverse{2}und gebot dem Aaron:
»Nimm dir ein junges Rind zum Sündopfer und einen Widder zum Brandopfer,
beides fehlerlose Tiere, und bringe sie vor dem HERRN dar.
\bibleverse{3}Den Israeliten aber gebiete folgendes: ›Nehmt einen
Ziegenbock zum Sündopfer sowie ein Kalb und ein Schaf, beides
einjährige, fehlerlose Tiere, zum Brandopfer; \bibleverse{4}ferner einen
Stier und einen Widder zum Heilsopfer, um sie vor dem HERRN zu
schlachten; dazu ein mit Öl gemengtes Speisopfer; denn heute wird der
HERR euch erscheinen.‹« \bibleverse{5}Da brachten sie das, was Mose
befohlen hatte, vor das Offenbarungszelt, und die ganze Gemeinde kam
herbei und stellte sich vor dem HERRN auf. \bibleverse{6}Dann sagte
Mose: »Dies ist es, was der HERR euch zu tun geboten hat, damit euch die
Herrlichkeit des HERRN erscheine.« \bibleverse{7}Hierauf sagte Mose zu
Aaron: »Tritt an den Altar und bringe dein Sündopfer und dein Brandopfer
dar, damit du für dich {[}und für das Volk{]} Sühne erwirkst; danach
bringe die Opfergabe des Volkes dar, damit du auch für sie Sühne
erwirkst, wie der HERR geboten hat.«

\hypertarget{bb-das-hohepriesterliche-suxfcnd--und-brandopfer}{%
\subparagraph{bb) Das hohepriesterliche Sünd- und
Brandopfer}\label{bb-das-hohepriesterliche-suxfcnd--und-brandopfer}}

\bibleverse{8}Da trat Aaron an den Altar heran und schlachtete das Kalb,
das zum Sündopfer für ihn selbst bestimmt war. \bibleverse{9}Hierauf
reichten seine Söhne ihm das Blut dar, und er tauchte seinen Finger in
das Blut und strich etwas davon an die Hörner des Altars; das (übrige)
Blut aber goß er an den Fuß des Altars. \bibleverse{10}Dann ließ er das
Fett sowie die Nieren und den Lappen an der Leber von dem Sündopfertier
auf dem Altar in Rauch aufgehen, wie der HERR dem Mose geboten hatte;
\bibleverse{11}das Fleisch aber und das Fell verbrannte man in einem
Feuer außerhalb des Lagers. \bibleverse{12}Hierauf schlachtete er das
Brandopfertier, und seine Söhne reichten ihm das Blut, das er ringsum an
den Altar sprengte. \bibleverse{13}Dann reichten sie ihm das in Stücke
zerlegte Brandopfer samt dem Kopf, und er ließ es auf dem Altar in Rauch
aufgehen. \bibleverse{14}Nachdem er dann die Eingeweide und die Beine
gewaschen hatte, ließ er sie auf dem Altar über dem Brandopfer in Rauch
aufgehen.

\hypertarget{cc-die-vier-opfer-fuxfcr-das-volk}{%
\subparagraph{cc) Die vier Opfer für das
Volk}\label{cc-die-vier-opfer-fuxfcr-das-volk}}

\bibleverse{15}Hierauf brachte er die Opfergabe des Volkes dar: er nahm
den Bock, der zum Sündopfer für das Volk bestimmt war, schlachtete ihn
und brachte ihn als Sündopfer dar wie das vorige Sündopfer.
\bibleverse{16}Dann brachte er das Brandopfer dar und verfuhr dabei auf
die vorgeschriebene✲ Weise. \bibleverse{17}Weiter brachte er das
Speisopfer dar, nahm von demselben eine Handvoll und ließ es auf dem
Altar in Rauch aufgehen, {[}außer dem Morgenbrandopfer{]}.
\bibleverse{18}Endlich schlachtete er das Rind und den Widder als
Heilsopfer für das Volk; seine Söhne reichten ihm dabei das Blut, das er
ringsum an den Altar sprengte. \bibleverse{19}Die Fettstücke von dem
Rind aber, ferner von dem Widder den Fettschwanz und das die Eingeweide
bedeckende Fett nebst den Nieren und dem Lappen an der Leber~--
\bibleverse{20}diese Fettstücke legten sie auf die Bruststücke, und er
ließ dann die Fettstücke auf dem Altar in Rauch aufgehen;
\bibleverse{21}die Bruststücke aber und die rechte Keule webte✲ Aaron
als Webeopfer\textless sup title=``vgl. 8,27''\textgreater✲ vor dem
HERRN, wie Mose geboten hatte.

\hypertarget{dd-zwiefache-segnung-des-volkes-erscheinen-der-herrlichkeit-des-herrn-das-feuer-gottes-verzehrt-die-opfer-aarons}{%
\subparagraph{dd) Zwiefache Segnung des Volkes; Erscheinen der
Herrlichkeit des Herrn; das Feuer Gottes verzehrt die Opfer
Aarons}\label{dd-zwiefache-segnung-des-volkes-erscheinen-der-herrlichkeit-des-herrn-das-feuer-gottes-verzehrt-die-opfer-aarons}}

\bibleverse{22}Hierauf erhob Aaron seine Hände gegen das Volk hin und
segnete es; dann stieg er (vom Altar) herab, nachdem er das Sündopfer,
das Brandopfer und das Heilsopfer dargebracht hatte. \bibleverse{23}Dann
begaben sich Mose und Aaron in das Offenbarungszelt hinein, und als sie
wieder herausgetreten waren, segneten sie das Volk. Da erschien die
Herrlichkeit des HERRN dem ganzen Volk: \bibleverse{24}Feuer ging von
dem HERRN aus und verzehrte das Brandopfer und die Fettstücke auf dem
Altar. Als das ganze Volk dies sah, jubelten sie und warfen sich auf ihr
Angesicht nieder.

\hypertarget{c-die-erste-priesterliche-verfehlung-im-gottesdienst-und-ihre-bestrafung-einige-vorschriften-fuxfcr-die-priester}{%
\paragraph{c) Die erste priesterliche Verfehlung im Gottesdienst und
ihre Bestrafung; einige Vorschriften für die
Priester}\label{c-die-erste-priesterliche-verfehlung-im-gottesdienst-und-ihre-bestrafung-einige-vorschriften-fuxfcr-die-priester}}

\hypertarget{aa-nadabs-und-abihus-versuxfcndigung-und-tod}{%
\subparagraph{aa) Nadabs und Abihus Versündigung und
Tod}\label{aa-nadabs-und-abihus-versuxfcndigung-und-tod}}

\hypertarget{section-9}{%
\section{10}\label{section-9}}

\bibleverse{1}Die Söhne Aarons aber, Nadab und Abihu, nahmen beide ihre
Räucherpfannen, taten glühende Kohlen hinein, legten Räucherwerk darauf
und brachten so dem HERRN ein ungehöriges Feueropfer dar, das er ihnen
nicht geboten hatte. \bibleverse{2}Da ging Feuer vom HERRN aus und
verzehrte✲ sie, so daß sie vor dem HERRN starben. \bibleverse{3}Da sagte
Mose zu Aaron: »Hier trifft das ein, was der HERR angekündigt hat mit
den Worten: ›An denen, die mir nahestehen, will ich mich als den
Heiligen erweisen und vor dem ganzen Volk meine Herrlichkeit
offenbaren.‹« Aaron aber sagte kein Wort. \bibleverse{4}Darauf rief Mose
den Misael und den Elzaphan, die Söhne Ussiels, des Oheims Aarons,
herbei und befahl ihnen: »Tretet herzu und tragt eure Verwandten aus dem
Heiligtum hinweg vor das Lager hinaus!« \bibleverse{5}Da traten sie
herzu und trugen sie in ihren (leinenen) Unterkleidern weg vor das Lager
hinaus, wie Mose ihnen befohlen hatte.

\hypertarget{bb-vorschriften-fuxfcr-die-priester-betreffs-ausuxfcbung-von-trauergebruxe4uchen}{%
\subparagraph{bb) Vorschriften für die Priester betreffs Ausübung von
Trauergebräuchen}\label{bb-vorschriften-fuxfcr-die-priester-betreffs-ausuxfcbung-von-trauergebruxe4uchen}}

\bibleverse{6}Darauf sagte Mose zu Aaron und dessen Söhnen Eleasar und
Ithamar: »Ihr dürft euer Haupthaar nicht auflösen\textless sup
title=``d.h. frei oder ungeordnet herabhängen lassen''\textgreater✲ und
eure Kleider nicht zerreißen\textless sup title=``vgl.
21,10''\textgreater✲; sonst müßtet ihr sterben, und der HERR würde der
ganzen Gemeinde zürnen. Doch eure Volksgenossen, das ganze Haus Israel,
mögen über den Brand weinen, den der HERR angerichtet hat.
\bibleverse{7}Auch dürft ihr euch nicht vom Eingang des
Offenbarungszeltes entfernen, damit ihr nicht sterbt; denn das Salböl
des HERRN ist auf euch (gekommen).« So taten sie denn nach der Weisung
Moses.

\hypertarget{cc-verbot-des-weingenusses-fuxfcr-die-priester-wuxe4hrend-ihrer-amtlichen-tuxe4tigkeit}{%
\subparagraph{cc) Verbot des Weingenusses für die Priester während ihrer
amtlichen
Tätigkeit}\label{cc-verbot-des-weingenusses-fuxfcr-die-priester-wuxe4hrend-ihrer-amtlichen-tuxe4tigkeit}}

\bibleverse{8}Hierauf gebot der HERR dem Aaron folgendes:
\bibleverse{9}»Wein und berauschendes Getränk dürft ihr, du und deine
Söhne mit dir, nicht trinken, wenn ihr in das Offenbarungszelt
eintretet, damit ihr nicht sterbt -- das ist eine ewiggültige Verordnung
für alle eure Geschlechter -- : \bibleverse{10}(ihr sollt lernen,) einen
Unterschied zwischen dem Heiligen und Unheiligen, zwischen dem Reinen
und Unreinen zu machen, \bibleverse{11}und sollt die Israeliten in allen
Satzungen unterweisen, die der HERR euch durch den Mund Moses verkündigt
hat.«

\hypertarget{dd-uxfcber-das-verzehren-des-priesteranteils-an-speis--und-friedensopfern}{%
\subparagraph{dd) Über das Verzehren des Priesteranteils an Speis- und
Friedensopfern}\label{dd-uxfcber-das-verzehren-des-priesteranteils-an-speis--und-friedensopfern}}

\bibleverse{12}Hierauf gebot Mose dem Aaron und dessen Söhnen Eleasar
und Ithamar, die ihm noch übriggeblieben waren: »Nehmt das Speisopfer,
das von den Feueropfern des HERRN noch übrig ist, und eßt es ungesäuert
neben dem Altar; denn es ist hochheilig. \bibleverse{13}Und zwar müßt
ihr es an heiliger Stätte verzehren, denn es ist der für dich und deine
Söhne bestimmte Anteil von den Feueropfern des HERRN -- so ist mir
geboten worden. \bibleverse{14}Die Webebrust aber und die Hebekeule
sollt ihr, du und deine Söhne und deine Töchter mit dir, an einer reinen
Stätte essen; denn als der für dich und deine Söhne\textless sup
title=``oder: Kinder''\textgreater✲ bestimmte Anteil von den Heilsopfern
der Israeliten sind sie euch überwiesen worden. \bibleverse{15}Die
Hebekeule und die Webebrust soll man mitsamt den zu Feueropfern
bestimmten Fettstücken herbringen, um sie als Webeopfer vor dem HERRN zu
weben\textless sup title=``=~zu schwingen''\textgreater✲; dann sollen
sie dir und deinen Söhnen\textless sup title=``oder:
Kindern''\textgreater✲ mit dir als eine für ewige Zeiten festgesetzte
Gebühr zufallen, wie der HERR geboten hat.«

\hypertarget{ee-uxfcber-den-genuuxdf-des-fleisches-des-fuxfcr-das-volk-dargebrachten-suxfcndopferbockes}{%
\subparagraph{ee) Über den Genuß des Fleisches des für das Volk
dargebrachten
Sündopferbockes}\label{ee-uxfcber-den-genuuxdf-des-fleisches-des-fuxfcr-das-volk-dargebrachten-suxfcndopferbockes}}

\bibleverse{16}Als Mose dann aber eifrig nach dem Sündopferbock suchte,
stellte es sich heraus, daß er verbrannt worden war! Da geriet er über
Eleasar und Ithamar, die übriggebliebenen Söhne Aarons, in heftigen Zorn
und fragte: \bibleverse{17}»Warum habt ihr das Sündopferfleisch nicht an
heiliger Stätte gegessen? Es ist ja doch hochheilig, und der HERR hat es
euch gegeben, damit ihr die Schuld der Gemeinde wegschafft, indem ihr
ihnen Sühne vor dem HERRN erwirkt. \bibleverse{18}Bedenkt doch! Das Blut
davon ist nicht ins Innere des Heiligtums gebracht worden; darum hättet
ihr es unbedingt im heiligen Bezirk essen müssen, wie ich geboten habe!«
\bibleverse{19}Da antwortete Aaron dem Mose: »Bedenke doch: heute haben
(meine Söhne) ihr Sündopfer und ihr Brandopfer vor dem HERRN
dargebracht, und mich hat trotzdem solches Geschick betroffen! Wenn ich
nun heute Sündopferfleisch genossen hätte, würde das dem HERRN
wohlgefällig gewesen sein?« \bibleverse{20}Als Mose das hörte, erkannte
er es als begründet an.

\hypertarget{reinigkeitsgesetze-und-der-grouxdfe-versuxf6hnungstag-kap.-11-16}{%
\subsubsection{3. Reinigkeitsgesetze und der große Versöhnungstag (Kap.
11-16)}\label{reinigkeitsgesetze-und-der-grouxdfe-versuxf6hnungstag-kap.-11-16}}

\hypertarget{a-verordnungen-uxfcber-die-reinen-und-unreinen-tiere}{%
\paragraph{a) Verordnungen über die reinen und unreinen
Tiere}\label{a-verordnungen-uxfcber-die-reinen-und-unreinen-tiere}}

\hypertarget{section-10}{%
\section{11}\label{section-10}}

\bibleverse{1}Hierauf gebot der HERR dem Mose und Aaron folgendes:
\bibleverse{2}»Teilt den Israeliten folgende Verordnung mit: Dies sind
die Tiere, die ihr von allen Vierfüßlern auf der Erde\textless sup
title=``=~größeren Landtieren''\textgreater✲ essen dürft:
\bibleverse{3}Alles, was unter den Vierfüßlern gespaltene
Klauen\textless sup title=``oder: Hufe''\textgreater✲ hat, und zwar ganz
durchgespaltene Klauen, und was zugleich wiederkäut, das dürft ihr
essen. \bibleverse{4}Nur folgende Tiere dürft ihr von den Wiederkäuern
und von denen, die gespaltene Klauen\textless sup title=``oder:
Hufe''\textgreater✲ haben, nicht essen: das Kamel, denn es ist zwar ein
Wiederkäuer, hat aber keine durchgespaltenen Klauen: als unrein soll es
euch gelten;~-- \bibleverse{5}ferner den Klippdachs\textless sup
title=``vgl. Psalm 104,18''\textgreater✲; denn er ist zwar ein
Wiederkäuer, hat aber keine gespaltenen Klauen: als unrein soll er euch
gelten;~-- \bibleverse{6}ferner den Hasen; denn er ist zwar ein
Wiederkäuer, hat aber keine gespaltenen Klauen: als unrein soll er euch
gelten;~-- \bibleverse{7}ferner das Schwein; denn es hat zwar gespaltene
Klauen und sogar ganz durchgespaltene Klauen, ist aber kein Wiederkäuer:
als unrein soll es euch gelten. \bibleverse{8}Vom Fleisch dieser Tiere
dürft ihr nicht essen und ihren toten Körper nicht berühren: als unrein
sollen sie euch gelten.

\bibleverse{9}Folgende Tiere dürft ihr von allen im Wasser lebenden
Tieren essen: Alles, was Flossen und Schuppen hat im Wasser, in den
Meeren und in den Flüssen, das dürft ihr essen; \bibleverse{10}alles
aber, was in den Meeren und Flüssen keine Flossen und Schuppen hat unter
allen Geschöpfen, von denen das Wasser wimmelt, und unter allen lebenden
Wesen, die sich im Wasser befinden, die sollen euch ein Greuel sein,
\bibleverse{11}ja ein Greuel sollen sie euch sein: von ihrem Fleisch
dürft ihr nichts genießen, und vor ihren toten Körpern sollt ihr einen
Abscheu haben. \bibleverse{12}Alle Wassertiere, die keine Flossen und
Schuppen haben, sollen euch ein Greuel sein.

\bibleverse{13}Von den Vögeln aber sollt ihr folgende verabscheuen, die
nicht gegessen werden dürfen, sondern ein Greuel sind: den Adler, den
Bartgeier, den Lämmergeier, \bibleverse{14}die Weihe, alle Falkenarten,
\bibleverse{15}alle Arten der Raben, \bibleverse{16}den Strauß, die
Schwalbe\textless sup title=``oder: den Kuckuck?''\textgreater✲, die
Möwe, alle Arten Habichte, \bibleverse{17}das Käuzchen, den
Sturzpelikan\textless sup title=``oder: den Kormoran''\textgreater✲, den
Uhu, \bibleverse{18}die Eule, den Pelikan, den Erdgeier,
\bibleverse{19}den Storch und alle Reiherarten, den Wiedehopf und die
Fledermaus.

\bibleverse{20}Alle kleinen geflügelten Tiere {[}die auf vier Beinen
gehen,{]} sollen euch ein Greuel sein. \bibleverse{21}Nur diejenigen von
allen geflügelten kleinen Tieren dürft ihr essen, die auf vier Beinen
gehen und oberhalb ihrer vier Beine noch ein Paar Springbeine haben, um
mit ihnen auf der Erde zu hüpfen. \bibleverse{22}Von diesen dürft ihr
die folgenden essen: alle Arten der Zugheuschrecke, alle Arten der
Solhamheuschrecke, alle Arten der Hargolheuschrecke und alle Arten der
Hagabheuschrecke. \bibleverse{23}Aber alle übrigen geflügelten kleinen
Tiere {[}die vier Beine haben,{]} sollen euch ein Greuel sein.«

\hypertarget{bestimmungen-uxfcber-die-verunreinigung-durch-beruxfchrung-der-toten-kuxf6rper-unreiner-und-reiner-tiere}{%
\paragraph{Bestimmungen über die Verunreinigung durch Berührung der
toten Körper unreiner (und reiner)
Tiere}\label{bestimmungen-uxfcber-die-verunreinigung-durch-beruxfchrung-der-toten-kuxf6rper-unreiner-und-reiner-tiere}}

\bibleverse{24}»Durch folgende Tiere also verunreinigt ihr euch --
jeder, der ihre toten Körper berührt, ist bis zum Abend unrein,
\bibleverse{25}und jeder, der etwas von ihrem toten Körper trägt, muß
seine Kleider waschen und ist bis zum Abend unrein --:
\bibleverse{26}ihr werdet also unrein durch alle Tiere, die zwar
gespaltene, aber nicht ganz durchgespaltene Klauen\textless sup
title=``oder: Hufe''\textgreater✲ haben und keine Wiederkäuer sind: sie
sollen euch als unrein gelten; wer sie berührt, wird dadurch unrein.
\bibleverse{27}Auch alle vierfüßigen Tiere, die auf Tatzen gehen, sollen
euch als unrein gelten; wer ihre toten Körper berührt, ist bis zum Abend
unrein, \bibleverse{28}und wer ihre toten Körper trägt, muß seine
Kleider waschen und ist bis zum Abend unrein; sie sollen euch als unrein
gelten. \bibleverse{29}Weiter sollen folgende Tiere euch unter den
kleinen Tieren, von denen die Erde wimmelt, als unrein gelten: das
Wiesel, die Maus, alle Arten Eidechsen\textless sup title=``oder:
Kröten''\textgreater✲, nämlich \bibleverse{30}der Mauergecko, der
Dornschwanz, der Schleuderschwanz, der Salamander und das
Chamäleon\textless sup title=``oder: der Feuermolch''\textgreater✲.
\bibleverse{31}Diese sollen euch unter allen kleinen Tieren als unrein
gelten; wer sie berührt, wenn sie tot sind, ist bis zum Abend unrein.
\bibleverse{32}Auch jeder Gegenstand, auf den eins von ihnen, wenn sie
tot sind, gefallen ist, wird unrein: jedes Holzgerät oder Kleidungsstück
oder Fell✲ oder Sack, jedes Gerät, mit dem irgendeine Arbeit verrichtet
wird, muß ins Wasser getan werden und ist bis zum Abend unrein; dann
wird es wieder rein. \bibleverse{33}Auch jedes irdene Geschirr, in das
eins von ihnen hineinfällt, wird samt seinem ganzen Inhalt unrein, und
ihr müßt es zerschlagen. \bibleverse{34}Jede Speise, die gegessen zu
werden pflegt und an die solches Wasser kommt, wird unrein; und jede
Flüssigkeit, die man zu trinken pflegt, wird in jedem (derartigen) Gefäß
unrein. \bibleverse{35}Auch alles, worauf eins von solchen toten Tieren
fällt, wird unrein; ein Backofen oder Kochherd muß eingerissen werden:
sie sind unrein und sollen euch als unrein gelten; \bibleverse{36}nur
Quellen und Zisternen, also Wasseransammlungen\textless sup
title=``oder: Wasserbehälter''\textgreater✲, bleiben rein; wer aber das
hineingefallene tote Tier berührt, wird unrein. \bibleverse{37}Wenn
ferner eins von solchen toten Tieren auf irgendwelche Sämereien fällt,
die gesät werden sollen, so bleiben diese rein; \bibleverse{38}wenn aber
Wasser auf die Sämereien gegossen worden ist und dann eins von solchen
toten Tieren darauffällt, so sollen sie euch als unrein gelten.

\bibleverse{39}Wenn ferner ein Stück von den Haustieren stirbt, deren
Fleisch ihr essen dürft, so wird der, welcher das tote Tier berührt, bis
zum Abend unrein; \bibleverse{40}wer aber etwas von einem solchen toten
Tiere genießt, muß seine Kleider waschen und ist bis zum Abend unrein;
ebenso muß der, welcher solch ein totes Tier trägt, seine Kleider
waschen und ist bis zum Abend unrein.«

\hypertarget{zusatz-betreffend-den-genuuxdf-von-kriechtieren}{%
\paragraph{Zusatz betreffend den Genuß von
Kriechtieren}\label{zusatz-betreffend-den-genuuxdf-von-kriechtieren}}

\bibleverse{41}»Alle kleinen Kriechtiere ferner, von denen die Erde
wimmelt, sind ein Greuel: sie dürfen nicht gegessen werden.
\bibleverse{42}Alles, was auf dem Bauche kriecht, und alles, was sich
auf vier Füßen bewegt, auch alle vielfüßigen Tiere, überhaupt alle
kleinen Kriechtiere, die auf der Erde wimmeln, dürft ihr nicht essen,
denn sie sind ein Greuel.«

\hypertarget{schluuxdfermahnungen}{%
\paragraph{Schlußermahnungen}\label{schluuxdfermahnungen}}

\bibleverse{43}»Macht euch nicht selbst zum Greuel durch irgendein
kleines Kriechtier, und verunreinigt euch nicht durch sie, so daß ihr
durch sie unrein werdet! \bibleverse{44}Denn ich bin der HERR, euer
Gott. Heiligt euch also und seid heilig; denn ich bin heilig.
Verunreinigt euch nicht selbst durch irgendein Gewürm, das auf der Erde
kriecht! \bibleverse{45}Denn ich bin der HERR, der euch aus Ägypten
hergeführt hat, um euer Gott zu sein; darum sollt ihr heilig sein, denn
ich bin heilig.«

\bibleverse{46}Dies sind die Vorschriften in betreff der vierfüßigen
Tiere, der Vögel und aller lebenden Wesen, die sich im Wasser regen, und
all der Wesen, von denen die Erde wimmelt, \bibleverse{47}damit man
einen Unterschied macht zwischen dem Unreinen und dem Reinen, zwischen
den Tieren, die man essen darf, und denen, die nicht gegessen werden
dürfen.

\hypertarget{b-vorschriften-bezuxfcglich-der-wuxf6chnerinnen}{%
\paragraph{b) Vorschriften bezüglich der
Wöchnerinnen}\label{b-vorschriften-bezuxfcglich-der-wuxf6chnerinnen}}

\hypertarget{section-11}{%
\section{12}\label{section-11}}

\bibleverse{1}Darauf gebot der HERR dem Mose folgendes:
\bibleverse{2}»Teile den Israeliten folgende Verordnungen mit: Wenn ein
Weib Mutter wird und einen Knaben gebiert, so ist sie sieben Tage lang
unrein! Ebenso lange wie in den Tagen ihrer Unreinheit infolge ihres
regelmäßigen Unwohlseins ist sie unrein. \bibleverse{3}Am achten Tage
soll dann das Kind an seiner Vorhaut beschnitten werden.
\bibleverse{4}Alsdann muß sie noch dreiunddreißig Tage während der Zeit
ihrer Blutreinigung (zu Hause) bleiben: sie darf nichts Heiliges
berühren und nicht ins Heiligtum kommen, bis die Tage ihrer Reinigung
abgelaufen sind. \bibleverse{5}Gebiert sie aber ein Mädchen, so ist sie
zwei Wochen lang unrein, wie bei ihrer regelmäßigen Unreinheit, und muß
dann noch sechsundsechzig Tage während der Zeit ihrer Blutreinigung (zu
Hause) bleiben. \bibleverse{6}Sobald dann die Tage ihrer Reinigung
abgelaufen sind, so soll sie, mag das Kind ein Knabe oder ein Mädchen
sein, ein einjähriges Lamm zum Brandopfer und eine junge Taube oder eine
Turteltaube zum Sündopfer an den Eingang des Offenbarungszeltes zu dem
Priester bringen. \bibleverse{7}Dieser soll dann die Opfertiere vor dem
HERRN darbringen und ihr dadurch Sühne erwirken, dann wird sie von ihrem
Blutfluß rein sein. Diese Vorschriften gelten für die Wöchnerinnen, mag
das Kind ein Knabe oder ein Mädchen sein. \bibleverse{8}Sollte ihr
Vermögen aber zur Beschaffung eines Lammes nicht ausreichen, so soll sie
zwei Turteltauben oder zwei junge Tauben nehmen, die eine zum
Brandopfer, die andere zum Sündopfer. Wenn der Priester ihr dann Sühne
erwirkt hat, wird sie rein sein.«

\hypertarget{c-verordnungen-in-betreff-des-aussatzes-und-aussatzuxe4hnlicher-krankheiten}{%
\paragraph{c) Verordnungen in betreff des Aussatzes und aussatzähnlicher
Krankheiten}\label{c-verordnungen-in-betreff-des-aussatzes-und-aussatzuxe4hnlicher-krankheiten}}

\hypertarget{aa-ausschlag-und-flecken-bei-menschen-auf-der-blouxdfen-haut}{%
\subparagraph{aa) Ausschlag und Flecken bei Menschen auf der bloßen
Haut}\label{aa-ausschlag-und-flecken-bei-menschen-auf-der-blouxdfen-haut}}

\hypertarget{section-12}{%
\section{13}\label{section-12}}

\bibleverse{1}Hierauf gebot der HERR dem Mose und dem Aaron folgendes:
\bibleverse{2}»Wenn sich bei einem Menschen auf seiner bloßen Haut eine
Anschwellung oder ein Ausschlag oder ein heller Fleck bildet und sich so
auf seiner Haut ein aussatzartiges Leiden entwickeln könnte, so soll der
Betreffende zu dem Priester Aaron oder zu einem von dessen Söhnen, den
Priestern, gebracht werden. \bibleverse{3}Wenn dann der Priester die
betroffene Stelle auf der Haut ansieht und dabei findet, daß die Haare
auf der betroffenen Stelle weiß geworden sind und daß die betroffene
Stelle tiefer liegend erscheint als die sie umgebende Haut, so handelt
es sich um wirklichen Aussatz; und sobald der Priester dies wahrnimmt,
soll er den Betreffenden für unrein erklären. \bibleverse{4}Wenn sich
aber ein weißer Fleck auf der Haut befindet, der nicht tiefer liegend
erscheint als die ihn umgebende Haut und auf dem die Haare nicht weiß
geworden sind, so soll der Priester den Betroffenen sieben Tage lang
einschließen\textless sup title=``oder: absondern''\textgreater✲.
\bibleverse{5}Wenn der Priester ihn dann am siebten Tage wieder
untersucht und dabei findet, daß die betroffene Stelle sich in ihrem
Aussehen gleichgeblieben ist, da das Übel sich auf der Haut nicht weiter
ausgebreitet hat, so soll der Priester ihn noch einmal sieben Tage lang
einschließen\textless sup title=``oder: absondern''\textgreater✲.
\bibleverse{6}Wenn dann der Priester ihn am siebten Tage wieder
untersucht und dabei findet, daß die betroffene Stelle blässer geworden
ist und das Übel sich auf der Haut nicht weiter ausgebreitet hat, so
soll ihn der Priester für rein erklären: es ist nur ein gewöhnlicher
Ausschlag; der Betreffende muß seine Kleider waschen und ist dann rein.
\bibleverse{7}Wenn aber der Ausschlag, nachdem der Betreffende, um rein
zu werden, sich dem Priester gezeigt hat, auf der Haut immer weiter um
sich greift, so soll er sich dem Priester zum zweitenmal zeigen;
\bibleverse{8}wenn ihn dann der Priester untersucht und dabei findet,
daß der Ausschlag sich auf der Haut weiter ausgebreitet hat, so soll der
Priester ihn für unrein erklären: es ist wirklicher Aussatz.«

\hypertarget{bb-veralteter-aussatz}{%
\subparagraph{bb) Veralteter Aussatz}\label{bb-veralteter-aussatz}}

\bibleverse{9}»Wenn sich ein aussatzartiges Leiden an einem Menschen
zeigt, so soll er zum Priester gebracht werden. \bibleverse{10}Wenn der
Priester ihn dann untersucht und dabei findet, daß sich eine weiße
Anschwellung auf der Haut befindet, auf der die Haare weiß geworden
sind, und daß wildes Fleisch auf der Anschwellung wuchert,
\bibleverse{11}so ist das ein bereits veralteter Aussatz auf seiner
Haut; darum soll der Priester ihn für unrein erklären, ohne ihn erst
einzuschließen\textless sup title=``oder: abzusondern''\textgreater✲;
denn er ist wirklich unrein. \bibleverse{12}Wenn aber der Aussatz auf
der Haut so wuchert, daß der Aussatz die ganze Haut des Betroffenen vom
Kopf bis zu den Füßen überzieht, soweit die Augen des Priesters blicken
mögen, \bibleverse{13}wenn also der Priester bei seiner Besichtigung
findet, daß der Aussatz den ganzen Körper überzogen hat, so soll er den
Betroffenen für rein erklären: er ist ganz und gar weiß geworden und
daher rein. \bibleverse{14}Sobald sich aber wildes Fleisch an ihm sehen
läßt, ist er unrein. \bibleverse{15}Sobald daher der Priester das wilde
Fleisch wahrnimmt, soll er ihn für unrein erklären: das wilde Fleisch
ist unrein, es ist wirklicher Aussatz. \bibleverse{16}Wenn jedoch das
wilde Fleisch wieder weggeht und der Betreffende wieder weiß wird, so
soll er zum Priester gehen. \bibleverse{17}Wenn dieser ihn dann
untersucht und dabei findet, daß die kranke Stelle wieder weiß geworden
ist, so soll der Priester den Betroffenen für rein erklären: er ist
wirklich rein.«

\hypertarget{cc-aussatz-nach-vorausgegangenem-geschwuxfcr}{%
\subparagraph{cc) Aussatz nach vorausgegangenem
Geschwür}\label{cc-aussatz-nach-vorausgegangenem-geschwuxfcr}}

\bibleverse{18}»Wenn sich ferner an jemandes Leibe auf der Haut ein
Geschwür bildet und wieder heilt, \bibleverse{19}dann aber an der Stelle
des Geschwürs eine weiße Anschwellung oder ein weiß-rötlicher Fleck
entsteht, so soll der Betreffende sich dem Priester zeigen.
\bibleverse{20}Wenn dieser ihn dann untersucht und dabei findet, daß die
Stelle tiefer liegend erscheint als die sie umgebende Haut und daß die
Haare darauf weiß geworden sind, so soll der Priester ihn für unrein
erklären: es ist wirklicher Aussatz, der in dem Geschwür zum Ausbruch
gekommen ist. \bibleverse{21}Wenn aber der Priester bei der Besichtigung
der Stelle findet, daß keine weißen Haare darauf vorhanden sind und daß
der Fleck nicht tiefer liegt als die ihn umgebende Haut und blaß
aussieht, so soll der Priester den Betreffenden sieben Tage lang
einschließen\textless sup title=``oder: absondern''\textgreater✲.
\bibleverse{22}Wenn sich dann (der Fleck) auf der Haut immer weiter
ausgebreitet hat, so soll der Priester ihn für unrein erklären: es ist
wirklicher Aussatz. \bibleverse{23}Wenn aber der Fleck ruhig
stehengeblieben ist, ohne weiter um sich gegriffen zu haben, so ist es
nur ein vernarbtes Geschwür, und der Priester darf ihn für rein
erklären.«

\hypertarget{dd-aussatz-in-einer-brandwunde}{%
\subparagraph{dd) Aussatz in einer
Brandwunde}\label{dd-aussatz-in-einer-brandwunde}}

\bibleverse{24}»Oder wenn sich auf jemandes Haut eine Brandwunde
befindet und das in der Brandwunde sich bildende Fleisch einen
weiß-rötlichen oder weißen Fleck darstellt \bibleverse{25}und der
Priester bei dessen Besichtigung findet, daß die Haare auf dem Fleck
weiß geworden sind und (der Fleck) tiefer liegend erscheint als die ihn
umgebende Haut, so liegt wirklicher Aussatz vor, der in der Brandwunde
zum Ausbruch gekommen ist; darum soll der Priester den Betreffenden für
unrein erklären: es ist wirklicher Aussatz. \bibleverse{26}Wenn aber der
Priester bei seiner Besichtigung findet, daß keine weißen Haare auf dem
Fleck vorhanden sind und daß (der Fleck) nicht tiefer liegt als die ihn
umgebende Haut und blaß aussieht, so soll der Priester den Betreffenden
sieben Tage lang einschließen\textless sup title=``oder:
absondern''\textgreater✲. \bibleverse{27}Wenn ihn dann der Priester am
siebten Tage untersucht, so soll er, falls (der Fleck) sich auf der Haut
weiter ausgebreitet hat, den Betreffenden für unrein erklären: es ist
wirklicher Aussatz. \bibleverse{28}Wenn aber der Fleck ruhig
stehengeblieben ist, ohne auf der Haut weiter um sich gegriffen zu
haben, und blaß aussieht, so liegt nur eine Anschwellung der Brandwunde
vor; darum soll der Priester den Betreffenden für rein erklären, denn es
ist nur eine vernarbte Brandwunde.«

\hypertarget{ee-kopf--und-bartgrind}{%
\subparagraph{ee) Kopf- und Bartgrind}\label{ee-kopf--und-bartgrind}}

\bibleverse{29}»Wenn ferner ein Mann oder eine Frau einen Ausschlag am
Kopf oder am Bart bekommt \bibleverse{30}und der Priester bei der
Untersuchung des Ausschlags findet, daß er tiefer liegend erscheint als
die ihn umgebende Haut und daß sich feine, goldgelbe Haare darauf
befinden, so soll der Priester den Betreffenden für unrein erklären: es
ist bösartiger Grind, der Aussatz des Kopfes oder des Bartes.
\bibleverse{31}Wenn der Priester aber den bösen Grind besichtigt und
dabei findet, daß (die kranke Stelle) zwar nicht tiefer liegend
erscheint als die sie umgebende Haut, daß aber keine dunklen Haare
darauf vorhanden sind, so soll der Priester den des Grindleidens
Verdächtigen sieben Tage lang einschließen\textless sup title=``oder:
absondern''\textgreater✲. \bibleverse{32}Wenn dann der Priester die
betroffene Stelle am siebten Tage wieder untersucht und dabei findet,
daß der Grind sich nicht weiter ausgebreitet hat und keine goldgelben
Haare darauf vorhanden sind und der Grind nicht tiefer liegend erscheint
als die ihn umgebende Haut, \bibleverse{33}so soll der Betreffende sich
scheren lassen -- nur die grindige Stelle darf er nicht scheren --; und
der Priester soll dann den des Grindleidens Verdächtigen noch einmal
sieben Tage lang einschließen\textless sup title=``oder:
absondern''\textgreater✲. \bibleverse{34}Wenn der Priester dann den
Grind am siebten Tage wieder untersucht und dabei findet, daß der Grind
sich auf der Haut nicht weiter ausgebreitet hat und daß er nicht tiefer
liegend erscheint als die ihn umgebende Haut, so erkläre der Priester
den Betreffenden für rein; er muß dann seine Kleider waschen und soll
als rein gelten. \bibleverse{35}Wenn aber, nachdem der Betreffende für
rein erklärt worden ist, der Grind sich auf der Haut weiter ausbreitet
\bibleverse{36}und der Priester bei der Untersuchung findet, daß der
Grind auf der Haut weiter um sich gegriffen hat, so braucht der Priester
nicht erst noch nach goldgelben Haaren zu suchen: der Betreffende ist
unrein. \bibleverse{37}Ist der Grind sich aber in seinem Aussehen gleich
geblieben und sind dunkle Haare darauf gewachsen, so ist der Grind
geheilt: der Betreffende ist rein, und der Priester soll ihn für rein
erklären.«

\hypertarget{ff-ungefuxe4hrlicher-ausschlag-aussatz-der-kahlkuxf6pfe}{%
\subparagraph{ff) Ungefährlicher Ausschlag; Aussatz der
Kahlköpfe}\label{ff-ungefuxe4hrlicher-ausschlag-aussatz-der-kahlkuxf6pfe}}

\bibleverse{38}»Wenn ferner ein Mann oder eine Frau auf der Haut helle
Flecke, weiße, helle Flecke bekommen \bibleverse{39}und der Priester bei
seiner Untersuchung nur blasse, weiße Flecke auf ihrer Haut findet, so
ist es ein ungefährlicher\textless sup title=``oder:
gutartiger''\textgreater✲ Ausschlag, der auf der Haut zum Ausbruch
gekommen ist: der Betreffende ist rein.

\bibleverse{40}Wenn ferner einem Manne das Haupthaar ausgeht, so ist er
ein Hinter-Kahlkopf: ein solcher ist rein; \bibleverse{41}und wenn ihm
das Haupthaar nach der Gesichtsseite zu ausgeht, so ist er ein
Vorder-Kahlkopf: ein solcher ist ebenfalls rein. \bibleverse{42}Wenn
sich aber vorn oder hinten auf seiner Glatze ein weiß-rötlicher
Ausschlag zeigt, so ist es der Aussatz, der vorn oder hinten auf seiner
Glatze zum Ausbruch gekommen ist. \bibleverse{43}Wenn also der Priester
bei seiner Untersuchung findet, daß die Anschwellung des Ausschlags vorn
oder hinten auf seiner Glatze weiß-rötlich aussieht, wie sonst der
Aussatz auf der Haut des Leibes aussieht, \bibleverse{44}so ist der
Betreffende ein Aussätziger und daher unrein; der Priester muß ihn
unbedingt für unrein erklären: er hat das Aussatzleiden an seinem
Kopfe.«

\hypertarget{gg-allgemeine-anordnungen-fuxfcr-aussuxe4tzige}{%
\subparagraph{gg) Allgemeine Anordnungen für
Aussätzige}\label{gg-allgemeine-anordnungen-fuxfcr-aussuxe4tzige}}

\bibleverse{45}»Was nun den Aussätzigen betrifft, der dieses Leiden an
sich hat, so soll er zerrissene Kleider tragen und sein Haupthaar ohne
Pflege wachsen lassen; seinen Lippenbart soll er verhüllen und ›unrein!
unrein!‹ ausrufen. \bibleverse{46}Solange die Krankheit an ihm haftet,
soll er unrein sein: er ist unrein und soll abgesondert wohnen;
außerhalb des Lagers soll seine Wohnung sein.«

\hypertarget{hh-aussatz-am-zeug-und-leder}{%
\subparagraph{hh) Aussatz am Zeug und
Leder}\label{hh-aussatz-am-zeug-und-leder}}

\bibleverse{47}»Wenn ferner eine aussätzige Stelle an einem
Kleidungsstück, es sei von Wolle oder von Leinen✲, zum Vorschein kommt
\bibleverse{48}oder an gewebten oder gewirkten Stoffen von Leinen oder
von Wolle oder an einer Lederhaut oder an irgendeinem Gegenstand von
Leder \bibleverse{49}und die betroffene Stelle an dem Kleide oder am
Leder oder an dem gewebten oder gewirkten Stoff\textless sup
title=``oder: Zeug''\textgreater✲ oder an irgendeinem Gegenstand von
Leder grünlich oder rötlich aussieht, so liegt Aussatz vor, und man soll
es vom Priester beschauen lassen. \bibleverse{50}Der Priester soll dann
den vorliegenden Schaden beschauen und den betroffenen Gegenstand sieben
Tage lang unter Verschluß halten. \bibleverse{51}Wenn er dann am siebten
Tage bei der Besichtigung des betroffenen Gegenstandes findet, daß der
Schaden an dem Kleidungsstück oder dem gewebten oder gewirkten Stoff
oder am Leder -- an irgendeinem Gegenstand, zu welchem Leder verarbeitet
wird -- weiter um sich gegriffen hat, so ist der Ausschlag ein
bösartiger Aussatz: (der betreffende Gegenstand) ist unrein,
\bibleverse{52}und man soll das Kleidungsstück oder den gewebten oder
gewirkten Stoff von Wolle oder von Leinen oder jeden ledernen
Gegenstand, an dem der Aussatz haftet, verbrennen; denn es ist ein
bösartiger Aussatz: (der betreffende Gegenstand) muß im Feuer verbrannt
werden. \bibleverse{53}Wenn der Priester aber bei der Besichtigung
findet, daß der Schaden an dem Kleidungsstück oder dem gewebten oder
gewirkten Stoff oder an irgendeinem Gegenstand von Leder nicht weiter um
sich gegriffen hat, \bibleverse{54}so soll der Priester anordnen, daß
man den Gegenstand, an welchem der Ausschlag haftet, wasche, und soll
ihn dann noch einmal sieben Tage lang unter Verschluß halten.
\bibleverse{55}Wenn der Priester dann den betroffenen Gegenstand nach
der Waschung besichtigt und dabei findet, daß die betroffene Stelle ihr
Aussehen nicht verändert hat, so ist der Gegenstand, wenn auch der
Ausschlag nicht weiter um sich gegriffen hat, dennoch unrein: du sollst
ihn im Feuer verbrennen; es ist eine eingefressene Vertiefung auf seiner
Hinter- oder seiner Vorderseite. \bibleverse{56}Findet der Priester aber
bei der Besichtigung, daß die betroffene Stelle nach der Waschung blaß
geworden ist, so soll er sie aus dem Kleidungsstück oder dem Leder oder
dem gewebten oder gewirkten Stoff herausreißen. \bibleverse{57}Wenn (der
Ausschlag) aber wiederum an dem Kleidungsstück oder dem gewebten oder
gewirkten Stoff oder an irgendeinem Gegenstande von Leder zum Vorschein
kommt, so ist es ein frisch ausbrechender Aussatz; darum mußt du den
(ganzen) Gegenstand, an dem der Ausschlag haftet, im Feuer vernichten.
\bibleverse{58}Das Kleidungsstück aber oder der gewebte oder gewirkte
Stoff oder jeder lederne Gegenstand, an welchem der Ausschlag nach der
Waschung verschwunden ist, muß noch einmal gewaschen werden und soll
dann als rein gelten.«

\bibleverse{59}Dies sind die Vorschriften bezüglich des Aussatzes an
Kleidungsstücken von Wolle oder von Leinen oder an gewebten oder
gewirkten Stoffen oder an irgendeinem Gegenstand von Leder, um
(derartige Sachen) für rein oder für unrein zu erklären.

\hypertarget{ii-die-reinigung-aussuxe4tziger-personen}{%
\subparagraph{ii) Die Reinigung aussätziger
Personen}\label{ii-die-reinigung-aussuxe4tziger-personen}}

\hypertarget{section-13}{%
\section{14}\label{section-13}}

\bibleverse{1}Weiter gab der HERR dem Mose folgende Weisungen:
\bibleverse{2}»Folgende Vorschriften gelten für einen Aussätzigen am
Tage, da er für rein erklärt wird: Er soll zu dem Priester geführt
werden; \bibleverse{3}und zwar muß der Priester vor das Lager
hinausgehen. Wenn der Priester ihn dann untersucht und dabei findet, daß
der bösartige Aussatz an dem Aussätzigen zur Heilung gekommen ist,
\bibleverse{4}so soll der Priester anordnen, daß man für den, der als
rein erklärt werden soll, zwei lebende reine Vögel sowie ein Stück
Zedernholz, Karmesinfäden\textless sup title=``oder:
-wolle''\textgreater✲ und Ysop bringe. \bibleverse{5}Sodann soll der
Priester anordnen, daß man den einen Vogel über einem irdenen Gefäß mit
Quell- oder Flußwasser schlachte. \bibleverse{6}Den lebenden Vogel aber
nebst dem Zedernholz, dem Karmesin und dem Ysop\textless sup
title=``2.Mose 12,22''\textgreater✲ soll er nehmen und dies alles, auch
den lebenden Vogel, in das Blut des über dem Quell- oder Flußwasser
geschlachteten Vogels eintauchen. \bibleverse{7}Hierauf soll er den,
welcher für rein vom Aussatz erklärt werden soll, siebenmal damit
besprengen und ihn so reinigen; den lebenden Vogel aber soll er ins
freie Feld fliegen lassen. \bibleverse{8}Hierauf soll der, welcher sich
reinigen läßt, seine Kleider waschen, sein gesamtes Haar abscheren und
ein Wasserbad nehmen: dann ist er rein. Danach darf er zwar wieder ins
Lager kommen, muß aber noch sieben Tage außerhalb seines Zeltes bleiben.
\bibleverse{9}Am siebten Tage sodann soll er alle seine Haare abermals
scheren, Kopfhaar, Bart und Augenbrauen, überhaupt sein gesamtes Haar
soll er abscheren, seine Kleider waschen und seinen Leib im Wasser
baden: dann ist er rein.«

\hypertarget{jj-die-opfer-und-gebruxe4uche-des-achten-tages}{%
\subparagraph{jj) Die Opfer und Gebräuche des achten
Tages}\label{jj-die-opfer-und-gebruxe4uche-des-achten-tages}}

\bibleverse{10}»Hierauf soll er am achten Tage zwei fehlerlose
(männliche) Lämmer und ein einjähriges, fehlerloses weibliches Lamm
nehmen, außerdem drei Zehntel Epha Feinmehl, das mit Öl gemengt ist, zum
Speisopfer, und ein Log Öl. \bibleverse{11}Der Priester aber, der die
Reinigung vollzieht, soll denjenigen, der sich reinigen läßt, samt jenen
Opfergaben vor den HERRN an den Eingang des Offenbarungszeltes stellen.
\bibleverse{12}Dann nehme der Priester das eine (männliche) Lamm, bringe
es als Schuldopfer dar nebst dem Log Öl und webe✲ beides als Webeopfer✲
vor dem HERRN. \bibleverse{13}Dann schlachte er das andere Lamm an der
Stätte, wo man die Sündopfer und die Brandopfer zu schlachten pflegt, an
heiliger Stätte; denn wie das Sündopfer, so gehört auch das Schuldopfer
dem Priester: es ist hochheilig. \bibleverse{14}Hierauf nehme der
Priester etwas von dem Blut des Schuldopfers und streiche es dem, der
sich reinigen läßt, an das rechte Ohrläppchen und an den Daumen seiner
rechten Hand sowie an die große Zehe seines rechten Fußes.
\bibleverse{15}Dann nehme der Priester etwas von dem Log Öl und gieße es
in seine\textless sup title=``d.h. des Priesters''\textgreater✲ linke
Hand, \bibleverse{16}tauche dann seinen rechten Finger in das Öl, das
sich in seiner linken Hand befindet, und sprenge von dem Öl mit seinem
Finger siebenmal vor den HERRN. \bibleverse{17}Von dem übrigen Öl aber,
das sich in seiner Hand befindet, streiche der Priester dem, der sich
reinigen läßt, etwas an das rechte Ohrläppchen und an den Daumen seiner
rechten Hand sowie an die große Zehe seines rechten Fußes auf die
Stelle, wo sich schon das Blut des Schuldopfers befindet.
\bibleverse{18}Was dann von dem Öl in der Hand des Priesters noch übrig
ist, das tue er dem, der sich reinigen läßt, auf den Kopf, um ihm so
Sühne vor dem HERRN zu erwirken. \bibleverse{19}Hierauf richte der
Priester das Sündopfer her und erwirke dadurch dem, welcher sich
reinigen läßt, Sühne wegen seiner Unreinheit; darauf schlachte er das
Brandopfertier. \bibleverse{20}Wenn dann der Priester das Brandopfer und
das Speisopfer auf dem Altar dargebracht und ihm so Sühne erwirkt hat,
so ist der Betreffende rein.«

\hypertarget{kk-ersatz-des-suxfcnd--und-brandopfers-fuxfcr-unbemittelte}{%
\subparagraph{kk) Ersatz des Sünd- und Brandopfers für
Unbemittelte}\label{kk-ersatz-des-suxfcnd--und-brandopfers-fuxfcr-unbemittelte}}

\bibleverse{21}»Wenn er aber arm ist und sein Vermögen nicht ausreicht,
so soll er nur ein einziges Lamm als Schuldopfer zur Vollziehung der
Webe nehmen, damit ihm Sühne erwirkt werde, ferner nur ein einziges
Zehntel Feinmehl, das mit Öl gemengt ist, zum Speisopfer und ein Log Öl,
\bibleverse{22}dazu zwei Turteltauben oder zwei junge Tauben, je nach
seinem Vermögen, und zwar die eine zum Sündopfer, die andere zum
Brandopfer. \bibleverse{23}Er bringe sie am achten Tage, nachdem er für
rein erklärt worden ist, zu dem Priester an den Eingang des
Offenbarungszeltes vor den HERRN. \bibleverse{24}Der Priester nehme dann
das zum Schuldopfer bestimmte Lamm sowie das Log Öl und webe✲ beides als
Webeopfer✲ vor dem HERRN. \bibleverse{25}Hierauf schlachte man das zum
Schuldopfer bestimmte Lamm, und der Priester nehme etwas von dem Blut
des Schuldopfers und streiche es dem, der sich reinigen läßt, an das
rechte Ohrläppchen und an den Daumen seiner rechten Hand sowie an die
große Zehe seines rechten Fußes. \bibleverse{26}Dann gieße der Priester
etwas von dem Öl in seine\textless sup title=``d.h. des
Priesters''\textgreater✲ linke Hand \bibleverse{27}und sprenge mit
seinem rechten Finger von dem Öl, das sich in seiner linken Hand
befindet, siebenmal vor den HERRN. \bibleverse{28}Dann streiche der
Priester etwas von dem Öl, das sich in seiner Hand befindet, dem,
welcher sich reinigen läßt, an das rechte Ohrläppchen und an den Daumen
seiner rechten Hand sowie an die große Zehe seines rechten Fußes auf die
Stelle, wo sich schon das Blut des Schuldopfers befindet.
\bibleverse{29}Was dann von dem Öl in der Hand des Priesters noch übrig
ist, das tue er dem, der sich reinigen läßt, auf den Kopf, um ihm so
Sühne vor dem HERRN zu erwirken. \bibleverse{30}Hierauf richte der
Priester die eine von den Turteltauben oder von den jungen Tauben her,
für deren Beschaffung das Vermögen des Betreffenden ausgereicht hat,
\bibleverse{31}die eine zum Sündopfer, die andere zum Brandopfer samt
dem Speisopfer. So erwirke der Priester dem, der sich reinigen läßt,
Sühne vor dem HERRN.«

\bibleverse{32}Dies sind die Vorschriften in betreff eines Aussätzigen,
dessen Vermögen bei seiner Reinigung für die regelmäßigen Opfer nicht
zureicht.

\hypertarget{ll-vorschriften-in-betreff-des-aussatzes-an-huxe4usern}{%
\subparagraph{ll) Vorschriften in betreff des Aussatzes an
Häusern}\label{ll-vorschriften-in-betreff-des-aussatzes-an-huxe4usern}}

\bibleverse{33}Hierauf gebot der HERR dem Mose und Aaron folgendes:
\bibleverse{34}»Wenn ihr in das Land Kanaan kommt, das ich euch zum
Eigentum geben will, und ich in dem Lande, das euch alsdann gehört, an
einem Hause einen Aussatzschaden entstehen lasse, \bibleverse{35}so soll
der Eigentümer des Hauses hingehen und dem Priester die Anzeige machen:
›Es zeigt sich mir an meinem Hause etwas, das wie Aussatzschaden
aussieht.‹ \bibleverse{36}Dann soll der Priester anordnen, daß man, noch
ehe er selbst zur Besichtigung des Schadens hineingeht, das Haus
ausräume, damit nicht alles, was sich im Hause befindet, unrein werde;
hierauf soll der Priester hingehen, um das Haus zu besichtigen.
\bibleverse{37}Wenn er dann bei der Besichtigung des Ausschlags findet,
daß der Ausschlag sich an den Wänden des Hauses in Gestalt grünlicher
oder rötlicher Vertiefungen zeigt und diese tiefer liegend erscheinen
als die sie umgebende Wand, \bibleverse{38}so soll der Priester aus dem
Hause hinaus an den Eingang des Hauses gehen und das Haus auf sieben
Tage verschließen. \bibleverse{39}Wenn der Priester dann am siebten Tage
wiederkommt und bei der Besichtigung findet, daß der Schaden sich an den
Wänden des Hauses weiter verbreitet hat, \bibleverse{40}so soll der
Priester anordnen, daß man die Steine, an denen sich der Schaden zeigt,
herausbreche und sie außerhalb der Ortschaft an einen unreinen Ort
hinwerfe. \bibleverse{41}Das Haus aber lasse er im Inneren überall
abkratzen, und den abgekratzten Bewurf schütte man außerhalb der
Ortschaft an einen unreinen Ort hin. \bibleverse{42}Dann soll man andere
Steine nehmen und sie an die Stelle der ausgebrochenen Steine einsetzen;
auch soll man andern Lehm\textless sup title=``oder:
Mörtel''\textgreater✲ nehmen und das Haus damit bewerfen.
\bibleverse{43}Wenn dann der Ausschlag wiederkehrt und am Hause
ausbricht, nachdem man die Steine ausgebrochen und das Haus abgekratzt
und neu beworfen hat, \bibleverse{44}so soll der Priester
wiederkommen\textless sup title=``oder: hineingehen''\textgreater✲. Wenn
er dann bei der Besichtigung findet, daß der Ausschlag am Hause weiter
um sich gegriffen hat, so ist es ein bösartiger Aussatz am Hause: es ist
unrein. \bibleverse{45}Man soll deshalb das Haus niederreißen, seine
Steine, sein Holzwerk und allen Bewurf des Hauses, und soll dies alles
außerhalb der Ortschaft an einen unreinen Ort schaffen.
\bibleverse{46}Und wer in das Haus hineingegangen ist, solange es
verschlossen ist, soll bis zum Abend als unrein gelten,
\bibleverse{47}und wer in dem Hause geschlafen hat, muß seine Kleider
waschen; ebenso muß der, welcher in dem Hause gegessen hat, seine
Kleider waschen. \bibleverse{48}Wenn aber der Priester hineingeht und
bei der Besichtigung findet, daß der Ausschlag am Hause, nachdem das
Haus neu beworfen worden ist, nicht weiter um sich gegriffen hat, so
soll der Priester das Haus für rein erklären; denn der Schaden ist
geheilt. \bibleverse{49}Er soll dann zur Entsündigung des Hauses zwei
Vögel nehmen, ferner ein Stück Zedernholz, Karmesinfäden\textless sup
title=``oder: -wolle; vgl. 14,4''\textgreater✲ und Ysop,
\bibleverse{50}und soll den einen Vogel über einem irdenen Gefäß mit
Quell- oder Flußwasser schlachten. \bibleverse{51}Dann nehme er das
Zedernholz, den Ysop, den Karmesin und den lebenden Vogel und tauche das
alles in das Blut des geschlachteten Vogels und in das Quell- oder
Flußwasser, besprenge damit das Haus siebenmal \bibleverse{52}und
entsündige so das Haus mit dem Blut des Vogels und mit dem Quell- oder
Flußwasser sowie mit dem lebenden Vogel und dem Zedernholz, mit dem Ysop
und dem Karmesin; \bibleverse{53}den lebenden Vogel aber lasse er
außerhalb der Ortschaft ins freie Feld fliegen. Wenn er so dem Hause
Sühne erwirkt hat, so ist es rein.«

\hypertarget{mm-abschluuxdf}{%
\subparagraph{mm) Abschluß}\label{mm-abschluuxdf}}

\bibleverse{54}Dies sind die Vorschriften bezüglich aller Arten von
Aussatz und bezüglich des Grindes, \bibleverse{55}ferner bezüglich des
Aussatzes an Kleidungsstücken und an Häusern, \bibleverse{56}ferner
bezüglich der Anschwellungen sowie des Ausschlags und der hellen
Flecken, \bibleverse{57}um Belehrung darüber zu geben, wann etwas für
unrein und wann für rein zu erklären ist. Dies sind die Vorschriften
über den Aussatz.

\hypertarget{d-verunreinigende-krankhafte-oder-natuxfcrliche-ausfluxfcsse-bei-muxe4nnern-und-frauen}{%
\paragraph{d) Verunreinigende (krankhafte oder natürliche) Ausflüsse bei
Männern und
Frauen}\label{d-verunreinigende-krankhafte-oder-natuxfcrliche-ausfluxfcsse-bei-muxe4nnern-und-frauen}}

\hypertarget{aa-unreinheit-der-muxe4nner}{%
\subparagraph{aa) Unreinheit der
Männer}\label{aa-unreinheit-der-muxe4nner}}

\hypertarget{section-14}{%
\section{15}\label{section-14}}

\bibleverse{1}Weiter gebot der HERR dem Mose und Aaron folgendes:
\bibleverse{2}»Teilt den Israeliten folgende Verordnungen mit: Wenn ein
Mann einen Ausfluß aus seiner Scham hat, so ist dieser sein Ausfluß
etwas Unreines; \bibleverse{3}und zwar steht es mit seiner Unreinheit
infolge seines Ausflusses so: Mag der Ausfluß aus seiner Scham dauernd
sein oder nur zeitweise eintreten: in beiden Fällen liegt Unreinheit für
ihn vor. \bibleverse{4}Jedes Lager, auf dem der mit einem Ausfluß
Behaftete liegt, wird unrein, und jedes Gerät, auf dem er sitzt, wird
unrein. \bibleverse{5}Wer sein Lager berührt, muß seine Kleider waschen
und ein Wasserbad nehmen und ist bis zum Abend unrein.
\bibleverse{6}Auch wer sich auf das Gerät setzt, auf dem der mit einem
Ausfluß Behaftete gesessen hat, muß seine Kleider waschen und ein
Wasserbad nehmen und ist bis zum Abend unrein. \bibleverse{7}Wer den
Leib eines mit einem Ausfluß Behafteten berührt, muß seine Kleider
waschen und ein Wasserbad nehmen und ist bis zum Abend unrein.
\bibleverse{8}Wenn ferner der mit einem Ausfluß Behaftete einen Reinen
anspeit\textless sup title=``oder: mit seinem Speichel
beschmutzt''\textgreater✲, so muß dieser seine Kleider waschen und ein
Wasserbad nehmen und ist bis zum Abend unrein. \bibleverse{9}Alles
Reitzeug, auf dem der mit einem Ausfluß Behaftete beim Reiten sitzt, ist
unrein; \bibleverse{10}und wer irgend etwas berührt, was er unter sich
gehabt hat, wird bis zum Abend unrein; und wer etwas Derartiges
wegträgt, muß seine Kleider waschen und ein Wasserbad nehmen und ist bis
zum Abend unrein. \bibleverse{11}Jeder, den der mit einem Ausfluß
Behaftete berührt, ohne sich zuvor die Hände mit Wasser abgespült zu
haben, muß seine Kleider waschen und ein Wasserbad nehmen und ist bis
zum Abend unrein. \bibleverse{12}Ein irdenes Gefäß, das der mit einem
Ausfluß Behaftete berührt, muß zerschlagen werden; und jedes Holzgerät
muß im Wasser abgespült werden.

\bibleverse{13}Wenn aber der mit einem Ausfluß Behaftete von seinem
Leiden befreit wird, so soll er von dem Zeitpunkte, wo er rein✲ geworden
ist, sieben Tage zählen; dann wasche er seine Kleider und bade seinen
Leib in Quell- oder Flußwasser: so ist er rein. \bibleverse{14}Am achten
Tage aber nehme er sich zwei Turteltauben oder zwei junge Tauben, trete
damit vor den HERRN an den Eingang des Offenbarungszeltes und übergebe
sie dem Priester; \bibleverse{15}dieser soll sie dann herrichten, die
eine zum Sündopfer, die andere zum Brandopfer, und der Priester soll ihm
so Sühne vor dem HERRN wegen seines Ausflusses erwirken.

\bibleverse{16}Wenn ferner ein Mann einen unfreiwilligen Samenerguß hat,
so soll er seinen ganzen Leib im Wasser baden und ist bis zum Abend
unrein; \bibleverse{17}und jedes Kleidungsstück und alles Leder, an
welches der Samenerguß gekommen ist, muß im Wasser gewaschen werden und
ist bis zum Abend unrein.~-- \bibleverse{18}Wenn ferner ein Mann bei
einer Frau liegt und ein Samenerguß erfolgt, so müssen beide ein
Wasserbad nehmen und sind bis zum Abend unrein.«

\hypertarget{bb-unreine-zustuxe4nde-bei-frauen}{%
\subparagraph{bb) Unreine Zustände bei
Frauen}\label{bb-unreine-zustuxe4nde-bei-frauen}}

\bibleverse{19}»Wenn aber eine weibliche Person einen Ausfluß hat, und
zwar ihren gewöhnlichen Blutfluß, so soll ihre Unreinheit sieben Tage
lang dauern, und jeder, der sie während dieser Zeit berührt, wird bis
zum Abend unrein. \bibleverse{20}Alles, worauf sie während ihrer
Unreinheit liegt, wird unrein, ebenso alles, worauf sie sitzt,
\bibleverse{21}und jeder, der ihr Lager berührt, muß seine Kleider
waschen und ein Wasserbad nehmen und ist bis zum Abend unrein.
\bibleverse{22}Und jeder, der irgendein Gerät berührt, auf dem sie
gesessen hat, muß seine Kleider waschen und ein Wasserbad nehmen und ist
bis zum Abend unrein. \bibleverse{23}Auch wenn einer etwas berührt, das
sich auf ihrem Lager oder auf dem Gerät befindet, auf dem sie gesessen
hat, so ist er bis zum Abend unrein. \bibleverse{24}Und wenn etwa ein
Mann ihr beiwohnt und etwas von ihrer Unreinigkeit an ihn kommt, so ist
er sieben Tage lang unrein, und jedes Lager, auf dem er liegt, wird
unrein.

\bibleverse{25}Wenn aber eine weibliche Person außer der Zeit ihrer
regelmäßigen Unreinheit an einem langdauernden Blutfluß leidet oder über
die Zeit ihrer regelmäßigen Unreinheit hinaus den Blutfluß hat, so soll
von ihr während der ganzen Dauer ihres unreinen Flusses dasselbe gelten
wie während der Zeit ihrer regelmäßigen Unreinheit: sie ist unrein.
\bibleverse{26}Jedes Lager, auf dem sie während der ganzen Dauer ihres
Flusses liegt, soll für sie gelten wie das Lager zur Zeit ihrer
regelmäßigen Unreinheit; und jedes Gerät, auf dem sie sitzt, wird gerade
so unrein wie während der Zeit ihrer regelmäßigen Unreinheit;
\bibleverse{27}wer also irgendein solches (Gerät) berührt, wird dadurch
unrein: er muß seine Kleider waschen und ein Wasserbad nehmen und bleibt
bis zum Abend unrein. \bibleverse{28}Wenn sie aber von ihrem Fluß rein
geworden ist, so soll sie noch sieben Tage zählen: dann soll sie als
rein gelten. \bibleverse{29}Am achten Tage aber nehme sie sich zwei
Turteltauben oder zwei junge Tauben und bringe sie zu dem Priester an
den Eingang des Offenbarungszeltes; \bibleverse{30}der Priester soll
dann die eine als Sündopfer, die andere als Brandopfer darbringen und
ihr so Sühne vor dem HERRN wegen ihres unreinen Flusses erwirken.«

\hypertarget{cc-schluuxdfworte}{%
\subparagraph{cc) Schlußworte}\label{cc-schluuxdfworte}}

\bibleverse{31}»So sollt ihr die Israeliten von ihrer Unreinheit
befreien, damit sie nicht infolge ihrer Unreinheit sterben, wenn sie
meine Wohnung, die sich mitten unter ihnen befindet, verunreinigen.«~--
\bibleverse{32}Diese Vorschriften gelten für den mit einem Ausfluß
Behafteten sowie für den, der durch Samenerguß verunreinigt wird,
\bibleverse{33}ferner für weibliche Personen, die an ihrer regelmäßigen
Unreinheit leiden, und für jeden, der einen Ausfluß hat, es sei ein Mann
oder eine weibliche Person, sowie für den Mann, der einer unreinen Frau
beiwohnt.

\hypertarget{e-der-juxe4hrliche-grouxdfe-versuxf6hnungstag}{%
\paragraph{e) Der jährliche große
Versöhnungstag}\label{e-der-juxe4hrliche-grouxdfe-versuxf6hnungstag}}

\hypertarget{aa-die-vorbereitungen-1}{%
\subparagraph{aa) Die Vorbereitungen}\label{aa-die-vorbereitungen-1}}

\hypertarget{section-15}{%
\section{16}\label{section-15}}

\bibleverse{1}Weiter redete der HERR mit Mose nach dem Tode der beiden
Söhne Aarons, die da hatten sterben müssen, als sie (mit einem
ungehörigen Opfer) vor den HERRN getreten waren. \bibleverse{2}Der HERR
gebot damals dem Mose: »Sage deinem Bruder Aaron, er dürfe nicht zu
jeder beliebigen Zeit in das (innerste) Heiligtum hinter den Vorhang
eintreten vor die Deckplatte, die über der Lade liegt, er müßte sonst
sterben; denn ich offenbare mich in der Wolke über der Deckplatte.
\bibleverse{3}Nur in dem Falle darf Aaron in das (innerste) Heiligtum
eintreten, wenn er mit einem jungen Stier zum Sündopfer und mit einem
Widder zum Brandopfer erscheint. \bibleverse{4}Er muß ferner ein
heiliges Unterkleid von Leinen anhaben und Unterbeinkleider von Leinen
auf dem bloßen Leibe tragen und mit einem Gürtel von Leinen umgürtet
sein und sich einen Kopfbund von Leinen umgebunden haben; das sind die
heiligen Kleider, die er anlegen muß, nachdem er ein Wasserbad genommen
hat. \bibleverse{5}Von der Gemeinde der Israeliten aber soll er sich
zwei Ziegenböcke zum Sündopfer und einen Widder zum Brandopfer geben
lassen.«

\hypertarget{bb-erste-kurze-darstellung-der-zu-beobachtenden-suxfchngebruxe4uche}{%
\subparagraph{bb) Erste kurze Darstellung der zu beobachtenden
Sühngebräuche}\label{bb-erste-kurze-darstellung-der-zu-beobachtenden-suxfchngebruxe4uche}}

\bibleverse{6}»Hierauf soll Aaron den jungen Stier, der zum Sündopfer
für ihn selbst bestimmt ist, herzubringen✲ und sich und seinem Hause
Sühne erwirken. \bibleverse{7}Dann soll er die beiden Böcke nehmen und
sie vor den HERRN an den Eingang des Offenbarungszeltes stellen.
\bibleverse{8}Hierauf soll Aaron Lose über die beiden Böcke werfen, das
eine Los für den HERRN, das andere Los für Asasel. \bibleverse{9}Dann
soll Aaron den Bock, auf den das für den HERRN bestimmte Los gefallen
ist, heranbringen und ihn als Sündopfer herrichten; \bibleverse{10}der
Bock aber, auf den das für Asasel bestimmte Los gefallen ist, soll
lebend vor den HERRN gestellt werden, damit man über ihm die
Sühnehandlungen vollziehe und ihn dann zu Asasel in die Wüste schicke.«

\hypertarget{cc-ins-einzelne-gehende-darstellung-der-suxfchngebruxe4uche}{%
\subparagraph{cc) Ins einzelne gehende Darstellung der
Sühngebräuche}\label{cc-ins-einzelne-gehende-darstellung-der-suxfchngebruxe4uche}}

\hypertarget{entsuxfcndigung-der-priesterschaft}{%
\paragraph{Entsündigung der
Priesterschaft}\label{entsuxfcndigung-der-priesterschaft}}

\bibleverse{11}»Aaron soll also den jungen Stier, der zum Sündopfer für
ihn selbst bestimmt ist, herzubringen, um für sich und sein Haus Sühne
zu erwirken. Er schlachte demnach den Stier, der zum Sündopfer für ihn
bestimmt ist, \bibleverse{12}nehme dann die Räucherpfanne voll glühender
Kohlen von dem Altar, der vor dem HERRN steht, und seine beiden Hände
voll feinzerstoßenen, wohlriechenden Räucherwerks und bringe es hinein
in den Raum hinter dem Vorhange. \bibleverse{13}Dort tue er das
Räucherwerk vor dem HERRN auf die Kohlenglut, damit die Wolke des
Räucherwerks die Deckplatte, die auf der Gesetzeslade liegt, verhülle
und er nicht zu sterben braucht. \bibleverse{14}Dann nehme er etwas von
dem Blut des Stieres und sprenge es mit seinem Finger auf die
Vorderseite der Deckplatte, und auf die Stelle vor der Deckplatte
sprenge er siebenmal von dem Blut mit seinem Finger.«

\hypertarget{entsuxfcndigung-des-heiligtums-und-des-brandopferaltars}{%
\paragraph{Entsündigung des Heiligtums und des
Brandopferaltars}\label{entsuxfcndigung-des-heiligtums-und-des-brandopferaltars}}

\bibleverse{15}»Hierauf schlachte er den Bock, der zum Sündopfer für das
Volk bestimmt ist, und bringe sein Blut in den Raum hinter dem Vorhang.
Dort verfahre er mit diesem Blut ebenso, wie er mit dem Blut des Stieres
verfahren ist: er sprenge es auf die Deckplatte und auf die Stelle vor
die Deckplatte \bibleverse{16}und erwirke so dem Heiligtum Sühne wegen
der Verunreinigungen durch die Israeliten und wegen aller Übertretungen,
die sie sich haben zuschulden kommen lassen; ebenso verfahre er dann
auch mit dem Offenbarungszelt dessen, der bei ihnen mitten unter ihren
Versündigungen wohnt. \bibleverse{17}Es darf aber kein Mensch im
Offenbarungszelt anwesend sein, wenn er hineingeht, um im Heiligtum die
Sühnehandlungen vorzunehmen, bis er wieder herausgekommen ist. So
erwirke er Sühne für sich und sein Haus und für die ganze Gemeinde
Israel. \bibleverse{18}Alsdann soll er an den Altar hinausgehen, der vor
dem HERRN steht, und auch an diesem die Entsündigung vollziehen. Er
nehme nämlich etwas von dem Blut des Stieres und vom Blut des Bockes und
streiche es an die Hörner des Altars ringsum; \bibleverse{19}dann
sprenge er etwas von dem Blut siebenmal mit seinem Finger
an\textless sup title=``oder: auf''\textgreater✲ den Altar und heilige
ihn so und reinige ihn von den Versündigungen der Israeliten.«

\hypertarget{entsuxfcndigung-der-gemeinde-entsendung-des-bockes-fuxfcr-asasel}{%
\paragraph{Entsündigung der Gemeinde: Entsendung des Bockes für
Asasel}\label{entsuxfcndigung-der-gemeinde-entsendung-des-bockes-fuxfcr-asasel}}

\bibleverse{20}»Nachdem er so die Sühnung des Heiligtums und des
Offenbarungszeltes und des Altars vollzogen hat, soll er den noch
lebenden Bock herbeiholen. \bibleverse{21}Aaron lege diesem Bock seine
beiden Hände fest auf den Kopf, bekenne über ihm alle Verschuldungen der
Israeliten und alle Übertretungen, die sie sich irgendwie haben
zuschulden kommen lassen; er lege sie auf den Kopf des Bockes und lasse
diesen durch einen bereitstehenden Mann in die Wüste fortschaffen.
\bibleverse{22}So soll der Bock alle ihre Verschuldungen auf sich nehmen
und sie in eine abgeschiedene Gegend tragen; (der Mann) soll ihn dann in
der Wüste loslassen.«

\hypertarget{die-schluuxdfopfer-besonders-die-darbringung-des-suxfcndopferbockes}{%
\paragraph{Die Schlußopfer, besonders die Darbringung des
Sündopferbockes}\label{die-schluuxdfopfer-besonders-die-darbringung-des-suxfcndopferbockes}}

\bibleverse{23}»Hierauf soll Aaron wieder in das Offenbarungszelt
hineingehen und die leinenen Kleider, die er bei seinem Eintritt in das
innerste Heiligtum angezogen hatte, ausziehen und sie dort niederlegen.
\bibleverse{24}Darauf soll er an heiliger Stätte ein Wasserbad nehmen,
dann seine gewöhnlichen Amtskleider anziehen, hierauf wieder hinausgehen
und das Brandopfer für sich und das Brandopfer für das Volk herrichten,
um dadurch sich und dem Volke Sühne zu erwirken. \bibleverse{25}Das Fett
des Sündopfers soll er auf dem Altar in Rauch aufgehen lassen.
\bibleverse{26}Der Mann aber, der den Bock zu Asasel hinausgeschafft
hat, muß seine Kleider waschen und ein Wasserbad nehmen: dann darf er
wieder ins Lager kommen. \bibleverse{27}Den Sündopferstier aber und den
Sündopferbock, deren Blut hineingebracht worden ist, um im innersten
Heiligtum zur Vollziehung der Sühnehandlungen zu dienen, soll man vor
das Lager hinausschaffen und ihre Felle, ihr Fleisch und ihren
Eingeweideinhalt im Feuer verbrennen. \bibleverse{28}Der (Mann), welcher
die Verbrennung besorgt hat, muß seine Kleider waschen und ein Wasserbad
nehmen: dann darf er wieder ins Lager kommen.«

\hypertarget{festsetzung-des-tages-und-bestimmungen-uxfcber-die-uxe4uuxdferliche-feier-des-festes}{%
\paragraph{Festsetzung des Tages und Bestimmungen über die äußerliche
Feier des
Festes}\label{festsetzung-des-tages-und-bestimmungen-uxfcber-die-uxe4uuxdferliche-feier-des-festes}}

\bibleverse{29}»Und folgendes soll für euch eine ewiggültige Verordnung
sein: Im siebten Monat, am zehnten Tage des Monats, sollt ihr eure
Seelen beugen\textless sup title=``oder: kasteien, d.h.
fasten''\textgreater✲ und dürft keinerlei Arbeit verrichten, weder der
Einheimische noch der Fremdling, der sich bei euch aufhält;
\bibleverse{30}denn an diesem Tage erwirkt man für euch Sühne, um euch
zu reinigen: von all euren Sünden sollt ihr da vor dem HERRN rein
werden. \bibleverse{31}Ein Tag völliger Ruhe\textless sup title=``=~ein
hoher Feiertag''\textgreater✲ soll es für euch sein, und ihr sollt
fasten; das ist eine ewiggültige Verordnung. \bibleverse{32}Die Sühnung
soll aber der Priester vollziehen, den man gesalbt und in sein Amt
eingesetzt haben wird, damit er an Stelle seines Vaters den
Priesterdienst versehe; und er soll die leinenen Kleider, die heiligen
Gewänder, anlegen \bibleverse{33}und das Allerheiligste entsündigen;
ebenso soll er das Offenbarungszelt und den Altar entsündigen und auch
den Priestern und der gesamten Volksgemeinde Sühne erwirken.
\bibleverse{34}Dies soll für euch eine ewiggültige Verordnung sein, daß
man den Israeliten einmal im Jahr Sühne für all ihre Sünden erwirken
soll.« -- Und (Aaron) tat, wie der HERR dem Mose geboten hatte.

\hypertarget{das-heiligkeitsgesetz-kap.-17-27}{%
\subsubsection{4. Das Heiligkeitsgesetz (Kap.
17-27)}\label{das-heiligkeitsgesetz-kap.-17-27}}

\hypertarget{a-vorschriften-uxfcber-schlachtung-der-haustiere-uxfcber-blutgenuuxdf-u.a.}{%
\paragraph{a) Vorschriften über Schlachtung der Haustiere, über
Blutgenuß
u.a.}\label{a-vorschriften-uxfcber-schlachtung-der-haustiere-uxfcber-blutgenuuxdf-u.a.}}

\hypertarget{section-16}{%
\section{17}\label{section-16}}

\bibleverse{1}Hierauf gebot der HERR dem Mose folgendes:
\bibleverse{2}»Rede mit Aaron und seinen Söhnen und mit allen Israeliten
und sage ihnen: Dies ist es, was der HERR geboten hat:«

\hypertarget{aa-einheit-der-schlachtstuxe4tte-opferbarer-tiere}{%
\subparagraph{aa) Einheit der Schlachtstätte opferbarer
Tiere}\label{aa-einheit-der-schlachtstuxe4tte-opferbarer-tiere}}

\bibleverse{3}»Jedermann vom Hause Israel, der ein Rind oder ein Schaf
oder eine Ziege innerhalb oder außerhalb des Lagers schlachten will,
\bibleverse{4}(das Tier) aber nicht an den Eingang des
Offenbarungszeltes bringt, um es dem HERRN als Opfergabe vor der Wohnung
des HERRN darzubringen, einem solchen Manne soll das als Blutschuld
angerechnet werden: Blut hat er vergossen, darum soll ein solcher Mann
aus der Mitte seines Volkes ausgerottet werden! \bibleverse{5}Die
Israeliten sollen also ihre Schlachttiere, die sie jetzt auf freiem
Felde zu schlachten pflegen, herbeibringen, und zwar sollen sie sie für
den HERRN an den Eingang des Offenbarungszeltes zu dem Priester bringen
und sie als Heilsopfer für den HERRN schlachten. \bibleverse{6}Der
Priester soll dann das Blut an den Altar des HERRN, der vor dem Eingang
des Offenbarungszeltes steht, sprengen und das Fett zum lieblichen
Geruch für den HERRN in Rauch aufgehen lassen. \bibleverse{7}Sie sollen
also ihre Schlachttiere hinfort nicht mehr den bösen
Geistern\textless sup title=``oder: den Feldteufeln''\textgreater✲
schlachten, deren Götzendienst sie jetzt treiben! Dies soll für sie eine
ewiggültige Verordnung von Geschlecht zu Geschlecht sein.«

\hypertarget{bb-einheit-der-opferstuxe4tte-beim-offenbarungszelt}{%
\subparagraph{bb) Einheit der Opferstätte beim
Offenbarungszelt}\label{bb-einheit-der-opferstuxe4tte-beim-offenbarungszelt}}

\bibleverse{8}»Weiter sollst du ihnen sagen: Wenn jemand vom Hause
Israel und von den Fremdlingen, die sich als Gäste bei ihnen aufhalten,
ein Brandopfer oder ein Schlachtopfer darbringen will \bibleverse{9}und
es nicht an den Eingang des Offenbarungszeltes bringt, um es für den
HERRN herzurichten, so soll ein solcher Mensch aus seinen Volksgenossen
ausgerottet werden.«

\hypertarget{cc-verbot-jeglichen-blutgenusses}{%
\subparagraph{cc) Verbot jeglichen
Blutgenusses}\label{cc-verbot-jeglichen-blutgenusses}}

\bibleverse{10}»Wenn ferner jemand vom Hause Israel und von den
Fremdlingen, die sich bei ihnen aufhalten, irgendwelches Blut genießt:
gegen einen solchen Menschen, der Blut genießt, will ich mein Angesicht
richten und ihn aus der Mitte seines Volkes ausrotten.
\bibleverse{11}Denn das Leben des Leibes liegt im Blut, und ich habe es
euch für den Altar bestimmt, damit ihr euch dadurch Sühne für eure
Sünden erwirkt; denn das Blut ist es, das Sühne durch das in ihm
enthaltene Leben bewirkt. \bibleverse{12}Darum habe ich den Israeliten
geboten: Niemand von euch darf Blut genießen! Auch der Fremdling, der
als Gast unter euch lebt, darf kein Blut genießen!«

\hypertarget{dd-behandlung-des-blutes-jagdbaren-wildes-und-des-fleisches-gefallener-oder-zerrissener-tiere}{%
\subparagraph{dd) Behandlung des Blutes jagdbaren Wildes und des
Fleisches gefallener oder zerrissener
Tiere}\label{dd-behandlung-des-blutes-jagdbaren-wildes-und-des-fleisches-gefallener-oder-zerrissener-tiere}}

\bibleverse{13}»Wenn also jemand von den Israeliten und von den
Fremdlingen, die als Gäste unter ihnen leben, ein Stück Wild oder
Geflügel erjagt, das gegessen werden darf, so soll er das Blut des
Tieres auslaufen lassen und es mit Erde bedecken. \bibleverse{14}Denn
was das Leben alles Fleisches\textless sup title=``=~jedes lebenden
Geschöpfes''\textgreater✲ betrifft, so liegt sein Leben in seinem Blut.
Darum habe ich den Israeliten geboten: Von keinem Fleisch\textless sup
title=``=~lebenden Geschöpf''\textgreater✲ dürft ihr das Blut genießen,
denn das Leben alles Fleisches\textless sup title=``=~jedes
Geschöpfes''\textgreater✲ liegt in seinem Blut: jeder, der es genießt,
soll ausgerottet werden.~-- \bibleverse{15}Und wer das Fleisch eines
gefallenen✲ oder von Raubtieren zerrissenen Tieres genießt, er sei ein
Einheimischer oder ein Fremdling, der soll seine Kleider waschen und ein
Wasserbad nehmen und ist bis zum Abend unrein: alsdann ist er wieder
rein. \bibleverse{16}Wenn er aber (seine Kleider) nicht wäscht und sich
nicht badet, so hat er seine Verschuldung zu tragen.«

\hypertarget{b-ehe--und-keuschheitsgesetze}{%
\paragraph{b) Ehe- und
Keuschheitsgesetze}\label{b-ehe--und-keuschheitsgesetze}}

\hypertarget{aa-gottes-mahnung-nicht-nach-der-weise-der-uxe4gypter-und-kanaanuxe4er-sondern-in-gottes-satzungen-zu-wandeln}{%
\subparagraph{aa) Gottes Mahnung, nicht nach der Weise der Ägypter und
Kanaanäer, sondern in Gottes Satzungen zu
wandeln}\label{aa-gottes-mahnung-nicht-nach-der-weise-der-uxe4gypter-und-kanaanuxe4er-sondern-in-gottes-satzungen-zu-wandeln}}

\hypertarget{section-17}{%
\section{18}\label{section-17}}

\bibleverse{1}Weiter gebot der HERR dem Mose folgendes:
\bibleverse{2}»Teile den Israeliten folgende Verordnungen mit: Ich bin
der HERR, euer Gott! \bibleverse{3}Nach der Weise der Bewohner des
Landes Ägypten, in dem ihr gewohnt habt, dürft ihr nicht verfahren; auch
nach der Weise der Bewohner des Landes Kanaan, wohin ich euch bringen
werde, dürft ihr nicht verfahren und nach ihren Satzungen nicht wandeln;
\bibleverse{4}nein, meine Gebote sollt ihr befolgen und meine Satzungen
beobachten, um in\textless sup title=``oder: nach''\textgreater✲ ihnen
zu wandeln: ich bin der HERR, euer Gott! \bibleverse{5}So beobachtet
denn meine Satzungen und meine Gebote; denn der Mensch, der nach ihnen
tut, wird durch sie das Leben haben: ich bin der HERR!«

\hypertarget{bb-verbot-der-blutschande-aufzuxe4hlung-der-verbotenen-ehen}{%
\subparagraph{bb) Verbot der Blutschande; Aufzählung der verbotenen
Ehen}\label{bb-verbot-der-blutschande-aufzuxe4hlung-der-verbotenen-ehen}}

\bibleverse{6}»Keiner von euch darf sich irgendeiner seiner nächsten
Blutsverwandten nahen, um mit ihr geschlechtlichen Umgang zu haben: ich
bin der HERR! \bibleverse{7}Mit deinem Vater und mit deiner Mutter
darfst du keinen geschlechtlichen Umgang haben; sie ist deine Mutter: du
darfst ihr nicht beiwohnen. \bibleverse{8}Mit deiner Stiefmutter darfst
du keinen geschlechtlichen Umgang haben: dein Vater allein hat ein Recht
an sie. \bibleverse{9}Mit deiner Schwester, der Tochter deines Vaters
oder der Tochter deiner Mutter, mag sie im Hause geboren oder auswärts
geboren\textless sup title=``=~von außen zugebracht''\textgreater✲ sein,
darfst du keinen geschlechtlichen Umgang haben. \bibleverse{10}Mit der
Tochter deines Sohnes oder mit der Tochter deiner Tochter darfst du
keinen geschlechtlichen Umgang haben; denn sie sind (wie) deine eigenen
Töchter✲. \bibleverse{11}Mit der Tochter der Frau deines Vaters, die
dein Vater gezeugt hat -- sie ist (wie) deine Schwester -- darfst du
keinen geschlechtlichen Umgang haben. \bibleverse{12}Mit der Schwester
deines Vaters darfst du keinen geschlechtlichen Umgang haben: sie ist
deines Vaters nächste Blutsverwandte. \bibleverse{13}Mit der Schwester
deiner Mutter darfst du keinen geschlechtlichen Umgang haben, denn sie
ist die nächste Blutsverwandte deiner Mutter. \bibleverse{14}Mit der
Frau des Bruders deines Vaters darfst du keinen geschlechtlichen Umgang
haben; seinem Weibe darfst du nicht nahen, sie ist deine
Muhme\textless sup title=``=~Ohmsfrau, Tante''\textgreater✲.
\bibleverse{15}Mit deiner Schwiegertochter darfst du keinen
geschlechtlichen Umgang haben; sie ist das Weib deines Sohnes: du darfst
ihr nicht beiwohnen. \bibleverse{16}Mit der Frau deines Bruders darfst
du keinen geschlechtlichen Umgang haben; nur dein Bruder hat ein Recht
an sie. \bibleverse{17}Mit einer Frau und zugleich ihrer Tochter darfst
du keinen geschlechtlichen Umgang haben; die Tochter ihres Sohnes und
die Tochter ihrer Tochter darfst du nicht nehmen, um ihnen beizuwohnen;
sie sind nächste Blutsverwandte: es wäre eine Schandtat✲.
\bibleverse{18}Auch darfst du eine Frau nicht zu ihrer Schwester als
Nebenfrau hinzunehmen, um ihr neben jener, solange sie lebt,
beizuwohnen.«

\hypertarget{cc-warnung-vor-unzuchtssuxfcnden}{%
\subparagraph{cc) Warnung vor
Unzuchtssünden}\label{cc-warnung-vor-unzuchtssuxfcnden}}

\bibleverse{19}»Du darfst ferner einer Frau während der Zeit ihrer
Unreinheit\textless sup title=``d.h. ihres monatlichen
Unwohlseins''\textgreater✲ nicht nahen, um ihr beizuwohnen.~--
\bibleverse{20}Mit der Ehefrau deines Nächsten\textless sup
title=``oder: Volksgenossen''\textgreater✲ darfst du nicht den Beischlaf
vollziehen, weil du dich dadurch verunreinigen würdest.~--
\bibleverse{21}Von deinen Kindern darfst du keines hingeben, um es dem
Moloch zur Opferung zu weihen, damit du den Namen deines Gottes nicht
entweihst: ich bin der HERR.~-- \bibleverse{22}Bei einem Manne darf man
nicht liegen, wie man bei einer Frau liegt; das wäre eine Greueltat.~--
\bibleverse{23}Auch mit keinem Tiere darfst du dich paaren und dich
dadurch verunreinigen; und eine weibliche Person darf sich nicht vor ein
Tier hinstellen, um sich von ihm begatten zu lassen; das wäre eine
schändliche Versündigung.«

\hypertarget{dd-ermahnender-und-drohender-schluuxdf}{%
\subparagraph{dd) Ermahnender und drohender
Schluß}\label{dd-ermahnender-und-drohender-schluuxdf}}

\bibleverse{24}»Verunreinigt euch nicht durch etwas Derartiges! Denn
durch alles dieses haben sich die Völkerschaften verunreinigt, die ich
vor euch vertreiben werde. \bibleverse{25}{[}Da das Land dadurch
verunreinigt wurde, habe ich seine Verschuldung an ihm heimgesucht, so
daß das Land seine Bewohner ausgespien hat.{]} \bibleverse{26}Ihr aber
sollt meine Satzungen und meine Gebote beobachten und dürft keinen von
allen solchen Greueln verüben, weder der Einheimische noch der
Fremdling, der als Gast unter euch lebt~-- \bibleverse{27}denn alle
diese Greuel haben die Leute verübt, die vor euch im Lande gewohnt
haben, und das Land ist dadurch verunreinigt worden --;
\bibleverse{28}das Land würde auch euch sonst ausspeien, wenn ihr es
verunreinigt, wie es das Volk ausgespien hat, das vor euch da war.
\bibleverse{29}Denn wer irgendeinen von diesen Greueln verübt: alle, die
Derartiges verüben, sollen aus der Mitte ihres Volkes ausgerottet
werden. \bibleverse{30}So beobachtet denn, was mir gegenüber zu
beobachten ist, daß ihr keinen von den greulichen Bräuchen übt, die vor
euch geübt worden sind, und euch dadurch nicht verunreinigt: ich bin der
HERR, euer Gott!«

\hypertarget{c-allerlei-vorschriften-auf-grund-der-zehn-gebote-besonders-in-betreff-heiliger-pflichten-im-tuxe4glichen-leben}{%
\paragraph{c) Allerlei Vorschriften auf Grund der Zehn Gebote, besonders
in betreff heiliger Pflichten im täglichen
Leben}\label{c-allerlei-vorschriften-auf-grund-der-zehn-gebote-besonders-in-betreff-heiliger-pflichten-im-tuxe4glichen-leben}}

\hypertarget{section-18}{%
\section{19}\label{section-18}}

\bibleverse{1}Weiter gebot der HERR dem Mose folgendes:
\bibleverse{2}»Teile der ganzen Gemeinde der Israeliten folgende
Verordnungen mit: Ihr sollt heilig sein, denn ich, der HERR euer Gott,
bin heilig! \bibleverse{3}Ihr sollt ein jeder Ehrfurcht vor seiner
Mutter und seinem Vater haben und meine Ruhetage beobachten: ich bin der
HERR, euer Gott.~-- \bibleverse{4}Wendet euch nicht den Götzen zu und
fertigt euch keine gegossenen Götterbilder an: ich bin der HERR, euer
Gott!

\bibleverse{5}Wenn ihr dem HERRN ein Heilsopfer schlachten✲ wollt, sollt
ihr es so opfern, daß ihr Wohlgefallen (beim HERRN) dadurch erlangt.
\bibleverse{6}Am Tage, an dem ihr es opfert, und am Tage darauf muß es
gegessen werden; was aber bis zum dritten Tage übriggeblieben ist, muß
im Feuer verbrannt werden. \bibleverse{7}Sollte dennoch am dritten Tage
davon gegessen werden, so würde das als verdorbenes Fleisch gelten und
nicht wohlgefällig aufgenommen werden: \bibleverse{8}wer es äße, würde
eine Verschuldung auf sich laden; denn er hätte das dem HERRN Geheiligte
entweiht, und ein solcher Mensch soll aus seinen Volksgenossen
ausgerottet werden.

\bibleverse{9}Wenn ihr die Ernte eures Landes schneidet, so sollst du
dein Feld nicht ganz bis an den äußersten Rand abernten und auch keine
Nachlese nach deiner Ernte halten. \bibleverse{10}Auch in deinem
Weinberge sollst du keine Nachlese vornehmen und die abgefallenen Beeren
in deinem Weinberge nicht auflesen; dem Armen und dem Fremdling sollst
du sie überlassen: ich bin der HERR, euer Gott!

\bibleverse{11}Ihr sollt nicht stehlen und nicht ableugnen und euch
nicht untereinander betrügen. \bibleverse{12}Ihr sollt bei meinem Namen
nicht falsch schwören, daß du den Namen deines Gottes entweihst: ich bin
der HERR!~-- \bibleverse{13}Du sollst deinen Nächsten✲ nicht
bedrücken\textless sup title=``oder: übervorteilen''\textgreater✲ und
nicht berauben; der Lohn eines Tagelöhners soll von dir nicht über Nacht
bis zum (andern) Morgen zurückbehalten werden.~-- \bibleverse{14}Du
sollst einem Tauben nicht fluchen und einem Blinden keinen Anstoß in den
Weg legen, sondern dich vor deinem Gott fürchten: ich bin der HERR!

\bibleverse{15}Begeht kein Unrecht beim Rechtsprechen; sieh die Person
eines Geringen nicht an, begünstige aber auch keinen Vornehmen, sondern
richte deinen Nächsten✲ dem Rechte gemäß. \bibleverse{16}Geh nicht als
Verleumder unter deinen Volksgenossen umher; tritt im Gericht nicht
gegen das Blut✲ deines Nächsten auf: ich bin der HERR!~--
\bibleverse{17}Du sollst gegen deinen Bruder keinen Haß in deinem Herzen
hegen, sondern sollst deinen Nächsten✲ ernstlich zurechtweisen, damit du
seinetwegen keine Verschuldung auf dich lädst. \bibleverse{18}Du sollst
den Angehörigen deines Volkes gegenüber nicht rachgierig sein und ihnen
nichts nachtragen, sondern sollst deinen Nächsten lieben wie dich
selbst: ich bin der HERR!«

\hypertarget{verschiedene-religiuxf6s-sittliche-vorschriften-besonders-verbote-heidnischer-bruxe4uche}{%
\paragraph{Verschiedene religiös-sittliche Vorschriften, besonders
Verbote heidnischer
Bräuche}\label{verschiedene-religiuxf6s-sittliche-vorschriften-besonders-verbote-heidnischer-bruxe4uche}}

\bibleverse{19}»Meine Gebote sollt ihr beobachten! Du darfst bei deinem
Vieh nicht zweierlei Arten sich paaren lassen, auch dein Feld nicht mit
zweierlei Samen besäen; und kein Kleid, das aus zweierlei
Stoffen\textless sup title=``oder: Arten von Garn''\textgreater✲ gewebt
ist, darf auf deinen Leib kommen.~-- \bibleverse{20}Wenn ein Mann bei
einem Weibe liegt und ihr beiwohnt, die eine Sklavin ist, die als
Nebenweib einem andern Manne angehört, aber weder losgekauft noch
freigelassen ist, so soll eine Bestrafung stattfinden; doch sie sollen
nicht den Tod erleiden, weil sie keine Freie gewesen ist.
\bibleverse{21}Er soll jedoch Gott dem HERRN als seine Buße einen Widder
als Schuldopfer an den Eingang des Offenbarungszeltes bringen;
\bibleverse{22}und der Priester soll ihm mittels des Schuldopferwidders
Sühne vor dem HERRN wegen seiner Sünde erwirken, die er begangen hat;
dann wird ihm seine Sünde, die er begangen hat, vergeben werden.~--

\bibleverse{23}Wenn ihr in das (verheißene) Land gekommen seid und
allerlei Obstbäume pflanzt, so sollt ihr deren Vorhaut -- d.h. ihren
Fruchtertrag -- als Vorhaut ansehen: drei Jahre lang sollen sie euch als
unbeschnitten gelten, so daß nichts von ihnen gegessen werden darf.
\bibleverse{24}Im vierten Jahr aber soll ihr ganzer Fruchtertrag dem
HERRN als Gabe des Dankes✲ geweiht sein. \bibleverse{25}Erst im fünften
Jahr dürft ihr Früchte von ihnen genießen und werdet alsdann einen um so
reicheren Ertrag erlangen: ich bin der HERR, euer Gott!~--

\bibleverse{26}Ihr dürft nichts essen, was Blut enthält. -- Ihr dürft
nicht Wahrsagerei noch Zauberei treiben.~-- \bibleverse{27}Ihr dürft
euer Haupthaar an den Schläfen nicht rund scheren; auch darfst du den
Rand deines Bartes nicht stutzen.~-- \bibleverse{28}Wegen eines Toten
dürft ihr euch keine Einschnitte an eurem Leibe machen und keine
Ätzschrift\textless sup title=``d.h. eingeätzten Bilder oder
Schriftzeichen''\textgreater✲ an euch anbringen: ich bin der HERR!~--
\bibleverse{29}Du sollst deine Tochter nicht entweihen, indem du eine
Buhldirne aus ihr machst; denn das Land soll keine Buhlerei treiben, und
das Land darf nicht voll von Unzucht werden.~--

\bibleverse{30}Meine Ruhetage sollt ihr beobachten und vor meinem
Heiligtum Ehrfurcht haben: ich bin der HERR!~-- \bibleverse{31}Wendet
euch nicht an die Totengeister\textless sup title=``oder:
Totenbeschwörer''\textgreater✲ und an die Wahrsagegeister\textless sup
title=``oder: Wahrsager''\textgreater✲; sucht sie nicht auf, damit ihr
nicht durch sie verunreinigt werdet: ich bin der HERR, euer Gott!«

\hypertarget{verschiedene-pflichten-gegen-den-nuxe4chsten}{%
\paragraph{Verschiedene Pflichten gegen den
Nächsten}\label{verschiedene-pflichten-gegen-den-nuxe4chsten}}

\bibleverse{32}»Vor einem grauen Haupte sollst du aufstehen und die
Person eines Greises ehren und dich vor deinem Gott fürchten: ich bin
der HERR!~-- \bibleverse{33}Wenn ein Fremdling sich bei dir in eurem
Lande als Gast aufhält, so sollt ihr ihn nicht bedrücken;
\bibleverse{34}wie ein Einheimischer aus eurer eigenen Mitte soll euch
der Fremdling gelten, der unter euch lebt, und du sollst ihn lieben wie
dich selbst; denn ihr seid ja selbst Fremdlinge im Lande Ägypten
gewesen: ich bin der HERR, euer Gott!

\bibleverse{35}Ihr sollt kein Unrecht verüben weder beim Rechtsprechen
noch mit dem Längenmaß\textless sup title=``=~der Elle''\textgreater✲,
mit dem Gewicht und dem Hohlmaß; \bibleverse{36}richtige Waage, richtige
Gewichtstücke, richtiges Getreide- und Flüssigkeitsmaß sollt ihr führen:
ich bin der HERR, euer Gott, der euch aus dem Lande Ägypten weggeführt
hat. \bibleverse{37}So beobachtet denn alle meine Satzungen und alle
meine Gebote und handelt nach ihnen: ich bin der HERR!«

\hypertarget{d-strafbestimmungen-besonders-fuxfcr-die-in-kap.-18-auch-19-verbotenen-vergehungen}{%
\paragraph{d) Strafbestimmungen besonders für die in Kap. 18 (auch 19)
verbotenen
Vergehungen}\label{d-strafbestimmungen-besonders-fuxfcr-die-in-kap.-18-auch-19-verbotenen-vergehungen}}

\hypertarget{section-19}{%
\section{20}\label{section-19}}

\bibleverse{1}Weiter gebot der HERR dem Mose folgendes:
\bibleverse{2}»Den Israeliten sollst du sagen: Wenn irgendeiner von den
Israeliten oder von den Fremdlingen, die in Israel als Gäste leben, eins
von seinen Kindern dem Moloch✲ hingibt, so soll er unfehlbar mit dem
Tode bestraft werden: das Volk des Landes soll ihn steinigen!
\bibleverse{3}Ich selbst will mein Angesicht gegen einen solchen
Menschen kehren und ihn aus der Mitte seines Volkes ausrotten, weil er
eins von seinen Kindern dem Moloch hingegeben und so mein Heiligtum
verunreinigt und meinen heiligen Namen entweiht hat. \bibleverse{4}Wenn
also das Volk des Landes die Augen vor einem solchen Menschen, der eins
von seinen Kindern dem Moloch hingibt, verschließen sollte, so daß es
ihn nicht tötet, \bibleverse{5}so will ich selbst mein Angesicht gegen
den betreffenden Menschen und gegen sein ganzes Geschlecht kehren und
ihn samt allen, die denselben Götzendienst treiben und sich dem Moloch
mit ihrem Götzendienst ergeben, aus der Mitte ihres Volkes ausrotten.
\bibleverse{6}Wenn sich ferner jemand an die Totenbeschwörer und die
Wahrsager\textless sup title=``vgl. 19,31''\textgreater✲ wendet, um
Götzendienst mit ihnen zu treiben, so will ich mein Angesicht gegen
einen solchen Menschen kehren und ihn aus der Mitte seines Volkes
ausrotten.«

\hypertarget{mahnung-zur-heiligung-strafen-fuxfcr-verschiedene-suxfcnden-besonders-der-unzucht-und-fuxfcr-verbrechen}{%
\paragraph{Mahnung zur Heiligung; Strafen für verschiedene Sünden
(besonders der Unzucht) und für
Verbrechen}\label{mahnung-zur-heiligung-strafen-fuxfcr-verschiedene-suxfcnden-besonders-der-unzucht-und-fuxfcr-verbrechen}}

\bibleverse{7}»So heiligt euch denn, daß ihr heilig werdet; denn ich bin
der HERR, euer Gott; \bibleverse{8}beobachtet meine Satzungen und tut
nach ihnen: ich, der HERR, bin es, der euch heiligt. \bibleverse{9}Ein
jeder, der seinem Vater oder seiner Mutter flucht, soll unfehlbar mit
dem Tode bestraft werden! Er hat seinem Vater oder seiner Mutter
geflucht: Blutschuld lastet auf ihm\textless sup title=``oder: sein Blut
komme über ihn!''\textgreater✲.

\bibleverse{10}Wenn ferner ein Mann Ehebruch mit einer verheirateten
Frau treibt, wenn er mit der Ehefrau seines Nächsten✲ Ehebruch treibt,
so sollen beide, der Ehebrecher und die Ehebrecherin, unfehlbar mit dem
Tode bestraft werden.~-- \bibleverse{11}Wenn ein Mann der Frau seines
Vaters\textless sup title=``d.h. seiner Stiefmutter''\textgreater✲
beiwohnt, so hat er seinen Vater geschändet; beide Schuldige sollen
unfehlbar mit dem Tode bestraft werden: Blutschuld lastet auf
ihnen\textless sup title=``oder: ihr Blut komme über sie''\textgreater✲.
\bibleverse{12}Wenn ein Mann seiner Schwiegertochter beiwohnt, so sollen
beide unfehlbar mit dem Tode bestraft werden; sie haben eine schändliche
Befleckung verübt: Blutschuld lastet auf ihnen\textless sup title=``vgl.
V.11''\textgreater✲.~-- \bibleverse{13}Wenn ein Mann bei einem andern
Manne liegt, wie man einem Weibe beiwohnt, so haben beide eine Greueltat
begangen; sie sollen unfehlbar mit dem Tode bestraft werden: Blutschuld
lastet auf ihnen.~-- \bibleverse{14}Wenn ein Mann eine Frau ehelicht und
zugleich ihre Mutter, so ist das eine Schandtat: man soll ihn und die
beiden Frauen verbrennen, damit keine solche Schandtat unter euch verübt
werde.~-- \bibleverse{15}Wenn sich ferner ein Mann mit einem Tiere
paart, so soll er unfehlbar mit dem Tode bestraft werden, und auch das
Tier sollt ihr töten. \bibleverse{16}Und wenn ein Weib sich irgendeinem
Tiere naht, um sich mit ihm zu paaren, so sollst du das Weib samt dem
Tiere töten; sie sollen unfehlbar mit dem Tode bestraft werden:
Blutschuld lastet auf ihnen\textless sup title=``vgl.
V.11''\textgreater✲.~-- \bibleverse{17}Wenn ein Mann seine
Schwester\textless sup title=``d.h. Stiefschwester''\textgreater✲ nimmt,
die Tochter seines Vaters oder die Tochter seiner Mutter, und den
Beischlaf im Einverständnis mit ihr vollzieht, so ist das Blutschande;
sie sollen vor den Augen ihrer Volksgenossen ausgerottet werden: er hat
seiner Schwester beigewohnt: er soll für seine Verschuldung büßen!~--
\bibleverse{18}Wenn ferner ein Mann mit einer weiblichen Person zur Zeit
ihres monatlichen Unwohlseins geschlechtlichen Umgang gehabt hat, wenn
er also den Quell ihres Blutes\textless sup title=``oder: ihren
Blutfluß''\textgreater✲ enthüllt und sie (selbst) den Quell ihres Blutes
aufgedeckt hat, so sollen beide aus der Mitte ihres Volkes ausgerottet
werden!~-- \bibleverse{19}Mit der Schwester deiner Mutter und mit der
Schwester deines Vaters darfst du keinen geschlechtlichen Umgang haben;
denn wer das tut, der hat sich mit seiner nächsten Blutsverwandten
geschlechtlich vergangen: sie sollen für ihre Verschuldung büßen.
\bibleverse{20}Wenn ferner ein Mann der Frau seines Oheims\textless sup
title=``=~seiner Muhme; vgl. 18,14''\textgreater✲ beiwohnt, so hat er
seinen Oheim geschändet; sie sollen für ihre Sünde büßen: kinderlos
werden sie sterben. \bibleverse{21}Und wenn ein Mann die Frau seines
Bruders zum Weibe nimmt, so ist das Blutschande; er hat damit seinen
Bruder geschändet: sie werden kinderlos bleiben.«

\hypertarget{abschluuxdf-mahnung-zum-heiligsein-an-israel-als-das-fuxfcr-gott-abgesonderte-volk}{%
\paragraph{Abschluß: Mahnung zum Heiligsein an Israel als das für Gott
abgesonderte
Volk}\label{abschluuxdf-mahnung-zum-heiligsein-an-israel-als-das-fuxfcr-gott-abgesonderte-volk}}

\bibleverse{22}»So beobachtet denn alle meine Satzungen und alle meine
Gebote und tut nach ihnen, damit euch das Land nicht ausspeie, in das
ich euch bringen will, damit ihr darin wohnt. \bibleverse{23}Ihr dürft
also nicht nach den Satzungen\textless sup title=``oder: in den
Bräuchen''\textgreater✲ der Völkerschaften wandeln, die ich vor euch
vertreiben werde; denn alle diese Sünden haben sie verübt, so daß sie
mir zum Ekel geworden sind. \bibleverse{24}Euch aber habe ich verheißen:
Ihr sollt ihr Land in Besitz nehmen, und ich will es euch zu eigen
geben, ein Land, das von Milch und Honig überfließt: ich bin der HERR,
euer Gott, der ich euch von den übrigen Völkern abgesondert habe.
\bibleverse{25}So macht denn einen Unterschied zwischen den reinen und
den unreinen vierfüßigen Tieren und zwischen den unreinen und den reinen
Vögeln, und macht euch nicht selbst zu einem Greuel durch (unreine)
vierfüßige Tiere oder durch Vögel oder durch irgendwelche Tiere, die
sich auf dem Erdboden regen, die ich euch als unrein bezeichnet und
ausgesondert habe. \bibleverse{26}Ihr sollt mir also heilig sein, denn
ich, der HERR, bin heilig und habe euch von den übrigen Völkern
abgesondert, damit ihr mir angehört.«

\hypertarget{zu-pv.6}{%
\paragraph{Zu \textbar pV.6}\label{zu-pv.6}}

\bibleverse{27}»Wenn ferner ein Mann oder ein Weib einen Geist der
Totenbeschwörung oder einen Wahrsagegeist in sich hat, so sollen sie
unfehlbar mit dem Tode bestraft werden; man soll sie steinigen:
Blutschuld lastet auf ihnen\textless sup title=``oder: ihr Blut komme
über sie!''\textgreater✲.«

\hypertarget{e-heilige-pflichten-der-priester-besonders-des-hohenpriesters-v.10-15}{%
\paragraph{e) Heilige Pflichten der Priester (besonders des
Hohenpriesters
V.10-15)}\label{e-heilige-pflichten-der-priester-besonders-des-hohenpriesters-v.10-15}}

\hypertarget{aa-verordnungen-in-betreff-der-verunreinigungen-durch-tote-trauergebruxe4uche-und-ehehindernisse}{%
\subparagraph{aa) Verordnungen in betreff der Verunreinigungen durch
Tote, Trauergebräuche und
Ehehindernisse}\label{aa-verordnungen-in-betreff-der-verunreinigungen-durch-tote-trauergebruxe4uche-und-ehehindernisse}}

\hypertarget{section-20}{%
\section{21}\label{section-20}}

\bibleverse{1}Weiter gebot der HERR dem Mose: »Teile den Priestern, den
Söhnen Aarons, folgende Verordnungen mit: (Ein Priester) darf sich unter
seinen Volksgenossen an keiner Leiche verunreinigen; \bibleverse{2}nur
an seinen nächsten Blutsverwandten, nämlich an seiner Mutter und seinem
Vater, an seinem Sohn und seiner Tochter und seinem Bruder,
\bibleverse{3}auch an seiner Schwester, wenn sie noch Jungfrau ist und
ihm darum nahesteht und noch keinem Manne angehört hat -- an dieser darf
er sich verunreinigen. \bibleverse{4}Er darf sich nicht als Gatte unter
seinen Volksgenossen verunreinigen, so daß er dadurch entweiht würde.
\bibleverse{5}Sie\textless sup title=``d.h. die Priester''\textgreater✲
dürfen sich an ihrem Haupt keine Glatze scheren und den Rand ihres
Bartes nicht stutzen und sich keine Einschnitte in ihren Leib machen.
\bibleverse{6}Sie sollen ihrem Gott heilig sein und den Namen ihres
Gottes nicht entweihen; denn sie haben die Feueropfer des HERRN, die
Speise ihres Gottes, darzubringen; darum sollen sie heilig sein.~--
\bibleverse{7}Eine Buhldirne oder eine Entehrte dürfen sie nicht zur
Ehefrau nehmen, ebensowenig eine von ihrem Manne verstoßene\textless sup
title=``oder: geschiedene''\textgreater✲ Frau; denn (der Priester) ist
seinem Gott geweiht. \bibleverse{8}Darum sollst du ihn für heilig
achten, denn er bringt die Speise deines Gottes dar: als heilig soll er
dir gelten, denn ich bin heilig, der HERR, der euch heiligt.
\bibleverse{9}Und wenn die Tochter eines Priesters sich durch Unzucht
entweiht, so entweiht sie dadurch ihren Vater: im Feuer soll sie
verbrannt werden!

\bibleverse{10}Der Priester aber, welcher der oberste unter seinen
Amtsgenossen ist, auf dessen Haupt das Salböl gegossen worden ist und
den man in sein Amt eingesetzt hat, damit er die heiligen Kleider
anziehe, darf sein Haupthaar (in der Trauer) nicht auflösen\textless sup
title=``d.h. frei und ungeordnet herabhangen lassen, vgl.
10,6''\textgreater✲ und seine Kleider nicht zerreißen. \bibleverse{11}Er
darf auch zu keiner Leiche hineingehen; sogar an seinem Vater und an
seiner Mutter darf er sich nicht verunreinigen. \bibleverse{12}Aus dem
Heiligtum darf er sich nicht entfernen, damit er das Heiligtum seines
Gottes nicht entweiht; denn die Weihe des Salböls seines Gottes befindet
sich auf ihm: ich bin der HERR. \bibleverse{13}Zur Ehefrau muß er sich
eine Jungfrau nehmen; \bibleverse{14}eine Witwe oder eine
verstoßene\textless sup title=``oder: geschiedene''\textgreater✲ Frau
oder eine Entehrte oder eine Buhldirne -- diese darf er nicht ehelichen;
sondern eine Jungfrau aus seinen Volksgenossen muß er sich zur Frau
nehmen, \bibleverse{15}damit er seine Nachkommenschaft unter seinen
Volksgenossen nicht entweiht; denn ich bin der HERR, der ihn heiligt.«

\hypertarget{bb-die-vom-priestertum-ausschlieuxdfenden-leibesfehler}{%
\subparagraph{bb) Die vom Priestertum ausschließenden
Leibesfehler}\label{bb-die-vom-priestertum-ausschlieuxdfenden-leibesfehler}}

\bibleverse{16}Weiter gebot der HERR dem Mose folgendes:
\bibleverse{17}»Teile dem Aaron folgende Verordnungen mit: Wenn
irgendeiner von deinen Nachkommen in ihren künftigen Geschlechtern einen
Leibesfehler an sich hat, so darf er nicht herantreten, um die Speise
seines Gottes darzubringen; \bibleverse{18}denn keiner, der ein
leibliches Gebrechen an sich hat, darf mir nahen, kein Blinder oder
Lahmer, kein im Gesicht Entstellter oder einer, an dem ein Glied zu lang
ist; \bibleverse{19}auch keiner, der einen Beinbruch oder einen Armbruch
hat, \bibleverse{20}auch kein Buckliger und kein Zwerg\textless sup
title=``oder: Schwindsüchtiger?''\textgreater✲, keiner, der weiße
Flecken im Auge hat oder der mit Krätze oder mit Flechten behaftet oder
der entmannt ist. \bibleverse{21}Keiner von den Nachkommen des Priesters
Aaron, der ein Gebrechen an sich hat, darf mir nahen, um die Feueropfer
des HERRN darzubringen; hat er ein Gebrechen an sich, so darf er mir
nicht nahen, um die Speise seines Gottes darzubringen.
\bibleverse{22}Von der Speise seines Gottes, sowohl von den hochheiligen
als auch von den heiligen Gaben, darf er essen; \bibleverse{23}doch zu
dem (inneren) Vorhang darf er nicht hineingehen und an den Altar nicht
treten, weil er ein Gebrechen an sich hat; sonst würde er meine
Heiligtümer entweihen; denn ich bin der HERR, der sie heiligt.«

\bibleverse{24}Mose teilte dies dann dem Aaron und dessen Söhnen und
allen Israeliten mit.

\hypertarget{cc-verhalten-gegenuxfcber-den-heiligen-gaben-seitens-der-priester-und-seitens-der-nichtpriester}{%
\subparagraph{cc) Verhalten gegenüber den heiligen Gaben seitens der
Priester und seitens der
Nichtpriester}\label{cc-verhalten-gegenuxfcber-den-heiligen-gaben-seitens-der-priester-und-seitens-der-nichtpriester}}

\hypertarget{section-21}{%
\section{22}\label{section-21}}

\bibleverse{1}Weiter gebot der HERR dem Mose folgendes:
\bibleverse{2}»Befiehl Aaron und seinen Söhnen, daß sie den heiligen
Gaben gegenüber, die die Israeliten mir weihen,
Zurückhaltung\textless sup title=``oder: Vorsicht''\textgreater✲
beobachten, damit sie meinen heiligen Namen nicht entweihen: ich bin der
HERR. \bibleverse{3}Sage zu ihnen: (Diese Verordnung gilt) für alle eure
künftigen Geschlechter: Wenn irgendeiner von all euren Nachkommen den
heiligen Gaben, welche die Israeliten dem HERRN weihen, nahekommt,
während er eine Unreinheit an sich hat, ein solcher Mensch soll von
meinem Angesicht hinweg ausgerottet werden: ich bin der HERR!
\bibleverse{4}Wer von den Nachkommen Aarons mit Aussatz oder mit einem
Ausfluß behaftet ist, darf von den heiligen Gaben nicht mitessen, bis er
wieder rein ist; und wer irgendeinen durch eine Leiche Verunreinigten
berührt oder wer einen Samenerguß gehabt hat \bibleverse{5}oder wer
irgendein kriechendes Tier berührt hat, durch das man unrein wird, oder
einen Menschen, durch den man in irgendeiner Beziehung unrein wird:~--
\bibleverse{6}einer, der etwas Derartiges berührt hat, ist bis zum Abend
unrein und darf nichts von den heiligen Gaben genießen, es sei denn, daß
er zuvor ein Wasserbad genommen hat. \bibleverse{7}Nach Sonnenuntergang
aber ist er wieder rein und darf alsdann von den heiligen Gaben essen;
denn sie sind die ihm zukommende Speise. \bibleverse{8}Ein
gefallenes\textless sup title=``oder: verendetes''\textgreater✲ oder
(von Raubtieren) zerrissenes Tier darf er nicht essen; er würde dadurch
unrein werden: ich bin der HERR. \bibleverse{9}So sollen sie denn meine
Verordnungen befolgen\textless sup title=``oder: beobachten, was mir
gegenüber zu beobachten ist''\textgreater✲, damit sie nicht wegen einer
Übertretung Sünde auf sich laden und infolge einer Entweihung des
Geheiligten sterben müssen: ich bin der HERR, der sie heiligt!
\bibleverse{10}Kein Unbefugter darf etwas Geheiligtes genießen; kein
Beisasse✲ eines Priesters und keiner von seinen Tagelöhnern darf etwas
Geheiligtes genießen. \bibleverse{11}Wenn aber ein Priester einen
Sklaven für Geld erwirbt, so darf dieser davon (mit)essen; ebenso dürfen
die in seinem Hause geborenen Sklaven von seiner Speise (mit)essen.
\bibleverse{12}Wenn ferner die Tochter eines Priesters einen
Nichtpriester geheiratet hat, so darf sie von den Hebeopfern der
heiligen Gaben nichts genießen. \bibleverse{13}Wenn aber die Tochter
eines Priesters Witwe oder (von ihrem Mann) geschieden wird, ohne Kinder
zu haben, und in das Haus ihres Vaters zurückkehrt, so darf sie, gerade
wie in ihrer Jugend, von der Speise ihres Vaters (mit)essen; dagegen
darf kein Unbefugter etwas davon genießen. \bibleverse{14}Wenn ferner
jemand aus Versehen etwas Geheiligtes genießt, so soll er den fünften
Teil des Wertes hinzufügen und das Geheiligte dem Priester erstatten.
\bibleverse{15}Die Priester aber sollen die heiligen Gaben der
Israeliten, die sie als Hebeopfer für den HERRN empfangen, nicht
entweihen, \bibleverse{16}damit sie ihnen\textless sup title=``d.h. den
Israeliten''\textgreater✲ kein Vergehen und keine Schuld dadurch
aufladen, daß sie ihre heiligen Gaben genießen; denn ich bin der HERR,
der sie heiligt.«

\hypertarget{f-heilige-pflichten-der-laien-bezuxfcglich-der-beschaffenheit-der-opfer}{%
\paragraph{f) Heilige Pflichten der Laien bezüglich der Beschaffenheit
der
Opfer}\label{f-heilige-pflichten-der-laien-bezuxfcglich-der-beschaffenheit-der-opfer}}

\hypertarget{aa-fehlerlosigkeit-der-opfertiere}{%
\subparagraph{aa) Fehlerlosigkeit der
Opfertiere}\label{aa-fehlerlosigkeit-der-opfertiere}}

\bibleverse{17}Weiter gebot der HERR dem Mose folgendes:
\bibleverse{18}»Teile dem Aaron und seinen Söhnen und allen Israeliten
folgende Verordnungen mit: Wenn irgend jemand vom Hause Israel oder von
den Fremdlingen in Israel seine Opfergabe darbringen will -- es seien
irgendwelche gelobte oder freiwillige Gaben, die sie dem HERRN als
Brandopfer darbringen wollen --, \bibleverse{19}so muß es, um euch
wohlgefällig zu machen, ein fehlerloses männliches Tier von den Rindern,
von den Schafen oder den Ziegen sein. \bibleverse{20}Kein Tier, das
einen Fehler an sich hat, dürft ihr darbringen; denn es würde euch nicht
wohlgefällig machen. \bibleverse{21}Und wenn jemand dem HERRN ein
Heilsopfer darbringen will, ein Rind oder ein Stück Kleinvieh, sei es um
ein Gelübde zu erfüllen oder als freiwillige Gabe, so darf es, um
wohlgefällig zu sein, keinen Fehler, keinerlei Gebrechen an sich haben.
\bibleverse{22}Ist es blind oder hat es ein gebrochenes Glied oder einen
Wundschaden oder ist es mit Geschwüren oder mit Krätze oder mit Flechten
behaftet: derartige Tiere dürft ihr dem HERRN nicht darbringen und kein
Feueropfer von ihnen dem HERRN auf den Altar legen. \bibleverse{23}Ein
Rind oder ein Stück Kleinvieh, an dem ein Glied zu lang oder zu kurz
(verkrüppelt) ist, magst du als freiwillige Gabe opfern, aber als
Gelübdeopfer würde es nicht wohlgefällig aufgenommen werden.
\bibleverse{24}Ein Tier ferner, dem die Hoden zerquetscht oder
zerschlagen oder ausgerissen oder ausgeschnitten sind, dürft ihr dem
HERRN nicht darbringen; weder dürft ihr Tiere in eurem Lande zu
verstümmelten (Tieren) machen \bibleverse{25}noch solche von einem
Ausländer kaufen und sie als Speise eurem Gott darbringen; denn eine
Verstümmlung, ein Gebrechen haftet ihnen an; sie würden euch nicht
wohlgefällig machen\textless sup title=``oder: wohlgefällig aufgenommen
werden''\textgreater✲.«

\hypertarget{bb-drei-weitere-opfervorschriften}{%
\subparagraph{bb) Drei weitere
Opfervorschriften}\label{bb-drei-weitere-opfervorschriften}}

\bibleverse{26}Weiter gebot der HERR dem Mose folgendes:
\bibleverse{27}»Ein Rind oder ein Schaf oder ein Ziegenlamm soll nach
der Geburt sieben Tage lang unter✲ seiner Mutter bleiben; erst vom
achten Tage an und weiterhin wird es (als Opfergabe) wohlgefällig sein,
wenn man es als Feueropfer dem HERRN darbringt.~-- \bibleverse{28}Ein
Rind oder ein Stück Kleinvieh dürft ihr nicht zugleich mit seinem Jungen
an einem und demselben Tage schlachten.~-- \bibleverse{29}Wenn ihr
ferner dem HERRN ein Dankschlachtopfer darbringen wollt, sollt ihr es so
opfern, daß ihr Wohlgefallen dadurch erlangt: \bibleverse{30}es muß noch
an demselben Tage verzehrt werden; ihr dürft nichts davon bis zum andern
Morgen übriglassen: ich bin der HERR!«

\hypertarget{cc-allgemeine-schluuxdfermahnung}{%
\subparagraph{cc) Allgemeine
Schlußermahnung}\label{cc-allgemeine-schluuxdfermahnung}}

\bibleverse{31}»So beobachtet denn meine Gebote und tut nach ihnen: ich
bin der HERR! \bibleverse{32}Und entweiht meinen heiligen Namen nicht,
damit ich inmitten der Israeliten geheiligt werde: ich bin der HERR, der
euch heiligt, \bibleverse{33}der euch aus Ägypten herausgeführt hat, um
euer Gott zu sein: ich bin der HERR!«

\hypertarget{g-gesetze-fuxfcr-die-feier-des-sabbats-und-der-jahresfeste}{%
\paragraph{g) Gesetze für die Feier des Sabbats und der
Jahresfeste}\label{g-gesetze-fuxfcr-die-feier-des-sabbats-und-der-jahresfeste}}

\hypertarget{section-22}{%
\section{23}\label{section-22}}

\bibleverse{1}Hierauf gebot der HERR dem Mose folgendes:
\bibleverse{2}»Teile den Israeliten folgende Verordnungen mit: Die Feste
des HERRN, die ihr als Festversammlungen am Heiligtum ausrufen sollt,
meine Feste, sind folgende:«

\hypertarget{aa-der-sabbat}{%
\subparagraph{aa) Der Sabbat}\label{aa-der-sabbat}}

\bibleverse{3}Sechs Tage hindurch soll gearbeitet werden, aber der
siebte Tag ist ein Tag völliger Ruhe\textless sup title=``=~ein hoher
Feiertag''\textgreater✲ mit Versammlung am Heiligtum; da dürft ihr
keinerlei Arbeit verrichten: es ist ein Ruhetag zu Ehren des HERRN in
allen euren Wohnsitzen.«

\hypertarget{bb-das-passah-und-das-fest-der-ungesuxe4uerten-brote}{%
\subparagraph{bb) Das Passah und das Fest der ungesäuerten
Brote}\label{bb-das-passah-und-das-fest-der-ungesuxe4uerten-brote}}

\bibleverse{4}»Folgendes sind die Feste des HERRN mit Versammlungen am
Heiligtum, die ihr zu dem für sie festgesetzten Zeitpunkt ausrufen
sollt: \bibleverse{5}Am vierzehnten Tag des ersten Monats gegen Abend
findet die Passahfeier für den HERRN statt; \bibleverse{6}und am
fünfzehnten Tage desselben Monats wird das Fest der ungesäuerten Brote
zu Ehren des HERRN gefeiert; da sollt ihr sieben Tage lang ungesäuerte
Brote essen. \bibleverse{7}Am ersten Tage habt ihr eine Festversammlung
am Heiligtum zu halten: da dürft ihr keinerlei Werktagsarbeit verrichten
\bibleverse{8}und sollt dem HERRN sieben Tage lang ein Feueropfer
darbringen. Am siebten Tage soll wieder eine Festversammlung am
Heiligtum stattfinden: da dürft ihr keinerlei Werktagsarbeit
verrichten.«

\hypertarget{cc-darbringung-der-erstlingsgarbe-in-der-woche-des-festes-der-ungesuxe4uerten-brote}{%
\subparagraph{cc) Darbringung der Erstlingsgarbe (in der Woche des
Festes der ungesäuerten
Brote)}\label{cc-darbringung-der-erstlingsgarbe-in-der-woche-des-festes-der-ungesuxe4uerten-brote}}

\bibleverse{9}Weiter gebot der HERR dem Mose folgendes:
\bibleverse{10}»Teile den Israeliten folgende Verordnungen mit: Wenn ihr
in das Land kommt, das ich euch geben werde, und ihr die Ernte dort
abhaltet, so sollt ihr von eurer Ernte die Erstlingsgarbe zum Priester
bringen. \bibleverse{11}Dieser soll dann die gespendete Garbe vor dem
HERRN weben✲, damit sie euch wohlgefällig mache; am Tage nach dem Sabbat
soll der Priester sie weben; \bibleverse{12}und ihr sollt an dem Tage,
an welchem ihr die Garbe weben laßt, dem HERRN ein fehlerloses
einjähriges Lamm als Brandopfer darbringen; \bibleverse{13}dazu als
Speisopfer für ihn zwei Zehntel Epha Feinmehl, das mit Öl gemengt ist,
als ein Feueropfer für den HERRN zu lieblichem Geruch; dazu als
Trankopfer für ihn ein Viertel Hin Wein. \bibleverse{14}Brot (vom neuen
Getreide) und geröstete oder zerstoßene Körner (der frischen Frucht)
dürft ihr bis zu eben diesem Tage, bis ihr eurem Gott die Opfergabe
dargebracht habt, nicht essen: diese Verordnung soll ewige Geltung für
eure künftigen Geschlechter in allen euren Wohnsitzen haben.«

\hypertarget{dd-das-wochen--oder-pfingstfest}{%
\subparagraph{dd) Das Wochen- oder
Pfingstfest}\label{dd-das-wochen--oder-pfingstfest}}

\bibleverse{15}»Hierauf sollt ihr euch vom Tage nach dem Sabbat an, von
dem Tage an, wo ihr die Webegarbe dargebracht habt, sieben Wochen
abzählen: volle Wochen sollen es sein; \bibleverse{16}bis zu dem Tage,
der auf den siebten Sabbat folgt, sollt ihr fünfzig Tage abzählen und
dann dem HERRN ein Speisopfer vom neuen Getreide darbringen.
\bibleverse{17}Aus euren Wohnsitzen sollt ihr zwei Webebrote darbringen,
die aus zwei Zehntel Epha Feinmehl hergestellt und mit Sauerteig
gebacken sein müssen, als Erstlingsgaben für den HERRN.
\bibleverse{18}Weiter sollt ihr zu den Broten sieben fehlerlose
einjährige Lämmer und einen jungen Stier und zwei Widder darbringen --
die sollen für den HERRN ein Brandopfer sein -- samt dem zugehörigen
Speisopfer und den erforderlichen Trankopfern, als ein Feueropfer zu
lieblichem Geruch für den HERRN. \bibleverse{19}Ferner sollt ihr einen
Ziegenbock zum Sündopfer und zwei einjährige Lämmer zum Heilsopfer
herrichten. \bibleverse{20}Der Priester soll sie dann samt den
Erstlingsbroten als Webeopfer vor dem HERRN weben✲ zugleich mit den
beiden Lämmern; sie sollen dem HERRN geheiligt sein und dem Priester
gehören. \bibleverse{21}Und ihr sollt an eben diesem Tage ausrufen
lassen: ›Eine Festversammlung am Heiligtum sollt ihr abhalten und
keinerlei Werktagsarbeit verrichten!‹ Diese Verordnung soll ewige
Geltung in allen euren Wohnsitzen für eure künftigen Geschlechter
haben!~-- \bibleverse{22}Wenn ihr aber die Ernte eures Landes
schneidet\textless sup title=``oder: abhaltet''\textgreater✲, sollst du
das Feld nicht bis zum äußersten Rand abernten und auch keine Nachlese
nach deiner Ernte vornehmen; nein, dem Armen und dem Fremdling sollst du
beides überlassen: ich bin der HERR, euer Gott.«\textless sup
title=``vgl. 19,9''\textgreater✲

\hypertarget{ee-der-neumondstag-des-siebenten-monats-oder-das-neujahrsfest}{%
\subparagraph{ee) Der Neumondstag des siebenten Monats (oder das
Neujahrsfest)}\label{ee-der-neumondstag-des-siebenten-monats-oder-das-neujahrsfest}}

\bibleverse{23}Weiter gebot der HERR dem Mose folgendes:
\bibleverse{24}»Teile den Israeliten folgende Verordnungen mit: Am
ersten Tage des siebten Monats soll bei euch ein Ruhetag sein, ein
Gedenktag mit Posaunenschall, eine Festversammlung am Heiligtum.
\bibleverse{25}Da dürft ihr keinerlei Werktagsarbeit verrichten und
sollt dem HERRN ein Feueropfer darbringen.«

\hypertarget{ff-der-grouxdfe-versuxf6hnungstag}{%
\subparagraph{ff) Der große
Versöhnungstag}\label{ff-der-grouxdfe-versuxf6hnungstag}}

\bibleverse{26}Weiter gebot der HERR dem Mose folgendes:
\bibleverse{27}»Sodann fällt auf den zehnten Tag desselben siebten
Monats der Versöhnungstag; da sollt ihr eine Festversammlung am
Heiligtum halten und sollt fasten\textless sup title=``eig.: eure Seelen
beugen, vgl. 16,29''\textgreater✲ und dem HERRN ein Feueropfer
darbringen. \bibleverse{28}Keinerlei Arbeit dürft ihr an eben diesem
Tage verrichten, denn es ist der Versöhnungstag, an dem man euch Sühne
vor dem HERRN, eurem Gott, erwirken soll. \bibleverse{29}Denn wer an
eben diesem Tage nicht fastet, der soll aus seinen Volksgenossen
ausgerottet werden; \bibleverse{30}und wer irgendeine Arbeit an eben
diesem Tage verrichtet, einen solchen Menschen will ich aus der Mitte
seines Volkes vertilgen. \bibleverse{31}Keinerlei Arbeit dürft ihr
verrichten; diese Verordnung soll ewige Geltung für eure künftigen
Geschlechter in allen euren Wohnsitzen haben. \bibleverse{32}Ein Tag
völliger Ruhe\textless sup title=``=~ein hoher oder: der höchste
Feiertag''\textgreater✲ soll es für euch sein, und ihr sollt fasten! Am
neunten Tage des Monats, am Abend, von einem Abend bis wieder zum Abend,
sollt ihr den euch gebotenen Ruhetag halten!«

\hypertarget{gg-das-laubhuxfcttenfest-oder-das-fest-der-lese}{%
\subparagraph{gg) Das Laubhüttenfest (oder: das Fest der
Lese)}\label{gg-das-laubhuxfcttenfest-oder-das-fest-der-lese}}

\bibleverse{33}Weiter gebot der HERR dem Mose folgendes:
\bibleverse{34}»Teile den Israeliten folgende Verordnungen mit: Am
fünfzehnten Tage desselben siebten Monats findet das (Laub-)Hüttenfest
sieben Tage lang zu Ehren des HERRN statt. \bibleverse{35}Am ersten Tage
soll eine Festversammlung am Heiligtum stattfinden; da dürft ihr
keinerlei Werktagsarbeit verrichten. \bibleverse{36}Sieben Tage hindurch
sollt ihr dem HERRN ein Feueropfer darbringen, dann am achten Tage
nochmals eine Festversammlung am Heiligtum abhalten und dem HERRN ein
Feueropfer darbringen. Es ist dies der Schlußfesttag, an dem ihr
keinerlei Werktagsarbeit verrichten dürft.«

\hypertarget{hh-abschluuxdf-des-festkalenders}{%
\subparagraph{hh) Abschluß des
Festkalenders}\label{hh-abschluuxdf-des-festkalenders}}

\bibleverse{37}»Dies sind die Feste des HERRN, zu denen ihr
Festversammlungen am Heiligtum ausrufen sollt, um dem HERRN Feueropfer
darzubringen: Brandopfer und Speisopfer, Schlachtopfer und Trankopfer,
wie sie für jeden einzelnen Tag geboten sind, \bibleverse{38}abgesehen
von den Sabbaten des HERRN und abgesehen von euren Gaben sowie von all
euren Gelübdeopfern und abgesehen von all euren freiwilligen Gaben, die
ihr dem HERRN darbringen werdet.«

\hypertarget{ii-nachtruxe4gliche-bestimmungen-uxfcber-das-laubhuxfcttenfest}{%
\subparagraph{ii) Nachträgliche Bestimmungen über das
Laubhüttenfest}\label{ii-nachtruxe4gliche-bestimmungen-uxfcber-das-laubhuxfcttenfest}}

\bibleverse{39}»Jedoch am fünfzehnten Tage des siebten Monats, wenn ihr
den Ertrag des Landes eingeerntet habt, sollt ihr das Fest des HERRN
sieben Tage lang feiern. Am ersten Tage soll Ruhetag sein und ebenso am
achten Tage; \bibleverse{40}und ihr sollt euch am ersten Tage schöne
Baumfrüchte holen, Palmenwedel und Zweige von dichtbelaubten Bäumen und
von Bachweiden und sollt sieben Tage lang vor dem HERRN, eurem Gott,
fröhlich sein. \bibleverse{41}Dies Fest sollt ihr alljährlich sieben
Tage lang zu Ehren des HERRN feiern; diese Verordnung hat ewige Geltung
für alle eure künftigen Geschlechter: im siebten Monat sollt ihr es
feiern. \bibleverse{42}Da sollt ihr sieben Tage lang in (Laub-)Hütten
wohnen; alle, die zum Volk Israel gehören, sollen in (Laub-)Hütten
wohnen, \bibleverse{43}damit eure künftigen Geschlechter erfahren, daß
ich die Kinder Israel habe in Hütten wohnen lassen, als ich sie aus
Ägypten wegführte, ich, der HERR, euer Gott.«

\bibleverse{44}So belehrte Mose die Israeliten über die Festzeiten des
HERRN.

\hypertarget{h-vorschriften-bezuxfcglich-der-zurichtung-des-heiligen-leuchters-und-der-schaubrote}{%
\paragraph{h) Vorschriften bezüglich der Zurichtung des heiligen
Leuchters und der
Schaubrote}\label{h-vorschriften-bezuxfcglich-der-zurichtung-des-heiligen-leuchters-und-der-schaubrote}}

\hypertarget{section-23}{%
\section{24}\label{section-23}}

\bibleverse{1}Weiter gebot der HERR dem Mose folgendes:
\bibleverse{2}»Gib den Israeliten die Weisung, dir reines Öl von
zerstoßenen Oliven für den Leuchter zu bringen, damit beständig Lampen
aufgesetzt werden können. \bibleverse{3}Außerhalb des Vorhangs vor der
Lade mit dem Gesetz im Offenbarungszelt soll Aaron (den Leuchter)
zurichten, damit er vom Abend bis zum Morgen ohne Unterbrechung vor dem
HERRN brenne\textless sup title=``vgl. 2.Mose 27,20-21''\textgreater✲.
Diese Verordnung hat ewige Geltung für alle eure künftigen Geschlechter:
\bibleverse{4}auf dem Leuchter von reinem Gold soll er die Lampen zu
beständigem Brennen vor dem HERRN zurichten.

\bibleverse{5}Sodann sollst du Feinmehl nehmen und daraus zwölf Kuchen
backen; zwei Zehntel Epha sollen auf jeden Kuchen kommen.
\bibleverse{6}Du sollst sie dann in zwei Schichten, je sechs in einer
Schicht, auf den Tisch von reinem Gold vor dem HERRN auflegen.
\bibleverse{7}Dann sollst du auf jede Schicht reinen Weihrauch tun, der
für das Brot als Duftteil dienen soll, als ein Feueropfer für den HERRN.
\bibleverse{8}An jedem Sabbattage soll er (die Brote) regelmäßig so vor
dem HERRN aufschichten; zu dieser Leistung sollen die Israeliten für
ewige Zeiten verpflichtet sein. \bibleverse{9}Die Brote sollen dann
Aaron und seinen Söhnen gehören; die sollen sie an heiliger Stätte
verzehren; denn als ein Hochheiliges sollen sie ihm von den Feueropfern
des HERRN zuteil werden als eine ewige Gebühr.«

\hypertarget{i-steinigung-eines-gottesluxe4sterers-bestrafung-des-totschlags-und-der-leibesverletzung-guxfcltigkeit-des-mosaischen-gesetzes-auch-fuxfcr-fremde}{%
\paragraph{i) Steinigung eines Gotteslästerers; Bestrafung des
Totschlags und der Leibesverletzung; Gültigkeit des mosaischen Gesetzes
auch für
Fremde}\label{i-steinigung-eines-gottesluxe4sterers-bestrafung-des-totschlags-und-der-leibesverletzung-guxfcltigkeit-des-mosaischen-gesetzes-auch-fuxfcr-fremde}}

\bibleverse{10}Der Sohn einer Israelitin -- er war aber der Sohn eines
Ägypters -- begab sich einst unter die Israeliten; da gerieten sie im
Lager in Streit miteinander, der Sohn der Israelitin und ein
israelitischer Mann. \bibleverse{11}Dabei lästerte der Sohn der
Israelitin den Namen (des HERRN) und fluchte dazu; da brachte man ihn
vor Mose -- seine Mutter aber hieß Selomith und war die Tochter Dibris,
vom Stamme Dan. \bibleverse{12}Hierauf legten sie ihn in Gewahrsam, bis
Mose ihnen Verhaltungsmaßregeln auf Grund einer Weisung des HERRN gäbe.
\bibleverse{13}Da gebot der HERR dem Mose: \bibleverse{14}»Laß den
Lästerer vor das Lager hinausführen, und alle, die es gehört haben,
sollen ihm die Hände fest auf den Kopf legen, und dann soll die ganze
Gemeinde ihn steinigen. \bibleverse{15}Zu den Israeliten aber sollst du
sagen: Wenn jemand seinem Gott flucht, so lädt er Sünde auf sich,
\bibleverse{16}und wer den Namen des HERRN lästert, soll unfehlbar mit
dem Tode bestraft werden: die ganze Gemeinde soll ihn steinigen; der
Fremde wie der Einheimische soll den Tod erleiden, wenn er den Namen
(des HERRN) lästert.~-- \bibleverse{17}Wenn ferner jemand irgendeinen
Menschen erschlägt, soll er unfehlbar mit dem Tode bestraft werden;
\bibleverse{18}wer aber ein Stück Vieh erschlägt, soll es ersetzen:
Leben um Leben\textless sup title=``d.h. ein lebendes Stück für das
tote''\textgreater✲. \bibleverse{19}Wenn ferner jemand seinem
Nächsten\textless sup title=``oder: Volksgenossen''\textgreater✲ einen
Leibesschaden zufügt, so soll man ihm ebenso tun, wie er getan hat:
\bibleverse{20}Bruch um Bruch, Auge um Auge, Zahn um Zahn; derselbe
Leibesschaden, den er dem andern zugefügt hat, soll auch ihm zugefügt
werden. \bibleverse{21}Wer also ein Stück Vieh erschlägt, soll es
ersetzen; wer aber einen Menschen erschlägt, soll den Tod erleiden.
\bibleverse{22}Das gleiche Recht soll bei euch für den Fremden wie für
den Einheimischen gelten; denn ich bin der HERR, euer Gott!«

\bibleverse{23}Als Mose dies den Israeliten verkündigt hatte, führten
sie den Lästerer vor das Lager hinaus und steinigten ihn dort; die
Israeliten taten so, wie der HERR dem Mose geboten hatte.

\hypertarget{k-vorschriften-bezuxfcglich-der-heiligen-jahre}{%
\paragraph{k) Vorschriften bezüglich der heiligen
Jahre}\label{k-vorschriften-bezuxfcglich-der-heiligen-jahre}}

\hypertarget{aa-das-sabbatjahr}{%
\subparagraph{aa) Das Sabbatjahr}\label{aa-das-sabbatjahr}}

\hypertarget{section-24}{%
\section{25}\label{section-24}}

\bibleverse{1}Und der HERR gebot dem Mose auf dem Berge Sinai folgendes:
\bibleverse{2}»Teile den Israeliten folgende Verordnungen mit: Wenn ihr
in das Land kommt, das ich euch geben werde, so soll das Land dem HERRN
einen Sabbat\textless sup title=``=~eine Ruhezeit''\textgreater✲ halten.
\bibleverse{3}Sechs Jahre sollst du dein Feld bestellen und sechs Jahre
deinen Weinberg beschneiden und den Ertrag des Landes einbringen;
\bibleverse{4}aber im siebten Jahre soll das Land einen Sabbat der
Ruhe\textless sup title=``=~völlige Ruhe''\textgreater✲ haben, eine dem
HERRN geweihte Ruhezeit: da darfst du dein Feld nicht bestellen und
deinen Weinberg nicht beschneiden; \bibleverse{5}auch den
Wildwuchs\textless sup title=``=~den Nachwuchs''\textgreater✲ deiner
(vorjährigen) Ernte darfst du nicht einheimsen und die Trauben von
deinem unbeschnittenen Weinstock nicht lesen: es soll ein Sabbatjahr✲
für das Land sein. \bibleverse{6}Was das Land während seiner Ruhezeit
von selbst hervorbringt, soll euch zur Nahrung dienen, dir sowie deinen
Knechten und Mägden, deinen Tagelöhnern und den Beisassen, die bei dir
als Gäste leben; \bibleverse{7}auch deinem Vieh und den wilden Tieren,
die in deinem Lande leben, soll der gesamte Ertrag (dieses Jahres) zur
Nahrung dienen.«

\hypertarget{bb-das-halljahr}{%
\subparagraph{bb) Das Halljahr}\label{bb-das-halljahr}}

\bibleverse{8}»Sodann sollst du dir sieben solcher
Sabbatjahre\textless sup title=``=~Ruhejahre oder:
Jahrsabbate''\textgreater✲, also siebenmal sieben Jahre, abzählen, so
daß dir die Zeit der sieben Sabbatjahre neunundvierzig Jahre beträgt.
\bibleverse{9}Dann sollst du am zehnten Tage des siebten Monats die
Lärmposaune erschallen lassen; am Versöhnungstage sollt ihr die Posaunen
überall in eurem Lande erschallen lassen \bibleverse{10}und so das
fünfzigste Jahr heiligen, und sollt im Lande Freiheit\textless sup
title=``oder: Befreiung''\textgreater✲ für alle seine Bewohner ausrufen:
ein Halljahr\textless sup title=``oder: Jobeljahr''\textgreater✲ soll es
für euch sein, in dem ein jeder von euch wieder zu seinem Besitz kommen
und ein jeder zu seiner Familie zurückkehren soll. \bibleverse{11}Ein
Halljahr soll also jedes fünfzigste Jahr für euch sein; da dürft ihr
weder säen, noch das, was von selbst gewachsen ist, einernten, noch
Trauben von den unbeschnittenen Weinstöcken lesen; \bibleverse{12}denn
ein Halljahr ist es: es soll euch heilig sein; vom Felde weg sollt ihr
essen, was es von selbst hervorbringt.

\bibleverse{13}In solchem Halljahr soll ein jeder von euch wieder zu
seinem Besitz kommen. \bibleverse{14}Wenn du also deinem
Nächsten\textless sup title=``oder: Volksgenossen''\textgreater✲ etwas
verkaufst oder von deinem Nächsten etwas kaufst, so sollt ihr einander
nicht übervorteilen, \bibleverse{15}sondern nach der Zahl der Jahre, die
seit dem (letzten) Halljahr verflossen sind, sollst du von deinem
Nächsten kaufen, und nach der Zahl der Erntejahre soll er dir verkaufen.
\bibleverse{16}Bei einer größeren Zahl von Jahren (bis zum nächsten
Halljahr) sollst du den Kaufpreis für ein Grundstück verhältnismäßig
erhöhen und bei einer kleineren Zahl von Jahren den Kaufpreis
verhältnismäßig verringern; denn er verkauft dir nur eine (bestimmte)
Anzahl von Ernten. \bibleverse{17}Keiner übervorteile also den andern,
sondern du sollst dich vor deinem Gott fürchten; denn ich bin der HERR,
euer Gott.

\bibleverse{18}So verfahrt denn nach meinen Anordnungen und beobachtet
meine Gebote und handelt nach ihnen, so werdet ihr in eurem Lande in
Sicherheit wohnen. \bibleverse{19}Dann wird das Land euch seinen Ertrag
geben, so daß ihr euch satt essen könnt und in Sicherheit darin wohnt.
\bibleverse{20}Wenn ihr aber fragt: ›Wovon sollen wir uns denn im
siebten Jahre nähren, wenn wir doch weder säen noch den erforderlichen
Ertrag einernten dürfen?‹, \bibleverse{21}so sollt ihr wissen: ich werde
euch im sechsten Jahre meinen Segen zuwenden, daß es euch den Ertrag für
drei Jahre liefern soll. \bibleverse{22}Obgleich ihr daher erst im
achten Jahre säet, werdet ihr doch immer noch von dem früheren Ertrage
altes Getreide zu essen haben; bis ins neunte Jahr, bis dessen Ernte
eingebracht ist, werdet ihr altes Getreide zu essen haben.

\bibleverse{23}Der Landbesitz darf also nicht für immer verkauft werden,
denn mir gehört das Land: ihr seid ja nur Fremdlinge und Beisassen bei
mir. \bibleverse{24}Daher sollt ihr in dem ganzen Lande, das ihr
innehabt, für euren Landbesitz die Wiedereinlösung gestatten.«

\hypertarget{cc-einluxf6sung-oder-ruxfcckfall-von-land--und-hausbesitz}{%
\subparagraph{cc) Einlösung oder Rückfall von Land- und
Hausbesitz}\label{cc-einluxf6sung-oder-ruxfcckfall-von-land--und-hausbesitz}}

\bibleverse{25}»Wenn einer deiner Volksgenossen verarmt und etwas von
seinem Grundbesitz verkauft, so soll sein nächster Verwandter als Löser
für ihn eintreten und das wieder einlösen\textless sup title=``=~für ihn
zurückkaufen''\textgreater✲ dürfen, was sein Verwandter verkauft hat.
\bibleverse{26}Wenn ferner jemand keinen Löser hat, aber selbst soviel
Geld aufzubringen vermag, als zur Wiedereinlösung seines Besitzes
erforderlich ist, \bibleverse{27}so soll er die Jahre, die seit seinem
Verkauf verflossen sind, in Anrechnung bringen und das Überschüssige
demjenigen zurückzahlen, an den er verkauft hat, damit er so wieder zu
seinem Besitz kommt. \bibleverse{28}Wenn er aber nicht soviel Geld
aufbringen kann, als zum Rückkauf erforderlich ist, so soll das von ihm
verkaufte Grundstück im Besitz des Käufers bis zum Halljahr verbleiben;
aber im Halljahr soll es frei werden, so daß er wieder zu seinem
Eigentum kommt.

\bibleverse{29}Wenn ferner jemand ein Wohnhaus in einer ummauerten Stadt
verkauft, so soll das Rückkaufsrecht für ihn bestehen, bis ein Jahr nach
dem Verkauf des Hauses vergangen ist: ein volles Jahr soll das
Rückkaufsrecht für ihn bestehen. \bibleverse{30}Wenn ein Rückkauf aber
bis zum Ablauf eines vollen Jahres nicht stattgefunden hat, so soll das
Haus, das in einer ummauerten Stadt liegt, dem Käufer und seinen
Nachkommen für immer verbleiben: es soll im Halljahr nicht frei werden.
\bibleverse{31}Dagegen die Häuser in den Gehöften\textless sup
title=``oder: Dörfern''\textgreater✲, die von keiner Mauer umgeben sind,
sollen als zum Feldbesitz gehörig angesehen werden: es soll
(unbeschränktes) Rückkaufsrecht für sie gelten, und im Halljahr sollen
sie frei werden. \bibleverse{32}Was ferner die Städte der Leviten, die
Häuser in den Städten betrifft, die ihnen zum Eigentum überwiesen sind,
so soll den Leviten ein ewiges Rückkaufsrecht zustehen.
\bibleverse{33}Wenn jedoch einer von den Leviten sein verkauftes Haus
nicht wieder einlöst, so soll es, wenn es in einer ihm zugewiesenen
Stadt liegt, im Halljahr wieder frei werden; denn die Häuser in den
Städten der Leviten sind ihr Erbbesitz inmitten der Israeliten.
\bibleverse{34}Das zu ihren Städten gehörende Weideland aber darf
überhaupt nicht verkauft werden, denn es gehört ihnen als ewiges
Eigentum.«

\hypertarget{dd-gebot-der-hilfsbereitschaft-gegen-verarmte-israeliten-loskauf-von-hebruxe4ischen-sklaven-oder-deren-freilassung-im-halljahr}{%
\subparagraph{dd) Gebot der Hilfsbereitschaft gegen verarmte Israeliten;
Loskauf von hebräischen Sklaven oder deren Freilassung im
Halljahr}\label{dd-gebot-der-hilfsbereitschaft-gegen-verarmte-israeliten-loskauf-von-hebruxe4ischen-sklaven-oder-deren-freilassung-im-halljahr}}

\bibleverse{35}»Wenn ferner einer deiner Volksgenossen verarmt, so daß
er sich neben dir nicht zu halten vermag, so sollst du ihn unterstützen,
so daß er wie ein Fremdling oder Beisasse neben dir lebt.
\bibleverse{36}Du darfst nicht Zins und Aufschlag\textless sup
title=``oder: Wucher''\textgreater✲ von ihm nehmen, sondern sollst dich
vor deinem Gott fürchten, damit dein Bruder neben dir leben kann.
\bibleverse{37}Du darfst ihm dein Geld nicht um Zins geben und deine
Nahrungsmittel nicht um Aufschlag\textless sup title=``oder:
Wucher''\textgreater✲: \bibleverse{38}ich bin der HERR, euer Gott, der
euch aus dem Lande Ägypten geführt hat, um euch das Land Kanaan zu geben
und euer Gott zu sein!

\bibleverse{39}Wenn ferner einer deiner Volksgenossen neben dir verarmt
und sich dir verkauft, so sollst du ihn keine Sklavenarbeit verrichten
lassen, \bibleverse{40}nein, wie ein Tagelöhner, wie ein Beisasse soll
er bei dir sein: nur bis zum Halljahr soll er bei dir dienen;
\bibleverse{41}dann aber soll er frei von dir ausgehen, er und seine
Kinder mit ihm, und zu seinem Geschlecht zurückkehren und wieder in den
Besitz seiner Väter eintreten. \bibleverse{42}Denn meine Dienstknechte
sind sie, die ich aus dem Lande Ägypten herausgeführt habe: sie dürfen
nicht wie gewöhnliche Sklaven verkauft werden. \bibleverse{43}Du sollst
nicht mit Härte über ihn herrschen, sondern dich vor deinem Gott
fürchten! \bibleverse{44}Was aber deine leibeigenen Knechte und Mägde
betrifft, die du besitzen wirst\textless sup title=``oder:
darfst''\textgreater✲, so mögt ihr euch solche Knechte und Mägde von den
Heidenvölkern kaufen, die rings um euch her wohnen. \bibleverse{45}Auch
aus den Kindern der Beisassen, die bei euch als Gäste leben -- aus ihnen
könnt ihr euch welche kaufen sowie aus ihren Nachkommen, die bei euch
leben und die sie in eurem Lande gezeugt haben: diese mögen euch als
Eigentum gehören, \bibleverse{46}und ihr könnt sie auf eure Kinder nach
euch vererben, damit diese sie als Eigentum besitzen: auf ewig könnt ihr
diese als Sklaven dienen lassen. Aber über eure Brüder, die Israeliten
-- da darfst du nicht, einer über den andern, mit Härte herrschen!

\bibleverse{47}Wenn ferner ein Fremdling oder Beisasse neben dir
Vermögen erwirbt, während einer deiner Volksgenossen neben ihm verarmt
und sich dem Fremdling, dem Beisassen neben dir, oder einem Abkömmling
aus der Familie eines Fremdlings als Sklaven verkauft, \bibleverse{48}so
soll, nachdem er sich verkauft hat, das Loskaufsrecht für ihn bestehen:
einer von seinen Volksgenossen mag\textless sup title=``oder:
darf''\textgreater✲ ihn loskaufen; \bibleverse{49}oder sein Oheim oder
der Sohn seines Oheims\textless sup title=``=~sein Vetter''\textgreater✲
mag\textless sup title=``oder: darf''\textgreater✲ ihn loskaufen, oder
sonst einer von seinen nächsten Blutsverwandten aus seinem Geschlecht
darf ihn loskaufen; oder wenn er selbst soviel Geld aufbringt, kann er
sich selbst loskaufen. \bibleverse{50}Und zwar soll er mit dem, der ihn
gekauft hat, die Zeit von dem Jahre ab, wo er sich verkauft hat, bis zum
Halljahr berechnen, und der Preis, um den er sich ihm verkauft hat, soll
auf die Zahl der Jahre gleichmäßig verteilt werden: wie bei einem
Lohnarbeiter soll die Dienstzeit bei ihm berechnet werden.
\bibleverse{51}Wenn also der Jahre noch viele (bis zum Halljahr) sind,
so soll er diesen entsprechend sein Lösegeld von der Kaufsumme
zurückzahlen; \bibleverse{52}wenn aber nur noch wenige Jahre bis zum
Halljahr übrig sind, so muß er danach die Berechnung anstellen: nach
Maßgabe\textless sup title=``oder: Verhältnis''\textgreater✲ seiner
Dienstjahre hat er sein Lösegeld zurückzuzahlen. \bibleverse{53}Wie ein
Mietling, der Jahr für Jahr um Lohn arbeitet, soll er bei seinem Herrn
sein, und dieser darf ihn vor deinen Augen nicht mit Härte als Herr
behandeln. \bibleverse{54}Wird er aber nicht auf diese Weise losgekauft,
so soll er im Halljahr frei ausgehen, er und seine Kinder mit ihm.
\bibleverse{55}Denn mir gehören die Israeliten als Dienstknechte; meine
Dienstknechte sind sie, die ich aus dem Lande Ägypten herausgeführt
habe, ich, der HERR, euer Gott!«

\hypertarget{l-segen-und-fluch-von-gott-zur-wahl-gestellt}{%
\paragraph{l) Segen und Fluch von Gott zur Wahl
gestellt}\label{l-segen-und-fluch-von-gott-zur-wahl-gestellt}}

\hypertarget{aa-einleitung-einschuxe4rfung-zweier-grundgebote}{%
\subparagraph{aa) Einleitung: Einschärfung zweier
Grundgebote}\label{aa-einleitung-einschuxe4rfung-zweier-grundgebote}}

\hypertarget{section-25}{%
\section{26}\label{section-25}}

\bibleverse{1}»Ihr sollt euch keine Götzen verfertigen und dürft euch
keine Schnitzbilder und Malsteine\textless sup title=``vgl. 5.Mose
7,5''\textgreater✲ aufrichten, auch keine Steine mit Bildwerk in eurem
Lande aufstellen, um euch davor niederzuwerfen; denn ich, der HERR, bin
euer Gott! \bibleverse{2}Meine Sabbate sollt ihr beobachten und mein
Heiligtum fürchten\textless sup title=``=~mit Ehrfurcht
scheuen''\textgreater✲: ich bin der Herr!«

\hypertarget{bb-segensverheiuxdfungen-fuxfcr-den-fall-des-gehorsams}{%
\subparagraph{bb) Segensverheißungen für den Fall des
Gehorsams}\label{bb-segensverheiuxdfungen-fuxfcr-den-fall-des-gehorsams}}

\bibleverse{3}»Wenn ihr in\textless sup title=``oder:
nach''\textgreater✲ meinen Satzungen wandelt und meine Gebote beobachtet
und nach ihnen tut, \bibleverse{4}so will ich euch Regen zu rechter Zeit
geben, damit das Land seinen Ertrag liefert und die Bäume auf dem Felde
ihre Früchte spenden. \bibleverse{5}Dann wird die Dreschzeit bei euch
bis an die Weinlese reichen und die Weinlese bis an die Saatzeit; und
ihr sollt Brot reichlich zu essen haben und sicher in eurem Lande
wohnen. \bibleverse{6}Dann will ich Frieden im Lande herrschen lassen,
daß ihr euch niederlegen könnt, ohne daß jemand euch aufschreckt; auch
die wilden Tiere will ich aus dem Lande verschwinden lassen, und kein
Schwert soll durch euer Land ziehen. \bibleverse{7}Ihr werdet eure
Feinde in die Flucht schlagen, und sie sollen vor euch durch das Schwert
fallen; \bibleverse{8}fünf von euch sollen hundert in die Flucht
schlagen und hundert von euch zehntausend vor sich her treiben, und eure
Feinde sollen vor euch durch das Schwert fallen. \bibleverse{9}Ich will
mich euch gnädig zuwenden, will euch zahlreich werden lassen und euch
mehren und meinen Bund mit euch aufrechthalten. \bibleverse{10}Ihr
werdet altes Getreide von der vorletzten Ernte zu essen haben und das
vorjährige wegschaffen müssen, um für das neue Raum zu schaffen.
\bibleverse{11}Und ich will meine Wohnung in eurer Mitte aufschlagen,
und mein Herz wird keine Abneigung gegen euch hegen,
\bibleverse{12}sondern ich will in eurer Mitte wandeln und euer Gott
sein, und ihr sollt mein Volk sein. \bibleverse{13}Ich bin der HERR,
euer Gott, der euch aus dem Lande Ägypten herausgeführt hat, damit ihr
ihnen nicht länger als Knechte dienen solltet; ich habe die Stäbe eures
Joches zerbrochen und euch aufrecht\textless sup title=``=~mit
aufgerichtetem Haupt''\textgreater✲ einhergehen lassen.«

\hypertarget{cc-fuxfcnf-strafandrohungen-fuxfcr-den-fall-des-ungehorsams-v.14-39}{%
\subparagraph{cc) Fünf Strafandrohungen für den Fall des Ungehorsams
(V.14-39)}\label{cc-fuxfcnf-strafandrohungen-fuxfcr-den-fall-des-ungehorsams-v.14-39}}

\hypertarget{tuxf6dliche-krankheiten-und-schimpfliche-niederlagen}{%
\paragraph{Tödliche Krankheiten und schimpfliche
Niederlagen}\label{tuxf6dliche-krankheiten-und-schimpfliche-niederlagen}}

\bibleverse{14}»Wenn ihr mir aber nicht gehorcht und nicht alle diese
Gebote erfüllt, \bibleverse{15}sondern meine Satzungen mißachtet und im
Herzen Widerwillen gegen meine Verordnungen hegt, so daß ihr nicht alle
meine Gebote befolgt, sondern den Bund mit mir brecht, \bibleverse{16}so
will auch ich dementsprechend mit euch verfahren und schreckliche
Heimsuchungen über euch verhängen: Schwindsucht und Fieber, daß euch das
Augenlicht erlöschen soll und das Leben qualvoll dahinschwindet.
Vergebens sollt ihr dann euren Samen aussäen, denn eure Feinde werden
ihn\textless sup title=``d.h. das Gesäte''\textgreater✲ verzehren;
\bibleverse{17}und ich werde mein Angesicht gegen euch kehren, daß ihr
vor euren Feinden die Flucht ergreifen müßt; und eure Widersacher sollen
über euch herrschen, und ihr sollt fliehen, auch wenn niemand euch
verfolgt.«

\hypertarget{duxfcrre-und-miuxdfwachs}{%
\paragraph{Dürre und Mißwachs}\label{duxfcrre-und-miuxdfwachs}}

\bibleverse{18}»Und wenn ihr mir auch dann noch nicht gehorcht, so will
ich euch noch siebenmal härter strafen um eurer Sünden willen;
\bibleverse{19}den trotzigen Hochmut werde ich euch dann brechen und
will den Himmel über euch hart wie Eisen machen und euren Erdboden wie
Erz, \bibleverse{20}so daß eure Kraft und Arbeit sich nutzlos erschöpfen
wird; denn euer Land wird euch keinen Ertrag geben und die Bäume auf dem
Felde keine Früchte spenden.«

\hypertarget{wilde-tiere}{%
\paragraph{Wilde Tiere}\label{wilde-tiere}}

\bibleverse{21}»Und wenn ihr mir auch dann noch widerstrebt und mir
nicht gehorchen wollt, so will ich fortfahren, euch noch siebenmal
härter um eurer Sünden willen zu schlagen. \bibleverse{22}Dann will ich
die wilden Tiere gegen euch loslassen, daß sie euch eure Kinder rauben
und euer Vieh zerreißen und eure Zahl vermindern, so daß eure Straßen
öde werden.«

\hypertarget{kriegsnot-im-verein-mit-pest-und-hunger}{%
\paragraph{Kriegsnot im Verein mit Pest und
Hunger}\label{kriegsnot-im-verein-mit-pest-und-hunger}}

\bibleverse{23}»Und wenn ihr euch auch dadurch nicht von mir warnen
laßt, sondern mir immer noch widerstrebt, \bibleverse{24}so will auch
ich euch widerstreben und euch auch meinerseits siebenfach für eure
Sünden schlagen. \bibleverse{25}Ich will das Schwert über euch kommen
lassen, das die Rache für den Bundesbruch vollziehen soll; und wenn ihr
euch dann in eure Städte zurückzieht, so werde ich die Pest unter euch
senden, und ihr sollt in Feindeshand fallen. \bibleverse{26}Wenn ich
euch dann noch die Stütze des Brotes zerbreche, so daß zehn Frauen Brot
für euch in einem einzigen Ofen backen und sie euch das Brot abgewogen
zurückbringen, so werdet ihr essen, ohne satt zu werden.«

\hypertarget{uxe4uuxdfere-und-innere-leiden-des-volkes-wuxe4hrend-der-verbannung-in-den-luxe4ndern-ihrer-feinde}{%
\paragraph{Äußere und innere Leiden des Volkes während der Verbannung in
den Ländern ihrer
Feinde}\label{uxe4uuxdfere-und-innere-leiden-des-volkes-wuxe4hrend-der-verbannung-in-den-luxe4ndern-ihrer-feinde}}

\bibleverse{27}»Und wenn ihr mir trotzdem nicht gehorsam seid und mir
immer noch widerstrebt, \bibleverse{28}so will auch ich im Grimm euch
widerstreben und euch siebenfach für eure Sünden züchtigen.
\bibleverse{29}Ihr sollt dann das Fleisch eurer eigenen Söhne essen und
das Fleisch eurer eigenen Töchter verzehren; \bibleverse{30}und ich
werde eure Höhentempel zerstören und eure Sonnensäulen umstürzen; eure
Leichname werde ich auf die Leichname eurer Götzen werfen, und mein Herz
wird euch verabscheuen. \bibleverse{31}Eure Städte will ich in
Trümmerstätten verwandeln und eure Heiligtümer verwüsten und euren
lieblichen Opferduft nicht mehr riechen. \bibleverse{32}Ja, ich selbst
werde das Land veröden, so daß eure Feinde, die dort ihren Wohnsitz
nehmen, sich darüber entsetzen sollen. \bibleverse{33}Euch aber werde
ich unter die (heidnischen) Völker zerstreuen und das Schwert hinter
euch her zücken; euer Land soll zur Wüste werden und eure Städte zu
Schutthaufen. \bibleverse{34}Da wird dann das Land seine
Ruhezeiten\textless sup title=``oder: Sabbatjahre''\textgreater✲ ersetzt
bekommen die ganze Zeit hindurch, in der es verwüstet daliegt, während
ihr im Lande eurer Feinde weilt; ja, da wird dann das Land Ruhe haben
und seine Ruhezeiten\textless sup title=``oder:
Sabbatjahre''\textgreater✲ nachholen; \bibleverse{35}die ganze Zeit
hindurch, in der es verwüstet daliegt, wird es die Ruhe haben, die ihm
in den euch gebotenen Ruhezeiten versagt war, als ihr in ihm wohntet.
\bibleverse{36}Die aber dann von euch noch übrig sind, denen will ich in
den Ländern ihrer Feinde Verzagtheit ins Herz legen, so daß das Rascheln
eines verwehten Blattes sie aufschreckt und sie davor fliehen sollen,
wie man sonst vor dem Schwerte flieht, und sie fallen sollen, obwohl
niemand sie verfolgt. \bibleverse{37}Sie sollen dann einer über den
andern hinstürzen, wie wenn es gälte, vor dem Schwerte zu fliehen,
obgleich doch niemand sie verfolgt; und es wird für euch kein
Standhalten vor euren Feinden geben: \bibleverse{38}ja, ihr sollt unter
den Heidenvölkern umkommen, und das Land eurer Feinde soll euch fressen.
\bibleverse{39}Und diejenigen von euch, die dann noch übrig sind, sollen
in den Ländern eurer Feinde infolge ihrer Sündenschuld verschmachten und
auch infolge der Sünden ihrer Väter hinschwinden mit ihnen\textless sup
title=``oder: wie diese''\textgreater✲.«

\hypertarget{ausblick-auf-israels-reue-und-bekehrung-in-der-verbannung-und-seine-wiederannahme-durch-gottes-bundestreue}{%
\paragraph{Ausblick auf Israels Reue und Bekehrung in der Verbannung und
seine Wiederannahme durch Gottes
Bundestreue}\label{ausblick-auf-israels-reue-und-bekehrung-in-der-verbannung-und-seine-wiederannahme-durch-gottes-bundestreue}}

\bibleverse{40}»Da werden sie dann ihre Schuld bekennen und auch die
Schuld ihrer Väter infolge ihres Treubruchs, den sie gegen mich begangen
haben, und werden auch eingestehen, daß, weil sie mir widerstrebt haben,
\bibleverse{41}auch ich ihnen widerstrebt und sie in das Land ihrer
Feinde gebracht habe. Wenn alsdann ihr unbeschnittenes\textless sup
title=``d.h. unempfängliches oder: schuldbeladenes''\textgreater✲ Herz
sich demütigt und sie dann die Strafe für ihre Verschuldung büßen,
\bibleverse{42}so will ich an meinen Bund mit Jakob gedenken und ebenso
an meinen Bund mit Isaak und an meinen Bund mit Abraham gedenken und
will des Landes gedenken. \bibleverse{43}Jedoch zuvor muß das Land von
ihnen verlassen sein und die ihm zukommenden Ruhezeiten\textless sup
title=``oder: Sabbatjahre''\textgreater✲ vergütet erhalten, solange es
nach ihrer Entfernung verödet liegt; und sie selbst müssen die Strafe
für ihre Verschuldung erleiden, weil sie ja doch meine Gebote mißachtet
und in ihrem Herzen gegen meine Satzungen einen Widerwillen gehegt
haben. \bibleverse{44}Aber selbst auch dann, wenn sie sich im Lande
ihrer Feinde befinden, will ich sie nicht so verwerfen und keinen
solchen Widerwillen gegen sie hegen, daß ich sie ganz vertilge und
meinen Bund mit ihnen breche, denn ich bin der HERR, ihr Gott.
\bibleverse{45}Nein, zu ihrem Heil will ich meines Bundes mit ihren
Vorfahren gedenken, die ich vor den Augen der Heidenvölker aus dem Lande
Ägypten weggeführt habe, um ihr Gott zu sein: ich, der HERR.«

\hypertarget{schluuxdf-der-vorhergehenden-gesetzsammlung}{%
\paragraph{Schluß der vorhergehenden
Gesetzsammlung}\label{schluuxdf-der-vorhergehenden-gesetzsammlung}}

\bibleverse{46}Dies sind die Satzungen, Verordnungen und Weisungen, die
der HERR auf dem Berge Sinai zwischen sich und den Israeliten durch
Vermittlung Moses gegeben hat.

\hypertarget{m-nachtrag-betreffend-weihegaben-geluxfcbde-und-abgaben-und-deren-luxf6sung}{%
\paragraph{m) Nachtrag betreffend Weihegaben (=~Gelübde) und Abgaben und
deren
Lösung}\label{m-nachtrag-betreffend-weihegaben-geluxfcbde-und-abgaben-und-deren-luxf6sung}}

\hypertarget{aa-angelobungen-und-deren-luxf6sung}{%
\subparagraph{aa) Angelobungen und deren
Lösung}\label{aa-angelobungen-und-deren-luxf6sung}}

\hypertarget{section-26}{%
\section{27}\label{section-26}}

\bibleverse{1}Hierauf gebot der HERR dem Mose folgendes:
\bibleverse{2}»Teile den Israeliten folgende Verordnungen mit: Wenn
jemand dem HERRN ein besonderes Gelübde erfüllen will, und zwar in
betreff von Personen nach dem Schätzungswert, \bibleverse{3}so soll der
Schätzungswert einer männlichen Person im Alter von zwanzig bis zu
sechzig Jahren fünfzig Silberschekel nach dem Gewicht des Heiligtums
betragen; \bibleverse{4}ist es aber eine weibliche Person, so soll der
Schätzungswert dreißig Schekel betragen. \bibleverse{5}Wenn es sich aber
um eine Person im Alter von fünf bis zu zwanzig Jahren handelt, so soll
der Schätzungswert einer männlichen Person zwanzig Schekel und bei einer
weiblichen Person zehn Schekel betragen. \bibleverse{6}Bei Kindern aber
im Alter von einem Monat bis zu fünf Jahren soll der Schätzungswert
eines Knaben fünf Silberschekel und der Schätzungswert eines Mädchens
drei Silberschekel betragen. \bibleverse{7}Wenn ferner die Person
sechzig Jahre alt ist oder darüber, so soll der Schätzungswert, wenn es
sich um einen Mann handelt, fünfzehn Schekel betragen, bei einer
weiblichen Person dagegen zehn Schekel. \bibleverse{8}Ist der
Betreffende aber für die Entrichtung dieses Schätzungswertes zu arm, so
stelle man ihn vor den Priester, damit dieser ihn abschätze; nach
Maßgabe dessen, was der Gelobende zu leisten vermag, soll der Priester
ihn abschätzen.

\bibleverse{9}Handelt es sich ferner um Vieh, von dem man dem HERRN eine
Opfergabe darbringen kann, so soll jedes Stück, das man von solchem Vieh
dem HERRN gibt, als geheiligt\textless sup title=``=~dem Heiligtum
verfallen''\textgreater✲ gelten: \bibleverse{10}man darf es nicht
umwechseln noch vertauschen, weder ein gutes Stück gegen ein schlechtes
noch ein schlechtes Stück gegen ein gutes; und wenn jemand dennoch ein
Stück Vieh gegen ein anderes vertauscht, so soll sowohl jenes als auch
das eingetauschte dem Heiligtum verfallen sein. \bibleverse{11}Handelt
es sich aber um irgendein Stück unreinen Viehes, von dem man dem HERRN
keine Opfergabe darbringen kann, so stelle man das betreffende Stück
Vieh vor den Priester; \bibleverse{12}dieser soll es dann abschätzen, je
nachdem es gut oder schlecht ist; bei der Abschätzung des Priesters soll
es dann sein Bewenden haben. \bibleverse{13}Will der Betreffende es aber
einlösen, so hat er noch ein Fünftel zu dem Schätzungswert hinzuzufügen.

\bibleverse{14}Wenn ferner jemand sein Haus als heilige Gabe dem HERRN
weiht, so soll der Priester es abschätzen, je nachdem es gut oder
schlecht ist; wie der Priester es abschätzt, so soll der Wert
festgesetzt sein. \bibleverse{15}Wenn dann der, welcher sein Haus
geweiht hat, es wieder einlösen will, so hat er noch ein Fünftel des
Betrags zu dem Schätzungswert hinzuzufügen: dann bleibt es sein
Eigentum.

\bibleverse{16}Wenn ferner jemand ein Stück von seinem ererbten
Grundbesitz dem HERRN weiht, so soll sich der Schätzungswert nach dem
Maß der dafür erforderlichen Aussaat richten: ein Homer Gerste Aussaat
soll zu fünfzig Silberschekel gerechnet werden. \bibleverse{17}Wenn der
Betreffende sein Stück Land vom Halljahr ab weiht, so soll es nach dem
vollen Schätzungswert zu stehen kommen; \bibleverse{18}wenn er aber sein
Stück Land erst nach dem Halljahr weiht, so soll der Priester ihm den
Geldbetrag mit Rücksicht auf die Zahl der Jahre berechnen, die bis zum
(nächsten) Halljahr noch ausstehen, so daß also von dem Schätzungswert
ein verhältnismäßiger Abzug gemacht wird. \bibleverse{19}Will aber der,
welcher das Stück Land geweiht hat, es wieder einlösen, so hat er noch
ein Fünftel des Betrages zu dem Schätzungswert hinzuzufügen: dann soll
es ihm verbleiben. \bibleverse{20}Hat er aber das Stück Land nicht
eingelöst, und er verkauft es trotzdem an einen andern, so kann es nicht
mehr eingelöst werden, \bibleverse{21}sondern das Stück Land soll, wenn
es im Halljahr frei wird, als dem HERRN geweiht gelten wie ein mit dem
Bann belegtes Stück Land: es soll dem Priester als Eigentum gehören.

\bibleverse{22}Wenn ferner jemand ein von ihm gekauftes Stück Land, das
nicht zu seinem ererbten Grundbesitz gehört, dem HERRN weiht,
\bibleverse{23}so soll ihm der Priester den Betrag✲ des Schätzungswertes
bis zum (nächsten) Halljahr berechnen, und der Betreffende soll diese
Schätzungssumme an demselben Tage als eine dem HERRN geweihte Gabe
entrichten. \bibleverse{24}Im Halljahr aber soll das Stück Land wieder
an den zurückfallen, von dem er es gekauft hatte, also an den, welchem
das Landstück als Erbbesitz zusteht.

\bibleverse{25}Jede (priesterliche) Schätzung aber soll nach dem Schekel
des Heiligtums erfolgen: der Schekel gilt zwanzig Gera.«

\hypertarget{bb-bestimmungen-betreffend-erstgeburten}{%
\subparagraph{bb) Bestimmungen betreffend
Erstgeburten}\label{bb-bestimmungen-betreffend-erstgeburten}}

\bibleverse{26}»Jedoch die Erstlinge vom Vieh, die als Erstgeburten dem
HERRN an sich schon zustehen, darf niemand weihen; sei es ein Rind oder
ein Stück Kleinvieh: es gehört dem HERRN. \bibleverse{27}Wenn es sich
aber um eine Erstgeburt von einem unreinen Haustier handelt, so muß der
Betreffende sie nach dem Schätzungswert lösen und noch ein Fünftel des
Betrags hinzufügen. Wird sie aber nicht gelöst, so soll sie nach dem
Schätzungswert verkauft werden.«

\hypertarget{cc-weihegaben-in-der-form-des-bannes}{%
\subparagraph{cc) Weihegaben in der Form des
Bannes}\label{cc-weihegaben-in-der-form-des-bannes}}

\bibleverse{28}»Jedoch alles mit dem Bann Belegte, das jemand von seinem
gesamten Besitz dem HERRN mittels des Bannes weiht, es seien Menschen
oder Vieh oder Stücke von seinem ererbten Grundbesitz, darf weder
verkauft noch eingelöst werden: alles Gebannte ist dem HERRN hochheilig.
\bibleverse{29}Handelt es sich dabei um Menschen, die mit dem Bann
belegt sind, so dürfen sie nicht losgekauft, sondern müssen unbedingt
getötet werden.«

\hypertarget{dd-bestimmungen-betreffend-die-frucht--und-viehzehnten}{%
\subparagraph{dd) Bestimmungen betreffend die Frucht- und
Viehzehnten}\label{dd-bestimmungen-betreffend-die-frucht--und-viehzehnten}}

\bibleverse{30}»Ferner sollen alle Zehnten des Landes, vom Saatertrag
des Feldes wie von den Früchten der Bäume, dem HERRN gehören: sie sind
dem HERRN geweiht. \bibleverse{31}Wenn aber jemand einen Teil von seinem
Zehnten einlösen will, so hat er ein Fünftel des Betrags mehr zu
bezahlen. \bibleverse{32}Was ferner allen Zehnten von Rindern und vom
Kleinvieh betrifft, so soll von allen Tieren, die unter dem Hirtenstabe
hindurchgehen, immer das zehnte Stück dem HERRN geheiligt sein.
\bibleverse{33}Man soll dabei nicht untersuchen, ob es gut oder schlecht
sei, und man darf es auch nicht vertauschen; wenn man es aber doch
vertauscht, so soll sowohl das betreffende als auch das eingetauschte
Tier dem Heiligtum verfallen sein und darf nicht gelöst werden.«

\hypertarget{schluuxdf-zu-den-sinaigesetzen}{%
\paragraph{Schluß zu den
Sinaigesetzen}\label{schluuxdf-zu-den-sinaigesetzen}}

\bibleverse{34}Dies sind die Gebote, die der HERR dem Mose auf dem Berge
Sinai für die Israeliten aufgetragen\textless sup title=``oder: zur
Pflicht gemacht''\textgreater✲ hat.
