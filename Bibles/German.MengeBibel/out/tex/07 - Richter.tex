\hypertarget{das-buch-der-richter}{%
\section{DAS BUCH DER RICHTER}\label{das-buch-der-richter}}

\hypertarget{i.-einleitung-abschluuxdf-der-erzuxe4hlung-von-der-eroberung-des-landes-11-25}{%
\subsection{I. Einleitung: Abschluß der Erzählung von der Eroberung des
Landes
(1,1-2,5)}\label{i.-einleitung-abschluuxdf-der-erzuxe4hlung-von-der-eroberung-des-landes-11-25}}

\hypertarget{eroberungskuxe4mpfe-einzelner-stuxe4mme-in-kanaan-nach-josuas-tod}{%
\subsubsection{1. Eroberungskämpfe einzelner Stämme in Kanaan nach
Josuas
Tod}\label{eroberungskuxe4mpfe-einzelner-stuxe4mme-in-kanaan-nach-josuas-tod}}

\hypertarget{a-kriegszuxfcge-und-waffentaten-der-juduxe4er-in-verbindung-mit-den-simeoniten}{%
\paragraph{a) Kriegszüge und Waffentaten der Judäer in Verbindung mit
den
Simeoniten}\label{a-kriegszuxfcge-und-waffentaten-der-juduxe4er-in-verbindung-mit-den-simeoniten}}

\hypertarget{section}{%
\section{1}\label{section}}

\bibleverse{1}Nach Josuas Tode aber fragten die Israeliten beim HERRN
an: »Wer von uns soll zuerst gegen die Kanaanäer hinaufziehen, um mit
ihnen zu kämpfen?« \bibleverse{2}Der HERR antwortete: »Juda soll
hinaufziehen; hiermit gebe ich das Land in seine Gewalt.«
\bibleverse{3}Da richtete Juda die Aufforderung an seinen Bruderstamm
Simeon: »Ziehe mit mir in das durchs Los mir zugeteilte Gebiet hinauf
und laß uns die Kanaanäer gemeinsam bekriegen; dann will auch ich mit
dir in deinen Losanteil ziehen.« So zog denn Simeon mit ihm.
\bibleverse{4}Als nun die Judäer hinaufgezogen waren, gab der HERR die
Kanaanäer und Pherissiter in ihre Gewalt: sie schlugen sie bei Besek,
zehntausend Mann. \bibleverse{5}Als sie dann bei Besek auf Adoni-Besek
stießen, griffen sie ihn an und besiegten die Kanaanäer und Pherissiter.
\bibleverse{6}Als dann Adoni-Besek die Flucht ergriff, verfolgten sie
ihn, nahmen ihn gefangen und hieben ihm die Daumen an seinen Händen und
die beiden großen Zehen ab. \bibleverse{7}Da sagte Adoni-Besek: »Siebzig
Könige, denen die Daumen an ihren Händen und die großen Zehen abgehauen
waren, haben Brocken\textless sup title=``oder: ihr Brot''\textgreater✲
unter meinem Tische auflesen müssen: wie ich getan habe, so hat Gott mir
wieder vergolten.« Man brachte ihn dann nach Jerusalem, wo er starb.~--
\bibleverse{8}Die Judäer belagerten hierauf Jerusalem, eroberten es,
machten die Einwohnerschaft mit dem Schwert nieder und steckten die
Stadt in Brand.

\hypertarget{eroberung-von-hebron-und-debir-durch-kaleb-und-othniel}{%
\paragraph{Eroberung von Hebron und Debir durch Kaleb und
Othniel}\label{eroberung-von-hebron-und-debir-durch-kaleb-und-othniel}}

\bibleverse{9}Darauf zogen die Judäer hinab, um die Kanaanäer zu
bekriegen, die im Gebirge sowie im Südland und in der Niederung wohnten.
\bibleverse{10}Die Judäer zogen also gegen die Kanaanäer, die in Hebron
wohnten -- Hebron hieß aber früher Kirjath-Arba --, und besiegten den
Sesai, den Ahiman und den Thalmai. \bibleverse{11}Von dort zog
er\textless sup title=``d.h. Kaleb''\textgreater✲ gegen die Einwohner
von Debir -- Debir hieß aber früher Kirjath-Sepher. \bibleverse{12}Da
machte Kaleb bekannt: »Wer Kirjath-Sepher bezwingt und erobert, dem gebe
ich meine Tochter Achsa zur Frau.« \bibleverse{13}Als nun Othniel, der
Sohn des Kenas, der jüngere Bruder Kalebs, die Stadt eroberte, gab er
ihm seine Tochter Achsa zur Frau. \bibleverse{14}Als sie ihm nun
zugeführt wurde, überredete sie ihn, ein Stück Ackerland von ihrem Vater
erbitten zu dürfen; und als sie dann vom Esel herabsprang und Kaleb sie
fragte: »Was wünschest du?«, \bibleverse{15}antwortete sie ihm: »Gib mir
doch ein Abschiedsgeschenk! Denn da du mich in das Südland verheiratet
hast, so gib mir auch Wasserquellen mit!« Da gab Kaleb ihr die oberen
und die unteren Quellen.

\hypertarget{anschluuxdf-der-keniter-an-juda}{%
\paragraph{Anschluß der Keniter an
Juda}\label{anschluuxdf-der-keniter-an-juda}}

\bibleverse{16}Auch die Söhne✲ des Keniters (Hobab), des Schwagers
Moses, waren mit den Judäern aus der Palmenstadt in die Steppe Juda
hinaufgezogen, die im Süden\textless sup title=``oder: beim
Südland''\textgreater✲ von Arad liegt; sie machten sich dann auf und
siedelten sich unter dem Volke\textless sup title=``d.h. unter den
Amalekitern''\textgreater✲ an.

\hypertarget{weitere-kriegerische-unternehmungen-der-juduxe4er}{%
\paragraph{Weitere kriegerische Unternehmungen der
Judäer}\label{weitere-kriegerische-unternehmungen-der-juduxe4er}}

\bibleverse{17}Darauf zogen die Judäer mit ihrem Bruderstamm Simeon zu
Felde; sie besiegten die Kanaanäer, die in Zephath wohnten,
vollstreckten den Bann an der Stadt und gaben ihr den Namen
Horma\textless sup title=``d.h. Bann; vgl. 4.Mose
21,3''\textgreater✲.~-- \bibleverse{18}Doch konnten die Judäer Gaza mit
dem zugehörigen Gebiete sowie Askalon und sein Gebiet und Ekron nebst
seinem Gebiet nicht erobern✲; \bibleverse{19}doch war der HERR mit den
Judäern, so daß sie das Bergland in Besitz nahmen; dagegen die Bewohner
der Niederung vermochten sie nicht zu vertreiben, weil diese eiserne
Streitwagen hatten. \bibleverse{20}Hebron aber gaben sie, wie Mose
bestimmt hatte, dem Kaleb, und dieser vertrieb dann von dort die drei
Enaksöhne. \bibleverse{21}Aber die Jebusiter, die Bewohner Jerusalems,
konnten die Benjaminiten nicht vertreiben; daher sind die Jebusiter
neben den Benjaminiten in Jerusalem wohnen geblieben bis auf den
heutigen Tag.

\hypertarget{b-unternehmungen-der-josephiten-und-anderer-stuxe4mme-die-kanaanuxe4er-werden-nicht-vollstuxe4ndig-vertrieben}{%
\paragraph{b) Unternehmungen der Josephiten und anderer Stämme; die
Kanaanäer werden nicht vollständig
vertrieben}\label{b-unternehmungen-der-josephiten-und-anderer-stuxe4mme-die-kanaanuxe4er-werden-nicht-vollstuxe4ndig-vertrieben}}

\bibleverse{22}Die vom Hause Joseph aber zogen ebenfalls in den Streit,
und zwar gegen Bethel, und der HERR war mit ihnen. \bibleverse{23}Das
Haus Joseph ließ nämlich Bethel auskundschaften -- die Stadt hieß
vormals Lus --; \bibleverse{24}und als die Späher einen Mann aus der
Stadt herauskommen sahen, sagten sie zu ihm: »Zeige uns doch einen
Zugang in die Stadt, wir wollen dich auch dafür belohnen.«
\bibleverse{25}Als er ihnen nun eine Stelle gezeigt hatte, wo man in die
Stadt eindringen konnte, eroberten sie den Ort und machten alle
Einwohner mit dem Schwert nieder; den Mann aber und seine ganze Familie
ließen sie frei abziehen. \bibleverse{26}Der Mann begab sich dann in das
Land der Hethiter und gründete dort eine Stadt, die er Lus nannte; so
heißt der Ort noch heutigen Tages.

\hypertarget{uxfcbersicht-uxfcber-die-nicht-eroberten-gebiete}{%
\paragraph{Übersicht über die nicht eroberten
Gebiete}\label{uxfcbersicht-uxfcber-die-nicht-eroberten-gebiete}}

\bibleverse{27}Der Stamm Manasse aber konnte die Bewohner von Beth-Sean
nebst den zugehörigen Ortschaften nicht vertreiben, ebensowenig die
Bewohner von Thaanach, von Dor, von Jibleam und von Megiddo nebst den zu
ihnen gehörenden Ortschaften; daher gelang es den Kanaanäern, in dieser
Gegend wohnen zu bleiben. \bibleverse{28}Als später aber die Israeliten
erstarkt waren, machten sie die Kanaanäer fronpflichtig, ohne sie jedoch
völlig vertreiben zu können.~-- \bibleverse{29}Ebensowenig vertrieb der
Stamm Ephraim die Kanaanäer, die in Geser wohnten; daher blieben die
Kanaanäer mitten unter ihnen in Geser wohnen.~-- \bibleverse{30}Der
Stamm Sebulon vertrieb nicht die Bewohner von Kitron und auch nicht die
Bewohner von Nahalol; daher blieben die Kanaanäer mitten unter ihnen
wohnen, wurden aber fronpflichtig.~-- \bibleverse{31}Asser vertrieb
nicht die Bewohner von Akko und die Bewohner von Sidon, sowie von
Ahlab✲, von Achsib, von Helba, von Aphik und von Rehob;
\bibleverse{32}daher blieben die Asseriten mitten unter den
einheimischen Kanaanäern wohnen, denn sie hatten sie nicht vertreiben
können.~-- \bibleverse{33}Der Stamm Naphthali vertrieb nicht die
Bewohner von Beth-Semes und auch nicht die Bewohner von Beth-Anath. So
blieb denn dieser Stamm mitten unter den einheimischen Kanaanäern
wohnen; doch wurden die Bewohner von Beth-Semes und von Beth-Anath ihnen
fronpflichtig.~-- \bibleverse{34}Die Amoriter aber drängten den Stamm
Dan ins Gebirge zurück und ließen ihn nicht in die Niederung
herabkommen. \bibleverse{35}So gelang es denn den Amoritern, in
Har-Heres, in Ajjalon und in Saalbim wohnen zu bleiben; doch lag die
Hand des Stammes Joseph immer schwerer auf ihnen, so daß sie schließlich
fronpflichtig wurden.~-- \bibleverse{36}Das Gebiet der Amoriter✲ aber
erstreckte sich von der Skorpionenhöhe bis nach Sela hin und weiter
hinauf.

\hypertarget{die-strafandrohung-des-engels-des-herrn-gegen-israel-wegen-des-bruches-der-bundespflicht}{%
\subsubsection{2. Die Strafandrohung des Engels des Herrn gegen Israel
wegen des Bruches der
Bundespflicht}\label{die-strafandrohung-des-engels-des-herrn-gegen-israel-wegen-des-bruches-der-bundespflicht}}

\hypertarget{section-1}{%
\section{2}\label{section-1}}

\bibleverse{1}Da kam der Engel des HERRN von Gilgal hinauf nach Bochim✲
und sprach: »Ich habe euch aus Ägypten hergeführt und euch in das Land
gebracht, das ich euren Vätern zugeschworen hatte mit der Verheißung:
›Ich werde meinen Bund mit euch in Ewigkeit nicht brechen!
\bibleverse{2}Ihr aber dürft mit den Bewohnern dieses Landes keinen
Vertrag schließen, sondern müßt ihre Altäre niederreißen!‹ Doch ihr seid
meinem Befehl nicht nachgekommen: was habt ihr da getan?
\bibleverse{3}So sage auch ich euch nun: ›Ich werde sie nicht mehr vor
euch vertreiben, damit sie euch zu Schlingen\textless sup title=``oder:
Bedrängern''\textgreater✲ und ihre Götter euch zum Fallstrick werden!‹«
\bibleverse{4}Als nun der Engel des HERRN diese Drohung gegen alle
Israeliten ausgesprochen hatte, fing das Volk laut zu weinen an.
\bibleverse{5}Daher nannte man jenen Ort Bochim\textless sup
title=``d.h. Ort der Weinenden''\textgreater✲; und sie brachten dort dem
HERRN Schlachtopfer dar.

\hypertarget{ii.-hauptteil-die-richtergeschichten-26-1631}{%
\subsection{II. Hauptteil: Die Richtergeschichten
(2,6-16,31)}\label{ii.-hauptteil-die-richtergeschichten-26-1631}}

\hypertarget{die-allgemeinen-zustuxe4nde-der-richterzeit}{%
\subsubsection{1. Die allgemeinen Zustände der
Richterzeit}\label{die-allgemeinen-zustuxe4nde-der-richterzeit}}

\hypertarget{a-nach-dem-tode-josuas-und-seiner-genossen-wendet-israel-sich-dem-guxf6tzendienst-zu}{%
\paragraph{a) Nach dem Tode Josuas und seiner Genossen wendet Israel
sich dem Götzendienst
zu}\label{a-nach-dem-tode-josuas-und-seiner-genossen-wendet-israel-sich-dem-guxf6tzendienst-zu}}

\bibleverse{6}Als nun Josua das Volk entlassen hatte\textless sup
title=``Jos 24,28''\textgreater✲ und die Israeliten weggegangen waren,
ein jeder in sein Erbteil, um das Land in Besitz zu nehmen,
\bibleverse{7}da diente das Volk dem HERRN, solange Josua lebte und
während der ganzen Lebenszeit der Ältesten, welche Josua überlebten und
all die großen Taten gesehen hatten, die der HERR an\textless sup
title=``oder: für''\textgreater✲ Israel vollbracht hatte.
\bibleverse{8}Dann starb aber Josua, der Sohn Nuns, der Knecht des
HERRN, im Alter von hundertundzehn Jahren, \bibleverse{9}und man begrub
ihn im Bereich seines Erbbesitzes zu Thimnath-Heres im Gebirge Ephraim,
nördlich vom Berge Gaas\textless sup title=``vgl. Jos
24,29-30''\textgreater✲. \bibleverse{10}Als dann auch jenes ganze
Geschlecht zu seinen Vätern versammelt war und ein anderes Geschlecht
nach ihnen erstand, das vom HERRN und von den Taten, die er
an\textless sup title=``oder: für''\textgreater✲ Israel vollbracht
hatte, nichts wußte, \bibleverse{11}da taten die Israeliten, was dem
HERRN mißfiel, indem sie den Baalen dienten \bibleverse{12}und den
HERRN, den Gott ihrer Väter, verließen, der sie aus dem Land Ägypten
herausgeführt hatte; sie gingen anderen Göttern nach, nämlich den
Göttern der Nachbarvölker ringsumher; sie erwiesen ihnen Anbetung und
reizten dadurch den HERRN zum Zorn.

\hypertarget{b-regelmuxe4uxdfiger-wechsel-von-abfall-und-strafe-von-buuxdfe-und-errettung-gottes-zorn}{%
\paragraph{b) Regelmäßiger Wechsel von Abfall und Strafe, von Buße und
Errettung; Gottes
Zorn}\label{b-regelmuxe4uxdfiger-wechsel-von-abfall-und-strafe-von-buuxdfe-und-errettung-gottes-zorn}}

\bibleverse{13}Wenn sie nun so vom HERRN abfielen und dem Baal und den
Astarten dienten, \bibleverse{14}dann entbrannte der Zorn des HERRN
gegen die Israeliten, und er gab sie der Gewalt von Räubern preis, die
sie ausplünderten, und ließ sie in die Hand ihrer Feinde ringsum fallen,
so daß sie vor ihren Feinden nicht mehr standzuhalten vermochten.
\bibleverse{15}Überall, wohin sie zogen\textless sup title=``oder:
Jedesmal, wenn sie ins Feld zogen''\textgreater✲, war die Hand des HERRN
gegen sie zum Unheil, wie der HERR es angedroht und wie er ihnen
zugeschworen hatte, so daß sie in sehr große Not gerieten.
\bibleverse{16}Da ließ dann der HERR Richter (unter ihnen) erstehen, die
sie aus der Gewalt ihrer Räuber befreiten. \bibleverse{17}Aber auch
ihren Richtern gehorchten sie nicht, sondern wandten sich treulos
anderen Göttern zu, denen sie Anbetung erwiesen; sie hatten gar schnell
den Weg verlassen, den ihre Väter im Gehorsam gegen die Gebote des HERRN
gewandelt waren: sie handelten nicht wie jene. \bibleverse{18}Sooft nun
der HERR Richter unter ihnen erstehen ließ, war der HERR mit dem
betreffenden Richter und errettete sie aus der Gewalt ihrer Feinde,
solange der Richter lebte; denn der HERR hatte Mitleid mit ihnen, wenn
sie über ihre Bedränger und Bedrücker wehklagten. \bibleverse{19}Sobald
aber der Richter gestorben war, trieben sie es aufs neue ärger als ihre
Väter, indem sie anderen Göttern nachgingen, um ihnen zu dienen und sie
anzubeten: sie ließen von ihrem bösen Tun und ihrem verstockten Wandel
nicht ab.

\bibleverse{20}Da entbrannte der Zorn des HERRN gegen Israel, so daß er
aussprach: »Zur Strafe dafür, daß dieses Volk die Verordnungen meines
Bundes, den ich ihren Vätern zur Pflicht gemacht habe, übertreten und
meine Weisungen nicht befolgt hat, \bibleverse{21}so will ich
meinerseits hinfort keinen einzigen mehr aus den Völkern, die Josua bei
seinem Tode übriggelassen hat, vor ihnen her vertreiben;
\bibleverse{22}nein, ich will Israel durch sie auf die Probe stellen, ob
sie den Weg des HERRN innehalten werden, um nach dem Vorgang ihrer Väter
darauf zu wandeln, oder nicht.« \bibleverse{23}Darum hatte also der HERR
diese Völkerschaften weiter bestehen lassen, statt sie sogleich zu
vertreiben, und hatte sie nicht in Josuas Hand fallen lassen.

\hypertarget{c-angabe-der-in-kanaan-uxfcbriggebliebenen-heidnischen-vuxf6lker-deren-gott-sich-zur-erprobung-und-leitung-der-israeliten-bediente}{%
\paragraph{c) Angabe der in Kanaan übriggebliebenen heidnischen Völker,
deren Gott sich zur Erprobung und Leitung der Israeliten
bediente}\label{c-angabe-der-in-kanaan-uxfcbriggebliebenen-heidnischen-vuxf6lker-deren-gott-sich-zur-erprobung-und-leitung-der-israeliten-bediente}}

\hypertarget{section-2}{%
\section{3}\label{section-2}}

\bibleverse{1}Folgendes sind aber die Völkerschaften, die der HERR hat
weiter bestehen lassen, um durch sie die Israeliten auf die Probe zu
stellen, nämlich alle die, welche die sämtlichen Kämpfe um Kanaan nicht
mitgemacht hatten~-- \bibleverse{2}{[}nur damit die Geschlechter der
Israeliten Kenntnis von denselben erhielten, um sie die Kriegsführung zu
lehren, und zwar nur die, welche von den früheren Kämpfen nichts
erlebt\textless sup title=``oder: kennengelernt hatten''\textgreater✲{]}
--: \bibleverse{3}die fünf Fürsten der Philister, alle Kanaanäer, die
Sidonier und Hewiter, die im Libanongebirge wohnen, vom Berge
Baal-Hermon an bis in die Gegend von Hamath. \bibleverse{4}Durch diese
wollte er nämlich Israel auf die Probe stellen, damit es sich zeige, ob
sie den Geboten des HERRN gehorchen würden, die er ihren Vätern durch
Mose zur Pflicht gemacht hatte.

\bibleverse{5}So hatten denn die Israeliten ihre Wohnsitze mitten unter
den Kanaanäern, Hethitern, Amoritern, Pherissitern, Hewitern und
Jebusitern; \bibleverse{6}sie nahmen sich deren Töchter zu Frauen und
ließen ihre eigenen Töchter Ehen mit den Söhnen jener eingehen und
verehrten die Götter jener.

\hypertarget{geschichten-der-einzelnen-richter-37-1631}{%
\subsubsection{2. Geschichten der einzelnen Richter
(3,7-16,31)}\label{geschichten-der-einzelnen-richter-37-1631}}

\hypertarget{a-die-ersten-richter-othniel-ehud-und-samgar}{%
\paragraph{a) Die ersten Richter: Othniel, Ehud und
Samgar}\label{a-die-ersten-richter-othniel-ehud-und-samgar}}

\bibleverse{7}Als nun die Israeliten taten, was dem HERRN mißfiel, und
den HERRN, ihren Gott, vergaßen und den Baalen und den
Astarten\textless sup title=``vgl. 2,13''\textgreater✲ dienten,
\bibleverse{8}da entbrannte der Zorn des HERRN gegen Israel, und er ließ
sie in die Gewalt des Kusan-Risathaim, des Königs von Mesopotamien,
fallen, so daß die Israeliten dem Kusan-Risathaim acht Jahre lang
dienstbar wurden. \bibleverse{9}Als aber die Israeliten den HERRN laut
um Hilfe anriefen, ließ der HERR ihnen einen Retter erstehen, der sie
befreite, nämlich Othniel, den Sohn des Kenas, den jüngeren Bruder
Kalebs. \bibleverse{10}Über diesen kam also der Geist des HERRN, und er
verhalf den Israeliten zu ihrem Recht. Als er nämlich zum Kampf
ausgezogen war, gab der HERR den Kusan-Risathaim, den König von
Mesopotamien, in seine Gewalt, so daß er ihn völlig besiegte
\bibleverse{11}und das Land vierzig Jahre lang Ruhe hatte.~--

Als dann aber Othniel, der Sohn des Kenas, gestorben war,
\bibleverse{12}taten die Israeliten wiederum, was dem HERRN mißfiel; da
verlieh der HERR dem Eglon, dem Könige der Moabiter, Macht über die
Israeliten, weil sie taten, was dem HERRN mißfiel. \bibleverse{13}Der
verbündete sich nämlich mit den Ammonitern und Amalekitern, zog heran
und besiegte die Israeliten, und sie bemächtigten sich der Palmenstadt
(Jericho). \bibleverse{14}So waren denn die Israeliten dem Moabiterkönig
Eglon achtzehn Jahre lang untertan. \bibleverse{15}Da riefen die
Israeliten den HERRN laut um Hilfe an, und der HERR ließ ihnen einen
Retter erstehen, nämlich Ehud, den Sohn des Benjaminiten Gera, einen
Mann, der linkshändig war. Durch diesen schickten nämlich die Israeliten
die ihnen auferlegte Abgabe\textless sup title=``=~den
Tribut''\textgreater✲ an den Moabiterkönig Eglon. \bibleverse{16}Ehud
hatte sich aber ein zweischneidiges Schwert, eine Elle\textless sup
title=``oder: Spanne?''\textgreater✲ lang, machen lassen und es unter
seinem Rock an seine rechte Hüfte gegürtet. \bibleverse{17}So
überreichte er dem Moabiterkönig Eglon die Abgabe; Eglon war aber ein
sehr beleibter Mann. \bibleverse{18}Als Ehud nun mit der Überreichung
der Abgabe fertig war, ging er in Begleitung der Leute, welche die
Abgabe getragen hatten, weg, \bibleverse{19}kehrte dann aber selbst bei
den Götzenbildern in der Nähe von Gilgal wieder um und ließ (dem Eglon)
sagen: »Ich habe einen geheimen Auftrag an dich, o König.« Als dieser
ihm nun zu schweigen geboten hatte, bis alle, die um ihn her standen,
hinausgegangen waren, \bibleverse{20}trat Ehud an ihn heran -- er saß
nämlich in dem kühlen Obergemach, das für ihn allein bestimmt war -- und
sagte zu ihm: »Ich habe einen Auftrag von Gott an dich!« Als jener nun
vom Sessel aufgestanden war, \bibleverse{21}griff Ehud mit seiner linken
Hand zu, nahm das Schwert von seiner rechten Hüfte und stieß es ihm in
den Bauch, \bibleverse{22}so daß sogar der Griff hinter der Klinge noch
eindrang und das Fett sich um die Klinge schloß; denn er hatte ihm das
Schwert nicht wieder aus dem Leibe herausgezogen. \bibleverse{23}Hierauf
trat Ehud auf den Söller hinaus, nachdem er die Tür des Obergemachs
hinter sich verschlossen und verriegelt hatte. \bibleverse{24}Kaum war
er nun hinausgegangen, als Eglons Diener kamen und nachsahen; als sie
aber die Tür des Obergemachs verriegelt fanden, dachten sie: »Er wird
wohl gerade seine Notdurft in dem kühlen Gemach verrichten.«
\bibleverse{25}So warteten sie sich denn zuschanden; schließlich aber,
als er die Tür des Obergemachs immer noch nicht öffnete, holten sie
einen Schlüssel und öffneten, und siehe: da lag ihr Herr als Leiche am
Boden. \bibleverse{26}Ehud aber war, während sie gezaudert hatten,
entronnen; er war schon über die Götzenbilder hinausgelangt und nach
Seira entkommen. \bibleverse{27}Sobald er dann heimgekehrt war, stieß er
im Gebirge Ephraim in die Posaune, und die Israeliten zogen mit ihm vom
Gebirge hinab, er an ihrer Spitze; \bibleverse{28}und er rief ihnen zu:
»Folgt mir eilends nach! Denn der HERR hat eure Feinde, die Moabiter, in
eure Hand gegeben!« Da zogen sie unter seiner Führung hinab, besetzten
die Jordanfurten, die nach Moab führten, und ließen niemand hinüber.
\bibleverse{29}Und sie erschlugen damals den Moabitern gegen zehntausend
Mann, lauter kräftige und streitbare Männer, und kein einziger entkam.
\bibleverse{30}So mußten die Moabiter sich damals unter die Gewalt der
Israeliten beugen, und das Land hatte achtzig Jahre lang Ruhe.~--

\bibleverse{31}Nach Ehud aber trat Samgar, der Sohn Anaths, auf und
erschlug den Philistern sechshundert Mann mit einem
Ochsentreiberstecken; und auch er brachte Israel Rettung.

\hypertarget{b-deboras-und-baraks-kampf-und-sieg-deboras-siegeslied}{%
\paragraph{b) Deboras und Baraks Kampf und Sieg; Deboras
Siegeslied}\label{b-deboras-und-baraks-kampf-und-sieg-deboras-siegeslied}}

\hypertarget{aa-der-kuxf6nig-jabin-und-sein-heerfuxfchrer-sisera-knechten-israel}{%
\subparagraph{aa) Der König Jabin und sein Heerführer Sisera knechten
Israel}\label{aa-der-kuxf6nig-jabin-und-sein-heerfuxfchrer-sisera-knechten-israel}}

\hypertarget{section-3}{%
\section{4}\label{section-3}}

\bibleverse{1}Als aber die Israeliten nach Ehuds Tode aufs neue taten,
was dem HERRN mißfiel, \bibleverse{2}ließ der HERR sie in die Gewalt des
kanaanäischen Königs Jabin fallen, der in Hazor regierte; sein
Feldhauptmann war Sisera, der in Haroseth-Goim wohnte. \bibleverse{3}Da
riefen die Israeliten den HERRN laut um Hilfe an; denn Jabin besaß
neunhundert eiserne\textless sup title=``d.h.
eisenbeschlagene''\textgreater✲ Kriegswagen und bedrückte die Israeliten
gewaltsam zwanzig Jahre lang.

\hypertarget{bb-deboras-und-baraks-verbindung-barak-fuxfchrt-das-heer-der-nuxf6rdlichen-stuxe4mme-zum-kampf-auf-den-berg-thabor}{%
\subparagraph{bb) Deboras und Baraks Verbindung; Barak führt das Heer
der nördlichen Stämme zum Kampf auf den Berg
Thabor}\label{bb-deboras-und-baraks-verbindung-barak-fuxfchrt-das-heer-der-nuxf6rdlichen-stuxe4mme-zum-kampf-auf-den-berg-thabor}}

\bibleverse{4}Nun wirkte\textless sup title=``oder:
waltete''\textgreater✲ damals Debora, eine Prophetin, die Frau
Lappidoths, als Richterin in Israel. \bibleverse{5}Sie hatte ihren Sitz
unter der Debora-Palme zwischen Rama und Bethel im Gebirge Ephraim, und
die Israeliten suchten sie dort oben auf, um sich von ihr Recht sprechen
zu lassen. \bibleverse{6}Diese nun schickte hin und ließ Barak, den Sohn
Abinoams, aus Kedes im Stamme Naphthali rufen und sagte zu ihm: »Wisse
wohl: der HERR, der Gott Israels, gebietet dir: ›Mache dich auf, ziehe
auf\textless sup title=``oder: an''\textgreater✲ den Berg Thabor und
nimm zehntausend Mann mit dir aus den beiden Stämmen Naphthali und
Sebulon. \bibleverse{7}Ich will dann Sisera, den Heerführer Jabins, samt
seinen Wagen und seinen Heerscharen zu dir an den Bach Kison führen und
ihn in deine Gewalt geben.‹« \bibleverse{8}Barak antwortete ihr: »Wenn
du mit mir gehst, so will ich gehen; wenn du aber nicht mit mir gehst,
so gehe auch ich nicht.« \bibleverse{9}Da erwiderte sie: »Gewiß will ich
mit dir gehen; doch wird alsdann der Ruhm des Zuges, den du unternimmst,
nicht dir zuteil werden, denn der HERR wird den Sisera in die Hand eines
Weibes fallen lassen.« So machte sich denn Debora auf den Weg und begab
sich mit Barak nach Kedes. \bibleverse{10}Hierauf entbot Barak die
Stämme Sebulon und Naphthali nach Kedes, und zehntausend Mann zogen
unter seiner Führung (auf den Berg Thabor) hinauf; auch Debora zog mit
ihm hinauf. \bibleverse{11}Der Keniter Heber aber hatte sich von den
übrigen Kenitern, von der Familie Hobabs, des Schwagers
Moses\textless sup title=``vgl. 1,16''\textgreater✲, getrennt und schlug
seine Zelte bis zur Eiche bei Zaanannim in der Nähe von Kedes auf.

\hypertarget{cc-siseras-niederlage-und-ermordung-in-der-kisonebene-jaels-furchtbare-heldentat}{%
\subparagraph{cc) Siseras Niederlage und Ermordung in der Kisonebene;
Jaels furchtbare
Heldentat}\label{cc-siseras-niederlage-und-ermordung-in-der-kisonebene-jaels-furchtbare-heldentat}}

\bibleverse{12}Als nun Sisera die Kunde erhielt, daß Barak, der Sohn
Abinoams, auf den Berg Thabor hinaufgezogen sei, \bibleverse{13}entbot
Sisera alle seine Wagen, neunhundert eiserne Streitwagen, und das
gesamte Kriegsvolk, das unter seinem Befehl stand, aus Haroseth-Goim an
den Bach Kison. \bibleverse{14}Da sagte Debora zu Barak: »Auf! Denn dies
ist der Tag, an dem der HERR den Sisera in deine Hand gibt! Der HERR ist
(selbst) schon vor dir her ausgezogen!« So stieg denn Barak an der
Spitze seiner zehntausend Mann vom Berge Thabor hinab,
\bibleverse{15}und der HERR setzte Sisera und alle seine Wagen und sein
ganzes Heer durch die Ankunft und den wilden Schwertangriff Baraks in
solche Verwirrung, daß Sisera vom Wagen stieg und zu Fuß floh.
\bibleverse{16}Barak aber verfolgte die Wagen und das Heer bis
Haroseth-Goim, und das ganze Heer Siseras wurde mit dem Schwert
niedergemacht: auch nicht ein einziger blieb übrig.
\bibleverse{17}Sisera aber war zu Fuß nach dem Zelte Jaels, der Frau des
Keniters Heber, geflohen; denn zwischen Jabin, dem Könige von Hazor, und
der Familie des Keniters Heber herrschte Friede. \bibleverse{18}Da trat
Jael aus dem Zelte hinaus Sisera entgegen und sagte zu ihm: »Kehre ein,
Herr, kehre ein bei mir: fürchte dich nicht!« Da trat er zu ihr ins
Zelt, und sie deckte ihn mit einer Schlafdecke\textless sup
title=``oder: einem Teppich''\textgreater✲ zu. \bibleverse{19}Dann bat
er sie: »Gib mir doch ein wenig Wasser zu trinken, denn ich bin
durstig!« Da öffnete sie den Milchschlauch, gab ihm zu trinken und
deckte ihn wieder zu. \bibleverse{20}Darauf bat er sie: »Stelle dich in
den Eingang des Zeltes, und wenn jemand kommt und dich fragt, ob jemand
hier sei, so antworte: ›Nein, kein Mensch!‹« \bibleverse{21}Nun holte
Jael, Hebers Frau, einen Zeltpflock, nahm einen Hammer in die Hand, trat
leise an ihn heran, während er vor Erschöpfung eingeschlafen war, und
schlug ihm den Pflock durch die Schläfe, so daß er noch in den Erdboden
eindrang; so starb er. \bibleverse{22}In diesem Augenblick kam Barak,
der den Sisera verfolgte; Jael trat hinaus ihm entgegen und rief ihm zu:
»Komm, ich will dir den Mann zeigen, den du suchst!« Als er dann bei ihr
eintrat, fand er wirklich Sisera als Leiche am Boden liegen, und der
Pflock steckte ihm noch in der Schläfe.

\bibleverse{23}So demütigte Gott an jenem Tage den Kanaanäerkönig Jabin
vor den Israeliten, \bibleverse{24}deren Hand dann immer schwerer auf
Jabin, dem Kanaanäerkönige, lastete, bis sie ihn völlig vernichtet
hatten.

\hypertarget{dd-deboras-und-baraks-siegeslied}{%
\subparagraph{dd) Deboras (und Baraks)
Siegeslied}\label{dd-deboras-und-baraks-siegeslied}}

\hypertarget{section-4}{%
\section{5}\label{section-4}}

\bibleverse{1}An jenem Tage sangen Debora und Barak, der Sohn Abinoams,
folgendes Lied:

\hypertarget{der-leitgedanke-und-die-aufforderung-an-die-huxf6rer}{%
\paragraph{Der Leitgedanke und die Aufforderung an die
Hörer}\label{der-leitgedanke-und-die-aufforderung-an-die-huxf6rer}}

\bibleverse{2}Daß Führer an der Spitze standen in Israel, daß das Volk
sich willig zeigte: drob preiset den HERRN! \bibleverse{3}Hört zu, ihr
Könige! Merkt auf, ihr Fürsten! Ich will, ja, ich will dem HERRN
lobsingen, will spielen dem HERRN, dem Gott Israels!

\hypertarget{gott-naht-im-gewitter}{%
\paragraph{Gott naht im Gewitter}\label{gott-naht-im-gewitter}}

\bibleverse{4}HERR, als du auszogst von Seir, als du schrittest von
Edoms Gefilden her, da bebte die Erde, es troffen die Himmel, ja, die
Wolken troffen von Wasser; \bibleverse{5}die Berge wankten vor dem
HERRN, der Sinai dort vor dem HERRN, dem Gott Israels\textless sup
title=``vgl. 5.Mose 33,2; Ps 68,8-9''\textgreater✲.

\hypertarget{die-betruxfcbenden-bisherigen-verhuxe4ltnisse}{%
\paragraph{Die betrübenden bisherigen
Verhältnisse}\label{die-betruxfcbenden-bisherigen-verhuxe4ltnisse}}

\bibleverse{6}In den Tagen Samgars, des Sohnes Anaths✲, in den Tagen
Jaels waren öde die Straßen, und die Wegewandrer gingen auf krummen
Pfaden\textless sup title=``d.h. Schleichwegen''\textgreater✲;
\bibleverse{7}es fehlte an Führern in Israel, gebrach, bis du auftratst,
Debora, auftratst, eine Mutter in Israel. \bibleverse{8}Man wählte sich
neue Götter; damals war Kampf schon vor den Toren, und weder Schild noch
Lanze war zu sehn bei Vierzigtausenden in Israel.

\hypertarget{die-gluxfcckliche-gegenwart}{%
\paragraph{Die glückliche Gegenwart}\label{die-gluxfcckliche-gegenwart}}

\bibleverse{9}Mein Herz gehört den Führern Israels, denen, die willig
sich zeigten im Volk: -- preiset den HERRN! \bibleverse{10}Die ihr
reitet auf weißglänzenden Eselinnen, die ihr sitzet auf Teppichen und
die zu Fuß ihr wandert: erzählt es euch! \bibleverse{11}Horch!~\ldots{}
zwischen den Tränkrinnen! Dort preist man die Heilstaten des HERRN, die
Heilstaten seiner Führerschaft in Israel. Da zog das Volk des HERRN zu
den Toren hinab.

\hypertarget{israels-stuxe4mme-in-der-schlacht}{%
\paragraph{Israels Stämme in der
Schlacht}\label{israels-stuxe4mme-in-der-schlacht}}

\bibleverse{12}Wach auf, erwache, Debora! Wach auf, erwache und stimme
den Sang an! Erhebe dich, Barak, und fange deine Fänger, Sohn Abinoams!
\bibleverse{13}Da zog Israel hinab samt seinen Edlen; das Volk des HERRN
zog hinab als Heldenschar. \bibleverse{14}Aus Ephraim kamen die, deren
Stammsitz unter Amalek ist, hinter ihnen Benjamin mit seinen Scharen;
aus Machir zogen Gebieter hinab und aus Sebulon die Träger des
Führerstabs, \bibleverse{15}und die Fürsten in Issaschar mit Debora und
wie Issaschar so Barak\textless sup title=``oder: der
Barakstamm''\textgreater✲: in die Ebene stürmte er hin zu Fuß. An Rubens
Bächen fanden schwere Erwägungen statt: \bibleverse{16}»Warum bliebst du
zwischen den Hürden sitzen, um das Herdengeblök\textless sup
title=``oder: Hirtengeflöt''\textgreater✲ zu hören?« An Rubens Bächen
fanden schwere Erwägungen statt. \bibleverse{17}Gliead blieb ruhig
jenseits des Jordans, und Dan -- warum weilte er bei den Schiffen? Asser
saß still am Gestade des Meeres und blieb ruhig an seinen Buchten;
\bibleverse{18}aber Sebulon ist ein Volk, das sein Leben dem Tode
preisgibt, auch Naphthali auf den Höhen seines Gefildes.

\hypertarget{die-schlacht}{%
\paragraph{Die Schlacht}\label{die-schlacht}}

\bibleverse{19}Könige kamen und stritten; damals stritten die Könige
Kanaans bei Thaanach an den Wassern Megiddos: Beute an Silber gewannen
sie nicht. \bibleverse{20}Vom Himmel her stritten die Sterne, von ihren
Bahnen her stritten sie gegen Sisera. \bibleverse{21}Der Kisonbach
spülte sie weg, der alte Schlachtenbach, der Kisonbach: tritt sie
nieder, meine Seele, mit aller Kraft! \bibleverse{22}Damals stampften
die Hufe der Rosse vom Rennen, dem Rennen ihrer Helden.
\bibleverse{23}»Verfluchet Meros!« ruft der Engel des HERRN, »ja,
fluchet seinen Bewohnern! Denn sie sind dem HERRN nicht zu Hilfe
gekommen, dem HERRN nicht zu Hilfe unter den Helden!«

\hypertarget{die-grouxdftat-der-jael}{%
\paragraph{Die Großtat der Jael}\label{die-grouxdftat-der-jael}}

\bibleverse{24}Gepriesen vor allen Weibern sei Jael, das Weib des
Keniters Heber, vor den Weibern im Zelt gepriesen! \bibleverse{25}Um
Wasser bat er, Milch gab sie, im Ehrenbecher reichte sie Sahne.
\bibleverse{26}Ihre Hand streckte sie aus nach dem Zeltpflock, ihre
Rechte nach einem Arbeitshammer, hämmerte los auf Sisera, zermalmte sein
Haupt, zerschmetterte und durchbohrte ihm die Schläfe; \bibleverse{27}zu
ihren Füßen brach er zusammen, sank hin, lag da; zu ihren Füßen brach er
zusammen, sank hin: wo er zusammenbrach, blieb entseelt er liegen.

\hypertarget{in-siseras-hause}{%
\paragraph{In Siseras Hause}\label{in-siseras-hause}}

\bibleverse{28}Durchs Fenster spähte sie aus und rief in Angst, Siseras
Mutter, durchs Gitter hindurch: »Warum zaudert sein Wagen heimzukommen?
Warum säumt der Hufschlag seiner Gespanne?« \bibleverse{29}Die klügste
ihrer Edelfrauen erwidert ihr, und auch sie selbst gibt sich die
Antwort: \bibleverse{30}»Sicherlich haben sie Beute zu teilen gefunden,
eine Dirne, zwei Dirnen für jeden Mann, Beute an bunten Stoffen für
Sisera, Beute an buntgestickten Gewändern, farbiges Zeug, ein Paar
gestickte Tücher für den Hals der Herrin.«

\hypertarget{der-abgesang}{%
\paragraph{Der Abgesang}\label{der-abgesang}}

\bibleverse{31}So müssen umkommen alle deine Feinde, HERR! Doch die ihn
lieben, sind wie der Sonne Aufgang in ihrer Kraft\textless sup
title=``oder: Pracht''\textgreater✲.~--

Darauf hatte das Land vierzig Jahre lang Ruhe.

\hypertarget{c-die-gideongeschichten-gideon-jerubbaal-und-sein-sohn-abimelech-kap.-6-8}{%
\paragraph{c) Die Gideongeschichten; Gideon-Jerubbaal und sein Sohn
Abimelech (Kap.
6-8)}\label{c-die-gideongeschichten-gideon-jerubbaal-und-sein-sohn-abimelech-kap.-6-8}}

\hypertarget{aa-der-erneute-abfall-des-volkes-hat-die-knechtung-und-auspluxfcnderung-durch-die-midianiter-zur-folge}{%
\subparagraph{aa) Der erneute Abfall des Volkes hat die Knechtung und
Ausplünderung durch die Midianiter zur
Folge}\label{aa-der-erneute-abfall-des-volkes-hat-die-knechtung-und-auspluxfcnderung-durch-die-midianiter-zur-folge}}

\hypertarget{section-5}{%
\section{6}\label{section-5}}

\bibleverse{1}Als dann die Israeliten wiederum taten, was dem HERRN
mißfiel, gab der HERR sie sieben Jahre lang in die Gewalt der
Midianiter; \bibleverse{2}und die Hand der Midianiter lag schwer auf
Israel. Um sich der Midianiter zu erwehren, richteten die Israeliten
sich die Schlupfwinkel ein, die sich in den Bergen befinden, und legten
die Höhlen und die Bergfesten\textless sup title=``oder:
Burgen''\textgreater✲ an. \bibleverse{3}Denn sooft die Israeliten gesät
hatten, zogen die Midianiter, die Amalekiter und die übrigen Horden des
Ostens gegen sie heran, \bibleverse{4}lagerten sich gegen sie im Lande
und verwüsteten den Ertrag der Felder bis nach Gaza hin und ließen keine
Lebensmittel in Israel übrig, auch kein Kleinvieh, keine Rinder und
Esel; \bibleverse{5}denn wenn sie mit ihren Herden und Zelten
heranzogen, kamen sie so zahlreich wie Heuschreckenschwärme, so daß sie
selbst und ihre Kamele nicht zu zählen waren; und wenn sie eingedrungen
waren, verheerten sie das Land. \bibleverse{6}Als Israel so von den
Midianitern arg mitgenommen wurde, schrien die Israeliten zum HERRN um
Hilfe.

\hypertarget{bb-strafrede-eines-propheten}{%
\subparagraph{bb) Strafrede eines
Propheten}\label{bb-strafrede-eines-propheten}}

\bibleverse{7}Als nun die Israeliten den HERRN um Hilfe gegen die
Midianiter angerufen hatten, \bibleverse{8}sandte der HERR einen
Propheten zu den Israeliten, der zu ihnen sagte: »So hat der HERR, der
Gott Israels, gesprochen: ›Ich selbst habe euch aus Ägypten hergebracht
und euch aus dem Hause der Knechtschaft herausgeführt; \bibleverse{9}ich
habe euch aus der Hand der Ägypter und aus der Gewalt aller eurer
Bedrücker errettet und sie vor euch her vertrieben und euch ihr Land
gegeben \bibleverse{10}und habe zu euch gesagt: Ich, der HERR, bin euer
Gott: ihr dürft die Götter der Amoriter, in deren Lande ihr wohnt, nicht
verehren! Aber ihr habt nicht auf meine Mahnung gehört.‹«

\hypertarget{cc-gideons-berufung-durch-einen-engel-seine-bedenken-durch-ein-gotteszeichen-niedergeschlagen}{%
\subparagraph{cc) Gideons Berufung durch einen Engel; seine Bedenken
durch ein Gotteszeichen
niedergeschlagen}\label{cc-gideons-berufung-durch-einen-engel-seine-bedenken-durch-ein-gotteszeichen-niedergeschlagen}}

\bibleverse{11}Da kam der Engel des HERRN und setzte sich unter die
Terebinthe in Ophra, die dem Abiesriten Joas gehörte, während dessen
Sohn Gideon gerade Weizen in der Kelter ausklopfte, um ihn vor den
Midianitern in Sicherheit zu bringen. \bibleverse{12}Diesem erschien
also der Engel des HERRN und redete ihn mit den Worten an: »Der HERR ist
mit dir, du tapferer Held!« \bibleverse{13}Gideon antwortete ihm: »Mit
Verlaub\textless sup title=``oder: Ach!''\textgreater✲, mein Herr! Wenn
der HERR wirklich mit uns wäre, wie hätte uns da dies alles widerfahren
können? Und wo sind alle seine Wundertaten, von denen unsere Väter uns
erzählt haben, indem sie sagten: ›Der HERR ist es gewesen, der uns aus
Ägypten hergeführt hat!‹? Jetzt aber hat der HERR uns verstoßen und uns
in die Hand der Midianiter fallen lassen!« \bibleverse{14}Da wandte der
HERR sich ihm zu und sagte: »Gehe hin in dieser deiner Kraft und rette
Israel aus der Gewalt der Midianiter! Ich sende dich ja!«
\bibleverse{15}Er aber entgegnete ihm: »Mit Verlaub\textless sup
title=``oder: Ach!''\textgreater✲, mein Herr! Wie könnte ich Israel
erretten, da doch mein Geschlecht das geringste\textless sup
title=``oder: ärmste''\textgreater✲ in Manasse ist und ich der Jüngste
in meines Vaters Hause bin?« \bibleverse{16}Da antwortete ihm der HERR:
»Ich werde ja mit dir sein, und du sollst die Midianiter schlagen wie
einen einzelnen Mann.« \bibleverse{17}Da entgegnete er ihm: »Wenn du mir
wirklich gnädig gesinnt bist, so gib mir ein Zeichen, daß du selbst es
bist, der mit mir redet! \bibleverse{18}Entferne dich doch nicht von
hier, bis ich zu dir zurückkehre und eine Gabe von mir herausbringe und
sie dir vorsetze!« Da antwortete er: »Ich will hier sitzen bleiben, bis
du wiederkommst.«

\bibleverse{19}Darauf ging Gideon ins Haus hinein und richtete ein
Ziegenböckchen zu und ungesäuerte Kuchen von einem Epha Mehl; das
Fleisch legte er in einen Korb, und die Brühe tat er in einen Topf und
brachte es dann zu ihm hinaus unter die Terebinthe und setzte es ihm
vor. \bibleverse{20}Doch der Engel Gottes sagte zu ihm: »Nimm das
Fleisch und die ungesäuerten Kuchen, lege sie auf den Felsen drüben und
gieße die Brühe darüber aus!« Als er es getan hatte,
\bibleverse{21}streckte der Engel des HERRN den Stab aus, den er in der
Hand hatte, und berührte mit der Spitze (des Stabes) das Fleisch und die
ungesäuerten Kuchen; da schlug Feuer aus dem Gestein hervor und
verzehrte das Fleisch und die Kuchen; der Engel des HERRN aber war vor
seinen Augen verschwunden. \bibleverse{22}Da erkannte Gideon, daß es der
Engel des HERRN gewesen war, und er rief aus: »Wehe, HERR mein Gott!
Ach, ich habe den Engel des HERRN von Angesicht zu Angesicht gesehen!«
\bibleverse{23}Aber der HERR antwortete ihm: »Friede dir\textless sup
title=``=~beruhige dich''\textgreater✲! Fürchte nichts! Du wirst nicht
sterben!« \bibleverse{24}Darauf erbaute Gideon dort dem HERRN einen
Altar und nannte ihn: »Der HERR ist Heil!« Bis auf den heutigen Tag
steht er noch in Ophra, dem Wohnort der Abiesriten.

\hypertarget{dd-gideons-auftreten-gegen-baal-seine-rettung-durch-seinen-vater-sammlung-eines-heeres-gegen-die-midianiter-seine-zwiefache-gotteserprobung}{%
\subparagraph{dd) Gideons Auftreten gegen Baal; seine Rettung durch
seinen Vater; Sammlung eines Heeres gegen die Midianiter; seine
zwiefache
Gotteserprobung}\label{dd-gideons-auftreten-gegen-baal-seine-rettung-durch-seinen-vater-sammlung-eines-heeres-gegen-die-midianiter-seine-zwiefache-gotteserprobung}}

\bibleverse{25}In derselben Nacht gebot ihm dann der HERR: »Nimm den
jungen Stier, den dein Vater hat, und außerdem den zweiten
siebenjährigen Stier, reiße den Altar Baals, der deinem Vater gehört,
nieder und haue den Götzenbaum um, der daneben steht!
\bibleverse{26}Dann baue dem HERRN, deinem Gott, auf der höchsten Stelle
dieser Bergfeste einen Altar aus aufgeschichteten Steinen und nimm den
Stier und bringe ihn als Brandopfer dar mit dem Holz des Götzenbaumes,
den du umhauen sollst!« \bibleverse{27}Da nahm Gideon zehn Männer von
seinen Knechten und tat, wie der HERR ihm befohlen hatte; weil er sich
aber vor seiner Familie und vor den Leuten der Stadt fürchtete, es bei
Tage zu tun, tat er es bei Nacht.

\bibleverse{28}Als nun die Männer der Stadt am nächsten Morgen
aufstanden, fanden sie den Altar Baals niedergerissen und den
Götzenbaum, der daneben gestanden hatte, umgehauen; der Stier aber war
als Brandopfer auf dem neuerbauten Altar dargebracht worden.
\bibleverse{29}Da fragten sie einer den andern: »Wer hat das getan?«,
und als sie dann nachforschten und sich erkundigten, hieß es: »Gideon,
der Sohn des Joas, hat das getan.« \bibleverse{30}Da sagten die Männer
der Stadt zu Joas: »Gib deinen Sohn heraus: er muß sterben, weil er den
Altar Baals niedergerissen und den heiligen Baum, der danebenstand,
umgehauen hat!« \bibleverse{31}Aber Joas entgegnete allen, die bei ihm
standen: »Wollt ihr etwa für Baal streiten, oder wollt ihr für ihn
eintreten? Wer für ihn streitet, soll noch diesen Morgen sterben! Wenn
Baal ein Gott ist, so mag er für sich selbst eintreten, weil man seinen
Altar niedergerissen hat!« \bibleverse{32}Daher gab man ihm\textless sup
title=``d.h. dem Gideon''\textgreater✲ an jenem Tage den Namen
»Jerubbaal«\textless sup title=``d.h. Baal möge streiten''\textgreater✲,
indem man sagte: »Baal möge gegen ihn\textless sup title=``d.h.
Gideon''\textgreater✲ streiten, weil er seinen Altar niedergerissen
hat.«

\bibleverse{33}Als sich nun alle Midianiter, dazu auch die Amalekiter
und die Horden des Ostens insgesamt vereinigt und sich nach
Überschreitung (des Jordans) in der Ebene Jesreel gelagert hatten,
\bibleverse{34}kam der Geist des HERRN über Gideon, so daß er in die
Posaune stieß, worauf die Abiesriter seinem Aufgebot folgten.
\bibleverse{35}Er sandte dann Boten in ganz Manasse umher, und auch
dieser Stamm folgte seinem Ruf; ebenso sandte er Boten durch Asser,
durch Sebulon und durch Naphthali, und auch diese Stämme zogen zur
Hilfeleistung heran.

\bibleverse{36}Nun sagte Gideon zu Gott: »Willst du wirklich Israel
durch meine Hand erretten, wie du verheißen hast~-- \bibleverse{37}gut,
so will ich ein Schaffel\textless sup title=``oder: die
Schafschur''\textgreater✲ auf der Tenne ausbreiten; wenn dann der Tau
bloß auf dem Fell\textless sup title=``oder: der
Wollschur''\textgreater✲ liegen wird, der ganze übrige Boden aber
trocken bleibt, so will ich daran erkennen, daß du Israel durch meine
Hand erretten willst, wie du verheißen hast.« \bibleverse{38}Und so
geschah es: als er am andern Morgen früh das Fell\textless sup
title=``oder: die Wolle''\textgreater✲ ausdrückte, preßte er Tau aus dem
Fell, eine ganze Schale voll Wasser. \bibleverse{39}Darauf sagte Gideon
zu Gott: »Gerate nicht in Zorn gegen mich, wenn ich nur diesmal noch
rede! Laß mich nur noch diesmal einen Versuch mit dem Fell\textless sup
title=``oder: der Wollschur''\textgreater✲ machen: das Fell allein möge
trocken bleiben, auf dem ganzen übrigen Boden aber Tau liegen!«
\bibleverse{40}Da ließ Gott es in jener Nacht so geschehen: das Fell
allein blieb trocken, während sonst auf dem Boden überall Tau lag.

\hypertarget{ee-gideons-sieg-uxfcber-die-midianiter}{%
\subparagraph{ee) Gideons Sieg über die
Midianiter}\label{ee-gideons-sieg-uxfcber-die-midianiter}}

\hypertarget{section-6}{%
\section{7}\label{section-6}}

\bibleverse{1}Da machte sich Jerubbaal, das ist Gideon, in der Frühe mit
der gesamten Mannschaft, die bei ihm war, auf den Weg und lagerte sich
bei der Quelle Harod, so daß sich das Lager der Midianiter nördlich von
ihm in der Ebene nach dem Hügel More hin befand.

\hypertarget{ff-gideons-streitmacht-wird-durch-zweimalige-sichtung-auf-300-mann-vermindert}{%
\subparagraph{ff) Gideons Streitmacht wird durch zweimalige Sichtung auf
300~Mann
vermindert}\label{ff-gideons-streitmacht-wird-durch-zweimalige-sichtung-auf-300-mann-vermindert}}

\bibleverse{2}Da sagte der HERR zu Gideon: »Das Volk, das du bei dir
hast, ist zu zahlreich, als daß ich die Midianiter in ihre Gewalt geben
sollte; die Israeliten könnten sich sonst mir gegenüber rühmen und
behaupten: ›Wir haben uns durch eigene Kraft gerettet!‹
\bibleverse{3}Darum laß vor dem Volk laut ausrufen: ›Wer sich fürchtet
und Angst hat, der kehre um {[}und entferne sich vom Gebirge
Gilead{]}!‹« Da zogen 22000~Mann von dem Kriegsvolk ab, und 10000
blieben zurück. \bibleverse{4}Der HERR aber sagte zu Gideon: »Das Volk
ist immer noch zu zahlreich; führe sie hinab ans Wasser: dort will ich
sie dir sichten; und von wem ich dir dann sagen werde: ›Dieser soll mit
dir ziehen!‹, der soll dich begleiten; aber jeder, von dem ich dir sage:
›Dieser soll nicht mit dir ziehen!‹, der soll dich nicht begleiten!«
\bibleverse{5}Als nun Gideon die Leute an das Wasser hinabgeführt hatte,
sagte der HERR zu ihm: »Jeden, der das Wasser mit der Zunge leckt, wie
die Hunde es machen, den stelle besonders, und ebenso jeden, der
niederkniet, um zu trinken.« \bibleverse{6}Es belief sich aber die Zahl
derer, die das Wasser {[}mit der Hand in den Mund{]} geleckt hatten, auf
dreihundert Mann; alle übrigen hatten sich auf die Knie niedergelassen,
um Wasser (mit der Hand in den Mund) zu trinken. \bibleverse{7}Darauf
sagte der HERR zu Gideon: »Durch die dreihundert Mann, die (das Wasser)
geleckt haben, will ich euch erretten und die Midianiter in deine Hand
geben; alle übrigen Leute aber sollen ein jeder in seinen Wohnort
zurückkehren!« \bibleverse{8}Da behielten sie den Mundvorrat der Leute
und deren Posaunen bei sich zurück, alle übrigen israelitischen Männer
aber entließ er, einen jeden in seine Heimat; nur die dreihundert Mann
behielt er bei sich. Das Lager der Midianiter aber befand sich unterhalb
von ihm in der Ebene.

\hypertarget{gg-gideons-zuversicht-wird-durch-beschleichung-des-feindlichen-lagers-gestuxe4rkt}{%
\subparagraph{gg) Gideons Zuversicht wird durch Beschleichung des
feindlichen Lagers
gestärkt}\label{gg-gideons-zuversicht-wird-durch-beschleichung-des-feindlichen-lagers-gestuxe4rkt}}

\bibleverse{9}In derselben Nacht nun gebot ihm der HERR: »Mache dich
auf, ziehe gegen das Lager hinab! Denn ich habe es in deine Hand
gegeben. \bibleverse{10}Fürchtest du dich aber hinabzuziehen, so steige
mit deinem Knappen\textless sup title=``oder: Burschen''\textgreater✲
Pura zum Lager hinab; \bibleverse{11}wenn du dann belauscht hast, was
man dort redet, wirst du alsdann den Mut in dir fühlen, gegen das Lager
hinabzuziehen.« Da begab er sich mit seinem Knappen Pura hinab bis zu
den Kriegern am äußersten Rande des Lagers. \bibleverse{12}Die
Midianiter aber sowie die Amalekiter und die sämtlichen Horden aus dem
Osten hatten sich in der Ebene gelagert so zahllos wie
Heuschreckenschwärme, und die Menge ihrer Kamele war unzählbar wie der
Sand am Gestade des Meeres. \bibleverse{13}Als nun Gideon dort ankam,
erzählte ein Mann gerade seinem Kameraden einen Traum mit den Worten:
»Denke dir: ich habe einen Traum gehabt! Ich sah, wie ein hartgebackenes
Gerstenbrot in das midianitische Lager rollte, bis es an ein Zelt kam
und dieses so traf, daß es umfiel: es wurde umgestürzt, das Unterste
zuoberst, und so lag das Zelt da.« \bibleverse{14}Da antwortete der
andere: »Das bedeutet nichts anderes als das Schwert des Israeliten
Gideon, des Sohnes des Joas! Gott hat die Midianiter und unser ganzes
Lager in seine Hand gegeben!« \bibleverse{15}Als nun Gideon die
Erzählung von dem Traum und seine Deutung gehört hatte, warf er sich
danksagend auf die Knie nieder, kehrte dann ins israelitische Lager
zurück und rief: »Auf! Der HERR hat das Lager der Midianiter in eure
Hand gegeben!«

\hypertarget{hh-gideons-siegreicher-uxfcberfall-des-midianitischen-lagers}{%
\subparagraph{hh) Gideons siegreicher Überfall des midianitischen
Lagers}\label{hh-gideons-siegreicher-uxfcberfall-des-midianitischen-lagers}}

\bibleverse{16}Hierauf teilte er seine dreihundert Mann in drei
Abteilungen und gab ihnen allen Posaunen in die Hand und leere Krüge, in
denen sich aber Fackeln befanden. \bibleverse{17}Dann befahl er ihnen:
»Achtet darauf, wie ich es mache, und macht es ebenso! Was ich tun
werde, sobald ich unmittelbar vor dem Lager angekommen bin, das tut auch
ihr! \bibleverse{18}Wenn ich also samt allen Leuten, die zu meiner
Abteilung gehören, in die Posaune stoße, so stoßt auch ihr in die
Posaunen rings um das ganze Lager und ruft: ›Für den HERRN und für
Gideon!‹«

\bibleverse{19}Als nun Gideon mit den hundert Mann, die seine Abteilung
bildeten, bei Beginn der mittleren Nachtwache -- soeben hatte man die
Wachen aufgestellt -- unmittelbar vor dem Lager angekommen war, stießen
sie in die Posaunen und zerschlugen die Krüge, die sie in der Hand
hatten; \bibleverse{20}und zwar stießen die drei Abteilungen
gleichzeitig in die Posaunen und zerschlugen die Krüge, nahmen dann die
Fackeln in die linke Hand und die Posaunen in die rechte, um
hineinzustoßen, und schrien: »Schwert für den HERRN und für Gideon!«
\bibleverse{21}Dabei blieben sie ein jeder ruhig an seiner Stelle rings
um das Lager herum stehen. Da geriet das ganze Lager in Bewegung, alles
schrie und suchte zu fliehen; \bibleverse{22}als jene dann aber in die
dreihundert Posaunen stießen, richtete der HERR das Schwert eines jeden
gegen den andern, und zwar im ganzen Lager, und das Heerlager floh bis
Beth-Sitta nach Zerera hin, bis an das (Jordan-) Ufer von Abel-Mehola
bei Tabbath.

\hypertarget{ii-erfolgreiche-verfolgung-die-eifersuxfcchtigen-ephraimiten-durch-gideon-beguxfctigt}{%
\subparagraph{ii) Erfolgreiche Verfolgung; die eifersüchtigen
Ephraimiten durch Gideon
begütigt}\label{ii-erfolgreiche-verfolgung-die-eifersuxfcchtigen-ephraimiten-durch-gideon-beguxfctigt}}

\bibleverse{23}Nun wurden die Mannschaften der Israeliten aus den
Stämmen Naphthali, Asser und ganz Manasse aufgeboten und verfolgten die
Midianiter. \bibleverse{24}Auch hatte Gideon Boten im ganzen Berglande
Ephraim umhergesandt mit der Aufforderung: »Kommt herab den Midianitern
entgegen und schneidet ihnen das Wasser ab bis Beth-Bara und den
Übergang über den Jordan!« Da folgten alle Männer vom Stamm Ephraim dem
Rufe und schnitten ihnen das Wasser ab bis Beth-Bara und den Übergang
über den Jordan. \bibleverse{25}Auch nahmen sie die beiden
midianitischen Fürsten\textless sup title=``oder:
Häuptlinge''\textgreater✲ Oreb\textless sup title=``d.h.
Rabe''\textgreater✲ und Seeb\textless sup title=``d.h.
Wolf''\textgreater✲ gefangen und hieben Oreb am Rabenfelsen nieder und
erschlugen Seeb bei der Wolfskelter\textless sup title=``oder:
Wolfsschlucht''\textgreater✲. Nach der Verfolgung der Midianiter aber
brachten sie die Köpfe Orebs und Seebs zu Gideon auf die andere Seite
des Jordans.

\hypertarget{section-7}{%
\section{8}\label{section-7}}

\bibleverse{1}Die Mannschaft der Ephraimiten aber sagte zu ihm: »Warum
hast du uns das zuleide getan, daß du uns nicht gleich gerufen hast, als
du zum Kampf gegen die Midianiter auszogst?« und machte ihm schwere
Vorwürfe. \bibleverse{2}Doch er entgegnete ihnen: »Was habe ich denn
jetzt geleistet im Vergleich mit euch? Ist nicht die Nachlese Ephraims
ergiebiger als die Weinlese Abiesers? \bibleverse{3}In eure Hand hat ja
Gott die Häuptlinge der Midianiter, Oreb und Seeb, fallen lassen. Was
habe ich also im Vergleich mit euch zu leisten vermocht?« Durch diese
Worte, die er an sie richtete, wurde ihr Unwille gegen ihn
beschwichtigt.

\hypertarget{jj-gideons-taten-im-ostjordanland}{%
\subparagraph{jj) Gideons Taten im
Ostjordanland}\label{jj-gideons-taten-im-ostjordanland}}

\hypertarget{gideons-bitte-in-sukkoth-und-pnuel-unfreundlich-zuruxfcckgewiesen}{%
\paragraph{Gideons Bitte in Sukkoth und Pnuel unfreundlich
zurückgewiesen}\label{gideons-bitte-in-sukkoth-und-pnuel-unfreundlich-zuruxfcckgewiesen}}

\bibleverse{4}Als nun Gideon an den Jordan gekommen und mit seinen
dreihundert Mann, erschöpft von der Verfolgung, übergesetzt war,
\bibleverse{5}bat er die Einwohner von Sukkoth: »Gebt doch den Leuten,
die mir folgen, einige Laibe Brot✲, denn sie sind erschöpft; und ich bin
auf der Verfolgung der Midianiterkönige Sebah und Zalmunna.«
\bibleverse{6}Aber die Vorsteher von Sukkoth antworteten: »Hast du etwa
die Faust Sebahs und Zalmunnas schon in deiner Hand, daß wir deiner
Mannschaft Brot geben sollten?« \bibleverse{7}Da erwiderte Gideon: »Nun
gut! Wenn der HERR den Sebah und Zalmunna in meine Gewalt gegeben hat,
will ich euch den Leib mit Wüstendornen und Stechdisteln zerdreschen!«
\bibleverse{8}Er zog dann von dort weiter nach Pnuel hinauf und richtete
an sie die gleiche Bitte; aber die Einwohner von Pnuel gaben ihm
dieselbe Antwort wie die Leute von Sukkoth. \bibleverse{9}Da erklärte er
auch den Einwohnern von Pnuel: »Wenn ich wohlbehalten zurückkomme, will
ich den Turm hier niederreißen!«

\hypertarget{gideon-nimmt-die-beiden-kuxf6nige-gefangen-und-ruxe4cht-sich-an-den-beiden-unfreundlichen-stuxe4dten}{%
\paragraph{Gideon nimmt die beiden Könige gefangen und rächt sich an den
beiden unfreundlichen
Städten}\label{gideon-nimmt-die-beiden-kuxf6nige-gefangen-und-ruxe4cht-sich-an-den-beiden-unfreundlichen-stuxe4dten}}

\bibleverse{10}Sebah und Zalmunna aber befanden sich mit ihren Heeren in
Karkor, etwa 15000~Mann, alle, die von dem gesamten Heere der Horden des
Ostens übriggeblieben waren; denn 120000~Mann hatten den Tod gefunden,
lauter schwertbewaffnete Männer. \bibleverse{11}Gideon zog nun auf der
Karawanenstraße östlich von Nobah und Jogbeha heran und überfiel das
Heer, das in seinem Lager keine Gefahr ahnte. \bibleverse{12}Sebah und
Zalmunna flohen, er aber verfolgte sie und nahm die beiden
midianitischen Könige Sebah und Zalmunna gefangen; das ganze Heer aber
hatte er zersprengt.

\bibleverse{13}Als dann Gideon, der Sohn des Joas, von der Anhöhe von
Heres her aus dem Kampf zurückkehrte, \bibleverse{14}griff er einen
jungen Mann auf, der in Sukkoth zu Hause war, und fragte ihn aus; dieser
mußte ihm die Namen der Vorsteher von Sukkoth und der dortigen Ältesten
aufschreiben, siebenundsiebzig Personen. \bibleverse{15}Als er dann in
Sukkoth ankam, sagte er zu den Männern dort: »Hier sind nun Sebah und
Zalmunna, derentwegen ihr mich verhöhnt habt mit der Frage: ›Hast du
etwa die Faust Sebahs und Zalmunnas schon in deiner Hand, daß wir deinen
erschöpften Leuten Brot geben sollten?‹« \bibleverse{16}Darauf ließ er
die Ältesten des Orts ergreifen, nahm Wüstendornen und Stechdisteln und
ließ damit den Männern von Sukkoth einen Denkzettel geben.
\bibleverse{17}Den Turm in Pnuel aber zerstörte er und ließ die Männer
der Stadt niederhauen.

\hypertarget{gideon-uxfcbt-an-den-beiden-midianiterkuxf6nigen-blutrache}{%
\paragraph{Gideon übt an den beiden Midianiterkönigen
Blutrache}\label{gideon-uxfcbt-an-den-beiden-midianiterkuxf6nigen-blutrache}}

\bibleverse{18}Darauf fragte er Sebah und Zalmunna: »Wie sahen die
Männer aus, die ihr am Thabor\textless sup title=``oder: in
Thabar''\textgreater✲ erschlagen habt?« Sie antworteten: »Ganz wie du,
so sahen sie aus, ein jeder wie ein Königssohn an Wuchs.«
\bibleverse{19}Da rief (Gideon) aus: »Das waren meine Brüder, die Söhne
meiner Mutter! So wahr der HERR lebt: hättet ihr sie am Leben gelassen,
so wollte ich euch auch nicht ums Leben bringen!« \bibleverse{20}Hierauf
sagte er zu seinem erstgeborenen Sohne Jether: »Auf! Haue sie nieder!«
Aber der Knabe zog sein Schwert nicht, denn er fürchtete sich, weil er
noch zu jung war. \bibleverse{21}Da sagten Sebah und Zalmunna: »Stehe du
auf und stoße uns nieder! Denn wie der Mann, so seine Kraft«. Da stand
Gideon auf und hieb Sebah und Zalmunna nieder; die kleinen Monde aber,
die ihre Kamele am Halse trugen, nahm er für sich.

\hypertarget{kk-gideon-lehnt-das-kuxf6nigtum-ab-seine-abguxf6tterei-und-sein-lebensende}{%
\subparagraph{kk) Gideon lehnt das Königtum ab; seine Abgötterei und
sein
Lebensende}\label{kk-gideon-lehnt-das-kuxf6nigtum-ab-seine-abguxf6tterei-und-sein-lebensende}}

\bibleverse{22}Hierauf baten die Israeliten den Gideon: »Sei unser
König, sowohl du als auch dein Sohn und deines Sohnes Sohn! Denn du hast
uns aus der Gewalt der Midianiter befreit.« \bibleverse{23}Aber Gideon
antwortete ihnen: »Ich will nicht euer König sein, und mein Sohn soll
auch nicht über euch herrschen: der HERR soll euer König sein!«
\bibleverse{24}Dann fuhr er fort: »Eine Bitte möchte ich an euch
richten: gebt mir ein jeder die Ringe, die er erbeutet hat!« Die
Midianiter hatten nämlich goldene Ringe getragen, weil sie Ismaeliter
waren. \bibleverse{25}Da antworteten sie: »Gewiß, die wollen wir dir
gern geben.« Da breiteten sie einen Mantel\textless sup title=``oder:
ein Tuch''\textgreater✲ aus, und ein jeder warf die von ihm erbeuteten
Ringe darauf. \bibleverse{26}Es betrug aber das Gewicht der goldenen
Ringe, die er sich erbeten hatte, 1700~Schekel Gold, abgesehen von den
kleinen Monden und den Ohrgehängen und den Purpurgewändern, welche die
midianitischen Könige getragen hatten, und abgesehen von den Halsketten,
die an den Hälsen ihrer Kamele gehangen hatten. \bibleverse{27}Gideon
ließ dann daraus einen kostbaren Ephod anfertigen und stellte diesen in
seinem Wohnort Ophra auf; und ganz Israel trieb dort Abgötterei mit ihm,
so daß er für Gideon und sein Haus zum Fallstrick wurde.~--
\bibleverse{28}Die Midianiter aber waren von den Israeliten gedemütigt
worden, so daß sie das Haupt nicht mehr hochtragen konnten; und das Land
hatte vierzig Jahre lang Ruhe, solange Gideon lebte.

\bibleverse{29}Hierauf ging Jerubbaal✲, der Sohn des Joas, hin und lebte
ruhig in seinem Hause. \bibleverse{30}Gideon hatte aber siebzig
vollbürtige Söhne, denn er hatte viele Frauen; \bibleverse{31}auch von
seinem Nebenweibe, die in Sichem wohnte, hatte er einen Sohn, dem er den
Namen Abimelech\textless sup title=``d.h. mein Vater ist
König''\textgreater✲ gab. \bibleverse{32}Gideon, der Sohn des Joas,
starb dann in hohem Alter und wurde im Grabe seines Vaters Joas zu
Ophra, dem Wohnort der Abiesriten, beigesetzt.

\hypertarget{ll-neuer-abfall-der-israeliten-von-gott-und-ihr-undank-gegen-gideon}{%
\subparagraph{ll) Neuer Abfall der Israeliten von Gott und ihr Undank
gegen
Gideon}\label{ll-neuer-abfall-der-israeliten-von-gott-und-ihr-undank-gegen-gideon}}

\bibleverse{33}Nach Gideons Tode aber trieben die Israeliten wiederum
Götzendienst mit den Baalen und machten den Bundesbaal✲ zu ihrem Gott;
\bibleverse{34}denn die Israeliten dachten nicht mehr an den HERRN,
ihren Gott, der sie aus der Gewalt aller ihrer Feinde ringsum errettet
hatte; \bibleverse{35}auch bewiesen sie sich nicht dankbar gegen das
Haus Jerubbaal-Gideons für alles Gute, das er an Israel getan hatte.

\hypertarget{d-gideons-sohn-abimelech-als-kuxf6nig-in-sichem}{%
\paragraph{d) Gideons Sohn Abimelech als König in
Sichem}\label{d-gideons-sohn-abimelech-als-kuxf6nig-in-sichem}}

\hypertarget{aa-abimelechs-brudermord-in-ophra-und-sein-kuxf6nigtum-in-sichem}{%
\subparagraph{aa) Abimelechs Brudermord in Ophra und sein Königtum in
Sichem}\label{aa-abimelechs-brudermord-in-ophra-und-sein-kuxf6nigtum-in-sichem}}

\hypertarget{section-8}{%
\section{9}\label{section-8}}

\bibleverse{1}Abimelech aber, der Sohn Jerubbaals, begab sich nach
Sichem zu den Brüdern\textless sup title=``oder:
Verwandten''\textgreater✲ seiner Mutter und stellte ihnen sowie allen,
die zu der Familie des Vaters seiner Mutter gehörten, folgendes vor:
\bibleverse{2}»Richtet doch an alle Bürger von Sichem laut die Frage:
›Was ist vorteilhafter für euch, wenn siebzig Männer, sämtliche Söhne
Jerubbaals, über euch herrschen oder wenn ein einziger Mann euer
Herrscher ist?‹ Bedenkt auch, daß ich von eurem Fleisch und Bein bin!«
\bibleverse{3}Als nun die Brüder\textless sup title=``oder:
Verwandten''\textgreater✲ seiner Mutter alle diese Äußerungen vor den
Bürgern von Sichem laut aussprachen und für ihn eintraten, da neigten
sich ihre Herzen dem Abimelech zu; denn sie sagten: »Er ist unser
Bruder✲.« \bibleverse{4}Sie gaben ihm also siebzig Schekel Silber aus
dem Tempel des Bundesbaals, und mit diesem Gelde nahm Abimelech eine
Anzahl nichtsnutziger und verwegener Leute in seinen Dienst: sie
bildeten sein Gefolge. \bibleverse{5}Er begab sich dann nach Ophra in
das Haus seines Vaters und ermordete dort seine Brüder, die Söhne
Jerubbaals, siebzig Männer, alle auf einem Steine\textless sup
title=``oder: auf einmal''\textgreater✲; nur Jotham, der jüngste Sohn
Jerubbaals, blieb am Leben, weil er sich versteckt hatte.
\bibleverse{6}Darauf versammelten sich alle Bürger von Sichem und alle
Bewohner des Millo\textless sup title=``d.h. der Burg''\textgreater✲,
begaben sich zu der Denkmalseiche, die in der Nähe von Sichem
steht\textless sup title=``vgl. Jos 24,26''\textgreater✲, und machten
dort Abimelech zum König.

\hypertarget{bb-jotham-truxe4gt-den-sichemiten-eine-ernst-warnende-fabel-vor-und-flucht-ihnen-sowie-dem-abimelech}{%
\subparagraph{bb) Jotham trägt den Sichemiten eine ernst warnende Fabel
vor und flucht ihnen sowie dem
Abimelech}\label{bb-jotham-truxe4gt-den-sichemiten-eine-ernst-warnende-fabel-vor-und-flucht-ihnen-sowie-dem-abimelech}}

\bibleverse{7}Als man das dem Jotham hinterbrachte, machte er sich auf,
trat auf den Gipfel des Berges Garizim und rief ihnen mit hocherhobener
Stimme zu: »Hört mich an, ihr Bürger von Sichem, damit auch Gott euch
anhöre!

\bibleverse{8}Einst gingen die Bäume hin, um einen König über sich zu
salben, und sagten zum Ölbaum: ›Sei du unser König!‹ \bibleverse{9}Aber
der Ölbaum erwiderte ihnen: ›Soll ich etwa meine Fettigkeit aufgeben, um
deretwillen Götter und Menschen mich preisen, und soll hingehen, um über
den Bäumen zu schweben?‹ \bibleverse{10}Da sagten die Bäume zum
Feigenbaum: ›Komm du, sei unser König!‹ \bibleverse{11}Aber der
Feigenbaum erwiderte ihnen: ›Soll ich etwa meine Süßigkeit und meine
köstlichen Früchte aufgeben und hingehen, um über den Bäumen zu
schweben?‹ \bibleverse{12}Da sagten die Bäume zum Weinstock: ›Komm du,
sei unser König!‹ \bibleverse{13}Aber der Weinstock erwiderte ihnen:
›Soll ich etwa meinen Wein aufgeben, der Götter und Menschen fröhlich
macht, und hingehen, um über den Bäumen zu schweben?‹ \bibleverse{14}Da
sagten die Bäume alle zum Dornstrauch: ›Komm du, sei unser König!‹
\bibleverse{15}Da antwortete der Dornstrauch den Bäumen: ›Wenn ihr mich
wirklich zum König über euch salben wollt, so kommt: vertraut euch
meinem Schatten an! Wo nicht, so soll Feuer vom Dornstrauch ausgehen und
die Zedern des Libanons verzehren!‹ \bibleverse{16}Und nun, wenn ihr
treu und redlich gehandelt habt, daß ihr Abimelech zum König gemacht,
und wenn ihr ehrenhaft an Jerubbaal und seinem Hause gehandelt und euch
dankbar für seine Verdienste um euch bewiesen habt~--
\bibleverse{17}denn mein Vater hat für euch gekämpft und sein Leben aufs
Spiel gesetzt und euch aus der Gewalt der Midianiter errettet,
\bibleverse{18}während ihr euch jetzt gegen das Haus meines Vaters
erhoben und seine Söhne ermordet habt, siebzig Männer, alle auf
einmal\textless sup title=``vgl. V.5''\textgreater✲, und habt Abimelech,
den Sohn seiner Magd✲, zum König über die Bürger von Sichem gemacht,
weil er euer Bruder\textless sup title=``oder:
Stammesgenosse''\textgreater✲ ist --; \bibleverse{19}wenn ihr also heute
treu und ehrenhaft an Jerubbaal und seinem Hause gehandelt habt, so
wünsche ich euch, Freude an Abimelech zu erleben, und auch er möge euer
froh werden! \bibleverse{20}Wenn aber nicht, so möge Feuer von Abimelech
ausgehen und die Bürger von Sichem samt den Bewohnern der Burg
verzehren, und ebenso möge Feuer von den Bürgern Sichems und von den
Bewohnern der Burg ausgehen und den Abimelech verzehren!«
\bibleverse{21}Darauf entfloh Jotham eilig und kam glücklich nach Beer,
wo er sich in weiter Entfernung von seinem Bruder Abimelech niederließ.

\hypertarget{cc-unheilvolle-entwicklung-der-verhuxe4ltnisse-in-sichem}{%
\subparagraph{cc) Unheilvolle Entwicklung der Verhältnisse in
Sichem}\label{cc-unheilvolle-entwicklung-der-verhuxe4ltnisse-in-sichem}}

\bibleverse{22}Als nun Abimelech drei Jahre über Israel geherrscht
hatte, \bibleverse{23}ließ Gott einen bösen Geist\textless sup
title=``=~Zwietracht und Zerwürfnis''\textgreater✲ zwischen Abimelech
und den Bürgern von Sichem entstehen, so daß die Bürger von Sichem von
Abimelech abfielen~-- \bibleverse{24}dies geschah nämlich, damit das an
den siebzig Söhnen Jerubbaals begangene Verbrechen gerächt würde und
Gott ihr Blut auf ihren Bruder Abimelech, der sie ermordet hatte, und
auf die Bürger von Sichem kommen ließe, die ihm bei der Ermordung seiner
Brüder behilflich gewesen waren. \bibleverse{25}Die Bürger von Sichem
legten also, um ihn verhaßt zu machen, auf den Bergeshöhen Wegelagerer
in Hinterhalt; die mußten jeden ausplündern, der auf der Straße an ihnen
vorüberzog. Das wurde dem Abimelech hinterbracht.

\bibleverse{26}Damals war nämlich Gaal, der Sohn Ebeds, mit seinen
Brüdern\textless sup title=``oder: Stammesgenossen''\textgreater✲
gekommen und hatte sich in Sichem niedergelassen, und die Bürger von
Sichem hatten Vertrauen zu ihm gewonnen. \bibleverse{27}Als sie nun
einst aufs Feld hinausgezogen waren und die Weinlese gehalten und
gekeltert hatten, veranstalteten sie ein Freudenfest, begaben sich in
den Tempel ihres Gottes, aßen und tranken und stießen Verwünschungen
gegen Abimelech aus. \bibleverse{28}Da rief Gaal, der Sohn Ebeds, aus:
»Wer ist Abimelech, und wer sind wir Sichemiten, daß wir seine
Untertanen sein sollten? Ist er nicht ein Sohn Jerubbaals und Sebul sein
Vogt? Müßte er nicht den Männern Hemors, des Stammvaters von Sichem,
untertan sein? Warum sollen nun wir ihm dienen? \bibleverse{29}Hätte ich
nur dies Volk unter meinem Befehl: ich wollte diesem Abimelech schon den
Weg weisen und dem Abimelech sagen lassen: ›Verstärke dein Heer und
rücke aus!‹«

\bibleverse{30}Als nun Sebul, der Stadthauptmann, Kunde von den Reden
Gaals, des Sohnes Ebeds, erhielt, geriet er in Zorn; \bibleverse{31}er
sandte insgeheim Boten an Abimelech und ließ ihm sagen: »Wisse wohl:
Gaal, der Sohn Ebeds, ist mit seinen Brüdern\textless sup title=``vgl.
V.26''\textgreater✲ nach Sichem gekommen, und sie wiegeln nun die Stadt
gegen dich auf. \bibleverse{32}Mache dich also in der Nacht\textless sup
title=``oder: noch in dieser Nacht''\textgreater✲ mit dem Kriegsvolk,
das du bei dir hast, auf den Weg und lege dich draußen auf dem Felde in
einen Hinterhalt. \bibleverse{33}Frühmorgens aber, sobald die Sonne
aufgeht, brich auf und überfalle die Stadt. Wenn er dann mit seinen
Leuten gegen dich ausrückt, so verfahre mit ihm nach den Umständen.«

\bibleverse{34}Da machte sich Abimelech mit allem Kriegsvolk, das er bei
sich hatte, in der Nacht auf den Weg, und sie legten sich in vier
Abteilungen in den Hinterhalt gegen Sichem. \bibleverse{35}Als nun Gaal,
der Sohn Ebeds, hinausging und an den Eingang des Stadttores trat, brach
Abimelech mit seinen Leuten aus dem Hinterhalt hervor.
\bibleverse{36}Sobald Gaal die Leute erblickte, sagte er zu Sebul: »Da
kommt ja Kriegsvolk von den Berghöhen herab!« Aber Sebul entgegnete ihm:
»Den Schatten der Berge siehst du für Männer an.« \bibleverse{37}Doch
Gaal blieb dabei: »Nein, es sind Krieger, die vom Nabel des
Landes\textless sup title=``oder: vom Erdnabel''\textgreater✲
herabziehen, und dort ist noch eine Schar, die des Weges von der
Wahrsager-Eiche her kommt«. \bibleverse{38}Da rief Sebul ihm zu: »Wo ist
denn nun dein großes Maul, mit dem du ausgerufen hast: ›Wer ist
Abimelech, daß wir seine Untertanen sein sollten?‹ Das ist ja das
Kriegsvolk, das du verachtet hast! So ziehe nun jetzt aus und kämpfe mit
ihm!« \bibleverse{39}Da zog Gaal an der Spitze der Bürger von Sichem aus
und wurde mit Abimelech handgemein; \bibleverse{40}dieser aber schlug
ihn zurück, und er mußte vor ihm fliehen, und viele blieben erschlagen
liegen bis an den Eingang des Stadttores. \bibleverse{41}Während
Abimelech dann in Aruma blieb, vertrieb Sebul den Gaal und dessen
Brüder\textless sup title=``oder: Genossen''\textgreater✲, so daß sie
nicht länger in Sichem bleiben konnten.

\hypertarget{dd-abimelechs-blutiger-sieg-zerstuxf6rung-der-stadt-sichem-abimelechs-schandtat-an-den-bewohnern-der-burg-und-sein-unruxfchmliches-ende-in-thebez}{%
\subparagraph{dd) Abimelechs blutiger Sieg, Zerstörung der Stadt Sichem;
Abimelechs Schandtat an den Bewohnern der Burg und sein unrühmliches
Ende in
Thebez}\label{dd-abimelechs-blutiger-sieg-zerstuxf6rung-der-stadt-sichem-abimelechs-schandtat-an-den-bewohnern-der-burg-und-sein-unruxfchmliches-ende-in-thebez}}

\bibleverse{42}Als dann am folgenden Morgen die Leute aufs Feld
hinausgegangen waren, nahm Abimelech, der Kunde davon erhalten hatte,
\bibleverse{43}sein Kriegsvolk, teilte es in drei Abteilungen und legte
sich draußen auf dem Felde in den Hinterhalt. Als er nun sah, wie die
Leute aus der Stadt herauskamen, fiel er über sie her und machte sie
nieder; \bibleverse{44}dann stürmte er mit seiner Abteilung vor und nahm
am Eingang des Stadttores Stellung, während die beiden anderen
Abteilungen über alle die herfielen, welche auf dem Felde waren, und sie
niederhieben. \bibleverse{45}Abimelech bestürmte dann die Stadt jenen
ganzen Tag hindurch, und als er sie erobert hatte, ließ er die Einwohner
töten, zerstörte die Stadt und streute Salz über die ganze Stätte.

\bibleverse{46}Als die Bewohner der Burg von Sichem das vernahmen,
begaben sie sich alle in den Keller des Tempels des Bundesgottes.
\bibleverse{47}Sobald nun dem Abimelech gemeldet wurde, daß alle
Bewohner der Burg von Sichem dort beisammen seien, \bibleverse{48}stieg
er mit seiner gesamten Mannschaft auf den Berg Zalmon; dort nahm
Abimelech seine Axt zur Hand, hieb einen Busch Holz ab, lud ihn sich auf
die Schultern und befahl seinen Leuten: »Was ihr mich habt tun sehen,
das tut auch ihr ohne Verzug!« \bibleverse{49}Da hieben auch sämtliche
Krieger Mann für Mann einen Busch für sich ab, zogen dann hinter
Abimelech her, warfen (das Holz) oben auf den Keller und setzten so den
Keller von obenher in Brand, so daß nun auch alle Bewohner der Burg von
Sichem ihren Tod fanden, ungefähr tausend Männer und Frauen.

\bibleverse{50}Hierauf rückte Abimelech gegen Thebez, belagerte den Ort
und eroberte ihn. \bibleverse{51}Mitten im Ort stand aber ein fester
Turm, in den sich alle Männer und Frauen, sämtliche Bürger des Orts,
geflüchtet hatten; sie hatten sich daselbst eingeschlossen und waren auf
das Dach des Turmes gestiegen. \bibleverse{52}Als nun Abimelech an den
Turm herangerückt war und ihn bestürmte und dabei bis an den Eingang des
Turmes heranging, um Feuer an ihn zu legen, \bibleverse{53}schleuderte
ihm eine Frau den oberen Stein einer Handmühle auf den Kopf und
zerschmetterte ihm den Schädel. \bibleverse{54}Da rief er schnell seinen
Waffenträger herbei und befahl ihm: »Ziehe dein Schwert und töte mich
vollends, damit man mir nicht nachsagen kann, ein Weib habe mich
umgebracht!« Da durchbohrte ihn sein Waffenträger, und er starb.

\bibleverse{55}Als nun die Israeliten sahen, daß Abimelech tot war,
kehrten sie ein jeder in seinen Wohnort zurück. \bibleverse{56}So
vergalt Gott dem Abimelech das Verbrechen, das er durch die Ermordung
seiner siebzig Brüder an seinem Vater begangen hatte; \bibleverse{57}und
ebenso ließ Gott alle Freveltaten der Einwohner von Sichem auf ihr Haupt
zurückfallen. So erfüllte sich an ihnen der Fluch Jothams, des Sohnes
Jerubbaals.

\hypertarget{e-die-kleinen-richter-thola-und-jair}{%
\paragraph{e) Die (kleinen) Richter Thola und
Jair}\label{e-die-kleinen-richter-thola-und-jair}}

\hypertarget{section-9}{%
\section{10}\label{section-9}}

\bibleverse{1}Nach Abimelech trat dann zur Errettung Israels ein Mann
aus dem Stamm Issaschar auf, nämlich Thola, der Sohn Puas, des Sohnes
Dodos; er wohnte in Samir im Gebirge Ephraim \bibleverse{2}und waltete
als Richter in Israel dreiundzwanzig Jahre lang; dann starb er und wurde
in Samir begraben.

\bibleverse{3}Nach ihm trat Jair aus Gilead auf und richtete Israel
zweiundzwanzig Jahre. \bibleverse{4}Er hatte dreißig Söhne, die auf
dreißig Eselsfüllen ritten und dreißig Ortschaften besaßen, die man bis
auf den heutigen Tag die ›Zeltdörfer Jairs‹ nennt; sie liegen in der
Landschaft Gilead\textless sup title=``vgl. 4.Mose 32,41''\textgreater✲.
\bibleverse{5}Dann starb Jair und wurde in Kamon begraben.

\hypertarget{f-die-jephthageschichten}{%
\paragraph{f) Die Jephthageschichten}\label{f-die-jephthageschichten}}

\hypertarget{aa-neuer-abfall-des-volkes-verursacht-neue-drangsale-durch-die-ammoniter-aufrichtige-buuxdfe-bewirkt-guxf6ttliche-gnade}{%
\subparagraph{aa) Neuer Abfall des Volkes verursacht neue Drangsale
durch die Ammoniter; aufrichtige Buße bewirkt göttliche
Gnade}\label{aa-neuer-abfall-des-volkes-verursacht-neue-drangsale-durch-die-ammoniter-aufrichtige-buuxdfe-bewirkt-guxf6ttliche-gnade}}

\bibleverse{6}Aber die Israeliten taten wiederum, was dem HERRN mißfiel;
denn sie dienten den Baalen und Astarten\textless sup title=``vgl.
2,13''\textgreater✲, den Göttern der Syrer und der Sidonier, den Göttern
der Moabiter, der Ammoniter und der Philister; den HERRN aber verließen
sie und dienten ihm nicht. \bibleverse{7}Da entbrannte der Zorn des
HERRN gegen die Israeliten, so daß er sie in die Gewalt der Philister
und in die Hand der Ammoniter fallen ließ; \bibleverse{8}die bedrückten
und mißhandelten die Israeliten achtzehn Jahre lang, nämlich alle
Israeliten jenseits des Jordans im Lande der Amoriter, die in Gilead
wohnten. \bibleverse{9}Als nun die Ammoniter sogar über den Jordan
zogen, um auch Juda, Benjamin und den Stamm Ephraim zu bekriegen, da
gerieten die Israeliten in sehr große Angst. \bibleverse{10}Da riefen
die Israeliten den HERRN laut um Hilfe an und bekannten: »Wir haben
gegen dich gesündigt, und zwar dadurch, daß wir unsern Gott verlassen
und den Baalen gedient haben!« \bibleverse{11}Da antwortete der HERR den
Israeliten: »Habe ich euch nicht von den Ägyptern und Amoritern, von den
Ammonitern und Philistern errettet? \bibleverse{12}Und als die Sidonier,
die Amalekiter und die Midianiter euch bedrängten und ihr mich um Hilfe
anrieft, habe ich euch da nicht aus ihrer Gewalt befreit?
\bibleverse{13}Ihr aber habt mich verlassen und anderen Göttern gedient;
darum will ich euch hinfort nicht mehr erretten. \bibleverse{14}Geht hin
und schreit um Hilfe zu den Göttern, die ihr euch erwählt habt: die
mögen euch helfen, wenn ihr in Not seid!« \bibleverse{15}Da beteten die
Israeliten zum HERRN: »Wir haben gesündigt! Verfahre mit uns ganz so,
wie es dir wohlgefällt, nur rette uns noch dies eine Mal!«
\bibleverse{16}Darauf entfernten sie die fremden Götter aus ihrer Mitte
und verehrten den HERRN; da konnte er sein Erbarmen mit der Not Israels
nicht länger zurückhalten.

\hypertarget{bb-jephthas-berufung-zum-richteramt}{%
\subparagraph{bb) Jephthas Berufung zum
Richteramt}\label{bb-jephthas-berufung-zum-richteramt}}

\bibleverse{17}Als nun die Ammoniter nach dem Aufgebot ihrer Kriegsmacht
sich in Gilead lagerten und die Israeliten sich gesammelt und ein Lager
bei Mizpa bezogen hatten, \bibleverse{18}da sagte das Volk, die Fürsten
von Gilead, zueinander: »Wer ist der Mann, der zuerst den Angriff gegen
die Ammoniter wagt? Er soll das Oberhaupt aller Bewohner Gileads sein!«

\hypertarget{section-10}{%
\section{11}\label{section-10}}

\bibleverse{1}Nun war der Gileaditer Jephtha ein tapferer Held, obwohl
der Sohn einer Dirne; sein Vater war irgendein Gileaditer.
\bibleverse{2}Als nun die (rechtmäßige) Frau des (betreffenden)
Gileaditers ihm Söhne gebar und die Söhne dieser Frau herangewachsen
waren, hatten sie Jephtha ausgestoßen und zu ihm gesagt: »Du sollst in
unserer Familie nicht miterben!, denn du bist der Sohn einer fremden
Frau!« \bibleverse{3}So war denn Jephtha vor seinen Brüdern geflohen und
hatte sich in der Landschaft Tob niedergelassen, wo sich nichtsnutzige
Leute um ihn sammelten, die mit ihm Raubzüge unternahmen.

\bibleverse{4}Nun begab es sich nach einiger Zeit, daß die Ammoniter
Krieg mit den Israeliten anfingen. \bibleverse{5}Als nun die Ammoniter
gegen die Israeliten zu Felde zogen, machten sich die Ältesten der
Gileaditer auf den Weg, um Jephtha aus der Landschaft Tob zu holen.
\bibleverse{6}Sie baten ihn: »Komm und werde unser Anführer, damit wir
gegen die Ammoniter kämpfen!« \bibleverse{7}Aber Jephtha antwortete den
Ältesten der Gileaditer: »Seid ihr es nicht, die mich gehaßt und aus
meines Vaters Hause vertrieben haben? Warum kommt ihr jetzt zu mir, wo
ihr in Not seid?« \bibleverse{8}Da erwiderten ihm die Ältesten der
Gileaditer: »Eben deshalb sind wir jetzt wieder zu dir gekommen, und
wenn du mit uns gehst und gegen die Ammoniter kämpfen willst, so sollst
du bei uns das Oberhaupt aller Bewohner Gileads sein!« \bibleverse{9}Da
antwortete Jephtha den Ältesten der Gileaditer: »Wenn ihr mich
zurückholt, damit ich gegen die Ammoniter kämpfe, und der HERR sie von
mir besiegt werden läßt, werde ich dann wirklich euer Oberhaupt sein?«
\bibleverse{10}Da erwiderten ihm die Ältesten von Gilead: »Der HERR sei
Zeuge zwischen uns (und strafe uns), wenn wir nicht so tun, wie du es
verlangst!« \bibleverse{11}So ging denn Jephtha mit den Ältesten von
Gilead, und das Kriegsvolk machte ihn zu seinem Oberhaupt und zum
Befehlshaber über sich {[}und Jephtha trug alles, was er zu sagen hatte,
dem HERRN in Mizpa vor{]}.

\hypertarget{cc-jephthas-erfolglose-unterhandlungen-mit-den-ammonitern}{%
\subparagraph{cc) Jephthas erfolglose Unterhandlungen mit den
Ammonitern}\label{cc-jephthas-erfolglose-unterhandlungen-mit-den-ammonitern}}

\bibleverse{12}Hierauf sandte Jephtha Boten an den König der Ammoniter
und ließ ihm sagen: »Was willst du von mir, daß du gegen mich
herangezogen bist, um mein Land zu bekriegen?« \bibleverse{13}Der König
der Ammoniter antwortete den Boten Jephthas: »Israel hat mir, als es aus
Ägypten heraufzog, mein Land weggenommen vom Arnon bis an den Jabbok und
bis an den Jordan: gib es mir also jetzt gutwillig zurück!«
\bibleverse{14}Darauf sandte Jephtha nochmals Boten an den König der
Ammoniter \bibleverse{15}und ließ ihm sagen: »Jephtha macht dich auf
folgendes aufmerksam: Die Israeliten haben den Moabitern und den
Ammonitern ihr Land nicht weggenommen, \bibleverse{16}sondern als die
Israeliten beim Auszug aus Ägypten durch die Wüste bis ans Schilfmeer
gewandert und in Kades angekommen waren, \bibleverse{17}schickten sie
Gesandte an den König der Edomiter und ließen ihn um freien Durchzug
durch sein Land bitten; aber der König der Edomiter wollte davon nichts
wissen. Darauf schickten sie auch an den König der Moabiter, aber auch
der wollte es nicht bewilligen. So mußten denn die Israeliten in Kades
bleiben, \bibleverse{18}dann durch die Wüste ziehen, um das Land der
Edomiter und das Land der Moabiter herumwandern und nach ihrer Ankunft
auf der Ostseite des Landes der Moabiter jenseits des Arnons lagern,
ohne das Gebiet der Moabiter betreten zu haben; denn der Arnon bildet
die Grenze der Moabiter. \bibleverse{19}Darauf schickten die Israeliten
Gesandte an den Amoriterkönig Sihon, der in Hesbon seinen Wohnsitz
hatte, und baten ihn um freien Durchzug durch sein Land, um an das Ziel
ihrer Wanderung zu gelangen. \bibleverse{20}Aber Sihon wollte den
Israeliten den Durchzug durch sein Gebiet aus Mißtrauen nicht gestatten,
sondern bot sein gesamtes Kriegsvolk auf, bezog ein Lager bei Jahaz und
griff die Israeliten an. \bibleverse{21}Da ließ der HERR, der Gott
Israels, Sihon mit seinem ganzen Heer in die Hand der Israeliten fallen,
so daß diese sie besiegten. So nahmen die Israeliten das ganze Land der
Amoriter, die in jenem Lande wohnten, in Besitz \bibleverse{22}und
bemächtigten sich des ganzen Gebiets der Amoriter vom Arnon bis an den
Jabbok und von der Wüste bis zum Jordan. \bibleverse{23}Und jetzt,
nachdem der HERR, der Gott Israels, die Amoriter vor seinem Volke Israel
vertrieben hat, willst du uns aus ihrem Besitz verdrängen?
\bibleverse{24}Nicht wahr? Was dein Gott Kamos dir zum Besitz gibt, das
nimmst du in Besitz; und alles, was der HERR, unser Gott, vor uns
vertrieben hat, in dessen Besitz treten wir ein! \bibleverse{25}Und nun:
bist du etwa besser\textless sup title=``oder: stärker''\textgreater✲
als der Moabiterkönig Balak, der Sohn Zippors? Hat er etwa mit Israel
gerechtet oder je Krieg gegen sie geführt, \bibleverse{26}während Israel
in Hesbon und den zugehörigen Ortschaften sowie in Aroer und den
zugehörigen Ortschaften und in allen Städten, die auf beiden Seiten des
Arnons liegen, dreihundert Jahre lang wohnte? Warum habt ihr sie denn in
jener Zeit nicht wieder an euch gerissen? \bibleverse{27}Ich habe dir
also nichts zuleide getan, du aber handelst unrecht gegen mich, indem du
Krieg mit mir anfängst: der HERR, der Richter, möge heute zwischen den
Israeliten und den Ammonitern richten\textless sup title=``oder:
entscheiden''\textgreater✲!« \bibleverse{28}Aber der König der Ammoniter
ließ die Vorstellungen unbeachtet, die Jephtha ihm hatte entbieten
lassen.

\hypertarget{dd-jephthas-geluxfcbde-sein-sieg-uxfcber-die-ammoniter}{%
\subparagraph{dd) Jephthas Gelübde; sein Sieg über die
Ammoniter}\label{dd-jephthas-geluxfcbde-sein-sieg-uxfcber-die-ammoniter}}

\bibleverse{29}Da kam der Geist des HERRN über Jephtha, und er zog durch
Gilead und Manasse, zog dann weiter nach Mizpe in Gilead, und von Mizpe
in Gilead zog er gegen die Ammoniter. \bibleverse{30}Damals brachte er
dem HERRN folgendes Gelübde dar: »Wenn du die Ammoniter wirklich in
meine Gewalt gibst, \bibleverse{31}so soll der, welcher mir (zuerst) aus
der Tür meines Hauses entgegenkommt, wenn ich wohlbehalten\textless sup
title=``oder: siegreich''\textgreater✲ aus dem Kriege mit den Ammonitern
heimkehre, der soll dem HERRN gehören, und ich will ihn als Brandopfer
darbringen!« \bibleverse{32}Hierauf zog Jephtha gegen die Ammoniter, um
ihnen eine Schlacht zu liefern, und der HERR gab sie in seine Hand:
\bibleverse{33}er brachte ihnen eine schwere Niederlage bei, von Aroer
an bis in die Gegend von Minnith {[}zwanzig Städte{]} und bis nach
Abel-Keramim\textless sup title=``d.h. Weingartenau''\textgreater✲. So
wurden die Ammoniter vor den Israeliten gedemütigt.

\hypertarget{ee-jephthas-ruxfcckkehr-und-die-ausfuxfchrung-des-geluxfcbdes-durch-die-opferung-seiner-tochter}{%
\subparagraph{ee) Jephthas Rückkehr und die Ausführung des Gelübdes
durch die Opferung seiner
Tochter}\label{ee-jephthas-ruxfcckkehr-und-die-ausfuxfchrung-des-geluxfcbdes-durch-die-opferung-seiner-tochter}}

\bibleverse{34}Als nun Jephtha nach Mizpe in sein Haus zurückkehrte,
siehe, da trat seine Tochter heraus ihm entgegen mit Handpauken und im
Reigentanz; sie war sein einziges Kind: außer ihr hatte er weder Sohn
noch Tochter. \bibleverse{35}Bei ihrem Anblick zerriß er seine Kleider
und rief aus: »Ach, meine Tochter! Du beugst mich tief darnieder! O daß
gerade du mich in solches Leid bringen mußt! Ich habe mich ja gegen den
HERRN verpflichtet und kann mein Gelübde nicht zurücknehmen!«
\bibleverse{36}Da erwiderte sie ihm: »Lieber Vater, hast du dich durch
ein Gelübde gegen den HERRN verpflichtet, so verfahre mit mir nach dem
Gelübde, das du ausgesprochen hast, nachdem der HERR dich Rache an
deinen Feinden, den Ammonitern, hat nehmen lassen!« \bibleverse{37}Dann
bat sie ihren Vater: »Nur dies eine möge mir noch gewährt werden: laß
mir noch zwei Monate Zeit, damit ich mich auf den Bergen ergehe und
meine Jungfrauschaft mit meinen Freundinnen beweine!« \bibleverse{38}Da
antwortete er ihr: »Ja, gehe hin!« und entließ sie auf zwei Monate; und
sie ging mit ihren Freundinnen hin und beweinte ihre Jungfrauschaft auf
den Bergen. \bibleverse{39}Aber nach Ablauf von zwei Monaten kehrte sie
zu ihrem Vater zurück, und er vollzog an ihr das Gelübde, das er getan
hatte; sie hatte aber nie mit einem Manne ein Verhältnis gehabt. Seitdem
ist die Sitte in Israel aufgekommen: \bibleverse{40}alljährlich ziehen
die israelitischen Mädchen aus, um die Tochter des Gileaditers Jephtha
in Liedern zu feiern, vier Tage im Jahr.

\hypertarget{ff-jephthas-siegreicher-kampf-mit-den-ephraimiten-und-sein-tod}{%
\subparagraph{ff) Jephthas siegreicher Kampf mit den Ephraimiten und
sein
Tod}\label{ff-jephthas-siegreicher-kampf-mit-den-ephraimiten-und-sein-tod}}

\hypertarget{section-11}{%
\section{12}\label{section-11}}

\bibleverse{1}Es wurden aber die Ephraimiten aufgeboten; sie zogen
nordwärts und ließen dem Jephtha sagen: »Warum bist du zum Krieg gegen
die Ammoniter ausgezogen, ohne uns zur Teilnahme am Feldzuge
aufzufordern? Nun wollen wir dir dein Haus über dem Kopf in Brand
stecken!« \bibleverse{2}Jephtha erwiderte ihnen: »Ich und mein Volk
haben einen schweren Streit mit den Ammonitern gehabt, und ich habe euch
um Hilfe angerufen, aber ihr habt mir keinen Beistand gegen sie
geleistet. \bibleverse{3}Als ich nun sah, daß du mir nicht zu Hilfe
kommen wolltest, setzte ich mein Leben aufs Spiel und zog gegen die
Ammoniter zu Felde, und der HERR gab sie in meine Gewalt. Warum zieht
ihr also jetzt gegen mich heran, um Händel mit mir anzufangen?«
\bibleverse{4}Darauf bot Jephtha alle Männer von Gilead auf und griff
die Ephraimiten an, und diese wurden von den Gileaditern geschlagen; sie
hatten nämlich die Behauptung ausgesprochen: »Flüchtige Ephraimiten seid
ihr; Gilead liegt nämlich in der Mitte von Ephraim und Manasse.«
\bibleverse{5}Die Gileaditer aber hatten die Jordanfurten nach Ephraim
zu besetzt. Sooft nun flüchtige Ephraimiten baten: »Laßt mich hinüber!«,
fragten die Männer von Gilead den Betreffenden, ob er ein Ephratiter✲
sei; antwortete er dann mit nein, \bibleverse{6}so forderte man ihn auf,
das Wort »Schibboleth«\textless sup title=``d.h. Strömung oder:
Ähre''\textgreater✲ auszusprechen. Sagte er dann »Sibboleth«, weil ihm
die richtige Aussprache unmöglich war, so ergriffen sie ihn und machten
ihn an den Jordanfurten nieder. So kamen damals 42000 Ephraimiten ums
Leben.

\bibleverse{7}Jephtha aber war sechs Jahre lang Richter in Israel; dann
starb Jephtha, der Gileaditer, und wurde in einer der Städte Gileads
begraben.

\hypertarget{g-die-kleinen-richter-ibzan-elon-und-abdon}{%
\paragraph{g) Die (kleinen) Richter Ibzan, Elon und
Abdon}\label{g-die-kleinen-richter-ibzan-elon-und-abdon}}

\bibleverse{8}Nach ihm war Ibzan aus Bethlehem Richter in Israel.
\bibleverse{9}Er hatte dreißig Söhne, und dreißig Töchter verheiratete
er nach auswärts, und dreißig Töchter✲ führte er seinen Söhnen von
auswärts als Gattinnen zu. Nachdem er sieben Jahre lang als Richter in
Israel gewaltet hatte, \bibleverse{10}starb Ibzan und wurde in Bethlehem
begraben.

\bibleverse{11}Nach ihm war Elon aus dem Stamme Sebulon Richter in
Israel und zwar zehn Jahre lang. \bibleverse{12}Als Elon aus Sebulon
dann starb, wurde er in Ajjalon im Lande Sebulon begraben.

\bibleverse{13}Nach ihm war Abdon aus Pirathon, der Sohn Hillels,
Richter in Israel. \bibleverse{14}Er hatte vierzig Söhne und dreißig
Enkel, die auf siebzig Eselsfüllen ritten. Nachdem er acht Jahre lang
Richter in Israel gewesen war, \bibleverse{15}starb Abdon aus Pirathon,
der Sohn Hillels, und wurde zu Pirathon im Lande Ephraim am
Amalekiterberge begraben.

\hypertarget{h-die-simsongeschichten-kap.-13-16}{%
\paragraph{h) Die Simsongeschichten (Kap.
13-16)}\label{h-die-simsongeschichten-kap.-13-16}}

\hypertarget{aa-die-vorgeschichte-philisterherrschaft-zweimalige-erscheinung-eines-engels-der-simsons-geburt-und-gottesweihe-ankuxfcndigt}{%
\subparagraph{aa) Die Vorgeschichte: Philisterherrschaft; zweimalige
Erscheinung eines Engels, der Simsons Geburt und Gottesweihe
ankündigt}\label{aa-die-vorgeschichte-philisterherrschaft-zweimalige-erscheinung-eines-engels-der-simsons-geburt-und-gottesweihe-ankuxfcndigt}}

\hypertarget{section-12}{%
\section{13}\label{section-12}}

\bibleverse{1}Als dann die Israeliten wiederum taten, was dem HERRN
mißfiel, ließ der HERR sie in die Hand der Philister fallen, vierzig
Jahre lang.

\bibleverse{2}Nun war da ein Mann aus Zora vom Geschlecht der Daniten
namens Manoah, dessen Frau unfruchtbar war und keine Kinder hatte.
\bibleverse{3}Da erschien der Engel des HERRN der Frau und sagte zu ihr:
»Du bist bis jetzt unfruchtbar gewesen und kinderlos geblieben, aber
wisse wohl: du wirst guter Hoffnung und Mutter eines Sohnes werden.
\bibleverse{4}So nimm dich nun fortan in acht, trinke keinen Wein und
keine berauschenden Getränke und iß nichts Unreines\textless sup
title=``vgl. 4.Mose 6,2-5''\textgreater✲. \bibleverse{5}Denn wisse wohl:
wenn du guter Hoffnung und Mutter eines Sohnes geworden bist, so darf
kein Schermesser auf sein Haupt kommen; denn der Knabe soll ein
Gottgeweihter von Geburt an sein, und er wird den Anfang damit machen,
Israel von der Herrschaft der Philister zu befreien.«

\bibleverse{6}Da ging die Frau hin und erzählte ihrem Manne: »Ein
Gottesmann ist zu mir gekommen, der ganz wie ein Engel Gottes aussah,
sehr furchterregend\textless sup title=``oder: ehrwürdig''\textgreater✲;
ich habe ihn aber nicht gefragt, woher er sei, und seinen Namen hat er
mir nicht angegeben. \bibleverse{7}Er hat mir aber gesagt: ›Du wirst
alsbald guter Hoffnung und Mutter eines Sohnes werden. So trinke denn
fortan keinen Wein und keine berauschenden Getränke und iß nichts
Unreines; denn ein Gottgeweihter soll der Knabe von Geburt an bis zu
seinem Todestage sein.‹« \bibleverse{8}Darauf betete Manoah zum HERRN
folgendermaßen: »Ach, Allherr, laß doch den Gottesmann, den du gesandt
hast, noch einmal zu uns kommen und uns darüber belehren, wie wir es mit
dem Knaben, der geboren werden soll, zu halten haben!« \bibleverse{9}Und
Gott erhörte das Gebet Manoahs, so daß der Engel Gottes nochmals zu der
Frau kam, während sie sich gerade auf dem Felde befand und ihr Mann
Manoah nicht bei ihr war. \bibleverse{10}Da lief die Frau eiligst hin
und berichtete es ihrem Manne mit den Worten: »Soeben ist mir der Mann
wieder erschienen, der schon neulich zu mir gekommen ist!«
\bibleverse{11}Da machte sich Manoah auf, hinter seiner Frau her, und
als er zu dem Manne gekommen war, fragte er ihn: »Bist du der Mann, der
meiner Frau die Verheißung gegeben hat?« \bibleverse{12}Er antwortete:
»Ja, ich bin es.« Da fragte Manoah weiter: »Wenn nun deine Verheißung
eintrifft, wie soll es dann mit dem Knaben gehalten werden, und was hat
er zu tun?« \bibleverse{13}Da antwortete der Engel des HERRN dem Manoah:
»Die Frau muß sich vor dem Genuß alles dessen hüten, was ich ihr
angegeben habe: \bibleverse{14}sie darf nichts genießen, was vom
Weinstock kommt; Wein und berauschende Getränke darf sie nicht trinken
und nichts Unreines essen; sie muß alles beobachten, was ich ihr geboten
habe.« \bibleverse{15}Da sagte Manoah zu dem Engel des HERRN: »Wir
möchten dich gern noch länger bei uns behalten und dir ein
Ziegenböckchen vorsetzen.« \bibleverse{16}Aber der Engel des HERRN
erwiderte dem Manoah: »Wenn du mich auch zum Bleiben veranlaßtest, würde
ich doch von deinem Mahl nichts genießen; willst du aber ein Brandopfer
zurüsten, so bringe es dem HERRN zu Ehren dar!« -- Manoah wußte nämlich
nicht, daß es der Engel des HERRN war. \bibleverse{17}Hierauf fragte
Manoah den Engel des HERRN: »Wie heißt du? Wir möchten dir gern eine
Ehre antun, wenn deine Verheißung eintrifft.« \bibleverse{18}Aber der
Engel des HERRN antwortete ihm: »Warum fragst du da nach meinem Namen,
der doch geheimnisvoll\textless sup title=``oder:
wunderbar''\textgreater✲ ist?« \bibleverse{19}Da holte Manoah das
Ziegenböckchen und das (zugehörige) Speisopfer und brachte es auf dem
Felsen dem HERRN dar, wobei dieser ein Wunder geschehen ließ, während
Manoah und seine Frau zusahen; \bibleverse{20}denn als die Flamme vom
Altar gen Himmel aufschlug, fuhr der Engel des HERRN in der Flamme des
Altars in die Höhe. Als Manoah und seine Frau das sahen, warfen sie sich
auf ihr Angesicht zur Erde nieder; \bibleverse{21}der Engel des HERRN
aber erschien dem Manoah und seiner Frau fortan nicht wieder. Doch
Manoah hatte nun erkannt, daß es der Engel des HERRN gewesen war,
\bibleverse{22}und sagte zu seiner Frau: »Wir müssen sicherlich sterben,
denn wir haben Gott gesehen!« \bibleverse{23}Aber seine Frau entgegnete
ihm: »Wenn der HERR uns hätte töten wollen, so hätte er kein Brand- und
Speisopfer von uns angenommen und hätte uns dies alles nicht sehen
lassen und jetzt uns nicht solche Ankündigungen gemacht.«~--
\bibleverse{24}Die Frau aber gebar einen Sohn und nannte ihn Simson; und
der Knabe wuchs heran, und der HERR segnete ihn.

\hypertarget{bb-simsons-werbung-um-eine-philisterin-seine-zerreiuxdfung-eines-luxf6wen-seine-hochzeit-sein-ruxe4tsel-und-seine-rache}{%
\subparagraph{bb) Simsons Werbung um eine Philisterin; seine Zerreißung
eines Löwen, seine Hochzeit, sein Rätsel und seine
Rache}\label{bb-simsons-werbung-um-eine-philisterin-seine-zerreiuxdfung-eines-luxf6wen-seine-hochzeit-sein-ruxe4tsel-und-seine-rache}}

\bibleverse{25}Als dann der Geist des HERRN sich in ihm zu regen begann
im Lager Dans\textless sup title=``vgl. 18,12''\textgreater✲ zwischen
Zora und Esthaol,

\hypertarget{section-13}{%
\section{14}\label{section-13}}

\bibleverse{1}ging Simson nach Thimna hinab und lernte dort ein Mädchen
unter den Töchtern der Philister kennen. \bibleverse{2}Nach seiner
Rückkehr erzählte er es seinen Eltern mit den Worten: »Ich habe in
Thimna ein Mädchen unter den Töchtern der Philister kennengelernt: nehmt
sie mir nun zur Frau!« \bibleverse{3}Da erwiderten ihm seine Eltern:
»Gibt's denn unter den Töchtern deiner Stammesgenossen und in unserem
ganzen Volke kein Weib mehr, daß du hingehen mußt, um dir eine Frau von
den heidnischen Philistern zu holen?« Doch Simson erwiderte seinem
Vater: »Diese nimm mir zur Frau! Denn gerade sie gefällt mir.«
\bibleverse{4}Seine Eltern wußten eben nicht, daß dies eine Fügung vom
HERRN war, der nach einem Anlaß zum Vorgehen gegen die Philister suchte;
denn damals waren die Philister Herren über Israel.

\bibleverse{5}So ging denn Simson mit seinen Eltern nach Thimna hinab,
und als sie bei den Weinbergen von Thimna angelangt waren, trat ihm
plötzlich ein junger Löwe brüllend in den Weg. \bibleverse{6}Da kam der
Geist des HERRN über ihn, so daß er den Löwen zerriß, wie man ein
Böckchen zerreißt, ohne daß er irgend etwas in der Hand hatte; seinen
Eltern erzählte er aber nichts von dem, was er getan hatte.
\bibleverse{7}Dann ging er (nach Thimna) hinab und besprach sich mit dem
Mädchen; denn sie gefiel ihm wohl. \bibleverse{8}Als er dann nach
einiger Zeit wieder hinging, um Hochzeit mit ihr zu machen, und vom Wege
abbog, um sich den toten Löwen noch einmal anzusehen, da befand sich im
Körper des Löwen ein Bienenschwarm und Honig. \bibleverse{9}Diesen nahm
er heraus in seine hohlen Hände und aß im Weitergehen davon; und als er
dann zu seinen Eltern gekommen war, gab er auch ihnen davon zu essen,
ohne ihnen jedoch mitzuteilen, daß er den Honig aus dem Körper des toten
Löwen herausgenommen hatte.

\bibleverse{10}Hierauf brachte sein Vater die Sache mit dem Mädchen in
Ordnung, und Simson richtete daselbst ein Gelage her; denn so pflegten
es die jungen Leute dort zu halten. \bibleverse{11}Als sie ihn nun
sahen, holten sie dreißig Brautgesellen herbei, die um ihn sein sollten.
\bibleverse{12}Zu diesen sagte Simson: »Ich will euch einmal ein Rätsel
aufgeben! Wenn ihr es mir innerhalb der sieben Tage des Gelages erraten
könnt und die Lösung findet, so gebe ich euch dreißig Unterkleider und
dreißig Festgewänder; \bibleverse{13}könnt ihr mir aber die Lösung nicht
angeben, so müßt ihr mir dreißig Unterkleider und dreißig Festgewänder
geben.« Sie antworteten: »Gib uns dein Rätsel auf, daß wir es hören!«
\bibleverse{14}Da sagte er zu ihnen: »Fraß✲ kam aus dem Fresser, und
Süßigkeit kam aus dem Starken.« Drei Tage lang waren sie nicht imstande,
das Rätsel zu lösen; \bibleverse{15}am vierten Tage aber sagten sie zu
Simsons Frau: »Berede deinen Mann, daß er uns die Lösung des Rätsels
angibt; sonst verbrennen wir dich samt deines Vaters Hause mit Feuer!
Ihr habt uns wohl hierher eingeladen, um uns arm zu machen?«
\bibleverse{16}Da brach die Frau Simsons in Tränen vor ihm aus und
sagte: »Nur Haß hegst du gegen mich, aber keine Liebe! Du hast meinen
Landsleuten das Rätsel aufgegeben und mir die Lösung nicht mitgeteilt!«
Da erwiderte er ihr: »Bedenke doch: meinen eigenen Eltern habe ich die
Lösung nicht verraten und sollte sie dir angeben?« \bibleverse{17}So
weinte sie denn vor ihm die sieben Tage hindurch, solange sie das Gelage
hielten; endlich am siebten Tage teilte er ihr die Lösung mit, weil sie
ihm keine Ruhe ließ, sie aber verriet die Lösung ihren Landsleuten.
\bibleverse{18}Da sagten denn die Männer der Stadt am siebten Tage zu
ihm, ehe die Sonne unterging: »Was ist süßer als Honig, und was ist
stärker als ein Löwe?« Er antwortete ihnen: »Hättet ihr nicht mit meinem
Rinde gepflügt, so hättet ihr mein Rätsel nicht erraten!«

\bibleverse{19}Da kam der Geist des HERRN über ihn, so daß er nach
Askalon hinabging und dort dreißig Mann von ihnen erschlug; diesen nahm
er alles ab, was sie an sich hatten, und gab die Festgewänder denen, die
das Rätsel gelöst hatten; dann kehrte er voller Zorn in das Haus seines
Vaters zurück. \bibleverse{20}Simsons Frau aber wurde an einen von
seinen Hochzeitsgenossen verheiratet, der sein Brautführer gewesen war.

\hypertarget{cc-mehrere-rache--und-krafttaten-simsons}{%
\subparagraph{cc) Mehrere Rache- und Krafttaten
Simsons}\label{cc-mehrere-rache--und-krafttaten-simsons}}

\hypertarget{simson-von-seinem-schwiegervater-betrogen-ruxe4cht-sich-an-den-philistern-durch-eine-fuchshetze}{%
\paragraph{Simson, von seinem Schwiegervater betrogen, rächt sich an den
Philistern durch eine
Fuchshetze}\label{simson-von-seinem-schwiegervater-betrogen-ruxe4cht-sich-an-den-philistern-durch-eine-fuchshetze}}

\hypertarget{section-14}{%
\section{15}\label{section-14}}

\bibleverse{1}Nach einiger Zeit aber, in den Tagen der Weizenernte,
wollte Simson seine Frau besuchen, indem er ihr ein Ziegenböckchen
mitbrachte; er dachte dabei: »Ich will zu meiner Frau in die
Frauenwohnung gehen.« Doch ihr Vater gestattete ihm nicht hineinzugehen,
\bibleverse{2}sondern sagte: »Ich mußte doch fest annehmen, daß du
nichts mehr von ihr wissen wolltest; daher habe ich sie einem von deinen
Hochzeitsgenossen zur Frau gegeben. Aber ihre jüngere Schwester ist noch
schöner als sie: die soll deine Frau werden an ihrer Statt!«
\bibleverse{3}Da erwiderte ihnen Simson: »Diesmal haben mir die
Philister nichts vorzuwerfen, wenn ich ihnen einen Denkzettel gebe!«
\bibleverse{4}Er ging also hin und fing dreihundert Füchse\textless sup
title=``oder: Schakale''\textgreater✲, nahm dann Feuerbrände, kehrte
Schwanz gegen Schwanz und brachte einen Feuerbrand mitten zwischen je
zwei Schwänzen an. \bibleverse{5}Hierauf zündete er die Feuerbrände an,
jagte die Tiere in die Kornfelder der Philister und setzte dadurch
sowohl die Garbenhaufen als auch das auf dem Halm stehende Getreide und
selbst die Weinberge und Ölbaumgärten in Brand. \bibleverse{6}Als nun
die Philister fragten: »Wer hat das getan?«, hieß es: »Simson, der
Schwiegersohn des Thimniters, weil der ihm seine Frau genommen und sie
einem von seinen Hochzeitsgenossen gegeben hat.« Da kamen die Philister
herangezogen und verbrannten sie\textless sup title=``d.h. die
Frau''\textgreater✲ samt ihres Vaters Hause. \bibleverse{7}Simson aber
sagte zu ihnen: »Wenn ihr es so treibt, will ich nicht eher ruhen, als
bis ich mich an euch gerächt habe!« \bibleverse{8}Hierauf richtete er
sie mit Schlägen so zu, daß kein gesundes Glied an ihnen blieb; dann
ging er hinab und nahm seinen Wohnsitz in der Felsenkluft von Etham.

\hypertarget{simsons-gefangennahme-und-heldentat-in-lehi}{%
\paragraph{Simsons Gefangennahme und Heldentat in
Lehi}\label{simsons-gefangennahme-und-heldentat-in-lehi}}

\bibleverse{9}Da zogen die Philister hinauf, lagerten sich in Juda und
breiteten sich bei Lehi aus. \bibleverse{10}Als nun die Judäer fragten:
»Warum seid ihr gegen uns heraufgezogen?«, antworteten sie: »Um Simson
gefangenzunehmen, sind wir hergekommen; wir wollen ihm Gleiches mit
Gleichem vergelten.« \bibleverse{11}Da zogen dreitausend Mann aus Juda
nach der Felsenkluft von Etham hinab und sagten zu Simson: »Weißt du
nicht, daß die Philister Herren über uns sind? Was hast du uns da
angerichtet!« Simson antwortete ihnen: »Wie sie mir getan, so habe ich
ihnen wieder getan.« \bibleverse{12}Da sagten sie zu ihm: »Wir sind
hergekommen, um dich zu binden und dich den Philistern auszuliefern.«
Simson entgegnete ihnen: »Leistet mir einen Schwur, daß ihr selbst mich
nicht erschlagen wollt!« \bibleverse{13}Sie antworteten ihm: »Nein, wir
wollen dich nur binden und dich ihnen dann ausliefern; aber töten wollen
wir dich nicht.« Darauf banden sie ihn mit zwei neuen Stricken und
führten ihn aus der Felsenkluft weg hinauf.

\bibleverse{14}Als er nun bis Lehi gekommen war und die Philister seine
Ankunft mit Jubelgeschrei begrüßten, da kam der Geist des HERRN über
ihn, so daß die Stricke an seinen Armen wie Flachsfäden wurden, die vom
Feuer versengt sind, und seine Fesseln ihm an den Händen zergingen.
\bibleverse{15}Als er dann einen noch frischen Eselskinnbacken fand,
streckte er seine Hand aus, ergriff ihn und erschlug damit tausend Mann.
\bibleverse{16}Da rief Simson aus: »Mit dem Eselskinnbacken habe ich sie
gründlich geschoren! Mit dem Eselskinnbacken habe ich tausend Mann
erschlagen!« \bibleverse{17}Als er diese Worte ausgerufen hatte, warf er
den Kinnbacken weg; und man nannte jenen Ort seitdem »Kinnbackenhöhe«.

\bibleverse{18}Nun aber dürstete ihn sehr; daher betete er laut zum
HERRN: »Du hast durch die Hand deines Knechtes diesen großen Sieg
herbeigeführt, und nun soll ich verdursten und den Heiden in die Hände
fallen?« \bibleverse{19}Da spaltete Gott die Vertiefung in dem
Kinnbacken, so daß Wasser daraus hervorfloß; und als er getrunken hatte,
kehrten seine Lebensgeister zurück, und er erholte sich wieder; daher
nannte man (jenen Ort) »Quelle des Rufers\textless sup title=``oder:
Beters''\textgreater✲«; sie befindet sich bei Lehi noch heutigen
Tages.~-- \bibleverse{20}Er war dann zwanzig Jahre lang Richter in
Israel zur Zeit der Herrschaft der Philister.

\hypertarget{simsons-krafttat-in-gaza}{%
\paragraph{Simsons Krafttat in Gaza}\label{simsons-krafttat-in-gaza}}

\hypertarget{section-15}{%
\section{16}\label{section-15}}

\bibleverse{1}Als Simson sich einst nach Gaza begeben hatte, sah er dort
eine Dirne und kehrte bei ihr ein. \bibleverse{2}Als nun den Einwohnern
von Gaza berichtet wurde, Simson sei dorthin gekommen, umstellten sie
ihn und lauerten ihm die ganze Nacht am Stadttor auf, verhielten sich
aber die ganze Nacht hindurch ruhig, weil sie dachten: »(Wir wollen
warten), bis es am Morgen hell wird, dann wollen wir ihn erschlagen!«
\bibleverse{3}Simson aber blieb nur bis Mitternacht liegen; um
Mitternacht aber stand er auf, faßte die beiden Flügel des Stadttors
samt den beiden Pfosten, riß sie mitsamt dem Riegel heraus, lud sie sich
auf die Schultern und trug sie auf den Gipfel des Berges, der gegen
Hebron hin\textless sup title=``=~östlich von Hebron''\textgreater✲
liegt.

\hypertarget{dd-simson-von-delila-verraten-von-den-philistern-geblendet-und-in-gaza-gefangengesetzt}{%
\subparagraph{dd) Simson von Delila verraten, von den Philistern
geblendet und in Gaza
gefangengesetzt}\label{dd-simson-von-delila-verraten-von-den-philistern-geblendet-und-in-gaza-gefangengesetzt}}

\bibleverse{4}Später gewann er ein Mädchen im Tale Sorek lieb, die hieß
Delila. \bibleverse{5}Zu dieser kamen die Fürsten der Philister hinauf
und sagten zu ihr: »Rede ihm zu und suche zu erfahren, woher seine große
Kraft stammt und wie wir ihn überwältigen können, um ihn zu binden und
unschädlich zu machen; wir würden dir dann auch jeder 1100 Silberstücke
geben.« \bibleverse{6}Da bat Delila den Simson: »Verrate mir doch, woher
deine große Kraft kommt und womit man dich binden müßte, um dich zu
überwältigen.« \bibleverse{7}Simson antwortete ihr: »Wenn man mich mit
sieben frischen, noch nicht ausgetrockneten Sehnen bände, so würde ich
schwach sein und wie jeder andere Mensch werden.« \bibleverse{8}Da
brachten die Fürsten der Philister sieben frische, noch nicht
ausgetrocknete Sehnen zu ihr hinauf, und sie band ihn damit,
\bibleverse{9}während sich Leute, die ihn überfallen sollten, bei ihr im
Frauengemach befanden. Als sie ihm nun zurief: »Die Philister überfallen
dich, Simson!«, da zerriß er die Sehnen, wie ein Wergfaden zerreißt,
wenn er Feuer riecht\textless sup title=``=~gefangen hat''\textgreater✲;
und seine Kraft blieb unerklärt.

\bibleverse{10}Da sagte Delila zu Simson: »Siehe, du hast mich betrogen
und mir Lügen vorgeredet. Verrate mir jetzt doch, womit man dich binden
kann!« \bibleverse{11}Da antwortete er ihr: »Wenn man mich fest mit
neuen Seilen bände, die noch zu keiner Arbeit benutzt sind, so würde ich
schwach sein und wie jeder andere Mensch werden.« \bibleverse{12}Da nahm
Delila neue Seile und band ihn damit; dann rief sie ihm zu: »Die
Philister überfallen dich, Simson!« -- es befanden sich aber (auch
diesmal) Leute, die ihn überfallen sollten, im Frauengemach --; da riß
er die Stricke von seinen Armen ab wie einen Faden.

\bibleverse{13}Nun sagte Delila zu Simson: »Bisher hast du mich betrogen
und mir Lügen vorgeredet; verrate mir doch, womit man dich binden kann!«
Da antwortete er ihr: »Wenn du die sieben Locken\textless sup
title=``oder: Strähnen''\textgreater✲ meines Kopfes in den Aufzug eines
Gewebes hineinwebtest und sie mit dem Pflock festschlügest, so würde ich
schwach sein und wie jeder andere Mensch werden.« Da ließ sie ihn
einschlafen, webte die sieben Locken seines Kopfes in den Aufzug eines
Gewebes hinein \bibleverse{14}und schlug den Aufzug mit dem Pflock fest.
Als sie ihm nun zurief: »Die Philister überfallen dich, Simson!« und er
aus seinem Schlaf erwachte, riß er den Webepflock samt dem Aufzug
heraus.

\bibleverse{15}Da sagte sie zu ihm: »Wie kannst du behaupten, du habest
mich lieb, während doch dein Herz mir gar nicht gehört? Du hast mich nun
schon dreimal betrogen und mir nicht verraten, woher deine große Kraft
rührt.« \bibleverse{16}Als sie ihm nun alle Tage mit ihren Reden
zusetzte und ihm keine Ruhe ließ, so daß er gar keine Freude mehr am
Leben hatte, \bibleverse{17}schüttete er ihr sein ganzes Herz aus, so
daß er zu ihr sagte: »Noch kein Schermesser ist auf mein Haupt gekommen;
denn ich bin ein Gottgeweihter von meiner Geburt an; würde ich
geschoren, so würde meine Kraft von mir weichen; ich würde dann schwach
sein und wie alle anderen Menschen werden.«

\bibleverse{18}Da nun Delila erkannte, daß er ihr sein ganzes Herz
ausgeschüttet hatte, ließ sie die Fürsten der Philister rufen und ihnen
sagen: »Diesmal müßt ihr heraufkommen, denn er hat mir sein ganzes Herz
entdeckt.« Da begaben sich die Fürsten der Philister zu ihr hinauf und
brachten auch das Geld mit. \bibleverse{19}Als sie ihn dann auf ihrem
Schoße hatte einschlafen lassen, rief sie einen Mann herbei, der die
sieben Locken auf seinem Haupt abscheren mußte; da wurde er schwächer
und schwächer, und seine Kraft wich von ihm. \bibleverse{20}Als sie nun
rief: »Die Philister überfallen dich, Simson!« und er aus seinem Schlaf
erwachte, dachte er: »Ich werde mich auch jetzt wie die vorigen Male
frei machen und glücklich davonkommen!« Er wußte ja nicht, daß der HERR
von ihm gewichen war. \bibleverse{21}Da ergriffen ihn die Philister,
stachen ihm die Augen aus und führten ihn nach Gaza hinab; dort legten
sie ihn in eherne Doppelketten, und er mußte im Gefängnis die Handmühle
drehen. \bibleverse{22}Allmählich wuchs ihm aber das Haupthaar wieder,
nachdem es abgeschoren worden war.

\hypertarget{ee-simsons-erniedrigung-letzte-rache-und-tod}{%
\subparagraph{ee) Simsons Erniedrigung, letzte Rache und
Tod}\label{ee-simsons-erniedrigung-letzte-rache-und-tod}}

\bibleverse{23}Nun kamen einst die Fürsten der Philister zusammen, um
ihrem Gott Dagon ein großes Schlachtopfer zu veranstalten und ein
Freudenfest zu feiern; denn sie sagten: »Unser Gott hat unsern Feind
Simson in unsere Hand gegeben!« \bibleverse{24}Als ihn nun das Volk
erblickte, priesen sie ihren Gott, indem sie ausriefen: »Unser Gott hat
unsern Feind in unsere Hand gegeben, der unsere Felder verwüstet und
viele von unsern Leuten erschlagen hat.« \bibleverse{25}Als nun ihr Herz
guter Dinge war, riefen sie: »Laßt Simson herkommen, damit er uns
belustige\textless sup title=``oder: uns etwas vortanze, oder: uns eins
aufspiele''\textgreater✲!« So ließ man denn Simson aus dem Gefängnis
holen, und er mußte vor ihnen spielen\textless sup title=``oder:
tanzen''\textgreater✲. Da man ihn nun zwischen die Säulen (der Halle)
gestellt hatte, \bibleverse{26}bat Simson den Burschen\textless sup
title=``oder: Diener''\textgreater✲, der ihn an der Hand gefaßt hielt:
»Laß mich doch mal los, damit ich die Säulen betaste, auf denen das Haus
ruht: ich möchte mich an sie anlehnen!« \bibleverse{27}Das Haus war aber
voll von Männern und Frauen; auch alle Fürsten der Philister waren dort
anwesend, und auf dem Dache befanden sich gegen dreitausend Männer und
Frauen, die dem Spiel\textless sup title=``oder: Tanz''\textgreater✲
Simsons zugesehen hatten. \bibleverse{28}Da betete Simson zum HERRN mit
den Worten: »O HERR, mein Gott! Gedenke doch meiner und verleihe mir nur
dies eine Mal noch Kraft, o Gott, damit ich Rache an den Philistern
nehme für eins von meinen beiden Augen!« \bibleverse{29}Darauf umfaßte
Simson die beiden Mittelsäulen, auf denen das Haus ruhte, die eine mit
seinem rechten, die andere mit seinem linken Arm, und stemmte sich gegen
sie; \bibleverse{30}und indem er ausrief: »Nun will ich mit den
Philistern sterben!«, neigte er sich mit aller Kraft vornüber. Da
stürzte das Haus auf die Fürsten und auf alle Leute, die darin waren,
und die Zahl der Toten, die er im Sterben tötete, war größer als die
Zahl derer, die er während seines Lebens getötet hatte.
\bibleverse{31}Darauf kamen seine Stammesgenossen und seine ganze
Familie hinab und holten seinen Leichnam; sie brachten ihn dann hinauf
und begruben ihn zwischen Zora und Esthaol im Grabe seines Vaters
Manoah. Er war aber zwanzig Jahre lang Richter in Israel gewesen.

\hypertarget{iii.-die-beiden-anhuxe4nge-des-richterbuches-innere-nuxf6te-in-israel-kap.-17-21}{%
\subsection{III. Die beiden Anhänge des Richterbuches (innere Nöte in
Israel) (Kap.
17-21)}\label{iii.-die-beiden-anhuxe4nge-des-richterbuches-innere-nuxf6te-in-israel-kap.-17-21}}

\hypertarget{erster-anhang-michas-bilderdienst-in-ephraim-gruxfcndung-des-heiligtums-in-dan}{%
\subsubsection{1. Erster Anhang: Michas Bilderdienst in Ephraim;
Gründung des Heiligtums in
Dan}\label{erster-anhang-michas-bilderdienst-in-ephraim-gruxfcndung-des-heiligtums-in-dan}}

\hypertarget{a-micha-und-seine-mutter-richten-guxf6tzendienst-auf-dem-gebirge-ephraim-ein}{%
\paragraph{a) Micha und seine Mutter richten Götzendienst auf dem
Gebirge Ephraim
ein}\label{a-micha-und-seine-mutter-richten-guxf6tzendienst-auf-dem-gebirge-ephraim-ein}}

\hypertarget{section-16}{%
\section{17}\label{section-16}}

\bibleverse{1}Es war einst ein Mann vom Gebirge Ephraim mit Namen Micha;
\bibleverse{2}der sagte zu seiner Mutter: »Die elfhundert Silberstücke,
die dir entwendet worden sind und um derenwillen du einen Fluch
ausgestoßen und ihn gar vor meinen Ohren ausgesprochen hast -- wisse,
das Geld ist in meinem Besitz: ich selbst habe es genommen; nun aber
will ich es dir zurückgeben!« Da antwortete seine Mutter: »Gesegnet
seist du vom HERRN, mein Sohn!« \bibleverse{3}Als er nun die elfhundert
Silberstücke seiner Mutter zurückgegeben hatte, sagte diese: »Ich will
das Geld dem HERRN weihen (und es) aus meiner Hand zugunsten meines
Sohnes (hingeben): es soll ein geschnitztes und gegossenes Gottesbild
davon angefertigt werden.« \bibleverse{4}Als er nun das Geld seiner
Mutter zurückgegeben hatte, nahm diese zweihundert Silberstücke und gab
sie einem Goldschmied, der davon ein geschnitztes und gegossenes
Gottesbild anfertigte, das nun im Hause Michas aufgestellt wurde.
\bibleverse{5}So besaß denn dieser Micha ein Gotteshaus; er ließ dann
noch ein kostbares Schulterkleid\textless sup title=``oder:
Priestergewand, vgl. 2.Mose 28,6-14''\textgreater✲ und einen Hausgott
anfertigen und stellte einen seiner Söhne an, daß er ihm (dem Micha) als
Priester diente. \bibleverse{6}Zu jener Zeit gab es noch keinen König in
Israel; ein jeder tat, was ihm beliebte.

\hypertarget{b-micha-bestellt-einen-wandernden-leviten-aus-juda-zum-priester-an-seinem-heiligtum}{%
\paragraph{b) Micha bestellt einen wandernden Leviten aus Juda zum
Priester an seinem
Heiligtum}\label{b-micha-bestellt-einen-wandernden-leviten-aus-juda-zum-priester-an-seinem-heiligtum}}

\bibleverse{7}Nun war da ein junger Mann aus Bethlehem in Juda aus dem
Geschlecht Judas; der war ein Levit und hielt sich dort als Fremder auf.
\bibleverse{8}Dieser Mann verließ die Stadt Bethlehem in Juda, um sich
an irgendeinem anderen Orte, wo es sich gerade träfe, als Fremder
niederzulassen, und kam bei seiner Wanderung im Gebirge Ephraim zum
Hause Michas. \bibleverse{9}Dieser fragte ihn: »Woher kommst du?« Er
antwortete ihm: »Ich bin ein Levit aus Bethlehem in Juda und bin
unterwegs, um mich an irgendeinem Ort niederzulassen, wo ich etwas
Geeignetes finde.« \bibleverse{10}Da sagte Micha zu ihm: »Bleibe bei mir
und sei mir ein Vater✲ und Priester, so will ich dir jährlich zehn
Silberstücke geben und für Kleidung und deinen Lebensunterhalt
aufkommen.« Als er dann dem Leviten weiter zuredete,
\bibleverse{11}erklärte dieser sich einverstanden, bei dem Manne zu
bleiben; und der junge Mann galt ihm wie einer seiner Söhne.
\bibleverse{12}Micha nahm also den Leviten in seinen Dienst, so daß der
junge Mann sein Priester wurde und im Hause Michas blieb.
\bibleverse{13}Micha aber dachte: »Jetzt bin ich gewiß, daß der HERR es
mir wird glücken lassen, weil ich einen Leviten zum Priester (gewonnen)
habe.«

\hypertarget{c-die-danitischen-kundschafter-im-hause-michas-das-ergebnis-ihrer-auskundschaftung-der-gegend-um-die-stadt-lais}{%
\paragraph{c) Die danitischen Kundschafter im Hause Michas; das Ergebnis
ihrer Auskundschaftung der Gegend um die Stadt
Lais}\label{c-die-danitischen-kundschafter-im-hause-michas-das-ergebnis-ihrer-auskundschaftung-der-gegend-um-die-stadt-lais}}

\hypertarget{section-17}{%
\section{18}\label{section-17}}

\bibleverse{1}Zu jener Zeit gab es noch keinen König in Israel; und der
Stamm der Daniten suchte sich damals gerade ein Gebiet zur Ansiedelung,
denn es war ihm bis dahin inmitten der israelitischen Stämme noch kein
Gebiet als Erbbesitz zugefallen. \bibleverse{2}Daher schickten die
Daniten fünf Männer aus der Gesamtheit ihrer Stammesgenossen ab,
kriegstüchtige Männer aus Zora und Esthaol, um das Land
auszukundschaften und zu erforschen, und gaben ihnen den Auftrag, sich
zur Erforschung des Landes aufzumachen. So kamen diese denn auf das
Gebirge Ephraim in den Wohnort Michas und übernachteten dort.
\bibleverse{3}Als sie nun in der Nähe von Michas Hause waren und den
jungen Mann, den Leviten, an seiner Sprache✲ erkannten, kehrten sie dort
ein und fragten ihn: »Wer hat dich hierher gebracht? Was tust du hier,
und wie geht's dir hier?« \bibleverse{4}Da antwortete er ihnen: »So und
so hat Micha mit mir verhandelt und mich dann in seinen Dienst genommen,
und ich bin sein Priester geworden.« \bibleverse{5}Da baten sie ihn:
»Befrage doch Gott, damit wir erfahren, ob das Unternehmen, für das wir
jetzt unterwegs sind, glücklichen Erfolg haben wird.« \bibleverse{6}Der
Priester gab ihnen hierauf den Bescheid: »Zieht getrost hin: euer
jetziges Unternehmen ist dem HERRN wohlgefällig!« \bibleverse{7}Da zogen
die fünf Männer weiter, gelangten nach Lais und sahen, daß die dortige
Bevölkerung nach der Weise der Sidonier sorglos lebte, friedlich und in
Sicherheit, und an nichts Mangel hatte, was es auf Erden gibt, vielmehr
im Besitz von Reichtum war; daß sie auch von den Sidoniern entfernt
wohnten und mit den Aramäern✲ in keiner Verbindung standen.
\bibleverse{8}Als sie daher zu ihren Stammesgenossen nach Zora und
Esthaol zurückgekehrt waren und diese sie fragten: »Welche Auskunft
bringt ihr uns?«, \bibleverse{9}antworteten sie: »Auf! Laßt uns gegen
sie zu Felde ziehen! Denn wir haben uns das Land angesehen und gefunden,
daß es ganz vortrefflich ist. Und da sitzt ihr noch untätig da? Säumt
nicht, euch aufzumachen und hinzuziehen, um das Land in Besitz zu
nehmen! \bibleverse{10}Wenn ihr hinkommt, werdet ihr ein sorgloses Volk
antreffen, dessen Land sich nach allen Seiten hin weit ausdehnt. Ja,
Gott hat es in eure Hand gegeben, eine Gegend, die an nichts Mangel hat,
was es auf Erden gibt.«

\hypertarget{d-die-zur-eroberung-der-stadt-lais-ausgesandten-daniten-rauben-unterwegs-die-heiligtuxfcmer-michas-und-nehmen-den-priester-mit}{%
\paragraph{d) Die zur Eroberung der Stadt Lais ausgesandten Daniten
rauben unterwegs die Heiligtümer Michas und nehmen den Priester
mit}\label{d-die-zur-eroberung-der-stadt-lais-ausgesandten-daniten-rauben-unterwegs-die-heiligtuxfcmer-michas-und-nehmen-den-priester-mit}}

\bibleverse{11}Da brachen sechshundert wohlbewaffnete Männer von dort
aus dem Stamme der Daniten, aus Zora und Esthaol, auf;
\bibleverse{12}sie zogen hinauf und lagerten bei Kirjath-Jearim in Juda;
daher führt der betreffende Platz den Namen ›Dans Lager‹ bis auf den
heutigen Tag; er liegt bekanntlich westlich von Kirjath-Jearim.
\bibleverse{13}Von dort zogen sie weiter in das Gebirge Ephraim und
kamen in den Wohnort Michas. \bibleverse{14}Da machten die fünf Männer,
die vordem zur Auskundschaftung des Gebietes der Stadt Lais ausgezogen
waren, ihren Stammesgenossen die Mitteilung: »Wißt ihr wohl, daß sich in
den Gehöften hier ein kostbares Priestergewand und ein Hausgott sowie
ein geschnitztes und gegossenes Gottesbild\textless sup title=``vgl.
17,4-5''\textgreater✲ befinden? Bedenkt also, was ihr zu tun habt!«
\bibleverse{15}Da bogen sie vom Wege dorthin ab, traten in die Wohnung
des jungen Leviten, in das Haus Michas ein, und begrüßten ihn.
\bibleverse{16}Während dann die sechshundert bewaffneten Männer, die zu
den Daniten gehörten, draußen am Eingang des Tores blieben,
\bibleverse{17}stiegen die fünf Männer, die vordem zur Auskundschaftung
des Landes hergekommen waren, hinauf, gingen in das Haus hinein und
nahmen das geschnitzte Bild sowie das Priestergewand, den Hausgott und
das gegossene Bild an sich, während der Priester draußen am Toreingang
bei den sechshundert bewaffneten Kriegern stand. \bibleverse{18}Als jene
nämlich in das Haus Michas eingetreten waren und das Schnitzbild sowie
das Priestergewand, den Hausgott und das Gußbild wegnahmen, sagte der
Priester zu ihnen: »Was macht ihr da?« \bibleverse{19}Sie antworteten
ihm: »Schweige still, lege dir die Hand auf den Mund, komm mit uns und
werde unser Vater✲ und Priester! Ist es besser für dich, Priester für
das Haus eines einzelnen Mannes zu sein oder Priester für einen ganzen
Stamm und für ein Geschlecht in Israel?« \bibleverse{20}Da erklärte sich
der Priester mit Freuden einverstanden: er nahm das Priestergewand, den
Hausgott und das geschnitzte Bild und trat mitten unter die Kriegsleute.
\bibleverse{21}Darauf wandten sie sich zum Abzug, nachdem sie noch die
Frauen mit den Kindern sowie das Herdenvieh und die wertvollen
Gegenstände an ihre Spitze gestellt hatten. \bibleverse{22}Kaum hatten
sie sich aber eine Strecke vom Hause Michas entfernt, als sich die
Männer, die in den Gehöften beim Hause Michas wohnten, zusammenscharten
und die Daniten einholten. \bibleverse{23}Als sie dann den Daniten
zuriefen, wandten diese sich um und fragten Micha: »Was soll dieser
Auflauf bedeuten?« \bibleverse{24}Er antwortete: »Meinen Gott, den ich
mir gemacht habe, habt ihr mit euch genommen samt dem Priester und seid
weggezogen: was bleibt mir da noch? Wie könnt ihr mich nur fragen, was
ich hier will?« \bibleverse{25}Aber die Daniten erwiderten ihm: »Laß
dein Geschrei uns hier nicht länger hören, sonst könnten erbitterte
Männer über euch herfallen, und es könnte dich und deine Angehörigen das
Leben kosten!« \bibleverse{26}Darauf zogen die Daniten ihres Weges, und
Micha, der wohl einsah, daß sie stärker waren als er, wandte sich um und
kehrte nach Hause zurück.

\hypertarget{e-die-daniten-erobern-lais-und-richten-dort-den-bilderdienst-michas-und-das-priestertum-des-mosaiden-jonathan-ein}{%
\paragraph{e) Die Daniten erobern Lais und richten dort den Bilderdienst
Michas und das Priestertum des Mosaiden Jonathan
ein}\label{e-die-daniten-erobern-lais-und-richten-dort-den-bilderdienst-michas-und-das-priestertum-des-mosaiden-jonathan-ein}}

\bibleverse{27}Nachdem aber die Daniten das Gottesbild✲, das Micha sich
angefertigt hatte, samt dem Priester, der bei ihm gewesen war,
mitgenommen hatten, überfielen sie Lais, dessen Bevölkerung friedlich
und sorglos war, machten alle Einwohner mit dem Schwert nieder und
ließen die Stadt in Flammen aufgehen, \bibleverse{28}ohne daß ihr jemand
zu Hilfe gekommen wäre; denn der Ort lag von Sidon weit entfernt und
stand auch mit den Aramäern\textless sup title=``=~Syrern;
V.7''\textgreater✲ in keiner Verbindung; er lag nämlich in dem Tal, das
sich nach Beth-Rehob hin erstreckt. Sie bauten dann die Stadt wieder auf
und siedelten sich in ihr an, \bibleverse{29}nannten aber die Stadt
›Dan‹ nach dem Namen ihres Ahnherrn Dan, dem Sohne Israels, während der
Ort früher Lais geheißen hatte. \bibleverse{30}Sodann stellten die
Daniten das geschnitzte Gottesbild bei sich auf, und Jonathan, der Sohn
Gersoms, des Sohnes Moses, er und seine Nachkommen, waren Priester bei
dem Stamme der Daniten bis zu der Zeit, wo die Bevölkerung in die
Verbannung\textless sup title=``oder: Gefangenschaft''\textgreater✲
ziehen mußte. \bibleverse{31}Das geschnitzte Bild aber, das Micha hatte
anfertigen lassen, war bei ihnen aufgestellt die ganze Zeit hindurch,
solange das Haus Gottes sich in Silo befand.

\hypertarget{zweiter-anhang-der-benjaminitische-buxfcrgerkrieg}{%
\subsubsection{2. Zweiter Anhang: Der benjaminitische
Bürgerkrieg}\label{zweiter-anhang-der-benjaminitische-buxfcrgerkrieg}}

\hypertarget{a-die-greueltat-der-einwohner-von-gibea}{%
\paragraph{a) Die Greueltat der Einwohner von
Gibea}\label{a-die-greueltat-der-einwohner-von-gibea}}

\hypertarget{aa-der-besuch-eines-leviten-in-bethlehem-zur-wiedergewinnung-seines-nebenweibes}{%
\subparagraph{aa) Der Besuch eines Leviten in Bethlehem zur
Wiedergewinnung seines
Nebenweibes}\label{aa-der-besuch-eines-leviten-in-bethlehem-zur-wiedergewinnung-seines-nebenweibes}}

\hypertarget{section-18}{%
\section{19}\label{section-18}}

\bibleverse{1}Zu jener Zeit, als es noch keinen König in Israel gab,
begab es sich, daß ein Levit, der ganz hinten im Gebirge Ephraim als
Fremdling wohnte, sich ein Mädchen aus Bethlehem in Juda zum Nebenweibe
nahm. \bibleverse{2}Aber sein Nebenweib überwarf sich mit ihm, verließ
ihn und kehrte in ihres Vaters Haus nach Bethlehem in Juda zurück; dort
blieb sie vier Monate lang. \bibleverse{3}Da machte ihr Mann sich auf
den Weg und zog ihr nach, um sie durch freundliches Zureden zur Rückkehr
zu bewegen; seinen Diener\textless sup title=``=~Burschen,
Knecht''\textgreater✲ und ein paar Esel hatte er bei sich.
Sie\textless sup title=``d.h. das junge Weib''\textgreater✲ führte ihn
dann in das Haus ihres Vaters, und als dieser ihn sah, kam er ihm
freundlich entgegen. \bibleverse{4}Sein Schwiegervater, der Vater des
jungen Weibes, hielt ihn dann zurück, so daß er drei Tage bei ihm blieb;
sie aßen und tranken miteinander und übernachteten dort.
\bibleverse{5}Am vierten Tage aber, als sie in der Frühe aufgestanden
waren und er aufbrechen wollte, sagte der Vater des jungen Weibes zu
seinem Schwiegersohn: »Stärke dich noch mit einem Imbiß, dann mögt ihr
euch auf den Weg machen!« \bibleverse{6}So setzten sie sich denn hin,
und die beiden aßen und tranken zusammen; dann bat der Vater des jungen
Weibes den Mann: »Entschließe dich doch, über Nacht noch hier zu
bleiben, und sei guter Dinge!« \bibleverse{7}Und als der Mann aufstand,
um sich auf den Weg zu machen, nötigte ihn sein Schwiegervater, so daß
er wiederum über Nacht dablieb. \bibleverse{8}Als er dann am fünften
Tage frühmorgens aufbrechen wollte, sagte der Vater des jungen Weibes
wiederum: »Stärke dich doch erst und wartet noch bis zum Nachmittag!«,
und so aßen sie beide nochmals zusammen. \bibleverse{9}Als dann der Mann
aufstand, um mit seinem Nebenweibe und seinem Diener aufzubrechen, sagte
sein Schwiegervater, der Vater des jungen Weibes, zu ihm: »Sieh doch,
der Tag geht zu Ende, es will Abend werden: übernachtet doch hier! Sieh
doch, wie der Tag schon zur Neige geht; bleibe über Nacht hier und laß
dir's bei mir gefallen! Morgen früh macht ihr euch dann auf euren Weg,
und du kehrst nach Hause zurück.« \bibleverse{10}Aber der Mann wollte
nicht noch einmal über Nacht bleiben, sondern brach auf und zog fort und
kam bis in die Gegend von Jebus, das ist Jerusalem; seine beiden
gesattelten\textless sup title=``oder: bepackten''\textgreater✲ Esel,
sein Nebenweib (und sein Diener) waren bei ihm.

\hypertarget{bb-einkehr-und-aufnahme-des-mannes-in-gibea}{%
\subparagraph{bb) Einkehr und Aufnahme des Mannes in
Gibea}\label{bb-einkehr-und-aufnahme-des-mannes-in-gibea}}

\bibleverse{11}Als sie nun bei Jebus waren und der Tag schon stark zu
Ende ging, sagte der Diener zu seinem Herrn: »Komm, laß uns hier in der
Jebusiterstadt einkehren und darin übernachten!« \bibleverse{12}Aber
sein Herr erwiderte ihm: »Nein, wir wollen in keiner Stadt von fremden
Leuten einkehren, die nicht zu den Israeliten gehören, sondern wollen
bis Gibea weiterziehen.« \bibleverse{13}Weiter sagte er zu seinem
Diener: »Komm, wir wollen eine von den Ortschaften dort zu erreichen
suchen und in Gibea oder in Rama über Nacht bleiben!« \bibleverse{14}Als
sie nun eiligst weiterzogen, ging die Sonne ihnen unter, als sie nahe
bei Gibea waren, das zu Benjamin gehört. \bibleverse{15}Da kehrten sie
dort ein, um zu einem Nachtquartier in Gibea zu gelangen, und er machte
nach seiner Ankunft auf dem Marktplatz der Stadt halt; aber da war
niemand, der sie zum Übernachten ins Haus aufgenommen hätte.
\bibleverse{16}Endlich kam ein alter Mann abends vom Felde von seiner
Arbeit heim; der stammte vom Gebirge Ephraim und lebte als Fremdling in
Gibea, während die Bewohner des Ortes Benjaminiten waren.
\bibleverse{17}Als dieser nun sich umsah und den Wandersmann auf dem
Marktplatz der Stadt erblickte, fragte er: »Wohin willst du, und woher
kommst du?« \bibleverse{18}Der antwortete ihm: »Wir sind auf der
Wanderung von Bethlehem in Juda nach dem äußersten Teil des Gebirges
Ephraim, wo ich zu Hause bin. Ich war nach Bethlehem in Juda gereist und
will jetzt nach meinem Wohnort zurückkehren; aber niemand nimmt mich in
sein Haus auf, \bibleverse{19}obgleich wir sowohl Stroh als auch Futter
für unsere Esel und auch Brot und Wein für mich und deine Magd und für
den Burschen, der hier bei uns, deinen Knechten, ist, bei uns haben und
nichts weiter bedürfen.« \bibleverse{20}Da sagte der alte Mann: »Friede
dir\textless sup title=``=~Sei mir willkommen''\textgreater✲! Nur, was
du bedarfst, laß meine Sorge sein! Doch auf dem Platz hier sollst du
nicht übernachten!« \bibleverse{21}Dann nahm er ihn in sein Haus mit und
mengte Futter für die Esel; und als sie sich die Füße gewaschen hatten,
aßen und tranken sie.

\hypertarget{cc-die-schandtat-an-dem-weibe-und-die-heimkehr-des-leviten}{%
\subparagraph{cc) Die Schandtat an dem Weibe und die Heimkehr des
Leviten}\label{cc-die-schandtat-an-dem-weibe-und-die-heimkehr-des-leviten}}

\bibleverse{22}Während sie so sich gütlich taten, umringten die Männer
der Stadt, nichtsnutzige Buben, das Haus, schlugen laut an die Tür und
riefen dem alten Manne, dem das Haus gehörte, die Worte zu: »Gib den
Mann heraus, der bei dir eingekehrt ist: wir wollen uns an ihn machen!«
\bibleverse{23}Da ging der Besitzer des Hauses zu ihnen hinaus und sagte
zu ihnen: »Nicht doch, meine Brüder! Begeht doch nichts so Böses!
Nachdem dieser Mann in mein Haus gekommen ist, dürft ihr eine solche
Schandtat nimmermehr verüben. \bibleverse{24}Da ist meine Tochter, die
Jungfrau, und das Nebenweib dieses Mannes: die will ich euch
herausbringen; denen mögt ihr Gewalt antun und mit ihnen machen, was
euch gefällt, aber an diesem Manne dürft ihr eine solche Schandtat nicht
verüben!« \bibleverse{25}Aber die Männer wollten nicht auf ihn hören. Da
nahm der Mann sein Nebenweib und führte sie zu ihnen hinaus auf die
Straße, und sie mißbrauchten sie und taten ihr Gewalt an die ganze Nacht
hindurch bis zum Morgen; erst bei Tagesanbruch ließen sie sie gehen.
\bibleverse{26}Als nun der Morgen tagte, kam das Weib heim und fiel am
Toreingang zum Hause des Mannes, woselbst ihr Herr war, nieder und blieb
da liegen, bis es hell wurde. \bibleverse{27}Als nun ihr Herr am Morgen
aufstand und die Haustür öffnete und hinaustrat, um seines Weges weiter
zu ziehen, fand er das Weib, sein Nebenweib, am Toreingang zum Hause
liegen mit den Händen auf der Schwelle. \bibleverse{28}Er rief ihr zu:
»Stehe auf, wir wollen weiterziehen!«, aber es erfolgte keine Antwort.
Da hob er sie auf den Esel, machte sich dann auf und zog nach seinem
Wohnort. \bibleverse{29}Als er dort in sein Haus gekommen war, nahm er
ein Messer, ergriff sein Nebenweib, zerschnitt sie Glied für Glied in
zwölf Stücke und schickte diese im ganzen Gebiet Israels umher.
\bibleverse{30}Da erklärte denn jeder, der das sah: »So etwas ist bisher
noch nicht vorgekommen und noch nicht erlebt worden seit der Zeit, wo
die Israeliten aus dem Lande Ägypten heraufgezogen sind, bis zum
heutigen Tage. Nehmt es zu Herzen, beratet euch und redet!«

\hypertarget{b-israels-strafgericht-an-benjamin}{%
\paragraph{b) Israels Strafgericht an
Benjamin}\label{b-israels-strafgericht-an-benjamin}}

\hypertarget{aa-die-beratung-der-israelitischen-stuxe4mme-in-mizpa-ihr-aufgebot-zum-kriege}{%
\subparagraph{aa) Die Beratung der israelitischen Stämme in Mizpa; ihr
Aufgebot zum
Kriege}\label{aa-die-beratung-der-israelitischen-stuxe4mme-in-mizpa-ihr-aufgebot-zum-kriege}}

\hypertarget{section-19}{%
\section{20}\label{section-19}}

\bibleverse{1}Da zogen alle Israeliten aus, und die Volksgemeinde
versammelte sich wie ein Mann von Dan bis Beerseba, auch das Land
Gilead, vor dem HERRN in Mizpa; \bibleverse{2}und die Häupter des ganzen
Volkes, der sämtlichen Stämme Israels, stellten sich in der Versammlung
des Volkes Gottes ein, 400000~Mann Fußvolk, mit Schwertern bewaffnet.
\bibleverse{3}Und die Benjaminiten hörten, daß die Israeliten nach Mizpa
hinaufgezogen seien. Nun fragten die Israeliten: »Sagt an, wie diese
Untat vor sich gegangen ist.« \bibleverse{4}Da nahm der Levit, der Mann
des ermordeten Weibes, das Wort und berichtete: »Ich war mit meinem
Nebenweibe nach Gibea im Stamme Benjamin gekommen, um dort zu
übernachten. \bibleverse{5}Da erhoben sich die Bürger von Gibea gegen
mich und umringten nachts das Haus in feindseliger Absicht gegen mich:
mich gedachten sie umzubringen, meinem Nebenweibe aber haben sie Gewalt
angetan, so daß sie gestorben ist. \bibleverse{6}Da habe ich mein
Nebenweib genommen, habe sie zerstückt und die Stücke in alle Teile des
israelitischen Erbbesitzes gesandt; denn man hat ein Verbrechen und eine
ruchlose Tat in Israel verübt. \bibleverse{7}Ihr seid jetzt hier alle
versammelt, ihr Israeliten: so beratet denn allhier die Sache und faßt
einen Beschluß!« \bibleverse{8}Da erhob sich das ganze Volk wie ein Mann
und rief: »Keiner von uns darf in seinen Wohnort zurückkehren und keiner
sich nach seinem Hause begeben! \bibleverse{9}Nein, so wollen wir jetzt
mit Gibea verfahren: Wir wollen gegen die Stadt nach dem Lose vorgehen!
\bibleverse{10}und zwar wollen wir von allen Stämmen Israels je zehn
Männer von hundert und je hundert von tausend und je tausend von
zehntausend nehmen, die sollen Lebensmittel für das Kriegsvolk holen,
damit wir dann nach ihrer Rückkehr mit Gibea im Stamme Benjamin ganz so
verfahren, wie die Schandtat es verdient, die es in\textless sup
title=``oder: an''\textgreater✲ Israel verübt hat.« \bibleverse{11}So
versammelte sich denn die gesamte Mannschaft der Israeliten gegen die
Stadt, wie ein Mann verbündet.

\hypertarget{bb-die-benjaminiten-anstatt-die-missetuxe4ter-auszuliefern-ruxfcsten-auch-ihrerseits-zum-kampf}{%
\subparagraph{bb) Die Benjaminiten, anstatt die Missetäter auszuliefern,
rüsten auch ihrerseits zum
Kampf}\label{bb-die-benjaminiten-anstatt-die-missetuxe4ter-auszuliefern-ruxfcsten-auch-ihrerseits-zum-kampf}}

\bibleverse{12}Hierauf sandten die israelitischen Stämme Männer durch
den ganzen Stamm Benjamin\textless sup title=``=~an alle
benjaminitischen Geschlechter''\textgreater✲ mit der Botschaft: »Was ist
das für eine Untat, die bei euch verübt worden ist! \bibleverse{13}So
gebt nun jetzt die Männer, die ruchlosen Buben, die sich in Gibea
befinden, heraus, damit wir sie töten und das Böse aus Israel
wegschaffen!« Aber die Benjaminiten wollten der Forderung ihrer Brüder✲,
der Israeliten, nicht nachkommen, \bibleverse{14}sie versammelten sich
vielmehr aus ihren Ortschaften in Gibea, um zum Kampf mit den Israeliten
auszurücken. \bibleverse{15}Als man damals die Benjaminiten aus den
Ortschaften musterte, belief sich ihre Zahl auf 26000 schwertbewaffnete
Männer, abgesehen von den Bewohnern Gibeas selbst, die bei der Musterung
700 auserlesene Krieger stellten. \bibleverse{16}Unter all diesen Leuten
waren 700 auserlesene Männer linkshändig: ein jeder von ihnen
schleuderte mit Steinen haarscharf, ohne zu fehlen. \bibleverse{17}Als
man dann auch die Mannschaft der Israeliten außer dem Stamme Benjamin
musterte, ergab sich bei ihnen die Zahl von 400000 schwertbewaffneten
Männern, lauter kampffähige Krieger.

\hypertarget{cc-blutige-niederlage-der-israeliten-an-den-ersten-beiden-schlachttagen-ihre-anfrage-in-bethel}{%
\subparagraph{cc) Blutige Niederlage der Israeliten an den ersten beiden
Schlachttagen; ihre Anfrage in
Bethel}\label{cc-blutige-niederlage-der-israeliten-an-den-ersten-beiden-schlachttagen-ihre-anfrage-in-bethel}}

\bibleverse{18}Hierauf machten die Israeliten sich auf den Weg und zogen
nach Bethel hinauf, um Gott zu befragen, wer von ihnen zuerst zum Kampf
mit den Benjaminiten ausziehen solle. Der HERR antwortete ihnen: »Juda
soll den Anfang machen!« \bibleverse{19}So brachen denn die Israeliten
am folgenden Morgen früh auf und lagerten sich vor\textless sup
title=``oder: gegen''\textgreater✲ Gibea; \bibleverse{20}dann zogen die
Männer von Israel zum Kampf mit den Benjaminiten aus und stellten sich
gegen sie in Schlachtordnung vor Gibea auf. \bibleverse{21}Die
Benjaminiten aber brachen aus Gibea hervor und streckten an diesem Tage
22000~Mann von Israel zu Boden. \bibleverse{22}Doch das Kriegsvolk, die
israelitische Mannschaft, ließ sich dadurch nicht entmutigen, sondern
stellte sich noch einmal in Schlachtordnung auf an derselben Stelle, wo
sie sich am ersten Tage aufgestellt hatten. \bibleverse{23}Zuvor aber
zogen die Israeliten (nach Bethel) hinauf und weinten vor dem HERRN bis
zum Abend und befragten dann den HERRN, ob sie noch einmal zum Kampfe
mit ihren Brüdern✲, den Benjaminiten, ausrücken sollten. Der HERR gab
ihnen die Antwort: »Zieht gegen sie aus!«

\bibleverse{24}Als nun die Israeliten an diesem zweiten Tage wieder
gegen die Benjaminiten anrückten, \bibleverse{25}zogen diese ihnen auch
an diesem zweiten Tage aus Gibea entgegen und streckten von den
Israeliten abermals 18000~Mann zu Boden, lauter Männer, die das Schwert
führten. \bibleverse{26}Da zogen die Israeliten allesamt, das ganze
Kriegsvolk, nach Bethel hinauf, weinten dort ununterbrochen vor dem
HERRN, fasteten während jenes Tages bis zum Abend und brachten Brand-
und Heilsopfer vor dem HERRN dar. \bibleverse{27}Als die Israeliten dann
den HERRN befragten -- dort befand sich nämlich zu jener Zeit die
Bundeslade Gottes, \bibleverse{28}und Pinehas, der Sohn Eleasars, des
Sohnes Aarons, versah damals den Dienst vor ihm -- und die Anfrage
stellten: »Soll ich noch einmal zum Kampf gegen meine Brüder, die
Benjaminiten, ausziehen oder davon abstehen?«, gab der HERR die Antwort:
»Zieht hin, denn morgen will ich sie in eure Gewalt geben!«

\hypertarget{dd-die-vernichtung-gibeas-und-die-fast-vuxf6llige-ausrottung-des-stammes-benjamin}{%
\subparagraph{dd) Die Vernichtung Gibeas und die fast völlige Ausrottung
des Stammes
Benjamin}\label{dd-die-vernichtung-gibeas-und-die-fast-vuxf6llige-ausrottung-des-stammes-benjamin}}

\bibleverse{29}Darauf legte Israel ringsum gegen Gibea Leute in den
Hinterhalt. \bibleverse{30}Als dann die Israeliten am dritten Tage
wieder gegen die Benjaminiten heranzogen und sich wie die beiden vorigen
Male in Schlachtordnung vor\textless sup title=``oder:
gegen''\textgreater✲ Gibea aufstellten, \bibleverse{31}rückten die
Benjaminiten heraus dem (feindlichen) Heere entgegen, ließen sich von
der Stadt weglocken und begannen, wie die beiden vorigen Male, einige
von dem Heere zu erschlagen auf den Landstraßen -- von denen die eine
nach Bethel hinaufgeht, die andere durch die Felder nach Gibeon führt
--, etwa dreißig Mann von den Israeliten. \bibleverse{32}Da dachten die
Benjaminiten: »Sie sind von uns geschlagen wie früher«; die Israeliten
dagegen hatten verabredet: »Wir wollen fliehen, um sie noch weiter von
der Stadt wegzulocken nach den Landstraßen hin!« \bibleverse{33}Daher
verließen sie, die gesamte Mannschaft der Israeliten, ihren Standort und
stellten sich erst wieder bei Baal-Thamar auf, während die im Hinterhalt
liegenden Israeliten aus ihrem Standort westlich von Gibea
hervorbrachen. \bibleverse{34}So rückten denn 10000~Mann, eine aus ganz
Israel erlesene Mannschaft, gegen Gibea heran, und ein wütender Kampf
entstand; jene aber hatten keine Ahnung davon, daß das Verderben daran
war, über sie hereinzubrechen. \bibleverse{35}Da ließ der HERR den Stamm
Benjamin von Israel besiegt werden, so daß die Israeliten von den
Benjaminiten 25100~Mann an diesem Tage niedermachten, lauter Männer, die
das Schwert führten. \bibleverse{36}Da sahen die Benjaminiten, daß sie
geschlagen waren; die Israeliten hatten sich nämlich vor den
Benjaminiten zurückgezogen, weil sie sich auf den Hinterhalt verließen,
den sie gegen Gibea gelegt hatten. \bibleverse{37}Die im Hinterhalt
Liegenden warfen sich dann auch eiligst auf Gibea, drangen in die Stadt
ein und machten die ganze Bevölkerung mit dem Schwerte nieder.
\bibleverse{38}Die Israeliten hatten aber mit den im Hinterhalt
Liegenden verabredet, daß sie eine große Rauchwolke aus der Stadt
aufsteigen lassen sollten. \bibleverse{39}Als sich nun das israelitische
Heer in der Schlacht zur Flucht gewandt und die Benjaminiten schon
angefangen hatten, unter den Israeliten einige, etwa dreißig Mann, zu
erschlagen, weil sie dachten, jene seien völlig von ihnen geschlagen wie
beim früheren Kampf~-- \bibleverse{40}da begann das verabredete Zeichen
aus der Stadt aufzusteigen, die Rauchsäule; und als die Benjaminiten
sich umwandten, sahen sie die Flammen der ganzen Stadt zum Himmel
aufsteigen. \bibleverse{41}Nun machte das israelitische Heer kehrt, die
Benjaminiten aber gerieten in Bestürzung, denn sie erkannten, daß das
Unheil sie erreicht hatte. \bibleverse{42}Sie zogen sich dann vor den
Israeliten in der Richtung nach der Steppe hin zurück; aber das
feindliche Heer folgte ihnen auf dem Fuße nach, und die aus der Stadt
Kommenden vernichteten sie in ihrer Mitte. \bibleverse{43}Sie
umzingelten die Benjaminiten, verfolgten sie, holten sie ein, wo sie
ausruhen wollten, bis in die Gegend östlich von Gibea✲;
\bibleverse{44}dabei fielen von den Benjaminiten 18000~Mann, lauter
tapfere Krieger. \bibleverse{45}Die anderen wandten sich zur Flucht
gegen die Wüste, nach dem Felsen Rimmon hin, jene aber erschlugen von
ihnen auf den Landstraßen nachträglich noch 5000~Mann und verfolgten sie
weiter bis Gideom\textless sup title=``oder: bis zu ihrer
Vernichtung''\textgreater✲ und machten von ihnen noch 2000~Mann nieder.
\bibleverse{46}So betrug die Gesamtzahl der an jenem Tage gefallenen
Benjaminiten 25000~Mann, lauter schwertbewaffnete, tapfere Krieger.

\bibleverse{47}600~Mann aber, die sich zur Flucht gewandt hatten, waren
in die Wüste nach dem Felsen Rimmon entkommen und blieben dort am Felsen
Rimmon vier Monate lang. \bibleverse{48}Die israelitische Mannschaft
aber kehrte ins Land der noch übrigen Benjaminiten zurück und machte mit
dem Schwert alles nieder, was sich vorfand, die Menschen in den
Ortschaften bis zum Vieh; auch alle Ortschaften, die sich da vorfanden,
steckten sie in Brand.

\hypertarget{c-versorgung-der-uxfcbriggebliebenen-benjaminiten-mit-frauen}{%
\paragraph{c) Versorgung der übriggebliebenen Benjaminiten mit
Frauen}\label{c-versorgung-der-uxfcbriggebliebenen-benjaminiten-mit-frauen}}

\hypertarget{aa-trauer-der-gemeinde-den-benjaminiten-werden-jungfrauen-aus-der-stadt-jabes-zugewiesen}{%
\subparagraph{aa) Trauer der Gemeinde; den Benjaminiten werden
Jungfrauen aus der Stadt Jabes
zugewiesen}\label{aa-trauer-der-gemeinde-den-benjaminiten-werden-jungfrauen-aus-der-stadt-jabes-zugewiesen}}

\hypertarget{section-20}{%
\section{21}\label{section-20}}

\bibleverse{1}Die Männer von Israel hatten aber in Mizpa folgenden
feierlichen Schwur abgelegt: »Keiner von uns darf seine Tochter einem
Benjaminiten zur Frau geben!« \bibleverse{2}Als nun das Volk nach Bethel
gezogen war und dort bis zum Abend vor Gott (beim Heiligtum) verweilte,
fingen sie an, laut zu weinen, \bibleverse{3}und wehklagten: »Warum, o
HERR, Gott Israels, ist dies in Israel geschehen, daß heute ein ganzer
Stamm aus Israel fehlt!« \bibleverse{4}Am folgenden Tage aber in der
Frühe baute das Volk dort einen Altar und brachte Brand- und Heilsopfer
dar. \bibleverse{5}Dann fragten die Israeliten: »Wer von allen Stämmen
Israels ist nicht mit der Gemeinde zum HERRN hergekommen?« Man hatte
nämlich in betreff eines jeden, der nicht zum HERRN nach Mizpa hinkommen
würde, einen feierlichen Eid ausgesprochen des Inhalts: »Er muß
unbedingt sterben.« \bibleverse{6}Nun tat es aber den Israeliten um
ihren Bruderstamm Benjamin leid, so daß sie klagten: »Heute ist ein
ganzer Stamm von Israel abgehauen! \bibleverse{7}Wie können wir nun den
Übriggebliebenen zu Frauen verhelfen, da wir doch beim HERRN geschworen
haben, daß wir ihnen keine von unsern Töchtern zu Frauen geben wollen?«
\bibleverse{8}Da fragten sie: »Ist etwa einer von den Stämmen Israels
nicht zum HERRN nach Mizpa hinaufgekommen?« Da ergab es sich, daß aus
Jabes in Gilead niemand ins Lager zur Versammlung gekommen war.
\bibleverse{9}Man hatte nämlich das Volk gemustert, und da stellte es
sich heraus, daß von den Einwohnern von Jabes in Gilead keiner anwesend
war. \bibleverse{10}Nun sandte die Volksgemeinde 12000~Mann von den
tapfersten Männern dorthin mit dem Befehl: »Geht hin und erschlagt die
Einwohner von Jabes in Gilead mit dem Schwerte, auch die Weiber und die
Kinder! \bibleverse{11}Verfahrt dabei aber so: an allen Männern sowie an
allen weiblichen Personen, die schon mit Männern zu tun gehabt haben,
sollt ihr den Blutbann vollstrecken (die Jungfrauen aber am Leben lassen
und sie hierher ins Lager bringen)!« \bibleverse{12}Sie\textless sup
title=``d.h. die Abgesandten''\textgreater✲ fanden aber unter den
Bewohnern von Jabes in Gilead vierhundert jungfräuliche Mädchen, die
noch mit keinem Mann zu tun gehabt hatten; die brachten sie ins Lager
nach Silo, das im Lande Kanaan liegt. \bibleverse{13}Hierauf sandte die
ganze Volksgemeinde hin und verhandelte mit den Benjaminiten, die sich
noch am Felsen Rimmon befanden, und ließ ihnen Sicherheit\textless sup
title=``oder: freies Geleit''\textgreater✲ entbieten. \bibleverse{14}So
kehrten denn die Benjaminiten damals zurück, und man gab ihnen die
Mädchen, die man von den weiblichen Personen aus Jabes in Gilead am
Leben gelassen hatte, zu Frauen; diese reichten jedoch für sie nicht
aus.

\hypertarget{bb-der-raub-der-jungfrauen-von-silo-durch-die-benjaminiten-abschluuxdf-der-erzuxe4hlung}{%
\subparagraph{bb) Der Raub der Jungfrauen von Silo durch die
Benjaminiten; Abschluß der
Erzählung}\label{bb-der-raub-der-jungfrauen-von-silo-durch-die-benjaminiten-abschluuxdf-der-erzuxe4hlung}}

\bibleverse{15}Weil nun die Benjaminiten dem Volke leid taten, da der
HERR einen Riß in den Stämmen Israels hatte entstehen lassen,
\bibleverse{16}sagten die Ältesten der Gemeinde: »Wie können wir den
Übriggebliebenen zu Frauen verhelfen, da ja die Frauen aus Benjamin
ausgerottet sind?« \bibleverse{17}Da sagten sie: »Der Erbbesitz soll den
Benjaminiten verbleiben, die mit dem Leben davongekommen sind, damit
nicht ein Stamm aus Israel ausgetilgt wird. \bibleverse{18}Wir aber
können ihnen keine von unsern Töchtern zu Frauen geben, denn die
Israeliten haben feierlich geschworen: ›Verflucht sei, wer seine Tochter
einem Benjaminiten zur Frau gibt!‹« \bibleverse{19}Da sagten sie: »Es
findet bekanntlich alle Jahre ein Fest des HERRN in Silo statt (auf dem
Platze) nördlich von Bethel, östlich von der Landstraße, die von Bethel
nach Sichem hinaufführt, und südlich von Lebona.« \bibleverse{20}Da
gaben sie den Benjaminiten folgende Weisung: »Geht hin und versteckt
euch in den Weinbergen! \bibleverse{21}Wenn ihr dann die Mädchen von
Silo herauskommen seht, um Reigentänze aufzuführen, so brecht aus den
Weinbergen hervor und erhascht euch ein jeder sein Weib aus den Mädchen
von Silo und kehrt mit ihr ins Land Benjamin zurück. \bibleverse{22}Wenn
dann ihre Väter oder Brüder kommen, um sich bei uns (über euch) zu
beschweren, so wollen wir zu ihnen sagen: ›Vergönnt sie ihnen um
unsertwillen! {[}Denn wir haben durch den Krieg (gegen Jabes) nicht für
jeden eine Frau gewonnen.{]} Nicht ihr habt sie ihnen ja gegeben; denn
in diesem Fall würdet ihr als schuldig dastehen.‹«

\bibleverse{23}Die Benjaminiten befolgten die Weisung: sie holten sich
die erforderliche Zahl von Frauen aus den tanzenden Mädchen, die sie
raubten, kehrten dann in ihren Erbbesitz zurück, bauten die Ortschaften
wieder auf und ließen sich darin nieder. \bibleverse{24}Nunmehr zogen
auch die Israeliten von dort heim, ein jeder in seinen Stamm und zu
seinem Geschlecht, und begaben sich von dort weg, ein jeder in seinen
Erbbesitz.

\bibleverse{25}Zu jener Zeit gab es noch keinen König in Israel; jeder
tat, was ihm gut dünkte.
