\hypertarget{der-brief-des-apostels-paulus-an-die-ruxf6mer}{%
\section{DER BRIEF DES APOSTELS PAULUS AN DIE
RÖMER}\label{der-brief-des-apostels-paulus-an-die-ruxf6mer}}

\hypertarget{eingang-des-briefes-und-angabe-des-themas}{%
\subsubsection{Eingang des Briefes und Angabe des
Themas}\label{eingang-des-briefes-und-angabe-des-themas}}

\hypertarget{a-bezeichnung-des-absenders-und-der-empfuxe4nger-des-briefes-und-apostolischer-segensgruuxdf-an-die-gemeinde}{%
\paragraph{a) Bezeichnung des Absenders und der Empfänger des Briefes
und apostolischer Segensgruß an die
Gemeinde}\label{a-bezeichnung-des-absenders-und-der-empfuxe4nger-des-briefes-und-apostolischer-segensgruuxdf-an-die-gemeinde}}

\hypertarget{section}{%
\section{1}\label{section}}

\bibleverse{1} Ich, Paulus, ein Knecht✲ Christi Jesu, bin durch Berufung
zum Apostel ausgesondert\textless sup title=``=~eigens dazu
bestellt''\textgreater✲, die Heilsbotschaft Gottes zu verkündigen,
\bibleverse{2} die er\textless sup title=``d.h. Gott''\textgreater✲
durch seine Propheten in (den) heiligen Schriften voraus verheißen hat,
\bibleverse{3} nämlich (die Heilsbotschaft) von seinem Sohne. Dieser ist
nach dem Fleische aus Davids Samen✲ hervorgegangen, \bibleverse{4} aber
als Sohn Gottes in Macht erwiesen nach dem Geist der Heiligkeit aufgrund
seiner Auferstehung aus den Toten. Durch ihn, unsern Herrn Jesus
Christus, \bibleverse{5} haben wir\textless sup title=``=~habe
ich''\textgreater✲ Gnade und das Apostelamt empfangen, um
Glaubensgehorsam zu seines Namens Ehre unter allen Heidenvölkern zu
wirken; \bibleverse{6} zu diesen gehört auch ihr, da ihr für Jesus
Christus (von Gott) berufen worden seid. \bibleverse{7} Euch allen, die
ihr als Geliebte Gottes, als berufene\textless sup title=``oder: durch
Berufung''\textgreater✲ Heilige in Rom wohnt, sende ich meinen Gruß:
Gnade werde euch zuteil und Friede von Gott unserm Vater und dem Herrn
Jesus Christus!

\hypertarget{b-danksagung-des-apostels-an-gott-fuxfcr-den-glaubensstand-der-gemeinde-und-ausdruck-des-wunsches-auch-in-rom-die-heilsbotschaft-verkuxfcnden-zu-duxfcrfen}{%
\paragraph{b) Danksagung des Apostels an Gott für den Glaubensstand der
Gemeinde und Ausdruck des Wunsches, auch in Rom die Heilsbotschaft
verkünden zu
dürfen}\label{b-danksagung-des-apostels-an-gott-fuxfcr-den-glaubensstand-der-gemeinde-und-ausdruck-des-wunsches-auch-in-rom-die-heilsbotschaft-verkuxfcnden-zu-duxfcrfen}}

\bibleverse{8} Zuerst sage ich meinem Gott durch Jesus Christus um euer
aller willen Dank dafür, daß man von eurem Glauben in der ganzen
Welt\textless sup title=``d.h. in allen christlichen
Gemeinden''\textgreater✲ mit Anerkennung redet. \bibleverse{9} Denn
Gott, dem ich in\textless sup title=``oder: mit''\textgreater✲ meinem
Geiste bei der Verkündigung der Heilsbotschaft seines
Sohnes\textless sup title=``oder: von seinem Sohne''\textgreater✲ diene,
ist mein Zeuge, daß ich euer ohne Unterlaß gedenke \bibleverse{10} und
jedesmal in meinen Gebeten die Bitte ausspreche, ob es mir wohl endlich
einmal durch Gottes Willen vergönnt sein möchte, zu euch zu kommen.
\bibleverse{11} Denn ich sehne mich danach, euch zu sehen; ich möchte
euch gern diese und jene geistliche Gabe zu eurer Stärkung mitteilen
\bibleverse{12} oder, besser gesagt, in eurer Mitte durch die
Wechselwirkung unsers beiderseitigen Glaubens gleichfalls eine
Kräftigung\textless sup title=``oder: Förderung''\textgreater✲ erfahren.
\bibleverse{13} Sodann will ich euch nicht in Unkenntnis darüber lassen,
liebe Brüder, daß ich mir schon oftmals vorgenommen habe, euch zu
besuchen -- ich bin bis jetzt nur immer wieder an der Ausführung (meiner
Absicht) gehindert worden --, um auch bei euch ebenso wie bei den
übrigen Heidenvölkern einige Frucht zu erlangen\textless sup
title=``oder: als Ertrag zu wirken''\textgreater✲. \bibleverse{14} Den
Griechen wie den Barbaren\textless sup title=``d.h.
Nichtgriechen''\textgreater✲, den Gebildeten wie den Ungelehrten bin ich
(zu dienen) verpflichtet; \bibleverse{15} der gute Wille ist also
meinerseits vorhanden, auch euch in Rom die Heilsbotschaft zu verkünden.

\hypertarget{c-angabe-des-themas-d.h.-des-zu-behandelnden-gegenstandes-die-rechtfertigungsup-titleoder-die-gottesgerechtigkeit}{%
\paragraph{c) Angabe des Themas (d.h. des zu behandelnden Gegenstandes):
Die Rechtfertigung\textless sup title=``oder: die
Gottesgerechtigkeit''\textgreater✲}\label{c-angabe-des-themas-d.h.-des-zu-behandelnden-gegenstandes-die-rechtfertigungsup-titleoder-die-gottesgerechtigkeit}}

\bibleverse{16} Denn ich schäme mich der Heilsbotschaft nicht; ist sie
doch eine Gotteskraft, die jedem, der da glaubt, die Rettung bringt, wie
zuerst\textless sup title=``oder: zunächst =~in erster
Linie''\textgreater✲ dem Juden, so auch dem Griechen\textless sup
title=``=~dem griechisch redenden Heiden''\textgreater✲. \bibleverse{17}
Denn Gottesgerechtigkeit wird in ihr\textless sup title=``oder: durch
sie''\textgreater✲ geoffenbart, aus Glauben zu Glauben, wie geschrieben
steht\textless sup title=``Hab 2,4''\textgreater✲: »Der Gerechte wird
aus Glauben\textless sup title=``oder: infolge von
Glauben''\textgreater✲ leben.«

\hypertarget{i.-die-gottesgerechtigkeit-aus-dem-glauben-an-jesus-christus-118-839}{%
\subsection{I. Die Gottesgerechtigkeit aus dem Glauben an Jesus Christus
(1,18-8,39)}\label{i.-die-gottesgerechtigkeit-aus-dem-glauben-an-jesus-christus-118-839}}

\hypertarget{a.-die-heilsbeduxfcrftigkeit-der-gesamten-menschheit-118-320}{%
\subsection{A. Die Heilsbedürftigkeit der gesamten Menschheit
(1,18-3,20)}\label{a.-die-heilsbeduxfcrftigkeit-der-gesamten-menschheit-118-320}}

\hypertarget{gottes-zorn-uxfcber-die-von-ihm-abgefallene-heidenwelt}{%
\subsubsection{1. Gottes Zorn über die von ihm abgefallene
Heidenwelt}\label{gottes-zorn-uxfcber-die-von-ihm-abgefallene-heidenwelt}}

\hypertarget{a-die-suxfcndenschuld-besonders-der-guxf6tzendienst-des-gesamten-heidentums}{%
\paragraph{a) Die Sündenschuld (besonders der Götzendienst) des gesamten
Heidentums}\label{a-die-suxfcndenschuld-besonders-der-guxf6tzendienst-des-gesamten-heidentums}}

\bibleverse{18} Denn Gottes Zorn offenbart sich vom Himmel her über alle
Gottlosigkeit und Ungerechtigkeit der Menschen, welche die
Wahrheit\textless sup title=``=~die wahre Erkenntnis
Gottes''\textgreater✲ in\textless sup title=``oder: mit''\textgreater✲
Ungerechtigkeit unterdrücken. \bibleverse{19} Denn was man von Gott
erkennen kann, das ist in\textless sup title=``oder:
unter''\textgreater✲ ihnen wohlbekannt; Gott selbst hat es ihnen ja
kundgetan. \bibleverse{20} Sein unsichtbares Wesen läßt sich ja doch
seit Erschaffung der Welt an seinen Werken mit dem geistigen Auge
deutlich ersehen, nämlich seine ewige Macht und göttliche Größe. Daher
gibt es keine Entschuldigung für sie, \bibleverse{21} weil sie Gott zwar
kannten, ihm aber doch nicht als Gott Verehrung und Dank dargebracht
haben, sondern in ihren Gedanken auf nichtige Dinge verfallen sind und
ihr unverständiges Herz in Verfinsterung haben geraten lassen.
\bibleverse{22} Während sie sich ihrer angeblichen Weisheit rühmten,
sind sie zu Toren geworden \bibleverse{23} und haben die Herrlichkeit
des unvergänglichen Gottes mit dem Abbild des vergänglichen Menschen und
der Gestalt von Vögeln, von vierfüßigen Tieren und kriechendem Gewürm
vertauscht.

\hypertarget{b-das-guxf6ttliche-strafgericht-uxfcber-die-heidenwelt-wegen-ihres-suxfcndenverderbens}{%
\paragraph{b) Das göttliche Strafgericht über die Heidenwelt wegen ihres
Sündenverderbens}\label{b-das-guxf6ttliche-strafgericht-uxfcber-die-heidenwelt-wegen-ihres-suxfcndenverderbens}}

\bibleverse{24} Daher hat Gott sie durch die Begierden ihrer Herzen in
den Schmutz der Unsittlichkeit versinken lassen, so daß ihre Leiber an
ihnen selbst geschändet wurden; \bibleverse{25} denn sie haben die
Wahrheit\textless sup title=``=~das wahre Wesen''\textgreater✲ Gottes
mit der Lüge vertauscht und Anbetung und Verehrung dem Geschaffenen
erwiesen anstatt dem Schöpfer, der da gepriesen ist in Ewigkeit. Amen.
\bibleverse{26} Deshalb hat Gott sie auch in schandbare Leidenschaften
fallen lassen; denn ihre Frauen haben den natürlichen Geschlechtsverkehr
mit dem widernatürlichen vertauscht; \bibleverse{27} und ebenso haben
auch die Männer den natürlichen Verkehr mit der Frau aufgegeben und sind
in ihrer wilden Gier zueinander entbrannt, so daß sie, Männer mit
Männern, die Schamlosigkeit verübten, aber auch die gebührende Strafe
für ihre Verirrung an sich selbst\textless sup title=``=~am eigenen
Leibe''\textgreater✲ empfingen. \bibleverse{28} Und weil sie es
verschmähten, Gott in rechter Erkenntnis festzuhalten\textless sup
title=``oder: zu besitzen''\textgreater✲, hat Gott sie in eine
verworfene Sinnesweise versinken lassen, so daß sie alle Ungebühr
verüben: \bibleverse{29} sie sind erfüllt mit jeglicher Ungerechtigkeit,
Schlechtigkeit, Habgier und Bosheit, voll von Neid, Mordlust,
Streitsucht, Arglist und Niedertracht; \bibleverse{30} sie sind
Ohrenbläser, Verleumder, Gottesfeinde, gewalttätige und hoffärtige
Leute, Prahler, erfinderisch im Bösen, ungehorsam gegen die Eltern,
\bibleverse{31} unverständig, treulos, ohne Liebe und Erbarmen;
\bibleverse{32} sie kennen zwar die göttliche Rechtsordnung genau, daß,
wer derartiges verübt, den Tod verdient, tun es aber trotzdem nicht nur
selbst, sondern spenden auch noch denen Beifall, die solche Dinge
verüben.

\hypertarget{gottes-zorn-auch-uxfcber-die-juden-wegen-ihrer-gesetzesuxfcbertretungen}{%
\subsubsection{2. Gottes Zorn auch über die Juden wegen ihrer
Gesetzesübertretungen}\label{gottes-zorn-auch-uxfcber-die-juden-wegen-ihrer-gesetzesuxfcbertretungen}}

\hypertarget{a-auch-den-juden-steht-das-zorngericht-bevor-ihr-richten-anderer-befreit-sie-nicht-vom-gericht-gottes}{%
\paragraph{a) Auch den Juden steht das Zorngericht bevor; ihr Richten
anderer befreit sie nicht vom Gericht
Gottes}\label{a-auch-den-juden-steht-das-zorngericht-bevor-ihr-richten-anderer-befreit-sie-nicht-vom-gericht-gottes}}

\hypertarget{section-1}{%
\section{2}\label{section-1}}

\bibleverse{1} Daher gibt es (auch) für dich, o Mensch, wer du auch sein
magst, der du dich zum Richter (über andere) machst, keine
Entschuldigung; denn worin du den anderen richtest, darin verurteilst du
dich selbst; du, sein Richter, begehst ja dieselben Sünden!
\bibleverse{2} Wir wissen aber, daß Gottes Gericht\textless sup
title=``oder: Urteil''\textgreater✲ der Wahrheit gemäß über die ergeht,
welche derartiges verüben. \bibleverse{3} Rechnest du etwa darauf, o
Mensch, der du dich zum Richter über solche Übeltäter machst und doch
selber das Gleiche verübst, daß du dem Urteil Gottes (beim jüngsten
Gericht) entrinnen werdest? \bibleverse{4} Oder verachtest✲ du den
Reichtum seiner Güte, Geduld und Langmut, und erkennst du nicht, daß
Gottes Güte dich zur Buße✲ führen will? \bibleverse{5} Mit deinem
Starrsinn und unbußfertigen Herzen aber häufst du dir selbst Zorn auf
für den Tag des Zorns und der Offenbarung des gerechten Gerichts Gottes,
\bibleverse{6} der einem jeden nach seinen Werken vergelten
wird\textless sup title=``Ps 62,13''\textgreater✲, \bibleverse{7}
nämlich ewiges Leben (wird er geben) denen, welche im guten
Werk\textless sup title=``oder: im Tun des Guten''\textgreater✲
standhaft ausharrend, nach Herrlichkeit, Ehre und Unvergänglichkeit
trachten; \bibleverse{8} dagegen (seinen) Zorn und Grimm denen, welche
starrsinnig\textless sup title=``oder: eigenwillig''\textgreater✲ sind
und der Wahrheit nicht gehorchen, sondern der Ungerechtigkeit dienen.
\bibleverse{9} Trübsal und Angst wird über die Seele jedes Menschen
kommen, der das Böse tut, wie zunächst über den Juden, so auch über den
Griechen\textless sup title=``vgl. 1,16''\textgreater✲; \bibleverse{10}
dagegen Herrlichkeit, Ehre und Friede (wird) einem jeden (zuteil
werden), der das Gute tut, wie zunächst dem Juden, so auch dem Griechen;
\bibleverse{11} denn bei Gott gibt es kein Ansehen der Person.

\hypertarget{b-wertlosigkeit-des-blouxdfen-gesetzesbesitzes-bzw.-des-besseren-sittlichen-wissens-und-der-beschneidung}{%
\paragraph{b) Wertlosigkeit des bloßen Gesetzesbesitzes (bzw. des
besseren sittlichen Wissens) und der
Beschneidung}\label{b-wertlosigkeit-des-blouxdfen-gesetzesbesitzes-bzw.-des-besseren-sittlichen-wissens-und-der-beschneidung}}

\hypertarget{aa-gottes-urteil-ist-fuxfcr-die-juden-dasselbe-wie-fuxfcr-die-heiden-ausschlieuxdflich-durch-das-tun-bzw.-nichttun-des-gesetzes-bestimmt}{%
\subparagraph{aa) Gottes Urteil ist für die Juden dasselbe wie für die
Heiden, ausschließlich durch das Tun (bzw. Nichttun) des Gesetzes
bestimmt}\label{aa-gottes-urteil-ist-fuxfcr-die-juden-dasselbe-wie-fuxfcr-die-heiden-ausschlieuxdflich-durch-das-tun-bzw.-nichttun-des-gesetzes-bestimmt}}

\bibleverse{12} Denn alle, die, ohne das (mosaische) Gesetz (zu
besitzen), gesündigt haben, werden auch ohne Zutun des Gesetzes
verlorengehen\textless sup title=``=~dem Verderben
anheimfallen''\textgreater✲, und alle, die innerhalb\textless sup
title=``=~im Besitz''\textgreater✲ des Gesetzes gesündigt haben, werden
durch das Gesetz gerichtet werden; \bibleverse{13} denn nicht die Hörer
des Gesetzes sind vor Gott gerecht, sondern (nur) die Täter des Gesetzes
werden gerechtfertigt✲ werden. \bibleverse{14} Sooft nämlich Heiden, die
das Gesetz nicht haben, von Natur\textless sup title=``=~von sich
aus''\textgreater✲ die Forderungen des Gesetzes erfüllen, so sind diese,
weil\textless sup title=``oder: wiewohl''\textgreater✲ sie das Gesetz
nicht haben, sich selbst (das) Gesetz; \bibleverse{15} sie liefern ja
dadurch den tatsächlichen Beweis, daß das vom Gesetz gebotene Tun ihnen
ins Herz geschrieben ist, wofür auch ihr Gewissen sein Zeugnis ablegt
und ebenso ihre Gedanken, die im Widerstreit miteinander Anklagen
erheben oder auch Entschuldigungen vorbringen~-- \bibleverse{16} (das
wird sich) an dem Tage (herausstellen), an welchem Gott das in den
(Herzen der) Menschen Verborgene richten wird, (und zwar) nach der
Heilsbotschaft, wie ich sie verkündige, durch Jesus Christus.

\hypertarget{bb-die-bessere-sittliche-erkenntnis-und-die-lehrbefuxe4higung-machen-den-juden-vor-gott-nicht-gerecht-sein-selbstruhm-wegen-des-gesetzes-ist-hinfuxe4llig-weil-er-es-uxfcbertritt}{%
\subparagraph{bb) Die bessere sittliche Erkenntnis und die
Lehrbefähigung machen den Juden vor Gott nicht gerecht; sein Selbstruhm
wegen des Gesetzes ist hinfällig, weil er es
übertritt}\label{bb-die-bessere-sittliche-erkenntnis-und-die-lehrbefuxe4higung-machen-den-juden-vor-gott-nicht-gerecht-sein-selbstruhm-wegen-des-gesetzes-ist-hinfuxe4llig-weil-er-es-uxfcbertritt}}

\bibleverse{17} Wenn andererseits du dich mit Stolz einen Juden nennst
und dich durch den Besitz des Gesetzes gesichert fühlst und dich deines
Verhältnisses zu Gott rühmst \bibleverse{18} und seinen Willen kennst
und infolge der aus dem Gesetz gewonnenen Unterweisung das, was in jedem
Fall das Richtige ist, wohl zu beurteilen verstehst \bibleverse{19} und
dir zutraust, ein Führer der Blinden zu sein, ein Licht für die in der
Finsternis Lebenden, \bibleverse{20} ein Erzieher der Unverständigen,
ein Lehrer der Unmündigen, weil du ja im Gesetz die Erkenntnis und
Wahrheit verkörpert\textless sup title=``=~deutlich
dargestellt''\textgreater✲ besitzest~-- \bibleverse{21} nun, andere
Leute belehrst du, und dich selbst belehrst du nicht? Du predigst, man
dürfe nicht stehlen, und stiehlst selbst? \bibleverse{22} Du sagst, man
dürfe nicht ehebrechen, und brichst selber die Ehe? Du verabscheust die
Götzenbilder und vergreifst dich doch räuberisch an ihren
Tempeln\textless sup title=``vgl. 5.Mose 7,25-26''\textgreater✲?
\bibleverse{23} Du rühmst dich des Gesetzes und verunehrst doch Gott
durch deine Übertretung des Gesetzes? \bibleverse{24} Denn »der Name
Gottes wird durch eure Schuld unter den Heiden gelästert«, wie
geschrieben steht\textless sup title=``Jes 52,5; Hes
36,20.23''\textgreater✲.

\hypertarget{cc-die-beschneidung-ist-fuxfcr-den-juden-wertlos-wenn-er-das-gesetz-uxfcbertritt-beschneidung-des-herzens-tut-not}{%
\subparagraph{cc) Die Beschneidung ist für den Juden wertlos, wenn er
das Gesetz übertritt; Beschneidung des »Herzens« tut
not}\label{cc-die-beschneidung-ist-fuxfcr-den-juden-wertlos-wenn-er-das-gesetz-uxfcbertritt-beschneidung-des-herzens-tut-not}}

\bibleverse{25} Denn was die Beschneidung betrifft, so ist sie zwar eine
heilsame Sache, wenn du nämlich das Gesetz tatsächlich hältst; bist du
aber ein Übertreter des Gesetzes, so ist dir die Beschneidung zum
Unbeschnittensein geworden. \bibleverse{26} Wenn umgekehrt der
Unbeschnittene die Forderungen des Gesetzes beobachtet, wird ihm dann
sein Unbeschnittensein nicht als Beschneidung angerechnet werden?
\bibleverse{27} Ja, der von Natur✲ Unbeschnittene, der das Gesetz
erfüllt, wird (am Gerichtstage) das Urteil über dich
sprechen\textless sup title=``=~deine Verurteilung
bewirken''\textgreater✲, der du trotz (deines) geschriebenen Gesetzes
und (trotz deiner) Beschneidung ein Übertreter des Gesetzes bist.
\bibleverse{28} Denn nicht der ist (in Wahrheit) ein Jude, der es
sichtbar✲ ist, und die (rechte) Beschneidung besteht nicht in dem, was
äußerlich am Fleisch vorgenommen wird; \bibleverse{29} nein, (nur) der
ist ein Jude, der es innerlich ist, und die Beschneidung muß am Herzen
vollzogen sein im Geist, nicht (äußerlich) nach dem Buchstaben -- das
Lob\textless sup title=``oder: die Anerkennung''\textgreater✲ eines
solchen kommt nicht von Menschen her, sondern von Gott.

\hypertarget{c-die-bevorzugte-stellung-der-juden-bleibt-dennoch-bestehen-durch-ihre-untreue-wird-die-treue-gottes-in-ein-um-so-helleres-licht-gestellt}{%
\paragraph{c) Die bevorzugte Stellung der Juden bleibt dennoch bestehen;
durch ihre Untreue wird die Treue Gottes in ein um so helleres Licht
gestellt}\label{c-die-bevorzugte-stellung-der-juden-bleibt-dennoch-bestehen-durch-ihre-untreue-wird-die-treue-gottes-in-ein-um-so-helleres-licht-gestellt}}

\hypertarget{section-2}{%
\section{3}\label{section-2}}

\bibleverse{1} Was bleibt hiernach überhaupt noch als
Vorzug\textless sup title=``oder: Vorrecht''\textgreater✲ der Juden (vor
den Heiden) oder als Nutzen der Beschneidung bestehen? \bibleverse{2}
Immerhin viel in jeder Hinsicht! Zuerst, daß ihnen die Verheißungen
Gottes anvertraut worden sind. \bibleverse{3} Denn nicht wahr, wenn
manche sich untreu erwiesen haben -- wird etwa deren Untreue die Treue
Gottes aufheben? \bibleverse{4} Nimmermehr! Es bleibt vielmehr dabei:
Gott ist wahrhaftig, ob auch jeder Mensch ein Lügner ist\textless sup
title=``Ps 116,11''\textgreater✲, wie es in der Schrift
heißt\textless sup title=``Ps 51,6''\textgreater✲: »Du sollst in deinen
Worten\textless sup title=``oder: Urteilssprüchen''\textgreater✲ als
gerecht erfunden werden und Sieger bleiben, wenn man mit dir rechtet.«
\bibleverse{5} Wenn aber so unsere Ungerechtigkeit die Gerechtigkeit
Gottes erweist\textless sup title=``oder: in um so helleres Licht
stellt''\textgreater✲, was sollen wir daraus folgern? Ist Gott dann
nicht ungerecht, wenn er seinen Zorn\textless sup title=``oder: sein
Zorngericht''\textgreater✲ verhängt? -- ich rede da nach gewöhnlicher
Menschenweise.~-- \bibleverse{6} Nimmermehr! Wie sollte Gott sonst wohl
Richter der ganzen Welt sein können? \bibleverse{7} Wenn aber Gottes
Wahrhaftigkeit infolge meines Lügens um so stärker zu seiner
Verherrlichung\textless sup title=``oder: Ehre''\textgreater✲
hervorgetreten ist, warum werde auch ich dann noch als Sünder gerichtet?
\bibleverse{8} Und warum halten wir uns dann nicht an den Grundsatz, den
manche Lästerzungen mir wirklich in den Mund legen: »Laßt uns das Böse
tun, damit das Gute dabei herauskomme?« Nun, die betreffenden Leute
trifft das verdammende Urteil mit Fug und Recht.

\hypertarget{d-ergebnis-das-suxfcndenverderben-erstreckt-sich-uxfcber-heiden-und-juden-und-wird-durch-zahlreiche-schriftstellen-bestuxe4tigt}{%
\paragraph{d) Ergebnis: Das Sündenverderben erstreckt sich über Heiden
und Juden und wird durch zahlreiche Schriftstellen
bestätigt}\label{d-ergebnis-das-suxfcndenverderben-erstreckt-sich-uxfcber-heiden-und-juden-und-wird-durch-zahlreiche-schriftstellen-bestuxe4tigt}}

\bibleverse{9} Wie steht es also? Haben wir (Juden) für uns etwas
voraus? Nicht unbedingt. Wir haben ja schon vorhin gegen Juden ebenso
wie gegen Griechen\textless sup title=``vgl. 2,9''\textgreater✲ die
Anklage erheben müssen, daß sie ausnahmslos unter (der Herrschaft) der
Sünde stehen, \bibleverse{10} wie es in der Schrift heißt: »Es gibt
keinen Gerechten, auch nicht einen; \bibleverse{11} es gibt keinen
Einsichtigen, keinen, der Gott mit Ernst sucht; \bibleverse{12} sie sind
alle abgewichen, allesamt entartet; keiner ist da, der das Gute tut,
auch nicht ein einziger.«\textless sup title=``Ps 14,1-3''\textgreater✲
\bibleverse{13} »Ein offenes Grab ist ihre Kehle, mit ihren Zungen reden
sie Trug.«\textless sup title=``Ps 5,10''\textgreater✲ »Otterngift ist
unter\textless sup title=``oder: hinter''\textgreater✲ ihren
Lippen.«\textless sup title=``Ps 140,4''\textgreater✲ \bibleverse{14}
»Ihr Mund ist voll Fluchens und Bitterkeit.«\textless sup title=``Ps
10,7''\textgreater✲ \bibleverse{15} »Schnell sind ihre Füße, Blut zu
vergießen; \bibleverse{16} Verwüstung und Unheil sind auf ihren Wegen,
\bibleverse{17} und den Weg des Friedens kennen sie nicht.«\textless sup
title=``Jes 59,7-8''\textgreater✲ \bibleverse{18} »Keine Furcht Gottes
steht ihnen vor Augen.«\textless sup title=``Ps 36,2''\textgreater✲
\bibleverse{19} Wir wissen aber, daß das Gesetz alles, was es
ausspricht, denen vorhält, die unter dem Gesetz\textless sup
title=``d.h. im Besitz des Gesetzes''\textgreater✲ sind; es soll eben
einem jeden der Mund gestopft\textless sup title=``=~zum Schweigen
gebracht''\textgreater✲ werden und die ganze Welt dem Gericht Gottes
verfallen sein; \bibleverse{20} denn aufgrund von Gesetzeswerken wird
kein Fleisch✲ vor Gott gerechtfertigt werden\textless sup title=``Ps
143,2''\textgreater✲; durch das Gesetz kommt ja (nur) Erkenntnis der
Sünde.

\hypertarget{b.-die-in-der-heilsbotschaft-verkuxfcndigte-neue-gerechtigkeit-ist-den-heiden-wie-den-juden-zuguxe4nglich-321-839}{%
\subsection{B. Die in der Heilsbotschaft verkündigte neue Gerechtigkeit
ist den Heiden wie den Juden zugänglich
(3,21-8,39)}\label{b.-die-in-der-heilsbotschaft-verkuxfcndigte-neue-gerechtigkeit-ist-den-heiden-wie-den-juden-zuguxe4nglich-321-839}}

\hypertarget{grund-und-recht-der-neuen-heilsordnung-beruht-auf-dem-glauben-an-christi-versuxf6hnungstod-und-auferstehung}{%
\subsubsection{1. Grund und Recht der neuen Heilsordnung beruht auf dem
Glauben (an Christi Versöhnungstod und
Auferstehung)}\label{grund-und-recht-der-neuen-heilsordnung-beruht-auf-dem-glauben-an-christi-versuxf6hnungstod-und-auferstehung}}

\hypertarget{a-die-gottesgerechtigkeit-wird-den-an-jesus-glaubenden-zuteil}{%
\paragraph{a) Die Gottesgerechtigkeit wird den an Jesus Glaubenden
zuteil}\label{a-die-gottesgerechtigkeit-wird-den-an-jesus-glaubenden-zuteil}}

\bibleverse{21} Jetzt aber ist, unabhängig vom Gesetz, jedoch bezeugt
von dem Gesetz und den Propheten, die Gottesgerechtigkeit geoffenbart
worden, \bibleverse{22} nämlich die Gottesgerechtigkeit, die durch den
Glauben an Jesus Christus für alle da ist und allen zukommt, die da
glauben. Denn hier gibt es keinen Unterschied; \bibleverse{23} alle
haben ja gesündigt und ermangeln des Ruhmes, den Gott verleiht;
\bibleverse{24} so werden sie umsonst\textless sup title=``oder:
geschenkweise =~ohne eigenes Verdienst''\textgreater✲ durch seine Gnade
gerechtfertigt vermöge\textless sup title=``oder:
aufgrund''\textgreater✲ der Erlösung, die in Christus Jesus (erfolgt)
ist. \bibleverse{25} Ihn hat Gott in seinem Blute\textless sup
title=``=~blutigen Tode''\textgreater✲ als ein durch den Glauben
wirksames Sühnemittel hingestellt, damit er\textless sup title=``d.h.
Gott''\textgreater✲ seine Gerechtigkeit erweise, weil die Sünden, die
früher während der Zeiten der Langmut Gottes begangen worden waren,
bisher ungestraft geblieben waren; \bibleverse{26} er wollte also seine
Gerechtigkeit in der gegenwärtigen Zeit erweisen, damit er selbst als
gerecht dastehe und (zugleich) jeden, der den Glauben an Jesus besitzt,
für gerecht erkläre.

\hypertarget{b-die-gottesgerechtigkeit-aus-glauben-schlieuxdft-allen-selbstruhm-aus-und-gilt-fuxfcr-heiden-wie-fuxfcr-juden}{%
\paragraph{b) Die Gottesgerechtigkeit aus Glauben schließt allen
Selbstruhm aus und gilt für Heiden wie für
Juden}\label{b-die-gottesgerechtigkeit-aus-glauben-schlieuxdft-allen-selbstruhm-aus-und-gilt-fuxfcr-heiden-wie-fuxfcr-juden}}

\bibleverse{27} Wo bleibt nun da das Rühmen\textless sup title=``=~der
Selbstruhm''\textgreater✲? Es ist ausgeschlossen! Durch was für ein
Gesetz? Durch das der Werke? Nein, sondern durch das Gesetz des
Glaubens\textless sup title=``d.h. durch den Weg des
Glaubens''\textgreater✲. \bibleverse{28} Denn wir halten dafür, daß der
Mensch durch den Glauben gerechtfertigt werde ohne Gesetzeswerke.
\bibleverse{29} Oder ist Gott etwa nur der Juden und nicht auch der
Heiden Gott? Jawohl, auch der Heiden, \bibleverse{30} so gewiß es nur
einen einzigen Gott gibt, der die Beschnittenen✲ aus
Glauben\textless sup title=``=~aufgrund des Glaubens''\textgreater✲ und
die Unbeschnittenen✲ durch den Glauben\textless sup title=``=~infolge
ihres Glaubens''\textgreater✲ rechtfertigen\textless sup title=``oder:
gerechtsprechen''\textgreater✲ wird.

\hypertarget{die-glaubensgerechtigkeit-ist-als-neue-heilsordnung-die-erfuxfcllung-der-schon-im-alten-testament-festgestellten-heilsordnung}{%
\subsubsection{2. Die Glaubensgerechtigkeit ist als neue Heilsordnung
die Erfüllung der schon im Alten Testament festgestellten
Heilsordnung}\label{die-glaubensgerechtigkeit-ist-als-neue-heilsordnung-die-erfuxfcllung-der-schon-im-alten-testament-festgestellten-heilsordnung}}

\hypertarget{a-das-gesetz-behuxe4lt-seine-geltung-aber-zu-einem-besonderen-zweck}{%
\paragraph{a) Das Gesetz behält seine Geltung, aber zu einem besonderen
Zweck}\label{a-das-gesetz-behuxe4lt-seine-geltung-aber-zu-einem-besonderen-zweck}}

\bibleverse{31} Heben wir demnach das Gesetz durch den Glauben auf?
Nimmermehr! Nein, wir geben dem Gesetz die rechte Stellung.

\hypertarget{b-nachweis-der-glaubensgerechtigkeit-bei-abraham-und-durch-ein-zeugnis-davids}{%
\paragraph{b) Nachweis der Glaubensgerechtigkeit bei Abraham und durch
ein Zeugnis
Davids}\label{b-nachweis-der-glaubensgerechtigkeit-bei-abraham-und-durch-ein-zeugnis-davids}}

\hypertarget{section-3}{%
\section{4}\label{section-3}}

\bibleverse{1} Was werden wir somit von unserm Stammvater Abraham sagen?
Was hat er nach dem Fleisch\textless sup title=``d.h. durch sein eigenes
menschliches Tun''\textgreater✲ erlangt? \bibleverse{2} Wenn Abraham
nämlich aufgrund von Werken gerechtfertigt worden ist, so hat er
allerdings Grund, sich zu rühmen, freilich (auch dann) nicht
vor\textless sup title=``oder: bei''\textgreater✲ Gott. \bibleverse{3}
Denn was sagt die Schrift? »Abraham glaubte Gott, und das wurde ihm zur
Gerechtigkeit gerechnet.«\textless sup title=``1.Mose
15,6''\textgreater✲ \bibleverse{4} Wenn nun jemand Werke verrichtet, so
erhält er den Lohn nicht aus Gnade angerechnet, sondern (zugeteilt) nach
Schuldigkeit; \bibleverse{5} wer dagegen keine Werke verrichtet, sondern
an den glaubt, der den Gottlosen rechtfertigt, dem wird sein Glaube zur
Gerechtigkeit gerechnet; \bibleverse{6} wie ja auch David die
Seligpreisung über den Menschen ausspricht, dem Gott Gerechtigkeit ohne
Rücksicht auf Werke anrechnet\textless sup title=``Ps
32,1-2''\textgreater✲: \bibleverse{7} »Glückselig sind die, denen die
Gesetzesübertretungen vergeben und deren Sünden zugedeckt worden sind;
\bibleverse{8} glückselig ist der Mann, dem der Herr (die) Sünde nicht
anrechnet.«

\hypertarget{c-abraham-als-vater-aller-gluxe4ubigen-auch-der-heiden}{%
\paragraph{c) Abraham als Vater aller Gläubigen, auch der
Heiden}\label{c-abraham-als-vater-aller-gluxe4ubigen-auch-der-heiden}}

\bibleverse{9} Gilt nun diese Seligpreisung nur den Beschnittenen✲ oder
auch den Unbeschnittenen\textless sup title=``=~Nichtjuden,
Heiden''\textgreater✲? Wir behaupten ja doch: »Dem Abraham wurde sein
Glaube zur Gerechtigkeit gerechnet.« \bibleverse{10} Unter welchen
Umständen hat denn diese Anrechnung stattgefunden? Als er schon
beschnitten oder als er noch unbeschnitten war? Nun: nicht als er schon
beschnitten, sondern als er noch unbeschnitten war; \bibleverse{11} und
das äußere Zeichen der Beschneidung empfing er dann als Siegel für die
Glaubensgerechtigkeit, die er im Zustande der
Unbeschnittenheit\textless sup title=``d.h. schon vor der
Beschneidung''\textgreater✲ besessen hatte\textless sup title=``1.Mose
17''\textgreater✲. So sollte er der Vater aller derer werden, die ohne
Beschneidung glauben, damit ihnen die Gerechtigkeit angerechnet werde,
\bibleverse{12} und (ebenso) der Vater der Beschnittenen, nämlich derer,
die nicht nur infolge der (leiblichen) Beschneidung ihm angehören,
sondern die auch in den Fußtapfen des Glaubens wandeln, den unser Vater
Abraham schon in unbeschnittenem Zustande besessen\textless sup
title=``oder: bewiesen''\textgreater✲ hat.

\hypertarget{d-die-heilsverheiuxdfung-ist-dem-abraham-nicht-durch-das-gesetz-zuteil-geworden-sondern-durch-den-glauben}{%
\paragraph{d) Die Heilsverheißung ist dem Abraham nicht durch das Gesetz
zuteil geworden, sondern durch den
Glauben}\label{d-die-heilsverheiuxdfung-ist-dem-abraham-nicht-durch-das-gesetz-zuteil-geworden-sondern-durch-den-glauben}}

\bibleverse{13} Denn die Verheißung, die Abraham oder sein
Samen\textless sup title=``=~seine leibliche
Nachkommenschaft''\textgreater✲ empfangen hat, daß er der Erbe der Welt
sein sollte, ist ihm nicht durch das Gesetz zuteil geworden, sondern
durch die Glaubensgerechtigkeit. \bibleverse{14} Wenn nämlich die
Gesetzesleute die Erben sind, so ist damit der Glaube
entleert\textless sup title=``oder: wertlos gemacht''\textgreater✲ und
die Verheißung entkräftet\textless sup title=``oder:
aufgehoben''\textgreater✲; \bibleverse{15} denn das Gesetz bringt (nur)
Zorn zustande; wo dagegen kein Gesetz ist, da gibt es auch keine
Übertretung. \bibleverse{16} Deshalb ist es\textless sup title=``d.h.
das verheißene Erbe''\textgreater✲ an den Glauben gebunden -- es soll ja
ein Gnadengeschenk sein --, damit die Verheißung für die gesamte
Nachkommenschaft Gültigkeit habe, und zwar nicht nur für die, welche es
aufgrund des Gesetzes ist, sondern auch für die, welche wie Abraham
glaubt, der ja unser aller Vater ist~-- \bibleverse{17} nach dem
Schriftwort\textless sup title=``1.Mose 17,5''\textgreater✲: »Zum Vater
vieler Völker habe ich dich gesetzt\textless sup title=``oder:
bestimmt''\textgreater✲« -- vor dem Gott, dem er geglaubt hat als dem,
welcher die Toten lebendig macht und das noch nicht Vorhandene
benennt\textless sup title=``oder: so ruft''\textgreater✲, als wäre es
schon vorhanden.

\hypertarget{e-der-vorbildliche-glaube-abrahams}{%
\paragraph{e) Der vorbildliche Glaube
Abrahams}\label{e-der-vorbildliche-glaube-abrahams}}

\bibleverse{18} Abraham hat da, wo nichts zu hoffen war, doch
hoffnungsvoll am Glauben festgehalten, damit er der Vater vieler Völker
würde nach der Verheißung\textless sup title=``d.h. weil ihm von Gott
zugesagt war; 1.Mose 15,5''\textgreater✲: »So (unzählbar) soll deine
Nachkommenschaft sein«; \bibleverse{19} und ohne im Glauben schwach zu
werden, nahm er, der fast hundertjährige Mann, die Erstorbenheit seines
eigenen Leibes und den schon erstorbenen Mutterschoß der Sara wahr.
\bibleverse{20} Trotzdem ließ er sich im Hinblick auf die Verheißung
Gottes nicht durch Unglauben irre machen, sondern vielmehr wurde er im
Glauben immer stärker, indem er Gott die Ehre gab \bibleverse{21} und
der festen Überzeugung lebte, daß Gott das, was er verheißen hatte, auch
zu verwirklichen vermöge. \bibleverse{22} Darum ist es\textless sup
title=``d.h. sein Glaube''\textgreater✲ ihm auch zur Gerechtigkeit
gerechnet worden\textless sup title=``1.Mose 15,6''\textgreater✲.

\hypertarget{f-auch-uns-bringt-solcher-glaube-gerechtigkeit-und-seligkeit}{%
\paragraph{f) Auch uns bringt solcher Glaube Gerechtigkeit und
Seligkeit}\label{f-auch-uns-bringt-solcher-glaube-gerechtigkeit-und-seligkeit}}

\bibleverse{23} Aber nicht nur um seinetwillen steht geschrieben, daß
es\textless sup title=``d.h. der Glaube''\textgreater✲ ihm angerechnet
worden ist, \bibleverse{24} sondern auch um unsertwillen; denn auch uns
soll es\textless sup title=``d.h. der Glaube''\textgreater✲ angerechnet
werden, uns, die wir an den glauben, der unsern Herrn Jesus von den
Toten auferweckt hat, \bibleverse{25} ihn, der um unserer Übertretungen
willen in den Tod gegeben\textless sup title=``Jes 53,4-5''\textgreater✲
und um unserer Rechtfertigung willen auferweckt worden ist.

\hypertarget{mit-der-rechtfertigung-ist-die-gewiuxdfheit-der-endlichen-rettung-und-vollendung-gegeben}{%
\subsubsection{3. Mit der Rechtfertigung ist die Gewißheit der endlichen
Rettung und Vollendung
gegeben}\label{mit-der-rechtfertigung-ist-die-gewiuxdfheit-der-endlichen-rettung-und-vollendung-gegeben}}

\hypertarget{a-die-kuxfcnftige-rettung-ist-fuxfcr-die-gerechtfertigten-trotz-aller-truxfcbsale-aufgrund-der-durch-den-opfertod-christi-bewiesenen-liebe-gottes-gewuxe4hrleistet}{%
\paragraph{a) Die künftige Rettung ist für die Gerechtfertigten trotz
aller Trübsale aufgrund der durch den Opfertod Christi bewiesenen Liebe
Gottes
gewährleistet}\label{a-die-kuxfcnftige-rettung-ist-fuxfcr-die-gerechtfertigten-trotz-aller-truxfcbsale-aufgrund-der-durch-den-opfertod-christi-bewiesenen-liebe-gottes-gewuxe4hrleistet}}

\hypertarget{section-4}{%
\section{5}\label{section-4}}

\bibleverse{1} Da wir nun aus Glauben\textless sup title=``=~aufgrund
des Glaubens''\textgreater✲ gerechtfertigt worden sind, so haben wir
Frieden mit Gott durch unsern Herrn Jesus Christus, \bibleverse{2} durch
den wir im Glauben auch den Zugang zu unserm jetzigen Gnadenstande
erlangt haben, und wir rühmen uns auch der Hoffnung auf die Herrlichkeit
Gottes. \bibleverse{3} Ja noch mehr als das: wir rühmen uns dessen sogar
in den Trübsalen, weil wir wissen, daß die Trübsal standhaftes
Ausharren\textless sup title=``oder: Geduld''\textgreater✲ wirkt,
\bibleverse{4} das standhafte Ausharren Bewährung, die Bewährung
Hoffnung; \bibleverse{5} die Hoffnung aber führt nicht zur Enttäuschung,
weil die Liebe Gottes in unsere Herzen ausgegossen ist durch den
heiligen Geist, der uns verliehen worden ist. \bibleverse{6} Denn
Christus ist ja, als wir nach Lage der Dinge noch schwach\textless sup
title=``=~in Sünden''\textgreater✲ waren, für Gottlose gestorben.
\bibleverse{7} Denn kaum wird (sonst wohl) jemand für einen Gerechten
den Tod erleiden -- doch für den Guten entschließt sich vielleicht noch
jemand dazu, sogar sein Leben hinzugeben --; \bibleverse{8} Gott aber
beweist seine Liebe zu uns dadurch, daß Christus für uns gestorben ist,
als wir noch Sünder waren. \bibleverse{9} So werden wir also jetzt,
nachdem wir durch sein Blut gerechtfertigt sind, noch viel gewisser
durch ihn vor dem Zorn (Gottes) gerettet werden. \bibleverse{10} Denn
wenn wir, als\textless sup title=``oder: obgleich''\textgreater✲ wir
noch Feinde Gottes waren, mit ihm durch den Tod seines Sohnes versöhnt
worden sind, so werden wir jetzt als Versöhnte noch viel gewisser
Rettung finden durch sein\textless sup title=``d.h.
Christi''\textgreater✲ Leben. \bibleverse{11} Aber noch mehr: wir rühmen
uns sogar Gottes durch unsern Herrn Jesus Christus, durch den wir jetzt
die Versöhnung empfangen haben.

\hypertarget{b-christus-als-gegenbild-von-adam-die-gnade-die-unverguxe4ngliches-leben-bringt-ist-muxe4chtiger-als-die-todbringende-suxfcnde}{%
\paragraph{b) Christus als Gegenbild von Adam; die Gnade, die
unvergängliches Leben bringt, ist mächtiger als die todbringende
Sünde}\label{b-christus-als-gegenbild-von-adam-die-gnade-die-unverguxe4ngliches-leben-bringt-ist-muxe4chtiger-als-die-todbringende-suxfcnde}}

\bibleverse{12} Darum, gleichwie durch einen Menschen die Sünde in die
Welt hineingekommen ist und durch die Sünde der Tod, und so der Tod zu
allen Menschen hindurchgedrungen ist, weil sie ja alle gesündigt
haben~-- \bibleverse{13} denn bis zum\textless sup title=``=~schon vor
dem''\textgreater✲ Gesetz war Sünde in der Welt vorhanden, die Sünde
wird nur nicht angerechnet, wenn\textless sup title=``oder:
weil''\textgreater✲ kein Gesetz vorhanden ist; \bibleverse{14} aber
trotzdem hat der Tod seine Herrschaft unbeschränkt von Adam bis Mose
sogar über die ausgeübt, welche sich nicht durch Übertretung (eines
vorliegenden Gebotes) in gleicher Weise versündigt hatten wie Adam, der
das Vorbild\textless sup title=``oder: ein Gegenbild''\textgreater✲ des
zukünftigen (Adam) ist. \bibleverse{15} Jedoch verhält es sich mit der
Gnadengabe (Jesu) nicht so wie mit der Übertretung (Adams). Denn wenn
(dort) die Übertretung des Einen den Tod der Vielen\textless sup
title=``=~aller Menschen''\textgreater✲ zur Folge gehabt hat, so hat
sich (hier) die Gnade Gottes und die Gnadengabe des einen Menschen Jesus
Christus erst recht an den Vielen\textless sup title=``oder: für die
Vielen''\textgreater✲ überreich erwiesen✲. \bibleverse{16} Auch ist bei
der Gabe die Wirkung nicht so wie dort, wo ein einziger Sünder den Anlaß
gegeben hat. Denn (dort) ist das Urteil aus Anlaß eines einzigen Sünders
zum Verdammungsurteil geworden, (hier) dagegen die Gnadengabe aus Anlaß
dieser Übertretungen zum Rechtfertigungsurteil✲. \bibleverse{17} Denn
wenn (dort) infolge der Übertretung des Einen der Tod durch die Schuld
jenes Einen seine Herrschaft unbeschränkt ausgeübt hat, so werden (hier)
noch viel gewisser die, welche die überschwengliche Fülle der Gnade und
des Geschenks der Gerechtigkeit empfangen, im (künftigen) Leben als
Könige herrschen durch den Einen, Jesus Christus.

\bibleverse{18} Also: wie es durch eine einzige Übertretung für alle
Menschen zum Verdammungsurteil gekommen ist, so kommt es auch durch eine
einzige Rechttat für alle Menschen zur lebenwirkenden Rechtfertigung✲.
\bibleverse{19} Wie nämlich durch den Ungehorsam des einen Menschen die
Vielen als Sünder hingestellt worden sind, ebenso werden auch durch den
Gehorsam des Einen\textless sup title=``Phil 2,8''\textgreater✲ die
Vielen als Gerechte hingestellt werden. \bibleverse{20} Das Gesetz aber
ist nur nebenbei hereingekommen, damit die Übertretung noch größer
würde. Wo aber die Sünde zugenommen hatte, da ist die Gnade erst recht
überreich hervorgetreten, \bibleverse{21} damit, gleichwie die Sünde
königlich geherrscht hat durch den Tod, so auch die Gnade ihre
Königsherrschaft ausübe durch (gottgewirkte) Gerechtigkeit zum ewigen
Leben durch Jesus Christus, unsern Herrn.

\hypertarget{in-christus-sind-wir-durch-seine-fleischwerdung-seine-kreuzigung-und-seine-auferweckung-von-der-herrschaft-des-gesetzes-der-suxfcnde-und-des-todes-freigemacht-und-erben-der-herrlichkeit-geworden}{%
\subsubsection{4. In Christus sind wir durch seine Fleischwerdung, seine
Kreuzigung und seine Auferweckung von der Herrschaft des Gesetzes, der
Sünde und des Todes freigemacht und Erben der Herrlichkeit
geworden}\label{in-christus-sind-wir-durch-seine-fleischwerdung-seine-kreuzigung-und-seine-auferweckung-von-der-herrschaft-des-gesetzes-der-suxfcnde-und-des-todes-freigemacht-und-erben-der-herrlichkeit-geworden}}

\hypertarget{a-die-heilsbedeutung-der-taufe-die-heilserkenntnis-der-erluxf6sung-in-christus-und-das-bleiben-der-wandel-darin}{%
\paragraph{a) Die Heilsbedeutung der Taufe; die Heilserkenntnis der
Erlösung in Christus und das Bleiben, der Wandel,
darin}\label{a-die-heilsbedeutung-der-taufe-die-heilserkenntnis-der-erluxf6sung-in-christus-und-das-bleiben-der-wandel-darin}}

\hypertarget{aa-wir-sind-mitgekreuzigt-mitgestorben-mitbegraben-durch-die-taufe-und-mitauferweckt-mit-christus-jesus}{%
\subparagraph{aa) Wir sind mitgekreuzigt, mitgestorben, mitbegraben
(durch die Taufe) und mitauferweckt mit Christus
Jesus}\label{aa-wir-sind-mitgekreuzigt-mitgestorben-mitbegraben-durch-die-taufe-und-mitauferweckt-mit-christus-jesus}}

\hypertarget{section-5}{%
\section{6}\label{section-5}}

\bibleverse{1} Was folgt nun daraus? Wollen\textless sup title=``oder:
sollen''\textgreater✲ wir in der Sünde verharren, damit die Gnade sich
um so reicher erweise? \bibleverse{2} Nimmermehr! Wie sollten wir, die
wir der Sünde gestorben\textless sup title=``oder: für die Sünde
tot''\textgreater✲ sind, in ihr noch weiterleben? \bibleverse{3} Oder
wißt ihr nicht, daß wir alle, die wir auf Christus Jesus\textless sup
title=``oder: in Jesus Christus hinein''\textgreater✲ getauft worden
sind, auf seinen Tod getauft\textless sup title=``oder: in seinen Tod
hineinversenkt''\textgreater✲ worden sind? \bibleverse{4} Wir sind also
deshalb durch die Taufe in den Tod mit ihm begraben worden, damit,
gleichwie Christus von den Toten auferweckt worden ist durch die
Herrlichkeit des Vaters, ebenso auch wir in einem neuen Leben wandeln.
\bibleverse{5} Denn wenn wir mit ihm zur Gleichheit des Todes
verwachsen\textless sup title=``=~aufs engste verbunden''\textgreater✲
sind, so werden wir es auch hinsichtlich seiner Auferstehung sein;
\bibleverse{6} wir erkennen ja dies, daß unser alter Mensch deshalb
mitgekreuzigt worden ist, damit der von der Sünde beherrschte Leib
vernichtet werde\textless sup title=``oder: abgetan sei''\textgreater✲,
auf daß wir hinfort nicht mehr der Sünde als Sklaven dienen;
\bibleverse{7} denn wer gestorben ist, der ist dadurch von (jedem
Rechtsanspruch) der Sünde freigesprochen.

\hypertarget{bb-das-mitleben-mit-dem-auferstandenen-christus}{%
\subparagraph{bb) Das Mitleben mit dem auferstandenen
Christus}\label{bb-das-mitleben-mit-dem-auferstandenen-christus}}

\bibleverse{8} Sind wir aber mit Christus gestorben, so glauben wir
zuversichtlich, daß wir auch mit ihm leben werden, \bibleverse{9} da
Christus, wie wir wissen, nach seiner Auferweckung von den Toten nicht
mehr stirbt: der Tod hat keine Herrschermacht✲ mehr über ihn.
\bibleverse{10} Denn den Tod, den er gestorben ist, hat er der Sünde ein
für allemal entrichtet, das Leben aber, das er (jetzt) lebt, ist Leben
für Gott. \bibleverse{11} Ebenso müßt auch ihr euch als tot für die
Sünde betrachten, aber als lebend für Gott in Christus Jesus, unserm
Herrn.

\hypertarget{cc-mahnung-des-apostels-an-die-gluxe4ubigen-in-dieser-heilserkenntnis-zu-bleiben-und-der-suxfcnde-nicht-mehr-zu-dienen}{%
\subparagraph{cc) Mahnung des Apostels an die Gläubigen, in dieser
Heilserkenntnis zu bleiben und der Sünde nicht mehr zu
dienen}\label{cc-mahnung-des-apostels-an-die-gluxe4ubigen-in-dieser-heilserkenntnis-zu-bleiben-und-der-suxfcnde-nicht-mehr-zu-dienen}}

\bibleverse{12} So darf also die Sünde in eurem sterblichen Leibe nicht
mehr so herrschen, daß ihr seinen Begierden Gehorsam leistet;
\bibleverse{13} und stellet auch eure Glieder nicht mehr als
Waffen\textless sup title=``oder: Werkzeuge''\textgreater✲ der
Ungerechtigkeit in den Dienst der Sünde; stellet euch vielmehr als
solche, die aus dem Tode zum Leben erstanden sind, in den Dienst Gottes,
und gebt (so) eure Glieder als Waffen\textless sup title=``oder:
Werkzeuge''\textgreater✲ der Gerechtigkeit an Gott hin! \bibleverse{14}
Denn die Sünde wird kein Herrscherrecht (mehr) über euch ausüben: ihr
steht ja nicht (mehr) unter dem Gesetz, sondern unter der Gnade.

\hypertarget{b-freiheit-vom-gesetz-ist-freiheit-von-der-suxfcnde-nicht-fuxfcr-die-suxfcnde-wir-sollen-knechte-der-gottesgerechtigkeit-sein}{%
\paragraph{b) Freiheit vom Gesetz ist Freiheit von der Sünde, nicht für
die Sünde; wir sollen Knechte der Gottesgerechtigkeit
sein}\label{b-freiheit-vom-gesetz-ist-freiheit-von-der-suxfcnde-nicht-fuxfcr-die-suxfcnde-wir-sollen-knechte-der-gottesgerechtigkeit-sein}}

\hypertarget{aa-der-suxfcndendienst-ist-dem-gerechtigkeitsdienst-gewichen}{%
\subparagraph{aa) Der Sündendienst ist dem Gerechtigkeitsdienst
gewichen}\label{aa-der-suxfcndendienst-ist-dem-gerechtigkeitsdienst-gewichen}}

\bibleverse{15} Was folgt nun daraus? Wollen wir sündigen, weil wir
nicht unter dem Gesetz, sondern unter der Gnade stehen? Nimmermehr!
\bibleverse{16} Ihr wißt ja doch, daß, wenn ihr euch jemand als Knechte
zum Gehorsam hingebt, ihr dann auch dessen Knechte seid und ihm Gehorsam
zu leisten habt, und zwar entweder (als Knechte) der Sünde, was zum Tode
führt, oder (als Knechte) des Gehorsams (gegen Gott), wodurch ihr zur
(lebenspendenden) Gerechtigkeit gelangt. \bibleverse{17} Gott aber sei
Dank, daß ihr früher zwar Knechte der Sünde gewesen seid, jetzt aber
euch von Herzen der Lehre in der Gestalt angeschlossen habt, wie ihr
derselben übergeben\textless sup title=``oder: zugewiesen''\textgreater✲
worden seid! \bibleverse{18} So seid ihr nunmehr von (der Herrschaft)
der Sünde frei geworden und in den Dienst der Gerechtigkeit getreten~--
\bibleverse{19} ich gebrauche da einen Ausdruck, der menschlichen
Verhältnissen entnommen ist, und zwar mit Rücksicht auf die Schwachheit
eures Fleisches. Denn wie ihr vordem eure Glieder in den Knechtsdienst
der Unsittlichkeit und der Gesetzlosigkeit zu einem gesetzlosen Leben
gestellt habt, ebenso stellet jetzt eure Glieder als Knechte in den
Dienst der Gerechtigkeit, um zur Heiligung zu gelangen. \bibleverse{20}
Denn damals, als ihr Knechte der Sünde waret, da waret ihr freie Leute
gegenüber der Gerechtigkeit. \bibleverse{21} Welche Frucht habt ihr nun
damals aufzuweisen gehabt? Nur solche (Früchte), deren ihr euch jetzt
schämt; denn das Ende davon ist der Tod. \bibleverse{22} Jetzt dagegen,
wo ihr von der Sünde frei und Knechte Gottes geworden seid, habt ihr als
eure Frucht die Heiligung und als Endergebnis das ewige Leben.
\bibleverse{23} Denn der Sold, den die Sünde zahlt, ist der Tod, die
Gnadengabe Gottes aber ist das ewige Leben in Christus Jesus, unserm
Herrn.

\hypertarget{bb-sind-wir-mit-christus-gestorben-und-auferstanden-so-sind-wir-rechtmuxe4uxdfig-frei-vom-gesetz-und-sind-verpflichtet-im-dienst-des-auferstandenen-uns-bezuxfcglich-der-suxfcnde-fuxfcr-tot-zu-halten}{%
\subparagraph{bb) Sind wir mit Christus gestorben und auferstanden, so
sind wir rechtmäßig frei vom Gesetz und sind verpflichtet, im Dienst des
Auferstandenen uns bezüglich der Sünde für tot zu
halten}\label{bb-sind-wir-mit-christus-gestorben-und-auferstanden-so-sind-wir-rechtmuxe4uxdfig-frei-vom-gesetz-und-sind-verpflichtet-im-dienst-des-auferstandenen-uns-bezuxfcglich-der-suxfcnde-fuxfcr-tot-zu-halten}}

\hypertarget{section-6}{%
\section{7}\label{section-6}}

\bibleverse{1} Oder wißt ihr nicht, meine Brüder -- ich rede ja doch zu
gesetzeskundigen Leuten --, daß das Gesetz für\textless sup
title=``oder: über''\textgreater✲ den Menschen nur, solange er lebt,
bindende Gewalt hat? \bibleverse{2} So ist z.B. eine verheiratete Frau
gesetzlich an ihren Mann so lange gebunden, als er lebt; wenn aber der
Mann stirbt, so ist sie frei von dem Gesetz, das sie an den Mann bindet.
\bibleverse{3} Demnach wird sie zwar, solange ihr Mann lebt, allgemein
als Ehebrecherin gelten, wenn sie sich einem andern Manne zu eigen gibt;
stirbt aber ihr Mann, so ist sie frei vom Gesetz und keine Ehebrecherin,
wenn sie sich einem andern Mann zu eigen gibt. \bibleverse{4} Mithin
seid auch ihr, meine Brüder, dem Gesetz gegenüber getötet worden, und
zwar durch (das Getötetwerden) des Leibes Christi, um hinfort einem
anderen, nämlich dem, der von den Toten auferweckt worden ist, als
Eigentum anzugehören, damit wir nunmehr für Gott Frucht
brächten\textless sup title=``oder: bringen''\textgreater✲.
\bibleverse{5} Denn solange wir im Fleische waren, wirkten sich die
durch das Gesetz erregten sündhaften Leidenschaften in unsern Gliedern
in der Weise aus, daß wir für den Tod Frucht brachten. \bibleverse{6}
Jetzt aber sind wir vom Gesetz losgekommen, da wir dem, was uns in
Banden hielt, gestorben sind, so daß wir nunmehr unsern Dienst im neuen
Wesen des Geistes und nicht mehr im alten Wesen des Buchstabens (des
Gesetzes) leisten.

\hypertarget{c-die-unheilvolle-wirkung-des-gesetzes-das-den-menschen-mit-der-suxfcnde-bekannt-macht-und-die-suxfcnde-zum-aufleben-im-fleische-bringt}{%
\paragraph{c) Die unheilvolle Wirkung des Gesetzes, das den Menschen mit
der Sünde bekannt macht und die Sünde zum Aufleben im Fleische
bringt}\label{c-die-unheilvolle-wirkung-des-gesetzes-das-den-menschen-mit-der-suxfcnde-bekannt-macht-und-die-suxfcnde-zum-aufleben-im-fleische-bringt}}

\bibleverse{7} Was folgt nun daraus? Ist das Gesetz (selbst)
Sünde\textless sup title=``=~etwas Sündhaftes''\textgreater✲?
Nimmermehr! Aber ich hätte die Sünde nicht kennengelernt außer durch das
Gesetz; denn ich hätte auch von der bösen Lust nichts gewußt, wenn das
Gesetz nicht gesagt hätte\textless sup title=``2.Mose
20,17''\textgreater✲: »Laß dich nicht gelüsten!« \bibleverse{8} Da hat
die Sünde eine Angriffsgelegenheit gegen mich gewonnen und durch das
Gebot jegliche böse Lust in mir zustande gebracht; denn ohne
Gesetz\textless sup title=``d.h. wo kein Gesetz ist''\textgreater✲ ist
die Sünde tot. \bibleverse{9} Ich lebte einst ohne das Gesetz; als dann
aber das Gebot (des Gesetzes) kam, lebte die Sünde (in mir) auf,
\bibleverse{10} für mich aber kam der Tod; und so erwies sich dasselbe
Gebot, das doch zum Leben verhelfen soll, für mich als todbringend;
\bibleverse{11} denn nachdem die Sünde eine Angriffsgelegenheit gegen
mich gewonnen hatte, betrog sie mich durch das Gebot und brachte mir
durch dieses den Tod. \bibleverse{12} Demnach ist das Gesetz (an sich)
heilig und ebenso das Gebot heilig, gerecht und gut. \bibleverse{13} So
hat also etwas Gutes mir den Tod gebracht? O nein, das hat vielmehr die
Sünde getan: sie sollte als Sünde zutage treten, indem sie mir durch das
Gute den Tod brachte; sie sollte sich eben durch das Gebot als über alle
Maßen sündig erweisen\textless sup title=``=~sich zu maßloser Sündigkeit
steigern''\textgreater✲.

\hypertarget{d-die-ohnmacht-des-gesetzes-und-des-guten-willens-gegenuxfcber-der-im-fleisch-als-macht-herrschenden-suxfcnde}{%
\paragraph{d) Die Ohnmacht des Gesetzes und des guten Willens gegenüber
der im Fleisch als Macht herrschenden
Sünde}\label{d-die-ohnmacht-des-gesetzes-und-des-guten-willens-gegenuxfcber-der-im-fleisch-als-macht-herrschenden-suxfcnde}}

\bibleverse{14} Wir wissen ja, daß das Gesetz geistlich ist\textless sup
title=``d.h. aus dem göttlichen Geist stammt''\textgreater✲; ich aber
bin von fleischlicher Art (und dadurch) unter die (Gewalt der) Sünde
verkauft. \bibleverse{15} Ja, mein ganzes Tun ist mir unbegreiflich;
denn ich vollbringe nicht das, was ich will, sondern tue das, was ich
hasse✲. \bibleverse{16} Wenn ich aber das tue, was ich nicht will, so
erkenne ich durch die innere Zustimmung zum Gesetz an, daß dieses gut
sei. \bibleverse{17} Jetzt\textless sup title=``=~in diesem
Falle''\textgreater✲ aber bin nicht mehr ich der, welcher
es\textless sup title=``d.h. das Böse''\textgreater✲ vollbringt, sondern
die in mir wohnende Sünde. \bibleverse{18} Denn ich weiß ja: in mir, das
heißt in meinem Fleische, wohnt nichts Gutes; denn der gute Wille ist
bei mir wohl vorhanden, dagegen das Vollbringen des Guten nicht;
\bibleverse{19} denn ich tue nicht das Gute, das ich tun will, sondern
vollbringe das Böse, das ich nicht tun will. \bibleverse{20} Wenn ich
aber das tue, was ich nicht will, so bin nicht mehr ich es, der es
vollbringt, sondern die in mir wohnende Sünde. \bibleverse{21} Ich finde
somit bei mir, der ich das Gute tun will, das Gesetz\textless sup
title=``=~den Zwang''\textgreater✲ vor, daß bei mir das Böse zustande
kommt. \bibleverse{22} Denn nach meinem inneren Menschen stimme ich dem
göttlichen Gesetz freudig zu, \bibleverse{23} nehme aber in meinen
Gliedern ein andersartiges Gesetz wahr, das dem Gesetz meiner Vernunft
widerstreitet und mich gefangennimmt unter das Gesetz der Sünde, das in
meinen Gliedern wirkt. \bibleverse{24} O ich unglückseliger Mensch! Wer
wird mich aus diesem Todesleibe erlösen? \bibleverse{25} Dank sei Gott;
(es ist geschehen) durch Jesus Christus, unsern Herrn!

Also ist es so: Auf mich selbst gestellt diene ich mit der Vernunft dem
Gesetz Gottes, mit dem Fleisch dagegen dem Gesetz der Sünde.

\hypertarget{e-die-im-auferstandenen-christus-den-gluxe4ubigen-dargebotene-lebenskraft-des-gottesgeistes-macht-frei-von-suxfcnde-und-tod}{%
\paragraph{e) Die im auferstandenen Christus den Gläubigen dargebotene
Lebenskraft des Gottesgeistes macht frei von Sünde und
Tod}\label{e-die-im-auferstandenen-christus-den-gluxe4ubigen-dargebotene-lebenskraft-des-gottesgeistes-macht-frei-von-suxfcnde-und-tod}}

\hypertarget{aa-der-christ-steht-unter-dem-gesetz-des-geistes}{%
\subparagraph{aa) Der Christ steht unter dem Gesetz des
Geistes}\label{aa-der-christ-steht-unter-dem-gesetz-des-geistes}}

\hypertarget{section-7}{%
\section{8}\label{section-7}}

\bibleverse{1} So gibt es also jetzt keine Verurteilung mehr für die,
welche in Christus Jesus sind; \bibleverse{2} denn das Gesetz des
Lebensgeistes in Christus Jesus hat uns von dem Gesetz der Sünde und des
Todes freigemacht. \bibleverse{3} Denn was dem (mosaischen) Gesetz
unmöglich war, das, worin es wegen (des Widerstandes) des Fleisches
ohnmächtig war -- Gott hat (es vollbracht), (nämlich) die Sünde im
Fleische verurteilt, indem er seinen Sohn in der Gleichgestalt des
Sündenfleisches und um der Sünde willen sandte, \bibleverse{4} damit die
Rechtsforderung des Gesetzes ihre Erfüllung fände in uns\textless sup
title=``oder: an uns''\textgreater✲, die wir nicht nach dem Fleische
wandeln, sondern nach dem Geiste.

\hypertarget{bb-der-gegensatz-zwischen-den-menschen-die-gott-im-geist-dienen-und-denen-welche-nach-den-trieben-des-fleisches-leben}{%
\subparagraph{bb) Der Gegensatz zwischen den Menschen, die Gott im Geist
dienen, und denen, welche nach den Trieben des Fleisches
leben}\label{bb-der-gegensatz-zwischen-den-menschen-die-gott-im-geist-dienen-und-denen-welche-nach-den-trieben-des-fleisches-leben}}

\bibleverse{5} Denn die fleischlich gesinnten (Menschen) haben ein
fleischliches Trachten, die geistlich gesinnten aber ein geistliches.
\bibleverse{6} Denn das Trachten des Fleisches bedeutet Tod, das
Trachten des Geistes dagegen Leben und Frieden, \bibleverse{7} und zwar
deshalb, weil das Trachten des Fleisches Feindschaft gegen Gott ist; es
unterwirft sich ja dem Gesetz Gottes nicht, vermag das auch gar nicht;
\bibleverse{8} so können denn die fleischlich gerichteten (Menschen)
Gott nicht gefallen.

\hypertarget{cc-der-christ-als-eine-wohnung-des-geistes}{%
\subparagraph{cc) Der Christ als eine Wohnung des
Geistes}\label{cc-der-christ-als-eine-wohnung-des-geistes}}

\bibleverse{9} Ihr dagegen seid\textless sup title=``oder:
lebt''\textgreater✲ nicht im Fleisch, sondern im Geist, wenn nämlich
Gottes Geist wirklich in euch wohnt; wenn aber jemand den Geist Christi
nicht hat, so gehört ein solcher (Mensch) ihm auch nicht an.
\bibleverse{10} Wohnt dagegen Christus in euch, so ist euer Leib zwar
tot\textless sup title=``oder: dem Tod verfallen''\textgreater✲ um der
Sünde willen, euer Geist aber ist Leben um der Gerechtigkeit willen.
\bibleverse{11} Und wenn der Geist dessen, der Jesus von den Toten
auferweckt hat, in euch wohnt, so wird er, der Christus von den Toten
auferweckt hat, auch eure sterblichen Leiber lebendig machen durch
seinen in euch wohnenden Geist.

\hypertarget{f-der-geistesbesitz-das-leben-im-geiste-verbuxfcrgt-den-kindern-gottes-die-leibeserluxf6sung-wenn-sie-in-den-leiden-dieser-zeit-standhalten}{%
\paragraph{f) Der Geistesbesitz (=~das Leben im Geiste) verbürgt den
Kindern Gottes die Leibeserlösung, wenn sie in den Leiden dieser Zeit
standhalten}\label{f-der-geistesbesitz-das-leben-im-geiste-verbuxfcrgt-den-kindern-gottes-die-leibeserluxf6sung-wenn-sie-in-den-leiden-dieser-zeit-standhalten}}

\bibleverse{12} Somit haben wir, liebe Brüder, nicht dem Fleische
gegenüber die Verpflichtung, nach dem Fleische\textless sup
title=``=~fleischlich; vgl. 7,5''\textgreater✲ zu leben; \bibleverse{13}
denn wenn ihr nach dem Fleische lebt, so ist euch der Tod gewiß; wenn
ihr dagegen durch den Geist die Geschäfte des Leibes tötet, so werdet
ihr leben. \bibleverse{14} Denn alle, die vom Geiste Gottes
geleitet\textless sup title=``oder: getrieben''\textgreater✲
werden\textless sup title=``oder: sich leiten lassen''\textgreater✲, die
sind Söhne Gottes. \bibleverse{15} Der Geist, den ihr empfangen habt,
ist ja doch nicht ein Geist der Knechtschaft, so daß ihr euch aufs neue
fürchten müßtet; sondern ihr habt den Geist der Sohnschaft empfangen, in
welchem\textless sup title=``oder: durch den''\textgreater✲ wir rufen:
»Abba, (lieber) Vater!« \bibleverse{16} Eben dieser Geist ist es, der
vereint mit unserm Geiste ihm bezeugt, daß wir Gottes Kinder sind.
\bibleverse{17} Sind wir aber Kinder, so sind wir auch Erben, und zwar
Erben Gottes und Miterben Christi, wenn wir nämlich mit ihm leiden, um
(einst) auch an seiner Herrlichkeit teilzunehmen.

\bibleverse{18} Ich halte nämlich dafür, daß die Leiden der Jetztzeit
nicht wert sind, verglichen zu werden mit der Herrlichkeit, die an uns
geoffenbart werden soll. \bibleverse{19} Denn das sehnsüchtige
Harren\textless sup title=``oder: Verlangen''\textgreater✲ des
Geschaffenen\textless sup title=``=~der ganzen Schöpfung''\textgreater✲
wartet auf das Offenbarwerden (der Herrlichkeit) der Söhne\textless sup
title=``oder: Kinder''\textgreater✲ Gottes. \bibleverse{20} Denn der
Nichtigkeit\textless sup title=``oder: Vergänglichkeit''\textgreater✲
ist die ganze Schöpfung unterworfen worden -- allerdings nicht
freiwillig\textless sup title=``oder: durch eigene
Schuld''\textgreater✲, sondern um dessen willen, der ihre Unterwerfung
bewirkt hat --, jedoch auf die Hoffnung hin, \bibleverse{21} daß auch
sie selbst, die Schöpfung, von der Knechtschaft der Vergänglichkeit
befreit werden wird zur (Teilnahme an der) Freiheit, welche die Kinder
Gottes im Stande der Verherrlichung besitzen werden. \bibleverse{22} Wir
wissen ja, daß die gesamte Schöpfung bis jetzt noch überall seufzt und
mit Schmerzen einer Neugeburt harrt.

\bibleverse{23} Aber nicht nur sie\textless sup title=``oder:
das''\textgreater✲, sondern auch wir selbst, die wir doch den Geist als
Erstlingsgabe bereits besitzen, seufzen gleichfalls in unserm Inneren
beim Warten auf (das Offenbarwerden) der Sohnschaft, nämlich auf die
Erlösung unsers Leibes. \bibleverse{24} Denn wir sind zwar gerettet
worden, aber doch (bisher) nur auf Hoffnung hin. Eine Hoffnung aber, die
man schon (verwirklicht) sieht, ist keine (rechte) Hoffnung mehr; denn
wozu braucht man noch auf etwas zu hoffen, das man schon (verwirklicht)
sieht? \bibleverse{25} Wenn wir dagegen auf das hoffen, was wir noch
nicht (verwirklicht) sehen, so warten wir darauf in Geduld.

\bibleverse{26} Gleicherweise kommt aber auch der Geist unserer
Schwachheit zu Hilfe; denn wir wissen nicht, was wir so, wie es gerade
not tut\textless sup title=``oder: sich gebührt''\textgreater✲, beten
sollen. Da tritt dann aber der Geist selbst mit
unaussprechlichen\textless sup title=``oder: wortlosen''\textgreater✲
Seufzern für uns ein; \bibleverse{27} der aber, der die Herzen
erforscht\textless sup title=``d.h. Gott''\textgreater✲, versteht die
Sprache des Geistes, weil dieser in einer dem Willen Gottes
entsprechenden Weise für Heilige✲ eintritt.

\hypertarget{der-tiefste-grund-fuxfcr-die-christliche-heils--und-hoffnungsgewiuxdfheit}{%
\subsubsection{5. Der tiefste Grund für die christliche Heils- und
Hoffnungsgewißheit}\label{der-tiefste-grund-fuxfcr-die-christliche-heils--und-hoffnungsgewiuxdfheit}}

\hypertarget{a-der-gottgewirkte-anfang-unserer-gottesgemeinschaft-verbuxfcrgt-ihre-schlieuxdfliche-vollendung}{%
\paragraph{a) Der gottgewirkte Anfang unserer Gottesgemeinschaft
verbürgt ihre schließliche
Vollendung}\label{a-der-gottgewirkte-anfang-unserer-gottesgemeinschaft-verbuxfcrgt-ihre-schlieuxdfliche-vollendung}}

\bibleverse{28} Wir wissen aber, daß denen, die Gott lieben, alle Dinge
zum Guten mitwirken\textless sup title=``oder: dienen''\textgreater✲,
nämlich denen, welche nach seinem Vorsatz\textless sup title=``oder:
seiner Vorherbestimmung''\textgreater✲ berufen sind. \bibleverse{29}
Denn die, welche er zuvor ersehen hat, die hat er auch im voraus dazu
bestimmt, (einst) dem Bilde seines Sohnes gleichgestaltet zu werden:
dieser sollte eben der Erstgeborene unter vielen Brüdern sein.
\bibleverse{30} Und die, welche er vorausbestimmt hat, die hat er auch
berufen; und die er berufen hat, die hat er auch gerechtfertigt; und die
er gerechtfertigt hat, denen hat er auch die (himmlische) Herrlichkeit
verliehen.\textless sup title=``Joh 17,22''\textgreater✲

\hypertarget{b-somit-ist-unser-heilsstand-gegen-alle-muxe4chte-guxf6ttlich-gesichert-und-unsere-glaubensgewiuxdfheit-und-rettungszuversicht-gerechtfertigt}{%
\paragraph{b) Somit ist unser Heilsstand gegen alle Mächte göttlich
gesichert und unsere Glaubensgewißheit und Rettungszuversicht
gerechtfertigt}\label{b-somit-ist-unser-heilsstand-gegen-alle-muxe4chte-guxf6ttlich-gesichert-und-unsere-glaubensgewiuxdfheit-und-rettungszuversicht-gerechtfertigt}}

\bibleverse{31} Was folgt nun hieraus? Wenn Gott für uns ist, wer kann
dann gegen uns sein? \bibleverse{32} Er, der seinen eigenen Sohn nicht
verschont, sondern ihn für uns alle (in den Tod) dahingegeben hat: wie
sollte er uns mit ihm nicht auch alles (andere) schenken?
\bibleverse{33} Wer will\textless sup title=``oder:
sollte''\textgreater✲ Anklage gegen die Auserwählten Gottes erheben?
Gott ist es ja, der sie rechtfertigt. \bibleverse{34} Wer
will\textless sup title=``oder: sollte''\textgreater✲ sie verurteilen?
Etwa Christus Jesus, der doch (für uns) gestorben ist, ja, mehr noch,
der auferweckt worden ist, der zur Rechten Gottes sitzt und auch für uns
eintritt? \bibleverse{35} Wer will\textless sup title=``oder:
sollte''\textgreater✲ uns von der Liebe Christi scheiden? Etwa Trübsal
oder Bedrängnis, Verfolgung oder Hunger oder Mangel an Kleidung, Gefahr
oder Henkerbeil? \bibleverse{36} Wie geschrieben steht\textless sup
title=``Ps 44,23''\textgreater✲: »Um deinetwillen werden wir den ganzen
Tag gemordet; wir sind geachtet wie Schlachtschafe.« \bibleverse{37}
Nein, in dem allem\textless sup title=``=~in allen diesen
Nöten''\textgreater✲ siegen wir weitaus\textless sup title=``oder:
überlegen''\textgreater✲ durch den, der uns geliebt hat. \bibleverse{38}
Denn ich bin dessen gewiß, daß weder Tod noch Leben, weder Engel noch
Gewalten✲, weder Gegenwärtiges noch Zukünftiges noch irgendwelche
Mächte, \bibleverse{39} weder Höhe noch Tiefe\textless sup title=``d.h.
Himmel noch Unterwelt''\textgreater✲ noch sonst irgendetwas anderes
Geschaffenes imstande sein wird, uns von der Liebe Gottes zu scheiden,
die da ist in Christus Jesus, unserm Herrn.

\hypertarget{ii.-die-gegenwuxe4rtige-verwerfung-der-grouxdfen-mehrzahl-des-juxfcdischen-volkes-in-ihrem-verhuxe4ltnis-zum-heilsplan-gottes-kap.-9-11}{%
\subsection{II. Die gegenwärtige Verwerfung der großen Mehrzahl des
jüdischen Volkes in ihrem Verhältnis zum Heilsplan Gottes (Kap.
9-11)}\label{ii.-die-gegenwuxe4rtige-verwerfung-der-grouxdfen-mehrzahl-des-juxfcdischen-volkes-in-ihrem-verhuxe4ltnis-zum-heilsplan-gottes-kap.-9-11}}

\hypertarget{einleitung-der-tiefe-schmerz-des-apostels-uxfcber-die-zeitweilige-ausschlieuxdfung-seines-volkes-vom-heile}{%
\subsubsection{1. Einleitung: Der tiefe Schmerz des Apostels über die
zeitweilige Ausschließung seines Volkes vom
Heile}\label{einleitung-der-tiefe-schmerz-des-apostels-uxfcber-die-zeitweilige-ausschlieuxdfung-seines-volkes-vom-heile}}

\hypertarget{section-8}{%
\section{9}\label{section-8}}

\bibleverse{1} Ich sage die Wahrheit in Christus, ich lüge nicht -- mein
Gewissen bezeugt es mir im heiligen Geist --: \bibleverse{2} ich trage
schweren Kummer und unaufhörlichen Schmerz in meinem Herzen.
\bibleverse{3} Gern wollte ich selbst durch einen Fluch aus der
Gemeinschaft mit Christus ausgestoßen sein, wenn ich dadurch meine
Brüder, meine Volksgenossen nach dem Fleische, retten könnte;
\bibleverse{4} sie sind ja doch Israeliten, denen der
Sohnesstand\textless sup title=``oder: das Kindschaftsrecht =~die
Annahme zum Gottesvolk''\textgreater✲ und die Herrlichkeit Gottes, die
Bündnisse und die Gesetzgebung, der Gottesdienst und die Verheißungen
zuteil geworden sind, \bibleverse{5} denen die Erzväter angehören und
aus denen der Messias dem Fleische nach stammt: der da Gott über allem
ist, gepriesen in Ewigkeit! Amen.

\hypertarget{rechtfertigung-gottes-bezuxfcglich-der-gegenwuxe4rtigen-verwerfung-der-juden}{%
\subsubsection{2. Rechtfertigung Gottes bezüglich der gegenwärtigen
Verwerfung der
Juden}\label{rechtfertigung-gottes-bezuxfcglich-der-gegenwuxe4rtigen-verwerfung-der-juden}}

\hypertarget{a-gottes-verheiuxdfungen-an-israel-sind-unverbruxfcchlich-gelten-aber-nicht-allen-leiblichen-sondern-nur-den-geistlichen-nachkommen-abrahams}{%
\paragraph{a) Gottes Verheißungen an Israel sind unverbrüchlich, gelten
aber nicht allen leiblichen, sondern nur den geistlichen Nachkommen
Abrahams}\label{a-gottes-verheiuxdfungen-an-israel-sind-unverbruxfcchlich-gelten-aber-nicht-allen-leiblichen-sondern-nur-den-geistlichen-nachkommen-abrahams}}

\bibleverse{6} Ich will damit aber nicht gesagt haben, daß Gottes
Verheißungswort hinfällig geworden\textless sup title=``oder: unerfüllt
geblieben''\textgreater✲ sei; denn nicht alle, die aus Israel stammen,
sind Israel, \bibleverse{7} und nicht alle sind schon deshalb, weil sie
Abrahams Same\textless sup title=``d.h. leibliche
Nachkommen''\textgreater✲ sind, auch seine Kinder; sondern\textless sup
title=``1.Mose 21,12''\textgreater✲: »In✲ Isaak soll dir
Nachkommenschaft genannt werden.« \bibleverse{8} Das will ich sagen:
Nicht die leiblichen Kinder (Abrahams) sind damit auch Gottes Kinder,
sondern (nur) die Kinder der Verheißung werden als Nachkommenschaft
(Abrahams) gerechnet. \bibleverse{9} Denn so lautet das Wort der
Verheißung\textless sup title=``1.Mose 18,10.14''\textgreater✲: »(Übers
Jahr) um diese Zeit werde ich (wieder-) kommen, da wird Sara einen Sohn
haben.« \bibleverse{10} Und nicht nur hier\textless sup title=``=~bei
Sara''\textgreater✲ ist es so gewesen, sondern auch bei Rebekka, die von
einem und demselben Manne, nämlich unserm Vater✲ Isaak, guter Hoffnung
war. \bibleverse{11} Denn ehe sie (ihre beiden Kinder) noch geboren
waren und irgend etwas Gutes oder Böses getan hatten, schon da wurde --
damit Gottes Vorherbestimmung aus freier Wahl bestehen bliebe,
\bibleverse{12} abhängig nicht von Werken, sondern (allein) von dem
(Willen des) Berufenden -- der Rebekka gesagt\textless sup
title=``1.Mose 25,23''\textgreater✲: »Der Ältere wird dem Jüngeren
dienstbar sein«; \bibleverse{13} wie ja auch (anderswo) geschrieben
steht\textless sup title=``Mal 1,2-3''\textgreater✲: »Jakob habe ich
geliebt, Esau aber habe ich gehaßt.«

\hypertarget{b-die-erwuxe4hlung-zum-heil-ist-das-freie-werk-der-gnade-gottes-die-versagung-des-heils-und-der-gnade-gestattet-dem-menschen-kein-hadern-mit-gott}{%
\paragraph{b) Die Erwählung zum Heil ist das freie Werk der Gnade
Gottes; die Versagung des Heils und der Gnade gestattet dem Menschen
kein Hadern mit
Gott}\label{b-die-erwuxe4hlung-zum-heil-ist-das-freie-werk-der-gnade-gottes-die-versagung-des-heils-und-der-gnade-gestattet-dem-menschen-kein-hadern-mit-gott}}

\bibleverse{14} Was folgt nun daraus? Liegt da etwa Ungerechtigkeit auf
seiten Gottes vor? Nimmermehr! \bibleverse{15} Zu Mose sagt er
ja\textless sup title=``2.Mose 33,19''\textgreater✲: »Ich werde Gnade
erweisen, wem ich gnädig bin, und werde Barmherzigkeit dem erzeigen,
dessen ich mich erbarme.« \bibleverse{16} Demnach kommt es nicht auf
jemandes Wollen oder Laufen✲ an, sondern auf Gottes Erbarmen.
\bibleverse{17} So sagt ja auch die Schrift zum Pharao\textless sup
title=``2.Mose 9,16''\textgreater✲: »Gerade dazu habe ich dich in die
Welt kommen lassen, um an dir meine Macht zu erweisen und damit mein
Name auf der ganzen Erde verkündet werde.« \bibleverse{18} Also: Gott
erbarmt sich, wessen er will, und verstockt auch, wen er will.
\bibleverse{19} Da wirst du mir nun einwenden: »Wie kann er dann noch
(jemand) tadeln? Wer vermöchte denn seinem Willen\textless sup
title=``oder: Ratschluß''\textgreater✲ Widerstand zu leisten?«
\bibleverse{20} Ja, o Mensch, wer bist denn du, daß du Gott zur
Verantwortung ziehen willst? Darf etwa das Gebilde zu seinem Bildner
sagen: »Warum hast du mich so gemacht?« \bibleverse{21} Oder hat der
Töpfer nicht Macht über den Ton, aus derselben Masse hier ein Gefäß zu
ehrenvoller Bestimmung und dort ein anderes zu gemeiner Verwendung zu
verfertigen?

\bibleverse{22} Wie aber, wenn Gott, obgleich er seinen Zorn offenbaren
und seine Macht an den Tag legen will, doch die Gefäße des Zornes, die
zur Vernichtung hergestellt sind\textless sup title=``oder: für den
Untergang, oder: zum Gericht reif waren''\textgreater✲, mit großer
Langmut getragen hat, \bibleverse{23} um zugleich den Reichtum seiner
Herrlichkeit an Gefäßen des Erbarmens zu erweisen, die er zur (Teilnahme
an seiner) Herrlichkeit zuvor bereitet hat? \bibleverse{24} Als solche
(Gefäße des Erbarmens) hat er auch uns berufen, und zwar nicht nur aus
den Juden, sondern auch aus den Heiden(völkern), \bibleverse{25} wie er
ja auch bei (dem Propheten) Hosea sagt\textless sup title=``Hos
2,25''\textgreater✲: »Ich werde das, was nicht mein Volk ist, mein Volk
nennen und der Ungeliebten den Namen ›Geliebte‹ beilegen«;
\bibleverse{26} und\textless sup title=``Hos 2,1''\textgreater✲: »Es
wird geschehen: an dem Orte, wo zu ihnen gesagt worden ist: ›Ihr seid
nicht mein Volk‹, dort werden sie ›Söhne des lebendigen Gottes‹ genannt
werden.« \bibleverse{27} Jesaja ferner ruft laut im Hinblick auf Israel
aus\textless sup title=``Jes 10,22-23''\textgreater✲: »Wenn auch die
Zahl der Söhne Israels wie der Sand am Meer wäre, wird doch nur der Rest
davon gerettet werden; \bibleverse{28} denn sein Wort\textless sup
title=``=~seine Drohung''\textgreater✲ wird der Herr, indem er die Dinge
sicher und Schlag auf Schlag verlaufen läßt, zur Ausführung auf der Erde
bringen.« \bibleverse{29} Und wie Jesaja vorhergesagt hat\textless sup
title=``Jes 1,9''\textgreater✲: »Hätte der Herr der Heerscharen uns
nicht einen Samen\textless sup title=``=~eine
Nachkommenschaft''\textgreater✲ übriggelassen, so wären wir wie Sodom
geworden und hätten gleiches Schicksal mit Gomorrha gehabt.«

\hypertarget{israels-verwerfung-ist-durch-eigene-schuld-herbeigefuxfchrt}{%
\subsubsection{3. Israels Verwerfung ist durch eigene Schuld
herbeigeführt}\label{israels-verwerfung-ist-durch-eigene-schuld-herbeigefuxfchrt}}

\hypertarget{a-die-schuld-der-juden-hat-in-der-verwerfung-der-glaubensgerechtigkeit-und-im-uxfcbereifrigen-trachten-nach-der-werkgerechtigkeit-bestanden}{%
\paragraph{a) Die Schuld der Juden hat in der Verwerfung der
Glaubensgerechtigkeit und im übereifrigen Trachten nach der
Werkgerechtigkeit
bestanden}\label{a-die-schuld-der-juden-hat-in-der-verwerfung-der-glaubensgerechtigkeit-und-im-uxfcbereifrigen-trachten-nach-der-werkgerechtigkeit-bestanden}}

\bibleverse{30} Was folgt nun daraus? Dieses: Heiden, die nicht nach
Gerechtigkeit trachteten, haben Gerechtigkeit erlangt, nämlich die
Gerechtigkeit, die aus dem Glauben kommt; \bibleverse{31} Israel
dagegen, das nach der vom Gesetz geforderten Gerechtigkeit trachtete,
hat das vom Gesetz gesteckte Ziel (der Rechtfertigung) nicht erreicht.
\bibleverse{32} Warum nicht? Weil sie es nicht auf dem Glaubensweg,
sondern es mit Werken haben erreichen wollen: da haben sie sich am Stein
des Anstoßes gestoßen, \bibleverse{33} von dem geschrieben
steht\textless sup title=``Jes 28,16; 8,14''\textgreater✲: »Siehe, ich
lege in Zion einen Stein des Anstoßes und einen Felsen des
Ärgernisses\textless sup title=``oder: des Strauchelns =~zum
Fallen''\textgreater✲; und wer auf ihn sein Vertrauen setzt\textless sup
title=``oder: an ihn glaubt''\textgreater✲, wird nicht
zuschanden\textless sup title=``oder: enttäuscht''\textgreater✲ werden.«

\hypertarget{section-9}{%
\section{10}\label{section-9}}

\bibleverse{1} Liebe Brüder! Der aufrichtige Wunsch meines Herzens und
mein Gebet zu Gott für sie geht dahin, daß sie gerettet werden;
\bibleverse{2} denn ich muß ihnen das Zeugnis ausstellen, daß sie Eifer
für Gott haben, aber leider nicht in der rechten Erkenntnis.
\bibleverse{3} Denn weil sie die Gottesgerechtigkeit verkannt haben und
dagegen beflissen sind, ihre eigene Gerechtigkeit zur Geltung zu
bringen, haben sie sich der Gottesgerechtigkeit nicht unterworfen.

\hypertarget{b-israels-verschulden-ist-um-so-schwerer-als-gott-nichts-unterlassen-hat-um-israel-schon-seit-moses-zeiten-zur-glaubensgerechtigkeit-zu-fuxfchren}{%
\paragraph{b) Israels Verschulden ist um so schwerer, als Gott nichts
unterlassen hat, um Israel schon seit Moses Zeiten zur
Glaubensgerechtigkeit zu
führen}\label{b-israels-verschulden-ist-um-so-schwerer-als-gott-nichts-unterlassen-hat-um-israel-schon-seit-moses-zeiten-zur-glaubensgerechtigkeit-zu-fuxfchren}}

\bibleverse{4} Denn dem Gesetz hat Christus ein Ende gemacht, damit
jeder, der da glaubt, zur Gerechtigkeit gelange. \bibleverse{5} Mose
schreibt nämlich\textless sup title=``3.Mose 18,5''\textgreater✲: »Der
Mensch, der die vom Gesetz geforderte Gerechtigkeit geübt hat, wird
durch sie das Leben haben.« \bibleverse{6} Die Gerechtigkeit dagegen,
die aus dem Glauben kommt, spricht so\textless sup title=``vgl. 5.Mose
30,12-13''\textgreater✲: »Denke nicht in deinem Herzen: ›Wer wird in den
Himmel hinaufsteigen?‹ -- nämlich um Christus herabzuholen --,
\bibleverse{7} oder: ›Wer wird in den Abgrund\textless sup title=``=~die
Unterwelt''\textgreater✲ hinabsteigen?‹ -- nämlich um Christus von den
Toten heraufzuholen« --, \bibleverse{8} sondern was sagt sie? »Nahe ist
dir das Wort: in deinem Munde und in deinem Herzen (hast du
es)«\textless sup title=``5.Mose 30,14''\textgreater✲, nämlich das Wort
vom Glauben, das wir verkündigen. \bibleverse{9} Denn wenn du »mit
deinem Munde« Jesus als den Herrn bekennst und »mit deinem Herzen«
glaubst, daß Gott ihn von den Toten auferweckt hat, so wirst du gerettet
werden. \bibleverse{10} Denn mit dem Herzen glaubt man (an ihn) zur
Gerechtigkeit\textless sup title=``=~und wird dadurch
gerecht''\textgreater✲, und mit dem Munde bekennt man (ihn) zur
Errettung\textless sup title=``=~und wird dadurch
gerettet''\textgreater✲. \bibleverse{11} Sagt doch die
Schrift\textless sup title=``Jes 28,16''\textgreater✲: »Keiner, der auf
ihn sein Vertrauen setzt\textless sup title=``oder: an ihn
glaubt''\textgreater✲, wird zuschanden✲ werden.« \bibleverse{12} Denn
hier gibt es keinen Unterschied zwischen dem Juden und dem Griechen✲:
sie alle haben ja einen und denselben Herrn, ihn, der sich reich erweist
an allen, die ihn anrufen; \bibleverse{13} denn »jeder, der den Namen
des Herrn anruft, wird gerettet werden«\textless sup title=``Joel
3,5''\textgreater✲. \bibleverse{14} Nun -- wie sollen sie den anrufen,
an den sie nicht zu glauben gelernt haben? Wie sollen sie aber an den
glauben, von dem sie nicht gehört haben? Wie sollen sie aber von ihm
hören ohne einen Verkündiger\textless sup title=``d.h. wenn keiner ihnen
die Heilsbotschaft verkündigt''\textgreater✲? \bibleverse{15} Und wie
soll ihnen jemand verkündigen, ohne dazu ausgesandt zu sein? -- wie es
in der Schrift heißt\textless sup title=``Jes 52,7''\textgreater✲: »Wie
lieblich\textless sup title=``oder: willkommen''\textgreater✲ sind die
Füße✲ derer, welche frohe Botschaft von guten Dingen bringen!«

\hypertarget{c-die-unentschuldbarkeit-des-ungluxe4ubigen-teiles-israels-welcher-das-auch-ihm-angebotene-heil-zuruxfcckgewiesen-hat}{%
\paragraph{c) Die Unentschuldbarkeit des ungläubigen Teiles Israels,
welcher das auch ihm angebotene Heil zurückgewiesen
hat}\label{c-die-unentschuldbarkeit-des-ungluxe4ubigen-teiles-israels-welcher-das-auch-ihm-angebotene-heil-zuruxfcckgewiesen-hat}}

\bibleverse{16} Aber freilich: nicht alle sind der Heilsbotschaft
gehorsam gewesen; sagt doch (schon) Jesaja\textless sup title=``Jes
53,1''\textgreater✲: »Herr, wer hat unserer Botschaft Glauben
geschenkt?« \bibleverse{17} Mithin kommt der Glaube aus der Botschaft✲,
die Predigt aber (erfolgt) durch Christi Wort\textless sup title=``oder:
im Auftrage Christi''\textgreater✲. \bibleverse{18} Aber, frage ich:
Haben sie (die Predigt) vielleicht nicht zu hören bekommen? O doch!
Ȇber die ganze Erde ist ihr Schall gedrungen und ihre Worte bis an die
Enden der bewohnten Welt.«\textless sup title=``Ps 19,5''\textgreater✲
\bibleverse{19} Aber, frage ich: Hat Israel sie vielleicht nicht
verstanden? O doch! (Schon) Mose sagt als erster Zeuge\textless sup
title=``5.Mose 32,21''\textgreater✲: »Ich will euch eifersüchtig machen
auf solche, die kein Volk sind; gegen ein unverständiges Volk will ich
euch erbittern.« \bibleverse{20} Jesaja ferner erkühnt sich zu
sagen\textless sup title=``Jes 65,1''\textgreater✲: »Ich bin gefunden
worden von denen, die mich nicht suchten; ich bin denen bekannt
geworden, die nicht nach mir fragten.« \bibleverse{21} Dagegen in bezug
auf Israel sagt er\textless sup title=``Jes 65,2''\textgreater✲: »Den
ganzen Tag habe ich meine Arme (vergebens) ausgestreckt nach einem
Volke, das ungehorsam ist und widerspricht.«

\hypertarget{wendung-des-gerichts-zum-heil-israels}{%
\subsubsection{4. Wendung des Gerichts zum Heil
Israels}\label{wendung-des-gerichts-zum-heil-israels}}

\hypertarget{a-der-gruxf6uxdfere-teil-der-juden-ist-zwar-verstockt-und-von-gott-verstouxdfen-aber-schon-jetzt-ist-ein-kleiner-teil-durch-gottes-gnade-zum-heil-bestimmt}{%
\paragraph{a) Der größere Teil der Juden ist zwar verstockt und von Gott
verstoßen, aber schon jetzt ist ein kleiner Teil durch Gottes Gnade zum
Heil
bestimmt}\label{a-der-gruxf6uxdfere-teil-der-juden-ist-zwar-verstockt-und-von-gott-verstouxdfen-aber-schon-jetzt-ist-ein-kleiner-teil-durch-gottes-gnade-zum-heil-bestimmt}}

\hypertarget{section-10}{%
\section{11}\label{section-10}}

\bibleverse{1} So frage ich nun: Hat Gott sein Volk etwa
verstoßen?\textless sup title=``Ps 94,14''\textgreater✲ Keineswegs! Ich
bin doch auch ein Israelit, aus Abrahams Nachkommenschaft, aus dem
Stamme Benjamin. \bibleverse{2} Nein, Gott hat sein Volk, das er zuvor
ersehen\textless sup title=``=~sich von Anfang an zum Eigentum
erwählt''\textgreater✲ hat, nicht verstoßen. Oder wißt ihr nicht, was
die Schrift bei (der Erzählung von) Elia sagt, als dieser vor Gott gegen
Israel mit der Klage auftritt\textless sup title=``1.Kön
19,10.14''\textgreater✲: \bibleverse{3} »Herr, deine Propheten haben sie
getötet, deine Altäre niedergerissen; ich bin allein übriggeblieben, und
nun trachten sie mir nach dem Leben«? \bibleverse{4} Aber wie lautet da
die göttliche Antwort an ihn? »Ich habe mir noch siebentausend Männer
übrigbehalten, die ihre Knie vor Baal nicht gebeugt haben.«\textless sup
title=``1.Kön 19,18''\textgreater✲ \bibleverse{5} Ebenso ist nun auch in
unserer Zeit ein Rest nach der göttlichen Gnadenauswahl vorhanden.
\bibleverse{6} Ist er aber durch Gnade (ausgesondert), so ist er es
nicht mehr aufgrund von Werken; sonst würde ja die Gnade keine Gnade
mehr sein.

\bibleverse{7} Wie steht es also? Was Israel erstrebt, das hat es (in
seiner Gesamtheit) nicht erreicht; der auserwählte Teil aber hat es
erreicht; die übrigen dagegen sind verstockt worden, \bibleverse{8} wie
geschrieben steht\textless sup title=``Jes 29,10; 5.Mose
29,3''\textgreater✲: »Gott hat ihnen den Geist der Betäubung✲ gegeben,
Augen des Nichtsehens\textless sup title=``=~um nicht zu
sehen''\textgreater✲ und Ohren des Nichthörens\textless sup title=``=~um
nicht zu hören''\textgreater✲, bis auf den heutigen Tag.« \bibleverse{9}
Und David sagt\textless sup title=``Ps 69,23-24''\textgreater✲: »Möge
ihr Tisch ihnen zur Schlinge und zum Fangnetz werden, zum Fallstrick und
zur Vergeltung! \bibleverse{10} Ihre Augen sollen verfinstert werden,
damit sie nicht sehen, und den Rücken beuge ihnen für immer!«

\hypertarget{b-die-guxf6ttliche-heilsabsicht-bei-der-berufung-der-heiden-war-die-den-unglauben-der-juden-durch-reizung-zur-nacheiferung-zu-besiegen-ihre-verwerfung-ist-nicht-endguxfcltig}{%
\paragraph{b) Die göttliche Heilsabsicht bei der Berufung der Heiden war
die, den Unglauben der Juden durch Reizung zur Nacheiferung zu besiegen;
ihre Verwerfung ist nicht
endgültig}\label{b-die-guxf6ttliche-heilsabsicht-bei-der-berufung-der-heiden-war-die-den-unglauben-der-juden-durch-reizung-zur-nacheiferung-zu-besiegen-ihre-verwerfung-ist-nicht-endguxfcltig}}

\bibleverse{11} So frage ich nun: Sind sie etwa deshalb gestrauchelt,
damit sie zu Fall kommen\textless sup title=``=~ins Verderben
fallen''\textgreater✲ sollten? Keineswegs! Vielmehr ist infolge ihrer
Verfehlung das Heil den Heiden zuteil geworden; das soll
sie\textless sup title=``d.h. die Juden''\textgreater✲ wiederum zur
Nacheiferung reizen. \bibleverse{12} Wenn aber schon ihre Verfehlung ein
reicher Segen für die Menschheit und ihr Zurückbleiben ein reicher Segen
für die Heiden geworden ist, um wieviel segensreicher wird (dann erst)
ihre Vollzahl\textless sup title=``oder: ihr vollzähliges
Eingehen''\textgreater✲ sein!

\bibleverse{13} Euch Heiden(christen) aber sage ich: Gerade weil ich
Heidenapostel bin, tue ich meinem Dienst um so größere Ehre an,
\bibleverse{14} (wenn ich bemüht bin) ob ich vielleicht meine
Volksgenossen zur Nacheiferung zu reizen und (wenigstens) einige von
ihnen zu retten vermag. \bibleverse{15} Denn wenn schon ihre Verwerfung
zur Versöhnung der Welt geführt hat, was wird da ihre Annahme anderes
sein als Leben aus den Toten? \bibleverse{16} Wenn aber das
Erstlingsbrot\textless sup title=``d.h. die Erstlingsgabe vom
Teig''\textgreater✲ heilig ist\textless sup title=``4.Mose
15,19-21''\textgreater✲, so ist es auch die (ganze übrige) Teigmasse;
und wenn die Wurzel heilig ist, so sind es auch die Zweige.

\bibleverse{17} Wenn nun aber einige von den Zweigen herausgebrochen
worden sind und du, der du ein wilder Ölbaum(zweig) warst, unter sie
eingepfropft worden bist und dadurch Anteil an der Wurzel, die dem
Ölbaum die Fettigkeit schafft, erhalten hast, \bibleverse{18} so
rühme\textless sup title=``oder: überhebe''\textgreater✲ dich deswegen
nicht gegen die (anderen) Zweige! Tust du es dennoch (so bedenke wohl):
nicht du trägst die Wurzel, sondern die Wurzel trägt dich.
\bibleverse{19} Du wirst nun einwenden: »Es sind ja doch Zweige
ausgebrochen worden, weil ich eingepfropft werden sollte.«
\bibleverse{20} Ganz recht! Infolge ihres Unglaubens sind sie
ausgebrochen worden, und du stehst infolge deines Glaubens (an ihrer
Stelle). Sei nicht hochmütig, sondern sei auf deiner Hut!
\bibleverse{21} Denn wenn Gott die natürlichen Zweige nicht verschont
hat, so wird er auch dich nicht verschonen. \bibleverse{22} Darum
beachte wohl die Güte, aber auch die Strenge Gottes: seine Strenge gegen
die Gefallenen, dagegen die Güte Gottes gegen dich, vorausgesetzt daß du
bei der (dir widerfahrenen) Güte verbleibst; denn sonst wirst auch du
(aus dem Ölbaum) wieder herausgeschnitten werden, \bibleverse{23}
während umgekehrt jene, wenn sie nicht im Unglauben verharren, wieder
eingepfropft werden; Gott hat ja die Macht\textless sup title=``oder:
das Recht''\textgreater✲ dazu, sie wieder einzupfropfen. \bibleverse{24}
Denn wenn du aus dem wilden Ölbaum, dem du von Haus aus angehörst,
herausgeschnitten und gegen die Natur in den edlen Ölbaum eingepfropft
worden bist: wieviel eher werden diese, die von Natur dahin gehören,
ihrem ursprünglichen Ölbaum (wieder) eingepfropft werden!

\hypertarget{c-der-ganze-rest-vom-volke-israel-wird-schlieuxdflich-nach-bekehrung-der-heidenauswahl-zum-glauben-gelangen-und-alles-wird-zur-rechtfertigung-und-verherrlichung-gottes-gereichen}{%
\paragraph{c) Der ganze Rest vom Volke Israel wird schließlich nach
Bekehrung der Heidenauswahl zum Glauben gelangen, und alles wird zur
Rechtfertigung und Verherrlichung Gottes
gereichen}\label{c-der-ganze-rest-vom-volke-israel-wird-schlieuxdflich-nach-bekehrung-der-heidenauswahl-zum-glauben-gelangen-und-alles-wird-zur-rechtfertigung-und-verherrlichung-gottes-gereichen}}

\bibleverse{25} Ich will euch nämlich, meine Brüder, über dieses
Geheimnis nicht in Unkenntnis lassen, damit ihr nicht in vermeintlicher
Klugheit auf eigene Gedanken verfallt: Verstockung ist über einen Teil
der Israeliten gekommen bis zu der Zeit, da die Vollzahl der Heiden (in
die Gemeinde Gottes) eingegangen sein wird; \bibleverse{26} und auf
diese Weise wird Israel in seiner Gesamtheit gerettet werden, wie
geschrieben steht\textless sup title=``Jes 59,20-21;
27,9''\textgreater✲: »Aus Zion wird der Retter\textless sup
title=``oder: Erlöser''\textgreater✲ kommen; er wird Jakob von allem
gottlosen Wesen frei machen; \bibleverse{27} und darin wird sich ihnen
der von mir herbeigeführte Bund zeigen, wenn ich ihre Sünden
wegnehme\textless sup title=``oder: tilge''\textgreater✲.«
\bibleverse{28} So sind sie im Hinblick auf die Heilsbotschaft zwar
Feinde (Gottes) um euretwillen, aber im Hinblick auf die Erwählung sind
sie Geliebte (Gottes) um der Väter willen; \bibleverse{29} denn
unwiderruflich sind die Gnadengaben\textless sup title=``oder:
Gnadenverheißungen''\textgreater✲ und die Berufung Gottes.
\bibleverse{30} Denn wie ihr einst ungehorsam gegen Gott gewesen seid,
jetzt aber infolge des Ungehorsams dieser Erbarmen erlangt habt,
\bibleverse{31} ebenso sind wiederum diese jetzt ungehorsam geworden, um
durch das euch gewährte Erbarmen (dereinst) ebenfalls Barmherzigkeit zu
erlangen. \bibleverse{32} Denn Gott hat alle zusammen in Ungehorsam
verschlossen, um allen Erbarmen widerfahren zu lassen. \bibleverse{33} O
welch eine Tiefe des Reichtums\textless sup title=``=~der
Gnadenfülle''\textgreater✲ und der Weisheit und der Erkenntnis Gottes!
Wie unbegreiflich sind seine Gerichte\textless sup title=``oder:
Urteile''\textgreater✲ und unerforschlich seine Wege! \bibleverse{34}
»Denn wer hat den Sinn\textless sup title=``=~die
Gedanken''\textgreater✲ des Herrn erkannt, oder wer ist sein Ratgeber
gewesen?«\textless sup title=``Jes 40,13''\textgreater✲ \bibleverse{35}
Oder »wer hat ihm zuerst etwas gegeben, wofür ihm Vergeltung zuteil
werden müßte?«\textless sup title=``Hiob 41,2; Jer 23,18''\textgreater✲
\bibleverse{36} Denn von ihm und durch ihn und zu ihm\textless sup
title=``oder: für ihn''\textgreater✲ sind alle Dinge: ihm gebührt die
Ehre in Ewigkeit! Amen.

\hypertarget{iii.-ermahnungen-an-die-gemeinde-fuxfcr-das-christliche-leben-121-1513}{%
\subsection{III. Ermahnungen an die Gemeinde für das christliche Leben
(12,1-15,13)}\label{iii.-ermahnungen-an-die-gemeinde-fuxfcr-das-christliche-leben-121-1513}}

\hypertarget{allgemeine-mahnung-als-eingang-heiligung-des-persuxf6nlichen-lebens-durch-vuxf6llige-hingabe-an-gott}{%
\subsubsection{1. Allgemeine Mahnung als Eingang: Heiligung des
persönlichen Lebens durch völlige Hingabe an
Gott}\label{allgemeine-mahnung-als-eingang-heiligung-des-persuxf6nlichen-lebens-durch-vuxf6llige-hingabe-an-gott}}

\hypertarget{section-11}{%
\section{12}\label{section-11}}

\bibleverse{1} So ermahne ich euch nun, liebe Brüder, durch (den Hinweis
auf) die Barmherzigkeit Gottes: Bringt eure Leiber als ein lebendiges,
heiliges und Gott wohlgefälliges Opfer dar: (das sei) euer vernünftiger
Gottesdienst! \bibleverse{2} Gestaltet eure Lebensführung nicht nach der
Weise dieser Weltzeit, sondern wandelt euch um durch die Erneuerung
eures Sinnes, damit ihr ein sicheres Urteil darüber gewinnt, welches der
Wille Gottes sei, nämlich das Gute und (Gott) Wohlgefällige und
Vollkommene.

\hypertarget{ermahnung-zur-selbstbescheidung-der-einzelnen-und-zur-treuen-verwendung-der-empfangenen-gnadengaben-im-dienst-der-gemeinde}{%
\subsubsection{2. Ermahnung zur Selbstbescheidung der Einzelnen und zur
treuen Verwendung der empfangenen Gnadengaben im Dienst der
Gemeinde}\label{ermahnung-zur-selbstbescheidung-der-einzelnen-und-zur-treuen-verwendung-der-empfangenen-gnadengaben-im-dienst-der-gemeinde}}

\bibleverse{3} So fordere ich denn kraft der mir verliehenen Gnade einen
jeden von euch auf, nicht höher von sich zu denken, als zu denken sich
gebührt, sondern auf eine besonnene Selbstschätzung bedacht zu sein nach
dem Maß des Glaubens, das Gott einem jeden zugeteilt hat. \bibleverse{4}
Denn wie wir an einem Leibe viele Glieder haben, die Glieder aber nicht
alle denselben Dienst verrichten, \bibleverse{5} so bilden auch wir
trotz unserer Vielheit einen einzigen Leib in Christus, im Verhältnis
zueinander aber sind wir Glieder, \bibleverse{6} doch so, daß wir
Gnadengaben besitzen, die nach der uns verliehenen Gnade verschieden
sind. Wer also die Gabe prophetischer Rede besitzt, bleibe in
Übereinstimmung mit dem Maß des Glaubens; \bibleverse{7} wem die Gabe
des Gemeindedienstes zuteil geworden ist, der betätige sie durch
Dienstleistungen; wer Lehrgabe besitzt, verwende sie als
Lehrer\textless sup title=``oder: zur Belehrung''\textgreater✲;
\bibleverse{8} hat jemand die Gabe des Ermahnens\textless sup
title=``=~der Seelsorge''\textgreater✲, so betätige er sich im
Ermahnen\textless sup title=``=~in der Seelsorge''\textgreater✲; wer
Mildtätigkeit übt, tue es in Einfalt; wer zu den Vorstehern gehört,
zeige rechten Eifer; wer Barmherzigkeit übt, tue es mit Freudigkeit!

\hypertarget{ermahnung-zur-bruderliebe-und-zur-betuxe4tigung-christlicher-gesinnung-gegen-freund-und-feind}{%
\subsubsection{3. Ermahnung zur Bruderliebe und zur Betätigung
christlicher Gesinnung gegen Freund und
Feind}\label{ermahnung-zur-bruderliebe-und-zur-betuxe4tigung-christlicher-gesinnung-gegen-freund-und-feind}}

\bibleverse{9} Die Liebe sei ungeheuchelt! Verabscheut das Böse, haltet
am Guten fest! \bibleverse{10} In der Bruderliebe zueinander seid voll
Herzlichkeit; in der Ehrerbietung komme einer dem andern zuvor!
\bibleverse{11} Seid unverdrossen, wo es Eifer gilt; seid feurig im
Geist, dem Herrn zu dienen bereit! \bibleverse{12} Seid fröhlich in der
Hoffnung, geduldig im Leiden, beharrlich im Gebet! \bibleverse{13} Für
die Bedürfnisse der Heiligen beweist Anteilnahme; übt die
Gastfreundschaft willig. \bibleverse{14} Segnet, die euch verfolgen,
segnet sie und flucht ihnen nicht! \bibleverse{15} Freuet euch mit den
Fröhlichen und weinet mit den Weinenden! \bibleverse{16} Seid
einträchtig untereinander gesinnt; richtet eure Gedanken nicht auf hohe
Dinge, sondern laßt euch zu den niedrigen herab; haltet euch nicht
selbst für klug! \bibleverse{17} Vergeltet niemand Böses mit Bösem; seid
auf das bedacht, was in den Augen aller Menschen löblich ist!
\bibleverse{18} Ist's möglich, soviel an euch liegt, so lebt mit allen
Menschen in Frieden! \bibleverse{19} Rächet euch nicht selbst, Geliebte,
sondern gebt Raum\textless sup title=``=~überlaßt das''\textgreater✲ dem
(göttlichen) Zorn\textless sup title=``oder:
Strafgericht''\textgreater✲; denn es steht geschrieben\textless sup
title=``5.Mose 32,35''\textgreater✲: »Mein ist die Rache, ich will
vergelten, spricht der Herr.« \bibleverse{20} Vielmehr: »Wenn deinen
Feind hungert, so speise ihn; wenn ihn dürstet, so gib ihm zu trinken;
denn wenn du das tust, wirst du feurige Kohlen auf sein Haupt
sammeln.«\textless sup title=``Spr 25,21-22''\textgreater✲
\bibleverse{21} Laß dich nicht vom Bösen überwinden, sondern überwinde
das Böse durch das Gute!

\hypertarget{ermahnung-zum-gehorsam-gegen-die-gottgesetzte-obrigkeit}{%
\subsubsection{4. Ermahnung zum Gehorsam gegen die gottgesetzte
Obrigkeit}\label{ermahnung-zum-gehorsam-gegen-die-gottgesetzte-obrigkeit}}

\hypertarget{section-12}{%
\section{13}\label{section-12}}

\bibleverse{1} Jedermann sei den obrigkeitlichen Gewalten\textless sup
title=``oder: den vorgesetzten Obrigkeiten''\textgreater✲ untertan; denn
es gibt keine Obrigkeit, ohne von Gott (bestellt zu sein), und wo immer
eine besteht, ist sie von Gott verordnet. \bibleverse{2} Wer sich also
der Obrigkeit widersetzt, der lehnt sich damit gegen Gottes Ordnung auf;
und die sich auflehnen, werden sich selbst ein Strafurteil\textless sup
title=``=~ihre gerechte Strafe''\textgreater✲ zuziehen. \bibleverse{3}
Denn die obrigkeitlichen Personen sind nicht für die guten
Taten\textless sup title=``=~für die, welche recht
handeln''\textgreater✲ ein Schrecken, sondern für die bösen. Willst du
also frei von Furcht vor der Obrigkeit sein, so tu das Gute: dann wirst
du Anerkennung von ihr erhalten; \bibleverse{4} denn sie ist Gottes
Dienerin dir zum Guten\textless sup title=``=~zu deinem
Besten''\textgreater✲. Tust du aber das Böse, so fürchte dich; denn sie
trägt das Schwert nicht umsonst: sie ist ja Gottes Dienerin, eine
Vergelterin zur Vollziehung des (göttlichen) Zornes\textless sup
title=``oder: Strafgerichts''\textgreater✲ an dem Übeltäter.
\bibleverse{5} Darum muß man ihr untertan sein, und zwar nicht nur aus
Furcht vor dem (göttlichen) Zorn, sondern auch um des Gewissens willen.
\bibleverse{6} Deshalb entrichtet ihr ja auch Steuern; denn
sie\textless sup title=``d.h. die Beamten''\textgreater✲ sind Gottes
Dienstleute, die für eben diesen Zweck unablässig tätig sind.

\hypertarget{ermahnungen-zu-allseitiger-pflichterfuxfcllung-besonders-zur-nuxe4chstenliebe-als-der-erfuxfcllung-des-gesetzes}{%
\subsubsection{5. Ermahnungen zu allseitiger Pflichterfüllung, besonders
zur Nächstenliebe als der Erfüllung des
Gesetzes}\label{ermahnungen-zu-allseitiger-pflichterfuxfcllung-besonders-zur-nuxe4chstenliebe-als-der-erfuxfcllung-des-gesetzes}}

\bibleverse{7} Lasset allen zukommen, was ihr ihnen schuldig seid: die
Steuer, wem die Steuer gebührt, den Zoll, wem der Zoll zukommt, die
Furcht, wem die Furcht, und die Ehre, wem die Ehre gebührt.
\bibleverse{8} Bleibt niemand etwas schuldig, außer daß ihr einander
liebt; denn wer den anderen liebt, hat damit das Gesetz
erfüllt\textless sup title=``Gal 5,14''\textgreater✲. \bibleverse{9}
Denn das Gebot: »Du sollst nicht ehebrechen, nicht töten, nicht stehlen,
laß dich nicht gelüsten!« und jedes andere derartige Gebot ist in diesem
Wort einheitlich zusammengefaßt\textless sup title=``3.Mose
19,18''\textgreater✲: »Du sollst deinen Nächsten lieben wie dich
selbst!« \bibleverse{10} Die Liebe tut dem Nächsten nichts Böses;
demnach ist die Liebe die Erfüllung des Gesetzes.

\hypertarget{das-nahe-weltende-mahnt-zum-wandel-im-licht-und-zur-heiligung-des-persuxf6nlichen-lebens}{%
\subsubsection{6. Das nahe Weltende mahnt zum Wandel im Licht und zur
Heiligung des persönlichen
Lebens}\label{das-nahe-weltende-mahnt-zum-wandel-im-licht-und-zur-heiligung-des-persuxf6nlichen-lebens}}

\bibleverse{11} Und zwar (verhaltet euch auf diese Weise) in richtiger
Erkenntnis der (gegenwärtigen) Zeit, daß nämlich die Stunde nunmehr für
uns da ist, aus dem Schlaf zu erwachen; denn jetzt ist die Rettung uns
näher als damals, als wir zum Glauben gekommen sind: \bibleverse{12} die
Nacht ist vorgerückt und der Tag nahegekommen. So lasset uns denn die
Werke der Finsternis abtun, dagegen die Waffen des Lichts anlegen!
\bibleverse{13} Lasset uns sittsam wandeln, wie es sich am Tage geziemt:
nicht in Schwelgereien und Trinkgelagen, nicht in Unzucht und
Ausschweifungen, nicht in Streit und Eifersucht; \bibleverse{14} nein,
ziehet den Herrn Jesus Christus an, und seid dem Fleisch\textless sup
title=``=~dem Leibe''\textgreater✲ nicht so zu Diensten, daß böse
Begierden dadurch erregt werden!

\hypertarget{besondere-ermahnungen-fuxfcr-das-verhalten-der-glaubensstarken-besonders-der-freieren-heidenchristen-und-der-glaubensschwachen-besonders-der-uxe4ngstlichen-judenchristen}{%
\subsubsection{7. Besondere Ermahnungen für das Verhalten der
Glaubensstarken (besonders der freieren Heidenchristen) und der
Glaubensschwachen (besonders der ängstlichen
Judenchristen)}\label{besondere-ermahnungen-fuxfcr-das-verhalten-der-glaubensstarken-besonders-der-freieren-heidenchristen-und-der-glaubensschwachen-besonders-der-uxe4ngstlichen-judenchristen}}

\hypertarget{a-beurteilung-der-die-gemeinde-bewegenden-streitfrage-und-warnung-vor-liebloser-verurteilung-der-uxe4uuxdferen-lebensfuxfchrung-des-nuxe4chsten}{%
\paragraph{a) Beurteilung der die Gemeinde bewegenden Streitfrage und
Warnung vor liebloser Verurteilung der äußeren Lebensführung des
Nächsten}\label{a-beurteilung-der-die-gemeinde-bewegenden-streitfrage-und-warnung-vor-liebloser-verurteilung-der-uxe4uuxdferen-lebensfuxfchrung-des-nuxe4chsten}}

\hypertarget{section-13}{%
\section{14}\label{section-13}}

\bibleverse{1} Auf den im Glauben Schwachen nehmet (liebevolle)
Rücksicht, ohne über Gewissensbedenken (mit ihm) zu streiten.
\bibleverse{2} Der eine ist überzeugt, alles essen zu dürfen, während
der Schwache nur Pflanzenkost genießt. \bibleverse{3} Wer (alles) ißt,
verachte den nicht, der nicht (alles) ißt; und wer nicht (alles) ißt,
soll über den, der (alles) ißt, nicht zu Gericht sitzen, denn Gott hat
ihn (als Angehörigen) angenommen. \bibleverse{4} Wie kommst du dazu,
dich zum Richter über den Knecht\textless sup title=``oder:
Diener''\textgreater✲ eines andern zu machen? Er steht oder fällt seinem
eigenen Herrn; und zwar wird er stehen bleiben, denn sein Herr ist stark
genug, ihn aufrecht zu halten. \bibleverse{5} Mancher macht einen
Unterschied zwischen den Tagen, während einem andern alle Tage als
gleich gelten: ein jeder möge nach seiner eigenen Denkweise zu einer
festen Überzeugung kommen! \bibleverse{6} Wer etwas auf einzelne Tage
gibt, der tut es für den Herrn\textless sup title=``=~um dem Herrn zu
dienen''\textgreater✲; und wer (alles) ißt, tut es für den Herrn, denn
er sagt Gott Dank dabei; und wer nicht (alles) ißt, tut es auch für den
Herrn und sagt Gott Dank dabei. \bibleverse{7} Keiner von uns lebt ja
für sich selbst\textless sup title=``=~gehört im Leben sich selbst
an''\textgreater✲, und keiner stirbt für sich selbst\textless sup
title=``=~gehört im Sterben sich selbst an''\textgreater✲;
\bibleverse{8} denn leben wir, so leben wir dem Herrn, und sterben wir,
so sterben wir dem Herrn; darum, mögen wir leben oder sterben, so
gehören wir dem Herrn als Eigentum an. \bibleverse{9} Dazu ist ja
Christus gestorben und wieder lebendig geworden, um sowohl über Tote als
auch über Lebende Herr zu sein. \bibleverse{10} Du aber: wie kannst du
dich zum Richter über deinen Bruder machen? Oder auch du: wie darfst du
deinen Bruder verachten? Wir werden ja alle (einmal) vor den
Richterstuhl Gottes treten müssen; \bibleverse{11} denn es steht
geschrieben\textless sup title=``Jes 45,23''\textgreater✲: »So wahr ich
lebe«, spricht der Herr, »mir (zu Ehren) wird jedes Knie sich beugen,
und jede Zunge wird Gott bekennen\textless sup title=``=~preisen, oder:
huldigen''\textgreater✲.« \bibleverse{12} Demnach wird ein jeder von uns
über\textless sup title=``oder: für''\textgreater✲ sich selbst
Rechenschaft vor Gott abzulegen haben.

\hypertarget{b-ermahnung-an-die-glaubensstarken-den-glaubensschwachen-kein-uxe4rgernis-zu-geben-und-bei-allem-tun-nach-gewissenszuversicht-zu-streben}{%
\paragraph{b) Ermahnung an die Glaubensstarken, den Glaubensschwachen
kein Ärgernis zu geben und bei allem Tun nach Gewissenszuversicht zu
streben}\label{b-ermahnung-an-die-glaubensstarken-den-glaubensschwachen-kein-uxe4rgernis-zu-geben-und-bei-allem-tun-nach-gewissenszuversicht-zu-streben}}

\bibleverse{13} Darum wollen wir nicht mehr einer den andern richten,
sondern haltet vielmehr das für das Richtige, dem Bruder keinen Anstoß
und kein Ärgernis zu bereiten! \bibleverse{14} Ich weiß und bin dessen
im Herrn Jesus gewiß, daß nichts an und für sich unrein\textless sup
title=``oder: verunreinigend''\textgreater✲ ist; jedoch wenn jemand
etwas für unrein hält, so ist es für ihn (tatsächlich) unrein.
\bibleverse{15} Denn wenn dein Bruder (durch dich) um einer Speise
willen in Betrübnis versetzt wird, so wandelst du nicht mehr nach (dem
Gebot) der Liebe. Bringe durch dein Essen nicht den ins Verderben, für
den Christus gestorben ist! \bibleverse{16} Verschuldet es also nicht,
daß euer Heilsgut der Verlästerung anheimfällt! \bibleverse{17} Das
Reich Gottes besteht ja nicht in Essen und Trinken, sondern in
Gerechtigkeit und Frieden und Freude im heiligen Geist; \bibleverse{18}
denn wer darin Christus dient, der ist Gott wohlgefällig und vor den
Menschen bewährt\textless sup title=``oder: von ihnen
wertgeschätzt''\textgreater✲.

\bibleverse{19} Darum wollen wir auf das bedacht sein, was zum Frieden
und zu gegenseitiger Erbauung dient! \bibleverse{20} Zerstöre nicht um
einer Speise willen das Werk Gottes! Zwar ist alles rein, aber zum
Unheil ist es für jemand, der es mit✲ inneren Bedenken genießt;
\bibleverse{21} da ist es löblich✲, kein Fleisch zu essen und keinen
Wein zu trinken, überhaupt nichts (zu tun), woran dein Bruder Anstoß
nimmt. \bibleverse{22} Du hast Glaubenszuversicht: halte sie für dich
selbst vor dem Angesicht Gottes fest! Wohl dem, der nicht mit sich
selbst ins Gericht zu gehen\textless sup title=``=~sich keinen Vorwurf
zu machen''\textgreater✲ braucht bei dem, was er für recht\textless sup
title=``oder: gut''\textgreater✲ hält! \bibleverse{23} Wer dagegen ißt,
obwohl er Bedenken hegt, der hat sich dadurch die Verurteilung
zugezogen, weil (er) nicht aus Glauben (gehandelt hat); alles aber, was
nicht aus Glauben geschieht, ist Sünde.

\hypertarget{c-ermahnung-zur-geduld-mit-den-schwachen-und-zu-christlicher-eintracht-nach-dem-vorbild-christi}{%
\paragraph{c) Ermahnung zur Geduld mit den Schwachen und zu christlicher
Eintracht nach dem Vorbild
Christi}\label{c-ermahnung-zur-geduld-mit-den-schwachen-und-zu-christlicher-eintracht-nach-dem-vorbild-christi}}

\hypertarget{section-14}{%
\section{15}\label{section-14}}

\bibleverse{1} Da haben wir Starken die Pflicht, die Schwächen derer,
die nicht so stark sind (wie wir), zu tragen und nicht wohlgefällig an
uns selbst zu denken: \bibleverse{2} nein, jeder von uns lebe dem
Nächsten zu Gefallen, ihm zum Besten, zu seiner Erbauung✲!
\bibleverse{3} Denn auch Christus hat nicht sich selbst zu Gefallen
gelebt, sondern wie geschrieben steht\textless sup title=``Ps
69,10''\textgreater✲: »Die Schmähungen derer, die dich schmähen, sind
auf mich gefallen\textless sup title=``=~haben mich
getroffen''\textgreater✲.« \bibleverse{4} So ist ja alles, was vor
Zeiten geschrieben worden ist, für uns zur Belehrung geschrieben, damit
wir durch standhaftes Ausharren\textless sup title=``oder:
Geduld''\textgreater✲ und durch den Trost, den die (heiligen) Schriften
gewähren, an der Hoffnung festhalten. \bibleverse{5} Der Gott aber, von
dem standhaftes Ausharren\textless sup title=``oder:
Geduld''\textgreater✲ und Trost kommen, möge euch dazu verhelfen, einen
einträchtigen Sinn untereinander nach der Weise\textless sup
title=``oder: dem Vorbilde''\textgreater✲ Christi Jesu zu besitzen,
\bibleverse{6} damit ihr einmütig mit einem Munde den Gott und Vater
unsers Herrn Jesus Christus preisen könnt.

\hypertarget{d-mahnung-an-beide-teile-der-gemeinde-zu-eintruxe4chtiger-gemeinschaft-und-freudigem-glauben}{%
\paragraph{d) Mahnung an beide Teile der Gemeinde zu einträchtiger
Gemeinschaft und freudigem
Glauben}\label{d-mahnung-an-beide-teile-der-gemeinde-zu-eintruxe4chtiger-gemeinschaft-und-freudigem-glauben}}

\bibleverse{7} Darum nehmet euch gegenseitig (in Liebe) an\textless sup
title=``oder: auf''\textgreater✲, wie auch Christus euch zu Gottes
Verherrlichung\textless sup title=``oder: Ehre''\textgreater✲ (in Liebe)
angenommen\textless sup title=``oder: aufgenommen''\textgreater✲ hat!
\bibleverse{8} Ich meine nämlich: Christus ist ein Diener der
Beschneidung\textless sup title=``=~der Juden''\textgreater✲ geworden
zum Erweis der Wahrhaftigkeit Gottes, um die den Vätern gegebenen
Verheißungen zu verwirklichen, \bibleverse{9} die Heiden andrerseits
sollen\textless sup title=``oder: müssen''\textgreater✲ Gott um seiner
Barmherzigkeit willen preisen, wie geschrieben steht\textless sup
title=``Ps 18,50''\textgreater✲: »Darum will ich dich preisen unter den
Heiden und deinem Namen lobsingen.« \bibleverse{10} Und an einer anderen
Stelle heißt es\textless sup title=``5.Mose 32,43''\textgreater✲:
»Freuet euch, ihr Heiden, im Verein mit seinem Volke!« \bibleverse{11}
und an einer anderen Stelle\textless sup title=``Ps
117,1''\textgreater✲: »Lobet, ihr Heiden alle, den Herrn, und alle
Völker sollen ihn preisen!« \bibleverse{12} Weiter sagt
Jesaja\textless sup title=``Jes 11,10''\textgreater✲: »Erscheinen wird
der Wurzelschoß✲ Isais, und zwar er, der da aufsteht, um über die Heiden
zu herrschen: auf ihn werden die Heiden ihre Hoffnung setzen.«
\bibleverse{13} Der Gott aber, der unsere Hoffnung ist, erfülle euch mit
aller Freude und mit Frieden auf dem Grunde des Glaubens, damit ihr
immer reicher an Hoffnung werdet durch die Kraft des heiligen Geistes!

\hypertarget{iv.-schluuxdf-des-briefes-persuxf6nliche-mitteilungen-gruxfcuxdfe-und-letzte-mahnung-1514-1627}{%
\subsection{IV. Schluß des Briefes; persönliche Mitteilungen, Grüße und
letzte Mahnung
(15,14-16,27)}\label{iv.-schluuxdf-des-briefes-persuxf6nliche-mitteilungen-gruxfcuxdfe-und-letzte-mahnung-1514-1627}}

\hypertarget{rechtfertigender-ruxfcckblick-des-apostels-auf-den-brief-und-hinweis-auf-sein-apostelamt-fuxfcr-die-heiden}{%
\subsubsection{1. Rechtfertigender Rückblick des Apostels auf den Brief
und Hinweis auf sein Apostelamt für die
Heiden}\label{rechtfertigender-ruxfcckblick-des-apostels-auf-den-brief-und-hinweis-auf-sein-apostelamt-fuxfcr-die-heiden}}

\bibleverse{14} Ich habe aber auch persönlich von euch, liebe Brüder,
die feste Überzeugung, daß ihr eurerseits mit dem besten Willen erfüllt
und, mit aller Erkenntnis voll ausgerüstet, wohlbefähigt seid, euch auch
untereinander zurechtzuweisen. \bibleverse{15} Trotzdem habe ich euch,
wenigstens stellenweise, etwas rückhaltlos geschrieben, um euch (an dies
und das) zu erinnern, und zwar aufgrund des mir von Gott verliehenen
Gnadenamtes. \bibleverse{16} Ich soll ja ein Diener Christi Jesu für die
Heiden sein und als solcher den Priesterdienst an der Heilsbotschaft
Gottes verrichten, damit die Heiden zu einer gottwohlgefälligen, durch
den heiligen Geist geheiligten Opfergabe werden. \bibleverse{17} In
Christus Jesus darf ich mich daher meines für die Sache Gottes
geleisteten Dienstes rühmen; \bibleverse{18} denn ich werde mich nicht
erkühnen, von irgendwelchen Erfolgen zu reden, die nicht Christus durch
mich gewirkt\textless sup title=``oder: wirklich
vollbracht''\textgreater✲ hat, um Heiden(völker) zum
Gehorsam\textless sup title=``=~zur Bekehrung''\textgreater✲ zu bringen
durch Wort und Tat, \bibleverse{19} durch die Kraft von Zeichen und
Wundern, durch die Kraft des heiligen Geistes. Auf diese Weise habe ich
nämlich von Jerusalem aus und in weitem Umkreise bis nach Illyrikum hin
die (Verkündigung der) Heilsbotschaft von Christus voll ausgerichtet.
\bibleverse{20} Dabei habe ich es mir aber zur Ehrensache gemacht, die
Heilsbotschaft nicht da zu verkündigen, wo der Name Christi bereits
bekannt war; denn ich habe nicht auf fremdem Grund und Boden (weiter)
bauen wollen, \bibleverse{21} sondern bin dem Schriftwort
gefolgt\textless sup title=``Jes 52,15''\textgreater✲: »Die sollen (ihn)
zu sehen bekommen, denen noch nichts über ihn verkündigt worden ist, und
die noch keine Kunde (von ihm) haben, die sollen (ihn) kennenlernen.«

\hypertarget{mitteilung-der-nuxe4chsten-reisepluxe4ne-des-apostels}{%
\subsubsection{2. Mitteilung der nächsten Reisepläne des
Apostels}\label{mitteilung-der-nuxe4chsten-reisepluxe4ne-des-apostels}}

\bibleverse{22} Das ist denn auch der Grund, weshalb ich so oft
verhindert worden bin, zu euch zu kommen. \bibleverse{23} Da ich jetzt
aber in den Gegenden hier kein Arbeitsfeld mehr habe\textless sup
title=``=~nicht mehr nötig bin''\textgreater✲, wohl aber seit vielen
Jahren mich danach sehne, zu euch zu kommen, \bibleverse{24} so werde
ich, sobald ich die Reise nach Spanien unternehme (meinen Plan
ausführen). Ich hoffe nämlich, euch auf der Durchreise zu sehen und von
euch das Geleit zur Weiterreise dorthin zu erhalten, nachdem ich mich
zunächst ein wenig bei euch\textless sup title=``oder: an
euch''\textgreater✲ erquickt habe. \bibleverse{25} Augenblicklich aber
befinde ich mich auf der Reise nach Jerusalem anläßlich eines
Liebesdienstes für die Heiligen. \bibleverse{26} Mazedonien und Achaja✲
haben nämlich beschlossen, eine Geldsammlung für die Armen unter den
Heiligen in Jerusalem zu veranstalten. \bibleverse{27} Ja, sie haben es
beschlossen und sind es ihnen ja auch schuldig; denn wenn die
Heiden(christen) Anteil an den geistlichen Gütern jener erhalten haben,
so sind sie dafür auch verpflichtet, ihnen mit ihren
irdischen\textless sup title=``oder: weltlichen''\textgreater✲ Gütern
auszuhelfen. \bibleverse{28} Wenn ich nun dieses Geschäft erledigt und
ihnen den Ertrag dieser Sammlung sicher übermittelt habe, dann werde ich
über euer Rom nach Spanien reisen. \bibleverse{29} Ich weiß aber, daß,
wenn ich zu euch komme, ich euch eine Fülle des Segens Christi
mitbringen werde.

\hypertarget{des-apostels-mahnung-an-die-gemeinde-zur-fuxfcrbitte-fuxfcr-ihn}{%
\subsubsection{3. Des Apostels Mahnung an die Gemeinde zur Fürbitte für
ihn}\label{des-apostels-mahnung-an-die-gemeinde-zur-fuxfcrbitte-fuxfcr-ihn}}

\bibleverse{30} Ich bitte euch aber dringend, liebe Brüder, bei unserm
Herrn Jesus Christus und bei der Liebe, die der (heilige) Geist wirkt:
steht mir mit den Gebeten, die ihr für mich an Gott richtet, im Kampfe
kräftig bei, \bibleverse{31} damit ich von\textless sup title=``oder:
vor''\textgreater✲ den Ungehorsamen in Judäa errettet werde und meine
Dienstleistung für Jerusalem bei den Heiligen dort eine gute Aufnahme
finden möge! \bibleverse{32} Dann kann ich, so Gott will, in freudiger
Stimmung zu euch kommen und mich im Zusammensein mit euch erquicken.
\bibleverse{33} Der Gott des Friedens aber sei mit euch allen! Amen.

\hypertarget{empfehlung-der-phuxf6be-der-uxfcberbringerin-des-briefes-gruxfcuxdfe-des-apostels-an-glaubensgenossen-in-rom}{%
\subsubsection{4. Empfehlung der Phöbe, der Überbringerin des Briefes;
Grüße des Apostels an Glaubensgenossen in
Rom}\label{empfehlung-der-phuxf6be-der-uxfcberbringerin-des-briefes-gruxfcuxdfe-des-apostels-an-glaubensgenossen-in-rom}}

\hypertarget{section-15}{%
\section{16}\label{section-15}}

\bibleverse{1} Ich empfehle euch aber unsere Schwester Phöbe, die im
Dienst der Gemeinde zu Kenchreä steht: \bibleverse{2} nehmt sie im Herrn
auf, wie es sich für Heilige geziemt, und steht ihr in allen Fällen, wo
sie euer bedarf, hilfreich zur Seite; denn sie hat gleichfalls vielen
Beistand geleistet, auch mir persönlich.

\bibleverse{3} Grüßt Priska und Aquila, meine Mitarbeiter in Christus
Jesus, \bibleverse{4} die für mein Leben ihren eigenen Hals✲ eingesetzt
haben, wofür nicht ich allein ihnen zu Dank verpflichtet bin, sondern
auch sämtliche heidenchristlichen Gemeinden; \bibleverse{5} grüßt auch
die Gemeinde in ihrem Hause. Grüßt meinen geliebten Epänetus, der die
Erstlingsgabe Asiens für Christus ist. \bibleverse{6} Grüßt Maria, die
sich hingebend für euch gemüht hat. \bibleverse{7} Grüßt Andronikus und
Junias, meine Volksgenossen und (einst) meine Mitgefangenen, die bei den
Aposteln in hohem Ansehen stehen und auch schon vor mir in Christus✲
gewesen sind. \bibleverse{8} Grüßt meinen im Herrn geliebten Ampliatus.
\bibleverse{9} Grüßt Urbanus, unsern Mitarbeiter in Christus, und meinen
geliebten Stachys. \bibleverse{10} Grüßt Apelles, der ein bewährter
Jünger Christi ist. Grüßt die Brüder unter den Leuten des Aristobulus.
\bibleverse{11} Grüßt meinen Volksgenossen Herodion. Grüßt diejenigen
unter den Leuten des Narcissus, die im Herrn✲ sind. \bibleverse{12}
Grüßt Tryphäna und Tryphosa, die eifrig im Herrn arbeiten. Grüßt die
geliebte Persis, die eine treue Arbeiterin im Herrn gewesen ist.
\bibleverse{13} Grüßt den im Herrn auserwählten Rufus und seine Mutter,
die auch mir eine Mutter gewesen ist. \bibleverse{14} Grüßt Asynkritus,
Phlegon, Hermes, Patrobas, Hermas und die (anderen) Brüder bei ihnen.
\bibleverse{15} Grüßt Philologus und Julias, Nereus nebst seiner
Schwester, auch Olympas und alle Heiligen bei ihnen. \bibleverse{16}
Grüßt einander mit dem heiligen Kuß. Alle Gemeinden Christi lassen euch
grüßen.

\hypertarget{warnung-vor-verfuxfchrern-welche-spaltungen-und-irrungen-in-der-gemeinde-anrichten}{%
\subsubsection{5. Warnung vor Verführern, welche Spaltungen und Irrungen
in der Gemeinde
anrichten}\label{warnung-vor-verfuxfchrern-welche-spaltungen-und-irrungen-in-der-gemeinde-anrichten}}

\bibleverse{17} Ich ermahne euch aber, liebe Brüder, auf der Hut vor
denen zu sein, welche die Spaltungen und Ärgernisse erregen im Gegensatz
zu der Lehre, in der ihr unterwiesen worden seid: geht ihnen aus dem
Wege\textless sup title=``oder: zieht euch von ihnen
zurück''\textgreater✲; \bibleverse{18} denn solche Menschen dienen nicht
unserm Herrn Christus, sondern ihrem Bauche und betören durch ihre schön
klingenden Reden und glatten\textless sup title=``oder:
salbungsvollen''\textgreater✲ Worte die Herzen der Arglosen.
\bibleverse{19} Die Kunde von eurem (Glaubens-) Gehorsam ist ja zu allen
gedrungen. Deshalb habe ich meine Freude an euch, wünsche aber, daß ihr
weise seid, wo es das Gute gilt, dagegen einfältig gegenüber dem
Bösen\textless sup title=``=~der Verführung zum Bösen
unzugänglich''\textgreater✲. \bibleverse{20} Der Gott des Friedens aber
wird den Satan unter euren Füßen zertreten, und zwar in Bälde. Die Gnade
unsers Herrn Jesus sei mit euch!

\hypertarget{gruxfcuxdfe-von-freunden-des-paulus-nach-rom-und-abschlieuxdfender-lobpreis-gottes}{%
\subsubsection{6. Grüße von Freunden des Paulus nach Rom und
abschließender Lobpreis
Gottes}\label{gruxfcuxdfe-von-freunden-des-paulus-nach-rom-und-abschlieuxdfender-lobpreis-gottes}}

\bibleverse{21} Es grüßen euch mein Mitarbeiter Timotheus und meine
Volksgenossen Lucius, Jason und Sosipater. \bibleverse{22} Ich, Tertius,
der Schreiber dieses Briefes, grüße euch im Herrn. \bibleverse{23} Es
grüßt euch Gaius, dessen Gastfreundschaft ich und die ganze Gemeinde
genießen. Es grüßen euch der Stadtkämmerer Erastus und der Bruder
Quartus. \bibleverse{24} Die Gnade unsers Herrn Jesus Christus sei mit
euch allen! Amen. \bibleverse{25} Ihm aber, der die Kraft hat, euch (im
Glauben) fest zu machen nach✲ meiner Heilsverkündigung und der Predigt
von Jesus Christus nach✲ der Offenbarung des Geheimnisses, das ewige
Zeiten hindurch verschwiegen geblieben, \bibleverse{26} jetzt aber
bekanntgegeben und auch durch prophetische Schriften nach dem
Auftrage\textless sup title=``oder: Befehl''\textgreater✲ des ewigen
Gottes bei allen Heidenvölkern verkündigt worden ist, um
Glaubensgehorsam (bei ihnen) zu wirken: \bibleverse{27} ihm, dem allein
weisen Gott, sei durch Jesus Christus die Herrlichkeit\textless sup
title=``oder: Ehre''\textgreater✲ in alle Ewigkeit! Amen.
