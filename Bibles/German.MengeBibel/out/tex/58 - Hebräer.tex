\hypertarget{der-brief-an-die-hebruxe4er}{%
\section{DER BRIEF AN DIE HEBRÄER}\label{der-brief-an-die-hebruxe4er}}

\hypertarget{i.-die-gruxf6uxdfe-des-gottessohnes-und-die-bedrohlichen-folgen-des-ungehorsams-gegen-sein-wort-11-413}{%
\subsection{I. Die Größe des Gottessohnes und die bedrohlichen Folgen
des Ungehorsams gegen sein Wort
(1,1-4,13)}\label{i.-die-gruxf6uxdfe-des-gottessohnes-und-die-bedrohlichen-folgen-des-ungehorsams-gegen-sein-wort-11-413}}

\hypertarget{die-einzigartige-hoheit-des-gottessohnes-gegenuxfcber-den-alttestamentlichen-gottesboten}{%
\subsubsection{1. Die einzigartige Hoheit des Gottessohnes gegenüber den
alttestamentlichen
Gottesboten}\label{die-einzigartige-hoheit-des-gottessohnes-gegenuxfcber-den-alttestamentlichen-gottesboten}}

\hypertarget{a-allgemeine-darlegung-dieser-hoheit}{%
\paragraph{a) Allgemeine Darlegung dieser
Hoheit}\label{a-allgemeine-darlegung-dieser-hoheit}}

\hypertarget{section}{%
\section{1}\label{section}}

\bibleverse{1} Nachdem Gott vorzeiten vielfältig\textless sup
title=``=~zu vielen Malen''\textgreater✲ und auf vielerlei Weise zu den
Vätern geredet hat in den Propheten, \bibleverse{2} hat er am Ende
dieser Tage\textless sup title=``d.h. in dieser Endzeit''\textgreater✲
zu uns geredet im Sohn, den er zum Erben von allem eingesetzt✲, durch
den er auch die Weltzeiten\textless sup title=``oder:
Welten''\textgreater✲ geschaffen hat. \bibleverse{3} Dieser ist der
Abglanz seiner Herrlichkeit und die Ausprägung\textless sup
title=``=~der Abdruck, oder: das Ebenbild''\textgreater✲ seines Wesens
und trägt das Weltall durch sein Allmachtswort; er hat sich, nachdem er
die Reinigung von den Sünden vollbracht hat, zur Rechten der
Erhabenheit\textless sup title=``=~der Majestät Gottes''\textgreater✲ in
den Himmelshöhen gesetzt \bibleverse{4} und ist dadurch um so viel
größer✲ geworden als die Engel, wie der Name, den er als Erbteil
erhalten hat, den ihrigen überragt.

\hypertarget{b-alttestamentliche-belege-fuxfcr-die-erhabenheit-des-gottessohnes-uxfcber-die-engel}{%
\paragraph{b) Alttestamentliche Belege für die Erhabenheit des
Gottessohnes über die
Engel}\label{b-alttestamentliche-belege-fuxfcr-die-erhabenheit-des-gottessohnes-uxfcber-die-engel}}

\bibleverse{5} Denn zu welchem von den Engeln hätte Gott jemals
gesagt\textless sup title=``Ps 2,7''\textgreater✲: »Mein Sohn bist du:
ich habe dich heute gezeugt«? Und ein andermal\textless sup
title=``2.Sam 7,14''\textgreater✲: »Ich will ihm Vater sein, und er soll
für mich Sohn sein?« \bibleverse{6} Weiter sagt er von der Zeit, in
welcher er den Erstgeborenen wiederum\textless sup title=``=~zum
zweitenmal''\textgreater✲ in die Menschenwelt einführen
wird\textless sup title=``Ps 97,7''\textgreater✲: »Alle Engel Gottes
sollen vor ihm huldigend sich neigen\textless sup title=``oder: anbetend
niederfallen''\textgreater✲.« \bibleverse{7} Und in bezug auf die Engel
heißt es zwar\textless sup title=``Ps 104,4''\textgreater✲: »Er macht
seine Engel zu Winden und seine Diener zur Feuerflamme«, \bibleverse{8}
aber in bezug auf den Sohn\textless sup title=``Ps
45,7-8''\textgreater✲: »Dein Thron, o Gott, steht fest in alle Ewigkeit,
und der Stab\textless sup title=``=~das Zepter''\textgreater✲ der
Geradheit ist der Stab deiner Königsherrschaft. \bibleverse{9} Du hast
Gerechtigkeit geliebt und Gesetzwidrigkeit gehaßt; darum hat dich, o
Gott, dein Gott mit Freudenöl gesalbt vor deinen Genossen\textless sup
title=``=~wie keinen deinesgleichen''\textgreater✲.« \bibleverse{10} Und
ferner\textless sup title=``Ps 102,26-28''\textgreater✲: »Du hast im
Anfang, Herr, die Erde gegründet, und die Himmel sind deiner Hände Werk;
\bibleverse{11} sie werden vergehen, du aber bleibst, und sie werden
alle veralten wie ein Gewand; \bibleverse{12} wie einen Mantel wirst du
sie zusammenrollen, und sie werden verwandelt werden; du aber bleibst
derselbe, und deine Jahre werden kein Ende nehmen.« \bibleverse{13} Zu
welchem Engel hätte er ferner jemals gesagt\textless sup title=``Ps
110,1''\textgreater✲: »Setze dich zu meiner Rechten, bis ich deine
Feinde hinlege zum Schemel deiner Füße«? \bibleverse{14} Sind sie nicht
allesamt (nur) dienstbare Geister, die zu Dienstleistungen ausgesandt
werden um derer willen, welche die Rettung\textless sup title=``oder:
das Heil''\textgreater✲ ererben sollen?

\hypertarget{c-daraus-ergibt-sich-fuxfcr-uns-die-verpflichtung-den-durch-diesen-sohn-zu-uns-geredeten-worten-willigen-gehorsam-entgegenzubringen}{%
\paragraph{c) Daraus ergibt sich für uns die Verpflichtung, den durch
diesen Sohn zu uns geredeten Worten willigen Gehorsam
entgegenzubringen}\label{c-daraus-ergibt-sich-fuxfcr-uns-die-verpflichtung-den-durch-diesen-sohn-zu-uns-geredeten-worten-willigen-gehorsam-entgegenzubringen}}

\hypertarget{section-1}{%
\section{2}\label{section-1}}

\bibleverse{1} Darum müssen wir uns um so fester an das halten, was wir
gehört haben, um seiner ja nicht verlustig zu gehen. \bibleverse{2} Denn
wenn schon das durch Vermittlung von Engeln verkündete Wort\textless sup
title=``=~das Gesetz; vgl. Apg 7,38.53; Gal 3,19''\textgreater✲
unverbrüchlich war und jede Übertretung und jeder Ungehorsam die
gebührende Vergeltung empfing✲: \bibleverse{3} wie sollten wir da (der
Strafe) entrinnen, wenn wir ein so hohes Heil unbeachtet lassen? Dieses
hat ja seinen Anfang von der Verkündigung durch den Herrn (selbst)
genommen und ist uns dann von den Ohrenzeugen zuverlässig bestätigt
worden, \bibleverse{4} wobei auch Gott noch Zeugnis dafür abgelegt hat
durch Zeichen und Wunder, durch mannigfache Krafttaten✲ und Austeilungen
des heiligen Geistes, nach seinem Ermessen\textless sup title=``=~wie es
ihm wohlgefällig war''\textgreater✲.

\hypertarget{die-der-erhabenheit-des-gottessohnes-scheinbar-widersprechende-erniedrigung-christi-unter-die-engel}{%
\subsubsection{2. Die der Erhabenheit des Gottessohnes scheinbar
widersprechende Erniedrigung Christi unter die
Engel}\label{die-der-erhabenheit-des-gottessohnes-scheinbar-widersprechende-erniedrigung-christi-unter-die-engel}}

\hypertarget{a-seine-erniedrigung-menschwerdung-und-todesleiden-beschruxe4nkt-seine-erhabenheit-nicht}{%
\paragraph{a) Seine Erniedrigung (Menschwerdung und Todesleiden)
beschränkt seine Erhabenheit
nicht}\label{a-seine-erniedrigung-menschwerdung-und-todesleiden-beschruxe4nkt-seine-erhabenheit-nicht}}

\bibleverse{5} Denn nicht Engeln hat er\textless sup title=``d.h.
Gott''\textgreater✲ die zukünftige Welt, von der wir hier reden,
unterstellt, \bibleverse{6} vielmehr hat jemand an einer Stelle
ausdrücklich bezeugt\textless sup title=``Ps 8,5-7''\textgreater✲: »Was
ist der Mensch, daß du seiner gedenkst, oder des Menschen Sohn, daß du
ihn beachtest? \bibleverse{7} Du hast ihn für eine kurze Zeit unter die
Engel erniedrigt, ihn (dann aber) mit Herrlichkeit und Ehre gekrönt;
\bibleverse{8} alles hast du ihm unter die Füße unterworfen.« Dadurch
nämlich, daß er »ihm alles unterworfen hat«, hat er nichts von der
Unterwerfung unter ihn ausgenommen. Bisher nehmen wir allerdings noch
nicht wahr, daß ihm alles\textless sup title=``oder: das
All''\textgreater✲ unterworfen ist; \bibleverse{9} wohl aber sehen wir
den, der für eine kurze Zeit unter die Engel erniedrigt gewesen ist,
nämlich Jesus, um seines Todesleidens willen mit Herrlichkeit und Ehre
gekrönt; er sollte ja durch Gottes Gnade für jeden\textless sup
title=``=~zum Besten eines jeden''\textgreater✲ den Tod schmecken.

\hypertarget{b-die-notwendigkeit-der-erniedrigung-besonders-des-todesleidens}{%
\paragraph{b) Die Notwendigkeit der Erniedrigung (besonders des
Todesleidens)}\label{b-die-notwendigkeit-der-erniedrigung-besonders-des-todesleidens}}

\bibleverse{10} Denn es geziemte ihm, um dessen willen alles ist und
durch den alles ist, nachdem er viele Söhne zur Herrlichkeit geführt
hatte, den Urheber\textless sup title=``vgl. 12,2''\textgreater✲ ihrer
Rettung\textless sup title=``oder: ihres Heils''\textgreater✲ durch
Leiden hindurch zur Vollendung zu bringen. \bibleverse{11} Denn beide,
sowohl der Heiligende\textless sup title=``Joh 17,19''\textgreater✲ als
auch die, welche (von ihm) geheiligt werden, (kommen =~stammen) alle von
dem gleichen Vater her; aus diesem Grunde schämt er sich auch nicht, sie
»Brüder« zu nennen, \bibleverse{12} indem er sagt\textless sup
title=``Ps 22,23''\textgreater✲: »Ich will deinen Namen meinen Brüdern
verkündigen, inmitten der Gemeinde will ich dich preisen«;
\bibleverse{13} und an einer andern Stelle\textless sup title=``Jes
8,17''\textgreater✲: »Ich will mein Vertrauen auf ihn setzen«; und
wiederum\textless sup title=``Jes 8,18''\textgreater✲: »Siehe, hier bin
ich und die Kinder, die Gott mir gegeben hat.«

\hypertarget{c-die-segensreichen-folgen-der-erniedrigung}{%
\paragraph{c) Die segensreichen Folgen der
Erniedrigung}\label{c-die-segensreichen-folgen-der-erniedrigung}}

\bibleverse{14} Weil nun die Kinder (leiblich) am Blut und Fleisch
Anteil haben, hat auch er gleichermaßen Anteil an ihnen erhalten, um
durch seinen Tod den zu vernichten, der die Macht des Todes\textless sup
title=``oder: Gewalt über den Tod''\textgreater✲ hat, nämlich den
Teufel, \bibleverse{15} und um alle die in Freiheit zu setzen, die durch
Furcht vor dem Tode während ihres ganzen Lebens in Knechtschaft gehalten
wurden. \bibleverse{16} Denn es sind doch sicherlich nicht Engel, deren
er sich anzunehmen hat, sondern der Nachkommenschaft Abrahams nimmt er
sich an; \bibleverse{17} und daher mußte er in allen Stücken seinen
Brüdern gleich werden, damit er barmherzig würde und ein treuer
Hoherpriester Gott gegenüber\textless sup title=``=~im Dienst vor
Gott''\textgreater✲, um für die Sünden des Volkes Vergebung zu erwirken.
\bibleverse{18} Denn eben deshalb, weil er selbst Versuchung erlitten
hat, vermag er denen zu helfen, die versucht werden.

\hypertarget{der-gottessohn-jesus-in-seiner-erhabenheit-uxfcber-den-gottesdiener-mose}{%
\subsubsection{3. Der Gottessohn Jesus in seiner Erhabenheit über den
Gottesdiener
Mose}\label{der-gottessohn-jesus-in-seiner-erhabenheit-uxfcber-den-gottesdiener-mose}}

\hypertarget{section-2}{%
\section{3}\label{section-2}}

\bibleverse{1} Darum, heilige Brüder, Genossen der himmlischen Berufung,
richtet euer Augenmerk auf den Gottesboten und Hohenpriester unsers
Bekenntnisses, auf Jesus, \bibleverse{2} der da »treu« war dem, der ihn
geschaffen\textless sup title=``oder: dazu gemacht''\textgreater✲ hat,
wie auch Mose (treu gewesen ist) »in Gottes ganzem Hause«\textless sup
title=``4.Mose 12,7''\textgreater✲. \bibleverse{3} Denn einer größeren
Herrlichkeit\textless sup title=``oder: Ehre''\textgreater✲ als Mose ist
dieser würdig erachtet worden, in dem Maße, als der, welcher das Haus
hergestellt hat, höher an Ehre steht als das Haus\textless sup
title=``d.h. die Hausgenossen''\textgreater✲. \bibleverse{4} Denn jedes
Haus wird von jemand hergestellt; der aber alles hergestellt
hat\textless sup title=``=~der Baumeister des Alls''\textgreater✲, das
ist Gott. \bibleverse{5} Und was Mose betrifft, so ist er zwar »in
seinem✲ ganzen Hause treu« gewesen, (aber doch nur) als »Diener«, um
Zeugnis abzulegen für das, was als Offenbarung\textless sup
title=``oder: als Gesetz''\textgreater✲ verkündigt werden sollte;
\bibleverse{6} Christus dagegen (ist treu) als »Sohn« über »sein eigenes
Haus«, und sein Haus sind wir, vorausgesetzt, daß wir an der freudigen
Zuversicht und an der Hoffnung, deren wir uns rühmen, bis ans Ende
unerschütterlich festhalten.

\hypertarget{mahnung-die-verheiuxdfene-gottesruhe-nicht-durch-unglauben-zu-verscherzen}{%
\subsubsection{4. Mahnung, die verheißene Gottesruhe nicht durch
Unglauben zu
verscherzen}\label{mahnung-die-verheiuxdfene-gottesruhe-nicht-durch-unglauben-zu-verscherzen}}

\hypertarget{a-hinweis-auf-die-warnung-des-psalmwortes-und-auf-das-strafgericht-an-dem-untreuen-gottesvolke-beim-zug-durch-die-wuxfcste}{%
\paragraph{a) Hinweis auf die Warnung des Psalmwortes und auf das
Strafgericht an dem untreuen Gottesvolke beim Zug durch die
Wüste}\label{a-hinweis-auf-die-warnung-des-psalmwortes-und-auf-das-strafgericht-an-dem-untreuen-gottesvolke-beim-zug-durch-die-wuxfcste}}

\hypertarget{aa-die-warnung-des-psalmwortes-vor-unglauben-und-abfall}{%
\subparagraph{aa) Die Warnung des Psalmwortes vor Unglauben und
Abfall}\label{aa-die-warnung-des-psalmwortes-vor-unglauben-und-abfall}}

\bibleverse{7} Deshalb (gilt uns) das Wort des heiligen
Geistes\textless sup title=``Ps 95,7-11''\textgreater✲: »Heute, wenn ihr
seine Stimme hört, \bibleverse{8} verhärtet eure Herzen nicht, wie (es
einst) bei der Erbitterung\textless sup title=``=~dem bitteren
Zerwürfnis''\textgreater✲ am Tage der Versuchung in der Wüste (geschah),
\bibleverse{9} wo eure Väter (mich) mit einer Erprobung versuchten; und
doch haben sie meine Werke✲ vierzig Jahre hindurch gesehen.
\bibleverse{10} Deshalb ward ich über dieses Geschlecht entrüstet und
sprach: ›Allezeit gehen sie mit ihrem Herzen irre!‹ Sie aber erkannten
meine Wege nicht, \bibleverse{11} so daß ich in meinem Zorn schwur: ›Sie
sollen nimmermehr in meine Ruhe eingehen!‹« \bibleverse{12} Gebt acht,
liebe Brüder, daß sich in keinem von euch ein böses Herz des Unglaubens
im Abfall von dem lebendigen Gott zeige! \bibleverse{13} Ermahnt euch
vielmehr selbst an jedem Tage, solange das »Heute« noch gilt, damit
keiner von euch durch den Betrug der Sünde verhärtet werde.

\hypertarget{bb-das-warnungsbeispiel-der-israeliten-in-der-wuxfcste}{%
\subparagraph{bb) Das Warnungsbeispiel der Israeliten in der
Wüste}\label{bb-das-warnungsbeispiel-der-israeliten-in-der-wuxfcste}}

\bibleverse{14} Denn Genossen Christi sind wir geworden, wenn anders wir
die anfängliche Glaubenszuversicht bis ans Ende unerschütterlich
festhalten. \bibleverse{15} Wenn es heißt\textless sup title=``Ps
95,7-8''\textgreater✲: »Heute, wenn ihr seine Stimme hört, verhärtet
eure Herzen nicht, wie es bei der Erbitterung✲ geschah«~--
\bibleverse{16} wer waren denn die Leute, die, obgleich sie (seine
Verheißung) gehört hatten, dennoch sich erbittern ließen? Waren es nicht
alle, die durch Moses Vermittlung aus Ägypten ausgezogen waren?
\bibleverse{17} Und wer waren die Leute, über die er vierzig Jahre lang
entrüstet gewesen ist? Doch wohl die, welche gesündigt hatten und deren
Glieder (dann) in der Wüste zerfallen sind. \bibleverse{18} Und wer
waren die Leute, denen er zugeschworen hat, sie sollten nicht in seine
Ruhe eingehen? Doch wohl die, welche sich ungehorsam bewiesen hatten.
\bibleverse{19} So sehen wir denn, daß sie nicht haben hineingelangen
können infolge (ihres) Unglaubens.

\hypertarget{b-ausdeutung-der-verheiuxdfung-des-psalmwortes-bezuxfcglich-der-ruhe-des-gottesvolkes}{%
\paragraph{b) Ausdeutung der Verheißung des Psalmwortes bezüglich der
Ruhe des
Gottesvolkes}\label{b-ausdeutung-der-verheiuxdfung-des-psalmwortes-bezuxfcglich-der-ruhe-des-gottesvolkes}}

\hypertarget{section-3}{%
\section{4}\label{section-3}}

\bibleverse{1} Da nun die Verheißung des Eingehens in seine Ruhe noch
unerfüllt geblieben ist\textless sup title=``oder: immer noch
besteht''\textgreater✲, so wollen wir ängstlich darauf bedacht sein, daß
es sich bei keinem von euch herausstelle, er sei
zurückgeblieben\textless sup title=``=~nicht ans Ziel
gekommen''\textgreater✲. \bibleverse{2} Denn die Heilsbotschaft ist an
uns ebensogut ergangen wie an jene; aber jenen hat das Wort, das sie zu
hören bekamen, nichts genützt, weil es bei den Hörern nicht mit dem
Glauben vereinigt\textless sup title=``=~fest verwachsen''\textgreater✲
war. \bibleverse{3} Wir dagegen, die wir zum Glauben gekommen sind,
gehen in die Ruhe ein, wie er✲ gesagt hat\textless sup title=``Ps
95,11''\textgreater✲: »So daß ich in meinem Zorn schwur: ›Sie sollen
nimmermehr in meine Ruhe eingehen!‹« -- wiewohl doch das Wirken (Gottes)
seit\textless sup title=``oder: mit''\textgreater✲ der Vollendung der
Weltschöpfung zum Abschluß gekommen war. \bibleverse{4} Er hat sich ja
an einer Stelle über den siebten Tag so ausgesprochen\textless sup
title=``1.Mose 2,2''\textgreater✲: »Gott ruhte am siebten Tage von allen
seinen Werken«; \bibleverse{5} an anderer Stelle dagegen heißt
es\textless sup title=``Ps 95,11''\textgreater✲: »Sie sollen nimmermehr
in meine Ruhe eingehen!« \bibleverse{6} Da also das Eingehen einiger in
die Ruhe bestehen bleibt, andrerseits die, welche zuerst die beglückende
Botschaft empfangen haben, infolge (ihres) Ungehorsams nicht
hineingelangt sind, \bibleverse{7} so setzt (Gott) aufs neue einen Tag
fest, ein »Heute«, indem er nach so langer Zeit durch David, wie schon
vorhin✲ gesagt worden ist, verkündigt: »Heute, wenn ihr seine Stimme
hört, verhärtet eure Herzen nicht!« \bibleverse{8} Denn wenn Josua sie
wirklich in die Ruhe eingeführt hätte, so würde (Gott) nicht von einem
anderen, späteren Tage reden. \bibleverse{9} Somit bleibt dem Volk
Gottes eine Sabbatruhe noch vorbehalten; \bibleverse{10} denn wer in
seine\textless sup title=``d.h. Gottes''\textgreater✲ Ruhe eingegangen
ist, der ist damit auch seinerseits zur Ruhe von seinen Werken gelangt,
geradeso wie Gott von den seinigen.

\hypertarget{c-abschlieuxdfende-ermahnung-mit-hinweis-auf-den-ernst-und-die-kraft-des-wortes-gottes}{%
\paragraph{c) Abschließende Ermahnung mit Hinweis auf den Ernst und die
Kraft des Wortes
Gottes}\label{c-abschlieuxdfende-ermahnung-mit-hinweis-auf-den-ernst-und-die-kraft-des-wortes-gottes}}

\bibleverse{11} So wollen wir also eifrig darauf bedacht sein, in jene
Ruhe einzugehen, damit keiner zu Fall kommt und dadurch das gleiche
warnende Beispiel des Ungehorsams darbiete. \bibleverse{12} Denn
lebendig\textless sup title=``=~voller Leben''\textgreater✲ ist das Wort
Gottes und wirkungskräftig und schärfer als jedes zweischneidige
Schwert\textless sup title=``oder: Messer''\textgreater✲: es dringt
hindurch, bis es Seele und Geist, Gelenke und Mark scheidet, und ist ein
Richter über die Regungen\textless sup title=``oder:
Gesinnungen''\textgreater✲ und Gedanken des Herzens; \bibleverse{13} und
es gibt nichts Geschaffenes, das sich vor ihm\textless sup title=``d.h.
vor Gott''\textgreater✲ verbergen könnte, nein, alles liegt entblößt und
aufgedeckt vor den Augen dessen, dem wir Rechenschaft abzulegen haben.

\hypertarget{ii.-das-vollkommene-hohepriestertum-jesu-die-heilslehre-von-der-vollkommenheit-und-die-pflicht-ausharrenden-glaubens-414-1229}{%
\subsection{II. Das vollkommene Hohepriestertum Jesu, die Heilslehre von
der Vollkommenheit und die Pflicht ausharrenden Glaubens
(4,14-12,29)}\label{ii.-das-vollkommene-hohepriestertum-jesu-die-heilslehre-von-der-vollkommenheit-und-die-pflicht-ausharrenden-glaubens-414-1229}}

\hypertarget{jesus-ist-der-vollkommene-von-gott-selbst-eingesetzte-und-vollen-vertrauens-wuxfcrdige-hohepriester}{%
\subsubsection{1. Jesus ist der vollkommene, von Gott selbst eingesetzte
und vollen Vertrauens würdige
Hohepriester}\label{jesus-ist-der-vollkommene-von-gott-selbst-eingesetzte-und-vollen-vertrauens-wuxfcrdige-hohepriester}}

\hypertarget{a-jesus-kennt-die-menschlichen-schwuxe4chen-aus-eigener-erfahrung}{%
\paragraph{a) Jesus kennt die menschlichen Schwächen aus eigener
Erfahrung}\label{a-jesus-kennt-die-menschlichen-schwuxe4chen-aus-eigener-erfahrung}}

\bibleverse{14} Da wir nun einen großen✲ Hohenpriester haben, der durch
die Himmel hindurchgegangen ist, Jesus, den Sohn Gottes, so wollen wir
am Bekenntnis (zu ihm) festhalten. \bibleverse{15} Wir haben ja (an ihm)
nicht einen Hohenpriester, der nicht Mitgefühl mit unsern Schwachheiten
haben könnte, sondern einen solchen, der in allen Stücken auf gleiche
Weise (wie wir) versucht worden ist, nur ohne Sünde\textless sup
title=``=~ohne zu sündigen''\textgreater✲. \bibleverse{16} So wollen wir
denn mit freudiger Zuversicht zum Thron der Gnade hinzutreten, um
Barmherzigkeit zu empfangen und Gnade zu finden zu rechtzeitiger
Hilfe\textless sup title=``d.h. so daß wir Hilfe zu rechter Zeit
finden''\textgreater✲.

\hypertarget{b-bei-christus-finden-sich-die-notwendigen-in-melchisedek-vorgeschatteten-erfordernisse-des-hohenpriesters}{%
\paragraph{b) Bei Christus finden sich die notwendigen, in Melchisedek
vorgeschatteten Erfordernisse des
Hohenpriesters}\label{b-bei-christus-finden-sich-die-notwendigen-in-melchisedek-vorgeschatteten-erfordernisse-des-hohenpriesters}}

\hypertarget{section-4}{%
\section{5}\label{section-4}}

\bibleverse{1} Denn jeder aus der Zahl der Menschen genommene
Hohepriester wird für Menschen zum Dienst vor Gott eingesetzt, um teils
unblutige, teils blutige Opfer für Sünden darzubringen; \bibleverse{2}
und er ist dabei wohl imstande, die Unwissenden und Irrenden
billig\textless sup title=``oder: mit Nachsicht''\textgreater✲ zu
beurteilen, weil er ja selbst mit Schwachheit behaftet ist.
\bibleverse{3} Und um dieser willen muß er wie für das Volk, so auch für
sich selbst Opfer der Sünden wegen darbringen. \bibleverse{4} Und
niemand kann sich selbst diese Würde zueignen, sondern er muß von Gott
dazu berufen werden, ganz so wie es auch bei Aaron der Fall gewesen ist.
\bibleverse{5} So hat denn auch Christus sich nicht selbst✲ die hohe
Würde des Hohenpriesters zugeeignet, sondern der (hat sie ihm
verliehen), der zu ihm gesagt hat\textless sup title=``Ps
2,7''\textgreater✲: »Mein Sohn bist du, ich selbst habe dich heute
gezeugt«; \bibleverse{6} wie er auch an einer anderen Stelle
sagt\textless sup title=``Ps 110,4''\textgreater✲: »Du bist Priester in
Ewigkeit nach der Ordnung Melchisedeks.« \bibleverse{7} Er hat in den
Tagen seines Fleisches✲ Gebete und flehentliche Bitten mit lautem
Schreien✲ und Tränen vor den gebracht, der ihn vom Tode zu erretten
vermochte, und hat auch Erhörung gefunden (und ist) aus seiner Angst
(befreit worden) \bibleverse{8} und hat trotz seiner Sohnesstellung an
seinem Leiden den Gehorsam gelernt. \bibleverse{9} Nachdem er so zur
Vollendung gelangt war, ist er für alle, die ihm gehorsam sind, der
Urheber ewigen Heils geworden, \bibleverse{10} er, der von Gott mit der
Bezeichnung »Hoherpriester nach der Ordnung Melchisedeks«
angeredet\textless sup title=``oder: begrüßt''\textgreater✲ worden ist.

\hypertarget{klage-uxfcber-die-unmuxfcndigkeit-uxfcber-die-geistige-truxe4gheit-und-ruxfcckstuxe4ndigkeit-der-leser}{%
\subsubsection{2. Klage über die Unmündigkeit, über die geistige
Trägheit und Rückständigkeit der
Leser}\label{klage-uxfcber-die-unmuxfcndigkeit-uxfcber-die-geistige-truxe4gheit-und-ruxfcckstuxe4ndigkeit-der-leser}}

\bibleverse{11} Darüber hätten wir noch viel zu sagen, doch es ist
schwer, euch das klarzumachen, weil eure Fassungskraft stumpf geworden
ist. \bibleverse{12} Denn während ihr nach (der Länge) der Zeit schon
Lehrer sein müßtet, bedürft ihr umgekehrt noch der Belehrung in den
Anfangsgründen der göttlichen Offenbarungsworte und seid dahin gekommen,
daß ihr Milch statt fester Nahrung nötig habt. \bibleverse{13} Denn
jeder, der noch auf Milch angewiesen ist, versteht sich noch nicht auf
das Wort der Gerechtigkeit; er ist eben noch ein unmündiges Kind.
\bibleverse{14} Für Gereifte\textless sup title=``oder: Vollkommene,
d.h. Erwachsene''\textgreater✲ dagegen ist die feste Nahrung da, nämlich
für die, welche infolge ihrer Gewöhnung geübte Sinne✲ besitzen, so daß
sie das Gute und das Schlechte zu unterscheiden vermögen.

\hypertarget{unterweisung-uxfcber-die-vollkommenheit-in-erkenntnis-glaube-und-hoffnungsgewiuxdfheit}{%
\subsubsection{3. Unterweisung über die Vollkommenheit in Erkenntnis,
Glaube und
Hoffnungsgewißheit}\label{unterweisung-uxfcber-die-vollkommenheit-in-erkenntnis-glaube-und-hoffnungsgewiuxdfheit}}

\hypertarget{a-es-gilt-fortzuschreiten-ruxfcckfall-ist-gefuxe4hrlich-und-kann-zu-unheilbarem-schaden-fuxfchren}{%
\paragraph{a) Es gilt fortzuschreiten; Rückfall ist gefährlich und kann
zu unheilbarem Schaden
führen}\label{a-es-gilt-fortzuschreiten-ruxfcckfall-ist-gefuxe4hrlich-und-kann-zu-unheilbarem-schaden-fuxfchren}}

\hypertarget{section-5}{%
\section{6}\label{section-5}}

\bibleverse{1} Darum wollen wir (jetzt) von den Anfangsgründen der Lehre
Christi\textless sup title=``oder: über Christus''\textgreater✲ absehen
und uns zur vollen Reife\textless sup title=``oder: Lehre für
Gereifte''\textgreater✲ erheben, wollen nicht noch einmal den Grund
legen mit Sinnesänderung, die sich von toten Werken abkehrt, und mit dem
Glauben an Gott, \bibleverse{2} mit der Belehrung über
Waschungen\textless sup title=``oder: Taufen''\textgreater✲ und
Handauflegung, über Totenauferstehung und ewiges Gericht. \bibleverse{3}
Ja, dies wollen wir tun, wenn anders Gott es gelingen läßt.
\bibleverse{4} Denn es ist unmöglich, solche, die einmal die Erleuchtung
empfangen und die himmlische Gabe geschmeckt haben und des heiligen
Geistes teilhaftig geworden sind \bibleverse{5} und das köstliche
Gotteswort und die Kräfte der zukünftigen Welt gekostet✲ haben
\bibleverse{6} und dann doch abgefallen sind, noch einmal zur
Sinnesänderung zu erneuern, weil sie für ihre Person den Sohn Gottes von
neuem kreuzigen und ihn der Beschimpfung preisgeben\textless sup
title=``oder: ihren Spott mit ihm treiben''\textgreater✲. \bibleverse{7}
Denn wenn ein Acker den oftmals\textless sup title=``oder:
reichlich''\textgreater✲ auf ihn fallenden Regen in sich eingesogen hat
und denen, für die er bestellt wird, nützlichen Ertrag hervorbringt, so
macht er sich den von Gott kommenden Segen zu eigen; \bibleverse{8}
bringt er dagegen Dornen und Disteln\textless sup title=``1.Mose
3,17-18''\textgreater✲ hervor, so ist er unbrauchbar und geht dem Fluch
entgegen, dessen Ende zum Feuerbrand führt.

\hypertarget{b-zuversichtliche-hoffnung-auf-uxfcberwindung-dieses-betruxfcbenden-zustandes-der-leser-und-der-ihnen-drohenden-gefahr}{%
\paragraph{b) Zuversichtliche Hoffnung auf Überwindung dieses
betrübenden Zustandes der Leser und der ihnen drohenden
Gefahr}\label{b-zuversichtliche-hoffnung-auf-uxfcberwindung-dieses-betruxfcbenden-zustandes-der-leser-und-der-ihnen-drohenden-gefahr}}

\bibleverse{9} Wir sind aber in bezug auf euch, Geliebte, wenn wir auch
so reden, doch eines Besseren gewiß, nämlich dessen, was in engster
Beziehung zur Errettung\textless sup title=``vgl. Phil
3,21''\textgreater✲ steht. \bibleverse{10} Denn Gott ist nicht
ungerecht, daß er eure Arbeit\textless sup title=``=~das, was ihr
geleistet habt''\textgreater✲ und die Liebe vergäße, die ihr für seinen
Namen dadurch an den Tag gelegt habt, daß ihr den Heiligen Dienste
geleistet habt und auch jetzt noch leistet. \bibleverse{11} Wir wünschen
aber innig, daß jeder einzelne von euch den gleichen Eifer an den Tag
legen möge, um die Hoffnung bis ans Ende mit voller Gewißheit
festzuhalten, \bibleverse{12} damit ihr nicht stumpf\textless sup
title=``oder: lässig''\textgreater✲ werdet, sondern dem Vorbild derer
nachfolgt, die durch Glauben und standhaftes Ausharren\textless sup
title=``oder: Geduld''\textgreater✲ die verheißenen Heilsgüter erben.

\hypertarget{c-der-feste-grund-der-hoffnung-auf-die-sicher-zu-erwartende-herrlichkeit-liegt-in-den-zuverluxe4ssigen-verheiuxdfungen-gottes}{%
\paragraph{c) Der feste Grund der Hoffnung auf die sicher zu erwartende
Herrlichkeit liegt in den zuverlässigen Verheißungen
Gottes}\label{c-der-feste-grund-der-hoffnung-auf-die-sicher-zu-erwartende-herrlichkeit-liegt-in-den-zuverluxe4ssigen-verheiuxdfungen-gottes}}

\bibleverse{13} Nachdem Gott nämlich dem Abraham die Verheißung gegeben
hatte, schwur er, weil er bei keinem Höheren schwören konnte, bei sich
selbst \bibleverse{14} mit den Worten\textless sup title=``1.Mose
22,16-17''\textgreater✲: »Fürwahr, ich will dich reichlich segnen und
dich überaus zahlreich machen!«, \bibleverse{15} und auf diese
Weise\textless sup title=``d.h. auf diesen Eidschwur hin''\textgreater✲
harrte jener geduldig aus und erlangte das Verheißene. \bibleverse{16}
Menschen schwören bekanntlich bei dem Höheren, und der Eid dient ihnen
zur Bekräftigung, so daß alle Widerrede ausgeschlossen ist.
\bibleverse{17} Aus diesem Grunde ist auch Gott, weil er den Erben
seiner Verheißung das Unabänderliche seines Ratschlusses in besonderem
Grade deutlich dartun wollte, als Bürge mit einem Eid eingetreten,
\bibleverse{18} damit wir durch zwei unabänderliche Tatsachen, bei denen
Gott unmöglich getäuscht haben kann, eine starke Ermutigung besäßen,
wir, die wir unsere Zuflucht dazu genommen haben, die uns eröffnete
Hoffnung zu ergreifen. \bibleverse{19} In dieser besitzen wir ja
gleichsam einen festen und zuverlässigen Anker für unsere Seele, der bis
hinter den Vorhang (in das himmlische Heiligtum) hineinreicht,
\bibleverse{20} wohin Jesus als Vorläufer uns zum Heil hineingegangen
ist, insofern er »Hoherpriester nach der Ordnung✲ Melchisedeks« geworden
ist in Ewigkeit.

\hypertarget{das-hohepriestertum-jesu-sein-vollkommenes-opfer-und-die-vollendung-der-gluxe4ubigen-in-christus}{%
\subsubsection{4. Das Hohepriestertum Jesu, sein vollkommenes Opfer und
die Vollendung der Gläubigen in
Christus}\label{das-hohepriestertum-jesu-sein-vollkommenes-opfer-und-die-vollendung-der-gluxe4ubigen-in-christus}}

\hypertarget{a-jesus-der-vollkommene-hohepriester-auf-ewig-nach-der-ordnung-melchisedeks}{%
\paragraph{a) Jesus der vollkommene Hohepriester auf ewig nach der
Ordnung
Melchisedeks}\label{a-jesus-der-vollkommene-hohepriester-auf-ewig-nach-der-ordnung-melchisedeks}}

\hypertarget{aa-angaben-uxfcber-die-person-melchisedeks}{%
\subparagraph{aa) Angaben über die Person
Melchisedeks}\label{aa-angaben-uxfcber-die-person-melchisedeks}}

\hypertarget{section-6}{%
\section{7}\label{section-6}}

\bibleverse{1} Dieser Melchisedek nämlich, König von Salem, Priester des
höchsten Gottes, ging dem Abraham entgegen, als dieser von der Besiegung
der Könige\textless sup title=``1.Mose 14,17-20''\textgreater✲
zurückkehrte, und segnete ihn; \bibleverse{2} dafür teilte Abraham ihm
dann auch den Zehnten von der ganzen Beute zu. Zunächst ist er, wenn man
seinen Namen deutet, ›König der Gerechtigkeit‹, sodann aber auch ›König
von Salem‹, das bedeutet ›König des Friedens‹; \bibleverse{3} er hat (im
biblischen Bericht) keinen Vater, keine Mutter, keine Ahnenreihe, weder
einen Anfang seiner Tage noch ein Ende seines Lebens, gleicht vielmehr
dem Sohne Gottes: er bleibt Priester für immer.

\hypertarget{bb-melchisedek-steht-an-wuxfcrde-huxf6her-als-die-levitischen-priester}{%
\subparagraph{bb) Melchisedek steht an Würde höher als die levitischen
Priester}\label{bb-melchisedek-steht-an-wuxfcrde-huxf6her-als-die-levitischen-priester}}

\bibleverse{4} Beachtet nun wohl, wie groß dieser Mann dasteht, dem der
Erzvater Abraham den Zehnten von den auserlesensten Beutestücken gegeben
hat! \bibleverse{5} Wohl sind auch diejenigen Nachkommen Levis, die das
Priestertum empfangen, berechtigt, den Zehnten von dem Volk in der vom
Gesetz vorgeschriebenen Weise zu erheben\textless sup title=``4.Mose
18,20-30''\textgreater✲, also von ihren Brüdern✲, obgleich doch auch
diese leibliche Nachkommen Abrahams sind; \bibleverse{6} jener aber hat,
wiewohl er seiner Abstammung nach mit ihnen in keiner Verbindung steht,
von Abraham (selbst) den Zehnten erhoben und den, der im Besitz der
Verheißungen war, gesegnet. \bibleverse{7} Nun ist es aber durchaus
unbestreitbar, daß das Geringere von dem Höheren gesegnet wird.
\bibleverse{8} Und hier\textless sup title=``oder: in dem einen
Fall''\textgreater✲ sind es sterbliche Menschen, welche die Zehnten
entgegennehmen, dort aber\textless sup title=``oder: in dem andern
Fall''\textgreater✲ ist es einer, dem\textless sup title=``oder: von
dem''\textgreater✲ bezeugt wird, daß er lebt✲. \bibleverse{9} Weiter: in
der Person Abrahams ist gewissermaßen auch vom Zehntenempfänger Levi der
Zehnte erhoben worden; \bibleverse{10} denn er (Levi) befand sich
(damals) noch in der Lende seines Stammvaters, als Melchisedek diesem
entgegenging.

\hypertarget{cc-die-durch-das-eigenartige-priestertum-jesu-bewirkte-uxe4nderung-aufhebung-der-priesterordnung}{%
\subparagraph{cc) Die durch das eigenartige Priestertum Jesu bewirkte
Änderung (Aufhebung) der
Priesterordnung}\label{cc-die-durch-das-eigenartige-priestertum-jesu-bewirkte-uxe4nderung-aufhebung-der-priesterordnung}}

\bibleverse{11} Freilich, wenn eine Vollendung\textless sup
title=``=~etwas Vollkommenes''\textgreater✲ durch das levitische
Priestertum möglich\textless sup title=``oder: zu erreichen
gewesen''\textgreater✲ wäre -- auf diesem (Priestertum) beruht ja die
ganze Gesetzgebung✲ des Volkes --: welches Bedürfnis hätte dann noch
vorgelegen, einen andersartigen Priester »nach der Ordnung Melchisedeks«
einzusetzen und ihn nicht (einfach) »nach der Ordnung Aarons« zu
benennen? \bibleverse{12} Denn mit einer Änderung\textless sup
title=``oder: Umgestaltung''\textgreater✲ des Priestertums tritt mit
Notwendigkeit auch eine Änderung des Gesetzes ein. \bibleverse{13} Der
nämlich, auf den sich jener Ausspruch\textless sup title=``Ps
110,4''\textgreater✲ bezieht, hat ja doch einem anderen Stamme angehört,
aus dem niemand (jemals) mit dem Altardienst zu tun gehabt hat.
\bibleverse{14} Es ist ja doch allbekannt, daß unser Herr (Jesus) aus
(dem Stamme) Juda hervorgegangen ist, und in bezug auf diesen Stamm hat
Mose nichts verlauten lassen, was sich auf Priester bezieht.
\bibleverse{15} Und vollends klar liegt die Sache dadurch, daß ein
andersartiger Priester, der dem Melchisedek ähnlich ist, eingesetzt
wird, \bibleverse{16} der es nicht nach der Bestimmung eines an
leibliche Abstammung bindenden Gebotes geworden ist, sondern nach der
Kraft unzerstörbaren Lebens. \bibleverse{17} Denn ihm\textless sup
title=``oder: von ihm =~über ihn''\textgreater✲ wird
bezeugt\textless sup title=``Ps 110,4''\textgreater✲: »Du bist Priester
in Ewigkeit nach der Ordnung Melchisedeks.«

\hypertarget{dd-der-grund-fuxfcr-den-wechsel-der-priesterordnung-ist-dauxdf-jesus-buxfcrge-eines-huxf6heren-bundes-werden-sollte}{%
\subparagraph{dd) Der Grund für den Wechsel der Priesterordnung ist, daß
Jesus Bürge eines höheren Bundes werden
sollte}\label{dd-der-grund-fuxfcr-den-wechsel-der-priesterordnung-ist-dauxdf-jesus-buxfcrge-eines-huxf6heren-bundes-werden-sollte}}

\bibleverse{18} Damit tritt einerseits zwar die Aufhebung eines bis
dahin gültigen Gebotes ein, weil es sich unwirksam und unbrauchbar
erwiesen hatte~-- \bibleverse{19} das (mosaische) Gesetz hat ja auch
wirklich keine Vollendung\textless sup title=``=~nichts
Vollkommenes''\textgreater✲ zustande gebracht --, andrerseits (tritt
dadurch) aber auch die Herbeiführung einer besseren Hoffnung (ein),
mittels derer\textless sup title=``=~bei deren
Verwirklichung''\textgreater✲ wir Gott (wirklich) nahen können.
\bibleverse{20} Und insofern (er) nicht ohne Eidesleistung (Priester
geworden ist) -- jene sind ja ohne Eidschwur Priester geworden,
\bibleverse{21} dieser dagegen mit einem Eidschwur von seiten dessen,
der zu ihm spricht\textless sup title=``Ps 110,4''\textgreater✲:
»Geschworen hat der Herr, und es wird ihn nicht gereuen: du bist
Priester in Ewigkeit« --: \bibleverse{22} dementsprechend ist Jesus um
so mehr der Bürge eines besseren Bundes geworden. \bibleverse{23}
Außerdem sind dort Priester als Vielheit\textless sup title=``=~in
beträchtlicher Anzahl''\textgreater✲ vorhanden gewesen, weil sie durch
den Tod daran gehindert wurden, (im Amt) zu verbleiben; \bibleverse{24}
hier aber ist es ein solcher, der, weil er »in Ewigkeit« bleibt, ein nie
wechselndes Priestertum im Besitz hat. \bibleverse{25} Daher vermag er
auch denen, die durch seine Vermittlung zu Gott hinzutreten, vollkommene
Rettung zu schaffen: er lebt ja immerdar, um fürbittend für sie (vor
Gott) einzutreten.

\hypertarget{ee-abschlieuxdfende-zusammenfassung-jesus-als-der-vollkommene-und-ewige-hohepriester}{%
\subparagraph{ee) Abschließende Zusammenfassung: Jesus als der
vollkommene und ewige
Hohepriester}\label{ee-abschlieuxdfende-zusammenfassung-jesus-als-der-vollkommene-und-ewige-hohepriester}}

\bibleverse{26} Denn einen solchen Hohenpriester mußten wir auch haben,
der da heilig, schuldlos, unbefleckt ist, von den Sündern geschieden und
über die Himmel hoch erhöht; \bibleverse{27} der nicht wie die
(menschlichen) Hohenpriester Tag für Tag nötig hat, zunächst für seine
eigenen Sünden Opfer darzubringen, danach für die des Volkes; denn
letzteres hat er ein für allemal dadurch geleistet, daß er sich selbst
(zum Opfer) dargebracht hat. \bibleverse{28} Denn das (mosaische) Gesetz
bestellt zu Hohenpriestern Menschen, die mit Schwachheit behaftet sind;
das Wort des Eidschwurs dagegen, der erst nach dem Gesetz erfolgt ist,
setzt einen\textless sup title=``oder: den''\textgreater✲ Sohn ein, der
für die Ewigkeit vollendet ist.

\hypertarget{b-jesus-waltet-als-hoherpriester-im-himmel-aufgrund-seines-abschlieuxdfenden-selbstopfers}{%
\paragraph{b) Jesus waltet als Hoherpriester im Himmel aufgrund seines
abschließenden
Selbstopfers}\label{b-jesus-waltet-als-hoherpriester-im-himmel-aufgrund-seines-abschlieuxdfenden-selbstopfers}}

\hypertarget{aa-die-uxfcberlegenheit-des-himmlischen-hohepriesterdienstes-jesu-und-des-neuen-bundes-dessen-mittler-er-ist}{%
\subparagraph{aa) Die Überlegenheit des himmlischen Hohepriesterdienstes
Jesu und des neuen Bundes, dessen Mittler er
ist}\label{aa-die-uxfcberlegenheit-des-himmlischen-hohepriesterdienstes-jesu-und-des-neuen-bundes-dessen-mittler-er-ist}}

\hypertarget{section-7}{%
\section{8}\label{section-7}}

\bibleverse{1} Die Hauptsache aber bei der vorliegenden Darlegung ist
folgendes: Einen solchen Hohenpriester haben wir, der sich im Himmel zur
Rechten des Thrones der göttlichen Erhabenheit✲ gesetzt hat,
\bibleverse{2} und zwar als Verwalter\textless sup title=``oder:
priesterlicher Diener''\textgreater✲ des Heiligtums, nämlich des wahren
Zeltes\textless sup title=``vgl. V.5''\textgreater✲, das der Herr, nicht
ein Mensch errichtet hat. \bibleverse{3} Denn jeder Hohepriester wird zu
dem Zweck bestellt, unblutige und blutige Opfer darzubringen; daher muß
auch dieser\textless sup title=``d.h. Jesus''\textgreater✲ etwas
darzubringen haben. \bibleverse{4} Befände er sich nun auf der Erde, so
würde er nicht einmal Priester sein, weil hier ja bereits Priester
vorhanden sind, welche die Gaben nach dem (mosaischen) Gesetz
darbringen. \bibleverse{5} Diese versehen freilich den Dienst nur an
einer Nachbildung und einem Schattenbild der himmlischen Dinge
entsprechend der göttlichen Weisung, die Mose erhielt, als er das
Zelt\textless sup title=``=~die Stiftshütte''\textgreater✲ herstellen
sollte; denn »Gib wohl acht«, sagt der Herr zu ihm\textless sup
title=``2.Mose 25,40''\textgreater✲, »daß du alles nach dem Vorbild✲
anfertigst, das dir auf dem Berge gezeigt worden ist«.

\bibleverse{6} Nun aber hat er\textless sup title=``d.h.
Jesus''\textgreater✲ einen um so vorzüglicheren Priesterdienst erlangt,
als er auch Mittler eines besseren\textless sup title=``oder:
höheren''\textgreater✲ Bundes ist, der auf der Grundlage besserer✲
Verheißungen festgesetzt worden ist. \bibleverse{7} Wenn nämlich jener
erste (Bund) tadellos gewesen wäre, so würde nicht die Möglichkeit,
einen zweiten (Bund) zu schließen, gesucht werden. \bibleverse{8} Denn
einen Tadel spricht (Gott) gegen sie (die Israeliten) aus mit den
Worten\textless sup title=``Jer 31,31-34''\textgreater✲: »Wisset wohl:
es kommen Tage« -- so lautet der Ausspruch des Herrn --, »da will ich
mit dem Hause Israel und mit dem Hause Juda einen neuen Bund
aufrichten✲, \bibleverse{9} nicht einen solchen Bund, wie ich ihn mit
ihren Vätern\textless sup title=``oder: für ihre Väter''\textgreater✲
damals geschlossen habe, als ich sie bei der Hand nahm, um sie aus dem
Lande Ägypten wegzuführen; denn sie sind meinem Bunde nicht treu
geblieben, und auch ich habe mich nicht (mehr) um sie gekümmert« -- so
lautet der Ausspruch des Herrn --. \bibleverse{10} »Nein, darin soll der
Bund bestehen, den ich mit dem Hause Israel nach jenen Tagen schließen
werde« -- so lautet der Ausspruch des Herrn --: »Ich will meine Gesetze
in ihren Sinn hineinlegen und sie ihnen ins Herz schreiben und will dann
ihr Gott sein, und sie sollen mein Volk sein. \bibleverse{11} Dann
braucht niemand mehr seinem Mitbürger und niemand seinem Bruder
Belehrung zu erteilen und ihm vorzuhalten: ›Lerne den Herrn kennen!‹
Denn sie werden mich allesamt kennen vom Kleinsten bis zum Größten unter
ihnen. \bibleverse{12} Denn ihren Übertretungen gegenüber werde ich
Nachsicht üben und ihrer Sünden nicht mehr gedenken.« \bibleverse{13}
Indem er hier von einem »neuen« (Bunde) redet, hat er den ersten für
veraltet erklärt; was aber veraltet ist und sich überlebt hat, das geht
dem Untergang entgegen.

\hypertarget{bb-die-unvollkommenheit-des-levitischen-priesterdienstes-und-die-vollkommenheit-oder-uxfcberlegenheit-des-hohepriesterlichen-dienstes-christi}{%
\subparagraph{bb) Die Unvollkommenheit des levitischen Priesterdienstes
und die Vollkommenheit (oder: Überlegenheit) des hohepriesterlichen
Dienstes
Christi}\label{bb-die-unvollkommenheit-des-levitischen-priesterdienstes-und-die-vollkommenheit-oder-uxfcberlegenheit-des-hohepriesterlichen-dienstes-christi}}

\hypertarget{section-8}{%
\section{9}\label{section-8}}

\bibleverse{1} Allerdings hatte auch der erste (Bund)
Satzungen\textless sup title=``=~feste Bestimmungen''\textgreater✲ für
den Gottesdienst und (hatte) auch das weltliche\textless sup
title=``oder: irdische''\textgreater✲ Heiligtum. \bibleverse{2} Es
wurde\textless sup title=``oder: war''\textgreater✲ nämlich ein Zelt
hergestellt, dessen Vorderraum, in welchem sich der Leuchter sowie der
Tisch mit den aufgelegten Schaubroten befinden, das sogenannte Heilige
ist. \bibleverse{3} Hinter dem zweiten Vorhang aber liegt der Teil des
Zeltes, der das Allerheiligste genannt wird, \bibleverse{4} mit dem
goldenen Räucheraltar und der ganz mit Gold überzogenen Bundeslade, in
welcher sich der goldene Krug mit dem Manna sowie der Stab Aarons, der
Blüten getrieben hatte, und die Bundestafeln befinden; \bibleverse{5}
oben über ihr aber stehen die (beiden) Cherube der
Herrlichkeit\textless sup title=``=~als Zeichen der Gegenwart Gottes;
Jes 37,16''\textgreater✲, welche (mit ihren Flügeln) die Deckplatte
überschatten -- doch hierüber soll jetzt nicht im einzelnen geredet
werden. \bibleverse{6} Seitdem nun dies so eingerichtet worden ist,
betreten die Priester, welche die gottesdienstlichen Handlungen zu
verrichten haben, den Vorderraum des Zeltes jederzeit; \bibleverse{7} in
den zweiten✲ Raum dagegen darf nur der Hohepriester einmal im Jahr
eintreten, (und zwar) nicht ohne Blut, das er für sich selbst und für
die Verfehlungen\textless sup title=``=~Unwissenheitssünden; vgl. 4.Mose
15,22-31''\textgreater✲ des Volkes darbringt. \bibleverse{8} Dadurch
weist der heilige Geist darauf hin, daß der Weg\textless sup
title=``oder: Zugang''\textgreater✲ zum wahrhaften Heiligtum✲ noch nicht
geoffenbart\textless sup title=``=~für die Allgemeinheit
freigegeben''\textgreater✲ ist, solange das vordere Zelt noch Bestand
hat. \bibleverse{9} So ist denn dieser Vorraum ein Sinnbild\textless sup
title=``=~sinnbildlicher Hinweis''\textgreater✲ auf\textless sup
title=``oder: für''\textgreater✲ die Gegenwart, insofern in ihm
unblutige und blutige Opfer dargebracht werden, die doch nicht imstande
sind, den, der (Gott mit ihnen) dient, in seinem Gewissen ans Ziel zu
bringen\textless sup title=``=~völlig zu befriedigen''\textgreater✲.
\bibleverse{10} Sie sind ja neben den (Verordnungen über) Speisen,
Getränke und mancherlei Waschungen ebenfalls nur als fleischliche✲
Satzungen\textless sup title=``oder: Verordnungen''\textgreater✲ bis zu
der Zeit auferlegt, wo etwas Besseres\textless sup title=``=~die
richtige Ordnung''\textgreater✲ in Geltung tritt.

\bibleverse{11} Christus dagegen ist, indem er als Hoherpriester der
zukünftigen Güter kam\textless sup title=``oder:
erschien''\textgreater✲, durch das größere\textless sup title=``oder:
erhabenere''\textgreater✲ und vollkommenere Zelt, das nicht mit Händen
gemacht ist, d.h. nicht dieser erschaffenen Welt angehört,
\bibleverse{12} auch nicht mittels des Blutes von Böcken und Kälbern,
sondern mittels seines eigenen Blutes ein für allemal in das
(himmlische) Heiligtum eingetreten und hat eine ewiggültige Erlösung
ausfindig gemacht. \bibleverse{13} Denn wenn schon das Blut von Böcken
und Stieren und die Asche einer Kuh\textless sup title=``vgl. 4.Mose
19''\textgreater✲, mit der man die Verunreinigten besprengt, Heiligung
zu leiblicher Reinheit bewirkt, \bibleverse{14} um wieviel mehr wird das
Blut Christi, der kraft ewigen Geistes sich selbst als ein fehlerloses
Opfer Gott dargebracht hat, unser Gewissen von toten Werken reinigen,
damit wir dem lebendigen Gott dienen!

\hypertarget{cc-christus-als-mittler-eines-neuen-bundes-und-sein-einmaliger-opfertod-als-ewigguxfcltiges-mittel-seines-himmlischen-hohepriesterdienstes}{%
\subparagraph{cc) Christus als Mittler eines neuen Bundes und sein
einmaliger Opfertod als ewiggültiges Mittel seines himmlischen
Hohepriesterdienstes}\label{cc-christus-als-mittler-eines-neuen-bundes-und-sein-einmaliger-opfertod-als-ewigguxfcltiges-mittel-seines-himmlischen-hohepriesterdienstes}}

\bibleverse{15} Und aus diesem Grunde ist er auch der Mittler eines
neuen Bundes, damit aufgrund eines Todes, der zum Erlaß\textless sup
title=``=~zur Sühnung''\textgreater✲ der während der Dauer des ersten
Bundes begangenen Übertretungen erfolgt ist, die Berufenen das
verheißene Gut des ewigen Erbes empfangen sollten\textless sup
title=``vgl. Kol 1,5''\textgreater✲. \bibleverse{16} Denn wo eine
letztwillige Stiftung\textless sup title=``=~ein
Testament''\textgreater✲ vorliegt, da muß unbedingt der Tod dessen, der
die Stiftung errichtet hat, (als eingetreten) nachgewiesen werden;
\bibleverse{17} eine Stiftung wird ja erst nach Eintritt des Todes
rechtskräftig, während sie durchaus keine Kraft✲ besitzt, solange der
Stifter noch lebt. \bibleverse{18} Daher ist ja auch der erste Bund
nicht ohne Blut eingeweiht worden. \bibleverse{19} Nachdem nämlich Mose
jedes Gebot, wie das Gesetz es vorschrieb, dem ganzen Volke vorgetragen
hatte, nahm er das Blut der Kälber und der Böcke nebst Wasser und
Scharlachwolle und einem Büschel Ysop und besprengte damit wie das Buch
selbst, so auch das gesamte Volk, \bibleverse{20} indem er dabei
ausrief: »Dies ist das Blut des Bundes, den Gott für euch angeordnet
hat!«\textless sup title=``2.Mose 24,6-8''\textgreater✲ \bibleverse{21}
Aber auch das Zelt und sämtliche gottesdienstlichen Geräte besprengte er
in gleicher Weise mit dem Blute; \bibleverse{22} überhaupt wird beinahe
alles nach dem Gesetz mit Blut gereinigt, und ohne Blutvergießen erfolgt
keine Vergebung.

\hypertarget{c-jesu-vollkommenes-opfer-und-die-vollendung-der-gluxe4ubigen-in-christus}{%
\paragraph{c) Jesu vollkommenes Opfer und die Vollendung der Gläubigen
in
Christus}\label{c-jesu-vollkommenes-opfer-und-die-vollendung-der-gluxe4ubigen-in-christus}}

\hypertarget{aa-das-einmalige-blutige-selbstopfer-christi-und-seine-gewaltige-heilsbedeutung-fuxfcr-die-gluxe4ubigen}{%
\subparagraph{aa) Das einmalige blutige Selbstopfer Christi und seine
gewaltige Heilsbedeutung für die
Gläubigen}\label{aa-das-einmalige-blutige-selbstopfer-christi-und-seine-gewaltige-heilsbedeutung-fuxfcr-die-gluxe4ubigen}}

\bibleverse{23} Es mußten also zwar die Nachbildungen der im Himmel
(befindlichen Heiligtümer) durch diese Mittel gereinigt werden, aber für
die himmlischen Heiligtümer selbst muß es bessere Opfer geben, als jene
es sind. \bibleverse{24} Denn Christus ist nicht in ein von
Menschenhänden hergestelltes Heiligtum eingegangen, das nur eine
Nachbildung des wahren\textless sup title=``oder:
eigentlichen''\textgreater✲ Heiligtums wäre, sondern in den Himmel
selbst, um jetzt uns zum Heil (persönlich) vor dem Angesicht Gottes zu
erscheinen; \bibleverse{25} auch hat er das nicht in der Absicht getan,
sich oftmals als Opfer darzubringen, wie der (irdische) Hohepriester
alljährlich mit fremdem Blut in das Heiligtum hineingeht;
\bibleverse{26} sonst hätte er ja seit Erschaffung der Welt oftmals
leiden müssen. So aber ist er nur einmal am Ende der Weltzeiten✲
offenbar geworden\textless sup title=``oder: erschienen''\textgreater✲,
um die Sünde durch sein Opfer aufzuheben\textless sup title=``=~zu
beseitigen''\textgreater✲. \bibleverse{27} Und so gewiß es den Menschen
bevorsteht\textless sup title=``oder: bestimmt ist''\textgreater✲,
einmal zu sterben, danach aber das Gericht, \bibleverse{28} ebenso wird
auch Christus, nachdem er ein einziges Mal als Opfer dargebracht worden
ist, um die Sünden vieler wegzunehmen, zum zweitenmal ohne (Beziehung
zur) Sünde denen, die auf ihn warten, zum Heil\textless sup
title=``oder: zur Errettung; vgl. Phil 3,20-21''\textgreater✲
erscheinen.

\hypertarget{bb-das-schattenhafte-vorbild-und-die-unzuluxe4nglichkeit-des-alljuxe4hrlichen-versuxf6hnungsopfers-des-levitischen-hohenpriesters-die-vollkommenheit-des-opfers-jesu}{%
\subparagraph{bb) Das schattenhafte Vorbild und die Unzulänglichkeit des
alljährlichen Versöhnungsopfers des levitischen Hohenpriesters; die
Vollkommenheit des Opfers
Jesu}\label{bb-das-schattenhafte-vorbild-und-die-unzuluxe4nglichkeit-des-alljuxe4hrlichen-versuxf6hnungsopfers-des-levitischen-hohenpriesters-die-vollkommenheit-des-opfers-jesu}}

\hypertarget{section-9}{%
\section{10}\label{section-9}}

\bibleverse{1} Denn weil das (mosaische) Gesetz nur das schattenhafte
Abbild der zukünftigen Heilsgüter enthält✲, nicht aber die Gestalt der
Dinge selbst\textless sup title=``d.h. die wirkliche Erscheinungsform
der Dinge''\textgreater✲, so ist es nimmermehr imstande,
alljährlich\textless sup title=``vgl. 9,25''\textgreater✲ durch
dieselben Opfer, die man immer wieder darbringt, die an den Opfern
Teilnehmenden ans Ziel\textless sup title=``=~zur
Vollendung''\textgreater✲ zu bringen. \bibleverse{2} Würde man sonst
nicht mit ihrer Darbringung aufgehört haben, weil doch die Teilnehmer am
Gottesdienst keinerlei Schuldbewußtsein mehr gehabt hätten, wenn sie ein
für allemal gereinigt gewesen wären? \bibleverse{3} Statt dessen tritt
durch diese Opfer alljährlich eine Erinnerung an (die) Sünden ein,
\bibleverse{4} denn Blut von Stieren und Böcken kann unmöglich
Sünden\textless sup title=``=~Gesetzesübertretungen; 9,15; Jer
11,15''\textgreater✲ wegschaffen. \bibleverse{5} Daher sagt
er\textless sup title=``d.h. der Messias''\textgreater✲ auch bei seinem
Eintritt in die Welt\textless sup title=``Ps 40,7-9''\textgreater✲:
»Schlachtopfer und Speisopfer hast du nicht gewollt\textless sup
title=``=~haben wollen''\textgreater✲, wohl aber hast du mir einen Leib
bereitet; \bibleverse{6} an Brandopfern und Sündopfern hast du kein
Wohlgefallen gehabt. \bibleverse{7} Da sprach ich: ›Siehe, ich komme --
in der Buchrolle\textless sup title=``Ps 40,8''\textgreater✲ steht über
mich geschrieben --, um deinen Willen, o Gott, zu tun.‹« \bibleverse{8}
Während er zu Anfang sagt: »Schlachtopfer und Speisopfer, Brandopfer und
Sündopfer hast du nicht gewollt und kein Wohlgefallen an ihnen gehabt«
-- obgleich diese Opfer doch dem Gesetz entsprechend dargebracht werden
--, \bibleverse{9} fährt er danach fort: »Siehe, ich komme, um deinen
Willen zu tun«: er hebt (also) das Erste auf, um das Zweite dafür als
gültig hinzustellen; \bibleverse{10} und auf Grund dieses Willens
(Gottes) sind wir durch die Darbringung✲ des Leibes Jesu Christi ein für
allemal geheiligt.

\hypertarget{cc-das-einmalige-und-in-vollkommenheit-guxfcltige-selbstopfer-jesu-macht-alle-anderen-suxfcndopfer-unnuxf6tig-weil-es-die-gluxe4ubigen-vor-gott-ganz-vollkommen-gemacht-hat}{%
\subparagraph{cc) Das einmalige und in Vollkommenheit gültige
Selbstopfer Jesu macht alle anderen Sündopfer unnötig, weil es die
Gläubigen vor Gott ganz vollkommen gemacht
hat}\label{cc-das-einmalige-und-in-vollkommenheit-guxfcltige-selbstopfer-jesu-macht-alle-anderen-suxfcndopfer-unnuxf6tig-weil-es-die-gluxe4ubigen-vor-gott-ganz-vollkommen-gemacht-hat}}

\bibleverse{11} Und jeder Priester zwar steht Tag für Tag da, indem er
seinen Dienst verrichtet und immer wieder dieselben Opfer darbringt, die
doch nimmermehr imstande sind, Sünden wegzuschaffen; \bibleverse{12}
dieser dagegen hat nur ein einziges Opfer für (die) Sünden dargebracht
und sich dann für immer zur Rechten Gottes gesetzt; \bibleverse{13}
hinfort wartet er, bis seine Feinde hingelegt sein werden zum Schemel
seiner Füße. \bibleverse{14} Denn durch eine einzige Darbringung✲ hat er
die, welche sich (von ihm) heiligen lassen (wollen), für immer ans
Ziel\textless sup title=``=~zur Vollendung''\textgreater✲ gebracht.
\bibleverse{15} Dafür gibt uns aber auch der heilige Geist sein Zeugnis;
denn nach den Worten\textless sup title=``Jer 31,33-34''\textgreater✲:
\bibleverse{16} »Dies ist der Bund, den ich nach jenen Tagen mit ihnen
schließen\textless sup title=``oder: für sie festsetzen''\textgreater✲
werde«, fährt der Herr fort: »Ich will meine Gesetze in ihre Herzen
hineinlegen und sie ihnen in den Sinn schreiben« \bibleverse{17} und
»ihrer Sünden und ihrer Gesetzlosigkeiten will ich nicht mehr gedenken«.
\bibleverse{18} Wo diese aber Vergebung gefunden haben, da ist keine
Darbringung✲ für Sünde\textless sup title=``=~kein
Sündopfer''\textgreater✲ mehr erforderlich.

\hypertarget{mahnungen-zu-einer-der-herrlichkeit-des-durch-den-neuen-bund-geschaffenen-heils-entsprechenden-treue-und-vollendung-im-glauben-um-die-verheiuxdfungen-zu-erlangen}{%
\subsubsection{5. Mahnungen zu einer der Herrlichkeit des durch den
neuen Bund geschaffenen Heils entsprechenden Treue und Vollendung im
Glauben, um die Verheißungen zu
erlangen}\label{mahnungen-zu-einer-der-herrlichkeit-des-durch-den-neuen-bund-geschaffenen-heils-entsprechenden-treue-und-vollendung-im-glauben-um-die-verheiuxdfungen-zu-erlangen}}

\hypertarget{a-allgemeine-mahnung-zum-ausharren-im-glauben-hoffen-und-lieben-und-zwar-in-gemeinschaft-mit-der-ganzen-gemeinde}{%
\paragraph{a) Allgemeine Mahnung zum Ausharren im Glauben, Hoffen und
Lieben, und zwar in Gemeinschaft mit der ganzen
Gemeinde}\label{a-allgemeine-mahnung-zum-ausharren-im-glauben-hoffen-und-lieben-und-zwar-in-gemeinschaft-mit-der-ganzen-gemeinde}}

\bibleverse{19} Da wir also, liebe Brüder, die freudige Zuversicht
haben, durch das Blut Jesu in das (himmlische) Heiligtum einzugehen~--
\bibleverse{20} das ist der neue, lebendige Weg, den er uns durch den
Vorhang hindurch, das heißt durch sein Fleisch, eingeweiht✲ hat --,
\bibleverse{21} und da wir einen großen\textless sup title=``oder:
erhabenen''\textgreater✲ Priester haben, der über das Haus Gottes
gesetzt ist\textless sup title=``oder: waltet''\textgreater✲,
\bibleverse{22} so laßt uns mit aufrichtigem Herzen in voller
Glaubensgewißheit hinzutreten, nachdem wir uns durch Besprengung der
Herzen vom bösen Gewissen✲ befreit und unsern Leib mit reinem Wasser
gewaschen\textless sup title=``oder: in reinem Wasser
gebadet''\textgreater✲ haben. \bibleverse{23} Laßt uns am Bekenntnis der
Hoffnung unerschütterlich festhalten; denn treu ist der, welcher die
Verheißung gegeben hat. \bibleverse{24} Und laßt uns auch aufeinander
achtgeben, um uns gegenseitig zur Liebe und zu guten Werken anzuregen,
\bibleverse{25} indem wir unsere Zusammenkünfte✲ nicht versäumen, wie
das bei etlichen Gewohnheit ist, sondern uns gegenseitig ermuntern, und
zwar um so mehr, als ihr den Tag (der Wiederkunft Jesu) schon nahen
seht.

\hypertarget{warnung-vor-abfall-und-vor-dem-guxf6ttlichen-gericht-welches-die-der-gnade-spottenden-treffen-wird}{%
\paragraph{Warnung vor Abfall und vor dem göttlichen Gericht, welches
die der Gnade Spottenden treffen
wird}\label{warnung-vor-abfall-und-vor-dem-guxf6ttlichen-gericht-welches-die-der-gnade-spottenden-treffen-wird}}

\bibleverse{26} Denn wenn wir vorsätzlich✲ sündigen, nachdem wir die
Erkenntnis der Wahrheit erlangt haben, so bleibt uns fortan kein Opfer
für die Sünden mehr übrig, \bibleverse{27} sondern nur ein angstvolles
Warten auf das Gericht und die Gier des Feuers, das die Widerspenstigen
verzehren wird. \bibleverse{28} Wenn jemand das mosaische Gesetz
verworfen\textless sup title=``=~freventlich übertreten''\textgreater✲
hat, so muß er ohne Erbarmen auf (die Aussage von) zwei oder drei Zeugen
hin sterben\textless sup title=``4.Mose 15,30-31; 5.Mose
17,6''\textgreater✲: \bibleverse{29} eine wieviel härtere Strafe, denkt
doch, wird dem zuerkannt werden, der den Sohn Gottes mit Füßen getreten
und das Blut des Bundes, durch das er geheiligt worden ist, für
wertlos\textless sup title=``oder: gemein''\textgreater✲ geachtet und
mit dem Geist der Gnade Spott getrieben hat! \bibleverse{30} Wir kennen
ja den, der gesagt hat\textless sup title=``5.Mose
32,35-36''\textgreater✲: »Mein ist die Rache\textless sup title=``=~das
Strafamt''\textgreater✲, ich will vergelten«, und an einer anderen
Stelle\textless sup title=``Ps 135,14''\textgreater✲: »Der Herr wird
sein Volk richten.« \bibleverse{31} Schrecklich ist es, dem lebendigen
Gott in die Hände zu fallen\textless sup title=``vgl. 5.Mose
32,39-41''\textgreater✲.

\hypertarget{b-besondere-mahnungen}{%
\paragraph{b) Besondere Mahnungen}\label{b-besondere-mahnungen}}

\hypertarget{aa-mahnung-zur-glaubenstreue-und-hoffnungszuversicht-bei-den-zunehmenden-leiden-im-hinblick-auf-den-verheiuxdfenen-lohn}{%
\subparagraph{aa) Mahnung zur Glaubenstreue und Hoffnungszuversicht bei
den zunehmenden Leiden im Hinblick auf den verheißenen
Lohn}\label{aa-mahnung-zur-glaubenstreue-und-hoffnungszuversicht-bei-den-zunehmenden-leiden-im-hinblick-auf-den-verheiuxdfenen-lohn}}

\bibleverse{32} Denkt aber an die früheren Tage zurück, in denen ihr
nach empfangener Erleuchtung einen harten Leidenskampf geduldig
bestanden habt, \bibleverse{33} indem ihr teils selbst durch
Beschimpfungen und Drangsale zum öffentlichen Schauspiel gemacht wurdet,
teils an den Geschicken derer teilnehmen mußtet, die in solche Lagen
versetzt waren. \bibleverse{34} Ihr habt ja doch mit den Gefangenen
mitgelitten und den Raub eurer Habe mit Freuden hingenommen in der
Erkenntnis, daß ihr selbst einen wertvolleren und bleibenden Besitz
habt. \bibleverse{35} Werft also eure freudige Zuversicht nicht weg: sie
bringt ja eine hohe Lohnvergeltung mit sich! \bibleverse{36} Denn
standhaftes Ausharren\textless sup title=``oder: Geduld''\textgreater✲
tut euch not, damit ihr nach Erfüllung des göttlichen Willens das
verheißene Gut\textless sup title=``vgl. zu Kol 1,5''\textgreater✲
erlangt. \bibleverse{37} Denn es währt »nur noch eine kleine, ganz kurze
Zeit, dann wird der kommen, der kommen soll, und nicht auf sich warten
lassen. \bibleverse{38} Mein Gerechter aber wird aus
Glauben\textless sup title=``=~infolge seines Glaubens''\textgreater✲
das Leben haben«, und »wenn er kleinmütig zurückweicht, hat mein Herz
kein Wohlgefallen an ihm«\textless sup title=``Jes 26,20; Hab
2,3-4''\textgreater✲. \bibleverse{39} Wir aber haben nichts mit dem
Zurückweichen✲ zu tun, das zum Verderben führt, sondern (halten es) mit
dem Glauben, der das Leben gewinnt.

\hypertarget{section-10}{%
\section{11}\label{section-10}}

\bibleverse{1} Es ist aber der Glaube ein zuversichtliches Vertrauen auf
das, was man hofft, ein festes Überzeugtsein von Dingen\textless sup
title=``oder: Tatsachen''\textgreater✲, die man (mit Augen) nicht
sieht\textless sup title=``vgl. Joh 20,29''\textgreater✲.

\hypertarget{bb-alttestamentliche-vorbilder-solchen-glaubens}{%
\subparagraph{bb) Alttestamentliche Vorbilder solchen
Glaubens}\label{bb-alttestamentliche-vorbilder-solchen-glaubens}}

\bibleverse{2} Im Besitz solchen Glaubens haben nämlich die Altvordern
(das ehrende) Zeugnis (von Gott) erlangt. \bibleverse{3} Durch Glauben
erkennen✲ wir, daß die Welt durch Gottes Wort ins Dasein gerufen worden
ist; es sollte eben das jetzt Sichtbare nicht aus dem sinnlich
Wahrnehmbaren entstanden sein.

\hypertarget{drei-beispiele-von-glaubenshelden-aus-der-zeit-der-urvuxe4ter-von-abel-bis-noah}{%
\paragraph{Drei Beispiele von Glaubenshelden aus der Zeit der Urväter
von Abel bis
Noah}\label{drei-beispiele-von-glaubenshelden-aus-der-zeit-der-urvuxe4ter-von-abel-bis-noah}}

\bibleverse{4} Durch Glauben hat Abel Gott ein wertvolleres Opfer als
Kain dargebracht und durch ihn das Zeugnis erhalten, er sei ein
Gerechter, indem Gott (selbst) Zeugnis für seine Opfergaben ablegte; und
durch ihn\textless sup title=``d.h. mittels solchen
Glaubens''\textgreater✲ redet er auch jetzt noch nach seinem Tode.~--
\bibleverse{5} Durch Glauben\textless sup title=``=~wegen seines
Glaubens''\textgreater✲ wurde Henoch\textless sup title=``vgl. Jud
14''\textgreater✲ entrückt, damit er den Tod nicht sähe, und »er war
(auf Erden) nicht mehr zu finden, weil Gott ihn entrückt
hatte«\textless sup title=``1.Mose 5,24''\textgreater✲; denn vor seiner
Entrückung ist ihm bezeugt worden, daß er Gottes Wohlgefallen besessen
habe. \bibleverse{6} Ohne Glauben aber kann man (Gott) unmöglich
wohlgefallen; denn wer sich Gott nahen will, muß glauben, daß es einen
Gott gibt und daß er denen, die ihn suchen, ihren Lohn zukommen läßt.~--
\bibleverse{7} Durch Glauben hat Noah, als er die (göttliche) Weisung
erhalten hatte, in Besorgnis um die Dinge, die noch nicht sichtbar vor
Augen lagen, eine Arche zur Rettung seiner Familie gebaut; durch solchen
Glauben hat er der Welt das Urteil gesprochen und ist ein Erbe der
glaubensgemäßen Gerechtigkeit geworden.

\hypertarget{beispiele-aus-der-zeit-abrahams-und-der-seinen}{%
\paragraph{Beispiele aus der Zeit Abrahams und der
Seinen}\label{beispiele-aus-der-zeit-abrahams-und-der-seinen}}

\bibleverse{8} Durch Glauben leistete Abraham dem Ruf Folge, der ihn in
ein Land ziehen hieß, das er zum Erbbesitz empfangen sollte: er wanderte
aus, ohne zu wissen wohin. \bibleverse{9} Durch Glauben siedelte er sich
als Beisasse\textless sup title=``=~ohne Besitzrecht''\textgreater✲ in
dem verheißenen Lande wie in der Fremde an und wohnte in Zelten
samt\textless sup title=``oder: im Verein mit''\textgreater✲ Isaak und
Jakob, den Miterben der gleichen Verheißung; \bibleverse{10} denn er
wartete auf die Stadt, welche die festen Grundmauern hat, deren Erbauer
und Werkmeister Gott ist. \bibleverse{11} Durch Glauben empfing ebenso
auch Sara die Kraft, trotz ihres hohen Alters Mutter zu werden, weil sie
den für zuverlässig ansah, der ihr die Verheißung gegeben hatte.
\bibleverse{12} Daher sind auch von einem einzigen und zwar einem
bereits erstorbenen Manne Nachkommen entsprossen so zahlreich wie die
Sterne des Himmels und wie der Sand am Gestade des Meeres, den niemand
zählen kann.~-- \bibleverse{13} Im Glauben sind diese alle gestorben,
ohne die (Erfüllung der) Verheißungen erlangt zu haben; nur von ferne
haben sie diese gesehen und freudig begrüßt und bekannt, daß sie nur
Fremdlinge und Gäste auf der Erde seien; \bibleverse{14} denn wer ein
solches Bekenntnis ablegt, gibt dadurch zu erkennen, daß er ein
Vaterland\textless sup title=``oder: eine Heimat''\textgreater✲ sucht.
\bibleverse{15} Hätten sie nun dabei an jenes (Vaterland) gedacht, aus
dem sie ausgewandert waren, so hätten sie Zeit\textless sup
title=``oder: Gelegenheit''\textgreater✲ zur Rückkehr dorthin gehabt;
\bibleverse{16} so aber tragen sie nach einem besseren (Vaterland)
Verlangen, nämlich nach dem himmlischen. Daher schämt sich auch Gott
ihrer nicht, ihr Gott genannt zu werden; er hat ihnen ja (bereits) eine
Stadt (als Wohnung) bereitet.~-- \bibleverse{17} Durch Glauben hat
Abraham, als er versucht wurde, den Isaak zur Opferung dargebracht; ja
er wollte seinen einzigen (Sohn) opfern, obgleich er die Verheißungen
empfangen hatte \bibleverse{18} und ihm zugesagt worden war\textless sup
title=``1.Mose 21,12''\textgreater✲: »Nach\textless sup title=``oder:
in''\textgreater✲ Isaak soll dir Nachkommenschaft genannt werden«;
\bibleverse{19} er bedachte eben, daß Gott die Macht habe, auch aus den
Toten zu erwecken; daher hat er ihn auch als ein Gleichnis\textless sup
title=``oder: Sinnbild''\textgreater✲ zurückerhalten.~-- \bibleverse{20}
Durch Glauben segnete Isaak auch den Jakob und Esau im Hinblick auf
zukünftige Geschicke\textless sup title=``oder: Güter''\textgreater✲.~--
\bibleverse{21} Durch Glauben segnete Jakob bei seinem Sterben jeden der
(beiden) Söhne Josephs und betete zu Gott, auf die Spitze seines Stabes
gelehnt.~-- \bibleverse{22} Durch Glauben gedachte Joseph bei seinem
Lebensende des (einstigen) Auszuges der Israeliten und traf Anordnungen
in betreff seiner Gebeine.

\hypertarget{beispiele-aus-der-zeit-moses-und-josuas}{%
\paragraph{Beispiele aus der Zeit Moses und
Josuas}\label{beispiele-aus-der-zeit-moses-und-josuas}}

\bibleverse{23} Durch Glauben geschah es, daß Mose nach seiner Geburt
drei Monate lang von seinen Eltern verborgen gehalten wurde, weil sie
die Schönheit des Knäbleins sahen, und daß sie sich vor dem Befehl des
Königs nicht fürchteten. \bibleverse{24} Durch Glauben verschmähte es
Mose, als er herangewachsen war, ein Sohn der Tochter des Pharaos zu
heißen; \bibleverse{25} lieber wollte er mit dem Volke Gottes Drangsale
erleiden, als einen vorübergehenden Genuß von der Sünde haben:
\bibleverse{26} er achtete die Schmach Christi für einen größeren
Reichtum als die Schätze Ägyptens; denn er hatte die (himmlische)
Belohnung im Auge. \bibleverse{27} Durch Glauben verließ er Ägypten,
ohne Furcht vor dem Zorn des Königs; denn er wurde stark\textless sup
title=``oder: harrte aus''\textgreater✲, als ob er den Unsichtbaren
sähe. \bibleverse{28} Durch Glauben hat er das Passah und die
Besprengung (der Türpfosten) mit dem Blute angeordnet, damit der
Würgengel die Erstgeburten der Israeliten nicht antaste. \bibleverse{29}
Durch Glauben sind sie durch das Rote Meer gezogen wie über trockenes
Land, während die Ägypter ertranken, als sie denselben Versuch
machten.~-- \bibleverse{30} Durch Glauben geschah es, daß die Mauern
Jerichos einstürzten, nachdem man sieben Tage lang um sie herumgezogen
war.~-- \bibleverse{31} Durch Glauben kam die Dirne Rahab nicht zugleich
mit den Ungehorsamen✲ ums Leben, weil sie die Kundschafter friedlich bei
sich aufgenommen hatte.

\hypertarget{beispiele-von-glaubenshelden-aus-der-spuxe4teren-geschichte-israels}{%
\paragraph{Beispiele von Glaubenshelden aus der späteren Geschichte
Israels}\label{beispiele-von-glaubenshelden-aus-der-spuxe4teren-geschichte-israels}}

\bibleverse{32} Und was soll ich noch weiter sagen? Die Zeit würde mir
ja fehlen, wenn ich von Gideon und Barak, von Simson und Jephtha, von
David und Samuel und den Propheten reden wollte: \bibleverse{33} Durch
Glauben haben diese (Männer) Königreiche überwältigt, (vergeltende)
Gerechtigkeit geübt, (Erfüllung von) Verheißungen erlangt, Löwenrachen
verschlossen, \bibleverse{34} die Kraft des Feuers ausgelöscht; sie sind
der Schärfe des Schwertes entronnen, aus Kraftlosigkeit wieder
erstarkt\textless sup title=``oder: von Krankheiten
geheilt''\textgreater✲, im Kampfe Helden\textless sup title=``oder:
Sieger''\textgreater✲ geworden, haben Heere fremder Völker in die Flucht
geschlagen; \bibleverse{35} Frauen haben ihre Toten durch Auferweckung
zurückerhalten. Andere haben sich martern lassen und die
Befreiung\textless sup title=``oder: jede Schonung''\textgreater✲
zurückgewiesen, um einer desto herrlicheren Auferstehung teilhaftig zu
werden. \bibleverse{36} Wieder andere haben Verhöhnung und Geißelung,
dazu noch Ketten und Kerker über sich ergehen lassen; \bibleverse{37}
sie sind gesteinigt, gefoltert, zersägt, mit dem Henkerbeil hingerichtet
worden, sind in Schaffellen, in Ziegenhäuten unter Entbehrungen,
Drangsalen und Mißhandlungen umhergezogen; \bibleverse{38} sie, deren
die Welt nicht wert war, haben in Einöden und Gebirgen, in Höhlen und
Erdklüften umherirren müssen. \bibleverse{39} Und diese alle, denen doch
durch den Glauben ihr Zeugnis zuteil geworden ist, haben die (Erfüllung
der) Verheißung nicht erlangt, \bibleverse{40} weil Gott für uns etwas
Besseres zuvor ersehen hatte: sie sollten nicht ohne uns zur
(himmlischen) Vollendung gelangen.

\hypertarget{cc-ermahnung-zur-bewuxe4hrung-der-glaubenstreue-besonders-im-hinblick-auf-das-vorbild-jesu}{%
\subparagraph{cc) Ermahnung zur Bewährung der Glaubenstreue besonders im
Hinblick auf das Vorbild
Jesu}\label{cc-ermahnung-zur-bewuxe4hrung-der-glaubenstreue-besonders-im-hinblick-auf-das-vorbild-jesu}}

\hypertarget{section-11}{%
\section{12}\label{section-11}}

\bibleverse{1} So wollen denn auch wir, da wir uns von einer solchen
Wolke von Zeugen umgeben sehen, alles, was uns beschwert, und
(besonders) die uns so leicht umstrickende Sünde ablegen und mit
standhafter Ausdauer in dem uns obliegenden Wettkampfe laufen,
\bibleverse{2} indem wir dabei hinblicken auf Jesus, den Anfänger und
Vollender des Glaubens, der um den Preis der Freude, die ihn (als
Siegeslohn) erwartete, den Kreuzestod erduldet und die Schmach für
nichts geachtet, dann sich aber zur Rechten des Thrones Gottes gesetzt
hat. \bibleverse{3} Ja, denkt an ihn, der ein derartiges
Widersprechen\textless sup title=``oder: solche
Anfeindungen''\textgreater✲ von den Sündern gegen sich geduldig ertragen
hat, damit ihr (im Lauf) nicht ermattet und euren Mut nicht sinken laßt!

\hypertarget{dd-mahnung-die-leidensanfechtungen-als-fuxf6rderungsmittel-fuxfcr-das-glaubensleben-dienen-zu-lassen}{%
\subparagraph{dd) Mahnung, die Leidensanfechtungen als Förderungsmittel
für das Glaubensleben dienen zu
lassen}\label{dd-mahnung-die-leidensanfechtungen-als-fuxf6rderungsmittel-fuxfcr-das-glaubensleben-dienen-zu-lassen}}

\bibleverse{4} Denn bis jetzt habt ihr im Kampf gegen die
Sünde\textless sup title=``vgl. V.3''\textgreater✲ noch nicht bis aufs
Blut Widerstand geleistet \bibleverse{5} und habt das Mahnwort
vergessen, das zu euch wie zu Söhnen spricht\textless sup title=``Spr
3,11-12''\textgreater✲: »Mein Sohn, achte die Züchtigung des Herrn nicht
gering und verzage nicht, wenn du von ihm zurechtgewiesen\textless sup
title=``oder: heimgesucht''\textgreater✲ wirst; \bibleverse{6} denn wen
der Herr lieb hat, den züchtigt er und geißelt jeden Sohn, den er als
den seinigen annimmt.« \bibleverse{7} Haltet standhaft\textless sup
title=``oder: geduldig''\textgreater✲ aus, um euch erziehen zu lassen!
Gott verfährt mit euch wie mit Söhnen; denn wo wäre wohl ein Sohn, den
sein Vater nicht züchtigt? \bibleverse{8} Wenn ihr dagegen ohne
Züchtigung bliebet, die doch allen (anderen Söhnen) zuteil geworden ist,
so wäret ihr ja unechte Kinder und keine Söhne. \bibleverse{9} Ferner
(bedenkt): wir haben doch unter der Zucht unserer leiblichen Väter
gestanden und ihnen Ehrerbietung erwiesen; wollten\textless sup
title=``oder: sollten''\textgreater✲ wir uns da nicht viel eher dem
Vater der Geister unterwerfen und dadurch zum Leben gelangen?
\bibleverse{10} Denn jene haben doch nur für kurze Zeit nach ihrem
Ermessen Zucht (an uns) geübt, er aber zu unserm wahren Besten, damit
wir seiner Heiligkeit teilhaftig würden. \bibleverse{11} Jede Züchtigung
scheint uns freilich für den Augenblick nicht erfreulich, sondern
betrübend zu sein; hinterher aber läßt sie denen, die sich durch sie
haben üben lassen, die friedvolle✲ Frucht der Gerechtigkeit erwachsen.

\hypertarget{ee-mahnung-an-die-gemeinde-sich-aufzuraffen-und-sich-der-schwachen-und-gefuxe4hrdeten-glieder-anzunehmen}{%
\subparagraph{ee) Mahnung an die Gemeinde, sich aufzuraffen und sich der
schwachen und gefährdeten Glieder
anzunehmen}\label{ee-mahnung-an-die-gemeinde-sich-aufzuraffen-und-sich-der-schwachen-und-gefuxe4hrdeten-glieder-anzunehmen}}

\bibleverse{12} Darum »richtet die erschlafften Hände\textless sup
title=``oder: Arme''\textgreater✲ und die ermatteten Knie wieder
auf«\textless sup title=``Jes 35,3''\textgreater✲ \bibleverse{13} und
»stellt für eure Füße gerade Bahnen her«\textless sup title=``Spr
4,26''\textgreater✲, damit das Lahme\textless sup title=``d.h. die
lahmen Gemeindeglieder''\textgreater✲ nicht ganz vom rechten Wege
abkomme, sondern vielmehr geheilt✲ werde. \bibleverse{14} Trachtet
eifrig nach dem Frieden mit jedermann und nach der Heiligung, ohne die
niemand den Herrn schauen wird; \bibleverse{15} und gebt acht darauf,
daß niemand hinter der Gnade Gottes zurückbleibe\textless sup
title=``=~die Gnade versäume''\textgreater✲, daß keine »Wurzel voll
Bitterkeit\textless sup title=``=~kein giftiger
Wurzelschoß''\textgreater✲« aufwachse und Unheil anrichte\textless sup
title=``5.Mose 29,17''\textgreater✲ und gar viele durch sie
befleckt\textless sup title=``oder: vergiftet''\textgreater✲ werden;
\bibleverse{16} daß niemand ein ehebrecherischer\textless sup
title=``=~von Gott abtrünniger''\textgreater✲ oder verworfener Mensch
sei wie Esau, der für eine einzige Mahlzeit sein Erstgeburtsrecht
verkauft hat. \bibleverse{17} Ihr wißt ja, daß er auch später, als er
den Segen zum Erbe erlangen wollte, abgewiesen wurde; denn er fand
keinen Raum\textless sup title=``=~keine Möglichkeit''\textgreater✲ zu
einer Gesinnungsumkehr, obgleich er sie unter Tränen suchte.

\hypertarget{ff-nochmaliger-hinweis-auf-die-hoheit-des-neuen-bundes-und-auf-die-nahende-endentscheidung}{%
\subparagraph{ff) Nochmaliger Hinweis auf die Hoheit des neuen Bundes
und auf die nahende
Endentscheidung}\label{ff-nochmaliger-hinweis-auf-die-hoheit-des-neuen-bundes-und-auf-die-nahende-endentscheidung}}

\bibleverse{18} Denn ihr seid nicht zu einem mit Händen greifbaren und
im Feuer lodernden Berge herangetreten, nicht zu Wolkendunkel,
Finsternis und Gewittersturm, \bibleverse{19} nicht zu Posaunenschall
und Donnerworten, bei denen die Zuhörer die Bitte aussprachen, es möchte
nicht weiter zu ihnen geredet werden~-- \bibleverse{20} sie konnten
nämlich die Verordnung nicht ertragen\textless sup title=``die an sie
erging; 2.Mose 19,12-13''\textgreater✲: »Selbst ein Tier, das den Berg
berührt, soll gesteinigt werden!« --; \bibleverse{21} ja, so furchtbar
war die Erscheinung, daß (sogar) Mose erklärte\textless sup
title=``5.Mose 9,19''\textgreater✲: »Ich bin voller Furcht und Zittern!«
\bibleverse{22} Nein, ihr seid zu dem Berge Zion und zur Stadt des
lebendigen Gottes, dem himmlischen Jerusalem, herangetreten und zu
vielen Tausenden von Engeln, zu einer Festversammlung \bibleverse{23}
und zur Gemeinde der im Himmel aufgeschriebenen Erstgeborenen und zu
Gott, dem Richter über alle, und zu den Geistern der vollendeten
Gerechten, \bibleverse{24} und zu Jesus, dem Mittler des neuen Bundes,
und zum Blute der Besprengung, das Besseres\textless sup title=``oder:
wirksamer''\textgreater✲ redet als (das Blut) Abels.

\hypertarget{gg-die-fuxfcr-die-widerstrebenden-erschreckende-und-fuxfcr-die-gehorsamen-beseligende-herrlichkeit-des-eintritts-der-endzeit}{%
\subparagraph{gg) Die für die Widerstrebenden erschreckende und für die
Gehorsamen beseligende Herrlichkeit des Eintritts der
Endzeit}\label{gg-die-fuxfcr-die-widerstrebenden-erschreckende-und-fuxfcr-die-gehorsamen-beseligende-herrlichkeit-des-eintritts-der-endzeit}}

\bibleverse{25} Hütet euch, daß ihr den nicht ablehnt\textless sup
title=``d.h. euch nicht weigert, den anzuhören''\textgreater✲, der (zu
euch) redet! Denn wenn jene nicht ungestraft geblieben sind, die den
ablehnten, der sich ihnen auf Erden kundgab: wieviel weniger werden wir
dann davonkommen, wenn wir uns von dem abwenden, der vom Himmel her (zu
uns redet)! \bibleverse{26} Seine Stimme hat damals die Erde
erschüttert; jetzt aber hat er diese Verheißung gegeben\textless sup
title=``Hag 2,6''\textgreater✲: »Noch einmal werde ich nicht nur die
Erde, sondern auch den Himmel erbeben machen.« \bibleverse{27} Das Wort
»noch einmal« weist auf die Umwandlung dessen hin, das erschüttert wird,
weil es etwas Geschaffenes ist; es soll eben etwas Bleibendes entstehen,
das nicht erschüttert werden kann. \bibleverse{28} Darum wollen wir,
weil wir ein unerschütterliches Reich empfangen sollen, dankbar dafür
sein; denn dadurch dienen wir Gott so, wie es ihm wohlgefällig ist, mit
frommer Scheu und Furcht; \bibleverse{29} denn auch unser Gott ist ein
verzehrendes Feuer\textless sup title=``5.Mose 4,24''\textgreater✲.

\hypertarget{iii.-schluuxdfworte}{%
\subsection{III. Schlußworte}\label{iii.-schluuxdfworte}}

\hypertarget{einzelmahnungen-zur-bruderliebe-zur-sittenreinheit-und-zur-fuxf6rderung-des-gemeindelebens}{%
\subsubsection{1. Einzelmahnungen zur Bruderliebe, zur Sittenreinheit
und zur Förderung des
Gemeindelebens}\label{einzelmahnungen-zur-bruderliebe-zur-sittenreinheit-und-zur-fuxf6rderung-des-gemeindelebens}}

\hypertarget{a-allgemeine-ermahnungen}{%
\paragraph{a) Allgemeine Ermahnungen}\label{a-allgemeine-ermahnungen}}

\hypertarget{section-12}{%
\section{13}\label{section-12}}

\bibleverse{1} Bleibt fest in der Bruderliebe. \bibleverse{2} Vergeßt
die Gastfreundschaft nicht; denn durch diese haben einige, ohne es zu
wissen, Engel beherbergt. \bibleverse{3} Gedenkt der Gefangenen, als ob
ihr mitgefangen wäret, der Mißhandelten als solche, die gleichfalls noch
im Leibe sind\textless sup title=``=~den Leib an sich
tragen''\textgreater✲. \bibleverse{4} Die Ehe werde von allen in Ehren
gehalten und das Ehebett bleibe unbefleckt; denn Unzüchtige und
Ehebrecher wird Gott richten. \bibleverse{5} Euer Sinn\textless sup
title=``oder: Verhalten''\textgreater✲ sei frei von Geldgier; begnügt
euch mit dem, was euch gerade zu Gebote steht, denn er\textless sup
title=``d.h. Gott''\textgreater✲ selbst hat gesagt\textless sup
title=``Jos 1,5''\textgreater✲: »Ich will dir nimmermehr meine Hilfe
versagen und dich nicht verlassen«; \bibleverse{6} daher dürfen wir auch
zuversichtlich sagen\textless sup title=``Ps 118,6''\textgreater✲: »Der
Herr ist meine Hilfe, ich will mich nicht fürchten: was können Menschen
mir antun?«

\hypertarget{b-hauptmahnung-zur-treue-gegen-die-vorsteher-und-gegen-jesus-den-in-ewigkeit-bleibenden-und-den-beendiger-des-juxfcdischen-suxfcndopferdienstes}{%
\paragraph{b) Hauptmahnung zur Treue gegen die Vorsteher und gegen
Jesus, den in Ewigkeit Bleibenden und den Beendiger des jüdischen
Sündopferdienstes}\label{b-hauptmahnung-zur-treue-gegen-die-vorsteher-und-gegen-jesus-den-in-ewigkeit-bleibenden-und-den-beendiger-des-juxfcdischen-suxfcndopferdienstes}}

\bibleverse{7} Bleibt eurer Führer\textless sup title=``oder:
Vorsteher''\textgreater✲ eingedenk, die euch das Wort Gottes verkündigt
haben! Betrachtet immer wieder den Ausgang ihres Wandels und nehmt ihren
Glauben zum Vorbild\textless sup title=``=~folgt ihrem Glauben
nach''\textgreater✲! \bibleverse{8} Jesus Christus ist gestern und heute
derselbe und (ist's auch =~bleibt's auch) in Ewigkeit!~-- \bibleverse{9}
Laßt euch nicht durch mancherlei und fremdartige Lehren
fortreißen\textless sup title=``=~vom rechten Wege
abbringen''\textgreater✲! Denn es ist gut\textless sup title=``oder:
heilsam''\textgreater✲, daß das Herz durch Gnade gefestigt wird, nicht
durch Speisen, mit denen sich zu befassen noch niemandem Nutzen gebracht
hat. \bibleverse{10} Wir besitzen einen Opferaltar, von dem zu essen die
kein Recht haben, welche dem Zelt✲ obliegen; \bibleverse{11} denn von
den Tieren, deren Blut zur Sühnung der Sünde durch den Hohenpriester in
das Heiligtum hineingebracht wird, werden die Leiber außerhalb des
Lagers verbrannt\textless sup title=``3.Mose 16,27''\textgreater✲.
\bibleverse{12} Deshalb hat auch Jesus, um das Volk durch sein eigenes
Blut zu heiligen, außerhalb des Stadttores gelitten. \bibleverse{13} So
wollen wir denn zu ihm vor das Lager hinausgehen und seine Schmach
tragen. \bibleverse{14} Denn wir haben hier keine bleibende
Stadt\textless sup title=``=~Wohnstätte, Heimat''\textgreater✲, sondern
suchen die zukünftige. \bibleverse{15} So wollen wir also durch ihn Gott
allezeit ein Lobopfer darbringen, das heißt die »Frucht\textless sup
title=``=~den Tribut''\textgreater✲ unserer Lippen«\textless sup
title=``Hos 14,3; Jes 57,19''\textgreater✲, die seinen Namen bekennen.

\hypertarget{c-nochmalige-einzelermahnungen-besonders-bezuxfcglich-des-verhaltens-gegen-die-gemeindevorsteher}{%
\paragraph{c) Nochmalige Einzelermahnungen (besonders bezüglich des
Verhaltens gegen die
Gemeindevorsteher)}\label{c-nochmalige-einzelermahnungen-besonders-bezuxfcglich-des-verhaltens-gegen-die-gemeindevorsteher}}

\bibleverse{16} Wohlzutun und mitzuteilen vergeßt nicht, denn das sind
Opfer, an denen Gott Wohlgefallen hat.~-- \bibleverse{17} Gehorcht euren
Führern\textless sup title=``oder: Vorstehern''\textgreater✲ und fügt
euch ihnen, denn sie wachen über eure Seelen als solche, die einst
Rechenschaft abzulegen haben: möchten sie das mit Freuden tun und nicht
mit Seufzen, denn das wäre für euch kein Gewinn\textless sup
title=``oder: nicht heilsam''\textgreater✲!

\bibleverse{18} Betet für uns; denn wir sind uns bewußt, ein gutes
Gewissen zu haben, weil wir bestrebt sind, in allen Beziehungen einen
ehrbaren Wandel zu führen. \bibleverse{19} Um so dringender aber fordere
ich euch dazu auf, damit ich euch um so schneller zurückgegeben werde.

\hypertarget{abschluuxdf-des-briefes-segenswunsch-persuxf6nliche-mitteilungen-gruxfcuxdfe}{%
\subsubsection{2. Abschluß des Briefes (Segenswunsch, persönliche
Mitteilungen,
Grüße)}\label{abschluuxdf-des-briefes-segenswunsch-persuxf6nliche-mitteilungen-gruxfcuxdfe}}

\bibleverse{20} Der Gott des Friedens aber, der den großen✲ Hirten der
Schafe, unsern Herrn Jesus, von den Toten\textless sup title=``=~aus der
Totenwelt''\textgreater✲ heraufgeführt\textless sup title=``oder:
wiedergebracht''\textgreater✲ hat durch das Blut des ewigen Bundes,
\bibleverse{21} der möge euch in\textless sup title=``oder:
mit''\textgreater✲ allem Guten zur Ausrichtung seines Willens ausrüsten
und in uns das wirken, was (vor) ihm wohlgefällig ist, durch Jesus
Christus, dem die Herrlichkeit\textless sup title=``oder:
Ehre''\textgreater✲ gebührt in alle Ewigkeit! Amen.

\bibleverse{22} Ich ermahne euch aber, liebe Brüder: laßt euch dies
(mein) Mahnwort✲ gefallen; ich habe euch ja auch nur kurz geschrieben.
\bibleverse{23} Vernehmt die Mitteilung, daß unser Bruder Timotheus
freigelassen ist; mit ihm zusammen werde ich euch besuchen, wenn er bald
kommt. \bibleverse{24} Grüßt alle eure Vorsteher und alle Heiligen. Die
(Brüder) aus Italien lassen euch grüßen. \bibleverse{25} Die Gnade sei
mit euch allen!
