\hypertarget{der-erste-brief-des-apostels-paulus-an-die-korinther}{%
\section{DER ERSTE BRIEF DES APOSTELS PAULUS AN DIE
KORINTHER}\label{der-erste-brief-des-apostels-paulus-an-die-korinther}}

\hypertarget{eingang-des-briefes}{%
\subsubsection{Eingang des Briefes}\label{eingang-des-briefes}}

\hypertarget{a-zuschrift-und-segensgruuxdf}{%
\paragraph{a) Zuschrift und
Segensgruß}\label{a-zuschrift-und-segensgruuxdf}}

\hypertarget{section}{%
\section{1}\label{section}}

\bibleverse{1} Ich, Paulus, der ich zum Apostel Jesu Christi durch den
Willen Gottes berufen bin, und der Bruder Sosthenes \bibleverse{2}
senden unsern Gruß der Gemeinde Gottes in Korinth, den in Christus Jesus
Geheiligten\textless sup title=``oder: Geweihten =~zu Gott
Gehörigen''\textgreater✲, den (von Gott) berufenen Heiligen, samt allen
denen, welche den Namen unsers Herrn Jesus Christus anrufen an jeglichem
Ort, bei ihnen wie bei uns. \bibleverse{3} Gnade sei mit euch und Friede
von Gott, unserm Vater, und dem Herrn Jesus Christus!

\hypertarget{b-danksagung-des-apostels-fuxfcr-die-reiche-den-korinthern-widerfahrene-gnade-gottes-zuversichtliche-hoffnung-fuxfcr-die-zukunft}{%
\paragraph{b) Danksagung des Apostels für die reiche, den Korinthern
widerfahrene Gnade Gottes; zuversichtliche Hoffnung für die
Zukunft}\label{b-danksagung-des-apostels-fuxfcr-die-reiche-den-korinthern-widerfahrene-gnade-gottes-zuversichtliche-hoffnung-fuxfcr-die-zukunft}}

\bibleverse{4} Ich danke meinem Gott allezeit euretwegen für die Gnade
Gottes, die euch in Christus Jesus widerfahren ist. \bibleverse{5} Ihr
seid ja in ihm an allem\textless sup title=``oder: nach allen Seiten
hin''\textgreater✲ reich geworden\textless sup title=``oder: reich
ausgestattet''\textgreater✲, an aller Redegabe\textless sup
title=``oder: Lehre''\textgreater✲ und aller Erkenntnis, \bibleverse{6}
wie denn das Zeugnis von Christus in euch\textless sup title=``oder: bei
euch''\textgreater✲ fest gegründet worden ist\textless sup title=``oder:
festen Boden gewonnen hat''\textgreater✲. \bibleverse{7} Ihr steht
infolgedessen an keiner Gnadengabe (hinter anderen Gemeinden) zurück,
während ihr auf die Offenbarung✲ unsers Herrn Jesus Christus wartet,
\bibleverse{8} der euch auch Festigkeit verleihen wird bis ans Ende, so
daß ihr am Tage unsers Herrn Jesus Christus frei von Tadel\textless sup
title=``oder: Anklage''\textgreater✲ dastehen könnt. \bibleverse{9} Treu
ist Gott, durch den ihr in die Gemeinschaft mit seinem Sohne Jesus
Christus, unserm Herrn, berufen worden seid.

\hypertarget{i.-miuxdfstuxe4nde-in-der-gemeinde-110-620}{%
\subsection{I. Mißstände in der Gemeinde
(1,10-6,20)}\label{i.-miuxdfstuxe4nde-in-der-gemeinde-110-620}}

\hypertarget{a.-das-parteiwesen-in-der-gemeinde-110-421}{%
\subsection{A. Das Parteiwesen in der Gemeinde
(1,10-4,21)}\label{a.-das-parteiwesen-in-der-gemeinde-110-421}}

\hypertarget{feststellung-des-tatbestandes-und-die-widersinnigkeit-des-parteitreibens}{%
\subsubsection{1. Feststellung des Tatbestandes und die Widersinnigkeit
des
Parteitreibens}\label{feststellung-des-tatbestandes-und-die-widersinnigkeit-des-parteitreibens}}

\bibleverse{10} Ich ermahne euch aber, liebe Brüder, unter
Berufung\textless sup title=``oder: Hinweis''\textgreater✲ auf den Namen
unsers Herrn Jesus Christus: Führt allesamt einerlei Rede und laßt keine
Spaltungen unter euch herrschen, sondern steht in gleicher Gesinnung und
in derselben Überzeugung fest geschlossen da! \bibleverse{11} Es ist mir
nämlich über euch, meine Brüder, von den Leuten der Chloe berichtet
worden, daß Streitigkeiten\textless sup title=``oder:
Parteiungen''\textgreater✲ unter euch bestehen; \bibleverse{12} ich
meine damit nämlich den Übelstand, daß jeder von euch (als seine Losung)
ausspricht: »Ich halte zu Paulus«, »Ich dagegen zu Apollos«, »Ich aber
zu Kephas✲«, »Ich aber zu Christus«. \bibleverse{13} Ist Christus in
Stücke zerteilt worden? Ist etwa Paulus für euch gekreuzigt worden? Oder
seid ihr auf den Namen des Paulus getauft worden? \bibleverse{14} Ich
sage (Gott) Dank dafür, daß ich niemand von euch außer Krispus und Gaius
getauft habe; \bibleverse{15} so kann niemand behaupten, ihr seiet auf
meinen Namen getauft worden. \bibleverse{16} Doch ja, ich habe
(außerdem) auch noch die Hausgenossen des Stephanas getauft; sonst aber
wüßte ich nicht, daß ich noch irgendeinen andern getauft hätte.
\bibleverse{17} Christus hat mich ja nicht ausgesandt, um zu taufen,
sondern um die Heilsbotschaft zu verkündigen, und zwar nicht mit hoher
Redeweisheit, damit das Kreuz Christi nicht entleert werde\textless sup
title=``d.h. seiner Kraft oder Bedeutung verlustig gehe''\textgreater✲.

\hypertarget{das-wesen-und-die-wirkung-des-wortes-vom-kreuz}{%
\subsubsection{2. Das Wesen und die Wirkung des Wortes vom
Kreuz}\label{das-wesen-und-die-wirkung-des-wortes-vom-kreuz}}

\hypertarget{a-das-wort-vom-kreuz-ist-eine-gotteskraft-der-weltweisheit-entgegengesetzt-und-von-der-welt-als-torheit-geachtet}{%
\paragraph{a) Das Wort vom Kreuz ist eine Gotteskraft, der Weltweisheit
entgegengesetzt und von der Welt als Torheit
geachtet}\label{a-das-wort-vom-kreuz-ist-eine-gotteskraft-der-weltweisheit-entgegengesetzt-und-von-der-welt-als-torheit-geachtet}}

\bibleverse{18} Denn das Wort vom Kreuz ist für die, welche
verlorengehen, eine Torheit, für die aber, welche gerettet werden, für
uns, ist es eine Gotteskraft. \bibleverse{19} Denn es steht
geschrieben\textless sup title=``Jes 29,14''\textgreater✲: »Ich will die
Weisheit der Weisen\textless sup title=``=~die Gelehrsamkeit der
Gelehrten''\textgreater✲ zuschanden machen und den Verstand der
Verständigen\textless sup title=``oder: die Klugheit der
Klugen''\textgreater✲ als nichtig abtun.« \bibleverse{20} Wo ist denn
ein Weiser? Wo ein Gelehrter? Wo ein Forscher\textless sup title=``oder:
Wortstreiter''\textgreater✲ dieser Weltzeit? Hat Gott nicht die Weisheit
der Welt als Torheit hingestellt? \bibleverse{21} Weil nämlich die Welt
da, wo Gottes Weisheit tatsächlich vorlag\textless sup title=``oder:
sich offenbarte''\textgreater✲, Gott vermittelst ihrer Weisheit nicht
erkannte, hat es Gott gefallen, durch die Torheit der Predigt die zu
retten, welche Glauben haben. \bibleverse{22} Denn während einerseits
die Juden Wunderzeichen fordern, andrerseits die Griechen\textless sup
title=``vgl. Röm 1,16''\textgreater✲ Weltweisheit verlangen,
\bibleverse{23} verkünden wir dagegen Christus als den Gekreuzigten, der
für die Juden ein Ärgernis und für die Heiden eine Torheit ist;
\bibleverse{24} denen aber, die berufen sind, sowohl den Juden als auch
den Griechen, (verkünden wir) Christus als Gotteskraft und
Gottesweisheit. \bibleverse{25} Denn die Torheit Gottes\textless sup
title=``d.h. die von Gott kommt oder stammt''\textgreater✲ ist weiser
als die Menschen (sind), und die Schwachheit Gottes\textless sup
title=``d.h. die Gott wirkt''\textgreater✲ ist der Stärke der Menschen
überlegen.

\hypertarget{b-beweis-aus-dem-tatsuxe4chlichen-bestand-der-von-gott-in-korinth-berufenen-christlichen-gemeinde}{%
\paragraph{b) Beweis aus dem tatsächlichen Bestand der von Gott in
Korinth berufenen christlichen
Gemeinde}\label{b-beweis-aus-dem-tatsuxe4chlichen-bestand-der-von-gott-in-korinth-berufenen-christlichen-gemeinde}}

\bibleverse{26} Seht euch doch einmal eure Berufung an, liebe Brüder! Da
sind nicht viele Weise✲ nach dem Fleisch\textless sup title=``d.h. im
Sinn der Welt''\textgreater✲ unter euch, nicht viele einflußreiche
Personen, nicht viele Hochgeborene; \bibleverse{27} nein, was der Welt
als töricht✲ gilt, das hat Gott erwählt, um die Weisen✲ zu beschämen;
und was der Welt als schwach gilt, das hat Gott erwählt, um das Starke
zu beschämen; \bibleverse{28} und was der Welt als niedrig und
verächtlich\textless sup title=``oder: bedeutungslos''\textgreater✲
gilt, das hat Gott erwählt, ja das, was der Welt nichts gilt, um das,
was ihr etwas gilt, zunichte zu machen: \bibleverse{29} es soll sich
eben kein Fleisch✲ vor Gott rühmen können. \bibleverse{30} Ihm habt ihr
es also zu verdanken, daß ihr in Christus Jesus seid, der uns von Gott
her zur Weisheit gemacht worden ist wie auch zur Gerechtigkeit und
Heiligung und zur Erlösung, \bibleverse{31} damit das Schriftwort seine
Geltung behalte\textless sup title=``Jer 9,23''\textgreater✲: »Wer sich
rühmen will, der rühme sich des Herrn!«

\hypertarget{dem-entsprach-auch-die-predigt-des-paulus-zu-korinth}{%
\subsubsection{3. Dem entsprach auch die Predigt des Paulus zu
Korinth}\label{dem-entsprach-auch-die-predigt-des-paulus-zu-korinth}}

\hypertarget{a-die-predigtweise-des-paulus-bei-der-gruxfcndung-der-gemeinde-war-anspruchslos-und-ohne-weltweisheit}{%
\paragraph{a) Die Predigtweise des Paulus bei der Gründung der Gemeinde
war anspruchslos und ohne
Weltweisheit}\label{a-die-predigtweise-des-paulus-bei-der-gruxfcndung-der-gemeinde-war-anspruchslos-und-ohne-weltweisheit}}

\hypertarget{section-1}{%
\section{2}\label{section-1}}

\bibleverse{1} So bin denn auch ich, als ich zu euch kam, liebe Brüder,
nicht in der Absicht gekommen, euch mit überwältigender Redekunst oder
Weisheit das Zeugnis Gottes\textless sup title=``oder: von Gott, oder:
über Gott''\textgreater✲ zu verkündigen; \bibleverse{2} nein, ich hatte
mir vorgenommen, kein anderes Wissen bei euch zu zeigen als das von
Jesus Christus, und zwar dem Gekreuzigten. \bibleverse{3} Dabei trat ich
mit (dem Gefühl der) Schwachheit und mit Furcht und großer Ängstlichkeit
bei euch auf, \bibleverse{4} und meine Rede und meine Predigt erfolgte
nicht mit eindrucksvollen Weisheitsworten, sondern mit dem Ausweis von
Geist und Kraft; \bibleverse{5} denn euer Glaube sollte nicht auf
Menschenweisheit, sondern auf Gotteskraft beruhen\textless sup
title=``oder: gegründet sein''\textgreater✲.

\hypertarget{b-gleichwohl-ist-die-predigt-des-paulus-mit-einer-huxf6heren-guxf6ttlichen-weisheit-ausgestattet-die-aber-nur-den-geistesmenschen-erkennbar-ist}{%
\paragraph{b) Gleichwohl ist die Predigt des Paulus mit einer höheren
(göttlichen) Weisheit ausgestattet, die aber nur den Geistesmenschen
erkennbar
ist}\label{b-gleichwohl-ist-die-predigt-des-paulus-mit-einer-huxf6heren-guxf6ttlichen-weisheit-ausgestattet-die-aber-nur-den-geistesmenschen-erkennbar-ist}}

\hypertarget{aa-die-geheimnisvolle-weisheit-gottes-fuxfcr-die-vollkommenen}{%
\subparagraph{aa) Die geheimnisvolle Weisheit Gottes für die
Vollkommenen}\label{aa-die-geheimnisvolle-weisheit-gottes-fuxfcr-die-vollkommenen}}

\bibleverse{6} Was wir aber vortragen, ist dennoch Weisheit -- bei den
Vollkommenen\textless sup title=``=~Vorgeschrittenen, geistlich
Gereiften''\textgreater✲, jedoch nicht die Weisheit dieser Weltzeit,
auch nicht die der Machthaber dieser Weltzeit, die dem Untergang
verfallen: \bibleverse{7} nein, wir tragen Gottes geheimnisvolle,
verborgene Weisheit vor, die Gott vor allen Weltzeiten zu unserer
Verherrlichung vorherbestimmt\textless sup title=``oder: im voraus
festgelegt''\textgreater✲ hat. \bibleverse{8} Diese (Weisheit) hat
keiner von den Machthabern dieser Weltzeit erkannt; denn hätten sie sie
erkannt, so hätten sie den Herrn der Herrlichkeit nicht ans Kreuz
geschlagen; \bibleverse{9} vielmehr (predigen wir so), wie geschrieben
steht: »Was kein Auge gesehen und kein Ohr gehört hat und wovon keines
Menschen Herz eine Ahnung gehabt hat, nämlich das, was Gott denen
bereitet hat, die ihn lieben.«

\hypertarget{bb-die-ergruxfcndung-und-aufnahme-dieser-weisheit-ist-nur-den-geistesmenschen-muxf6glich}{%
\subparagraph{bb) Die Ergründung und Aufnahme dieser Weisheit ist nur
den Geistesmenschen
möglich}\label{bb-die-ergruxfcndung-und-aufnahme-dieser-weisheit-ist-nur-den-geistesmenschen-muxf6glich}}

\bibleverse{10} Uns aber hat Gott dies durch den Geist geoffenbart; denn
der Geist erforscht alles, selbst die Tiefen Gottes. \bibleverse{11}
Denn wer von den Menschen kennt das innere Wesen eines Menschen? Doch
nur der Geist, der in dem betreffenden Menschen wohnt. Ebenso hat auch
niemand das innere Wesen Gottes erkannt als nur der Geist Gottes.
\bibleverse{12} Wir aber haben nicht den Geist der Welt empfangen,
sondern den Geist, der aus Gott ist, um das zu erkennen, was uns von
Gott aus Gnaden geschenkt worden ist. \bibleverse{13} Und davon reden
wir auch, (aber) nicht mit Worten, wie menschliche Weisheit sie lehrt,
sondern mit solchen, wie der Geist sie lehrt✲, indem wir geistgewirkten
Inhalt mit geistgewirkter Sprache verbinden. \bibleverse{14} Der
seelische✲ Mensch aber nimmt nichts an, was vom Geiste Gottes kommt,
denn es gilt ihm als Torheit, und er ist nicht imstande, es zu
verstehen, weil es geistlich beurteilt werden muß. \bibleverse{15} Der
Geistesmensch dagegen beurteilt alles zutreffend, während er selbst von
niemand zutreffend beurteilt wird. \bibleverse{16} »Denn wer hat den
Sinn des Herrn erkannt, daß er ihn unterweisen\textless sup
title=``oder: beraten''\textgreater✲ könnte?«\textless sup title=``Jes
40,13''\textgreater✲ Wir aber haben den Sinn Christi.

\hypertarget{c-paulus-hat-bisher-den-korinthern-wegen-ihrer-durch-das-parteiunwesen-erwiesenen-unreife-nicht-die-volle-weisheit-verkuxfcnden-kuxf6nnen}{%
\paragraph{c) Paulus hat bisher den Korinthern wegen ihrer durch das
Parteiunwesen erwiesenen Unreife nicht die volle Weisheit verkünden
können}\label{c-paulus-hat-bisher-den-korinthern-wegen-ihrer-durch-das-parteiunwesen-erwiesenen-unreife-nicht-die-volle-weisheit-verkuxfcnden-kuxf6nnen}}

\hypertarget{section-2}{%
\section{3}\label{section-2}}

\bibleverse{1} So habe denn auch ich, liebe Brüder, (damals) zu euch
nicht als zu Geistesmenschen✲ reden können, sondern nur als zu
fleischlich gesinnten Menschen, nur als zu unmündigen Kindern in
Christus. \bibleverse{2} Milch habe ich euch zu trinken gegeben, nicht
feste Speise; denn die konntet ihr noch nicht vertragen. Ja, ihr könnt
sie auch jetzt noch nicht vertragen; \bibleverse{3} ihr seid ja immer
noch fleischlich gesinnt. Denn solange noch Eifersucht und Streit unter
euch herrschen, seid ihr da nicht fleischlich gerichtet und wandelt wie
Menschen (gewöhnlichen Schlages)? \bibleverse{4} Wenn nämlich der eine
erklärt: »Ich halte zu Paulus«, der andere: »Ich zu Apollos«, seid ihr
da nicht Menschen (gewöhnlichen Schlages)?

\hypertarget{richtige-schuxe4tzung-der-lehrer-d.h.-missionare}{%
\subsubsection{4. Richtige Schätzung der Lehrer (d.h.
Missionare)}\label{richtige-schuxe4tzung-der-lehrer-d.h.-missionare}}

\hypertarget{a-sie-sind-diener-und-mitarbeiter-gottes}{%
\paragraph{a) Sie sind Diener und Mitarbeiter
Gottes}\label{a-sie-sind-diener-und-mitarbeiter-gottes}}

\bibleverse{5} Was ist denn Apollos, und was ist Paulus? Diener sind
sie, durch die ihr zum Glauben gekommen seid; und zwar dient jeder (von
uns beiden) so, wie der Herr es ihm verliehen hat: \bibleverse{6} ich
habe gepflanzt, Apollos hat begossen, Gott aber hat das Wachstum
gegeben. \bibleverse{7} Somit ist weder der Pflanzende noch der
Begießende (für sich) etwas, sondern nur Gott, der das Wachstum
verleiht. \bibleverse{8} Der Pflanzende hingegen und der Begießende sind
einer wie der andere, doch wird jeder seinen besonderen\textless sup
title=``=~den ihm zustehenden''\textgreater✲ Lohn empfangen nach seiner
besonderen Arbeit. \bibleverse{9} Denn Gottes Mitarbeiter✲ sind wir;
Gottes Ackerfeld, Gottes Bau seid ihr.

\hypertarget{b-jeder-lehrer-sehe-zu-dauxdf-seine-arbeit-im-feuer-des-dereinstigen-guxf6ttlichen-gerichts-bestehe}{%
\paragraph{b) Jeder Lehrer sehe zu, daß seine Arbeit im Feuer des
dereinstigen göttlichen Gerichts
bestehe!}\label{b-jeder-lehrer-sehe-zu-dauxdf-seine-arbeit-im-feuer-des-dereinstigen-guxf6ttlichen-gerichts-bestehe}}

\bibleverse{10} Nach der mir von Gott verliehenen Gnade habe ich als ein
kundiger Baumeister den Grund (bei euch) gelegt; ein anderer baut darauf
weiter; jeder aber möge zusehen, wie er darauf weiterbaut!
\bibleverse{11} Denn einen anderen Grund kann niemand legen als den, der
gelegt ist, und der ist Jesus Christus. \bibleverse{12} Ob aber jemand
auf diesen Grund weiterbaut mit Gold, Silber und kostbaren Steinen,
(oder aber) mit Holz, Heu und Stroh~-- \bibleverse{13} eines jeden
Arbeit wird (dereinst) offenbar werden; denn der Gerichtstag wird es
ausweisen, weil er sich in Feuer\textless sup title=``oder: als ein
Feuer''\textgreater✲ offenbart; und wie die Arbeit eines jeden
beschaffen ist, wird eben das Feuer erproben✲. \bibleverse{14} Wenn das
Werk jemandes, das er darauf weitergebaut hat, (in dem Feuer)
standhält\textless sup title=``=~Bestand hat''\textgreater✲, so wird er
Lohn empfangen; \bibleverse{15} wenn aber das Werk jemandes verbrennt,
so wird er den Schaden zu tragen haben\textless sup title=``oder: den
Lohn einbüßen''\textgreater✲: er selbst zwar wird gerettet
werden\textless sup title=``=~mit dem Leben davonkommen''\textgreater✲,
aber nur so, wie durchs Feuer hindurch.

\hypertarget{c-um-die-gemeinde-handelt-es-sich-und-dauxdf-ihr-richtig-gedient-werde}{%
\paragraph{c) Um die Gemeinde handelt es sich und daß ihr richtig
gedient
werde}\label{c-um-die-gemeinde-handelt-es-sich-und-dauxdf-ihr-richtig-gedient-werde}}

\bibleverse{16} Wißt ihr nicht, daß ihr (als Gemeinde) ein Tempel Gottes
seid und daß der Geist Gottes in\textless sup title=``oder:
bei''\textgreater✲ euch wohnt? \bibleverse{17} Wenn jemand den Tempel
Gottes verderbt, den wird Gott verderben; denn der Tempel Gottes ist
heilig, und der seid ihr! \bibleverse{18} Niemand betrüge sich selbst!
Wenn jemand unter euch in (den Dingen) dieser Weltzeit weise zu sein
vermeint, so muß er erst ein Tor werden, um dann wirklich zur Weisheit
zu gelangen; \bibleverse{19} denn die Weisheit dieser Welt ist in Gottes
Augen Torheit. Es steht ja doch geschrieben\textless sup title=``Hiob
5,13''\textgreater✲: »Er\textless sup title=``d.h. Gott''\textgreater✲
fängt die Weisen in ihrer Schlauheit«; \bibleverse{20} und an einer
andern Stelle\textless sup title=``Ps 94,11''\textgreater✲: »Der Herr
kennt die Gedanken der Weisen, daß sie nichtig sind.« \bibleverse{21}
Daher mache niemand viel Rühmens von Menschen! Alles gehört ja euch zu
eigen: \bibleverse{22} Paulus ebensowohl wie Apollos und Kephas✲, die
ganze Welt, das Leben ebensowohl wie der Tod, das Gegenwärtige wie das
Zukünftige: alles gehört euch; \bibleverse{23} ihr aber gehört Christus
an, und Christus gehört zu Gott.

\hypertarget{section-3}{%
\section{4}\label{section-3}}

\bibleverse{1} Dafür halte uns jedermann, nämlich für Diener Christi und
für Verwalter der Geheimnisse Gottes. \bibleverse{2} Bei dieser Sachlage
verlangt man allerdings von den Verwaltern, daß ein solcher treu
erfunden werde.

\hypertarget{persuxf6nliche-schluuxdfbemerkungen}{%
\subsubsection{5. Persönliche
Schlußbemerkungen}\label{persuxf6nliche-schluuxdfbemerkungen}}

\hypertarget{a-paulus-weiuxdf-sich-nur-dem-herrn-verantwortlich}{%
\paragraph{a) Paulus weiß sich nur dem Herrn
verantwortlich}\label{a-paulus-weiuxdf-sich-nur-dem-herrn-verantwortlich}}

\bibleverse{3} Doch was mich betrifft, so ist es mir etwas ganz
Geringes\textless sup title=``=~durchaus gleichgültig''\textgreater✲, ob
ich von euch oder von (sonst) einem menschlichen Gerichtstage✲ ein
Urteil empfange; ja, ich gebe nicht einmal selbst ein Urteil über mich
ab. \bibleverse{4} Denn ich bin mir wohl keiner Schuld bewußt, aber
dadurch bin ich noch nicht gerechtfertigt; nein, der Herr ist's, der das
Urteil über mich abgibt. \bibleverse{5} Daher urteilet über nichts vor
der Zeit, bis der Herr kommt, der auch das im Dunkel Verborgene ans
Licht bringen und die Gedanken der Herzen offenbar machen wird; und dann
wird einem jeden das ihm gebührende Lob von Gott her zuteil werden.

\hypertarget{b-paulus-huxe4lt-den-korinthern-ihre-selbstuxfcberhebung-gegenuxfcber-der-leidvollen-lage-der-apostel-vor}{%
\paragraph{b) Paulus hält den Korinthern ihre Selbstüberhebung gegenüber
der leidvollen Lage der Apostel
vor}\label{b-paulus-huxe4lt-den-korinthern-ihre-selbstuxfcberhebung-gegenuxfcber-der-leidvollen-lage-der-apostel-vor}}

\bibleverse{6} Ich habe das Gesagte aber, liebe Brüder, auf mich und
Apollos bezogen (und zwar) mit Rücksicht auf euch, damit ihr an uns
(beiden) das Wort\textless sup title=``oder: die Mahnung''\textgreater✲
verstehen lernt: »Nicht hinausgehen über das, was geschrieben steht!«,
damit sich niemand (von euch) für den einen (Lehrer) gegen den andern
aufblähen\textless sup title=``oder: ereifern =~in die Brust
werfen''\textgreater✲ möge. \bibleverse{7} Denn wer ist es, der dir ein
Vorrecht\textless sup title=``oder: den Vorrang''\textgreater✲ gibt? Was
besitzest du aber, das du nicht empfangen hättest? Wenn du es aber
empfangen hast, was rühmst du dich, als ob du es nicht empfangen
hättest?

\bibleverse{8} Ihr (freilich) seid bereits gesättigt, seid bereits im
Besitz des Reichtums, habt es ohne unser Zutun zu königlicher Herrschaft
(im Gottesreich) gebracht! Wollte Gott, ihr hättet es wirklich schon zu
königlicher Herrschaft gebracht, damit auch wir mit euch zum Herrschen
kämen! \bibleverse{9} Denn ich bin der Ansicht, Gott habe uns Aposteln
den letzten Platz zugewiesen wie zum Tode verurteilten (Verbrechern);
wir sind ja der (ganzen) Welt, Engeln sowohl wie Menschen, ein
Schaustück geworden! \bibleverse{10} Wir stehen als Toren da um Christi
willen, ihr aber seid kluge Leute in Christus; wir sind schwach, ihr
aber stark; ihr steht in Ehren und wir in Verachtung. \bibleverse{11}
Bis zur jetzigen Stunde leiden wir Hunger und Durst, haben keine
Kleidung und müssen uns mit Fäusten schlagen lassen, führen ein unstetes
Leben \bibleverse{12} und mühen uns ab, um mit eigenen Händen das
tägliche Brot zu verdienen. Schmäht man uns, so segnen wir; verfolgt man
uns, so halten wir geduldig still; \bibleverse{13} beschimpft man uns,
so geben wir gute Worte: wie der Kehricht der Welt, wie der allgemeine
Auswurf sind wir bis heute geworden.

\hypertarget{c-hinweis-des-paulus-auf-sein-persuxf6nliches-verhuxe4ltnis-zur-gemeinde-als-ihres-geistlichen-vaters}{%
\paragraph{c) Hinweis des Paulus auf sein persönliches Verhältnis zur
Gemeinde (als ihres geistlichen
Vaters)}\label{c-hinweis-des-paulus-auf-sein-persuxf6nliches-verhuxe4ltnis-zur-gemeinde-als-ihres-geistlichen-vaters}}

\bibleverse{14} Ich schreibe dies nicht, um euch zu beschämen, sondern
um euch als meine geliebten Kinder zu ermahnen\textless sup
title=``oder: zurechtzuweisen''\textgreater✲. \bibleverse{15} Denn wenn
ihr auch viele tausend Lehrmeister in Christus hättet, so habt ihr doch
nicht viele Väter; denn in Christus Jesus bin ich durch die
(Verkündigung der) Heilsbotschaft euer (geistlicher) Vater geworden.
\bibleverse{16} Daher rufe ich euch mahnend zu: »Nehmt mich zum
Vorbild!«

\hypertarget{d-sendung-des-timotheus-paulus-kuxfcndigt-den-korinthern-seinen-besuch-behufs-feststellung-ihres-christenstandes-an}{%
\paragraph{d) Sendung des Timotheus; Paulus kündigt den Korinthern
seinen Besuch behufs Feststellung ihres Christenstandes
an}\label{d-sendung-des-timotheus-paulus-kuxfcndigt-den-korinthern-seinen-besuch-behufs-feststellung-ihres-christenstandes-an}}

\bibleverse{17} Eben deswegen habe ich auch Timotheus zu euch gesandt,
mein im Herrn geliebtes und treues Kind; er wird euch an meine Wege in
Christus Jesus erinnern, wie ich sie überall in jeder Gemeinde lehre.
\bibleverse{18} Freilich haben sich einige (bei euch) in der Annahme,
daß ich nicht zu euch kommen würde, in die Brust geworfen;
\bibleverse{19} ich werde aber, wenn es des Herrn Wille ist, bald zu
euch kommen und dann nicht die Worte derer prüfen, die sich so in die
Brust geworfen haben, sondern ihre Kraft; \bibleverse{20} denn nicht auf
Worten beruht\textless sup title=``oder: gründet sich''\textgreater✲ das
Reich Gottes, sondern auf Kraft. \bibleverse{21} Was wünscht ihr nun?
Soll ich mit dem Stock\textless sup title=``oder: der
Rute''\textgreater✲ zu euch kommen oder mit Liebe und dem Geiste der
Sanftmut✲?

\hypertarget{b.-sittliche-miuxdfstuxe4nde-in-der-gemeinde-und-ihre-beseitigung-51-620}{%
\subsection{B. Sittliche Mißstände in der Gemeinde und ihre Beseitigung
(5,1-6,20)}\label{b.-sittliche-miuxdfstuxe4nde-in-der-gemeinde-und-ihre-beseitigung-51-620}}

\hypertarget{ruxfcge-wegen-eines-besonders-schreienden-falles-von-unzucht}{%
\subsubsection{1. Rüge wegen (eines besonders schreienden Falles von)
Unzucht}\label{ruxfcge-wegen-eines-besonders-schreienden-falles-von-unzucht}}

\hypertarget{a-schwerer-tadel-der-von-der-gemeinde-gegen-einen-unzuxfcchtigen-geuxfcbten-nachsicht}{%
\paragraph{a) Schwerer Tadel der von der Gemeinde gegen einen
Unzüchtigen geübten
Nachsicht}\label{a-schwerer-tadel-der-von-der-gemeinde-gegen-einen-unzuxfcchtigen-geuxfcbten-nachsicht}}

\hypertarget{section-4}{%
\section{5}\label{section-4}}

\bibleverse{1} Allgemein hört man von Unzucht bei euch, und noch dazu
von einer solchen Unzucht, wie sie selbst bei den Heiden unerhört ist,
daß nämlich einer die Frau seines Vaters zum Weibe genommen hat!
\bibleverse{2} Und da wollt ihr euch noch in die Brust werfen und habt
nicht vielmehr Leid getragen, damit der Schuldige aus eurer Mitte
entfernt werde? \bibleverse{3} Nun -- ich, der ich leiblich zwar
abwesend, aber mit meinem Geiste bei euch anwesend bin, habe über diesen
Menschen, der sich so schwer vergangen hat, bereits Gericht gehalten,
als ob ich persönlich bei euch wäre. \bibleverse{4} Wir wollen uns
nämlich im Namen des Herrn Jesus versammeln, ihr und mein Geist im
Verein mit der Kraft unsers Herrn Jesus, \bibleverse{5} und wollen den
betreffenden Menschen dem Satan zur Vernichtung des Fleisches übergeben,
damit der Geist am Tage des Herrn Jesus gerettet werde.

\hypertarget{b-allgemeine-mahnung-zu-sittlicher-reinheit-unter-hinweis-auf-den-opfertod-jesu-des-passahlammes}{%
\paragraph{b) Allgemeine Mahnung zu sittlicher Reinheit unter Hinweis
auf den Opfertod Jesu, »des
Passahlammes«}\label{b-allgemeine-mahnung-zu-sittlicher-reinheit-unter-hinweis-auf-den-opfertod-jesu-des-passahlammes}}

\bibleverse{6} Euer Ruhm✲ ist nicht schön! Wißt ihr nicht, daß schon ein
wenig Sauerteig den ganzen Teig durchsäuert? \bibleverse{7} Schafft den
alten Sauerteig weg, damit ihr (durchweg) ein neuer Teig seid; ihr seid
ja doch (als Christen) frei von allem Sauerteig; denn es ist ja auch
unser Passahlamm geschlachtet worden: Christus. \bibleverse{8} Darum
laßt uns Festfeier halten nicht im\textless sup title=``oder: mit
dem''\textgreater✲ alten Sauerteig, auch nicht im\textless sup
title=``oder: mit dem''\textgreater✲ Sauerteig der Schlechtigkeit und
Bosheit, sondern im\textless sup title=``oder: mit dem''\textgreater✲
Süßteig der Lauterkeit✲ und Wahrheit.

\hypertarget{c-berichtigung-eines-miuxdfverstuxe4ndnisses-der-korinther-bezuxfcglich-der-warnung-vor-unzuxfcchtigen}{%
\paragraph{c) Berichtigung eines Mißverständnisses der Korinther
bezüglich der Warnung vor
Unzüchtigen}\label{c-berichtigung-eines-miuxdfverstuxe4ndnisses-der-korinther-bezuxfcglich-der-warnung-vor-unzuxfcchtigen}}

\bibleverse{9} Ich habe euch in meinem (vorigen) Briefe geschrieben, ihr
möchtet keinen Verkehr mit unzüchtigen Leuten haben; \bibleverse{10}
(das heißt) nicht überhaupt mit den Unzüchtigen dieser Welt oder mit den
Betrügern und Räubern oder Götzendienern; sonst müßtet ihr ja aus der
Welt auswandern. \bibleverse{11} Jetzt aber schreibe ich euch
(unmißverständlich) so: Ihr dürft keinen Verkehr mit jemand haben, der
den (christlichen) Brudernamen führt und dabei ein unzüchtiger Mensch
oder ein Betrüger, ein Götzendiener, ein Verleumder, ein Trunkenbold
oder ein Räuber ist; mit einem solchen Menschen dürft ihr nicht einmal
Tischgemeinschaft haben. \bibleverse{12} Denn was habe ich mit dem
Richten von Leuten außerhalb der Gemeinde zu tun? Habt nicht auch ihr
(nur) die zu eurer Gemeinde Gehörigen zu richten? \bibleverse{13} Die
draußen Stehenden wird Gott richten. Schafft den bösen Menschen aus
eurer Mitte weg!

\hypertarget{ruxfcge-des-prozessierens-vor-heidnischen-gerichten-und-uxfcberhaupt-des-prozessefuxfchrens}{%
\subsubsection{2. Rüge des Prozessierens vor heidnischen Gerichten und
überhaupt des
Prozesseführens}\label{ruxfcge-des-prozessierens-vor-heidnischen-gerichten-und-uxfcberhaupt-des-prozessefuxfchrens}}

\hypertarget{section-5}{%
\section{6}\label{section-5}}

\bibleverse{1} Gewinnt es wirklich jemand von euch über sich, wenn er
einen Rechtshandel mit einem andern (Bruder) hat, sein Recht
vor\textless sup title=``oder: bei''\textgreater✲ den
Ungerechten\textless sup title=``=~heidnischen Richtern''\textgreater✲
anstatt vor\textless sup title=``oder: bei''\textgreater✲ den Heiligen
zu suchen? \bibleverse{2} Wißt ihr denn nicht, daß die Heiligen (einst)
die Welt richten werden? Wenn euch also das Gericht über die Welt
zusteht, seid ihr da nicht geeignet für die Entscheidung der
geringfügigsten Rechtshändel? \bibleverse{3} Wißt ihr nicht, daß wir
sogar Engel richten werden, geschweige denn Rechtshändel um mein und
dein\textless sup title=``oder: über Dinge des gewöhnlichen
Lebens''\textgreater✲? \bibleverse{4} Wenn ihr also Streitsachen über
mein und dein\textless sup title=``=~über alltägliche
Dinge''\textgreater✲ habt, da laßt ihr solche Leute über euch zu Gericht
sitzen, die sonst in der Gemeinde keine Achtung genießen! \bibleverse{5}
Euch zur Beschämung muß ich das sagen! Gibt es denn wirklich keinen
einzigen einsichtigen Mann unter euch, der befähigt wäre, zwischen
Brüdern als Schiedsrichter zu entscheiden? \bibleverse{6} Aber nein,
statt dessen streitet ein Bruder mit dem andern vor den Richtern, und
noch dazu vor Ungläubigen!

\bibleverse{7} Es ist überhaupt das schon ein sittlicher Mangel an euch,
daß ihr Rechtshändel miteinander habt. Warum laßt ihr euch nicht lieber
Unrecht zufügen, warum laßt ihr euch nicht lieber übervorteilen?
\bibleverse{8} Aber statt dessen verübt ihr selber Unrecht und
Übervorteilung, und noch dazu an Brüdern! \bibleverse{9} Wißt ihr nicht,
daß keiner, der Unrecht tut, das Reich Gottes erben wird? Irret euch
nicht! Weder Unzüchtige noch Götzendiener, weder Ehebrecher noch
Lüstlinge und Knabenschänder, \bibleverse{10} weder Diebe noch Betrüger,
auch keine Trunkenbolde, keine Verleumder und Räuber werden das Reich
Gottes erben. \bibleverse{11} Und Leute solcher Art sind manche (von
euch früher) gewesen. Doch ihr habt euch (in der Taufe) reinwaschen
lassen, seid geheiligt worden, habt die Rechtfertigung erlangt durch den
Namen des Herrn Jesus Christus und durch den Geist unsers Gottes.

\hypertarget{unzuchtsuxfcnden-haben-mit-christlicher-freiheit-nichts-zu-tun-warnung-vor-unzucht}{%
\subsubsection{3. Unzuchtsünden haben mit christlicher Freiheit nichts
zu tun; Warnung vor
Unzucht}\label{unzuchtsuxfcnden-haben-mit-christlicher-freiheit-nichts-zu-tun-warnung-vor-unzucht}}

\bibleverse{12} »Alles ist mir erlaubt!« -- Ja, aber nicht alles ist
zuträglich. »Alles ist mir erlaubt!« -- Ja, aber ich darf mich nicht von
irgend etwas beherrschen lassen. \bibleverse{13} »Die Speisen sind für
den Bauch da, und der Bauch ist für die Speisen da.« -- Ja, aber Gott
wird sowohl diesen als auch jenem ein Ende bereiten. Doch der Leib ist
nicht für die Unzucht da, sondern für den Herrn, und der Herr ist da für
den Leib; \bibleverse{14} Gott aber hat den Herrn auferweckt und wird
auch uns durch seine Macht auferwecken. \bibleverse{15} Wißt ihr nicht,
daß eure Leiber Glieder Christi sind? Soll\textless sup title=``oder:
darf''\textgreater✲ ich nun die Glieder Christi\textless sup
title=``d.h. welche Christus gehören''\textgreater✲ nehmen und Glieder
einer Buhlerin aus ihnen machen? Nimmermehr! \bibleverse{16} Oder wißt
ihr nicht, daß, wer sich an eine Buhlerin hängt, ein Leib mit ihr ist?
Es heißt ja\textless sup title=``1.Mose 2,24''\textgreater✲: »Die beiden
werden ein Fleisch sein.« \bibleverse{17} Wer dagegen dem Herrn anhängt,
der ist ein Geist mit ihm. \bibleverse{18} Fliehet die Unzucht! Jede
(andere) Sünde, die ein Mensch begeht, bleibt außerhalb seines Leibes,
der Unzüchtige aber sündigt gegen seinen eigenen Leib. \bibleverse{19}
Oder wißt ihr nicht, daß euer Leib ein Tempel des in euch wohnenden
heiligen Geistes ist, den ihr von Gott empfangen habt, und daß ihr
(somit) nicht euch selbst gehört? \bibleverse{20} Denn ihr seid teuer
erkauft worden. Macht also Gott Ehre mit eurem Leibe!

\hypertarget{ii.-antworten-des-apostels-auf-anfragen-des-gemeindebriefes-kap.-7-15}{%
\subsection{II. Antworten des Apostels auf Anfragen des Gemeindebriefes
(Kap.
7-15)}\label{ii.-antworten-des-apostels-auf-anfragen-des-gemeindebriefes-kap.-7-15}}

\hypertarget{uxfcber-ehe-und-ehelosigkeit}{%
\subsubsection{1. Über Ehe und
Ehelosigkeit}\label{uxfcber-ehe-und-ehelosigkeit}}

\hypertarget{a-vom-wert-und-beduxfcrfnis-der-ehe-und-vom-ehelichen-leben-im-allgemeinen}{%
\paragraph{a) Vom Wert und Bedürfnis der Ehe und vom ehelichen Leben im
allgemeinen}\label{a-vom-wert-und-beduxfcrfnis-der-ehe-und-vom-ehelichen-leben-im-allgemeinen}}

\hypertarget{section-6}{%
\section{7}\label{section-6}}

\bibleverse{1} Auf die Anfragen in eurem Briefe aber (antworte ich
folgendes): Ein Mann tut gut, (überhaupt) kein Weib zu berühren;
\bibleverse{2} aber um der (Vermeidung der) Unzuchtsünden willen mag
jeder (Mann) eine Ehefrau\textless sup title=``=~eine eigene
Gattin''\textgreater✲ und jede (Frau) ihren Ehemann\textless sup
title=``=~einen eigenen Gatten''\textgreater✲ haben. \bibleverse{3} Der
Mann leiste seiner Frau die schuldige Ehepflicht, ebenso auch die Frau
ihrem Manne! \bibleverse{4} Die Frau hat nicht über ihren Leib zu
verfügen, sondern ihr Mann; gleicherweise besitzt aber auch der Mann
kein Verfügungsrecht über seinen Leib, sondern die Frau. \bibleverse{5}
Entzieht euch einander nicht, höchstens aufgrund beiderseitigen
Einverständnisses für eine (bestimmte) Zeit, um euch (ungestört) dem
Gebet zu widmen, aber dann wieder zusammenzukommen, damit der Satan euch
nicht infolge eurer Unenthaltsamkeit in Versuchung führe! \bibleverse{6}
Übrigens spreche ich dies nur als ein Zugeständnis aus, nicht als ein
Gebot. \bibleverse{7} Ich möchte freilich wünschen, daß alle Menschen so
wären wie ich; doch jeder hat hierin eine besondere Gabe von Gott, der
eine so, der andere anders.

\hypertarget{b-vorschriften-uxfcber-verschiedene-verhuxe4ltnisse-des-ehelichen-und-ehelosen-lebens}{%
\paragraph{b) Vorschriften über verschiedene Verhältnisse des ehelichen
und ehelosen
Lebens}\label{b-vorschriften-uxfcber-verschiedene-verhuxe4ltnisse-des-ehelichen-und-ehelosen-lebens}}

\hypertarget{aa-vom-verhalten-der-ledigen-und-von-scheidung-christlicher-ehen}{%
\subparagraph{aa) Vom Verhalten der Ledigen und von Scheidung
christlicher
Ehen}\label{aa-vom-verhalten-der-ledigen-und-von-scheidung-christlicher-ehen}}

\bibleverse{8} Den Unverheirateten aber und den Witwen sage ich: Sie tun
gut, wenn sie so\textless sup title=``d.h. ehelos''\textgreater✲
bleiben, wie auch ich es bin; \bibleverse{9} können sie aber nicht
enthaltsam leben, so mögen sie heiraten; denn in der Ehe leben ist
besser als entflammt sein\textless sup title=``d.h. von Begier verzehrt
werden''\textgreater✲.

\bibleverse{10} Den Verheirateten aber gebiete ich -- nein, nicht ich,
sondern der Herr --, daß eine Frau sich von ihrem Manne nicht
scheiden\textless sup title=``oder: trennen''\textgreater✲ soll;
\bibleverse{11} hat sie sich aber doch geschieden\textless sup
title=``oder: getrennt''\textgreater✲, so soll sie unverheiratet bleiben
oder sich mit ihrem Mann wieder versöhnen; und ebenso soll auch der Mann
seine Frau nicht entlassen\textless sup title=``oder: verstoßen; vgl. Mk
10,11-12''\textgreater✲.

\hypertarget{bb-verhalten-in-der-mischehe}{%
\subparagraph{bb) Verhalten in der
Mischehe}\label{bb-verhalten-in-der-mischehe}}

\bibleverse{12} Den übrigen aber sage ich von mir aus, nicht der Herr:
Wenn ein (christlicher) Bruder eine Ungläubige✲ zur Frau hat und diese
einverstanden ist, mit ihm weiterzuleben, so soll er sie nicht
entlassen. \bibleverse{13} Ebenso, wenn eine (gläubige) Frau einen
ungläubigen✲ Mann hat und dieser einverstanden ist, mit ihr
weiterzuleben, so soll sie ihren Mann nicht entlassen. \bibleverse{14}
Denn der ungläubige Mann ist durch seine Frau geweiht✲, und die
ungläubige Frau ist durch den (gläubigen) Bruder geweiht✲; sonst wären
ja auch eure Kinder unrein, und sie sind doch in Wirklichkeit geweiht✲.
\bibleverse{15} Wenn jedoch der ungläubige Teil durchaus die Trennung
will, so mag er sich trennen: in solchen Fällen ist der Bruder oder die
Schwester nicht sklavisch (an eine Ehe) gebunden; vielmehr hat Gott uns
zu einem Leben in Frieden berufen. \bibleverse{16} Denn wie kannst du, o
Frau, wissen, ob du deinen Mann wirklich retten\textless sup
title=``=~zur Bekehrung bringen''\textgreater✲ wirst? Oder wie kannst
du, o Mann, wissen, ob du deine Frau wirklich retten wirst?

\hypertarget{c-allgemeine-vorschrift-uxfcber-die-stellung-des-christen-zu-den-bestehenden-uxe4uuxdferen-verhuxe4ltnissen-jeder-gluxe4ubige-bleibe-in-dem-stande-in-dem-er-berufen-worden-ist}{%
\paragraph{c) Allgemeine Vorschrift über die Stellung des Christen zu
den bestehenden äußeren Verhältnissen: Jeder Gläubige bleibe in dem
Stande, in dem er berufen worden
ist!}\label{c-allgemeine-vorschrift-uxfcber-die-stellung-des-christen-zu-den-bestehenden-uxe4uuxdferen-verhuxe4ltnissen-jeder-gluxe4ubige-bleibe-in-dem-stande-in-dem-er-berufen-worden-ist}}

\bibleverse{17} Nur wie der Herr einem jeden sein Los zugeteilt, wie
Gott einen jeden berufen hat, so lebe er weiter: diese Vorschrift gebe
ich in allen Gemeinden. \bibleverse{18} Ist jemand als Beschnittener
berufen worden, so suche er seine Beschneidung nicht rückgängig zu
machen; und wer als Unbeschnittener berufen worden ist, lasse sich nicht
beschneiden: \bibleverse{19} die Beschneidung hat keinen Wert, und auch
das Unbeschnittensein hat keinen Wert, sondern nur die Beobachtung der
Gebote Gottes. \bibleverse{20} Jeder bleibe in dem Stande, in dem er
berufen worden ist! \bibleverse{21} Bist du als Sklave berufen worden:
laß dich's nicht anfechten, nein, selbst wenn du frei werden kannst, so
bleibe nur um so lieber dabei. \bibleverse{22} Denn der im Herrn
berufene Sklave ist ein Freigelassener des Herrn\textless sup
title=``oder: vom Herrn freigekauft''\textgreater✲, und ebenso ist der
Freie nach seiner Berufung ein Sklave\textless sup title=``oder:
Knecht''\textgreater✲ Christi. \bibleverse{23} Ihr seid teuer✲ erkauft
worden: werdet nicht Knechte der Menschen! \bibleverse{24} Ein jeder,
liebe Brüder, möge in dem Stande, in dem er berufen worden ist, bei Gott
verbleiben!

\hypertarget{d-uxfcber-ehelosigkeit-beider-geschlechter-ratschluxe4ge-fuxfcr-die-verheiratung-lediger-muxe4dchen-und-fuxfcr-die-wiederverheiratung-der-witwen}{%
\paragraph{d) Über Ehelosigkeit beider Geschlechter; Ratschläge für die
Verheiratung lediger Mädchen und für die Wiederverheiratung der
Witwen}\label{d-uxfcber-ehelosigkeit-beider-geschlechter-ratschluxe4ge-fuxfcr-die-verheiratung-lediger-muxe4dchen-und-fuxfcr-die-wiederverheiratung-der-witwen}}

\bibleverse{25} In betreff der Mädchen\textless sup title=``oder:
unverheirateten Haustöchter''\textgreater✲ aber habe ich kein
(ausdrückliches) Gebot des Herrn, spreche aber meine eigene Ansicht aus
als einer, der Barmherzigkeit vom Herrn erfahren hat, so daß ich
Vertrauen verdiene. \bibleverse{26} Ich halte also dafür, daß dieser
Stand (der Ehelosigkeit) empfehlenswert ist wegen der gegenwärtigen
schweren Zeitlage, daß (nämlich) ein jeder gut tut, so zu bleiben (wie
er ist). \bibleverse{27} Bist du an eine Gattin gebunden, so suche keine
Lösung des Verhältnisses; bist du ledig, so suche keine Gattin;
\bibleverse{28} doch hast du, wenn du heiratest, damit keine Sünde
begangen, und auch ein Mädchen sündigt nicht, wenn es sich verheiratet.
Freilich -- Not im äußeren Leben werden die Betreffenden (durchzumachen)
haben, und ich möchte euch doch davon verschont sehen.

\bibleverse{29} Das aber sage ich euch, liebe Brüder: Die Frist ist nur
noch kurz bemessen; künftighin müssen auch die, welche eine Frau haben,
sich so verhalten, als hätten sie keine, \bibleverse{30} ebenso die
Weinenden, als weinten sie nicht, die Fröhlichen, als wären sie nicht
fröhlich, die Kaufenden, als ob sie (das Gekaufte) nicht behielten,
\bibleverse{31} und die mit der Welt Verkehrenden, als hätten sie nichts
mit ihr zu schaffen; denn die Welt in ihrer jetzigen Gestalt geht dem
Untergang entgegen! \bibleverse{32} Da möchte ich nun wünschen, daß ihr
frei von Sorgen bliebet. Der Unverheiratete ist um die Sache des Herrn
besorgt: er möchte dem Herrn gefallen; \bibleverse{33} der Verheiratete
dagegen sorgt sich um die Dinge der Welt: er möchte seiner Frau
gefallen; \bibleverse{34} so ist er geteilten Herzens. Ebenso richtet
die Frau, die keinen Mann mehr hat, und die Jungfrau ihr Sorgen auf die
Sache des Herrn: sie möchten an Leib und Geist heilig sein; die
verheiratete Frau dagegen sorgt sich um die Dinge der Welt: sie möchte
ihrem Manne gefallen. \bibleverse{35} Diesen Rat (ledig zu bleiben) gebe
ich euch aber zu eurem eigenen Besten, nicht um euch eine Schlinge um
den Hals zu legen, sondern zur Förderung guter Sitte und zu treuem
Festhalten am Herrn. \bibleverse{36} Meint jedoch jemand, an seiner
unverheirateten Tochter nicht recht zu handeln, falls sie die Jahre
ihrer Jugendblüte überschreite, und liegt demnach ein Anlaß (zu ihrer
Verheiratung) vor, so tue er, was er will; er versündigt sich nicht: sie
mögen sich heiraten. \bibleverse{37} Wer dagegen in seinem Herzen fest
geworden ist und sich in keiner Zwangslage befindet, sondern freier Herr
über seinen Willen\textless sup title=``=~seine
Entschließungen''\textgreater✲ ist und sich fest vorgenommen hat, seine
jungfräuliche Tochter unverheiratet zu lassen, der wird gut daran tun.
\bibleverse{38} Also: wer seine unverheiratete Tochter verheiratet, tut
gut daran, und wer sie nicht verheiratet, wird noch besser tun.

\hypertarget{uxfcber-die-wiederverheiratung-der-witwen}{%
\paragraph{Über die Wiederverheiratung der
Witwen}\label{uxfcber-die-wiederverheiratung-der-witwen}}

\bibleverse{39} Eine Ehefrau ist so lange gebunden, als ihr Mann lebt;
wenn aber ihr Mann entschlafen ist, so steht es ihr frei, sich zu
verheiraten, mit wem sie will, nur geschehe es im Herrn! \bibleverse{40}
Glücklicher aber ist sie zu preisen, wenn sie so bleibt (wie sie ist);
das ist meine Ansicht, und ich denke doch auch den Geist Gottes zu
besitzen.

\hypertarget{uxfcber-den-genuuxdf-von-guxf6tzenopferfleisch-und-uxfcber-den-gebrauch-der-christlichen-freiheit}{%
\subsubsection{2. Über den Genuß von Götzenopferfleisch und über den
Gebrauch der christlichen
Freiheit}\label{uxfcber-den-genuuxdf-von-guxf6tzenopferfleisch-und-uxfcber-den-gebrauch-der-christlichen-freiheit}}

\hypertarget{a-behandlung-der-von-den-korinthern-geuxe4uuxdferten-ansichten}{%
\paragraph{a) Behandlung der von den Korinthern geäußerten
Ansichten}\label{a-behandlung-der-von-den-korinthern-geuxe4uuxdferten-ansichten}}

\hypertarget{aa-die-erkenntnis-fuxfcr-sich-allein-steht-an-wert-hinter-der-liebe-zuruxfcck}{%
\subparagraph{aa) Die Erkenntnis für sich allein steht an Wert hinter
der Liebe
zurück}\label{aa-die-erkenntnis-fuxfcr-sich-allein-steht-an-wert-hinter-der-liebe-zuruxfcck}}

\hypertarget{section-7}{%
\section{8}\label{section-7}}

\bibleverse{1} »Was sodann das Götzenopferfleisch betrifft, so wissen
wir, daß wir allesamt im Besitz der (erforderlichen) Erkenntnis sind.«
Ja, aber die Erkenntnis macht dünkelhaft, die Liebe dagegen erbaut.
\bibleverse{2} Wer sich auf seine Erkenntnis etwas einbildet, der hat
noch nicht so erkannt, wie man erkennen muß; \bibleverse{3} wer dagegen
Gott liebt, der ist von ihm erkannt.

\hypertarget{bb-nicht-jedermann-ist-im-besitz-der-vollkommenen-erkenntnis}{%
\subparagraph{bb) Nicht jedermann ist im Besitz der vollkommenen
Erkenntnis}\label{bb-nicht-jedermann-ist-im-besitz-der-vollkommenen-erkenntnis}}

\bibleverse{4} »Was nun den Genuß des Götzenopferfleisches betrifft, so
wissen wir, daß es keinen Götzen in der Welt gibt und daß es keinen
(anderen) Gott gibt als den einen. \bibleverse{5} Denn mag es auch
sogenannte Götter, sei es im Himmel oder auf der Erde geben -- es gibt
ja (wirklich) viele solche Götter und viele Herren --, \bibleverse{6} so
gibt es doch für uns (Christen) nur einen Gott, nämlich den Vater, von
dem alle Dinge sind und wir zu ihm\textless sup title=``oder: für
ihn''\textgreater✲, und nur einen Herrn, nämlich Jesus Christus, durch
den\textless sup title=``=~durch dessen Vermittlung''\textgreater✲ alle
Dinge (geworden) sind und wir durch ihn.«~-- \bibleverse{7} Ja, aber es
besitzen nicht alle (Christen) solche Erkenntnis, vielmehr gibt es
manche, die infolge ihrer (früheren) Gewöhnung an den Götzendienst (das
Fleisch) noch als ein dem Götzen geweihtes Opfer essen, und so wird ihr
Gewissen, schwach wie es ist, dadurch befleckt.

\hypertarget{cc-fuxfcr-den-gebrauch-christlicher-freiheit-ist-liebevolle-ruxfccksichtnahme-auf-die-schwachen-mauxdfgebend}{%
\subparagraph{cc) Für den Gebrauch christlicher Freiheit ist liebevolle
Rücksichtnahme auf die Schwachen
maßgebend}\label{cc-fuxfcr-den-gebrauch-christlicher-freiheit-ist-liebevolle-ruxfccksichtnahme-auf-die-schwachen-mauxdfgebend}}

\bibleverse{8} »(Der Genuß von) Speise wird für unsere Stellung zu Gott
nicht maßgebend sein: essen wir nicht, so haben wir dadurch keinen
Nachteil, und essen wir, so haben wir dadurch keinen Vorteil.«~--
\bibleverse{9} Ja, aber sehet wohl zu, daß diese eure Freiheit für die
Schwachen nicht zu einem Anstoß\textless sup title=``oder: zum
Ärgernis''\textgreater✲ werde! \bibleverse{10} Denn wenn jemand dich mit
deiner »Erkenntnis« in einem Götzentempel am Mahl teilnehmen sieht, muß
da nicht sein Gewissen, wenn\textless sup title=``oder:
weil''\textgreater✲ er schwach ist, dazu »erbaut«\textless sup
title=``=~bewogen, oder: ermutigt''\textgreater✲ werden, (ebenfalls) das
Götzenopferfleisch zu essen? \bibleverse{11} So wird dann der Schwache
durch deine Erkenntnis ins Verderben gebracht, der Bruder, um dessen
willen Christus gestorben ist! \bibleverse{12} Wenn ihr euch aber auf
diese Weise an den Brüdern versündigt und ihr schwaches Gewissen
mißhandelt\textless sup title=``oder: verwundet''\textgreater✲, so
versündigt ihr euch an Christus. \bibleverse{13} Darum, wenn
Speise\textless sup title=``d.h. das, was ich esse''\textgreater✲ meinem
Bruder zum Anstoß wird\textless sup title=``=~ihn zur Sünde
verführt''\textgreater✲, so will ich in Ewigkeit kein Fleisch genießen,
um meinem Bruder kein Ärgernis zu bereiten.

\hypertarget{b-paulus-als-vorbild-der-selbstlosigkeit-durch-seinen-freiwilligen-verzicht-auf-die-apostelrechte}{%
\paragraph{b) Paulus als Vorbild der Selbstlosigkeit durch seinen
freiwilligen Verzicht auf die
Apostelrechte}\label{b-paulus-als-vorbild-der-selbstlosigkeit-durch-seinen-freiwilligen-verzicht-auf-die-apostelrechte}}

\hypertarget{aa-darlegung-und-begruxfcndung-der-dem-paulus-als-einem-apostel-zustehenden-rechte}{%
\subparagraph{aa) Darlegung und Begründung der dem Paulus als einem
Apostel zustehenden
Rechte}\label{aa-darlegung-und-begruxfcndung-der-dem-paulus-als-einem-apostel-zustehenden-rechte}}

\hypertarget{section-8}{%
\section{9}\label{section-8}}

\bibleverse{1} Bin ich nicht ein freier (Mann)? Bin ich nicht ein
Apostel? Habe ich nicht unsern Herrn Jesus gesehen? Seid ihr nicht mein
Werk im Herrn\textless sup title=``=~als christliche
Gemeinde''\textgreater✲? \bibleverse{2} Mag ich für andere kein Apostel
sein, so bin ich es doch sicherlich für euch; denn das
Siegel\textless sup title=``=~die Beglaubigung''\textgreater✲ für mein
Apostelamt seid ihr im Herrn\textless sup title=``=~als christliche
Gemeinde''\textgreater✲. \bibleverse{3} Das ist meine
Rechtfertigung\textless sup title=``=~mein Ausweisschein''\textgreater✲
denen gegenüber, die über mich zu Gericht sitzen (wollen).

\bibleverse{4} Haben wir (Apostel) etwa nicht das Recht, Essen und
Trinken zu beanspruchen? \bibleverse{5} Haben wir nicht das Recht, eine
Schwester✲ als Ehefrau (auf unsern Reisen) bei uns zu haben wie die
übrigen Apostel und die Brüder des Herrn und Kephas✲? \bibleverse{6}
Oder sind wir allein, ich und Barnabas, nicht berechtigt, die Handarbeit
zu unterlassen? \bibleverse{7} Wer leistet jemals Kriegsdienste für
eigenen Sold? Wer legt einen Weinberg an, ohne von dessen Früchten zu
essen? Oder wer hütet eine Herde, ohne von der Milch der Herde zu
genießen? \bibleverse{8} Behaupte ich dies etwa als einen bloß von
Menschen aufgestellten Grundsatz, oder enthält nicht das Gesetz dieselbe
Verordnung? \bibleverse{9} Im mosaischen Gesetz steht ja doch
geschrieben\textless sup title=``5.Mose 25,4''\textgreater✲: »Du sollst
einem Ochsen, der zu dreschen hat, das Maul nicht verbinden!« Ist es
Gott etwa um die Ochsen zu tun? \bibleverse{10} Oder gehen seine Worte
nicht ohne allen Zweifel auf uns? Ja, um unsertwillen steht geschrieben,
daß der Pflüger auf Hoffnung hin pflügen und der Drescher in der
Hoffnung auf Mitgenuß (des Ertrages) arbeiten soll. \bibleverse{11} Wenn
wir für euch das Geistliche\textless sup title=``d.h. die geistlichen
Güter''\textgreater✲ ausgesät haben, ist es da etwas Absonderliches,
wenn wir von euch das Irdische\textless sup title=``d.h. die irdischen
Güter''\textgreater✲ ernten? \bibleverse{12} Wenn andere an dem
Verfügungsrecht über euch\textless sup title=``=~über euer
Vermögen''\textgreater✲ Anteil haben, sind wir dann nicht in noch
höherem Grade dazu berechtigt? Aber wir haben von diesem Recht keinen
Gebrauch gemacht, sondern ertragen lieber alles, um nur der
Heilsbotschaft Christi\textless sup title=``oder: von
Christus''\textgreater✲ kein Hindernis in den Weg zu legen.

\hypertarget{bb-darlegung-der-gruxfcnde-weshalb-paulus-auf-seine-rechte-verzichtet}{%
\subparagraph{bb) Darlegung der Gründe, weshalb Paulus auf seine Rechte
verzichtet}\label{bb-darlegung-der-gruxfcnde-weshalb-paulus-auf-seine-rechte-verzichtet}}

\bibleverse{13} Wißt ihr nicht, daß die, welche den Tempeldienst
verrichten, von den Einkünften des Tempels ihren Unterhalt haben? Daß
die, welche beständig am Opferaltar tätig sind, ihren Anteil vom Altar
erhalten? \bibleverse{14} Ebenso hat auch der Herr über die Verkündiger
der Heilsbotschaft die Anordnung getroffen, daß sie von der
Heilsverkündigung leben sollen. \bibleverse{15} Ich aber habe von keinem
dieser Rechte für mich Gebrauch gemacht, und ich schreibe euch dieses
auch nicht deshalb, damit es fortan mit mir so gehalten werde; denn
lieber wollte ich sterben, als -- nein, meinen Ruhm soll mir niemand
zunichte machen! \bibleverse{16} Denn wenn ich die Heilsbotschaft
verkündige, so habe ich daran keinen Grund zum Rühmen, denn ich stehe
dabei unter einem Zwang; ein ›Wehe!‹ träfe mich ja, wenn ich die
Heilsbotschaft nicht verkündigte! \bibleverse{17} Denn nur, wenn ich
dies aus freiem Entschluß tue, habe ich (Anspruch auf) Lohn; wenn ich es
aber unfreiwillig tue, so ist es nur ein Haushalteramt, mit dem ich
betraut bin. \bibleverse{18} Worin besteht demnach mein Lohn? Darin, daß
ich als Verkündiger der Heilsbotschaft diese unentgeltlich darbiete, so
daß ich von meinem Recht bei der Verkündigung der Heilsbotschaft keinen
Gebrauch mache.

\hypertarget{cc-paulus-obwohl-uxe4uuxdferlich-vuxf6llig-frei-ist-dennoch-ein-knecht-aller-menschen}{%
\subparagraph{cc) Paulus, obwohl äußerlich völlig frei, ist dennoch ein
Knecht aller
Menschen}\label{cc-paulus-obwohl-uxe4uuxdferlich-vuxf6llig-frei-ist-dennoch-ein-knecht-aller-menschen}}

\bibleverse{19} Denn obwohl ich von allen Menschen unabhängig bin, habe
ich mich doch allen zum Knecht gemacht, um die Mehrzahl\textless sup
title=``oder: recht viele''\textgreater✲ von ihnen zu gewinnen.
\bibleverse{20} So bin ich denn für die Juden\textless sup title=``oder:
den Juden gegenüber''\textgreater✲ zu einem Juden geworden, um Juden zu
gewinnen; für die Gesetzesleute zu einem Mann des Gesetzes -- obgleich
ich selbst nicht unter dem Gesetz stehe --, um die Gesetzesleute zu
gewinnen; \bibleverse{21} für die (Heiden), die das Gesetz nicht haben,
zu einem Manne, der ohne das Gesetz lebt -- obgleich ich nicht ohne
Gottes Gesetz lebe, vielmehr dem Gesetz Christi unterworfen bin --, um
die, welche das Gesetz nicht haben, zu gewinnen; \bibleverse{22} für die
Schwachen bin ich ein Schwacher geworden, um die Schwachen zu gewinnen;
kurz: für alle bin ich alles geworden, um auf jeden Fall einige zu
retten. \bibleverse{23} Alles (das) aber tue ich um der Heilsbotschaft
willen, damit auch ich Anteil an ihr\textless sup title=``oder: ihrem
Segen, d.h. an dem in ihr verheißenen Heilsgut''\textgreater✲ erlange.

\hypertarget{dd-der-apostel-als-wettkuxe4mpfer-um-den-himmlischen-siegespreis}{%
\subparagraph{dd) Der Apostel als Wettkämpfer um den himmlischen
Siegespreis}\label{dd-der-apostel-als-wettkuxe4mpfer-um-den-himmlischen-siegespreis}}

\bibleverse{24} Wißt ihr nicht, daß die, welche in der Rennbahn laufen,
zwar alle laufen, daß aber nur einer den Siegespreis erhält? Lauft ihr
nun in der Weise, daß ihr ihn erlangt! \bibleverse{25} Jeder aber, der
sich am Wettkampf beteiligen will, legt sich Enthaltsamkeit in allen
Beziehungen auf, jene, um einen vergänglichen Kranz zu empfangen, wir
aber einen unvergänglichen. \bibleverse{26} So laufe ich denn nicht
ziellos\textless sup title=``=~ins Blaue hinein''\textgreater✲ und
treibe den Faustkampf so, daß ich keine Lufthiebe führe; \bibleverse{27}
sondern ich zerschlage meinen Leib und mache ihn mir dienstbar, um
nicht, nachdem ich als Herold andere zum Kampf aufgerufen habe, mich
selbst als untüchtig\textless sup title=``oder: des Preises
unwürdig''\textgreater✲ zu erweisen.

\hypertarget{c-die-tatsuxe4chlichen-gefahren-der-teilnahme-an-den-guxf6tzenopfermahlzeiten}{%
\paragraph{c) Die tatsächlichen Gefahren der Teilnahme an den
Götzenopfermahlzeiten}\label{c-die-tatsuxe4chlichen-gefahren-der-teilnahme-an-den-guxf6tzenopfermahlzeiten}}

\hypertarget{aa-das-durch-guxf6ttliche-gnadenerweise-in-der-wuxfcste-gesegnete-und-zur-rettung-ins-heilige-land-berufene-israel}{%
\subparagraph{aa) Das durch göttliche Gnadenerweise in der Wüste
gesegnete und zur Rettung ins heilige Land berufene
Israel}\label{aa-das-durch-guxf6ttliche-gnadenerweise-in-der-wuxfcste-gesegnete-und-zur-rettung-ins-heilige-land-berufene-israel}}

\hypertarget{section-9}{%
\section{10}\label{section-9}}

\bibleverse{1} Ich will euch nämlich nicht in Unkenntnis darüber lassen,
liebe Brüder, daß unsere Väter allesamt unter (dem Schutz) der Wolke
gestanden haben und allesamt durch das Meer hindurchgezogen sind
\bibleverse{2} und sämtlich die Taufe auf Mose in der Wolke und im Meer
empfangen, \bibleverse{3} auch allesamt dieselbe geistliche Speise
gegessen \bibleverse{4} und sämtlich denselben geistlichen Trank
getrunken haben: sie tranken nämlich aus einem geistlichen Felsen, der
sie begleitete, und dieser Fels war Christus.

\hypertarget{bb-trotzdem-wurden-sie-weil-sie-ihrer-fleischeslust-willig-dienten-verworfen-uns-zum-warnenden-beispiel}{%
\subparagraph{bb) Trotzdem wurden sie, weil sie ihrer Fleischeslust
willig dienten, verworfen, uns zum warnenden
Beispiel}\label{bb-trotzdem-wurden-sie-weil-sie-ihrer-fleischeslust-willig-dienten-verworfen-uns-zum-warnenden-beispiel}}

\bibleverse{5} Doch an den meisten von ihnen hatte Gott kein
Wohlgefallen, denn sie sind in der Wüste niedergestreckt worden.
\bibleverse{6} Diese Dinge\textless sup title=``oder:
Vorgänge''\textgreater✲ aber sind zum warnenden Vorbild für uns
geschehen, damit wir unsre Gelüste nicht auf Böses richten, wie jene
sich haben gelüsten lassen\textless sup title=``4.Mose
11,4''\textgreater✲. \bibleverse{7} Werdet auch keine Götzendiener, wie
manche von jenen; es steht ja geschrieben\textless sup title=``2.Mose
32,6''\textgreater✲: »Das Volk setzte sich nieder, um zu essen und zu
trinken, und stand wieder auf, um sich zu belustigen.« \bibleverse{8}
Wir wollen auch keine Unzucht treiben, wie manche von jenen es getan
haben; sind doch (deshalb) von ihnen an einem einzigen Tage
dreiundzwanzigtausend gefallen\textless sup title=``4.Mose
25,1.9''\textgreater✲. \bibleverse{9} Wir wollen auch den Herrn nicht
versuchen, wie manche von ihnen es getan haben und dafür von den
Schlangen umgebracht worden sind\textless sup title=``4.Mose
21,5-6''\textgreater✲. \bibleverse{10} Murret auch nicht, wie manche von
ihnen getan und dafür den Tod durch den Verderber erlitten
haben\textless sup title=``4.Mose 14,2.37''\textgreater✲.
\bibleverse{11} Dies alles ist jenen aber vorbildlicherweise widerfahren
und ist niedergeschrieben worden zur Warnung für uns, denen das Ende der
Weltzeiten\textless sup title=``Hebr 1,2''\textgreater✲ nahe bevorsteht.
\bibleverse{12} Wer daher festzustehen meint, der sehe wohl zu, daß er
nicht falle! \bibleverse{13} Es hat euch (bisher) noch keine andere als
menschliche Versuchung betroffen; und Gott ist treu: er wird nicht
zulassen, daß ihr über euer Vermögen hinaus versucht werdet, sondern
wird zugleich mit der Versuchung auch einen solchen Ausgang schaffen,
daß ihr sie bestehen könnt.

\hypertarget{cc-teilnahme-am-guxf6tzendienst-und-an-opfermahlzeiten-ist-mit-der-feier-des-christlichen-abendmahls-unvereinbar-und-deshalb-zu-meiden}{%
\subparagraph{cc) Teilnahme am Götzendienst und an Opfermahlzeiten ist
mit der Feier des christlichen Abendmahls unvereinbar und deshalb zu
meiden}\label{cc-teilnahme-am-guxf6tzendienst-und-an-opfermahlzeiten-ist-mit-der-feier-des-christlichen-abendmahls-unvereinbar-und-deshalb-zu-meiden}}

\bibleverse{14} Darum, meine Geliebten, fliehet vor dem Götzendienst!
\bibleverse{15} Ich rede ja doch zu euch als zu verständigen Leuten:
urteilt selbst über das, was ich sage! \bibleverse{16} Der Kelch des
Segens, den wir segnen: ist✲ er nicht die Gemeinschaft mit dem Blute
Christi? Das Brot, das wir brechen: ist✲ es nicht die Gemeinschaft mit
dem Leibe Christi? \bibleverse{17} Weil es ein einziges Brot ist, sind
wir trotz unserer Vielheit doch ein einziger Leib, denn wir alle teilen
uns in das eine Brot\textless sup title=``oder: haben Anteil an dem
einen Brote''\textgreater✲. \bibleverse{18} Sehet das irdische Israel
an: stehen nicht die, welche die Opfer essen, in engster Gemeinschaft
mit dem Opferaltar? \bibleverse{19} Was behaupte\textless sup
title=``oder: meine''\textgreater✲ ich nun damit? Daß das
Götzenopferfleisch etwas sei? Oder daß ein Götze etwas sei?
\bibleverse{20} Nein, wohl aber (behaupte ich), daß die Heiden die
Opfer, die sie darbringen, dämonischen Wesen\textless sup title=``oder:
Geistern''\textgreater✲ und nicht Gott darbringen. Ich will aber nicht,
daß ihr in Verbindung mit den Dämonen tretet. \bibleverse{21} Ihr könnt
nicht (zugleich) den Kelch des Herrn und den Kelch der Dämonen trinken;
ihr könnt nicht (zugleich) am Tisch des Herrn und am Tisch der Dämonen
Gäste sein. \bibleverse{22} Oder wollen wir den Herrn zu
Eifersucht\textless sup title=``oder: zum Zorn''\textgreater✲ reizen?
Sind wir etwa stärker als er?

\hypertarget{dd-wann-ist-der-genuuxdf-von-guxf6tzenopferfleisch-unbedenklich-beschruxe4nkung-der-christlichen-freiheit-durch-ruxfccksicht-auf-die-bruderliebe}{%
\subparagraph{dd) Wann ist der Genuß von Götzenopferfleisch
unbedenklich? Beschränkung der christlichen Freiheit durch Rücksicht auf
die
Bruderliebe}\label{dd-wann-ist-der-genuuxdf-von-guxf6tzenopferfleisch-unbedenklich-beschruxe4nkung-der-christlichen-freiheit-durch-ruxfccksicht-auf-die-bruderliebe}}

\bibleverse{23} »Alles ist (uns Christen) erlaubt!« -- Ja, aber nicht
alles ist zuträglich. »Alles ist erlaubt!« -- Ja, aber nicht alles
erbaut\textless sup title=``=~fördert das geistliche
Leben''\textgreater✲. \bibleverse{24} Niemand sei (nur) auf seinen
eigenen Vorteil bedacht, sondern (jeder) auf die Förderung des anderen.
\bibleverse{25} Alles, was auf dem Fleischmarkt zum Verkauf steht, das
esset, ohne um des Gewissens willen Nachforschungen anzustellen;
\bibleverse{26} denn »dem Herrn gehört die (ganze) Erde und alle ihre
Fülle«\textless sup title=``Ps 24,1''\textgreater✲. \bibleverse{27} Wenn
einer von den Ungläubigen\textless sup title=``=~ein
Nichtchrist''\textgreater✲ euch zu Gast einlädt und ihr hingehen wollt,
so esset alles, was man euch vorsetzt, ohne um des Gewissens willen
Nachforschungen anzustellen. \bibleverse{28} Wenn aber jemand euch
(ausdrücklich) sagt: »Dies ist Opferfleisch!«, so esset nicht davon mit
Rücksicht auf den, der euch darauf hingewiesen hat, und um des Gewissens
willen~-- \bibleverse{29} ich meine damit jedoch nicht euer eigenes
Gewissen, sondern das des anderen; denn warum soll ich meine Freiheit
von dem Gewissen eines anderen richten\textless sup title=``=~schuldig
sprechen''\textgreater✲ lassen? \bibleverse{30} Wenn ich für meine
Person etwas mit Danksagung (gegen Gott) genieße, warum soll ich mich da
bezüglich dessen, wofür ich ein Dankgebet spreche, schmähen\textless sup
title=``=~als verwerflich beurteilen''\textgreater✲ lassen?

\hypertarget{ee-abschlieuxdfende-mahnung-zu-allezeit-rechtem-christenwandel}{%
\subparagraph{ee) Abschließende Mahnung zu allezeit rechtem
Christenwandel}\label{ee-abschlieuxdfende-mahnung-zu-allezeit-rechtem-christenwandel}}

\bibleverse{31} Nun: mögt ihr essen oder trinken oder sonst etwas tun,
tut alles zur Verherrlichung\textless sup title=``oder:
Ehre''\textgreater✲ Gottes! \bibleverse{32} Gebt weder den Juden noch
den Griechen noch der Gemeinde Gottes einen Anstoß, \bibleverse{33} wie
auch ich allen in jeder Hinsicht zu Gefallen lebe, indem ich nicht
meinen Vorteil suche, sondern den der vielen\textless sup title=``d.h.
der großen Mehrheit''\textgreater✲, damit sie gerettet werden.

\hypertarget{section-10}{%
\section{11}\label{section-10}}

\bibleverse{1} Nehmt mich zum Vorbild✲, gleichwie ich meinerseits dem
Vorbild Christi nachfolge!

\hypertarget{uxfcber-das-wohlanstuxe4ndige-verhalten-der-muxe4nner-und-die-verschleierung-der-frauen-beim-gebet-und-beim-gottesdienst}{%
\subsubsection{3. Über das wohlanständige Verhalten der Männer und die
Verschleierung der Frauen beim Gebet und beim
Gottesdienst}\label{uxfcber-das-wohlanstuxe4ndige-verhalten-der-muxe4nner-und-die-verschleierung-der-frauen-beim-gebet-und-beim-gottesdienst}}

\bibleverse{2} Ich erkenne es aber lobend bei euch an, daß »ihr in allen
Beziehungen meiner eingedenk seid und an den
Überlieferungen\textless sup title=``oder: Weisungen''\textgreater✲
festhaltet, wie ich sie euch mitgeteilt\textless sup title=``oder:
übergeben''\textgreater✲ habe.« \bibleverse{3} Ich möchte euch aber zu
bedenken geben, daß das Haupt✲ jedes Mannes Christus ist, das Haupt der
Frau aber ist der Mann, und das Haupt Christi ist Gott. \bibleverse{4}
Jeder Mann, der beim Beten oder beim prophetischen Reden eine
Kopfbedeckung trägt, entehrt dadurch sein Haupt✲; \bibleverse{5} jede
Frau dagegen, die mit unverhülltem Haupte betet oder prophetisch redet,
entehrt dadurch ihr Haupt, denn sie steht damit auf völlig gleicher
Stufe mit einer Geschorenen\textless sup title=``=~öffentlichen
Dirne''\textgreater✲. \bibleverse{6} Denn wenn eine Frau sich nicht
verschleiert, so mag sie sich auch das Haar abschneiden lassen; ist es
aber für eine Frau schimpflich, sich das Haar kurz zu schneiden oder es
sich ganz abscheren zu lassen, so soll sie sich verschleiern!
\bibleverse{7} Der Mann dagegen darf das Haupt nicht verhüllt haben,
weil er Gottes Abbild und Abglanz ist; die Frau aber ist der Abglanz des
Mannes. \bibleverse{8} Der Mann stammt ja doch nicht von der Frau,
sondern die Frau vom Manne; \bibleverse{9} auch ist der Mann ja nicht um
der Frau willen geschaffen, sondern die Frau um des Mannes willen.
\bibleverse{10} Deshalb muß die Frau (ein Zeichen der)
Herrschaft\textless sup title=``d.h. der Vollmacht des Mannes über
sie''\textgreater✲ auf dem Haupte tragen um der Engel willen.

\hypertarget{abweisung-einer-geringschuxe4tzung-der-frau-und-alles-streitens-uxfcber-den-gegenstand}{%
\paragraph{Abweisung einer Geringschätzung der Frau und alles Streitens
über den
Gegenstand}\label{abweisung-einer-geringschuxe4tzung-der-frau-und-alles-streitens-uxfcber-den-gegenstand}}

\bibleverse{11} Sonst steht jedoch weder die Frau gesondert vom Mann,
noch der Mann gesondert von der Frau im Herrn da; \bibleverse{12} denn
wie die Frau aus dem Manne entstanden ist, so wird wiederum der Mann
durch die Frau geboren; alles aber ist von Gott ausgegangen\textless sup
title=``=~so geordnet''\textgreater✲. \bibleverse{13} Urteilt für euch
selbst: Ist es schicklich, daß eine Frau unverhüllt zu Gott betet?
\bibleverse{14} Und lehrt euch nicht schon euer natürliches Gefühl, daß,
wenn ein Mann langes Haar trägt, es eine Schmach für ihn ist,
\bibleverse{15} während, wenn eine Frau langes Haar trägt, es eine Ehre
für sie ist? Denn das lange Haar ist ihr als Schleier\textless sup
title=``=~an Stelle eines Schleiers''\textgreater✲ gegeben.
\bibleverse{16} Will aber jemand durchaus auf seiner abweichenden
Meinung bestehen (so wisse er): Wir kennen eine solche Sitte nicht, und
auch die Gemeinden Gottes überhaupt.

\hypertarget{ernster-tadel-der-miuxdfstuxe4nde-bei-den-gemeinsamen-mahlzeiten-und-anweisung-fuxfcr-die-wuxfcrdige-feier-des-abendmahls}{%
\subsubsection{4. Ernster Tadel der Mißstände bei den gemeinsamen
Mahlzeiten und Anweisung für die würdige Feier des
Abendmahls}\label{ernster-tadel-der-miuxdfstuxe4nde-bei-den-gemeinsamen-mahlzeiten-und-anweisung-fuxfcr-die-wuxfcrdige-feier-des-abendmahls}}

\hypertarget{a-die-miuxdfstuxe4nde}{%
\paragraph{a) Die Mißstände}\label{a-die-miuxdfstuxe4nde}}

\bibleverse{17} Die folgenden Anordnungen aber treffe ich, weil ich es
nicht löblich finde, daß eure Zusammenkünfte euch nicht zum Segen,
sondern zur Schädigung gereichen. \bibleverse{18} Zunächst nämlich höre
ich, daß, wenn ihr in einer Gemeindeversammlung zusammenkommt,
Spaltungen unter euch bestehen, und zum Teil glaube ich es wirklich;
\bibleverse{19} es muß ja doch auch Parteiungen bei euch geben, damit
die Bewährten\textless sup title=``oder: Tüchtigen''\textgreater✲ unter
euch erkennbar werden! \bibleverse{20} Wenn ihr also an einem Ort
zusammenkommt, so ist es nicht möglich, das Herrenmahl zu
essen\textless sup title=``=~in rechter Weise halten''\textgreater✲;
\bibleverse{21} denn jeder nimmt beim Essen seine eigene Mahlzeit
vorweg, so daß der eine hungrig bleibt, während der andere trunken
ist\textless sup title=``oder: in Wein schlemmt''\textgreater✲.
\bibleverse{22} Habt ihr denn keine Häuser, um dort zu essen und zu
trinken? Oder verachtet ihr die Gemeinde Gottes und geht ihr darauf aus,
die Unbemittelten zu beschämen? Was soll ich dazu sagen? Soll ich euch
etwa loben? In diesem Punkte sicherlich nicht!

\hypertarget{b-die-rechte-feier-des-abendmahls-und-die-uxfcblen-folgen-des-unwuxfcrdigen-genusses-abschlieuxdfende-mahnung}{%
\paragraph{b) Die rechte Feier des Abendmahls und die üblen Folgen des
unwürdigen Genusses; abschließende
Mahnung}\label{b-die-rechte-feier-des-abendmahls-und-die-uxfcblen-folgen-des-unwuxfcrdigen-genusses-abschlieuxdfende-mahnung}}

\bibleverse{23} Denn ich habe es meinerseits vom Herrn her so
überkommen\textless sup title=``=~mitgeteilt erhalten''\textgreater✲,
wie ich es euch auch überliefert habe: Der Herr Jesus, in der Nacht, in
der er verraten wurde, nahm er Brot, \bibleverse{24} sprach das
Dankgebet, brach das Brot und sagte: »Dies ist mein Leib, (der) für euch
(gebrochen oder: dahingegeben wird); dies tut zu meinem Gedächtnis!«
\bibleverse{25} Ebenso (nahm er) auch den Kelch nach dem Mahl und sagte:
»Dieser Kelch ist der neue Bund in meinem Blut\textless sup
title=``oder: durch mein Blut''\textgreater✲; dies tut, sooft ihr (ihn)
trinkt, zu meinem Gedächtnis!« \bibleverse{26} Denn sooft ihr dieses
Brot eßt und den Kelch trinkt, verkündigt ihr (damit) den Tod des Herrn,
bis er (wieder-) kommt.

\bibleverse{27} Wer daher in unwürdiger Weise das Brot ißt oder den
Kelch des Herrn trinkt, der wird sich am Leibe und am Blute des Herrn
versündigen. \bibleverse{28} Jedermann prüfe sich also selbst und esse
dann erst von dem Brot und trinke von\textless sup title=``oder:
aus''\textgreater✲ dem Kelch! \bibleverse{29} Denn wer da ißt und
trinkt, der zieht sich selbst durch sein Essen und Trinken ein
(göttliches) Strafurteil zu, wenn er den Leib (des Herrn) nicht
unterscheidet. \bibleverse{30} Deshalb gibt es unter euch auch Schwache
und Kranke in so großer Zahl, und gar viele sind schon entschlafen.
\bibleverse{31} Wenn wir aber mit uns selbst ins Gericht
gingen\textless sup title=``=~uns prüften''\textgreater✲, so würden wir
kein Strafurteil empfangen. \bibleverse{32} Indem wir jedoch ein
Strafurteil empfangen, werden wir vom Herrn gezüchtigt\textless sup
title=``=~in Zucht genommen''\textgreater✲, damit wir nicht mit der Welt
zusammen verurteilt werden. \bibleverse{33} Darum, meine Brüder, wenn
ihr zum Essen\textless sup title=``=~zum heiligen Mahle''\textgreater✲
zusammenkommt, so wartet aufeinander! \bibleverse{34} Wenn jemand Hunger
hat, so esse er (vorher) zu Hause, damit ihr durch eure Zusammenkünfte
euch kein Strafgericht zuzieht. Das Weitere werde ich anordnen, wenn ich
(zu euch) komme.

\hypertarget{unterweisung-uxfcber-die-geistesgaben}{%
\subsubsection{5. Unterweisung über die
Geistesgaben}\label{unterweisung-uxfcber-die-geistesgaben}}

\hypertarget{a-allgemeines-uxfcber-die-auuxdferordentlichen-geistesgaben-ihren-ursprung-ihre-mannigfaltigkeit-und-ihren-zweck}{%
\paragraph{a) Allgemeines über die außerordentlichen Geistesgaben, ihren
Ursprung, ihre Mannigfaltigkeit und ihren
Zweck}\label{a-allgemeines-uxfcber-die-auuxdferordentlichen-geistesgaben-ihren-ursprung-ihre-mannigfaltigkeit-und-ihren-zweck}}

\hypertarget{aa-vorbemerkung-das-kennzeichen-der-gottgewirkten-geistesgaben}{%
\subparagraph{aa) Vorbemerkung: Das Kennzeichen der gottgewirkten
Geistesgaben}\label{aa-vorbemerkung-das-kennzeichen-der-gottgewirkten-geistesgaben}}

\hypertarget{section-11}{%
\section{12}\label{section-11}}

\bibleverse{1} In betreff der Geistesgaben aber will ich euch, liebe
Brüder, nicht im unklaren lassen. \bibleverse{2} Ihr wißt von eurer
Heidenzeit her: da waren es die stummen Götzenbilder, zu denen ihr mit
unwiderstehlicher Gewalt hingezogen\textless sup title=``oder:
fortgerissen''\textgreater✲ wurdet. \bibleverse{3} Darum tue ich euch
kund, daß niemand, der im Geiste\textless sup title=``oder: durch den
Geist''\textgreater✲ Gottes redet, sagt: »Verflucht ist\textless sup
title=``oder: sei''\textgreater✲ Jesus!« und keiner zu sagen vermag:
»Jesus ist der Herr!«, außer im heiligen Geist\textless sup
title=``oder: durch den heiligen Geist''\textgreater✲.

\hypertarget{bb-verschiedenheit-der-geistesgaben-aber-nur-ein-alles-wirkender-geist-und-ein-zweck}{%
\subparagraph{bb) Verschiedenheit der Geistesgaben, aber nur ein alles
wirkender Geist und ein
Zweck}\label{bb-verschiedenheit-der-geistesgaben-aber-nur-ein-alles-wirkender-geist-und-ein-zweck}}

\bibleverse{4} Es gibt nun zwar verschiedene Arten von Gnadengaben, aber
nur einen und denselben Geist; \bibleverse{5} und es gibt verschiedene
Arten von Dienstleistungen, doch nur einen und denselben Herrn;
\bibleverse{6} und es gibt verschiedene Arten von Kraftwirkungen, aber
nur einen und denselben Gott, der alles in allen wirkt. \bibleverse{7}
Jedem wird aber die Offenbarung des Geistes zum allgemeinen
Besten\textless sup title=``=~zum Nutzen der Gemeinde''\textgreater✲
verliehen. \bibleverse{8} So wird dem einen durch den Geist
Weisheitsrede verliehen, einem andern Erkenntnisrede nach Maßgabe
desselben Geistes, \bibleverse{9} einem andern Glaube in
demselben\textless sup title=``oder: durch denselben''\textgreater✲
Geist, einem andern Heilungsgaben in dem einen Geiste, \bibleverse{10}
einem andern Verrichtung von Wundertaten, einem andern
Weissagung\textless sup title=``oder: prophetische Rede''\textgreater✲,
einem andern Unterscheidung der Geister, einem andern mancherlei Arten
von Zungenreden, einem andern die Auslegung der Zungenreden.
\bibleverse{11} Dies alles wirkt aber ein und derselbe Geist, indem er
jedem eine besondere Gabe zuteilt, wie er will.

\hypertarget{cc-veranschaulichung-durch-das-gleichnis-vom-menschenleib-und-seinen-vielen-gliedern}{%
\subparagraph{cc) Veranschaulichung durch das Gleichnis vom Menschenleib
und seinen vielen
Gliedern}\label{cc-veranschaulichung-durch-das-gleichnis-vom-menschenleib-und-seinen-vielen-gliedern}}

\bibleverse{12} Denn wie der Leib eine Einheit\textless sup
title=``oder: nur einer''\textgreater✲ ist und doch viele Glieder hat,
alle Glieder des Leibes aber trotz ihrer Vielheit einen Leib bilden, so
ist es auch mit Christus. \bibleverse{13} Denn durch einen
Geist\textless sup title=``oder: in einem Geist''\textgreater✲ sind wir
alle durch die Taufe zu einem Leibe zusammengeschlossen worden, wir
mögen Juden oder Griechen, Sklaven oder Freie sein, und wir sind alle
mit einem Geist getränkt worden. \bibleverse{14} Auch der Leib besteht
ja nicht aus einem einzigen Gliede, sondern aus vielen. \bibleverse{15}
Wenn der Fuß sagte: »Weil ich nicht Hand bin, gehöre ich nicht zum
Leibe«, so gehört er darum doch zum Leibe; \bibleverse{16} und wenn das
Ohr sagte: »Weil ich kein Auge bin, gehöre ich nicht zum Leibe«, so
gehört es darum doch zum Leibe. \bibleverse{17} Wenn der ganze Leib nur
Auge wäre, wo bliebe da das Gehör? Wenn er ganz Gehör wäre, wo bliebe da
der Geruchssinn? \bibleverse{18} Nun aber hat Gott jedem einzelnen
Gliede seine besondere Stelle am Leibe angewiesen, wie es seinem Willen
entsprach. \bibleverse{19} Wäre das Ganze nur ein einziges Glied, wo
bliebe da der Leib? \bibleverse{20} So aber sind zwar viele Glieder
vorhanden, aber es besteht doch nur ein Leib. \bibleverse{21} Das Auge
kann aber nicht zu der Hand sagen: »Ich habe dich nicht nötig«,
ebensowenig der Kopf zu den Füßen: »Ich habe euch nicht nötig«;
\bibleverse{22} ganz im Gegenteil: die scheinbar schwächsten Glieder des
Leibes sind gerade notwendig, \bibleverse{23} und denjenigen
Körperteilen, die wir für weniger edel halten, erweisen wir besondere
Ehre, und die weniger anständigen Teile unsers Leibes erhalten eine
besonders wohlanständige Ausstattung, \bibleverse{24} deren unsere
anständigen Glieder nicht bedürfen. Ja, Gott hat den Leib so
zusammengefügt, daß er dem weniger wichtigen Gliede desto größere Ehre
zugeteilt hat, \bibleverse{25} damit keine Uneinigkeit im Leibe
herrsche, sondern die Glieder einträchtig füreinander sorgen.
\bibleverse{26} Und wenn ein Glied leidet, so leiden alle Glieder mit,
und wenn ein Glied besonders geehrt\textless sup title=``oder: herrlich
gehalten''\textgreater✲ wird, so freuen sich alle Glieder mit.

\hypertarget{dd-anwendung-des-bildes-auf-die-gottgeordnete-gliederung-der-gemeinde}{%
\subparagraph{dd) Anwendung des Bildes auf die gottgeordnete Gliederung
der
Gemeinde}\label{dd-anwendung-des-bildes-auf-die-gottgeordnete-gliederung-der-gemeinde}}

\bibleverse{27} Ihr aber seid Christi Leib, und jeder einzelne ist ein
Glied daran nach seinem Teil; \bibleverse{28} und zwar hat Gott in der
Gemeinde eingesetzt erstens die einen zu Aposteln, zweitens (andere) zu
Propheten\textless sup title=``oder: geisterfüllten
Predigern''\textgreater✲, drittens (noch andere) zu Lehrern; sodann
Wunderkräfte, sodann Gaben der Heilungen, Hilfeleistungen,
Verwaltungsgeschäfte, mancherlei Arten von Zungenreden. \bibleverse{29}
Sind etwa alle (Gemeindeglieder) Apostel? Etwa alle
Propheten\textless sup title=``oder: geisterfüllte
Prediger''\textgreater✲? Alle Lehrer? Besitzen etwa alle Wunderkräfte?
\bibleverse{30} Haben etwa alle Heilungsgaben? Reden alle mit Zungen?
Können alle die Zungensprachen auslegen?

\hypertarget{b-das-hohelied-von-der-liebe-als-der-huxf6chsten-geistesgabe}{%
\paragraph{b) Das Hohelied von der Liebe als der höchsten
Geistesgabe}\label{b-das-hohelied-von-der-liebe-als-der-huxf6chsten-geistesgabe}}

\hypertarget{aa-ohne-liebe-sind-auch-die-huxf6chsten-geistesgaben-wertlos}{%
\subparagraph{aa) Ohne Liebe sind auch die höchsten Geistesgaben
wertlos}\label{aa-ohne-liebe-sind-auch-die-huxf6chsten-geistesgaben-wertlos}}

\bibleverse{31} Strebet nun eifrig nach den höchsten✲ Gnadengaben! Und
jetzt will ich euch noch einen ganz unvergleichlichen Weg zeigen:

\hypertarget{section-12}{%
\section{13}\label{section-12}}

\bibleverse{1} Wenn ich in den Zungensprachen der Menschen und der Engel
reden könnte, aber die Liebe nicht besäße, so wäre ich nur ein tönendes
Erz oder eine klingende Schelle. \bibleverse{2} Und wenn ich die Gabe
prophetischer Rede besäße und alle Geheimnisse wüßte und alle Erkenntnis
und wenn ich allen Glauben besäße, so daß ich Berge versetzen könnte,
aber die Liebe mir fehlte, so wäre ich nichts. \bibleverse{3} Und wenn
ich alle meine Habe (an die Armen) austeilte und meinen Leib dem
Feuertode preisgäbe, aber keine Liebe besäße, so würde es mir nichts
nützen.

\hypertarget{bb-das-wesen-der-liebe}{%
\subparagraph{bb) Das Wesen der Liebe}\label{bb-das-wesen-der-liebe}}

\bibleverse{4} Die Liebe ist langmütig, ist gütig\textless sup
title=``oder: freundlich''\textgreater✲; die Liebe ist frei von
Eifersucht (und Neid), die Liebe prahlt nicht, sie bläht sich nicht auf,
\bibleverse{5} sie ist nicht rücksichtslos\textless sup title=``oder:
tut nichts Unschickliches''\textgreater✲, sie sucht nicht den eigenen
Vorteil, läßt sich nicht erbittern, rechnet das Böse nicht
an\textless sup title=``=~trägt es nicht nach''\textgreater✲;
\bibleverse{6} sie freut sich nicht über die Ungerechtigkeit, freut sich
vielmehr (im Bunde) mit der Wahrheit; \bibleverse{7} sie deckt alles
zu\textless sup title=``=~entschuldigt alles''\textgreater✲, sie glaubt
alles, sie hofft alles, sie erträgt\textless sup title=``oder:
erduldet''\textgreater✲ alles.

\hypertarget{cc-die-vollkommenheit-der-ewig-bleibenden-liebe-gegenuxfcber-dem-stuxfcckwerk-der-anderen-gnadengaben}{%
\subparagraph{cc) Die Vollkommenheit der ewig bleibenden Liebe gegenüber
dem Stückwerk der anderen
Gnadengaben}\label{cc-die-vollkommenheit-der-ewig-bleibenden-liebe-gegenuxfcber-dem-stuxfcckwerk-der-anderen-gnadengaben}}

\bibleverse{8} Die Liebe hört niemals auf. Die Gabe prophetischer Rede
wird ein Ende nehmen, die Zungenreden werden aufhören, die
Erkenntnis\textless sup title=``oder: das Wissen''\textgreater✲ wird ein
Ende haben. \bibleverse{9} Denn Stückwerk ist unser Erkennen und
Stückwerk unsere prophetische Redegabe, \bibleverse{10} und wenn das
Vollkommene\textless sup title=``oder: die Vollendung''\textgreater✲
kommt, dann wird das Stückwerk ein Ende haben. \bibleverse{11} Als ich
ein Kind war, redete ich wie ein Kind, hatte einen Sinn wie ein Kind,
urteilte wie ein Kind; seit ich aber ein Mann geworden bin, habe ich das
kindische Wesen abgetan. \bibleverse{12} Denn jetzt sehen wir in einem
Spiegel nur undeutliche Bilder, dann✲ aber von Angesicht zu Angesicht.
Jetzt ist mein Erkennen nur Stückwerk; dann✲ aber werde ich ganz
erkennen, wie auch ich ganz erkannt worden bin. \bibleverse{13} Nun aber
bleiben Glaube, Hoffnung, Liebe, diese drei; die größte unter diesen
aber ist die Liebe.

\hypertarget{section-13}{%
\section{14}\label{section-13}}

\bibleverse{1} Jaget also der Liebe nach! Doch bemüht euch (daneben)
auch um die Geistesgaben, besonders aber um die Gabe der
prophetischen\textless sup title=``oder: geisterfüllten''\textgreater✲
Rede.

\hypertarget{c-von-der-vorzuxfcglichkeit-der-prophetischen-oder-geisterfuxfcllten-beredsamkeit-besonders-dem-zungenreden-gegenuxfcber}{%
\paragraph{c) Von der Vorzüglichkeit der prophetischen (oder:
geisterfüllten) Beredsamkeit, besonders dem Zungenreden
gegenüber}\label{c-von-der-vorzuxfcglichkeit-der-prophetischen-oder-geisterfuxfcllten-beredsamkeit-besonders-dem-zungenreden-gegenuxfcber}}

\hypertarget{aa-der-unterschied-zwischen-der-prophetischen-rede-und-der-zungenrede}{%
\subparagraph{aa) Der Unterschied zwischen der prophetischen Rede und
der
Zungenrede}\label{aa-der-unterschied-zwischen-der-prophetischen-rede-und-der-zungenrede}}

\bibleverse{2} Denn der Zungenredner redet nicht zu Menschen, sondern zu
Gott; niemand versteht ihn ja, vielmehr redet er im Geist\textless sup
title=``oder: unter der Einwirkung des Geistes''\textgreater✲
Geheimnisse. \bibleverse{3} Der prophetisch\textless sup title=``oder:
geisterfüllt''\textgreater✲ Redende dagegen redet zu Menschen zu ihrer
Erbauung und Ermahnung und Tröstung. \bibleverse{4} Der Zungenredner
erbaut sich selbst, der prophetisch Redende dagegen erbaut die Gemeinde.
\bibleverse{5} Ich möchte, daß ihr allesamt mit Zungen
redetet\textless sup title=``oder: reden könntet''\textgreater✲, aber
noch lieber, daß ihr prophetische Redegabe besäßet; denn der prophetisch
Redende steht über dem Zungenredner, es müßte denn sein, daß er (das von
ihm Geredete) auch auslegt, damit die Gemeinde Erbauung dadurch
empfängt. \bibleverse{6} So aber, liebe Brüder -- wenn ich als
Zungenredner zu euch käme, was würde ich euch da nützen, wenn ich an
euch nicht (auch) Worte der Offenbarung oder Erkenntnis, der
prophetischen Zusprache oder der Belehrung richtete?

\hypertarget{bb-die-nutzlosigkeit-und-zweckwidrigkeit-alles-unverstuxe4ndlichen-tuxf6nens-und-redens}{%
\subparagraph{bb) Die Nutzlosigkeit und Zweckwidrigkeit alles
unverständlichen Tönens und
Redens}\label{bb-die-nutzlosigkeit-und-zweckwidrigkeit-alles-unverstuxe4ndlichen-tuxf6nens-und-redens}}

\bibleverse{7} Wenn die seelenlosen\textless sup title=``oder:
unbelebten''\textgreater✲ Musikwerkzeuge, obwohl sie einen Klang✲ von
sich geben, z.B. eine Flöte oder Zither, die einzelnen Töne nicht
unterscheiden lassen, wie soll man da erkennen, was auf der Flöte oder
auf der Zither gespielt wird? \bibleverse{8} Ebenso auch, wenn eine
Trompete nur einen undeutlichen Schall hören läßt, wer wird sich da zum
Kampfe rüsten? \bibleverse{9} So steht es auch bei euch: wenn ihr beim
Zungenreden keine deutlichen Worte vernehmen laßt, wie soll man da das
Gesprochene verstehen? Ihr werdet dann eben nur in den Wind\textless sup
title=``oder: in die Luft''\textgreater✲ reden. \bibleverse{10} Es gibt
wer weiß wie viele verschiedene Sprachen in der Welt, und keine ist (an
und für sich) unverständlich; \bibleverse{11} wenn ich aber die
Bedeutung (der Wörter) einer Sprache nicht kenne, so werde ich für den
in ihr Redenden ein Fremdling sein, und der in ihr Redende bleibt für
mich ein Fremdling. \bibleverse{12} So steht es auch mit euch: weil ihr
nach den Erweisungen des Geistes\textless sup title=``=~nach
Geistesgaben''\textgreater✲ eifrig trachtet, so seid darauf bedacht, zur
Erbauung der Gemeinde eine reiche Fülle von ihnen zu haben\textless sup
title=``oder: zu erhalten''\textgreater✲. \bibleverse{13} Deshalb möge
der Zungenredner auch um die Gabe der Auslegung beten; \bibleverse{14}
denn wenn ich in Zungenrede bete, so betet dabei wohl mein Geist, aber
mein Verstand bleibt unbeteiligt. \bibleverse{15} Was folgt nun daraus?
Ich will mit dem Geist\textless sup title=``d.h. in der
Verzückung''\textgreater✲ beten, will aber auch mit dem Verstande beten;
ich will Psalmen\textless sup title=``=~geisterfüllte Lieder; vgl. Eph
5,19''\textgreater✲ mit dem Geist singen, aber auch mit dem Verstande;
\bibleverse{16} sonst wenn du ein Dankgebet nur mit dem Geist sprichst,
wie soll da einer, der den Platz des Laien\textless sup
title=``=~Unkundigen; vgl. V.23''\textgreater✲ einnimmt, das Amen zu
deinem Dankgebet sprechen? Er versteht ja gar nicht, was du sagst.
\bibleverse{17} Du für deine Person magst wohl ein treffliches Dankgebet
sprechen, aber der andere wird dadurch nicht erbaut. \bibleverse{18} Ich
danke Gott: mehr als ihr alle rede ich in Zungen; \bibleverse{19} aber
in einer Gemeindeversammlung will ich lieber fünf Worte mit meinem
Verstande reden, um auch andere zu unterweisen, als viele tausend Worte
in Zungenrede.

\hypertarget{cc-auch-das-alte-testament-und-die-nichtchristliche-auuxdfenwelt-urteilt-verwerflich-uxfcber-das-unverstuxe4ndliche-reden}{%
\subparagraph{cc) Auch das Alte Testament und die nichtchristliche
Außenwelt urteilt verwerflich über das unverständliche
Reden}\label{cc-auch-das-alte-testament-und-die-nichtchristliche-auuxdfenwelt-urteilt-verwerflich-uxfcber-das-unverstuxe4ndliche-reden}}

\bibleverse{20} Liebe Brüder, zeigt euch nicht als Kinder in der
Urteilskraft! Nein, in der Bosheit sollt ihr Kinder sein, aber
hinsichtlich der Urteilskraft zeigt euch als Erwachsene\textless sup
title=``=~gereifte Menschen''\textgreater✲. \bibleverse{21} Im Gesetz
steht geschrieben\textless sup title=``Jes 28,11-12''\textgreater✲:
»Durch Menschen mit fremder Sprache und durch die Lippen von Fremden
will ich zu diesem Volke reden, und auch so werden sie nicht auf mich
hören, spricht der Herr.« \bibleverse{22} Mithin sind die Zungenreden
ein Zeichen nicht für die Gläubigen, sondern für die Ungläubigen; die
prophetische Beredsamkeit✲ dagegen ist ein solches nicht für die
Ungläubigen, sondern für die Gläubigen. \bibleverse{23} Wenn also die
ganze Gemeinde sich an einem Ort versammelte und alle in Zungensprachen
redeten und dann Laien\textless sup title=``=~Nicht-Unterrichtete,
Uneingeweihte''\textgreater✲ oder Ungläubige hereinkämen, würden diese
da nicht sagen, ihr seiet von Sinnen? \bibleverse{24} Wenn dagegen alle
prophetisch reden\textless sup title=``d.h. sich in geisterfüllten
Ansprachen ergehen''\textgreater✲ und dann ein Ungläubiger oder ein Laie
dazukommt, so wird ihm von allen ins Gewissen geredet\textless sup
title=``=~seine Sünde vorgehalten''\textgreater✲, er fühlt sich von
allen ins Gericht genommen, \bibleverse{25} die geheimen Gedanken seines
Herzens werden aufgedeckt, und so wird er auf sein Angesicht fallen und
Gott anbeten und offen bekennen, daß Gott tatsächlich in euch (wirksam)
ist.

\hypertarget{d-vorschriften-fuxfcr-die-ordnung-in-den-erbaulichen-versammlungen}{%
\paragraph{d) Vorschriften für die Ordnung in den erbaulichen
Versammlungen}\label{d-vorschriften-fuxfcr-die-ordnung-in-den-erbaulichen-versammlungen}}

\hypertarget{aa-ordnung-fuxfcr-die-redner}{%
\subparagraph{aa) Ordnung für die
Redner}\label{aa-ordnung-fuxfcr-die-redner}}

\bibleverse{26} Was folgt nun daraus, ihr Brüder? Sooft ihr euch
versammelt, hat ein jeder (etwas in Bereitschaft): ein geistliches
Lied\textless sup title=``vgl. Eph 5,19''\textgreater✲, einen
belehrenden Vortrag, eine Offenbarung, eine Zungenrede, eine Auslegung
(derselben) -- das alles laßt zur Erbauung (der Gemeinde) dienen!
\bibleverse{27} Will man in Zungensprachen reden, so sollen es jedesmal
nur zwei oder höchstens drei sein, und zwar der Reihe nach, und einer
soll die Auslegung geben. \bibleverse{28} Ist jedoch kein Ausleger da,
so soll er (der Zungenredner) in der Versammlung schweigen: er mag dann
für sich allein und zu\textless sup title=``oder: für''\textgreater✲
Gott reden. \bibleverse{29} Propheten\textless sup title=``oder:
geisterfüllte Redner''\textgreater✲ sollen gleichfalls nur zwei oder
drei zu Worte kommen und die anderen\textless sup title=``d.h. die
Zuhörer''\textgreater✲ sich ein Urteil darüber bilden\textless sup
title=``oder: abgeben''\textgreater✲. \bibleverse{30} Wenn aber einem
anderen, der noch dasitzt, eine Offenbarung zuteil wird, so soll der
erste schweigen; \bibleverse{31} denn ihr könnt alle einzeln✲ als
prophetische Redner auftreten, damit alle Belehrung empfangen und alle
ermahnt werden; \bibleverse{32} und die prophetischen Geister sind ja
auch den Propheten gehorsam~-- \bibleverse{33} denn Gott ist nicht ein
Gott der Unordnung, sondern des Friedens -- wie in allen Gemeinden der
Heiligen.

\hypertarget{bb-gegen-unziemliches-reden-der-frauen-in-den-versammlungen}{%
\subparagraph{bb) Gegen unziemliches Reden der Frauen in den
Versammlungen}\label{bb-gegen-unziemliches-reden-der-frauen-in-den-versammlungen}}

\bibleverse{34} Die Frauen sollen in den Gemeindeversammlungen
schweigen, denn es kann ihnen nicht gestattet werden zu reden, sondern
sie haben sich unterzuordnen, wie auch das (mosaische) Gesetz es
gebietet\textless sup title=``1.Mose 3,16''\textgreater✲.
\bibleverse{35} Wünschen sie aber Belehrung über irgend etwas, so mögen
sie daheim ihre Ehemänner befragen; denn es steht einer Frau übel an,
sich in einer Gemeindeversammlung hören zu lassen. \bibleverse{36} Oder
ist etwa das Wort Gottes von euch ausgegangen oder zu euch allein
hingekommen?

\hypertarget{e-schluuxdfwort-zu-dem-in-den-kapiteln-12-14-gesagten}{%
\paragraph{e) Schlußwort zu dem in den Kapiteln 12-14
Gesagten}\label{e-schluuxdfwort-zu-dem-in-den-kapiteln-12-14-gesagten}}

\bibleverse{37} Wenn jemand sich für einen Propheten\textless sup
title=``oder: geisterfüllten Redner''\textgreater✲ oder (überhaupt) für
einen Geistbegabten hält, so muß er erkennen, daß das, was ich euch hier
schreibe, das Gebot des Herrn ist. \bibleverse{38} Wenn jemand es aber
nicht anerkennen will, so wird er (auch von Gott) nicht (an)erkannt.
\bibleverse{39} Also, meine Brüder: bemüht euch eifrig um die Gabe
prophetischer Beredsamkeit und hindert\textless sup title=``oder:
unterdrückt''\textgreater✲ auch das Zungenreden nicht! \bibleverse{40}
Laßt aber alles mit Anstand und in Ordnung vor sich gehen!

\hypertarget{belehrung-uxfcber-die-auferweckung-der-toten}{%
\subsubsection{6. Belehrung über die Auferweckung der
Toten}\label{belehrung-uxfcber-die-auferweckung-der-toten}}

\hypertarget{a-von-den-tatsachen-und-zeugen-durch-welche-die-auferweckung-christi-beglaubigt-ist}{%
\paragraph{a) Von den Tatsachen und Zeugen, durch welche die
Auferweckung Christi beglaubigt
ist}\label{a-von-den-tatsachen-und-zeugen-durch-welche-die-auferweckung-christi-beglaubigt-ist}}

\hypertarget{section-14}{%
\section{15}\label{section-14}}

\bibleverse{1} Ich weise euch aber, liebe Brüder, auf die Heilsbotschaft
hin, die ich euch (seinerzeit) getreulich verkündigt habe, die ihr auch
angenommen habt, in der ihr auch fest steht \bibleverse{2} und durch die
ihr auch die Rettung\textless sup title=``oder: das Heil''\textgreater✲
erlangt, wenn ihr sie in der Gestalt festhaltet, in welcher ich sie euch
getreulich verkündigt habe; es müßte sonst sein, daß ihr vergeblich zum
Glauben gekommen wäret. \bibleverse{3} Ich habe euch nämlich an erster
Stelle mitgeteilt, was ich auch überkommen habe, daß Christus für unsere
Sünden gestorben ist, den Schriften gemäß\textless sup title=``Jes
53''\textgreater✲, \bibleverse{4} und daß er begraben und daß er am
dritten Tage auferweckt worden ist, den Schriften gemäß\textless sup
title=``Hos 6,2; Ps 16,10''\textgreater✲, \bibleverse{5} und daß er dem
Kephas✲ erschienen ist, danach den Zwölfen. \bibleverse{6} Darauf ist er
mehr als fünfhundert Brüdern auf einmal erschienen, von denen die
meisten jetzt noch leben, einige aber entschlafen sind. \bibleverse{7}
Darauf ist er dem Jakobus erschienen, danach sämtlichen Aposteln.
\bibleverse{8} Zuallerletzt aber ist er gleichsam als der
Fehlgeburt\textless sup title=``=~einer unzeitigen Geburt''\textgreater✲
auch mir erschienen. \bibleverse{9} Denn ich bin der geringste unter den
Aposteln und des Apostelnamens nicht würdig, weil ich die Gemeinde
Gottes verfolgt habe. \bibleverse{10} Durch Gottes Gnade aber bin ich,
was ich bin, und seine Gnade gegen mich hat sich nicht erfolglos
erwiesen, sondern ich habe weit mehr geschafft\textless sup
title=``oder: Arbeit geleistet''\textgreater✲ als sie alle, doch nicht
ich, sondern die Gnade Gottes, die mit mir ist\textless sup
title=``oder: gewesen ist''\textgreater✲. \bibleverse{11} Ganz gleich
nun, ob ich (es bin) oder jene: so lautet unsere Verkündigung✲, und so
seid ihr zum Glauben gekommen.

\hypertarget{b-auf-christi-auferweckung-von-den-toten-beruht-der-gesamte-glaube-und-die-feste-hoffnung-der-christen}{%
\paragraph{b) Auf Christi Auferweckung von den Toten beruht der gesamte
Glaube und die feste Hoffnung der
Christen}\label{b-auf-christi-auferweckung-von-den-toten-beruht-der-gesamte-glaube-und-die-feste-hoffnung-der-christen}}

\bibleverse{12} Wenn aber unsere Predigt die Auferweckung Christi von
den Toten verkündigt, wie kommen da einige unter euch zu der Behauptung,
daß es eine Auferstehung der Toten nicht gebe? \bibleverse{13} Gibt es
nämlich keine Auferstehung der Toten, so ist auch Christus nicht
auferweckt worden; \bibleverse{14} ist aber Christus nicht auferweckt
worden, so ist unsere Predigt leer✲ und leer auch euer Glaube.
\bibleverse{15} Dann werden aber auch wir als falsche Zeugen in Gottes
Sache erfunden, weil wir gegen Gott das Zeugnis abgelegt haben, daß er
Christus auferweckt habe, während er ihn doch nicht auferweckt hat, wenn
es wirklich keine Auferweckung der Toten gibt. \bibleverse{16} Denn wenn
Tote (überhaupt) nicht auferweckt werden, so ist auch Christus nicht
auferweckt worden; \bibleverse{17} wenn aber Christus nicht auferweckt
worden ist, so ist euer Glaube nichtig\textless sup title=``oder:
wertlos''\textgreater✲; dann seid ihr noch in euren Sünden;
\bibleverse{18} dann sind also auch die in Christus Entschlafenen
verloren(gegangen)! \bibleverse{19} Wenn wir weiter nichts sind als
solche, die in diesem Leben ihre Hoffnung auf Christus gesetzt haben, so
sind wir die beklagenswertesten unter allen Menschen.

\hypertarget{c-darlegung-der-folgen-der-auferweckung-christi-die-vorguxe4nge-in-denen-sich-die-auferweckung-bis-zur-vollendung-vollzieht}{%
\paragraph{c) Darlegung der Folgen der Auferweckung Christi; die
Vorgänge, in denen sich die Auferweckung bis zur Vollendung
vollzieht}\label{c-darlegung-der-folgen-der-auferweckung-christi-die-vorguxe4nge-in-denen-sich-die-auferweckung-bis-zur-vollendung-vollzieht}}

\bibleverse{20} Nun aber ist Christus von den Toten auferweckt worden
(und zwar) als Erstling der Entschlafenen. \bibleverse{21} Denn weil der
Tod durch einen Menschen gekommen\textless sup title=``=~verursacht
worden''\textgreater✲ ist, erfolgt auch die Auferstehung der Toten durch
einen Menschen. \bibleverse{22} Wie nämlich in Adam alle sterben, so
werden auch✲ in Christus alle wieder zum Leben gebracht werden,
\bibleverse{23} ein jeder aber in seiner besonderen Abteilung: als
Erstling Christus, hierauf die, welche Christus angehören, bei seiner
Ankunft✲, \bibleverse{24} danach das Ende\textless sup title=``oder:
Endergebnis =~der Abschluß''\textgreater✲, wenn er Gott {[}und{]} dem
Vater das Reich\textless sup title=``oder: Königtum''\textgreater✲
übergibt, sobald er jede (andere) Herrschaft und jede Gewalt und Macht
vernichtet hat; \bibleverse{25} denn er muß als König herrschen, »bis er
ihm alle Feinde unter die Füße gelegt hat«\textless sup title=``Ps
110,1''\textgreater✲. \bibleverse{26} Der letzte Feind, der vernichtet
wird, ist der Tod; \bibleverse{27} denn »alles hat er ihm unter die Füße
gelegt«\textless sup title=``Ps 8,7''\textgreater✲. Wenn er dann aber
aussprechen wird: »Alles ist unterworfen!«, so ist doch
selbstverständlich der ausgenommen, der ihm alles unterworfen hat.
\bibleverse{28} Sobald ihm aber alles unterworfen ist, dann wird auch
der Sohn selbst sich dem unterwerfen, der ihm alles unterworfen hat,
damit Gott (alsdann) alles sei in allen\textless sup title=``oder: in
allem''\textgreater✲.

\hypertarget{d-nur-beim-glauben-an-die-auferweckung-ist-vieles-was-die-christen-tun-und-leiden-begruxfcndet-und-verstuxe4ndlich}{%
\paragraph{d) Nur beim Glauben an die Auferweckung ist vieles, was die
Christen tun und leiden, begründet und
verständlich}\label{d-nur-beim-glauben-an-die-auferweckung-ist-vieles-was-die-christen-tun-und-leiden-begruxfcndet-und-verstuxe4ndlich}}

\bibleverse{29} Wie kämen sonst manche dazu, sich für die Toten taufen
zu lassen? Wenn Tote überhaupt nicht auferweckt werden, wozu läßt man
sich da noch für sie taufen? \bibleverse{30} Und wir? Wozu setzen wir
uns da Stunde für Stunde Gefahren aus? \bibleverse{31} Tagtäglich sterbe
ich\textless sup title=``=~muß ich auf den Tod gefaßt
sein''\textgreater✲, so wahr ihr, liebe Brüder, mein Ruhm seid, den ich
in Christus Jesus, unserm Herrn, habe. \bibleverse{32} Wenn ich nach der
Weise der Menschen in Ephesus mit wilden Tieren gekämpft habe, was hilft
mir das? Wenn die Toten nicht auferweckt werden, so »laßt uns essen und
trinken, denn morgen sind wir tot!«\textless sup title=``Jes
22,13''\textgreater✲ \bibleverse{33} Laßt euch nicht irreführen!
»Schlechter Umgang verderbt gute Sitten.« \bibleverse{34} Werdet
nüchtern, wie es sich gehört, und sündigt nicht; denn manchen fehlt die
richtige Gotteserkenntnis: zur Beschämung muß ich euch das sagen!

\hypertarget{e-von-der-art-der-auferweckung-und-von-dem-auferstehungsleibe}{%
\paragraph{e) Von der Art der Auferweckung und von dem
Auferstehungsleibe}\label{e-von-der-art-der-auferweckung-und-von-dem-auferstehungsleibe}}

\hypertarget{aa-das-bild-vom-samenkorn}{%
\subparagraph{aa) Das Bild vom
Samenkorn}\label{aa-das-bild-vom-samenkorn}}

\bibleverse{35} »Aber«, wird mancher fragen, »wie werden die Toten
auferweckt, und mit was für einem Leibe erscheinen sie?« \bibleverse{36}
Du Tor! Der Same, den du säst, bekommt doch auch nur dann Leben, wenn er
(zuvor) erstorben ist; \bibleverse{37} und was du säen magst: du säst
damit doch nicht schon den Leib, der erst noch entstehen wird, sondern
ein nacktes\textless sup title=``oder: bloßes''\textgreater✲ Samenkorn,
zum Beispiel von Weizen oder von sonst einem Gewächs. \bibleverse{38}
Gott aber gibt ihm einen Leib nach seinem Belieben, und zwar einer jeden
Samenart einen besonderen Leib.

\hypertarget{bb-die-ganze-schuxf6pfung-zeigt-die-gruxf6uxdfte-mannigfaltigkeit-der-stoffe-der-gestalt-und-der-beschaffenheit-der-dinge}{%
\subparagraph{bb) Die ganze Schöpfung zeigt die größte Mannigfaltigkeit
der Stoffe, der Gestalt und der Beschaffenheit der
Dinge}\label{bb-die-ganze-schuxf6pfung-zeigt-die-gruxf6uxdfte-mannigfaltigkeit-der-stoffe-der-gestalt-und-der-beschaffenheit-der-dinge}}

\bibleverse{39} Nicht jedes Fleisch hat die gleiche Beschaffenheit,
sondern anders ist das Fleisch der Menschen beschaffen, anders das der
vierfüßigen Tiere, anders das Fleisch der Vögel, anders das der Fische.
\bibleverse{40} Auch gibt es himmlische Leiber und irdische Leiber; aber
andersartig ist die Herrlichkeit der himmlischen, andersartig die
(äußere Erscheinung) der irdischen Leiber. \bibleverse{41} Einen anderen
Glanz hat die Sonne, einen anderen der Mond, und einen anderen Glanz
haben die Sterne; denn jeder Stern ist von dem anderen an Glanz
verschieden. \bibleverse{42} Ebenso verhält es sich auch mit der
Auferstehung der Toten: Es wird gesät in Vergänglichkeit, auferweckt in
Unvergänglichkeit; \bibleverse{43} es wird gesät in Unehre\textless sup
title=``oder: Armseligkeit''\textgreater✲, auferweckt in Herrlichkeit;
gesät wird in Schwachheit, auferweckt in Kraft; \bibleverse{44} gesät
wird ein seelischer\textless sup title=``oder:
natürlicher''\textgreater✲ Leib, auferweckt ein geistlicher Leib. So gut
es einen seelischen\textless sup title=``oder:
natürlichen''\textgreater✲ Leib gibt, so gibt es auch einen geistlichen.

\hypertarget{cc-die-wirklichkeit-eines-himmlischen-unverweslichen-leibes}{%
\subparagraph{cc) Die Wirklichkeit eines himmlischen (=~unverweslichen)
Leibes}\label{cc-die-wirklichkeit-eines-himmlischen-unverweslichen-leibes}}

\bibleverse{45} So\textless sup title=``=~in diesem Sinn''\textgreater✲
steht auch geschrieben\textless sup title=``1.Mose 2,7''\textgreater✲:
»Der erste Mensch Adam wurde zu einem lebendigen\textless sup
title=``=~Leben habenden''\textgreater✲ Seelenwesen«, der letzte
Adam\textless sup title=``d.h. Jesus; vgl. Röm 5,12-19''\textgreater✲ zu
einem lebenschaffenden Geisteswesen. \bibleverse{46} Doch nicht das
Geistliche kommt dabei zuerst, sondern das Seelische\textless sup
title=``oder: Natürliche''\textgreater✲, danach erst das Geistliche.
\bibleverse{47} Der erste Mensch ist von der Erde her, ist
erdig\textless sup title=``oder: irdisch''\textgreater✲, der zweite
Mensch (nämlich Christus) ist himmlischen Ursprungs. \bibleverse{48} Wie
der irdische Mensch (Adam) beschaffen ist, so sind auch die irdischen
(Menschen) beschaffen; und wie der himmlische Mensch (Christus)
beschaffen ist, so sind auch die himmlischen (Menschen) beschaffen;
\bibleverse{49} und wie wir das Bild des irdischen (Adam) an uns
getragen haben, so werden wir auch das Bild des himmlischen (Christus)
an uns tragen.

\hypertarget{f-die-schlieuxdfliche-verwandlung-bei-der-vollendung-der-gluxe4ubigen}{%
\paragraph{f) Die schließliche Verwandlung bei der Vollendung der
Gläubigen}\label{f-die-schlieuxdfliche-verwandlung-bei-der-vollendung-der-gluxe4ubigen}}

\bibleverse{50} Das aber versichere ich (euch), liebe Brüder: Fleisch
und Blut können das Reich Gottes nicht ererben; auch kann das
Vergängliche nicht die Unvergänglichkeit ererben. \bibleverse{51}
Seht\textless sup title=``oder: Wisset wohl''\textgreater✲, ich sage
euch ein Geheimnis: Wir werden nicht alle entschlafen, wohl aber werden
wir alle verwandelt werden, \bibleverse{52} (und zwar) im Nu, in einem
Augenblick, beim letzten Posaunenstoß; denn die Posaune wird erschallen,
und sofort werden die Toten in Unvergänglichkeit auferweckt werden, und
wir werden verwandelt werden. \bibleverse{53} Denn dieser vergängliche
Leib muß die Unvergänglichkeit anziehen, und dieser sterbliche Leib muß
die Unsterblichkeit anziehen.

\bibleverse{54} Wenn aber dieser vergängliche Leib die Unvergänglichkeit
angezogen hat und dieser sterbliche Leib die Unsterblichkeit, dann wird
sich das Wort erfüllen, das geschrieben steht\textless sup title=``Jes
25,8; Hos 13,14''\textgreater✲: »Verschlungen ist der Tod in
Sieg\textless sup title=``oder: zum Sieg''\textgreater✲: \bibleverse{55}
Tod, wo ist dein Sieg? \bibleverse{56} Tod, wo ist dein Stachel?« Der
Stachel des Todes ist aber die Sünde, und die Kraft der Sünde liegt im
Gesetz. \bibleverse{57} Gott aber sei Dank, der uns den Sieg verleiht
durch unsern Herrn Jesus Christus! \bibleverse{58} Daher, meine
geliebten Brüder, werdet fest, unerschütterlich, und beteiligt euch
allezeit eifrig am Werk des Herrn; ihr wißt ja, daß eure Arbeit nicht
vergeblich ist im Herrn.

\hypertarget{iii.-geschuxe4ftliches-und-persuxf6nliches-kap.-16}{%
\subsection{III. Geschäftliches und Persönliches (Kap.
16)}\label{iii.-geschuxe4ftliches-und-persuxf6nliches-kap.-16}}

\hypertarget{aufforderung-zur-beteiligung-an-der-geldsammlung-fuxfcr-jerusalem}{%
\subsubsection{1. Aufforderung zur Beteiligung an der Geldsammlung für
Jerusalem}\label{aufforderung-zur-beteiligung-an-der-geldsammlung-fuxfcr-jerusalem}}

\hypertarget{section-15}{%
\section{16}\label{section-15}}

\bibleverse{1} Was sodann die Sammlung für die Heiligen (in Jerusalem)
betrifft, so haltet auch ihr es damit ebenso, wie ich es für die
galatischen Gemeinden angeordnet habe: \bibleverse{2} Am ersten Tage
jeder Woche\textless sup title=``d.h. allsonntäglich''\textgreater✲ lege
jeder von euch (in seinem Hause) etwas beiseite und spare soviel
zusammen, wie seine Verhältnisse es gestatten, damit die Sammlungen
nicht erst nach meiner Ankunft stattzufinden brauchen. \bibleverse{3}
Wenn ich dann bei euch eingetroffen bin, werde ich die von euch
bezeichneten Vertrauensmänner mit Briefen entsenden, damit sie eure
Liebesgabe nach Jerusalem überbringen; \bibleverse{4} ist es aber der
Mühe wert, daß auch ich hinreise, so sollen sie mit mir zusammen reisen.

\hypertarget{die-reisepluxe4ne-des-paulus-und-nachrichten-uxfcber-das-kommen-des-timotheus-und-apollos}{%
\subsubsection{2. Die Reisepläne des Paulus und Nachrichten über das
Kommen des Timotheus und
Apollos}\label{die-reisepluxe4ne-des-paulus-und-nachrichten-uxfcber-das-kommen-des-timotheus-und-apollos}}

\bibleverse{5} Ich werde aber zu euch kommen, wenn ich Mazedonien
durchreist habe, denn Mazedonien durchreise ich nur, \bibleverse{6} bei
euch aber werde ich wohl länger bleiben, vielleicht sogar den ganzen
Winter zubringen, damit ihr mir dann bei meiner Weiterreise das Geleit
geben könnt. \bibleverse{7} Ich möchte euch nämlich diesmal nicht nur
auf der Durchreise sehen, sondern kann hoffentlich eine Zeitlang bei
euch verweilen, wenn es des Herrn Wille ist. \bibleverse{8} Hier in
Ephesus bleibe ich noch bis zum Pfingstfest, \bibleverse{9} denn es hat
sich mir hier Gelegenheit zu vielseitiger und erfolgreicher Wirksamkeit
geboten; freilich fehlt es auch nicht an Gegnern.~-- \bibleverse{10}
Wenn aber Timotheus (zu euch) kommt, so sorgt dafür, daß er ohne Furcht
bei euch verweilen kann; er arbeitet ja am Werk des Herrn ebenso wie
ich. \bibleverse{11} Niemand möge ihn also über die Achsel ansehen!
Entlaßt ihn dann in Frieden, damit er wieder zu mir komme; denn ich
warte auf ihn samt den Brüdern.~-- \bibleverse{12} Was sodann den Bruder
Apollos betrifft, so habe ich ihm dringend zugeredet, er möchte sich mit
den Brüdern zu euch begeben; doch er will jetzt die Reise durchaus nicht
unternehmen; er wird aber kommen, sobald es ihm gelegen ist.

\hypertarget{schluuxdfermahnungen-persuxf6nliche-empfehlungen-gruxfcuxdfe-und-segenswunsch}{%
\subsubsection{3. Schlußermahnungen, persönliche Empfehlungen, Grüße und
Segenswunsch}\label{schluuxdfermahnungen-persuxf6nliche-empfehlungen-gruxfcuxdfe-und-segenswunsch}}

\bibleverse{13} Seid wachsam, steht fest im Glauben, seid mannhaft,
werdet stark! \bibleverse{14} Laßt alles bei euch in Liebe zugehen!~--
\bibleverse{15} Ich habe euch noch auf eines aufmerksam zu machen, liebe
Brüder: Ihr wißt vom Hause des Stephanas, daß er der erste gewesen ist,
der in Achaja✲ bekehrt worden ist und daß sie (er und die Seinen) sich
in den Dienst für die Heiligen gestellt haben. \bibleverse{16} So ordnet
denn auch ihr euch solchen Leuten unter und überhaupt einem jeden, der
da mitarbeitet und sich abmüht! \bibleverse{17} Ich bin erfreut über die
Ankunft des Stephanas, des Fortunatus und Achaikus: sie haben mir für
euer Fernsein Ersatz geleistet \bibleverse{18} und mir wie auch euch
geistige Ruhe\textless sup title=``oder: Erquickung''\textgreater✲
gebracht. Solchen Männern müßt ihr Anerkennung zollen!

\bibleverse{19} Es grüßen euch die Gemeinden der Provinz Asien. Es
grüßen euch herzlich im Herrn Aquila und Priska samt der Gemeinde in
ihrem Hause. \bibleverse{20} Es grüßen euch die Brüder alle. Grüßt ihr
einander mit dem heiligen Kuß.

\bibleverse{21} Hier mein, des Paulus, eigenhändiger Gruß!
\bibleverse{22} Wer den Herrn nicht liebt, der sei verflucht! Maranatha!
\bibleverse{23} Die Gnade des Herrn Jesus sei mit euch! \bibleverse{24}
Meine Liebe ist mit euch allen in Christus Jesus.
