\hypertarget{der-prophet-jeremia}{%
\section{DER PROPHET JEREMIA}\label{der-prophet-jeremia}}

\hypertarget{a.-reden-und-erlebnisse-jeremias-bis-zur-zerstuxf6rung-jerusalems-kap.-1-39}{%
\subsection{A. Reden und Erlebnisse Jeremias bis zur Zerstörung
Jerusalems (Kap.
1-39)}\label{a.-reden-und-erlebnisse-jeremias-bis-zur-zerstuxf6rung-jerusalems-kap.-1-39}}

\hypertarget{i.-uxfcberschrift.-berufung-des-propheten}{%
\subsection{I. Überschrift. Berufung des
Propheten}\label{i.-uxfcberschrift.-berufung-des-propheten}}

\hypertarget{die-uxfcberschrift}{%
\subsubsection{1. Die Überschrift}\label{die-uxfcberschrift}}

\hypertarget{section}{%
\section{1}\label{section}}

\bibleverse{1}(Dies sind) die Reden\textless sup title=``oder:
Aussprüche''\textgreater✲ Jeremias, des Sohnes Hilkias, der zu der
Priesterschaft in Anathoth im Lande✲ Benjamin gehörte; \bibleverse{2}an
ihn erging das Wort des HERRN in den Tagen des jüdäischen Königs Josia,
des Sohnes Amons, im dreizehnten Jahre seiner Regierung,
\bibleverse{3}und erging dann auch noch weiter an ihn in den Tagen des
judäischen Königs Jojakim, des Sohnes Josias, bis zum Ablauf des elften
Regierungsjahres des judäischen Königs Zedekia, des Sohnes Josias, bis
zur Wegführung (der Bewohner) Jerusalems im fünften Monat (des elften
Regierungsjahres Zedekias).

\hypertarget{jeremias-berufung-und-weihe-zum-prophetenamt}{%
\subsubsection{2. Jeremias Berufung und Weihe zum
Prophetenamt}\label{jeremias-berufung-und-weihe-zum-prophetenamt}}

\bibleverse{4}Es erging aber das Wort des HERRN an mich folgendermaßen:
\bibleverse{5}»Noch ehe ich dich im Mutterschoße bildete, habe ich dich
erwählt\textless sup title=``oder: ersehen''\textgreater✲, und ehe du
das Licht der Welt erblicktest, habe ich dich geweiht: zum Propheten für
die Völker habe ich dich bestimmt.« \bibleverse{6}Da antwortete ich:
»Ach, HERR, mein Gott, sieh doch: ich verstehe ja nicht zu reden, denn
ich bin noch so jung!« \bibleverse{7}Doch der HERR erwiderte mir: »Sage
nicht, du seiest noch so jung! Denn\textless sup title=``oder:
sondern''\textgreater✲ zu allen, wohin ich dich senden werde, sollst du
gehen, und alles, was ich dir auftragen werde, sollst du reden.
\bibleverse{8}Fürchte dich nicht vor ihnen; denn ich bin mit dir, um
dich zu behüten!« -- so lautet der Ausspruch des HERRN.
\bibleverse{9}Hierauf streckte der HERR seine Hand aus und berührte
meinen Mund mit ihr; dann sagte der HERR zu mir: »Hiermit lege ich meine
Worte in deinen Mund! \bibleverse{10}Wisse wohl: ich bestelle dich heute
über\textless sup title=``oder: für''\textgreater✲ die Völker und
über\textless sup title=``oder: für''\textgreater✲ die
Königreiche\textless sup title=``oder: Königshäuser''\textgreater✲, um
auszureißen und niederzureißen, zu vernichten und zu zerstören, (aber
auch) um aufzubauen und zu pflanzen.«

\hypertarget{zwei-berufungsgesichte-der-wache-baum-und-der-siedende-kessel}{%
\paragraph{Zwei Berufungsgesichte (der wache Baum und der siedende
Kessel)}\label{zwei-berufungsgesichte-der-wache-baum-und-der-siedende-kessel}}

\bibleverse{11}Weiter erging das Wort des HERRN an mich folgendermaßen:
»Was siehst du, Jeremia?« Ich antwortete: »Einen Zweig vom wachen Baum
sehe ich.« \bibleverse{12}Da sagte der HERR zu mir: »Du hast richtig
gesehen: ja, ich wache\textless sup title=``=~halte die Augen
offen''\textgreater✲ über meinem Wort, um es in Erfüllung gehen zu
lassen!«

\bibleverse{13}Hierauf erging das Wort des HERRN an mich noch einmal
folgendermaßen: »Was siehst du?« Ich antwortete: »Einen siedenden Kessel
sehe ich, dessen Vorderseite✲ von Norden her (gegen Süden) gerichtet
ist.« \bibleverse{14}Da sagte der HERR zu mir: »Ja, von Norden her wird
das Unglück sich siedend über alle Bewohner des Landes ergießen.
\bibleverse{15}Denn gib acht: ich will alle Völkerstämme\textless sup
title=``oder: Horden''\textgreater✲ der Reiche im Norden entbieten« --
so lautet der Ausspruch des HERRN --, »daß sie heranziehen und ein jeder
seinen Thron aufstellt an den Eingang der Tore\textless sup
title=``=~dicht vor die Tore''\textgreater✲ Jerusalems und gegen alle
Mauern der Stadt ringsum und gegen alle Städte Judas.
\bibleverse{16}Dann will ich Abrechnung mit ihnen\textless sup
title=``d.h. den Judäern''\textgreater✲ halten wegen all ihrer Bosheit,
daß sie von mir abgefallen sind und anderen Göttern geopfert und die
Machwerke ihrer Hände angebetet haben.«

\hypertarget{ans-werk-und-hinaus-ins-amt}{%
\paragraph{Ans Werk und hinaus ins
Amt!}\label{ans-werk-und-hinaus-ins-amt}}

\bibleverse{17}»Du aber, gürte dir die Hüften, mache dich auf und
verkünde ihnen alles, was ich dir gebieten werde! Erschrick nicht vor
ihnen, sonst setze ich dich vor ihnen in Schrecken! \bibleverse{18}Denn
wisse wohl: Ich selbst mache dich heute zu einer festen Burg, zu einer
eisernen Säule und zu einer ehernen Mauer gegen das ganze Land, sowohl
gegen die Könige von Juda als auch gegen dessen Fürsten\textless sup
title=``oder: oberste Beamte''\textgreater✲, gegen dessen Priester und
gegen die ganze Bevölkerung des Landes. \bibleverse{19}Wenn sie auch
gegen dich anstürmen, sollen sie dich doch nicht bezwingen; denn ich bin
mit dir« -- so lautet der Ausspruch des HERRN --, »um dich zu behüten!«

\hypertarget{ii.-mahn--und-strafreden-kap.-2-29}{%
\subsection{II. Mahn- und Strafreden (Kap.
2-29)}\label{ii.-mahn--und-strafreden-kap.-2-29}}

\hypertarget{strafreden-hauptsuxe4chlich-aus-der-zeit-des-kuxf6nigs-josia-kap.-2-6}{%
\subsubsection{1. Strafreden hauptsächlich aus der Zeit des Königs Josia
(Kap.
2-6)}\label{strafreden-hauptsuxe4chlich-aus-der-zeit-des-kuxf6nigs-josia-kap.-2-6}}

\hypertarget{a-jeremias-erste-strafrede}{%
\paragraph{a) Jeremias erste
Strafrede}\label{a-jeremias-erste-strafrede}}

\hypertarget{aa-israels-anfuxe4ngliche-treue-gegen-seinen-gott-und-sein-spuxe4terer-schnuxf6der-abfall-mit-seinen-unheilvollen-folgen}{%
\subparagraph{aa) Israels anfängliche Treue gegen seinen Gott und sein
späterer schnöder Abfall mit seinen unheilvollen
Folgen}\label{aa-israels-anfuxe4ngliche-treue-gegen-seinen-gott-und-sein-spuxe4terer-schnuxf6der-abfall-mit-seinen-unheilvollen-folgen}}

\hypertarget{section-1}{%
\section{2}\label{section-1}}

\bibleverse{1}Nun erging das Wort des HERRN an mich folgendermaßen:
\bibleverse{2}»Gehe hin und rufe (dem Volk in) Jerusalem laut in die
Ohren: ›So hat der HERR gesprochen: Ich gedenke an die
Holdseligkeit\textless sup title=``oder: Zuneigung''\textgreater✲ deiner
Jugend, deine bräutliche Liebe, wie du hinter mir herzogst in der Wüste,
im unwirtlichen Lande. \bibleverse{3}Geheiligt war Israel (damals) dem
HERRN, sein Erstlingsabhub von der Ernte; alle, die sich an ihm
vergriffen, mußten es büßen: Unheil kam über sie‹« -- so lautet der
Ausspruch des HERRN. \bibleverse{4}Vernehmt das Wort des HERRN, ihr vom
Hause Jakob und alle ihr Geschlechter des Hauses Israel!
\bibleverse{5}So hat der HERR gesprochen: »Was haben eure Väter
Unrechtes an mir gefunden, daß sie sich von mir losgesagt haben und der
Nichtigkeit\textless sup title=``=~den nichtigen Götzen''\textgreater✲
nachgelaufen und auf Nichtiges verfallen\textless sup title=``oder:
zunichte geworden''\textgreater✲ sind? \bibleverse{6}Sie fragten nicht
mehr: ›Wo ist der HERR, der uns aus Ägyptenland hergeführt, der uns
durch die Wüste geleitet hat, durch ein Land der Steppen und Schluchten,
durch ein Land der Dürre und des Dunkels, durch ein Land, das kein
Wanderer durchzieht und in welchem kein Mensch Wohnung nimmt?‹
\bibleverse{7}Als ich euch dann in das Land der Fruchtgefilde gebracht
hatte, damit ihr dessen Früchte und Segen✲ genösset, da seid ihr
hineingekommen und habt mein Land entweiht und mein Besitztum zu einer
Greuelstätte gemacht! \bibleverse{8}Die Priester fragten nicht: ›Wo ist
der HERR?‹, und die Hüter des Gesetzes kannten mich nicht, die Hirten✲
des Volkes fielen von mir ab, und die Propheten weissagten durch den
Baal\textless sup title=``oder: im Namen Baals''\textgreater✲ und liefen
den Götzen nach, die doch nicht zu helfen vermögen. \bibleverse{9}Darum
muß ich noch weiter mit euch ins Gericht gehen« -- so lautet der
Ausspruch des HERRN -- »und werde noch mit euren Kindeskindern ins
Gericht gehen!«

\hypertarget{bb-israels-verhalten-ist-unerhuxf6rt-und-beispiellos}{%
\subparagraph{bb) Israels Verhalten ist unerhört und
beispiellos}\label{bb-israels-verhalten-ist-unerhuxf6rt-und-beispiellos}}

\bibleverse{10}»Denn fahrt doch nach den Gestaden der Kitthäer hinüber
und überzeugt euch dort, sendet nach Kedar, erkundigt euch genau und
seht zu, ob so etwas jemals geschehen ist: \bibleverse{11}ob je ein Volk
seine Götter umgetauscht hat -- und die da sind nicht einmal Götter!
Aber mein Volk hat seinen Ruhm\textless sup title=``oder: seine Ehre
=~seinen herrlichen Gott''\textgreater✲ vertauscht gegen ohnmächtige
Götzen! \bibleverse{12}Entsetzt euch darüber, ihr Himmel, schaudert und
werdet starr vor Erstaunen!« -- so lautet der Ausspruch des HERRN.
\bibleverse{13}»Denn zwiefaches Unrecht hat mein Volk begangen: mich,
den Born lebendigen Wassers, haben sie verlassen, um sich gegrabene
Brunnen\textless sup title=``d.h. Zisternen''\textgreater✲ anzulegen,
löcherige Brunnen, die das Wasser nicht halten!«

\hypertarget{cc-die-unheilvollen-folgen-buuxdfmahnung}{%
\subparagraph{cc) Die unheilvollen Folgen;
Bußmahnung}\label{cc-die-unheilvollen-folgen-buuxdfmahnung}}

\bibleverse{14}»Ist denn Israel ein Knecht✲ oder ein
Sklavensohn\textless sup title=``=~ein im Hause geborener
Leibeigener''\textgreater✲? Warum ist er denn der Plünderung
preisgegeben worden? \bibleverse{15}Löwen haben über ihm\textless sup
title=``oder: gegen ihn''\textgreater✲ gebrüllt, haben ihr Geheul
erschallen lassen und sein Land zur Wüste gemacht; seine Städte sind
verbrannt, leer von Bewohnern; \bibleverse{16}auch die Ägypter von Noph✲
und Thapanches✲ haben dir den Scheitel abgeweidet. \bibleverse{17}Trägt
nicht die Schuld daran deine Abkehr vom HERRN, deinem Gott, schon zur
Zeit, wo er dich auf der Wanderung führte? \bibleverse{18}Und jetzt --
was hast du nach Ägypten zu laufen, um das Wasser des Nils zu trinken?
Und was brauchst du nach Assyrien zu laufen, um das Wasser des
Euphratstromes zu trinken? \bibleverse{19}Deine Bosheit\textless sup
title=``=~eigene Schuld''\textgreater✲ bringt dich ins Unglück, und dein
treuloses Treiben führt die Strafe für dich herbei! So erkenne es denn
und bedenke wohl, wie schlimm und unheilvoll es ist, daß du den HERRN,
deinen Gott, verlassen hast und keine Furcht vor mir in dir wohnt!« --
so lautet der Ausspruch Gottes, des HERRN der Heerscharen.

\hypertarget{dd-der-verderbliche-baalsdienst-und-der-zuxfcgellose-hang-zur-abguxf6tterei}{%
\subparagraph{dd) Der verderbliche Baalsdienst und der zügellose Hang
zur
Abgötterei}\label{dd-der-verderbliche-baalsdienst-und-der-zuxfcgellose-hang-zur-abguxf6tterei}}

\bibleverse{20}»Denn von alters her hast du dein Joch zerbrochen, deine
Bande zerrissen und hast gesagt: ›Ich will nicht (länger) dienstbar
sein!‹ Nein, auf jedem hohen Hügel und unter jedem dichtbelaubten Baum
hast du dich als Buhlerin hingestreckt. \bibleverse{21}Und doch hatte
ich dich als Edelrebe eingepflanzt, als ganz echtes Gewächs: ach, wie
bist du mir in die wilden Schosse eines fremden Weinstocks ausgeartet!
\bibleverse{22}Ja, wenn du dich auch mit Laugensalz wüschest und noch so
viel Seife an dich wendetest: deine Schuld würde doch als Schmutzfleck
vor mir bleiben!« -- so lautet der Ausspruch Gottes des HERRN.

\bibleverse{23}»Wie kannst du nur behaupten: ›Ich habe mich nicht
verunreinigt, bin den Baalen nicht nachgelaufen!‹ Sieh doch dein Treiben
im Tal an, bedenke, was du dort verübt hast, du leichtfüßige Kamelin,
die toll in die Kreuz und Quere rennt! \bibleverse{24}Eine Wildeselin,
die, an die Steppe gewöhnt, in wilder Lustgier nach Luft schnappt: wer
vermag ihre Brunst zu dämpfen? Wer immer sie begehrt, braucht sich nicht
müde zu laufen: in ihrer Brunstzeit findet er sie mühelos.
\bibleverse{25}Nimm deinen Fuß in acht, daß er sich nicht
barfuß\textless sup title=``oder: wund''\textgreater✲ läuft, und deine
Kehle, daß sie nicht vor Durst lechzt! Doch du entgegnest: ›Vergebliche
Mühe, nein! Ich habe nun einmal die Fremden\textless sup title=``oder:
Buhlen''\textgreater✲ gern, und ihnen will ich nachlaufen!‹«

\hypertarget{ee-der-unwuxfcrdige-guxf6tzendienst-und-unbegreifliche-abfall-die-schlechte-rechtspflege-und-verkehrte-staatsleitung}{%
\subparagraph{ee) Der unwürdige Götzendienst und unbegreifliche Abfall,
die schlechte Rechtspflege und verkehrte
Staatsleitung}\label{ee-der-unwuxfcrdige-guxf6tzendienst-und-unbegreifliche-abfall-die-schlechte-rechtspflege-und-verkehrte-staatsleitung}}

\bibleverse{26}»Wie ein Dieb beschämt dasteht, wenn er ertappt wird, so
werden\textless sup title=``oder: müssen''\textgreater✲ die zum Hause
Israel Gehörigen sich schämen\textless sup title=``oder: sich enttäuscht
sehen''\textgreater✲, sie samt ihren Königen und Fürsten\textless sup
title=``oder: obersten Beamten''\textgreater✲, ihren Priestern und
Propheten, \bibleverse{27}sie, die zu einem Stück Holz\textless sup
title=``d.h. hölzernen Gottesbilde''\textgreater✲ sagen: ›Mein Vater
bist du!‹ und zu einem Stein: ›Dir verdanke ich mein Leben!‹ Dagegen mir
haben sie den Rücken zugekehrt und nicht mehr das Angesicht. Wenn aber
das Unglück über sie kommt, dann rufen sie: ›Stehe auf und hilf uns!‹
\bibleverse{28}Wo sind denn deine Götter, die du dir selbst angefertigt
hast? Sie mögen doch aufstehen, ob sie dir helfen können zur Zeit deiner
Not! Denn so zahlreich wie deine Städte sind auch deine Götter geworden,
Juda. \bibleverse{29}Warum beklagt ihr euch über mich? Ihr seid ja
allesamt treulos von mir abgefallen!« -- so lautet der Ausspruch des
HERRN. \bibleverse{30}»Vergebens habe ich eure Söhne\textless sup
title=``oder: Kinder''\textgreater✲ geschlagen: sie haben sich keine
Lehre daraus gezogen\textless sup title=``=~nicht warnen
lassen''\textgreater✲; das Schwert hat eure Propheten gefressen wie ein
reißender Löwe. \bibleverse{31}O (entartetes) Geschlecht, das ihr seid!
Achtet doch auf das Wort des HERRN! Bin ich etwa eine Wüste für Israel
gewesen oder ein Land tiefer Finsternis? Warum sagt denn mein Volk: ›Wir
haben die Freiheit gewonnen! Wir kehren nicht wieder zu dir zurück!‹?
\bibleverse{32}Vergißt wohl eine Jungfrau ihren Schmuck, eine Braut
ihren Gürtel? Mein Volk aber hat mich vergessen schon seit unzähligen
Tagen!«

\bibleverse{33}»Wie geschickt weißt du deinen Gang einzurichten, um
Liebschaften anzuknüpfen! Darum hast du dich auch auf deinen Gängen
sogar an Verbrechen gewöhnt: \bibleverse{34}auch an den Säumen deines
Gewandes findet sich das Blut von schuldlosen unglücklichen Menschen,
die du nicht bei einem Einbruch betroffen hast. \bibleverse{35}Und
trotzdem behauptest du: ›Ich bin unschuldig: sein Zorn wendet sich von
mir ab!‹ Wisse wohl: ich will mit dir ins Gericht gehen wegen dieses
deines Wortes: ›Ich habe nichts Böses getan!‹«

\bibleverse{36}»Warum hast du es so eilig, in der Staatsleitung✲ einen
andern Weg einzuschlagen? Auch an Ägypten wirst du ebenso enttäuscht
werden, wie du an Assyrien enttäuscht worden bist; \bibleverse{37}auch
von dort wirst du abziehen, indem du die Hände über dem Kopfe
zusammenschlägst; denn der HERR hat die verworfen, auf die du dein
Vertrauen gesetzt hast, und so wirst du kein Glück mit ihnen haben.«

\hypertarget{ff-ist-eine-wiederannahme-des-durch-guxf6tzendienst-geschuxe4ndeten-volkes-uxfcberhaupt-noch-muxf6glich}{%
\subparagraph{ff) Ist eine Wiederannahme des durch Götzendienst
geschändeten Volkes überhaupt noch
möglich?}\label{ff-ist-eine-wiederannahme-des-durch-guxf6tzendienst-geschuxe4ndeten-volkes-uxfcberhaupt-noch-muxf6glich}}

\hypertarget{section-2}{%
\section{3}\label{section-2}}

\bibleverse{1}Er fuhr dann fort: »Wenn ein Mann seine Ehefrau entläßt
und diese von ihm weggegangen und die Frau eines andern Mannes geworden
ist, darf sie dann wieder zu ihm zurückkehren?\textless sup title=``vgl.
5.Mose 24,1-4''\textgreater✲ Würde nicht das betreffende Land dadurch
ganz entweiht werden? Du aber hast schon mit vielen Liebhabern Ehebruch
getrieben und solltest doch zu mir zurückkehren dürfen?« -- so lautet
der Ausspruch des HERRN. \bibleverse{2}»Erhebe doch deine Augen zu den
kahlen Höhen und halte Umschau: wo hast du dich nicht schänden lassen?
An den Wegen hast du gesessen und ihnen aufgelauert wie ein Araber in
der Wüste und hast das Land entweiht durch deine Buhlerei und deine
Verworfenheit! \bibleverse{3}Und ob dir auch die Regenschauer (von mir)
vorenthalten wurden und der Spätregen✲ ausblieb, behieltest du doch die
Stirn eines buhlerischen Weibes bei und wolltest nicht in dich gehen.
\bibleverse{4}Freilich, nunmehr rufst du mir zu: ›Mein Vater! Du bist ja
der Vertraute meiner Jugend! \bibleverse{5}Wird er denn immerdar
grollen, mir's ewig nachtragen?‹ Ja, so hast du geredet, dabei aber das
Böse verübt und es durchgesetzt.«

\hypertarget{b-israel-und-juda-als-abtruxfcnnige-schwestern-mahnung-zur-umkehr-und-heimkehr}{%
\paragraph{b) Israel und Juda als abtrünnige Schwestern; Mahnung zur
Umkehr und
Heimkehr}\label{b-israel-und-juda-als-abtruxfcnnige-schwestern-mahnung-zur-umkehr-und-heimkehr}}

\hypertarget{aa-beide-schwestern-haben-schwer-gesuxfcndigt-aber-judas-schuld-ist-gruxf6uxdfer}{%
\subparagraph{aa) Beide Schwestern haben schwer gesündigt, aber Judas
Schuld ist
größer}\label{aa-beide-schwestern-haben-schwer-gesuxfcndigt-aber-judas-schuld-ist-gruxf6uxdfer}}

\bibleverse{6}Der HERR sprach weiter zu mir in den Tagen des Königs
Josia folgendermaßen: »Hast du gesehen, wie Israel, das abtrünnige Weib,
es getrieben hat? Sie ist auf jeden hohen Berg und unter jeden
dichtbelaubten Baum gegangen und hat dort Ehebruch getrieben.
\bibleverse{7}Zwar dachte ich: ›Sie wird zu mir zurückkehren, nachdem
sie dies alles verübt\textless sup title=``oder: so getrieben
hat''\textgreater✲‹; aber sie kehrte nicht zurück. Ihre treulose
Schwester Juda sah das nun wohl; \bibleverse{8}doch obgleich sie gesehen
hatte, daß ich das abtrünnige Weib Israel wegen ihres ehebrecherischen
Treibens verstoßen und ihr den Scheidebrief gegeben hatte, nahm ihre
treulose Schwester Juda es sich doch nicht zu Herzen, sondern ging hin
und trieb ebenfalls Unzucht. \bibleverse{9}So kam es denn, daß sie durch
ihre leichtfertige Unzucht das Land entweihte; denn sie trieb Ehebruch
mit dem Stein und mit dem Holz\textless sup title=``vgl.
2,27''\textgreater✲. \bibleverse{10}Trotz alledem ist aber ihre treulose
Schwester Juda nicht mit ihrem ganzen Herzen zu mir zurückgekehrt,
sondern nur mit Heuchelei\textless sup title=``=~zum
Schein''\textgreater✲!« -- so lautet der Ausspruch des HERRN.

\bibleverse{11}Hierauf sagte der HERR weiter zu mir: »Israel, das
abtrünnige Weib, steht weniger schuldig da als die treulose Juda.
\bibleverse{12}Gehe hin und rufe diese Worte laut nach Norden hin:
›Kehre zurück (zu mir), Israel, du Abtrünnige!‹ -- so lautet der
Ausspruch des HERRN --; ›ich will euch nicht mehr zornig anblicken, denn
ich bin liebevoll‹ -- so lautet der Ausspruch des HERRN --; ›ich will
(es dir) nicht ewig nachtragen! \bibleverse{13}Nur erkenne deine
Verschuldung, daß du dem HERRN, deinem Gott, die Treue gebrochen und
dich immer wieder den Fremden preisgegeben hast unter jedem
dichtbelaubten Baum; aber auf meinen Ruf habt ihr\textless sup
title=``oder: hast du''\textgreater✲ nicht gehört!‹« -- so lautet der
Ausspruch des HERRN.

\hypertarget{bb-aufforderung-zur-umkehr-heilsspruch-uxfcber-jerusalem-und-juda}{%
\subparagraph{bb) Aufforderung zur Umkehr; Heilsspruch über Jerusalem
und
Juda}\label{bb-aufforderung-zur-umkehr-heilsspruch-uxfcber-jerusalem-und-juda}}

\bibleverse{14}»Kehrt um, ihr abtrünnigen Söhne\textless sup
title=``oder: Kinder''\textgreater✲!« -- so lautet der Ausspruch des
HERRN --; »denn ich habe Herrenrecht über euch und will euch holen, je
einen aus jeder Ortschaft und je zwei aus jedem Geschlecht, und will
euch nach Zion heimkehren lassen; \bibleverse{15}und ich will euch
Hirten nach meinem Herzen geben, die euch mit Einsicht und Besonnenheit
weiden sollen. \bibleverse{16}Wenn ihr euch dann im Lande vermehrt habt
und zahlreich geworden seid in jenen Tagen« -- so lautet der Ausspruch
des HERRN --, »so wird man nicht mehr sagen: ›O die Lade mit dem
Bundesgesetz des HERRN!‹, denn sie wird keinem mehr in den Sinn kommen,
und man wird ihrer nicht mehr gedenken und sie nicht mehr vermissen;
auch wird niemals wieder eine solche angefertigt werden.
\bibleverse{17}In jener Zeit wird man Jerusalem den Thron des HERRN
nennen, und es werden dort alle Heidenvölker zusammenströmen um des
Namens des HERRN willen {[}in Jerusalem{]} und in ihrem Wandel nicht
länger dem Starrsinn ihres eigenen bösen Herzens folgen.
\bibleverse{18}In jenen Tagen wird das Haus Juda mit dem Hause Israel
Hand in Hand gehen, und sie werden vereint aus dem Nordlande in das Land
heimkehren, das ich euren\textless sup title=``oder:
ihren''\textgreater✲ Vätern zum Erbbesitz gegeben habe.«

\hypertarget{cc-wehmuxfctiger-ruxfcckblick-auf-die-untreue-des-volkes}{%
\subparagraph{cc) Wehmütiger Rückblick auf die Untreue des
Volkes}\label{cc-wehmuxfctiger-ruxfcckblick-auf-die-untreue-des-volkes}}

\bibleverse{19}»Zwar hatte ich gedacht: ›Wie will ich dich an Sohnes
Statt halten und dir ein herrliches Land, den kostbarsten Besitz der
ganzen Völkerwelt, verleihen!‹ Und weiter hatte ich gedacht, ihr würdet
mich ›Vater‹ nennen und euch von meiner Nachfolge nicht mehr abkehren.
\bibleverse{20}Aber ach! Wie ein Weib ihrem Genossen die Treue bricht,
so habt auch ihr treulos an mir gehandelt, ihr vom Hause Israel!« -- so
lautet der Ausspruch des HERRN.

\hypertarget{dd-buuxdflied-des-volkes}{%
\subparagraph{dd) Bußlied des Volkes}\label{dd-buuxdflied-des-volkes}}

\bibleverse{21}Horch! Auf den kahlen Höhen vernimmt man Weinen, das
inständige Flehen der Kinder Israel, weil sie auf verkehrtem Wege
gewandelt sind und den HERRN, ihren Gott, vergessen haben.
\bibleverse{22}»Kehrt um, ihr abtrünnigen Söhne\textless sup
title=``oder: Kinder''\textgreater✲: ich will euren Abfall
wiedergutmachen!« -- »Ja hier sind wir, wir kommen zu dir; denn du,
HERR, bist unser Gott. \bibleverse{23}Wahrlich, nur Trug sind die Hügel,
das Lärmen\textless sup title=``=~die lärmenden Feste''\textgreater✲ auf
den Höhen! Wahrlich, nur beim HERRN, unserm Gott, steht das Heil für
Israel! \bibleverse{24}Ach, der Schandgötze\textless sup title=``vgl.
11,13''\textgreater✲ hat seit unserer Jugend das verschlungen, was
unsere Väter erworben hatten, ihr Kleinvieh und ihre Rinder, ihre Söhne
und ihre Töchter! \bibleverse{25}So müssen wir uns denn in unsere
Schande betten, und unsere Schmach muß uns zudecken! Denn wir haben
gegen den HERRN, unsern Gott, gesündigt, wir selbst und unsere Väter,
von unserer Jugend an bis auf diesen Tag und haben auf die Stimme des
HERRN, unsers Gottes, nicht gehört!«

\hypertarget{ee-verheiuxdfung-der-wiederannahme-nach-aufrichtiger-buuxdfe}{%
\subparagraph{ee) Verheißung der Wiederannahme nach aufrichtiger
Buße}\label{ee-verheiuxdfung-der-wiederannahme-nach-aufrichtiger-buuxdfe}}

\hypertarget{section-3}{%
\section{4}\label{section-3}}

\bibleverse{1}»Wenn du umkehrst, Israel« -- so lautet der Ausspruch des
HERRN --, »sollst du zu mir zurückkehren dürfen; und wenn du deine
greulichen Götzen mir aus den Augen schaffst, sollst du nicht verstoßen
werden; \bibleverse{2}und wenn du ›So wahr der HERR lebt!‹ in Wahrheit,
in Treue und Aufrichtigkeit schwörst, sollen die Heidenvölker mit deinem
Namen sich segnen und dein Glück sich wünschen!« \bibleverse{3}Denn so
hat der HERR zu den Männern von Juda und zu den Bewohnern Jerusalems
gesprochen: »Brecht euch einen Neubruch\textless sup title=``=~brecht
das Brachland eurer Herzen um''\textgreater✲ und säet nicht in die
Dornen hinein! \bibleverse{4}Beschneidet euch für den HERRN und schafft
die Vorhaut eurer Herzen weg, ihr Männer von Juda und ihr Bewohner
Jerusalems, damit mein Zorn nicht ausbricht wie Feuer und unauslöschlich
brennt wegen der Verworfenheit eures ganzen Tuns!«

\hypertarget{c-judas-unverbesserliche-verderbnis-und-das-nahende-gottesgericht}{%
\paragraph{c) Judas unverbesserliche Verderbnis und das nahende
Gottesgericht}\label{c-judas-unverbesserliche-verderbnis-und-das-nahende-gottesgericht}}

\hypertarget{aa-krieg-das-erscheinen-des-furchtbaren-nordischen-feindes-sammlung-verschiedenartiger-stuxfccke-kriegsbilder-betrachtungen-und-mahnungen}{%
\subparagraph{aa) Krieg! Das Erscheinen des furchtbaren nordischen
Feindes (Sammlung verschiedenartiger Stücke: Kriegsbilder, Betrachtungen
und
Mahnungen)}\label{aa-krieg-das-erscheinen-des-furchtbaren-nordischen-feindes-sammlung-verschiedenartiger-stuxfccke-kriegsbilder-betrachtungen-und-mahnungen}}

\bibleverse{5}Verkündet es in Juda und laßt in Jerusalem die Botschaft
ausrufen: »Stoßt in die Posaune im Lande umher! Ruft mit voller Kraft
aus und macht bekannt: ›Schart euch zusammen und laßt uns in die festen
Städte ziehen!‹ \bibleverse{6}Pflanzt ein Banner\textless sup
title=``oder: Panier''\textgreater✲ auf nach Zion hin, flüchtet euch,
ohne Halt zu machen! Denn Unheil lasse ich von Norden her kommen und
gewaltige Zerschmetterung.« \bibleverse{7}Heraufgestiegen ist schon der
Löwe aus seinem Dickicht, und der Völkerwürger ist aufgebrochen, ist
ausgezogen aus seiner Wohnstätte, um dein Land zur Wüste zu machen, daß
deine Städte in Trümmer sinken, menschenleer werden! \bibleverse{8}Darum
gürtet euch Sackleinen✲ um, wehklagt und jammert: »Ach, der lodernde
Zorn des HERRN hat sich nicht von uns abgewandt!«

\bibleverse{9}»An jenem Tage« -- so lautet der Ausspruch des HERRN --
»wird dem Könige und den Fürsten\textless sup title=``oder: obersten
Beamten''\textgreater✲ der Mut entschwinden, werden die Priester ratlos
sein und die Propheten fassungslos dastehen.« \bibleverse{10}Da sagte
ich: »Ach, HERR, mein Gott! Wahrlich, arg hast du dieses Volk und
Jerusalem getäuscht, als du verhießest: ›Heil soll euch widerfahren!‹,
und nun ist das Schwert ihnen gedrungen bis ans Leben!«~--
\bibleverse{11}In jener Zeit wird man zu\textless sup title=``oder:
von''\textgreater✲ diesem Volk und zu\textless sup title=``oder:
von''\textgreater✲ Jerusalem sagen: »Ein Glutwind von den kahlen Höhen
in der Wüste kommt dahergestürmt auf die Tochter meines Volkes zu,
ungeeignet zum Worfeln und ungeeignet zum Sieben des Korns,
\bibleverse{12}ein Wind zu scharf, als daß er dazu\textless sup
title=``=~für beides''\textgreater✲ taugte, kommt, von mir bestellt: nun
will ich selbst mit ihnen ins Gericht gehen!«~--

\bibleverse{13}Sehet, wie Wetterwolken zieht er\textless sup
title=``d.h. der Löwe =~der Feind''\textgreater✲ herauf, und wie der
Sturmwind sind seine Kriegswagen, schneller als Adler seine Rosse: wehe
uns, wir sind verloren!~--

\bibleverse{14}Wasche dein Herz vom Bösen rein, Jerusalem, auf daß du
Rettung erlangst! Wie lange soll noch dein unheilvolles Sinnen in deinem
Inneren wohnen?~--

\bibleverse{15}Ach, horch! Eine Botschaft kommt von Dan her, eine
Unglückskunde vom Gebirge Ephraim! \bibleverse{16}»Verkündet es den
Völkern, meldet es weiter nach Jerusalem: ›Belagerer kommen aus fernem
Lande und erheben ihren Kriegsruf gegen die Städte Judas!
\bibleverse{17}Wie Feldhüter stellen sie sich ringsum gegen Jerusalem
auf, weil es widerspenstig gegen mich gewesen ist!‹« -- so lautet der
Ausspruch des HERRN.~--

\bibleverse{18}Dein Wandel und dein ganzes Tun haben dir solches Unheil
eingetragen: deiner Verworfenheit verdankst du dies, daß es so bitterweh
ist, ja daß es dir bis ans Leben geht.~--

\bibleverse{19}O meine Brust, meine Brust! Mir ist so angst! O ihr Wände
meines Herzens! Mein Inneres ist mir in wildem Aufruhr: ich kann nicht
schweigen\textless sup title=``oder: mich nicht
beruhigen''\textgreater✲! Denn Posaunenschall vernimmst du, meine Seele,
das Getöse des Krieges. \bibleverse{20}Zerstörung über
Zerstörung\textless sup title=``=~Schlag auf Schlag''\textgreater✲ wird
gemeldet, denn verheert ist das ganze Land! Urplötzlich sind meine
Hütten zerstört, in einem Augenblick meine Zelte! \bibleverse{21}Wie
lange noch soll ich Paniere sehen, Posaunenschall vernehmen?~--

\bibleverse{22}Ach, verblendet ist mein Volk: mich kennen sie nicht!
Törichte Kinder sind sie und ohne Einsicht: klug sind sie zum Bösestun,
aber auf das Tun des Guten verstehen sie sich nicht.~--

\bibleverse{23}Ich blicke die Erde an: ach, sie ist wüst und öde! und
zum Himmel empor: sein Licht ist verschwunden! \bibleverse{24}Ich blicke
die Berge an: ach, sie beben, und alle Hügel schwanken!
\bibleverse{25}Ich blicke umher: ach, kein Mensch ist da, und alle Vögel
des Himmels sind entflohen! \bibleverse{26}Ich blicke umher: ach, das
Fruchtgefilde (Juda) ist eine Wüste, und alle seine Städte sind
zerstört: nach dem Willen des HERRN, infolge der Glut seines Zorns!~--

\bibleverse{27}Denn so hat der HERR gesprochen: »Zur Wüste soll das
ganze Land werden, doch seine völlige Vernichtung will ich nicht
herbeiführen. \bibleverse{28}Darob trauert die Erde\textless sup
title=``oder: das Land''\textgreater✲, und der Himmel droben hüllt sich
in Schwarz, weil ich (das) ausgesprochen und beschlossen habe; und ich
lasse es mich nicht gereuen und gehe nicht davon ab.«~--

\bibleverse{29}Vor dem Ruf »Reiter und Bogenschützen!« ist das ganze
Stadtvolk auf der Flucht; in die Dickichte haben sie sich begeben und
die Felsen erklommen: jede Ortschaft steht verlassen da, und kein Mensch
ist darin wohnen geblieben. \bibleverse{30}Du aber, vergewaltigte
(Stadt), was willst du tun? Magst du dich auch in Purpur kleiden, magst
du dich mit Goldgeschmeide putzen, magst du deine Augenränder mit
Schminke umzeichnen: vergebens machst du dich schön! Die Liebhaber haben
dich satt, trachten dir nach dem Leben! \bibleverse{31}Ach! Geschrei
höre ich wie von einem Weibe in Kindesnöten, Angstrufe wie von einer
Frau, die zum erstenmal Mutter wird: das Geschrei der Tochter Zion, die
da aufstöhnt, ihre Hände flehend ausbreitet: »Wehe mir! Ach, erschöpft
ist meine Seele\textless sup title=``oder: mein Leben''\textgreater✲,
eine Beute der Mörder!«

\hypertarget{bb-die-grauenvolle-verderbnis-des-gesamten-volkes-in-jerusalem-nuxf6tigt-den-herrn-zum-schonungslosen-vollzug-der-strafe-durch-den-furchtbaren-feind}{%
\subparagraph{bb) Die grauenvolle Verderbnis des gesamten Volkes in
Jerusalem nötigt den Herrn zum schonungslosen Vollzug der Strafe durch
den furchtbaren
Feind}\label{bb-die-grauenvolle-verderbnis-des-gesamten-volkes-in-jerusalem-nuxf6tigt-den-herrn-zum-schonungslosen-vollzug-der-strafe-durch-den-furchtbaren-feind}}

\hypertarget{section-4}{%
\section{5}\label{section-4}}

\bibleverse{1}Streift in den Straßen Jerusalems umher und seht euch um!
Erkundigt euch und sucht auf den Plätzen der Stadt, ob ihr jemand
findet, ob einer da ist, der Recht übt, der auf Treue\textless sup
title=``oder: Redlichkeit''\textgreater✲ hält: dann will ich ihr
vergeben. \bibleverse{2}Aber wenn sie auch sagen: »So wahr der HERR
lebt!«, so schwören sie darum doch falsch. \bibleverse{3}Sind denn deine
Augen, HERR, nicht auf Treue\textless sup title=``oder:
Ehrlichkeit''\textgreater✲ gerichtet? Du hast sie zwar geschlagen, aber
es hat ihnen nicht wehe getan; du hast sie der Vernichtung preisgegeben,
aber sie haben keine Zucht annehmen wollen: sie haben ihr Angesicht
härter gemacht als Felsgestein und eine Umkehr von sich gewiesen.
\bibleverse{4}Da dachte ich: »Nur die kleinen Leute sind so; die
benehmen sich töricht, weil sie den Weg des HERRN, das Recht ihres
Gottes\textless sup title=``oder: die Gebühr gegen ihren
Gott''\textgreater✲ nicht kennen. \bibleverse{5}Ich will doch einmal zu
den Großen gehen und mit ihnen reden; denn die müssen doch den Weg des
HERRN, das Recht ihres Gottes\textless sup title=``oder: die Gebühr
gegen ihren Gott''\textgreater✲ kennen.« Doch sie haben insgesamt das
Joch zerbrochen, die Bande zerrissen! \bibleverse{6}Darum schlägt sie
der Löwe aus dem Walde nieder, überwältigt sie der Steppenwolf; der
Panther lauert ihnen auf vor ihren Städten: jeder, der aus ihnen
hinausgeht, wird zerrissen; denn zahlreich sind ihre Übertretungen,
vielfältig ihre Abfallsünden.

\hypertarget{cc-anfang-der-drohrede-gottes}{%
\subparagraph{cc) Anfang der Drohrede
Gottes}\label{cc-anfang-der-drohrede-gottes}}

\bibleverse{7}»Weshalb sollte ich dir verzeihen? Deine Söhne haben mich
verlassen und schwören bei Nichtgöttern; und obwohl ich sie den Bund
hatte beschwören lassen, haben sie doch Ehebruch begangen und sind im
Hurenhause heimisch geworden. \bibleverse{8}Wie
wohlgenährte\textless sup title=``oder: brünstige''\textgreater✲ Rosse
schweifen sie umher: ein jeder wiehert nach dem Eheweibe des andern.
\bibleverse{9}Sollte ich so etwas ungestraft lassen?« -- so lautet der
Ausspruch des HERRN --, »oder sollte an einem solchen Volk meine
Seele\textless sup title=``oder: mein Zorn''\textgreater✲ sich nicht
rächen?«

\hypertarget{dd-das-guxf6ttliche-strafgericht-fuxfcr-die-abtruxfcnnigen-und-ungluxe4ubig-trotzigen}{%
\subparagraph{dd) Das göttliche Strafgericht für die Abtrünnigen und
ungläubig
Trotzigen}\label{dd-das-guxf6ttliche-strafgericht-fuxfcr-die-abtruxfcnnigen-und-ungluxe4ubig-trotzigen}}

\bibleverse{10}Steigt auf ihre Mauern hinauf und richtet Verwüstungen
an, doch vernichtet sie nicht völlig! Haut ihre Ranken ab; denn dem
HERRN gehören sie nicht (mehr) an. \bibleverse{11}»Ach, sie haben gar
treulos an mir gehandelt, das Haus Israel und das Haus Juda!« -- so
lautet der Ausspruch des HERRN. \bibleverse{12}Sie haben den HERRN
verleugnet und gesagt: »Es ist nichts mit ihm, und kein Unglück wird
über uns kommen: weder Schwert noch Hungersnot werden wir zu sehen
bekommen! \bibleverse{13}Und die Propheten? Die sind für den Wind; denn
das Wort (des HERRN) ist nicht in ihnen: möge es ihnen selbst so
ergehen!« \bibleverse{14}Darum hat Gott, der HERR der Heerscharen, so
gesprochen: »Weil ihr solche Reden führt, will ich nunmehr meine Worte
in deinem Munde zu Feuer machen und dieses Volk zu Brennholz, daß es sie
verzehren soll!«

\hypertarget{ee-der-furchtbare-feind}{%
\subparagraph{ee) Der furchtbare Feind}\label{ee-der-furchtbare-feind}}

\bibleverse{15}»Wisset wohl: ich lasse ein Volk aus der Ferne über euch
kommen, ihr vom Hause Israel!« -- so lautet der Ausspruch des HERRN --;
»ein Volk von unverwüstlicher Kraft ist es, ein Volk von uraltem Stamm,
ein Volk, dessen Sprache du nicht kennst und dessen Rede du nicht
verstehst. \bibleverse{16}Sein Köcher ist wie ein offenes Grab: allesamt
sind sie Kriegshelden. \bibleverse{17}Es wird deine Ernte und dein
Brotkorn verzehren, verzehren deine Söhne und Töchter, verzehren dein
Kleinvieh und deine Rinder, verzehren deinen Weinstock und deinen
Feigenbaum; deine festen Städte, auf die du dein Vertrauen setzt, wird
es mit dem Schwert zerstören.«

\hypertarget{ff-die-ursache-der-guxf6ttlichen-strafe-nuxe4mlich-der-verbannung}{%
\subparagraph{ff) Die Ursache der göttlichen Strafe (nämlich der
Verbannung)}\label{ff-die-ursache-der-guxf6ttlichen-strafe-nuxe4mlich-der-verbannung}}

\bibleverse{18}»Doch auch in jenen Tagen« -- so lautet der Ausspruch des
HERRN -- »will ich euch nicht völlig vernichten. \bibleverse{19}Aber
wenn ihr alsdann fragt: ›Wofür hat der HERR, unser Gott, uns dies alles
widerfahren lassen?‹, so sollst du ihnen antworten: ›Gleichwie ihr mich
verlassen und fremden Göttern im eigenen Lande gedient habt, ebenso
sollt ihr nun Fremden dienstbar sein in einem Lande, das nicht euch
gehört\textless sup title=``=~in fremdem Lande''\textgreater✲!‹«

\hypertarget{gg-der-unverstand-des-volkes-die-habgier-der-oberen-schichten-und-die-unredlichkeit-der-geistlichkeit}{%
\subparagraph{gg) Der Unverstand des Volkes, die Habgier der oberen
Schichten und die Unredlichkeit der
Geistlichkeit}\label{gg-der-unverstand-des-volkes-die-habgier-der-oberen-schichten-und-die-unredlichkeit-der-geistlichkeit}}

\bibleverse{20}Verkündet dies im Hause Jakob und macht es in Juda
bekannt mit den Worten: \bibleverse{21}»Hört doch dies, ihr törichtes
Volk voll Unverstand, die ihr Augen habt und nicht seht, die ihr Ohren
habt und nicht hört! \bibleverse{22}Mich wollt ihr nicht fürchten« -- so
lautet der Ausspruch des HERRN -- »und vor mir nicht zittern? der ich
dem Meere den Sand zur Grenze gesetzt habe als ewige Schranke, die es
nicht überschreiten darf, so daß seine Wogen, wenn sie auch branden,
doch ohnmächtig sind und, wenn sie auch brausen, doch nicht ungebührlich
vordringen. \bibleverse{23}Aber dieses Volk besitzt ein trotziges und
widerspenstiges Herz; sie sind abgefallen und davongegangen
\bibleverse{24}und haben niemals in ihrem Herzen gedacht: ›Laßt uns doch
den HERRN, unsern Gott, fürchten, der den Regen spendet, Frühregen wie
Spätregen\textless sup title=``=~Herbstregen wie
Frühlingsregen''\textgreater✲ zu rechter Zeit, der die festbestimmten
Wochen der Erntezeit uns zugute einhält!‹ \bibleverse{25}Eure
Verschuldungen haben das unmöglich gemacht und eure Sünden euch um den
Segen gebracht.

\bibleverse{26}Denn unter meinem Volke gibt es Gottlose, die auf der
Lauer liegen, wie Vogelfänger sich ducken: sie stellen Fallen auf und
treiben Menschenfang. \bibleverse{27}Wie ein Käfig sich mit Vögeln
füllt, so füllen sich ihre Häuser mit ungerechtem Gut; auf solche Weise
sind sie hoch gekommen und reich geworden; \bibleverse{28}fett sind sie
geworden und feist, ja, ihre Verworfenheit überschreitet jedes Maß. An
das Recht halten sie sich nicht; für die Sache der Waisen treten sie
nicht ein, um sie zum Siege zu führen, und der Rechtssache der Armen
nehmen sie sich nicht an. \bibleverse{29}Sollte ich so etwas ungestraft
lassen?« -- so lautet der Ausspruch des HERRN --, »oder sollte meine
Seele\textless sup title=``oder: mein Zorn''\textgreater✲ sich an einem
solchen Volk nicht rächen?«\textless sup title=``vgl. V.9''\textgreater✲

\bibleverse{30}Entsetzliche und greuliche Dinge haben sich im Lande
zugetragen: \bibleverse{31}die Propheten prophezeien als Lügendiener,
und die Priester schalten mit ihnen Hand in Hand, und mein Volk hat es
gern so! Was werdet ihr aber tun, wenn es damit zu Ende geht?

\hypertarget{hh-erneute-ankuxfcndigung-des-bevorstehenden-krieges-neue-schilderung-der-gruxf6uxdfe-des-inneren-schadens}{%
\subparagraph{hh) Erneute Ankündigung des bevorstehenden Krieges; neue
Schilderung der Größe des inneren
Schadens}\label{hh-erneute-ankuxfcndigung-des-bevorstehenden-krieges-neue-schilderung-der-gruxf6uxdfe-des-inneren-schadens}}

\hypertarget{section-5}{%
\section{6}\label{section-5}}

\bibleverse{1}Flüchtet, ihr Benjaminiten, aus Jerusalem hinweg, stoßt zu
Thekoa in die Trompete und richtet ein Sturmzeichen über Beth-Kerem auf!
Denn Unheil droht von Norden her, und zwar eine gewaltige
Zerschmetterung: \bibleverse{2}die Liebliche und Verwöhnte, die Tochter
Zion, will ich vertilgen. \bibleverse{3}Auf sie zu kommen Hirten gezogen
mitsamt ihren Herden, schlagen Zelte ringsum gegen sie auf und weiden
ein jeder sein Teil\textless sup title=``=~seinen Bereich''\textgreater✲
ab. \bibleverse{4}»Weihet euch zum Kampf gegen sie! Auf, wir wollen noch
am Mittag hinaufziehen! Wehe uns, der Tag neigt sich schon, denn die
Abendschatten dehnen sich schon lang! \bibleverse{5}Auf, laßt uns bei
Nacht hinaufziehen und ihre Paläste zerstören! \bibleverse{6}Denn so hat
der HERR der Heerscharen gesprochen: ›Fället die Bäume dort und führt
einen Wall gegen Jerusalem auf!‹« -- »Das ist die Stadt, die gestraft
werden soll: überall herrscht Gewalttat\textless sup title=``oder:
Unrecht''\textgreater✲ in ihrem Innern! \bibleverse{7}Wie ein Brunnen
sein Wasser sprudeln läßt, so läßt sie ihre Bosheit sprudeln: von Frevel
und Unrecht hört man in ihr; Wunden und Striemen treten mir allezeit (in
ihr) vor Augen. \bibleverse{8}Laß dich warnen, Jerusalem, damit mein
Herz sich nicht von dir losreißt, damit ich dich nicht zur Wüste mache,
zu einem unbewohnten Lande!«

\hypertarget{ii-heiliger-zorn-des-propheten-uxfcber-die-verderbtheit-des-gesamten-volkes-besonders-der-geistigen-leiter-unheilsdrohung}{%
\subparagraph{ii) Heiliger Zorn des Propheten über die Verderbtheit des
gesamten Volkes (besonders der geistigen Leiter);
Unheilsdrohung}\label{ii-heiliger-zorn-des-propheten-uxfcber-die-verderbtheit-des-gesamten-volkes-besonders-der-geistigen-leiter-unheilsdrohung}}

\bibleverse{9}So hat der HERR der Heerscharen gesprochen: »Gründliche
Nachlese wie am Weinstock wird man am Überrest Israels halten: lege
deine Hand immer wieder an, wie der Weingärtner an die Ranken!«
\bibleverse{10}Ja, an wen soll ich meine Worte noch richten und wem ins
Gewissen reden, daß sie darauf hören? Ach, ihr Ohr ist unbeschnitten✲,
so daß sie nichts vernehmen können! Ach, das Wort des HERRN ist ihnen
zum Spott geworden, so daß sie kein Gefallen daran haben!
\bibleverse{11}Ja, ich bin mit der Zornglut des HERRN erfüllt, nur mit
Mühe vermag ich sie zurückzuhalten! »So laß sie sich denn über die
Kinder auf der Straße ergießen und über die Schar der Jünglinge
allzumal! Denn Männer wie Weiber sollen von ihr getroffen werden, Alte
mitsamt den Vollbetagten; \bibleverse{12}und ihre Häuser sollen an
andere übergehen, die Äcker und die Weiber allesamt, wenn ich meine Hand
ausstrecke gegen die Bewohner des Landes!« -- so lautet der Ausspruch
des HERRN. \bibleverse{13}Denn vom Jüngsten bis zum Ältesten sind sie
alle gierig nach Gewinn, und vom Propheten bis zum Priester gehen sie
alle mit Falschheit um; \bibleverse{14}die schwere Wunde meines Volkes
wollen sie leichtfertig obenhin heilen, indem sie verheißen: »Heil,
Heil!«, wo doch kein Heil vorhanden ist. \bibleverse{15}Beschämt werden
sie dastehen müssen, weil sie Greuel verübt haben; und doch schämen sie
sich keineswegs, und Erröten kennen sie nicht. »Darum werden sie fallen,
wenn alles fällt: zur Zeit, wo ich sie zur Rechenschaft ziehe, werden
sie stürzen!« -- so hat der HERR gesprochen.

\hypertarget{jj-alle-muxfche-gottes-ist-umsonst-gewesen-das-gericht-ist-unvermeidlich}{%
\subparagraph{jj) Alle Mühe Gottes ist umsonst gewesen; das Gericht ist
unvermeidlich}\label{jj-alle-muxfche-gottes-ist-umsonst-gewesen-das-gericht-ist-unvermeidlich}}

\bibleverse{16}So hat der HERR gesprochen: »Tretet hin an die Wege und
haltet Umschau und forscht nach den Pfaden der Vorzeit, welches der Weg
des Heils\textless sup title=``oder: zum Glück''\textgreater✲ sei, und
dann wandelt auf ihm, so werdet ihr Ruhe finden für eure Seelen!« Aber
sie haben geantwortet: »Nein, wir wollen auf ihm nicht wandeln!«
\bibleverse{17}»Dann\textless sup title=``oder: trotzdem''\textgreater✲
habe ich Wächter über euch bestellt (und euch gemahnt): Merkt auf den
Schall der Posaune! Aber sie haben geantwortet: ›Nein, wir wollen auf
ihn nicht merken!‹ \bibleverse{18}Darum höret, ihr Völker, und gib wohl
acht, du Gemeinde, was mit ihnen geschehen wird! \bibleverse{19}Vernimm
du es, Erde! Jetzt will ich Unheil über dies Volk bringen, den Lohn
ihrer (bösen) Anschläge! Denn auf meine Worte haben sie nicht geachtet,
und meine Weisung -- die haben sie verworfen. \bibleverse{20}Wozu soll
mir da der Weihrauch aus Saba dienen und das kostbare Würzrohr aus
fernem Lande? Eure Brandopfer sind mir nicht wohlgefällig und eure
Schlachtopfer mir nicht angenehm!« \bibleverse{21}Darum hat der HERR so
gesprochen: »Nunmehr lege ich diesem Volk Steine des Anstoßes in den
Weg, daß Väter und Söhne zugleich darüber straucheln, daß ein Nachbar
mit dem andern zugrunde geht!«

\hypertarget{kk-der-furchtbare-feind-aus-dem-norden-vor-welchem-juda-in-trauer-und-verderben-dahinsinkt}{%
\subparagraph{kk) Der furchtbare Feind aus dem Norden, vor welchem Juda
in Trauer und Verderben
dahinsinkt}\label{kk-der-furchtbare-feind-aus-dem-norden-vor-welchem-juda-in-trauer-und-verderben-dahinsinkt}}

\bibleverse{22}So hat der HERR gesprochen: »Gebt acht! Es kommt ein Volk
vom Nordlande her, und ein gewaltiges Heer setzt sich in Bewegung vom
äußersten Ende der Erde her. \bibleverse{23}Bogen und Wurfspieß führen
sie, grausam sind sie und ohne Erbarmen; ihr Lärmen ist wie
Meeresbrausen, und auf Rossen reiten sie, gerüstet wie ein Mann✲ zum
Kampf gegen dich, Tochter Zion!«

\bibleverse{24}»Wir haben die Kunde von ihm vernommen: die Hände sind
uns schlaff herabgesunken, Angst hat uns erfaßt, Zittern wie ein Weib in
Kindesnöten! \bibleverse{25}Geht nicht aufs Feld\textless sup
title=``oder: ins Freie''\textgreater✲ hinaus und wandert nicht auf der
Landstraße; denn da droht euch das Schwert des Feindes -- Grauen
ringsum!« \bibleverse{26}O Tochter meines Volkes, umgürte dich mit dem
Sackleinen✲ und wälze dich in der Asche! Stelle Trauer an wie um den
einzigen Sohn, eine bittere Wehklage; denn jählings wird der Verwüster
über uns kommen!

\hypertarget{ll-jeremia-als-metallpruxfcfer-seine-vergebliche-arbeit-an-seinem-volke}{%
\subparagraph{ll) Jeremia als Metallprüfer: seine vergebliche Arbeit an
seinem
Volke}\label{ll-jeremia-als-metallpruxfcfer-seine-vergebliche-arbeit-an-seinem-volke}}

\bibleverse{27}»Zum Prüfer habe ich dich bei meinem Volk\textless sup
title=``oder: für mein Volk''\textgreater✲ bestellt, zum Metallprüfer,
damit du ihren Wandel erkennen lernst und prüfst.
\bibleverse{28}Allesamt sind sie widerspenstige Empörer, gehen als
Verleumder umher, sind gemeines Kupfer und Eisen, allesamt schändliche
Bösewichter!« \bibleverse{29}Der Blasebalg hat wohl geschnaubt, aber aus
dem Feuer ist nur Blei herausgekommen: vergeblich ist alles Schmelzen
gewesen, die Schlacken haben sich nicht ausscheiden lassen.
\bibleverse{30}›Verworfenes Silber‹ nennt man sie nun, denn der HERR hat
sie verworfen.

\hypertarget{drohreden-aus-dem-anfang-der-regierung-jojakims-kap.-7-10}{%
\subsubsection{2. Drohreden aus dem Anfang der Regierung Jojakims (Kap.
7-10)}\label{drohreden-aus-dem-anfang-der-regierung-jojakims-kap.-7-10}}

\hypertarget{a-die-tempelrede-jeremias}{%
\paragraph{a) Die Tempelrede Jeremias}\label{a-die-tempelrede-jeremias}}

\hypertarget{aa-gegen-den-rein-uxe4uuxdferlichen-gottesdienst-und-den-offenkundigen-ungehorsam-des-volkes}{%
\subparagraph{aa) Gegen den rein äußerlichen Gottesdienst und den
offenkundigen Ungehorsam des
Volkes}\label{aa-gegen-den-rein-uxe4uuxdferlichen-gottesdienst-und-den-offenkundigen-ungehorsam-des-volkes}}

\hypertarget{section-6}{%
\section{7}\label{section-6}}

\bibleverse{1}Das Wort, das vom HERRN an Jeremia ergangen ist, lautet
folgendermaßen: \bibleverse{2}»Stelle dich auf im Tor des Hauses des
HERRN und halte dort mit lauter Stimme folgende Ansprache: Vernehmt das
Wort des HERRN, ihr Judäer alle, die ihr durch diese Tore eintretet, um
den HERRN anzubeten! \bibleverse{3}So hat der HERR der Heerscharen, der
Gott Israels, gesprochen: Bessert euren Wandel und euer ganzes Tun, so
will ich euch an diesem Orte wohnen lassen! \bibleverse{4}Setzt euer
Vertrauen nicht auf Trugworte, daß ihr sagt: ›Der Tempel des HERRN, der
Tempel des HERRN, der Tempel des HERRN ist dies!‹ \bibleverse{5}Denn
nur, wenn ihr ernstlich euren Wandel und euer ganzes Tun bessert, wenn
ihr wirklich das Recht bei den Streitigkeiten des einen mit dem anderen
gelten laßt, \bibleverse{6}wenn ihr Fremdlinge, Waisen und Witwen nicht
bedrückt und kein unschuldiges Blut an diesem Orte vergießt und nicht
anderen Göttern nachlauft zu eurem eigenen Schaden: \bibleverse{7}nur
dann will ich euch an diesem Orte wohnen lassen, in diesem Lande, das
ich euren Vätern gegeben habe, von Ewigkeit bis in Ewigkeit.

\bibleverse{8}Aber seht: ihr verlaßt euch auf Trugworte, die keinen Wert
haben! \bibleverse{9}Nicht wahr? Stehlen, morden und Ehebruch treiben,
falsch schwören, dem Baal opfern und anderen Göttern nachlaufen, die
euch nichts angehen! \bibleverse{10}Und dann kommt ihr her und tretet
vor mein Angesicht in diesem Hause, das nach meinem Namen genannt ist,
und sagt: ›Wir sind errettet\textless sup title=``oder:
wohlgeborgen''\textgreater✲!‹, um dann alle jene Greuel weiter zu
verüben. \bibleverse{11}Ist denn dieses Haus, das meinen Namen trägt, in
euren Augen zu einer Räuberhöhle geworden? Ja wahrlich, auch ich sehe es
so an!« -- so lautet der Ausspruch des HERRN.

\hypertarget{bb-der-tempel-wird-bei-fortgesetzten-uxfcbertretungen-des-volkes-ebenso-zerstuxf6rt-werden-wie-einst-das-heiligtum-in-silo}{%
\subparagraph{bb) Der Tempel wird bei fortgesetzten Übertretungen des
Volkes ebenso zerstört werden wie einst das Heiligtum in
Silo}\label{bb-der-tempel-wird-bei-fortgesetzten-uxfcbertretungen-des-volkes-ebenso-zerstuxf6rt-werden-wie-einst-das-heiligtum-in-silo}}

\bibleverse{12}»Geht doch einmal nach meiner heiligen Stätte, die in
Silo war, wo ich meinen Namen zu Anfang habe wohnen lassen, und seht,
was ich aus ihr wegen der Bosheit meines Volkes Israel gemacht habe!
\bibleverse{13}Nun aber, weil ihr ganz die gleichen Taten verübt habt«
-- so lautet der Ausspruch des HERRN -- »und, wiewohl ich unaufhörlich
und immer wieder zu euch geredet habe, dennoch nicht gehört und, obschon
ich euch zugerufen, dennoch nicht geantwortet habt: \bibleverse{14}so
will ich mit diesem Hause, das meinen Namen trägt und auf das ihr euer
Vertrauen setzt, und mit der Wohnstätte, die ich euch und euren Vätern
gegeben habe, ebenso verfahren, wie ich mit Silo verfahren bin:
\bibleverse{15}verstoßen will ich euch von meinem Angesicht hinweg, wie
ich alle eure Brüder, die gesamte Nachkommenschaft Ephraims, bereits
verstoßen habe!«

\hypertarget{cc-abweisung-der-fuxfcrbitte-des-propheten-die-abguxf6ttische-verehrung-der-heidnischen-himmelskuxf6nigin}{%
\subparagraph{cc) Abweisung der Fürbitte des Propheten; die abgöttische
Verehrung der heidnischen
Himmelskönigin}\label{cc-abweisung-der-fuxfcrbitte-des-propheten-die-abguxf6ttische-verehrung-der-heidnischen-himmelskuxf6nigin}}

\bibleverse{16}»Du aber, lege keine Fürbitte für dieses Volk ein, laß
kein Flehen und kein Gebet für sie laut werden und dringe nicht in mich;
denn ich würde dich doch nicht erhören. \bibleverse{17}Siehst du denn
nicht, was sie in den Ortschaften Judas und auf den Straßen Jerusalems
treiben? \bibleverse{18}Die Kinder lesen Holz zusammen, und die Väter
zünden das Feuer an; die Frauen aber kneten den Teig, um Kuchen für die
Himmelskönigin zu backen; und Trankopfer spenden sie fremden Göttern, um
mir wehe zu tun. \bibleverse{19}Indes -- tun sie mir damit wehe?« -- so
lautet der Ausspruch des HERRN --, »nicht vielmehr sich selbst zu ihrer
offenkundigen Beschämung?« \bibleverse{20}Darum hat Gott der HERR so
gesprochen: »Wahrlich, mein Zorn und mein Grimm wird sich über diesen
Ort ergießen, über die Menschen und über das Vieh, über die Bäume des
Feldes und die Früchte des Erdbodens, und er wird unauslöschlich
brennen!«

\hypertarget{dd-gehorsam-will-gott-nicht-opfer-und-nicht-selbstgewuxe4hlten-gottesdienst}{%
\subparagraph{dd) Gehorsam will Gott, nicht Opfer und nicht
selbstgewählten
Gottesdienst}\label{dd-gehorsam-will-gott-nicht-opfer-und-nicht-selbstgewuxe4hlten-gottesdienst}}

\bibleverse{21}So hat der HERR der Heerscharen, der Gott Israels,
gesprochen: »Fügt immerhin eure Brandopfer zu euren Schlachtopfern hinzu
und eßt das Fleisch davon! \bibleverse{22}Denn ich habe euren Vätern
damals, als ich sie aus Ägypten wegführte, nichts von Brandopfern und
Schlachtopfern gesagt und geboten, \bibleverse{23}sondern habe ihnen
dies Gebot gegeben: ›Gehorcht meinen Weisungen, so will ich euer Gott
sein, und ihr sollt mein Volk sein; und haltet den ganzen Weg inne, den
ich euch gebiete, damit es euch wohlergehe!‹ \bibleverse{24}Aber sie
haben nicht gehorcht und mir kein Gehör geschenkt, sondern sind nach den
Ratschlägen, nach dem Starrsinn ihres bösen Herzens gewandelt, indem sie
mir den Rücken und nicht mehr das Angesicht zukehrten.
\bibleverse{25}Wohl habe ich seit dem Tage, als eure Väter aus Ägypten
auszogen, bis auf den heutigen Tag alle meine Knechte, die Propheten,
tagtäglich unermüdlich früh und spät zu euch gesandt,
\bibleverse{26}aber sie haben mir nicht gehorcht und mir kein Gehör
geschenkt, sondern sich halsstarrig gezeigt und es noch ärger getrieben
als ihre Väter. \bibleverse{27}Wenn du ihnen nun auch dies alles in
deiner Rede vorhältst, so werden sie doch nicht auf dich hören; und wenn
du ihnen zurufst, werden sie dir keine Antwort geben. \bibleverse{28}So
sage denn zu ihnen: ›Dies ist das Volk, das auf die Stimme des HERRN,
seines Gottes, nicht hört und sich nicht warnen läßt; dahin ist die
Treue und verschwunden aus ihrem Munde!‹«

\hypertarget{ee-der-greuelvolle-guxf6tzendienst-wird-furchtbare-suxfchnung-finden}{%
\subparagraph{ee) Der greuelvolle Götzendienst wird furchtbare Sühnung
finden}\label{ee-der-greuelvolle-guxf6tzendienst-wird-furchtbare-suxfchnung-finden}}

\bibleverse{29}Schere deine Kopfzier\textless sup title=``=~dein schönes
Haupthaar''\textgreater✲ ab, Tochter Zion, und wirf es weg, und stimme
ein Klagelied auf den kahlen Höhen an! Denn verworfen hat der HERR und
verstoßen das Geschlecht, dem er zürnt!

\bibleverse{30}»Denn die Kinder Juda haben getan, was mir mißfällt« --
so lautet der Ausspruch des HERRN --: »sie haben ihre scheußlichen
Götzen in dem Hause, das meinen Namen trägt, aufgestellt, um es zu
entweihen, \bibleverse{31}und haben die Opferstätte des Thopheths im
Tale Ben-Hinnom\textless sup title=``vgl. 2,23''\textgreater✲ angelegt,
um ihre Söhne und Töchter dort als Brandopfer darzubringen, was ich
ihnen niemals geboten habe und was mir nie in den Sinn gekommen ist.
\bibleverse{32}Darum gebt acht: es kommt die Zeit« -- so lautet der
Ausspruch des HERRN --, »da wird man nicht mehr vom Thopheth und vom Tal
Ben-Hinnom reden, sondern vom Würgetal\textless sup title=``=~Mordtal;
vgl. 19,6''\textgreater✲, und man wird im Thopheth begraben, weil sonst
kein Platz mehr da ist. \bibleverse{33}Dann werden die Leichname dieses
Volkes den Vögeln des Himmels und den Tieren des Feldes zum Fraß dienen,
ohne daß jemand sie verscheucht; \bibleverse{34}und ich werde in den
Ortschaften Judas und auf den Straßen Jerusalems aller lauten Freude und
Fröhlichkeit, allem Jubel des Bräutigams und allem Brautgesang ein Ende
machen; denn zur Wüste soll das Land werden!«

\hypertarget{ff-das-schmuxe4hliche-los-der-guxf6tzendiener-nach-der-eroberung-des-landes}{%
\subparagraph{ff) Das schmähliche Los der Götzendiener nach der
Eroberung des
Landes}\label{ff-das-schmuxe4hliche-los-der-guxf6tzendiener-nach-der-eroberung-des-landes}}

\hypertarget{section-7}{%
\section{8}\label{section-7}}

\bibleverse{1}»Zu jener Zeit« -- so lautet der Ausspruch des HERRN --
»wird man die Gebeine der Könige von Juda und die Gebeine seiner
Fürsten\textless sup title=``oder: Oberen''\textgreater✲, die Gebeine
der Priester und Propheten und (überhaupt) die Gebeine der Bewohner
Jerusalems aus ihren Gräbern herausholen \bibleverse{2}und wird sie
hinbreiten vor der Sonne und dem Mond und vor dem ganzen Heer der
Gestirne, welche sie geliebt und verehrt haben und denen sie
nachgelaufen sind, die sie befragt und angebetet haben; man wird sie
dann nicht wieder sammeln noch begraben, nein, zu Dünger sollen sie auf
offenem Felde werden! \bibleverse{3}Und der ganze Rest, der von diesem
bösen Geschlecht noch übriggeblieben ist, wird lieber sterben als leben
wollen an allen Orten, wohin ich sie verstoßen haben werde« -- so lautet
der Ausspruch Gottes, des HERRN der Heerscharen.

\hypertarget{gg-gegen-die-unbuuxdffertigkeit-des-volkes-und-den-duxfcnkel-der-geistlichen-leiter-die-schrecken-des-bevorstehenden-gerichts}{%
\subparagraph{gg) Gegen die Unbußfertigkeit des Volkes und den Dünkel
der geistlichen Leiter; die Schrecken des bevorstehenden
Gerichts}\label{gg-gegen-die-unbuuxdffertigkeit-des-volkes-und-den-duxfcnkel-der-geistlichen-leiter-die-schrecken-des-bevorstehenden-gerichts}}

\bibleverse{4}»Sage zu ihnen auch: So hat der HERR gesprochen: Fällt
jemand wohl hin und steht nicht wieder auf? Oder wendet sich jemand ab
(auf einen falschen Weg) und kehrt nicht wieder um? \bibleverse{5}Warum
hat sich denn dieses Volk zu Jerusalem abgewandt in immerwährender
Abkehr? Sie halten fest am Irrtum und wollen durchaus nicht umkehren.
\bibleverse{6}Ich habe hingehorcht und gelauscht: sie reden die
Unwahrheit; kein einziger bereut seine Schlechtigkeit, daß er
sagte\textless sup title=``oder: dächte''\textgreater✲: ›Was habe ich
getan!‹ Nein, jeder stürmt dahin in seinem Jagen wie ein in die Schlacht
stürmendes Roß. \bibleverse{7}Sogar der Storch oben am Himmel kennt
seine bestimmten Zeiten, auch die Turteltaube, die Schwalbe und der
Kranich halten die Zeit ihrer Wiederkehr ein; aber mein Volk weiß nichts
von der Rechtsordnung des HERRN!«

\bibleverse{8}Wie könnt ihr nur sagen: »Wir sind weise, wir sind ja im
Besitz des göttlichen Gesetzes!« Ja freilich! Aber zur Lüge hat es der
Lügengriffel der Schriftgelehrten gemacht. \bibleverse{9}Beschämt werden
die Weisen dastehen, werden bestürzt sein und sich gefangen✲ sehen; sie
haben ja das Wort des HERRN verworfen: welcherlei Weisheit besitzen sie
da noch?

\hypertarget{hh-drohwort-gegen-die-geistlichen-fuxfchrer-des-volkes}{%
\subparagraph{hh) Drohwort gegen die geistlichen Führer des
Volkes}\label{hh-drohwort-gegen-die-geistlichen-fuxfchrer-des-volkes}}

\bibleverse{10}»Darum will ich ihre Weiber anderen geben, ihre Äcker
Eroberern\textless sup title=``oder: anderen Besitzern''\textgreater✲;
denn vom Jüngsten bis zum Ältesten sind sie alle gierig nach Gewinn, vom
Propheten bis zum Priester gehen sie alle mit Lug und Trug um.
\bibleverse{11}Die schwere Wunde der Tochter meines Volkes wollen sie
leichtfertig obenhin heilen, indem sie verheißen: ›Heil, Heil!‹, wo doch
kein Heil vorhanden ist. \bibleverse{12}Beschämt werden sie dastehen
müssen, weil sie Greuel verübt haben; und doch schämen sie sich
keineswegs und Erröten kennen sie nicht mehr. Darum werden sie fallen,
wenn alles fällt: zur Zeit, da ich sie zur Rechenschaft ziehe, werden
sie stürzen!« -- so hat der HERR gesprochen. \bibleverse{13}»Will ich
Lese bei ihnen halten« -- so lautet der Ausspruch des HERRN --, »so sind
keine Trauben am Weinstock und keine Feigen am Feigenbaum, und das Laub
ist verwelkt. So will ich denn Leute für sie bestellen, die sie
wegräumen sollen!«

\hypertarget{ii-die-schrecken-des-durch-das-herannahen-des-nordischen-feindes-sich-vollziehenden-gottesgerichts}{%
\subparagraph{ii) Die Schrecken des durch das Herannahen des nordischen
Feindes sich vollziehenden
Gottesgerichts}\label{ii-die-schrecken-des-durch-das-herannahen-des-nordischen-feindes-sich-vollziehenden-gottesgerichts}}

\bibleverse{14}Wozu sitzen wir müßig da? Scharet euch zusammen und laßt
uns in die festen Städte flüchten und dort umkommen! Denn der HERR,
unser Gott, will unsern Untergang und hat uns Giftwasser zu trinken
gegeben, weil wir gegen den HERRN gesündigt haben. \bibleverse{15}Wozu
hoffen wir noch auf Rettung? Es wird ja doch nichts Gutes! Wozu auf eine
Zeit der Heilung? Es gibt ja doch nur Schrecken! \bibleverse{16}Von Dan
her vernimmt man schon das Schnauben seiner Rosse, vom lauten Gewieher
seiner Hengste erbebt das ganze Land! Ja, sie kommen heran und verzehren
das Land und seine Fülle, die Städte samt ihren Bewohnern.
\bibleverse{17}»Denn gebt acht: ich lasse Schlangen gegen euch los,
Giftschlangen, gegen welche es keine Beschwörung gibt: die sollen euch
stechen!« -- so lautet der Ausspruch des HERRN.

\hypertarget{jj-des-propheten-hoffnungslosigkeit-und-sein-schmerz-uxfcber-die-sittliche-zerruxfcttung-des-volkes}{%
\subparagraph{jj) Des Propheten Hoffnungslosigkeit und sein Schmerz über
die sittliche Zerrüttung des
Volkes}\label{jj-des-propheten-hoffnungslosigkeit-und-sein-schmerz-uxfcber-die-sittliche-zerruxfcttung-des-volkes}}

\bibleverse{18}Schwerer Kummer ist über mich gekommen, krank ist mein
Herz in mir! \bibleverse{19}Horch! Laut erschallt das Wehgeschrei der
Tochter meines Volkes aus fernem Lande: »Ist denn der HERR nicht in
Zion? Ist sein\textless sup title=``d.h. Zions''\textgreater✲ König
nicht drinnen?« »Warum haben sie mich erbittert durch ihren
Bilderdienst, durch die nichtigen Götzen des Auslands?«
\bibleverse{20}»Vorüber ist die Ernte, zu Ende die Obstlese\textless sup
title=``oder: der Sommer''\textgreater✲, doch wir haben keine Rettung
erlangt!«

\bibleverse{21}Über den Zusammenbruch der Tochter meines Volkes bin ich
gebrochen; ich gehe trauernd einher, Entsetzen hat mich ergriffen!
\bibleverse{22}Gibt es denn keinen Balsam mehr in Gilead, und ist kein
Arzt mehr dort? Ach, warum ist der Tochter meines Volkes noch keine
Heilung zuteil geworden! \bibleverse{23}O daß doch mein Haupt zu Wasser
würde und meine Augen zum Tränenquell! Dann wollte ich Tag und Nacht
weinen um die Erschlagenen der Tochter meines Volkes!

\hypertarget{kk-jeremia-verzweifelt-an-seinem-sittlich-zerruxfctteten-volke}{%
\subparagraph{kk) Jeremia verzweifelt an seinem sittlich zerrütteten
Volke}\label{kk-jeremia-verzweifelt-an-seinem-sittlich-zerruxfctteten-volke}}

\hypertarget{section-8}{%
\section{9}\label{section-8}}

\bibleverse{1}O hätte ich doch eine Wanderer-Herberge✲ fern in der
Wüste, so wollte ich mein Volk verlassen und von ihnen weggehen! Denn
sie sind allesamt Ehebrecher, eine Gesellschaft von
Treulosen\textless sup title=``oder: Betrügern''\textgreater✲.
\bibleverse{2}»Sie spannen ihre Zunge wie einen Bogen: mit Lüge und
nicht durch Wahrhaftigkeit schalten sie als Herren im Lande; denn von
einer Bosheit schreiten sie zur andern fort, mich aber kennen sie
nicht!« -- so lautet der Ausspruch des HERRN. \bibleverse{3}Seid auf der
Hut, ein jeder vor seinem Freunde, und schenkt auch keinem Bruder
Vertrauen! Denn jeder Bruder übt Lug und Trug, und jeder Freund geht auf
Verleumdung aus; \bibleverse{4}sie hintergehen einer den andern, und
keiner redet ein wahres Wort; sie haben ihre Zunge an Lügenreden gewöhnt
und mühen sich ab, verkehrt zu handeln, können nicht anders:
\bibleverse{5}»Gewalttat über Gewalttat, Arglist über Arglist! Sie
wollen mich nicht kennen!« -- so lautet der Ausspruch des HERRN.
\bibleverse{6}Darum hat der HERR der Heerscharen so gesprochen: »Gib
acht: ich will sie schmelzen und läutern! Denn wie sollte ich anders mit
der Tochter meines Volks verfahren? \bibleverse{7}Ein todbringender
Pfeil ist ihre Zunge, Trug sind die Worte ihres Mundes; man redet
freundlich mit seinem Nächsten, aber im Herzen stellt man ihm eine
Falle. \bibleverse{8}Sollte ich so etwas bei ihnen ungestraft lassen?«
-- so lautet der Ausspruch des HERRN --, »oder sollte ich mich an einem
solchen Volk nicht rächen?«

\hypertarget{ll-die-wehklage-des-propheten-gottes-gerichtsspruch-klagelieder}{%
\subparagraph{ll) Die Wehklage des Propheten; Gottes Gerichtsspruch;
Klagelieder}\label{ll-die-wehklage-des-propheten-gottes-gerichtsspruch-klagelieder}}

\bibleverse{9}Um die Berge will ich ein Weinen und eine Wehklage erheben
und um die Auen der Trift ein Trauerlied! Denn sie sind verödet, so daß
niemand sie durchwandert und man die Stimmen der Herden nicht mehr
vernimmt; die Vögel des Himmels und das Wild -- alles ist entflohen, ist
weggezogen! \bibleverse{10}»Auch Jerusalem will ich zu Steinhaufen
machen, zur Behausung der Schakale; und die Ortschaften Judas verwandle
ich in eine Einöde, in der kein Mensch mehr wohnt!«

\hypertarget{mm-die-frage-des-propheten-und-der-gerichtsspruch-gottes-als-antwort}{%
\subparagraph{mm) Die Frage des Propheten und der Gerichtsspruch Gottes
als
Antwort}\label{mm-die-frage-des-propheten-und-der-gerichtsspruch-gottes-als-antwort}}

\bibleverse{11}Wer ist ein so weiser Mann, daß er dies verstehen mag?
Und zu wem hat der Mund des HERRN geredet, daß er es kundtue, warum das
Land zugrunde gegangen ist und verödet daliegt wie die Wüste, so daß
niemand es durchwandert?

\bibleverse{12}Der HERR hat gesagt: »Weil sie mein Gesetz, das ich ihnen
vorgelegt hatte, unbeachtet gelassen und meinen Weisungen nicht gehorcht
und nicht nach ihnen gelebt haben, \bibleverse{13}vielmehr dem Starrsinn
ihres eigenen Herzens gefolgt und den Baalen nachgelaufen sind, wie sie
es von ihren Vätern gelernt hatten; \bibleverse{14}darum« -- so hat der
HERR der Heerscharen, der Gott Israels, gesprochen: »Nunmehr will ich
sie, dieses Volk da, mit Wermut speisen und ihnen Giftwasser zu trinken
geben \bibleverse{15}und will sie unter die Heidenvölker zerstreuen, die
weder sie noch ihre Väter gekannt haben, und will das Schwert hinter
ihnen her senden, bis ich sie vertilgt habe!«

\hypertarget{nn-klagelieder-uxfcber-das-volk}{%
\subparagraph{nn) Klagelieder über das
Volk}\label{nn-klagelieder-uxfcber-das-volk}}

\bibleverse{16}So hat der HERR der Heerscharen gesprochen: »Merkt auf
und ruft die Klageweiber herbei, daß sie kommen, und schickt zu den
weisen Frauen\textless sup title=``=~den wehgesangskundigen
Weibern''\textgreater✲, daß sie herkommen \bibleverse{17}und eilends ein
Klagelied\textless sup title=``oder: das Totenlied''\textgreater✲ über
uns anstimmen, damit unsere Augen in Tränen zerfließen und unsere
Wimpern von Zähren triefen!«~--

\bibleverse{18}Ach horch! Eine Wehklage vernimmt man von Zion her:
»Wehe, wie sind wir vergewaltigt und schmählich in Schande geraten! Ach,
wir müssen das Land verlassen! Ach, man hat unsere Wohnungen
niedergerissen!«~--

\bibleverse{19}Ach hört, ihr Frauen, das Wort des HERRN, und euer Ohr
vernehme das Wort seines Mundes, und lehrt eure Töchter das Klagelied
und eine jede die andere den Grabgesang: \bibleverse{20}»Ach, der Tod
ist in unsere Fenster eingestiegen, in unsere Paläste eingedrungen, hat
die Kinder von der Straße weggerafft, die Jünglinge\textless sup
title=``oder: jungen Männer''\textgreater✲ von den Marktplätzen!«~--

\bibleverse{21}Rede: »So lautet der Ausspruch des HERRN: Es liegen die
Leichen der Menschen da wie Dünger auf dem Felde und wie Ährenbündel
hinter dem Schnitter, ohne daß jemand sie aufliest!«

\hypertarget{oo-anhang-der-wahre-selbstruhm-die-rechte-beschneidung}{%
\subparagraph{oo) Anhang: Der wahre Selbstruhm; die rechte
Beschneidung}\label{oo-anhang-der-wahre-selbstruhm-die-rechte-beschneidung}}

\bibleverse{22}So hat der HERR gesprochen: »Nicht rühme sich der Weise
seiner Weisheit, und der Starke rühme sich nicht seiner Stärke, nicht
rühme sich der Reiche seines Reichtums! \bibleverse{23}Sondern wer sich
rühmen will, der rühme sich dessen, daß er Einsicht besitzt und von mir
erkennt, daß ich, der HERR, es bin, der Gnade, Recht und Gerechtigkeit
auf Erden übt\textless sup title=``oder: walten läßt''\textgreater✲;
denn an solchen\textless sup title=``oder: daran''\textgreater✲ habe ich
Wohlgefallen« -- so lautet der Ausspruch des HERRN.

\hypertarget{pp-israel-ist-unbeschnitten-am-herzen}{%
\subparagraph{pp) Israel ist unbeschnitten am
Herzen}\label{pp-israel-ist-unbeschnitten-am-herzen}}

\bibleverse{24}»Gebt acht: es kommt die Zeit« -- so lautet der Ausspruch
des HERRN --, »da werde ich alle, die, obgleich beschnitten, doch
unbeschnitten sind, zur Rechenschaft ziehen: \bibleverse{25}Ägypten und
Juda, Edom, die Ammoniter und Moabiter und alle, die sich das Haar an
den Schläfen stutzen, die in der Wüste wohnen. Denn wohl sind alle
Heidenvölker unbeschnitten, aber das ganze Haus Israel ist unbeschnitten
am Herzen!«

\hypertarget{b-die-nichtigkeit-der-guxf6tzen-und-die-erhabenheit-des-allein-wahren-gottes}{%
\paragraph{b) Die Nichtigkeit der Götzen und die Erhabenheit des allein
wahren
Gottes}\label{b-die-nichtigkeit-der-guxf6tzen-und-die-erhabenheit-des-allein-wahren-gottes}}

\hypertarget{section-9}{%
\section{10}\label{section-9}}

\bibleverse{1}Vernehmt das Wort, das der HERR euch verkünden läßt, ihr
vom Hause Israel! \bibleverse{2}So hat der HERR gesprochen: »Gewöhnt
euch nicht an den Weg\textless sup title=``=~die Weise''\textgreater✲
der Heidenvölker und laßt euch nicht durch die Zeichen am Himmel
erschrecken, weil\textless sup title=``oder: wennschon''\textgreater✲
die Heidenvölker vor ihnen erschrecken! \bibleverse{3}Denn der
Gottesdienst der Heidenvölker ist nichts als Wahn: ein Stück Holz ist es
ja, das man im Walde gehauen hat, ein Werk von Künstlerhänden, mit dem
Schnitzmesser hergestellt. \bibleverse{4}Mit Silber und Gold verziert er
es\textless sup title=``d.h. das geschnitzte Holzbild''\textgreater✲,
mit Nägeln und Hämmern befestigt man es, damit es nicht wackelt.
\bibleverse{5}Wie eine Vogelscheuche im Gemüsegarten stehen sie da und
können nicht reden; man muß sie tragen, denn sie können nicht gehen.
Fürchtet euch nicht vor ihnen, denn sie können kein Unheil anrichten,
aber auch Gutes zu tun\textless sup title=``oder: Glück zu
bringen''\textgreater✲ steht nicht in ihrer Macht.«

\hypertarget{die-erhabenheit-gottes-gegenuxfcber-der-veruxe4chtlichkeit-der-guxf6tzen}{%
\paragraph{Die Erhabenheit Gottes gegenüber der Verächtlichkeit der
Götzen}\label{die-erhabenheit-gottes-gegenuxfcber-der-veruxe4chtlichkeit-der-guxf6tzen}}

\bibleverse{6}Dir, o HERR, ist niemand gleich! Groß bist du, und groß
ist dein Name ob deiner Kraft: \bibleverse{7}wer sollte dich nicht
fürchten, du König der Völker? Ja, dir gebührt dies; denn unter allen
Weisen der Heidenvölker und in all ihren Königreichen ist keiner dir
gleich, \bibleverse{8}sondern allesamt sind sie dumm und töricht: die
ganze Weisheit der Götzen ist -- Holz, \bibleverse{9}dünngehämmertes
Silber, das man aus Tharsis geholt hat, und Gold aus Uphas, eine Arbeit
des Bildschnitzers und der Hände des Goldschmieds, blauer und roter
Purpur ist ihr Gewand: Machwerke kunstfertiger Meister sind sie
allesamt. \bibleverse{10}Aber der HERR ist Gott in Wahrheit, ist der
lebendige Gott und ein ewiger König; vor seinem Zürnen erbebt die Erde,
und seinen Grimm vermögen die Völker nicht zu ertragen.
\bibleverse{11}{[}So sollt ihr von\textless sup title=``oder:
zu''\textgreater✲ ihnen sagen: »Die Götter, die den Himmel und die Erde
nicht geschaffen haben, diese werden von der Erde und unter diesem
Himmel hinweg verschwinden.«{]} \bibleverse{12}Der HERR ist es, der die
Erde durch seine Kraft geschaffen, den Erdkreis durch seine Weisheit
fest gegründet und durch seine Einsicht den Himmel ausgespannt hat.
\bibleverse{13}Wenn er beim Schall des Donners Wasserrauschen am Himmel
entstehen läßt und Gewölk vom Ende der Erde heraufführt, wenn er Blitze
beim Regen schafft und den Sturmwind aus seinen Vorratskammern
herausläßt~-- \bibleverse{14}starr steht alsdann jeder Mensch da, ohne
es begreifen zu können, und schämen muß sich jeder Goldschmied seines
Bildwerks; denn Trug ist sein gegossener Götze, kein Odem✲ wohnt in ihm:
\bibleverse{15}nichts als Wahn✲ sind sie, lächerliche Gebilde; wenn die
Zeit des Strafgerichts für sie kommt, ist es zu Ende mit ihnen.
\bibleverse{16}Aber nicht wie diese ist Jakobs Erbteil; nein, er ist es,
der das All geschaffen hat, und Israel ist der Stamm seines Erbbesitzes:
HERR der Heerscharen ist sein Name.

\hypertarget{c-die-not-des-fuxfcr-die-verbannung-bestimmten-volkes-seine-klage-uxfcber-das-verhuxe4ngte-strafgericht-und-seine-demuxfctige-ergebung-in-gottes-willen}{%
\paragraph{c) Die Not des für die Verbannung bestimmten Volkes; seine
Klage über das verhängte Strafgericht und seine demütige Ergebung in
Gottes
Willen}\label{c-die-not-des-fuxfcr-die-verbannung-bestimmten-volkes-seine-klage-uxfcber-das-verhuxe4ngte-strafgericht-und-seine-demuxfctige-ergebung-in-gottes-willen}}

\bibleverse{17}Raffe dein Bündel von der Erde\textless sup title=``oder:
aus dem Lande''\textgreater✲ zusammen, Tochter Zion, die du in
Belagerungsnot\textless sup title=``oder: Bedrängnis''\textgreater✲
sitzest! \bibleverse{18}Denn so hat der HERR gesprochen: »Wisse wohl:
diesmal will ich die Bewohner des Landes hinwegschleudern und sie in
Bedrängnis versetzen, damit sie endlich zur Erkenntnis kommen!«

\bibleverse{19}»Wehe mir ob meiner Wunde: qualvoll ist der Schlag, der
mich getroffen! Doch ich denke: ›Das ist nun einmal mein Leiden: so will
ich es denn tragen!‹ \bibleverse{20}Mein Zelt ist zerstört, und alle
meine Zeltstricke sind zerrissen; meine Kinder sind mir entführt, keines
ist mehr da; niemand schlägt mir hinfort mein Zelt wieder auf und
breitet meine Decken darüber aus! \bibleverse{21}Denn meine Hirten sind
verdummt gewesen und haben nicht nach dem HERRN gefragt; darum haben sie
kein Gelingen\textless sup title=``oder: keinen Erfolg''\textgreater✲
gehabt, und ihre ganze Herde ist zerstoben.«

\bibleverse{22}Horch, eine Kunde: siehe da, sie kommt, und ein
gewaltiges Getöse vom Nordlande her, um die Städte Judas zur Wüste zu
machen, zur Behausung der Schakale!

\hypertarget{gebet-des-volkes-um-gottes-gnade-und-um-bestrafung-der-rohen-heiden}{%
\paragraph{Gebet des Volkes um Gottes Gnade und um Bestrafung der rohen
Heiden}\label{gebet-des-volkes-um-gottes-gnade-und-um-bestrafung-der-rohen-heiden}}

\bibleverse{23}Ich weiß, HERR, daß des Menschen Schicksal nicht in
seiner Hand liegt und daß ein Mann, der da wandelt, seinen Gang nicht
fest zu richten vermag. \bibleverse{24}Züchtige mich, HERR, aber nach
Billigkeit\textless sup title=``oder: mit Maßen''\textgreater✲, nicht in
deinem Zorn, auf daß du mich nicht ganz vernichtest!
\bibleverse{25}Gieße deinen Grimm aus über die Heidenvölker, die dich
nicht kennen, und über die Stämme, die deinen Namen nicht anrufen! Denn
sie haben Jakob verschlungen, ja ganz und gar aufgezehrt und seine Aue✲
zur Wüste gemacht!

\hypertarget{reden-aus-der-zeit-jojakims-als-der-einbruch-des-heeres-nebukadnezars-drohte-kap.-11-13}{%
\subsubsection{3. Reden aus der Zeit Jojakims, als der Einbruch des
Heeres Nebukadnezars drohte (Kap.
11-13)}\label{reden-aus-der-zeit-jojakims-als-der-einbruch-des-heeres-nebukadnezars-drohte-kap.-11-13}}

\hypertarget{a-die-untreue-des-volkes-gegen-den-gottesbund-und-ihre-vernichtenden-folgen}{%
\paragraph{a) Die Untreue des Volkes gegen den Gottesbund und ihre
vernichtenden
Folgen}\label{a-die-untreue-des-volkes-gegen-den-gottesbund-und-ihre-vernichtenden-folgen}}

\hypertarget{aa-das-band-zwischen-gott-und-dem-volk-ist-zerschnitten}{%
\subparagraph{aa) Das Band zwischen Gott und dem Volk ist
zerschnitten}\label{aa-das-band-zwischen-gott-und-dem-volk-ist-zerschnitten}}

\hypertarget{section-10}{%
\section{11}\label{section-10}}

\bibleverse{1}Das Wort, das an Jeremia vom HERRN erging, lautete
folgendermaßen: \bibleverse{2}»Vernehmet die Worte dieses Bundes und
verkündet sie den Männern von Juda und besonders den Bewohnern
Jerusalems \bibleverse{3}und sage du noch folgendes zu ihnen: So hat der
HERR, der Gott Israels, gesprochen: ›Verflucht ist jeder, der den
Bestimmungen dieses Bundes nicht nachkommt, \bibleverse{4}den ich euren
Vätern zur Pflicht gemacht habe zu der Zeit, als ich sie aus dem Lande
Ägypten, aus dem Eisen-Schmelzofen, wegführte, indem ich ihnen sagte:
Gehorcht meinen Weisungen und handelt nach ihnen ganz so, wie ich euch
gebiete, so sollt ihr mein Volk sein, und ich will euer Gott sein,
\bibleverse{5}damit ich den Eid aufrechterhalten✲ kann, den ich euren
Vätern zugeschworen habe, nämlich ihnen ein Land zu geben, das von Milch
und Honig überfließt, wie es heute noch der Fall ist!‹« Da antwortete
ich mit den Worten: »Amen\textless sup title=``=~so sei
es''\textgreater✲, HERR!«

\bibleverse{6}Darauf sagte der HERR zu mir: »Mache alle diese Worte in
den Ortschaften Judas und auf den Straßen Jerusalems bekannt, indem du
sagst: ›Vernehmet die Worte dieses Bundes und handelt nach ihnen!
\bibleverse{7}Denn ich habe eure Väter zu der Zeit, als ich sie aus dem
Lande Ägypten wegführte, und bis auf den heutigen Tag immer und immer
wieder aufs ernstlichste ermahnt und sie aufgefordert, meinen Weisungen
zu gehorchen. \bibleverse{8}Aber sie haben nicht gehorcht und mir kein
Gehör geschenkt, sondern sind ein jeder nach dem Starrsinn seines
eigenen bösen Herzens gewandelt. Deshalb habe ich alle Drohworte dieses
Bundes, dessen Beobachtung ich ihnen geboten hatte, an ihnen in
Erfüllung gehen lassen, weil sie nicht danach gehandelt hatten.‹«

\hypertarget{bb-des-volkes-bundesbruch-und-verwerfung}{%
\subparagraph{bb) Des Volkes Bundesbruch und
Verwerfung}\label{bb-des-volkes-bundesbruch-und-verwerfung}}

\bibleverse{9}Weiter sagte der HERR zu mir: »Es besteht eine
Verschwörung unter den Männern von Juda und den Bewohnern Jerusalems.
\bibleverse{10}Sie sind in die Sünden ihrer Vorväter zurückgefallen, die
meinen Weisungen den Gehorsam versagt haben; sind doch auch sie anderen
Göttern nachgelaufen, um sie zu verehren: gebrochen haben die vom Hause
Israel und vom Hause Juda meinen Bund, den ich mit ihren Vätern
geschlossen habe!« \bibleverse{11}Darum hat der HERR so gesprochen:
»Nunmehr will ich ein Unheil über sie bringen, aus dem herauszukommen
ihnen nicht möglich sein soll; und wenn sie dann zu mir um Hilfe
schreien, will ich sie nicht erhören. \bibleverse{12}Wenn dann die
Ortschaften Judas und die Bewohner Jerusalems hingehen und zu den
Göttern, denen sie geopfert haben, um Hilfe schreien, so werden diese
ihnen zur Zeit ihres Unglücks nimmermehr helfen können.
\bibleverse{13}Denn so zahlreich, wie deine Städte sind, ebenso
zahlreich sind auch deine Götter geworden, Juda; und so viele Straßen es
in Jerusalem gibt, ebenso viele Altäre habt ihr dem Schandgötzen
errichtet, Altäre, um dem Baal zu räuchern. \bibleverse{14}Du aber
(Jeremia) sollst keine Fürbitte für dieses Volk einlegen und kein Flehen
und kein Gebet für sie laut werden lassen! Denn ich würde doch nicht
hören, wenn sie mich zur Zeit ihrer Not um Hilfe anriefen!«

\hypertarget{cc-opfer-und-selbstgewuxe4hlte-weihegaben-halten-den-untergang-des-einst-gottgeliebten-volkes-nicht-auf}{%
\subparagraph{cc) Opfer und selbstgewählte Weihegaben halten den
Untergang des einst gottgeliebten Volkes nicht
auf}\label{cc-opfer-und-selbstgewuxe4hlte-weihegaben-halten-den-untergang-des-einst-gottgeliebten-volkes-nicht-auf}}

\bibleverse{15}»Was will mein Geliebter✲ in meinem Hause? Etwa böse
Anschläge ausführen? Werden Gelübde und Opferfleisch deine
Bosheit\textless sup title=``oder: dein Unglück''\textgreater✲ von dir
wegschaffen, daß du alsdann frohlocken dürftest?« \bibleverse{16}Einen
›immergrünen Ölbaum im Schmuck herrlicher Früchte‹ -- so hat der HERR
dich einst genannt; aber wegen seines gewaltig lauten Rauschens legt er
jetzt Feuer an ihn, und es brechen\textless sup title=``oder:
verbrennen''\textgreater✲ seine Zweige. \bibleverse{17}Denn der HERR der
Heerscharen, der dich gepflanzt, hat Unheil über dich beschlossen wegen
der Bosheit, die das Haus Israel und das Haus Juda verübt haben, um mich
zu erbittern, indem sie dem Baal opferten.«

\hypertarget{b-der-mordanschlag-der-bewohner-von-anathoth-gegen-jeremia-und-ihre-strafe}{%
\paragraph{b) Der Mordanschlag der Bewohner von Anathoth gegen Jeremia
und ihre
Strafe}\label{b-der-mordanschlag-der-bewohner-von-anathoth-gegen-jeremia-und-ihre-strafe}}

\bibleverse{18}Der HERR hat es mich wissen lassen, da habe ich es
erfahren; damals hast du mich ihr Treiben durchschauen lassen.
\bibleverse{19}Ich selbst war wie ein argloses Lamm, das zur
Schlachtbank geführt wird, und ahnte nicht, daß sie böse Anschläge gegen
mich schmiedeten: »Laßt uns den Baum samt seinen Früchten vernichten und
ihn aus dem Lande der Lebenden ausrotten, daß seines Namens nicht mehr
gedacht wird!«

\bibleverse{20}Und nun, HERR der Heerscharen, gerechter Richter, der du
Nieren und Herz prüfst: laß mich deine Rache an ihnen sehen! Denn dir
habe ich meine Sache anheimgestellt.

\bibleverse{21}Darum hat der HERR so gesprochen in bezug auf die Männer
von Anathoth, die dir nach dem Leben getrachtet und zu dir gesagt haben:
»Du sollst nicht im Namen des HERRN weissagen, sonst mußt du durch
unsere Hand sterben!«~-- \bibleverse{22}darum hat der HERR der
Heerscharen so gesprochen: »Wisse wohl: ich will es sie büßen lassen!
Ihre jungen Männer sollen durchs Schwert umkommen, ihre Söhne und
Töchter Hungers sterben! \bibleverse{23}Kein Überrest soll ihnen
verbleiben, denn ich will Unglück über die Männer von Anathoth bringen
in dem Jahre, wo ich die Strafe an ihnen vollziehe!«

\hypertarget{c-das-ruxe4tselhafte-walten-gottes}{%
\paragraph{c) Das rätselhafte Walten
Gottes}\label{c-das-ruxe4tselhafte-walten-gottes}}

\hypertarget{aa-jeremias-anfrage-an-gott-bezuxfcglich-des-gluxfccks-der-gottlosen}{%
\subparagraph{aa) Jeremias Anfrage an Gott bezüglich des Glücks der
Gottlosen}\label{aa-jeremias-anfrage-an-gott-bezuxfcglich-des-gluxfccks-der-gottlosen}}

\hypertarget{section-11}{%
\section{12}\label{section-11}}

\bibleverse{1}»Du behältst recht, HERR, wenn ich mit dir streite, und
doch möchte ich über (dein) richterliches Walten mit dir reden: Warum
ist das Tun und Lassen der Gottlosen erfolgreich, und warum bleiben
alle, die treulos handeln, unangefochten? \bibleverse{2}Du selbst
pflanzest sie ein, sie schlagen auch Wurzel; sie gedeihen und bringen
auch Frucht: nahe bist du ihnen ihrem Munde nach, doch fern von ihrem
Herzen\textless sup title=``vgl. Jes 29,13''\textgreater✲.
\bibleverse{3}Du aber, HERR, du kennst mich durch und durch und hast
erprobt, wie mein Herz zu dir steht: raffe sie hinweg wie Schafe zur
Schlachtung und weihe sie für den Tag, an dem sie abgetan werden!
\bibleverse{4}Wie lange soll das Land noch trauern und die Gewächse auf
der ganzen Flur verdorren? Wie lange noch sollen infolge der Bosheit
seiner Bewohner Vieh und Vögel hinschwinden? Sie sagen\textless sup
title=``oder: denken''\textgreater✲ ja doch: ›Er\textless sup
title=``d.h. Jeremia''\textgreater✲ wird unser Ende nicht zu sehen
bekommen!‹«

\hypertarget{bb-die-guxf6ttliche-antwort}{%
\subparagraph{bb) Die göttliche
Antwort}\label{bb-die-guxf6ttliche-antwort}}

\bibleverse{5}»Wenn du mit Fußgängern wettläufst und die dich schon müde
machen, wie willst du da mit Rossen um die Wette rennen? Und wenn du
dich nur in einem friedlichen Lande sicher fühlst, wie willst du es da
machen im hohen Dickicht des Jordans? \bibleverse{6}Denn selbst deine
Verwandten und deines Vaters Haus -- sogar die sind treulos gegen dich,
auch die haben hinter dir her geschrien aus voller Kehle; traue ihnen
nicht, wenn sie auch freundlich mit dir reden!«

\hypertarget{cc-gottes-klagelied-uxfcber-sein-durch-die-nachbarvuxf6lker-verwuxfcstetes-land}{%
\subparagraph{cc) Gottes Klagelied über sein durch die Nachbarvölker
verwüstetes
Land}\label{cc-gottes-klagelied-uxfcber-sein-durch-die-nachbarvuxf6lker-verwuxfcstetes-land}}

\bibleverse{7}»Ich habe mich von meinem Hause losgesagt, meinen
Erbbesitz hingegeben, habe den Liebling meines Herzens in die Gewalt
seiner Feinde fallen lassen. \bibleverse{8}Mein Erbteil ist mir geworden
wie ein Löwe im Walde: es hat sein Gebrüll gegen mich erhoben, darum
mußte es mir verhaßt werden. \bibleverse{9}Ist denn mein Erbbesitz für
mich zu einem bunten Vogel geworden, daß die Vögel sich ringsum dawider
sammeln? Auf! Laßt alle Tiere des Feldes zusammenkommen! Bringt sie zum
Fressen herbei! \bibleverse{10}Viele Hirten haben meinen Weinberg
verwüstet, meinen Grund und Boden zertreten; sie haben den Acker, der
meine Lust war, zur öden Trift gemacht. \bibleverse{11}In eine Einöde
haben sie ihn verwandelt, verödet trauert er um mich her; verwüstet ist
das ganze Land, weil niemand es sich hat zu Herzen gehen lassen.«
\bibleverse{12}Über alle kahlen Höhen in der Trift sind Verwüster
eingebrochen; denn ein Schwert hat der HERR, das von einem Ende des
Landes bis zum andern frißt: da gibt es keine Rettung für alles
Fleisch\textless sup title=``=~für irgendein Geschöpf''\textgreater✲.
\bibleverse{13}Sie haben Weizen gesät, aber Dornen geerntet, haben sich
abgemüht, ohne etwas auszurichten. So werdet denn zuschanden an euren
Ernteerträgen infolge des lodernden Zornes des HERRN!

\hypertarget{dd-gerichts--und-heilsankuxfcndigung-fuxfcr-die-heidnischen-nachbarvuxf6lker}{%
\subparagraph{dd) Gerichts- und Heilsankündigung für die heidnischen
Nachbarvölker}\label{dd-gerichts--und-heilsankuxfcndigung-fuxfcr-die-heidnischen-nachbarvuxf6lker}}

\bibleverse{14}So hat der HERR über alle meine bösen Nachbarn
gesprochen, die den Erbbesitz angetastet haben, den ich meinem Volke
Israel zu eigen gegeben habe: »Wisse wohl: ich will sie aus ihrem Boden
herausreißen, wie ich das Haus Juda aus ihrer Mitte wegreiße!
\bibleverse{15}Wenn ich sie aber herausgerissen habe, alsdann will ich
mich ihrer wieder erbarmen und will sie zurückführen, einen jeden in
seinen Erbbesitz und einen jeden in sein Land. \bibleverse{16}Wenn sie
sich dann an die Wege\textless sup title=``oder: Weise, d.h.
Gottesverehrung''\textgreater✲ meines Volkes fest gewöhnen, so daß sie
bei meinem Namen schwören: ›So wahr der HERR lebt!‹, gleichwie sie mein
Volk daran gewöhnt haben, beim Baal zu schwören, so sollen sie inmitten
meines Volkes aufgebaut werden. \bibleverse{17}Wollen sie aber nicht
gehorchen, so will ich ein solches Volk mit Stumpf und Stiel für immer
ausreißen!« -- so lautet der Ausspruch des HERRN.

\hypertarget{d-ankuxfcndigung-des-strafgerichts-uxfcber-das-unverbesserliche-volk-mahnung-zur-umkehr}{%
\paragraph{d) Ankündigung des Strafgerichts über das unverbesserliche
Volk; Mahnung zur
Umkehr}\label{d-ankuxfcndigung-des-strafgerichts-uxfcber-das-unverbesserliche-volk-mahnung-zur-umkehr}}

\hypertarget{aa-gleichnis-d.h.-sinnbildliche-handlung-vom-linnenen-guxfcrtel}{%
\subparagraph{aa) Gleichnis (d.h. sinnbildliche Handlung) vom linnenen
Gürtel}\label{aa-gleichnis-d.h.-sinnbildliche-handlung-vom-linnenen-guxfcrtel}}

\hypertarget{section-12}{%
\section{13}\label{section-12}}

\bibleverse{1}So hat der HERR zu mir gesprochen: »Gehe hin und kaufe dir
einen linnenen Gürtel und lege ihn dir um die Hüften, aber laß ihn nicht
ins Wasser geraten!« \bibleverse{2}Da kaufte ich den Gürtel nach dem
Befehl des HERRN und legte ihn mir um die Hüften. \bibleverse{3}Hierauf
erging das Wort des HERRN zum zweitenmal an mich folgendermaßen:
\bibleverse{4}»Nimm den Gürtel, den du dir gekauft hast und um deine
Hüften trägst, und mache dich auf, gehe an den Euphrat und verstecke ihn
dort in einer Felsspalte!« \bibleverse{5}Da ging ich hin und versteckte
ihn am Euphrat, wie der HERR mir geboten hatte. \bibleverse{6}Nach
längerer Zeit aber sagte der HERR zu mir: »Mache dich auf, gehe an den
Euphrat und hole von dort den Gürtel, den du dort auf meinen Befehl
versteckt hast!« \bibleverse{7}Da begab ich mich an den Euphrat, grub
nach und nahm den Gürtel von der Stelle, wo ich ihn versteckt hatte;
aber siehe da: der Gürtel war verdorben, war zu nichts mehr zu
gebrauchen!

\hypertarget{bb-die-deutung-des-gleichnisses}{%
\subparagraph{bb) Die Deutung des
Gleichnisses}\label{bb-die-deutung-des-gleichnisses}}

\bibleverse{8}Da erging das Wort des HERRN an mich folgendermaßen:
\bibleverse{9}»So spricht der HERR: Ebenso will ich den Hochmut Judas
und den Hochmut Jerusalems, der so gewaltig ist, ins Verderben stürzen!
\bibleverse{10}Dieses böse Volk, das sich weigert, auf meine Worte zu
hören, das da wandelt im Starrsinn seines Herzens und anderen Göttern
nachläuft, um ihnen zu dienen und sie anzubeten: es soll ihm ebenso
ergehen wie diesem Gürtel, der zu nichts mehr taugt. \bibleverse{11}Denn
gleichwie der Gürtel sich eng an die Hüften eines Mannes anschließt, so
hatte ich das ganze Haus Israel und das ganze Haus Juda sich eng an mich
anschließen lassen« -- so lautet der Ausspruch des HERRN --, »damit sie
mein Volk würden, mir zum Ruhm und zum Lobpreis und zur Zierde; aber sie
haben nicht gehorcht.«

\hypertarget{cc-das-gleichnis-von-den-gefuxfcllten-und-zerschlagenen-weinkruxfcgen-warnungsruf}{%
\subparagraph{cc) Das Gleichnis von den gefüllten und zerschlagenen
Weinkrügen;
Warnungsruf}\label{cc-das-gleichnis-von-den-gefuxfcllten-und-zerschlagenen-weinkruxfcgen-warnungsruf}}

\bibleverse{12}»So richte denn folgende Worte an sie: ›So hat der HERR,
der Gott Israels, gesprochen: Jeder Krug wird mit Wein gefüllt.‹ Wenn
sie dann zu dir sagen: ›Sollten wir das wirklich nicht selbst wissen,
daß jeder Krug mit Wein gefüllt wird?‹, \bibleverse{13}so erwidere
ihnen: ›So hat der HERR gesprochen: Fürwahr, ich will alle Bewohner
dieses Landes, sowohl die Könige, die auf dem Throne Davids sitzen, als
auch die Priester und Propheten und alle Bewohner Jerusalems mit
Trunkenheit füllen \bibleverse{14}und sie sich dann einen an dem andern
zerschmettern lassen, und zwar die Väter zugleich mit den Söhnen!‹ -- so
lautet der Ausspruch des HERRN --; ›ich will dabei keine Schonung üben
und kein Mitleid haben, und kein Erbarmen soll mich abhalten, sie zu
vernichten!‹«

\hypertarget{dd-warnung-vor-selbstsicherheit}{%
\subparagraph{dd) Warnung vor
Selbstsicherheit}\label{dd-warnung-vor-selbstsicherheit}}

\bibleverse{15}Höret und merkt auf! Seid nicht hochmütig, denn der HERR
ist's, der geredet hat! \bibleverse{16}Gebt dem HERRN, eurem Gott, die
Ehre, bevor es Nacht wird und bevor eure Füße sich an den Bergen in der
Dunkelheit stoßen und ihr dann auf Licht wartet, er es aber zu tiefster
Finsternis macht und es in Wolkendunkel verwandelt! \bibleverse{17}Wenn
ihr aber nicht gehorcht, so muß ich vor Kummer im Verborgenen weinen ob
eurem Hochmut, und mein Auge muß unaufhörlich in Tränen zerfließen, weil
die Herde des HERRN gefangen weggeführt wird.

\hypertarget{ee-drohrede-an-den-kuxf6nig-und-die-kuxf6nigin-mutter}{%
\subparagraph{ee) Drohrede an den König und die
Königin-Mutter}\label{ee-drohrede-an-den-kuxf6nig-und-die-kuxf6nigin-mutter}}

\bibleverse{18}Sage\textless sup title=``oder: sagt''\textgreater✲ zum
König und zur Herrin\textless sup title=``oder:
Königin-Mutter''\textgreater✲: »Setzt euch tief herunter, denn vom Haupt
ist euch eure herrliche Krone herabgefallen. \bibleverse{19}Die Städte
des Südlands sind verschlossen, und niemand ist da, der sie öffnet; in
Gefangenschaft wird Juda weggeführt insgesamt, weggeführt in voller
Zahl!«

\hypertarget{ff-klagelied-und-wehe-uxfcber-jerusalem}{%
\subparagraph{ff) Klagelied und Wehe über
Jerusalem}\label{ff-klagelied-und-wehe-uxfcber-jerusalem}}

\bibleverse{20}Hebe deine Augen auf, Jerusalem, und sieh, wie sie von
Norden daherkommen! Wo ist die Herde, die dir anvertraut war, deine
prächtigen Schafe? \bibleverse{21}Was wirst du sagen, wenn er die,
welche du selbst als vertraute Freunde an dich gewöhnt hast, zum
Oberhaupt\textless sup title=``=~zu Herren''\textgreater✲ über dich
bestellt? Werden dich da nicht Wehen erfassen wie ein Weib in
Geburtsnöten? \bibleverse{22}Und wenn du alsdann bei dir selber denkst:
»Warum hat solches Leid mich getroffen?«, so wisse: Wegen deiner
schweren Verschuldung wird dir die Schleppe aufgehoben, werden dir die
Füße mit Gewalt entblößt. \bibleverse{23}Kann wohl ein Mohr seine Haut
verwandeln und ein Pardel sein buntes Fell? Dann würdet auch ihr
imstande sein gut zu handeln, die ihr an Bösestun gewöhnt seid.
\bibleverse{24}»Darum will ich sie zerstreuen wie Spreu, die vor dem
Wüstenwinde verfliegt. \bibleverse{25}Das ist dein Los, dein Teil, das
ich dir zugemessen habe« -- so lautet der Ausspruch des HERRN --, »weil
du mich vergessen und dein Vertrauen auf Trug\textless sup title=``=~den
Truggötzen''\textgreater✲ gesetzt hast. \bibleverse{26}Darum will auch
ich dir deine Schleppen vorn bis über das Gesicht hochziehen, damit
deine Scham sichtbar wird. \bibleverse{27}Deine Ehebrecherei und dein
brünstiges Wiehern, die Schmach deiner Buhlerei -- auf den Hügeln wie im
freien Felde habe ich deine Greuel gesehen! Wehe dir, Jerusalem, daß du
dich nicht reinigst! Wie lange wird's noch währen?«

\hypertarget{ankuxfcndigung-schwerer-leiden-und-des-gerichts-uxfcber-juda-zurechtweisung-des-verzagenden-propheten-kap.-14-17}{%
\subsubsection{4. Ankündigung schwerer Leiden und des Gerichts über
Juda; Zurechtweisung des verzagenden Propheten (Kap.
14-17)}\label{ankuxfcndigung-schwerer-leiden-und-des-gerichts-uxfcber-juda-zurechtweisung-des-verzagenden-propheten-kap.-14-17}}

\hypertarget{a-schilderung-der-grouxdfen-duxfcrre}{%
\paragraph{a) Schilderung der großen
Dürre}\label{a-schilderung-der-grouxdfen-duxfcrre}}

\hypertarget{section-13}{%
\section{14}\label{section-13}}

\bibleverse{1}(Dies ist) das Wort des HERRN, das an Jeremia erging aus
Anlaß der großen Dürre: \bibleverse{2}»Juda trauert, und in seinen Toren
verschmachten (die Menschen), liegen im Trauergewand am Boden, und
Jerusalems Wehgeschrei steigt empor! \bibleverse{3}Die Vornehmen unter
ihnen schicken ihre Diener nach Wasser aus; aber wenn diese an die
Brunnen\textless sup title=``d.h. Zisternen''\textgreater✲ gekommen
sind, finden sie kein Wasser und kehren mit leeren Gefäßen heim:
enttäuscht und bestürzt sind sie und verhüllen ihr Haupt.
\bibleverse{4}Wegen des Erdreichs, das unbestellt daliegt, weil kein
Regen im Lande gefallen ist, sind die Ackerbauer (in ihrer Hoffnung)
getäuscht und verhüllen sich das Haupt. \bibleverse{5}Ja, selbst die
Hirschkuh auf dem Felde läßt ihr Junges, das sie eben geboren hat, im
Stich, weil sie nichts Grünes mehr findet; \bibleverse{6}und die
Wildesel stehen auf den kahlen Höhen, schnappen nach Luft wie die
Schakale; ihre Augen verschmachten✲, denn nirgends ist grünes Futter.«

\hypertarget{b-bitte-des-volkes}{%
\paragraph{b) Bitte des Volkes}\label{b-bitte-des-volkes}}

\bibleverse{7}»Wenn unsere Sünden uns anklagen, HERR, so handle du uns
zugut um deines Namens willen! Denn groß ist die Zahl unserer
Treubrüche, mit denen wir gegen dich gesündigt haben. \bibleverse{8}O du
Hoffnung Israels, du sein Retter zur Zeit der Not! Warum willst du
sein\textless sup title=``oder: bist du geworden''\textgreater✲ wie ein
Fremdling im Lande und wie ein Wanderer, der nur zum Übernachten Halt
macht? \bibleverse{9}Warum willst du sein wie ein verzagter Mann, wie
ein Held, der nicht zu helfen vermag? Du bist ja doch in unserer Mitte,
o HERR, und nach deinem Namen sind wir genannt: verlaß uns nicht!«

\hypertarget{c-gott-weist-die-fuxfcrbitte-des-propheten-zuruxfcck-und-bedroht-die-falschen-propheten-und-das-ganze-volk-mit-noch-gruxf6uxdferer-not}{%
\paragraph{c) Gott weist die Fürbitte des Propheten zurück und bedroht
die falschen Propheten und das ganze Volk mit noch größerer
Not}\label{c-gott-weist-die-fuxfcrbitte-des-propheten-zuruxfcck-und-bedroht-die-falschen-propheten-und-das-ganze-volk-mit-noch-gruxf6uxdferer-not}}

\bibleverse{10}So hat der HERR in bezug auf dieses Volk gesprochen: »So
hin und her zu schweifen, das lieben sie: ihre Füße haben sie nie
geschont; der HERR hat kein Wohlgefallen an ihnen gehabt; doch jetzt
gedenkt er ihrer Schuld und straft sie für ihre Sünden.«

\bibleverse{11}Weiter hat der HERR zu mir gesagt: »Du sollst keine
Fürbitte für dieses Volk einlegen, daß es ihm gut ergehen möge!
\bibleverse{12}Wenn sie fasten, so höre ich nicht auf ihr Flehen, und
wenn sie Brandopfer und Speisopfer darbringen, so nehme ich sie nicht
wohlgefällig an, vielmehr will ich sie durch Schwert und Hungersnot und
durch die Pest ausrotten!« \bibleverse{13}Da sagte ich: »Ach, HERR, mein
Gott, siehe, die Propheten sagen doch zu ihnen: ›Ihr werdet kein Schwert
zu sehen bekommen, und Hungersnot wird euch nicht treffen; nein,
dauerndes Heil will ich euch an dieser Stätte verleihen!‹«
\bibleverse{14}Da antwortete mir der HERR: »Lüge weissagen die Propheten
in meinem Namen; ich habe sie nicht gesandt und sie nicht entboten und
ihnen keinen Auftrag erteilt; erlogene Gesichte und Trugweissagung und
selbstersonnene Täuschung weissagen sie euch!« \bibleverse{15}Darum hat
der HERR so gesprochen: »Die Propheten, die in meinem Namen weissagen
und die, obgleich ich sie nicht gesandt habe, dennoch verkünden: ›Weder
Schwert✲ noch Hungersnot wird dieses Land treffen‹ -- durch das Schwert
und durch Hungersnot sollen diese Propheten den Tod finden!
\bibleverse{16}Das Volk aber, dem sie weissagen, wird auf den Straßen
Jerusalems vom Hunger und Schwert niedergestreckt daliegen, ohne daß
jemand sie zu Grabe trägt -- sie selbst und ihre Frauen, ihre Söhne und
ihre Töchter; so will ich die Strafe für ihre Bosheit über sie
ausgießen!«

\hypertarget{d-jeremia-beweint-die-grouxdfe-not-judas}{%
\paragraph{d) Jeremia beweint die große Not
Judas}\label{d-jeremia-beweint-die-grouxdfe-not-judas}}

\bibleverse{17}»Du aber richte dieses Wort an sie: ›Bei Tag und bei
Nacht zerfließen meine Augen in Tränen und kommen nicht zur Ruhe! Denn
einen furchtbaren Schlag hat die Jungfrau, die Tochter meines Volkes,
erlitten, eine qualvolle\textless sup title=``oder:
unheilbare''\textgreater✲ Wunde. \bibleverse{18}Gehe ich aufs Feld
hinaus, so erblicke ich dort vom Schwert Erschlagene, und komme ich in
die Stadt zurück, so gewahre ich die Qualen des Hungers! Ja, auch die
Propheten und auch die Priester müssen in ein Land ziehen, das sie nicht
kennen.‹«

\hypertarget{e-erneute-klage-und-dringende-bitte-des-propheten}{%
\paragraph{e) Erneute Klage und dringende Bitte des
Propheten}\label{e-erneute-klage-und-dringende-bitte-des-propheten}}

\bibleverse{19}Hast du denn Juda ganz verworfen? Oder bist du Zions im
Herzen überdrüssig geworden? Warum hast du uns so geschlagen, daß keine
Heilung für uns mehr vorhanden ist? Wir warten auf Rettung\textless sup
title=``oder: Genesung''\textgreater✲, aber es kommt nichts Gutes, und
auf die Stunde der Heilung, aber ach, da ist nichts als Schrecken!
\bibleverse{20}O HERR, wir erkennen unsere Gottlosigkeit, auch die
Schuld unserer Väter, daß wir gegen dich gesündigt haben.
\bibleverse{21}Verwirf (uns) nicht um deines Namens willen! Laß den
Thronsitz deiner Herrlichkeit nicht in Unehre fallen! Behalte im
Gedächtnis, brich nicht deinen Bund mit uns! \bibleverse{22}Gibt es etwa
unter den nichtigen Götzen der Heiden Regenspender? Oder schickt etwa
der Himmel von selbst die Regengüsse? Bist du es nicht, HERR, unser
Gott? So hoffen wir denn auf dich; denn du bist es, der dies alles
tut\textless sup title=``oder: getan hat''\textgreater✲.

\hypertarget{f-nochmalige-zuruxfcckweisung-einer-fuxfcrbitte-jeremias-und-neue-drohung-gottes}{%
\paragraph{f) Nochmalige Zurückweisung einer Fürbitte Jeremias und neue
Drohung
Gottes}\label{f-nochmalige-zuruxfcckweisung-einer-fuxfcrbitte-jeremias-und-neue-drohung-gottes}}

\hypertarget{section-14}{%
\section{15}\label{section-14}}

\bibleverse{1}Aber der HERR antwortete mir: »Wenn auch Mose und Samuel
vor mich träten, würde mein Herz sich doch diesem Volke nicht zuwenden:
schaffe sie mir aus den Augen, sie sollen weggehen! \bibleverse{2}Wenn
sie dich dann fragen: ›Wohin sollen wir gehen?‹, so antworte ihnen: ›So
hat der HERR gesprochen: Wer für den Tod\textless sup title=``d.h. die
Pest''\textgreater✲ bestimmt ist, gehe zum Tode! Wer für das Schwert
(bestimmt ist), zum Schwerte! Wer für den Hunger, zum Hunger! Und wer
für die Gefangenschaft, zur Gefangenschaft!‹ \bibleverse{3}Denn ich will
vier Arten (von Verderbern) gegen sie aufbieten« -- so lautet der
Ausspruch des HERRN --: »das Schwert zum Morden, die Hunde zum
Zerren\textless sup title=``oder: Fortschleppen''\textgreater✲, die
Vögel des Himmels und die Tiere des Feldes zum Fressen und zum Vertilgen
\bibleverse{4}und will sie so zu einem abschreckenden Beispiel für alle
Reiche der Erde machen um des judäischen Königs Manasse, des Sohnes
Hiskias, willen, wegen alles dessen, was er in Jerusalem verübt hat.«

\hypertarget{g-jeremias-klage-uxfcber-die-schweren-kriegsleiden-die-jerusalem-heimgesucht-haben-und-noch-heimsuchen-werden}{%
\paragraph{g) Jeremias Klage über die schweren Kriegsleiden, die
Jerusalem heimgesucht haben und noch heimsuchen
werden}\label{g-jeremias-klage-uxfcber-die-schweren-kriegsleiden-die-jerusalem-heimgesucht-haben-und-noch-heimsuchen-werden}}

\bibleverse{5}Ach, wer wird Mitleid mit dir haben, Jerusalem, und wer
dir Teilnahme bezeigen? Und wer wird bei dir einkehren, um sich nach
deinem Ergehen zu erkundigen? \bibleverse{6}»Du selbst hast mich ja
verworfen« -- so lautet der Ausspruch des HERRN --, »hast mir den Rücken
zugekehrt; darum habe ich meine Hand gegen dich erhoben und dich
zerschmettert: ich war's müde, mich erbitten zu lassen!
\bibleverse{7}Ich habe sie mit der Worfschaufel hinausgeworfelt in den
Toren des Landes, habe mein Volk seiner Kinder beraubt und es zugrunde
gerichtet: sie sind von ihren bösen Wegen doch nicht umgekehrt!
\bibleverse{8}Ihre Witwen sind mir zahlreicher geworden als der Sand am
Meer; ich habe ihnen über die Mütter der jungen Männer den Würgengel am
hellen Mittag gebracht, habe jählings Angst und Schrecken auf sie fallen
lassen. \bibleverse{9}Die Mutter, welche sieben Söhne geboren hat,
vergeht nun vor Trauer, haucht ihr Leben aus: die Sonne ist ihr noch bei
Tage untergegangen, fassungslos und gebrochen steht sie da! Was jetzt
aber von ihnen noch übrig ist, will ich dem Schwert preisgeben (auf der
Flucht) vor ihren Feinden!« -- so lautet der Ausspruch des HERRN.

\hypertarget{h-bittere-klage-jeremias-uxfcber-seinen-beruf-seine-zurechtweisung-durch-gott}{%
\paragraph{h) Bittere Klage Jeremias über seinen Beruf; seine
Zurechtweisung durch
Gott}\label{h-bittere-klage-jeremias-uxfcber-seinen-beruf-seine-zurechtweisung-durch-gott}}

\hypertarget{aa-jeremia-in-seiner-kraft-erschuxf6pft-und-an-gott-irre-geworden}{%
\subparagraph{aa) Jeremia in seiner Kraft erschöpft und an Gott irre
geworden}\label{aa-jeremia-in-seiner-kraft-erschuxf6pft-und-an-gott-irre-geworden}}

\bibleverse{10}Wehe mir, meine Mutter, daß du mich geboren hast, einen
Mann der Anfeindung und des Haders für das ganze Land\textless sup
title=``oder: für alle Welt''\textgreater✲! Ich habe kein Geld
ausgeliehen, und niemand hat mir geborgt, und doch fluchen sie mir alle!
\bibleverse{11}Der HERR hat verheißen: »Wahrlich, ich will dich zum
Guten stärken, wahrlich, ich will es so fügen, daß zur Zeit des
Unglücks, zur Zeit der Not, der Feind dich bittend angehen soll!«

\bibleverse{12}{[}Kann man Eisen zerbrechen, Eisen aus dem Norden, und
Erz? \bibleverse{13}Dein Vermögen und deine Schätze will ich der
Plünderung preisgeben ohne Entgelt, und zwar wegen aller Sünden, die du
in allen Teilen deines Landes begangen hast; \bibleverse{14}und ich will
dich deinen Feinden dienstbar machen in einem Lande, das du nicht
kennst; denn ein Feuer ist entbrannt durch meinen Zorn und lodert gegen
euch\textless sup title=``vgl. 17,3-4''\textgreater✲.{]}
\bibleverse{15}HERR, du selbst weißt es: gedenke mein und nimm dich
meiner an und schaffe mir Rache an meinen Widersachern! Laß mich nicht
infolge deiner Langmut (gegen sie) hinweggerafft werden! Bedenke, daß
ich um deinetwillen Schmach erdulde! \bibleverse{16}Sooft deine Befehle
erfolgten, habe ich sie meine Speise sein lassen, und deine Weisungen
sind mir eine Wonne und Herzensfreude gewesen; ich bin ja nach deinem
Namen genannt, HERR, du Gott der Heerscharen. \bibleverse{17}Ich habe
nie im Kreise der Scherzenden gesessen, um mich zu erlustigen; nein, von
deiner Hand gebeugt, habe ich einsam gesessen, weil du mich mit
Unwillen\textless sup title=``oder: Unmut''\textgreater✲ erfüllt
hattest. \bibleverse{18}Warum ist mein Schmerz endlos geworden und meine
Wunde tödlich, daß sie eine Heilung zurückweist? Willst du mir wirklich
wie ein trügerischer Bach sein, wie ein Wasserlauf, auf den kein Verlaß
ist?

\hypertarget{bb-des-propheten-zurechtweisung-und-wiederannahme-durch-gott}{%
\subparagraph{bb) Des Propheten Zurechtweisung und Wiederannahme durch
Gott}\label{bb-des-propheten-zurechtweisung-und-wiederannahme-durch-gott}}

\bibleverse{19}Darum hat der HERR so (zu mir) gesprochen: »Wenn du
umkehrst\textless sup title=``d.h. andern Sinnes wirst''\textgreater✲,
so will ich dich zurückkehren lassen, daß du mir aufs neue dienen
darfst; und wenn du nur Edles, nichts Gemeines hören läßt, sollst du
wieder wie mein Mund sein. Jene sollen sich dann zu dir umwenden, du
aber sollst dich nicht zu ihnen umwenden. \bibleverse{20}Dann will ich
dich diesem Volke gegenüber zu einer hochragenden Mauer von Erz machen,
daß, wenn sie gegen dich anstürmen, sie dir doch nichts anhaben können;
denn ich bin mit dir, um dir zu helfen und dir den Sieg zu verleihen« --
so lautet der Ausspruch des HERRN --; \bibleverse{21}»und ich will dich
aus der Hand der Bösen erretten und dich aus der Faust der Gewalttätigen
befreien!«

\hypertarget{i-jeremia-soll-den-untergang-des-volkes-und-den-ernst-der-zeit-durch-persuxf6nliche-entsagung-augenfuxe4llig-darstellen}{%
\paragraph{i) Jeremia soll den Untergang des Volkes und den Ernst der
Zeit durch persönliche Entsagung augenfällig
darstellen}\label{i-jeremia-soll-den-untergang-des-volkes-und-den-ernst-der-zeit-durch-persuxf6nliche-entsagung-augenfuxe4llig-darstellen}}

\hypertarget{aa-jeremia-soll-keine-familie-begruxfcnden}{%
\subparagraph{aa) Jeremia soll keine Familie
begründen}\label{aa-jeremia-soll-keine-familie-begruxfcnden}}

\hypertarget{section-15}{%
\section{16}\label{section-15}}

\bibleverse{1}Das Wort des HERRN erging dann an mich folgendermaßen:
\bibleverse{2}»Du sollst dir kein Weib nehmen und weder Söhne noch
Töchter an diesem Orte haben!« \bibleverse{3}Denn so hat der HERR
gesprochen in betreff der Söhne und Töchter, die an diesem Orte geboren
werden, und in betreff ihrer Mütter, die sie gebären, und in betreff
ihrer Väter, die sie in diesem Lande zeugen: \bibleverse{4}»An
qualvollen Todesarten sollen sie sterben, ohne betrauert und bestattet
zu werden! Zu Dünger auf offenem Felde sollen sie werden! Durch Schwert
und Hunger sollen sie ums Leben kommen, und ihre Leichen sollen den
Vögeln des Himmels und den Tieren des Feldes zum Fraß dienen!«

\hypertarget{bb-jeremia-soll-sich-von-leichenfeierlichkeiten-und-fruxf6hlichen-gelagen-fernhalten}{%
\subparagraph{bb) Jeremia soll sich von Leichenfeierlichkeiten und
fröhlichen Gelagen
fernhalten}\label{bb-jeremia-soll-sich-von-leichenfeierlichkeiten-und-fruxf6hlichen-gelagen-fernhalten}}

\bibleverse{5}Weiter gebot der HERR mir: »Du sollst in kein Trauerhaus
eintreten und zu keiner Totenklage hingehen und keinem von ihnen Beileid
bezeigen! Denn ich habe diesem Volke meine Freundschaft entzogen« -- so
lautet der Ausspruch des HERRN --, »die Liebe und das Erbarmen.
\bibleverse{6}So sollen sie denn in diesem Lande sterben, groß und
klein, ohne bestattet zu werden, und niemand wird um sie trauern noch
sich blutig ritzen oder sich ihretwegen kahl scheren. \bibleverse{7}Auch
wird man keinem das Trauerbrot brechen\textless sup title=``oder: die
Trauerspeise reichen''\textgreater✲, um ihn wegen eines Verstorbenen zu
trösten, und wird keinem den Trostbecher zu trinken geben wegen seines
Vaters oder seiner Mutter.~-- \bibleverse{8}Auch in ein Haus, wo man ein
Gastmahl abhält, sollst du nicht eintreten, um dich zum Schmausen und
Trinken mit ihnen niederzusetzen!« \bibleverse{9}Denn so hat der HERR
der Heerscharen, der Gott Israels, gesprochen: »Fürwahr, ich will an
diesem Orte vor euren Augen und in euren Tagen aller lauten Freude und
aller Fröhlichkeit, allem Jubel des Bräutigams und allem Brautgesang ein
Ende machen!«

\hypertarget{cc-begruxfcndung-dieser-heimsuchungen-und-ankuxfcndigung-der-wegfuxfchrung-des-volkes-in-die-gefangenschaft}{%
\subparagraph{cc) Begründung dieser Heimsuchungen und Ankündigung der
Wegführung des Volkes in die
Gefangenschaft}\label{cc-begruxfcndung-dieser-heimsuchungen-und-ankuxfcndigung-der-wegfuxfchrung-des-volkes-in-die-gefangenschaft}}

\bibleverse{10}»Wenn du nun diesem Volk alle diese Worte verkündigst und
sie dich dann fragen: ›Warum hat der HERR uns all dieses große Unheil
angedroht? Und worin besteht unsere Verschuldung und worin unsere Sünde,
die wir gegen den HERRN, unsern Gott, begangen haben?‹,
\bibleverse{11}so antworte ihnen: ›Darin, daß eure Väter mich verlassen
haben‹ -- so lautet der Ausspruch des HERRN -- ›und anderen Göttern
nachgelaufen sind und ihnen gedient und sie angebetet, mich aber
verlassen und mein Gesetz nicht beobachtet haben. \bibleverse{12}Und ihr
habt es noch ärger getrieben als eure Väter; ihr geht ja ein jeder dem
Starrsinn seines bösen Herzens nach, ohne auf mich zu hören!
\bibleverse{13}So will ich euch denn aus diesem Lande wegschleudern in
ein Land, das weder ihr noch eure Väter gekannt haben; dort sollt ihr
dann anderen Göttern Tag und Nacht dienen, weil ich für euch kein
Erbarmen mehr übrig habe!‹«

\hypertarget{k-eingeschobene-heilsweissagung}{%
\paragraph{k) Eingeschobene
Heilsweissagung}\label{k-eingeschobene-heilsweissagung}}

\bibleverse{14}»Darum\textless sup title=``oder: jedoch''\textgreater✲
wisset wohl: es kommt die Zeit« -- so lautet der Ausspruch des HERRN --,
»da wird man nicht mehr sagen: ›So wahr der HERR lebt, der die Kinder
Israel aus dem Lande Ägypten hergeführt hat!‹, \bibleverse{15}sondern:
›So wahr der HERR lebt, der die Kinder Israel hergeführt hat aus dem
Nordlande und aus all den Ländern, wohin er sie verstoßen hatte!‹ Denn
ich werde sie in ihr Land zurückbringen, das ich ihren Vätern gegeben
habe.«

\hypertarget{l-fischer-und-juxe4ger-von-gott-gesendet-werden-bald-das-volk-grausam-verfolgen}{%
\paragraph{l) Fischer und Jäger, von Gott gesendet, werden bald das Volk
grausam
verfolgen}\label{l-fischer-und-juxe4ger-von-gott-gesendet-werden-bald-das-volk-grausam-verfolgen}}

\bibleverse{16}»Wisset wohl: ich will zahlreiche Fischer entbieten« --
so lautet der Ausspruch des HERRN --, »die sollen sie wie Fische fangen;
und danach will ich zahlreiche Jäger entbieten, die sollen sie aufjagen
von jedem Berge hinweg und von jedem Hügel weg und aus den Felsenklüften
heraus; \bibleverse{17}denn meine Augen sind auf alle ihre Wege
gerichtet: sie bleiben mir nicht verborgen, und ihre Schuld ist vor
meinen Augen nicht verhüllt. \bibleverse{18}Zunächst also will ich ihnen
ihre Schuld und ihre Sünde zwiefach vergelten, weil sie mein Land durch
die Leichen ihrer scheußlichen Götzen entweiht und meinen Erbbesitz mit
ihren Greueln erfüllt haben.«

\hypertarget{m-die-heiden-erkennen-den-einen-gott}{%
\paragraph{m) Die Heiden erkennen den einen
Gott}\label{m-die-heiden-erkennen-den-einen-gott}}

\bibleverse{19}O HERR, du meine Stärke und meine Burg, meine Zuflucht in
der Zeit der Not! Zu dir werden die Heidenvölker von den Enden der Erde
her kommen und sagen: »Nichts als Trug haben unsere Väter zum Besitz
gehabt, nichtige Götzen, von denen keiner zu helfen vermag!
\bibleverse{20}Kann etwa ein Mensch sich Götter anfertigen? Das sind
doch keine Götter!«~-- \bibleverse{21}»Darum wisset wohl: diesmal will
ich sie zur Erkenntnis führen, will sie meine Hand und meine
Stärke\textless sup title=``oder: Macht''\textgreater✲ fühlen lassen:
dann werden sie erkennen, daß mein Name ist ›der HERR‹!«

\hypertarget{n-schuld-und-strafe-vertrauen-und-hoffnung}{%
\paragraph{n) Schuld und Strafe, Vertrauen und
Hoffnung}\label{n-schuld-und-strafe-vertrauen-und-hoffnung}}

\hypertarget{aa-judas-unverzeihliche-schuld-und-gottes-schwere-strafe}{%
\subparagraph{aa) Judas unverzeihliche Schuld und Gottes schwere
Strafe}\label{aa-judas-unverzeihliche-schuld-und-gottes-schwere-strafe}}

\hypertarget{section-16}{%
\section{17}\label{section-16}}

\bibleverse{1}Die Sünde Judas ist aufgeschrieben mit eisernem Griffel,
mit diamantener Spitze eingegraben in die Tafel ihres Herzens und an die
Hörner ihrer Altäre. \bibleverse{2}Wie ihrer Kinder, so gedenken sie
ihrer Altäre und ihrer Ascheren✲ bei den dichtbelaubten Bäumen, auf den
hohen Hügeln. \bibleverse{3}»Meinen Berg im Gefilde, deine Habe, alle
deine Schätze gebe ich der Plünderung preis, deine Höhen als Entgelt für
deine Versündigung in allen Teilen deines Gebiets. \bibleverse{4}Da mußt
du denn, und zwar durch eigene Schuld, deinen Erbbesitz fahren lassen,
den ich dir verliehen habe, und ich will dich deinen Feinden zum Knecht✲
machen in einem Lande, das du nicht kennst; denn ein Feuer habt ihr in
meiner Nase\textless sup title=``oder: durch meinen Zorn''\textgreater✲
angezündet, das bis in Ewigkeit brennen wird.«

\hypertarget{bb-falsches-menschenvertrauen-und-rechtes-gottvertrauen-und-ihre-fruxfcchte}{%
\subparagraph{bb) Falsches Menschenvertrauen und rechtes Gottvertrauen
und ihre
Früchte}\label{bb-falsches-menschenvertrauen-und-rechtes-gottvertrauen-und-ihre-fruxfcchte}}

\bibleverse{5}So hat der HERR gesprochen: »Verflucht ist der Mann, der
sich auf Menschen verläßt und Fleisch zu seinem Arm macht und dessen
Herz sich vom HERRN abkehrt! \bibleverse{6}Der gleicht einem kahlen
Baume\textless sup title=``oder: Wacholderstrauche''\textgreater✲ in der
Steppe und wird nicht erleben, daß Gutes kommt; nein, er muß in dürren
Wüstenstrichen wohnen, auf dem Salzboden der unwirtlichen Heide.
\bibleverse{7}Gesegnet aber ist der Mann, der sich auf den HERRN verläßt
und dessen Zuversicht der HERR ist! \bibleverse{8}Der gleicht einem
Baume, der am Wasser gepflanzt ist und seine Wurzeln nach dem Bache hin
ausstreckt; er hat nichts zu fürchten, wenn Hitze kommt, und sein Laub
bleibt grün; auch in dürren Jahren ist ihm nicht bange, und ohne
Aufhören trägt er Früchte.«

\hypertarget{cc-zwei-spruxfcche-der-lebenserfahrung-des-menschen-herz-und-die-unsicherheit-des-ungerechten-gewinnes}{%
\subparagraph{cc) Zwei Sprüche der Lebenserfahrung: des Menschen Herz
und die Unsicherheit des ungerechten
Gewinnes}\label{cc-zwei-spruxfcche-der-lebenserfahrung-des-menschen-herz-und-die-unsicherheit-des-ungerechten-gewinnes}}

\bibleverse{9}Arglistig ist das Herz, mehr als alles andere, und
verschlagen ist es: wer kann es ergründen? \bibleverse{10}»Ich, der
HERR, erforsche das Herz und prüfe die Nieren, und zwar um einem jeden
zu vergelten nach seinem Wandel, nach der Frucht seiner
Taten\textless sup title=``=~wie sein ganzes Tun es
verdient''\textgreater✲.«

\bibleverse{11}Wie ein Rebhuhn, das Eier bebrütet, die es nicht gelegt
hat, so ist ein Mensch, der Reichtum erwirbt, aber nicht auf rechtmäßige
Weise: in der Mitte seiner Lebenstage muß er ihn wieder fahren lassen,
und an seinem Ende steht er da als Narr.

\hypertarget{dd-israels-herrlicher-besitz}{%
\subparagraph{dd) Israels herrlicher
Besitz}\label{dd-israels-herrlicher-besitz}}

\bibleverse{12}O Thron der Herrlichkeit, hocherhaben von Anbeginn an, du
Stätte unsers Heiligtums! \bibleverse{13}O Hoffnung Israels, HERR! Alle,
die dich verlassen, werden zuschanden, und die von dir abfallen, deren
Namen werden auf die Erde\textless sup title=``=~in den Erdboden oder:
Staub''\textgreater✲ geschrieben; denn verlassen haben sie den
Brunnquell lebendigen Wassers, den HERRN.

\hypertarget{ee-jeremias-rachegebet-gegen-spuxf6tter-und-gegner}{%
\subparagraph{ee) Jeremias Rachegebet gegen Spötter und
Gegner}\label{ee-jeremias-rachegebet-gegen-spuxf6tter-und-gegner}}

\bibleverse{14}Heile mich, HERR, so werde ich heil, hilf mir, so ist mir
geholfen! Denn mein Lobpreis\textless sup title=``oder:
Ruhm''\textgreater✲ bist du. \bibleverse{15}Siehe, jene sagen zu mir:
»Wo bleibt denn das Drohwort des HERRN? Möge es doch eintreffen!«
\bibleverse{16}Ich aber habe mich nicht dem Hirtenamt in deinem Dienst
entzogen und habe den Tag des Unheils nicht herbeigewünscht: du weißt es
wohl! Was über meine Lippen gekommen ist, liegt offen vor deinen Augen.
\bibleverse{17}Mache mich nicht völlig hoffnungslos, du bist meine
Zuflucht am Tage des Unheils! \bibleverse{18}Laß meine Verfolger
zuschanden\textless sup title=``oder: enttäuscht''\textgreater✲ werden,
aber nicht mich! Laß sie verzagt dastehen, aber nicht mich! Bringe über
sie den Tag des Unheils und zerschmettere sie mit doppelter Vernichtung!

\hypertarget{o-einschuxe4rfung-der-sabbatheiligung-mit-entsprechenden-verheiuxdfungen-und-drohungen}{%
\paragraph{o) Einschärfung der Sabbatheiligung mit entsprechenden
Verheißungen und
Drohungen}\label{o-einschuxe4rfung-der-sabbatheiligung-mit-entsprechenden-verheiuxdfungen-und-drohungen}}

\bibleverse{19}So hat der HERR mir geboten: »Gehe hin und stelle dich
auf im Tor der Söhne des Volkes\textless sup title=``oder: der
Volksgenossen''\textgreater✲, durch das die Könige von Juda aus- und
einziehen, und in allen übrigen Toren Jerusalems \bibleverse{20}und sage
zu ihnen: Vernehmt das Wort des HERRN, ihr Könige von Juda und ihr
Judäer insgesamt und alle ihr Bewohner Jerusalems, die ihr durch diese
Tore eingeht! \bibleverse{21}So hat der HERR gesprochen: ›Hütet euch um
eures Lebens willen, am Sabbattage eine Last auf euch zu nehmen und sie
in die Tore Jerusalems hineinzubringen! \bibleverse{22}Tragt am
Sabbattage auch keine Last aus euren Häusern hinaus und verrichtet
überhaupt keinerlei Arbeit, sondern haltet den Sabbattag heilig, wie ich
euren Vätern geboten habe!‹ \bibleverse{23}Doch sie haben nicht gehorcht
und mir kein Gehör geschenkt, sondern haben sich halsstarrig gezeigt, so
daß sie nicht gehorsam gewesen sind und sich nicht haben warnen lassen.
\bibleverse{24}›Wenn ihr nun willig auf mich hört‹ -- so lautet der
Ausspruch des HERRN --, ›daß ihr am Sabbattage keine Last durch die Tore
dieser Stadt hereintragt, vielmehr den Sabbattag heilig haltet, indem
ihr keinerlei Arbeit an ihm verrichtet, \bibleverse{25}so werden Könige
{[}und Fürsten{]}, die auf dem Throne Davids sitzen, durch die Tore
dieser Stadt zu Wagen und zu Roß einziehen, sie samt ihren
Fürsten\textless sup title=``oder: Oberen''\textgreater✲, die Männer von
Juda samt den Bewohnern Jerusalems, und diese Stadt wird ewig bewohnt
bleiben. \bibleverse{26}Dazu werden aus den Ortschaften Judas und aus
der Umgegend von Jerusalem sowie aus dem Stamme Benjamin und aus der
Niederung, vom Bergland und aus dem Südgau Leute kommen, die Brand- und
Schlachtopfer, Speisopfer und Weihrauch darbringen und mit Dankopfern im
Tempel des HERRN erscheinen. \bibleverse{27}Wenn ihr aber nicht auf mich
hört, den Sabbattag heilig zu halten, so daß ihr am Sabbattage keinerlei
Last tragt, noch mit einer solchen durch die Tore Jerusalems eingeht, so
will ich Feuer an die Tore der Stadt legen, das soll die Paläste
Jerusalems verzehren und nicht erlöschen!‹«

\hypertarget{gerichtsankuxfcndigungen-und-unheilsdrohungen-jeremias-persuxf6nliche-leidenserfahrungen-kap.-18-26}{%
\subsubsection{5. Gerichtsankündigungen und Unheilsdrohungen; Jeremias
persönliche Leidenserfahrungen (Kap.
18-26)}\label{gerichtsankuxfcndigungen-und-unheilsdrohungen-jeremias-persuxf6nliche-leidenserfahrungen-kap.-18-26}}

\hypertarget{a-des-tuxf6pfers-arbeit-als-sinnbild-des-guxf6ttlichen-waltens-uxfcber-dem-geschick-der-vuxf6lker}{%
\paragraph{a) Des Töpfers Arbeit als Sinnbild des göttlichen Waltens
über dem Geschick der
Völker}\label{a-des-tuxf6pfers-arbeit-als-sinnbild-des-guxf6ttlichen-waltens-uxfcber-dem-geschick-der-vuxf6lker}}

\hypertarget{section-17}{%
\section{18}\label{section-17}}

\bibleverse{1}Das Wort, welches an Jeremia vom HERRN erging, lautete
folgendermaßen: \bibleverse{2}»Mache dich auf und gehe in das Haus des
Töpfers hinab, denn dort will ich dir meine Weisungen kundtun!«
\bibleverse{3}So ging ich denn in das Haus des Töpfers hinab und fand
ihn gerade mit einer Arbeit auf der Töpferscheibe beschäftigt;
\bibleverse{4}und wenn das Gefäß, das er anfertigte, mißriet, wie das
bei dem Ton unter der Hand des Töpfers vorkommt, so machte er wieder ein
anderes Gefäß daraus, wie es dem Töpfer eben gut schien.

\bibleverse{5}Da erging das Wort des HERRN an mich folgendermaßen:
\bibleverse{6}»Habe ich nicht das Recht, wie dieser Töpfer da mit euch
zu verfahren, ihr vom Hause Israel?« -- so lautet der Ausspruch des
HERRN. »Wisset wohl: wie der Ton in der Hand des Töpfers, ebenso seid
ihr in meiner Hand, ihr vom Hause Israel. \bibleverse{7}Einmal drohe ich
einem Volke oder einem Königshause, daß ich es ausrotten, vernichten und
vertilgen wolle; \bibleverse{8}wenn dann aber das betreffende Volk,
gegen das meine Drohung gerichtet war, sich von seiner Bosheit bekehrt,
so lasse ich mir das Unheil leid sein, das ich ihm zuzufügen beschlossen
hatte. \bibleverse{9}Und ein andermal verheiße ich einem Volke oder
einem Königshause, es aufbauen und einpflanzen zu wollen;
\bibleverse{10}wenn es dann aber tut, was mir mißfällt, indem es meinen
Weisungen nicht nachkommt, so lasse ich mir das Gute leid sein, das ich
ihm zu erweisen gedacht hatte. \bibleverse{11}Darum verkünde nun den
Männern von Juda und besonders den Bewohnern Jerusalems folgendes: ›So
hat der HERR gesprochen: Wisset wohl: ich habe Böses gegen euch im Sinn
und hege Unheilsgedanken gegen euch. Kehrt doch um, ein jeder von seinem
bösen Wege, und bessert euren Wandel und euer ganzes Tun!‹
\bibleverse{12}Doch sie werden antworten: ›Vergebliche Mühe! Nein, wir
wollen unsern eigenen Gedanken nachgehen und ein jeder nach dem
Starrsinn seines bösen Herzens handeln!‹«

\hypertarget{unbegreiflich-und-unnatuxfcrlich-ist-der-abfall-des-volkes}{%
\paragraph{Unbegreiflich und unnatürlich ist der Abfall des
Volkes}\label{unbegreiflich-und-unnatuxfcrlich-ist-der-abfall-des-volkes}}

\bibleverse{13}Darum hat der HERR so gesprochen: »Erkundigt euch doch
bei den Heidenvölkern, ob jemand so etwas jemals gehört habe! Etwas ganz
Abscheuliches hat die Jungfrau Israel verübt! \bibleverse{14}Verläßt
wohl jemals der Schnee des Libanons den Felsen im Gefild? Oder versiegen
jemals die weither kommenden Wasser, die kalten, rieselnden?
\bibleverse{15}Doch mich hat mein Volk vergessen: den nichtigen Götzen
opfern sie, und diese haben sie zu Fall gebracht auf ihren Wegen, den
Pfaden der Vorzeit, so daß sie wandeln auf den Steigen eines ungebahnten
Weges, \bibleverse{16}um so ihr Land zu einem abschreckenden Beispiel zu
machen, zu ewigem Hohn: jeder, der daran vorüberzieht, entsetzt sich und
schüttelt den Kopf. \bibleverse{17}Wie der Ostwind werde ich sie (auf
der Flucht) vor dem Feinde her zerstreuen, werde ihnen den Rücken, aber
nicht mein Angesicht zukehren am Tage ihres Untergangs!«

\hypertarget{b-feindselige-angriffe-der-unbuuxdffertigen-priester-und-propheten-auf-jeremia-rachegebet-des-propheten}{%
\paragraph{b) Feindselige Angriffe der unbußfertigen Priester und
Propheten auf Jeremia; Rachegebet des
Propheten}\label{b-feindselige-angriffe-der-unbuuxdffertigen-priester-und-propheten-auf-jeremia-rachegebet-des-propheten}}

\bibleverse{18}Sie haben gesagt: »Kommt, laßt uns Anschläge gegen
Jeremia ersinnen! Denn noch fehlt es den Priestern nicht an Belehrung
und den Weisen nicht an Rat, noch den Propheten an Wort\textless sup
title=``d.h. an geistlicher Beredsamkeit''\textgreater✲. Kommt, wir
wollen ihn mit (seiner eigenen) Zunge schlagen und alle seine Reden
unbeachtet lassen\textless sup title=``oder: belauern''\textgreater✲!«
\bibleverse{19}Gib doch acht auf mich, HERR, und vernimm die Worte
meiner Widersacher! \bibleverse{20}Soll denn Gutes mit Bösem vergolten
werden? Sie haben ja meinem Leben eine Grube gegraben! Denke daran, wie
ich vor dir gestanden habe, Fürbitte für sie einzulegen, um deinen Zorn
von ihnen abzuwenden! \bibleverse{21}Darum gib ihre Kinder dem Hunger
preis und überliefere sie der Gewalt des Schwertes! Laß ihre Weiber
kinderlos und zu Witwen werden, wenn ihre Männer von der Pest hingerafft
sind, und laß ihre Jünglinge vom Schwert erschlagen werden im Kriege!
\bibleverse{22}Wehgeschrei möge aus ihren Häusern erschallen, wenn du
unversehens Kriegerscharen über sie kommen läßt; denn sie haben eine
Grube gegraben, um mich zu fangen, und meinen Füßen haben sie heimlich
Schlingen gelegt. \bibleverse{23}Du aber, HERR, kennst alle ihre
Mordanschläge gegen mich: vergib ihnen ihre Missetat nicht und tilge
ihre Sünde nicht aus vor deinem Angesicht, sondern laß sie
niedergestürzt\textless sup title=``oder: verurteilt''\textgreater✲ vor
deinen Augen liegen! Zur Zeit deines Zorngerichts rechne mit ihnen ab!

\hypertarget{c-der-greuel-des-thopheth-die-vernichtung-judas-sinnbildlich-durch-die-zerschmetterung-eines-kruges-angekuxfcndigt}{%
\paragraph{c) Der Greuel des Thopheth; die Vernichtung Judas
sinnbildlich durch die Zerschmetterung eines Kruges
angekündigt}\label{c-der-greuel-des-thopheth-die-vernichtung-judas-sinnbildlich-durch-die-zerschmetterung-eines-kruges-angekuxfcndigt}}

\hypertarget{section-18}{%
\section{19}\label{section-18}}

\bibleverse{1}So hat der HERR zu mir gesprochen: »Gehe hin und kaufe dir
beim Töpfer einen Krug, nimm dann einige von den Ältesten\textless sup
title=``oder: Vornehmsten''\textgreater✲ des Volkes und von den
vornehmsten Priestern mit dir \bibleverse{2}und gehe in das Tal
Ben-Hinnom\textless sup title=``vgl. 2,23''\textgreater✲ hinaus, das vor
dem Eingang des Scherbentors liegt, und rufe dort laut die Worte aus,
die ich dir sagen werde! \bibleverse{3}So sprich zu ihnen: Vernehmt das
Wort des HERRN, ihr Könige von Juda und ihr Bewohner Jerusalems! So hat
der HERR der Heerscharen, der Gott Israels, gesprochen: ›Wisset wohl:
ich will Unheil über diesen Ort bringen, daß jedem, der davon hört, die
Ohren gellen sollen! \bibleverse{4}Zur Strafe dafür, daß sie mich
verlassen und diese Stätte entehrt und an ihr anderen Göttern geopfert
haben, von denen weder sie noch ihre Väter noch die Könige von Juda
etwas gewußt haben, und weil sie diese Stätte mit dem Blute Unschuldiger
erfüllt \bibleverse{5}und die Baalshöhen erbaut haben, um ihre Kinder
als Brandopfer für den Baal zu verbrennen, was ich nie geboten noch
angeordnet habe und was mir nie in den Sinn gekommen ist:~--
\bibleverse{6}darum wisset wohl: die Zeit kommt‹ -- so lautet der
Ausspruch des HERRN --, ›da wird dieser Ort nicht mehr Thopheth und
dieses Tal nicht mehr Tal Ben-Hinnom, sondern Würgetal✲ genannt werden.
\bibleverse{7}Da will ich dann die Klugheit Judas und Jerusalems an
diesem Orte ausschütten\textless sup title=``d.h. zunichte
machen''\textgreater✲ und will sie fallen lassen durch das Schwert (auf
der Flucht) vor ihren Feinden her und durch die Hand derer, die ihnen
nach dem Leben trachten\textless sup title=``d.h. durch die Hand ihrer
Todfeinde''\textgreater✲, und will ihre Leichen den Vögeln des Himmels
und den Tieren des Feldes zum Fraß geben. \bibleverse{8}Diese Stadt aber
will ich zum abschreckenden Beispiel und zum Gespött machen, so daß
jeder, der an ihr vorübergeht, sich entsetzen und wegen aller ihrer
Leiden zischen soll. \bibleverse{9}Auch will ich sie das Fleisch ihrer
Söhne und das Fleisch ihrer Töchter essen lassen, und sie sollen einer
das Fleisch des andern verzehren infolge der Belagerung\textless sup
title=``oder: Bedrängnis''\textgreater✲ und Not, in die sie von ihren
Feinden und von denen, die ihnen nach dem Leben trachten, versetzt
werden.‹ \bibleverse{10}Hierauf sollst du den Krug vor den Augen der
Männer, die mit dir gegangen sind, zerschlagen \bibleverse{11}und zu
ihnen sagen: ›So hat der HERR der Heerscharen gesprochen: Ebenso werde
ich dieses Volk und diese Stadt zerschmettern, wie man Töpfergeschirr
zerschmettert, das dann nicht wiederhergestellt werden kann; und im
Thopheth wird man begraben, weil sonst kein Platz mehr zum Begraben
vorhanden ist. \bibleverse{12}Auf diese Weise will ich mit diesem Ort
verfahren‹ -- so lautet der Ausspruch des HERRN -- ›und mit seinen
Bewohnern: ein Thopheth will ich aus dieser Stadt machen.
\bibleverse{13}Da sollen dann die Häuser Jerusalems und die Paläste der
Könige von Juda ebenso unrein werden wie die Stätte des Thopheth:
nämlich alle die Häuser, auf deren Dächern sie dem gesamten Sternenheer
des Himmels geräuchert und fremden Göttern Trankspenden ausgegossen
haben.‹«

\hypertarget{d-jeremias-rede-vor-dem-volke-im-tempelhofe-seine-miuxdfhandlung-durch-den-tempelobersten-pashur}{%
\paragraph{d) Jeremias Rede vor dem Volke im Tempelhofe; seine
Mißhandlung durch den Tempelobersten
Pashur}\label{d-jeremias-rede-vor-dem-volke-im-tempelhofe-seine-miuxdfhandlung-durch-den-tempelobersten-pashur}}

\bibleverse{14}Als Jeremia dann vom Thopheth, wohin der HERR ihn zur
Verkündigung des Prophetenspruchs gesandt hatte, zurückkehrte, trat er
in den Vorhof des Tempels des HERRN und sprach zum ganzen Volk:
\bibleverse{15}»So hat der HERR der Heerscharen, der Gott Israels,
gesprochen: ›Wisset wohl: ich will über diese Stadt und über sämtliche
Ortschaften, die zu ihr gehören, all das Unheil kommen lassen, das ich
ihr angedroht habe! denn sie haben sich halsstarrig gezeigt, um auf
meine Worte nicht zu hören.‹«

\hypertarget{section-19}{%
\section{20}\label{section-19}}

\bibleverse{1}Als aber der Priester Pashur, der Sohn Immers, der
damalige Oberaufseher im Tempel des HERRN, Jeremia diese Weissagung
aussprechen hörte, \bibleverse{2}ließ er den Propheten Jeremia stäupen
und ihn in den Block legen\textless sup title=``oder:
schließen''\textgreater✲, der sich im Benjaminstor, dem oberen Tor am
Tempel des HERRN, befand. \bibleverse{3}Am folgenden Morgen aber, als
Pashur den Jeremia aus dem Block freigelassen hatte, sagte Jeremia zu
ihm: »Nicht Pashur nennt der HERR hinfort deinen Namen, sondern ›Grauen
ringsum‹! \bibleverse{4}Denn so hat der HERR gesprochen: ›Wisse wohl:
ich mache dich zum Grauen für dich selbst und für alle deine Freunde!
Denn sie werden durch das Schwert ihrer Feinde fallen, und zwar so, daß
du es mit eigenen Augen ansiehst; und ganz Juda will ich der Gewalt des
Königs von Babylon preisgeben, damit er sie gefangen nach Babylon
wegführt und sie mit dem Schwert erschlägt. \bibleverse{5}Dazu will ich
den ganzen Reichtum dieser Stadt, ihren gesamten Besitz samt allen ihren
Kostbarkeiten, dahingeben, auch alle Schätze der Könige von Juda ihren
Feinden zu eigen geben: die sollen sie plündern und mitnehmen und nach
Babylon bringen. \bibleverse{6}Du aber, Pashur, und alle deine
Hausgenossen -- ihr sollt in die Gefangenschaft wandern, und nach
Babylon sollst du kommen und dort sterben und dort auch begraben werden,
du und alle deine Freunde, denen du falsch geweissagt hast!‹«

\hypertarget{e-des-propheten-bittere-klage-uxfcber-die-leiden-seines-berufs-seine-inneren-kuxe4mpfe-und-sein-trost}{%
\paragraph{e) Des Propheten bittere Klage über die Leiden seines Berufs;
seine inneren Kämpfe und sein
Trost}\label{e-des-propheten-bittere-klage-uxfcber-die-leiden-seines-berufs-seine-inneren-kuxe4mpfe-und-sein-trost}}

\bibleverse{7}Du hast mich betört\textless sup title=``oder:
verlockt''\textgreater✲, HERR, und ich habe mich betören\textless sup
title=``oder: verlocken''\textgreater✲ lassen; du hast mich überwältigt
und bist Sieger geblieben! Zum Gelächter bin ich geworden tagaus tagein,
alle Welt verhöhnt mich! \bibleverse{8}Ach, sooft ich rede, muß ich
aufschreien, muß ich »Unrecht und Vergewaltigung!« rufen; denn das Wort
des HERRN hat mir Hohn und Schmach eingebracht den ganzen Tag!
\bibleverse{9}Doch wenn ich mir vornehme: »Ich will seiner nicht mehr
gedenken und in seinem Namen nicht mehr reden«, so ist es mir im Innern,
als wäre ein loderndes Feuer in meinen Gebeinen eingeschlossen; und mühe
ich mich ab, es auszuhalten, so vermag ich es nicht! \bibleverse{10}Ach,
ich habe viele schon flüstern hören -- Grauen ringsum! --: »Zeigt ihn
an!« und »Wir wollen ihn anzeigen!«\textless sup title=``vgl.
18,18''\textgreater✲ Alle, die zu meiner Freundschaft gehören, lauern
auf einen Fehltritt von mir: »Vielleicht läßt er sich betören, daß wir
ihn in der Gewalt haben und Rache an ihm nehmen können!«

\bibleverse{11}Aber der HERR steht mir bei wie\textless sup
title=``oder: als''\textgreater✲ ein gewaltiger Held; darum werden meine
Verfolger zu Fall kommen und nichts ausrichten; sie werden sich ganz
enttäuscht sehen, weil es ihnen nicht gelungen ist: eine ewige Schmach,
die unvergeßlich bleiben wird! \bibleverse{12}Und nun, HERR der
Heerscharen, der du den Gerechten prüfst, Nieren und Herz ansiehst: laß
mich deine Rache an ihnen sehen, denn dir habe ich meine Sache
anheimgestellt!\textless sup title=``vgl. 11,20''\textgreater✲~--
\bibleverse{13}Singet dem HERRN, preiset den HERRN, denn er errettet das
Leben des Armen aus der Hand der Übeltäter!

\hypertarget{jeremia-verflucht-sein-leben}{%
\paragraph{Jeremia verflucht sein
Leben}\label{jeremia-verflucht-sein-leben}}

\bibleverse{14}Verflucht sei der Tag, an dem ich geboren ward! Der Tag,
an dem meine Mutter mich geboren hat, bleibe ungesegnet!
\bibleverse{15}Verflucht sei der Mann, der meinem Vater die frohe
Botschaft brachte: »Ein Kind, ein Sohn ist dir geboren!« und ihn dadurch
hoch erfreute\textless sup title=``oder: ihm Glück dazu
wünschte''\textgreater✲! \bibleverse{16}Diesem Manne möge es ergehen wie
den Städten, die der HERR erbarmungslos zerstört hat: er höre
Wehgeschrei am Morgen und Kriegslärm zur Mittagszeit!
\bibleverse{17}Warum hat er\textless sup title=``d.h.
Gott''\textgreater✲ mich nicht schon im Mutterschoß sterben lassen, so
daß meine Mutter mein Grab geworden wäre und ihr Schoß mich immerfort
getragen hätte! \bibleverse{18}Warum nur bin ich aus dem Mutterschoß zur
Welt gekommen? Doch nur, um Mühsal und Herzeleid zu erleben und damit
meine Tage in Schande vergingen!

\hypertarget{f-ungluxfccksankuxfcndigungen-an-die-stadt-jerusalem-an-die-herrscher-und-das-volk}{%
\paragraph{f) Unglücksankündigungen an die Stadt Jerusalem, an die
Herrscher und das
Volk}\label{f-ungluxfccksankuxfcndigungen-an-die-stadt-jerusalem-an-die-herrscher-und-das-volk}}

\hypertarget{aa-unheilvolle-antwort-jeremias-an-zedekias-gesandte-und-mahnungen-an-das-volk-wuxe4hrend-der-belagerung-jerusalems}{%
\subparagraph{aa) Unheilvolle Antwort Jeremias an Zedekias Gesandte und
Mahnungen an das Volk während der Belagerung
Jerusalems}\label{aa-unheilvolle-antwort-jeremias-an-zedekias-gesandte-und-mahnungen-an-das-volk-wuxe4hrend-der-belagerung-jerusalems}}

\hypertarget{section-20}{%
\section{21}\label{section-20}}

\bibleverse{1}(Dies ist) das Wort, das vom HERRN an Jeremia erging, als
der König Zedekia Pashur, den Sohn Malkijas, und den Priester Zephanja,
den Sohn Maasejas, zu ihm gesandt hatte mit dem Auftrage:
\bibleverse{2}»Befrage doch den HERRN für uns! Denn Nebukadnezar, der
König von Babylon, führt Krieg mit uns\textless sup title=``=~belagert
uns''\textgreater✲; vielleicht tut der HERR wie schon oft ein Wunder an
uns, daß jener von uns abziehen muß!«

\bibleverse{3}Da sagte Jeremia zu ihnen: »Bringt dem Zedekia folgende
Antwort: \bibleverse{4}So hat der HERR, der Gott Israels, gesprochen:
›Wisset wohl: die Kriegswaffen in eurer Hand, mit denen ihr bisher gegen
den König von Babylon und gegen die Chaldäer, die euch belagern,
außerhalb der Stadtmauer gekämpft habt, die will ich umwenden und sie
ins Innere dieser Stadt zuhauf hineinbringen; \bibleverse{5}und ich will
selbst gegen euch kämpfen mit hocherhobener Hand und starkem Arm, mit
Zorn und Grimm und voller Wut; \bibleverse{6}und ich will die Bewohner
dieser Stadt niederschlagen, sowohl Menschen als Vieh: an einer
verheerenden Seuche sollen sie sterben! \bibleverse{7}Hierauf aber‹ --
so lautet der Ausspruch des HERRN -- ›will ich Zedekia, den König von
Juda, samt seinen Dienern und dem Volke, soweit sie in dieser Stadt von
der Seuche, vom Schwert und vom Hunger verschont geblieben sind, in die
Hand Nebukadnezars, des Königs von Babylon, und in die Hand ihrer Feinde
und in die Hand derer, die ihnen nach dem Leben trachten, fallen lassen,
damit er sie mit der Schärfe des Schwertes niederhaue, ohne Mitleid mit
ihnen zu haben und ohne Schonung und Erbarmen zu üben!‹«

\hypertarget{bb-rat-an-das-volk}{%
\subparagraph{bb) Rat an das Volk}\label{bb-rat-an-das-volk}}

\bibleverse{8}»Zu dem Volke hier aber sollst du sagen: ›So hat der HERR
gesprochen: Wisset wohl: ich lasse euch die Wahl zwischen dem Wege, der
zum Leben führt, und dem Wege zum Tode: \bibleverse{9}Wer hier in der
Stadt bleibt, der wird durch das Schwert, durch den Hunger oder durch
die Pest ums Leben kommen; wer aber hinausgeht und sich den Chaldäern
ergibt, die euch belagern, der wird erhalten bleiben und sein Leben in
Sicherheit bringen. \bibleverse{10}Denn ich habe mein Angesicht gegen
diese Stadt gerichtet zum Unheil und nicht zum Segen‹ -- so lautet der
Ausspruch des HERRN --: ›sie soll in die Hand des Königs von Babylon
gegeben werden, damit er sie in Flammen aufgehen läßt!‹«

\hypertarget{cc-mahnwort-an-das-kuxf6nigshaus}{%
\subparagraph{cc) Mahnwort an das
Königshaus}\label{cc-mahnwort-an-das-kuxf6nigshaus}}

\bibleverse{11}»Sodann sollst du zum Hause des Königs von Juda sagen:
›Vernehmt das Wort des HERRN, ihr vom Hause Davids! \bibleverse{12}So
hat der HERR gesprochen: Haltet an jedem Morgen gerechtes Gericht und
rettet den Bedrückten aus der Hand des Gewalttätigen, damit mein Zorn
nicht wie Feuer hervorbricht und unauslöschlich brennt infolge der
Bosheit eurer Taten!‹«

\hypertarget{dd-gerichtsankuxfcndigung-fuxfcr-die-stadt-jerusalem}{%
\subparagraph{dd) Gerichtsankündigung für die Stadt
Jerusalem}\label{dd-gerichtsankuxfcndigung-fuxfcr-die-stadt-jerusalem}}

\bibleverse{13}»Wisse wohl: ich will an dich\textless sup title=``d.h.
gegen dich vorgehen''\textgreater✲, Bewohnerin des Tals, du Fels in der
Ebene!« -- so lautet der Ausspruch des HERRN. »Ihr sagt: ›Wer sollte
über uns herfallen und wer in unsere Wohnungen eindringen?‹
\bibleverse{14}Nun, ich will euch heimsuchen, wie eure Taten es
verdienen« -- so lautet der Ausspruch des HERRN --, »und will ein Feuer
in ihrem Walde entfachen, das ihre ganze Umgebung verzehren soll!«

\hypertarget{ee-warnung-an-das-kuxf6nigshaus-von-juda-ankuxfcndigung-des-gerichts-uxfcber-die-kuxf6nige-sallum-joahas-und-jojakim}{%
\subparagraph{ee) Warnung an das Königshaus von Juda; Ankündigung des
Gerichts über die Könige Sallum (=~Joahas) und
Jojakim}\label{ee-warnung-an-das-kuxf6nigshaus-von-juda-ankuxfcndigung-des-gerichts-uxfcber-die-kuxf6nige-sallum-joahas-und-jojakim}}

\hypertarget{section-21}{%
\section{22}\label{section-21}}

\bibleverse{1}So hat der HERR zu mir gesprochen: »Gehe zum Palast des
Königs von Juda hinab und richte dort folgende Botschaft aus:
\bibleverse{2}›Vernimm das Wort des HERRN, König von Juda, der du auf
dem Throne Davids sitzest, du samt deinen Dienern und deinen Untertanen,
die ihr durch diese Tore (der Königsburg) eingeht! \bibleverse{3}So hat
der HERR gesprochen: Laßt Recht und Gerechtigkeit walten, rettet den
Beraubten aus der Hand des Gewalttätigen, bedrückt und vergewaltigt
keinen Fremdling, keine Waise und Witwe und vergießt kein unschuldiges
Blut an diesem Ort! \bibleverse{4}Denn nur, wenn ihr dieses Gebot
wirklich befolgt, werden durch die Tore dieses Palastes Könige
einziehen, die auf Davids Thron sitzen und auf Wagen einherfahren und
auf Rossen reiten, er selbst und seine Diener und seine Untertanen.
\bibleverse{5}Wenn ihr aber diesen Weisungen nicht nachkommt, so habe
ich bei mir selbst geschworen‹ -- so lautet der Ausspruch des HERRN --,
›daß dieser Palast zu einer Trümmerstätte werden soll!‹«

\hypertarget{ff-fluch-uxfcber-den-kuxf6nigspalast}{%
\subparagraph{ff) Fluch über den
Königspalast}\label{ff-fluch-uxfcber-den-kuxf6nigspalast}}

\bibleverse{6}Denn so hat der HERR in betreff des Palastes des Königs
von Juda gesprochen: »Ein Gilead bist du mir, der Gipfel des Libanons;
doch wahrlich, ich will dich zur Wüste machen, zu unbewohnten Städten,
\bibleverse{7}und ich will Verwüster gegen dich in Dienst nehmen, einen
jeden mit seinen Gerätschaften\textless sup title=``d.h.
Beilen''\textgreater✲: die sollen deine prachtvollen Zedern umhauen und
ins Feuer werfen! \bibleverse{8}Wenn dann viele Völkerschaften an dieser
Stadt vorüberziehen und einer den andern fragt: ›Warum ist doch der HERR
mit dieser großen Stadt so schlimm verfahren?‹, \bibleverse{9}so wird
man antworten: ›Zur Strafe dafür, daß sie dem Bunde mit dem HERRN, ihrem
Gott, untreu geworden sind und andere Götter angebetet und ihnen gedient
haben.‹«

\hypertarget{gg-die-kuxf6nigsspruxfcche-2210-238.-zunuxe4chst-ein-wort-des-beileids-fuxfcr-den-ungluxfccklichen-sallum-joahas}{%
\subparagraph{gg) Die Königssprüche (22,10-23,8). Zunächst ein Wort des
Beileids für den unglücklichen Sallum
(=~Joahas)}\label{gg-die-kuxf6nigsspruxfcche-2210-238.-zunuxe4chst-ein-wort-des-beileids-fuxfcr-den-ungluxfccklichen-sallum-joahas}}

\bibleverse{10}Weint nicht um den Toten und klagt nicht um ihn! Weint
vielmehr um den, der weggezogen ist! Denn er kehrt nie wieder zurück und
sieht das Land seiner Geburt nicht wieder. \bibleverse{11}Denn der HERR
hat über Sallum, den Sohn Josias, des Königs von Juda, der seinem Vater
Josia in der Regierung gefolgt war und aus diesem Ort weggezogen ist,
folgendermaßen gesprochen: »Er wird nicht wieder hierher zurückkehren.
\bibleverse{12}Nein, an dem Orte, wohin man ihn in die Gefangenschaft
geführt hat, dort wird er sterben und dieses Land nicht wiedersehen.«

\hypertarget{hh-schwere-anklage-und-strafandrohung-gegen-den-kuxf6nig-jojakim}{%
\subparagraph{hh) Schwere Anklage und Strafandrohung gegen den König
Jojakim}\label{hh-schwere-anklage-und-strafandrohung-gegen-den-kuxf6nig-jojakim}}

\bibleverse{13}Wehe dem, der sein Haus mit Ungerechtigkeit baut und
seine Obergemächer mit Unrecht! Der seinen Nächsten ohne Entgelt
arbeiten läßt und ihm seinen Lohn vorenthält! \bibleverse{14}Der da
ausspricht: »Ich will mir ein geräumiges Haus bauen mit
weitgedehnten\textless sup title=``oder: luftigen''\textgreater✲
Gemächern, von Fenstern durchbrochen und mit Zedernholz getäfelt und mit
Zinnober rot gestrichen!« \bibleverse{15}Siehst du dein Königtum darin
bestehen, daß du dich für Zedernholz begeisterst? Dein Vater hat ja auch
gegessen und getrunken, aber er hat Recht und Gerechtigkeit geübt: da
erging es ihm gut; \bibleverse{16}er hat den Armen und Elenden zu ihrem
Recht verholfen: da stand alles gut. »Heißt nicht das mich recht
erkennen?« -- so lautet der Ausspruch des HERRN. \bibleverse{17}Dagegen
deine Augen und dein Herz sind nur auf Gewinn für dich gerichtet und auf
das Blut Unschuldiger, um es zu vergießen, und auf Bedrückung und
Erpressung, um sie zu verüben.

\bibleverse{18}Darum hat der HERR über Jojakim, den Sohn Josias, den
König von Juda, also gesprochen: »Man wird nicht um ihn klagen: ›Ach,
mein Bruder!‹ und ›Ach, seine Bruderschaft!‹, man wird nicht um ihn
klagen: ›Ach, Gebieter!‹ und ›Ach, seine Hoheit✲!‹ \bibleverse{19}Nein,
eines Esels Bestattung wird man an ihm vollziehen, wird ihn
hinausschleifen und hinwerfen weit außerhalb der Tore Jerusalems!«

\hypertarget{ii-unheilsverkuxfcndigung-fuxfcr-jerusalem-und-seinen-kuxf6nig-jojachin}{%
\subparagraph{ii) Unheilsverkündigung für Jerusalem und seinen König
Jojachin}\label{ii-unheilsverkuxfcndigung-fuxfcr-jerusalem-und-seinen-kuxf6nig-jojachin}}

\bibleverse{20}»Steige (Volk von Jerusalem) auf den Libanon und schreie
laut! Laß deinen Klageruf in Basan erschallen und schreie vom Berge
Abarim\textless sup title=``4.Mose 27,12; 5.Mose 32,49''\textgreater✲
herab! Denn zerschmettert sind alle deine Liebhaber. \bibleverse{21}Ich
habe zu dir geredet zur Zeit deines ungetrübten Glücks, doch du
antwortetest: ›Ich will nichts davon hören!‹ Das war so deine Art von
Jugend auf, du wolltest auf mich nicht hören. \bibleverse{22}Alle deine
Hirten\textless sup title=``d.h. Häupter und Führer''\textgreater✲ wird
nun der Sturmwind auf die Weide führen, und deine Liebhaber\textless sup
title=``oder: Lieblinge''\textgreater✲ müssen in die Gefangenschaft
wandern! Ja, alsdann wirst du beschämt und enttäuscht dastehen ob all
deiner Bosheit! \bibleverse{23}Die du jetzt auf dem Libanon thronst, in
Zedern eingenistet: wie wirst du stöhnen, wenn Wehen dich überfallen,
Krämpfe wie ein Weib in Kindesnöten!«

\hypertarget{jj-drei-ausspruxfcche-uxfcber-den-kuxf6nig-konja-jojachin}{%
\subparagraph{jj) Drei Aussprüche über den König Konja
(=~Jojachin)}\label{jj-drei-ausspruxfcche-uxfcber-den-kuxf6nig-konja-jojachin}}

\bibleverse{24}»So wahr ich lebe!« -- so lautet der Ausspruch des HERRN
--: »wäre auch Konja, der Sohn Jojakims, der König von Juda, ein
Siegelring an meiner rechten Hand, so wollte ich dich doch von da
abreißen \bibleverse{25}und dich in die Hand derer geben, die dir nach
dem Leben trachten, und in die Hand derer, vor denen dir graut, und zwar
in die Hand Nebukadnezars, des Königs von Babylon, und in die Hand der
Chaldäer. \bibleverse{26}Und ich will dich samt deiner Mutter, die dich
geboren hat, in ein fremdes Land schleudern, in dem ihr nicht geboren
seid\textless sup title=``=~das nicht eure Heimat ist''\textgreater✲,
und dort werdet ihr sterben! \bibleverse{27}In das Land aber, in das sie
sich sehnen zurückzukehren, dahin werden sie nie zurückkehren!«~--

\bibleverse{28}Ist denn dieser Mann Konja ein verächtliches,
zerschlagenes Gefäß oder ein Gerät, an dem niemand Gefallen findet?
Warum sind sie weggeschleudert worden, er samt seinen Kindern, und in
ein Land geworfen, das sie nicht kannten?~--

\bibleverse{29}O Land, Land, Land, vernimm das Wort des HERRN!
\bibleverse{30}So hat der HERR gesprochen: »Schreibt diesen Mann in die
Listen als kinderlos ein, als einen Mann, der in seinen Lebenstagen kein
Gelingen\textless sup title=``=~kein Glück''\textgreater✲ haben wird!
Denn keinem von seinen Nachkommen wird es gelingen, auf Davids Thron zu
sitzen und noch einmal über Juda zu herrschen!«

\hypertarget{kk-weheruf-uxfcber-die-untreuen-hirten-und-verheiuxdfung-des-wahren-hirten-aus-dem-hause-davids}{%
\subparagraph{kk) Weheruf über die untreuen Hirten und Verheißung des
wahren Hirten aus dem Hause
Davids}\label{kk-weheruf-uxfcber-die-untreuen-hirten-und-verheiuxdfung-des-wahren-hirten-aus-dem-hause-davids}}

\hypertarget{section-22}{%
\section{23}\label{section-22}}

\bibleverse{1}»Wehe den Hirten, welche die Schafe meiner Weide zugrunde
richten und sich zerstreuen lassen!« -- so lautet der Ausspruch des
HERRN. \bibleverse{2}Darum hat der HERR, der Gott Israels, in betreff
der Hirten, die mein Volk weiden, so gesprochen: »Ihr seid es, die meine
Schafe zerstreut und versprengt und nicht acht auf sie gegeben haben;
darum will ich euch jetzt wegen der Bosheit eures ganzen Tuns zur
Rechenschaft ziehen!« -- so lautet der Ausspruch des HERRN.
\bibleverse{3}»Ich selbst will aber auch den Überrest meiner Herde aus
all den Ländern, wohin ich sie versprengt habe, sammeln und sie auf ihre
Trift zurückführen: da werden sie fruchtbar sein und gedeihen.
\bibleverse{4}Dann will ich Hirten über sie erstehen lassen, die sie
weiden sollen, daß sie sich nicht weiterhin zu fürchten brauchen und
nicht erschrecken müssen und daß keines vermißt wird!« -- so lautet der
Ausspruch des HERRN.

\hypertarget{ll-verheiuxdfung-des-davidssprossen}{%
\subparagraph{ll) Verheißung des
Davidssprossen}\label{ll-verheiuxdfung-des-davidssprossen}}

\bibleverse{5}»Wisset wohl: es kommt die Zeit« -- so lautet der
Ausspruch des HERRN --, »da will ich dem David einen rechten Sproß
erwecken: der wird als König herrschen und mit Weisheit handeln und
Recht und Gerechtigkeit im Lande walten lassen! \bibleverse{6}In seinen
Tagen wird Juda gerettet werden\textless sup title=``=~Glück
erleben''\textgreater✲ und Israel in Sicherheit wohnen, und der Name,
den man ihm beilegt, wird lauten: ›Der HERR unsere
Gerechtigkeit‹\textless sup title=``=~Hort des Heils''\textgreater✲.
\bibleverse{7}Darum wisset wohl: es kommt die Zeit« -- so lautet der
Ausspruch des HERRN --, »da wird man nicht mehr sagen: ›So wahr der HERR
lebt, der die Kinder Israel aus dem Lande Ägypten hergeführt hat!‹,
\bibleverse{8}sondern: ›So wahr der HERR lebt, der die zum Hause Israel
Gehörigen aus dem Nordland und aus all den Ländern, wohin ich sie
versprengt hatte, hergeführt und heimgebracht hat, damit sie wieder auf
ihrem Grund und Boden wohnten!‹«\textless sup title=``vgl.
16,14-15''\textgreater✲

\hypertarget{g-gegen-das-verwerfliche-treiben-der-falschen-propheten}{%
\paragraph{g) Gegen das verwerfliche Treiben der falschen
Propheten}\label{g-gegen-das-verwerfliche-treiben-der-falschen-propheten}}

\hypertarget{aa-klage-uxfcber-die-allgemeine-verderbnis-und-die-verworfenheit-der-geistlichen-fuxfchrer}{%
\subparagraph{aa) Klage über die allgemeine Verderbnis und die
Verworfenheit der geistlichen
Führer}\label{aa-klage-uxfcber-die-allgemeine-verderbnis-und-die-verworfenheit-der-geistlichen-fuxfchrer}}

\bibleverse{9}

\bibleverse{10}Ach, das Land ist voll von Ehebrechern! Ach, unter dem
Fluch liegt das Land in Trauer darnieder, sind die Auen der Trift
verdorrt, weil ihr ganzes Trachten Bosheit ist und ihr Schalten
Unredlichkeit. \bibleverse{11}»Denn beide, Propheten und Priester, sind
ruchlos: sogar in meinem Tempel habe ich ihr böses Treiben angetroffen!«
-- so lautet der Ausspruch des HERRN. \bibleverse{12}»Darum soll ihr Weg
für sie werden wie schlüpfriger Boden: in der Dunkelheit sollen sie
anstoßen, daß sie auf ihm zu Fall kommen; denn ich will Unheil, will das
Jahr ihrer Heimsuchung über sie bringen!« -- so lautet der Ausspruch des
HERRN.

\hypertarget{bb-ausspruxfcche-uxfcber-die-falschen-propheten}{%
\subparagraph{bb) Aussprüche über die falschen
Propheten}\label{bb-ausspruxfcche-uxfcber-die-falschen-propheten}}

\bibleverse{13}»Schon an den Propheten Samarias habe ich Ärgerliches
erlebt: sie weissagten im Namen\textless sup title=``vgl.
2,8''\textgreater✲ des Baal und führten mein Volk Israel irre;
\bibleverse{14}aber an den Propheten Jerusalems habe ich Grauenvolles
erlebt: Ehebruch und Wandel in der Lüge, und sie bestärken die Übeltäter
in ihrem Tun, damit sich ja keiner von ihnen von seiner Bosheit bekehre.
Ich achte sie allesamt den Leuten von Sodom gleich und die Bewohner
ihrer Stadt den Leuten von Gomorrha!« \bibleverse{15}Darum hat der HERR
der Heerscharen über die Propheten so gesprochen: »Fürwahr, ich will sie
mit Wermut speisen und ihnen Giftwasser zu trinken geben\textless sup
title=``vgl. 9,14''\textgreater✲; denn von den Propheten Jerusalems hat
sich Verworfenheit über das ganze Land verbreitet!«~--

\bibleverse{16}So hat der HERR der Heerscharen gesprochen: »Hört nicht
auf die Worte der Propheten, die euch weissagen! Sie machen euch nur
Wind vor: selbstersonnene Gesichte verkünden sie euch ohne den Auftrag
des HERRN. \bibleverse{17}Sie sagen immerdar zu denen, die mich
verachten: ›Der HERR hat verheißen: Es wird euch wohl ergehen!‹, und zu
allen, die im Starrsinn ihres Herzens dahinwandeln, sagen sie: ›Es wird
euch kein Unheil widerfahren! \bibleverse{18}Denn wer hat im Ratskreise
des HERRN gestanden, daß er ihn gesehen und sein Wort gehört hätte? Wer
hat sein Wort erlauscht und gehört?‹ \bibleverse{19}Wisset wohl: ein
Sturmwind des HERRN, sein Grimm, bricht los und wirbelnde Windsbraut,
die auf das Haupt der Gottlosen niederfährt! \bibleverse{20}Nicht
nachlassen wird der lodernde Zorn des HERRN, bis er's
vollbracht\textless sup title=``oder: bis er ihn völlig
gestillt''\textgreater✲ und die Gedanken\textless sup title=``oder:
Entschlüsse''\textgreater✲ seines Herzens vollführt hat: am Ende der
Tage werdet ihr das schon klar erkennen! \bibleverse{21}Ich habe diese
Propheten nicht gesandt, und doch haben sie es eilig! Ich habe ihnen
keinen Auftrag gegeben, und doch weissagen sie! \bibleverse{22}Hätten
sie wirklich in meinem Ratskreise gestanden, so würden sie meinem Volk
meine Worte verkünden und es von seinem bösen Wandel und seinem
gottlosen Tun abbringen!«~--

\bibleverse{23}»Bin ich denn ein Gott, der nur in die Nähe sieht« -- so
lautet der Ausspruch des HERRN --, »und nicht ein Gott auch aus der
Ferne? \bibleverse{24}Oder kann sich jemand in Schlupfwinkeln so
verstecken, daß ich ihn nicht sähe?« -- so lautet der Ausspruch des
HERRN. »Bin ich es nicht, der den Himmel und die Erde erfüllt?« -- so
lautet der Ausspruch des HERRN.

\hypertarget{cc-warnung-vor-den-truxe4umen-der-luxfcgenpropheten}{%
\subparagraph{cc) Warnung vor den Träumen der
Lügenpropheten}\label{cc-warnung-vor-den-truxe4umen-der-luxfcgenpropheten}}

\bibleverse{25}»Ich habe wohl gehört, was die Propheten sagen, die in
meinem Namen Lügen weissagen, wenn sie verkünden: ›Ich habe einen Traum
gehabt, einen Traum!‹ \bibleverse{26}Wie lange soll das bei ihnen noch
so fortgehen? Haben etwa diese Lügenpropheten, die selbstersonnenen Trug
weissagen, im Sinn, \bibleverse{27}ja, haben sie die Absicht, durch ihre
Träume, die sie einander erzählen, meinen Namen bei meinem Volke ebenso
in Vergessenheit zu bringen, wie ihre Väter meinen Namen über dem Baal
vergessen haben? \bibleverse{28}Der Prophet, dem (wirklich) ein Traum
zuteil geworden ist, erzähle ihn als Traum, und wem mein Wort zuteil
geworden ist, verkünde mein Wort der Wahrheit gemäß! Was hat das Stroh
mit dem Korn gemein?« -- so lautet der Ausspruch des HERRN.
\bibleverse{29}»Ist mein Wort nicht also: wie Feuer?« -- so lautet der
Ausspruch des HERRN -- »und wie ein Hammer, der Felsen zerschlägt?
\bibleverse{30}Darum wisset wohl: ich will an die Propheten\textless sup
title=``d.h. gegen die Propheten vorgehen''\textgreater✲« -- so lautet
der Ausspruch des HERRN --, »die meine Worte einer dem andern stehlen!
\bibleverse{31}Ja, wisset wohl: ich will an die Propheten« -- so lautet
der Ausspruch des HERRN --, »die ihre Zunge dazu mißbrauchen,
Gottessprüche zu verkünden! \bibleverse{32}Ja, wisset wohl: ich will an
die (Propheten), welche Lügenträume weissagen« -- so lautet der
Ausspruch des HERRN -- »und sie anderen erzählen und mein Volk durch
ihre Lügen und ihre Gaukelei irreführen, während ich sie doch nicht
gesandt und ihnen keinen Auftrag gegeben habe und sie diesem Volke gar
keinen Nutzen schaffen!« -- so lautet der Ausspruch des HERRN.

\hypertarget{h-warnung-vor-dem-ungebuxfchrlichen-ausdruck-last-des-herrn}{%
\paragraph{h) Warnung vor dem ungebührlichen Ausdruck ›Last des
Herrn‹}\label{h-warnung-vor-dem-ungebuxfchrlichen-ausdruck-last-des-herrn}}

\bibleverse{33}»Wenn aber dieses Volk oder ein Prophet oder ein Priester
dich fragen sollte: ›Was ist die Last des HERRN?‹, so antworte ihnen:
›\textless sup title=``Was die Last sei?''\textgreater✲Ihr seid die
Last, und ich will euch abwerfen! -- so lautet der Ausspruch des HERRN.
\bibleverse{34}Der Prophet aber und der Priester und wer vom Volk noch
von der Last des HERRN redet -- einen solchen Menschen will ich es büßen
lassen samt seinem Hause! \bibleverse{35}Ihr sollt vielmehr zueinander
und untereinander so sagen: ›Was hat der HERR geantwortet?‹ und ›Was hat
der HERR verkündigt?‹ \bibleverse{36}Aber den Ausdruck ›Last des HERRN‹
sollt ihr nicht mehr gebrauchen, sonst soll einem jeden diese seine
Redeweise zur Last werden! Denn ihr würdet damit die Worte des
lebendigen Gottes, des HERRN der Heerscharen, unsers Gottes,
verkehren\textless sup title=``oder: verdrehen''\textgreater✲.
\bibleverse{37}So sollst du den Propheten fragen: ›Was hat der HERR dir
geantwortet?‹ oder ›Was hat der HERR verkündigt?‹ \bibleverse{38}Wenn
ihr aber den Ausdruck ›Last des HERRN‹ gebraucht -- nun, so hat der HERR
folgendermaßen gesprochen: Zur Strafe dafür, daß ihr diesen Ausdruck
›Last des HERRN‹ gebraucht, obgleich ich euch habe gebieten lassen, den
Ausdruck ›Last des HERRN‹ nicht zu gebrauchen, \bibleverse{39}darum
wisset wohl: Ich will euch aufheben wie eine Last und wegwerfen euch
samt der Stadt, die ich euch und euren Vätern gegeben habe, von meinem
Angesicht hinweg, \bibleverse{40}und will ewige Schmach über euch
verhängen und ewige Schande, die nie vergessen werden soll!‹«

\hypertarget{i-das-gesicht-von-den-zwei-feigenkuxf6rben-und-die-bedeutung-der-kuxf6rbe}{%
\paragraph{i) Das Gesicht von den zwei Feigenkörben und die Bedeutung
der
Körbe}\label{i-das-gesicht-von-den-zwei-feigenkuxf6rben-und-die-bedeutung-der-kuxf6rbe}}

\hypertarget{section-23}{%
\section{24}\label{section-23}}

\bibleverse{1}Der HERR hat mich (folgendes Gesicht) schauen lassen: Ich
gewahrte zwei Körbe mit Feigen, die vor dem Tempel des HERRN aufgestellt
waren -- nachdem Nebukadnezar, der König von Babylon, Jechonja✲, den
Sohn Jojakims, den König von Juda, und die Oberen\textless sup
title=``oder: Fürsten''\textgreater✲ von Juda samt den Schmieden und
Schlossern aus Jerusalem in die Gefangenschaft geführt und sie nach
Babylon gebracht hatte. \bibleverse{2}Der eine Korb enthielt sehr gute
Feigen, wie Frühfeigen zu sein pflegen; in dem andern Korbe aber
befanden sich sehr schlechte Feigen, die wegen ihrer schlechten
Beschaffenheit ungenießbar waren.

\bibleverse{3}Da fragte mich der HERR: »Was siehst du, Jeremia?« Ich
antwortete: »Feigen! Die guten Feigen sind sehr gut, aber die schlechten
ganz schlecht, so daß man sie vor Schlechtigkeit nicht genießen kann.«

\bibleverse{4}Da erging das Wort des HERRN an mich folgendermaßen:
\bibleverse{5}»So spricht der HERR, der Gott Israels: Wie diese guten
Feigen hier, so will ich die gefangenen Judäer, die ich aus diesem Ort
in das Land der Chaldäer habe wegführen lassen, freundlich ansehen:
\bibleverse{6}ich will mein Auge zum Guten\textless sup title=``oder:
freundlich''\textgreater✲ auf sie richten und sie in dieses Land
zurückkehren lassen, um sie neu aufzubauen, ohne sie wieder
niederzureißen, und um sie einzupflanzen, ohne sie wieder auszureißen.
\bibleverse{7}Und ich will ihnen ein Herz\textless sup title=``oder:
Einsicht''\textgreater✲ verleihen, mich zu erkennen, daß ich der HERR
bin; und sie sollen mein Volk sein, und ich will ihr Gott sein, denn sie
werden sich mit ihrem ganzen Herzen zu mir bekehren.~--
\bibleverse{8}Aber wie die schlechten Feigen, die so schlecht sind, daß
man sie nicht genießen kann« -- ja, so hat der HERR gesprochen --,
»ebenso will ich Zedekia, den König von Juda, machen samt seinen
Oberen\textless sup title=``oder: Fürsten''\textgreater✲ und denen, die
von den Bewohnern Jerusalems in diesem Lande zurückgeblieben sind, und
auch denen, die sich in Ägypten niedergelassen haben: \bibleverse{9}ich
will sie zum abschreckenden Beispiel des Unglücks für alle Reiche der
Erde machen, zum Schimpf und zum Hohn, zur Spottrede und zum Fluchwort
an allen Orten, wohin ich sie verstoßen werde; \bibleverse{10}und ich
will das Schwert, den Hunger und die Pest gegen sie loslassen, bis sie
ganz aus dem Lande vertilgt sind, das ich ihnen und ihren Vätern gegeben
habe.«

\hypertarget{k-gerichtsdrohung-gegen-juda-und-die-gesamte-vuxf6lkerwelt}{%
\paragraph{k) Gerichtsdrohung gegen Juda und die gesamte
Völkerwelt}\label{k-gerichtsdrohung-gegen-juda-und-die-gesamte-vuxf6lkerwelt}}

\hypertarget{aa-zeitangabe-jeremias-hinweis-auf-seine-23juxe4hrige-erfolglose-wirksamkeit}{%
\subparagraph{aa) Zeitangabe; Jeremias Hinweis auf seine 23jährige
erfolglose
Wirksamkeit}\label{aa-zeitangabe-jeremias-hinweis-auf-seine-23juxe4hrige-erfolglose-wirksamkeit}}

\hypertarget{section-24}{%
\section{25}\label{section-24}}

\bibleverse{1}(Dies ist) das Wort, das an Jeremia über das ganze Volk
Juda ergangen ist im vierten Regierungsjahre Jojakims, des Sohnes
Josias, des Königs von Juda -- es war dies das erste Regierungsjahr
Nebukadnezars, des Königs von Babylon --; \bibleverse{2}der Prophet
Jeremia hat dies Wort an das ganze Volk von Juda und an alle Einwohner
Jerusalems gerichtet, indem er sprach:

\bibleverse{3}»Seit dem dreizehnten Regierungsjahre Josias, des Sohnes
Amons, des Königs von Juda, bis auf den heutigen Tag, nun schon
dreiundzwanzig Jahre lang, ist das Wort des HERRN an mich ergangen, und
ich habe unermüdlich früh und spät zu euch geredet, aber ihr habt nicht
darauf gehört. \bibleverse{4}Dazu hat der HERR alle seine
Knechte\textless sup title=``oder: Diener''\textgreater✲, die Propheten,
unermüdlich früh und spät zu euch gesandt, aber ihr habt ihnen nicht
gehorcht und ihnen kein Gehör geschenkt, um euch warnen zu lassen,
\bibleverse{5}indem er euch sagen ließ: ›Kehrt doch um, ein jeder von
seinem bösen Wandel und von seinem verwerflichen Tun, dann sollt ihr in
dem Lande, das der HERR euch und euren Vätern gegeben hat, wohnen
bleiben bis in alle Ewigkeit! \bibleverse{6}Lauft also nicht anderen
Göttern nach, um ihnen zu dienen und sie anzubeten, und reizt mich nicht
zum Zorn durch die Machwerke eurer Hände, damit ich kein Unglück über
euch verhänge! \bibleverse{7}Aber ihr habt nicht auf mich gehört‹ -- so
lautet der Ausspruch des HERRN --, ›sondern habt mich geflissentlich zum
Zorn gereizt durch die Machwerke eurer Hände, euch selbst zum Unheil.‹«

\hypertarget{bb-ankuxfcndigung-der-vernichtung-judas-sowie-der-siebzigjuxe4hrigen-babylonischen-gefangenschaft-und-der-spuxe4teren-bestrafung-der-chalduxe4er}{%
\subparagraph{bb) Ankündigung der Vernichtung Judas sowie der
siebzigjährigen babylonischen Gefangenschaft und der späteren Bestrafung
der
Chaldäer}\label{bb-ankuxfcndigung-der-vernichtung-judas-sowie-der-siebzigjuxe4hrigen-babylonischen-gefangenschaft-und-der-spuxe4teren-bestrafung-der-chalduxe4er}}

\bibleverse{8}Darum hat der HERR der Heerscharen so gesprochen: »Zur
Strafe dafür, daß ihr auf meine Worte nicht gehört habt,
\bibleverse{9}will ich nunmehr alle Völkerschaften des Nordens
herbeiholen« -- so lautet der Ausspruch des HERRN -- »und an meinen
Knecht Nebukadnezar, den König von Babylon, Botschaft senden und sie
gegen dies Land und seine Bewohner und gegen alle diese Völker ringsum
hereinbrechen lassen; und ich will den Bann über sie
verhängen\textless sup title=``=~sie dem Untergang weihen''\textgreater✲
und sie zum Gegenstand des Entsetzens und des Spottes und zu ewigen
Einöden machen; \bibleverse{10}und will unter ihnen jeder lauten Freude
und Fröhlichkeit, jedem Bräutigamsjubel und jedem Brautgesang, dem
Schall der Handmühlen und dem Licht der Lampen ein Ende machen.
\bibleverse{11}Dieses ganze Land soll zur Einöde, zur Wüste werden, und
diese Völkerschaften sollen dem Könige von Babylon dienstbar sein
siebzig Jahre lang. \bibleverse{12}Wenn aber die siebzig Jahre um sind,
dann will ich auch am König von Babylon und an jenem Volk« -- so lautet
der Ausspruch des HERRN -- »das Strafgericht wegen ihrer Verschuldung
vollziehen, auch am Lande der Chaldäer, und will es auf ewig zu
Wüsteneien machen. \bibleverse{13}Ich will dann an jenem Lande alle
meine Drohungen, die ich gegen dasselbe ausgesprochen habe, in Erfüllung
gehen lassen, alles, was in diesem Buche geschrieben steht, was Jeremia
über alle Völker geweissagt hat. \bibleverse{14}Denn sie\textless sup
title=``d.h. die Chaldäer''\textgreater✲ sollen gleichfalls mächtigen
Völkern und gewaltigen Königen dienstbar werden, und ich werde ihnen
nach Verdienst und nach ihrem ganzen Tun vergelten.«

\hypertarget{cc-gottes-zornbecher-und-schwert-fuxfcr-alle-vuxf6lker}{%
\subparagraph{cc) Gottes Zornbecher und Schwert für alle
Völker}\label{cc-gottes-zornbecher-und-schwert-fuxfcr-alle-vuxf6lker}}

\bibleverse{15}Denn so hat der HERR, der Gott Israels, zu mir
gesprochen: »Nimm diesen Becher voll Zornweins aus meiner Hand und laß
alle Völker, zu denen ich dich senden werde, daraus trinken!
\bibleverse{16}Sie sollen trinken, daß sie hin und her taumeln und in
Tollheit rasen ob\textless sup title=``oder: vor''\textgreater✲ dem
Schwert, das ich unter sie sende!«

\bibleverse{17}Da nahm ich den Becher aus der Hand des HERRN und ließ
alle Völker daraus trinken, zu denen der HERR mich gesandt hatte:
\bibleverse{18}Jerusalem und die anderen Städte Judas, ihre Könige und
ihre Fürsten\textless sup title=``oder: Oberen''\textgreater✲, um sie
zur Einöde, zum abschreckenden Beispiel, zum Gegenstand des Spottes und
zum Fluchwort zu machen, wie es heutzutage der Fall ist;
\bibleverse{19}sodann den Pharao, den König von Ägypten, samt seinen
Dienern und obersten Beamten und seinem ganzen Volk \bibleverse{20}und
das gesamte Völkergemisch dort; sodann alle Könige des Landes
Uz\textless sup title=``vgl. Hiob 1,1''\textgreater✲ und alle Könige des
Philisterlandes, nämlich Askalon, Gaza, Ekron und den Überrest von
Asdod; \bibleverse{21}Edom, Moab und die Ammoniter; \bibleverse{22}alle
Könige von Tyrus, alle Könige von Sidon und die Könige der Küstenländer
jenseits des Meeres; \bibleverse{23}ferner Dedan, Thema, Bus und alle,
die sich das Haar an den Schläfen stutzen; \bibleverse{24}sodann alle
Könige von Arabien und alle Könige der Mischvölker, die in der Wüste
wohnen; \bibleverse{25}ferner alle Könige von Simri und alle Könige von
Elam und alle Könige von Medien; \bibleverse{26}sodann alle Könige des
Nordens, die nahen wie die fernen, einen nach dem andern, überhaupt alle
Königreiche der Welt, soviele ihrer auf dem ganzen Erdboden sind; der
König von Sesach✲ aber muß nach ihnen trinken.

\bibleverse{27}»Du sollst dabei zu ihnen sagen: ›So hat der HERR der
Heerscharen, der Gott Israels, gesprochen: Trinkt, bis ihr trunken seid
und euch erbrecht! Stürzt hin, ohne wieder aufzustehen --
ob\textless sup title=``oder: vor''\textgreater✲ dem Schwert, das ich
unter euch sende!‹ \bibleverse{28}Sollten sie sich aber weigern, den
Becher aus deiner Hand zu nehmen, um aus ihm zu trinken, so sollst du zu
ihnen sagen: ›So hat der HERR der Heerscharen gesprochen: Trinken müßt
ihr dennoch! \bibleverse{29}Denn wisset wohl: Bei der Stadt, die nach
meinem Namen genannt ist, habe ich mit dem Strafgericht den Anfang
gemacht, und da solltet ihr frei ausgehen? Nein, ihr sollt nicht
ungestraft bleiben; denn das Schwert biete ich gegen alle Bewohner der
Erde auf!‹« -- so lautet der Ausspruch des HERRN der Heerscharen.

\hypertarget{dd-gott-erscheint-zum-weltgericht-vernichtung-der-vuxf6lker}{%
\subparagraph{dd) Gott erscheint zum Weltgericht; Vernichtung der
Völker}\label{dd-gott-erscheint-zum-weltgericht-vernichtung-der-vuxf6lker}}

\bibleverse{30}»Du aber sollst bei der Verkündigung aller dieser
Drohworte zu ihnen sagen: ›Der HERR brüllt aus der Höhe und läßt seine
Stimme erschallen✲ aus seiner heiligen Wohnstätte! Laut brüllt er über
seine Aue hin, läßt ein Jauchzen erschallen wie die Keltertreter gegen
alle Bewohner der Erde. \bibleverse{31}Bis ans Ende der Erde dringt der
Schall; denn der HERR geht mit den Völkern ins Gericht; er bringt seine
Sache mit der ganzen Menschheit zum Austrag: die Gottlosen gibt er dem
Schwerte preis!‹« -- so lautet der Ausspruch des HERRN.

\bibleverse{32}So hat der HERR der Heerscharen gesprochen: »Fürwahr,
Unheil schreitet von Volk zu Volk, und ein gewaltiger Sturm zieht heran
von den Enden der Erde!« \bibleverse{33}An jenem Tage werden die vom
HERRN Erschlagenen von einem Ende der Erde bis zum andern daliegen,
unbetrauert und ohne aufgehoben und begraben zu werden: zu Dünger müssen
sie auf offenem Felde werden.

\bibleverse{34}»Heult, ihr Völkerhirten, und schreit! Und wälzt euch (in
der Asche), ihr Führer\textless sup title=``oder: Herren''\textgreater✲
der Herde! Denn eure Zeit ist erfüllt, daß man euch schlachte, und ich
zerschmettere euch, daß ihr zu Boden fallen sollt wie kostbares
Geschirr!« \bibleverse{35}Da gibt es kein Entfliehen mehr für die Hirten
und kein Entrinnen für die Führer\textless sup title=``oder:
Herren''\textgreater✲ der Herde. \bibleverse{36}Horch! Angstgeschrei der
Hirten und Geheul der Führer\textless sup title=``oder:
Herren''\textgreater✲ der Herde! Denn der HERR verwüstet ihre Weide,
\bibleverse{37}und verheert werden die friedlichen Auen vor dem
lodernden Zorn des HERRN! \bibleverse{38}Wie ein Löwe hat er sein
Dickicht verlassen: ach, ihr Land ist zur Wüste geworden vor dem
verheerenden Schwert und vor seinem lodernden Zorn!

\hypertarget{l-jeremias-verhaftung-und-freisprechung-wegen-seiner-tempelrede-unter-jojakim-hinrichtung-des-propheten-uria}{%
\paragraph{l) Jeremias Verhaftung und Freisprechung wegen seiner
Tempelrede unter Jojakim; Hinrichtung des Propheten
Uria}\label{l-jeremias-verhaftung-und-freisprechung-wegen-seiner-tempelrede-unter-jojakim-hinrichtung-des-propheten-uria}}

\hypertarget{aa-einleitung-angabe-des-hauptinhalts-der-rede-gefangennahme-jeremias}{%
\subparagraph{aa) Einleitung; Angabe des Hauptinhalts der Rede;
Gefangennahme
Jeremias}\label{aa-einleitung-angabe-des-hauptinhalts-der-rede-gefangennahme-jeremias}}

\hypertarget{section-25}{%
\section{26}\label{section-25}}

\bibleverse{1}Im Anfang der Regierung Jojakims, des Sohnes Josias, des
Königs von Juda, erging folgendes Wort des HERRN (an Jeremia):
\bibleverse{2}So spricht der HERR: »Stelle dich auf im Vorhof des
Tempels des HERRN und verkünde denen, die aus allen Ortschaften Judas
herkommen, um im Tempel des HERRN anzubeten, alle Worte, deren
Verkündigung ich dir geboten habe: laß kein Wort davon weg!
\bibleverse{3}Vielleicht hören sie darauf und bekehren sich, ein jeder
von seinem bösen Wandel; dann würde ich mir auch das Unheil leid sein
lassen, das ich ihnen wegen ihres bösen Tuns zuzufügen gedenke.
\bibleverse{4}Und zwar sollst du zu ihnen sagen: ›So hat der HERR
gesprochen: Wenn ihr mir nicht gehorcht und nicht nach meinem Gesetz
wandelt, das ich euch vorgelegt habe, \bibleverse{5}und wenn ihr nicht
auf die Worte meiner Knechte, der Propheten, hört, die ich früh und spät
immer wieder zu euch sende, ohne daß ihr auf sie hört: \bibleverse{6}so
will ich mit diesem Tempel hier verfahren wie einst mit dem zu Silo und
will den Namen dieser Stadt zum Fluchwort für alle Völker der Erde
machen!‹«

\bibleverse{7}Als nun die Priester und Propheten und das gesamte Volk
den Jeremia diese Worte im Tempel des HERRN verkündigen hörten
\bibleverse{8}und Jeremia mit der Verkündigung alles dessen, was er dem
ganzen Volke nach dem Befehl des HERRN vorhalten sollte, zu Ende war, da
ergriffen ihn die Priester, die Propheten und das gesamte Volk und
riefen: »Jetzt mußt du sterben! \bibleverse{9}Warum hast du im Namen des
HERRN die Weissagung ausgesprochen, es werde diesem Hause ergehen wie
dem zu Silo und diese Stadt werde so wüst werden, daß niemand mehr darin
wohne?« So rottete sich denn das gesamte Volk im Tempel des HERRN gegen
Jeremia zusammen.

\hypertarget{bb-die-gerichtsverhandlung-vor-den-oberen-jeremias-freisprechung-die-fuxfcrsprache-einiger-volksuxe4ltesten-fuxfcr-ihn}{%
\subparagraph{bb) Die Gerichtsverhandlung vor den Oberen; Jeremias
Freisprechung; die Fürsprache einiger Volksältesten für
ihn}\label{bb-die-gerichtsverhandlung-vor-den-oberen-jeremias-freisprechung-die-fuxfcrsprache-einiger-volksuxe4ltesten-fuxfcr-ihn}}

\bibleverse{10}Als nun die Fürsten\textless sup title=``oder:
Oberen''\textgreater✲ von Juda von diesen Vorgängen Kunde erhielten,
begaben sie sich aus dem Palast des Königs zum Tempel des HERRN hinauf
und ließen sich zum Gericht nieder im Eingang des neuen Tores am Tempel
des HERRN. \bibleverse{11}Hierauf gaben die Priester und die Propheten
vor den Fürsten\textless sup title=``oder: Oberen''\textgreater✲ und dem
gesamten Volke die Erklärung ab: »Dieser Mann ist des Todes schuldig;
denn er hat gegen diese Stadt geweissagt, wie ihr mit eigenen Ohren
gehört habt.«

\bibleverse{12}Jeremia aber richtete an alle Fürsten\textless sup
title=``oder: Oberen''\textgreater✲ und an das gesamte Volk folgende
Worte: »Der HERR hat mich gesandt, damit ich gegen diesen Tempel und
gegen diese Stadt alle die Drohworte ausspreche, die ihr vernommen habt.
\bibleverse{13}Und nun -- bessert euren Wandel und euer ganzes Tun und
gehorcht den Weisungen des HERRN, eures Gottes, damit der HERR sich das
Unheil leid sein läßt, das er euch angedroht hat! \bibleverse{14}Was
mich aber betrifft, so stehe ich hier in eurer Gewalt: verfahrt mit mir,
wie es euch gut und recht dünkt! \bibleverse{15}Nur das sollt ihr
bestimmt wissen: Wenn ihr mich tötet, werdet ihr unschuldiges Blut über
euch, über diese Stadt und ihre Bewohner bringen; denn der HERR hat mich
wahrhaftig zu euch gesandt, damit ich alle diese Worte laut an euch
richte.«

\bibleverse{16}Da sagten die Fürsten\textless sup title=``oder:
Oberen''\textgreater✲ und das gesamte Volk zu den Priestern und den
Propheten: »Dieser Mann ist des Todes nicht schuldig, denn er hat im
Auftrage des HERRN, unsers Gottes, zu uns geredet.«
\bibleverse{17}Hierauf traten auch Männer von den Ältesten des Landes
auf und sagten zu der ganzen versammelten Volksmenge:
\bibleverse{18}»Micha aus Moreseth\textless sup title=``vgl. Mi
1,1''\textgreater✲ ist unter der Regierung Hiskias, des Königs von Juda,
als Prophet aufgetreten und hat zum ganzen Volk von Juda gesagt: ›So hat
der HERR der Heerscharen gesprochen: Zion wird zu Ackerland umgepflügt
werden, und Jerusalem wird zu einem Trümmerhaufen und der Tempelberg zu
einer bewaldeten Höhe werden!‹\textless sup title=``Mi
3,12''\textgreater✲ \bibleverse{19}Haben nun etwa Hiskia, der König von
Juda, und ganz Juda ihn dafür getötet? Hat Hiskia nicht vielmehr den
HERRN gefürchtet und den HERRN zu versöhnen gewußt, so daß der HERR sich
das Unheil leid sein ließ, das er ihnen angedroht hatte? Und wir wollen
unser Gewissen mit einer so schweren Schuld beladen?«

\hypertarget{cc-unheilvolles-geschick-des-propheten-uria}{%
\subparagraph{cc) Unheilvolles Geschick des Propheten
Uria}\label{cc-unheilvolles-geschick-des-propheten-uria}}

\bibleverse{20}Es war aber damals noch ein anderer Mann da, der im Namen
des HERRN als Prophet wirkte, nämlich Uria, der Sohn Semajas, aus
Kirjath-Jearim; und zwar weissagte er gegen diese Stadt und gegen dieses
Land mit denselben Worten wie Jeremia. \bibleverse{21}Als nun der König
Jojakim und alle seine Heerführer und alle obersten Beamten von seinen
Reden hörten, suchte der König ihn zu töten; Uria aber erhielt Kunde
davon, und da er sich fürchtete, ergriff er die Flucht und entkam nach
Ägypten. \bibleverse{22}Da sandte der König Jojakim Männer nach Ägypten,
nämlich Elnathan, den Sohn Achbors, und noch einige andere mit ihm;
\bibleverse{23}die holten Uria aus Ägypten und brachten ihn zum König
Jojakim, der ihn mit dem Schwert hinrichten und seinen Leichnam auf den
Begräbnisplatz des gemeinen Volkes werfen ließ. \bibleverse{24}Aber
Jeremias hatte sich (damals) Ahikam, der Sohn Saphans, tatkräftig
angenommen, so daß man ihn dem Volk nicht zur Tötung✲ preisgab.

\hypertarget{jeremia-im-kampf-mit-den-falschen-propheten-kap.-27-29}{%
\subsubsection{6. Jeremia im Kampf mit den falschen Propheten (Kap.
27-29)}\label{jeremia-im-kampf-mit-den-falschen-propheten-kap.-27-29}}

\hypertarget{a-jeremias-aufforderung-an-die-nachbarvuxf6lker-sowie-an-seine-volksgenossen-sich-unter-das-joch-nebukadnezars-zu-beugen}{%
\paragraph{a) Jeremias Aufforderung an die Nachbarvölker sowie an seine
Volksgenossen, sich unter das Joch Nebukadnezars zu
beugen}\label{a-jeremias-aufforderung-an-die-nachbarvuxf6lker-sowie-an-seine-volksgenossen-sich-unter-das-joch-nebukadnezars-zu-beugen}}

\hypertarget{aa-jeremia-mit-dem-joch-auf-dem-nacken-warnt-die-gesandten-einiger-auswuxe4rtigen-staaten}{%
\subparagraph{aa) Jeremia, mit dem Joch auf dem Nacken, warnt die
Gesandten einiger auswärtigen
Staaten}\label{aa-jeremia-mit-dem-joch-auf-dem-nacken-warnt-die-gesandten-einiger-auswuxe4rtigen-staaten}}

\hypertarget{section-26}{%
\section{27}\label{section-26}}

\bibleverse{1}Im Anfang der Regierung Zedekias, des Sohnes Josias, des
Königs von Juda, erging folgendes Wort an Jeremia vom HERRN:
\bibleverse{2}so gebot mir der HERR: »Mache dir Stricke und Jochstäbe
und lege sie dir auf den Nacken \bibleverse{3}und sende (Botschaft) an
den König von Edom sowie an den König von Moab, an den König der
Ammoniter, an den König von Tyrus und an den König von Sidon durch
Vermittlung der Gesandten, die nach Jerusalem zu Zedekia, dem König von
Juda, gekommen sind, \bibleverse{4}und trage ihnen folgende Botschaft an
ihre Gebieter auf: ›So hat der HERR der Heerscharen, der Gott Israels,
gesprochen: Berichtet euren Gebietern folgendes: \bibleverse{5}Ich habe
die Erde, die Menschen und die Tiere, die es auf der ganzen Erde gibt,
durch meine große Kraft und meinen ausgestreckten Arm geschaffen und
gebe sie, wem es mir beliebt. \bibleverse{6}So habe ich nunmehr alle
diese Länder der Gewalt meines Kneches\textless sup title=``oder:
Dieners''\textgreater✲ Nebukadnezar, des Königs von Babylon, übergeben
und sogar die Tiere des Feldes ihm gegeben, daß sie ihm dienstbar seien.
\bibleverse{7}So sollen denn alle Völker ihm und seinem Sohne und seinem
Enkel untertan sein, bis auch für sein Land die Zeit gekommen ist, wo
mächtige Völker und große Könige ihn sich untertan machen.
\bibleverse{8}Dasjenige Volk und Reich aber, das ihm, dem babylonischen
König Nebukadnezar, sich nicht unterwirft und seinen Nacken nicht
in\textless sup title=``oder: unter''\textgreater✲ das Joch des
babylonischen Königs stecken will, ein solches Volk‹ -- so lautet der
Ausspruch des HERRN -- ›will ich mit dem Schwert✲, mit Hunger und mit
der Pest heimsuchen, bis ich es durch seine Hand gänzlich vernichtet
habe. \bibleverse{9}So hört ihr nun nicht auf eure Propheten und
Wahrsager, auch nicht auf eure Träume\textless sup title=``oder:
Träumer''\textgreater✲, eure Zauberer und Beschwörer, wenn sie euch
bestimmt versichern: Ihr werdet dem König von Babylon nicht untertan
sein müssen; \bibleverse{10}denn eine Lüge ist es, die sie euch
weissagen, um euch aus eurem Lande in die Verbannung zu bringen, weil
ich euch alsdann verstoßen werde und ihr zugrunde geht.
\bibleverse{11}Dasjenige Volk aber, das seinen Nacken in\textless sup
title=``oder: unter''\textgreater✲ das Joch des babylonischen Königs
steckt und ihm untertan ist, das will ich ruhig in seinem Lande
belassen‹ -- so lautet der Ausspruch des HERRN --, ›damit es dasselbe
bebaut und darin wohnen bleibt.‹«

\hypertarget{bb-jeremia-richtet-dieselbe-warnung-an-den-juxfcdischen-kuxf6nig-zedekia}{%
\subparagraph{bb) Jeremia richtet dieselbe Warnung an den jüdischen
König
Zedekia}\label{bb-jeremia-richtet-dieselbe-warnung-an-den-juxfcdischen-kuxf6nig-zedekia}}

\bibleverse{12}Hierauf richtete ich an Zedekia, den König von Juda,
folgende Worte in ganz demselben Sinn: »Steckt euren Nacken
in\textless sup title=``oder: unter''\textgreater✲ das Joch des Königs
von Babylon und unterwerft euch ihm und seinem Volk, so werdet ihr am
Leben bleiben! \bibleverse{13}Warum wollt ihr, du und dein Volk, durch
das Schwert, durch den Hunger und durch die Pest zugrunde gehen, wie der
HERR dem Volke angedroht hat, das sich dem Könige von Babylon nicht
unterwerfen will? \bibleverse{14}Hört nur nicht auf die Reden der
Propheten, die euch bestimmt versichern: ›Ihr werdet dem König von
Babylon nicht untertan sein müssen!‹, denn eine Lüge ist es, die sie
euch weissagen. \bibleverse{15}›Denn ich habe sie nicht gesandt‹ -- so
lautet der Ausspruch des HERRN --, ›vielmehr weissagen sie Falsches in
meinem Namen, damit ich euch verstoße und ihr elend zugrunde geht, ihr
mitsamt den Propheten, die euch weissagen.‹«

\hypertarget{cc-jeremias-mahnung-an-die-priester-und-an-das-volk}{%
\subparagraph{cc) Jeremias Mahnung an die Priester und an das
Volk}\label{cc-jeremias-mahnung-an-die-priester-und-an-das-volk}}

\bibleverse{16}Hierauf wandte ich mich an die Priester und an das ganze
hiesige Volk mit folgenden Worten: »So hat der HERR gesprochen: Hört
nicht auf die Reden eurer Propheten, die vor euch die Weissagung
aussprechen: ›Fürwahr, die Tempelgeräte des HERRN werden nun gar bald
aus Babylon zurückgebracht werden!‹, denn eine Lüge ist es, die sie euch
weissagen. \bibleverse{17}Hört nicht auf sie, werdet vielmehr dem König
von Babylon untertan, so werdet ihr am Leben bleiben: warum soll diese
Stadt zu einer Wüste werden? \bibleverse{18}Sind sie aber wirklich
Propheten und sind sie im Besitz des Wortes des HERRN, so mögen sie doch
Fürbitte beim HERRN der Heerscharen einlegen, daß die Geräte, die noch
im Tempel des HERRN und im Palast des Königs von Juda und in Jerusalem
übriggeblieben sind, nicht auch noch nach Babylon kommen!
\bibleverse{19}Denn so hat der HERR der Heerscharen bezüglich der Säulen
und bezüglich des großen Wasserbeckens sowie bezüglich der Gestühle und
der sonstigen Geräte gesprochen, die in dieser Stadt noch
zurückgeblieben sind, \bibleverse{20}weil Nebukadnezar, der König von
Babylon, sie nicht mitgenommen hat, als er Jechonja✲, den Sohn Jojakims,
den König von Juda, aus Jerusalem nach Babylon in die Gefangenschaft
wegführte samt allen vornehmen Männern Judas und Jerusalems~--
\bibleverse{21}ja, so hat der HERR der Heerscharen, der Gott Israels,
bezüglich der Geräte gesprochen, die im Tempel des HERRN und im Palast
des Königs von Juda und in Jerusalem noch zurückgeblieben sind:
\bibleverse{22}›Nach Babylon sollen sie gebracht werden und dort bleiben
bis zu dem Tage, an dem ich wieder nach ihnen sehen werde‹ -- so lautet
der Ausspruch des HERRN -- ›und ich sie wieder herschaffe und an diesen
Ort zurückbringe.‹«

\hypertarget{b-jeremia-und-der-falsche-prophet-hananja}{%
\paragraph{b) Jeremia und der falsche Prophet
Hananja}\label{b-jeremia-und-der-falsche-prophet-hananja}}

\hypertarget{aa-hananjas-ausspruch-und-jeremias-antwort}{%
\subparagraph{aa) Hananjas Ausspruch und Jeremias
Antwort}\label{aa-hananjas-ausspruch-und-jeremias-antwort}}

\hypertarget{section-27}{%
\section{28}\label{section-27}}

\bibleverse{1}Es begab sich aber in demselben Jahre, im Anfang der
Regierung Zedekias, des Königs von Juda, im fünften Monat des vierten
Jahres, da sagte der Prophet Hananja, der Sohn Assurs aus Gibeon, im
Tempel des HERRN in Gegenwart der Priester und des ganzen Volkes so zu
mir: \bibleverse{2}»So hat der HERR der Heerscharen, der Gott Israels,
gesprochen: ›Ich zerbreche das Joch des Königs von Babylon!
\bibleverse{3}Noch vor Ablauf von zwei Jahren will ich alle Tempelgeräte
des HERRN, die Nebukadnezar, der König von Babylon, von dieser Stätte
weggenommen und nach Babylon gebracht hat, wieder an diese Stätte
zurückbringen; \bibleverse{4}auch Jechonja, den Sohn Jojakims, den König
von Juda, samt allen Judäern, die nach Babylon in die
Verbannung\textless sup title=``oder: Gefangenschaft''\textgreater✲
weggeführt sind, will ich an diesen Ort zurückbringen‹ -- so lautet der
Ausspruch des HERRN --; ›denn ich will das Joch des Königs von Babylon
zerbrechen.‹«

\bibleverse{5}Da gab der Prophet Jeremia dem Propheten Hananja in
Gegenwart der Priester und des gesamten Volkes, das im Tempel des HERRN
anwesend war, \bibleverse{6}folgende Antwort: »Ja, so sei es! Der HERR
möge es so fügen! Der HERR möge deine Weissagung, die du ausgesprochen
hast, in Erfüllung gehen lassen, daß er nämlich die Tempelgeräte des
HERRN und alle in die Gefangenschaft Weggeführten aus Babylon an diesen
Ort zurückbringt! \bibleverse{7}Jedoch vernimm folgendes Wort, das ich
vor deinen Ohren und vor dem gesamten Volk hier laut ausspreche:
\bibleverse{8}›Die Propheten, die vor mir und vor dir seit den ältesten
Zeiten aufgetreten sind, die haben über mächtige Länder und über große
Reiche von Krieg, von Unheil und von Pest geweissagt; \bibleverse{9}der
Prophet also, der eine Glücksverheißung ausspricht, wird erst dann, wenn
seine Prophezeiung eingetroffen ist, als ein Prophet anerkannt werden,
den der HERR wirklich gesandt hat!‹«

\hypertarget{bb-hananjas-gewalttuxe4tiges-vorgehen-und-jeremias-gottesspruch-todesurteil-uxfcber-ihn}{%
\subparagraph{bb) Hananjas gewalttätiges Vorgehen und Jeremias
Gottesspruch (=~Todesurteil) über
ihn}\label{bb-hananjas-gewalttuxe4tiges-vorgehen-und-jeremias-gottesspruch-todesurteil-uxfcber-ihn}}

\bibleverse{10}Da nahm der Prophet Hananja die Jochstäbe vom Nacken des
Propheten Jeremia und zerbrach sie; \bibleverse{11}sodann sagte Hananja
vor dem ganzen Volke: »So hat der HERR gesprochen: ›Ebenso will ich das
Joch Nebukadnezars, des Königs von Babylon, noch vor Ablauf von zwei
Jahren zerbrechen und es vom Nacken aller Völker wegnehmen!‹« Der
Prophet Jeremia aber ging seines Weges.

\bibleverse{12}Nachdem aber der Prophet Hananja die Jochstäbe vom Nacken
des Propheten Jeremia (genommen und sie) zerbrochen hatte, erging das
Wort des HERRN an Jeremia folgendermaßen: \bibleverse{13}»Gehe hin und
sage zu Hananja: So hat der HERR gesprochen: ›Jochstäbe von Holz hast du
zerbrochen, aber Jochstäbe von Eisen an ihre Stelle gesetzt.‹
\bibleverse{14}Denn so hat der HERR der Heerscharen, der Gott Israels,
gesprochen: ›Ein eisernes Joch lege ich allen diesen Völkern auf den
Nacken, daß sie Nebukadnezar, dem König von Babylon, dienstbar sein
müssen; ja, sie sollen ihm dienen, und sogar die wilden Tiere des Feldes
habe ich ihm übergeben.‹«

\bibleverse{15}Weiter sagte der Prophet Jeremia zu dem Propheten
Hananja: »Höre doch, Hananja! Der HERR hat dich nicht gesandt, und doch
hast du dieses Volk dazu verführt, sich auf eine Lüge zu verlassen!
\bibleverse{16}Darum hat der HERR so gesprochen: ›Wisse wohl: ich will
dich vom Erdboden wegschaffen; noch in diesem Jahre sollst du sterben,
weil du zum Ungehorsam gegen den HERRN aufgefordert hast!‹«
\bibleverse{17}Und der Prophet Hananja starb wirklich noch in demselben
Jahre im siebten Monat.

\hypertarget{c-jeremias-schreiben-an-die-in-babylon-gefangenen-juduxe4er-buxf6ser-ausgang-zweier-falschen-propheten-in-babylon}{%
\paragraph{c) Jeremias Schreiben an die in Babylon gefangenen Judäer;
böser Ausgang zweier falschen Propheten in
Babylon}\label{c-jeremias-schreiben-an-die-in-babylon-gefangenen-juduxe4er-buxf6ser-ausgang-zweier-falschen-propheten-in-babylon}}

\hypertarget{aa-erkluxe4rende-einleitung}{%
\subparagraph{aa) Erklärende
Einleitung}\label{aa-erkluxe4rende-einleitung}}

\hypertarget{section-28}{%
\section{29}\label{section-28}}

\bibleverse{1}Dies ist der Wortlaut des Schreibens, das der Prophet
Jeremia von Jerusalem aus an die am Leben gebliebenen Ältesten unter den
in die Gefangenschaft Weggeführten und an die Priester und Propheten und
überhaupt an das gesamte Volk sandte, das Nebukadnezar von Jerusalem
nach Babylon in die Gefangenschaft geführt hatte~--
\bibleverse{2}nachdem der König Jechonja und die Königin-Mutter nebst
den Hofbeamten, den Fürsten\textless sup title=``oder:
Oberen''\textgreater✲ von Juda und Jerusalem, den Schmieden und
Schlossern aus Jerusalem weggezogen waren --, \bibleverse{3}und zwar
durch Vermittlung Eleasas, des Sohnes Saphans, und Gemarjas, des Sohnes
Hilkias, die Zedekia, der König von Juda, nach Babylon zu Nebukadnezar,
dem König von Babylon, sandte.

\hypertarget{bb-wortlaut-des-schreibens-jeremias}{%
\subparagraph{bb) Wortlaut des Schreibens
Jeremias}\label{bb-wortlaut-des-schreibens-jeremias}}

\bibleverse{4}(Dies ist der Wortlaut des Schreibens:) »So spricht der
HERR der Heerscharen, der Gott Israels, zu allen Verbannten, die ich aus
Jerusalem nach Babylon habe wegführen lassen: \bibleverse{5}›Baut Häuser
und wohnt in ihnen! Legt Gärten an und genießt ihre Früchte!
\bibleverse{6}Nehmt euch Frauen und zeugt Söhne und Töchter! Nehmt auch
für eure Söhne Frauen und verheiratet eure Töchter an Männer, damit sie
Mütter von Söhnen und Töchtern werden und ihr euch dort vermehrt und an
Zahl nicht abnehmt! \bibleverse{7}Bemüht euch um die Wohlfahrt der
Stadt\textless sup title=``oder: des Landes''\textgreater✲, wohin ich
euch in die Verbannung habe führen lassen, und betet für sie zum HERRN,
denn auf seiner Wohlfahrt beruht euer eigenes Wohl.‹ \bibleverse{8}Denn
so spricht der HERR der Heerscharen, der Gott Israels: ›Laßt euch von
euren Propheten, die in eurer Mitte leben, und von euren Wahrsagern
nicht täuschen und schenkt auch euren Träumen, die ihr euch träumen
laßt, keinen Glauben! \bibleverse{9}Denn Lügen sind es, die sie euch in
meinem Namen weissagen: ich habe sie nicht gesandt‹ -- so lautet der
Ausspruch des HERRN. \bibleverse{10}Vielmehr, so spricht der HERR: ›Erst
wenn volle siebzig Jahre für Babylon vergangen sind, werde ich mich euer
wieder annehmen und meine Glücksverheißung an euch in Erfüllung gehen
lassen, daß ich euch an diesen Ort zurückbringe. \bibleverse{11}Denn ich
weiß wohl, was für Gedanken ich gegen✲ euch hege‹ -- so lautet der
Ausspruch des HERRN --, ›nämlich Gedanken des Heils und nicht des Leids,
euch eine Zukunft und Hoffnung zu gewähren. \bibleverse{12}Wenn ihr mich
alsdann anruft, so will ich euch antworten, und wenn ihr zu mir betet,
will ich euch erhören, \bibleverse{13}und wenn ihr mich sucht, werdet
ihr mich finden; ja, wenn ihr dann von ganzem Herzen Verlangen nach mir
tragt, \bibleverse{14}so will ich mich von euch finden lassen‹ -- so
lautet der Ausspruch des HERRN -- ›und will euer Schicksal wenden und
euch aus allen Völkern und von allen Orten her sammeln, wohin ich euch
verstoßen habe‹ -- so lautet der Ausspruch des HERRN --, ›und will euch
an den Ort zurückbringen, von wo ich euch habe wegführen lassen!‹«

\hypertarget{cc-die-traurige-lage-der-in-jerusalem-zuruxfcckgebliebenen-volksgenossen-bescheltung-zweier-ehebrecherischer-luxfcgenpropheten-in-babylon}{%
\subparagraph{cc) Die traurige Lage der in Jerusalem zurückgebliebenen
Volksgenossen; Bescheltung zweier ehebrecherischer Lügenpropheten in
Babylon}\label{cc-die-traurige-lage-der-in-jerusalem-zuruxfcckgebliebenen-volksgenossen-bescheltung-zweier-ehebrecherischer-luxfcgenpropheten-in-babylon}}

\bibleverse{15}»Wenn ihr aber sagt: ›Der HERR hat uns (auch) in Babylon
Propheten erstehen lassen‹~-- \bibleverse{16}{[}ja, so sagt der HERR in
betreff des Königs gesprochen, der auf dem Throne Davids sitzt, und in
betreff des gesamten Volkes, das in dieser Stadt hier wohnt, in betreff
eurer Volksgenossen, die nicht mit euch in die Verbannung gezogen
sind~-- \bibleverse{17}so spricht der HERR der Heerscharen: ›Wisset
wohl: ich entbiete gegen sie das Schwert, den Hunger und die Pest und
will sie machen wie ekelhafte Feigen, die so schlecht sind, daß man sie
nicht genießen kann, \bibleverse{18}ich will sie mit dem Schwert, mit
Hunger und mit der Pest verfolgen und sie zum abschreckenden Beispiel
für alle Reiche der Erde machen, zum Fluchwort und zum Entsetzen, zum
Spott und Hohn bei allen Völkern, unter die ich sie verstoßen habe,
\bibleverse{19}zur Strafe dafür, daß sie auf meine Worte nicht gehört
haben‹ -- so lautet der Ausspruch des HERRN --, ›da ich doch meine
Knechte, die Propheten, früh und spät immer wieder zu ihnen gesandt
habe, ohne daß ihr auf sie gehört hättet‹ -- so lautet der Ausspruch des
HERRN. \bibleverse{20}›So vernehmt nun doch ihr das Wort des HERRN, ihr
Weggeführten alle, die ich aus Jerusalem nach Babylon in die Verbannung
habe wegführen lassen.‹{]} --: \bibleverse{21}so hat der HERR der
Heerscharen, der Gott Israels, in betreff Ahabs, des Sohnes Kolajas, und
in betreff Zedekias, des Sohnes Maasejas, gesprochen, die euch Lügen
weissagen in meinem Namen: ›Fürwahr, ich will sie in die Gewalt
Nebukadnezars, des Königs von Babylon, geben, damit er sie vor euren
Augen hinrichten läßt!‹~-- \bibleverse{22}es wird dann bei allen in die
Verbannung weggeführten Judäern, die in Babylon leben, infolge dieses
Geschicks ein Fluchwort in Aufnahme kommen, daß man sagt: ›Der HERR
lasse es dir ergehen wie dem Zedekia und dem Ahab, die der König von
Babylon im Feuer hat rösten lassen!‹ --, \bibleverse{23}›zur Strafe
dafür, daß sie Gottlosigkeit in Israel verübt und mit den Frauen ihrer
Volksgenossen Ehebruch getrieben und in meinem Namen Lügenworte
verkündigt haben, wozu sie keinen Auftrag von mir hatten: mir selbst ist
das wohlbekannt, und ich bin Zeuge dafür!‹ -- so lautet der Ausspruch
des HERRN.«

\hypertarget{d-semajas-beschwerde-uxfcber-das-schreiben-jeremias-vor-der-priesterschaft-in-jerusalem-die-ihm-von-gott-angedrohte-strafe}{%
\paragraph{d) Semajas Beschwerde über das Schreiben Jeremias vor der
Priesterschaft in Jerusalem; die ihm von Gott angedrohte
Strafe}\label{d-semajas-beschwerde-uxfcber-das-schreiben-jeremias-vor-der-priesterschaft-in-jerusalem-die-ihm-von-gott-angedrohte-strafe}}

\bibleverse{24}»Zu Semaja aus Nehalam aber sollst du folgendes sagen:
\bibleverse{25}So hat der HERR der Heerscharen, der Gott Israels,
gesprochen: Weil du in deinem eigenen Namen einen Brief an das gesamte
Volk in Jerusalem und an den Priester Zephanja, den Sohn Maasejas, und
an sämtliche Priester gerichtet hast folgenden Wortlauts:
\bibleverse{26}›Der HERR hat dich an Stelle des Priesters Jojada zum
Priester bestellt, damit ein Aufseher im Tempel des HERRN für jeden
Irrsinnigen, der als Weissager auftritt, vorhanden sei, damit du einen
solchen Menschen in den Block und in Halseisen legest:
\bibleverse{27}nun, warum bist du nicht gegen Jeremia aus Anathoth
vorgegangen, der sich herausnimmt, bei euch als Prophet zu wirken?
\bibleverse{28}Er hat ja doch ein Schreiben an uns nach Babylon
geschickt, worin er sagt: Es wird noch lange dauern; baut (euch also)
Häuser und wohnt in ihnen! Legt Gärten an und genießt ihre
Früchte!‹~\ldots«

\bibleverse{29}Als nun der Priester Zephanja diesen Brief dem Propheten
Jeremia persönlich vorgelesen hatte, \bibleverse{30}erging das Wort des
HERRN an Jeremia folgendermaßen: \bibleverse{31}»Laß allen in die
Verbannung Weggeführten folgende Botschaft zugehen: ›So hat der HERR in
betreff Semajas aus Nehalam gesprochen: Weil Semaja als Prophet bei euch
aufgetreten ist, ohne daß ich ihn gesandt habe, und er euch dazu
verführt hat, euch auf Lügen zu verlassen, \bibleverse{32}darum hat der
HERR so gesprochen: Wisset wohl, ich will Semaja aus Nehalam und seine
Nachkommen dafür büßen lassen: er soll keinen Nachkommen haben, der
inmitten dieses Volkes wohnen bleibt, auch soll er das Glück nicht
miterleben, das ich meinem Volke zugedacht habe!‹ -- so lautet der
Ausspruch des HERRN --; ›denn er hat zum Ungehorsam gegen den HERRN
aufgefordert.‹«

\hypertarget{iii.-trostreden-vom-kuxfcnftigen-heil-kap.-30-33}{%
\subsection{III. Trostreden vom künftigen Heil (Kap.
30-33)}\label{iii.-trostreden-vom-kuxfcnftigen-heil-kap.-30-33}}

\hypertarget{truxf6stlicher-ausblick-auf-das-geschick-des-volkes}{%
\subsubsection{1. Tröstlicher Ausblick auf das Geschick des
Volkes}\label{truxf6stlicher-ausblick-auf-das-geschick-des-volkes}}

\hypertarget{a-einleitung-jeremia-soll-alle-an-ihn-ergangenen-gottesworte-aufzeichnen}{%
\paragraph{a) Einleitung: Jeremia soll alle an ihn ergangenen
Gottesworte
aufzeichnen}\label{a-einleitung-jeremia-soll-alle-an-ihn-ergangenen-gottesworte-aufzeichnen}}

\hypertarget{section-29}{%
\section{30}\label{section-29}}

\bibleverse{1}Das Wort, das vom HERRN an Jeremia erging, lautete
folgendermaßen: \bibleverse{2}So spricht der HERR, der Gott Israels:
»Schreibe dir alle Worte, die ich zu dir geredet habe, in ein Buch!
\bibleverse{3}Denn wisse wohl: es kommt die Zeit« -- so lautet der
Ausspruch des HERRN --, »da werde ich das Geschick meines Volkes Israel
und Juda wenden« -- so spricht der HERR -- »und sie in das Land
zurückführen, das ich ihren Vätern gegeben habe: sie sollen es (wieder)
in Besitz nehmen.«

\hypertarget{b-die-angstvolle-wende}{%
\paragraph{b) Die angstvolle Wende}\label{b-die-angstvolle-wende}}

\bibleverse{4}Dies aber sind die Worte, die der HERR in betreff Israels
und Judas ausgesprochen hat; \bibleverse{5}ja, so hat der HERR
gesprochen: »Banges Geschrei vernehmen wir, Entsetzen voller Unheil!
\bibleverse{6}Fragt doch nach und seht zu, ob auch ein Mannsbild in
Kindesnöte kommen kann! Warum sehe ich denn alle Männer die Hände an die
Hüften stemmen wie Frauen in Kindesnöten und alle Gesichter in
Totenblässe verwandelt? \bibleverse{7}Ach wehe! Gewaltig ist jener Tag,
keiner ist ihm gleich! Und eine Zeit der Not ist's für Jakob, doch er
wird aus ihr gerettet werden!«

\hypertarget{c-die-erluxf6sung-aus-der-not}{%
\paragraph{c) Die Erlösung aus der
Not}\label{c-die-erluxf6sung-aus-der-not}}

\bibleverse{8}»An jenem Tage wird's geschehen« -- so lautet der
Ausspruch des HERRN der Heerscharen --, »da werde ich sein Joch, das auf
deinem Nacken liegt, zerbrechen und deine Fesseln zerreißen; und Fremde
sollen sie nicht länger knechten, \bibleverse{9}sondern dem HERRN, ihrem
Gott, werden sie dienen und ihrem König David, den ich ihnen erwecken
will.

\bibleverse{10}Du aber fürchte dich nicht, mein Knecht Jakob« -- so
lautet der Ausspruch des HERRN --, »und laß dir nicht bange sein,
Israel! Denn wisse wohl: ich will dich erretten aus der Ferne und deine
Kinder aus dem Lande ihrer Gefangenschaft; und Jakob soll heimkehren und
in Ruhe und Sicherheit wohnen, ohne daß jemand ihn aufschreckt;
\bibleverse{11}denn ich bin mit dir« -- so lautet der Ausspruch des
HERRN --, »um dir zu helfen. Denn über alle Völker, unter die ich dich
zerstreut habe, will ich völlige Vernichtung bringen; dich allein will
ich nicht völlig vernichten, sondern dich nur nach Gebühr\textless sup
title=``oder: deiner Verschuldung entsprechend''\textgreater✲ züchtigen;
denn ganz ungestraft will\textless sup title=``oder: kann''\textgreater✲
ich dich nicht lassen.«

\hypertarget{d-israels-untergang-infolge-seiner-suxfcnden}{%
\paragraph{d) Israels Untergang infolge seiner
Sünden}\label{d-israels-untergang-infolge-seiner-suxfcnden}}

\bibleverse{12}Ja, so hat der HERR gesprochen: »Tödlich ist deine Wunde,
unheilbar der Schlag, der dich getroffen! \bibleverse{13}Niemand nimmt
sich deiner Sache an, für dein Geschwür gibt es keine Heilmittel, kein
Verband ist für dich da! \bibleverse{14}Alle deine Liebhaber haben dich
vergessen und kümmern sich nicht um dich; denn wie ein Feind schlägt, so
habe ich dich geschlagen mit erbarmungsloser Züchtigung wegen der Größe
deiner Schuld und wegen der Menge deiner Sünden! \bibleverse{15}Was
schreist du ob deiner Wunde, daß dein Schmerz unheilbar sei? Wegen der
Größe deiner Schuld und wegen der Menge deiner Sünden habe ich dir dies
Leid angetan!«

\hypertarget{e-die-wiederherstellung-des-volkes-und-des-landes}{%
\paragraph{e) Die Wiederherstellung des Volkes und des
Landes}\label{e-die-wiederherstellung-des-volkes-und-des-landes}}

\bibleverse{16}»Darum\textless sup title=``oder: jedoch''\textgreater✲
sollen alle, die dich gefressen haben, wieder gefressen werden und alle
deine Bedränger insgesamt in die Gefangenschaft wandern; die dich
ausgeplündert haben, sollen der Plünderung anheimfallen, und alle, die
dich ausgeraubt haben, will ich der Beraubung preisgeben!
\bibleverse{17}Denn ich will dir einen Verband anlegen und dich von
deinen Wunden heilen« -- so lautet der Ausspruch des HERRN --, »weil man
dich, die du doch Zion bist, ›die Verstoßene‹ genannt hat, ›nach der
niemand fragt‹.« \bibleverse{18}So hat der HERR gesprochen: »Nunmehr
will ich das Geschick der Zelte Jakobs wenden und mich seiner Wohnungen
erbarmen: die Stadt soll auf ihrem Hügel wieder aufgebaut und die
Königsburg\textless sup title=``oder: der Palast''\textgreater✲ in der
alten Weise bewohnt werden! \bibleverse{19}Lobgesänge und der Jubel
fröhlicher Menschen sollen wieder aus ihnen erschallen, und ich will sie
mehren, daß ihre Zahl nicht klein bleibt, und ich will sie zu Ehren
bringen, daß sie nicht länger verachtet sein sollen!
\bibleverse{20}Jakobs Söhne sollen wieder zu mir stehen wie vordem, und
seine Volksgemeinde wird festen Bestand vor mir haben; alle seine
Bedränger aber werde ich zur Rechenschaft ziehen! \bibleverse{21}Sein
Machthaber\textless sup title=``=~mächtiger Fürst''\textgreater✲ soll
aus ihm selbst stammen und sein Herrscher aus seiner Mitte hervorgehen,
und ich will ihm Zutritt zu mir gewähren, daß er mir nahen darf; denn
wer würde sonst wohl sein Leben daransetzen, um mir zu nahen?« -- so
lautet der Ausspruch des HERRN. \bibleverse{22}»So werdet ihr denn mein
Volk sein, und ich will euer Gott sein.«

\hypertarget{f-wiederholung-aus-p2319-20}{%
\paragraph{f) Wiederholung aus
\textbar p23,19-20}\label{f-wiederholung-aus-p2319-20}}

\bibleverse{23}Wisset wohl: ein Sturmwind des HERRN, sein Grimm, bricht
los und wirbelnde Windsbraut, auf das Haupt der Gottlosen fährt sie
nieder! \bibleverse{24}Nicht nachlassen wird der lodernde Zorn des
HERRN, bis er's vollbracht und die Gedanken seines Herzens ausgeführt
hat: am Ende der Tage werdet ihr das schon erkennen!

\hypertarget{g-die-begegnung-gottes-und-israels-in-der-wuxfcste-die-hoffnungsreichen-begruxfcuxdfungsworte}{%
\paragraph{g) Die Begegnung Gottes und Israels in der Wüste; die
hoffnungsreichen
Begrüßungsworte}\label{g-die-begegnung-gottes-und-israels-in-der-wuxfcste-die-hoffnungsreichen-begruxfcuxdfungsworte}}

\hypertarget{section-30}{%
\section{31}\label{section-30}}

\bibleverse{1}»In jener Zeit« -- so lautet der Ausspruch des HERRN --
»will ich der Gott sein für alle Geschlechter Israels, und sie sollen
mein Volk sein.« \bibleverse{2}So hat der HERR gesprochen: »Das Volk der
dem Schwert Entronnenen hat Gnade gefunden in der Wüste: ich will
hingehen, um Israel zu seiner Ruhestätte zu führen!« \bibleverse{3}Von
fern her ist der HERR mir erschienen: »Ja, mit ewiger Liebe habe ich
dich geliebt; darum habe ich dir meine Gnade\textless sup title=``oder:
Güte''\textgreater✲ so lange treu bewahrt. \bibleverse{4}Ich will dich
noch einmal aufbauen, daß du neuerbaut dastehst, Jungfrau Israel! Du
sollst dich noch einmal mit deinen Handpauken schmücken und im Reigen
der Tanzenden ausziehen! \bibleverse{5}Du sollst noch einmal Weingärten
auf den Bergen Samarias anlegen, und die sie angelegt haben, sollen auch
die Früchte genießen. \bibleverse{6}Denn es kommt ein Tag, da werden die
Wächter im Gebirge Ephraim rufen: ›Macht euch auf, laßt uns nach Zion
hinaufziehen zum HERRN, unserm Gott!‹«

\hypertarget{h-der-heimzug}{%
\paragraph{h) Der Heimzug}\label{h-der-heimzug}}

\bibleverse{7}Denn so hat der HERR gesprochen: »Erhebt ein
Freudengeschrei über Jakob und jauchzt über das Haupt\textless sup
title=``=~das erste''\textgreater✲ der Völker! Laßt Lobgesang erschallen
und betet: ›Rette dein Volk, HERR, den Überrest Israels!‹
\bibleverse{8}Seht, ich bringe sie heim aus dem Lande des Nordens und
sammle sie von den Enden der Erde, unter ihnen Blinde und Lahme,
Schwangere und Wöchnerinnen allzumal: als große Volksgemeinde kehren sie
hierher zurück. \bibleverse{9}Mit Weinen kommen sie, und unter
flehentlichen Gebeten geleite ich sie; ich führe sie zu Wasserbächen auf
ebenem Wege, auf dem sie nicht straucheln sollen; denn ich bin (jetzt
wieder) Israels Vater geworden, und Ephraim ist mein erstgeborener
Sohn!«

\hypertarget{i-in-der-heimat}{%
\paragraph{i) In der Heimat}\label{i-in-der-heimat}}

\bibleverse{10}Vernehmt das Wort des HERRN, ihr Völker, und verkündet in
den fernsten Meeresländern folgende Botschaft: »Er, der Israel zerstreut
hat, sammelt es wieder und hütet es wie ein Hirt seine Herde!«
\bibleverse{11}Denn der HERR hat Jakob losgekauft und ihn befreit aus
der Gewalt dessen, der stärker war als er. \bibleverse{12}So werden sie
denn kommen und auf Zions Höhe jubeln und strahlen vor Freude über die
Segensgaben des HERRN, über das Korn und den Most und das Öl, über die
jungen Schafe und Rinder; und ihre Seele wird sein wie ein
wohlbewässerter Garten, und sie werden fortan nicht mehr zu darben
brauchen. \bibleverse{13}Alsdann wird die Jungfrau sich wieder am
Reigentanz erfreuen, Jünglinge und Greise allzumal. »Ja, ich will ihre
Trauer in Freude verwandeln und sie trösten und fröhlich machen nach
ihrem Leid. \bibleverse{14}Und ich will das Herz der Priester mit fetter
Speise laben, und mein Volk soll sich an meinen Segensgaben sättigen!«
-- so lautet der Ausspruch des HERRN.

\hypertarget{k-rahels-schmerz-und-klage-ephraims-reue-und-gottes-liebe-aufforderung-zur-ruxfcckkehr}{%
\paragraph{k) Rahels Schmerz und Klage; Ephraims Reue und Gottes Liebe;
Aufforderung zur
Rückkehr}\label{k-rahels-schmerz-und-klage-ephraims-reue-und-gottes-liebe-aufforderung-zur-ruxfcckkehr}}

\hypertarget{aa-rahel-weint-um-ihre-kinder-und-wird-von-gott-getruxf6stet}{%
\subparagraph{aa) Rahel weint um ihre Kinder und wird von Gott
getröstet}\label{aa-rahel-weint-um-ihre-kinder-und-wird-von-gott-getruxf6stet}}

\bibleverse{15}So hat der HERR gesprochen: »Horch! In Rama wird Wehklage
laut, bitterliches Weinen! Rahel weint um ihre Kinder, will sich nicht
trösten lassen wegen ihrer Kinder: ach, sie sind nicht mehr da!«
\bibleverse{16}Doch so hat der HERR gesprochen: »Wehre deiner Stimme das
Klagen und deinen Augen die Tränen! Denn es gibt noch einen Lohn für
deine Mühsal« -- so lautet der Ausspruch des HERRN --; »denn sie sollen
aus dem Lande des Feindes wieder heimkehren! \bibleverse{17}Ja, es ist
noch eine Hoffnung für deine Zukunft vorhanden« -- so lautet der
Ausspruch des HERRN --; »denn deine Kinder kehren in ihre Heimat
zurück!«

\hypertarget{bb-ephraims-buuxdfe-und-gottes-gnade}{%
\subparagraph{bb) Ephraims Buße und Gottes
Gnade}\label{bb-ephraims-buuxdfe-und-gottes-gnade}}

\bibleverse{18}»Ich habe wohl gehört, wie Ephraim klagte: ›Du hast mich
gezüchtigt, und ich habe Zucht gelernt wie ein nicht ans Joch gewöhnter
Jungstier: o laß mich heimkehren, so will ich mich bekehren! Du bist ja
doch der HERR, mein Gott! \bibleverse{19}Denn seitdem ich mich von dir
abgewandt habe, fühle ich Reue; und nachdem ich zur Erkenntnis gekommen
bin, schlage ich mich auf die Hüften: ich schäme mich, stehe zerknirscht
da, denn ich habe die Schmach meiner Jugend zu büßen!‹
\bibleverse{20}Ist mir denn Ephraim ein so teurer Sohn oder mein
Lieblingskind, daß, sooft ich ihm auch gedroht habe, ich seiner doch
immer wieder freundlich gedenken muß? Darum schlägt mein Herz so warm
für ihn: ich muß mich seiner erbarmen!« -- so lautet der Ausspruch des
HERRN.

\hypertarget{cc-gottes-aufforderung-an-israel-zur-ruxfcckkehr}{%
\subparagraph{cc) Gottes Aufforderung an Israel zur
Rückkehr}\label{cc-gottes-aufforderung-an-israel-zur-ruxfcckkehr}}

\bibleverse{21}»Stelle dir Wegweiser auf, setze dir Merksteine hin! Gib
wohl acht auf die Straße, auf den Weg, den du einst gezogen bist! Kehre
heim, Jungfrau Israel, kehre heim zu deinen Städten hier!
\bibleverse{22}Wie lange willst du dich noch hierhin und dorthin
wenden\textless sup title=``=~dich spröde gebärden''\textgreater✲, du
abtrünnige Tochter? Der HERR schafft ja doch etwas Neues im
Lande\textless sup title=``oder: auf Erden''\textgreater✲: das Weib
umwirbt\textless sup title=``oder: muß umwerben''\textgreater✲ den
Mann.«

\hypertarget{l-drei-spruxfcche-verschiedenartigen-inhalts}{%
\paragraph{l) Drei Sprüche verschiedenartigen
Inhalts}\label{l-drei-spruxfcche-verschiedenartigen-inhalts}}

\hypertarget{aa-gottes-segen-nach-der-ruxfcckkehr-der-zerstreuten}{%
\subparagraph{aa) Gottes Segen nach der Rückkehr der
Zerstreuten}\label{aa-gottes-segen-nach-der-ruxfcckkehr-der-zerstreuten}}

\bibleverse{23}So hat der HERR der Heerscharen, der Gott Israels,
gesprochen: »Noch wird man im Lande Juda und in seinen Städten, wenn ich
ihr Geschick gewandt habe, diesen Gruß aussprechen: ›Der HERR segne
dich, du Gefilde der Gerechtigkeit, du heiliger Berg!‹
\bibleverse{24}Und Juda wird darin wohnen samt allen seinen Städten ohne
Ausnahme, die Ackerleute und solche, die mit der Herde umherziehen;
\bibleverse{25}denn ich will die lechzenden Seelen reichlich tränken und
jegliche schmachtende Seele sättigen!« \bibleverse{26}Darüber erwachte
ich und schaute mich um, und mein Schlaf war mir köstlich gewesen.

\hypertarget{bb-gott-will-bauen-und-pflanzen}{%
\subparagraph{bb) Gott will bauen und
pflanzen}\label{bb-gott-will-bauen-und-pflanzen}}

\bibleverse{27}»Wisset wohl: es kommt die Zeit« -- so lautet der
Ausspruch des HERRN --, »da will ich über das Haus Israel und über das
Haus Juda eine Saat von Menschen und eine Saat von Vieh ausstreuen;
\bibleverse{28}und wie ich die Augen offen über ihnen gehalten habe, um
auszureißen und zu zerstören, um niederzureißen und zu verderben und
Unheil anzurichten, ebenso will ich alsdann über ihnen wachen, um
aufzubauen und zu pflanzen!« -- so lautet der Ausspruch des HERRN.

\hypertarget{cc-das-sprichwort-von-den-herlingen-soll-auuxdfer-gebrauch-kommen}{%
\subparagraph{cc) Das Sprichwort von den Herlingen soll außer Gebrauch
kommen}\label{cc-das-sprichwort-von-den-herlingen-soll-auuxdfer-gebrauch-kommen}}

\bibleverse{29}In jenen Tagen wird man nicht mehr sagen: »Die Väter
haben Herlinge\textless sup title=``=~unreife oder: saure
Trauben''\textgreater✲ gegessen, und den Kindern werden die Zähne stumpf
davon«; \bibleverse{30}sondern ein jeder wird um seiner eigenen
Verschuldung willen sterben: nur wer Herlinge ißt, dem sollen die
(eigenen) Zähne stumpf werden\textless sup title=``vgl. Hes
18,2-4''\textgreater✲.

\hypertarget{m-der-neue-gnadenbund-mit-beiden-huxe4usern-israels}{%
\paragraph{m) Der neue Gnadenbund mit beiden Häusern
Israels}\label{m-der-neue-gnadenbund-mit-beiden-huxe4usern-israels}}

\bibleverse{31}»Wisset wohl: es kommt die Zeit« -- so lautet der
Ausspruch des HERRN --, »da will ich mit dem Hause Israel und mit dem
Hause Juda einen neuen Bund schließen, \bibleverse{32}nicht einen
solchen Bund, wie ich ihn mit ihren Vätern damals geschlossen habe, als
ich sie bei der Hand nahm, um sie aus Ägyptenland wegzuführen, einen
Bund, den sie gebrochen haben, wiewohl ich Herrenrecht über sie hatte!«
-- so lautet der Ausspruch des HERRN. \bibleverse{33}»Nein, darin soll
der Bund bestehen, den ich mit dem Hause Israel nach dieser Zeit
schließen werde« -- so lautet der Ausspruch des HERRN --: »Ich will mein
Gesetz in ihr Inneres hineinlegen und es ihnen ins Herz schreiben und
will dann ihr Gott sein, und sie sollen mein Volk sein.
\bibleverse{34}Da braucht dann niemand mehr seinem Genossen und niemand
seinem Bruder Belehrung zu erteilen und ihm vorzuhalten: ›Lernt den
HERRN erkennen!‹, denn sie werden mich allesamt erkennen, die Kleinsten
wie die Größten« -- so lautet der Ausspruch des HERRN --; »denn ich will
ihnen ihre Schuld vergeben und ihrer Sünde nicht mehr gedenken!«

\hypertarget{n-der-ewige-bestand-des-heils}{%
\paragraph{n) Der ewige Bestand des
Heils}\label{n-der-ewige-bestand-des-heils}}

\bibleverse{35}So hat der HERR gesprochen, der die Sonne zur Leuchte am
Tage bestellt hat, die Ordnungen\textless sup title=``=~festgeordneten
Erscheinungen''\textgreater✲ des Mondes und der Sterne zur Erleuchtung
bei Nacht, der das Meer aufwühlt, so daß seine Wogen brausen -- HERR der
Heerscharen ist sein Name --: \bibleverse{36}»Wenn diese festen
Ordnungen jemals vor mir zu bestehen aufhören« -- so lautet der
Ausspruch des HERRN --, »dann (erst) soll auch die Nachkommenschaft
Israels aufhören, ein Volk vor meinen Augen zu sein für alle Zeiten!«
\bibleverse{37}So hat der HERR gesprochen: »So wenig der Himmel droben
ausgemessen und die Grundfesten der Erde drunten durchspäht✲ werden
können, so wenig will ich auch die gesamte Nachkommenschaft Israels
verwerfen wegen alles dessen, was sie begangen haben« -- so lautet der
Ausspruch des HERRN.

\hypertarget{o-wiederherstellung-jerusalems-zu-einer-vollkommen-heiligen-stadt}{%
\paragraph{o) Wiederherstellung Jerusalems zu einer vollkommen heiligen
Stadt}\label{o-wiederherstellung-jerusalems-zu-einer-vollkommen-heiligen-stadt}}

\bibleverse{38}»Wisset wohl: es kommt die Zeit« -- so lautet der
Ausspruch des HERRN --, »da wird diese Stadt für den HERRN wieder
aufgebaut werden vom Turm Hananeel\textless sup title=``=~Neh
3,1''\textgreater✲ bis zum Ecktor\textless sup title=``2.Kön
14,13''\textgreater✲; \bibleverse{39}und weiter wird die Meßschnur (von
da) geradeaus über den Hügel Gareb fortlaufen und sich dann nach Goah
wenden. \bibleverse{40}Und das ganze Tal der Leichen und der Opferasche
und das gesamte Feld bis an den Bach Kidron, bis an die Ecke des
Roßtores\textless sup title=``Neh 3,27-28''\textgreater✲ gegen Osten,
wird dem HERRN heilig sein; es wird dort alsdann nie wieder eingerissen
und zerstört werden in Ewigkeit!«

\hypertarget{jeremias-ackerkauf-in-anathoth-sein-gebet-und-die-guxf6ttliche-antwort}{%
\subsubsection{2. Jeremias Ackerkauf in Anathoth, sein Gebet und die
göttliche
Antwort}\label{jeremias-ackerkauf-in-anathoth-sein-gebet-und-die-guxf6ttliche-antwort}}

\hypertarget{a-jeremia-kauft-als-gefangener-nach-gottes-weisung-einen-acker-in-anathoth}{%
\paragraph{a) Jeremia kauft als Gefangener nach Gottes Weisung einen
Acker in
Anathoth}\label{a-jeremia-kauft-als-gefangener-nach-gottes-weisung-einen-acker-in-anathoth}}

\hypertarget{section-31}{%
\section{32}\label{section-31}}

\bibleverse{1}(Dies ist) das Wort, das vom HERRN an Jeremia erging im
zehnten Regierungsjahr des judäischen Königs Zedekia -- dieses Jahr war
das achtzehnte Regierungsjahr Nebukadnezars. \bibleverse{2}Damals
belagerte nämlich das Heer des Königs von Babylon Jerusalem, und der
Prophet Jeremia wurde im Wachthof, der zum Palast des Königs von Juda
gehörte, in Haft gehalten. \bibleverse{3}Denn Zedekia, der König von
Juda, hatte ihn dort gefangengesetzt mit dem Vorhalt: »Warum trittst du
als Prophet auf und sagst: ›So hat der HERR gesprochen: Fürwahr, ich
will diese Stadt in die Gewalt des Königs von Babylon geben, daß er sie
erobert; \bibleverse{4}auch Zedekia, der König von Juda, wird den Händen
der Chaldäer nicht entrinnen, sondern unfehlbar dem König von Babylon in
die Hände übergeben werden und von Mund zu Mund mit ihm reden und ihm
Auge in Auge gegenüberstehen; \bibleverse{5}der wird Zedekia dann nach
Babylon bringen lassen, und dort wird er bleiben, bis ich mich seiner
wieder annehme -- so lautet der Ausspruch des HERRN. Wenn ihr also mit
den Chaldäern kämpft, werdet ihr kein Glück haben.‹«

\bibleverse{6}Da sagte Jeremia: »Das Wort des HERRN ist an mich
folgendermaßen ergangen: \bibleverse{7}Demnächst wird Hanamel, der Sohn
deines Oheims Sallum, zu dir kommen mit der Aufforderung: ›Kaufe dir
meinen Acker, der bei Anathoth liegt! Denn du hast das
Löserecht\textless sup title=``oder: Vorkaufsrecht''\textgreater✲ und
bist zum Kauf verpflichtet.‹« \bibleverse{8}Und wirklich kam Hanamel,
der Sohn meines Oheims, zu mir in den Wachthof, wie der HERR mir
angekündigt hatte, und sagte zu mir: »Kaufe doch meinen Acker, der bei
Anathoth im Stamm Benjamin liegt, denn dir steht das Besitz- und
Vorkaufsrecht zu; kaufe ihn dir!« Da wurde mir klar, daß es eine Weisung
vom HERRN gewesen war; \bibleverse{9}und so kaufte ich denn den Acker,
der bei Anathoth lag, von meinem Vetter Hanamel und wog ihm das Geld
dar, siebzehn Schekel Silber. \bibleverse{10}Hierauf brachte ich einen
Kaufvertrag zu Papier, versah ihn mit einem Siegel, ließ ihn durch
Zeugen beglaubigen und wog ihm das Geld auf der Waage dar.
\bibleverse{11}Hierauf nahm ich den Kaufvertrag, den versiegelten und
mit der Abmachung und den Bedingungen versehenen und auch den offenen,
\bibleverse{12}und übergab den Kaufvertrag Baruch, dem Sohne Nerijas,
des Sohnes Mahsejas, in Gegenwart meines Vetters Hanamel und in
Gegenwart der Zeugen, die den Kaufvertrag unterschrieben hatten, und in
Gegenwart aller Judäer, die im Wachthof anwesend waren.
\bibleverse{13}Hierauf erteilte ich dem Baruch in ihrer Gegenwart
folgenden Auftrag: \bibleverse{14}»So hat der HERR der Heerscharen, der
Gott Israels, gesprochen: ›Nimm diese Urkunden, und zwar sowohl diesen
versiegelten Kaufvertrag als auch dieses offene Schriftstück, und lege
sie in ein irdenes Gefäß, damit sie lange Zeit erhalten bleiben!‹
\bibleverse{15}Denn so hat der HERR der Heerscharen, der Gott Israels,
gesprochen: ›Man wird künftig wieder Häuser, Äcker und Weinberge in
diesem Lande kaufen!‹«

\hypertarget{b-jeremias-gebet-und-bitte-um-aufkluxe4rung}{%
\paragraph{b) Jeremias Gebet und Bitte um
Aufklärung}\label{b-jeremias-gebet-und-bitte-um-aufkluxe4rung}}

\bibleverse{16}Nachdem ich so Baruch, dem Sohne Nerijas, den Kaufvertrag
übergeben hatte, richtete ich folgendes Gebet an den HERRN:
\bibleverse{17}»Ach HERR, mein Gott! Du bist's, der den Himmel und die
Erde durch deine große Kraft und deinen ausgestreckten Arm geschaffen
hat: dir ist kein Ding unmöglich. \bibleverse{18}Du übst
Gnade\textless sup title=``oder: Güte''\textgreater✲ an Tausenden und
läßt die Strafe für die Schuld der Väter in den Schoß\textless sup
title=``=~auf das Haupt''\textgreater✲ ihrer Kinder nach ihnen fallen,
du großer, starker Gott, dessen Name ›HERR der Heerscharen‹ ist,
\bibleverse{19}groß an Rat und mächtig an Tat, du, dessen Augen offen
stehen über allen Wegen der Menschenkinder, damit du einem jeden nach
seinem Wandel und nach den Früchten seines Tuns vergiltst.
\bibleverse{20}Du hast Zeichen und Wunder im Lande Ägypten und bis auf
diesen Tag sowohl an Israel als auch an den (anderen) Menschen gewirkt
und dir dadurch einen Namen gemacht, wie es heute klar zu Tage liegt.
\bibleverse{21}Du hast dein Volk Israel aus dem Lande Ägypten ausziehen
lassen unter Zeichen und Wundern, mit starker Hand, mit hocherhobenem
Arm und großem Schrecken, \bibleverse{22}und hast ihnen dies Land
gegeben, dessen Besitz du ihren Vätern zugeschworen hattest, ein Land,
das von Milch und Honig überfließt. \bibleverse{23}Als sie aber
hineingekommen waren und es in Besitz genommen hatten, hörten sie nicht
auf deine Weisungen und lebten nicht nach deinem Gesetz und taten nichts
von allem, was du ihnen zu tun geboten hattest; darum hast du ihnen all
dies Unglück widerfahren lassen. \bibleverse{24}Ach, die
Belagerungswälle sind schon bis an die Stadt herangekommen, um sie zu
erobern, und die Stadt ist der Gewalt der Chaldäer, die sie belagern,
durch das Schwert, durch den Hunger und die Pest preisgegeben, und was
du angedroht hast, ist eingetreten: du siehst es ja selbst.
\bibleverse{25}Und doch hast du, HERR, mein Gott, mir geboten: ›Kaufe
dir den Acker für Geld und ziehe Zeugen hinzu!‹, und dabei ist die Stadt
schon der Gewalt der Chaldäer preisgegeben!«

\hypertarget{c-gottes-antwort-aufkluxe4rung-und-heilsverkuxfcndigung}{%
\paragraph{c) Gottes Antwort (Aufklärung und
Heilsverkündigung)}\label{c-gottes-antwort-aufkluxe4rung-und-heilsverkuxfcndigung}}

\bibleverse{26}Da erging das Wort des HERRN an Jeremia folgendermaßen:
\bibleverse{27}»Fürwahr, ich bin der HERR, der Gott alles Fleisches:
sollte mir irgend etwas unmöglich sein? \bibleverse{28}Darum spricht der
HERR so: Allerdings lasse ich diese Stadt in die Gewalt der Chaldäer und
zwar in die Gewalt des babylonischen Königs Nebukadnezar fallen, der sie
erobern soll; \bibleverse{29}und die Chaldäer, die diese Stadt belagern,
werden in sie eindringen und Feuer an diese Stadt legen und sie
einäschern, eben die Häuser, auf deren Dächern man dem Baal Rauchopfer
dargebracht und fremden Göttern Trankspenden ausgegossen hat, um mich zu
erbittern. \bibleverse{30}Denn die Israeliten und die Judäer haben von
ihrer Jugend an immer nur das getan, was mir mißfällt; ja, die
Israeliten haben mich immer nur erbittert durch die Machwerke ihrer
Hände!« -- so lautet der Ausspruch des HERRN --; \bibleverse{31}»ja ein
Gegenstand des Zornes und des Grimms ist diese Stadt für mich vom Tage
ihrer Gründung an bis auf den heutigen Tag gewesen, so daß ich sie mir
aus den Augen schaffen muß \bibleverse{32}wegen all des Bösen, das die
Israeliten und die Judäer mir zum Ärgernis verübt haben, sie selbst,
ihre Könige und ihre Fürsten\textless sup title=``oder:
Oberen''\textgreater✲, ihre Priester und ihre Propheten, sowohl die
Männer von Juda als die Bewohner Jerusalems. \bibleverse{33}Sie haben
mir den Rücken zugekehrt, statt auf mich zu blicken; und obgleich ich
sie früh und spät immer wieder habe belehren lassen, haben sie doch
nicht darauf gehört und keine Zucht annehmen wollen.
\bibleverse{34}Nein, sie haben ihre scheußlichen Götzen sogar in dem
Hause, das meinen Namen trägt, aufgestellt, um es dadurch zu entweihen,
\bibleverse{35}und haben dem Baal die Opferstätten im Tal Ben-Hinnom
erbaut, um ihre Söhne und Töchter dort dem Moloch als Opfer zu
verbrennen, was ich ihnen niemals geboten habe und was mir nie in den
Sinn gekommen ist, daß sie solche Greuel verüben sollten, um Juda zur
Sünde zu verführen.«

\bibleverse{36}Nun aber -- trotz alledem spricht der HERR, der Gott
Israels, in bezug auf diese Stadt, von der ihr sagt, sie sei der Gewalt
des babylonischen Königs durch das Schwert, durch den Hunger und durch
die Pest preisgegeben, folgendermaßen: \bibleverse{37}»Fürwahr, ich will
sie\textless sup title=``d.h. die Judäer''\textgreater✲ aus allen
Ländern, wohin ich sie in meinem Zorn und Grimm und in heftiger Ungnade
verstoßen habe, wieder sammeln und sie an diesen Ort zurückbringen und
sie hier in Sicherheit wohnen lassen. \bibleverse{38}Sie sollen dann
mein Volk sein, und ich will ihr Gott sein \bibleverse{39}und ihnen
einerlei Sinn und einerlei Wandel verleihen, auf daß sie mich allezeit
fürchten, zu ihrem eigenen Heil und zum Segen ihrer Kinder nach ihnen.
\bibleverse{40}Und ich will einen ewigen Bund mit ihnen schließen, daß
ich niemals von ihnen ablassen will, ihnen Gutes zu erweisen, und ich
will ihnen Furcht vor mir ins Herz legen, damit sie mir nie wieder
untreu werden. \bibleverse{41}Ich werde dann meine Freude an ihnen
haben, so daß ich ihnen Liebe erweise, und will sie in dieses Land
einpflanzen in Treue, mit ganzem Herzen und mit ganzer Seele.«
\bibleverse{42}Denn so spricht der HERR: »Wie ich all dieses große
Unheil über dieses Volk gebracht habe, ebenso will ich ihnen all das
Gute widerfahren lassen, das ich ihnen jetzt verheiße.
\bibleverse{43}Denn es sollen wieder Äcker gekauft werden in diesem
Lande, von dem ihr sagt, es sei eine Einöde ohne Menschen und ohne Vieh
und sei der Gewalt der Chaldäer preisgegeben. \bibleverse{44}Man wird
wieder Äcker für Geld kaufen und Kaufverträge schreiben und versiegeln
und sie durch Zeugen beglaubigen lassen im Stamm Benjamin wie im Bezirk
Jerusalems und in den Ortschaften Judas, sowohl in den Ortschaften des
Berglandes als auch in denen der Niederung und in den Ortschaften des
Südlandes; denn ich werde ihr Geschick wenden!« -- so lautet der
Ausspruch des HERRN.

\hypertarget{weitere-weissagungen-von-der-wiederherstellung-des-volkes-des-landes-und-der-stadt}{%
\subsubsection{3. Weitere Weissagungen von der Wiederherstellung des
Volkes, des Landes und der
Stadt}\label{weitere-weissagungen-von-der-wiederherstellung-des-volkes-des-landes-und-der-stadt}}

\hypertarget{a-heilsweissagungen-fuxfcr-jerusalem-und-juda}{%
\paragraph{a) Heilsweissagungen für Jerusalem und
Juda}\label{a-heilsweissagungen-fuxfcr-jerusalem-und-juda}}

\hypertarget{section-32}{%
\section{33}\label{section-32}}

\bibleverse{1}Hierauf erging das Wort des HERRN zum zweitenmal an
Jeremia, während er noch im Wachthof eingeschlossen gehalten wurde,
folgendermaßen: \bibleverse{2}»So spricht der HERR, der (alles)
ausführt, der HERR, der es ersinnt, um es auch zu verwirklichen, ›HERR‹
ist sein Name: \bibleverse{3}Rufe mich an, so will ich dich erhören und
dir große und unglaubliche\textless sup title=``oder:
geheime''\textgreater✲ Dinge kundtun, von denen du bisher nichts gewußt
hast. \bibleverse{4}Denn so spricht der HERR, der Gott Israels, in
betreff der Häuser dieser Stadt und in betreff der Paläste der Könige
von Juda, die niedergerissen worden sind, um gegen die Belagerungswälle
und zu Kriegszwecken verwandt zu werden, \bibleverse{5}zum Kampf gegen
die Chaldäer und um sie mit den Leichen der Menschen anzufüllen, die ich
in meinem Zorn und Grimm erschlagen habe, weil ich mein Angesicht vor
dieser Stadt um all ihrer Bosheit willen verhüllt hatte --:
\bibleverse{6}Wisse wohl: ich will ihr einen Verband und Heilmittel
auflegen und ihr Heilung verschaffen und ihnen eine Fülle von Glück und
Sicherheit erscheinen lassen. \bibleverse{7}Und ich will das Geschick
Judas und das Geschick Israels wenden und sie wieder aufbauen✲ wie
vordem \bibleverse{8}und will sie von all ihrer Verschuldung reinigen,
mit der sie gegen mich gesündigt haben, und will ihnen alle Missetaten
vergeben, die sie gegen mich begangen haben und durch die sie mir untreu
geworden sind. \bibleverse{9}Dann wird Jerusalem für mich ein
Freudenname werden, ein Ruhm und eine Verherrlichung bei allen Völkern
der Erde, die, wenn sie von all dem Guten hören, das ich dieser Stadt
erweise, erschrecken und erzittern werden über all das Gute und über all
das Glück, das ich ihr erweisen werde!«~--

\bibleverse{10}So hat der HERR gesprochen: »An diesem Orte, von dem ihr
sagt, er sei verödet, menschenleer und ohne Vieh in den Ortschaften
Judas und auf den Straßen Jerusalems, die jetzt vereinsamt sind,
menschenleer, ohne Bewohner und ohne Vieh, \bibleverse{11}da wird man
künftig wieder Freudenrufe und laute Fröhlichkeit vernehmen, den Jubel
des Bräutigams und den Jubel der Braut, den Jubel derer, die da rufen:
›Lobet den HERRN der Heerscharen! Denn gütig ist der HERR, und seine
Gnade währet ewiglich!‹, und den Jubel derer, welche Dankopfer im Tempel
des HERRN darbringen. Denn ich will das Geschick des Landes wenden, daß
es wieder wird wie vordem!« -- so hat der HERR gesprochen.~--

\bibleverse{12}So hat der HERR der Heerscharen gesprochen: »An diesem
Ort, der jetzt verödet ist, menschenleer und ohne Vieh, und in allen
seinen Ortschaften soll künftig wieder eine Trift\textless sup
title=``oder: Weide''\textgreater✲ für Hirten sein, die ihre Herden sich
lagern lassen. \bibleverse{13}In den Ortschaften des Berglandes und in
denen der Niederung, in den Ortschaften des Südlandes und im Stamme
Benjamin, im Bezirk Jerusalems und in den übrigen Ortschaften Judas
sollen künftig die Herden wieder unter den Händen des (Hirten), der sie
zählt, vorüberziehen!« -- so hat der HERR gesprochen.

\hypertarget{b-weissagung-vom-davidssprossen-der-zukunft-und-vom-ewigen-bestand-des-volkes-des-davidischen-kuxf6nigtums-und-des-levitischen-priestertums}{%
\paragraph{b) Weissagung vom Davidssprossen der Zukunft und vom ewigen
Bestand des Volkes, des davidischen Königtums und des levitischen
Priestertums}\label{b-weissagung-vom-davidssprossen-der-zukunft-und-vom-ewigen-bestand-des-volkes-des-davidischen-kuxf6nigtums-und-des-levitischen-priestertums}}

\bibleverse{14}»Wisset wohl: es kommt die Zeit« -- so lautet der
Ausspruch des HERRN --, »da will ich die Segensverheißung, die ich über
das Haus Israel und über das Haus Juda ausgesprochen habe, in Erfüllung
gehen lassen. \bibleverse{15}In jenen Tagen und zu jener Zeit will ich
dem David einen Sproß der Gerechtigkeit\textless sup title=``d.h. einen
rechten =~rechtbeschaffenen Sprößling''\textgreater✲ ersprießen lassen,
der Recht und Gerechtigkeit im Lande walten läßt. \bibleverse{16}In
jenen Tagen wird Juda Rettung erlangen und Jerusalem in Sicherheit
wohnen, und der Name, den man der Stadt beilegt, wird lauten: ›der HERR
unsere Gerechtigkeit‹\textless sup title=``oder: unser
Heil''\textgreater✲.« \bibleverse{17}Denn so hat der HERR gesprochen:
»Nie soll es dem David an einem (Nachkommen) fehlen, der auf dem Throne
des Hauses Israel sitzt; \bibleverse{18}und auch den levitischen
Priestern soll es vor meinem Angesicht nie an einem (Nachkommen) fehlen,
der Brandopfer darbringt und Speisopfer in Rauch aufgehen läßt und
Schlachtopfer zurichtet immerdar!«

\bibleverse{19}Darauf erging das Wort des HERRN an Jeremia
folgendermaßen: \bibleverse{20}»So spricht der HERR: So wenig ihr meinen
Bund mit dem Tage und meinen Bund mit der Nacht aufheben könnt, so daß
Tag und Nacht nicht mehr zu ihrer Zeit eintreten würden,
\bibleverse{21}ebenso wenig wird auch mein Bund mit meinem Knecht David
aufgehoben werden, daß er keinen Sohn✲ mehr haben sollte, der als König
auf seinem Throne säße, und ebenso wenig mein Bund mit meinen Dienern,
den priesterlichen Leviten\textless sup title=``oder: den levitischen
Priestern''\textgreater✲! \bibleverse{22}Wie das Sternenheer am Himmel
nicht gezählt und der Sand am Meer nicht gemessen werden kann, ebenso
unzählbar will ich die Nachkommenschaft meines Knechtes David und die
Leviten machen, die meinen Dienst versehen.«

\bibleverse{23}Weiter erging das Wort des HERRN an Jeremia
folgendermaßen: \bibleverse{24}»Hast du nicht darauf geachtet, was diese
Leute da behaupten, wenn sie sagen: ›Die beiden
Geschlechter\textless sup title=``oder: Häuser''\textgreater✲, die der
HERR einst erwählt hatte, die hat er jetzt verworfen!‹, und wie sie mein
Volk verachten, so daß es in ihren Augen gar kein Volk mehr ist?
\bibleverse{25}So spricht der HERR: So gewiß mein Bund mit Tag und Nacht
besteht, so gewiß ich die Ordnungen\textless sup title=``d.h.
Naturgesetze''\textgreater✲ des Himmels und der Erde festgesetzt habe,
\bibleverse{26}ebenso gewiß will ich auch die Nachkommenschaft Jakobs
und meines Knechtes David nicht verwerfen, daß ich aus seiner
Nachkommenschaft keine Herrscher mehr für die Nachkommenschaft Abrahams,
Isaaks und Jakobs entnehmen sollte; denn ich werde ihr Geschick wenden
und mich ihrer erbarmen!«

\hypertarget{iv.-verschiedenes-aus-der-zeit-vor-der-zerstuxf6rung-jerusalems-kap.-34-39}{%
\subsection{IV. Verschiedenes aus der Zeit vor der Zerstörung Jerusalems
(Kap.
34-39)}\label{iv.-verschiedenes-aus-der-zeit-vor-der-zerstuxf6rung-jerusalems-kap.-34-39}}

\hypertarget{ankuxfcndigung-des-schicksals-zedekias}{%
\subsubsection{1. Ankündigung des Schicksals
Zedekias}\label{ankuxfcndigung-des-schicksals-zedekias}}

\hypertarget{section-33}{%
\section{34}\label{section-33}}

\bibleverse{1}Das Wort, das vom HERRN an Jeremia erging, als
Nebukadnezar, der König von Babylon, mit seiner ganzen Heeresmacht und
allen Königreichen der Erde, soweit sie seiner Herrschaft unterworfen
waren, und mit allen (übrigen) Völkern Jerusalem belagerte und alle
zugehörigen Städte bekriegte, lautete folgendermaßen: \bibleverse{2}»So
spricht der HERR, der Gott Israels: Gehe hin und verkünde dem judäischen
Könige Zedekia folgende Botschaft: ›So hat der HERR gesprochen: Wisse
wohl: ich gebe diese Stadt in die Gewalt des Königs von Babylon, damit
er sie in Flammen aufgehen läßt. \bibleverse{3}Und auch du wirst seiner
Hand nicht entrinnen, sondern unfehlbar ergriffen und in seine Hand
überliefert werden; du wirst dann dem Könige von Babylon Auge in Auge
gegenüberstehen, und er wird Mund gegen Mund mit dir reden; darauf wirst
du nach Babylon kommen.‹ \bibleverse{4}Vernimm jedoch das Wort des
HERRN, Zedekia, König von Juda! Folgende Verheißung hat der HERR über
dich ausgesprochen: ›Du sollst den Tod nicht durch das Schwert erleiden;
\bibleverse{5}in Frieden sollst du sterben! Und wie man deinen Vätern,
den früheren Königen, deinen Vorgängern, Leichenbrände veranstaltet hat,
so wird man auch dir ein Totenfeuer anzünden und dich mit dem Klagerufe
betrauern: Ach, Gebieter!, denn ich habe es so bestimmt! -- so lautet
der Ausspruch des HERRN.‹«

\bibleverse{6}Alle diese Worte verkündete der Prophet Jeremia dem
judäischen König Zedekia in Jerusalem, \bibleverse{7}während das Heer
des Königs von Babylon Jerusalem und alle noch übriggebliebenen Städte
von Juda, nämlich Lachis und Aseka, belagerte; denn diese waren die
einzigen von den festen Plätzen Judas, die noch standhielten.

\hypertarget{strafrede-jeremias-und-strafandrohung-gottes-wegen-des-an-freigelassenen-hebruxe4ischen-sklaven-in-jerusalem-begangenen-treubruchs}{%
\subsubsection{2. Strafrede Jeremias und Strafandrohung Gottes wegen des
an freigelassenen hebräischen Sklaven in Jerusalem begangenen
Treubruchs}\label{strafrede-jeremias-und-strafandrohung-gottes-wegen-des-an-freigelassenen-hebruxe4ischen-sklaven-in-jerusalem-begangenen-treubruchs}}

\hypertarget{a-darlegung-des-sachverhalts}{%
\paragraph{a) Darlegung des
Sachverhalts}\label{a-darlegung-des-sachverhalts}}

\bibleverse{8}(Dies ist) das Wort, das vom HERRN an Jeremia erging,
nachdem der König Zedekia sich durch ein Abkommen mit dem gesamten Volk
in Jerusalem dazu verpflichtet hatte, eine Freilassung für sie ausrufen
zu lassen, \bibleverse{9}daß nämlich ein jeder seinen hebräischen
Sklaven und ein jeder seine hebräische Sklavin frei ziehen lassen solle,
so daß niemand künftig einen judäischen Volksgenossen zu Sklavendiensten
zwingen dürfe. \bibleverse{10}Da hatten alle Fürsten\textless sup
title=``oder: Oberen''\textgreater✲ und das gesamte Volk, die diesen
Beschluß gemeinsam gefaßt hatten, ihre Sklaven und Sklavinnen
freizulassen und sie nicht länger zu Sklavendiensten zu zwingen, den
Beschluß auch pflichtgemäß ausgeführt und die Freilassung vollzogen;
\bibleverse{11}nachher aber waren sie anderen Sinnes geworden und hatten
die Sklaven und Sklavinnen, die sie bereits in Freiheit gesetzt hatten,
zurückgeholt und zwangsweise wieder zu Sklaven und Sklavinnen gemacht.

\hypertarget{b-das-gerichtswort-gottes}{%
\paragraph{b) Das Gerichtswort Gottes}\label{b-das-gerichtswort-gottes}}

\bibleverse{12}Da erging das Wort des HERRN an Jeremia folgendermaßen:
\bibleverse{13}»So spricht der HERR, der Gott Israels: Ich selbst habe
mit euren Vätern damals, als ich sie aus dem Lande Ägypten, dem
Sklavenhause, wegführte, einen Bund geschlossen und geboten\textless sup
title=``2.Mose 21,2-6; 5.Mose 15,1-12''\textgreater✲:
\bibleverse{14}›Nach Ablauf von sieben Jahren sollt ihr ein jeder seinen
hebräischen Volksgenossen, der sich dir verkauft hat, in Freiheit
setzen: sechs Jahre lang soll er dir Dienste leisten, dann aber sollst
du ihn frei von dir ziehen lassen!‹ Aber eure Väter sind mir nicht
gehorsam gewesen und haben mir kein Gehör geschenkt. \bibleverse{15}Nun
wart ihr jetzt zwar umgekehrt und hattet getan, was in meinen Augen
recht war, indem ihr ein jeder die Freilassung für seinen Volksgenossen
ausrufen ließt und in dem Tempel, der meinen Namen trägt, vor meinem
Angesicht einen gemeinsamen Beschluß faßtet. \bibleverse{16}Dann aber
seid ihr wieder anderen Sinnes geworden und habt meinen Namen entehrt,
indem ihr ein jeder seinen Sklaven und seine Sklavin, die ihr auf ihr
Verlangen bereits freigelassen hattet, zurückgeholt und zwangsweise
wieder zu euren Sklaven und Sklavinnen gemacht habt.

\bibleverse{17}Darum spricht der HERR so: Ihr habt nicht auf mich
gehört, daß ihr ein jeder für seinen Volks- und Stammesgenossen die
Freilassung hättet ausrufen lassen: nun, so will ich jetzt über euch
eine Freilassung ausrufen« -- so lautet der Ausspruch des HERRN --,
»nämlich für das Schwert, für die Pest und für den Hunger, und will euch
zum abschreckenden Beispiel für alle Reiche der Erde machen
\bibleverse{18}und will die Männer, die das vor mir geschlossene
Abkommen übertreten haben, indem sie den Bestimmungen des Beschlusses,
den sie vor meinem Angesicht gefaßt hatten, nicht nachgekommen sind, dem
Opferkalb gleich machen, das sie in zwei Hälften zerschnitten haben und
zwischen dessen Stücken sie hindurchgeschritten sind: \bibleverse{19}die
Fürsten\textless sup title=``oder: Oberen''\textgreater✲ von Juda und
die Fürsten Jerusalems, die Kammerherren\textless sup title=``oder:
Hofbeamten''\textgreater✲ und die Priester und das gesamte Volk des
Landes, die zwischen den Stücken des Kalbes hindurchgeschritten sind~--
\bibleverse{20}ja, die will ich der Gewalt ihrer Feinde preisgeben und
sie in die Hand derer fallen lassen, die ihnen nach dem Leben trachten;
und ihre Leichname sollen den Vögeln des Himmels und den Tieren des
Feldes zum Fraß dienen. \bibleverse{21}Auch Zedekia, den König von Juda,
und seine Fürsten\textless sup title=``oder: obersten
Beamten''\textgreater✲ will ich der Gewalt ihrer Feinde preisgeben und
sie in die Hand derer fallen lassen, die ihnen nach dem Leben
trachten\textless sup title=``d.h. in die Hand ihrer
Todfeinde''\textgreater✲, nämlich in die Gewalt des Heeres des Königs
von Babylon, das jetzt von euch abgezogen ist. \bibleverse{22}Wisset
wohl: ich gebiete« -- so lautet der Ausspruch des HERRN -- »und bringe
sie wieder zu dieser Stadt zurück, damit sie sie belagern und erobern
und in Flammen aufgehen lassen! Und auch die (übrigen) Städte Judas will
ich zu einer unbewohnten Einöde machen!«

\hypertarget{der-gehorsam-der-rechabiten-im-gegensatz-zu-dem-ungehorsam-judas}{%
\subsubsection{3. Der Gehorsam der Rechabiten im Gegensatz zu dem
Ungehorsam
Judas}\label{der-gehorsam-der-rechabiten-im-gegensatz-zu-dem-ungehorsam-judas}}

\hypertarget{a-jeremia-erprobt-auf-guxf6ttlichen-befehl-die-treue-der-rechabiten}{%
\paragraph{a) Jeremia erprobt auf göttlichen Befehl die Treue der
Rechabiten}\label{a-jeremia-erprobt-auf-guxf6ttlichen-befehl-die-treue-der-rechabiten}}

\hypertarget{section-34}{%
\section{35}\label{section-34}}

\bibleverse{1}Das Wort, das vom HERRN unter der Regierung des judäischen
Königs Jojakim, des Sohnes Josias, an Jeremia erging, lautete
folgendermaßen: \bibleverse{2}»Begib dich zur
Genossenschaft\textless sup title=``oder: Familie''\textgreater✲ der
Rechabiten und lade sie ein; führe sie dann in den Tempel des HERRN in
eine der Zellen und setze ihnen Wein zum Trinken vor!« \bibleverse{3}Da
holte ich Jaasanja, den Sohn Jeremias, des Sohnes Habazinjas, nebst
seinen Brüdern und allen seinen Söhnen, überhaupt die ganze
Genossenschaft der Rechabiten, \bibleverse{4}und führte sie zum Tempel
des HERRN in die Zelle der Söhne des Gottesmannes Hanan, des Sohnes
Jigdaljas, die neben der Zelle der Fürsten, oberhalb der Zelle des
Schwellenhüters Maaseja, des Sohnes Sallums, lag. \bibleverse{5}Dort
setzte ich den zur Familie der Rechabiten gehörenden Männern mit Wein
gefüllte Krüge und Becher vor und forderte sie auf, Wein zu trinken.
\bibleverse{6}Doch sie antworteten: »Wir trinken keinen Wein, denn unser
Stammvater Jonadab, der Sohn Rechabs\textless sup title=``vgl. 2.Kön
10,15-16''\textgreater✲, hat uns das Gebot erteilt: ›Ihr dürft keinen
Wein trinken, weder ihr noch euere Nachkommen, in alle Zukunft;
\bibleverse{7}auch dürft ihr euch keine Häuser bauen, keine Saatfelder
bestellen und keine Weinberge anlegen oder in Besitz haben, sondern
sollt während eures ganzen Lebens in Zelten wohnen, damit ihr lange in
dem Lande lebt, in welchem ihr euch als Fremdlinge aufhaltet.‹
\bibleverse{8}So sind wir denn dem Gebot unsers Stammvaters Jonadab, des
Sohnes Rechabs, in allen Stücken genau nachgekommen, so daß wir
zeitlebens keinen Wein trinken, weder wir noch unsere Frauen, noch
unsere Söhne und Töchter, \bibleverse{9}und daß wir uns keine Häuser
bauen, um darin zu wohnen, und weder Weinberge noch Äcker und Saatfelder
besitzen; \bibleverse{10}vielmehr wohnen wir in Zelten und erfüllen
gehorsam alles, was unser Stammvater Jonadab uns geboten hat.
\bibleverse{11}Nur als Nebukadnezar, der König von Babylon, gegen dies
Land herangezogen kam, da sagten wir: ›Kommt, wir wollen uns vor dem
Heere der Chaldäer und vor dem Heere der Syrer nach Jerusalem
zurückziehen!‹, und so haben wir jetzt in Jerusalem Wohnung genommen.«

\hypertarget{b-jeremias-ansprache-an-die-umstehenden}{%
\paragraph{b) Jeremias Ansprache an die
Umstehenden}\label{b-jeremias-ansprache-an-die-umstehenden}}

\bibleverse{12}Da erging das Wort des HERRN an Jeremia folgendermaßen:
\bibleverse{13}»So spricht der HERR der Heerscharen, der Gott Israels:
Gehe hin und sage zu den Männern von Juda und zu den Bewohnern
Jerusalems: ›Wollt ihr denn keine Zucht annehmen, daß ihr meinen
Weisungen gehorcht?‹ -- so lautet der Ausspruch des HERRN.
\bibleverse{14}›Die Weisung, die Jonadab, der Sohn Rechabs, seinen
Nachkommen gegeben hat, keinen Wein zu trinken, die ist befolgt worden:
sie haben keinen Wein bis auf den heutigen Tag getrunken, weil sie dem
Gebot ihres Stammvaters gehorsam gewesen sind. Ich aber habe früh und
spät immer wieder zu euch geredet, doch ihr habt nicht auf mich gehört.
\bibleverse{15}Und dabei habe ich alle meine Knechte, die Propheten,
früh und spät immer wieder zu euch gesandt mit der Mahnung: Kehrt doch
alle von euren bösen Wegen um, bessert euren Wandel und lauft nicht
anderen Göttern nach, um ihnen zu dienen! Dann sollt ihr in dem Lande
wohnen bleiben, das ich euch und euren Vätern gegeben habe; aber ihr
habt nicht hören wollen und seid mir nicht gehorsam gewesen.
\bibleverse{16}Ja, die Nachkommen Jonadabs, des Sohnes Rechabs, sind dem
Gebot nachgekommen, das ihr Stammvater ihnen gegeben hat; dieses Volk
aber hat nicht auf mich gehört!‹ \bibleverse{17}Darum spricht der HERR,
der Gott der Heerscharen, der Gott Israels, folgendermaßen: ›Fürwahr,
ich will über Juda und über alle Bewohner Jerusalems all das Unheil
kommen lassen, das ich ihnen angedroht habe, weil sie nicht haben hören
wollen, als ich zu ihnen redete, und nicht geantwortet haben, als ich
ihnen zurief.‹«

\bibleverse{18}Zur Genossenschaft der Rechabiten aber sagte Jeremia: »So
spricht der HERR der Heerscharen, der Gott Israels: ›Weil ihr dem Gebot
eures Stammvaters Jonadab gehorsam gewesen seid, indem ihr alle seine
Gebote beobachtet und alles getan habt, was er euch befohlen hat:
\bibleverse{19}darum spricht der HERR der Heerscharen, der Gott Israels,
also: Es soll Jonadab, dem Sohne Rechabs, in Zukunft niemals an einem
(Nachkommen) fehlen, der in meinem Dienst vor mir steht!‹«

\hypertarget{das-weissagungsbuch-jeremias-und-seine-schicksale}{%
\subsubsection{4. Das Weissagungsbuch Jeremias und seine
Schicksale}\label{das-weissagungsbuch-jeremias-und-seine-schicksale}}

\hypertarget{a-herstellung-des-buches-und-seine-verlesung-vor-dem-volk-und-vor-den-oberen}{%
\paragraph{a) Herstellung des Buches und seine Verlesung vor dem Volk
und vor den
Oberen}\label{a-herstellung-des-buches-und-seine-verlesung-vor-dem-volk-und-vor-den-oberen}}

\hypertarget{section-35}{%
\section{36}\label{section-35}}

\bibleverse{1}Im vierten Regierungsjahre des judäischen Königs Jojakim,
des Sohnes Josias, erging folgendes Wort des HERRN an Jeremia:
\bibleverse{2}»Nimm dir eine Buchrolle und schreibe auf sie alle die
Worte, die ich in betreff Israels und Judas und in betreff aller Völker
zu dir gesprochen habe seit dem Tage, an dem ich dir Offenbarungen habe
zuteil werden lassen, nämlich seit der Regierung Josias bis auf den
heutigen Tag! \bibleverse{3}Vielleicht hören dann die vom Hause Juda auf
all das Unheil, das ich über sie zu verhängen gedenke, und bekehren sich
alle von ihrem bösen Wandel, so daß ich ihnen ihre Verschuldung und
Sünde vergeben kann.«

\bibleverse{4}Da berief Jeremia den Baruch, den Sohn Nerijas, und dieser
schrieb alle die Worte, welche der HERR zu ihm gesprochen hatte, so, wie
Jeremia sie ihm vorsagte, auf eine Buchrolle. \bibleverse{5}Hierauf gab
Jeremia dem Baruch folgenden Auftrag: »Mir ist's verwehrt\textless sup
title=``oder: ich selbst bin verhindert''\textgreater✲: ich darf nicht
in den Tempel des HERRN gehen. \bibleverse{6}So gehe denn du hin und
lies aus der Rolle die Worte des HERRN, die du so, wie ich sie dir
vorsagte, aufgeschrieben hast, dem Volke im Tempel des HERRN an einem
Fasttage\textless sup title=``d.h. Buß- und Bettage''\textgreater✲ laut
vor; auch allen Judäern, die aus ihren Ortschaften herkommen, sollst du
sie laut vorlesen: \bibleverse{7}vielleicht dringt dann ihr Flehen zum
HERRN empor, und sie bekehren sich alle von ihrem bösen Wandel; denn
groß ist der Zorn und Grimm, mit dem der HERR diesem Volke gedroht hat.«
\bibleverse{8}Da tat Baruch, der Sohn Nerijas, genau so, wie der Prophet
Jeremia ihm aufgetragen hatte, indem er aus dem Buche die Worte des
HERRN im Tempel des HERRN vorlas.

\bibleverse{9}Es begab sich nämlich im fünften Regierungsjahr des
judäischen Königs Jojakim, des Sohnes Josias, im neunten
Monat\textless sup title=``=~am neunten Neumond''\textgreater✲, daß man
die ganze Bevölkerung Jerusalems und alles Volk, das aus den Ortschaften
Judas nach Jerusalem gekommen war, zu einem Fasten vor dem HERRN
aufrief. \bibleverse{10}Da las Baruch aus dem Buche die Worte Jeremias
im Tempel des HERRN in der Zelle Gemarjas, des Sohnes des
Staatsschreibers Saphan, im oberen Vorhof, am Eingang des neuen Tores am
Tempel des HERRN dem ganzen Volke laut vor. \bibleverse{11}Als nun
Michaja, der Sohn Gemarjas, des Sohnes Saphans, alle Worte des HERRN aus
dem Buche hatte vorlesen hören, \bibleverse{12}begab er sich zum
königlichen Palast hinab in das Zimmer des Staatsschreibers✲, wo gerade
alle Fürsten\textless sup title=``oder: obersten Beamten''\textgreater✲
eine Sitzung abhielten, nämlich der Staatsschreiber Elisama, Delaja, der
Sohn Semajas, Elnathan, der Sohn Achbors, Gemarja, der Sohn Saphans,
Zedekia, der Sohn Hananjas kurz alle Fürsten\textless sup title=``oder:
Oberen''\textgreater✲. \bibleverse{13}Als ihnen nun Michaja alle die
Worte mitgeteilt hatte, die er Baruch dem Volk aus dem Buch hatte laut
vorlesen hören, \bibleverse{14}sandten alle Fürsten\textless sup
title=``oder: Oberen''\textgreater✲ Jehudi, den Sohn Nethanjas, des
Sohnes Selemjas, des Sohnes Kusis, zu Baruch und ließen ihm sagen: »Nimm
die Buchrolle, aus der du dem Volke laut vorgelesen hast, mit dir und
komm hierher!« Da nahm Baruch, der Sohn Nerijas, die Rolle mit sich und
kam zu ihnen. \bibleverse{15}Nun sagten sie zu ihm: »Setze dich hin und
lies sie auch uns laut vor!« Da las Baruch sie ihnen laut vor.
\bibleverse{16}Als sie nun alle die Worte gehört hatten, sahen sie
einander erschrocken an und sagten zu Baruch: »Wir müssen alle diese
Worte\textless sup title=``oder: diesen ganzen Vorfall''\textgreater✲
unbedingt dem König berichten.« \bibleverse{17}Hierauf wandten sie sich
an Baruch mit der Frage: »Teile uns doch mit, wie du alle diese Worte,
die er dir vorgesagt hat, niedergeschrieben hast.« \bibleverse{18}Baruch
antwortete ihnen: »Jeremia selber hat mir alle diese Worte
vorgesprochen, und ich habe sie mit Tinte in das Buch eingetragen.«
\bibleverse{19}Da sagten die Fürsten zu Baruch: »Geh, verbirg dich, du
selbst und Jeremia! Niemand darf wissen, wo ihr seid!«

\hypertarget{b-kuxf6nig-jojakim-zerschneidet-und-verbrennt-das-weissagungsbuch-jeremias}{%
\paragraph{b) König Jojakim zerschneidet und verbrennt das
Weissagungsbuch
Jeremias}\label{b-kuxf6nig-jojakim-zerschneidet-und-verbrennt-das-weissagungsbuch-jeremias}}

\bibleverse{20}Hierauf begaben sie sich zum König in den Palasthof --
die Buchrolle hatten sie im Zimmer des Staatsschreibers Elisama
zurückgelassen -- und erstatteten dem Könige Bericht über den ganzen
Vorfall. \bibleverse{21}Da schickte der König den Jehudi hin, die Rolle
zu holen, und dieser brachte sie aus dem Zimmer des Staatsschreibers
Elisama herbei; sodann las Jehudi sie dem Könige und allen Fürsten vor,
die um den König standen. \bibleverse{22}Der König saß aber gerade in
der Winterwohnung -- es war nämlich der neunte Monat --, während das
Feuer im Kohlenbecken vor ihm brannte. \bibleverse{23}Sooft nun Jehudi
drei oder vier Spalten der Schriftrolle vorgelesen hatte, schnitt der
König sie mit dem Federmesser ab und warf sie in das Feuer, das im
Kohlenbecken brannte, bis die ganze Rolle im Feuer des Kohlenbeckens
vernichtet war. \bibleverse{24}Der König aber und alle seine
Diener\textless sup title=``oder: Hofbeamten''\textgreater✲, welche
diese ganze Vorlesung anhörten, wurden nicht in Bestürzung versetzt und
zerrissen ihre Kleider nicht; \bibleverse{25}ja sogar als Elnathan,
Delaja und Gemarja den König dringend baten, die Rolle nicht zu
verbrennen, hörte er doch nicht auf sie; \bibleverse{26}vielmehr gab der
König dem Prinzen\textless sup title=``=~seinem Sohn''\textgreater✲
Jerahmeel sowie Seraja, dem Sohne Asriels, und Selemja, dem Sohne
Abdeels, den Befehl, den Schreiber Baruch und den Propheten Jeremia zu
verhaften; aber der HERR hielt sie verborgen.

\hypertarget{c-erneuerung-des-buches-und-gottes-drohung-gegen-jojakim}{%
\paragraph{c) Erneuerung des Buches und Gottes Drohung gegen
Jojakim}\label{c-erneuerung-des-buches-und-gottes-drohung-gegen-jojakim}}

\bibleverse{27}Da erging das Wort des HERRN an Jeremia, nachdem der
König jene Rolle verbrannt hatte mitsamt den Worten, die Baruch so, wie
Jeremia sie ihm vorsagte, aufgeschrieben hatte, folgendermaßen:
\bibleverse{28}»Nimm dir noch einmal eine andere Rolle und schreibe auf
sie alle die vorigen Worte, welche auf der vorigen Rolle gestanden
haben, die Jojakim, der König von Juda, verbrannt hat.
\bibleverse{29}Gegen Jojakim aber, den König von Juda, sollst du
folgende Drohung aussprechen: ›So hat der HERR gesprochen: Du hast jene
Rolle verbrannt, indem du fragtest: Warum hast du auf sie (die Worte)
geschrieben, der König von Babylon werde unfehlbar kommen und dieses
Land verwüsten und Menschen und Vieh darin vertilgen?
\bibleverse{30}Darum bestimmt der HERR für Jojakim, den König von Juda,
folgendes: Er soll keinen (Nachkommen) haben, der auf dem Throne Davids
sitzt, und sein Leichnam soll hingeworfen daliegen, der Hitze bei Tage
und der Kälte bei Nacht preisgegeben! \bibleverse{31}Und ich werde an
ihm und seinen Nachkommen und seinen Dienern ihre Verschuldung
heimsuchen und will über sie und über die Bewohner Jerusalems und über
die Männer von Juda all das Unheil kommen lassen, das ich ihnen
angedroht habe, ohne daß sie darauf hörten!‹«

\bibleverse{32}So nahm denn Jeremia eine andere Buchrolle und gab sie
dem Schreiber Baruch, dem Sohne Nerijas; dieser schrieb dann auf sie
alle Worte\textless sup title=``oder: Aussprüche''\textgreater✲, die in
dem vom judäischen Könige Jojakim verbrannten Buche gestanden hatten,
so, wie Jeremia sie ihm vorsagte; außerdem wurden auch noch viele andere
gleichartige Worte\textless sup title=``oder: Aussprüche''\textgreater✲
hinzugefügt.

\hypertarget{jeremias-schicksale-und-spruxfcche-wuxe4hrend-der-belagerung-jerusalems-kap.-37-39}{%
\subsubsection{5. Jeremias Schicksale und Sprüche während der Belagerung
Jerusalems (Kap.
37-39)}\label{jeremias-schicksale-und-spruxfcche-wuxe4hrend-der-belagerung-jerusalems-kap.-37-39}}

\hypertarget{a-jeremias-antwort-an-die-gesandtschaft-zedekias}{%
\paragraph{a) Jeremias Antwort an die Gesandtschaft
Zedekias}\label{a-jeremias-antwort-an-die-gesandtschaft-zedekias}}

\hypertarget{section-36}{%
\section{37}\label{section-36}}

\bibleverse{1}Zur Zeit, als Zedekia, der Sohn Josias, regierte, den
Nebukadnezar, der König von Babylon, zum König über das Land Juda an
Stelle Konjas✲, des Sohnes Jojakims, eingesetzt hatte,
\bibleverse{2}hörte weder er noch seine Diener\textless sup
title=``oder: Hofbeamten''\textgreater✲, noch die Bevölkerung des Landes
auf die Worte, die der HERR durch den Propheten Jeremia an sie richtete.

\bibleverse{3}Da sandte (einst) der König Zedekia den Juchal, den Sohn
Selemjas, und den Priester Zephanja, den Sohn Maasejas, zum Propheten
Jeremia und ließ ihm sagen: »Bete doch für uns zum HERRN, unserm Gott!«
\bibleverse{4}Jeremia bewegte sich aber damals noch in voller Freiheit
inmitten des Volkes, da man ihn noch nicht ins Gefängnis geworfen hatte.
\bibleverse{5}Das Heer des Pharaos war nämlich aus Ägypten aufgebrochen,
und die Chaldäer, die Jerusalem belagerten, waren, als sie die Kunde
davon erhielten, von Jerusalem abgezogen.

\bibleverse{6}Da erging das Wort des HERRN an den Propheten Jeremia
folgendermaßen: \bibleverse{7}»So spricht der HERR, der Gott Israels:
Verkündet dem Könige von Juda, der euch zu mir gesandt hat, um mich zu
befragen, folgende Botschaft: ›Wisse wohl: das Heer des Pharaos, das
euch zur Hilfe ausgezogen ist, wird alsbald in sein Land Ägypten
zurückkehren, \bibleverse{8}und zurückkehren werden die Chaldäer, um
diese Stadt zu belagern, und werden sie erobern und in Flammen aufgehen
lassen.‹ \bibleverse{9}So spricht der HERR: ›Täuscht euch nicht selbst
mit der Meinung, daß die Chaldäer jetzt wirklich von euch abziehen
werden; denn sie werden nicht abziehen. \bibleverse{10}Nein, wenn ihr
auch das ganze Heer der Chaldäer, die Krieg mit euch führen, besiegtet
und nur einige schwerverwundete Männer von ihnen übrigblieben, so würden
diese doch ein jeder in seinem Zelt aufstehen und diese Stadt in Flammen
aufgehen lassen!‹«

\hypertarget{b-jeremias-verhaftung-und-gefangensetzung-durch-einen-milituxe4rischen-beamten}{%
\paragraph{b) Jeremias Verhaftung und Gefangensetzung durch einen
militärischen
Beamten}\label{b-jeremias-verhaftung-und-gefangensetzung-durch-einen-milituxe4rischen-beamten}}

\bibleverse{11}Nun begab es sich, als das Heer der Chaldäer wegen des
Heeres des Pharaos von Jerusalem abgezogen war, \bibleverse{12}daß
Jeremia Jerusalem verlassen und sich in die Landschaft Benjamin begeben
wollte, um dort eine Erbschaftssache im Kreise seiner Verwandten zu
erledigen. \bibleverse{13}Als er nun im Benjaminstor angelangt war, wo
ein Mann namens Jirja, der Sohn Selemjas, des Sohnes Hananjas, als
Hauptmann der Wache stand, hielt dieser den Propheten Jeremia an unter
dem Vorwand: »Du willst zu den Chaldäern übergehen!«
\bibleverse{14}Jeremia entgegnete: »Das ist eine Lüge! Ich will nicht zu
den Chaldäern übergehen!« und hörte nicht weiter nach ihm hin; aber
Jirja ließ Jeremia festnehmen und vor die Fürsten\textless sup
title=``oder: Oberen''\textgreater✲ führen. \bibleverse{15}Diese waren
Jeremia feindlich gesinnt, ließen ihn stäupen und im Hause des
Staatsschreibers Jonathan, das man zum Gefängnis hergerichtet hatte,
gefangensetzen.

\hypertarget{c-jeremia-aufs-neue-vom-kuxf6nig-befragt-und-aus-dem-gefuxe4ngnis-in-den-wachthof-gebracht}{%
\paragraph{c) Jeremia aufs neue vom König befragt und aus dem Gefängnis
in den Wachthof
gebracht}\label{c-jeremia-aufs-neue-vom-kuxf6nig-befragt-und-aus-dem-gefuxe4ngnis-in-den-wachthof-gebracht}}

\bibleverse{16}Als Jeremia so in das Brunnengebäude, und zwar in die
unterirdischen Gewölbe gekommen war und dort lange Zeit zugebracht
hatte, \bibleverse{17}sandte der König Zedekia (eines Tages) hin, ließ
ihn holen und richtete in seinem Palast insgeheim die Frage an ihn: »Ist
ein Wort vom HERRN ergangen?« Jeremia antwortete: »Jawohl! Nämlich der
Gewalt des Königs von Babylon wirst du preisgegeben\textless sup
title=``oder: ausgeliefert''\textgreater✲ werden.« \bibleverse{18}Weiter
sagte Jeremia zum König Zedekia: »Was habe ich gegen dich und gegen
deine Diener und gegen unser Volk verschuldet, daß ihr mich ins
Gefängnis gesetzt habt? \bibleverse{19}Wo sind denn jetzt eure
Propheten, die euch mit Bestimmtheit geweissagt haben, der König von
Babylon werde nicht gegen euch und gegen dieses Land heranziehen?
\bibleverse{20}Und nun -- so höre doch, mein Herr und König, laß meine
Bitte Gehör bei dir finden, laß mich nicht wieder in das Haus des
Staatsschreibers Jonathan bringen, damit ich dort nicht sterben muß!«
\bibleverse{21}Da gab der König Zedekia Befehl, und man brachte Jeremia
im Wachthof in Gewahrsam und gab ihm täglich einen Laib Brot aus der
Bäckergasse, bis alles Brot in der Stadt aufgezehrt war. So blieb denn
Jeremia im Wachthof.

\hypertarget{d-der-anschlag-der-oberen-auf-das-leben-des-propheten-vereitelt}{%
\paragraph{d) Der Anschlag der Oberen auf das Leben des Propheten
vereitelt}\label{d-der-anschlag-der-oberen-auf-das-leben-des-propheten-vereitelt}}

\hypertarget{aa-jeremia-von-den-oberen-als-ein-hochverruxe4ter-in-eine-zisterne-geworfen}{%
\subparagraph{aa) Jeremia von den Oberen als ein Hochverräter in eine
Zisterne
geworfen}\label{aa-jeremia-von-den-oberen-als-ein-hochverruxe4ter-in-eine-zisterne-geworfen}}

\hypertarget{section-37}{%
\section{38}\label{section-37}}

\bibleverse{1}Sephatja aber, der Sohn Matthans, und Gedalja, der Sohn
Pashurs, und Juchal, der Sohn Selemjas, und Pashur, der Sohn Malkijas,
hörten die Worte, die Jeremia an das ganze Volk richtete, daß er nämlich
sagte: \bibleverse{2}»So hat der HERR gesprochen: ›Wer hier in der Stadt
verbleibt, wird durch das Schwert, durch den Hunger und die Pest ums
Leben kommen; wer dagegen zu den Chaldäern hinausgeht✲, wird erhalten
bleiben und sein Leben in Sicherheit bringen.‹ \bibleverse{3}Denn so hat
der HERR gesprochen: ›Diese Stadt wird unfehlbar in die Gewalt des
Heeres des Königs von Babylon gegeben werden, der sie erobern wird.‹«
\bibleverse{4}Da sagten die Fürsten\textless sup title=``oder:
Oberen''\textgreater✲ zum König: »Dieser Mensch sollte\textless sup
title=``oder: muß''\textgreater✲ hingerichtet werden, er macht ja die
Kriegsleute, die hier in der Stadt noch übriggeblieben sind, und die
ganze Bevölkerung mutlos, indem er solche Worte vor ihnen ausspricht;
denn dieser Mensch hat nicht das Wohl unsers Volkes im Auge, sondern
dessen Unglück!« \bibleverse{5}Da gab der König Zedekia ihnen zur
Antwort: »Nun gut! Ihr habt freie Verfügung über ihn; der König ist ja
euch gegenüber machtlos.« \bibleverse{6}Da ließen sie Jeremia festnehmen
und ihn in die Zisterne des Königssohnes✲ Malkija werfen, die sich im
Wachthof befand; in diese ließen sie Jeremia an Stricken hinab. In der
Zisterne war aber kein Wasser, sondern nur Schlamm, in welchen Jeremia
einsank.

\hypertarget{bb-jeremias-rettung-durch-den-uxe4thiopier-mohren-ebedmelech}{%
\subparagraph{bb) Jeremias Rettung durch den Äthiopier (=~Mohren)
Ebedmelech}\label{bb-jeremias-rettung-durch-den-uxe4thiopier-mohren-ebedmelech}}

\bibleverse{7}Als aber der Äthiopier Ebedmelech, ein Beamter am
königlichen Hof, erfuhr, daß man Jeremia in die Zisterne verbracht habe,
\bibleverse{8}verließ er den königlichen Palast und machte dem König,
der sich gerade im Benjaminstor aufhielt, folgende Meldung:
\bibleverse{9}»Mein Herr und König! Jene Männer haben in allem unrecht
gehandelt, was sie dem Propheten Jeremia zugefügt haben, der von ihnen
in die Zisterne geworfen worden ist; er muß ja da, wo er sich befindet,
Hungers sterben!« Denn es war kein Brot mehr in der Stadt vorhanden.
\bibleverse{10}Da erteilte der König dem Äthiopier Ebedmelech den
Befehl: »Nimm von hier drei Männer mit dir und laß den Propheten Jeremia
aus der Zisterne heraufziehen, ehe er stirbt!« \bibleverse{11}Da nahm
Ebedmelech die Männer mit sich, begab sich in den königlichen Palast in
den Raum unter der Schatzkammer\textless sup title=``oder: dem
Vorratshause''\textgreater✲, nahm von dort Lappen von zerrissenen und
abgetragenen Kleidungsstücken und ließ sie an Stricken zu Jeremia in die
Zisterne hinab. \bibleverse{12}Alsdann rief er dem Jeremia zu, er möge
diese Lappen von den zerrissenen und abgetragenen Kleidungsstücken sich
unter die Achselhöhlen um die Stricke legen; und als Jeremia dies getan
hatte, \bibleverse{13}zogen sie ihn an den Stricken aus der Zisterne
herauf. Jeremia blieb dann im Wachthofe.

\hypertarget{e-jeremias-letzte-geheime-unterredung-mit-dem-kuxf6nige}{%
\paragraph{e) Jeremias letzte geheime Unterredung mit dem
Könige}\label{e-jeremias-letzte-geheime-unterredung-mit-dem-kuxf6nige}}

\bibleverse{14}Hierauf sandte der König Zedekia hin und ließ den
Propheten Jeremia zu sich holen in den dritten Eingang am Tempel des
HERRN; und der König sagte zu Jeremia: »Ich habe eine Frage an dich zu
richten: verschweige mir nichts!« \bibleverse{15}Jeremia antwortete dem
Zedekia: »Wenn ich es dir kundtue, wirst du mich sicherlich töten
lassen, und wenn ich dir einen Rat gebe, wirst du doch nicht auf mich
hören.« \bibleverse{16}Da gab der König Zedekia dem Jeremia insgeheim
die eidliche Zusicherung: »So wahr der HERR lebt, der uns diese
Seele\textless sup title=``=~unser Leben''\textgreater✲ geschaffen hat:
ich werde dich nicht töten lassen und werde dich nicht jenen Männern in
die Hände liefern, die dir nach dem Leben trachten!« \bibleverse{17}Da
sagte Jeremia zu Zedekia: »So hat der HERR, der Gott der Heerscharen,
der Gott Israels, gesprochen: ›Wenn du dich den Heeresobersten des
Königs von Babylon ergibst, so wirst du am Leben bleiben, und diese
Stadt wird nicht mit Feuer zerstört werden, und zwar wirst du samt
deiner Familie das Leben behalten. \bibleverse{18}Wenn du dich aber den
Heeresobersten des Königs von Babylon nicht ergibst, so wird diese Stadt
in die Gewalt der Chaldäer gegeben, die sie mit Feuer verbrennen werden;
und du selbst wirst ihrer Hand nicht entgehen.‹« \bibleverse{19}Da
erwiderte der König Zedekia dem Jeremia: »Ich fürchte, daß man mich den
Judäern, die schon zu den Chaldäern übergegangen sind, ausliefern wird
und daß diese sich an mir vergreifen.« \bibleverse{20}Jeremia aber
entgegnete: »Man wird dich ihnen nicht ausliefern! Höre doch bei dem,
was ich dir sage, auf die Weisung des HERRN, so wird es dir gut ergehen,
und du wirst am Leben bleiben. \bibleverse{21}Weigerst du dich aber
hinauszugehen\textless sup title=``=~dich zu ergeben''\textgreater✲, so
ist dies das Wort, das der HERR mir geoffenbart hat:
\bibleverse{22}Wisse wohl: alle Frauen, die im Palast des Königs von
Juda noch übriggeblieben sind, werden zu den Heeresobersten des Königs
von Babylon hinausgeführt werden und dabei ausrufen: ›Betrogen haben sie
dich und überlistet, deine vertrauten Freunde! Nun deine Füße im Schlamm
versinken, haben sie sich davongemacht!‹ \bibleverse{23}Alle deine
Frauen aber samt deinen Kindern wird man zu den Chaldäern hinausführen,
und du selbst wirst ihren Händen nicht entgehen, sondern von der Hand
des Königs von Babylon ergriffen werden und die Verbrennung dieser Stadt
herbeiführen!«

\hypertarget{auf-befehl-des-kuxf6nigs-verschweigt-jeremia-den-oberen-den-inhalt-der-unterredung}{%
\paragraph{Auf Befehl des Königs verschweigt Jeremia den Oberen den
Inhalt der
Unterredung}\label{auf-befehl-des-kuxf6nigs-verschweigt-jeremia-den-oberen-den-inhalt-der-unterredung}}

\bibleverse{24}Hierauf sagte Zedekia zu Jeremia: »Kein Mensch darf von
dieser Unterredung etwas erfahren, sonst wärst du des Todes!
\bibleverse{25}Wenn aber die Fürsten\textless sup title=``oder:
Oberen''\textgreater✲ erfahren sollten, daß ich mich mit dir besprochen
habe, und sie zu dir kommen und zu dir sagen: ›Teile uns doch mit, was
du zum König gesagt hast, verschweige uns ja nichts, sonst lassen wir
dich hinrichten! Und was hat der König zu dir gesagt?‹,
\bibleverse{26}so antworte ihnen: ›Ich habe dem König meine inständige
Bitte vorgetragen, er möge mich nicht wieder in das Haus Jonathans
bringen lassen, damit ich dort nicht sterbe.‹« \bibleverse{27}Als nun
wirklich alle Fürsten\textless sup title=``oder: Oberen''\textgreater✲
zu Jeremia kamen und ihn fragten, gab er ihnen genau nach jener Weisung
des Königs Bescheid; da ließen sie ihn in Ruhe; denn von der Unterredung
war nichts weiter in die Öffentlichkeit gedrungen. \bibleverse{28}So
verblieb denn Jeremia im Wachthof bis zu dem Tage, an dem Jerusalem
erobert wurde.

\hypertarget{f-jeremias-geschick-bei-der-eroberung-jerusalems-schicksal-zedekias-sowie-der-stadt-und-des-landes}{%
\paragraph{f) Jeremias Geschick bei der Eroberung Jerusalems; Schicksal
Zedekias sowie der Stadt und des
Landes}\label{f-jeremias-geschick-bei-der-eroberung-jerusalems-schicksal-zedekias-sowie-der-stadt-und-des-landes}}

\hypertarget{section-38}{%
\section{39}\label{section-38}}

\bibleverse{1}Als aber Jerusalem erobert war -- im neunten
Regierungsjahre des judäischen Königs Zedekia, im zehnten Monat, war
Nebukadnezar, der König von Babylon, mit seiner ganzen Heeresmacht vor
Jerusalem gerückt und hatte die Belagerung begonnen; \bibleverse{2}im
elften Regierungsjahre Zedekias aber, am neunten Tage des vierten
Monats, wurde Bresche in die Stadtmauer gelegt --, \bibleverse{3}da
kamen alle Fürsten\textless sup title=``oder: Heerführer''\textgreater✲
des Königs von Babylon und ließen sich im\textless sup title=``oder:
am''\textgreater✲ Mitteltor nieder, nämlich {[}Nergal-Sarezer,{]} der
Oberkämmerer Samgar-Nebusarsekim, der Obermagier Nergal-Sarezer und alle
übrigen Fürsten\textless sup title=``oder: Heerführer''\textgreater✲ des
Königs von Babylon. \bibleverse{4}Als nun Zedekia, der König von Juda,
und alle Kriegsleute das sahen, ergriffen sie die Flucht und verließen
bei Nacht die Stadt auf dem Wege nach dem Königsgarten durch das Tor
zwischen den beiden Mauern und zogen dann weiter der Jordan-Ebene zu.
\bibleverse{5}Aber das Heer der Chaldäer setzte ihnen nach, und sie
holten Zedekia in den Steppen von Jericho ein; sie nahmen ihn fest und
brachten ihn zu Nebukadnezar, dem König von Babylon, nach Ribla in der
Landschaft Hamath; der hielt dann Gericht über ihn. \bibleverse{6}Der
König von Babylon ließ die Söhne Zedekias in Ribla vor dessen Augen
schlachten\textless sup title=``=~grausam hinrichten''\textgreater✲, und
ebenso verfuhr er mit allen vornehmen Judäern; \bibleverse{7}Zedekia
aber ließ er blenden und in Ketten legen, um ihn nach Babylon zu
bringen. \bibleverse{8}Den königlichen Palast aber und die Häuser der
Einwohnerschaft ließen die Chaldäer in Flammen aufgehen und rissen die
Mauern Jerusalems nieder. \bibleverse{9}Den Rest des Volkes aber, sowohl
die, welche in der Stadt übriggeblieben waren, als auch die Überläufer,
die zu ihm übergegangen waren, und was vom Volk sonst noch am Leben war,
ließ Nebusaradan, der Befehlshaber der Leibwache, nach Babylon führen;
\bibleverse{10}von den geringen Leuten jedoch, die keinen Besitz hatten,
ließ Nebusaradan, der Befehlshaber der Leibwache, einen Teil im Lande
Juda zurück und wies ihnen an jenem Tage Weinberge und Äcker an.

\hypertarget{freundliche-behandlung-jeremias-von-seiten-der-chalduxe4er}{%
\paragraph{Freundliche Behandlung Jeremias von seiten der
Chaldäer}\label{freundliche-behandlung-jeremias-von-seiten-der-chalduxe4er}}

\bibleverse{11}In betreff Jeremias aber ließ Nebukadnezar, der König von
Babylon, Nebusaradan, dem Befehlshaber der Leibwache, folgenden Befehl
zugehen: \bibleverse{12}»Nimm ihn, trage Sorge für ihn und tu ihm nichts
zuleide, sondern verfahre mit ihm nach den Wünschen, die er gegen dich
äußern wird!« \bibleverse{13}Da sandten Nebusaradan, der Befehlshaber
der Leibwache, und Nebusasban, der Oberkämmerer, und Nergal-Sarezer, der
Obermagier, und alle übrigen Großen\textless sup title=``oder:
Obersten''\textgreater✲ des Königs von Babylon hin,
\bibleverse{14}ließen Jeremia aus dem Wachthofe holen und übergaben ihn
Gedalja, dem Sohne Ahikams, des Sohnes Saphans, daß er ihn frei nach
Hause gehen lasse; so blieb er denn inmitten des Volkes wohnen.

\hypertarget{g-heilsspruch-fuxfcr-den-uxe4thiopier-ebedmelech}{%
\paragraph{g) Heilsspruch für den Äthiopier
Ebedmelech}\label{g-heilsspruch-fuxfcr-den-uxe4thiopier-ebedmelech}}

\bibleverse{15}An Jeremia war aber, als er noch im Wachthofe in Haft
gehalten wurde, folgendes Wort des HERRN ergangen: \bibleverse{16}»Gehe
hin und sage zu dem Äthiopier Ebedmelech: ›So hat der HERR der
Heerscharen, der Gott Israels, gesprochen: Nunmehr lasse ich meine
Drohworte gegen diese Stadt in Erfüllung gehen zum Unheil, nicht zum
Segen, und ihre Erfüllung wird dir an jenem\textless sup title=``=~dem
betreffenden''\textgreater✲ Tage vor Augen treten. \bibleverse{17}Dich
aber will ich an jenem Tage erretten‹ -- so lautet der Ausspruch des
HERRN --, ›und du sollst nicht den Männern in die Hände fallen, vor
denen du in Angst bist; \bibleverse{18}vielmehr will ich dich entrinnen
lassen, und du sollst nicht durch das Schwert umkommen, sondern sollst
dein Leben in Sicherheit bringen, weil du auf mich vertraut hast!‹« --
so lautet der Ausspruch des HERRN.

\hypertarget{b.-reden-und-erlebnisse-jeremias-meist-nach-der-zerstuxf6rung-jerusalems-kap.-40-52}{%
\subsection{B. Reden und Erlebnisse Jeremias meist nach der Zerstörung
Jerusalems (Kap.
40-52)}\label{b.-reden-und-erlebnisse-jeremias-meist-nach-der-zerstuxf6rung-jerusalems-kap.-40-52}}

\hypertarget{i.-geschicke-jeremias-und-seiner-volksgenossen-in-der-verwuxfcsteten-heimat-und-in-uxe4gypten-kap.-40-45}{%
\subsection{I. Geschicke Jeremias und seiner Volksgenossen in der
verwüsteten Heimat und in Ägypten (Kap.
40-45)}\label{i.-geschicke-jeremias-und-seiner-volksgenossen-in-der-verwuxfcsteten-heimat-und-in-uxe4gypten-kap.-40-45}}

\hypertarget{jeremias-freilassung-aus-der-chalduxe4ischen-gefangenschaft-und-ruxfcckkehr-zum-statthalter-gedalja}{%
\subsubsection{1. Jeremias Freilassung aus der chaldäischen
Gefangenschaft und Rückkehr zum Statthalter
Gedalja}\label{jeremias-freilassung-aus-der-chalduxe4ischen-gefangenschaft-und-ruxfcckkehr-zum-statthalter-gedalja}}

\hypertarget{section-39}{%
\section{40}\label{section-39}}

\bibleverse{1}(Dies ist) das Wort, das vom HERRN an Jeremia erging,
nachdem Nebusaradan, der Befehlshaber der Leibwache, ihn von Rama aus
entlassen hatte, wo er ihn, und zwar mit Ketten gefesselt, ausfindig
gemacht hatte inmitten aller gefangenen Bewohner Jerusalems und Judas,
die nach Babylon weggeführt werden sollten. \bibleverse{2}Als nämlich
der Befehlshaber der Leibwache den Jeremia dort ausfindig
gemacht\textless sup title=``oder: zu sich beschieden''\textgreater✲
hatte, sagte er zu ihm: »Der HERR, dein Gott, hatte diesem Ort dieses
Unglück angedroht, \bibleverse{3}und der HERR hat es nun auch eintreten
lassen und seine Drohung zur Ausführung gebracht: weil ihr gegen den
HERRN gesündigt und auf seine Weisung nicht gehört habt, darum ist es
euch so ergangen. \bibleverse{4}Und nun, siehe: ich mache dich jetzt
frei von den Ketten an deinen Händen. Gefällt es dir, mit mir nach
Babylon zu gehen, so komm: ich werde Sorge für dich tragen; hast du aber
keine Lust, mit mir nach Babylon zu gehen, so laß es! Wisse wohl: das
ganze Land steht dir offen: du kannst gehen, wohin es dir beliebt und
gut dünkt!« \bibleverse{5}Als Jeremia sich dann nicht sofort
entschließen konnte, fuhr er fort: »So kehre doch zurück zu Gedalja, dem
Sohne Ahikams, des Sohnes Saphans, den der König von Babylon zum
Statthalter über die Städte von Juda eingesetzt hat, und bleibe bei ihm
inmitten des Volkes wohnen; oder gehe, wohin du sonst Lust hast!«
Hierauf gab ihm der Befehlshaber der Leibwache Lebensmittel✲ und ein
Geschenk und entließ ihn. \bibleverse{6}Jeremia begab sich dann nach
Mizpa zu Gedalja, dem Sohne Ahikams, und blieb dort bei ihm inmitten des
Volkes, das im Lande übriggeblieben war.

\hypertarget{gedalja-als-statthalter-in-der-neugegruxfcndeten-juxfcdischen-niederlassung-mizpa-seine-ermordung-und-ihre-folgen}{%
\subsubsection{2. Gedalja als Statthalter in der neugegründeten
jüdischen Niederlassung Mizpa; seine Ermordung und ihre
Folgen}\label{gedalja-als-statthalter-in-der-neugegruxfcndeten-juxfcdischen-niederlassung-mizpa-seine-ermordung-und-ihre-folgen}}

\hypertarget{a-gedalja-sammelt-die-juduxe4er-zu-einer-kolonie-in-mizpa}{%
\paragraph{a) Gedalja sammelt die Judäer zu einer Kolonie in
Mizpa}\label{a-gedalja-sammelt-die-juduxe4er-zu-einer-kolonie-in-mizpa}}

\bibleverse{7}Als nun alle Truppenführer, die sich mit ihren
Mannschaften noch im offenen Lande befanden, erfuhren, daß der König von
Babylon Gedalja, den Sohn Ahikams, zum Statthalter über das Land
eingesetzt und ihm die Obhut über Männer, Frauen und Kinder
und\textless sup title=``oder: nämlich''\textgreater✲ über die geringen
Leute im Lande, die nicht nach Babylon weggeführt worden waren,
anvertraut habe, \bibleverse{8}da kamen sie zu Gedalja nach Mizpa,
nämlich Ismael, der Sohn Nethanjas, sowie Johanan und Jonathan, die
Söhne Kareahs, ferner Seraja, der Sohn Thanhumeths, die Söhne Ophais aus
Netopha, und Jesanja, der Sohn des Maachathiters, samt ihren Leuten.
\bibleverse{9}Da richtete Gedalja, der Sohn Ahikams, des Sohnes Saphans,
an sie und ihre Leute unter feierlicher Anrufung Gottes folgende
Ansprache: »Fürchtet euch nicht davor, den Chaldäern\textless sup
title=``oder: den chaldäischen Beamten; vgl. 2.Kön 25,24''\textgreater✲
untertan zu sein! Bleibt im Lande wohnen und unterwerft euch dem König
von Babylon: ihr werdet euch gut dabei stehen! \bibleverse{10}Seht, ich
selbst bleibe hier in Mizpa, um euch vor den Chaldäern, die zu uns
kommen werden, zu vertreten; ihr aber mögt Wein, Obst und Öl
sammeln\textless sup title=``oder: ernten''\textgreater✲ und in euren
Behältern\textless sup title=``oder: Speichern''\textgreater✲ einheimsen
und könnt ruhig in euren Ortschaften wohnen, die ihr in Besitz genommen
habt!« \bibleverse{11}Ebenso erhielten alle Judäer, die sich in Moab und
unter den Ammonitern sowie in Edom und in allen übrigen Ländern
aufhielten, Kunde davon, daß der König von Babylon einen Rest (der
Bevölkerung) in Juda übriggelassen und daß er Gedalja, den Sohn Ahikams,
des Sohnes Saphans, über ihn als Statthalter eingesetzt habe.
\bibleverse{12}Da kehrten alle diese Judäer aus allen Gegenden, wohin
sie versprengt worden waren, zurück und kamen ins Land Juda zu Gedalja
nach Mizpa; sie hatten dann eine reiche Wein- und Obsternte.

\hypertarget{b-gedalja-von-ismael-ermordet}{%
\paragraph{b) Gedalja von Ismael
ermordet}\label{b-gedalja-von-ismael-ermordet}}

\bibleverse{13}Als aber Johanan, der Sohn Kareahs, und alle
Truppenführer, die noch im offenen Lande gestanden hatten, zu Gedalja
nach Mizpa gekommen waren, \bibleverse{14}sagten sie zu ihm: »Weißt du
wohl, daß Baalis, der König der Ammoniter, den Ismael, den Sohn
Nethanjas, abgesandt hat, um dich zu ermorden?« Doch Gedalja, der Sohn
Ahikams, schenkte ihnen keinen Glauben. \bibleverse{15}Darauf besprach
sich Johanan, der Sohn Kareahs, heimlich mit Gedalja in Mizpa und sagte:
»Laß mich doch hingehen und Ismael, den Sohn Nethanjas, erschlagen: kein
Mensch soll etwas davon erfahren! Warum soll er dich ermorden, so daß
alle Judäer, die sich hier bei dir gesammelt haben, wieder zerstreut
werden und der letzte Rest von Juda zugrunde geht?« \bibleverse{16}Aber
Gedalja, der Sohn Ahikams, antwortete Johanan, dem Sohne Kareahs: »Du
darfst das nicht tun; denn was du da von Ismael sagst, ist nicht wahr!«

\hypertarget{section-40}{%
\section{41}\label{section-40}}

\bibleverse{1}Im siebten Monat\textless sup title=``oder: am siebten
Neumond''\textgreater✲ aber kam Ismael, der Sohn Nethanjas, des Sohnes
Elisamas, (ein Mann) von königlicher Abkunft und einer von den
Großen\textless sup title=``oder: Würdenträgern''\textgreater✲ des
früheren Königs, in Begleitung von zehn Männern zu Gedalja, dem Sohne
Ahikams, nach Mizpa. Als sie dort mit ihm zusammen beim Mahl saßen,
\bibleverse{2}erhob sich Ismael, der Sohn Nethanjas, und die zehn
Männer, die bei ihm waren, und erschlugen Gedalja, den Sohn Ahikams, des
Sohnes Saphans, mit dem Schwert: so ermordete er den Mann, den der König
von Babylon zum Statthalter über das Land eingesetzt hatte.
\bibleverse{3}Auch alle Judäer, die bei Gedalja in Mizpa waren, sowie
die chaldäischen Kriegsleute, die sich dort befanden, ließ Ismael
niedermachen.

\hypertarget{c-ismael-ermordet-israelitische-tempelpilger-und-zieht-mit-zahlreichen-gefangenen-von-mizpa-ab}{%
\paragraph{c) Ismael ermordet israelitische Tempelpilger und zieht mit
zahlreichen Gefangenen von Mizpa
ab}\label{c-ismael-ermordet-israelitische-tempelpilger-und-zieht-mit-zahlreichen-gefangenen-von-mizpa-ab}}

\bibleverse{4}Am anderen Tage nach der Ermordung Gedaljas aber, als noch
niemand etwas von der Sache erfahren hatte, \bibleverse{5}kamen Männer
aus Sichem, aus Silo und Samaria, achtzig an der Zahl, mit geschorenen
Bärten, zerrissenen Kleidern und mit Schnittwunden am Leibe; sie hatten
Opfergaben und Weihrauch bei sich, um diese (Gaben) in den Tempel des
HERRN zu bringen. \bibleverse{6}Da ging Ismael, der Sohn Nethanjas, von
Mipza aus ihnen entgegen, indem er beim Gehen immerfort weinte; und als
er mit ihnen zusammentraf, sagte er zu ihnen: »Kommt herein zu Gedalja,
dem Sohne Ahikams!« \bibleverse{7}Als sie dann aber ins Innere der Stadt
gekommen waren, machte Ismael, der Sohn Nethanjas, sie nieder und warf
sie in eine Zisterne hinein, er und die Leute, die bei ihm waren.
\bibleverse{8}Es befanden sich aber unter ihnen zehn Männer, die zu
Ismael sagten: »Töte uns nicht! Denn wir haben noch im Felde vergrabene
Vorräte von Weizen, Gerste, Öl und Honig!« Da verschonte er sie und
tötete sie nicht wie die anderen. \bibleverse{9}Die Zisterne aber, in
welche Ismael alle Leichen der ermordeten Männer werfen ließ, war eine
große Zisterne, dieselbe, welche der König Asa zu Kriegszwecken gegen
Baesa, den König von Israel, hatte anlegen lassen\textless sup
title=``vgl. 1.Kön 15,22''\textgreater✲; diese füllte jetzt Ismael, der
Sohn Nethanjas, mit den (Leichen der) Erschlagenen an \bibleverse{10}und
führte hierauf den gesamten Überrest der Bevölkerung, der sich in Mizpa
befand, gefangen weg: die königlichen Frauen und alle in Mizpa
übriggebliebenen Personen, die Nebusaradan, der Befehlshaber der
Leibwache, der Obhut Gedaljas, des Sohnes Ahikams, überwiesen hatte --
die führte Ismael, der Sohn Nethanjas, gefangen weg und machte sich auf
den Weg, um zu den Ammonitern hinüberzuziehen.

\hypertarget{d-johanan-befreit-die-von-ismael-gefangenen-juduxe4er-bei-gibeon-und-bricht-zur-auswanderung-nach-uxe4gypten-auf}{%
\paragraph{d) Johanan befreit die von Ismael gefangenen Judäer bei
Gibeon und bricht zur Auswanderung nach Ägypten
auf}\label{d-johanan-befreit-die-von-ismael-gefangenen-juduxe4er-bei-gibeon-und-bricht-zur-auswanderung-nach-uxe4gypten-auf}}

\bibleverse{11}Als aber Johanan, der Sohn Kareahs, und alle
Truppenführer, die sich bei ihm befanden, die ganze Freveltat erfuhren,
die Ismael, der Sohn Nethanjas, verübt hatte, \bibleverse{12}boten sie
alle ihre Leute auf und zogen zum Kampf gegen Ismael, den Sohn
Nethanjas, aus, und sie trafen ihn am großen Teich bei Gibeon.
\bibleverse{13}Als nun die ganze Volksmenge, die sich bei Ismael befand,
Johanan, den Sohn Kareahs, und alle Truppenführer, die bei ihm waren,
erblickten, freuten sie sich; \bibleverse{14}und die ganze Volksmenge,
die Ismael gefangen aus Mizpa weggeführt hatte, machte kehrt und ging zu
Johanan, dem Sohn Kareahs, über; \bibleverse{15}Ismael aber, der Sohn
Nethanjas, ergriff mit acht Männern die Flucht vor Johanan und gelangte
zu den Ammonitern.

\bibleverse{16}Hierauf nahm Johanan, der Sohn Kareahs, samt allen
Truppenführern, die bei ihm waren, den gesamten Rest des Volkes, den
Ismael, der Sohn Nethanjas, nach der Ermordung Gedaljas, des Sohnes
Ahikams, gefangen aus Mizpa weggeführt hatte, die Männer und
Kriegsleute, die Weiber, Kinder und Hofbeamten\textless sup title=``vgl.
38,7''\textgreater✲, die er von Gibeon zurückgebracht hatte.
\bibleverse{17}Sie machten sich dann auf den Weg und lagerten
sich\textless sup title=``=~machten Halt''\textgreater✲ in\textless sup
title=``oder: bei''\textgreater✲ der Herberge Kimhams, die in der Nähe
von Bethlehem liegt, um von dort nach Ägypten zu ziehen
\bibleverse{18}(aus Furcht) vor den Chaldäern, vor denen sie sich
fürchteten, weil Ismael, der Sohn Nethanjas, Gedalja, den Sohn Ahikams,
der vom Könige von Babylon zum Statthalter über das Land eingesetzt
worden war, ermordet hatte.

\hypertarget{erfolglose-warnung-jeremias-vor-der-auswanderung-nach-uxe4gypten}{%
\subsubsection{3. Erfolglose Warnung Jeremias vor der Auswanderung nach
Ägypten}\label{erfolglose-warnung-jeremias-vor-der-auswanderung-nach-uxe4gypten}}

\hypertarget{a-jeremia-befragt-gott-im-auftrage-seiner-volksgenossen-bezuxfcglich-der-auswanderung}{%
\paragraph{a) Jeremia befragt Gott im Auftrage seiner Volksgenossen
bezüglich der
Auswanderung}\label{a-jeremia-befragt-gott-im-auftrage-seiner-volksgenossen-bezuxfcglich-der-auswanderung}}

\hypertarget{section-41}{%
\section{42}\label{section-41}}

\bibleverse{1}Nun traten alle Truppenführer, auch Johanan, der Sohn
Kareahs, und Asarja, der Sohn Hosajas, mit der gesamten Volksmenge, groß
und klein, an den Propheten Jeremia heran \bibleverse{2}und sagten zu
ihm: »Schenke doch unserer inständigen Bitte Gehör und bete für uns zum
HERRN, deinem Gott, für diesen ganzen Überrest! Wir sind ja nur als ein
kleines Häuflein von der früheren großen Zahl übriggeblieben, wie du an
uns mit eigenen Augen hier siehst. \bibleverse{3}Der HERR, dein Gott,
wolle uns den Weg angeben, den wir einschlagen sollen, und wie wir uns
zu verhalten haben!« \bibleverse{4}Der Prophet Jeremia antwortete ihnen:
»Ich bin einverstanden. Gut, ich will zum HERRN, eurem Gott, beten, wie
ihr es wünscht, und will euch dann alles kundtun, was der HERR euch
antworten wird: kein Wort will ich euch verschweigen.« \bibleverse{5}Da
sagten sie zu Jeremia: »Der HERR soll ein wahrhaftiger und zuverlässiger
Zeuge gegen uns sein, wenn wir nicht genau der Weisung folgen werden,
die der HERR, dein Gott, uns durch dich wird zukommen lassen!
\bibleverse{6}Mag es uns erwünscht oder unerwünscht sein: -- dem Gebot
des HERRN, unsers Gottes, zu dem wir dich senden, wollen wir gehorchen,
damit es uns gut ergehe, wenn wir der Weisung des HERRN, unsers Gottes,
gehorsam sind!«

\hypertarget{b-jeremia-warnt-im-namen-gottes-vor-der-auswanderung}{%
\paragraph{b) Jeremia warnt im Namen Gottes vor der
Auswanderung}\label{b-jeremia-warnt-im-namen-gottes-vor-der-auswanderung}}

\bibleverse{7}Als dann nach Ablauf von zehn Tagen das Wort des HERRN an
Jeremia ergangen war, \bibleverse{8}ließ er Johanan, den Sohn Kareahs,
samt allen Truppenführern, die bei ihm waren, und das gesamte Volk, groß
und klein, rufen \bibleverse{9}und sagte zu ihnen: »So hat der HERR, der
Gott Israels, zu dem ihr mich gesandt habt, um eure inständige Bitte vor
ihn zu bringen, zu mir gesprochen: \bibleverse{10}›Wenn ihr ruhig in
diesem Lande wohnen bleibt, so will ich euch aufbauen, ohne wieder
abzubrechen, und will euch einpflanzen, ohne wieder auszureißen; denn
mir tut das Unheil leid, das ich euch habe widerfahren lassen.
\bibleverse{11}Fürchtet euch nicht vor dem Könige von Babylon, vor dem
ihr jetzt in Angst seid! Fürchtet euch nicht vor ihm!‹ -- so lautet der
Ausspruch des HERRN --; ›denn ich bin mit euch, um euch zu helfen und
euch aus seiner Hand zu erretten. \bibleverse{12}Ich will euch Gnade bei
ihm finden lassen, daß er sich euer erbarmt und euch auf eurem Grund und
Boden wohnen läßt!‹ \bibleverse{13}Wenn ihr aber sagt: ›Wir wollen nicht
in diesem Lande bleiben‹, so daß ihr der Weisung des HERRN, eures
Gottes, nicht gehorcht, \bibleverse{14}sondern sagt: ›Nein! Vielmehr
nach Ägypten wollen wir ziehen, wo wir keinen Krieg mehr sehen und
keinen Trompetenschall mehr hören und nicht mehr nach Brot zu hungern
brauchen, und dort wollen wir uns niederlassen!‹~-- \bibleverse{15}nun
denn, so vernehmt das Wort des HERRN, ihr von Juda Übriggebliebenen! So
hat der HERR der Heerscharen, der Gott Israels, gesprochen: ›Wenn ihr
wirklich den Beschluß faßt, nach Ägypten zu ziehen, und euch dorthin
begebt, um euch dort in fremdem Lande anzusiedeln, \bibleverse{16}so
wird das Schwert, vor dem ihr euch jetzt fürchtet, euch dort im Lande
Ägypten erreichen; und der Hunger, vor dem euch jetzt bange ist, wird
sich euch dort in Ägypten an die Fersen heften, so daß ihr dort umkommen
werdet! \bibleverse{17}Ja alle Männer, welche den Beschluß fassen, nach
Ägypten zu ziehen, um dort in fremdem Lande zu leben, werden durch das
Schwert, durch den Hunger und die Pest umkommen, und es soll keinen
unter ihnen geben, der am Leben bleibt und dem Unheil entrinnt, das ich
über sie verhängen werde!‹«

\hypertarget{jeremia-wiederholt-die-guxf6ttliche-drohung}{%
\paragraph{Jeremia wiederholt die göttliche
Drohung}\label{jeremia-wiederholt-die-guxf6ttliche-drohung}}

\bibleverse{18}»Denn so hat der HERR der Heerscharen, der Gott Israels,
gesprochen: ›Wie mein Zorn und Grimm sich über die Bewohner Jerusalems
ergossen hat, ebenso wird mein Grimm sich über euch ergießen, wenn ihr
nach Ägypten zieht, und ihr sollt zur Verwünschung und zum
abschreckenden Beispiel, zum Fluchwort und zur Beschimpfung werden und
diese Stätte\textless sup title=``oder: Gegend''\textgreater✲ nicht
wiedersehen!‹ \bibleverse{19}Dies ist es, was der HERR euch, den von
Juda Übriggebliebenen, sagen läßt: ›Zieht nicht nach Ägypten!‹ Bedenkt
wohl, daß ich euch heute ernstlich gewarnt habe! \bibleverse{20}Ihr habt
euch ja selbst um den Preis des eigenen Lebens auf einen Irrweg begeben;
denn ihr habt mich zum HERRN, eurem Gott, gesandt mit dem Auftrag: ›Bete
für uns zum HERRN, unserm Gott! Und genau so, wie der HERR, unser Gott,
es gebieten wird, so verkünde es uns, damit wir dann danach tun!‹
\bibleverse{21}Nun habe ich es euch heute verkündigt, aber ihr wollt der
Weisung des HERRN, eures Gottes, nicht gehorchen, und zwar so, daß ihr
alles zurückweist, was er mir an euch aufgetragen hat. \bibleverse{22}So
wisset denn bestimmt, daß ihr durch das Schwert, durch den Hunger und
die Pest den Tod finden werdet an dem Orte, wohin es euch zu ziehen
gelüstet, um dort als Fremdlinge zu wohnen!«

\hypertarget{c-ungehorsam-der-gewarnten-jeremia-und-baruch-werden-wider-ihren-willen-nach-uxe4gypten-mitgenommen}{%
\paragraph{c) Ungehorsam der Gewarnten; Jeremia und Baruch werden wider
ihren Willen nach Ägypten
mitgenommen}\label{c-ungehorsam-der-gewarnten-jeremia-und-baruch-werden-wider-ihren-willen-nach-uxe4gypten-mitgenommen}}

\hypertarget{section-42}{%
\section{43}\label{section-42}}

\bibleverse{1}Als nun Jeremia dem ganzen✲ Volk alle Worte, deren
Verkündigung ihm vom HERRN, ihrem Gott, aufgetragen worden war, bis zu
Ende mitgeteilt hatte, alle jene Worte, \bibleverse{2}da sagten Asarja,
der Sohn Hosajas, und Johanan, der Sohn Kareahs, und alle übrigen
widerspenstigen Männer, die mit Jeremia redeten: »Du redest die
Unwahrheit! Der HERR, unser Gott, hat dich nicht gesandt, um (uns) zu
gebieten: ›Ihr sollt nicht nach Ägypten ziehen, um dort in fremdem Lande
zu wohnen!‹, \bibleverse{3}sondern Baruch, der Sohn Nerijas, hetzt dich
gegen uns auf, in der Absicht, uns den Chaldäern in die Hände zu
liefern, damit sie uns töten und uns nach Babylon in die
Verbannung\textless sup title=``oder: Gefangenschaft''\textgreater✲
führen!« \bibleverse{4}So kamen denn Johanan, der Sohn Kareahs, samt
allen Truppenführern und das gesamte Volk der Weisung des HERRN, im
Lande Juda zu bleiben, nicht nach; \bibleverse{5}vielmehr nahm Johanan,
der Sohn Kareahs, samt allen Truppenführern den gesamten Überrest der
Judäer, alle, die aus allen Völkerschaften, wohin sie sich zerstreut
hatten, zurückgekehrt waren, um sich im Lande Juda niederzulassen,
\bibleverse{6}die Männer samt den Weibern und Kindern, die königlichen
Frauen, überhaupt alle die Personen, die Nebusaradan, der Befehlshaber
der Leibwache, bei Gedalja, dem Sohne Ahikams, des Sohnes Saphans,
zurückgelassen hatte, unter ihnen auch den Propheten Jeremia und Baruch,
den Sohn Nerijas, \bibleverse{7}und zogen im Ungehorsam gegen die
Weisung des HERRN nach Ägypten. Sie kamen dann nach
Thachpanches\textless sup title=``=~Daphne; vgl. 2,16''\textgreater✲.

\hypertarget{jeremia-kuxfcndigt-in-thachpanches-die-baldige-unterwerfung-uxe4gyptens-durch-nebukadnezar-an}{%
\subsubsection{4. Jeremia kündigt in Thachpanches die baldige
Unterwerfung Ägyptens durch Nebukadnezar
an}\label{jeremia-kuxfcndigt-in-thachpanches-die-baldige-unterwerfung-uxe4gyptens-durch-nebukadnezar-an}}

\bibleverse{8}Da erging das Wort des HERRN an Jeremia in
Thachpanches\textless sup title=``=~Daphne; vgl. 2,16''\textgreater✲
folgendermaßen: \bibleverse{9}»Hole große Steine herbei und grabe sie in
den Lehmboden\textless sup title=``oder: Schutt''\textgreater✲ ein beim
Ziegel-Ofen, der am Eingang zum Palast des Pharaos in Thachpanches
steht, im Beisein judäischer Männer, \bibleverse{10}und sage zu ihnen:
›So hat der HERR der Heerscharen, der Gott Israels, gesprochen: Wisset
wohl: ich will meinen Knecht Nebukadnezar, den König von Babylon,
herkommen lassen und seinen Thron auf diesen Steinen, die du hier
eingesenkt hast, aufstellen; er wird dann sein Thronzelt über ihnen
ausspannen. \bibleverse{11}Er wird herkommen und das Land Ägypten
schlagen: was für den Tod\textless sup title=``=~die Pest''\textgreater✲
bestimmt ist, verfällt dem Tode, was für die Gefangenschaft bestimmt
ist, der Gefangenschaft, und was für das Schwert bestimmt ist, dem
Schwert. \bibleverse{12}Dann wird er Feuer an die Tempel der Götter
Ägyptens legen und sie verbrennen, und er wird sie wegführen und das
Land Ägypten um sich wickeln, wie der Hirt seinen Mantel um sich
wickelt, und dann unbehelligt wieder abziehen. \bibleverse{13}Die
Obelisken des Sonnentempels (von Heliopolis) im Lande Ägypten wird er
zertrümmern und die Tempel der ägyptischen Götter in Flammen aufgehen
lassen.‹«

\hypertarget{jeremias-letzter-kampf-gegen-die-abguxf6tterei-des-volkes-in-uxe4gypten}{%
\subsubsection{5. Jeremias letzter Kampf gegen die Abgötterei des Volkes
in
Ägypten}\label{jeremias-letzter-kampf-gegen-die-abguxf6tterei-des-volkes-in-uxe4gypten}}

\hypertarget{a-jeremias-drohrede-gegen-den-guxf6tzendienst-der-juden}{%
\paragraph{a) Jeremias Drohrede gegen den Götzendienst der
Juden}\label{a-jeremias-drohrede-gegen-den-guxf6tzendienst-der-juden}}

\hypertarget{section-43}{%
\section{44}\label{section-43}}

\bibleverse{1}(Dies ist) das Wort, das an Jeremia erging in betreff
aller in Ägypten wohnenden Judäer, die sich in Migdol und in
Thachpanches, in Noph und im Gebiet von Pathros niedergelassen hatten;
dasselbe lautet: \bibleverse{2}»So hat der HERR der Heerscharen, der
Gott Israels, gesprochen: ›Ihr selbst habt all das Unglück gesehen✲, das
ich über Jerusalem und alle Städte Judas verhängt habe; ihr wißt, sie
liegen heutigestags in Trümmern und sind unbewohnt \bibleverse{3}infolge
ihrer Bosheit, die sie verübt haben, um mich zu erbittern, indem sie
hingingen, um anderen Göttern zu opfern und zu dienen, die sie nicht
kannten, weder sie noch ihr, noch eure Väter. \bibleverse{4}Wohl hatte
ich alle meine Knechte, die Propheten, früh und spät immer wieder zu
euch gesandt mit der Mahnung: Verübt doch solchen Greuel nicht, den ich
hasse! \bibleverse{5}Aber sie wollten nicht gehorchen und schenkten mir
kein Gehör, daß sie von ihrem bösen Tun abgelassen und anderen Göttern
nicht mehr geopfert hätten. \bibleverse{6}Da ergoß sich denn mein Grimm
und mein Zorn und loderte in den Städten Judas und in den Straßen
Jerusalems auf, so daß sie zu öden Trümmerstätten wurden, wie sie es
heutigestags noch sind. \bibleverse{7}Und nun‹ -- so hat der HERR der
Heerscharen, der Gott Israels, gesprochen --: ›Warum richtet ihr ein so
großes Unheil gegen euch selbst an, daß ihr bei euch Männer und Weiber,
Kinder und Säuglinge aus Juda ausrottet, so daß ihr keinen Rest mehr für
euch übriglaßt? \bibleverse{8}Ihr reizt mich ja zum Zorn durch das
Tun\textless sup title=``oder: die Machwerke''\textgreater✲ eurer Hände,
indem ihr anderen Göttern in Ägypten opfert, wohin ihr euch begeben
habt, um dort als Fremdlinge zu wohnen -- (allerdings mit dem Ergebnis),
daß ihr der Vernichtung verfallt und zu einem Fluchwort und zur
Beschimpfung bei allen Völkern der Erde werdet. \bibleverse{9}Habt ihr
die Übeltaten eurer Väter vergessen und die Übeltaten der Könige von
Juda und die Übeltaten ihrer Weiber und eure eigenen Übeltaten und das
viele Böse, das eure Weiber im Lande Juda und in den Straßen Jerusalems
verübt haben? \bibleverse{10}Noch heutigestags sind sie nicht
zerknirscht und fürchten sich nicht und wandeln nicht nach meinem Gesetz
und nach meinen Geboten, die ich euch und euren Vätern zur Pflicht
gemacht habe. \bibleverse{11}Darum‹ -- so hat der HERR der Heerscharen,
der Gott Israels, gesprochen --: ›nunmehr will ich mein Angesicht gegen
euch richten zum Unheil, und zwar um ganz Juda auszurotten;
\bibleverse{12}und ich will den Überrest der Judäer hinwegraffen, deren
Absicht darauf gerichtet (gewesen) ist, nach Ägypten zu ziehen, um dort
als Fremdlinge zu wohnen: sie sollen alle vertilgt werden! in Ägypten
sollen sie fallen, durch das Schwert und durch den Hunger sollen sie
aufgerieben werden, klein und groß, durch das Schwert und durch den
Hunger sollen sie umkommen und zu einem abschreckenden Beispiel, zu
einem Fluchwort, zur Verwünschung und Beschimpfung werden!
\bibleverse{13}Ja, heimsuchen will ich die, welche sich in Ägypten
niedergelassen haben, wie ich Jerusalem heimgesucht habe, durchs
Schwert, durch Hunger und durch die Pest; \bibleverse{14}und unter dem
Überrest der Judäer, die hergekommen sind, um hier in Ägypten als
Fremdlinge zu wohnen, soll es keinen geben, der seinem Geschick entgeht
und am Leben bleibt, um ins Land Juda zurückzukehren, wohin sie sich
zurücksehnen und wo sie gern wieder wohnen möchten; denn sie sollen
nicht dorthin zurückkehren außer einigen Entronnenen\textless sup
title=``oder: Flüchtlingen''\textgreater✲!‹«

\hypertarget{b-die-gemeinde-besonders-die-frauen-erkluxe4ren-offen-der-himmelskuxf6nigin-dienen-zu-wollen}{%
\paragraph{b) Die Gemeinde, besonders die Frauen, erklären offen, der
Himmelskönigin dienen zu
wollen}\label{b-die-gemeinde-besonders-die-frauen-erkluxe4ren-offen-der-himmelskuxf6nigin-dienen-zu-wollen}}

\bibleverse{15}Da antworteten dem Jeremia alle Männer, welche wußten,
daß ihre Frauen anderen Göttern räucherten, und alle Frauen, die in
großer Schar dabei standen, und das gesamte Volk, das in Ägypten (und)
Pathros\textless sup title=``=~in Unter- und Oberägypten; vgl.
V.1''\textgreater✲ wohnte, folgendermaßen: \bibleverse{16}»Was die
Forderung betrifft, die du im Namen des HERRN an uns gerichtet hast, so
wisse, daß wir auf dich nicht hören! \bibleverse{17}Wir wollen vielmehr
das Gelübde, das wir geleistet haben, nämlich der Himmelskönigin zu
räuchern\textless sup title=``oder: Opfer zu verbrennen''\textgreater✲
und ihr Trankopfer zu spenden, getreulich ausführen, ganz so wie wir und
unsere Väter, unsere Könige und Fürsten\textless sup title=``oder:
Oberen''\textgreater✲ es in den Ortschaften Judas und auf den Straßen
Jerusalems getan haben! Damals hatten wir Brot in Fülle, befanden uns
wohl und wußten nichts von Unglück. \bibleverse{18}Aber seitdem wir
aufgehört haben, der Himmelskönigin zu räuchern\textless sup
title=``oder: Opfer zu verbrennen''\textgreater✲ und ihr Trankopfer zu
spenden, haben wir Mangel an allem gelitten und sind durch das Schwert
und durch den Hunger aufgerieben worden. \bibleverse{19}Und wenn wir der
Himmelskönigin jetzt (wieder) Opfer verbrennen und ihr Trankopfer
spenden -- geschieht es etwa ohne die Zustimmung unserer Ehemänner, daß
wir ihr zu Ehren Kuchen backen, indem wir ihre Gestalt
darauf\textless sup title=``oder: dadurch''\textgreater✲ abbilden, und
ihr Trankopfer spenden?«

\hypertarget{c-jeremias-zuruxfcckweisung-ihrer-ausreden-und-seine-abkehr-von-ihnen}{%
\paragraph{c) Jeremias Zurückweisung ihrer Ausreden und seine Abkehr von
ihnen}\label{c-jeremias-zuruxfcckweisung-ihrer-ausreden-und-seine-abkehr-von-ihnen}}

\bibleverse{20}Da gab Jeremia dem gesamten✲ Volk, den Männern und Frauen
und allen denen, die ihm mit solchen Reden entgegengetreten waren,
folgende Antwort: \bibleverse{21}»Jawohl, die Räucherei, die ihr in den
Ortschaften Judas und in den Straßen Jerusalems getrieben habt, ihr und
eure Väter, eure Könige und Fürsten\textless sup title=``oder:
Oberen''\textgreater✲ und die Bevölkerung des Landes -- hat der HERR
ihrer etwa nicht gedacht und sie nicht in Erinnerung behalten?
\bibleverse{22}Ja, weil der HERR es wegen eures verwerflichen Treibens
und wegen der Greuel, die ihr verübtet, nicht länger ertragen konnte,
darum ist euer Land zur Einöde, zum abschreckenden Beispiel und zu einem
Fluchwort geworden, leer von Bewohnern, wie es jetzt noch der Fall ist!
\bibleverse{23}Eben zur Strafe dafür, daß ihr (den Götzen) Opfer
verbrannt und dadurch gegen den HERRN gesündigt und auf die Weisungen
des HERRN nicht gehört und nicht nach seinem Gesetz und seinen Geboten
und Vorschriften gelebt habt: eben darum ist dieses Unglück, in dem ihr
euch gegenwärtig befindet, über euch gekommen!«~-- \bibleverse{24}Weiter
sagte Jeremia zu dem gesamten Volk und besonders zu allen Frauen:
»Vernehmt das Wort des HERRN, ihr Judäer alle, die ihr in Ägypten wohnt!
\bibleverse{25}So hat der HERR der Heerscharen, der Gott Israels,
gesprochen: ›Ihr und eure Frauen, ihr habt es mit eurem Munde gelobt und
führt es auch tatsächlich aus! Ihr sagt: Wir wollen unsere Gelübde, die
wir geleistet haben, nämlich der Himmelskönigin Opfer zu verbrennen und
ihr Trankopfer zu spenden, getreulich ausführen! So erfüllt denn ja eure
Gelübde und führt getreulich das aus, was ihr gelobt habt!‹«

\hypertarget{d-jeremias-letzte-ungluxfccksweissagung-uxfcber-die-uxe4gyptische-gemeinde}{%
\paragraph{d) Jeremias letzte Unglücksweissagung über die ägyptische
Gemeinde}\label{d-jeremias-letzte-ungluxfccksweissagung-uxfcber-die-uxe4gyptische-gemeinde}}

\bibleverse{26}»Darum vernehmt das Wort des HERRN, ihr Judäer alle, die
ihr in Ägypten wohnt: ›Fürwahr, ich schwöre bei meinem großen Namen‹ --
so hat der HERR gesprochen --: ›Niemals soll fortan noch mein Name in
ganz Ägypten von irgend einem Judäer in den Mund genommen werden, daß er
etwa sagte: So wahr Gott der HERR lebt! \bibleverse{27}Wisset wohl: ich
will die Augen über ihnen offen halten zum Verderben, nicht zum Heil!
Und es sollen alle Judäer, die im Lande Ägypten weilen, durch das
Schwert und durch den Hunger umkommen, bis sie völlig vernichtet sind!
\bibleverse{28}Ja die dem Schwert Entronnenen, die aus dem Lande Ägypten
ins Land Juda heimkehren, sollen nur wenige an Zahl sein; dann wird der
gesamte Überrest der Judäer, die nach Ägypten gezogen sind, um sich dort
als Fremdlinge aufzuhalten, -- der wird dann erkennen, wessen Wort sich
verwirklicht, das meine oder das ihre! \bibleverse{29}Und dies soll für
euch das Zeichen\textless sup title=``=~die Bürgschaft''\textgreater✲
sein‹ -- so lautet der Ausspruch des HERRN --, ›daß ich euch an diesem
Orte heimsuchen werde -- damit ihr erkennt, daß meine Unheilsdrohungen
gegen euch unfehlbar in Erfüllung gehen werden‹: \bibleverse{30}so hat
der HERR gesprochen: ›Fürwahr, ich will den Pharao Hophra, den König von
Ägypten, in die Hand seiner Gegner und Todfeinde fallen lassen, so wie
ich Zedekia, den König von Juda, in die Hand Nebukadnezars, des Königs
von Babylon, seines Gegners und Todfeindes, habe fallen lassen.‹«

\hypertarget{jeremias-mahn--und-trostworte-an-baruch}{%
\subsubsection{6. Jeremias Mahn- und Trostworte an
Baruch}\label{jeremias-mahn--und-trostworte-an-baruch}}

\hypertarget{section-44}{%
\section{45}\label{section-44}}

\bibleverse{1}(Dies ist) das Wort, das der Prophet Jeremia an Baruch,
den Sohn Nerijas richtete, als dieser im vierten Regierungsjahre
Jojakims, des Sohnes Josias, des Königs von Juda, die betreffenden
Worte\textless sup title=``oder: Reden''\textgreater✲ so, wie Jeremia
sie ihm vorsagte, in ein Buch niedergeschrieben hatte\textless sup
title=``vgl. Kap. 36''\textgreater✲; das Wort lautete: \bibleverse{2}»So
hat der HERR, der Gott Israels, in bezug auf dich, Baruch, gesprochen:
\bibleverse{3}›Du klagst: O wehe mir! Der HERR fügt noch Kummer zu
meinem Schmerz hinzu! Müde bin ich von allem Seufzen und finde keine
Ruhe!‹ \bibleverse{4}Sage zu ihm: ›So hat der HERR gesprochen: Wisse
wohl: was ich selbst gebaut habe, das breche ich wieder ab, und was ich
gepflanzt habe, das reiße ich wieder aus {[}und zwar betrifft dies die
ganze Erde{]}; \bibleverse{5}und da willst du Großes für dich verlangen?
Verlange es nicht! Denn bedenke wohl: ich verhänge Unglück über alles
Fleisch\textless sup title=``=~alle Menschen''\textgreater✲‹, -- so
lautet der Ausspruch des HERRN --; ›dir aber gewähre ich, mit dem Leben
davonzukommen an allen Orten, wohin du dich begeben wirst!‹«

\hypertarget{ii.-weissagungen-gegen-fremde-heidnische-vuxf6lker-kap.-46-51}{%
\subsection{II. Weissagungen gegen fremde (heidnische) Völker (Kap.
46-51)}\label{ii.-weissagungen-gegen-fremde-heidnische-vuxf6lker-kap.-46-51}}

\hypertarget{section-45}{%
\section{46}\label{section-45}}

\bibleverse{1}Was als Wort des HERRN an den Propheten Jeremia in betreff
der (heidnischen) Völker ergangen ist:

\hypertarget{zwei-ausspruxfcche-uxfcber-uxe4gypten}{%
\subsubsection{1. Zwei Aussprüche über
Ägypten}\label{zwei-ausspruxfcche-uxfcber-uxe4gypten}}

\hypertarget{a-erster-ausspruch-der-stolze-aufmarsch-der-uxe4gypter-ihre-niederlage-bei-karchemisch}{%
\paragraph{a) Erster Ausspruch: Der stolze Aufmarsch der Ägypter; ihre
Niederlage bei
Karchemisch}\label{a-erster-ausspruch-der-stolze-aufmarsch-der-uxe4gypter-ihre-niederlage-bei-karchemisch}}

\bibleverse{2}Über Ägypten: in betreff des Heeres des Pharaos Necho, des
Königs von Ägypten, das am Euphratstrom bei Karchemisch stand und von
Nebukadnezar, dem König von Babylon, im vierten Regierungsjahre des
judäischen Königs Jojakim, des Sohnes Josias, geschlagen wurde:

\bibleverse{3}Rüstet Schild und Tartsche\textless sup title=``=~Groß-
und Kleinschild''\textgreater✲ und tretet an zum Kampf!
\bibleverse{4}Schirret die Rosse an und sitzet auf, ihr Reiter! Stellt
euch auf im Helmschmuck, macht die Lanzen scharf, legt euch die Panzer
an! \bibleverse{5}Warum sehe ich sie verzagt zurückweichen? Warum sind
ihre Mannen mutlos und ergreifen die Flucht, ohne sich umzuwenden?
»Entsetzen ringsum!« -- so lautet der Ausspruch des HERRN --;
\bibleverse{6}»der Behendeste kann nicht entfliehen und der Tapferste
nicht entrinnen! Dort im Norden, am Ufer des Euphratstromes, sind sie
gestrauchelt und zu Fall gekommen!«

\bibleverse{7}Wer war's doch, der wie der Nil emporstieg, daß seine
Fluten wie Ströme wogten? \bibleverse{8}Ägypten stieg wie der Nil empor,
daß seine Fluten wie Ströme wogten, und es drohte: »Ich will
emporsteigen, das Land\textless sup title=``oder: die
Erde''\textgreater✲ überschwemmen, will Städte vertilgen samt ihren
Bewohnern!« \bibleverse{9}Stürmt heran, ihr Rosse, und rast daher, ihr
Wagen! Und die Mannen\textless sup title=``oder: Krieger''\textgreater✲
mögen ausrücken, die Äthiopier und die schildbewehrten Putäer und die
Luditer, die den Bogen führen und spannen! \bibleverse{10}Ja, dieser Tag
ist für Gott, den HERRN der Heerscharen, ein Tag der Rache, um seinen
Widersachern zu vergelten: da frißt das Schwert, bis es satt ist, und
berauscht sich an ihrem Blut, denn ein Schlachtfest✲ hält Gott, der HERR
der Heerscharen, im Nordland am Euphratstrom. \bibleverse{11}Gehe nach
Gilead hinauf und hole Balsam, jungfräuliche Tochter Ägypten! Umsonst
wendest du ein Heilmittel nach dem andern an: für dich gibt's keine
Heilung mehr! \bibleverse{12}Die Völker vernehmen deine Schande, und die
Erde hallt von deinem Wehgeschrei wider; denn ein Krieger ist über den
andern gefallen: miteinander sind beide niedergestürzt\textless sup
title=``oder: beisammen liegen beide da''\textgreater✲!

\hypertarget{b-zweiter-ausspruch-schwere-kriegsnot-uxe4gyptens-infolge-der-eroberung-durch-nebukadnezar-trostworte-an-israel}{%
\paragraph{b) Zweiter Ausspruch: Schwere Kriegsnot Ägyptens infolge der
Eroberung durch Nebukadnezar; Trostworte an
Israel}\label{b-zweiter-ausspruch-schwere-kriegsnot-uxe4gyptens-infolge-der-eroberung-durch-nebukadnezar-trostworte-an-israel}}

\bibleverse{13}(Dies ist) das Wort, das der HERR an den Propheten
Jeremia gerichtet hat, als Nebukadnezar, der König von Babylon, kommen
sollte, um das Land Ägypten niederzuwerfen: \bibleverse{14}Verkündet es
in Ägypten und ruft es in Migdol\textless sup title=``vgl.
44,1''\textgreater✲ aus, ruft es auch in Memphis und Daphne aus!
Gebietet: Stelle dich auf (zur Wehr) und mache dich bereit, denn schon
frißt das Schwert rings um dich her! \bibleverse{15}Warum sind deine
Helden niedergeworfen? Sie haben nicht standgehalten, denn der HERR hat
sie niedergestoßen. \bibleverse{16}Er hat viele straucheln lassen; ja,
einer stürzt über den andern, so daß sie ausrufen: »Auf! Laßt uns
heimkehren zu unserm Volk und in unser Heimatland vor dem gewalttätigen✲
Schwert!« \bibleverse{17}Nennt den Namen des Pharaos, des Königs von
Ägypten: ›Toben, das den richtigen Zeitpunkt versäumt hat‹.
\bibleverse{18}»So wahr ich lebe« -- so lautet der Ausspruch des Königs,
dessen Name ›HERR der Heerscharen‹ ist --: »Wie der Thabor unter den
Bergen und wie der Karmel am Meer, so wird er\textless sup title=``d.h.
Nebukadnezar''\textgreater✲ heranziehen! \bibleverse{19}Setze dir die
Geräte zur Auswanderung in Bereitschaft, du Einwohnerschaft, Tochter
Ägypten! Denn Memphis wird zur Einöde werden, wird eingeäschert,
menschenleer!« \bibleverse{20}Eine wunderschöne junge Kuh ist Ägypten;
aber es kommt, ja es kommt die Bremse von Norden her.
\bibleverse{21}Auch seine Söldner, die es in seiner Mitte wie Mastkälber
hat, ja, auch sie haben kehrt gemacht, haben sich insgesamt zur Flucht
gewandt und nicht standgehalten, denn ihr Unglückstag ist über sie
hereingebrochen, die Zeit ihrer Heimsuchung\textless sup title=``oder:
Strafe''\textgreater✲! \bibleverse{22}Man hört etwas daherkommen wie das
Rascheln\textless sup title=``oder: Zischeln''\textgreater✲ einer
Schlange, die davoneilt; denn sie\textless sup title=``d.h. die
Feinde''\textgreater✲ rücken mit Heeresmacht heran und fallen mit Äxten
über das Land her wie Holzhauer. \bibleverse{23}»Sie hauen seinen Wald
um« -- so lautet der Ausspruch des HERRN --, »der
unübersehbar\textless sup title=``oder: undurchdringlich''\textgreater✲
ist; denn ihrer sind mehr als der Heuschrecken, und unzählbar ist ihre
Menge!« \bibleverse{24}Zuschanden wird die Tochter Ägypten, der Gewalt
des nordischen Volkes wird sie preisgegeben! \bibleverse{25}Gesprochen
hat der HERR der Heerscharen, der Gott Israels: »Wisset wohl: ich halte
(jetzt) Abrechnung mit dem Amon von No\textless sup title=``=~Theben in
Oberägypten''\textgreater✲ sowie mit dem Pharao und ganz Ägypten samt
seinen Göttern und Königen, ja mit dem Pharao samt denen, die sich auf
ihn verlassen. \bibleverse{26}Und ich gebe sie in die Gewalt ihrer
Todfeinde, und zwar in die Gewalt Nebukadnezars, des Königs von Babylon,
und in die Gewalt seiner Knechte\textless sup title=``oder:
Diener''\textgreater✲. Nachmals aber wird das Land wieder bewohnt sein
wie in den Tagen der Vorzeit« -- so lautet der Ausspruch des HERRN.

\hypertarget{trostwort-fuxfcr-israel}{%
\paragraph{Trostwort für Israel}\label{trostwort-fuxfcr-israel}}

\bibleverse{27}»Du aber, fürchte dich nicht, mein Knecht Jakob, und laß
dir nicht bange sein, Israel! Denn wisse wohl: ich will dich erretten
aus fernen Landen und deine Kinder\textless sup title=``oder:
Angehörigen''\textgreater✲ aus dem Lande ihrer Gefangenschaft, dann wird
Jakob heimkehren und in Ruhe und Sicherheit leben, ohne daß jemand ihn
aufschreckt. \bibleverse{28}Du also, fürchte dich nicht, mein Knecht
Jakob!« -- so lautet der Ausspruch des HERRN --, »ich bin ja mit dir;
denn über alle Völker, unter die ich dich zerstreut habe, will ich
völlige Vernichtung bringen; dich aber will ich nicht völlig vernichten,
sondern dich mit Maßen\textless sup title=``oder: nach
Billigkeit''\textgreater✲ züchtigen; denn ganz ungestraft kann ich dich
nicht lassen.«

\hypertarget{ausspruch-uxfcber-das-philisterland}{%
\subsubsection{2. Ausspruch über das
Philisterland}\label{ausspruch-uxfcber-das-philisterland}}

\hypertarget{section-46}{%
\section{47}\label{section-46}}

\bibleverse{1}(Dies ist) das Wort des HERRN, das an den Propheten
Jeremia in betreff der Philister ergangen ist, bevor der Pharao Gaza
erobert hatte: \bibleverse{2}So hat der HERR gesprochen: Sehet, Wasser
fluten heran von Norden her und werden zu einem überwallenden Wildbach!
Sie überschwemmen das Land und alles, was darin ist, die Städte samt
ihren Bewohnern, so daß die Menschen laut schreien und alle Bewohner des
Landes heulen! \bibleverse{3}Vor dem dröhnenden Hufschlag seiner
Hengste, vor dem Rasseln seiner Kriegswagen, dem Rollen seiner Räder
wenden Väter sich nicht nach ihren Kindern um, weil die Angst ihre Arme
lähmt; \bibleverse{4}denn der Tag ist gekommen, der allen Philistern den
Untergang bringt und für Tyrus und Sidon den letzten Helfer ausrottet,
der noch übriggeblieben ist; denn der HERR will die Philister
vernichten, die Abkömmlinge der von der Insel Kaphtor✲ Gekommenen.
\bibleverse{5}Gaza hat sich kahl geschoren, zerstört ist Askalon; o
Überrest der Enakiter, wie lange wirst du dich noch blutig ritzen
müssen! \bibleverse{6}Wehe, Schwert des HERRN! Wann wirst du endlich zur
Ruhe kommen? Fahre in deine Scheide zurück, mache ein Ende und halte
dich still! \bibleverse{7}Aber wie sollte es zur Ruhe kommen, da doch
der HERR es entboten hat? Gegen Askalon und gegen das ganze Gestade des
Meeres, dorthin hat er es beordert!

\hypertarget{ausspruxfcche-uxfcber-moab}{%
\subsubsection{3. Aussprüche über
Moab}\label{ausspruxfcche-uxfcber-moab}}

\hypertarget{a-die-verwuxfcstung-des-landes-und-die-allgemeine-flucht}{%
\paragraph{a) Die Verwüstung des Landes und die allgemeine
Flucht}\label{a-die-verwuxfcstung-des-landes-und-die-allgemeine-flucht}}

\hypertarget{section-47}{%
\section{48}\label{section-47}}

\bibleverse{1}Über Moab: So hat der HERR der Heerscharen, der Gott
Israels, gesprochen: Wehe über Nebo, denn es ist verwüstet! Zuschanden
geworden, erobert ist Kirjathaim: zuschanden geworden ist die hohe Feste
und gestürzt! \bibleverse{2}Dahin ist der Ruhm der Moabiter! In Hesbon
sinnt man auf Unheil gegen sie: »Kommt, wir wollen sie ausrotten, daß
sie kein Volk mehr sind!« Auch du, Madmen, wirst vernichtet werden: das
Schwert fährt hinter dir her! \bibleverse{3}Horch! Wehgeschrei schallt
von Horonaim her: »Verwüstung und gewaltiger Zusammenbruch!«
\bibleverse{4}Zertrümmert ist das Moabiterland: sein Wehgeschrei
erschallt bis Zoar hin! \bibleverse{5}Ach, die Anhöhe von Luhith steigt
man unter Weinen hinan! Ach, am Abhang von Horonaim hört man
Angstgeschrei über die Vernichtung: \bibleverse{6}»Fliehet, rettet euer
Leben und fristet es gleich dem Wacholderstrauch in der Wüste!«

\hypertarget{b-moabs-schuld-und-strafe}{%
\paragraph{b) Moabs Schuld und Strafe}\label{b-moabs-schuld-und-strafe}}

\bibleverse{7}Denn weil du dich auf deine Machwerke\textless sup
title=``d.h. Götzenbilder''\textgreater✲ und auf deine Schätze verlassen
hast, sollst nun auch du erobert werden, und Kamos muß in die
Verbannung\textless sup title=``oder: Gefangenschaft''\textgreater✲
wandern, seine Priester und Oberen\textless sup title=``oder:
Häuptlinge''\textgreater✲ allzumal; \bibleverse{8}und es kommt der
Verwüster über alle deine Städte; keine einzige wird verschont bleiben;
auch das Tal unten geht zugrunde, und die Ebene oben wird verheert, wie
der HERR angedroht hat. \bibleverse{9}Gebt Moab Flügel, damit es auf und
davon fliege! Und seine Städte werden zur Einöde werden, so daß niemand
mehr darin wohnt. \bibleverse{10}Verflucht sei, wer das Werk des HERRN
lässig betreibt, und verflucht, wer sein Schwert vom Blutvergießen
zurückhält!

\hypertarget{c-auf-moabs-lange-ruhezeit-und-sorglosigkeit-folgt-not-und-vernichtung}{%
\paragraph{c) Auf Moabs lange Ruhezeit und Sorglosigkeit folgt Not und
Vernichtung}\label{c-auf-moabs-lange-ruhezeit-und-sorglosigkeit-folgt-not-und-vernichtung}}

\bibleverse{11}Sorglos hat Moab von Jugend auf gelebt und ungestört auf
seinen Hefen geruht: es ist nie aus einem Faß in ein anderes umgegossen
worden und niemals in Gefangenschaft gewandert; daher hat es auch seinen
Geschmack beibehalten, und sein Duft ist unverändert geblieben.
\bibleverse{12}»Darum, wisset wohl: es kommt die Zeit« -- so lautet der
Ausspruch des HERRN --, »da will ich ihm Küfer senden, die sollen es
umschütten und seine Fässer entleeren und seine Krüge zerschlagen.
\bibleverse{13}Da wird dann Moab am Kamos✲ zuschanden werden, gleichwie
die vom Hause Israel an Bethel, auf das sie ihre Zuversicht setzten,
zuschanden geworden sind. \bibleverse{14}Wie könnt ihr nur sagen: ›Wir
sind tapfere Krieger und wehrhafte Männer zum Kampf!‹ \bibleverse{15}Der
Verwüster Moabs und seiner Städte ist im Anzug, und seine auserlesene
junge Mannschaft sinkt hin\textless sup title=``oder: steigt
hinab''\textgreater✲ zur Schlachtung« -- so lautet der Ausspruch des
Königs, dessen Name ›HERR der Heerscharen‹ ist --: \bibleverse{16}»Moabs
Untergang steht nahe bevor, und sein Verderben eilt schnell herbei.«

\hypertarget{d-das-bedauerliche-elend-des-landes-und-aller-seiner-stuxe4dte}{%
\paragraph{d) Das bedauerliche Elend des Landes und aller seiner
Städte}\label{d-das-bedauerliche-elend-des-landes-und-aller-seiner-stuxe4dte}}

\bibleverse{17}Bezeugt ihm Beileid, ihr seine Umwohner insgesamt und
alle, die ihr von seinem Ruhm gehört habt! Ruft aus: »Wie ist doch der
starke Herrscherstab zerbrochen, das ruhmvolle Zepter!«
\bibleverse{18}Steige von deinem Ehrenplatz herab und setze dich auf die
nackte Erde, du Bewohnerschaft, Tochter Dibon! Denn der Verwüster Moabs
ist im Anzug gegen dich und zerstört deine Burgen. \bibleverse{19}Tritt
an die Straße und spähe aus, Bewohnerschaft von Aroer! Richte an die
Flüchtlinge und an die Entronnenen die Frage: »Was ist geschehen?«
\bibleverse{20}Moab ist zuschanden geworden, ach, es ist in
Verzweiflung: wehklagt und jammert! Verkündet am Arnon, daß Moab
verwüstet ist! \bibleverse{21}Ja, das Strafgericht ist ergangen über die
weite Ebene, über Holon, über Jahza und Mephaath, \bibleverse{22}über
Dibon, Nebo und Beth-Diblathaim, \bibleverse{23}über Kirjathaim,
Beth-Gamul und Beth-Meon, \bibleverse{24}über Kerioth, Bozra und alle
(anderen) Ortschaften des Moabiterlandes, die fernen wie die nahen!
\bibleverse{25}»Abgehauen ist Moabs Horn und sein Arm zerschmettert!« --
so lautet der Ausspruch des HERRN.

\hypertarget{e-erneute-ankuxfcndigung-der-verwuxfcstung-als-der-strafe-fuxfcr-moabs-spott-uxfcber-israel-und-fuxfcr-moabs-hochmut}{%
\paragraph{e) Erneute Ankündigung der Verwüstung als der Strafe für
Moabs Spott über Israel und für Moabs
Hochmut}\label{e-erneute-ankuxfcndigung-der-verwuxfcstung-als-der-strafe-fuxfcr-moabs-spott-uxfcber-israel-und-fuxfcr-moabs-hochmut}}

\bibleverse{26}Macht Moab trunken! Denn gegen den HERRN hat es sich
überhoben: möge es kopfüber in sein Gespei stürzen und selber auch zum
Gespött werden! \bibleverse{27}Oder ist dir etwa Israel nicht zum
Gespött gewesen? Hat man es etwa unter Dieben ertappt, daß du, sooft du
von ihm sprachst, höhnisch den Kopf schütteltest? \bibleverse{28}Verlaßt
die Städte und macht euch in den Felsklüften heimisch, ihr Bewohner
Moabs, und tut es der Wildtaube gleich, die an den Hängen der gähnenden
Abgründe nistet!

\bibleverse{29}Wir haben von Moabs übergroßem Stolz gehört, von seinem
Hochmut und seinem Stolz, von seinem Trotz und seinem hochfahrenden
Sinn. \bibleverse{30}»Ich kenne wohl« -- so lautet der Ausspruch des
HERRN -- »seinen Übermut und seine eitlen Prahlereien: ihr ganzes Tun
ist unehrlich.« \bibleverse{31}Darum muß ich um Moab wehklagen und um
ganz Moab jammern; über die Bewohner von Kir-Heres seufzt man.
\bibleverse{32}Mehr als man um Jaser geweint hat, muß ich um dich
weinen, Weinstock von Sibma, um dich, dessen Ranken über den
See\textless sup title=``d.h. das Tote Meer''\textgreater✲
hinüberwanderten, ja bis zum See von Jaser reichten: in deine Obsternte
und deine Weinlese ist der Verwüster hereingebrochen, \bibleverse{33}und
verschwunden sind Freude und Jubel aus dem Fruchtgefilde und aus dem
Lande Moab. Dem Wein in den Kufen mache ich ein Ende: man tritt die
Kelter nicht mehr unter Jubelruf; der laute Ruf ist jetzt kein Jubelruf
mehr!

\hypertarget{f-neue-schilderung-der-kriegsnot-moabs-verschiedenartige-zusuxe4tze}{%
\paragraph{f) Neue Schilderung der Kriegsnot Moabs; verschiedenartige
Zusätze}\label{f-neue-schilderung-der-kriegsnot-moabs-verschiedenartige-zusuxe4tze}}

\bibleverse{34}Vom wehklagenden Hesbon her lassen sie ihr Geschrei bis
Eleale, bis Jahaz hin erschallen, von Zoar her bis Horonaim, bis
Eglath-Schelischija hin; ja auch die Wasser von Nimrim sollen zu
Wüsteneien werden. \bibleverse{35}»Und ich will dagegen einschreiten« --
so lautet der Ausspruch des HERRN --, »daß man in Moab noch zur
Opferhöhe hinaufsteigt und seinem Gott dort Opfer anzündet.«
\bibleverse{36}Darum klagt mein Herz um Moab laut wie Flötenschall, und
um die Einwohner von Kir-Heres klagt mein Herz laut wie Flötenschall
darum, daß alles, was sie an Hab und Gut erworben hatten, verloren
gegangen ist. \bibleverse{37}Denn alle Häupter sind zur Glatze geschoren
und alle Bärte abgeschnitten; an allen Händen\textless sup title=``oder:
Armen''\textgreater✲ sind Schnittwunden sichtbar und
Sackleinen\textless sup title=``oder: Trauergewandung''\textgreater✲
umgürtet die Hüften. \bibleverse{38}Auf allen Dächern Moabs und auf
seinen Straßen ist nichts als Trauerklage; denn »ich habe Moab
zerschlagen wie ein Gefäß, an dem niemand Gefallen hat!« -- so lautet
der Ausspruch des HERRN. \bibleverse{39}Wie ist es doch voller
Verzweiflung! Wehklaget! wie hat doch Moab schmählich den Rücken
gewandt! Ja, ein Gegenstand des Spottes und Entsetzens ist Moab für alle
Nachbarvölker geworden!

\hypertarget{g-das-gericht-und-seine-vernichtenden-folgen-trostreicher-hinweis-auf-die-wiederherstellung-moabs}{%
\paragraph{g) Das Gericht und seine vernichtenden Folgen; trostreicher
Hinweis auf die Wiederherstellung
Moabs}\label{g-das-gericht-und-seine-vernichtenden-folgen-trostreicher-hinweis-auf-die-wiederherstellung-moabs}}

\bibleverse{40}Denn so hat der HERR gesprochen: »Seht, einem Adler
gleich fliegt (der Feind) heran und breitet seine Schwingen
über\textless sup title=``oder: gegen''\textgreater✲ Moab aus!«
\bibleverse{41}Die Städte sind bezwungen und die Burgen erobert, und den
moabitischen Kriegern wird an jenem Tage zumute sein wie einem Weibe in
Kindesnöten. \bibleverse{42}Vernichtet wird Moab, daß es kein Volk mehr
ist; denn gegen den HERRN hat es sich überhoben. \bibleverse{43}»Grauen
und Grube und Garn kommen über euch, Bewohner von Moab!« -- so lautet
der Ausspruch des HERRN. \bibleverse{44}»Wer dem Grauen entrinnt, stürzt
in die Grube, und wer der Grube entstiegen ist, fängt sich im
Garn\textless sup title=``Jes 24,17-18''\textgreater✲, wenn ich diese
Schrecken über die Moabiter hereinbrechen lasse im Jahre ihrer
Heimsuchung!« -- so lautet der Ausspruch des HERRN.

\bibleverse{45}Im Schatten Hesbons machen die Flüchtlinge erschöpft
halt; doch Feuer bricht aus Hesbon hervor und eine Flamme mitten aus
Sihons Palast: die versengt die Schläfen der Moabiter und den Scheitel
der Söhne✲ des Kampfgetümmels. \bibleverse{46}Wehe dir, Moab! Verloren
ist das Volk des Kamos!✲; denn deine Söhne sind in die Gefangenschaft
weggeführt und deine Töchter in die Knechtschaft.~--
\bibleverse{47}»Doch ich will das Geschick Moabs am Ende der Tage wieder
wenden!« -- so lautet der Ausspruch des HERRN.

Bis hierher geht der Gerichtsspruch\textless sup title=``oder: das
Strafgericht''\textgreater✲ über Moab.

\hypertarget{ausspruch-uxfcber-die-ammoniter}{%
\subsubsection{4. Ausspruch über die
Ammoniter}\label{ausspruch-uxfcber-die-ammoniter}}

\hypertarget{section-48}{%
\section{49}\label{section-48}}

\bibleverse{1}Über die Ammoniter: So hat der HERR gesprochen: »Hat denn
Israel keine Söhne mehr, oder hat es keinen Erben? Wie kommt es, daß
Milkom die Erbschaft in Gad angetreten und sein Volk in den dortigen
Städten Wohnung genommen hat? \bibleverse{2}Darum wisset wohl: es kommt
die Zeit« -- so lautet der Ausspruch des HERRN --, »da lasse ich gegen
die Ammoniterstadt Rabba Kriegsgeschrei erschallen; sie soll dann zum
Schutthaufen werden, und ihre Tochterstädte sollen in Flammen aufgehen:
da soll dann Israel seine Erben\textless sup title=``d.h. die, welche
ihm sein Erbe genommen haben''\textgreater✲ wieder beerben!« -- so
lautet der Ausspruch des HERRN.

\bibleverse{3}Erhebe Wehgeschrei, Hesbon, denn Ai ist zerstört! Jammert,
ihr Tochterstädte Rabbas, umgürtet euch mit Sackleinen\textless sup
title=``oder: Trauergewändern''\textgreater✲, wehklagt und lauft in den
Hürden hin und her! Denn Milkom✲ muß in die Gefangenschaft wandern,
seine Priester und Oberen\textless sup title=``oder:
Fürsten''\textgreater✲ allzumal! \bibleverse{4}Was prahlst du mit deinen
Tälern? Dein Tal ist überströmt, du abtrünnige Tochter, die im Vertrauen
auf ihre Schätze sich rühmt: »Wer sollte an mich herankommen?«
\bibleverse{5}»Wisse wohl: ich will Schrecken über dich hereinbrechen
lassen von allen Seiten ringsum!« -- so lautet der Ausspruch Gottes, des
HERRN der Heerscharen --; »und ihr sollt weggetrieben werden, ein jeder,
ohne daß er sich umzublicken vermag, und niemand soll die Flüchtigen
wieder sammeln!

\bibleverse{6}Doch nachmals will ich das Geschick der Ammoniter wieder
wenden!« -- so lautet der Ausspruch des HERRN.

\hypertarget{ausspruxfcche-uxfcber-die-edomiter}{%
\subsubsection{5. Aussprüche über die
Edomiter}\label{ausspruxfcche-uxfcber-die-edomiter}}

\bibleverse{7}Über Edom: So hat der HERR der Heerscharen gesprochen:
»Gibt's denn keine Weisheit mehr in Theman? Ist denn den Verständigen
die Klugheit abhanden gekommen und ihnen die Weisheit ausgegangen?
\bibleverse{8}Fliehet, macht euch davon, verkriecht euch in tiefe
Verstecke, ihr Bewohner Dedans! Denn den Untergang lasse ich über Esau
hereinbrechen, die Zeit, wo ich mit ihm abrechne. \bibleverse{9}Wenn
Weingärtner bei dir einbrechen, lassen sie da nicht eine Nachlese übrig?
Wenn Diebe in der Nacht (kommen), rauben sie doch nur so viel, bis sie
genug haben. \bibleverse{10}Doch ich selbst durchsuche Esau und decke
seine Schlupfwinkel auf; und will er sich verstecken, so kann er es
nicht: vernichtet wird seine Nachkommenschaft samt seinen Bruderstämmen
und seinen Nachbarn, so daß nichts mehr von ihm vorhanden ist.
\bibleverse{11}Überlaß mir deine Waisen: ich will sie am Leben erhalten,
und deine Witwen mögen auf mich vertrauen!«

\bibleverse{12}Denn so hat der HERR gesprochen: »Fürwahr, solche, die es
nicht verdienten, den Becher zu trinken, haben ihn trinken müssen, und
du solltest frei ausgehen? Nein, du sollst nicht ungestraft bleiben,
sondern mußt unweigerlich trinken! \bibleverse{13}Denn ich habe bei mir
selbst geschworen« -- so lautet der Ausspruch des HERRN --: »Bozra soll
zum abschreckenden Beispiel, zum Gespött, zur Wüste und zum Fluchwort
werden und alle zugehörigen Ortschaften zu Einöden auf ewig!«~--

\bibleverse{14}Eine Kunde habe ich vom HERRN her vernommen, und eine
Botschaft ist unter die Völker gesandt worden: »Versammelt euch und
zieht gegen Edom heran und macht euch auf zum Kampf!«
\bibleverse{15}Denn wisse wohl: klein mache ich dich unter den Völkern,
verachtet unter den Menschen deine Furchtbarkeit! \bibleverse{16}Betört
hat dich dein vermessener Sinn, weil du in Felsenklüften wohnst und
Bergeshöhen besetzt hältst. »Wenn du auch dein Nest so hoch anlegst wie
der Adler: ich stürze dich doch von dort hinab!« -- so lautet der
Ausspruch des HERRN. \bibleverse{17}»Und Edom soll zum Gegenstand des
Erstarrens werden: jeder, der an ihm vorüberwandert, soll sich entsetzen
und über alle seine Leiden zischen! \bibleverse{18}Wie Sodom und
Gomorrha und ihre Nachbarstädte einst von Grund aus zerstört worden
sind« -- so lautet der Ausspruch des HERRN --, »ebenso soll auch dort
niemand mehr wohnen und kein Menschenkind sich darin aufhalten.
\bibleverse{19}Fürwahr, wie ein Löwe aus dem Dickicht des Jordans zu der
immergrünen Aue hinaufsteigt, so will ich Edom im Nu von dort
vertreiben, und wer dazu ausersehen ist, den werde ich zum Herrn dort
einsetzen. Denn wer ist mir gleich, und wer will mich zur Rechenschaft
ziehen? Und wo wäre ein Völkerhirt✲, der es mit mir aufnehmen
könnte?«~--

\bibleverse{20}Darum vernehmt den Ratschluß, den der HERR
über\textless sup title=``oder: gegen''\textgreater✲ Edom gefaßt hat,
und die Absichten, mit denen er sich gegen die Bewohner von Theman
trägt: Fürwahr, die Hirtenbuben werden sie wegschleppen! Fürwahr, ihre
eigene Trift wird sich über sie entsetzen! \bibleverse{21}Vom Gedröhn
ihres Sturzes erbebt die Erde; ihr Wehgeschrei -- am Schilfmeer wird
sein Schall vernommen! \bibleverse{22}Seht, einem Adler gleich steigt
(der Feind) herauf und fliegt daher und breitet seine Schwingen
über\textless sup title=``oder: gegen''\textgreater✲ Bozra aus; da wird
den edomitischen Kriegern an jenem Tage zumute sein wie einem Weibe in
Kindesnöten\textless sup title=``vgl. 48,40-41''\textgreater✲.

\hypertarget{ausspruch-uxfcber-damaskus}{%
\subsubsection{6. Ausspruch über
Damaskus}\label{ausspruch-uxfcber-damaskus}}

\bibleverse{23}Über Damaskus: Enttäuscht✲ sind Hamath und Arpad, denn
eine schlimme Kunde haben sie vernommen; sie sind verzagt, in
ängstlicher Erregung wie das Meer, das nicht zur Ruhe kommen kann.
\bibleverse{24}Damaskus ist mutlos geworden, hat sich zur Flucht
gewandt, und Zittern hat es ergriffen; Angst und Krämpfe haben es erfaßt
wie ein Weib in Kindesnöten. \bibleverse{25}Wie ist sie doch so ganz
verlassen, die ruhmreiche Stadt, die Burg meiner Wonne!
\bibleverse{26}»Darum werden ihre jungen Männer auf ihren Straßen fallen
und alle kriegstüchtigen Männer an jenem Tage umkommen!« -- so lautet
der Ausspruch des HERRN der Heerscharen --; \bibleverse{27}»und ich
werde Feuer an die Mauern von Damaskus legen, das die Paläste Benhadads
verzehren soll!«

\hypertarget{uxfcber-die-kedarener-und-andere-arabische-stuxe4mme}{%
\subsubsection{7. Über die Kedarener und andere arabische
Stämme}\label{uxfcber-die-kedarener-und-andere-arabische-stuxe4mme}}

\bibleverse{28}Über Kedar und über die Königreiche von Hazor, die
Nebukadnezar, der König von Babylon, besiegte, hat der HERR so
gesprochen: »Auf! Zieht gegen Kedar zu Felde und überwältigt die Söhne
des Ostens! \bibleverse{29}Ihre Zelte und ihre Herden raube man ihnen,
ihre Zeltbehänge und ihren gesamten Hausrat, auch ihre Kamele nehme man
ihnen weg und rufe über sie aus: ›Grauen ringsum!‹
\bibleverse{30}Fliehet, macht euch eilends davon, verkriecht euch in
tiefe Verstecke, ihr Bewohner von Hazor!« -- so lautet der Ausspruch des
HERRN --; »denn Nebukadnezar, der König von Babylon, hat es auf euch
abgesehen und einen Anschlag gegen euch ersonnen. \bibleverse{31}Auf!
Zieht zu Felde gegen das sorglose Volk, das in Sicherheit lebt!« -- so
lautet der Ausspruch des HERRN --, »das weder Tore noch Riegel hat: für
sich allein wohnen sie. \bibleverse{32}Ihre Kamele sollen zur Beute und
ihre vielen Herden zum Raube werden; und ich will sie, die sich das
Haupthaar an der Schläfe stutzen, in alle Winde zerstreuen und von allen
Seiten her Verderben über sie hereinbrechen lassen!« -- so lautet der
Ausspruch des HERRN. \bibleverse{33}»Da wird dann Hazor eine Behausung
für Schakale werden, eine Einöde für ewige Zeiten; niemand wird mehr
dort wohnen und kein Menschenkind sich darin aufhalten!«

\hypertarget{ausspruch-uxfcber-elam}{%
\subsubsection{8. Ausspruch über Elam}\label{ausspruch-uxfcber-elam}}

\bibleverse{34}Das Wort, das über Elam an den Propheten Jeremia im
Anfang der Regierung des judäischen Königs Zedekia erging, lautet
folgendermaßen: \bibleverse{35}So hat der HERR der Heerscharen
gesprochen: »Fürwahr, ich zerbreche den Bogen Elams, den Hauptteil
seiner Kraft, \bibleverse{36}und lasse die vier Winde von den vier Enden
des Himmels über die Elamiter hereinbrechen und zerstreue sie nach allen
diesen Windrichtungen hin, so daß es kein Volk geben soll, zu dem nicht
elamitische Flüchtlinge gelangen werden! \bibleverse{37}Und ich will den
Elamitern bange Angst vor ihren Feinden einflößen und vor denen, die
ihnen ans Leben wollen, und verhänge Unglück über sie, die Glut meines
Zorns!« -- so lautet der Ausspruch des HERRN --, »und ich lasse das
Schwert hinter ihnen herfahren, bis ich sie ausgerottet habe!
\bibleverse{38}Dann will ich meinen Richterstuhl in Elam aufstellen und
den König samt den Fürsten daraus vertilgen!« -- so lautet der Ausspruch
des HERRN.

\bibleverse{39}»Doch am Ende der Tage will ich das Geschick Elams wieder
wenden!« -- so lautet der Ausspruch des HERRN.

\hypertarget{ausspruxfcche-uxfcber-babylon}{%
\subsubsection{9. Aussprüche über
Babylon}\label{ausspruxfcche-uxfcber-babylon}}

\hypertarget{a-der-sturz-babylons-und-seine-bedeutung-fuxfcr-das-leidgepruxfcfte-juxfcdische-volk}{%
\paragraph{a) Der Sturz Babylons und seine Bedeutung für das
leidgeprüfte jüdische
Volk}\label{a-der-sturz-babylons-und-seine-bedeutung-fuxfcr-das-leidgepruxfcfte-juxfcdische-volk}}

\hypertarget{section-49}{%
\section{50}\label{section-49}}

\bibleverse{1}(Dies ist) das Wort, das der HERR über\textless sup
title=``oder: gegen''\textgreater✲ Babylon, über das Land der Chaldäer,
durch den Mund des Propheten Jeremia ausgesprochen hat:
\bibleverse{2}»Verkündigt es unter den Völkern und macht es bekannt und
pflanzt ein Banner\textless sup title=``=~ein Panier oder: eine
Flagge''\textgreater✲ auf! Macht es bekannt und verheimlicht es nicht!
Verkündigt: ›Erobert ist Babylon, zuschanden geworden Bel ! Merodach
steht fassungslos da! Ihre Bilder✲ sind zuschanden geworden, ihre Götzen
stehen fassungslos da!‹ \bibleverse{3}Denn es zieht gegen Babylon von
Norden her ein Volk heran: das wird sein Land zur Wüste machen, so daß
kein Bewohner mehr darin zu finden ist: sowohl Menschen als Vieh sind
entflohen, haben sich davongemacht!~--

\bibleverse{4}In jenen Tagen und zu jener Zeit« -- so lautet der
Ausspruch des HERRN -- »werden die Kinder Israel heimkehren, sie im
Verein mit den Kindern Juda; unter unaufhörlichem Weinen werden sie
daherkommen und den HERRN, ihren Gott, suchen. \bibleverse{5}Zum Zion
erfragen sie den Weg, dorthin sind ihre Blicke gerichtet: ›Kommt und
schließt euch an den HERRN an zu einem ewigen, unvergeßlichen Bunde!‹
\bibleverse{6}Mein Volk war wie eine verlorene Schafherde; ihre Hirten
hatten sie auf Abwege geleitet, auf den Bergen sie in der Irre
umhergeführt; von Berg zu Hügel mußten sie ziehen und hatten ihre
Lagerstätte vergessen. \bibleverse{7}Jeder, der auf sie stieß, fraß sie,
und ihre Widersacher sagten: ›Wir tun kein Unrecht damit!‹ -- zur Strafe
dafür, daß sie sich am HERRN versündigt hatten, der Trift der
Gerechtigkeit\textless sup title=``d.h. dem rechten
Weideplatz''\textgreater✲ und der Hoffnung ihrer Väter.~--

\bibleverse{8}Flieht aus dem Bereiche Babylons und verlaßt das Land der
Chaldäer! Werdet den Widdern an der Spitze der Herde gleich!
\bibleverse{9}Denn wisset wohl: ich will gegen Babylon ein großes
Völkerheer aufbieten und aus dem Nordland heranziehen lassen; die sollen
sich gegen die Stadt aufstellen: von dorther wird sie erobert werden.
Ihre Pfeile sind wie die eines tüchtigen Kriegshelden, der nie mit
leeren Händen heimkehrt. \bibleverse{10}So wird denn das Chaldäerland
ausgeraubt werden: alle, die es plündern, sollen satt
werden\textless sup title=``=~genug bekommen''\textgreater✲!« -- so
lautet der Ausspruch des HERRN.~--

\bibleverse{11}»Ja, freut euch nur, ja, jubelt nur, ihr Räuber meines
Erbbesitzes! Ja, hüpft nur lustig wie Rinder\textless sup title=``oder:
eine junge Kuh''\textgreater✲ beim Dreschen und wiehert gleich den
Hengsten! \bibleverse{12}Dennoch wird eure Mutter ganz
zuschanden\textless sup title=``oder: zu Schmach und
Schande''\textgreater✲ werden und die euch geboren hat, beschämt
dastehen; ja, das letzte unter den Völkern soll jetzt zur Wüste und zu
einer dürren Steppe werden! \bibleverse{13}Infolge des Zorns des HERRN
wird es unbewohnt sein und ganz zur Wüste werden, so daß jeder, der an
Babylon vorüberzieht, sich entsetzen und über alle seine Leiden zischen
soll!«~--

\bibleverse{14}Stellt euch ringsum zum Kampf gegen Babylon auf, ihr
Bogenschützen alle! Schießt nach ihm, spart die Pfeile nicht! Denn am
HERRN hat es sich versündigt. \bibleverse{15}Erhebt ringsum ein
Jubelgeschrei über es: »Es hat sich ergeben! Gefallen sind seine
Festungswerke, niedergerissen seine Mauern!« Weil dies die Rache des
HERRN ist, nun, so vollzieht die Rache an ihm! Verfahret mit ihm, wie es
selbst verfahren ist! \bibleverse{16}Rottet aus Babylon jeden Sämann aus
und jeden, der die Sichel in der Erntezeit ergreift! Vor dem
gewalttätigen✲ Schwert werden sie sich ein jeder zu seinem Volke wenden
und ein jeder in seine Heimat fliehen.

\hypertarget{b-israels-bisheriges-miuxdfgeschick-und-spuxe4teres-heil}{%
\paragraph{b) Israels bisheriges Mißgeschick und späteres
Heil}\label{b-israels-bisheriges-miuxdfgeschick-und-spuxe4teres-heil}}

\bibleverse{17}Israel ist wie ein verscheuchtes Schaf, das Löwen verjagt
haben: zuerst hat der König von Assyrien es angefressen, und nun zuletzt
hat Nebukadnezar, der König von Babylon, ihm die Knochen abgenagt.
\bibleverse{18}Darum hat der HERR der Heerscharen, der Gott Israels, so
gesprochen: »Fürwahr, ich will den König von Babylon und sein Land
strafen, wie ich den König von Assyrien gestraft habe.
\bibleverse{19}Alsdann will ich Israel zu seiner Trift heimkehren
lassen, damit es wieder auf dem Karmel\textless sup title=``vgl. Am
1,2''\textgreater✲ und in Basan weide und auf dem Gebirge Ephraim und in
Gilead seinen Hunger stille. \bibleverse{20}In jenen Tagen und zu jener
Zeit« -- so lautet der Ausspruch des HERRN -- »wird man nach der
Verschuldung Israels suchen, aber sie wird nicht mehr vorhanden sein,
und nach den Sünden Judas, aber sie werden nicht mehr zu finden sein;
denn ich habe denen vergeben, die ich als Rest übrig lasse.«

\hypertarget{c-gegen-das-land-doppeltrotz}{%
\paragraph{c) Gegen das Land
›Doppeltrotz‹}\label{c-gegen-das-land-doppeltrotz}}

\bibleverse{21}»Zieh heran gegen das Land ›Doppeltrotz‹ und gegen die
Bewohner von Pekod\textless sup title=``oder: der Stadt
›Heimsuchung‹''\textgreater✲! Morde und vollziehe den Bann hinter ihnen
her« -- so lautet der Ausspruch des HERRN -- »und führe alles so aus,
wie ich dir geboten habe!« \bibleverse{22}Horch! Krieg ist im Lande und
gewaltiger Einsturz! \bibleverse{23}Wie ist doch zerschlagen und
zertrümmert der Hammer, der die ganze Erde schlug! Wie ist doch Babylon
zum Schreckbild unter den Völkern geworden! \bibleverse{24}»Ich habe dir
Schlingen gelegt, Babylon, und du bist auch gefangen worden, ohne daß du
dich dessen versahst: du bist ertappt und auch gefaßt, denn mit dem
HERRN hast du dich in Kampf eingelassen.« \bibleverse{25}Der HERR hat
seine Rüstkammer aufgetan und die Waffen seines Zornes daraus
hervorgeholt; denn Arbeit gibt es zu tun für Gott, den HERRN der
Heerscharen, im Chaldäerlande. \bibleverse{26}Rückt von allen Seiten
gegen das Land heran, öffnet seine Speicher! Schüttet alles in ihm zu
Haufen auf wie Garben und vollzieht den Bann an ihm, daß kein Rest von
ihm übrig bleibt! \bibleverse{27}Stecht alle seine Rinder nieder, laßt
sie zur Schlachtung niedersinken! Wehe ihnen, denn ihr Tag ist gekommen,
die Stunde ihrer Bestrafung! \bibleverse{28}Horch! Flüchtlinge und
Entronnene aus dem Lande Babylon rufen, um in Zion die Rache des HERRN,
unsers Gottes, zu verkünden, die Rache für seinen Tempel!~--

\bibleverse{29}Bietet Schützen gegen Babylon auf, alle, die den Bogen
spannen! Lagert euch rings um die Stadt, laßt ihr kein Entrinnen zuteil
werden! Zahlt ihr ihre Böstaten nach Gebühr heim, verfahrt mit ihr ganz
so, wie sie selbst verfahren ist! Denn gegen den HERRN, den Heiligen
Israels, hat sie sich vermessen aufgelehnt. \bibleverse{30}»Darum sollen
ihre jungen Männer auf ihren Straßen fallen und alle ihre
kriegstüchtigen Leute umkommen an jenem Tage!« -- so lautet der
Ausspruch des HERRN. \bibleverse{31}»Siehe, ich will an
dich\textless sup title=``d.h. gegen dich vorgehen''\textgreater✲, du
Freche!« -- so lautet der Ausspruch Gottes, des HERRN der Heerscharen
--; »denn dein Tag ist gekommen, die Stunde, da ich dich strafe:
\bibleverse{32}da soll die Freche straucheln und zu Fall kommen, ohne
daß jemand ihr aufhilft; und ich will Feuer an ihre Städte legen, das
soll alles rings um sie her verzehren!«

\hypertarget{d-heilsverkuxfcndigung-fuxfcr-israel}{%
\paragraph{d) Heilsverkündigung für
Israel}\label{d-heilsverkuxfcndigung-fuxfcr-israel}}

\bibleverse{33}So hat der HERR der Heerscharen gesprochen: »Wohl leiden
die Söhne Israels und die Söhne Judas insgesamt Gewalt, und alle, die
sie in Gefangenschaft geschleppt haben, halten sie fest und wollen sie
nicht wieder freigeben; \bibleverse{34}doch ihr Erlöser ist stark, ›HERR
der Heerscharen‹ ist sein Name; er wird ihre Sache mit Nachdruck führen,
damit er der Erde Ruhe schaffe, aber Unruhe den Bewohnern Babylons.«

\hypertarget{e-schwertspruch}{%
\paragraph{e) Schwertspruch}\label{e-schwertspruch}}

\bibleverse{35}»Das Schwert komme über die Chaldäer« -- so lautet der
Ausspruch des HERRN --, »über die Bewohner Babylons, über seine Fürsten
und über seine Gelehrten! \bibleverse{36}Das Schwert über die
Schwätzer\textless sup title=``oder: Wahrsager''\textgreater✲, daß sie
als Narren dastehen! Das Schwert über seine tapferen Krieger, daß sie zu
Feiglingen werden! \bibleverse{37}Das Schwert über seine Rosse und
Kriegswagen und über das ganze Völkergemisch innerhalb seines Bereichs,
daß sie zu Weibern werden! Das Schwert über seine Schätze, daß sie der
Plünderung anheimfallen! \bibleverse{38}Das Schwert über seine Gewässer,
daß sie vertrocknen! Denn es ist ein Land der Götzenbilder, und durch
die Abgötter haben sie den Verstand verloren.«

\hypertarget{f-verschiedene-zusuxe4tze-und-wiederholungen}{%
\paragraph{f) Verschiedene Zusätze und
Wiederholungen}\label{f-verschiedene-zusuxe4tze-und-wiederholungen}}

\bibleverse{39}Darum sollen Wildkatzen im Verein mit Schakalen dort
hausen und Strauße darin wohnen, und niemals soll es wieder besiedelt
werden, sondern unbewohnt bleiben von Geschlecht zu Geschlecht!
\bibleverse{40}»Wie Gott einst Sodom und Gomorrha und ihre Nachbarstädte
von Grund aus zerstört hat« -- so lautet der Ausspruch des HERRN --,
»ebenso soll auch dort niemand mehr wohnen und kein Menschenkind sich
darin aufhalten!«✲~--

\bibleverse{41}Gebt acht! Es kommt ein Volk von Norden her, und eine
gewaltige Völkerschaft und viele Könige setzen sich in Bewegung von den
Enden der Erde her. \bibleverse{42}Bogen und Wurfspieß führen sie,
grausam sind sie und ohne Erbarmen; ihr Lärmen ist wie Meeresbrausen,
und auf Rossen reiten sie: gerüstet wie ein Kriegsmann zum Kampfe gegen
dich, Tochter Babylon! \bibleverse{43}Wenn der König von Babylon die
Kunde von ihnen erhält, sinken ihm die Arme schlaff herab; Angst erfaßt
ihn, Krampf wie ein Weib in Kindesnöten. \bibleverse{44}»Fürwahr, wie
ein Löwe aus dem Dickicht des Jordans zu der immergrünen Aue
hinaufsteigt, so will ich sie im Nu von dort vertreiben, und wer dazu
ausersehen ist, den werde ich zum Herrn dort einsetzen. Denn wer ist mir
gleich, und wer darf mich zur Rechenschaft ziehen? Und wo wäre ein
Völkerhirt✲, der es mit mir aufnehmen könnte?«✲

\bibleverse{45}Darum vernehmt den Ratschluß, den der HERR gegen Babylon
gefaßt hat, und die Absichten, mit denen er sich gegen das Land der
Chaldäer trägt: Fürwahr, die Hirtenbuben werden sie wegschleppen!
fürwahr, ihre eigene Trift wird sich über sie entsetzen!\textless sup
title=``vgl, 49,20''\textgreater✲ \bibleverse{46}Von dem Rufe: »Babylon
ist erobert!« erbebt die Erde, und Geschrei vernimmt man unter den
Völkern.

\hypertarget{g-babylons-macht-und-fall}{%
\paragraph{g) Babylons Macht und Fall}\label{g-babylons-macht-und-fall}}

\hypertarget{aa-das-gericht-uxfcber-babylon-ist-beschlossen}{%
\subparagraph{aa) Das Gericht über Babylon ist
beschlossen}\label{aa-das-gericht-uxfcber-babylon-ist-beschlossen}}

\hypertarget{section-50}{%
\section{51}\label{section-50}}

\bibleverse{1}So hat der HERR gesprochen: »Fürwahr, ich lasse gegen
Babylon und gegen die, welche im ›Herzen\textless sup title=``=~Zentrum,
Mittelpunkt''\textgreater✲ meiner Widersacher‹ wohnen, die Wut eines
Verderbers losbrechen, \bibleverse{2}und ich entsende Worfler nach
Babylon, die sollen es worfeln und sein Land ausplündern!« Wenn die am
Tage des Unglücks (die Stadt) von allen Seiten umzingeln,
\bibleverse{3}dann spanne kein Schütze seinen Bogen mehr, und niemand
erhebe sich zum Widerstand in seinem Panzer! Doch schont ihre jungen
Männer nicht, vollzieht den Bann an ihrem gesamten Heer,
\bibleverse{4}so daß Erschlagene daliegen im Lande der Chaldäer und
Durchbohrte✲ auf ihren Straßen! \bibleverse{5}Denn weder Israel noch
Juda ist als Witwe von seinem Gott, vom HERRN der Heerscharen,
verlassen; dagegen das Land jener ist voll von Verschuldung gegen den
Heiligen Israels. \bibleverse{6}Fliehet aus dem Bereich Babylons hinweg
und rettet ein jeder sein Leben, damit ihr nicht den Tod findet um
seiner Verschuldung willen! Denn die Zeit der Rache ist dies für den
HERRN: was es verübt hat, vergilt er ihm.

\hypertarget{bb-babylon-der-taumelbecher-gottes-das-todesurteil-uxfcber-babylon}{%
\subparagraph{bb) Babylon der Taumelbecher Gottes: das Todesurteil über
Babylon}\label{bb-babylon-der-taumelbecher-gottes-das-todesurteil-uxfcber-babylon}}

\bibleverse{7}Ein goldener Becher war Babylon in der Hand des HERRN, der
die ganze Erde trunken machte; von seinem Wein haben die Völker
getrunken, darum haben die Völker den Verstand verloren\textless sup
title=``oder: sich wie toll gebärdet''\textgreater✲.
\bibleverse{8}Plötzlich ist Babylon gefallen und zerschmettert:
»Wehklagt über die Stadt, holt Balsam für ihre Schmerzen\textless sup
title=``=~schmerzenden Wunden''\textgreater✲: vielleicht ist noch
Heilung möglich!« \bibleverse{9}»Wir haben Babylon heilen wollen, aber
es war nicht zu heilen: überlaßt es sich selbst! Laßt uns abziehen, ein
jeder in sein Land! Denn bis an den Himmel reicht das Strafgericht über
die Stadt und ragt bis zu den Wolken!« \bibleverse{10}»Der HERR hat die
Gerechtigkeit unserer Sache ans Licht gebracht: kommt, laßt uns in Zion
das Walten des HERRN, unsers Gottes, verkündigen!«

\hypertarget{cc-die-stadt-wird-nach-beschluuxdf-gottes-erstuxfcrmt}{%
\subparagraph{cc) Die Stadt wird nach Beschluß Gottes
erstürmt}\label{cc-die-stadt-wird-nach-beschluuxdf-gottes-erstuxfcrmt}}

\bibleverse{11}Schärft die Pfeile, ergreift die Schilde! Der HERR hat
die Wut der Könige von Medien erweckt, denn sein Absehen ist gegen
Babylon gerichtet, es zu vernichten; denn die Rache des HERRN ist da,
die Rache für seinen Tempel. \bibleverse{12}Gegen die Mauern Babylons
pflanzt ein Banner\textless sup title=``=~Flagge oder:
Panier''\textgreater✲ auf! Verschärft die Bewachung, stellt Wachtposten
auf, legt Mannschaften in Hinterhalt! Denn wie der HERR es sich
vorgenommen hat, so führt er es auch aus, was er den Bewohnern Babylons
angedroht hat. \bibleverse{13}O Stadt, die du wohnst✲ an großen Wassern,
reich an Schätzen: gekommen ist dein Ende, das Maß ist voll zum
Abschneiden! \bibleverse{14}Der HERR der Heerscharen hat bei sich selbst
geschworen: »Habe ich dich auch mit Menschen angefüllt wie mit
Heuschrecken, so wird man doch Siegesgeschrei über dich erheben!«

\hypertarget{dd-lobpreis-des-herrn-des-gottes-israels}{%
\subparagraph{dd) Lobpreis des Herrn, des Gottes
Israels}\label{dd-lobpreis-des-herrn-des-gottes-israels}}

\bibleverse{15}Er ist es, der die Erde durch seine Kraft geschaffen, den
Erdkreis durch seine Weisheit fest gegründet und durch seine Einsicht
den Himmel ausgespannt hat. \bibleverse{16}Wenn er beim Schall des
Donners Wasserrauschen am Himmel entstehen läßt und Gewölk vom Ende der
Erde heraufführt, wenn er Blitze beim Regen schafft und den Sturmwind
aus seinen Vorratskammern herausläßt~-- \bibleverse{17}starr steht
alsdann jeder Mensch da, ohne es begreifen zu können, und schämen muß
sich jeder Goldschmied seines Bildwerks; denn Trug ist sein gegossener
Götze, und kein Odem\textless sup title=``oder: Leben''\textgreater✲
wohnt in ihm: \bibleverse{18}nichts als Wahn✲ sind sie, lächerliche
Gebilde; wenn die Zeit des Strafgerichts für sie kommt, ist es zu Ende
mit ihnen. \bibleverse{19}Aber nicht wie diese ist Jakobs Erbteil; nein,
er ist es, der das All gebildet hat, und Israel ist der Stamm seines
Erbbesitzes: HERR der Heerscharen ist sein Name!

\hypertarget{ee-hammerspruch-das-gericht-an-babylon-in-seiner-weltgeschichtlichen-bedeutung}{%
\subparagraph{ee) Hammerspruch; das Gericht an Babylon in seiner
weltgeschichtlichen
Bedeutung}\label{ee-hammerspruch-das-gericht-an-babylon-in-seiner-weltgeschichtlichen-bedeutung}}

\bibleverse{20}»Ein Hammer bist du mir gewesen, eine Kriegswaffe; und
ich habe mit dir Völker zerhämmert und Königreiche mit dir zertrümmert.
\bibleverse{21}Zerhämmert habe ich mit dir Rosse samt ihren Reitern,
zerhämmert mit dir Kriegswagen samt den darauf Fahrenden;
\bibleverse{22}zerhämmert habe ich mit dir Männer und Weiber, zerhämmert
mit dir Greise und Kinder, zerhämmert Jünglinge und Jungfrauen;
\bibleverse{23}zerhämmert habe ich mit dir Hirten samt ihren Herden,
zerhämmert mit dir Ackerleute samt ihren Gespannen, zerhämmert
Landpfleger und Statthalter. \bibleverse{24}Aber jetzt will ich Babylon
und allen Bewohnern des Chaldäerlandes alle ihre Bosheit, die sie an
Zion verübt haben, vor euren Augen vergelten!« -- so lautet der
Ausspruch des HERRN.~-- \bibleverse{25}»Nunmehr will ich an
dich\textless sup title=``d.h. gegen dich vorgehen''\textgreater✲« -- so
lautet der Ausspruch des HERRN --, »du Berg des Verderbens, der du über
die ganze Erde Verderben gebracht hast! Ja, ich will meine Hand gegen
dich ausstrecken und dich von der Felsenhöhe hinabwälzen und dich zu
einem verbrannten\textless sup title=``oder:
ausgebrannten''\textgreater✲ Berge machen, \bibleverse{26}so daß man von
dir weder Ecksteine noch Grundsteine mehr nehmen\textless sup
title=``oder: holen''\textgreater✲ kann; nein, eine öde Wüste sollst du
sein auf ewige Zeit!« -- so lautet der Ausspruch des HERRN.

\hypertarget{h-babylons-schuld-und-strafe}{%
\paragraph{h) Babylons Schuld und
Strafe}\label{h-babylons-schuld-und-strafe}}

\hypertarget{aa-schilderung-der-eroberung-der-stadt}{%
\subparagraph{aa) Schilderung der Eroberung der
Stadt}\label{aa-schilderung-der-eroberung-der-stadt}}

\bibleverse{27}Pflanzt ein Panier✲ auf der Erde auf, stoßt in die
Posaune unter den Völkern, weiht✲ Völker zum Kampf gegen Babylon, bietet
gegen es die Königreiche von Ararat, Minni und Askenas auf, bestellt
einen Heerführer gegen es, laßt Reiterei anrücken so zahlreich wie
borstige Heuschrecken! \bibleverse{28}Weiht Völker zum Kampf gegen es,
die Könige von Medien, ihre Landpfleger und alle ihre Statthalter und
das ganze Gebiet ihrer Herrschaft! \bibleverse{29}Da erbebt und zittert
die Erde; denn die Ratschlüsse des HERRN gehen an Babylon in Erfüllung,
um das Land Babylon zu einer menschenleeren Einöde zu machen.
\bibleverse{30}Babylons Mannen ziehen nicht mehr ins Feld, sitzen
tatenlos in den Burgen✲; ihr Mut ist geschwunden, sie sind zu Weibern
geworden; schon hat man die Wohnungen in der Stadt in Flammen aufgehen
lassen, ihre\textless sup title=``d.h. Babylons''\textgreater✲ Riegel
sind zerbrochen. \bibleverse{31}Ein Läufer läuft dem andern entgegen und
ein Bote dem andern, um dem König von Babylon zu melden, daß seine Stadt
an allen Ecken erobert, \bibleverse{32}daß die Furten besetzt seien und
man die Sümpfe mit Feuer ausgebrannt habe und dem Kriegsvolk der Mut
entsunken sei.

\bibleverse{33}Denn so hat der HERR der Heerscharen, der Gott Israels,
gesprochen: »Die Tochter Babylon gleicht einer Tenne zur Zeit, da man
sie feststampft: nur noch eine kleine Weile, so kommt für sie (für
Babylon) die Zeit der Ernte!«

\hypertarget{bb-die-verschuldung-babylons-an-jerusalem-und-die-rache-gottes}{%
\subparagraph{bb) Die Verschuldung Babylons an Jerusalem und die Rache
Gottes}\label{bb-die-verschuldung-babylons-an-jerusalem-und-die-rache-gottes}}

\bibleverse{34}»Nebukadnezar, der König von Babylon, hat mich gefressen,
hat mich vernichtet, mich hingestellt als ein leeres Gefäß! Er hat mich
verschlungen wie ein Drache und seinen Bauch mit mir gefüllt, hat mich
aus meinem Paradies\textless sup title=``=~meiner wonnigen
Heimat''\textgreater✲ hinausgestoßen. \bibleverse{35}Die an mir verübte
Gewalttätigkeit und meine Zerfleischung komme über Babylon« -- so
spreche die Bewohnerschaft Zions -- »und mein Blut über die Bewohner des
Chaldäerlandes!« -- so spreche Jerusalem. \bibleverse{36}Darum hat der
HERR so gesprochen: »Wisse wohl: ich führe deine Sache und vollziehe die
Rache für dich: ich trockne Babylons Strom aus und lasse seine Brunnen
versiegen! \bibleverse{37}Babylon soll zum Trümmerhaufen werden, zur
Behausung der Schakale, zum abschreckenden Beispiel und Gespött, ohne
Bewohner! \bibleverse{38}Jetzt brüllen sie noch allesamt wie junge
Löwen, knurren wie Löwenkätzchen; \bibleverse{39}aber wenn sie (vor
Gier) glühen, will ich ihnen ein Mahl\textless sup title=``oder:
Gelage''\textgreater✲ herrichten und sie trunken machen, daß sie taumeln
und einschlafen zu ewigem Schlaf, aus dem sie nicht wieder erwachen!« --
so lautet der Ausspruch des HERRN. \bibleverse{40}»Ich lasse sie wie
Lämmer zur Schlachtung niedersinken, wie Widder samt den Böcken!«

\hypertarget{cc-klagelied-uxfcber-den-fall-der-stadt-verbunden-mit-mahnungen-an-israel}{%
\subparagraph{cc) Klagelied über den Fall der Stadt verbunden mit
Mahnungen an
Israel}\label{cc-klagelied-uxfcber-den-fall-der-stadt-verbunden-mit-mahnungen-an-israel}}

\bibleverse{41}Ach, wie ist doch Sesach\textless sup title=``=~Babylon;
vgl. 25,26''\textgreater✲ eingenommen und erobert der Stolz der ganzen
Erde! Ach, wie ist doch Babylon zum Gegenstand des Entsetzens unter den
Völkern geworden! \bibleverse{42}Das Meer ist gegen Babylon
heraufgestiegen, von seinen brausenden Wellen ist es überflutet;
\bibleverse{43}seine Städte sind zur Wüste geworden, zu dürrem Land und
zur Steppe, zu einem Land, in welchem niemand wohnt und das kein
Menschenkind durchwandert. \bibleverse{44}Auch am Bel✲ zu Babylon will
ich das Strafgericht vollziehen und aus seinem Rachen das wieder
herausholen, was er verschlungen hat, und nicht mehr sollen künftig die
Völker zu ihm hinströmen! Auch die Mauer Babylons ist gefallen!

\bibleverse{45}Ziehet aus seinem Bereich hinweg, mein Volk, und rettet
ein jeder sein Leben vor der Zornglut des HERRN! \bibleverse{46}Doch
laßt euer Herz nicht verzagen und geratet nicht in Angst bei den
Gerüchten, die im Lande im Umlauf sind, wenn in dem einen Jahre dieses
Gerücht sich verbreitet und im Jahre darauf jenes Gerücht, und
Gewalttätigkeit im Lande herrscht, und ein Machthaber sich gegen den
andern erhebt!

\bibleverse{47}»Darum wisset wohl: es kommt die Zeit, da werde ich das
Strafgericht an den Götzenbildern Babylons vollziehen! Da wird dann sein
ganzes Land zuschanden werden und alle seine (Bewohner) erschlagen in
seiner Mitte fallen! \bibleverse{48}Dann werden Himmel und Erde samt
allem, was in ihnen ist, über Babylon jubeln, denn von Norden her
brechen ihm die Verwüster ins Land ein« -- so lautet der Ausspruch des
HERRN.

\bibleverse{49}Auch Babylon muß fallen um der erschlagenen Israeliten
willen, wie um Babylons willen Erschlagene auf der ganzen Erde gefallen
sind. \bibleverse{50}Ihr, die ihr dem Schwert entronnen seid, ziehet ab
und steht nicht still! Bleibt des HERRN auch in der Ferne eingedenk und
haltet die Erinnerung an Jerusalem getreulich fest: \bibleverse{51}»Wir
haben uns schämen müssen, denn wir haben Schmähreden zu hören bekommen;
Schamröte hat unser Antlitz bedeckt, denn Fremde sind über die
Heiligtümer im Tempel des HERRN hergefallen!« \bibleverse{52}»Darum
wisset wohl: es kommt die Zeit« -- so lautet der Ausspruch des HERRN --,
»da werde ich das Strafgericht an den Götzen Babylons vollziehen, und
überall in seinem Lande werden dann tödlich Verwundete röcheln!
\bibleverse{53}Wenn Babylon auch bis zum Himmel emporstiege und seine
Festung unersteiglich hoch baute, so werden ihm doch von mir her die
Verwüster kommen!« -- so lautet der Ausspruch des HERRN.

\hypertarget{dd-abschluuxdf-und-ruxfcckblick}{%
\subparagraph{dd) Abschluß und
Rückblick}\label{dd-abschluuxdf-und-ruxfcckblick}}

\bibleverse{54}Horch! Geschrei von Babylon her und ein gewaltiger
Einsturz aus dem Lande der Chaldäer! \bibleverse{55}Denn der HERR
verwüstet Babylon und macht dem lauten Lärmen dort ein Ende: es brausen
ihre\textless sup title=``d.h. der Feinde''\textgreater✲ Wogen wie
gewaltige Fluten, laut erschallen ihre wilden Kriegsrufe.
\bibleverse{56}Denn der Verwüster bricht über die Stadt, über Babylon
herein, und ihre Krieger werden gefangen, ihre Bogen sind zerbrochen;
denn ein Gott der Vergeltung ist der HERR, er zahlt sicher heim.
\bibleverse{57}»Ihre Oberen\textless sup title=``oder:
Fürsten''\textgreater✲ aber und Gelehrten, ihre Landpfleger, Statthalter
und Krieger mache ich trunken, daß sie entschlafen zu ewigem Schlaf und
nicht wieder erwachen« -- so lautet der Ausspruch des Königs, dessen
Name ›HERR der Heerscharen‹ ist.~-- \bibleverse{58}So hat der HERR der
Heerscharen gesprochen: »Babylons Mauern, so breit sie sind, sollen bis
auf den Grund niedergerissen und seine Tore, so hoch sie sind, mit Feuer
verbrannt werden!«, und so\textless sup title=``trifft das Wort zu: Hab
2,13''\textgreater✲: »Völker mühen sich für nichts ab, und
Völkerschaften arbeiten sich für das Feuer ab.«

\hypertarget{i-der-fluch-uxfcber-babylon-wird-von-seraja-im-auftrage-jeremias-in-den-euphrat-versenkt}{%
\paragraph{i) Der Fluch über Babylon wird von Seraja im Auftrage
Jeremias in den Euphrat
versenkt}\label{i-der-fluch-uxfcber-babylon-wird-von-seraja-im-auftrage-jeremias-in-den-euphrat-versenkt}}

\bibleverse{59}(Dies ist) der Auftrag, den der Prophet Jeremia Seraja,
dem Sohne Nerijas, des Sohnes Mahsejas, erteilte, als dieser mit
Zedekia, dem König von Juda, im vierten Jahre von dessen Regierung nach
Babylon reiste; Seraja war damals Quartiermeister\textless sup
title=``oder: Reisemarschall''\textgreater✲. \bibleverse{60}Jeremia
hatte aber alles Unglück, das über Babylon hereinbrechen sollte, nämlich
alle Aussprüche, die hier über Babylon aufgezeichnet stehen, auf eine
einzige Buchrolle\textless sup title=``oder: ein Blatt''\textgreater✲
geschrieben \bibleverse{61}und zu Seraja gesagt: »Wenn du nach Babylon
kommst, so sieh dich nach einem passenden Orte um und verlies alle diese
Worte laut \bibleverse{62}und sage dann: ›HERR, du selbst hast diesem
Orte angedroht, ihn vernichten zu wollen, so daß kein Bewohner mehr in
ihm sein solle, weder Menschen noch Vieh, sondern daß er zu einer Einöde
für ewige Zeiten werden solle.‹ \bibleverse{63}Wenn du dann diese
Buchrolle zu Ende gelesen hast, so binde einen Stein daran und wirf sie
mitten in den Euphrat hinein \bibleverse{64}und rufe aus: ›So soll auch
Babylon versinken und nicht wieder hochkommen infolge des Unglücks, das
ich über es verhänge!‹«

Bis hierher gehen die Aussprüche\textless sup title=``vgl.
1,1''\textgreater✲ Jeremias.

\hypertarget{iii.-anhang-nachrichten-uxfcber-die-zerstuxf6rung-jerusalems-sowie-uxfcber-die-wegfuxfchrung-der-gefangenen-und-uxfcber-jojachins-begnadigung-kap.-52}{%
\subsection{III. Anhang: Nachrichten über die Zerstörung Jerusalems
sowie über die Wegführung der Gefangenen und über Jojachins Begnadigung
(Kap.
52)}\label{iii.-anhang-nachrichten-uxfcber-die-zerstuxf6rung-jerusalems-sowie-uxfcber-die-wegfuxfchrung-der-gefangenen-und-uxfcber-jojachins-begnadigung-kap.-52}}

\hypertarget{zedekias-abfall-jerusalems-belagerung-flucht-und-gefangennahme-des-kuxf6nigs-strafgericht-in-ribla}{%
\subsubsection{1. Zedekias Abfall; Jerusalems Belagerung; Flucht und
Gefangennahme des Königs; Strafgericht in
Ribla}\label{zedekias-abfall-jerusalems-belagerung-flucht-und-gefangennahme-des-kuxf6nigs-strafgericht-in-ribla}}

\hypertarget{section-51}{%
\section{52}\label{section-51}}

\bibleverse{1}Im Alter von einundzwanzig Jahren kam Zedekia auf den
Thron und regierte elf Jahre in Jerusalem; seine Mutter hieß Hamutal und
war die Tochter Jeremias, aus Libna. \bibleverse{2}Er tat, was dem HERRN
mißfiel, ganz wie Jojakim getan hatte. \bibleverse{3}Denn infolge des
Zornes des HERRN kam es mit Jerusalem und Juda dahin, daß der HERR sie
von seinem Angesicht verstieß.

Als Zedekia aber vom König von Babylon abgefallen war, \bibleverse{4}da
-- es war im neunten Jahre seiner Regierung, am zehnten Tage des zehnten
Monats -- kam Nebukadnezar, der König von Babylon, in eigener Person mit
seiner ganzen Heeresmacht gegen Jerusalem herangezogen, und man
eröffnete die Belagerung der Stadt und führte rings um sie
Belagerungswerke auf; \bibleverse{5}und die Stadt blieb dann
eingeschlossen bis ins elfte Regierungsjahr Zedekias. \bibleverse{6}Am
neunten Tage des vierten Monats, als die Hungersnot in der Stadt
übermächtig geworden war und auch die Landbevölkerung kein Brot mehr
hatte, \bibleverse{7}da wurde die Stadtmauer durchbrochen, und alle
Kriegsleute ergriffen die Flucht und verließen die Stadt nachts auf dem
Wege durch das Tor zwischen den beiden Mauern, das am Königsgarten lag,
während die Chaldäer noch rings um die Stadt her lagen, und sie wandten
sich dann der Jordan-Ebene zu. \bibleverse{8}Aber das Heer der Chaldäer
setzte dem Könige nach und holte Zedekia in den Steppen von Jericho ein,
nachdem sein ganzes Heer sich zerstreut und ihn verlassen hatte.
\bibleverse{9}So wurde denn der König gefangengenommen und zum König von
Babylon nach Ribla in der Landschaft Hamath hinaufgeführt; der hielt
dann Gericht über ihn. \bibleverse{10}Der König von Babylon ließ die
Söhne Zedekias vor dessen Augen schlachten\textless sup
title=``=~grausam hinrichten''\textgreater✲, ebenso auch alle
Fürsten\textless sup title=``oder: Oberen''\textgreater✲ von Juda in
Ribla. \bibleverse{11}Zedekia aber ließ er blenden und in zwei Ketten
legen und ihn dann nach Babylon bringen, wo er ihn bis zu seinem
Todestage gefangenhielt.

\hypertarget{eroberung-und-zerstuxf6rung-der-stadt-pluxfcnderung-und-verbrennung-des-tempels-wegfuxfchrung-von-einwohnern-nach-babylon-hinrichtungen-zu-ribla}{%
\subsubsection{2. Eroberung und Zerstörung der Stadt; Plünderung und
Verbrennung des Tempels; Wegführung von Einwohnern nach Babylon;
Hinrichtungen zu
Ribla}\label{eroberung-und-zerstuxf6rung-der-stadt-pluxfcnderung-und-verbrennung-des-tempels-wegfuxfchrung-von-einwohnern-nach-babylon-hinrichtungen-zu-ribla}}

\bibleverse{12}Am zehnten Tage des fünften Monats aber -- das war das
neunzehnte Regierungsjahr Nebukadnezars, des Königs von Babylon -- zog
Nebusaradan, der Befehlshaber der Leibwache, der zur persönlichen
Umgebung des Königs von Babylon gehörte, in Jerusalem ein
\bibleverse{13}und verbrannte den Tempel des HERRN sowie den königlichen
Palast und alle anderen Häuser in Jerusalem: alle größeren Häuser ließ
er in Flammen aufgehen. \bibleverse{14}Sodann mußte das ganze
chaldäische Heer, welches der Befehlshaber der Leibwache bei sich hatte,
alle Mauern rings um Jerusalem niederreißen; \bibleverse{15}hierauf ließ
Nebusaradan, der Befehlshaber der Leibwache, {[}einen Teil des niederen
Volkes und{]} den Rest des Volkes, was an Einwohnern in der Stadt noch
übriggeblieben war, ebenso die Überläufer, die zum König von Babylon
übergegangen waren, {[}sowie den Rest der Werkleute{]} in die
Verbannung\textless sup title=``oder: Gefangenschaft''\textgreater✲ nach
Babylon führen; \bibleverse{16}von der niederen Bevölkerung des Landes
aber ließ Nebusaradan, der Befehlshaber der Leibwache, einen Teil als
Weingärtner und Ackerleute zurück. \bibleverse{17}Aber die ehernen
Säulen, die am Tempel des HERRN standen, sowie die Gestühle und das
große eherne Wasserbecken, die beim Tempel des HERRN waren, zerschlugen
die Chaldäer und nahmen alles Erz davon mit sich nach Babylon.
\bibleverse{18}Auch die Töpfe, Schaufeln, Messer zum Lichtputzen,
Schöpfkellen, Schüsseln, überhaupt alle ehernen Geräte, die man beim
Gottesdienst gebraucht hatte, nahmen sie weg; \bibleverse{19}auch die
Becken, Kohlenpfannen, Sprengschalen, Töpfe, Leuchter, Schüsseln und
Becher, alles, was ganz aus Gold oder ganz aus Silber bestand, nahm der
Befehlshaber der Leibwache weg. \bibleverse{20}Was die beiden Säulen
sowie das eine große Wasserbecken und die zwölf ehernen Rinder, die sich
unter ihm befanden, und die Gestühle betrifft, die der König Salomo
einst für den Tempel des HERRN hatte anfertigen lassen, so war es
unmöglich, das Erz aller dieser Kunstwerke zu wägen. \bibleverse{21}Was
aber die Säulen betrifft, so war die eine Säule achtzehn Ellen hoch, und
ein Faden von zwölf Ellen umspannte sie; ihre Dicke betrug vier Finger,
inwendig war sie hohl. \bibleverse{22}Oben auf ihr befand sich ein
eherner Knauf, dessen Höhe bei der einen Säule fünf Ellen betrug; und
ein Flechtwerk und Granatäpfel waren ringsum an dem Knauf angebracht,
alles von Erz; ebenso war auch die andere Säule beschaffen. Was aber die
Granatäpfel betrifft, \bibleverse{23}so waren sechsundneunzig
Granatäpfel da, die frei in der Luft hingen; die Zahl sämtlicher
Granatäpfel an dem Flechtwerk ringsum betrug hundert.

\bibleverse{24}Weiter ließ der Befehlshaber der Leibwache den
Oberpriester Seraja, den Unterpriester Zephanja und die drei
Schwellenhüter verhaften; \bibleverse{25}ferner nahm er aus der Stadt
den einen Kämmerer fest, der den Oberbefehl über das Kriegsvolk gehabt
hatte, sowie sieben Männer von denen, die zu der ständigen Umgebung des
Königs gehört hatten und die in der Stadt vorgefunden waren, außerdem
den Schreiber des Feldhauptmanns, der das Landvolk zum Heeresdienst
ausgehoben hatte, außerdem sechzig Personen aus der Landbevölkerung, die
noch in der Stadt angetroffen waren. \bibleverse{26}Diese also nahm
Nebusaradan, der Befehlshaber der Leibwache, und brachte sie zum König
von Babylon nach Ribla; \bibleverse{27}der König von Babylon ließ sie
dann zu Ribla in der Landschaft Hamath grausam hinrichten. So wurde Juda
von seinem heimischen Boden in die Gefangenschaft\textless sup
title=``oder: Verbannung''\textgreater✲ weggeführt.

\hypertarget{die-zahlen-der-weggefuxfchrten}{%
\paragraph{Die Zahlen der
Weggeführten}\label{die-zahlen-der-weggefuxfchrten}}

\bibleverse{28}Dies ist die Zahl der Personen, die Nebukadnezar in die
Gefangenschaft hat wegführen lassen: im siebten Jahre seiner Regierung
3023 Judäer; \bibleverse{29}im achtzehnten Regierungsjahr Nebukadnezars
832 Personen aus Jerusalem; \bibleverse{30}im dreiundzwanzigsten
Regierungsjahr Nebukadnezars führte Nebusaradan, der Befehlshaber der
Leibwache, von den Judäern 745 Personen weg: im ganzen waren es 4600
Personen.

\hypertarget{jojachins-begnadigung-nach-siebenunddreiuxdfigjuxe4hriger-gefangenschaft}{%
\subsubsection{3. Jojachins Begnadigung nach siebenunddreißigjähriger
Gefangenschaft}\label{jojachins-begnadigung-nach-siebenunddreiuxdfigjuxe4hriger-gefangenschaft}}

\bibleverse{31}Aber im siebenunddreißigsten Jahre nach der
Wegführung\textless sup title=``oder: der Gefangenschaft''\textgreater✲
Jojachins, des Königs von Juda, am fünfundzwanzigsten Tage des zwölften
Monats, begnadigte Ewil-Merodach, der König von Babylon -- im Jahre
seines Regierungsantritts -- den König Jojachin von Juda und entließ ihn
aus dem Kerker. \bibleverse{32}Er redete freundlich mit ihm und wies ihm
seinen Sitz an über den Sitzen der anderen Könige, die mit ihm in
Babylon waren. \bibleverse{33}Er durfte nun auch seine
Gefangenenkleidung ablegen und speiste regelmäßig an der königlichen
Tafel, solange er noch lebte. \bibleverse{34}Sein Unterhalt wurde ihm
als ständiger Unterhalt, soviel er täglich bedurfte, von seiten des
Königs von Babylon bis zu seinem Todestage gewährt, solange er noch
lebte.
