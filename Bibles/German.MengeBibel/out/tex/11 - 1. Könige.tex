\hypertarget{das-erste-buch-der-kuxf6nige}{%
\section{DAS ERSTE BUCH DER KÖNIGE}\label{das-erste-buch-der-kuxf6nige}}

\hypertarget{i.-die-geschichte-salomos-kap.-1-11}{%
\subsection{I. Die Geschichte Salomos (Kap.
1-11)}\label{i.-die-geschichte-salomos-kap.-1-11}}

\hypertarget{wie-salomo-kuxf6nig-wurde}{%
\subsubsection{1. Wie Salomo König
wurde}\label{wie-salomo-kuxf6nig-wurde}}

\hypertarget{a-davids-altersschwuxe4che-abisag-von-sunem-zur-pflege-des-kuxf6nigs-bestellt}{%
\paragraph{a) Davids Altersschwäche; Abisag von Sunem zur Pflege des
Königs
bestellt}\label{a-davids-altersschwuxe4che-abisag-von-sunem-zur-pflege-des-kuxf6nigs-bestellt}}

\hypertarget{section}{%
\section{1}\label{section}}

\bibleverse{1}Als nun der König David alt (und) hochbetagt geworden war,
hüllte man ihn in Decken ein, aber er konnte trotzdem nicht warm werden.
\bibleverse{2}Da sagten seine Diener zu ihm: »Man muß sich für den
König, unsern Herrn, nach einem jungfräulichen Mädchen umsehen, die ihn
zu bedienen hat und als Pflegerin bei ihm ist; wenn die dann in seinen
Armen ruht, wird der König, unser Herr, gewiß warm werden.«
\bibleverse{3}Da suchte man denn im ganzen Bereiche Israels nach dem
schönsten Mädchen; und man fand Abisag von Sunem\textless sup
title=``1.Sam 28,4''\textgreater✲ und brachte sie zum König.
\bibleverse{4}Sie war ein Mädchen von außerordentlicher Schönheit und
hatte nun den König zu bedienen und zu pflegen; aber der König hatte
keinen ehelichen Umgang mit ihr.

\hypertarget{b-adonias-trachten-nach-der-herrschaft-seine-veranstaltung-eines-opfermahls}{%
\paragraph{b) Adonias Trachten nach der Herrschaft; seine Veranstaltung
eines
Opfermahls}\label{b-adonias-trachten-nach-der-herrschaft-seine-veranstaltung-eines-opfermahls}}

\bibleverse{5}Adonia aber, der Sohn der Haggith, dachte voller
Überhebung: »Ich bin's, der König wird!« Daher schaffte er sich Wagen
und Pferde\textless sup title=``oder: Reiter''\textgreater✲ an und
fünfzig Mann, die als Leibdiener✲ vor ihm herliefen. \bibleverse{6}Sein
Vater hatte ihm darüber nie, solange er lebte, Vorstellungen gemacht,
daß er zu ihm gesagt hätte: »Warum tust du so etwas?« Dazu war er ein
Mann von großer Schönheit und war gleich nach Absalom geboren.
\bibleverse{7}So setzte er sich denn ins Einvernehmen mit Joab, dem Sohn
der Zeruja, und mit dem Priester Abjathar, so daß diese auf der Seite
Adonias standen, \bibleverse{8}während der Priester Zadok und Benaja,
der Sohn Jojadas, und der Prophet Nathan sowie Simei und Rei und die
Helden\textless sup title=``oder: Leibwache, vgl. 2.Sam
10,7''\textgreater✲ Davids es nicht mit Adonia hielten.
\bibleverse{9}Als nun Adonia beim Schlangenstein, der neben der
Walkerquelle\textless sup title=``2.Sam 17,17''\textgreater✲ liegt,
Kleinvieh, Rinder und Mastkälber zu einem großen Opfermahl schlachtete,
lud er alle seine Brüder, die Königssöhne (königlichen Prinzen), dazu
ein, auch alle Männer von Juda, soweit sie im Dienste des Königs
standen; \bibleverse{10}aber den Propheten Nathan und Benaja sowie die
Helden✲ und seinen Bruder Salomo ließ er uneingeladen.

\hypertarget{c-nathans-verabredung-mit-bathseba}{%
\paragraph{c) Nathans Verabredung mit
Bathseba}\label{c-nathans-verabredung-mit-bathseba}}

\bibleverse{11}Da sagte Nathan zu Bathseba, der Mutter Salomos: »Hast du
nicht gehört, daß Adonia, der Sohn der Haggith, sich zum König gemacht
hat, ohne daß David, unser Herr, etwas davon weiß? \bibleverse{12}Nun
denn, komm, ich will dir einen Rat geben, wie du dir und deinem Sohne
Salomo das Leben retten kannst! \bibleverse{13}Gehe sofort zum König
David hinein und sage zu ihm: ›Du selbst hast ja doch, mein Herr und
König, deiner Magd eidlich versprochen, daß mein\textless sup
title=``oder: dein''\textgreater✲ Sohn Salomo König nach dir werden und
er der Erbe deines Thrones sein solle. Wie kommt es denn, daß jetzt
Adonia König geworden ist?‹ \bibleverse{14}Während du dann dort noch mit
dem Könige redest, werde ich selbst nach dir hineinkommen und deine
Aussagen bestätigen\textless sup title=``oder:
vervollständigen''\textgreater✲.«

\hypertarget{d-bathseba-erinnert-den-kuxf6nig-an-seine-zusage-nathans-eingreifen}{%
\paragraph{d) Bathseba erinnert den König an seine Zusage; Nathans
Eingreifen}\label{d-bathseba-erinnert-den-kuxf6nig-an-seine-zusage-nathans-eingreifen}}

\bibleverse{15}So ging denn Bathseba zum König in das Schlafgemach
hinein; der König aber war schon sehr alt, und Abisag von Sunem war
seine Pflegerin. \bibleverse{16}Als Bathseba sich nun vor dem Könige
verneigt und niedergeworfen hatte, fragte der König sie: »Was wünschest
du?« \bibleverse{17}Sie antwortete ihm: »Mein Herr, du selbst hast
deiner Magd bei dem HERRN, deinem Gott, zugeschworen, daß mein Sohn
Salomo König nach dir werden und er der Erbe deines Thrones sein solle.
\bibleverse{18}Aber nun hat sich ja doch Adonia zum König gemacht, ohne
daß du, mein Herr und König, etwas davon weißt. \bibleverse{19}Er hat
nämlich Rinder, Mastkälber und Kleinvieh in Menge (zu einem großen
Opfermahl) schlachten lassen und alle königlichen Prinzen, auch den
Priester Abjathar, sowie den obersten Heerführer Joab dazu eingeladen;
aber deinen Knecht Salomo hat er uneingeladen gelassen!
\bibleverse{20}Nun aber, mein Herr und König, sind die Augen aller
Israeliten auf dich gerichtet, daß du ihnen kundgebest, wer auf dem
Throne meines Herrn, des Königs, als sein Nachfolger sitzen soll.
\bibleverse{21}Sonst wird es dahin kommen, sobald mein Herr und König
sich zu seinen Vätern (schlafen) gelegt hat, daß ich und mein Sohn
Salomo als Schuldige\textless sup title=``oder:
Verbrecher''\textgreater✲ dastehen.«

\bibleverse{22}Während sie noch mit dem Könige redete, erschien der
Prophet Nathan; \bibleverse{23}man meldete also dem Könige: »Der Prophet
Nathan ist da!« Und er trat vor den König, und als er sich vor ihm mit
dem Angesicht zur Erde verneigt und sich niedergeworfen hatte,
\bibleverse{24}sagte Nathan: »Mein Herr und König! Du hast wohl selbst
angeordnet, daß Adonia König nach dir sein und auf deinem Throne sitzen
soll? \bibleverse{25}Denn er ist heute hinabgegangen und hat Rinder,
Mastkälber und Kleinvieh in Menge schlachten lassen und hat alle
königlichen Prinzen, außerdem die obersten Heerführer und den Priester
Abjathar eingeladen; und nun essen und trinken sie vor ihm und rufen:
›Es lebe der König Adonia!‹ \bibleverse{26}Aber mich, deinen Knecht, und
den Priester Zadok sowie Benaja, den Sohn Jojadas, und deinen Knecht
Salomo hat er nicht dazu eingeladen! \bibleverse{27}Wenn dies alles mit
Wissen und Willen meines Herrn, des Königs, geschehen ist, so hättest du
also deine Diener nicht wissen lassen, wer auf dem Throne meines Herrn,
des Königs, als sein Nachfolger sitzen soll?«

\hypertarget{e-david-bestuxe4tigt-seine-fruxfcher-gegebene-zusage-durch-einen-schwur-setzt-salomo-zu-seinem-nachfolger-ein-und-bestimmt-dessen-sofortige-salbung}{%
\paragraph{e) David bestätigt seine früher gegebene Zusage durch einen
Schwur, setzt Salomo zu seinem Nachfolger ein und bestimmt dessen
sofortige
Salbung}\label{e-david-bestuxe4tigt-seine-fruxfcher-gegebene-zusage-durch-einen-schwur-setzt-salomo-zu-seinem-nachfolger-ein-und-bestimmt-dessen-sofortige-salbung}}

\bibleverse{28}Da antwortete der König David: »Ruft mir Bathseba
(wieder)!« Als sie nun erschienen und vor den König getreten war,
\bibleverse{29}schwur der König mit den Worten: »So wahr der HERR lebt,
der mich aus aller Not errettet hat! \bibleverse{30}Wie ich dir beim
HERRN, dem Gott Israels, geschworen und gelobt habe, daß nämlich dein
Sohn Salomo König nach mir sein und als mein Nachfolger auf meinem
Throne sitzen soll, so will ich es am heutigen Tage wahr machen!«
\bibleverse{31}Da verneigte sich Bathseba mit dem Angesicht bis zum
Boden, warf sich vor dem Könige nieder und rief aus: »Möge mein Herr,
der König David, ewiglich leben!«

\bibleverse{32}Hierauf befahl der König David: »Ruft mir den Priester
Zadok, den Propheten Nathan und Benaja, den Sohn Jojadas!« Als sie vor
dem Könige erschienen waren, befahl dieser ihnen: \bibleverse{33}»Nehmt
die Knechte✲ eures Herrn mit euch und laßt meinen Sohn Salomo mein
eigenes Maultier besteigen und geleitet ihn hinab an den Gihon.
\bibleverse{34}Dort sollen der Priester Zadok und der Prophet Nathan ihn
zum König über Israel salben; ihr aber laßt in die Posaune stoßen und
ruft: ›Es lebe der König Salomo!‹ \bibleverse{35}Alsdann zieht hinter
ihm her wieder (zur Burg) herauf; und wenn er dort angekommen ist, soll
er sich auf meinen Thron setzen, und dann soll er König sein an meiner
Statt; denn ihn habe ich zum Fürsten über Israel und Juda bestellt!«
\bibleverse{36}Da antwortete Benaja, der Sohn Jojadas, dem König mit den
Worten: »So sei es! Möge der HERR, der Gott meines Herrn, des Königs,
sein Amen dazu sprechen! \bibleverse{37}Wie Gott der HERR mit meinem
Herrn, dem König, gewesen ist, so möge er auch mit Salomo sein und möge
seinen Thron noch mehr erhöhen als den Thron meines Herrn, des Königs
David!«

\hypertarget{f-salomos-feierliche-salbung-wirkung-der-betreffenden-nachricht-auf-die-beim-opfermahl-versammelten}{%
\paragraph{f) Salomos feierliche Salbung; Wirkung der betreffenden
Nachricht auf die beim Opfermahl
Versammelten}\label{f-salomos-feierliche-salbung-wirkung-der-betreffenden-nachricht-auf-die-beim-opfermahl-versammelten}}

\bibleverse{38}So gingen denn der Priester Zadok, der Prophet Nathan und
Benaja, der Sohn Jojadas, samt der Leibwache der Krethi und
Plethi\textless sup title=``2.Sam 8,18''\textgreater✲ hinab, ließen
Salomo das Maultier des Königs David besteigen und geleiteten ihn an den
Gihon. \bibleverse{39}Der Priester Zadok hatte aber das mit Öl gefüllte
Horn aus dem (heiligen) Zelt\textless sup title=``2.Sam
6,17''\textgreater✲ mitgenommen und salbte nun Salomo; dann ließen sie
in die Posaune stoßen, und alles Volk rief: »Es lebe der König Salomo!«
\bibleverse{40}Hierauf geleitete ihn alles Volk (nach der Burg) hinauf,
indem die Leute alle dabei auf Flöten bliesen und so laut und freudig
jubelten, daß die Erde vor ihrem Geschrei schier bersten wollte.

\bibleverse{41}Das hörten Adonia und alle seine Gäste, als sie eben mit
dem Mahl zu Ende waren. Als nun Joab den Posaunenschall vernahm, fragte
er: »Was bedeutet das Geschrei und der Lärm in der Stadt?«
\bibleverse{42}Während er noch redete, kam Jonathan, der Sohn des
Priesters Abjathar; und Adonia rief ihm zu: »Komm herein✲, du bist ein
zuverlässiger Mann und bringst gewiß gute Botschaft!«
\bibleverse{43}Aber Jonathan antwortete dem Adonia: »Im Gegenteil! Unser
Herr, der König David, hat Salomo zum König gemacht! \bibleverse{44}Der
König hat nämlich den Priester Zadok, den Propheten Nathan und Benaja,
den Sohn Jojadas, samt der Leibwache der Krethi und Plethi✲ mit ihm
entsandt; diese haben ihn das Maultier des Königs besteigen lassen,
\bibleverse{45}und der Priester Zadok und der Prophet Nathan haben ihn
am Gihon zum König gesalbt und sind von dort jubelnd (auf die Burg)
hinaufgezogen, und die ganze Stadt ist dadurch in Aufregung geraten;
daher rührt das Geschrei, das ihr gehört habt. \bibleverse{46}Salomo hat
sich dann auch auf den Königsthron gesetzt; \bibleverse{47}außerdem sind
auch die Diener\textless sup title=``=~höchsten
Staatsbeamten''\textgreater✲ des Königs bereits hineingegangen, um
unserm Herrn, dem König David, Glück zu wünschen mit den Worten: ›Dein
Gott möge den Namen Salomos noch ruhmvoller machen als den deinigen und
seinen Thron noch mehr erhöhen als den deinigen!‹ Dabei hat der König
sich auf seinem Lager verneigt \bibleverse{48}und auch noch die Worte
hinzugefügt: ›Gepriesen sei der HERR, der Gott Israels, der es heute so
gefügt hat, daß ich einen Nachfolger (aus meinem Geschlecht) auf meinem
Throne mit eigenen Augen sitzen sehe!‹« \bibleverse{49}Da erschraken
alle, die von Adonia eingeladen worden waren; sie brachen auf und gingen
ein jeder seines Weges.

\hypertarget{g-adonias-begnadigung}{%
\paragraph{g) Adonias Begnadigung}\label{g-adonias-begnadigung}}

\bibleverse{50}Adonia selbst aber, der sich vor Salomo fürchtete, eilte
sofort hin und umfaßte die Hörner des Altars. \bibleverse{51}Als man
dies dem Salomo meldete mit den Worten: »Adonia hat jetzt aus Furcht vor
dem König Salomo die Hörner des Altars umfaßt und erklärt, der König
Salomo möge ihm erst schwören, daß er seinen Knecht nicht hinrichten
lassen wolle«, \bibleverse{52}sagte Salomo: »Wenn er sich als ein
ehrenhafter Mann erweist, so soll ihm kein Haar gekrümmt werden; läßt er
sich aber Böses zuschulden kommen, dann ist er ein Kind des Todes!«
\bibleverse{53}Darauf ließ ihn der König Salomo vom Altar wegholen; und
als er kam und sich vor dem König Salomo niederwarf, sagte dieser zu
ihm: »Gehe in dein Haus!«

\hypertarget{davids-weisungen-an-salomo-und-sein-tod}{%
\subsubsection{2. Davids Weisungen an Salomo und sein
Tod}\label{davids-weisungen-an-salomo-und-sein-tod}}

\hypertarget{section-1}{%
\section{2}\label{section-1}}

\bibleverse{1}Als es nun mit Davids Leben zu Ende ging, gab er seinem
Sohne Salomo folgende Weisungen: \bibleverse{2}»Ich gehe jetzt den Weg
alles Irdischen; so sei denn stark und zeige dich als Mann!
\bibleverse{3}Beobachte alles, was der HERR, dein Gott, von dir fordert,
indem du auf seinen Wegen wandelst, seine Satzungen und Gebote, seine
Rechte und Verordnungen so hältst, wie im Gesetz Moses geschrieben
steht; dann wirst du Glück haben in allem, was du unternimmst, und
überall, wohin du dich wendest; \bibleverse{4}dann wird auch der HERR
seine Verheißung in Erfüllung gehen lassen, die er mir gegeben hat mit
den Worten\textless sup title=``2.Sam 7,12-15''\textgreater✲: ›Wenn
deine Söhne✲ auf ihren Weg achthaben, so daß sie in Treue vor mir
wandeln mit ihrem ganzen Herzen und mit ihrer ganzen Seele, so soll es
dir nie an einem Manne\textless sup title=``d.h. Nachkommen und
Nachfolger''\textgreater✲ auf dem Thron Israels fehlen.‹
\bibleverse{5}Nun weißt du selbst ja auch, wie Joab, der Sohn der
Zeruja, gegen mich gehandelt hat, wie er sich nämlich gegen die beiden
obersten Heerführer Israels, gegen Abner, den Sohn Ners, und gegen
Amasa, den Sohn Jethers, benommen hat, indem er sie ermordete und für
das im Kriege vergossene Blut mitten im Frieden Rache nahm und
unschuldiges Blut an den Gürtel um seine Hüften und an\textless sup
title=``oder: in''\textgreater✲ die Schuhe an seinen Füßen gebracht hat.
\bibleverse{6}So handle nun nach deiner Weisheit und laß sein graues
Haar nicht in Frieden✲ in das Totenreich hinabfahren!
\bibleverse{7}Dagegen sollst du den Söhnen\textless sup title=``oder:
Kindern''\textgreater✲ des Gileaditers Barsillai Liebe erweisen: sie
sollen zu denen gehören, die an deinem Tisch speisen; das haben sie
durch ihr Entgegenkommen verdient, als ich vor deinem Bruder Absalom
fliehen mußte. \bibleverse{8}Ferner befindet sich in deiner Nähe der
Benjaminit Simei, der Sohn Geras, aus Bahurim; der hat die frechsten
Flüche gegen mich ausgestoßen an dem Tage, als ich mich nach Mahanaim
begab. Aber er ist mir dann an den Jordan entgegengekommen, und ich habe
ihm mit einem Schwur beim HERRN zugesagt: ›Ich will dich nicht
hinrichten lassen.‹ \bibleverse{9}Nunmehr aber laß du ihn nicht
ungestraft! Du bist ja ein kluger Mann und wirst schon wissen, wie du
mit ihm zu verfahren hast, um sein graues Haar blutbefleckt in das
Totenreich hinabfahren zu lassen.«

\hypertarget{davids-tod-salomos-regierungsantritt}{%
\paragraph{Davids Tod; Salomos
Regierungsantritt}\label{davids-tod-salomos-regierungsantritt}}

\bibleverse{10}Hierauf legte sich David zu seinen Vätern und wurde (zu
Jerusalem) in der Davidsstadt begraben. \bibleverse{11}Die Zeit aber,
die David über Israel regiert hat, betrug vierzig Jahre: sieben Jahre
hat er in Hebron und dreiunddreißig Jahre in Jerusalem regiert.
\bibleverse{12}Nunmehr saß Salomo auf dem Thron seines Vaters David, und
sein Königtum erstarkte zu außerordentlicher Macht.

\hypertarget{salomos-gewaltsame-mauxdfregeln-beim-regierungsantritt}{%
\subsubsection{3. Salomos gewaltsame Maßregeln beim
Regierungsantritt}\label{salomos-gewaltsame-mauxdfregeln-beim-regierungsantritt}}

\hypertarget{a-adonia-wegen-seines-unbesonnenen-verlangens-getuxf6tet}{%
\paragraph{a) Adonia wegen seines unbesonnenen Verlangens
getötet}\label{a-adonia-wegen-seines-unbesonnenen-verlangens-getuxf6tet}}

\bibleverse{13}Adonia aber, der Sohn der Haggith, begab sich zu
Bathseba, der Mutter Salomos. Als diese ihn fragte: »Bedeutet dein
Kommen Gutes?«, antwortete er: »Ja, etwas Gutes.« \bibleverse{14}Dann
fuhr er fort: »Ich habe eine Bitte an dich.« Sie erwiderte: »Rede!«
\bibleverse{15}Da sagte er: »Du weißt selbst, daß das Königtum
(eigentlich) mir zukam und daß ganz Israel in mir den zukünftigen König
sah; aber die Sache kam dann anders, und das Königtum ist meinem Bruder
zugefallen, weil es vom HERRN für ihn bestimmt war. \bibleverse{16}Nun
aber habe ich eine einzige Bitte an dich; schlage sie mir nicht ab!« Sie
entgegnete ihm: »Rede!« \bibleverse{17}Da fuhr er fort: »Bitte doch den
König Salomo -- denn dich wird er sicherlich nicht abweisen --, er möge
mir Abisag von Sunem zur Frau geben.« \bibleverse{18}Bathseba erwiderte
darauf: »Gut! Ich will deinethalben mit dem König reden.«

\bibleverse{19}Als Bathseba sich nun zum König Salomo begab, um wegen
Adonias mit ihm zu reden, erhob sich der König, ging ihr entgegen,
verneigte sich vor ihr und setzte sich dann wieder auf seinen Stuhl;
dann ließ er auch für die Königin-Mutter einen Stuhl hinstellen, und sie
setzte sich zu seiner Rechten. \bibleverse{20}Darauf sagte sie: »Ich
habe eine kleine Bitte an dich: schlage sie mir nicht ab!« Der König
erwiderte ihr: »Bitte nur, liebe Mutter! Denn ich werde dich nicht
abweisen.« \bibleverse{21}Da sagte sie: »Möchte doch Abisag von Sunem
deinem Bruder Adonia zur Frau gegeben werden!« \bibleverse{22}Da gab der
König Salomo seiner Mutter folgende Antwort: »Warum bittest du für
Adonia nur um Abisag von Sunem? Bitte für ihn doch lieber gleich um das
Königtum! Er ist ja mein älterer Bruder, und auf seiner Seite stehen der
Priester Abjathar und Joab, der Sohn der Zeruja!« \bibleverse{23}Hierauf
schwur der König Salomo beim HERRN: »Gott strafe mich jetzt und künftig,
wenn diese Bitte den Adonia nicht das Leben kostet! \bibleverse{24}So
wahr der HERR lebt, der mich (als König) eingesetzt und mich den Thron
meines Vaters David hat besteigen lassen und der mir nach seiner
Verheißung ein Haus gebaut hat: heute noch muß Adonia sterben!«
\bibleverse{25}Hierauf beauftragte der König Salomo Benaja, den Sohn
Jojadas; der stieß ihn nieder; so starb er.

\hypertarget{b-der-priester-abjathar-abgesetzt-und-verbannt}{%
\paragraph{b) Der Priester Abjathar abgesetzt und
verbannt}\label{b-der-priester-abjathar-abgesetzt-und-verbannt}}

\bibleverse{26}Dem Priester Abjathar aber befahl der König: »Begib dich
nach Anathoth auf dein Landgut! Denn du hast eigentlich den Tod
verdient; aber ich will dich heute✲ nicht töten lassen, weil du die Lade
Gottes des HERRN vor meinem Vater David getragen und weil du an allen
Leiden, die meinen Vater betroffen haben, teilgenommen hast.«
\bibleverse{27}So verstieß also Salomo den Abjathar, so daß er nicht
mehr Priester des HERRN war; und damit ging die Drohung in Erfüllung,
die der HERR einst gegen das Haus Elis in Silo ausgesprochen
hatte\textless sup title=``vgl. 1.Sam 2,31-36''\textgreater✲.

\hypertarget{c-joab-hingerichtet}{%
\paragraph{c) Joab hingerichtet}\label{c-joab-hingerichtet}}

\bibleverse{28}Als nun die Kunde von diesem allem zu Joab drang -- Joab
hatte nämlich zu Adonia gehalten, während er sich an Absalom nicht
angeschlossen hatte --, da floh Joab in das Zelt des HERRN und umfaßte
die Hörner des Altars. \bibleverse{29}Als nun dem König Salomo gemeldet
wurde, daß Joab in das Zelt des HERRN geflohen sei und dort neben dem
Altar stehe, da sandte Salomo Benaja, den Sohn Jojadas, hin mit dem
Befehl: »Gehe hin, stoße ihn nieder!« \bibleverse{30}Benaja begab sich
also in das Zelt des HERRN und sagte zu Joab: »So lautet des Königs
Befehl: ›Komm heraus!‹« Aber er antwortete: »Nein, hier will ich
sterben!« Da meldete Benaja dies dem König mit den Worten: »So hat Joab
gesprochen, und so hat er mir geantwortet.« \bibleverse{31}Da befahl ihm
der König: »Tu ihm, wie er gesagt hat: stoße ihn nieder und begrabe ihn
und schaffe so das Blut, das Joab ohne Grund vergossen hat, von mir und
von meines Vaters Hause weg! \bibleverse{32}Der HERR lasse ihm seine
Blutschuld auf sein eigenes Haupt zurückfallen, weil er zwei Männer, die
ehrenhafter und besser waren als er, niedergestoßen und mit dem Schwert
ermordet hat, ohne daß mein Vater David etwas davon wußte, nämlich
Abner, den Sohn Ners, den Heerführer der Israeliten, und Amasa, den Sohn
Jethers, den Heerführer der Judäer: \bibleverse{33}deren Blut möge auf
das Haupt Joabs und auf das Haupt seiner Nachkommen für alle Zukunft
zurückfallen! David aber und seinen Nachkommen, seinem Hause und seinem
Thron möge für alle Zeiten Heil vom HERRN zuteil werden!«
\bibleverse{34}Da ging Benaja, der Sohn Jojadas, hinauf und versetzte
ihm den Todesstoß; er wurde alsdann bei seinem Hause in der Steppe
(Juda) begraben. \bibleverse{35}Der König machte hierauf Benaja, den
Sohn Jojadas, an Joabs Statt zum obersten Heerführer und übertrug dem
Priester Zadok die Stelle Abjathars.

\hypertarget{d-simei-zuerst-in-jerusalem-festgehalten-dann-getuxf6tet}{%
\paragraph{d) Simei zuerst in Jerusalem festgehalten, dann
getötet}\label{d-simei-zuerst-in-jerusalem-festgehalten-dann-getuxf6tet}}

\bibleverse{36}Weiter ließ der König den Simei rufen und sagte zu ihm:
»Baue dir ein Haus in Jerusalem und bleibe dort wohnen! Du darfst von
dort nicht weggehen, wohin es auch sei! \bibleverse{37}Sobald du dich
nämlich von da entfernst und auch nur über den Bach Kidron gehst, so
kannst du dich darauf verlassen, daß du sterben mußt: dein Blut kommt
dann über dein eigenes Haupt!« \bibleverse{38}Simei erwiderte dem König:
»Gut so! Wie mein Herr, der König, befohlen hat, so wird dein Knecht
tun.« So wohnte denn Simei lange Zeit in Jerusalem. \bibleverse{39}Aber
nach Ablauf von drei Jahren begab es sich, daß dem Simei zwei Sklaven zu
Achis, dem König von Gath, dem Sohne Maachas, entliefen. Als man nun dem
Simei mitteilte, daß seine Sklaven sich in Gath befänden,
\bibleverse{40}machte Simei sich auf, ließ seinen Esel satteln und begab
sich nach Gath zu Achis, um seine Sklaven ausfindig zu machen; er zog
also hin und holte seine Sklaven aus Gath zurück. \bibleverse{41}Sobald
nun Salomo erfuhr, daß Simei sich von Jerusalem nach Gath begeben habe
und von dort wieder zurückgekommen sei, \bibleverse{42}ließ der König
den Simei vor sich kommen und sagte zu ihm: »Habe ich dich nicht beim
HERRN schwören lassen und dir die bestimmte Versicherung gegeben:
›Sobald du dich von hier entfernst, wohin es auch sei, so kannst du dich
darauf verlassen, daß du sterben mußt?‹ Damals erklärtest du mir: ›Gut
so! Ich habe es gehört.‹ \bibleverse{43}Warum hast du dich nun an den
beim HERRN geschworenen Eid und an den von mir erteilten Befehl nicht
gehalten?« \bibleverse{44}Der König fügte dann noch hinzu: »Du weißt
selbst, wie schwer du dich gegen meinen Vater David vergangen hast -- du
erinnerst dich dessen noch recht gut --; so läßt denn der HERR jetzt
deine Bosheit auf dein Haupt zurückfallen. \bibleverse{45}Der König
Salomo dagegen wird gesegnet sein und der Thron Davids immerdar
feststehen vor dem HERRN!« \bibleverse{46}Darauf erteilte der König dem
Benaja, dem Sohn Jojadas, Befehl; der ging hinaus und stieß ihn nieder:
so starb er.

\hypertarget{salomos-verheiratung-mit-einer-uxe4gyptischen-prinzessin-sein-antrittsopfer-und-sein-traum-in-gibeon}{%
\subsubsection{4. Salomos Verheiratung mit einer ägyptischen Prinzessin;
sein Antrittsopfer und sein Traum in
Gibeon}\label{salomos-verheiratung-mit-einer-uxe4gyptischen-prinzessin-sein-antrittsopfer-und-sein-traum-in-gibeon}}

Als nun das Königtum sich in Salomos Hand befestigt hatte,

\hypertarget{section-2}{%
\section{3}\label{section-2}}

\bibleverse{1}verschwägerte sich Salomo mit dem Pharao, dem König von
Ägypten; er heiratete nämlich eine Tochter des Pharaos und führte sie in
die Davidsstadt, bis er mit dem Bau seines Palastes sowie mit dem Bau
des Tempels des HERRN und der Ringmauer um Jerusalem fertig war.
\bibleverse{2}Das Volk mußte damals leider noch auf den Höhen opfern,
weil bis zu dieser Zeit dem Namen des HERRN noch kein eigenes Haus
erbaut worden war. \bibleverse{3}Salomo hatte zwar den HERRN lieb, so
daß er nach den Weisungen seines Vaters David wandelte, doch opferte und
räucherte auch er noch auf den Höhen.

\hypertarget{salomos-opfer-und-die-gotteserscheinung-in-gibeon}{%
\paragraph{Salomos Opfer und die Gotteserscheinung in
Gibeon}\label{salomos-opfer-und-die-gotteserscheinung-in-gibeon}}

\bibleverse{4}So begab sich denn der König nach Gibeon, um dort zu
opfern; denn dort befand sich das vornehmste Höhenheiligtum; tausend
Brandopfer(-tiere) brachte Salomo auf dem dortigen Altar dar.
\bibleverse{5}Da erschien der HERR dem Salomo zu Gibeon nachts im Traum,
und Gott sagte: »Bitte, was ich dir geben soll!« \bibleverse{6}Salomo
antwortete: »Du hast deinem Knecht, meinem Vater David, große Huld
erwiesen, weil er vor dir in Wahrheit\textless sup title=``oder:
Treue''\textgreater✲ und Gerechtigkeit und mit aufrichtigem Herzen gegen
dich gewandelt ist, und du hast ihm auch diese große Huld bewahrt und
ihm einen Sohn gegeben, der auf seinem Throne sitzt, wie dieser Tag
beweist. \bibleverse{7}Nun denn, HERR, mein Gott, weil du selbst deinen
Knecht an meines Vaters David Statt zum König gemacht hast, ich aber
noch ein junger Mann bin, der weder aus noch ein weiß, \bibleverse{8}und
weil dein Knecht in der Mitte\textless sup title=``=~im
Mittelpunkt''\textgreater✲ deines Volkes steht, das du erwählt hast,
eines so großen Volkes, daß man es vor Menge nicht zählen noch berechnen
kann: \bibleverse{9}so wollest du deinem Knecht ein verständiges Herz
geben, damit er dein Volk zu regieren versteht und zwischen gut und böse
zu unterscheiden weiß; denn wer wäre sonst imstande, dieses dein so
zahlreiches Volk zu regieren?« \bibleverse{10}Diese Rede gefiel dem
HERRN wohl, daß Salomo nämlich eine solche Bitte ausgesprochen hatte;
\bibleverse{11}darum antwortete Gott ihm: »Weil du diese Bitte
ausgesprochen und dir nicht ein langes Leben oder Reichtum gewünscht
oder auch um den Tod deiner Feinde gebeten, sondern dir Einsicht erbeten
hast, um Verständnis für das Recht zu besitzen, \bibleverse{12}so will
ich deine Bitte erfüllen: Siehe, ich will dir ein weises und
einsichtsvolles Herz geben, so daß deinesgleichen nicht vor dir gewesen
ist und deinesgleichen nach dir nicht erstehen wird. \bibleverse{13}Aber
auch das will ich dir verleihen, was du dir nicht erbeten hast, sowohl
Reichtum als auch Ehre✲, so daß kein anderer König dir gleich sein soll,
solange du lebst. \bibleverse{14}Und wenn du auf meinen Wegen wandelst,
indem du meine Satzungen und Gebote beobachtest, wie dein Vater David
gewandelt ist, so will ich dir auch ein langes Leben verleihen.«
\bibleverse{15}Als nun Salomo erwachte, erkannte er, daß es ein
bedeutungsvoller Traum gewesen war. Als er dann nach Jerusalem
zurückgekehrt war, trat er vor die Bundeslade des HERRN, brachte
Brandopfer dar und richtete Heilsopfer her und veranstaltete ein
Festmahl für alle seine Diener✲.

\hypertarget{salomos-weiser-richterspruch}{%
\subsubsection{5. Salomos weiser
Richterspruch}\label{salomos-weiser-richterspruch}}

\bibleverse{16}Damals kamen zwei Dirnen zum König und traten vor ihn;
\bibleverse{17}und das eine Weib sagte: »Mit Vergunst, Herr! Ich und
dieses Weib wohnen in demselben Hause, und ich gebar ein Kind in ihrer
Gegenwart im Hause. \bibleverse{18}Da geschah es zwei Tage nach meiner
Niederkunft, daß auch dieses Weib ein Kind gebar, und wir beide waren
allein, kein Fremder war sonst bei uns im Hause, nur wir beide befanden
uns im Hause. \bibleverse{19}Da starb das Kind dieses Weibes in der
Nacht, weil sie es im Schlaf erdrückt hatte. \bibleverse{20}Sie aber
stand mitten in der Nacht auf, nahm mein Kind von meiner Seite weg,
während deine Magd schlief, und legte es an ihre Brust, dagegen ihr
totes Kind legte sie mir in den Arm. \bibleverse{21}Als ich nun gegen
Morgen aufstand, um meinem Kinde die Brust zu geben, sah ich, daß es tot
war; als ich es aber bei Tagesanbruch genau betrachtete, sah ich, daß es
gar nicht mein Kind war, das ich geboren hatte.« \bibleverse{22}Da sagte
das andere Weib: »Nein, mein Kind ist das lebende, und dein Kind ist das
tote!«, jene aber versicherte: »Nein, dein Kind ist das tote und mein
Kind das lebende!« So stritten sie vor dem Könige. \bibleverse{23}Da
sagte der König: »Die eine behauptet: ›Dieses, das lebende Kind, gehört
mir, und dein Kind ist das tote‹; die andere behauptet: ›Nein, dein Kind
ist das tote und mein Kind das lebende!‹« \bibleverse{24}Dann befahl der
König: »Holt mir ein Schwert!« Als man nun das Schwert vor den König
gebracht hatte, \bibleverse{25}befahl er: »Teilt das lebende Kind in
zwei Teile und gebt dieser Frau die eine Hälfte und jener die andere
Hälfte!« \bibleverse{26}Da rief die Frau, der das lebende Kind gehörte
-- denn die mütterliche Liebe zu ihrem Kinde kam bei ihr zum Durchbruch
--, dem König die Worte zu: »Mit Vergunst, Herr! Gebt ihr das lebende
Kind und tötet es ja nicht!« Die andere aber rief: »Es soll weder mir
noch dir gehören: zerteilt es!« \bibleverse{27}Da entschied der König:
»Die da, welche gerufen hat: ›Gebt ihr das lebende Kind und tötet es ja
nicht!‹, die ist seine Mutter.« \bibleverse{28}Als nun ganz Israel den
Richterspruch vernahm, den der König gefällt hatte, fühlte man Ehrfurcht
vor dem König, denn man erkannte, daß eine göttliche Weisheit in ihm
wohnte, um Recht zu sprechen.

\hypertarget{salomos-oberste-beamte-und-vuxf6gte-seine-hofhaltung-macht-und-weisheit}{%
\subsubsection{6. Salomos oberste Beamte und Vögte; seine Hofhaltung,
Macht und
Weisheit}\label{salomos-oberste-beamte-und-vuxf6gte-seine-hofhaltung-macht-und-weisheit}}

\hypertarget{section-3}{%
\section{4}\label{section-3}}

\bibleverse{1}So herrschte also Salomo als König über ganz Israel;
\bibleverse{2}und dies waren seine Fürsten\textless sup
title=``=~obersten Beamten''\textgreater✲: Asarja, der Sohn Zadoks, war
(Hoher-) Priester; \bibleverse{3}Elihoreph und Ahija, die Söhne Sisas,
waren Staatsschreiber; Josaphat, der Sohn Ahiluds, war Kanzler;
\bibleverse{4}Benaja, der Sohn Jojadas, war oberster Heerführer;
{[}Zadok und Abjathar waren Priester;{]} \bibleverse{5}Asarja, der Sohn
Nathans, war über die Vögte gesetzt; Sabud, der Sohn Nathans, war der
priesterliche Freund✲ des Königs; \bibleverse{6}Ahisar war
Palastoberster, und Adoniram, der Sohn Addas, hatte die Oberaufsicht
über die Fronarbeiten.

\bibleverse{7}Salomo hatte zwölf Vögte, die über ganz Israel gesetzt
waren und den König und seinen Hof zu versorgen hatten, und zwar oblag
jedem von ihnen die Versorgung einen Monat lang im Jahre.
\bibleverse{8}Folgendes sind ihre Namen: Der Sohn Hurs, im Gebirge
Ephraim; \bibleverse{9}der Sohn Dekers in Makaz; (ihm unterstand)
Saalbim, Beth-Semes und Elon bis Beth-Hanan; \bibleverse{10}der Sohn
Heseds in Arubboth; ihm war Socho und die ganze Landschaft Hepher
überwiesen; \bibleverse{11}der Sohn Abinadabs, der das ganze Hügelland
von Dor unter sich hatte; er war mit Salomos Tochter Taphath verheiratet
worden; \bibleverse{12}Baana, der Sohn Ahiluds, in Thaanach, Megiddo und
ganz Beth-Sean, das neben Zarthan liegt unterhalb Jesreels, von
Beth-Sean bis nach Abel-Mehola, bis über Jokmeam hinaus;
\bibleverse{13}der Sohn Gebers zu Ramoth in Gilead, dem die Zeltdörfer
Jairs, des Sohnes Manasses, die in Gilead liegen, überwiesen waren; dazu
gehörte auch der Landstrich Argob in Basan, sechzig große Städte mit
Mauern und ehernen Riegeln; \bibleverse{14}Ahinadab, der Sohn Iddos, in
Mahanaim; \bibleverse{15}Ahimaaz in Naphthali; auch er war mit einer
Tochter Salomos, mit Basmath, verheiratet; \bibleverse{16}Baana, der
Sohn Husais, in Asser und Bealoth; \bibleverse{17}Josaphat, der Sohn
Pharuahs, in Issaschar; \bibleverse{18}Simei, der Sohn Elas, in
Benjamin; \bibleverse{19}Geber, der Sohn Uris, im Lande Gilead, dem
Lande Sihons, des Königs der Amoriter, und Ogs, des Königs von Basan;
außerdem war noch ein Vogt\textless sup title=``oder:
Statthalter''\textgreater✲ über alle Vögte im Lande eingesetzt.~--
\bibleverse{20}(Die Bewohner von) Juda und Israel waren so zahlreich wie
der Sand am Meer an Menge; sie aßen und tranken und waren guter Dinge.

\hypertarget{section-4}{%
\section{5}\label{section-4}}

\bibleverse{1}Salomo war aber Beherrscher aller Reiche vom Euphratstrom
bis zum Philisterland und bis an die Grenze Ägyptens; sie zahlten Tribut
und waren Salomo untertan, solange er lebte.

\bibleverse{2}Der tägliche Speisebedarf Salomos betrug dreißig Kor
Feinmehl und sechzig Kor gewöhnliches Mehl, \bibleverse{3}zehn gemästete
Rinder, zwanzig Rinder von der Weide und hundert Stück Kleinvieh,
ungerechnet die Hirsche, Gazellen, Damhirsche und das gemästete
Geflügel.

\hypertarget{salomos-macht-und-herrlichkeit-weisheit-und-dichterische-begabung}{%
\paragraph{Salomos Macht und Herrlichkeit, Weisheit und dichterische
Begabung}\label{salomos-macht-und-herrlichkeit-weisheit-und-dichterische-begabung}}

\bibleverse{4}Denn er herrschte über alle Länder diesseits des Euphrats
von Thiphsah✲ bis nach Gaza, über alle Könige diesseits des Euphrats,
und lebte in Frieden mit allen Völkern ringsum, \bibleverse{5}so daß
Juda und Israel von Dan bis Beerseba in Sicherheit wohnten, ein jeder
unter seinem Weinstock und unter seinem Feigenbaum, solange Salomo
lebte. \bibleverse{6}Salomo besaß auch viertausend Paar✲ Pferde für
seine Kriegswagen und zwölftausend Reitpferde, \bibleverse{7}und jene
Vögte versorgten den König Salomo und alle, deren Unterhalt dem König
Salomo oblag, ein jeder während seines Monats, und ließen es an nichts
fehlen. \bibleverse{8}Auch die Gerste und das Stroh für die Wagen- und
die Reitpferde hatten sie an den Ort zu liefern, wo er sich gerade
aufhielt, ein jeder, wie es ihn der Reihe nach traf. \bibleverse{9}Gott
verlieh aber dem Salomo Weisheit und Einsicht in sehr hohem Maße und
einen Verstand so weitreichend wie der Sand, der am Ufer des Meeres
liegt, \bibleverse{10}so daß die Weisheit Salomos größer war als die
Weisheit aller Bewohner des Morgenlandes und als alle Weisheit Ägyptens;
\bibleverse{11}ja, er war weiser als alle Menschen, auch weiser als der
Esrahiter Ethan und als Heman, Kalkol und Darda, die Söhne Mahols, und
sein Ruhm war unter allen Völkern ringsum verbreitet. \bibleverse{12}Er
verfaßte dreitausend Sprüche, und die Zahl seiner Lieder betrug tausend
und fünf. \bibleverse{13}Er besang die Bäume von der Zeder auf dem
Libanon an bis zum Ysop, der (aus den Fugen) an der Mauer hervorwächst;
er besang auch die vierfüßigen Tiere und die Vögel, das Gewürm und die
Fische; \bibleverse{14}und aus allen Völkern kamen die Leute, um die
Weisheit Salomos zu hören, von allen Königen der Erde her Leute, die von
seiner Weisheit gehört hatten.

\hypertarget{salomos-vertrag-mit-hiram-von-tyrus-und-vorbereitungen-zum-tempelbau}{%
\subsubsection{7. Salomos Vertrag mit Hiram von Tyrus und Vorbereitungen
zum
Tempelbau}\label{salomos-vertrag-mit-hiram-von-tyrus-und-vorbereitungen-zum-tempelbau}}

\bibleverse{15}Hiram aber, der König von Tyrus, schickte seine Knechte✲
an Salomo; denn er hatte gehört, daß man ihn als Nachfolger seines
Vaters zum König gesalbt hatte; Hiram war nämlich zeitlebens mit David
befreundet gewesen.

\hypertarget{salomos-botschaft-und-bitte-an-hiram}{%
\paragraph{Salomos Botschaft und Bitte an
Hiram}\label{salomos-botschaft-und-bitte-an-hiram}}

\bibleverse{16}Da ließ auch Salomo dem Hiram durch eine Gesandtschaft
sagen: \bibleverse{17}»Du weißt selbst, daß mein Vater David dem Namen
des HERRN, seines Gottes, keinen Tempel hat erbauen können wegen der
Kriege, in die seine Feinde ihn rings verwickelten, bis der HERR sie ihm
unter seine Fußsohlen legte. \bibleverse{18}Da mir jetzt aber der HERR,
mein Gott, auf allen Seiten Ruhe verschafft hat, so daß kein Widersacher
und keine Schwierigkeit mehr vorhanden ist, \bibleverse{19}so
beabsichtige ich nun, dem Namen des HERRN, meines Gottes, einen Tempel
zu erbauen, wie der HERR dies meinem Vater David verheißen hat mit den
Worten: ›Dein Sohn, den ich als deinen Nachfolger auf deinen Thron
setzen werde, der soll meinem Namen das Haus bauen.‹ \bibleverse{20}So
gib nun Befehl, daß man mir Zedern auf dem Libanon fälle; meine Leute
sollen dabei mit den deinigen zusammenarbeiten, und den Lohn für deine
Leute will ich dir geben, ganz wie du es bestimmst. Du weißt ja selbst,
daß es keinen bei uns gibt, der Bauholz so zu hauen verstände wie die
Sidonier.«

\hypertarget{hirams-antwort-und-zusage-abschluuxdf}{%
\paragraph{Hirams Antwort und Zusage;
Abschluß}\label{hirams-antwort-und-zusage-abschluuxdf}}

\bibleverse{21}Als Hiram diese Botschaft Salomos vernahm, freute er sich
außerordentlich und rief aus: »Gepriesen sei heute der HERR, der dem
David einen weisen Sohn zum Herrscher über dies große Volk gegeben hat!«
\bibleverse{22}Hierauf ließ Hiram dem Salomo durch eine Gesandtschaft
sagen: »Ich habe deine Botschaft an mich vernommen. Ich werde alle deine
Wünsche bezüglich des Zedern- und Zypressenholzes erfüllen.
\bibleverse{23}Meine Leute sollen die Hölzer vom Libanon ans Meer
hinabschaffen; dann will ich Flöße daraus auf dem Meer herstellen lassen
und sie bis an den Ort schaffen, den du mir angeben wirst; dort lasse
ich sie wieder auseinandernehmen, und du läßt sie dann abholen. Dafür
mußt du aber auch meine Wünsche erfüllen, indem du meinen Hofhalt mit
Speisebedarf versorgst.« \bibleverse{24}So lieferte also Hiram dem
Salomo Zedern- und Zypressenholz, soviel er wünschte;
\bibleverse{25}Salomo dagegen lieferte dem Hiram 20000 Kor Weizen für
den Unterhalt seines Hofes und 20000 Bath feinstes Öl von zerstoßenen
Oliven; so viel hatte Salomo dem Hiram Jahr für Jahr zu liefern.
\bibleverse{26}Der HERR aber verlieh dem Salomo Weisheit, wie er ihm
verheißen hatte; und es bestand ein freundschaftliches Verhältnis
zwischen Salomo und Hiram, denn sie hatten einen Bund\textless sup
title=``oder: Vertrag''\textgreater✲ miteinander geschlossen.

\hypertarget{salomos-fronarbeiter-und-die-letzten-vorarbeiten-fuxfcr-den-tempelbau}{%
\paragraph{Salomos Fronarbeiter und die letzten Vorarbeiten für den
Tempelbau}\label{salomos-fronarbeiter-und-die-letzten-vorarbeiten-fuxfcr-den-tempelbau}}

\bibleverse{27}Hierauf hob der König Salomo aus ganz Israel Fronarbeiter
aus, so daß die Fronenden sich auf dreißigtausend Mann beliefen.
\bibleverse{28}Er schickte sie abwechselnd auf den Libanon, jeden Monat
zehntausend, so daß sie einen Monat auf dem Libanon waren und zwei
Monate jeder zu Hause blieb; Adoniram hatte die Oberaufsicht über die
Fronarbeiter.~-- \bibleverse{29}Außerdem hatte Salomo 70000 Lastträger
und 80000 Steinhauer im Gebirge (Juda), \bibleverse{30}ungerechnet die
3300 von Salomo bestellten Werkführer\textless sup title=``oder:
Aufseher''\textgreater✲, welche die Arbeiten zu leiten und die Leute,
die mit der Arbeit beschäftigt waren, zu beaufsichtigen hatten.
\bibleverse{31}Der König gab auch Befehl, große und schwere\textless sup
title=``oder: kostbare''\textgreater✲ Steine zu brechen, um die
Grundmauern des Tempels mit Quadersteinen zu legen. \bibleverse{32}Die
Bauleute Salomos und die Bauleute Hirams behieben sie dann gemeinsam und
richteten die Hölzer und die Steine zum Bau des Tempels her.

\hypertarget{bau-des-tempels-und-der-kuxf6niglichen-paluxe4ste}{%
\subsubsection{8. Bau des Tempels und der königlichen
Paläste}\label{bau-des-tempels-und-der-kuxf6niglichen-paluxe4ste}}

\hypertarget{a-allgemeines-uxfcber-das-tempelhaus-mit-seinen-ruxe4umen-und-dem-anbau}{%
\paragraph{a) Allgemeines über das Tempelhaus mit seinen Räumen und dem
Anbau}\label{a-allgemeines-uxfcber-das-tempelhaus-mit-seinen-ruxe4umen-und-dem-anbau}}

\hypertarget{section-5}{%
\section{6}\label{section-5}}

\bibleverse{1}Im vierhundertachtzigsten Jahr nach dem Auszug der
Israeliten aus Ägypten, im vierten Jahre der Regierung Salomos über
Israel, im Monat Siw✲ -- das ist der zweite Monat --, da begann Salomo
den Bau des Tempels für den HERRN. \bibleverse{2}Der Tempel, den der
König Salomo für den HERRN baute, war sechzig Ellen lang, zwanzig Ellen
breit und dreißig Ellen hoch. \bibleverse{3}Die Halle an der Vorderseite
des Großraumes des Gebäudes war zwanzig Ellen breit, entsprechend der
Breite des Gebäudes, und zehn Ellen tief in der Längsrichtung des
Gebäudes. \bibleverse{4}(Salomo) ließ an dem Tempel Fenster\textless sup
title=``d.h. Lichtöffnungen''\textgreater✲ mit unbeweglichem Gitterwerk
anbringen \bibleverse{5}und führte an der Wand\textless sup
title=``oder: Mauer''\textgreater✲ des Tempels ringsum, sowohl um den
Großraum als auch um den Hinterraum, einen Anbau auf und richtete so
ringsum Seitenräume\textless sup title=``oder:
Seitenstockwerke''\textgreater✲ ein. \bibleverse{6}Das unterste
Stockwerk (dieses Anbaus) war fünf Ellen breit, das mittlere sechs, das
dritte sieben Ellen breit; denn Salomo hatte außen am Tempel ringsum
Absätze anbringen lassen, damit die Querbalken nicht in die
Mauern\textless sup title=``oder: Wände''\textgreater✲ des Hauptgebäudes
eingriffen. \bibleverse{7}Beim Bau des Tempels verwandte man aber nur
Steine, die beim Brechen (im Steinbruch) schon fertig behauen waren, so
daß man während der Errichtung des Gebäudes von Hämmern und Meißeln,
überhaupt von eisernen Werkzeugen im Tempel nichts hörte.
\bibleverse{8}Der Eingang zum untersten Stockwerk des Anbaus befand sich
an der Südseite des Gebäudes; und auf einer Wendeltreppe stieg man in
das mittlere und vom mittleren in das dritte Stockwerk hinauf.
\bibleverse{9}Als er so den Bau des Tempels vollendet hatte, deckte er
das Gebäude mit Balken und Bohlenreihen von Zedernholz.
\bibleverse{10}Den Anbau aber führte er um den ganzen Tempel her auf,
jedes Stockwerk fünf Ellen hoch, und verband ihn mit dem Hauptgebäude
durch Zedernbalken.

\hypertarget{b-eine-verheiuxdfung-gottes-an-salomo}{%
\paragraph{b) Eine Verheißung Gottes an
Salomo}\label{b-eine-verheiuxdfung-gottes-an-salomo}}

\bibleverse{11}Und es erging das Wort des HERRN an Salomo
folgendermaßen: \bibleverse{12}»So steht es mit diesem Hause, das du
eben erbaust: Wenn du nach meinen Satzungen wandelst und meinen
Weisungen nachkommst, alle meine Gebote hältst, so daß du nach ihnen
wandelst, so will ich die Verheißung an dir wahrmachen, die ich deinem
Vater David gegeben habe: \bibleverse{13}ich werde alsdann inmitten der
Israeliten wohnen und mein Volk Israel nicht verlassen.«

\hypertarget{c-die-innere-ausstattung-des-tempels}{%
\paragraph{c) Die innere Ausstattung des
Tempels}\label{c-die-innere-ausstattung-des-tempels}}

\bibleverse{14}Als nun Salomo mit dem Bau des Hauses fertig war,
\bibleverse{15}bekleidete er die Wände des Tempels im Inneren des Hauses
mit Brettern von Zedernholz: vom Fußboden des Gebäudes bis zu den Balken
der Decke täfelte er die inneren Wände mit Holz und belegte den Fußboden
des Tempels mit Bohlen von Zypressenholz. \bibleverse{16}Ebenso
bekleidete er die zwanzig Ellen an der Hinterseite des Tempels mit
Zedernbrettern vom Fußboden bis an die Decke und baute sich so im
Inneren des Tempels den Hinterraum als das Allerheiligste aus.
\bibleverse{17}Aber das vordere Tempelhaus, das heißt der Großraum vorn
vor dem Allerheiligsten, maß vierzig Ellen. \bibleverse{18}So bestand
denn das ganze Haus im Inneren aus Zedernholz, verziert mit Schnitzwerk
in Form von wilden Gurken und Blumengewinden, alles von Zedernholz, so
daß kein Stein zu sehen war. \bibleverse{19}Den Hinterraum aber im
Inneren des Gebäudes richtete er so ein, daß er die Bundeslade des HERRN
dort unterbringen konnte. \bibleverse{20}Der Hinterraum war nämlich
zwanzig Ellen lang, zwanzig Ellen breit und zwanzig Ellen hoch, und er
ließ ihn mit feinem Gold überziehen; auch den Zedernholz-Altar vor dem
Hinterraum überzog er damit. \bibleverse{21}Weiter überzog Salomo das
Gebäude\textless sup title=``d.h. den Großraum''\textgreater✲ im Inneren
mit feinem Gold und zog vor dem Hinterraum goldene Ketten her und
überzog auch ihn mit Gold. \bibleverse{22}Den ganzen Tempel überzog er
also im Inneren mit Gold, das ganze Gebäude vollständig; auch den Altar,
der vor dem Allerheiligsten stand, überzog er ganz mit Gold.

\hypertarget{d-die-ausstattung-des-allerheiligsten}{%
\paragraph{d) Die Ausstattung des
Allerheiligsten}\label{d-die-ausstattung-des-allerheiligsten}}

\bibleverse{23}Weiter ließ er im Allerheiligsten zwei Cherube aus
Ölbaumholz anfertigen, je zehn Ellen hoch. \bibleverse{24}Jeder Flügel
der beiden Cherube maß fünf Ellen, so daß von der einen Flügelspitze bis
zu der andern ein Abstand von zehn Ellen war. \bibleverse{25}Und zehn
Ellen maß auch der andere Cherub: beide Cherube hatten dasselbe Maß und
dieselbe Gestalt; \bibleverse{26}die Höhe des einen Cherubs betrug zehn
Ellen und ebenso die des anderen Cherubs. \bibleverse{27}Er stellte die
beiden Cherube dann in der Mitte des innersten Tempelraumes auf, und sie
hielten ihre Flügel so ausgebreitet, daß der Flügel des einen Cherubs
diese Wand des Allerheiligsten und der Flügel des andern Cherubs die
andere Wand berührte, während ihre inneren Flügel in der Richtung nach
der Mitte des Raumes aneinander stießen. \bibleverse{28}Auch die Cherube
überzog er mit Gold. \bibleverse{29}Weiter ließ er an allen Wänden des
Gebäudes ringsum Schnitzwerk von Cheruben, Palmen und Blumengewinden
anbringen. \bibleverse{30}Auch den Fußboden des Tempels ließ er mit Gold
überziehen sowohl im Hinterraum als auch im Vorderraum.

\hypertarget{e-die-tuxfcren-und-der-innere-vorhof}{%
\paragraph{e) Die Türen und der innere
Vorhof}\label{e-die-tuxfcren-und-der-innere-vorhof}}

\bibleverse{31}Und für den Eingang in den Hinterraum\textless sup
title=``=~das Allerheiligste''\textgreater✲ ließ er eine Türeinfassung
von Ölbaumholz anfertigen; die Oberschwelle und die Pfosten bildeten ein
Fünfeck. \bibleverse{32}Auf den beiden Türflügeln von Ölbaumholz aber
ließ er Schnitzwerk von Cheruben, Palmen und Blumengewinden anbringen
und sie dann mit Gold überziehen, und zwar die Cherube und die Palmen
mit breitgeschlagenem Gold. \bibleverse{33}Ebenso ließ er auch für den
Eingang zum Großraum eine Türeinfassung von Ölbaumholz anfertigen, die
ein Viereck bildete, \bibleverse{34}dazu zwei Türflügel von
Zypressenholz, von denen jeder aus zwei drehbaren Blättern bestand.
\bibleverse{35}Er ließ dann Schnitzwerk darauf anbringen, nämlich
Cherube, Palmen und Blumengewinde, und ließ sie mit Goldblech
überziehen, das dem Schnitzwerk genau angepaßt war.
\bibleverse{36}Darauf ließ er den inneren Vorhof mit einer Mauer
umziehen, die aus drei Lagen von Quadersteinen und einer Lage von
Zedernbalken bestand.

\hypertarget{f-die-bauzeit}{%
\paragraph{f) Die Bauzeit}\label{f-die-bauzeit}}

\bibleverse{37}Im vierten Jahre (der Regierung Salomos), im Monat Siw✲,
war der Grund zum Tempel des HERRN gelegt worden, \bibleverse{38}und im
elften Jahre, im Monat Bul -- das ist der achte Monat --, war der Tempel
vollendet in allen seinen Teilen und mit allem, was dazu gehörte; sieben
Jahre also hatte man daran gebaut.

\hypertarget{beschreibung-der-uxfcbrigen-weltlichen-bauten-salomos}{%
\subsubsection{9. Beschreibung der übrigen (weltlichen) Bauten
Salomos}\label{beschreibung-der-uxfcbrigen-weltlichen-bauten-salomos}}

\hypertarget{section-6}{%
\section{7}\label{section-6}}

\bibleverse{1}Aber an seinem eigenen Hause baute Salomo dreizehn Jahre
lang, bis er seinen Palast ganz vollendet hatte.

\hypertarget{a-das-libanonwaldhaus}{%
\paragraph{a) Das Libanonwaldhaus}\label{a-das-libanonwaldhaus}}

\bibleverse{2}Er erbaute nämlich das Libanonwaldhaus, das hundert Ellen
lang, fünfzig Ellen breit und dreißig Ellen hoch war, auf vier Reihen
von Zedernsäulen, und auf den Säulen ruhten Balken\textless sup
title=``oder: Kapitäle''\textgreater✲ von Zedernholz. \bibleverse{3}Eine
Decke von Zedernholz befand sich über den Tragbalken, die auf den Säulen
lagen, zusammen fünfundvierzig, fünfzehn in jeder Reihe.
\bibleverse{4}Und Durchblicke waren da in drei Reihen, und die
Lichtöffnungen lagen einander gegenüber, dreimal. \bibleverse{5}Alle
Türen und Lichtöffnungen waren viereckig, im Durchblick; und die
Lichtöffnungen lagen einander gegenüber, dreimal.

\hypertarget{b-die-uxfcbrigen-palastbauten-salomos}{%
\paragraph{b) Die übrigen Palastbauten
Salomos}\label{b-die-uxfcbrigen-palastbauten-salomos}}

\bibleverse{6}Sodann erbaute er den Säulensaal, der fünfzig Ellen lang
und dreißig Ellen breit war; eine Halle mit Säulen und eine Treppe
befand sich davor.~-- \bibleverse{7}Weiter erbaute er den Thronsaal, in
dem er Recht sprach, den Gerichtssaal; der war mit Zedernholz getäfelt
vom Fußboden bis an die Decke.~-- \bibleverse{8}Sodann sein eigener
Palast, in dem er wohnte, im zweiten Vorhof einwärts vom Thronsaal, war
ein Bau von derselben Art. Auch für die Tochter des Pharaos, die Salomo
geheiratet hatte, baute er einen Palast gleich jener Halle.

\bibleverse{9}Alle diese Bauten waren aus Prachtsteinen aufgeführt, die
nach Quadermaß behauen und innen wie außen mit der Säge zugeschnitten
waren, und zwar vom Grund✲ an bis zu den Gesimsen und von außen bis zu
dem großen Vorhof. \bibleverse{10}Die Fundamente aber bestanden aus
gewaltig großen Steinen, aus Blöcken von zehn Ellen und solchen von acht
Ellen Länge; \bibleverse{11}und darüber lagen Prachtsteine, die nach
Quadermaß zugehauen waren, und Zedernbalken. \bibleverse{12}Der große
Vorhof aber war rings mit einer Mauer umgeben, die aus drei Lagen
Quadersteinen und einer Lage Zedernbalken bestand; ebenso war es auch
bei dem inneren Vorhof am Tempel des HERRN und bei dem Vorhof am
Säulensaal des Palastes.

\hypertarget{die-ausstattung-des-tempels}{%
\subsubsection{10. Die Ausstattung des
Tempels}\label{die-ausstattung-des-tempels}}

\hypertarget{a-angaben-uxfcber-den-kuxfcnstler-hiram-von-tyrus}{%
\paragraph{a) Angaben über den Künstler Hiram von
Tyrus}\label{a-angaben-uxfcber-den-kuxfcnstler-hiram-von-tyrus}}

\bibleverse{13}Darauf sandte der König Salomo hin und ließ Hiram von
Tyrus holen. \bibleverse{14}Dieser war der Sohn einer Witwe aus dem
Stamme Naphthali, sein Vater aber war ein tyrischer Kupferschmied. Er
war ein hochbegabter, einsichtsvoller und kunstsinniger Mann, der sich
auf die Herstellung von Erzarbeiten jeder Art verstand. Dieser Mann kam
also zum König Salomo und führte alle Erzarbeiten für ihn aus.

\hypertarget{b-die-beiden-ehernen-suxe4ulen-jachin-und-boas}{%
\paragraph{b) Die beiden ehernen Säulen (Jachin und
Boas)}\label{b-die-beiden-ehernen-suxe4ulen-jachin-und-boas}}

\bibleverse{15}So stellte er die beiden ehernen Säulen her: achtzehn
Ellen war die eine Säule hoch, ein Faden von zwölf Ellen umspannte sie,
und ihre Dicke betrug vier Finger; inwendig war sie hohl; und ebenso
machte er die zweite Säule. \bibleverse{16}Auch fertigte er zwei aus Erz
gegossene Knäufe\textless sup title=``oder: Kapitelle''\textgreater✲ an,
um sie oben auf die Säulen zu setzen; jedes dieser beiden Kapitelle
hatte fünf Ellen Höhe. \bibleverse{17}Zur Bekleidung der Kapitelle, die
sich oben auf den Säulen befanden, dienten Flechtwerke von kettenartig
gearbeiteten Schnüren. \bibleverse{18}b und ebenso verfertigte er sie
für das andere Kapitell. \bibleverse{18}b und ebenso verfertigte er sie
für das andere Kapitell. \bibleverse{19}Die Kapitelle, die sich oben auf
den Säulen befanden, waren in Form von Lilien gearbeitet, in der
Vorhalle, vier Ellen. \bibleverse{20}a Und Kapitelle waren auf den
beiden Säulen auch oberhalb nahe bei dem Wulst, der nach der Seite des
Flechtwerks ging. \bibleverse{20}a Und Kapitelle waren auf den beiden
Säulen auch oberhalb nahe bei dem Wulst, der nach der Seite des
Flechtwerks ging. \bibleverse{21}Er stellte diese Säulen am Eingang zur
Vorhalle des Tempels auf; die eine Säule, die er rechts aufstellte,
nannte er Jachin\textless sup title=``d.h. er gründet
fest''\textgreater✲, und der anderen, die er links aufstellte, gab er
den Namen Boas\textless sup title=``d.h. in ihm ist
Kraft''\textgreater✲. \bibleverse{22}Oben auf den Säulen aber war ein
lilienförmiges Gebilde; und damit war die Herstellung der Säulen
vollendet.

\hypertarget{c-das-eherne-meer-oder-grouxdfe-wasserbecken}{%
\paragraph{c) Das eherne Meer (oder große
Wasserbecken)}\label{c-das-eherne-meer-oder-grouxdfe-wasserbecken}}

\bibleverse{23}Hiram fertigte auch das aus Erz gegossene
Meer\textless sup title=``oder: große Wasserbecken''\textgreater✲, das
von einem Rande bis zum andern zehn Ellen maß, ringsum gerundet und fünf
Ellen hoch; eine Schnur von dreißig Ellen war erforderlich, um es ganz
zu umspannen. \bibleverse{24}Unterhalb seines Randes waren
Gebilde\textless sup title=``oder: Verzierungen''\textgreater✲ von
wilden Gurken✲ angebracht, die es rings umgaben, je zehn auf die Elle;
sie bildeten einen Kranz um das Becken, zwei Reihen Gurken, die gleich
beim Guß mitgegossen worden waren. \bibleverse{25}Es ruhte auf zwölf
Rindern, von denen drei nach Norden, drei nach Westen, drei nach Süden
und drei nach Osten gewandt waren; das Becken aber lag oben auf ihnen,
und die Hinterseite war bei allen Rindern nach innen gekehrt.
\bibleverse{26}Die Dicke (der Wand) des Beckens betrug eine Handbreite,
und sein Rand war wie die Arbeit eines Becherrandes geformt, nach Art
einer blühenden Lilie. Es faßte zweitausend Bath.

\hypertarget{d-die-zehn-gestuxfchle-und-die-zehn-opferkessel}{%
\paragraph{d) Die zehn Gestühle und die zehn
Opferkessel}\label{d-die-zehn-gestuxfchle-und-die-zehn-opferkessel}}

\bibleverse{27}Er fertigte auch die zehn Gestühle\textless sup
title=``d.h. fahrbare Becken oder: Kessel''\textgreater✲ aus Erz; jedes
dieser Gestühle war vier Ellen lang, vier Ellen breit und drei Ellen
hoch. \bibleverse{28}Die Gestühle waren aber folgendermaßen gearbeitet:
sie hatten Stege\textless sup title=``oder: Verschlußstücke,
Schlußleisten''\textgreater✲, und zwar Stege auch zwischen den
Leitersprossen; \bibleverse{29}auf den Stegen aber, die zwischen den
Leitersprossen waren, befanden sich Löwen, Rinder und Cherube, und
ebenso an den Leitersprossen oben und unten; unterhalb der Löwen und
Rinder waren Kränze in Form von Gewinden angebracht.
\bibleverse{30}Jedes Gestühl hatte vier eherne Räder und eherne Achsen,
und an seinen vier Ecken befanden sich Aufsätze\textless sup
title=``oder: Ansätze''\textgreater✲, die unterhalb des Beckens
angegossen waren; jenseits eines jeden waren Gewinde.
\bibleverse{31}Sein Auflager\textless sup title=``oder:
Mundstück''\textgreater✲ befand sich innerhalb der Aufsätze und ragte
eine Elle darüber hinaus; sein Auflager war rund, in Gestell-Arbeit,
anderthalb Ellen im Durchmesser; und auch an dem Auflager\textless sup
title=``oder: Mundstück''\textgreater✲ war Bildwerk angebracht, seine
Stege aber waren viereckig, nicht rund. \bibleverse{32}Die vier Räder
befanden sich unten an den Stegen, und die Halter der Räder waren an dem
Gestühl befestigt, und jedes Rad war anderthalb Ellen hoch.
\bibleverse{33}Die Räder waren wie Wagenräder gearbeitet; ihre Halter
und Felgen, ihre Speichen und Naben -- alles war Gußwerk.
\bibleverse{34}An den vier Eckpfosten jedes Gestühls befanden sich vier
Aufsätze, die aus einem Guß mit dem Gestühl waren. \bibleverse{35}Oben
auf dem Gestühl war eine Art Gestell, eine halbe Elle hoch, rund
ringsum; und oben auf dem Gestühl war der Aufsatz mit seinen Haltern,
die das Becken stützten, seine Halter und Stege aus einem Guß mit ihm.
\bibleverse{36}Auf die Tafeln und Stege grub er Bildwerk von Cheruben,
Löwen und Palmen ein, soweit leerer Raum bei ihnen vorhanden war, und
Gewinde ringsum. \bibleverse{37}Auf diese Weise stellte er die zehn
Gestühle her; alle hatten denselben Guß, dasselbe Maß und dieselbe
Gestaltung.~-- \bibleverse{38}Dann fertigte er zehn eherne Kessel, von
denen jeder vierzig Bath faßte und jeder einen Durchmesser von vier
Ellen hatte; auf jedes der zehn Gestühle kam ein solcher Kessel.
\bibleverse{39}Von den Gestühlen stellte er fünf auf der Südseite und
fünf auf der Nordseite des Tempels auf; das große Wasserbecken aber
erhielt seinen Platz auf der Südostseite des Tempels.

\hypertarget{e-weitere-opfergeruxe4te-von-erz-zusammenfassende-uxfcbersicht}{%
\paragraph{e) Weitere Opfergeräte von Erz; zusammenfassende
Übersicht}\label{e-weitere-opfergeruxe4te-von-erz-zusammenfassende-uxfcbersicht}}

\bibleverse{40}Weiter fertigte Hiram die Töpfe, Schaufeln und
Sprengschalen an und vollendete so alle Arbeiten, die er für den König
Salomo am Gotteshause herzustellen hatte, \bibleverse{41}nämlich zwei
Säulen mit den beiden kugelförmigen Knäufen\textless sup title=``oder:
Kapitellen''\textgreater✲ oben auf den Säulen sowie die zwei Geflechte
zur Bekleidung der beiden kugelförmigen Kapitelle oben auf den Säulen;
\bibleverse{42}und die vierhundert Granatäpfel für die beiden
Flechtwerke: zwei Reihen Granatäpfel für jedes Flechtwerk;
\bibleverse{43}ferner die zehn Gestühle nebst den zehn Kesseln auf den
Gestühlen, \bibleverse{44}und das eine große Wasserbecken mit den zwölf
Rindern unter dem Becken; \bibleverse{45}sodann die Töpfe, Schaufeln und
Sprengschalen. Alle diese Kunstwerke, die Hiram dem König Salomo für den
Tempel des HERRN fertigte, waren von geplättetem✲ Erz. \bibleverse{46}In
der Jordanaue hatte der König sie gießen lassen an der Furt von Adama,
zwischen Sukkoth und Zarethan\textless sup title=``oder:
Zorthan''\textgreater✲. \bibleverse{47}Salomo ließ aber alle diese
Metallarbeiten ungewogen wegen ihrer übergroßen Menge; das Gewicht des
Erzes wurde nicht festgestellt.

\hypertarget{f-nachtrag-die-goldgeruxe4te-des-tempels-abschluuxdf}{%
\paragraph{f) Nachtrag: die Goldgeräte des Tempels;
Abschluß}\label{f-nachtrag-die-goldgeruxe4te-des-tempels-abschluuxdf}}

\bibleverse{48}Ferner ließ Salomo alle die Ausstattungsgegenstände und
Geräte anfertigen, die zum Tempel des HERRN gehörten: den vergoldeten
Altar und den vergoldeten Tisch, auf dem die Schaubrote lagen;
\bibleverse{49}weiter die Leuchter, nämlich fünf auf der rechten und
fünf auf der linken Seite vor dem Allerheiligsten aus gediegenem Gold,
dazu die Blüten, Lampen und Lichtscheren aus Gold; \bibleverse{50}weiter
die Becken, Messer, Sprengschalen, Schüsseln und Räucherpfannen aus
gediegenem Gold. Was schließlich die Angeln an den Türflügeln im
Innenraum des Tempels, des Allerheiligsten, sowie an den Türflügeln des
Großraumes des Tempels betrifft, so waren sie von Gold.

\bibleverse{51}Als nun alle Arbeiten, die der König Salomo für den
Tempel des HERRN hatte herstellen lassen, fertig waren, ließ Salomo auch
die Weihgeschenke seines Vaters David hineinbringen; und zwar legte er
das Silber, das Gold und die Geräte in die Schatzkammern des Tempels des
HERRN.

\hypertarget{die-tempelweihe-salomos}{%
\subsubsection{11. Die Tempelweihe
Salomos}\label{die-tempelweihe-salomos}}

\hypertarget{a-uxfcberfuxfchrung-der-bundeslade-in-den-tempel}{%
\paragraph{a) Überführung der Bundeslade in den
Tempel}\label{a-uxfcberfuxfchrung-der-bundeslade-in-den-tempel}}

\hypertarget{section-7}{%
\section{8}\label{section-7}}

\bibleverse{1}Damals ließ Salomo die Ältesten✲ der Israeliten und (zwar
besonders) alle Häupter der Stämme und die Obersten\textless sup
title=``oder: Fürsten''\textgreater✲ der israelitischen Geschlechter bei
sich in Jerusalem zusammenkommen, um die Bundeslade des HERRN aus der
Davidstadt, das ist Zion, hinaufzubringen. \bibleverse{2}So versammelten
sich denn alle israelitischen Männer beim König Salomo am
Fest\textless sup title=``d.h. Laubhüttenfest''\textgreater✲ im Monat
Ethanim, das ist der siebte Monat. \bibleverse{3}Als nun alle Ältesten✲
der Israeliten sich eingefunden hatten, hoben die Priester die Lade des
HERRN auf \bibleverse{4}und trugen sie hinauf, ebenso das
Offenbarungszelt und alle heiligen Geräte, die sich im Zelt befanden:
die Priester und Leviten trugen sie hinauf. \bibleverse{5}Der König
Salomo aber und die ganze Volksgemeinde Israel, die sich um ihn
versammelt hatte, opferten vor der Lade so viele Stück Kleinvieh und
Rinder, daß ihre Menge geradezu unzählbar war. \bibleverse{6}Alsdann
brachten die Priester die Lade mit dem Bundesgesetz des HERRN an die für
sie bestimmte Stätte, nämlich in den Hinterraum des Tempels, in das
Allerheiligste, unter die Flügel der Cherube; \bibleverse{7}die Cherube
hielten nämlich ihre Flügel ausgebreitet über die Stätte, wo die Lade
stand, so daß die Cherube eine Decke oben über der Lade und deren
Tragstangen bildeten. \bibleverse{8}Die Tragstangen aber waren so lang,
daß die Spitzen der Stangen im Heiligtum an der Vorderseite des
Allerheiligsten eben noch sichtbar waren; weiter außen aber waren sie
nicht zu sehen; und sie sind dort geblieben bis auf den heutigen Tag.
\bibleverse{9}In der Lade befand sich nichts als nur die beiden
steinernen Tafeln, die Mose am Horeb hineingelegt hatte, die Tafeln des
Bundes, den der HERR mit den Israeliten nach ihrem Auszug aus Ägypten
geschlossen hatte.

\hypertarget{b-die-erscheinung-der-herrlichkeit-gottes}{%
\paragraph{b) Die Erscheinung der Herrlichkeit
Gottes}\label{b-die-erscheinung-der-herrlichkeit-gottes}}

\bibleverse{10}Als aber die Priester aus dem Heiligtum hinausgetreten
waren, da erfüllte die Wolke den Tempel des HERRN, \bibleverse{11}so daß
die Priester wegen der Wolke nicht hintreten konnten, um ihren Dienst zu
verrichten; denn die Herrlichkeit\textless sup title=``d.h. der
Lichtglanz''\textgreater✲ des HERRN erfüllte den Tempel des HERRN.
\bibleverse{12}Damals sprach Salomo: »Der HERR hat gesagt, er wolle im
Dunkeln wohnen. \bibleverse{13}So habe ich dir nun ein Haus zur Wohnung
gebaut, eine Stätte zum Wohnsitz für dich auf ewige Zeiten.«

\hypertarget{c-salomos-weihespruch-und-weiherede-an-das-volk}{%
\paragraph{c) Salomos Weihespruch und Weiherede an das
Volk}\label{c-salomos-weihespruch-und-weiherede-an-das-volk}}

\bibleverse{14}Hierauf wandte der König sich um und segnete die ganze
Volksgemeinde Israel, wobei die ganze Gemeinde Israel dastand.
\bibleverse{15}Dann sagte er: »Gepriesen sei der HERR, der Gott Israels,
der die Verheißung, die er meinem Vater David mündlich gegeben, nun
tatsächlich erfüllt hat, da er sagte\textless sup title=``2.Sam
7,6-13''\textgreater✲: \bibleverse{16}›Seit der Zeit, wo ich mein Volk
Israel aus Ägypten hinausgeführt, habe ich aus allen Stämmen Israels nie
eine Stadt dazu erwählt, daß mir daselbst ein Haus gebaut würde, an dem
mein Name haften sollte; David aber habe ich dazu ersehen, Herrscher
über mein Volk Israel zu sein.‹ \bibleverse{17}Nun hatte zwar mein Vater
David den Wunsch, dem Namen des HERRN, des Gottes Israels, ein Haus zu
bauen; \bibleverse{18}aber der HERR ließ meinem Vater David verkünden:
›Daß du den Wunsch gehegt hast, meinem Namen ein Haus zu bauen, an
diesem Vorhaben hast du wohl getan; \bibleverse{19}jedoch nicht du
sollst das Haus mir bauen, sondern dein leiblicher Sohn, der dir geboren
werden wird, der soll meinem Namen das Haus bauen.‹ \bibleverse{20}Nun
hat der HERR diese Verheißung, die er gegeben hat, in Erfüllung gehen
lassen; denn ich bin an die Stelle meines Vaters David getreten und habe
den Thron Israels bestiegen, wie der HERR es verheißen hatte, und habe
dem Namen des HERRN, des Gottes Israels, den Tempel erbaut,
\bibleverse{21}und ich habe darin eine Stätte geschaffen für die Lade,
in der die Urkunde des Bundes liegt, den der HERR mit unsern Vätern
geschlossen hat, als er sie aus dem Lande Ägypten hinausführte.«

\hypertarget{d-salomos-weihegebet}{%
\paragraph{d) Salomos Weihegebet}\label{d-salomos-weihegebet}}

\bibleverse{22}Nunmehr trat Salomo angesichts der ganzen Gemeinde Israel
vor den Altar des HERRN, breitete seine Hände gen Himmel aus
\bibleverse{23}und betete: »HERR, du Gott Israels! Kein Gott weder im
Himmel droben noch auf der Erde unten ist dir gleich, der du den Bund
und die Gnade deinen Knechten bewahrst, die mit ihrem ganzen Herzen vor
dir wandeln. \bibleverse{24}Du hast deinem Knechte David, meinem Vater,
das Versprechen gehalten, was du ihm gegeben hattest; ja, was du
mündlich zugesagt hattest, das hast du tatsächlich erfüllt, wie es heute
sichtbar zutage liegt. \bibleverse{25}Und nun, HERR, du Gott Israels,
halte deinem Knechte David, meinem Vater, auch die Verheißung, die du
ihm gegeben hast mit den Worten: ›Es soll dir nie an einem (Nachkommen)
fehlen, der vor meinem Angesicht auf dem Throne Israels sitze, wofern
nur deine Söhne✲ auf ihren Weg achthaben, daß sie vor meinen Augen
wandeln, wie du vor mir gewandelt bist.‹ \bibleverse{26}Nun also, Gott
Israels, laß deine Verheißung, die du deinem Knecht David, meinem Vater,
gegeben hast, in Erfüllung gehen!

\bibleverse{27}Wie aber? Sollte Gott wirklich auf der Erde Wohnung
nehmen? Siehe, der Himmel und aller Himmel Himmel\textless sup
title=``=~die höchsten oder: obersten Himmel''\textgreater✲ können dich
nicht fassen: wieviel weniger dieses Haus, das ich gebaut habe!
\bibleverse{28}Und doch wende dich dem Gebet deines Knechtes und seinem
Flehen zu, HERR, mein Gott, und höre auf das laute Rufen und das Gebet,
das dein Knecht heute an dich richtet! \bibleverse{29}Laß deine Augen
bei Tag und bei Nacht offenstehen über diesem Hause, über der Stätte,
von der du verheißen hast: ›Mein Name soll daselbst wohnen!‹, daß du auf
das Gebet hörest, das dein Knecht an dieser Stätte verrichten wird.
\bibleverse{30}So höre denn auf das Flehen deines Knechtes und deines
Volkes Israel, sooft sie an dieser Stätte beten werden! Ja, erhöre du es
an der Stätte, wo du thronst, im Himmel, und wenn du es hörst, so
vergib! \bibleverse{31}Wenn sich jemand gegen seinen Nächsten vergeht
und man ihm einen Eid auferlegt, den er schwören soll, und er kommt und
schwört\textless sup title=``=~spricht den Fluch aus''\textgreater✲ vor
deinem Altar in diesem Hause: \bibleverse{32}so wollest du es im Himmel
hören und eingreifen und deinen Knechten Recht schaffen, indem du den
Schuldigen dadurch für schuldig erklärst, daß du sein Tun auf sein Haupt
zurückfallen läßt, dem Unschuldigen aber dadurch zu seinem Recht
verhilfst, daß du ihm zuteil werden läßt nach seiner
Gerechtigkeit\textless sup title=``oder: Unschuld''\textgreater✲!

\bibleverse{33}Wenn dein Volk Israel von einem Feinde geschlagen wird,
weil es sich gegen dich versündigt hat, sich dann aber wieder zu dir
bekehrt und deinen Namen bekennt und in diesem Hause zu dir betet und
fleht: \bibleverse{34}so wollest du es im Himmel hören und deinem Volk
Israel die Sünde vergeben und sie in dem Lande wohnen
lassen\textless sup title=``oder: in das Land
zurückbringen''\textgreater✲, das du ihren Vätern gegeben hast!

\bibleverse{35}Wenn der Himmel verschlossen bleibt und kein Regen fällt,
weil sie gegen dich gesündigt haben, und sie dann an dieser
Stätte\textless sup title=``vgl. V.30''\textgreater✲ beten und deinen
Namen bekennen und sich von ihrer Sünde bekehren, weil du sie gedemütigt
hast: \bibleverse{36}so wollest du es im Himmel hören und deinen
Knechten, deinem Volk Israel, die Sünde vergeben, indem du sie auf den
rechten Weg weisest, auf dem sie wandeln sollen, und wollest Regen
fallen lassen auf dein Land, das du deinem Volk zum Erbbesitz gegeben
hast!

\bibleverse{37}Wenn eine Hungersnot im Lande herrscht, wenn die Pest
ausbricht, wenn Getreidebrand oder Vergilben des Getreides, Heuschrecken
oder Ungeziefer über das Land kommen, wenn seine Feinde es in einer
seiner Ortschaften bedrängen oder sonst irgendeine Plage, irgendeine
Krankheit sie heimsucht: \bibleverse{38}was man alsdann bittet und
fleht, es geschehe von einem einzelnen Menschen oder von deinem ganzen
Volk Israel, wenn ein jeder sich in seinem Gewissen getroffen fühlt und
er seine Hände nach diesem Hause hin ausstreckt: \bibleverse{39}so
wollest du es im Himmel hören an der Stätte, wo du thronst, und wollest
Verzeihung gewähren und einem jeden ganz nach Verdienst vergelten, wie
du sein Herz kennst -- denn du allein kennst das Herz aller
Menschenkinder --, \bibleverse{40}damit sie dich allezeit fürchten,
solange sie auf dem Boden des Landes leben, das du unsern Vätern gegeben
hast!

\bibleverse{41}Aber auch den Fremdling, der nicht zu deinem Volk Israel
gehört, sondern aus fernem Lande um deines Namens willen hergekommen
ist~-- \bibleverse{42}denn sie werden von deinem großen Namen, von
deiner starken Hand und deinem hocherhobenen Arm hören --, wenn er also
kommt und vor diesem Tempel\textless sup title=``vgl.
V.30''\textgreater✲ betet: \bibleverse{43}so wollest du ihn im Himmel
hören an der Stätte, wo du thronst, und alles das tun, um was der
Fremdling dich anruft, auf daß alle Völker der Erde deinen Namen
kennenlernen, damit sie dich ebenso fürchten wie dein Volk Israel und
damit sie innewerden, daß dieses Haus, das ich erbaut habe, deinem Namen
als Besitz zugesprochen ist.

\bibleverse{44}Wenn dein Volk gegen seine Feinde zum Kampf auszieht auf
dem Wege, auf den du sie senden wirst, und sie sich im Gebet zu dem
HERRN nach der Stadt hin wenden, die du erwählt hast, und nach dem
Tempel hin, den ich zu Ehren deines Namens erbaut habe:
\bibleverse{45}so wollest du ihr Gebet und Flehen im Himmel hören und
ihnen zu ihrem Recht verhelfen!

\bibleverse{46}Wenn sie sich an dir versündigt haben -- es gibt ja
keinen Menschen, der nicht sündigt -- und du ihnen zürnst und sie dem
Feinde preisgibst, so daß ihre Besieger sie gefangen wegführen in
Feindesland, es liege fern oder nahe, \bibleverse{47}und sie dann in dem
Lande, wohin sie in die Gefangenschaft geführt worden sind, in sich
gehen und sich bekehren und dich im Lande ihrer Zwingherren mit dem
Bekenntnis anrufen: ›Wir haben gesündigt und uns vergangen, wir haben
gottlos gehandelt!‹, \bibleverse{48}wenn sie sich also im Lande ihrer
Feinde, die sie in Gefangenschaft halten, mit ganzem Herzen und mit
ganzer Seele dir wieder zuwenden und zu dir beten in der Richtung nach
ihrem Lande hin, das du ihren Vätern gegeben hast, und nach der Stadt
hin, die du dir erwählt hast, und nach dem Tempel hin, den ich zu Ehren
deines Namens erbaut habe: \bibleverse{49}so wollest du ihr Gebet und
Flehen im Himmel an der Stätte, wo du thronst, hören und ihnen zu ihrem
Recht verhelfen \bibleverse{50}und wollest deinem Volke vergeben, was
sie gegen dich gesündigt haben, und alle ihre Übertretungen, mit denen
sie sich gegen dich vergangen haben, und wollest sie Barmherzigkeit
finden lassen bei ihren Zwingherren, so daß diese Erbarmen mit ihnen
haben! \bibleverse{51}Denn sie sind dein Volk und dein Eigentum, das du
aus Ägypten, mitten aus dem Eisen-Schmelzofen, weggeführt hast.

\bibleverse{52}So laß denn deine Augen offenstehen für das Flehen deines
Knechtes und für das Flehen deines Volkes Israel, daß du sie erhörest,
sooft sie dich anrufen! \bibleverse{53}Denn du selbst hast sie dir zum
Eigentum ausgesondert aus allen Völkern der Erde, wie du es durch den
Mund deines Knechtes Mose ausgesprochen hast\textless sup title=``2.Mose
19,5''\textgreater✲, als du unsere Väter aus Ägypten wegführtest, HERR,
unser Gott!«

\hypertarget{e-salomos-segnendes-und-mahnendes-schluuxdfwort}{%
\paragraph{e) Salomos segnendes und mahnendes
Schlußwort}\label{e-salomos-segnendes-und-mahnendes-schluuxdfwort}}

\bibleverse{54}Als nun Salomo mit diesem ganzen Gebet und Flehen, das er
an den HERRN gerichtet hatte, zu Ende war, erhob er sich von dem Platze
vor dem Altar des HERRN, wo er mit zum Himmel ausgebreiteten Händen auf
den Knien gelegen hatte; \bibleverse{55}er trat dann hin und segnete die
ganze Volksgemeinde Israel, indem er mit lauter Stimme ausrief:
\bibleverse{56}»Gepriesen sei der HERR, der seinem Volk Israel Ruhe
verschafft hat, ganz wie er es verheißen hat! Von all seinen herrlichen
Verheißungen, die er durch den Mund seines Knechtes Mose gegeben hat,
ist keine einzige unerfüllt geblieben. \bibleverse{57}Der HERR, unser
Gott, sei mit uns, wie er mit unsern Vätern gewesen ist! Er verlasse uns
nicht und verwerfe uns nicht, \bibleverse{58}sondern lasse unsere Herzen
auf ihn gerichtet sein, damit wir allezeit auf seinen Wegen wandeln und
seine Gebote, Satzungen und Rechte beobachten, zu denen er unsere Väter
verpflichtet hat! \bibleverse{59}Und diese meine Worte, mit denen ich
den HERRN angefleht habe, mögen dem HERRN, unserm Gott, bei Tag und bei
Nacht gegenwärtig sein, auf daß er seinem Knecht und seinem Volk Israel
so, wie jeder Tag es erfordert, Recht schaffe, \bibleverse{60}damit alle
Völker der Erde erkennen, daß der HERR Gott ist und sonst keiner.
\bibleverse{61}Euer Herz aber möge dem HERRN, unserm Gott, ungeteilt
ergeben sein, daß ihr nach seinen Satzungen wandelt und seine Gebote so
haltet, wie es heute der Fall ist!«

\hypertarget{f-abschluuxdf-der-feierlichkeiten-durch-ein-grouxdfes-opferfest}{%
\paragraph{f) Abschluß der Feierlichkeiten durch ein großes
Opferfest}\label{f-abschluuxdf-der-feierlichkeiten-durch-ein-grouxdfes-opferfest}}

\bibleverse{62}Hierauf brachten der König und ganz Israel mit ihm
Schlachtopfer vor dem HERRN dar, \bibleverse{63}und zwar ließ Salomo als
Heilsopfer, das er dem HERRN darbrachte, 22000 Rinder und 120000 Stück
Kleinvieh schlachten: so weihten der König und alle Israeliten den
Tempel des HERRN ein. \bibleverse{64}An jenem Tage weihte der König den
mittleren Teil des Vorhofes, der vor dem Tempel des HERRN liegt, zur
Opferstätte; denn er brachte dort die Brandopfer, die Speisopfer und die
Fettstücke der Heilsopfer dar, weil der eherne Altar, der vor dem Tempel
des HERRN steht, zu klein war, um die Brand- und Speisopfer und die
Fettstücke der Heilsopfer zu fassen.

\bibleverse{65}Auch beging Salomo damals das (Laubhütten-) Fest und ganz
Israel mit ihm -- eine gewaltige Festgemeinde, die zusammengekommen war
von der Gegend bei Hamath an bis an den Bach Ägyptens -- vor dem HERRN,
unserm Gott, sieben Tage lang {[}und noch einmal sieben Tage, im ganzen
vierzehn Tage lang{]}. \bibleverse{66}Am achten Tage aber entließ er das
Volk; sie nahmen Abschied vom König und kehrten zu ihren
Zelten\textless sup title=``=~in ihre Heimat''\textgreater✲ zurück,
fröhlich und wohlgemut wegen all des Guten, mit dem der HERR seinen
Knecht David und sein Volk Israel gesegnet hatte.

\hypertarget{gottes-abermalige-erscheinung-und-seine-antwort-auf-salomos-gebet}{%
\subsubsection{12. Gottes abermalige Erscheinung und seine Antwort auf
Salomos
Gebet}\label{gottes-abermalige-erscheinung-und-seine-antwort-auf-salomos-gebet}}

\hypertarget{section-8}{%
\section{9}\label{section-8}}

\bibleverse{1}Als nun Salomo den Bau des Tempels des HERRN und des
königlichen Palastes vollendet hatte und mit allem, was er sonst noch
auszuführen gewünscht hatte, fertig war, \bibleverse{2}da erschien ihm
der HERR zum zweitenmal, wie er ihm vorher in Gibeon erschienen war;
\bibleverse{3}und der HERR sagte zu ihm: »Ich habe dein Gebet und dein
Flehen gehört, das du an mich gerichtet hast. Ich habe diesen Tempel,
den du erbaut hast, dazu geweiht, meinen Namen für alle Zeiten daran
haften\textless sup title=``oder: dort wohnen''\textgreater✲ zu lassen,
und meine Augen und mein Herz sollen immerdar dort zugegen sein.
\bibleverse{4}Wenn du nun vor mir ebenso wandelst, wie dein Vater David
es getan hat, in Herzenseinfalt und Aufrichtigkeit, so daß du alles
tust, was ich dir geboten habe, und meine Satzungen und Rechte
beobachtest, \bibleverse{5}so will ich den Thron deines Königtums über
Israel auf ewige Zeiten bestätigen, wie ich es deinem Vater David
feierlich zugesagt habe mit den Worten: ›Es soll dir nie an einem Manne✲
auf dem Throne Israels fehlen!‹ \bibleverse{6}Wenn ihr aber von mir
abfallt, ihr oder eure Kinder, und meine Gebote und Satzungen, die ich
euch zur Pflicht gemacht habe, nicht beobachtet, sondern anderen Göttern
zu dienen und sie anzubeten anfangt, \bibleverse{7}so werde ich Israel
aus dem Lande, das ich ihnen gegeben habe, ausrotten und den Tempel, den
ich meinem Namen geheiligt habe, keines Blickes mehr würdigen, und
Israel soll für alle Völker ein Gegenstand des Hohns und Spottes werden.
\bibleverse{8}Und dieser Tempel soll zu einem Trümmerhaufen werden, so
daß alle, die an ihm vorübergehen, sich entsetzen und zischeln; und wenn
man dann fragt: ›Warum hat der HERR diesem Land und diesem Hause solches
Geschick widerfahren lassen?‹, \bibleverse{9}so wird man antworten: ›Zur
Strafe dafür, daß sie den HERRN, ihren Gott, der ihre Väter aus Ägypten
herausgeführt hatte, verlassen und sich anderen Göttern zugewandt und
sie angebetet und ihnen gedient haben; darum hat der HERR all dieses
Unglück über sie kommen lassen.‹«

\hypertarget{allerlei-auf-salomo-bezuxfcgliche-angaben}{%
\subsubsection{13. Allerlei auf Salomo bezügliche
Angaben}\label{allerlei-auf-salomo-bezuxfcgliche-angaben}}

\hypertarget{a-landabtretung-an-hiram-als-gegenleistung}{%
\paragraph{a) Landabtretung an Hiram als
Gegenleistung}\label{a-landabtretung-an-hiram-als-gegenleistung}}

\bibleverse{10}Nach Ablauf der zwanzig Jahre nun, während deren Salomo
die beiden Bauwerke, den Tempel des HERRN und den königlichen Palast,
erbaut hatte~-- \bibleverse{11}Hiram, der König von Tyrus, hatte nämlich
dem König Salomo Zedern- und Zypressenholz sowie Gold geliefert, soviel
er gewünscht hatte --, damals schenkte der König Salomo dem Hiram
zwanzig Städte in der Landschaft Galiläa. \bibleverse{12}Als aber Hiram
aus Tyrus herüberkam, um sich die Städte anzusehen, die Salomo ihm
überwiesen hatte, gefielen sie ihm nicht, \bibleverse{13}und er sagte:
»Was sind das für Städte, die du mir da abgetreten hast, mein Bruder!«
Daher nennt man diese Landschaft »Kabul«\textless sup title=``d.h. so
gut wie nichts''\textgreater✲ bis auf den heutigen Tag.
\bibleverse{14}Hiram hatte nämlich dem Könige hundertundzwanzig Talente
Gold gesandt.

\hypertarget{b-von-salomos-fronarbeitern-festungsbauten-vorratsstuxe4dten-regelmuxe4uxdfigen-opfern-u.a.}{%
\paragraph{b) Von Salomos Fronarbeitern, Festungsbauten, Vorratsstädten,
regelmäßigen Opfern
u.a.}\label{b-von-salomos-fronarbeitern-festungsbauten-vorratsstuxe4dten-regelmuxe4uxdfigen-opfern-u.a.}}

\bibleverse{15}Folgendermaßen aber verhielt es sich mit den
Fronarbeitern, die der König Salomo ausgehoben hatte, um den Tempel des
HERRN und seinen eigenen Palast sowie die Burg Millo und die Mauer
Jerusalems zu bauen und um Hazor, Megiddo und Geser zu befestigen:
\bibleverse{16}Der Pharao nämlich, der König von Ägypten, war
heraufgezogen, hatte Geser erobert und eingeäschert und die Kanaanäer,
die in der Stadt wohnten, niedergemacht und dann (den Ort) seiner
Tochter, der Gemahlin Salomos, als Mitgift geschenkt;
\bibleverse{17}hierauf hatte Salomo Geser wieder aufgebaut, ebenso
(baute er) auch das untere Beth-Horon, \bibleverse{18}Baalath, Thamar in
der Steppe im Lande Juda, \bibleverse{19}dazu alle Vorratsstädte, die
Salomo besaß, und die Ortschaften für die Kriegswagen und die Reitpferde
und überhaupt alle Bauten, die Salomo in Jerusalem, auf dem Libanon und
im ganzen Bereich seines Königreiches auszuführen wünschte.
\bibleverse{20}Alles, was noch an Nachkommen von den Amoritern,
Hethitern, Pherissitern, Hewitern und Jebusitern vorhanden war, die
nicht zu den Israeliten gehörten~-- \bibleverse{21}deren Nachkommen,
soweit sie im Lande noch übriggeblieben waren, weil die Israeliten den
Blutbann an ihnen nicht hatten vollstrecken können, die hob Salomo zum
Frondienst aus, und sie sind Fronarbeiter geblieben bis auf den heutigen
Tag. \bibleverse{22}Von den Israeliten dagegen machte Salomo keinen zum
Leibeigenen, sondern diese dienten ihm als Krieger und Hofbeamte, als
hohe und niedere Offiziere und als Befehlshaber über seine Kriegswagen
und seine Reiterei. \bibleverse{23}Die Zahl der Oberaufseher, die bei
den Arbeiten Salomos beschäftigt waren, belief sich auf 550; sie hatten
die bei den Arbeiten beschäftigten Leute zu beaufsichtigen.

\bibleverse{24}Sobald die Tochter des Pharaos aus der Davidstadt in
ihren eigenen Palast eingezogen war, den Salomo für sie hatte bauen
lassen, machte er sich an den Bau der Burg Millo.~--
\bibleverse{25}Dreimal jährlich pflegte Salomo Brand- und Heilsopfer auf
dem Altar darzubringen, den er dem HERRN erbaut hatte, und ebenso auf
dem (Altar) zu räuchern, der vor dem HERRN stand.

\hypertarget{c-salomos-flottenbau-und-schiffahrt}{%
\paragraph{c) Salomos Flottenbau und
Schiffahrt}\label{c-salomos-flottenbau-und-schiffahrt}}

Als er den Häuserbau vollendet hatte, \bibleverse{26}schuf König Salomo
auch eine Flotte in Ezjon-Geber, das bei Elath am Ufer des Schilfmeeres
im Lande der Edomiter liegt. \bibleverse{27}Hiram sandte dann auf dieser
Flotte als Bemannung seine Leute -- Seeleute, die mit dem Meer vertraut
waren -- zusammen mit den Leuten Salomos aus. \bibleverse{28}Sie fuhren
bis nach Ophir und holten von dort Gold, 420 Talente, und brachten es
dem König Salomo.

\hypertarget{d-besuch-der-kuxf6nigin-von-saba}{%
\paragraph{d) Besuch der Königin von
Saba}\label{d-besuch-der-kuxf6nigin-von-saba}}

\hypertarget{section-9}{%
\section{10}\label{section-9}}

\bibleverse{1}Als aber die Königin von Saba den Ruhm Salomos vernahm und
von dem Tempel hörte, den er dem Namen des HERRN erbaut hatte, kam sie,
um ihn mit Rätselfragen auf die Probe zu stellen. \bibleverse{2}Sie kam
also nach Jerusalem mit einem sehr großen Gefolge und mit Kamelen,
welche Spezereien✲ und Gold in sehr großer Menge und Edelsteine trugen.
Als sie nun bei Salomo angekommen war, trug sie ihm alles vor, was sie
sich vorgenommen hatte. \bibleverse{3}Salomo aber wußte ihr auf alle
Fragen Antwort zu geben, und nichts war dem Könige verborgen, daß er ihr
nicht hätte Auskunft geben können. \bibleverse{4}Als nun die Königin von
Saba sich von der allseitigen Weisheit Salomos überzeugt hatte, dazu den
Palast sah, den er erbaut hatte, \bibleverse{5}und die Speisen auf
seiner Tafel und wie seine Hofleute dasaßen, ferner die Aufwartung
seiner Dienerschaft und ihre Tracht, seine Mundschenken sowie seine
Brandopfer, die er im Tempel des HERRN darzubringen pflegte, da geriet
sie vor Erstaunen außer sich \bibleverse{6}und sagte zum König: »Wahr
ist das gewesen, was ich in meiner Heimat über dich und deine Weisheit
gehört habe. \bibleverse{7}Ich wollte dem, was man mir erzählte, nicht
glauben, bis ich jetzt hergekommen bin und mich mit eigenen Augen
überzeugt habe. Und dabei hat man mir noch nicht einmal die Hälfte
berichtet: deine Weisheit und deine Vorzüge übertreffen noch das
Gerücht, das ich vernommen habe. \bibleverse{8}Beneidenswert sind deine
Leute, beneidenswert diese deine Diener, die beständig um dich sind und
deine Weisheit hören können! \bibleverse{9}Gepriesen sei der HERR, dein
Gott, der Wohlgefallen an dir gehabt hat, so daß er dich auf den Thron
Israels gesetzt hat! Weil der HERR Israel allezeit liebt, darum hat er
dich zum König bestellt, damit du Recht und Gerechtigkeit übest.«
\bibleverse{10}Hierauf schenkte sie dem König hundertundzwanzig Talente
Gold und Spezereien✲ in sehr großer Menge, sowie Edelsteine; niemals
wieder ist eine solche Fülle von Spezereien ins Land gekommen, wie die
Königin von Saba sie damals dem König Salomo schenkte.
\bibleverse{11}Allerdings brachten auch die Schiffe Hirams, die Gold aus
Ophir geholt hatten, Sandelholz in sehr großer Menge und Edelsteine aus
Ophir mit; \bibleverse{12}und der König ließ aus dem Sandelholz Geländer
für den Tempel des HERRN und für den königlichen Palast herstellen sowie
Zithern und Harfen für die Sänger. In solcher Menge ist Sandelholz nie
wieder ins Land gekommen und dort zu sehen gewesen bis auf den heutigen
Tag. \bibleverse{13}Der König Salomo aber schenkte der Königin von Saba
alles, wonach sie Verlangen trug und was sie sich wünschte, abgesehen
von den Geschenken, die er ihr aus freien Stücken mit königlicher
Freigebigkeit gab. Hierauf trat sie mit ihrem Gefolge den Rückweg an und
zog heim.

\hypertarget{e-verschiedene-angaben-uxfcber-salomos-einkuxfcnfte-kostbarkeiten-ansehen-und-macht-pferdehandel}{%
\paragraph{e) Verschiedene Angaben über Salomos Einkünfte,
Kostbarkeiten, Ansehen und Macht,
Pferdehandel}\label{e-verschiedene-angaben-uxfcber-salomos-einkuxfcnfte-kostbarkeiten-ansehen-und-macht-pferdehandel}}

\bibleverse{14}Das Gewicht des Goldes, das für Salomo in einem einzigen
Jahre einging, betrug 666 Talente Gold, \bibleverse{15}ungerechnet die
Abgaben der reisenden Großkaufleute und die Steuern der Kleinhändler
sowie die Tribute aller Könige Arabiens\textless sup title=``=~der
Beduinen''\textgreater✲ und was von den Statthaltern des Landes
einkam.~-- \bibleverse{16}Der König Salomo ließ auch zweihundert
Langschilde von getriebenem Gold anfertigen: sechshundert Schekel Gold
verwandte er auf jeden Schild; \bibleverse{17}ferner dreihundert
Kleinschilde von getriebenem Gold; drei Minen Gold verwandte er auf
jeden Schild. Der König brachte sie dann im Libanonwaldhause unter.~--
\bibleverse{18}Weiter ließ der König einen großen Thron von Elfenbein
anfertigen und ihn mit feinem Gold überziehen. \bibleverse{19}Der Thron
hatte sechs Stufen, und das Kopfstück des Thrones hinten war gerundet;
auf beiden Seiten des Sitzplatzes befanden sich Armlehnen, und neben den
Armlehnen standen zwei Löwen; \bibleverse{20}außerdem standen zwölf
Löwen auf den sechs Stufen zu beiden Seiten: ein derartiges Kunstwerk
ist noch nie für ein Königreich hergestellt worden.~--
\bibleverse{21}Alle Trinkgefäße des Königs Salomo bestanden aus Gold;
auch alle Geräte im Libanonwaldhause waren von feinem Gold, nichts von
Silber, das man zu Salomos Lebzeiten für wertlos ansah.
\bibleverse{22}Denn der König hatte Tharsisschiffe, die mit den Schiffen
Hirams über Meer fuhren; alle drei Jahre einmal kamen die Tharsisschiffe
heim und brachten Gold und Silber, Elfenbein, Affen und Pfauen mit.

\bibleverse{23}So übertraf denn der König Salomo alle Könige der Erde an
Reichtum und an Weisheit; \bibleverse{24}und alle Welt suchte Salomo zu
sehen, um sich persönlich von seiner Weisheit zu überzeugen, die Gott
ihm ins Herz gelegt hatte. \bibleverse{25}Dabei brachte jeder von ihnen
ein entsprechendes Geschenk mit: silberne und goldene Kunstwerke,
Gewänder, Waffen und Spezereien✲, Rosse und Maultiere, Jahr für Jahr.~--
\bibleverse{26}Salomo brachte auch zahlreiche Kriegswagen und Reitpferde
zusammen, so daß er 1400 Wagen und 12000 Reitpferde besaß, die er in den
Wagenstädten oder in seiner Nähe zu Jerusalem unterbrachte.
\bibleverse{27}Und der König brachte es dahin, daß es in Jerusalem
soviel Silber gab wie Steine und daß die Zedernstämme an Menge den
Maulbeerfeigenbäumen in der Niederung gleichkamen.~-- \bibleverse{28}Der
Bezug der Pferde für Salomo erfolgte aus Ägypten, und zwar aus Koa; die
Händler des Königs kauften sie dort in Koa auf, \bibleverse{29}so daß
ein Wagen bei der Ausfuhr aus Ägypten auf sechshundert Schekel Silber zu
stehen kam und ein Pferd auf einhundertundfünfzig. Auf gleiche Weise
wurden sie durch Vermittelung (seiner Händler) für alle Könige der
Hethiter und für die Könige von Syrien bezogen\textless sup title=``vgl.
2.Chr 1,16-17''\textgreater✲.

\hypertarget{salomos-versuxfcndigungen-seine-widersacher-und-sein-unruxfchmlicher-ausgang}{%
\subsubsection{14. Salomos Versündigungen; seine Widersacher und sein
unrühmlicher
Ausgang}\label{salomos-versuxfcndigungen-seine-widersacher-und-sein-unruxfchmlicher-ausgang}}

\hypertarget{a-salomos-vielweiberei-und-abguxf6tterei-gottes-drohung}{%
\paragraph{a) Salomos Vielweiberei und Abgötterei; Gottes
Drohung}\label{a-salomos-vielweiberei-und-abguxf6tterei-gottes-drohung}}

\hypertarget{section-10}{%
\section{11}\label{section-10}}

\bibleverse{1}Der König Salomo liebte aber zahlreiche ausländische
Frauen, nämlich neben der Tochter des Pharaos auch moabitische,
ammonitische, edomitische, phönizische und hethitische,
\bibleverse{2}also Frauen aus solchen Völkern, bezüglich deren der HERR
den Israeliten geboten hatte\textless sup title=``2.Mose 34,15-16;
5.Mose 7,3''\textgreater✲: »Ihr sollt keinen ehelichen Verkehr mit ihnen
haben, und sie sollen nicht mit euch verkehren, sie würden sonst
sicherlich eure Herzen ihren Göttern zuwenden!« An diesen Frauen hing
Salomo mit Vorliebe; \bibleverse{3}und zwar hatte er siebenhundert
fürstliche Frauen und dreihundert Nebenweiber, und seine Frauen zogen
sein Herz von dem HERRN ab. \bibleverse{4}Als Salomo nämlich alt
geworden war, wandten seine Frauen sein Herz anderen Göttern zu, so daß
sein Herz dem HERRN, seinem Gott, nicht mehr ungeteilt ergeben war wie
das Herz seines Vaters David. \bibleverse{5}So verehrte er z.B. die
phönizische Göttin Astarte und den greulichen Götzen der Ammoniter,
Milkom; \bibleverse{6}und Salomo tat so, was dem HERRN mißfiel, indem er
dem HERRN nicht volle Hingabe bewies wie sein Vater David.
\bibleverse{7}Damals\textless sup title=``d.h. im Alter''\textgreater✲
baute Salomo für Kamos, den Götzen der Moabiter, ein Höhenheiligtum auf
dem Berge östlich von Jerusalem, und ebenso für Moloch, den Götzen der
Ammoniter; \bibleverse{8}und dasselbe tat er für alle seine
ausländischen Frauen, die ihren Göttern Rauch- und Schlachtopfer
darbrachten.

\bibleverse{9}So wurde denn der HERR zornig auf Salomo, weil er sein
Herz vom HERRN, dem Gott Israels, abgewandt hatte, der ihm doch zweimal
erschienen war \bibleverse{10}und ihm gerade dieses Gebot gegeben hatte,
keine fremden Götter zu verehren; trotzdem hatte er dieses Gebot des
HERRN unbeachtet gelassen. \bibleverse{11}Darum sagte der HERR zu
Salomo: »Weil es soweit mit dir gekommen ist, daß du meinen Bund und
meine Satzungen, die ich dir zur Pflicht gemacht habe, nicht mehr
beachtest, so will ich dir das Königtum entreißen und es einem deiner
Knechte geben. \bibleverse{12}Doch will ich es noch nicht bei deinen
Lebzeiten tun um deines Vaters David willen; erst deinem Sohne will ich
es entreißen. \bibleverse{13}Doch will ich ihm nicht das ganze Reich
entreißen; nein, einen Stamm will ich deinem Sohne geben um meines
Knechtes David willen und um Jerusalems willen, das ich erwählt habe.«

\hypertarget{b-salomos-uxe4uuxdfere-feinde-der-edomiter-hadad-und-der-syrer-reson}{%
\paragraph{b) Salomos äußere Feinde (der Edomiter Hadad und der Syrer
Reson)}\label{b-salomos-uxe4uuxdfere-feinde-der-edomiter-hadad-und-der-syrer-reson}}

\bibleverse{14}So ließ denn der HERR dem Salomo einen Widersacher
erstehen in dem Edomiter Hadad, der aus dem Königsgeschlecht in Edom
stammte. \bibleverse{15}Als David nämlich die Edomiter besiegt hatte und
der Feldhauptmann Joab hingezogen war, um die gefallenen Israeliten zu
begraben, und er dabei alles, was männlichen Geschlechts in Edom war,
niedermetzeln ließ~-- \bibleverse{16}denn ein halbes Jahr lang war Joab
mit dem ganzen Heer der Israeliten dort geblieben, bis er alles, was
männlichen Geschlechts in Edom war, ausgerottet hatte --,
\bibleverse{17}da entfloh Hadad mit mehreren Edomitern, die zu den
Dienern seines Vaters gehört hatten, um sich nach Ägypten zu begeben;
Hadad war aber damals noch ein kleiner Knabe\textless sup title=``oder:
junger Mann''\textgreater✲. \bibleverse{18}Sie machten sich also aus
Midian auf und gelangten nach Paran; aus Paran nahmen sie dann Leute mit
sich und kamen so nach Ägypten zum Pharao, dem ägyptischen Könige.
Dieser gab ihm ein Haus und wies ihm seinen Unterhalt an und schenkte
ihm auch ein Landgut. \bibleverse{19}Hadad gewann dann die Gunst des
Pharaos in hohem Grade, so daß er ihm die Schwester seiner Gemahlin, die
Schwester der Thachpenes, zur Frau gab. \bibleverse{20}Diese Schwester
der Thachpenes gebar ihm dann seinen Sohn Genubath, den sie im Palast
des Pharaos erziehen ließ; und fortan lebte Genubath im Palast des
Pharaos unter den Söhnen des Königs. \bibleverse{21}Als Hadad dann in
Ägypten erfuhr, daß David sich zu seinen Vätern gelegt habe und daß auch
der Feldhauptmann Joab tot sei, bat Hadad den Pharao: »Laß mich gehen,
daß ich in meine Heimat ziehe!« \bibleverse{22}Als ihn der Pharao
fragte: »Was fehlt dir denn bei mir, daß du durchaus in deine Heimat
ziehen willst?«, antwortete er: »Nichts! Doch du mußt mich ziehen
lassen!« \bibleverse{23}Und Gott ließ dem Salomo noch einen andern
Widersacher erstehen, nämlich Reson, den Sohn Eljadas, der aus der
Umgebung Hadad-Esers, seines Herrn, des Königs von Zoba, entflohen war.
\bibleverse{24}Dieser sammelte Leute um sich und wurde der Führer einer
Freibeuterschar damals, als David das Blutbad unter den Syrern
anrichtete\textless sup title=``vgl. 2.Sam 8,3-4''\textgreater✲. Er zog
dann nach Damaskus, setzte sich dort fest, machte sich zum König in
Damaskus \bibleverse{25}a und war ein Widersacher Israels, solange
Salomo lebte.

\bibleverse{25}a und war ein Widersacher Israels, solange Salomo lebte.

\hypertarget{c-die-empuxf6rung-des-ephraimiten-jerobeam}{%
\paragraph{c) Die Empörung des Ephraimiten
Jerobeam}\label{c-die-empuxf6rung-des-ephraimiten-jerobeam}}

\bibleverse{26}Auch Jerobeam, der Sohn Nebats, ein Ephraimit aus Zereda,
der Sohn einer Witwe namens Zerua, ein Beamter Salomos, empörte sich
gegen den König; \bibleverse{27}und zwar war er folgendermaßen dazu
gekommen, sich gegen den König zu empören: Salomo baute die Burg Millo
aus, um die Lücke zu schließen, die an der Davidsstadt seines Vaters
noch geblieben war. \bibleverse{28}Nun war jener Jerobeam ein tüchtiger
Mann, und als Salomo sah, wie eifrig der junge Mann bei der Arbeit war,
machte er ihn zum Aufseher über die gesamten Fronarbeiten des Hauses
Joseph. \bibleverse{29}Damals begab es sich nun, als Jerobeam eines
Tages Jerusalem verlassen hatte, daß ihn der Prophet Ahia von Silo
unterwegs traf, der gerade einen neuen Mantel umhatte; und sie beide
waren allein auf freiem Felde. \bibleverse{30}Da nahm Ahia den neuen
Mantel, den er anhatte, zerriß ihn in zwölf Stücke \bibleverse{31}und
sagte zu Jerobeam: »Nimm dir zehn Stücke davon; denn so hat der HERR,
der Gott Israels, gesprochen: ›Siehe, ich will das Reich der Hand
Salomos entreißen und will dir zehn Stämme geben~-- \bibleverse{32}aber
den einen Stamm soll er behalten um meines Knechtes David willen und um
Jerusalems willen, der Stadt, die ich aus allen Stämmen Israels erwählt
habe --, \bibleverse{33}zur Strafe dafür, daß er mich verlassen und sich
vor Astarte, der Gottheit der Phönizier, vor Kamos, dem Gott der
Moabiter, und vor Milkom, dem Gott der Ammoniter, niedergeworfen hat und
nicht auf meinen Wegen gewandelt ist, um das zu tun, was mir
wohlgefällt, und meine Satzungen und Rechte zu beobachten, wie sein
Vater David es getan hat. \bibleverse{34}Doch will ich nicht ihm selbst
die ganze Königsmacht entreißen, sondern will ihm, solange er lebt, die
Herrschaft lassen um meines Knechtes David willen, den ich erwählt habe
und der meine Gebote und Satzungen beobachtet hat. \bibleverse{35}Aber
seinem Sohne will ich das Königtum nehmen und es dir geben, nämlich zehn
Stämme; \bibleverse{36}seinem Sohne dagegen will ich nur einen einzigen
Stamm geben, damit meinem Knecht David allezeit eine
Leuchte\textless sup title=``oder: ein Fortleuchten''\textgreater✲ vor
mir verbleibe in Jerusalem, der Stadt, die ich mir erwählt habe, um
meinen Namen daselbst dauernd wohnen zu lassen. \bibleverse{37}Dich aber
will ich nehmen, damit du herrschest über alles, wonach du Verlangen
trägst, und König über Israel seiest. \bibleverse{38}Wenn du dann allen
meinen Befehlen nachkommst und auf meinen Wegen wandelst und das tust,
was mir wohlgefällt, indem du meine Satzungen und meine Gebote
beobachtest, wie mein Knecht David es getan hat, so will ich mit dir
sein und dir ein Haus bauen, das Bestand haben soll, wie ich David ein
solches gebaut habe, und will dir Israel übergeben; \bibleverse{39}die
Nachkommenschaft Davids aber will ich um deswillen demütigen, doch nicht
für alle Zeiten.‹«~-- \bibleverse{40}Als Salomo dann dem Jerobeam nach
dem Leben trachtete, machte dieser sich auf und floh nach Ägypten zu
Sisak, dem König von Ägypten, und blieb in Ägypten bis zu Salomos Tode.

\hypertarget{d-die-quellen-der-geschichte-salomos-sein-tod}{%
\paragraph{d) Die Quellen der Geschichte Salomos; sein
Tod}\label{d-die-quellen-der-geschichte-salomos-sein-tod}}

\bibleverse{41}Die übrige Geschichte Salomos aber und alle seine Taten
und seine Weisheit, das findet sich bekanntlich schon aufgeschrieben im
Buch der Denkwürdigkeiten Salomos. \bibleverse{42}Die Zeit aber, während
welcher Salomo in Jerusalem über ganz Israel geherrscht hat, betrug
vierzig Jahre. \bibleverse{43}Dann legte Salomo sich zu seinen Vätern
und wurde in der Stadt Davids, seines Vaters, begraben. Sein Sohn
Rehabeam folgte ihm in der Regierung nach.

\hypertarget{ii.-geschichte-der-getrennten-reiche-juda-und-israel-bis-josaphat-von-juda-und-ahasja-von-israel-kap.-12-22}{%
\subsection{II. Geschichte der getrennten Reiche Juda und Israel bis
Josaphat von Juda und Ahasja von Israel (Kap.
12-22)}\label{ii.-geschichte-der-getrennten-reiche-juda-und-israel-bis-josaphat-von-juda-und-ahasja-von-israel-kap.-12-22}}

\hypertarget{von-der-teilung-des-reiches-bis-zum-regierungsantritt-ahabs-von-israel-kap.-12-16}{%
\subsubsection{1. Von der Teilung des Reiches bis zum Regierungsantritt
Ahabs von Israel (Kap.
12-16)}\label{von-der-teilung-des-reiches-bis-zum-regierungsantritt-ahabs-von-israel-kap.-12-16}}

\hypertarget{a-die-spaltung-des-reiches}{%
\paragraph{a) Die Spaltung des
Reiches}\label{a-die-spaltung-des-reiches}}

\hypertarget{der-reichstag-zu-sichem-bitte-der-israeliten-um-erleichterung}{%
\paragraph{Der Reichstag zu Sichem; Bitte der Israeliten um
Erleichterung}\label{der-reichstag-zu-sichem-bitte-der-israeliten-um-erleichterung}}

\hypertarget{section-11}{%
\section{12}\label{section-11}}

\bibleverse{1}Rehabeam begab sich nach Sichem; denn in Sichem hatten
sich alle Israeliten eingefunden, um ihn zum König zu machen.
\bibleverse{2}Sobald nun Jerobeam, der Sohn Nebats, Kunde davon erhielt
-- er befand sich nämlich noch in Ägypten, wohin er vor dem König Salomo
geflohen war --, da kehrte Jerobeam aus Ägypten zurück;
\bibleverse{3}sie hatten nämlich hingesandt und ihn rufen lassen. So
kamen denn Jerobeam und die ganze Volksgemeinde Israels (nach Sichem)
und trugen dem Rehabeam folgendes vor: \bibleverse{4}»Dein Vater hat uns
ein hartes✲ Joch auferlegt; so erleichtere du uns jetzt deines Vaters
harten Dienst und das schwere Joch, das er uns aufgelegt hat, so wollen
wir dir untertan sein.« \bibleverse{5}Er antwortete ihnen: »Geduldet
euch noch drei Tage, dann kommt wieder zu mir!«

\hypertarget{die-beratung-rehabeams}{%
\paragraph{Die Beratung Rehabeams}\label{die-beratung-rehabeams}}

Nachdem sich nun das Volk entfernt hatte, \bibleverse{6}beriet sich der
König Rehabeam mit den alten Räten, die seinem Vater Salomo während
dessen Lebzeiten gedient hatten, und fragte sie: »Welche Antwort ratet
ihr mir diesen Leuten zu erteilen?« \bibleverse{7}Sie gaben ihm folgende
Antwort: »Wenn du heute diesen Leuten zu Willen bist, dich nachgiebig
zeigst und auf sie hörst und ihnen eine freundliche Antwort gibst, so
werden sie dir stets gehorsame Untertanen sein.« \bibleverse{8}Aber er
ließ den Rat, den ihm die Alten gegeben hatten, unbeachtet und beriet
sich mit den jungen Männern, die mit ihm aufgewachsen waren und jetzt in
seinen Diensten standen. \bibleverse{9}Er fragte sie: »Welche Antwort
müssen wir nach eurer Ansicht diesen Leuten geben, die von mir eine
Erleichterung des Joches verlangen, das mein Vater ihnen aufgelegt hat?«
\bibleverse{10}Da gaben ihm die jungen Männer, die mit ihm aufgewachsen
waren, folgende Antwort: »So mußt du diesen Leuten antworten, die von
dir eine Erleichterung des schweren Joches verlangen, das dein Vater
ihnen aufgelegt hat -- so mußt du ihnen antworten: ›Mein kleiner Finger
ist dicker als meines Vaters Lenden. \bibleverse{11}Und nun: hat mein
Vater euch ein schweres Joch aufgeladen, so will ich euer Joch noch
schwerer machen; hat mein Vater euch mit Peitschen gezüchtigt, so will
ich euch mit Skorpionen✲ züchtigen.‹«

\hypertarget{abfall-der-zehn-stuxe4mme-jerobeams-wahl-zum-kuxf6nig-von-israel}{%
\paragraph{Abfall der zehn Stämme; Jerobeams Wahl zum König von
Israel}\label{abfall-der-zehn-stuxe4mme-jerobeams-wahl-zum-kuxf6nig-von-israel}}

\bibleverse{12}Als nun Jerobeam mit dem ganzen Volk am dritten Tage zu
Rehabeam kam, wie der König ihnen befohlen hatte mit den Worten: »Kommt
am dritten Tage wieder zu mir!«, \bibleverse{13}gab der König dem Volke
eine harte Antwort und ließ den Rat außer acht, den die Alten ihm
gegeben hatten; \bibleverse{14}er gab ihnen vielmehr nach dem Rat der
jungen Männer folgende Antwort: »Hat mein Vater euch ein schweres Joch
aufgelegt, so will ich euer Joch noch schwerer machen; hat mein Vater
euch mit Peitschen gezüchtigt, so will ich euch mit Skorpionen
züchtigen.« \bibleverse{15}So schenkte also der König dem Volke kein
Gehör; denn vom HERRN war es so gefügt worden, damit er seine Verheißung
in Erfüllung gehen ließe, die der HERR durch den Mund Ahias von Silo dem
Jerobeam, dem Sohne Nebats, gegeben hatte.

\bibleverse{16}Als nun ganz Israel sah, daß der König ihnen kein Gehör
schenkte, ließ das Volk dem König folgende Erklärung zugehen: »Was haben
wir mit David zu schaffen? Wir haben nichts gemein mit dem Sohne Isais.
Auf, ihr Israeliten, zu euren Zelten\textless sup title=``=~in eure
Heimat''\textgreater✲! Nun sorge (selbst) für dein Haus, David!« So
begaben sich denn die Israeliten in ihre Heimat, \bibleverse{17}so daß
Rehabeam nur über die Israeliten, die in den Ortschaften Judas wohnten,
König blieb. \bibleverse{18}Als dann der König Rehabeam
Adoram\textless sup title=``oder: Adoniram''\textgreater✲, den
Oberaufseher über die Fronarbeiten, hinsandte, warf ihn ganz Israel mit
Steinen zu Tode. Da hatte der König nichts Eiligeres zu tun, als seinen
Wagen zu besteigen, um (aus Sichem) nach Jerusalem zu fliehen.
\bibleverse{19}So fiel Israel vom Hause Davids ab bis auf den heutigen
Tag.

\bibleverse{20}Als nun ganz Israel vernahm, daß Jerobeam zurückgekehrt
sei, ließen sie ihn in die Gemeindeversammlung holen und machten ihn zum
König über ganz Israel; dem Hause Davids blieb nur der eine Stamm Juda
treu.

\hypertarget{rehabeam-steht-auf-gottes-weisung-vom-kriege-gegen-israel-ab}{%
\paragraph{Rehabeam steht auf Gottes Weisung vom Kriege gegen Israel
ab}\label{rehabeam-steht-auf-gottes-weisung-vom-kriege-gegen-israel-ab}}

\bibleverse{21}Als nun Rehabeam in Jerusalem angekommen war, bot er das
ganze Haus Juda und den Stamm Benjamin, 180000 auserlesene Krieger, zum
Kampf gegen das Haus Israel auf, um das Königtum für Rehabeam, den Sohn
Salomos, wiederzugewinnen. \bibleverse{22}Aber das Wort Gottes erging an
den Gottesmann Semaja also: \bibleverse{23}»Sage zu Rehabeam, dem Sohne
Salomos, dem König von Juda, und zum ganzen Hause Juda und Benjamin und
zu dem übrigen Volk: \bibleverse{24}So hat der HERR gesprochen: ›Ihr
sollt nicht hinziehen, um mit euren Brüdern, den Israeliten, Krieg zu
führen: kehrt allesamt nach Hause zurück! Denn von mir ist dieses alles
so gefügt worden.‹« Da gehorchten sie dem Befehl des HERRN und machten
sich auf den Heimweg, wie der HERR es geboten hatte.

\hypertarget{b-jerobeams-i.-regierung-und-abguxf6tterei}{%
\paragraph{b) Jerobeams I. Regierung und
Abgötterei}\label{b-jerobeams-i.-regierung-und-abguxf6tterei}}

\bibleverse{25}Jerobeam aber befestigte Sichem im Gebirge Ephraim und
machte es zu seiner Residenz; darauf zog er von dort aus und befestigte
Pnuel. \bibleverse{26}Er dachte aber bei sich: »Das Königtum wird nun
wohl an das Haus Davids zurückfallen. \bibleverse{27}Wenn nämlich das
Volk hier hinaufziehen muß, um im Tempel des HERRN zu Jerusalem Opfer
darzubringen, so wird das Herz des Volkes hier sich wieder dem König
Rehabeam von Juda als ihrem Herrn zuwenden; sie werden mich dann
umbringen und dem König Rehabeam von Juda wieder zufallen.«

\hypertarget{die-einfuxfchrung-des-stierdienstes-in-bethel-und-dan}{%
\paragraph{Die Einführung des Stierdienstes in Bethel und
Dan}\label{die-einfuxfchrung-des-stierdienstes-in-bethel-und-dan}}

\bibleverse{28}Als der König dann mit sich zu Rate gegangen war, ließ er
zwei goldene Stierbilder anfertigen und sagte zum Volk: »Ihr seid nun
lange genug nach Jerusalem hinaufgezogen. Seht, dies hier ist euer Gott,
Israeliten, der euch aus Ägypten hergeführt hat!« \bibleverse{29}Das
eine Stierbild stellte er dann in Bethel auf, das andere ließ er nach
Dan bringen. \bibleverse{30}Dies wurde aber eine Veranlassung zur Sünde;
und das Volk ging zu dem einen hin (nach Bethel und zu dem andern) nach
Dan. \bibleverse{31}Er richtete auch Opferstätten auf den Höhen ein und
bestellte zu Priestern beliebige Leute aus dem Volk, die nicht zu den
Leviten gehörten. \bibleverse{32}Ferner ordnete Jerobeam ein Fest an am
fünfzehnten Tage des achten Monats, das dem Laubhüttenfest in Juda
entsprach, und opferte selbst auf dem Altar. Ebenso machte er es zu
Bethel, um den Stierbildern zu opfern, die er hatte anfertigen lassen,
und er ließ in Bethel die Priester der Höhentempel, die er eingerichtet
hatte, den heiligen Dienst verrichten.

\hypertarget{c-drohung-eines-propheten-gegen-den-altar-zu-bethel-ungehorsam-und-tod-dieses-propheten}{%
\paragraph{c) Drohung eines Propheten gegen den Altar zu Bethel;
Ungehorsam und Tod dieses
Propheten}\label{c-drohung-eines-propheten-gegen-den-altar-zu-bethel-ungehorsam-und-tod-dieses-propheten}}

\hypertarget{aa-der-vorgang-am-altar-zu-bethel}{%
\subparagraph{aa) Der Vorgang am Altar zu
Bethel}\label{aa-der-vorgang-am-altar-zu-bethel}}

\bibleverse{33}Als er nun zu dem Altar hinaufgestiegen war, den er in
Bethel errichtet hatte, am fünfzehnten Tage des achten Monats, in
demselben Monat, den er willkürlich\textless sup title=``oder:
eigenmächtig''\textgreater✲ zu einem Festtag der Israeliten bestimmt
hatte, -- als er also zu dem Altar hinaufgestiegen war, um zu opfern,

\hypertarget{section-12}{%
\section{13}\label{section-12}}

\bibleverse{1}da kam plötzlich auf Geheiß\textless sup title=``oder: im
Auftrage''\textgreater✲ des HERRN ein Gottesmann aus Juda nach Bethel,
während Jerobeam gerade am Altar stand, um zu opfern.
\bibleverse{2}Dieser rief auf Geheiß des HERRN folgende Worte gegen den
Altar aus: »Altar! Altar! So hat der HERR gesprochen: ›Einst wird dem
Hause Davids ein Sohn geboren werden namens Josia; der wird auf dir die
Höhenpriester schlachten, die auf dir opfern, und Menschengebeine wird
man auf dir verbrennen!‹« \bibleverse{3}Gleichzeitig kündigte er ein
Wunderzeichen an mit den Worten: »Dies ist das Wahrzeichen dafür, daß
der HERR es ist, der jetzt durch mich geredet hat: Der Altar da wird
jetzt bersten und die Fettasche, die darauf liegt, verschüttet werden.«
\bibleverse{4}Sobald nun der König die Worte hörte, die der Gottesmann
gegen den Altar zu Bethel ausgerufen hatte, streckte Jerobeam seinen Arm
aus vom Altar herab und rief: »Nehmt ihn fest!« Aber sein Arm, den er
gegen ihn ausgestreckt hatte, erstarrte, so daß er ihn nicht wieder
zurückziehen konnte. \bibleverse{5}Der Altar aber barst, und die
Fettasche, die darauf lag, wurde vom Altar herab verschüttet, wie der
Gottesmann es als Wahrzeichen auf Geheiß des HERRN angekündigt hatte.
\bibleverse{6}Da richtete der König an den Gottesmann die Bitte:
»Besänftige doch den HERRN, deinen Gott, und bitte für mich, daß ich
meinen Arm wieder an mich ziehen kann!« Da legte der Gottesmann
Fürsprache beim HERRN ein, so daß der König seinen Arm wieder an sich
ziehen konnte und dieser wieder so wurde wie zuvor. \bibleverse{7}Darauf
sagte der König zu dem Gottesmann: »Komm mit mir ins Haus und erquicke
dich! Ich will dir auch ein Geschenk geben.« \bibleverse{8}Aber der
Gottesmann antwortete dem Könige: »Wenn du mir auch deinen halben Besitz
gäbest, so würde ich doch nicht mit dir gehen, würde auch an diesem Ort
weder Brot essen noch Wasser trinken. \bibleverse{9}Denn so ist mir
geboten worden durch das Wort des HERRN, das da lautete: ›Du darfst dort
weder Brot essen noch Wasser trinken, darfst auch nicht auf dem Wege
zurückkehren, auf dem du hingegangen bist!‹« \bibleverse{10}Darauf zog
er auf einem andern Wege von dannen und kehrte nicht auf dem Wege
zurück, auf dem er nach Bethel gekommen war.

\hypertarget{bb-der-vorgang-im-hause-des-alten-propheten-zu-bethel}{%
\subparagraph{bb) Der Vorgang im Hause des alten Propheten zu
Bethel}\label{bb-der-vorgang-im-hause-des-alten-propheten-zu-bethel}}

\bibleverse{11}Nun wohnte in Bethel ein alter Prophet; dessen Söhne
kamen heim und erzählten ihm alles, was der Gottesmann an jenem Tage in
Bethel getan, und die Worte, die er an den König gerichtet hatte. Als
sie dies ihrem Vater erzählt hatten, \bibleverse{12}fragte ihr Vater
sie, auf welchem Wege er weggegangen sei. Als seine Söhne ihm den Weg
bezeichnet hatten, den der Gottesmann, der aus Juda gekommen war,
eingeschlagen hatte, \bibleverse{13}befahl er seinen Söhnen: »Sattelt
mir den Esel!«, und als sie ihm den Esel gesattelt hatten, bestieg er
ihn \bibleverse{14}und ritt hinter dem Gottesmanne her. Er fand ihn
unter der Terebinthe sitzen und fragte ihn: »Bist du der Gottesmann, der
aus Juda gekommen ist?« Er antwortete: »Ja.« \bibleverse{15}Da bat er
ihn: »Komm mit mir nach Hause und nimm einen Imbiß bei mir ein!«
\bibleverse{16}Doch er entgegnete: »Ich darf nicht mit dir umkehren und
bei dir einkehren, werde auch an diesem Ort kein Brot essen noch Wasser
mit dir trinken; \bibleverse{17}denn durch das Wort des HERRN ist mir
der Befehl erteilt: ›Du darfst dort weder Brot essen noch Wasser
trinken, darfst auch nicht auf dem Wege zurückkehren, auf dem du
gekommen bist.‹« \bibleverse{18}Aber jener entgegnete ihm: »Auch ich bin
ein Prophet wie du, und ein Engel hat mir auf Geheiß des HERRN geboten:
›Bringe ihn mit dir in dein Haus zurück, auf daß er Brot esse und Wasser
trinke!‹« Damit sagte er ihm aber die Unwahrheit. \bibleverse{19}Da
kehrte er mit ihm um und aß Brot in seinem Hause und trank Wasser.

\hypertarget{cc-der-tod-und-die-bestattung-des-ungehorsamen-propheten}{%
\subparagraph{cc) Der Tod und die Bestattung des ungehorsamen
Propheten}\label{cc-der-tod-und-die-bestattung-des-ungehorsamen-propheten}}

\bibleverse{20}Während sie aber bei Tisch saßen, erging das Wort des
HERRN an den Propheten, der ihn zurückgeholt hatte; \bibleverse{21}und
er rief dem Gottesmanne, der aus Juda gekommen war, die Worte zu: »So
hat der HERR gesprochen: ›Zur Strafe dafür, daß du gegen den Befehl des
HERRN ungehorsam gewesen bist und auf das Gebot, das der HERR, dein
Gott, dir gegeben hat, nicht geachtet hast, \bibleverse{22}sondern
umgekehrt bist und Brot gegessen und Wasser getrunken hast an dem Ort,
an welchem der HERR dir zu essen und zu trinken verboten hatte, wird
deine Leiche nicht in das Grab deiner Väter kommen!‹«
\bibleverse{23}Nachdem er nun mit Essen und Trinken fertig war, sattelte
er sich den Esel des Propheten, der ihn zurückgeholt hatte.
\bibleverse{24}Als er dann weggeritten war, stieß unterwegs ein Löwe auf
ihn und tötete ihn, und sein Leichnam lag auf dem Wege hingestreckt,
während der Esel neben ihm stehenblieb und auch der Löwe neben dem Toten
stand. \bibleverse{25}Als nun Leute, die dort vorübergingen, den
Leichnam auf dem Wege hingestreckt und den Löwen neben dem Toten stehen
sahen, gingen sie hin und erzählten es in der Stadt, in welcher der alte
Prophet wohnte. \bibleverse{26}Sobald nun der Prophet, der ihn von der
Heimkehr zurückgeholt hatte, dies vernahm, sagte er: »Das ist der
Gottesmann, der gegen den Befehl des HERRN ungehorsam gewesen ist; darum
hat der HERR ihn dem Löwen preisgegeben; der hat ihn zerrissen und
getötet, wie der HERR es ihm vorher angekündigt hatte.«
\bibleverse{27}Sodann befahl er seinen Söhnen: »Sattelt mir den Esel!«,
und als sie ihn gesattelt hatten, \bibleverse{28}ritt er hin und fand
seine Leiche auf dem Wege hingestreckt, während der Esel und der Löwe
neben dem Toten standen: der Löwe hatte den Toten nicht gefressen und
den Esel nicht zerrissen. \bibleverse{29}Der Prophet hob nun die Leiche
des Gottesmannes auf, lud ihn auf den Esel und brachte ihn in den
Wohnort des alten Propheten zurück, um ihm eine Totenfeier zu
veranstalten und ihn zu begraben. \bibleverse{30}Er ließ aber die Leiche
in sein eigenes Grab legen, und man hielt ihm die Totenklage: »Ach, mein
Bruder!« \bibleverse{31}Nachdem er ihn aber begraben hatte, sagte er zu
seinen Söhnen: »Wenn ich einst gestorben bin, so begrabt mich in
demselben Grabe, in welchem der Gottesmann begraben liegt; neben seine
Gebeine legt auch die meinigen! \bibleverse{32}Denn die Drohung, die er
auf Befehl des HERRN gegen den Altar zu Bethel und gegen alle
Höhentempel in den Ortschaften Samarias ausgestoßen hat, wird sicherlich
in Erfüllung gehen.«\textless sup title=``vgl. 2.Kön
23,16-18''\textgreater✲

\hypertarget{dd-jerobeam-setzt-sein-suxfcndiges-tun-fort}{%
\subparagraph{dd) Jerobeam setzt sein sündiges Tun
fort}\label{dd-jerobeam-setzt-sein-suxfcndiges-tun-fort}}

\bibleverse{33}Auch nach dieser Begebenheit ließ Jerobeam von seinem
bösen Wandel nicht ab, sondern fuhr fort, alle beliebigen Leute zu
Höhenpriestern zu machen; wer nur immer Lust dazu hatte, den setzte er
ein und bestellte ihn zum Höhenpriester. \bibleverse{34}Dieses Verfahren
führte aber beim Hause Jerobeams zur Versündigung und weiter zu seiner
Vernichtung und Vertilgung vom Erdboden hinweg.

\hypertarget{d-strafandrohung-des-propheten-ahia-jerobeams-tod}{%
\paragraph{d) Strafandrohung des Propheten Ahia; Jerobeams
Tod}\label{d-strafandrohung-des-propheten-ahia-jerobeams-tod}}

\hypertarget{section-13}{%
\section{14}\label{section-13}}

\bibleverse{1}Zu jener Zeit wurde Abia, der Sohn Jerobeams, krank.
\bibleverse{2}Da sagte Jerobeam zu seiner Gemahlin: »Mache dich auf,
verkleide dich, damit man die Gemahlin Jerobeams in dir nicht erkenne,
und begib dich nach Silo. Dort wohnt nämlich der Prophet Ahia, derselbe,
der mir einst angekündigt hat, daß ich König über dieses Volk werden
würde\textless sup title=``vgl. 11,29-39''\textgreater✲.
\bibleverse{3}Nimm zehn Brote und Kuchen und einen Krug Honig mit und
gehe zu ihm: er wird dir verkünden, wie es mit dem Knaben gehen wird.«
\bibleverse{4}Da tat die Gemahlin Jerobeams so; sie machte sich auf den
Weg nach Silo und kam in das Haus Ahias. Ahia aber konnte nicht mehr
sehen, denn seine Augen waren infolge seines hohen Alters erblindet.
\bibleverse{5}Der HERR aber hatte zu Ahia gesagt: »Soeben kommt die
Gemahlin Jerobeams, um von dir Auskunft über ihren Sohn zu erhalten,
denn er ist krank. So und so sollst du zu ihr sagen.«

\hypertarget{strafrede-und-drohung-ahias-gegen-jerobeam}{%
\paragraph{Strafrede und Drohung Ahias gegen
Jerobeam}\label{strafrede-und-drohung-ahias-gegen-jerobeam}}

Als sie nun in ihrer Verkleidung eintrat \bibleverse{6}und Ahia, während
sie zur Tür hereintrat, das Geräusch ihrer Schritte vernahm, rief er ihr
zu: »Komm herein, Gattin Jerobeams! Warum doch verstellst du dich so?
Mir ist eine harte Botschaft an dich aufgetragen. \bibleverse{7}Gehe
heim und sage zu Jerobeam: So hat der HERR, der Gott Israels,
gesprochen: ›Ich habe dich mitten aus dem Volk emporgehoben und dich zum
Fürsten über mein Volk Israel bestellt; \bibleverse{8}ich habe das
Königtum dem Hause Davids entrissen und es dir gegeben; du aber bist
nicht gewesen wie mein Knecht David, der meine Gebote beobachtet hat und
mir von ganzem Herzen gehorsam gewesen ist, so daß er nur das tat, was
mir wohlgefiel; \bibleverse{9}nein, du hast mehr Böses getan als alle
deine Vorgänger; denn du bist hingegangen und hast dir andere Götter
gemacht, nämlich Gußbilder, um mich zum Zorn zu reizen, mich aber hast
du hinter deinen Rücken geworfen. \bibleverse{10}Darum will ich nunmehr
Unglück über das Haus Jerobeams kommen lassen und von den Angehörigen
Jerobeams alles ausrotten, was männlichen Geschlechts ist, Unmündige und
Mündige in Israel, und will das Haus Jerobeams wegfegen, wie man Unrat
wegfegt, bis nichts mehr von ihm übrig ist. \bibleverse{11}Wer von
Jerobeams Angehörigen in der Stadt stirbt, den sollen die Hunde fressen,
und wer auf dem freien Felde stirbt, den sollen die Vögel des Himmels
fressen; denn der HERR hat gesprochen! \bibleverse{12}Du aber mache dich
auf und kehre heim: sowie deine Füße die Stadt betreten, wird der Knabe
sterben. \bibleverse{13}Ganz Israel wird dann um ihn klagen, und man
wird ihn begraben; denn von den Angehörigen Jerobeams wird dieser allein
in ein Grab kommen, weil sich an ihm noch etwas Gutes vor dem HERRN, dem
Gott Israels, im Hause Jerobeams gefunden hat. \bibleverse{14}Der HERR
aber wird sich einen König über Israel erstehen lassen, der das Haus
Jerobeams ausrotten soll an jenem Tage\textless sup title=``vgl.
15,29-30''\textgreater✲; und was geschieht schon jetzt?✲
\bibleverse{15}Und der HERR wird Israel schlagen, daß es schwankt, wie
das Schilfrohr im Wasser schwankt; er wird Israel aus diesem schönen
Lande verstoßen, das er ihren Vätern gegeben hat, und wird sie
zerstreuen jenseits des Euphratstroms zur Strafe dafür, daß sie sich
Götzenbilder angefertigt und den HERRN dadurch erzürnt haben.
\bibleverse{16}Ja, er wird Israel dahingeben um der Sünden Jerobeams
willen, die er selbst begangen und zu denen er Israel verführt hat!«

\hypertarget{erfuxfcllung-der-weissagung-schluuxdfwort}{%
\paragraph{Erfüllung der Weissagung;
Schlußwort}\label{erfuxfcllung-der-weissagung-schluuxdfwort}}

\bibleverse{17}Da machte sich die Gemahlin Jerobeams auf den Weg und
kehrte nach Thirza zurück; kaum hatte sie dort die Schwelle des
Königshauses betreten, als der Knabe starb. \bibleverse{18}Man begrub
ihn, und ganz Israel hielt ihm die Totenklage, wie der HERR es durch
seinen Knecht, den Propheten Ahia, hatte ankündigen lassen.

\bibleverse{19}Die übrige Geschichte Jerobeams aber, wie er Kriege
geführt und wie er regiert hat, das findet sich bekanntlich bereits
aufgezeichnet im Buch der Denkwürdigkeiten\textless sup title=``oder:
Chronik''\textgreater✲ der Könige von Israel. \bibleverse{20}Die Dauer
der Regierung Jerobeams betrug zweiundzwanzig Jahre; dann legte er sich
zu seinen Vätern, und sein Sohn Nadab folgte ihm in der Regierung nach.

\hypertarget{e-die-regierung-des-kuxf6nigs-rehabeam-von-juda-und-seiner-nachfolger-abia-und-asa}{%
\paragraph{e) Die Regierung des Königs Rehabeam von Juda und seiner
Nachfolger Abia und
Asa}\label{e-die-regierung-des-kuxf6nigs-rehabeam-von-juda-und-seiner-nachfolger-abia-und-asa}}

\hypertarget{aa-judas-abguxf6tterei-unter-rehabeam}{%
\subparagraph{aa) Judas Abgötterei unter
Rehabeam}\label{aa-judas-abguxf6tterei-unter-rehabeam}}

\bibleverse{21}Rehabeam aber, der Sohn Salomos, war König über Juda; 41
Jahre war Rehabeam alt, als er König wurde, und 17 Jahre regierte er in
Jerusalem, der Stadt, die der HERR aus allen Stämmen Israels erwählt
hatte, um seinen Namen dort wohnen zu lassen. Seine Mutter hieß Naama
und war eine Ammonitin. \bibleverse{22}Juda tat aber, was dem HERRN
mißfiel, und sie reizten ihn durch die Sünden, die sie begingen, in noch
höherem Grade zum Eifer, als er erregt worden war durch alles, was ihre
Väter getan hatten; \bibleverse{23}denn auch sie errichteten sich
Höhentempel, Malsteine\textless sup title=``oder:
Steinsäulen''\textgreater✲ und Götzenbilder auf jedem hohen Hügel und
unter jedem dichtbelaubten Baume; \bibleverse{24}ja, auch
Heiligtumsbuhler\textless sup title=``oder: geweihte
Buhler''\textgreater✲ gab es im Lande, (kurz) sie taten es allen Greueln
der Heidenvölker gleich, die der HERR vor den Israeliten vertrieben
hatte.

\hypertarget{bb-einfall-und-pluxfcnderung-des-uxe4gyptischen-kuxf6nigs-sisak-schluuxdfwort}{%
\subparagraph{bb) Einfall und Plünderung des ägyptischen Königs Sisak;
Schlußwort}\label{bb-einfall-und-pluxfcnderung-des-uxe4gyptischen-kuxf6nigs-sisak-schluuxdfwort}}

\bibleverse{25}Aber im fünften Regierungsjahre Rehabeams zog Sisak, der
König von Ägypten, gegen Jerusalem heran \bibleverse{26}und raubte die
Schätze des Tempels des HERRN und die Schätze des königlichen Palastes,
überhaupt alles raubte er; auch die goldenen Schilde nahm er weg, die
Salomo hatte anfertigen lassen. \bibleverse{27}An deren Stelle ließ der
König Rehabeam eherne Schilde herstellen und übergab sie der Obhut der
Befehlshaber seiner Leibwache, die am Eingang zum königlichen Palast die
Wache hatte. \bibleverse{28}Sooft sich nun der König in den Tempel des
HERRN begab, mußten die Leibwächter die Schilde tragen und brachten sie
dann wieder in das Wachtzimmer der Leibwache zurück.

\bibleverse{29}Die übrige Geschichte Rehabeams aber und alles, was er
unternommen hat, findet sich bekanntlich aufgezeichnet im Buche der
Denkwürdigkeiten\textless sup title=``oder: Chronik''\textgreater✲ der
Könige von Juda. \bibleverse{30}Es bestand aber Krieg zwischen Rehabeam
und Jerobeam, solange sie lebten. \bibleverse{31}Rehabeam legte sich
dann zu seinen Vätern und wurde bei seinen Vätern in der Davidsstadt
begraben {[}seine Mutter hieß Naama und war eine Ammonitin{]}; und sein
Sohn Abia folgte ihm in der Regierung nach.

\hypertarget{cc-regierung-des-kuxf6nigs-abia-von-juda}{%
\subparagraph{cc) Regierung des Königs Abia von
Juda}\label{cc-regierung-des-kuxf6nigs-abia-von-juda}}

\hypertarget{section-14}{%
\section{15}\label{section-14}}

\bibleverse{1}Im achtzehnten Regierungsjahre Jerobeams, des Sohnes
Nebats, wurde Abia König über Juda. \bibleverse{2}Er regierte drei Jahre
in Jerusalem; seine Mutter hieß Maacha und war eine Enkeltochter
Absaloms\textless sup title=``oder: eine Tochter Uriels von Gibea; vgl.
2.Chr 13,2''\textgreater✲. \bibleverse{3}Abia wandelte in allen Sünden
seines Vaters, die dieser vor ihm begangen hatte, und sein Herz war dem
HERRN, seinem Gott, nicht ungeteilt ergeben wie das Herz seines Ahnherrn
David. \bibleverse{4}Doch um Davids willen hatte der HERR, sein Gott,
ihm eine Leuchte\textless sup title=``vgl. 11,36''\textgreater✲ in
Jerusalem scheinen lassen, indem er seinen Sohn nach ihm (auf den Thron)
erhob und Jerusalem bestehen ließ, \bibleverse{5}weil David getan hatte,
was dem HERRN wohlgefiel, und während seines ganzen Lebens von allem,
was er ihm geboten, nicht abgewichen war, abgesehen von dem Vorkommnis
mit dem Hethiter Uria. \bibleverse{6}Es bestand aber Krieg zwischen Abia
und Jerobeam, solange er lebte. \bibleverse{7}Die übrige Geschichte
Abias aber und alles, was er unternommen hat, findet sich bekanntlich
aufgezeichnet in dem Buche der Denkwürdigkeiten\textless sup
title=``oder: Chronik''\textgreater✲ der Könige von Juda. Es bestand
aber Krieg zwischen Abia und Jerobeam. \bibleverse{8}Als Abia sich dann
zu seinen Vätern gelegt und man ihn in der Davidsstadt begraben hatte,
folgte ihm sein Sohn Asa in der Regierung nach.

\hypertarget{dd-regierung-des-kuxf6nigs-asa-von-juda}{%
\subparagraph{dd) Regierung des Königs Asa von
Juda}\label{dd-regierung-des-kuxf6nigs-asa-von-juda}}

\hypertarget{asas-einschreiten-gegen-den-guxf6tzendienst}{%
\paragraph{Asas Einschreiten gegen den
Götzendienst}\label{asas-einschreiten-gegen-den-guxf6tzendienst}}

\bibleverse{9}Im zwanzigsten Jahre der Regierung Jerobeams, des Königs
von Israel, wurde Asa König über Juda \bibleverse{10}und regierte 41
Jahre zu Jerusalem; seine Mutter\textless sup title=``=~Großmutter; vgl.
15,2''\textgreater✲ hieß Maacha und war eine Tochter✲ Absaloms.
\bibleverse{11}Asa tat, was dem HERRN wohlgefiel, wie sein Ahnherr
David. \bibleverse{12}So jagte er denn die Heiligtumsbuhler\textless sup
title=``oder: geweihten Buhler''\textgreater✲ aus dem Lande und
entfernte alle Götzenbilder, die seine Vorfahren hatten anfertigen
lassen. \bibleverse{13}Sogar seiner Mutter Maacha entzog er den Rang der
Königin-Mutter, weil sie der Aschera ein Schandmal\textless sup
title=``oder: Götzenbild''\textgreater✲ hatte anfertigen lassen; Asa
ließ ihr Schandmal umhauen und im Kidrontal verbrennen.
\bibleverse{14}Der Höhendienst wurde allerdings nicht abgeschafft, doch
war das Herz Asas dem HERRN zeitlebens ungeteilt ergeben.
\bibleverse{15}Er ließ auch die Geschenke, die sein Vater und die er
selbst geweiht hatte, in den Tempel des HERRN bringen, Silber, Gold und
Geräte.

\hypertarget{asas-krieg-mit-dem-kuxf6nig-baesa-von-israel-schluuxdfwort}{%
\paragraph{Asas Krieg mit dem König Baesa von Israel;
Schlußwort}\label{asas-krieg-mit-dem-kuxf6nig-baesa-von-israel-schluuxdfwort}}

\bibleverse{16}Es bestand aber zwischen Asa und dem König Baesa von
Israel Krieg, solange sie lebten. \bibleverse{17}Dabei zog Baesa, der
König von Israel, gegen Juda hinauf und befestigte Rama, damit niemand
mehr bei Asa, dem König von Juda, ungehindert aus- und eingehen könne.
\bibleverse{18}Da nahm Asa alles Silber und Gold, das noch in den
Schatzkammern des Tempels des HERRN vorhanden war, und die Schätze des
königlichen Palastes, übergab sie seinen Dienern und sandte sie an
Benhadad, den Sohn Tabrimmons, den Enkel Hesjons, den König von Syrien,
der in Damaskus seinen Sitz hatte✲, und ließ ihm sagen:
\bibleverse{19}»Ein Bündnis besteht zwischen mir und dir, zwischen
meinem Vater und deinem Vater. Hier übersende ich dir ein Geschenk an
Silber und Gold. So löse denn dein Bündnis mit dem König Baesa von
Israel, damit er aus meinem Lande abziehe!« \bibleverse{20}Benhadad
schenkte der Aufforderung des Königs Asa Gehör, ließ seine Heerführer
gegen die Städte Israels ziehen und eroberte Ijjon, Dan,
Abel-Beth-Maacha und ganz Kinneroth nebst der ganzen Landschaft
Naphthali. \bibleverse{21}Sobald nun Baesa Kunde davon erhielt, gab er
die Befestigung Ramas auf und kehrte nach Thirza zurück.
\bibleverse{22}Nun bot der König Asa ganz Juda auf bis auf den letzten
Mann; die mußten die Steine und Balken wegschaffen, mit denen Baesa Rama
hatte befestigen wollen; und der König Asa ließ damit Geba im Stamme
Benjamin und Mizpa befestigen.

\bibleverse{23}Die ganze übrige Geschichte Asas aber und alle seine
tapferen Taten und alles, was er unternommen hat, sowie seine
Festungsbauten, das findet sich bekanntlich aufgezeichnet im Buche der
Denkwürdigkeiten\textless sup title=``oder: Chronik''\textgreater✲ der
Könige von Juda. Doch hatte er in seinem Alter an einer Fußkrankheit zu
leiden. \bibleverse{24}Als Asa sich dann zu seinen Vätern gelegt und man
ihn bei seinen Vätern in der Stadt Davids, seines Ahnherrn, begraben
hatte, folgte ihm sein Sohn Josaphat in der Regierung nach.

\hypertarget{f-die-kuxf6nige-in-israel-von-nadab-bis-ahab}{%
\paragraph{f) Die Könige in Israel von Nadab bis
Ahab}\label{f-die-kuxf6nige-in-israel-von-nadab-bis-ahab}}

\hypertarget{aa-regierung-des-kuxf6nigs-nadab-von-israel-sein-sturz-durch-baesa}{%
\subparagraph{aa) Regierung des Königs Nadab von Israel; sein Sturz
durch
Baesa}\label{aa-regierung-des-kuxf6nigs-nadab-von-israel-sein-sturz-durch-baesa}}

\bibleverse{25}Nadab aber, der Sohn Jerobeams, wurde König über Israel
im zweiten Jahre der Regierung Asas, des Königs von Juda, und regierte
zwei Jahre über Israel. \bibleverse{26}Er tat, was dem HERRN mißfiel,
und wandelte auf dem Wege seines Vaters und in dessen Sünde, zu der er
Israel verführt hatte. \bibleverse{27}Aber Baesa, der Sohn Ahias, aus
dem Hause Issaschar, zettelte eine Verschwörung gegen ihn an und
ermordete ihn bei der Philisterstadt Gibbethon, während Nadab und ganz
Israel Gibbethon gerade belagerten. \bibleverse{28}Baesa brachte ihn
also im dritten Regierungsjahre Asas, des Königs von Juda, ums Leben und
wurde König an seiner Statt. \bibleverse{29}Sobald er nun die Herrschaft
erlangt hatte, rottete er das ganze Haus Jerobeams aus: er ließ von den
Angehörigen Jerobeams keine lebende Seele übrig, bis er sie vertilgt
hatte, wie der HERR es durch den Mund seines Knechtes Ahia von Silo
hatte ankündigen lassen\textless sup title=``vgl. 14,14''\textgreater✲,
\bibleverse{30}zur Strafe für die Sünden, die Jerobeam selbst verübt und
zu denen er die Israeliten verführt hatte, indem er den HERRN, den Gott
Israels, zum Zorn reizte. \bibleverse{31}Die übrige Geschichte Nadabs
aber und alles, was er unternommen hat, findet sich bekanntlich
aufgezeichnet im Buche der Denkwürdigkeiten\textless sup title=``oder:
Chronik''\textgreater✲ der Könige von Israel. \bibleverse{32}Es bestand
aber Krieg zwischen Asa und dem König Baesa von Israel, solange sie
lebten.

\hypertarget{bb-regierung-des-kuxf6nigs-baesa-von-israel}{%
\subparagraph{bb) Regierung des Königs Baesa von
Israel}\label{bb-regierung-des-kuxf6nigs-baesa-von-israel}}

\bibleverse{33}Im dritten Regierungsjahre Asas, des Königs von Juda,
wurde Baesa, der Sohn Ahias, König über ganz Israel und regierte in
Thirza vierundzwanzig Jahre lang. \bibleverse{34}Er tat, was dem HERRN
mißfiel, und wandelte auf dem Wege Jerobeams und in dessen Sünde, zu der
er Israel verführt hatte.

\hypertarget{cc-des-propheten-jehu-gerichtsdrohung-gegen-baesa}{%
\subparagraph{cc) Des Propheten Jehu Gerichtsdrohung gegen
Baesa}\label{cc-des-propheten-jehu-gerichtsdrohung-gegen-baesa}}

\hypertarget{section-15}{%
\section{16}\label{section-15}}

\bibleverse{1}Da erging das Wort des HERRN an Jehu, den Sohn Hananis,
gegen Baesa also: \bibleverse{2}»Weil ich dich aus dem Staub erhoben und
dich zum Fürsten über mein Volk Israel gemacht habe, du aber auf dem
Wege Jerobeams gewandelt bist und mein Volk Israel zur Sünde verführt
hast, so daß sie mich durch ihre Sünden zum Zorn reizen,
\bibleverse{3}so will ich nun Baesa und sein Haus wegfegen und will es
mit deinem Hause machen wie mit dem Hause Jerobeams, des Sohnes Nebats.
\bibleverse{4}Wer von den Angehörigen Baesas in der Stadt stirbt, den
sollen die Hunde fressen, und wer von ihnen auf dem freien Felde stirbt,
den sollen die Vögel des Himmels fressen!«\textless sup title=``vgl.
14,11''\textgreater✲~-- \bibleverse{5}Die übrige Geschichte Baesas aber
und was er unternommen hat sowie seine tapferen Taten, das findet sich
bekanntlich aufgezeichnet im Buch der Denkwürdigkeiten\textless sup
title=``oder: Chronik''\textgreater✲ der Könige von Israel.~--
\bibleverse{6}Als Baesa sich dann zu seinen Vätern gelegt und man ihn in
Thirza begraben hatte, folgte ihm sein Sohn Ela in der Regierung nach.
\bibleverse{7}Übrigens war das Wort des HERRN gegen Baesa und dessen
Haus durch den Mund des Propheten Jehu, des Sohnes Hananis, sowohl wegen
alles des Bösen ergangen, durch das er sich gegen den HERRN versündigt
hatte, um ihn durch sein ganzes Tun zum Zorn zu reizen, so daß er es dem
Hause Jerobeams gleichtat, als auch deshalb, weil er dieses ausgerottet
hatte.

\hypertarget{regierung-des-kuxf6nigs-ela-von-israel}{%
\paragraph{Regierung des Königs Ela von
Israel}\label{regierung-des-kuxf6nigs-ela-von-israel}}

\bibleverse{8}Im sechsundzwanzigsten Jahre der Regierung Asas, des
Königs von Juda, wurde Ela, der Sohn Baesas, König über Israel und
regierte in Thirza zwei Jahre. \bibleverse{9}Gegen ihn verschwor sich
sein Diener Simri, einer seiner Obersten, der über die Hälfte der
Kriegswagen gesetzt war. Als sich Ela nun einst in Thirza bei einem
Gelage im Hause Arzas, seines Haushofmeisters in Thirza, berauscht
hatte, \bibleverse{10}drang Simri dort ein und erschlug ihn im
siebenundzwanzigsten Jahre der Regierung Asas, des Königs von Juda, und
wurde König an seiner Statt. \bibleverse{11}Als er nun König geworden
war, ermordete er, sobald er auf seinem Throne saß, alle zum Hause
Baesas Gehörigen; er ließ keinen von ihnen übrig, der männlichen
Geschlechts war, weder seine Blutsverwandten noch seine Freunde.
\bibleverse{12}So rottete Simri das ganze Haus Baesas aus, gemäß der
Drohung, die der HERR durch den Mund des Propheten Jehu gegen Baesa
ausgesprochen hatte, \bibleverse{13}wegen all der Sünden, die Baesa und
sein Sohn Ela begangen und zu denen sie Israel verführt hatten, um den
HERRN, den Gott Israels, durch ihren Götzendienst zum Zorn zu reizen.
\bibleverse{14}Die übrige Geschichte Elas aber und alles, was er
unternommen hat, findet sich bekanntlich aufgezeichnet im Buch der
Denkwürdigkeiten\textless sup title=``oder: Chronik''\textgreater✲ der
Könige von Israel.

\hypertarget{dd-regierung-des-kuxf6nigs-simri-von-israel}{%
\subparagraph{dd) Regierung des Königs Simri von
Israel}\label{dd-regierung-des-kuxf6nigs-simri-von-israel}}

\bibleverse{15}Im siebenundzwanzigsten Jahre der Regierung Asas, des
Königs von Juda, wurde Simri König für sieben Tage in Thirza, während
das Heer die Philisterstadt Gibbethon belagerte. \bibleverse{16}Als nun
das Heer im Lager die Kunde erhielt, Simri habe eine Verschwörung
angestiftet und den König schon ermordet, da erhob das ganze
israelitische Heer noch an demselben Tage Omri, den obersten
Befehlshaber des israelitischen Heeres, im Lager zum König über Israel.
\bibleverse{17}Darauf zog Omri mit allen Israeliten von Gibbethon ab und
belagerte Thirza. \bibleverse{18}Als nun Simri sah, daß die Stadt
erobert war, begab er sich in die Burg des königlichen Palastes, steckte
den königlichen Palast über sich in Brand und fand so den Tod
\bibleverse{19}um seiner Sünden willen, die er begangen hatte, indem er
tat, was dem HERRN mißfiel, indem er auf dem Wege Jerobeams wandelte und
in dessen Sünde, durch die er sich verschuldet hatte, indem er Israel
zur Sünde verleitete. \bibleverse{20}Die übrige Geschichte Simris aber
und seine Verschwörung, die er angestiftet hatte, das findet sich
bekanntlich aufgezeichnet im Buche der Denkwürdigkeiten\textless sup
title=``oder: Chronik''\textgreater✲ der Könige von Israel.

\hypertarget{ee-spaltung-des-reiches-israel-omri-alleinherrscher}{%
\subparagraph{ee) Spaltung des Reiches Israel; Omri
Alleinherrscher}\label{ee-spaltung-des-reiches-israel-omri-alleinherrscher}}

\bibleverse{21}Damals spaltete sich das Volk Israel in zwei Parteien:
die eine Hälfte des Volkes hielt es mit Thibni, dem Sohne Ginaths, um
ihn zum König zu machen, die andere Hälfte dagegen war für Omri.
\bibleverse{22}Aber die Partei Omris gewann die Oberhand über die
Anhänger Thibnis, des Sohnes Ginaths; und als Thibni starb\textless sup
title=``oder: im Kampf den Tod fand''\textgreater✲, wurde Omri König.

\hypertarget{ff-regierung-des-kuxf6nigs-omri-von-israel-gruxfcndung-der-hauptstadt-samaria}{%
\subparagraph{ff) Regierung des Königs Omri von Israel; Gründung der
Hauptstadt
Samaria}\label{ff-regierung-des-kuxf6nigs-omri-von-israel-gruxfcndung-der-hauptstadt-samaria}}

\bibleverse{23}Im einunddreißigsten Jahre der Regierung Asas, des Königs
von Juda, wurde Omri König über Israel und regierte zwölf Jahre. Als er
in Thirza sechs Jahre regiert hatte, \bibleverse{24}kaufte er den Berg
Samaria von Semer für zwei Talente Silber, befestigte dann den Berg und
nannte die Stadt, die er dort gründete, Samaria nach dem Namen Semers,
des früheren Besitzers des Berges. \bibleverse{25}Omri tat aber, was dem
HERRN mißfiel, ja, er trieb es noch ärger als alle seine Vorgänger;
\bibleverse{26}er wandelte ganz auf dem Wege Jerobeams, des Sohnes
Nebats, und in den Sünden, zu denen jener die Israeliten verführt hatte,
so daß sie den HERRN, den Gott Israels, durch ihren Götzendienst
erzürnten. \bibleverse{27}Die übrige Geschichte Omris aber, seine
Unternehmungen und die tapferen Taten, die er vollführt hat, das alles
findet sich bekanntlich aufgezeichnet im Buch der
Denkwürdigkeiten\textless sup title=``oder: Chronik''\textgreater✲ der
Könige von Israel. \bibleverse{28}Als Omri sich dann zu seinen Vätern
gelegt und man ihn in Samaria begraben hatte, folgte ihm sein Sohn Ahab
in der Regierung nach.

\hypertarget{g-die-suxfcnden-des-kuxf6nigs-ahab-von-israel-und-seiner-gemahlin-isebel}{%
\paragraph{g) Die Sünden des Königs Ahab von Israel und seiner Gemahlin
Isebel}\label{g-die-suxfcnden-des-kuxf6nigs-ahab-von-israel-und-seiner-gemahlin-isebel}}

\bibleverse{29}Ahab, der Sohn Omris, wurde König über Israel im
achtunddreißigsten Jahre der Regierung Asas, des Königs von Juda; und
Ahab, der Sohn Omris, regierte über Israel zweiundzwanzig Jahre in
Samaria. \bibleverse{30}Er tat aber, was dem HERRN mißfiel, und trieb es
noch ärger als alle seine Vorgänger. \bibleverse{31}Und nicht genug, daß
er in den Sünden Jerobeams, des Sohnes Nebats, wandelte, heiratete er
auch noch Isebel, die Tochter des Sidonierkönigs Ethbaal, und wandte
sich dann dem Dienste Baals zu und betete ihn an. \bibleverse{32}Er
errichtete dem Baal auch einen Altar in dem Baaltempel, den er in
Samaria erbaut hatte, \bibleverse{33}und ließ das Ascherabild anfertigen
und verübte noch andere Greuel, um den HERRN, den Gott Israels, noch
heftiger zu erzürnen als alle israelitischen Könige, die vor ihm
geherrscht hatten.

\hypertarget{wiederaufbau-der-stadt-jericho}{%
\paragraph{Wiederaufbau der Stadt
Jericho}\label{wiederaufbau-der-stadt-jericho}}

\bibleverse{34}Während seiner Regierung baute Hiel von Bethel die Stadt
Jericho wieder auf. Die Grundlegung kostete seinem ältesten Sohne Abiram
das Leben, und die Einsetzung der Tore brachte seinem jüngsten Sohne
Segub den Tod, wie der HERR es durch den Mund Josuas, des Sohnes Nuns,
vorausgesagt hatte\textless sup title=``vgl. Jos 6,26''\textgreater✲.

\hypertarget{ahab-von-israel-und-der-prophet-elia-kap.-17-22}{%
\subsubsection{2. Ahab von Israel und der Prophet Elia (Kap.
17-22)}\label{ahab-von-israel-und-der-prophet-elia-kap.-17-22}}

\hypertarget{a-das-auftreten-elias-die-wunder-am-bache-krith-und-in-zarpath}{%
\paragraph{a) Das Auftreten Elias; die Wunder am Bache Krith und in
Zarpath}\label{a-das-auftreten-elias-die-wunder-am-bache-krith-und-in-zarpath}}

\hypertarget{aa-elia-vor-dem-kuxf6nig-ahab-und-am-bache-krith}{%
\subparagraph{aa) Elia vor dem König Ahab und am Bache
Krith}\label{aa-elia-vor-dem-kuxf6nig-ahab-und-am-bache-krith}}

\hypertarget{section-16}{%
\section{17}\label{section-16}}

\bibleverse{1}Da sagte Elia, der Thisbiter, aus Thisbe in Gilead, zu
Ahab: »So wahr der HERR, der Gott Israels, lebt, in dessen Dienst ich
stehe: es soll in den nächsten Jahren weder Tau noch Regen fallen, es
sei denn auf mein Wort!« \bibleverse{2}Hierauf erging das Wort des HERRN
an ihn also: \bibleverse{3}»Gehe weg von hier und wende dich ostwärts
und verbirg dich am Bache Krith, der östlich vom Jordan fließt.
\bibleverse{4}Aus dem Bache sollst du trinken, und den Raben habe ich
geboten, dich dort mit Nahrung zu versorgen.« \bibleverse{5}Da ging er
weg und tat nach dem Befehl des HERRN: er ging hin und ließ sich am
Bache Krith nieder, der auf der Ostseite des Jordans fließt;
\bibleverse{6}und die Raben brachten ihm Brot und Fleisch am Morgen und
ebenso am Abend, und er trank aus dem Bache.

\hypertarget{bb-das-wunder-elias-bei-der-witwe-in-zarpath-sarepta-in-phuxf6nizien}{%
\subparagraph{bb) Das Wunder Elias bei der Witwe in Zarpath (Sarepta) in
Phönizien}\label{bb-das-wunder-elias-bei-der-witwe-in-zarpath-sarepta-in-phuxf6nizien}}

\bibleverse{7}Als dann aber der Bach nach einiger Zeit trocken wurde,
weil kein Regen im Lande gefallen war, \bibleverse{8}erging das Wort des
HERRN an ihn also: \bibleverse{9}»Mache dich auf, begib dich nach
Zarpath✲, das zu Sidon gehört, und bleibe daselbst! Ich habe einer Witwe
dort geboten, für deinen Unterhalt zu sorgen.« \bibleverse{10}Da machte
er sich auf den Weg und begab sich nach Zarpath; und als er am Stadttor
ankam, war dort eine Witwe gerade damit beschäftigt, Holz
zusammenzulesen. Er rief sie an mit den Worten: »Hole mir doch ein wenig
Wasser in einem Kruge, damit ich trinke!« \bibleverse{11}Als sie nun
hinging, um es zu holen, rief er ihr die Worte nach: »Bring mir doch
auch einen Bissen Brot mit!« \bibleverse{12}Aber sie antwortete: »So
wahr der HERR, dein Gott, lebt! Ich besitze nichts Gebackenes; nur noch
eine Handvoll Mehl ist im Topf und ein wenig Öl im Kruge. Eben lese ich
ein paar Stücke Holz zusammen, dann will ich heimgehen und es für mich
und meinen Sohn zubereiten, damit wir es essen und dann sterben.«
\bibleverse{13}Doch Elia antwortete ihr: »Fürchte dich nicht, gehe heim
und tu, wie du gesagt hast; doch zuerst bereite mir davon einen kleinen
Kuchen und bringe ihn mir her! Darnach magst du für dich und deinen Sohn
auch etwas zubereiten. \bibleverse{14}Denn so hat der HERR, der Gott
Israels, gesprochen: ›Das Mehl im Topf soll nicht ausgehen und das Öl im
Kruge nicht abnehmen bis zu dem Tage, wo der HERR wieder Regen auf den
Erdboden fallen läßt.‹« \bibleverse{15}Da ging sie hin und kam der
Weisung Elias nach; und sie hatten lange Zeit zu essen, er und sie und
ihr Sohn: \bibleverse{16}das Mehl im Topf ging nicht aus, und das Öl im
Kruge nahm nicht ab, wie der HERR es durch den Mund Elias hatte
ankündigen lassen.

\hypertarget{cc-die-wiederbelebung-des-sohnes-der-witwe}{%
\subparagraph{cc) Die Wiederbelebung des Sohnes der
Witwe}\label{cc-die-wiederbelebung-des-sohnes-der-witwe}}

\bibleverse{17}Nachmals aber begab es sich, daß der Sohn jener Frau, der
das Haus gehörte, krank wurde, und seine Krankheit verschlimmerte sich
so, daß kein Atem mehr in ihm blieb. \bibleverse{18}Da sagte sie zu
Elia: »Was haben wir miteinander zu schaffen, du Mann Gottes? Du bist
nur deshalb zu mir gekommen, um meine Verschuldung bei Gott in
Erinnerung zu bringen und den Tod meines Sohnes herbeizuführen!«
\bibleverse{19}Er antwortete ihr: »Gib mir deinen Sohn her!« Er nahm ihn
dann von ihrem Schoß, trug ihn in das Obergemach hinauf, wo er selbst
wohnte, und legte ihn auf sein Bett; \bibleverse{20}dann rief er den
HERRN an und betete: »HERR, mein Gott, hast du wirklich die Witwe, bei
der ich zu Gast bin, so unglücklich gemacht, daß du ihren Sohn hast
sterben lassen?« \bibleverse{21}Darauf streckte er sich dreimal über den
Knaben hin und rief den HERRN mit den Worten an: »HERR, mein Gott, laß
doch die Seele\textless sup title=``oder: das Leben''\textgreater✲
dieses Knaben wieder in ihn zurückkehren!« \bibleverse{22}Da erhörte der
HERR das Gebet Elias, und die Seele des Knaben kehrte in ihn zurück, so
daß er wieder auflebte. \bibleverse{23}Elia aber nahm den Knaben, trug
ihn aus dem Obergemach ins Haus hinunter und übergab ihn seiner Mutter
mit den Worten: »Sieh her, dein Sohn lebt!« \bibleverse{24}Da antwortete
die Frau dem Elia: »Ja, nun weiß ich, daß du ein Mann Gottes bist und
daß das Wort des HERRN in deinem Munde Wahrheit ist!«

\hypertarget{b-elia-stellt-sich-dem-ahab-das-gottesurteil-auf-dem-karmel}{%
\paragraph{b) Elia stellt sich dem Ahab; das Gottesurteil auf dem
Karmel}\label{b-elia-stellt-sich-dem-ahab-das-gottesurteil-auf-dem-karmel}}

\hypertarget{aa-gottes-befehl-an-elia-obadjas-zusammentreffen-mit-elia}{%
\subparagraph{aa) Gottes Befehl an Elia; Obadjas Zusammentreffen mit
Elia}\label{aa-gottes-befehl-an-elia-obadjas-zusammentreffen-mit-elia}}

\hypertarget{section-17}{%
\section{18}\label{section-17}}

\bibleverse{1}Lange Zeit darauf aber erging das Wort des HERRN an Elia
im dritten Jahr der Dürre also: »Gehe hin, zeige dich dem Ahab; denn ich
will auf Erden regnen lassen«. \bibleverse{2}Da machte sich Elia auf den
Weg, um dem Ahab vor die Augen zu treten.

Die Hungersnot war aber in Samaria immer drückender geworden;
\bibleverse{3}da hatte Ahab seinen Haushofmeister Obadja rufen lassen --
dieser war ein treuer Verehrer des HERRN; \bibleverse{4}als daher Isebel
die Propheten des HERRN ausrottete, hatte Obadja hundert Propheten
genommen und sie, je fünfzig Mann, in einer Höhle versteckt und sie mit
Brot und Wasser versorgt. \bibleverse{5}Ahab hatte also zu Obadja
gesagt: »Komm, wir wollen durch das Land an alle Wasserquellen und an
alle Bäche gehen; vielleicht finden wir noch Futter, so daß wir Pferde
und Maultiere am Leben erhalten können und nicht einen Teil des Viehs
eingehen zu lassen brauchen.« \bibleverse{6}Dann hatten sie das Land
unter sich geteilt, um es zu durchwandern: Ahab war für sich allein in
der einen Richtung gegangen und Obadja auch für sich allein in der
anderen Richtung.

\bibleverse{7}Während nun Obadja unterwegs war, trat ihm plötzlich Elia
entgegen. Als er ihn erkannte, warf er sich auf sein Angesicht nieder
und rief aus: »Bist du es wirklich, mein Herr Elia?« \bibleverse{8}Er
antwortete ihm: »Jawohl! Gehe hin und sage deinem Herrn: ›Elia ist da!‹«
\bibleverse{9}Doch er erwiderte: »Womit habe ich es verdient, daß du
deinen Knecht dem Ahab ausliefern willst, damit er mich umbringe?
\bibleverse{10}So wahr der HERR, dein Gott, lebt: es gibt kein Volk und
kein Königreich, wohin mein Herr nicht gesandt hätte, um dich zu suchen;
sagte man dann: ›Er ist nicht hier‹, so ließ er das Königshaus und das
Volk schwören, daß er dich wirklich nicht ausfindig machen würde.
\bibleverse{11}Und nun forderst du mich auf: ›Gehe hin und melde deinem
Herrn: Elia ist da!‹ \bibleverse{12}Wenn ich jetzt von dir weggehe und
der Geist des HERRN dich an einen mir unbekannten Ort entführt und ich
dann zu Ahab käme, um es ihm zu melden, und er dich dann nicht fände, so
würde er mich hinrichten lassen! Und dein Knecht hat doch den HERRN von
Jugend auf gefürchtet. \bibleverse{13}Ist es denn meinem Herrn unbekannt
geblieben, was ich getan habe, als Isebel die Propheten des HERRN
ermorden ließ? Daß ich von den Propheten des HERRN hundert Mann, je
fünfzig in einer Höhle versteckt und sie mit Speise und Trank versorgt
habe? \bibleverse{14}Und jetzt forderst du mich auf: ›Gehe hin und sage
deinem Herrn: Elia ist da!‹ Er würde mich ja umbringen!«
\bibleverse{15}Aber Elia entgegnete: »So wahr Gott, der HERR der
Heerscharen, lebt, in dessen Dienst ich stehe: noch heute will ich ihm
vor die Augen treten!«

\hypertarget{bb-elia-vor-ahab-berufung-der-guxf6tzenpropheten-auf-den-berg-karmel}{%
\subparagraph{bb) Elia vor Ahab; Berufung der Götzenpropheten auf den
Berg
Karmel}\label{bb-elia-vor-ahab-berufung-der-guxf6tzenpropheten-auf-den-berg-karmel}}

\bibleverse{16}Da ging Obadja dem Ahab entgegen und berichtete es ihm,
und Ahab machte sich auf, um mit Elia zusammenzutreffen.
\bibleverse{17}Sobald nun Ahab den Elia erblickte, rief er ihm zu: »Bist
du wirklich da, du Unglücksstifter für Israel?« \bibleverse{18}Er
antwortete: »Nicht ich bin es, der Israel ins Unglück gestürzt hat,
sondern du und dein Haus, weil ihr die Gebote des HERRN verlassen habt
und den Baalen nachgelaufen seid. \bibleverse{19}Nun aber sende hin und
laß ganz Israel bei mir auf dem Berge Karmel zusammenkommen, dazu die
vierhundertfünfzig Propheten Baals und die vierhundert Propheten der
Aschera, die vom Tisch der Isebel essen.« \bibleverse{20}Da sandte Ahab
Boten in alle Teile Israels und ließ die Propheten auf dem Berge Karmel
zusammenkommen.

\hypertarget{cc-das-gottesurteil-auf-dem-karmel-die-tuxf6tung-der-baalspropheten}{%
\subparagraph{cc) Das Gottesurteil auf dem Karmel; die Tötung der
Baalspropheten}\label{cc-das-gottesurteil-auf-dem-karmel-die-tuxf6tung-der-baalspropheten}}

\bibleverse{21}Da trat Elia vor das gesamte Volk hin und sagte: »Wie
lange wollt ihr nach beiden Seiten hinken? Wenn der HERR Gott ist, so
haltet euch zu ihm; ist es aber der Baal, so folgt diesem nach!« Aber
das Volk antwortete ihm kein Wort. \bibleverse{22}Hierauf sagte Elia zum
Volk: »Ich bin allein noch als Prophet des HERRN übriggeblieben, der
Propheten Baals dagegen sind vierhundertfünfzig Mann. \bibleverse{23}So
gebe man uns nun zwei Stiere; sie mögen sich dann einen von den Stieren
auswählen und ihn zerstücken und auf die Holzscheite legen, jedoch ohne
Feuer daranzubringen. Ich aber will den andern Stier zurichten und ihn
auf die Holzscheite legen, ebenfalls ohne Feuer daranzulegen.
\bibleverse{24}Dann ruft ihr den Namen eures Gottes an, während ich den
Namen des HERRN anrufen werde; und der Gott, der dann mit Feuer
antwortet, der soll als Gott gelten!« Da rief das ganze Volk: »Der
Vorschlag ist gut!« \bibleverse{25}Hierauf sagte Elia zu den Propheten
Baals: »Wählt euch einen von den Stieren aus und richtet ihn zuerst zu;
denn ihr seid in der Mehrzahl; ruft dann den Namen eures Gottes an, aber
ihr dürft kein Feuer daranlegen.« \bibleverse{26}Da nahmen sie den
Stier, dessen Wahl er ihnen freigestellt hatte, richteten ihn zu und
riefen den Namen Baals vom Morgen bis zum Mittag an, indem sie riefen:
»Baal, erhöre uns!«, aber es erfolgte kein Laut, und niemand antwortete.
Dabei tanzten sie um den Altar herum, den sie errichtet hatten.
\bibleverse{27}Als es nun Mittag geworden war, da verhöhnte Elia sie mit
den Worten: »Ruft recht laut, er ist ja doch ein Gott! Vielleicht ist er
eben in Gedanken versunken oder ist beiseite gegangen oder befindet sich
auf Reisen; vielleicht schläft er gar und muß erst aufwachen.«
\bibleverse{28}Da riefen sie recht laut und brachten sich nach ihrem
Brauch Wunden mit Schwertern und Spießen bei, bis das Blut an ihnen
herabfloß. \bibleverse{29}Als dann der Mittag vorüber war, gerieten sie
ins Rasen bis zur Zeit, da man das Speisopfer darzubringen pflegt; aber
kein Laut, keine Antwort und keine Erhörung war erfolgt.

\bibleverse{30}Nunmehr sagte Elia zu dem ganzen Volk: »Tretet zu mir
heran!« Als nun das ganze Volk zu ihm getreten war, stellte er den Altar
des HERRN, der niedergerissen worden war, wieder her; \bibleverse{31}er
nahm nämlich zwölf Steine nach der Zahl der Stämme der Söhne Jakobs --
an den einst das Wort des HERRN also ergangen war: »Israel soll dein
Name sein!«~-- \bibleverse{32}und baute von den Steinen einen Altar im
Namen des HERRN; alsdann zog er rings um den Altar einen Graben, der
einen Umfang hatte wie ein Feld für zwei Maß Aussaat.
\bibleverse{33}Hierauf schichtete er die Holzscheite auf, zerstückte den
Stier, legte ihn auf den Holzstoß \bibleverse{34}und sagte: »Füllet vier
Krüge mit Wasser und gießt es über das Brandopfer und über das Holz!«
Dann befahl er: »Wiederholt es noch einmal!« Da taten sie es noch
einmal. Hierauf befahl er: »Tut es zum drittenmal!« Da taten sie es zum
drittenmal, \bibleverse{35}so daß das Wasser rings um den Altar
herumlief; und auch den Graben ließ er mit Wasser füllen.

\hypertarget{dd-elias-gebet-von-gott-erhuxf6rt-abschlachtung-der-baalspfaffen}{%
\subparagraph{dd) Elias Gebet von Gott erhört; Abschlachtung der
Baalspfaffen}\label{dd-elias-gebet-von-gott-erhuxf6rt-abschlachtung-der-baalspfaffen}}

\bibleverse{36}Als dann die Zeit da war, wo man das Speisopfer
darzubringen pflegt, trat der Prophet Elia herzu und betete: »HERR, Gott
Abrahams, Isaaks und Israels, laß es heute kund werden, daß du Gott in
Israel bist und ich dein Knecht bin und daß ich dies alles nach deinem
Befehl getan habe. \bibleverse{37}Erhöre mich, HERR, erhöre mich, damit
dieses Volk erkennt, daß du, HERR, der wahre Gott bist und du selbst
ihre Herzen zur Umkehr gebracht hast!« \bibleverse{38}Da fiel das Feuwer
des HERRN herab und verzehrte das Brandopfer und das Holz, die Steine
und das Erdreich und leckte sogar das Wasser im Graben auf.
\bibleverse{39}Als das ganze Volk das sah, warfen sie sich auf ihr
Angesicht nieder und riefen aus: »Der HERR, er ist der wahre Gott! Der
HERR, er ist der wahre Gott!« \bibleverse{40}Elia aber befahl ihnen:
»Ergreift die Propheten Baals, laßt keinen von ihnen entrinnen!« Als man
sie nun ergriffen hatte, führte Elia sie an den Bach Kison hinab und
ließ sie dort abschlachten\textless sup title=``oder:
niedermetzeln''\textgreater✲.

\hypertarget{ee-schilderung-des-aufsteigenden-gewitters-ahabs-fahrt-und-elias-dauerlauf-nach-jesreel}{%
\subparagraph{ee) Schilderung des aufsteigenden Gewitters; Ahabs Fahrt
und Elias Dauerlauf nach
Jesreel}\label{ee-schilderung-des-aufsteigenden-gewitters-ahabs-fahrt-und-elias-dauerlauf-nach-jesreel}}

\bibleverse{41}Hierauf sagte Elia zu Ahab: »Gehe hinauf, iß und trink!
Denn ich höre schon das Rauschen des Regens.« \bibleverse{42}Während nun
Ahab hinaufging, um zu essen und zu trinken, stieg Elia der Spitze des
Karmels zu und kauerte sich tief zur Erde nieder, indem er sein Gesicht
zwischen seine Knie legte. \bibleverse{43}Dann befahl er seinem Diener:
»Steige höher hinauf, schaue aus nach dem Meere hin!« Der ging hinauf
und schaute aus und meldete: »Es ist nichts zu sehen.« Er antwortete:
»Gehe wieder hin!«, und so siebenmal. \bibleverse{44}Beim siebten Male
aber meldete er: »Soeben steigt eine Wolke, so klein wie eines Mannes
Hand, aus dem Meere auf!« Da befahl er ihm: »Gehe hin und sage zu Ahab:
›Laß anspannen und fahre hinab, damit dich der Regen nicht zurückhält!‹«
\bibleverse{45}Und es dauerte nicht lange, da wurde der Himmel schwarz
von Wolken und Sturm\textless sup title=``=~von
Gewitterwolken''\textgreater✲, und es erfolgte ein gewaltiger Regen;
Ahab aber bestieg den Wagen und fuhr nach Jesreel. \bibleverse{46}Über
Elia aber kam die Hand des HERRN, so daß er seine Lenden gürtete und vor
Ahab herlief bis nach Jesreel hin.

\hypertarget{c-flucht-elias-vor-isebel-und-die-gotteserscheinung-am-horeb-die-berufung-elias}{%
\paragraph{c) Flucht Elias vor Isebel und die Gotteserscheinung am
Horeb; die Berufung
Elias}\label{c-flucht-elias-vor-isebel-und-die-gotteserscheinung-am-horeb-die-berufung-elias}}

\hypertarget{aa-isebels-drohung-elias-verzagtheit-seine-stuxe4rkung-durch-einen-engel-und-seine-wanderung-nach-dem-horeb}{%
\subparagraph{aa) Isebels Drohung; Elias Verzagtheit; seine Stärkung
durch einen Engel und seine Wanderung nach dem
Horeb}\label{aa-isebels-drohung-elias-verzagtheit-seine-stuxe4rkung-durch-einen-engel-und-seine-wanderung-nach-dem-horeb}}

\hypertarget{section-18}{%
\section{19}\label{section-18}}

\bibleverse{1}Als nun Ahab der Isebel alles mitteilte, was Elia getan,
und vor allem, wie er alle Propheten mit dem Schwert umgebracht hatte,
\bibleverse{2}da schickte Isebel einen Boten an Elia und ließ ihm sagen:
»Die Götter sollen mich jetzt und künftig strafen, wenn ich nicht morgen
um diese Zeit mit deinem Leben ebenso verfahre, wie du mit dem Leben
eines jeden von ihnen verfahren bist!« \bibleverse{3}Da geriet er in
Furcht und machte sich schnell auf den Weg, um sein Leben zu retten.

Als er dann nach Beerseba, das schon zu Juda gehört, gekommen war, ließ
er seinen Diener dort zurück; \bibleverse{4}er selbst aber ging eine
Tagereise weit in die Wüste hinein; dort angekommen, setzte er sich
unter einem Ginsterstrauch nieder. Da wünschte er sich den Tod und
betete: »Es ist genug! Nimm nunmehr, HERR, mein Leben hin, denn ich bin
nicht besser als meine Väter.« \bibleverse{5}Hierauf legte er sich
nieder und schlief unter dem Ginsterstrauch ein; aber plötzlich rührte
ihn ein Engel an und sagte zu ihm: \bibleverse{6}»Stehe auf, iß!« Als er
nun hinblickte, sah er zu seinen Häupten einen auf heißen Steinen
gerösteten Brotkuchen liegen, und daneben stand ein Krug mit Wasser. Er
aß also und trank und legte sich wieder schlafen. \bibleverse{7}Aber der
Engel des HERRN kam zum zweitenmal wieder, rührte ihn an und sagte:
»Stehe auf, iß! Sonst ist der Weg für dich zu weit.« \bibleverse{8}Da
stand er auf, aß und trank und wanderte, durch diese Speise gestärkt,
vierzig Tage und vierzig Nächte lang bis zum Gottesberge Horeb,
\bibleverse{9}wo er in eine Höhle ging und darin über Nacht blieb.

\hypertarget{bb-die-gottesoffenbarung-am-horeb}{%
\subparagraph{bb) Die Gottesoffenbarung am
Horeb}\label{bb-die-gottesoffenbarung-am-horeb}}

Da nun erging an ihn das Wort des HERRN, der zu ihm sagte: »Was willst
du hier, Elia?« \bibleverse{10}Er antwortete: »Ich habe für den HERRN,
den Gott der Heerscharen, unerschrocken geeifert; denn die Israeliten
haben deinen Bund verlassen, deine Altäre niedergerissen und deine
Propheten mit dem Schwert getötet; ich allein bin übriggeblieben, und
nun trachten sie auch mir nach dem Leben.« \bibleverse{11}Da erwiderte
er: »Gehe hinaus und tritt auf dem Berge vor den HERRN hin!« Und siehe,
der HERR zog an ihm vorüber: ein Sturmwind, gewaltig und stark, der die
Berge zerriß und die Felsen spaltete, ging vor dem HERRN her; aber der
HERR war nicht in dem Sturme. Nach dem Sturm kam ein Erdbeben: aber der
HERR war nicht in dem Erdbeben; \bibleverse{12}und nach dem Erdbeben kam
ein Feuer: aber der HERR war nicht in dem Feuer. Nach dem Feuer aber kam
ein leises, sanftes Säuseln. \bibleverse{13}Als Elia dieses hörte,
verhüllte er sich das Antlitz mit seinem Mantel, ging hinaus und trat an
den Eingang der Höhle. Da redete ihn eine Stimme an, die fragte: »Was
willst du hier, Elia?« \bibleverse{14}Er antwortete: »Ich habe für den
HERRN, den Gott der Heerscharen, unerschrocken geeifert; denn die
Israeliten haben deinen Bund verlassen, deine Altäre niedergerissen und
deine Propheten mit dem Schwert getötet; ich allein bin übriggeblieben,
und nun trachten sie auch mir nach dem Leben.«

\hypertarget{cc-elia-erhuxe4lt-den-befehl-die-werkzeuge-der-guxf6ttlichen-rache-hasael-jehu-elisa-bereitzumachen}{%
\subparagraph{cc) Elia erhält den Befehl, die Werkzeuge der göttlichen
Rache (Hasael, Jehu, Elisa)
bereitzumachen}\label{cc-elia-erhuxe4lt-den-befehl-die-werkzeuge-der-guxf6ttlichen-rache-hasael-jehu-elisa-bereitzumachen}}

\bibleverse{15}Da sagte der HERR zu ihm: »Kehre jetzt auf demselben Wege
nach der Steppe von Damaskus zurück, gehe in die Stadt hinein und salbe
Hasael zum König über Syrien; \bibleverse{16}und Jehu, den Sohn Nimsis,
sollst du zum König über Israel salben, und Elisa, den Sohn Saphats, aus
Abel-Mehola, sollst du zum Propheten an deiner Statt salben.
\bibleverse{17}Wer dann dem Schwert Hasaels entrinnt, den wird Jehu
töten, und wer dem Schwert Jehus entrinnt, den wird Elisa töten.
\bibleverse{18}Doch will ich in Israel siebentausend (Männer)
übriglassen: alle, deren Knie sich vor dem Baal nicht gebeugt haben, und
alle, deren Mund ihn\textless sup title=``d.h. sein
Götzenbild''\textgreater✲ nicht geküßt hat.«

\hypertarget{dd-die-berufung-elisas}{%
\subparagraph{dd) Die Berufung Elisas}\label{dd-die-berufung-elisas}}

\bibleverse{19}Als Elia nun von dort weggegangen war, traf er Elisa, den
Sohn Saphats, der gerade pflügte; zwölf Joch\textless sup title=``oder:
Gespanne''\textgreater✲ Ochsen waren vor ihm her, er selbst aber befand
sich bei dem zwölften. Während nun Elia an ihm vorüberschritt, warf er
ihm sein Fell\textless sup title=``oder: seinen haarigen
Mantel''\textgreater✲ über. \bibleverse{20}Da verließ er die Rinder,
eilte dem Elia nach und sagte: »Laß mich nur noch Abschied von meinem
Vater und meiner Mutter nehmen; dann will ich dir nachfolgen.« Elia
antwortete ihm: »Gehe immerhin noch einmal zurück; denn was habe ich dir
getan?« \bibleverse{21}Da kehrte er von ihm zurück, nahm das eine Joch
Rinder und schlachtete es, mit dem Geschirr der Rinder aber kochte er
ihr Fleisch und gab es den Leuten zu essen; dann machte er sich auf den
Weg, schloß sich an Elia an und wurde sein Diener.

\hypertarget{d-zweimaliger-sieg-ahabs-uxfcber-die-syrer}{%
\paragraph{d) Zweimaliger Sieg Ahabs über die
Syrer}\label{d-zweimaliger-sieg-ahabs-uxfcber-die-syrer}}

\hypertarget{aa-benhadad-belagert-samaria-ahabs-anfuxe4ngliche-schwuxe4che-sodann-entschiedenes-auftreten}{%
\subparagraph{aa) Benhadad belagert Samaria; Ahabs anfängliche Schwäche,
sodann entschiedenes
Auftreten}\label{aa-benhadad-belagert-samaria-ahabs-anfuxe4ngliche-schwuxe4che-sodann-entschiedenes-auftreten}}

\hypertarget{section-19}{%
\section{20}\label{section-19}}

\bibleverse{1}Benhadad aber, der König von Syrien, bot seine ganze
Streitmacht auf, zweiunddreißig Könige leisteten ihm Heeresfolge mit
Rossen und Wagen; so zog er heran, belagerte Samaria und bestürmte es.
\bibleverse{2}Dann schickte er Gesandte in die Stadt zu Ahab, dem König
von Israel, \bibleverse{3}und ließ ihm sagen: »So spricht Benhadad:
›Dein Silber und Gold gehört mir, und deine Frauen und deine
{[}schönsten{]} Kinder gehören ebenfalls mir.‹« \bibleverse{4}Der König
von Israel gab ihm zur Antwort: »Wie du befiehlst, mein Herr und König:
dein bin ich mit allem, was mir gehört.« \bibleverse{5}Darauf kamen die
Gesandten noch einmal und sagten: »So spricht Benhadad: ›Ich habe zu dir
gesandt und dir sagen lassen: Dein Silber und dein Gold, deine Frauen
und deine Kinder sollst du mir geben; \bibleverse{6}morgen also um diese
Zeit werde ich meine Diener zu dir senden, damit sie deinen Palast und
die Häuser deiner Diener durchsuchen; und sie sollen dann alles, was
ihnen beachtenswert erscheint, mit Beschlag belegen und mitnehmen.‹«

\bibleverse{7}Da berief der König von Israel alle Ältesten des Landes
und sagte: »Da könnt ihr nun klar erkennen, wie böse dieser Mensch es
meint; denn als er zu mir sandte, um meine Frauen und meine Kinder, mein
Silber und mein Gold zu fordern, habe ich es ihm nicht verweigert.«
\bibleverse{8}Da antworteten ihm alle Ältesten und das gesamte Volk: »Da
gehorchst du nicht und willigst nicht ein!« \bibleverse{9}So erwiderte
er denn den Gesandten Benhadads: »Meldet meinem Herrn, dem Könige:
›Alles, was du von deinem Knecht zuerst verlangt hast, will ich tun,
aber auf diese letzte Forderung kann ich nicht eingehen.‹« Als nun die
Gesandten hingegangen waren und den Bescheid überbracht hatten,
\bibleverse{10}sandte Benhadad zu ihm und ließ ihm sagen: »Die Götter
sollen mich jetzt und künftig strafen, wenn der Schutt von Samaria
hinreicht, allen Kriegern, an deren Spitze ich stehe, die hohlen Hände
zu füllen!« \bibleverse{11}Darauf gab ihm der König von Israel zur
Antwort: »Sagt ihm nur: ›Wer sich das Schwert umgürtet, darf sich nicht
brüsten, als ob er es schon wieder ablegte.‹« \bibleverse{12}Als
Benhadad diesen Bescheid vernahm, während er gerade mit den Königen in
den Weinberglauben bei einem Trinkgelage zechte, rief er seinen Leuten
zu: »Zum Angriff!«, und sie stellten sich zum Angriff gegen die Stadt
auf.

\hypertarget{bb-die-weisungen-eines-propheten-an-ahab-grouxdfer-sieg-der-israeliten-uxfcber-benhadad}{%
\subparagraph{bb) Die Weisungen eines Propheten an Ahab; großer Sieg der
Israeliten über
Benhadad}\label{bb-die-weisungen-eines-propheten-an-ahab-grouxdfer-sieg-der-israeliten-uxfcber-benhadad}}

\bibleverse{13}Da trat plötzlich ein Prophet zu Ahab, dem König von
Israel, und sagte: »So hat der HERR gesprochen: ›Siehst du diesen ganzen
gewaltigen Heerhaufen? Wisse wohl: ich gebe ihn dir heute in die Hand,
damit du erkennst, daß ich der HERR bin!‹« \bibleverse{14}Ahab fragte:
»Durch wen?« Er antwortete: »So hat der HERR gesprochen: ›Durch die
Leute der Landvögte.‹« Da fragte er weiter: »Wer soll den Kampf
eröffnen?« Er erwiderte: »Du selbst.« \bibleverse{15}Darauf musterte
Ahab die Leute der Landvögte: es waren ihrer 232 Mann. Nach ihnen
musterte er das gesamte übrige Kriegsvolk, alle Israeliten: es waren
7000~Mann. \bibleverse{16}Sie machten dann zur Mittagszeit einen
Ausfall, als Benhadad sich gerade in den Lauben samt den zweiunddreißig
mit ihm verbündeten Königen einen Rausch antrank. \bibleverse{17}Als nun
die Leute der Landvögte als Vortrab ausrückten und Benhadad durch
Kundschafter, die er ausgesandt hatte, die Meldung erhielt, daß Leute
aus Samaria ausgerückt seien, \bibleverse{18}befahl er: »Mögen sie in
friedlicher oder in feindlicher Absicht ausgerückt sein: nehmt sie
lebendig gefangen!« \bibleverse{19}Sobald aber jene aus der Stadt
ausgerückt waren, nämlich die Leute der Landvögte und das ihnen folgende
Heer, \bibleverse{20}schlugen sie wütend drein, so daß die Syrer flohen.
Die Israeliten verfolgten sie, doch Benhadad, der König von Syrien,
rettete sich zu Pferde mit einigen Reitern. \bibleverse{21}Nun rückte
auch der König von Israel selber aus, fiel über die Rosse und Wagen her
und brachte den Syrern eine schwere Niederlage bei.

\hypertarget{cc-benhadads-zweiter-feldzug-neue-weisung-des-propheten-an-ahab-beratung-der-syrer-sieg-der-israeliten}{%
\subparagraph{cc) Benhadads zweiter Feldzug; neue Weisung des Propheten
an Ahab; Beratung der Syrer; Sieg der
Israeliten}\label{cc-benhadads-zweiter-feldzug-neue-weisung-des-propheten-an-ahab-beratung-der-syrer-sieg-der-israeliten}}

\bibleverse{22}Da trat der Prophet (wiederum) zum König von Israel und
sagte zu ihm: »Wohlan, nimm dich zusammen und sieh wohl zu, was du zu
tun hast! Denn übers Jahr wird der König von Syrien wieder gegen dich
heranziehen.«

\bibleverse{23}Die Heerführer des Königs von Syrien aber sagten zu ihm:
»Ihr Gott ist ein Berggott, darum sind sie uns überlegen gewesen; wenn
wir dagegen in der Ebene mit ihnen kämpfen könnten, würden wir sie gewiß
besiegen. \bibleverse{24}Gehe also folgendermaßen zu Werke: Entferne die
Könige sämtlich von ihren Stellen und ersetze sie durch Statthalter;
\bibleverse{25}sodann biete ein ebenso großes Heer auf, wie das vorige
war, das dir verlorengegangen ist, ebenso Rosse und Wagen wie das vorige
Mal; dann wollen wir mit ihnen in der Ebene kämpfen, so werden wir sie
gewiß besiegen.« Benhadad ging auf ihren Vorschlag ein und verfuhr
demgemäß.

\bibleverse{26}Als dann das Jahr um war, bot Benhadad die Syrer auf und
zog nach Aphek, um dort den Israeliten eine Schlacht zu liefern.
\bibleverse{27}Nachdem auch die Israeliten gemustert worden waren und
sich mit Mundvorrat versorgt hatten, zogen sie ihnen entgegen und
lagerten sich ihnen gegenüber wie ein Paar kleine Ziegenherden, während
die Syrer die ganze Gegend füllten. \bibleverse{28}Da trat der
Gottesmann abermals herzu und sagte zum König von Israel: »So hat der
HERR gesprochen: ›Weil die Syrer gesagt haben, der HERR sei ein Gott der
Berge, aber kein Gott der Ebenen, so will ich diesen ganzen gewaltigen
Heerhaufen in deine Hand geben, damit ihr erkennt, daß ich der HERR
bin.‹« \bibleverse{29}So lagerten sie denn sieben Tage lang einander
gegenüber; am siebten Tage aber kam es zur Schlacht, und die Israeliten
erschlugen von den Syrern hunderttausend Mann Fußvolk an einem einzigen
Tage; \bibleverse{30}die Übriggebliebenen flüchteten sich in die Stadt
Aphek. Da fiel die Stadtmauer über den 27000~Mann zusammen, die
übriggeblieben waren.

\hypertarget{dd-benhadad-in-aphek-belagert-und-zur-ergebung-gezwungen-ahabs-unbesonnene-milde-gegen-ihn}{%
\subparagraph{dd) Benhadad in Aphek belagert und zur Ergebung gezwungen;
Ahabs unbesonnene Milde gegen
ihn}\label{dd-benhadad-in-aphek-belagert-und-zur-ergebung-gezwungen-ahabs-unbesonnene-milde-gegen-ihn}}

Auch Benhadad war geflohen und in die Stadt gekommen (und versteckte
sich) von einem Gemach in das andere. \bibleverse{31}Da sagten seine
Diener zu ihm: »Wir haben oft genug gehört, daß die Könige der
Israeliten großmütige Könige seien. So wollen wir uns denn Bußgewänder
um unsere Lenden legen und Stricke um unseren Kopf und so zum König von
Israel hinausgehen; vielleicht schenkt er dir das Leben.«
\bibleverse{32}Sie gürteten sich also Bußgewänder um ihre Lenden und
legten Stricke um ihren Kopf; und als sie zum König von Israel kamen,
sagten sie: »Dein Knecht Benhadad läßt dich bitten, ihm das Leben zu
schenken.« Er antwortete: »Lebt er noch? Er ist ja mein Bruder.«
\bibleverse{33}Die Männer nahmen das als eine gute Vorbedeutung; darum
beeilten sie sich, ihn beim Wort zu nehmen, und sagten: »Benhadad ist
also dein Bruder?« Da antwortete er: »Geht, bringt ihn her!«, und als
Benhadad dann zu ihm hinauskam, ließ er ihn zu sich auf den Wagen
steigen. \bibleverse{34}Da sagte Benhadad zu ihm: »Die Städte, die mein
Vater deinem Vater weggenommen hat, will ich dir zurückgeben; auch magst
du dir Handelshäuser\textless sup title=``oder: Kaufläden''\textgreater✲
in Damaskus anlegen, wie mein Vater sich solche in Samaria angelegt
hat.« »Und ich«, entgegnete Ahab, »will dich auf Grund dieses Abkommens
freilassen.« Er schloß also einen Vertrag mit ihm und ließ ihn frei.

\hypertarget{ee-ein-prophetenschuxfcler-stellt-dem-ahab-seine-verfehlung-vor-augen}{%
\subparagraph{ee) Ein Prophetenschüler stellt dem Ahab seine Verfehlung
vor
Augen}\label{ee-ein-prophetenschuxfcler-stellt-dem-ahab-seine-verfehlung-vor-augen}}

\bibleverse{35}Einer aber von den Prophetenschülern forderte auf
Geheiß\textless sup title=``=~auf eine Eingebung''\textgreater✲ des
HERRN seinen Genossen auf: »Bringe mir eine Wunde bei!« Als jener sich
weigerte, ihn zu verwunden, \bibleverse{36}sagte er zu ihm: »Zur Strafe
dafür, daß du dem Befehl des HERRN nicht nachgekommen bist, wird dich,
sobald du von mir weggegangen bist, ein Löwe töten.« Kaum war er dann
von ihm weggegangen, da fiel ein Löwe ihn an und tötete ihn.
\bibleverse{37}Als jener hierauf einen anderen (Genossen) traf, forderte
er ihn (wieder) auf: »Bringe mir eine Wunde bei!« Da verwundete ihn
dieser durch einen Schlag. \bibleverse{38}Nun ging der Prophet hin und
stellte sich auf den Weg, den der König kommen mußte, machte sich aber
durch eine Binde über seinen Augen unkenntlich. \bibleverse{39}Als dann
der König vorüberkam, rief er ihn laut an mit den Worten: »Dein Knecht
war mit in die Schlacht gezogen; da trat plötzlich ein Mann, ein Oberer,
an mich heran und brachte mir einen Gefangenen mit der Weisung: ›Bewache
diesen Mann! Sollte er dir abhanden kommen, so haftest du mit deinem
Leben für ihn, oder du hast ein Talent Silber zu bezahlen.‹
\bibleverse{40}Während nun dein Knecht hier und da zu tun hatte, war der
Mensch verschwunden.« Da sagte der König von Israel zu ihm: »Da hast du
dein Urteil; du hast es dir selbst gesprochen.« \bibleverse{41}Da
entfernte jener schnell die Binde von seinen Augen, und der König von
Israel erkannte in ihm einen von den Propheten. \bibleverse{42}Der aber
sagte zu ihm: »So hat der HERR gesprochen: ›Weil du den Mann, der von
mir dem Tode geweiht war, aus der Hand gelassen hast, mußt du mit deinem
Leben für ihn haften und dein Volk für sein Volk.‹« \bibleverse{43}Da
zog der König von Israel mißmutig und verstört nach Hause und gelangte
nach Samaria.

\hypertarget{e-ahabs-schuxe4ndliche-gewalttat-an-naboth}{%
\paragraph{e) Ahabs schändliche Gewalttat an
Naboth}\label{e-ahabs-schuxe4ndliche-gewalttat-an-naboth}}

\hypertarget{aa-ahabs-miuxdfmut-uxfcber-naboth}{%
\subparagraph{aa) Ahabs Mißmut über
Naboth}\label{aa-ahabs-miuxdfmut-uxfcber-naboth}}

\hypertarget{section-20}{%
\section{21}\label{section-20}}

\bibleverse{1}Nachmals aber begab sich folgendes: Naboth, (ein Bürger)
von Jesreel, besaß einen Weinberg in Jesreel nahe bei dem Palaste Ahabs,
des Königs von Samaria. \bibleverse{2}Ahab machte nun dem Naboth
folgenden Vorschlag: »Tritt mir deinen Weinberg ab, damit ich mir einen
Gemüsegarten daraus mache, weil er nahe bei meinem Palaste liegt; ich
will dir einen besseren Weinberg dafür geben oder, wenn dir das lieber
ist, dir den Preis bar bezahlen.« \bibleverse{3}Aber Naboth erwiderte
dem Ahab: »Der HERR bewahre mich davor, dir den Erbbesitz meiner Väter
abzutreten!« \bibleverse{4}Da kehrte Ahab in seinen Palast zurück,
mißmutig und verstört über die Antwort, die Naboth, der Jesreeliter, ihm
gegeben hatte mit den Worten: »Ich will dir den Erbbesitz meiner Väter
nicht abtreten.« Er legte sich auf sein Bett, wandte das Gesicht gegen
die Wand ab und wollte keine Nahrung zu sich nehmen.

\hypertarget{bb-isebels-unheilvolles-eingreifen-ihr-nichtswuxfcrdiger-brief-an-die-uxe4ltesten-der-stadt-jesreel}{%
\subparagraph{bb) Isebels unheilvolles Eingreifen; ihr nichtswürdiger
Brief an die Ältesten der Stadt
Jesreel}\label{bb-isebels-unheilvolles-eingreifen-ihr-nichtswuxfcrdiger-brief-an-die-uxe4ltesten-der-stadt-jesreel}}

\bibleverse{5}Da trat seine Gemahlin Isebel zu ihm und fragte ihn:
»Warum bist du denn so mißmutig, daß du nichts essen willst?«
\bibleverse{6}Er antwortete ihr: »Ich habe Naboth von Jesreel den
Vorschlag gemacht: ›Tritt mir deinen Weinberg gegen Bezahlung ab, oder,
wenn du das vorziehst, will ich dir einen anderen Weinberg dafür geben‹;
aber er hat mir geantwortet: ›Ich will dir meinen Weinberg nicht
abtreten.‹« \bibleverse{7}Da erwiderte ihm seine Gemahlin Isebel: »Jetzt
mußt du zeigen, daß du König in Israel bist! Stehe auf, iß und sei guten
Muts: ich will dir den Weinberg des Jesreeliters Naboth schon
verschaffen.« \bibleverse{8}Darauf schrieb sie einen Brief unter Ahabs
Namen, versiegelte ihn mit seinem Siegel und schickte den Brief an die
Ältesten und Obersten\textless sup title=``oder:
Vornehmsten''\textgreater✲, die in seiner Stadt wohnten und Mitbeisitzer
Naboths waren. \bibleverse{9}In dem Briefe schrieb sie folgendes: »Laßt
ein Fasten ausrufen und setzt Naboth obenan unter dem Volke!
\bibleverse{10}Dann setzt zwei nichtswürdige Menschen ihm gegenüber, die
gegen ihn auftreten und das Zeugnis ablegen sollen, er habe Gott und den
König gelästert; dann führt ihn vor die Stadt hinaus und steinigt ihn zu
Tode!«

\hypertarget{cc-ruchlose-ermordung-naboths-ahabs-gewaltsame-besitznahme-des-weinbergs}{%
\subparagraph{cc) Ruchlose Ermordung Naboths; Ahabs gewaltsame
Besitznahme des
Weinbergs}\label{cc-ruchlose-ermordung-naboths-ahabs-gewaltsame-besitznahme-des-weinbergs}}

\bibleverse{11}Und die Männer seiner Stadt, die Ältesten und Vornehmen,
die in seiner Stadt die Beisitzer waren, taten, wie Isebel ihnen in dem
Briefe, den sie ihnen hatte zugehen lassen, aufgetragen hatte:
\bibleverse{12}sie ließen ein Fasten ausrufen und setzten Naboth obenan
unter dem Volke; \bibleverse{13}dann kamen die zwei nichtswürdigen
Buben, setzten sich ihm gegenüber und legten vor dem Volke das Zeugnis
gegen ihn ab: »Naboth hat Gott und den König gelästert!« Hierauf führte
man ihn zur Stadt hinaus und steinigte ihn; so fand er seinen Tod.
\bibleverse{14}Dann sandten sie zu Isebel und ließen ihr melden, Naboth
sei gesteinigt worden und sei nun tot. \bibleverse{15}Sobald Isebel die
Nachricht von Naboths Steinigung und Tod erhalten hatte, sagte sie zu
Ahab: »Auf! Nimm den Weinberg des Jesreeliters Naboth, den er dir für
Geld nicht hat überlassen wollen, in Besitz! Denn Naboth lebt nicht
mehr, sondern ist tot.« \bibleverse{16}Sobald nun Ahab den Tod Naboths
erfuhr, machte er sich auf den Weg, um in den Weinberg Naboths
hinabzugehen und ihn in Besitz zu nehmen.

\hypertarget{dd-elia-kuxfcndigt-dem-kuxf6nigspaar-das-guxf6ttliche-strafgericht-an}{%
\subparagraph{dd) Elia kündigt dem Königspaar das göttliche Strafgericht
an}\label{dd-elia-kuxfcndigt-dem-kuxf6nigspaar-das-guxf6ttliche-strafgericht-an}}

\bibleverse{17}Da erging das Wort des HERRN an Elia, den Thisbiter,
also: \bibleverse{18}»Mache dich auf, gehe hinab und tritt vor Ahab, den
König von Israel, der in Samaria wohnt✲; er befindet sich gerade im
Weinberge Naboths, wohin er hinabgegangen ist, um ihn in Besitz zu
nehmen. \bibleverse{19}Sage dann zu ihm folgendes: ›So hat der HERR
gesprochen: Du hast gemordet und hast nun auch schon die Erbschaft
angetreten?‹ Sage dann weiter zu ihm: ›So hat der HERR gesprochen: An
der Stelle, wo die Hunde das Blut Naboths geleckt haben, sollen die
Hunde auch dein Blut lecken!‹«\textless sup title=``vgl.
22,38''\textgreater✲ \bibleverse{20}Da sagte Ahab zu Elia: »Hast du mich
wieder herausgefunden, mein Feind?« Er antwortete: »Ja, ich habe dich
herausgefunden. Weil du dich so weit vergessen hast, zu tun, was dem
HERRN mißfällt: \bibleverse{21}›so will ich nunmehr Unglück über dich
bringen und dich wegfegen und will von Ahabs Angehörigen alles
ausrotten, was männlichen Geschlechts ist, Unmündige wie Mündige, in
Israel; \bibleverse{22}und ich will es mit deinem Hause machen wie mit
dem Hause Jerobeams, des Sohnes Nebats, und wie mit dem Hause Baesas,
des Sohnes Ahias, weil du mich zum Zorn gereizt und Israel zur Sünde
verführt hast!‹ \bibleverse{23}Auch in betreff Isebels hat der HERR
geredet und angekündigt: ›Die Hunde sollen Isebel am
Stadtgraben\textless sup title=``oder: auf der Feldmark''\textgreater✲
von Jesreel fressen\textless sup title=``vgl. 2.Kön
9,33-36''\textgreater✲. \bibleverse{24}Wer von Ahabs Angehörigen in der
Stadt stirbt, den sollen die Hunde fressen, und wer auf dem Felde
stirbt, den sollen die Vögel des Himmels fressen!‹« \bibleverse{25}Es
hat niemals einen Menschen gegeben, der sich so weit wie Ahab vergessen
hätte, um das zu tun, was dem HERRN mißfällt, weil ihn sein Weib Isebel
dazu verführte. \bibleverse{26}Er verübte zahllose Greuel, indem er den
Götzen nachging, ganz wie die Amoriter es getan hatten, die der HERR vor
den Israeliten vertrieben hatte.

\hypertarget{ee-ahabs-reue-gottes-milderung-der-unheilsdrohung}{%
\subparagraph{ee) Ahabs Reue; Gottes Milderung der
Unheilsdrohung}\label{ee-ahabs-reue-gottes-milderung-der-unheilsdrohung}}

\bibleverse{27}Als nun Ahab diese Worte hörte, zerriß er seine Kleider,
legte ein härenes Bußgewand um den bloßen Leib und fastete; er schlief
sogar in dem Bußgewand und ging bekümmert einher. \bibleverse{28}Da
erging das Wort des HERRN an Elia, den Thisbiter, also:
\bibleverse{29}»Hast du gesehen, wie Ahab sich vor mir gedemütigt hat?
Weil er sich vor mir gedemütigt hat, will ich das Unglück nicht schon
bei seinen Lebzeiten hereinbrechen lassen: erst unter der Regierung
seines Sohnes will ich das Unglück über sein Haus kommen lassen.«

\hypertarget{f-ahabs-und-josaphats-ungluxfccklicher-feldzug-gegen-ramoth-ahabs-tod}{%
\paragraph{f) Ahabs und Josaphats unglücklicher Feldzug gegen Ramoth;
Ahabs
Tod}\label{f-ahabs-und-josaphats-ungluxfccklicher-feldzug-gegen-ramoth-ahabs-tod}}

\hypertarget{aa-ahab-und-josaphat-verbuxfcnden-sich-zum-kriege-gegen-syrien}{%
\subparagraph{aa) Ahab und Josaphat verbünden sich zum Kriege gegen
Syrien}\label{aa-ahab-und-josaphat-verbuxfcnden-sich-zum-kriege-gegen-syrien}}

\hypertarget{section-21}{%
\section{22}\label{section-21}}

\bibleverse{1}Drei Jahre vergingen ruhig, ohne daß Krieg zwischen Syrien
und Israel stattfand. \bibleverse{2}Im dritten Jahre aber, als Josaphat,
der König von Juda, zum König von Israel (auf Besuch) gekommen war,
\bibleverse{3}sagte der König von Israel zu seinen Dienern: »Ihr wißt
doch wohl, daß Ramoth in Gilead uns gehört? Wir aber sitzen hier müßig,
anstatt es dem König von Syrien zu entreißen!« \bibleverse{4}Als er
hierauf an Josaphat die Frage richtete: »Willst du mit mir gegen Ramoth
in Gilead zu Felde ziehen?«, antwortete ihm Josaphat: »Ich will sein wie
du: mein Volk\textless sup title=``oder: Heer''\textgreater✲ wie dein
Volk\textless sup title=``oder: Heer''\textgreater✲, meine Rosse wie
deine Rosse!«

\hypertarget{bb-der-guxfcnstige-bescheid-der-400-propheten-micha-soll-befragt-werden}{%
\subparagraph{bb) Der günstige Bescheid der 400 Propheten; Micha soll
befragt
werden}\label{bb-der-guxfcnstige-bescheid-der-400-propheten-micha-soll-befragt-werden}}

\bibleverse{5}Als Josaphat dann dem König von Israel riet, zunächst doch
den Willen des HERRN zu erforschen, \bibleverse{6}ließ der König von
Israel die Propheten zusammenkommen, ungefähr vierhundert Mann, und
fragte sie: »Soll ich gegen Ramoth in Gilead zu Felde ziehen, oder soll
ich es unterlassen?« Sie antworteten: »Ziehe hin, denn der HERR wird es
dem König in die Hand geben.« \bibleverse{7}Da fragte Josaphat: »Gibt es
hier sonst keinen Propheten des HERRN mehr, durch den wir Auskunft
erhalten könnten?« \bibleverse{8}Der König von Israel erwiderte dem
Josaphat: »Es ist wohl noch einer da, durch den wir den HERRN befragen
könnten, aber ich habe nicht gern mit ihm zu tun; denn er weissagt mir
niemals Gutes, sondern immer nur Unglück: Micha, der Sohn Jimlas.« Aber
Josaphat entgegnete: »Der König wolle nicht so reden!«

\bibleverse{9}Da rief der König von Israel einen Kammerherrn und befahl
ihm, schleunigst Micha, den Sohn Jimlas, zu holen.
\bibleverse{10}Während nun der König von Israel und der König Josaphat
von Juda ein jeder auf seinem Throne in ihren Königsgewändern auf dem
Platz am Eingang des Stadttores von Samaria dasaßen und alle Propheten
vor ihnen weissagten, \bibleverse{11}machte sich Zedekia, der Sohn
Kenaanas, eiserne Hörner und rief aus: »So spricht der HERR: ›Mit
solchen (Hörnern) wirst du die Syrer niederstoßen, bis du sie vernichtet
hast!‹« \bibleverse{12}Ebenso weissagten auch alle anderen Propheten,
indem sie riefen: »Ziehe hin gegen Ramoth in Gilead: du wirst Glück
haben! Denn der HERR wird es dem König in die Hand fallen lassen.«

\hypertarget{cc-michas-anfuxe4nglicher-gluxfccksspruch-sodann-seine-unheilsverkuxfcndigung}{%
\subparagraph{cc) Michas anfänglicher Glücksspruch, sodann seine
Unheilsverkündigung}\label{cc-michas-anfuxe4nglicher-gluxfccksspruch-sodann-seine-unheilsverkuxfcndigung}}

\bibleverse{13}Der Bote aber, der hingegangen war, um Micha zu holen,
sagte zu ihm: »Wisse: die (übrigen) Propheten haben dem Könige
einstimmig Glück verheißen; schließe dich doch ihrem einmütigen
Ausspruch an und prophezeie ebenfalls Glück!« \bibleverse{14}Micha aber
antwortete: »So wahr der HERR lebt: nur was der HERR mir eingeben wird,
das werde ich verkünden!« \bibleverse{15}Als er nun zum Könige kam,
fragte dieser ihn: »Micha, sollen wir gegen Ramoth in Gilead zu Felde
ziehen, oder sollen wir es unterlassen?« Er antwortete ihm: »Ziehe hin,
du wirst Glück haben! Denn der HERR wird es dem Könige in die Hand
fallen lassen.« \bibleverse{16}Da entgegnete ihm der König: »Wie oft
soll ich dich noch beschwören, mir nichts zu verkünden als nur die reine
Wahrheit im Namen des HERRN?« \bibleverse{17}Da sagte Micha: »Ich habe
ganz Israel zerstreut auf den Bergen gesehen wie Schafe, die keinen
Hirten haben; der HERR aber sagte: ›Diese haben keinen Herrscher mehr:
ein jeder von ihnen möge in Frieden nach Hause zurückkehren!‹«
\bibleverse{18}Da sagte der König von Israel zu Josaphat: »Habe ich dir
nicht gesagt, daß er mir niemals Glück, sondern immer nur Unheil
prophezeit?« \bibleverse{19}Micha aber fuhr fort: »Darum vernimm das
Wort des HERRN! Ich habe den HERRN auf seinem Throne sitzen sehen,
während das ganze himmlische Heer zur Rechten und zur Linken neben ihm
stand. \bibleverse{20}Und der HERR fragte: ›Wer will Ahab betören, daß
er zu Felde ziehe und bei Ramoth in Gilead falle?‹ Da erwiderte der eine
dies, der andere das, \bibleverse{21}bis endlich der\textless sup
title=``oder: ein''\textgreater✲ Geist vortrat und sich vor den HERRN
stellte und sagte: ›Ich will ihn betören.‹ Der HERR fragte ihn: ›Auf
welche Weise?‹ \bibleverse{22}Da antwortete er: ›Ich will hingehen und
zum Lügengeist im Munde aller seiner Propheten werden.‹ Da sagte der
HERR: ›Du sollst ihn betören, und es wird dir auch gelingen: gehe hin
und mache es so.‹ \bibleverse{23}Nun denn, siehe, der HERR hat allen
diesen deinen Propheten einen Lügengeist in den Mund gelegt; denn der
HERR hat Unglück für dich beschlossen.«

\hypertarget{dd-michas-miuxdfhandlung-durch-zedekia-und-seine-gefangennahme-durch-ahab}{%
\subparagraph{dd) Michas Mißhandlung durch Zedekia und seine
Gefangennahme durch
Ahab}\label{dd-michas-miuxdfhandlung-durch-zedekia-und-seine-gefangennahme-durch-ahab}}

\bibleverse{24}Da trat Zedekia, der Sohn Kenaanas, auf Micha zu und gab
ihm einen Backenstreich mit den Worten: »Wie? Ist etwa der Geist des
HERRN von mir gewichen, um mit✲ dir zu reden?« \bibleverse{25}Micha
entgegnete: »Du wirst es an jenem Tage erfahren, an dem du dich aus
einem Gemach in das andere begeben wirst, um dich zu verstecken.«
\bibleverse{26}Hierauf befahl der König von Israel (dem Kammerherrn):
»Nimm Micha fest und führe ihn zu dem Stadthauptmann Amon und zu dem
königlichen Prinzen Joas zurück \bibleverse{27}und melde dort: ›So hat
der König befohlen: Setzt diesen Menschen ins Gefängnis und erhaltet ihn
notdürftig mit Brot und Wasser am Leben, bis ich wohlbehalten
heimkehre!‹« \bibleverse{28}Micha antwortete: »Wenn du wirklich
wohlbehalten heimkehrst, dann hat der HERR nicht in mir\textless sup
title=``oder: durch mich''\textgreater✲ geredet.« Er fügte dann noch
hinzu: »Hört dies, ihr Völker alle!«

\hypertarget{ee-niederlage-der-verbuxfcndeten-bei-ramoth-ahabs-tod-in-der-schlacht-seine-bestattung-in-samaria-schluuxdfwort}{%
\subparagraph{ee) Niederlage der Verbündeten bei Ramoth; Ahabs Tod in
der Schlacht; seine Bestattung in Samaria;
Schlußwort}\label{ee-niederlage-der-verbuxfcndeten-bei-ramoth-ahabs-tod-in-der-schlacht-seine-bestattung-in-samaria-schluuxdfwort}}

\bibleverse{29}Als hierauf der König von Israel und der König Josaphat
von Juda gegen Ramoth in Gilead zu Felde gezogen waren,
\bibleverse{30}sagte der König von Israel zu Josaphat: »Ich will mich
verkleiden und so in die Schlacht gehen; du aber magst deine gewöhnliche
Kleidung anbehalten.« So nahm denn der König von Israel verkleidet an
der Schlacht teil. \bibleverse{31}Der König von Syrien hatte aber den
zweiunddreißig Befehlshabern seiner Kriegswagen den bestimmten Befehl
erteilt: »Ihr sollt niemand angreifen, er sei gering oder vornehm,
sondern nur den König von Israel!« \bibleverse{32}Als nun die
Befehlshaber der Kriegswagen Josaphat zu Gesicht bekamen, dachten sie,
daß es gewiß der König von Israel sei, und wandten sich gegen ihn, um
ihn anzugreifen. Da erhob Josaphat ein Geschrei\textless sup
title=``oder: den judäischen Kriegsruf''\textgreater✲;
\bibleverse{33}und sobald die Befehlshaber der Wagen erkannt hatten, daß
er nicht der König von Israel sei, wandten sie sich von ihm ab.
\bibleverse{34}Ein Mann aber spannte seinen Bogen aufs Geratewohl und
traf den König von Israel zwischen dem Ringelgurt und dem Panzer. Da
befahl er seinem Wagenlenker: »Wende um und bringe mich vom Schlachtfeld
weg, denn ich bin verwundet!« \bibleverse{35}Da aber der Kampf an jenem
Tage immer heftiger wurde, blieb der König dann doch den Syrern
gegenüber aufrecht im Wagen stehen, bis er am Abend starb; das Blut war
aus der Schußwunde ins Innere des Wagens geflossen. \bibleverse{36}Da
erscholl gegen Sonnenuntergang der laute Ruf durch das Lager: »Jeder
(kehre heim) in seine Stadt und jeder in sein Land! Denn der König ist
tot!« \bibleverse{37}Als man dann nach Samaria gekommen war, begruben
sie den König in Samaria; \bibleverse{38}und als man den Wagen am Teich
von Samaria abspülte, leckten die Hunde sein Blut {[}und die Dirnen
wuschen sich damit{]}, wie der HERR es zuvor angekündigt
hatte\textless sup title=``vgl. 21,19''\textgreater✲.

\bibleverse{39}Die übrige Geschichte Ahabs aber sowie alle seine Taten
und das Elfenbeinhaus, das er gebaut, und sämtliche Städte, die er
befestigt hat, das findet sich bekanntlich aufgezeichnet im Buch der
Denkwürdigkeiten\textless sup title=``oder: Chronik''\textgreater✲ der
Könige von Israel.~-- \bibleverse{40}Als Ahab sich aber zu seinen Vätern
gelegt hatte, folgte ihm sein Sohn Ahasja in der Regierung nach.

\hypertarget{g-josaphat-kuxf6nig-von-juda}{%
\paragraph{g) Josaphat König von
Juda}\label{g-josaphat-kuxf6nig-von-juda}}

\bibleverse{41}Josaphat, der Sohn Asas, wurde König über Juda im vierten
Jahre der Regierung des Königs Ahab von Israel.
\bibleverse{42}Fünfunddreißig Jahre war Josaphat beim Regierungsantritt
alt, und fünfundzwanzig Jahre regierte er in Jerusalem. Seine Mutter
hieß Asuba und war eine Tochter Silhis. \bibleverse{43}Er wandelte ganz
auf dem Wege seines Vaters Asa, ohne davon abzuweichen, so daß er tat,
was dem HERRN wohlgefiel. \bibleverse{44}Nur der Höhendienst wurde nicht
beseitigt: das Volk brachte immer noch Schlacht- und Rauchopfer auf den
Höhen dar. \bibleverse{45}Mit dem König von Israel aber lebte Josaphat
in Frieden.

\bibleverse{46}Die übrige Geschichte Josaphats aber und seine tapferen
Taten, die er ausgeführt, und wie er Krieg geführt hat, das findet sich
bekanntlich bereits aufgezeichnet im Buche der
Denkwürdigkeiten\textless sup title=``oder: Chronik''\textgreater✲ der
Könige von Juda. \bibleverse{47}Auch den Rest der
Heiligtumsbuhler\textless sup title=``oder: der geweihten
Buhler''\textgreater✲, der unter der Regierung seines Vaters Asa noch
übriggeblieben war, schaffte er aus dem Lande weg.

\bibleverse{48}In Edom gab es damals keinen König; als König galt der
(judäische) Statthalter✲. \bibleverse{49}Josaphat hatte Tharsisschiffe
bauen lassen, die nach Ophir fahren und Gold holen sollten; aber man
fuhr nicht dahin, weil die Schiffe bei Ezjon-Geber scheiterten.
\bibleverse{50}Damals machte Ahasja, der Sohn Ahabs, dem Josaphat den
Vorschlag: »Laß meine Leute die Fahrt zur See mit den deinigen
mitmachen!«, aber Josaphat ging nicht darauf ein. \bibleverse{51}Als
Josaphat sich dann zu seinen Vätern gelegt und man ihn bei seinen Vätern
in der Stadt seines Ahnherrn David begraben hatte, folgte ihm sein Sohn
Joram in der Regierung nach.

\hypertarget{h-ahasja-kuxf6nig-von-israel-2252-2.kuxf6n-118}{%
\paragraph{h) Ahasja König von Israel (22,52 -- 2.Kön
1,18)}\label{h-ahasja-kuxf6nig-von-israel-2252-2.kuxf6n-118}}

\bibleverse{52}Ahasja, der Sohn Ahabs, wurde König über Israel zu
Samaria im siebzehnten Jahre der Regierung des Königs Josaphat von Juda
und regierte zwei Jahre über Israel. \bibleverse{53}Er tat, was dem
HERRN mißfiel, und wandelte auf dem Wege seines Vaters und seiner Mutter
und auf dem Wege Jerobeams, des Sohnes Nebats, der Israel zur Sünde
verführt hatte. \bibleverse{54}Er diente dem Baal und betete ihn an und
erzürnte dadurch den HERRN, den Gott Israels, ganz wie sein Vater getan
hatte.
