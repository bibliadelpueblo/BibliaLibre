\hypertarget{der-brief-des-apostels-paulus-an-philemon}{%
\section{DER BRIEF DES APOSTELS PAULUS AN
PHILEMON}\label{der-brief-des-apostels-paulus-an-philemon}}

\hypertarget{a-zuschrift-und-segensgruuxdf}{%
\paragraph{a) Zuschrift und
Segensgruß}\label{a-zuschrift-und-segensgruuxdf}}

\hypertarget{section}{%
\section{1}\label{section}}

\bibleverse{1} Ich, Paulus, ein Gefangener (um) Christi Jesu (willen),
und der Bruder Timotheus senden dem geliebten Philemon, unserm
Mitarbeiter, \bibleverse{2} sowie der Schwester Appia und unserm
Mitstreiter Archippus nebst der Gemeinde in deinem Hause unsern Gruß:
\bibleverse{3} Gnade (sei mit) euch und Friede von Gott unserm Vater und
dem Herrn Jesus Christus!

\hypertarget{b-dank-an-gott-und-fuxfcrbitte-fuxfcr-philemon}{%
\paragraph{b) Dank an Gott und Fürbitte für
Philemon}\label{b-dank-an-gott-und-fuxfcrbitte-fuxfcr-philemon}}

\bibleverse{4} Ich sage meinem Gott allezeit Dank, sooft ich deiner in
meinen Gebeten gedenke; \bibleverse{5} ich höre ja, welche Liebe und
welchen Glauben\textless sup title=``oder: welche Treue''\textgreater✲
du dem Herrn Jesus und allen Heiligen gegenüber bewährst. \bibleverse{6}
(Dahin geht aber mein Gebet,) daß die aus deinem Glauben erwachsene
gemeinnützige Tätigkeit sich in der Erkenntnis all des Guten wirksam
erweise, das in uns vorhanden ist auf Christus hin. \bibleverse{7} Ja,
große Freude und reichen Trost habe ich deiner Liebe zu verdanken, weil
die Herzen der Heiligen durch dich, lieber Bruder, erquickt\textless sup
title=``=~ermutigt, oder: gestärkt''\textgreater✲ worden sind.

\hypertarget{c-fuxfcrsprache-fuxfcr-onesimus}{%
\paragraph{c) Fürsprache für
Onesimus}\label{c-fuxfcrsprache-fuxfcr-onesimus}}

\bibleverse{8} Darum, wenn ich es mir auch in Christus unbedenklich
herausnehmen dürfte, dir vorzuschreiben, was sich gebührt\textless sup
title=``=~was deine Pflicht ist''\textgreater✲, \bibleverse{9} so ziehe
ich es doch um der Liebe willen vor, nur eine Bitte gegen dich
auszusprechen: ein Mann, wie ich es bin, ein alter Paulus und jetzt
obendrein ein Gefangener (um) Christi Jesu (willen), \bibleverse{10} ich
bitte dich für mein (Glaubens-) Kind, dessen Vater ich in meiner
Gefangenschaft geworden bin, für Onesimus, \bibleverse{11} der sich dir
zwar ehemals als ein Nichtsnutz erwiesen hat, jetzt aber dir ebenso wie
mir von großem Nutzen ist. \bibleverse{12} Ich sende ihn, das will sagen
mein eigenes Herz, hiermit dir zurück. \bibleverse{13} Gern hätte ich
ihn hier bei mir behalten, damit er mir als dein Vertreter in der
Gefangenschaft, die ich für die Heilsbotschaft leide, Dienste leisten
möchte; \bibleverse{14} doch ohne deine Einwilligung habe ich nichts tun
wollen, damit deine Guttat nicht erzwungen erscheine, sondern als eine
freiwillige Leistung erfolge. \bibleverse{15} Vielleicht ist er nämlich
nur deswegen eine Zeitlang (von dir) getrennt gewesen, damit du ihn auf
ewig wieder erhieltest, \bibleverse{16} (und zwar) nicht mehr als einen
Sklaven, sondern als einen, der etwas Besseres darstellt, nämlich einen
geliebten Bruder, als der er mir schon in besonderem Maße gilt, wieviel
mehr noch dir, dem er sowohl mit seinem Leibe als nun auch im
Herrn\textless sup title=``=~als Mitchrist''\textgreater✲ angehört!
\bibleverse{17} Wenn du dich nun mit mir eng verbunden weißt, so nimm
ihn auf wie mich selbst! \bibleverse{18} Und hat er dich irgendwie
geschädigt oder ist er dir etwas schuldig, so setze das mir auf
Rechnung! \bibleverse{19} Ich, Paulus, gebe es dir hier schriftlich: Ich
will es bezahlen! Dabei brauche ich dir nicht zu sagen, daß auch du mir
etwas\textless sup title=``oder: noch mehr''\textgreater✲ schuldig bist,
nämlich dich selbst. \bibleverse{20} Ja, lieber Bruder, ich möchte dich
gern ein wenig ausnutzen im Herrn: erfülle mir einen Herzenswunsch in
Christus!

\hypertarget{d-briefschluuxdf-besuchsansage-gruxfcuxdfe-und-segenswunsch}{%
\paragraph{d) Briefschluß, Besuchsansage, Grüße und
Segenswunsch}\label{d-briefschluuxdf-besuchsansage-gruxfcuxdfe-und-segenswunsch}}

\bibleverse{21} Im Vertrauen auf deinen Gehorsam schreibe ich dir; ich
weiß ja, du wirst sogar noch mehr tun, als ich verlange. \bibleverse{22}
Zugleich rüste dich aber auch, mich als Gast bei dir aufzunehmen, denn
ich hoffe, infolge eurer Gebete euch geschenkt zu werden.

\bibleverse{23} Es grüßen dich Epaphras, mein Mitgefangener in Christus
Jesus, \bibleverse{24} Markus, Aristarchus, Demas und Lukas, meine
Mitarbeiter. \bibleverse{25} Die Gnade des Herrn Jesus Christus sei mit
eurem Geiste!
