\hypertarget{der-zweite-brief-des-apostels-paulus-an-die-korinther}{%
\section{DER ZWEITE BRIEF DES APOSTELS PAULUS AN DIE
KORINTHER}\label{der-zweite-brief-des-apostels-paulus-an-die-korinther}}

\hypertarget{zuschrift-und-segensgruuxdf}{%
\subsubsection{Zuschrift und
Segensgruß}\label{zuschrift-und-segensgruuxdf}}

\hypertarget{section}{%
\section{1}\label{section}}

\bibleverse{1} Ich, Paulus, ein Apostel Christi Jesu durch den Willen
Gottes, und der Bruder Timotheus senden der Gemeinde Gottes in Korinth
samt allen Heiligen in ganz Achaja✲ unsern Gruß: \bibleverse{2} Gnade
sei mit euch und Friede von Gott unserm Vater und dem Herrn Jesus
Christus!

\hypertarget{einleitung-mit-lobpreis-gottes}{%
\paragraph{Einleitung mit Lobpreis
Gottes}\label{einleitung-mit-lobpreis-gottes}}

\hypertarget{a-dankgebet-des-apostels-fuxfcr-den-trost-den-sowohl-er-als-auch-die-leser-von-gott-in-leiden-empfangen}{%
\paragraph{a) Dankgebet des Apostels für den Trost, den sowohl er als
auch die Leser von Gott in Leiden
empfangen}\label{a-dankgebet-des-apostels-fuxfcr-den-trost-den-sowohl-er-als-auch-die-leser-von-gott-in-leiden-empfangen}}

\bibleverse{3} Gepriesen sei der Gott und Vater unsers Herrn Jesus
Christus, der Vater der Barmherzigkeit und der Gott alles Trostes,
\bibleverse{4} der uns in aller unserer Trübsal tröstet, damit wir dann
(unserseits) alle, die sich in irgendeiner Trübsal befinden, mit dem
Trost zu erquicken vermögen, mit dem wir selbst von Gott getröstet
werden. \bibleverse{5} Denn wie die Leiden Christi sich überaus
reichlich über uns ergießen, so ergießt sich durch Christus auch unser
Trost überaus reichlich. \bibleverse{6} Werden wir nun (von Trübsal)
bedrängt, so dient das zur Tröstung und zum Heil für euch; und empfangen
wir Trost (von Gott), so dient das (gleichfalls) euch zum Trost, und
dieser erweist sich dann wirksam\textless sup title=``oder: wirkt sich
dann aus''\textgreater✲ in standhaftem Ertragen der gleichen Leiden, die
auch wir durchzumachen haben; \bibleverse{7} und so steht unsere
Hoffnung für euch unerschütterlich fest, weil wir wissen, daß ihr wie an
den Leiden, so auch am Troste✲ Anteil habt.

\hypertarget{b-mitteilung-uxfcber-die-errettung-des-paulus-und-seiner-mitarbeiter-aus-todesgefahr}{%
\paragraph{b) Mitteilung über die Errettung des Paulus und seiner
Mitarbeiter aus
Todesgefahr}\label{b-mitteilung-uxfcber-die-errettung-des-paulus-und-seiner-mitarbeiter-aus-todesgefahr}}

\bibleverse{8} Wir möchten euch nämlich, liebe Brüder, über die Trübsal,
die uns in der Provinz Asien betroffen hat, nicht in Unkenntnis lassen,
daß nämlich das Leid so übergewaltig, so unerträglich schwer auf uns
gelastet hat, daß wir sogar unser Leben verloren gaben; \bibleverse{9}
ja, wir selber hatten es schon für ausgemacht gehalten, daß wir sterben
müßten; wir sollten eben lernen, unser Vertrauen nicht auf uns selbst zu
setzen, sondern auf den Gott, der die Toten auferweckt. \bibleverse{10}
Er hat uns denn auch aus einer so großen Todesgefahr errettet und wird
uns auch fernerhin erretten; auf ihn setzen wir unsere Hoffnung, daß er
uns auch in Zukunft erretten wird, \bibleverse{11} wenn\textless sup
title=``oder: indem''\textgreater✲ auch ihr mit eurer Fürbitte hilfreich
für uns eintretet, damit aus dem Munde vieler (Anteilnehmenden) der uns
zuteil gewordene Gnadenerweis durch viele auch die Dankesweihe im
Hinblick auf uns erhalte.

\hypertarget{i.-selbstrechtfertigung-des-apostels-gegen-erhobene-vorwuxfcrfe-persuxf6nliche-auseinandersetzungen-112-217}{%
\subsection{I. Selbstrechtfertigung des Apostels gegen erhobene
Vorwürfe; persönliche Auseinandersetzungen
(1,12-2,17)}\label{i.-selbstrechtfertigung-des-apostels-gegen-erhobene-vorwuxfcrfe-persuxf6nliche-auseinandersetzungen-112-217}}

\hypertarget{der-lautere-lebenswandel-des-apostels-und-seine-wahrhaftigkeit-im-brieflichen-verkehr}{%
\subsubsection{1. Der lautere Lebenswandel des Apostels und seine
Wahrhaftigkeit im brieflichen
Verkehr}\label{der-lautere-lebenswandel-des-apostels-und-seine-wahrhaftigkeit-im-brieflichen-verkehr}}

\bibleverse{12} Denn darin besteht unser Ruhm\textless sup title=``oder:
Rühmen''\textgreater✲: in dem Zeugnis unsers Gewissens, daß wir in
Sittenreinheit und Lauterkeit Gottes, nicht in fleischlicher
Weisheit\textless sup title=``vgl. Phil 3,3-9''\textgreater✲, sondern in
der Gnade Gottes unsern Wandel in der Welt und ganz besonders euch
gegenüber\textless sup title=``=~im Verkehr mit euch''\textgreater✲
geführt haben. \bibleverse{13} Denn wir schreiben euch (in unsern
Briefen) nichts anderes, als was ihr da lest und auch (daraus) versteht;
ich hoffe aber, daß ihr es bis ans Ende verstehen werdet\textless sup
title=``oder: daß ihr schließlich zum vollen Verständnis gelangen
werdet''\textgreater✲, \bibleverse{14} wie ihr uns ja, wenigstens zum
Teil, schon verstanden habt, nämlich, daß wir euer Ruhm\textless sup
title=``oder: für euch der Grund des Ruhmes''\textgreater✲ sind, ebenso
wie ihr es für uns seid am Tage unsers Herrn Jesus.

\hypertarget{bericht-des-apostels-uxfcber-den-wechsel-seiner-reisepluxe4ne-hinweis-auf-seine-zuverluxe4ssigkeit-als-die-eines-apostels-christi-und-des-treuen-gottes}{%
\subsubsection{2. Bericht des Apostels über den Wechsel seiner
Reisepläne; Hinweis auf seine Zuverlässigkeit als die eines Apostels
Christi und des treuen
Gottes}\label{bericht-des-apostels-uxfcber-den-wechsel-seiner-reisepluxe4ne-hinweis-auf-seine-zuverluxe4ssigkeit-als-die-eines-apostels-christi-und-des-treuen-gottes}}

\bibleverse{15} Auf Grund dieser festen Überzeugung nun war ich willens,
schon früher\textless sup title=``oder: zuerst''\textgreater✲ zu euch zu
kommen\textless sup title=``vgl. 1.Kor 16,5''\textgreater✲, damit ihr
eine zweite\textless sup title=``oder: neue''\textgreater✲ Freude
erhieltet, \bibleverse{16} und über euer Korinth nach Mazedonien
weiterzureisen; alsdann wollte ich von Mazedonien aus nochmals zu euch
kommen und mir von euch das Geleit nach Judäa geben lassen.
\bibleverse{17} Habe ich mich nun etwa, wenn\textless sup title=``oder:
als''\textgreater✲ ich diese Absicht hatte, der Leichtfertigkeit
schuldig gemacht? Oder sind meine Entschlüsse überhaupt Entschlüsse nach
dem Fleisch\textless sup title=``=~Entschlüsse eines weltlich
gerichteten Menschen''\textgreater✲, damit das »Ja ja« und das »Nein
nein« mir (gleichzeitig) zur Verfügung stehe? \bibleverse{18} Aber Gott
ist Bürge dafür, daß unser Wort, das an euch ergeht, nicht Ja und Nein
(zu gleicher Zeit) ist. \bibleverse{19} Denn Gottes Sohn Jesus Christus,
der unter euch durch uns gepredigt worden ist, nämlich durch mich und
Silvanus und Timotheus, ist auch nicht Ja und Nein (zugleich) gewesen,
sondern in ihm ist das »Ja« geschehen\textless sup title=``oder:
erschienen =~verwirklicht worden''\textgreater✲; \bibleverse{20} denn
für alle Verheißungen Gottes liegt in ihm das »Ja«\textless sup
title=``d.h. die Erfüllung''\textgreater✲; daher ist durch ihn auch das
»Amen« erfolgt, Gott zur Verherrlichung\textless sup title=``oder:
Ehre''\textgreater✲ durch uns. \bibleverse{21} Der uns aber samt euch
auf Christus fest gründet und uns gesalbt hat, das ist Gott,
\bibleverse{22} er, der uns auch sein Siegel aufgedrückt und uns den
Geist als Unterpfand✲ in unsere Herzen gegeben hat.

\hypertarget{angabe-des-wahren-grundes-weshalb-paulus-nicht-nach-korinth-gekommen-sei}{%
\subsubsection{3. Angabe des wahren Grundes, weshalb Paulus nicht nach
Korinth gekommen
sei}\label{angabe-des-wahren-grundes-weshalb-paulus-nicht-nach-korinth-gekommen-sei}}

\bibleverse{23} Meinerseits aber rufe ich Gott zum Zeugen gegen meine
Seele an, daß ich (nur) aus Schonung für euch noch nicht wieder nach
Korinth gekommen bin. \bibleverse{24} Nicht daß wir als die Herren über
euren Glauben zu bestimmen hätten, nein, wir sind
Mitarbeiter\textless sup title=``oder: Helfer''\textgreater✲ an eurer
Freude; denn im Glauben steht ihr fest.

\hypertarget{section-1}{%
\section{2}\label{section-1}}

\bibleverse{1} Ich habe aber aus Rücksicht auf mich selbst diesen
Beschluß gefaßt, nicht nochmals zu euch zu kommen, wenn mein Besuch nur
unter Betrübnis möglich ist. \bibleverse{2} Denn wenn ich euch in
Betrübnis versetze -- ja, wer ist dann da, der mich noch erfreuen
könnte? Wer sonst als der, welcher von mir in Betrübnis versetzt wird?
\bibleverse{3} Und eben dies habe ich auch in meinem Briefe
ausgesprochen, um nicht nach meiner Ankunft Betrübnis an denen zu
erleben, von denen mir doch billigerweise Freude widerfahren müßte; ich
darf ja doch zu euch allen das Vertrauen haben, daß meine Freude euer
aller Freude ist. \bibleverse{4} Denn aus großer Bedrängnis und
Herzensangst heraus habe ich euch unter vielen Tränen (meinen Brief)
geschrieben, nicht damit ihr in Betrübnis versetzt würdet, sondern damit
ihr die Liebe erkennen möchtet, die ich in besonders hohem Maße gerade
zu euch habe.

\hypertarget{beseitigung-des-zerwuxfcrfnisses-zwischen-paulus-und-den-korinthern-empfehlung-von-milde-gegen-den-buuxdffertigen-uxfcbeltuxe4ter}{%
\subsubsection{4. Beseitigung des Zerwürfnisses zwischen Paulus und den
Korinthern; Empfehlung von Milde gegen den bußfertigen
Übeltäter}\label{beseitigung-des-zerwuxfcrfnisses-zwischen-paulus-und-den-korinthern-empfehlung-von-milde-gegen-den-buuxdffertigen-uxfcbeltuxe4ter}}

\bibleverse{5} Wenn aber jemand Betrübnis verursacht hat, so hat er
nicht mich (persönlich) betrübt, sondern mehr oder weniger -- damit ich
ihn nicht zu schwer belaste -- euch alle. \bibleverse{6} Für den
Betreffenden genügt nun diese von der Mehrheit (der Gemeinde) ihm
zuerkannte Strafe, \bibleverse{7} so daß ihr im Gegenteil ihm jetzt
lieber verzeihen und Trost zusprechen solltet, damit der Betreffende
nicht durch die übergroße Traurigkeit in Verzweiflung versetzt wird.
\bibleverse{8} Deshalb empfehle ich euch, Liebe gegen ihn walten zu
lassen. \bibleverse{9} Denn ich habe mich ja bei meinem Schreiben auch
von der Absicht leiten lassen, euch auf die Probe zu stellen, ob euer
Gehorsam sich in allen Stücken bewähren würde. \bibleverse{10} Wem ihr
aber eine Verfehlung verzeiht, dem verzeihe auch ich; denn auch ich habe
das, was ich verziehen habe -- wenn ich überhaupt etwas zu verzeihen
hatte --, um euretwillen vor dem Angesicht Christi verziehen.
\bibleverse{11} Wir wollen uns doch nicht vom Satan überlisten lassen,
dessen Gedanken\textless sup title=``oder: Anschläge''\textgreater✲ uns
ja wohlbekannt sind.

\hypertarget{des-apostels-erlebnisse-in-troas-und-mazedonien-sein-lobpreis-gottes-fuxfcr-die-siegreiche-wirkung-der-heilsverkuxfcndigung}{%
\subsubsection{5. Des Apostels Erlebnisse in Troas und Mazedonien; sein
Lobpreis Gottes für die siegreiche Wirkung der
Heilsverkündigung}\label{des-apostels-erlebnisse-in-troas-und-mazedonien-sein-lobpreis-gottes-fuxfcr-die-siegreiche-wirkung-der-heilsverkuxfcndigung}}

\bibleverse{12} Als ich aber nach Troas gekommen war, um die
Heilsbotschaft Christi\textless sup title=``oder: von
Christus''\textgreater✲ zu verkünden, stand mir dort wohl eine
Tür\textless sup title=``=~günstige Gelegenheit zur
Wirksamkeit''\textgreater✲ im Herrn offen, \bibleverse{13} aber ich kam
doch innerlich zu keiner Ruhe, weil ich meinen Bruder Titus dort nicht
antraf; ich nahm vielmehr Abschied von ihnen und zog weiter nach
Mazedonien. \bibleverse{14} Gott aber sei gedankt, der uns in
Christus\textless sup title=``=~im Dienste Christi''\textgreater✲
allezeit (wie) in einem Triumphzug mit sich einherführt und den
Wohlgeruch seiner Erkenntnis durch uns an allen Orten
offenbart\textless sup title=``d.h. wahrnehmbar aufsteigen
läßt''\textgreater✲! \bibleverse{15} Denn ein Wohlgeruch Christi sind
wir für Gott bei denen, die gerettet werden, und auch bei denen, die
verlorengehen: \bibleverse{16} für die letzteren ein Geruch vom Tode her
zum Tod, für die ersteren ein Geruch vom Leben her zum Leben. Und wer
ist dazu\textless sup title=``d.h. zu solchem Dienst''\textgreater✲
tüchtig? \bibleverse{17} Nun, wir machen es nicht wie so
viele\textless sup title=``oder: die meisten?''\textgreater✲, die mit
Gottes Wort hausieren gehen\textless sup title=``oder: Gottes Wort
verfälscht darbieten''\textgreater✲; nein, aus reinem Herzen, nein, auf
Antrieb Gottes reden wir vor Gottes Angesicht in Christus.

\hypertarget{ii.-die-herrlichkeit-des-neuen-bundes-und-des-apostolischen-amtes-bei-uxe4uuxdferer-armseligkeit-und-verfolgung-31-610}{%
\subsection{II. Die Herrlichkeit des neuen Bundes und des apostolischen
Amtes bei äußerer Armseligkeit und Verfolgung
(3,1-6,10)}\label{ii.-die-herrlichkeit-des-neuen-bundes-und-des-apostolischen-amtes-bei-uxe4uuxdferer-armseligkeit-und-verfolgung-31-610}}

\hypertarget{die-herrlichkeit}{%
\subsubsection{1. Die Herrlichkeit}\label{die-herrlichkeit}}

\hypertarget{a-einleitende-bemerkung-die-gemeinde-zu-korinth-als-empfehlungsbrief-fuxfcr-paulus-und-gott-als-sicherer-grund-der-zuversicht-fuxfcr-den-apostel}{%
\paragraph{a) Einleitende Bemerkung: Die Gemeinde zu Korinth als
Empfehlungsbrief für Paulus und Gott als sicherer Grund der Zuversicht
für den
Apostel}\label{a-einleitende-bemerkung-die-gemeinde-zu-korinth-als-empfehlungsbrief-fuxfcr-paulus-und-gott-als-sicherer-grund-der-zuversicht-fuxfcr-den-apostel}}

\hypertarget{section-2}{%
\section{3}\label{section-2}}

\bibleverse{1} Fangen wir schon wieder an, »uns selbst zu
empfehlen\textless sup title=``=~Empfehlungsbriefe
auszustellen''\textgreater✲«? Nein; oder haben wir etwa, wie gewisse
Leute, Empfehlungsbriefe an euch oder von euch nötig? \bibleverse{2}
Nein, unser Empfehlungsbrief seid ihr: der ist uns ins Herz
hineingeschrieben, der wird von aller Welt zur Kenntnis
genommen\textless sup title=``oder: anerkannt''\textgreater✲ und
gelesen; \bibleverse{3} bei euch liegt es ja klar zutage, daß ihr ein
Brief Christi seid, der von uns in seinem Dienst ausgefertigt ist,
geschrieben nicht mit Tinte, sondern mit dem Geiste des lebendigen
Gottes, nicht auf Tafeln von Stein, sondern auf Herzenstafeln von
Fleisch. \bibleverse{4} Solche Zuversicht haben wir aber durch Christus
zu Gott; \bibleverse{5} nicht als ob wir von uns selbst aus tüchtig
wären, etwas auszudenken\textless sup title=``oder: über etwas zu
urteilen''\textgreater✲, als stamme es von uns selbst; nein, unsere
Tüchtigkeit stammt von Gott.

\hypertarget{b-die-herrlichkeit-des-neuen-bundes-und-des-apostolischen-amts-gegenuxfcber-dem-alten-bunde-und-dem-dienst-des-mose}{%
\paragraph{b) Die Herrlichkeit des neuen Bundes und des apostolischen
Amts gegenüber dem alten Bunde und dem Dienst des
Mose}\label{b-die-herrlichkeit-des-neuen-bundes-und-des-apostolischen-amts-gegenuxfcber-dem-alten-bunde-und-dem-dienst-des-mose}}

\bibleverse{6} Er ist es auch, der uns tüchtig gemacht hat zu Dienern
des neuen Bundes, (der ein Bund) nicht des Buchstabens, sondern des
Geistes (ist); denn der Buchstabe (des Gesetzes) tötet, der Geist aber
macht lebendig. \bibleverse{7} Wenn nun aber (schon) der Dienst, der den
Tod bringt, mit seiner auf Stein eingegrabenen Buchstabenschrift solche
Herrlichkeit besaß, daß die Israeliten das Angesicht Moses nicht
anzuschauen vermochten wegen des auf seinem Antlitz liegenden Glanzes,
der doch wieder verschwand\textless sup title=``2.Mose
34,29-35''\textgreater✲: \bibleverse{8} wie sollte da der Dienst des
Geistes\textless sup title=``=~der im Geiste geschieht''\textgreater✲
nicht eine noch weit größere Herrlichkeit besitzen? \bibleverse{9} Denn
wenn (schon) der Dienst, der die Verurteilung (zum Tode) bringt,
Herrlichkeit besitzt\textless sup title=``oder: besessen
hat''\textgreater✲, so muß der Dienst, der die Gerechtsprechung
vermittelt, in noch viel höherem Grade überreich an Herrlichkeit sein;
\bibleverse{10} ja, die auch dort vorhandene Herrlichkeit verschwindet
in dieser Beziehung völlig gegenüber der überschwenglichen Herrlichkeit
(dieses Dienstes). \bibleverse{11} Denn wenn (schon) das Vergängliche
Herrlichkeit besitzt\textless sup title=``oder: besessen
hat''\textgreater✲, so muß das Bleibende in einer noch viel größeren
Herrlichkeit dastehen.

\hypertarget{c-die-verschiedenheit-der-beiderlei-dienste-tritt-sowohl-bei-ihren-dienern-als-auch-in-ihren-wirkungen-zutage}{%
\paragraph{c) Die Verschiedenheit der beiderlei Dienste tritt sowohl bei
ihren Dienern als auch in ihren Wirkungen
zutage}\label{c-die-verschiedenheit-der-beiderlei-dienste-tritt-sowohl-bei-ihren-dienern-als-auch-in-ihren-wirkungen-zutage}}

\bibleverse{12} Weil wir nun eine solche Hoffnung haben, treten wir auch
mit rückhaltlosem Freimut auf \bibleverse{13} und (machen es) nicht wie
Mose (,der) eine Decke auf sein Gesicht legte, damit die Israeliten
nicht das Ende des verschwindenden (Glanzes) wahrnehmen
könnten\textless sup title=``2.Mose 34,29-35''\textgreater✲.
\bibleverse{14} Indessen ihr geistliches Denken ist verhärtet worden;
denn bis auf den heutigen Tag ist dieselbe Decke immer noch da, wenn die
Schriften des alten Bundes vorgelesen\textless sup title=``oder: von
ihnen gelesen''\textgreater✲ werden, und wird nicht abgetan✲, weil sie
nur in Christus weggenommen wird. \bibleverse{15} Ja, bis heute liegt,
sooft Mose vorgelesen wird, eine Decke über ihrem Herzen.
\bibleverse{16} Sobald Israel sich aber zum Herrn bekehrt, wird die
Decke weggezogen\textless sup title=``2.Mose 34,34''\textgreater✲.
\bibleverse{17} Der Herr aber ist der Geist; wo aber der Geist des Herrn
ist, da ist Freiheit. \bibleverse{18} Wir alle aber, die wir mit
unverhülltem Angesicht die Herrlichkeit des Herrn
widerspiegeln\textless sup title=``oder: sich in uns spiegeln
lassen''\textgreater✲, werden dadurch in das gleiche Bild\textless sup
title=``oder: in sein Ebenbild''\textgreater✲ umgestaltet von
Herrlichkeit zu Herrlichkeit\textless sup title=``=~von einer
Herrlichkeit zur anderen''\textgreater✲, wie das\textless sup
title=``oder: da es ja''\textgreater✲ vom Herrn des Geistes geschieht.

\hypertarget{d-paulus-und-die-seinen-treten-als-rechte-christusboten-mit-furchtlosigkeit-wahrhaftigkeit-und-guxf6ttlicher-erleuchtung-auf}{%
\paragraph{d) Paulus und die Seinen treten als rechte Christusboten mit
Furchtlosigkeit, Wahrhaftigkeit und göttlicher Erleuchtung
auf}\label{d-paulus-und-die-seinen-treten-als-rechte-christusboten-mit-furchtlosigkeit-wahrhaftigkeit-und-guxf6ttlicher-erleuchtung-auf}}

\hypertarget{section-3}{%
\section{4}\label{section-3}}

\bibleverse{1} Deshalb werden wir, weil wir infolge des uns
widerfahrenen (göttlichen) Erbarmens dieses Amt zu verwalten haben,
nicht mutlos, \bibleverse{2} sondern haben uns von aller schändlichen
Heimlichtuerei losgesagt; denn wir gehen nicht mit Arglist\textless sup
title=``oder: Verschlagenheit''\textgreater✲ um, verfälschen auch das
Wort Gottes nicht, empfehlen uns vielmehr durch die offene Verkündigung
der Wahrheit jedem Gewissensurteil der Menschen vor den Augen Gottes.
\bibleverse{3} Wenn trotzdem die von uns verkündigte Heilsbotschaft
»verhüllt« ist\textless sup title=``d.h. dunkel bleibt''\textgreater✲,
so ist sie doch nur bei denen\textless sup title=``oder: für
die''\textgreater✲ verhüllt, welche verlorengehen, \bibleverse{4} weil
in ihnen der Gott dieser Weltzeit\textless sup title=``d.h. der
Satan''\textgreater✲ das Denkvermögen der Ungläubigen verdunkelt hat,
damit ihnen das helle Licht der Heilsbotschaft von der Herrlichkeit
Christi, der das Ebenbild Gottes ist, nicht leuchte. \bibleverse{5} Denn
nicht »uns selbst« verkündigen wir, sondern Christus Jesus als den
Herrn, uns selbst aber als eure Knechte✲ um Jesu willen. \bibleverse{6}
Denn Gott, der da geboten hat\textless sup title=``1.Mose
1,3''\textgreater✲: »Aus der Finsternis strahle das Licht hervor!«, der
ist es auch, der das Licht in unsern Herzen hat aufstrahlen\textless sup
title=``oder: in unsere Herzen hat hineinstrahlen''\textgreater✲ lassen,
um (uns) die Erkenntnis der Herrlichkeit Gottes im Angesicht Christi
erglänzen zu lassen.

\hypertarget{die-kuxf6rperliche-schwachheit-und-uxe4uuxdfere-bedruxe4ngnis-der-diener-christi-und-doch-dabei-die-innere-herrlichkeit-des-apostolischen-dienstes-und-der-endhoffnung}{%
\subsubsection{2. Die körperliche Schwachheit und äußere Bedrängnis der
Diener Christi und doch dabei die innere Herrlichkeit des apostolischen
Dienstes und der
Endhoffnung}\label{die-kuxf6rperliche-schwachheit-und-uxe4uuxdfere-bedruxe4ngnis-der-diener-christi-und-doch-dabei-die-innere-herrlichkeit-des-apostolischen-dienstes-und-der-endhoffnung}}

\hypertarget{a-die-leidensvolle-uxe4uuxdfere-lage-der-apostel-neben-ihrer-glaubenszuversicht}{%
\paragraph{a) Die leidensvolle äußere Lage der Apostel neben ihrer
Glaubenszuversicht}\label{a-die-leidensvolle-uxe4uuxdfere-lage-der-apostel-neben-ihrer-glaubenszuversicht}}

\bibleverse{7} Wir besitzen aber diesen Schatz in irdenen Gefäßen, damit
die überschwengliche (Fülle der) Kraft sich als Gott angehörend und
nicht als von uns stammend erweise. \bibleverse{8} Allenthalben sind wir
bedrängt, aber nicht erdrückt, in Ratlosigkeit versetzt, aber nicht in
Verzagtheit\textless sup title=``oder: Verzweiflung''\textgreater✲,
\bibleverse{9} verfolgt, aber nicht im Stich gelassen, zu Boden
niedergeworfen, aber nicht ums Leben gebracht; \bibleverse{10} allezeit
tragen wir das Sterben\textless sup title=``oder:
Todesleiden''\textgreater✲ Jesu an unserm Leibe mit uns umher, damit
auch das Leben Jesu an unserm Leibe sichtbar werde. \bibleverse{11} Denn
immerfort werden wir mitten im Leben in den Tod dahingegeben um Jesu
willen, damit auch das Leben Jesu an unserm sterblichen Fleische
sichtbar werde. \bibleverse{12} Somit tut der Tod sein Werk
in\textless sup title=``oder: an''\textgreater✲ uns, das Leben aber
in\textless sup title=``oder: an''\textgreater✲ euch.

\bibleverse{13} Weil wir aber denselben Geist des Glaubens besitzen --
nach dem Wort der Schrift\textless sup title=``Ps 116,10''\textgreater✲:
»Ich habe geglaubt, darum habe ich geredet« --, so glauben auch wir und
deshalb reden wir auch; \bibleverse{14} denn wir wissen, daß der,
welcher den Herrn Jesus auferweckt hat, auch uns mit Jesus auferwecken
und uns mit euch zusammen (vor ihm, d.h. vor dem Richterstuhl Christi)
darstellen wird. \bibleverse{15} Denn das alles geschieht um
euretwillen, damit die Gnade, gemehrt durch immer weiteren Zuwachs (der
Begnadeten), die Danksagung in immer stärkerem Strom sich ergießen lasse
zur Ehre Gottes.

\hypertarget{b-im-sterben-des-uxe4uuxdferen-menschen-vollzieht-sich-die-erneuerung-des-geistlichen-menschen}{%
\paragraph{b) Im Sterben des äußeren Menschen vollzieht sich die
Erneuerung des geistlichen
Menschen}\label{b-im-sterben-des-uxe4uuxdferen-menschen-vollzieht-sich-die-erneuerung-des-geistlichen-menschen}}

\bibleverse{16} Darum werden wir auch nicht verzagt; nein, wenn auch
unser äußerer Mensch aufgerieben wird, so empfängt doch unser innerer
Mensch Tag für Tag neue Kraft. \bibleverse{17} Denn die augenblickliche,
leicht wiegende Last unserer Leiden bringt uns in überschwenglicher
Weise über alles Maß hinaus ein ewiges Vollgewicht von Herrlichkeit ein,
\bibleverse{18} weil wir den Blick nicht auf das Sichtbare, sondern auf
das Unsichtbare richten; denn das Sichtbare ist zeitlich✲, das
Unsichtbare aber bleibt ewig.

\hypertarget{c-die-hoffnung-und-sehnsucht-des-paulus-nach-der-himmlischen-leiblichkeit-und-der-himmlischen-heimat}{%
\paragraph{c) Die Hoffnung und Sehnsucht des Paulus nach der himmlischen
Leiblichkeit und der himmlischen
Heimat}\label{c-die-hoffnung-und-sehnsucht-des-paulus-nach-der-himmlischen-leiblichkeit-und-der-himmlischen-heimat}}

\hypertarget{section-4}{%
\section{5}\label{section-4}}

\bibleverse{1} Wir wissen ja, daß, wenn unser irdisches Haus, das
Leibeszelt, abgebrochen sein wird, wir einen von Gott bereiteten Bau
erhalten, ein nicht von Menschenhänden hergestelltes, ewiges Haus im
Himmel. \bibleverse{2} In diesem (gegenwärtigen) Zustande\textless sup
title=``oder: aus diesem Grunde''\textgreater✲ seufzen wir ja auch, weil
wir danach verlangen, mit unserer himmlischen Behausung überkleidet zu
werden, \bibleverse{3} da wir ja (erst dann), wenn wir diese angelegt
haben, nicht unbekleidet werden erfunden werden. \bibleverse{4} Denn
solange wir uns noch in dem Leibeszelte (hier) befinden, haben wir zu
seufzen und fühlen uns bedrückt, weil wir lieber nicht erst entkleidet,
sondern (sogleich) überkleidet werden möchten, damit das Sterbliche vom
Leben verschlungen werde. \bibleverse{5} Der uns aber eben dafür
zubereitet\textless sup title=``=~tüchtig gemacht''\textgreater✲ hat,
das ist Gott, der uns den Geist als Unterpfand\textless sup
title=``oder: Angeld; 1,22''\textgreater✲ gegeben hat. \bibleverse{6} So
haben wir denn allezeit guten Mut, und da wir wissen, daß, solange wir
unsere Heimat im Leibe haben, wir fern vom Herrn in der Fremde leben
\bibleverse{7} -- denn wir wandeln (hier noch) in (der Welt des)
Glaubens, nicht schon in (der Welt des) Schauens --, \bibleverse{8} so
haben wir guten Mut, möchten jedoch lieber aus dem Leibe auswandern und
in die Heimat zum Herrn kommen. \bibleverse{9} Darum bieten wir auch
allen Eifer auf, mögen wir uns (schon) in der Heimat oder noch in der
Fremde befinden, ihm wohlgefällig zu sein. \bibleverse{10} Denn wir
müssen alle vor dem Richterstuhl Christi offenbar werden\textless sup
title=``=~persönlich erscheinen''\textgreater✲, damit ein jeder (seinen
Lohn) empfange, je nachdem er während seines leiblichen Lebens gehandelt
hat, es sei gut oder böse.

\hypertarget{paulus-bezeugt-die-lauterkeit-und-selbstlosigkeit-seines-apostolischen-wirkens-im-gegensatz-zu-den-angriffen-seiner-uxe4uuxdferlich-bevorzugten-gegner}{%
\subsubsection{3. Paulus bezeugt die Lauterkeit und Selbstlosigkeit
seines apostolischen Wirkens im Gegensatz zu den Angriffen seiner
äußerlich bevorzugten
Gegner}\label{paulus-bezeugt-die-lauterkeit-und-selbstlosigkeit-seines-apostolischen-wirkens-im-gegensatz-zu-den-angriffen-seiner-uxe4uuxdferlich-bevorzugten-gegner}}

\hypertarget{a-persuxf6nliche-bemerkungen-besonders-bezuxfcglich-seines-verhuxe4ltnisses-zur-gemeinde}{%
\paragraph{a) Persönliche Bemerkungen, besonders bezüglich seines
Verhältnisses zur
Gemeinde}\label{a-persuxf6nliche-bemerkungen-besonders-bezuxfcglich-seines-verhuxe4ltnisses-zur-gemeinde}}

\bibleverse{11} Weil wir also die Furcht vor dem Herrn kennen, suchen
wir »Menschen zu gewinnen«, für\textless sup title=``oder:
vor''\textgreater✲ Gott aber sind wir offenbar; doch hoffe ich, auch in
euren Gewissen offenbar zu sein. \bibleverse{12} Wir bringen uns damit
nicht schon wieder bei euch in Empfehlung, sondern wollen euch einen
Anlaß zum Ruhmeszeugnis für uns geben, damit ihr denen zu antworten
wißt, die sich nur äußerer Vorzüge, nicht aber ihrer Herzensverfassung
rühmen können. \bibleverse{13} Denn »sind wir um den Verstand gekommen«,
so ist es für Gott\textless sup title=``=~im Dienst
Gottes''\textgreater✲ geschehen, und »sind wir bei gesundem Verstande«,
so (sind wir's) zum Segen für euch.

\hypertarget{b-hinweis-auf-den-eigenartigen-inhalt-seiner-predigt-und-auf-die-herrlichkeit-seines-versuxf6hnungsdienstes}{%
\paragraph{b) Hinweis auf den eigenartigen Inhalt seiner Predigt und auf
die Herrlichkeit seines
Versöhnungsdienstes}\label{b-hinweis-auf-den-eigenartigen-inhalt-seiner-predigt-und-auf-die-herrlichkeit-seines-versuxf6hnungsdienstes}}

\bibleverse{14} Denn die Liebe Christi drängt uns\textless sup
title=``oder: hält uns in ihrer Gewalt''\textgreater✲, weil wir uns von
der Überzeugung leiten lassen: Einer ist für alle gestorben, folglich
sind sie allesamt gestorben; \bibleverse{15} und er ist darum für alle
gestorben, damit die, welche leben, nicht mehr sich selbst leben,
sondern dem, der für sie gestorben und auferweckt ist. \bibleverse{16}
Daher kennen wir von jetzt ab niemand mehr nach dem Fleisch; nein, sogar
wenn wir (früher) Christus nach dem Fleisch gekannt haben, so kennen wir
ihn doch jetzt nicht mehr so. \bibleverse{17} Wenn also jemand in
Christus ist, so ist er eine neue Schöpfung\textless sup title=``oder:
neu geschaffen''\textgreater✲: das Alte ist vergangen, siehe, ein Neues
ist entstanden! \bibleverse{18} Das alles ist aber das Werk Gottes, der
uns durch Christus mit sich versöhnt hat und uns (Aposteln) den Dienst
der Versöhnung\textless sup title=``d.h. die Versöhnung zu
verkündigen''\textgreater✲ übertragen hat. \bibleverse{19} Denn (so
steht es:) Gott war in Christus und hat die Welt mit sich versöhnt,
indem er ihnen ihre Übertretungen nicht anrechnete und in uns das Wort
von der Versöhnung niedergelegt hat. \bibleverse{20} Für Christus also
reden wir\textless sup title=``=~sind wir tätig''\textgreater✲ als seine
Gesandten, da ja Gott durch uns ermahnt; wir bitten für Christus: »Laßt
euch mit Gott versöhnen!« \bibleverse{21} Er hat den, der Sünde nicht
kannte\textless sup title=``=~von keiner Sünde wußte''\textgreater✲, für
uns zur Sünde\textless sup title=``d.h. zum Sündenträger; vgl. Jes
53,6''\textgreater✲ gemacht, damit wir in ihm Gottes Gerechtigkeit
würden.

\hypertarget{c-paulus-als-apostel-vorbildlich-durch-opferreiche-und-selbstlose-berufserfuxfcllung-im-dienste-gottes}{%
\paragraph{c) Paulus als Apostel vorbildlich durch opferreiche und
selbstlose Berufserfüllung im Dienste
Gottes}\label{c-paulus-als-apostel-vorbildlich-durch-opferreiche-und-selbstlose-berufserfuxfcllung-im-dienste-gottes}}

\hypertarget{section-5}{%
\section{6}\label{section-5}}

\bibleverse{1} Als (Gottes) Mitarbeiter\textless sup title=``vgl. 1.Kor
3,9''\textgreater✲ aber ermahnen wir euch auch: (Seid darauf bedacht)
die Gnade Gottes nicht vergeblich✲ anzunehmen\textless sup title=``oder:
empfangen zu haben''\textgreater✲!~-- \bibleverse{2} Es steht ja
geschrieben\textless sup title=``Jes 49,8''\textgreater✲: »Zur
willkommenen\textless sup title=``=~mir wohlgefälligen''\textgreater✲
Zeit habe ich dich erhört und am Tage des Heils dir geholfen.« Seht,
jetzt ist die hochwillkommene\textless sup title=``=~ihm
wohlgefällige''\textgreater✲ Zeit, seht, jetzt ist der Tag des Heils!
\bibleverse{3} Und dabei geben wir niemand irgendwelchen Anstoß, damit
kein Tadel unsern Dienst treffe; \bibleverse{4} vielmehr suchen wir uns
in jeder Hinsicht als Diener Gottes zu empfehlen: durch große
Standhaftigkeit in Leiden, in Nöten, in Bedrängnissen, \bibleverse{5}
bei Schlägen, bei Gefangenschaften, bei Volksaufständen, in Mühsalen, in
durchwachten Nächten, bei Mangel an Nahrung, \bibleverse{6} in
Sittenreinheit, durch Erkenntnis, durch Langmut, durch Gütigkeit, durch
heiligen Geist, durch ungeheuchelte Liebe, \bibleverse{7} im Wort der
Wahrheit\textless sup title=``=~durch wahrhaftige Lehre''\textgreater✲,
durch die Kraft Gottes, durch die Waffen der Gerechtigkeit zur Rechten
und zur Linken\textless sup title=``=~zum Angriff und zur
Abwehr''\textgreater✲, \bibleverse{8} unter Ehre und Schande, bei übler
und guter Nachrede, als wären wir Verführer\textless sup title=``oder:
Irrlehrer''\textgreater✲ und doch wahrhaftig, \bibleverse{9} als die
Unbekannten und doch wohlbekannt, als die Sterbenden und seht, wir
leben; als die Gezüchtigten und doch nicht zu Tode gepeinigt,
\bibleverse{10} als die Leidtragenden, aber doch allezeit Fröhlichen,
als Bettler, die aber viele reich machen; als solche, die nichts haben
und doch alles besitzen.

\hypertarget{iii.-ermahnung-zu-reinem-christenwandel-611-71}{%
\subsection{III. Ermahnung zu reinem Christenwandel
(6,11-7,1)}\label{iii.-ermahnung-zu-reinem-christenwandel-611-71}}

\hypertarget{feierlich-liebevolle-bitte-an-die-korinther-um-volle-wiederherstellung-der-gemeinschaft}{%
\subsubsection{1. Feierlich liebevolle Bitte an die Korinther um volle
Wiederherstellung der
Gemeinschaft}\label{feierlich-liebevolle-bitte-an-die-korinther-um-volle-wiederherstellung-der-gemeinschaft}}

\bibleverse{11} Liebe Korinther! Unser Mund hat sich euch gegenüber
aufgetan, das Herz ist uns weit geworden! \bibleverse{12} Ihr nehmt in
unserm Herzen keinen engen Raum ein, aber eng ist der Raum in eurem
Inneren (für uns)! \bibleverse{13} So vergeltet (uns) nun Gleiches mit
Gleichem -- ich rede zu euch wie\textless sup title=``oder:
als''\textgreater✲ zu Kindern --: laßt auch eure Herzen sich weit
erschließen!

\hypertarget{warnung-vor-allem-heidnischen-wesen-und-forderung-der-vollkommenen-heiligung}{%
\subsubsection{2. Warnung vor allem heidnischen Wesen und Forderung der
vollkommenen
Heiligung}\label{warnung-vor-allem-heidnischen-wesen-und-forderung-der-vollkommenen-heiligung}}

\bibleverse{14} Gebt euch nicht dazu her, mit Ungläubigen✲ an einem
fremdartigen Joch zu ziehen! Denn was haben Gerechtigkeit und
Gesetzlosigkeit miteinander gemein? Oder was hat das Licht mit der
Finsternis zu schaffen? \bibleverse{15} Wie stimmt Christus mit Beliar
überein, oder welche Gemeinschaft besteht zwischen einem Gläubigen und
einem Ungläubigen? \bibleverse{16} Wie verträgt sich der Tempel Gottes
mit den Götzen? Wir sind ja doch der Tempel des lebendigen Gottes, wie
Gott gesagt hat\textless sup title=``3.Mose 26,11-12''\textgreater✲:
»Ich werde unter ihnen wohnen und wandeln; ich will ihr Gott sein, und
sie sollen mein Volk sein.« \bibleverse{17} Darum\textless sup
title=``Jes 52,11''\textgreater✲: »Geht aus ihrer Mitte hinweg und
sondert euch (von ihnen) ab«, gebietet der Herr, »und rührt nichts
Unreines an, so will ich euch aufnehmen« und\textless sup title=``2.Sam
7,14''\textgreater✲: \bibleverse{18} »Ich will euch ein Vater sein, und
ihr sollt mir Söhne und Töchter sein«, sagt der Herr, der Allmächtige.

\hypertarget{section-6}{%
\section{7}\label{section-6}}

\bibleverse{1} Da wir nun solche Verheißungen haben, Geliebte, wollen
wir uns von jeder Befleckung des Fleisches und des Geistes\textless sup
title=``=~unsers äußeren und inneren Menschen''\textgreater✲
reinigen\textless sup title=``oder: reinhalten''\textgreater✲ und
völlige Heiligung\textless sup title=``oder: Heiligkeit''\textgreater✲
bei uns schaffen in der Furcht Gottes!

\hypertarget{iv.-freude-des-apostels-uxfcber-die-aussuxf6hnung-mit-der-gemeinde-und-uxfcber-das-bereitwillige-und-buuxdffertige-verhalten-der-korinther-72-16}{%
\subsection{IV. Freude des Apostels über die Aussöhnung mit der Gemeinde
und über das bereitwillige und bußfertige Verhalten der Korinther
(7,2-16)}\label{iv.-freude-des-apostels-uxfcber-die-aussuxf6hnung-mit-der-gemeinde-und-uxfcber-das-bereitwillige-und-buuxdffertige-verhalten-der-korinther-72-16}}

\hypertarget{des-apostels-liebesbitte-liebesversicherung-und-vertrauensbezeugung}{%
\subsubsection{1. Des Apostels Liebesbitte, Liebesversicherung und
Vertrauensbezeugung}\label{des-apostels-liebesbitte-liebesversicherung-und-vertrauensbezeugung}}

\bibleverse{2} Lasset uns Eingang (in eure Herzen) finden! Wir haben
niemand (von euch) unrecht getan, niemand zugrunde gerichtet, niemand
übervorteilt. \bibleverse{3} Ich sage das nicht, um eine Verurteilung
auszusprechen\textless sup title=``oder: euch Vorwürfe zu
machen''\textgreater✲; ich habe euch ja vorhin schon erklärt, daß wir
euch in unserm Herzen tragen, um zusammen mit euch zu sterben und
zusammen zu leben. \bibleverse{4} Ich spreche mich mit voller Offenheit
euch gegenüber aus, ich bin voll Rühmens über euch, habe Trost in Fülle
und bin überreich an Freude bei aller meiner Trübsal.

\hypertarget{freude-des-apostels-uxfcber-die-ankunft-und-die-botschaft-des-titus}{%
\subsubsection{2. Freude des Apostels über die Ankunft und die Botschaft
des
Titus}\label{freude-des-apostels-uxfcber-die-ankunft-und-die-botschaft-des-titus}}

\bibleverse{5} Denn auch nach unserer Ankunft in Mazedonien fanden wir
durchaus keine leibliche Ruhe, sondern überall gab es Bedrängnis; von
außen Kämpfe, im Inneren Ängste. \bibleverse{6} Aber Gott, der die
Gebeugten tröstet, hat auch uns✲ getröstet durch die Ankunft des Titus,
\bibleverse{7} und zwar nicht nur durch seine Ankunft, sondern auch
durch den Trost, den er bei\textless sup title=``oder: an''\textgreater✲
euch gefunden hatte; denn er berichtete uns von eurer Sehnsucht nach
mir, von euren Klagen, von eurem Eifer für mich, so daß meine Freude
noch größer wurde.

\hypertarget{freude-des-apostels-uxfcber-die-heilsame-wirkung-des-strafbriefes-uxfcber-das-vuxf6llig-wiederhergestellte-einvernehmen-und-uxfcber-den-guxfcnstigen-bericht-des-titus}{%
\subsubsection{3. Freude des Apostels über die heilsame Wirkung des
Strafbriefes, über das völlig wiederhergestellte Einvernehmen und über
den günstigen Bericht des
Titus}\label{freude-des-apostels-uxfcber-die-heilsame-wirkung-des-strafbriefes-uxfcber-das-vuxf6llig-wiederhergestellte-einvernehmen-und-uxfcber-den-guxfcnstigen-bericht-des-titus}}

\bibleverse{8} Denn wenn ich euch auch durch meinen (vorigen) Brief
betrübt habe, so tut das mir doch nicht leid. Wenn es mir (früher) auch
leid getan hat -- ich sehe ja, daß jener Brief euch, wenn auch nur
vorübergehend, betrübt hat --, \bibleverse{9} so freue ich mich doch
jetzt, allerdings nicht darüber, daß ihr in Betrübnis versetzt worden
seid, wohl aber darüber, daß ihr durch die Betrübnis zur Reue geführt
worden seid; denn eure Betrübnis ist so gewesen, wie Gott sie haben
will, damit ihr von unserer Seite in keiner Weise Schaden erlittet.
\bibleverse{10} Denn die Betrübnis, wie Gott sie haben will, wirkt eine
Reue zum Heil, die niemand (später) zu bereuen hat; die Betrübnis der
Welt dagegen wirkt den Tod. \bibleverse{11} Denn seht doch: eben diese
eure gottwohlgefällige Betrübnis -- welche Bereitwilligkeit hat sie bei
euch gewirkt, ja noch mehr: Entschuldigung, Unwillen, Furcht, Sehnsucht,
Eifer, Bestrafung (des Schuldigen)! In jeder Beziehung habt ihr
bewiesen, daß ihr in der (bewußten) Sache vorwurfsfrei dasteht.
\bibleverse{12} Darum, wenn ich euch auch geschrieben habe, so (habe ich
es) doch nicht wegen des Beleidigers und auch nicht wegen des
Beleidigten (getan), sondern zu dem Zweck, daß euer Eifer für uns bei
euch vor Gottes Angesicht offenbar würde. Dadurch haben wir Trost
gefunden.

\bibleverse{13} Zu diesem unserm Trost kam aber noch ein
außerordentlicher Zuwachs an Freude hinzu im Hinblick auf die Freude des
Titus, weil ihm eine geistige Erquickung von euch allen zuteil geworden
ist. \bibleverse{14} Denn wenn ich mich (früher) ihm gegenüber mehrfach
rühmend über euch ausgesprochen hatte, so habe ich nun damit keine
Enttäuschung erlebt; vielmehr, wie alles, was ich zu euch geredet habe,
wahr gewesen ist, so hat sich nun auch mein Rühmen dem Titus gegenüber
als Wahrheit erwiesen; \bibleverse{15} und sein Herz ist euch (jetzt)
noch hingebender zugewandt, weil er an euer aller Gehorsam zurückdenkt,
wie ihr ihn mit Furcht und Zittern aufgenommen habt. \bibleverse{16} Ich
freue mich, daß ich mich in jeder Beziehung auf euch verlassen kann!

\hypertarget{v.-die-geldsammlung-fuxfcr-die-notleidende-urgemeinde-in-jerusalem-kap.-8-9}{%
\subsection{V. Die Geldsammlung für die notleidende Urgemeinde in
Jerusalem (Kap.
8-9)}\label{v.-die-geldsammlung-fuxfcr-die-notleidende-urgemeinde-in-jerusalem-kap.-8-9}}

\hypertarget{der-erfreuliche-vorbildliche-erfolg-der-sammlung-bei-den-mazedonischen-gemeinden}{%
\subsubsection{1. Der erfreuliche (vorbildliche) Erfolg der Sammlung bei
den mazedonischen
Gemeinden}\label{der-erfreuliche-vorbildliche-erfolg-der-sammlung-bei-den-mazedonischen-gemeinden}}

\hypertarget{section-7}{%
\section{8}\label{section-7}}

\bibleverse{1} Wir können\textless sup title=``oder:
wollen''\textgreater✲ euch nun auch, liebe Brüder, Mitteilung von der
Gnade Gottes machen, die (den Brüdern) in den mazedonischen Gemeinden
verliehen worden ist, \bibleverse{2} daß nämlich trotz schwerer
Leidensprüfung die überschwengliche Fülle ihrer Freude und ihre
abgrundtiefe Armut sich in den reichen Erweis ihrer Mildtätigkeit
ergossen haben. \bibleverse{3} Denn nach Vermögen, ich bezeuge es ihnen,
ja über Vermögen haben sie aus eigenem Antrieb gespendet, \bibleverse{4}
indem sie uns inständig um die Vergünstigung baten, sich an dem
Liebeswerk für die Heiligen (in Jerusalem) beteiligen zu dürfen;
\bibleverse{5} und sie haben dann nicht nur, wie wir gehofft hatten,
(gespendet,) nein, sie haben geradezu sich selbst hingegeben, in erster
Linie dem Herrn und (dann) auch uns nach Gottes Willen.

\hypertarget{aufforderung-an-die-korinther-zu-reger-beteiligung-an-der-sammlung-behufs-durchfuxfchrung-des-begonnenen-liebeswerks}{%
\subsubsection{2. Aufforderung an die Korinther zu reger Beteiligung an
der Sammlung behufs Durchführung des begonnenen
Liebeswerks}\label{aufforderung-an-die-korinther-zu-reger-beteiligung-an-der-sammlung-behufs-durchfuxfchrung-des-begonnenen-liebeswerks}}

\bibleverse{6} Wir haben daher dem Titus zugeredet, er möchte, wie er
schon früher damit begonnen habe, so dieses Liebeswerk bei euch jetzt
auch zum Abschluß bringen. \bibleverse{7} Aber wie ihr euch in allen
Beziehungen hervortut, durch Glauben und Redegabe, durch Erkenntnis und
Eifer in jeder Hinsicht und durch die Liebe, die von uns her in euch
(geweckt oder: wirksam ist), so tut euch nun auch bei diesem Liebeswerk
durch reiche Betätigung hervor! \bibleverse{8} Ich sage das nicht als
einen Befehl, nein, ich möchte am Eifer der anderen die Echtheit auch
eurer Liebe erproben. \bibleverse{9} Ihr kennt ja die Gnade unsers Herrn
Jesus Christus, daß er, obschon er reich war, doch um euretwillen arm
geworden ist, damit ihr durch seine Armut reich würdet. \bibleverse{10}
Nur einen Rat will ich euch hierbei geben, nämlich: dieses\textless sup
title=``d.h. die Beteiligung an diesem Liebeswerk''\textgreater✲ ist für
euch selbst rätlich\textless sup title=``oder:
empfehlenswert''\textgreater✲, weil ihr ja schon seit vorigem Jahr nicht
nur mit der Ausführung (der Sammlung), sondern auch mit dem Entschluß
dazu den anderen vorangegangen seid. \bibleverse{11} So bringt denn
jetzt auch das begonnene Werk zum Abschluß, damit, wie der gute Wille
vorhanden ist, so auch die Ausführung dem Maße eures Vermögens
entspricht. \bibleverse{12} Denn wenn der gute Wille vorhanden ist, so
ist er wohlgefällig\textless sup title=``oder: willkommen''\textgreater✲
nach dem Maß dessen, was einer hat\textless sup title=``=~zu leisten
vermag''\textgreater✲, nicht nach dem Maß dessen, was er nicht
hat\textless sup title=``=~zu leisten vermag''\textgreater✲.
\bibleverse{13} Denn nicht soll anderen eine Entlastung, euch selbst
aber eine Belastung geschaffen werden; nein, des Ausgleichs wegen
\bibleverse{14} soll diesmal euer Überfluß dem Mangel jener abhelfen,
damit (ein andermal) der Überfluß jener eurem Mangel zugute komme und so
ein Ausgleich stattfinde, \bibleverse{15} wie geschrieben
steht\textless sup title=``2.Mose 16,18''\textgreater✲: »Wer viel (Manna
gesammelt) hatte, besaß doch keinen Überschuß, und wer nur wenig besaß,
hatte keinen Mangel.«

\hypertarget{empfehlung-des-titus-und-der-beiden-anderen-abgeordneten-des-paulus}{%
\subsubsection{3. Empfehlung des Titus und der beiden anderen
Abgeordneten des
Paulus}\label{empfehlung-des-titus-und-der-beiden-anderen-abgeordneten-des-paulus}}

\bibleverse{16} Dank aber sei Gott, der dem Titus den gleichen Eifer für
euch ins Herz legt (wie mir)! \bibleverse{17} Denn er hat unserer
Aufforderung (euch zu besuchen) bereitwillig Gehör geschenkt, ist aber,
weil sein Eifer noch größer ist, aus freiem Entschluß zu euch abgereist.
\bibleverse{18} Wir haben ihm aber den Bruder mitgegeben, dessen Lob
bezüglich der (Verkündigung der) Heilsbotschaft durch alle Gemeinden
verbreitet ist. \bibleverse{19} Aber davon abgesehen ist er auch von den
Gemeinden zu unserm Reisegefährten bei der Überbringung dieser
Liebesgabe gewählt worden, die von uns ins Werk gesetzt wird zur Ehre
des Herrn selbst und zum Erweis unsers guten Willens; \bibleverse{20}
denn dadurch verhüten wir den Übelstand, daß jemand uns
verdächtigt\textless sup title=``oder: in üble Nachrede
bringt''\textgreater✲ um dieser reichen Spende willen, die durch unsern
Dienst vermittelt wird. \bibleverse{21} Wir sind ja darauf bedacht, daß
alles löblich\textless sup title=``=~in guter Ordnung''\textgreater✲
dabei zugehe, nicht nur vor dem Herrn, sondern auch vor den Menschen.
\bibleverse{22} Wir haben aber mit jenen beiden zusammen auch noch
unsern Bruder gesandt, dessen Eifer wir schon oftmals bei vielen
Gelegenheiten erprobt haben, der jetzt aber bei seinem vollen Vertrauen
zu euch noch viel eifriger ist. \bibleverse{23} Wenn ich hier für Titus
eintrete, so geschieht es, weil er mein Genosse und in bezug auf euch
mein Mitarbeiter ist; und was unsere (beiden anderen) Brüder betrifft,
so sind sie Abgeordnete von Gemeinden, ein Abglanz\textless sup
title=``oder: eine Ehre''\textgreater✲ Christi. \bibleverse{24} Erweist
ihnen also eure Liebe und liefert den Gemeinden den offenkundigen
Beweis, daß ihr unser Rühmen ihnen gegenüber wirklich verdient habt.

\hypertarget{was-paulus-von-den-korinthern-bisher-geruxfchmt-hat-und-nunmehr-erwartet-und-welche-gruxfcnde-ihn-zur-voraussendung-der-bruxfcder-bestimmt-haben}{%
\subsubsection{4. Was Paulus von den Korinthern bisher gerühmt hat und
nunmehr erwartet und welche Gründe ihn zur Voraussendung der Brüder
bestimmt
haben}\label{was-paulus-von-den-korinthern-bisher-geruxfchmt-hat-und-nunmehr-erwartet-und-welche-gruxfcnde-ihn-zur-voraussendung-der-bruxfcder-bestimmt-haben}}

\hypertarget{section-8}{%
\section{9}\label{section-8}}

\bibleverse{1} Denn in betreff der Liebesgabe selbst, die für die
Heiligen\textless sup title=``in Jerusalem; vgl. 8,4''\textgreater✲
bestimmt ist, brauche ich euch nicht weiter zu schreiben; \bibleverse{2}
ich kenne ja eure Bereitwilligkeit, von der ich den Mazedoniern
gegenüber zu eurer Empfehlung rühmend hervorhebe, daß Achaja✲ schon seit
vorigem Jahre in Bereitschaft sei; und euer Eifer hat die meisten von
ihnen angespornt. \bibleverse{3} Die Brüder aber habe ich deshalb
abgesandt, damit das Lob, das wir euch erteilt haben, sich in diesem
Punkte nicht als unberechtigt erweist, damit ihr vielmehr, wie ich
angegeben habe, wirklich in Bereitschaft seid. \bibleverse{4} Ich möchte
nicht, daß, wenn Mazedonier mit mir (nach Korinth) kommen und euch noch
nicht fertig finden, wir -- um nicht zu sagen »ihr« -- mit dieser
zuversichtlichen Versicherung beschämt dastehen. \bibleverse{5} Deshalb
habe ich es für erforderlich gehalten, den Brüdern zuzureden, zu euch
vorauszureisen und die von euch früher angekündigte Segensgabe schon
vorher fertigzustellen, damit diese dann wirklich als ein
Segen\textless sup title=``d.h. eine reiche Darbietung''\textgreater✲
bereitliegt und nicht nach Geiz aussieht.

\hypertarget{nochmalige-aufforderung-zu-reger-beteiligung-an-der-sammlung-mit-hinweis-auf-die-segensreiche-wirkung-des-liebeswerkes}{%
\subsubsection{5. Nochmalige Aufforderung zu reger Beteiligung an der
Sammlung mit Hinweis auf die segensreiche Wirkung des
Liebeswerkes}\label{nochmalige-aufforderung-zu-reger-beteiligung-an-der-sammlung-mit-hinweis-auf-die-segensreiche-wirkung-des-liebeswerkes}}

\bibleverse{6} Ich meine das aber so: Wer kärglich sät, der wird auch
kärglich ernten, und wer reichlich sät, der wird auch reichlich ernten.
\bibleverse{7} Jeder (gebe), wie er es sich im Herzen vorgenommen
hat\textless sup title=``=~wie sein Herz ihn treibt''\textgreater✲,
nicht mit Unlust oder aus Zwang; denn (nur) »einen freudigen Geber hat
Gott lieb«\textless sup title=``Spr 22,8''\textgreater✲. \bibleverse{8}
Gott hat aber die Macht, euch mit jeglicher Gnadengabe reichlich zu
segnen, auf daß ihr allezeit in jeder Hinsicht vollauf genug habt und
(außerdem noch) reiche Mittel besitzt zu guten Werken\textless sup
title=``=~zu Wohltätigkeitswerken''\textgreater✲ jeder Art,
\bibleverse{9} wie geschrieben steht\textless sup title=``Ps
112,9''\textgreater✲: »Er hat reichlich ausgeteilt, hat den Armen
gespendet; seine Gerechtigkeit bleibt ewig bestehen.« \bibleverse{10} Er
aber, der dem Sämann Samen darreicht und Brot zur Speise, der wird auch
euch die (Mittel zur) Aussaat darreichen und mehren und die Früchte
eurer Gerechtigkeit✲ wachsen lassen, \bibleverse{11} so daß ihr mit
allem reichlich ausgestattet werdet zur Erweisung jeder Mildtätigkeit,
welche durch unsere Vermittlung die Danksagung (der Empfänger) gegen
Gott bewirkt. \bibleverse{12} Denn der durch diese Liebesgabe geleistete
Dienst hilft nicht nur dem Mangel der Heiligen\textless sup title=``vgl.
8,4''\textgreater✲ ab, sondern schafft auch reichen Segen durch viele an
Gott gerichtete Dankgebete. \bibleverse{13} Jene werden ja infolge eurer
Bewährung bei diesem Liebeswerk Gott dafür preisen, daß ihr in eurem
Bekenntnis zu der Heilsbotschaft Christi Gehorsam und in der Teilnahme
für sie und für alle (anderen) Aufrichtigkeit bewiesen habt.
\bibleverse{14} Dabei werden sie auch im Gebet für euch ihrer Sehnsucht
nach euch Ausdruck geben wegen der Gnade Gottes, die sich überreich an
euch erweist. \bibleverse{15} Dank sei Gott für seine unaussprechlich
reiche Gabe!

\hypertarget{vi.-paulus-rechtfertigt-seine-apostolische-wuxfcrde-und-wirksamkeit-gegen-seine-judaistischen-gegner-101-1218}{%
\subsection{VI. Paulus rechtfertigt seine apostolische Würde und
Wirksamkeit gegen seine judaistischen Gegner
(10,1-12,18)}\label{vi.-paulus-rechtfertigt-seine-apostolische-wuxfcrde-und-wirksamkeit-gegen-seine-judaistischen-gegner-101-1218}}

\hypertarget{eingang-gegenuxfcber-dem-vorwurf-der-charakterschwuxe4che-und-des-fleischlichen-wandels-weist-paulus-seine-gegner-auf-die-erprobte-kraft-seines-wirkens-hin}{%
\subsubsection{1. Eingang: Gegenüber dem Vorwurf der Charakterschwäche
und des fleischlichen Wandels weist Paulus seine Gegner auf die erprobte
Kraft seines Wirkens
hin}\label{eingang-gegenuxfcber-dem-vorwurf-der-charakterschwuxe4che-und-des-fleischlichen-wandels-weist-paulus-seine-gegner-auf-die-erprobte-kraft-seines-wirkens-hin}}

\hypertarget{section-9}{%
\section{10}\label{section-9}}

\bibleverse{1} Persönlich aber ermahne ich, Paulus, euch mit dem Hinweis
auf die Sanftmut und Milde Christi, ich, der ich (angeblich) »Auge in
Auge zwar demütig bei euch bin, aus der Ferne\textless sup title=``oder:
abwesend''\textgreater✲ aber mich selbstbewußt✲ gegen euch zeige«.
\bibleverse{2} Ich bitte (euch) nur: (Zwingt mich nicht dazu) bei meiner
Anwesenheit Selbstbewußtsein✲ beweisen zu müssen in der Zuversicht, die
ich mir gegen gewisse Leute herauszunehmen gedenke, die da von uns die
Meinung haben, daß »wir einen Wandel nach dem Fleische führen«.
\bibleverse{3} Ja, wir wandeln wohl im Fleische, führen aber unsern
Kampf nicht nach Fleischesart; \bibleverse{4} denn die Waffen, mit denen
wir kämpfen, sind nicht fleischlicher\textless sup title=``oder:
menschlicher''\textgreater✲ Art, sondern starke Gotteswaffen zur
Zerstörung von Bollwerken: wir zerstören mit ihnen klug ausgedachte
Anschläge \bibleverse{5} und jede hohe Burg, die sich gegen die
Erkenntnis Gottes erhebt, und nehmen alles Sinnen\textless sup
title=``oder: jedes Denken''\textgreater✲ in\textless sup title=``oder:
für''\textgreater✲ den Gehorsam gegen Christus gefangen \bibleverse{6}
und halten uns bereit, jeden Ungehorsam zu bestrafen, sobald nur erst
euer Gehorsam völlig wiederhergestellt ist.

\hypertarget{auseinandersetzung-mit-den-gegnern-im-einzelnen}{%
\subsubsection{2. Auseinandersetzung mit den Gegnern im
einzelnen}\label{auseinandersetzung-mit-den-gegnern-im-einzelnen}}

\hypertarget{a-des-apostels-recht-sich-seines-amtes-zu-ruxfchmen-und-verwahrung-gegen-den-vorwurf-mangelnden-persuxf6nlichen-mutes}{%
\paragraph{a) Des Apostels Recht, sich seines Amtes zu rühmen, und
Verwahrung gegen den Vorwurf mangelnden persönlichen
Mutes}\label{a-des-apostels-recht-sich-seines-amtes-zu-ruxfchmen-und-verwahrung-gegen-den-vorwurf-mangelnden-persuxf6nlichen-mutes}}

\bibleverse{7} Sehet doch auf das, was vor Augen liegt! Wenn jemand von
sich selbst die feste Überzeugung hat, daß er für seine Person Christus
angehört\textless sup title=``=~im Dienste Christi steht''\textgreater✲,
so möge er andererseits auch dies bei sich bedenken, daß ebenso gut wie
er selbst auch wir Christus angehören. \bibleverse{8} Ja, wenn ich mich
noch etwas stärker bezüglich meiner Befugnis rühmen wollte, die der Herr
mir zu eurer »Auferbauung«✲, nicht zu eurer »Zerstörung« verliehen hat,
so würde ich damit nicht zuschanden werden, \bibleverse{9} und es würde
sich nicht herausstellen, daß ich euch durch meine Briefe gewissermaßen
einzuschüchtern suche. \bibleverse{10} Denn »seine Briefe«, sagt man,
»sind allerdings wuchtig und kraftvoll, aber sein persönliches Auftreten
ist schwächlich, und reden kann er gar nicht«. \bibleverse{11} Wer so
redet, möge sich folgendes gesagt sein lassen: Wie wir uns aus der Ferne
brieflich mit Worten zeigen, ebenso werden wir uns auch bei unserer
Anwesenheit mit der Tat beweisen!

\hypertarget{b-die-verschiedenheit-des-von-paulus-mit-recht-und-von-seinen-gegnern-mit-anmauxdfung-geuxfcbten-selbstruhmes}{%
\paragraph{b) Die Verschiedenheit des von Paulus mit Recht und von
seinen Gegnern mit Anmaßung geübten
Selbstruhmes}\label{b-die-verschiedenheit-des-von-paulus-mit-recht-und-von-seinen-gegnern-mit-anmauxdfung-geuxfcbten-selbstruhmes}}

\bibleverse{12} Wir nehmen uns allerdings nicht heraus, uns mit gewissen
Leuten unter denen, die sich selbst empfehlen✲, auf eine Stufe zu
stellen oder uns mit ihnen zu vergleichen; nein, sie sind unverständig
genug, sich an sich selbst zu messen und sich mit sich selbst zu
vergleichen. \bibleverse{13} Wir dagegen wollen uns nicht ins Maßlose
rühmen, sondern nach dem Maße des Arbeitsfeldes, das Gott uns als
Maßstab zugewiesen hat, daß wir nämlich auch bis zu euch gelangen
sollten. \bibleverse{14} Denn wir strecken uns nicht über Gebühr aus,
als reichten wir nicht bis zu euch hin; wir sind ja doch tatsächlich mit
der (Verkündung der) Heilsbotschaft Christi\textless sup title=``oder:
von Christus''\textgreater✲ auch bis zu euch hingelangt. \bibleverse{15}
Und dabei rühmen wir uns nicht maßlos aufgrund\textless sup
title=``oder: auf dem Felde''\textgreater✲ fremder Arbeitsleistungen,
hegen aber die Hoffnung, beim Wachstum eures Glaubens dem uns
zugewiesenen Wirkungsgebiet entsprechend bei euch noch ungleich größer
dazustehen, \bibleverse{16} wenn wir die Heilsbotschaft noch in die über
eure Grenzen hinausliegenden Länder tragen, ohne uns dabei auf fremdem
Arbeitsfelde dessen zu rühmen, was (dort schon von anderen)
fertiggestellt ist. \bibleverse{17} Nein, »wer sich rühmen will, der
rühme sich des Herrn«\textless sup title=``Jer 9,22-23; 1.Kor
1,31''\textgreater✲; \bibleverse{18} denn nicht wer sich selbst
empfiehlt, ist bewährt\textless sup title=``oder:
erprobt''\textgreater✲, sondern der, den der Herr empfiehlt.

\hypertarget{c-der-selbstruhm-des-apostels-gegen-seine-gegner}{%
\paragraph{c) Der Selbstruhm des Apostels gegen seine
Gegner}\label{c-der-selbstruhm-des-apostels-gegen-seine-gegner}}

\hypertarget{aa-weshalb-und-mit-welchem-recht-der-apostel-sich-selbst-ruxfchmt}{%
\subparagraph{aa) Weshalb und mit welchem Recht der Apostel sich selbst
rühmt}\label{aa-weshalb-und-mit-welchem-recht-der-apostel-sich-selbst-ruxfchmt}}

\hypertarget{section-10}{%
\section{11}\label{section-10}}

\bibleverse{1} Möchtet ihr euch doch ein klein wenig Torheit von mir
gefallen lassen! Nicht wahr? Ihr laßt sie euch auch von mir gefallen;
\bibleverse{2} denn ich eifere um euch mit göttlichem Eifer\textless sup
title=``oder: ich bin eifersüchtig auf euch mit göttlicher
Eifersucht''\textgreater✲; ich habe euch ja einem einzigen Manne
verlobt, um euch Christus\textless sup title=``oder: dem
Messias''\textgreater✲ als eine reine Jungfrau zuzuführen.
\bibleverse{3} Ich fürchte aber, daß, wie die Schlange einst Eva mit
ihrer Arglist verführt hat, so auch eure Gedanken von der Einfalt und
lauteren Gesinnung gegen Christus zum Argen hingezogen werden.
\bibleverse{4} Denn wenn irgend jemand daherkommt und euch einen anderen
Jesus verkündigt, den wir nicht verkündigt haben, oder wenn ihr einen
andersartigen Geist empfangt, den ihr (durch uns) nicht empfangen habt,
oder eine andersartige Heilsbotschaft, die ihr (durch uns) nicht
erhalten habt, so laßt ihr euch das bestens gefallen. \bibleverse{5} Ich
denke doch, in keiner Beziehung hinter den »unvergleichlichen« Aposteln
zurückgeblieben zu sein. \bibleverse{6} Denn mag ich auch im Reden
ungeschult sein, so bin ich es doch nicht in der Erkenntnis, die wir ja
doch in jeder Hinsicht euch gegenüber bei allen erwiesen haben.

\hypertarget{bb-der-ruhm-seiner-uneigennuxfctzigen-unentgeltlichen-wirksamkeit-im-gegensatz-zu-den-im-dienste-satans-arbeitenden-gegnern}{%
\subparagraph{bb) Der Ruhm seiner uneigennützigen (unentgeltlichen)
Wirksamkeit im Gegensatz zu den im Dienste Satans arbeitenden
Gegnern}\label{bb-der-ruhm-seiner-uneigennuxfctzigen-unentgeltlichen-wirksamkeit-im-gegensatz-zu-den-im-dienste-satans-arbeitenden-gegnern}}

\bibleverse{7} Oder habe ich etwa dadurch eine Sünde begangen, daß ich
mich selbst erniedrigt habe, damit ihr erhöht würdet, insofern ich euch
die Heilsbotschaft Gottes ohne Entgelt getreulich verkündigt habe?
\bibleverse{8} Andere Gemeinden habe ich ausgebeutet, indem ich
Belohnung von ihnen genommen habe, um euch zu dienen; \bibleverse{9} und
während meines Aufenthalts bei euch bin ich, auch als ich Mangel litt,
doch keinem zur Last gefallen; denn meinem Mangel haben die Brüder
abgeholfen, die damals aus Mazedonien gekommen waren; und in jeder
Beziehung habe ich mich so gehalten, daß ich euch nicht beschwerlich
gefallen bin, und werde es auch (in Zukunft) so halten. \bibleverse{10}
So gewiß die Wahrhaftigkeit Christi in mir wohnt: dieser Ruhm soll mir
in den Gebieten von Achaja nicht verkümmert werden! \bibleverse{11}
Warum das? Etwa weil ich keine Liebe zu euch habe? Das weiß Gott.
\bibleverse{12} Doch was ich (jetzt) tue, werde ich auch (fernerhin)
tun, um denen, die gern eine Möglichkeit haben möchten, bei ihrem Rühmen
ebenso erfunden zu werden wie ich, diese Möglichkeit abzuschneiden.
\bibleverse{13} Denn diese Leute sind Lügenapostel, unredliche Arbeiter,
die nur die Maske von Aposteln Christi tragen. \bibleverse{14} Und das
ist kein Wunder, denn der Satan selbst nimmt ja das Aussehen eines
Lichtengels an. \bibleverse{15} Da ist es denn nichts Verwunderliches,
wenn auch seine Diener mit der Maske von Dienern der Gerechtigkeit
auftreten. Doch ihr Ende wird ihrem ganzen Tun entsprechen.

\hypertarget{cc-nochmalige-herbe-bitte-des-apostels-um-entschuldigung-seines-tuxf6richten-selbstruhmes}{%
\subparagraph{cc) Nochmalige (herbe) Bitte des Apostels um
Entschuldigung seines törichten
Selbstruhmes}\label{cc-nochmalige-herbe-bitte-des-apostels-um-entschuldigung-seines-tuxf6richten-selbstruhmes}}

\bibleverse{16} Nochmals wiederhole ich es: Niemand möge mich für einen
(wirklichen) Toren halten! Wenn aber doch -- nun, so laßt euch meine
Torheit einmal gefallen, damit auch ich ein kleines Loblied von mir
anstimme! \bibleverse{17} Was ich (jetzt) rede, das rede ich nicht im
Sinne des Herrn, sondern eben in Torheit, weil das Rühmen nun einmal an
der Tagesordnung ist. \bibleverse{18} Weil so viele sich nach dem
Fleische\textless sup title=``oder: äußerer Vorzüge''\textgreater✲
rühmen, will auch ich es einmal tun. \bibleverse{19} Ihr laßt euch ja
die Toren gern gefallen, ihr klugen Leute; \bibleverse{20} ihr haltet ja
still, wenn man euch als Knechte behandelt, wenn man euch
aufzehrt\textless sup title=``=~völlig ausbeutet''\textgreater✲, euch
listig einfängt, wenn man selbstbewußt auftritt, ja euch ins Gesicht
schlägt\textless sup title=``=~Ohrfeigen versetzt''\textgreater✲.
\bibleverse{21} Zu meiner Schande muß ich gestehen: Dazu sind wir
freilich zu schwach\textless sup title=``oder: schüchtern''\textgreater✲
gewesen.

\hypertarget{dd-der-selbstruhm-des-apostels-im-blick-auf-seine-lebenserfahrungen}{%
\subparagraph{dd) Der Selbstruhm des Apostels im Blick auf seine
Lebenserfahrungen}\label{dd-der-selbstruhm-des-apostels-im-blick-auf-seine-lebenserfahrungen}}

\hypertarget{der-apostel-ruxfchmt-sich-seiner-abstammung-seines-amtes-der-fuxfclle-seiner-leiden-im-apostolischen-dienst}{%
\paragraph{Der Apostel rühmt sich seiner Abstammung, seines Amtes, der
Fülle seiner Leiden im apostolischen
Dienst}\label{der-apostel-ruxfchmt-sich-seiner-abstammung-seines-amtes-der-fuxfclle-seiner-leiden-im-apostolischen-dienst}}

Worauf sich aber sonst jemand ohne Scheu etwas einbildet -- ich rede in
Torheit --, darauf kann auch ich es mir ohne Scheu herausnehmen.
\bibleverse{22} Sie sind Hebräer? Ich auch. Sie sind Israeliten? Ich
auch. Sie sind Nachkommen Abrahams? Ich auch. \bibleverse{23} Sie sind
Diener Christi? Ich rede im Aberwitz: Ich bin's noch mehr: In mühevollen
Arbeiten überreichlich, in Gefangenschaften überreichlich, unter
Schlägen mehr als genug, in Todesgefahren gar oft; \bibleverse{24} von
Juden habe ich fünfmal die vierzig (Geißelhiebe) weniger
einen\textless sup title=``5.Mose 25,3''\textgreater✲ erhalten;
\bibleverse{25} dreimal bin ich ausgepeitscht, einmal gesteinigt worden;
dreimal habe ich Schiffbruch gelitten, einen Tag und eine Nacht bin ich
ein Spielball der Wellen gewesen; \bibleverse{26} wie viele
beschwerliche Fußwanderungen habe ich gemacht, wie viele Gefahren
bestanden durch Flüsse, Gefahren durch Räuber, Gefahren durch meine
eigenen Volksgenossen, Gefahren durch Heiden, Gefahren in Städten,
Gefahren in Einöden, Gefahren auf dem Meer, Gefahren unter falschen
Brüdern! \bibleverse{27} Wie oft habe ich Mühsale und Beschwerden
bestanden, wie oft durchwachte Nächte, Hunger und Durst, wie oft
Entbehrungen (jeder Art), Kälte und Mangel an Kleidung! \bibleverse{28}
Dazu -- abgesehen von allem Außergewöhnlichen -- das Überlaufenwerden
tagaus tagein, die Sorge für alle (meine) Gemeinden! \bibleverse{29} Wo
ist jemand schwach (in seinem Glaubensleben), und ich wäre nicht auch
schwach\textless sup title=``=~ich nähme nicht Anteil an seinem
Zustand''\textgreater✲? Wo wird jemandem Anstoß bereitet, ohne daß ich
brennenden Schmerz empfände? \bibleverse{30} Wenn einmal gerühmt sein
muß, so will ich mich der Erweisungen meiner Schwachheit rühmen.
\bibleverse{31} Der Gott und Vater des Herrn Jesus, der in alle Ewigkeit
Hochgelobte, weiß, daß ich nicht lüge. \bibleverse{32} In Damaskus hat
der Statthalter des Königs Aretas die Stadt (Damaskus) bewachen lassen,
um mich festzunehmen; \bibleverse{33} da hat man mich durch eine
Öffnung\textless sup title=``oder: ein Fenster''\textgreater✲ in einem
Korbe über die Stadtmauer hinabgelassen, und so bin ich seinen Händen
entronnen.

\hypertarget{der-apostel-ruxfchmt-sich-der-huxf6chsten-begnadungen-durch-himmlische-offenbarungen-und-der-tiefsten-demuxfctigung-durch-kuxf6rperlichen-leidenszustand}{%
\paragraph{Der Apostel rühmt sich der höchsten Begnadungen (durch
himmlische Offenbarungen) und der tiefsten Demütigung (durch
körperlichen
Leidenszustand)}\label{der-apostel-ruxfchmt-sich-der-huxf6chsten-begnadungen-durch-himmlische-offenbarungen-und-der-tiefsten-demuxfctigung-durch-kuxf6rperlichen-leidenszustand}}

\hypertarget{section-11}{%
\section{12}\label{section-11}}

\bibleverse{1} Gerühmt muß sein; es ist zwar nicht heilsam, aber ich
will doch auf die Gesichte und Offenbarungen des Herrn zu sprechen
kommen. \bibleverse{2} Ich weiß von einem Menschen in Christus, daß er
vor vierzehn Jahren bis zum\textless sup title=``oder: in
den''\textgreater✲ dritten Himmel entrückt wurde; ob er dabei im Leibe
gewesen ist, weiß ich nicht, ob außerhalb des Leibes, weiß ich auch
nicht, Gott weiß es. \bibleverse{3} Und ich weiß von dem betreffenden
Menschen -- ob er im Leibe oder ohne den Leib gewesen ist, weiß ich
nicht, Gott weiß es --, \bibleverse{4} daß er in das Paradies entrückt
wurde und unsagbare\textless sup title=``oder:
unaussprechliche''\textgreater✲ Worte hörte, die ein Mensch nicht
aussprechen\textless sup title=``oder: mitteilen''\textgreater✲ darf.
\bibleverse{5} Als ein solcher\textless sup title=``d.h. so hoch
begnadeter''\textgreater✲ Mensch will ich mich rühmen, in bezug auf mich
selbst aber will ich mich nicht rühmen als nur wegen der Schwachheiten.
\bibleverse{6} Wenn ich mich nämlich wirklich entschlösse, mich zu
rühmen, wäre ich deshalb kein Tor, denn ich würde die Wahrheit sagen;
doch ich unterlasse es, damit niemand höher von mir denke als dem
entsprechend, was er an mir sieht oder von mir hört, \bibleverse{7} und
auch wegen der außerordentlichen Größe der Offenbarungen. Deswegen ist
mir auch, damit ich mich nicht überhebe, ein Dorn\textless sup
title=``oder: Stachel''\textgreater✲ ins\textless sup title=``oder: für
das''\textgreater✲ Fleisch gegeben worden, ein Engel\textless sup
title=``oder: Sendling''\textgreater✲ Satans, der mich mit Fäusten
schlagen muß, damit ich mich nicht überhebe. \bibleverse{8} Dreimal habe
ich um seinetwillen den Herrn angefleht, er\textless sup title=``d.h.
der Satansengel''\textgreater✲ möchte von mir ablassen; \bibleverse{9}
doch er\textless sup title=``d.h. der Herr''\textgreater✲ hat zu mir
gesagt: »Meine Gnade ist für dich genügend\textless sup title=``=~muß
dir genügen''\textgreater✲, denn meine Kraft gelangt in der Schwachheit
zur Vollendung\textless sup title=``=~zu voller
Auswirkung''\textgreater✲.« Daher will ich mich am liebsten um so mehr
meiner Schwachheiten rühmen, damit die Kraft Christi Wohnung bei mir
nimmt\textless sup title=``=~sich auf mich niederläßt''\textgreater✲.
\bibleverse{10} Darum bin ich freudigen Muts in Schwachheiten, bei
Mißhandlungen, in Notlagen, in Verfolgungen und Bedrängnissen, die ich
um Christi willen erleide; denn gerade wenn ich schwach bin, dann bin
ich stark.

\hypertarget{d-abschluuxdf-des-selbstruhms-und-der-selbstverteidigung-des-paulus}{%
\paragraph{d) Abschluß des Selbstruhms und der Selbstverteidigung des
Paulus}\label{d-abschluuxdf-des-selbstruhms-und-der-selbstverteidigung-des-paulus}}

\hypertarget{aa-hinweis-auf-das-unrecht-der-korinther}{%
\subparagraph{aa) Hinweis auf das Unrecht der
Korinther}\label{aa-hinweis-auf-das-unrecht-der-korinther}}

\bibleverse{11} Ich bin ein Tor geworden; ihr habt mich dazu gezwungen;
denn (eigentlich) hätte ich von euch empfohlen werden müssen; ich bin ja
doch in keiner Beziehung hinter den »unvergleichlichen« Aposteln
zurückgeblieben, wenn ich auch nichts bin; \bibleverse{12} wenigstens
sind die Zeichen✲ des Apostels unter euch in aller Ausdauer erbracht
worden durch Zeichen, Wunder und Machttaten; \bibleverse{13} denn was
wäre es, worin ihr im Vergleich mit den anderen Gemeinden verkürzt
worden wäret? Höchstens das eine, daß ich persönlich euch nicht zur Last
gefallen bin: vergebt mir dieses Unrecht!

\hypertarget{bb-ankuxfcndigung-der-bevorstehenden-ankunft-des-apostels-abweisung-einer-verleumdung}{%
\subparagraph{bb) Ankündigung der bevorstehenden Ankunft des Apostels;
Abweisung einer
Verleumdung}\label{bb-ankuxfcndigung-der-bevorstehenden-ankunft-des-apostels-abweisung-einer-verleumdung}}

\bibleverse{14} Seht, ich halte mich jetzt zu einem dritten Besuche bei
euch bereit und werde euch (auch diesmal) nicht zur Last fallen; denn
ich suche nicht euer Hab und Gut, sondern euch selbst; die Kinder sind
ja nicht verpflichtet, für die Eltern Schätze zu sammeln, sondern
umgekehrt die Eltern für die Kinder. \bibleverse{15} Ich aber will
herzlich gern (Geld und Gut) zum Opfer bringen, ja mich selbst völlig
aufopfern lassen, wenn es sich um euer Seelenheil handelt. Soll ich,
wenn ich euch in besonderem Maße liebe, darum weniger Gegenliebe (bei
euch) finden?

\bibleverse{16} Doch gut: ich persönlich bin euch nicht beschwerlich
gefallen; aber weil ich ein »schlauer« Mann bin, habe ich euch »mit List
eingefangen«✲? \bibleverse{17} Habe ich euch etwa durch einen von denen
ausbeuten\textless sup title=``oder: ausnutzen''\textgreater✲ lassen,
die ich zu euch gesandt habe? \bibleverse{18} Ich habe Titus zur Reise
veranlaßt und den Bruder mitgesandt: hat euch nun Titus irgendwie
ausgebeutet? Sind wir (beide) nicht in dem gleichen Geiste gewandelt?
Nicht in den gleichen Fußtapfen?

\hypertarget{vii.-schluuxdfermahnungen-an-die-gemeinde-1219-1313}{%
\subsection{VII. Schlußermahnungen an die Gemeinde
(12,19-13,13)}\label{vii.-schluuxdfermahnungen-an-die-gemeinde-1219-1313}}

\hypertarget{a-berichtigung-einer-meinung-der-korinther-befuxfcrchtung-des-apostels-wegen-des-sittlichen-standes-der-gemeinde}{%
\paragraph{a) Berichtigung einer Meinung der Korinther; Befürchtung des
Apostels wegen des sittlichen Standes der
Gemeinde}\label{a-berichtigung-einer-meinung-der-korinther-befuxfcrchtung-des-apostels-wegen-des-sittlichen-standes-der-gemeinde}}

\bibleverse{19} Schon lange seid ihr der Meinung, daß wir uns euch
gegenüber verantworten wollen. Nein, vor Gottes Angesicht in Christus
reden wir, und zwar soll das alles, Geliebte, zu eurer Auferbauung✲
dienen. \bibleverse{20} Ich fürchte nämlich, euch bei meinem Kommen
nicht so zu finden, wie ich es wünsche, und (selbst) von euch so
erfunden zu werden, wie ihr es nicht wünscht; ich fürchte,
Streitigkeiten und Eifersucht, Zerwürfnisse und Parteiwesen,
Verleumdungen und Ohrenbläsereien, Überhebung und Unordnung (bei euch)
vorzufinden; \bibleverse{21} (ich fürchte) daß mein Gott mich nach
meiner Ankunft aufs neue demütigende Erfahrungen bei euch machen läßt
und daß ich um viele von denen Leid tragen muß, die früher gesündigt
haben und wegen der Unsittlichkeit, der Unzucht und ausschweifenden
Lebensweise, die sie getrieben haben, unbußfertig geblieben sind.

\hypertarget{b-ankuxfcndigung-unparteiischen-verfahrens-und-schonungslosen-gerichts}{%
\paragraph{b) Ankündigung unparteiischen Verfahrens und schonungslosen
Gerichts}\label{b-ankuxfcndigung-unparteiischen-verfahrens-und-schonungslosen-gerichts}}

\hypertarget{section-12}{%
\section{13}\label{section-12}}

\bibleverse{1} Zum drittenmal komme ich jetzt zu euch: »auf Grund der
Aussagen von zwei oder drei Zeugen wird jede Sache
festgestellt\textless sup title=``oder: entschieden''\textgreater✲
werden«\textless sup title=``5.Mose 19,15''\textgreater✲. \bibleverse{2}
Ich habe es denen, die früher gesündigt haben, und allen anderen im
voraus angekündigt und gebe wie bei meinem zweiten Besuch, so auch
jetzt, während ich noch abwesend bin, die Erklärung ab: »Wenn ich noch
einmal komme, werde ich keine Schonung üben!«~-- \bibleverse{3} Ihr
verlangt ja den Beweis\textless sup title=``oder: eine
Bezeugung''\textgreater✲ dafür, daß Christus in mir redet, und der ist
gegen euch nicht schwach, sondern stark unter\textless sup title=``oder:
bei''\textgreater✲ euch. \bibleverse{4} Denn er ist (wohl) infolge von
Schwachheit gekreuzigt worden, lebt aber durch die Kraft Gottes. So sind
auch wir wohl schwach in ihm, werden uns aber im Verein mit ihm durch
die Kraft Gottes lebendig✲ an euch erweisen. \bibleverse{5} Macht an
euch selbst die Probe, ob ihr im Glauben steht, prüft euch selbst! Oder
könnt ihr nicht an euch selbst erkennen, daß Jesus Christus in euch ist?
Da müßtet ihr ja unbewährt\textless sup title=``=~unechte
Christen''\textgreater✲ sein. \bibleverse{6} Daß wir aber nicht
unbewährt✲ sind, sollt ihr hoffentlich erkennen! \bibleverse{7} Doch wir
beten zu Gott, daß ihr nichts Böses tun mögt, nicht zu dem Zweck, daß
wir uns als bewährt\textless sup title=``oder: echt''\textgreater✲
offenbaren, sondern damit ihr das Gute tut, wir aber wie Nichtbewährte
dastehen. \bibleverse{8} Denn wir vermögen nichts wider die Wahrheit,
sondern nur für die Wahrheit. \bibleverse{9} Wir freuen uns ja, wenn wir
schwach sind, ihr aber stark seid; und dahin geht auch unser Gebet,
nämlich daß ihr euch völlig zurechtbringen laßt. \bibleverse{10} Aus
diesem Grunde schreibe ich euch dieses noch als Abwesender, um bei
meiner Anwesenheit nicht mit Strenge vorgehen zu müssen in Ausübung der
Machtbefugnis, die der Herr mir zur Auferbauung und nicht »zur
Zerstörung« verliehen hat.

\hypertarget{c-schluuxdfermahnungen-gruxfcuxdfe-und-segenswunsch}{%
\paragraph{c) Schlußermahnungen, Grüße und
Segenswunsch}\label{c-schluuxdfermahnungen-gruxfcuxdfe-und-segenswunsch}}

\bibleverse{11} Im übrigen, liebe Brüder, freuet euch\textless sup
title=``=~gehabt euch wohl''\textgreater✲! Laßt euch zurechtbringen,
nehmt Ermahnungen an, seid eines Sinnes und haltet Frieden; dann wird
der Gott der Liebe und des Friedens mit euch sein. \bibleverse{12} Grüßt
einander mit dem heiligen Kuß. Es grüßen euch die Heiligen\textless sup
title=``d.h. die Christen in Mazedonien''\textgreater✲ alle.~--
\bibleverse{13} Die Gnade des Herrn Jesus Christus und die Liebe Gottes
und die Gemeinschaft des heiligen Geistes sei mit euch allen!
