\hypertarget{das-buch-josua}{%
\section{DAS BUCH JOSUA}\label{das-buch-josua}}

\hypertarget{i.-die-eroberung-des-westjordanlandes-kap.-1-12}{%
\subsection{I. Die Eroberung des Westjordanlandes (Kap.
1-12)}\label{i.-die-eroberung-des-westjordanlandes-kap.-1-12}}

\hypertarget{die-vorbereitungen-zur-eroberung-kap.-1-5}{%
\subsubsection{1. Die Vorbereitungen zur Eroberung (Kap.
1-5)}\label{die-vorbereitungen-zur-eroberung-kap.-1-5}}

\hypertarget{a-gottes-eroberungsauftrag-und-ermutigung-josuas-vorbereitungen-zum-uxfcbergang-uxfcber-den-jordan}{%
\paragraph{a) Gottes Eroberungsauftrag und Ermutigung Josuas;
Vorbereitungen zum Übergang über den
Jordan}\label{a-gottes-eroberungsauftrag-und-ermutigung-josuas-vorbereitungen-zum-uxfcbergang-uxfcber-den-jordan}}

\hypertarget{section}{%
\section{1}\label{section}}

1Nach dem Tode Moses, des Knechtes des HERRN, sagte der HERR zu Josua,
dem Sohne Nuns, dem Diener Moses: 2»Mein Knecht Mose ist tot; so mache
du dich nun auf und ziehe über den Jordan dort, du und das ganze Volk
da, in das Land hinüber, das ich ihnen, den Israeliten, geben will.
3Allen Grund und Boden, auf den eure Fußsohle treten wird, gebe ich
euch, wie ich es Mose zugesagt habe. 4Von der Wüste und dem Libanon dort
bis an den großen Strom, den Euphratstrom, das ganze Land der Hethiter,
bis zu dem großen Meer im Westen soll euer Gebiet reichen. 5Niemand soll
vor dir standhalten können, solange du lebst: wie ich mit Mose gewesen
bin, so will ich auch mit dir sein; ich will dir meine Hilfe nicht
entziehen und dich nicht verlassen. 6Sei mutig und stark! Denn du sollst
diesem Volk das Land als Erbe austeilen, dessen Verleihung ich ihren
Vätern zugeschworen habe. 7Nur sei stark und fest entschlossen, auf die
Beobachtung aller Weisungen des Gesetzes bedacht zu sein, das mein
Knecht Mose dir zur Pflicht gemacht hat; weiche davon weder nach rechts
noch nach links ab, damit du bei allen deinen Unternehmungen glücklichen
Erfolg hast. 8Höre nicht auf, von diesem Gesetzbuch zu reden, und sinne
Tag und Nacht darüber nach, damit du auf die Beobachtung alles dessen,
was darin geschrieben steht, bedacht bist; denn alsdann wirst du
glücklichen Erfolg bei deinen Unternehmungen haben, und alsdann wird dir
alles gelingen. 9Ich habe dir also zur Pflicht gemacht: Sei stark und
entschlossen! Habe keine Angst und verzage nicht! Denn mit dir ist der
HERR, dein Gott, bei allem, was du unternimmst.«

\hypertarget{josua-ordnet-die-marschbereitschaft-des-volkes-an}{%
\paragraph{Josua ordnet die Marschbereitschaft des Volkes
an}\label{josua-ordnet-die-marschbereitschaft-des-volkes-an}}

10Da gab Josua den Obmännern\textless sup title=``vgl. 5.Mose
1,15''\textgreater✲ des Volkes folgenden Befehl: 11»Geht hin und her im
Lager und gebt dem Volke folgenden Befehl: ›Versorgt euch mit
Lebensmitteln! Denn (schon) in drei Tagen werdet ihr über den Jordan
dort ziehen, um in den Besitz des Landes zu kommen, das der HERR, euer
Gott, euch zum Eigentum geben will.‹«

\hypertarget{pflichtgetreues-verhalten-der-ostjordanischen-stuxe4mme}{%
\paragraph{Pflichtgetreues Verhalten der ostjordanischen
Stämme}\label{pflichtgetreues-verhalten-der-ostjordanischen-stuxe4mme}}

12Zu den Stämmen Ruben und Gad und dem halben Stamm Manasse aber sagte
Josua: 13»Denkt an das Wort, das Mose, der Knecht des HERRN, euch zur
Pflicht gemacht hat, als er sagte: ›Der HERR, euer Gott, hat euch ans
Ziel gelangen lassen und euch dieses Land gegeben.‹ 14Eure Frauen, eure
kleinen Kinder und euer Vieh mögen in dem Lande bleiben, das euch Mose
auf dieser Seite des Jordans angewiesen hat; ihr aber, alle
kriegstüchtigen Männer, müßt kampfgerüstet an der Spitze eurer Brüder✲
hinüberziehen und ihnen Beistand leisten, 15bis der HERR eure Brüder
ebenso wie euch ans Ziel gebracht hat und auch sie das Land in Besitz
genommen haben, das der HERR, euer Gott, ihnen geben will. Alsdann sollt
ihr in euer eigenes Land zurückkehren und es als Besitz innehaben, das
Mose, der Knecht des HERRN, euch im Ostjordanlande angewiesen hat.« 16Da
gaben sie dem Josua folgende Antwort: »Alles, was du uns befohlen hast,
wollen wir tun, und wohin du uns auch sendest, dahin wollen wir gehen.
17Ganz so, wie wir Mose gehorcht haben, wollen wir auch dir gehorchen.
Nur möge der HERR, dein Gott, mit dir sein, wie er mit Mose gewesen ist!
18Jeder, der sich deinen Befehlen widersetzt und deinen Anordnungen
nicht gehorcht, sooft du uns etwas gebietest, soll mit dem Tode bestraft
werden. Nur sei stark und entschlossen!«

\hypertarget{b-die-auskundschaftung-von-jericho-die-errettung-der-beiden-kundschafter-durch-die-dirne-rahab}{%
\paragraph{b) Die Auskundschaftung von Jericho; die Errettung der beiden
Kundschafter durch die Dirne
Rahab}\label{b-die-auskundschaftung-von-jericho-die-errettung-der-beiden-kundschafter-durch-die-dirne-rahab}}

\hypertarget{section-1}{%
\section{2}\label{section-1}}

1Hierauf sandte Josua, der Sohn Nuns, von Sittim aus heimlich zwei
Männer als Kundschafter aus mit der Weisung: »Geht hin, seht euch das
Land und besonders Jericho an!« Da machten sie sich auf den Weg und
kamen in das Haus einer Dirne namens Rahab und legten sich dort
schlafen. 2Als nun dem König von Jericho die Meldung zuging: »Wisse
wohl: es sind heute abend einige Männer von den Israeliten hierher
gekommen, um das ganze Land auszukundschaften«, 3da sandte der König von
Jericho zu Rahab und ließ ihr sagen: »Gib die Männer heraus, die zu dir
gekommen und in deinem Hause eingekehrt sind; denn sie sind hergekommen,
um das ganze Land auszukundschaften.« 4Das Weib aber hatte die beiden
Männer genommen und sie versteckt und sagte nun: »Jawohl, die Männer
sind bei mir gewesen, aber ich wußte nicht, woher sie waren; 5und als
nun das Stadttor beim Dunkelwerden geschlossen werden sollte, sind die
Männer weggegangen; ich weiß nicht, wohin die Männer sich begeben haben.
Jagt ihnen schnell nach; denn ihr könnt sie noch einholen.« 6Sie hatte
sie jedoch auf das Dach hinaufgeführt und sie dort unter Flachsbündeln
versteckt, die sie sich auf dem Dache aufgeschichtet hatte. 7Die Männer
verfolgten sie nun in der Richtung nach dem Jordan hin bis zu den
Furten; das Stadttor aber schloß man, sobald die zur Verfolgung
ausgesandten Leute draußen waren.

\hypertarget{die-verhandlungen-und-festen-verabredungen-rahabs-mit-den-kundschaftern}{%
\paragraph{Die Verhandlungen und festen Verabredungen Rahabs mit den
Kundschaftern}\label{die-verhandlungen-und-festen-verabredungen-rahabs-mit-den-kundschaftern}}

8Bevor aber jene sich schlafen legten, stieg das Weib zu ihnen auf das
Dach hinauf 9und sagte zu den Männern: »Ich weiß, daß der HERR euch das
Land verliehen und daß uns ein Schrecken vor euch befallen hat, so daß
alle Bewohner des Landes verzagt vor euch sind. 10Denn wir haben davon
gehört, daß der HERR bei eurem Auszug aus Ägypten das Wasser im
Schilfmeer vor euch hat vertrocknen lassen und wie ihr mit den beiden
Königen der Amoriter jenseits des Jordans, mit Sihon und Og, verfahren
seid, an denen ihr den Bann vollstreckt habt. 11Als wir das vernahmen,
ist uns alles Vertrauen entschwunden, und niemand hat euch gegenüber
noch Mut behalten; denn der HERR, euer Gott, ist Gott im Himmel oben und
auf der Erde unten. 12Und nun schwört mir doch beim HERRN, daß, weil ich
euch Gutes erwiesen habe, auch ihr meines Vaters Hause Gutes erweisen
wollt, und gebt mir ein sicheres Zeichen, 13daß ihr meine Eltern und
meine Geschwister samt allen ihren Angehörigen am Leben lassen und uns
vor dem Tode retten wollt.« 14Da antworteten ihr die Männer: »Wir bürgen
mit unserem Leben für das eurige! Wenn ihr an dieser unserer Verabredung
nicht zu Verrätern werdet, so wollen wir, wenn der HERR uns das Land
gibt, dir Liebe und Treue widerfahren lassen!« 15Hierauf ließ sie die
beiden an einem Seil durch das Fenster hinunter; denn ihr Haus war an
die Stadtmauer angebaut, weil sie an der Stadtmauer wohnte. 16Dann sagte
sie zu ihnen: »Geht ins Gebirge, sonst könnten die Verfolger euch
begegnen, und haltet euch dort drei Tage lang verborgen, bis die
Verfolger zurückgekehrt sind; alsdann könnt ihr eures Weges ziehen.«
17Da sagten die Männer zu ihr: »Wir werden uns dieses Eides entledigen,
den du uns hast schwören lassen. 18Wisse wohl: wenn wir ins Land kommen,
mußt du diese purpurrote Schnur am Fenster befestigen, durch das du uns
hinuntergelassen hast, und mußt deine Eltern und Geschwister, überhaupt
alle zu deines Vaters Haus Gehörenden bei dir in deinem Hause
versammeln. 19Jeder, der dann aus der Tür deines Hauses auf die Straße
hinausgeht, ist selbst für sein Leben verantwortlich, während wir frei
von Schuld sind; wer aber bei dir im Hause sein wird, für dessen Leben
tragen wir die Verantwortung, wenn Hand an ihn gelegt wird. 20Auch wenn
du treulos gegen diese unsere Verabredung handelst, sind wir des dir
geleisteten Eides ledig, den du uns hast schwören lassen.« 21Sie
antwortete: »Wie ihr gesagt habt, so soll es abgemacht sein!« Dann ließ
sie sie gehen, und sie entfernten sich; sie aber knüpfte die
Purpurschnur an das Fenster.

\hypertarget{gluxfcckliche-ruxfcckkehr-der-kundschafter-zu-josua-mit-guter-botschaft}{%
\paragraph{Glückliche Rückkehr der Kundschafter zu Josua mit guter
Botschaft}\label{gluxfcckliche-ruxfcckkehr-der-kundschafter-zu-josua-mit-guter-botschaft}}

22Als jene nun weggegangen und in das Gebirge gekommen waren, blieben
sie dort drei Tage, bis die Verfolger zurückgekehrt waren; diese hatten
(nämlich) bei der Verfolgung überall auf dem Wege nach ihnen gesucht,
aber sie nicht gefunden. 23Nun traten die beiden Männer den Heimweg an,
stiegen vom Gebirge hinab, setzten über den Jordan und begaben sich zu
Josua, dem Sohne Nuns, dem sie alles erzählten, was ihnen begegnet war.
24Sie berichteten aber dem Josua: »Der HERR hat das ganze Land in unsere
Gewalt gegeben; auch sind alle Bewohner des Landes vor uns verzagt.«

\hypertarget{c-wunderbarer-durchzug-der-israeliten-durch-den-jordan-aufrichtung-von-zwuxf6lf-denksteinen}{%
\paragraph{c) Wunderbarer Durchzug der Israeliten durch den Jordan;
Aufrichtung von zwölf
Denksteinen}\label{c-wunderbarer-durchzug-der-israeliten-durch-den-jordan-aufrichtung-von-zwuxf6lf-denksteinen}}

\hypertarget{aa-ankunft-am-jordan-bekanntmachung-der-obmuxe4nner-und-zwei-befehle-josuas-aufbruch-des-volkes}{%
\subparagraph{aa) Ankunft am Jordan; Bekanntmachung der Obmänner und
zwei Befehle Josuas; Aufbruch des
Volkes}\label{aa-ankunft-am-jordan-bekanntmachung-der-obmuxe4nner-und-zwei-befehle-josuas-aufbruch-des-volkes}}

\hypertarget{section-2}{%
\section{3}\label{section-2}}

1Darauf ließ Josua am andern Morgen früh aufbrechen, und sie kamen von
Sittim an den Jordan, er mit allen Israeliten; und sie blieben dort über
Nacht, ehe sie hinüberzogen. 2Nach drei Tagen aber gingen die
Obmänner\textless sup title=``vgl. 5.Mose 1,15''\textgreater✲ im ganzen
Lager hin und her 3und gaben dem Volke folgenden Befehl: »Sobald ihr die
Bundeslade des HERRN, eures Gottes, erblickt, wie sie von den
levitischen Priestern aufgehoben\textless sup title=``oder:
weggetragen''\textgreater✲ wird, so brecht auch ihr von eurem Standort
auf und zieht hinter ihr her; 4doch muß zwischen euch und ihr ein
Abstand von etwa zweitausend Ellen bleiben -- ihr dürft ihr nicht zu
nahe kommen --, damit ihr den Weg wisset, den ihr einzuschlagen habt;
denn ihr seid bisher noch nie auf solchem Wege gezogen.« 5Weiter befahl
Josua dem Volke: »Heiligt euch, denn morgen wird der HERR Wunder unter
euch tun!« 6Dann gab Josua den Priestern die Weisung: »Hebt die
Bundeslade auf und zieht vor dem Volke her hinüber!« Da hoben sie die
Bundeslade auf und zogen vor dem Volke einher.

\hypertarget{bb-gottes-heilsverheiuxdfung-an-josua-ankuxfcndigung-des-guxf6ttlichen-wunders-durch-josua}{%
\subparagraph{bb) Gottes Heilsverheißung an Josua; Ankündigung des
göttlichen Wunders durch
Josua}\label{bb-gottes-heilsverheiuxdfung-an-josua-ankuxfcndigung-des-guxf6ttlichen-wunders-durch-josua}}

7Der HERR aber sagte zu Josua: »Heute will ich anfangen, dich in den
Augen von ganz Israel groß zu machen\textless sup title=``=~zu
verherrlichen''\textgreater✲, damit sie erkennen, daß ich mit dir sein
werde, wie ich mit Mose gewesen bin. 8Gib du nun den Priestern, welche
die Bundeslade tragen, folgenden Befehl: ›Wenn ihr beim Jordan an den
Rand des Wassers gekommen seid, so bleibt am Jordan stehen!‹«

9Dann sagte Josua zu den Israeliten: »Tretet herzu und vernehmt die
Worte des HERRN, eures Gottes!« 10Dann hielt Josua folgende Ansprache:
»Daran sollt ihr erkennen, daß ein lebendiger Gott in eurer Mitte ist
und daß er die Kanaanäer, Hethiter, Hewiter, Pherissiter, Girgasiter,
Amoriter und Jebusiter gewißlich vor euch her vertreiben wird: 11sehet,
die Bundeslade des Herrn der ganzen Erde wird vor euch her durch den
Jordan ziehen. 12So wählt euch nun zwölf Männer aus den Stämmen Israels,
aus jedem Stamm einen Mann. 13Sobald dann die Fußsohlen der Priester,
welche die Lade Gottes, des Herrn der ganzen Erde, tragen, in das Wasser
des Jordans eintauchen\textless sup title=``=~sich
niedersenken''\textgreater✲, wird das Wasser des Jordans, nämlich das
Wasser, das von oben her zufließt, auf einmal von dem übrigen Wasser
abgeschnitten werden\textless sup title=``=~sich trennen''\textgreater✲
und wie ein einziger Damm stehenbleiben.«

\hypertarget{cc-der-jordan-bleibt-stehen-und-teilt-sich}{%
\subparagraph{cc) Der Jordan bleibt stehen und teilt
sich}\label{cc-der-jordan-bleibt-stehen-und-teilt-sich}}

14Als nun das Volk aus seinen Zelten aufbrach, um über den Jordan zu
ziehen, indem die Priester die Bundeslade vor dem Volke her trugen,
15und als die Träger der Lade an den Jordan kamen und die Füße der
Priester, der Träger der Lade, in den Rand des Wassers eintauchten --
der Jordan ist aber während der ganzen Erntezeit vollströmend bis über
alle seine Ufer hinaus --, 16da blieb das von oben her zufließende
Wasser stehen: es erhob sich wie ein einziger Damm in weiter Entfernung
bei der Ortschaft Adam, die seitwärts von Zarthan liegt; dagegen das
nach dem Steppensee, dem Toten Meer, hinabfließende Wasser verlief sich
völlig. So zog denn das Volk hindurch, Jericho gegenüber. 17Die Priester
aber, welche die Bundeslade des HERRN trugen, standen festen Fußes
mitten im Jordan auf trockenem Boden, während ganz Israel trockenen
Fußes hindurchzog, bis das ganze Volk den Übergang über den Jordan
vollständig bewerkstelligt hatte.

\hypertarget{dd-errichtung-eines-steindenkmals-im-jordanbett-und-eines-andern-am-jenseitigen-ufer-in-gilgal}{%
\subparagraph{dd) Errichtung eines Steindenkmals im Jordanbett und eines
andern am jenseitigen Ufer in
Gilgal}\label{dd-errichtung-eines-steindenkmals-im-jordanbett-und-eines-andern-am-jenseitigen-ufer-in-gilgal}}

\hypertarget{section-3}{%
\section{4}\label{section-3}}

1Als nun das ganze Volk den Übergang über den Jordan vollständig
bewerkstelligt hatte, gebot der HERR dem Josua folgendes: 2»Wählt euch
aus dem Volke zwölf Männer, aus jedem Stamme einen Mann, 3und erteilt
ihnen folgenden Befehl: ›Nehmt euch von hier, aus der Mitte des Jordans,
von der Stelle, wo die Füße der Priester festgestanden\textless sup
title=``=~Halt gemacht''\textgreater✲ haben, zwölf Steine auf, schafft
sie mit euch hinüber und legt sie auf dem Lagerplatz nieder, wo ihr
diese Nacht zubringen werdet.‹« 4Da berief Josua die zwölf Männer, die
er aus den Israeliten hatte wählen lassen, aus jedem Stamme einen Mann,
5und befahl ihnen: »Geht hin vor die Lade des HERRN, eures Gottes,
mitten in den Jordan hinein und ladet euch ein jeder einen Stein auf
seine Schulter, entsprechend der Zahl der israelitischen Stämme, 6damit
dies ein Wahrzeichen✲ unter euch werde! Wenn dann eure Kinder euch
künftig fragen: ›Was haben diese Steine für euch zu bedeuten?‹, 7so
sollt ihr ihnen antworten: ›Daß das Wasser des Jordans vor der
Bundeslade des HERRN zu fließen aufgehört hat; als die durch den Jordan
zog, hat das Wasser des Jordans zu fließen aufgehört; deshalb sollen
diese Steine für die Israeliten ein ewiges Denkmal sein!‹« 8Da taten die
Israeliten so, wie Josua befohlen hatte: sie hoben zwölf Steine mitten
aus dem Jordan auf, wie der HERR dem Josua geboten hatte, entsprechend
der Zahl der israelitischen Stämme, nahmen sie mit sich nach dem Platze
des Nachtlagers hinüber und legten sie dort nieder. 9Zwölf andere Steine
aber ließ Josua mitten im Jordan an der Stelle aufrichten, wo die Füße
der Priester, welche die Bundeslade trugen, gestanden hatten; dort
befinden sie sich noch heutigen Tages. 10Die Priester aber, welche die
Lade trugen, blieben mitten im Jordan stehen, bis alle Arbeit ausgeführt
war, die Josua nach dem Befehl des HERRN dem Volk geboten hatte, genau
so, wie Mose dem Josua geboten hatte; und das Volk zog eilends hinüber.
11Als dann das ganze Volk den Übergang vollständig bewerkstelligt hatte,
zog auch die Lade des HERRN hinüber und die Priester angesichts des
Volkes. 12Es waren aber die Stämme Ruben und Gad und der halbe Stamm
Manasse kampfgerüstet an der Spitze der Israeliten hinübergezogen, wie
Mose ihnen geboten hatte. 13In der Zahl von etwa vierzigtausend
kampfgerüsteten Männern waren sie vor dem HERRN her zum Kampf in die
Steppen von Jericho hinübergezogen.

\hypertarget{ee-wirkung-des-wunderbaren-ereignisses-auf-die-israeliten-und-auf-alle-vuxf6lker-abschlieuxdfende-angaben}{%
\subparagraph{ee) Wirkung des wunderbaren Ereignisses auf die Israeliten
und auf alle Völker; abschließende
Angaben}\label{ee-wirkung-des-wunderbaren-ereignisses-auf-die-israeliten-und-auf-alle-vuxf6lker-abschlieuxdfende-angaben}}

14An jenem Tage machte der HERR den Josua groß\textless sup title=``vgl.
3,7''\textgreater✲ in den Augen von ganz Israel, so daß sie Ehrfurcht
vor ihm hatten während seines ganzen Lebens, wie sie vor Mose Ehrfurcht
gehabt hatten.

15Hierauf gebot der HERR dem Josua folgendes: 16»Befiehl den Priestern,
welche die Gesetzeslade tragen, aus dem Jordan heraufzusteigen.« 17Da
gab Josua den Priestern den Befehl: »Steigt aus dem Jordan herauf!«
18Als nun die Priester, welche die Bundeslade des HERRN trugen, aus dem
Jordan heraufgestiegen waren und die Fußsohlen der Priester kaum das
trockene Ufer betreten hatten, da kehrte das Wasser des Jordans in sein
Bett zurück und ergoß sich wie früher überall über seine Ufer.

19Das Volk war aber aus dem Jordan am zehnten Tage des ersten Monats
heraufgestiegen und lagerte in Gilgal an der Ostgrenze des Gebiets von
Jericho. 20Da ließ Josua jene zwölf Steine, die sie aus dem Jordan
mitgebracht hatten, in Gilgal aufrichten 21und sagte dabei zu den
Israeliten: »Wenn eure Kinder künftig an ihre Väter die Frage richten:
›Was haben diese Steine zu bedeuten?‹, 22so gebt euren Kindern folgende
Auskunft: ›Trockenen Fußes ist Israel hier durch den Jordan gezogen,
23weil der HERR, euer Gott, das Wasser des Jordans vor euch her hat
vertrocknen lassen, bis ihr hindurchgezogen wart, ebenso wie der HERR,
euer Gott, es mit dem Schilfmeer gemacht hat, das er vor uns hat
vertrocknen lassen, bis wir hindurchgezogen waren; 24denn alle Völker
der Erde sollten erkennen, daß die Hand des HERRN stark ist, damit ihr
den HERRN, euren Gott, allezeit fürchtet.‹«

\hypertarget{d-ereignisse-im-lager-zu-gilgal}{%
\paragraph{d) Ereignisse im Lager zu
Gilgal}\label{d-ereignisse-im-lager-zu-gilgal}}

\hypertarget{aa-vornahme-der-beschneidung-israels}{%
\subparagraph{aa) Vornahme der Beschneidung
Israels}\label{aa-vornahme-der-beschneidung-israels}}

\hypertarget{section-4}{%
\section{5}\label{section-4}}

1Als nun alle Könige der Amoriter, die jenseits des Jordans nach Westen
hin wohnten, und alle Könige der Kanaanäer an der Meeresküste die Kunde
erhielten, daß der HERR das Wasser des Jordans vor den Israeliten hatte
vertrocknen lassen, bis sie hinübergezogen waren, da schwand ihnen alles
Vertrauen, und sie hatten den Israeliten gegenüber keinen Mut mehr.

2Damals gebot der HERR dem Josua: »Fertige dir Steinmesser an und
beschneide wiederum die Israeliten zum zweitenmal!« 3Da machte sich
Josua Steinmesser und beschnitt mit ihnen die Israeliten am Hügel
Araloth\textless sup title=``d.h. Hügel der Vorhäute''\textgreater✲.
4Der Grund aber, weshalb Josua die Beschneidung vornahm, war folgender:
Das gesamte Volk männlichen Geschlechts, das aus Ägypten ausgezogen war,
alle Kriegsleute, waren nach ihrem Auszug aus Ägypten während der
Wanderung in der Wüste gestorben. 5Das ganze Volk nämlich, welches
auszog, war beschnitten gewesen; aber das gesamte Volk, das nach dem
Auszug aus Ägypten während der Wanderung in der Wüste geboren war, hatte
die Beschneidung nicht empfangen. 6Denn vierzig Jahre lang waren die
Israeliten in der Wüste gewandert, bis die Gesamtheit der Kriegsleute
nach dem Auszug aus Ägypten weggestorben war, weil sie den Weisungen des
HERRN nicht nachgekommen waren; deshalb hatte der HERR ihnen geschworen,
er wolle sie das Land nicht sehen lassen, das er uns, wie er ihren
Vätern zugeschworen hatte, geben wollte, ein Land, das von Milch und
Honig überfließt. 7Aber ihre Söhne, die der HERR an ihrer Statt hatte
aufwachsen lassen, diese beschnitt jetzt Josua; denn sie waren
unbeschnitten geblieben, weil man sie während der Wanderung nicht
beschnitten hatte. 8Als nun die Beschneidung am ganzen Volke vorgenommen
worden war, blieben sie an Ort und Stelle gelagert, bis sie genesen
waren. 9Der HERR aber sagte zu Josua: »Heute habe ich die ägyptische
Schmach\textless sup title=``oder: die Schmähung der
Ägypter''\textgreater✲ von euch abgewälzt.« Daher heißt dieser Ort
Gilgal\textless sup title=``d.h. Abwälzung''\textgreater✲ bis auf den
heutigen Tag.

\hypertarget{bb-erste-passahfeier-in-kanaan-aufhuxf6ren-des-manna}{%
\subparagraph{bb) Erste Passahfeier in Kanaan; Aufhören des
Manna}\label{bb-erste-passahfeier-in-kanaan-aufhuxf6ren-des-manna}}

10Während nun die Israeliten in Gilgal gelagert waren, feierten sie das
Passah am vierzehnten Tage des Monats am Abend in den Steppen von
Jericho 11und aßen am anderen Tage nach dem Passah von den Erzeugnissen
des Landes, nämlich ungesäuertes Brot und geröstetes Getreide, an eben
diesem Tage. 12Da hörte am folgenden Morgen das Manna auf, weil sie
jetzt von den Erzeugnissen des Landes zu essen hatten, und es gab
hinfort für die Israeliten kein Manna mehr, sondern sie nährten sich in
jenem Jahre von den Erzeugnissen des Landes Kanaan.

\hypertarget{cc-josua-wird-durch-die-erscheinung-des-guxf6ttlichen-heerfuxfchrers-ermutigt}{%
\subparagraph{cc) Josua wird durch die Erscheinung des göttlichen
Heerführers
ermutigt}\label{cc-josua-wird-durch-die-erscheinung-des-guxf6ttlichen-heerfuxfchrers-ermutigt}}

13Während sich nun Josua bei Jericho befand, begab es sich, daß er seine
Augen aufschlug und einen Mann, der ein gezücktes Schwert in der Hand
hatte, sich gegenüber stehen sah. Josua ging auf ihn zu und fragte ihn:
»Gehörst du zu uns oder zu unsern Feinden?« 14Da antwortete er: »Nein,
sondern ich bin der Oberste des Heeres des HERRN; soeben bin ich
gekommen.« Da warf sich Josua auf sein Angesicht nieder, um ihm zu
huldigen, und fragte ihn dann: »Was hat mein Herr seinem Knecht zu
sagen?« 15Da antwortete der Heeresoberste des HERRN dem Josua: »Ziehe
dir die Schuhe aus von deinen Füßen! Denn die Stätte, auf der du stehst,
ist heilig.« Da tat Josua so.

\hypertarget{die-siegreichen-kuxe4mpfe-mit-den-kanaanuxe4ern-kap.-6-12}{%
\subsubsection{2. Die siegreichen Kämpfe mit den Kanaanäern (Kap.
6-12)}\label{die-siegreichen-kuxe4mpfe-mit-den-kanaanuxe4ern-kap.-6-12}}

\hypertarget{a-die-eroberung-und-zerstuxf6rung-jerichos}{%
\paragraph{a) Die Eroberung und Zerstörung
Jerichos}\label{a-die-eroberung-und-zerstuxf6rung-jerichos}}

\hypertarget{aa-josua-wird-von-gott-uxfcber-die-art-und-weise-der-eroberung-jerichos-belehrt}{%
\subparagraph{aa) Josua wird von Gott über die Art und Weise der
Eroberung Jerichos
belehrt}\label{aa-josua-wird-von-gott-uxfcber-die-art-und-weise-der-eroberung-jerichos-belehrt}}

\hypertarget{section-5}{%
\section{6}\label{section-5}}

1Jericho aber hatte (seine Tore) geschlossen und blieb den Israeliten
gegenüber verriegelt✲, so daß niemand aus- oder eingehen konnte. 2Da
sagte der HERR zu Josua: »Hiermit gebe ich Jericho und seinen König samt
den streitbaren Männern in deine Gewalt. 3So zieht denn um die Stadt
herum, alle Kriegsleute, einmal rings um die Stadt her. So sollst du es
sechs Tage lang tun; 4dabei sollen sieben Priester sieben Lärmposaunen
vor der Lade her tragen. Am siebten Tage aber sollt ihr siebenmal um die
Stadt herumziehen, und die Priester sollen dabei in die Posaunen stoßen.
5Wenn man dann ein Zeichen mit dem Lärmhorn gibt, soll das gesamte Volk,
sobald ihr den Posaunenschall hört, ein lautes Kriegsgeschrei erheben;
dann wird die Stadtmauer von selbst in sich zusammenstürzen, und das
Volk soll sie ersteigen, wo ein jeder gerade steht.«

\hypertarget{bb-die-tuxe4glich-einmaligen-umzuxfcge-um-die-stadt-wuxe4hrend-der-ersten-sechs-tage}{%
\subparagraph{bb) Die täglich einmaligen Umzüge um die Stadt während der
ersten sechs
Tage}\label{bb-die-tuxe4glich-einmaligen-umzuxfcge-um-die-stadt-wuxe4hrend-der-ersten-sechs-tage}}

6Darauf berief Josua, der Sohn Nuns, die Priester und befahl ihnen:
»Hebt die Bundeslade auf, und sieben Priester sollen sieben Lärmposaunen
vor der Lade des HERRN her tragen!« 7Hierauf befahl er dem Volk: »Zieht
rings um die Stadt herum, und zwar sollen die Gewappneten vor der Lade
des HERRN her ziehen!« 8Als nun Josua dem Volk diesen Befehl erteilt
hatte, da setzten sich die sieben Priester in Bewegung, welche die
sieben Lärmposaunen vor dem HERRN her trugen, und stießen in die
Posaunen, während die Bundeslade des HERRN ihnen nachfolgte; 9die
Gewappneten aber zogen vor den Priestern einher, die in die Posaunen
stießen, und die Nachhut zog hinter der Lade her, während man dabei
fortwährend in die Posaunen stieß. 10Dem Volk aber hatte Josua streng
geboten: »Ihr dürft kein Kriegsgeschrei erheben und eure Stimme nicht
hören lassen, und kein Wort darf aus eurem Munde kommen bis zu dem Tage,
an dem ich euch zurufe: ›Laßt ein Geschrei erschallen!‹ Dann müßt ihr
das Kriegsgeschrei erheben.« 11So ließ er denn die Lade des HERRN einmal
den Umzug rings um die Stadt machen; hierauf begaben sie sich wieder ins
Lager und blieben über Nacht im Lager.~-- 12Am folgenden Morgen machte
sich Josua früh auf, und die Priester trugen wiederum die Lade des
HERRN; 13und die sieben Priester, welche die sieben Lärmposaunen vor der
Lade des HERRN her trugen, stießen beim Gehen fortwährend in die
Posaunen, während die Gewappneten vor ihnen herzogen und die Nachhut
hinter der Lade des HERRN folgte, indem man dabei immerfort in die
Posaunen stieß. 14So zogen sie am zweiten Tage einmal um die Stadt herum
und kehrten dann wieder ins Lager zurück. So machten sie es sechs Tage
lang.

\hypertarget{cc-die-sieben-umzuxfcge-am-siebten-tag-eroberung-und-zerstuxf6rung-der-stadt}{%
\subparagraph{cc) Die sieben Umzüge am siebten Tag; Eroberung und
Zerstörung der
Stadt}\label{cc-die-sieben-umzuxfcge-am-siebten-tag-eroberung-und-zerstuxf6rung-der-stadt}}

15Am siebten Tage aber machten sie sich früh beim Aufgang der Morgenröte
auf und zogen in derselben Weise siebenmal um die Stadt herum; nur an
diesem Tage umzogen sie die Stadt siebenmal. 16Beim siebten Umzug aber,
als die Priester in die Posaunen gestoßen hatten, rief Josua dem Volke
zu: »Erhebt das Kriegsgeschrei! Denn der HERR hat die Stadt in eure
Gewalt gegeben! 17Aber die Stadt mit allem, was darin ist, soll dem
Bann✲ für den HERRN geweiht sein; nur die Dirne Rahab soll am Leben
bleiben, sie nebst allen denen, die sich bei ihr im Hause befinden; denn
sie hat die Kundschafter versteckt, die wir ausgesandt hatten. 18Nehmt
ihr euch aber ja vor dem gebannten Gut in acht, daß ihr nicht, obgleich
ihr es dem Bann geweiht habt, euch doch etwas von dem gebannten Gut
aneignet und dadurch das Lager der Israeliten dem Bannfluch überliefert
und es ins Unglück stürzt! 19Alles Silber und Gold nebst den kupfernen
und eisernen Geräten soll dem HERRN geheiligt sein und in den Schatz des
HERRN kommen!«~-- 20Da erhob das Volk das Kriegsgeschrei, und die
Posaunen ertönten; und als das Volk den Posaunenschall vernahm und ein
lautes Kriegsgeschrei erhoben hatte, da stürzte die Mauer in sich
zusammen, und das Volk drang in die Stadt ein, ein jeder da, wo er
gerade stand. Als sie so die Stadt eingenommen hatten, 21vollstreckten
sie den Bann an allem, was sich in der Stadt befand, an Männern wie an
Weibern, an jung und alt, an den Rindern wie am Kleinvieh und an den
Eseln: alles wurde mit der Schärfe des Schwertes niedergemacht.

\hypertarget{dd-verschonung-rahabs-und-ihrer-angehuxf6rigen-verfluchung-des-wiederaufbaus-der-stadt}{%
\subparagraph{dd) Verschonung Rahabs und ihrer Angehörigen; Verfluchung
des Wiederaufbaus der
Stadt}\label{dd-verschonung-rahabs-und-ihrer-angehuxf6rigen-verfluchung-des-wiederaufbaus-der-stadt}}

22Den beiden Männern aber, die das Land ausgekundschaftet hatten, hatte
Josua befohlen: »Geht in das Haus der Dirne und führt das Weib mit allen
ihren Angehörigen von dort heraus, wie ihr es ihr zugeschworen habt!«
23Da gingen die jungen Männer, die beiden Kundschafter, hin und führten
Rahab nebst ihren Eltern und Geschwistern und allen ihren Angehörigen
hinaus: alle ihre Verwandten führten sie hinaus und brachten sie an
einem Orte außerhalb des israelitischen Lagers unter. 24Die Stadt aber
mit allem, was darin war, ließen sie in Flammen aufgehen; nur das Silber
und Gold sowie die kupfernen und eisernen Geräte taten sie in den Schatz
im Hause des HERRN. 25Die Dirne Rahab aber nebst ihrer Familie und allen
ihren Angehörigen ließ Josua am Leben, und sie ist inmitten der
Israeliten bis auf den heutigen Tag wohnen geblieben, weil sie die Boten
versteckt hatte, die Josua zur Auskundschaftung Jerichos ausgesandt
hatte.

26Damals ließ Josua (das Volk) folgenden Eid schwören: »Verflucht vor
dem HERRN sei der Mann, der es unternimmt, diese Stadt Jericho wieder
aufzubauen! Um den Preis seines Erstgeborenen wird\textless sup
title=``oder: soll''\textgreater✲ er ihren Grundstein legen und um den
Preis seines jüngsten Sohnes ihre Tore einsetzen!«\textless sup
title=``vgl. 1.Kön 16,34''\textgreater✲

27Der HERR aber war mit Josua, so daß sein Ruhm sich durch das Land
verbreitete.

\hypertarget{b-entdeckung-und-bestrafung-von-achans-diebstahl}{%
\paragraph{b) Entdeckung und Bestrafung von Achans
Diebstahl}\label{b-entdeckung-und-bestrafung-von-achans-diebstahl}}

\hypertarget{aa-miuxdflingen-des-sorgsam-vorbereiteten-zuges-gegen-ai-verzagtheit-des-volkes-josuas-flehentliches-gebet}{%
\subparagraph{aa) Mißlingen des sorgsam vorbereiteten Zuges gegen Ai;
Verzagtheit des Volkes; Josuas flehentliches
Gebet}\label{aa-miuxdflingen-des-sorgsam-vorbereiteten-zuges-gegen-ai-verzagtheit-des-volkes-josuas-flehentliches-gebet}}

\hypertarget{section-6}{%
\section{7}\label{section-6}}

1Die Israeliten hatten sich aber eine Veruntreuung an dem gebannten Gut
zuschulden kommen lassen; denn Achan, der Sohn Karmis, des Sohnes
Sabdis, des Sohnes Serahs, vom Stamme Juda, hatte sich etwas von dem
gebannten Gut angeeignet. Darob entbrannte der Zorn des HERRN gegen die
Israeliten.~-- 2Nun sandte Josua einige Männer von Jericho aus nach Ai,
das bei Beth-Awen östlich von Bethel liegt, mit der Weisung: »Geht
hinauf und kundschaftet die Gegend aus!« Als nun die Männer
hinaufgegangen waren und Ai ausgekundschaftet hatten, 3berichteten sie
dem Josua nach ihrer Rückkehr: »Nicht das gesamte Volk braucht
hinaufzuziehen; zwei- bis dreitausend Mann genügen, um Ai zu erobern.
Bemühe nicht das ganze Volk dorthin; denn die Zahl ihrer Leute ist
gering.« 4So zogen denn von dem Volk etwa dreitausend Mann dorthin,
wurden aber von den Aiten in die Flucht geschlagen, 5und die Aiten
erschlugen etwa sechsunddreißig Mann von ihnen, verfolgten sie dann von
dem Stadttor bis an die Steinbrüche und schlugen sie am Bergabhang. Da
schwand dem Volk aller Mut und schlug in Verzagtheit um; 6Josua aber
zerriß seine Kleider, warf sich vor der Lade des HERRN auf sein
Angesicht zur Erde nieder bis zum Abend, er samt den Ältesten der
Israeliten, und sie streuten sich Staub\textless sup title=``oder:
Asche''\textgreater✲ aufs Haupt. 7Darauf betete Josua: »Ach, HERR, mein
Gott! Warum hast du dieses Volk über den Jordan geführt? Um uns in die
Hand der Amoriter fallen zu lassen, damit sie uns vernichten? O hätten
wir uns doch daran genügen lassen, jenseits des Jordans wohnen zu
bleiben! 8Verzeihe, HERR! Was soll ich sagen, nachdem Israel sich vor
seinen Feinden zur Flucht gewandt hat? 9Wenn das die Kanaanäer und alle
übrigen Bewohner des Landes erfahren, so werden sie von allen Seiten
über uns herfallen und unsern Namen von der Erde vertilgen! Was willst
du nun\textless sup title=``oder: dann''\textgreater✲ für deinen großen
Namen tun?«

\hypertarget{bb-gott-teilt-dem-josua-den-grund-seines-zornes-mit-und-erteilt-anweisung-zur-ermittlung-des-schuldigen}{%
\subparagraph{bb) Gott teilt dem Josua den Grund seines Zornes mit und
erteilt Anweisung zur Ermittlung des
Schuldigen}\label{bb-gott-teilt-dem-josua-den-grund-seines-zornes-mit-und-erteilt-anweisung-zur-ermittlung-des-schuldigen}}

10Da antwortete der HERR dem Josua: »Stehe auf! Wozu liegst du da auf
deinem Angesicht? 11Israel hat sich versündigt! Denn sie haben sowohl
mein Bundesgebot übertreten, das ich ihnen zur Pflicht gemacht habe, als
auch sich etwas von dem gebannten Gut angeeignet; so haben sie sowohl
einen Diebstahl begangen als auch das Gestohlene versteckt und unter
ihre eigenen Geräte getan. 12Daher vermögen die Israeliten jetzt vor
ihren Feinden nicht mehr standzuhalten, sondern müssen vor ihren Feinden
die Flucht ergreifen; denn sie sind selbst dem Bann verfallen. Ich werde
hinfort nicht mehr mit euch sein, wenn ihr das gebannte Gut nicht aus
eurer Mitte wegschafft. 13Stehe auf, laß das Volk sich heiligen und
mache bekannt: ›Heiligt euch auf morgen!‹ Denn so hat der HERR, der Gott
Israels, gesprochen: ›Gebanntes Gut befindet sich in deiner Mitte,
Israel; du wirst deinen Feinden nicht eher zu widerstehen vermögen, als
bis ihr das gebannte Gut aus eurer Mitte weggeschafft habt.‹ 14Darum
sollt ihr morgen früh antreten, Stamm für Stamm; und der Stamm, den der
HERR durch das Los bezeichnet, soll herantreten, ein Geschlecht nach dem
andern; und das Geschlecht, das der HERR durch das Los bezeichnet, soll
herantreten, eine Familie nach der andern; und die Familie, die der HERR
durch das Los bezeichnet, soll Mann für Mann herantreten. 15Wer dann als
dem Bann verfallen durch das Los bezeichnet wird, soll mit allem, was er
besitzt, im Feuer verbrannt werden, weil er das Bundesgebot des HERRN
übertreten und eine Schandtat in Israel begangen hat!«

\hypertarget{cc-achan-wird-durch-das-los-als-frevler-ausfindig-gemacht-und-nach-eingestuxe4ndnis-seiner-schuld-gesteinigt}{%
\subparagraph{cc) Achan wird durch das Los als Frevler ausfindig gemacht
und nach Eingeständnis seiner Schuld
gesteinigt}\label{cc-achan-wird-durch-das-los-als-frevler-ausfindig-gemacht-und-nach-eingestuxe4ndnis-seiner-schuld-gesteinigt}}

16Da ließ Josua am andern Morgen in der Frühe die Israeliten antreten,
Stamm für Stamm; da wurde der Stamm Juda durch das Los getroffen. 17Als
er dann die Geschlechter von Juda herantreten ließ, wurde das Geschlecht
der Sarchiten getroffen. Als er dann das Geschlecht der Sarchiten
herantreten ließ, eine Familie nach der andern, wurde die Familie Sabdis
getroffen; 18und als er dessen Familie Mann für Mann herantreten ließ,
wurde Achan getroffen, der Sohn Karmis, des Sohnes Sabdis, des Sohnes
Serahs, vom Stamme Juda. 19Da sagte Josua zu Achan: »Mein Sohn, gib doch
dem HERRN, dem Gott Israels, die Ehre und lege ein offenes Bekenntnis
vor ihm ab: gestehe mir, was du getan hast: verhehle mir nichts!« 20Da
antwortete Achan dem Josua: »Fürwahr, ich habe mich am HERRN, dem Gott
Israels, versündigt! So und so habe ich getan: 21ich sah unter den
Beutestücken einen schönen babylonischen Mantel, dazu zweihundert
Schekel Silber und eine Goldstange, fünfzig Schekel an Gewicht; da
gelüstete mich nach diesen Sachen, und ich eignete sie mir an; sie sind
jetzt mitten in meinem Zelt im Boden vergraben, und das Silber liegt
zuunterst.« 22Da sandte Josua Boten hin, die liefen zum Zelt, und man
fand die Sachen wirklich in seinem Zelt vergraben, und das Silber lag
zuunterst. 23Da nahmen sie die Sachen aus dem Zelte mit, brachten sie zu
Josua und zu allen Israeliten und legten sie vor den HERRN hin. 24Nun
nahm Josua, und ganz Israel mit ihm, Achan, den Sohn Serahs, und das
Silber, den Mantel und die Goldstange, dazu seine Söhne und Töchter,
auch seine Rinder, seine Esel und sein Kleinvieh, ferner sein Zelt und
alles, was er sonst noch besaß, und brachten das alles in das Tal Achor
hinauf. 25Dort sagte Josua zu Achan: »Wie hast du uns ins Unglück
gestürzt! Dafür möge der HERR auch dich heute ins Unglück stürzen!«
Hierauf steinigten ihn alle Israeliten {[}und man verbrannte sie im
Feuer und vollzog die Steinigung an ihnen{]}; 26dann errichteten sie
über ihm einen großen Steinhaufen, der noch heutigentags dort liegt. Da
ließ der HERR von seiner Zornesglut ab. Daher heißt jener Ort bis auf
den heutigen Tag das Tal Achor\textless sup title=``d.h.
Unglückstal''\textgreater✲.

\hypertarget{c-eroberung-und-zerstuxf6rung-von-ai}{%
\paragraph{c) Eroberung und Zerstörung von
Ai}\label{c-eroberung-und-zerstuxf6rung-von-ai}}

\hypertarget{aa-auf-guxf6ttliche-anweisung-zieht-josua-gegen-ai-und-legt-einen-hinterhalt-im-westen-der-stadt}{%
\subparagraph{aa) Auf göttliche Anweisung zieht Josua gegen Ai und legt
einen Hinterhalt im Westen der
Stadt}\label{aa-auf-guxf6ttliche-anweisung-zieht-josua-gegen-ai-und-legt-einen-hinterhalt-im-westen-der-stadt}}

\hypertarget{section-7}{%
\section{8}\label{section-7}}

1Hierauf sagte der HERR zu Josua: »Fürchte dich nicht und sei ohne
Angst! Nimm das gesamte Kriegsvolk mit dir und mache dich auf, ziehe
gegen Ai hinauf: hiermit gebe ich den König von Ai samt seinem Volk,
seiner Stadt und seinem Lande in deine Gewalt. 2Du sollst mit Ai und
seinem König so verfahren, wie du mit Jericho und seinem König verfahren
bist; jedoch die dortige Beute und das Vieh dürft ihr für euch behalten.
Lege dir einen Hinterhalt gegen die Stadt auf ihrer Westseite!«

3Da machte sich Josua mit dem gesamten Kriegsvolk auf, um gegen Ai
hinaufzuziehen. Er wählte aber 30000~Mann tapfere Krieger aus und sandte
sie bei Nacht ab, 4nachdem er ihnen folgenden Befehl gegeben hatte:
»Wisset wohl: ihr sollt euch gegen die Stadt in den Hinterhalt legen,
und zwar auf der Westseite der Stadt; entfernt euch nicht zu weit von
der Stadt und haltet euch allesamt bereit! 5Ich aber und das ganze
Kriegsvolk, das bei mir ist, werden gegen die Stadt anrücken, und wenn
sie dann wie das vorige Mal einen Ausfall gegen uns machen, wollen wir
vor ihnen fliehen. 6Sie werden dann herauskommen, um uns zu verfolgen,
bis wir sie von der Stadt ganz abgezogen haben; denn sie werden denken:
›Die fliehen vor uns wie das vorige Mal‹; und wir wollen ja auch
wirklich vor ihnen fliehen. 7Dann müßt ihr aus dem Hinterhalt
hervorbrechen und euch der Stadt bemächtigen; denn der HERR, euer Gott,
wird sie in eure Gewalt geben. 8Wenn ihr aber die Stadt eingenommen
habt, so steckt sie in Brand: nach dem Gebot des HERRN sollt ihr
verfahren! Das sind meine Befehle für euch.« 9Als Josua sie so entlassen
hatte, zogen sie in den Hinterhalt und nahmen Stellung zwischen Bethel
und Ai, und zwar auf der Westseite von Ai, während Josua jene Nacht
inmitten des übrigen Kriegsvolkes verbrachte.

10Josua machte sich dann am folgenden Morgen früh auf und zog, nachdem
er das Kriegsvolk gemustert hatte, mit den Ältesten\textless sup
title=``oder: Vornehmsten''\textgreater✲ der Israeliten an der Spitze
des Heeres gegen Ai hinauf. 11Das gesamte Kriegsvolk, das bei ihm war,
zog hinauf und rückte heran und kam vor der Stadt an; sie lagerten
nördlich von Ai, so daß das Tal zwischen ihnen und Ai lag. 12{[}Er hatte
aber etwa fünftausend Mann genommen und sie als Hinterhalt zwischen
Bethel und Ai, auf der Westseite der Stadt, Stellung nehmen lassen.
13Dann stellten sie das Volk auf, das ganze Heer, das sich nördlich von
der Stadt befand, und dessen Hinterhalt westlich von der Stadt; Josua
aber begab sich in jener Nacht mitten in das Tal.{]}

\hypertarget{bb-verlauf-des-kampfes-verbrennung-der-unbewachten-stadt}{%
\subparagraph{bb) Verlauf des Kampfes; Verbrennung der unbewachten
Stadt}\label{bb-verlauf-des-kampfes-verbrennung-der-unbewachten-stadt}}

14Als nun der König von Ai dies sah, machten sich die Männer der Stadt
in der Frühe eilends auf und zogen aus, den Israeliten zum Kampf
entgegen, er und sein ganzes Volk, an den bestimmten Ort vor der Steppe,
ohne zu wissen, daß ihm auf der Westseite der Stadt ein Hinterhalt
gelegt war. 15Da ließen sich Josua und ganz Israel von ihnen schlagen
und flohen in der Richtung nach der Steppe zu. 16Darauf wurde das
gesamte Volk, das noch in der Stadt war, zu ihrer Verfolgung aufgeboten,
und sie verfolgten Josua und ließen sich so von der Stadt weglocken:
17kein Mann blieb in Ai {[}und Bethel{]} zurück, der nicht zur
Verfolgung der Israeliten ausgezogen wäre; die Stadt aber ließen sie
hinter sich offen stehen und verfolgten die Israeliten. 18Da sagte der
HERR zu Josua: »Strecke die Lanze, die du in der Hand hast, gegen Ai
aus, denn ich gebe es in deine Gewalt.« Da streckte Josua die Lanze, die
er in der Hand hatte, gegen die Stadt aus. 19Nun verließ der Hinterhalt
in aller Eile seinen Standort und gelangte, als Josua seine Hand
ausgestreckt hatte, im Lauf zur Stadt, die sie einnahmen und sofort in
Brand steckten. 20Als nun die Aiten sich umwandten und den Rauch von der
Stadt zum Himmel aufsteigen sahen, hatten sie keine Möglichkeit mehr,
hierhin oder dorthin zu fliehen; denn das israelitische Heer, das nach
der Wüste hin geflohen war, wandte sich gegen seine Verfolger um. 21Denn
als Josua und alle Israeliten sahen, daß der Hinterhalt die Stadt
eingenommen hatte und daß der Rauch von der Stadt aufstieg, machten sie
kehrt und schlugen die Aiten. 22Jene anderen aber waren ihnen aus der
Stadt entgegengezogen, so daß sich die Aiten mitten zwischen den
Israeliten befanden, da die einen von dieser, die anderen von jener
Seite kamen; und sie machten sie nieder, bis kein einziger von ihnen
übrig war, der sich hätte retten oder entfliehen können. 23Den König von
Ai aber nahmen sie lebendig gefangen und brachten ihn zu Josua.

\hypertarget{cc-vollziehung-des-bannes-an-der-stadt-der-kuxf6nig-getuxf6tet-und-bis-zum-abend-aufgehuxe4ngt}{%
\subparagraph{cc) Vollziehung des Bannes an der Stadt; der König getötet
und bis zum Abend
aufgehängt}\label{cc-vollziehung-des-bannes-an-der-stadt-der-kuxf6nig-getuxf6tet-und-bis-zum-abend-aufgehuxe4ngt}}

24Als nun die Israeliten alle Bewohner von Ai auf dem freien Felde und
in der Wüste, wohin die sie verfolgt hatten, niedergemacht hatten und
diese alle bis auf den letzten Mann durch das Schwert gefallen waren, da
wandten sich alle Israeliten wieder gegen Ai und machten alle Einwohner
ohne Gnade nieder. 25Die Zahl aller, die an diesem Tage fielen, Männer
und Weiber, betrug zwölftausend, die ganze Einwohnerschaft von Ai.
26Josua hatte nämlich seine Hand, die er mit der Lanze ausgestreckt
hatte, nicht eher zurückgezogen, als bis der Bann an allen Bewohnern von
Ai vollstreckt war. 27Nur das Vieh und die Beute dieser Stadt behielten
die Israeliten für sich, nach der Weisung, die der HERR dem Josua
erteilt hatte. 28Josua verbrannte dann Ai und machte es für ewige Zeiten
zu einem Schutthaufen, zu einer wüsten Stätte, bis auf den heutigen Tag.
29Den König von Ai aber ließ er bis zur Abendzeit an einen
Baum\textless sup title=``oder: Pfahl''\textgreater✲ hängen; doch bei
Sonnenuntergang nahm man seinen Leichnam auf Befehl Josuas von dem Baume
ab und warf ihn an den Eingang des Stadttores, wo man über ihm einen
großen Steinhaufen errichtete, der dort noch bis zum heutigen Tage
vorhanden ist.

\hypertarget{d-erbauung-eines-altars-auf-dem-berge-ebal-und-vorlesung-des-gesetzes-durch-josua-nach-dem-opferfest}{%
\paragraph{d) Erbauung eines Altars auf dem Berge Ebal und Vorlesung des
Gesetzes durch Josua nach dem
Opferfest}\label{d-erbauung-eines-altars-auf-dem-berge-ebal-und-vorlesung-des-gesetzes-durch-josua-nach-dem-opferfest}}

30Damals baute Josua dem HERRN, dem Gott Israels, einen Altar auf dem
Berge Ebal, 31wie Mose, der Knecht des HERRN, den Israeliten geboten
hatte, wie im Gesetzbuch Moses geschrieben steht\textless sup
title=``2.Mose 20,25''\textgreater✲, einen Altar aus unbehauenen
Steinen, an die noch kein eisernes Werkzeug gekommen war; und sie
brachten auf ihm dem HERRN Brandopfer dar und schlachteten Dankopfer.
32Dann schrieb er dort auf Steine eine Abschrift des mosaischen
Gesetzes, das (Mose) einst in Gegenwart der Israeliten geschrieben
hatte. 33Und ganz Israel mit seinen Ältesten und Obmännern\textless sup
title=``vgl. 5.Mose 1,15''\textgreater✲ und seinen Richtern stand zu
beiden Seiten der Lade den levitischen Priestern gegenüber, welche die
Bundeslade des HERRN zu tragen hatten, sowohl die Fremdlinge als auch
die Einheimischen, die eine Hälfte von ihnen nach dem Berge Garizim zu,
die andere Hälfte nach dem Berge Ebal hin, wie Mose, der Knecht des
HERRN, einstmals geboten hatte, das Volk Israel zu segnen. 34Darnach las
Josua alle Worte des Gesetzes laut vor, den Segen wie den Fluch, genau
so, wie es im Gesetzbuch geschrieben steht: 35es gab kein Wort von
allem, was Mose geboten hatte, das Josua nicht vor der ganzen
Versammlung der Israeliten, auch vor den Frauen und Kindern und den
Fremdlingen, die unter ihnen mitzogen, laut vorgelesen hätte.

\hypertarget{e-uxfcberlistung-der-israeliten-durch-die-gibeoniten}{%
\paragraph{e) Überlistung der Israeliten durch die
Gibeoniten}\label{e-uxfcberlistung-der-israeliten-durch-die-gibeoniten}}

\hypertarget{aa-die-kanaanuxe4erkuxf6nige-schlieuxdfen-einen-bund-gegen-israel}{%
\subparagraph{aa) Die Kanaanäerkönige schließen einen Bund gegen
Israel}\label{aa-die-kanaanuxe4erkuxf6nige-schlieuxdfen-einen-bund-gegen-israel}}

\hypertarget{section-8}{%
\section{9}\label{section-8}}

1Als nun alle Könige dies vernahmen, die diesseits des Jordans im
Berglande, in der Niederung und an der ganzen Küste des großen Meeres
nach dem Libanon hin wohnten, nämlich die Hethiter und die Amoriter, die
Kanaanäer, Pherissiter, Hewiter und Jebusiter, 2da taten sie sich alle
zusammen, um einmütig gegen Josua und die Israeliten zu kämpfen.

\hypertarget{bb-die-gibeoniten-senden-eine-gesandtschaft-und-erlangen-durch-betrug-ein-friedliches-abkommen-mit-den-israeliten}{%
\subparagraph{bb) Die Gibeoniten senden eine Gesandtschaft und erlangen
durch Betrug ein friedliches Abkommen mit den
Israeliten}\label{bb-die-gibeoniten-senden-eine-gesandtschaft-und-erlangen-durch-betrug-ein-friedliches-abkommen-mit-den-israeliten}}

3Als aber die Einwohner von Gibeon vernahmen, wie Josua mit Jericho und
Ai verfahren war, 4gingen sie ihrerseits mit List zu Werke: sie machten
sich auf den Weg, und zwar versorgten sie sich mit Lebensmitteln, nahmen
alte Säcke für ihre Esel und alte, geborstene und geflickte
Weinschläuche, 5zogen alte, geflickte Schuhe und abgetragene Kleider an,
und alles Brot, das sie als Mundvorrat bei sich hatten, war vertrocknet
und zerbröckelt. 6Als sie so zu Josua ins Lager nach Gilgal gekommen
waren, sagten sie zu ihm und zu den Israeliten: »Aus einem fernen Lande
sind wir gekommen; so schließt denn jetzt einen Vertrag mit uns.« 7Da
antworteten die Israeliten den Hewitern: »Vielleicht wohnt ihr mitten
unter uns: wie könnten wir da einen Vertrag mit euch schließen?« 8Da
sagten sie zu Josua: »Wir sind deine Knechte.« Als Josua sie nun fragte:
»Wer seid ihr, und woher kommt ihr?«, 9antworteten sie ihm: »Aus einem
ganz fernen Lande sind deine Knechte infolge des Ruhmes des HERRN,
deines Gottes, gekommen; denn wir haben die Kunde von ihm vernommen,
sowohl alles, was er an Ägypten getan hat, 10als auch alles, was er an
den beiden Königen der Amoriter jenseits des Jordans getan hat, an
Sihon, dem König von Hesbon, und an Og, dem König von Basan, der zu
Astaroth wohnte. 11Da haben unsere Ältesten und alle Bewohner unsers
Landes zu uns gesagt: ›Nehmt Lebensmittel mit euch auf den Weg, geht
ihnen entgegen und sagt zu ihnen: ›Wir sind eure Knechte; so schließt
nun einen Vertrag mit uns!‹ 12Hier ist unser Brot: es war noch warm, als
wir es aus unseren Häusern als Reisekost mitnahmen an dem Tage, als wir
aufbrachen, um zu euch zu ziehen; nun aber seht ihr, daß es vertrocknet
und zerbröckelt geworden ist. 13Und hier sind unsere Weinschläuche, die
neu waren, als wir sie füllten; nun aber sind sie geborsten, wie ihr
seht, und unsere Kleider und Schuhe hier sind von dem sehr weiten Wege
ganz abgenutzt.« 14Da ließen sich die israelitischen Männer etwas von
ihrem Mundvorrat geben, aber einen Ausspruch des HERRN holten sie nicht
ein, 15sondern Josua gewährte ihnen Frieden✲ und schloß einen Vertrag
mit ihnen, daß er sie am Leben lassen wolle; und die Fürsten der
Gemeinde leisteten ihnen einen Eid.

\hypertarget{cc-die-gibeoniten-nach-der-entdeckung-ihres-betrugs-zu-gemeinde--und-tempelknechten-gemacht}{%
\subparagraph{cc) Die Gibeoniten nach der Entdeckung ihres Betrugs zu
Gemeinde- und Tempelknechten
gemacht}\label{cc-die-gibeoniten-nach-der-entdeckung-ihres-betrugs-zu-gemeinde--und-tempelknechten-gemacht}}

16Als aber nach Abschluß des Vertrags mit ihnen drei Tage vergangen
waren, da erfuhr man, daß jene ganz aus der Nähe waren und mitten unter
ihnen wohnten. 17Denn als die Israeliten weiter zogen, kamen sie am
dritten Tage zu ihren Städten, nämlich zu den Ortschaften Gibeon,
Kephira, Beeroth und Kirjath-Jearim. 18Die Israeliten taten ihnen aber
nichts zuleide, weil die Fürsten der Gemeinde ihnen beim HERRN, dem Gott
Israels, geschworen hatten. Als nun die ganze Gemeinde über die Fürsten
murrte, 19sagten diese alle zu der ganzen Gemeinde: »Wir haben ihnen
beim HERRN, dem Gott Israels, einen Eid geleistet; darum dürfen wir uns
jetzt nicht an ihnen vergreifen. 20Wir wollen so mit ihnen verfahren,
daß wir sie am Leben lassen, damit kein Zorngericht\textless sup
title=``oder: Strafgericht''\textgreater✲ über uns komme wegen des
Eides, den wir ihnen geschworen haben.« 21Die Fürsten gaben also (vor
dem Volk) die Erklärung ab: »Sie sollen am Leben bleiben, aber Holzhauer
und Wasserträger für die ganze Gemeinde werden.« (Da taten die
Israeliten,) wie die Fürsten ihnen vorgeschlagen hatten.

22Hierauf ließ Josua sie rufen und sagte bei der Unterredung zu ihnen:
»Warum habt ihr uns getäuscht, indem ihr vorgabt: ›Wir wohnen sehr weit
von euch weg‹, während ihr doch mitten unter uns wohnt? 23Nun denn, so
sollt ihr verflucht sein und in Zukunft immer Knechte, sowohl Holzhauer
als Wasserträger, für das Haus meines Gottes sein!« 24Da gaben sie dem
Josua zur Antwort: »Deinen Knechten wurde als gewiß mitgeteilt, daß der
HERR, dein Gott, seinem Knechte Mose geboten hat, euch dies ganze Land
zu geben, und ihr solltet alle Bewohner des Landes vor euch her
vertilgen. Da gerieten wir vor euch in große Furcht für unser Leben und
haben so gehandelt. 25Und nun -- wir sind ja in deiner Gewalt: verfahre
mit uns so, wie es dich gut und recht dünkt!« 26Da verfuhr er in der
angegebenen Weise mit ihnen und rettete sie aus der Hand der Israeliten,
so daß diese sie nicht ums Leben brachten. 27Josua machte sie also
damals zu Holzhauern und Wasserträgern für die Gemeinde und für den
Altar des HERRN\textless sup title=``d.h. zu
Tempelsklaven''\textgreater✲ an der Stätte, die der HERR erwählen würde;
und so ist's geblieben bis auf den heutigen Tag.

\hypertarget{f-sieg-der-israeliten-uxfcber-die-fuxfcnf-amoriterkuxf6nige-bei-gibeon-und-eroberung-der-suxfcdlichen-landeshuxe4lfte}{%
\paragraph{f) Sieg der Israeliten über die fünf Amoriterkönige bei
Gibeon und Eroberung der südlichen
Landeshälfte}\label{f-sieg-der-israeliten-uxfcber-die-fuxfcnf-amoriterkuxf6nige-bei-gibeon-und-eroberung-der-suxfcdlichen-landeshuxe4lfte}}

\hypertarget{aa-der-zug-der-fuxfcnf-kuxf6nige-gegen-gibeon-josuas-sieg-bei-gibeon}{%
\subparagraph{aa) Der Zug der fünf Könige gegen Gibeon; Josuas Sieg bei
Gibeon}\label{aa-der-zug-der-fuxfcnf-kuxf6nige-gegen-gibeon-josuas-sieg-bei-gibeon}}

\hypertarget{section-9}{%
\section{10}\label{section-9}}

1Als nun Adoni-Zedek, der König von Jerusalem, die Kunde erhielt, daß
Josua Ai erobert und den Bann an der Stadt vollstreckt habe, daß er mit
Ai und seinem König ebenso wie mit Jericho und seinem König verfahren
sei und daß die Bewohner von Gibeon Frieden mit den Israeliten
geschlossen hätten und mitten unter ihnen wohnen geblieben seien, 2da
fürchteten sie sich sehr; denn Gibeon war eine bedeutende Stadt, so groß
wie irgendeine von den Königstädten und noch größer als Ai, und alle
ihre Männer tapfere Krieger. 3Daher schickte Adoni-Zedek, der König von
Jerusalem, Gesandte an Hoham, den König von Hebron, sowie an Piream, den
König von Jarmuth, an Japhia, den König von Lachis, und an Debir, den
König von Eglon, und ließ ihnen sagen: 4»Zieht zu mir herauf und helft
mir, Gibeon dafür zu strafen, daß es mit Josua und den Israeliten
Frieden geschlossen hat.« 5Da vereinigten sich und zogen hinauf die fünf
Amoriterkönige: der König von Jerusalem, der König von Hebron, der König
von Jarmuth, der König von Lachis und der König von Eglon mit allen
ihren Heeren und belagerten Gibeon und bestürmten es. 6Da sandten die
Gibeoniten Boten zu Josua in das Lager nach Gilgal und ließen ihm sagen:
»Laß deine Knechte nicht im Stich! Komm eilends zu uns herauf, rette uns
und hilf uns! Denn alle Könige der Amoriter, die das Bergland bewohnen,
haben sich gegen uns verbündet.« 7Da zog Josua mit seinem gesamten
Kriegsvolk, lauter tapferen Männern, von Gilgal aus hinauf; 8und der
HERR sagte zu ihm: »Fürchte dich nicht vor ihnen! Denn ich habe sie in
deine Gewalt gegeben: kein einziger von ihnen soll vor dir standhalten
können.« 9Als Josua sie nun plötzlich überfiel -- die ganze Nacht
hindurch war er von Gilgal hinaufgezogen --, 10ließ der HERR unter ihnen
einen plötzlichen Schrecken vor den Israeliten entstehen, so daß diese
ihnen eine schwere Niederlage bei Gibeon beibrachten und sie in der
Richtung nach dem Berghang von Beth-Horon verfolgten und sie bis Aseka
und Makkeda schlugen.

\hypertarget{bb-die-zwei-grouxdfen-wundertaten-gottes-steinhagel-und-sonnenstillstand}{%
\subparagraph{bb) Die zwei großen Wundertaten Gottes: Steinhagel und
Sonnenstillstand}\label{bb-die-zwei-grouxdfen-wundertaten-gottes-steinhagel-und-sonnenstillstand}}

11Als sie sich nun auf der Flucht vor den Israeliten am Abhang von
Beth-Horon befanden, ließ der HERR große Steine\textless sup
title=``d.h. Hagelstücke''\textgreater✲ vom Himmel bis nach Aseka hin
auf sie herabfallen, so daß sie dadurch den Tod fanden; die Zahl derer,
welche durch den Steinhagel das Leben verloren, war größer als die Zahl
derer, welche durch das Schwert der Israeliten gefallen waren. 12Damals
betete Josua zum HERRN, an dem Tage, an dem der HERR die Amoriter den
Israeliten preisgab, und zwar rief er angesichts der Israeliten aus:
»Sonne, stehe still zu Gibeon und du, Mond, im Tal von Ajjalon!« 13Da
stand die Sonne still, und der Mond blieb stehen, bis das Volk Rache an
seinen Feinden genommen hatte. Das steht bekanntlich im »Buch des
Braven« geschrieben. Die Sonne blieb also mitten am Himmel stehen und
eilte beinahe einen ganzen Tag lang nicht zum Untergang. 14Einen Tag wie
diesen hat es weder vorher noch später gegeben, daß der HERR auf die
Stimme eines Menschen gehört hätte; denn der HERR stritt für Israel.
15{[}Hierauf kehrte Josua und ganz Israel mit ihm in das Lager nach
Gilgal zurück.{]}

\hypertarget{cc-fortsetzung-der-verfolgung-die-fuxfcnf-in-einer-huxf6hle-gefangenen-amoriterkuxf6nige-werden-getuxf6tet-und-aufgehuxe4ngt}{%
\subparagraph{cc) Fortsetzung der Verfolgung; die fünf in einer Höhle
gefangenen Amoriterkönige werden getötet und
aufgehängt}\label{cc-fortsetzung-der-verfolgung-die-fuxfcnf-in-einer-huxf6hle-gefangenen-amoriterkuxf6nige-werden-getuxf6tet-und-aufgehuxe4ngt}}

16Jene fünf Könige aber waren geflohen und hatten sich in
der\textless sup title=``oder: einer''\textgreater✲ Höhle bei Makkeda
versteckt. 17Als nun Josua die Meldung erhielt, daß man die fünf Könige
versteckt in der Höhle bei Makkeda gefunden habe, 18gab er den Befehl:
»Wälzt große Steine vor den Eingang der Höhle und stellt Leute zu ihrer
Bewachung daneben auf! 19Ihr anderen aber steht nicht still, sondern
verfolgt eure Feinde und haut ihre Nachzügler nieder! Laßt sie nicht in
ihre Städte entkommen; denn der HERR, euer Gott, hat sie in eure Gewalt
gegeben.«

20Als nun Josua und die Israeliten eine sehr schwere Niederlage unter
ihnen bis zur völligen Vernichtung angerichtet hatten -- diejenigen von
ihnen, welche entkommen waren, hatten sich in die festen Plätze
geflüchtet~-- 21und als das ganze Kriegsvolk unangefochten zu Josua in
das Lager nach Makkeda zurückgekehrt war, ohne daß jemand den Israeliten
auch nur das Geringste hatte anhaben können, 22da gab Josua den Befehl:
»Macht den Eingang der Höhle frei und bringt jene fünf Könige aus der
Höhle zu mir heraus!« 23Man kam dem Befehle nach und brachte jene fünf
Könige aus der Höhle zu ihm heraus, die Könige von Jerusalem, von
Hebron, von Jarmuth, von Lachis und von Eglon. 24Als man diese Könige
nun zu Josua herausgeführt hatte, rief Josua alle Israeliten herbei und
sagte zu den Anführern der Kriegsleute, die mit ihm gezogen waren:
»Tretet heran und setzt diesen Königen den Fuß auf den Nacken!« Da
traten sie heran und setzten ihnen den Fuß auf den Nacken. 25Dann fuhr
Josua fort: »Fürchtet euch nicht und habt keine Angst, seid mutig und
stark! Denn ebenso wird der HERR es mit allen euren Feinden machen, mit
denen ihr noch zu kämpfen habt.« 26Hierauf ließ Josua sie totschlagen
und an fünf Bäumen\textless sup title=``oder: Pfählen''\textgreater✲
aufhängen; und sie blieben an den Bäumen bis zum Abend hängen. 27Bei
Sonnenuntergang aber nahm man sie auf Befehl Josuas von den Bäumen ab
und warf sie in die Höhle, in der sie sich versteckt hatten, und legte
große Steine an den Eingang der Höhle, die noch bis zum heutigen Tage
dort liegen.

\hypertarget{dd-unterwerfung-der-ganzen-suxfcdhuxe4lfte-kanaans-josuas-ruxfcckkehr-nach-gilgal}{%
\subparagraph{dd) Unterwerfung der ganzen Südhälfte Kanaans; Josuas
Rückkehr nach
Gilgal}\label{dd-unterwerfung-der-ganzen-suxfcdhuxe4lfte-kanaans-josuas-ruxfcckkehr-nach-gilgal}}

28Josua eroberte dann an jenem Tage noch Makkeda und schlug es mit der
Schärfe des Schwertes, indem er am dortigen König, an der Stadt und der
gesamten Einwohnerschaft den Bann vollstreckte; er verschonte keinen
einzigen von ihnen und verfuhr mit dem König von Makkeda, wie er mit dem
König von Jericho verfahren war.~-- 29Hierauf zog Josua mit allen
Israeliten von Makkeda weiter nach Libna und bestürmte es; 30und der
HERR ließ auch diese Stadt in die Hand der Israeliten fallen samt ihrem
König, und er schlug sie mit der Schärfe des Schwertes samt der ganzen
Einwohnerschaft; er verschonte keinen einzigen von ihnen und verfuhr mit
dem dortigen König, wie er mit dem König von Jericho verfahren war.~--
31Dann zog Josua mit allen Israeliten von Libna weiter nach Lachis, das
er belagerte und bestürmte. 32Und der HERR ließ auch Lachis in die
Gewalt der Israeliten fallen, so daß Josua es am zweiten Tage eroberte
und die Stadt samt der ganzen Einwohnerschaft mit der Schärfe des
Schwertes schlug, genau so, wie er es mit Libna gemacht hatte. 33Damals
zog Horam, der König von Geser, herauf, um Lachis Hilfe zu leisten; aber
Josua schlug ihn und sein Kriegsvolk so, daß kein einziger von ihnen am
Leben blieb.~-- 34Dann zog Josua mit allen Israeliten von Lachis weiter
nach Eglon, das sie belagerten und bestürmten. 35Sie eroberten es noch
an demselben Tage und schlugen es mit der Schärfe des Schwertes; an der
ganzen Einwohnerschaft vollstreckte er an jenem Tage den Bann, genau so,
wie er es mit Lachis gemacht hatte.~-- 36Hierauf zog Josua mit allen
Israeliten von Eglon nach Hebron hinauf, das sie bestürmten; 37sie
eroberten es und schlugen es mit der Schärfe des Schwertes samt seinem
König und allen zugehörigen Ortschaften und deren gesamten
Einwohnerschaft, ohne einen einzigen zu verschonen, genau so, wie er es
mit Eglon gemacht hatte, indem er an der Stadt und allen ihren
Einwohnern den Bann vollstreckte.~-- 38Dann wandte sich Josua mit allen
Israeliten gegen Debir, das er bestürmte. 39Er nahm die Stadt ein samt
ihrem König und allen zugehörigen Ortschaften; und sie schlugen sie mit
der Schärfe des Schwertes und vollzogen an der gesamten Einwohnerschaft
den Bann, ohne einen einzigen zu verschonen; wie er es mit Hebron
gemacht hatte und wie er mit Libna und dem dortigen König verfahren war,
ebenso machte er es auch mit Debir und dem dortigen König.

40So unterwarf Josua das ganze Land, nämlich das Bergland und das
Südland, die Niederung und das Hügelland samt allen ihren Königen, ohne
auch nur einen einzigen übrigzulassen; und an allem Lebenden
vollstreckte er den Bann, wie der HERR, der Gott Israels, geboten hatte.
41Josua unterwarf sie von Kades-Barnea bis Gaza und das ganze Land Gosen
bis Gibeon; 42und zwar brachte Josua alle diese Könige und ihr Land auf
einmal in seine Gewalt, denn der HERR, der Gott Israels, stritt für
Israel. 43Hierauf kehrte Josua mit allen Israeliten in das Lager nach
Gilgal zurück.

\hypertarget{g-besiegung-der-verbuxfcndeten-nordkanaanuxe4ischen-kuxf6nige-in-der-schlacht-am-see-merom-eroberung-des-nordens}{%
\paragraph{g) Besiegung der verbündeten nordkanaanäischen Könige in der
Schlacht am See Merom; Eroberung des
Nordens}\label{g-besiegung-der-verbuxfcndeten-nordkanaanuxe4ischen-kuxf6nige-in-der-schlacht-am-see-merom-eroberung-des-nordens}}

\hypertarget{aa-die-von-jabin-zum-bunde-vereinigten-kuxf6nige-werden-von-josua-vernichtet}{%
\subparagraph{aa) Die von Jabin zum Bunde vereinigten Könige werden von
Josua
vernichtet}\label{aa-die-von-jabin-zum-bunde-vereinigten-kuxf6nige-werden-von-josua-vernichtet}}

\hypertarget{section-10}{%
\section{11}\label{section-10}}

1Als nun Jabin, der König von Hazor, Kunde davon erhielt, schickte er
Gesandte an Jobab, den König von Madon, sowie an den König von Simron
und an den König von Achsaph 2und an die Könige, die im Norden im
Berglande sowie in der Jordanebene südlich vom See Genezareth und in der
Niederung und im Hügelgelände von Dor gegen Westen wohnten, 3an die
Kanaanäer gegen Osten und gegen Westen und an die Amoriter, Hethiter,
Pherissiter und Jebusiter im Berglande und an die Hewiter am Fuß des
Hermon in der Landschaft Mizpa. 4So zogen sie denn mit allen ihren
Heeren ins Feld, ein Kriegsvolk unzählig wie der Sand am Meeresufer,
auch mit sehr vielen Rossen und Streitwagen. 5Alle diese Könige, die
sich verbündet hatten, rückten ins Feld und schlugen gemeinsam ihr Lager
am Wasser von Merom auf, um den Israeliten eine Schlacht zu liefern.

6Da sagte der HERR zu Josua: »Fürchte dich nicht vor ihnen! Denn morgen
um diese Zeit will ich es fügen, daß sie alle erschlagen vor den
Israeliten daliegen; ihre Rosse aber sollst du lähmen und ihre
Streitwagen im Feuer verbrennen.« 7Als dann Josua sie mit seiner ganzen
Heeresmacht plötzlich am Wasser von Merom überfiel und sich auf sie
warf, 8gab der HERR sie in die Gewalt der Israeliten, so daß diese sie
besiegten und bis Groß-Sidon und bis Misrephoth-Majim und bis in die
Talebene von Mizpe gegen Osten verfolgten und ein Blutbad unter ihnen
anrichteten, bis kein einziger von ihnen mehr übrig war. 9Josua verfuhr
dann mit ihnen, wie der HERR ihm geboten hatte: ihre Rosse lähmte er,
und ihre Kriegswagen verbrannte er im Feuer.

\hypertarget{bb-unterwerfung-der-ganzen-nordhuxe4lfte-kanaans}{%
\subparagraph{bb) Unterwerfung der ganzen Nordhälfte
Kanaans}\label{bb-unterwerfung-der-ganzen-nordhuxe4lfte-kanaans}}

10Dann machte Josua wieder kehrt zu jener Zeit, eroberte Hazor und ließ
den dortigen König mit dem Schwert erschlagen; Hazor war nämlich ehedem
die Hauptstadt aller jener Reiche. 11Sie machten dann die ganze dortige
Einwohnerschaft mit der Schärfe des Schwertes nieder, indem sie den Bann
an ihnen vollstreckten: nichts blieb übrig, was Odem hatte, und Hazor
selbst ließ er in Flammen aufgehen. 12Alle Städte jener Könige samt
allen ihren Königen brachte Josua in seine Gewalt; er schlug sie mit der
Schärfe des Schwertes und vollstreckte den Bann an ihnen, wie Mose, der
Knecht Gottes, geboten hatte. 13Jedoch alle Städte, die auf den dortigen
Anhöhen lagen, verbrannten die Israeliten nicht, mit alleiniger Ausnahme
von Hazor, das Josua in Flammen aufgehen ließ. 14Alle Beute aber aus
diesen Ortschaften und das Vieh nahmen die Israeliten für sich hin, doch
alle Menschen machten sie mit der Schärfe des Schwertes nieder bis zu
völliger Vernichtung: sie verschonten nichts Lebendes. 15Wie der HERR
seinem Knecht Mose geboten hatte, so hatte Mose es dem Josua zur Pflicht
gemacht, und so führte Josua es auch aus: er ließ nicht das Geringste
unbefolgt von allem, was der HERR dem Mose geboten hatte.

\hypertarget{cc-ruxfcckblick-erfuxfcllung-des-guxf6ttlichen-vernichtungswillens-durch-verstockung-der-kanaanuxe4er}{%
\subparagraph{cc) Rückblick; Erfüllung des göttlichen
Vernichtungswillens durch Verstockung der
Kanaanäer}\label{cc-ruxfcckblick-erfuxfcllung-des-guxf6ttlichen-vernichtungswillens-durch-verstockung-der-kanaanuxe4er}}

16So unterwarf Josua dieses ganze Land, nämlich das Bergland wie das
ganze Südland, die ganze Landschaft Gosen, die Niederung, das Jordantal
und das Bergland von Israel mit der dazugehörigen Niederung, 17von dem
kahlen Gebirge an, das sich nach Seir hin erhebt, bis nach Baal-Gad in
der Talebene des Libanons am Fuße des Hermongebirges. Auch alle dortigen
Könige bekam er in seine Gewalt: er schlug sie und ließ sie sterben.
18Lange Zeit führte Josua mit allen diesen Königen Krieg. 19Es gab keine
Stadt, die sich freiwillig den Israeliten ergeben hätte, außer den
Hewitern, die in Gibeon wohnten; alles andere mußten sie mit
Waffengewalt erobern. 20Denn vom HERRN geschah es, daß er ihr Herz zum
Krieg mit den Israeliten verhärtete, damit der Bann an ihnen vollstreckt
werden könnte, ohne daß ihnen Gnade gewährt würde, vielmehr damit sie
ausgerottet würden, wie der HERR dem Mose geboten hatte.

\hypertarget{dd-ausrottung-der-enakiter-d.h.-riesen-aus-dem-lande-abschluuxdf}{%
\subparagraph{dd) Ausrottung der Enakiter (d.h. Riesen) aus dem Lande;
Abschluß}\label{dd-ausrottung-der-enakiter-d.h.-riesen-aus-dem-lande-abschluuxdf}}

21Zu jener Zeit zog Josua hin und vernichtete die Enakiter\textless sup
title=``d.h. Riesen''\textgreater✲ im Gebirge: in Hebron, in Debir, in
Anab sowie im ganzen Berglande Juda und im ganzen Berglande Israel; an
ihnen wie an ihren Ortschaften vollstreckte Josua den Bann. 22Es blieben
keine Enakiter im Lande der Israeliten übrig; nur in Gaza, in Gath und
in Asdod erhielten sich Reste von ihnen.~-- 23So eroberte denn Josua das
ganze Land genau so, wie der HERR dem Mose geboten hatte, und Josua gab
es den Israeliten zum Erbbesitz nach ihren Abteilungen in den einzelnen
Stämmen; das Land bekam dann Ruhe vom Kriege.

\hypertarget{h-verzeichnis-der-von-den-israeliten-besiegten-kuxf6nige}{%
\paragraph{h) Verzeichnis der von den Israeliten besiegten
Könige}\label{h-verzeichnis-der-von-den-israeliten-besiegten-kuxf6nige}}

\hypertarget{aa-die-zwei-von-mose-besiegten-ostjordanischen-kuxf6nige}{%
\subparagraph{aa) Die zwei von Mose besiegten ostjordanischen
Könige}\label{aa-die-zwei-von-mose-besiegten-ostjordanischen-kuxf6nige}}

\hypertarget{section-11}{%
\section{12}\label{section-11}}

1Dies sind die Könige des Landes, welche die Israeliten besiegt und
deren Land sie in Besitz genommen haben: im Ostjordanlande (die Gebiete)
vom Fluß Arnon an bis zum Hermongebirge nebst der ganzen Steppe östlich
vom Jordan: 2Sihon, der König der Amoriter, der in Hesbon seinen Sitz
hatte; er herrschte von Aroer an, das am Ufer des Flusses Arnon liegt,
und zwar von der Mitte des Flußtals an und über die Hälfte von Gilead
bis zum Fluß Jabbok, der Grenze der Ammoniter, 3und über das Jordantal
bis an die Ostseite des Sees Genezareth und bis an die Ostseite des
Meeres der Steppe, des Salzsees, nach Beth-Jesimoth hin und südwärts
(über das Land) am Fuß der Abhänge des Pisgagebirges. 4Sodann das Gebiet
Ogs, des Königs von Basan, der zu den Überresten der Rephaiter✲ gehörte
und in Astaroth und Edrei seinen Sitz hatte; 5er herrschte über das
Hermongebirge, über Salcha und ganz Basan bis an die Grenze der
Gesuriter und Maachathiter und über die Hälfte von Gilead bis an das
Gebiet Sihons, des Königs von Hesbon. 6Mose, der Knecht des HERRN, und
die Israeliten hatten sie besiegt, und Mose, der Knecht des HERRN, hatte
ihr Gebiet den beiden Stämmen Ruben und Gad und dem halben Stamm Manasse
zum Besitz gegeben.

\hypertarget{bb-die-von-josua-im-westjordanlande-besiegten-31-kuxf6nige}{%
\subparagraph{bb) Die von Josua im Westjordanlande besiegten 31
Könige}\label{bb-die-von-josua-im-westjordanlande-besiegten-31-kuxf6nige}}

7Und dies sind die Könige des Landes, welche Josua und die Israeliten im
Westjordanlande besiegt haben, von Baal-Gad im Libanontal an bis zu dem
kahlen Gebirge, das nach Seir hin ansteigt, und deren Land Josua den
israelitischen Stämmen zum Besitz überwies, jedem Stamme einen Teil, 8im
Berglande wie in der Niederung und im Jordantal, an den Bergabhängen wie
in der Wüste und im Südlande, die (Gebiete der) Hethiter, Amoriter,
Kanaanäer, Pherissiter, Hewiter und Jebusiter: 9der König von Jericho
(war) einer; der König von Ai, das seitwärts von Bethel liegt, einer;
10der König von Jerusalem einer; der König von Hebron einer; 11der König
von Jarmuth einer; der König von Lachis einer; 12der König von Eglon
einer; der König von Geser einer; 13der König von Debir einer; der König
von Geder einer; 14der König von Horma einer; der König von Arad einer;
15der König von Libna einer; der König von Adullam einer; 16der König
von Makkeda einer; der König von Bethel einer; 17der König von Thappuah
einer; der König von Hepher einer; 18der König von Aphek einer; der
König von Saron einer; 19der König von Madon einer; der König von Hazor
einer; 20der König von Simron-Meron einer; der König von Achsaph einer;
21der König von Thaanach einer; der König von Megiddo einer; 22der König
von Kedes einer; der König von Jokneam am Karmel einer; 23der König von
Dor in dem Hügelgelände von Dor einer; der König von Gojim\textless sup
title=``oder: der Heiden''\textgreater✲ in Gilgal einer; 24der König von
Thirza einer. Im ganzen waren es einunddreißig Könige.

\hypertarget{ii.-die-verteilung-des-landes-und-einrichtung-von-sechs-freistuxe4dten-und-achtundvierzig-levitenstuxe4dten-kap.-13-21}{%
\subsection{II. Die Verteilung des Landes und Einrichtung von sechs
Freistädten und achtundvierzig Levitenstädten (Kap.
13-21)}\label{ii.-die-verteilung-des-landes-und-einrichtung-von-sechs-freistuxe4dten-und-achtundvierzig-levitenstuxe4dten-kap.-13-21}}

\hypertarget{einleitung}{%
\subsubsection{1. Einleitung}\label{einleitung}}

\hypertarget{a-aufzuxe4hlung-der-bisher-unerobert-gebliebenen-gebiete-der-verteilungsauftrag-gottes}{%
\paragraph{a) Aufzählung der bisher unerobert gebliebenen Gebiete; der
Verteilungsauftrag
Gottes}\label{a-aufzuxe4hlung-der-bisher-unerobert-gebliebenen-gebiete-der-verteilungsauftrag-gottes}}

\hypertarget{section-12}{%
\section{13}\label{section-12}}

1Als nun Josua alt und hochbetagt geworden war, sagte der HERR zu ihm:
»Du bist nun alt und hochbetagt, und von dem Lande sind sehr viele Teile
bisher noch unerobert geblieben. 2Dies ist das noch uneroberte Land:
sämtliche Bezirke der Philister und das ganze Gesuriterland. 3Vom
Sihorbach an, der östlich von Ägypten fließt, bis an das Gebiet von
Ekron im Norden -- es wird zum Gebiet der Kanaanäer gerechnet --: die
fünf Fürsten der Philister, nämlich der von Gaza, der von Asdod, der von
Askalon, der von Gath und der von Ekron, sowie die Awwiter im Süden;
4ferner das ganze Land der Kanaanäer und Meara, das den Sidoniern
gehört, bis nach Aphek, bis an die Grenze der Amoriter; 5sodann das Land
der Gibliter und der ganze Libanon östlichen Teils, von Baal-Gad am Fuß
des Hermongebirges bis dahin, wo man nach Hamath kommt. 6Alle
Gebirgsbewohner vom Libanon an bis Misrephoth-Majim, alle Sidonier,
werde ich selbst vor den Israeliten her vertreiben; verlose es immerhin
an Israel als Erbbesitz, wie ich dir geboten habe. 7So verteile also
jetzt dieses Land als Erbbesitz an die neun Stämme und an den halben
Stamm Manasse.«

\hypertarget{b-allgemeine-angaben-uxfcber-die-verteilung-des-ostjordanlandes-durch-mose-nachtruxe4gliche-zusuxe4tze-v.13-14}{%
\paragraph{b) Allgemeine Angaben über die Verteilung des Ostjordanlandes
durch Mose; nachträgliche Zusätze
(V.13-14)}\label{b-allgemeine-angaben-uxfcber-die-verteilung-des-ostjordanlandes-durch-mose-nachtruxe4gliche-zusuxe4tze-v.13-14}}

8Mit dem andern halben Stamm Manasse nämlich hatten die Stämme Ruben und
Gad ihren Erbbesitz bereits empfangen, den ihnen Mose im Ostjordanlande
angewiesen hatte, wie ihn Mose, der Knecht des HERRN, ihnen angewiesen
hatte: 9von Aroer an, das am Ufer des Flusses Arnon liegt, und überhaupt
von den Städten an, die mitten im Flußtal liegen, dazu die ganze
Hochebene von Medeba bis Dibon, 10ferner sämtliche Städte des
Amoriterkönigs Sihon, der in Hesbon geherrscht hatte, bis an das Gebiet
der Ammoniter; 11sodann Gilead und das Gebiet der Gesuriter und
Maachathiter sowie das ganze Hermongebirge und ganz Basan bis nach
Salcha, 12das ganze Reich Ogs in Basan, der in Astharoth und Edrei
geherrscht hatte -- er war vom Überrest der Rephaiter noch
übriggeblieben --: diese hatte Mose besiegt und aus ihrem Besitz
vertrieben. 13Die Israeliten dagegen haben die Gesuriter und
Maachathiter nicht aus ihrem Besitz vertrieben, sondern diese beiden
Völkerschaften sind mitten unter den Israeliten bis auf den heutigen Tag
wohnen geblieben.~-- 14Nur dem Stamme Levi hatte (Mose) keinen Erbbesitz
verliehen: die Feueropfer des HERRN, des Gottes der Israeliten, die sind
dessen Erbbesitz, wie er ihm zugesagt hat.

\hypertarget{genauere-angaben-uxfcber-die-von-mose-verteilten-gebiete-der-stuxe4mme-ruben-gad-und-halb-manasse}{%
\subsubsection{2. Genauere Angaben über die von Mose verteilten Gebiete
der Stämme Ruben, Gad und halb
Manasse}\label{genauere-angaben-uxfcber-die-von-mose-verteilten-gebiete-der-stuxe4mme-ruben-gad-und-halb-manasse}}

15Mose hatte aber dem Stamme der Rubeniten Landbesitz nach ihren
Geschlechtern angewiesen, 16so daß ihnen folgendes Gebiet zuteil wurde:
das Gebiet von Aroer an, das am Ufer des Flusses Arnon liegt, und
überhaupt von den Städten an, die mitten im Flußtal liegen, dazu die
ganze Hochebene bis Medeba; 17Hesbon mit allen zugehörigen Ortschaften,
die auf der Hochebene liegen, nämlich Dibon, Bamoth-Baal,
Beth-Baal-Meon, 18Jahza, Kedemoth, Mephaath, 19Kirjathaim, Sibma,
Zereth-Sahar auf dem Berge in der Talebene, 20Beth-Peor, die Abhänge des
Pisga, Beth-Jesimoth 21und alle übrigen Ortschaften der Hochebene;
ferner das ganze Reich des Amoriterkönigs Sihon, der in Hesbon
geherrscht hatte -- Mose hatte ihn besiegt, ihn mitsamt den Fürsten der
Midianiter, nämlich Ewi, Rekem, Zur, Hur und Reba, den Häuptlingen
Sihons, die im Lande gewohnt hatten. 22Auch den Wahrsager Bileam, den
Sohn Beors, hatten die Israeliten mit dem Schwert getötet außer den
anderen von ihnen Erschlagenen. 23Die (westliche) Grenze des Stammes
Ruben aber bildete der Jordan und sein Uferland. Dies war der Erbbesitz
der Rubeniten nach ihren Geschlechtern: die genannten Städte mit den
zugehörigen Gehöften\textless sup title=``oder: Dörfern''\textgreater✲.

24Sodann hatte Mose dem Stamme Gad, den Gaditen, Landbesitz nach ihren
Geschlechtern angewiesen, 25so daß ihnen folgendes Gebiet zuteil wurde:
Jaser und sämtliche Ortschaften in Gilead und das halbe Ammoniterland
bis Aroer, das östlich von Rabba liegt; 26ferner das Land von Hesbon bis
Ramath-Mizpe und Betonim und von Mahanaim bis an das Gebiet von Lidebir;
27sodann in der Talebene: Beth-haram, Beth-Nimra, Sukkoth und Zaphon,
der Rest vom Reiche Sihons, des Königs von Hesbon, der Jordan mit seinem
Uferland bis an das Ende des Sees Genezareth auf der Ostseite des
Jordans. 28Das war der Erbbesitz der Gaditen nach ihren Geschlechtern,
die genannten Städte mit den zugehörigen Dörfern.

29Ferner hatte Mose dem halben Stamm Manasse Landbesitz angewiesen, so
daß dem halben Stamme der Manassiten nach ihren Geschlechtern folgendes
Gebiet zuteil wurde: 30das Land von Mahanaim an, das ganze Basan, das
ganze Reich Ogs, des Königs von Basan, und sämtliche Zeltdörfer Jairs,
die in Basan liegen, sechzig Ortschaften; 31ferner die Hälfte von Gilead
sowie Astaroth und Edrei, die Hauptstädte des Reiches Ogs in Basan. Dies
alles wurde den Nachkommen Machirs, des Sohnes Manasses, und zwar der
Hälfte der Nachkommen Machirs, nach ihren Geschlechtern zuteil.

\hypertarget{c-abschluuxdf-v.32-wiederholung-von-v.14}{%
\paragraph{c) Abschluß (V.32); Wiederholung von
V.14}\label{c-abschluuxdf-v.32-wiederholung-von-v.14}}

32Diese Gebiete sind es, die Mose in den Steppen der Moabiter jenseits
des Jordans, östlich von Jericho, ausgeteilt hatte. 33Dem Stamme Levi
aber hatte Mose keinen Erbbesitz gegeben: der HERR, der Gott Israels,
der sollte ihr Erbbesitz sein, wie er ihnen zugesagt hatte.

\hypertarget{verteilung-des-westjordanlandes-kanaans-kap.-14-21}{%
\subsubsection{3. Verteilung des Westjordanlandes (=~Kanaans) (Kap.
14-21)}\label{verteilung-des-westjordanlandes-kanaans-kap.-14-21}}

\hypertarget{a-uxfcberschrift-und-einleitende-bemerkungen-der-erbbesitz-kalebs-in-hebron}{%
\paragraph{a) Überschrift und einleitende Bemerkungen; der Erbbesitz
Kalebs in
Hebron}\label{a-uxfcberschrift-und-einleitende-bemerkungen-der-erbbesitz-kalebs-in-hebron}}

\hypertarget{section-13}{%
\section{14}\label{section-13}}

1Und dies sind die Gebiete, welche die Israeliten im Lande Kanaan als
Erbbesitz empfangen haben, die ihnen der Priester Eleasar und Josua, der
Sohn Nuns, und die Häupter der israelitischen Stämme 2als ihren
Erbbesitz durch das Los zugeteilt haben, wie der HERR es durch Mose
bezüglich der neuneinhalb Stämme verordnet hatte; 3denn den zweieinhalb
Stämmen hatte Mose ihren Erbbesitz jenseits des Jordans angewiesen, den
Leviten aber keinen Erbbesitz unter den Israeliten verliehen. 4Die
Nachkommen Josephs bildeten nämlich zwei Stämme, Manasse und Ephraim;
den Leviten aber gab man keinen Anteil am Landbesitz, sondern nur
einzelne Städte zum Bewohnen nebst den zugehörigen Weidetriften für ihr
Vieh und ihre Habe. 5Wie der HERR dem Mose geboten hatte, so verfuhren
die Israeliten bei der Verteilung des Landes.

\hypertarget{kaleb-erhuxe4lt-auf-seine-bitte-den-bezirk-von-hebron-als-erbteil}{%
\paragraph{Kaleb erhält auf seine Bitte den Bezirk von Hebron als
Erbteil}\label{kaleb-erhuxe4lt-auf-seine-bitte-den-bezirk-von-hebron-als-erbteil}}

6Da traten die Judäer vor Josua in Gilgal, und Kaleb, der Sohn
Jephunnes, der Kenissite, sagte zu Josua: »Du weißt selbst, was der HERR
zu Mose, dem Manne Gottes, in bezug auf mich und auf dich in
Kades-Barnea gesagt hat. 7Vierzig Jahre war ich alt, als Mose, der
Knecht des HERRN, mich von Kades-Barnea zur Auskundschaftung des Landes
aussandte, und ich erstattete ihm Bericht, wie ich in meinem Herzen
wirklich dachte. 8Meine Volksgenossen aber, die mit mir hinaufgezogen
waren, machten dem Volk das Herz verzagt, während ich dem HERRN, meinem
Gott, vollen Gehorsam bewies. 9Da sprach Mose an jenem Tage folgenden
Schwur aus: ›Fürwahr das Land, das dein Fuß betreten hat, soll dir und
deinen Nachkommen auf ewige Zeiten als Erbbesitz zuteil werden, weil du
dem HERRN, meinem Gott, vollkommen gehorsam gewesen bist!‹ 10Und nun hat
mich der HERR, wie du siehst, seiner Verheißung gemäß noch
fünfundvierzig Jahre am Leben erhalten seit der Zeit, als der HERR jenes
Wort zu Mose gesprochen hat und während die Israeliten in der Wüste
umhergezogen sind; und so bin ich jetzt fünfundachtzig Jahre alt. 11Ich
bin heute noch so rüstig wie damals, als Mose mich aussandte; wie meine
Kraft damals war, so ist sie jetzt noch zum Kriegsdienst, zum Ausmarsch
und zur Heimkehr\textless sup title=``oder: zu jeder Arbeit und
Leistung''\textgreater✲ ausreichend. 12So überweise mir nun dieses
Bergland, von dem der HERR damals gesprochen hat; du selbst hast ja
damals gehört, daß es dort noch Enakiter✲ und große, feste Städte gibt;
vielleicht ist der HERR mit mir, so daß ich sie nach der Verheißung des
HERRN aus ihrem Besitz vertreiben kann.« 13Da segnete Josua den Kaleb,
den Sohn Jephunnes, und verlieh ihm Hebron als Erbbesitz. 14Auf diese
Weise ist Hebron an Kaleb, den Sohn Jephunnes, den Kenissiten, als
Erbbesitz bis auf den heutigen Tag gekommen, weil er nämlich dem HERRN,
dem Gott Israels, vollen Gehorsam bewiesen hatte. 15Hebron hieß aber
ehemals Kirjath-Arba\textless sup title=``d.h. die Stadt
Arbas''\textgreater✲; Arba war der größte Mann unter den Enakitern
gewesen. -- Und das Land hatte Ruhe vom Kriege.

\hypertarget{b-das-gebiet-des-stammes-juda}{%
\paragraph{b) Das Gebiet des Stammes
Juda}\label{b-das-gebiet-des-stammes-juda}}

\hypertarget{section-14}{%
\section{15}\label{section-14}}

1Für die einzelnen Geschlechter des Stammes Juda aber fiel das Los
südwärts nach dem Gebiet der Edomiter, nach der Wüste Zin hin im
äußersten Süden des Landes; 2und zwar geht ihre Südgrenze vom Ende des
Salzmeeres, von seiner Südspitze aus, 3läuft dann weiter gegen die
Südseite der Skorpionenhöhe, dann nach Zin hinüber, steigt aufwärts
südlich von Kades-Barnea, geht dann weiter nach Hezron, zieht sich
aufwärts nach Addar, wendet sich herum nach Karka, 4geht dann nach Azmon
hinüber und setzt sich fort bis an den Bach Ägyptens, bis sie nach dem
Meere hin ihr Ende erreicht: dies soll eure Südgrenze sein.~-- 5Die
Ostgrenze aber bildet das Salzmeer bis zur Jordanmündung -- und die
Nordgrenze geht vom Nordende des Salzmeeres, von der Jordanmündung aus;
6dann zieht sich die Grenze hinauf nach Beth-Holga und läuft weiter bis
nördlich von Beth-Araba; dann zieht die Grenze sich aufwärts zum Felsen
Bohans, des Sohnes Rubens, 7geht dann vom Tal Achor aufwärts nach Debir,
läuft mit veränderter Richtung nordwärts nach Gilgal, das der Anhöhe
Adummim gegenüber liegt, die sich südlich von dem Bache befindet; dann
zieht sich die Grenze hinüber nach dem Wasser En-Semes\textless sup
title=``d.h. Sonnenquell''\textgreater✲ und läuft weiter nach der Quelle
Rogel\textless sup title=``d.h. Walkerquelle''\textgreater✲; 8sodann
geht sie im Tale Ben-Hinnom hinauf südlich vom Bergrücken der Jebusiter,
das ist Jerusalem; weiter zieht die Grenze sich hinauf zu dem Gipfel des
Berges, der westlich vor dem Tale Hinnom am Nordende der Talebene
Rephaim liegt; 9dann biegt die Grenze vom Gipfel des Berges um nach der
Quelle des Wassers Nephthoah, läuft weiter nach den Ortschaften des
Ephrongebirges hin und zieht mit veränderter Richtung nach Baala, das
ist Kirjath-Jearim; 10von Baala wendet sie sich dann westwärts nach dem
Gebirge Seir, geht hierauf hinüber nach der Nordseite des Berges Jearim,
das ist Kesalon, senkt sich hinab nach Beth-Semes und weiter nach
Thimna; 11dann läuft die Grenze weiter an den Nordabhang des Berges
Ekron und mit veränderter Richtung nach Sikkeron, geht dann hinüber nach
dem Berge von Baala, läuft weiter bis Jabneel und erreicht schließlich
ihr Ende am Meer.~-- 12Die Westgrenze aber bildet das große Meer und
sein Küstenland. Das ist die Grenze des Stammes der Judäer ringsum für
ihre Geschlechter.

\hypertarget{kalebs-besitz-und-erfolgreiche-tuxe4tigkeit}{%
\paragraph{Kalebs Besitz und erfolgreiche
Tätigkeit}\label{kalebs-besitz-und-erfolgreiche-tuxe4tigkeit}}

13Kaleb aber, dem Sohne Jephunnes, gab Josua einen Landbesitz mitten im
Stamme Juda nach dem Befehl des HERRN an Josua, nämlich die Stadt Arbas,
des Stammvaters der Enakiter\textless sup title=``vgl.
14,12''\textgreater✲, das ist Hebron. 14Kaleb vertrieb dann von dort die
drei Enakssöhne Sesai, Ahiman und Thalmai, die Abkömmlinge Enaks, 15und
zog von dort weiter gegen die Bewohner von Debir, das ehemals
Kirjath-Sepher geheißen hatte. 16Als nun Kaleb bekanntmachte: »Wer
Kirjath-Sepher bezwingt und erobert, dem gebe ich meine Tochter Achsa
zur Frau« 17und Othniel, der Sohn des Kenas, ein Bruder Kalebs, die
Stadt eroberte, gab er ihm seine Tochter Achsa zur Frau. 18Als sie ihm
nun zugeführt wurde, überredete sie ihn, ein Stück Ackerland von ihrem
Vater erbitten zu dürfen, und als sie dann vom Esel herabsprang und
Kaleb sie fragte: »Was wünschest du?«, 19antwortete sie: »Gib mir doch
ein Abschiedsgeschenk! Weil du mich in das Südland verheiratet hast, so
gib mir auch Wasserquellen!« Da gab er ihr die oberen und die unteren
Brunnen\textless sup title=``oder: Quellen''\textgreater✲.

\hypertarget{die-stuxe4dte-von-juda}{%
\paragraph{Die Städte von Juda}\label{die-stuxe4dte-von-juda}}

20Folgendes ist der Erbbesitz der einzelnen Geschlechter des Stammes
Juda. 21Es liegen nämlich im südlichen Teil des Stammes Juda nach dem
Gebiet der Edomiter hin die Ortschaften: Kabzeel, Eder, Jagur, 22Kina,
Dimona, Adada, 23Kedes, Hazor und Jithnan; 24Siph, Telem, Bealoth,
25Hazor-Hadattha und Kerioth-Hezron, das ist Hazor; 26Amam, Sema,
Molada, 27Hazar-Gadda, Hesmon, Beth-Pelet, 28Hazar-Sual, Beerseba,
Bisjothja; 29Baala, Ijjim, Ezem, 30Eltholad, Kesil, Horma, 31Ziklag,
Madmanna, Sansanna, 32Lebaoth, Silhim, Ain und Rimmon: im ganzen 29
Ortschaften nebst den zugehörigen Dörfern\textless sup title=``oder:
Gehöften''\textgreater✲.

33In der Niederung: Esthaol, Zora, Asna, 34Sanoah und En-Gannim,
Thappuah und Enam, 35Jarmuth und Adullam, Socho, Aseka, 36Saaraim,
Adithaim, Gedera und Gederothaim: 14 Ortschaften nebst den zugehörigen
Dörfern. 37Zenan, Hadasa, Migdal-Gad, 38Dilgan, Mizpe und Joktheel;
39Lachis, Bozkath, Eglon, 40Kabbon, Lahmas, Kithlis 41und Gederoth,
Beth-Dagon, Naama und Makkeda: 16 Ortschaften mit den zugehörigen
Dörfern. 42Libna, Ether, Asan, 43Jiphthah, Asna, Nezib, 44Kegila, Achsib
und Maresa: 9 Ortschaften mit den zugehörigen Dörfern. 45Ekron mit den
zugehörigen Ortschaften und Dörfern. 46Von Ekron an, und zwar nach dem
Meere zu, alles, was seitlich von Asdod und den zugehörigen Dörfern
liegt: 47Asdod mit den zugehörigen Ortschaften und Dörfern; Gaza mit den
zugehörigen Ortschaften und Dörfern bis an den Bach Ägyptens; die
Westgrenze aber bildet das große Meer nebst seinem Küstenlande.

48Ferner im Berglande: Samir, Jatthir, Socho, 49Danna, Kirjath-Sanna,
das ist Debir, 50Anab, Esthemo, Anim, 51Gosen, Holon und Gilo: 11
Ortschaften mit den zugehörigen Dörfern. 52Arab, Duma, Esgan, 53Janum,
Beth-Thappuah, Apheka, 54Humta, Kirjath-Arba, das ist Hebron, und Zior:
9 Ortschaften mit den zugehörigen Dörfern. 55Maon, Karmel, Siph, Juta,
56Jesreel, Jokdeam, Sanoah, 57Kain, Gibea und Thimna: 10 Ortschaften mit
den zugehörigen Dörfern. 58Halhul, Beth-Zur, Gedor, 59Maarath,
Beth-Anoth und Elthekon: 6 Ortschaften mit den zugehörigen Dörfern.
60Kirjath-Baal, das ist Kirjath-Jearim, und Rabba: 2 Ortschaften mit den
zugehörigen Dörfern.

61In der Steppe: Beth-Araba, Middin, Sechacha, 62Nibsan und die
Salzstadt und Engedi: 6 Ortschaften mit den zugehörigen Dörfern.

63Was aber die Jebusiter, die Bewohner von Jerusalem, anbetrifft, so
vermochte der Stamm Juda sie nicht zu vertreiben; daher sind die
Jebusiter in Jerusalem neben den Judäern wohnen geblieben bis auf den
heutigen Tag.

\hypertarget{c-das-gebiet-der-nachkommen-josephs}{%
\paragraph{c) Das Gebiet der Nachkommen
Josephs}\label{c-das-gebiet-der-nachkommen-josephs}}

\hypertarget{aa-die-suxfcdgrenze-des-gebiets}{%
\subparagraph{aa) Die Südgrenze des
Gebiets}\label{aa-die-suxfcdgrenze-des-gebiets}}

\hypertarget{section-15}{%
\section{16}\label{section-15}}

1Sodann fiel den Nachkommen Josephs durch das Los ihr Anteil zu vom
Jordan bei Jericho an und begriff im Osten das Uferland von Jericho,
dann die Wüste\textless sup title=``oder: Steppe''\textgreater✲, die
sich von Jericho an aufsteigend im Gebirge nach Bethel erstreckt. 2Die
Grenze läuft dann von Bethel weiter nach Lus und zieht sich hinüber nach
dem Gebiet der Arkiter, nach Atharoth, 3läuft hierauf westwärts hinab
zum Gebiet der Japhletiter bis zum Gebiet von Unter-Beth-Horon und bis
Geser; ihr Endpunkt liegt dann am Meer. 4Dies ist der Erbbesitz, den die
Nachkommen Josephs, die Stämme Manasse und Ephraim, erhielten.

\hypertarget{bb-gebiet-des-stammes-ephraim}{%
\subparagraph{bb) Gebiet des Stammes
Ephraim}\label{bb-gebiet-des-stammes-ephraim}}

5Dies aber ist das Gebiet der Geschlechter des Stammes Ephraim: Die
Grenze ihres Erbteils geht nämlich im Osten von Ateroth-Addar bis
Ober-Beth-Horon; 6von da läuft die Grenze dem Meere zu. -- Im Norden
bildet Michmethath die Grenze, und zwar wendet sich die Grenze ostwärts
nach Thaanath-Silo und geht daran östlich vorbei nach Janoha. 7Von
Janoha senkt sie sich nach Ataroth und Naarath hinab, berührt dann
Jericho und endet am Jordan. 8Von Thappuah aus geht die Grenze westwärts
an den Bach Kana und erreicht ihr Ende am Meer. Dies ist der Erbbesitz
des Stammes Ephraim, auf die einzelnen Geschlechter verteilt. 9Dazu
kommen noch die Städte, die für die Ephraimiten mitten im Erbteil der
Manassiten abgesondert waren, sämtliche Städte mit den zugehörigen
Dörfern. 10Sie vertrieben aber die Kanaanäer nicht, die in Geser
wohnten; daher sind die Kanaanäer mitten unter den Ephraimiten wohnen
geblieben bis auf den heutigen Tag und sind nur fronpflichtig geworden.

\hypertarget{cc-gebiet-des-stammes-manasse}{%
\subparagraph{cc) Gebiet des Stammes
Manasse}\label{cc-gebiet-des-stammes-manasse}}

\hypertarget{section-16}{%
\section{17}\label{section-16}}

1Sodann fiel dem Stamm Manasse -- dieser war nämlich der erstgeborene
Sohn Josephs -- sein Losanteil zu. Machir, dem erstgeborenen Sohne
Manasses, dem Vater Gileads, wurde Gilead und Basan zuteil, denn er war
ein Kriegsmann. 2Sodann erhielten die übrigen Nachkommen Manasses,
Geschlecht für Geschlecht, ihren Anteil, nämlich die Nachkommen
Abiesers, die Nachkommen Heleks, die Nachkommen Asriels, die Nachkommen
Sichems, die Nachkommen Hephers und die Nachkommen Semidas; dies waren
die männlichen Nachkommen Manasses, des Sohnes Josephs, Geschlecht für
Geschlecht.

\hypertarget{dd-die-tuxf6chter-zelophhads-werden-zu-erbtuxf6chtern-gemacht}{%
\subparagraph{dd) Die Töchter Zelophhads werden zu Erbtöchtern
gemacht}\label{dd-die-tuxf6chter-zelophhads-werden-zu-erbtuxf6chtern-gemacht}}

3Aber Zelophhad, der Sohn Hephers, des Sohnes Gileads, des Sohnes
Machirs, des Sohnes Manasses, hatte keine Söhne, sondern nur Töchter;
deren Namen waren Mahla, Noa, Hogla, Milka und Thirza. 4Diese traten nun
vor den Priester Eleasar und vor Josua, den Sohn Nuns, und vor die
Fürsten und sagten: »Der HERR hat Mose geboten, uns ein Erbteil unter
unsern Stammesgenossen zu geben.« Da überwies er\textless sup
title=``oder: man''\textgreater✲ ihnen nach dem Befehl des HERRN ein
Erbteil unter den Stammesgenossen ihres Vaters; 5und so fielen dem Stamm
Manasse zehn Losanteile zu, abgesehen von der Landschaft Gilead und
Basan, die jenseits des Jordans liegen; 6denn die weiblichen Nachkommen
Manasses erhielten ein Erbteil gerade wie seine männlichen Nachkommen;
die Landschaft Gilead aber kam an die übrigen Nachkommen Manasses.

\hypertarget{ee-grenzen-und-stuxe4dte-des-stammes-manasse}{%
\subparagraph{ee) Grenzen und Städte des Stammes
Manasse}\label{ee-grenzen-und-stuxe4dte-des-stammes-manasse}}

7Es geht aber die Grenze des Stammes Manasse von Asser nach Michmethath,
das östlich von Sichem liegt; dann geht die Grenze südwärts zu den
Bewohnern von En-Thappuah hin. 8Die Landschaft Thappuah gehört zum Stamm
Manasse; die Stadt Thappuah selbst aber, an der Grenze von Manasse,
gehört zum Stamme Ephraim. 9Dann geht die Grenze hinab zum Bache Kana,
südlich von dem Bache -- die dort mitten unter den Städten Manasses
liegenden Ortschaften gehören zum Stamm Ephraim --; alsdann aber läuft
die Grenze nördlich vom Bache und endet am Meer; 10die Südseite ist
ephraimitisch, die Nordseite manassitisch, und das Meer bildet hier die
Grenze; nordwärts aber stoßen sie an Asser, ostwärts an Issaschar.
11Außerdem erhielt der Stamm Manasse in Issaschar und in Asser folgende
Städte: Beth-Sean mit den zugehörigen Ortschaften, ferner Jibleam mit
den zugehörigen Ortschaften sowie die Bewohner von Dor und von En-Dor
mit den zugehörigen Ortschaften, die Bewohner von Thaanach mit den
zugehörigen Ortschaften und die Bewohner von Megiddo mit den zugehörigen
Ortschaften: das Dreihügelgebiet.~-- 12Der Stamm Manasse vermochte
jedoch nicht diese Städte zu erobern; daher gelang es den Kanaanäern, in
dieser Gegend wohnen zu bleiben. 13Als später aber die Israeliten
erstarkten, machten sie die Kanaanäer fronpflichtig, ohne sie jedoch
ganz vertreiben zu können.

\hypertarget{ff-josua-fordert-die-sich-beklagenden-josephstuxe4mme-auf-sich-den-wald-auszuroden}{%
\subparagraph{ff) Josua fordert die sich beklagenden Josephstämme auf,
sich den Wald
auszuroden}\label{ff-josua-fordert-die-sich-beklagenden-josephstuxe4mme-auf-sich-den-wald-auszuroden}}

14Da wandten sich die beiden Josephstämme an Josua und trugen ihm
folgendes vor: »Warum hast du mir nur ein Los und einen einzigen Anteil
als Erbbesitz gegeben, obgleich ich doch ein zahlreiches Volk bin, da
der HERR mich bisher gesegnet hat?« 15Josua antwortete ihnen: »Wenn du
ein zahlreiches Volk bist, so ziehe doch in den Wald hinauf und schaffe
dir dort im Lande der Pherissiter und der Rephaiter durch Ausroden des
Waldes Raum zum Wohnen, wenn\textless sup title=``oder:
weil''\textgreater✲ dir das Bergland Ephraim zu enge ist.« 16Da
entgegneten die Nachkommen Josephs: »Das Bergland reicht für uns nicht
aus; alle Kanaanäer aber, die unten in der Ebene wohnen, haben eiserne
Streitwagen, sowohl die, welche in Beth-Sean und den zugehörigen
Ortschaften, als auch die, welche in der Ebene Jesreel wohnen.« 17Da
antwortete Josua den beiden Josephstämmen Ephraim und Manasse: »Du bist
ein sehr zahlreiches und sehr starkes Volk; so sollst du nicht nur ein
Los erhalten; 18denn ein Bergland wird dir zuteil werden. Da es Wald
ist, mußt du ihn ausroden: dann werden auch die Ausläufer\textless sup
title=``d.h. die anliegenden Gegenden''\textgreater✲ dir zuteil werden;
denn du wirst die Kanaanäer vertreiben, wenn sie auch eiserne Wagen
haben und wenn sie auch stark sind.«

\hypertarget{d-die-gebiete-der-uxfcbrigen-stuxe4mme}{%
\paragraph{d) Die Gebiete der übrigen
Stämme}\label{d-die-gebiete-der-uxfcbrigen-stuxe4mme}}

\hypertarget{aa-das-offenbarungszelt-in-silo-aufgeschlagen-schriftliche-aufnahme-und-verteilung-des-noch-unbesetzten-landes}{%
\subparagraph{aa) Das Offenbarungszelt in Silo aufgeschlagen;
schriftliche Aufnahme und Verteilung des noch unbesetzten
Landes}\label{aa-das-offenbarungszelt-in-silo-aufgeschlagen-schriftliche-aufnahme-und-verteilung-des-noch-unbesetzten-landes}}

\hypertarget{section-17}{%
\section{18}\label{section-17}}

1Hierauf versammelte sich die ganze Volksgemeinde der Israeliten in Silo
und schlug dort das Offenbarungszelt auf; denn das Land lag unterworfen
vor ihnen da. 2Nun waren aber unter den Israeliten noch sieben Stämme
übrig, deren Erbbesitz man noch nicht ausgeteilt hatte. 3Da sagte Josua
zu den Israeliten: »Wie lange wollt ihr noch lässig bleiben, statt
hinzugehen, um das Land in Besitz zu nehmen, das der HERR, der Gott
eurer Väter, euch gegeben hat? 4Bestimmt doch drei Männer aus jedem
Stamm, so will ich sie aussenden, damit sie sich daranmachen, das Land
zu durchwandern und es schriftlich aufzunehmen mit Rücksicht auf den für
sie erforderlichen Erbbesitz. Wenn sie dann zu mir zurückgekommen sind,
5mögen sie es in sieben Teile unter sich verteilen. Juda soll sein
Gebiet im Süden behalten und das Haus Josephs auf seinem Gebiet im
Norden bleiben; 6ihr aber fertigt schriftlich eine Übersicht des Landes
bei Zerlegung in sieben Teile an und bringt die Aufzeichnung mir
hierher, so will ich das Los für euch werfen hier vor dem HERRN, unserm
Gott. 7Denn die Leviten erhalten keinen Landbesitz unter euch, weil das
Priestertum des HERRN ihr Erbteil ist; Gad aber und Ruben und der halbe
Stamm Manasse haben ihren Erbbesitz bereits im Ostjordanlande empfangen,
den ihnen Mose, der Knecht Gottes, angewiesen hat.«

8Da machten sich die Männer auf den Weg, und Josua gab ihnen, als sie
zur schriftlichen Aufnahme des Landes aufbrachen, die Weisung: »Geht
hin, durchwandert das Land und nehmt es schriftlich auf; dann kommt
wieder zu mir, so will ich hier zu Silo das Los für euch vor dem HERRN
werfen.« 9So machten sich denn die Männer auf den Weg, zogen durch das
Land und trugen es Stadt für Stadt unter Zerlegung in sieben Teile in
ein Buch ein; dann kehrten sie zu Josua ins Lager nach Silo zurück. 10Da
warf Josua das Los für sie zu Silo vor dem HERRN, und Josua verteilte
dort das Land unter die Israeliten, wie es ihren Anteilen\textless sup
title=``oder: Abteilungen''\textgreater✲ entsprach.

\hypertarget{bb-das-gebiet-des-stammes-benjamin}{%
\subparagraph{bb) Das Gebiet des Stammes
Benjamin}\label{bb-das-gebiet-des-stammes-benjamin}}

11So kam denn das Los für die Geschlechter des Stammes Benjamin heraus,
und zwar kam das Gebiet, das ihnen durchs Los zufiel, zwischen die
Stämme Juda und Joseph zu liegen. 12Ihre Nordgrenze begann aber am
Jordan, zieht sich dann aufwärts nach dem Bergzuge nördlich von Jericho
und von da auf das Gebirge nach Westen zu und endet nach der Wüste von
Beth-Awen hin. 13Von dort geht die Grenze dann nach Lus hinüber, nach
dem Höhenzuge südlich von Lus, das ist Bethel; dann senkt sich die
Grenze hinab nach Ateroth-Addar über den Berg, der südlich von
Unter-Beth-Horon liegt. 14Sodann zieht die Grenze in veränderter
Richtung auf ihrer Westseite nach Süden von dem Berg an, der südlich
Beth-Horon gegenüber liegt, und endet bei Kirjath-Baal, das ist die
judäische Stadt Kirjath-Jearim. Dies ist die Westseite. 15Die Südseite
aber beginnt bei der Stadtgrenze von Kirjath-Jearim und setzt sich dann
westwärts fort nach der Quelle des Wassers von Nephthoah; 16dann läuft
die Grenze hinab bis an das Ende des Berges, der östlich vom Tal
Ben-Hinnom und nördlich von der Talebene Rephaim liegt, zieht dann in
das Hinnomtal hinab südlich vom Bergrücken der Jebusiter und weiter
hinab zur Quelle Rogel\textless sup title=``d.h.
Walkerquelle''\textgreater✲; 17alsdann läuft sie mit veränderter
Richtung nordwärts, und zwar nach En-Semes und weiter nach Geliloth✲
hin, das der Anhöhe Adummim gegenüber liegt, senkt sich dann hinab zum
Felsen Bohans, des Sohnes Rubens, 18und geht hinüber zu dem Höhenzug,
der nordwärts der Araba gegenüber liegt; hierauf senkt die Grenze sich
in die Araba hinab, 19läuft dann hinüber bis nördlich vom Bergrücken von
Beth-Hogla und erreicht ihr Ende an der Nordspitze des Salzmeeres, am
südlichen Ende des Jordans. Dies ist die Südgrenze. 20Der Jordan aber
bildet die Grenze auf der Ostseite. Dies ist der Erbbesitz der
Geschlechter des Stammes Benjamin nach seinen Grenzen ringsum.

21Die Städte der Geschlechter des Stammes Benjamin aber sind: Jericho,
Beth-Hogla, Emek-Keziz, 22Beth-Araba, Zemaraim, Bethel, 23Awwim, Para,
Ophra, 24Kephar-Ammoni, Ophni und Geba: 12 Städte mit den zugehörigen
Dörfern; 25Gibeon, Rama, Beeroth, 26Mizpe, Kephira, Moza, 27Rekem,
Jirpeel, Tharala, 28Zela, Eleph und die Jebusiterstadt, das ist
Jerusalem, Gibeath, Kirjath: 14 Städte mit den zugehörigen Dörfern. Das
ist der Erbbesitz der Geschlechter des Stammes Benjamin.

\hypertarget{cc-das-gebiet-des-stammes-simeon}{%
\subparagraph{cc) Das Gebiet des Stammes
Simeon}\label{cc-das-gebiet-des-stammes-simeon}}

\hypertarget{section-18}{%
\section{19}\label{section-18}}

1Dann kam das zweite Los heraus für Simeon, für die Geschlechter des
Stammes der Simeoniten; und zwar lag ihr Erbteil mitten im Erbbesitz der
Judäer. 2Es wurde ihnen aber als ihr Erbbesitz zuteil: Beerseba, Seba,
Molada, 3Hazar-Sual, Bala, Ezem, 4Eltholad, Bethul, Horma, 5Ziklag,
Beth-Markaboth, Hazar-Susa, 6Beth-Lebaoth und Saruhen: 13 Städte mit den
zugehörigen Dörfern. 7Ferner Ain, Rimmon, Ether und Asan: 4 Städte mit
den zugehörigen Dörfern; 8außerdem alle Dörfer, die rings um diese
Städte liegen bis nach Baalath-Beer, dem Rama des Südlandes. Dies ist
der Erbbesitz der Geschlechter des Stammes Simeon. 9Von dem Anteil der
Judäer war der Erbbesitz der Simeoniten genommen; denn der Anteil der
Judäer war für diese zu groß; daher erhielten die Simeoniten ihren
Erbbesitz mitten in deren Besitztum.

\hypertarget{dd-das-gebiet-des-stammes-sebulon}{%
\subparagraph{dd) Das Gebiet des Stammes
Sebulon}\label{dd-das-gebiet-des-stammes-sebulon}}

10Hierauf kam das dritte Los heraus für die Geschlechter des Stammes
Sebulon; und die Grenze ihres Erbbesitzes reichte bis Sarid. 11Ihre
Grenze geht aber westwärts hinauf, und zwar nach Marala hin, berührt
Dabbeseth und stößt an den Bach, der östlich von Jokneam fließt. 12Aber
auf der östlichen Seite von Sarid, gegen Sonnenaufgang, wendet sie sich
nach dem Gebiet von Kisloth-Thabor, geht weiter nach Daberath, zieht
sich aufwärts nach Japhia, 13läuft von da ostwärts, gegen Sonnenaufgang,
nach Gath-Hepher, nach Eth-Kazin hinüber, läuft aus bei Rimmon und
erstreckt sich nach Nea hin; 14dann zieht sich die Grenze mit
veränderter Richtung um dasselbe herum nördlich von Hannathon und
erreicht ihr Ende im Tal von Jiphthah-El. 15~\ldots{} und Kattath,
Nahalal, Simron, Jidala und Bethlehem: 12 Städte mit den zugehörigen
Dörfern. 16Dies war der Erbbesitz der Geschlechter des Stammes Sebulon:
die genannten Städte mit den zugehörigen Dörfern.

\hypertarget{ee-das-gebiet-des-stammes-issaschar}{%
\subparagraph{ee) Das Gebiet des Stammes
Issaschar}\label{ee-das-gebiet-des-stammes-issaschar}}

17Für Issaschar kam das vierte Los heraus, für die Geschlechter der
Issaschariten. 18Ihr Gebiet erstreckte sich über Jesreel, Kesulloth,
Sunem, 19Hapharaim, Sion, Anaharath, 20Rabbith, Kisjon, Ebez, 21Remeth,
En-Gannim, En-Hadda und Beth-Pazzez. 22Die Grenze berührt Thabor,
Sahazima und Beth-Semes, und ihre Grenze endet am Jordan: 16 Städte mit
den zugehörigen Dörfern. 23Dies war der Erbbesitz der Geschlechter des
Stammes der Issaschariten: die Städte mit den zugehörigen Dörfern.

\hypertarget{ff-das-gebiet-des-stammes-asser}{%
\subparagraph{ff) Das Gebiet des Stammes
Asser}\label{ff-das-gebiet-des-stammes-asser}}

24Dann kam das fünfte Los heraus für die Geschlechter des Stammes Asser.
25Ihr Gebiet umfaßte Helkath, Hali, Beten, Achsaph, 26Allammelech, Amgad
und Miseal und stößt westwärts an den Karmel und an den Sihor von
Libnath. 27Sodann wendet sich die Grenze ostwärts nach Beth-Dagon,
berührt Sebulon und das Tal Jiphthah-El im Norden, dann Beth-Emek und
Negiel und setzt sich nordwärts fort nach Kabul, 28Ebron, Rehob, Hammon
und Kana bis zu der großen Stadt Sidon; 29dann zieht die Grenze mit
veränderter Richtung nach Rama und bis zu der festen Stadt Tyrus, sodann
mit veränderter Richtung nach Hossa und endet am Meer. Mahaleb, Aksib,
30Umma, Aphek und Rehob: 22 Städte mit den zugehörigen Dörfern. 31Dies
war der Erbbesitz der Geschlechter des Stammes Asser: die genannten
Städte mit den zugehörigen Dörfern.

\hypertarget{gg-das-gebiet-des-stammes-naphthali}{%
\subparagraph{gg) Das Gebiet des Stammes
Naphthali}\label{gg-das-gebiet-des-stammes-naphthali}}

32Für den Stamm Naphthali kam das sechste Los heraus, für die
Geschlechter der Naphthaliten. 33Ihre Grenze geht von Heleph, von der
Eiche\textless sup title=``oder: dem Eichenwald''\textgreater✲ bei
Zaanannim, über Adami-Nekeb und Jabneel bis Lakkum und endet am Jordan;
34dann geht die Grenze mit veränderter Richtung westwärts nach
Asnoth-Thabor, setzt sich fort nach Hukkok hin, berührt dann im Süden
Sebulon, stößt im Westen an Asser und im Osten an den Jordan. 35Die
festen Städte waren: Ziddim, Zer und Hammath, Rakkath, Kinnereth,
36Adama, Rama, Hazor, 37Kedes, Edrei, En-Hazor, 38Jireon, Migdal-El,
Horem, Beth-Anath und Beth-Semes: 19 Städte mit den zugehörigen Dörfern.
39Dies war der Erbbesitz der Geschlechter des Stammes der Naphthaliten:
die Städte mit den zugehörigen Dörfern.

\hypertarget{hh-das-gebiet-des-stammes-dan}{%
\subparagraph{hh) Das Gebiet des Stammes
Dan}\label{hh-das-gebiet-des-stammes-dan}}

40Für die Geschlechter des Stammes Dan kam das siebte Los heraus. 41Das
Gebiet ihres Erbteils umfaßte: Zorga, Esthaol, Ir-Semes, 42Saalabbin,
Ajjalon, Jithla, 43Elon, Thimnath, Ekron, 44Eltheke, Gibbethon, Baalath,
45Jehud, Bene-Berak, Gath-Rimmon, 46Me-Jarkon und Rakkon samt dem
Gebiete gegen Japho hin. 47Als das Gebiet den Daniten später zu enge
wurde, zogen sie hinauf und bekriegten Lesem (Lais); und nachdem sie es
erobert und die Einwohner mit der Schärfe des Schwertes niedergemacht
hatten, nahmen sie es in Besitz, siedelten sich dort an und gaben Lesem
den Namen »Dan« nach ihrem Stammvater Dan\textless sup title=``vgl. Ri
18,27-29''\textgreater✲. 48Dies war der Erbbesitz der Geschlechter des
Stammes Dan: die genannten Städte mit den zugehörigen Dörfern.

\hypertarget{ii-das-besitztum-josuas-abschluuxdf-des-berichts}{%
\subparagraph{ii) Das Besitztum Josuas; Abschluß des
Berichts}\label{ii-das-besitztum-josuas-abschluuxdf-des-berichts}}

49Als nun die Israeliten mit der Verteilung des Landes nach seinem
ganzen Umfang fertig waren, gaben sie Josua, dem Sohne Nuns, ein
Besitztum in ihrer Mitte. 50Nach dem Befehl des HERRN gaben sie ihm die
Stadt, die er sich erbeten hatte, nämlich Thimnath-Serah im Berglande
Ephraim; er befestigte dann die Stadt und ließ sich in ihr nieder.

51Dies sind die Erbteile, die der Priester Eleasar und Josua, der Sohn
Nuns, und die Stammeshäupter der Israeliten in Silo vor dem HERRN am
Eingang des Offenbarungszeltes durch das Los verteilten.

\hypertarget{e-die-sechs-zufluchts--oder-freistuxe4dte}{%
\paragraph{e) Die sechs Zufluchts- oder
Freistädte}\label{e-die-sechs-zufluchts--oder-freistuxe4dte}}

\hypertarget{aa-der-guxf6ttliche-befehl}{%
\subparagraph{aa) Der göttliche
Befehl}\label{aa-der-guxf6ttliche-befehl}}

Als sie dann mit der Verteilung des Landes fertig waren,

\hypertarget{section-19}{%
\section{20}\label{section-19}}

1gebot der HERR dem Josua folgendes: 2»Mache den Israeliten folgende
Mitteilung: ›Bestimmt euch noch die Zufluchtsstädte\textless sup
title=``oder: Freistädte''\textgreater✲, von denen ich zu euch durch
Mose gesagt habe\textless sup title=``4.Mose 35,9-15; 5.Mose
19,1-13''\textgreater✲, 3daß ein Totschläger dahin fliehen solle, der
jemand aus Versehen, unvorsätzlich, getötet hat; die sollen euch als
Zufluchtsstätten vor dem Bluträcher dienen.‹ 4Flüchtet er sich dann in
eine von diesen Städten und bleibt am Eingang des Stadttores stehen und
trägt den Ältesten der betreffenden Stadt seine Sache vor, so sollen sie
ihn bei sich in der Stadt aufnehmen und ihm einen Ort anweisen, daß er
bei ihnen wohnen kann. 5Wenn dann der Bluträcher ihn verfolgt, so dürfen
sie ihm den Totschläger nicht ausliefern, weil er den andern
unvorsätzlich getötet hat, ohne ihm schon früher feind gewesen zu sein.
6Er soll vielmehr in der betreffenden Stadt wohnen bleiben, bis er vor
der Gemeinde gestanden hat, um abgeurteilt zu werden, und dann bis zum
Tode des derzeitigen Hohenpriesters; alsdann darf der Totschläger wieder
in seine Ortschaft und in sein Haus zurückkehren, in die Ortschaft, aus
der er geflohen war.«

\hypertarget{bb-ausfuxfchrung-des-befehls}{%
\subparagraph{bb) Ausführung des
Befehls}\label{bb-ausfuxfchrung-des-befehls}}

7Da machten die Israeliten Kedes in Galiläa im Gebirge Naphthali und
Sichem im Berglande auf dem Gebirge Ephraim und Kirjath-Arba, das ist
Hebron, im Gebirge Juda zu geweihten Zufluchtsstätten. 8Jenseits des
Jordans aber, östlich von Jericho, bestimmten sie zu derartigen Städten
Bezer in der Steppe, auf der Hochebene im Stamme Ruben, ferner Ramoth in
Gilead im Stamme Gad, und Golan in Basan im Stamme Manasse. 9Dies waren
die Städte, die man für alle Israeliten und für die unter ihnen lebenden
Fremdlinge dazu bestimmte, daß jeder, der einen andern unvorsätzlich
getötet hätte, sich dorthin flüchten sollte, damit er nicht durch die
Hand des Bluträchers den Tod fände, ehe er vor der Gemeinde gestanden
hätte.

\hypertarget{f-die-achtundvierzig-priester--und-levitenstuxe4dte}{%
\paragraph{f) Die achtundvierzig Priester- und
Levitenstädte}\label{f-die-achtundvierzig-priester--und-levitenstuxe4dte}}

\hypertarget{section-20}{%
\section{21}\label{section-20}}

1Hierauf traten die Familienhäupter der Leviten zu dem Priester Eleasar
und zu Josua, dem Sohne Nuns, und zu den Häuptern der israelitischen
Stämme 2und sagten zu ihnen in Silo im Lande Kanaan: »Der HERR hat durch
Mose geboten, daß man uns Städte zu Wohnsitzen und die zugehörigen
Weidetriften für unser Vieh anweise.« 3So überwiesen denn die Israeliten
von ihrem Erbbesitz den Leviten nach dem Gebot des HERRN folgende Städte
nebst den zugehörigen Weidetriften:

4Als das Los für die Familien der Kahathiten herauskam, erhielten unter
den Leviten die Nachkommen des Priesters Aaron von den Stämmen Juda,
Simeon und Benjamin durch das Los dreizehn Städte\textless sup
title=``oder: Ortschaften''\textgreater✲; 5die übrigen Nachkommen
Kahaths aber erhielten durch das Los zehn Städte von den Geschlechtern
der Stämme Ephraim, Dan und halb Manasse.~-- 6Die Nachkommen Gersons
erhielten durch das Los dreizehn Städte von den Geschlechtern der Stämme
Issaschar, Asser, Naphthali und halb Manasse in Basan.~-- 7Die Familien
der Nachkommen Meraris erhielten zwölf Städte von den Stämmen Ruben, Gad
und Sebulon. 8Diese Städte also mit den zugehörigen Weidetriften wiesen
die Israeliten den Leviten durch das Los zu, wie der HERR es durch Mose
geboten hatte.

9Sie traten aber von seiten der Stämme Juda und Simeon folgende mit
Namen bezeichnete Städte ab: 10den Nachkommen Aarons -- denn auf sie
fiel das erste Los -- wurde von den Familien der Kahathiten unter den
Leviten folgendes zuteil: 11man gab ihnen die Stadt Arbas, des Vaters
Enaks, das ist Hebron, im Gebirge Juda, samt den zugehörigen
Weidetriften rings um die Stadt her; 12aber das zur Stadt gehörige
Ackerland nebst den zugehörigen Dörfern hatten sie Kaleb, dem Sohne
Jephunnes, als seinen Besitz gegeben. 13Sie überwiesen also den
Nachkommen des Priesters Aaron Hebron, die Zufluchtsstadt für
Totschläger, nebst den zugehörigen Weidetriften, außerdem Libna nebst
den zugehörigen Weidetriften, 14Jatthir nebst den zugehörigen
Weidetriften, Esthemoa nebst den zugehörigen Weidetriften, 15Holon nebst
den zugehörigen Weidetriften, Debir nebst den zugehörigen Weidetriften,
16Asan\textless sup title=``1.Chr 6,44''\textgreater✲ nebst den
zugehörigen Weidetriften, Jutta nebst den zugehörigen Weidetriften,
Beth-Semes nebst den zugehörigen Weidetriften: neun Städte von diesen
beiden Stämmen. 17Ferner vom Stamme Benjamin: Gibeon nebst den
zugehörigen Weidetriften, Geba nebst den zugehörigen Weidetriften,
18Anathoth nebst den zugehörigen Weidetriften und Almon nebst den
zugehörigen Weidetriften: vier Städte. 19Demnach erhielten die
Nachkommen Aarons, die Priester, im ganzen dreizehn Städte nebst den
zugehörigen Weidetriften.

20Was sodann die Familien der übrigen zu den Leviten gehörenden
Nachkommen Kahaths betrifft, so waren die ihnen durch das Los
zugefallenen Städte vom Gebiet des Stammes Ephraim genommen; 21und zwar
überwies man ihnen Sichem, die Zufluchtsstadt für Totschläger, nebst den
zugehörigen Weidetriften im Gebirge Ephraim; ferner Geser nebst den
zugehörigen Weidetriften, 22Kibzaim nebst den zugehörigen Weidetriften
und Beth-Horon nebst den zugehörigen Weidetriften: vier Städte. 23Ferner
vom Stamme Dan: Eltheke nebst den zugehörigen Weidetriften, Gibbethon
nebst den zugehörigen Weidetriften, 24Ajjalon nebst den zugehörigen
Weidetriften, Gath-Rimmon nebst den zugehörigen Weidetriften: vier
Städte. 25Sodann vom halben Stamme Manasse: Thaanach nebst den
zugehörigen Weidetriften und Gath-Rimmon nebst den zugehörigen
Weidetriften: zwei Städte. 26Demnach erhielten die Familien der übrigen
Nachkommen Kahaths im ganzen zehn Städte nebst den zugehörigen
Weidetriften.

27Weiter erhielten unter den Geschlechtern der Leviten die Nachkommen
Gersons vom halben Stamm Manasse: Golan in Basan, die Zufluchtsstadt für
Totschläger, nebst den zugehörigen Weidetriften, und Beesthera nebst den
zugehörigen Weidetriften: zwei Städte. 28Ferner vom Stamme Issaschar:
Kisjon nebst den zugehörigen Weidetriften, Daberath nebst den
zugehörigen Weidetriften, 29Jarmuth nebst den zugehörigen Weidetriften
und En-Gannim nebst den zugehörigen Weidetriften: vier Städte. 30Weiter
vom Stamme Asser: Miseal nebst den zugehörigen Weidetriften, Abdon nebst
den zugehörigen Weidetriften, 31Helkath nebst den zugehörigen
Weidetriften und Rehob nebst den zugehörigen Weidetriften: vier Städte.
32Sodann vom Stamme Naphthali: Kedes in Galiläa, die Zufluchtsstadt für
Totschläger, nebst den zugehörigen Weidetriften, Hammoth-Dor nebst den
zugehörigen Weidetriften und Karthan nebst den zugehörigen Weidetriften:
drei Städte. 33Demnach erhielten die Familien der Gersoniten im ganzen
dreizehn Städte nebst den zugehörigen Weidetriften.

34Die Familien der Nachkommen Meraris aber, die noch übrigen Leviten,
erhielten vom Stamme Sebulon: Jokneam nebst den zugehörigen
Weidetriften, Kartha nebst den zugehörigen Weidetriften, 35Dimna nebst
den zugehörigen Weidetriften und Nahalal nebst den zugehörigen
Weidetriften: vier Städte. 36Ferner vom Stamme Ruben: Bezer, (die
Zufluchtsstadt für Totschläger,) nebst den zugehörigen Weidetriften,
Jahza nebst den zugehörigen Weidetriften, 37Kedemoth nebst den
zugehörigen Weidetriften und Mephaath nebst den zugehörigen
Weidetriften: vier Städte. 38Weiter vom Stamme Gad: Ramoth in Gilead,
die Zufluchtsstadt für Totschläger, nebst den zugehörigen Weidetriften,
Mahanaim nebst den zugehörigen Weidetriften, 39Hesbon nebst den
zugehörigen Weidetriften, Jaser nebst den zugehörigen Weidetriften: im
ganzen vier Städte. 40Demnach erhielten die Familien der Nachkommen
Meraris, die von den Geschlechtern der Leviten noch übrig waren, als
Losanteil zwölf Städte.

41Die Gesamtzahl der levitischen Städte inmitten des Erbbesitzes der
Israeliten betrug achtundvierzig Städte nebst den zugehörigen
Weidetriften. 42Diese Städte hatten ausnahmslos ihre Weidetriften rings
um sich her: so verhielt es sich bei allen diesen Städten\textless sup
title=``oder: Ortschaften''\textgreater✲.

\hypertarget{g-abschlieuxdfender-ruxfcckblick}{%
\paragraph{g) Abschließender
Rückblick}\label{g-abschlieuxdfender-ruxfcckblick}}

43So hatte also der HERR den Israeliten das ganze Land gegeben, dessen
Verleihung er ihren Vätern zugeschworen hatte; sie hatten es in Besitz
genommen und sich darin niedergelassen. 44Und der HERR verschaffte ihnen
Ruhe auf allen Seiten, ganz so, wie er es ihren Vätern zugeschworen
hatte; denn keiner von all ihren Feinden hatte ihnen widerstehen können:
alle ihre Feinde hatte der HERR in ihre Gewalt gegeben. 45Von allen
Segensverheißungen, die der HERR dem Hause Israel gegeben hatte, war
keine einzige unerfüllt geblieben: alle waren eingetroffen.

\hypertarget{iii.-schluuxdferzuxe4hlungen-von-der-wirksamkeit-josuas-kap.-22-24}{%
\subsection{III. Schlußerzählungen von der Wirksamkeit Josuas (Kap.
22-24)}\label{iii.-schluuxdferzuxe4hlungen-von-der-wirksamkeit-josuas-kap.-22-24}}

\hypertarget{entlassung-der-ostjordanischen-stuxe4mme-und-deren-altarbau-am-jordan}{%
\subsubsection{1. Entlassung der ostjordanischen Stämme und deren
Altarbau am
Jordan}\label{entlassung-der-ostjordanischen-stuxe4mme-und-deren-altarbau-am-jordan}}

\hypertarget{a-josua-entluxe4uxdft-die-stuxe4mme-mit-worten-der-anerkennung-mit-mahnung-und-segen}{%
\paragraph{a) Josua entläßt die Stämme mit Worten der Anerkennung, mit
Mahnung und
Segen}\label{a-josua-entluxe4uxdft-die-stuxe4mme-mit-worten-der-anerkennung-mit-mahnung-und-segen}}

\hypertarget{section-21}{%
\section{22}\label{section-21}}

1Damals berief Josua die Stämme Ruben, Gad und halb Manasse 2und sagte
zu ihnen: »Ihr habt alles erfüllt, was euch Mose, der Knecht des HERRN,
geboten hat, und habt euch auch mir gegenüber in allem, was ich euch
befohlen habe, gehorsam bewiesen: 3ihr habt eure Volksgenossen diese
lange Zeit hindurch bis auf den heutigen Tag nicht im Stich gelassen und
das Gebot des HERRN, eures Gottes, treu erfüllt. 4Da nun aber der HERR,
euer Gott, euren Volksgenossen Ruhe\textless sup title=``d.h.
Ruhesitze''\textgreater✲ verschafft hat, wie er ihnen verheißen hatte,
so kehrt jetzt zu euren Zelten\textless sup title=``=~in eure
Heimat''\textgreater✲ zurück, in das euch gehörige Land, das euch Mose,
der Knecht des HERRN, jenseits des Jordans zugewiesen hat. 5Nur seid auf
die genaue Beobachtung des Gesetzes und der Gebote bedacht, die euch
Mose, der Knecht des HERRN, zur Pflicht gemacht hat, daß ihr den HERRN,
euren Gott, liebt und allezeit auf seinen Wegen wandelt, seine Gebote
beobachtet und ihm treu bleibt und ihm von ganzem Herzen und mit ganzer
Seele dient!« 6Als Josua sie dann mit Segenswünschen verabschiedet
hatte, kehrten sie zu ihren Zelten zurück.

7Der einen Hälfte des Stammes Manasse aber hatte Mose in Basan
Landbesitz gegeben; der andern Hälfte dagegen hatte Josua bei ihren
Volksgenossen im Westjordanlande ihren Erbbesitz angewiesen. Als Josua
sie nun in ihre Heimat entließ, segnete er auch sie 8und sagte zu ihnen:
»Kehrt zu euren Zelten zurück mit vielen Schätzen und mit einem sehr
großen Viehbesitz, mit Silber und Gold, mit Kupfer, Eisen und Kleidern
in großer Menge. Teilt das, was ihr von euren Feinden erbeutet habt, mit
euren Brüdern✲!«

\hypertarget{b-der-altarbau-der-ostjordanischen-stuxe4mme-in-gilgal-und-seine-uxfcblen-folgen-rede-des-priesters-pinehas}{%
\paragraph{b) Der Altarbau der ostjordanischen Stämme in Gilgal und
seine üblen Folgen; Rede des Priesters
Pinehas}\label{b-der-altarbau-der-ostjordanischen-stuxe4mme-in-gilgal-und-seine-uxfcblen-folgen-rede-des-priesters-pinehas}}

9So kehrten denn die Stämme Ruben, Gad und halb Manasse zurück und
verließen die übrigen Israeliten in Silo, das im Lande Kanaan liegt, um
in das ihnen gehörige Land Gilead heimzuziehen, wo sie sich nach dem
durch Mose verkündigten Befehl des HERRN ansässig gemacht hatten. 10Als
sie nun an die Steinkreise am Jordan gekommen waren, die noch im Lande
Kanaan liegen, da bauten die Stämme Ruben, Gad und halb Manasse dort
einen Altar am Jordan, einen großen, weithin sichtbaren Altar. 11Als
aber die (übrigen) Israeliten die Kunde erhielten: »Die Stämme Ruben,
Gad und halb Manasse haben den Altar gerade dem Lande Kanaan gegenüber
bei den Steinkreisen am Jordan, auf der anderen Seite des Gebietes der
Israeliten, gebaut«~-- 12als die Israeliten das erfuhren, da versammelte
sich die ganze Gemeinde der Israeliten in Silo, um gegen sie zu Felde zu
ziehen. 13Da sandten die Israeliten an die Stämme Ruben, Gad und halb
Manasse in das Land Gilead Pinehas, den Sohn des Priesters Eleasar,
14und zehn Fürsten mit ihm, je einen Fürsten von jedem Stamme, von
sämtlichen israelitischen Stämmen, von denen jeder das Oberhaupt der
Familien seines Stammes unter den Tausendschaften Israels war. 15Als
diese zu den Stämmen Ruben, Gad und halb Manasse in das Land Gilead
gekommen waren, besprachen sie sich mit ihnen und sagten: 16»Die ganze
Gemeinde des HERRN läßt euch folgendes sagen: Was ist das für eine
Treulosigkeit, die ihr euch gegen den Gott Israels habt zuschulden
kommen lassen, daß ihr euch heute vom HERRN abwendet, indem ihr euch
einen Altar baut und euch so gegen den HERRN auflehnt? 17Haben wir noch
nicht genug an der Versündigung bezüglich Peors\textless sup
title=``vgl. 4.Mose 25,1-15''\textgreater✲, von der wir uns bis auf den
heutigen Tag noch nicht gereinigt haben und um deretwillen das Sterben
über die Gemeinde des HERRN kam? 18Und ihr wollt euch dennoch heute vom
HERRN lossagen? Die Folge davon wird sein: wenn ihr euch heute gegen den
HERRN auflehnt, so wird er morgen seinen Zorn an der ganzen Gemeinde
Israel auslassen. 19Wenn übrigens das Land, das ihr innehabt, nach eurer
Ansicht unrein ist, so kommt doch in das Eigentumsland des HERRN
herüber, wo sich die Wohnstätte des HERRN befindet, und macht euch
mitten unter uns ansässig! Aber lehnt euch nicht gegen den HERRN auf und
lehnt euch nicht gegen uns auf, indem ihr euch außer dem Altar des
HERRN, unsres Gottes, noch einen besonderen Altar erbaut! 20Ist nicht
damals, als Achan, der Sohn Serahs, sich treuloserweise an dem gebannten
Gut vergriffen hatte, ein Zorngericht über die ganze Gemeinde Israel
ergangen, obgleich er nur ein einzelner Mann war? Hat er nicht sein
Vergehen mit dem Tode büßen müssen?«

\hypertarget{c-die-ostjordanischen-stuxe4mme-rechtfertigen-sich-erfolgreich}{%
\paragraph{c) Die ostjordanischen Stämme rechtfertigen sich
erfolgreich}\label{c-die-ostjordanischen-stuxe4mme-rechtfertigen-sich-erfolgreich}}

21Da antworteten die Stämme Ruben, Gad und halb Manasse den Häuptern der
Tausendschaften Israels folgendermaßen: 22»Der Starke, Gott der HERR, ja
der Starke, Gott der HERR, der weiß es, und Israel soll es wissen: Wenn
es aus Auflehnung und wenn es aus Treulosigkeit gegen den HERRN
geschehen ist, so möge uns am heutigen Tage keine Rettung von dir (o
Pinehas) zuteil werden! 23Wenn wir uns einen Altar erbaut haben, um uns
vom HERRN abzuwenden, oder wenn es zu dem Zweck geschehen ist, daß wir
auf ihm Brandopfer und Speisopfer darbringen oder Heilsopfer auf ihm
herrichten wollten, so möge der HERR selbst es ahnden! 24Nein, nur aus
Besorgnis vor einer gewissen Sache\textless sup title=``oder: einem
möglichen Gerede''\textgreater✲ haben wir das getan, weil wir nämlich
dachten, künftig würden eure Kinder zu unsern Kindern sagen: ›Was habt
ihr mit dem HERRN, dem Gott Israels, gemein? 25Der HERR hat ja doch den
Jordan zur Grenze zwischen uns und euch Rubeniten und Gaditen gemacht:
ihr habt keinen Anteil am HERRN!‹ Dadurch würden eure Kinder dann die
unsrigen davon abbringen, den HERRN zu fürchten. 26Darum dachten wir:
Wir wollen uns doch daranmachen, den Altar zu bauen, nicht für
Brandopfer und nicht für Schlachtopfer; 27sondern er soll zwischen uns
und euch und zwischen unsern künftigen Geschlechtern ein Zeugnis sein,
daß wir dem Dienst des HERRN vor ihm mit unsern Brandopfern, unsern
Schlachtopfern und unsern Heilsopfern obliegen (wollen); sonst könnten
eure Kinder künftig zu unsern Kindern sagen: ›Ihr habt keinen Anteil am
HERRN!‹ 28Wir dachten also: Wenn sie künftig zu uns oder unsern
Nachkommen so sprechen sollten, dann wollen wir entgegnen: ›Seht euch
doch die Bauart des Gottesaltars an, den unsere Väter errichtet haben,
nicht für Brandopfer und nicht für Schlachtopfer, sondern zum Zeugnis
zwischen uns und euch!‹ 29Fern sei es von uns, daß wir uns gegen den
HERRN auflehnen (wollen) und uns heute vom HERRN abwenden, indem wir für
Brandopfer, für Speisopfer und für Schlachtopfer noch einen besonderen
Altar bauen außer dem Altar des HERRN, unseres Gottes, der vor seiner
Wohnung steht!«

30Als nun der Priester Pinehas und die Fürsten der Gemeinde, die Häupter
der Tausendschaften Israels, die ihn begleiteten, diese Erklärung der
Rubeniten, Gaditen und Manassiten vernommen hatten, waren sie dadurch
zufriedengestellt. 31Daher antwortete Pinehas, der Sohn des Priesters
Eleasar, den Rubeniten, Gaditen und Manassiten: »Heute erkennen wir, daß
der HERR wirklich in unserer Mitte ist, weil ihr euch eine solche
Treulosigkeit gegen den HERRN nicht habt zuschulden kommen lassen.
Dadurch habt ihr die Israeliten vor der Hand\textless sup title=``=~vor
dem Strafgericht''\textgreater✲ des HERRN behütet.« 32Als hierauf
Pinehas, der Sohn des Priesters Eleasar, mit den Fürsten aus dem Lande
Gilead von den Rubeniten und Gaditen ins Land Kanaan zu den Israeliten
zurückgekehrt war und sie ihnen Bericht erstattet hatten, 33waren die
Israeliten dadurch zufriedengestellt; sie priesen Gott und dachten nicht
mehr daran, gegen sie mit Heeresmacht auszuziehen, um das Land zu
verwüsten, in welchem die Rubeniten und Gaditen wohnten. 34Die Rubeniten
und Gaditen aber gaben dem Altar den Namen ›Zeuge\textless sup
title=``oder: Zeugnis''\textgreater✲‹; denn (sie sagten): »Er soll als
Zeuge zwischen uns dienen, daß der HERR (der wahre) Gott ist.«

\hypertarget{josuas-abschied-vom-volk-und-sein-tod}{%
\subsubsection{2. Josuas Abschied vom Volk und sein
Tod}\label{josuas-abschied-vom-volk-und-sein-tod}}

\hypertarget{a-josuas-erste-ermahnende-ansprache-an-die-vertreter-israels}{%
\paragraph{a) Josuas erste ermahnende Ansprache an die Vertreter
Israels}\label{a-josuas-erste-ermahnende-ansprache-an-die-vertreter-israels}}

\hypertarget{section-22}{%
\section{23}\label{section-22}}

1Geraume Zeit später, nachdem der HERR den Israeliten Ruhe vor allen
ihren Feinden ringsum verschafft hatte und Josua alt und hochbetagt
geworden war, 2berief Josua das gesamte Israel, die
Ältesten\textless sup title=``oder: Vornehmsten''\textgreater✲ und die
Oberhäupter des Volkes, seine Richter und Obmänner\textless sup
title=``vgl. 5.Mose 1,15''\textgreater✲, und sprach zu ihnen: »Ich bin
nun alt geworden und hochbetagt; 3ihr aber habt alles gesehen, was der
HERR, euer Gott, allen diesen Völkerschaften vor euch her hat
widerfahren lassen; denn der HERR, euer Gott, ist es, der für euch
gestritten hat. 4Wisset wohl: ich habe diese Völkerschaften, die noch
übriggeblieben sind, durch das Los euren einzelnen Stämmen als Erbteil
zugewiesen, gerade so wie alle Völkerschaften, die ich ausgerottet habe,
vom Jordan an bis an das große Meer gegen Sonnenuntergang; 5und der
HERR, euer Gott, selbst wird sie vor euch verjagen und vor euch her
vertreiben, und ihr werdet ihr Land in Besitz nehmen, wie der HERR, euer
Gott, es euch verheißen hat. 6So haltet denn unerschütterlich fest
daran, alles, was im Gesetzbuch Moses geschrieben steht, zu beobachten
und zu befolgen, ohne nach rechts und nach links davon abzuweichen,
7damit ihr euch mit diesen Völkerschaften, die bei euch noch
übriggeblieben sind, nicht vermischt und den Namen ihrer Götter nicht
anruft\textless sup title=``oder: in den Mund nehmt''\textgreater✲, auch
bei ihnen nicht schwört und ihnen nicht dient und sie nicht anbetet;
8sondern am HERRN, eurem Gott, sollt ihr festhalten, wie ihr es bis auf
den heutigen Tag getan habt. 9Der HERR hat ja große und mächtige
Völkerschaften vor euch her vertrieben, und niemand hat vor euch bis auf
den heutigen Tag standhalten können. 10Ein einziger Mann von euch treibt
tausend von ihnen in die Flucht; denn der HERR, euer Gott, ist es, der
für euch streitet, wie er euch verheißen hat. 11So seid denn um eures
Lebens willen eifrig darauf bedacht, den HERRN, euren Gott, zu lieben!
12Denn wenn ihr euch irgendwie von ihm abwendet und euch dem Überrest
dieser Völkerschaften, die bei euch noch übriggeblieben sind, anschließt
und Heiraten mit ihnen eingeht, so daß ihr euch mit ihnen vermischt und
sie mit euch: 13so sollt ihr bestimmt wissen, daß dann der HERR, euer
Gott, diese Völkerschaften nicht länger vor euch her vertreiben wird,
sondern sie werden für euch zur Schlinge und zum Fallstrick werden, zur
Geißel in euren Seiten und zu Dornen in euren Augen, bis ihr aus diesem
schönen Lande vertilgt seid, das der HERR, euer Gott, euch gegeben hat.

14Seht, ich gehe jetzt den Weg alles Irdischen\textless sup
title=``=~den alle Welt gehen muß''\textgreater✲; so bedenkt denn mit
ganzem Herzen und mit ganzer Seele, daß von all den Segensverheißungen,
die der HERR, euer Gott, in bezug auf euch gegeben hat, keine einzige
unerfüllt geblieben ist; nein, alle sind bei euch eingetroffen, keine
einzige von ihnen ist unerfüllt geblieben. 15Aber wie alle
Segensverheißungen, die der HERR, euer Gott, euch gegeben hat, bei euch
eingetroffen sind, ebenso wird der HERR auch alle Drohungen an euch in
Erfüllung gehen lassen, bis er euch aus diesem schönen Lande vertilgt
hat, das ihr vom HERRN, eurem Gott, empfangen habt. 16Wenn ihr den Bund,
zu dem der HERR, euer Gott, euch verpflichtet hat, übertretet und
hingeht, um anderen Göttern zu dienen und sie anzubeten, so wird der
Zorn des HERRN gegen euch entbrennen, und ihr werdet schnell aus dem
schönen Lande verschwunden sein, das er euch gegeben hat.«

\hypertarget{b-josua-nimmt-auf-dem-landtage-zu-sichem-abschied-vom-volk}{%
\paragraph{b) Josua nimmt auf dem Landtage zu Sichem Abschied vom
Volk}\label{b-josua-nimmt-auf-dem-landtage-zu-sichem-abschied-vom-volk}}

\hypertarget{section-23}{%
\section{24}\label{section-23}}

1Weiter versammelte Josua alle Stämme der Israeliten in Sichem und
berief dorthin die Ältesten\textless sup title=``oder:
Vornehmsten''\textgreater✲ der Israeliten sowie ihre Oberhäupter, ihre
Richter und Obmänner. Als sie sich dann vor Gott aufgestellt hatten,
2sagte Josua zu dem ganzen Volke: »So spricht der HERR, der Gott
Israels: ›Eure Väter haben vor alters jenseits des Euphratstromes
gewohnt, nämlich Tharah, der Vater Abrahams und Nahors, und haben andere
Götter verehrt. 3Da holte ich euren Vater Abraham aus dem Lande jenseits
des Euphratstromes und ließ ihn im ganzen Lande Kanaan umherwandern und
gab ihm zahlreiche Nachkommenschaft, nachdem ich ihm Isaak geschenkt
hatte. 4Dem Isaak aber ließ ich Jakob und Esau geboren werden und gab
dem Esau das Gebirge Seir zum Besitz, während Jakob und seine Söhne nach
Ägypten hinabzogen. 5Dann sandte ich Mose und Aaron und suchte Ägypten
heim mit den Wundertaten, die ich inmitten des Landes verrichtete, und
führte euch danach von dort weg. 6Als ich aber eure Väter aus Ägypten
wegführte und ihr an das Meer gekommen wart, verfolgten die Ägypter eure
Väter mit Kriegswagen und Reitern bis ans Schilfmeer. 7Als sie nun zum
HERRN um Hilfe schrien, ließ er dichte Finsternis zwischen euch und die
Ägypter treten und ließ das Meer über sie hinströmen, so daß es sie
überflutete; ihr habt ja mit eigenen Augen gesehen, was ich den Ägyptern
habe widerfahren lassen. Dann habt ihr geraume Zeit in der Wüste
zugebracht. 8Hierauf führte ich euch in das Land der Amoriter, die
jenseits des Jordans ansässig waren; und als sie gegen euch kämpften,
gab ich sie in eure Gewalt, so daß ihr in den Besitz ihres Landes kamet,
und ich vernichtete sie vor euch. 9Als dann der Moabiterkönig Balak, der
Sohn Zippors, sich erhob und gegen Israel kämpfen wollte und Bileam, den
Sohn Beors, durch Gesandte rufen ließ, damit er euch verfluche, 10war
ich nicht gewillt, auf Bileam zu hören; er mußte euch vielmehr segnen,
und so errettete ich euch aus seiner Gewalt. 11Als ihr hierauf über den
Jordan gezogen und nach Jericho gekommen wart und die Bürger von Jericho
sowie die Amoriter, Pherissiter, Kanaanäer, Hethiter, Girgasiter,
Hewiter und Jebusiter feindlich gegen euch auftraten, gab ich sie in
eure Gewalt 12und sandte die Hornissen\textless sup title=``2.Mose
23,28''\textgreater✲ vor euch her: die trieben sie vor euch her in die
Flucht, die beiden Amoriterkönige, ohne Zutun deines Schwertes und
deines Bogens. 13So habe ich euch ein Land gegeben, um das du dich nicht
hast zu mühen brauchen, und Städte, in denen ihr jetzt wohnt, ohne sie
gebaut zu haben; von Weinbergen und Ölbaumgärten, die ihr nicht angelegt
habt, genießt ihr die Früchte.‹

14So fürchtet nun den HERRN und dient ihm aufrichtig und treu! Schafft
die Götter weg, denen eure Väter jenseits des Euphratstromes und in
Ägypten gedient haben, und dient dem HERRN! 15Wollt ihr euch aber nicht
dazu verstehen, dem HERRN zu dienen, so entscheidet euch heute, wem ihr
dienen wollt, ob den Göttern, denen eure Väter jenseits des
Euphratstromes gedient haben, oder den Göttern der Amoriter, in deren
Lande ihr wohnt. Ich aber und mein Haus, wir wollen dem HERRN dienen!«

\hypertarget{das-volk-gelobt-treuen-gehorsam-und-wird-von-josua-aufs-neue-feierlich-fuxfcr-gott-verpflichtet}{%
\paragraph{Das Volk gelobt treuen Gehorsam und wird von Josua aufs neue
feierlich für Gott
verpflichtet}\label{das-volk-gelobt-treuen-gehorsam-und-wird-von-josua-aufs-neue-feierlich-fuxfcr-gott-verpflichtet}}

16Da gab das Volk die Erklärung ab: »Fern sei es von uns, den HERRN zu
verlassen und anderen Göttern zu dienen! 17Denn der HERR, unser Gott,
ist es, der uns und unsere Väter aus dem Lande Ägypten, aus dem Hause
der Knechtschaft, hergeführt und der vor unsern Augen jene großen Wunder
verrichtet und uns auf dem ganzen Wege, den wir gezogen sind, und unter
all den Völkerschaften, durch deren Land unsere Wanderung gegangen ist,
behütet hat. 18Ja, der HERR ist es, der alle Völker, auch die Amoriter,
die Bewohner des Landes, vor uns her vertrieben hat. Auch wir wollen dem
HERRN dienen, denn er ist unser Gott!«

19Da sagte Josua zum Volk: »Ihr seid nicht imstande, dem HERRN zu
dienen; denn er ist ein heiliger Gott; ein eifersüchtiger Gott ist er,
der euch eure Übertretungen und eure Sünden nicht vergeben wird. 20Wenn
ihr den HERRN verlaßt und fremden Göttern dient, so wird er sich (von
euch) abwenden und euch Unheil widerfahren lassen und euch vernichten,
nachdem er euch Gutes getan hat.« 21Das Volk aber erklärte dem Josua:
»Nein, dem HERRN wollen wir dienen!« 22Da sagte Josua zum Volk: »Ihr
seid Zeugen gegen euch selbst, daß ihr euch den HERRN erwählt habt, ihm
zu dienen.« Sie antworteten: »Ja, wir sind Zeugen!« 23»So schafft nun
die fremden Götter weg, die noch unter euch sind, und neigt euer Herz
dem HERRN, dem Gott Israels, zu!« 24Da erklärte das Volk dem Josua: »Dem
HERRN, unserm Gott, wollen wir dienen und seinen Weisungen gehorchen!«

25So verpflichtete denn Josua das Volk an jenem Tage feierlich auf den
Bund und setzte für die Israeliten in Sichem Gesetz und
Recht\textless sup title=``=~Verordnungen und Pflichten''\textgreater✲
fest. 26Darauf trug Josua dies alles in das Gesetzbuch Gottes ein, nahm
dann einen großen Stein und richtete ihn dort unter der
Terebinthe\textless sup title=``1.Mose 35,4''\textgreater✲ auf, die
im\textless sup title=``oder: beim''\textgreater✲ Heiligtum des HERRN
stand. 27Dann sagte Josua zu dem ganzen Volk: »Wisset wohl: dieser Stein
da soll als Zeuge gegen uns dienen, denn er hat alle Worte gehört, die
der HERR zu uns geredet hat; darum soll er Zeuge gegen euch sein, damit
ihr gegen euren Gott nicht treulos handelt!« 28Hierauf entließ Josua das
Volk, einen jeden in sein Besitztum.

\hypertarget{c-abschluuxdf-des-buches-josuas-tod-und-begruxe4bnis-vgl.-ri-28-9-beisetzung-der-gebeine-josephs-tod-eleasars}{%
\paragraph{c) Abschluß des Buches; Josuas Tod und Begräbnis (vgl. Ri
2,8-9); Beisetzung der Gebeine Josephs; Tod
Eleasars}\label{c-abschluuxdf-des-buches-josuas-tod-und-begruxe4bnis-vgl.-ri-28-9-beisetzung-der-gebeine-josephs-tod-eleasars}}

29Nach diesen Begebenheiten starb Josua, der Sohn Nuns, der Knecht des
HERRN, im Alter von hundertundzehn Jahren; 30und man begrub ihn im
Bereich seines Erbbesitzes zu Thimnath-Serah auf dem Gebirge Ephraim
nördlich vom Berge Gaas. 31Die Israeliten aber dienten dem HERRN,
solange Josua lebte und während der ganzen Lebenszeit der Ältesten, die
Josua noch lange überlebten und die alle die Taten kannten, die der HERR
an\textless sup title=``oder: für''\textgreater✲ Israel vollbracht
hatte.

32Die Gebeine Josephs aber, welche die Israeliten aus Ägypten
mitgebracht hatten\textless sup title=``vgl. 2.Mose
13,19''\textgreater✲, begrub man zu Sichem auf dem Stück Land, das Jakob
einst von den Söhnen Hemors, des Vaters Sichems, um den Preis von
hundert Kesita gekauft hatte\textless sup title=``vgl. 1.Mose
33,19''\textgreater✲ und das dann in den Besitz der Nachkommen Josephs
übergegangen war.

33Als hierauf auch Eleasar, der Sohn Aarons, gestorben war, begrub man
ihn in Gibea, der Stadt seines Sohnes Pinehas, die diesem auf dem
Gebirge Ephraim zugewiesen worden war.
