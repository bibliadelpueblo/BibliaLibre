\hypertarget{das-zweite-buch-mose}{%
\section{DAS ZWEITE BUCH MOSE}\label{das-zweite-buch-mose}}

\emph{(genannt Exodus, d.h. das Buch vom Auszug)}

\hypertarget{i.-bedruxfcckung-und-errettung-der-israeliten-in-uxe4gypten-11-1521}{%
\subsection{I. Bedrückung und Errettung der Israeliten in Ägypten
(1,1-15,21)}\label{i.-bedruxfcckung-und-errettung-der-israeliten-in-uxe4gypten-11-1521}}

\hypertarget{die-schnelle-vermehrung-der-israeliten-trotz-der-druxfcckenden-knechtung-durch-die-uxe4gypter}{%
\subsubsection{1. Die schnelle Vermehrung der Israeliten trotz der
drückenden Knechtung durch die
Ägypter}\label{die-schnelle-vermehrung-der-israeliten-trotz-der-druxfcckenden-knechtung-durch-die-uxe4gypter}}

\hypertarget{section}{%
\section{1}\label{section}}

\bibleverse{1} Dies sind die Namen der Söhne Israels, die nach Ägypten
gekommen waren -- mit Jakob waren sie gekommen, ein jeder mit seiner
Familie --: \bibleverse{2} Ruben, Simeon, Levi und Juda; \bibleverse{3}
Issaschar, Sebulon und Benjamin; \bibleverse{4} Dan und Naphthali, Gad
und Asser. \bibleverse{5} Die Gesamtzahl der leiblichen Nachkommen
Jakobs betrug siebzig Seelen; Joseph aber hatte sich (bereits) in
Ägypten befunden.

\bibleverse{6} Als aber Joseph und alle seine Brüder, überhaupt alle
gestorben waren, welche in jener Zeit gelebt hatten, \bibleverse{7}
vermehrten sich die Israeliten gewaltig und wurden über alle Maßen
zahlreich und stark, so daß das Land voll von ihnen wurde.
\bibleverse{8} Da kam ein neuer König in Ägypten zur Regierung, der
Joseph nicht gekannt hatte\textless sup title=``oder: von Joseph nichts
wußte''\textgreater✲. \bibleverse{9} Der sagte zu seinem Volk: »Seht,
das Volk der Israeliten wird uns zu zahlreich und zu stark.
\bibleverse{10} Wohlan, wir wollen klug gegen sie zu Werke gehen, damit
ihrer nicht noch mehr werden; sonst könnte es geschehen, daß, wenn ein
Krieg ausbräche, sie sich auch noch zu unsern Feinden schlügen und gegen
uns kämpften und aus dem Lande wegzögen.« \bibleverse{11} So setzten sie
denn Fronvögte über das Volk, um es mit den Fronarbeiten, die sie ihm
auferlegten, zu bedrücken; und es mußte für den Pharao Vorratsstädte
bauen, nämlich Pithom und Ramses. \bibleverse{12} Aber je mehr man das
Volk bedrückte, desto zahlreicher wurde es und desto mehr breitete es
sich aus, so daß die Ägypter ein Grauen vor den Israeliten empfanden.
\bibleverse{13} Daher zwangen die Ägypter die Israeliten gewaltsam zum
Knechtsdienst \bibleverse{14} und verleideten ihnen das Leben durch
harte Fronarbeit in Lehm- und Ziegelsteinen und durch allerlei
Feldarbeit, lauter Dienstleistungen, die sie zwangsweise von ihnen
verrichten ließen.

\hypertarget{das-gottesfuxfcrchtige-verhalten-der-beiden-hebruxe4ischen-hebammen}{%
\paragraph{Das gottesfürchtige Verhalten der beiden hebräischen
Hebammen}\label{das-gottesfuxfcrchtige-verhalten-der-beiden-hebruxe4ischen-hebammen}}

\bibleverse{15} Da erteilte der König von Ägypten den hebräischen
Hebammen, von denen die eine Siphra, die andere Pua hieß, folgenden
Befehl: \bibleverse{16} »Wenn ihr den Hebräerinnen bei der Geburt Hilfe
leistet, so gebt bei der Entbindung wohl acht: wenn das Kind ein Knabe
ist, so tötet ihn! ist es aber ein Mädchen, so mag es am Leben bleiben!«
\bibleverse{17} Aber die Hebammen waren gottesfürchtig und befolgten den
Befehl des Königs von Ägypten nicht, sondern ließen die Knaben am Leben.
\bibleverse{18} Da rief der König von Ägypten die Hebammen zu sich und
fragte sie: »Warum verfahrt ihr so und laßt die Knaben am Leben?«
\bibleverse{19} Die Hebammen antworteten dem Pharao: »Ja, die
hebräischen Frauen sind nicht so (schwächlich) wie die ägyptischen,
sondern haben eine kräftige Natur; ehe noch die Hebamme zu ihnen kommt,
haben sie schon geboren.« \bibleverse{20} Gott aber ließ es den Hebammen
gut ergehen. So vermehrte sich denn das Volk stark und wurde sehr
zahlreich; \bibleverse{21} und weil die Hebammen gottesfürchtig waren,
verlieh Gott ihnen reichen Kindersegen. \bibleverse{22} Da befahl der
Pharao seinem ganzen Volke: »Jeden neugeborenen Knaben (der Hebräer)
werft in den Nil, alle Mädchen aber laßt am Leben!«

\hypertarget{geburt-moses-seine-schicksale-bis-zu-seiner-verheiratung}{%
\subsubsection{2. Geburt Moses; seine Schicksale bis zu seiner
Verheiratung}\label{geburt-moses-seine-schicksale-bis-zu-seiner-verheiratung}}

\hypertarget{a-geburt-und-aussetzung-rettung-und-erziehung-moses}{%
\paragraph{a) Geburt und Aussetzung, Rettung und Erziehung
Moses}\label{a-geburt-und-aussetzung-rettung-und-erziehung-moses}}

\hypertarget{section-1}{%
\section{2}\label{section-1}}

\bibleverse{1} Nun ging ein Mann aus dem Stamme Levi hin und heiratete
eine Levitin. \bibleverse{2} Diese Frau wurde Mutter eines Sohnes; und
als sie sah, daß es ein schönes Kind war, verbarg sie ihn drei Monate
lang. \bibleverse{3} Als sie ihn dann nicht länger verborgen halten
konnte, nahm sie für ihn ein Kästchen von Papyrusrohr, machte es mit
Erdharz und Pech dicht, legte das Knäblein hinein und setzte es in das
Schilf am Ufer des Nils. \bibleverse{4} Seine Schwester aber mußte sich
in einiger Entfernung hinstellen, um zu sehen, was mit ihm geschehen
würde. \bibleverse{5} Da kam die Tochter des Pharaos an den Nil hinab,
um zu baden, während ihre Dienerinnen am Ufer des Stromes hin und her
gingen. Da erblickte sie das Kästchen mitten im Schilf und ließ es durch
ihre Leibmagd holen. \bibleverse{6} Als sie es dann öffnete, siehe, da
lag ein weinendes Knäblein darin! Da fühlte sie Mitleid mit ihm und
sagte\textless sup title=``oder: dachte''\textgreater✲: »Das ist eins
von den Kindern der Hebräer.« \bibleverse{7} Da fragte seine Schwester
die Tochter des Pharaos: »Soll ich hingehen und dir eine Amme von den
Hebräerinnen holen, damit sie dir den Knaben nährt?« \bibleverse{8} Die
Tochter des Pharaos antwortete ihr: »Ja, gehe hin!« Da ging das Mädchen
hin und holte die Mutter des Kindes. \bibleverse{9} Die Tochter des
Pharaos sagte zu dieser: »Nimm dieses Knäblein mit und nähre es mir! Ich
will dir den Lohn dafür geben.« So nahm denn die Frau das Knäblein und
nährte es. \bibleverse{10} Als der Knabe dann größer geworden war,
brachte sie ihn der Tochter des Pharaos; die nahm ihn als Sohn an und
gab ihm den Namen Mose; »denn«, sagte sie, »ich habe ihn aus dem Wasser
gezogen«.

\hypertarget{b-moses-uxfcbeltat-totschlag-als-erster-beweis-seiner-liebe-zu-seinem-volk}{%
\paragraph{b) Moses Übeltat (Totschlag) als erster Beweis seiner Liebe
zu seinem
Volk}\label{b-moses-uxfcbeltat-totschlag-als-erster-beweis-seiner-liebe-zu-seinem-volk}}

\bibleverse{11} Zu jener Zeit nun, als Mose zum Mann geworden war, ging
er (einmal) zu seinen Volksgenossen hinaus und sah ihren Fronarbeiten
zu. Da sah er, wie ein Ägypter einen Hebräer, einen von seinen
Volksgenossen, schlug. \bibleverse{12} Da blickte er sich nach allen
Seiten um, und als er sah, daß kein Mensch sonst zugegen war, erschlug
er den Ägypter und verscharrte ihn im Sand. \bibleverse{13} Am folgenden
Tage ging er wieder hinaus und sah, wie zwei Hebräer sich miteinander
zankten. Da sagte er zu dem, der im Unrecht war: »Warum schlägst du
deinen Volksgenossen?« \bibleverse{14} Der gab zur Antwort: »Wer hat
dich zum Obmann und Richter über uns bestellt? Willst du mich etwa auch
totschlagen, wie du den Ägypter totgeschlagen hast?« Da erschrak Mose,
denn er sagte sich: »So ist also die Sache doch ruchbar geworden!«

\hypertarget{c-mose-flieht-nach-midian-und-heiratet-dort-zippora-die-tochter-des-priesters-reguel}{%
\paragraph{c) Mose flieht nach Midian und heiratet dort Zippora, die
Tochter des Priesters
Reguel}\label{c-mose-flieht-nach-midian-und-heiratet-dort-zippora-die-tochter-des-priesters-reguel}}

\bibleverse{15} Als nun auch der Pharao von dem Vorfall erfuhr und Mose
töten lassen wollte, floh Mose vor dem Pharao und nahm seinen Wohnsitz
im Lande Midian. Er hatte sich nämlich (nach seiner Ankunft dort) am
Brunnen niedergesetzt. \bibleverse{16} Nun hatte der Priester der
Midianiter sieben Töchter; die kamen und wollten Wasser schöpfen und die
Tränkrinnen füllen, um das Kleinvieh ihres Vaters zu tränken.
\bibleverse{17} Aber die Hirten kamen dazu und wollten sie wegdrängen.
Da erhob sich Mose, leistete ihnen Beistand und tränkte ihre Herde.
\bibleverse{18} Als sie nun zu ihrem Vater Reguel heimkamen, fragte er
sie: »Warum kommt ihr heute so früh heim?« \bibleverse{19} Sie
antworteten: »Ein ägyptischer Mann hat uns gegen die Hirten in Schutz
genommen, ja, er hat sogar das Schöpfen für uns besorgt und die Herde
getränkt.« \bibleverse{20} Da sagte er zu seinen Töchtern: »Und wo ist
er? Warum habt ihr denn den Mann dort draußen gelassen? Ladet ihn doch
zum Essen ein!« \bibleverse{21} Mose entschloß sich dann, bei dem Manne
zu bleiben, und dieser gab ihm seine Tochter Zippora zur Frau.
\bibleverse{22} Als sie ihm einen Sohn gebar, gab er ihm den Namen
Gersom\textless sup title=``d.h. Gast der Fremde?''\textgreater✲;
»denn«, sagte er, »ein Gast bin ich in einem fremden Lande geworden«.

\hypertarget{d-gott-huxf6rt-die-wehklagen-der-bedruxfcckten-israeliten}{%
\paragraph{d) Gott hört die Wehklagen der bedrückten
Israeliten}\label{d-gott-huxf6rt-die-wehklagen-der-bedruxfcckten-israeliten}}

\bibleverse{23} Es begab sich dann während jener langen Zeit, daß der
König von Ägypten starb. Die Israeliten aber seufzten unter dem
Frondienst und schrien auf, und ihr Hilferuf wegen des Frondienstes
stieg zu Gott empor. \bibleverse{24} Als Gott nun ihr Wehklagen hörte,
gedachte er seines Bundes mit Abraham, Isaak und Jakob; \bibleverse{25}
und Gott sah die Israeliten an, und Gott nahm Kenntnis davon.

\hypertarget{die-berufung-moses-am-gottesberge-horeb-seine-ruxfcckkehr-nach-uxe4gypten}{%
\subsubsection{3. Die Berufung Moses am Gottesberge Horeb; seine
Rückkehr nach
Ägypten}\label{die-berufung-moses-am-gottesberge-horeb-seine-ruxfcckkehr-nach-uxe4gypten}}

\hypertarget{a-gott-offenbart-sich-dem-mose-im-dornbusch-als-der-ich-bin-der-sein-volk-aus-der-uxe4gyptischen-knechtschaft-befreien-wolle}{%
\paragraph{a) Gott offenbart sich dem Mose im Dornbusch als der ›Ich
bin‹, der sein Volk aus der ägyptischen Knechtschaft befreien
wolle}\label{a-gott-offenbart-sich-dem-mose-im-dornbusch-als-der-ich-bin-der-sein-volk-aus-der-uxe4gyptischen-knechtschaft-befreien-wolle}}

\hypertarget{section-2}{%
\section{3}\label{section-2}}

\bibleverse{1} Mose aber weidete das Kleinvieh seines Schwiegervaters
Jethro, des Priesters der Midianiter. Als er nun einst die Herde über
die Steppe hinaus getrieben hatte, kam er an den Berg Gottes, an den
Horeb. \bibleverse{2} Da erschien ihm der Engel des HERRN als eine
Feuerflamme, die mitten aus einem Dornbusch hervorschlug; und als er
hinblickte, sah er, daß der Dornbusch im Feuer brannte, ohne jedoch vom
Feuer verzehrt zu werden. \bibleverse{3} Da dachte Mose: »Ich will doch
hingehen und mir diese wunderbare Erscheinung ansehen, warum der
Dornbusch nicht verbrennt.« \bibleverse{4} Als nun der HERR sah, daß er
herankam, um nachzusehen, rief Gott ihm aus dem Dornbusch heraus die
Worte zu: »Mose, Mose!« Er antwortete: »Hier bin ich!« \bibleverse{5} Da
sagte er: »Tritt nicht näher heran! Ziehe dir die Schuhe aus von den
Füßen; denn die Stätte, auf der du stehst, ist heiliger Boden.«
\bibleverse{6} Dann fuhr er fort: »Ich bin der Gott deines Vaters, der
Gott Abrahams, der Gott Isaaks und der Gott Jakobs.« Da verhüllte Mose
sein Gesicht; denn er fürchtete sich, Gott anzuschauen. \bibleverse{7}
Hierauf sagte der HERR: »Ich habe das Elend meines Volkes in Ägypten
gesehen und ihr Geschrei über ihre Fronvögte gehört; ja, ich kenne ihre
Leiden! \bibleverse{8} Daher bin ich herabgekommen, um sie aus der
Gewalt der Ägypter zu erretten und sie aus jenem Lande in ein schönes,
geräumiges Land zu führen, in ein Land, das von Milch und Honig
überfließt, in die Wohnsitze der Kanaanäer, Hethiter, Amoriter,
Pherissiter, Hewiter und Jebusiter. \bibleverse{9} Weil also jetzt das
Wehgeschrei der Israeliten zu mir gedrungen ist und ich auch gesehen
habe, wie schwer die Ägypter sie bedrücken, \bibleverse{10} so gehe
jetzt hin! Denn ich will dich zum Pharao senden, damit du mein Volk, die
Israeliten, aus Ägypten hinausführst.« \bibleverse{11} Da sagte Mose zu
Gott: »Wer bin ich, daß ich zum Pharao gehen und die Israeliten aus
Ägypten hinausführen sollte?« \bibleverse{12} Er antwortete: »Ich selbst
werde mit dir sein! Und dies soll dir das Wahrzeichen dafür sein, daß
ich dich gesandt habe: Wenn du das Volk aus Ägypten
wegführst\textless sup title=``oder: weggeführt hast''\textgreater✲,
werdet ihr an diesem Berge Gott dienen\textless sup title=``=~Gott
verehren''\textgreater✲.«

\hypertarget{die-offenbarung-des-gottesnamens}{%
\paragraph{Die Offenbarung des
Gottesnamens}\label{die-offenbarung-des-gottesnamens}}

\bibleverse{13} Da sagte Mose zu Gott: »Wenn ich nun aber zu den
Israeliten komme und ihnen sage: ›Der Gott eurer Väter hat mich zu euch
gesandt‹, und wenn sie mich dann fragen: ›Wie heißt er denn?‹, was soll
ich ihnen dann antworten?« \bibleverse{14} Da sagte Gott zu Mose: »Ich
bin, der ich bin.« Dann fuhr er fort: »So sollst du zu den Israeliten
sagen: Der ›Ich bin‹ hat mich zu euch gesandt!« \bibleverse{15} Und
weiter sagte Gott zu Mose: »So sollst du zu den Israeliten sagen: ›Der
HERR, der Gott eurer Väter, der Gott Abrahams, der Gott Isaaks und der
Gott Jakobs, hat mich zu euch gesandt.‹ Das ist mein Name in Ewigkeit
und meine Benennung von Geschlecht zu Geschlecht.«

\hypertarget{gottes-auftrag-und-verheiuxdfung-an-mose}{%
\paragraph{Gottes Auftrag und Verheißung an
Mose}\label{gottes-auftrag-und-verheiuxdfung-an-mose}}

\bibleverse{16} »Gehe hin und versammle die Ältesten der Israeliten und
sage zu ihnen: ›Der HERR, der Gott eurer Väter, ist mir erschienen, der
Gott Abrahams, Isaaks und Jakobs, und hat gesagt: Ich habe auf euch und
auf das, was euch in Ägypten widerfahren ist, genau achtgegeben
\bibleverse{17} und habe beschlossen, euch aus dem Elend Ägyptens in das
Land der Kanaanäer, Hethiter, Amoriter, Pherissiter, Hewiter und
Jebusiter wegzuführen, in ein Land, das von Milch und Honig überfließt.‹
\bibleverse{18} Wenn sie dann auf dich hören, sollst du mit den Ältesten
der Israeliten zum König von Ägypten hingehen, und ihr sollt zu ihm
sagen: ›Der HERR, der Gott der Hebräer, ist uns erschienen; und nun
möchten wir drei Tagereisen weit in die Wüste ziehen, um dort dem HERRN,
unserm Gott, zu opfern.‹ \bibleverse{19} Ich weiß aber, daß der König
von Ägypten euch nicht wird ziehen lassen, wenn er nicht durch eine
starke Hand\textless sup title=``=~mit Gewalt''\textgreater✲ dazu
gezwungen wird. \bibleverse{20} Darum werde ich dann meine Hand
ausstrecken und das Ägyptervolk mit all meinen Wundertaten schlagen, die
ich in seiner Mitte verrichten werde; daraufhin wird er euch ziehen
lassen. \bibleverse{21} Auch will ich dieses Volk bei den Ägyptern Gunst
finden lassen, so daß ihr bei eurem Auszug nicht mit leeren Händen
ausziehen sollt, \bibleverse{22} nein, jede Frau soll sich von ihrer
Nachbarin und ihrer Hausgenossin silberne und goldene Schmucksachen und
Kleider geben lassen; die sollt ihr dann euren Söhnen und Töchtern
anlegen und so die Ägypter ausplündern.«

\hypertarget{b-nach-anfuxe4nglichem-widerstreben-versteht-sich-mose-zur-ausfuxfchrung-des-guxf6ttlichen-auftrages}{%
\paragraph{b) Nach anfänglichem Widerstreben versteht sich Mose zur
Ausführung des göttlichen
Auftrages}\label{b-nach-anfuxe4nglichem-widerstreben-versteht-sich-mose-zur-ausfuxfchrung-des-guxf6ttlichen-auftrages}}

\hypertarget{aa-die-beglaubigungswunder}{%
\subparagraph{aa) Die
Beglaubigungswunder}\label{aa-die-beglaubigungswunder}}

\hypertarget{section-3}{%
\section{4}\label{section-3}}

\bibleverse{1} Mose aber entgegnete: »Ach, sie werden mir nicht glauben
und auf meine Aussagen nicht hören, sondern behaupten: ›Der HERR ist dir
nicht erschienen!‹« \bibleverse{2} Da erwiderte ihm der HERR: »Was hast
du da in deiner Hand?« Er antwortete: »Einen Stab.« \bibleverse{3} Da
sagte er: »Wirf ihn auf die Erde!« Als er ihn nun auf die Erde geworfen
hatte, wurde er zu einer Schlange, vor welcher Mose die Flucht ergriff.
\bibleverse{4} Da sagte der HERR zu Mose: »Strecke deine Hand aus und
ergreif sie beim Schwanz!« Er streckte seine Hand aus und faßte sie: da
wurde sie wieder zum Stab in seiner Hand~-- \bibleverse{5} »damit sie
glauben, daß dir der HERR erschienen ist, der Gott ihrer Väter, der Gott
Abrahams, der Gott Isaaks und der Gott Jakobs.« \bibleverse{6} Weiter
sagte der HERR zu ihm: »Stecke deine Hand in deinen Busen!« Er steckte
seine Hand in den Busen, und als er sie wieder herauszog, war seine Hand
vom Aussatz weiß wie Schnee. \bibleverse{7} Dann sagte er: »Stecke deine
Hand noch einmal in deinen Busen!« Als er es getan hatte und die Hand
dann wieder aus seinem Busen hervorzog, da war sie wieder wie sein
übriges Fleisch geworden. \bibleverse{8} »Wenn sie dir also nicht
glauben und sich von dem ersten Zeichen nicht überzeugen lassen, so
werden sie doch auf das zweite Zeichen hin glauben. \bibleverse{9}
Sollten sie aber selbst auf diese beiden Zeichen hin nicht glauben und
auf deine Aussagen nicht hören, so nimm etwas Wasser aus dem Nil und
schütte es auf den trockenen Boden, dann wird das Wasser, das du aus dem
Strom genommen hast, auf dem trocknen Boden zu Blut werden.«

\hypertarget{bb-neue-einwendungen-moses-bestellung-aarons-zum-sprecher}{%
\subparagraph{bb) Neue Einwendungen Moses; Bestellung Aarons zum
Sprecher}\label{bb-neue-einwendungen-moses-bestellung-aarons-zum-sprecher}}

\bibleverse{10} Mose aber sagte zum HERRN: »Bitte, HERR! Ich bin kein
Mann, der zu reden versteht; ich bin es früher nicht gewesen und bin es
auch jetzt nicht, seitdem du zu deinem Knecht redest, sondern ich bin
mit Mund und Zunge unbeholfen.« \bibleverse{11} Da antwortete ihm der
HERR: »Wer hat dem Menschen den Mund geschaffen, oder wer macht ihn
stumm oder taub, sehend oder blind? Bin ich es nicht, der HERR?
\bibleverse{12} So gehe also hin! Ich will schon mit deinem Munde sein
und dich lehren, was du reden sollst.« \bibleverse{13} Doch er
antwortete: »Bitte, HERR! Sende lieber einen andern, wen du willst!«
\bibleverse{14} Da entbrannte der Zorn des HERRN gegen Mose, und er
sagte: »Ist nicht dein Bruder Aaron da, der Levit? Ich weiß, daß der
trefflich zu reden versteht; auch ist er schon im Begriff, dir
entgegenzugehen, und wenn er dich sieht, wird er sich herzlich freuen.
\bibleverse{15} Dann sollst du dich mit ihm besprechen und ihm die Worte
in den Mund legen; ich aber will mit deinem und mit seinem Munde sein
und euch angeben, was ihr zu tun habt. \bibleverse{16} Er soll also für
dich zum Volk reden, und zwar so, daß er für dich der Mund ist und du
für ihn an Gottes Statt bist. \bibleverse{17} Und den Stab da nimm in
die Hand, um mit ihm die Wunderzeichen zu tun!«

\hypertarget{c-ruxfcckkehr-moses-nach-uxe4gypten}{%
\paragraph{c) Rückkehr Moses nach
Ägypten}\label{c-ruxfcckkehr-moses-nach-uxe4gypten}}

\hypertarget{aa-moses-abschied-von-seinem-schwiegervater-jethro-gottes-weisung}{%
\subparagraph{aa) Moses Abschied von seinem Schwiegervater Jethro;
Gottes
Weisung}\label{aa-moses-abschied-von-seinem-schwiegervater-jethro-gottes-weisung}}

\bibleverse{18} Hierauf kehrte Mose zu seinem Schwiegervater Jethro
zurück und sagte zu ihm: »Ich möchte doch einmal zu meinen Angehörigen
nach Ägypten zurückkehren, um zu sehen, ob sie noch am Leben sind.«
Jethro antwortete ihm: »Ziehe hin in Frieden!« \bibleverse{19} Da sagte
der HERR zu Mose im Midianiterlande: »Kehre nunmehr nach Ägypten zurück;
denn alle die Leute, die dir nach dem Leben getrachtet haben, sind tot.«
\bibleverse{20} So nahm denn Mose seine Frau und seine Söhne, setzte sie
auf Esel und trat die Rückkehr nach Ägypten an; den Gottesstab aber nahm
er in die Hand. \bibleverse{21} Da sagte der HERR zu Mose: »Wenn du
jetzt nach Ägypten zurückkommst, so sieh wohl zu, daß du alle die
Wunderzeichen, deren Vollführung ich dir aufgetragen habe, vor dem
Pharao verrichtest! Ich aber werde sein Herz verhärten, daß er das Volk
nicht ziehen läßt. \bibleverse{22} Dann sollst du zum Pharao sagen: ›So
hat der HERR gesprochen: Israel ist mein erstgeborener Sohn;
\bibleverse{23} daher fordere ich dich auf: Laß meinen Sohn ziehen,
damit er mir diene! Weigerst du dich aber, ihn ziehen zu lassen, so
werde ich deinen erstgeborenen Sohn sterben lassen!‹«

\hypertarget{bb-der-geheimnisvolle-vorgang-in-der-nachtherberge}{%
\subparagraph{bb) Der geheimnisvolle Vorgang in der
Nachtherberge}\label{bb-der-geheimnisvolle-vorgang-in-der-nachtherberge}}

\bibleverse{24} Unterwegs aber, in der Nachtherberge, überfiel der HERR
den Mose und wollte ihn töten. \bibleverse{25} Da nahm Zippora einen
scharfen Stein, schnitt damit die Vorhaut ihres Sohnes ab, warf sie ihm
vor die Füße und sagte: »Ein Blutbräutigam bist du mir!« \bibleverse{26}
Da ließ er von ihm ab. Damals sagte sie ›Blutbräutigam‹ im Hinblick auf
die Beschneidung.

\hypertarget{cc-mose-und-aaron-finden-in-uxe4gypten-glauben-bei-den-israeliten}{%
\subparagraph{cc) Mose und Aaron finden in Ägypten Glauben bei den
Israeliten}\label{cc-mose-und-aaron-finden-in-uxe4gypten-glauben-bei-den-israeliten}}

\bibleverse{27} Der HERR aber hatte dem Aaron geboten: »Gehe Mose
entgegen nach der Wüste zu!« Da machte er sich auf und traf ihn am Berge
Gottes und küßte ihn. \bibleverse{28} Mose teilte nun dem Aaron alles
mit, was der HERR ihm bei der Sendung aufgetragen, und alle
Wunderzeichen, die er ihm geboten hatte. \bibleverse{29} Darauf gingen
Mose und Aaron hin und versammelten alle Ältesten der Israeliten;
\bibleverse{30} und Aaron teilte ihnen alles mit, was der HERR dem Mose
aufgetragen hatte, und dieser verrichtete die Wunderzeichen vor den
Augen des Volkes. \bibleverse{31} Da schenkte ihm das Volk Glauben, und
als sie hörten, daß der HERR sich der Israeliten gnädig angenommen und
ihr Elend angesehen habe, verneigten sie sich und warfen sich zur Erde
nieder.

\hypertarget{die-erste-erfolglose-verhandlung-mit-dem-pharao-huxe4rtere-bedruxfcckung-israels}{%
\subsubsection{4. Die erste erfolglose Verhandlung mit dem Pharao;
härtere Bedrückung
Israels}\label{die-erste-erfolglose-verhandlung-mit-dem-pharao-huxe4rtere-bedruxfcckung-israels}}

\hypertarget{a-die-verhandlung}{%
\paragraph{a) Die Verhandlung}\label{a-die-verhandlung}}

\hypertarget{section-4}{%
\section{5}\label{section-4}}

\bibleverse{1} Hierauf gingen Mose und Aaron hin und sagten zum Pharao:
»So hat der HERR, der Gott Israels, gesprochen: ›Laß mein Volk ziehen,
damit sie mir ein Fest in der Wüste feiern!‹« \bibleverse{2} Der Pharao
aber antwortete: »Wer ist der HERR, daß ich seinen Befehlen gehorchen
und Israel ziehen lassen müßte? Ich kenne (diesen) HERRN nicht und will
auch Israel nicht ziehen lassen.« \bibleverse{3} Da entgegneten sie:
»Der Gott der Hebräer ist uns erschienen; wir möchten nun drei
Tagereisen weit in die Wüste ziehen und dem HERRN, unserm Gott, dort
Schlachtopfer darbringen, damit er uns nicht mit der Pest oder mit dem
Schwert heimsucht!« \bibleverse{4} Aber der König von Ägypten erwiderte
ihnen: »Warum wollt ihr, Mose und Aaron, das Volk von seiner Arbeit
abziehen? Geht an eure Frondienste!« \bibleverse{5} Dann fuhr der Pharao
fort: »Es gibt schon genug Gesindel im Land; und da wollt ihr sie noch
von ihren Frondiensten feiern lassen?!«

\hypertarget{b-das-volk-wird-noch-huxe4rter-bedruxfcckt-die-israelitischen-aufseher-vom-pharao-abgewiesen-machen-mose-und-aaron-bittere-vorwuxfcrfe}{%
\paragraph{b) Das Volk wird noch härter bedrückt; die israelitischen
Aufseher, vom Pharao abgewiesen, machen Mose und Aaron bittere
Vorwürfe}\label{b-das-volk-wird-noch-huxe4rter-bedruxfcckt-die-israelitischen-aufseher-vom-pharao-abgewiesen-machen-mose-und-aaron-bittere-vorwuxfcrfe}}

\bibleverse{6} An demselben Tage erteilte dann der Pharao den Fronvögten
und Aufsehern des Volkes den Befehl: \bibleverse{7} »Ihr sollt dem Volk
nicht mehr wie bisher Stroh\textless sup title=``oder:
Häckerling''\textgreater✲ zur Anfertigung der Ziegel liefern! Sie sollen
selbst hingehen und sich Stroh zusammensuchen! \bibleverse{8} Dabei
sollt ihr ihnen aber dieselbe Zahl von Ziegeln, die sie bisher gefertigt
haben, auferlegen, ohne etwas davon zu erlassen! Denn sie sind träge;
darum schreien sie immerfort: ›Wir wollen hinziehen und unserm Gott
Opfer darbringen!‹ \bibleverse{9} Die Arbeit soll den Leuten erschwert
werden, damit sie daran zu schaffen haben und nicht auf Lügenreden
achten!« \bibleverse{10} Da gingen die Fronvögte und Aufseher des Volkes
hinaus und sagten zum Volk: »So hat der Pharao befohlen: ›Ich lasse euch
hinfort kein Stroh mehr liefern: \bibleverse{11} geht selbst hin und
holt euch Stroh, wo ihr es findet! Doch von eurer Arbeit wird euch
nichts erlassen.‹«

\bibleverse{12} Da zerstreute sich das Volk im ganzen Lande Ägypten, um
Stoppeln zu sammeln zu Häckerling; \bibleverse{13} die Fronvögte aber
drängten sie mit der Forderung: »Ihr müßt Tag für Tag die volle Arbeit
leisten wie früher, als es noch Stroh gab.« \bibleverse{14} Und die
israelitischen Aufseher, welche die Fronvögte des Pharaos über sie
gesetzt hatten, erhielten Stockschläge, und man sagte zu ihnen: »Warum
habt ihr weder gestern noch heute euren bestimmten Satz Ziegel
fertiggestellt wie früher?« \bibleverse{15} Da gingen die israelitischen
Aufseher hin und wehklagten beim Pharao mit den Worten: »Warum
behandelst du deine Knechte so? \bibleverse{16} Stroh wird deinen
Knechten nicht mehr geliefert, und doch heißt es: ›Schafft Ziegel!‹ Und
nun werden deine Knechte sogar geschlagen, und die Schuld wird auf dein
Volk geschoben!« \bibleverse{17} Er aber antwortete: »Träge seid ihr,
träge! Darum sagt ihr: ›Wir möchten hinziehen, um dem HERRN zu opfern.‹
\bibleverse{18} Und nun marsch an die Arbeit! Stroh wird euch nicht
geliefert, aber die festgesetzte Zahl von Ziegeln habt ihr zu liefern!«
\bibleverse{19} So sahen sich denn die israelitischen Aufseher in eine
üble Lage versetzt, nämlich (ihren Volksgenossen) sagen zu müssen: »Von
den Ziegeln, die ihr Tag für Tag zu liefern habt, dürft ihr keinen Abzug
machen!« \bibleverse{20} Als sie nun aus dem Palast des Pharaos
herauskamen, stießen sie auf Mose und Aaron, die auf sie warteten.
\bibleverse{21} Da sagten sie zu ihnen: »Der HERR möge es euch gedenken
und euch dafür richten✲, daß ihr uns beim Pharao und seinen Beamten ganz
verhaßt gemacht und ihnen das Schwert in die Hand gegeben habt, uns
umzubringen!«

\hypertarget{c-die-klage-moses-und-die-verheiuxdfung-gottes}{%
\paragraph{c) Die Klage Moses und die Verheißung
Gottes}\label{c-die-klage-moses-und-die-verheiuxdfung-gottes}}

\bibleverse{22} Da wandte sich Mose wieder an den HERRN und sagte:
»Herr! Warum läßt du diesem Volk solches Unheil widerfahren? Warum hast
du mich hergesandt? \bibleverse{23} Denn seitdem ich zum Pharao gegangen
bin, um in deinem Namen zu reden, hat er dies Volk erst recht
mißhandelt, und du hast zur Rettung deines Volkes nichts getan!«

\hypertarget{section-5}{%
\section{6}\label{section-5}}

\bibleverse{1} Da sagte der HERR zu Mose: »Jetzt sollst du sehen, was
ich mit dem Pharao machen werde: Durch eine starke Hand gezwungen, wird
er sie ziehen lassen, ja durch eine starke Hand gezwungen, wird er sie
aus seinem Lande wegtreiben!«

\hypertarget{nochmalige-berufung-moses-stammbaum-moses}{%
\subsubsection{5. Nochmalige Berufung Moses; Stammbaum
Moses}\label{nochmalige-berufung-moses-stammbaum-moses}}

\hypertarget{a-gott-offenbart-sich-mose-aufs-neue-bestuxe4tigt-das-bundesverhuxe4ltnis-und-verheiuxdft-die-rettung-des-volkes}{%
\paragraph{a) Gott offenbart sich Mose aufs neue, bestätigt das
Bundesverhältnis und verheißt die Rettung des
Volkes}\label{a-gott-offenbart-sich-mose-aufs-neue-bestuxe4tigt-das-bundesverhuxe4ltnis-und-verheiuxdft-die-rettung-des-volkes}}

\bibleverse{2} Da redete Gott mit Mose und sagte zu ihm: »Ich bin der
HERR. \bibleverse{3} Ich bin dem Abraham, Isaak und Jakob als ›der
allmächtige Gott‹ erschienen, aber mit\textless sup title=``oder:
unter''\textgreater✲ meinem Namen ›Gott der HERR‹ habe ich mich ihnen
nicht geoffenbart. \bibleverse{4} Auch habe ich meinen Bund mit ihnen
geschlossen, ihnen das Land Kanaan zu geben, das Land ihrer
Fremdlingschaft, in dem sie als Gäste\textless sup title=``oder:
Fremdlinge''\textgreater✲ sich aufgehalten haben. \bibleverse{5} Ich
habe auch die Klagen der Israeliten gehört, die von den Ägyptern
geknechtet werden, und habe meines Bundes\textless sup title=``=~meiner
Bundeszusage''\textgreater✲ gedacht. \bibleverse{6} Darum sage zu den
Israeliten: ›Ich bin der HERR und will euch von dem Druck der
Fronarbeiten der Ägypter frei machen und euch aus ihrem Zwangsdienst
erretten und euch erlösen mit hoch erhobenem Arm und mit gewaltigen
Strafgerichten. \bibleverse{7} Und ich will euch zu meinem Volk annehmen
und will euer Gott sein, und ihr sollt erkennen, daß ich der HERR, euer
Gott, bin, der euch vom Druck des Frondienstes der Ägypter frei macht.
\bibleverse{8} Ich will euch auch in das Land bringen, dessen Verleihung
ich dem Abraham, Isaak und Jakob durch einen feierlichen Eid zugesagt
habe, und will es euch zum erblichen Besitz geben, ich, der HERR!‹«

\hypertarget{b-mose-von-seinem-verzagten-volk-abgewiesen-erhuxe4lt-von-gott-neue-weisung}{%
\paragraph{b) Mose, von seinem verzagten Volk abgewiesen, erhält von
Gott neue
Weisung}\label{b-mose-von-seinem-verzagten-volk-abgewiesen-erhuxe4lt-von-gott-neue-weisung}}

\bibleverse{9} Mose berichtete dies den Israeliten; aber sie hörten
nicht auf ihn aus Kleinmut und wegen des harten Frondienstes.
\bibleverse{10} Da sagte Gott zu Mose: \bibleverse{11} »Gehe hin,
fordere den Pharao, den König von Ägypten, auf, die Israeliten aus
seinem Lande ziehen zu lassen!« \bibleverse{12} Aber Mose sprach sich
vor dem HERRN offen so aus: »Nicht einmal die Israeliten haben auf mich
gehört: wie sollte da der Pharao mich anhören, zumal da ich im Reden
ungewandt bin!« \bibleverse{13} Da redete der HERR mit Mose und Aaron
und ordnete sie ab an die Israeliten und an den Pharao, den König von
Ägypten, um die Israeliten aus Ägypten wegzuführen.

\hypertarget{c-stammbaum-aarons-und-moses}{%
\paragraph{c) Stammbaum Aarons und
Moses}\label{c-stammbaum-aarons-und-moses}}

\bibleverse{14} Dies sind ihre Familienhäupter: Die Söhne Rubens, des
erstgeborenen Sohnes Israels, waren: Hanoch und Pallu, Hezron und Karmi;
dies sind die Geschlechter Rubens. \bibleverse{15} Und die Söhne Simeons
waren: Jemuel, Jamin, Ohad, Jachin, Zohar und Saul, der Sohn der
Kanaanäerin; dies sind die Geschlechter Simeons. \bibleverse{16} Und
dies sind die Namen der Söhne Levis nach ihren Geschlechtern: Gerson,
Kehath und Merari; Levi aber wurde 137~Jahre alt. \bibleverse{17} Die
Söhne Gersons waren: Libni und Simei nach ihren Familien.
\bibleverse{18} Die Söhne Kehaths waren: Amram, Jizhar, Hebron und
Ussiel; Kehath aber wurde 133~Jahre alt. \bibleverse{19} Und die Söhne
Meraris waren: Mahli und Musi; das sind die Familien Levis nach ihren
Geschlechtern. \bibleverse{20} Amram aber heiratete seine Muhme✲
Jochebed; die gebar ihm Aaron und Mose; Amram wurde dann 137~Jahre alt.
\bibleverse{21} Die Söhne Jizhars aber waren: Korah, Nepheg und Sichri;
\bibleverse{22} und die Söhne Ussiels waren: Misael, Elzaphan und
Sithri. \bibleverse{23} Aaron aber heiratete Eliseba, die Tochter
Amminadabs, die Schwester Nahsons; die gebar ihm Nadab und Abihu,
Eleasar und Ithamar. \bibleverse{24} Und die Söhne Korahs waren: Assir,
Elkana und Abiasaph; dies sind die Familien der Korahiten.
\bibleverse{25} Eleasar aber, der Sohn Aarons, heiratete eine von den
Töchtern Putiels; die gebar ihm den Pinehas. Dies sind die Stammhäupter
der Leviten nach ihren Geschlechtern.~-- \bibleverse{26} Dieser Aaron
und dieser Mose sind es, denen der HERR geboten hatte: »Führt die
Israeliten aus dem Lande Ägypten hinaus nach ihren Heerscharen!«
\bibleverse{27} Diese sind es, die mit dem Pharao, dem König von
Ägypten, verhandelten, um die Israeliten aus Ägypten wegzuführen: dieser
Mose und dieser Aaron.

\hypertarget{d-die-neue-sendung-moses-und-des-zum-sprecher-bestellten-aaron-an-den-pharao}{%
\paragraph{d) Die neue Sendung Moses und des zum Sprecher bestellten
Aaron an den
Pharao}\label{d-die-neue-sendung-moses-und-des-zum-sprecher-bestellten-aaron-an-den-pharao}}

\bibleverse{28} Damals nun, als der HERR mit Mose im Lande Ägypten
redete, \bibleverse{29} sagte der HERR zu Mose folgendes: »Ich bin der
HERR! Vermelde dem Pharao, dem König von Ägypten, alles, was ich dir
sagen werde.« \bibleverse{30} Mose aber antwortete vor dem HERRN: »Ach,
ich bin im Reden ungewandt: wie sollte da der Pharao auf mich hören!«

\hypertarget{section-6}{%
\section{7}\label{section-6}}

\bibleverse{1} Da erwiderte der HERR dem Mose: »Siehe, ich mache dich
für den Pharao zu einem Gott, und dein Bruder Aaron soll dein Prophet
sein\textless sup title=``vgl. 4,16''\textgreater✲. \bibleverse{2} Du
sollst ihm alles sagen, was ich dir auftragen werde, doch dein Bruder
Aaron soll mit dem Pharao verhandeln, daß er die Israeliten aus seinem
Lande ziehen lasse. \bibleverse{3} Ich aber will das Herz des Pharaos
verhärten, um viele Zeichen und Wunder im Lande Ägypten zu verrichten.
\bibleverse{4} Wenn der Pharao nun auf euch nicht hört, so will ich
meine Hand an\textless sup title=``oder: auf''\textgreater✲ die Ägypter
legen und meine Heerscharen, mein Volk, die Israeliten, aus Ägypten
unter gewaltigen Strafgerichten wegführen. \bibleverse{5} Dann werden
die Ägypter zur Erkenntnis kommen, daß ich der HERR bin, wenn ich meine
Hand gegen die Ägypter ausgestreckt und die Israeliten aus ihrer Mitte
weggeführt habe.« \bibleverse{6} Da taten Mose und Aaron so, wie der
HERR ihnen geboten hatte, genau so taten sie. \bibleverse{7} Mose war
aber achtzig und Aaron dreiundachtzig Jahre alt, als sie mit dem Pharao
verhandelten.

\hypertarget{die-wunder-und-plagen-in-uxe4gypten-der-auszug-78-1316}{%
\subsubsection{6. Die Wunder und Plagen in Ägypten; der Auszug
(7,8-13,16)}\label{die-wunder-und-plagen-in-uxe4gypten-der-auszug-78-1316}}

\hypertarget{a-das-beglaubigungswunder-verwandlung-des-stabes-in-eine-schlange}{%
\paragraph{a) Das Beglaubigungswunder (Verwandlung des Stabes in eine
Schlange)}\label{a-das-beglaubigungswunder-verwandlung-des-stabes-in-eine-schlange}}

\bibleverse{8} Hierauf sagte der HERR zu Mose und zu Aaron:
\bibleverse{9} »Wenn der Pharao euch auffordert, ein Wunder zu eurer
Beglaubigung zu verrichten, so sollst du zu Aaron sagen: ›Nimm deinen
Stab und wirf ihn vor den Pharao hin!‹, dann wird er zu einer großen
Schlange werden.« \bibleverse{10} Da gingen Mose und Aaron zum Pharao
und taten so, wie der HERR ihnen geboten hatte: Aaron warf seinen Stab
vor den Pharao und dessen Hofleute hin, und er verwandelte sich in eine
große Schlange. \bibleverse{11} Aber der Pharao ließ auch seinerseits
die Weisen und Zauberer kommen, und auch sie, die ägyptischen
Zauberkünstler, taten dasselbe vermittels ihrer Geheimkünste:
\bibleverse{12} jeder warf seinen Stab hin, da verwandelten diese sich
in Schlangen; jedoch Aarons Stab verschlang ihre Stäbe. \bibleverse{13}
Aber das Herz des Pharaos blieb hart, so daß er nicht auf sie hörte, wie
der HERR es vorausgesagt hatte.

\hypertarget{b-die-erste-plage-verwandlung-des-nilwassers-in-blut}{%
\paragraph{b) Die erste Plage: Verwandlung des Nilwassers in
Blut}\label{b-die-erste-plage-verwandlung-des-nilwassers-in-blut}}

\bibleverse{14} Hierauf sagte der HERR zu Mose: »Das Herz des Pharaos
ist verstockt: er weigert sich, das Volk ziehen zu lassen.
\bibleverse{15} Begib dich morgen früh zum Pharao -- da geht er nämlich
an den Fluß -- und tritt ihm am Ufer des Nils entgegen; den Stab, der
sich in eine Schlange verwandelt hat, nimm in deine Hand \bibleverse{16}
und sage zu ihm: ›Der HERR, der Gott der Hebräer, hat mich zu dir
gesandt mit der Weisung: Laß mein Volk ziehen, damit es mir in der Wüste
diene! Doch du hast bisher nicht gehorchen wollen. \bibleverse{17} Daher
spricht der HERR so: Daran sollst du erkennen, daß ich der HERR bin: ich
werde jetzt mit dem Stabe, den ich hier in der Hand habe, auf das Wasser
im Nil schlagen, dann wird es sich in Blut verwandeln, \bibleverse{18}
die Fische im Strom werden sämtlich sterben, und der Strom wird stinkend
werden, so daß die Ägypter vor Ekel kein Wasser mehr aus dem Strom
trinken werden.‹« \bibleverse{19} Weiter sagte der HERR zu Mose:
»Befiehl dem Aaron: ›Nimm deinen Stab und strecke deine Hand aus über
die Gewässer in Ägypten, über seine Stromarme, seine Kanäle und Teiche
und über alle seine Wasserbehälter, damit sie zu Blut werden! Und Blut
soll überall in Ägypten sein, selbst in den hölzernen und steinernen
Gefäßen!‹« \bibleverse{20} Mose und Aaron taten so, wie der HERR ihnen
geboten hatte: Aaron hob den Stab hoch und schlug mit ihm auf das Wasser
im Nil vor den Augen des Pharaos und seiner Diener: da verwandelte sich
alles Wasser im Strom in Blut; \bibleverse{21} die Fische im Strom
starben sämtlich, und der Strom wurde stinkend, so daß die Ägypter das
Wasser aus dem Strom nicht mehr trinken konnten; und das Blut war
überall im Land Ägypten. \bibleverse{22} Aber die ägyptischen Zauberer
taten dasselbe vermittels ihrer Geheimkünste; daher blieb das Herz des
Pharaos hart, und er hörte nicht auf sie, wie der HERR es vorausgesagt
hatte: \bibleverse{23} der Pharao wandte sich ab und ging nach Hause und
nahm sich auch dieses nicht zu Herzen. \bibleverse{24} Alle Ägypter aber
gruben rings um den Nil nach Trinkwasser; denn von dem Nilwasser konnten
sie nicht trinken. \bibleverse{25} So vergingen volle sieben Tage,
nachdem der HERR den Strom geschlagen hatte.

\hypertarget{c-die-zweite-plage-angedroht-und-verwirklicht-fruxf6sche}{%
\paragraph{c) Die zweite Plage angedroht und verwirklicht:
Frösche}\label{c-die-zweite-plage-angedroht-und-verwirklicht-fruxf6sche}}

\bibleverse{26} Hierauf gebot der HERR dem Mose: »Gehe zum Pharao und
sage zu ihm: ›So hat der HERR gesprochen: Laß mein Volk ziehen, damit es
mir diene! \bibleverse{27} Wenn du dich aber weigerst, es ziehen zu
lassen, so will ich dein ganzes Gebiet mit Fröschen heimsuchen.
\bibleverse{28} Der Nil soll dann von Fröschen wimmeln; die sollen
heraufkommen und in deinen Palast, in dein Schlafgemach und auf dein
Bett kriechen und in die Häuser deiner Diener und unter dein Volk, auch
in deine Backöfen und Backtröge dringen; \bibleverse{29} ja an dir
selbst und deinen Untertanen und an all deinen Dienern sollen die
Frösche hinaufkriechen!‹«

\hypertarget{section-7}{%
\section{8}\label{section-7}}

\bibleverse{1} Hierauf gebot der HERR dem Mose: »Sage zu Aaron: ›Strecke
deine Hand mit deinem Stabe aus über die Stromarme, die Kanäle und
Teiche, und laß die Frösche über das Land Ägypten heraufkommen!‹«
\bibleverse{2} Da streckte Aaron seine Hand über die Gewässer Ägyptens
aus, und die Frösche kamen herauf und bedeckten das Land Ägypten.
\bibleverse{3} Aber auch die Zauberer taten dasselbe vermittels ihrer
Geheimkünste: auch sie ließen die Frösche über das Land Ägypten kommen.

\bibleverse{4} Da ließ der Pharao Mose und Aaron kommen und sagte: »Legt
beim HERRN Fürbitte für mich ein, daß er die Frösche von mir und meinem
Volk entferne! Dann will ich das Volk ziehen lassen, damit es dem HERRN
opfert.« \bibleverse{5} Mose antwortete dem Pharao: »Verfüge über mich!
Auf wann soll ich für dich, für deine Diener und dein Volk die
Vertilgung der Frösche erbitten, damit sie von dir und aus deinen
Palästen verschwinden und nur noch im Nil verbleiben?« \bibleverse{6} Er
antwortete: »Auf morgen!« Da sagte Mose: »Wie du wünschest, so sei es!
Du sollst erkennen, daß niemand dem HERRN, unserm Gott, gleich ist.
\bibleverse{7} Die Frösche sollen also von dir und aus deinen Palästen,
von deinen Dienern und deinem Volk weichen; nur im Nil sollen sie
verbleiben!« \bibleverse{8} Als Mose und Aaron dann vom Pharao
weggegangen waren, betete Mose laut zum HERRN wegen der Frösche, mit
denen er den Pharao heimgesucht hatte. \bibleverse{9} Da tat der HERR
nach der Bitte Moses, so daß die Frösche in den Häusern, in den Gehöften
und auf den Feldern hinwegstarben; \bibleverse{10} man schüttete sie
überall in Haufen zusammen, und das Land stank davon. \bibleverse{11}
Als aber der Pharao merkte, daß er Luft✲ bekommen hatte, verstockte er
sein Herz weiter und hörte nicht auf sie, wie der HERR es vorausgesagt
hatte.

\hypertarget{d-dritte-und-vierte-plage-stechmuxfccken-und-hundsfliegen}{%
\paragraph{d) Dritte und vierte Plage: Stechmücken und
Hundsfliegen}\label{d-dritte-und-vierte-plage-stechmuxfccken-und-hundsfliegen}}

\bibleverse{12} Hierauf sagte der HERR zu Mose: »Befiehl dem Aaron:
›Strecke deinen Stab aus und schlage mit ihm den Staub auf dem Erdboden,
damit er sich in ganz Ägypten in Stechmücken verwandelt!‹«
\bibleverse{13} Und sie taten so: Aaron streckte seine Hand mit dem
Stabe aus und schlug damit den Staub auf dem Erdboden; da kamen die
Stechmücken an die Menschen und an das Vieh; aller Staub auf dem
Erdboden wurde zu Stechmücken in ganz Ägypten. \bibleverse{14} Die
ägyptischen Zauberer bemühten sich mit ihren Geheimkünsten ebenso,
Stechmücken hervorzubringen, vermochten es aber nicht; die Stechmücken
aber saßen an Menschen und Vieh. \bibleverse{15} Da sagten die Zauberer
zum Pharao: »Das ist eines Gottes Finger!« Doch das Herz des Pharaos
blieb hart, und er hörte nicht auf sie, wie der HERR es vorausgesagt
hatte.

\bibleverse{16} Hierauf gebot der HERR dem Mose: »Mache dich morgen in
der Frühe auf und tritt vor den Pharao hin, wenn er hinaus an den Fluß
geht, und sage zu ihm: ›So hat der HERR gesprochen: Laß mein Volk
ziehen, damit es mir diene! \bibleverse{17} Denn wenn du mein Volk nicht
ziehen läßt, so will ich Hundsfliegen über dich und deine Diener, über
dein Volk und deine Paläste kommen lassen, so daß die Häuser der Ägypter
und sogar der Erdboden, auf dem sie stehen, voll von Hundsfliegen sein
werden. \bibleverse{18} Aber an demselben Tage will ich das Land Gosen,
wo mein Volk wohnt, absondern, so daß es dort keine Hundsfliegen geben
soll, damit du erkennst, daß ich der HERR bin inmitten dieses Landes.
\bibleverse{19} Ich will also eine Scheidung zwischen meinem und deinem
Volk eintreten lassen: morgen soll dies Zeichen geschehen!‹«
\bibleverse{20} Und der HERR tat so: es kamen Hundsfliegen in gewaltiger
Menge in den Palast des Pharaos und in die Wohnungen seiner Diener und
über das ganze Land Ägypten, und das Land litt schwer unter den
Hundsfliegen. \bibleverse{21} Da ließ der Pharao Mose und Aaron rufen
und sagte: »Geht hin und opfert eurem Gott hier im Lande!«
\bibleverse{22} Da antwortete Mose: »Es geht nicht an, daß wir das tun;
denn wir bringen dem HERRN, unserm Gott, Opfer dar, die den Ägyptern ein
Greuel sind. Wenn wir nun vor den Augen der Ägypter Opfer darbrächten,
die ihnen ein Greuel sind, würden sie uns da nicht steinigen?
\bibleverse{23} Nein, drei Tagereisen weit wollen wir in die Wüste
ziehen und dem HERRN, unserm Gott, dort opfern, wie er uns geboten hat.«
\bibleverse{24} Da sagte der Pharao: »Ich will euch ziehen lassen, damit
ihr dem HERRN, eurem Gott, in der Wüste opfern könnt; nur entfernt euch
nicht zu weit und legt Fürbitte für mich ein!« \bibleverse{25} Mose
antwortete: »Sobald ich dich jetzt verlassen habe, will ich beim HERRN
Fürbitte einlegen, daß die Hundsfliegen morgen vom Pharao, von seinen
Dienern und seinem Volk verschwinden; nur möge dann der Pharao uns nicht
abermals täuschen, indem er das Volk doch nicht ziehen läßt, damit es
dem HERRN opfern kann!« \bibleverse{26} Als Mose hierauf vom Pharao
weggegangen war und zum HERRN gebetet hatte, \bibleverse{27} erfüllte
der HERR dem Mose seine Bitte: er ließ die Hundsfliegen vom Pharao, von
seinen Dienern und seinem Volk verschwinden, so daß keine einzige
übrigblieb. \bibleverse{28} Aber der Pharao verstockte sein Herz auch
diesmal und ließ das Volk nicht ziehen.

\hypertarget{e-fuxfcnfte-sechste-und-siebente-plage-viehpest-beulen-und-hagel}{%
\paragraph{e) Fünfte, sechste und siebente Plage: Viehpest, Beulen und
Hagel}\label{e-fuxfcnfte-sechste-und-siebente-plage-viehpest-beulen-und-hagel}}

\hypertarget{section-8}{%
\section{9}\label{section-8}}

\bibleverse{1} Hierauf sagte der HERR zu Mose: »Gehe zum Pharao und sage
zu ihm: ›So hat der HERR, der Gott der Hebräer, gesprochen: Laß mein
Volk ziehen, damit es mir diene! \bibleverse{2} Denn wenn du dich
weigerst, es ziehen zu lassen, und sie noch länger zurückhältst,
\bibleverse{3} so wird die Hand des HERRN über dein Vieh auf dem Felde
kommen, über die Pferde, Esel und Kamele, über die Rinder und das
Kleinvieh mit einer sehr schlimmen Seuche. \bibleverse{4} Der HERR wird
dabei aber einen Unterschied zwischen dem Vieh der Israeliten und dem
Vieh der Ägypter machen, so daß von dem gesamten Besitz der Israeliten
kein Stück fallen wird.‹« \bibleverse{5} Darauf setzte der HERR eine
bestimmte Zeit fest mit den Worten: »Morgen schon wird der HERR dies im
Lande geschehen lassen!« \bibleverse{6} Und am andern Tage ließ der HERR
dies wirklich eintreten: alles Vieh der Ägypter starb, während vom Vieh
der Israeliten kein einziges Stück fiel. \bibleverse{7} Als der Pharao
nämlich hinsandte, um nachzusehen, stellte es sich heraus, daß vom Vieh
der Israeliten kein einziges Stück gefallen war. Aber das Herz des
Pharaos blieb trotzdem verstockt, so daß er das Volk nicht ziehen ließ.

\bibleverse{8} Hierauf gebot der HERR dem Mose und Aaron: »Nehmt euch
eure beiden Hände voll Ofenruß, und Mose soll ihn vor den Augen des
Pharaos himmelwärts streuen! \bibleverse{9} Dann wird er sich als feiner
Staub über das ganze Land Ägypten verbreiten und an Menschen und am Vieh
zu Beulen\textless sup title=``oder: Blattern''\textgreater✲ werden, die
als Geschwüre aufbrechen, im ganzen Lande Ägypten!« \bibleverse{10} Da
nahmen sie Ofenruß und traten vor den Pharao, und Mose streute ihn
himmelwärts; da wurde er zu Beulen\textless sup title=``oder:
Blattern''\textgreater✲, die als Geschwüre an den Menschen und am Vieh
aufbrachen. \bibleverse{11} Die Zauberer aber konnten nicht vor Mose
treten wegen der Beulen; denn die Beulen waren an den Zauberern ebenso
wie an allen anderen Ägyptern aufgebrochen. \bibleverse{12} Doch der
HERR verhärtete das Herz des Pharaos, so daß er nicht auf sie hörte, wie
der HERR es dem Mose vorausgesagt hatte.~--

\hypertarget{die-siebte-plage-der-hagel}{%
\paragraph{Die siebte Plage: der
Hagel}\label{die-siebte-plage-der-hagel}}

\bibleverse{13} Hierauf gebot der HERR dem Mose: »Tritt morgen in der
Frühe vor den Pharao und sage zu ihm: ›So hat der HERR, der Gott der
Hebräer, gesprochen: Laß mein Volk ziehen, damit es mir diene!
\bibleverse{14} Denn diesmal will ich alle meine Plagen gegen dich
selbst sowie gegen deine Diener und dein Volk loslassen, damit du
erkennst, daß niemand mir gleichkommt auf der ganzen Erde!
\bibleverse{15} Denn schon jetzt hätte ich meine Hand ausstrecken und
dich samt deinem Volk mit der Pest schlagen können, so daß du von der
Erde vertilgt worden wärst; \bibleverse{16} aber ich habe dich
absichtlich leben lassen, um an dir meine Kraft zu erweisen und damit
mein Name auf der ganzen Erde gepriesen wird. \bibleverse{17} Wenn du
dich noch länger dagegen sträubst, mein Volk ziehen zu lassen,
\bibleverse{18} so will ich morgen um diese Zeit einen sehr schweren
Hagel niedergehen lassen, wie ein solcher nie zuvor in Ägypten dagewesen
ist vom Tage seiner Gründung an bis jetzt. \bibleverse{19} Sende also
hin und laß dein Vieh und alles, was du im Freien hast, in Sicherheit
bringen: denn alle Menschen und alle Tiere, die sich im Freien befinden
und nicht unter Dach und Fach gebracht worden sind, werden sterben, wenn
der Hagel auf sie niederfällt!‹« \bibleverse{20} Wer nun von den Leuten
des Pharaos die Drohung des HERRN fürchtete, der brachte seine Knechte
und sein Vieh unter Dach und Fach in Sicherheit; \bibleverse{21} wer
aber die Drohung des HERRN nicht beachtete, der ließ seine Knechte und
sein Vieh im Freien.

\bibleverse{22} Da gebot der HERR dem Mose: »Strecke deine Hand gen
Himmel aus, damit Hagel in ganz Ägypten falle auf Menschen und Vieh und
auf alles, was in Ägypten auf den Feldern gewachsen ist!«
\bibleverse{23} Als nun Mose seinen Stab gen Himmel ausstreckte, ließ
der HERR donnern und hageln, und Feuer fuhr zur Erde nieder, und der
HERR ließ Hagel auf Ägypten regnen; \bibleverse{24} mit dem Hagel aber
kamen unaufhörliche Blitze mitten in den Hagel hinein so furchtbar, wie
man etwas Derartiges in ganz Ägypten noch nicht erlebt hatte, seit es
von einem Volk bewohnt war. \bibleverse{25} Der Hagel erschlug in ganz
Ägypten alles, was sich im Freien befand, Menschen wie Tiere; auch alle
Feldgewächse zerschlug der Hagel und zerschmetterte alle Bäume auf dem
Felde. \bibleverse{26} Nur im Lande Gosen, wo die Israeliten wohnten,
fiel kein Hagel. \bibleverse{27} Da ließ der Pharao Mose und Aaron rufen
und sagte zu ihnen: »Diesmal habe ich mich versündigt\textless sup
title=``=~bekenne ich mich schuldig''\textgreater✲: der HERR ist im
Recht, ich aber und mein Volk sind im Unrecht! \bibleverse{28} Legt
Fürbitte beim HERRN ein; denn der Donnerschläge Gottes und des Hagels
ist nun mehr als genug: ich will euch ziehen lassen, und ihr sollt nicht
länger hier bleiben!« \bibleverse{29} Da antwortete ihm Mose: »Sobald
ich zur Stadt hinausgehe, will ich meine Hände zum HERRN ausbreiten;
dann werden die Donnerschläge aufhören, und kein Hagel wird mehr fallen,
damit du erkennst, daß die Erde dem HERRN gehört. \bibleverse{30} Aber
ich weiß wohl: du und deine Diener, ihr fürchtet euch immer noch nicht
vor Gott dem HERRN.« \bibleverse{31} Der Flachs und die Gerste waren
zerschlagen, denn die Gerste stand schon in Ähren und der Flachs in
Blüte; \bibleverse{32} aber der Weizen und der Spelt waren nicht
zerschlagen, weil sie spätzeitig sind. \bibleverse{33} Als Mose dann vom
Pharao hinweg aus der Stadt hinausgegangen war, breitete er seine Hände
zum HERRN aus; da hörten die Donnerschläge und der Hagel auf, und auch
der Regen strömte nicht mehr auf die Erde nieder. \bibleverse{34} Als
nun der Pharao sah, daß der Regen, der Hagel und der Donner aufgehört
hatten, fuhr er fort zu sündigen und verstockte sein Herz, er samt
seinen Dienern. \bibleverse{35} So blieb denn das Herz des Pharaos hart,
und er ließ die Israeliten nicht ziehen, wie der HERR es durch Mose
vorausgesagt hatte.

\hypertarget{f-die-achte-und-neunte-plage-heuschrecken-und-finsternis}{%
\paragraph{f) Die achte und neunte Plage: Heuschrecken und
Finsternis}\label{f-die-achte-und-neunte-plage-heuschrecken-und-finsternis}}

\hypertarget{section-9}{%
\section{10}\label{section-9}}

\bibleverse{1} Hierauf sagte der HERR zu Mose: »Gehe zum Pharao! Denn
ich selbst habe ihm und seinen Dienern das Herz verhärtet, um diese
meine Zeichen in ihrer Mitte zu verrichten, \bibleverse{2} und damit du
deinen Kindern und Kindeskindern einst erzählen kannst, wie ich gegen
die Ägypter vorgegangen bin und welche Zeichen ich unter ihnen vollführt
habe: erkennen sollt ihr, daß ich der HERR bin!« \bibleverse{3} Da
gingen Mose und Aaron zum Pharao und sagten zu ihm: »So hat der HERR,
der Gott der Hebräer, gesprochen: ›Wie lange willst du dich noch
sträuben, dich vor mir zu demütigen? Laß mein Volk ziehen, damit es mir
diene! \bibleverse{4} Denn wenn du dich weigerst, mein Volk ziehen zu
lassen, so will ich morgen Heuschrecken in dein Land kommen lassen;
\bibleverse{5} die werden die Oberfläche des Erdbodens so bedecken, daß
man den Erdboden nicht mehr wird sehen können, und sollen alles
auffressen, was von dem Hagelwetter verschont geblieben und euch noch
übriggelassen ist; sie sollen auch alle Bäume abfressen, die euch auf
dem Felde wachsen; \bibleverse{6} sie sollen auch deine Paläste und die
Häuser aller deiner Diener und die Häuser aller Ägypter anfüllen, wie es
deine Väter und die Väter deiner Väter, seitdem sie auf dem Erdboden
gewesen sind, bis auf den heutigen Tag nicht erlebt haben!‹« Damit
wandte er sich und verließ den Pharao.

\bibleverse{7} Da sagten die Diener des Pharaos zu ihm: »Wie lange soll
dieser Mensch uns noch unglücklich machen? Laß doch die Leute ziehen,
damit sie dem HERRN, ihrem Gott, dienen! Siehst du noch nicht ein, daß
Ägypten zugrunde gerichtet wird?« \bibleverse{8} Hierauf holte man Mose
und Aaron zum Pharao zurück, und er sagte zu ihnen: »Zieht hin und dient
dem HERRN, eurem Gott! Wer soll denn alles hinziehen?« \bibleverse{9} Da
antwortete Mose: »Mit jung und alt wollen wir hinausziehen, mit unsern
Söhnen und unsern Töchtern, mit unserm Kleinvieh und unsern Rindern
wollen wir hinausziehen; denn wir haben ein Fest des HERRN zu feiern.«
\bibleverse{10} Da antwortete er ihnen: »Möge der HERR ebenso mit euch
sein, wie ich euch mit Weib und Kind ziehen lasse! Seht ihr wohl, daß
ihr Böses im Sinn habt? \bibleverse{11} Daraus wird nichts! Ihr Männer
mögt hinziehen und dem HERRN dienen: das ist ja auch euer Begehr
gewesen!« Hierauf wies man sie vom Pharao weg.

\bibleverse{12} Da gebot der HERR dem Mose: »Strecke deine Hand über das
Land Ägypten aus, damit die Heuschrecken über das Land kommen und alle
Feldgewächse abfressen, alles, was der Hagel übriggelassen hat!«
\bibleverse{13} Da streckte Mose seinen Stab über das Land Ägypten aus,
und der HERR ließ einen Ostwind über das Land hin wehen jenen ganzen Tag
und die ganze Nacht; als es dann Morgen wurde, hatte der Ostwind die
Heuschrecken herbeigebracht. \bibleverse{14} So kamen denn die
Heuschrecken über das ganze Land Ägypten und ließen sich in allen Teilen
Ägyptens in gewaltiger Menge nieder; nie zuvor waren so viele
Heuschrecken dagewesen wie damals, und künftig wird es nie wieder so
viele geben. \bibleverse{15} Sie bedeckten die Oberfläche des ganzen
Landes, so daß der Erdboden nicht mehr zu sehen war, und sie fraßen alle
Feldgewächse ab und alle Baumfrüchte, die der Hagel übriggelassen hatte,
so daß nichts Grünes mehr an den Bäumen und an den Feldgewächsen im
ganzen Lande Ägypten übrigblieb.

\bibleverse{16} Da ließ der Pharao in aller Eile Mose und Aaron rufen
und sagte: »Ich habe mich am HERRN, eurem Gott, und an euch versündigt!
\bibleverse{17} Und nun vergib mir meine Verfehlung nur noch dies eine
Mal und betet zum HERRN, eurem Gott, daß er wenigstens dieses Verderben
von mir abwende!« \bibleverse{18} Als nun (Mose) vom Pharao weggegangen
war und zum HERRN gebetet hatte, \bibleverse{19} da wandte der HERR den
Wind, so daß er sehr stark aus dem Westen wehte; der hob die
Heuschrecken auf und warf sie ins Schilfmeer, so daß keine einzige
Heuschrecke im ganzen Bereich von Ägypten übrigblieb. \bibleverse{20}
Aber der HERR verhärtete das Herz des Pharaos, so daß er die Israeliten
nicht ziehen ließ.

\hypertarget{die-neunte-plage-finsternis}{%
\paragraph{Die neunte Plage:
Finsternis}\label{die-neunte-plage-finsternis}}

\bibleverse{21} Hierauf gebot der HERR dem Mose: »Strecke deine Hand gen
Himmel aus, damit eine Finsternis über das Land Ägypten komme, so dicht,
daß man sie greifen kann.« \bibleverse{22} Als nun Mose seine Hand gen
Himmel ausgestreckt hatte, entstand eine Finsternis im ganzen Land
Ägypten drei Tage lang. \bibleverse{23} Kein Mensch konnte den andern
sehen, und keiner erhob sich von seinem Platz drei Tage lang; aber die
Israeliten hatten alle hellen Tag in ihren Wohnsitzen. \bibleverse{24}
Da ließ der Pharao Mose rufen und sagte: »Zieht hin, dient dem HERRN!
Nur euer Kleinvieh und eure Rinder sollen hier zurückbleiben; auch eure
Frauen und Kinder mögen mit euch gehen!« \bibleverse{25} Da antwortete
Mose: »Nicht nur mußt du selbst uns Tiere zu Schlacht- und Brandopfern
mitgeben, damit wir sie dem HERRN, unserm Gott, darbringen,
\bibleverse{26} sondern auch unser Vieh muß mit uns ziehen: keine Klaue
darf zurückbleiben! Denn davon müssen wir Tiere zur Verehrung des HERRN,
unsers Gottes, nehmen; wir wissen ja nicht, was wir dem HERRN zu opfern
haben, ehe wir an Ort und Stelle sind.« \bibleverse{27} Aber der HERR
verhärtete das Herz des Pharaos, so daß er sie nicht ziehen lassen
wollte, \bibleverse{28} sondern zu Mose sagte: »Hinweg von mir! Hüte
dich, mir nochmals vor die Augen zu treten! Denn sobald du dich wieder
vor mir sehen läßt, bist du des Todes!« \bibleverse{29} Da antwortete
Mose: »Du hast recht geredet: ich werde dir nicht wieder vor die Augen
treten!«

\hypertarget{g-ankuxfcndigung-der-zehnten-letzten-und-entscheidenden-plage-des-sterbens-der-erstgeburten}{%
\paragraph{g) Ankündigung der zehnten (letzten und entscheidenden)
Plage, des Sterbens der
Erstgeburten}\label{g-ankuxfcndigung-der-zehnten-letzten-und-entscheidenden-plage-des-sterbens-der-erstgeburten}}

\hypertarget{section-10}{%
\section{11}\label{section-10}}

\bibleverse{1} Darauf sagte der HERR zu Mose: »Noch eine einzige Plage
will ich über den Pharao und über Ägypten kommen lassen; alsdann wird er
euch von hier ziehen lassen, ja er wird, wenn er euch bedingungslos
entläßt, euch sogar gewaltsam von hier wegtreiben. \bibleverse{2} Gib
nun dem Volke die bestimmte Weisung, daß sie sich insgesamt, Männer wie
Weiber, silberne und goldene Wertsachen von ihren Nachbarn und
Nachbarinnen erbitten\textless sup title=``oder: leihen''\textgreater✲.«
\bibleverse{3} Der HERR stimmte dann die Ägypter günstig gegen das Volk;
auch stand Mose in den Augen der Diener des Pharaos und des ganzen
ägyptischen Volkes als ein großer Mann da. \bibleverse{4} Hierauf sagte
Mose: »So hat der HERR gesprochen: ›Um Mitternacht will ich mitten durch
Ägypten schreiten; \bibleverse{5} da soll dann jede Erstgeburt in
Ägypten sterben, vom erstgeborenen Sohn des Pharaos an, der auf seinem
Thron sitzt, bis zum Erstgeborenen der Magd, die hinter der Handmühle
sitzt, auch alles Erstgeborene vom Vieh. \bibleverse{6} Da wird sich ein
großes Wehgeschrei im ganzen Land Ägypten erheben, wie ein solches noch
nie dagewesen ist und nie wieder stattfinden wird. \bibleverse{7} Aber
gegen keinen Israeliten, weder gegen einen Menschen noch gegen das Vieh,
soll auch nur ein Hund ein Knurren hören lassen, damit ihr erkennt, daß
der HERR eine Scheidung\textless sup title=``oder: einen
Unterschied''\textgreater✲ zwischen den Ägyptern und den Israeliten
macht.‹ \bibleverse{8} Dann werden alle diese deine Diener zu mir
herabkommen, sich vor mir niederwerfen und bitten: ›Ziehe weg, du und
das ganze Volk, das deiner Leitung folgt!‹, und danach werde ich
wegziehen.« Hierauf ging (Mose) vom Pharao weg in glühendem Zorn.

\hypertarget{zusammenfassender-schluuxdf-der-zeichen--und-wundergeschichte}{%
\paragraph{Zusammenfassender Schluß der Zeichen- und
Wundergeschichte}\label{zusammenfassender-schluuxdf-der-zeichen--und-wundergeschichte}}

\bibleverse{9} Der HERR hatte aber zu Mose gesagt: »Der Pharao wird
nicht auf euch hören, damit meine Wunder in Ägypten zahlreich werden.«
\bibleverse{10} So haben denn Mose und Aaron alle diese Wunder vor dem
Pharao vollführt; aber der HERR verstockte das Herz des Pharaos, so daß
er die Israeliten aus seinem Lande nicht ziehen ließ.

\hypertarget{h-einsetzung-des-passah}{%
\paragraph{h) Einsetzung des Passah}\label{h-einsetzung-des-passah}}

\hypertarget{aa-anordnungen-uxfcber-die-zubereitung-und-das-essen-des-passahlammes}{%
\subparagraph{aa) Anordnungen über die Zubereitung und das Essen des
Passahlammes}\label{aa-anordnungen-uxfcber-die-zubereitung-und-das-essen-des-passahlammes}}

\hypertarget{section-11}{%
\section{12}\label{section-11}}

\bibleverse{1} Darauf gebot der HERR dem Mose und Aaron im Lande Ägypten
folgendes: \bibleverse{2} »Der gegenwärtige Monat soll euch als
Anfangsmonat gelten! Der erste soll er euch unter den Monaten des Jahres
sein! \bibleverse{3} Gebt der ganzen Gemeinde Israel folgende Weisungen:
Am zehnten Tage dieses Monats, da nehme sich jeder (Hausvater) ein Lamm,
für je eine Familie✲ ein Lamm; \bibleverse{4} und wenn eine Familie zu
klein für ein ganzes Lamm ist, so nehme er und sein ihm zunächst
wohnender Nachbar eins gemeinsam nach der Zahl der Seelen! Ihr sollt auf
das Lamm so viele Personen rechnen, als zum Verzehren erforderlich sind!
\bibleverse{5} Es müssen fehlerlose, männliche, einjährige Lämmer sein;
von den Schafen oder von den Ziegen sollt ihr sie nehmen. \bibleverse{6}
Bis zum vierzehnten Tage dieses Monats sollt ihr sie in Verwahrung
haben; dann soll die gesamte Volksgemeinde Israel sie zwischen den
beiden Abenden\textless sup title=``d.h. zwischen Sonnenuntergang und
Dunkelwerden''\textgreater✲ schlachten! \bibleverse{7} Hierauf sollen
sie etwas von dem Blut nehmen und es an die beiden Türpfosten und an die
Oberschwelle an den Häusern streichen, in denen sie die Mahlzeit halten.
\bibleverse{8} Sie sollen dann das Fleisch noch in derselben Nacht
essen, und zwar am Feuer gebraten, und dazu ungesäuertes Brot; mit
bitteren Kräutern sollen sie es essen. \bibleverse{9} Ihr dürft nichts
davon roh oder im Wasser gekocht genießen, sondern am Feuer gebraten,
und zwar so, daß der Kopf noch mit den Beinen und mit dem Rumpf
zusammenhängt! \bibleverse{10} Ihr dürft nichts davon bis zum andern
Morgen übriglassen, sondern was etwa davon bis zum Morgen übrigbleibt,
sollt ihr im Feuer verbrennen. \bibleverse{11} Und auf folgende Weise
sollt ihr es essen: eure Hüften gegürtet, eure Schuhe✲ an den Füßen und
euren Stab in der Hand; und in ängstlicher Hast sollt ihr es essen: ein
Vorübergehen des HERRN ist es. \bibleverse{12} Denn ich will in dieser
Nacht durch das Land Ägypten schreiten und alle Erstgeburt in Ägypten
sterben lassen sowohl von den Menschen als vom Vieh, und ich will an
allen ägyptischen Göttern ein Strafgericht vollziehen, ich, der HERR!
\bibleverse{13} Dabei soll dann das Blut an den Häusern, in denen ihr
euch befindet, ein Zeichen zu eurem Schutz sein; denn wenn ich das Blut
sehe, will ich schonend an euch vorübergehen, und es soll euch kein
tödliches Verderben treffen, wenn ich den Schlag gegen das Land Ägypten
führe.«

\hypertarget{bb-anordnungen-uxfcber-die-siebentuxe4gige-feier-des-festes-der-ungesuxe4uerten-brote}{%
\subparagraph{bb) Anordnungen über die siebentägige Feier des Festes der
ungesäuerten
Brote}\label{bb-anordnungen-uxfcber-die-siebentuxe4gige-feier-des-festes-der-ungesuxe4uerten-brote}}

\bibleverse{14} »Dieser Tag soll dann für euch ein Gedächtnistag sein,
den ihr zu Ehren des HERRN festlich begehen sollt! Von Geschlecht zu
Geschlecht sollt ihr ihn als eine ewige Satzung feiern! \bibleverse{15}
Sieben Tage lang sollt ihr ungesäuertes Brot essen; gleich am ersten
Tage sollt ihr allen Sauerteig aus euren Häusern entfernen; denn jeder,
der vom ersten bis zum siebten Tage Gesäuertes ißt, ein solcher Mensch
soll aus Israel ausgerottet werden! \bibleverse{16} Weiter soll am
ersten Tage eine heilige Festversammlung bei euch stattfinden und ebenso
am siebten Tage eine heilige Festversammlung: keinerlei Arbeit darf an
diesen (beiden Tagen) verrichtet werden! Nur was ein jeder zum Essen
nötig hat, das allein darf von euch zubereitet werden! \bibleverse{17}
So beobachtet denn das Fest der ungesäuerten Brote! Denn an eben diesem
Tage habe ich eure Heerscharen aus dem Land Ägypten hinausgeführt; darum
sollt ihr diesen Tag von Geschlecht zu Geschlecht als eine ewige Satzung
beobachten! \bibleverse{18} Im ersten Monat, am vierzehnten Tage des
Monats, am Abend sollt ihr ungesäuertes Brot essen bis zum Abend des
einundzwanzigsten Tages des Monats! \bibleverse{19} Sieben Tage lang
darf kein Sauerteig in euren Häusern zu finden sein; denn wer da
Gesäuertes ißt, ein solcher Mensch soll aus der Gemeinde Israel
ausgerottet werden, er sei ein Fremder oder ein Einheimischer im Lande!
\bibleverse{20} Nichts Gesäuertes dürft ihr essen: überall, wo ihr auch
wohnen mögt, sollt ihr ungesäuertes Brot essen!«

\hypertarget{cc-mose-teilt-den-uxe4ltesten-die-vorschriften-uxfcber-das-passah-mit}{%
\subparagraph{cc) Mose teilt den Ältesten die Vorschriften über das
Passah
mit}\label{cc-mose-teilt-den-uxe4ltesten-die-vorschriften-uxfcber-das-passah-mit}}

\bibleverse{21} Da berief Mose alle Ältesten der Israeliten und sagte zu
ihnen: »Geht hin und holt euch Kleinvieh, für jede Familie ein Stück,
und schlachtet es als Passah! \bibleverse{22} Dann nehmt einen Büschel
Ysop, taucht ihn in das Blut im Becken und streicht etwas von dem Blut
im Becken an die Oberschwelle und an die beiden Pfosten der Tür; keiner
von euch darf aber bis zum andern Morgen aus der Tür seines Hauses
hinausgehen! \bibleverse{23} Wenn dann der HERR einherschreitet, um die
Ägypter sterben zu lassen, und er das Blut an der Oberschwelle und an
den beiden Türpfosten sieht, so wird der HERR an der Tür schonend
vorübergehen und dem Würgengel nicht gestatten, in eure Häuser
einzutreten, um euch sterben zu lassen. \bibleverse{24} Ihr sollt aber
dieses Gebot als eine Satzung für euch und eure Kinder auf ewige Zeiten
beobachten! \bibleverse{25} Auch wenn ihr in das Land kommt, das der
HERR euch nach seiner Verheißung geben wird, sollt ihr diesen heiligen
Brauch stets beobachten! \bibleverse{26} Wenn eure Kinder euch dann
fragen: ›Was bedeutet dieser Brauch bei euch?‹, \bibleverse{27} so sollt
ihr antworten: ›Es ist das Passahopfer für den HERRN, der in Ägypten an
den Häusern der Israeliten schonend vorübergegangen ist: während er die
Ägypter sterben ließ, hat er unsere Häuser verschont.‹« Da verneigte
sich das Volk und warf sich zur Erde nieder. \bibleverse{28} Hierauf
gingen die Israeliten hin und taten so; wie der HERR dem Mose und Aaron
geboten hatte, so taten sie.

\hypertarget{i-die-zehnte-plage-sterben-der-uxe4gyptischen-erstgeburten-und-der-anfang-des-auszugs}{%
\paragraph{i) Die zehnte Plage (Sterben der ägyptischen Erstgeburten)
und der Anfang des
Auszugs}\label{i-die-zehnte-plage-sterben-der-uxe4gyptischen-erstgeburten-und-der-anfang-des-auszugs}}

\bibleverse{29} Um Mitternacht aber begab es sich, daß der HERR alle
Erstgeburten im Lande Ägypten sterben ließ, vom erstgeborenen Sohn des
Pharaos an, der auf seinem Thron saß, bis zum Erstgeborenen des
Gefangenen, der im Kerker lag, auch alles Erstgeborene des Viehs.
\bibleverse{30} Da stand der Pharao in dieser Nacht auf, er und alle
seine Diener und alle übrigen Ägypter, und es erhob sich ein großes
Wehgeschrei in Ägypten; denn es gab kein Haus, in dem nicht ein Toter
gelegen hätte. \bibleverse{31} Da ließ (der Pharao) noch in der Nacht
Mose und Aaron rufen und sagte: »Macht euch auf, zieht aus meinem Volk
hinweg, sowohl ihr als auch die Israeliten! Geht hin und dient dem
HERRN, wie ihr gesagt habt! \bibleverse{32} Auch euer Kleinvieh und eure
Rinder nehmt mit, wie ihr gesagt habt: geht hin und bittet auch für mich
um Segen!« \bibleverse{33} Auch die Ägypter drängten das Volk zu
schleunigem Aufbruch aus dem Lande; denn sie dachten: »Wir sind (sonst)
alle des Todes!« \bibleverse{34} Da nahm das Volk seinen Brotteig, noch
ehe er gesäuert war, ihre Backschüsseln, die sie, in ihre Mäntel
gewickelt, auf den Schultern trugen. \bibleverse{35} Die Israeliten
hatten aber (zuvor) die Weisung Moses befolgt und sich von den Ägyptern
silberne und goldene Wertsachen sowie Kleider erbeten; \bibleverse{36}
und der HERR hatte dabei die Ägypter gegen das Volk günstig gestimmt, so
daß sie ihnen das Erbetene gewährten; und so plünderten sie die Ägypter
aus.

\hypertarget{der-auszug-israels-aus-uxe4gypten}{%
\paragraph{Der Auszug Israels aus
Ägypten}\label{der-auszug-israels-aus-uxe4gypten}}

\bibleverse{37} So brachen denn die Israeliten von Ramses nach Sukkoth
zu auf, ungefähr 600000~Mann zu Fuß, die Männer allein, ungerechnet die
Weiber und Kinder. \bibleverse{38} Auch viel zusammengelaufenes Volk zog
mit ihnen, dazu Kleinvieh und Rinder, eine gewaltige Menge Vieh.
\bibleverse{39} Aus dem Teig aber, den sie aus Ägypten mitgenommen
hatten, buken sie (unterwegs) ungesäuerte Brotkuchen; denn er war
ungesäuert, weil man sie aus Ägypten vertrieben und ihnen keine Zeit
gelassen hatte; daher hatten sie auch für keine Wegzehrung sorgen
können.

\bibleverse{40} Die Zeit aber, während welcher die Israeliten in Ägypten
gewohnt hatten, betrug 430~Jahre; \bibleverse{41} und nach Ablauf dieser
430~Jahre, und zwar an eben jenem Tage, zogen alle Heerscharen des HERRN
aus dem Land Ägypten weg. \bibleverse{42} Eine Nacht des
Wachens\textless sup title=``=~eine durchwachte Nacht''\textgreater✲ für
den HERRN war das, damit er sie aus Ägypten wegführe; eben diese Nacht
ist dem HERRN geweiht als ein von allen Israeliten für alle ihre
künftigen Geschlechter zu beobachtendes Wachen.

\hypertarget{k-nachtrag-zur-passahverordnung-heiligung-der-erstgeburt}{%
\paragraph{k) Nachtrag zur Passahverordnung; Heiligung der
Erstgeburt}\label{k-nachtrag-zur-passahverordnung-heiligung-der-erstgeburt}}

\bibleverse{43} Da sagte der HERR zu Mose und Aaron: »Dies ist die
Verordnung für das Passah: Kein Fremder darf davon essen\textless sup
title=``oder: mitessen''\textgreater✲; \bibleverse{44} aber jeder für
Geld gekaufte Knecht✲ darf davon essen, sobald er beschnitten worden
ist. \bibleverse{45} Ein Beisasse oder Lohnarbeiter dürfen nicht davon
essen. \bibleverse{46} In einem und demselben Hause muß es gegessen
werden; man darf nichts von dem Fleisch aus dem Hause hinausbringen, und
keinen Knochen dürft ihr an ihm zerbrechen\textless sup title=``vgl. Joh
19,36''\textgreater✲. \bibleverse{47} Die ganze Gemeinde der Israeliten
soll es feiern; \bibleverse{48} und wenn ein Fremder sich unter euch
aufhält und das Passah zu Ehren des HERRN feiern will, so müssen zuvor
alle männlichen Personen seiner Familie beschnitten werden: alsdann darf
er an der Feier teilnehmen und soll den Einheimischen des Landes
gleichgeachtet sein; aber kein Unbeschnittener darf davon essen.
\bibleverse{49} Ein und dasselbe Gesetz soll für den Einheimischen und
für den Fremden gelten, der in eurer Mitte weilt!« \bibleverse{50} Da
taten alle Israeliten so; wie der HERR dem Mose und Aaron geboten hatte,
genau so taten sie.~-- \bibleverse{51} An eben diesem Tage, an welchem
der HERR die Israeliten aus Ägypten nach ihren Heerscharen✲
hinwegführte,

\hypertarget{section-12}{%
\section{13}\label{section-12}}

\bibleverse{1} gebot der HERR dem Mose folgendes: \bibleverse{2}
»Heilige\textless sup title=``oder: weihe''\textgreater✲ mir alles
Erstgeborene, alles, was bei den Israeliten zuerst aus dem Mutterschoß
ans Tageslicht hervortritt, von Menschen wie vom Vieh: es gehört mir!«

\hypertarget{die-bekanntmachung-der-vorschriften-uxfcber-die-feier-des-festes-der-ungesuxe4uerten-brote}{%
\paragraph{Die Bekanntmachung der Vorschriften über die Feier des Festes
der ungesäuerten
Brote}\label{die-bekanntmachung-der-vorschriften-uxfcber-die-feier-des-festes-der-ungesuxe4uerten-brote}}

\bibleverse{3} Hierauf sagte Mose zum Volk: »Gedenkt des heutigen Tages,
an dem ihr aus Ägypten weggezogen seid, aus dem Hause der Knechtschaft!
Denn mit starker Hand hat der HERR euch von dort weggeführt; daher darf
nichts Gesäuertes gegessen werden! \bibleverse{4} Heute zieht ihr aus im
Monat Abib\textless sup title=``d.h. im Ährenmonat''\textgreater✲.
\bibleverse{5} Wenn dich nun der HERR in das Land der Kanaanäer,
Hethiter, Amoriter, Hewiter und Jebusiter gebracht hat, dessen
Verleihung er deinen Vätern einst zugeschworen hat, ein Land, das von
Milch und Honig überfließt, so sollst du diesen heiligen Brauch in
diesem Monat beobachten: \bibleverse{6} Sieben Tage lang sollst du
ungesäuertes Brot essen, und am siebten Tage findet ein Fest zu Ehren
des HERRN statt! \bibleverse{7} Während der sieben Tage soll
ungesäuertes Brot gegessen werden, und nichts Gesäuertes und kein
Sauerteig darf in deinem ganzen Gebiet zu finden sein! \bibleverse{8}
Deinen Kindern aber sollst du an diesem Tage folgendes kundtun: ›(Diesen
Brauch beobachte ich) zur Erinnerung an das, was der HERR an
mir\textless sup title=``oder: für mich''\textgreater✲ getan hat, als
ich aus Ägypten auszog.‹ \bibleverse{9} Und (dieser Brauch) soll dir
gleichsam ein Denkzeichen an deiner Hand und ein Erinnerungsmal auf
deiner Stirn sein, damit das Gesetz des HERRN in deinem Munde lebendig
bleibt; denn mit starker Hand hat der HERR dich aus Ägypten weggeführt.
\bibleverse{10} Darum sollst du diese Satzung zur bestimmten Zeit Jahr
für Jahr beobachten!«

\hypertarget{heiligung-der-erstgeburten}{%
\paragraph{Heiligung der
Erstgeburten}\label{heiligung-der-erstgeburten}}

\bibleverse{11} »Wenn der HERR dich dann, wie er es dir und deinen
Vätern zugeschworen hat, in das Land der Kanaanäer gebracht und es dir
zum Besitz gegeben hat, \bibleverse{12} so sollst du dem HERRN alles
weihen, was zuerst aus dem Mutterschoß ans Tageslicht hervortritt! Auch
jeder erste Wurf des Viehs, der dir zuteil wird\textless sup
title=``oder: das du besitzest''\textgreater✲, soweit er männlich ist,
gehört dem HERRN. \bibleverse{13} Jedes erstgeborene Eselfüllen aber
sollst du mit einem Stück Kleinvieh\textless sup title=``oder: einem
Lamm''\textgreater✲ loskaufen; oder, wenn du es nicht loskaufen willst,
so brich ihm das Genick! Weiter sollst du jede Erstgeburt vom Menschen
bei deinen Söhnen loskaufen! \bibleverse{14} Wenn dann dein Sohn künftig
die Frage an dich richtet: ›Was bedeutet dieser Brauch?‹, so antworte
ihm: ›Mit starker Hand hat der HERR uns aus Ägypten, aus dem Hause der
Knechtschaft, weggeführt. \bibleverse{15} Denn weil der Pharao sich
hartnäckig dagegen sträubte, uns ziehen zu lassen, hat der HERR alle
Erstgeburt in Ägypten sterben lassen, von den Erstgeborenen der Menschen
an bis zur Erstgeburt des Viehs; darum opfere ich dem HERRN alle
männlichen Erstgeburten, aber jeden Erstgeborenen meiner Söhne kaufe ich
los.‹ \bibleverse{16} Und dieser Brauch soll dir gleichsam ein
Denkzeichen an deiner Hand und ein Erinnerungsmal auf deiner Stirn sein;
denn mit starker Hand hat der HERR uns aus Ägypten weggeführt!«

\hypertarget{der-durchzug-der-israeliten-durch-das-schilfmeer-1317-1521}{%
\subsubsection{7. Der Durchzug der Israeliten durch das Schilfmeer
(13,17-15,21)}\label{der-durchzug-der-israeliten-durch-das-schilfmeer-1317-1521}}

\hypertarget{a-der-zug-nach-der-wuxfcste-und-dem-schilfmeer-zu-bis-etham-die-bestuxe4ndige-guxf6ttliche-fuxfchrung-des-heereszuges}{%
\paragraph{a) Der Zug nach der Wüste und dem Schilfmeer zu bis Etham;
die beständige göttliche Führung des
Heereszuges}\label{a-der-zug-nach-der-wuxfcste-und-dem-schilfmeer-zu-bis-etham-die-bestuxe4ndige-guxf6ttliche-fuxfchrung-des-heereszuges}}

\bibleverse{17} Als nun der Pharao das Volk hatte ziehen lassen, führte
Gott sie nicht in der Richtung nach dem Lande der Philister, obgleich
dies der nächste Weg gewesen wäre; denn Gott dachte: Das Volk könnte es
sich gereuen lassen, wenn es Krieg in Aussicht hätte, und möchte wieder
nach Ägypten zurückkehren. \bibleverse{18} Darum ließ Gott das Volk sich
seitwärts in der Richtung nach der Wüste, gegen das Schilfmeer hin,
wenden, und kampfgerüstet zogen die Israeliten aus Ägypten ab.
\bibleverse{19} Mose nahm aber die Gebeine Josephs mit; denn dieser
hatte die Israeliten feierlich schwören lassen und gebeten\textless sup
title=``vgl. 1.Mose 50,25''\textgreater✲: »Wenn Gott sich einst an euch
gnädig erweist, dann nehmt meine Gebeine von hier mit euch!«

\bibleverse{20} So brachen sie denn von Sukkoth auf und lagerten in
Etham am Rande der Wüste. \bibleverse{21} Der HERR aber zog vor ihnen
her, bei Tage in einer Wolkensäule, um ihnen den Weg zu zeigen, und
nachts in einer Feuersäule, um ihnen zu leuchten, damit sie bei Tag und
bei Nacht wandern könnten: \bibleverse{22} nicht wich die Wolkensäule
bei Tage und nicht die Feuersäule nachts von der Spitze des Zuges.

\hypertarget{b-gott-befiehlt-die-uxe4nderung-der-marschrichtung}{%
\paragraph{b) Gott befiehlt die Änderung der
Marschrichtung}\label{b-gott-befiehlt-die-uxe4nderung-der-marschrichtung}}

\hypertarget{section-13}{%
\section{14}\label{section-13}}

\bibleverse{1} Da gebot der HERR dem Mose folgendes: \bibleverse{2}
»Befiehl den Israeliten umzukehren und östlich von Pi-Hahiroth zwischen
Migdol und dem Meer zu lagern! Gerade gegenüber von Baal-Zephon sollt
ihr am Meer lagern! \bibleverse{3} Dann wird der Pharao von den
Israeliten denken: ›Ratlos irren sie im Lande umher, die Wüste hält sie
umschlossen!‹ \bibleverse{4} Dann will ich das Herz des Pharaos
verhärten, daß er sie verfolgt, damit ich mich am Pharao und an seiner
ganzen Heeresmacht verherrliche und damit die Ägypter erkennen, daß ich
der HERR bin.« Und sie taten so.

\hypertarget{c-der-pharao-verfolgt-die-israeliten-und-holt-sie-ein}{%
\paragraph{c) Der Pharao verfolgt die Israeliten und holt sie
ein}\label{c-der-pharao-verfolgt-die-israeliten-und-holt-sie-ein}}

\bibleverse{5} Als nun dem König von Ägypten gemeldet wurde, daß das
Volk entwichen sei, trat bei ihm und seinen Dienern eine Sinnesänderung
dem Volk gegenüber ein, und sie sagten: »Was haben wir da getan, daß wir
die Israeliten aus unserm Dienst entlassen haben!« \bibleverse{6} So
ließ er denn seinen Streitwagen anschirren und nahm sein Kriegsvolk mit
sich; \bibleverse{7} sechshundert auserlesene Kriegswagen nahm er mit
und was sonst an Kriegswagen in Ägypten vorhanden war und die besten
Kämpfer auf einem jeden von ihnen. \bibleverse{8} Denn der HERR hatte
das Herz des Pharaos, des Königs von Ägypten, verhärtet, so daß er die
Israeliten verfolgte, obgleich diese mit hocherhobener Hand\textless sup
title=``d.h. kampfbereit''\textgreater✲ ausgezogen waren. \bibleverse{9}
So setzten denn die Ägypter ihnen nach und holten sie ein, als sie sich
eben am Meer gelagert hatten, alle Rosse und Wagen des Pharaos, seine
Reiter und überhaupt seine Heeresmacht, bei Pi-Hahiroth, Baal-Zephon
gegenüber.

\hypertarget{d-die-verzagten-israeliten-werden-von-mose-ermutigt-der-zuspruch-und-das-tatsuxe4chliche-eingreifen-gottes}{%
\paragraph{d) Die verzagten Israeliten werden von Mose ermutigt; der
Zuspruch und das tatsächliche Eingreifen
Gottes}\label{d-die-verzagten-israeliten-werden-von-mose-ermutigt-der-zuspruch-und-das-tatsuxe4chliche-eingreifen-gottes}}

\bibleverse{10} Als nun der Pharao nahe herangekommen war und die
Israeliten hinschauten und die Ägypter erblickten, die hinter ihnen
herzogen, da gerieten die Israeliten in große Angst und schrien zum
HERRN \bibleverse{11} und sagten zu Mose: »Hast du uns etwa deshalb,
weil es in Ägypten keine Gräber gab, mitgenommen, damit wir in der Wüste
sterben? Was hast du uns da angetan, daß du uns aus Ägypten weggeführt
hast! \bibleverse{12} Haben wir dir nicht schon in Ägypten aufs
bestimmteste erklärt: ›Laß uns in Ruhe: wir wollen den Ägyptern
dienen!‹; denn besser wäre es für uns, den Ägyptern zu dienen, als hier
in der Wüste zu sterben!« \bibleverse{13} Da entgegnete Mose dem Volk:
»Fürchtet euch nicht! Haltet nur stand, so werdet ihr sehen, welche
Rettung euch der HERR heute noch schaffen wird! Denn so, wie ihr die
Ägypter heute seht, werdet ihr sie in alle Ewigkeit nicht wieder sehen.
\bibleverse{14} Der HERR wird für euch streiten, verhaltet ihr euch nur
ruhig!«

\bibleverse{15} Da sagte der HERR zu Mose: »Was schreist du zu mir?
Befiehl den Israeliten aufzubrechen. \bibleverse{16} Du aber hebe deinen
Stab empor, strecke deine Hand über das Meer aus und spalte es, damit
die Israeliten mitten durch das Meer hindurch auf trockenem Boden ziehen
können. \bibleverse{17} Ich aber will dann das Herz der Ägypter
verhärten, daß sie hinter ihnen herziehen, und will mich am Pharao und
an seiner ganzen Heeresmacht, an seinen Wagen und Reitern,
verherrlichen; \bibleverse{18} und die Ägypter sollen erkennen, daß ich
der HERR bin, wenn ich mich am Pharao, an seinen Wagen und Reitern
verherrlicht habe.« \bibleverse{19} Da änderte der Engel Gottes, der
(bisher) vor dem Heer der Israeliten hergezogen war, seine Stellung und
trat hinter sie; infolgedessen ging auch die Wolkensäule vorn vor ihnen
weg und trat hinter sie, \bibleverse{20} so daß sie zwischen das Heer
der Ägypter und das Heer der Israeliten zu stehen kam; und sie zeigte
sich dort als Wolke und Finsternis, während sie hier die Nacht
erleuchtete; so gerieten beide Heere die ganze Nacht hindurch nicht
feindlich aneinander.

\hypertarget{e-durchzug-der-israeliten-durch-das-schilfmeer-untergang-der-uxe4gypter}{%
\paragraph{e) Durchzug der Israeliten durch das Schilfmeer; Untergang
der
Ägypter}\label{e-durchzug-der-israeliten-durch-das-schilfmeer-untergang-der-uxe4gypter}}

\bibleverse{21} Als dann Mose seine Hand über das Meer ausstreckte,
drängte der HERR das Meer durch einen starken Ostwind die ganze Nacht
hindurch zurück und legte den Meeresboden trocken, und die Wasser
spalteten sich. \bibleverse{22} So gingen denn die Israeliten trocknen
Fußes mitten durch das Meer, während die Wasser ihnen wie eine Wand zur
Rechten und zur Linken standen. \bibleverse{23} Die Ägypter aber eilten
ihnen nach und zogen hinter ihnen her, alle Rosse des Pharaos, seine
Wagen und seine Reiter, mitten ins Meer hinein. \bibleverse{24} Zur Zeit
der Morgenwache aber schaute der HERR in der Feuer- und Wolkensäule hin
auf das Heer der Ägypter und brachte ihren Zug in Verwirrung;
\bibleverse{25} er ließ die Räder ihrer Wagen abspringen und machte, daß
sie nur mühsam vorwärts kamen. Da riefen die Ägypter: »Laßt uns vor den
Israeliten fliehen, denn der HERR streitet für sie gegen die Ägypter!«
\bibleverse{26} Da gebot der HERR dem Mose: »Strecke deine Hand über das
Meer aus: damit die Wasser auf die Ägypter, auf ihre Wagen und ihre
Reiter, zurückströmen!« \bibleverse{27} So streckte denn Mose seine Hand
über das Meer aus, da kehrte das Meer bei Tagesanbruch in sein altes
Bett zurück, während die Ägypter ihm gerade entgegen flohen; und der
HERR stürzte die Ägypter mitten ins Meer hinein. \bibleverse{28} Denn
als die Wasser zurückgeströmt waren, bedeckten sie die Wagen und die
Reiter der ganzen Heeresmacht des Pharaos, die hinter ihnen her ins Meer
gezogen waren, so daß auch nicht einer von ihnen am Leben blieb.
\bibleverse{29} Die Israeliten aber waren trocknen Fußes mitten durch
das Meer gezogen, während die Wasser ihnen wie eine Wand zur Rechten und
zur Linken standen. \bibleverse{30} So rettete der HERR die Israeliten
an diesem Tage aus der Hand der Ägypter, und Israel sah die Ägypter tot
am Meeresufer liegen. \bibleverse{31} Als die Israeliten aber die große
Wundertat sahen, die der HERR an den Ägyptern vollbracht hatte, da
fürchtete das Volk den HERRN, und sie glaubten an den HERRN und an
seinen Knecht Mose.

\hypertarget{f-siegeslied-der-israeliten-am-schilfmeer}{%
\paragraph{f) Siegeslied der Israeliten am
Schilfmeer}\label{f-siegeslied-der-israeliten-am-schilfmeer}}

\hypertarget{section-14}{%
\section{15}\label{section-14}}

\bibleverse{1} Damals sangen Mose und die Israeliten zum Preise des
HERRN folgendes Lied:

Singen will ich dem HERRN, denn hocherhaben ist er; Rosse und Reiter hat
er ins Meer gestürzt.

\bibleverse{2} Meine Stärke und mein Lobgesang ist der HERR, der mir
Rettung geschafft hat; er ist mein Gott: ihn will ich preisen, meiner
Väter Gott: ihn will ich erheben! \bibleverse{3} Der HERR ist ein
Kriegsheld, HERR ist sein Name. \bibleverse{4} Die Wagen des Pharaos und
seine Macht hat er ins Meer gestürzt, seine auserlesenen Krieger sind im
Schilfmeer versunken. \bibleverse{5} Die Fluten haben sie bedeckt, wie
Steine sind sie in die Tiefen gefahren.

\bibleverse{6} Deine Rechte, o HERR, ist herrlich durch Kraft; deine
Rechte, o HERR, zerschmettert den Feind. \bibleverse{7} Durch die Fülle
deiner Hoheit vernichtest du deine Gegner; du läßt deine Zornglut
ausgehn: die verzehrt sie wie Spreu. \bibleverse{8} Durch den Hauch
deiner Nase türmten die Wasser sich hoch, wie ein Wall standen die
Fluten aufrecht, die Wogen erstarrten mitten im Meer. \bibleverse{9} Da
dachte der Feind: »Ich will nachsetzen, einholen, will Beute verteilen,
meine Gier soll sich letzen an ihnen! zücken will ich mein Schwert,
meine Hand soll sie tilgen!« \bibleverse{10} Da bliesest du mit deinem
Odem: das Meer bedeckte sie; wie Blei versanken sie in den gewaltigen
Wogen.

\bibleverse{11} Wer ist dir gleich, HERR, unter den Göttern? wer ist wie
du so herrlich an Majestät, furchtbar an Ruhmeswerken, ein Wundertäter?
\bibleverse{12} Du hast deine Rechte ausgestreckt: da verschlang sie die
Erde. \bibleverse{13} Mit deiner Huld hast du das Volk geleitet, das du
erlöst hast; mit deiner Kraft hast du es geführt zu deiner heiligen
Wohnstatt. \bibleverse{14} Die Völker vernahmen's und bebten, Angst
befiel die Bewohner des Philisterlandes. \bibleverse{15} Da\textless sup
title=``oder: damals?''\textgreater✲ erschraken die Fürsten von Edom,
Zittern ergriff die Häuptlinge Moabs, die Bewohner Kanaans verzagten
alle; \bibleverse{16} Entsetzen und Angst überfiel sie; ob der Kraft
deines Armes wurden sie starr wie ein Stein, bis dein Volk hindurchzog,
HERR, bis das Volk hindurchzog, das du erworben. \bibleverse{17} Du
brachtest sie hinein und pflanztest sie ein auf den Berg deines
Eigentums, an die Stätte, die du, HERR, zur Wohnung dir bereitet, in das
Heiligtum, Herr, das deine Hände gegründet. \bibleverse{18} Der HERR
ist\textless sup title=``oder: bleibt''\textgreater✲ König immer und
ewig!

\bibleverse{19} Als nämlich die Rosse des Pharaos mit seinen Wagen und
Reitern ins Meer gekommen waren, hatte der HERR die Fluten des Meeres
über sie zurückströmen lassen, während die Israeliten trocknen Fußes
mitten durchs Meer gezogen waren.

\hypertarget{beteiligung-der-frauen-besonders-mirjams-am-preise-des-herrn}{%
\paragraph{Beteiligung der Frauen (besonders Mirjams) am Preise des
Herrn}\label{beteiligung-der-frauen-besonders-mirjams-am-preise-des-herrn}}

\bibleverse{20} Darauf nahm die Prophetin Mirjam, Aarons Schwester, die
Handpauke zur Hand, und alle Frauen zogen mit Handpauken und im
Reigenschritt tanzend hinter ihr her. \bibleverse{21} Und Mirjam sang
den Männern als Antwort zu: Singet dem HERRN! Denn hocherhaben ist er;
Rosse und Reiter hat er ins Meer gestürzt!

\hypertarget{ii.-zug-zum-sinai-bundesschlieuxdfung-und-gesetzgebung-1522-4038}{%
\subsection{II. Zug zum Sinai, Bundesschließung und Gesetzgebung
(15,22-40,38)}\label{ii.-zug-zum-sinai-bundesschlieuxdfung-und-gesetzgebung-1522-4038}}

\hypertarget{zug-vom-schilfmeer-bis-zum-sinai-1522-1827}{%
\subsubsection{1. Zug vom Schilfmeer bis zum Sinai
(15,22-18,27)}\label{zug-vom-schilfmeer-bis-zum-sinai-1522-1827}}

\hypertarget{a-das-bittere-wasser-in-mara-genieuxdfbar-gemacht-die-ankunft-im-lieblichen-elim}{%
\paragraph{a) Das bittere Wasser in Mara genießbar gemacht; die Ankunft
im lieblichen
Elim}\label{a-das-bittere-wasser-in-mara-genieuxdfbar-gemacht-die-ankunft-im-lieblichen-elim}}

\bibleverse{22} Hierauf ließ Mose die Israeliten vom Schilfmeer
aufbrechen, und sie zogen weiter in die Wüste Sur hinein; drei Tage lang
wanderten sie in der Wüste, ohne Wasser zu finden. \bibleverse{23} Als
sie dann nach Mara kamen, konnten sie das Wasser dort nicht trinken,
weil es bitter war; daher hieß der Ort Mara\textless sup title=``d.h.
Bitterkeit''\textgreater✲. \bibleverse{24} Da murrte das Volk gegen Mose
und sagte: »Was sollen wir trinken?« \bibleverse{25} Da flehte er laut
zum HERRN, und der HERR zeigte ihm ein Holz; als Mose dieses in das
Wasser geworfen hatte, wurde das Wasser süß. Dort gab er\textless sup
title=``d.h. der HERR''\textgreater✲ dem Volk Gesetze und Verordnungen
und stellte es dort auf die Probe, \bibleverse{26} indem er sagte: »Wenn
du auf die Weisungen des HERRN, deines Gottes, willig hörst und das
tust, was ihm wohlgefällt, wenn du seinen Befehlen gehorchst und alle
seine Gebote beobachtest, so will ich von allen Heimsuchungen, die ich
über die Ägypter verhängt habe, keine über dich kommen lassen; denn ich,
der HERR, bin dein Arzt\textless sup title=``=~der dich
heilt''\textgreater✲.« \bibleverse{27} Hierauf kamen sie nach Elim; dort
waren zwölf Wasserquellen und siebzig Palmbäume; und sie lagerten dort
am Wasser.

\hypertarget{b-wachteln-und-manna-als-speisen-in-der-wuxfcste-sin}{%
\paragraph{b) Wachteln und Manna als Speisen in der Wüste
Sin}\label{b-wachteln-und-manna-als-speisen-in-der-wuxfcste-sin}}

\hypertarget{section-15}{%
\section{16}\label{section-15}}

\bibleverse{1} Dann brachen sie von Elim auf, und die ganze Gemeinde der
Israeliten gelangte in die Wüste Sin, die zwischen Elim und dem Sinai
liegt, am fünfzehnten Tage des zweiten Monats nach ihrem Auszug aus dem
Lande Ägypten.

\hypertarget{aa-das-murren-des-volkes-die-guxf6ttliche-erhuxf6rung-durch-die-wachtel--und-mannaspende}{%
\subparagraph{aa) Das Murren des Volkes; die göttliche Erhörung durch
die Wachtel- und
Mannaspende}\label{aa-das-murren-des-volkes-die-guxf6ttliche-erhuxf6rung-durch-die-wachtel--und-mannaspende}}

\bibleverse{2} Da murrte die ganze Gemeinde der Israeliten gegen Mose
und Aaron in\textless sup title=``oder: wegen''\textgreater✲ der Wüste;
\bibleverse{3} und die Israeliten sagten zu ihnen: »Wären wir doch durch
die Hand des HERRN in Ägypten gestorben, als wir bei den Fleischtöpfen
saßen und reichlich Brot zu essen hatten! Jetzt habt ihr uns in diese
Wüste hinausgeführt, um diese ganze Volksgemeinde Hungers sterben zu
lassen!« \bibleverse{4} Da sagte der HERR zu Mose: »Gut! Ich will euch
Brot vom Himmel regnen lassen; das Volk braucht dann nur hinauszugehen
und sich seinen täglichen Bedarf Tag für Tag zu sammeln; damit will ich
es auf die Probe stellen, ob es nach meinen Weisungen wandeln will oder
nicht. \bibleverse{5} Wenn sie aber am sechsten Tage das, was sie
heimgebracht haben, zubereiten, so wird es das Doppelte von dem sein,
was sie sonst tagtäglich gesammelt haben.«

\bibleverse{6} Da sagten Mose und Aaron zu allen Israeliten: »Heute
abend werdet ihr erkennen, daß der HERR es gewesen ist, der euch aus
Ägypten weggeführt hat; \bibleverse{7} und morgen früh, da werdet ihr
die Herrlichkeit des HERRN zu sehen bekommen! Denn er hat gehört, wie
ihr gegen ihn gemurrt habt; wir dagegen -- was sind wir, daß ihr gegen
uns murren könntet?« \bibleverse{8} Dann fuhr Mose fort: »Ja, daran
werdet ihr (die Herrlichkeit des HERRN) erkennen, daß der HERR euch
heute abend Fleisch zu essen gibt und morgen früh Brot zum Sattwerden,
weil der HERR gehört hat, wie ihr gegen ihn laut gemurrt habt. Denn was
sind wir? Euer Murren ist nicht gegen uns gerichtet, sondern gegen den
HERRN.« \bibleverse{9} Hierauf sagte Mose zu Aaron: »Befiehl der ganzen
Gemeinde der Israeliten: ›Tretet heran vor den HERRN; denn er hat euer
Murren gehört!‹« \bibleverse{10} Als dann Aaron dies der ganzen Gemeinde
der Israeliten mitgeteilt hatte und sie sich nach der Wüste hin gewandt
hatten, da erschien plötzlich die Herrlichkeit des HERRN in der Wolke.

\bibleverse{11} Darauf sagte der HERR zu Mose: \bibleverse{12} »Ich habe
das Murren der Israeliten gehört; mache ihnen folgendes bekannt: ›Gegen
Abend\textless sup title=``genauer: zwischen den beiden Abenden; vgl.
12,6''\textgreater✲ sollt ihr Fleisch zu essen bekommen und morgen früh
euch an Brot satt essen und sollt erkennen, daß ich, der HERR, euer Gott
bin.‹« \bibleverse{13} Und wirklich: am Abend kamen Wachteln
herangezogen und bedeckten das Lager; und am anderen Morgen lag eine
Tauschicht rings um das Lager her; \bibleverse{14} und als die
Tauschicht vergangen war, da lag überall auf der Wüstenfläche etwas
Feines, Körniges, fein wie der Reif auf der Erde. \bibleverse{15} Als
das die Israeliten sahen, fragten sie einer den andern: »Was ist das?«;
denn sie wußten nicht, was es war. Da sagte Mose zu ihnen: »Dies ist das
Brot, das der HERR euch zum Essen gegeben hat.«

\hypertarget{bb-vorschriften-uxfcber-das-einsammeln-des-manna-mose-erkluxe4rt-eine-wundererscheinung-die-dabei-vorkam}{%
\subparagraph{bb) Vorschriften über das Einsammeln des Manna; Mose
erklärt eine Wundererscheinung, die dabei
vorkam}\label{bb-vorschriften-uxfcber-das-einsammeln-des-manna-mose-erkluxe4rt-eine-wundererscheinung-die-dabei-vorkam}}

\bibleverse{16} »Folgendes ist es, was der HERR euch gebietet: ›Sammelt
euch davon, jeder soviel er für sich zum Essen nötig hat, je einen
Gomer\textless sup title=``vgl. 16,36''\textgreater✲ für den Kopf; nach
der Zahl der Seelen✲, die jeder in seinem Zelt hat, sollt ihr euch
holen.‹« \bibleverse{17} Da taten die Israeliten so und sammelten, der
eine viel, der andere wenig; \bibleverse{18} als sie es dann aber mit
dem Gomer maßen, da hatte der, welcher viel gesammelt hatte, keinen
Überschuß, und wer wenig gesammelt hatte, dem mangelte nichts: jeder
hatte so viel gesammelt, als er zu seiner Nahrung bedurfte.
\bibleverse{19} Hierauf befahl ihnen Mose: »Niemand hebe etwas davon bis
zum anderen Morgen auf!« \bibleverse{20} Aber sie hörten nicht auf Mose,
sondern manche hoben etwas davon bis zum anderen Morgen auf; aber da
waren Würmer darin gewachsen, und es roch übel; Mose aber wurde zornig
über sie. \bibleverse{21} So sammelten sie es denn alle Morgen, ein
jeder nach seinem Bedarf; sobald aber die Sonne heiß schien, zerschmolz
es.

\bibleverse{22} Am sechsten Tage aber hatten sie doppelt so viel Brot
gesammelt, zwei Gomer für jede Person. Da kamen alle Vorsteher der
Gemeinde und berichteten es dem Mose. \bibleverse{23} Dieser antwortete
ihnen: »Folgendes ist es, was der HERR geboten hat: ›Ein Ruhetag, ein
dem HERRN heiliger Feiertag (Sabbat) ist morgen\textless sup
title=``oder: soll morgen sein''\textgreater✲!‹ Was ihr backen wollt,
das backt, und was ihr kochen wollt, das kocht; alles aber, was
übrigbleibt, legt beiseite und hebt es euch für morgen auf!«
\bibleverse{24} Da hoben sie es bis zum folgenden Morgen auf, wie Mose
angeordnet hatte, und diesmal wurde es nicht übelriechend, und auch kein
Wurm war darin. \bibleverse{25} Da sagte Mose: »Eßt es heute! Denn heute
ist Sabbatfeier für den HERRN: heute werdet ihr auf dem Felde nichts
finden. \bibleverse{26} Sechs Tage sollt ihr es sammeln; aber am siebten
Tage ist Sabbat✲, an diesem gibt es keins.« \bibleverse{27} Als trotzdem
am siebten Tage einige vom Volk hinausgingen, um zu sammeln, fanden sie
nichts. \bibleverse{28} Da sagte der HERR zu Mose: »Wie lange wollt ihr
euch noch weigern, meine Gebote und Weisungen zu befolgen?
\bibleverse{29} Seht doch! Weil der HERR euch den Sabbat eingesetzt hat,
darum gibt er euch am sechsten Tage Brot für zwei Tage. Bleibt also alle
daheim: niemand verlasse am siebten Tage seine Wohnung!« \bibleverse{30}
So ruhte denn das Volk am siebten Tage.

\hypertarget{cc-nuxe4here-angaben-uxfcber-das-manna-gottes-gebot-bezuxfcglich-der-aufbewahrung-eines-mit-manna-gefuxfcllten-kruges}{%
\subparagraph{cc) Nähere Angaben über das Manna; Gottes Gebot bezüglich
der Aufbewahrung eines mit Manna gefüllten
Kruges}\label{cc-nuxe4here-angaben-uxfcber-das-manna-gottes-gebot-bezuxfcglich-der-aufbewahrung-eines-mit-manna-gefuxfcllten-kruges}}

\bibleverse{31} Die Israeliten nannten es aber Manna; es sah weißlich
aus wie Koriandersamen und schmeckte wie Honigkuchen. \bibleverse{32}
Hierauf sagte Mose: »Folgendes hat der HERR geboten: ›Ein Gomer voll
soll davon für eure künftigen Geschlechter aufbewahrt werden, damit sie
das Brot sehen, mit dem ich euch in der Wüste gespeist habe, als ich
euch aus dem Lande Ägypten wegführte.‹« \bibleverse{33} Da befahl Mose
dem Aaron: »Nimm einen Krug, tu einen Gomer Manna hinein und stelle ihn
hin vor den HERRN zur Aufbewahrung für eure künftigen Geschlechter!«
\bibleverse{34} Nach dem Befehl, den der HERR dem Mose gegeben hatte,
stellte Aaron (den Krug später) vor die Gesetzestafeln in der Bundeslade
zur Aufbewahrung. \bibleverse{35} Die Israeliten haben aber das Manna
vierzig Jahre lang gegessen, bis sie in bewohntes Land kamen; sie haben
das Manna gegessen, bis sie an die Grenze des Landes Kanaan
kamen\textless sup title=``vgl. Jos 5,12''\textgreater✲. \bibleverse{36}
Ein Gomer aber ist der zehnte Teil eines Epha.

\hypertarget{c-von-der-wuxfcste-sin-nach-rephidim-sieg-uxfcber-die-amalekiter}{%
\paragraph{c) Von der Wüste Sin nach Rephidim; Sieg über die
Amalekiter}\label{c-von-der-wuxfcste-sin-nach-rephidim-sieg-uxfcber-die-amalekiter}}

\hypertarget{aa-die-wunderbare-wasserspende-aus-dem-felsen-bei-massa-und-meriba}{%
\subparagraph{aa) Die wunderbare Wasserspende aus dem Felsen bei Massa
und
Meriba}\label{aa-die-wunderbare-wasserspende-aus-dem-felsen-bei-massa-und-meriba}}

\hypertarget{section-16}{%
\section{17}\label{section-16}}

\bibleverse{1} Hierauf zog die ganze Gemeinde der Israeliten nach dem
Befehl des HERRN aus der Wüste Sin weiter, einen Tagemarsch nach dem
andern, und lagerte in Rephidim, wo es aber kein Trinkwasser für das
Volk gab. \bibleverse{2} Da haderte das Volk mit Mose und rief: »Gebt
uns Wasser zum Trinken!« Aber Mose antwortete ihnen: »Was hadert ihr mit
mir? Was versucht ihr den HERRN?« \bibleverse{3} Weil aber das Volk dort
infolge des Wassermangels Durst litt, murrte es gegen Mose und sagte:
»Warum hast du uns nur aus Ägypten hergeführt? Etwa um mich und meine
Kinder und mein Vieh hier verdursten zu lassen?« \bibleverse{4} Da
betete Mose laut zum HERRN mit den Worten: »Was soll ich mit diesem Volk
machen? Es fehlt nicht viel, so steinigen sie mich!« \bibleverse{5} Da
antwortete der HERR dem Mose: »Tritt an die Spitze des Volkes und nimm
einige von den Ältesten der Israeliten mit dir! Auch deinen Stab, mit
dem du den Nil geschlagen hast, nimm in die Hand und gehe!
\bibleverse{6} Dann will ich dort vor dich auf den Felsen am Horeb
treten, und wenn du dann an den Felsen geschlagen hast, wird Wasser aus
ihm hervorfließen, so daß das Volk zu trinken hat.« Mose tat so vor den
Augen der Ältesten Israels. \bibleverse{7} Darauf nannte er den Ort
Massa\textless sup title=``=~Prüfung, Versuchung''\textgreater✲ und
Meriba✲, weil die Israeliten dort gehadert und den HERRN geprüft (oder
versucht) hatten, indem sie sagten: »Ist der HERR in unserer Mitte oder
nicht?«

\hypertarget{bb-die-amalekiter-bei-rephidim-besiegt}{%
\subparagraph{bb) Die Amalekiter bei Rephidim
besiegt}\label{bb-die-amalekiter-bei-rephidim-besiegt}}

\bibleverse{8} Als darauf die Amalekiter heranrückten, um mit den
Israeliten bei Rephidim zu kämpfen, \bibleverse{9} befahl Mose dem
Josua: »Wähle uns\textless sup title=``oder: dir''\textgreater✲ Männer
aus und ziehe zum Kampf mit den Amalekitern aus! Morgen will ich mich
mit dem Gottesstabe in der Hand auf die Spitze des Hügels stellen.«
\bibleverse{10} Josua tat, wie Mose ihm befohlen hatte, (und zog aus,)
um mit den Amalekitern zu kämpfen, während Mose, Aaron und Hur auf die
Spitze des Hügels stiegen. \bibleverse{11} Solange nun Mose seinen Arm
hochhielt, hatten die Israeliten die Oberhand; sobald er aber seinen Arm
ruhen✲ ließ, waren die Amalekiter siegreich. \bibleverse{12} Als nun
schließlich die Arme Moses erlahmten, nahmen sie einen Stein und legten
den unter ihn, und er setzte sich darauf; dann stützten Aaron und Hur
seine Arme, der eine auf dieser, der andere auf jener Seite; so blieben
seine Arme fest bis zum Sonnenuntergang, \bibleverse{13} so daß Josua
die Amalekiter und ihr Kriegsvolk mit der Schärfe des Schwertes
niederhieb. \bibleverse{14} Da sagte der HERR zu Mose: »Schreibe dies zu
dauernder Erinnerung in ein Buch und schärfe es dem Josua ein, daß ich
das Andenken an die Amalekiter unter dem Himmel ganz und gar austilgen
werde!« \bibleverse{15} Darauf baute Mose einen Altar und nannte ihn
›der HERR ist mein Banner\textless sup title=``oder:
Panier''\textgreater✲‹; \bibleverse{16} »denn«, sagte er, »die Hand an
das Banner\textless sup title=``oder: Panier''\textgreater✲ des HERRN!
Krieg führt der HERR mit den Amalekitern von Geschlecht zu Geschlecht!«

\hypertarget{d-jethros-besuch-bei-mose-am-gottesberge-einsetzung-von-richtern}{%
\paragraph{d) Jethros Besuch bei Mose am Gottesberge; Einsetzung von
Richtern}\label{d-jethros-besuch-bei-mose-am-gottesberge-einsetzung-von-richtern}}

\hypertarget{section-17}{%
\section{18}\label{section-17}}

\bibleverse{1} Jethro aber, der Priester der Midianiter, der
Schwiegervater Moses, hatte alles erfahren, was Gott an Mose und an
seinem Volke Israel getan hatte, daß der HERR nämlich die Israeliten aus
Ägypten weggeführt hatte. \bibleverse{2} Da nahm Jethro, der
Schwiegervater Moses, Zippora, Moses Frau, die dieser zurückgesandt
hatte, \bibleverse{3} samt ihren beiden Söhnen, von denen der eine
Gersom hieß, weil Mose gesagt hatte: »Ein Gast bin ich in einem fremden
Lande geworden«\textless sup title=``vgl. 2,22''\textgreater✲,
\bibleverse{4} während der andere Elieser✲ hieß, denn »der Gott meines
Vaters ist meine Hilfe gewesen und hat mich vor dem Schwert des Pharaos
errettet«~-- \bibleverse{5} Jethro also, der Schwiegervater Moses, kam
mit den Söhnen Moses und dessen Frau zu Mose in die Wüste, wo jener sich
am Berge Gottes gelagert hatte, \bibleverse{6} und ließ dem Mose sagen:
»Ich, dein Schwiegervater Jethro, komme zu dir mit deiner Frau und ihren
beiden Söhnen, die bei ihr sind.« \bibleverse{7} Da ging Mose seinem
Schwiegervater entgegen, verneigte sich vor ihm, und (jener) küßte ihn;
nachdem sie dann einander begrüßt hatten, traten sie in das Zelt ein.
\bibleverse{8} Hierauf erzählte Mose seinem Schwiegervater alles, was
der HERR am Pharao und an den Ägyptern um der Israeliten willen getan
hatte, und alle die Leiden, die ihnen unterwegs zugestoßen waren, und
wie der HERR sie errettet hatte. \bibleverse{9} Da freute sich Jethro
über alles Gute, das der HERR den Israeliten erwiesen hatte, indem er
sie aus der Gewalt der Ägypter errettete. \bibleverse{10} Und Jethro
rief aus: »Gepriesen sei der HERR, der euch aus der Gewalt der Ägypter
und aus der Gewalt des Pharaos errettet und der das Volk aus der
Gewaltherrschaft der Ägypter befreit hat! \bibleverse{11} Nun erkenne
ich, daß der HERR größer ist als alle Götter; er hat es bewiesen, als
(jene) sich übermütig gegen sie benahmen.« \bibleverse{12} Darauf ließ
Jethro, Moses Schwiegervater, Tiere zu einem Brand- und Schlachtopfer
für Gott herbeibringen, und Aaron nebst allen Ältesten der Israeliten
kam herbei, um mit dem Schwiegervater Moses das Opfermahl vor Gott zu
halten.

\hypertarget{die-neuordnung-des-gerichtswesens}{%
\paragraph{Die Neuordnung des
Gerichtswesens}\label{die-neuordnung-des-gerichtswesens}}

\bibleverse{13} Am folgenden Tage aber hielt Mose eine Gerichtssitzung
ab, um dem Volke Recht zu sprechen; und das Volk stand vor Mose vom
Morgen bis zum Abend. \bibleverse{14} Als nun der Schwiegervater Moses
sah, was er alles mit dem Volk zu tun hatte, sagte er: »Was machst du
dir da mit dem Volk so viel zu schaffen? Warum sitzest du allein zu
Gericht, während das ganze Volk vom Morgen bis zum Abend vor dir steht?«
\bibleverse{15} Mose antwortete seinem Schwiegervater: »Ja, das Volk
kommt zu mir, um Gott zu befragen; \bibleverse{16} sooft sie einen
Rechtshandel haben, kommen sie zu mir, damit ich Schiedsrichter zwischen
den Parteien sei und ihnen Gottes Rechtssprüche und Entscheidungen
kundtue.« \bibleverse{17} Da sagte sein Schwiegervater zu ihm: »Dein
Verfahren ist nicht zweckmäßig; \bibleverse{18} dabei mußt du selbst und
ebenso auch diese Leute, die bei dir stehen, ganz erschöpft werden; denn
die Sache ist zu schwer für dich, und du allein kannst sie nicht
durchführen. \bibleverse{19} Nun höre mich an: ich will dir einen Rat
geben, und Gott möge mit dir sein! Sei du der Vertreter des Volkes Gott
gegenüber und bringe du ihre Anliegen vor Gott! \bibleverse{20} Mache
ihnen daneben die Rechtssprüche und Entscheidungen klar und gib ihnen
den Weg an, den sie innezuhalten haben, und das Verfahren, das sie
beobachten müssen. \bibleverse{21} Zugleich sieh dich aber unter dem
ganzen Volke nach tüchtigen, gottesfürchtigen und zuverlässigen Männern
um, die keiner Bestechung zugänglich sind, und setze diese als Obmänner
über sie, die einen über tausend, andere über hundert, andere über
fünfzig und andere über zehn, \bibleverse{22} damit sie dem Volke
jederzeit Recht sprechen, und zwar so, daß sie alle wichtigen Sachen vor
dich bringen, alle geringfügigen Sachen aber selbst entscheiden! Auf
diese Weise verschaffe dir Erleichterung und laß sie die Last mit dir
tragen! \bibleverse{23} Wenn du es so machst und Gott es dir gestattet,
so wirst du dabei bestehen können, und auch alle diese Leute werden
befriedigt nach Hause zurückkehren.«

\bibleverse{24} Mose befolgte den Rat seines Schwiegervaters und tat
alles, was er ihm vorgeschlagen hatte: \bibleverse{25} er wählte
tüchtige Männer aus allen Israeliten aus und setzte sie zu Obmännern
über das Volk ein, die einen über tausend, andere über hundert, andere
über fünfzig und über zehn. \bibleverse{26} Diese hatten dem Volk zu
jeder Zeit Recht zu sprechen: die schwierigen Sachen legten sie dem Mose
vor, aber alle geringfügigen Sachen entschieden sie selbst.
\bibleverse{27} Hierauf ließ Mose seinen Schwiegervater ziehen, und
dieser kehrte in sein Land zurück.

\hypertarget{die-bundesschlieuxdfung-am-sinai-die-verkuxfcndigung-der-zehn-gebote-die-grundordnungen-des-israelitischen-gemeinwesens-191-2418}{%
\subsubsection{2. Die Bundesschließung am Sinai; die Verkündigung der
zehn Gebote; die Grundordnungen des israelitischen Gemeinwesens
(19,1-24,18)}\label{die-bundesschlieuxdfung-am-sinai-die-verkuxfcndigung-der-zehn-gebote-die-grundordnungen-des-israelitischen-gemeinwesens-191-2418}}

\hypertarget{a-ankunft-des-volkes-am-sinai-vorbereitung-der-gesetzgebung}{%
\paragraph{a) Ankunft des Volkes am Sinai; Vorbereitung der
Gesetzgebung}\label{a-ankunft-des-volkes-am-sinai-vorbereitung-der-gesetzgebung}}

\hypertarget{section-18}{%
\section{19}\label{section-18}}

\bibleverse{1} Im dritten Monat\textless sup title=``oder: am dritten
Neumond''\textgreater✲ nach dem Auszug der Israeliten aus Ägypten, an
diesem Tage\textless sup title=``oder: genau auf den Tag''\textgreater✲
kamen sie in die Wüste Sinai. \bibleverse{2} Sie waren nämlich von
Rephidim aufgebrochen und in die Wüste Sinai gelangt und lagerten sich
dort in der Wüste, und zwar dem Berg gegenüber. \bibleverse{3} Als Mose
dann zu Gott hinaufstieg, rief der HERR ihm vom Berge herab die Worte
zu: »So sollst du zum Hause Jakobs sprechen und den Kindern Israels
verkündigen: \bibleverse{4} ›Ihr habt selbst gesehen, was ich an den
Ägyptern getan und wie ich euch auf Adlersflügeln getragen und euch
hierher zu mir gebracht habe. \bibleverse{5} Und nun, wenn ihr meinen
Weisungen willig gehorcht und meinen Bund haltet, so sollt ihr
aus\textless sup title=``oder: vor''\textgreater✲ allen Völkern mein
besonderes Eigentum sein; denn mir gehört die ganze Erde; \bibleverse{6}
ihr aber sollt mir ein Königreich von Priestern und ein heiliges Volk
sein.‹ Das sind die Worte, die du den Israeliten verkünden sollst.«

\bibleverse{7} Da ging Mose hin, berief die Ältesten des Volkes und
teilte ihnen alle diese Worte mit, die der HERR ihm aufgetragen hatte.
\bibleverse{8} Das ganze Volk aber antwortete einmütig: »Alles, was der
HERR geboten hat, wollen wir tun!« Als hierauf Mose dem HERRN die
Antwort des Volkes überbracht hatte, \bibleverse{9} sagte der HERR zu
Mose: »Ich werde diesmal in dichtem Gewölk zu dir kommen, damit das Volk
es hört, wenn ich mit dir rede, und dir für immer Glauben\textless sup
title=``oder: Vertrauen''\textgreater✲ schenkt.« {[}Mose aber berichtete
dem HERRN die Antwort des Volkes.{]}

\bibleverse{10} Dann sagte der HERR weiter zu Mose: »Gehe zum Volk und
laß sie sich heute und morgen heiligen und ihre Kleider waschen,
\bibleverse{11} damit sie übermorgen bereit sind! Denn übermorgen wird
der HERR vor den Augen des ganzen Volkes auf den Berg Sinai herabfahren.
\bibleverse{12} Bestimme daher dem Volk ringsum eine Grenze und sage
ihnen: ›Hütet euch wohl, an dem Berge emporzusteigen oder auch nur
seinen Fuß zu berühren! Wer den Berg berührt, der ist des Todes!
\bibleverse{13} Niemandes Hand darf ihn berühren, sondern ein solcher
soll gesteinigt oder erschossen\textless sup title=``d.h. mit einem
Pfeil oder Spieß durchbohrt''\textgreater✲ werden: weder ein Tier noch
ein Mensch darf am Leben bleiben! Erst wenn das Widderhorn geblasen
wird, dürfen sie am Berge emporsteigen.‹« \bibleverse{14} Darauf stieg
Mose vom Berge zum Volk hinab und ließ das Volk sich heiligen, und sie
wuschen ihre Kleider\textless sup title=``vgl. V.10''\textgreater✲;
\bibleverse{15} auch gebot er dem Volke: »Haltet euch für übermorgen
bereit: keiner nahe sich bis dahin einem Weibe!«

\hypertarget{die-erschreckende-gotteserscheinung-auf-dem-sinai}{%
\paragraph{Die erschreckende Gotteserscheinung auf dem
Sinai}\label{die-erschreckende-gotteserscheinung-auf-dem-sinai}}

\bibleverse{16} Am dritten Tage aber, als es Morgen wurde, entstand ein
Donnern und Blitzen; schweres Gewölk lag auf dem Berge, und gewaltiger
Posaunenschall ertönte, so daß das ganze Volk, das sich im Lager befand,
zitterte. \bibleverse{17} Da führte Mose das Volk aus dem Lager hinaus,
Gott entgegen, und sie stellten sich am Fuß des Berges auf.
\bibleverse{18} Der Berg Sinai aber war ganz in Rauch gehüllt, weil der
HERR im Feuer auf ihn herabgefahren war; Rauch stieg von ihm auf wie der
Rauch von einem Schmelzofen, und der ganze Berg erbebte stark.
\bibleverse{19} Und der Posaunenschall wurde immer stärker: Mose redete,
und Gott antwortete ihm mit lauter Stimme. \bibleverse{20} Als nun der
HERR auf den Berg Sinai, auf den Gipfel des Berges, hinabgefahren war,
berief er Mose auf den Gipfel des Berges, und Mose stieg hinauf.
\bibleverse{21} Da befahl der HERR dem Mose: »Steige hinab, warne das
Volk, daß sie ja nicht zum HERRN durchbrechen, um ihn zu schauen, sonst
würde eine große Zahl von ihnen ums Leben kommen! \bibleverse{22} Auch
die Priester, die sonst dem HERRN nahen dürfen, müssen die
Heiligung\textless sup title=``oder: eine Reinigung''\textgreater✲ an
sich vollziehen, damit der HERR nicht gegen sie losfährt.«
\bibleverse{23} Da erwiderte Mose dem HERRN: »Das Volk kann ja nicht auf
den Berg Sinai hinaufsteigen; denn du selbst hast uns gewarnt und mir
geboten, eine Grenze um den Berg festzusetzen und ihn für unnahbar zu
erklären.« \bibleverse{24} Doch der HERR antwortete ihm: »Steige hinab
und komm dann mit Aaron wieder herauf! Die Priester aber und das Volk
dürfen die festgesetzte Grenze nicht überschreiten, um zum HERRN
hinaufzusteigen, damit er nicht gegen sie losfährt.« \bibleverse{25} Da
stieg Mose zum Volk hinab und kündigte es ihnen an.

\hypertarget{b-die-verkuxfcndigung-der-zehn-gebote}{%
\paragraph{b) Die Verkündigung der Zehn
Gebote}\label{b-die-verkuxfcndigung-der-zehn-gebote}}

\hypertarget{section-19}{%
\section{20}\label{section-19}}

\bibleverse{1} Hierauf redete Gott alle diese Worte und sprach:
\bibleverse{2} »Ich bin der HERR, dein Gott\textless sup title=``oder:
Ich, der HERR, bin dein Gott''\textgreater✲, der dich aus dem Land
Ägypten hinausgeführt hat, aus dem Diensthause\textless sup
title=``oder: dem Hause der Knechtschaft''\textgreater✲.

\bibleverse{3} Du sollst keine anderen Götter haben neben mir!

\bibleverse{4} Du sollst dir kein Gottesbild anfertigen noch irgendein
Abbild weder von dem, was oben im Himmel, noch von dem, was unten auf
der Erde, noch von dem, was im Wasser unterhalb der Erde ist!
\bibleverse{5} Du sollst dich vor ihnen nicht niederwerfen und ihnen
nicht dienen\textless sup title=``oder: sie nicht
anbeten''\textgreater✲! Denn ich, der HERR, dein Gott, bin ein
eifriger\textless sup title=``d.h. eifersüchtiger''\textgreater✲ Gott,
der die Verschuldung der Väter heimsucht an den Kindern, an den Enkeln
und Urenkeln bei denen, die mich hassen, \bibleverse{6} der aber Gnade
erweist an Tausenden von Nachkommen\textless sup title=``oder: ins
tausendste Geschlecht''\textgreater✲ derer, die mich lieben und meine
Gebote halten.

\bibleverse{7} Du sollst den Namen des HERRN, deines Gottes, nicht
mißbrauchen! Denn der HERR wird den nicht ungestraft lassen, der seinen
Namen mißbraucht. \bibleverse{8} Gedenke des Sabbattages, daß du ihn
heilig hältst! \bibleverse{9} Sechs Tage sollst du arbeiten und alle
deine Geschäfte verrichten! \bibleverse{10} Aber der siebte Tag ist ein
Feiertag\textless sup title=``oder: Ruhetag''\textgreater✲ zu Ehren des
HERRN, deines Gottes: da darfst du keinerlei Geschäft verrichten, weder
du selbst noch dein Sohn oder deine Tochter, weder dein Knecht, noch
deine Magd, noch dein Vieh, noch der Fremdling, der bei dir in deinen
Ortschaften weilt! \bibleverse{11} Denn in sechs Tagen hat der HERR den
Himmel und die Erde geschaffen, das Meer und alles, was in ihnen ist;
aber am siebten Tage hat er geruht; darum hat der HERR den Sabbattag
gesegnet und ihn für heilig erklärt.

\bibleverse{12} Ehre deinen Vater und deine Mutter, damit du lange lebst
in dem Lande, das der HERR, dein Gott, dir geben wird!

\bibleverse{13} Du sollst nicht töten!

\bibleverse{14} Du sollst nicht ehebrechen!

\bibleverse{15} Du sollst nicht stehlen!

\bibleverse{16} Du sollst kein falsches Zeugnis ablegen gegen deinen
Nächsten!

\bibleverse{17} Du sollst nicht begehren deines Nächsten Haus! Du sollst
nicht begehren deines Nächsten Weib, noch seinen Knecht, noch seine
Magd, noch sein Rind, noch seinen Esel, noch irgend etwas, was deinem
Nächsten gehört.«

\hypertarget{die-wirkung-der-gotteserscheinung-auf-das-volk-moses-beruhigende-rede}{%
\paragraph{Die Wirkung der Gotteserscheinung auf das Volk; Moses
beruhigende
Rede}\label{die-wirkung-der-gotteserscheinung-auf-das-volk-moses-beruhigende-rede}}

\bibleverse{18} Als aber das ganze Volk die Donnerschläge und die
flammenden Blitze, den Posaunenschall und den rauchenden Berg wahrnahm,
da zitterten sie und blieben in der Ferne stehen \bibleverse{19} und
sagten zu Mose: »Rede du mit uns, dann wollen wir zuhören; Gott aber
möge nicht mit uns reden, sonst müssen wir sterben!« \bibleverse{20} Da
antwortete Mose dem Volk: »Fürchtet euch nicht! Denn Gott ist nur
deshalb gekommen, um euch auf die Probe zu stellen und damit die Furcht
vor ihm euch gegenwärtig bleibt, auf daß ihr nicht sündigt.«
\bibleverse{21} So blieb denn das Volk in der Ferne stehen; Mose aber
trat an das dunkle Gewölk heran, in welchem Gott war.

\hypertarget{c-vorluxe4ufige-ordnung-der-gottesverehrung-besonders-des-altarbaues}{%
\paragraph{c) Vorläufige Ordnung der Gottesverehrung (besonders des
Altarbaues)}\label{c-vorluxe4ufige-ordnung-der-gottesverehrung-besonders-des-altarbaues}}

\bibleverse{22} Hierauf gebot der HERR dem Mose: »So sollst du zu den
Israeliten sagen: ›Ihr habt selbst gesehen, daß ich vom Himmel her mit
euch geredet habe. \bibleverse{23} Darum sollt ihr keine anderen Götter
neben mir anfertigen: Götter von Silber und Götter von Gold sollt ihr
euch nicht anfertigen! \bibleverse{24} Einen Altar von Erde sollst du
mir herstellen und auf ihm deine Brandopfer und Heilsopfer, dein
Kleinvieh und deine Rinder darbringen! An jeder Stätte, wo ich ein
Gedächtnis meines Namens stiften werde, will ich zu dir kommen und dich
segnen. \bibleverse{25} Willst du mir aber einen Altar von Steinen
errichten, so darfst du ihn nicht von behauenen Steinen bauen! Denn
sobald du mit deinem Meißel über sie hingefahren bist, hast du sie
entweiht. \bibleverse{26} Auch darfst du nicht auf Stufen zu meinem
Altar hinaufsteigen, damit deine Blöße nicht vor ihm sichtbar wird!‹«

\hypertarget{d-gesetze-aus-dem-sogenannten-bundesbuch-kap.-21-23}{%
\paragraph{d) Gesetze aus dem sogenannten Bundesbuch (Kap.
21-23)}\label{d-gesetze-aus-dem-sogenannten-bundesbuch-kap.-21-23}}

\hypertarget{aa-die-rechte-der-hebruxe4ischen-sklaven-und-sklavinnen}{%
\subparagraph{aa) Die Rechte der hebräischen Sklaven und
Sklavinnen}\label{aa-die-rechte-der-hebruxe4ischen-sklaven-und-sklavinnen}}

\hypertarget{section-20}{%
\section{21}\label{section-20}}

\bibleverse{1} »Und dies sind die Rechtssatzungen, die du ihnen vorlegen
sollst: \bibleverse{2} Wenn du einen hebräischen Knecht✲ kaufst, soll er
dir sechs Jahre lang dienen; aber im siebten Jahre soll er unentgeltlich
freigelassen werden. \bibleverse{3} Ist er allein\textless sup
title=``d.h. ohne Frau''\textgreater✲ gekommen, so soll er auch
allein\textless sup title=``d.h. ohne Frau''\textgreater✲ wieder gehen;
war er aber verheiratet, so soll auch seine Frau mit ihm freigelassen
werden. \bibleverse{4} Hat ihm dagegen sein Herr eine Frau gegeben und
diese ihm Söhne oder Töchter geboren, so soll die Frau samt ihren
Kindern ihrem Herrn verbleiben, und er soll allein entlassen werden.
\bibleverse{5} Erklärt aber der Knecht ausdrücklich: ›Ich habe meinen
Herrn, meine Frau und meine Kinder lieb, ich will nicht freigelassen
werden‹, \bibleverse{6} so soll sein Herr ihn vor Gott hintreten lassen
und ihn an die Tür oder an den Türpfosten stellen: dort soll sein Herr
ihm das Ohr mit einer Pfrieme durchbohren, und er soll dann zeitlebens
sein Knecht bleiben.~-- \bibleverse{7} Wenn aber jemand seine Tochter
als Magd✲ verkauft, so darf sie nicht wie die Knechte freigelassen
werden. \bibleverse{8} Wenn sie ihrem Herrn, der sie für sich bestimmt
hatte, mißfällt, so lasse er sie loskaufen; jedoch sie an fremde Leute
zu verkaufen, dazu hat er kein Recht, weil er treulos an ihr gehandelt
hat. \bibleverse{9} Wenn er sie aber für seinen Sohn bestimmt, so hat er
sie nach dem Recht der Töchter\textless sup title=``=~der freien
Mädchen''\textgreater✲ zu behandeln. \bibleverse{10} Nimmt er sich noch
eine andere, so darf er ihr doch die Fleischkost, die Kleidung und die
Beiwohnung nicht verkürzen. \bibleverse{11} Will er ihr aber diese drei
Verpflichtungen nicht gewähren, so soll sie umsonst, ohne Entgelt, frei
ausgehen.«

\hypertarget{bb-bestimmungen-zum-schutz-des-menschenlebens}{%
\subparagraph{bb) Bestimmungen zum Schutz des
Menschenlebens}\label{bb-bestimmungen-zum-schutz-des-menschenlebens}}

\bibleverse{12} »Wer einen andern so schlägt, daß er stirbt, soll mit
dem Tode bestraft werden. \bibleverse{13} Hat er es aber nicht
vorsätzlich getan, sondern hat Gott es seiner Hand widerfahren lassen,
so will ich dir eine Stätte bestimmen, wohin er fliehen soll.
\bibleverse{14} Wenn sich aber jemand gegen einen andern so weit
vergißt, daß er ihn hinterlistig ums Leben bringt, so sollst du ihn
sogar von meinem Altar wegholen, damit er stirbt!~-- \bibleverse{15} Wer
seinen Vater oder seine Mutter schlägt, soll mit dem Tode bestraft
werden! \bibleverse{16} Wer einen Menschen raubt, sei es, daß er ihn
verkauft hat oder daß der Betreffende noch in seiner Gewalt gefunden
wird, soll mit dem Tode bestraft werden.« \bibleverse{17} Auch wer
seinem Vater oder seiner Mutter flucht, soll mit dem Tode bestraft
werden!~--

\hypertarget{cc-ersatzpflicht-bei-kuxf6rperverletzungen-durch-menschen}{%
\subparagraph{cc) Ersatzpflicht bei Körperverletzungen durch
Menschen}\label{cc-ersatzpflicht-bei-kuxf6rperverletzungen-durch-menschen}}

\bibleverse{18} »Wenn Männer in Streit geraten und einer den andern mit
einem Stein oder mit der Faust so schlägt, daß er zwar nicht stirbt,
aber doch bettlägerig wird, \bibleverse{19} so soll, wenn er wieder
aufkommt und draußen an seiner Krücke umhergehen kann, der andere, der
ihn geschlagen hat, straflos bleiben; nur soll er ihm den Schaden
ersetzen, der ihm aus der Arbeitsunfähigkeit erwachsen ist, und für die
Heilkosten aufkommen! \bibleverse{20} Wenn jemand seinen Knecht oder
seine Magd mit dem Stock so schlägt, daß sie ihm unter der Hand sterben,
so muß das bestraft werden; \bibleverse{21} wenn jedoch der Betreffende
noch einen oder zwei Tage am Leben bleibt, so soll keine Bestrafung
stattfinden, denn es handelt sich um sein eigenes Geld.~--
\bibleverse{22} Wenn Männer in Streit geraten und dabei eine schwangere
Frau so stoßen, daß eine Frühgeburt eintritt, ihr sonst aber kein
Schaden entsteht, so soll der Schuldige diejenige Geldbuße zahlen, die
der Ehemann der Frau ihm auferlegt, und zwar\textless sup title=``oder:
jedoch''\textgreater✲ nach Anhörung von Schiedsrichtern.~--
\bibleverse{23} Wenn aber ein bleibender Leibesschaden entsteht, so
sollst du geben\textless sup title=``oder: so gilt''\textgreater✲: Leben
um Leben, \bibleverse{24} Auge um Auge, Zahn um Zahn, Hand um Hand, Fuß
um Fuß, \bibleverse{25} Brandmal um Brandmal, Wunde um Wunde, Strieme um
Strieme!~-- \bibleverse{26} Schlägt jemand seinen Knecht oder seine Magd
so ins Auge, daß er es zugrunde richtet, so soll er sie zur
Entschädigung für ihr Auge freilassen! \bibleverse{27} Und schlägt er
seinem Knecht oder seiner Magd einen Zahn aus, so soll er sie zur
Entschädigung für ihren Zahn freilassen!«

\hypertarget{dd-ersatzpflicht-bei-tuxf6tung-oder-verletzung-eines-menschen-durch-tiere}{%
\subparagraph{dd) Ersatzpflicht bei Tötung oder Verletzung eines
Menschen durch
Tiere}\label{dd-ersatzpflicht-bei-tuxf6tung-oder-verletzung-eines-menschen-durch-tiere}}

\bibleverse{28} »Wenn ein Rind einen Mann oder eine Frau so stößt, daß
der Tod eintritt, so soll das Rind gesteinigt, sein Fleisch aber nicht
gegessen werden; der Eigentümer des Rindes jedoch bleibt straflos.
\bibleverse{29} Wenn aber das Rind schon früher stößig war und sein
Eigentümer trotz erfolgter Verwarnung es nicht gehörig gehütet hatte, so
soll das Rind, wenn es einen Mann oder eine Frau tötet, gesteinigt und
auch sein Eigentümer mit dem Tode bestraft werden. \bibleverse{30} Wenn
ihm nur eine Geldbuße auferlegt wird, so soll er als Lösegeld für sein
Leben so viel bezahlen, als ihm auferlegt wird. \bibleverse{31} Wenn das
Rind einen Knaben oder ein Mädchen stößt, so soll mit ihm nach derselben
Rechtsbestimmung verfahren werden. \bibleverse{32} Stößt das Rind aber
einen Knecht oder eine Magd, so soll sein Eigentümer ihrem Herrn dreißig
Schekel Silber bezahlen; das Rind aber soll gesteinigt werden.«

\hypertarget{ee-bestimmungen-zum-schutz-des-eigentums-des-viehstandes-oder-feldertrags-oder-des-anvertrauten-gutes}{%
\subparagraph{ee) Bestimmungen zum Schutz des Eigentums (des Viehstandes
oder Feldertrags oder des anvertrauten
Gutes)}\label{ee-bestimmungen-zum-schutz-des-eigentums-des-viehstandes-oder-feldertrags-oder-des-anvertrauten-gutes}}

\bibleverse{33} »Läßt jemand eine Grube offen stehen, oder gräbt jemand
eine Grube aus und deckt sie nicht zu, und es fällt ein Rind oder ein
Esel hinein, \bibleverse{34} so soll der Eigentümer der Grube Ersatz
leisten: mit Geld soll er den Eigentümer des Tieres entschädigen; das
tote Tier aber gehört dann ihm.~-- \bibleverse{35} Wenn jemandes Rind
das Rind eines andern so stößt, daß es stirbt, so sollen sie das lebende
Rind verkaufen und sich in den dadurch gewonnenen Erlös teilen, und auch
das tote Tier sollen sie unter sich teilen. \bibleverse{36} War es
jedoch bekannt, daß das Rind schon vorher stößig war, und hatte sein
Eigentümer trotzdem es nicht gehörig gehütet, so soll er unweigerlich
ein Rind für das Rind als Ersatz geben, das tote Tier aber soll ihm
gehören.

\bibleverse{37} Wenn jemand ein Rind oder ein Stück Kleinvieh stiehlt
und es schlachtet oder verkauft, so soll er für das eine Rind fünf
Rinder und für das eine Stück Kleinvieh vier Stück erstatten.

\bibleverse{1} Wird ein Dieb beim Einbruch betroffen und totgeschlagen,
so ist dies für den Täter nicht als Mord\textless sup title=``oder:
Blutschuld''\textgreater✲ anzusehen; \bibleverse{2} a wenn aber die
Sonne zur Zeit der Tat über ihm schon aufgegangen war, so soll ein
Mord\textless sup title=``oder: Blutschuld''\textgreater✲ für ihn als
vorliegend angenommen werden. \bibleverse{2} a wenn aber die Sonne zur
Zeit der Tat über ihm schon aufgegangen war, so soll ein
Mord\textless sup title=``oder: Blutschuld''\textgreater✲ für ihn als
vorliegend angenommen werden.

\hypertarget{section-21}{%
\section{22}\label{section-21}}

\bibleverse{3} Wird das gestohlene Tier, es sei ein Rind oder ein Esel
oder ein Stück Kleinvieh, noch lebend in seinem Besitz vorgefunden, so
soll er nur doppelten Ersatz leisten; \bibleverse{4} Läßt jemand ein
Feld oder einen Baumgarten abweiden und sein Vieh dabei frei laufen, so
daß es auf dem Felde eines andern weidet, so hat er mit dem besten
Ertrage seines Feldes und mit dem besten Ertrage seines Baumgartens
Ersatz zu leisten.~-- \bibleverse{5} Wenn Feuer ausbricht und Dornhecken
ergreift und es wird ein Garbenhaufen oder das auf dem Halm stehende
Getreide oder das Feld dadurch verbrannt, so soll der, welcher den Brand
verursacht hat, den Schaden ersetzen.«

\hypertarget{ff-veruntreuung-verlust-oder-beschuxe4digung-anvertrauten-gutes}{%
\subparagraph{ff) Veruntreuung, Verlust oder Beschädigung anvertrauten
Gutes}\label{ff-veruntreuung-verlust-oder-beschuxe4digung-anvertrauten-gutes}}

\bibleverse{6} »Wenn jemand einem andern Geld oder Wertsachen zur
Aufbewahrung übergibt und es wird dies aus dem Hause des Betreffenden
gestohlen, so hat der Dieb, wenn er ausfindig gemacht wird, doppelten
Ersatz zu leisten. \bibleverse{7} Wird aber der Dieb nicht ausfindig
gemacht, so soll der Besitzer des Hauses vor Gott\textless sup
title=``vgl. 21,6''\textgreater✲ hintreten, damit festgestellt wird, ob
er sich nicht an dem Eigentum des andern vergriffen hat!~--
\bibleverse{8} Bei jedem Fall von Veruntreuung, mag es sich um ein Rind,
einen Esel, ein Stück Kleinvieh, ein Kleidungsstück, kurz um irgend
etwas abhanden Gekommenes handeln, wovon jemand behauptet, daß es sein
Eigentum sei, soll die Angelegenheit beider vor Gott\textless sup
title=``vgl. 21,6''\textgreater✲ kommen, und wen Gott für schuldig
erklärt, der soll dem andern doppelten Ersatz leisten.~-- \bibleverse{9}
Wenn jemand einem andern einen Esel oder ein Rind oder ein Stück
Kleinvieh oder sonst irgendein Stück Vieh zur Hut\textless sup
title=``oder: zum Hüten''\textgreater✲ übergibt und dieses zu Tode oder
zu Schaden kommt oder geraubt wird, ohne daß jemand es gesehen hat,
\bibleverse{10} so soll ein Eid bei Gott zwischen den beiden Parteien
entscheiden, ob der Betreffende sich nicht am Eigentum des andern
vergriffen hat. Der Eigentümer des Tieres muß dann den Verlust
hinnehmen, und (der andere) braucht keinen Ersatz zu leisten.
\bibleverse{11} Ist es ihm aber wirklich gestohlen worden, so hat er dem
Eigentümer Ersatz zu leisten. \bibleverse{12} Ist es dagegen (von einem
Raubtier) zerrissen worden, so mag er das zerrissene Tier zum Beweis
beibringen: für das Zerrissene braucht er keinen Ersatz zu leisten.~--
\bibleverse{13} Wenn ferner jemand ein Stück Vieh von einem andern
entlehnt hat und dieses zu Schaden oder zu Tode kommt, so muß er, wenn
sein Eigentümer nicht zugegen war, unweigerlich\textless sup
title=``oder: vollen''\textgreater✲ Ersatz leisten; \bibleverse{14} war
jedoch der Eigentümer zugegen, so braucht er keinen Ersatz zu leisten.
Wenn das Tier (um Geld) gemietet war, so ist der Schaden in dem Mietgeld
eingeschlossen.«

\hypertarget{gg-mannigfache-vorschriften-betreffend-gott-sittlichkeit-und-nuxe4chstenliebe}{%
\subparagraph{gg) Mannigfache Vorschriften betreffend Gott, Sittlichkeit
und
Nächstenliebe}\label{gg-mannigfache-vorschriften-betreffend-gott-sittlichkeit-und-nuxe4chstenliebe}}

\bibleverse{15} »Wenn jemand eine Jungfrau verführt, die noch nicht
verlobt ist, und ihr beiwohnt, so muß er sie sich durch Erlegung der
Heiratsgabe zur Ehefrau erkaufen. \bibleverse{16} Wenn ihr Vater sich
aber durchaus weigert, sie ihm zu geben, so soll er das als Heiratsgabe
für Jungfrauen übliche Kaufgeld bezahlen.«

\bibleverse{17} Eine Zauberin sollst du nicht am Leben lassen.~--
\bibleverse{18} Wer einem Tiere beiwohnt, soll mit dem Tode bestraft
werden.~-- \bibleverse{19} Wer den Göttern opfert außer dem HERRN
allein, soll dem Bann verfallen.

\bibleverse{20} Einen Fremdling sollst du nicht übervorteilen und nicht
bedrücken; denn ihr selbst seid Fremdlinge im Land Ägypten gewesen.~--
\bibleverse{21} Keine Witwe oder Waise sollt ihr bedrücken.
\bibleverse{22} Wenn du sie irgendwie bedrückst und sie dann zu mir
schreien, so werde ich ihr Schreien gewißlich erhören, \bibleverse{23}
und mein Zorn wird entbrennen, und ich werde euch durch das Schwert
sterben lassen, so daß eure Frauen zu Witwen und eure Kinder zu Waisen
werden.~-- \bibleverse{24} Wenn du jemandem aus meinem Volk, einem
Armen, der neben dir wohnt, Geld leihst, so sollst du dich nicht als
Wucherer gegen ihn benehmen: ihr sollt keine Zinsen von ihm fordern.~--
\bibleverse{25} Wenn du dir von einem andern den Mantel als Pfand geben
läßt, so sollst du ihm diesen bis zum Sonnenuntergang zurückgeben;
\bibleverse{26} denn das ist seine einzige Decke, die einzige Hülle für
seinen Leib: worauf soll er sonst beim Schlafen liegen? Wenn er zu mir
schriee, so würde ich ihn erhören; denn ich bin gnädig.~--
\bibleverse{27} Gott sollst du nicht lästern und einem Fürsten in deinem
Volk nicht fluchen\textless sup title=``Apg 23,5''\textgreater✲.~--
\bibleverse{28} Mit der Abgabe von dem Überfluß deiner Tenne und von dem
Abfluß deiner Kelter sollst du nicht zögern. -- Den Erstgeborenen deiner
Söhne sollst du mir geben. \bibleverse{29} Ebenso sollst du es mit
deinen Rindern und deinem Kleinvieh halten! Sieben Tage soll (das
Erstgeborene) bei seiner Mutter bleiben, und am achten Tage sollst du es
mir darbringen.~-- \bibleverse{30} Heilige Männer sollt ihr mir sein und
Fleisch von einem auf dem Feld zerrissenen Tiere nicht essen: den Hunden
sollt ihr es vorwerfen.«

\hypertarget{hh-wahrhaftiges-und-ehrenhaftes-verhalten-besonders-vor-gericht}{%
\subparagraph{hh) Wahrhaftiges und ehrenhaftes Verhalten, besonders vor
Gericht}\label{hh-wahrhaftiges-und-ehrenhaftes-verhalten-besonders-vor-gericht}}

\hypertarget{section-22}{%
\section{23}\label{section-22}}

\bibleverse{1} »Du sollst kein falsches Gerücht verbreiten. -- Biete
dem, der eine ungerechte Sache hat, nicht die Hand, daß du ein falscher
Zeuge für ihn würdest.~-- \bibleverse{2} Du sollst dich nicht der großen
Menge zu bösem Tun anschließen und bei einem Rechtsstreit nicht so
aussagen, daß du dich nach der großen Menge richtest, um das Recht zu
beugen.~-- \bibleverse{3} Den Vornehmen sollst du in seinem Rechtshandel
nicht begünstigen.~-- \bibleverse{4} Wenn du das Rind deines Feindes
oder seinen Esel umherirrend antriffst, so sollst du ihm das Tier
unweigerlich wieder zuführen. \bibleverse{5} Wenn du den Esel deines
Widersachers unter seiner Last zusammengebrochen siehst, so hüte dich,
ihn bei dem Tier allein\textless sup title=``=~ohne
Beistand''\textgreater✲ zu lassen! Du sollst unweigerlich im Verein mit
ihm die Hilfeleistung vollbringen.~-- \bibleverse{6} Beuge nicht das
Recht eines von den Armen deines Volkes in einem Rechtshandel!~--
\bibleverse{7} Von falscher Anklage halte dich fern und hilf nicht dazu,
einen Unschuldigen, der im Recht ist, ums Leben zu bringen! denn ich
lasse den Schuldigen nicht Recht haben\textless sup title=``oder: nicht
ungestraft''\textgreater✲. \bibleverse{8} Nimm keine
Bestechungsgeschenke an; denn Geschenke machen die Sehenden blind und
verdrehen die Sache der Unschuldigen. \bibleverse{9} Einen Fremdling
sollst du nicht bedrücken! Ihr wißt ja selbst, wie einem Fremdling
zumute ist; denn ihr seid selbst Fremdlinge im Land Ägypten gewesen.«

\hypertarget{ii-bestimmungen-fuxfcr-die-sabbatjahre-festzeiten-und-opfer}{%
\subparagraph{ii) Bestimmungen für die Sabbatjahre, Festzeiten und
Opfer}\label{ii-bestimmungen-fuxfcr-die-sabbatjahre-festzeiten-und-opfer}}

\bibleverse{10} »Sechs Jahre sollst du dein Land bestellen und seinen
Ertrag einernten; \bibleverse{11} aber im siebten Jahre sollst du es
ruhen✲ lassen und es freigeben, damit die Armen deines Volkes sich davon
nähren können; und was diese übriglassen, soll das Getier des Feldes
fressen. Ebenso sollst du es mit deinen Weinbergen und mit deinen
Ölbaumgärten halten.~-- \bibleverse{12} Sechs Tage hindurch sollst du
deine Arbeit verrichten; aber am siebten Tage sollst du feiern, damit
dein Ochs und dein Esel ausruhen und der Sohn deiner Magd sowie der
Fremdling Atem schöpfen können.~-- \bibleverse{13} Auf alles, was ich
euch geboten habe, sollt ihr achtgeben. Den Namen anderer Götter aber
sollt ihr nicht aussprechen: er soll nicht über deine Lippen kommen.

\bibleverse{14} Dreimal im Jahre sollst du mir ein Fest feiern.
\bibleverse{15} Das Fest der ungesäuerten Brote sollst du beobachten:
sieben Tage lang sollst du ungesäuertes Brot essen, wie ich dir geboten
habe, zur bestimmten Zeit im Monat Abib! Denn in diesem Monat bist du
aus Ägypten ausgezogen. Man darf aber nicht mit leeren Händen vor meinem
Angesicht erscheinen. \bibleverse{16} Sodann das Fest der Ernte, der
Erstlinge deines Ackerbaus, dessen, was du auf dem Feld ausgesät hast,
und das Fest der Lese beim Ausgang des Jahres, wenn du deinen Ertrag vom
Feld einsammelst. \bibleverse{17} Dreimal im Jahr sollen alle deine
männlichen Personen vor dem Angesicht Gottes, des HERRN, erscheinen.~--
\bibleverse{18} Du sollst das Blut meiner Schlachtopfer nicht zusammen
mit gesäuertem Brot opfern, und von dem Fett meiner Festopfer soll
nichts bis zum andern Morgen übrigbleiben. \bibleverse{19} Das Beste von
den Erstlingen deiner Felder sollst du in das Haus des HERRN, deines
Gottes, bringen. -- Ein Böckchen sollst du nicht in der Milch seiner
Mutter kochen.«

\hypertarget{jj-schluuxdfermahnung-bezuxfcglich-der-austreibung-der-kanaanuxe4er-verheiuxdfung-von-beistand-und-segen-im-falle-treuen-gehorsams}{%
\subparagraph{jj) Schlußermahnung bezüglich der Austreibung der
Kanaanäer; Verheißung von Beistand und Segen im Falle treuen
Gehorsams}\label{jj-schluuxdfermahnung-bezuxfcglich-der-austreibung-der-kanaanuxe4er-verheiuxdfung-von-beistand-und-segen-im-falle-treuen-gehorsams}}

\bibleverse{20} »Wisse wohl: ich will einen Engel vor dir hergehen
lassen, um dich unterwegs zu behüten und dich an den Ort zu bringen, den
ich dir bestimmt habe. \bibleverse{21} Nimm dich vor ihm in acht,
gehorche seinen Weisungen und sei nicht widerspenstig gegen ihn; denn er
würde euch eure Verschuldungen nicht vergeben, weil ich persönlich in
ihm bin. \bibleverse{22} Doch wenn du seinen Weisungen willig gehorchst
und alles tust, was ich (dir durch ihn) gebieten werde, so will ich der
Feind deiner Feinde und der Bedränger deiner Bedränger sein.
\bibleverse{23} Wenn mein Engel nun vor dir hergeht und dich in das Land
der Amoriter, Hethiter, Pherissiter, Kanaanäer, Hewiter und Jebusiter
bringt und ich sie ausrotte: \bibleverse{24} dann wirf dich vor ihren
Göttern nicht nieder, diene ihnen nicht und ahme ihr Tun nicht nach!
Nein, du sollst ihre Götzenbilder allesamt niederreißen und ihre
Malsteine\textless sup title=``vgl. 5.Mose 7,5''\textgreater✲
zertrümmern. \bibleverse{25} Dient vielmehr dem HERRN, eurem Gott, so
will ich dein Brot und dein Wasser segnen und Krankheiten von dir
fernhalten; \bibleverse{26} keine Frau soll in deinem Lande eine
Fehlgeburt tun oder kinderlos bleiben, und ich will deine Lebenstage auf
die volle Zahl bringen.

\bibleverse{27} Meinen Schrecken will ich vor dir hergehen lassen und
alle Völker, zu denen du kommst, verzagt machen und alle deine Feinde
vor dir die Flucht ergreifen lassen. \bibleverse{28} Die Hornissen will
ich vor dir hersenden, damit sie die Hewiter, Kanaanäer und Hethiter vor
dir vertreiben. \bibleverse{29} Nicht in einem Jahr will ich sie vor dir
her vertreiben, sonst würde das Land zur Wüste\textless sup
title=``oder: Öde''\textgreater✲ werden und die wilden Tiere zu deinem
Schaden überhandnehmen; \bibleverse{30} nein, nach und nach will ich sie
vor dir vertreiben, bis du so zahlreich geworden bist, daß du das ganze
Land in Besitz nehmen kannst. \bibleverse{31} Und ich will dein Gebiet
sich ausdehnen lassen vom Schilfmeer bis zum Meer der Philister und von
der Wüste bis an den Euphratstrom; denn ich will die Bewohner des Landes
in deine Gewalt geben, daß du sie vor dir her vertreibst.
\bibleverse{32} Du darfst mit ihnen und mit ihren Göttern keinen Vertrag
schließen; \bibleverse{33} sie dürfen in deinem Lande nicht wohnen
bleiben, damit sie dich nicht zur Sünde gegen mich verführen; denn wenn
du ihren Göttern dientest, so würde das dich ins Verderben stürzen.«

\hypertarget{e-feierlicher-abschluuxdf-des-bundes-abfassung-und-verlesung-des-bundesbuches-das-bundesopfer-nebst-blutbesprengung}{%
\paragraph{e) Feierlicher Abschluß des Bundes; Abfassung und Verlesung
des Bundesbuches; das Bundesopfer nebst
Blutbesprengung}\label{e-feierlicher-abschluuxdf-des-bundes-abfassung-und-verlesung-des-bundesbuches-das-bundesopfer-nebst-blutbesprengung}}

\hypertarget{section-23}{%
\section{24}\label{section-23}}

\bibleverse{1} Hierauf gebot er dem Mose: »Steige zum HERRN\textless sup
title=``=~zu mir''\textgreater✲ herauf, du nebst Aaron, Nadab und Abihu
und siebzig von den Ältesten der Israeliten, und bringt von fern eure
Verehrung dar. \bibleverse{2} Mose aber soll dann allein nahe an den
HERRN herantreten; die anderen dagegen sollen nicht näher hinzutreten,
und auch das Volk darf nicht mit ihm heraufsteigen.«

\bibleverse{3} Hierauf kam Mose und teilte dem Volke alle Verordnungen
des HERRN und alle Rechtssatzungen mit. Da gab das ganze Volk einstimmig
die Erklärung ab: »Alle Verordnungen, die der HERR erlassen hat, wollen
wir ausführen.« \bibleverse{4} Da schrieb Mose alles, was der HERR
geboten hatte, nieder und baute am andern Morgen früh einen Altar am Fuß
des Berges und (errichtete) zwölf Malsteine entsprechend den zwölf
Stämmen Israels. \bibleverse{5} Dann erteilte er den jungen
israelitischen Männern den Auftrag, Brandopfer(tiere) herzubringen und
junge Stiere als Heilsopfer für den HERRN zu schlachten. \bibleverse{6}
Hierauf nahm Mose die eine Hälfte des Blutes und goß es in die
Opferschalen; die andere Hälfte des Blutes aber sprengte er an den
Altar. \bibleverse{7} Hierauf nahm er das Bundesbuch und las es dem
Volke laut vor; und sie erklärten: »Alles, was der HERR geboten hat,
wollen wir tun und willig erfüllen.« \bibleverse{8} Dann nahm Mose das
Blut und besprengte mit ihm das Volk, wobei er ausrief: »Dies ist das
Blut des Bundes, den der HERR mit euch auf Grund aller dieser Gebote
geschlossen hat!«

\hypertarget{die-siebzig-uxe4ltesten-der-israeliten-auf-dem-sinai-vor-gott}{%
\paragraph{Die siebzig Ältesten der Israeliten auf dem Sinai vor
Gott}\label{die-siebzig-uxe4ltesten-der-israeliten-auf-dem-sinai-vor-gott}}

\bibleverse{9} Als hierauf Mose und Aaron, Nadab und Abihu und siebzig
von den Ältesten der Israeliten hinaufgestiegen waren, \bibleverse{10}
schauten sie den Gott Israels: (der Boden) unter seinen Füßen war wie
ein Gebilde von Saphirplatten\textless sup title=``oder:
-fliesen''\textgreater✲ und wie der Himmel selbst an hellem Glanz.
\bibleverse{11} Er streckte aber seine Hand nicht aus gegen die
Auserwählten der Israeliten: nein, sie schauten Gott und aßen und
tranken.

\hypertarget{mose-verweilt-vierzig-tage-auf-dem-sinai-um-die-von-gott-beschriebenen-gesetzestafeln-in-empfang-zu-nehmen}{%
\paragraph{Mose verweilt vierzig Tage auf dem Sinai, um die von Gott
beschriebenen Gesetzestafeln in Empfang zu
nehmen}\label{mose-verweilt-vierzig-tage-auf-dem-sinai-um-die-von-gott-beschriebenen-gesetzestafeln-in-empfang-zu-nehmen}}

\bibleverse{12} Hierauf gebot der HERR dem Mose: »Steige zu mir auf den
Berg herauf und verweile dort, damit ich dir die Steintafeln mit dem
Gesetz und den Geboten gebe, die ich zu ihrer Unterweisung
aufgeschrieben habe.« \bibleverse{13} Da machte sich Mose mit seinem
Diener Josua auf den Weg und stieg auf den Berg Gottes hinauf.
\bibleverse{14} Zu den Ältesten aber hatte er gesagt: »Wartet hier auf
uns, bis wir zu euch zurückkehren. Aaron und Hur sind ja bei euch: wer
irgendeinen Rechtshandel hat, wende sich an sie.« \bibleverse{15} Als
Mose dann auf den Berg gestiegen war, verhüllte Gewölk den Berg,
\bibleverse{16} und die Herrlichkeit des HERRN ließ sich auf den Berg
Sinai nieder, und das Gewölk verhüllte den Berg sechs Tage lang; erst am
siebten Tage rief er dem Mose aus dem Gewölk heraus zu. \bibleverse{17}
Die Herrlichkeit des HERRN zeigte sich aber vor den Augen der Israeliten
wie ein verzehrendes Feuer auf der Spitze des Berges. \bibleverse{18} Da
begab sich Mose mitten in das Gewölk hinein und stieg auf den Berg
hinauf. Und Mose verweilte auf dem Berge vierzig Tage und vierzig
Nächte.

\hypertarget{die-sinaigesetze-betreffend-den-gottesdienst-und-das-priestertum-der-israeliten-kap.-25-31}{%
\subsubsection{3. Die Sinaigesetze betreffend den Gottesdienst und das
Priestertum der Israeliten (Kap.
25-31)}\label{die-sinaigesetze-betreffend-den-gottesdienst-und-das-priestertum-der-israeliten-kap.-25-31}}

\hypertarget{a-vorschriften-uxfcber-bau-und-ausstattung-des-heiligtums-stiftshuxfctte}{%
\paragraph{a) Vorschriften über Bau und Ausstattung des Heiligtums
(=~Stiftshütte)}\label{a-vorschriften-uxfcber-bau-und-ausstattung-des-heiligtums-stiftshuxfctte}}

\hypertarget{aa-aufforderung-zu-einer-freiwilligen-beisteuer-fuxfcr-die-herstellung-des-heiligtums}{%
\subparagraph{aa) Aufforderung zu einer freiwilligen Beisteuer für die
Herstellung des
Heiligtums}\label{aa-aufforderung-zu-einer-freiwilligen-beisteuer-fuxfcr-die-herstellung-des-heiligtums}}

\hypertarget{section-24}{%
\section{25}\label{section-24}}

\bibleverse{1} Der HERR sprach dann zu Mose folgendermaßen:
\bibleverse{2} »Fordere die Israeliten auf, eine Beisteuer\textless sup
title=``oder: Abgabe''\textgreater✲ an mich zu entrichten! Von einem
jeden, den sein Herz dazu treibt, sollt ihr die Abgabe an mich
annehmen\textless sup title=``oder: erheben''\textgreater✲!
\bibleverse{3} Und zwar besteht die Abgabe, die ihr von ihnen erheben
sollt, in folgendem: in Gold, Silber und Kupfer; \bibleverse{4} in
blauem und rotem Purpur und Karmesin\textless sup title=``d.h.
karmesinfarbenen Garnen oder Stoffen''\textgreater✲, in
Byssus\textless sup title=``1.Mose 41,42''\textgreater✲ und Ziegenhaar;
\bibleverse{5} in rotgefärbten Widderfellen und Seekuhhäuten; in
Akazienholz; \bibleverse{6} in Öl zur Beleuchtung\textless sup
title=``oder: für den Leuchter''\textgreater✲, in Gewürzkräutern für das
Salböl und für das wohlriechende Räucherwerk; \bibleverse{7} in
Onyxsteinen und anderen Edelsteinen zum Besatz für das Schulterkleid und
für das Brustschild. \bibleverse{8} Sie sollen mir nämlich ein Heiligtum
herstellen, damit ich mitten unter ihnen wohne. \bibleverse{9} Genau so,
wie ich dir das Musterbild der Wohnung und das Musterbild aller ihrer
Geräte zeigen werde, so sollt ihr es herstellen.«

\hypertarget{bb-anweisung-uxfcber-die-anfertigung-der-heiligen-geruxe4te-lade-tisch-leuchter}{%
\subparagraph{bb) Anweisung über die Anfertigung der heiligen Geräte
(Lade, Tisch,
Leuchter)}\label{bb-anweisung-uxfcber-die-anfertigung-der-heiligen-geruxe4te-lade-tisch-leuchter}}

\bibleverse{10} »Sie sollen\textless sup title=``oder: Du
sollst''\textgreater✲ also eine Lade\textless sup title=``vgl.
37,1-9''\textgreater✲ aus Akazienholz anfertigen, zweieinhalb Ellen
lang, anderthalb Ellen breit und anderthalb Ellen hoch. \bibleverse{11}
Du sollst sie mit feinem Gold überziehen, und zwar inwendig und
auswendig, und oben einen goldenen Kranz ringsum an ihr anbringen.
\bibleverse{12} Sodann gieße für sie vier goldene Ringe und befestige
sie unten an ihren vier Ecken\textless sup title=``oder:
Füßen''\textgreater✲, und zwar zwei Ringe an ihrer einen Seite und zwei
Ringe an ihrer andern Seite. \bibleverse{13} Weiter fertige zwei Stangen
von Akazienholz an, überziehe sie mit Gold \bibleverse{14} und stecke
diese Stangen in die Ringe an den Seiten der Lade, damit man die Lade
vermittels ihrer tragen kann. \bibleverse{15} Die Stangen sollen in den
Ringen der Lade verbleiben: sie dürfen nicht daraus entfernt werden.
\bibleverse{16} In die Lade sollst du dann das Gesetz legen, das ich dir
geben werde. \bibleverse{17} Sodann fertige eine Deckplatte aus feinem
Gold an, zweieinhalb Ellen lang und anderthalb Ellen breit.
\bibleverse{18} Weiter sollst du zwei goldene Cherube anfertigen, und
zwar in getriebener Arbeit, an den beiden Enden der Deckplatte.
\bibleverse{19} Den einen Cherub sollst du am Ende der einen Seite und
den andern Cherub am Ende der andern Seite anbringen; mit der Deckplatte
zu einem Stück verbunden sollt ihr die Cherube an den beiden Enden der
Deckplatte anbringen. \bibleverse{20} Die Cherube sollen die Flügel nach
oben hin ausgebreitet halten, so daß sie die Deckplatte mit ihren
Flügeln überdecken; ihre Gesichter sollen einander zugekehrt und
zugleich zur Deckplatte hin gerichtet sein. \bibleverse{21} Die
Deckplatte sollst du dann oben auf die Lade legen; und in die Lade
sollst du das Gesetz tun, das ich dir geben werde. \bibleverse{22}
Daselbst will ich mit dir dann zusammenkommen; und von der Deckplatte
herab, aus dem Raum zwischen den beiden Cheruben hervor, die auf der
Gesetzeslade stehen, will ich dir alles mitteilen, was ich den
Israeliten durch dich aufzutragen habe.

\bibleverse{23} Ferner sollst du einen Tisch\textless sup title=``vgl.
37,10-16''\textgreater✲ aus Akazienholz anfertigen, zwei Ellen lang,
eine Elle breit und anderthalb Ellen hoch. \bibleverse{24} Überziehe ihn
mit feinem Gold und bringe an ihm ringsum einen goldenen Kranz an.
\bibleverse{25} Sodann bringe an ihm ringsum eine
Einfassung\textless sup title=``oder: Leiste''\textgreater✲ an, die eine
Handbreit hoch ist, und an dieser Einfassung wiederum einen goldenen
Kranz ringsum. \bibleverse{26} Dann fertige für ihn vier goldene Ringe
an und befestige diese Ringe an den vier Ecken bei seinen vier Füßen.
\bibleverse{27} Dicht an der Einfassung sollen sich die Ringe befinden
zur Aufnahme der Stangen, mit denen man den Tisch tragen kann.
\bibleverse{28} Die Stangen verfertige aus Akazienholz und überziehe sie
mit Gold; mit ihnen soll der Tisch getragen werden. \bibleverse{29}
Weiter fertige die für ihn erforderlichen Schüsseln und Schalen, die
Kannen und Becher an, die zu den Trankopfern gebraucht werden; aus
feinem Gold sollst du sie herstellen. \bibleverse{30} Auf den Tisch aber
sollst du beständig Schaubrote vor mich hinlegen.

\bibleverse{31} Weiter sollst du einen Leuchter\textless sup
title=``vgl. 37,17-24''\textgreater✲ aus feinem Gold anfertigen; in
getriebener Arbeit soll der Leuchter, sein Fuß und sein Schaft,
angefertigt werden; seine Blumenkelche -- Knäufe mit Blüten -- sollen
aus einem Stück mit ihm gearbeitet sein. \bibleverse{32} Sechs
Arme\textless sup title=``oder: Röhren''\textgreater✲ sollen von seinen
Seiten ausgehen, drei Arme auf jeder Seite des Leuchters.
\bibleverse{33} Drei mandelblütenförmige Blumenkelche -- je ein Knauf
mit einer Blüte -- sollen sich an jedem Arm befinden; so soll es bei
allen sechs Armen sein, die von dem Leuchter ausgehen. \bibleverse{34}
Am Schaft selbst aber sollen sich vier mandelblütenförmige Blumenkelche
-- Knäufe mit Blüten -- befinden, \bibleverse{35} und zwar soll sich an
ihm immer ein Knauf unter jedem Paar der sechs Arme befinden, die vom
Schaft des Leuchters ausgehen. \bibleverse{36} Ihre Knäufe und Arme
sollen aus einem Stück mit ihm bestehen: der ganze Leuchter soll eine
einzige getriebene Arbeit von feinem Gold sein. \bibleverse{37} Sodann
sollst du sieben Lampen für ihn anfertigen; und man soll ihm diese
Lampen so aufsetzen, daß sie den vor dem Leuchter liegenden Raum
erleuchten. \bibleverse{38} Auch die zugehörigen Lichtscheren und
Pfannen sollen aus feinem Gold bestehen. \bibleverse{39} Aus einem
Talent feinen Goldes soll man ihn nebst allen diesen Geräten herstellen.
\bibleverse{40} Gib wohl acht, daß du alles genau nach dem Musterbild✲
anfertigst, das dir auf dem Berge gezeigt werden soll.«

\hypertarget{cc-anweisung-uxfcber-die-herstellung-der-wohnung-die-vier-decken-die-prachtdecke-und-die-drei-uxfcberdecken}{%
\subparagraph{cc) Anweisung über die Herstellung der Wohnung: Die vier
Decken (die Prachtdecke und die drei
Überdecken)}\label{cc-anweisung-uxfcber-die-herstellung-der-wohnung-die-vier-decken-die-prachtdecke-und-die-drei-uxfcberdecken}}

\hypertarget{section-25}{%
\section{26}\label{section-25}}

\bibleverse{1} »Die Wohnung\textless sup title=``d.h. das
Offenbarungszelt''\textgreater✲ aber sollst du aus zehn Teppichen
herstellen; von gezwirntem Byssus, blauem und rotem Purpur und Karmesin,
mit Cherubbildern, wie sie der Kunstweber wirkt, sollst du sie
herstellen. \bibleverse{2} Die Länge eines jeden Teppichs soll 28~Ellen
und die Breite 4~Ellen betragen; alle Teppiche sollen dieselbe Größe
haben. \bibleverse{3} Je fünf dieser Teppiche sollen zu einem Stück
zusammengefügt werden, einer an den andern. \bibleverse{4} Sodann sollst
du Schleifen von blauem Purpur am Saum des äußersten Teppichs des einen
zusammengefügten Stücks anbringen; und ebenso sollst du es am Saum des
äußersten Teppichs bei dem andern zusammengefügten Stück machen.
\bibleverse{5} Fünfzig Schleifen sollst du an dem einen Teppich
anbringen und ebenso fünfzig Schleifen am Saum des Teppichs, der zu dem
andern zusammengesetzten Stück gehört, die Schleifen müssen einander
genau gegenüberstehen. \bibleverse{6} Sodann fertige fünfzig goldene
Haken an und verbinde die Teppiche durch die Haken miteinander, so daß
die Wohnung ein Ganzes bildet.

\bibleverse{7} Weiter sollst du Teppiche aus Ziegenhaar zu einer
Zeltdecke über der Wohnung anfertigen; elf Teppiche sollst du zu diesem
Zweck anfertigen. \bibleverse{8} Die Länge eines jeden Teppichs soll
30~Ellen und die Breite 4 Ellen betragen; die elf Teppiche sollen
dieselbe Größe haben. \bibleverse{9} Dann sollst du fünf von diesen
Teppichen zu einem Stück für sich zusammenfügen und ebenso die anderen
sechs Teppiche für sich, und zwar sollst du den sechsten Teppich an der
Vorderseite des Zeltes doppelt legen. \bibleverse{10} Weiter sollst du
fünfzig Schleifen am Saum des äußersten Teppichs des einen
zusammengefügten Stückes anbringen und ebenso fünfzig Schleifen am Saum
des äußersten Teppichs des andern zusammengesetzten Stückes.
\bibleverse{11} Dann fertige fünfzig kupferne Haken an, stecke diese
Haken in die Schleifen und füge die Zeltdecke so zusammen, daß sie ein
Ganzes bildet. \bibleverse{12} Was aber das Überhangen des Überschusses
an den Zeltteppichen betrifft, so soll die Hälfte des überschüssigen
Teppichs an der Hinterseite der Wohnung herabhangen; \bibleverse{13} und
von dem, was an der Länge der Zeltteppiche überschüssig ist, soll auf
beiden Seiten je eine Elle auf den Langseiten der Wohnung überhangen und
so zu ihrer Bedeckung dienen.~-- \bibleverse{14} Außerdem sollst du für
das Zeltdach noch eine Schutzdecke von rotgefärbten Widderfellen und
oben darüber noch eine andere Schutzdecke von Seekuhhäuten anfertigen.«

\hypertarget{dd-das-holzgeruxfcst-aus-brettern-und-fuxfcnf-riegeln-bestehend}{%
\subparagraph{dd) Das Holzgerüst, aus Brettern und fünf Riegeln
bestehend}\label{dd-das-holzgeruxfcst-aus-brettern-und-fuxfcnf-riegeln-bestehend}}

\bibleverse{15} »Weiter sollst du die Bretter für die Wohnung aus
Akazienholz anfertigen; sie sollen aufrecht stehen; \bibleverse{16} die
Länge jedes Brettes soll zehn Ellen und die Breite anderthalb Ellen
betragen. \bibleverse{17} An jedem Brett sollen zwei Zapfen sitzen,
einer dem andern gegenüber eingefügt; so sollst du es an allen Brettern
der Wohnung machen. \bibleverse{18} Und zwar sollst du an Brettern für
die Wohnung herrichten: zwanzig Bretter für die Mittagsseite südwärts,
\bibleverse{19} und unter diesen zwanzig Brettern sollst du vierzig
silberne Füße\textless sup title=``oder: Fußgestelle,
Sockel''\textgreater✲ anbringen, nämlich je zwei Füße unter jedem Brett
für seine beiden Zapfen. \bibleverse{20} Ebenso für die andere Seite der
Wohnung, nämlich für die Nordseite, zwanzig Bretter \bibleverse{21} und
vierzig silberne Füße, nämlich je zwei Füße unter jedes Brett.
\bibleverse{22} Für die Hinterseite der Wohnung aber, nach Westen zu,
sollst du sechs Bretter anfertigen, \bibleverse{23} außerdem noch zwei
Bretter für die Ecken der Wohnung an der Hinterseite. \bibleverse{24}
Diese sollen unten und gleicherweise oben vollständig sein bis an den
ersten Ring hin✲. So soll es bei beiden sein: die beiden Eckstücke
sollen sie bilden. \bibleverse{25} Demnach sollen es im ganzen acht
Bretter sein und ihre Füße von Silber: sechzehn Füße, immer zwei Füße
unter jedem Brett.

\bibleverse{26} Sodann sollst du Riegel aus Akazienholz anfertigen, fünf
für die Bretter der einen Langseite der Wohnung \bibleverse{27} und fünf
Riegel für die Bretter der andern Langseite der Wohnung und fünf Riegel
für die Bretter an der Hinterseite der Wohnung gegen Westen;
\bibleverse{28} und der mittlere Riegel soll in der Mitte der Bretter
von einem Ende bis zum andern durchlaufen. \bibleverse{29} Die Bretter
aber sollst du mit Gold überziehen und die dazu gehörigen Ringe, die zur
Aufnahme der Riegel dienen, aus Gold anfertigen und auch die Riegel mit
Gold überziehen. \bibleverse{30} Dann sollst du die Wohnung in der
erforderlichen Weise aufrichten, wie es dir auf dem Berg gezeigt worden
ist.«

\hypertarget{ee-die-beiden-vorhuxe4nge-und-die-innere-ausstattung-der-wohnung}{%
\subparagraph{ee) Die beiden Vorhänge und die innere Ausstattung der
Wohnung}\label{ee-die-beiden-vorhuxe4nge-und-die-innere-ausstattung-der-wohnung}}

\bibleverse{31} »Weiter sollst du einen Vorhang von blauem und rotem
Purpur, von Karmesin und gezwirntem Byssus anfertigen, und zwar in
Kunstweberarbeit mit (eingewirkten) Cherubbildern. \bibleverse{32} Du
sollst ihn dann an vier mit Gold überzogenen Ständern\textless sup
title=``oder: Säulen''\textgreater✲ von Akazienholz aufhängen, deren
Haken von Gold sind und die auf vier silbernen Füßen✲ stehen;
\bibleverse{33} und zwar sollst du den Vorhang unter den Teppichhaken
anbringen und dorthin\textless sup title=``d.h. in den
Raum''\textgreater✲, hinter den Vorhang, die Lade mit dem
Gesetz\textless sup title=``vgl. 25,16''\textgreater✲ bringen, so daß
der Vorhang für euch eine Scheidewand zwischen dem Heiligen und dem
Allerheiligsten bildet. \bibleverse{34} Dann sollst du die Deckplatte
auf die Gesetzeslade im Allerheiligsten legen, \bibleverse{35} den Tisch
aber draußen vor dem Vorhang aufstellen und den Leuchter dem Tisch
gegenüber an die Südseite der Wohnung setzen, während du den Tisch an
die Nordseite stellst. \bibleverse{36} Ferner fertige für den Eingang
des Zeltes einen Vorhang von blauem und rotem Purpur, von Karmesin und
gezwirntem Byssus in Buntwirkerarbeit an. \bibleverse{37} Für diesen
Vorhang fertige fünf Ständer\textless sup title=``oder:
Säulen''\textgreater✲ von Akazienholz an und überziehe sie mit Gold;
ihre Haken sollen von Gold sein, und fünf kupferne Fußgestelle sollst du
für sie gießen.«

\hypertarget{ff-anweisung-uxfcber-den-brandopferaltar-den-vorhof-und-die-lieferung-von-uxf6l-fuxfcr-den-leuchter}{%
\subparagraph{ff) Anweisung über den Brandopferaltar, den Vorhof und die
Lieferung von Öl für den
Leuchter}\label{ff-anweisung-uxfcber-den-brandopferaltar-den-vorhof-und-die-lieferung-von-uxf6l-fuxfcr-den-leuchter}}

\hypertarget{section-26}{%
\section{27}\label{section-26}}

\bibleverse{1} »Den (Brandopfer-) Altar\textless sup title=``vgl.
38,1-7''\textgreater✲ sollst du aus Akazienholz anfertigen, fünf Ellen
lang und fünf Ellen breit -- viereckig✲ soll der Altar sein -- und drei
Ellen hoch. \bibleverse{2} Die zu ihm gehörenden Hörner sollst du an
seinen vier Ecken anbringen; sie sollen mit ihm aus einem Stück
bestehen; und du sollst ihn mit Kupfer überziehen. \bibleverse{3} Sodann
verfertige die zugehörigen, zur Wegräumung der Fettasche dienenden
Töpfe, sowie die zugehörigen Schaufeln, Becken, Gabeln und Pfannen; alle
erforderlichen Geräte sollst du aus Kupfer herstellen. \bibleverse{4}
Weiter fertige ein Gitterwerk, netzartig, aus Kupfer für den Altar an,
und setze an das Netzwerk vier kupferne Ringe an seine vier Ecken,
\bibleverse{5} und bringe es unterhalb der Einfassung des Altars von
unten auf an, so daß das Netzwerk bis zur halben Höhe des Altars
hinaufgeht. \bibleverse{6} Sodann fertige Tragstangen für den Altar an,
Stangen von Akazienholz, und überziehe sie mit Kupfer. \bibleverse{7}
Diese seine Stangen sollen dann in die Ringe gesteckt werden, so daß
sich die Stangen an den beiden Seiten des Altars befinden, wenn man ihn
trägt. \bibleverse{8} Du sollst ihn aus Brettern so herstellen, daß er
inwendig hohl ist; wie man es dir auf dem Berge gezeigt hat, so soll man
ihn anfertigen.

\bibleverse{9} Den Vorhof\textless sup title=``vgl.
38,9-20''\textgreater✲ der Wohnung aber sollst du so herstellen: auf der
Mittagseite, nach Süden zu, Umhänge für den Vorhof aus gezwirntem
Byssus, hundert Ellen lang, für die eine Seite; \bibleverse{10} dazu
zwanzig Ständer\textless sup title=``oder: Säulen''\textgreater✲ nebst
den zugehörigen zwanzig kupfernen Füßen\textless sup title=``oder:
Sockeln, Fußgestellen''\textgreater✲; die Nägel und Ringbänder der
Säulen müssen von Silber sein. \bibleverse{11} Ebenso auf der nördlichen
Langseite: Umhänge von hundert Ellen Länge; dazu zwanzig
Ständer\textless sup title=``oder: Säulen''\textgreater✲ nebst den
zugehörigen zwanzig kupfernen Füßen; die Nägel und Ringbänder der Säulen
müssen von Silber sein. \bibleverse{12} Ferner auf der westlichen
Breitseite des Vorhofs: Umhänge von fünfzig Ellen Länge; dazu zehn
Ständer\textless sup title=``oder: Säulen''\textgreater✲ nebst den
zugehörigen zehn Füßen. \bibleverse{13} Die Breite der östlichen
Vorderseite des Vorhofs soll fünfzig Ellen betragen, \bibleverse{14} und
zwar fünfzehn Ellen Umhänge für die eine Seite mit ihren drei Säulen und
deren drei Füßen, \bibleverse{15} und ebenso fünfzehn Ellen Umhänge für
die andere Seite mit ihren drei Säulen und deren drei Füßen.
\bibleverse{16} Am Eingang zum Vorhof aber soll sich ein Vorhang von
zwanzig Ellen Breite befinden, aus blauem und rotem Purpur, aus Karmesin
und gezwirntem Byssus in Buntwirkerarbeit gefertigt; dazu vier Säulen
nebst deren vier Füßen. \bibleverse{17} Alle Säulen rings um den Vorhof
sollen mit silbernen Ringbändern versehen sein, auch ihre Nägel von
Silber, ihre Füße aber von Kupfer. \bibleverse{18} Die Länge des Vorhofs
soll hundert Ellen, die Breite je fünfzig und die Höhe fünf Ellen
betragen, nämlich Umhänge von gezwirntem Byssus; ihre Füße aber sollen
von Kupfer sein. \bibleverse{19} Alle Gerätschaften der Wohnung für den
gesamten Dienst an ihr, auch alle ihre Pflöcke und alle Pflöcke des
Vorhofs sollen von Kupfer sein.

\bibleverse{20} Sodann befiehl du den Israeliten, dir ganz reines Öl aus
zerstoßenen Oliven für den Leuchter zu bringen, damit man beständig
Lampen aufsetzen kann. \bibleverse{21} Im Offenbarungszelt, außerhalb
des Vorhangs, der sich vor (der Lade mit) dem Gesetz befindet, soll
Aaron mit seinen Söhnen die Lampen zurechtmachen, damit sie vom Abend
bis zum Morgen vor dem HERRN brennen. Diese Verordnung soll ewige
Geltung für die Israeliten von Geschlecht zu Geschlecht haben!«

\hypertarget{b-gesetzgebung-uxfcber-das-aaronitische-priestertum-kap.-28-31}{%
\paragraph{b) Gesetzgebung über das aaronitische Priestertum (Kap.
28-31)}\label{b-gesetzgebung-uxfcber-das-aaronitische-priestertum-kap.-28-31}}

\hypertarget{aa-anweisung-uxfcber-die-priesterliche-kleidung-aarons-und-seiner-suxf6hne}{%
\subparagraph{aa) Anweisung über die priesterliche Kleidung Aarons und
seiner
Söhne}\label{aa-anweisung-uxfcber-die-priesterliche-kleidung-aarons-und-seiner-suxf6hne}}

\hypertarget{section-27}{%
\section{28}\label{section-27}}

\bibleverse{1} »Du aber sollst deinen Bruder Aaron nebst seinen Söhnen
aus der Mitte der Israeliten zu dir herantreten lassen, damit er mir als
Priester diene, nämlich Aaron und seine Söhne Nadab und Abihu, Eleasar
und Ithamar. \bibleverse{2} Du sollst also für deinen Bruder Aaron
heilige Kleider anfertigen lassen, ihm zur Ehre\textless sup
title=``oder: Würde''\textgreater✲ und zum Schmuck. \bibleverse{3}
Besprich dich also mit allen kunstverständigen Personen, die ich mit
kunstsinnigem Geist erfüllt habe, daß sie die Kleider Aarons anfertigen,
damit man ihn darin weihe und er mir als Priester dienen kann.
\bibleverse{4} Folgendes aber sind die Kleidungsstücke, die sie
anfertigen sollen: das Brustschild\textless sup title=``oder: die
Tasche''\textgreater✲, das Schulterkleid, das Obergewand, das Unterkleid
aus gewürfeltem Stoff, der Kopfbund und der Gürtel. Diese heiligen
Kleider sollen sie für deinen Bruder Aaron und seine Söhne anfertigen,
damit er mir als Priester diene; \bibleverse{5} und zwar sollen sie
Gold, blauen und roten Purpur, Karmesin und Byssus\textless sup
title=``vgl. 1.Mose 41,42''\textgreater✲ dazu verwenden.«

\hypertarget{bb-das-schulterkleid-ephod}{%
\subparagraph{bb) Das Schulterkleid
(Ephod)}\label{bb-das-schulterkleid-ephod}}

\bibleverse{6} »Das Schulterkleid sollen sie aus Gold, blauem und rotem
Purpur, Karmesin und gezwirntem Byssus in Kunstweberarbeit anfertigen.
\bibleverse{7} Es soll ein Paar zusammenfügbarer Schulterstücke haben
und an seinen beiden (oberen) Enden mittels derselben zusammengefügt
werden. \bibleverse{8} Die Binde, die sich an ihm befindet und dazu
dient, es fest anzulegen, soll von gleicher Arbeit und aus einem Stück
mit ihm bestehen, nämlich aus Goldfäden, blauem und rotem Purpur,
Karmesin und gezwirntem Byssus. \bibleverse{9} Ferner sollst du zwei
Onyxsteine nehmen und die Namen der Söhne Israels in sie
eingraben\textless sup title=``oder: einstechen''\textgreater✲,
\bibleverse{10} sechs ihrer Namen auf den einen Stein und die sechs
übrigen Namen auf den andern Stein, und zwar nach ihrer Geburtsfolge.
\bibleverse{11} In Steinschneiderarbeit, mittels Siegelstecherkunst,
sollst du die beiden Steine mit den Namen der Söhne Israels stechen
lassen, und mit einer Einfassung von goldenem Flechtwerk sollst du sie
versehen. \bibleverse{12} Dann sollst du die beiden Steine auf die
beiden Schulterstücke des Schulterkleides als Steine des Gedenkens
an\textless sup title=``oder: für''\textgreater✲ die Israeliten setzen,
damit Aaron ihre Namen vor dem HERRN auf seinen beiden Schultern zur
Erinnerung trägt. \bibleverse{13} Ferner sollst du Geflechte von
Golddraht anfertigen \bibleverse{14} und zwei Kettchen von feinem Gold;
in Gestalt von gedrehten Schnüren sollst du sie anfertigen und diese aus
Schnüren geflochtenen Kettchen an den Goldgeflechten befestigen.«

\hypertarget{cc-das-brustschild-mit-zubehuxf6r}{%
\subparagraph{cc) Das Brustschild mit
Zubehör}\label{cc-das-brustschild-mit-zubehuxf6r}}

\bibleverse{15} »Sodann fertige das Orakel-Brustschild in
Kunstweberarbeit an; ganz so, wie das Schulterkleid gearbeitet ist,
sollst du es anfertigen, nämlich aus Gold, blauem und rotem Purpur,
Karmesin und gezwirntem Byssus sollst du es herstellen. \bibleverse{16}
Viereckig\textless sup title=``oder: quadratförmig''\textgreater✲ soll
es sein, doppelt gelegt, eine Spanne lang und eine Spanne breit.
\bibleverse{17} Besetze es mit einem Besatz von Edelsteinen in vier
Reihen von Steinen; eine Reihe: ein Karneol, ein Topas und ein Smaragd
sollen die erste Reihe bilden; \bibleverse{18} die zweite Reihe: ein
Rubin, ein Saphir und ein Jaspis; \bibleverse{19} die dritte Reihe: ein
Hyazinth, ein Achat und ein Amethyst; \bibleverse{20} die vierte Reihe:
ein Chrysolith, ein Soham\textless sup title=``1.Mose
2,12''\textgreater✲ und ein Onyx; in Goldgeflecht sollen sie bei ihrer
Einsetzung gefaßt sein. \bibleverse{21} Die Steine sollen also
entsprechend den Namen der Söhne Israels zwölf sein, nach deren Namen;
mittels Siegelstecherkunst sollen sie, ein jeder mit seinem besonderen
Namen nach den zwölf Stämmen versehen sein. \bibleverse{22} An dem
Brustschild sollst du dann schnurähnlich geflochtene Kettchen von feinem
Gold befestigen. \bibleverse{23} Du sollst nämlich zwei goldene Ringe an
dem Brustschilde anbringen und diese zwei Ringe an den beiden (oberen)
Ecken des Brustschildes befestigen. \bibleverse{24} Hierauf befestige
die beiden goldenen Schnüre an den beiden Ringen, die sich an den
(oberen) Ecken des Brustschildes befinden, \bibleverse{25} und befestige
die beiden (anderen) Enden der beiden Schnüre an den beiden Geflechten
und diese wiederum an den Schulterstücken des Schulterkleides auf dessen
Vorderseite. \bibleverse{26} Dann fertige noch zwei goldene Ringe an und
setze sie an die beiden (unteren) Ecken des Brustschildes, und zwar an
seinen inneren Saum, der dem Schulterkleide zugekehrt ist.
\bibleverse{27} Fertige dann noch zwei goldene Ringe an und setze sie an
die beiden Schulterstücke des Schulterkleides, unten an seine
Vorderseite, dicht bei der Stelle, wo das Schulterkleid zusammengeht,
oberhalb der Binde des Schulterkleides. \bibleverse{28} Hierauf knüpfe
man das Brustschild mit seinen Ringen vermittels einer Schnur von blauem
Purpur an die Ringe des Schulterkleides, so daß das Brustschild über der
Binde des Schulterkleides fest anliegt und das Brustschild sich nicht
von seiner Stelle auf dem Schulterkleide verschieben kann.
\bibleverse{29} Aaron soll so die Namen der Söhne Israels an dem
Orakel-Brustschild\textless sup title=``vgl. V.15''\textgreater✲ auf
seinem Herzen tragen, sooft er in das Heiligtum hineingeht, zur
beständigen Erinnerung vor dem HERRN. \bibleverse{30} In das
Orakel-Brustschild aber sollst du die (heiligen Lose) Urim und Thummim
tun, damit sie auf dem Herzen\textless sup title=``=~auf der
Brust''\textgreater✲ Aarons liegen, sooft er vor den HERRN tritt; und
Aaron soll so das Orakel für die Israeliten beständig vor dem HERRN auf
seinem Herzen tragen.«

\hypertarget{dd-das-obergewand-zu-dem-schulterkleid}{%
\subparagraph{dd) Das Obergewand zu dem
Schulterkleid}\label{dd-das-obergewand-zu-dem-schulterkleid}}

\bibleverse{31} »Sodann sollst du das Obergewand zu dem Schulterkleide
ganz aus blauem Purpur anfertigen. \bibleverse{32} Seine Kopföffnung
soll sich in der Mitte befinden, und rings um diese Öffnung soll ein
Saum in Weberarbeit gehen; eine Öffnung soll es haben wie die eines
Panzerhemdes, damit es nicht einreißt. \bibleverse{33} Unten an seinem
Saum sollst du Granatäpfel aus blauem und rotem Purpur und Karmesin
ringsum anbringen und zwischen ihnen goldene Glöckchen ringsum,
\bibleverse{34} so daß am ganzen Saum des Obergewandes ringsum immer auf
ein goldenes Glöckchen ein Granatapfel folgt. \bibleverse{35} Aaron soll
(dieses Kleid) tragen, um den heiligen Dienst darin zu versehen, damit
man es klingeln hört, sooft er in das Heiligtum vor den HERRN hineingeht
und sooft er hinausgeht, damit er nicht stirbt.«

\hypertarget{ee-stirnblatt-unterkleid-kopfbund-und-guxfcrtel}{%
\subparagraph{ee) Stirnblatt, Unterkleid, Kopfbund und
Gürtel}\label{ee-stirnblatt-unterkleid-kopfbund-und-guxfcrtel}}

\bibleverse{36} »Weiter sollst du ein Stirnblatt aus feinem Gold
anfertigen und auf ihm mittels Siegelstecherarbeit die Worte
eingraben\textless sup title=``oder: einstechen''\textgreater✲: ›Dem
HERRN geweiht‹. \bibleverse{37} Du sollst es dann mit einer Schnur von
blauem Purpur versehen, damit es am Kopfbund angebracht werden kann: an
dessen Vorderseite soll es sich befinden, \bibleverse{38} und zwar soll
es auf der Stirn Aarons liegen, damit Aaron die Verfehlungen in betreff
der heiligen Gaben, welche die Israeliten weihen, auf sich nimmt, welche
heiligen Gaben sie auch darbringen mögen. Es soll also beständig auf
seiner Stirn liegen, um die Israeliten wohlgefällig vor dem HERRN zu
machen.~-- \bibleverse{39} Sodann sollst du das Unterkleid aus Byssus
würfelförmig gemustert weben und einen Kopfbund aus Byssus anfertigen,
auch einen Gürtel in Buntwirkerarbeit herstellen.«

\hypertarget{ff-die-kleidung-der-suxf6hne-aarons-allgemeines}{%
\subparagraph{ff) Die Kleidung der Söhne Aarons;
Allgemeines}\label{ff-die-kleidung-der-suxf6hne-aarons-allgemeines}}

\bibleverse{40} »Ferner sollst du für die Söhne Aarons Unterkleider
herstellen und ihnen Gürtel und hohe Mützen zur Ehre\textless sup
title=``oder: Würde''\textgreater✲ und zum Schmuck anfertigen.
\bibleverse{41} Dann sollst du deinen Bruder Aaron und ebenso seine
Söhne damit bekleiden und sollst sie salben, sie in ihr Amt einsetzen
und sie weihen, damit sie mir als Priester dienen. \bibleverse{42} Auch
fertige ihnen leinene Unterbeinkleider an zur Verhüllung ihrer Blöße;
diese sollen von den Hüften bis an die Schenkel reichen, \bibleverse{43}
und Aaron sowie seine Söhne sollen sie tragen, sooft sie in das
Offenbarungszelt hineingehen oder an den Altar treten, um den Dienst im
Heiligtum zu verrichten, damit sie keine Verschuldung auf sich laden und
nicht sterben müssen. Diese Verordnung soll für ihn und seine Nachkommen
nach ihm ewige Gültigkeit haben.«

\hypertarget{gg-anweisung-fuxfcr-die-weihe-der-priester}{%
\subparagraph{gg) Anweisung für die Weihe der
Priester}\label{gg-anweisung-fuxfcr-die-weihe-der-priester}}

\hypertarget{section-28}{%
\section{29}\label{section-28}}

\bibleverse{1} »Behufs ihrer Weihe für den Priesterdienst, den sie mir
zu verrichten haben, sollst du folgendermaßen verfahren: Nimm einen
jungen Stier und zwei fehlerlose Widder, \bibleverse{2} ungesäuertes
Brot und ungesäuerte, mit Öl gemengte\textless sup title=``oder:
eingerührte''\textgreater✲ Kuchen sowie ungesäuerte, mit Öl bestrichene
Fladen; aus feinem Weizenmehl sollst du sie bereiten. \bibleverse{3}
Lege sie dann in einen Korb und bringe sie in dem Korbe herbei, dazu den
jungen Stier und die beiden Widder. \bibleverse{4} Dann laß Aaron und
seine Söhne an den Eingang des Offenbarungszeltes treten und laß sie
eine Abwaschung mit Wasser an sich vornehmen. \bibleverse{5} Hierauf
nimm die heiligen Kleider und laß Aaron das Unterkleid und das zum
Schulterkleid gehörige Obergewand sowie das Schulterkleid selbst nebst
dem Brustschild anlegen und gürte ihm die Binde des Schulterkleides fest
um; \bibleverse{6} setze ihm dann den Kopfbund aufs Haupt und befestige
das heilige Diadem am Kopfbund. \bibleverse{7} Hierauf nimm das Salböl,
gieße ihm (etwas davon) aufs Haupt und salbe ihn so. \bibleverse{8}
Sodann laß auch seine Söhne herantreten und sich die Unterkleider
anlegen; \bibleverse{9} umgürte sie mit dem Gürtel {[}Aaron und seine
Söhne{]} und laß sie die hohen Mützen aufsetzen, damit ihnen das
Priestertum kraft einer ewig gültigen Einsetzung zusteht. Wenn du so
Aaron und seine Söhne in ihr Amt eingesetzt hast\textless sup
title=``vgl. 28,41''\textgreater✲, \bibleverse{10} sollst du den jungen
Stier vor das Offenbarungszelt führen lassen, und Aaron nebst seinen
Söhnen sollen ihre Hände fest auf den Kopf des Stieres legen.
\bibleverse{11} Dann schlachte den Stier vor dem HERRN am Eingang des
Offenbarungszeltes, \bibleverse{12} nimm etwas von dem Blut des Stieres
und streiche es mit deinem Finger an die Hörner des Altars; alles übrige
Blut aber schütte an den Fuß des Altars. \bibleverse{13} Sodann nimm das
gesamte Fett, welches die Eingeweide überzieht, sowie den Leberlappen
und die beiden Nieren samt dem daransitzenden Fett und laß es auf dem
Altar in Rauch aufgehen; \bibleverse{14} das Fleisch des Stieres aber
sowie sein Fell und den Inhalt seiner Gedärme verbrenne im Feuer
außerhalb des Lagers: es ist ein Sündopfer. \bibleverse{15} Dann nimm
den einen Widder, und Aaron nebst seinen Söhnen sollen ihre Hände fest
auf den Kopf des Widders legen. \bibleverse{16} Hierauf schlachte den
Widder, nimm sein Blut und sprenge es an den Altar ringsum;
\bibleverse{17} den Widder aber zerlege in seine Stücke, wasche seine
Eingeweide und Beine ab und lege sie zu seinen übrigen Stücken und zu
seinem Kopf \bibleverse{18} und laß dann den ganzen Widder auf dem Altar
in Rauch aufgehen: es ist ein Brandopfer für den HERRN, ein lieblicher
Geruch; ein Feueropfer ist es für den HERRN. \bibleverse{19} Hierauf
nimm den zweiten Widder, und Aaron nebst seinen Söhnen sollen ihre Hände
fest auf den Kopf des Widders legen. \bibleverse{20} Darauf schlachte
den Widder, nimm etwas von seinem Blut und bestreiche damit das rechte
Ohrläppchen Aarons und das rechte Ohrläppchen seiner Söhne sowie den
Daumen ihrer rechten Hand und die große Zehe ihres rechten Fußes; das
übrige Blut aber sprenge auf den Altar ringsum. \bibleverse{21} Sodann
nimm etwas von dem Blut, das sich auf dem Altar befindet, sowie von dem
Salböl und besprenge mit ihm Aaron und seine Kleider, ebenso seine Söhne
und deren Kleider, damit er und seine Kleider und ebenso seine Söhne und
deren Kleider geheiligt\textless sup title=``oder:
geweiht''\textgreater✲ sind. \bibleverse{22} Nimm dann von dem Widder
das Fett, nämlich den Fettschwanz und das Fett, welches die Eingeweide
überzieht, sowie den Leberlappen und die beiden Nieren samt dem
daransitzenden Fett sowie die rechte Keule -- denn es ist ein
Einweihungswidder --, \bibleverse{23} dazu einen Laib Brot, einen mit Öl
gemengten Brotkuchen und einen Fladen aus dem Korb mit den ungesäuerten
Broten, der vor den HERRN hingestellt ist; \bibleverse{24} lege dies
alles dem Aaron und seinen Söhnen in\textless sup title=``oder:
auf''\textgreater✲ die Hände und laß es als Webeopfer\textless sup
title=``vgl. 3.Mose 8,27''\textgreater✲ vor dem HERRN weben✲.
\bibleverse{25} Dann nimm es ihnen wieder aus den Händen und laß es auf
dem Altar über dem Brandopfer in Rauch aufgehen zu einem lieblichen
Geruch vor dem HERRN: ein Feueropfer ist es für den HERRN.
\bibleverse{26} Hierauf nimm die Brust von dem Einweihungswidder, der
für Aaron bestimmt ist, und webe sie als Webeopfer vor dem HERRN: dann
soll sie dir als Anteil zufallen. \bibleverse{27} Darum sollst du die
Brust des Webeopfers und die Hebekeule, die von dem für Aaron und für
seine Söhne bestimmten Einweihungswidder gewebt und als Hebe entnommen
sind, für geweiht erklären: \bibleverse{28} sie sollen für Aaron und
seine Söhne eine von seiten der Israeliten ewig zu leistende Gebühr
sein; denn es ist ein Hebeopfer und soll seitens der Israeliten als
Hebeopfer von ihren Heilsopfern abgegeben werden, als ihr Hebeopfer für
den HERRN.

\bibleverse{29} Die heiligen Kleider Aarons aber sollen nach ihm an
seine Söhne kommen, damit man sie darin salbe und sie darin in ihr Amt
einsetze\textless sup title=``vgl. 28,41''\textgreater✲. \bibleverse{30}
Sieben Tage lang soll sie derjenige von seinen Söhnen tragen, der an
seiner Statt Priester wird und in das Offenbarungszelt hineingehen wird,
um den Dienst im Heiligtum zu verrichten.

\bibleverse{31} Den Einweihungswidder aber sollst du nehmen und sein
Fleisch an heiliger Stätte kochen. \bibleverse{32} Dann sollen Aaron und
seine Söhne das Fleisch des Widders samt dem Brot, das sich in dem Korbe
befindet, am Eingang des Offenbarungszeltes verzehren: \bibleverse{33}
sie sollen die Stücke verzehren, durch welche die Sühne für sie bewirkt
wurde, als man sie in ihr Amt einsetzte\textless sup title=``vgl.
28,41''\textgreater✲ und sie weihte; aber ein Fremder✲ darf nicht davon
essen, denn sie sind heilig. \bibleverse{34} Und wenn von dem Fleisch
des Einweihungsopfers oder von dem Brot etwas bis zum Morgen
übrigbleibt, so sollst du das Übriggebliebene im Feuer verbrennen: es
darf nicht mehr gegessen werden, denn es ist heilig. \bibleverse{35} So
also sollst du mit Aaron und seinen Söhnen verfahren, genau so, wie ich
dir geboten habe: sieben Tage soll die Einweihung dauern!«

\hypertarget{hh-die-entsuxfcndigung-und-salbung-des-brandopferaltars}{%
\subparagraph{hh) Die Entsündigung und Salbung des
Brandopferaltars}\label{hh-die-entsuxfcndigung-und-salbung-des-brandopferaltars}}

\bibleverse{36} »Und für jeden Tag sollst du einen jungen Stier als
Sündopfer zum Vollzug der Sühne darbringen und den Altar entsündigen,
indem du die Sühngebräuche an ihm vollziehst, und sollst ihn salben, um
ihn zu weihen. \bibleverse{37} Sieben Tage hindurch sollst du die
Sühngebräuche am Altar vornehmen, um ihn zu weihen, damit der Altar
hochheilig werde: jeder, der den Altar berührt, soll dem Heiligtum
verfallen sein!«

\hypertarget{ii-das-tuxe4gliche-morgen--und-abend-brand--trank--und-speisopfer}{%
\subparagraph{ii) Das tägliche Morgen- und Abend-Brand-, Trank- und
Speisopfer}\label{ii-das-tuxe4gliche-morgen--und-abend-brand--trank--und-speisopfer}}

\bibleverse{38} »Folgendes aber ist es, was du auf dem Altar opfern
sollst: zwei einjährige Lämmer an jedem Tage ohne Ausnahme.
\bibleverse{39} Das eine Lamm sollst du am Morgen opfern, das andere
gegen Abend\textless sup title=``vgl. 12,6''\textgreater✲,
\bibleverse{40} und außerdem ein Zehntel Epha Feinmehl, das mit einem
Viertel Hin Öl von zerstoßenen Oliven gemengt ist; und als Trankopfer
soll ein Viertel Hin Wein zu dem einen\textless sup title=``oder:
ersten''\textgreater✲ Lamm kommen. \bibleverse{41} Das zweite Lamm aber
sollst du gegen Abend opfern -- mit dem Speisopfer und dem zugehörigen
Trankopfer sollst du es dabei halten wie am Morgen -- zu einem
lieblichen Geruch, als ein Feueropfer für den HERRN: \bibleverse{42} ein
regelmäßiges Brandopfer (soll es bei euch sein) für eure Geschlechter
vor dem HERRN am Eingang zum Offenbarungszelt, wo ich mit euch in
Verkehr treten werde, um dort mit dir zu reden.«

\hypertarget{jj-schluuxdfwort}{%
\subparagraph{jj) Schlußwort}\label{jj-schluuxdfwort}}

\bibleverse{43} »Ich will dort nämlich mit den Israeliten in Verkehr
treten\textless sup title=``oder: mich den Israeliten
offenbaren''\textgreater✲, und es (das Zelt) wird durch meine
Herrlichkeit geheiligt werden. \bibleverse{44} Ich will also das
Offenbarungszelt und den Altar heiligen; auch Aaron und seine Söhne will
ich heiligen, damit sie mir als Priester dienen; \bibleverse{45} und ich
will inmitten der Israeliten wohnen und will ihr Gott sein,
\bibleverse{46} und sie sollen erkennen, daß ich, der HERR, ihr Gott
bin, der ich sie aus Ägypten hinausgeführt habe, um mitten unter ihnen
zu wohnen, ich, der HERR, ihr Gott.«

\hypertarget{kk-nachtruxe4ge-zur-gesetzgebung-uxfcber-das-heiligtum-kap.-30-31}{%
\subparagraph{kk) Nachträge zur Gesetzgebung über das Heiligtum (Kap.
30-31)}\label{kk-nachtruxe4ge-zur-gesetzgebung-uxfcber-das-heiligtum-kap.-30-31}}

\hypertarget{vorschriften-bezuxfcglich-des-ruxe4ucheraltars}{%
\paragraph{Vorschriften bezüglich des
Räucheraltars}\label{vorschriften-bezuxfcglich-des-ruxe4ucheraltars}}

\hypertarget{section-29}{%
\section{30}\label{section-29}}

\bibleverse{1} »Sodann sollst du einen Altar\textless sup title=``vgl.
37,25-28''\textgreater✲ herstellen, um Räucherwerk auf ihm zu
verbrennen; aus Akazienholz sollst du ihn anfertigen; \bibleverse{2}
eine Elle lang und eine Elle breit, viereckig✲ soll er sein und zwei
Ellen hoch; seine Hörner sollen aus einem Stück mit ihm bestehen.
\bibleverse{3} Du sollst ihn, sowohl seine Platte als auch die Wände
ringsum und die Hörner, mit feinem Gold überziehen und einen goldenen
Kranz\textless sup title=``oder: Leiste''\textgreater✲ ringsum an ihm
anbringen. \bibleverse{4} Zwei goldene Ringe sollst du für ihn
anfertigen und sie unterhalb seines Kranzes an seinen beiden Seiten
anbringen; die sollen zur Aufnahme der Stangen dienen, mittels derer man
ihn tragen kann. \bibleverse{5} Die Stangen sollst du aus Akazienholz
anfertigen und sie mit Gold überziehen. \bibleverse{6} Du sollst den
Altar dann vor dem Vorhang aufstellen, der sich vor der Gesetzeslade
befindet\textless sup title=``vgl. 25,16''\textgreater✲, der Deckplatte
gegenüber, die über dem Gesetz liegt, woselbst ich mit dir in Verkehr
treten\textless sup title=``vgl. 29,43''\textgreater✲ werde.
\bibleverse{7} Aaron soll dann auf ihm wohlriechendes Räucherwerk
verbrennen; an jedem Morgen, wenn er die Lampen zurechtmacht, soll er es
verbrennen; \bibleverse{8} ebenso soll Aaron es verbrennen, wenn er
gegen Abend die Lampen aufsetzt: ein regelmäßiges Rauchopfer vor dem
HERRN soll es für ewige Zeiten sein. \bibleverse{9} Ihr dürft kein
ungehöriges Räucherwerk und kein Brand- oder Speisopfer auf ihm
darbringen; auch kein Trankopfer dürft ihr auf ihm ausgießen.
\bibleverse{10} Aaron soll einmal im Jahr die Sühnehandlung an seinen
Hörnern vornehmen; mit dem Blut des Versöhnungsopfers soll er einmal im
Jahr die Sühnehandlung an ihm vornehmen, von Geschlecht zu Geschlecht:
hochheilig ist er dem HERRN.«

\hypertarget{vorschriften-bezuxfcglich-der-erhebung-einer-kopfsteuer-fuxfcr-das-heiligtum-bei-der-musterung-des-volkes}{%
\paragraph{Vorschriften bezüglich der Erhebung einer Kopfsteuer für das
Heiligtum bei der Musterung des
Volkes}\label{vorschriften-bezuxfcglich-der-erhebung-einer-kopfsteuer-fuxfcr-das-heiligtum-bei-der-musterung-des-volkes}}

\bibleverse{11} Hierauf sagte der HERR zu Mose folgendes:
\bibleverse{12} »Wenn du die Kopfzahl der Israeliten, soweit sie
gemustert werden, aufnimmst, so sollen sie ein jeder ein Lösegeld für
ihr Leben dem HERRN bei ihrer Musterung entrichten, damit keine schlimme
Heimsuchung bei\textless sup title=``oder: trotz''\textgreater✲ ihrer
Musterung über sie kommt. \bibleverse{13} Es soll also ein jeder, der
sich der Musterung zu unterziehen hat, einen halben Schekel entrichten
-- nach dem Schekel des Heiligtums, wobei zwanzig Gera auf den Schekel
gehen --, einen halben Schekel als Abgabe an den HERRN. \bibleverse{14}
Jeder, der sich der Musterung zu unterziehen hat, von zwanzig Jahren an
und darüber, soll die Abgabe an den HERRN entrichten: \bibleverse{15}
der Reiche soll nicht mehr und der Arme nicht weniger als einen halben
Schekel geben, wenn ihr die Abgabe an den HERRN entrichtet, um die
Deckung✲ eures Lebens zu bewirken. \bibleverse{16} Du sollst also dieses
Sühnegeld von den Israeliten erheben und es zur Bestreitung der Kosten
des Dienstes am Offenbarungszelt verwenden; so wird es den Israeliten
zum (gnädigen) Gedenken vor dem HERRN dienen, um Deckung eures Lebens zu
bewirken.«

\hypertarget{vorschriften-bezuxfcglich-des-kupfernen-waschbeckens-fuxfcr-die-priester}{%
\paragraph{Vorschriften bezüglich des kupfernen Waschbeckens für die
Priester}\label{vorschriften-bezuxfcglich-des-kupfernen-waschbeckens-fuxfcr-die-priester}}

\bibleverse{17} Weiter sagte der HERR zu Mose folgendes: \bibleverse{18}
»Fertige auch ein kupfernes Becken\textless sup title=``vgl.
37,8''\textgreater✲ nebst einem kupfernen Gestell dazu für die
Waschungen an, stelle es zwischen dem Offenbarungszelt und dem Altar auf
und tu Wasser hinein, \bibleverse{19} damit Aaron und seine Söhne ihre
Hände und Füße daraus\textless sup title=``oder: darin''\textgreater✲
waschen; \bibleverse{20} sooft sie in das Offenbarungszelt hineingehen,
sollen sie sich mit Wasser waschen, damit sie nicht sterben, oder auch,
wenn sie an den Altar treten, um ihren Dienst zu verrichten, indem sie
Feueropfer für den HERRN in Rauch aufgehen lassen. \bibleverse{21} Da
sollen sie sich dann ihre Hände und Füße waschen, damit sie nicht
sterben; und dies soll eine ewiggültige Verordnung für sie sein, für
Aaron und seine Nachkommen von Geschlecht zu Geschlecht.«

\hypertarget{zubereitung-und-verwendung-des-heiligen-salbuxf6ls}{%
\paragraph{Zubereitung und Verwendung des heiligen
Salböls}\label{zubereitung-und-verwendung-des-heiligen-salbuxf6ls}}

\bibleverse{22} Weiter gebot der HERR dem Mose folgendes:
\bibleverse{23} »Nimm du dir Wohlgerüche von der besten Sorte, nämlich
Stakte\textless sup title=``oder: Tropfharz, d.h. von selbst
ausgeflossene Myrrhe''\textgreater✲ fünfhundert Schekel, wohlriechenden
Zimt halb soviel, also zweihundertundfünfzig Schekel, ferner
wohlriechenden Kalmus\textless sup title=``oder:
Balsamrohr''\textgreater✲ ebenfalls zweihundertundfünfzig Schekel
\bibleverse{24} und Kassia fünfhundert Schekel nach dem Gewicht des
Heiligtums, dazu ein Hin Olivenöl, \bibleverse{25} und stelle daraus ein
heiliges Salböl her, eine Salbenmischung, wie sie der Salbenmischer
herstellt: heiliges Salböl soll es sein. \bibleverse{26} Du sollst damit
das Offenbarungszelt und die Gesetzeslade salben, \bibleverse{27} ferner
den Tisch samt allen seinen Geräten, den Leuchter samt den zugehörigen
Geräten und den Räucheraltar, \bibleverse{28} ferner den Brandopferaltar
samt allen seinen Geräten und das Becken nebst seinem Gestell.
\bibleverse{29} So sollst du sie heiligen, damit sie hochheilig werden:
jeder, der sie berührt, soll dem Heiligtum verfallen sein!
\bibleverse{30} Auch Aaron und seine Söhne sollst du salben und sie
dadurch zu Priestern für meinen Dienst weihen. \bibleverse{31} Den
Israeliten aber sollst du folgendes gebieten: ›Als ein mir heiliges
Salböl soll dieses euch für alle eure Geschlechter gelten!
\bibleverse{32} Auf keines Menschen Leib darf es gegossen werden! und
ihr dürft solches Salböl nicht in der gleichen Zusammensetzung für euren
eigenen Gebrauch bereiten: es ist heilig und soll euch auch als heilig
gelten! \bibleverse{33} Wer ein gleiches durch Mischung herstellt und
etwas davon an eine unbefugte Person bringt, der soll aus seinen
Volksgenossen ausgerottet werden!‹«

\hypertarget{zubereitung-und-verwendung-des-heiligen-ruxe4ucherwerks}{%
\paragraph{Zubereitung und Verwendung des heiligen
Räucherwerks}\label{zubereitung-und-verwendung-des-heiligen-ruxe4ucherwerks}}

\bibleverse{34} Weiter gebot der HERR dem Mose folgendes: »Nimm dir
Gewürzkräuter, nämlich Stakte\textless sup title=``vgl.
V.23''\textgreater✲, Räucherklaue, Galban, {[}Gewürzkräuter{]} und
reinen Weihrauch, alle zu gleichen Teilen, \bibleverse{35} und stelle
daraus ein Räucherwerk her, eine würzige Mischung, wie sie der
Salbenmischer herstellt, mit (etwas) Salz vermengt, sonst rein, zu
heiligem Gebrauch bestimmt. \bibleverse{36} Zerstoße etwas davon zu
feinem Pulver und lege etwas davon vor die Gesetzeslade im
Offenbarungszelt, woselbst ich in Verkehr mit dir treten
werde\textless sup title=``vgl. 29,43''\textgreater✲: als hochheilig
soll es euch gelten! \bibleverse{37} Das Räucherwerk aber, das ihr für
euch selbst bereitet, dürft ihr nicht in dem gleichen
Mischungsverhältnis herstellen; nein, es soll dir als dem HERRN
geheiligt gelten! \bibleverse{38} Wer sich das gleiche Räucherwerk
bereitet, um seinen Wohlgeruch zu genießen, soll aus seinen
Volksgenossen ausgerottet werden!«

\hypertarget{berufung-zweier-werkmeister-und-ihrer-gehilfen}{%
\paragraph{Berufung zweier Werkmeister und ihrer
Gehilfen}\label{berufung-zweier-werkmeister-und-ihrer-gehilfen}}

\hypertarget{section-30}{%
\section{31}\label{section-30}}

\bibleverse{1} Weiter sagte der HERR zu Mose folgendes: \bibleverse{2}
»Wisse wohl: ich habe Bezaleel, den Sohn Uris, den Enkel Hurs, aus dem
Stamme Juda, namentlich\textless sup title=``=~mit
Namennennung''\textgreater✲ berufen \bibleverse{3} und ihn mit
göttlichem Geist erfüllt, mit Kunstsinn und Einsicht, mit Verstand und
allerlei Fertigkeiten, \bibleverse{4} um Kunstwerke zu ersinnen,
Arbeiten in Gold, in Silber und in Kupfer auszuführen, \bibleverse{5}
Edelsteine zu schneiden, um Kunstwerke damit zu besetzen, Holz zu
schnitzen, kurz Werke jeder Art kunstvoll auszuführen. \bibleverse{6}
Zugleich habe ich ihm Oholiab, den Sohn Ahisamachs, aus dem Stamme Dan,
beigegeben und allen Kunstverständigen die erforderliche Begabung
verliehen, damit sie alles, was ich dir geboten habe, herstellen,
\bibleverse{7} nämlich das Offenbarungszelt und die Gesetzeslade mit der
Deckplatte darauf und alle anderen Geräte des Zeltes, \bibleverse{8}
nämlich den Tisch mit allen seinen Geräten, den Leuchter aus feinem Gold
samt allen zugehörigen Geräten, den Räucheraltar, \bibleverse{9} den
Brandopferaltar mit allen seinen Geräten, das Becken mit seinem Gestell;
\bibleverse{10} sodann die {[}Prachtkleider und{]} heiligen Kleider für
den Priester Aaron sowie die Kleider seiner Söhne, die mir als Priester
dienen sollen; \bibleverse{11} ferner das Salböl und das wohlriechende
Räucherwerk für das Heiligtum. Genau so, wie ich dir geboten habe,
sollen sie alles ausführen!«

\hypertarget{einschuxe4rfung-des-sabbatgebotes}{%
\paragraph{Einschärfung des
Sabbatgebotes}\label{einschuxe4rfung-des-sabbatgebotes}}

\bibleverse{12} Weiter gebot der HERR dem Mose folgendes:
\bibleverse{13} »Du aber präge den Israeliten folgendes Gebot ein:
›Beobachtet ja meine Sabbate\textless sup title=``oder:
Ruhetage''\textgreater✲! Denn sie sind ein Zeichen (des Bundes) zwischen
mir und euch für eure künftigen Geschlechter, damit ihr erkennt, daß
ich, der HERR, es bin, der euch heiligt. \bibleverse{14} Beobachtet also
den Sabbat! Denn er soll euch heilig sein; wer ihn entweiht, soll
unfehlbar mit dem Tode bestraft werden; ja wer irgendeine Arbeit an
diesem Tage verrichtet, ein solcher Mensch soll aus der Mitte seiner
Volksgenossen ausgerottet werden: \bibleverse{15} Sechs Tage lang darf
gearbeitet werden, aber am siebten Tage ist der dem HERRN geheiligte
Feiertag mit vollständiger Ruhe: wer irgend am Sabbattag eine Arbeit
verrichtet, soll unfehlbar mit dem Tode bestraft werden! \bibleverse{16}
So sollen also die Israeliten den Sabbat beobachten, indem sie den
Ruhetag von Geschlecht zu Geschlecht feiern, als eine ewige
Verpflichtung. \bibleverse{17} Für ewige Zeiten soll er ein Zeichen (des
Bundes) zwischen mir und den Israeliten sein! Denn in sechs Tagen hat
der HERR den Himmel und die Erde geschaffen, aber am siebten Tag hat er
gefeiert und geruht.‹«

\hypertarget{mose-empfuxe4ngt-die-gesetzestafeln}{%
\paragraph{Mose empfängt die
Gesetzestafeln}\label{mose-empfuxe4ngt-die-gesetzestafeln}}

\bibleverse{18} Als der HERR nun seine Unterredung mit Mose auf dem
Berge Sinai beendet hatte, übergab er ihm die beiden Gesetzestafeln,
steinerne Tafeln, die vom Finger Gottes beschrieben waren.

\hypertarget{abfall-des-volkes-und-erneuerung-des-bundes-kap.-32-34}{%
\subsubsection{4. Abfall des Volkes und Erneuerung des Bundes (Kap.
32-34)}\label{abfall-des-volkes-und-erneuerung-des-bundes-kap.-32-34}}

\hypertarget{a-der-abfall-des-volkes-und-seine-bestrafung}{%
\paragraph{a) Der Abfall des Volkes und seine
Bestrafung}\label{a-der-abfall-des-volkes-und-seine-bestrafung}}

\hypertarget{aa-herstellung-und-anbetung-des-goldenen-stierbildes}{%
\subparagraph{aa) Herstellung und Anbetung des goldenen
Stierbildes}\label{aa-herstellung-und-anbetung-des-goldenen-stierbildes}}

\hypertarget{section-31}{%
\section{32}\label{section-31}}

\bibleverse{1} Als aber das Volk sah, daß Mose mit seiner Rückkehr vom
Berge auf sich warten ließ, sammelte sich das Volk um Aaron und sagte zu
ihm: »Auf! Mache uns einen Gott, der vor uns herziehen soll! Denn von
diesem Mose, dem Mann, der uns aus dem Land Ägypten hierher geführt hat,
wissen wir nicht, was aus ihm geworden ist.« \bibleverse{2} Da
antwortete ihnen Aaron: »Reißt die goldenen Ringe ab, die eure Frauen
und eure Söhne und Töchter in den Ohren tragen, und bringt sie mir her!«
\bibleverse{3} Da riß das gesamte Volk sich die goldenen Ringe ab, die
sie in den Ohren trugen, und brachten sie zu Aaron. \bibleverse{4} Der
nahm sie von ihnen in Empfang, bearbeitete das Gold mit dem
Meißel\textless sup title=``oder: in einer Gußform?''\textgreater✲ und
machte ein gegossenes Kalb✲ daraus. Da riefen sie: »Dies ist dein Gott,
Israel, der dich aus dem Land Ägypten hergeführt hat!« \bibleverse{5}
Als Aaron das sah, errichtete er einen Altar vor dem Stierbild und ließ
ausrufen: »Morgen findet ein Fest statt zu Ehren des HERRN!«
\bibleverse{6} Da machten sie sich am andern Morgen früh auf, opferten
Brandopfer und brachten Heilsopfer dar, und das Volk setzte sich nieder,
um zu essen und zu trinken; dann standen sie auf, um sich zu belustigen.

\hypertarget{bb-mose-steigt-durch-gott-von-dem-abfall-des-volkes-unterrichtet-vom-berge-hinab}{%
\subparagraph{bb) Mose steigt, durch Gott von dem Abfall des Volkes
unterrichtet, vom Berge
hinab}\label{bb-mose-steigt-durch-gott-von-dem-abfall-des-volkes-unterrichtet-vom-berge-hinab}}

\bibleverse{7} Da sagte der HERR zu Mose: »Auf! Gehe hinab! Denn dein
Volk, das du aus Ägypten hergeführt hast, begeht eine große Sünde:
\bibleverse{8} gar schnell sind sie von dem Wege abgewichen, den ich
ihnen geboten habe; sie haben sich ein gegossenes Stierbild gemacht und
es angebetet, haben ihm geopfert und ausgerufen: ›Dies ist dein Gott,
Israel, der dich aus dem Land Ägypten hergeführt hat!‹« \bibleverse{9}
Dann fuhr der HERR fort: »Ich habe dieses Volk beobachtet und sehe wohl:
es ist ein halsstarriges Volk. \bibleverse{10} Nun so laß mich, daß mein
Zorn gegen sie entbrenne und ich sie vernichte! Dich aber will ich zu
einem großen Volk machen!« \bibleverse{11} Mose aber suchte den HERRN,
seinen Gott, zu besänftigen, indem er sagte: »Warum, o HERR, soll dein
Zorn gegen dein Volk entbrennen, das du mit großer Kraft und starkem Arm
aus dem Land Ägypten herausgeführt hast? \bibleverse{12} Warum sollen
die Ägypter sagen: ›In böser Absicht\textless sup title=``oder: zum
Unheil''\textgreater✲ hat er sie hinausgeführt, um sie in den Bergen
umkommen zu lassen und sie vom Erdboden zu vertilgen‹? Laß ab von deiner
Zornesglut und laß dir das Unheil leid sein, das du deinem Volk
zugedacht hast! \bibleverse{13} Denke an deine Knechte Abraham, Isaak
und Israel, denen du bei dir selbst zugeschworen und verheißen hast:
›Ich will eure Nachkommenschaft so zahlreich machen wie die Sterne am
Himmel und will dies ganze Land, von dem ich geredet habe, euren
Nachkommen zu ewigem Besitz geben.‹« \bibleverse{14} Da ließ der HERR
sich das Unheil leid sein, das er seinem Volk zugedacht hatte.

\bibleverse{15} Mose aber machte sich auf den Rückweg und stieg vom
Berge hinab mit den beiden Gesetzestafeln in der Hand, Tafeln, die auf
ihren beiden Seiten, vorn und hinten, beschrieben waren. \bibleverse{16}
Diese Tafeln waren von Gott selbst angefertigt, und die Schrift war
Gottes Schrift, in die Tafeln eingegraben. \bibleverse{17} Als nun Josua
das laute Jubelgeschrei des Volkes hörte, sagte er zu Mose: »Kriegslärm
ist im Lager!« \bibleverse{18} Der aber antwortete: »Das klingt nicht
wie Geschrei von Siegern und auch nicht wie Geschrei von Besiegten:
nein, lautes Singen höre ich!«

\hypertarget{cc-moses-eifer-fuxfcr-gott-er-straft-das-volk-durch-die-leviten}{%
\subparagraph{cc) Moses Eifer für Gott; er straft das Volk durch die
Leviten}\label{cc-moses-eifer-fuxfcr-gott-er-straft-das-volk-durch-die-leviten}}

\bibleverse{19} Als er sich dann dem Lager genähert hatte und das
Stierbild und die Reigentänze sah, da geriet Mose in lodernden Zorn, so
daß er die Tafeln aus seinen Händen schleuderte und sie am Fuß des
Berges zertrümmerte. \bibleverse{20} Dann nahm er das Stierbild, das sie
angefertigt hatten, verbrannte es im Feuer und zerstieß es zu feinem
Staub, den streute er aufs Wasser und ließ es die Israeliten trinken.
\bibleverse{21} Hierauf sagte Mose zu Aaron: »Was hat dir dieses Volk
getan, daß du es zu einer so großen Sünde verführt hast?«
\bibleverse{22} Aaron antwortete: »Mein Herr möge nicht in Zorn geraten!
Du weißt selbst, wie das Volk zum Bösen geneigt ist. \bibleverse{23} Sie
forderten mich auf: ›Mache uns einen Gott, der vor uns herziehen soll!
Denn von diesem Mose, dem Mann, der uns aus dem Land Ägypten hergeführt
hat, wissen wir nicht, was aus ihm geworden ist.‹ \bibleverse{24} Da
antwortete ich ihnen: ›Wer Goldschmuck hat, der reiße ihn von sich ab!‹
Sie gaben es mir dann, und ich warf es ins Feuer; da kam dieses
Stierbild heraus.« \bibleverse{25} Als nun Mose sah, daß das Volk
zügellos geworden war -- denn Aaron hatte ihm die Zügel schießen lassen
zur Schadenfreude für ihre Feinde --, \bibleverse{26} trat Mose in das
Tor des Lagers und rief aus: »Her zu mir, wer es mit dem HERRN hält!« Da
scharten sich alle Leviten um ihn. \bibleverse{27} Zu diesen sagte er:
»So spricht der HERR, der Gott Israels: ›Gürtet euch ein jeder sein
Schwert an die Hüfte, geht im Lager hin und her von einem Tor zum andern
und erschlagt ein jeder den eigenen Bruder und ein jeder seine Freunde
und Verwandten!‹« \bibleverse{28} Die Leviten kamen dem Befehl Moses
nach, und so fielen an diesem Tage von dem Volk gegen dreitausend Mann.
\bibleverse{29} Da sagte Mose: »Weiht euch heute dem HERRN zu Priestern!
Denn ein jeder ist sogar gegen den eigenen Sohn und gegen den eigenen
Bruder schonungslos vorgegangen; darum soll euch heute Segen verliehen
werden!«

\hypertarget{dd-moses-fuxfcrbitte-fuxfcr-das-volk-die-einen-aufschub-gewuxe4hrende-antwort-gottes}{%
\subparagraph{dd) Moses Fürbitte für das Volk; die einen Aufschub
gewährende Antwort
Gottes}\label{dd-moses-fuxfcrbitte-fuxfcr-das-volk-die-einen-aufschub-gewuxe4hrende-antwort-gottes}}

\bibleverse{30} Am andern Tage aber sagte Mose zum Volk: »Ihr habt eine
schwere Sünde begangen; darum will ich jetzt zum HERRN hinaufsteigen!
Vielleicht kann ich euch Sühne für eure Sünde erwirken.« \bibleverse{31}
So kehrte denn Mose zum HERRN zurück und sagte: »Ach bitte! Dieses Volk
hat eine schwere Sünde begangen: es hat sich einen Gott aus Gold
angefertigt! \bibleverse{32} Und nun -- vergib ihnen doch ihre Sünde! Wo
nicht, so streiche lieber mich aus deinem Buche aus, das du geschrieben
hast!« \bibleverse{33} Der HERR aber antwortete dem Mose: »Wer gegen
mich gesündigt hat, nur den werde ich aus meinem Buche ausstreichen.
\bibleverse{34} Jetzt aber gehe hin und führe das Volk dahin, wohin ich
dir geboten habe! Jedoch nur mein Engel wird vor dir hergehen; und am
Tage meines Strafgerichts will ich sie für ihre Versündigung büßen
lassen!« \bibleverse{35} So suchte denn der HERR fortan das Volk heim
(zur Strafe) dafür, daß sie das Stierbild hatten machen lassen, welches
Aaron angefertigt hatte.

\hypertarget{b-folgen-der-versuxfcndigung-des-volkes-das-offenbarungszelt-verhandlungen-moses-mit-gott}{%
\paragraph{b) Folgen der Versündigung des Volkes; das Offenbarungszelt;
Verhandlungen Moses mit
Gott}\label{b-folgen-der-versuxfcndigung-des-volkes-das-offenbarungszelt-verhandlungen-moses-mit-gott}}

\hypertarget{aa-gottes-befehl-zum-aufbruch-in-das-land-der-verheiuxdfung-trauer-des-volkes-uxfcber-gottes-abkehr}{%
\subparagraph{aa) Gottes Befehl zum Aufbruch in das Land der Verheißung;
Trauer des Volkes über Gottes
Abkehr}\label{aa-gottes-befehl-zum-aufbruch-in-das-land-der-verheiuxdfung-trauer-des-volkes-uxfcber-gottes-abkehr}}

\hypertarget{section-32}{%
\section{33}\label{section-32}}

\bibleverse{1} Hierauf sagte der HERR zu Mose: »Ziehe nunmehr weiter,
hinweg von hier, du und das Volk, das du aus dem Land Ägypten hergeführt
hast, in das Land, das ich Abraham, Isaak und Jakob zugeschworen habe
mit den Worten: ›Deinen Nachkommen will ich es geben!‹ \bibleverse{2}
Ich will aber einen Engel vor dir hersenden und die Kanaanäer, Amoriter,
Hethiter, Pherissiter, Hewiter und Jebusiter vertreiben, -- in ein Land,
das von Milch und Honig überfließt. \bibleverse{3} Doch ich selbst will
nicht in deiner Mitte hinaufziehen, weil du ein halsstarriges Volk bist;
ich müßte dich sonst unterwegs vertilgen.« \bibleverse{4} Als das Volk
diese schlimme Botschaft vernahm, wurde es betrübt, und keiner legte
seinen Schmuck an; \bibleverse{5} denn der HERR hatte zu Mose gesagt:
»Sage den Israeliten: ›Ihr seid ein halsstarriges Volk; wenn ich auch
nur einen Augenblick in deiner Mitte einherzöge, müßte ich dich
vertilgen. So lege denn jetzt deinen Schmuck von dir ab, dann will ich
sehen, was ich für dich tun kann!‹« \bibleverse{6} Da legten die
Israeliten ihren Schmuck ab, vom Berge Horeb an.

\hypertarget{bb-die-stiftung-und-verwendung-des-vor-dem-lager-befindlichen-offenbarungszeltes}{%
\subparagraph{bb) Die Stiftung und Verwendung des vor dem Lager
befindlichen
Offenbarungszeltes}\label{bb-die-stiftung-und-verwendung-des-vor-dem-lager-befindlichen-offenbarungszeltes}}

\bibleverse{7} Mose aber nahm jedesmal das Zelt und schlug es für
sich\textless sup title=``oder: für ihn, d.h. für den
HERRN?''\textgreater✲ außerhalb des Lagers in einiger Entfernung vom
Lager auf und gab ihm den Namen ›Offenbarungszelt‹\textless sup
title=``oder: Zelt der Zusammenkunft''\textgreater✲. Sooft nun jemand
den HERRN befragen wollte, ging er zu dem Offenbarungszelt hinaus, das
außerhalb des Lagers lag. \bibleverse{8} Sooft aber Mose zu dem Zelt
hinausging, standen alle Leute auf und traten ein jeder in den Eingang
seines Zeltes und blickten hinter Mose her, bis er in das Zelt
eingetreten war. \bibleverse{9} Sobald dann Mose in das Zelt getreten
war, senkte sich die Wolkensäule herab und nahm ihren Stand am Eingang
des Zeltes, solange der HERR mit Mose redete. \bibleverse{10} Wenn nun
das ganze Volk die Wolkensäule am Eingang des Zeltes stehen sah, erhob
sich das ganze Volk, und jeder warf sich im Eingang seines Zeltes
nieder. \bibleverse{11} Der HERR aber redete mit Mose von Angesicht zu
Angesicht, wie jemand mit seinem Freunde redet. Mose kehrte dann wieder
ins Lager zurück, während sein Diener Josua, der Sohn Nuns, ein junger
Mann, das Innere des Zeltes nie verließ.

\hypertarget{cc-weitere-verhandlungen-moses-mit-gott-uxfcber-die-fernere-guxf6ttliche-leitung-des-volkes}{%
\subparagraph{cc) Weitere Verhandlungen Moses mit Gott über die fernere
göttliche Leitung des
Volkes}\label{cc-weitere-verhandlungen-moses-mit-gott-uxfcber-die-fernere-guxf6ttliche-leitung-des-volkes}}

\bibleverse{12} Hierauf sagte Mose zum HERRN: »Siehe, du hast mir wohl
geboten, dieses Volk (nach Kanaan) hinaufzuführen, hast mich aber nicht
wissen lassen, wen du mit mir senden willst; und doch hast du zu mir
gesagt: ›Ich kenne dich mit Namen, und du hast auch Gnade bei mir
gefunden!‹ \bibleverse{13} Wenn ich denn wirklich Gnade bei dir gefunden
habe, so laß mich doch deine Pläne wissen, damit ich dich erkenne und
damit ich (inne werde, daß ich) Gnade bei dir gefunden habe! Bedenke
doch auch, daß dies Volk dein Volk ist!« \bibleverse{14} Da antwortete
der HERR: »Wenn ich in Person mitzöge, würde ich dir dadurch Beruhigung
verschaffen?« \bibleverse{15} Da entgegnete ihm Mose: »Wenn du nicht in
Person mitziehst, so laß uns lieber nicht von hier wegziehen!
\bibleverse{16} Woran soll man denn sonst erkennen, daß ich samt deinem
Volk Gnade bei dir gefunden habe? Doch eben daran, daß du mit uns ziehst
und daß wir, ich und dein Volk, dadurch vor allen Völkern auf dem
Erdboden ausgezeichnet werden.« \bibleverse{17} Da antwortete der HERR
dem Mose: »Auch diese Bitte, die du jetzt ausgesprochen, will ich dir
erfüllen; denn du hast Gnade bei mir gefunden, und ich kenne dich mit
Namen.«

\hypertarget{dd-gott-verspricht-mose-das-schauen-seiner-herrlichkeit-als-gnadenbeweis}{%
\subparagraph{dd) Gott verspricht Mose das Schauen seiner Herrlichkeit
als
Gnadenbeweis}\label{dd-gott-verspricht-mose-das-schauen-seiner-herrlichkeit-als-gnadenbeweis}}

\bibleverse{18} Als Mose nun bat: »Laß mich doch deine Herrlichkeit
schauen!«, \bibleverse{19} antwortete der HERR: »Ich will all meine
Schöne vor deinen Augen vorüberziehen lassen und will den Namen des
HERRN laut vor dir ausrufen, nämlich daß ich Gnade erweise, wem ich eben
gnädig bin, und Barmherzigkeit dem erzeige, dessen ich mich erbarmen
will.« \bibleverse{20} Dann fuhr er fort: »Mein Angesicht kannst du
nicht schauen; denn kein Mensch, der mich schaut, bleibt am Leben.«
\bibleverse{21} Doch der HERR fuhr fort: »Siehe, es ist ein Platz neben
mir\textless sup title=``oder: eine Stätte bei mir''\textgreater✲: da
magst du dich auf den Felsen stellen! \bibleverse{22} Wenn ich dann in
meiner Herrlichkeit vorüberziehe, will ich dich in die Höhlung des
Felsens stellen und meine Hand schirmend über dich halten, bis ich
vorübergezogen bin. \bibleverse{23} Habe ich dann meine Hand
zurückgezogen, so wirst du meine Rückseite schauen; mein Angesicht aber
kann nicht geschaut werden.«

\hypertarget{c-erneuerung-des-bundes-und-der-gesetzestafeln-mose-schaut-die-herrlichkeit-des-herrn}{%
\paragraph{c) Erneuerung des Bundes und der Gesetzestafeln; Mose schaut
die Herrlichkeit des
Herrn}\label{c-erneuerung-des-bundes-und-der-gesetzestafeln-mose-schaut-die-herrlichkeit-des-herrn}}

\hypertarget{aa-mose-steigt-auf-gottes-gebot-mit-zwei-unbeschriebenen-steintafeln-auf-den-sinai}{%
\subparagraph{aa) Mose steigt auf Gottes Gebot mit zwei unbeschriebenen
Steintafeln auf den
Sinai}\label{aa-mose-steigt-auf-gottes-gebot-mit-zwei-unbeschriebenen-steintafeln-auf-den-sinai}}

\hypertarget{section-33}{%
\section{34}\label{section-33}}

\bibleverse{1} Darauf gebot der HERR dem Mose: »Haue dir zwei
Steintafeln zurecht, wie die ersten waren, dann will ich auf die Tafeln
die Worte schreiben, die auf den ersten Tafeln gestanden haben, die du
zertrümmert hast! \bibleverse{2} Halte dich für morgen bereit, gleich
früh auf den Berg Sinai zu steigen und dort auf der Spitze des Berges
vor mich zu treten! \bibleverse{3} Es soll aber niemand mit dir
heraufsteigen, und es darf sich auch niemand am ganzen Berge blicken
lassen; sogar das Kleinvieh und die Rinder dürfen nicht gegen diesen
Berg hin weiden!« \bibleverse{4} So hieb sich denn Mose zwei Steintafeln
zurecht, wie die ersten gewesen waren, und machte sich dann am andern
Morgen früh auf und stieg zum Berge Sinai hinauf, wie der HERR ihm
geboten hatte; die beiden Steintafeln trug er in der Hand.

\hypertarget{bb-die-gotteserscheinung-und-die-fuxfcrbitte-moses}{%
\subparagraph{bb) Die Gotteserscheinung und die Fürbitte
Moses}\label{bb-die-gotteserscheinung-und-die-fuxfcrbitte-moses}}

\bibleverse{5} Da fuhr der HERR im Gewölk hernieder, und er (Mose) trat
dort neben ihn und rief den Namen des HERRN an. \bibleverse{6} Da zog
der HERR vor seinen Augen vorüber und rief aus: »Der HERR, der HERR ist
ein barmherziger und gnädiger Gott, langmütig und reich an Gnade und
Treue, \bibleverse{7} der Gnade auf Tausende hin\textless sup
title=``oder: Tausenden; vgl. 20,6''\textgreater✲ bewahrt, der Unrecht,
Übertretung und Sünde vergibt, doch auch (den Schuldigen) keineswegs
ungestraft läßt, sondern die Schuld der Väter an Kindern und
Kindeskindern heimsucht, am dritten und am vierten Glied.«
\bibleverse{8} Da verneigte sich Mose eilends bis zur Erde, warf sich
nieder \bibleverse{9} und sagte: »Habe ich irgend Gnade bei dir, o Herr,
gefunden, so wolle mein Herr doch in unserer Mitte einherziehen! Denn es
ist ein halsstarriges Volk. Aber vergib uns unsere Schuld und Sünde und
laß uns dein Eigentum sein!«

\hypertarget{cc-gott-willigt-unter-ernsten-warnungen-in-die-erneuerung-des-bundesverhuxe4ltnisses}{%
\subparagraph{cc) Gott willigt unter ernsten Warnungen in die Erneuerung
des
Bundesverhältnisses}\label{cc-gott-willigt-unter-ernsten-warnungen-in-die-erneuerung-des-bundesverhuxe4ltnisses}}

\bibleverse{10} Da antwortete der HERR: »Wohlan, ich schließe einen
Bund: vor deinem ganzen Volk will ich Wunder tun, wie sie auf der ganzen
Erde und unter allen Völkern noch nie vollführt sind, und das ganze
Volk, in dessen Mitte du lebst, soll das Walten des HERRN wahrnehmen;
denn wunderbar soll das sein, was ich an dir tun werde. \bibleverse{11}
Beobachte wohl, was ich dir heute gebiete! Siehe, ich will vor dir die
Amoriter, Kanaanäer, Hethiter, Pherissiter, Hewiter und Jebusiter
vertreiben. \bibleverse{12} Hüte dich wohl, mit den Bewohnern des
Landes, in das du kommen wirst, einen Vertrag zu schließen, damit sie
für dich nicht, wenn sie unter dir wohnen bleiben, ein Fallstrick
werden! \bibleverse{13} Ihr sollt vielmehr ihre Altäre niederreißen,
ihre Malsteine\textless sup title=``1.Mose 28,18''\textgreater✲
zertrümmern und ihre Götzenbäume umhauen. \bibleverse{14} Denn du sollst
keinen andern Gott anbeten! Denn der HERR heißt ›Eiferer‹ und ist ein
eifersüchtiger Gott. \bibleverse{15} Schließe daher ja keinen Vertrag
mit den Bewohnern des Landes, damit du nicht, wenn sie ihren
Götzendienst treiben und ihren Göttern opfern und sie dich dazu
einladen, an ihren Opfermahlen teilnimmst. \bibleverse{16} Du würdest
dann auch von ihren Töchtern manche für deine Söhne zu Frauen nehmen,
und diese würden, wenn sie ihren Götzendienst treiben, deine Söhne zu
demselben Götzendienst verführen.«

\hypertarget{dd-die-neuen-bundesvorschriften-uxfcber-die-rechte-gottesverehrung}{%
\subparagraph{dd) Die neuen Bundesvorschriften über die rechte
Gottesverehrung}\label{dd-die-neuen-bundesvorschriften-uxfcber-die-rechte-gottesverehrung}}

\bibleverse{17} »Du sollst dir keine gegossenen Gottesbilder machen!~--
\bibleverse{18} Das Fest der ungesäuerten Brote sollst du beobachten!
Sieben Tage lang sollst du ungesäuertes Brot essen, wie ich dir geboten
habe, zur bestimmten Zeit des Monats Abib; denn im Monat
Abib\textless sup title=``oder: am Neumond des Monats
Abib''\textgreater✲ bist du aus Ägypten ausgezogen.~-- \bibleverse{19}
Alle Erstgeburt gehört mir, auch von all deinem Vieh jede männliche
Erstgeburt, der erste Wurf vom Rind- und Kleinvieh. \bibleverse{20} Aber
das Erstgeborene vom Esel sollst du entweder mit einem Lamm lösen✲ oder,
wenn du das nicht willst, ihm das Genick brechen. Jeden Erstgeborenen
von deinen Söhnen sollst du lösen! Und man darf vor mir nicht mit leeren
Händen erscheinen!~-- \bibleverse{21} Sechs Tage lang sollst du
arbeiten, aber am siebten Tage sollst du ruhen! Auch während der Zeit
des Pflügens und der Ernte sollst du ruhen!~-- \bibleverse{22} Auch das
Wochenfest sollst du feiern, nämlich das Fest der Erstlinge der
Weizenernte, sowie das Fest der Lese an der Wende des Jahres!~--
\bibleverse{23} Dreimal im Jahr sollen alle deine männlichen Personen
vor Gott dem HERRN, dem Gott Israels, erscheinen. \bibleverse{24} Denn
ich werde die Heidenvölker vor dir austreiben und dein Gebiet weit
ausdehnen; und niemand soll nach deinem Land Verlangen tragen, während
du hinaufziehst, um dich vor dem HERRN, deinem Gott, sehen zu lassen
dreimal im Jahr.~-- \bibleverse{25} Du sollst das Blut meines
Schlachtopfers nicht zusammen mit gesäuertem Brot darbringen! Und vom
Opferfleisch des Passahfestes darf nichts über Nacht bis zum andern
Morgen übrigbleiben!~-- \bibleverse{26} Das Vorzüglichste von den
Erstlingen deines Feldes sollst du in das Haus des HERRN, deines Gottes,
bringen! -- Du sollst kein Böckchen in der Milch seiner Mutter kochen!«

\hypertarget{ee-mose-schreibt-die-bundesgebote-auf-gott-erneuert-die-gesetzestafeln}{%
\subparagraph{ee) Mose schreibt die Bundesgebote auf; Gott erneuert die
Gesetzestafeln}\label{ee-mose-schreibt-die-bundesgebote-auf-gott-erneuert-die-gesetzestafeln}}

\bibleverse{27} Weiter gebot der HERR dem Mose: »Schreibe dir diese
Verordnungen auf! Denn auf Grund dieser Verordnungen habe ich mit dir
und mit Israel einen Bund geschlossen.« \bibleverse{28} Hierauf
verweilte Mose dort beim HERRN vierzig Tage und vierzig Nächte, ohne
Brot zu essen und Wasser zu trinken; und er\textless sup title=``d.h.
Gott; vgl. V.1''\textgreater✲ schrieb auf die Tafeln die Gebote des
Bundes, die zehn Gebote.

\hypertarget{ff-der-abstieg-moses-das-strahlen-seiner-gesichtshaut}{%
\subparagraph{ff) Der Abstieg Moses; das Strahlen seiner
Gesichtshaut}\label{ff-der-abstieg-moses-das-strahlen-seiner-gesichtshaut}}

\bibleverse{29} Als Mose dann vom Berge Sinai hinabstieg -- die beiden
Gesetzestafeln hatte er in der Hand, als er vom Berge hinabstieg --, da
wußte Mose nicht, daß die Haut seines Angesichts infolge\textless sup
title=``oder: während''\textgreater✲ seiner Unterredung mit dem HERRN
strahlend geworden war. \bibleverse{30} Als nun Aaron und alle
Israeliten Mose ansahen (und wahrnahmen), daß die Haut seines Angesichts
strahlte, da fürchteten sie sich, ihm nahe zu kommen. \bibleverse{31}
Als Mose sie aber herbeirief, wandten sich Aaron und alle Vorsteher der
Gemeinde ihm wieder zu, und Mose redete mit ihnen. \bibleverse{32}
Darauf traten auch alle Israeliten nahe an ihn heran, und er teilte
ihnen alles mit, was der HERR ihm auf dem Berge Sinai aufgetragen hatte.
\bibleverse{33} Als er dann mit seinen Mitteilungen zu Ende war, legte
er eine Hülle auf sein Angesicht. \bibleverse{34} Sooft Mose nun vor den
HERRN trat, um mit ihm zu reden, legte er die Hülle ab, bis er wieder
hinausging; und wenn er hinausgekommen war, teilte er den Israeliten
alles mit, was ihm geboten worden war. \bibleverse{35} Dabei bekamen
dann die Israeliten das Gesicht Moses zu sehen (und machten die
Beobachtung), daß die Haut in seinem Gesicht strahlend geworden war;
Mose aber legte dann die Hülle wieder auf sein Gesicht, bis er wieder
hineinging, um mit dem HERRN zu reden.

\hypertarget{herstellung-und-weihe-des-offenbarungszeltes-der-stiftshuxfctte-der-heiligen-geruxe4te-und-der-priesterkleider-kap.-35-40}{%
\subsubsection{5. Herstellung und Weihe des Offenbarungszeltes (=~der
Stiftshütte), der heiligen Geräte und der Priesterkleider (Kap.
35-40)}\label{herstellung-und-weihe-des-offenbarungszeltes-der-stiftshuxfctte-der-heiligen-geruxe4te-und-der-priesterkleider-kap.-35-40}}

\hypertarget{a-mitteilung-des-sabbatgebots-aufforderung-zur-leistung-der-beisteuer-fuxfcr-die-stiftshuxfctte}{%
\paragraph{a) Mitteilung des Sabbatgebots; Aufforderung zur Leistung der
Beisteuer für die
Stiftshütte}\label{a-mitteilung-des-sabbatgebots-aufforderung-zur-leistung-der-beisteuer-fuxfcr-die-stiftshuxfctte}}

\hypertarget{section-34}{%
\section{35}\label{section-34}}

\bibleverse{1} Hierauf versammelte Mose die ganze Gemeinde der
Israeliten und sagte zu ihnen: »Dies ist es, was der HERR zu tun geboten
hat: \bibleverse{2} Sechs Tage lang soll gearbeitet werden; aber der
siebte Tag soll euch heilig sein als ein Feiertag mit völliger Ruhe zu
Ehren des HERRN! Wer an diesem Tage eine Arbeit verrichtet, soll den Tod
erleiden! \bibleverse{3} Am Sabbattage dürft ihr kein Feuer in allen
euren Wohnungen anzünden!«

\bibleverse{4} Weiter sagte Mose zu der ganzen Gemeinde der Israeliten:
»Dies ist es, was der HERR geboten hat: \bibleverse{5} Bringt von eurem
Besitz eine Abgabe für den HERRN! Jeder, den sein Herz dazu treibt, möge
die Abgabe für den HERRN herbringen: Gold, Silber und Kupfer,
\bibleverse{6} blauen und roten Purpur und Karmesin\textless sup
title=``=~karmesinfarbige Garne''\textgreater✲, Byssus und Ziegenhaar,
\bibleverse{7} rotgefärbte Widderfelle und Seekuhhäute, Akazienholz,
\bibleverse{8} Öl für den Leuchter, sowie Gewürzkräuter✲ für das Salböl
und für das wohlriechende Räucherwerk, \bibleverse{9} Onyxsteine und
andere Edelsteine zum Besatz für das Schulterkleid und für das
Brustschild\textless sup title=``oder: die Brusttasche''\textgreater✲.
\bibleverse{10} Und alle, die kunstverständig unter euch sind, mögen
kommen und alles herstellen, was der HERR geboten hat, \bibleverse{11}
nämlich die Wohnung mit ihrem Zelt und ihrer Überdachung, ihren Haken
und Brettern, ihren Riegeln, Säulen\textless sup title=``oder:
Ständern''\textgreater✲ und deren Füßen\textless sup title=``oder:
Fußgestellen, Sockeln''\textgreater✲, \bibleverse{12} die Lade mit ihren
Tragstangen, die Deckplatte und den abschließenden Vorhang;
\bibleverse{13} den Tisch mit seinen Tragstangen und all seinen Geräten
und den Schaubroten; \bibleverse{14} den Leuchter zur Beleuchtung mit
seinen Geräten und seinen Lampen und das Öl für den Leuchter;
\bibleverse{15} den Räucheraltar mit seinen Tragstangen, das Salböl und
das wohlriechende Räucherwerk; \bibleverse{16} den Türvorhang für den
Eingang der Wohnung; den Brandopferaltar mit dem kupfernen Gitterwerk
daran, samt seinen Tragstangen und all seinen Geräten; das Becken mit
seinem Gestell; \bibleverse{17} die Umhänge für den Vorhof samt den dazu
gehörigen Säulen und deren Füßen sowie den Vorhang für das Tor am
Vorhof; \bibleverse{18} die Pflöcke der Wohnung und die Pflöcke des
Vorhofs nebst den erforderlichen Stricken; \bibleverse{19}

\hypertarget{b-das-volk-betuxe4tigt-seine-bereitwilligkeit}{%
\paragraph{b) Das Volk betätigt seine
Bereitwilligkeit}\label{b-das-volk-betuxe4tigt-seine-bereitwilligkeit}}

\bibleverse{20} Hierauf ging die ganze Gemeinde der Israeliten von Mose
weg; \bibleverse{21} dann aber kam ein jeder, den sein Herz dazu trieb,
und jeder, der einen willigen Sinn besaß, brachte die Beisteuer für den
HERRN zur Herstellung des Offenbarungszeltes und für den gesamten
heiligen Dienst in ihm und für die heiligen Kleider; \bibleverse{22} und
zwar kamen sowohl die Männer als auch die Frauen, jeder, den sein Herz
dazu trieb; sie brachten Spangen, Ohrringe, Fingerringe und
Halsgeschmeide, goldene Schmucksachen aller Art; und jeder, der dem
HERRN etwas von Gold als Weihgabe darzubringen beschlossen hatte,
brachte es herbei; \bibleverse{23} und jeder, in dessen Besitz sich
blauer und roter Purpur und Karmesin, Byssus und Ziegenhaar, rotgefärbte
Widderfelle und Seekuhhäute befanden, brachte sie herbei;
\bibleverse{24} jeder, der eine Beisteuer an Silber und Kupfer leisten
wollte, brachte die Spende für den HERRN herbei; und jeder, in dessen
Besitz sich Akazienholz zu irgendeiner Verwendung befand, brachte es
herbei. \bibleverse{25} Und alle Frauen, welche die erforderliche
Geschicklichkeit besaßen, spannen eigenhändig und brachten das Gespinst
herbei: blauen und roten Purpur, karmesinfarbige Garne und Byssus;
\bibleverse{26} und alle Frauen, die sich infolge ihrer Geschicklichkeit
dazu getrieben fühlten, verspannen die Ziegenhaare. \bibleverse{27} Die
Stammfürsten aber brachten Onyxsteine und andere Edelsteine zum Besatz
für das Schulterkleid und für das Brustschild, \bibleverse{28} ferner
die Gewürzkräuter und das Öl zur Beleuchtung und zum Salböl und zum
wohlriechenden Räucherwerk. \bibleverse{29} So brachten die Israeliten,
alle Männer und Frauen, die sich dazu getrieben fühlten, zu dem ganzen
Werk, dessen Ausführung der HERR durch Mose geboten hatte, etwas
beizutragen, freiwillige Gaben für den HERRN dar.

\hypertarget{c-berufung-der-werkmeister-und-der-anderen-kunstfertigen-reichliche-spenden-und-freiwillige-leistungen-des-volkes}{%
\paragraph{c) Berufung der Werkmeister und der anderen Kunstfertigen;
reichliche Spenden und freiwillige Leistungen des
Volkes}\label{c-berufung-der-werkmeister-und-der-anderen-kunstfertigen-reichliche-spenden-und-freiwillige-leistungen-des-volkes}}

\bibleverse{30} Hierauf sagte Mose zu den Israeliten: »Wisset wohl: der
HERR hat Bezaleel, den Sohn Uris, den Enkel Hurs, vom Stamme Juda,
namentlich\textless sup title=``d.h. mit Namennennung''\textgreater✲
berufen \bibleverse{31} und ihn mit göttlichem Geist erfüllt, mit
Kunstsinn, Einsicht, Verstand und allerlei Fertigkeiten, \bibleverse{32}
nämlich um Kunstwerke zu ersinnen, Arbeiten in Gold, Silber und Kupfer
auszuführen, \bibleverse{33} Edelsteine zu schneiden, um Kunstwerke
damit zu besetzen, Holz zu schnitzen, kurz Werke jeder Art kunstvoll
herzustellen. \bibleverse{34} Aber auch die Gabe, andere zu unterweisen,
hat er ihm verliehen, ihm und Oholiab, dem Sohn Ahisamachs, vom Stamme
Dan. \bibleverse{35} Er hat sie beide mit Kunstsinn reich ausgestattet,
um alle Arten von Arbeiten auszuführen, wie sie die Künstler in festen
Stoffen sowie die Kunstweber und Buntwirker in blauem und rotem Purpur,
in Karmesin und Byssus schaffen und die (einfachen) Weber liefern, indem
sie Arbeiten aller Art ausführen und Kunstwerke ersinnen!

\hypertarget{section-35}{%
\section{36}\label{section-35}}

\bibleverse{1} So werden denn Bezaleel und Oholiab und alle
kunstverständigen Männer, denen der HERR Kunstsinn und Einsicht
verliehen hat, so daß sie sich auf die Ausführung aller für das
Heiligtum erforderlichen Arbeiten verstehen, alles nach den Anordnungen
des HERRN ausführen.«

\bibleverse{2} Hierauf berief Mose den Bezaleel und Oholiab und alle
kunstverständigen Männer, denen der HERR Kunstsinn verliehen hatte,
alle, die sich dazu getrieben fühlten, sich an der Ausführung des Werkes
zu beteiligen. \bibleverse{3} Sie empfingen dann von Mose alle die
Gaben, welche die Israeliten zur Ausführung der Arbeiten für das
Heiligtum beigesteuert hatten; man brachte ihm aber an jedem Morgen
immer noch freiwillige Geschenke. \bibleverse{4} Da kamen alle Künstler,
welche die ganze Arbeit für das Heiligtum auszuführen hatten, Mann für
Mann, von der besonderen Arbeit, mit der sie beschäftigt waren,
\bibleverse{5} und sagten zu Mose: »Das Volk bringt viel mehr, als zur
Verwendung für die Arbeiten, deren Ausführung der HERR geboten hat,
erforderlich ist.« \bibleverse{6} Da ließ Mose durch Ausruf im Lager
bekanntmachen: »Niemand, weder Mann noch Frau, möge fernerhin noch eine
Arbeit als Beisteuer für das Heiligtum anfertigen!« Da mußte das Volk
aufhören, noch weitere Spenden zu bringen; \bibleverse{7} denn der
gelieferte Vorrat reichte für die Ausführung aller Arbeiten aus, ja es
blieb davon noch übrig.

\hypertarget{d-die-herstellung-der-heiligen-wohnung}{%
\paragraph{d) Die Herstellung der heiligen
Wohnung}\label{d-die-herstellung-der-heiligen-wohnung}}

\hypertarget{aa-die-herstellung-der-vier-zeltdecken}{%
\subparagraph{aa) Die Herstellung der vier
Zeltdecken}\label{aa-die-herstellung-der-vier-zeltdecken}}

\bibleverse{8} So stellten denn alle kunstverständigen Männer unter den
Werkleuten die Wohnung aus zehn Teppichen her; aus gezwirntem Byssus,
blauem und rotem Purpur und Karmesin, mit Cherubbildern in
Kunstweberarbeit stellte er\textless sup title=``d.h. Bezaleel; vgl.
37,1''\textgreater✲ sie her. \bibleverse{9} Die Länge jedes einzelnen
Teppichs betrug achtundzwanzig Ellen und die Breite vier Ellen: alle
Teppiche hatten dieselbe Größe. \bibleverse{10} Je fünf Teppiche fügte
er dann zu einem Stück zusammen, einen an den andern. \bibleverse{11}
Sodann brachte er Schleifen von blauem Purpur am Saum des äußersten
Teppichs des einen zusammengesetzten Stückes an, und ebenso machte er es
am Saum des äußersten Teppichs bei dem andern zusammengesetzten Stück.
\bibleverse{12} Fünfzig Schleifen brachte er an dem einen Teppich an und
ebenso fünfzig Schleifen am Saum des Teppichs, der zu dem andern
zusammengesetzten Stück gehörte, so daß die Schleifen einander genau
gegenüberstanden. \bibleverse{13} Sodann fertigte er fünfzig goldene
Haken an und verband die Teppiche vermittels der Haken miteinander, so
daß die Wohnung ein Ganzes bildete.

\bibleverse{14} Weiter fertigte er Teppiche aus Ziegenhaar zu einer
Zeltdecke über der Wohnung an; elf Teppiche fertigte er dazu an.
\bibleverse{15} Die Länge jedes Teppichs betrug dreißig Ellen und die
Breite vier Ellen: alle elf Teppiche hatten dieselbe Größe.
\bibleverse{16} Dann fügte er fünf von diesen Teppichen zu einem Stück
für sich zusammen und ebenso die sechs anderen Teppiche für sich.
\bibleverse{17} Dann brachte er fünfzig Schleifen am Saum des äußersten
Teppichs des einen zusammengesetzten Stückes an und ebenso fünfzig
Schleifen am Saum des äußersten Teppichs des andern zusammengesetzten
Stückes. \bibleverse{18} Hierauf fertigte er fünfzig kupferne Haken an,
um die Zeltdecke zusammenzufügen, so daß sie ein Ganzes bildete.~--
\bibleverse{19} Sodann verfertigte er für das Zeltdach eine Schutzdecke
von rotgefärbten Widderfellen und darüber noch eine andere Schutzdecke
von Seekuhhäuten.

\hypertarget{bb-herstellung-des-holzgeruxfcstes-der-bretter-und-fuxfcnf-riegel}{%
\subparagraph{bb) Herstellung des Holzgerüstes (der Bretter und fünf
Riegel)}\label{bb-herstellung-des-holzgeruxfcstes-der-bretter-und-fuxfcnf-riegel}}

\bibleverse{20} Weiter fertigte er die Bretter für die Wohnung aus
Akazienholz an, sie standen aufrecht; \bibleverse{21} die Länge jedes
Brettes betrug zehn Ellen und die Breite anderthalb Ellen;
\bibleverse{22} an jedem Brett saßen zwei Zapfen, einer dem andern
gegenüber eingefügt; so machte er es an allen Brettern der Wohnung.
\bibleverse{23} Und zwar stellte er an Brettern für die Wohnung her:
zwanzig Bretter für die Mittagsseite südwärts; \bibleverse{24} und er
brachte unter diesen zwanzig Brettern vierzig silberne Füße\textless sup
title=``oder: Fußgestelle, Sockel''\textgreater✲ an, je zwei Füße unter
jedem Brett für seine beiden Zapfen. \bibleverse{25} Ebenso fertigte er
für die andere Seite der Wohnung, nämlich für die Nordseite, zwanzig
Bretter an \bibleverse{26} und vierzig silberne Füße dazu, je zwei Füße
unter jedem Brett. \bibleverse{27} Aber für die Hinterseite der Wohnung,
nach Westen zu, fertigte er sechs Bretter an, \bibleverse{28} außerdem
noch zwei Bretter für die Ecken der Wohnung an der Hinterseite.
\bibleverse{29} Diese waren unten und gleicherweise oben vollständig bis
an den ersten Ring hin\textless sup title=``?; vgl.
26,24''\textgreater✲; so machte er es bei beiden für die beiden Ecken.
\bibleverse{30} Demnach waren es im ganzen acht Bretter, dazu ihre Füße
von Silber: sechzehn Füße, immer zwei Füße unter jedem Brett.

\bibleverse{31} Sodann fertigte er Riegel aus Akazienholz an: fünf für
die Bretter der einen Seitenwand der Wohnung \bibleverse{32} und fünf
Riegel für die Bretter der andern Seitenwand der Wohnung und fünf Riegel
für die Bretter an der Hinterseite der Wohnung gegen Westen.
\bibleverse{33} Den mittleren Riegel aber machte er so, daß er in der
Mitte der Bretter von einem Ende bis zum andern durchlief.
\bibleverse{34} Die Bretter aber überzog er mit Gold, und auch die
dazugehörigen Ringe, die zur Aufnahme der Riegel dienten, stellte er aus
Gold her und überzog die Riegel gleichfalls mit Gold.

\hypertarget{cc-herstellung-der-beiden-vorhuxe4nge}{%
\subparagraph{cc) Herstellung der beiden
Vorhänge}\label{cc-herstellung-der-beiden-vorhuxe4nge}}

\bibleverse{35} Weiter fertigte er den Vorhang aus blauem und rotem
Purpur, aus Karmesin und gezwirntem Byssus an, und zwar in
Kunstweberarbeit mit Cherubbildern. \bibleverse{36} Dann fertigte er für
ihn vier Säulen\textless sup title=``oder: Ständer''\textgreater✲ von
Akazienholz an und überzog sie mit Gold; auch ihre Nägel waren von Gold;
und er goß für sie vier silberne Füße\textless sup title=``oder:
Fußgestelle, Sockel''\textgreater✲.~-- \bibleverse{37} Sodann fertigte
er für den Eingang des Zeltes einen Vorhang von blauem und rotem Purpur,
von Karmesin und gezwirntem Byssus in Buntwirkerarbeit; \bibleverse{38}
außerdem die dazugehörigen fünf Säulen und ihre Nägel. Ihre Köpfe und
Ringbänder überzog er mit Gold; die fünf dazu gehörigen
Füße\textless sup title=``oder: Sockel''\textgreater✲ aber waren von
Kupfer.

\hypertarget{e-herstellung-der-heiligen-geruxe4te-und-des-vorhofes}{%
\paragraph{e) Herstellung der heiligen Geräte und des
Vorhofes}\label{e-herstellung-der-heiligen-geruxe4te-und-des-vorhofes}}

\hypertarget{aa-die-lade-mit-der-deckplatte}{%
\subparagraph{aa) Die Lade mit der
Deckplatte}\label{aa-die-lade-mit-der-deckplatte}}

\hypertarget{section-36}{%
\section{37}\label{section-36}}

\bibleverse{1} Weiter fertigte Bezaleel die Lade aus Akazienholz an,
zweieinhalb Ellen lang, anderthalb Ellen breit und anderthalb Ellen
hoch. \bibleverse{2} Er überzog sie inwendig und auswendig mit feinem
Gold und brachte an ihr einen goldenen Kranz ringsum an. \bibleverse{3}
Dann goß er für sie vier goldene Ringe und befestigte diese an ihren
vier Ecken\textless sup title=``oder: Füßen''\textgreater✲, und zwar
zwei Ringe an ihrer einen Seite und zwei Ringe an ihrer andern Seite.
\bibleverse{4} Sodann fertigte er Tragstangen von Akazienholz an und
überzog sie mit Gold; \bibleverse{5} darauf steckte er die Stangen in
die Ringe an den Seiten der Lade, so daß man die Lade tragen konnte.~--
\bibleverse{6} Weiter fertigte er eine Deckplatte von feinem Gold an,
zweieinhalb Ellen lang und anderthalb Ellen breit. \bibleverse{7} Auch
fertigte er zwei goldene Cherube an, und zwar in getriebener Arbeit, an
den beiden Enden der Deckplatte; \bibleverse{8} den einen Cherub brachte
er am Ende der einen Seite und den andern Cherub am Ende der andern
Seite an, und zwar mit der Deckplatte zu einem Stück verbunden, an ihren
beiden Enden. \bibleverse{9} Die Cherube breiteten ihre Flügel nach oben
hin aus, so daß sie die Deckplatte mit ihren Flügeln überdeckten und
ihre Gesichter einander zugekehrt waren; nach der Deckplatte hin waren
die Gesichter der Cherube gerichtet.

\hypertarget{bb-der-tisch-fuxfcr-die-schaubrote-und-trankopfer}{%
\subparagraph{bb) Der Tisch für die Schaubrote und
Trankopfer}\label{bb-der-tisch-fuxfcr-die-schaubrote-und-trankopfer}}

\bibleverse{10} Sodann fertigte er den Tisch aus Akazienholz an, zwei
Ellen lang, eine Elle breit und anderthalb Ellen hoch. \bibleverse{11}
Er überzog ihn mit feinem Gold und brachte einen goldenen Kranz ringsum
an ihm an. \bibleverse{12} Außerdem brachte er an ihm ringsum noch eine
Einfassung\textless sup title=``oder: Leiste''\textgreater✲ an, die eine
Handbreite hoch war, und an dieser Einfassung wieder einen goldenen
Kranz ringsherum. \bibleverse{13} Weiter goß er für ihn vier goldene
Ringe und befestigte diese Ringe an den vier Ecken bei seinen vier
Füßen. \bibleverse{14} Dicht an der Einfassung befanden sich die Ringe
und dienten zur Aufnahme der Stangen, damit man den Tisch tragen konnte.
\bibleverse{15} Die Stangen aber verfertigte er aus Akazienholz und
überzog sie mit Gold, damit man den Tisch tragen konnte. \bibleverse{16}
Weiter fertigte er die Geräte an, die auf dem Tische stehen sollten,
nämlich die erforderlichen Schüsseln und Schalen, die Becher und Kannen,
mit denen Trankopfer gespendet werden sollten, (alles) aus feinem Gold.

\hypertarget{cc-der-goldene-leuchter}{%
\subparagraph{cc) Der goldene Leuchter}\label{cc-der-goldene-leuchter}}

\bibleverse{17} Sodann fertigte er den Leuchter aus feinem Gold an; in
getriebener Arbeit stellte er den Leuchter, seinen Fuß und seinen
Schaft, her; seine Blumenkelche -- Knäufe mit Blüten -- waren aus einem
Stück mit ihm gearbeitet. \bibleverse{18} Sechs Arme\textless sup
title=``oder: Röhren''\textgreater✲ gingen von seinen beiden Seiten aus,
drei Arme auf jeder Seite des Leuchters. \bibleverse{19} Drei
mandelblütenförmige Blumenkelche -- je ein Knauf mit einer Blüte --
befanden sich an jedem Arm; so war es bei allen sechs Armen, die von dem
Leuchter ausgingen. \bibleverse{20} Am Schaft selbst aber befanden sich
vier mandelblütenförmige Blumenkelche -- Knäufe mit Blüten --,
\bibleverse{21} und zwar befand sich an ihm immer ein Knauf unter jedem
Paar der sechs Arme, die vom Schaft des Leuchters ausgingen.
\bibleverse{22} Ihre Knäufe und Arme bestanden aus einem Stück mit ihm:
der ganze Leuchter war eine einzige getriebene Arbeit aus feinem Gold.
\bibleverse{23} Sodann fertigte er auch die sieben Lampen für ihn an
nebst den erforderlichen Lichtputzscheren und den Pfannen aus feinem
Gold. \bibleverse{24} Aus einem Talent feinen Goldes stellte er ihn
nebst allen zugehörigen Geräten her.

\hypertarget{dd-der-ruxe4ucheraltar}{%
\subparagraph{dd) Der Räucheraltar}\label{dd-der-ruxe4ucheraltar}}

\bibleverse{25} Weiter fertigte er den Räucheraltar aus Akazienholz an,
eine Elle lang und eine Elle breit, viereckig✲, und zwei Ellen hoch;
seine Hörner waren aus einem Stück mit ihm. \bibleverse{26} Er überzog
ihn, sowohl die Platte als auch die Wände ringsum, sowie die Hörner mit
feinem Gold und brachte an ihm einen goldenen Kranz ringsherum an.
\bibleverse{27} Außerdem brachte er zwei goldene Ringe unterhalb seines
Kranzes an den beiden Ecken auf jeder der beiden Seiten an, zur Aufnahme
der Stangen, mittels derer man ihn tragen sollte. \bibleverse{28} Die
Stangen fertigte er aus Akazienholz an und überzog sie mit Gold.

\hypertarget{ee-das-heilige-salbuxf6l-und-ruxe4ucherwerk}{%
\subparagraph{ee) Das heilige Salböl und
Räucherwerk}\label{ee-das-heilige-salbuxf6l-und-ruxe4ucherwerk}}

\bibleverse{29} Dann bereitete er das heilige Salböl und das
wohlriechende Räucherwerk aus reinen Gewürzkräutern, wie es der
Salbenmischer herstellt.

\hypertarget{ff-der-brandopferaltar.-das-kupferne-becken}{%
\subparagraph{ff) Der Brandopferaltar. Das kupferne
Becken}\label{ff-der-brandopferaltar.-das-kupferne-becken}}

\hypertarget{section-37}{%
\section{38}\label{section-37}}

\bibleverse{1} Hierauf fertigte er den Brandopferaltar aus Akazienholz
an, fünf Ellen lang und fünf Ellen breit, viereckig✲ und drei Ellen
hoch. \bibleverse{2} Die zu ihm gehörenden Hörner brachte er an seinen
vier Ecken an; diese Hörner waren mit ihm aus einem Stück gearbeitet;
und er überzog ihn mit Kupfer. \bibleverse{3} Sodann verfertigte er alle
für den Altar erforderlichen Geräte: die Töpfe und Schaufeln, die
Becken, Gabeln und Pfannen; alle erforderlichen Geräte stellte er aus
Kupfer her. \bibleverse{4} Weiter verfertigte er für den Altar ein
Gitterwerk, eine netzartige Arbeit aus Kupfer, und brachte es unterhalb
seiner Einfassung von unten her bis zur halben Höhe (des Altars) an.
\bibleverse{5} Ferner goß er vier Ringe und setzte sie an die vier Ecken
des kupfernen Gitterwerks zur Aufnahme der Stangen. \bibleverse{6} Die
Stangen fertigte er aus Akazienholz und überzog sie mit Kupfer.
\bibleverse{7} Sodann steckte er die Stangen in die Ringe an den Seiten
des Altars, damit man ihn mittels ihrer tragen könnte; aus Brettern
stellte er ihn so her, daß er (inwendig) hohl war.

\bibleverse{8} Hierauf verfertigte er das Becken\textless sup
title=``oder: den Kessel''\textgreater✲ aus Kupfer und sein Gestell
gleichfalls aus Kupfer {[}aus den Spiegeln der diensttuenden Frauen, die
am Eingang des Offenbarungszeltes den Dienst versahen{]}.

\hypertarget{gg-der-vorhof}{%
\subparagraph{gg) Der Vorhof}\label{gg-der-vorhof}}

\bibleverse{9} Sodann stellte er den Vorhof so her: auf der Mittagseite,
nach Süden zu, die Umhänge für den Vorhof aus gezwirntem Byssus, hundert
Ellen lang; \bibleverse{10} dazu zwanzig Säulen\textless sup
title=``oder: Ständer''\textgreater✲ nebst den zugehörigen zwanzig
kupfernen Füßen; die Nägel der Säulen aber und die zugehörigen
Ringbänder waren von Silber. \bibleverse{11} Ebenso fertigte er für die
Nordseite die Umhänge an, hundert Ellen lang; dazu zwanzig Säulen nebst
den zugehörigen zwanzig kupfernen Füßen; die Nägel der Säulen aber und
die zugehörigen Ringbänder waren von Silber. \bibleverse{12} Ferner
verfertigte er für die Westseite Umhänge von fünfzig Ellen; dazu zehn
Säulen nebst den zugehörigen zehn Füßen; die Nägel der Säulen aber und
die zugehörigen Ringbänder waren von Silber. \bibleverse{13} Weiter für
die Ostseite, gegen Morgen, Umhänge von fünfzig Ellen, \bibleverse{14}
nämlich fünfzehn Ellen Umhänge für die eine Seite mit ihren drei Säulen
und deren drei Füßen, \bibleverse{15} und ebenso für die andere Seite
fünfzehn Ellen Umhänge; dazu drei Säulen nebst den zugehörigen drei
Füßen. \bibleverse{16} Alle Umhänge des Vorhofes ringsum waren von
gezwirntem Byssus; \bibleverse{17} und die Füße der Säulen bestanden aus
Kupfer, die Haken der Säulen aber und die zugehörigen Ringbänder aus
Silber und der Überzug ihrer Köpfe gleichfalls aus Silber; es waren aber
die Säulen des Vorhofes alle mit silbernen Ringbändern versehen.
\bibleverse{18} Der Vorhang aber für den Eingang zum Vorhof war
Buntwirkerarbeit von blauem und rotem Purpur, Karmesin und gezwirntem
Byssus; und zwar betrug die Länge zwanzig Ellen, die Höhe,
beziehungsweise die Breite, fünf Ellen, entsprechend den übrigen
Umhängen des Vorhofs. \bibleverse{19} Die zugehörigen vier Säulen aber
nebst ihren vier Füßen waren von Kupfer, ihre Nägel von Silber und der
Überzug ihrer Köpfe und ihre Ringbänder auch von Silber. \bibleverse{20}
Alle Pflöcke aber an der Wohnung und am Vorhof ringsum waren von Kupfer.

\hypertarget{f-die-kostenberechnung-der-fuxfcr-das-heiligtum-verbrauchten-metalle}{%
\paragraph{f) Die Kostenberechnung der für das Heiligtum verbrauchten
Metalle}\label{f-die-kostenberechnung-der-fuxfcr-das-heiligtum-verbrauchten-metalle}}

\bibleverse{21} Folgendes ist die Kostenberechnung für die Wohnung,
nämlich für die Wohnung des Gesetzes, wie sie auf Befehl Moses durch die
Dienstleistung der Leviten unter der Leitung des Priesters Ithamar, des
Sohnes Aarons, aufgestellt worden ist. \bibleverse{22} Bezaleel, der
Sohn Uris, der Enkel Hurs, aus dem Stamm Juda, hatte alles angefertigt,
was der HERR dem Mose geboten hatte; \bibleverse{23} und mit ihm
Oholiab, der Sohn Ahisamachs, aus dem Stamm Dan, ein Künstler in festen
Stoffen sowie ein Kunstweber und Buntwirker in blauem und rotem Purpur,
in Karmesin und Byssus.

\bibleverse{24} Was das gesamte Gold betrifft, das bei der Herstellung
des Heiligtums für alle Arbeiten verbraucht worden ist, so betrug das
freiwillig beigesteuerte Gold 29~Talente und 730~Schekel nach dem
Gewicht des Heiligtums. \bibleverse{25} Das Silber aber, das infolge der
Musterung der Gemeinde eingegangen war, betrug 100~Talente und
1775~Schekel nach dem Gewicht des Heiligtums, \bibleverse{26} je ein
Beka auf den Kopf, die Hälfte eines Schekels, nach dem Gewicht des
Heiligtums, von allen, die sich der Musterung unterzogen hatten, von
zwanzig Jahren an und darüber, im ganzen von 603550~Mann.
\bibleverse{27} Es dienten aber die 100~Talente Silber zum Gießen der
Füße des Heiligtums und der Füße des Vorhangs, 100~Talente für 100~Füße,
also ein Talent für jeden Fuß. \bibleverse{28} Aus den 1775~Schekeln
aber fertigte (Bezaleel) die Nägel für die Säulen an und überzog damit
ihre Köpfe und versah sie mit Ringbändern. \bibleverse{29} Das
freiwillig beigesteuerte Kupfer aber betrug 70~Talente und 2400~Schekel.
\bibleverse{30} Daraus verfertigte er die Füße (der Säulen) am Eingang
des Offenbarungszeltes sowie den kupfernen Altar nebst seinem kupfernen
Gitterwerk und alle Geräte des Altars, \bibleverse{31} ferner die Füße
des Vorhofs ringsum und die Füße (der Säulen) am Eingang des Vorhofs
sowie alle Pflöcke der Wohnung und alle Pflöcke des Vorhofs ringsum.

\hypertarget{g-anfertigung-der-priesterkleider}{%
\paragraph{g) Anfertigung der
Priesterkleider}\label{g-anfertigung-der-priesterkleider}}

\hypertarget{section-38}{%
\section{39}\label{section-38}}

\bibleverse{1} Aus dem blauen und roten Purpur aber und dem Karmesin
verfertigten sie die Prachtkleider für den Dienst im Heiligtum, und zwar
stellten sie die heiligen Kleidungsstücke für Aaron so her, wie der HERR
dem Mose geboten hatte.

\hypertarget{aa-das-schulterkleid-ephod}{%
\subparagraph{aa) Das Schulterkleid
(Ephod)}\label{aa-das-schulterkleid-ephod}}

\bibleverse{2} Das Schulterkleid fertigten sie also aus Gold, blauem und
rotem Purpur, Karmesin und gezwirntem Byssus an. \bibleverse{3} Sie
hämmerten nämlich das Gold zu breiten Blechen und zerschnitten sie dann
in dünne Fäden, um diese mittels Kunstweberarbeit mit dem blauen und
roten Purpur, dem Karmesin und dem Byssus zu verweben. \bibleverse{4}
Sie fertigten (zwei) zusammenfügbare Schulterstücke an, mittels derer
das Kleid an seinen beiden (oberen) Enden zusammengefügt wurde.
\bibleverse{5} Die Binde aber, die sich an ihm befand und dazu diente,
es fest anzulegen, war aus einem Stück mit ihm angefertigt und von
gleicher Arbeit, nämlich aus Goldfäden, blauem und rotem Purpur,
Karmesin und gezwirntem Byssus gearbeitet, wie der HERR dem Mose geboten
hatte.~-- \bibleverse{6} Sodann richteten sie die (beiden) Onyxsteine so
zu, daß sie in ein Geflecht von Golddraht eingefaßt und mittels
Siegelstecherkunst mit den (eingestochenen) Namen der Söhne Israels
versehen waren. \bibleverse{7} Diese setzte man auf die Schulterstücke
des Schulterkleides als Steine des Gedenkens an die
Israeliten\textless sup title=``vgl. 28,12''\textgreater✲, wie der HERR
dem Mose geboten hatte.

\hypertarget{bb-der-brustschmuck}{%
\subparagraph{bb) Der Brustschmuck}\label{bb-der-brustschmuck}}

\bibleverse{8} Hierauf fertigten sie das Brustschild\textless sup
title=``oder: die Brusttasche''\textgreater✲ in Kunstweberarbeit an,
ganz so, wie das Schulterkleid gearbeitet war, nämlich aus Goldfäden,
blauem und rotem Purpur, Karmesin und gezwirntem Byssus. \bibleverse{9}
Viereckig✲ war es; doppelt gelegt fertigten sie das Brustschild an, eine
Spanne lang und eine Spanne breit, doppelt gelegt. \bibleverse{10} Sie
besetzten es dann mit vier Reihen von Edelsteinen: ein Karneol, ein
Topas und ein Smaragd bildeten die erste Reihe; \bibleverse{11} die
zweite Reihe bestand aus einem Rubin, einem Saphir und einem Jaspis;
\bibleverse{12} die dritte Reihe aus einem Hyazinth, einem Achat und
einem Amethyst; \bibleverse{13} die vierte Reihe aus einem Chrysolith,
einem Soham\textless sup title=``1.Mose 2,12''\textgreater✲ und einem
Onyx; in ein Geflecht von Gold gefaßt, bildeten sie den Besatz.
\bibleverse{14} Die Steine waren aber, entsprechend den Namen der Söhne
Israels, nach deren Namen, zwölf (an Zahl); mittels Siegelstecherkunst
waren sie, ein jeder mit seinem besonderen Namen, für die zwölf Stämme
versehen. \bibleverse{15} Dann befestigten sie an dem Brustschilde
schnurähnlich geflochtene Kettchen von feinem Gold. \bibleverse{16}
Weiter verfertigten sie zwei goldene Geflechte und zwei goldene Ringe
und setzten die beiden Ringe an die beiden (oberen) Ecken des
Brustschildes. \bibleverse{17} Dann befestigten sie die beiden goldenen
Schnüre an den beiden Ringen an den (oberen) Ecken des Brustschildes;
\bibleverse{18} die beiden anderen Enden der beiden Schnüre aber
befestigten sie an den beiden Geflechten und (hefteten) diese wiederum
an die beiden Schulterstücke des Schulterkleides auf dessen Vorderseite.
\bibleverse{19} Dann fertigten sie noch zwei goldene Ringe an und
setzten sie an die beiden (unteren) Ecken des Brustschildes, und zwar an
seinen (inneren) Saum, der dem Schulterkleide zugekehrt war.
\bibleverse{20} Dann fertigten sie noch zwei goldene Ringe an und
setzten sie an die beiden Schulterstücke des Schulterkleides, unten auf
seine Vorderseite, dicht bei der Stelle, wo das Schulterkleid
zusammenging, oberhalb der Binde des Schulterkleides. \bibleverse{21}
Hierauf knüpften sie das Brustschild mit seinen Ringen vermittels einer
Schnur von blauem Purpur an die Ringe des Schulterkleides an, so daß das
Brustschild oberhalb der Binde des Schulterkleides fest anlag und sich
nicht von seiner Stelle auf dem Schulterkleide verschieben konnte: so
wie der HERR dem Mose geboten hatte.

\hypertarget{cc-das-obergewand-zum-schulterkleide}{%
\subparagraph{cc) Das Obergewand zum
Schulterkleide}\label{cc-das-obergewand-zum-schulterkleide}}

\bibleverse{22} Sodann fertigten sie das zu dem Schulterkleid gehörige
Obergewand in Weberarbeit an, ganz aus blauem Purpur. \bibleverse{23}
Die Halsöffnung des Obergewandes befand sich in seiner Mitte (und war)
wie die Öffnung eines Panzers; einen Saum hatte die Halsöffnung ringsum,
damit sie nicht einrisse. \bibleverse{24} Hierauf brachten sie unten am
Saum des Obergewandes Granatäpfel aus blauem und rotem Purpur, aus
Karmesin und gezwirntem Byssus an. \bibleverse{25} Weiter verfertigten
sie Glöckchen aus feinem Gold und setzten diese Glöckchen zwischen die
Granatäpfel an den Saum des Obergewandes ringsum zwischen die
Granatäpfel, \bibleverse{26} so daß immer auf ein Glöckchen ein
Granatapfel am Saum des Obergewandes ringsum folgte, zur Verwendung beim
heiligen Dienst, wie der HERR dem Mose geboten hatte.

\hypertarget{dd-die-uxfcbrigen-amtskleider-der-priester}{%
\subparagraph{dd) Die übrigen Amtskleider der
Priester}\label{dd-die-uxfcbrigen-amtskleider-der-priester}}

\bibleverse{27} Dann fertigten sie die Unterkleider aus Byssus in
Weberarbeit für Aaron und seine Söhne an, \bibleverse{28} ferner den
Kopfbund aus Byssus und die hohen Mützen aus Byssus und die leinenen
Unterbeinkleider aus gezwirntem Byssus; \bibleverse{29} endlich den
Gürtel aus gezwirntem Byssus, aus blauem und rotem Purpur und Karmesin
in Buntwirkerarbeit, wie der HERR dem Mose geboten hatte.

\hypertarget{ee-das-stirnblatt-fuxfcr-den-hohenpriester}{%
\subparagraph{ee) Das Stirnblatt für den
Hohenpriester}\label{ee-das-stirnblatt-fuxfcr-den-hohenpriester}}

\bibleverse{30} Hierauf fertigten sie das Stirnblatt, das heilige
Diadem, aus feinem Gold an und gruben auf ihm in Siegelstecherschrift
die Worte ein: »Dem HERRN geweiht!« \bibleverse{31} Sie banden daran
eine Schnur von blauem Purpur fest, um es oben am Kopfbund zu
befestigen, wie der HERR dem Mose geboten hatte.

\hypertarget{ff-uxfcbergabe-der-gefertigten-gegenstuxe4nde-an-mose}{%
\subparagraph{ff) Übergabe der gefertigten Gegenstände an
Mose}\label{ff-uxfcbergabe-der-gefertigten-gegenstuxe4nde-an-mose}}

\bibleverse{32} So wurde die ganze Arbeit für die Wohnung des
Offenbarungszeltes fertiggestellt: die Israeliten hatten alles genau so
gemacht, wie der HERR dem Mose geboten hatte. \bibleverse{33} So
brachten sie denn alles zur Wohnung Gehörige zu Mose: das Zelt mit allen
seinen Geräten, seinen Haken, Brettern, Riegeln, Säulen und Füßen;
\bibleverse{34} ferner die Schutzdecke von rotgefärbten Widderfellen,
die Schutzdecke von Seekuhhäuten und den abschließenden (inneren)
Vorhang; \bibleverse{35} die Gesetzeslade mit ihren Tragstangen und der
Deckplatte; \bibleverse{36} den Tisch mit allen seinen Geräten und den
Schaubroten; \bibleverse{37} den Leuchter aus feinem Gold mit seinen in
einer Reihe aufgesetzten Lampen und allen seinen Geräten sowie das Öl
zur Beleuchtung; \bibleverse{38} ferner den goldenen Altar, das Salböl,
das wohlriechende Räucherwerk und den Vorhang für den Eingang zum Zelt;
\bibleverse{39} den kupfernen Altar mit dem zugehörigen kupfernen
Gitterwerk, seinen Tragstangen und allen seinen Geräten; das
Becken\textless sup title=``oder: den Kessel''\textgreater✲ nebst seinem
Gestell; \bibleverse{40} die Umhänge des Vorhofs nebst seinen Säulen und
Füßen; den Vorhang für den Eingang des Vorhofs nebst den zugehörigen
Stricken und Pflöcken sowie alle Geräte für den Dienst in der Wohnung
des Offenbarungszeltes; \bibleverse{41} die heiligen Kleider für den
Priester Aaron sowie die Kleider seiner Söhne für den priesterlichen
Dienst. \bibleverse{42} Genau so wie der HERR dem Mose geboten hatte,
war die ganze Arbeit von den Israeliten ausgeführt worden.~--
\bibleverse{43} Als Mose dann alles von ihnen Hergestellte besichtigt
und sich überzeugt hatte, daß sie es genau nach den Anordnungen des
HERRN ausgeführt hatten, dankte er ihnen unter Segenswünschen.

\hypertarget{h-aufrichtung-und-einweihung-des-heiligtums-erfuxfcllung-der-heiligen-wohnung-mit-der-herrlichkeit-des-herrn}{%
\paragraph{h) Aufrichtung und Einweihung des Heiligtums; Erfüllung der
heiligen Wohnung mit der Herrlichkeit des
Herrn}\label{h-aufrichtung-und-einweihung-des-heiligtums-erfuxfcllung-der-heiligen-wohnung-mit-der-herrlichkeit-des-herrn}}

\hypertarget{aa-der-guxf6ttliche-befehl}{%
\subparagraph{aa) Der göttliche
Befehl}\label{aa-der-guxf6ttliche-befehl}}

\hypertarget{section-39}{%
\section{40}\label{section-39}}

\bibleverse{1} Hierauf gebot der HERR dem Mose folgendes: \bibleverse{2}
»Am ersten Tage des ersten Monats sollst du die Wohnung des
Offenbarungszeltes aufschlagen, \bibleverse{3} die Lade mit dem Gesetz
hineinstellen und den Vorhang vor der Lade aufhängen! \bibleverse{4}
Dann sollst du den Tisch hineinbringen, die erforderlichen
(Schaubrotschichten) auf ihm zurechtlegen, auch den Leuchter
hineinbringen und die zu ihm gehörigen Lampen aufsetzen! \bibleverse{5}
Dann stelle den goldenen Räucheraltar vor (den Vorhang vor der)
Gesetzeslade\textless sup title=``vgl. V.26''\textgreater✲ und hänge den
Vorhang am Eingang zur Wohnung auf! \bibleverse{6} Hierauf stelle den
Brandopferaltar vor dem Eingang zur Wohnung des Offenbarungszeltes auf
\bibleverse{7} und setze das Becken\textless sup title=``oder: den
Kessel''\textgreater✲ zwischen das Offenbarungszelt und den Altar und tu
Wasser hinein! \bibleverse{8} Weiter laß den Vorhof ringsum aufrichten
und den Vorhang am Eingang des Vorhofs anbringen! \bibleverse{9} Dann
nimm das Salböl und salbe die Wohnung nebst allem, was sich darin
befindet: weihe sie dadurch samt allen ihren Geräten, damit sie heilig
sei! \bibleverse{10} Ebenso salbe den Brandopferaltar samt allen seinen
Geräten und weihe so den Altar, damit er hochheilig sei! \bibleverse{11}
Auch das Becken\textless sup title=``oder: den Kessel''\textgreater✲
samt seinem Gestell mußt du salben und es dadurch heiligen!
\bibleverse{12} Hierauf laß Aaron und seine Söhne an den Eingang des
Offenbarungszeltes herantreten und laß sie eine Abwaschung mit Wasser an
sich vornehmen! \bibleverse{13} Sodann laß Aaron die heiligen Kleider
anlegen und salbe ihn und weihe ihn so zur Ausübung des Priesterdienstes
für mich! \bibleverse{14} Ebenso laß seine Söhne herantreten und die
Unterkleider anziehen; \bibleverse{15} dann salbe sie, wie du ihren
Vater gesalbt hast, damit sie mir als Priester dienen! Und diese Salbung
soll ihnen das Priestertum für ewige Zeiten von Geschlecht zu Geschlecht
verleihen!«

\hypertarget{bb-die-ausfuxfchrung-des-guxf6ttlichen-befehls}{%
\subparagraph{bb) Die Ausführung des göttlichen
Befehls}\label{bb-die-ausfuxfchrung-des-guxf6ttlichen-befehls}}

\bibleverse{16} Da Mose alles genau nach den Anordnungen des HERRN
ausführte, \bibleverse{17} wurde die Wohnung im zweiten Jahr (nach dem
Auszuge aus Ägypten), am ersten Tage des ersten Monats aufgeschlagen.
\bibleverse{18} Als Mose damals die Wohnung aufrichten ließ, legte er
ihre Füße\textless sup title=``=~die Fußgestelle''\textgreater✲ fest,
stellte die zugehörigen Bretter darauf, brachte ihre Riegel an und
richtete ihre Säulen\textless sup title=``oder: Ständer''\textgreater✲
auf; \bibleverse{19} dann breitete er das Zeltdach über der Wohnung aus
und legte die Schutzdecke des Zeltes oben darüber, wie der HERR dem Mose
geboten hatte. \bibleverse{20} Dann nahm er die (beiden) Gesetzestafeln
und legte sie in die Lade, steckte die Tragstangen an die Lade und legte
die Deckplatte oben auf die Lade; \bibleverse{21} alsdann brachte er die
Lade in die Wohnung hinein und hängte den abschließenden Vorhang so auf,
daß er die Lade mit dem Gesetz abschloß\textless sup title=``oder:
verdeckte''\textgreater✲, wie der HERR dem Mose geboten hatte.
\bibleverse{22} Darauf stellte er den Tisch in das Offenbarungszelt an
die Nordseite der Wohnung, außerhalb des Vorhangs, \bibleverse{23} und
legte auf ihm die Schichten der Schaubrote vor dem HERRN zurecht, wie
der HERR dem Mose geboten hatte. \bibleverse{24} Dann stellte er den
Leuchter in das Offenbarungszelt dem Tisch gegenüber, an die Südseite
der Wohnung, \bibleverse{25} und setzte die Lampen vor dem HERRN auf,
wie der HERR dem Mose geboten hatte. \bibleverse{26} Darauf stellte er
den goldenen Altar in das Offenbarungszelt vor den Vorhang
\bibleverse{27} und verbrannte wohlriechendes Räucherwerk auf ihm, wie
der HERR ihm geboten hatte. \bibleverse{28} Dann hängte er den Vorhang
für den Eingang der Wohnung auf, \bibleverse{29} stellte den
Brandopferaltar vor den Eingang der Wohnung des Offenbarungszeltes und
brachte das Brandopfer und das Speisopfer auf ihm dar, wie der HERR ihm
geboten hatte. \bibleverse{30} Dann ließ er den Kessel\textless sup
title=``oder: das Becken''\textgreater✲ zwischen dem Offenbarungszelt
und dem Altar aufstellen und Wasser für die Waschungen hineintun,
\bibleverse{31} damit Mose und Aaron nebst dessen Söhnen sich die Hände
und Füße daraus wuschen; \bibleverse{32} sooft sie in das
Offenbarungszelt hineingingen oder an den Altar herantraten, wuschen sie
sich daraus, wie der HERR dem Mose geboten hatte. \bibleverse{33}
Schließlich richtete er den Vorhof rings um die Wohnung und um den Altar
auf und brachte den Vorhang am Tor des Vorhofs an.

\hypertarget{cc-die-herrlichkeit-des-herrn-nimmt-von-der-wohnung-besitz}{%
\subparagraph{cc) Die Herrlichkeit des Herrn nimmt von der Wohnung
Besitz}\label{cc-die-herrlichkeit-des-herrn-nimmt-von-der-wohnung-besitz}}

\bibleverse{34} Als Mose so das ganze Werk vollendet hatte,
verhüllte\textless sup title=``oder: bedeckte''\textgreater✲ die Wolke
das Offenbarungszelt, und die Herrlichkeit des HERRN erfüllte die
Wohnung, \bibleverse{35} so daß Mose nicht in das Offenbarungszelt
hineingehen konnte, weil die Wolke sich auf dasselbe niedergelassen
hatte und die Herrlichkeit des HERRN die Wohnung erfüllte.
\bibleverse{36} Sooft sich nun die Wolke von der Wohnung erhob, brachen
die Israeliten auf während der ganzen Dauer ihrer Wanderungen;
\bibleverse{37} wenn aber die Wolke sich nicht erhob, brachen sie nicht
auf bis zu dem Tage, wo sie sich erhob. \bibleverse{38} Denn bei Tage
lag die Wolke des HERRN über der Wohnung; bei Nacht aber war sie, mit
Feuerschein erfüllt, dem ganzen Hause Israel sichtbar während der ganzen
Dauer ihrer Wanderzüge.
