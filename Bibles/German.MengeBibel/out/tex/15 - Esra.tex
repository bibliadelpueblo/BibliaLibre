\hypertarget{das-buch-esra}{%
\section{DAS BUCH ESRA}\label{das-buch-esra}}

\hypertarget{i.-die-ereignisse-von-der-ersten-ruxfcckkehr-der-verbannten-juden-bis-zur-vollendung-des-tempelbaues-kap.-1-6}{%
\subsection{I. Die Ereignisse von der (ersten) Rückkehr der verbannten
Juden bis zur Vollendung des Tempelbaues (Kap.
1-6)}\label{i.-die-ereignisse-von-der-ersten-ruxfcckkehr-der-verbannten-juden-bis-zur-vollendung-des-tempelbaues-kap.-1-6}}

\hypertarget{die-erlaubnis-des-cyrus-zur-ruxfcckkehr-der-noch-uxfcbriggebliebenen-juden-und-zur-erbauung-des-tempels}{%
\subsubsection{1. Die Erlaubnis des Cyrus zur Rückkehr der noch
übriggebliebenen Juden und zur Erbauung des
Tempels}\label{die-erlaubnis-des-cyrus-zur-ruxfcckkehr-der-noch-uxfcbriggebliebenen-juden-und-zur-erbauung-des-tempels}}

\hypertarget{a-wortlaut-der-kuxf6niglichen-verfuxfcgung}{%
\paragraph{a) Wortlaut der königlichen
Verfügung}\label{a-wortlaut-der-kuxf6niglichen-verfuxfcgung}}

\hypertarget{section}{%
\section{1}\label{section}}

1Im ersten Regierungsjahre des Kores✲, des Königs von Persien -- damit
das durch den Mund Jeremias ergangene Wort des HERRN in Erfüllung ginge
-- regte der HERR den Geist des Perserkönigs Kores dazu an, folgende
Verfügung in seinem ganzen Reiche ausrufen und auch durch schriftlichen
Erlaß bekanntmachen zu lassen: 2»So spricht\textless sup
title=``=~Folgendes verfügt''\textgreater✲ Kores, der König von Persien:
Alle Reiche der Erde hat der HERR, der Gott des Himmels, mir übergeben,
und er ist's auch, der mir aufgetragen hat, ihm zu Jerusalem in Juda ein
Haus\textless sup title=``=~einen Tempel''\textgreater✲ zu erbauen. 3Wer
also unter euch allen zu seinem Volke gehört, mit dem sei sein Gott, und
er ziehe hinauf nach Jerusalem in Juda und baue dort das
Haus\textless sup title=``=~den Tempel''\textgreater✲ des HERRN, des
Gottes Israels; das ist der Gott, der in Jerusalem wohnt. 4Und jeder,
der noch übriggeblieben ist, den sollen an allen Orten, wo er sich als
Fremdling aufhält, die betreffenden Ortsbewohner mit Silber und Gold,
mit beweglicher Habe und Vieh sowie mit freiwilligen Gaben für das
Gotteshaus in Jerusalem unterstützen.«

\hypertarget{b-die-wirkung-und-ausfuxfchrung-der-verfuxfcgung}{%
\paragraph{b) Die Wirkung und Ausführung der
Verfügung}\label{b-die-wirkung-und-ausfuxfchrung-der-verfuxfcgung}}

5Da machten sich die Familienhäupter von Juda und Benjamin sowie die
Priester und die Leviten auf den Weg, alle, denen Gott es in den Sinn
gegeben hatte, hinaufzuziehen, um den Tempel des HERRN in Jerusalem
wieder aufzubauen; 6und alle, die um sie her wohnten, unterstützten sie
auf jede Weise mit Gaben, mit Silber und Gold, mit beweglicher Habe und
Vieh und mit Kostbarkeiten, außerdem mit freiwilligen Gaben aller Art.

\hypertarget{herausgabe-und-aufzuxe4hlung-der-an-sesbazzar-serubbabel-ausgelieferten-tempelgeruxe4te}{%
\paragraph{Herausgabe und Aufzählung der an Sesbazzar (=~Serubbabel)
ausgelieferten
Tempelgeräte}\label{herausgabe-und-aufzuxe4hlung-der-an-sesbazzar-serubbabel-ausgelieferten-tempelgeruxe4te}}

7Auch gab der König Kores die Tempelgeräte wieder heraus, die
Nebukadnezar einst aus Jerusalem weggeführt und im Tempel seines Gottes
untergebracht hatte: 8der König Kores von Persien ließ sie unter der
Aufsicht des Schatzmeisters Mithredath hervorholen\textless sup
title=``oder: herausgeben''\textgreater✲, und dieser zählte sie
Sesbazzar, dem Fürsten von Juda, zu. 9Ihre Zahl war folgende: 30 goldene
und 1000\textless sup title=``oder: 2029''\textgreater✲ silberne Becken,
1030 goldene Becher, 2410 silberne Becher, 1000 andere Geräte, 11im
ganzen 5400 goldene und silberne Geräte. Das alles nahm Sesbazzar mit
hinauf, als die in die Verbannung\textless sup title=``oder:
Gefangenschaft''\textgreater✲ Weggeführten von Babylon nach Jerusalem
hinaufgeführt wurden✲.

\hypertarget{verzeichnis-der-zuruxfcckkehrenden-juden}{%
\subsubsection{2. Verzeichnis der zurückkehrenden
Juden}\label{verzeichnis-der-zuruxfcckkehrenden-juden}}

\hypertarget{a-eingangswort-und-angabe-der-leitenden-muxe4nner}{%
\paragraph{a) Eingangswort und Angabe der leitenden
Männer}\label{a-eingangswort-und-angabe-der-leitenden-muxe4nner}}

\hypertarget{section-1}{%
\section{2}\label{section-1}}

1Folgendes nun sind die Bewohner der Landschaft\textless sup
title=``oder: Provinz''\textgreater✲ Juda, die aus der Gefangenschaft
der in der Verbannung Lebenden, welche Nebukadnezar, der König von
Babylon, (einst) nach Babylon weggeführt hatte, hinaufgezogen sind und
(nun) nach Jerusalem und Juda zurückkehrten, ein jeder in
seine\textless sup title=``d.h. ihm zustehende oder: ihm
angewiesene''\textgreater✲ Ortschaft. 2Sie sind dorthin gekommen
zusammen mit Serubbabel, Jesua, Nehemia, Seraja, Reelaja, Mordochai,
Bilsan, Mispar, Bigwai, Rehum und Baana.

\hypertarget{b-aufzuxe4hlung-der-ruxfcckwandernden}{%
\paragraph{b) Aufzählung der
Rückwandernden}\label{b-aufzuxe4hlung-der-ruxfcckwandernden}}

Die Zahl der Männer des Volkes Israel betrug: 3die Familie Parhos 2172;
4die Familie Sephatja 372; 5die Familie Arah 775; 6die Familie
Pahath-Moab, nämlich die Familien Jesua und Joab, 2812; 7die Familie
Elam 1254; 8die Familie Satthu 945; 9die Familie Sakkai 760; 10die
Familie Bani 642; 11die Familie Bebai 623; 12die Familie Asgad 1222;
13die Familie Adonikam 666; 14die Familie Bigwai 2056; 15die Familie
Adin 454; 16die Familie Ater, nämlich der Zweig Hiskia, 98; 17die
Familie Bezai 323; 18die Familie Jora\textless sup title=``oder:
Hariph''\textgreater✲ 112; 19die Familie Hasum 223; 20die Leute von
Gibeon 95; 21die Leute von Bethlehem 123; 22die Männer von Netopha 56;
23die Männer von Anathoth 128; 24die Leute von Asmaweth 42; 25die Leute
von Kirjath-Arim, Kephira und Beeroth 743; 26die Leute von Rama und Geba
621; 27die Männer von Michmas 122; 28die Männer von Bethel und Ai 223;
29die Familie Nebo 52; 30die Familie Magbis 156; 31die Familie des
andern Elam 1254; 32die Familie Harim 320; 33die Leute von Lod, Hadid
und Ono 725; 34die Leute von Jericho 345; 35die Familie Senaa 3630.

36Die Priester: die Familie Jedaja, nämlich das Haus Jesua 973; 37die
Familie Immer 1052; 38die Familie Pashur 1247; 39die Familie Harim 1017.

40Die Leviten: die Familien Jesua und Kadmiel, Binnui und Hodawja 74;~--
41die Sänger: die Familie Asaph 128;~-- 42die Familien der Torhüter: die
Familien Sallum, Ater, Talmon, Akkub, Hatita und Sobai, im ganzen 139.

43Die Tempelhörigen: die Familie Ziha, die Familie Hasupha, die Familie
Tabbaoth, 44die Familie Keros, die Familie Siaha, die Familie Padon,
45die Familie Lebana, die Familie Hagaba, die Familie Akkub, 46die
Familie Hagab, die Familie Salmai, die Familie Hanan, 47die Familie
Giddel, die Familie Gahar, die Familie Reaja, 48die Familie Rezin, die
Familie Nekoda, die Familie Gassam, 49die Familie Ussa, die Familie
Paseah, die Familie Besai, 50die Familie Asna, die Familie der
Mehuniter, die Familie der Nephisiter, 51die Familie Bakbuk, die Familie
Hakupha, die Familie Harhur, 52die Familie Bazluth, die Familie Mehida,
die Familie Harsa, 53die Familie Barkos, die Familie Sisera, die Familie
Themah, 54die Familie Neziah, die Familie Hatipha.

55Die Söhne\textless sup title=``oder: Nachkommen''\textgreater✲ der
Sklaven\textless sup title=``oder: Leibeigenen''\textgreater✲ Salomos:
die Familie Sotai, die Familie Sophereth, die Familie Peruda, 56die
Familie Jaala, die Familie Darkon, die Familie Giddel, 57die Familie
Sephatja, die Familie Hattil, die Familie Pochereth-Hazzebaim, die
Familie Ami\textless sup title=``oder: Amon''\textgreater✲. 58Die
Gesamtzahl der Tempelhörigen und der Nachkommen der Sklaven Salomos
betrug 392.

59Und dies sind die, welche aus Thel-Melah, Thel-Harsa, Cherub-Addan und
Immer mit hinaufzogen, aber ihre Familie und ihre Herkunft nicht
nachweisen konnten, ob sie nämlich aus Israel stammten: 60die Familie
Delaja, die Familie Tobija, die Familie Nekoda, 652.~-- 61Sodann von den
Priesterfamilien: die Familie Habaja, die Familie Hakkoz, die Familie
jenes Barsillais, der eine Frau von den Töchtern des Gileaditen
Barsillai geheiratet und deren\textless sup title=``oder:
dessen''\textgreater✲ Namen angenommen hatte. 62Diese hatten zwar nach
ihrer Geschlechtsurkunde gesucht, aber diese hatte sich nicht finden
lassen; infolgedessen wurden sie als unrein vom Priestertum
ausgeschlossen, 63und der Statthalter hatte ihnen erklärt, daß sie von
dem Hochheiligen nicht essen dürften, bis wieder ein Priester für die
Befragung des Urim- und Thummimorakels\textless sup title=``2.Mose
28,30''\textgreater✲ da wäre.

\hypertarget{c-gesamtzahl-der-personen-und-der-lasttiere-der-gemeinde}{%
\paragraph{c) Gesamtzahl der Personen und der Lasttiere der
Gemeinde}\label{c-gesamtzahl-der-personen-und-der-lasttiere-der-gemeinde}}

64Die ganze Gemeinde insgesamt belief sich auf 42360 Seelen,
65ungerechnet ihre Sklaven und Sklavinnen, deren 7337 da waren. Dazu
kamen noch 200 Sänger und Sängerinnen.~-- 66Die Zahl ihrer Pferde betrug
736, ihrer Maultiere 245, 67ihrer Kamele 435, der Esel 6720.

\hypertarget{d-beitruxe4ge-fuxfcr-den-tempelbau-in-jerusalem-schluuxdfwort}{%
\paragraph{d) Beiträge für den Tempelbau in Jerusalem;
Schlußwort}\label{d-beitruxe4ge-fuxfcr-den-tempelbau-in-jerusalem-schluuxdfwort}}

68Als sie dann beim Tempel des HERRN in Jerusalem angekommen waren,
spendeten einige von den Familienhäuptern freiwillige Gaben für das Haus
Gottes, damit man es an seiner früheren Stätte wieder aufrichten könne.
69Nach ihrem Vermögen gaben sie für den Bauschatz: an Gold 61000
Dariken, an Silber 5000 Minen und 100 Priestergewänder. 70Und es
siedelten sich die Priester und die Leviten sowie ein Teil des Volkes in
Jerusalem und dessen Gebieten an, die Sänger dagegen und die Torhüter
und die Tempelhörigen in ihren Ortschaften, und alle übrigen Israeliten
in ihren Ortschaften.

\hypertarget{der-anfang-des-tempelbaus}{%
\subsubsection{3. Der Anfang des
Tempelbaus}\label{der-anfang-des-tempelbaus}}

\hypertarget{a-erbauung-des-brandopferaltars-und-einrichtung-des-regelmuxe4uxdfigen-opferdienstes-feier-des-laubhuxfcttenfestes}{%
\paragraph{a) Erbauung des Brandopferaltars und Einrichtung des
regelmäßigen Opferdienstes; Feier des
Laubhüttenfestes}\label{a-erbauung-des-brandopferaltars-und-einrichtung-des-regelmuxe4uxdfigen-opferdienstes-feier-des-laubhuxfcttenfestes}}

\hypertarget{section-2}{%
\section{3}\label{section-2}}

1Als nun der siebte Monat herangekommen war -- die Israeliten befanden
sich bereits in ihren Ortschaften --, da kam das Volk wie ein
Mann\textless sup title=``=~bis auf den letzten Mann''\textgreater✲ in
Jerusalem zusammen. 2Da machten sich Jesua, der Sohn Jozadaks, mit
seinen Genossen, den Priestern, und Serubbabel, der Sohn Sealthiels, mit
seinen Genossen daran, den Altar des Gottes Israels wieder aufzubauen,
um Brandopfer auf ihm darzubringen, wie es im Gesetz Moses, des Mannes
Gottes, vorgeschrieben war; 3und zwar errichteten sie den Altar auf
seinem alten Unterbau; denn wenn sie auch in Angst vor den heidnischen
Bewohnern der (umliegenden) Landschaften waren, brachten sie doch dem
HERRN Brandopfer auf ihm dar, und zwar Brandopfer am Morgen und am
Abend. 4Dann begingen sie das Laubhüttenfest vorschriftsgemäß und
brachten dabei Brandopfer Tag für Tag in richtiger Zahl nach der
Verordnung der für jeden einzelnen Tag bestimmten Opfer dar\textless sup
title=``vgl. 4.Mose 29,12-38''\textgreater✲, 5danach auch das tägliche
Brandopfer und die Opfer (für die Sabbate und) für die Neumonde und für
alle heiligen Festzeiten des HERRN sowie für jeden, der dem HERRN eine
freiwillige Gabe darbrachte. 6Am ersten Tage des siebten Monats hatten
sie angefangen, dem HERRN Brandopfer darzubringen, obgleich damals der
Grundstein zum Tempel des HERRN noch nicht gelegt war.

\hypertarget{b-vorbereitungen-zum-tempelbau-feierliche-grundsteinlegung}{%
\paragraph{b) Vorbereitungen zum Tempelbau; feierliche
Grundsteinlegung}\label{b-vorbereitungen-zum-tempelbau-feierliche-grundsteinlegung}}

7Dann aber gaben sie den Steinhauern und Zimmerleuten Geld, außerdem
Speise, Trank und Öl den Sidoniern und Tyriern, damit sie Zedernstämme
vom Libanon auf dem Meere nach Japho✲ brächten, wozu der König Kores von
Persien ihnen die Erlaubnis erteilt hatte.

8Im zweiten Jahre aber nach ihrer Rückkehr zum Gotteshause in Jerusalem,
im zweiten Monat, machten Serubbabel, der Sohn Sealthiels, und Jesua,
der Sohn Jozadaks, nebst ihren übrigen\textless sup title=``=~allen
ihren''\textgreater✲ Genossen, den Priestern und den Leviten, überhaupt
alle, die aus der Verbannung\textless sup title=``oder:
Gefangenschaft''\textgreater✲ nach Jerusalem zurückgekehrt waren, den
Anfang mit dem Bau und bestellten die Leviten von zwanzig Jahren an und
darüber zu Aufsehern über die Arbeiten am Tempel des HERRN. 9So standen
denn Jesua nebst seinen Söhnen und Genossen, Kadmiel mit seinen Söhnen,
die Söhne Hodawjas\textless sup title=``vgl. 2,40''\textgreater✲ wie ein
Mann da, um die Aufsicht über die Arbeiter am Hause Gottes zu führen,
ebenso die Familie Henadad mit ihren Söhnen und Genossen, die Leviten.

10Als nun die Bauleute den Grund zum Tempel des HERRN legten, nahmen die
Priester in ihrer Amtstracht mit Trompeten und die Leviten, die
Nachkommen Asaphs, mit Zimbeln Aufstellung, um dem HERRN nach der
Anordnung Davids, des Königs von Israel, zu lobsingen. 11So stimmten sie
denn zu Ehren des HERRN das Lob- und Danklied an: »Denn er ist
freundlich, und seine Güte währet ewiglich über Israel.« Das ganze Volk
erhob dann ein lautes Jubelgeschrei, während man\textless sup
title=``oder: indem es''\textgreater✲ den HERRN pries wegen der
Grundsteinlegung zum Tempel des HERRN. 12Viele aber von den Priestern,
den Leviten und den Familienhäuptern, nämlich die alten Leute, die den
früheren Tempel noch (mit eigenen Augen) gesehen hatten, begannen, als
man den Grund zu diesem Hause legte, laut zu weinen, während die Menge
ihre Stimmen zu freudigem Jubel erhob. 13Man konnte aber den Schall des
Freudengeschreis von dem lauten Weinen im Volke nicht unterscheiden,
denn das Volk erhob ein gewaltiges Jubelgeschrei, so daß der Schall
weithin zu hören war.

\hypertarget{stuxf6rung-des-tempelbaus-durch-die-feindlichen-samaritaner}{%
\subsubsection{4. Störung des Tempelbaus durch die feindlichen
Samaritaner}\label{stuxf6rung-des-tempelbaus-durch-die-feindlichen-samaritaner}}

\hypertarget{a-abweisung-der-samaritaner-von-der-teilnahme-am-tempelbau}{%
\paragraph{a) Abweisung der Samaritaner von der Teilnahme am
Tempelbau}\label{a-abweisung-der-samaritaner-von-der-teilnahme-am-tempelbau}}

\hypertarget{section-3}{%
\section{4}\label{section-3}}

1Als aber die Widersacher Judas und Benjamins vernahmen, daß die aus der
Verbannung\textless sup title=``oder: Gefangenschaft''\textgreater✲
Zurückgekehrten dabei waren, dem HERRN, dem Gott Israels, einen Tempel
zu bauen, 2begaben sie sich zu Serubbabel und den Familienhäuptern und
sagten zu ihnen: »Wir möchten zusammen mit euch bauen; denn wir verehren
euren Gott ebenso wie ihr und wir opfern ihm seit der Zeit des
Assyrerkönigs Assarhaddon, der uns hierhergebracht (hier angesiedelt)
hat.« 3Aber Serubbabel und Jesua und die übrigen Familienhäupter der
Israeliten antworteten ihnen: »Es geht nicht an, daß ihr und wir
zusammen unserm Gott einen Tempel bauen, sondern wir wollen allein dem
HERRN, dem Gott Israels, (einen Tempel) bauen, wie der König Kores, der
König von Persien, uns geboten hat.« 4Da suchte die Bevölkerung des
Landes dem jüdischen Volke Schwierigkeiten für sein Unternehmen zu
schaffen und sie vom Bauen abzuschrecken; 5auch gewannen sie durch
Bestechung einflußreiche Männer (vom persischen Hofe) gegen sie, um ihr
Vorhaben zu hintertreiben, solange der König Kores von Persien lebte und
bis zur Regierung des Perserkönigs Darius.

\hypertarget{b-verschiedene-anklageschriften-gegen-die-juden-und-deren-tempel--und-mauerbau-unter-der-regierung-des-xerxes-und-artaxerxes}{%
\paragraph{b) Verschiedene Anklageschriften gegen die Juden und deren
Tempel- und Mauerbau unter der Regierung des Xerxes und
Artaxerxes}\label{b-verschiedene-anklageschriften-gegen-die-juden-und-deren-tempel--und-mauerbau-unter-der-regierung-des-xerxes-und-artaxerxes}}

6Als aber Ahasveros\textless sup title=``d.h. Xerxes''\textgreater✲ zur
Herrschaft gekommen war, verfaßten sie zu Anfang seiner Regierung eine
Anklageschrift gegen die Bewohner von Juda und Jerusalem; 7und unter der
Regierung Arthasasthas\textless sup title=``d.h.
Artaxerxes'''\textgreater✲ richteten Bislam, Mithredath, Tabeel und alle
seine Genossen an Arthasastha, den König von Persien, einen Bericht, der
mit aramäischer Schrift geschrieben und ins Aramäische
übersetzt\textless sup title=``d.h. in aramäischer Sprache
abgefaßt''\textgreater✲ war. 8Der Statthalter Rehum und der
Staatsschreiber✲ Simsai verfaßten einen Bericht an den König Arthasastha
gegen Jerusalem mit folgendem Wortlaut: 9Der Statthalter✲ Rehum und der
Staatsschreiber Simsai und alle ihre Genossen, die Dinäer {[}und
Apharsathchäer{]}, die Tarpeläer, Apharsachäer\textless sup
title=``=~Perser; vgl. 5,6 u. 6,6''\textgreater✲, Arkewäer, Babylonier,
Susaniter, Dehiter und Elamiter 10und alle übrigen Völkerschaften, die
der große und erlauchte Osnappar weggeführt und in den Ortschaften
Samarias und in den übrigen Gebieten jenseits✲ des Euphrats angesiedelt
hat\textless sup title=``vgl. 2.Kön 17,24''\textgreater✲, und so
weiter:~-- 11dies ist die Abschrift\textless sup title=``=~der
Wortlaut''\textgreater✲ des Berichts, den sie an ihn sandten:

»An den König Arthasastha: Deine Knechte, die Männer jenseits✲ des
Euphrats und so weiter. 12Es sei dem König zu wissen getan, daß die
Juden, die von dir zu uns heraufgezogen (und) nach Jerusalem gekommen
sind, die aufrührerische und böse Stadt wieder aufbauen und ihre Mauern
wieder herstellen und die Grundlagen ausbessern. 13Darum sei dem Könige
kundgetan, daß, wenn diese Stadt wieder aufgebaut wird und deren Mauern
wiederhergestellt werden, sie keine Abgaben, keine Steuern und Zölle
mehr entrichten werden, so daß schließlich das Königshaus Schaden davon
haben wird. 14Weil wir nun in Amt und Sold des königlichen Hofes stehen
und es uns nicht geziemt, eine Schädigung des Königs ruhig mitanzusehen,
deswegen senden wir diesen Bericht an den König, 15damit man im Buche
der Denkwürdigkeiten deiner Väter nachforsche. Du wirst dann im Buche
der Denkwürdigkeiten finden und dich überzeugen, daß diese Stadt eine
aufrührerische und für die Könige und Länder unheilvolle Stadt gewesen
ist und daß seit den ältesten Zeiten Empörungen in ihr stattgefunden
haben, weshalb diese Stadt ja auch zerstört worden ist. 16Wir machen
also den König darauf aufmerksam, daß, wenn diese Stadt wieder aufgebaut
wird und ihre Mauern wiederhergestellt werden, du infolgedessen nicht im
Besitz der Länder jenseits✲ des Euphrats verbleiben wirst.«

\hypertarget{c-stillstand-des-tempelbaues-infolge-einer-kuxf6niglichen-verfuxfcgung}{%
\paragraph{c) Stillstand des Tempelbaues infolge einer königlichen
Verfügung}\label{c-stillstand-des-tempelbaues-infolge-einer-kuxf6niglichen-verfuxfcgung}}

17Da sandte der König folgende Antwort: »An den Statthalter\textless sup
title=``vgl. V.8''\textgreater✲ Rehum und den Staatsschreiber Simsai und
alle ihre Genossen, die in Samaria und den übrigen Gegenden jenseits des
Euphrats ansässig sind: »Gruß euch! und so weiter. 18Das Schreiben, das
ihr uns habt zugehen lassen, ist mir Wort für Wort vorgelesen worden;
19und nachdem Befehl von mir erteilt worden war, nachzuforschen, hat es
sich herausgestellt, daß die betreffende Stadt sich seit den ältesten
Zeiten gegen die Könige aufgelehnt hat und daß Aufruhr und Empörungen in
ihr angestiftet worden sind. 20Auch haben mächtige Könige in Jerusalem
regiert und über alle Länder jenseits des Euphrats geherrscht, und
Abgaben, Steuern und Zölle sind ihnen entrichtet worden. 21So erlaßt nun
den Befehl, daß jenen Männern der Wiederaufbau ihrer Stadt untersagt
werde, bis von mir die Erlaubnis dazu erteilt wird; 22und seid auf der
Hut, eine Nachlässigkeit in dieser Sache vorkommen zu lassen, damit
nicht große Schädigung zum Nachteil des Königshauses daraus erwächst!«

23Sobald hierauf die Abschrift des Erlasses des Königs Arthasastha vor
Rehum und dem Staatsschreiber Simsai und ihren Genossen verlesen worden
war, begaben sie sich eiligst nach Jerusalem zu den Juden und zwangen
sie unter rücksichtsloser Anwendung von Gewalt zur Einstellung des
Baues. 24Damals hörte die Arbeit am Hause Gottes in Jerusalem auf und
blieb eingestellt bis zum zweiten Regierungsjahr des Königs Darius von
Persien.

\hypertarget{die-wiederaufnahme-und-vollendung-des-tempelbaus}{%
\subsubsection{5. Die Wiederaufnahme und Vollendung des
Tempelbaus}\label{die-wiederaufnahme-und-vollendung-des-tempelbaus}}

\hypertarget{a-guxfcnstige-weissagungen-zweier-propheten-erlaubnis-des-statthalters-zur-wiederaufnahme-des-baues}{%
\paragraph{a) Günstige Weissagungen zweier Propheten; Erlaubnis des
Statthalters zur Wiederaufnahme des
Baues}\label{a-guxfcnstige-weissagungen-zweier-propheten-erlaubnis-des-statthalters-zur-wiederaufnahme-des-baues}}

\hypertarget{section-4}{%
\section{5}\label{section-4}}

1Es weissagten aber der Prophet Haggai und der Prophet Sacharja, der
Sohn Iddos, den Juden in Juda und Jerusalem im Namen des Gottes Israels,
dessen Geist über ihnen war. 2Daraufhin machten sich Serubbabel, der
Sohn Sealthiels, und Jesua, der Sohn Jozadaks, daran, den Bau des
Gotteshauses in Jerusalem aufs neue in Angriff zu nehmen, und mit ihnen
waren die (beiden) Propheten Gottes, welche sie unterstützten.

3Zu jener Zeit kamen Thathnai, der Statthalter der Provinz diesseits✲
des Euphrats, und Sethar-Bosnai und deren Genossen zu ihnen und
richteten die Frage an sie: »Wer hat euch die Erlaubnis gegeben, dieses
Haus zu bauen und diese Mauer\textless sup title=``=~oder: dieses
Heiligtum''\textgreater✲ wiederherzustellen, 4und welches sind die Namen
der Männer, die diesen Bau betreiben?« 5Aber das Auge ihres Gottes war
auf die Ältesten der Juden gerichtet, so daß (jene) von ihnen nicht die
Einstellung der Arbeit verlangten, bis ein Befehl des Darius vorläge und
man ihnen dann einen schriftlichen Bescheid darüber zugehen ließe.

\hypertarget{b-bericht-und-anfrage-des-statthalters-an-den-kuxf6nig-darius-bezuxfcglich-des-tempelbaus}{%
\paragraph{b) Bericht und Anfrage des Statthalters an den König Darius
bezüglich des
Tempelbaus}\label{b-bericht-und-anfrage-des-statthalters-an-den-kuxf6nig-darius-bezuxfcglich-des-tempelbaus}}

6Abschrift✲ des Berichts, den Thathnai, der Statthalter der Provinz
diesseits✲ des Euphrats, und Sethar-Bosnai und seine Genossen, die
Apharsachäer✲, die in der Provinz diesseits des Euphrats wohnten, an den
König Darius sandten. 7Sie sandten nämlich einen Bericht an ihn, dessen
Wortlaut folgender war:

»Dem König Darius alles Heil! 8Es sei dem Könige zu wissen getan, daß
wir uns in die Landschaft✲ Juda zum Hause\textless sup title=``oder:
Tempel''\textgreater✲ des großen Gottes begeben haben; es wird aus
Quadersteinen erbaut, und in die Wände werden Balken eingelegt; diese
Arbeit wird eifrig betrieben und geht unter ihren Händen erfolgreich
vonstatten. 9Da haben wir an die Ältesten dort die Frage gerichtet: ›Wer
hat euch den Befehl\textless sup title=``oder: die
Erlaubnis''\textgreater✲ gegeben, dieses Haus zu bauen und diese
Mauer\textless sup title=``oder: dieses Heiligtum''\textgreater✲
wiederherzustellen?‹ 10Und auch nach ihren Namen haben wir sie gefragt,
um sie dir mitzuteilen, indem wir die Namen der Männer aufschrieben, die
an ihrer Spitze stehen. 11Und folgendes ist die Auskunft, die sie uns
gegeben haben: ›Wir sind die Knechte✲ des Gottes des Himmels und der
Erde und bauen das Haus wieder auf, das ehedem vor vielen Jahren hier
gestanden hat und das ein großer israelitischer König erbaut und
aufgeführt hatte. 12Weil aber unsere Väter den Gott des Himmels erzürnt
hatten, hat er sie der Gewalt des Königs Nebukadnezar von Babylon, des
Chaldäers, preisgegeben; der hat dieses Haus zerstört und das Volk
gefangen\textless sup title=``oder: in die Verbannung''\textgreater✲
nach Babylon geführt. 13Doch im ersten Jahre der Regierung des Königs
Kores über Babylon hat der König Kores den Befehl gegeben, dieses
Gotteshaus wieder aufzubauen. 14Auch die goldenen und silbernen Geräte
des Gotteshauses, die Nebukadnezar aus dem Tempel zu Jerusalem
weggenommen und in (seinen) Tempel zu Babylon gebracht hatte, hat der
König Kores aus dem Tempel zu Babylon hervornehmen lassen und sie einem
Manne namens Sesbazzar ausgeliefert, den er zum Statthalter (in Judäa)
eingesetzt hatte, 15indem er diesem auftrug: Nimm diese Geräte, gehe hin
und lege sie im Tempel zu Jerusalem nieder; das Gotteshaus aber soll an
seiner früheren Stelle wieder aufgebaut werden! 16Daraufhin ist der
betreffende Sesbazzar hergekommen und hat den Grundstein\textless sup
title=``oder: die Grundmauern''\textgreater✲ zum Gotteshause in
Jerusalem gelegt; und seit jener Zeit wird bis heute daran gebaut, es
ist aber noch nicht vollendet.‹ 17Demnach möge man, wenn es dem Könige
beliebt, im königlichen Schatzhause\textless sup title=``vgl.
6,1''\textgreater✲ dort in Babylon nachforschen, ob es sich wirklich so
verhält, daß vom Könige Kores der Befehl gegeben worden ist, dieses
Gotteshaus in Jerusalem wieder aufzubauen. Der König wolle uns dann
seine Entscheidung in dieser Sache zukommen lassen.«

\hypertarget{c-auffindung-der-verfuxfcgung-des-cyrus-in-ekbatana-und-mitteilungen-daraus}{%
\paragraph{c) Auffindung der Verfügung des Cyrus in Ekbatana und
Mitteilungen
daraus}\label{c-auffindung-der-verfuxfcgung-des-cyrus-in-ekbatana-und-mitteilungen-daraus}}

\hypertarget{section-5}{%
\section{6}\label{section-5}}

1Als hierauf der König Darius anordnete, im Schatzhause, in welchem man
auch die Urkunden in Babylon aufbewahrte, nachzusehen, 2fand man in
Ahmetha\textless sup title=``d.h. Ekbatana''\textgreater✲, in der
Königsstadt✲, die in der Landschaft✲ Medien liegt, eine Schriftrolle, in
der folgendes geschrieben stand:

3»Urkunde: Im ersten Regierungsjahre des Königs Kores erließ der König
Kores den Befehl: ›Was das Gotteshaus in Jerusalem betrifft, so soll
dieses Haus wieder aufgebaut werden als eine Stätte, wo man
Schlachtopfer schlachtet und Feueropfer darbringt; seine Höhe soll
sechzig Ellen, seine Breite auch sechzig Ellen betragen; 4Schichten von
Quadersteinen sollen drei vorhanden sein und eine Schicht von
Holzbalken; die Kosten aber sollen aus der königlichen Kasse bestritten
werden. 5Auch die goldenen und silbernen Geräte des Gotteshauses, die
Nebukadnezar aus dem Tempel zu Jerusalem weggenommen und nach Babylon
gebracht hat, sollen zurückgegeben werden, so daß jedes Stück wieder in
den Tempel zu Jerusalem an seinen früheren Ort gelangt; und man soll sie
in dem Gotteshause niederlegen!‹«

\hypertarget{d-verfuxfcgung-des-darius-zur-ungehinderten-fortsetzung-und-zur-fuxf6rderung-des-tempelbaues}{%
\paragraph{d) Verfügung des Darius zur ungehinderten Fortsetzung und zur
Förderung des
Tempelbaues}\label{d-verfuxfcgung-des-darius-zur-ungehinderten-fortsetzung-und-zur-fuxf6rderung-des-tempelbaues}}

6»Nun denn, Thathnai, Statthalter der Provinz jenseits✲ des Euphrats,
und du, Sethar-Bosnai, und eure Genossen, die Apharsachäer, die ihr
jenseits des Euphrats wohnt, haltet euch fern von dort! 7Laßt die
Arbeiten an diesem Gotteshause ungestört fortgehen: der Statthalter von
Judäa und die Ältesten der Juden mögen dieses Gotteshaus an seiner alten
Stelle wieder aufbauen! 8Auch ist von mir eine Verfügung erlassen worden
bezüglich der Leistungen, die ihr jenen Ältesten der Juden für den Bau
dieses Gotteshauses zu gewähren habt, nämlich daß von den Steuererträgen
des Königs aus der Provinz jenseits des Euphrats die Kosten jenen
Männern genau\textless sup title=``oder: pünktlich''\textgreater✲
erstattet werden, und zwar unverzüglich. 9Und was (sonst) erforderlich
ist sowohl an jungen Stieren als auch an Widdern und Lämmern zu
Brandopfern für den Gott des Himmels sowie an Weizen, Salz, Wein und Öl,
das soll ihnen nach der Anforderung der Priester zu Jerusalem Tag für
Tag unweigerlich\textless sup title=``oder: unverkürzt''\textgreater✲
geliefert werden, 10damit sie dem Gott des Himmels lieblichen Opferduft
darbringen und für das Leben des Königs und seiner Söhne\textless sup
title=``=~der Prinzen''\textgreater✲ beten. 11Weiter ist von mir
verordnet worden, daß jedem, der an dieser Verfügung etwas ändern
sollte, ein Pfosten\textless sup title=``oder: Balken''\textgreater✲ aus
seinem Hause herausgerissen und er selbst gepfählt daran aufgehängt und
sein Haus wegen solchen Vergehens zu einem Schutthaufen gemacht werden
soll. 12Der Gott aber, der seinen Namen dort dauernd wohnen läßt, möge
jeden König und jedes Volk stürzen, die es unternehmen sollten, diese
Verfügung zu übertreten, um dieses Gotteshaus in Jerusalem zu zerstören.
Ich, Darius, habe den Befehl erlassen: er soll genau\textless sup
title=``oder: pünktlich''\textgreater✲ vollzogen werden!«

\hypertarget{e-vollendung-und-feierliche-einweihung-des-tempels}{%
\paragraph{e) Vollendung und feierliche Einweihung des
Tempels}\label{e-vollendung-und-feierliche-einweihung-des-tempels}}

13Darauf verfuhren Thathnai, der Statthalter der Provinz Syrien, und
Sethar-Bosnai nebst ihren Genossen genau nach dem Befehl, den der König
Darius ihnen hatte zugehen lassen. 14So konnten denn die Ältesten der
Juden weiterbauen, und die Arbeit ging erfolgreich vonstatten gemäß der
Weissagung der Propheten, nämlich Haggais und Sacharjas, des Sohnes
Iddos, so daß sie den Bau zu Ende führten nach dem Befehl des Gottes
Israels und nach dem Befehl des Kores und des Darius und des Königs
Arthasastha von Persien. 15Sie stellten aber dieses Haus fertig bis zum
dritten Tage des Monats Adar, und zwar war es das sechste Regierungsjahr
des Königs Darius. 16Da feierten denn die Israeliten, die Priester, die
Leviten und die übrigen aus der Gefangenschaft\textless sup
title=``oder: Verbannung''\textgreater✲ Zurückgekehrten die Einweihung
dieses Gotteshauses mit einem Freudenfest 17und opferten
zur\textless sup title=``oder: bei der''\textgreater✲ Einweihung dieses
Gotteshauses hundert Stiere, zweihundert Widder, vierhundert Lämmer als
Brandopfer und als Sündopfer für ganz Israel zwölf Ziegenböcke nach der
Zahl der israelitischen Stämme. 18Sie setzten dann auch die Priester in
ihr Amt ein nach ihren Abteilungen und die Leviten nach ihren Klassen
für den Gottesdienst in Jerusalem, wie es im Buche Moses vorgeschrieben
ist.

\hypertarget{f-feier-des-passahfestes}{%
\paragraph{f) Feier des Passahfestes}\label{f-feier-des-passahfestes}}

19Hierauf begingen die aus der Gefangenschaft\textless sup title=``oder:
Verbannung''\textgreater✲ Zurückgekehrten das Passah am vierzehnten Tage
des ersten Monats; 20denn die Priester und die Leviten hatten sich ohne
Ausnahme gereinigt; sie waren allesamt rein und schlachteten das Passah
für alle aus der Gefangenschaft\textless sup title=``oder:
Verbannung''\textgreater✲ Zurückgekehrten und für ihre
Geschlechtsgenossen, die Priester, und für sich selbst. 21Es aßen aber
(das Passah) sowohl die Israeliten, die aus der
Gefangenschaft\textless sup title=``oder: Verbannung''\textgreater✲
zurückgekehrt waren, als auch alle, die sich von der Unreinheit der
(heidnischen) Bevölkerung des Landes losgesagt und sich ihnen
angeschlossen hatten, um den HERRN, den Gott Israels, zu verehren.
22Dann begingen sie das Fest der ungesäuerten Brote sieben Tage lang mit
Freuden; denn der HERR hatte sie Freude erleben lassen, indem er ihnen
das Herz des Königs von Assyrien zugewandt hatte, so daß er ihre Arbeit
beim Bau des Hauses Gottes, des Gottes Israels, kräftig unterstützte.

\hypertarget{ii.-der-priester-esra-kap.-7-10}{%
\subsection{II. Der Priester Esra (Kap.
7-10)}\label{ii.-der-priester-esra-kap.-7-10}}

\hypertarget{esras-reise-nach-jerusalem}{%
\subsubsection{1. Esras Reise nach
Jerusalem}\label{esras-reise-nach-jerusalem}}

\hypertarget{a-die-ruxfcckwanderung-esras-und-seiner-schar-von-babylon-nach-jerusalem}{%
\paragraph{a) Die Rückwanderung Esras und seiner Schar von Babylon nach
Jerusalem}\label{a-die-ruxfcckwanderung-esras-und-seiner-schar-von-babylon-nach-jerusalem}}

\hypertarget{section-6}{%
\section{7}\label{section-6}}

1Nach diesen Begebenheiten nun unter der Regierung des Perserkönigs
Arthasastha\textless sup title=``d.h. Artaxerxes''\textgreater✲ zog
Esra, der Sohn Serajas, des Sohnes Asarjas, des Sohnes Hilkias, 2des
Sohnes Sallums, des Sohnes Zadoks, des Sohnes Ahitubs, 3des Sohnes
Amarjas, des Sohnes Asarjas, des Sohnes Merajoths, 4des Sohnes Serahjas,
des Sohnes Ussis, des Sohnes Bukkis, 5des Sohnes Abisuas, des Sohnes des
Pinehas, des Sohnes Eleasars, des Sohnes des Oberpriesters Aaron~--
6eben dieser Esra zog von Babylon herauf. Er war ein Schriftgelehrter,
wohlbewandert im mosaischen Gesetz, das der HERR, der Gott Israels,
gegeben hatte; und weil die gütige Hand des HERRN, seines Gottes, über
ihm waltete, hatte der König ihm alles bewilligt, um was er gebeten
hatte. 7So zog denn mit ihm ein Teil der Israeliten und der Priester,
Leviten, Sänger, Torhüter und Tempelhörigen im siebten Regierungsjahre
des Königs Arthasastha nach Jerusalem hinauf; 8und er gelangte nach
Jerusalem im fünften Monat dieses siebten Regierungsjahres des Königs.
9Auf den ersten Tag des ersten Monats nämlich hatte Esra den Aufbruch
von Babylon festgesetzt, und am ersten Tage des fünften Monats kam er in
Jerusalem an, weil die gütige Hand seines Gottes über ihm waltete;
10denn Esra hatte sein Streben fest darauf gerichtet, das Gesetz des
HERRN zu erforschen und durchzuführen und in Israel Satzung und Recht zu
lehren.

\hypertarget{b-wortlaut-des-kuxf6niglichen-schreibens-geleitbriefes-mit-angabe-der-dem-esra-zustehenden-vollmachten}{%
\paragraph{b) Wortlaut des königlichen Schreibens (=~Geleitbriefes) mit
Angabe der dem Esra zustehenden
Vollmachten}\label{b-wortlaut-des-kuxf6niglichen-schreibens-geleitbriefes-mit-angabe-der-dem-esra-zustehenden-vollmachten}}

11Folgendes ist nun die Abschrift\textless sup title=``=~der
Wortlaut''\textgreater✲ des Schreibens, das der König Arthasastha dem
schriftgelehrten Priester Esra, der ein Gelehrter in den Worten der
Gebote des HERRN und seiner Satzungen für Israel war, mitgegeben hatte:
12»Arthasastha, der König der Könige✲, wünscht dem Priester Esra, der
ein vollkommener Gelehrter im Gesetz des Himmelsgottes ist, Heil und so
weiter. 13Von mir ergeht hiermit der Befehl, daß ein jeder, der in
meinem Reiche vom Volk Israel und von seinen Priestern und Leviten
gewillt ist, nach Jerusalem zu ziehen, mit dir soll ziehen dürfen,
14weil du ja doch vom Könige und seinen sieben Räten gesandt bist, um
über die Verhältnisse in Juda und Jerusalem eine Untersuchung
anzustellen auf Grund des Gesetzes deines Gottes, das du in deinen
Händen hast, 15und um das Silber und Gold dorthin zu bringen, welches
der König und seine Räte dem Gott Israels, dessen Wohnsitz in Jerusalem
ist, als Weihgeschenk gespendet haben 16sowie alles Silber und Gold,
welches du in der ganzen Landschaft✲ Babylon erhalten wirst, mitsamt den
Weihgeschenken des Volkes und der Priester, welche freiwillige Gaben für
das Haus ihres Gottes in Jerusalem spenden werden. 17Dementsprechend
sollst du für dieses Geld gewissenhaft Stiere, Widder und Lämmer nebst
den zugehörigen Speisopfern und den erforderlichen Trankopfern kaufen
und sollst sie auf dem Altar eures Gotteshauses in Jerusalem darbringen.
18Was dann dir und deinen Genossen mit dem übrigen Silber und Gold zu
tun gut erscheint, das mögt ihr nach dem Willen eures Gottes tun.
19Ferner sollst du die Geräte, die dir für den Dienst im Hause deines
Gottes übergeben werden, vollzählig vor dem Gott zu Jerusalem abliefern.
20Und den weiteren Bedarf des Hauses deines Gottes, alle Kosten, deren
Bestreitung dir etwa obliegen wird, darfst du aus der königlichen
Schatzkammer decken lassen; 21es wird nämlich von mir, dem Könige
Arthasastha, hiermit allen Schatzmeistern✲ in der Provinz jenseits des
Euphrats der Befehl erteilt: Alles, was der Priester Esra, der im Gesetz
des Himmelsgottes gelehrte Mann, von euch verlangen wird, soll pünktlich
geleistet werden, 22bis zu hundert Talenten Silber und bis zu hundert
Kor Weizen und bis zu hundert Bath Wein und hundert Bath Öl, dazu Salz
in unbeschränkter Menge. 23Alles, was nach dem Befehl des Himmelsgottes
erforderlich ist, soll für das Haus des Himmelsgottes sorgfältig
geleistet werden, damit nicht etwa ein Zorngericht das Reich des Königs
und seiner Söhne trifft. 24Weiter sei euch hiermit kundgetan, daß
niemand berechtigt sein soll, irgendeinem Priester und Leviten, Sänger,
Torhüter, Tempelhörigen, kurz irgendeinem Diener dieses Gotteshauses
eine Geldabgabe, eine Steuer oder Zölle aufzuerlegen. 25Du aber, Esra,
setze gemäß der Weisheit (des Gesetzes) deines Gottes, das in deinen
Händen ist, Richter und Rechtspfleger\textless sup title=``oder:
Schöffen?''\textgreater✲ ein, die dem gesamten Volke in der Provinz
jenseits des Euphrats Recht sprechen sollen, nämlich allen denen, welche
die Gesetze deines Gottes kennen, und wer sie noch nicht kennt, dem
sollt ihr Belehrung zuteil werden lassen! 26Jeder aber, der dem Gesetz
deines Gottes und dem Gesetz des Königs nicht nachkommt, gegen den soll
gerichtlich mit Strenge vorgegangen werden, es sei mit Todesstrafe oder
Verbannung, mit Geldbuße oder Gefängnis!«

\hypertarget{c-dankgebet-esras-und-beginn-seiner-tuxe4tigkeit}{%
\paragraph{c) Dankgebet Esras und Beginn seiner
Tätigkeit}\label{c-dankgebet-esras-und-beginn-seiner-tuxe4tigkeit}}

27»Gepriesen sei der HERR, der Gott unserer Väter, der dem Könige den
Entschluß ins Herz gegeben hat, das Haus des HERRN in Jerusalem zu
verherrlichen, 28und der mich Gnade bei dem Könige und seinen Räten und
bei allen einflußreichen Würdenträgern des Königs hat finden lassen, so
daß ich, weil die Hand des HERRN, meines Gottes, schützend über mir
waltete, den Mut gewann, Familienhäupter aus Israel um mich zu sammeln,
damit sie mit mir nach Jerusalem hinaufzögen!«

\hypertarget{d-verzeichnis-der-mit-esra-ruxfcckwandernden-juduxe4ischen-familienhuxe4upter}{%
\paragraph{d) Verzeichnis der mit Esra rückwandernden judäischen
Familienhäupter}\label{d-verzeichnis-der-mit-esra-ruxfcckwandernden-juduxe4ischen-familienhuxe4upter}}

\hypertarget{section-7}{%
\section{8}\label{section-7}}

1Folgendes ist das Verzeichnis der Familienhäupter nebst der Angabe
ihrer Geschlechtszugehörigkeit, nämlich derer, die unter der Regierung
des Königs Arthasastha mit mir von Babylon (nach Jerusalem)
hinaufgezogen sind: 2Von den Nachkommen des Pinehas: Gersom; von den
Nachkommen Ithamars: Daniel; von den Nachkommen Davids: Hattus, 3(der
Sohn) Sechanjas; von den Nachkommen des Parhos: Sacharja, und mit ihm
waren aufgezeichnet an männlichen Personen 150; 4von den Nachkommen
Pahath-Moabs: Eljehoenai, der Sohn Serahjas, und mit ihm 200 männliche
Personen; 5von den Nachkommen Satthus: Sechanja, der Sohn Jahasiels, und
mit ihm 300 männliche Personen; 6von den Nachkommen Adins: Ebed, der
Sohn Jonathans, und mit ihm 50 männliche Personen; 7von den Nachkommen
Elams: Jesaja, der Sohn Athaljas, und mit ihm 70 männliche Personen;
8von den Nachkommen Sephatjas: Sebadja, der Sohn Michaels, und mit ihm
80 männliche Personen; 9von den Nachkommen Joabs: Obadja, der Sohn
Jehiels, und mit ihm 218 männliche Personen; 10von den Nachkommen
(Banis): Selomith, der Sohn Josiphjas, und mit ihm 160 männliche
Personen; 11von den Nachkommen Bebais: Sacharja, der Sohn Bebais, und
mit ihm 28 männliche Personen; 12von den Nachkommen Asgads: Johanan, der
Sohn Hakkatans, und mit ihm 110 männliche Personen; 13von den Nachkommen
Adonikams: Spätlinge, und dies sind ihre Namen: Eliphelet, Jehiel und
Semaja, und mit ihnen 60 männliche Personen; 14von den Nachkommen
Bigwais: Uthai und Sabbud, und mit ihnen 70 männliche Personen.

\hypertarget{e-die-letzten-vorbereitungen-zur-abreise}{%
\paragraph{e) Die letzten Vorbereitungen zur
Abreise}\label{e-die-letzten-vorbereitungen-zur-abreise}}

15Ich ließ sie dann an dem Flusse\textless sup title=``oder:
Kanal''\textgreater✲, der nach Ahawa fließt, zusammenkommen, und wir
lagerten dort drei Tage lang. Als ich da nun wohl das Volk und die
Priester wahrnahm, aber von den Leviten keinen einzigen
dort\textless sup title=``=~unter ihnen''\textgreater✲ fand, 16sandte
ich Elieser, Ariel, Semaja, Elnathan, Jarib, Elnathan, Nathan, Sacharja
und Mesullam, lauter Familienhäupter, sowie Jojarib und Elnathan, beides
einsichtige Männer\textless sup title=``oder: Lehrer''\textgreater✲, ab
17und beauftragte sie, sich zu Iddo, dem Vorsteher in der Ortschaft
Kasiphja, zu begeben; dabei gab ich ihnen genau die Worte an, die sie an
Iddo und seine Genossen in der Ortschaft Kasiphja richten sollten,
nämlich daß sie uns Diener für das Haus unseres Gottes zuführen möchten.
18Da brachten sie uns, weil die gütige Hand unsers Gottes über uns
waltete, einen einsichtsvollen Mann von den Nachkommen Mahlis, des
Sohnes Levis, des Sohnes Israels, nämlich Serebja samt seinen Söhnen und
Genossen, 18 an der Zahl, 19ferner Hasabja und mit ihm Jesaja von den
Nachkommen Meraris samt ihren Genossen und deren Söhnen, 20~Mann; 20und
von den Tempelhörigen, die David und die Fürsten den Leviten zu
Dienstleistungen überwiesen hatten, 220~Mann; sie sind alle mit Namen
verzeichnet.

\hypertarget{fasten-und-gebet-der-heimkehrenden-uxfcbergabe-der-weihgeschenke-fuxfcr-den-tempel-an-zuverluxe4ssige-muxe4nner}{%
\paragraph{Fasten und Gebet der Heimkehrenden; Übergabe der
Weihgeschenke für den Tempel an zuverlässige
Männer}\label{fasten-und-gebet-der-heimkehrenden-uxfcbergabe-der-weihgeschenke-fuxfcr-den-tempel-an-zuverluxe4ssige-muxe4nner}}

21Ich ließ nun dort am Flusse\textless sup title=``oder:
Kanal''\textgreater✲ Ahawa ein Fasten ausrufen, damit wir uns vor unserm
Gott demütigten, um von ihm eine glückliche Reise für uns und unsere
Familien\textless sup title=``d.h. Frauen und Kinder''\textgreater✲ und
für alle unsere Habe zu erflehen. 22Ich hatte mich nämlich geschämt, den
König um eine bewaffnete Mannschaft und um Reiter zu bitten, die uns
unterwegs gegen Feinde hätten schützen können; wir hatten vielmehr dem
Könige erklärt: »Die Hand unsers Gottes waltet über allen, die ihn
suchen\textless sup title=``=~sich an ihn wenden''\textgreater✲, zu
ihrem Heil, aber seine Macht und sein Zorn trifft alle, die ihn
verlassen.« 23So fasteten wir denn und flehten unsern Gott in dieser
Sache um Hilfe an, und er ließ sich von uns erbitten.

24Hierauf wählte ich aus den obersten Priestern zwölf aus, ferner auch
Serebja und Hasabja und mit ihnen zehn von ihren Genossen, 25und wog
ihnen das Silber und das Gold und die Geräte dar, das Weihgeschenk für
das Haus unsers Gottes, das der König samt seinen Räten und
Würdenträgern sowie alle dort wohnhaften Israeliten gespendet hatten.
26Ich wog ihnen also in ihre Hand\textless sup title=``=~ihre
Obhut''\textgreater✲ dar: an Silber 650 Talente, an silbernen Geräten
100 Talente, an Gold 100 Talente; 27ferner 20 goldene Becher im Werte
von 1000 Goldstücken und zwei Gefäße von Golderz✲, kostbar wie Gold.
28Dabei sagte ich zu ihnen: »Ihr seid dem HERRN heilig, und die Geräte
sind auch heilig, und das Silber und das Gold ist ein Weihgeschenk für
den HERRN, den Gott eurer Väter. 29Seid also wachsam und hütet es, bis
ihr es vor den obersten Priestern und Leviten und den obersten
Familienhäuptern der Israeliten zu Jerusalem in den Zellen des Tempels
des HERRN darwägen könnt!« 30Darauf nahmen die Priester und die Leviten
das ihnen zugewogene Silber und Gold und die Geräte in Empfang, um sie
nach Jerusalem in das Haus unsers Gottes zu bringen.

\hypertarget{f-ankunft-in-jerusalem-ablieferung-der-weihgeschenke-darbringung-von-opfern-unterstuxfctzung-von-seiten-der-kuxf6niglichen-beamten}{%
\paragraph{f) Ankunft in Jerusalem; Ablieferung der Weihgeschenke;
Darbringung von Opfern; Unterstützung von seiten der königlichen
Beamten}\label{f-ankunft-in-jerusalem-ablieferung-der-weihgeschenke-darbringung-von-opfern-unterstuxfctzung-von-seiten-der-kuxf6niglichen-beamten}}

31Hierauf brachen wir am zwölften Tage des ersten Monats vom
Flusse\textless sup title=``oder: Kanal''\textgreater✲ Ahawa auf, um
nach Jerusalem zu ziehen; und die Hand unsers Gottes beschützte uns, und
er behütete uns vor (Angriffen von) Feinden und Straßenräubern. 32So
kamen wir denn in Jerusalem an und ruhten dort drei Tage lang aus. 33Am
vierten Tage aber wurden das Silber und das Gold und die Geräte im Hause
unsers Gottes dem Priester Meremoth, dem Sohne Urias, in die Hand
dargewogen -- außer diesem war noch Eleasar, der Sohn des Pinehas,
zugegen und außer diesen beiden noch die Leviten Josabad, der Sohn
Jesuas, und Noadja, der Sohn Binnuis --; 34alles wurde damals gezählt
und nachgewogen und das Gesamtgewicht aufgeschrieben.~-- 35Als dann die
aus der Gefangenschaft heimgekehrten Verbannten dem Gott Israels
Brandopfer dargebracht hatten, nämlich 12 Stiere für ganz Israel, 96
Widder, 77\textless sup title=``oder: 72?''\textgreater✲ Lämmer und zum
Sündopfer 12 Böcke, das alles als Brandopfer für den HERRN, 36übergaben
sie die Befehle✲ des Königs den königlichen Satrapen✲ und Statthaltern
der Provinz auf der Westseite des Euphrats, worauf diese das Volk und
das Gotteshaus unterstützten.

\hypertarget{esra-reinigt-die-gemeinde-von-den-mischehen}{%
\subsubsection{2. Esra reinigt die Gemeinde von den
Mischehen}\label{esra-reinigt-die-gemeinde-von-den-mischehen}}

\hypertarget{a-esra-erhuxe4lt-kenntnis-von-den-mischheiraten-seine-bestuxfcrzung-daruxfcber}{%
\paragraph{a) Esra erhält Kenntnis von den Mischheiraten; seine
Bestürzung
darüber}\label{a-esra-erhuxe4lt-kenntnis-von-den-mischheiraten-seine-bestuxfcrzung-daruxfcber}}

\hypertarget{section-8}{%
\section{9}\label{section-8}}

1Als nun dieses abgemacht war, traten die Obersten✲ zu mir und sagten:
»Das Volk Israel, auch die Priester und die Leviten haben sich von den
Völkerschaften des Landes, trotz deren greulichem Götzendienst, nicht
abgesondert gehalten, nämlich von den Kanaanäern, Hethitern,
Pherissitern, Jebusitern, Ammonitern, Moabitern, Ägyptern und
Amoritern\textless sup title=``oder: Edomitern''\textgreater✲; 2sie
haben vielmehr von deren Töchtern Frauen für sich und ihre Söhne
genommen, und so hat sich der heilige Same\textless sup title=``oder:
das heilige Geschlecht''\textgreater✲ mit den (heidnischen)
Völkerschaften des Landes vermischt, und die Obersten und Vorsteher
haben zu dieser Treulosigkeit zuerst die Hand geboten.« 3Als ich diese
Mitteilung vernahm, zerriß ich mir das Gewand und den Mantel, raufte mir
das Haar aus Kopf und Bart aus und setzte mich erstarrt✲ nieder. 4Da
versammelten sich um mich alle, die in Angst waren vor den Worten✲ des
Gottes Israels wegen des Frevels\textless sup title=``=~der
Untreue''\textgreater✲ der aus der Gefangenschaft Zurückgekehrten; ich
aber saß erstarrt da bis zum Abendopfer.

\hypertarget{b-esras-buuxdfgebet}{%
\paragraph{b) Esras Bußgebet}\label{b-esras-buuxdfgebet}}

5Um die Zeit des Abendopfers aber erhob ich mich von meiner
Selbstdemütigung\textless sup title=``oder: aus meiner
Bußstellung''\textgreater✲, in der ich mein Gewand und meinen Mantel
zerrissen hatte, warf mich auf die Knie nieder, breitete meine Hände zum
HERRN, meinem Gott, aus 6und betete: »Mein Gott! Ich schäme mich und
erröte, mein Angesicht zu dir, mein Gott, zu erheben; denn unsere
Missetaten sind uns über das Haupt gewachsen, und unsere Schuld ist groß
geworden bis an den Himmel! 7Seit den Tagen unserer Väter stehen wir in
großer Schuld bis auf den heutigen Tag, und um unserer Missetaten willen
sind wir, unsere Könige und unsere Priester, der Gewalt der Könige der
(heidnischen) Länder preisgegeben worden, dem Schwert, der
Gefangenschaft, der Plünderung und schmachvollsten Entehrung, wie es
noch heutigen Tages der Fall ist. 8Jetzt ist uns zwar für einen kurzen
Augenblick Gnade\textless sup title=``oder: Erbarmen''\textgreater✲ vom
HERRN, unserm Gott, dadurch widerfahren, daß er uns einen Rest
Geretteter übriggelassen und uns an der Stätte seines Heiligtums einen
Zeltpflock\textless sup title=``=~sicheren Wohnsitz''\textgreater✲
geschenkt hat, damit unser Gott unsere Augen wieder leuchten mache und
uns in unserer Knechtschaft ein wenig aufleben lasse. 9Denn ob wir auch
Knechte✲ sind, hat unser Gott uns doch in unserer Knechtschaft nicht
verlassen, sondern hat uns die Huld der Könige von Persien zugewandt, so
daß er uns ein Aufleben vergönnt hat, um das Haus unsers Gottes wieder
aufzubauen und es aus seinen Trümmern wieder erstehen zu lassen und uns
eine Mauer\textless sup title=``d.h. einen ummauerten Ort, sicheren
Wohnsitz''\textgreater✲ in Juda und Jerusalem zu gewähren. 10Jetzt aber,
o unser Gott -- was sollen wir nach solchen Vorkommnissen sagen? Wir
haben ja deine Gebote unbeachtet gelassen, 11die du uns durch deine
Knechte, die Propheten, zur Pflicht gemacht hast mit den
Worten\textless sup title=``3.Mose 18,24-25''\textgreater✲: ›Das Land,
in welches ihr zieht, um es in Besitz zu nehmen, ist ein Land, das
infolge der Unreinheit der heidnischen Völkerschaften befleckt ist
infolge ihrer Götzengreuel, mit denen sie es bei ihrer Unreinheit von
einem Ende bis zum andern angefüllt haben. 12So sollt ihr nun eure
Töchter nicht ihren Söhnen zu Frauen geben und ihre Töchter nicht für
eure Söhne zu Frauen nehmen und nun und nimmer auf ihre Wohlfahrt und
ihr Wohlergehen bedacht sein, damit ihr stark bleibt und die Güter des
Landes genießt und es auf eure Söhne\textless sup title=``oder:
Kinder''\textgreater✲ für ewige Zeiten vererbt.‹ 13Und nun nach allem
Unheil, das uns infolge unserer bösen Taten und unserer großen Schuld
widerfahren ist -- wiewohl du, unser Gott, größere Schonung gegen uns
geübt hast, als unsere Sünden verdient haben, und uns diesen geretteten
Rest hier geschenkt hast --: 14sollten wir da aufs neue deine Gebote
übertreten und uns mit diesen Greuelvölkern verschwägern? Müßtest du uns
da nicht bis zur Vernichtung zürnen, so daß niemand mehr (von uns) übrig
bliebe noch entrinnen könnte? 15O HERR, Gott Israels! Du bist gerecht
darin, daß wir nur noch als ein Rest von Geretteten übriggeblieben sind,
wie es heutigentags der Fall ist: Ach, siehe, wir stehen hier vor dir in
unserer Schuld! Bei solchem Verhalten\textless sup title=``oder: bei
solcher Sachlage''\textgreater✲ kann unmöglich jemand vor dir bestehen!«

\hypertarget{c-das-vorgehen-gegen-die-mischehen}{%
\paragraph{c) Das Vorgehen gegen die
Mischehen}\label{c-das-vorgehen-gegen-die-mischehen}}

\hypertarget{section-9}{%
\section{10}\label{section-9}}

1Während nun Esra so betete und weinend und vor dem Hause Gottes
hingestreckt sein Bußbekenntnis ablegte, sammelte sich um ihn eine sehr
große Schar von Israeliten, Männer, Frauen und Kinder; denn das Volk war
in heftiges Weinen ausgebrochen. 2Da nahm Sechanja, der Sohn Jehiels,
aus der Familie Elam, das Wort und sagte zu Esra: »Wir haben uns an
unserm Gott versündigt, daß wir fremde Frauen aus den
Völkerschaften\textless sup title=``vgl. V.11''\textgreater✲ des Landes
geheiratet haben; aber trotzdem ist auch jetzt noch Hoffnung für Israel
vorhanden. 3Wir wollen uns jetzt unserm Gott gegenüber feierlich
verpflichten, alle fremden Frauen und die von diesen geborenen Kinder
aus dem Hause zu entfernen nach deinem Rat, o Herr, und nach dem Rat
derer, die das Gebot unsers Gottes fürchten, damit nach dem Gesetz
verfahren wird! 4Stehe auf, denn dir obliegt die Sache! Wir wollen zu
dir halten: sei entschlossen und handle!«

5Da stand Esra auf und ließ die Obersten der Priester und der Leviten
und des gesamten Israels schwören, daß sie wirklich nach diesem
Vorschlage verfahren wollten; und sie legten den Eid ab. 6Hierauf stand
Esra von dem Platz vor dem Hause Gottes auf und begab sich in die Zelle
Johanans, des Sohnes Eljasibs; er brachte dort die Nacht zu, ohne Speise
und Trank zu sich zu nehmen; denn er war tief betrübt über den Frevel
der aus der Gefangenschaft\textless sup title=``oder:
Verbannung''\textgreater✲ Zurückgekehrten.

7Hierauf ließ man in ganz Juda und Jerusalem durch Ausruf bekannt
machen, alle aus der Gefangenschaft\textless sup title=``oder:
Verbannung''\textgreater✲ Zurückgekehrten sollten sich in Jerusalem
versammeln; 8und jeder, der nicht binnen drei Tagen gemäß dem Beschluß
der Obersten und Ältesten erschiene, dessen gesamtes Vermögen solle dem
Bann verfallen\textless sup title=``vgl. 3.Mose 27,21.28''\textgreater✲
und er selbst aus der Gemeinde der aus der Gefangenschaft\textless sup
title=``oder: Verbannung''\textgreater✲ Heimgekehrten ausgeschlossen
werden.

\hypertarget{die-beschluxfcsse-der-volksversammlung-und-ihre-ausfuxfchrung}{%
\paragraph{Die Beschlüsse der Volksversammlung und ihre
Ausführung}\label{die-beschluxfcsse-der-volksversammlung-und-ihre-ausfuxfchrung}}

9Da versammelten sich denn alle Männer von Juda und Benjamin in
Jerusalem nach Ablauf dreier Tage, nämlich am zwanzigsten Tage des
neunten Monats\textless sup title=``d.h. des Dezembers''\textgreater✲;
und das ganze Volk saß auf dem freien Platze vor dem Hause Gottes,
zitternd wegen der vorliegenden Angelegenheit und infolge des kalten
Regenwetters. 10Da stand der Priester Esra auf und sprach zu ihnen: »Ihr
habt euch schwer versündigt, indem ihr fremde Frauen ins Haus genommen
und dadurch die Schuld Israels noch vergrößert habt! 11So legt nun ein
Schuldbekenntnis vor dem HERRN, dem Gott eurer Väter, ab und tut, was
ihm wohlgefällig ist! Sondert euch von den Völkerschaften des Landes und
von den fremden Frauen ab!« 12Da antwortete die ganze Versammlung und
rief laut: »Ja, es ist unsere Pflicht, so zu tun, wie du gesagt hast!
13Aber das Volk ist zahlreich, und dazu ist jetzt die Regenzeit, so daß
man sich unmöglich im Freien aufhalten kann; auch läßt sich die Sache
nicht an einem oder zwei Tagen abmachen, denn wir haben uns allzu
vielfacher Übertretungen in dieser Beziehung schuldig gemacht. 14So
mögen denn unsere Obersten für die ganze Gemeinde eintreten, und dann
sollen alle, die in unsern Ortschaften fremde Frauen heimgeführt haben,
zu festzusetzenden Zeiten kommen und mit ihnen die Ältesten der
betreffenden Ortschaften und deren Richter, bis die Zornglut unsers
Gottes um dieser Sache willen von uns abgewandt ist.« 15Nur Jonathan,
der Sohn Asahels, und Jahseja, der Sohn Thikwas, traten gegen diesen
Vorschlag auf, und Masullam und der Levit Sabbethai unterstützten sie;
16aber die übrigen aus der Gefangenschaft\textless sup title=``oder:
Verbannung''\textgreater✲ Zurückgekehrten verfuhren nach dem Vorschlage,
und der Priester Esra erwählte sich Männer, nämlich Familienhäupter nach
den einzelnen Familien -- sie sind alle mit Namen aufgezeichnet. Diese
hielten dann am ersten Tage des zehnten Monats (ihre erste) Sitzung ab,
um die Sache zu untersuchen, 17und sie erledigten die ganze
Angelegenheit bezüglich der Männer, welche fremde Frauen geheiratet
hatten, bis zum ersten Tage des ersten Monats.

\hypertarget{verzeichnis-der-priester-leviten-und-laien-welche-fremde-weiber-geheiratet-hatten}{%
\paragraph{Verzeichnis der Priester, Leviten und Laien, welche fremde
Weiber geheiratet
hatten}\label{verzeichnis-der-priester-leviten-und-laien-welche-fremde-weiber-geheiratet-hatten}}

18Es hatten sich aber unter den zur Priesterschaft Gehörigen folgende
gefunden, die fremde✲ Frauen geheiratet hatten: aus der Familie Jesuas,
des Sohnes Jozadaks, und von seinen Genossen: Maaseja, Elieser, Jarib
und Gedalja; 19sie versprachen mit Handschlag, ihre Frauen entlassen zu
wollen, und sie brachten als Schuldopfer einen Widder für ihre
Verschuldung dar. 20Ferner aus der Familie Immer: Hanani und Sebadja;
21aus der Familie Harim: Maaseja, Elia, Semaja, Jehiel und Ussia; 22aus
der Familie Pashur: Eljoenai, Maaseja, Ismael, Nethaneel, Josabad und
Elasa.~-- 23Sodann von den Leviten: Josabad, Simei, Kelaja (das ist
Kelita), Pethahja, Juda und Elieser.~-- 24Sodann von den Sängern:
Eljasib; und von den Torhütern: Sallum, Telem und Uri.~-- 25Von den
übrigen Israeliten aber: aus der Familie Parhos: Ramja, Jissia, Malkia,
Mijamin, Eleasar, Malkia und Benaja; 26aus der Familie Elam: Matthanja,
Sacharja, Jehiel, Abdi, Jeremoth und Elia; 27aus der Familie Satthu:
Eljoenai, Eljasib, Matthanja, Jeremoth, Sabad und Asisa; 28aus der
Familie Bebai: Johanan, Hananja, Sabbai und Athlai; 29aus der Familie
Bani: Mesullam, Malluch und Adaja, Jasub und Seal, Jeremoth; 30aus der
Familie Pahath-Moab: Adna, Kelal, Benaja, Maaseja, Matthanja, Bezaleel
und Binnui und Manasse; 31aus der Familie Harim: Elieser, Jissia,
Malkia, Semaja, Simeon, 32Benjamin, Malluch, Semarja; 33aus der Familie
Hasum: Matthenai, Matthattha, Sabad, Eliphelet, Jeremai, Manasse, Simei;
34aus der Familie Bani✲: Maadai, Amram und Uel, 35Benaja, Bedja, Keluhi,
36Wanja, Meremoth, Eljasib, 37Matthanja, Matthenai, Jaasai, 38Bani,
Binnui, Simei, 39Selemja, Nathan, Adaja, 40Machnadbai, Sasai, Sarai,
41Asrael und Selemja, Semarja, 42Sallum, Amarja, Joseph; 43aus der
Familie Nebo: Jehiel, Matthithja, Sabad, Sebina, Jaddai und Jole,
Benaja. 44Alle diese hatten fremde Frauen geheiratet, und es befanden
sich unter ihnen Frauen, welche Kinder geboren hatten.
