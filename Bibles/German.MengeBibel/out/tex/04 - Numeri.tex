\hypertarget{das-vierte-buch-mose}{%
\section{DAS VIERTE BUCH MOSE}\label{das-vierte-buch-mose}}

\emph{(genannt Numeri, d.h. Zahlangaben, Zählungen des Volkes)}

\hypertarget{die-letzten-verordnungen-und-ereignisse-am-sinai-11-1010}{%
\subsubsection{1. Die letzten Verordnungen und Ereignisse am Sinai
(1,1-10,10)}\label{die-letzten-verordnungen-und-ereignisse-am-sinai-11-1010}}

\hypertarget{a-erste-zuxe4hlung-der-kriegstuxfcchtigen-muxe4nner-mit-ausnahme-der-leviten}{%
\paragraph{a) Erste Zählung der kriegstüchtigen Männer (mit Ausnahme der
Leviten)}\label{a-erste-zuxe4hlung-der-kriegstuxfcchtigen-muxe4nner-mit-ausnahme-der-leviten}}

\hypertarget{aa-der-guxf6ttliche-befehl-und-seine-ausfuxfchrung}{%
\subparagraph{aa) Der göttliche Befehl und seine
Ausführung}\label{aa-der-guxf6ttliche-befehl-und-seine-ausfuxfchrung}}

\hypertarget{section}{%
\section{1}\label{section}}

\bibleverse{1}Hierauf gebot der HERR dem Mose in der Wüste Sinai, im
Offenbarungszelt, am ersten Tage des zweiten Monats im zweiten Jahr nach
ihrem Auszug aus dem Lande Ägypten folgendes: \bibleverse{2}»Nehmt die
Kopfzahl der gesamten Gemeinde der Israeliten auf nach ihren
Geschlechtern und nach ihren Familien, mit Aufzählung der einzelnen
Namen, alle männlichen Personen Kopf für Kopf; \bibleverse{3}von zwanzig
Jahren an und darüber, alles, was in Israel zum Heeresdienst tauglich
ist, die sollt ihr mustern, Schar für Schar, du und Aaron.
\bibleverse{4}Dabei soll euch je ein Mann von jedem Stamm Beistand
leisten, nämlich der Obmann\textless sup title=``oder: das
Oberhaupt''\textgreater✲ der Familien seines Stammes. \bibleverse{5}Und
dies sind die Namen der Männer, die euch dabei zur Seite stehen sollen:
für Ruben: Elizur, der Sohn Sedeurs; \bibleverse{6}für Simeon: Selumiel,
der Sohn Zurisaddais; \bibleverse{7}für Juda: Nahson, der Sohn
Amminadabs; \bibleverse{8}für Issaschar: Nethaneel, der Sohn Zuars;
\bibleverse{9}für Sebulon: Eliab, der Sohn Helons; \bibleverse{10}für
die Söhne Josephs, für Ephraim: Elisama, der Sohn Ammihuds; für Manasse:
Gamliel, der Sohn Pedazurs; \bibleverse{11}für Benjamin: Abidan, der
Sohn Gideonis; \bibleverse{12}für Dan: Ahieser, der Sohn Ammisaddais;
\bibleverse{13}für Asser: Pagiel, der Sohn Ochrans; \bibleverse{14}für
Gad: Eljasaph, der Sohn Deguels; \bibleverse{15}für Naphthali: Ahira,
der Sohn Enans.« \bibleverse{16}Dies waren die aus der Gemeinde
Berufenen, die Fürsten ihrer väterlichen Stämme, die Häupter der
Tausendschaften Israels. \bibleverse{17}Hierauf ließen Mose und Aaron
diese Männer, die ihnen mit Namen bezeichnet worden waren, zu sich
kommen \bibleverse{18}und versammelten dann die ganze Gemeinde am ersten
Tage des zweiten Monats. Da ließen sie sich in die Geburtsverzeichnisse
eintragen nach ihren Geschlechtern und nach ihren Familien, mit
Aufzählung der einzelnen Namen, von zwanzig Jahren an und darüber, Kopf
für Kopf; \bibleverse{19}wie der HERR dem Mose geboten hatte, so
musterte er sie in der Wüste Sinai.

\hypertarget{bb-die-ergebnisse-der-zuxe4hlung}{%
\subparagraph{bb) Die Ergebnisse der
Zählung}\label{bb-die-ergebnisse-der-zuxe4hlung}}

\bibleverse{20}Es beliefen sich aber die Nachkommen Rubens, des
erstgeborenen Sohnes Israels, ihre Abkömmlinge nach ihren Geschlechtern
und nach ihren Familien, nach der Zahl der einzelnen Namen, Kopf für
Kopf, alle männlichen Personen von zwanzig Jahren an und darüber, alles,
was zum Heeresdienst tauglich war, \bibleverse{21}soviele ihrer vom
Stamme Ruben gemustert wurden, auf 46500.

\bibleverse{22}Bei den Nachkommen Simeons beliefen sich ihre Abkömmlinge
nach ihren Geschlechtern und nach ihren Familien, {[}seine
Gemusterten{]} nach der Zahl der einzelnen Namen, Kopf für Kopf, alle
männlichen Personen von zwanzig Jahren an und darüber, alles, was zum
Heeresdienst tauglich war, \bibleverse{23}soviele ihrer vom Stamme
Simeon gemustert wurden, auf 59300.

\bibleverse{24}Bei den Nachkommen Gads beliefen sich ihre Abkömmlinge
nach ihren Geschlechtern und nach ihren Familien, nach der Zahl der
einzelnen Namen, von zwanzig Jahren an und darüber, alles, was zum
Heeresdienst tauglich war, \bibleverse{25}soviele ihrer vom Stamme Gad
gemustert wurden, auf 45650.

\bibleverse{26}Bei den Nachkommen Judas beliefen sich ihre Abkömmlinge
nach ihren Geschlechtern und nach ihren Familien, nach der Zahl der
einzelnen Namen, von zwanzig Jahren an und darüber, alles, was zum
Heeresdienst tauglich war, \bibleverse{27}soviele ihrer vom Stamme Juda
gemustert wurden, auf 74600.

\bibleverse{28}Bei den Nachkommen Issaschars beliefen sich ihre
Abkömmlinge nach ihren Geschlechtern und nach ihren Familien, nach der
Zahl der einzelnen Namen, von zwanzig Jahren an und darüber, alles, was
zum Heeresdienst tauglich war, \bibleverse{29}soviele ihrer vom Stamme
Issaschar gemustert wurden, auf 54400.

\bibleverse{30}Bei den Nachkommen Sebulons beliefen sich ihre
Abkömmlinge nach ihren Geschlechtern und nach ihren Familien, nach der
Zahl der einzelnen Namen, von zwanzig Jahren an und darüber, alles, was
zum Heeresdienst tauglich war, \bibleverse{31}soviele ihrer vom Stamme
Sebulon gemustert wurden, auf 57400.

\bibleverse{32}Was die Nachkommen Josephs betrifft, so beliefen sich bei
den Nachkommen Ephraims ihre Abkömmlinge nach ihren Geschlechtern und
nach ihren Familien, nach der Zahl der einzelnen Namen, von zwanzig
Jahren an und darüber, alles, was zum Heeresdienst tauglich war,
\bibleverse{33}soviele ihrer vom Stamme Ephraim gemustert wurden, auf
40500.

\bibleverse{34}Bei den Nachkommen Manasses beliefen sich ihre
Abkömmlinge nach ihren Geschlechtern und nach ihren Familien, nach der
Zahl der einzelnen Namen, von zwanzig Jahren an und darüber, alles, was
zum Heeresdienst tauglich war, \bibleverse{35}soviele ihrer vom Stamme
Manasse gemustert wurden, auf 32200.

\bibleverse{36}Bei den Nachkommen Benjamins beliefen sich ihre
Abkömmlinge nach ihren Geschlechtern und nach ihren Familien, nach der
Zahl der einzelnen Namen, von zwanzig Jahren an und darüber, alles, was
zum Heeresdienst tauglich war, \bibleverse{37}soviele ihrer vom Stamme
Bejamin gemustert wurden, auf 35400.

\bibleverse{38}Bei den Nachkommen Dans beliefen sich ihre Abkömmlinge
nach ihren Geschlechtern und nach ihren Familien, nach der Zahl der
einzelnen Namen, von zwanzig Jahren an und darüber, alles, was zum
Heeresdienst tauglich war, \bibleverse{39}soviele ihrer vom Stamme Dan
gemustert wurden, auf 62700.

\bibleverse{40}Bei den Nachkommen Assers beliefen sich ihre Abkömmlinge
nach ihren Geschlechtern und nach ihren Familien, nach der Zahl der
einzelnen Namen, von zwanzig Jahren an und darüber, alles, was zum
Heeresdienst tauglich war, \bibleverse{41}soviele ihrer vom Stamme Asser
gemustert wurden, auf 41500.

\bibleverse{42}Bei den Nachkommen Naphthalis beliefen sich ihre
Abkömmlinge nach ihren Geschlechtern und nach ihren Familien, nach der
Zahl der einzelnen Namen, von zwanzig Jahren an und darüber, alles, was
zum Heeresdienst tauglich war, \bibleverse{43}soviele ihrer vom Stamme
Naphthali gemustert wurden, auf 53400.

\bibleverse{44}Dies sind die Gemusterten, die Mose und Aaron mit den
Fürsten der Israeliten musterten; die letzteren waren zwölf Männer, und
zwar je ein Mann für die zu seinem Stamm gehörigen Familien.
\bibleverse{45}Und es beliefen sich alle, die von den Israeliten nach
ihren Familien, von zwanzig Jahren an und darüber, gemustert worden
waren, alles, was in Israel zum Heeresdienst tauglich war~--
\bibleverse{46}es beliefen sich also alle Gemusterten auf 603550.

\hypertarget{cc-die-ausnahmestellung-der-leviten}{%
\subparagraph{cc) Die Ausnahmestellung der
Leviten}\label{cc-die-ausnahmestellung-der-leviten}}

\bibleverse{47}Die Leviten aber nach ihrem väterlichen Stamm waren in
dieser Musterung nicht inbegriffen. \bibleverse{48}Der HERR hatte
nämlich dem Mose folgendes geboten: \bibleverse{49}»Nur die zum Stamm
Levi Gehörigen sollst du nicht mustern und ihre Kopfzahl nicht mit den
übrigen Israeliten aufnehmen; \bibleverse{50}sondern bestelle du die
Leviten zum Dienst an der Wohnung des Gesetzes und über ihre sämtlichen
Geräte und über alles, was zu ihr gehört: sie sollen die Wohnung und
ihre sämtlichen Geräte tragen, und sie sollen den Dienst an ihr versehen
und rings um die Wohnung her lagern. \bibleverse{51}Und wenn die Wohnung
weiter wandert, sollen die Leviten sie abbrechen; und wenn die Wohnung
halt macht, sollen die Leviten sie aufschlagen; jeder
Nichtzugehörige\textless sup title=``=~Unbefugte oder:
Nichtlevit''\textgreater✲ aber, der an sie herantritt, soll sterben.
\bibleverse{52}Und die übrigen Israeliten sollen nach ihren Heerscharen
ein jeder auf seinem Lagerplatz und ein jeder bei seiner
Fahne\textless sup title=``oder: seinem Panier''\textgreater✲ lagern;
\bibleverse{53}die Leviten aber sollen rings um die Wohnung des Gesetzes
her lagern, damit kein Zorn Gottes über die Gemeinde der Israeliten
hereinbricht. Die Leviten sollen also den Dienst an der Wohnung des
Gesetzes versehen.«

\bibleverse{54}So machten denn die Israeliten alles genau so, wie der
HERR dem Mose geboten hatte.

\hypertarget{b-die-lager--und-zugordnung-der-stuxe4mme}{%
\paragraph{b) Die Lager- und Zugordnung der
Stämme}\label{b-die-lager--und-zugordnung-der-stuxe4mme}}

\hypertarget{section-1}{%
\section{2}\label{section-1}}

\bibleverse{1}Weiter gebot der HERR dem Mose und Aaron folgendes:
\bibleverse{2}»Die Israeliten sollen ein jeder bei seinem Panier, bei
den Feldzeichen ihrer Stämme, lagern; dem Offenbarungszelt gegenüber
sollen sie ringsum lagern. \bibleverse{3}Und zwar sollen folgende
ostwärts✲ gegen Sonnenaufgang lagern: das Panier des Lagers Judas nach
seinen Heerscharen, und als Anführer des Stammes Juda Nahson, der Sohn
Amminadabs; \bibleverse{4}sein Heer beläuft sich auf 74600~gemusterte
Männer. \bibleverse{5}Neben ihm soll der Stamm Issaschar lagern, und als
Anführer des Stammes Issaschar Nethaneel, der Sohn Zuars;
\bibleverse{6}sein Heer beläuft sich auf 54400~gemusterte Männer.
\bibleverse{7}Ferner der Stamm Sebulon, und als Anführer des Stammes
Sebulon Eliab, der Sohn Helons; \bibleverse{8}sein Heer beläuft sich auf
57400~gemusterte Männer. \bibleverse{9}Alle Gemusterten im Lager Judas
machen nach ihren Heerscharen 186400~Mann aus: sie sollen (beim
Abmarsch) zuerst aufbrechen.

\bibleverse{10}Das Panier des Lagers Rubens soll südwärts nach seinen
Heerscharen lagern, und als Anführer des Stammes Ruben Elizur, der Sohn
Sedeurs; \bibleverse{11}sein Heer beläuft sich auf 46500~gemusterte
Männer. \bibleverse{12}Neben ihm soll der Stamm Simeon lagern, und als
Anführer des Stammes Simeon Selumiel, der Sohn Zurisaddais;
\bibleverse{13}sein Heer beläuft sich auf 59300~gemusterte Männer.
\bibleverse{14}Ferner der Stamm Gad, und als Anführer des Stammes Gad
Eljasaph, der Sohn Reguels; \bibleverse{15}sein Heer beläuft sich auf
45650~gemusterte Männer. \bibleverse{16}Alle Gemusterten im Lager Rubens
machen nach ihren Heerscharen 151450~Mann aus: sie sollen an zweiter
Stelle aufbrechen.

\bibleverse{17}Dann soll das Offenbarungszelt, das Lager der Leviten, in
der Mitte der übrigen Lager aufbrechen; wie sie gelagert sind, so sollen
sie aufbrechen, ein jeder an seiner Stelle, nach ihren Panieren.

\bibleverse{18}Das Panier des Lagers Ephraims soll nach seinen
Heerscharen westwärts lagern, und als Anführer des Stammes Ephraim
Elisama, der Sohn Ammihuds; \bibleverse{19}sein Heer beläuft sich auf
40500~gemusterte Männer. \bibleverse{20}Neben ihm soll sich der Stamm
Manasse lagern, und als Anführer des Stammes Manasse Gamliel, der Sohn
Pedazurs; \bibleverse{21}sein Heer beläuft sich auf 32200~gemusterte
Männer. \bibleverse{22}Ferner der Stamm Benjamin, und als Anführer des
Stammes Benjamin Abidan, der Sohn Gideonis; \bibleverse{23}sein Heer
beläuft sich auf 35400~gemusterte Männer. \bibleverse{24}Alle
Gemusterten im Lager Ephraims machen nach ihren Heerscharen 108100~Mann
aus: sie sollen an dritter Stelle aufbrechen.

\bibleverse{25}Das Panier des Lagers Dans soll nordwärts nach seinen
Heerscharen lagern, und als Anführer des Stammes Dan Ahieser, der Sohn
Ammisaddais; \bibleverse{26}sein Heer beläuft sich auf 62700~gemusterte
Männer. \bibleverse{27}Neben ihm soll der Stamm Asser lagern, und als
Anführer des Stammes Asser Pagiel, der Sohn Ochrans; \bibleverse{28}sein
Heer beläuft sich auf 41500~gemusterte Männer. \bibleverse{29}Ferner der
Stamm Naphthali, und als Anführer des Stammes Naphthali Ahira, der Sohn
Enans; \bibleverse{30}sein Heer beläuft sich auf 53400~gemusterte
Männer. \bibleverse{31}Alle Gemusterten im Lager Dans machen 157600~Mann
aus: sie sollen zuletzt nach ihren Panieren aufbrechen.«

\bibleverse{32}Dies sind die Gemusterten der Israeliten nach ihren
Stämmen; sämtliche Gemusterte der einzelnen Lager nach ihren Heerscharen
beliefen sich auf 603550~Mann. \bibleverse{33}Die Leviten aber waren in
dieser Musterung der Israeliten nicht inbegriffen, wie der HERR dem Mose
geboten hatte. \bibleverse{34}So taten denn die Israeliten genau so, wie
der HERR dem Mose geboten hatte: also lagerten sie nach ihren Panieren,
und also brachen sie auf ein jeder nach seinem Geschlecht bei seiner
Familie.

\hypertarget{c-musterung-der-leviten-luxf6sung-der-israelitischen-erstgeborenen}{%
\paragraph{c) Musterung der Leviten; Lösung der israelitischen
Erstgeborenen}\label{c-musterung-der-leviten-luxf6sung-der-israelitischen-erstgeborenen}}

\hypertarget{aa-die-suxf6hne-aarons}{%
\subparagraph{aa) Die Söhne Aarons}\label{aa-die-suxf6hne-aarons}}

\hypertarget{section-2}{%
\section{3}\label{section-2}}

\bibleverse{1}Dies sind die Nachkommen Aarons {[}und Moses{]} zu der
Zeit, als der HERR mit Mose auf dem Berge Sinai redete.
\bibleverse{2}Und zwar sind dies die Namen der Söhne Aarons: der
Erstgeborene war Nadab, sodann Abihu, Eleasar und Ithamar.
\bibleverse{3}Dies sind die Namen der Söhne Aarons, der gesalbten
Priester, die in ihr Amt eingesetzt worden waren\textless sup
title=``vgl. 3.Mose 8,28''\textgreater✲, um den Priesterdienst zu
verrichten. \bibleverse{4}Aber Nadab und Abihu starben (im Dienst) vor
dem HERRN, als sie in der Wüste Sinai ein ungehöriges Feueropfer vor dem
HERRN darbrachten, und hatten keine Söhne. So versahen denn Eleasar und
Ithamar den Priesterdienst unter der Aufsicht ihres Vaters Aaron.

\hypertarget{bb-die-leviten-zu-gehilfen-der-priester-und-zur-dienstleistung-am-heiligtum-bestimmt}{%
\subparagraph{bb) Die Leviten zu Gehilfen der Priester und zur
Dienstleistung am Heiligtum
bestimmt}\label{bb-die-leviten-zu-gehilfen-der-priester-und-zur-dienstleistung-am-heiligtum-bestimmt}}

\bibleverse{5}Der HERR gebot aber dem Mose folgendes: \bibleverse{6}»Laß
den Stamm Levi herantreten und stelle ihn vor den Priester Aaron, damit
sie ihm Dienste leisten; \bibleverse{7}denn sie sollen alles das
besorgen, was ihm und was der ganzen Gemeinde vor dem Offenbarungszelt
zu beobachten obliegt, so daß sie den Dienst an der heiligen Wohnung
verrichten. \bibleverse{8}Sie sollen also die sämtlichen Geräte des
Offenbarungszeltes und alles besorgen, was den Israeliten zu besorgen
obliegt, und so den Dienst an der heiligen Wohnung versehen.
\bibleverse{9}Du sollst also die Leviten dem Aaron und seinen Söhnen
beigeben\textless sup title=``oder: überweisen''\textgreater✲; ganz zu
eigen sollen sie ihm von seiten der Israeliten überwiesen sein.
\bibleverse{10}Aaron aber und seine Söhne sollst du dazu bestellen, daß
sie ihres Priesteramts warten; ein Unbefugter aber, der das Amt ausübt,
soll sterben.«

\hypertarget{cc-die-leviten-zur-ausluxf6sung-der-israelitischen-erstgeborenen-bestimmt}{%
\subparagraph{cc) Die Leviten zur Auslösung der israelitischen
Erstgeborenen
bestimmt}\label{cc-die-leviten-zur-ausluxf6sung-der-israelitischen-erstgeborenen-bestimmt}}

\bibleverse{11}Weiter sagte der HERR zu Mose folgendes:
\bibleverse{12}»Wisse wohl: ich selbst habe die Leviten aus der Mitte
der Israeliten herausgenommen\textless sup title=``oder:
ausgesondert''\textgreater✲ als Ersatz für alle Erstgeborenen, als
Ersatz für alle Kinder, die unter den Israeliten zuerst zur Welt
gekommen sind, damit die Leviten mir gehören. \bibleverse{13}Denn mir
gehören alle Erstgeborenen; an dem Tage, als ich alle Erstgeburten im
Lande Ägypten sterben ließ, habe ich mir alles Erstgeborene in Israel
geweiht sowohl von Menschen als auch von Tieren: mir sollen sie gehören,
mir, dem HERRN.«

\hypertarget{dd-zuxe4hlung-lagerplatz-anfuxfchrer-und-dienstordnung-der-muxe4nnlichen-leviten}{%
\subparagraph{dd) Zählung, Lagerplatz, Anführer und Dienstordnung der
männlichen
Leviten}\label{dd-zuxe4hlung-lagerplatz-anfuxfchrer-und-dienstordnung-der-muxe4nnlichen-leviten}}

\bibleverse{14}Hierauf gebot der HERR dem Mose in der Wüste Sinai
folgendes: \bibleverse{15}»Mustere die Leviten nach ihren Familien und
nach ihren Geschlechtern; alle männlichen Personen von einem Monat an
und darüber sollst du bei ihnen mustern.« \bibleverse{16}Da nahm Mose
nach dem Befehl des HERRN die Musterung bei ihnen vor, wie ihm geboten
worden war. \bibleverse{17}Und dies waren die Söhne Levis mit ihren
Namen: Gerson, Kehath und Merari. \bibleverse{18}Und dies sind die Namen
der Söhne Gersons nach ihren Geschlechtern: Libni und Simei;
\bibleverse{19}und die Söhne Kehaths nach ihren Geschlechtern: Amram und
Jizhar, Hebron und Ussiel; \bibleverse{20}und die Söhne Meraris nach
ihren Geschlechtern: Mahli und Musi. Das sind die Geschlechter der
Leviten nach ihren Familien.

\bibleverse{21}Von Gerson stammt also das Geschlecht der Libniten und
das Geschlecht der Simeiten; dies sind die Geschlechter der Gersoniten.
\bibleverse{22}Ihre Gemusterten, nach der Zahl aller männlichen Personen
bei ihnen von einem Monat an und darüber, beliefen sich auf 7500.
\bibleverse{23}Die Geschlechter der Gersoniten lagerten hinter der
heiligen Wohnung gegen Westen; \bibleverse{24}und der Fürst\textless sup
title=``oder: das Oberhaupt''\textgreater✲ des Hauses der Gersoniten war
Eljasaph, der Sohn Laels. \bibleverse{25}Die Dienstleistung der
Gersoniten im Offenbarungszelt betraf die heilige Wohnung und das
Zeltdach, seine Überdecke und den Vorhang am Eingang des
Offenbarungszeltes, \bibleverse{26}ferner die Umhänge des Vorhofs und
den Vorhang am Eingang des Vorhofs, der die heilige Wohnung und den
Altar rings umgab, sowie die zugehörigen Seile nebst allem, was es dabei
zu tun gab.

\bibleverse{27}Von Kehath stammt das Geschlecht der Amramiten sowie das
Geschlecht der Jizhariten, das Geschlecht der Hebroniten und das
Geschlecht der Ussieliten: dies sind die Geschlechter der Kehathiten.
\bibleverse{28}Ihre Gemusterten, nach der Zahl aller männlichen Personen
von einem Monat an und darüber, beliefen sich auf 8600, denen der Dienst
im Heiligtum oblag. \bibleverse{29}Die Geschlechter der Kehathiten
lagerten an der südlichen Längsseite der heiligen Wohnung;
\bibleverse{30}und das Familienoberhaupt der Geschlechter der Kehathiten
war Elizaphan, der Sohn Ussiels. \bibleverse{31}Ihre Dienstleistung
betraf die Lade und den Tisch, den Leuchter, die Altäre, die heiligen
Geräte, die beim Dienst gebraucht wurden, ferner den Vorhang (vor dem
Allerheiligsten) nebst allem, was es dabei zu tun gab.
\bibleverse{32}Das Oberhaupt der Levitenoberhäupter aber war Eleasar,
der Sohn des Priesters Aaron; er hatte die Aufsicht über die, welche
Dienste am Heiligtum zu leisten hatten.

\bibleverse{33}Von Merari stammte das Geschlecht der Mahliten und das
Geschlecht der Musiten; dies sind die Geschlechter Meraris.
\bibleverse{34}Ihre Gemusterten, nach der Zahl aller männlichen Personen
von einem Monat an und darüber, beliefen sich auf 6200.
\bibleverse{35}Das Familienoberhaupt der Geschlechter Meraris war
Zuriel, der Sohn Abihails. Sie lagerten an der nördlichen Längsseite der
heiligen Wohnung; \bibleverse{36}und die Dienstleistung der Merariten
betraf die Bretter der heiligen Wohnung sowie ihre Riegel, ihre Säulen
und Füße\textless sup title=``oder: Ständer und Sockel''\textgreater✲
nebst allen zugehörigen Geräten und allem, was es dabei zu tun gab,
\bibleverse{37}ferner die Säulen\textless sup title=``oder:
Ständer''\textgreater✲ des Vorhofs ringsum nebst ihren Füßen, ihren
Pflöcken und Seilen.

\bibleverse{38}Diejenigen aber, welche vor der heiligen Wohnung nach
Osten hin, vor dem Offenbarungszelt gegen Sonnenaufgang, lagerten, waren
Mose und Aaron mit seinen Söhnen, die den Dienst am Heiligtum zu
verrichten hatten, nämlich alles das, was den Israeliten dabei zu
besorgen oblag. -- Der Unbefugte aber, der den Dienst ausübte, sollte
sterben.

\bibleverse{39}Die Kopfzahl sämtlicher Leviten, die Mose und Aaron auf
Befehl des HERRN nach ihren Geschlechtern gemustert hatten, alle
männlichen Personen von einem Monat an und darüber, belief sich auf
22000.

\hypertarget{ee-musterung-und-luxf6sung-der-muxe4nnlichen-erstgeborenen-in-israel}{%
\subparagraph{ee) Musterung und Lösung der männlichen Erstgeborenen in
Israel}\label{ee-musterung-und-luxf6sung-der-muxe4nnlichen-erstgeborenen-in-israel}}

\bibleverse{40}Weiter gebot der HERR dem Mose: »Mustere nun auch alle
männlichen Erstgeborenen bei den Israeliten von einem Monat an und
darüber und nimm die Zahl ihrer Namen auf. \bibleverse{41}Du sollst aber
die Leviten für mich, den HERRN, als Ersatz für alle Erstgeborenen unter
den Israeliten nehmen und ebenso das Vieh der Leviten als Ersatz für
alle Erstgeburten unter dem Vieh der Israeliten.« \bibleverse{42}So
musterte denn Mose alle Erstgeborenen unter den Israeliten, wie der HERR
ihm geboten hatte; \bibleverse{43}und es belief sich die Gesamtzahl der
männlichen Erstgeborenen, nach der Zahl der Namen von einem Monat an und
darüber, soviele ihrer gemustert waren, auf 22273.

\bibleverse{44}Hierauf gebot der HERR dem Mose folgendes:
\bibleverse{45}»Nimm die Leviten als Ersatz für alle Erstgeborenen unter
den Israeliten und das Vieh der Leviten als Ersatz für das Vieh jener,
damit die Leviten mir gehören, mir, dem HERRN. \bibleverse{46}Was aber
den Loskauf der 273~israelitischen Erstgeborenen betrifft, welche über
die Zahl der Leviten überschüssig sind, \bibleverse{47}so sollst du fünf
Schekel für jeden Kopf erheben; nach dem Gewicht des Heiligtums sollst
du sie erheben, den Schekel zu zwanzig Gera gerechnet.
\bibleverse{48}Dies Geld sollst du als Lösegeld für die, welche
überzählig bei ihnen sind, Aaron und seinen Söhnen übergeben.«
\bibleverse{49}So erhob denn Mose das Loskaufgeld von denen, welche über
die durch die Leviten Ausgelösten überzählig waren; \bibleverse{50}von
den Erstgeborenen der Israeliten erhob er das Geld, 1365~Schekel, nach
dem Gewicht des Heiligtums. \bibleverse{51}Dann übergab Mose das
Lösegeld dem Aaron und seinen Söhnen nach dem Befehl des HERRN, wie der
HERR dem Mose geboten hatte.

\hypertarget{d-dienstordnung-fuxfcr-die-leviten-alter-und-zahl-der-dienenden-leviten}{%
\paragraph{d) Dienstordnung für die Leviten; Alter und Zahl der
dienenden
Leviten}\label{d-dienstordnung-fuxfcr-die-leviten-alter-und-zahl-der-dienenden-leviten}}

\hypertarget{aa-der-guxf6ttliche-befehl-zur-musterung-der-dienstfuxe4higen-leviten-vom-30.-bis-50.-jahre-nebst-dienstvorschriften}{%
\subparagraph{aa) Der göttliche Befehl zur Musterung der dienstfähigen
Leviten vom 30. bis 50.~Jahre nebst
Dienstvorschriften}\label{aa-der-guxf6ttliche-befehl-zur-musterung-der-dienstfuxe4higen-leviten-vom-30.-bis-50.-jahre-nebst-dienstvorschriften}}

\hypertarget{section-3}{%
\section{4}\label{section-3}}

\bibleverse{1}Hierauf gebot der HERR dem Mose und Aaron folgendes:
\bibleverse{2}»Nehmt unter den Leviten die Kopfzahl der Kehathiten nach
ihren Geschlechtern und nach ihren Familien auf, \bibleverse{3}von
dreißig Jahren an und darüber bis zu fünfzig Jahren, alle, die zum
Dienst tauglich sind, so daß sie am Offenbarungszelt tätig sein können.

\bibleverse{4}Dies soll der Dienst der Kehathiten am Offenbarungszelt
sein: (die Sorge für) das Hochheilige. \bibleverse{5}Wenn nämlich das
Lager aufbricht, sollen Aaron und seine Söhne hineingehen und den
abschließenden Vorhang abnehmen und die Gesetzeslade mit ihm
bedecken\textless sup title=``oder: in ihn hüllen''\textgreater✲.
\bibleverse{6}Dann sollen sie eine Decke von Seekuhfell darauflegen und
ein ganz aus blauem Purpur bestehendes Tuch oben darüberbreiten und dann
die zugehörigen Tragstangen anlegen. \bibleverse{7}Auch über den
Schaubrottisch sollen sie ein Tuch von blauem Purpur breiten und die
Schüsseln, Schalen und Becher sowie die Kannen für das Trankopfer
daraufstellen; auch das regelmäßig aufzulegende Brot soll sich auf ihm
befinden. \bibleverse{8}Dann sollen sie über das alles ein Tuch von
Karmesin breiten, dieses mit einer Decke von Seekuhfell überdecken und
dann die zugehörigen Tragstangen anlegen. \bibleverse{9}Hierauf sollen
sie ein Tuch von blauem Purpur nehmen und den lichtspendenden Leuchter
damit überdecken\textless sup title=``oder: darin
einhüllen''\textgreater✲ samt den zugehörigen Lampen, Lichtputzscheren,
Näpfen und allen Ölgefäßen, mit denen man ihn zu bedienen pflegt;
\bibleverse{10}dann sollen sie ihn nebst allen seinen Geräten in eine
Decke von Seekuhfell tun und auf die Tragbahre legen.
\bibleverse{11}Auch über den goldenen Altar sollen sie ein Tuch von
blauem Purpur breiten und ihn mit einer Decke von Seekuhfell
überdecken\textless sup title=``oder: darin einhüllen''\textgreater✲ und
die zugehörigen Tragstangen anlegen. \bibleverse{12}Weiter sollen sie
alle für den Dienst erforderlichen Geräte, mit denen man den Dienst im
Heiligtum zu verrichten pflegt, nehmen, sie in ein Tuch von blauem
Purpur tun, sie dann mit einer Decke von Seekuhfell überdecken und auf
die Tragbahre legen. \bibleverse{13}Hierauf sollen sie den (Brandopfer-)
Altar von der Fettasche reinigen und ein Tuch von rotem Purpur über ihn
breiten; \bibleverse{14}auf dieses sollen sie dann alle zugehörigen
Geräte legen, mit denen man den Dienst auf ihm zu verrichten pflegt: die
Kohlenpfannen, Gabeln, Schaufeln, Becken, überhaupt alle zum Altar
gehörigen Geräte; darüber sollen sie eine Decke von Seekuhfell breiten
und die zugehörigen Tragstangen anlegen. \bibleverse{15}Wenn dann Aaron
und seine Söhne beim Aufbruch des Lagers mit der Einhüllung des
Heiligen\textless sup title=``d.h. der heiligen
Gegenstände''\textgreater✲ und aller heiligen Geräte fertig sind, so
sollen danach die Kehathiten kommen, um alles zu tragen; sie dürfen aber
die heiligen Gegenstände nicht berühren, sonst würden sie sterben. Dies
ist es, was die Kehathiten vom Offenbarungszelt zu tragen haben.

\bibleverse{16}Eleasar aber, dem Sohne des Priesters Aaron, ist das Öl
für den Leuchter und das wohlriechende Räucherwerk sowie das regelmäßige
Speisopfer und das Salböl anvertraut, ferner die Aufsicht über die ganze
heilige Wohnung und über alles das, was sich an heiligen Gegenständen
und an zugehörigen Geräten in ihr befindet.«

\bibleverse{17}Hierauf gebot der HERR dem Mose und Aaron folgendes:
\bibleverse{18}»Laßt den Stamm der Geschlechter der Kehathiten nicht aus
der Mitte der Leviten der Vernichtung anheimfallen,
\bibleverse{19}sondern tut folgendes für sie, damit sie am Leben bleiben
und nicht sterben müssen, wenn sie an die hochheiligen Gegenstände
herantreten: Aaron und seine Söhne sollen hineingehen und jeden
einzelnen von ihnen anstellen, um das zu verrichten und das zu tragen,
was ihm obliegt; \bibleverse{20}aber sie selbst sollen nicht
hineingehen, um auch nur einen Augenblick die heiligen Gegenstände
anzusehen; sonst müssen sie sterben.«

\bibleverse{21}Weiter gebot der HERR dem Mose folgendes:
\bibleverse{22}»Nimm nun auch die Kopfzahl der Gersoniten nach ihren
Familien und nach ihren Geschlechtern auf; \bibleverse{23}von dreißig
Jahren an und darüber bis zu fünfzig Jahren sollst du sie mustern, alle,
die zum Dienst tauglich sind, so daß sie Verwendung bei den
Verrichtungen am Offenbarungszelt finden können. \bibleverse{24}Darin
soll der Dienst der Geschlechter der Gersoniten bezüglich der ihnen
obliegenden Verrichtungen und der zu tragenden Gegenstände bestehen:
\bibleverse{25}sie haben die Teppiche der heiligen Wohnung und das
Offenbarungszelt zu tragen, auch seine Decke und die Decke von
Seekuhfell, die oben darüber liegt, sowie den Vorhang am Eingang zum
Offenbarungszelt, \bibleverse{26}ferner die Umhänge des Vorhofes, und
den Vorhang am Eingangstor des Vorhofes, der die heilige Wohnung und den
Brandopferaltar rings umgibt, samt den zugehörigen Seilen und allen
Geräten, die zu diesem Dienst nötig sind; überhaupt alles, was dabei zu
tun ist, sollen sie besorgen. \bibleverse{27}Nach der Anordnung Aarons
und seiner Söhne soll der ganze Dienst der {[}Söhne der{]} Gersoniten
vor sich gehen bei allem, was sie zu tragen, und bei allem, was sie zu
verrichten haben; und zwar sollt ihr ihnen alles, was sie zu tragen
haben, Stück für Stück überweisen. \bibleverse{28}Dies ist der Dienst,
den die Geschlechter der {[}Söhne der{]} Gersoniten am Offenbarungszelt
zu verrichten haben; und die Aufsicht über ihre Verrichtungen soll
Ithamar, dem Sohn des Priesters Aaron, obliegen.

\bibleverse{29}Auch die Merariten sollst du nach ihren Geschlechtern und
nach ihren Familien mustern; \bibleverse{30}von dreißig Jahren an und
darüber bis zu fünfzig Jahren sollst du sie mustern, alle, die zum
Dienst tauglich sind, so daß sie Verwendung bei den Verrichtungen am
Offenbarungszelt finden können. \bibleverse{31}Und dies ist es, was
ihnen zu tragen obliegt und worin ihr ganzer Dienst am Offenbarungszelt
besteht: die Bretter der heiligen Wohnung sowie ihre Riegel, Säulen und
Füße\textless sup title=``oder: Ständer und Sockel''\textgreater✲,
\bibleverse{32}ferner die Säulen\textless sup title=``oder:
Ständer''\textgreater✲ rings um den Vorhof samt ihren Füßen, ihren
Pflöcken und Seilen, samt allen zugehörigen Geräten und nebst allem, was
es dabei zu tun gibt; und zwar sollt ihr ihnen die Geräte, deren
Fortschaffung ihnen obliegt, Stück für Stück zuweisen.
\bibleverse{33}Dies ist der Dienst der Geschlechter der Merariten,
alles, was sie am Offenbarungszelt unter der Aufsicht Ithamars, des
Sohnes des Priesters Aaron, zu verrichten haben.«

\hypertarget{bb-ausfuxfchrung-des-befehls-ergebnisse-der-musterung}{%
\subparagraph{bb) Ausführung des Befehls; Ergebnisse der
Musterung}\label{bb-ausfuxfchrung-des-befehls-ergebnisse-der-musterung}}

\bibleverse{34}So musterten denn Mose und Aaron samt den Stammesfürsten
der Gemeinde die {[}Söhne der{]} Kehathiten nach ihren Geschlechtern und
nach ihren Familien, \bibleverse{35}von dreißig Jahren an und darüber
bis zu fünfzig Jahren, alle, die zum Dienst tauglich waren, so daß sie
Verwendung bei den Verrichtungen am Offenbarungszelt finden konnten.
\bibleverse{36}Und es belief sich die Zahl derer, welche aus ihnen nach
ihren Geschlechtern gemustert wurden, auf 2750. \bibleverse{37}So viele
waren die\textless sup title=``=~dies war die Zahl der''\textgreater✲
aus den Geschlechtern der Kehathiten Gemusterten, alle, welche Dienste
am Offenbarungszelte zu leisten hatten und welche Mose und Aaron nach
dem durch Mose übermittelten Befehl des HERRN gemustert hatten.

\bibleverse{38}Was sodann diejenigen betrifft, welche aus den Gersoniten
nach ihren Geschlechtern und nach ihren Familien gemustert wurden,
\bibleverse{39}von dreißig Jahren an und darüber bis zu fünfzig Jahren,
alle, die zum Dienst tauglich waren, so daß sie Verwendung bei den
Verrichtungen am Offenbarungszelt finden konnten, \bibleverse{40}so
belief sich die Zahl derer, welche aus ihnen nach ihren Geschlechtern
und nach ihren Familien gemustert wurden, auf 2630. \bibleverse{41}So
viele waren die\textless sup title=``=~dies war die Zahl
der''\textgreater✲ aus den Geschlechtern der Gersoniten Gemusterten,
alle, welche Dienste am Offenbarungszelt zu leisten hatten und welche
Mose und Aaron nach dem Befehl des HERRN gemustert hatten.

\bibleverse{42}Was ferner diejenigen betrifft, welche aus den
Geschlechtern der Merariten nach ihren Geschlechtern und nach ihren
Familien gemustert wurden, \bibleverse{43}von dreißig Jahren an und
darüber bis zu fünfzig Jahren, alle, die zum Dienst tauglich waren, so
daß sie Verwendung bei den Verrichtungen am Offenbarungszelt finden
konnten, \bibleverse{44}so beliefen sich die, welche aus ihnen nach
ihren Geschlechtern gemustert wurden, auf 3200. \bibleverse{45}So viele
waren die\textless sup title=``=~dies war die Zahl der''\textgreater✲
aus den Geschlechtern der Merariten Gemusterten, welche Mose und Aaron
nach dem durch Mose übermittelten Befehl des HERRN gemustert hatten.

\bibleverse{46}Was aber die Gesamtzahl der gemusterten Leviten betrifft,
welche Mose und Aaron samt den Stammesfürsten der Israeliten nach ihren
Geschlechtern und nach ihren Familien gemustert hatten,
\bibleverse{47}von dreißig Jahren an und darüber bis zu fünfzig Jahren,
alle, welche tauglich waren, um bei den Dienstleistungen und bei der
Verrichtung des Tragens Verwendung am Offenbarungszelt zu finden,
\bibleverse{48}so belief sich die Zahl der aus ihnen Gemusterten auf
8580. \bibleverse{49}Nach dem durch Mose übermittelten Befehl des HERRN
wies man dem einzelnen von ihnen das zu, was er zu verrichten und zu
tragen hatte; und sie wurden so angestellt, wie es der HERR dem Mose
geboten hatte.

\hypertarget{e-allerlei-gesetze-und-verordnungen}{%
\paragraph{e) Allerlei Gesetze und
Verordnungen}\label{e-allerlei-gesetze-und-verordnungen}}

\hypertarget{aa-entfernung-der-unreinen-aus-dem-lager}{%
\subparagraph{aa) Entfernung der Unreinen aus dem
Lager}\label{aa-entfernung-der-unreinen-aus-dem-lager}}

\hypertarget{section-4}{%
\section{5}\label{section-4}}

\bibleverse{1}Weiter gebot der HERR dem Mose folgendes:
\bibleverse{2}»Befiehl den Israeliten, daß sie alle Aussätzigen und alle
mit einem Ausfluß Behafteten sowie alle, die sich an einer Leiche
verunreinigt haben, aus dem Lager entfernen; \bibleverse{3}sowohl Männer
als Weiber sollt ihr hinausschaffen, vor das Lager hinaus sollt ihr sie
schaffen, damit sie ihre Lager nicht verunreinigen, da doch ich in deren
Mitte wohne.« \bibleverse{4}Da taten die Israeliten so und schafften sie
vor das Lager hinaus; wie der HERR dem Mose geboten hatte, so taten die
Israeliten.

\hypertarget{bb-veruntreuungen-und-deren-suxfchne-nebst-anhang-v.9-10}{%
\subparagraph{bb) Veruntreuungen und deren Sühne (nebst Anhang
V.9-10)}\label{bb-veruntreuungen-und-deren-suxfchne-nebst-anhang-v.9-10}}

\bibleverse{5}Weiter gebot der HERR dem Mose folgendes:
\bibleverse{6}»Teile den Israeliten folgende Verordnungen mit: Wenn ein
Mann oder ein Weib irgendeine von den mannigfachen Sünden begeht, wie
Menschen sie zu begehen pflegen, so daß sie eine Veruntreuung gegen den
HERRN verüben und der Betreffende dadurch eine Verschuldung auf sich
lädt, \bibleverse{7}so sollen sie die Sünde, die sie begangen haben,
bekennen, und der Betreffende soll das von ihm Veruntreute nach seinem
vollen Wert zurückerstatten und noch ein Fünftel des Betrags hinzufügen
und es dem geben, an dem er sich verschuldet hat. \bibleverse{8}Wenn
dieser Mann aber keinen nächsten Verwandten hat, dem der Ersatz
geleistet werden könnte, so soll die Buße, die dem HERRN zu erstatten
ist, dem Priester gehören, abgesehen von dem Sühnewidder, mittels dessen
man\textless sup title=``d.h. der Priester''\textgreater✲ ihm Sühne
erwirkt. \bibleverse{9}Ebenso soll jede Hebe von allen heiligen Gaben,
welche die Israeliten dem Priester bringen, diesem gehören.
\bibleverse{10}Ihm sollen also die heiligen Gaben eines jeden gehören:
was irgend jemand dem Priester gibt, soll diesem gehören.«

\hypertarget{cc-eiferopfer-und-fluchwasser-bei-einer-des-ehebruchs-verduxe4chtigen-frau}{%
\subparagraph{cc) Eiferopfer und Fluchwasser bei einer des Ehebruchs
verdächtigen
Frau}\label{cc-eiferopfer-und-fluchwasser-bei-einer-des-ehebruchs-verduxe4chtigen-frau}}

\bibleverse{11}Weiter gebot der HERR dem Mose folgendes:
\bibleverse{12}»Teile den Israeliten folgende Verordnungen mit: Wenn
jemandes Ehefrau sich vergeht und ihrem Manne untreu wird,
\bibleverse{13}so daß ein anderer fleischlichen Umgang mit ihr hat, ohne
daß jedoch ihr Mann etwas davon weiß, und ihr tatsächlicher Ehebruch
unentdeckt geblieben ist, auch kein Zeuge gegen sie da ist, weil man sie
bei der Tat nicht ertappt hat, \bibleverse{14}es kommt nun aber der
Geist der Eifersucht über den Mann, so daß er auf seine Frau, die sich
wirklich vergangen hat, eifersüchtig wird -- oder auch es kommt der
Geist der Eifersucht über den Mann, so daß er auf seine Frau
eifersüchtig wird, ohne daß sie sich tatsächlich vergangen hat --:
\bibleverse{15}so soll der Mann seine Frau zum Priester bringen und
zugleich die ihrethalben erforderliche Opfergabe mitbringen, nämlich ein
Zehntel Epha Gerstenmehl; doch darf er kein Öl darübergießen und keinen
Weihrauch darauftun, denn es ist ein Speisopfer der Eifersucht, ein
Speisopfer der Offenbarung, das eine Verschuldung offenbar machen soll.
\bibleverse{16}Hierauf soll der Priester sie herantreten lassen und sie
vor den HERRN stellen. \bibleverse{17}Dann nehme der Priester heiliges
Wasser in einem irdenen Gefäß; weiter nehme der Priester etwas von dem
Staube, der auf dem Fußboden der Wohnung liegt, und tue es in das
Wasser. \bibleverse{18}Dann lasse der Priester die Frau vor den HERRN
treten, löse ihr das Haupthaar auf und gebe ihr das
Offenbarungs-Speisopfer auf die Hände -- es ist ein Speisopfer der
Eifersucht --; aber das fluchbringende Wasser des bitteren Wehs behalte
der Priester in der Hand. \bibleverse{19}Sodann beschwöre sie der
Priester und sage zu der Frau: ›Wenn kein (fremder) Mann dir beigewohnt
hat und du, die du deinem Gatten angehörst, dich nicht vergangen und
verunreinigt hast, so sollst du durch dieses fluchbringende Wasser des
bitteren Wehs unversehrt bleiben! \bibleverse{20}Wenn du dich aber,
obgleich du deinem Gatten angehörst, vergangen und verunreinigt hast, so
daß ein Mann außer deinem Gatten dir beigewohnt hat‹~--
\bibleverse{21}es beschwöre also der Priester die Frau mit feierlicher
Verfluchung und spreche so zu der Frau --, ›so mache der HERR dich zum
abschreckenden Beispiel einer feierlichen Verfluchung inmitten deines
Volkes, indem der HERR deine Hüften schwinden macht und deinen Leib
anschwellen läßt; \bibleverse{22}und es möge dieses fluchbringende
Wasser in deine Eingeweide dringen, so daß dein Leib anschwillt und
deine Hüften schwinden!‹ Die Frau aber soll sagen: ›Ja, so geschehe es!
Amen!‹ \bibleverse{23}Dann schreibe der Priester diese Verfluchungen auf
ein Blatt und wasche die Schrift wieder ab in das Wasser des bitteren
Wehs hinein. \bibleverse{24}Hierauf gebe er der Frau das fluchbringende
Wasser des bitteren Wehs zu trinken, damit das fluchbringende Wasser in
sie eindringe zu bitterem Weh. \bibleverse{25}Dann nehme der Priester
der Frau das Eifersuchts-Speisopfer aus der Hand, webe✲ das Speisopfer
von dem HERRN und bringe es hin zum Altar; \bibleverse{26}dort nehme der
Priester von dem Speisopfer eine Handvoll, nämlich den zum Duftopfer
bestimmten Teil, und lasse es auf dem Altar in Rauch aufgehen; danach
gebe er der Frau das Wasser zu trinken. \bibleverse{27}Hat er sie nun
das Wasser trinken lassen, so wird, falls sie sich vergangen hat und
ihrem Mann untreu gewesen ist, das fluchbringende Wasser zu bitterem Weh
in sie eindringen: ihr Leib wird anschwellen, und ihre Hüften werden
schwinden, und die Frau wird zum abschreckenden Beispiel einer
Verfluchung inmitten ihres Volkes werden. \bibleverse{28}Wenn aber die
Frau sich nicht verunreinigt hat, sondern schuldlos ist, so wird sie
unversehrt bleiben und kann wieder Mutter von Kindern werden.«

\bibleverse{29}Dies ist die Verordnung bezüglich der Eifersucht: Wenn
eine Frau, die einem Ehemann angehört, sich vergeht und sich
verunreinigt, \bibleverse{30}oder auch wenn über einen Mann der Geist
der Eifersucht kommt, so daß er auf seine Frau eifersüchtig wird, so
soll er die Frau vor den HERRN treten lassen, und der Priester soll
genau nach dieser Verordnung mit ihr verfahren. \bibleverse{31}Der Mann
soll dabei außer Schuld bleiben, eine solche Frau aber muß ihre Schuld
büßen.

\hypertarget{dd-vorschriften-bezuxfcglich-der-nasiruxe4er-d.h.-der-gottgeweihten}{%
\subparagraph{dd) Vorschriften bezüglich der Nasiräer (d.h. der
Gottgeweihten)}\label{dd-vorschriften-bezuxfcglich-der-nasiruxe4er-d.h.-der-gottgeweihten}}

\hypertarget{allgemeine-bestimmungen}{%
\paragraph{Allgemeine Bestimmungen}\label{allgemeine-bestimmungen}}

\hypertarget{section-5}{%
\section{6}\label{section-5}}

\bibleverse{1}Weiter gebot der HERR dem Mose folgendes:
\bibleverse{2}»Teile den Israeliten folgende Verordnungen mit: Wenn ein
Mann oder ein Weib das Absonderungsgelübde eines Nasiräers\textless sup
title=``d.h. eines Gottgeweihten''\textgreater✲ ablegen will, um sich
dem HERRN zu weihen, \bibleverse{3}so muß er sich des Weines und jedes
andern berauschenden Getränks enthalten; auch Essig\textless sup
title=``oder: Gegorenes''\textgreater✲ von Wein oder Essig von
berauschendem Getränk darf er nicht trinken; ebenso darf er keinerlei
aus Trauben hergestellte Flüssigkeit trinken und keine Trauben, weder
frische noch getrocknete, genießen. \bibleverse{4}Während der ganzen
Dauer seiner Weihezeit darf er von allem, was vom Weinstock
kommt\textless sup title=``=~gewonnen wird''\textgreater✲, nichts
genießen, auch keine unreifen Trauben und keine Rankenspitzen.
\bibleverse{5}Während der ganzen Dauer seines Weihegelübdes darf kein
Schermesser auf sein Haupt kommen; bis zum Ablauf der Zeit, für die er
sich dem HERRN geweiht hat, soll er als geheiligt gelten: er hat sein
Haupthaar frei wachsen zu lassen. \bibleverse{6}Während der ganzen Zeit,
für die er sich dem HERRN geweiht hat, darf er zu keinem Toten
hineingehen; \bibleverse{7}selbst wenn ihm Vater oder Mutter, Bruder
oder Schwester sterben, darf er sich an ihren Leichen nicht
verunreinigen; denn die Weihe seines Gottes ruht auf seinem Haupt:
\bibleverse{8}solange seine Weihe\textless sup title=``=~gelobte
Zeit''\textgreater✲ dauert, ist er dem HERRN heilig.«

\hypertarget{vorschriften-betreffend-die-verunreinigung-des-nasiruxe4ers}{%
\paragraph{Vorschriften betreffend die Verunreinigung des
Nasiräers}\label{vorschriften-betreffend-die-verunreinigung-des-nasiruxe4ers}}

\bibleverse{9}»Sollte aber jemand ganz plötzlich neben ihm sterben und
er dadurch sein geweihtes Haupt verunreinigen, so soll er an dem Tage,
an dem er wieder rein wird, sein Haupt scheren, also am siebten Tage;
\bibleverse{10}und am achten Tage soll er zwei Turteltauben oder zwei
junge Tauben zu dem Priester an den Eingang des Offenbarungszeltes
bringen. \bibleverse{11}Der Priester soll dann die eine zum Sündopfer
und die andere zum Brandopfer herrichten und ihm so Sühne erwirken
dafür, daß er sich an der Leiche (verunreinigt und sich dadurch)
versündigt hat. Dann soll er an demselben Tage sein Haupt nochmals für
geheiligt erklären \bibleverse{12}und sich dem HERRN für die gelobte
Zeit von neuem weihen, auch ein einjähriges Lamm als Schuldopfer
darbringen; die vorige Zeit aber soll als verfallen gelten, weil seine
Weihe unrein geworden war.«

\hypertarget{verordnungen-uxfcber-die-opferfeier-am-ende-des-nasiruxe4ats}{%
\paragraph{Verordnungen über die Opferfeier am Ende des
Nasiräats}\label{verordnungen-uxfcber-die-opferfeier-am-ende-des-nasiruxe4ats}}

\bibleverse{13}»Folgende Vorschriften aber gelten für den Nasiräer✲: An
dem Tage, an welchem die von ihm gelobte Weihezeit zu Ende ist, führe
man ihn an den Eingang des Offenbarungszeltes; \bibleverse{14}und er hat
dem HERRN seine Opfergabe darzubringen, nämlich ein einjähriges,
fehlerloses Lamm zum Brandopfer und ein einjähriges, fehlerloses
weibliches Lamm zum Sündopfer, ferner einen fehlerlosen Widder zum
Heilsopfer; \bibleverse{15}außerdem einen Korb mit ungesäuertem Backwerk
von Feinmehl, nämlich mit Öl gemengte Kuchen und ungesäuerte, mit Öl
bestrichene Fladen, nebst dem zugehörigen Speisopfer und den
erforderlichen Trankopfern. \bibleverse{16}Der Priester soll dann dies
alles vor den HERRN bringen und das Sündopfer und das Brandopfer für ihn
herrichten; \bibleverse{17}den Widder aber soll er als Heilsopfer für
den HERRN herrichten samt dem Korbe mit dem ungesäuerten Backwerk; auch
das Speisopfer und das Trankopfer soll der Priester für ihn darbringen.
\bibleverse{18}Sodann schere der Nasiräer sein geweihtes Haupt am
Eingang des Offenbarungszeltes, nehme sein geweihtes Haupthaar und lege
es in das Feuer, das unter dem Heilsopfer brennt. \bibleverse{19}Hierauf
nehme der Priester den gekochten Bug von dem Widder nebst einem
ungesäuerten Kuchen und einem ungesäuerten Fladen aus dem Korbe und lege
dies alles dem Nasiräer auf die Hände, nachdem dieser sich das geweihte
Haar abgeschoren hat. \bibleverse{20}Sodann webe✲ der Priester dieses
alles als Webeopfer vor dem HERRN: es ist eine dem Priester zufallende
heilige Gabe samt der Webebrust und der Hebekeule. Danach darf der
Nasiräer wieder Wein trinken.«

\bibleverse{21}Diese Vorschriften gelten für den Gottgeweihten, der ein
Gelübde abgelegt hat, nämlich bezüglich seiner Opfergabe, die er dem
HERRN auf Grund seiner Weihe darzubringen hat, abgesehen von dem, was er
sonst noch zu leisten vermag. Auf Grund seines Gelübdes, das er abgelegt
hat, soll er nach den für seine Weihe geltenden Vorschriften verfahren.

\hypertarget{ee-anordnung-des-priesterlichen-segens}{%
\subparagraph{ee) Anordnung des priesterlichen
Segens}\label{ee-anordnung-des-priesterlichen-segens}}

\bibleverse{22}Der HERR gebot alsdann dem Mose folgendes:
\bibleverse{23}»Gib Aaron und seinen Söhnen folgende Weisung: Mit diesen
Worten sollt ihr den Segen über die Israeliten aussprechen:
\bibleverse{24}›Der HERR segne dich und behüte dich! \bibleverse{25}Der
HERR lasse sein Angesicht leuchten über dir und sei dir gnädig!
\bibleverse{26}Der HERR erhebe sein Angesicht zu dir hin\textless sup
title=``oder: auf dich''\textgreater✲ und gewähre dir Frieden!‹
\bibleverse{27}Wenn sie so meinen Namen auf die Israeliten legen, will
ich sie segnen.«

\hypertarget{f-die-weihegeschenke-der-stammesfuxfcrsten-fuxfcr-das-heiligtum}{%
\paragraph{f) Die Weihegeschenke der Stammesfürsten für das
Heiligtum}\label{f-die-weihegeschenke-der-stammesfuxfcrsten-fuxfcr-das-heiligtum}}

\hypertarget{section-6}{%
\section{7}\label{section-6}}

\bibleverse{1}An dem Tage nun, als Mose mit der Errichtung der heiligen
Wohnung fertig war und sie gesalbt und geheiligt\textless sup
title=``oder: geweiht''\textgreater✲ hatte samt allen ihren Geräten,
auch den Brandaltar mit allen seinen Geräten gesalbt und geweiht hatte,
\bibleverse{2}da brachten die Fürsten der Israeliten, die Häupter der
einzelnen Stämme -- das sind die Stammesfürsten, die als Vorsteher die
Musterung vorgenommen hatten --, \bibleverse{3}da brachten sie ihre
Opfergabe vor den HERRN, nämlich sechs überdeckte Wagen und zwölf
Rinder, je einen Wagen auf zwei Fürsten und je ein Rind von jedem: die
brachten sie vor die heilige Wohnung. \bibleverse{4}Da gebot der HERR
dem Mose folgendes: \bibleverse{5}»Nimm sie von ihnen an, damit sie beim
Dienst am Offenbarungszelt Verwendung finden, und übergib sie den
Leviten unter Berücksichtigung des von jedem zu leistenden Dienstes.«
\bibleverse{6}So nahm denn Mose die Wagen und die Rinder und übergab sie
den Leviten. \bibleverse{7}Zwei von den Wagen und vier Rinder übergab er
den Gersoniten mit Rücksicht auf den von ihnen zu leistenden Dienst;
\bibleverse{8}die andern vier Wagen und acht Rinder aber übergab er den
Merariten mit Rücksicht auf den Dienst, den sie unter der
Aufsicht\textless sup title=``oder: Leitung''\textgreater✲ Ithamars, des
Sohnes des Priesters Aaron, zu leisten hatten. \bibleverse{9}Den
Kehathiten aber übergab er nichts; denn ihnen oblag die Besorgung der
heiligsten Gegenstände, die sie auf der Schulter tragen mußten.

\bibleverse{10}Sodann brachten die Fürsten die Einweihungsgaben für den
Altar an dem Tage dar, an welchem er gesalbt wurde, und zwar brachten
die Fürsten ihre Opfergaben vor den Altar. \bibleverse{11}Da gebot der
HERR dem Mose: »Tag für Tag soll jedesmal nur einer der Fürsten seine
Opfergabe zur Einweihung des Altars darbringen.«

\bibleverse{12}Derjenige nun, welcher am ersten Tage seine Opfergabe
darbrachte, war Nahson, der Sohn Amminadabs, vom Stamme Juda.
\bibleverse{13}Seine Opfergabe war: eine silberne Schüssel, 130~Schekel
schwer, ein silbernes Becken, 70~Schekel schwer nach dem Gewicht des
Heiligtums, beide gefüllt mit Feinmehl, das mit Öl gemengt war, zum
Speisopfer; \bibleverse{14}eine Schale von Gold, 10~Schekel schwer, mit
Räucherwerk gefüllt; \bibleverse{15}ein junger Stier, ein Widder und ein
einjähriges Lamm zum Brandopfer; \bibleverse{16}ein Ziegenbock zum
Sündopfer; \bibleverse{17}ferner zum Heilsopfer zwei Rinder, fünf
Widder, fünf Böcke und fünf einjährige Lämmer. Das war die Opfergabe
Nahsons, des Sohnes Amminadabs.

\bibleverse{18}Am zweiten Tage opferte Nethaneel, der Sohn Zuars, der
Fürst von Issaschar. \bibleverse{19}Er brachte als seine Opfergabe dar:
eine silberne Schüssel, 130~Schekel schwer, ein silbernes Becken,
70~Schekel schwer nach dem Gewicht des Heiligtums, beide gefüllt mit
Feinmehl, das mit Öl gemengt war, zum Speisopfer; \bibleverse{20}eine
Schale von Gold, 10~Schekel schwer, mit Räucherwerk gefüllt;
\bibleverse{21}einen jungen Stier, einen Widder und ein einjähriges Lamm
zum Brandopfer; \bibleverse{22}einen Ziegenbock zum Sündopfer;
\bibleverse{23}ferner zum Heilsopfer zwei Rinder, fünf Widder, fünf
Böcke und fünf einjährige Lämmer. Das war die Opfergabe Nethaneels, des
Sohnes Zuars.

\bibleverse{24}Am dritten Tage opferte der Fürst des Stammes Sebulon,
Eliab, der Sohn Helons. \bibleverse{25}Seine Opfergabe war: eine
silberne Schüssel, 130~Schekel schwer, ein silbernes Becken, 70~Schekel
schwer nach dem Gewicht des Heiligtums, beide gefüllt mit Feinmehl, das
mit Öl gemengt war, zum Speisopfer; \bibleverse{26}eine Schale von Gold,
10~Schekel schwer, mit Räucherwerk gefüllt; \bibleverse{27}ein junger
Stier, ein Widder und ein einjähriges Lamm zum Brandopfer;
\bibleverse{28}ein Ziegenbock zum Sündopfer; \bibleverse{29}ferner zum
Heilsopfer zwei Rinder, fünf Widder, fünf Böcke und fünf einjährige
Lämmer. Das war die Opfergabe Eliabs, des Sohnes Helons.

\bibleverse{30}Am vierten Tage opferte der Fürst des Stammes Ruben,
Elizur, der Sohn Sedeurs. \bibleverse{31}Seine Opfergabe war: eine
silberne Schüssel, 130~Schekel schwer, ein silbernes Becken, 70~Schekel
schwer nach dem Gewicht des Heiligtums, beide gefüllt mit Feinmehl, das
mit Öl gemengt war, zum Speisopfer; \bibleverse{32}eine Schale von Gold,
10~Schekel schwer, mit Räucherwerk gefüllt; \bibleverse{33}ein junger
Stier, ein Widder und ein einjähriges Lamm zum Brandopfer;
\bibleverse{34}ein Ziegenbock zum Sündopfer; \bibleverse{35}ferner zum
Heilsopfer zwei Rinder, fünf Widder, fünf Böcke und fünf einjährige
Lämmer. Das war die Opfergabe Elizurs, des Sohnes Sedeurs.

\bibleverse{36}Am fünften Tage opferte der Fürst des Stammes Simeon,
Selumiel, der Sohn Zurisaddais. \bibleverse{37}Seine Opfergabe war: eine
silberne Schüssel, 130~Schekel schwer, ein silbernes Becken, 70~Schekel
schwer nach dem Gewicht des Heiligtums, beide gefüllt mit Feinmehl, das
mit Öl gemengt war, zum Speisopfer; \bibleverse{38}eine Schale von Gold,
10~Schekel schwer, mit Räucherwerk gefüllt; \bibleverse{39}ein junger
Stier, ein Widder und ein einjähriges Lamm zum Brandopfer;
\bibleverse{40}ein Ziegenbock zum Sündopfer; \bibleverse{41}ferner zum
Heilsopfer zwei Rinder, fünf Widder, fünf Böcke und fünf einjährige
Lämmer. Das war die Opfergabe Selumiels, des Sohnes Zurisaddais.

\bibleverse{42}Am sechsten Tage opferte der Fürst des Stammes Gad,
Eljasaph, der Sohn Deguels. \bibleverse{43}Seine Opfergabe war: eine
silberne Schüssel, 130~Schekel schwer, ein silbernes Becken, 70~Schekel
schwer nach dem Gewicht des Heiligtums, beide gefüllt mit Feinmehl, das
mit Öl gemengt war, zum Speisopfer; \bibleverse{44}eine Schale von Gold,
10~Schekel schwer, mit Räucherwerk gefüllt; \bibleverse{45}ein junger
Stier, ein Widder und ein einjähriges Lamm zum Brandopfer;
\bibleverse{46}ein Ziegenbock zum Sündopfer; \bibleverse{47}ferner zum
Heilsopfer zwei Rinder, fünf Widder, fünf Böcke und fünf einjährige
Lämmer. Das war die Opfergabe Eljasaphs, des Sohnes Deguels.

\bibleverse{48}Am siebenten Tage opferte der Fürst des Stammes Ephraim,
Elisama, der Sohn Ammihuds. \bibleverse{49}Seine Opfergabe war: eine
silberne Schüssel, 130~Schekel schwer, ein silbernes Becken, 70~Schekel
schwer nach dem Gewicht des Heiligtums, beide gefüllt mit Feinmehl, das
mit Öl gemengt war, zum Speisopfer; \bibleverse{50}eine Schale von Gold,
10~Schekel schwer, mit Räucherwerk gefüllt; \bibleverse{51}ein junger
Stier, ein Widder und ein einjähriges Lamm zum Brandopfer;
\bibleverse{52}ein Ziegenbock zum Sündopfer; \bibleverse{53}ferner zum
Heilsopfer zwei Rinder, fünf Widder, fünf Böcke und fünf einjährige
Lämmer. Das war die Opfergabe Elisamas, des Sohnes Ammihuds.

\bibleverse{54}Am achten Tage opferte der Fürst des Stammes Manasse,
Gamliel, der Sohn Pedazurs. \bibleverse{55}Seine Opfergabe war: eine
silberne Schüssel, 130~Schekel schwer, ein silbernes Becken, 70~Schekel
schwer nach dem Gewicht des Heiligtums, beide gefüllt mit Feinmehl, das
mit Öl gemengt war, zum Speisopfer; \bibleverse{56}eine Schale von Gold,
10~Schekel schwer, mit Räucherwerk gefüllt; \bibleverse{57}ein junger
Stier, ein Widder und ein einjähriges Lamm zum Brandopfer;
\bibleverse{58}ein Ziegenbock zum Sündopfer; \bibleverse{59}ferner zum
Heilsopfer zwei Rinder, fünf Widder, fünf Böcke und fünf einjährige
Lämmer. Das war die Opfergabe Gamliels, des Sohnes Pedazurs.

\bibleverse{60}Am neunten Tage opferte der Fürst des Stammes Benjamin,
Abidan, der Sohn Gideonis. \bibleverse{61}Seine Opfergabe war: eine
silberne Schüssel, 130~Schekel schwer, ein silbernes Becken, 70~Schekel
schwer nach dem Gewicht des Heiligtums, beide gefüllt mit Feinmehl, das
mit Öl gemengt war, zum Speisopfer; \bibleverse{62}eine Schale von Gold,
10~Schekel schwer, mit Räucherwerk gefüllt; \bibleverse{63}ein junger
Stier, ein Widder und ein einjähriges Lamm zum Brandopfer;
\bibleverse{64}ein Ziegenbock zum Sündopfer; \bibleverse{65}ferner zum
Heilsopfer zwei Rinder, fünf Widder, fünf Böcke und fünf einjährige
Lämmer. Das war die Opfergabe Abidans, des Sohnes Gideonis.

\bibleverse{66}Am zehnten Tage opferte der Fürst des Stammes Dan,
Ahieser, der Sohn Ammisaddais. \bibleverse{67}Seine Opfergabe war: eine
silberne Schüssel, 130~Schekel schwer, ein silbernes Becken, 70~Schekel
schwer nach dem Gewicht des Heiligtums, beide gefüllt mit Feinmehl, das
mit Öl gemengt war, zum Speisopfer; \bibleverse{68}eine Schale von Gold,
10~Schekel schwer, mit Räucherwerk gefüllt; \bibleverse{69}ein junger
Stier, ein Widder und ein einjähriges Lamm zum Brandopfer;
\bibleverse{70}ein Ziegenbock zum Sündopfer; \bibleverse{71}ferner zum
Heilsopfer zwei Rinder, fünf Widder, fünf Böcke und fünf einjährige
Lämmer. Das war die Opfergabe Ahiesers, des Sohnes Ammisaddais.

\bibleverse{72}Am elften Tage opferte der Fürst des Stammes Asser,
Pagiel, der Sohn Ochrans. \bibleverse{73}Seine Opfergabe war: eine
silberne Schüssel, 130~Schekel schwer, ein silbernes Becken, 70~Schekel
schwer nach dem Gewicht des Heiligtums, beide gefüllt mit Feinmehl, das
mit Öl gemengt war, zum Speisopfer; \bibleverse{74}eine Schale von Gold,
10~Schekel schwer, mit Räucherwerk gefüllt; \bibleverse{75}ein junger
Stier, ein Widder und ein einjähriges Lamm zum Brandopfer;
\bibleverse{76}ein Ziegenbock zum Sündopfer; \bibleverse{77}ferner zum
Heilsopfer zwei Rinder, fünf Widder, fünf Böcke und fünf einjährige
Lämmer. Das war die Opfergabe Pagiels, des Sohnes Ochrans.

\bibleverse{78}Am zwölften Tage opferte der Fürst des Stammes Naphthali,
Ahira, der Sohn Enans. \bibleverse{79}Seine Opfergabe war: eine silberne
Schüssel, 130~Schekel schwer, ein silbernes Becken, 70~Schekel schwer
nach dem Gewicht des Heiligtums, beide gefüllt mit Feinmehl, das mit Öl
gemengt war, zum Speisopfer; \bibleverse{80}eine Schale von Gold,
10~Schekel schwer, mit Räucherwerk gefüllt; \bibleverse{81}ein junger
Stier, ein Widder und ein einjähriges Lamm zum Brandopfer;
\bibleverse{82}ein Ziegenbock zum Sündopfer; \bibleverse{83}ferner zum
Heilsopfer zwei Rinder, fünf Widder, fünf Böcke und fünf einjährige
Lämmer. Das war die Opfergabe Ahiras, des Sohnes Enans.

\bibleverse{84}Dies waren von seiten der Fürsten der Israeliten die
Gaben zur Einweihung des Altars an dem Tage, an welchem er gesalbt
wurde, nämlich 12 silberne Schüsseln, 12 silberne Becken, 12 goldene
Schalen, \bibleverse{85}jede silberne Schüssel 130~Schekel, jedes Becken
70~Schekel schwer; das gesamte Silber der Gefäße betrug also
2400~Schekel nach dem Gewicht des Heiligtums; \bibleverse{86}ferner
zwölf goldene Schalen, mit Räucherwerk gefüllt, jede Schale 10~Schekel
schwer nach dem Gewicht des Heiligtums; das gesamte Gold der Schalen
betrug also 120~Schekel. \bibleverse{87}Die Gesamtzahl der Rinder zum
Brandopfer belief sich auf 12 junge Stiere, dazu 12 Widder, 12
einjährige Lämmer nebst dem zugehörigen Speisopfer, und 12 Ziegenböcke
zum Sündopfer. \bibleverse{88}Die sämtlichen Rinder zum Heilsopfer
beliefen sich auf 24 junge Stiere, dazu 60 Widder, 60 Böcke, 60
einjährige Lämmer. Dies waren die Gaben zur Einweihung des Altars,
nachdem er gesalbt worden war.

\hypertarget{schluuxdfsatz-erfuxfcllung-der-in-p2.mose-2522}{%
\paragraph{Schlußsatz: Erfüllung der in \textbar p2.Mose
25,22}\label{schluuxdfsatz-erfuxfcllung-der-in-p2.mose-2522}}

\bibleverse{89}Wenn nun Mose in das Offenbarungszelt hineinging, um mit
dem HERRN zu reden, hörte er die Stimme zu sich reden von der Deckplatte
her, die über der Gesetzeslade lag, und zwar von dem Raum zwischen den
beiden Cheruben her; und so redete er (der HERR) zu ihm.

\hypertarget{g-besorgung-des-goldenen-leuchters-weihe-und-dienstalter-der-leviten}{%
\paragraph{g) Besorgung des goldenen Leuchters; Weihe und Dienstalter
der
Leviten}\label{g-besorgung-des-goldenen-leuchters-weihe-und-dienstalter-der-leviten}}

\hypertarget{aa-die-sieben-lampen-am-leuchter}{%
\subparagraph{aa) Die sieben Lampen am
Leuchter}\label{aa-die-sieben-lampen-am-leuchter}}

\hypertarget{section-7}{%
\section{8}\label{section-7}}

\bibleverse{1}Hierauf gebot der HERR dem Mose folgendes:
\bibleverse{2}»Gib dem Aaron folgende Weisung: Wenn du die Lampen
aufsetzt, so sollen die sieben Lampen ihr Licht nach der Vorderseite des
Leuchters hin werfen.« \bibleverse{3}Da machte Aaron es so: an der
Vorderseite des Leuchters setzte er dessen Lampen auf, wie der HERR dem
Mose geboten hatte. \bibleverse{4}Der Leuchter war aber aus Gold in
getriebener Arbeit hergestellt: sowohl sein Schaft als auch seine Blüten
-- alles war getriebene Arbeit; nach dem Muster, das der HERR dem Mose
gezeigt hatte, so hatte er den Leuchter anfertigen lassen.

\hypertarget{bb-die-darbringung-weihe-der-leviten-als-einer-heiligen-gabe-an-gott}{%
\subparagraph{bb) Die Darbringung (=~Weihe) der Leviten als einer
heiligen Gabe an
Gott}\label{bb-die-darbringung-weihe-der-leviten-als-einer-heiligen-gabe-an-gott}}

\bibleverse{5}Weiter gebot der HERR dem Mose folgendes:
\bibleverse{6}»Sondere die Leviten aus der Mitte der Israeliten aus und
reinige sie; \bibleverse{7}und zwar sollst du behufs ihrer Reinigung so
mit ihnen verfahren: Besprenge sie mit Entsündigungswasser; dann sollen
sie ein Schermesser über ihren ganzen Leib gehen lassen und ihre Kleider
waschen und so sich reinigen. \bibleverse{8}Hierauf sollen sie einen
jungen Stier nehmen nebst dem zugehörigen Speisopfer, nämlich Feinmehl,
das mit Öl gemengt ist; und einen zweiten jungen Stier sollst du zum
Sündopfer nehmen. \bibleverse{9}Dann sollst du die Leviten vor das
Offenbarungszelt treten lassen und die ganze Gemeinde der Israeliten
dort versammeln. \bibleverse{10}Hierauf laß die Leviten vor den HERRN
treten, und die Israeliten sollen ihre Hände fest auf die Leviten legen;
\bibleverse{11}dann soll Aaron die Leviten vor dem HERRN als ein von
seiten der Israeliten dargebrachtes Webeopfer weben✲, damit sie den
Dienst des HERRN zu verrichten geeignet werden. \bibleverse{12}Nachdem
dann die Leviten ihre Hände fest auf den Kopf der Stiere gelegt haben,
sollst du den einen Stier zum Sündopfer, den andern zum Brandopfer für
den HERRN herrichten, um den Leviten Sühne zu erwirken.
\bibleverse{13}Dann laß die Leviten vor Aaron und seine Söhne treten und
webe✲ sie als Webeopfer für den HERRN: \bibleverse{14}auf diese Weise
sollst du die Leviten aus der Mitte der Israeliten aussondern, damit die
Leviten mir gehören. \bibleverse{15}Darnach sollen die Leviten
hineingehen, um den Dienst am Offenbarungszelt zu verrichten. So sollst
du sie reinigen und als Webeopfer weben; \bibleverse{16}denn sie sind
mir aus der Mitte der Israeliten ganz zu eigen gegeben: als Ersatz für
alle Erstgeburten, für alle Kinder, die unter den Israeliten zuerst zur
Welt kommen, habe ich sie für mich genommen. \bibleverse{17}Denn mir
gehört alles Erstgeborene in Israel, von den Menschen wie vom Vieh; an
dem Tage, als ich alle Erstgeburten im Lande Ägypten sterben ließ, habe
ich sie mir geheiligt. \bibleverse{18}So habe ich denn die Leviten als
Ersatz für alle Erstgeborenen unter den Israeliten genommen
\bibleverse{19}und habe die Leviten dem Aaron und seinen Söhnen aus der
Mitte der Israeliten zu eigen gegeben, damit sie den von den Israeliten
zu leistenden Dienst am Offenbarungszelt verrichten und den Israeliten
als Deckung dienen, auf daß kein Unglücksschlag die Israeliten trifft,
wenn sie selbst sich dem Heiligtum nähern würden\textless sup
title=``oder: an das Heilige heranträten''\textgreater✲.«

\bibleverse{20}Hierauf verfuhren Mose und Aaron und die ganze Gemeinde
der Israeliten mit den Leviten genau so, wie der HERR dem Mose bezüglich
der Leviten geboten hatte: genau so verfuhren die Israeliten mit ihnen.
\bibleverse{21}Die Leviten ließen sich nämlich entsündigen und wuschen
ihre Kleider; Aaron webte✲ sie dann als Webeopfer vor dem HERRN, und
Aaron vollzog zu ihrer Reinigung die Sühnehandlungen für sie.
\bibleverse{22}Danach gingen die Leviten hinein, um ihren Dienst am
Offenbarungszelt unter der Aufsicht Aarons und seiner Söhne zu
verrichten; wie der HERR dem Mose in betreff der Leviten geboten hatte,
so verfuhren sie mit ihnen.

\hypertarget{cc-die-zeit-der-dienstpflicht-der-leviten}{%
\subparagraph{cc) Die Zeit der Dienstpflicht der
Leviten}\label{cc-die-zeit-der-dienstpflicht-der-leviten}}

\bibleverse{23}Hierauf sagte der HERR weiter zu Mose:
\bibleverse{24}»Folgende Vorschriften sollen in betreff der Leviten
gelten: Im Alter von fünfundzwanzig Jahren und darüber soll der Levit im
Dienst zur Besorgung der Verrichtungen am Offenbarungszelt
tätig\textless sup title=``oder: verpflichtet''\textgreater✲ sein;
\bibleverse{25}aber von fünfzig Jahren an soll er von den
Dienstleistungen befreit sein und nicht mehr dienen. \bibleverse{26}Er
mag alsdann seinen Brüdern im Offenbarungszelt behilflich sein, wenn sie
ihren Dienst verrichten, aber regelmäßigen Dienst soll er nicht mehr
leisten. So sollst du es mit den Leviten bezüglich ihrer Amtsgeschäfte
halten.«

\hypertarget{h-verordnungen-bezuxfcglich-der-passahfeier-die-wolkensuxe4ule-uxfcber-der-heiligen-wohnung}{%
\paragraph{h) Verordnungen bezüglich der Passahfeier; die Wolkensäule
über der heiligen
Wohnung}\label{h-verordnungen-bezuxfcglich-der-passahfeier-die-wolkensuxe4ule-uxfcber-der-heiligen-wohnung}}

\hypertarget{aa-die-nachtruxe4gliche-passahfeier-fuxfcr-unreine-und-reisende-das-passah-der-fremdlinge}{%
\subparagraph{aa) Die nachträgliche Passahfeier für Unreine und
Reisende; das Passah der
Fremdlinge}\label{aa-die-nachtruxe4gliche-passahfeier-fuxfcr-unreine-und-reisende-das-passah-der-fremdlinge}}

\hypertarget{section-8}{%
\section{9}\label{section-8}}

\bibleverse{1}Weiter gebot der HERR dem Mose in der Wüste Sinai im
zweiten Jahr nach ihrem Auszug aus dem Lande Ägypten, im ersten Monat,
folgendes: \bibleverse{2}»Die Israeliten sollen das Passah zur
festgesetzten Zeit feiern: \bibleverse{3}am vierzehnten Tage in diesem
Monat gegen Abend sollt ihr es zur festgesetzten Zeit feiern; nach allen
darauf bezüglichen Bestimmungen und nach allen dafür geltenden
Vorschriften sollt ihr es feiern.« \bibleverse{4}Da befahl Mose den
Israeliten, das Passah zu feiern; \bibleverse{5}und sie feierten es im
ersten Monat, am vierzehnten Tage des Monats, gegen Abend in der Wüste
Sinai; ganz wie der HERR dem Mose geboten hatte, so verfuhren die
Israeliten dabei.

\bibleverse{6}Nun waren aber einige Männer da, die sich an einer
Menschenleiche verunreinigt hatten und deshalb das Passah an jenem Tage
nicht feiern konnten. Diese traten nun an jenem Tage vor Mose und Aaron
\bibleverse{7}und sagten zu ihm: »Wir sind durch eine Menschenleiche
unrein geworden: warum sollen wir benachteiligt werden, daß wir die
Opfergabe für den HERRN nicht zur festgesetzten Zeit inmitten der
Israeliten darbringen dürfen?« \bibleverse{8}Mose antwortete ihnen:
»Wartet, ich will hören, was der HERR euretwegen anordnet!«

\bibleverse{9}Der HERR aber gebot dem Mose folgendes:
\bibleverse{10}»Teile den Israeliten folgende Verordnungen mit: Wenn
irgend jemand von euch oder von euren Nachkommen sich an einer Leiche
verunreinigt hat oder sich auf einer Reise in der Ferne befindet, so
soll er doch das Passah zu Ehren des HERRN feiern: \bibleverse{11}im
zweiten Monat, am vierzehnten Tage, gegen Abend sollen die Betreffenden
es feiern; mit ungesäuertem Brot und bitteren Kräutern sollen sie es
verzehren. \bibleverse{12}Sie dürfen nichts davon bis zum folgenden
Morgen übriglassen, auch keinen Knochen an ihm zerbrechen; nach allen
für das Passah geltenden Bestimmungen sollen sie es feiern.
\bibleverse{13}Wer aber rein ist und sich auf keiner Reise befindet und
die Passahfeier trotzdem unterläßt, ein solcher Mensch soll aus seinen
Volksgenossen ausgerottet werden, weil er dem HERRN die Opfergabe nicht
zur festgesetzten Zeit dargebracht hat: ein solcher Mensch soll für
seine Sünde büßen!~-- \bibleverse{14}Wenn ferner ein Fremdling sich
unter euch aufhält und dem HERRN das Passah feiern will, so soll er es
nach den für das Passah geltenden Bestimmungen und Vorschriften feiern;
die gleichen Bestimmungen sollen für euch gelten, sowohl für den
Fremdling als auch für den im Lande Geborenen\textless sup title=``oder:
Einheimischen''\textgreater✲.«

\hypertarget{bb-das-erscheinen-der-wolken--und-feuersuxe4ule-uxfcber-dem-heiligtum}{%
\subparagraph{bb) Das Erscheinen der Wolken- und Feuersäule über dem
Heiligtum}\label{bb-das-erscheinen-der-wolken--und-feuersuxe4ule-uxfcber-dem-heiligtum}}

\bibleverse{15}An dem Tage aber, an welchem man die heilige Wohnung
aufgeschlagen hatte, bedeckte die Wolke die Wohnung, nämlich das Zelt
mit dem Gesetz; doch am Abend lag sie über der Wohnung wie ein
Feuerschein bis zum Morgen. \bibleverse{16}So blieb es die ganze
Folgezeit hindurch: die Wolke bedeckte die Wohnung, und zwar nachts als
ein Feuerschein. \bibleverse{17}Sobald sich nun die Wolke von dem Zelt
erhob, brachen die Israeliten alsbald danach auf; und an dem Orte, wo
die Wolke sich wieder niederließ, da lagerten die Israeliten:
\bibleverse{18}nach dem Befehl des HERRN brachen die Israeliten auf, und
nach dem Befehl des HERRN lagerten sie; solange die Wolke ruhig über der
Wohnung lag, so lange blieben sie gelagert. \bibleverse{19}Auch wenn die
Wolke viele Tage lang über der Wohnung stehen blieb, beobachteten die
Israeliten die Weisung\textless sup title=``oder: das
Geheiß''\textgreater✲ des HERRN und zogen nicht weiter.
\bibleverse{20}Es kam aber auch vor, daß die Wolke nur wenige Tage über
der Wohnung stehen blieb -- nach dem Befehl des HERRN lagerten sie, und
nach dem Befehl des HERRN brachen sie auf. \bibleverse{21}Es kam auch
vor, daß die Wolke nur vom Abend bis zum Morgen blieb; wenn sie sich
dann am Morgen erhob, so brach man auf; oder wenn die Wolke einen Tag
und eine Nacht blieb und sich dann erhob, so brach man auf;
\bibleverse{22}oder wenn die Wolke zwei Tage oder einen Monat oder noch
längere Zeit blieb, indem die Wolke über der Wohnung Halt machte und auf
ihr ruhte, so blieben die Israeliten gelagert und brachen nicht auf;
sobald sie sich aber erhob, brachen sie auf: \bibleverse{23}nach dem
Geheiß des HERRN blieben sie gelagert, und nach dem Geheiß des HERRN
brachen sie auf: sie beobachteten die Weisung des HERRN nach dem durch
Mose übermittelten Geheiß des HERRN.

\hypertarget{i-verordnung-bezuxfcglich-zweier-silberner-trompeten}{%
\paragraph{i) Verordnung bezüglich zweier silberner
Trompeten}\label{i-verordnung-bezuxfcglich-zweier-silberner-trompeten}}

\hypertarget{section-9}{%
\section{10}\label{section-9}}

\bibleverse{1}Weiter gebot der HERR dem Mose folgendes:
\bibleverse{2}»Fertige dir zwei silberne Trompeten an; in getriebener
Arbeit sollst du sie anfertigen, und sie sollen dir dazu dienen, die
Gemeinde zusammenzurufen und das Zeichen zum Aufbruch der Lager zu
geben. \bibleverse{3}Sobald mit ihnen (beiden) geblasen wird, soll sich
die ganze Gemeinde bei dir am Eingang des Offenbarungszeltes versammeln;
\bibleverse{4}wird aber nur mit einer geblasen, so sollen sich die
Fürsten, die Häupter der Tausendschaften\textless sup title=``oder: die
Stammeshäupter''\textgreater✲ der Israeliten, bei dir versammeln.
\bibleverse{5}Wenn aber Alarm✲ geblasen wird, so sollen die ostwärts
liegenden Lager aufbrechen; \bibleverse{6}und wenn zum zweitenmal Alarm
geblasen wird, sollen die südwärts liegenden Lager aufbrechen: Alarm
soll geblasen werden zu ihrem Aufbruch. \bibleverse{7}Wenn es sich aber
um die Versammlung der Gemeinde handelt, sollt ihr einfache
Trompetenzeichen geben, aber keinen Alarm blasen. \bibleverse{8}Das
Blasen der Trompeten soll den Söhnen Aarons, den Priestern, obliegen;
diese Vorschriften sollen bei euch ewige Geltung für eure künftigen
Geschlechter haben. \bibleverse{9}Und wenn ihr in eurem Lande gegen den
Feind, der euch bedrängt, in den Krieg zieht und Alarm mit den Trompeten
blast, so wird euer beim HERRN, eurem Gott, gedacht werden, so daß ihr
Rettung von euren Feinden erlangt. \bibleverse{10}Auch an euren
Freudentagen und Festen sowie an euren Neumonden sollt ihr zu euren
Brandopfern und zu euren Heilsopfern die Trompeten blasen, damit sie
euch zu gnädigem Gedenken bei eurem Gott verhelfen: ich bin der HERR,
euer Gott!«

\hypertarget{wanderung-der-israeliten-vom-sinai-bis-zum-lande-der-moabiter-1011-221}{%
\subsubsection{2. Wanderung der Israeliten vom Sinai bis zum Lande der
Moabiter
(10,11-22,1)}\label{wanderung-der-israeliten-vom-sinai-bis-zum-lande-der-moabiter-1011-221}}

\hypertarget{a-aufbruch-vom-sinai-nach-der-wuxfcste-paran}{%
\paragraph{a) Aufbruch vom Sinai nach der Wüste
Paran}\label{a-aufbruch-vom-sinai-nach-der-wuxfcste-paran}}

\bibleverse{11}Im zweiten Jahr, am zwanzigsten Tage des zweiten Monats,
erhob sich die Wolke von der Wohnung des Gesetzes. \bibleverse{12}Da
brachen die Israeliten aus der Wüste Sinai auf, nach ihren
Zügen\textless sup title=``=~ein Lager nach dem andern''\textgreater✲;
und die Wolke ließ sich in der Wüste Paran nieder.

\hypertarget{beschreibung-der-zugordnung}{%
\paragraph{Beschreibung der
Zugordnung}\label{beschreibung-der-zugordnung}}

\bibleverse{13}So brachen sie denn zum erstenmal nach dem durch Mose
übermittelten Befehl des HERRN auf; \bibleverse{14}und zwar brach zuerst
das Panier des Lagers des Stammes Juda auf, eine Heerschar nach der
andern; das Heer dieses Stammes befehligte Nahson, der Sohn Amminadabs.
\bibleverse{15}Das Heer des Stammes Issaschar aber befehligte Nethaneel,
der Sohn Zuars; \bibleverse{16}und das Heer des Stammes Sebulon
befehligte Eliab, der Sohn Helons. \bibleverse{17}Als dann die heilige
Wohnung niedergelegt war, brachen die Gersoniten und die Merariten auf,
welche die (heilige) Wohnung zu tragen hatten. \bibleverse{18}Hierauf
brach das Panier des Lagers Rubens auf, eine Heerschar nach der andern;
das Heer dieses Stammes befehligte Elizur, der Sohn Sedeurs.
\bibleverse{19}Das Heer des Stammes Simeon aber befehligte Selumiel, der
Sohn Zurisaddais; \bibleverse{20}und das Heer des Stammes Gad befehligte
Eljasaph, der Sohn Deguels. \bibleverse{21}Dann brachen die Kehathiten
auf, die das (Hoch-) Heilige zu tragen hatten; man hatte aber die
(heilige) Wohnung bis zu der Ankunft dieser schon aufgerichtet.
\bibleverse{22}Hierauf brach das Panier des Lagers der Ephraimiten auf,
eine Heerschar nach der andern; das Heer dieses Stammes befehligte
Elisama, der Sohn Ammihuds. \bibleverse{23}Das Heer des Stammes Manasse
aber befehligte Gamliel, der Sohn Pedazurs; \bibleverse{24}und das Heer
des Stammes Benjamin befehligte Abidan, der Sohn Gideonis.
\bibleverse{25}Hierauf brach das Panier des Lagers des Stammes Dan auf,
das die Nachhut sämtlicher Lager bildete, eine Heerschar nach der
andern; das Heer dieses Stammes befehligte Ahieser, der Sohn
Ammisaddais. \bibleverse{26}Das Heer des Stammes Asser aber befehligte
Pagiel, der Sohn Ochrans; \bibleverse{27}und das Heer des Stammes
Naphthali befehligte Ahira, der Sohn Enans. \bibleverse{28}Dies war die
Marschordnung, in der die Israeliten aufbrachen, eine Heerschar nach der
andern.

\hypertarget{b-mose-sucht-vergeblich-seinen-schwager-hobab-zum-fuxfchrer-fuxfcr-die-weiterwanderung-zu-gewinnen}{%
\paragraph{b) Mose sucht (vergeblich?) seinen Schwager Hobab zum Führer
für die Weiterwanderung zu
gewinnen}\label{b-mose-sucht-vergeblich-seinen-schwager-hobab-zum-fuxfchrer-fuxfcr-die-weiterwanderung-zu-gewinnen}}

\bibleverse{29}Da sagte Mose zu Hobab, dem Sohn des Midianiters Reguel,
des Schwiegervaters Moses: »Wir brechen jetzt nach dem Lande auf, von
dem der HERR verheißen hat: ›Ich will es euch geben.‹ Ziehe mit uns, wir
wollen es dir gut lohnen; denn der HERR hat Israel Gutes verheißen.«
\bibleverse{30}Der aber antwortete ihm: »Nein, ich mag nicht mitziehen,
sondern will in meine Heimat und zu meiner Verwandtschaft zurückkehren.«
\bibleverse{31}Da bat ihn Mose: »Verlaß uns doch nicht! Denn da gerade
du die Plätze kennst, wo wir in der Wüste lagern können, so sollst du
unser Auge sein\textless sup title=``=~uns als Führer
dienen''\textgreater✲. \bibleverse{32}Wenn du mit uns ziehst und jenes
Glück uns zuteil wird, mit dem der HERR uns segnen will, so wollen wir
es dir gut lohnen.«

\hypertarget{c-der-aufbruch-vom-gottesberge-unter-fuxfchrung-der-bundeslade-die-sogenannten-signalworte}{%
\paragraph{c) Der Aufbruch vom Gottesberge unter Führung der Bundeslade;
die sogenannten
Signalworte}\label{c-der-aufbruch-vom-gottesberge-unter-fuxfchrung-der-bundeslade-die-sogenannten-signalworte}}

\bibleverse{33}So brachen sie denn vom Berge des HERRN auf, drei
Tagereisen weit, indem die Bundeslade des HERRN vor ihnen herzog, drei
Tagereisen weit, um einen Lagerplatz für sie ausfindig zu machen;
\bibleverse{34}dabei stand die Wolke des HERRN bei Tage über ihnen, wenn
sie aus dem Lager aufbrachen. \bibleverse{35}Und sooft die Lade sich in
Bewegung setzte, rief Mose aus: »Erhebe dich, HERR, auf daß deine Feinde
zerstieben und deine Widersacher vor dir fliehen!« \bibleverse{36}Und
sooft sie\textless sup title=``d.h. die Lade''\textgreater✲ Halt machte,
rief er aus: »Kehre zurück, HERR, zu den Zehntausenden\textless sup
title=``oder: Mengen''\textgreater✲ der Tausende Israels!«

\hypertarget{d-mehrfaches-widerstreben-des-volkes-und-seine-uxfcberwindung-durch-strafgerichte-und-heilsame-mauxdfregeln}{%
\paragraph{d) Mehrfaches Widerstreben des Volkes und seine Überwindung
durch Strafgerichte und heilsame
Maßregeln}\label{d-mehrfaches-widerstreben-des-volkes-und-seine-uxfcberwindung-durch-strafgerichte-und-heilsame-mauxdfregeln}}

\hypertarget{aa-das-murren-des-volkes-und-der-lagerbrand-zu-thabera}{%
\subparagraph{aa) Das Murren des Volkes und der Lagerbrand zu
Thabera}\label{aa-das-murren-des-volkes-und-der-lagerbrand-zu-thabera}}

\hypertarget{section-10}{%
\section{11}\label{section-10}}

\bibleverse{1}Da erging sich das Volk in lauten Klagen über sein
Ungemach vor den Ohren des HERRN. Als der HERR es hörte, entbrannte sein
Zorn, und das Feuer des HERRN\textless sup title=``d.h. der
Blitz''\textgreater✲ zündete unter ihnen und richtete am Ende des Lagers
Verheerung an. \bibleverse{2}Da schrie das Volk zu Mose, und dieser
betete zum HERRN: da erlosch das Feuer. \bibleverse{3}Man gab deshalb
diesem Orte den Namen Thabera\textless sup title=``d.h.
Brandstätte''\textgreater✲, weil dort das Feuer des HERRN gegen sie
aufgelodert war.

\hypertarget{bb-die-vorguxe4nge-in-kibroth-hattaawa-ankunft-in-hazeroth}{%
\subparagraph{bb) Die Vorgänge in Kibroth-Hattaawa; Ankunft in
Hazeroth}\label{bb-die-vorguxe4nge-in-kibroth-hattaawa-ankunft-in-hazeroth}}

\bibleverse{4}Das Gesindel aber, das sich unter ihnen befand, wurde
lüstern; da fingen auch die Israeliten wieder an zu jammern und klagten:
»Wer gibt uns Fleisch zu essen? \bibleverse{5}Wir denken an die Fische
zurück, die wir in Ägypten umsonst zu essen hatten, an die Gurken und
Melonen, an den Lauch, die Zwiebeln und den Knoblauch!
\bibleverse{6}Jetzt aber sind wir ganz ausgehungert; gar nichts ist da!
Nichts bekommen wir zu sehen als das Manna!« \bibleverse{7}Das Manna
aber sah aus wie Koriandersamen und glich dem Bedolachharz\textless sup
title=``2.Mose 16,31''\textgreater✲. \bibleverse{8}Das Volk streifte
umher und las es auf; hierauf zermahlten sie es mit Handmühlen oder
zerstießen es in Mörsern, kochten es dann in Töpfen oder bereiteten
Kuchen daraus; es schmeckte wie Ölbackwerk. \bibleverse{9}Wenn der Tau
nachts auf das Lager herabfiel, fiel das Manna mit auf ihn\textless sup
title=``d.h. den Tau''\textgreater✲ herab.

\hypertarget{cc-die-klage-moses-vor-gott}{%
\subparagraph{cc) Die Klage Moses vor
Gott}\label{cc-die-klage-moses-vor-gott}}

\bibleverse{10}Als nun Mose das Volk in allen Familien wehklagen hörte,
einen jeden am Eingang seines Zeltes, und der Zorn des HERRN heftig
entbrannt war, da regte sich der Unwille in Mose, \bibleverse{11}so daß
er zum HERRN sagte: »Warum verfährst du so übel mit deinem Knecht, und
warum nimmst du so wenig Rücksicht auf mich, daß du die Last (der Sorge)
für dieses ganze Volk auf mich legst? \bibleverse{12}Habe ich denn
Mutterpflichten gegen dieses ganze Volk zu erfüllen, oder habe ich es in
die Welt gesetzt, daß du zu mir sagen dürftest: ›Trage es auf deinen
Armen, wie die Wärterin den Säugling trägt, und bringe es in das Land,
das du ihren Vätern zugeschworen hast?‹ \bibleverse{13}Woher soll ich
Fleisch nehmen, um es diesem ganzen Volke zu geben? Mir jammern sie ja
ihre Not vor und rufen: ›Gib uns Fleisch zu essen!‹ \bibleverse{14}Ich
allein vermag die Last (der Sorge) für dieses ganze Volk nicht zu
tragen; sie ist für mich zu schwer! \bibleverse{15}Willst du trotzdem so
mit mir verfahren, so bringe mich doch lieber gleich um, wenn du es gut
mit mir meinst, damit ich mein Unglück nicht länger anzusehen brauche!«

\hypertarget{dd-gottes-anordnung-bestellung-von-siebzig-gehilfen-moses-gottes-verheiuxdfung-der-fleischspende-die-ungluxe4ubige-entgegnung-moses}{%
\subparagraph{dd) Gottes Anordnung (Bestellung von siebzig Gehilfen
Moses); Gottes Verheißung der Fleischspende; die ungläubige Entgegnung
Moses}\label{dd-gottes-anordnung-bestellung-von-siebzig-gehilfen-moses-gottes-verheiuxdfung-der-fleischspende-die-ungluxe4ubige-entgegnung-moses}}

\bibleverse{16}Da antwortete der HERR dem Mose: »Versammle mir siebzig
Männer aus den Ältesten der Israeliten, von denen du weißt, daß sie
wirklich Älteste des Volkes und seine Obmänner\textless sup
title=``oder: Vorsteher''\textgreater✲ sind; führe sie dann zum
Offenbarungszelt und laß sie sich dort neben dir aufstellen.
\bibleverse{17}Ich will dann herabkommen und dort zu dir reden und will
von dem Geist, der auf dir ruht\textless sup title=``oder: in dir
lebt''\textgreater✲, etwas nehmen und es ihnen mitteilen, damit sie im
Verein mit dir die Last (der Sorge) für das Volk tragen und du sie nicht
mehr allein zu tragen brauchst. \bibleverse{18}Zu dem Volke aber sollst
du sagen: ›Heiligt\textless sup title=``vgl. 2.Mose 19,10''\textgreater✲
euch für morgen! Da sollt ihr Fleisch zu essen bekommen; denn ihr habt
vor den Ohren des HERRN gejammert und ausgerufen: Wer gibt uns Fleisch
zu essen? In Ägypten hatten wir es so gut! Darum wird der HERR euch
Fleisch geben, damit ihr zu essen habt. \bibleverse{19}Nicht nur einen
Tag sollt ihr es zu essen haben, auch nicht nur zwei oder fünf oder zehn
oder zwanzig Tage: \bibleverse{20}nein, einen ganzen Monat lang, bis ihr
es nicht mehr riechen könnt und es euch zum Ekel wird! Denn ihr habt den
HERRN, der in eurer Mitte weilt, mißachtet und vor ihm gejammert und
geklagt: Warum sind wir nur aus Ägypten weggezogen!‹« \bibleverse{21}Da
erwiderte Mose: »Sechshunderttausend Mann zu Fuß zählt das Volk, unter
dem ich lebe, und doch sagst du: ›Fleisch will ich ihnen geben, daß sie
einen ganzen Monat lang zu essen haben‹? \bibleverse{22}Können so viele
Stück Kleinvieh und Rinder für sie geschlachtet werden, daß es für sie
ausreicht? Oder sollen alle Fische des Meeres für sie eingefangen
werden, daß es für sie ausreicht?« \bibleverse{23}Da antwortete der HERR
dem Mose: »Ist etwa der Arm des HERRN zu kurz? Jetzt sollst du sehen, ob
mein Wort sich dir erfüllt oder nicht!«

\hypertarget{ee-die-prophetische-begeisterung-der-siebzig-uxe4ltesten}{%
\subparagraph{ee) Die prophetische Begeisterung der siebzig
Ältesten}\label{ee-die-prophetische-begeisterung-der-siebzig-uxe4ltesten}}

\bibleverse{24}Hierauf ging Mose hinaus und teilte dem Volk die Worte
des HERRN mit; dann berief er siebzig Männer aus den Ältesten des Volkes
und ließ sie sich rings um das (heilige) Zelt aufstellen.
\bibleverse{25}Da fuhr der HERR in der Wolke herab und redete zu ihm,
nahm dann etwas von dem Geist, der auf ihm ruhte, und teilte ihn den
siebzig Ältesten zu. Sobald nun der Geist auf sie gekommen war, gerieten
sie in prophetische Begeisterung, später aber nicht wieder.
\bibleverse{26}Es waren aber zwei Männer im Lager zurückgeblieben, von
denen der eine Eldad hieß, der andere Medad; auch auf diese ließ der
Geist sich nieder -- sie gehörten nämlich zu der Zahl der
Aufgeschriebenen, waren aber nicht ans Zelt hinausgegangen --; diese
gerieten nun im Lager in prophetische Begeisterung. \bibleverse{27}Da
kam ein Jüngling gelaufen und meldete dem Mose: »Eldad und Medad sind im
Lager in prophetische Begeisterung geraten!« \bibleverse{28}Da brach
Josua, der Sohn Nuns, der schon von seiner Jünglingszeit an der Diener
Moses gewesen war, in die Worte aus: »O Mose, mein Herr, gebiete ihnen
Einhalt!« \bibleverse{29}Aber Mose entgegnete ihm: »Gerätst du aus Sorge
für mich in solchen Eifer? Möchte doch das ganze Volk des HERRN zu
Propheten werden, daß der HERR seinen Geist auf sie kommen ließe!«
\bibleverse{30}Hierauf begab sich Mose mit den Ältesten der Israeliten
ins Lager zurück.

\hypertarget{ff-speisung-durch-wachteln-gottes-strafgericht-die-lustgruxe4ber}{%
\subparagraph{ff) Speisung durch Wachteln; Gottes Strafgericht; die
Lustgräber}\label{ff-speisung-durch-wachteln-gottes-strafgericht-die-lustgruxe4ber}}

\bibleverse{31}Da erhob sich ein vom HERRN gesandter Wind, der führte
Wachteln vom Meere herüber und ließ sie auf das Lager hineinfallen,
ungefähr eine Tagereise weit nach allen Seiten rings um das Lager, und
sie flogen nur etwa zwei Ellen hoch über der Erde. \bibleverse{32}Da
machte sich das Volk jenen ganzen Tag und die ganze Nacht und den ganzen
folgenden Tag daran und sammelte Wachteln; wer auch nur wenig sammelte,
brachte es doch auf zehn Homer; dann breiteten sie sich diese (zum
Dörren) weithin aus rings um das Lager her. \bibleverse{33}Als sie aber
das Fleisch noch zwischen ihren Zähnen hatten, noch ehe es
verzehrt\textless sup title=``oder: zerkaut?''\textgreater✲ war, da
entbrannte der Zorn des HERRN gegen das Volk, und der HERR ließ ein
verheerendes Sterben unter dem Volke ausbrechen. \bibleverse{34}Daher
gab man diesem Orte den Namen Kibroth Hattaawa\textless sup title=``d.h.
Lustgräber''\textgreater✲, weil man dort die Leute begraben hatte, die
ihrem Gelüst gefrönt hatten.~-- \bibleverse{35}Von den Lustgräbern zog
das Volk dann weiter nach Hazeroth, woselbst sie längere Zeit blieben.

\hypertarget{e-mirjams-und-aarons-auflehnung-gegen-mose}{%
\paragraph{e) Mirjams und Aarons Auflehnung gegen
Mose}\label{e-mirjams-und-aarons-auflehnung-gegen-mose}}

\hypertarget{section-11}{%
\section{12}\label{section-11}}

\bibleverse{1}Mirjam und Aaron aber redeten gegen\textless sup
title=``=~übel von''\textgreater✲ Mose wegen des kuschitischen Weibes,
das er zur Frau genommen hatte; er hatte nämlich eine Kuschitin
geheiratet. \bibleverse{2}Außerdem sagten sie: »Hat der HERR etwa nur
mit Mose geredet? Hat er nicht auch mit uns geredet?« \bibleverse{3}Der
HERR hörte dies; Mose aber war ein überaus sanftmütiger Mann,
sanftmütiger als irgendein anderer Mensch auf der Erde.

\hypertarget{gottes-eintreten-fuxfcr-mose-mirjams-bestrafung}{%
\paragraph{Gottes Eintreten für Mose; Mirjams
Bestrafung}\label{gottes-eintreten-fuxfcr-mose-mirjams-bestrafung}}

\bibleverse{4}Da sagte der HERR sofort zu Mose, zu Aaron und zu Mirjam:
»Begebt euch alle drei zum Offenbarungszelt hinaus!« Als nun die drei
hinausgegangen waren, \bibleverse{5}fuhr der HERR in einer Wolkensäule
herab und trat an den Eingang des Zeltes; als er dann Aaron und Mirjam
gerufen hatte und die beiden hinausgegangen waren, \bibleverse{6}sagte
er: »Hört jetzt meine Worte! Wenn ein Prophet des HERRN unter euch ist,
so offenbare ich mich ihm durch Gesichte und rede zu ihm durch Träume.
\bibleverse{7}So steht es aber nicht bei meinem Knecht Mose; der ist mit
meinem ganzen Hause betraut; \bibleverse{8}von Mund zu Mund rede ich mit
ihm, unzweideutig und nicht in Rätseln, und er darf die Gestalt des
HERRN selbst schauen. Warum habt ihr euch also nicht gescheut, gegen
meinen Knecht, gegen Mose, übel zu reden?« \bibleverse{9}Darauf
entbrannte der Zorn des HERRN gegen sie, und er verschwand.
\bibleverse{10}Als aber die Wolke sich vom Zelt entfernt hatte, war
Mirjam plötzlich vom Aussatz weiß wie Schnee geworden; und als Aaron
sich zu Mirjam hinwandte, sah er, daß sie aussätzig war.

\hypertarget{aarons-und-moses-fuxfcrbitte-gottes-antwort-mirjams-heilung-ankunft-in-der-wuxfcste-paran}{%
\paragraph{Aarons und Moses Fürbitte; Gottes Antwort; Mirjams Heilung;
Ankunft in der Wüste
Paran}\label{aarons-und-moses-fuxfcrbitte-gottes-antwort-mirjams-heilung-ankunft-in-der-wuxfcste-paran}}

\bibleverse{11}Da sagte Aaron zu Mose: »Ach, bitte, mein Herr, laß uns
doch nicht für die Sünde büßen, daß wir in Unbesonnenheit gehandelt und
uns vergangen haben! \bibleverse{12}Laß doch Mirjam nicht wie ein totes
Kind\textless sup title=``=~eine Fehlgeburt''\textgreater✲ sein, dessen
Leib beim Austritt aus dem Mutterschoß schon halb verwest ist!«
\bibleverse{13}Darauf flehte Mose laut zum HERRN mit den Worten: »Ach
Gott! Laß sie doch wieder gesund werden!« \bibleverse{14}Da antwortete
der HERR dem Mose: »Wenn ihr Vater ihr ins Gesicht gespien hätte, müßte
sie sich da nicht sieben Tage lang schämen? Sie soll sieben Tage lang
außerhalb des Lagers eingeschlossen bleiben; alsdann mag sie wieder
Aufnahme im Lager finden.« \bibleverse{15}So wurde denn Mirjam sieben
Tage lang außerhalb des Lagers eingeschlossen; das Volk aber zog nicht
eher weiter, als bis Mirjam wieder ins Lager aufgenommen✲ war.
\bibleverse{16}Danach brach das Volk von Hazeroth auf und lagerte in der
Wüste Paran.

\hypertarget{f-aussendung-der-zwuxf6lf-kundschafter-und-ihr-bericht}{%
\paragraph{f) Aussendung der zwölf Kundschafter und ihr
Bericht}\label{f-aussendung-der-zwuxf6lf-kundschafter-und-ihr-bericht}}

\hypertarget{aa-gottes-auftrag-an-mose-und-auswahl-der-kundschafter}{%
\subparagraph{aa) Gottes Auftrag an Mose und Auswahl der
Kundschafter}\label{aa-gottes-auftrag-an-mose-und-auswahl-der-kundschafter}}

\hypertarget{section-12}{%
\section{13}\label{section-12}}

\bibleverse{1}Darauf gebot der HERR dem Mose folgendes:
\bibleverse{2}»Sende Männer aus, damit sie das Land Kanaan
auskundschaften, das ich den Israeliten geben will; je einen Mann aus
jedem Stamme seiner Väter sollt ihr aussenden, lauter Fürsten unter
ihnen.« \bibleverse{3}So sandte denn Mose sie nach dem Befehl des HERRN
aus der Wüste Paran aus, allesamt Männer, welche Stammeshäupter der
Israeliten waren, \bibleverse{4}und dies sind ihre Namen: vom Stamme
Ruben: Sammua, der Sohn Sakkurs; \bibleverse{5}vom Stamme Simeon:
Saphat, der Sohn Horis; \bibleverse{6}vom Stamme Juda: Kaleb, der Sohn
Jephunnes; \bibleverse{7}vom Stamme Issaschar: Jigal, der Sohn Josephs;
\bibleverse{8}vom Stamme Ephraim: Hosea, der Sohn Nuns;
\bibleverse{9}vom Stamme Benjamin: Palti, der Sohn Raphus;
\bibleverse{10}vom Stamme Sebulon: Gaddiel, der Sohn Sodis;
\bibleverse{11} \bibleverse{12}vom Stamme Dan: Ammiel, der Sohn
Gemallis; \bibleverse{13}vom Stamme Asser: Sethur, der Sohn Michaels;
\bibleverse{14}vom Stamme Naphthali: Nahbi, der Sohn Wophsis;
\bibleverse{15}vom Stamme Gad: Geuel, der Sohn Machis.
\bibleverse{16}Dies sind die Namen der Männer, die Mose entsandte, um
das Land auszukundschaften. Mose hatte aber Hosea, dem Sohne Nuns, den
Namen Josua gegeben.

\hypertarget{bb-die-weisung-moses-an-die-kundschafter}{%
\subparagraph{bb) Die Weisung Moses an die
Kundschafter}\label{bb-die-weisung-moses-an-die-kundschafter}}

\bibleverse{17}Mose entsandte sie also, um das Land Kanaan
auszukundschaften, und befahl ihnen: »Zieht hier hinauf im Südland und
steigt dann auf das Hochland hinauf; \bibleverse{18}seht zu, wie das
Land beschaffen ist und ob das Volk, das darin wohnt, stark oder
schwach, spärlich oder zahlreich ist; \bibleverse{19}auch wie das Land
beschaffen ist, in dem es wohnt, ob fruchtbar oder unfruchtbar, und wie
die Ortschaften beschaffen sind, in denen es wohnt, ob in offenen Lagern
oder in festen Plätzen; \bibleverse{20}auch wie der Boden beschaffen
ist, ob fett oder mager, ob Bäume darauf stehen oder nicht. Zeigt euch
mutig und bringt auch von den Früchten des Landes einige mit!« Es war
aber damals gerade die Zeit der ersten Trauben.

\hypertarget{cc-die-auskundschaftung-des-landes}{%
\subparagraph{cc) Die Auskundschaftung des
Landes}\label{cc-die-auskundschaftung-des-landes}}

\bibleverse{21}So zogen sie denn hinauf und kundschafteten das Land aus
von der Wüste Zin bis Rehob, wo der Weg nach Hamath geht.
\bibleverse{22}Sie zogen also im Südland hinauf und kamen bis Hebron, wo
die Enakssöhne Ahiman, Sesai und Thalmai wohnten; Hebron aber war sieben
Jahre vor der ägyptischen Stadt Zoan gegründet worden.
\bibleverse{23}Als sie dann ins Tal Eskol✲ gekommen waren, schnitten sie
dort eine Rebe mit einer einzigen Weintraube ab, die sie zu zweien an
einer Stange trugen, auch einige Granatäpfel und Feigen.
\bibleverse{24}Die betreffende Örtlichkeit nannte man Tal
Eskol\textless sup title=``d.h. Traubental''\textgreater✲ wegen der
Traube, welche die Israeliten dort abgeschnitten hatten.

\hypertarget{dd-ruxfcckkehr-und-bericht-der-ausgesandten}{%
\subparagraph{dd) Rückkehr und Bericht der
Ausgesandten}\label{dd-ruxfcckkehr-und-bericht-der-ausgesandten}}

\bibleverse{25}Nach Verlauf von vierzig Tagen machten sie sich dann auf
den Rückweg, nachdem sie das Land ausgekundschaftet hatten.
\bibleverse{26}Als sie nun zu Mose und Aaron und zur ganzen Gemeinde der
Israeliten in die Wüste Paran nach Kades zurückgekehrt waren,
erstatteten sie ihnen und der ganzen Gemeinde Bericht und zeigten ihnen
die Früchte des Landes. \bibleverse{27}Sie trugen ihm aber folgenden
Bericht vor: »Wir haben uns in das Land begeben, in das du uns gesandt
hast: es fließt wirklich von Milch und Honig über, und dies hier sind
Früchte von dort. \bibleverse{28}Jedoch das Volk, das im Lande wohnt,
ist stark, und die Städte sind befestigt und sehr groß; auch die
Enakssöhne haben wir dort gesehen. \bibleverse{29}Die Amalekiter
bewohnen das Südland, die Hethiter, Jebusiter und Amoriter wohnen im
Berglande, und die Kanaanäer wohnen am Meer und an der
Seite\textless sup title=``oder: am Ufer''\textgreater✲ des Jordans.«

\hypertarget{ee-kalebs-beschwichtigende-und-der-uxfcbrigen-kundschafter-entmutigende-worte}{%
\subparagraph{ee) Kalebs beschwichtigende und der übrigen Kundschafter
entmutigende
Worte}\label{ee-kalebs-beschwichtigende-und-der-uxfcbrigen-kundschafter-entmutigende-worte}}

\bibleverse{30}Kaleb suchte nun den Unwillen des Volkes gegen Mose zu
beschwichtigen, indem er ausrief: »Laßt uns nur hinaufziehen und (das
Land) in Besitz nehmen! Denn wir können es sicherlich überwältigen.«
\bibleverse{31}Jedoch die Männer, die mit ihm hinaufgezogen waren,
erklärten: »Wir sind nicht imstande, gegen das Volk hinaufzuziehen; denn
es ist uns zu stark.« \bibleverse{32}Dann entwarfen sie den Israeliten
eine schlimme Schilderung von dem Lande, das sie ausgekundschaftet
hatten, mit den Worten: »Das Land, das wir durchzogen haben, um es
auszukundschaften, ist ein Land, das seine Bewohner frißt; und alles
Volk, das wir darin gesehen haben, sind hochgewachsene Leute;
\bibleverse{33}auch die Riesen haben wir dort gesehen, die Enakssöhne
vom Geschlecht der Riesen; wir kamen uns selbst gegen sie wie
Heuschrecken vor, und ebenso erschienen wir ihnen.«

\hypertarget{g-murren-und-aufruhr-des-volkes-strafgericht-gottes-verurteilung-des-volkes-zu-langer-wuxfcstenzeit}{%
\paragraph{g) Murren und Aufruhr des Volkes; Strafgericht Gottes;
Verurteilung des Volkes zu langer
Wüstenzeit}\label{g-murren-und-aufruhr-des-volkes-strafgericht-gottes-verurteilung-des-volkes-zu-langer-wuxfcstenzeit}}

\hypertarget{aa-die-wirkung-des-berichts-auf-das-volk}{%
\subparagraph{aa) Die Wirkung des Berichts auf das
Volk}\label{aa-die-wirkung-des-berichts-auf-das-volk}}

\hypertarget{section-13}{%
\section{14}\label{section-13}}

\bibleverse{1}Da erhob die ganze Gemeinde ein lautes Geschrei, und das
Volk weinte in jener Nacht; \bibleverse{2}alle Israeliten murrten über
Mose und Aaron, und die ganze Gemeinde klagte vor ihnen: »Ach wären wir
doch im Lande Ägypten gestorben! Oder wären wir doch hier in der Wüste
gestorben! \bibleverse{3}Warum führt uns der HERR in dieses Land? Damit
wir durch das Schwert fallen? Damit unsere Frauen und kleinen Kinder
(den Feinden) zur Beute werden? Ist es nicht das Beste für uns, wir
kehren nach Ägypten zurück?« \bibleverse{4}Und sie sagten einer zum
andern: »Wir wollen uns einen Anführer wählen und nach Ägypten
zurückkehren!« \bibleverse{5}Da warfen sich Mose und Aaron vor der
ganzen versammelten Gemeinde der Israeliten auf ihr Angesicht nieder.

\hypertarget{bb-josuas-und-kalebs-vergeblicher-beschwichtigungsversuch}{%
\subparagraph{bb) Josuas und Kalebs vergeblicher
Beschwichtigungsversuch}\label{bb-josuas-und-kalebs-vergeblicher-beschwichtigungsversuch}}

\bibleverse{6}Josua aber, der Sohn Nuns, und Kaleb, der Sohn Jephunnes,
die zu denen gehörten, welche das Land ausgekundschaftet hatten,
zerrissen ihre Kleider \bibleverse{7}und gaben vor der ganzen Gemeinde
der Israeliten folgende Erklärung ab: »Das Land, das wir als
Kundschafter durchzogen haben, ist ein außerordentlich schönes Land!
\bibleverse{8}Wenn der HERR uns wohlwill, so wird er uns schon in dieses
Land bringen und es uns geben, ein Land, das von Milch und Honig
überfließt. \bibleverse{9}Nur empört euch nicht gegen den HERRN und
fürchtet euch ja nicht vor den Bewohnern des Landes! Denn wie einen
Bissen Brot werden wir sie verspeisen. Der Schutz ihrer Götter ist von
ihnen gewichen, aber mit uns ist der HERR: fürchtet euch nicht vor
ihnen!«

\hypertarget{cc-gottes-zorn-die-erfolgreiche-fuxfcrbitte-moses-das-guxf6ttliche-strafurteil}{%
\subparagraph{cc) Gottes Zorn; die erfolgreiche Fürbitte Moses; das
göttliche
Strafurteil}\label{cc-gottes-zorn-die-erfolgreiche-fuxfcrbitte-moses-das-guxf6ttliche-strafurteil}}

\bibleverse{10}Als nun die ganze Gemeinde schon daran dachte, sie zu
steinigen, erschien die Herrlichkeit des HERRN allen Israeliten am
Offenbarungszelt; \bibleverse{11}und der HERR sagte zu Mose: »Wie lange
will dieses Volk mich noch verhöhnen und wie lange noch mir kein
Vertrauen schenken trotz aller Wunderzeichen, die ich unter ihnen getan
habe? \bibleverse{12}Ich will sie mit der Pest schlagen und sie
ausrotten; dich aber will ich zu einem Volk machen, das größer und
stärker ist als sie!«

\bibleverse{13}Da erwiderte Mose dem HERRN: »Die Ägypter haben es
vernommen, daß du dieses Volk mit deiner Kraft aus ihrer Mitte
hergeführt hast, \bibleverse{14}und man wird es auch den Bewohnern
dieses Landes sagen, welche gehört haben, daß du, HERR, inmitten dieses
Volkes weilst, daß du dich ihm Auge in Auge offenbarst und daß deine
Wolke über ihnen steht und du in einer Wolkensäule bei Tage und in einer
Feuersäule bei Nacht vor ihnen herziehst. \bibleverse{15}Wenn du nun
dieses Volk wie einen Mann tötest, so werden die Völkerschaften, welche
die Kunde von dir vernommen haben, laut aussprechen:
\bibleverse{16}›Weil der HERR nicht imstande war, dieses Volk in das
Land zu bringen, das er ihnen zugeschworen hatte, darum hat er sie in
der Wüste abgeschlachtet!‹ \bibleverse{17}So möge sich denn jetzt deine
Kraft, o HERR, groß erweisen, wie du es verheißen hast, als du
sagtest\textless sup title=``2.Mose 34,6-7''\textgreater✲:
\bibleverse{18}›Der HERR ist langmütig und reich an Gnade; er vergibt
Unrecht und Übertretung, läßt aber auch den Schuldigen keineswegs
ungestraft, sondern sucht die Schuld der Väter an den Kindern heim bis
ins dritte und vierte Glied.‹ \bibleverse{19}So vergib doch diesem Volk
seine Schuld nach deiner großen Gnade und so, wie du diesem Volke von
Ägypten an bis hierher verziehen hast!« \bibleverse{20}Da antwortete der
HERR: »Ich habe ihm vergeben, wie du es erbeten hast.
\bibleverse{21}Aber wahrlich, so wahr ich lebe und so wahr die ganze
Erde von der Herrlichkeit des HERRN erfüllt werden soll:
\bibleverse{22}alle die Männer, die meine Herrlichkeit und meine
Wunderzeichen, die ich in Ägypten und in der Wüste getan habe, gesehen
und mich trotzdem nun schon zehnmal versucht und nicht auf meine
Weisungen gehört haben~-- \bibleverse{23}sie sollen das Land, das ich
ihren Vätern zugeschworen habe, nicht zu sehen bekommen! Nein, keiner
von allen, die mich verhöhnt haben, soll es zu sehen bekommen!
\bibleverse{24}Nur meinen Knecht Kaleb, der einen anderen Geist gezeigt
und mir vollen Gehorsam bewiesen hat, den will ich in das Land bringen,
das er schon einmal betreten hat, und seine Nachkommen sollen es
besitzen; \bibleverse{25}die Amalekiter aber und die Kanaanäer bleiben
in der Niederung wohnen! Morgen kehrt um und brecht nach der Wüste auf
in der Richtung nach dem Schilfmeer zu!«

\hypertarget{dd-die-strafe-gottes-fuxfcr-volk-und-kundschafter-noch-genauer-angegeben}{%
\subparagraph{dd) Die Strafe Gottes für Volk und Kundschafter noch
genauer
angegeben}\label{dd-die-strafe-gottes-fuxfcr-volk-und-kundschafter-noch-genauer-angegeben}}

\bibleverse{26}Hierauf sagte der HERR zu Mose und Aaron:
\bibleverse{27}»Wie lange soll es noch dauern, daß diese nichtswürdige
Gemeinde gegen mich murrt? Ich habe das Murren wohl gehört, das die
Israeliten gegen mich erheben. \bibleverse{28}Sage ihnen: ›So wahr ich
lebe!‹ -- so lautet der Ausspruch des HERRN --: ›Ich will es euch genau
so ergehen lassen, wie ihr es laut vor meinen Ohren ausgesprochen habt!
\bibleverse{29}In der Wüste hier sollen eure Leiber tot hinfallen, und
zwar ihr alle, die gemustert worden sind, ihr alle vollzählig von
zwanzig Jahren an und darüber, die ihr gegen mich gemurrt habt.
\bibleverse{30}Ihr sollt nimmermehr in das Land kommen, das ich euch
durch einen Eid zum Wohnsitz zugesagt habe, keiner außer Kaleb, dem
Sohne Jephunnes, und Josua, dem Sohne Nuns. \bibleverse{31}Eure kleinen
Kinder aber, von denen ihr gesagt habt, sie würden (den Feinden) zur
Beute werden, die will ich hineinbringen, und sie sollen das Land kennen
lernen, das ihr verschmäht habt; \bibleverse{32}eure eigenen Leiber aber
sollen in der Wüste hier tot hinfallen, \bibleverse{33}und eure Söhne
sollen vierzig Jahre lang als Hirten in der Wüste umherziehen und für
euren treulosen Abfall von mir büßen, bis eure Leiber in der Wüste
aufgerieben sind. \bibleverse{34}Nach der Zahl der vierzig Tage, in
denen ihr das Land ausgekundschaftet habt -- immer ein Tag für ein Jahr
gerechnet --, sollt ihr vierzig Jahre lang für eure Verschuldungen büßen
und sollt erfahren, was es auf sich hat, wenn ich mich von euch abwende!
\bibleverse{35}Ich, der HERR, habe es ausgesprochen! Wahrlich, so will
ich mit dieser ganzen nichtswürdigen Gemeinde verfahren, die sich gegen
mich zusammengerottet hat: in der Wüste hier sollen sie aufgerieben
werden, und hier sollen sie sterben!‹«

\hypertarget{ee-pluxf6tzlicher-tod-der-kundschafter-auuxdfer-josua-und-kaleb}{%
\subparagraph{ee) Plötzlicher Tod der Kundschafter außer Josua und
Kaleb}\label{ee-pluxf6tzlicher-tod-der-kundschafter-auuxdfer-josua-und-kaleb}}

\bibleverse{36}Die Männer aber, die Mose zur Auskundschaftung des Landes
ausgesandt hatte und die nach ihrer Rückkehr die ganze Gemeinde zum
Murren gegen ihn durch ihren ungünstigen Bericht über das Land verleitet
hatten~-- \bibleverse{37}diese Männer, die über das Land ungünstig
berichtet hatten, starben eines plötzlichen Todes vor dem HERRN.
\bibleverse{38}Nur Josua, der Sohn Nuns, und Kaleb, der Sohn Jephunnes,
blieben am Leben von jenen Männern, die zur Auskundschaftung des Landes
ausgezogen waren.

\hypertarget{ff-die-reue-des-volkes-der-verungluxfcckte-versuch-in-das-feindliche-land-einzudringen}{%
\subparagraph{ff) Die Reue des Volkes; der verunglückte Versuch, in das
feindliche Land
einzudringen}\label{ff-die-reue-des-volkes-der-verungluxfcckte-versuch-in-das-feindliche-land-einzudringen}}

\bibleverse{39}Als nun Mose diese Drohworte des HERRN allen Israeliten
mitgeteilt hatte, da geriet das Volk in tiefe Betrübnis.
\bibleverse{40}Sie machten sich also am andern Morgen in der Frühe
fertig, auf die Höhe des Gebirges hinaufzuziehen, und sagten: »Wir sind
jetzt bereit, in die Gegend hinaufzuziehen, von welcher der HERR geredet
hat; denn wir haben gesündigt.« \bibleverse{41}Aber Mose antwortete:
»Warum wollt ihr doch den Befehl des HERRN übertreten? Das wird euch
nicht gelingen! \bibleverse{42}Zieht ja nicht hinauf, denn der HERR ist
nicht in eurer Mitte; ihr werdet sonst von euren Feinden geschlagen
werden; \bibleverse{43}denn die Amalekiter und Kanaanäer stehen euch
dort gegenüber, und ihr werdet durch das Schwert fallen: durch euren
Abfall vom HERRN habt ihr nun einmal verschuldet, daß der HERR nicht mit
euch sein wird!« \bibleverse{44}Trotzdem blieben sie in ihrer
Vermessenheit dabei, auf die Höhe des Gebirges hinaufzuziehen; doch die
Lade mit dem Bundesgesetz des HERRN und Mose verließen das Innere des
Lagers nicht. \bibleverse{45}Da kamen die Amalekiter und Kanaanäer, die
in jenem Gebirge wohnten, herab, schlugen sie und zersprengten sie bis
Horma.

\hypertarget{h-verschiedene-verordnungen-aus-der-zeit-der-wuxfcstenwanderung}{%
\paragraph{h) Verschiedene Verordnungen aus der Zeit der
Wüstenwanderung}\label{h-verschiedene-verordnungen-aus-der-zeit-der-wuxfcstenwanderung}}

\hypertarget{aa-vorschriften-bezuxfcglich-der-speis--und-trankopfer-als-beigabe-zu-den-brand--und-heilsopfern}{%
\subparagraph{aa) Vorschriften bezüglich der Speis- und Trankopfer als
Beigabe zu den Brand- und
Heilsopfern}\label{aa-vorschriften-bezuxfcglich-der-speis--und-trankopfer-als-beigabe-zu-den-brand--und-heilsopfern}}

\hypertarget{section-14}{%
\section{15}\label{section-14}}

\bibleverse{1}Weiter gebot der HERR dem Mose folgendes:
\bibleverse{2}»Teile den Israeliten folgende Verordnungen mit: Wenn ihr
in das Land kommt, das ich euch zum Wohnsitz geben will,
\bibleverse{3}und ihr dem HERRN ein Feueropfer darbringen wollt, ein
Brandopfer oder ein Schlachtopfer, sei es, um ein besonderes Gelübde zu
erfüllen oder bei einer freiwilligen Gabe oder an euren Festen, um dem
HERRN einen lieblichen Duft zu bereiten, von den Rindern oder vom
Kleinvieh: \bibleverse{4}so soll der, welcher dem HERRN seine Opfergabe
darbringt, zugleich auch als Speisopfer ein Zehntel Epha Feinmehl
darbringen, das mit einem Viertel Hin Öl gemengt ist; \bibleverse{5}und
an Wein zum Trankopfer sollst du neben dem Brandopfer oder zu dem
Schlachtopfer noch ein Viertel Hin für jedes Lamm herrichten.
\bibleverse{6}Für einen Widder dagegen sollst du als Speisopfer zwei
Zehntel Epha Feinmehl, das mit einem Drittel Hin Öl gemengt ist,
herrichten \bibleverse{7}und an Wein zum Trankopfer ein Drittel Hin als
lieblichen Duft für den HERRN darbringen. \bibleverse{8}Wenn du ferner
ein junges Rind als Brandopfer oder als Schlachtopfer herrichten willst,
um ein besonderes Gelübde zu erfüllen oder als Heilsopfer für den HERRN,
\bibleverse{9}so soll man neben dem jungen Rind noch drei Zehntel Epha
Feinmehl, das mit einem halben Hin Öl gemengt ist, als Speisopfer
darbringen; \bibleverse{10}und an Wein sollst du als Trankopfer ein
halbes Hin beim Feueropfer zu lieblichem Duft für den HERRN darbringen.
\bibleverse{11}So soll es bei jedem Rind sowie bei jedem Widder und bei
jedem Schaf- oder Ziegenlamm gehalten werden: \bibleverse{12}nach der
Zahl (der Tiere), die ihr darbringt, sollt ihr bei jedem einzelnen Stück
so verfahren. \bibleverse{13}Jeder Einheimische soll hierbei in dieser
Weise verfahren, wenn er dem HERRN ein Feueropfer zu lieblichem Geruch
darbringt; \bibleverse{14}und wenn sich ein Fremdling als Gast bei euch
aufhält oder wenn sonst einmal jemand in eurer Mitte weilt und ein
Feueropfer zu lieblichem Geruch für den HERRN herrichtet, so soll er
ebenso verfahren, wie ihr es tut. \bibleverse{15}Für die ganze
Bevölkerung gilt eine und dieselbe Satzung, für euch wie für den
Fremdling, der sich als Gast bei euch aufhält; eine ewiggültige Satzung
soll es für euch von Geschlecht zu Geschlecht sein: vor dem
HERRN\textless sup title=``d.h. beim Opfer''\textgreater✲ soll der
Fremdling ebenso dastehen wie ihr selbst. \bibleverse{16}Das gleiche
Gesetz und das gleiche Recht soll für euch und für den Fremdling gelten,
der sich als Gast bei euch aufhält.«

\hypertarget{bb-bestimmung-uxfcber-die-erstlingskuchen}{%
\subparagraph{bb) Bestimmung über die
Erstlingskuchen}\label{bb-bestimmung-uxfcber-die-erstlingskuchen}}

\bibleverse{17}Weiter gebot der HERR dem Mose folgendes:
\bibleverse{18}»Teile den Israeliten folgende Verordnungen mit: Wenn ihr
in das Land kommt, in das ich euch bringen werde, \bibleverse{19}so
sollt ihr, wenn ihr von dem Brotkorn des Landes eßt, für den HERRN ein
Hebeopfer abgeben\textless sup title=``oder: erheben''\textgreater✲.
\bibleverse{20}Als Erstling eures Schrotmehls sollt ihr einen Kuchen als
Hebeopfer abgeben\textless sup title=``oder: erheben''\textgreater✲; wie
die Hebe von der Tenne, ebenso sollt ihr auch diese abgeben.
\bibleverse{21}Von den Erstlingen eures Schrotmehls sollt ihr dem HERRN
ein Hebeopfer abgeben in allen euren künftigen Geschlechtern.«

\hypertarget{cc-vorschriften-bezuxfcglich-der-suxfcndopfer-fuxfcr-unwissentliche-verfehlungen-unsuxfchnbarkeit-vorsuxe4tzlicher-uxfcbertretungen}{%
\subparagraph{cc) Vorschriften bezüglich der Sündopfer für
unwissentliche Verfehlungen; Unsühnbarkeit vorsätzlicher
Übertretungen}\label{cc-vorschriften-bezuxfcglich-der-suxfcndopfer-fuxfcr-unwissentliche-verfehlungen-unsuxfchnbarkeit-vorsuxe4tzlicher-uxfcbertretungen}}

\bibleverse{22}»Und wenn ihr euch unabsichtlich\textless sup
title=``oder: aus Versehen''\textgreater✲ vergeht und irgendeines von
diesen Geboten, die der HERR dem Mose aufgetragen hat, unbefolgt laßt,
\bibleverse{23}irgend etwas von dem, was der HERR euch durch Mose
geboten hat seit dem Tage, wo der HERR euch Gebote gegeben hat, und
weiterhin von Geschlecht zu Geschlecht, \bibleverse{24}so soll, wenn das
Vergehen unabsichtlich von der Gemeinde in Übereilung begangen ist, die
ganze Gemeinde einen jungen Stier als Brandopfer zu lieblichem Geruch
für den HERRN samt dem zugehörigen Speisopfer und dem erforderlichen
Trankopfer nach der vorgeschriebenen Weise herrichten, außerdem einen
Ziegenbock als Sündopfer. \bibleverse{25}Wenn der Priester so der ganzen
Gemeinde der Israeliten Sühne erwirkt hat, so wird ihnen Vergebung
zuteil werden; denn es hat nur ein Versehen stattgefunden, und sie haben
dem HERRN ihre Opfergabe in Gestalt eines Feueropfers, dazu auch ihr
Sündopfer vor dem HERRN wegen ihres Versehens dargebracht.
\bibleverse{26}So wird dann der ganzen Gemeinde der Israeliten sowie den
Fremdlingen, die sich als Gäste bei ihnen aufhalten, Vergebung zuteil
werden; denn es lag ein Versehen des ganzen Volkes vor.

\bibleverse{27}Wenn sich aber ein einzelner unabsichtlich verfehlt, so
soll er eine einjährige Ziege als Sündopfer darbringen.
\bibleverse{28}Wenn dann der Priester einem solchen, der sich
unvorsätzlich ein Vergehen gegen den HERRN hat zuschulden kommen lassen,
Sühne durch Vollziehung der Sühnegebräuche erwirkt hat, so wird ihm
Vergebung zuteil werden. \bibleverse{29}Für den Einheimischen unter den
Israeliten und für den als Gast im Lande weilenden Fremdling soll die
gleiche Bestimmung bei euch gelten, wenn sich jemand unvorsätzlich ein
Vergehen zuschulden kommen läßt. \bibleverse{30}Wenn sich aber jemand
vorsätzlich vergeht, mag es ein Einheimischer oder ein Fremdling sein,
ein solcher gilt als Gotteslästerer, und der Betreffende soll aus seinem
Volk ausgerottet werden; \bibleverse{31}denn er hat das Wort des HERRN
verachtet und sein Gebot übertreten; ein solcher Mensch soll ohne Gnade
ausgerottet werden: seine Schuld komme über ihn\textless sup
title=``oder: lastet auf ihm''\textgreater✲!«

\hypertarget{dd-bericht-uxfcber-die-steinigung-eines-sabbatschuxe4nders}{%
\subparagraph{dd) Bericht über die Steinigung eines
Sabbatschänders}\label{dd-bericht-uxfcber-die-steinigung-eines-sabbatschuxe4nders}}

\bibleverse{32}Als die Israeliten sich in der Wüste befanden, trafen sie
einen Mann, der am Sabbattage Holz auflas. \bibleverse{33}Da brachten
die, welche ihn beim Holzlesen angetroffen hatten, ihn zu Mose und Aaron
und zu der ganzen Gemeinde, \bibleverse{34}und man legte ihn in
Gewahrsam; denn es lag noch keine Bestimmung darüber vor, was mit ihm
geschehen solle. \bibleverse{35}Da gebot der HERR dem Mose: »Der Mann
soll unbedingt mit dem Tode bestraft werden: die ganze Gemeinde soll ihn
außerhalb des Lagers steinigen!« \bibleverse{36}So führte ihn denn die
ganze Gemeinde vor das Lager hinaus, und man warf ihn mit Steinen tot,
wie der HERR dem Mose geboten hatte.

\hypertarget{ee-verordnung-uxfcber-die-an-den-kleiderzipfeln-anzubringenden-quasten}{%
\subparagraph{ee) Verordnung über die an den Kleiderzipfeln
anzubringenden
Quasten}\label{ee-verordnung-uxfcber-die-an-den-kleiderzipfeln-anzubringenden-quasten}}

\bibleverse{37}Weiter gebot der HERR dem Mose folgendes:
\bibleverse{38}»Gib den Israeliten die Weisung, daß sie sich Quasten an
die Zipfel ihrer Obergewänder setzen, sie und ihre kommenden
Geschlechter, und daß sie an jeder Zipfelquaste eine Schnur von blauem
Purpur anbringen. \bibleverse{39}Die Quasten sollen euch dann dazu
dienen, daß ihr bei ihrem Anblick aller Gebote des HERRN gedenkt, um
nach ihnen zu tun und nicht von mir abzufallen nach den Gelüsten eures
Herzens und eurer Augen, durch die ihr euch zum Treubruch verführen
laßt. \bibleverse{40}Ihr sollt vielmehr aller meiner Gebote eingedenk
bleiben und nach ihnen tun und so eurem Gott geheiligt sein:
\bibleverse{41}ich bin der HERR, euer Gott, der euch aus dem Land
Ägypten weggeführt hat, um euer Gott zu sein, ich, der HERR, euer Gott!«

\hypertarget{i-aufruhr-und-bestrafung-des-leviten-korah-und-der-rubeniten-dathan-und-abiram}{%
\paragraph{i) Aufruhr und Bestrafung des Leviten Korah und der Rubeniten
Dathan und
Abiram}\label{i-aufruhr-und-bestrafung-des-leviten-korah-und-der-rubeniten-dathan-und-abiram}}

\hypertarget{aa-die-empuxf6rung-korahs-und-der-rubeniten}{%
\subparagraph{aa) Die Empörung Korahs und der
Rubeniten}\label{aa-die-empuxf6rung-korahs-und-der-rubeniten}}

\hypertarget{section-15}{%
\section{16}\label{section-15}}

\bibleverse{1}Es empörte sich aber Korah, der Sohn Jizhars, des Sohnes
Kahaths, des Sohnes Levis, und mit ihm Dathan und Abiram, die Söhne
Eliabs, des Sohnes Pallus, des Sohnes Rubens. \bibleverse{2}Diese
empörten sich gegen Mose und mit ihnen zweihundertundfünfzig Israeliten,
welche Vorsteher\textless sup title=``oder: Fürsten''\textgreater✲ der
Gemeinde, aus der Gemeindeversammlung berufene\textless sup title=``vgl.
1,16''\textgreater✲ und hochangesehene Männer waren. \bibleverse{3}Sie
rotteten sich also gegen Mose und Aaron zusammen und sagten zu ihnen:
»Ihr beansprucht für euch zu viel; denn die ganze Gemeinde, alle ohne
Ausnahme sind heilig\textless sup title=``oder: geweiht''\textgreater✲,
weil der HERR in ihrer Mitte weilt: warum erhebt ihr euch da über die
Gemeinde des HERRN?«

\hypertarget{bb-mose-tritt-der-rotte-korahs-entgegen-und-kuxfcndigt-ihnen-ein-gottesurteil-am-heiligtum-an}{%
\subparagraph{bb) Mose tritt der Rotte Korahs entgegen und kündigt ihnen
ein Gottesurteil am Heiligtum
an}\label{bb-mose-tritt-der-rotte-korahs-entgegen-und-kuxfcndigt-ihnen-ein-gottesurteil-am-heiligtum-an}}

\bibleverse{4}Als Mose das hörte, warf er sich auf sein Angesicht
nieder; \bibleverse{5}dann sagte er zu Korah und seinem ganzen Anhang:
»Morgen, da wird der HERR kundtun, wer ihm angehört und wer geweiht ist,
so daß er ihn zu sich nahen läßt; wen er sich dann erwählt, den wird er
zu sich nahen lassen. \bibleverse{6}Tut folgendes: Nehmt euch
Räucherpfannen, du, Korah, und ihr, sein ganzer Anhang,
\bibleverse{7}tut morgen Feuer\textless sup title=``=~feurige
Kohlen''\textgreater✲ hinein und legt vor dem HERRN Räucherwerk darauf;
wen dann der HERR erwählt, der soll als geweiht gelten! Damit gebt euch
zufrieden, ihr Söhne Levis!« \bibleverse{8}Mose sagte dann weiter zu
Korah: »Hört doch, ihr Söhne Levis! \bibleverse{9}Genügt es euch nicht,
daß der Gott Israels euch aus der Gemeinde der Israeliten ausgesondert
hat, um euch zu sich nahen zu lassen, damit ihr den Dienst an der
Wohnung des HERRN verrichtet und im Dienst der Gemeinde dieses Amt
verwaltet? \bibleverse{10}Dich und alle deine Brüder mit dir, die
Leviten, hat er zu sich nahen lassen, und nun verlangt ihr auch noch die
Priesterwürde? \bibleverse{11}Somit rottet ihr euch gegen den HERRN
zusammen, du und dein ganzer Anhang; denn was ist Aaron, daß ihr gegen
ihn murrt?«

\hypertarget{cc-dathan-und-abiram-setzen-der-aufforderung-moses-hohn-entgegen-moses-gebet-zu-gott}{%
\subparagraph{cc) Dathan und Abiram setzen der Aufforderung Moses Hohn
entgegen; Moses Gebet zu
Gott}\label{cc-dathan-und-abiram-setzen-der-aufforderung-moses-hohn-entgegen-moses-gebet-zu-gott}}

\bibleverse{12}Hierauf ließ Mose Dathan und Abiram, die Söhne Eliabs,
durch Boten rufen; aber sie ließen sagen: »Wir kommen nicht zu dir
hinauf! \bibleverse{13}Ist es nicht genug, daß du uns aus einem Lande,
das von Milch und Honig überfließt, hierher geführt hast, um uns in der
Wüste sterben zu lassen? Willst du dich gar noch zum Herrscher über uns
aufwerfen? \bibleverse{14}Du hast uns wahrlich nicht in ein Land
gebracht, das von Milch und Honig überfließt, und uns keine Äcker und
Weinberge zum Erbbesitz gegeben! Willst du etwa den Leuten hier Sand in
die Augen streuen? Wir kommen nicht zu dir hinauf!«
\bibleverse{15}Darüber geriet Mose in heftigen Zorn und betete zum
HERRN: »Wende dich nicht zu ihrer Opfergabe! Keinem von ihnen habe ich
auch nur einen Esel genommen und keinem von ihnen etwas zuleide getan!«

\hypertarget{dd-mose-ladet-korah-und-seine-genossen-zur-opfervollziehung-vor-gottes-herrlichkeitserscheinung-fuxfcrbitte-moses}{%
\subparagraph{dd) Mose ladet Korah und seine Genossen zur
Opfervollziehung vor; Gottes Herrlichkeitserscheinung; Fürbitte
Moses}\label{dd-mose-ladet-korah-und-seine-genossen-zur-opfervollziehung-vor-gottes-herrlichkeitserscheinung-fuxfcrbitte-moses}}

\bibleverse{16}Hierauf sagte Mose zu Korah: »Du und dein ganzer Anhang,
ihr mögt morgen vor dem HERRN erscheinen, du und sie und auch Aaron.
\bibleverse{17}Nehmt dann ein jeder seine Räucherpfanne und legt
Räucherwerk darauf; dann bringe ein jeder seine Räucherpfanne vor den
HERRN, zweihundertundfünfzig Räucherpfannen, auch du und Aaron, ein
jeder seine Räucherpfanne!« \bibleverse{18}So nahmen sie denn ein jeder
seine Räucherpfanne, taten feurige Kohlen hinein und legten Räucherwerk
darauf; dann traten sie an den Eingang des Offenbarungszeltes, auch Mose
und Aaron. \bibleverse{19}Korah hatte aber die ganze Gemeinde am Eingang
des Offenbarungszeltes gegen sie versammelt. Da erschien der ganzen
Gemeinde die Herrlichkeit des HERRN; \bibleverse{20}und der HERR gab dem
Mose und Aaron folgende Weisung: \bibleverse{21}»Sondert euch von dieser
Gemeinde ab: ich will sie in einem Augenblick vertilgen!«
\bibleverse{22}Da warfen sie sich (beide) auf ihr Angesicht nieder und
beteten: »O Gott, du Herr über Leben und Tod alles Fleisches! Willst du
denn, wenn ein einzelner Mann gesündigt hat, der ganzen Gemeinde
zürnen?« \bibleverse{23}Da gebot der HERR dem Mose:
\bibleverse{24}»Befiehl der Gemeinde, sich aus der Umgebung der Wohnung
Korahs, Dathans und Abirams zu entfernen!«

\hypertarget{ee-mose-bei-dathan-und-abiram-das-gottesgericht-an-ihnen-sowie-an-den-250-genossen-korahs}{%
\subparagraph{ee) Mose bei Dathan und Abiram; das Gottesgericht an ihnen
sowie an den 250 Genossen
Korahs}\label{ee-mose-bei-dathan-und-abiram-das-gottesgericht-an-ihnen-sowie-an-den-250-genossen-korahs}}

\bibleverse{25}Nun erhob sich Mose und begab sich zu Dathan und Abiram,
und es folgten ihm die Ältesten der Israeliten. \bibleverse{26}Dann gab
er der Gemeinde die Weisung: »Entfernt euch ja von den Zelten dieser
gottlosen Männer und rührt nichts von dem an, was ihnen gehört, damit
ihr nicht auch hinweggerafft werdet um all ihrer Sünden willen!«
\bibleverse{27}Da entfernten sie sich aus der Umgebung der Wohnung
Korahs, Dathans und Abirams. Dathan und Abiram aber waren herausgetreten
und standen mit ihren Frauen und ihren großen und kleinen Kindern im
Eingang ihrer Zelte. \bibleverse{28}Da sagte Mose: »Daran sollt ihr
erkennen, daß der HERR es ist, der mich gesandt hat, um alle diese Taten
zu vollbringen, und daß ich nicht nach eigenem Ermessen gehandelt habe:
\bibleverse{29}wenn diese hier so sterben, wie alle anderen Menschen
sterben, und von dem gewöhnlichen Schicksal der Menschen betroffen
werden, dann hat der HERR mich nicht gesandt; \bibleverse{30}wenn aber
der HERR etwas noch nie Vorgekommenes geschehen läßt und der Erdboden
seinen Mund auftut und sie mit allem, was ihnen gehört, verschlingt, so
daß sie lebendig in das Totenreich hinabfahren, so werdet ihr daran
erkennen, daß diese Männer Verächter\textless sup title=``oder:
Verhöhner''\textgreater✲ des HERRN gewesen sind!« \bibleverse{31}Kaum
hatte er diese Worte zu Ende gesprochen, da spaltete sich der Erdboden
unter ihren Füßen, \bibleverse{32}und die Erde tat ihren Mund auf und
verschlang sie samt ihren Familien sowie alle Anhänger Korahs mit ihrer
gesamten Habe: \bibleverse{33}lebend fuhren sie mit allem, was sie
besaßen, in das Totenreich hinab, die Erde schloß sich über ihnen, und
sie waren aus der Mitte der Gemeinde verschwunden. \bibleverse{34}Alle
Israeliten aber, die rings um sie her standen, flohen bei ihrem
Geschrei; denn sie dachten, die Erde würde auch sie verschlingen.
\bibleverse{35}Und es ging Feuer vom HERRN aus und verzehrte die
zweihundertundfünfzig Männer, die das Räucherwerk dargebracht hatten.

\hypertarget{k-die-auf-den-vereitelten-aufruhr-folgenden-begebenheiten}{%
\paragraph{k) Die auf den vereitelten Aufruhr folgenden
Begebenheiten}\label{k-die-auf-den-vereitelten-aufruhr-folgenden-begebenheiten}}

\hypertarget{aa-die-verwendung-der-250-ruxe4ucherpfannen-korahs-und-seiner-genossen-zum-uxfcberzug-fuxfcr-den-opferaltar}{%
\subparagraph{aa) Die Verwendung der 250 Räucherpfannen Korahs und
seiner Genossen zum Überzug für den
Opferaltar}\label{aa-die-verwendung-der-250-ruxe4ucherpfannen-korahs-und-seiner-genossen-zum-uxfcberzug-fuxfcr-den-opferaltar}}

\hypertarget{section-16}{%
\section{17}\label{section-16}}

\bibleverse{1}Hierauf gebot der HERR dem Mose folgendes:
\bibleverse{2}»Befiehl Eleasar, dem Sohne des Priesters Aaron, er solle
die Räucherpfannen von der Brandstätte aufheben und die feurigen Kohlen
in einiger Entfernung von hier umherstreuen; \bibleverse{3}denn dem
Heiligtum sind die Räucherpfannen dieser Männer verfallen, die durch
ihre Sünde ihr Leben verwirkt haben. Man mache daraus breitgehämmerte
Bleche zu einem Überzug für den Altar; denn da sie die Pfannen vor den
HERRN gebracht haben, sind sie dem Heiligtum verfallen und sollen nun
ein Wahrzeichen\textless sup title=``oder: Mahnzeichen''\textgreater✲
für die Israeliten bleiben.« \bibleverse{4}Da nahm der Priester Eleasar
die kupfernen Räucherpfannen, welche die vom Feuer Getöteten vor den
HERRN gebracht hatten, und man hämmerte sie breit zu einem Überzug für
den Altar. \bibleverse{5}So bildeten sie denn ein Erinnerungszeichen für
die Israeliten, damit kein Unbefugter, der nicht zu den Nachkommen
Aarons gehört, herantrete, um Räucherwerk vor dem HERRN zu verbrennen,
und damit es ihm nicht ergehe wie Korah und seiner Rotte, wie der HERR
es ihm durch Mose hatte ankündigen lassen.

\hypertarget{bb-bestrafung-der-uxfcber-den-untergang-der-aufruxfchrer-murrenden-gemeinde-die-durch-mose-und-aaron-bewirkte-suxfchne}{%
\subparagraph{bb) Bestrafung der über den Untergang der Aufrührer
murrenden Gemeinde; die durch Mose und Aaron bewirkte
Sühne}\label{bb-bestrafung-der-uxfcber-den-untergang-der-aufruxfchrer-murrenden-gemeinde-die-durch-mose-und-aaron-bewirkte-suxfchne}}

\bibleverse{6}Am folgenden Tage aber murrte die ganze Gemeinde der
Israeliten gegen\textless sup title=``oder: über''\textgreater✲ Mose und
Aaron; sie riefen: »Ihr habt das Volk des HERRN umgebracht!«
\bibleverse{7}Als sich nun die Gemeinde gegen Mose und Aaron
zusammenrottete, wandten (diese) sich dem Offenbarungszelt zu, und
siehe: die Wolke bedeckte es, und die Herrlichkeit des HERRN wurde
sichtbar. \bibleverse{8}Als dann Mose und Aaron vor das Offenbarungszelt
getreten waren, \bibleverse{9}gebot der HERR dem Mose:
\bibleverse{10}»Entfernt euch aus der Mitte dieser Gemeinde: ich will
sie in einem Augenblick vernichten!« Da warfen sie sich auf ihr
Angesicht nieder, \bibleverse{11}und Mose rief dem Aaron zu: »Nimm die
Räucherpfanne, tu feurige Kohlen vom Altar hinein und leg Räucherwerk
darauf! trage es dann eilends unter die Gemeinde und erwirke ihnen
dadurch Sühne! Denn das Zorngericht ist bereits vom HERRN ausgegangen,
und das Sterben hat schon begonnen!« \bibleverse{12}Da nahm Aaron (die
Räucherpfanne), wie Mose ihm befohlen hatte, und lief mitten in die
Volksmenge hinein, wo das Sterben beim Volk wirklich schon begonnen
hatte. Dann legte er Räucherwerk auf und erwirkte dem Volk dadurch
Sühne; \bibleverse{13}denn als er mitten zwischen den Toten und den
Lebenden dastand, wurde dem Sterben Einhalt getan. \bibleverse{14}Es
belief sich aber die Zahl derer, die durch das Sterben ums Leben
gekommen waren, auf 14700, abgesehen von denen, die um Korahs willen ihr
Leben verloren hatten. \bibleverse{15}Hierauf kehrte Aaron zu Mose an
den Eingang des Offenbarungszeltes zurück, als das Sterben ein Ende
genommen hatte.

\hypertarget{cc-erweis-des-priesterrechts-aarons-durch-das-wunderbare-sprossen-seines-stabes}{%
\subparagraph{cc) Erweis des Priesterrechts Aarons durch das wunderbare
Sprossen seines
Stabes}\label{cc-erweis-des-priesterrechts-aarons-durch-das-wunderbare-sprossen-seines-stabes}}

\bibleverse{16}Darauf gebot der HERR dem Mose folgendes:
\bibleverse{17}»Rede mit den Israeliten und laß dir von ihnen je einen
Stab für jeden Stamm geben, von allen ihren Fürsten Stamm für Stamm,
zwölf Stäbe. Schreibe auf jeden Stab den Namen des betreffenden Fürsten;
\bibleverse{18}auf den Stab Levis aber schreibe den Namen Aarons; denn
ein Stab soll für jedes Haupt ihrer Stämme sein. \bibleverse{19}Dann
lege sie im Offenbarungszelt vor der Gesetzeslade nieder, da, wo ich
mich euch zu offenbaren pflege. \bibleverse{20}Da soll dann der Stab des
Mannes, den ich mir erwähle, grünen\textless sup title=``oder:
ausschlagen''\textgreater✲, und ich will so dem Murren der Israeliten,
das sie gegen euch erheben, ein Ende machen {[}damit es mich nicht
nochmals belästigt{]}.« \bibleverse{21}Als nun Mose dies den Israeliten
mitgeteilt hatte, übergaben ihm alle ihre Fürsten, Stamm für Stamm, je
einen Stab für jeden Fürsten, (im ganzen) zwölf Stäbe; auch der Stab
Aarons war unter ihren Stäben. \bibleverse{22}Darauf legte Mose die
Stäbe vor dem HERRN im Offenbarungszelt nieder. \bibleverse{23}Als nun
Mose am folgenden Tage in das Offenbarungszelt hineinging, siehe, da
hatte der Stab Aarons, der dem Stamme Levi angehörte, gesproßt, und zwar
hatte er Schosse getrieben und Blüten hervorgebracht und trug reife
Mandeln. \bibleverse{24}Mose holte nun alle Stäbe aus dem Heiligtum des
HERRN heraus und legte sie allen Israeliten vor; diese besahen sie und
nahmen ein jeder seinen Stab zurück. \bibleverse{25}Der HERR aber gebot
dem Mose: »Bringe den Stab Aarons wieder vor die Gesetzeslade zurück; er
soll dort als ein Mahnzeichen für die Widerspenstigen aufbewahrt werden,
damit du ihrem Murren gegen mich ein Ende machst und sie nicht zu
sterben brauchen.« \bibleverse{26}Mose tat es; wie der HERR ihm geboten
hatte, so tat er.

\hypertarget{dd-angst-des-volkes-vor-dem-unter-ihnen-befindlichen-heiligtum}{%
\subparagraph{dd) Angst des Volkes vor dem unter ihnen befindlichen
Heiligtum}\label{dd-angst-des-volkes-vor-dem-unter-ihnen-befindlichen-heiligtum}}

\bibleverse{27}Die Israeliten aber sagten zu Mose: »Wehe! Wir kommen um,
wir sind verloren, alle sind wir verloren! \bibleverse{28}Wer irgend der
Wohnung des HERRN nahe kommt, muß sterben! Sollen wir denn rettungslos
ums Leben kommen?«

\hypertarget{l-amtspflichten-rechte-und-einkuxfcnfte-der-priester-und-der-leviten}{%
\paragraph{l) Amtspflichten, Rechte und Einkünfte der Priester und der
Leviten}\label{l-amtspflichten-rechte-und-einkuxfcnfte-der-priester-und-der-leviten}}

\hypertarget{aa-allgemeine-verordnungen-uxfcber-die-pflichten-der-priester-und-ihrer-gehilfen-der-leviten}{%
\subparagraph{aa) Allgemeine Verordnungen über die Pflichten der
Priester und ihrer Gehilfen, der
Leviten}\label{aa-allgemeine-verordnungen-uxfcber-die-pflichten-der-priester-und-ihrer-gehilfen-der-leviten}}

\hypertarget{section-17}{%
\section{18}\label{section-17}}

\bibleverse{1}Der HERR sagte dann zu Aaron: »Du und deine Söhne und das
Haus deines Vaters\textless sup title=``=~dein väterlicher
Stamm''\textgreater✲ mit dir, ihr sollt die Verfehlungen am Heiligtum
auf euch nehmen; und du und deine Söhne mit dir, ihr habt die
Verfehlungen an eurem Priesteramt zu tragen. \bibleverse{2}Aber auch
deine Brüder, den Stamm Levi, deinen väterlichen Stamm, laß mit dir
herantreten: sie sollen sich dir anschließen und dir behilflich sein,
während du und deine Söhne mit dir den Dienst vor dem Offenbarungszelt
verrichtet. \bibleverse{3}Und zwar sollen sie das besorgen, was zu
deiner Bedienung und zur Bedienung des ganzen Zeltes erforderlich ist;
nur an die heiligen Geräte und an den Altar dürfen sie nicht
herantreten, sonst müßt sowohl ihr als auch sie sterben.
\bibleverse{4}Sie sollen sich also dir anschließen und den Dienst am
Offenbarungszelt in seinem ganzen Umfang besorgen, und kein Unbefugter
darf sich neben euch daran beteiligen. \bibleverse{5}Ihr aber sollt den
Dienst im Heiligtum und den Dienst am Altar verrichten, damit nicht
nochmals ein Strafgericht über die Israeliten ergeht.
\bibleverse{6}Bedenkt wohl: ich selbst habe eure Brüder, die Leviten,
aus der Mitte der Israeliten als ein Geschenk für euch herausgenommen,
als solche, die dem HERRN zu eigen gegeben sind, um den Dienst am
Offenbarungszelt zu verrichten. \bibleverse{7}Du aber und deine Söhne
mit dir, ihr habt eures Priesteramts zu warten bezüglich aller
Geschäfte, die den Altar betreffen, und bezüglich aller Verrichtungen
drinnen hinter dem Vorhange und sollt so den Dienst versehen; als einen
geschenkten Dienst\textless sup title=``=~als Geschenk''\textgreater✲
überweise ich euch das Priestertum; der Unbefugte aber, der es ausübt,
soll den Tod erleiden.«

\hypertarget{bb-die-einkuxfcnfte-der-priester}{%
\subparagraph{bb) Die Einkünfte der
Priester}\label{bb-die-einkuxfcnfte-der-priester}}

\bibleverse{8}Weiter sagte der HERR zu Aaron: »Ich selbst überweise dir
hiermit das, was von meinen Hebeopfern (nicht verbrannt wird, sondern)
den Abhub bildet: von allen heiligen Gaben der Israeliten verleihe ich
sie dir und deinen Söhnen als Anteil, als eine ewige Gebühr.
\bibleverse{9}Folgendes soll von den hochheiligen Gaben, soweit sie
nicht verbrannt werden, dir zustehen: alles, was von ihnen dargebracht
wird an Speisopfern sowie an Sündopfern und an Schuldopfern, die sie mir
als Ersatz für Veruntreuungen darbringen; diese (Opfergaben) sollen dir
und deinen Söhnen als etwas Hochheiliges zustehen. \bibleverse{10}An
einer hochheiligen Stätte sollst du sie verzehren; jede männliche Person
darf davon essen; als heilig sollen sie dir gelten. \bibleverse{11}Auch
folgendes soll dir als Hebe von ihren Gaben zustehen: alle Webeopfer der
Israeliten; dir und deinen Söhnen und Töchtern mit dir weise ich sie als
eine ewige Gebühr zu: wer rein ist in deinem Hause, darf davon essen.
\bibleverse{12}Alles Beste vom Öl und alles Beste vom Most und vom
Getreide, ihre Erstlingsgaben, die sie dem HERRN weihen, das weise ich
dir zu: \bibleverse{13}die Erstlinge von allen Früchten, die in ihrem
Lande wachsen und die sie dem HERRN darbringen, sollen dir gehören: wer
rein ist in deinem Hause, darf davon essen. \bibleverse{14}Alles, was in
Israel mit dem Bann belegt worden ist, soll dir gehören.
\bibleverse{15}Alles Erstgeborene von allem Fleisch, das man dem HERRN
darzubringen hat, vom Menschen wie vom Vieh, soll dir gehören; jedoch
sollst du die Erstgeburten der Menschen unbedingt lösen lassen, und auch
die Erstgeburten der unreinen Tiere sollst du lösen lassen.
\bibleverse{16}Was aber die Lösung (einer menschlichen Erstgeburt)
betrifft, so sollst du sie im Alter von einem Monat an lösen lassen, und
zwar nach deiner Schätzung, für einen Betrag von fünf Schekeln Silber
nach dem Schekel des Heiligtums, der zwanzig Gera beträgt\textless sup
title=``vgl. 2.Mose 30,13''\textgreater✲. \bibleverse{17}Jedoch das
Erstgeborene eines Rindes oder das Erstgeborene eines Schafes oder einer
Ziege darfst du nicht lösen lassen, sie sind heilig\textless sup
title=``=~gehören dem Heiligtum''\textgreater✲; ihr Blut sollst du an
den Altar sprengen und ihr Fett als Feueropfer zum lieblichen Geruch für
den HERRN in Rauch aufgehen lassen; \bibleverse{18}ihr Fleisch aber soll
dir gehören: wie die Webebrust und wie die rechte Keule soll es dir
gehören. \bibleverse{19}Alle Hebeopfer von heiligen Gaben, welche die
Israeliten dem HERRN von ihrem Besitz darbringen, überweise ich dir und
deinen Söhnen und Töchtern bei dir als ewige Gebühr: das soll ein ewiger
Salzbund\textless sup title=``d.h. eine ewiggültige und unverbrüchliche
Bundesbestimmung; vgl. 3.Mose 2,13''\textgreater✲ vor dem HERRN für dich
und deine Nachkommen bei dir sein!«

\hypertarget{cc-die-zuweisung-des-zehnten-an-die-leviten-bei-versagung-von-landbesitz}{%
\subparagraph{cc) Die Zuweisung des Zehnten an die Leviten bei Versagung
von
Landbesitz}\label{cc-die-zuweisung-des-zehnten-an-die-leviten-bei-versagung-von-landbesitz}}

\bibleverse{20}Weiter sagte der HERR zu Aaron: »In ihrem Lande sollst du
keinen Erbbesitz haben und keinen Teil des Landes unter ihnen besitzen:
ich bin dein Anteil und dein Erbbesitz inmitten der Israeliten!
\bibleverse{21}(Als Entgelt) aber überweise ich hiermit den Leviten alle
Zehnten in Israel zum Erbbesitz für ihren Dienst, den sie zu verrichten
haben, für den Dienst am Offenbarungszelte. \bibleverse{22}Die
Israeliten nämlich dürfen künftighin am Offenbarungszelt nicht mehr
tätig sein, weil sie sonst Sünde auf sich laden würden und sterben
müßten. \bibleverse{23}Vielmehr sollen die Leviten den Dienst am
Offenbarungszelt verrichten, und sie sollen für etwaige Verfehlungen
verantwortlich sein: diese Bestimmung hat ewige Geltung für eure
künftigen Geschlechter. Aber einen Erbbesitz\textless sup
title=``=~eigenen Besitz''\textgreater✲ sollen sie inmitten der
Israeliten nicht haben; \bibleverse{24}denn den Zehnten, den die
Israeliten als Hebeopfer an den HERRN abgeben, habe ich den Leviten als
Erbbesitz überwiesen; darum habe ich in bezug auf sie verordnet, daß sie
keinen Erbbesitz inmitten der Israeliten zu eigen haben sollen.«

\hypertarget{dd-die-abgabe-des-zehnten-von-den-einkuxfcnften-der-leviten-an-die-priester}{%
\subparagraph{dd) Die Abgabe des Zehnten von den Einkünften der Leviten
an die
Priester}\label{dd-die-abgabe-des-zehnten-von-den-einkuxfcnften-der-leviten-an-die-priester}}

\bibleverse{25}Weiter gebot der HERR dem Mose folgendes:
\bibleverse{26}»Teile den Leviten folgende Verordnungen mit: Wenn ihr
von den Israeliten den Zehnten in Empfang nehmt, den ich euch von ihnen
als euren Erbbesitz zugewiesen habe, so sollt ihr davon ein Hebeopfer
für den HERRN abgeben, nämlich den Zehnten vom Zehnten.
\bibleverse{27}Dann soll dies euer Hebeopfer euch so angerechnet werden,
wie (wenn die anderen Israeliten) das Getreide von der Tenne und die
Fülle\textless sup title=``d.h. die Abfüllung''\textgreater✲ von der
Kelter (abgeben). \bibleverse{28}Ebenso sollt auch ihr ein Hebeopfer für
den HERRN von allen euren Zehnten abgeben, die ihr von den Israeliten in
Empfang nehmt, und sollt davon das Hebeopfer, das dem HERRN gebührt, an
den Priester Aaron abführen. \bibleverse{29}Von allen Gaben, die euch
zufallen, sollt ihr ein unverkürztes Hebeopfer für den HERRN abgeben,
und zwar von allem Besten davon, als die davon zu entrichtende heilige
Gebühr. \bibleverse{30}Sage also zu ihnen: ›Wenn ihr das Beste davon
abgebt, so soll (das übrige) euch, den Leviten, so angerechnet werden
wie (den anderen Israeliten) der Ertrag der Tenne und wie der Ertrag der
Kelter, \bibleverse{31}so daß ihr es an jedem beliebigen Ort verzehren
dürft, ihr und eure Familie; denn es ist euer Lohn für euren Dienst am
Offenbarungszelt.‹ \bibleverse{32}Wenn ihr also das Beste davon abgebt,
so werdet ihr seinetwegen keine Sünde auf euch laden und werdet die
heiligen Gaben der Israeliten nicht entweihen und darum auch nicht
sterben müssen.«

\hypertarget{m-herstellung-und-gebrauch-des-reinigungswassers-aus-der-asche-einer-ruxf6tlichen-kuh-behufs-reinigung-der-durch-leichenberuxfchrung-verunreinigten}{%
\paragraph{m) Herstellung und Gebrauch des Reinigungswassers aus der
Asche einer rötlichen Kuh behufs Reinigung der durch Leichenberührung
Verunreinigten}\label{m-herstellung-und-gebrauch-des-reinigungswassers-aus-der-asche-einer-ruxf6tlichen-kuh-behufs-reinigung-der-durch-leichenberuxfchrung-verunreinigten}}

\hypertarget{section-18}{%
\section{19}\label{section-18}}

\bibleverse{1}Der HERR sagte dann weiter zu Mose und Aaron folgendes:
\bibleverse{2}»Dies ist die Gesetzesbestimmung, die der HERR erlassen
hat durch die Verordnung: Sage den Israeliten, sie sollen dir eine
fehlerlose, rötliche Kuh bringen, die kein Gebrechen an sich hat und auf
die noch kein Joch gekommen ist. \bibleverse{3}Die sollt ihr dem
Priester Eleasar übergeben, und man soll sie dann vor das Lager
hinausführen und sie vor seinen Augen schlachten. \bibleverse{4}Hierauf
nehme der Priester Eleasar mit seinem Finger etwas von ihrem Blut und
sprenge siebenmal von ihrem Blut in der Richtung nach der Vorderseite
des Offenbarungszeltes hin. \bibleverse{5}Dann verbrenne man die Kuh vor
seinen Augen: alles, ihre Haut, ihr Fleisch und ihr Blut samt dem Inhalt
ihrer Eingeweide soll man verbrennen. \bibleverse{6}Hierauf nehme der
Priester Zedernholz, Ysop und Karmesinwolle und werfe es mitten in das
Feuer, in welchem die Kuh verbrannt wird. \bibleverse{7}Dann wasche der
Priester seine Kleider und nehme ein Wasserbad; hierauf darf er wieder
ins Lager kommen, bleibt jedoch bis zum Abend unrein. \bibleverse{8}Auch
derjenige, welcher die Kuh verbrannt hat, muß seine Kleider im Wasser
waschen und ein Wasserbad nehmen und bleibt bis zum Abend unrein.
\bibleverse{9}Dann soll ein Mann, der rein ist, die Asche der Kuh
sammeln und sie außerhalb des Lagers an einen reinen Platz hinschütten,
damit sie dort für die Gemeinde der Israeliten zur Herstellung von
Reinigungswasser aufbewahrt werde: es ist ein Entsündigungsmittel.
\bibleverse{10}Auch der Mann, welcher die Asche der Kuh gesammelt hat,
muß seine Kleider waschen und ist dann noch bis zum Abend unrein.

Es soll dann aber für die Israeliten und für die Fremdlinge, die sich
als Gäste unter ihnen aufhalten, folgende Vorschrift ewige Geltung
haben: \bibleverse{11}Wer einen Toten, irgendeine Menschenleiche,
berührt, soll sieben Tage lang unrein sein. \bibleverse{12}Ein solcher
Mensch soll sich damit\textless sup title=``d.h. mit solchem
Wasser''\textgreater✲ am dritten und am siebten Tage entsündigen, dann
ist er wieder rein; wenn er sich aber am dritten und am siebten Tage
nicht entsündigt, so wird er nicht rein. \bibleverse{13}Wer einen Toten,
die Leiche irgendeines gestorbenen Menschen, berührt und sich danach
nicht entsündigt, der hat die Wohnung des HERRN verunreinigt, und ein
solcher Mensch soll aus Israel ausgerottet werden. Weil er nicht mit
Reinigungswasser besprengt worden ist, bleibt er unrein: seine
Unreinheit bleibt an ihm haften.«

\hypertarget{anweisungen-uxfcber-besondere-fuxe4lle-der-verunreinigung-und-deren-behandlung}{%
\paragraph{Anweisungen über besondere Fälle der Verunreinigung und deren
Behandlung}\label{anweisungen-uxfcber-besondere-fuxe4lle-der-verunreinigung-und-deren-behandlung}}

\bibleverse{14}»Folgende Bestimmung gilt, wenn jemand in einem Zelte
stirbt: Jeder, der in das Zelt hineingeht, und jeder, der sich im Zelte
befindet, ist sieben Tage lang unrein; \bibleverse{15}auch jedes offene
Gefäß, auf dem sich kein festschließender Deckel befindet, ist unrein.
\bibleverse{16}Ebenso soll jeder, der auf freiem Feld einen mit dem
Schwert Erschlagenen oder sonst einen Toten oder menschliche Gebeine
oder ein Grab anrührt, sieben Tage lang unrein sein. \bibleverse{17}Für
einen so unrein Gewordenen nehme man etwas von der Asche des zur
Entsündigung verbrannten Opfertieres und gieße lebendiges
Wasser\textless sup title=``=~Quell- oder Flußwasser''\textgreater✲ in
ein Gefäß darüber. \bibleverse{18}Dann nehme ein reiner Mann einen
Ysopbüschel, tauche ihn in das Wasser und besprenge damit das Zelt samt
allen Geräten und die darin befindlichen Personen sowie den, der mit
Totengebeinen oder einem Erschlagenen oder einem Toten oder einem Grabe
in Berührung gekommen ist. \bibleverse{19}Und zwar soll der Reine den
Unreinen am dritten und am siebten Tage besprengen und ihn so am siebten
Tage entsündigen. Alsdann soll der Betreffende seine Kleider waschen und
ein Wasserbad nehmen, dann wird er am Abend wieder rein sein.
\bibleverse{20}Wenn aber jemand unrein wird und sich nicht entsündigt,
so soll ein solcher Mensch aus der Gemeinde ausgerottet werden; denn er
hat das Heiligtum des HERRN verunreinigt und ist nicht mit dem
Reinigungswasser besprengt worden: er ist unrein. \bibleverse{21}Diese
Verordnung soll bei euch ewige Geltung haben. Und auch der, welcher die
Besprengung mit dem Reinigungswasser vorgenommen hat, muß seine Kleider
waschen, und wer das Reinigungswasser berührt, soll bis zum Abend unrein
sein. \bibleverse{22}Auch alles, was der Unreine anrührt, wird unrein,
und ebenso wird jeder, der ihn berührt, bis zum Abend unrein.«

\hypertarget{n-ankunft-in-kades-und-mirjams-tod-erneutes-murren-des-volkes-die-fuxfcr-mose-und-aaron-verhuxe4ngnisvolle-wasserspende-aus-dem-felsen}{%
\paragraph{n) Ankunft in Kades und Mirjams Tod; erneutes Murren des
Volkes; die für Mose und Aaron verhängnisvolle Wasserspende aus dem
Felsen}\label{n-ankunft-in-kades-und-mirjams-tod-erneutes-murren-des-volkes-die-fuxfcr-mose-und-aaron-verhuxe4ngnisvolle-wasserspende-aus-dem-felsen}}

\hypertarget{section-19}{%
\section{20}\label{section-19}}

\bibleverse{1}Hierauf gelangte die ganze Gemeinde der Israeliten in die
Wüste Zin im ersten Monat (des vierzigsten Jahres), und das Volk ließ
sich in Kades nieder. Dort starb Mirjam und wurde dort begraben.

\bibleverse{2}Weil aber die Gemeinde kein Wasser hatte, rottete sie sich
gegen Mose und Aaron zusammen; \bibleverse{3}und das Volk haderte mit
Mose und rief laut aus: »Ach, wären wir doch auch umgekommen, als unsere
Brüder vor dem HERRN umkamen! \bibleverse{4}Warum habt ihr nur die
Gemeinde des HERRN in diese Wüste geführt, daß wir hier mit unserem Vieh
sterben müssen! \bibleverse{5}Und warum habt ihr uns aus Ägypten hierher
gebracht und uns in diese traurige Gegend geführt, an einen Ort, wo man
nicht säen kann und wo kein Feigenbaum, kein Weinstock und kein
Granatbaum zu finden ist und wo es nicht einmal Trinkwasser gibt!«

\bibleverse{6}Da gingen Mose und Aaron aus der Versammlung weg an den
Eingang des Offenbarungszeltes und warfen sich auf ihr Angesicht nieder;
da erschien ihnen die Herrlichkeit des HERRN. \bibleverse{7}Und der HERR
gebot dem Mose folgendes: »Nimm den Stab und versammle die Gemeinde, du
und dein Bruder Aaron; redet dann den Felsen vor ihren Augen an, daß er
sein Wasser hergeben solle, \bibleverse{8}so wirst du Wasser für sie aus
dem Felsen hervorfließen lassen und so der Gemeinde und ihrem Vieh
Trinkwasser verschaffen.« \bibleverse{9}Da holte Mose den Stab vor dem
HERRN weg\textless sup title=``=~aus dem Heiligtum''\textgreater✲, wie
der HERR ihm geboten hatte. \bibleverse{10}Darauf ließen Mose und Aaron
die Gemeinde vor dem Felsen zusammenkommen, und er sagte zu ihnen: »Hört
doch, ihr Widerspenstigen! Können wir wohl Wasser für euch aus diesem
Felsen hervorfließen lassen?« \bibleverse{11}Als Mose dann seine Hand
erhoben und zweimal mit seinem Stabe an den Felsen geschlagen hatte, da
strömte Wasser in Fülle heraus, so daß die Gemeinde und ihr Vieh zu
trinken hatten. \bibleverse{12}Der HERR aber sagte zu Mose und Aaron:
»Zur Strafe dafür, daß ihr mir kein Vertrauen geschenkt und mir nicht
als dem Heiligen die Ehre vor den Augen der Israeliten gegeben habt,
darum sollt ihr diese Gemeinde nicht in das Land bringen, das ich für
sie bestimmt habe!« \bibleverse{13}Das ist das Haderwasser (von Kades),
wo die Israeliten mit dem HERRN gehadert haben und er sich an ihnen als
der Heilige erwies\textless sup title=``=~sich an ihnen
verherrlichte''\textgreater✲.

\hypertarget{o-die-edomiter-verweigern-die-erlaubnis-zum-durchzug-aarons-tod}{%
\paragraph{o) Die Edomiter verweigern die Erlaubnis zum Durchzug; Aarons
Tod}\label{o-die-edomiter-verweigern-die-erlaubnis-zum-durchzug-aarons-tod}}

\bibleverse{14}Von Kades aus sandte Mose dann Boten an den König der
Edomiter: »So lassen dir deine Brüder, die Israeliten, sagen: Du kennst
selbst alle Leiden, die uns betroffen haben, \bibleverse{15}wie unsere
Väter nach Ägypten hinabgezogen sind und wir lange Zeit in Ägypten
gewohnt haben. Als dann die Ägypter uns und unsere Väter mißhandelten,
\bibleverse{16}haben wir zum HERRN um Hilfe geschrien, und er hat unser
Flehen gehört und einen Engel gesandt, der uns aus Ägypten hinausgeführt
hat. Jetzt befinden wir uns nun in Kades, einer Stadt an der Grenze
deines Gebietes. \bibleverse{17}Gestatte uns doch den Durchzug durch
dein Land! Wir wollen nicht durch die Äcker und durch die Weinberge
ziehen, auch kein Wasser aus den Brunnen✲ trinken; nein, auf der
Königsstraße wollen wir ziehen, ohne nach rechts und nach links
abzubiegen, bis wir dein Gebiet durchzogen haben.« \bibleverse{18}Aber
der Edomiter antwortete ihm: »Du darfst nicht durch mein Land ziehen,
sonst trete ich dir mit bewaffneter Hand entgegen.«
\bibleverse{19}Darauf ließen ihm die Israeliten sagen: »Auf der
Landstraße wollen wir ziehen, und wenn wir von deinem Wasser trinken,
ich und meine Herden, so will ich den vollen Preis dafür bezahlen; ich
will nur -- das ist die ganze Sache -- zu Fuß hindurchziehen.«
\bibleverse{20}Doch er antwortete: »Nein, du darfst nicht
hindurchziehen!« Zugleich zogen die Edomiter ihm mit zahlreichem
Kriegsvolk und mit bewaffneter Hand entgegen. \bibleverse{21}Da die
Edomiter also den Israeliten den Durchzug durch ihr Gebiet nicht
gestatten wollten, mußten die Israeliten seitwärts von ihnen abbiegen.

\hypertarget{der-zug-von-kades-bis-zum-berge-hor-aarons-tod}{%
\paragraph{Der Zug von Kades bis zum Berge Hor; Aarons
Tod}\label{der-zug-von-kades-bis-zum-berge-hor-aarons-tod}}

\bibleverse{22}So brachen sie denn von Kades auf, und die Israeliten,
die ganze Gemeinde, kamen an den Berg Hor. \bibleverse{23}Da sagte der
HERR zu Mose und zu Aaron am Berge Hor, an der Grenze des
Edomiterlandes: \bibleverse{24}»Aaron soll jetzt zu seinen Volksgenossen
versammelt werden; denn er soll nicht in das Land kommen, das ich für
die Israeliten bestimmt habe, weil ihr beim Haderwasser meinem Gebot
nicht nachgekommen seid. \bibleverse{25}Nimm Aaron und seinen Sohn
Eleasar und laß sie auf den Berg Hor hinaufsteigen; \bibleverse{26}laß
dort Aaron seine Gewänder ausziehen und lege sie seinem Sohne Eleasar
an; denn Aaron soll (zu seinen Volksgenossen) versammelt werden und dort
sterben.« \bibleverse{27}Mose tat, wie der HERR ihm geboten hatte, und
sie stiegen vor den Augen der ganzen Gemeinde auf den Berg Hor hinauf.
\bibleverse{28}Dort ließ Mose den Aaron seine Gewänder ausziehen und
legte sie dessen Sohne Eleasar an; darauf starb Aaron dort auf dem
Gipfel des Berges; Mose aber und Eleasar stiegen wieder vom Berge hinab.
\bibleverse{29}Als nun die ganze Gemeinde erfuhr, daß Aaron gestorben
sei, trauerte das ganze Haus Israel um Aaron dreißig Tage lang.

\hypertarget{p-siegreicher-kampf-mit-dem-kuxf6nig-von-arad-die-eherne-schlange-eroberung-des-ostjordanlandes}{%
\paragraph{p) Siegreicher Kampf mit dem König von Arad; die eherne
Schlange; Eroberung des
Ostjordanlandes}\label{p-siegreicher-kampf-mit-dem-kuxf6nig-von-arad-die-eherne-schlange-eroberung-des-ostjordanlandes}}

\hypertarget{aa-sieg-und-vollstreckung-des-bannes}{%
\subparagraph{aa) Sieg und Vollstreckung des
Bannes}\label{aa-sieg-und-vollstreckung-des-bannes}}

\hypertarget{section-20}{%
\section{21}\label{section-20}}

\bibleverse{1}Als nun der kanaanäische König von Arad, der im Südland
wohnte, die Kunde erhielt, daß die Israeliten auf dem Wege von Atharim
heranzögen, griff er die Israeliten an und nahm einige von ihnen
gefangen. \bibleverse{2}Da legten die Israeliten ein Gelübde vor dem
HERRN ab und versprachen: »Wenn du dieses Volk in unsere Gewalt gibst,
so wollen wir an ihren Ortschaften den Bann vollstrecken.«
\bibleverse{3}Da erhörte der HERR die Bitte der Israeliten und gab ihnen
die Kanaanäer preis. Man vollstreckte dann an ihnen und ihren
Ortschaften den Bann und nannte die Stätte seitdem Horma\textless sup
title=``d.h. Bann''\textgreater✲.

\hypertarget{bb-murren-des-volkes-die-giftschlangen-und-die-eherne-schlange}{%
\subparagraph{bb) Murren des Volkes; die Giftschlangen und die eherne
Schlange}\label{bb-murren-des-volkes-die-giftschlangen-und-die-eherne-schlange}}

\bibleverse{4}Dann brachen sie vom Berge Hor auf in der Richtung nach
dem Schilfmeer, um das Land der Edomiter zu umgehen. Unterwegs aber
wurde das Volk mißmutig \bibleverse{5}und erhob Anklage\textless sup
title=``oder: lehnte sich auf''\textgreater✲ gegen Gott und gegen Mose:
»Warum habt ihr uns aus Ägypten hierher geführt? Um uns in der Wüste
sterben zu lassen? Es gibt hier ja weder Brot noch Wasser, und uns ekelt
vor diesem erbärmlichen Brotzeug!« \bibleverse{6}Da sandte der HERR
feurige✲ Schlangen unter das Volk; die bissen die Leute, so daß
zahlreiche Israeliten starben. \bibleverse{7}Da kam das Volk zu Mose und
bekannte: »Wir haben gesündigt, daß wir Anklagen gegen den HERRN und
gegen dich erhoben haben; lege Fürbitte beim HERRN ein, daß er uns von
den Schlangen befreie!« Als Mose nun Fürbitte für das Volk einlegte,
\bibleverse{8}sagte der HERR zu ihm: »Fertige dir ein Schlangenbild an
und befestige es an einer Stange; wer dann gebissen ist und es anschaut,
soll am Leben bleiben.« \bibleverse{9}Da fertigte Mose eine eherne✲
Schlange an und befestigte sie oben an einer Stange. Wenn nun eine
Schlange jemanden gebissen hatte und er auf die eherne Schlange
hinschaute, so blieb er am Leben.

\hypertarget{cc-der-zug-bis-an-den-arnon-und-bis-in-die-steppen-der-moabiter-das-brunnenlied}{%
\subparagraph{cc) Der Zug bis an den Arnon und bis in die Steppen der
Moabiter; das
Brunnenlied}\label{cc-der-zug-bis-an-den-arnon-und-bis-in-die-steppen-der-moabiter-das-brunnenlied}}

\bibleverse{10}Darauf zogen die Israeliten weiter und lagerten bei
Oboth. \bibleverse{11}Dann zogen sie von Oboth weiter und lagerten bei
Ijje-Abarim in der Wüste, die vor dem Lande der Moabiter nach Osten
liegt. \bibleverse{12}Von dort zogen sie weiter und lagerten im Bachtal
Sered. \bibleverse{13}Von dort zogen sie weiter und lagerten jenseits
des Arnon, {[}der in der Wüste ist,{]} der im Lande der Amoriter
entspringt; der Arnon bildet nämlich die Grenze Moabs zwischen den
Moabitern und Amoritern. \bibleverse{14}Daher heißt es im Buch der
Kriege des HERRN: »Waheb in Supha, dazu die Bachtäler des Arnon
\bibleverse{15}und den Abhang der Bachtäler, der bis in die Gegend von
Ar reicht und sich an die Grenze von Moab anlehnt.«

\bibleverse{16}Von dort (zogen sie) dann nach Beer; das ist der Brunnen,
den der HERR meinte, als er zu Mose sagte: »Versammle das Volk, damit
ich ihm Wasser gebe.« \bibleverse{17}Damals sangen die Israeliten
folgendes Lied: »Quill empor, o Brunnen! Singt ihm zu: \bibleverse{18}›O
Brunnen, den Fürsten gegraben, den die Edlen des Volks erschlossen haben
mit dem Zepter, mit ihren Stäben!‹« Aus der Wüste (zogen sie) dann nach
Matthana, \bibleverse{19}von Matthana nach Nahaliel, von Nahaliel nach
Bamoth, \bibleverse{20}von Bamoth in das Tal, das in der moabitischen
Ebene liegt an der Höhe des Pisga, der auf die weite Einöde hinabblickt.

\hypertarget{dd-besiegung-des-amoriterkuxf6nigs-sihon-und-eroberung-seines-landes-triumphlied-der-israeliten}{%
\subparagraph{dd) Besiegung des Amoriterkönigs Sihon und Eroberung
seines Landes; Triumphlied der
Israeliten}\label{dd-besiegung-des-amoriterkuxf6nigs-sihon-und-eroberung-seines-landes-triumphlied-der-israeliten}}

\bibleverse{21}Darauf sandten die Israeliten Boten zu Sihon, dem König
der Amoriter, und ließen ihm sagen: \bibleverse{22}»Gestatte uns den
Durchzug durch dein Land! Wir wollen nicht auf die Äcker und in die
Weinberge abbiegen, auch kein Wasser aus den Brunnen✲ trinken; nein, wir
wollen auf der Königsstraße ziehen, bis wir dein Gebiet durchzogen
haben.« \bibleverse{23}Aber Sihon gestattete den Israeliten den Durchzug
durch sein Gebiet nicht, sondern sammelte sein gesamtes Kriegsvolk und
zog den Israeliten in die Wüste entgegen; und als er nach Jahaz gekommen
war, griff er die Israeliten an. \bibleverse{24}Diese aber schlugen ihn
mit der Schärfe des Schwertes und eroberten sein Land vom Arnon bis zum
Jabbok, bis an das Land der Ammoniter; denn Jaser liegt an der Grenze
des Ammoniterlandes. \bibleverse{25}So nahmen denn die Israeliten alle
dortigen Städte ein und ließen sich in allen Städten der Amoriter
nieder, in Hesbon und allen dazugehörigen Ortschaften.
\bibleverse{26}Denn Hesbon war die Hauptstadt des Amoriterkönigs Sihon;
dieser hatte nämlich mit dem vorigen Könige der Moabiter Krieg geführt
und ihm sein ganzes Land bis an den Arnon abgenommen.
\bibleverse{27}Darum singen die Dichter: Kommt nach Hesbon! Aufgebaut
und befestigt werde die Stadt Sihons! \bibleverse{28}Denn Feuer ging
(einst) aus von Hesbon, eine Flamme von der Stadt Sihons; die fraß die
Städte Moabs und verbrannte die Höhen am Arnon. \bibleverse{29}Wehe dir,
Moab! Dem Untergang geweiht bist du, Volk des Kamos, der seine Söhne zu
Flüchtlingen gemacht hat und seine Töchter zu Gefangenen für Sihon, den
Amoriterkönig. \bibleverse{30}Da haben wir (Moab) niedergeschossen,
Hesbon ist verlorengegangen bis Dibon; da haben wir verwüstet bis
Nophah: Feuer ging aus bis Medeba.

\hypertarget{ee-weiteres-vorruxfccken-der-israeliten-besiegung-des-kuxf6nigs-og-von-basan}{%
\subparagraph{ee) Weiteres Vorrücken der Israeliten; Besiegung des
Königs Og von
Basan}\label{ee-weiteres-vorruxfccken-der-israeliten-besiegung-des-kuxf6nigs-og-von-basan}}

\bibleverse{31}Als sich nun die Israeliten im Lande der Amoriter
festgesetzt hatten, \bibleverse{32}sandte Mose Männer aus, um Jaser
auszukundschaften, und sie nahmen dann die Stadt und die dazugehörigen
Ortschaften ein, und man vertrieb die dort ansässigen Amoriter.
\bibleverse{33}Hierauf wandten sie sich und zogen in der Richtung nach
Basan hinauf. Da rückte Og, der König von Basan, mit seinem gesamten
Kriegsvolk ihnen nach Edrei entgegen, um ihnen eine Schlacht zu liefern.
\bibleverse{34}Der HERR aber sagte zu Mose: »Fürchte dich nicht vor ihm!
Denn ich habe ihn mit seinem ganzen Volk und seinem Land in deine Gewalt
gegeben: verfahre mit ihm so, wie du mit dem Amoriterkönig Sihon, der in
Hesbon wohnte, verfahren bist.« \bibleverse{35}Da erschlugen sie ihn
nebst seinen Söhnen und seinem ganzen Kriegsvolk, so daß ihm auch nicht
einer übrigblieb, der entronnen wäre, und nahmen sein Land in Besitz.

\hypertarget{section-21}{%
\section{22}\label{section-21}}

\bibleverse{1}Hierauf zogen die Israeliten weiter und lagerten in den
Steppen der Moabiter jenseits des Jordans, Jericho gegenüber.

\hypertarget{ereignisse-in-den-steppen-der-moabiter-222-3613}{%
\subsubsection{3. Ereignisse in den Steppen der Moabiter
(22,2-36,13)}\label{ereignisse-in-den-steppen-der-moabiter-222-3613}}

\hypertarget{a-die-geschichte-des-sehers-bileam-222-2425}{%
\paragraph{a) Die Geschichte des Sehers Bileam
(22,2-24,25)}\label{a-die-geschichte-des-sehers-bileam-222-2425}}

\hypertarget{aa-der-moabiterkuxf6nig-balak-beschlieuxdft-gesandte-an-bileam-zu-schicken}{%
\subparagraph{aa) Der Moabiterkönig Balak beschließt, Gesandte an Bileam
zu
schicken}\label{aa-der-moabiterkuxf6nig-balak-beschlieuxdft-gesandte-an-bileam-zu-schicken}}

\bibleverse{2}Als nun Balak, der Sohn Zippors, alles Unheil sah, das die
Israeliten den Amoritern zugefügt hatten, \bibleverse{3}überkam die
Moabiter eine große Angst vor dem Volk (Israel), weil es so zahlreich
war, und sie empfanden ein Grauen vor den Israeliten. \bibleverse{4}Da
sagten die Moabiter zu den Ältesten der Midianiter: »Nun wird dieser
Schwarm alles rings um uns her kahlfressen, wie die Rinder das Grün des
Feldes abfressen!« Damals war aber Balak, der Sohn Zippors, König der
Moabiter. \bibleverse{5}Dieser sandte Boten zu Bileam, dem Sohne Beors,
nach Pethor, das am Euphratstrom liegt, ins Land seiner Volksgenossen,
um ihn holen zu lassen, und ließ ihm sagen: »Da ist ein Volk aus Ägypten
ausgezogen, das hat sich jetzt über das ganze Land ausgebreitet und sich
mir gegenüber festgesetzt. \bibleverse{6}So komm nun doch her und
verfluche mir dieses Volk; denn mir ist es zu stark; vielleicht gelingt
es mir dann, eine Niederlage unter ihnen anzurichten und es aus dem
Lande zu vertreiben; denn ich weiß: wen du segnest, der ist gesegnet,
und wem du fluchst, der ist verflucht.«

\hypertarget{bb-balaks-erste-gesandtschaft-ohne-erfolg-bei-bileam-seine-nochmalige-botschaft}{%
\subparagraph{bb) Balaks erste Gesandtschaft ohne Erfolg bei Bileam;
seine nochmalige
Botschaft}\label{bb-balaks-erste-gesandtschaft-ohne-erfolg-bei-bileam-seine-nochmalige-botschaft}}

\bibleverse{7}Da machten sich die Ältesten der Moabiter samt den
Ältesten der Midianiter, nachdem sie sich mit Wahrsagerlohn versehen
hatten, auf den Weg, kamen glücklich bei Bileam an und teilten ihm das
Anliegen Balaks mit. \bibleverse{8}Er aber antwortete ihnen: »Bleibt
diese Nacht hier, dann will ich euch Bescheid geben, je nachdem der HERR
mir Weisung erteilt.« So blieben denn die Fürsten\textless sup
title=``oder: Häuptlinge''\textgreater✲ der Moabiter bei Bileam.

\bibleverse{9}Da erschien Gott dem Bileam im Traum und fragte ihn: »Was
sind das für Männer bei dir?« \bibleverse{10}Bileam antwortete Gott:
»Balak, der Sohn Zippors, der König der Moabiter, hat mir durch sie
sagen lassen: \bibleverse{11}›Da ist ein Volk, das aus Ägypten
ausgezogen ist und sich über das ganze Land ausgebreitet hat. So komm
nun her und verfluche es mir; vielleicht vermag ich dann den Kampf mit
ihm aufzunehmen und es zu vertreiben.‹« \bibleverse{12}Gott aber sagte
zu Bileam: »Du darfst nicht mit ihnen gehen! Du darfst das Volk nicht
verfluchen; denn es ist gesegnet.« \bibleverse{13}Als Bileam nun am
Morgen aufgestanden war, sagte er zu den Häuptlingen Balaks: »Kehrt in
euer Land zurück; denn der HERR hat mir die Erlaubnis versagt, mit euch
zu ziehen.« \bibleverse{14}So machten sich denn die Häuptlinge der
Moabiter auf den Weg, kehrten zu Balak zurück und berichteten ihm:
»Bileam hat sich geweigert, mit uns zu gehen.«

\bibleverse{15}Da sandte Balak noch einmal Fürsten\textless sup
title=``oder: Häuptlinge''\textgreater✲ ab, die zahlreicher und
vornehmer waren als die ersten. \bibleverse{16}Als diese zu Bileam
kamen, sagten sie zu ihm: »So läßt Balak, der Sohn Zippors, dir sagen:
›Weigere dich doch nicht, zu mir zu kommen! \bibleverse{17}Denn ich will
dich gar hoch ehren\textless sup title=``oder: reichlich
belohnen''\textgreater✲ und alles tun, was du von mir verlangen wirst.
So komm doch her und verfluche mir dieses Volk!‹« \bibleverse{18}Bileam
aber gab den Gesandten Balaks die Antwort: »Wenn Balak mir auch alles
Silber und Gold geben wollte, soviel in seinen Palast geht, so vermöchte
ich doch den Befehl des HERRN, meines Gottes, nicht zu übertreten, weder
im Großen noch im Kleinen. \bibleverse{19}Indessen -- bleibt doch auch
ihr diese Nacht hier, damit ich in Erfahrung bringe, was der HERR mir
weiter zu sagen hat.« \bibleverse{20}Da erschien Gott dem Bileam in der
Nacht und sagte zu ihm: »Wenn die Männer gekommen sind, um dich zu
holen, so mache dich auf und gehe mit ihnen; aber du darfst nur das tun,
was ich dir sagen werde.« \bibleverse{21}Da machte sich Bileam am Morgen
auf, sattelte seine Eselin und machte sich mit den Häuptlingen der
Moabiter auf den Weg.

\hypertarget{cc-bileams-reise-nach-moab-und-das-vorkommnis-mit-der-eselin}{%
\subparagraph{cc) Bileams Reise nach Moab und das Vorkommnis mit der
Eselin}\label{cc-bileams-reise-nach-moab-und-das-vorkommnis-mit-der-eselin}}

\bibleverse{22}Da entbrannte aber der Zorn Gottes darüber, daß er sich
aufgemacht hatte, und der Engel des HERRN stellte sich ihm in den Weg,
um ihm feindlich entgegenzutreten, während er auf seiner Eselin
dahinritt und zwei seiner Diener ihn begleiteten. \bibleverse{23}Als nun
die Eselin den Engel des HERRN sah, der mit gezücktem Schwert in der
Hand ihm den Weg vertrat, bog sie vom Wege ab und ging ins Feld; da
schlug Bileam die Eselin, um sie wieder auf den Weg zu bringen.
\bibleverse{24}Hierauf stellte sich der Engel des HERRN in einem
Hohlwege zwischen den Weinbergen auf, wo zu beiden Seiten eine Mauer
war. \bibleverse{25}Als nun die Eselin den Engel des HERRN erblickte,
drückte sie sich fest an die Mauer und preßte dabei den Fuß Bileams
gegen die Mauer; da schlug er sie zum zweitenmal. \bibleverse{26}Hierauf
ging der Engel des HERRN nochmals eine Strecke weiter und blieb an einer
engen Stelle stehen, wo ein Ausweichen nach rechts oder links unmöglich
war. \bibleverse{27}Als nun die Eselin den Engel des HERRN erblickte,
legte sie sich unter Bileam auf den Boden nieder. Da geriet Bileam in
Zorn, so daß er die Eselin mit dem Stock schlug. \bibleverse{28}Der HERR
aber tat der Eselin den Mund auf, und sie sagte zu Bileam: »Was habe ich
dir getan, daß du mich nun schon dreimal geschlagen hast?«
\bibleverse{29}Bileam antwortete der Eselin: »Weil du mich zum besten
gehabt hast! Hätte ich nur ein Schwert in der Hand, so hätte ich dich
längst umgebracht!« \bibleverse{30}Da sagte die Eselin zu Bileam: »Bin
ich nicht deine Eselin, auf der du zeit deines Lebens bis auf den
heutigen Tag geritten bist? Ist es denn jemals meine Art gewesen, mich
so gegen dich zu benehmen?« Er antwortete: »Nein.«

\bibleverse{31}Nun tat der HERR dem Bileam die Augen auf, so daß er den
Engel des HERRN auf dem Wege mit dem gezückten Schwert in der Hand
stehen sah. Da verneigte er sich und warf sich auf sein Angesicht
nieder. \bibleverse{32}Der Engel des HERRN aber sagte zu ihm: »Warum
hast du deine Eselin nun schon dreimal geschlagen? Wisse wohl: ich habe
mich aufgemacht, um dir feindlich entgegenzutreten; denn diese deine
Reise ist unheilvoll\textless sup title=``oder:
überstürzt''\textgreater✲ und gegen meinen Willen. \bibleverse{33}Die
Eselin aber hat mich gesehen und ist alle drei Male vor mir ausgewichen;
hätte sie das nicht getan, so hätte ich dich längst erschlagen, sie aber
am Leben gelassen.« \bibleverse{34}Da sagte Bileam zu dem Engel des
HERRN: »Ich habe mich vergangen; ich wußte ja nicht, daß du mir auf dem
Wege entgegenstandest. Nun aber will ich, wenn mein Vorhaben dir
mißfällt, wieder umkehren!« \bibleverse{35}Da antwortete der Engel des
HERRN dem Bileam: »Gehe mit den Männern hin; aber du darfst nur das
reden, was ich dir eingeben werde.« So zog nun Bileam mit den
Häuptlingen Balaks weiter.

\hypertarget{dd-bileams-ankunft-bei-balak}{%
\subparagraph{dd) Bileams Ankunft bei
Balak}\label{dd-bileams-ankunft-bei-balak}}

\bibleverse{36}Als nun Balak hörte, daß Bileam komme, zog er ihm bis
Ar-Moab entgegen, das am Grenzfluß Arnon, an der äußersten Grenze seines
Landes, lag. \bibleverse{37}Da sagte Balak zu Bileam: »Habe ich nicht in
angemessener Weise zu dir gesandt, um dich rufen zu lassen? Warum bist
du nicht sofort zu mir gekommen? Meinst du vielleicht, ich sei nicht
imstande, dich zu ehren\textless sup title=``oder: zu
belohnen''\textgreater✲?« \bibleverse{38}Da antwortete ihm Bileam: »Ich
bin jetzt ja doch zu dir gekommen; aber werde ich wohl irgend etwas
kundtun können? Nur die Worte, die der HERR mir in den Mund legt, die
werde ich kundtun.« \bibleverse{39}So ging denn Bileam mit Balak, bis
sie nach Kirjath-Huzoth kamen. \bibleverse{40}Dort
schlachtete\textless sup title=``oder: opferte''\textgreater✲ Balak
Rinder und Kleinvieh und schickte davon an Bileam und an die
Fürsten\textless sup title=``oder: Häuptlinge''\textgreater✲, die bei
ihm waren. \bibleverse{41}Am folgenden Morgen aber nahm Balak den Bileam
mit sich und führte ihn nach Bamoth-Baal hinauf, von wo er den äußersten
Teil des (israelitischen) Volkes sehen konnte.

\hypertarget{ee-die-vorbereitungen-zu-der-gottesoffenbarung-der-erste-spruch-bileams}{%
\subparagraph{ee) Die Vorbereitungen zu der Gottesoffenbarung; der erste
Spruch
Bileams}\label{ee-die-vorbereitungen-zu-der-gottesoffenbarung-der-erste-spruch-bileams}}

\hypertarget{section-22}{%
\section{23}\label{section-22}}

\bibleverse{1}Da sagte Bileam zu Balak: »Baue mir hier sieben Altäre und
stelle mir hier sieben junge Stiere und sieben Widder bereit.«
\bibleverse{2}Balak kam der Aufforderung Bileams nach, und beide
opferten je einen Stier und einen Widder auf jedem Altar.
\bibleverse{3}Hierauf sagte Bileam zu Balak: »Bleibe du hier bei deinem
Brandopfer stehen, während ich einen Gang mache; vielleicht kommt der
HERR mir entgegen\textless sup title=``d.h. gibt er sich mir
kund''\textgreater✲, und was immer er mir offenbart, das will ich dir
kundtun.« Dann begab er sich auf eine kahle Anhöhe\textless sup
title=``oder: in die Einsamkeit''\textgreater✲.

\bibleverse{4}Da kam Gott dem Bileam wirklich entgegen, und dieser sagte
zu ihm: »Die sieben Altäre habe ich errichtet und auf jedem Altar einen
jungen Stier und einen Widder geopfert.« \bibleverse{5}Da legte der HERR
dem Bileam Worte in den Mund und gebot ihm: »Kehre zu Balak zurück und
sprich so zu ihm!« \bibleverse{6}Als er nun zu ihm zurückgekehrt war,
stand jener immer noch neben seinem Brandopfer, er mit allen Häuptlingen
der Moabiter.

\hypertarget{ff-bileam-segnet-israel-von-bamoth-baal-2241-herab}{%
\subparagraph{ff) Bileam segnet Israel von Bamoth-Baal (22,41)
herab}\label{ff-bileam-segnet-israel-von-bamoth-baal-2241-herab}}

\bibleverse{7}Da trug er seinen Spruch folgendermaßen vor: »Aus Aram hat
Balak mich holen lassen, Moabs König von den Bergen des Ostens: ›Komm
her, verfluche mir Jakob! ja komm, verwünsche Israel!‹~--
\bibleverse{8}Wie soll ich den verfluchen, den Gott nicht verflucht? Und
wie den verwünschen, den der HERR nicht verwünscht? \bibleverse{9}Ja,
vom Felsengipfel erblicke ich es, und von den Höhen herab erschaue ich
es; ein Volk zeigt sich mir, das für sich abgesondert wohnt und sich
nicht zu den übrigen Völkern rechnet. \bibleverse{10}Wer könnte den
Staub Jakobs zählen\textless sup title=``1.Mose 13,16''\textgreater✲
oder nur den vierten Teil Israels berechnen? Möchte ich doch den Tod
dieser Gerechten sterben und mein Ende dem ihren gleichen!«
\bibleverse{11}Da sagte Balak zu Bileam: »Was hast du mir da angetan! Um
meine Feinde zu verfluchen, habe ich dich holen lassen, und nun hast du
sie sogar gesegnet!« \bibleverse{12}Da gab jener ihm zur Antwort: »Ich
muß doch das, was der HERR mir in den Mund legt, getreulich
aussprechen!«

\hypertarget{gg-die-vorbereitungen-zu-der-neuen-gottesoffenbarung-der-zweite-spruch-bileams}{%
\subparagraph{gg) Die Vorbereitungen zu der neuen Gottesoffenbarung; der
zweite Spruch
Bileams}\label{gg-die-vorbereitungen-zu-der-neuen-gottesoffenbarung-der-zweite-spruch-bileams}}

\bibleverse{13}Nun sagte Balak zu ihm: »Komm doch mit mir an einen
anderen Ort, von wo aus du (das Volk) sehen kannst; du siehst hier nur
den äußersten Teil von ihm und kannst es nicht ganz überblicken;
verfluche es mir dann von dort aus!« \bibleverse{14}Hierauf nahm er ihn
mit sich nach dem Späherfeld, auf den Gipfel des Pisga, baute dort
sieben Altäre und opferte auf jedem Altar einen jungen Stier und einen
Widder. \bibleverse{15}Dann sagte Bileam zu Balak: »Bleibe hier neben
deinem Brandopfer stehen; ich aber will dort einer Kundgebung
entgegengehen\textless sup title=``=~eine Offenbarung
erwarten''\textgreater✲.«

\bibleverse{16}Da kam der HERR dem Bileam entgegen, legte ihm Worte in
den Mund und sagte: »Kehre zu Balak zurück und sprich so zu ihm!«
\bibleverse{17}Als er nun zu ihm zurückgekehrt war, stand jener immer
noch neben seinem Brandopfer, und die Häuptlinge der Moabiter waren bei
ihm.

\hypertarget{hh-bileam-segnet-israel-vom-berge-pisga-herab}{%
\subparagraph{hh) Bileam segnet Israel vom Berge Pisga
herab}\label{hh-bileam-segnet-israel-vom-berge-pisga-herab}}

\bibleverse{18}Als nun Balak ihn fragte: »Was hat der HERR gesagt?«,
trug er seinen Spruch folgendermaßen vor: »Wohlan, Balak, höre zu! Leihe
mir dein Ohr, Sohn Zippors! \bibleverse{19}Gott ist nicht ein Mensch,
daß er lüge, noch ein Menschenkind, daß ihn etwas gereue: sollte er
etwas sagen und es nicht ausführen? Sollte er etwas verheißen und es
nicht erfüllen? \bibleverse{20}Siehe, zu segnen ist mir aufgetragen, und
hat er gesegnet, so kann ich's nicht ändern. \bibleverse{21}Man erblickt
kein Unheil in Jakob und gewahrt kein Leid in Israel: der HERR, sein
Gott, ist mit ihm, und Königsjubel erschallt in seiner Mitte.
\bibleverse{22}Gott, der sie aus Ägypten geführt, ist ihm wie die Hörner
eines Wildstiers! \bibleverse{23}Ja, keine Zauberei haftet an
Jakob\textless sup title=``=~hilft gegen Jakob''\textgreater✲ und keine
Beschwörung gegen Israel. Jetzt sagt man von Jakob und Israel: ›Wie
Großes hat Gott vollführt!‹ \bibleverse{24}Sieh, welch ein Volk! Es
steht auf wie die Löwin und erhebt sich wie der Leu; es legt sich nicht
nieder, ehe es Raub verzehrt und das Blut der Erschlagnen getrunken.«

\bibleverse{25}Da sagte Balak zu Bileam: »Wenn du es nicht verfluchen
willst, so brauchst du es doch nicht zu segnen!« \bibleverse{26}Bileam
aber antwortete dem Balak: »Habe ich dir nicht bestimmt erklärt: Alles,
was der HERR (mir) mitteilen\textless sup title=``oder:
gebieten''\textgreater✲ wird, das werde ich tun?«

\hypertarget{ii-die-vorbereitungen-zu-der-dritten-gottesoffenbarung-der-dritte-spruch-bileams}{%
\subparagraph{ii) Die Vorbereitungen zu der dritten Gottesoffenbarung;
der dritte Spruch
Bileams}\label{ii-die-vorbereitungen-zu-der-dritten-gottesoffenbarung-der-dritte-spruch-bileams}}

\bibleverse{27}Nun sagte Balak zu Bileam: »Komm, ich will dich an einen
andern Ort mitnehmen; vielleicht gefällt es Gott dann, daß du mir das
Volk von dort aus verfluchst.« \bibleverse{28}So nahm denn Balak den
Bileam mit sich auf den Gipfel des Peor, der über die weite Einöde
emporragt. \bibleverse{29}Dort sagte Bileam zu Balak: »Baue mir hier
sieben Altäre und stelle mir hier sieben junge Stiere und sieben Widder
bereit.« \bibleverse{30}Balak kam der Aufforderung Bileams nach und
opferte einen Stier und einen Widder auf jedem Altar.

\hypertarget{section-23}{%
\section{24}\label{section-23}}

\bibleverse{1}Da Bileam aber erkannte, daß es der Wille des HERRN war,
Israel zu segnen, ging er nicht, wie die vorigen Male, auf Wahrzeichen
aus, sondern wandte sein Gesicht nach der Wüste hin.

\hypertarget{jj-bileam-segnet-israel-vom-berge-peor-herab}{%
\subparagraph{jj) Bileam segnet Israel vom Berge Peor
herab}\label{jj-bileam-segnet-israel-vom-berge-peor-herab}}

\bibleverse{2}Als er nun seine Augen aufschlug und Israel nach seinen
Stämmen gelagert sah, da kam der Geist Gottes über ihn,
\bibleverse{3}und er trug seinen Spruch folgendermaßen vor: »So spricht
Bileam, der Sohn Beors, und so spricht der Mann, dessen Auge geschlossen
ist; \bibleverse{4}so spricht der, der Gottes Worte vernimmt und die
Gedanken des Höchsten kennt, der die Offenbarungen des Allmächtigen
schaut, der hingesunken ist und dessen Augen enthüllt sind:
\bibleverse{5}Wie schön sind deine Zelte, Jakob, deine Wohnungen,
Israel! \bibleverse{6}Wie Täler, die sich weithin dehnen, wie Gärten an
einem Strom, wie Aloebäume, die der HERR gepflanzt, wie Zedern am
Wasser! \bibleverse{7}Wasser rinnt aus seinen Eimern, und seine Saat ist
reichlich getränkt. Sein König ist mächtiger als Agag\textless sup
title=``vgl. 1.Sam 15,8.20.32.33''\textgreater✲, und sein Königtum
steigt stolz empor. \bibleverse{8}Gott, der aus Ägypten es geführt, ist
ihm wie die Hörner eines Wildstiers. Es frißt die ihm feindlichen Völker
und zermalmt ihre Gebeine, es zerschmettert sie mit seinen Pfeilen✲.
\bibleverse{9}Es hat sich hingestreckt, liegt da wie ein Leu und wie
eine Löwin: wer will\textless sup title=``oder: darf''\textgreater✲ es
aufstören? Wer dich segnet, ist gesegnet, und wer dir flucht, ist
verflucht!«

\hypertarget{kk-balaks-zorn-und-bileams-entschuldigung}{%
\subparagraph{kk) Balaks Zorn und Bileams
Entschuldigung}\label{kk-balaks-zorn-und-bileams-entschuldigung}}

\bibleverse{10}Da geriet Balak in Zorn gegen Bileam, so daß er die Hände
zusammenschlug. Dann sagte Balak zu Bileam: »Um meine Feinde zu
verfluchen, habe ich dich holen lassen, und nun hast du sie sogar
gesegnet, nun schon dreimal! \bibleverse{11}Kehre jetzt nur sofort in
deine Heimat zurück! Ich hatte gedacht, dich hoch zu ehren\textless sup
title=``oder: reichlich zu belohnen''\textgreater✲; doch nun hat der
HERR dich um die Ehre\textless sup title=``oder: den Lohn''\textgreater✲
gebracht!« \bibleverse{12}Da antwortete Bileam dem Balak: »Habe ich
nicht schon deinen Boten, die du an mich gesandt hattest, ausdrücklich
erklärt✲: \bibleverse{13}›Wenn Balak mir auch sein ganzes Haus voll
Silber und Gold geben wollte, so vermöchte ich doch den Befehl des HERRN
nicht zu übertreten, um nach eigenem Belieben etwas zu tun, sei es zum
Guten oder zum Bösen; sondern nur, was der HERR mir eingäbe, das würde
ich kundtun‹? \bibleverse{14}Weil ich denn jetzt zu meinem Volk
zurückkehre, so komm: ich will dir verkünden, was dieses Volk deinem
Volk in künftigen Zeiten antun wird!«

\hypertarget{ll-der-vierte-spruch-bileams-der-stern-aus-jakob-dessen-sieg-uxfcber-moab-und-edom}{%
\subparagraph{ll) Der vierte Spruch Bileams: der Stern aus Jakob; dessen
Sieg über Moab und
Edom}\label{ll-der-vierte-spruch-bileams-der-stern-aus-jakob-dessen-sieg-uxfcber-moab-und-edom}}

\bibleverse{15}Hierauf trug er seinen Spruch folgendermaßen vor: »So
spricht Bileam, der Sohn Beors, und so spricht der Mann, dessen Auge
geschlossen ist; \bibleverse{16}So spricht der, welcher Gottes Worte
vernimmt und die Gedanken des Höchsten kennt, der die Offenbarungen des
Allmächtigen schaut, der hingesunken ist und dessen Augen enthüllt sind:
\bibleverse{17}Ich sehe ihn, doch nicht schon jetzt, ich gewahre ihn,
doch noch nicht in der Nähe; es geht ein Stern aus Jakob auf, und ein
Herrscherstab ersteht\textless sup title=``oder: erhebt
sich''\textgreater✲ aus Israel, der zerschmettert die Schläfen Moabs,
den Scheitel aller Söhne Seths. \bibleverse{18}Und Edom wird
sein\textless sup title=``d.h. Jakobs''\textgreater✲ Eigentum werden und
Seir sein Eigentum, sie, seine Feinde; Israel aber wird große Taten
verrichten. \bibleverse{19}Von Jakob wird der Herrscher ausgehn und
Entronnene\textless sup title=``oder: die Flüchtlinge''\textgreater✲
vertilgen aus den Städten.«

\hypertarget{mm-die-spruxfcche-uxfcber-die-amalekiter-keniter-und-assur-nebst-eber-schluuxdf-der-bileamgeschichte}{%
\subparagraph{mm) Die Sprüche über die Amalekiter, Keniter und Assur
(nebst Eber); Schluß der
Bileamgeschichte}\label{mm-die-spruxfcche-uxfcber-die-amalekiter-keniter-und-assur-nebst-eber-schluuxdf-der-bileamgeschichte}}

\bibleverse{20}Als er dann die Amalekiter erblickte, trug er seinen
Spruch folgendermaßen vor: »Der Völker erstes ist Amalek, doch sein Ende
fällt dem Untergang anheim!« \bibleverse{21}Als er dann die
Keniter\textless sup title=``1.Sam 15,6''\textgreater✲ erblickte, trug
er seinen Spruch folgendermaßen vor: »Fest ist dein Wohnsitz, (Kain,)
und auf Felsen gebaut dein Nest; \bibleverse{22}gleichwohl ist Kain dem
Untergang geweiht: wie lange noch, so führt Assur dich in
Gefangenschaft!« \bibleverse{23}Dann trug er nochmals seinen Spruch
folgendermaßen vor: »Wehe! Wer wird\textless sup title=``oder:
möchte''\textgreater✲ am Leben bleiben, wenn Gott dies eintreten läßt?
\bibleverse{24}Denn Schiffe kommen vom Strand der Kittäer\textless sup
title=``d.h. von Zypern''\textgreater✲, die demütigen Assur und
demütigen Eber; doch auch der wird dem Untergang verfallen!«

\bibleverse{25}Hierauf machte Bileam sich auf den Weg und kehrte in
seine Heimat zurück; und auch Balak ging seines Weges.

\hypertarget{b-versuxfcndigung-und-bestrafung-der-israeliten-in-sittim-im-lande-der-moabiter-des-pinehas-eifer-und-belohnung}{%
\paragraph{b) Versündigung und Bestrafung der Israeliten in Sittim (im
Lande der Moabiter?); des Pinehas Eifer und
Belohnung}\label{b-versuxfcndigung-und-bestrafung-der-israeliten-in-sittim-im-lande-der-moabiter-des-pinehas-eifer-und-belohnung}}

\hypertarget{aa-israels-verschuldung-durch-unzucht-und-abguxf6tterei-abfall-zu-baal-peor}{%
\subparagraph{aa) Israels Verschuldung durch Unzucht und Abgötterei
(Abfall zu
Baal-Peor)}\label{aa-israels-verschuldung-durch-unzucht-und-abguxf6tterei-abfall-zu-baal-peor}}

\hypertarget{section-24}{%
\section{25}\label{section-24}}

\bibleverse{1}Als aber die Israeliten sich in Sittim niedergelassen
hatten, fing das Volk an, mit den Moabitinnen Unzucht zu treiben.
\bibleverse{2}Diese luden das Volk zu den Opferfesten ihrer Götter ein,
und das Volk nahm an ihren Opfermahlen teil und betete ihre Götter an.
\bibleverse{3}Als nun die Israeliten sich so an Baal-Peor gehängt
hatten, entbrannte der Zorn des HERRN gegen Israel, \bibleverse{4}so daß
der HERR dem Mose gebot: »Nimm alle Häupter des Volkes und hänge sie
angesichts der Sonne auf, damit der lodernde Zorn des HERRN sich von
Israel abwende.« \bibleverse{5}Da befahl Mose den Richtern\textless sup
title=``oder: Oberen''\textgreater✲ der Israeliten: »Tötet ein jeder
diejenigen von seinen Leuten, die sich an Baal-Peor gehängt haben!«

\hypertarget{bb-das-einschreiten-des-pinehas-seine-belehnung-durch-gott-mit-einem-ewigen-priestertum}{%
\subparagraph{bb) Das Einschreiten des Pinehas; seine Belehnung durch
Gott mit einem ewigen
Priestertum}\label{bb-das-einschreiten-des-pinehas-seine-belehnung-durch-gott-mit-einem-ewigen-priestertum}}

\bibleverse{6}Da kam gerade einer von den Israeliten (ins Lager) und
brachte eine Midianitin zu seinen Volksgenossen mit vor den Augen Moses
und vor den Augen der ganzen Gemeinde der Israeliten, während diese am
Eingang des Offenbarungszeltes wehklagten. \bibleverse{7}Als Pinehas,
der Sohn des Priesters Eleasar, des Sohnes Aarons, das sah, trat er aus
der Mitte der Gemeinde heraus, nahm einen Speer in seine Hand,
\bibleverse{8}ging dann dem Israeliten in das Schlafgemach nach und
durchbohrte beide, den israelitischen Mann und das Weib, und zwar
letzteres durch ihren Unterleib. Da wurde dem Sterben unter den
Israeliten Einhalt getan. \bibleverse{9}Es belief sich aber die Zahl
derer, die durch das Sterben umgekommen waren, auf 24000.

\bibleverse{10}Hierauf sagte der HERR zu Mose: \bibleverse{11}»Pinehas,
der Sohn des Priesters Eleasar, des Sohnes Aarons, hat meinen Zorn von
den Israeliten dadurch abgewandt, daß er denselben Eifer, der mir eigen
ist, unter ihnen bewiesen hat; darum habe ich die Israeliten trotz
meines Eifers nicht ganz vertilgt. \bibleverse{12}So mache denn bekannt:
Ich schließe hierdurch mit ihm meinen Bund, daß ihm Heil widerfahren
soll; \bibleverse{13}und zwar soll ihm und seinen Nachkommen nach ihm
das Priestertum auf ewige Zeiten zustehen, zum Lohn dafür, daß er für
seinen Gott geeifert und den Israeliten Sühne erwirkt hat.«
\bibleverse{14}Der damals getötete Israelit aber, der mitsamt der
Midianitin getötet worden war, hieß Simri; er war der Sohn Salus und das
Haupt eines Geschlechts\textless sup title=``d.h. ein
Stammesfürst''\textgreater✲ der Simeoniten; \bibleverse{15}die damals
getötete Midianitin aber hieß Kosbi; sie war die Tochter Zurs, des
Hauptes einer Familie, eines midianitischen Geschlechts.

\hypertarget{cc-gottes-gebot-an-den-midianitern-rache-zu-nehmen}{%
\subparagraph{cc) Gottes Gebot, an den Midianitern Rache zu
nehmen}\label{cc-gottes-gebot-an-den-midianitern-rache-zu-nehmen}}

\bibleverse{16}Darauf gebot der HERR dem Mose folgendes:
\bibleverse{17}»Behandelt die Midianiter als Feinde und erschlagt sie!
\bibleverse{18}Denn sie haben feindlich gegen euch gehandelt durch ihre
Ränke\textless sup title=``oder: Arglist''\textgreater✲, die sie gegen
euch verübt haben in betreff des Peor und in betreff ihrer Landsmännin
Kosbi, der Tochter eines midianitischen Fürsten, die am Tage des wegen
des Peor verhängten Sterbens getötet worden ist.«

\hypertarget{c-die-zweite-zuxe4hlung-des-volkes-in-der-ebene-der-moabiter-behufs-der-verteilung-des-landes}{%
\paragraph{c) Die zweite Zählung des Volkes in der Ebene der Moabiter
behufs der Verteilung des
Landes}\label{c-die-zweite-zuxe4hlung-des-volkes-in-der-ebene-der-moabiter-behufs-der-verteilung-des-landes}}

\hypertarget{section-25}{%
\section{26}\label{section-25}}

\bibleverse{1}Nachdem aber das Sterben zu Ende war, gebot der HERR dem
Mose und Eleasar, dem Sohne des Priesters Aaron, folgendes:
\bibleverse{2}»Stellt die Kopfzahl der ganzen Gemeinde der Israeliten
fest, von zwanzig Jahren an und darüber, Geschlecht für Geschlecht,
alle, die zum Kriegsdienst in Israel tauglich sind!« \bibleverse{3}Da
nahmen Mose und der Priester Eleasar die Musterung vor in den Steppen
der Moabiter am Jordan, Jericho gegenüber, \bibleverse{4}von zwanzig
Jahren an und darüber, wie der HERR dem Mose geboten hatte.

\hypertarget{aa-die-ergebnisse-der-zuxe4hlung}{%
\subparagraph{aa) Die Ergebnisse der
Zählung}\label{aa-die-ergebnisse-der-zuxe4hlung}}

Es waren aber die Israeliten, die aus Ägypten ausgezogen waren:
\bibleverse{5}Ruben, der Erstgeborene Israels. Die Söhne Rubens waren:
Hanok, von dem das Geschlecht der Hanokiten stammt; von Pallu das
Geschlecht der Palluiten; \bibleverse{6}von Hezron das Geschlecht der
Hezroniten; von Karmi das Geschlecht der Karmiten. \bibleverse{7}Dies
sind die Geschlechter der Rubeniten, und die Zahl ihrer Gemusterten
betrug 43730. \bibleverse{8}Der Sohn Pallus war Eliab; \bibleverse{9}und
die Söhne Eliabs: Nemuel, Dathan und Abiram. Dieser Dathan und Abiram
waren die zur Gemeindeversammlung Berufenen, die sich gegen Mose und
Aaron mit der Rotte Korahs aufgelehnt hatten, als sie sich gegen den
HERRN auflehnten, \bibleverse{10}worauf die Erde ihren Mund auftat und
sie samt Korah verschlang, während die Rotte dadurch umkam, daß das
Feuer die 250 Männer verzehrte, so daß sie zu einem abschreckenden
Beispiel wurden. \bibleverse{11}Die Söhne Korahs aber waren nicht mit
umgekommen.

\bibleverse{12}Die Söhne Simeons nach ihren Geschlechtern waren diese:
von Nemuel stammte das Geschlecht der Nemueliten; von Jamin das
Geschlecht der Jaminiten; von Jachin das Geschlecht der Jachiniten;
\bibleverse{13}von Serah das Geschlecht der Sarchiten; von Saul das
Geschlecht der Sauliten. \bibleverse{14}Dies sind die Geschlechter der
Simeoniten: 22200.

\bibleverse{15}Die Söhne Gads nach ihren Geschlechtern waren: von Zephon
das Geschlecht der Zephoniten; von Haggi das Geschlecht der Haggiten;
von Suni das Geschlecht der Suniten; \bibleverse{16}von Osni das
Geschlecht der Osniten; von Eri das Geschlecht der Eriten;
\bibleverse{17}von Arod das Geschlecht der Aroditen; von Areli das
Geschlecht der Areliten. \bibleverse{18}Dies sind die Geschlechter der
Söhne Gads, soviele von ihnen gemustert wurden: 40500.

\bibleverse{19}Die Söhne Judas waren: Er und Onan, die beide im Lande
Kanaan starben. \bibleverse{20}Es waren aber die Söhne Judas nach ihren
Geschlechtern: von Sela das Geschlecht der Selaniten; von Perez das
Geschlecht der Parziten; von Serah das Geschlecht der Sariten.
\bibleverse{21}Die Söhne des Perez aber waren: von Hezron das Geschlecht
der Hezroniten; von Hamul das Geschlecht der Hamuliten.
\bibleverse{22}Dies sind die Geschlechter Judas, soviele von ihnen
gemustert wurden: 76500.

\bibleverse{23}Die Söhne Issaschars nach ihren Geschlechtern waren: von
Thola das Geschlecht der Tholaiten; von Puwwa das Geschlecht der
Puwwiten; \bibleverse{24}von Jasub das Geschlecht der Jasubiten; von
Simron das Geschlecht der Simroniten. \bibleverse{25}Dies sind die
Geschlechter Issaschars, soviele von ihnen gemustert wurden: 64300.

\bibleverse{26}Die Söhne Sebulons nach ihren Geschlechtern waren: von
Sered das Geschlecht der Sarditen; von Elon das Geschlecht der Eloniten;
von Jahleel das Geschlecht der Jahleeliten. \bibleverse{27}Dies sind die
Geschlechter der Sebuloniten, soviele von ihnen gemustert wurden: 60500.

\bibleverse{28}Die Söhne Josephs nach ihren Geschlechtern waren: Manasse
und Ephraim. \bibleverse{29}Die Söhne Manasses waren: von Machir das
Geschlecht der Machiriten. Machir war der Vater Gileads; von Gilead
stammt das Geschlecht der Gileaditen. \bibleverse{30}Dies sind die Söhne
Gileads: von Jeser stammt das Geschlecht der Jesriten; von Helek das
Geschlecht der Helkiten; \bibleverse{31}von Asriel das Geschlecht der
Asrieliten; von Sichem das Geschlecht der Sichmiten; \bibleverse{32}von
Semida das Geschlecht der Semidaiten; von Hepher das Geschlecht der
Hephriten. \bibleverse{33}Zelophhad aber, der Sohn Hephers, hatte keine
Söhne, sondern nur Töchter, die hießen Mahla und Noa, Hogla, Milka und
Thirza. \bibleverse{34}Dies sind die Geschlechter Manasses, soviele von
ihnen gemustert wurden: 52700.

\bibleverse{35}Dies waren die Söhne Ephraims nach ihren Geschlechtern:
von Suthelah das Geschlecht der Suthalhiten; von Becher das Geschlecht
der Bachriten; von Thachan das Geschlecht der Thachaniten.
\bibleverse{36}Und dies waren die Söhne Suthelahs: von Eran das
Geschlecht der Eraniten. \bibleverse{37}Dies sind die Geschlechter der
Söhne Ephraims, soviele von ihnen gemustert wurden: 32500. Dies sind die
Söhne Josephs nach ihren Geschlechtern.

\bibleverse{38}Die Söhne Benjamins nach ihren Geschlechtern waren: von
Bela das Geschlecht der Baliten; von Asbel das Geschlecht der Asbeliten;
von Ahiram das Geschlecht der Ahiramiten; \bibleverse{39}von Supham das
Geschlecht der Suphamiten; von Hupham das Geschlecht der Huphamiten.
\bibleverse{40}Und die Söhne Belas waren: Ard und Naaman; von Ard stammt
das Geschlecht der Arditen, von Naaman das Geschlecht der Naamaniten.
\bibleverse{41}Dies sind die Söhne Benjamins nach ihren Geschlechtern,
soviele von ihnen gemustert wurden: 45600.

\bibleverse{42}Dies waren die Söhne Dans nach ihren Geschlechtern: von
Suham das Geschlecht der Suhamiten. \bibleverse{43}Dies sind die Söhne
Dans nach ihren Geschlechtern. Alle Geschlechter der Suhamiten, soviele
von ihnen gemustert wurden, beliefen sich auf 64400.

\bibleverse{44}Die Söhne Assers nach ihren Geschlechtern waren: von
Jimna das Geschlecht der Jimniten; von Jiswi das Geschlecht der
Jiswiten; von Beria das Geschlecht der Beriiten. \bibleverse{45}Von den
Söhnen Berias: von Heber das Geschlecht der Hebriten; von Malkiel das
Geschlecht der Malkieliten. \bibleverse{46}Und die Tochter Assers hieß
Serah. \bibleverse{47}Dies sind die Geschlechter der Söhne Assers,
soviele von ihnen gemustert wurden: 53400.

\bibleverse{48}Die Söhne Naphthalis nach ihren Geschlechtern waren: von
Jahzeel das Geschlecht der Jahzeeliten; von Guni das Geschlecht der
Guniten; \bibleverse{49}von Jezer das Geschlecht der Jizriten; von
Sillem das Geschlecht der Sillemiten. \bibleverse{50}Dies sind die Söhne
Naphthalis nach ihren Geschlechtern; und ihre Gemusterten beliefen sich
auf 45400.

\bibleverse{51}Dies ist die Gesamtzahl der gemusterten Israeliten:
601730.

\hypertarget{bb-anweisung-betreffs-der-landverteilung}{%
\subparagraph{bb) Anweisung betreffs der
Landverteilung}\label{bb-anweisung-betreffs-der-landverteilung}}

\bibleverse{52}Hierauf sagte der HERR zu Mose folgendes:
\bibleverse{53}»Unter diese soll das Land als erblicher Besitz nach der
Kopfzahl verteilt werden; \bibleverse{54}den größeren Stämmen sollst du
einen größeren Erbbesitz geben, dagegen den kleineren einen weniger
großen Erbbesitz zuteilen; jedem Stamme soll sein Erbbesitz nach der
Zahl der aus ihm Gemusterten zugeteilt werden. \bibleverse{55}Doch soll
die Verteilung des Landes durch das Los erfolgen: nach den Namen ihrer
väterlichen Stämme sollen sie es in Besitz nehmen; \bibleverse{56}nach
der Entscheidung des Loses soll der Erbbesitz zwischen den größeren und
den kleineren Stämmen verteilt werden.«

\hypertarget{cc-die-zuxe4hlung-der-leviten}{%
\subparagraph{cc) Die Zählung der
Leviten}\label{cc-die-zuxe4hlung-der-leviten}}

\bibleverse{57}Und folgendes sind die Leviten, soviele von ihnen nach
ihren Geschlechtern gemustert wurden: von Gerson das Geschlecht der
Gersoniten, von Kehath das Geschlecht der Kehathiten, von Merari das
Geschlecht der Merariten. \bibleverse{58}Dies sind die Geschlechter
Levis: das Geschlecht der Libniten, das Geschlecht der Hebroniten, das
Geschlecht der Mahliten, das Geschlecht der Musiten, das Geschlecht der
Korhiten. Kehath aber war der Vater Amrams. \bibleverse{59}Und die Frau
Amrams hieß Jochebed, eine Tochter Levis, die dem Levi in Ägypten
geboren war; diese gebar dem Amram Aaron und Mose und deren Schwester
Mirjam. \bibleverse{60}Dem Aaron aber wurden Nadab und Abihu, Eleasar
und Ithamar geboren; \bibleverse{61}aber Nadab und Abihu kamen ums
Leben, als sie ein ungehöriges Feueropfer vor dem HERRN darbrachten.
\bibleverse{62}Es belief sich aber die Zahl der aus ihnen Gemusterten
auf 23000 Seelen, alle männlichen Personen von einem Monat an und
darüber; sie waren nämlich nicht mit unter den übrigen Israeliten
gemustert worden, weil ihnen kein Erbbesitz inmitten der Israeliten
zugeteilt wurde.

\hypertarget{dd-schluuxdfsatz-mit-ruxfcckblick}{%
\subparagraph{dd) Schlußsatz mit
Rückblick}\label{dd-schluuxdfsatz-mit-ruxfcckblick}}

\bibleverse{63}Dies war die Musterung, die Mose und der Priester Eleasar
bei den Israeliten in den Steppen der Moabiter am Jordan, Jericho
gegenüber, vorgenommen haben. \bibleverse{64}Unter diesen befand sich
aber kein einziger Mann mehr von denen, die von Mose und dem Priester
Aaron einst in der Wüste am Sinai gemustert worden waren.
\bibleverse{65}Der HERR hatte ihnen ja angekündigt, daß sie in der Wüste
sterben sollten. So war denn kein einziger von ihnen übriggeblieben
außer Kaleb, dem Sohn Jephunnes, und Josua, dem Sohne Nuns.

\hypertarget{d-bestimmungen-bezuxfcglich-des-grundbesitzes-der-erbtuxf6chter}{%
\paragraph{d) Bestimmungen bezüglich des Grundbesitzes der
Erbtöchter}\label{d-bestimmungen-bezuxfcglich-des-grundbesitzes-der-erbtuxf6chter}}

\hypertarget{section-26}{%
\section{27}\label{section-26}}

\bibleverse{1}Da traten herzu die Töchter Zelophhads, des Sohnes
Hephers, des Sohnes Gileads, des Sohnes Machirs, des Sohnes Manasses,
aus den Geschlechtern Manasses, des Sohnes Josephs; und die Namen seiner
Töchter waren Mahla, Noa, Hogla, Milka und Thirza. \bibleverse{2}Die
traten also vor Mose und vor den Priester Eleasar, vor die
Stammesfürsten und die ganze Gemeinde am Eingang des Offenbarungszeltes
und sagten: \bibleverse{3}»Unser Vater ist in der Wüste gestorben, hat
aber nicht zu der Rotte derer gehört, die sich gegen den HERRN
zusammengetan haben, zu der Rotte Korahs, sondern er ist wegen seiner
eigenen Sündhaftigkeit gestorben, ohne Söhne zu hinterlassen.
\bibleverse{4}Warum soll nun der Name unseres Vaters aus seinem
Geschlecht gestrichen werden✲, weil er keinen Sohn gehabt hat? Gib uns
einen Erbbesitz unter den Geschlechtsgenossen unseres Vaters!«

\bibleverse{5}Als nun Mose ihre Rechtssache dem HERRN (zur Entscheidung)
vorlegte, \bibleverse{6}gab der HERR dem Mose folgenden Bescheid:
\bibleverse{7}»Die Töchter Zelophhads haben recht; du sollst ihnen daher
unweigerlich einen Erbbesitz unter den Geschlechtsgenossen ihres Vaters
geben und sollst den ihrem Vater zustehenden Erbbesitz auf sie übergehen
lassen. \bibleverse{8}Den Israeliten aber sollst du folgende Verordnung
kundtun: ›Wenn ein Mann stirbt, ohne einen Sohn zu hinterlassen, so
sollt ihr seinen Erbbesitz auf seine Tochter übergehen lassen;
\bibleverse{9}und wenn er keine Tochter hat, so sollt ihr seinen
Erbbesitz seinen Brüdern geben; \bibleverse{10}wenn er auch keine Brüder
hat, so sollt ihr seinen Erbbesitz den Brüdern seines Vaters geben;
\bibleverse{11}und wenn sein Vater keine Brüder hat, so sollt ihr seinen
Erbbesitz seinem nächsten Blutsverwandten aus seinem Geschlecht geben,
damit er ihn in Besitz nimmt.‹« Dies soll für die Israeliten eine
Rechtsbestimmung sein, wie der HERR dem Mose geboten hat.

\hypertarget{e-ankuxfcndigung-des-bevorstehenden-todes-an-mose-einsetzung-josuas-zu-seinem-nachfolger}{%
\paragraph{e) Ankündigung des bevorstehenden Todes an Mose; Einsetzung
Josuas zu seinem
Nachfolger}\label{e-ankuxfcndigung-des-bevorstehenden-todes-an-mose-einsetzung-josuas-zu-seinem-nachfolger}}

\bibleverse{12}Hierauf gebot der HERR dem Mose\textless sup title=``vgl.
5.Mose 32,48-52''\textgreater✲: »Steige auf das Gebirge Abarim hier und
sieh dir das Land an, das ich für die Israeliten bestimmt habe.
\bibleverse{13}Wenn du es dir angeschaut hast, sollst auch du zu deinen
Volksgenossen versammelt werden, wie dein Bruder Aaron zu ihnen bereits
versammelt worden ist, \bibleverse{14}weil ihr in der Wüste Zin, als die
Gemeinde mit mir haderte, meinem Geheiß, mich durch Beschaffung von
Wasser vor ihnen zu verherrlichen, nicht nachgekommen seid« -- das
bezieht sich auf das Haderwasser von Kades in der Wüste Zin\textless sup
title=``vgl. 20,1-13''\textgreater✲. \bibleverse{15}Da antwortete Mose
dem HERRN folgendermaßen: \bibleverse{16}»Gott, der HERR über Leben und
Tod alles Fleisches, wolle einen Mann über die Gemeinde bestellen,
\bibleverse{17}der an ihrer Spitze aus- und einziehe, der sie ins Feld
hinausführe und wieder zurückführe, damit die Gemeinde des HERRN nicht
wie eine Herde ohne Hirten sei!« \bibleverse{18}Da gebot der HERR dem
Mose: »Nimm Josua zu dir, den Sohn Nuns, einen Mann, in dem (mein) Geist
wohnt, und lege deine Hand fest auf ihn; \bibleverse{19}laß ihn dann vor
den Priester Eleasar und vor die ganze Gemeinde treten, setze ihn vor
ihren Augen in sein Amt ein \bibleverse{20}und teile ihm etwas von
deiner Hoheit\textless sup title=``oder: Würde''\textgreater✲ mit, damit
die ganze Gemeinde der Israeliten ihm gehorsam ist. \bibleverse{21}Er
soll sich dann (bei jeder Gelegenheit) an den Priester Eleasar wenden,
damit dieser für ihn das Urim-Orakel\textless sup title=``vgl. 2.Mose
28,29-30''\textgreater✲ vor dem HERRN befragt: nach dessen Weisung
sollen sie zum Kampf ausziehen und ebenso nach dessen Weisung wieder
heimkehren, er und alle Israeliten mit ihm, überhaupt die ganze
Gemeinde.« \bibleverse{22}Mose tat, wie der HERR ihm geboten hatte: er
nahm Josua, ließ ihn vor den Priester Eleasar und vor die ganze Gemeinde
treten, \bibleverse{23}legte dann seine Hände fest auf ihn und setzte
ihn so in sein Amt ein, wie der HERR es durch Mose angeordnet hatte.

\hypertarget{f-vorschriften-bezuxfcglich-der-tuxe4glichen-und-der-festtuxe4glichen-gemeindeopfer}{%
\paragraph{f) Vorschriften bezüglich der täglichen und der festtäglichen
Gemeindeopfer}\label{f-vorschriften-bezuxfcglich-der-tuxe4glichen-und-der-festtuxe4glichen-gemeindeopfer}}

\hypertarget{section-27}{%
\section{28}\label{section-27}}

\bibleverse{1}Weiter gebot der HERR dem Mose folgendes:
\bibleverse{2}»Teile den Israeliten folgende Verordnungen mit: Ihr sollt
darauf achten, meine Opfergaben, meine Speise in Gestalt der mir
zukommenden Feueropfer zu lieblichem Geruch für mich, mir zu rechter
Zeit darzubringen!«

\hypertarget{aa-das-tuxe4gliche-morgen--und-abend-brandopfer}{%
\subparagraph{aa) Das tägliche Morgen- und
Abend-Brandopfer}\label{aa-das-tuxe4gliche-morgen--und-abend-brandopfer}}

\bibleverse{3}»Sage ihnen also: Dies ist das Feueropfer, das ihr dem
HERRN darbringen sollt: täglich zwei fehlerlose, einjährige Lämmer als
regelmäßiges Brandopfer. \bibleverse{4}Das eine Lamm sollst du am Morgen
opfern und das andere Lamm gegen Abend herrichten; \bibleverse{5}dazu
als Speisopfer ein Zehntel Epha Feinmehl, das mit einem Viertel Hin Öl
von zerstoßenen Oliven gemengt ist. \bibleverse{6}Dies ist das
regelmäßige Brandopfer, das am Berge Sinai zum lieblichen Geruch als
Feueropfer für den HERRN eingesetzt\textless sup title=``oder: zum
erstenmal hergerichtet''\textgreater✲ worden ist; \bibleverse{7}dazu als
zugehöriges Trankopfer ein Viertel Hin Wein für das eine Lamm; im
Heiligtum sollst du das Trankopfer von starkem Getränk dem HERRN
spenden. \bibleverse{8}Das zweite Lamm aber sollst du gegen Abend
herrichten; mit einem Speisopfer wie am Morgen und dem zugehörigen
Trankopfer sollst du es herrichten als ein Feueropfer zu lieblichem
Geruch für den HERRN.«

\hypertarget{bb-das-zusatzopfer-am-sabbat}{%
\subparagraph{bb) Das Zusatzopfer am
Sabbat}\label{bb-das-zusatzopfer-am-sabbat}}

\bibleverse{9}»Ferner am Sabbattage: zwei fehlerlose, einjährige Lämmer
und zwei Zehntel Epha mit Öl gemengtes Feinmehl als Speisopfer nebst dem
zugehörigen Trankopfer. \bibleverse{10}Dies ist das Sabbat-Brandopfer,
das an jedem Sabbat außer dem regelmäßigen Brandopfer und dem
zugehörigen Trankopfer darzubringen ist.«

\hypertarget{cc-das-zusatzopfer-am-neumondstage}{%
\subparagraph{cc) Das Zusatzopfer am
Neumondstage}\label{cc-das-zusatzopfer-am-neumondstage}}

\bibleverse{11}»Ferner sollt ihr an jedem ersten Tage eurer Monate dem
HERRN als Brandopfer darbringen: zwei junge Stiere und einen Widder,
sieben fehlerlose, einjährige Lämmer \bibleverse{12}und zu jedem Stier
drei Zehntel Epha mit Öl gemengtes Feinmehl als Speisopfer und zu dem
einen Widder zwei Zehntel Epha mit Öl gemengtes Feinmehl als Speisopfer;
\bibleverse{13}und zu jedem Lamm ein Zehntel Epha mit Öl gemengtes
Feinmehl als Speisopfer -- das (alles) als Brandopfer zu lieblichem
Geruch, als Feueropfer für den HERRN. \bibleverse{14}Was ferner die
zugehörigen Trankopfer betrifft, so soll ein halbes Hin Wein auf jeden
Stier und ein Drittel Hin auf den Widder und ein Viertel Hin auf jedes
Lamm kommen. Das ist das an jedem Neumondstage des Jahres, Monat für
Monat, darzubringende Brandopfer. \bibleverse{15}Außerdem soll neben dem
regelmäßigen Brandopfer und dem zugehörigen Trankopfer dem HERRN ein
Ziegenbock als Sündopfer dargebracht werden.«

\hypertarget{dd-die-zusatzopfer-fuxfcr-die-sieben-tage-des-festes-der-ungesuxe4uerten-brote}{%
\subparagraph{dd) Die Zusatzopfer für die sieben Tage des Festes der
ungesäuerten
Brote}\label{dd-die-zusatzopfer-fuxfcr-die-sieben-tage-des-festes-der-ungesuxe4uerten-brote}}

\bibleverse{16}»Sodann findet am vierzehnten Tage des ersten Monats das
Passahfest zu Ehren des HERRN statt. \bibleverse{17}Und am fünfzehnten
Tage dieses Monats wird Festfeier gehalten; sieben Tage lang soll
ungesäuertes Brot gegessen werden. \bibleverse{18}Am ersten Tage findet
eine Festversammlung am Heiligtum statt; da dürft ihr keinerlei
Werktagsarbeit verrichten \bibleverse{19}und habt als Feueropfer, als
Brandopfer, für den HERRN darzubringen: zwei junge Stiere, einen Widder
und sieben einjährige Lämmer -- fehlerlose Tiere müssen es sein --;
\bibleverse{20}sodann als zugehöriges Speisopfer von Feinmehl, das mit
Öl gemengt ist: drei Zehntel Epha sollt ihr zu jedem Stiere und zwei
Zehntel zu dem Widder herrichten; \bibleverse{21}und je ein Zehntel
sollst du zu jedem Lamm von den sieben Lämmern herrichten;
\bibleverse{22}außerdem noch einen Ziegenbock zum Sündopfer, um euch
Sühne zu erwirken. \bibleverse{23}Außer dem Morgen-Brandopfer, welches
das regelmäßige Brandopfer bildet, sollt ihr dies alles darbringen.
\bibleverse{24}Dieselben Opfer sollt ihr täglich, alle sieben Tage
hindurch, als eine Feueropferspeise zu lieblichem Geruch für den HERRN
herrichten; außer dem regelmäßigen Brandopfer und dem zugehörigen
Trankopfer soll es hergerichtet werden. \bibleverse{25}Am siebten Tage
aber soll bei euch wieder eine Festversammlung am Heiligtum stattfinden;
da dürft ihr keinerlei Werktagsarbeit verrichten.«

\hypertarget{ee-die-zusatzopfer-am-fest-der-erstlinge-oder-erntefest}{%
\subparagraph{ee) Die Zusatzopfer am Fest der Erstlinge (oder:
Erntefest)}\label{ee-die-zusatzopfer-am-fest-der-erstlinge-oder-erntefest}}

\bibleverse{26}»Ferner am Tage der Erstlingsfrüchte, wenn ihr dem HERRN
ein Speisopfer vom neuen Getreide darbringt, an eurem Wochenfest, soll
bei euch eine Versammlung am Heiligtum stattfinden; da dürft ihr
keinerlei Werktagsarbeit verrichten. \bibleverse{27}Und als Brandopfer
zu lieblichem Geruch für den HERRN sollt ihr darbringen: zwei junge
Stiere, einen Widder, sieben einjährige Lämmer (lauter fehlerlose
Tiere); \bibleverse{28}dazu als zugehöriges Speisopfer Feinmehl, das mit
Öl gemengt ist, nämlich drei Zehntel zu jedem Stier, zwei Zehntel zu dem
einen Widder, \bibleverse{29}je ein Zehntel zu jedem Lamm von den sieben
Lämmern; \bibleverse{30}außerdem einen Ziegenbock, um euch Sühne zu
erwirken; \bibleverse{31}außer dem regelmäßigen Brandopfer und dem
zugehörigen Speisopfer sollt ihr dies alles herrichten -- fehlerlose
Tiere müssen es sein -- nebst den erforderlichen Trankopfern.«

\hypertarget{ff-die-zusatzopfer-am-neujahrstage}{%
\subparagraph{ff) Die Zusatzopfer am
Neujahrstage}\label{ff-die-zusatzopfer-am-neujahrstage}}

\hypertarget{section-28}{%
\section{29}\label{section-28}}

\bibleverse{1}»Ferner am ersten Tage des siebten Monats soll bei euch
eine Festversammlung am Heiligtum stattfinden; da dürft ihr keinerlei
Werktagsarbeit verrichten: der Tag des Posaunenblasens soll es euch
sein. \bibleverse{2}Da sollt ihr als Brandopfer zu lieblichem Geruch für
den HERRN darbringen: einen jungen Stier, einen Widder und sieben
fehlerlose, einjährige Lämmer; \bibleverse{3}dazu als zugehöriges
Speisopfer von Feinmehl, das mit Öl gemengt ist: drei Zehntel Epha zu
dem Stier, zwei Zehntel zu dem Widder \bibleverse{4}und ein Zehntel zu
jedem Lamm von den sieben Lämmern; \bibleverse{5}außerdem einen
Ziegenbock als Sündopfer, um euch Sühne zu erwirken; \bibleverse{6}außer
dem Neumond-Brandopfer nebst dem zugehörigen Speisopfer und außer dem
regelmäßigen Brandopfer nebst dem zugehörigen Speisopfer und den
erforderlichen Trankopfern, in der vorgeschriebenen Weise, zu lieblichem
Geruch, als ein Feueropfer für den HERRN.«

\hypertarget{gg-die-zusatzopfer-am-grouxdfen-versuxf6hnungstage}{%
\subparagraph{gg) Die Zusatzopfer am großen
Versöhnungstage}\label{gg-die-zusatzopfer-am-grouxdfen-versuxf6hnungstage}}

\bibleverse{7}»Ferner am zehnten Tage desselben siebten Monats soll bei
euch eine Festversammlung am Heiligtum stattfinden, und ihr sollt
fasten\textless sup title=``vgl. 3.Mose 16,29''\textgreater✲; keinerlei
Werktagsarbeit dürft ihr da verrichten. \bibleverse{8}Als Brandopfer
sollt ihr dabei für den HERRN zu lieblichem Geruch herrichten: einen
jungen Stier, einen Widder, sieben einjährige Lämmer -- fehlerlose Tiere
müssen es sein --; \bibleverse{9}außerdem als zugehöriges Speisopfer von
Feinmehl, das mit Öl gemengt ist: drei Zehntel zu dem Stier, zwei
Zehntel zu dem einen Widder, \bibleverse{10}je ein Zehntel zu jedem Lamm
von den sieben Lämmern; \bibleverse{11}auch einen Ziegenbock als
Sündopfer, außer dem Versöhnungs-Sündopfer und dem regelmäßigen
Brandopfer nebst dem zugehörigen Speisopfer und den erforderlichen
Trankopfern.«

\hypertarget{hh-die-zusatzopfer-fuxfcr-die-sieben-tage-des-laubhuxfcttenfestes}{%
\subparagraph{hh) Die Zusatzopfer für die sieben Tage des
Laubhüttenfestes}\label{hh-die-zusatzopfer-fuxfcr-die-sieben-tage-des-laubhuxfcttenfestes}}

\bibleverse{12}»Ferner am fünfzehnten Tage des siebten Monats soll bei
euch eine Festversammlung am Heiligtum stattfinden; da dürft ihr
keinerlei Werktagsarbeit verrichten, sondern sollt dem HERRN ein Fest
sieben Tage lang feiern. \bibleverse{13}Dabei sollt ihr als Brandopfer,
als Feueropfer zu lieblichem Geruch für den HERRN, darbringen: dreizehn
junge Stiere, zwei Widder, vierzehn einjährige Lämmer -- fehlerlose
Tiere müssen es sein --; \bibleverse{14}dazu als zugehöriges Speisopfer
von Feinmehl, das mit Öl gemengt ist: drei Zehntel zu jedem von den
dreizehn Stieren, zwei Zehntel zu jedem von den beiden Widdern
\bibleverse{15}und je ein Zehntel zu jedem Lamm von den vierzehn
Lämmern; \bibleverse{16}auch einen Ziegenbock als Sündopfer, außer dem
regelmäßigen Brandopfer nebst dem zugehörigen Speisopfer und dem
erforderlichen Trankopfer.

\bibleverse{17}Sodann am zweiten Tage: zwölf junge Stiere, zwei Widder,
vierzehn fehlerlose, einjährige Lämmer \bibleverse{18}nebst dem
zugehörigen Speisopfer und den erforderlichen Trankopfern zu den
Stieren, zu den Widdern und zu den Lämmern, nach ihrer Zahl, der
vorgeschriebenen Weise gemäß; \bibleverse{19}auch einen Ziegenbock als
Sündopfer, außer dem regelmäßigen Brandopfer nebst dem zugehörigen
Speisopfer und den erforderlichen Trankopfern.

\bibleverse{20}Sodann am dritten Tage: elf junge Stiere, zwei Widder,
vierzehn fehlerlose, einjährige Lämmer \bibleverse{21}nebst dem
zugehörigen Speisopfer und den erforderlichen Trankopfern zu den
Stieren, zu den Widdern und zu den Lämmern, nach ihrer Zahl, der
vorgeschriebenen Weise gemäß; \bibleverse{22}auch einen Ziegenbock als
Sündopfer, außer dem regelmäßigen Brandopfer nebst dem zugehörigen
Speisopfer und dem erforderlichen Trankopfer.

\bibleverse{23}Sodann am vierten Tage: zehn junge Stiere, zwei Widder,
vierzehn fehlerlose, einjährige Lämmer \bibleverse{24}nebst dem
zugehörigen Speisopfer und den erforderlichen Trankopfern zu den
Stieren, zu den Widdern und zu den Lämmern nach ihrer Zahl, der
vorgeschriebenen Weise gemäß; \bibleverse{25}auch einen Ziegenbock als
Sündopfer, außer dem regelmäßigen Brandopfer, dem zugehörigen Speisopfer
und dem erforderlichen Trankopfer.

\bibleverse{26}Sodann am fünften Tage: neun junge Stiere, zwei Widder,
vierzehn fehlerlose, einjährige Lämmer \bibleverse{27}nebst dem
zugehörigen Speisopfer und den erforderlichen Trankopfern zu den
Stieren, zu den Widdern und zu den Lämmern, nach ihrer Zahl, der
vorgeschriebenen Weise gemäß; \bibleverse{28}auch einen Ziegenbock als
Sündopfer, außer dem regelmäßigen Brandopfer nebst dem zugehörigen
Speisopfer und dem erforderlichen Trankopfer.

\bibleverse{29}Sodann am sechsten Tage: acht junge Stiere, zwei Widder,
vierzehn fehlerlose, einjährige Lämmer \bibleverse{30}nebst dem
zugehörigen Speisopfer und den erforderlichen Trankopfern zu den
Stieren, zu den Widdern und zu den Lämmern, nach ihrer Zahl, der
vorgeschriebenen Weise gemäß; \bibleverse{31}auch einen Ziegenbock als
Sündopfer, außer dem regelmäßigen Brandopfer, dem zugehörigen Speisopfer
und den erforderlichen Trankopfern.

\bibleverse{32}Sodann am siebten Tage: sieben junge Stiere, zwei Widder,
vierzehn fehlerlose, einjährige Lämmer \bibleverse{33}nebst dem
zugehörigen Speisopfer und den erforderlichen Trankopfern zu den
Stieren, zu den Widdern und zu den Lämmern, nach ihrer Zahl, der
vorgeschriebenen Weise gemäß; \bibleverse{34}auch einen Ziegenbock als
Sündopfer, außer dem regelmäßigen Brandopfer, dem zugehörigen Speisopfer
und dem erforderlichen Trankopfer.

\bibleverse{35}Am achten Tage soll bei euch eine Festversammlung
stattfinden; da dürft ihr keinerlei Werktagsarbeit verrichten.
\bibleverse{36}Und als Brandopfer, als Feueropfer zu lieblichem Geruch
für den HERRN, sollt ihr darbringen: einen jungen Stier, einen Widder,
sieben fehlerlose, einjährige Lämmer \bibleverse{37}nebst dem
zugehörigen Speisopfer und den erforderlichen Trankopfern zu dem Stier,
zu dem Widder und zu den Lämmern, nach ihrer Zahl, der vorgeschriebenen
Weise gemäß; \bibleverse{38}auch einen Ziegenbock als Sündopfer, außer
dem regelmäßigen Brandopfer nebst dem zugehörigen Speisopfer und dem
erforderlichen Trankopfer.«

\hypertarget{ii-schluuxdfsatz-der-opfergesetze}{%
\subparagraph{ii) Schlußsatz der
Opfergesetze}\label{ii-schluuxdfsatz-der-opfergesetze}}

\bibleverse{39}»Diese Opfer sollt ihr an euren Festen für den HERRN
herrichten, abgesehen von dem, was ihr infolge von Gelübden und als
freiwillige Gaben sowohl an Brandopfern und Speisopfern als auch an
Trankopfern und Heilsopfern darbringen werdet.«

\hypertarget{section-29}{%
\section{30}\label{section-29}}

\bibleverse{1}Mose teilte dies alles den Israeliten genau so mit, wie
der HERR es ihm geboten hatte.

\hypertarget{g-vorschriften-bezuxfcglich-der-verbindlichkeit-bzw.-unguxfcltigkeit-von-geluxfcbden-besonders-von-geluxfcbden-weiblicher-personen}{%
\paragraph{g) Vorschriften bezüglich der Verbindlichkeit bzw.
Ungültigkeit von Gelübden (besonders von Gelübden weiblicher
Personen)}\label{g-vorschriften-bezuxfcglich-der-verbindlichkeit-bzw.-unguxfcltigkeit-von-geluxfcbden-besonders-von-geluxfcbden-weiblicher-personen}}

\bibleverse{2}Darauf sagte Mose zu den Stammeshäuptern der Israeliten:
»Folgendes hat der HERR geboten:

\bibleverse{3}Wenn ein Mann dem HERRN ein Gelübde ablegt oder einen Eid
schwört, durch den er sich zu einer Enthaltung verpflichtet, so darf er
sein Wort nicht brechen: genau so, wie er es ausgesprochen hat, soll er
es auch ausführen. \bibleverse{4}Wenn aber eine weibliche Person, die in
ihrer Jugend✲ im Hause ihres Vaters lebt, dem HERRN ein Gelübde ablegt
oder sich zu einer Enthaltung verpflichtet \bibleverse{5}und ihr Vater
von ihrem Gelübde oder von der Enthaltung, zu der sie sich verpflichtet
hat, Kunde erhält und dann ihr gegenüber dazu schweigt, so sollen alle
ihre Gelübde gültig sein, und auch jede Enthaltung, zu der sie sich
verpflichtet hat, soll zu Recht bestehen. \bibleverse{6}Wenn aber ihr
Vater an dem Tage, an dem er davon erfährt, ihr die Genehmigung versagt,
so sollen alle ihre Gelübde und alle ihre Enthaltungen, zu denen sie
sich verpflichtet hat, ungültig sein; und der HERR wird ihr verzeihen,
weil ihr Vater ihr die Genehmigung versagt hat.

\bibleverse{7}Wenn sie sich dann aber verheiraten sollte, während ihre
Gelübde oder ein unbedachter Ausspruch ihrer Lippen, durch den sie sich
eine Enthaltung auferlegt hat, für sie noch verbindlich sind,
\bibleverse{8}und ihr Mann erhält Kunde davon, schweigt aber ihr
gegenüber an dem Tage, an dem er davon hört, so sollen ihre Gelübde
gültig sein, und die Enthaltungen, zu denen sie sich verpflichtet hat,
sollen zu Recht bestehen. \bibleverse{9}Wenn aber ihr Mann an dem Tage,
an dem er davon erfährt, ihr die Genehmigung versagt, so macht er
dadurch das Gelübde, das auf ihr ruht, und den unbedachten Ausspruch
ihrer Lippen, durch den sie sich eine Enthaltung auferlegt hat,
ungültig, und der HERR wird ihr verzeihen. \bibleverse{10}Aber das
Gelübde einer Witwe oder einer verstoßenen✲ Frau, alles, wozu sie sich
verpflichtet hat, soll für sie verbindlich sein.

\bibleverse{11}Wenn ferner eine Frau im Hause ihres Mannes etwas gelobt
oder sich durch einen Eid zu einer Enthaltung verpflichtet hat
\bibleverse{12}und ihr Mann Kunde davon erhält, aber ihr gegenüber dazu
schweigt und ihr nicht die Genehmigung versagt, so sollen alle ihre
Gelübde gültig sein, und jede Enthaltung, zu der sie sich verpflichtet
hat, soll zu Recht bestehen. \bibleverse{13}Wenn ihr Mann aber an dem
Tage, an dem er davon hört, sie für ungültig erklärt hat, so soll nichts
von dem, was sie bezüglich ihrer Gelübde und bezüglich der
Verpflichtungen zu einer Enthaltung ausgesprochen hat, Gültigkeit haben:
ihr Mann hat sie für ungültig erklärt, darum wird der HERR ihr
verzeihen.«

\hypertarget{nochmalige-einschuxe4rfung-des-rechtes-des-gatten-schluuxdfsatz}{%
\paragraph{Nochmalige Einschärfung des Rechtes des Gatten;
Schlußsatz}\label{nochmalige-einschuxe4rfung-des-rechtes-des-gatten-schluuxdfsatz}}

\bibleverse{14}»Jedes Gelübde und jedes eidliche Versprechen einer
Enthaltung behufs einer Selbstkasteiung kann ihr Mann für gültig und
ebenso für ungültig erklären. \bibleverse{15}Wenn aber ihr Mann von
einem Tage bis zum andern stillschweigt, so bestätigt er dadurch alle
ihre Gelübde oder alle die Enthaltungen, die auf ihr ruhen; er hat sie
dadurch bestätigt, daß er ihr gegenüber an dem Tage, an welchem er Kunde
davon erhielt, geschwiegen hat. \bibleverse{16}Wenn er sie aber erst
später aufhebt, nachdem er schon vorher davon gehört hat, so hat er die
Verschuldung (seiner Frau) zu tragen.«

\bibleverse{17}Das sind die Verordnungen, die der HERR dem Mose geboten
hat, damit sie Geltung haben zwischen einem Manne und seiner Frau und
zwischen einem Vater und seiner Tochter, solange diese in ihrer
Jugend\textless sup title=``d.h. unverheiratet; V.4''\textgreater✲ im
Hause ihres Vaters lebt.

\hypertarget{h-rachekrieg-der-israeliten-gegen-die-midianiter-verteilung-der-beute}{%
\paragraph{h) Rachekrieg der Israeliten gegen die Midianiter; Verteilung
der
Beute}\label{h-rachekrieg-der-israeliten-gegen-die-midianiter-verteilung-der-beute}}

\hypertarget{section-30}{%
\section{31}\label{section-30}}

\bibleverse{1}Der HERR gebot hierauf dem Mose folgendes:
\bibleverse{2}»Nimm Rache für die Israeliten an den Midianitern! Danach
sollst du zu deinen Volksgenossen versammelt werden.« \bibleverse{3}Da
befahl Mose dem Volke folgendes: »Rüstet aus eurer Mitte Männer zu einem
Kriegszuge aus, sie sollen gegen die Midianiter ziehen, um die Rache des
HERRN an den Midianitern zu vollstrecken. \bibleverse{4}Je tausend Mann
von einem Stamm, und zwar von allen Stämmen der Israeliten, sollt ihr zu
dem Kriegszuge entsenden.« \bibleverse{5}So wurden denn aus den Stämmen
der Israeliten je tausend Mann von jedem Stamme ausgehoben, im ganzen
zwölftausend kriegsgerüstete Männer. \bibleverse{6}Diese sandte Mose,
tausend Mann von jedem Stamm, zum Kriegszuge aus und mit ihnen Pinehas,
den Sohn des Priesters Eleasar, zum Kriegszuge; der hatte die heiligen
Geräte und die Alarmtrompeten\textless sup title=``vgl.
10,1-10''\textgreater✲ bei sich. \bibleverse{7}So zogen sie denn gegen
die Midianiter zu Felde, wie der HERR dem Mose geboten hatte, und
töteten alle männlichen Personen. \bibleverse{8}Auch die Könige der
Midianiter töteten sie zu den anderen hinzu, die von ihnen erschlagen
waren, nämlich: Ewi, Rekem, Zur, Hur und Reba, die fünf Könige der
Midianiter; auch Bileam, den Sohn Beors, machten sie mit dem Schwert
nieder. \bibleverse{9}Hierauf führten die Israeliten die midianitischen
Frauen und ihre Kinder gefangen weg, erbeuteten ihr sämtliches Lastvieh
und alle ihre Herden und ihre gesamte Habe \bibleverse{10}und
verbrannten alle Ortschaften in ihren Wohnsitzen und alle ihre Zeltlager
mit Feuer. \bibleverse{11}Dann nahmen sie die gesamte Beute und alles,
was sie an Menschen und Vieh geraubt hatten, \bibleverse{12}und brachten
die Gefangenen sowie den Raub und die Beute zu Mose und zu dem Priester
Eleasar und zu der Gemeinde der Israeliten ins Lager, in die
moabitischen Steppen, die am Jordan, Jericho gegenüber, lagen.

\hypertarget{verordnung-uxfcber-die-tuxf6tung-aller-muxe4nnlichen-kinder-uxfcber-die-behandlung-der-weiblichen-gefangenen-und-kinder-und-uxfcber-die-vor-der-ruxfcckkehr-zu-vollziehende-reinigung}{%
\paragraph{Verordnung über die Tötung aller männlichen Kinder, über die
Behandlung der weiblichen Gefangenen und Kinder und über die vor der
Rückkehr zu vollziehende
Reinigung}\label{verordnung-uxfcber-die-tuxf6tung-aller-muxe4nnlichen-kinder-uxfcber-die-behandlung-der-weiblichen-gefangenen-und-kinder-und-uxfcber-die-vor-der-ruxfcckkehr-zu-vollziehende-reinigung}}

\bibleverse{13}Da gingen Mose und der Priester Eleasar und alle Fürsten
der Gemeinde ihnen entgegen vor das Lager hinaus. \bibleverse{14}Mose
aber geriet in Zorn über die Befehlshaber des Heeres, die Hauptleute der
Tausendschaften und die Hauptleute der Hundertschaften, die von dem
Feldzuge zurückkehrten; \bibleverse{15}und Mose sagte zu ihnen: »Habt
ihr denn alle Frauen am Leben gelassen? \bibleverse{16}Sie sind es ja
gerade, die auf den Rat Bileams den Israeliten ein Anlaß geworden sind,
Untreue gegen den HERRN zu begehen {[}in betreff des Peor{]}, so daß das
Sterben über die Gemeinde des HERRN kam. \bibleverse{17}So tötet nun
alle männlichen Kinder unter ihnen, und ebenso tötet alle weiblichen
Personen, die schon mit einem Manne zu tun gehabt haben;
\bibleverse{18}aber alle Kinder weiblichen Geschlechts, die noch mit
keinem Manne zu tun gehabt haben, laßt für euch am Leben.
\bibleverse{19}Ihr selbst aber müßt sieben Tage lang außerhalb des
Lagers bleiben: jeder, der einen Menschen getötet, und jeder, der einen
Erschlagenen berührt hat, muß sich am dritten und am siebten Tage
entsündigen, ihr selbst und eure Gefangenen\textless sup title=``vgl.
19,17-19''\textgreater✲. \bibleverse{20}Auch alle Kleider und alle
Gegenstände von Leder sowie alles aus Ziegenhaaren Gefertigte und alle
hölzernen Geräte müßt ihr entsündigen!«

\bibleverse{21}Hierauf sagte der Priester Eleasar zu den Kriegsleuten,
die an dem Feldzuge teilgenommen hatten: »Dies ist eine Bestimmung des
Gesetzes, das der HERR dem Mose zur Pflicht gemacht hat:
\bibleverse{22}das Gold und das Silber, das Kupfer, Eisen, Zinn und
Blei, \bibleverse{23}überhaupt alles, was das Feuer verträgt, müßt ihr
durchs Feuer gehen lassen, dann wird es rein sein; jedoch muß es auch
noch mit dem Reinigungswasser\textless sup title=``vgl.
19,18-20''\textgreater✲ entsündigt werden. Alles aber, was kein Feuer
verträgt, müßt ihr durchs Wasser gehen lassen. \bibleverse{24}Wenn ihr
also am siebten Tage eure Kleider gewaschen habt, dann werdet ihr rein
sein und dürft danach wieder ins Lager kommen.«

\hypertarget{verteilung-der-lebenden-beute-menschen-und-vieh-weihgeschenk-der-anfuxfchrer}{%
\paragraph{Verteilung der lebenden Beute (Menschen und Vieh);
Weihgeschenk der
Anführer}\label{verteilung-der-lebenden-beute-menschen-und-vieh-weihgeschenk-der-anfuxfchrer}}

\bibleverse{25}Hierauf gebot der HERR dem Mose folgendes:
\bibleverse{26}»Nimm von allem, was ihr als Beute an Menschen und an
Vieh weggeführt habt, die Gesamtzahl auf, du mit dem Priester Eleasar
und den Stammeshäuptern der Gemeinde; \bibleverse{27}und teile die Beute
zur Hälfte zwischen denen, die als Teilnehmer am Kriege ins Feld gezogen
sind, und zwischen der ganzen übrigen Gemeinde. \bibleverse{28}Dann
erhebe von den Kriegsleuten, die ins Feld gezogen sind, eine Abgabe für
den HERRN, nämlich je ein Stück von fünf Hunderten, sowohl von den
Menschen als auch von den Rindern, von den Eseln und vom Kleinvieh;
\bibleverse{29}von der ihnen zufallenden Hälfte sollt ihr sie nehmen,
und du sollst sie dem Priester Eleasar als ein Hebeopfer für den HERRN
übergeben. \bibleverse{30}Von der den (anderen) Israeliten zufallenden
Hälfte aber sollst du aus fünfzig Stück immer eins herausgreifen, sowohl
aus den Menschen als auch aus den Rindern, den Eseln und dem Kleinvieh,
also aus dem gesamten Vieh, und sollst sie den Leviten übergeben, die
den Dienst an der Wohnung des HERRN zu verrichten haben.«
\bibleverse{31}Da taten Mose und der Priester Eleasar, wie der HERR dem
Mose geboten hatte.

\bibleverse{32}Es betrug aber das Erbeutete, das, was von der Beute, die
das Kriegsvolk gemacht hatte, übriggeblieben war: 675000 Stück
Kleinvieh, \bibleverse{33}72000 Rinder \bibleverse{34}und 61000 Esel;
\bibleverse{35}und was die Menschen betrifft, so belief sich die Zahl
der Mädchen, die noch mit keinem Manne zu tun gehabt hatten, auf
insgesamt 32000 Seelen. \bibleverse{36}So betrug also die Hälfte,
nämlich der Anteil derer, die als Kriegsleute ins Feld gezogen waren,
337500 Stück Kleinvieh, \bibleverse{37}und die Abgabe vom Kleinvieh für
den HERRN betrug 675 Stück; \bibleverse{38}die Zahl der Rinder betrug
36000 Stück und die Abgabe davon für den HERRN 72; \bibleverse{39}ferner
die Zahl der Esel 30500 und die Abgabe davon für den HERRN 61;
\bibleverse{40}sodann die Zahl der Menschen 16000 und die Abgabe davon
für den HERRN 32 Seelen. \bibleverse{41}Mose übergab dann die zum
Hebeopfer für den HERRN bestimmte Abgabe dem Priester Eleasar, wie der
HERR dem Mose geboten hatte. \bibleverse{42}Endlich von der den
Israeliten zufallenden Hälfte, die Mose von dem für die Kriegsleute
bestimmten Anteil abgesondert hatte~-- \bibleverse{43}es belief sich
aber die der Gemeinde zukommende Hälfte ebenfalls auf 337500 Stück
Kleinvieh, \bibleverse{44}36000 Rinder, \bibleverse{45}30500 Esel
\bibleverse{46}und 16000 Menschen --, \bibleverse{47}von der den
Israeliten zukommenden Hälfte also nahm Mose immer ein Stück, das er aus
je fünfzig Stück herausgegriffen hatte, sowohl von den Menschen als auch
vom Vieh und übergab sie den Leviten, die den Dienst an der Wohnung des
HERRN zu verrichten hatten, wie der HERR dem Mose geboten hatte.

\bibleverse{48}Nun traten zu Mose die Befehlshaber über die Abteilungen
des Heeres, die Hauptleute der Tausendschaften und die Hauptleute der
Hundertschaften, \bibleverse{49}und sagten zu Mose: »Deine Knechte haben
die Gesamtzahl der Kriegsleute festgestellt, die unter unserm Befehl
gestanden haben, und es ist dabei kein einziger von unseren Leuten
vermißt worden. \bibleverse{50}Darum bringen wir jetzt als Opfergabe für
den HERRN das dar, was ein jeder an Goldschmuck erbeutet hat: Armketten
und Armspangen, Fingerringe, Ohrringe und Geschmeide, um für unsere
Seelen Sühne\textless sup title=``oder: für unser Leben
Deckung''\textgreater✲ vor dem HERRN zu erlangen.« \bibleverse{51}Da
nahmen Mose und der Priester Eleasar das Gold, allerlei künstlich
gearbeitete Sachen, von ihnen in Empfang; \bibleverse{52}und es belief
sich alles Gold, das sie zum Hebeopfer für den HERRN bestimmt hatten,
auf 16750~Schekel von seiten der Hauptleute über tausend und der
Hauptleute über hundert Mann; \bibleverse{53}die gemeinen Kriegsleute
aber hatten ein jeder für sich Beute gemacht. \bibleverse{54}Mose und
der Priester Eleasar nahmen also das Gold von den Hauptleuten der
Tausendschaften und der Hundertschaften in Empfang und brachten es in
das Offenbarungszelt, damit es dort den Israeliten vor dem HERRN zum
gnädigen Gedächtnis diene.

\hypertarget{i-verteilung-des-ostjordanlandes-an-die-stuxe4mme-ruben-gad-und-halb-manasse}{%
\paragraph{i) Verteilung des Ostjordanlandes an die Stämme Ruben, Gad
und halb
Manasse}\label{i-verteilung-des-ostjordanlandes-an-die-stuxe4mme-ruben-gad-und-halb-manasse}}

\hypertarget{aa-die-bitte-der-rubeniten-und-gaditen-von-mose-in-einer-strafrede-zuruxfcckgewiesen}{%
\subparagraph{aa) Die Bitte der Rubeniten und Gaditen von Mose in einer
Strafrede
zurückgewiesen}\label{aa-die-bitte-der-rubeniten-und-gaditen-von-mose-in-einer-strafrede-zuruxfcckgewiesen}}

\hypertarget{section-31}{%
\section{32}\label{section-31}}

\bibleverse{1}Die Stämme Ruben und Gad besaßen aber außerordentlich viel
Vieh. Als sie sich nun die Landschaft Jaser und die Landschaft Gilead
ansahen, erkannten sie, daß die Gegend zur Viehzucht geeignet war.
\bibleverse{2}So gingen denn die Stämme Gad und Ruben hin und sagten zu
Mose und zu dem Priester Eleasar und zu den Häuptern✲ der Gemeinde:
\bibleverse{3}»Ataroth, Dibon, Jaser, Nimra, Hesbon, Eleale, Sebam, Nebo
und Beon, \bibleverse{4}das Land, das der HERR der Gemeinde Israel
unterworfen hat, ist ein zur Viehzucht geeignetes Land, und deine
Knechte besitzen viel Vieh.« \bibleverse{5}Dann fuhren sie fort: »Wenn
du uns eine Liebe erweisen willst, so möge diese Landschaft deinen
Knechten als Erbbesitz überwiesen werden: laß uns nicht mit über den
Jordan ziehen!« \bibleverse{6}Darauf antwortete Mose den Stämmen Gad und
Ruben: »Eure Brüder sollen also in den Krieg ziehen, und ihr wollt ruhig
hier bleiben? \bibleverse{7}Warum wollt ihr denn die (übrigen)
Israeliten von dem Entschluß abbringen, in das Land hinüberzuziehen, das
der HERR für sie bestimmt hat? \bibleverse{8}Ebenso haben es auch eure
Väter gemacht, als ich sie von Kades-Barnea zur Besichtigung des Landes
aussandte\textless sup title=``vgl. 13,17-33''\textgreater✲.
\bibleverse{9}Als sie nämlich bis zum Tal Eskol hinaufgezogen waren und
sich das Land angesehen hatten, brachten sie die Israeliten von ihrem
Entschluß ab, so daß sie nicht in das Land ziehen wollten, das der HERR
für sie bestimmt hatte. \bibleverse{10}Da entbrannte aber der Zorn des
HERRN an jenem Tage, so daß er mit einem Schwur aussprach:
\bibleverse{11}›Die Männer, die aus Ägypten ausgezogen sind, von zwanzig
Jahren an und darüber, die sollen nimmermehr das Land zu sehen bekommen,
das ich Abraham, Isaak und Jakob zugeschworen habe, denn sie sind mir
nicht in allen Stücken gehorsam gewesen, \bibleverse{12}außer dem
Kenissiter Kaleb, dem Sohn Jephunnes, und außer Josua, dem Sohne Nuns;
denn diese beiden sind dem HERRN in allen Stücken gehorsam gewesen.‹
\bibleverse{13}So entbrannte also der Zorn des HERRN gegen Israel, und
er ließ sie vierzig Jahre lang in der Wüste umherirren, bis das ganze
Geschlecht, das in den Augen des HERRN übel gehandelt hatte,
ausgestorben war. \bibleverse{14}Und jetzt seid ihr an Stelle eurer
Väter aufgetreten, ein Nachwuchs von Sündern, um den lodernden Zorn des
HERRN gegen Israel noch zu mehren! \bibleverse{15}Wenn ihr euch vom
Gehorsam gegen ihn abwendet, so wird er das Volk noch länger in der
Wüste belassen, und ihr werdet so dies ganze Volk zugrunde richten!«

\hypertarget{bb-die-antwort-der-rubeniten-und-gaditen}{%
\subparagraph{bb) Die Antwort der Rubeniten und
Gaditen}\label{bb-die-antwort-der-rubeniten-und-gaditen}}

\bibleverse{16}Da traten sie an ihn heran und sagten: »Wir wollen hier
nur Viehhürden für unsere Herden und feste Wohnorte für unsere Frauen
und Kinder bauen; \bibleverse{17}wir selbst aber wollen kampfgerüstet an
der Spitze der Israeliten einherziehen, bis wir sie in ihre Wohnsitze
gebracht haben; unterdessen sollen unsere Frauen und Kinder wegen der
(feindseligen) Bewohner des Landes in den befestigten Ortschaften
bleiben. \bibleverse{18}Wir wollen nicht eher in unsere Häuser
zurückkehren, als bis die Israeliten sämtlich ihren Erbbesitz empfangen
haben; \bibleverse{19}denn wir beanspruchen keinen Besitz neben ihnen
jenseits des Jordans und weiterhin, wenn uns unser Erbbesitz diesseits
des Jordans gegen Osten zuteil geworden ist.«

\hypertarget{cc-die-zusage-moses-unter-festsetzung-der-bedingungen-verleihung-des-ostjordanlandes-an-die-bittenden-stuxe4mme}{%
\subparagraph{cc) Die Zusage Moses unter Festsetzung der Bedingungen;
Verleihung des Ostjordanlandes an die bittenden
Stämme}\label{cc-die-zusage-moses-unter-festsetzung-der-bedingungen-verleihung-des-ostjordanlandes-an-die-bittenden-stuxe4mme}}

\bibleverse{20}Da sagte Mose zu ihnen: »Wenn ihr das wirklich tut, daß
ihr euch vor dem HERRN zum Kriege rüstet \bibleverse{21}und ihr alle
gerüstet vor den Augen des HERRN über den Jordan zieht, bis er seine
Feinde vor sich her ausgetrieben hat, \bibleverse{22}und ihr erst dann
wieder heimkehrt, nachdem das Land vor dem HERRN unterworfen ist, so
sollt ihr eurer Verpflichtungen gegen den HERRN und gegen Israel los und
ledig sein, und dieses Land soll euch als Erbbesitz vor dem HERRN zuteil
werden. \bibleverse{23}Wenn ihr aber nicht so handelt, so habt ihr euch
damit gegen den HERRN versündigt und werdet die Folgen eurer
Versündigung zu fühlen bekommen! \bibleverse{24}Baut euch also Städte
für eure Frauen und Kinder und Hürden für euer Kleinvieh, tut aber auch,
was ihr versprochen habt!« \bibleverse{25}Da gaben die Stämme Gad und
Ruben dem Mose folgende Versicherung: »Deine Knechte werden tun, wie du,
Herr, befiehlst: \bibleverse{26}unsere Kinder, unsere Frauen, unsere
Herden und all unsere Lasttiere sollen hier in den Ortschaften Gileads
bleiben, \bibleverse{27}wir aber, deine Knechte, werden alle, soweit sie
waffenfähig sind, vor dem HERRN her zum Kampf hinüberziehen, wie du,
Herr, befohlen hast.«

\bibleverse{28}Darauf gab Mose ihretwegen dem Priester Eleasar und
Josua, dem Sohne Nuns, und den Stammeshäuptern der Israeliten Anweisung,
\bibleverse{29}indem er ihnen auftrug: »Wenn die Gaditen und Rubeniten,
alle Waffenfähigen, mit euch zum Kriege vor dem HERRN her über den
Jordan ziehen, so sollt ihr, wenn das Land unterworfen vor euch daliegt,
ihnen die Landschaft Gilead zum Erbbesitz geben; \bibleverse{30}wenn sie
aber nicht kampfgerüstet mit euch hinüberziehen, so sollen sie unter
euch im Lande Kanaan angesiedelt\textless sup title=``=~ansässig
gemacht''\textgreater✲ werden.« \bibleverse{31}Da gaben die Gaditen und
Rubeniten folgende Erklärung ab: »Wie der HERR deinen Knechten geboten
hat, so wollen wir tun: \bibleverse{32}wir wollen kampfgerüstet vor dem
HERRN her in das Land Kanaan hinüberziehen, damit uns unser Erbbesitz
diesseits des Jordans verbleibt.«

\bibleverse{33}Darauf überwies Mose ihnen -- den Stämmen Gad und Ruben
und dem halben Stamme Manasses, des Sohnes Josephs -- das Reich des
Amoriterkönigs Sihon und das Reich des Königs Og von Basan, das Land mit
den darin liegenden Städten samt den zugehörigen Gebieten ringsum.

\hypertarget{dd-uxfcbersicht-uxfcber-die-von-den-gaditen-und-rubeniten-wieder-aufgebauten-stuxe4dte}{%
\subparagraph{dd) Übersicht über die von den Gaditen und Rubeniten
wieder aufgebauten
Städte}\label{dd-uxfcbersicht-uxfcber-die-von-den-gaditen-und-rubeniten-wieder-aufgebauten-stuxe4dte}}

\bibleverse{34}Da bauten die Gaditen Dibon, Ataroth, Aroer,
\bibleverse{35}Ateroth-Sophan, Jaser, Jogbeha, \bibleverse{36}Beth-Nimra
und Beth-Haran wieder auf, feste Städte und Hürden für das Kleinvieh.
\bibleverse{37}Die Rubeniten aber bauten Hesbon, Eleale, Kirjathaim,
\bibleverse{38}Nebo, Baal-Meon {[}mit verändertem Namen zu sprechen!{]}
und Sibma wieder auf; sie gaben aber den Städten, die sie wieder
aufgebaut hatten, andere Namen.

\hypertarget{ee-nachkommen-manasses-setzen-sich-im-ostjordanlande-fest}{%
\subparagraph{ee) Nachkommen Manasses setzen sich im Ostjordanlande
fest}\label{ee-nachkommen-manasses-setzen-sich-im-ostjordanlande-fest}}

\bibleverse{39}Die Söhne Machirs aber, des Sohnes Manasses, zogen nach
Gilead, eroberten es und trieben die Amoriter aus, die darin wohnten.
\bibleverse{40}Mose verlieh dann Gilead dem Machir, dem Sohne Manasses,
und dieser siedelte sich dort an. \bibleverse{41}Jair aber, der Sohn
Manasses, zog hin, bemächtigte sich ihrer (der Amoriter) Zeltdörfer und
gab ihnen den Namen »Zeltdörfer Jairs«.~-- \bibleverse{42}Auch Nobah zog
hin, nahm Kenath und die dazugehörigen Ortschaften ein und benannte es
nach seinem Namen Nobah.

\hypertarget{k-verzeichnis-der-vierzig-lagerstuxe4tten-an-denen-die-israeliten-in-den-vierzig-wuxfcstenjahren-geweilt-haben}{%
\paragraph{k) Verzeichnis der (vierzig) Lagerstätten, an denen die
Israeliten in den vierzig Wüstenjahren geweilt
haben}\label{k-verzeichnis-der-vierzig-lagerstuxe4tten-an-denen-die-israeliten-in-den-vierzig-wuxfcstenjahren-geweilt-haben}}

\hypertarget{section-32}{%
\section{33}\label{section-32}}

\bibleverse{1}Folgendes sind die einzelnen Züge\textless sup
title=``oder: Wanderungen''\textgreater✲ der Israeliten, in denen sie
aus Ägypten nach ihren Heerscharen unter der Führung Moses und Aarons
ausgezogen sind. \bibleverse{2}Mose hatte nämlich auf Befehl des HERRN
die Orte aufgeschrieben, von denen ihre Auszüge erfolgt waren; und
folgendes sind ihre Züge von einem Aufbruchsort zum andern:
\bibleverse{3}Sie brachen von Ramses am fünfzehnten Tage des ersten
Monats auf; am Tage nach dem Passah zogen die Israeliten mit hoch
erhobener Hand\textless sup title=``vgl. 2.Mose 14,8''\textgreater✲ vor
den Augen aller Ägypter aus, \bibleverse{4}während die Ägypter alle
Erstgeborenen begruben, die der HERR unter ihnen hatte sterben lassen;
denn der HERR hatte auch an ihren Göttern ein Strafgericht
vollzogen\textless sup title=``d.h. sie als ohnmächtig
erwiesen''\textgreater✲.

\bibleverse{5}Die Israeliten brachen also von Ramses auf und lagerten in
Sukkoth. \bibleverse{6}Von Sukkoth zogen sie dann weiter und lagerten in
Etham, das am Rande der Wüste liegt. \bibleverse{7}Von Etham zogen sie
weiter und wandten sich nach Pi-Hahiroth, das Baal-Zephon gegenüber
liegt, und lagerten östlich von Migdol. \bibleverse{8}Von Pi-Hahiroth
brachen sie auf und zogen mitten durch das Meer nach der Wüste hin; sie
wanderten dann drei Tagereisen weit in der Wüste Etham und lagerten in
Mara. \bibleverse{9}Von Mara zogen sie weiter und kamen nach Elim; dort
waren zwölf Wasserquellen und siebzig Palmbäume, und sie lagerten
daselbst. \bibleverse{10}Von Elim zogen sie weiter und lagerten am
Schilfmeer. \bibleverse{11}Vom Schilfmeer zogen sie weiter und lagerten
in der Wüste Sin. \bibleverse{12}Aus der Wüste Sin zogen sie weiter und
lagerten in Dophka. \bibleverse{13}Von Dophka zogen sie weiter und
lagerten in Alus. \bibleverse{14}Von Alus zogen sie weiter und lagerten
in Rephidim; dort hatte das Volk kein Wasser zu trinken.
\bibleverse{15}Von Rephidim zogen sie weiter und lagerten in der Wüste
Sinai. \bibleverse{16}Aus der Wüste Sinai zogen sie weiter und lagerten
bei den Lustgräbern. \bibleverse{17}Von den Lustgräbern zogen sie weiter
und lagerten in Hazeroth. \bibleverse{18}Von Hazeroth zogen sie weiter
und lagerten in Rithma. \bibleverse{19}Von Rithma zogen sie weiter und
lagerten in Rimmon-Perez. \bibleverse{20}Von Rimmon-Perez zogen sie
weiter und lagerten in Libna. \bibleverse{21}Von Libna zogen sie weiter
und lagerten in Rissa. \bibleverse{22}Von Rissa zogen sie weiter und
lagerten in Kehelatha. \bibleverse{23}Von Kehelatha zogen sie weiter und
lagerten am Berge Sepher. \bibleverse{24}Vom Berge Sepher zogen sie
weiter und lagerten in Harada. \bibleverse{25}Von Harada zogen sie
weiter und lagerten in Makheloth. \bibleverse{26}Von Makheloth zogen sie
weiter und lagerten in Thahath. \bibleverse{27}Von Thahath zogen sie
weiter und lagerten in Therah. \bibleverse{28}Von Therah zogen sie
weiter und lagerten in Mithka. \bibleverse{29}Von Mithka zogen sie
weiter und lagerten in Hasmona. \bibleverse{30}Von Hasmona zogen sie
weiter und lagerten in Moseroth. \bibleverse{31}Von Moseroth zogen sie
weiter und lagerten in Bene-Jaakan. \bibleverse{32}Von Bene-Jaakan zogen
sie weiter und lagerten in Hor-Hagidgad. \bibleverse{33}Von Hor-Hagidgad
zogen sie weiter und lagerten in Jotbatha. \bibleverse{34}Von Jotbatha
zogen sie weiter und lagerten in Abrona. \bibleverse{35}Von Abrona zogen
sie weiter und lagerten in Ezjon-Geber. \bibleverse{36}Von Ezjon-Geber
zogen sie weiter und lagerten in der Wüste Zin, das ist Kades.
\bibleverse{37}Von Kades zogen sie weiter und lagerten am Berge Hor, an
der Grenze des Landes der Edomiter. \bibleverse{38}Da stieg der Priester
Aaron nach dem Befehl des HERRN auf den Berg Hor hinauf und starb
daselbst im vierzigsten Jahr nach dem Auszug der Israeliten aus dem
Lande Ägypten am ersten Tage des fünften Monats; \bibleverse{39}Aaron
war aber 123 Jahre alt, als er auf dem Berge Hor starb.
\bibleverse{40}{[}Und der Kanaanäer, der König von Arad, der im
südlichen Teile des Landes Kanaan wohnte, hörte vom Heranrücken der
Israeliten.{]} \bibleverse{41}Vom Berge Hor zogen sie dann weiter und
lagerten in Zalmona. \bibleverse{42}Von Zalmona zogen sie weiter und
lagerten in Phunon. \bibleverse{43}Von Phunon zogen sie weiter und
lagerten in Oboth. \bibleverse{44}Von Oboth zogen sie weiter und
lagerten in Ijje-Abarim an der Grenze des Moabiterlandes.
\bibleverse{45}Von Ijjim zogen sie weiter und lagerten in Dibon-Gad.
\bibleverse{46}Von Dibon-Gad zogen sie weiter und lagerten in
Almon-Diblathaim. \bibleverse{47}Von Almon-Diblathaim zogen sie weiter
und lagerten am Gebirge Abarim östlich vom Nebo. \bibleverse{48}Vom
Gebirge Abarim zogen sie weiter und lagerten sich in den Steppen der
Moabiter am Jordan, Jericho gegenüber; \bibleverse{49}und zwar lagerten
sie am Jordan von Beth-Jesimoth bis Abel-Sittim in den Steppen der
Moabiter.

\hypertarget{l-vorluxe4ufige-bestimmungen-gottes-bezuxfcglich-der-eroberung-und-verteilung-des-westjordanlandes-kanaan}{%
\paragraph{l) Vorläufige Bestimmungen Gottes bezüglich der Eroberung und
Verteilung des Westjordanlandes
Kanaan}\label{l-vorluxe4ufige-bestimmungen-gottes-bezuxfcglich-der-eroberung-und-verteilung-des-westjordanlandes-kanaan}}

\bibleverse{50}Der HERR gebot dann dem Mose in den Steppen der Moabiter
am Jordan, Jericho gegenüber: »Teile den Israeliten folgende
Verordnungen mit: \bibleverse{51}Wenn ihr über den Jordan in das Land
Kanaan hinübergezogen seid, \bibleverse{52}sollt ihr alle Bewohner des
Landes vor euch her austreiben und alle ihre Götzenbilder\textless sup
title=``d.h. Holz- und Steinbilder''\textgreater✲ vernichten; auch alle
ihre Gußbilder sollt ihr vernichten und alle ihre Höhen zerstören.
\bibleverse{53}Ihr sollt dann das Land in Besitz nehmen und darin
wohnen; denn euch habe ich das Land als Eigentum verliehen.
\bibleverse{54}Und zwar sollt ihr euch das Land durch das Los als
Erbbesitz zuteilen entsprechend euren Stämmen: den größeren Stämmen
sollt ihr einen größeren Erbbesitz geben und den kleineren einen weniger
großen Erbbesitz zuteilen; doch wohin immer einem jeden das Los fällt,
da soll es ihm als Eigentum zuteil werden: nach euren väterlichen
Stämmen sollt ihr euch das Land als Erbbesitz zuteilen.
\bibleverse{55}Wenn ihr aber die Bewohner des Landes nicht vor euch her
austreibt, so werden die, welche ihr von ihnen übriglaßt, zu Dornen in
euren Augen und zu Stacheln in euren Seiten werden und euch in dem
Lande, in dem ihr wohnen werdet, bedrängen. \bibleverse{56}Die Folge
wird dann sein, daß ich euch das Geschick widerfahren lasse, das ich
ihnen zugedacht hatte.«

\hypertarget{festsetzung-der-grenzen-des-in-besitz-zu-nehmenden-landes-kanaan}{%
\paragraph{Festsetzung der Grenzen des in Besitz zu nehmenden Landes
Kanaan}\label{festsetzung-der-grenzen-des-in-besitz-zu-nehmenden-landes-kanaan}}

\hypertarget{section-33}{%
\section{34}\label{section-33}}

\bibleverse{1}Weiter sagte der HERR zu Mose: \bibleverse{2}»Folgende
Verordnungen sollst du den Israeliten mitteilen: Wenn ihr in das Land
Kanaan kommt, so soll dies das Gebiet sein, das euch als Erbbesitz
zufällt: das Land Kanaan in seinem ganzen Umfang. \bibleverse{3}Und zwar
soll die Südseite sich euch von der Wüste Zin an, Edom entlang,
hinziehen, so daß eure Südgrenze im Osten am (südlichen) Ende des
Salzmeeres beginnt. \bibleverse{4}Dann soll eure Grenze südlich vom
Skorpionenstieg umbiegen und sich nach Zin hinüberziehen, wo sie südlich
von Kades-Barnea endigt; von dort soll sie weiter nach Hazar-Addar hin
laufen und nach Azmon hinübergehen; \bibleverse{5}und von Azmon wende
sich die Grenze nach dem Bach Ägyptens hin und endige nach dem Westmeer
zu.~-- \bibleverse{6}Was sodann die Westgrenze betrifft, so gelte euch
da das große Meer zugleich als Grenze; das soll eure Westgrenze sein.~--
\bibleverse{7}Und folgendes soll eure Nordgrenze sein: vom großen
Westmeer an sollt ihr euch eine Grenzlinie bis zum Berge Hor ziehen;
\bibleverse{8}vom Berge Hor an sollt ihr euch eine Grenzlinie in der
Richtung auf Hamath hin ziehen, und das Ende der Grenze soll nach Zedad
hin sein; \bibleverse{9}dann laufe die Grenze weiter nach Siphron hin
und erreiche ihr Ende bei Hazar-Enan. Das soll eure Nordgrenze sein.~--
\bibleverse{10}Als Ostgrenze aber sollt ihr euch eine Linie von
Hazar-Enan nach Sepham festsetzen; \bibleverse{11}und von Sepham gehe
die Grenze nach Ha-Ribla hinab östlich von Ain; dann ziehe sich die
Grenze noch weiter hinab und stoße auf den Höhenzug östlich vom See
Genezareth; \bibleverse{12}dann ziehe sich die Grenze den Jordan entlang
hinab und erreiche ihr Ende am Salzmeer. Dies soll euer Land nach seinen
Grenzen ringsum sein.«

\bibleverse{13}Mose gab dann den Israeliten folgende Weisung: »Dies ist
das Land, das ihr euch durchs Los als euren Erbbesitz zuteilen sollt und
das der HERR den neun Stämmen und dem halben Stamm (Manasse) zu geben
geboten hat. \bibleverse{14}Denn alle zu den beiden Stämmen Ruben und
Gad gehörenden Familien und der halbe Stamm Manasse haben ihren
Erbbesitz bereits empfangen. \bibleverse{15}Diese zweieinhalb Stämme
haben ihren Erbbesitz auf der andern Seite des Jordans, Jericho
gegenüber, im Osten, nach Sonnenaufgang hin, bereits erhalten.«

\hypertarget{aufzuxe4hlung-der-muxe4nner-welche-die-verteilung-des-landes-besorgen-sollen}{%
\paragraph{Aufzählung der Männer, welche die Verteilung des Landes
besorgen
sollen}\label{aufzuxe4hlung-der-muxe4nner-welche-die-verteilung-des-landes-besorgen-sollen}}

\bibleverse{16}Weiter sagte der HERR zu Mose folgendes:
\bibleverse{17}»Dies sind die Namen der Männer, die das Land als
Erbbesitz unter euch verteilen sollen: der Priester Eleasar und Josua,
der Sohn Nuns; \bibleverse{18}dazu sollt ihr von jedem Stamm einen
Fürsten zu der Verteilung des Landes hinzuziehen. \bibleverse{19}Und
dies sind die Namen der Männer: für den Stamm Juda: Kaleb, der Sohn
Jephunnes; \bibleverse{20}für den Stamm Simeon: Samuel, der Sohn
Ammihuds; \bibleverse{21}für den Stamm Benjamin: Elidad, der Sohn
Kislons; \bibleverse{22}für den Stamm Dan als Fürst: Bukki, der Sohn
Joglis; \bibleverse{23}für die Nachkommen Josephs: für den Stamm Manasse
als Fürst: Hanniel, der Sohn Ephods, \bibleverse{24}und für den Stamm
Ephraim als Fürst: Kemuel, der Sohn Siphtans; \bibleverse{25}für den
Stamm Sebulon als Fürst: Elizaphan, der Sohn Parnachs;
\bibleverse{26}für den Stamm Issaschar als Fürst: Paltiel, der Sohn
Assans; \bibleverse{27}für den Stamm Asser als Fürst: Ahihud, der Sohn
Selomis; \bibleverse{28}für den Stamm Naphthali als Fürst: Pedahel, der
Sohn Ammihuds.« \bibleverse{29}Diese waren es, denen der HERR gebot, den
Israeliten ihren Erbbesitz im Lande Kanaan zuzuteilen.

\hypertarget{m-bestimmungen-bezuxfcglich-der-42-levitenstuxe4dte-und-der-sechs-fuxfcr-totschluxe4ger-bestimmten-freistuxe4dte}{%
\paragraph{m) Bestimmungen bezüglich der (42) Levitenstädte und der
sechs für Totschläger bestimmten
Freistädte}\label{m-bestimmungen-bezuxfcglich-der-42-levitenstuxe4dte-und-der-sechs-fuxfcr-totschluxe4ger-bestimmten-freistuxe4dte}}

\hypertarget{section-34}{%
\section{35}\label{section-34}}

\bibleverse{1}Weiter gebot der HERR dem Mose in den Steppen der Moabiter
am Jordan, Jericho gegenüber, folgendes: \bibleverse{2}»Befiehl den
Israeliten, daß sie von ihrem Erbbesitz den Leviten Städte zum Wohnen
geben; und zu den Städten sollt ihr auch eine Weidetrift rings um sie
her den Leviten geben. \bibleverse{3}Die Städte sollen ihnen dann als
Wohnsitze dienen, während die zugehörigen Weidetriften für ihr Lastvieh
und ihre Herden und für alle ihre übrigen Tiere bestimmt sind.
\bibleverse{4}Es sollen sich aber die Weidetriften bei den Städten, die
ihr den Leviten zu geben habt, von der Stadtmauer nach außen hin tausend
Ellen weit ringsum erstrecken; \bibleverse{5}und ihr sollt außerhalb der
Stadt auf der Ostseite zweitausend Ellen abmessen und ebenso auf der
Südseite zweitausend Ellen und auf der Westseite zweitausend Ellen und
auf der Nordseite zweitausend Ellen, so daß die Stadt selbst in der
Mitte liegt; dies sollen für sie die Weideplätze bei den Städten sein.
\bibleverse{6}Unter den Städten aber, die ihr den Leviten zu geben habt,
sollen sich die sechs Freistädte befinden, die ihr ihnen abtreten sollt,
damit dorthin fliehen kann, wer einen Totschlag begangen hat; außer
diesen aber sollt ihr ihnen noch zweiundvierzig Städte geben:
\bibleverse{7}die Gesamtzahl der Städte, die ihr den Leviten zu geben
habt, soll sich auf achtundvierzig Städte nebst den zugehörigen
Weidetriften belaufen. \bibleverse{8}Was aber die Städte betrifft, die
ihr von dem Besitztum der Israeliten abzugeben habt, so sollt ihr von
den größeren Stämmen eine größere Anzahl und von den kleineren Stämmen
eine kleinere Anzahl von Städten nehmen: jeder Stamm soll nach Maßgabe
seines Erbbesitzes, den er erhalten wird, von seinen Städten eine
bestimmte Zahl an die Leviten abtreten.«

\bibleverse{9}Der HERR gebot dann weiter dem Mose folgendes:
\bibleverse{10}»Teile den Israeliten folgende Verordnungen mit: Wenn ihr
über den Jordan in das Land Kanaan hinüberkommt, \bibleverse{11}sollt
ihr euch passend gelegene Städte auswählen, die euch als Freistädte
dienen sollen, damit ein Totschläger dahin fliehen kann, der einen
Menschen unvorsätzlich erschlagen hat. \bibleverse{12}Diese Städte
sollen euch nämlich als Zufluchtsorte vor dem Bluträcher dienen, damit
der Totschläger nicht getötet wird, ehe er vor der Gemeinde gestanden
hat, um abgeurteilt zu werden. \bibleverse{13}Es sollen aber die Städte,
die ihr zu solchen Freistädten herzugeben habt, sechs an Zahl sein:
\bibleverse{14}drei Städte sollt ihr diesseits des Jordans dazu
bestimmen und drei Städte im Lande Kanaan dazu hergeben: Freistädte
sollen es sein. \bibleverse{15}Für die Israeliten wie für die Fremdlinge
und die Beisassen, die unter ihnen leben, sollen diese sechs Städte als
Zufluchtsorte dienen, damit jeder dorthin fliehen kann, der einen
Menschen unvorsätzlich erschlagen hat.«

\hypertarget{die-bestrafung-des-muxf6rders}{%
\paragraph{Die Bestrafung des
Mörders}\label{die-bestrafung-des-muxf6rders}}

\bibleverse{16}»Wenn er ihn aber mit einem eisernen Werkzeug geschlagen
hat, so daß er daran gestorben ist, so ist er ein Mörder: als solcher
muß er unbedingt mit dem Tode bestraft werden. \bibleverse{17}Wenn er
ihn ferner mit einem Stein, den er in der Hand hatte und der den Tod
verursachen kann, geschlagen hat, so daß jener daran gestorben ist, so
ist er ein Mörder: als solcher muß er unbedingt mit dem Tode bestraft
werden. \bibleverse{18}Oder wenn er ihn mit einem hölzernen Werkzeug,
das er in der Hand hatte und durch das jemand getötet werden kann,
geschlagen hat, so daß jener daran gestorben ist, so ist er ein Mörder:
als solcher muß er unbedingt mit dem Tode bestraft werden;
\bibleverse{19}der Bluträcher soll den Mörder töten; sobald er ihn
antrifft, soll er ihn töten. \bibleverse{20}Wenn ferner jemand einem
andern aus Haß einen Stoß versetzt oder vorsätzlich nach ihm geworfen
hat, so daß jener daran gestorben ist, \bibleverse{21}oder wenn er ihn
aus Feindschaft mit der Hand geschlagen hat, so daß jener daran
gestorben ist, so muß der, welcher ihn geschlagen hat, unbedingt mit dem
Tode bestraft werden: er ist ein Mörder: der Bluträcher soll den Mörder
töten, sobald er ihn antrifft.

\bibleverse{22}Hat er ihn aber von ungefähr, nicht aus Feindschaft,
gestoßen oder unabsichtlich irgendeinen Gegenstand auf ihn geworfen
\bibleverse{23}oder, ohne ihn zu sehen, irgendeinen Stein, durch den
jemand getötet werden kann, auf ihn fallen lassen und jener ist daran
gestorben, obgleich er ihm nicht feindlich gesinnt war und ohne daß er
ihm Schaden zufügen wollte: \bibleverse{24}so soll die Gemeinde zwischen
dem Totschläger und dem Bluträcher nach diesen Rechtsbestimmungen
entscheiden, \bibleverse{25}und die Gemeinde soll den Totschläger vor
der Gewalttätigkeit des Bluträchers schützen, und die Gemeinde soll ihn
in die ihm zustehende Freistadt, in die er sich geflüchtet hatte,
zurückbringen, und er soll darin wohnen bleiben bis zum Tode des
Hohenpriesters, den man mit dem heiligen Öl gesalbt hat.
\bibleverse{26}Wenn aber der Totschläger aus dem Gebiet der ihm
zustehenden Freistadt, in die er geflohen ist, hinausgeht
\bibleverse{27}und der Bluträcher ihn außerhalb des Gebiets der
betreffenden Freistadt antrifft und den Totschläger tötet, so lädt er
keine Blutschuld auf sich; \bibleverse{28}denn jener hat in der ihm
zustehenden Freistadt bis zum Tode des Hohenpriesters zu bleiben, und
erst nach dem Tod des Hohenpriesters darf der Totschläger an den Ort, wo
sein Erbbesitz liegt, zurückkehren. \bibleverse{29}Diese
Rechtsbestimmungen sollen bei euch für eure künftigen Geschlechter in
allen euren Wohnsitzen zu Recht bestehen.

\bibleverse{30}Wenn irgend jemand einen Menschen erschlägt, so soll man
den Mörder auf die Aussage von Zeugen mit dem Tode bestrafen; aber das
Zeugnis eines einzelnen soll nicht hinreichen, um jemand zum Tode zu
verurteilen. \bibleverse{31}Ihr dürft auch kein Lösegeld für das Leben
eines Mörders, der des Todes schuldig ist, annehmen, vielmehr soll er
unbedingt mit dem Tode bestraft werden. \bibleverse{32}Auch dürft ihr
kein Lösegeld für jemand, der in die ihm zustehende Freistadt geflohen
ist, zu dem Zweck annehmen, daß er schon vor dem Tode des Hohenpriesters
zurückkehren und irgendwo im Lande wohnen dürfe. \bibleverse{33}Ihr
sollt das Land, in dem ihr wohnt, nicht entweihen; denn das Blut
entweiht das Land, und dem Lande kann für das Blut, das darin vergossen
worden ist, keine Sühne geschafft werden außer durch das Blut dessen,
der es vergossen hat. \bibleverse{34}So verunreinigt denn das Land
nicht, in dem ihr wohnt und in dessen Mitte auch ich wohne; denn ich,
der HERR, wohne inmitten der Israeliten.«

\hypertarget{n-nachtrag-zu-dem-gesetz-bezuxfcglich-der-erbtuxf6chter}{%
\paragraph{n) Nachtrag zu dem Gesetz bezüglich der
Erbtöchter}\label{n-nachtrag-zu-dem-gesetz-bezuxfcglich-der-erbtuxf6chter}}

\hypertarget{section-35}{%
\section{36}\label{section-35}}

\bibleverse{1}Die Familienhäupter des Geschlechts, das von Gilead, dem
Sohne Machirs, dem Enkel Manasses, stammte und zu den von Joseph
stammenden Geschlechtern gehörte, traten auf und trugen dem Mose und den
Fürsten, den Stammeshäuptern der Israeliten, \bibleverse{2}folgendes
Anliegen vor: »Gott hat dir, o Herr, geboten, den Israeliten das Land
durch das Los zum Erbbesitz zu geben; auch hast du, Herr, von Gott die
Weisung erhalten, den Erbbesitz unseres Stammgenossen Zelophhad seinen
Töchtern zu überweisen. \bibleverse{3}Wenn diese sich nun aber mit einem
Manne verheiraten, der zu einem andern Stamme der Israeliten gehört, so
geht ihr Erbbesitz dem Erbbesitz unserer Väter verloren und wird zu dem
Erbbesitz des Stammes geschlagen, dem sie (nach ihrer Verheiratung)
angehören, während er dem Losteile unseres Erbbesitzes verlorengeht.
\bibleverse{4}Und auch wenn das Halljahr für die Israeliten kommt, so
bleibt ihr Erbbesitz mit dem Erbbesitz des Stammes vereinigt, dem sie
(nach ihrer Verheiratung) angehören; aber dem Erbbesitz des Stammes
unserer Väter geht ihr Erbbesitz verloren.«

\hypertarget{die-neue-gemeinguxfcltige-verordnung-bezuxfcglich-der-verheiratung-der-erbtuxf6chter-und-die-ausfuxfchrung-dieser-verordnung-im-vorliegenden-fall}{%
\paragraph{Die neue gemeingültige Verordnung bezüglich der Verheiratung
der Erbtöchter und die Ausführung dieser Verordnung im vorliegenden
Fall}\label{die-neue-gemeinguxfcltige-verordnung-bezuxfcglich-der-verheiratung-der-erbtuxf6chter-und-die-ausfuxfchrung-dieser-verordnung-im-vorliegenden-fall}}

\bibleverse{5}Da gab Mose den Israeliten nach dem Geheiß des HERRN
folgende Weisung: »Der Stamm der Nachkommen Josephs hat recht.
\bibleverse{6}So lautet die Verordnung, die der HERR in betreff der
Töchter Zelophhads erlassen hat: Sie mögen sich ganz nach ihrem Gefallen
verheiraten; jedoch dürfen sie nur mit einem Manne aus einem Geschlecht
ihres väterlichen Stammes die Ehe eingehen, \bibleverse{7}damit nicht
ein Erbbesitz der Israeliten von einem Stamm an einen andern übergeht;
denn die Israeliten sollen ein jeder den Erbbesitz ihres väterlichen
Stammes ungeschmälert erhalten. \bibleverse{8}Darum darf jedes Mädchen,
das in einem von den israelitischen Stämmen zu Grundbesitz gelangt, nur
einen Mann aus einem Geschlecht ihres väterlichen Stammes heiraten,
damit die Israeliten allesamt ihren väterlichen Erbbesitz ungeschmälert
erhalten \bibleverse{9}und kein Erbbesitz von einem Stamm an einen
andern übergeht; denn die Stämme der Israeliten sollen ein jeder seinen
Erbbesitz ungeschmälert erhalten.«

\bibleverse{10}Wie der HERR dem Mose geboten hatte, so taten die Töchter
Zelophhads; \bibleverse{11}denn Mahla, Thirza, Hogla, Milka und Noa, die
Töchter Zelophhads, verheirateten sich mit den Söhnen ihrer
Oheime\textless sup title=``=~mit ihren Vettern''\textgreater✲;
\bibleverse{12}mit Männern aus den Geschlechtern, die von Manasse, dem
Sohne Josephs, stammten, verheirateten sie sich, und so verblieb ihr
Erbbesitz bei dem Stamme, dem das Geschlecht ihres Vaters angehörte.

\hypertarget{schluuxdfsatz-fuxfcr-den-abschnitt-221-3612}{%
\paragraph{Schlußsatz für den Abschnitt
22,1-36,12}\label{schluuxdfsatz-fuxfcr-den-abschnitt-221-3612}}

\bibleverse{13}Dies sind die Gebote und Verordnungen, die der HERR an
die Israeliten in den Steppen der Moabiter am Jordan, Jericho gegenüber,
durch Vermittlung Moses hat ergehen lassen.
