\hypertarget{das-hohelied}{%
\section{DAS HOHELIED}\label{das-hohelied}}

\hypertarget{section}{%
\section{1}\label{section}}

\bibleverse{1}Das Lied der Lieder, von Salomo.

\hypertarget{aus-sulammiths-71-muxe4dchenzeit}{%
\subsubsection{1. Aus Sulammiths (7,1)
Mädchenzeit}\label{aus-sulammiths-71-muxe4dchenzeit}}

\hypertarget{a-erstes-lied-sulammiths-selbstgespruxe4ch-und-liebessehnsucht-im-kuxf6nigspalast-salomos}{%
\paragraph{a) Erstes Lied: Sulammiths Selbstgespräch und Liebessehnsucht
(im Königspalast
Salomos?)}\label{a-erstes-lied-sulammiths-selbstgespruxe4ch-und-liebessehnsucht-im-kuxf6nigspalast-salomos}}

\bibleverse{2}O möcht' er mich küssen mit seines Mundes Küssen! Denn
deine Liebe ist wonniger als Wein! \bibleverse{3}Köstlich ist der Duft
deiner Salben; wie ausgegossenes✲ Salböl ist dein Name: drum haben die
Mädchen dich lieb\textless sup title=``oder: gern''\textgreater✲.
\bibleverse{4}Zieh mich dir nach, komm, laß uns eilen! Führe mich,
König, in deine Gemächer! »Wir wollen jubeln und deiner uns freuen,
wollen preisen deine Liebe mehr als Wein!« Ach, inniglich\textless sup
title=``oder: Ja, mit Recht''\textgreater✲ lieben sie dich.

\hypertarget{b-zweites-lied-klage-uxfcber-gefuxe4hrdete-muxe4dchenschuxf6nheit}{%
\paragraph{b) Zweites Lied: Klage über gefährdete
Mädchenschönheit}\label{b-zweites-lied-klage-uxfcber-gefuxe4hrdete-muxe4dchenschuxf6nheit}}

\bibleverse{5}Gebräunt bin ich, aber doch schön, ihr Töchter Jerusalems,
wie die Zelte von Kedar, wie Salomos Teppiche. \bibleverse{6}Seht mich
nicht an, daß so gebräunt ich bin, daß die Sonne mich so verbrannt hat!
Meiner Mutter Söhne waren böse auf mich, bestellten mich zur Hüterin der
Weinberge; meinen eignen Weinberg hab' ich nicht gehütet\textless sup
title=``oder: hüten können''\textgreater✲.

\hypertarget{c-drittes-lied-bitte-der-braut-um-ein-stelldichein}{%
\paragraph{c) Drittes Lied: Bitte der Braut um ein
Stelldichein}\label{c-drittes-lied-bitte-der-braut-um-ein-stelldichein}}

\bibleverse{7}Tu mir kund, du, den meine Seele liebt: wo weidest du, wo
lagerst du zur Mittagszeit? Denn warum soll als Verirrte\textless sup
title=``oder: Landstreicherin''\textgreater✲ ich erscheinen bei den
Herden deiner Genossen?~-- \bibleverse{8}»Wenn du das nicht weißt, du
schönste unter den Weibern, so geh nur hinaus, den Spuren der Herde
nach, und weide deine Zicklein bei den Zelten der Hirten!«

\hypertarget{der-werdende-und-der-gewordene-brautstand}{%
\subsubsection{2. Der werdende und der gewordene
Brautstand}\label{der-werdende-und-der-gewordene-brautstand}}

\hypertarget{a-erstes-lied-trautes-liebesgespruxe4ch}{%
\paragraph{a) Erstes Lied: Trautes
Liebesgespräch}\label{a-erstes-lied-trautes-liebesgespruxe4ch}}

\bibleverse{9}»Einem Prachtroß an Pharaos Prunkwagen vergleiche ich
dich, meine Freundin: \bibleverse{10}reizend sind deine Wangen im
Schmuck der Kettchen, dein Hals in den Perlenschnüren!
\bibleverse{11}Goldene Kettchen lassen wir dir machen mit
Kügelchen\textless sup title=``oder: Glöckchen''\textgreater✲ von
Silber.«~-- \bibleverse{12}Solange der König noch in seinem Kreise
weilte\textless sup title=``oder: an seiner Tafel saß''\textgreater✲,
gab meine Narde ihren Duft. \bibleverse{13}Mein Geliebter ist mir wie
ein Myrrhenbündlein, das am Busen mir ruht; \bibleverse{14}ein
Cyprusgebinde ist mir mein Geliebter in den Weinbergen von Engedi.~--
\bibleverse{15}»O schön bist du, meine Freundin, ja, du bist schön!
Deine Augen sind Taubenaugen.«~-- \bibleverse{16}»O schön bist du, mein
Geliebter, ja holdselig! Sieh, unser Lager ist frisches Grün;
\bibleverse{17}unsres Hauses Gebälk sind Zedern, unser Getäfel
Zypressen.«

\hypertarget{b-zweites-lied-wechselgesang-und-verlobung}{%
\paragraph{b) Zweites Lied: Wechselgesang und
Verlobung}\label{b-zweites-lied-wechselgesang-und-verlobung}}

\hypertarget{section-1}{%
\section{2}\label{section-1}}

\bibleverse{1}Ich bin eine Narzisse\textless sup title=``oder:
Herbstzeitlose''\textgreater✲ in Saron, eine Lilie der Täler.~--
\bibleverse{2}»Wie eine Lilie unter den Dornen, so ist meine Freundin
inmitten der Mädchen.«~-- \bibleverse{3}Wie ein Apfelbaum unter den
Bäumen des Waldes, so ist mein Geliebter inmitten der Burschen; in
seinem Schatten begehr' ich zu weilen: seine Frucht ist meinem Gaumen
gar süß. \bibleverse{4}In ein Weinhaus hat er mich (jetzt) geführt, doch
sein Panier über mir ist die Liebe. \bibleverse{5}Stärkt mich doch mit
Traubenkuchen! Erquickt mich mit Äpfeln, denn ich bin liebeskrank!
\bibleverse{6}Seine Linke liegt unter meinem Haupt, und seine Rechte
umfängt mich. \bibleverse{7}O laßt euch beschwören, ihr Töchter
Jerusalems, bei den Gazellen oder den Hinden der Flur: störet die Liebe
nicht auf und wecket sie nicht, bis es ihr selber gefällt!\textless sup
title=``vgl. 8,3-4''\textgreater✲

\hypertarget{c-drittes-lied-liebesfruxfchling}{%
\paragraph{c) Drittes Lied:
Liebesfrühling}\label{c-drittes-lied-liebesfruxfchling}}

\bibleverse{8}Horch! mein Geliebter! Siehe, da kommt er, springt daher
über die Berge, hüpft über die Hügel! \bibleverse{9}Mein Geliebter
gleicht einer Gazelle oder dem jungen Hirsch. Ach sieh, da steht er
hinter unsrer Mauer! Ich schaue durchs Fenster, gucke durchs Gitter!
\bibleverse{10}Mein Geliebter hebt an und ruft mir zu: »Steh auf, meine
Freundin, meine Schöne, und komm! \bibleverse{11}Sieh nur: der Winter
ist dahin, die Regenzeit vorüber, ist vergangen! \bibleverse{12}Die
Blumen zeigen sich auf der Flur, die Zeit der Gesänge ist da, und der
Turteltaube Ruf läßt sich im Land wieder hören; \bibleverse{13}der
Feigenbaum setzt seine Knospen\textless sup title=``oder:
Jungfrüchte''\textgreater✲ an, und der Reben Blüte spendet ihren Duft.
Steh auf, meine Freundin, meine Schöne, und komm! \bibleverse{14}Mein
Täubchen im Felsengeklüft, im Versteck der Felswand, laß mich schauen
dein Antlitz, deine Stimme mich hören! Denn süß ist deine Stimme, dein
Antlitz so lieblich!«

\hypertarget{d-zwei-liebesseufzer-der-braut}{%
\paragraph{d) Zwei Liebesseufzer der
Braut}\label{d-zwei-liebesseufzer-der-braut}}

\bibleverse{15}Fangt uns die Füchse, die kleinen Füchse, die
Weinbergverwüster! Unsre Reben stehn ja in Blüte!~-- \bibleverse{16}Mein
Geliebter ist mein, und ich bin sein; er weidet auf der Lilienau.
\bibleverse{17}Bis der Abendwind haucht und die Schatten fliehn, ergehe
dich frei, mein Geliebter, der Gazelle gleich, oder wie der junge Hirsch
auf zerklüfteten✲ Bergen\textless sup title=``vgl. 8,14''\textgreater✲.

\hypertarget{vom-brautstand-zur-vermuxe4hlung}{%
\subsubsection{3. Vom Brautstand zur
Vermählung}\label{vom-brautstand-zur-vermuxe4hlung}}

\hypertarget{a-erstes-lied-sehnsuchtstraum-der-braut}{%
\paragraph{a) Erstes Lied: Sehnsuchtstraum der
Braut}\label{a-erstes-lied-sehnsuchtstraum-der-braut}}

\hypertarget{section-2}{%
\section{3}\label{section-2}}

\bibleverse{1}Auf meinem Lager in den Nächten, da sucht ich ihn, den
meine Seele liebt: ich suchte ihn und fand ihn nicht. \bibleverse{2}»Ich
will mich doch aufmachen und die Stadt durchstreifen, in den Straßen und
auf den Plätzen will ich ihn suchen, den meine Seele liebt!« Ich suchte
ihn und fand ihn nicht. \bibleverse{3}Mich trafen die Wächter, die in
der Stadt umhergehn: »Habt ihr ihn nicht gesehn, den meine Seele liebt?«
\bibleverse{4}Kaum war ich an ihnen vorüber, da fand ich ihn, den meine
Seele liebt. Ich hielt ihn fest und ließ ihn nicht los, bis ich ihn
gebracht ins Haus meiner Mutter und ins Gemach der Guten, die mir das
Leben gegeben. \bibleverse{5}Ich beschwöre euch, ihr Töchter Jerusalems,
bei den Gazellen oder den Hinden der Flur: störet die Liebe nicht auf
und wecket sie nicht, bis es ihr selber gefällt!

\hypertarget{b-zweites-lied-der-hochzeitszug-des-bruxe4utigams}{%
\paragraph{b) Zweites Lied: Der Hochzeitszug des
Bräutigams}\label{b-zweites-lied-der-hochzeitszug-des-bruxe4utigams}}

\bibleverse{6}Was ist's, das da heraufkommt aus der Trift
wie\textless sup title=``oder: mit''\textgreater✲ Säulen von Rauch,
umduftet von Myrrhe und Weihrauch, von allem Gewürzstaub des Krämers?
\bibleverse{7}Siehe da, es ist Salomos Tragbett✲, rings umgeben von
sechzig Helden aus Israels Kriegern, \bibleverse{8}schwertbewaffnet sie
alle und kriegsgeübt, ein jeder mit seinem Schwert an der Seite zum
Schutz gegen nächtliche Schrecken! \bibleverse{9}Eine Prachtsänfte hat
der König {[}Salomo{]} sich fertigen lassen aus Holz vom Libanon;
\bibleverse{10}ihre Säulen hat er von Silber machen lassen, ihre Lehne
von Gold; ihr Sitz ist von Purpurzeug, das Innere kunstvoll gestickt,
ein Liebesbeweis der Töchter Jerusalems. \bibleverse{11}Kommt heraus,
ihr Töchter Zions, beschaut euch den König Salomo in der
Krone\textless sup title=``=~dem Bräutigamskranz''\textgreater✲, mit der
seine Mutter ihn gekrönt am Tage seiner Hochzeit und am Tage seiner
Herzensfreude!

\hypertarget{c-drittes-lied-beschreibung-der-braut-durch-den-bruxe4utigam}{%
\paragraph{c) Drittes Lied: Beschreibung der Braut durch den
Bräutigam}\label{c-drittes-lied-beschreibung-der-braut-durch-den-bruxe4utigam}}

\hypertarget{section-3}{%
\section{4}\label{section-3}}

\bibleverse{1}O du bist schön, meine Freundin, ja schön bist du! Deine
Augen sind Taubenaugen hinter deinem Schleier hervor; dein Haar gleicht
einer Ziegenherde, die vom Gileadberge herabwallt. \bibleverse{2}Deine
Zähne sind gleich einer Herde von Schafen, die frischgeschoren der
Schwemme entsteigen, allesamt zwillingsträchtig und keins von ihnen
kinderlos. \bibleverse{3}Deine Lippen gleichen einer Purpurschnur, und
dein Mund ist voll Anmut; wie der Spalt\textless sup title=``oder: die
Schnitte''\textgreater✲ eines Granatapfels schimmern die Schläfen dir
hinter dem Schleier hervor. \bibleverse{4}Wie Davids Turm ragt dein
Hals, zur Fernsicht gebaut: tausend Schilde hangen an ihm, lauter Wehren
von Helden. \bibleverse{5}Deine Brüste sind gleich einem Zwillingspaar
junger Gazellen, die unter den Lilien\textless sup title=``oder: auf
Lilienauen''\textgreater✲ weiden. \bibleverse{6}Bis der Abendwind haucht
und die Schatten entfliehn, will zum Myrrhenberge ich gehn und zum
Weihrauchhügel. \bibleverse{7}Wunderschön bist du, meine Freundin;
nichts, nichts fehlt deinen Reizen!

\hypertarget{d-viertes-lied-die-vermuxe4hlung}{%
\paragraph{d) Viertes Lied: Die
Vermählung}\label{d-viertes-lied-die-vermuxe4hlung}}

\bibleverse{8}Komm mit mir vom Libanon, Braut, mit mir vom Libanon, o
komm! Steig herab vom Gipfel des Amana, vom Gipfel des
Senir\textless sup title=``vgl. Hes 27,5''\textgreater✲ und Hermon, von
den Lagerstätten der Löwen, von den Bergen der Panther. \bibleverse{9}Du
hast mich bezaubert, meine bräutliche Schwester, du hast mich bezaubert
mit einem deiner Blicke, mit einem der Kettchen an deinem Halsschmuck!
\bibleverse{10}Wie schön ist deine Liebe, meine bräutliche Schwester,
viel süßer\textless sup title=``oder: köstlicher''\textgreater✲ ist
deine Liebe als Wein, und deiner Salben Duft geht über alle Wohlgerüche!
\bibleverse{11}Von Honigseim triefen deine Lippen, meine Braut; Honig
und Milch birgst du unter deiner Zunge, und deiner Gewänder Duft ist wie
der Duft des Libanons!

\hypertarget{e-fuxfcnftes-lied-vergleich-der-bruxe4utlichen-gattin-mit-einem-herrlichen-garten}{%
\paragraph{e) Fünftes Lied: Vergleich der bräutlichen Gattin mit einem
herrlichen
Garten}\label{e-fuxfcnftes-lied-vergleich-der-bruxe4utlichen-gattin-mit-einem-herrlichen-garten}}

\bibleverse{12}Ein wohlverschloßner Garten ist meine bräutliche
Schwester, ein verschloßner Born, ein versiegelter Quell.
\bibleverse{13}Alles, was an dir sproßt, ist ein Lusthain\textless sup
title=``oder: Paradies''\textgreater✲ von Granaten mit den köstlichsten
Früchten, Zyperblumen samt Narden, \bibleverse{14}Narde und Safran,
Würzrohr und Zimt samt allerlei Weihrauchstauden, Myrrhe und Aloe nebst
allen edelsten Balsamgewächsen. \bibleverse{15}Eine Quelle im Garten
bist du, ein Born voll sprudelnden Wassers, und Bäche, die vom Libanon
rieseln. \bibleverse{16}Erwache, du Nordwind, und komm, du Südwind!
Durchhauche meinen Garten, daß seine Düfte zerfließen! Mein Geliebter
komme in seinen Garten und genieße seine köstlichen Früchte!~--

\hypertarget{der-junge-gatte-nimmt-besitz-von-seinem-garten-das-hochzeitsmahl}{%
\paragraph{Der junge Gatte nimmt Besitz von seinem Garten; das
Hochzeitsmahl}\label{der-junge-gatte-nimmt-besitz-von-seinem-garten-das-hochzeitsmahl}}

\hypertarget{section-4}{%
\section{5}\label{section-4}}

\bibleverse{1}»Ich komme in meinen Garten, meine bräutliche Schwester;
ich pflücke meine Myrrhe samt meinem Balsam, koste meine Wabe samt
meinem Honig, ich trinke meinen Wein samt meiner Milch. Esset, ihr
Freunde, trinkt und sättiget euch an Liebe!«

\hypertarget{eine-neue-gleichartige-reihe-von-liedern-uxfcber-brautstand-und-vermuxe4hlung}{%
\subsubsection{4. Eine neue, gleichartige Reihe von Liedern über
Brautstand und
Vermählung}\label{eine-neue-gleichartige-reihe-von-liedern-uxfcber-brautstand-und-vermuxe4hlung}}

\hypertarget{a-erstes-lied-besuch-des-bruxe4utigams}{%
\paragraph{a) Erstes Lied: Besuch des
Bräutigams}\label{a-erstes-lied-besuch-des-bruxe4utigams}}

\bibleverse{2}Ich schlief, doch mein Herz war wach. Horch! Da klopft
mein Geliebter! »Mach mir auf, meine Schwester, meine Freundin, mein
Täubchen, meine Reine\textless sup title=``oder: Traute''\textgreater✲!
Ach, mein Haupthaar ist voll von Tau, meine Locken voll von den Tropfen
der Nacht!« \bibleverse{3}»Ich habe mein Kleid schon ausgezogen: wie
sollt' ich's wieder anziehn?! Ich habe mir schon die Füße gewaschen: wie
sollt' ich sie wieder beschmutzen?!« \bibleverse{4}Da streckte mein
Geliebter die Hand durch das Guckloch (der Tür), da wallte das Herz mir
auf vor Sehnsucht nach ihm, und die Sinne vergingen mir ob seiner Rede;
\bibleverse{5}ich stand auf, um meinem Geliebten zu öffnen: da troffen
meine Hände von Myrrhe und meine Finger vom köstlichsten Öl am Griff des
Riegels. \bibleverse{6}Ich öffnete meinem Geliebten, doch mein Geliebter
war fort, war verschwunden. Ich suchte ihn und fand ihn nicht; ich rief
nach ihm, doch er gab mir keine Antwort. \bibleverse{7}Es trafen mich
die Wächter, die in der Stadt umhergehn; sie schlugen mich, verwundeten
mich, rissen den Schleier\textless sup title=``oder:
Umhang''\textgreater✲ mir ab, die Wächter der Stadtmauer!

\hypertarget{b-zweites-lied-beschreibung-des-bruxe4utigams-durch-die-braut}{%
\paragraph{b) Zweites Lied: Beschreibung des Bräutigams durch die
Braut}\label{b-zweites-lied-beschreibung-des-bruxe4utigams-durch-die-braut}}

\bibleverse{8}O laßt euch beschwören, ihr Töchter Jerusalems: wenn ihr
antrefft meinen Geliebten, was sollt ihr ihm sagen? Daß krank ich bin
vor Liebe! \bibleverse{9}»Was hat denn dein Geliebter vor andern
Geliebten voraus, du Schönste unter den Weibern? Was hat dein Geliebter
vor andern Geliebten voraus, daß du uns so beschwörst?«
\bibleverse{10}Mein Geliebter ist blendend weiß und braun, kenntlich
unter vielen Tausenden. \bibleverse{11}Sein Haupt ist geläutertes
Feingold, seine Locken wallende Ranken, schwarz wie Raben;
\bibleverse{12}seine Augen wie Tauben an Wasserbächen, die, milchweiß
gebadet, am Teichesrand sitzen; \bibleverse{13}seine Wangen wie
Balsambeete, Gelände duftender Kräuter; seine Lippen sind wie Lilien,
triefend von köstlichster Myrrhe; \bibleverse{14}seine Arme\textless sup
title=``oder: Finger''\textgreater✲ goldene Walzen, mit Edelsteinen
dicht besetzt; sein Leib ein Kunstwerk von Elfenbein, mit Saphiren
übersät; \bibleverse{15}seine Beine Säulen von weißem Marmor, ruhend auf
Sockeln von Feingold; seine Gestalt (ragend) wie der Libanon, großartig
wie die Zedern; \bibleverse{16}sein Gaumen\textless sup title=``oder:
Mund''\textgreater✲ lauter Süße und alles an ihm entzückend! Das ist
mein Geliebter und das mein Freund, ihr Töchter Jerusalems!

\hypertarget{section-5}{%
\section{6}\label{section-5}}

\bibleverse{1}»Wohin ist denn dein Geliebter gegangen, du Schönste unter
den Weibern? Wohin hat dein Geliebter sich begeben, damit wir ihn mit
dir suchen?« \bibleverse{2}Mein Geliebter ist in seinen Garten
hinabgegangen zu den Balsambeeten, um sich in den Anlagen zu ergehen und
Lilien zu pflücken. \bibleverse{3}Ich gehöre meinem Geliebten, und mein
Geliebter gehört mir: er weidet auf der Lilienau.

\hypertarget{c-drittes-lied-beschreibung-der-braut-durch-den-bruxe4utigam-1}{%
\paragraph{c) Drittes Lied: Beschreibung der Braut durch den
Bräutigam}\label{c-drittes-lied-beschreibung-der-braut-durch-den-bruxe4utigam-1}}

\bibleverse{4}Schön bist du, meine Freundin, wie Thirza\textless sup
title=``1.Kön 14,17''\textgreater✲, lieblich wie Jerusalem, doch
furchtbar wie Kriegerscharen! \bibleverse{5}Wende deine Augen weg von
mir, denn sie bringen mich von Sinnen! Dein Haar gleicht einer
Ziegenherde, die vom Gileadberge herabwallt. \bibleverse{6}Deine Zähne
sind wie eine Herde Mutterschafe, die der Schwemme\textless sup
title=``=~dem Bade''\textgreater✲ entsteigen, allesamt zwillingsträchtig
und keins von ihnen ist kinderlos. \bibleverse{7}Wie der
Spalt\textless sup title=``oder: die Schnitte''\textgreater✲ eines
Granatapfels schimmern die Schläfen dir hinter dem Schleier hervor.
\bibleverse{8}Ihrer sechzig sind Königinnen (bei Salomo) und achtzig
Nebenfrauen und zahllos die Jungfrauen. \bibleverse{9}Eine einzige ist
meine Taube, meine Reine\textless sup title=``oder:
Traute''\textgreater✲, die einziggeliebte Tochter ihrer Mutter, das
Herzblatt der Guten, die ihr das Leben gegeben. Wenn die Mädchen sie
sehen, so preisen diese sie glücklich, Königinnen und Nebenfrauen, und
künden ihren Ruhm.

\hypertarget{d-viertes-lied-der-hochzeitszug}{%
\paragraph{d) Viertes Lied: Der Hochzeitszug
(?)}\label{d-viertes-lied-der-hochzeitszug}}

\bibleverse{10}Wer ist diese, die da hervorglänzt wie das Morgenrot,
schön wie der Vollmond, strahlend wie die Sonne, furchtbar wie
Kriegerscharen? \bibleverse{11}In den Nußgarten war ich hinabgegangen,
um mich zu erfreun am jungen Grün des Tales, um nachzusehn, wie der
Weinstock gesproßt, ob die Granaten Blüten getrieben hätten.
\bibleverse{12}Unvermutet hat mein Verlangen mich geführt zu der Tochter
eines Edlen✲.

\hypertarget{e-fuxfcnftes-lied-beschreibung-des-schwert--oder-lager-tanzes-der-braut-lobpreis-ihrer-schuxf6nheit}{%
\paragraph{e) Fünftes Lied: Beschreibung des Schwert- (oder:
Lager-)tanzes der Braut; Lobpreis ihrer
Schönheit}\label{e-fuxfcnftes-lied-beschreibung-des-schwert--oder-lager-tanzes-der-braut-lobpreis-ihrer-schuxf6nheit}}

\hypertarget{section-6}{%
\section{7}\label{section-6}}

\bibleverse{1}Wende dich, wende dich, Sulammith!\textless sup
title=``vgl. 1.Sam 28,4''\textgreater✲ Wende dich, wende dich, daß wir
dich beschauen! »Was wollt ihr schauen an Sulammith beim kriegerischen
Tanz?« \bibleverse{2}Wie schön sind deine Füße\textless sup
title=``oder: Schritte''\textgreater✲ in den Schuhen, du Fürstenkind!
Die Wölbungen deiner Hüften sind wie Halsgeschmeide, ein Werk von
Künstlerhand; \bibleverse{3}dein Schoß eine runde\textless sup
title=``oder: verschlossene''\textgreater✲ Schale, der nie der
Mischtrank fehlen darf; dein Leib ein Weizenhaufen, umsäumt von Lilien.
\bibleverse{4}Deine Brüste sind gleich einem Zwillingspaar junger
Gazellen; \bibleverse{5}dein Hals wie ein Turm von Elfenbein, deine
Augen wie die Teiche von Hesbon am volksbelebten Tor; deine Nase wie der
Libanonturm, der nach Damaskus schaut; \bibleverse{6}dein Haupt droben
wie der Karmel und das herabwallende Haar deines Hauptes wie dunkler
Purpur: ein König liegt gefangen in den Locken\textless sup
title=``oder: Schlingen''\textgreater✲!

\hypertarget{lobpreis-der-braut-durch-den-bruxe4utigam}{%
\paragraph{Lobpreis der Braut durch den
Bräutigam}\label{lobpreis-der-braut-durch-den-bruxe4utigam}}

\bibleverse{7}Wie bist du so schön und so hold, du Geliebte, du
Wonnevolle! \bibleverse{8}Dein Wuchs da gleicht einer Palme und deine
Brüste den Datteltrauben. \bibleverse{9}Ich dachte: Ersteigen will ich
die Palme, ihre Fruchtrispen ergreifen; dann sollen deine Brüste mir
sein wie Trauben am Weinstock und dein Atem süß wie der Duft von Äpfeln
\bibleverse{10}und dein Mund wie der köstlichste Wein, der meinem Gaumen
glatt eingeht und mir über die Lippen und Zähne sanft hinfließt.

\hypertarget{im-ehestande}{%
\subsubsection{5. Im Ehestande}\label{im-ehestande}}

\hypertarget{a-erstes-lied-in-der-heimat-der-gattin}{%
\paragraph{a) Erstes Lied: In der Heimat der
Gattin}\label{a-erstes-lied-in-der-heimat-der-gattin}}

\bibleverse{11}Ich gehöre meinem Geliebten, und nach mir sehnt sich sein
Herz! \bibleverse{12}Komm, mein Geliebter, laß uns aufs Feld hinausgehn,
in den Dörfern übernachten! \bibleverse{13}Frühmorgens brechen wir nach
den Weinbergen auf, wollen nachsehn, ob der Weinstock sproßt, ob die
Blüten sich erschließen, die Granaten blühen: dort will ich dir meine
Liebe schenken. \bibleverse{14}Die Liebesäpfel duften süß, und über
unsrer Tür sind köstliche Früchte jeder Art, heurige und jährige: für
dich, mein Geliebter, hab' ich sie aufbewahrt.

\hypertarget{anhang-ein-lied-aus-der-brautzeit-geliebter-und-bruder}{%
\paragraph{Anhang (ein Lied aus der Brautzeit): Geliebter und
Bruder}\label{anhang-ein-lied-aus-der-brautzeit-geliebter-und-bruder}}

\hypertarget{section-7}{%
\section{8}\label{section-7}}

\bibleverse{1}Ach, wärst du doch mein Bruder, hättest die Brust meiner
Mutter gesogen! Träf' ich dich dann auf der Straße, so dürft' ich dich
küssen, ohne daß jemand mich deshalb mißachtete. \bibleverse{2}Ich nähme
dich dann mit mir ins Haus meiner Mutter; du müßtest mich unterweisen,
ich gäbe dir Würzwein zu trinken, den Most meiner Granaten.~--
\bibleverse{3}Seine Linke liegt unter meinem Haupt, und seine Rechte
umfängt mich. \bibleverse{4}Ich beschwöre euch, ihr Töchter Jerusalems:
was wollt ihr die Liebe aufstören und wecken, eh' es ihr selber
gefällt!\textless sup title=``vgl. 2,6-7''\textgreater✲

\hypertarget{b-zweites-lied-am-ziel-in-der-heimat-des-gatten}{%
\paragraph{b) Zweites Lied: Am Ziel in der Heimat des
Gatten}\label{b-zweites-lied-am-ziel-in-der-heimat-des-gatten}}

\bibleverse{5}Wer ist's, die da heraufkommt aus der Trift, gelehnt an
ihren Geliebten? »Unter dem Apfelbaum hab' ich dich aufgeweckt; dort hat
deine Mutter dich mit Schmerzen geboren, dort dich mit Schmerzen ans
Licht der Welt gebracht.« \bibleverse{6}»O lege mich an dein Herz wie
einen Siegelring, wie einen Siegelring an deinen Arm! Denn stark wie der
Tod ist die Liebe und ihre Leidenschaft hart\textless sup
title=``=~unerbittlich oder: unbezwinglich''\textgreater✲ wie die
Unterwelt; ihre Gluten sind Feuergluten, ihre Flammen wie Flammen
Gottes. \bibleverse{7}Die mächtigsten Fluten vermögen die Liebe nicht
auszulöschen und Ströme sie nicht fortzuschwemmen\textless sup
title=``oder: zu überfluten''\textgreater✲; böt' einer auch alles Gut
seines Hauses (als Kaufpreis) für die Liebe: man würde sein nur
spotten.«

\hypertarget{anhang-von-drei-scherzworten}{%
\subsubsection{6. Anhang von drei
Scherzworten}\label{anhang-von-drei-scherzworten}}

\hypertarget{a-lied-vom-schwesterlein-das-die-pluxe4ne-der-habgierigen-bruxfcder-vereitelt}{%
\paragraph{a) Lied vom Schwesterlein, das die Pläne der habgierigen
Brüder
vereitelt}\label{a-lied-vom-schwesterlein-das-die-pluxe4ne-der-habgierigen-bruxfcder-vereitelt}}

\bibleverse{8}Ein Schwesterlein haben wir, die noch keine Brüste hat;
was sollen wir nun mit unserer Schwester tun am Tage, wo man um sie
freit? \bibleverse{9}Ist sie eine Mauer, so bauen wir eine
Krönung\textless sup title=``oder: Zinne''\textgreater✲ von Silber auf
ihr; ist sie aber ein Tor, so verrammeln wir es mit Zederbohlen.«~--
\bibleverse{10}Ich bin\textless sup title=``oder: war''\textgreater✲
eine Mauer, und meine Brüste sind\textless sup title=``oder:
waren''\textgreater✲ wie Türme; doch ich habe mich ihm gezeigt als
friedlich übergebene Burg.

\hypertarget{b-lied-von-den-zwei-weinbergen}{%
\paragraph{b) Lied von den zwei
Weinbergen}\label{b-lied-von-den-zwei-weinbergen}}

\bibleverse{11}Einen Weinberg besaß Salomo in Baal-Hamon; er übergab den
Weinberg den Hütern; jeder hatte für seinen Ertrag tausend Silberstücke
zu zahlen. \bibleverse{12}Über meinen Weinberg verfüge ich allein. Die
tausend Silberstücke gehören dir, (mein) Salomo, und dazu noch
zweihundert den Hütern seiner Früchte.

\hypertarget{c-schluuxdfwort}{%
\paragraph{c) Schlußwort}\label{c-schluuxdfwort}}

\bibleverse{13}Die du wohnst in den Gärten, die Freunde lauschen: deine
Stimme laß mich hören! \bibleverse{14}»Enteile, mein Geliebter, und
mache es wie die Gazelle oder wie der junge Hirsch auf den
balsamduftenden Bergen!«\textless sup title=``vgl. 2,17''\textgreater✲
