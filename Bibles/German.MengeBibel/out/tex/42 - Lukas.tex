\hypertarget{die-heilsbotschaft-nach-lukas}{%
\section{DIE HEILSBOTSCHAFT NACH
LUKAS}\label{die-heilsbotschaft-nach-lukas}}

\hypertarget{vorwort}{%
\subsection{Vorwort}\label{vorwort}}

\hypertarget{section}{%
\section{1}\label{section}}

\bibleverse{1} Weil\textless sup title=``oder: nachdem''\textgreater✲
bekanntlich schon viele es unternommen haben, einen Bericht über die
Begebenheiten, die sich unter uns erfüllt✲ haben, so abzufassen,
\bibleverse{2} wie die Männer sie uns überliefert haben, die von
Anbeginn an Augenzeugen und (alsdann) Diener des Wortes gewesen sind,
\bibleverse{3} habe auch ich mich entschlossen, nachdem ich allen
Tatsachen von den Anfängen an sorgfältig nachgegangen bin\textless sup
title=``=~nachgeforscht habe''\textgreater✲, alles für dich, hochedler
Theophilus, in richtiger\textless sup title=``oder:
sachgemäßer''\textgreater✲ Reihenfolge aufzuzeichnen, \bibleverse{4}
damit du dich von der Zuverlässigkeit der Nachrichten\textless sup
title=``oder: Lehren''\textgreater✲, in denen du unterwiesen worden
bist, überzeugen kannst.

\hypertarget{i.-die-vorgeschichte-15-252}{%
\subsection{I. Die Vorgeschichte
(1,5-2,52)}\label{i.-die-vorgeschichte-15-252}}

\hypertarget{ankuxfcndigung-der-geburt-johannes-des-tuxe4ufers}{%
\subsubsection{1. Ankündigung der Geburt Johannes des
Täufers}\label{ankuxfcndigung-der-geburt-johannes-des-tuxe4ufers}}

\bibleverse{5} Es lebte zur Zeit des jüdischen Königs Herodes ein
Priester namens Zacharias aus der Priesterabteilung Abia; der hatte eine
Frau aus der Zahl der Töchter\textless sup title=``=~der weiblichen
Nachkommen''\textgreater✲ Aarons, die Elisabeth hieß. \bibleverse{6} Sie
waren beide gerecht✲ vor Gott und wandelten in allen Geboten und
Satzungen des Herrn ohne Tadel. \bibleverse{7} Sie hatten jedoch kein
Kind, weil der Elisabeth Mutterfreuden versagt waren, und beide standen
schon in vorgerücktem Alter.

\bibleverse{8} Da begab es sich einst, als er nach der Ordnung seiner
Abteilung den Priesterdienst vor Gott zu verrichten hatte,
\bibleverse{9} daß er nach dem Brauch der Priesterschaft durch das Los
dazu bestimmt wurde, in den Tempel des Herrn zu gehen und dort das
Rauchopfer darzubringen\textless sup title=``2.Mose 30,7; 1.Chr
24,19''\textgreater✲, \bibleverse{10} während die ganze Volksmenge
draußen zur Stunde des Rauchopfers dem Gebet oblag. \bibleverse{11} Da
erschien ihm ein Engel des Herrn, der stand auf der rechten Seite des
Rauchopferaltars. \bibleverse{12} Bei seinem Anblick erschrak Zacharias,
und Furcht befiel ihn; \bibleverse{13} der Engel aber sagte zu ihm:
»Fürchte dich nicht, Zacharias, denn dein Gebet hat Erhörung gefunden,
und deine Frau Elisabeth wird dir Mutter eines Sohnes werden, dem du den
Namen Johannes geben sollst. \bibleverse{14} Du wirst Freude und Jubel✲
darüber empfinden, und viele werden sich über seine Geburt freuen,
\bibleverse{15} denn er wird groß vor dem Herrn sein; Wein und (andere)
berauschende Getränke wird er nicht genießen\textless sup title=``Ri
13,4-5; 1.Sam 1,11''\textgreater✲, und mit heiligem Geist wird er schon
von Geburt an erfüllt werden. \bibleverse{16} Viele von den Söhnen
Israels wird er zum Herrn, ihrem Gott, zurückführen; \bibleverse{17} und
er ist es, der vor ihm\textless sup title=``d.h. dem
Herrn''\textgreater✲ einhergehen wird im Geist und in der Kraft des
Elia, um die Herzen der Väter den Kindern wieder zuzuwenden\textless sup
title=``Mal 3,1.23-24''\textgreater✲ und die Ungehorsamen zur Gesinnung
der Gerechten (zu führen), um dem Herrn ein wohlbereitetes Volk zu
schaffen.« \bibleverse{18} Da sagte Zacharias zu dem Engel: »Wie soll
ich das für möglich halten? Ich selbst bin ja ein alter Mann, und meine
Frau ist auch schon betagt.«\textless sup title=``1.Mose 15,8;
18,11''\textgreater✲ \bibleverse{19} Da antwortete ihm der Engel: »Ich
bin Gabriel, der (als Diener) vor Gottes Angesicht steht, und bin
gesandt, um zu dir zu reden und dir diese frohe Botschaft zu
verkündigen. \bibleverse{20} Doch nun vernimm: Du wirst stumm sein und
nicht reden können bis zu dem Tage, an dem diese meine Verheißung sich
erfüllt, weil du meinen Worten nicht geglaubt hast, die zu ihrer Zeit in
Erfüllung gehen werden.«

\bibleverse{21} Das Volk wartete unterdessen auf Zacharias und wunderte
sich darüber, daß er so lange im Tempel verweilte. \bibleverse{22} Als
er endlich heraustrat, konnte er nicht zu ihnen reden; da merkten sie,
daß er eine Erscheinung im Tempel gesehen hatte; und er seinerseits
suchte sich ihnen durch Kopfnicken\textless sup title=``oder:
Winke''\textgreater✲ verständlich zu machen, blieb aber stumm.
\bibleverse{23} Als dann die (sieben) Tage seines Priesterdienstes zu
Ende waren, kehrte er heim in sein Haus.

\bibleverse{24} Nach diesen Tagen aber wurde seine Frau Elisabeth guter
Hoffnung; sie hielt sich fünf Monate lang in tiefer Zurückgezogenheit
und dachte: \bibleverse{25} »So hat der Herr an mir getan in der Zeit,
die er dazu ersehen hat, meine Schmach bei den Menschen von mir
hinwegzunehmen.«\textless sup title=``1.Mose 30,23''\textgreater✲

\hypertarget{ankuxfcndigung-der-geburt-jesu}{%
\subsubsection{2. Ankündigung der Geburt
Jesu}\label{ankuxfcndigung-der-geburt-jesu}}

\bibleverse{26} Im sechsten Monat aber wurde der Engel Gabriel von Gott
nach Galiläa in eine Stadt namens Nazareth gesandt \bibleverse{27} zu
einer Jungfrau, die mit einem Manne namens Joseph aus dem Hause Davids
verlobt war; die Jungfrau hieß Maria. \bibleverse{28} Als nun der Engel
bei ihr eintrat, sagte er: »Sei gegrüßt, du Begnadete: der Herr ist mit
dir!« \bibleverse{29} Sie wurde über diese Anrede bestürzt und
überlegte\textless sup title=``=~konnte sich nicht
erklären''\textgreater✲, was dieser Gruß zu bedeuten habe.
\bibleverse{30} Da sagte der Engel zu ihr: »Fürchte dich nicht, Maria,
denn du hast Gnade bei Gott gefunden! \bibleverse{31} Wisse wohl: du
wirst guter Hoffnung werden und Mutter eines Sohnes, dem du den Namen
Jesus\textless sup title=``vgl. Mt 1,21''\textgreater✲ geben sollst.
\bibleverse{32} Dieser wird groß sein und Sohn des Höchsten genannt
werden, und Gott der Herr wird ihm den Thron seines Vaters David geben,
\bibleverse{33} und er wird als König über das Haus Jakobs in alle
Ewigkeit herrschen, und sein Königtum wird kein Ende
haben.«\textless sup title=``Jes 9,7; 2.Sam 7,12-13''\textgreater✲

\bibleverse{34} Da sagte Maria zu dem Engel: »Wie soll das möglich sein?
Ich weiß doch von keinem Manne.« \bibleverse{35} Da gab der Engel ihr
zur Antwort: »Heiliger Geist wird über dich kommen und die Kraft des
Höchsten dich überschatten; daher wird auch das Heilige, das (von dir)
geboren werden soll, Gottes Sohn genannt werden. \bibleverse{36} Und nun
vernimm: Elisabeth, deine Verwandte, ist ebenfalls trotz ihres hohen
Alters mit einem Sohn gesegnet und steht jetzt schon im sechsten Monat,
sie, die man unfruchtbar nennt; \bibleverse{37} denn bei Gott ist kein
Ding unmöglich.«\textless sup title=``1.Mose 18,14''\textgreater✲
\bibleverse{38} Da sagte Maria: »Siehe, ich bin des Herrn Magd: mir
geschehe nach deinem Wort!« Damit schied der Engel von ihr.

\hypertarget{besuch-der-maria-bei-elisabeth}{%
\subsubsection{3. Besuch der Maria bei
Elisabeth}\label{besuch-der-maria-bei-elisabeth}}

\hypertarget{a-die-begegnung-der-beiden-muxfctter-elisabeths-begruxfcuxdfung}{%
\paragraph{a) Die Begegnung der beiden Mütter; Elisabeths
Begrüßung}\label{a-die-begegnung-der-beiden-muxfctter-elisabeths-begruxfcuxdfung}}

\bibleverse{39} Maria aber machte sich in jenen Tagen auf und wanderte
eilends in das Bergland nach einer Stadt (im Stamme) Juda;
\bibleverse{40} dort trat sie in das Haus des Zacharias und begrüßte
Elisabeth. \bibleverse{41} Da begab es sich, als Elisabeth den Gruß der
Maria vernahm, da bewegte sich das Kind lebhaft in ihrem Leibe; und
Elisabeth wurde mit heiligem Geist erfüllt \bibleverse{42} und brach mit
lauter Stimme in die Worte aus: »Gesegnet\textless sup title=``oder:
gepriesen''\textgreater✲ bist du unter den Frauen, und gesegnet ist die
Frucht deines Leibes! \bibleverse{43} Doch woher wird mir die
Ehre\textless sup title=``oder: das Glück''\textgreater✲ zuteil, daß die
Mutter meines Herrn zu mir kommt? \bibleverse{44} Denn wisse: als der
Klang deines Grußes mir ins Ohr drang, bewegte sich das Kind vor Freude
lebhaft in meinem Leibe. \bibleverse{45} O selig die, welche geglaubt
hat, denn die Verheißung, die der Herr ihr gegeben hat, wird in
Erfüllung gehen!«

\hypertarget{b-lobgesang-der-maria-das-sogenannte-magnifikat}{%
\paragraph{b) Lobgesang der Maria (das sogenannte
›Magnifikat‹)}\label{b-lobgesang-der-maria-das-sogenannte-magnifikat}}

\bibleverse{46} Darauf sprach Maria\textless sup title=``vgl. 1.Sam
2,1-10''\textgreater✲: »Meine Seele erhebt den Herrn, \bibleverse{47}
und mein Geist jubelt über Gott, meinen Retter\textless sup
title=``oder: Heiland; Hab 3,18''\textgreater✲, \bibleverse{48} weil er
die Niedrigkeit seiner Magd angesehen hat!\textless sup title=``1.Sam
1,11''\textgreater✲ Denn siehe: von nun an werden alle Geschlechter mich
selig preisen, \bibleverse{49} weil der Allmächtige Großes an mir getan
hat\textless sup title=``5.Mose 10,21''\textgreater✲. Ja, heilig ist
sein Name\textless sup title=``Ps 111,9''\textgreater✲, \bibleverse{50}
und sein Erbarmen wird von Geschlecht zu Geschlecht denen zuteil, die
ihn fürchten\textless sup title=``Ps 103,17''\textgreater✲.
\bibleverse{51} Er wirkt seine Kraft aus mit seinem Arm, er zerstreut,
die da hoffärtig sind in ihres Herzens Sinn\textless sup title=``Ps
89,11''\textgreater✲ \bibleverse{52} er stürzt Machthaber von den
Thronen und erhöht Niedrige\textless sup title=``Ps 147,6; Hiob
5,11''\textgreater✲; \bibleverse{53} Hungrige sättigt er mit Gütern und
läßt Reiche leer ausgehen\textless sup title=``Ps 107,9; 34,11; 1.Sam
2,5.7-8''\textgreater✲. \bibleverse{54} Er hat sich Israels angenommen,
seines Knechts, um der Barmherzigkeit zu gedenken\textless sup
title=``Jes 41,8; Ps 98,3''\textgreater✲, \bibleverse{55} wie er es
unsern Vätern verheißen hat, dem Abraham und seinen Nachkommen in
Ewigkeit.«\textless sup title=``Mi 7,20; 1.Mose 17,7''\textgreater✲
\bibleverse{56} Maria blieb dann etwa drei Monate bei Elisabeth und
kehrte hierauf in ihr Haus zurück.

\hypertarget{geburt-beschneidung-und-jugend-des-johannes-lobgesang-des-zacharias}{%
\subsubsection{4. Geburt, Beschneidung und Jugend des Johannes;
Lobgesang des
Zacharias}\label{geburt-beschneidung-und-jugend-des-johannes-lobgesang-des-zacharias}}

\bibleverse{57} Für Elisabeth aber erfüllte sich die Zeit ihrer
Niederkunft, und sie wurde Mutter eines Sohnes. \bibleverse{58} Als nun
ihre Nachbarn und Verwandten hörten, daß der Herr ihr so große
Barmherzigkeit erwiesen hatte, freuten sie sich mit ihr. \bibleverse{59}
Am achten Tage kamen sie zur Beschneidung des Knäbleins und wollten es
mit dem Namen seines Vaters Zacharias benennen; \bibleverse{60} doch
seine Mutter sagte abwehrend: »Nein, er soll Johannes heißen!«
\bibleverse{61} Sie entgegneten ihr: »In deiner Verwandtschaft gibt es
doch keinen, der diesen Namen führt.« \bibleverse{62} Sie winkten nun
seinem Vater die Frage zu, wie er ihn benannt haben wolle.
\bibleverse{63} Der forderte ein Täfelchen und schrieb die Worte darauf:
»Johannes ist sein Name!«, und alle verwunderten sich darüber.
\bibleverse{64} In demselben Augenblick aber wurde ihm der Mund
aufgetan, und das Band seiner Zunge (löste sich): er konnte wieder reden
und pries Gott. \bibleverse{65} Da kam Furcht über alle, die in ihrer
Nachbarschaft wohnten, und im ganzen Bergland von Judäa wurden alle
diese Begebenheiten viel besprochen, \bibleverse{66} und alle, die von
ihnen hörten, nahmen sie sich zu Herzen und sagten\textless sup
title=``oder: dachten''\textgreater✲: »Was wird wohl aus diesem Kinde
werden?« Denn auch die Hand des Herrn war mit ihm.

\hypertarget{prophetisches-loblied-des-zacharias-das-sogenannte-benediktus}{%
\paragraph{Prophetisches Loblied des Zacharias (das sogenannte
›Benediktus‹)}\label{prophetisches-loblied-des-zacharias-das-sogenannte-benediktus}}

\bibleverse{67} Und sein Vater Zacharias wurde mit heiligem Geist
erfüllt und sprach die prophetischen Worte aus: \bibleverse{68}
»Gepriesen sei der Herr, der Gott Israels!\textless sup title=``Ps
41,14; 72,18''\textgreater✲, Denn er hat sein Volk gnädig angesehen und
ihm eine Erlösung geschaffen\textless sup title=``Ps
111,4''\textgreater✲ \bibleverse{69} und hat uns ein Horn des Heils
aufgerichtet im Hause Davids, seines Knechtes\textless sup title=``Ps
132,17; 1.Sam 2,10''\textgreater✲. \bibleverse{70} So hat er es durch
den Mund seiner heiligen Propheten von alters her verheißen:
\bibleverse{71} retten will er uns von unsern Feinden und aus der Hand
aller, die uns hassen\textless sup title=``Ps 106,10''\textgreater✲,
\bibleverse{72} um unsern Vätern Barmherzigkeit zu erweisen\textless sup
title=``Mi 7,20''\textgreater✲ und seines heiligen Bundes zu
gedenken\textless sup title=``Ps 105,8-9; 106,45; 1.Mose
17,7''\textgreater✲, \bibleverse{73} des Eides, den er unserm Vater
Abraham geschworen hat\textless sup title=``1.Mose 22,16-17; Jer
11,5''\textgreater✲, er wolle uns erretten aus der Hand unserer Feinde
\bibleverse{74} und uns verleihen, daß wir ihm furchtlos dienen
\bibleverse{75} in Heiligkeit und Gerechtigkeit vor seinen Augen alle
Tage unsers Lebens. \bibleverse{76} Aber auch du, Knäblein, wirst ein
Prophet des Höchsten genannt werden; denn du wirst vor dem Herrn
einhergehen, ihm die Wege zu bereiten\textless sup title=``Mal
3,1''\textgreater✲, \bibleverse{77} um seinem Volke die Erkenntnis des
Heils zu verschaffen, die ihnen durch Vergebung ihrer Sünden zuteil
werden wird\textless sup title=``Jer 31,34''\textgreater✲.
\bibleverse{78} So will es das herzliche Erbarmen unsers Gottes, mit dem
uns der Aufgang aus der Höhe erschienen ist\textless sup title=``Jes
60,1-2; Mal 3,20''\textgreater✲, \bibleverse{79} um denen Licht zu
spenden, die in Finsternis und Todesschatten sitzen\textless sup
title=``Jes 9,2''\textgreater✲, und unsere Füße✲ auf den Weg des
Friedens zu leiten.«~-- \bibleverse{80} Das Knäblein aber wuchs heran
und wurde stark am Geist und hielt sich in der Einöde auf bis zum Tage
seines öffentlichen Auftretens vor Israel.

\hypertarget{geburt-jesu-in-bethlehem}{%
\subsubsection{5. Geburt Jesu in
Bethlehem}\label{geburt-jesu-in-bethlehem}}

\hypertarget{a-der-erlauxdf-des-kaisers-augustus-und-seine-bedeutung-fuxfcr-die-geburt-jesu}{%
\paragraph{a) Der Erlaß des Kaisers Augustus und seine Bedeutung für die
Geburt
Jesu}\label{a-der-erlauxdf-des-kaisers-augustus-und-seine-bedeutung-fuxfcr-die-geburt-jesu}}

\hypertarget{section-1}{%
\section{2}\label{section-1}}

\bibleverse{1} Es begab sich aber in jenen Tagen, daß eine Verordnung
vom Kaiser Augustus ausging, es solle eine Volkszählung\textless sup
title=``oder: Einschätzung''\textgreater✲ im ganzen römischen Reich
vorgenommen werden. \bibleverse{2} Es war dies die erste
Zählung\textless sup title=``oder: Schätzung''\textgreater✲, die zu der
Zeit stattfand, als Quirinius Statthalter in Syrien war. \bibleverse{3}
Da machten alle sich auf, um sich in die Listen eintragen\textless sup
title=``oder: sich einschätzen''\textgreater✲ zu lassen, ein jeder in
seinem (Heimats-) Ort. \bibleverse{4} So zog denn auch Joseph von
Galiläa aus der Stadt Nazareth nach Judäa hinauf nach der Stadt Davids,
die Bethlehem heißt, weil er aus Davids Hause und Geschlecht stammte,
\bibleverse{5} um sich daselbst mit Maria, seiner jungen Ehefrau, die
guter Hoffnung war, einschätzen zu lassen. \bibleverse{6} Während ihres
dortigen Aufenthalts kam aber für Maria die Stunde ihrer Niederkunft,
\bibleverse{7} und sie gebar ihren ersten Sohn, den sie in Windeln
wickelte und in eine Krippe legte, weil es sonst keinen Platz in der
Herberge für sie gab.

\hypertarget{b-die-hirten-auf-dem-felde-und-die-engelerscheinung}{%
\paragraph{b) Die Hirten auf dem Felde und die
Engelerscheinung}\label{b-die-hirten-auf-dem-felde-und-die-engelerscheinung}}

\bibleverse{8} Nun waren Hirten in derselben Gegend auf freiem Felde und
hielten in jener Nacht Wache bei ihrer Herde. \bibleverse{9} Da trat ein
Engel des Herrn zu ihnen, und die Herrlichkeit\textless sup
title=``=~der Lichtglanz''\textgreater✲ des Herrn umleuchtete sie, und
sie gerieten in große Furcht. \bibleverse{10} Der Engel aber sagte zu
ihnen: »Fürchtet euch nicht! Denn wisset wohl: ich verkündige euch große
Freude, die dem ganzen Volke widerfahren wird; \bibleverse{11} denn euch
ist heute ein Retter\textless sup title=``oder: Heiland''\textgreater✲
geboren, welcher ist Christus\textless sup title=``=~der Messias; vgl.
Mt 1,16''\textgreater✲, der Herr, in der Stadt Davids. \bibleverse{12}
Und dies sei das Erkennungszeichen für euch: Ihr werdet ein neugeborenes
Kind finden, das in Windeln gewickelt ist und in einer Krippe liegt.«
\bibleverse{13} Und plötzlich war bei dem Engel die Menge der
himmlischen Heerscharen, die Gott priesen mit den Worten:
\bibleverse{14} »Ehre sei Gott in Himmelshöhen und Friede auf Erden
in\textless sup title=``oder: unter''\textgreater✲ den Menschen des
(göttlichen) Wohlgefallens!«

\hypertarget{c-die-hirten-bei-dem-jesuskind-in-bethlehem}{%
\paragraph{c) Die Hirten bei dem Jesuskind in
Bethlehem}\label{c-die-hirten-bei-dem-jesuskind-in-bethlehem}}

\bibleverse{15} Als hierauf die Engel von ihnen weg in den Himmel
zurückgekehrt waren, sagten die Männer, die Hirten, zueinander: »Wir
wollen doch bis Bethlehem hinübergehen und uns die Sache ansehen, die
sich dort begeben hat und die der Herr uns hat verkünden lassen!«
\bibleverse{16} So gingen sie denn eilends hin und fanden Maria und
Joseph, dazu das Kind, das in der Krippe lag. \bibleverse{17} Als sie es
gesehen hatten, teilten sie ihnen die Verkündigung mit, die sie über
dieses Kind vernommen hatten; \bibleverse{18} und alle, die es hörten,
verwunderten sich über den Bericht der Hirten. \bibleverse{19} Maria
aber bewahrte alle diese Mitteilungen im Gedächtnis und bedachte sie in
ihrem Herzen. \bibleverse{20} Die Hirten aber kehrten wieder zurück; sie
priesen und lobten Gott für alles, was sie gehört und gesehen hatten
genau so, wie es ihnen (von den Engeln) verkündigt worden war.

\hypertarget{jesu-beschneidung-und-darstellung-im-tempel}{%
\subsubsection{6. Jesu Beschneidung und Darstellung im
Tempel}\label{jesu-beschneidung-und-darstellung-im-tempel}}

\bibleverse{21} Als dann acht Tage vergangen waren, so daß man das Kind
beschneiden mußte\textless sup title=``3.Mose 12,3''\textgreater✲, gab
man ihm den Namen Jesus✲, der schon vor seiner Empfängnis von dem Engel
angegeben worden war.

\bibleverse{22} Als dann die (vierzig) nach dem mosaischen
Gesetz\textless sup title=``3.Mose 12,2-8''\textgreater✲ für ihre
Reinigung vorgeschriebenen Tage zu Ende waren, brachten sie das Kind
nach Jerusalem hinauf, um es dem Herrn darzustellen\textless sup
title=``=~zu heiligen oder: zu weihen''\textgreater✲~-- \bibleverse{23}
wie im Gesetz des Herrn geschrieben steht\textless sup title=``2.Mose
13,2.12''\textgreater✲: »Jedes erstgeborene männliche Kind, das zur Welt
kommt, soll als dem Herrn geheiligt\textless sup title=``=~geweiht; vgl.
1,35''\textgreater✲ gelten« --; \bibleverse{24} zugleich wollten sie das
Opfer nach der Vorschrift im Gesetz des Herrn\textless sup
title=``3.Mose 12,8''\textgreater✲ darbringen, nämlich ein Paar
Turteltauben oder zwei junge Tauben.

\hypertarget{des-greisen-simeon-begruxfcuxdfung-lobgesang-und-prophezeiung}{%
\subsubsection{7. Des greisen Simeon Begrüßung, Lobgesang und
Prophezeiung}\label{des-greisen-simeon-begruxfcuxdfung-lobgesang-und-prophezeiung}}

\bibleverse{25} Und siehe, da lebte ein Mann in Jerusalem namens Simeon;
dieser Mann war gerecht✲ und gottesfürchtig; er wartete auf die Tröstung
Israels, und heiliger Geist war auf ihm. \bibleverse{26} Vom heiligen
Geist war ihm auch geoffenbart worden, er solle den Tod nicht eher
sehen, bevor er den Gesalbten des Herrn gesehen hätte. \bibleverse{27}
So kam er denn damals, vom Geist getrieben, in den Tempel; und als die
Eltern das Jesuskind hineinbrachten, um nach dem Brauch\textless sup
title=``oder: der Vorschrift''\textgreater✲ des Gesetzes mit ihm zu
verfahren, \bibleverse{28} da nahm auch er es in seine Arme und pries
Gott mit den Worten: \bibleverse{29} »Herr, nun entläßt du deinen
Knecht, wie du ihm verheißen hast\textless sup title=``vgl.
V.26''\textgreater✲, im Frieden; \bibleverse{30} denn meine Augen haben
dein Heil gesehen\textless sup title=``Jes 40,5''\textgreater✲,
\bibleverse{31} das du vor den Augen aller Völker\textless sup
title=``Jes 52,10''\textgreater✲ bereitet hast, \bibleverse{32} ein
Licht zur Erleuchtung der Heiden\textless sup title=``Jes 42,6;
49,6''\textgreater✲ und zur Verherrlichung deines Volkes Israel.«
\bibleverse{33} Die beiden Eltern Jesu verwunderten sich über das, was
da über das Kind gesagt wurde. \bibleverse{34} Simeon aber segnete sie
und sagte zu Maria, seiner Mutter: »Wisse wohl: dieser ist vielen zum
Fallen und (vielen) zum Aufstehen in Israel bestimmt und zu einem
Zeichen, das Widerspruch erfährt~-- \bibleverse{35} und auch dir selbst
wird ein Schwert durch die Seele dringen --, auf daß aus vielen Herzen
die Gedanken offenbar werden.«

\hypertarget{begruxfcuxdfung-des-kindes-durch-die-greise-hanna-ruxfcckkehr-der-heiligen-familie-nach-nazareth}{%
\subsubsection{8. Begrüßung des Kindes durch die greise Hanna; Rückkehr
der heiligen Familie nach
Nazareth}\label{begruxfcuxdfung-des-kindes-durch-die-greise-hanna-ruxfcckkehr-der-heiligen-familie-nach-nazareth}}

\bibleverse{36} Es war da auch eine Prophetin Hanna, eine Tochter
Phanuels aus dem Stamme Asser, die war hochbetagt; nur sieben Jahre
hatte sie nach ihrer Mädchenzeit mit ihrem Manne gelebt \bibleverse{37}
und war dann Witwe geblieben bis (zum Alter von) vierundachtzig Jahren.
Sie verließ den Tempel nicht und diente Gott mit Fasten und Beten bei
Tag und bei Nacht. \bibleverse{38} Diese trat auch in eben dieser Stunde
hinzu, pries Gott und redete von ihm\textless sup title=``d.h. von dem
Kinde''\textgreater✲ zu allen, die auf die Erlösung
Jerusalems\textless sup title=``oder: Israels''\textgreater✲ warteten.

\bibleverse{39} Nachdem sie dann alles nach den Vorschriften im Gesetz
des Herrn erfüllt hatten, kehrten sie nach Galiläa in ihre Stadt
Nazareth zurück. \bibleverse{40} Der Knabe aber wuchs heran und wurde
kräftig und mit Weisheit erfüllt, und die Gnade Gottes war über
ihm\textless sup title=``=~ruhte auf ihm''\textgreater✲.

\hypertarget{der-zwuxf6lfjuxe4hrige-jesusknabe-im-tempel}{%
\subsubsection{9. Der zwölfjährige Jesusknabe im
Tempel}\label{der-zwuxf6lfjuxe4hrige-jesusknabe-im-tempel}}

\bibleverse{41} Seine Eltern pflegten aber alle Jahre zum Passahfest
nach Jerusalem zu wandern\textless sup title=``2.Mose
23,14-17''\textgreater✲. \bibleverse{42} Als er nun zwölf Jahre alt
geworden war und sie wie gewöhnlich zur Festzeit hinaufgezogen waren,
\bibleverse{43} blieb, als sie die Festtage dort zugebracht hatten und
sie sich auf den Heimweg machten, der Knabe Jesus in Jerusalem zurück,
ohne daß seine Eltern es bemerkten. \bibleverse{44} In der Meinung, er
befinde sich unter der Reisegesellschaft, gingen sie eine Tagereise weit
und suchten ihn bei den Verwandten und Bekannten; \bibleverse{45} als
sie ihn aber dort nicht fanden, kehrten sie nach Jerusalem zurück und
suchten ihn dort. \bibleverse{46} Nach drei Tagen endlich fanden sie
ihn, wie er im Tempel mitten unter den Lehrern saß und ihnen zuhörte und
auch Fragen an sie richtete; \bibleverse{47} und alle, die ihn hörten,
staunten über sein Verständnis und seine Antworten. \bibleverse{48} Als
sie\textless sup title=``d.h. seine Eltern''\textgreater✲ ihn dort
erblickten, wurden sie betroffen, und seine Mutter sagte zu ihm: »Kind,
warum hast du uns das angetan? Bedenke doch: dein Vater und ich suchen
dich mit Angst!« \bibleverse{49} Da antwortete er ihnen: »Wie habt ihr
mich nur suchen können? Wußtet ihr nicht, daß ich im Hause meines Vaters
sein muß?« \bibleverse{50} Sie verstanden aber das Wort nicht, das er zu
ihnen gesagt hatte.~-- \bibleverse{51} Er kehrte dann mit ihnen nach
Nazareth zurück und war ihnen untertan\textless sup title=``=~ein
gehorsamer Sohn''\textgreater✲, und seine Mutter bewahrte alle diese
Worte\textless sup title=``oder: Vorkommnisse''\textgreater✲ in ihrem
Herzen. \bibleverse{52} Jesus aber nahm an Weisheit, Körpergröße und
Gnade✲ bei Gott und den Menschen zu\textless sup title=``1.Sam
2,26''\textgreater✲.

\hypertarget{ii.-einleitung-zur-uxf6ffentlichen-wirksamkeit-jesu-31-413}{%
\subsection{II. Einleitung zur öffentlichen Wirksamkeit Jesu
(3,1-4,13)}\label{ii.-einleitung-zur-uxf6ffentlichen-wirksamkeit-jesu-31-413}}

\hypertarget{auftreten-buuxdfpredigt-wirksamkeit-und-gefangennahme-johannes-des-tuxe4ufers}{%
\subsubsection{1. Auftreten, Bußpredigt, Wirksamkeit und Gefangennahme
Johannes des
Täufers}\label{auftreten-buuxdfpredigt-wirksamkeit-und-gefangennahme-johannes-des-tuxe4ufers}}

\hypertarget{section-2}{%
\section{3}\label{section-2}}

\bibleverse{1} Im fünfzehnten Regierungsjahre des Kaisers Tiberius, als
Pontius Pilatus Statthalter von Judäa war und Herodes
Vierfürst\textless sup title=``oder: Kleinfürst; vgl. Mt
14,1''\textgreater✲ von Galiläa, sein Bruder Philippus Vierfürst von
Ituräa und der Landschaft Trachonitis und Lysanias Vierfürst von
Abilene, \bibleverse{2} zur Zeit des Hohenpriesters Hannas und Kaiphas:
da erging das Wort Gottes an Johannes, den Sohn des Zacharias, in der
Wüste. \bibleverse{3} Er durchzog also die ganze Gegend am Jordan und
verkündigte eine Taufe der Buße\textless sup title=``oder: Bekehrung;
vgl. Mt 3,2''\textgreater✲ zur Vergebung der Sünden, \bibleverse{4} wie
im Buche der Aussprüche des Propheten Jesaja geschrieben
steht\textless sup title=``Jes 40,3-5''\textgreater✲: »Eine Stimme ruft
laut in der Wüste: ›Bereitet dem Herrn den Weg, ebnet ihm seine Pfade!
\bibleverse{5} Alle Vertiefungen\textless sup title=``oder:
Schluchten''\textgreater✲ sollen ausgefüllt und alle Berge und Hügel
geebnet werden! Was krumm ist, soll gerade und was uneben ist, soll zu
glattem Wege werden, \bibleverse{6} und die gesamte Menschheit soll das
Heil Gottes sehen!‹« \bibleverse{7} So sprach Johannes denn zu den
Volksscharen, die zu ihm hinauszogen, um sich von ihm taufen zu lassen:
»Ihr Schlangenbrut! Wer hat euch darauf gebracht, dem drohenden
Zorngericht entfliehen zu wollen? \bibleverse{8} So bringet denn
Früchte, die der Buße würdig sind✲, und laßt euch nicht in den Sinn
kommen, bei euch zu sagen\textless sup title=``oder: zu
denken''\textgreater✲: ›Wir haben ja doch Abraham zum Vater!‹, denn ich
sage euch: Gott vermag dem Abraham aus den Steinen hier Kinder zu
erwecken. \bibleverse{9} Schon ist aber auch die Axt den Bäumen an die
Wurzel gelegt, und jeder Baum, der nicht gute Früchte bringt, wird
abgehauen und ins Feuer geworfen.« \bibleverse{10} Da fragte ihn die
Volksmenge: »Was sollen wir denn tun?« \bibleverse{11} Er gab ihnen zur
Antwort: »Wer zwei Röcke\textless sup title=``oder:
Anzüge''\textgreater✲ hat, der gebe einen davon dem ab, der keinen hat,
und wer zu essen hat, mache es ebenso!« \bibleverse{12} Es kamen auch
Zöllner, um sich taufen zu lassen, und fragten ihn: »Meister, was sollen
wir tun?« \bibleverse{13} Er antwortete ihnen: »Fordert nicht mehr (Geld
von den Leuten), als euch vorgeschrieben ist!« \bibleverse{14} Es
fragten ihn auch Kriegsleute: »Was sollen wir tun?« Er antwortete ihnen:
»Tut niemand Gewald an, verübt keine Erpressungen und begnügt euch mit
eurer Löhnung!«

\bibleverse{15} Als nun das Volk in gespannter Erwartung war und alle
sich in ihren Herzen Gedanken über Johannes machten, ob er nicht
vielleicht selbst der Christus\textless sup title=``=~der Gesalbte, der
Messias''\textgreater✲ sei, \bibleverse{16} antwortete Johannes allen
mit den Worten: »Ich taufe euch (nur) mit Wasser; es kommt aber der,
welcher stärker ist als ich und für den ich nicht gut genug bin, ihm die
Riemen seiner Schuhe aufzubinden: der wird euch mit heiligem Geist und
mit Feuer taufen. \bibleverse{17} Er hat seine Worfschaufel in der Hand,
um seine Tenne gründlich zu reinigen, und er wird den Weizen in seine
Scheuer sammeln, die Spreu aber mit unauslöschlichem Feuer verbrennen.«
\bibleverse{18} Auch noch viele andere Ermahnungen richtete er an das
Volk und verkündigte ihm die Heilsbotschaft. \bibleverse{19} Der
Vierfürst Herodes✲ aber, dem er wegen Herodias, der Frau seines Bruders,
und wegen alles Bösen, das Herodes verübt hatte, Vorhaltungen gemacht
hatte, \bibleverse{20} fügte zu allen Übeltaten auch noch die hinzu, daß
er Johannes ins Gefängnis werfen ließ.

\hypertarget{taufe-und-messiasweihe-jesu}{%
\subsubsection{2. Taufe und Messiasweihe
Jesu}\label{taufe-und-messiasweihe-jesu}}

\bibleverse{21} Es begab sich aber, als das gesamte Volk sich taufen
ließ und auch Jesus getauft worden war und betete, daß der Himmel sich
auftat \bibleverse{22} und der heilige Geist in leiblicher Gestalt wie
eine Taube auf ihn herabschwebte und eine Stimme aus dem Himmel
erscholl: »Du bist mein geliebter Sohn, an dir habe ich Wohlgefallen
gefunden!«

\hypertarget{ahnentafel-oder-stammbaum-jesu}{%
\subsubsection{3. Ahnentafel (oder Stammbaum)
Jesu}\label{ahnentafel-oder-stammbaum-jesu}}

\bibleverse{23} Und er, Jesus, war bei seinem Auftreten etwa dreißig
Jahre alt und war, wie man meinte, der Sohn Josephs, \bibleverse{24} des
Sohnes des Eli, des Matthat, des Levi, des Melchi, des Jannai, des
Joseph, \bibleverse{25} des Mattathias, des Amos, des Nahum,
\bibleverse{26} des Esli, des Naggai, des Maath, des Mattathias, des
Semein, des Jose, des Joda, \bibleverse{27} des Johanan, des Resa, des
Serubbabel, des Salathiel, des Neri, \bibleverse{28} des Melchi, des
Addi, des Kosam, des Elmadam, des Er, \bibleverse{29} des Josua, des
Elieser, des Jorim, des Matthath, des Levi, \bibleverse{30} des Simeon,
des Juda, des Joseph, des Jona, des Eljakim, \bibleverse{31} des Melea,
des Menna, des Matthatha, des Nathan, des David, \bibleverse{32} des
Jesse\textless sup title=``oder: Isai''\textgreater✲, des Jobed, des
Boas, des Sala, des Nahason, \bibleverse{33} des Amminadab, des Admin,
des Arni, des Hezron, des Phares, des Juda, \bibleverse{34} des Jakob,
des Isaak, des Abraham, des Tharah, des Nahor, \bibleverse{35} des
Serug, des Regu, des Peleg, des Eber, des Selah, \bibleverse{36} des
Kainan, des Arphachsad, des Sem, des Noah, des Lamech, \bibleverse{37}
des Methusalah, des Henoch, des Jared, des Mahalaleel, des Kenan,
\bibleverse{38} des Enos, des Seth, des Adam, -- Gottes.

\hypertarget{die-versuchung-jesu-als-seine-messiasprobe}{%
\subsubsection{4. Die Versuchung Jesu als seine
Messiasprobe}\label{die-versuchung-jesu-als-seine-messiasprobe}}

\hypertarget{section-3}{%
\section{4}\label{section-3}}

\bibleverse{1} Jesus kehrte dann, voll heiligen Geistes, vom Jordan
zurück und wurde vom Geist vierzig Tage lang in der Wüste (umher)
geführt \bibleverse{2} und dabei vom Teufel versucht. Er aß in diesen
Tagen nichts, so daß ihn hungerte, als sie zu Ende waren. \bibleverse{3}
Da sagte der Teufel zu ihm: »Bist du Gottes Sohn, so gebiete diesem
Steine hier, er solle zu Brot werden!« \bibleverse{4} Doch Jesus
antwortete ihm: »Es steht geschrieben\textless sup title=``5.Mose
8,3''\textgreater✲: ›Nicht vom Brot allein wird\textless sup
title=``oder: soll''\textgreater✲ der Mensch leben!‹« \bibleverse{5}
Hierauf führte ihn der Teufel in die Höhe\textless sup title=``=~auf
einen hohen Berg''\textgreater✲, zeigte ihm in einem Augenblick alle
Reiche des Erdkreises \bibleverse{6} und sagte zu ihm: »Dir will ich
diese ganze Macht und ihre Herrlichkeit geben; denn mir ist sie
übergeben, und ich kann sie geben, wem ich will. \bibleverse{7} Wenn du
also vor mir (niederfällst und mich) anbetest, so soll sie ganz dir
gehören.« \bibleverse{8} Da gab ihm Jesus zur Antwort: »Es steht
geschrieben\textless sup title=``5.Mose 6,13-14''\textgreater✲: ›Du
sollst den Herrn, deinen Gott, anbeten und ihm allein dienen!‹«
\bibleverse{9} Hierauf führte der Teufel ihn nach Jerusalem, stellte ihn
auf die Zinne des Tempels und sagte zu ihm: »Bist du Gottes Sohn, so
stürze dich von hier hinab! \bibleverse{10} Denn es steht
geschrieben\textless sup title=``Ps 91,11-12''\textgreater✲: ›Er wird
seine Engel für dich entbieten, daß sie dich behüten, \bibleverse{11}
und sie werden dich auf den Armen tragen, damit du mit deinem Fuß an
keinen Stein stoßest.‹« \bibleverse{12} Da antwortete ihm Jesus: »Es ist
gesagt\textless sup title=``5.Mose 6,16''\textgreater✲: ›Du sollst den
Herrn, deinen Gott, nicht versuchen!« \bibleverse{13} Als der Teufel nun
mit allen Versuchungen zu Ende war, ließ er von ihm ab bis zu einer
gelegenen Zeit.

\hypertarget{iii.-jesu-wirken-in-galiluxe4a-414-950}{%
\subsection{III. Jesu Wirken in Galiläa
(4,14-9,50)}\label{iii.-jesu-wirken-in-galiluxe4a-414-950}}

\hypertarget{erstes-auftreten-jesu-in-galiluxe4a-seine-predigt-und-verwerfung-in-seiner-vaterstadt-nazareth}{%
\subsubsection{1. Erstes Auftreten Jesu in Galiläa; seine Predigt und
Verwerfung in seiner Vaterstadt
Nazareth}\label{erstes-auftreten-jesu-in-galiluxe4a-seine-predigt-und-verwerfung-in-seiner-vaterstadt-nazareth}}

\bibleverse{14} Jesus kehrte dann in der Kraft des Geistes nach Galiläa
zurück, und die Kunde von ihm verbreitete sich in der ganzen Umgegend.
\bibleverse{15} Er lehrte in ihren\textless sup title=``=~den
dortigen''\textgreater✲ Synagogen und wurde (wegen seiner Lehre) von
allen gepriesen.

\bibleverse{16} So kam er denn auch nach Nazareth, wo er aufgewachsen
war, ging dort nach seiner Gewohnheit am nächsten Sabbattage in die
Synagoge und stand auf, um vorzulesen. \bibleverse{17} Da reichte man
ihm das Buch des Propheten Jesaja; und als er das Buch aufrollte, traf
er auf die Stelle, wo geschrieben steht\textless sup title=``Jes 61,1-2;
58,6''\textgreater✲: \bibleverse{18} »Der Geist des Herrn ist über
mir\textless sup title=``oder: ruht auf mir''\textgreater✲, weil er mich
gesalbt✲ hat, damit ich den Armen die frohe Botschaft bringe; er hat
mich gesandt, um den Gefangenen die Freilassung und den Blinden die
Verleihung des Augenlichts zu verkünden, die Unterdrückten in Freiheit
zu entlassen, \bibleverse{19} ein Gnadenjahr des Herrn auszurufen.«
\bibleverse{20} Nachdem er dann das Buch wieder zusammengerollt und es
dem Diener zurückgegeben hatte, setzte er sich, und aller Augen in der
Synagoge waren gespannt auf ihn gerichtet. \bibleverse{21} Da begann er
seine Ansprache an sie mit den Worten: »Heute ist dieses Schriftwort,
das ihr soeben vernommen habt, zur Erfüllung gekommen!«

\bibleverse{22} Und alle stimmten ihm zu und staunten über die Worte der
Gnade\textless sup title=``oder: über die holdseligen
Worte''\textgreater✲, die aus seinem Munde kamen, und sagten: »Ist
dieser nicht der Sohn Josephs?« \bibleverse{23} Da antwortete er ihnen:
»Jedenfalls werdet ihr mir das Sprichwort vorhalten: ›Arzt, mache dich
selber gesund!‹ Alle die großen Taten, die (von dir), wie wir gehört
haben, in Kapernaum vollbracht worden sind, die vollführe auch hier in
deiner Vaterstadt!« \bibleverse{24} Er fuhr dann aber fort: »Wahrlich
ich sage euch: Kein Prophet ist in seiner Vaterstadt willkommen.
\bibleverse{25} In Wahrheit aber sage ich euch: Viele Witwen gab es in
Israel in den Tagen Elias, als der Himmel drei Jahre und sechs Monate
lang verschlossen blieb, so daß eine große Hungersnot über die ganze
Erde kam; \bibleverse{26} und doch wurde Elia zu keiner einzigen von
ihnen gesandt, sondern nur nach Sarepta im Gebiet von Sidon zu einer
Witwe\textless sup title=``1.Kön 17,1.9''\textgreater✲. \bibleverse{27}
Und viele Aussätzige gab es in Israel zur Zeit des Propheten Elisa, und
doch wurde kein einziger von ihnen gereinigt, sondern nur der Syrer
Naeman\textless sup title=``2.Kön 5,14''\textgreater✲.« \bibleverse{28}
Als sie das hörten, gerieten alle, die in der Synagoge anwesend waren,
in heftigen Zorn: \bibleverse{29} sie standen auf, stießen ihn aus der
Stadt hinaus und führten ihn an den Rand\textless sup title=``oder: auf
einen Vorsprung''\textgreater✲ des Berges, auf dem ihre Stadt erbaut
war, um ihn dort hinabzustürzen. \bibleverse{30} Er ging aber mitten
durch sie hindurch und wanderte weiter.

\hypertarget{jesu-wirken-in-kapernaum-und-in-der-umgegend}{%
\subsubsection{2. Jesu Wirken in Kapernaum und in der
Umgegend}\label{jesu-wirken-in-kapernaum-und-in-der-umgegend}}

\hypertarget{a-jesus-lehrt-in-der-synagoge-zu-kapernaum-und-heilt-einen-besessenen}{%
\paragraph{a) Jesus lehrt in der Synagoge zu Kapernaum und heilt einen
Besessenen}\label{a-jesus-lehrt-in-der-synagoge-zu-kapernaum-und-heilt-einen-besessenen}}

\bibleverse{31} Er begab sich dann nach der galiläischen Stadt Kapernaum
hinab und lehrte sie dort am Sabbat. \bibleverse{32} Da gerieten sie
über seine Lehre in Staunen, denn seine Rede beruhte auf (göttlicher)
Vollmacht. \bibleverse{33} Nun war da in der Synagoge ein Mann, der von
einem unreinen✲ Geiste besessen war; der schrie laut auf:
\bibleverse{34} »Ha! Was willst du von uns, Jesus von Nazareth? Du bist
gekommen, um uns zu vernichten! Ich weiß wohl, wer du bist: der Heilige
Gottes!« \bibleverse{35} Jesus bedrohte ihn mit den Worten: »Verstumme
und fahre von ihm aus!« Da warf der böse Geist den Mann mitten unter sie
zu Boden und fuhr von ihm aus, ohne ihm irgendwelchen Schaden zuzufügen.
\bibleverse{36} Da gerieten sie alle in Staunen; sie besprachen sich
miteinander und sagten: »Was ist das für ein Machtwort? Mit (göttlicher)
Vollmacht\textless sup title=``oder: Herrschergewalt''\textgreater✲ und
Kraft gebietet er den unreinen Geistern, und sie fahren aus!«
\bibleverse{37} Und die Kunde von ihm verbreitete sich in alle Orte der
Umgegend.

\hypertarget{b-heilung-der-schwiegermutter-des-simon-petrus-und-anderer-kranken-in-kapernaum}{%
\paragraph{b) Heilung der Schwiegermutter des Simon Petrus und anderer
Kranken in
Kapernaum}\label{b-heilung-der-schwiegermutter-des-simon-petrus-und-anderer-kranken-in-kapernaum}}

\bibleverse{38} Nachdem er dann die Synagoge verlassen hatte, begab er
sich in das Haus Simons. Dort war die Schwiegermutter Simons von hohem
Fieber befallen, und man wandte sich ihretwegen an ihn. \bibleverse{39}
Er trat also zu ihr, beugte sich über sie und bedrohte das Fieber: da
wich es von ihr; sie stand sogleich vom Lager auf und bediente sie (bei
der Mahlzeit).

\bibleverse{40} Als dann die Sonne unterging, brachten alle, welche
Kranke mit mancherlei Leiden hatten, sie zu ihm; er aber legte einem
jeden von ihnen die Hände auf und heilte sie. \bibleverse{41} Auch böse
Geister fuhren von vielen aus, wobei sie laut schrien und ausriefen: »Du
bist der Sohn Gottes!« Er bedrohte sie jedoch und ließ sie nicht zu
Worte kommen; denn sie wußten, daß er Christus\textless sup
title=``=~der Messias''\textgreater✲ war.

\hypertarget{c-jesu-wanderpredigt-in-der-umgegend-von-kapernaum}{%
\paragraph{c) Jesu Wanderpredigt in der Umgegend von
Kapernaum}\label{c-jesu-wanderpredigt-in-der-umgegend-von-kapernaum}}

\bibleverse{42} Bei Tagesanbruch aber entwich er von dort und begab sich
an einen einsamen Ort; doch die Volksmenge suchte nach ihm und kam zu
ihm hin und wollte ihn zurückhalten, damit er nicht von ihnen wegginge.
\bibleverse{43} Er aber sagte zu ihnen: »Ich muß auch den anderen
Städten die Heilsbotschaft vom Reiche Gottes verkünden, denn dazu bin
ich gesandt.« \bibleverse{44} So predigte er denn in den Synagogen des
jüdischen Landes.

\hypertarget{schiffspredigt-jesu-wunderbarer-fischzug-des-petrus-berufung-der-ersten-vier-juxfcnger}{%
\subsubsection{3. Schiffspredigt Jesu; wunderbarer Fischzug des Petrus;
Berufung der ersten vier
Jünger}\label{schiffspredigt-jesu-wunderbarer-fischzug-des-petrus-berufung-der-ersten-vier-juxfcnger}}

\hypertarget{section-4}{%
\section{5}\label{section-4}}

\bibleverse{1} Es begab sich aber (eines Tages), als das Volk ihn
umdrängte und das Wort Gottes hörte, während er selbst am See Gennesaret
stand, \bibleverse{2} da sah er zwei Boote am Ufer des Sees liegen; die
Fischer aber waren aus ihnen ausgestiegen und wuschen ihre Netze.
\bibleverse{3} Da trat er in eins der Boote, das Simon gehörte, und bat
ihn, ein wenig vom Lande abzustoßen; darauf setzte er sich nieder und
lehrte die Volksscharen vom Boote aus. \bibleverse{4} Als er dann seine
Ansprache beendet hatte, sagte er zu Simon: »Fahre auf die Höhe (des
Sees) hinaus und werft eure Netze aus, damit ihr einen Zug✲ tut!«
\bibleverse{5} Da antwortete Simon: »Meister, die ganze Nacht hindurch
haben wir gearbeitet und nichts gefangen; doch auf dein Wort hin will
ich die Netze auswerfen.« \bibleverse{6} Als sie das getan hatten,
fingen sie eine so große Menge Fische, daß ihre Netze zerreißen wollten.
\bibleverse{7} Da winkten sie ihren Genossen, die in dem andern Boot
waren, sie möchten kommen und ihnen helfen; die kamen auch, und man
füllte beide Boote, so daß sie tiefgingen. \bibleverse{8} Als Simon
Petrus das sah, warf er sich vor Jesus auf die Knie nieder und rief aus:
»Herr, gehe weg von mir, denn ich bin ein sündiger Mensch!«
\bibleverse{9} Denn Schrecken hatte ihn und alle, die bei ihm waren,
wegen dieses ihres Fischfangs befallen, \bibleverse{10} ebenso auch den
Jakobus und Johannes, die Söhne des Zebedäus, welche Simons
Genossen\textless sup title=``oder: Teilhaber''\textgreater✲ waren. Doch
Jesus sagte zu Simon: »Fürchte dich nicht! Von nun an wirst du ein
Menschenfischer sein.« \bibleverse{11} Sie brachten nun die Boote an
Land, verließen alles und folgten ihm nach.

\hypertarget{jesus-heilt-einen-aussuxe4tzigen-und-entweicht-in-die-einsamkeit}{%
\subsubsection{4. Jesus heilt einen Aussätzigen und entweicht in die
Einsamkeit}\label{jesus-heilt-einen-aussuxe4tzigen-und-entweicht-in-die-einsamkeit}}

\bibleverse{12} Es begab sich darauf, während er sich in einer der
Städte aufhielt, daß ein Mann da war, über und über mit Aussatz
behaftet. Als dieser Jesus sah, warf er sich vor ihm auf sein Angesicht
nieder und bat ihn mit den Worten: »Herr, wenn du willst, kannst du mich
reinigen!« \bibleverse{13} Jesus streckte die Hand aus, faßte ihn an und
sagte: »Ich will's: werde rein!« Da verschwand der Aussatz sogleich von
ihm. \bibleverse{14} Jesus gebot ihm dann, niemand etwas davon zu sagen,
und gab ihm die Weisung: »Gehe hin, zeige dich dem Priester und bringe
für deine Reinigung das Opfer dar, wie Mose es geboten hat\textless sup
title=``3.Mose 13,49; 14,10''\textgreater✲, zum Zeugnis✲ für sie!«
\bibleverse{15} Aber die Kunde über ihn breitete sich immer weiter aus,
und das Volk strömte in großen Scharen zusammen, um ihn zu hören und
sich von ihren Krankheiten heilen zu lassen. \bibleverse{16} Er jedoch
zog sich in die Einsamkeit zurück und betete dort.

\hypertarget{zusammenstuxf6uxdfe-mit-den-fuxfchrern-des-volkes-schriftgelehrten-und-pharisuxe4ern}{%
\subsubsection{5. Zusammenstöße mit den Führern des Volkes
(Schriftgelehrten und
Pharisäern)}\label{zusammenstuxf6uxdfe-mit-den-fuxfchrern-des-volkes-schriftgelehrten-und-pharisuxe4ern}}

\hypertarget{a-heilung-eines-geluxe4hmten-jesus-vergibt-suxfcnden}{%
\paragraph{a) Heilung eines Gelähmten; Jesus vergibt
Sünden}\label{a-heilung-eines-geluxe4hmten-jesus-vergibt-suxfcnden}}

\bibleverse{17} Eines Tages, als er der Lehrtätigkeit oblag, saßen auch
Pharisäer und Gesetzeslehrer da, die aus allen Ortschaften Galiläas und
Judäas und (besonders) aus Jerusalem gekommen waren; und die Kraft des
Herrn war durch ihn wirksam, so daß er Heilungen vollbrachte.
\bibleverse{18} Da brachten Männer auf einem Tragbett einen Mann, der
gelähmt war, und suchten ihn in das Haus hineinzubringen und vor Jesus
niederzusetzen. \bibleverse{19} Weil sie aber wegen der Volksmenge keine
Möglichkeit fanden, ihn hineinzubringen, stiegen sie auf das Dach und
ließen ihn samt dem Tragbett durch die Ziegel hindurch mitten unter die
Leute vor Jesus hinab. \bibleverse{20} Als dieser ihren Glauben sah,
sagte er: »Mensch, deine Sünden sind dir vergeben!« \bibleverse{21} Da
begannen die Schriftgelehrten und Pharisäer sich Gedanken darüber zu
machen und sagten: »Wer ist dieser? Er spricht ja Gotteslästerungen aus!
Wer kann Sünden vergeben außer Gott allein?« \bibleverse{22} Weil nun
Jesus ihre Gedanken durchschaute, redete er sie mit den Worten an: »Was
denkt ihr da in euren Herzen? \bibleverse{23} Was ist leichter, zu
sagen: ›Deine Sünden sind dir vergeben‹, oder zu sagen: ›Stehe auf und
gehe umher‹? \bibleverse{24} Damit ihr aber wißt\textless sup
title=``oder: erkennt''\textgreater✲, daß der Menschensohn Vollmacht
hat, auf der Erde Sünden zu vergeben« -- hierauf sagte er zu dem
Gelähmten: »Ich sage dir: Stehe auf, nimm dein Bett auf dich und gehe
heim in dein Haus!« \bibleverse{25} Da stand er augenblicklich vor ihren
Augen auf, nahm das (Tragbett), auf dem er gelegen hatte, und ging Gott
preisend heim in sein Haus. \bibleverse{26} Da gerieten alle außer sich
vor Erstaunen; sie priesen Gott und sagten voller Furcht: »Unglaubliches
haben wir heute gesehen!«

\hypertarget{b-berufung-des-zuxf6llners-levi-matthuxe4us-jesus-als-tischgenosse-der-zuxf6llner-und-suxfcnder}{%
\paragraph{b) Berufung des Zöllners Levi (=~Matthäus); Jesus als
Tischgenosse der Zöllner und
Sünder}\label{b-berufung-des-zuxf6llners-levi-matthuxe4us-jesus-als-tischgenosse-der-zuxf6llner-und-suxfcnder}}

\bibleverse{27} Hierauf ging er (aus dem Hause) hinaus und sah einen
Zöllner namens Levi an der Zollstätte sitzen und sagte zu ihm: »Folge
mir nach!« \bibleverse{28} Da verließ jener alles, stand auf und folgte
ihm nach. \bibleverse{29} Und Levi richtete ihm zu Ehren ein großes
Gastmahl in seinem Hause zu, und eine große Schar von Zöllnern und
anderen Leuten waren da, die mit ihnen am Mahl teilnahmen.
\bibleverse{30} Da sagten die Pharisäer und die zu ihnen gehörenden
Schriftgelehrten unwillig zu seinen Jüngern: »Warum eßt und trinkt ihr
mit den Zöllnern und Sündern?« \bibleverse{31} Jesus antwortete ihnen
mit den Worten: »Die Gesunden haben keinen Arzt nötig, wohl aber die
Kranken; \bibleverse{32} ich bin nicht gekommen, Gerechte zu berufen zur
Buße\textless sup title=``oder: Bekehrung; vgl. Mt 3,2''\textgreater✲,
sondern Sünder.«

\hypertarget{c-die-fastenfrage-der-johannesjuxfcnger-und-pharisuxe4er-jesus-rechtfertigt-das-neue-in-seinem-verhalten}{%
\paragraph{c) Die Fastenfrage der Johannesjünger und Pharisäer; Jesus
rechtfertigt das Neue in seinem
Verhalten}\label{c-die-fastenfrage-der-johannesjuxfcnger-und-pharisuxe4er-jesus-rechtfertigt-das-neue-in-seinem-verhalten}}

\bibleverse{33} Sie aber sagten zu ihm: »Die Jünger des Johannes fasten
häufig und verrichten (dabei) Gebete, ebenso auch die (Schüler) der
Pharisäer, während die deinigen essen und trinken.« \bibleverse{34}
Jesus antwortete ihnen: »Könnt ihr etwa von den Hochzeitsgästen
verlangen, daß sie fasten, solange der Bräutigam noch bei ihnen weilt?
\bibleverse{35} Es werden aber Tage kommen, wo der Bräutigam ihnen
genommen ist: dann, an jenen Tagen, werden sie fasten.«

\bibleverse{36} Er legte ihnen aber auch noch ein Gleichnis vor:
»Niemand reißt\textless sup title=``oder: schneidet''\textgreater✲ doch
von einem neuen Kleid ein Stück Zeug ab und setzt es auf ein altes
Kleid, sonst würde er nur das neue (Kleid) zerreißen, und zu dem alten
Kleide würde das Stück Zeug von dem neuen (Kleide) doch nicht passen.
\bibleverse{37} Auch füllt niemand neuen✲ Wein in alte Schläuche; sonst
sprengt der junge Wein die Schläuche und läuft selbst aus, und auch die
Schläuche gehen verloren. \bibleverse{38} Nein, jungen Wein muß man in
neue Schläuche füllen. \bibleverse{39} Und niemand, der alten Wein
getrunken hat, mag jungen Wein trinken; denn er sagt: ›Der alte ist
bekömmlich\textless sup title=``oder: schmeckt gut''\textgreater✲.‹«

\hypertarget{d-das-uxe4hrenraufen-der-juxfcnger-am-sabbat-der-erste-streit-jesu-mit-den-pharisuxe4ern-uxfcber-die-sabbatheiligung}{%
\paragraph{d) Das Ährenraufen der Jünger am Sabbat; der erste Streit
Jesu mit den Pharisäern über die
Sabbatheiligung}\label{d-das-uxe4hrenraufen-der-juxfcnger-am-sabbat-der-erste-streit-jesu-mit-den-pharisuxe4ern-uxfcber-die-sabbatheiligung}}

\hypertarget{section-5}{%
\section{6}\label{section-5}}

\bibleverse{1} Es begab sich aber an einem Sabbat, daß er durch die
Kornfelder wanderte; dabei pflückten seine Jünger Ähren ab, zerrieben
sie in den Händen und aßen sie\textless sup title=``d.h. die
Körner''\textgreater✲. \bibleverse{2} Da sagten einige von den
Pharisäern zu ihnen: »Warum tut ihr da etwas, das man am Sabbat nicht
tun darf?« \bibleverse{3} Jesus antwortete ihnen: »Habt ihr auch davon
nichts gelesen\textless sup title=``1.Sam 21,2-7''\textgreater✲, was
David getan hat, als ihn samt seinen Begleitern hungerte? \bibleverse{4}
Wie er da ins Gotteshaus hineinging, dort die Schaubrote nahm und sie aß
und auch seinen Begleitern davon gab, obgleich doch niemand außer den
Priestern sie essen darf?« \bibleverse{5} Er schloß mit den Worten: »Der
Menschensohn ist Herr (auch) über den Sabbat.«

\hypertarget{e-heilung-des-mannes-mit-dem-geluxe4hmten-arm-am-sabbat-der-zweite-streit-uxfcber-die-sabbatheiligung}{%
\paragraph{e) Heilung des Mannes mit dem gelähmten Arm am Sabbat; der
zweite Streit über die
Sabbatheiligung}\label{e-heilung-des-mannes-mit-dem-geluxe4hmten-arm-am-sabbat-der-zweite-streit-uxfcber-die-sabbatheiligung}}

\bibleverse{6} An einem anderen Sabbat aber ging er in die Synagoge und
lehrte. Dort war ein Mann, dessen rechter Arm verdorrt\textless sup
title=``d.h. gelähmt''\textgreater✲ war. \bibleverse{7} Da lauerten die
Schriftgelehrten und Pharisäer ihm auf, ob er wohl am Sabbat heilen
würde, um dann einen Grund zu einer Anklage gegen ihn zu haben;
\bibleverse{8} er aber kannte ihre Gedanken wohl. Er sagte nun zu dem
Manne mit dem gelähmten Arm: »Stehe auf und tritt vor in die Mitte!«
Jener stand auf und trat hin. \bibleverse{9} Dann sagte Jesus zu ihnen:
»Ich frage euch: Darf man am Sabbat Gutes tun, oder soll man Böses tun?
Darf man ein Leben erhalten, oder soll man es zugrunde gehen lassen?«
\bibleverse{10} Nachdem er sie dann alle ringsum (zornig) angeblickt
hatte, sagte er zu ihm: »Strecke deinen Arm aus!« Jener tat es, und sein
Arm wurde wieder hergestellt. \bibleverse{11} Jene aber wurden ganz
sinnlos vor Wut und besprachen sich miteinander, was sie Jesus antun
könnten.

\hypertarget{berufung-und-namen-der-zwuxf6lf-apostel-zulauf-des-volkes-viele-heilungen}{%
\subsubsection{6. Berufung und Namen der zwölf Apostel; Zulauf des
Volkes; viele
Heilungen}\label{berufung-und-namen-der-zwuxf6lf-apostel-zulauf-des-volkes-viele-heilungen}}

\bibleverse{12} Es begab sich aber in diesen Tagen, daß er hinausging
auf den Berg, um zu beten, und er verbrachte dort die (ganze) Nacht im
Gebet zu Gott. \bibleverse{13} Als es dann Tag geworden war, rief er
seine Jünger✲ zu sich und wählte zwölf aus ihnen aus, die er auch
Apostel\textless sup title=``d.h. Sendboten''\textgreater✲ nannte:
\bibleverse{14} Simon, den er auch Petrus\textless sup title=``d.h. Fels
oder Felsenmann''\textgreater✲ nannte, und dessen Bruder Andreas; ferner
Jakobus und Johannes, Philippus und Bartholomäus, \bibleverse{15}
Matthäus und Thomas, Jakobus, den Sohn des Alphäus, und Simon mit dem
Beinamen ›der Eiferer‹, \bibleverse{16} Judas, den Sohn des Jakobus, und
Judas Iskarioth, der (später) zum Verräter an ihm wurde.

\bibleverse{17} Als er dann mit ihnen (vom Berge) wieder hinabgestiegen
war, blieb er auf einem ebenen Platz stehen samt einer großen Schar
seiner Jünger✲ und einer zahlreichen Volksmenge aus dem ganzen jüdischen
Lande, besonders aus Jerusalem, auch aus dem Küstenlande von Tyrus und
Sidon. \bibleverse{18} Alle diese waren gekommen, um ihn zu hören und
sich von ihren Krankheiten heilen zu lassen; auch die von unreinen
Geistern Geplagten fanden Heilung; \bibleverse{19} und die ganze
Volksmenge suchte ihn anzurühren, denn eine Kraft ging von ihm aus und
heilte alle.

\hypertarget{die-bergpredigt-bzw.-feldpredigt}{%
\subsubsection{7. Die Bergpredigt (bzw.
Feldpredigt?)}\label{die-bergpredigt-bzw.-feldpredigt}}

\hypertarget{a-die-vier-seligpreisungen-der-irdisch-und-geistlich-armen-und-die-vier-weherufe-uxfcber-die-irdisch-und-geistlich-reichen}{%
\paragraph{a) Die vier Seligpreisungen der irdisch (und geistlich) Armen
und die vier Weherufe über die irdisch (und geistlich)
Reichen}\label{a-die-vier-seligpreisungen-der-irdisch-und-geistlich-armen-und-die-vier-weherufe-uxfcber-die-irdisch-und-geistlich-reichen}}

\bibleverse{20} Da richtete er seine Augen auf seine Jünger und sagte:
»Selig seid ihr Armen, denn euer Teil ist das Reich Gottes!
\bibleverse{21} Selig seid ihr, die ihr jetzt hungert, denn ihr werdet
gesättigt werden! Selig seid ihr, die ihr jetzt weint, denn ihr werdet
lachen! \bibleverse{22} Selig seid ihr, wenn die Menschen euch hassen
und wenn sie euch aus ihrer Gemeinschaft ausschließen und euch schmähen
und euren Namen als ein Schimpfwort verwerfen um des Menschensohnes
willen! \bibleverse{23} Freuet euch alsdann und jubelt! Denn wisset
wohl: euer Lohn ist groß im Himmel. Ihre Väter haben ja an den Propheten
ebenso gehandelt. \bibleverse{24} Doch wehe euch Reichen, denn ihr habt
euren Trost dahin\textless sup title=``=~bereits
empfangen''\textgreater✲! \bibleverse{25} Wehe euch, die ihr jetzt satt
seid, denn ihr werdet Hunger leiden! Wehe euch, die ihr jetzt lacht,
denn ihr werdet trauern und weinen! \bibleverse{26} Wehe euch, wenn alle
Welt mit Lobesworten von euch redet! Ihre Väter haben ja an den falschen
Propheten ebenso gehandelt.«

\hypertarget{b-gebot-der-feindesliebe-verzicht-auf-wiedervergeltung}{%
\paragraph{b) Gebot der Feindesliebe; Verzicht auf
Wiedervergeltung}\label{b-gebot-der-feindesliebe-verzicht-auf-wiedervergeltung}}

\bibleverse{27} »Euch aber, meinen Hörern, sage ich: Liebet eure Feinde,
tut denen Gutes, die euch hassen, \bibleverse{28} segnet die, welche
euch fluchen, betet für die, welche euch anfeinden\textless sup
title=``oder: kränken''\textgreater✲! \bibleverse{29} Wer dich auf die
Wange schlägt, dem halte auch die andere hin, und wer dir den Mantel
wegnimmt, dem verweigere auch den Rock nicht! \bibleverse{30} Jedem, der
dich (um etwas) bittet, dem gib, und wer dir das Deine nimmt, von dem
fordere es nicht zurück! \bibleverse{31} Und wie ihr von den Leuten
behandelt werden wollt, ebenso behandelt auch ihr sie! \bibleverse{32}
Denn wenn ihr (nur) die liebt, die euch lieben: welchen (Anspruch auf)
Dank habt ihr dann? Auch die Sünder lieben ja die, welche ihnen Liebe
erweisen. \bibleverse{33} Und wenn ihr (nur) denen Gutes erweist, die
euch Gutes tun: welchen (Anspruch auf) Dank habt ihr dann? Auch die
Sünder tun dasselbe. \bibleverse{34} Und wenn ihr denen leiht, von denen
ihr (das Geliehene) zurückzuerhalten hofft: welchen (Anspruch auf) Dank
habt ihr dann? Auch die Sünder leihen den Sündern, um ebensoviel
zurückzuerhalten. \bibleverse{35} Nein, liebet eure Feinde, tut Gutes
und leihet aus, ohne etwas zurückzuerwarten! Dann wird euer Lohn groß
sein, und ihr werdet Söhne des Höchsten sein; denn er ist gütig (auch)
gegen die Undankbaren und Bösen. \bibleverse{36} Seid barmherzig, wie
euer Vater barmherzig ist!«

\hypertarget{c-warnung-vor-dem-richten-und-heuchlerischen-bessernwollen}{%
\paragraph{c) Warnung vor dem Richten und heuchlerischen
Bessernwollen}\label{c-warnung-vor-dem-richten-und-heuchlerischen-bessernwollen}}

\bibleverse{37} »Und richtet nicht, dann werdet ihr auch nicht gerichtet
werden; und verurteilt nicht, dann werdet ihr auch nicht verurteilt
werden; laßt (eure Schuldner) frei, dann werdet ihr auch freigelassen
werden. \bibleverse{38} Gebt, dann wird auch euch gegeben werden: ein
reichliches, festgedrücktes, gerütteltes und übervolles Maß wird man
euch in den Schoß schütten; denn mit demselben Maß, mit dem ihr zumeßt,
wird euch wieder zugemessen werden.«~-- \bibleverse{39} Er legte ihnen
dann auch ein Gleichnis vor: »Kann wohl ein Blinder einen Blinden
führen? Werden sie nicht beide in die Grube fallen?\textless sup
title=``Mt 15,14''\textgreater✲ \bibleverse{40} Der Jünger\textless sup
title=``oder: Schüler''\textgreater✲ steht nicht über seinem
Meister\textless sup title=``oder: Lehrer''\textgreater✲: jeder (Jünger)
wird, wenn er völlig ausgebildet ist, immer nur wie sein Meister
sein\textless sup title=``Mt 10,24-25''\textgreater✲. \bibleverse{41}
Was siehst du aber den Splitter im Auge\textless sup title=``vgl. Mt
7,5''\textgreater✲ deines Bruders, während du den Balken in deinem
eignen Auge nicht wahrnimmst? \bibleverse{42} Oder wie darfst du zu
deinem Bruder sagen: ›Bruder, laß mich den Splitter, der in deinem Auge
steckt, herausziehen‹, während du den Balken in deinem eignen Auge nicht
gewahrst? Du Heuchler! Ziehe zuerst den Balken aus deinem Auge, dann
magst du zusehen, daß du den Splitter herausziehst, der im Auge deines
Bruders steckt.«

\hypertarget{d-der-glaubensgehorsam-sowie-der-unglaube-der-menschen-kommt-aus-dem-herzen-wie-die-fruxfcchte-aus-der-art-des-baumes}{%
\paragraph{d) Der Glaubensgehorsam sowie der Unglaube der Menschen kommt
aus dem Herzen, wie die Früchte aus der Art des
Baumes}\label{d-der-glaubensgehorsam-sowie-der-unglaube-der-menschen-kommt-aus-dem-herzen-wie-die-fruxfcchte-aus-der-art-des-baumes}}

\bibleverse{43} »Denn es gibt keinen guten Baum, der schlechte Früchte
bringt, und umgekehrt keinen schlechten Baum, der gute Früchte bringt.
\bibleverse{44} Jeden Baum erkennt man ja an seinen Früchten; denn von
Dornen sammelt man keine Feigen, und von einem Dornbusch kann man keine
Trauben lesen. \bibleverse{45} Ein guter Mensch bringt aus der guten
Schatzkammer seines Herzens das Gute hervor, während ein böser Mensch
aus der bösen (Schatzkammer seines Herzens) das Böse hervorbringt; denn
wovon das Herz voll ist, davon redet sein Mund.«

\hypertarget{e-der-gehorsam-oder-der-ungehorsam-gegen-jesu-wort-bewuxe4hrt-oder-ruxe4cht-sich-wie-ein-hausbau-mit-festem-grund-oder-ohne-solchen}{%
\paragraph{e) Der Gehorsam oder der Ungehorsam gegen Jesu Wort bewährt
oder rächt sich wie ein Hausbau mit festem Grund oder ohne
solchen}\label{e-der-gehorsam-oder-der-ungehorsam-gegen-jesu-wort-bewuxe4hrt-oder-ruxe4cht-sich-wie-ein-hausbau-mit-festem-grund-oder-ohne-solchen}}

\bibleverse{46} »Was nennt ihr mich aber ›Herr, Herr!‹ und tut doch
nicht, was ich (euch) sage? \bibleverse{47} Wer zu mir kommt und meine
Worte hört und nach ihnen tut -- ich will euch zeigen, wem der zu
vergleichen ist: \bibleverse{48} Er gleicht einem Manne, der, als er ein
Haus bauen wollte, bis in die Tiefe ausgraben ließ und die Grundmauer
auf den Felsen legte. Als nun Hochwasser kam, stieß die Flut an jenes
Haus, vermochte es aber wegen seiner festen Bauart nicht zu erschüttern.
\bibleverse{49} Wer aber (meine Worte) hört und nicht nach ihnen tut,
der gleicht einem Manne, der ein Haus ohne feste Grundmauer auf den
(lockeren) Erdboden baute. Als dann die Flut dagegen stieß, stürzte es
sogleich in sich zusammen, und der Einsturz\textless sup title=``oder:
Trümmerhaufen''\textgreater✲ dieses Hauses war gewaltig.«

\hypertarget{heilung-des-knechts-des-heidnischen-hauptmanns-von-kapernaum}{%
\subsubsection{8. Heilung des Knechts des (heidnischen) Hauptmanns von
Kapernaum}\label{heilung-des-knechts-des-heidnischen-hauptmanns-von-kapernaum}}

\hypertarget{section-6}{%
\section{7}\label{section-6}}

\bibleverse{1} Nachdem Jesus alle seine Reden an das Volk, das ihm
zuhörte, beendet hatte, ging er nach Kapernaum hinein. \bibleverse{2}
Dort lag der Diener✲ eines Hauptmanns, der diesem besonders wert war,
todkrank darnieder. \bibleverse{3} Weil nun der Hauptmann von Jesus
gehört hatte, sandte er Älteste der Juden zu ihm mit der Bitte, er
möchte kommen und seinen Diener gesund machen. \bibleverse{4} Als diese
zu Jesus kamen, baten sie ihn inständig mit den Worten: »Er verdient es,
daß du ihm diese Bitte erfüllst; \bibleverse{5} denn er hat unser Volk
lieb, und er ist es, der uns unsere Synagoge gebaut hat.« \bibleverse{6}
Da machte sich Jesus mit ihnen auf den Weg. Als er aber nicht mehr weit
von dem Hause entfernt war, sandte der Hauptmann Freunde ab und ließ ihm
sagen: »Herr, bemühe dich nicht, denn ich bin nicht wert, daß du unter
mein Dach trittst. \bibleverse{7} Darum habe ich mich auch nicht für
würdig gehalten, selbst zu dir zu kommen; sprich vielmehr nur ein Wort,
so muß mein Diener gesund werden. \bibleverse{8} Denn auch ich bin ein
Mensch, der unter Vorgesetzten steht, und habe Mannschaften unter mir;
und wenn ich zu einem sage: ›Geh!‹, so geht er, und zu einem anderen:
›Komm!‹, so kommt er, und zu meinem Diener: ›Tu das!‹, so tut er's.«
\bibleverse{9} Als Jesus das hörte, wunderte er sich über ihn und sagte,
zu der ihn begleitenden Volksmenge gewandt: »Ich sage euch: Selbst in
Israel habe ich solchen Glauben nicht gefunden!« \bibleverse{10} Als
dann die Abgesandten in das Haus (des Hauptmanns) zurückkehrten, fanden
sie den Diener von seiner Krankheit genesen.

\hypertarget{auferweckung-des-juxfcnglings-zu-nain}{%
\subsubsection{9. Auferweckung des Jünglings zu
Nain}\label{auferweckung-des-juxfcnglings-zu-nain}}

\bibleverse{11} Kurze Zeit darauf begab es sich, daß Jesus nach einer
Stadt namens Nain wanderte, und mit ihm zogen seine Jünger und eine
große Volksschar. \bibleverse{12} Als er sich nun dem Stadttor näherte,
da trug man gerade einen Toten heraus, den einzigen Sohn seiner Mutter,
und die war eine Witwe; und eine große Volksmenge aus der Stadt gab ihr
das Geleit. \bibleverse{13} Als der Herr sie sah, ging ihr Unglück ihm
zu Herzen, und er sagte zu ihr: »Weine nicht!« \bibleverse{14} Dann trat
er hinzu und faßte die Bahre an; da standen die Träger still, und er
sprach: »Jüngling, ich sage dir: stehe auf!« \bibleverse{15} Da setzte
der Tote sich aufrecht hin und fing an zu reden; und Jesus gab ihn
seiner Mutter wieder\textless sup title=``1.Kön 17,23''\textgreater✲.
\bibleverse{16} Da kam Furcht über alle, und sie priesen Gott und
sagten: »Ein großer Prophet ist unter uns erstanden!« und: »Gott hat
sein Volk gnädig angesehen!« \bibleverse{17} Die Kunde von dieser seiner
Tat aber verbreitete sich im ganzen jüdischen Lande und in allen
umliegenden Gegenden.

\hypertarget{gesandtschaft-johannes-des-tuxe4ufers-jesu-antwort-und-sein-zeugnis-uxfcber-johannes}{%
\subsubsection{10. Gesandtschaft Johannes des Täufers; Jesu Antwort und
sein Zeugnis über
Johannes}\label{gesandtschaft-johannes-des-tuxe4ufers-jesu-antwort-und-sein-zeugnis-uxfcber-johannes}}

\bibleverse{18} Auch dem Johannes erstatteten seine Jünger Bericht über
dies alles. Da rief Johannes zwei von seinen Jüngern zu sich,
\bibleverse{19} sandte sie zum Herrn und ließ ihn fragen: »Bist du es,
der da kommen soll, oder sollen wir auf einen andern warten?«
\bibleverse{20} Als nun die Männer bei Jesus eintrafen, sagten sie:
»Johannes der Täufer hat uns zu dir gesandt und läßt dich fragen: ›Bist
du es, der da kommen soll, oder sollen wir auf einen andern warten?‹«
\bibleverse{21} Jesus heilte in eben jener Stunde viele von Krankheiten,
von schmerzhaften Leiden und bösen Geistern und schenkte vielen Blinden
das Augenlicht. \bibleverse{22} So gab er ihnen denn zur Antwort: »Geht
hin und berichtet dem Johannes, was ihr (hier) gesehen und gehört habt:
Blinde werden sehend, Lahme gehen, Aussätzige werden rein, Taube hören,
Tote werden auferweckt, Armen wird die Heilsbotschaft
verkündigt\textless sup title=``Jes 35,5; 61,1''\textgreater✲,
\bibleverse{23} und selig ist, wer an mir nicht irre wird.«
\bibleverse{24} Als nun die Boten des Johannes wieder weggegangen waren,
begann Jesus zu der Volksmenge über Johannes zu reden: »Was wolltet ihr
sehen, als ihr (jüngst) in die Wüste hinauszogt? Etwa ein Schilfrohr,
das vom Winde hin und her bewegt wird? \bibleverse{25} Nein; aber wozu
seid ihr hinausgezogen? Wolltet ihr einen Mann in weichen Gewändern
sehen? Nein; die Leute, welche prächtige Kleidung tragen und in
Üppigkeit leben, sind in den Königspalästen zu finden. \bibleverse{26}
Aber wozu seid ihr hinausgezogen? Wolltet ihr einen Propheten sehen? Ja,
ich sage euch: Mehr noch als einen Propheten! \bibleverse{27} Dieser ist
es, über den geschrieben steht\textless sup title=``Mal
3,1''\textgreater✲: ›Siehe, ich sende meinen Boten vor dir her, der
deinen Weg vor dir her bereiten soll.‹ \bibleverse{28} Ja, ich sage
euch: Unter den von Frauen Geborenen gibt es keinen größeren (Propheten)
als Johannes; aber der Kleinste im Reiche Gottes ist größer als er.
\bibleverse{29} Und das gesamte Volk, das ihn hörte, auch die Zöllner
sind dem Willen Gottes nachgekommen, indem sie sich mit der Taufe des
Johannes taufen ließen; \bibleverse{30} aber die Pharisäer und die
Gesetzeslehrer haben den Heilsratschluß Gottes für ihre Person
verworfen, indem sie sich von ihm nicht taufen ließen.

\bibleverse{31} Wem soll ich nun\textless sup title=``oder:
demnach''\textgreater✲ die Menschen des gegenwärtigen Zeitalters
vergleichen? Wem sind sie gleich? \bibleverse{32} Sie sind wie Kinder,
die auf einem öffentlichen Platze sitzen und einander zurufen: ›Wir
haben euch gepfiffen, doch ihr habt nicht getanzt! Wir haben Klagelieder
angestimmt, doch ihr habt nicht geweint!‹ \bibleverse{33} Denn Johannes
der Täufer ist gekommen, der kein Brot aß und keinen Wein trank; da sagt
ihr: ›Er ist von Sinnen!‹ \bibleverse{34} Nun ist der Menschensohn
gekommen, welcher ißt und trinkt; da sagt ihr: ›Seht, ein Fresser und
Weintrinker, ein Freund von Zöllnern und Sündern!‹ \bibleverse{35} Und
doch ist die (göttliche) Weisheit gerechtfertigt worden von\textless sup
title=``oder: an''\textgreater✲ allen ihren Kindern.«

\hypertarget{jesu-salbung-durch-die-grouxdfe-suxfcnderin}{%
\subsubsection{11. Jesu Salbung durch die große
Sünderin}\label{jesu-salbung-durch-die-grouxdfe-suxfcnderin}}

\bibleverse{36} Es lud ihn aber einer von den Pharisäern ein, bei ihm zu
speisen; er ging denn auch in die Wohnung des Pharisäers und nahm bei
Tische Platz. \bibleverse{37} Und siehe, eine Frau, die in der Stadt als
Sünderin lebte und erfahren hatte, daß Jesus im Hause des Pharisäers zu
Gaste sei, brachte ein Alabasterfläschchen mit Myrrhenöl \bibleverse{38}
und begann, indem sie von hinten an seine Füße herantrat und weinte,
seine Füße mit ihren Tränen zu benetzen und sie mit ihrem Haupthaar zu
trocknen; dann küßte sie seine Füße und salbte sie mit dem Myrrhenöl.
\bibleverse{39} Als nun der Pharisäer, der ihn eingeladen hatte, das
sah, dachte er bei sich: »Wenn dieser wirklich ein Prophet wäre, so
müßte er wissen, wer und was für eine Frau das ist, die ihn da berührt,
daß sie nämlich eine Sünderin ist.« \bibleverse{40} Da nahm Jesus das
Wort und sagte zu ihm: »Simon, ich habe dir etwas zu sagen.« Jener
erwiderte: »Meister, sprich!« \bibleverse{41} »Ein Geldverleiher hatte
zwei Schuldner; der eine war ihm fünfhundert Denare✲ schuldig, der
andere fünfzig; \bibleverse{42} weil sie aber nicht zurückzahlen
konnten, schenkte er beiden die Schuld. Wer von ihnen wird ihn nun am
meisten lieben?« \bibleverse{43} Simon antwortete: »Ich denke der, dem
er das meiste geschenkt hat.« Jesus erwiderte ihm: »Du hast richtig
geurteilt.« \bibleverse{44} Indem er sich dann zu der Frau wandte, sagte
er zu Simon: »Siehst du diese Frau hier? Ich bin in dein Haus gekommen:
du hast mir kein Wasser für die Füße gegeben, sie aber hat mir die Füße
mit ihren Tränen genetzt und sie mit ihrem Haar getrocknet.
\bibleverse{45} Du hast mir keinen Kuß gegeben, sie aber hat, seitdem
ich eingetreten bin, mir die Füße unaufhörlich geküßt. \bibleverse{46}
Du hast mir das Haupt nicht mit Öl gesalbt, sie aber hat mir mit
Myrrhenöl die Füße gesalbt. \bibleverse{47} Deshalb sage ich dir: Ihre
vielen Sünden sind ihr vergeben, denn sie hat viel Liebe erwiesen; wem
aber nur wenig vergeben wird, der erweist auch nur wenig Liebe.«
\bibleverse{48} Dann sagte er zu ihr: »Deine Sünden sind (dir)
vergeben!« \bibleverse{49} Da begannen die Tischgenossen bei sich zu
denken\textless sup title=``oder: untereinander zu sagen''\textgreater✲:
»Wer ist dieser, daß er sogar Sünden vergibt?« \bibleverse{50} Er aber
sagte zu der Frau: »Dein Glaube hat dich gerettet: gehe hin in Frieden!«

\hypertarget{die-bestuxe4ndigen-begleiter-jesu-bei-seinen-wanderungen-die-dienenden-galiluxe4ischen-frauen}{%
\subsubsection{12. Die beständigen Begleiter Jesu bei seinen
Wanderungen; die dienenden galiläischen
Frauen}\label{die-bestuxe4ndigen-begleiter-jesu-bei-seinen-wanderungen-die-dienenden-galiluxe4ischen-frauen}}

\hypertarget{section-7}{%
\section{8}\label{section-7}}

\bibleverse{1} In der folgenden Zeit durchwanderte er dann das Land von
Stadt zu Stadt und von Dorf zu Dorf, indem er öffentlich lehrte und die
Heilsbotschaft vom Reiche Gottes verkündigte. In seiner Begleitung
befanden sich die zwölf Jünger \bibleverse{2} sowie auch einige Frauen,
die er von bösen Geistern und Krankheiten geheilt hatte, z.B. Maria, die
Magdalena genannt wurde, aus der sieben böse Geister ausgefahren waren,
\bibleverse{3} ferner Johanna, die Frau des Chuza, eines Verwalters des
Herodes, und Susanna und noch viele andere, die ihnen mit den ihnen zu
Gebote stehenden Mitteln Dienste leisteten.

\hypertarget{gleichnis-vom-suxe4mann-und-dem-vierfachen-ackerfeld}{%
\subsubsection{13. Gleichnis vom Sämann und dem vierfachen
Ackerfeld}\label{gleichnis-vom-suxe4mann-und-dem-vierfachen-ackerfeld}}

\bibleverse{4} Als nun eine große Volksmenge zusammenkam und die Leute
aus allen Städten ihm zuströmten, sprach er in der Form eines
Gleichnisses: \bibleverse{5} »Der Sämann ging aus, um seinen Samen zu
säen; und beim Säen fiel einiges (von dem Saatkorn) auf den Weg
längshin\textless sup title=``oder: daneben''\textgreater✲ und wurde
zertreten, und die Vögel des Himmels fraßen es auf. \bibleverse{6}
Anderes fiel auf felsigen Boden, und als es aufgegangen war, verdorrte
es, weil ihm die Feuchtigkeit fehlte. \bibleverse{7} Wieder anderes fiel
mitten unter die Dornen, und die Dornen wuchsen mit auf und erstickten
es. \bibleverse{8} Anderes aber fiel auf den guten Boden, wuchs auf und
brachte hundertfältigen Ertrag.« Bei diesen Worten rief er laut aus:
»Wer Ohren hat zu hören, der höre!«

\hypertarget{sinn-und-zweck-der-gleichnisse-deutung-des-gleichnisses-vom-suxe4mann}{%
\paragraph{Sinn und Zweck der Gleichnisse; Deutung des Gleichnisses vom
Sämann}\label{sinn-und-zweck-der-gleichnisse-deutung-des-gleichnisses-vom-suxe4mann}}

\bibleverse{9} Da fragten ihn seine Jünger nach dem Sinn dieses
Gleichnisses; \bibleverse{10} und er antwortete: »Euch ist es gegeben,
die Geheimnisse des Reiches Gottes zu erkennen, den anderen aber (werden
sie) nur in Gleichnissen (vorgetragen), damit ›sie mit sehenden Augen
doch nicht sehen und mit hörenden Ohren doch nicht
verstehen‹\textless sup title=``Jes 6,9-10''\textgreater✲.
\bibleverse{11} Dies ist aber die Deutung des Gleichnisses: Der
Same\textless sup title=``=~das Saatkorn''\textgreater✲ ist das Wort
Gottes. \bibleverse{12} Die, bei denen der Same auf den Weg
längshin\textless sup title=``oder: daneben''\textgreater✲ fiel, sind
solche, die (das Wort wohl) gehört haben, darauf aber kommt der Teufel
und nimmt das Wort aus ihrem Herzen weg, damit sie nicht zum Glauben
gelangen und dadurch gerettet werden. \bibleverse{13} Die, bei denen der
Same auf den felsigen Boden fiel, sind solche, die das Wort, wenn sie es
gehört haben, mit Freuden annehmen; doch es kann nicht Wurzel bei ihnen
schlagen: eine Zeitlang glauben sie wohl, aber zur Zeit der Versuchung
fallen sie ab. \bibleverse{14} Was dann unter die Dornen fiel, das
deutet auf solche, die das Wort gehört haben, dann aber hingehen und es
von den Sorgen und dem Reichtum und den Freuden des Lebens ersticken
lassen, so daß sie die Frucht nicht zur Reife bringen. \bibleverse{15}
Was aber auf den guten Boden fiel, das deutet auf solche, die das Wort,
welches sie gehört haben, in einem feinen und guten Herzen festhalten
und mit Beharrlichkeit Frucht bringen.«

\hypertarget{einzelspruxfcche-uxfcber-die-juxfcngerpflicht-bezuxfcglich-der-kuxfcnftigen-klaren-und-reichlichen-verkuxfcndigung-der-heilsbotschaft}{%
\subsubsection{14. Einzelsprüche über die Jüngerpflicht (bezüglich der
künftigen klaren und reichlichen Verkündigung der
Heilsbotschaft)}\label{einzelspruxfcche-uxfcber-die-juxfcngerpflicht-bezuxfcglich-der-kuxfcnftigen-klaren-und-reichlichen-verkuxfcndigung-der-heilsbotschaft}}

\bibleverse{16} »Niemand aber, der ein Licht angezündet hat, deckt es
mit einem Gefäß zu oder stellt es unter ein Bett, sondern er stellt es
auf einen Leuchter✲, damit die Eintretenden den hellen Schein
sehen\textless sup title=``Mt 5,15''\textgreater✲. \bibleverse{17} Denn
nichts ist verborgen, was nicht offenbar werden wird, und nichts ist
geheim, was nicht bekannt werden und ans Tageslicht kommen
wird\textless sup title=``Mt 10,26; Lk 12,2''\textgreater✲.
\bibleverse{18} Darum gebt wohl acht, wie ihr hört\textless sup
title=``=~daß ihr recht hört''\textgreater✲! Denn wer da hat, dem wird
noch dazugegeben werden, und wer nicht hat, dem wird auch das noch
genommen werden, was er zu haben meint.«\textless sup title=``Mt 13,12;
25,29''\textgreater✲

\hypertarget{die-wahren-verwandten-jesu}{%
\subsubsection{15. Die wahren Verwandten
Jesu}\label{die-wahren-verwandten-jesu}}

\bibleverse{19} Es stellten sich dann seine Mutter und seine Brüder bei
ihm ein, konnten jedoch wegen der Volksmenge nicht zu ihm gelangen.
\bibleverse{20} Da wurde ihm gemeldet: »Deine Mutter und deine Brüder
stehen draußen und wünschen dich zu sehen.« \bibleverse{21} Er aber
antwortete ihnen mit den Worten: »Meine Mutter und meine Brüder sind
diese da, die das Wort Gottes hören und (danach) tun.«

\hypertarget{jesu-macht-uxfcber-sturm-buxf6se-geister-und-tod}{%
\subsubsection{16. Jesu Macht über Sturm, böse Geister und
Tod}\label{jesu-macht-uxfcber-sturm-buxf6se-geister-und-tod}}

\hypertarget{a-jesus-beschwichtigt-den-seesturm}{%
\paragraph{a) Jesus beschwichtigt den
Seesturm}\label{a-jesus-beschwichtigt-den-seesturm}}

\bibleverse{22} Eines Tages begab es sich, daß er mit seinen Jüngern in
ein Boot stieg und zu ihnen sagte: »Wir wollen an die andere Seite des
Sees hinüberfahren!« So stießen sie denn vom Lande ab. \bibleverse{23}
Während der Fahrt aber schlief er ein. Da fuhr ein Sturmwind auf den See
herab, das Boot füllte sich mit Wasser, und sie gerieten in
Lebensgefahr. \bibleverse{24} Da traten sie zu ihm und weckten ihn mit
den Worten: »Meister, Meister, wir gehen unter!« Er aber stand auf und
bedrohte den Wind und das Gewoge des Wassers: da legten sie sich, und es
trat Windstille ein. \bibleverse{25} Er sagte dann zu ihnen: »Wo ist nun
euer Glaube (geblieben)?« Sie waren aber in Furcht und Staunen geraten
und sagten zueinander: »Wer ist denn dieser, daß er sogar den Winden und
dem Wasser gebietet und sie ihm gehorsam sind?«

\hypertarget{b-jesus-heilt-den-besessenen-im-lande-der-gergesener}{%
\paragraph{b) Jesus heilt den Besessenen im Lande der
Gergesener}\label{b-jesus-heilt-den-besessenen-im-lande-der-gergesener}}

\bibleverse{26} Sie fuhren dann nach dem Lande der
Gergesener\textless sup title=``vgl. Mt 8,28''\textgreater✲, das Galiläa
gegenüber liegt. \bibleverse{27} Als er dort ans Land gestiegen war, kam
ihm ein Mann aus der Stadt entgegen, der von bösen Geistern besessen
war; schon seit langer Zeit hatte er keine Kleider mehr angezogen, auch
hielt er sich in keinem Hause mehr auf, sondern in den Gräbern.
\bibleverse{28} Als er Jesus sah, schrie er auf, warf sich vor ihm
nieder und rief laut: »Was willst du von mir, Jesus, du Sohn Gottes, des
Höchsten? Ich bitte dich! Quäle mich nicht\textless sup title=``=~laß
mich in Ruhe''\textgreater✲!« \bibleverse{29} Jesus war nämlich im
Begriff, dem unreinen Geist zu gebieten, aus dem Manne auszufahren; denn
dieser hatte ihn schon seit langer Zeit in seiner Gewalt, und man hatte
ihn mit Ketten und Fußfesseln gebunden und in Gewahrsam gehalten; doch
er hatte die Bande allemal zerrissen und wurde von dem bösen Geiste in
die Einöden getrieben. \bibleverse{30} Jesus fragte ihn nun: »Wie heißt
du?« Er antwortete: »Legion«\textless sup title=``d.h.
Heerschar''\textgreater✲; denn viele böse Geister waren in ihn gefahren.
\bibleverse{31} Diese baten ihn nun, er möchte ihnen nicht gebieten, in
den Abgrund\textless sup title=``oder: die Hölle''\textgreater✲ zu
fahren. \bibleverse{32} Nun befand sich dort eine große Herde Schweine
auf der Weide an dem Berge; deshalb baten die Geister ihn um die
Erlaubnis, in diese fahren zu dürfen, und er erlaubte es ihnen.
\bibleverse{33} So fuhren denn die Geister aus dem Manne aus und in die
Schweine hinein; und die Herde stürmte den Abhang hinab in den See und
ertrank dort.

\bibleverse{34} Als nun die Hirten sahen, was geschehen war, ergriffen
sie die Flucht und erstatteten Meldung in der Stadt und in den Gehöften.
\bibleverse{35} Da zogen die Leute hinaus, um zu sehen, was vorgefallen
war; sie kamen zu Jesus und fanden den Mann, aus dem die Geister
ausgefahren waren, bekleidet und ganz vernünftig zu den Füßen Jesu
sitzen und gerieten darüber in Furcht. \bibleverse{36} Die Augenzeugen
erzählten ihnen dann, wie der (früher) Besessene geheilt worden war.
\bibleverse{37} Da bat ihn die gesamte Bevölkerung der Umgegend von
Gergesa, er möchte ihr Gebiet verlassen; denn sie waren in große Furcht
geraten. So stieg er denn wieder ins Boot und machte sich auf den
Rückweg.

\bibleverse{38} Hierauf richtete der Mann, von dem die bösen Geister
ausgefahren waren, die Bitte an ihn, bei ihm bleiben zu dürfen; doch
Jesus hieß ihn gehen mit den Worten: \bibleverse{39} »Kehre in dein Haus
zurück und erzähle dort, wie Großes Gott an dir getan hat!« Da ging er
denn auch hin und verkündete in der ganzen Stadt, wie Großes Jesus an
ihm getan hatte.

\hypertarget{c-jesus-heilt-die-blutfluxfcssige-frau-und-erweckt-das-tuxf6chterlein-des-jairus}{%
\paragraph{c) Jesus heilt die blutflüssige Frau und erweckt das
Töchterlein des
Jairus}\label{c-jesus-heilt-die-blutfluxfcssige-frau-und-erweckt-das-tuxf6chterlein-des-jairus}}

\bibleverse{40} Als Jesus dann zurückkehrte, nahm die Volksmenge ihn mit
Freuden in Empfang, denn sie hatten alle auf ihn gewartet.
\bibleverse{41} Da kam ein Mann namens Jairus, ein Vorsteher der
(dortigen) Synagoge, warf sich vor Jesus nieder und bat ihn, in sein
Haus zu kommen; \bibleverse{42} er hatte nämlich eine einzige Tochter
von ungefähr zwölf Jahren, und diese lag im Sterben.

Während Jesus nun hinging, umdrängte ihn die Volksmenge. \bibleverse{43}
Und eine Frau, die schon seit zwölf Jahren am Blutfluß litt und
{[}obgleich sie ihr ganzes Vermögen an Ärzte aufgewandt hatte{]} bei
keinem Heilung hatte finden können, \bibleverse{44} die trat von hinten
an ihn heran und faßte die Quaste\textless sup title=``vgl. 4.Mose
15,38-41''\textgreater✲ seines Mantels an, und augenblicklich kam der
Blutfluß bei ihr zum Stillstand. \bibleverse{45} Da fragte Jesus: »Wer
hat mich angefaßt?« Als nun alle es in Abrede stellten, sagte Petrus:
»Meister, die Volksmenge umdrängt und stößt dich von allen Seiten!«
\bibleverse{46} Jesus aber erwiderte: »Es hat mich jemand angefaßt, ich
habe ja gefühlt, daß eine Kraft von mir ausgegangen ist.«
\bibleverse{47} Als nun die Frau sah, daß sie nicht unbemerkt geblieben
war, kam sie zitternd herbei, warf sich vor ihm nieder und bekannte vor
dem ganzen Volk, aus welchem Grunde sie ihn angefaßt habe und wie sie
augenblicklich gesund geworden sei. \bibleverse{48} Da sagte er zu ihr:
»Meine Tochter, dein Glaube hat dir Heilung verschafft: gehe in
Frieden!«

\bibleverse{49} Während er noch redete, kam einer von den Leuten des
Synagogenvorstehers mit der Meldung: »Deine Tochter ist gestorben:
bemühe den Meister nicht weiter!« \bibleverse{50} Als Jesus das hörte,
sagte er zu Jairus: »Fürchte dich nicht, glaube nur, dann wird sie
gerettet werden!« \bibleverse{51} Als er dann an das Haus gekommen war,
ließ er niemand (von den Seinen) mit sich eintreten außer Petrus,
Johannes, Jakobus und den Eltern des Mädchens. \bibleverse{52} Alle
weinten aber und wehklagten laut um sie; er jedoch sagte: »Weinet nicht!
Sie ist nicht tot, sondern schläft nur«; \bibleverse{53} da verlachten
sie ihn, weil sie wohl wußten, daß sie tot war. \bibleverse{54} Er aber
faßte sie bei der Hand und rief ihr laut zu: »Mädchen, stehe auf!«
\bibleverse{55} Da kehrte ihr Geist zu ihr zurück, und sie stand
sogleich auf; und er ordnete an, man solle ihr zu essen geben.
\bibleverse{56} Und ihre Eltern waren vor Erregung ganz außer sich; er
aber gebot ihnen, keinem etwas von dem Geschehenen zu erzählen.

\hypertarget{aussendung-der-zwuxf6lf-juxfcnger-und-anweisungen-fuxfcr-sie}{%
\subsubsection{17. Aussendung der zwölf Jünger und Anweisungen für
sie}\label{aussendung-der-zwuxf6lf-juxfcnger-und-anweisungen-fuxfcr-sie}}

\hypertarget{section-8}{%
\section{9}\label{section-8}}

\bibleverse{1} Er rief dann die Zwölf zusammen und gab ihnen Kraft und
Vollmacht über alle bösen Geister sowie zur Heilung von Krankheiten,
\bibleverse{2} hierauf sandte er sie aus, das Reich Gottes zu verkünden
und (die Kranken) zu heilen. \bibleverse{3} Dabei gab er ihnen die
Weisung: »Nehmt nichts mit auf den Weg, weder einen Stock noch einen
Ranzen\textless sup title=``oder: eine Reisetasche''\textgreater✲, weder
Brot noch Geld; auch sollt ihr nicht jeder zwei Röcke haben!
\bibleverse{4} Wo ihr in ein Haus eingetreten seid, dort bleibt und von
dort zieht weiter! \bibleverse{5} Und wo man euch nicht aufnimmt, da
geht aus einer solchen Stadt weg und schüttelt den Staub von euren Füßen
ab zum Zeugnis wider sie!« \bibleverse{6} So machten sie sich denn auf
den Weg und wanderten von Dorf zu Dorf, indem sie überall die
Heilsbotschaft verkündeten und Heilungen vollführten.

\hypertarget{abschluuxdf-der-wirksamkeit-jesu-in-galiluxe4a}{%
\subsubsection{18. Abschluß der Wirksamkeit Jesu in
Galiläa}\label{abschluuxdf-der-wirksamkeit-jesu-in-galiluxe4a}}

\hypertarget{a-urteil-des-herodes-uxfcber-jesus}{%
\paragraph{a) Urteil des Herodes über
Jesus}\label{a-urteil-des-herodes-uxfcber-jesus}}

\bibleverse{7} Es hörte aber der Vierfürst Herodes von allen diesen
Begebenheiten und fühlte sich dadurch beunruhigt; denn manche
behaupteten, Johannes sei von den Toten auferweckt worden;
\bibleverse{8} andere wieder meinten, Elia sei erschienen; noch andere,
einer von den alten Propheten sei auferstanden. \bibleverse{9} Herodes
aber sagte\textless sup title=``oder: dachte''\textgreater✲: »Den
Johannes habe ich enthaupten lassen; wer mag nun dieser sein, über den
ich solche Dinge höre?« So hegte er denn den Wunsch, Jesus persönlich zu
sehen.

\hypertarget{b-ruxfcckkehr-der-apostel-jesus-zieht-sich-in-die-einsamkeit-zuruxfcck-speisung-der-fuxfcnftausend}{%
\paragraph{b) Rückkehr der Apostel; Jesus zieht sich in die Einsamkeit
zurück; Speisung der
Fünftausend}\label{b-ruxfcckkehr-der-apostel-jesus-zieht-sich-in-die-einsamkeit-zuruxfcck-speisung-der-fuxfcnftausend}}

\bibleverse{10} Nach ihrer Rückkehr berichteten ihm die Apostel alles,
was sie getan hatten. Da nahm er sie mit sich und zog sich in die Stille
zurück in eine Ortschaft namens Bethsaida. \bibleverse{11} Als aber die
Volksmenge das in Erfahrung gebracht hatte, zogen sie ihm nach, und er
ließ sie auch zu sich kommen, redete zu ihnen vom Reiche Gottes und
machte die gesund, welche der Heilung bedurften. \bibleverse{12} Als der
Tag sich dann zu neigen begann, traten die Zwölf an ihn heran und sagten
zu ihm: »Laß das Volk ziehen, damit sie in die umliegenden Ortschaften
und Gehöfte gehen und dort Unterkunft und Verpflegung finden; denn hier
sind wir in einer öden Gegend.« \bibleverse{13} Doch er antwortete
ihnen: »Gebt ihr ihnen doch zu essen!« Da erwiderten sie: »Wir haben
nicht mehr als fünf Brote und zwei Fische; wir müßten sonst hingehen und
Lebensmittel für dieses ganze Volk einkaufen«~-- \bibleverse{14} es
waren nämlich gegen fünftausend Männer. Er sagte aber zu seinen Jüngern:
»Laßt sie sich in Gruppen von etwa je fünfzig Personen lagern.«
\bibleverse{15} Sie taten so und brachten alle dazu, sich zu lagern.
\bibleverse{16} Darauf nahm er die fünf Brote und die beiden Fische,
blickte zum Himmel auf, sprach den Lobpreis (Gottes), brach die Brote
und gab sie\textless sup title=``d.h. die Stücke''\textgreater✲ immer
wieder den Jüngern, damit diese sie dem Volk vorlegten. \bibleverse{17}
Und sie aßen und wurden alle satt; dann las man die Brocken auf, die sie
übriggelassen hatten, zwölf Körbe voll.

\hypertarget{c-das-messiasbekenntnis-des-petrus-und-die-erste-leidensankuxfcndigung-jesu}{%
\paragraph{c) Das Messiasbekenntnis des Petrus und die erste
Leidensankündigung
Jesu}\label{c-das-messiasbekenntnis-des-petrus-und-die-erste-leidensankuxfcndigung-jesu}}

\bibleverse{18} Es begab sich hierauf, als er für sich allein betete,
daß nur die Jünger sich bei ihm befanden; da fragte er sie: »Für wen
halten mich die Volksscharen?« \bibleverse{19} Sie gaben ihm zur
Antwort: »Für Johannes den Täufer, andere für Elia, noch andere meinen,
einer von den alten Propheten sei auferstanden.« \bibleverse{20} Darauf
fragte er sie weiter: »Ihr aber -- für wen haltet ihr mich?« Da
antwortete Petrus: »Für Christus, den Gottgesalbten!« \bibleverse{21} Da
gab er ihnen die strenge Weisung und gebot ihnen, sie sollten das
niemand sagen, \bibleverse{22} und fügte noch hinzu: »Der Menschensohn
muß vieles leiden und von den Ältesten und Hohenpriestern und
Schriftgelehrten verworfen werden und den Tod erleiden und am dritten
Tage auferweckt werden.«

\hypertarget{d-spruxfcche-uxfcber-die-nachfolge-der-juxfcnger}{%
\paragraph{d) Sprüche über die Nachfolge der
Jünger}\label{d-spruxfcche-uxfcber-die-nachfolge-der-juxfcnger}}

\bibleverse{23} Dann sagte er zu allen: »Will jemand mein Nachfolger✲
sein, so verleugne er sich selbst und nehme sein Kreuz Tag für Tag auf
sich und folge so mir nach!~-- \bibleverse{24} Denn wer sein Leben
retten will, der wird es verlieren; wer aber sein Leben um meinetwillen
verliert, der wird es retten. \bibleverse{25} Denn was hülfe es einem
Menschen, wenn er die ganze Welt gewänne, sich selbst aber verlöre oder
einbüßte?~-- \bibleverse{26} Denn wer sich meiner und meiner Worte
schämt, dessen wird auch der Menschensohn sich schämen, wenn er in
seiner Herrlichkeit und in der Herrlichkeit des Vaters und der heiligen
Engel kommt. \bibleverse{27} Ich sage euch aber der Wahrheit gemäß:
Einige unter denen, die hier stehen, werden\textless sup title=``oder:
sollen''\textgreater✲ den Tod nicht schmecken, bis sie das Reich Gottes
gesehen haben.«

\hypertarget{e-jesu-verkluxe4rung-auf-dem-berge}{%
\paragraph{e) Jesu Verklärung auf dem
Berge}\label{e-jesu-verkluxe4rung-auf-dem-berge}}

\bibleverse{28} Etwa acht Tage nach diesen Unterredungen nahm er Petrus,
Johannes und Jakobus mit sich und stieg auf den Berg, um zu beten.
\bibleverse{29} Während er nun betete, veränderte sich das Aussehen
seines Angesichts, und seine Kleidung wurde leuchtend weiß.
\bibleverse{30} Und siehe, zwei Männer besprachen sich mit ihm, das
waren Mose und Elia; \bibleverse{31} sie erschienen in (himmlischer)
Herrlichkeit✲ und redeten davon, wie sein Lebensausgang sich in
Jerusalem vollziehen sollte. \bibleverse{32} Petrus aber und seine
Genossen waren von schwerer Schläfrigkeit befallen; weil sie sich aber
mit Gewalt wach hielten, sahen sie seine Herrlichkeit und die beiden
Männer, die bei ihm standen. \bibleverse{33} Als diese von ihm scheiden
wollten, sagte Petrus zu Jesus: »Meister, hier sind wir gut
aufgehoben\textless sup title=``vgl. Mt 17,4''\textgreater✲; wir wollen
drei Hütten bauen, eine für dich, eine für Mose und eine für Elia« -- er
wußte nämlich nicht, was er da sagte. \bibleverse{34} Während er noch so
redete, kam eine Wolke und überschattete sie; und sie gerieten in
Furcht, als sie in die Wolke hineinkamen. \bibleverse{35} Da erscholl
eine Stimme aus der Wolke, die rief: »Dies ist mein auserwählter Sohn:
höret auf ihn!«, \bibleverse{36} und während die Stimme erscholl, fand
es sich, daß Jesus allein da war. Und die Jünger blieben verschwiegen
und teilten in jenen Tagen niemand etwas von dem mit, was sie gesehen
hatten.

\hypertarget{f-heilung-eines-fallsuxfcchtigen-knaben}{%
\paragraph{f) Heilung eines fallsüchtigen
Knaben}\label{f-heilung-eines-fallsuxfcchtigen-knaben}}

\bibleverse{37} Als sie aber am folgenden Tage von dem Berge wieder
hinabgestiegen waren, kam ihm eine große Volksmenge entgegen.
\bibleverse{38} Da rief ein Mann aus der Volksmenge heraus: »Meister,
ich bitte dich: nimm dich meines Sohnes an, er ist ja mein einziger!
\bibleverse{39} Siehe, ein Geist packt ihn, so daß er plötzlich
aufschreit; und er zerrt ihn hin und her, so daß ihm Schaum vor den Mund
tritt, und läßt nur schwer von ihm ab: er reibt seine Kräfte ganz auf!
\bibleverse{40} Ich habe deine Jünger gebeten, sie möchten ihn
austreiben, doch sie haben es nicht gekonnt.« \bibleverse{41} Da
antwortete Jesus: »O ihr ungläubige und verkehrte Art von Menschen! Wie
lange soll ich noch bei euch sein und es mit euch aushalten? Bringe
deinen Sohn hierher!« \bibleverse{42} Während nun der Knabe noch auf ihn
zuging, riß der böse Geist ihn hin und her und zog ihn krampfhaft
zusammen. Jesus aber bedrohte den unreinen Geist, heilte den Knaben und
gab ihn seinem Vater (gesund) zurück. \bibleverse{43} Da gerieten alle
außer sich vor Staunen über die große Macht Gottes.

\hypertarget{g-zweite-leidensankuxfcndigung-jesu}{%
\paragraph{g) Zweite Leidensankündigung
Jesu}\label{g-zweite-leidensankuxfcndigung-jesu}}

Während nun alle voll Verwunderung über alle seine Taten waren, sagte er
zu seinen Jüngern: \bibleverse{44} »Laßt ihr die Worte, die ich euch
jetzt sage, in eure Ohren dringen! Denn der Menschensohn wird in die
Hände der Menschen überantwortet werden.« \bibleverse{45} Sie verstanden
aber diesen Ausspruch nicht, sondern er blieb vor ihnen verhüllt, damit
sie ihn nicht begriffen\textless sup title=``vgl. Mk
4,12''\textgreater✲; doch scheuten sie sich, ihn wegen dieses Ausspruchs
zu befragen.

\hypertarget{h-der-juxfcnger-uxfcberhebung-belehrung-uxfcber-die-demut-und-uxfcber-die-duldsamkeit}{%
\paragraph{h) Der Jünger Überhebung; Belehrung über die Demut und über
die
Duldsamkeit}\label{h-der-juxfcnger-uxfcberhebung-belehrung-uxfcber-die-demut-und-uxfcber-die-duldsamkeit}}

\bibleverse{46} Es stieg aber (einmal) der Gedanke in ihnen auf, wer
wohl der Größte unter ihnen wäre. \bibleverse{47} Da Jesus nun den
Gedanken kannte, der sie beschäftigte, nahm er ein Kind, stellte es
neben sich \bibleverse{48} und sagte zu ihnen: »Wenn jemand dieses Kind
auf meinen Namen hin aufnimmt, so nimmt er mich auf, und wer mich
aufnimmt, der nimmt den auf, der mich gesandt hat; denn wer der Kleinste
unter euch allen ist, der ist groß\textless sup title=``=~der
Größte''\textgreater✲.«

\bibleverse{49} Da nahm Johannes das Wort und sagte: »Meister, wir haben
jemand gesehen, der mit deinem Namen böse Geister austrieb, und haben es
ihm untersagt, weil er dir nicht mit uns nachfolgt\textless sup
title=``d.h. nicht zu unserm Jüngerkreise gehört''\textgreater✲.«
\bibleverse{50} Jesus aber erwiderte ihm: »Untersagt es ihm nicht! Denn
wer nicht gegen euch ist, der ist für euch.«

\hypertarget{iv.-wirksamkeit-jesu-auf-seiner-langen-wanderung-nach-jerusalem-951-1927}{%
\subsection{IV. Wirksamkeit Jesu auf seiner langen Wanderung nach
Jerusalem
(9,51-19,27)}\label{iv.-wirksamkeit-jesu-auf-seiner-langen-wanderung-nach-jerusalem-951-1927}}

\hypertarget{aufbruch-zur-reise-das-ungastliche-samariterdorf}{%
\subsubsection{1. Aufbruch zur Reise; das ungastliche
Samariterdorf}\label{aufbruch-zur-reise-das-ungastliche-samariterdorf}}

\bibleverse{51} Als dann aber die Zeit seines Hingangs\textless sup
title=``d.h. seiner Aufnahme in den Himmel''\textgreater✲ herankam,
richtete er fest entschlossen sein Augenmerk darauf, nach Jerusalem zu
ziehen, \bibleverse{52} und er sandte Boten vor sich her. Diese machten
sich auf den Weg und kamen in ein Dorf der Samariter, um dort ein
Unterkommen✲ für ihn zu besorgen; \bibleverse{53} doch man nahm ihn
nicht auf, weil er die Absicht hatte, nach Jerusalem zu ziehen.
\bibleverse{54} Als die Jünger Jakobus und Johannes das sahen, fragten
sie: »Herr, willst du, daß wir aussprechen, es solle Feuer vom Himmel
fallen und sie verzehren, wie auch Elia getan hat?«\textless sup
title=``2.Kön 1,10.12.''\textgreater✲ \bibleverse{55} Er aber wandte
sich um und verwies es ihnen mit den Worten: »Wißt ihr nicht, welches
Geistes Kinder ihr seid? Der Menschensohn ist nicht gekommen, um
Menschenleben\textless sup title=``oder: Seelen''\textgreater✲ zu
vernichten, sondern um sie zu retten.« \bibleverse{56} So begaben sie
sich denn in ein anderes Dorf.

\hypertarget{drei-verschiedene-nachfolger-jesu-spruxfcche-uxfcber-die-nachfolge}{%
\subsubsection{2. Drei verschiedene Nachfolger Jesu; Sprüche über die
Nachfolge}\label{drei-verschiedene-nachfolger-jesu-spruxfcche-uxfcber-die-nachfolge}}

\bibleverse{57} Als sie dann des Weges weiterzogen, sagte einer zu ihm:
»Ich will dir folgen, wohin du auch gehst.« \bibleverse{58} Jesus
antwortete ihm: »Die Füchse haben Gruben und die Vögel des Himmels
Nester, der Menschensohn aber hat keine Stätte, wohin er sein Haupt
legen kann.«~-- \bibleverse{59} Zu einem anderen sagte er: »Folge mir
nach!« Der entgegnete: »Erlaube mir, zunächst noch hinzugehen und meinen
Vater zu begraben.« \bibleverse{60} Da antwortete er ihm: »Laß die Toten
ihre Toten begraben! Du aber gehe hin und verkündige das Reich
Gottes!«~-- \bibleverse{61} Noch ein anderer sagte: »Herr, ich will dir
folgen; zunächst aber gestatte mir, von meinen Hausgenossen Abschied zu
nehmen!« \bibleverse{62} Da sagte Jesus zu ihm: »Niemand, der die Hand
an den Pflug gelegt hat und dann noch rückwärts blickt, ist für das
Reich Gottes tauglich.«

\hypertarget{aussendung-der-siebzig-juxfcnger-und-anweisung-fuxfcr-sie-wehruf-uxfcber-die-unbuuxdffertigen-galiluxe4ischen-stuxe4dte}{%
\subsubsection{3. Aussendung der siebzig Jünger und Anweisung für sie;
Wehruf über die unbußfertigen galiläischen
Städte}\label{aussendung-der-siebzig-juxfcnger-und-anweisung-fuxfcr-sie-wehruf-uxfcber-die-unbuuxdffertigen-galiluxe4ischen-stuxe4dte}}

\hypertarget{section-9}{%
\section{10}\label{section-9}}

\bibleverse{1} Hierauf aber bestellte der Herr noch siebzig andere
(Jünger) und sandte sie paarweise vor sich her in alle Städte und
Ortschaften, in die er selbst zu gehen gedachte\textless sup
title=``oder: kommen würde''\textgreater✲. \bibleverse{2} Er sagte zu
ihnen: »Die Ernte ist groß, aber klein die Zahl der Arbeiter; darum
bittet den Herrn der Ernte, daß er Arbeiter auf sein Erntefeld sende!
\bibleverse{3} Geht hin! Seht, ich sende euch wie Lämmer mitten unter
Wölfe. \bibleverse{4} Nehmt keinen Geldbeutel mit euch, auch keinen
Ranzen\textless sup title=``oder: keine Reisetasche''\textgreater✲ und
keine Schuhe, und laßt euch unterwegs mit niemand in lange Begrüßungen
ein. \bibleverse{5} Wo ihr in ein Haus eintretet, da sagt zuerst:
›Friede (sei) mit diesem Hause!‹ \bibleverse{6} Wenn dann dort ein Sohn
des Friedens wohnt, wird der Friede, den ihr ihm gewünscht habt, auf ihm
ruhen; andernfalls wird euer Friedensgruß zu euch zurückkehren.
\bibleverse{7} In demselben Hause bleibt dann und eßt und trinkt, was
man euch bietet; denn der Arbeiter ist seines Lohnes wert\textless sup
title=``=~hat Anspruch auf Lohn''\textgreater✲. Geht nicht aus einem
Hause weg in ein anderes; \bibleverse{8} und wo ihr in einer Stadt
einkehrt und man euch aufnimmt, so eßt, was man euch vorsetzt,
\bibleverse{9} und heilt die Kranken daselbst und sagt zu den
Stadtbewohnern: ›Das Reich Gottes ist nahe zu euch herbeigekommen!‹
\bibleverse{10} Wo ihr aber in einer Stadt einkehrt und man euch nicht
aufnimmt, so geht auf ihre Straßen hinaus und sagt: \bibleverse{11}
›Sogar den Staub, der sich uns aus eurer Stadt an die Füße gehängt hat,
wischen wir ab, damit er euch verbleibt, doch das sollt ihr wissen, daß
das Reich Gottes nahe herbeigekommen ist!‹ \bibleverse{12} Ich sage
euch: Es wird Sodom an jenem Tage erträglicher ergehen als der
betreffenden Stadt!

\bibleverse{13} Wehe dir, Chorazin! Wehe dir, Bethsaida! Denn wenn in
Tyrus und Sidon die Wundertaten geschehen wären, die bei euch geschehen
sind, so hätten sie längst, in Sack und Asche sitzend, Buße
getan\textless sup title=``oder: sich bekehrt; Mt 3,2''\textgreater✲.
\bibleverse{14} Doch es wird Tyrus und Sidon beim Gericht erträglicher
ergehen als euch. \bibleverse{15} Und du, Kapernaum, wirst doch nicht
etwa bis zum Himmel erhöht werden? Nein, bis zum Totenreich wirst du
hinabgestoßen werden!\textless sup title=``Jes 14,13-15''\textgreater✲
\bibleverse{16} Wer euch hört, der hört mich, und wer euch verwirft,
verwirft mich; wer aber mich verwirft, verwirft den, der mich gesandt
hat.«

\hypertarget{ruxfcckkehr-der-siebzig-juxfcnger-jesu-jubelruf-und-seine-seligpreisung-der-juxfcnger}{%
\subsubsection{4. Rückkehr der siebzig Jünger; Jesu Jubelruf und seine
Seligpreisung der
Jünger}\label{ruxfcckkehr-der-siebzig-juxfcnger-jesu-jubelruf-und-seine-seligpreisung-der-juxfcnger}}

\bibleverse{17} Die Siebzig kehrten dann voller Freude zurück und
sagten: »Herr, auch die bösen Geister sind uns kraft\textless sup
title=``=~infolge der Verwendung''\textgreater✲ deines Namens
gehorsam✲!« \bibleverse{18} Da antwortete er ihnen: »Ich habe den Satan
wie einen Blitz aus dem Himmel herabgestürzt gesehen. \bibleverse{19}
Ihr wißt: ich habe euch die Macht verliehen, auf Schlangen und Skorpione
zu treten\textless sup title=``Ps 91,13''\textgreater✲, und Macht über
das ganze Heer des Widersachers, und keinen Schaden wird er euch
irgendwie zufügen können. \bibleverse{20} Doch nicht darüber freuet
euch, daß die Geister euch gehorsam✲ sind; freut euch vielmehr darüber,
daß eure Namen im Himmel eingeschrieben stehen!«

\bibleverse{21} In eben dieser Stunde jubelte Jesus durch den heiligen
Geist mit den Worten: »Ich preise dich\textless sup title=``oder: danke
dir''\textgreater✲, Vater, Herr des Himmels und der Erde, daß du dies
vor Weisen und Klugen verborgen und es Unmündigen geoffenbart hast; ja,
Vater, denn so ist es dir wohlgefällig gewesen. \bibleverse{22} Alles
ist mir von meinem Vater übergeben worden, und niemand erkennt, wer der
Sohn ist, als nur der Vater, und wer der Vater ist, als nur der Sohn,
und wem der Sohn ihn\textless sup title=``oder: es''\textgreater✲
offenbaren will.«~-- \bibleverse{23} Dann wandte er sich zu den Jüngern
besonders und sagte: »Selig sind die Augen, die da sehen, was ihr seht!
\bibleverse{24} Denn ich sage euch: Viele Propheten und Könige haben
gewünscht, das zu sehen, was ihr seht, und haben es nicht gesehen, und
das zu hören, was ihr hört, und haben es nicht gehört.«

\hypertarget{jesus-und-der-gesetzeslehrer-wesen-der-nuxe4chstenliebe-erzuxe4hlung-vom-barmherzigen-samariter}{%
\subsubsection{5. Jesus und der Gesetzeslehrer; Wesen der Nächstenliebe;
Erzählung vom barmherzigen
Samariter}\label{jesus-und-der-gesetzeslehrer-wesen-der-nuxe4chstenliebe-erzuxe4hlung-vom-barmherzigen-samariter}}

\bibleverse{25} Da trat ein Gesetzeslehrer auf, um ihn zu versuchen, und
fragte: »Meister, was muß ich tun, um ewiges Leben zu ererben?«
\bibleverse{26} Jesus erwiderte ihm: »Was steht im Gesetz geschrieben?
Wie lauten da die Worte?« \bibleverse{27} Er gab zur Antwort: »Du sollst
den Herrn, deinen Gott, lieben von ganzem Herzen, mit deiner ganzen
Seele, mit aller deiner Kraft und mit deinem ganzen Denken«\textless sup
title=``5.Mose 6,5''\textgreater✲ und »deinen Nächsten wie dich
selbst«\textless sup title=``3.Mose 19,18''\textgreater✲.
\bibleverse{28} Jesus sagte zu ihm: »Du hast richtig geantwortet; tu
das, so wirst du leben!« \bibleverse{29} Jener wollte sich aber
rechtfertigen und sagte zu Jesus: »Ja, wer ist denn mein Nächster?«
\bibleverse{30} Da erwiderte Jesus: »Ein Mann ging von Jerusalem nach
Jericho hinab und fiel Räubern in die Hände; die plünderten ihn aus,
schlugen ihn blutig, ließen ihn halbtot liegen und gingen davon.
\bibleverse{31} Zufällig kam ein Priester jene Straße hinabgezogen und
sah ihn liegen, ging aber vorüber. \bibleverse{32} Ebenso kam auch ein
Levit an die Stelle und sah ihn, ging aber vorüber. \bibleverse{33} Ein
Samariter aber, der auf der Reise war, kam in seine Nähe, und als er ihn
sah, fühlte er Mitleid mit ihm; \bibleverse{34} er trat an ihn heran und
verband ihm die Wunden, wobei er Öl und Wein darauf goß; dann setzte er
ihn auf sein Maultier, brachte ihn in eine Herberge und verpflegte ihn.
\bibleverse{35} Am folgenden Morgen holte er zwei Denare✲ heraus (aus
seinem Beutel), gab sie dem Wirt und sagte: ›Verpflege ihn, und was es
dich etwa mehr kostet, will ich dir bei meiner Rückkehr ersetzen.‹
\bibleverse{36} Wer von diesen dreien hat sich nun nach deiner Ansicht
dem unter die Räuber Gefallenen als Nächster erwiesen?« \bibleverse{37}
Jener antwortete: »Der, welcher die Barmherzigkeit an ihm geübt hat.« Da
sagte Jesus zu ihm: »So gehe hin und handle du ebenso!«

\hypertarget{martha-und-maria-in-bethanien}{%
\subsubsection{6. Martha und Maria in
Bethanien}\label{martha-und-maria-in-bethanien}}

\bibleverse{38} Als sie dann weiterwanderten, kam er in ein Dorf, und
eine Frau namens Martha nahm ihn in ihr Haus auf. \bibleverse{39} Diese
hatte eine Schwester namens Maria, die sich zu den Füßen des Herrn
niederließ und seinen Worten zuhörte; \bibleverse{40} Martha dagegen
ließ sich durch vielerlei Dienstleistungen für die Bewirtung in Anspruch
nehmen. Nun trat sie zu ihm und sagte: »Herr, machst du dir nichts
daraus, daß meine Schwester die Bedienung mir allein überlassen hat?
Sage ihr doch, sie möge mir zur Hand gehen!« \bibleverse{41} Aber der
Herr gab ihr zur Antwort: »Martha, Martha! Du machst dir {[}Sorge und{]}
Unruhe um vielerlei; \bibleverse{42} aber nur eins ist nötig. Denn Maria
hat das gute Teil erwählt: das soll ihr nicht genommen werden.«

\hypertarget{vom-gebet}{%
\subsubsection{7. Vom Gebet}\label{vom-gebet}}

\hypertarget{a-jesus-lehrt-die-juxfcnger-beten-das-vaterunser}{%
\paragraph{a) Jesus lehrt die Jünger beten: das
Vaterunser}\label{a-jesus-lehrt-die-juxfcnger-beten-das-vaterunser}}

\hypertarget{section-10}{%
\section{11}\label{section-10}}

\bibleverse{1} Jesus betete (einst unterwegs) an einem Orte✲, und als er
damit zu Ende war, sagte einer seiner Jünger zu ihm: »Herr, lehre uns
beten\textless sup title=``=~ein Gebet''\textgreater✲, wie auch Johannes
seine Jünger (Gebete) gelehrt hat!« \bibleverse{2} Da sagte er zu ihnen:
»Wenn ihr beten wollt, so sprecht: ›Vater, geheiligt werde dein Name!
Dein Reich komme! \bibleverse{3} Unser auskömmliches Brot\textless sup
title=``vgl. Mt 6,11''\textgreater✲ gib uns Tag für Tag! \bibleverse{4}
Und vergib uns unsere Sünden, denn auch wir vergeben jedem, der sich an
uns verschuldet! Und führe uns nicht in Versuchung!‹«

\hypertarget{b-das-gleichnis-vom-bittenden-freund-jesus-ermuntert-zum-instuxe4ndigen-und-anhaltenden-beten}{%
\paragraph{b) Das Gleichnis vom bittenden Freund; Jesus ermuntert zum
inständigen und anhaltenden
Beten}\label{b-das-gleichnis-vom-bittenden-freund-jesus-ermuntert-zum-instuxe4ndigen-und-anhaltenden-beten}}

\bibleverse{5} Dann fuhr er fort: »Wer unter euch hätte wohl einen
Freund und ginge (nicht) mitten in der Nacht zu ihm und sagte zu ihm:
›Freund, hilf mir mit drei Broten aus! \bibleverse{6} Denn ein Freund
von mir ist auf der Reise zu mir gekommen, und ich habe ihm nichts
vorzusetzen‹; \bibleverse{7} und jener würde von drinnen antworten:
›Belästige mich nicht! Die Tür ist schon verschlossen, und meine Kinder
liegen schon bei mir im Bett; ich kann nicht aufstehen und es dir
geben!‹ \bibleverse{8} Ich sage euch: Wenn er auch nicht deshalb
aufstehen und ihm das Gewünschte geben mag, weil jener sein Freund ist,
so wird er doch wegen dessen Hartnäckigkeit aufstehen und ihm geben,
soviel er bedarf. \bibleverse{9} So sage denn auch ich euch: Bittet, so
wird euch gegeben werden; suchet, so werdet ihr finden; klopft an, so
wird man euch auftun! \bibleverse{10} Denn wer da bittet, empfängt, und
wer da sucht, findet, und wer anklopft, dem wird man auftun.
\bibleverse{11} Wo wäre aber unter euch ein Vater, der seinem Sohne,
wenn er ihn um Brot bittet, einen Stein reichte? Oder wenn er ihn um
einen Fisch bittet, wird er ihm statt dessen wohl eine Schlange reichen?
\bibleverse{12} Oder auch einen Skorpion statt eines Eies?
\bibleverse{13} Wenn nun ihr, die ihr doch böse seid, euren Kindern gute
Gaben zu geben versteht: wieviel mehr wird der Vater vom Himmel her
Heiligen Geist denen geben, die ihn darum bitten!«

\hypertarget{jesus-verteidigt-sich-gegen-die-beelzebul-luxe4sterung-der-pharisuxe4er-gleichnis-vom-ruxfcckfall}{%
\subsubsection{8. Jesus verteidigt sich gegen die Beelzebul-Lästerung
der Pharisäer; Gleichnis vom
Rückfall}\label{jesus-verteidigt-sich-gegen-die-beelzebul-luxe4sterung-der-pharisuxe4er-gleichnis-vom-ruxfcckfall}}

\bibleverse{14} Er trieb dann einen bösen Geist aus, der stumm war; und
als der böse Geist ausgefahren war, konnte der Stumme reden. Da geriet
die Volksmenge in Erstaunen. \bibleverse{15} Einige von ihnen aber
sagten: »Im Bunde mit Beelzebul\textless sup title=``vgl. 2.Kön
1,2''\textgreater✲, dem Obersten✲ der bösen Geister, treibt er die bösen
Geister aus.« \bibleverse{16} Andere wieder stellten ihn auf die Probe,
indem sie von ihm ein Wunderzeichen vom Himmel her verlangten.
\bibleverse{17} Weil er jedoch ihre Gedanken kannte, sagte er zu ihnen:
»Jedes Reich, das gegen sich selbst entzweit\textless sup title=``oder:
in sich selbst uneinig''\textgreater✲ ist, wird verwüstet, und ein Haus
stürzt auf das andere. \bibleverse{18} Wenn nun auch der Satan
gegen\textless sup title=``oder: mit''\textgreater✲ sich selbst in
Zwiespalt geraten ist, wie wird dann seine Herrschaft Bestand haben
können? Ihr behauptet ja, daß ich die bösen Geister im Bunde mit
Beelzebul austreibe! \bibleverse{19} Wenn ich aber die Geister im Bunde
mit Beelzebul austreibe, mit wessen Hilfe treiben eure
Söhne\textless sup title=``oder: eigenen Leute''\textgreater✲ sie aus?
Darum werden diese eure Richter sein\textless sup title=``=~euch das
Urteil sprechen''\textgreater✲. \bibleverse{20} Wenn ich aber die bösen
Geister durch Gottes Finger austreibe, dann ist ja das Reich Gottes
(schon) zu euch gekommen.~-- \bibleverse{21} Solange der Starke in
voller Waffenrüstung sein Schloß bewacht, ist sein Besitztum in
Sicherheit; \bibleverse{22} wenn aber ein Stärkerer ihn überfällt und
besiegt, so nimmt er ihm seine Waffenrüstung, auf die er sich verlassen
hatte, und teilt die ihm abgenommene Beute aus.~-- \bibleverse{23} Wer
nicht mit mir ist, der ist gegen mich, und wer nicht mit mir sammelt,
der zerstreut.

\bibleverse{24} Wenn der unreine Geist von einem Menschen ausgefahren
ist, durchwandert er wüste Gegenden und sucht eine Ruhestätte; und wenn
er keine findet, so sagt\textless sup title=``oder: denkt''\textgreater✲
er: ›Ich will in mein Haus zurückkehren, das ich verlassen habe.‹
\bibleverse{25} Wenn er dann hinkommt, findet er es sauber gefegt und
schön aufgeräumt. \bibleverse{26} Hierauf geht er hin und nimmt noch
sieben andere Geister, die bösartiger sind als er selbst; und sie ziehen
ein und nehmen dort Wohnung; und das Ende wird bei einem solchen
Menschen schlimmer als der Anfang war.«

\hypertarget{jesu-seligpreisung-der-wahren-gotteskinder}{%
\paragraph{Jesu Seligpreisung der wahren
Gotteskinder}\label{jesu-seligpreisung-der-wahren-gotteskinder}}

\bibleverse{27} Als er so redete, erhob eine Frau aus der Volksmenge
ihre Stimme und rief ihm zu: »Selig (zu preisen) ist der Mutterschoß,
der dich getragen, und die Brüste, die dich genährt haben!«
\bibleverse{28} Er aber erwiderte: »Selig sind vielmehr die, welche das
Wort Gottes hören und bewahren!«

\hypertarget{rede-jesu-gegen-die-zeichenforderer-das-jonazeichen}{%
\subsubsection{9. Rede Jesu gegen die Zeichenforderer; das
Jonazeichen}\label{rede-jesu-gegen-die-zeichenforderer-das-jonazeichen}}

\bibleverse{29} Als dann immer noch mehr Volksscharen sich sammelten,
begann er folgende Rede: »Das gegenwärtige Geschlecht ist ein böses
Geschlecht! Es verlangt ein Zeichen, doch es wird ihm kein Zeichen
gegeben werden als nur das Zeichen des (Propheten) Jona. \bibleverse{30}
Denn wie Jona einst für die Bewohner von Ninive zu einem Zeichen
geworden ist\textless sup title=``Jona 3,3-5''\textgreater✲, so wird es
auch der Menschensohn für dieses Geschlecht sein. \bibleverse{31} Die
Königin aus dem Süden\textless sup title=``1.Kön 10,1-10''\textgreater✲
wird im Gericht mit✲ den Männern dieses Geschlechts (als Zeugin)
auftreten und ihre Verurteilung herbeiführen; denn sie ist von den Enden
der Erde gekommen, um die Weisheit Salomos zu hören; und hier steht
Größeres\textless sup title=``d.h. einer, der größer ist''\textgreater✲
als Salomo. \bibleverse{32} Die Männer von Ninive werden im Gericht mit✲
diesem Geschlecht (als Zeugen) auftreten und seine Verurteilung
herbeiführen; denn sie haben auf Jonas Predigt hin Buße getan; und hier
steht Größeres\textless sup title=``d.h. einer, der größer
ist''\textgreater✲ als Jona.~-- \bibleverse{33} Niemand zündet ein Licht
an und stellt es dann in einen verborgenen Winkel oder unter den
Scheffel, sondern auf den Leuchter✲, damit die Eintretenden den hellen
Schein sehen. \bibleverse{34} Die Leuchte des Leibes ist dein Auge.
Solange nun dein Auge richtig\textless sup title=``oder:
gesund''\textgreater✲ ist, hat auch dein ganzer Leib Licht; wenn es aber
nichts taugt, so befindet sich auch dein (ganzer) Leib in Finsternis.
\bibleverse{35} Darum gib wohl acht, daß das Licht in dir sich nicht
verfinstert! \bibleverse{36} Ist nun dein ganzer Leib
beleuchtet\textless sup title=``oder: ins Licht getreten''\textgreater✲
und kein Teil an ihm im Finstern geblieben, dann wird er ganz so in
Helligkeit sein, wie wenn die Lampe dich mit ihrem hellen Schein
bestrahlt.«

\hypertarget{strafrede-jesu-gegen-die-pharisuxe4er-und-schriftgelehrten}{%
\subsubsection{10. Strafrede Jesu gegen die Pharisäer und
Schriftgelehrten}\label{strafrede-jesu-gegen-die-pharisuxe4er-und-schriftgelehrten}}

\hypertarget{a-gegen-die-heuchelei-der-pharisuxe4er}{%
\paragraph{a) Gegen die Heuchelei der
Pharisäer}\label{a-gegen-die-heuchelei-der-pharisuxe4er}}

\bibleverse{37} Im Anschluß an diese Rede bat ihn ein Pharisäer, zu
Mittag bei ihm zu speisen; er ging auch zu ihm ins Haus und setzte sich
(ohne weiteres) zu Tisch. \bibleverse{38} Als der Pharisäer das sah,
äußerte er seine Verwunderung darüber, daß Jesus sich vor der Mahlzeit
nicht zunächst gewaschen habe. \bibleverse{39} Da sagte der Herr zu ihm:
»Nun ja, ihr Pharisäer haltet die Außenseite des Bechers und der
Schüssel rein, euer Inneres aber ist voll von Raub (-gier) und Bosheit.
\bibleverse{40} Ihr Toren! Hat nicht der, welcher das Auswendige
geschaffen hat, auch das Inwendige geschaffen? \bibleverse{41} Gebt nur
das, was sich darin befindet, als Almosen: seht, dann habt ihr sofort
alles rein. \bibleverse{42} Aber wehe euch Pharisäern! Ihr entrichtet
den Zehnten von Minze, Raute und jedem anderen Gartengewächs, laßt aber
das Recht\textless sup title=``oder: die Rechtspflege, oder: das
Gericht''\textgreater✲ und die Liebe zu Gott außer acht. Vielmehr sollte
man diese (beiden) üben und jenes nicht unterlassen. \bibleverse{43}
Wehe euch Pharisäern! Ihr liebt den Ehrenplatz in den Synagogen und
wollt auf den Märkten\textless sup title=``oder: Straßen''\textgreater✲
gegrüßt sein. \bibleverse{44} Wehe euch! Ihr seid wie die unkenntlich
gewordenen Gräber, über welche die Leute, ohne es zu wissen✲, hingehen.«

\hypertarget{b-wehrufe-uxfcber-die-bosheit-der-gesetzeslehrer-wirkung-der-rede}{%
\paragraph{b) Wehrufe über die Bosheit der Gesetzeslehrer; Wirkung der
Rede}\label{b-wehrufe-uxfcber-die-bosheit-der-gesetzeslehrer-wirkung-der-rede}}

\bibleverse{45} Da nahm einer von den Gesetzeslehrern das Wort und sagte
zu ihm: »Meister, mit diesen Worten beleidigst du auch uns!«
\bibleverse{46} Er aber entgegnete: »Wehe auch euch Gesetzeslehrern!
Denn ihr bürdet den Menschen schwer zu tragende Lasten auf, rührt aber
selber die Lasten mit keinem Finger an. \bibleverse{47} Wehe euch! Ihr
erbaut den Propheten Grabmäler, während eure Väter sie getötet haben.
\bibleverse{48} Damit tretet ihr als Zeugen für die Taten eurer Väter
auf und zollt ihnen Beifall; denn jene haben sie getötet, und ihr
errichtet ihnen Bauwerke. \bibleverse{49} Darum hat auch die Weisheit
Gottes gesagt: ›Ich will Propheten und Apostel zu ihnen senden, von
denen sie einige töten und verfolgen werden, \bibleverse{50} damit das
Blut aller Propheten, das seit Grundlegung der Welt vergossen worden
ist, an diesem Geschlecht gerächt werde, \bibleverse{51} vom Blute Abels
an\textless sup title=``1.Mose 4,8''\textgreater✲ bis zum Blute
Sacharjas\textless sup title=``2.Chr 24,20-22''\textgreater✲, der
zwischen dem Brandopferaltar und dem Tempelhause den Tod erlitten hat.‹
Ja, ich sage euch: es wird an diesem Geschlecht gerächt werden!
\bibleverse{52} Wehe euch Gesetzeslehrern! Ihr habt den Schlüssel zur
Erkenntnis (des Heils) weggenommen✲; ihr selbst seid nicht
hineingegangen, und die, welche hineingehen wollten, habt ihr daran
gehindert.«

\bibleverse{53} Als er dann von dort weggegangen war, begannen die
Schriftgelehrten und Pharisäer ihm mit Erbitterung
zuzusetzen\textless sup title=``oder: nachzustellen''\textgreater✲ und
ihn über immer mehr Dinge auszufragen, \bibleverse{54} wobei sie ihn
belauerten, um ein (unbedachtes) Wort aus seinem Munde aufzufangen.

\hypertarget{warnungen-und-mahnungen-jesu-an-die-juxfcnger}{%
\subsubsection{11. Warnungen und Mahnungen Jesu an die
Jünger}\label{warnungen-und-mahnungen-jesu-an-die-juxfcnger}}

\hypertarget{a-warnung-vor-pharisuxe4ischer-heuchelei}{%
\paragraph{a) Warnung vor pharisäischer
Heuchelei}\label{a-warnung-vor-pharisuxe4ischer-heuchelei}}

\hypertarget{section-11}{%
\section{12}\label{section-11}}

\bibleverse{1} Als sich unterdessen eine Volksmenge von vielen Tausenden
angesammelt hatte, so daß sie einander auf die Füße traten, begann er,
zuerst zu seinen Jüngern zu sagen: »Hütet euch vor dem Sauerteig der
Pharisäer, das heißt vor der Heuchelei!~-- \bibleverse{2} Nichts aber
ist verhüllt, das nicht enthüllt werden wird, und nichts verborgen, was
nicht bekannt werden wird. \bibleverse{3} Daher wird alles, was ihr im
Dunkeln geredet habt, im Licht (der Öffentlichkeit) gehört werden; und
was ihr in den Kammern ins Ohr geflüstert habt, wird auf den Dächern
ausgerufen werden.«

\hypertarget{b-warnung-vor-menschenfurcht}{%
\paragraph{b) Warnung vor
Menschenfurcht}\label{b-warnung-vor-menschenfurcht}}

\bibleverse{4} »Ich sage aber euch, meinen Freunden: Fürchtet euch nicht
vor denen, die den Leib zwar töten, danach aber euch nichts weiter antun
können! \bibleverse{5} Ich will euch aber angeben, vor wem ihr euch zu
fürchten habt: Fürchtet euch vor dem, der die Macht besitzt zu töten und
dann auch noch in die Hölle zu werfen! Ja, ich sage euch: Vor diesem
fürchtet euch!~-- \bibleverse{6} Verkauft man nicht fünf Sperlinge für
zwei Kupferstücke\textless sup title=``=~ein paar
Pfennige''\textgreater✲? Und doch ist kein einziger von ihnen bei Gott
vergessen. \bibleverse{7} Nun sind aber (bei euch) sogar die Haare auf
eurem Haupt alle gezählt. Fürchtet euch nicht: ihr seid mehr wert als
viele Sperlinge!

\bibleverse{8} Ich sage euch aber: Wer sich zu mir vor den Menschen
bekennt, zu dem wird sich auch der Menschensohn vor den Engeln Gottes
bekennen; \bibleverse{9} wer mich aber vor den Menschen verleugnet, der
wird auch vor den Engeln Gottes verleugnet werden.«

\hypertarget{c-warnung-vor-luxe4sterung-des-heiligen-geistes-hinweis-auf-den-beistand-des-geistes}{%
\paragraph{c) Warnung vor Lästerung des heiligen Geistes; Hinweis auf
den Beistand des
Geistes}\label{c-warnung-vor-luxe4sterung-des-heiligen-geistes-hinweis-auf-den-beistand-des-geistes}}

\bibleverse{10} »Und wer immer ein Wort gegen den Menschensohn
ausspricht, der wird Vergebung finden; wer aber gegen den heiligen Geist
eine Lästerung begeht, der wird keine Vergebung finden.~--
\bibleverse{11} Wenn man euch aber vor die Synagogen✲ und vor die
Obrigkeiten und die Behörden stellt, so macht euch keine Sorge darüber,
wie oder womit ihr euch verteidigen oder was ihr sagen sollt!
\bibleverse{12} Denn der heilige Geist wird euch in eben der Stunde
lehren✲, was ihr sagen sollt.«

\hypertarget{uxfcber-die-rechte-stellung-zu-den-dingen-dieser-welt}{%
\subsubsection{12. Über die rechte Stellung zu den Dingen dieser
Welt}\label{uxfcber-die-rechte-stellung-zu-den-dingen-dieser-welt}}

\hypertarget{a-warnung-vor-einmischung-in-irdische-huxe4ndel}{%
\paragraph{a) Warnung vor Einmischung in irdische
Händel}\label{a-warnung-vor-einmischung-in-irdische-huxe4ndel}}

\bibleverse{13} Es sagte aber einer aus der Volksmenge zu ihm: »Meister,
sage doch meinem Bruder, er solle die Erbschaft mit mir teilen!«
\bibleverse{14} Jesus aber antwortete ihm: »Mensch, wer hat mich zum
Richter oder Erbschaftsteiler über euch bestellt?«

\hypertarget{b-warnung-vor-habsucht-gleichnis-vom-reichen-toren}{%
\paragraph{b) Warnung vor Habsucht; Gleichnis vom reichen
Toren}\label{b-warnung-vor-habsucht-gleichnis-vom-reichen-toren}}

\bibleverse{15} Dann fuhr er fort: »Seht euch vor und hütet euch vor
aller\textless sup title=``oder: jeder Art von''\textgreater✲ Habsucht!
Denn wenn jemand auch Überfluß hat, so ist das Leben für ihn doch durch
all sein Besitztum nicht gesichert✲.«

\bibleverse{16} Er legte ihnen dann folgendes Gleichnis vor: »Einem
reichen Manne hatten seine Felder eine ergiebige Ernte gebracht.
\bibleverse{17} Da überlegte er bei sich folgendermaßen: ›Was soll ich
tun? Ich habe keinen Raum, meine Ernte unterzubringen.‹ \bibleverse{18}
Dann sagte er: ›So will ich's machen: Ich will meine Scheunen abreißen
und größere bauen und dort meinen gesamten Ernteertrag und meine
Güter\textless sup title=``oder: Vorräte''\textgreater✲ unterbringen
\bibleverse{19} und will dann zu meiner Seele sagen: Liebe Seele, du
hast nun einen reichen Vorrat auf viele Jahre daliegen; gönne dir also
Ruhe, iß und trink und laß dir's wohl sein!‹ \bibleverse{20} Aber Gott
sprach zu ihm: ›Du Narr! Noch in dieser Nacht fordert man dir deine
Seele ab; wem wird dann das gehören, was du aufgespeichert hast?‹
\bibleverse{21} So geht es jedem, der für sich selbst Schätze sammelt
und nicht reich für\textless sup title=``oder: bei''\textgreater✲ Gott
ist.«

\hypertarget{c-der-juxfcnger-stellung-zum-sorgen-und-irdischen-gut}{%
\paragraph{c) Der Jünger Stellung zum Sorgen und irdischen
Gut}\label{c-der-juxfcnger-stellung-zum-sorgen-und-irdischen-gut}}

\bibleverse{22} Weiter sagte er zu seinen Jüngern: »Deshalb sage ich
euch: Seid nicht besorgt um euer Leben, was ihr essen sollt, auch nicht
um euren Leib, was ihr anziehen sollt! \bibleverse{23} Das Leben ist
doch wertvoller als die Nahrung und der Leib wertvoller als die
Kleidung. \bibleverse{24} Sehet die Raben an: sie säen nicht und ernten
nicht, sie haben keine Vorratskammern und keine Scheunen, und Gott
ernährt sie doch. Wieviel mehr seid ihr doch wert als die Vögel!
\bibleverse{25} Wer aber von euch vermöchte durch all seine Sorgen der
Länge seiner Lebenszeit\textless sup title=``vgl. Mt 6,27''\textgreater✲
auch nur eine Spanne zuzusetzen? \bibleverse{26} Wenn ihr also nicht
einmal etwas ganz Geringes vermögt, wozu macht ihr euch da Sorge um das
Übrige? \bibleverse{27} Sehet die Lilien an, wie sie weder spinnen noch
weben, und doch sage ich euch: Auch Salomo in aller seiner Pracht ist
nicht so herrlich gekleidet gewesen wie eine von diesen. \bibleverse{28}
Wenn nun Gott das Gras auf dem Felde, das heute steht und morgen in den
Ofen geworfen wird, so kleidet: wieviel eher wird er es euch tun, ihr
Kleingläubigen! \bibleverse{29} So fragt denn auch ihr nicht ängstlich,
was ihr essen und was ihr trinken sollt, und regt euch nicht darüber
auf! \bibleverse{30} Denn nach allen diesen Dingen trachten die
Heidenvölker der Welt; euer Vater weiß ja, daß ihr dies bedürft.
\bibleverse{31} Trachtet vielmehr nach seinem Reich, dann wird euch
dieses obendrein gegeben werden. \bibleverse{32} Fürchte dich nicht, du
kleine Herde! Denn eurem Vater hat es gefallen, euch das Reich (Gottes)
zu geben. \bibleverse{33} Verkauft euren Besitz und gebt ihn als Almosen
hin! Verschafft euch Geldbeutel, die sich nicht abnützen, einen Schatz,
der nie zu Ende geht, im Himmel, wo kein Dieb hineinkommt und keine
Motte etwas zernagt! \bibleverse{34} Denn wo euer Schatz ist, da wird
auch euer Herz sein.«

\hypertarget{mahnungen-zur-wachsamkeit-und-bereitschaft-gleichnis-von-den-treuen-knechten-und-vom-buxf6sen-knecht}{%
\subsubsection{13. Mahnungen zur Wachsamkeit und Bereitschaft; Gleichnis
von den treuen Knechten und vom bösen
Knecht}\label{mahnungen-zur-wachsamkeit-und-bereitschaft-gleichnis-von-den-treuen-knechten-und-vom-buxf6sen-knecht}}

\bibleverse{35} »Laßt eure Hüften gegürtet sein und eure Lampen
brennen\textless sup title=``Mt 25,1-13''\textgreater✲! \bibleverse{36}
Denn ihr sollt Leuten gleichen, die auf ihren Herrn warten, wann er vom
Hochzeitsmahl heimkehren werde, um ihm, wenn er kommt und anklopft,
sogleich zu öffnen. \bibleverse{37} Selig zu preisen sind solche
Knechte, die der Herr bei seiner Rückkehr wachend antrifft! Wahrlich ich
sage euch: Er wird sich das Gewand hochschürzen, wird sie sich zu Tische
setzen lassen und herantreten, um sie zu bedienen. \bibleverse{38} Und
mag er erst in der zweiten oder in der dritten Nachtwache kommen und sie
so vorfinden: selig sind sie zu preisen! \bibleverse{39} Das aber seht
ihr ein: Wenn der Hausherr wüßte, in welcher Stunde der Dieb kommt, so
würde er keinen Einbruch in sein Haus zulassen. \bibleverse{40} Darum
haltet auch ihr euch bereit, denn der Menschensohn kommt zu einer
Stunde, in der ihr es nicht vermutet.«

\bibleverse{41} Da fragte Petrus: »Herr, hast du dies Gleichnis nur für
uns bestimmt oder auch für alle anderen?« \bibleverse{42} Der Herr
antwortete: »Wer ist demnach der treue Haushalter, der kluge, den sein
Herr über seine Dienerschaft setzen wird, damit er ihnen das gebührende
Speisemaß zu rechter Zeit gebe? \bibleverse{43} Selig zu preisen ist ein
solcher Knecht, den sein Herr bei seiner Rückkehr in solcher Tätigkeit
findet. \bibleverse{44} Wahrlich ich sage euch: Über seine sämtlichen
Güter wird er ihn setzen. \bibleverse{45} Wenn aber ein solcher Knecht
in seinem Herzen denkt: ›Mein Herr kommt noch lange nicht!‹ und dann
anfängt, die Knechte und Mägde zu schlagen, zu schmausen und zu zechen
und sich zu betrinken: \bibleverse{46} so wird der Herr eines solchen
Knechtes an einem Tage kommen, an dem er ihn nicht erwartet, und zu
einer Stunde, die er nicht kennt; und wird ihn zerhauen\textless sup
title=``Mt 24,51''\textgreater✲ lassen und ihm seinen Platz\textless sup
title=``oder: sein gebührendes Teil''\textgreater✲ bei den
Ungetreuen\textless sup title=``oder: Ungläubigen''\textgreater✲
anweisen. \bibleverse{47} Ein solcher Knecht aber, der den Willen seines
Herrn gekannt und doch nichts ausgeführt und nichts nach seinem Willen
getan hat, wird viele Schläge erhalten; \bibleverse{48} wer dagegen
seinen Willen nicht gekannt und Dinge getan hat, die Züchtigung
verdienen, wird nur wenige Schläge erhalten. Wem aber viel gegeben ist,
von dem wird auch viel gefordert werden, und wem viel anvertraut ist,
von dem wird man auch um so mehr verlangen.«

\hypertarget{mahnung-auf-gottes-weg-zu-achten}{%
\subsubsection{14. Mahnung, auf Gottes Weg zu
achten}\label{mahnung-auf-gottes-weg-zu-achten}}

\hypertarget{a-jesus-bringt-ein-feuer-und-den-zwiespalt-auf-die-erde}{%
\paragraph{a) Jesus bringt ein Feuer und den Zwiespalt auf die
Erde}\label{a-jesus-bringt-ein-feuer-und-den-zwiespalt-auf-die-erde}}

\bibleverse{49} »Ich bin dazu gekommen, ein Feuer auf die Erde zu
werfen, und was sollte ich lieber wünschen, als daß es schon brennte!
\bibleverse{50} Doch mit einer Taufe habe ich mich (vorher) noch taufen
zu lassen, und wie ist mir so bange (und doch zugleich: wie drängt es
mich), bis sie vollzogen ist! \bibleverse{51} Meint ihr, ich sei
gekommen, um Frieden auf die Erde zu bringen? Nein, sage ich euch,
vielmehr Zwiespalt. \bibleverse{52} Denn von nun an werden fünf, die in
einem Hause wohnen, entzweit sein: drei werden gegen zwei und zwei gegen
drei stehen, \bibleverse{53} der Vater gegen den Sohn und der Sohn gegen
den Vater, die Mutter gegen die Tochter und die Tochter gegen die
Mutter, die Schwiegermutter gegen ihre Schwiegertochter und die
Schwiegertochter gegen die Schwiegermutter.«\textless sup title=``Mi
7,6''\textgreater✲

\hypertarget{b-die-zeichen-der-zeit-mahnen-zum-ernst-auf-rettung-der-seele-bedacht-zu-sein}{%
\paragraph{b) Die Zeichen der Zeit mahnen zum Ernst, auf Rettung der
Seele bedacht zu
sein}\label{b-die-zeichen-der-zeit-mahnen-zum-ernst-auf-rettung-der-seele-bedacht-zu-sein}}

\bibleverse{54} Dann sagte er auch noch zu der Volksmenge: »Wenn ihr
Gewölk im Westen aufsteigen seht, dann sagt ihr sogleich: ›Es gibt
Regen‹, und es kommt auch so; \bibleverse{55} und wenn ihr den Südwind
wehen seht, so sagt ihr: ›Es wird heiß✲ werden‹, und es kommt auch so.
\bibleverse{56} Ihr Heuchler✲! Das Aussehen der Erde und des Himmels
versteht ihr richtig zu beurteilen; wie kommt es denn, daß ihr die
gegenwärtige Zeit nicht richtig beurteilt? \bibleverse{57} Warum könnt
ihr auch nicht von euch selbst aus zu einem Urteil über das, was recht
ist, gelangen? \bibleverse{58} Denn wenn du mit deinem Widersacher vor
Gericht gehst, so gib dir noch unterwegs Mühe, dich gütlich mit ihm
abzufinden, damit er dich nicht etwa vor den Richter schleppt und der
Richter dich dem Gerichtsdiener übergibt und der Gerichtsdiener dich ins
Gefängnis wirft. \bibleverse{59} Ich sage dir: du wirst von dort
sicherlich nicht loskommen, bis du auch den letzten Heller bezahlt
hast.«\textless sup title=``vgl. Mt 5,25-26''\textgreater✲

\hypertarget{buuxdfmahnungen-an-die-fuxfcr-das-gericht-reifen-volksgenossen-gleichnis-vom-unfruchtbaren-feigenbaum}{%
\subsubsection{15. Bußmahnungen an die für das Gericht reifen
Volksgenossen; Gleichnis vom unfruchtbaren
Feigenbaum}\label{buuxdfmahnungen-an-die-fuxfcr-das-gericht-reifen-volksgenossen-gleichnis-vom-unfruchtbaren-feigenbaum}}

\hypertarget{section-12}{%
\section{13}\label{section-12}}

\bibleverse{1} Es waren aber zu eben dieser Zeit einige Leute (bei ihm)
eingetroffen, die ihm von den Galiläern erzählten, deren Blut Pilatus
zusammen mit dem ihrer Schlachtopfer vergossen hatte. \bibleverse{2} Da
antwortete ihnen Jesus: »Meint ihr etwa, diese Galiläer seien größere
Sünder gewesen als alle anderen Galiläer, weil sie dies Schicksal
erlitten haben? \bibleverse{3} Nein, sage ich euch; sondern wenn ihr
euren Sinn nicht ändert, werdet ihr alle ebenso umkommen. \bibleverse{4}
Oder meint ihr, daß jene achtzehn, auf die der Turm am (Teich) Siloah
gestürzt ist und sie erschlagen hat, schuldbeladener gewesen seien als
alle anderen Bewohner von Jerusalem? \bibleverse{5} Nein, sage ich euch;
sondern wenn ihr euren Sinn nicht ändert, werdet ihr alle ebenso
umkommen.«

\bibleverse{6} Er sagte ihnen dann noch folgendes Gleichnis: »Jemand
hatte einen Feigenbaum in seinem Weinberge stehen, und er kam und suchte
Frucht an ihm, fand jedoch keine. \bibleverse{7} Da sagte er zu dem
Weingärtner: ›Sieh, ich komme nun schon drei Jahre her und suche Frucht
an diesem Feigenbaum, finde jedoch keine; haue ihn ab! Wozu soll er noch
den Platz wegnehmen?‹ \bibleverse{8} Da antwortete ihm jener: ›Herr, laß
ihn noch dieses Jahr stehen! Ich will noch einmal das Land um ihn herum
graben und ihn düngen: \bibleverse{9} vielleicht bringt er künftig doch
noch Frucht; andernfalls laß ihn umhauen!‹«

\hypertarget{eine-krankenheilung-am-sabbat-streit-uxfcber-sabbatheiligung}{%
\subsubsection{16. Eine Krankenheilung am Sabbat; Streit über
Sabbatheiligung}\label{eine-krankenheilung-am-sabbat-streit-uxfcber-sabbatheiligung}}

\bibleverse{10} Jesus lehrte einst in einer der Synagogen am Sabbat.
\bibleverse{11} Da war gerade eine Frau anwesend, die schon seit
achtzehn Jahren einen Geist der Schwäche hatte; sie war zusammengekrümmt
und unfähig, sich ordentlich aufzurichten. \bibleverse{12} Als Jesus sie
erblickte, rief er sie herbei und sagte zu ihr: »Frau, du bist von
deiner Schwäche befreit!« \bibleverse{13} Dann legte er ihr die Hände
auf, und sie richtete sich augenblicklich gerade empor und pries Gott.
\bibleverse{14} Weil nun der Vorsteher der Synagoge unwillig darüber
war, daß Jesus am Sabbat eine Heilung vollzogen hatte, sagte er zu der
(versammelten) Menge: »Sechs Tage sind da, an denen man arbeiten soll;
an diesen also kommt und laßt euch heilen, aber nicht (gerade) am
Sabbattage!« \bibleverse{15} Der Herr aber antwortete ihm mit den
Worten: »Ihr Heuchler✲! Bindet nicht ein jeder von euch am Sabbat seinen
Ochsen oder Esel von der Krippe los und führt ihn zur Tränke?
\bibleverse{16} Diese Frau aber, eine Tochter Abrahams, die der Satan
nun schon achtzehn Jahre lang in Fesseln gehalten hat, die sollte von
dieser Fessel am Sabbattage nicht befreit werden dürfen?«
\bibleverse{17} Durch diese seine Worte wurden alle seine Gegner
beschämt; die ganze Volksmenge aber freute sich über alle die herrlichen
Taten, die durch ihn geschahen.

\hypertarget{zwei-himmelreichsgleichnisse-vom-senfkorn-und-vom-sauerteig}{%
\subsubsection{17. Zwei Himmelreichsgleichnisse (vom Senfkorn und vom
Sauerteig)}\label{zwei-himmelreichsgleichnisse-vom-senfkorn-und-vom-sauerteig}}

\bibleverse{18} Dann sagte er: »Wem ist das Reich Gottes gleich, und
womit soll ich es vergleichen? \bibleverse{19} Es ist einem Senfkorn
gleich, das ein Mann nahm und in seinem Garten einlegte; dort wuchs es
und wurde zu einem Baume, und die Vögel des Himmels nisteten in seinen
Zweigen.«

\bibleverse{20} Und weiter sagte er: »Womit soll ich das Reich Gottes
vergleichen? \bibleverse{21} Es ist einem Sauerteig gleich, den eine
Frau nahm und unter drei Scheffel Mehl mengte, bis der ganze Teig
durchsäuert war.«

\hypertarget{jesus-auf-dem-wege-nach-jerusalem}{%
\subsubsection{18. Jesus auf dem Wege nach
Jerusalem}\label{jesus-auf-dem-wege-nach-jerusalem}}

\hypertarget{a-mahnung-zum-eifrigen-und-rechtzeitigen-trachten-nach-dem-gottesreich-die-enge-pforte-wertlosigkeit-einer-blouxdf-uxe4uuxdferlichen-zugehuxf6rigkeit-zu-jesus}{%
\paragraph{a) Mahnung zum eifrigen und rechtzeitigen Trachten nach dem
Gottesreich; die enge Pforte; Wertlosigkeit einer bloß äußerlichen
Zugehörigkeit zu
Jesus}\label{a-mahnung-zum-eifrigen-und-rechtzeitigen-trachten-nach-dem-gottesreich-die-enge-pforte-wertlosigkeit-einer-blouxdf-uxe4uuxdferlichen-zugehuxf6rigkeit-zu-jesus}}

\bibleverse{22} So wanderte er von Stadt zu Stadt und von Dorf zu Dorf,
indem er lehrte und seine Wanderung nach Jerusalem vollzog.
\bibleverse{23} Da fragte ihn jemand: »Herr, es sind wohl nur wenige,
die gerettet werden?« Er antwortete ihnen: \bibleverse{24} »Ringet
danach, durch die enge Pforte\textless sup title=``vgl. Mt
7,13-14''\textgreater✲ einzugehen! Denn viele, sage ich euch, werden
hineinzukommen suchen und es nicht vermögen. \bibleverse{25} Wenn ihr
erst dann, nachdem der Hausherr sich schon erhoben und die Tür
abgeschlossen hat, draußen zu stehen und an die Tür zu klopfen beginnt
und ihm zuruft: ›Herr, mache uns auf!‹, so wird er euch antworten: ›Ich
weiß von euch nicht, woher ihr seid.‹\textless sup title=``vgl. Mt
25,11-12''\textgreater✲ \bibleverse{26} Dann werdet ihr anfangen zu
versichern: ›Wir haben doch vor deinen Augen (mit dir) gegessen und
getrunken, und du hast bei uns auf den Straßen gelehrt‹\textless sup
title=``Mt 7,22-23''\textgreater✲; \bibleverse{27} aber er wird
erwidern: ›Ich sage euch: ich weiß nicht, woher ihr seid; hinweg von mir
alle, die ihr die Ungerechtigkeit übt!‹\textless sup title=``Ps
6,9''\textgreater✲ \bibleverse{28} Dort wird's dann ein lautes Weinen
und Zähneknirschen geben, wenn ihr Abraham, Isaak, Jakob und alle
Propheten im Reiche Gottes sehen werdet, während ihr selbst
hinausgestoßen seid\textless sup title=``Mt 8,11-12''\textgreater✲.
\bibleverse{29} Und sie werden von Osten und Westen, von Norden und
Süden kommen und sich im Reiche Gottes zum Mahl niedersetzen.
\bibleverse{30} Und wisset wohl: Es gibt Letzte, die werden Erste sein,
und es gibt Erste, die werden Letzte sein.«\textless sup title=``Mt
19,30''\textgreater✲

\hypertarget{b-des-herodes-versuch-der-einschuxfcchterung-jesu-vereitelt-bedrohung-jerusalems}{%
\paragraph{b) Des Herodes Versuch der Einschüchterung Jesu vereitelt;
Bedrohung
Jerusalems}\label{b-des-herodes-versuch-der-einschuxfcchterung-jesu-vereitelt-bedrohung-jerusalems}}

\bibleverse{31} In eben dieser Stunde traten einige Pharisäer herzu und
sagten zu ihm: »Entferne dich von hier und ziehe weiter, denn Herodes
will dich töten!« \bibleverse{32} Doch er antwortete ihnen: »Geht hin
und meldet diesem Fuchs: ›Wisse wohl: ich treibe böse Geister aus und
vollführe Heilungen heute noch und morgen, und übermorgen komme ich
damit zum Abschluß. \bibleverse{33} Jedoch heute und morgen und
übermorgen muß ich weiterziehen; denn es geht nicht an, daß ein Prophet
anderswo als in Jerusalem ums Leben kommt.‹ \bibleverse{34} Jerusalem,
Jerusalem, das du die Propheten tötest und die steinigst, die zu dir
gesandt sind! Wie oft habe ich deine Kinder sammeln wollen, wie eine
Henne ihre Küchlein unter ihre Flügel sammelt, doch ihr habt nicht
gewollt! \bibleverse{35} Seht, euer Haus wird euch nun
überlassen\textless sup title=``Jer 22,5''\textgreater✲. Ich sage euch
aber: Ihr werdet mich nicht sehen, bis die Zeit kommt, wo ihr ausruft:
›Gepriesen\textless sup title=``oder: gesegnet''\textgreater✲ sei, der
da kommt im Namen des Herrn!‹«\textless sup title=``Ps
118,26''\textgreater✲

\hypertarget{jesus-beim-pharisuxe4ergastmahl}{%
\subsubsection{19. Jesus beim
Pharisäergastmahl}\label{jesus-beim-pharisuxe4ergastmahl}}

\hypertarget{a-neuer-streit-wegen-einer-krankenheilung-am-sabbat}{%
\paragraph{a) Neuer Streit wegen einer Krankenheilung am
Sabbat}\label{a-neuer-streit-wegen-einer-krankenheilung-am-sabbat}}

\hypertarget{section-13}{%
\section{14}\label{section-13}}

\bibleverse{1} Als er dann an einem Sabbat in das Haus eines der
Obersten\textless sup title=``=~der führenden Männer''\textgreater✲ der
Pharisäer gekommen war, um dort am Mahl teilzunehmen, lauerten sie ihm
auf. \bibleverse{2} Und siehe, ein wassersüchtiger Mann erschien dort
vor ihm. \bibleverse{3} Da richtete Jesus die Frage an die
Gesetzeslehrer und Pharisäer: »Darf man am Sabbat heilen oder nicht?«
\bibleverse{4} Sie aber schwiegen. Da faßte er ihn an, heilte ihn und
hieß ihn gehen. \bibleverse{5} Hierauf sagte er zu ihnen: »Wem von euch
wird sein Sohn oder sein Rind in einen Brunnen fallen, und er wird ihn
nicht sofort auch am Sabbattage herausziehen?« \bibleverse{6} Und sie
vermochten ihm auf diese Frage keine widersprechende Antwort zu geben.

\hypertarget{b-tischgespruxe4che-uxfcber-bescheidenheit-und-rechte-gastfreiheit}{%
\paragraph{b) Tischgespräche (über Bescheidenheit und rechte
Gastfreiheit)}\label{b-tischgespruxe4che-uxfcber-bescheidenheit-und-rechte-gastfreiheit}}

\bibleverse{7} Er legte aber den Gästen ein Gleichnis vor, weil er
beobachtete, wie sie sich die obersten Plätze aussuchten, und sagte zu
ihnen: \bibleverse{8} »Wenn du von jemand zu einem Festmahl\textless sup
title=``vgl. 12,36''\textgreater✲ eingeladen bist, so setze dich nicht
obenan; es könnte sonst jemand, der noch vornehmer ist als du, von ihm
geladen sein, \bibleverse{9} und dann würde der, welcher dich und ihn
geladen hat, kommen und zu dir sagen: ›Tritt diesem da den Platz ab!‹,
und du müßtest dich alsdann dazu verstehen, beschämt untenan zu sitzen.
\bibleverse{10} Nein, wenn du eingeladen bist, so gehe hin und setze
dich untenan; dann wird der Gastgeber kommen und zu dir sagen: ›Freund,
rücke weiter nach oben!‹, dann wirst du in den Augen aller deiner
Tischgenossen geehrt dastehen. \bibleverse{11} Denn wer sich selbst
erhöht, wird erniedrigt werden, und wer sich selbst erniedrigt, wird
erhöht werden.«\textless sup title=``Mt 23,12''\textgreater✲

\bibleverse{12} Er sagte dann auch zu dem, der ihn eingeladen hatte:
»Wenn du ein Mittagsmahl oder ein Abendessen veranstaltest, so lade
nicht deine Freunde und deine Brüder, nicht deine Verwandten und reichen
Nachbarn dazu ein; sonst laden auch sie dich wieder ein, und dir wird
Gleiches mit Gleichem vergolten. \bibleverse{13} Nein, wenn du ein
Gastmahl veranstalten willst, so lade Arme und Krüppel, Lahme und Blinde
dazu ein, \bibleverse{14} dann wirst du glückselig sein, weil sie es dir
nicht vergelten können; denn es wird dir bei der Auferstehung der
Gerechten vergolten werden.«

\hypertarget{c-das-gleichnis-vom-grouxdfen-gastmahl}{%
\paragraph{c) Das Gleichnis vom großen
Gastmahl}\label{c-das-gleichnis-vom-grouxdfen-gastmahl}}

\bibleverse{15} Als einer von den Tischgenossen dies hörte, sagte er zu
ihm: »Glückselig ist, wer am Mahl im Reiche Gottes teilnehmen wird!«
\bibleverse{16} Jesus aber antwortete ihm: »Ein Mann veranstaltete ein
großes Gastmahl und lud viele dazu ein. \bibleverse{17} Er sandte dann
seinen Knecht zur Stunde des Gastmahls aus und ließ den Geladenen sagen,
sie möchten kommen, denn es sei nunmehr alles bereit. \bibleverse{18} Da
begannen alle ohne Ausnahme sich zu entschuldigen✲. Der erste ließ ihm
sagen: ›Ich habe einen Acker gekauft und muß notwendigerweise hingehen,
um ihn zu besichtigen; ich bitte dich: sieh mich als entschuldigt an!‹
\bibleverse{19} Ein anderer sagte: ›Ich habe fünf Joch✲ Ochsen gekauft
und muß hingehen, um sie zu erproben; ich bitte dich: sieh mich als
entschuldigt an!‹ \bibleverse{20} Wieder ein anderer sagte: ›Ich habe
mich verheiratet, kann also nicht kommen.‹ \bibleverse{21} Als nun der
Knecht zurückkam, berichtete er dies seinem Herrn. Da wurde der Hausherr
zornig und gab seinem Knecht die Weisung: ›Gehe schnell hinaus auf die
Straßen und Gassen der Stadt und bringe die Armen und Krüppel, die
Blinden und Lahmen hierher.‹ \bibleverse{22} Der Knecht meldete dann:
›Herr, dein Befehl ist ausgeführt, doch es ist noch Platz vorhanden.‹
\bibleverse{23} Da sagte der Herr zu dem Knecht: ›Gehe auf die
Landstraßen und an die Zäune hinaus und nötige die Leute dort
hereinzukommen, damit mein Haus voll werde! \bibleverse{24} Denn ich
sage euch: Keiner von jenen Männern, die (zuerst) geladen waren, wird
mein Gastmahl zu kosten bekommen.‹«

\hypertarget{von-den-bedingungen-der-juxfcngerschaft-und-vom-ernst-der-nachfolge-jesu}{%
\subsubsection{20. Von den Bedingungen der Jüngerschaft und vom Ernst
der Nachfolge
Jesu}\label{von-den-bedingungen-der-juxfcngerschaft-und-vom-ernst-der-nachfolge-jesu}}

\bibleverse{25} Es zogen aber große Volksscharen mit ihm; da wandte er
sich um und sagte zu ihnen: \bibleverse{26} »Wenn jemand zu mir kommt
und nicht seinen Vater und seine Mutter, sein Weib und seine Kinder,
seine Brüder und seine Schwestern, ja sogar sein eigenes Leben haßt, so
kann er nicht mein Jünger sein. \bibleverse{27} Wer nicht sein Kreuz
trägt und mir nachfolgt, der kann nicht mein Jünger sein.«

\hypertarget{mahnung-zu-ernster-pruxfcfung-und-warnung-vor-uxfcbereilter-nachfolge-der-ausspruch-vom-salz}{%
\paragraph{Mahnung zu ernster Prüfung und Warnung vor übereilter
Nachfolge; der Ausspruch vom
Salz}\label{mahnung-zu-ernster-pruxfcfung-und-warnung-vor-uxfcbereilter-nachfolge-der-ausspruch-vom-salz}}

\bibleverse{28} »Denn wer unter euch, der einen Turm zu bauen
beabsichtigt, setzt sich nicht zuerst hin und berechnet die Kosten, ob
er auch die Mittel zur Ausführung des Planes habe? \bibleverse{29}
Sonst, wenn er den Grund gelegt hat, und er den Bau nicht zu Ende führen
kann, werden alle, die es sehen, anfangen über ihn zu spotten
\bibleverse{30} und werden sagen: ›Dieser Mensch hat den Bau begonnen,
doch ihn nicht zu Ende führen können.‹ \bibleverse{31} Oder welcher
König, der zum Kriege mit einem andern König ausziehen will, setzt sich
nicht zuerst hin und geht mit sich zu Rat, ob er imstande ist, mit
zehntausend Mann dem entgegenzutreten, der mit zwanzigtausend gegen ihn
anrückt? \bibleverse{32} Sonst muß er, solange jener noch weit entfernt
ist, eine Gesandtschaft an ihn schicken und um Friedensverhandlungen
bitten. \bibleverse{33} Ebenso kann keiner von euch mein Jünger sein,
der sich nicht von allem lossagt, was er besitzt.~-- \bibleverse{34} Das
Salz ist also etwas Gutes; wenn aber einmal auch das Salz fade geworden
ist, womit soll man es wieder zu Salz machen? \bibleverse{35} Weder für
den Acker noch für den Düngerhaufen ist es brauchbar: man wirft es eben
aus dem Hause hinaus. Wer Ohren hat zu hören, der höre!«\textless sup
title=``Mt 5,13; Mk 9,50''\textgreater✲

\hypertarget{drei-gleichnisse-von-der-suxfcnderliebe-jesu}{%
\subsubsection{21. Drei Gleichnisse von der Sünderliebe
Jesu}\label{drei-gleichnisse-von-der-suxfcnderliebe-jesu}}

\hypertarget{a-einleitung-gleichnisse-vom-verlorenen-schaf-und-vom-verlorenen-silberstuxfcck}{%
\paragraph{a) Einleitung; Gleichnisse vom verlorenen Schaf und vom
verlorenen
Silberstück}\label{a-einleitung-gleichnisse-vom-verlorenen-schaf-und-vom-verlorenen-silberstuxfcck}}

\hypertarget{section-14}{%
\section{15}\label{section-14}}

\bibleverse{1} Es waren aber gerade die Zöllner und Sünder die, die ihm
nahe zu kommen suchten, um ihn zu hören. \bibleverse{2} Darüber murrten
die Pharisäer und die Schriftgelehrten laut und sagten: »Dieser nimmt
Sünder (in seine Umgebung) auf und ißt mit ihnen.« \bibleverse{3} Da
antwortete ihnen Jesus durch folgendes Gleichnis: \bibleverse{4} »Wo ist
jemand unter euch, der hundert Schafe besitzt und, wenn ihm eins von
ihnen verloren geht, nicht die neunundneunzig in der Einöde zurückläßt
und dem verlorenen nachgeht, bis er es findet? \bibleverse{5} Wenn er es
dann gefunden hat, nimmt er es voller Freude auf seine Schultern
\bibleverse{6} und ruft, wenn er nach Hause gekommen ist, seine Freunde
und Nachbarn zusammen und sagt zu ihnen: ›Freuet euch mit mir! Denn ich
habe mein Schaf wiedergefunden, das verloren gegangen war.‹
\bibleverse{7} Ich sage euch: Ebenso wird im Himmel über einen einzigen
Sünder, der sich bekehrt\textless sup title=``vgl. Mt
3,2''\textgreater✲, mehr Freude herrschen als über neunundneunzig
Gerechte, die der Bekehrung nicht bedürfen.

\bibleverse{8} Oder wo ist eine Frau, die zehn Drachmen✲ besitzt und,
wenn sie eine von ihnen verliert, nicht ein Licht anzündet und das Haus
fegt und eifrig sucht, bis sie (das Geldstück) findet? \bibleverse{9}
Wenn sie es dann gefunden hat, ruft sie ihre Freundinnen und
Nachbarinnen zusammen und sagt: ›Freuet euch mit mir, denn ich habe die
Drachme wiedergefunden, die ich verloren hatte.‹ \bibleverse{10} Ebenso,
sage ich euch, herrscht Freude bei den Engeln Gottes über einen einzigen
Sünder, der sich bekehrt.«

\hypertarget{b-das-gleichnis-vom-verlorenen-sohn-bzw.-von-den-zwei-verlorenen-suxf6hnen}{%
\paragraph{b) Das Gleichnis vom verlorenen Sohn (bzw. von den zwei
verlorenen
Söhnen)}\label{b-das-gleichnis-vom-verlorenen-sohn-bzw.-von-den-zwei-verlorenen-suxf6hnen}}

\bibleverse{11} Dann fuhr er fort: »Ein Mann hatte zwei Söhne.
\bibleverse{12} Der jüngere von ihnen sagte zum Vater: ›Vater, gib mir
den auf mich entfallenden Teil des Vermögens!‹ Da verteilte jener das
Hab und Gut unter sie. \bibleverse{13} Kurze Zeit darauf packte der
jüngere Sohn alles, was ihm gehörte, zusammen und zog in ein fernes
Land; dort brachte er sein Vermögen in einem ausschweifenden Leben
durch. \bibleverse{14} Als er nun alles aufgebraucht hatte, entstand
eine schwere Hungersnot in jenem Lande, und auch er begann Not zu
leiden. \bibleverse{15} Da ging er hin und stellte sich einem der Bürger
jenes Landes zur Verfügung; der schickte ihn auf seine Felder, die
Schweine zu hüten, \bibleverse{16} und er hätte sich gern an den Schoten
des Johannesbrotbaumes satt gegessen, welche die Schweine als Futter
bekamen, doch niemand gab sie ihm. \bibleverse{17} Da ging er in sich
und sagte: ›Wie viele Tagelöhner meines Vaters haben Brot im Überfluß,
während ich hier vor Hunger umkomme! \bibleverse{18} Ich will mich
aufmachen und zu meinem Vater gehen und zu ihm sagen: Vater, ich habe
gegen den Himmel✲ und dir gegenüber gesündigt; \bibleverse{19} ich bin
nicht mehr wert, dein Sohn zu heißen: halte mich wie einen von deinen
Tagelöhnern.‹ \bibleverse{20} So machte er sich denn auf den Weg zu
seinem Vater. Als er aber noch weit entfernt war, sah ihn sein Vater
kommen und fühlte Mitleid: er eilte (ihm entgegen), fiel ihm um den Hals
und küßte ihn. \bibleverse{21} Da sagte der Sohn zu ihm: ›Vater, ich
habe gegen den Himmel und dir gegenüber gesündigt; ich bin nicht mehr
wert, dein Sohn zu heißen!‹ \bibleverse{22} Der Vater aber befahl seinen
Knechten: ›Holt schnell das beste Gewand aus dem Hause und legt es ihm
an; gebt ihm auch einen Ring an seine Hand und Schuhe an seine Füße
\bibleverse{23} und bringt das gemästete Kalb her, schlachtet es und
laßt uns essen und fröhlich sein! \bibleverse{24} Denn dieser mein Sohn
war tot und ist wieder lebendig geworden, er war verloren und ist
wiedergefunden!‹ Und sie fingen an, fröhlich zu sein.

\bibleverse{25} Sein älterer Sohn aber war währenddessen auf dem Felde.
Als er nun heimkehrte und sich dem Hause näherte, hörte er Musik und
Reigenchöre. \bibleverse{26} Da rief er einen von den Knechten herbei
und erkundigte sich, was das zu bedeuten habe. \bibleverse{27} Der gab
ihm zur Antwort: ›Dein Bruder ist heimgekommen; da hat dein Vater das
gemästete Kalb schlachten lassen, weil er ihn gesund wiedererhalten
hat.‹ \bibleverse{28} Da wurde er zornig und wollte nicht ins Haus
hineingehen; sein Vater aber kam heraus und redete ihm gut zu.
\bibleverse{29} Da antwortete er dem Vater: ›Du weißt: schon so viele
Jahre diene ich dir und habe noch nie ein Gebot von dir übertreten; doch
mir hast du noch nie auch nur ein Böcklein gegeben, daß ich mit meinen
Freunden ein fröhliches Mahl hätte halten können. \bibleverse{30} Nun
aber dieser dein Sohn heimgekehrt ist, der dein Vermögen mit Dirnen
durchgebracht hat, da hast du ihm das Mastkalb schlachten lassen!‹
\bibleverse{31} Er aber erwiderte ihm: ›Mein Sohn, du bist allezeit bei
mir, und alles, was mein ist, ist auch dein. \bibleverse{32} Wir mußten
doch fröhlich sein und uns freuen! Denn dieser dein Bruder war tot und
ist wieder lebendig geworden, er war verloren gegangen und ist
wiedergefunden worden.‹«

\hypertarget{vermahnung-der-juxfcnger}{%
\subsubsection{22. Vermahnung der
Jünger}\label{vermahnung-der-juxfcnger}}

\hypertarget{a-gleichnis-vom-untreuen-aber-klugen-verwalter}{%
\paragraph{a) Gleichnis vom untreuen, aber klugen
Verwalter}\label{a-gleichnis-vom-untreuen-aber-klugen-verwalter}}

\hypertarget{section-15}{%
\section{16}\label{section-15}}

\bibleverse{1} Er sagte dann noch zu seinen Jüngern: »Es war ein reicher
Mann, der einen Verwalter hatte; über diesen wurde ihm hinterbracht, daß
er ihm sein Vermögen veruntreue. \bibleverse{2} Da ließ er ihn rufen und
sagte zu ihm: ›Was muß ich da über dich hören? Lege Rechnung ab über
deine Verwaltung, denn du kannst nicht länger mein Verwalter sein!‹
\bibleverse{3} Da überlegte der Verwalter bei sich: ›Was soll ich tun,
da mein Herr mir die Verwaltung abnimmt? Zum Graben\textless sup
title=``d.h. zu harter Handarbeit''\textgreater✲ bin ich zu schwach, und
zu betteln schäme ich mich. \bibleverse{4} Nun, ich weiß schon, was ich
tun will, damit die Leute mich, wenn ich meines Amtes enthoben bin, in
ihre Häuser aufnehmen.‹ \bibleverse{5} Er ließ also die Schuldner seines
Herrn alle einzeln zu sich kommen und fragte den ersten: ›Wieviel bist
du meinem Herrn schuldig?‹ \bibleverse{6} Der antwortete: ›Hundert
Tonnen Öl.‹ Da sagte er zu ihm: ›Nimm hier deinen
Pachtvertrag\textless sup title=``oder: Schuldschein''\textgreater✲,
setze dich hin und schreibe schnell fünfzig!‹ \bibleverse{7} Darauf
fragte er einen andern: ›Du aber, wieviel bist du schuldig?‹ Der
antwortete: ›Hundert Zentner Weizen.‹ Er sagte zu ihm: ›Nimm hier deinen
Pachtvertrag\textless sup title=``oder: Schuldschein''\textgreater✲ und
schreibe achtzig.‹« \bibleverse{8} Und der Herr lobte den unehrlichen
Verwalter, daß er klug gehandelt habe; denn -- sagte er -- »die Kinder
dieser Weltzeit sind im Verkehr mit ihresgleichen\textless sup
title=``=~ihren Mitmenschen''\textgreater✲ klüger als die Kinder des
Lichts. \bibleverse{9} Auch ich sage euch: Macht euch Freunde mit dem
ungerechten Mammon\textless sup title=``=~Reichtum; vgl. Mt
6,24''\textgreater✲, damit, wenn er euch ausgeht, ihr Aufnahme in den
ewigen Hütten findet.«

\hypertarget{b-vom-wert-der-irdischen-und-der-himmlischen-guxfcter-und-von-der-treue}{%
\paragraph{b) Vom Wert der irdischen und der himmlischen Güter und von
der
Treue}\label{b-vom-wert-der-irdischen-und-der-himmlischen-guxfcter-und-von-der-treue}}

\bibleverse{10} »Wer im Kleinsten treu ist, der ist auch im Großen treu,
und wer im Kleinsten ungerecht\textless sup title=``oder:
unredlich''\textgreater✲ ist, der ist auch im Großen
ungerecht\textless sup title=``oder: unredlich''\textgreater✲.
\bibleverse{11} Wenn ihr euch nun in der Verwaltung des ungerechten
Mammons nicht treu erwiesen habt, wer wird euch da das wahre Gut
anvertrauen? \bibleverse{12} Und wenn ihr euch am fremden Gut nicht treu
erwiesen habt, wer wird euch da euer eigenes geben? \bibleverse{13} Kein
Knecht kann zwei Herren (zugleich) dienen; denn entweder wird er den
einen hassen und den andern lieben, oder er wird dem einen anhangen und
den andern mißachten. Ihr könnt nicht Gott dienen und (zugleich) dem
Mammon.«\textless sup title=``Mt 6,24''\textgreater✲

\hypertarget{verwarnung-der-pharisuxe4er}{%
\subsubsection{23. Verwarnung der
Pharisäer}\label{verwarnung-der-pharisuxe4er}}

\hypertarget{a-jesus-weist-die-habgierigen-und-spottenden-pharisuxe4er-zurecht}{%
\paragraph{a) Jesus weist die habgierigen und spottenden Pharisäer
zurecht}\label{a-jesus-weist-die-habgierigen-und-spottenden-pharisuxe4er-zurecht}}

\bibleverse{14} Dies alles hörten aber die Pharisäer, die geldgierig
waren, und rümpften die Nase über ihn. \bibleverse{15} Da sagte er zu
ihnen: »Ihr seid die Leute, die sich selbst vor den Menschen als gerecht
hinstellen, Gott aber kennt eure Herzen; denn was vor den Menschen hoch
dasteht, ist ein Greuel vor Gott. \bibleverse{16} Das Gesetz und die
Propheten (reichen) bis auf Johannes; von da an wird das Reich Gottes
durch die Heilsbotschaft verkündigt, und ein jeder drängt sich mit
Gewalt hinein\textless sup title=``Mt 11,12-13''\textgreater✲.
\bibleverse{17} Es ist aber eher möglich, daß Himmel und Erde vergehen,
als daß vom Gesetz ein einziges Strichlein hinfällig✲ wird\textless sup
title=``Mt 5,18''\textgreater✲. \bibleverse{18} Wer seine Frau
entläßt\textless sup title=``oder: sich von seiner Frau
scheidet''\textgreater✲ und eine andere heiratet, begeht Ehebruch, und
wer eine von ihrem Gatten entlassene\textless sup title=``oder:
geschiedene''\textgreater✲ Frau heiratet, begeht auch
Ehebruch.«\textless sup title=``Mt 5,32; 19,9''\textgreater✲

\hypertarget{b-erzuxe4hlung-vom-reichen-mann-und-vom-armen-lazarus}{%
\paragraph{b) Erzählung vom reichen Mann und vom armen
Lazarus}\label{b-erzuxe4hlung-vom-reichen-mann-und-vom-armen-lazarus}}

\bibleverse{19} »Es war aber ein reicher Mann, der kleidete sich in
Purpur und kostbare Leinwand und lebte alle Tage herrlich und in
Freuden. \bibleverse{20} Ein Armer aber namens Lazarus lag vor seiner
Türhalle; der war mit Geschwüren bedeckt \bibleverse{21} und hatte nur
den Wunsch, sich von den Abfällen vom Tisch des Reichen zu sättigen;
aber\textless sup title=``oder: ja''\textgreater✲ es kamen sogar die
Hunde herbei und beleckten seine Geschwüre. \bibleverse{22} Nun begab es
sich, daß der Arme starb und von den Engeln in Abrahams
Schoß\textless sup title=``=~an die Brust Abrahams''\textgreater✲
getragen wurde; auch der Reiche starb und wurde begraben.
\bibleverse{23} Als dieser nun im Totenreich, wo er Qualen litt, seine
Augen aufschlug, erblickte er Abraham in der Ferne und Lazarus in seinem
Schoß\textless sup title=``=~an seiner Brust''\textgreater✲.
\bibleverse{24} Da rief er mit lauter Stimme: ›Vater Abraham! Erbarme
dich meiner und sende Lazarus, damit er seine Fingerspitze ins Wasser
tauche und mir die Zunge kühle! Denn ich leide Qualen in dieser
Feuerglut.‹ \bibleverse{25} Aber Abraham antwortete: ›Mein Sohn, denke
daran, daß du dein Gutes während deines Erdenlebens empfangen hast, und
Lazarus gleicherweise das Üble; jetzt aber wird er hier getröstet,
während du Qualen leiden mußt. \bibleverse{26} Und zu alledem ist
zwischen uns und euch eine große Kluft festgelegt, damit die, welche von
hier zu euch hinübergehen wollen, es nicht können und man auch von dort
nicht zu uns herüberkommen kann.‹ \bibleverse{27} Da erwiderte er: ›So
bitte ich dich denn, Vater: sende ihn in meines Vaters Haus~--
\bibleverse{28} denn ich habe noch fünf Brüder --, damit er sie
ernstlich warne, damit sie nicht auch an diesen Ort der Qual kommen.‹
\bibleverse{29} Abraham aber antwortete: ›Sie haben Mose und die
Propheten; auf diese mögen sie hören!‹ \bibleverse{30} Jener jedoch
entgegnete: ›Nein, Vater Abraham! Sondern wenn einer von den Toten zu
ihnen kommt, dann werden sie sich bekehren.‹ \bibleverse{31} Abraham
aber antwortete ihm: ›Wenn sie nicht auf Mose und die Propheten hören,
so werden sie sich auch nicht überzeugen lassen, wenn einer von den
Toten aufersteht.‹«

\hypertarget{gespruxe4che-jesu-mit-den-juxfcngern}{%
\subsubsection{24. Gespräche Jesu mit den
Jüngern}\label{gespruxe4che-jesu-mit-den-juxfcngern}}

\hypertarget{a-warnung-vor-verfuxfchrungen-zum-unglauben-und-zur-suxfcnde-und-mahnung-zum-vergeben}{%
\paragraph{a) Warnung vor Verführungen (zum Unglauben und zur Sünde) und
Mahnung zum
Vergeben}\label{a-warnung-vor-verfuxfchrungen-zum-unglauben-und-zur-suxfcnde-und-mahnung-zum-vergeben}}

\hypertarget{section-16}{%
\section{17}\label{section-16}}

\bibleverse{1} Weiter sagte Jesus zu seinen Jüngern: »Es kann nicht
anders sein, als daß Ärgernisse\textless sup title=``oder:
Verführungen''\textgreater✲ kommen; wehe aber dem, durch den sie kommen!
\bibleverse{2} Es wäre besser für ihn, wenn ihm ein Mühlstein um den
Hals gelegt und er ins Meer geworfen wäre, als daß er für einen von
diesen geringen Leuten zum Ärgernis wird\textless sup title=``oder: ihn
zum Bösen verführt''\textgreater✲.~-- \bibleverse{3} Gebt auf euch
selbst acht! Wenn dein Bruder sich (gegen dich) vergangen hat, so halte
es ihm vor; und wenn er es bereut, so vergib ihm. \bibleverse{4} Selbst
wenn er sich siebenmal am Tage gegen dich vergeht und siebenmal wieder
zu dir kommt und erklärt: ›Es tut mir leid!‹, so sollst du ihm
vergeben.«

\hypertarget{b-von-der-kraft-des-glaubens}{%
\paragraph{b) Von der Kraft des
Glaubens}\label{b-von-der-kraft-des-glaubens}}

\bibleverse{5} Die Apostel baten alsdann den Herrn: »Mehre uns den
Glauben!« \bibleverse{6} Da antwortete der Herr: »Wenn ihr Glauben wie
ein Senfkorn hättet und ihr diesem Maulbeerbaum gebötet: ›Entwurzle dich
und verpflanze dich ins Meer!‹, so würde er euch gehorsam sein.«

\hypertarget{c-gleichnis-vom-herrn-und-seinem-zur-arbeit-verpflichteten-knecht-gegen-die-lohnsucht}{%
\paragraph{c) Gleichnis vom Herrn und seinem zur Arbeit verpflichteten
Knecht (Gegen die
Lohnsucht)}\label{c-gleichnis-vom-herrn-und-seinem-zur-arbeit-verpflichteten-knecht-gegen-die-lohnsucht}}

\bibleverse{7} »Wer von euch aber, der einen Knecht beim Pflügen oder
beim Viehhüten hat, wird zu ihm bei seiner Heimkehr vom Felde sagen:
›Komm sogleich her und setze dich zu Tisch!‹? \bibleverse{8} Wird er
nicht vielmehr zu ihm sagen: ›Bereite mir mein Abendessen, schürze dich
und bediene mich, bis ich gegessen und getrunken habe; nachher magst
auch du essen und trinken‹? \bibleverse{9} Er wird doch wohl dem Knecht
nicht noch dankbar dafür sein, daß er die ihm erteilten Befehle
ausgeführt hat? \bibleverse{10} Ebenso steht's auch bei euch: Wenn ihr
alles getan habt, was euch befohlen war, so sagt: ›Wir sind
armselige\textless sup title=``oder: geringe''\textgreater✲ Knechte; wir
haben nur unsere Schuldigkeit getan.‹«

\hypertarget{heilung-von-zehn-aussuxe4tzigen-der-dankbare-samariter}{%
\subsubsection{25. Heilung von zehn Aussätzigen; der dankbare
Samariter}\label{heilung-von-zehn-aussuxe4tzigen-der-dankbare-samariter}}

\bibleverse{11} Auf seiner Wanderung nach Jerusalem durchzog Jesus das
Grenzgebiet von Samaria und Galiläa. \bibleverse{12} Als er dort in ein
Dorf eintrat, kamen ihm zehn aussätzige Männer entgegen, die in der
Ferne\textless sup title=``d.h. weit abseits vom Wege''\textgreater✲
stehen blieben \bibleverse{13} und ihre Stimme erhoben und riefen:
»Jesus, (lieber) Meister, erbarme dich unser!« \bibleverse{14} Als er
sie erblickte, sagte er zu ihnen: »Geht hin und zeigt euch den
Priestern.« Während sie dann hingingen, wurden sie rein. \bibleverse{15}
Einer von ihnen aber, als er sich geheilt sah, kehrte zurück, pries Gott
mit lauter Stimme, \bibleverse{16} warf sich zu Jesu Füßen auf sein
Angesicht nieder und dankte ihm; und das war ein Samariter.
\bibleverse{17} Da sagte Jesus: »Sind ihrer nicht zehn rein geworden? Wo
sind denn die anderen neun? \bibleverse{18} Hat sich sonst keiner
gefunden, der zurückgekehrt ist, um Gott die Ehre zu geben, außer diesem
Fremdling?« \bibleverse{19} Zu ihm sagte er dann: »Stehe auf und gehe!
Dein Glaube hat dir Rettung✲ verschafft.«

\hypertarget{rede-vom-kommen-des-gottesreiches-und-von-der-erscheinung-des-menschensohnes-an-seinem-tage}{%
\subsubsection{26. Rede vom Kommen des Gottesreiches und von der
Erscheinung des Menschensohnes an seinem
Tage}\label{rede-vom-kommen-des-gottesreiches-und-von-der-erscheinung-des-menschensohnes-an-seinem-tage}}

\bibleverse{20} Als er aber von den Pharisäern aufs neue gefragt wurde,
wann das Reich Gottes käme, gab er ihnen zur Antwort: »Das Reich Gottes
kommt nicht mit äußerlichem Gebaren\textless sup title=``=~unter
augenfälligen Erscheinungen''\textgreater✲; \bibleverse{21} man wird
auch nicht sagen können: ›Siehe, hier ist es!‹ oder ›dort ist es!‹ Denn
wisset wohl: Das Reich Gottes ist (bereits) mitten unter euch.«
\bibleverse{22} Weiter sagte er zu seinen Jüngern: »Es werden Tage
kommen, wo ihr euch danach sehnen werdet, einen einzigen von den Tagen
des Menschensohnes zu sehen, doch ihr werdet ihn nicht sehen.
\bibleverse{23} Und wird man dann zu euch sagen: ›Seht dort! Seht
hier!‹, so geht nicht hin und gebt nichts darauf! \bibleverse{24} Denn
wie der Blitz, wenn er aufblitzt, am Himmel hin von einem Ende bis zum
andern leuchtet, so wird es auch mit dem Menschensohn an seinem Tage
sein. \bibleverse{25} Zuerst✲ muß er aber noch vieles leiden und von
diesem Geschlecht verworfen werden. \bibleverse{26} Und wie es in den
Tagen Noahs zugegangen ist, so wird es auch in den Tagen des
Menschensohnes sein: \bibleverse{27} Man aß und trank, man heiratete und
wurde verheiratet bis zu dem Tage, an welchem Noah in die Arche
ging\textless sup title=``1.Mose 7,7''\textgreater✲ und die Sintflut kam
und allen den Untergang brachte. \bibleverse{28} Ebenso wie es in den
Tagen Lots zugegangen ist: Man aß und trank, man kaufte und verkaufte,
man pflanzte und baute; \bibleverse{29} aber an dem Tage, an welchem Lot
aus Sodom wegging, regnete es Feuer und Schwefel vom Himmel und
vernichtete alle~-- \bibleverse{30} ebenso wird es auch an dem Tage
sein, an welchem der Menschensohn sich offenbart. \bibleverse{31} Wer an
diesem Tage auf dem Dache ist, während seine Geräte sich im Hause
befinden, der steige nicht erst noch hinab, um sie zu holen; und ebenso,
wer auf dem Felde ist, kehre nicht zurück! \bibleverse{32} Denkt an Lots
Frau! \bibleverse{33} Wer sein Leben zu erhalten sucht, der wird es
verlieren, und wer es verliert, dem wird es erhalten bleiben.
\bibleverse{34} Ich sage euch: In der betreffenden Nacht werden zwei
(Männer) auf einem Lager liegen: der eine wird angenommen\textless sup
title=``oder: mitgenommen''\textgreater✲, der andere zurückgelassen
werden; \bibleverse{35} zwei (Frauen) werden an derselben Handmühle
mahlen: die eine wird angenommen\textless sup title=``oder:
mitgenommen''\textgreater✲, die andere zurückgelassen werden.«
\bibleverse{37} Da erwiderten ihm die Jünger mit der Frage: »Herr, wo
denn?« Er aber antwortete ihnen: »Wo das Aas\textless sup title=``=~ein
verendetes Tier''\textgreater✲ liegt, da sammeln sich auch die
Geier.«\textless sup title=``Mt 24,28''\textgreater✲

\hypertarget{gleichnis-vom-gottlosen-richter-und-der-bittenden-witwe-von-der-kraft-des-anhaltenden-gebets}{%
\subsubsection{27. Gleichnis vom gottlosen Richter und der bittenden
Witwe (von der Kraft des anhaltenden
Gebets)}\label{gleichnis-vom-gottlosen-richter-und-der-bittenden-witwe-von-der-kraft-des-anhaltenden-gebets}}

\hypertarget{section-17}{%
\section{18}\label{section-17}}

\bibleverse{1} Er legte ihnen dann ein Gleichnis vor, um sie darauf
hinzuweisen, daß man allezeit beten müsse und nicht müde darin werden
dürfe. \bibleverse{2} »In einer Stadt«, so sagte er, »lebte ein Richter,
der Gott nicht fürchtete und auf keinen Menschen Rücksicht nahm.
\bibleverse{3} Nun wohnte in jener Stadt eine Witwe, die (immer wieder)
zu ihm kam mit dem Anliegen: ›Schaffe mir Recht gegen meinen
Widersacher!‹ \bibleverse{4} Lange Zeit wollte er nicht; schließlich
aber dachte er bei sich: ›Wenn ich auch Gott nicht fürchte und auf
keinen Menschen Rücksicht nehme, \bibleverse{5} will ich dieser Witwe
doch zu ihrem Recht verhelfen, weil sie mir lästig fällt; sonst kommt
sie schließlich noch und wird handgreiflich gegen mich.‹« \bibleverse{6}
Dann fuhr der Herr fort: »Hört, was (hier) der ungerechte Richter sagt!
\bibleverse{7} Sollte nun Gott nicht auch seinen Auserwählten Recht
schaffen, die Tag und Nacht zu ihm rufen, auch wenn er Langmut bei ihnen
übt? \bibleverse{8} Ich sage euch: Er wird ihnen gar bald ihr Recht
schaffen! Doch wird wohl der Menschensohn bei seinem Kommen den Glauben
auf Erden vorfinden?«

\hypertarget{das-gleichnis-vom-pharisuxe4er-und-zuxf6llner}{%
\subsubsection{28. Das Gleichnis vom Pharisäer und
Zöllner}\label{das-gleichnis-vom-pharisuxe4er-und-zuxf6llner}}

\bibleverse{9} Er legte dann auch einigen, die von ihrer eigenen
Gerechtigkeit überzeugt waren und auf die anderen mit Geringschätzung
herabsahen, folgendes Gleichnis vor: \bibleverse{10} »Zwei Männer gingen
in den Tempel hinauf, um zu beten, der eine ein Pharisäer, der andere
ein Zöllner. \bibleverse{11} Der Pharisäer trat hin und betete bei
sich\textless sup title=``oder: mit Bezug auf sich''\textgreater✲ so: ›O
Gott, ich danke dir, daß ich nicht bin wie die anderen Menschen, Räuber,
Betrüger, Ehebrecher oder auch wie der Zöllner dort. \bibleverse{12} Ich
faste zweimal in der Woche und gebe den Zehnten von allem, was ich
erwerbe.‹ \bibleverse{13} Der Zöllner dagegen stand von ferne und mochte
nicht einmal die Augen zum Himmel erheben, sondern schlug sich an die
Brust und sagte: ›Gott, sei mir Sünder gnädig!‹ \bibleverse{14} Ich sage
euch: Dieser ging gerechtfertigt in sein Haus hinab, ganz anders, als es
bei jenem der Fall war! Denn wer sich selbst erhöht, wird erniedrigt
werden; wer sich aber selbst erniedrigt, wird erhöht
werden.«\textless sup title=``Lk 14,11; Mt 23,12''\textgreater✲

\hypertarget{jesus-segnet-die-kinder}{%
\subsubsection{29. Jesus segnet die
Kinder}\label{jesus-segnet-die-kinder}}

\bibleverse{15} Die Leute brachten aber auch ihre kleinen Kinder zu ihm,
damit er sie anrühre; als die Jünger das sahen, verwiesen sie es ihnen
in barscher Weise. \bibleverse{16} Jesus aber rief sie\textless sup
title=``d.h. die Kinder''\textgreater✲ zu sich heran und sagte: »Laßt
die Kinder zu mir kommen und hindert sie nicht daran, \bibleverse{17}
denn für ihresgleichen ist das Reich Gottes bestimmt\textless sup
title=``vgl. Mk 10,14''\textgreater✲. Wahrlich ich sage euch: Wer das
Reich Gottes nicht wie ein Kind annimmt, wird sicherlich nicht
hineinkommen.«

\hypertarget{jesu-gespruxe4ch-mit-dem-reichen-vorsteher-von-der-gefahr-des-reichtums}{%
\subsubsection{30. Jesu Gespräch mit dem reichen Vorsteher; von der
Gefahr des
Reichtums}\label{jesu-gespruxe4ch-mit-dem-reichen-vorsteher-von-der-gefahr-des-reichtums}}

\bibleverse{18} Hierauf richtete ein Oberster\textless sup title=``oder:
Vorsteher''\textgreater✲ die Frage an ihn: »Guter Meister, was muß ich
tun, um ewiges Leben zu ererben✲?« \bibleverse{19} Jesus antwortete ihm:
»Was nennst du mich gut? Niemand ist gut als Gott allein.
\bibleverse{20} Du kennst die Gebote: Du sollst nicht ehebrechen, nicht
töten, nicht stehlen, nicht falsches Zeugnis ablegen, ehre deinen Vater
und deine Mutter!« \bibleverse{21} Darauf erwiderte jener: »Dies alles
habe ich von Jugend an gehalten.« \bibleverse{22} Als Jesus das hörte,
sagte er zu ihm: »Eins fehlt dir noch: Verkaufe alles, was du besitzest,
und verteile den Erlös an die Armen, so wirst du einen Schatz im Himmel
haben; dann komm und folge mir nach.« \bibleverse{23} Als jener das
hörte, wurde er tief betrübt; denn er war sehr reich. \bibleverse{24}
Als Jesus ihn so sah, sagte er: »Wie schwer ist es doch für die
Begüterten, in das Reich Gottes einzugehen! \bibleverse{25} Ja, es ist
leichter\textless sup title=``=~eher möglich''\textgreater✲, daß ein
Kamel durch ein Nadelöhr hindurchgeht, als daß ein Reicher in das Reich
Gottes eingeht.« \bibleverse{26} Da sagten die Zuhörer: »Ja, wer kann
dann gerettet werden?« \bibleverse{27} Jesus aber antwortete: »Was bei
Menschen unmöglich ist, das ist bei Gott möglich.«

\hypertarget{vom-lohn-der-entsagung-bzw.-der-nachfolge-jesu}{%
\subsubsection{31. Vom Lohn der Entsagung (bzw. der Nachfolge
Jesu)}\label{vom-lohn-der-entsagung-bzw.-der-nachfolge-jesu}}

\bibleverse{28} Darauf sagte Petrus: »Siehe, wir haben alles Unsrige
verlassen und sind dir nachgefolgt.« \bibleverse{29} Da sagte Jesus zu
ihnen: »Wahrlich ich sage euch: Niemand hat Haus oder Weib, Geschwister,
Eltern oder Kinder um des Reiches Gottes willen verlassen,
\bibleverse{30} der nicht vielmal Wertvolleres wiederempfinge (schon) in
dieser Zeitlichkeit, und in der zukünftigen Weltzeit ewiges Leben.«

\hypertarget{aufbruch-nach-jerusalem-dritte-leidensankuxfcndigung-jesu}{%
\subsubsection{32. Aufbruch nach Jerusalem; dritte Leidensankündigung
Jesu}\label{aufbruch-nach-jerusalem-dritte-leidensankuxfcndigung-jesu}}

\bibleverse{31} Er nahm dann die Zwölf zu sich (abseits) und sagte zu
ihnen: »Wir ziehen jetzt nach Jerusalem hinauf, und es wird alles in
Erfüllung gehen, was durch die Propheten von dem Menschensohn
geschrieben ist. \bibleverse{32} Denn er wird den Heiden überliefert und
verspottet, mißhandelt und angespien werden, \bibleverse{33} und sie
werden ihn geißeln und töten, und am dritten Tage wird er auferstehen.«
\bibleverse{34} Doch sie verstanden nichts hiervon, sondern dieser
Ausspruch war ihnen dunkel, und sie begriffen nicht, was er mit diesem
Wort hatte sagen wollen.

\hypertarget{heilung-des-blinden-bei-jericho}{%
\subsubsection{33. Heilung des Blinden bei
Jericho}\label{heilung-des-blinden-bei-jericho}}

\bibleverse{35} Als er dann in die Nähe von Jericho kam, saß da ein
Blinder am Wege und bettelte. \bibleverse{36} Als dieser nun die vielen
Leute vorüberziehen hörte, erkundigte er sich, was das zu bedeuten habe.
\bibleverse{37} Man teilte ihm mit, daß Jesus von Nazareth vorübergehe.
\bibleverse{38} Da rief er laut: »Jesus, Sohn Davids, erbarme dich
meiner!« \bibleverse{39} Die an der Spitze des Zuges Gehenden riefen ihm
drohend zu, er solle still sein; doch er rief nur noch lauter: »Sohn
Davids, erbarme dich meiner!« \bibleverse{40} Da blieb Jesus stehen und
ließ ihn zu sich führen. Als er nun nahe herangekommen war, fragte Jesus
ihn: \bibleverse{41} »Was wünschest du von mir?« Er antwortete: »Herr,
ich möchte sehen können.« \bibleverse{42} Jesus erwiderte ihm: »Werde
sehend! Dein Glaube hat dir Rettung verschafft.« \bibleverse{43} Da
konnte er augenblicklich sehen und schloß sich ihm an, indem er Gott
pries; auch das gesamte Volk, das zugesehen hatte, gab Gott die Ehre
durch Lobpreis.

\hypertarget{jesu-einkehr-beim-oberzuxf6llner-zachuxe4us-in-jericho}{%
\subsubsection{34. Jesu Einkehr beim Oberzöllner Zachäus in
Jericho}\label{jesu-einkehr-beim-oberzuxf6llner-zachuxe4us-in-jericho}}

\hypertarget{section-18}{%
\section{19}\label{section-18}}

\bibleverse{1} Jesus kam dann nach Jericho hinein und zog durch die
Stadt hindurch. \bibleverse{2} Dort wohnte aber ein Mann namens Zachäus,
der war ein Oberzöllner und ein reicher Mann. \bibleverse{3} Er hätte
Jesus gern von Person gesehen, konnte es aber wegen der Volksmenge
nicht, weil er klein von Gestalt war. \bibleverse{4} So eilte er denn
auf dem Wege voraus und stieg auf einen Maulbeerfeigenbaum hinauf, um
ihn zu sehen; denn dort mußte er vorbeikommen. \bibleverse{5} Als nun
Jesus an die Stelle kam, blickte er in die Höhe und rief ihm zu:
»Zachäus! Steige schnell herunter; denn ich muß heute in deinem Hause
einkehren.« \bibleverse{6} Da stieg er schnell herab und nahm ihn mit
Freuden bei sich auf. \bibleverse{7} Und alle, die es sahen, murrten
laut und sagten: »Bei einem sündigen Manne ist er eingekehrt, um bei ihm
zu herbergen.« \bibleverse{8} Zachäus aber trat zum Herrn und sagte:
»Siehe, Herr, die Hälfte meines Vermögens will ich den Armen geben, und
wenn ich jemand in etwas übervorteilt habe, will ich es ihm vierfach
ersetzen!« \bibleverse{9} Da sagte Jesus zu ihm: »Heute ist diesem Hause
Heil widerfahren, weil ja auch er ein Sohn Abrahams ist. \bibleverse{10}
Denn der Menschensohn ist gekommen, das Verlorene zu suchen und zu
retten.«

\hypertarget{das-gleichnis-von-den-anvertrauten-geldern-minen-oder-pfunden}{%
\subsubsection{35. Das Gleichnis von den anvertrauten Geldern (Minen
oder
Pfunden)}\label{das-gleichnis-von-den-anvertrauten-geldern-minen-oder-pfunden}}

\bibleverse{11} Als sie dies hörten, fügte er noch ein Gleichnis hinzu,
weil er sich in der Nähe von Jerusalem befand und weil sie meinten, das
Reich Gottes würde jetzt sofort erscheinen. \bibleverse{12} Er sagte
also: »Ein Mann von vornehmer Abkunft reiste in ein fernes Land, um für
sich dort die Königswürde zu gewinnen und dann wieder heimzukehren.
\bibleverse{13} Er berief nun zehn seiner Knechte, gab ihnen zehn
Minen\textless sup title=``oder: Pfunde''\textgreater✲ und sagte zu
ihnen: ›Macht Geschäfte (mit dem Gelde) in der Zeit, während ich
verreist bin!‹ \bibleverse{14} Seine Mitbürger aber haßten ihn und
schickten eine Abordnung hinter ihm her, durch die sie erklären ließen:
›Wir wollen diesen Mann nicht als König über uns haben!‹ \bibleverse{15}
Als er nun nach Empfang der Königswürde heimkehrte, ließ er jene
Knechte, denen er das Geld gegeben hatte, zu sich rufen, um zu erfahren,
was für Geschäfte ein jeder gemacht hätte. \bibleverse{16} Da erschien
der erste und sagte: ›Herr, dein Pfund hat zehn weitere Pfunde
eingebracht.‹ \bibleverse{17} Der Herr antwortete ihm: ›Schön, du guter
Knecht! Weil du im Kleinen\textless sup title=``=~über
Wenigem''\textgreater✲ treu gewesen bist, sollst du die Verwaltung von
zehn Städten erhalten.‹ \bibleverse{18} Dann kam der zweite und sagte:
›Herr, dein Pfund hat fünf Pfunde hinzugewonnen.‹ \bibleverse{19} Er
sagte auch zu diesem: ›Auch du sollst über fünf Städte gesetzt sein!‹
\bibleverse{20} Hierauf kam der dritte und sagte: ›Herr, hier ist dein
Pfund, das ich in einem Schweißtuch wohlverwahrt gehalten habe;
\bibleverse{21} denn ich hatte Furcht vor dir, weil du ein strenger Mann
bist: du hebst ab, was du nicht eingelegt hast, und erntest, was du
nicht gesät hast.‹ \bibleverse{22} Da antwortete er ihm: ›Nach deiner
eigenen Aussage will ich dir das Urteil sprechen, du nichtswürdiger
Knecht! Du wußtest, daß ich ein strenger Mann bin, daß ich abhebe, was
ich nicht eingelegt habe, und ernte, was ich nicht gesät habe?
\bibleverse{23} Warum hast du da mein Geld nicht auf eine Bank gebracht?
Dann hätte ich es bei meiner Rückkehr mit Zinsen abgehoben.‹
\bibleverse{24} Darauf befahl er den Dabeistehenden: ›Nehmt ihm das
Pfund weg und gebt es dem, der die zehn Pfund hat.‹ \bibleverse{25} Sie
erwiderten ihm: ›Herr, er hat ja schon zehn Pfunde.‹ \bibleverse{26} Ich
sage euch: Jedem, der da hat, wird (noch dazu) gegeben werden; wer aber
nicht hat, dem wird auch das genommen werden, was er hat.
\bibleverse{27} Doch jene meine Feinde, die mich nicht zum König über
sich gewollt haben, führt hierher und macht sie vor meinen Augen
nieder!«

\hypertarget{v.-jesu-einzug-in-jerusalem-und-letztes-wirken-1928-2138}{%
\subsection{V. Jesu Einzug in Jerusalem und letztes Wirken
(19,28-21,38)}\label{v.-jesu-einzug-in-jerusalem-und-letztes-wirken-1928-2138}}

\hypertarget{jesus-vor-den-toren-jerusalems-sein-einzug-in-jerusalem}{%
\subsubsection{1. Jesus vor den Toren Jerusalems; sein Einzug in
Jerusalem}\label{jesus-vor-den-toren-jerusalems-sein-einzug-in-jerusalem}}

\bibleverse{28} Nach diesen Worten zog Jesus weiter auf dem Wege nach
Jerusalem hinauf. \bibleverse{29} Als er nun in die Nähe von Bethphage
und Bethanien am sogenannten Ölberge gekommen war, sandte er zwei von
seinen Jüngern ab \bibleverse{30} mit der Weisung: »Geht in das Dorf,
das dort vor euch liegt! Wenn ihr hineinkommt, werdet ihr ein Eselfüllen
angebunden finden, auf dem noch nie ein Mensch gesessen hat: bindet es
los und führt es her! \bibleverse{31} Und wenn euch jemand fragen
sollte: ›Warum bindet ihr es los?‹, so antwortet ihm: ›Der Herr hat es
nötig.‹« \bibleverse{32} Als nun die Abgesandten hingegangen waren,
fanden sie es so, wie er ihnen gesagt hatte; \bibleverse{33} und als sie
das Füllen losbanden, sagten dessen Eigentümer zu ihnen: »Wozu bindet
ihr das Füllen los?« \bibleverse{34} Sie antworteten: »Der Herr hat es
nötig.« \bibleverse{35} Sie führten es darauf zu Jesus, legten ihre
Mäntel auf das Füllen und ließen Jesus sich daraufsetzen.
\bibleverse{36} Während er dann weiterzog, breiteten sie ihre Mäntel auf
den Weg aus. \bibleverse{37} Als er nunmehr an den Abstieg vom Ölberg
herankam, begann die gesamte Menge der Jünger freudig Gott mit lauter
Stimme um all der Wundertaten willen, die sie gesehen hatten, Lobpreis
darzubringen, \bibleverse{38} indem sie ausriefen:
»Gepriesen\textless sup title=``oder: gesegnet''\textgreater✲ sei, der
da kommt als König im Namen des Herrn!\textless sup title=``Ps
118,26''\textgreater✲ Im Himmel ist Friede\textless sup title=``oder:
Heil''\textgreater✲ und Ehre\textless sup title=``oder:
Herrlichkeit''\textgreater✲ in Himmelshöhen!«\textless sup title=``vgl.
2,14''\textgreater✲ \bibleverse{39} Da sagten einige Pharisäer aus der
Volksmenge zu ihm: »Meister, untersage das deinen Jüngern✲!«
\bibleverse{40} Doch er gab zur Antwort: »Ich sage euch: Wenn diese
schwiegen, würden die Steine schreien!«

\hypertarget{jesu-truxe4nen-uxfcber-jerusalem-und-weissagung-von-jerusalems-zerstuxf6rung}{%
\paragraph{Jesu Tränen über Jerusalem und Weissagung von Jerusalems
Zerstörung}\label{jesu-truxe4nen-uxfcber-jerusalem-und-weissagung-von-jerusalems-zerstuxf6rung}}

\bibleverse{41} Als er dann nähergekommen war und die Stadt erblickte,
weinte er über sie \bibleverse{42} und sagte: »Wenn doch auch du an
diesem Tage erkennen möchtest, was zu deinem Frieden dient! Nun aber ist
es deinen Augen verborgen geblieben. \bibleverse{43} Denn es werden Tage
über dich kommen, da werden deine Feinde einen Wall gegen dich
aufführen, dich ringsum einschließen und dich von allen Seiten
bedrängen; \bibleverse{44} sie werden dich und deine Kinder✲ in dir dem
Erdboden gleichmachen\textless sup title=``Ps 137,9''\textgreater✲ und
keinen Stein in dir auf dem andern lassen zur Strafe dafür, daß du die
Zeit deiner (gnadenreichen) Heimsuchung nicht erkannt hast.«

\hypertarget{jesu-tempelreinigung-mordplan-der-fuxfchrer-des-volkes}{%
\subsubsection{2. Jesu Tempelreinigung; Mordplan der Führer des
Volkes}\label{jesu-tempelreinigung-mordplan-der-fuxfchrer-des-volkes}}

\bibleverse{45} Als er sich darauf in den Tempel begeben hatte, machte
er sich daran, die Verkäufer hinauszutreiben, \bibleverse{46} indem er
ihnen zurief: »Es steht geschrieben\textless sup title=``Jes
56,7''\textgreater✲: ›Mein Haus soll ein Bethaus sein!‹ Ihr aber habt es
zu einer ›Räuberhöhle‹ gemacht.«\textless sup title=``Jer
7,11''\textgreater✲

\bibleverse{47} Er lehrte dann täglich im Tempel. Die Hohenpriester aber
und die Schriftgelehrten samt den Obersten\textless sup title=``oder:
Vornehmsten''\textgreater✲ des Volkes trachteten ihm nach dem Leben,
\bibleverse{48} fanden jedoch keine Möglichkeit, ihre Absicht gegen ihn
auszuführen; denn das gesamte Volk hing an ihm\textless sup
title=``=~fühlte sich zu ihm hingezogen''\textgreater✲, sooft es ihn
hörte.

\hypertarget{die-frage-des-hohen-rates-nach-jesu-vollmacht}{%
\subsubsection{3. Die Frage des Hohen Rates nach Jesu
Vollmacht}\label{die-frage-des-hohen-rates-nach-jesu-vollmacht}}

\hypertarget{section-19}{%
\section{20}\label{section-19}}

\bibleverse{1} Eines Tages nun, als er das Volk im Tempel lehrte und die
Heilsbotschaft verkündigte, traten die Hohenpriester und
Schriftgelehrten samt den Ältesten an ihn heran \bibleverse{2} und
sagten zu ihm: »Sage uns, aufgrund welcher Vollmacht du hier in dieser
Weise auftrittst oder wer es ist, der dir die Vollmacht\textless sup
title=``=~das Recht''\textgreater✲ dazu gegeben hat?« \bibleverse{3} Da
antwortete er ihnen: »Auch ich will euch eine Frage vorlegen; sagt mir:
\bibleverse{4} Stammte die Taufe des Johannes vom Himmel oder von
Menschen?« \bibleverse{5} Da überlegten sie bei sich\textless sup
title=``oder: berieten miteinander''\textgreater✲ folgendermaßen: »Sagen
wir: ›Vom Himmel‹, so wird er fragen: ›Warum habt ihr ihm dann keinen
Glauben geschenkt?‹ \bibleverse{6} Sagen wir dagegen: ›Von Menschen‹, so
wird das ganze Volk uns steinigen; denn es ist überzeugt, daß Johannes
ein Prophet (gewesen) ist.« \bibleverse{7} So gaben sie ihm denn zur
Antwort, sie wüßten nicht, woher sie stamme. \bibleverse{8} Da sagte
Jesus zu ihnen: »Dann sage auch ich euch nicht, aufgrund welcher
Vollmacht ich hier so auftrete.«

\hypertarget{das-gleichnis-von-den-treulosen-weinguxe4rtnern}{%
\subsubsection{4. Das Gleichnis von den treulosen
Weingärtnern}\label{das-gleichnis-von-den-treulosen-weinguxe4rtnern}}

\bibleverse{9} Er begann dann dem Volk folgendes Gleichnis vorzutragen:
»Ein Mann legte einen Weinberg an, verpachtete ihn an Weingärtner und
ging dann für längere Zeit ins Ausland. \bibleverse{10} Als nun die Zeit
da war, sandte er einen Knecht zu den Weingärtnern, damit sie ihm
(seinen Teil) vom Ertrag des Weinbergs abgäben; aber die Weingärtner
mißhandelten diesen und schickten ihn mit leeren Händen zurück.
\bibleverse{11} Da sandte er nochmals einen andern Knecht; sie aber
mißhandelten und beschimpften auch diesen und schickten ihn mit leeren
Händen zurück. \bibleverse{12} Er sandte darauf noch einen dritten; sie
aber schlugen auch diesen blutig und warfen ihn hinaus. \bibleverse{13}
Da sagte\textless sup title=``oder: dachte''\textgreater✲ der Herr des
Weinbergs: ›Was soll ich tun? Ich will meinen geliebten Sohn hinsenden;
vor diesem werden sie sich doch wohl scheuen.‹ \bibleverse{14} Als die
Weingärtner ihn aber erblickten, überlegten sie miteinander und sagten:
›Dies ist der Erbe! Wir wollen ihn töten: dann fällt das Erbgut uns zu.‹
\bibleverse{15} So stießen sie ihn denn aus dem Weinberge hinaus und
schlugen ihn tot. Was wird nun der Herr des Weinbergs mit ihnen machen?
\bibleverse{16} Er wird kommen und diese Weingärtner ums Leben bringen
und den Weinberg an andere vergeben.« Als sie das hörten, sagten sie:
»Nimmermehr!« \bibleverse{17} Jesus aber blickte sie an und sagte: »Was
bedeutet denn dieses Schriftwort\textless sup title=``Ps
118,22''\textgreater✲: ›Der Stein, den die Bauleute (als unbrauchbar)
verworfen haben, der ist zum Eckstein geworden‹? \bibleverse{18} Jeder,
der an diesem Steine zu Fall kommt, wird zerschmettert werden; auf wen
aber der Stein fällt, den wird er zermalmen.«\textless sup title=``vgl.
Jes 8,14-15; Dan 2,34.44''\textgreater✲ \bibleverse{19} Da suchten die
Schriftgelehrten und Hohenpriester ihn noch in derselben Stunde
festzunehmen, fürchteten sich jedoch vor dem Volk; sie hatten nämlich
wohl gemerkt, daß er dieses Gleichnis gegen sie gerichtet hatte.

\hypertarget{die-steuerfrage-der-pharisuxe4er-oder-das-gespruxe4ch-vom-zinsgroschen}{%
\subsubsection{5. Die Steuerfrage der Pharisäer (oder das Gespräch vom
Zinsgroschen)}\label{die-steuerfrage-der-pharisuxe4er-oder-das-gespruxe4ch-vom-zinsgroschen}}

\bibleverse{20} So lauerten sie ihm denn auf und sandten Aufpasser ab,
die sich das Aussehen gesetzesstrenger Leute geben sollten, damit sie
ihn durch einen seiner Aussprüche fingen und ihn dann der Obrigkeit und
der Gewalt des Statthalters überliefern könnten. \bibleverse{21} Die
fragten ihn also: »Meister, wir wissen, daß du offen\textless sup
title=``oder: aufrichtig''\textgreater✲ redest und lehrst und die Person
nicht ansiehst, sondern den Weg Gottes mit Wahrhaftigkeit lehrst:
\bibleverse{22} ist es recht, daß wir dem Kaiser Steuern entrichten,
oder nicht?« \bibleverse{23} Da er nun ihre böse Absicht durchschaute,
sagte er zu ihnen: \bibleverse{24} »Zeigt mir einen Denar! Wessen Bild
und Aufschrift trägt er?« Sie antworteten: »Des Kaisers.«
\bibleverse{25} Da sagte er zu ihnen: »Nun, so gebt dem Kaiser, was dem
Kaiser zukommt, und Gott, was Gott zukommt.« \bibleverse{26} Und sie
vermochten ihn nicht bei einem Ausspruch vor dem Volk zu fangen und
wußten, voll Verwunderung über seine Antwort, nichts mehr zu sagen.

\hypertarget{die-sadduzuxe4erfrage-uxfcber-die-auferstehung-der-toten}{%
\subsubsection{6. Die Sadduzäerfrage (über die Auferstehung der
Toten)}\label{die-sadduzuxe4erfrage-uxfcber-die-auferstehung-der-toten}}

\bibleverse{27} Hierauf traten einige Sadduzäer herzu, die da behaupten,
es gebe keine Auferstehung, und legten ihm eine Frage vor
\bibleverse{28} mit den Worten: »Meister, Mose hat uns
vorgeschrieben\textless sup title=``5.Mose 25,5''\textgreater✲: ›Wenn
jemandem sein Bruder stirbt, der eine Frau hat, jedoch kinderlos
geblieben ist, so soll sein Bruder die Frau ehelichen und für seinen
Bruder das Geschlecht fortpflanzen.‹ \bibleverse{29} Nun waren da sieben
Brüder. Der erste✲ nahm eine Frau und starb kinderlos; \bibleverse{30}
der zweite heiratete sie darauf, \bibleverse{31} dann der dritte und in
derselben Weise alle sieben, hinterließen aber keine Kinder und starben;
\bibleverse{32} zuletzt starb auch die Frau. \bibleverse{33} Wem von
ihnen wird diese nun bei der Auferstehung als Frau angehören? Alle
sieben haben sie ja zur Frau gehabt.« \bibleverse{34} Da sagte Jesus zu
ihnen: »Die Kinder✲ der jetzigen Weltzeit heiraten und werden
verheiratet; \bibleverse{35} diejenigen aber, welche würdig befunden
worden sind, an jener Weltzeit und an der Auferstehung der
Toten\textless sup title=``=~aus der Totenwelt''\textgreater✲
teilzunehmen, die heiraten weder noch werden sie verheiratet;
\bibleverse{36} sie können dann ja auch nicht mehr sterben, denn sie
sind den Engeln gleich und sind Söhne✲ Gottes, weil sie Söhne der
Auferstehung sind. \bibleverse{37} Daß aber die Toten auferweckt werden,
das hat auch Mose bei (der Erzählung von) dem Dornbusch erkennen
lassen\textless sup title=``2.Mose 3,6''\textgreater✲, indem er dort den
Herrn ›den Gott Abrahams, den Gott Isaaks und den Gott Jakobs‹ nennt.
\bibleverse{38} Gott ist doch nicht ein Gott von Toten, sondern von
Lebenden, denn alle leben ihm\textless sup title=``oder: für
ihn''\textgreater✲.« \bibleverse{39} Da antworteten mehrere
Schriftgelehrte: »Meister, du hast trefflich gesprochen!«
\bibleverse{40} Sie wagten auch hinfort nicht mehr, ihm eine Frage
vorzulegen.

\hypertarget{die-gegenfrage-jesu-nach-dem-messias-als-dem-sohne-davids}{%
\subsubsection{7. Die Gegenfrage Jesu nach dem Messias als dem Sohne
Davids}\label{die-gegenfrage-jesu-nach-dem-messias-als-dem-sohne-davids}}

\bibleverse{41} Er sagte dann aber zu ihnen: »Wie kann man behaupten,
Christus\textless sup title=``oder: der Messias''\textgreater✲ sei
Davids Sohn? \bibleverse{42} David selbst sagt ja doch im
Psalmbuch\textless sup title=``Ps 110,1''\textgreater✲: ›Der Herr hat zu
meinem Herrn gesagt: Setze dich zu meiner Rechten, \bibleverse{43} bis
ich deine Feinde hinlege zum Schemel deiner Füße.‹ \bibleverse{44} David
nennt ihn\textless sup title=``d.h. den Messias''\textgreater✲ also
›Herr‹; wie kann er da sein Sohn sein?«

\hypertarget{jesu-warnung-vor-dem-ehrgeiz-und-der-habgier-der-schriftgelehrten}{%
\subsubsection{8. Jesu Warnung vor dem Ehrgeiz und der Habgier der
Schriftgelehrten}\label{jesu-warnung-vor-dem-ehrgeiz-und-der-habgier-der-schriftgelehrten}}

\bibleverse{45} Zu seinen Jüngern aber sagte er, während das ganze Volk
zuhörte: \bibleverse{46} »Hütet euch vor den Schriftgelehrten, die es
lieben, in langen Gewändern einherzugehen, und sich auf den öffentlichen
Plätzen gern begrüßen lassen; die auf die vordersten Sitze in den
Synagogen und auf die obersten Plätze bei den Gastmählern Anspruch
machen; \bibleverse{47} die die Häuser der Witwen
verschlingen\textless sup title=``=~gierig an sich
bringen''\textgreater✲ und zum Schein lange Gebete verrichten; diese
werden ein besonders strenges Gericht erfahren.«

\hypertarget{jesu-lob-der-zwei-scherflein-der-armen-witwe}{%
\subsubsection{9. Jesu Lob der zwei Scherflein der armen
Witwe}\label{jesu-lob-der-zwei-scherflein-der-armen-witwe}}

\hypertarget{section-20}{%
\section{21}\label{section-20}}

\bibleverse{1} Als er dann aufblickte\textless sup title=``=~Umschau
hielt''\textgreater✲, sah er, wie die Reichen ihre Gaben in den
Opferkasten einlegten. \bibleverse{2} Da sah er auch eine arme Witwe
dort zwei Scherflein\textless sup title=``vgl. Mk 12,42''\textgreater✲
hineintun \bibleverse{3} und sagte: »Wahrlich ich sage euch: Diese arme
Witwe hat mehr als alle anderen eingelegt; \bibleverse{4} denn jene
haben alle aus ihrem Überfluß eine Gabe in den Gotteskasten getan, sie
aber hat aus ihrer Dürftigkeit alles eingelegt, was sie zum
Lebensunterhalt besaß.«

\hypertarget{jesu-rede-uxf6lbergrede-an-die-apostel-von-der-zerstuxf6rung-des-tempels-und-jerusalems-vom-ende-dieser-weltzeit-und-von-seiner-erscheinung-am-juxfcngsten-tage}{%
\subsubsection{10. Jesu Rede (Ölbergrede) an die Apostel von der
Zerstörung des Tempels und Jerusalems, vom Ende dieser Weltzeit und von
seiner Erscheinung am jüngsten
Tage}\label{jesu-rede-uxf6lbergrede-an-die-apostel-von-der-zerstuxf6rung-des-tempels-und-jerusalems-vom-ende-dieser-weltzeit-und-von-seiner-erscheinung-am-juxfcngsten-tage}}

\hypertarget{a-einleitung-anlauxdf-der-rede}{%
\paragraph{a) Einleitung: Anlaß der
Rede}\label{a-einleitung-anlauxdf-der-rede}}

\bibleverse{5} Als einige dann vom Tempel sagten, er sei (ein Prachtbau)
mit herrlichen Steinen und Weihgeschenken geschmückt, antwortete er:
\bibleverse{6} »Was ihr da anschaut -- es werden Tage kommen, an denen
kein Stein auf dem andern liegen bleibt, der nicht niedergerissen wird.«
\bibleverse{7} Da richteten sie die Frage an ihn: »Meister, wann wird
dies denn geschehen, und welches ist das Anzeichen dafür, wann dies
eintreten wird?«

\hypertarget{b-die-ersten-vorzeichen-des-endes}{%
\paragraph{b) Die ersten Vorzeichen des
Endes}\label{b-die-ersten-vorzeichen-des-endes}}

\bibleverse{8} Da antwortete er: »Seht zu, daß ihr nicht irregeführt
werdet! Denn viele werden unter meinem Namen kommen und sagen: ›Ich bin
es\textless sup title=``d.h. Christus, oder: der
Messias''\textgreater✲‹, und ›Die Zeit ist nahe!‹ Lauft ihnen nicht
nach! \bibleverse{9} Wenn ihr ferner von Kriegen und Aufständen hört, so
laßt euch dadurch nicht erschrecken! Denn das muß zuerst kommen, aber
das Ende ist dann noch nicht sogleich da.« \bibleverse{10} Hierauf fuhr
er fort: »Ein Volk wird sich gegen das andere erheben und ein Reich
gegen das andere\textless sup title=``Jes 19,2''\textgreater✲;
\bibleverse{11} auch gewaltige Erdbeben werden stattfinden und hier und
da Hungersnöte und Seuchen; auch schreckhafte Erscheinungen und große
Zeichen vom Himmel her werden erfolgen.«

\hypertarget{c-die-juxfcngerverfolgungen}{%
\paragraph{c) Die
Jüngerverfolgungen}\label{c-die-juxfcngerverfolgungen}}

\bibleverse{12} »Aber ehe alles dies geschieht, wird man Hand an euch
legen und euch verfolgen, indem man euch an die Synagogen und
Gefängnisse überantwortet und euch vor Könige und Statthalter führt um
meines Namens willen. \bibleverse{13} Da wird euch dann Gelegenheit
geboten werden, Zeugnis (für mich) abzulegen. \bibleverse{14} So
beherzigt denn (die Warnung) wohl, daß ihr euch nicht im voraus Sorge
über die Art eurer Verteidigung machet; \bibleverse{15} denn ich selbst
werde euch Redegabe und Weisheit verleihen, der alle eure Widersacher
nicht zu widerstehen noch zu widersprechen imstande sein sollen.
\bibleverse{16} Ihr werdet aber sogar von Eltern und Geschwistern, von
Verwandten und Freunden überantwortet werden, ja man wird manche von
euch töten, \bibleverse{17} und ihr werdet allen um meines Namens willen
verhaßt sein. \bibleverse{18} Doch es soll kein Haar von eurem Haupte
verlorengehen: \bibleverse{19} durch standhaftes Ausharren werdet ihr
euch das Leben gewinnen.«

\hypertarget{d-die-zerstuxf6rung-jerusalems-und-die-not-des-juxfcdischen-volkes}{%
\paragraph{d) Die Zerstörung Jerusalems und die Not des jüdischen
Volkes}\label{d-die-zerstuxf6rung-jerusalems-und-die-not-des-juxfcdischen-volkes}}

\bibleverse{20} »Wenn ihr aber Jerusalem von Kriegsheeren umlagert seht,
dann erkennet daran, daß seine Zerstörung nahe bevorsteht.
\bibleverse{21} Dann sollen die (Gläubigen) in Judäa ins Gebirge fliehen
und die Bewohner (der Hauptstadt) auswandern und die auf dem Lande
Wohnenden nicht in die Stadt hineinziehen; \bibleverse{22} denn dies
sind die Tage der Vergeltung\textless sup title=``5.Mose
32,35''\textgreater✲, damit alles in Erfüllung gehe, was in der Schrift
steht. \bibleverse{23} Wehe den Frauen, die in jenen Tagen guter
Hoffnung sind, und den Müttern, die ein Kind zu nähren haben! Denn große
Not wird im Lande herrschen und ein Zorngericht über dieses Volk
ergehen; \bibleverse{24} und sie werden durch die Schärfe des Schwertes
fallen und in die Gefangenschaft unter alle Heidenvölker weggeführt
werden, und Jerusalem wird von Heiden zertreten werden\textless sup
title=``Sach 12,3''\textgreater✲, bis die Zeiten der Heiden abgelaufen
sind.«

\hypertarget{e-die-letzten-vorzeichen-des-endes-und-die-erscheinung-des-menschensohnes-am-juxfcngsten-tage}{%
\paragraph{e) Die letzten Vorzeichen des Endes und die Erscheinung des
Menschensohnes (am Jüngsten
Tage)}\label{e-die-letzten-vorzeichen-des-endes-und-die-erscheinung-des-menschensohnes-am-juxfcngsten-tage}}

\bibleverse{25} »Dann werden Zeichen an Sonne, Mond und Sternen in
Erscheinung treten und auf der Erde wird Verzweiflung der Völker in
ratloser Angst beim Brausen des Meeres und seines Wogenschwalls
herrschen, \bibleverse{26} indem Menschen den Geist aufgeben vor Furcht
und in banger Erwartung der Dinge, die über den Erdkreis kommen werden;
denn (sogar) die Kräfte des Himmels werden in Erschütterung
geraten\textless sup title=``Jes 34,4''\textgreater✲. \bibleverse{27}
Und hierauf wird man den Menschensohn in\textless sup title=``oder:
auf''\textgreater✲ einer Wolke kommen sehen mit großer Macht und
Herrlichkeit\textless sup title=``Dan 7,13''\textgreater✲.
\bibleverse{28} Wenn dies nun zu geschehen beginnt, dann richtet euch
auf und hebt eure Häupter empor; denn eure Erlösung naht.«

\bibleverse{29} Er sagte ihnen dann noch ein Gleichnis: »Seht den
Feigenbaum und alle anderen Bäume an: \bibleverse{30} sobald sie
ausschlagen, erkennt ihr, wenn ihr es seht, von selbst, daß nunmehr der
Sommer nahe ist. \bibleverse{31} So sollt auch ihr, wenn ihr alles
dieses eintreten seht, erkennen, daß das Reich Gottes nahe ist.
\bibleverse{32} Wahrlich ich sage euch: Dieses\textless sup title=``d.h.
das gegenwärtige''\textgreater✲ Geschlecht wird nicht vergehen, bis
alles geschieht. \bibleverse{33} Himmel und Erde werden vergehen, aber
meine Worte werden nimmermehr vergehen!«

\hypertarget{f-schluuxdfermahnung-zur-nuxfcchternheit-und-wachsamkeit}{%
\paragraph{f) Schlußermahnung zur Nüchternheit und
Wachsamkeit}\label{f-schluuxdfermahnung-zur-nuxfcchternheit-und-wachsamkeit}}

\bibleverse{34} »Habt aber auf euch selbst acht, daß eure Herzen nicht
etwa durch Schlemmerei und Trunkenheit und Sorgen des Lebens beschwert
werden und jener Tag euch unvermutet überfalle wie eine Schlinge;
\bibleverse{35} denn hereinbrechen wird er über alle Bewohner der ganzen
Erde. \bibleverse{36} Seid also allezeit wachsam und betet darum, daß
ihr die Kraft empfanget, diesem allem, was da kommen soll, zu entrinnen
und vor den Menschensohn hinzutreten!«

\hypertarget{abschluuxdf-des-berichts-von-dem-aufenthalt-und-dem-uxf6ffentlichen-wirken-jesu-in-jerusalem}{%
\subsubsection{11. Abschluß des Berichts von dem Aufenthalt und dem
öffentlichen Wirken Jesu in
Jerusalem}\label{abschluuxdf-des-berichts-von-dem-aufenthalt-und-dem-uxf6ffentlichen-wirken-jesu-in-jerusalem}}

\bibleverse{37} Tagsüber war Jesus im Tempel, wo er lehrte; an jedem
Abend aber ging er (aus der Stadt) hinaus und übernachtete am
sogenannten Ölberg; \bibleverse{38} und das ganze Volk kam schon
frühmorgens zu ihm, um ihm im Tempel zuzuhören.

\hypertarget{vi.-jesu-leidensgeschichte-und-die-auferstehungsberichte-kap.-22-24}{%
\subsection{VI. Jesu Leidensgeschichte und die Auferstehungsberichte
(Kap.
22-24)}\label{vi.-jesu-leidensgeschichte-und-die-auferstehungsberichte-kap.-22-24}}

\hypertarget{mordanschlag-der-fuxfchrer-des-volks}{%
\subsubsection{1. Mordanschlag der Führer des
Volks}\label{mordanschlag-der-fuxfchrer-des-volks}}

\hypertarget{section-21}{%
\section{22}\label{section-21}}

\bibleverse{1} So kam denn das Fest der ungesäuerten Brote, das
sogenannte Passah, heran; \bibleverse{2} und die Hohenpriester und
Schriftgelehrten suchten Mittel und Wege, wie sie ihn
beseitigen\textless sup title=``oder: umbringen''\textgreater✲ könnten;
denn sie fürchteten sich vor dem Volke.

\hypertarget{verrat-des-judas}{%
\subsubsection{2. Verrat des Judas}\label{verrat-des-judas}}

\bibleverse{3} Da fuhr der Satan in Judas, der den Beinamen Iskariot
führte und zur Zahl der Zwölf gehörte: \bibleverse{4} er ging hin und
verabredete mit den Hohenpriestern und den Hauptleuten der Tempelwache,
wie er ihnen Jesus in die Hände liefern wollte\textless sup
title=``oder: könnte''\textgreater✲. \bibleverse{5} Darüber freuten sie
sich und kamen mit ihm überein, ihm Geld zu geben; \bibleverse{6} er war
einverstanden und suchte nun nach einer guten Gelegenheit, um ihnen
Jesus hinter dem Rücken des Volkes in die Hände zu liefern.

\hypertarget{vorbereitung-des-ostermahles}{%
\subsubsection{3. Vorbereitung des
Ostermahles}\label{vorbereitung-des-ostermahles}}

\bibleverse{7} Als dann der Tag der ungesäuerten Brote gekommen war, an
dem man das Passahlamm schlachten mußte, \bibleverse{8} sandte er Petrus
und Johannes ab mit der Weisung: »Geht hin und richtet uns das
Passahmahl zu, damit wir es essen können!« \bibleverse{9} Auf ihre
Frage: »Wo sollen wir es zurichten?« \bibleverse{10} antwortete er
ihnen: »Gebt acht: sobald ihr in die Stadt hineinkommt, wird euch ein
Mann begegnen, der einen Krug mit Wasser trägt; folgt ihm in das Haus,
in das er hineingeht, \bibleverse{11} und sagt dem Eigentümer des
Hauses: ›Der Meister läßt dich fragen: Wo ist der Speisesaal, in welchem
ich das Passahlamm mit meinen Jüngern essen kann?‹ \bibleverse{12} Dann
wird er euch ein geräumiges, mit Tischpolstern ausgestattetes Obergemach
zeigen: dort richtet das Mahl zu!« \bibleverse{13} Sie gingen hin und
fanden es so, wie er ihnen gesagt hatte, und richteten das Passahmahl
zu.

\hypertarget{jesu-letztes-mahl-im-juxfcngerkreise-einsetzung-des-heiligen-abendmahls}{%
\subsubsection{4. Jesu letztes Mahl im Jüngerkreise; Einsetzung des
heiligen
Abendmahls}\label{jesu-letztes-mahl-im-juxfcngerkreise-einsetzung-des-heiligen-abendmahls}}

\bibleverse{14} Als dann die Stunde gekommen war, setzte er sich zu
Tisch und die Apostel mit ihm. \bibleverse{15} Da sagte er zu ihnen:
»Herzlich habe ich mich danach gesehnt, dieses Passahmahl vor meinem
Leiden noch mit euch zu essen; \bibleverse{16} denn ich sage euch: ich
werde es nicht mehr essen, bis es im Reiche Gottes seine
Vollendung\textless sup title=``oder: volle Erfüllung''\textgreater✲
findet.« \bibleverse{17} Dann nahm er einen Becher, sprach das Dankgebet
und sagte: »Nehmt diesen (Becher) und teilt ihn unter euch!
\bibleverse{18} Denn ich sage euch: Ich werde von nun an von dem
Erzeugnis des Weinstocks nicht mehr trinken, bis das Reich Gottes
kommt.« \bibleverse{19} Dann nahm er Brot, sprach den Lobpreis (Gottes),
brach das Brot und gab es ihnen mit den Worten: »Dies ist mein Leib
{[}der für euch dahingegeben wird; das tut zu meinem Gedächtnis!«
\bibleverse{20} Ebenso tat er mit dem Becher nach dem Mahl und sagte:
»Dieser Kelch ist der neue Bund in meinem Blut, das für euch vergossen
wird{]}. \bibleverse{21} Doch wisset wohl: Die Hand meines Verräters ist
mit mir zusammen\textless sup title=``=~neben mir''\textgreater✲ auf dem
Tische. \bibleverse{22} Denn der Menschensohn geht zwar dahin, wie es
bestimmt ist; doch wehe dem Menschen, durch den er verraten wird!«
\bibleverse{23} Da fingen sie an, sich untereinander zu besprechen, wer
von ihnen es wohl sein möchte, der dies tun würde.

\hypertarget{worte-des-abschieds-an-die-juxfcnger}{%
\subsubsection{5. Worte des Abschieds an die
Jünger}\label{worte-des-abschieds-an-die-juxfcnger}}

\hypertarget{a-jesus-tadelt-den-ehrgeiz-der-juxfcnger-erkennt-aber-ihr-treues-ausharren-bei-ihm-lohnverheiuxdfend-an}{%
\paragraph{a) Jesus tadelt den Ehrgeiz der Jünger, erkennt aber ihr
treues Ausharren bei ihm lohnverheißend
an}\label{a-jesus-tadelt-den-ehrgeiz-der-juxfcnger-erkennt-aber-ihr-treues-ausharren-bei-ihm-lohnverheiuxdfend-an}}

\bibleverse{24} Da entstand auch noch ein Streit unter ihnen darüber,
wer von ihnen als der Größte zu gelten habe. \bibleverse{25} Er aber
sagte zu ihnen: »Die Könige der Völker herrschen gewaltsam über sie, und
ihre Machthaber lassen sich ›Wohltäter‹\textless sup title=``=~gnädige
Herren''\textgreater✲ nennen. \bibleverse{26} Bei euch aber darf es
nicht so sein, sondern der Größte unter euch muß wie der Jüngste sein
und wer obenan sitzt, wie der Aufwartende. \bibleverse{27} Denn wer ist
der Größere: der zu Tische sitzt oder der dabei bedient? Doch wohl der
zu Tische Sitzende. Ich aber bin in eurer Mitte wie der Aufwartende.
\bibleverse{28} Ihr aber seid es, die in meinen Anfechtungen bei mir
ausgeharrt haben. \bibleverse{29} So vermache ich euch denn die
Königswürde\textless sup title=``oder: Königsherrschaft''\textgreater✲,
wie mein Vater sie mir vermacht✲ hat: \bibleverse{30} ihr sollt
(dereinst) in meinem Reiche an meinem Tische essen und trinken und sollt
auf Thronen sitzen, um die zwölf Stämme Israels zu richten\textless sup
title=``=~als Herrscher zu leiten''\textgreater✲.«

\hypertarget{b-warnung-an-den-selbstbewuuxdften-petrus-und-weissagung-seiner-verleugnung}{%
\paragraph{b) Warnung an den selbstbewußten Petrus und Weissagung seiner
Verleugnung}\label{b-warnung-an-den-selbstbewuuxdften-petrus-und-weissagung-seiner-verleugnung}}

\bibleverse{31} »Simon, Simon! Wisse wohl: der Satan hat sich (von Gott)
ausgebeten, Gewalt über euch zu erhalten, um euch zu
sichten\textless sup title=``eig. zu sieben =~im Siebe zu
schütteln''\textgreater✲, wie man Weizen siebt; \bibleverse{32} ich aber
habe für dich gebeten, daß dein Glaube nicht ausgehe\textless sup
title=``oder: ganz aufhöre''\textgreater✲; und du, wenn du dich einst
bekehrt hast, stärke deine Brüder!« \bibleverse{33} Da antwortete ihm
Petrus: »Herr, ich bin bereit, mit dir sowohl ins Gefängnis als auch in
den Tod zu gehen!« \bibleverse{34} Jesus aber entgegnete: »Ich sage dir,
Petrus: Der Hahn wird heute nicht krähen, bis du dreimal geleugnet hast,
mich zu kennen!«

\hypertarget{c-hinweis-auf-die-von-den-juxfcngern-bisher-in-sicherheit-durchlebte-zeit-und-auf-die-ernste-und-schwere-zukunft}{%
\paragraph{c) Hinweis auf die von den Jüngern bisher in Sicherheit
durchlebte Zeit und auf die ernste und schwere
Zukunft}\label{c-hinweis-auf-die-von-den-juxfcngern-bisher-in-sicherheit-durchlebte-zeit-und-auf-die-ernste-und-schwere-zukunft}}

\bibleverse{35} Dann fuhr er fort: »Als ich euch ohne Geldbeutel, ohne
Ranzen\textless sup title=``oder: Reisetasche''\textgreater✲ und Schuhe
aussandte, habt ihr da Mangel an irgend etwas gelitten?« Sie
antworteten: »Nein, an nichts!« \bibleverse{36} Er fuhr fort: »Jetzt
aber -- wer einen Beutel (mit Geld) hat, der nehme ihn mit sich, ebenso
auch einen Ranzen, und wer nichts (derartiges) hat, verkaufe seinen
Mantel und kaufe sich ein Schwert! \bibleverse{37} Denn ich sage euch:
Folgendes Schriftwort muß sich an mir erfüllen\textless sup title=``Jes
53,12''\textgreater✲: ›Er ist unter die Gesetzlosen✲ gerechnet worden‹;
denn in der Tat: das mir bestimmte Geschick kommt jetzt zum Abschluß.«
\bibleverse{38} Da sagten sie: »Herr, siehe, hier sind zwei Schwerter!«
Er antwortete ihnen: »Das genügt.«

\hypertarget{jesu-seelenkampf-und-gebet-am-uxf6lberg}{%
\subsubsection{6. Jesu Seelenkampf und Gebet am
Ölberg}\label{jesu-seelenkampf-und-gebet-am-uxf6lberg}}

\bibleverse{39} Er ging dann (aus der Stadt) hinaus und begab sich nach
seiner Gewohnheit an den Ölberg; es begleiteten ihn auch seine Jünger.
\bibleverse{40} Als er an Ort und Stelle angelangt war, sagte er zu
ihnen: »Betet darum, daß ihr nicht in Versuchung geratet!«
\bibleverse{41} Darauf entfernte er sich etwa einen Steinwurf weit von
ihnen, kniete nieder und betete \bibleverse{42} mit den Worten: »Vater,
wenn du willst, so laß diesen Kelch an mir vorübergehen! Doch nicht mein
Wille, sondern der deine geschehe!« \bibleverse{43} Da erschien ihm ein
Engel vom Himmel und stärkte ihn. \bibleverse{44} Und als er in
angstvollen Seelenkampf geraten war, betete er noch inbrünstiger; und
sein Schweiß wurde wie Blutstropfen, die zur Erde niederfielen.
\bibleverse{45} Nach dem Gebet stand er auf, und als er zu seinen
Jüngern kam, fand er sie vor Traurigkeit eingeschlafen \bibleverse{46}
und sagte zu ihnen: »Was schlaft ihr? Steht auf und betet, damit ihr
nicht in Versuchung geratet!«

\hypertarget{gefangennahme-jesu}{%
\subsubsection{7. Gefangennahme Jesu}\label{gefangennahme-jesu}}

\bibleverse{47} Während er noch (zu ihnen) redete, erschien plötzlich
eine Volksschar, und der mit dem Namen Judas, einer von den Zwölfen,
ging an ihrer Spitze und trat auf Jesus zu, um ihn zu küssen.
\bibleverse{48} Jesus aber sagte zu ihm: »Judas, mit einem Kuß verrätst
du den Menschensohn?« \bibleverse{49} Als nun die Begleiter Jesu sahen,
was da kommen würde, sagten sie: »Herr, sollen wir mit dem Schwert
dreinschlagen?«, \bibleverse{50} und einer von ihnen schlug (wirklich)
nach dem Knecht des Hohenpriesters und hieb ihm das rechte Ohr ab.
\bibleverse{51} Jesus aber antwortete: »Laßt ab! Bis hierher und nicht
weiter!« Dann rührte er das Ohr an und heilte ihn. \bibleverse{52} Zu
den Hohenpriestern aber und den Hauptleuten der Tempelwache und den
Ältesten, die gegen ihn hergekommen waren, sagte Jesus: »Wie gegen einen
Räuber seid ihr mit Schwertern und Knütteln ausgezogen. \bibleverse{53}
Während ich täglich bei euch im Tempel war, habt ihr die Hände nicht
gegen mich ausgestreckt. Aber dies ist eure Stunde und die Macht der
Finsternis!«

\hypertarget{verleugnung-und-reue-des-petrus}{%
\subsubsection{8. Verleugnung und Reue des
Petrus}\label{verleugnung-und-reue-des-petrus}}

\bibleverse{54} Als sie ihn dann festgenommen hatten, führten sie ihn ab
und brachten ihn in das Haus des Hohenpriesters; Petrus aber folgte von
weitem. \bibleverse{55} Als sie dann mitten im Hof ein Feuer angezündet
und sich zusammengesetzt hatten, nahm auch Petrus mitten unter ihnen
Platz. \bibleverse{56} Da sah ihn eine Magd am Feuer sitzen; sie blickte
ihn scharf an und sagte: »Dieser ist auch bei ihm gewesen.«
\bibleverse{57} Petrus aber leugnete mit den Worten: »Weib, ich kenne
ihn nicht!« \bibleverse{58} Nach einer kleinen Weile bemerkte ihn ein
anderer und sagte: »Du gehörst auch zu ihnen!« Petrus aber entgegnete:
»Mensch, ich nicht!« \bibleverse{59} Nach Verlauf von etwa einer Stunde
versicherte ein anderer bestimmt: »Wahrhaftig, dieser ist auch mit ihm
zusammen gewesen, er ist ja auch ein Galiläer!« \bibleverse{60} Da
entgegnete Petrus: »Mensch, ich verstehe nicht, was du sagst!«; und
unmittelbar darauf, während er noch redete, krähte der Hahn.
\bibleverse{61} Da wandte der Herr sich um und blickte Petrus an; und
Petrus dachte an das Wort des Herrn, wie er zu ihm gesagt hatte: »Ehe
noch der Hahn heute kräht, wirst du mich dreimal verleugnen.«
\bibleverse{62} Und er ging hinaus und weinte bitterlich.

\hypertarget{verspottung-und-miuxdfhandlung-jesu-verhuxf6r-vor-dem-hohen-rat}{%
\subsubsection{9. Verspottung und Mißhandlung Jesu; Verhör vor dem Hohen
Rat}\label{verspottung-und-miuxdfhandlung-jesu-verhuxf6r-vor-dem-hohen-rat}}

\bibleverse{63} Die Männer aber, die Jesus zu bewachen hatten, trieben
ihren Spott mit ihm und schlugen ihn; \bibleverse{64} sie verhüllten ihm
das Gesicht und richteten dann die Frage an ihn: »Weissage uns: Wer
ist's, der dich (eben) geschlagen hat?« \bibleverse{65} Auch noch viele
andere Schmähungen stießen sie gegen ihn aus.

\bibleverse{66} Als es dann Tag geworden war, versammelte sich der Rat
der Ältesten des Volkes, Hohepriester und Schriftgelehrte; sie ließen
ihn in ihre Versammlung führen \bibleverse{67} und sagten: »Wenn du
Christus\textless sup title=``=~der Messias''\textgreater✲ bist, so sage
es uns!« Doch er erwiderte ihnen: »Wenn ich es euch sage, werdet ihr es
mir doch nicht glauben, \bibleverse{68} und wenn ich Fragen an euch
richte, werdet ihr mir keine Antwort geben. \bibleverse{69} Aber von nun
an wird der Menschensohn zur Rechten der Macht Gottes
sitzen!«\textless sup title=``Dan 7,13; Ps 110,1''\textgreater✲
\bibleverse{70} Da sagten sie alle: »So bist du also der Sohn Gottes?«
Er antwortete ihnen: »Ja, ihr selbst sagt es: ich bin's.«
\bibleverse{71} Da erklärten sie: »Wozu haben wir noch weitere
Zeugenaussagen nötig? Wir haben es ja selbst aus seinem Munde gehört!«

\hypertarget{jesus-vor-pilatus-und-herodes-seine-verurteilung-durch-pilatus}{%
\subsubsection{10. Jesus vor Pilatus und Herodes; seine Verurteilung
durch
Pilatus}\label{jesus-vor-pilatus-und-herodes-seine-verurteilung-durch-pilatus}}

\hypertarget{a-die-anklage-der-juden-und-das-verhuxf6r-jesu-vor-pilatus}{%
\paragraph{a) Die Anklage der Juden und das Verhör Jesu vor
Pilatus}\label{a-die-anklage-der-juden-und-das-verhuxf6r-jesu-vor-pilatus}}

\hypertarget{section-22}{%
\section{23}\label{section-22}}

\bibleverse{1} Nun erhob sich ihre ganze Versammlung, und sie führten
ihn dem Pilatus vor. \bibleverse{2} Dort erhoben sie folgende Anklagen
gegen ihn: »Wir haben festgestellt, daß dieser Mensch unser Volk
aufwiegelt und es davon abhalten will, dem Kaiser Steuern zu entrichten,
und daß er behauptet, er sei Christus\textless sup title=``oder: der
Messias''\textgreater✲, ein König.« \bibleverse{3} Pilatus fragte ihn
nun: »Bist du der König der Juden?«, und Jesus antwortete ihm: »Ja, ich
bin es!« \bibleverse{4} Da sagte Pilatus zu den Hohenpriestern und der
Volksmenge: »Ich finde keine Schuld an diesem Manne.« \bibleverse{5} Sie
aber versicherten immer heftiger: »Er wiegelt das Volk auf, indem er
seine Lehre im ganzen jüdischen Lande verbreitet: in Galiläa hat er
damit begonnen und bis hierher es fortgesetzt!« \bibleverse{6} Als
Pilatus das hörte, fragte er, ob der Mann ein Galiläer wäre;
\bibleverse{7} und als er vernahm, daß er aus dem Machtbereich des
Herodes sei, sandte er ihn dem Herodes zu, der in diesen Tagen ebenfalls
in Jerusalem weilte.

\hypertarget{b-jesus-vor-herodes}{%
\paragraph{b) Jesus vor Herodes}\label{b-jesus-vor-herodes}}

\bibleverse{8} Herodes aber war sehr erfreut darüber, Jesus zu sehen;
denn er hätte ihn längst gern gesehen, weil er viel über ihn gehört
hatte; er hoffte auch, ein Wunderzeichen von ihm vollführt zu sehen.
\bibleverse{9} So richtete er denn mancherlei Fragen an ihn, doch Jesus
gab ihm keinerlei Antwort; \bibleverse{10} die Hohenpriester und
Schriftgelehrten aber standen dabei und verklagten ihn leidenschaftlich.
\bibleverse{11} Da behandelte ihn denn Herodes samt den Herren seines
Gefolges mit Verachtung und Hohn und sandte ihn, nachdem er ihm ein
Prachtgewand hatte anlegen lassen, zu Pilatus zurück. \bibleverse{12} An
diesem Tage wurden Herodes und Pilatus miteinander befreundet, während
sie vordem in Feindschaft gegeneinander gestanden hatten.

\hypertarget{c-jesus-wieder-vor-pilatus}{%
\paragraph{c) Jesus wieder vor
Pilatus}\label{c-jesus-wieder-vor-pilatus}}

\bibleverse{13} Pilatus ließ nun die Hohenpriester und die Mitglieder
des Hohen Rates und das Volk zusammenrufen \bibleverse{14} und sagte zu
ihnen: »Ihr habt mir diesen Mann als einen Volksverführer vorgeführt.
Nun, ihr seht, ich habe ihn in eurem Beisein verhört, habe aber an ihm
durchaus nicht die Schuld gefunden, deren ihr ihn anklagt.
\bibleverse{15} Ebensowenig auch Herodes, denn er hat ihn an uns
zurückverwiesen. Ihr seht also: nichts, was die Todesstrafe verdient,
ist von ihm begangen worden. \bibleverse{16} Darum will ich ihn geißeln
lassen und dann freigeben.«

\hypertarget{jesus-und-barabbas-die-verurteilung}{%
\paragraph{Jesus und Barabbas; die
Verurteilung}\label{jesus-und-barabbas-die-verurteilung}}

\bibleverse{17} {[}Er war aber verpflichtet, ihnen an jedem Fest einen
(Gefangenen) freizugeben.{]} \bibleverse{18} Da schrien sie aber
allesamt: »Hinweg\textless sup title=``=~zum Tode''\textgreater✲ mit
diesem! Gib uns dagegen Barabbas frei!«~-- \bibleverse{19} dieser saß
nämlich wegen eines Aufruhrs, der in der Stadt vorgekommen war, und
wegen Mordes im Gefängnis. \bibleverse{20} Da redete Pilatus zum
zweitenmal auf sie ein, weil er Jesus gern freigeben wollte;
\bibleverse{21} sie aber riefen ihm dagegen zu: »Kreuzige, kreuzige
ihn!« \bibleverse{22} Zum drittenmal fragte er sie dann: »Was hat denn
dieser Mann Böses getan? Ich habe keine todeswürdige Schuld an ihm
gefunden! Ich will ihn also geißeln lassen und dann freigeben.«
\bibleverse{23} Sie aber bestürmten ihn mit lautem Geschrei und
verlangten seine Kreuzigung; und ihr Geschrei drang durch\textless sup
title=``oder: hatte Erfolg''\textgreater✲. \bibleverse{24} So fällte
denn Pilatus das Urteil, ihr Verlangen solle erfüllt werden.
\bibleverse{25} Er gab also den Mann frei, der wegen Aufruhrs und Mordes
ins Gefängnis geworfen worden war, wie sie es verlangten; Jesus dagegen
gab er ihrem Willen preis.

\hypertarget{jesu-todesweg-nach-golgatha-und-seine-worte-an-die-klagenden-frauen-von-jerusalem-seine-kreuzigung-und-sein-tod}{%
\subsubsection{11. Jesu Todesweg nach Golgatha und seine Worte an die
klagenden Frauen von Jerusalem; seine Kreuzigung und sein
Tod}\label{jesu-todesweg-nach-golgatha-und-seine-worte-an-die-klagenden-frauen-von-jerusalem-seine-kreuzigung-und-sein-tod}}

\bibleverse{26} Als sie ihn dann (zur Richtstätte) abführten, griffen
sie einen gewissen Simon aus Cyrene auf, der vom Felde heimkam, und
bürdeten ihm das Kreuz auf, damit er es Jesu nachtrüge. \bibleverse{27}
Es folgte ihm aber eine große Volksmenge, auch viele Frauen, die um ihn
wehklagten und weinten. \bibleverse{28} Da wandte Jesus sich zu ihnen um
und sagte: »Ihr Töchter Jerusalems, weint nicht über mich, weint
vielmehr über euch selbst und über eure Kinder! \bibleverse{29} Denn
wisset wohl: es kommen Tage, an denen man sagen wird: ›Glücklich zu
preisen sind die Unfruchtbaren und die Frauen, die nicht Mutter geworden
sind und die kein Kind an der Brust genährt haben!‹ \bibleverse{30} Dann
wird man anfangen, den Bergen zuzurufen: ›Fallet auf uns!‹ und den
Hügeln: ›Bedecket uns!‹\textless sup title=``Hos 10,8''\textgreater✲
\bibleverse{31} Denn wenn man dies am grünen Holze tut, was wird da erst
am dürren geschehen?« \bibleverse{32} Es wurden aber außerdem noch zwei
Verbrecher mit ihm zur Hinrichtung abgeführt.

\bibleverse{33} Als sie nun an den Platz gekommen waren, der
›Schädel(stätte)‹ heißt, kreuzigten sie dort ihn und die beiden
Verbrecher, den einen zu seiner Rechten, den anderen zu seiner Linken.
\bibleverse{34} Jesus aber sprach: »Vater, vergib ihnen, denn sie wissen
nicht, was sie tun!« Darauf verteilten sie seine Kleidungsstücke unter
sich, indem sie das Los darüber warfen\textless sup title=``Ps
22,19''\textgreater✲; \bibleverse{35} und das Volk stand dabei und sah
zu. Es verhöhnten (ihn) aber auch die Mitglieder des Hohen Rates mit den
Worten: »Anderen hat er geholfen; so helfe er nun sich selbst, wenn er
wirklich Christus\textless sup title=``oder: der Messias''\textgreater✲,
der Gesalbte Gottes, ist, der Auserwählte!« \bibleverse{36} Auch die
Soldaten verspotteten ihn: sie traten hinzu, reichten ihm
Essig\textless sup title=``Ps 69,22''\textgreater✲ \bibleverse{37} und
sagten: »Bist du der König der Juden, so hilf dir selbst!«
\bibleverse{38} Über ihm war aber auch eine Inschrift angebracht {[}in
griechischer, lateinischer und hebräischer Schrift{]}: »Dies ist der
König der Juden.«

\hypertarget{jesus-und-die-beiden-schuxe4cher}{%
\paragraph{Jesus und die beiden
Schächer}\label{jesus-und-die-beiden-schuxe4cher}}

\bibleverse{39} Einer aber von den Verbrechern, die da gehenkt waren,
schmähte ihn mit den Worten: »Du willst Christus\textless sup
title=``oder: der Messias''\textgreater✲ sein? So hilf dir doch selbst
und uns!« \bibleverse{40} Da antwortete ihm der andere mit lautem
Vorwurf: »Hast du denn nicht einmal Furcht vor Gott, da dich doch
derselbe Urteilsspruch\textless sup title=``=~die gleiche
Strafe''\textgreater✲ getroffen hat? \bibleverse{41} Und zwar uns beide
mit Recht, denn wir empfangen den Lohn für unsere Taten; dieser aber hat
nichts Unrechtes getan!« \bibleverse{42} Dann fuhr er fort: »Jesus,
denke an mich, wenn du in deine Königsherrschaft\textless sup
title=``oder: mit deinem Reiche''\textgreater✲ kommst!« \bibleverse{43}
Da sagte Jesus zu ihm: »Wahrlich ich sage dir: Heute (noch) wirst du mit
mir im Paradiese sein!«

\hypertarget{jesu-sterben-die-wunderzeichen-bei-seinem-tode}{%
\paragraph{Jesu Sterben; die Wunderzeichen bei seinem
Tode}\label{jesu-sterben-die-wunderzeichen-bei-seinem-tode}}

\bibleverse{44} Es war nunmehr um die sechste Stunde\textless sup
title=``d.h. die Mittagszeit''\textgreater✲: da kam eine Finsternis über
das ganze Land bis zur neunten Stunde, \bibleverse{45} indem die Sonne
ihren Schein verlor; und der Vorhang im Tempel riß mitten entzwei.
\bibleverse{46} Da rief Jesus mit lauter Stimme die Worte aus: »Vater,
in deine Hände befehle ich meinen Geist!«\textless sup title=``Ps
31,6''\textgreater✲, und nach diesen Worten verschied er.
\bibleverse{47} Als nun der Hauptmann sah, was geschehen war, pries er
Gott und sagte: »Dieser Mann ist wirklich ein Gerechter gewesen!«
\bibleverse{48} Und die ganze Volksmenge, die zu diesem Schauspiel
zusammengekommen war und alles sah, was sich zugetragen hatte, schlug
sich an die Brust und kehrte heim. \bibleverse{49} Alle seine Bekannten
aber standen von ferne\textless sup title=``Ps 38,12''\textgreater✲,
auch die Frauen, die ihm aus Galiläa nachgefolgt waren und dies alles
mit ansahen.

\hypertarget{die-grablegung-jesu}{%
\subsubsection{12. Die Grablegung Jesu}\label{die-grablegung-jesu}}

\bibleverse{50} Und siehe, ein Mann namens Joseph, der ein Mitglied des
Hohen Rates war, ein guter und gerechter Mann~-- \bibleverse{51} er war
mit ihrem Beschluß und ihrer Handlungsweise nicht einverstanden gewesen
-- aus der jüdischen Stadt Arimathäa, der auf das Reich Gottes wartete:
\bibleverse{52} dieser ging zu Pilatus und bat ihn um den Leichnam Jesu.
\bibleverse{53} Dann nahm er ihn (vom Kreuz) herab, wickelte ihn in
feine Leinwand und legte ihn in ein Grab, das in den Felsen gehauen und
in welchem bisher noch niemand beigesetzt worden war. \bibleverse{54} Es
war aber der Rüsttag✲, und der Sabbat wollte anbrechen. \bibleverse{55}
Die Frauen aber, die ihn aus Galiläa begleitet hatten, waren mitgegangen
und hatten sich das Grab und die Beisetzung seines Leichnams angesehen.
\bibleverse{56} Nachdem sie hierauf (in die Stadt) zurückgekehrt waren,
besorgten sie wohlriechende Stoffe und Salben und brachten dann den
Sabbat nach der Vorschrift des Gesetzes in der Stille zu.

\hypertarget{entdeckung-des-leeren-grabes-am-ostermorgen-die-offenbarung-an-die-frauen}{%
\subsubsection{13. Entdeckung des leeren Grabes am Ostermorgen; die
Offenbarung an die
Frauen}\label{entdeckung-des-leeren-grabes-am-ostermorgen-die-offenbarung-an-die-frauen}}

\hypertarget{section-23}{%
\section{24}\label{section-23}}

\bibleverse{1} Am ersten Tage nach dem Sabbat\textless sup title=``oder:
am ersten Tage der Woche''\textgreater✲ aber kamen sie beim Morgengrauen
zum Grabe mit den wohlriechenden Stoffen, die sie zubereitet hatten.
\bibleverse{2} Da fanden sie den Stein vom Grabe weggewälzt,
\bibleverse{3} doch als sie hineingetreten waren, fanden sie den
Leichnam des Herrn Jesus nicht. \bibleverse{4} Während sie nun hierüber
ratlos waren, standen plötzlich zwei Männer in strahlenden Gewändern bei
ihnen; \bibleverse{5} und als sie in Furcht gerieten und den Blick zu
Boden schlugen, sagten diese zu ihnen: »Was sucht ihr den Lebenden bei
den Toten? \bibleverse{6} Er ist nicht (mehr) hier, sondern ist
auferweckt worden. Denkt daran, wie er zu euch geredet hat, als er noch
in Galiläa war, \bibleverse{7} und aussagte, der Menschensohn müsse in
die Hände sündiger Menschen ausgeliefert und ans Kreuz geschlagen werden
und am dritten Tage auferstehen.« \bibleverse{8} Da erinnerten sie sich
seiner Worte, \bibleverse{9} kehrten vom Grabe zurück und berichteten
dies alles den elf (Jüngern) und allen übrigen. \bibleverse{10} Es waren
dies aber Maria von Magdala und Johanna und Maria, die Mutter des
Jakobus, und die anderen Frauen mit ihnen, die den Aposteln dies
berichteten; \bibleverse{11} doch diese Mitteilungen erschienen ihnen
als leeres Gerede, und sie schenkten ihnen\textless sup title=``d.h. den
Frauen''\textgreater✲ keinen Glauben. \bibleverse{12} Petrus aber machte
sich auf und eilte zum Grabe, und als er sich hineinbückte, sah er nur
die Leintücher daliegen; so kehrte er denn nach Hause zurück, voll
Verwunderung über das Geschehene.

\hypertarget{die-emmausjuxfcnger}{%
\subsubsection{14. Die Emmausjünger}\label{die-emmausjuxfcnger}}

\bibleverse{13} Und siehe, zwei von ihnen waren an demselben Tage auf
der Wanderung nach einem Dorf begriffen, das sechzig
Stadien\textless sup title=``d.h. etwa zwölf Kilometer oder zweieinhalb
Stunden''\textgreater✲ von Jerusalem entfernt lag und Emmaus hieß.
\bibleverse{14} Sie unterhielten sich miteinander über alle diese
Begebenheiten. \bibleverse{15} Während sie sich nun so unterhielten und
sich gegeneinander aussprachen, kam Jesus selbst hinzu und schloß sich
ihnen auf der Wanderung an; \bibleverse{16} ihre Augen jedoch wurden
gehalten, so daß sie ihn nicht erkannten. \bibleverse{17} Er fragte sie
nun: »Was sind das für Gespräche, die ihr da auf eurer Wanderung
miteinander führt?« Da blieben sie betrübten Angesichts stehen.
\bibleverse{18} Der eine aber von ihnen, namens Kleopas, erwiderte ihm:
»Du bist wohl der einzige, der sich in Jerusalem aufhält und nichts von
dem erfahren hat, was in diesen Tagen dort geschehen ist?«
\bibleverse{19} Er fragte sie: »Was denn?« Sie antworteten ihm: »Das,
was mit Jesus von Nazareth geschehen ist, der ein Prophet war, gewaltig
in Tat und Wort vor Gott und dem ganzen Volk. \bibleverse{20} Ihn haben
unsere Hohenpriester und der Hohe Rat zur Todesstrafe ausgeliefert und
ans Kreuz gebracht. \bibleverse{21} Wir aber hatten gehofft, daß er es
sei, der Israel erlösen würde; aber nun ist bei dem allem heute schon
der dritte Tag, seit dies geschehen ist. \bibleverse{22} Dazu haben uns
aber auch noch einige Frauen, die zu uns gehören, in Bestürzung
versetzt: sie sind heute in der Frühe am Grabe gewesen \bibleverse{23}
und haben, als sie seinen Leichnam nicht gefunden hatten, nach ihrer
Rückkehr erzählt, sie hätten auch noch eine Erscheinung von Engeln
gesehen, und diese hätten gesagt, daß er lebe. \bibleverse{24} Da sind
denn einige der Unseren zum Grabe hingegangen und haben es so gefunden,
wie die Frauen berichtet hatten, ihn selbst aber haben sie nicht
gesehen.« \bibleverse{25} Da sagte er zu ihnen: »O ihr Gedankenlosen,
wie ist doch euer Herz so träge\textless sup title=``oder:
stumpf''\textgreater✲, um an alles das zu glauben, was die Propheten
verkündigt haben! \bibleverse{26} Mußte denn Christus\textless sup
title=``oder: der Messias''\textgreater✲ dies nicht leiden und dann in
seine Herrlichkeit eingehen?« \bibleverse{27} Darauf fing er bei Mose
und allen Propheten an und legte ihnen alle Schriftstellen aus, die sich
auf ihn bezogen.

\bibleverse{28} So kamen sie in die Nähe des Dorfes, wohin die Wanderung
ging, und er tat so, als wollte er weiterwandern. \bibleverse{29} Da
nötigten sie ihn mit den Worten: »Bleibe bei uns, denn es will Abend
werden, und der Tag hat sich schon geneigt!« So trat er denn ein, um bei
ihnen zu bleiben. \bibleverse{30} Als er sich hierauf mit ihnen zu Tisch
gesetzt hatte, nahm er das Brot, sprach den Lobpreis (Gottes), brach das
Brot und gab es ihnen: \bibleverse{31} da gingen ihnen die Augen auf,
und sie erkannten ihn; doch er entschwand ihren Blicken. \bibleverse{32}
Da sagten sie zueinander: »Brannte nicht unser Herz in uns, als er
unterwegs mit uns redete und uns den Sinn der Schriftstellen erschloß?«
\bibleverse{33} Und sie machten sich noch in derselben Stunde auf,
kehrten nach Jerusalem zurück und fanden dort die Elf nebst ihren
Genossen versammelt; \bibleverse{34} diese teilten ihnen mit: »Der Herr
ist wirklich auferweckt worden und ist dem Simon erschienen!«
\bibleverse{35} Da erzählten auch sie, was sich unterwegs zugetragen
hatte und wie er von ihnen am Brechen des Brotes erkannt worden war.

\hypertarget{erscheinung-jesu-im-juxfcngerkreise-am-ostersonntagabend-sein-missionsbefehl-und-abschied-von-den-juxfcngern}{%
\subsubsection{15. Erscheinung Jesu im Jüngerkreise am
Ostersonntagabend; sein Missionsbefehl und Abschied von den
Jüngern}\label{erscheinung-jesu-im-juxfcngerkreise-am-ostersonntagabend-sein-missionsbefehl-und-abschied-von-den-juxfcngern}}

\bibleverse{36} Während sie hierüber noch sprachen, trat Jesus selbst
mitten unter sie mit den Worten: »Friede sei mit euch!« \bibleverse{37}
Da gerieten sie in Angst und Furcht und meinten, einen Geist zu sehen.
\bibleverse{38} Doch er sagte zu ihnen: »Was seid ihr so bestürzt, und
warum steigen Zweifel in euren Herzen auf? \bibleverse{39} Seht meine
Hände und meine Füße an, daß ich es leibhaftig bin! Betastet mich und
beschaut mich; ein Geist hat ja doch kein Fleisch und keine Knochen, wie
ihr solche an mir wahrnehmt.« \bibleverse{40} Nach diesen Worten zeigte
er ihnen seine Hände und Füße. \bibleverse{41} Als sie aber vor Freude
immer noch ungläubig und voll Verwunderung waren, fragte er sie:
\bibleverse{42} »Habt ihr hier nicht etwas zu essen?« Da reichten sie
ihm ein Stück von einem gebratenen Fisch; \bibleverse{43} das nahm er
und aß es vor ihren Augen.

\bibleverse{44} Dann sagte er zu ihnen: »Dies besagen meine Worte, die
ich zu euch gesprochen habe, als ich noch bei euch war: es müsse alles
in Erfüllung gehen, was im mosaischen Gesetz, bei den Propheten und in
den Psalmen über mich geschrieben steht.« \bibleverse{45} Hierauf
erschloß er ihnen den Sinn für das Verständnis der Schriften
\bibleverse{46} und sagte zu ihnen: »So steht geschrieben:
Christus\textless sup title=``=~der Messias''\textgreater✲ muß leiden
und am dritten Tage von den Toten auferstehen, \bibleverse{47} und auf
Grund seines Namens muß Buße zur Vergebung der Sünden bei allen Völkern
gepredigt werden, zuerst aber in Jerusalem. \bibleverse{48} Ihr seid die
Zeugen hierfür. \bibleverse{49} Und wisset wohl: Ich sende das
Verheißungsgut meines Vaters auf euch herab; ihr aber bleibt hier in der
Stadt, bis ihr mit Kraft aus der Höhe ausgerüstet worden seid!«

\hypertarget{jesu-himmelfahrt}{%
\paragraph{Jesu Himmelfahrt}\label{jesu-himmelfahrt}}

\bibleverse{50} Hierauf führte er sie (aus der Stadt) hinaus bis in die
Nähe von Bethanien, erhob dann seine Hände und segnete sie;
\bibleverse{51} und es begab sich: während er sie segnete, schied er von
ihnen und wurde in den Himmel emporgehoben.

\hypertarget{schluuxdf-des-evangeliums}{%
\subsubsection{16. Schluß des
Evangeliums}\label{schluuxdf-des-evangeliums}}

\bibleverse{52} Und sie warfen sich anbetend vor ihm nieder und kehrten
hocherfreut nach Jerusalem zurück \bibleverse{53} und hielten sich
beständig im Tempel auf und priesen Gott.
