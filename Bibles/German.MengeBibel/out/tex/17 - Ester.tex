\hypertarget{das-buch-esther}{%
\section{DAS BUCH ESTHER}\label{das-buch-esther}}

\hypertarget{verstouxdfung-der-kuxf6nigin-wasthi}{%
\subsubsection{1. Verstoßung der Königin
Wasthi}\label{verstouxdfung-der-kuxf6nigin-wasthi}}

\hypertarget{a-das-fest-des-perserkuxf6nigs-ahasveros-xerxes-in-susa-fuxfcr-die-wuxfcrdentruxe4ger-und-huxf6chsten-beamten-seines-reiches}{%
\paragraph{a) Das Fest des Perserkönigs Ahasveros (=~Xerxes) in Susa für
die Würdenträger und höchsten Beamten seines
Reiches}\label{a-das-fest-des-perserkuxf6nigs-ahasveros-xerxes-in-susa-fuxfcr-die-wuxfcrdentruxe4ger-und-huxf6chsten-beamten-seines-reiches}}

\hypertarget{section}{%
\section{1}\label{section}}

1Es begab sich unter der Regierung des Ahasveros -- desselben Ahasveros,
der von Indien bis Äthiopien über hundertundsiebenundzwanzig Provinzen
herrschte --, 2zu jener Zeit also, als der König Ahasveros auf seinem
Königsthron in der Burg\textless sup title=``=~Königspalast,
Residenz''\textgreater✲ Susa saß, 3im dritten Jahre seiner Regierung: da
veranstaltete er für alle seine Fürsten und Diener\textless sup
title=``oder: Beamten''\textgreater✲ ein Festmahl, wobei die persischen
und medischen Heerführer, die Edlen und die höchsten Beamten der
Provinzen vor ihm versammelt waren, 4indem er dabei den Reichtum seiner
königlichen Herrlichkeit und die glanzvolle Pracht seiner Größe✲ viele
Tage lang, nämlich hundertundachtzig Tage, zur Schau stellte.

\hypertarget{b-das-mahl-fuxfcr-die-bewohnerschaft-der-kuxf6nigsstadt-residenz-susa-das-festmahl-der-kuxf6nigin-wasthi}{%
\paragraph{b) Das Mahl für die Bewohnerschaft der Königsstadt
(=~Residenz) Susa; das Festmahl der Königin
Wasthi}\label{b-das-mahl-fuxfcr-die-bewohnerschaft-der-kuxf6nigsstadt-residenz-susa-das-festmahl-der-kuxf6nigin-wasthi}}

5Als dann diese Tage zu Ende waren, veranstaltete der König für die
gesamte Bewohnerschaft, die sich in der Residenz Susa befand, vom
Größten bis zum Kleinsten, ein siebentägiges Festmahl auf dem Hofe vor
dem königlichen Schloßgarten. 6Weiße und purpurblaue Vorhänge von
Baumwolle waren mit Schnüren von Byssus und rotem Purpur mittels
silberner Ringe an Marmorsäulen aufgehängt; Ruhepolster von schwerem,
mit Gold- und Silberfäden durchwirktem Seidenzeug standen auf einem
Pflaster aus Einlegewerk✲ von Alabaster, von weißem und dunkelfarbigem
Marmor und Perlmutterstein. 7Die Getränke reichte man in goldenen
Gefäßen, die an Formen immer wieder verschieden waren; und Wein aus des
Königs Kellern war reichlich vorhanden, wie es der Freigebigkeit eines
Königs entspricht. 8Das Trinken aber ging der Verordnung gemäß ohne
jeden Zwang vor sich; denn der König hatte an alle seine
Palastbeamten\textless sup title=``oder: Hofmarschälle''\textgreater✲
die Weisung ergehen lassen, man solle jedem gestatten, es beim Trinken
nach seinem Belieben zu halten.~-- 9Auch die Königin Wasthi
veranstaltete ein Festmahl für die Frauen im Inneren des königlichen
Palastes des Königs Ahasveros.

\hypertarget{c-wasthi-weigert-sich-im-festsaal-zu-erscheinen}{%
\paragraph{c) Wasthi weigert sich, im Festsaal zu
erscheinen}\label{c-wasthi-weigert-sich-im-festsaal-zu-erscheinen}}

10Am siebten Tage nun, als der König durch den Wein in fröhliche
Stimmung versetzt war, befahl er Mehuman, Bistha, Harbona, Bigtha und
Abagtha, Sethar und Karkas, den sieben Kammerherren, die den Dienst beim
König Ahasveros versahen, 11sie sollten die Königin Wasthi im Schmuck
der Königskrone vor den König bringen, um den Völkern und Fürsten ihre
Schönheit zu zeigen; denn sie war in der Tat eine schöne Frau. 12Doch
die Königin Wasthi weigerte sich, auf den Befehl, den der König ihr
durch die Kammerherren hatte zugehen lassen, zu erscheinen.

\hypertarget{d-beratung-und-beschluuxdffassung-uxfcber-wasthis-bestrafung-bekanntgabe-der-verstouxdfung-im-ganzen-reiche}{%
\paragraph{d) Beratung und Beschlußfassung über Wasthis Bestrafung;
Bekanntgabe der Verstoßung im ganzen
Reiche}\label{d-beratung-und-beschluuxdffassung-uxfcber-wasthis-bestrafung-bekanntgabe-der-verstouxdfung-im-ganzen-reiche}}

Darüber ärgerte sich der König sehr und geriet in glühenden Zorn, 13so
daß er den Weisen, die sich auf die Zeiten verstanden -- denn so wurde
jede den König betreffende Angelegenheit dem Rat\textless sup
title=``oder: Kollegium''\textgreater✲ der Gesetz- und Rechtskundigen
vorgelegt; 14unter diesen standen ihm Karsna, Sethar, Admatha, Tharsis,
Meres, Marsna und Memuchan, die sieben persischen und medischen Fürsten,
am nächsten, die jederzeit Zutritt zum Könige hatten und die erste
Stelle im Reiche einnahmen --, die Frage vorlegte: 15»Wie ist nach dem
Gesetz mit der Königin Wasthi zu verfahren dafür, daß sie dem Befehl,
den der König Ahasveros ihr durch die Kammerherren hatte zugehen lassen,
nicht nachgekommen ist?« 16Da antwortete Memuchan vor dem Könige und den
Fürsten: »Nicht gegen den König allein hat sich die Königin Wasthi
vergangen, sondern zugleich gegen alle Fürsten und alle Völker, die in
sämtlichen Provinzen des Königs Ahasveros wohnen. 17Denn die Kunde von
dem Verhalten der Königin wird zu allen Frauen dringen, und es werden
ihnen ihre Eheherren verächtlich erscheinen, da sie sagen werden: ›Der
König Ahasveros befahl, man solle die Königin Wasthi vor ihn bringen,
aber sie kam nicht!‹ 18Schon heute werden die persischen und medischen
Fürstinnen, die von dem Verhalten der Königin Kenntnis erhalten haben,
allen Fürsten des Königs davon erzählen, woraus dann Verachtung und
Verdruß genug entstehen wird. 19Wenn es also dem König genehm ist, so
möge eine königliche Verordnung von ihm ausgehen und unter die
persischen und medischen Gesetze aufgenommen werden, und zwar mit
unwiderruflicher Geltung, daß Wasthi vor dem König Ahasveros nicht mehr
erscheinen dürfe, und der König möge ihre königliche Würde auf eine
andere übertragen, die besser ist als sie. 20Wenn dann die Verfügung,
die der König erlassen wird, in seinem ganzen Reiche, das ja groß ist,
zu allgemeiner Kenntnis gelangt, so werden alle Frauen ihren Eheherren,
vom größten bis zum kleinsten, die schuldige Ehre erweisen.«

21Dieser Vorschlag fand den Beifall des Königs und der Fürsten, und der
König ging auf den Rat Memuchans ein. 22Er sandte also Erlasse in alle
Provinzen des Königs, in jede Provinz nach ihrer Schrift und an jedes
Volk in der betreffenden Sprache: jeder Mann solle Herr in seinem Hause
sein und dürfe anordnen, was ihm beliebe✲.

\hypertarget{esthers-erhebung-zur-kuxf6nigin-mardochais-verdienst}{%
\subsubsection{2. Esthers Erhebung zur Königin; Mardochais
Verdienst}\label{esthers-erhebung-zur-kuxf6nigin-mardochais-verdienst}}

\hypertarget{a-veranstaltung-einer-grouxdfen-brautschau-fuxfcr-den-kuxf6nig}{%
\paragraph{a) Veranstaltung einer großen Brautschau für den
König}\label{a-veranstaltung-einer-grouxdfen-brautschau-fuxfcr-den-kuxf6nig}}

\hypertarget{section-1}{%
\section{2}\label{section-1}}

1Als nach diesen Vorkommnissen die Aufregung des Königs Ahasveros sich
gelegt hatte und er an Wasthi und ihr Verhalten und an den gegen sie
gefaßten Beschluß zurückdachte, 2sagten die Höflinge, die den König
persönlich zu bedienen hatten: »Man sollte für den König junge Mädchen,
Jungfrauen von besonderer Schönheit, suchen. 3Der König möge also in
allen Provinzen seines Reiches Beamte bestellen, welche alle jungen
Mädchen, Jungfrauen von besonderer Schönheit, im Frauenhause der
Residenz Susa zusammenbringen sollen unter die Obhut des königlichen
Kammerherrn Hegai, des Aufsehers über die Frauen. Man lasse ihnen dann
die zur Schönheitspflege erforderliche Behandlung zuteil werden, 4und
das Mädchen, das dem König am meisten gefällt, soll als Königin an
Wasthis Stelle treten.« Dieser Vorschlag fand den Beifall des Königs,
und er brachte ihn zur Ausführung.

\hypertarget{b-mitteilungen-zur-vorgeschichte-der-esther}{%
\paragraph{b) Mitteilungen zur Vorgeschichte der
Esther}\label{b-mitteilungen-zur-vorgeschichte-der-esther}}

5Nun lebte in der Residenz Susa ein jüdischer Mann namens Mardochai, der
Sohn Jairs, des Sohnes Simeis, des Sohnes des Kis, aus dem Stamme
Benjamin; 6er war aus Jerusalem mitweggeführt worden mit den
Gefangenen\textless sup title=``oder: Verbannten''\textgreater✲, die
zusammen mit dem jüdischen Könige Jechonja\textless sup
title=``=~Jojachin; vgl. 2.Kön 24,10-16''\textgreater✲ von Nebukadnezar,
dem Babylonierkönige, in die Gefangenschaft\textless sup title=``oder:
Verbannung''\textgreater✲ geführt worden waren. 7Er war der
Pflegevater\textless sup title=``oder: Vormund''\textgreater✲ der
Hadassa✲, das ist Esther✲, der Tochter seines Oheims; denn sie hatte
weder Vater noch Mutter, war aber ein Mädchen von herrlicher Gestalt und
außerordentlicher Schönheit; und nach dem Tode ihrer beiden Eltern hatte
Mardochai sie als Tochter angenommen.

\hypertarget{c-esthers-vorbereitungsjahr-im-kuxf6niglichen-palast-und-ihre-erhebung-zur-kuxf6nigin}{%
\paragraph{c) Esthers Vorbereitungsjahr im königlichen Palast und ihre
Erhebung zur
Königin}\label{c-esthers-vorbereitungsjahr-im-kuxf6niglichen-palast-und-ihre-erhebung-zur-kuxf6nigin}}

8Als nun der Erlaß und Befehl des Königs bekanntgemacht war und viele
junge Mädchen in der Residenz Susa zusammengebracht und unter die Obhut
Hegais gestellt wurden, da wurde auch Esther in den königlichen Palast
aufgenommen und kam unter die Aufsicht Hegais, des Hüters der
Frauen\textless sup title=``=~des Haremsaufsehers''\textgreater✲. 9Weil
nun das junge Mädchen ihm gefiel und seine Gunst gewann, war er eifrig
darauf bedacht, ihr die wirksamsten Schönheitsmittel und eine passende
Kost zukommen zu lassen und ihr die sieben auserlesensten Dienerinnen
aus dem königlichen Schlosse zu überweisen und sie mit ihren Dienerinnen
im schönsten Teile des Frauenhauses unterzubringen. 10Esther teilte aber
niemandem etwas von ihrer jüdischen Herkunft und ihren
Familienverhältnissen mit, weil Mardochai ihr verboten hatte, etwas
davon zu verraten. 11Mardochai ging aber Tag für Tag vor dem Hofe am
Frauenhause auf und ab, um sich zu erkundigen, ob es der Esther wohl
erginge und was mit ihr geschähe.

12Wenn aber die Reihe an irgendein junges Mädchen kam, sich zum König
Ahasveros zu begeben, nachdem sie zwölf Monate lang nach den für die
Frauen geltenden Verordnungen gelebt hatte -- denn so lange Zeit
erforderte ihre Schönheitspflege, nämlich sechs Monate lang eine
Behandlung mit Myrrhenöl und sechs Monate mit Spezereien und anderen
Schönheitsmitteln der Frauen --, 13und wenn sich das junge Mädchen
alsdann zum Könige hineinbegab, so wurde ihr alles, was sie verlangte,
aus dem Frauenhause in den königlichen Palast mitgegeben; 14am Abend
ging sie hinein, und am Morgen kehrte sie zurück, und zwar in das zweite
Frauenhaus unter die Aufsicht des königlichen Kammerherrn Saasgas, des
Hüters der Nebenfrauen; sie durfte dann nicht wieder zum König kommen,
außer wenn der König Gefallen an ihr gefunden hatte und sie ausdrücklich
berufen wurde.

15Als nun Esther, die Tochter Abihails, des Oheims Mardochais, der sie
als Tochter angenommen hatte, an der Reihe war, sich zum König zu
begeben, da verlangte sie nichts, als was Hegai, der königliche
Kammerherr, der Hüter der Frauen, ihr riet; denn Esther hatte sich die
Zuneigung aller erworben, die sie sahen. 16Als nun Esther zum König
Ahasveros in seinen königlichen Palast im zehnten Monat -- das ist der
Monat Tebeth -- im siebten Jahre seiner Regierung gebracht war, 17gewann
der König sie lieber als alle anderen Frauen, und sie erlangte seine
Gunst und Zuneigung in höherem Grade als alle übrigen Jungfrauen, so daß
er ihr die Königskrone aufs Haupt setzte und sie zur Königin an Wasthis
Stelle erhob. 18Darauf veranstaltete der König für alle seine
Würdenträger und Beamten ein großes Festmahl zu Ehren der Esther,
gewährte den Provinzen einen Steuererlaß und bewilligte eine
Getreidespende mit königlicher Freigebigkeit.

\hypertarget{d-mardochai-entdeckt-eine-verschwuxf6rung-gegen-den-kuxf6nig-sein-verdienst-wird-in-der-reichschronik-aufgezeichnet}{%
\paragraph{d) Mardochai entdeckt eine Verschwörung gegen den König; sein
Verdienst wird in der Reichschronik
aufgezeichnet}\label{d-mardochai-entdeckt-eine-verschwuxf6rung-gegen-den-kuxf6nig-sein-verdienst-wird-in-der-reichschronik-aufgezeichnet}}

19Als dann zum zweitenmal Jungfrauen zusammengeholt wurden, während
Mardochai sich gerade im Tor des Königs(palastes) aufhielt~-- 20Esther
hatte aber von ihrer Familie und ihrer jüdischen Herkunft nichts
kundwerden lassen, wie Mardochai ihr geboten hatte; sie tat überhaupt
alles, was Mardochai ihr auftrug, gerade wie früher, als sie noch in
Pflege bei ihm war --, 21in jenen Tagen also, während Mardochai sich
gerade im Tor des Königs(palastes) aufhielt, gerieten Bigthan und
Theres, zwei königliche Kammerherren aus der Zahl der Schwellenhüter, in
solchen Zorn, daß sie dem König Ahasveros nach dem Leben trachteten.
22Aber Mardochai erhielt Kunde von der Sache und machte der Königin
Esther Mitteilung davon; Esther aber erstattete dem Könige Anzeige im
Auftrage Mardochais. 23Als nun die Sache untersucht wurde und sich als
wahr herausstellte, wurden jene beiden ans Kreuz geschlagen und das
Vorkommnis ins Buch der Zeitgeschichte\textless sup title=``oder: in die
Reichschronik''\textgreater✲ für den König eingetragen.

\hypertarget{hamans-erhuxf6hung-sein-anschlag-gegen-die-juden}{%
\subsubsection{3. Hamans Erhöhung; sein Anschlag gegen die
Juden}\label{hamans-erhuxf6hung-sein-anschlag-gegen-die-juden}}

\hypertarget{a-hamans-befuxf6rderung-zur-huxf6chsten-ehrenstellung-mardochai-verweigert-ihm-die-kniebeugung-haman-beschlieuxdft-die-ausrottung-aller-juden}{%
\paragraph{a) Hamans Beförderung zur höchsten Ehrenstellung; Mardochai
verweigert ihm die Kniebeugung; Haman beschließt die Ausrottung aller
Juden}\label{a-hamans-befuxf6rderung-zur-huxf6chsten-ehrenstellung-mardochai-verweigert-ihm-die-kniebeugung-haman-beschlieuxdft-die-ausrottung-aller-juden}}

\hypertarget{section-2}{%
\section{3}\label{section-2}}

1Nach diesen Begebenheiten erhob der König Ahasveros den Agagiten Haman,
den Sohn Hammedathas, zu den höchsten Ehren und Würden und wies ihm
seinen Stuhl\textless sup title=``oder: Sitz''\textgreater✲ über dem
aller Fürsten in seiner Umgebung\textless sup title=``=~am
Hofe''\textgreater✲ an. 2So mußten denn alle Diener\textless sup
title=``oder: Beamten''\textgreater✲ des Königs, die sich am königlichen
Hofe befanden, die Knie vor Haman beugen und sich vor ihm
niederwerfen\textless sup title=``oder: verneigen''\textgreater✲; denn
so hatte der König es ihm zu Ehren befohlen. Mardochai aber beugte die
Knie nicht und warf sich auch nicht nieder. 3Da fragten ihn die Diener
des Königs, die am Hofe des Königs waren, warum er den Befehl des Königs
unbeachtet lasse. 4Als sie diese Frage Tag für Tag an ihn richteten,
ohne daß er auf sie hörte, meldeten sie es dem Haman, um zu sehen, ob
die Rechtfertigung Mardochais anerkannt werden würde; er hatte ihnen
nämlich mitgeteilt, daß er ein Jude sei. 5Als nun Haman selbst sah, daß
Mardochai vor ihm die Knie nicht beugte und sich nicht niederwarf,
geriet er in die höchste Wut; 6doch da er es unter seiner Würde hielt,
an Mardochai allein die Hand zu legen -- man hatte ihm nämlich
mitgeteilt, welchem Volke Mardochai angehöre --, so faßte er den Plan,
alle Juden, die im ganzen Reiche des Ahasveros lebten, zugleich mit
Mardochai auszurotten.

\hypertarget{b-haman-setzt-seinen-beschluuxdf-beim-kuxf6nige-durch}{%
\paragraph{b) Haman setzt seinen Beschluß beim Könige
durch}\label{b-haman-setzt-seinen-beschluuxdf-beim-kuxf6nige-durch}}

7Im ersten Monat -- das ist der Monat Nisan --, im zwölften
Regierungsjahr des Königs Ahasveros, warf man das Pur, d.h. das Los, in
Gegenwart Hamans von einem Tage zum andern und von einem Monat zum
andern bis zum zwölften Monat, das ist der Monat Adar (und das Los fiel
auf den dreizehnten Tag). 8Dann sagte Haman zum König Ahasveros: »Da ist
ein einzigartiges Volk, das unter den Völkern in allen Provinzen deines
Reiches zerstreut und abgesondert lebt und dessen Gesetze von denen
aller anderen Völker abweichen; da sie sich nun nach den Gesetzen des
Königs nicht richten, so ist es für den König nicht geziemend, sie ruhig
gewähren zu lassen. 9Wenn es dem König genehm ist, so möge ihre
Ausrottung durch einen schriftlichen Erlaß verfügt werden; ich werde
dann auch zehntausend Talente Silber in die Hände der Schatzmeister
darwägen können, damit diese sie in die königlichen Schatzhäuser
abführen.« 10Da zog der König seinen Siegelring von der Hand und reichte
ihn dem Agagiten Haman, dem Sohne Hammedathas, dem Judenfeinde, 11indem
er zu ihm sagte: »Das Geld sei dir überlassen und das Volk ebenso: du
magst mit ihm nach deinem Belieben verfahren.«

\hypertarget{c-die-ausrottung-der-juden-im-ganzen-reiche-durch-den-kuxf6nig-angeordnet}{%
\paragraph{c) Die Ausrottung der Juden im ganzen Reiche durch den König
angeordnet}\label{c-die-ausrottung-der-juden-im-ganzen-reiche-durch-den-kuxf6nig-angeordnet}}

12Da wurden die königlichen Staatsschreiber am dreizehnten Tage des
ersten Monats berufen, und es wurde genau nach der Weisung Hamans an die
königlichen Landpfleger und die Statthalter der einzelnen Provinzen und
an die Fürsten eines jeden Volkes mit der Schrift jeder einzelnen
Provinz und in der besonderen Sprache jedes Volkes eine schriftliche
Verfügung im Namen des Königs Ahasveros erlassen und mit dem Siegelringe
des Königs untersiegelt. 13Die Schreiben wurden dann durch die Eilboten
in alle Provinzen des Königs versandt, daß alle Juden, jung und alt,
Kinder und Weiber, an einem Tage, nämlich am dreizehnten des zwölften
Monats -- das ist der Monat Adar --, ausgerottet, ermordet und
umgebracht werden sollten; ihr Vermögen solle der Plünderung
anheimfallen. 14Damit aber die Verfügung in jeder einzelnen Provinz
erlassen würde, ward eine Abschrift des Schreibens allen Völkern
bekanntgemacht, damit sie sich auf den genannten Tag bereithielten.
15Die Eilboten machten sich nach dem Befehl des Königs eilends auf den
Weg, während der Erlaß in der Residenz Susa veröffentlicht wurde. Der
König und Haman aber setzten sich hin, um zu zechen, während in der
Stadt Susa Bestürzung herrschte.

\hypertarget{mardochais-trauer-sein-bemuxfchen-esther-zur-rettung-der-juden-zu-bewegen}{%
\subsubsection{4. Mardochais Trauer; sein Bemühen, Esther zur Rettung
der Juden zu
bewegen}\label{mardochais-trauer-sein-bemuxfchen-esther-zur-rettung-der-juden-zu-bewegen}}

\hypertarget{section-3}{%
\section{4}\label{section-3}}

1Als nun Mardochai alles erfuhr, was vorgegangen war, zerriß er seine
Kleider, legte ein Trauergewand an und (streute sich) Asche (aufs
Haupt), ging dann aus seinem Hause mitten in die Stadt hinein und
wehklagte dabei laut und schmerzlich. 2So kam er bis vor das Tor des
Königs(palastes); denn in das Tor des Palastes selbst durfte man in
einem Trauergewande nicht treten. 3Auch in allen einzelnen Provinzen,
überall, wohin die Verfügung und der Erlaß des Königs gelangte,
herrschte bei den Juden große Trauer; man fastete, weinte und wehklagte;
die meisten saßen in Sackleinen✲ und auf Asche da.

\hypertarget{a-esther-wird-von-mardochai-uxfcber-das-drohende-unheil-aufgekluxe4rt-und-von-ihm-gebeten-beim-kuxf6nig-um-gnade-zu-flehen}{%
\paragraph{a) Esther wird von Mardochai über das drohende Unheil
aufgeklärt und von ihm gebeten, beim König um Gnade zu
flehen}\label{a-esther-wird-von-mardochai-uxfcber-das-drohende-unheil-aufgekluxe4rt-und-von-ihm-gebeten-beim-kuxf6nig-um-gnade-zu-flehen}}

4Als nun die Dienerinnen der Esther und ihre Kammerherren kamen und es
ihr meldeten, geriet die Königin in große Aufregung und sandte dem
Mardochai Kleider, damit er sie anzöge und das Trauergewand ablegte;
aber er nahm sie nicht an. 5Da ließ Esther den Hathach kommen, einen von
den Kammerherren, den der König zu ihrem Dienst bestellt hatte, und gab
ihm den Auftrag, sich bei Mardochai zu erkundigen, was das zu bedeuten
habe und warum es geschehe. 6Als nun Hathach zu Mardochai hinauskam auf
den öffentlichen Platz, der vor dem Tor des königlichen Palastes lag,
7teilte Mardochai ihm alles mit, was ihn betroffen hatte, auch den
bestimmten Betrag der Geldsumme, die Haman als Entgelt für die Ermordung
der Juden an die königliche Schatzkammer zu zahlen versprochen hatte.
8Außerdem übergab er ihm eine Abschrift der in Susa zu ihrer
Niedermetzelung erlassenen schriftlichen Verordnung, damit er sie der
Esther zeige und ihr alles mitteile und sie veranlasse, sich zum Könige
zu begeben, um ihn um Gnade anzuflehen und bei ihm Fürbitte für ihr Volk
einzulegen.

\hypertarget{b-esthers-weigerung-wird-von-mardochai-besiegt-sie-verlangt-jedoch-dauxdf-die-juden-zu-ihren-gunsten-ein-strenges-fasten-abhalten}{%
\paragraph{b) Esthers Weigerung wird von Mardochai besiegt; sie verlangt
jedoch, daß die Juden zu ihren Gunsten ein strenges Fasten
abhalten}\label{b-esthers-weigerung-wird-von-mardochai-besiegt-sie-verlangt-jedoch-dauxdf-die-juden-zu-ihren-gunsten-ein-strenges-fasten-abhalten}}

9Als Hathach nun zurückgekommen war und der Esther die Botschaft
Mardochais mitgeteilt hatte, 10sandte Esther den Hathach nochmals an
Mardochai mit der Meldung: 11»Alle Diener des Königs und die Leute in
den königlichen Provinzen wissen, daß für jedermann, es sei Mann oder
Frau, der zum König in den inneren Hof eintritt, ohne gerufen zu sein,
ein und dasselbe Gesetz gilt, nämlich daß er sterben muß, es sei denn,
daß der König ihm sein goldenes Zepter entgegenstreckt, damit er am
Leben bleibe. Ich aber bin seit nunmehr schon dreißig Tagen nicht zum
König berufen worden.« 12Als man nun Mardochai die Meldung der Esther
mitgeteilt hatte, 13ließ dieser ihr folgende Antwort zukommen: »Bilde
dir nicht ein, daß infolge deiner Zugehörigkeit zum königlichen Hofe du
allein von allen Juden mit dem Leben davonkommen werdest! 14Denn wenn du
wirklich zu dieser Zeit stille sitzen wolltest, so wird den Juden Hilfe
und Rettung von einer andern Seite her erstehen; du aber und deine ganze
Familie, ihr werdet umkommen! Und wer weiß, ob du nicht gerade für eine
Zeit, wie diese ist, zur königlichen Würde gelangt bist?« 15Da ließ
Esther dem Mardochai zurücksagen: 16»Gehe hin, versammle alle Juden, die
sich in Susa befinden, und fastet um meinetwillen, und zwar drei Tage
lang bei Tag und Nacht, ohne zu essen und zu trinken. Auch ich will mit
meinen Dienerinnen ebenso fasten und mich alsdann zum König begeben,
wenn es auch gegen das Gesetz ist. Muß ich dann sterben, nun, so sterbe
ich!« 17Da ging Mardochai weg und tat ganz so, wie Esther ihm angegeben
hatte.

\hypertarget{esthers-freundliche-aufnahme-beim-kuxf6nige-und-hamans-verblendung}{%
\subsubsection{5. Esthers freundliche Aufnahme beim Könige und Hamans
Verblendung}\label{esthers-freundliche-aufnahme-beim-kuxf6nige-und-hamans-verblendung}}

\hypertarget{section-4}{%
\section{5}\label{section-4}}

1Am dritten Tage nun legte Esther königliche Gewandung an und trat in
den inneren Vorhof des königlichen Palastes, dem Palast des Königs
gegenüber, während der König eben im königlichen Palast, dem Eingang zum
Palast gegenüber, auf seinem Königsthrone saß. 2Als nun der König die
Königin Esther im Vorhofe stehen sah, fand sie Gnade vor ihm, so daß er
ihr das goldene Zepter entgegenstreckte, das er in der Hand hielt. Da
trat Esther hinzu und berührte die Spitze des Zepters. 3Darauf sagte der
König zu ihr: »Was ist mit dir, Königin Esther, und was ist dein Begehr?
Wäre es auch die Hälfte des Königreichs -- sie sollte dir gegeben
werden!« 4Da antwortete Esther: »Wenn es dem König genehm ist, so wolle
der König sich heute mit Haman zu dem Gastmahl einfinden, das ich für
ihn bereitet habe!« 5Da befahl der König: »Ruft sofort Haman herbei,
damit wir den Wunsch Esthers erfüllen!«

\hypertarget{a-der-kuxf6nig-mit-haman-von-esther-zum-mahl-geladen-nimmt-eine-nochmalige-einladung-zum-mahl-an}{%
\paragraph{a) Der König, mit Haman von Esther zum Mahl geladen, nimmt
eine nochmalige Einladung zum Mahl
an}\label{a-der-kuxf6nig-mit-haman-von-esther-zum-mahl-geladen-nimmt-eine-nochmalige-einladung-zum-mahl-an}}

Als dann der König sich mit Haman zu dem Mahl eingefunden hatte, das von
Esther zugerichtet worden war, 6sagte der König zu Esther beim
Weintrinken: »Was ist deine Bitte? Sie soll dir gewährt werden! Und was
ist dein Wunsch? Wäre es auch die Hälfte des Königreichs -- es soll
geschehen!« 7Da antwortete Esther mit den Worten: »Meine Bitte und mein
Wunsch ist: 8Wenn ich beim König Gnade gefunden habe und wenn es dem
König genehm ist, mir meine Bitte zu gewähren und meinen Wunsch zu
erfüllen, so wolle der König mit Haman (auch morgen) zu dem Gastmahl
kommen, das ich für sie herrichten will: morgen werde ich dann der
Aufforderung des Königs nachkommen.«

\hypertarget{b-hamans-hochmuxfctige-verblendung-seine-absicht-mardochai-huxe4ngen-zu-lassen}{%
\paragraph{b) Hamans hochmütige Verblendung; seine Absicht, Mardochai
hängen zu
lassen}\label{b-hamans-hochmuxfctige-verblendung-seine-absicht-mardochai-huxe4ngen-zu-lassen}}

9Haman nun begab sich an diesem Tage in fröhlicher Stimmung und guten
Mutes auf den Heimweg; als er aber Mardochai im Tor des Königs(palastes)
erblickte und dieser sich weder erhob noch ihn überhaupt beachtete,
geriet er in volle Wut über Mardochai; 10doch er bezwang sich. Als er
aber nach Hause gekommen war, ließ er seine Freunde und seine Gattin
Seres holen. 11Diesen erzählte er von seinem großartigen Reichtum und
von seinen vielen Söhnen und besonders davon, wie der König ihn
ausgezeichnet und ihn über die Fürsten und Beamten des Königs erhoben
habe. 12Dann fuhr er fort: »Auch hat die Königin Esther zu dem Gastmahl,
das sie ausgerichtet hatte, keinen mit dem König geladen als nur mich;
und auch für morgen bin ich mit dem König von ihr eingeladen. 13Aber das
alles befriedigt mich nicht, solange ich noch den Juden Mardochai im Tor
des Königs(palastes) sitzen sehen muß!« 14Da sagten seine Frau Seres und
alle seine Freunde zu ihm: »Man richte doch einen Pfahl von fünfzig
Ellen Höhe auf; dann sprich morgen früh mit dem König, daß man Mardochai
daran aufhängen möge; danach kannst du vergnügt mit dem König zum
Gastmahl gehen.« Dieser Vorschlag gefiel dem Haman so, daß er den Pfahl
aufrichten ließ.

\hypertarget{mardochai-durch-haman-zu-hohen-ehren-erhoben}{%
\subsubsection{6. Mardochai durch Haman zu hohen Ehren
erhoben}\label{mardochai-durch-haman-zu-hohen-ehren-erhoben}}

\hypertarget{a-des-kuxf6nigs-erlebnis-in-einer-schlaflosen-nacht}{%
\paragraph{a) Des Königs Erlebnis in einer schlaflosen
Nacht}\label{a-des-kuxf6nigs-erlebnis-in-einer-schlaflosen-nacht}}

\hypertarget{section-5}{%
\section{6}\label{section-5}}

1Als in der folgenden Nacht der König nicht schlafen konnte, ließ er
sich das Buch der Denkwürdigkeiten, die Reichschronik, holen; aus dieser
wurde ihm dann vorgelesen. 2Da fand sich darin verzeichnet, daß
Mardochai den Bigthana und Theres, die beiden königlichen Kammerherren
aus der Zahl der Schwellenhüter, zur Anzeige gebracht hatte, weil sie
mit dem Plan umgegangen waren, den König Ahasveros aus dem Wege zu
räumen\textless sup title=``vgl. 2,21-23''\textgreater✲. 3Als nun der
König fragte: »Welche Ehre und Auszeichnung ist dem Mardochai dafür
zuteil geworden?« und die Leibdiener, die den Dienst beim Könige hatten,
ihm die Antwort gaben, es sei ihm gar keine Belohnung zuteil geworden,
4fragte der König weiter: »Wer ist im Vorhofe?« Nun war Haman gerade in
den äußeren Vorhof des königlichen Palastes getreten, um beim König zu
beantragen, man möge Mardochai an den Pfahl hängen, den er für ihn hatte
aufrichten lassen.

\hypertarget{b-haman-veranlauxdft-den-kuxf6nig-unabsichtlich-dazu-fuxfcr-mardochai-eine-auuxdferordentliche-ehrung-zu-beschlieuxdfen-und-diese-durch-ihn-persuxf6nlich-ausfuxfchren-zu-lassen}{%
\paragraph{b) Haman veranlaßt den König unabsichtlich dazu, für
Mardochai eine außerordentliche Ehrung zu beschließen und diese durch
ihn persönlich ausführen zu
lassen}\label{b-haman-veranlauxdft-den-kuxf6nig-unabsichtlich-dazu-fuxfcr-mardochai-eine-auuxdferordentliche-ehrung-zu-beschlieuxdfen-und-diese-durch-ihn-persuxf6nlich-ausfuxfchren-zu-lassen}}

5Als nun die Leibdiener dem König sagten, Haman sei es, der im Vorhof
stehe, befahl der König, er solle eintreten. 6Als nun Haman eingetreten
war, fragte ihn der König: »Was kann man einem Manne tun, den der König
auszuzeichnen wünscht?« Da dachte Haman bei sich: »Wen anders sollte der
König eher auszuzeichnen wünschen als mich?« 7Er antwortete also dem
König: »Wenn der König jemanden auszuzeichnen wünscht, 8so bringe man
ein königliches Gewand herbei, das der König selbst bereits getragen,
und ein Pferd, auf dem der König selbst schon geritten hat und auf
dessen Kopfe die Königskrone angebracht ist; 9man übergebe dann das
Gewand und das Pferd einem der vornehmsten Fürsten des Königs, damit
dieser den Mann, den der König auszuzeichnen wünscht, damit bekleide und
ihn auf dem Pferde über den Hauptplatz der Stadt reiten lasse und dabei
vor ihm her ausrufe: ›So tut man dem Manne, den der König auszuzeichnen
wünscht!‹« 10Da sagte der König zu Haman: »Nimm sofort ein solches
Gewand und das Pferd, so wie du gesagt hast, und mache es so mit dem
Juden Mardochai, der im Tor des Königs(palastes) sitzt! Unterlaß nichts
von allem, was du vorgeschlagen hast!« 11Da holte Haman das
erforderliche Kleid und das Pferd, bekleidete Mardochai (damit), ließ
ihn auf dem Hauptplatz der Stadt umherreiten und rief vor ihm her aus:
»So tut man dem Manne, den der König auszuzeichnen wünscht!«

\hypertarget{c-hamans-betruxfcbnis-mit-buxf6sen-ahnungen-erfuxfcllt-begibt-er-sich-zum-gastmahl-der-kuxf6nigin}{%
\paragraph{c) Hamans Betrübnis; mit bösen Ahnungen erfüllt, begibt er
sich zum Gastmahl der
Königin}\label{c-hamans-betruxfcbnis-mit-buxf6sen-ahnungen-erfuxfcllt-begibt-er-sich-zum-gastmahl-der-kuxf6nigin}}

12Hierauf kehrte Mardochai an das Tor des Königspalastes zurück; Haman
aber eilte traurig und mit verhülltem Haupt nach Hause 13und erzählte
seiner Gattin Seres und seinen sämtlichen Freunden alles, was ihm
widerfahren war. Da sagten seine klugen Freunde und seine Frau Seres zu
ihm: »Wenn Mardochai, vor dem du jetzt zum erstenmal den kürzeren
gezogen hast, ein geborener Jude ist, so wirst du nichts gegen ihn
ausrichten, sondern ihm gegenüber ganz den kürzeren ziehen!« 14Während
sie so noch mit ihm sprachen, erschienen die Kammerherren des Königs und
geleiteten Haman eiligst zu dem Mahl, das Esther zugerichtet hatte.

\hypertarget{hamans-sturz-und-hinrichtung}{%
\subsubsection{7. Hamans Sturz und
Hinrichtung}\label{hamans-sturz-und-hinrichtung}}

\hypertarget{a-esther-eruxf6ffnet-dem-kuxf6nige-beim-mahl-die-mordpluxe4ne-hamans-der-kuxf6nig-erhebt-sich-zornerfuxfcllt-vom-mahl}{%
\paragraph{a) Esther eröffnet dem Könige beim Mahl die Mordpläne Hamans;
der König erhebt sich zornerfüllt vom
Mahl}\label{a-esther-eruxf6ffnet-dem-kuxf6nige-beim-mahl-die-mordpluxe4ne-hamans-der-kuxf6nig-erhebt-sich-zornerfuxfcllt-vom-mahl}}

\hypertarget{section-6}{%
\section{7}\label{section-6}}

1Als nun der König und Haman sich zum festlichen Mahl bei der Königin
Esther begeben hatten, 2sagte der König zu Esther auch am zweiten Tage
beim Weintrinken: »Was ist deine Bitte, Königin Esther? Sie soll dir
gewährt werden! Und was ist dein Wunsch? Wäre es auch die Hälfte des
Königreichs -- es soll geschehen!« 3Da antwortete die Königin Esther mit
den Worten: »Wenn ich Gnade vor dir gefunden habe, o König, und wenn es
dem König genehm ist, so möge mir geschenkt werden mein Leben: das ist
meine Bitte, und mein Volk: das ist mein Wunsch! 4Denn wir sind verkauft
worden, ich und mein Volk, um ausgerottet, ermordet und umgebracht zu
werden! Und wenn wir nur als Sklaven und Sklavinnen verkauft wären, so
würde ich geschwiegen haben, denn unsere Bedrängnis wäre nicht
hinreichend, um den König zu belästigen.« 5Da fragte der König Ahasveros
die Königin Esther: »Wer ist der, und wo ist der, welcher es sich in den
Sinn hat kommen lassen, so zu handeln?« 6Esther erwiderte: »Der
Widersacher und Feind ist der Bösewicht Haman da!« Da fuhr Haman vor dem
König und der Königin erschreckt zusammen; 7der König aber stand in
höchster Erregung vom Weingelage auf und begab sich in den Garten am
Palast, während Haman zurückblieb, um bei der Königin Esther um sein
Leben zu flehen; denn er sah wohl ein, daß sein Verderben beim Könige
fest beschlossene Sache war.

\hypertarget{b-der-kuxf6nig-verurteilt-nach-seiner-ruxfcckkehr-haman-zum-tode-und-luxe4uxdft-ihn-sofort-an-dem-fuxfcr-mardochai-errichteten-pfahl-aufhuxe4ngen}{%
\paragraph{b) Der König verurteilt nach seiner Rückkehr Haman zum Tode
und läßt ihn sofort an dem für Mardochai errichteten Pfahl
aufhängen}\label{b-der-kuxf6nig-verurteilt-nach-seiner-ruxfcckkehr-haman-zum-tode-und-luxe4uxdft-ihn-sofort-an-dem-fuxfcr-mardochai-errichteten-pfahl-aufhuxe4ngen}}

8Als dann der König aus dem Schloßgarten wieder in den Saal, in dem das
Gastmahl stattgefunden, zurückkehrte, hatte Haman sich gerade an dem
Polster\textless sup title=``oder: Divan''\textgreater✲ niedergeworfen,
auf dem Esther lag. Da rief der König aus: »Will er etwa gar der Königin
in meinem eigenen Hause Gewalt antun?« Das Wort war kaum dem Munde des
Königs entfahren, als man auch schon dem Haman das Gesicht verhüllte.
9Da sagte Harbona, einer von den Kammerherren, die den König zu bedienen
hatten: »Es steht ja auch schon der Pfahl, den Haman für Mardochai, den
Wohltäter des Königs, bei seinem Hause hat errichten lassen, fünfzig
Ellen hoch!« Da befahl der König: »Hängt ihn daran!« 10So hängte man
denn Haman an den Pfahl, den er für Mardochai hatte errichten lassen;
darauf legte sich der Zorn des Königs.

\hypertarget{die-vorbereitungen-zur-rettung-der-juden}{%
\subsubsection{8. Die Vorbereitungen zur Rettung der
Juden}\label{die-vorbereitungen-zur-rettung-der-juden}}

\hypertarget{a-esthers-beschenkung-und-mardochais-erhuxf6hung-von-seiten-des-kuxf6nigs}{%
\paragraph{a) Esthers Beschenkung und Mardochais Erhöhung von seiten des
Königs}\label{a-esthers-beschenkung-und-mardochais-erhuxf6hung-von-seiten-des-kuxf6nigs}}

\hypertarget{section-7}{%
\section{8}\label{section-7}}

1An demselben Tage schenkte der König Ahasveros der Königin Esther das
Haus des Judenfeindes Haman; Mardochai aber erhielt Zutritt beim König;
denn Esther hatte dem König mitgeteilt, wie er mit ihr verwandt war.
2Der König zog dann seinen Siegelring, den er dem Haman hatte abnehmen
lassen, vom Finger ab und übergab ihn Mardochai; Esther aber übertrug
dem Mardochai die Verwaltung des Hauses Hamans.

\hypertarget{b-festsetzung-und-bekanntmachung-von-schutzmauxdfregeln-fuxfcr-die-juden-gegen-ihre-feinde}{%
\paragraph{b) Festsetzung und Bekanntmachung von Schutzmaßregeln für die
Juden gegen ihre
Feinde}\label{b-festsetzung-und-bekanntmachung-von-schutzmauxdfregeln-fuxfcr-die-juden-gegen-ihre-feinde}}

3Hierauf verhandelte Esther nochmals mit dem Könige, indem sie sich ihm
zu Füßen warf und ihn unter Tränen anflehte, er möge doch die Bosheit
des Agagiters Haman und den Anschlag, den dieser gegen die Juden ins
Werk gesetzt hatte, rückgängig machen. 4Als der König nun der Esther das
goldene Zepter entgegenstreckte, stand Esther auf, trat vor den König
5und sagte: »Wenn es dem König genehm ist und ich Gnade vor ihm gefunden
habe und es dem König gut erscheint und ich seine Zuneigung besitze, so
möge durch einen schriftlichen Erlaß verordnet werden, daß die auf den
Anschlag des Agagiters Haman, des Sohnes Hammedathas, bezüglichen
Schreiben, in denen er die Ermordung der in allen Provinzen des Reiches
lebenden Juden angeordnet hat, widerrufen werden. 6Denn wie vermöchte
ich das Unglück mit anzusehen, das meine Volksgenossen treffen soll? Und
wie vermöchte ich den Untergang meines Geschlechts mit anzusehen?« 7Da
antwortete der König Ahasveros der Königin Esther und dem Juden
Mardochai: »Wie ihr wißt, habe ich das Haus Hamans der Esther geschenkt,
und ihn selbst hat man an den Pfahl gehängt zur Strafe dafür, daß er
sich an den Juden hat vergreifen wollen. 8So mögt ihr nun im Namen des
Königs in betreff der Juden schriftlich verfügen, wie ihr es für
angemessen haltet, und es dann mit dem Siegelring des Königs
untersiegeln; dagegen eine Verfügung, die schriftlich im Namen des
Königs erlassen und mit dem Siegelring des Königs untersiegelt ist, kann
nicht rückgängig gemacht werden.«

9So wurden denn damals -- es war am dreiundzwanzigsten Tage des dritten
Monats, d.h. des Monats Siwan -- die königlichen Staatsschreiber
berufen, und es wurde genau nach der Anweisung Mardochais in betreff der
Juden an die Landpfleger und Statthalter und Fürsten der
hundertundsiebenundzwanzig Provinzen von Indien bis Äthiopien, und zwar
einer jeden Provinz mit ihrer Schrift und einem jeden Volke in seiner
Sprache, geschrieben, auch an die Juden mit ihrer Schrift und in ihrer
Sprache; 10und zwar ließ er im Namen des Königs Ahasveros schreiben und
mit dem Siegelring des Königs untersiegeln und sandte dann die Schreiben
durch die reitenden Eilboten ab, die auf den vorzüglichsten, aus den
Gestüten stammenden Rennpferden ritten. 11In den Schreiben war verfügt,
daß der König den Juden in allen einzelnen Städten gestatte, sich
zusammenzutun und ihr Leben zu verteidigen, indem sie jedes Aufgebot
eines Volkes oder einer Provinz, das sie angreifen würde, samt ihren
Kindern und Frauen, vernichteten, ermordeten und umbrächten; auch
sollten sie deren Vermögen plündern dürfen, 12und zwar an ein und
demselben Tage in allen Provinzen des Königs Ahasveros, nämlich am
dreizehnten Tage des zwölften Monats, d.h. des Monats Adar. 13Damit aber
die Verfügung in jeder einzelnen Provinz erlassen würde, wurde eine
Abschrift des Schreibens allen Völkern bekanntgemacht, damit die Juden
an dem betreffenden Tage bereit wären, an ihren Feinden Rache zu nehmen.
14Die auf den vorzüglichsten Rennpferden reitenden Eilboten machten sich
dann auf Befehl des Königs schleunigst und in aller Eile auf den Weg,
während die Verfügung in der Residenz Susa veröffentlicht wurde.

\hypertarget{c-mardochai-zeigt-sich-in-susa-in-fuxfcrstlichem-aufzuge-freude-der-juden-im-ganzen-reiche}{%
\paragraph{c) Mardochai zeigt sich in Susa in fürstlichem Aufzuge;
Freude der Juden im ganzen
Reiche}\label{c-mardochai-zeigt-sich-in-susa-in-fuxfcrstlichem-aufzuge-freude-der-juden-im-ganzen-reiche}}

15Mardochai aber trat aus dem Palast des Königs hervor in königlicher
Kleidung von purpurblauer und weißer Baumwolle, mit einem großen
goldenen Stirnreif und einem Mantel von Byssus und rotem Purpur; und die
Stadt Susa jauchzte und freute sich: 16den Juden war Glück und Freude,
Jubel und Ehre zuteil geworden. 17Auch in allen Provinzen und in jeder
Stadt, überall, wohin die Verfügung und der Erlaß des Königs gelangte,
war bei den Juden Freude und Jubel, Festmahl und Feiertag; und viele von
der heidnischen Bevölkerung des Landes traten zum Judentum über, weil
Furcht vor den Juden über sie gekommen war.

\hypertarget{rettung-und-rache-der-juden}{%
\subsubsection{9. Rettung und Rache der
Juden}\label{rettung-und-rache-der-juden}}

\hypertarget{a-ausrottung-der-judenfeinde-am-13.-tage-des-monats-adar-im-ganzen-reiche}{%
\paragraph{a) Ausrottung der Judenfeinde am 13. Tage des Monats Adar im
ganzen
Reiche}\label{a-ausrottung-der-judenfeinde-am-13.-tage-des-monats-adar-im-ganzen-reiche}}

\hypertarget{section-8}{%
\section{9}\label{section-8}}

1Im zwölften Monat nun, d.h. im Monat Adar, am dreizehnten Tage dieses
Monats, als die Verfügung und der Erlaß des Königs zur Ausführung kommen
sollte, an eben dem Tage, an welchem die Feinde der Juden gehofft
hatten, sie zu überwältigen -- die Sache wandte sich aber so, daß die
Juden ihrerseits ihre Feinde überwältigten --, 2da taten sich die Juden
in ihren Städten in allen Provinzen des Königs Ahasveros zusammen, um
Hand an die zu legen, welche ihnen Unheil zuzufügen gedacht hatten; und
niemand konnte ihnen Widerstand leisten, denn die Furcht vor ihnen hatte
sich aller Völker bemächtigt. 3Auch alle Fürsten in den Provinzen sowie
die Landpfleger und Statthalter und Beamten des Königs gewährten den
Juden Unterstützung, weil die Furcht vor Mardochai sie befallen hatte;
4denn Mardochai stand am königlichen Hofe groß da, und sein Ruf war in
alle Provinzen gedrungen; denn Mardochais Ansehen war unaufhaltsam im
Steigen. 5So richteten denn die Juden unter allen ihren Feinden mit dem
Schwerte, durch Morden und Niederhauen, ein Blutbad an und gingen gegen
ihre Widersacher nach Herzenslust vor. 6Auch in der Residenz Susa
mordeten die Juden und brachten fünfhundert Mann ums Leben; 7dazu
töteten sie Parsandatha, Dalphon, Aspatha, 8Poratha, Adalja, Aridatha,
9Parmastha, Arisai, Aridai und Wajesatha, 10die zehn Söhne Hamans, des
Sohnes Hammedathas, des Judenverfolgers; aber fremdes Hab und Gut
rührten sie nicht an.

\hypertarget{b-fortsetzung-des-gemetzels-am-14.-tage-des-monats-freudenfest-der-juden-zur-feier-ihrer-rettung}{%
\paragraph{b) Fortsetzung des Gemetzels am 14. Tage des Monats;
Freudenfest der Juden zur Feier ihrer
Rettung}\label{b-fortsetzung-des-gemetzels-am-14.-tage-des-monats-freudenfest-der-juden-zur-feier-ihrer-rettung}}

11An demselben Tage kam die Zahl der in der Residenz Susa Getöteten zur
Kenntnis des Königs. 12Da sagte der König zu der Königin Esther: »In der
Residenz Susa haben die Juden gemordet und fünfhundert Mann ums Leben
gebracht, auch die zehn Söhne Hamans; was mögen sie da wohl in den
übrigen Provinzen des Reiches angerichtet haben? Doch was ist deine
Bitte? Sie soll dir gewährt werden. Und was ist weiter noch dein Wunsch?
Er soll erfüllt werden.« 13Da antwortete Esther: »Wenn es dem König
genehm ist, so möge auch morgen noch den Juden in Susa gestattet sein,
in derselben Weise wie heute zu verfahren; die zehn Söhne Hamans aber
möge man an den Pfahl hängen.« 14Da gebot der König, daß so verfahren
werden sollte; und der betreffende Befehl wurde in Susa erlassen, und
die zehn Söhne Hamans wurden aufgehängt. 15So taten sich denn die Juden
in Susa auch am vierzehnten Tage des Monats Adar zusammen und brachten
in Susa noch dreihundert Mann um; aber fremdes Hab und Gut rührten sie
nicht an.

16Auch die übrigen Juden, die in den Provinzen des Reiches wohnten,
hatten sich zusammengetan, um ihr Leben zu verteidigen und sich Ruhe vor
ihren Feinden zu verschaffen; sie hatten 75000 von ihren Feinden
umgebracht, ohne jedoch fremdes Hab und Gut anzurühren. 17Das war am
dreizehnten Tage des Monats Adar geschehen, aber am vierzehnten Tage des
Monats hatten sie sich ruhig verhalten und ihn zu einem Tage der
Festgelage\textless sup title=``oder: des Schmausens''\textgreater✲ und
der Freude gemacht. 18In Susa dagegen hatten sich die Juden sowohl am
dreizehnten als auch am vierzehnten Tage dieses Monats zusammengetan und
erst am fünfzehnten Tage Ruhe gehalten und diesen Tag zu einem Tage der
Festgelage\textless sup title=``oder: des Schmausens''\textgreater✲ und
der Freude gemacht. 19Darum feiern die Juden auf dem Lande, die in den
offenen Ortschaften wohnen, den vierzehnten Tag des Monats Adar als
einen Tag der Freude und der Schmausereien und als einen Festtag, an dem
man sich gegenseitig (leckere) Gerichte zusendet.

\hypertarget{einsetzung-des-purimfestes-mardochais-gruxf6uxdfe}{%
\subsubsection{10. Einsetzung des Purimfestes; Mardochais
Größe}\label{einsetzung-des-purimfestes-mardochais-gruxf6uxdfe}}

\hypertarget{a-mardochai-ordnet-die-feier-des-purimfestes-fuxfcr-alle-zukunft-an}{%
\paragraph{a) Mardochai ordnet die Feier des Purimfestes für alle
Zukunft
an}\label{a-mardochai-ordnet-die-feier-des-purimfestes-fuxfcr-alle-zukunft-an}}

20Mardochai schrieb hierauf diese Begebenheiten auf und sandte Schreiben
an alle Juden in allen Provinzen des Königs Ahasveros, die nahen und die
fernen, 21um sie zu dem Brauche zu verpflichten, daß sie sei es den
vierzehnten, sei es den fünfzehnten Tag des Monats Adar Jahr für Jahr
feierten 22als die Tage, an denen die Juden Ruhe vor ihren Feinden
erlangt hatten, und als den Monat, in dem sich der Kummer für sie in
Freude verwandelt hatte und die Trauer in einen Festtag, so daß sie
diese (Tage) feierten als Tage der Festgelage und der Freude, an denen
man sich gegenseitig (leckere) Gerichte zusendet und die Armen
beschenkt. 23So nahmen denn die Juden das, was sie damals zum erstenmal
getan und was Mardochai ihnen brieflich geraten hatte, als stehenden
Brauch an. 24Weil der Agagiter Haman, der Sohn Hammedathas, der Feind
aller Juden, den Plan gegen die Juden gefaßt hatte, sie zu vernichten,
und weil er das Pur, d.h. das Los, hatte werfen lassen, um sie zu
verderben und auszurotten, 25der König aber, als Esther vor ihn getreten
war, durch einen schriftlichen Erlaß angeordnet hatte, daß sein
boshafter Anschlag, den er gegen die Juden ersonnen hatte, auf sein
eigenes Haupt zurückfallen und daß man ihn und seine Söhne an den Pfahl
hängen solle: 26darum hat man diese Tage ›Purim‹ genannt nach dem Worte
Pur. Aus diesen Gründen also -- wegen des gesamten Inhalts dieses
Briefes und wegen alles dessen, was sie selbst erlebt oder was durch
andere zu ihrer Kenntnis gekommen war~-- 27ordneten die Juden an und
setzten für sich und ihre Nachkommen und für alle, die sich ihnen
anschließen würden, als unumstößliche Satzung fest, diese beiden Tage in
der für sie vorgeschriebenen Weise und zu der für sie bestimmten Zeit
Jahr für Jahr festlich zu begehen; 28und diese beiden Tage sollten im
Gedächtnis festgehalten und von Geschlecht zu Geschlecht in allen
Familien, Provinzen und Ortschaften gefeiert werden, so daß diese
Purimtage unter den Juden niemals in Abgang kämen und die Erinnerung an
sie bei ihren Nachkommen niemals verschwände.

\hypertarget{b-zweiter-purimbrief-esthers-und-mardochais-an-die-juden}{%
\paragraph{b) Zweiter Purimbrief Esthers und Mardochais an die
Juden}\label{b-zweiter-purimbrief-esthers-und-mardochais-an-die-juden}}

29Weiter schrieb die Königin Esther, die Tochter Abihails, mit allem
Nachdruck, um dieses Purimschreiben zur Geltung zu bringen, 30und sandte
Briefe an alle Juden in die hundertundsiebenundzwanzig Provinzen, in das
ganze Königreich des Ahasveros, freundliche und treugemeinte Worte, 31um
die Feier dieser Purimtage in betreff der für sie bestimmten Zeiten zur
festen Satzung zu erheben, ganz so wie der Jude Mardochai und die
Königin Esther es für sie angeordnet und wie sie es auch für sich selbst
und für ihre Nachkommen festgesetzt hatten, nämlich die Vorschriften
über die Fasten und ihre Wehklage. 32Der Erlaß Esthers erhob diese
Purimvorschriften zum Gesetz\textless sup title=``oder: zu fester
Satzung''\textgreater✲ und wurde in einer Urkunde aufgezeichnet.

\hypertarget{c-die-machtstellung-und-die-verdienste-mardochais-um-das-wohl-der-juden}{%
\paragraph{c) Die Machtstellung und die Verdienste Mardochais um das
Wohl der
Juden}\label{c-die-machtstellung-und-die-verdienste-mardochais-um-das-wohl-der-juden}}

\hypertarget{section-9}{%
\section{10}\label{section-9}}

1Der König Ahasveros legte dann dem Festlande und den Inseln des Meeres
eine Abgabe auf. 2Und alle Erweise seines gewaltigen und machtvollen
Wirkens und die genaue Schilderung der hohen Stellung Mardochais, zu der
ihn der König erhob, das alles findet sich bekanntlich aufgezeichnet im
Buch der Denkwürdigkeiten\textless sup title=``oder:
Chronik''\textgreater✲ der medischen und persischen Könige. 3Denn der
Jude Mardochai war der Erste im Range nach dem König Ahasveros und
hochgeachtet bei den Juden und beliebt bei der Menge seiner
Volksgenossen, weil er das Beste seines Volkes suchte und für das Wohl
seines ganzen Stammes eintrat.
