\hypertarget{der-brief-des-jakobus}{%
\section{DER BRIEF DES JAKOBUS}\label{der-brief-des-jakobus}}

\hypertarget{zuschrift-und-gruuxdf}{%
\subsubsection{Zuschrift und Gruß}\label{zuschrift-und-gruuxdf}}

\hypertarget{section}{%
\section{1}\label{section}}

\bibleverse{1} Ich, Jakobus, ein Knecht Gottes und des Herrn Jesus
Christus, sende den zwölf in der Zerstreuung (unter den Heiden) lebenden
Stämmen meinen Gruß.

\hypertarget{vom-rechten-verhalten-in-versuchungen-d.h.-anfechtungen-mahnung-zu-rechter-gesinnung-belehrung-uxfcber-versuchungen}{%
\subsubsection{1. Vom rechten Verhalten in Versuchungen (d.h.
Anfechtungen); Mahnung zu rechter Gesinnung; Belehrung über
Versuchungen}\label{vom-rechten-verhalten-in-versuchungen-d.h.-anfechtungen-mahnung-zu-rechter-gesinnung-belehrung-uxfcber-versuchungen}}

\hypertarget{a-zweck-und-segen-der-anfechtungen}{%
\paragraph{a) Zweck und Segen der
Anfechtungen}\label{a-zweck-und-segen-der-anfechtungen}}

\bibleverse{2} Erachtet es für lauter Freude, meine Brüder, wenn ihr in
mancherlei Versuchungen\textless sup title=``d.h. Prüfungen,
Anfechtungen''\textgreater✲ geratet; \bibleverse{3} ihr erkennt ja, daß
die Bewährung eures Glaubens standhaftes Ausharren\textless sup
title=``oder: Geduld''\textgreater✲ bewirkt. \bibleverse{4} Das
standhafte Ausharren muß aber zu voller Betätigung führen, damit ihr
vollkommen und tadellos seid und sich in keiner Beziehung ein Mangel an
euch zeigt.

\hypertarget{b-mahnung-zu-ausdauernder-bitte-um-weisheit}{%
\paragraph{b) Mahnung zu ausdauernder Bitte um
Weisheit}\label{b-mahnung-zu-ausdauernder-bitte-um-weisheit}}

\bibleverse{5} Sollte aber jemand von euch Mangel an Weisheit haben, so
erbitte er sie sich von Gott, der allen ohne weiteres und ohne laute
Vorwürfe\textless sup title=``=~barsche Abweisung''\textgreater✲ gibt:
dann wird sie ihm zuteil werden. \bibleverse{6} Nur bitte er im
Glauben\textless sup title=``oder: mit Zuversicht''\textgreater✲, ohne
irgendeinen Zweifel zu hegen; denn wer da zweifelt, der gleicht einer
vom Wind getriebenen und hin und her geworfenen Meereswoge.
\bibleverse{7} Ein solcher Mensch darf nicht erwarten, daß er etwas vom
Herrn empfangen werde, \bibleverse{8} er, ein Mann mit zwei
Seelen\textless sup title=``=~mit geteiltem Herzen''\textgreater✲,
unbeständig auf allen seinen Wegen\textless sup title=``d.h. in seiner
ganzen Lebensführung''\textgreater✲.

\hypertarget{c-die-rechte-gesinnung-bei-armut-und-reichtum-segen-der-bewuxe4hrung}{%
\paragraph{c) Die rechte Gesinnung bei Armut und Reichtum; Segen der
Bewährung}\label{c-die-rechte-gesinnung-bei-armut-und-reichtum-segen-der-bewuxe4hrung}}

\bibleverse{9} Es rühme sich aber der niedrig stehende Bruder seiner
Höhe\textless sup title=``oder: Hoheit''\textgreater✲, \bibleverse{10}
der reiche dagegen seiner Niedrigkeit✲, weil er wie die Blumen des
Grases vergehen wird. \bibleverse{11} Denn die Sonne geht mit ihrer Glut
auf und versengt das Gras; dann fallen seine Blumen ab, und seine ganze
Schönheit ist dahin\textless sup title=``Jes 40,6-7''\textgreater✲: so
wird auch der Reiche in seinen Wegen\textless sup title=``=~in seinen
Unternehmungen''\textgreater✲ verwelken.~-- \bibleverse{12} Selig ist
der Mann, der die Versuchung✲ standhaft erträgt! Denn nachdem er sich
bewährt hat, wird er das Leben als Siegeskranz empfangen, den
er\textless sup title=``d.h. Gott''\textgreater✲ denen verheißen hat,
die ihn lieben.

\hypertarget{d-versuchungen-zum-buxf6sen-stammen-von-der-eigenen-lust-nicht-von-gott-dem-urquell-alles-guten}{%
\paragraph{d) Versuchungen zum Bösen stammen von der eigenen Lust, nicht
von Gott, dem Urquell alles
Guten}\label{d-versuchungen-zum-buxf6sen-stammen-von-der-eigenen-lust-nicht-von-gott-dem-urquell-alles-guten}}

\bibleverse{13} Niemand sage\textless sup title=``oder:
meine''\textgreater✲, wenn er (zum Bösen) versucht wird: »Von Gott werde
ich versucht«; denn Gott kann nicht vom Bösen\textless sup title=``oder:
zum Bösen''\textgreater✲ versucht werden, versucht aber auch seinerseits
niemand. \bibleverse{14} Nein, ein jeder wird (zum Bösen) versucht,
indem er von seiner eigenen Lust\textless sup title=``oder:
Begierde''\textgreater✲ gereizt und gelockt wird. \bibleverse{15}
Sodann, wenn die Lust empfangen hat\textless sup title=``=~befruchtet
ist''\textgreater✲, gebiert sie Sünde; die Sünde aber gebiert, wenn sie
zur Vollendung gekommen ist, den Tod.~-- \bibleverse{16} Irret euch
nicht, meine geliebten Brüder: \bibleverse{17} lauter gute Gabe und
lauter vollkommenes Geschenk kommt von oben herab, vom Vater der
Himmelslichter, bei dem keine Veränderung und keine zeitweilige
Verdunkelung stattfindet. \bibleverse{18} Aus freiem Liebeswillen hat er
uns durch das Wort der Wahrheit ins Dasein gerufen\textless sup
title=``oder: neu geboren''\textgreater✲, damit wir gewissermaßen die
Erstlingsfrucht unter seinen Geschöpfen wären.

\hypertarget{mahnung-den-gluxe4ubigen-stand-durch-bewahren-des-wortes-und-durch-gute-glaubens-werke-zu-beweisen}{%
\subsubsection{2. Mahnung, den gläubigen Stand durch Bewahren des Wortes
und durch gute (Glaubens-)Werke zu
beweisen}\label{mahnung-den-gluxe4ubigen-stand-durch-bewahren-des-wortes-und-durch-gute-glaubens-werke-zu-beweisen}}

\hypertarget{a-seid-nicht-nur-huxf6rer-sondern-auch-tuxe4ter-des-wortes}{%
\paragraph{a) Seid nicht nur Hörer, sondern auch Täter des
Wortes}\label{a-seid-nicht-nur-huxf6rer-sondern-auch-tuxe4ter-des-wortes}}

\bibleverse{19} Wisset\textless sup title=``=~laßt es euch gesagt
sein''\textgreater✲, meine geliebten Brüder: es sei {[}aber{]} jeder
Mensch schnell (bereit) zum Hören, langsam zum Reden und langsam zum
Zorn; \bibleverse{20} denn der Zorn des Menschen tut nichts, was vor
Gott recht ist. \bibleverse{21} Darum legt alle
Unsauberkeit\textless sup title=``=~schmutzige Gesinnung''\textgreater✲
und den letzten Rest der Bosheit ab, und nehmt mit Sanftmut das euch
eingepflanzte Wort an, das eure Seelen zu retten vermag. \bibleverse{22}
Seid aber Täter des Wortes und nicht bloß Hörer, sonst betrügt ihr euch
selbst. \bibleverse{23} Denn wer nur ein Hörer des Wortes ist, aber kein
Täter, der gleicht einem Menschen, der sein leibliches Gesicht im
Spiegel beschaut; \bibleverse{24} denn nachdem er sich beschaut hat und
weggegangen ist, vergißt er alsbald, wie er ausgesehen hat.
\bibleverse{25} Wer dagegen in das vollkommene Gesetz der Freiheit
hineingeschaut hat und bei ihm verbleibt, indem er nicht ein
vergeßlicher Hörer, sondern ein wirklicher Täter\textless sup
title=``oder: ein Täter des Werkes Gottes''\textgreater✲ ist, der wird
in seinem Tun selig sein.

\hypertarget{einige-beispiele-fuxfcr-das-tun-der-rechten-werke}{%
\paragraph{Einige Beispiele für das Tun der rechten
Werke}\label{einige-beispiele-fuxfcr-das-tun-der-rechten-werke}}

\bibleverse{26} Wenn jemand Gott zu dienen meint und dabei seine Zunge
nicht im Zaume hält, vielmehr sein Herz\textless sup title=``=~sich
selbst''\textgreater✲ betrügt, dessen Gottesdienst\textless sup
title=``oder: Frömmigkeit''\textgreater✲ ist nichtig. \bibleverse{27}
Ein reiner und fleckenloser Gottesdienst vor Gott dem Vater besteht
darin, daß man Waisen und Witwen in ihrer Trübsal besucht und sich
selbst von der Welt unbefleckt erhält.

\hypertarget{b-huxfctet-euch-vor-dem-ansehen-der-person-besonders-vor-verachtung-der-armen}{%
\paragraph{b) Hütet euch vor dem Ansehen der Person, besonders vor
Verachtung der
Armen}\label{b-huxfctet-euch-vor-dem-ansehen-der-person-besonders-vor-verachtung-der-armen}}

\hypertarget{section-1}{%
\section{2}\label{section-1}}

\bibleverse{1} Meine Brüder, habt den Glauben an unsern Herrn Jesus
Christus, (den Herrn) der Herrlichkeit, nicht so, daß Ansehen der
Person✲ damit verbunden ist. \bibleverse{2} Wenn z.B. in eure
(gottesdienstliche) Versammlung ein Mann mit goldenen Ringen an den
Fingern und in prächtiger Kleidung tritt und zugleich ein armer in
unsauberem Anzug erscheint, \bibleverse{3} und ihr dann eure Blicke auf
den prächtig Gekleideten richtet und zu ihm sagt: »Setze du dich hierher
auf den guten Platz«, während ihr zu dem Armen sagt: »Stelle du dich
dorthin oder setze dich hier unten auf meinen Fußschemel!«~--
\bibleverse{4} seid ihr da nicht in Zwiespalt\textless sup title=``oder:
Widerspruch''\textgreater✲ mit euch selbst geraten und zu Richtern mit
bösen Erwägungen\textless sup title=``oder:
Hintergedanken''\textgreater✲ geworden? \bibleverse{5} Hört (mich an),
meine geliebten Brüder! Hat Gott nicht gerade die, welche für die
Welt\textless sup title=``=~in den Augen der Welt''\textgreater✲ arm
sind, dazu erwählt, reich im\textless sup title=``oder:
durch''\textgreater✲ Glauben und Erben des Reiches zu sein, das er denen
verheißen hat, die ihn lieben? \bibleverse{6} Ihr aber habt den Armen
mißachtet. Sind es nicht gerade die Reichen, die euch gewalttätig
behandeln, und ziehen nicht gerade sie euch vor die Gerichte?
\bibleverse{7} Sind nicht gerade sie es, die den guten✲ Namen lästern,
der (bei der Taufe) über euch angerufen\textless sup title=``oder:
ausgesprochen''\textgreater✲ worden ist?

\hypertarget{die-erfuxfcllung-des-mosaischen-gesetzes-muuxdf-einheitlich-d.h.-ausnahmslos-sein}{%
\paragraph{Die Erfüllung des mosaischen Gesetzes muß einheitlich (d.h.
ausnahmslos)
sein}\label{die-erfuxfcllung-des-mosaischen-gesetzes-muuxdf-einheitlich-d.h.-ausnahmslos-sein}}

\bibleverse{8} Allerdings\textless sup title=``oder:
gewiß''\textgreater✲, wenn ihr das königliche Gesetz nach dem
Schriftwort erfüllt\textless sup title=``3.Mose 19,18''\textgreater✲:
»Du sollst deinen Nächsten lieben wie dich selbst«, so tut ihr recht
daran; \bibleverse{9} wenn ihr aber die Person anseht, so begeht ihr
Sünde und werdet vom Gesetz als Übertreter überführt\textless sup
title=``oder: erwiesen''\textgreater✲. \bibleverse{10} Denn wer das
ganze Gesetz erfüllt, aber gegen ein einziges Gebot verstößt, der hat
sich damit gegen das ganze (Gesetz) vergangen. \bibleverse{11} Denn der
da geboten hat: »Du sollst nicht ehebrechen«, der hat auch geboten: »Du
sollst nicht töten.« Wenn du nun zwar kein Ehebrecher bist, wohl aber
ein Mörder, so bist du ein Übertreter des (ganzen) Gesetzes geworden.
\bibleverse{12} Redet so und handelt so wie Leute, die (einst) durch das
Gesetz der Freiheit gerichtet werden sollen. \bibleverse{13} Denn das
Gericht verfährt erbarmungslos gegen den, der kein Erbarmen geübt hat;
die Barmherzigkeit dagegen rühmt sich gegen das Gericht\textless sup
title=``=~erweist sich dem Gericht überlegen, oder: triumphiert über das
Gericht''\textgreater✲.

\hypertarget{c-der-glaube-ohne-werke-ist-tot-und-nutzlos-wahrer-glaube-erweist-sich-in-der-selbstverleugnung-und-in-guten-werken}{%
\paragraph{c) Der Glaube ohne Werke ist tot und nutzlos; wahrer Glaube
erweist sich in der Selbstverleugnung und in guten
Werken}\label{c-der-glaube-ohne-werke-ist-tot-und-nutzlos-wahrer-glaube-erweist-sich-in-der-selbstverleugnung-und-in-guten-werken}}

\bibleverse{14} Was nützt es, meine Brüder, wenn jemand behauptet,
Glauben zu besitzen, dabei aber keine Werke (aufzuweisen) hat? Vermag
etwa der Glaube ihn zu retten? \bibleverse{15} Wenn z.B. ein Bruder oder
eine Schwester keine Kleidung hat und an der täglichen Nahrung Mangel
leidet \bibleverse{16} und dann jemand von euch zu ihnen sagt: »Geht hin
in Frieden, wärmt euch\textless sup title=``=~kleidet euch
warm''\textgreater✲ und eßt euch satt!«, ohne ihnen jedoch das zu geben,
was ihr Leib bedarf: welchen Nutzen hat das für sie? \bibleverse{17} So
steht es auch mit dem Glauben: hat er keine Werke (aufzuweisen), so ist
er an sich selbst\textless sup title=``=~für sich allein''\textgreater✲
tot. \bibleverse{18} Doch es wird jemand einwenden: »Du hast Glauben,
und ich habe Werke; weise mir deinen Glauben nach, der ohne Werke ist,
und ich will dir aus meinen Werken den Glauben nachweisen!«
\bibleverse{19} Du glaubst, daß es nur einen Gott gibt? Du tust recht
daran; aber das glauben auch die Teufel\textless sup title=``=~die bösen
Geister''\textgreater✲ und -- schaudern dabei. \bibleverse{20} Willst du
wohl einsehen, du gedankenloser Mensch, daß der Glaube ohne die Werke
unnütz\textless sup title=``oder: wertlos''\textgreater✲ ist?

\hypertarget{zwei-alttestamentliche-beispiele-als-schriftbeweis-fuxfcr-die-werke-die-zur-glaubensvollendung-fuxfchren}{%
\paragraph{Zwei alttestamentliche Beispiele als Schriftbeweis für die
Werke, die zur Glaubensvollendung
führen}\label{zwei-alttestamentliche-beispiele-als-schriftbeweis-fuxfcr-die-werke-die-zur-glaubensvollendung-fuxfchren}}

\bibleverse{21} Ist nicht unser Vater Abraham aus Werken\textless sup
title=``=~aufgrund von Werken''\textgreater✲ gerechtfertigt worden, da
er seinen Sohn Isaak auf dem Opferaltar darbrachte? \bibleverse{22}
Daran siehst du, daß der Glaube mit seinen Werken zusammengewirkt hat
und der Glaube erst durch die Werke zur Vollendung✲ gebracht ist,
\bibleverse{23} und daß so erst das Schriftwort sich erfüllt hat, das da
lautet\textless sup title=``1.Mose 15,6''\textgreater✲: »Abraham glaubte
aber Gott, und das wurde ihm als Gerechtigkeit angerechnet«, und er
wurde ›Gottes Freund‹ genannt\textless sup title=``Jes
41,8''\textgreater✲. \bibleverse{24} So seht ihr, daß der Mensch aus
Werken gerechtfertigt wird und nicht aus Glauben allein. \bibleverse{25}
Ist nicht ebenso auch die Dirne Rahab aufgrund von Werken gerechtfertigt
worden, weil sie die Kundschafter bei sich aufgenommen und sie auf einem
anderen Wege wieder (aus dem Hause) entlassen hatte? \bibleverse{26}
Denn ebenso wie der Leib ohne Geist tot ist, ebenso ist auch der Glaube
ohne Werke tot.

\hypertarget{warnung-vor-unberufenem-zudrang-zum-lehreramt-und-vor-den-zungensuxfcnden}{%
\subsubsection{3. Warnung vor unberufenem Zudrang zum Lehreramt und vor
den
Zungensünden}\label{warnung-vor-unberufenem-zudrang-zum-lehreramt-und-vor-den-zungensuxfcnden}}

\hypertarget{section-2}{%
\section{3}\label{section-2}}

\bibleverse{1} Drängt euch nicht zum Lehrerberuf, meine Brüder! Bedenkt
wohl, daß wir (Lehrer) ein um so strengeres Urteil\textless sup
title=``oder: Gericht''\textgreater✲ empfangen werden\textless sup
title=``=~zu erwarten haben''\textgreater✲. \bibleverse{2} Wir fehlen ja
allesamt vielfach; wer sich beim Reden nicht verfehlt, der ist ein
vollkommener Mann und vermag auch den ganzen Leib im Zaume zu halten.
\bibleverse{3} Wenn wir den Pferden die Zäume ins Maul legen, um sie uns
gehorsam zu machen, so haben wir damit auch ihren ganzen Leib in der
Gewalt. \bibleverse{4} Seht, auch die Schiffe, die doch so groß sind und
von starken Winden getrieben werden, lassen sich durch ein ganz kleines
Steuerruder dahin lenken, wohin das Belieben des Steuermannes sie haben
will. \bibleverse{5} So ist auch die Zunge nur ein kleines Glied und
kann sich doch großer Dinge✲ rühmen. Seht, wie klein ist das Feuer und
wie groß der Wald, den es in Brand setzt! \bibleverse{6} Auch die Zunge
ist ein Feuer; als der Inbegriff der Ungerechtigkeit steht die Zunge
unter unsern Gliedern da, sie, die den ganzen Leib befleckt, die sowohl
das (rollende) Rad des Seins\textless sup title=``d.h. den ganzen Lauf
des Lebens =~die ganze Lebensbahn''\textgreater✲ in Brand
setzt\textless sup title=``=~zur Hölle macht''\textgreater✲ als auch
(selbst) von der Hölle in Brand gesetzt wird. \bibleverse{7} Denn jede
Art der vierfüßigen Tiere\textless sup title=``oder: wilden
Landtiere''\textgreater✲ und Vögel, der Schlangen und Seetiere wird von
der menschlichen Natur gebändigt und ist von ihr gebändigt worden;
\bibleverse{8} aber die Zunge vermag kein Mensch zu bändigen, dies
ruhelose Übel, voll todbringenden Giftes. \bibleverse{9} Mit ihr segnen✲
wir den Herrn und Vater, und mit ihr fluchen wir den Menschen, die doch
nach Gottes Bild geschaffen sind: \bibleverse{10} aus demselben Munde
gehen Segen und Fluch hervor. Das darf nicht so sein, meine Brüder.
\bibleverse{11} Läßt etwa eine Quelle aus derselben Öffnung süßes und
bitteres Wasser sprudeln? \bibleverse{12} Kann etwa, meine Brüder, ein
Feigenbaum Oliven tragen oder ein Weinstock Feigen? Ebensowenig kann
eine Salzquelle süßes Wasser geben.

\hypertarget{von-der-falschen-seelisch-irdischen-und-der-wahren-geistlich-himmlischen-weisheit}{%
\subsubsection{4. Von der falschen, seelisch-irdischen, und der wahren,
geistlich-himmlischen
Weisheit}\label{von-der-falschen-seelisch-irdischen-und-der-wahren-geistlich-himmlischen-weisheit}}

\bibleverse{13} Wer ist weise und einsichtsvoll unter euch? Der beweise
durch seinen guten Wandel seine Werke\textless sup title=``oder: was er
leisten kann''\textgreater✲ in sanftmütiger Weisheit\textless sup
title=``oder: durch die Gelassenheit eines Weisen''\textgreater✲!
\bibleverse{14} Wenn ihr aber bittere Eifersucht und
Zanksucht\textless sup title=``oder: Rechthaberei''\textgreater✲ in
eurem Herzen hegt, so rühmt euch nicht lügnerisch im Widerspruch mit der
Wahrheit. \bibleverse{15} Das ist nicht die Weisheit, die von oben her
kommt, sondern ist eine irdische, sinnliche\textless sup title=``oder:
natürliche''\textgreater✲, teuflische. \bibleverse{16} Denn wo
Eifersucht und Zanksucht\textless sup title=``oder:
Rechthaberei''\textgreater✲ herrschen, da gibt's Unfrieden\textless sup
title=``oder: Unordnung''\textgreater✲ und alle Arten bösen Tuns.
\bibleverse{17} Die Weisheit dagegen, die von oben kommt, ist fürs erste
lauter✲, sodann friedfertig✲, freundlich, nachgiebig, reich an Erbarmen
und guten Früchten, frei von Zweifel✲ und ohne Heuchelei.
\bibleverse{18} (Der Same) aber, (der) die Frucht der Gerechtigkeit
(hervorbringt), wird in Frieden für die\textless sup title=``oder: von
denen''\textgreater✲ gesät, die Frieden stiften✲.

\hypertarget{ermahnungen}{%
\subsubsection{5. Ermahnungen}\label{ermahnungen}}

\hypertarget{a-warnungen-vor-unfrieden-unzufriedenheit-und-weltsinn-vor-schmuxe4hsucht-und-lieblosem-richten}{%
\paragraph{a) Warnungen vor Unfrieden, Unzufriedenheit und Weltsinn, vor
Schmähsucht und lieblosem
Richten}\label{a-warnungen-vor-unfrieden-unzufriedenheit-und-weltsinn-vor-schmuxe4hsucht-und-lieblosem-richten}}

\hypertarget{section-3}{%
\section{4}\label{section-3}}

\bibleverse{1} Woher kommen die Kämpfe und woher die Streitigkeiten bei
euch? Doch wohl daher, daß eure Lüste einen Kampf in euren Gliedern
führen? \bibleverse{2} Ihr seid begehrlich -- und gelangt doch nicht zum
Besitz; ihr mordet\textless sup title=``=~haßt auf den
Tod''\textgreater✲ und seid neidisch, ohne doch eure Wünsche erfüllt zu
sehen; ihr lebt in Kampf und Streitigkeiten und gelangt doch nicht zum
Besitz, weil ihr nicht betet; \bibleverse{3} ihr betet wohl, empfangt
aber nichts, weil ihr in böser Absicht betet, nämlich um (das Erbetene)
in euren Lüsten wieder durchzubringen. \bibleverse{4} Ihr
gottabtrünnigen Seelen! Wißt ihr nicht, daß die Freundschaft mit der
Welt Feindschaft gegen Gott ist? Wer also ein Freund der Welt sein will,
erweist sich als Feind Gottes. \bibleverse{5} Oder meint ihr, die
Schrift mache leere Worte, wenn sie sagt: »Eifersüchtiges Verlangen hegt
der Geist, den er\textless sup title=``d.h. Gott''\textgreater✲ Wohnung
in uns hat nehmen lassen«? \bibleverse{6} Um so reicher ist aber die
Gnade, die er zuteilt. Darum heißt es\textless sup title=``Spr
3,34''\textgreater✲: »Gott widersteht den Hoffärtigen, den Demütigen
aber gibt er Gnade.« \bibleverse{7} Unterwerft euch also Gott und
widersteht dem Teufel, so wird er von euch fliehen. \bibleverse{8} Nahet
euch zu Gott, so wird er sich zu euch nahen; reinigt euch die Hände, ihr
Sünder, und heiligt eure Herzen, ihr Doppelherzigen\textless sup
title=``=~Menschen mit geteiltem Sinn; vgl. 1,8''\textgreater✲!
\bibleverse{9} Fühlt euer Elend, trauert und weint! Euer Lachen
verwandle sich in Traurigkeit und eure Freude in Betrübnis!
\bibleverse{10} Demütigt euch vor dem Herrn, so wird er euch erhöhen!

\bibleverse{11} Redet nicht feindselig gegeneinander, liebe Brüder! Wer
feindselig gegen seinen Bruder redet oder seinen Bruder richtet, der
redet feindselig gegen das Gesetz und richtet das Gesetz; wenn du aber
das Gesetz richtest, so bist du nicht ein Täter des Gesetzes, sondern
ein Richter\textless sup title=``=~machst dich zum
Richter''\textgreater✲. \bibleverse{12} Nur einer ist Gesetzgeber und
Richter; er, der die Macht hat zu erretten und zu verderben. Wer aber
bist du, daß du dich zum Richter über den Nächsten machst?

\hypertarget{b-gegen-vermessenes-selbstvertrauen-bei-weltlichen-unternehmungen}{%
\paragraph{b) Gegen vermessenes Selbstvertrauen bei (weltlichen)
Unternehmungen}\label{b-gegen-vermessenes-selbstvertrauen-bei-weltlichen-unternehmungen}}

\bibleverse{13} Weiter nun: Ihr, die ihr sagt: »Heute oder morgen wollen
wir in die und die Stadt ziehen, wollen dort ein Jahr bleiben, Geschäfte
machen und Geld verdienen«,~-- \bibleverse{14} und ihr wißt doch nicht,
was der morgende Tag bringen wird, wie es dann um euer Leben steht. Ihr
seid doch nur ein Rauch\textless sup title=``oder: Hauch''\textgreater✲,
der für kurze Zeit sichtbar wird und dann verschwindet. \bibleverse{15}
Statt dessen solltet ihr sagen: »Wenn es der Wille des Herrn ist, werden
wir am Leben bleiben und dies oder jenes tun.« \bibleverse{16} So aber
tut ihr noch groß mit euren hochfahrenden Gedanken! Alle derartige
Großtuerei ist verwerflich. \bibleverse{17} Wer also weiß, wie er sich
richtig zu verhalten hat, es aber nicht tut, für den ist es Sünde.

\hypertarget{c-ankuxfcndigung-des-nahenden-gerichts-an-die-uxfcppigen-reichen-und-die-gott-vergessenden-mammonsdiener}{%
\paragraph{c) Ankündigung des nahenden Gerichts an die üppigen Reichen
und die Gott vergessenden
Mammonsdiener}\label{c-ankuxfcndigung-des-nahenden-gerichts-an-die-uxfcppigen-reichen-und-die-gott-vergessenden-mammonsdiener}}

\hypertarget{section-4}{%
\section{5}\label{section-4}}

\bibleverse{1} Weiter nun: Ihr Reichen, weinet und jammert über die
Leiden, die euch bevorstehen! \bibleverse{2} Euer Reichtum ist
vermodert, und eure Gewänder sind ein Fraß für die Motten geworden,
\bibleverse{3} euer Gold und Silber ist vom Rost angefressen, und ihr
Rost wird ein Zeugnis für euch sein, und der Rost wird euer Fleisch
fressen wie Feuer. Noch jetzt in der Endzeit habt ihr euch Schätze
gesammelt! \bibleverse{4} Wisset wohl: Der Lohn, den ihr den Arbeitern,
die eure Ernte eingebracht haben, vorenthalten habt, schreit (aus euren
Häusern zum Himmel empor), und die Klagerufe eurer Schnitter sind zu den
Ohren des Herrn der Heerscharen gedrungen. \bibleverse{5} Ihr habt hier
auf Erden geschwelgt und gepraßt, habt euch noch am Tage der Schlachtung
nach Herzenslust gütlich getan. \bibleverse{6} Ihr habt den Gerechten
verurteilt, ihn gemordet; er leistet euch keinen Widerstand.

\hypertarget{d-ermahnung-an-die-gluxe4ubigen-zu-geduldigem-ausharren-im-hinblick-auf-die-nah-bevorstehende-wiederkunft-des-herrn}{%
\paragraph{d) Ermahnung an die Gläubigen zu geduldigem Ausharren im
Hinblick auf die nah bevorstehende Wiederkunft des
Herrn}\label{d-ermahnung-an-die-gluxe4ubigen-zu-geduldigem-ausharren-im-hinblick-auf-die-nah-bevorstehende-wiederkunft-des-herrn}}

\bibleverse{7} So harret denn standhaft aus, liebe Brüder, bis zur
Ankunft✲ des Herrn! Bedenket: Der Landmann wartet auf die köstliche
Frucht der Erde\textless sup title=``oder: seines Feldes''\textgreater✲
und geduldet sich ihretwegen, bis sie den Früh- und
Spätregen\textless sup title=``=~Herbst- und
Frühlingsregen''\textgreater✲ empfängt. \bibleverse{8} So haltet auch
ihr geduldig aus und macht eure Herzen fest, denn die Ankunft des Herrn
steht nahe bevor. \bibleverse{9} Seufzt nicht\textless sup
title=``=~werdet nicht ungehalten''\textgreater✲ gegeneinander, liebe
Brüder, damit ihr nicht gerichtet werdet! Bedenkt wohl: Der Richter
steht (schon) vor der Tür! \bibleverse{10} Nehmt euch, liebe Brüder, für
die Leiden und das geduldige Aushalten die Propheten zum Vorbild, die im
Namen des Herrn geredet haben! \bibleverse{11} Seht, wir preisen die
selig, welche geduldig ausgeharrt haben. Vom standhaften Ausharren Hiobs
habt ihr gehört und von dem Ausgang, den der Herr ihm bereitet
hat\textless sup title=``Hiob 42,10-17''\textgreater✲; erkennet daraus,
daß der Herr reich an Mitleid und voll Erbarmens ist.

\hypertarget{e-schluuxdfermahnungen-bezuxfcglich-des-schwuxf6rens-und-des-gebets-bezuxfcglich-des-verhaltens-zu-freude-und-leid-in-krankheit-und-gegen-abgeirrte}{%
\paragraph{e) Schlußermahnungen bezüglich des Schwörens und des Gebets,
bezüglich des Verhaltens zu Freude und Leid, in Krankheit und gegen
Abgeirrte}\label{e-schluuxdfermahnungen-bezuxfcglich-des-schwuxf6rens-und-des-gebets-bezuxfcglich-des-verhaltens-zu-freude-und-leid-in-krankheit-und-gegen-abgeirrte}}

\bibleverse{12} Vor allem aber, meine Brüder, schwöret nicht, weder beim
Himmel noch bei der Erde noch sonst irgendeinen Eid; es sei vielmehr
euer Ja ein Ja und euer Nein ein Nein, damit ihr nicht dem Gericht
verfallt.

\bibleverse{13} Hat jemand unter euch zu leiden, so bete er; geht es
jemandem gut, so singe er Loblieder. \bibleverse{14} Ist jemand unter
euch krank, so lasse er die Ältesten der Gemeinde zu sich kommen; diese
sollen dann über ihm beten, nachdem sie ihn im Namen des Herrn mit Öl
gesalbt haben; \bibleverse{15} alsdann wird das gläubige Gebet den
Kranken retten, und der Herr wird ihn aufrichten\textless sup
title=``=~aufstehen lassen''\textgreater✲, und wenn er Sünden begangen
hat, wird ihm Vergebung zuteil werden. \bibleverse{16} Bekennet also
einander die Sünden und betet füreinander, damit ihr Heilung erlangt;
das Gebet eines Gerechten besitzt eine große Kraft, wenn es ernstlich
ist. \bibleverse{17} Elia war ein Mensch von gleicher Art wie wir und
betete inständig, es möchte nicht regnen; da regnete es drei und ein
halbes Jahr lang nicht im Lande\textless sup title=``oder: auf
Erden''\textgreater✲. \bibleverse{18} Er betete dann nochmals: da gab
der Himmel wieder Regen, und die Erde ließ ihre Frucht sprossen.

\bibleverse{19} Meine Brüder, wenn jemand unter euch von der Wahrheit
abgeirrt ist und einer ihn zur Umkehr bringt, \bibleverse{20} so soll er
wissen: Wer einen Sünder von seinem Irrweg bekehrt, der wird damit seine
Seele vom Tode retten und eine Menge von Sünden bedecken\textless sup
title=``vgl. Spr 10,12; Jes 55,7''\textgreater✲.
