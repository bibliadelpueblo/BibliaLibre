\hypertarget{der-prediger}{%
\section{DER PREDIGER}\label{der-prediger}}

\emph{(oder Weisheitslehrer)}

\hypertarget{i.-eingang}{%
\subsection{I. Eingang}\label{i.-eingang}}

\hypertarget{a-die-uxfcberschrift}{%
\paragraph{a) Die Überschrift}\label{a-die-uxfcberschrift}}

\hypertarget{section}{%
\section{1}\label{section}}

\bibleverse{1}(Dies sind) die Worte des Predigers, des Sohnes Davids,
des Königs in Jerusalem.

\hypertarget{b-die-nutzlosigkeit-alles-menschlichen-muxfchens-infolge-des-bestuxe4ndigen-einerleis-im-kreislauf-der-dinge}{%
\paragraph{b) Die Nutzlosigkeit alles menschlichen Mühens infolge des
beständigen Einerleis im Kreislauf der
Dinge}\label{b-die-nutzlosigkeit-alles-menschlichen-muxfchens-infolge-des-bestuxe4ndigen-einerleis-im-kreislauf-der-dinge}}

\bibleverse{2}O Nichtigkeit der Nichtigkeiten! sagt der Prediger; o
Nichtigkeit der Nichtigkeiten: alles ist nichtig! \bibleverse{3}Welchen
Gewinn hat der Mensch von all seiner Mühe, mit der er sich unter der
Sonne abmüht? \bibleverse{4}Ein Geschlecht geht dahin, und ein anderes
kommt, doch die Erde steht ewig unbewegt. \bibleverse{5}Die Sonne geht
auf, und die Sonne geht unter und eilt an denselben Ort zurück, wo sie
aufging\textless sup title=``oder: wieder aufgehn soll''\textgreater✲.
\bibleverse{6}Der Wind geht nach Süden und dreht sich nach Norden;
immerfort kreisend weht der Wind, und zu seinen\textless sup
title=``=~den alten''\textgreater✲ Kreisläufen kehrt der Wind zurück.
\bibleverse{7}Alle Flüsse laufen ins Meer, und das Meer wird doch nicht
voll; an den Ort, wohin die Flüsse einmal fließen, dahin fließen sie
immer wieder. \bibleverse{8}Alle Dinge mühen sich ab: kein Mensch vermag
es auszusprechen\textless sup title=``=~mit Worten zu
erschöpfen''\textgreater✲; das Auge wird des Sehens nicht satt und das
Ohr nicht voll vom Hören. \bibleverse{9}Was gewesen ist, dasselbe wird
wieder sein, und was geschehen ist, dasselbe wird wieder geschehen; es
gibt nichts Neues unter der Sonne. \bibleverse{10}Kommt (einmal) etwas
vor, von dem man sagen möchte: »Siehe, dies hier ist etwas Neues!«, so
ist es doch längst dagewesen in den Zeitläuften, die vor uns waren:
\bibleverse{11}es ist nur kein Andenken an die früheren Zeiten
geblieben, und auch für die späteren, die künftig sein werden, wird kein
Andenken übrigbleiben bei denen, die noch später kommen werden.

\hypertarget{ii.-der-hauptteil-erfahrungen-und-erlebnisse-des-predigers-und-daran-angeknuxfcpfte-ermahnungen-und-lebensregeln-112-128}{%
\subsection{II. Der Hauptteil: Erfahrungen und Erlebnisse des Predigers
und daran angeknüpfte Ermahnungen und Lebensregeln
(1,12-12,8)}\label{ii.-der-hauptteil-erfahrungen-und-erlebnisse-des-predigers-und-daran-angeknuxfcpfte-ermahnungen-und-lebensregeln-112-128}}

\hypertarget{die-nutzlosigkeit-des-strebens-nach-weisheit-und-erkenntnis}{%
\subsubsection{1. Die Nutzlosigkeit des Strebens nach Weisheit und
Erkenntnis}\label{die-nutzlosigkeit-des-strebens-nach-weisheit-und-erkenntnis}}

\hypertarget{a-das-menschenleben-erweist-sich-dem-betrachter-als-sinnlos}{%
\paragraph{a) Das Menschenleben erweist sich dem Betrachter als
sinnlos}\label{a-das-menschenleben-erweist-sich-dem-betrachter-als-sinnlos}}

\bibleverse{12}Ich, der Prediger, bin König über Israel in Jerusalem
gewesen \bibleverse{13}und habe es mir angelegen sein lassen, vermittels
der Weisheit alles zu erforschen und zu ergründen, was unter dem Himmel
geschieht: ein leidiges\textless sup title=``oder:
mühseliges''\textgreater✲ Geschäft, das Gott den Menschenkindern
auferlegt hat, sich damit abzuquälen. \bibleverse{14}Ich habe alles
Arbeiten beobachtet, das unter der Sonne betrieben wird, und siehe da:
alles war\textless sup title=``oder: ist''\textgreater✲ nichtig und ein
Haschen nach Wind. \bibleverse{15}Krummes kann doch nicht als gerade
gelten, und was lückenhaft ist, darf man nicht als voll rechnen.

\hypertarget{b-das-streben-nach-klarer-erkenntnis-fuxfchrt-zur-enttuxe4uschung}{%
\paragraph{b) Das Streben nach klarer Erkenntnis führt zur
Enttäuschung}\label{b-das-streben-nach-klarer-erkenntnis-fuxfchrt-zur-enttuxe4uschung}}

\bibleverse{16}Ich dachte bei mir in meinem Herzen also: »Fürwahr, ich
habe mir größere Schätze der Weisheit erworben als alle, die vor mir
über\textless sup title=``oder: in''\textgreater✲ Jerusalem gewesen
sind, und mein Geist hat sich eine Fülle von Weisheit und Erkenntnis
angeeignet!« \bibleverse{17}Als ich mich aber daranmachte, zu erkennen,
was Weisheit sei, und zu erkennen, was Torheit und Unverstand sei, da
wurde es mir klar, daß auch dies nur ein Haschen nach Wind ist;
\bibleverse{18}denn wo viel Weisheit ist, da ist auch viel Verdruß, und
mit der Zunahme der Erkenntnis wächst auch der Schmerz\textless sup
title=``oder: die Enttäuschung''\textgreater✲.

\hypertarget{die-nutzlosigkeit-des-versuchs-durch-sinnliche-freuden-und-lebensgenuuxdf-oder-durch-schuxf6pferische-tuxe4tigkeit-befriedigung-zu-erlangen}{%
\subsubsection{2. Die Nutzlosigkeit des Versuchs, durch sinnliche
Freuden und Lebensgenuß oder durch schöpferische Tätigkeit Befriedigung
zu
erlangen}\label{die-nutzlosigkeit-des-versuchs-durch-sinnliche-freuden-und-lebensgenuuxdf-oder-durch-schuxf6pferische-tuxe4tigkeit-befriedigung-zu-erlangen}}

\hypertarget{section-1}{%
\section{2}\label{section-1}}

\bibleverse{1}Da dachte ich bei mir in meinem Herzen: »Wohlan denn, ich
will es einmal mit der Freude und dem Lebensgenuß versuchen!« Aber
siehe, auch das war nichtig. \bibleverse{2}Vom Lachen mußte ich sagen:
»Unsinn ist das!« und von der Freude: »Wozu soll die dienen?«
\bibleverse{3}Ich faßte den Entschluß, meinem Leibe mit Wein gütlich zu
tun -- allerdings so, daß mein Verstand die Leitung mit Besonnenheit
behielte -- und mich an die Torheit zu halten, bis ich sähe, was für die
Menschenkinder das Beste sei, daß sie es täten unter dem Himmel während
der ganzen\textless sup title=``oder: kurzbemessenen''\textgreater✲
Dauer ihres Lebens. \bibleverse{4}Ich unternahm große Werke: ich baute
mir Häuser, pflanzte mir Weinberge, \bibleverse{5}legte mir Gärten und
Parke an und pflanzte darin Fruchtbäume jeder Art; \bibleverse{6}ich
legte mir Wasserteiche an, um aus ihnen den Wald\textless sup
title=``oder: Hain''\textgreater✲ mit seinem üppigen Baumwuchs zu
bewässern; \bibleverse{7}ich kaufte Knechte und Mägde, hatte auch
Gesinde, das in meinem Hause geboren war, und besaß auch große Herden
von Rindern und Kleinvieh, größere als irgend jemand vor mir sie in
Jerusalem besessen hatte. \bibleverse{8}Ich häufte mir auch Silber und
Gold an, die Schätze von Königen und Ländern, schaffte mir Sänger und
Sängerinnen an und, was die Hauptlust der Menschen ist: Frauen über
Frauen. \bibleverse{9}So stand ich groß da und tat es allen zuvor, die
vor mir in Jersualem gelebt hatten; dabei war mir auch meine Weisheit
verblieben. \bibleverse{10}Nichts von allem, wonach meine Augen
Verlangen trugen, versagte ich ihnen, keinen Wunsch ließ ich meinem
Herzen unerfüllt; denn mein Herz sollte Freude haben von all meinem
Schaffen, und das sollte mir der Lohn für alle meine Mühe sein.
\bibleverse{11}Doch als ich nun alle Werke prüfend betrachtete, die
meine Hände geschaffen, und die Mühe erwog, die ich auf ihre Ausführung
verwandt hatte: ach, da war das alles nichtig und ein Haschen nach Wind,
und es kommt nirgends ein Gewinn heraus unter der Sonne.
\bibleverse{12}b Denn was wird der Mensch tun, der nach mir, dem Könige,
kommen wird? Dasselbe, was man immer schon getan hat.

\hypertarget{auch-die-weisheit-ist-zuletzt-ebenso-nichtig-wie-die-torheit-weil-das-endgeschick-des-weisen-und-des-toren-das-gleiche-ist}{%
\subsubsection{3. Auch die Weisheit ist zuletzt ebenso nichtig wie die
Torheit, weil das Endgeschick des Weisen und des Toren das gleiche
ist}\label{auch-die-weisheit-ist-zuletzt-ebenso-nichtig-wie-die-torheit-weil-das-endgeschick-des-weisen-und-des-toren-das-gleiche-ist}}

a Hierauf wandte ich mich dazu, den Wert der Weisheit neben der Torheit
und dem Unverstand festzustellen. \bibleverse{13}Da sah ich denn ein,
daß die Weisheit einen Vorzug vor der Torheit hat, wie das Licht einen
Vorzug vor der Finsternis besitzt; \bibleverse{14}der Weise hat ja Augen
im Kopf, während der Tor im Finstern wandelt. Zugleich erkannte ich aber
auch, daß das gleiche Geschick alle (beide) trifft. \bibleverse{15}Da
dachte ich bei mir in meinem Herzen: »Wenn mich dasselbe Geschick trifft
wie den Toren, wozu bin ich dann so besonders weise gewesen?« So mußte
ich mir denn sagen, daß auch dies nichtig sei. \bibleverse{16}Denn der
Weise hinterläßt ebensowenig wie der Tor ein ewiges Gedenken, weil ja in
den künftigen Tagen alles längst vergessen sein wird; ach ja, wie stirbt
doch der Weise samt dem Toren dahin! \bibleverse{17}So wurde mir denn
das Leben verhaßt\textless sup title=``oder: verleidet''\textgreater✲,
denn mir mißfiel alles Tun, das unter der Sonne stattfindet; alles ist
ja nichtig und ein Haschen nach Wind!

\hypertarget{hinweis-auf-den-uxfcbelstand-dauxdf-der-weise-den-ertrag-und-genuuxdf-seiner-muxfchevollen-arbeit-einem-vielleicht-tuxf6richten-erben-hinterlassen-muuxdf}{%
\paragraph{Hinweis auf den Übelstand, daß der Weise den Ertrag und Genuß
seiner mühevollen Arbeit einem vielleicht törichten Erben hinterlassen
muß}\label{hinweis-auf-den-uxfcbelstand-dauxdf-der-weise-den-ertrag-und-genuuxdf-seiner-muxfchevollen-arbeit-einem-vielleicht-tuxf6richten-erben-hinterlassen-muuxdf}}

\bibleverse{18}Da wurde mir alles Bemühen, das ich bis dahin unter der
Sonne aufgewandt hatte, verleidet, weil ich ja das durch meine Mühe
Geschaffene einem (andern) überlassen muß, der mein Nachfolger sein
wird; \bibleverse{19}und wer kann wissen, ob der weise sein wird oder
ein Tor? Und doch wird er schalten und walten über alle meine Mühe, über
das, was ich durch meine Weisheit unter der Sonne zustande gebracht
habe. Auch das ist nichtig. \bibleverse{20}So kam es denn mit mir dahin,
daß ich mich der Verzweiflung überließ wegen all der Mühe, die ich unter
der Sonne aufgewandt hatte. \bibleverse{21}Denn es kommt vor, daß ein
Mensch sich mit Weisheit, Einsicht und Tüchtigkeit abgemüht hat und dann
den Ertrag seiner Arbeit einem (andern) überlassen muß, der sich gar
nicht darum gemüht hat. Auch das ist nichtig und ein großer Übelstand.
\bibleverse{22}Denn welchen Nutzen hat der Mensch von all seiner Mühe
und von dem Streben seines Geistes, womit er sich unter der Sonne
abmüht, \bibleverse{23}wenn alle seine Tage leidvoll sind und
Widerwärtigkeit sein ganzes Schaffen und nicht einmal bei Nacht sein
Geist Ruhe findet? Auch das ist nichtig.

\hypertarget{das-beste-ist-also-fuxfcr-den-menschen-den-augenblick-zu-genieuxdfen-soweit-gott-es-ihm-verguxf6nnt}{%
\subsubsection{4. Das beste ist also für den Menschen, den Augenblick zu
genießen, soweit Gott es ihm
vergönnt}\label{das-beste-ist-also-fuxfcr-den-menschen-den-augenblick-zu-genieuxdfen-soweit-gott-es-ihm-verguxf6nnt}}

\bibleverse{24}So gibt es denn für den Menschen nichts Besseres, als daß
er ißt und trinkt und sein Herz bei seiner Mühsal guter Dinge sein läßt.
Freilich habe ich erkannt, daß auch dies von der Hand Gottes abhängt;
\bibleverse{25}denn wer kann essen und wer genießen ohne sein Zutun?
\bibleverse{26}Denn einem Menschen, der ihm wohlgefällt, gibt Gott
Weisheit, Einsicht und Freude\textless sup title=``oder:
Genuß''\textgreater✲, dem Sünder aber gibt er das leidige Geschäft, zu
sammeln und zusammenzuscharren, um es hernach dem zu überlassen, der
Gott wohlgefällig ist. Auch das ist nichtig und ein Haschen nach Wind.

\hypertarget{die-vuxf6llige-abhuxe4ngigkeit-des-menschlichen-tuns-von-einer-unabuxe4nderlichen-weltordnung}{%
\subsubsection{5. Die völlige Abhängigkeit des menschlichen Tuns von
einer unabänderlichen
Weltordnung}\label{die-vuxf6llige-abhuxe4ngigkeit-des-menschlichen-tuns-von-einer-unabuxe4nderlichen-weltordnung}}

\hypertarget{a-alles-hat-seine-zeit}{%
\paragraph{a) Alles hat seine Zeit}\label{a-alles-hat-seine-zeit}}

\hypertarget{section-2}{%
\section{3}\label{section-2}}

\bibleverse{1}Jegliches Ding hat seine Zeit und alles Vornehmen unter
dem Himmel seine Stunde. \bibleverse{2}Das Geborenwerden hat seine Zeit
und ebenso das Sterben; das Pflanzen hat seine Zeit und ebenso das
Ausraufen des Gepflanzten; \bibleverse{3}das Töten\textless sup
title=``oder: Zerstören''\textgreater✲ hat seine Zeit und ebenso das
Heilen; das Einreißen hat seine Zeit und ebenso das Aufbauen;
\bibleverse{4}das Weinen hat seine Zeit und ebenso das Lachen; das
Klagen\textless sup title=``oder: Trauern''\textgreater✲ hat seine Zeit
und ebenso das Tanzen; \bibleverse{5}das Hinwerfen von Steinen hat seine
Zeit und ebenso das Sammeln von Steinen; das Liebkosen hat seine Zeit
und ebenso das Meiden der Liebkosung; \bibleverse{6}das Suchen hat seine
Zeit und ebenso das Verlieren; das Aufbewahren hat seine Zeit und ebenso
das Wegwerfen; \bibleverse{7}das Zerreißen hat seine Zeit und ebenso das
Zusammennähen\textless sup title=``oder: Flicken''\textgreater✲; das
Schweigen hat seine Zeit und ebenso das Reden; \bibleverse{8}das Lieben
hat seine Zeit und ebenso das Hassen; der Krieg hat seine Zeit und
ebenso der Friede.

\hypertarget{b-aber-der-mensch-kennt-die-von-gott-festgesetzte-zeit-nicht-und-ist-ihr-gegenuxfcber-ohnmuxe4chtig}{%
\paragraph{b) Aber der Mensch kennt die von Gott festgesetzte Zeit nicht
und ist ihr gegenüber
ohnmächtig}\label{b-aber-der-mensch-kennt-die-von-gott-festgesetzte-zeit-nicht-und-ist-ihr-gegenuxfcber-ohnmuxe4chtig}}

\bibleverse{9}Welchen Gewinn hat also der Tätige davon, daß er sich
abmüht? \bibleverse{10}Ich habe die (leidige) Aufgabe betrachtet, die
Gott den Menschenkindern gestellt hat, sich damit abzuplagen.
\bibleverse{11}Alles hat Gott vortrefflich eingerichtet zu seiner Zeit,
ja auch die Ewigkeit hat er ihnen ins Herz gelegt, nur daß der Mensch
das Tun Gottes von Anfang bis zu Ende nicht zu durchschauen\textless sup
title=``oder: verstehen''\textgreater✲ vermag. \bibleverse{12}So habe
ich denn erkannt, daß es nichts Besseres für den Menschen gibt, als sich
der Freude hinzugeben und sich gütlich zu tun in seinem Leben;
\bibleverse{13}freilich auch, daß, sooft jemand ißt und trinkt und zum
Genießen bei all seiner Mühsal kommt, daß das auch eine Gabe Gottes ist.
\bibleverse{14}Ich habe erkannt, daß alles, was Gott tut\textless sup
title=``oder: bestimmt hat''\textgreater✲, ewige Geltung hat: man kann
da nichts hinzufügen und nichts davon wegnehmen; und das hat Gott so
eingerichtet, damit man sich vor ihm fürchte. \bibleverse{15}Was da ist,
das ist schon längst gewesen, und was geschehen wird, ist längst
dagewesen; denn Gott sucht das Entschwundene\textless sup title=``oder:
in Vergessenheit Geratene''\textgreater✲ wieder hervor.

\hypertarget{das-ausbleiben-der-sittlichen-weltordnung-neben-der-alles-beherrschenden-naturordnung}{%
\subsubsection{6. Das Ausbleiben der sittlichen Weltordnung neben der
alles beherrschenden
Naturordnung}\label{das-ausbleiben-der-sittlichen-weltordnung-neben-der-alles-beherrschenden-naturordnung}}

\hypertarget{a-in-der-menschenwelt-herrscht-gottlosigkeit-und-unrecht-aber-gott-ist-der-weltrichter}{%
\paragraph{a) In der Menschenwelt herrscht Gottlosigkeit und Unrecht,
aber Gott ist der
Weltrichter}\label{a-in-der-menschenwelt-herrscht-gottlosigkeit-und-unrecht-aber-gott-ist-der-weltrichter}}

\bibleverse{16}Weiter aber habe ich unter der Sonne wahrgenommen: an der
Stätte des Rechts\textless sup title=``d.h. wo Recht sein
sollte''\textgreater✲, da herrschte das Unrecht, und an der Stätte der
Gerechtigkeit, da herrschte die Gesetzlosigkeit. \bibleverse{17}Da
dachte ich bei mir in meinem Sinn: »Den Gerechten wie den Gottlosen wird
Gott richten; denn er hat für jedes Vorhaben und für alles Tun eine Zeit
festgesetzt.«

\hypertarget{b-das-gesetz-der-verguxe4nglichkeit-besteht-fuxfcr-menschen-wie-fuxfcr-tiere-und-mahnt-zum-lebensgenuuxdf}{%
\paragraph{b) Das Gesetz der Vergänglichkeit besteht für Menschen wie
für Tiere und mahnt zum
Lebensgenuß}\label{b-das-gesetz-der-verguxe4nglichkeit-besteht-fuxfcr-menschen-wie-fuxfcr-tiere-und-mahnt-zum-lebensgenuuxdf}}

\bibleverse{18}Da dachte ich bei mir selbst: »Um der Menschenkinder
willen ist das so gefügt, damit Gott sie prüft und damit sie einsehen,
daß sie an und für sich den Tieren gleichstehen.« \bibleverse{19}Denn
das Schicksal der Menschen und das Schicksal der Tiere ist ein und
dasselbe: die einen sterben so gut wie die anderen, und sie haben alle
den gleichen Odem, und einen Vorzug des Menschen vor den Tieren gibt es
nicht: \bibleverse{20}alles geht an denselben Ort; alles ist vom Staube
geworden\textless sup title=``oder: genommen''\textgreater✲, und alles
kehrt zum Staube zurück. \bibleverse{21}Wer weiß denn vom Lebensodem des
Menschen, ob er aufwärts in die Luft emporsteigt, und vom Lebensodem des
Tieres, ob er nach unten zur Erde hinabfährt? \bibleverse{22}So habe ich
denn eingesehen, daß es für den Menschen nichts Besseres gibt, als daß
er sich freue bei seinem Tun; ja das ist sein Teil\textless sup
title=``oder: Lohn''\textgreater✲; denn wer wird ihn dahin bringen, daß
er Einsicht in das gewinnt, was nach ihm sein wird?

\hypertarget{weitere-uxfcberlstuxe4nde-des-irdischen-lebensgluxfccks}{%
\subsubsection{7. Weitere Überlstände des irdischen
Lebensglücks}\label{weitere-uxfcberlstuxe4nde-des-irdischen-lebensgluxfccks}}

\hypertarget{a-unterdruxfcckung-eifersucht-und-teils-ruhelose-arbeit-teils-truxe4ge-ruhe-entwerten-das-leben}{%
\paragraph{a) Unterdrückung, Eifersucht und teils ruhelose Arbeit, teils
träge Ruhe entwerten das
Leben}\label{a-unterdruxfcckung-eifersucht-und-teils-ruhelose-arbeit-teils-truxe4ge-ruhe-entwerten-das-leben}}

\hypertarget{section-3}{%
\section{4}\label{section-3}}

\bibleverse{1}Und wiederum betrachtete ich alle Bedrückungen, die unter
der Sonne verübt werden; ich sah da die Tränen der Bedrückten, die
keinen Tröster hatten und von seiten ihrer Bedrücker
Gewalttat\textless sup title=``oder: Mißhandlung''\textgreater✲
erlitten, ohne daß jemand Trost für sie hatte. \bibleverse{2}Da pries
ich die Toten, die längst gestorben sind, glücklicher als die Lebenden,
die jetzt noch am Leben sind; \bibleverse{3}aber glücklicher als beide
pries ich den, der noch nicht ins Dasein getreten ist und deshalb das
böse Treiben noch nicht gesehen hat, das unter der Sonne stattfindet.

\bibleverse{4}Weiter habe ich eingesehen, daß alle Mühe und aller
Erfolg, den man bei seiner Tätigkeit hat, nur eine Folge der
Eifersucht\textless sup title=``oder: des Neides''\textgreater✲ des
einen gegen den andern ist. Auch das ist nichtig und ein Haschen nach
Wind. \bibleverse{5}Der Tor dagegen legt die Hände
ineinander\textless sup title=``=~in den Schoß''\textgreater✲ und zehrt
von seinem eigenen Fleisch: \bibleverse{6}»Besser ist eine Hand voll
Ruhe als beide Fäuste voll Arbeit und Haschen nach Wind.«

\hypertarget{b-das-sichmuxfchen-des-alleinstehenden-ist-zwecklos-zwei-vereint-arbeitende-sind-besser-daran}{%
\paragraph{b) Das Sichmühen des Alleinstehenden ist zwecklos; zwei
vereint Arbeitende sind besser
daran}\label{b-das-sichmuxfchen-des-alleinstehenden-ist-zwecklos-zwei-vereint-arbeitende-sind-besser-daran}}

\bibleverse{7}Ich habe auch noch ein anderes Beispiel eitlen Mühens
unter der Sonne gesehen: \bibleverse{8}Da ist einer, der ganz allein
steht ohne Freunde und Genossen; auch einen Sohn und Bruder hat er
nicht; gleichwohl wird er nicht müde, sich zu plagen, und seine Augen
sehen sich am Reichtum nicht satt (er müßte sich doch sagen): »Für wen
mühe ich mich ab und versage mir jeden Genuß?« Auch das ist nichtig und
ein verfehltes Tun.

\bibleverse{9}Besser sind zwei daran als ein Einzelner, weil ihnen ein
guter Lohn für ihre Mühe zuteil wird; \bibleverse{10}denn wenn sie
fallen, so hilft der eine dem andern wieder auf. Wehe aber dem
Einzelnen! Wenn er hinfällt, ist kein Zweiter da, um ihm wieder
aufzuhelfen! \bibleverse{11}So auch, wenn zwei zusammen schlafen, so
wärmen sie sich gegenseitig; aber ein Einzelner, wie soll dem warm
werden? \bibleverse{12}Und während jemand einen Einzelnen überwältigen
mag, so werden sie zu zweit vor ihm standhalten, und (gar) eine
dreifache Schnur wird nicht so bald zerreißen.

\hypertarget{c-mitteilung-eines-geschichtlichen-vorgangs-durch-den-die-beobachtung-des-predigers-dauxdf-volksgunst-unzuverluxe4ssig-sei-bestuxe4tigt-wird}{%
\paragraph{c) Mitteilung eines geschichtlichen Vorgangs, durch den die
Beobachtung des Predigers, daß Volksgunst unzuverlässig sei, bestätigt
wird}\label{c-mitteilung-eines-geschichtlichen-vorgangs-durch-den-die-beobachtung-des-predigers-dauxdf-volksgunst-unzuverluxe4ssig-sei-bestuxe4tigt-wird}}

\bibleverse{13}Ein armer, aber weiser Jüngling ist mehr wert als ein
alter, jedoch törichter König, der keine Belehrung\textless sup
title=``oder: Warnung''\textgreater✲ mehr annimmt. \bibleverse{14}Denn
aus dem Gefängnis gelangte er auf den Thron, obgleich er unter der
Regierung jenes in Armut geboren war. \bibleverse{15}Ich sah alle
Lebenden, die unter der Sonne wandelten, die Partei des Jünglings
ergreifen, der an jenes Stelle treten sollte: \bibleverse{16}endlos war
die Menge aller derer, die ihn sich zum Führer erkoren hatten.
Gleichwohl freuten die Späteren sich seiner nicht mehr. So war denn auch
dieses nichtig und ein Haschen nach Wind.

\hypertarget{mahnung-zur-vorsicht-bei-der-ausuxfcbung-gottesdienstlicher-pflichten-beim-opfer-beim-gebet-und-bei-geluxfcbden}{%
\subsubsection{8. Mahnung zur Vorsicht bei der Ausübung
gottesdienstlicher Pflichten (beim Opfer, beim Gebet und bei
Gelübden)}\label{mahnung-zur-vorsicht-bei-der-ausuxfcbung-gottesdienstlicher-pflichten-beim-opfer-beim-gebet-und-bei-geluxfcbden}}

\bibleverse{17}Gib acht auf deinen Fuß, wenn du zum Hause Gottes gehst;
denn hintreten, um zu hören\textless sup title=``oder: gehorsam zu
sein''\textgreater✲, ist besser, als wenn die Toren Opfer darbringen:
sie wissen ja nichts weiter als Böses zu tun.~--

\hypertarget{section-4}{%
\section{5}\label{section-4}}

\bibleverse{1}Sei nicht vorschnell mit deinem Munde, und laß dich durch
den Drang deines Herzens nicht dazu bringen, ein Wort vor Gott
auszusprechen; denn Gott ist im Himmel, du aber bist auf der Erde; darum
mache wenig Worte! \bibleverse{2}Denn wo Vielgeschäftigkeit ist, da
kommen Träume; und wo viele Worte sind, da entsteht Torengeschwätz.~--
\bibleverse{3}Hast du Gott ein Gelübde dargebracht, so säume nicht, es
zu erfüllen! Denn er hat kein Wohlgefallen an den Toren. Was du gelobt
hast, das erfülle auch! \bibleverse{4}Besser ist es, kein Gelübde zu
tun, als etwas zu geloben und es nicht zu erfüllen.
\bibleverse{5}Gestatte deinem Munde nicht, deine Person in Schuld zu
bringen, und sage nicht vor dem Gottesdiener✲ aus, daß eine Übereilung
vorliege: warum soll Gott über etwas von dir Ausgesprochenes zürnen und
das Werk deiner Hände mißlingen lassen? \bibleverse{6}Denn wo viele
Träume sind, da ist auch viel eitler Wortschwall. Vielmehr fürchte Gott!

\hypertarget{bedruxfcckungen-im-staate-sind-zu-bedauern-aber-begreiflich-segen-des-kuxf6nigtums-fuxfcr-ackerbautreibende-staaten}{%
\paragraph{Bedrückungen im Staate sind zu bedauern, aber begreiflich;
Segen des Königtums für ackerbautreibende
Staaten}\label{bedruxfcckungen-im-staate-sind-zu-bedauern-aber-begreiflich-segen-des-kuxf6nigtums-fuxfcr-ackerbautreibende-staaten}}

\bibleverse{7}Wenn du siehst, wie der Arme bedrückt wird und wie es mit
Recht und Gerechtigkeit in der Landschaft\textless sup title=``=~in
deinem Lande oder: Volke''\textgreater✲ übel bestellt ist, so rege dich
darüber nicht auf; denn über dem Hohen steht ein noch Höherer auf der
Lauer, und ein Allerhöchster hält Wacht über sie alle.
\bibleverse{8}Doch ein Vorteil für ein Land ist jedenfalls dies: ein
König über bebautes Land\textless sup title=``oder: Feld''\textgreater✲.

\hypertarget{die-nichtigkeit-und-die-beschwerden-des-reichtums}{%
\subsubsection{9. Die Nichtigkeit und die Beschwerden des
Reichtums}\label{die-nichtigkeit-und-die-beschwerden-des-reichtums}}

\bibleverse{9}Wer das Geld liebt, wird des Geldes nie satt, und wer am
Reichtum✲ seine Freude hat, ist unersättlich nach Einkünften; auch das
ist nichtig. \bibleverse{10}Wenn das Gut sich mehrt, so mehren sich auch
die, welche davon zehren; und welchen Nutzen hat sein Besitzer davon,
als daß er die Augen daran weidet? \bibleverse{11}Süß ist der Schlaf des
Arbeiters, mag er wenig oder viel zu essen haben; den Reichen aber läßt
die Übersättigung nicht zum Schlaf kommen.~-- \bibleverse{12}Es gibt ein
ganz schlimmes Übel, das ich unter der Sonne beobachtet habe: Reichtum,
der von seinem Besitzer zu seinem eigenen Unheil gehütet wird.
\bibleverse{13}Geht nämlich solcher Reichtum durch irgendeinen
Unglücksfall verloren, so behält der Sohn, den er erzeugt hat, nichts
mehr im Besitz. \bibleverse{14}Nackt, wie er aus dem Schoß seiner Mutter
hervorgekommen ist, muß er wieder davon, wie er gekommen ist, und kann
für seine Mühe\textless sup title=``oder: von seinem mühsam
Erworbenen''\textgreater✲ nicht das Geringste mitnehmen, um es in seinem
Besitz zu behalten. \bibleverse{15}Ja, das ist auch ein schlimmer
Übelstand: ganz so, wie er gekommen ist, muß er wieder davon. Welchen
Gewinn hat er nun davon, daß er sich für den Wind abgemüht hat?
\bibleverse{16}Dazu verlebt er alle seine Tage im Dunkel und trübselig,
bei viel Verdruß, Krankheit und Aufregung.

\hypertarget{empfehlung-des-lebensgenusses-neben-der-arbeit-und-dem-reichtum}{%
\paragraph{Empfehlung des Lebensgenusses neben der Arbeit und dem
Reichtum}\label{empfehlung-des-lebensgenusses-neben-der-arbeit-und-dem-reichtum}}

\bibleverse{17}(Vernimm dagegen) was ich als gut, als schön befunden
habe: daß der Mensch ißt und trinkt und es sich wohl sein läßt bei all
seiner Mühe, mit der er sich unter der Sonne plagt während der geringen
Zahl der Lebenstage, die Gott ihm beschieden hat; denn das ist sein
Teil\textless sup title=``=~seine Bestimmung''\textgreater✲.
\bibleverse{18}Allerdings, wenn Gott irgendeinem Menschen Reichtum und
irdische Güter verliehen und ihn in die glückliche Lage versetzt hat,
davon zu genießen und sein Teil hinzunehmen und sich bei seiner Mühsal
zu freuen, so ist das eine Gnadengabe Gottes. \bibleverse{19}Denn ein
solcher wird nicht viel an (die Kürze) seiner Lebenstage denken, weil
Gott (ihm) sein Wohlgefallen an der Freude seines Herzens bezeigt.

\hypertarget{einzelne-besondere-uxfcbelstuxe4nde-des-reichtums-und-der-begierde}{%
\subsubsection{10. Einzelne besondere Übelstände des Reichtums und der
Begierde}\label{einzelne-besondere-uxfcbelstuxe4nde-des-reichtums-und-der-begierde}}

\hypertarget{a-jemand-besitzt-reiche-guxfcter-von-denen-er-aber-keinen-genuuxdf-hat}{%
\paragraph{a) Jemand besitzt reiche Güter, von denen er aber keinen
Genuß
hat}\label{a-jemand-besitzt-reiche-guxfcter-von-denen-er-aber-keinen-genuuxdf-hat}}

\hypertarget{section-5}{%
\section{6}\label{section-5}}

\bibleverse{1}Es gibt einen Übelstand, den ich unter der Sonne
beobachtet habe und der schwer auf dem Menschen lastet: \bibleverse{2}Da
verleiht Gott jemandem Reichtum, irdische Güter und Ehre, so daß ihm für
seine Person nichts fehlt von allem, wonach er Verlangen trägt; aber
Gott gestattet ihm nicht, es zu genießen, sondern ein Fremder hat den
Genuß davon: das ist bedauerlich und ein schwerer Übelstand!
\bibleverse{3}Wenn jemand Vater von hundert Kindern würde und viele
Jahre lebte, so daß die Zahl seiner Lebenstage groß wäre, er aber nicht
dazu käme, seines Lebens froh zu werden {[}und ihm sogar kein Begräbnis
zuteil würde{]}, so sage ich: Besser als er ist ein Totgeborener daran.
\bibleverse{4}Denn ein solcher kommt als ein Nichts auf die Welt und
geht im Dunkel hinweg, und sein Name bleibt mit Dunkel bedeckt;
\bibleverse{5}auch hat er die Sonne nicht gesehen und weiß nichts von
ihr; aber in Beziehung auf Ruhe hat er einen Vorzug vor jenem.
\bibleverse{6}Ja, wenn jemand auch zweimal tausend Jahre lebte, ohne
jedoch seines Lebens froh zu werden -- fährt nicht alles dahin an
denselben Ort?\textless sup title=``vgl. 3,20''\textgreater✲

\hypertarget{b-die-unersuxe4ttlichkeit-der-begierde-und-des-strebens-nach-genuuxdf}{%
\paragraph{b) Die Unersättlichkeit der Begierde und des Strebens nach
Genuß}\label{b-die-unersuxe4ttlichkeit-der-begierde-und-des-strebens-nach-genuuxdf}}

\bibleverse{7}Alles Mühen des Menschen geschieht für den Mund, und
dennoch wird dessen Begierde nicht gestillt. \bibleverse{8}Doch welchen
Vorzug hat hierin der Weise vor dem Toren? Den des Armen, der sich auf
die richtige Lebensführung versteht. \bibleverse{9}Besser ist das
Anschauen mit den Augen als das Umherschweifen mit der Begierde. Auch
das ist nichtig und ein Haschen nach Wind.

\hypertarget{c-die-menschliche-ohnmacht-gegenuxfcber-der-guxf6ttlichen-vorherbestimmung-aller-dinge-besonders-des-lebens-der-einzelnen-menschen}{%
\paragraph{c) Die menschliche Ohnmacht gegenüber der göttlichen
Vorherbestimmung aller Dinge (besonders des Lebens der einzelnen
Menschen)}\label{c-die-menschliche-ohnmacht-gegenuxfcber-der-guxf6ttlichen-vorherbestimmung-aller-dinge-besonders-des-lebens-der-einzelnen-menschen}}

\bibleverse{10}Alles, was geschieht\textless sup title=``oder:
entsteht''\textgreater✲, ist längst im voraus bestimmt, und von
vornherein steht fest, wie es einem Menschen ergehen wird, und niemand
vermag den zur Rechenschaft zu ziehen, der stärker ist als er.
\bibleverse{11}Wohl findet da vieles Gerede statt, aber das schafft nur
noch mehr Nichtigkeit: welchen Nutzen hat der Mensch davon?
\bibleverse{12}Denn wer weiß, was dem Menschen im Leben gut ist während
der wenigen Tage seines nichtigen Lebens, die er dem Schatten
vergleichbar verbringt? Denn wer tut dem Menschen kund, was nach ihm
sein wird unter der Sonne?

\hypertarget{allerlei-spruxfcche-der-lebensweisheit}{%
\subsubsection{11. Allerlei Sprüche der
Lebensweisheit}\label{allerlei-spruxfcche-der-lebensweisheit}}

\hypertarget{a-mahnungen-zu-rechtem-lebensernst-und-zu-geduldiger-ergebung-in-die-guxf6ttlichen-fuxfcgungen}{%
\paragraph{a) Mahnungen zu rechtem Lebensernst und zu geduldiger
Ergebung in die göttlichen
Fügungen}\label{a-mahnungen-zu-rechtem-lebensernst-und-zu-geduldiger-ergebung-in-die-guxf6ttlichen-fuxfcgungen}}

\hypertarget{section-6}{%
\section{7}\label{section-6}}

\bibleverse{1}Besser ist ein guter Name als kostbares
Salböl\textless sup title=``=~als Wohlgeruch''\textgreater✲, und besser
der Todestag als der Geburtstag.~-- \bibleverse{2}Besser ist es, in ein
Trauerhaus zu gehen, als zu einem fröhlichen Gastmahl\textless sup
title=``oder: zum Hochzeitsschmaus''\textgreater✲; denn jenes weist auf
das Ende aller Menschen hin, und wer noch im Leben steht, möge sich das
zu Herzen nehmen! \bibleverse{3}Besser Unmut als Lachen; denn bei
ernstem Angesicht steht es gut um das Herz. \bibleverse{4}Das Herz der
Weisen weilt im Trauerhause, aber das Herz der Toren im Hause der
Freude.~-- \bibleverse{5}Besser ist es, auf das Schelten eines Weisen zu
hören, als daß man die Lieder der Toren anhört; \bibleverse{6}denn wie
das Knistern\textless sup title=``oder: Prasseln''\textgreater✲ des
Reisigs unter dem Kessel, so ist das Lachen des Toren. Auch das ist
nichtig.~-- \bibleverse{7}Denn unredlicher Gewinn macht den Weisen zum
Toren, und Bestechungsgeschenke verderben das Herz\textless sup
title=``=~die Gesinnung''\textgreater✲.~-- \bibleverse{8}Besser ist der
Ausgang einer Sache als ihr Anfang, besser Langmut als Hochmut.
\bibleverse{9}Übereile dich nicht, in ärgerliche Stimmung zu geraten;
denn der Ärger hat seine Wohnung im Busen der Toren.~--
\bibleverse{10}Frage nicht, wie es komme, daß die früheren Zeiten besser
waren als die jetzigen; denn nicht die Weisheit gibt dir diese Frage
ein. \bibleverse{11}Weisheit ist so gut wie ein Erbbesitz, und Einsicht
ein Gewinn für die, welche das Sonnenlicht sehen; \bibleverse{12}denn im
Schatten\textless sup title=``=~unter dem Schutz''\textgreater✲ der
Weisheit ist man ebenso geborgen wie im Schatten\textless sup
title=``=~unter dem Schutz''\textgreater✲ des Geldes; aber der Vorzug
der Erkenntnis besteht darin, daß die Weisheit ihrem Besitzer das Leben
erhält.~-- \bibleverse{13}Betrachte das Walten Gottes; denn wer kann
etwas gerade machen, was er gekrümmt hat? \bibleverse{14}Am guten Tage
sei guter Dinge, und am bösen Tage, da erwäge: auch diesen hat Gott
ebenso wie jenen gemacht, damit der Mensch nicht ausfindig mache, was
nach ihm geschieht\textless sup title=``oder: ihm
bevorsteht''\textgreater✲.

\hypertarget{b-warnung-vor-aller-mauxdflosigkeit-und-mahnung-zu-wahrer-weisheit}{%
\paragraph{b) Warnung vor aller Maßlosigkeit und Mahnung zu wahrer
Weisheit}\label{b-warnung-vor-aller-mauxdflosigkeit-und-mahnung-zu-wahrer-weisheit}}

\bibleverse{15}Alles (beides) habe ich in den Tagen meines eitlen
(Erdenlebens) gesehen: mancher Gerechte geht trotz seiner Gerechtigkeit
zugrunde, und mancher Gottlose bringt es trotz seiner Bosheit zu langem
Leben. \bibleverse{16}Verhalte dich nicht allzu gerecht und gebärde dich
nicht übertrieben weise: warum willst du selbst Schaden
nehmen\textless sup title=``oder: dich zugrunde richten''\textgreater✲?
\bibleverse{17}Handle aber auch nicht allzu gottlos und zu töricht:
warum willst du vor der Zeit sterben? \bibleverse{18}Es ist am besten,
wenn du an dem einen festhältst und auch das andere nicht fahren läßt;
denn der Gottesfürchtige entgeht allem beidem\textless sup title=``oder:
kommt weiter als sie alle?''\textgreater✲.~-- \bibleverse{19}Die
Weisheit verleiht dem Weisen mehr Kraft als zehn Machthaber, die in der
Stadt sind. \bibleverse{20}Denn kein Mensch auf Erden ist so gerecht,
daß er nur Gutes täte und niemals sündigte. \bibleverse{21}Gib auch
nicht auf alles Gerede acht, das man führt; du könntest sonst einmal
deinen eigenen Knecht dich schmähen hören; \bibleverse{22}denn gar
manchmal -- du wirst dir dessen wohl bewußt sein -- hast du selbst
andere geschmäht.

\bibleverse{23}Alles dies habe ich mit\textless sup title=``oder: im
Streben nach''\textgreater✲ der Weisheit erprobt; ich dachte: »Ich will
die Weisheit gewinnen!«, doch sie blieb fern von mir. \bibleverse{24}In
weiter Ferne liegt der Grund aller Dinge und tief, ja tief verborgen:
wer kann ihn ausfindig machen?

\hypertarget{des-predigers-uxfcble-erfahrungen-mit-dem-weiblichen-geschlecht}{%
\subsubsection{12. Des Predigers üble Erfahrungen mit dem weiblichen
Geschlecht}\label{des-predigers-uxfcble-erfahrungen-mit-dem-weiblichen-geschlecht}}

\bibleverse{25}Immer wieder, wenn ich mich dazu wandte und mein Streben
darauf richtete, Erkenntnis und ein richtiges Urteil zu gewinnen und mit
dem Suchen nach Weisheit zu einem Abschluß zu kommen und einzusehen, daß
die Gottlosigkeit Torheit ist und die Torheit Wahnsinn,
\bibleverse{26}da fand ich etwas, das bitterer ist als der Tod, nämlich
das Weib, das einem Fangnetz gleicht und dessen Herz Schlingen, dessen
Arme Fesseln sind. Wer Gott wohlgefällt, der entgeht ihr, doch wer
sündigt\textless sup title=``oder: doch wer ihm mißfällt''\textgreater✲,
wird von ihr gefangen. \bibleverse{27}»Siehe, dies habe ich gefunden«,
sagt der Prediger, »indem ich eine Erfahrung zu der andern fügte, um ein
sicheres Urteil zu gewinnen; \bibleverse{28}was aber meine Seele immer
noch sucht und was ich nicht gefunden habe, ist dies: Unter tausend habe
ich wohl einen Mann gefunden, aber ein Weib habe ich unter ihnen allen
nicht gefunden. \bibleverse{29}Allerdings, wisse wohl: dies habe ich
gefunden, daß Gott die Menschen gerade\textless sup title=``=~recht,
richtig''\textgreater✲ geschaffen hat; sie selbst aber suchen viele
verwerfliche Künste.«

\hypertarget{weitere-lebensregeln-und-neue-lebensruxe4tsel}{%
\subsubsection{13. Weitere Lebensregeln und neue
Lebensrätsel}\label{weitere-lebensregeln-und-neue-lebensruxe4tsel}}

\hypertarget{a-des-weisen-verhalten-gegenuxfcber-dem-herrscher-und-in-tagen-der-unterdruxfcckung}{%
\paragraph{a) Des Weisen Verhalten gegenüber dem Herrscher und in Tagen
der
Unterdrückung}\label{a-des-weisen-verhalten-gegenuxfcber-dem-herrscher-und-in-tagen-der-unterdruxfcckung}}

\hypertarget{section-7}{%
\section{8}\label{section-7}}

\bibleverse{1}Wer ist wie der Weise und wer versteht sich auf die
Deutung der Dinge? Die Weisheit erleuchtet\textless sup title=``oder:
verklärt''\textgreater✲ das Angesicht eines Menschen, so daß die Härte
seiner Gesichtszüge verwandelt wird. \bibleverse{2}Ich sage: Beobachte
das Gebot des Königs, und zwar wegen des bei Gott geleisteten Treueides.
\bibleverse{3}Übereile dich nicht, ihm aus den Augen zu
gehen\textless sup title=``=~von ihm wegzugehen''\textgreater✲, und laß
dich auf keine böse Sache ein; denn er setzt alles durch, was er will,
\bibleverse{4}weil ja das Wort des Königs eine Macht ist; und wer darf
zu ihm sagen: »Was tust du da?« \bibleverse{5}Wer das Gebot beobachtet,
wird nichts Schlimmes erleben; wohl aber wird das Herz des Weisen die
zur bestimmten Zeit eintretende richterliche Entscheidung erleben.

\hypertarget{b-ohnmacht-und-ratlosigkeit-des-menschen}{%
\paragraph{b) Ohnmacht und Ratlosigkeit des
Menschen}\label{b-ohnmacht-und-ratlosigkeit-des-menschen}}

\bibleverse{6}Denn für jede Sache gibt es eine zur bestimmten Zeit
eintretende Entscheidung; doch der Übelstand lastet schwer auf dem
Menschen, \bibleverse{7}daß er die Zukunft nicht kennt; denn wer könnte
ihm ansagen, wie es in Zukunft sein wird? \bibleverse{8}Kein Mensch hat
Macht über den Wind, so daß er den Wind aufhalten könnte; ebensowenig
ist jemand Herr über den Tag seines Todes; auch gibt es im Kriege keine
Entlassung\textless sup title=``oder: keinen Urlaub''\textgreater✲; und
ebenso läßt die Gesetzesübertretung den nicht entkommen, der sie übt.

\hypertarget{c-gerechte-und-gottlose-trifft-das-gleiche-geschick-in-abhuxe4ngigkeit-von-einer-huxf6heren-gewalt-es-genuxfcgt-wenn-man-bei-seiner-arbeit-zum-lebensgenuuxdf-gelangt}{%
\paragraph{c) Gerechte und Gottlose trifft das gleiche Geschick in
Abhängigkeit von einer höheren Gewalt; es genügt, wenn man bei seiner
Arbeit zum Lebensgenuß
gelangt}\label{c-gerechte-und-gottlose-trifft-das-gleiche-geschick-in-abhuxe4ngigkeit-von-einer-huxf6heren-gewalt-es-genuxfcgt-wenn-man-bei-seiner-arbeit-zum-lebensgenuuxdf-gelangt}}

\bibleverse{9}Alles dieses habe ich gesehen, indem ich mein Augenmerk
auf alles Geschehen\textless sup title=``oder: Tun''\textgreater✲
richtete, das unter der Sonne stattfindet, solange ein Mensch über
andere herrscht zu ihrem Unglück. \bibleverse{10}Dabei habe ich auch
gesehen, daß Gottlose begraben wurden und zur Ruhe eingingen, während
Leute, die rechtschaffen gelebt hatten, von der heiligen\textless sup
title=``oder: geweihten''\textgreater✲ Stätte wegziehen mußten und in
der Stadt in Vergessenheit gerieten; auch das ist nichtig.
\bibleverse{11}Weil der Urteilsspruch über böse Taten nicht schnell
vollstreckt wird, darum ist das Herz der Menschen mit Mut erfüllt, Böses
zu tun; \bibleverse{12}außerdem (auch aus dem Grunde), weil ein Sünder
hundertmal Böses tut und doch lange am Leben bleibt -- obgleich ich
weiß, daß es den Gottesfürchtigen gutgehen wird, weil sie sich vor
ihm\textless sup title=``d.h. vor Gott''\textgreater✲ fürchten,
\bibleverse{13}während es dem Gottlosen nicht gutgehen und er seine Tage
nicht wie ein Schatten in die Länge ziehen wird, weil er sich vor Gott
nicht fürchtet. \bibleverse{14}Es gibt etwas Nichtiges, das auf Erden
vorkommt, nämlich daß es Gerechte gibt, denen es so ergeht, wie es den
Gottlosen nach ihrem Tun ergehen müßte, und daß es manchen Gottlosen so
ergeht, wie es bei den Gerechten nach ihrem Tun der Fall sein müßte. Da
habe ich mir gesagt, daß auch dies nichtig sei. \bibleverse{15}So lobe
ich mir denn die Freude, weil es für den Menschen nichts Besseres unter
der Sonne gibt als zu essen und zu trinken und guter Dinge zu sein; und
dies möge ihn bei seiner Mühsal begleiten während der Tage seines
Lebens, die Gott ihm unter der Sonne vergönnt.

\hypertarget{d-das-walten-gottes-in-der-weltregierung-ist-fuxfcr-den-menschen-unergruxfcndlich}{%
\paragraph{d) Das Walten Gottes in der Weltregierung ist für den
Menschen
unergründlich}\label{d-das-walten-gottes-in-der-weltregierung-ist-fuxfcr-den-menschen-unergruxfcndlich}}

\bibleverse{16}Sooft ich mein Streben darauf richtete, zur Erkenntnis
der Weisheit zu gelangen und alles Tun, das auf der Erde vor sich geht,
zu beobachten, \bibleverse{17}habe ich bezüglich des ganzen göttlichen
Waltens erkannt, daß der Mensch, mag er auch seinen Augen weder bei Tag
noch bei Nacht Schlaf zu finden vergönnen, das Walten, das sich unter
der Sonne vollzieht, nicht zu ergründen vermag {[}insofern der Mensch
trotz aller Mühe, mit der er es zu erforschen sucht, es doch nicht
ergründet{]}. Denn auch wenn der Weise es zu erkennen vermeint, vermag
er es doch nicht zu ergründen.

\hypertarget{gleiches-los-fuxfcr-alle-im-leben-und-im-tode-die-menschliche-ohnmacht-gegenuxfcber-der-gottheit-gottgefuxe4lliger-lebensgenuuxdf-ehe-der-tod-aller-freude-und-tuxe4tigkeit-ein-ziel-setzt}{%
\subsubsection{14. Gleiches Los für alle im Leben und im Tode; die
menschliche Ohnmacht gegenüber der Gottheit; gottgefälliger Lebensgenuß,
ehe der Tod aller Freude und Tätigkeit ein Ziel
setzt}\label{gleiches-los-fuxfcr-alle-im-leben-und-im-tode-die-menschliche-ohnmacht-gegenuxfcber-der-gottheit-gottgefuxe4lliger-lebensgenuuxdf-ehe-der-tod-aller-freude-und-tuxe4tigkeit-ein-ziel-setzt}}

\hypertarget{section-8}{%
\section{9}\label{section-8}}

\bibleverse{1}Ja, auf dies alles habe ich mein Augenmerk gerichtet und
dies alles mir klar zu machen gesucht, daß nämlich die Gerechten und die
Weisen mit ihrem ganzen Tun in der Hand Gottes sind. Der Mensch weiß
weder, ob ihm Liebe oder Haß begegnen wird: alles ist vor ihm (in der
Zukunft) verhüllt. \bibleverse{2}Dasselbe Geschick trifft alle ohne
Unterschied: das gleiche Los wird allen zuteil, dem Gerechten wie dem
Gottlosen, dem Reinen wie dem Unreinen, dem, der opfert, wie dem, der
nicht opfert; dem Guten geht es wie dem Sünder und dem, der schwört, wie
dem, der sich vor dem Schwören scheut. \bibleverse{3}Das ist ein
Übelstand bei allem, was unter der Sonne geschieht, daß allen das
gleiche Geschick beschieden ist und auch daß das Herz der Menschenkinder
voll Bosheit ist und Unverstand in ihrem Herzen wohnt, solange sie
leben; danach aber geht's zu den Toten. \bibleverse{4}Denn solange einer
überhaupt noch zu den Lebenden gehört, so lange hat er noch etwas zu
hoffen; denn ein lebender Hund ist mehr wert\textless sup title=``oder:
besser daran''\textgreater✲ als ein toter Löwe. \bibleverse{5}Die
Lebenden wissen doch noch, daß sie sterben werden, die Toten aber wissen
überhaupt nichts und haben auch keinen Lohn mehr zu erwarten; sogar ihr
Andenken wird ja vergessen. \bibleverse{6}Sowohl Lieben als Hassen und
Eifern\textless sup title=``oder: Neiden''\textgreater✲ ist für sie
längst vorbei, und sie nehmen in Ewigkeit keinen Anteil mehr an irgend
etwas, das unter der Sonne vor sich geht.

\bibleverse{7}Wohlan denn, iß dein Brot mit Freuden und trinke deinen
Wein mit wohlgemutem Herzen! Denn Gott hat solches Tun bei dir von
vornherein gutgeheißen. \bibleverse{8}Trage allezeit weiße Kleider und
laß das Salböl deinem Haupte nicht mangeln. \bibleverse{9}Genieße das
Leben mit dem Weibe, das du liebgewonnen hast, an all deinen eitlen✲
Lebenstagen, die Gott dir unter der Sonne vergönnt, alle deine eitlen
Tage hindurch; denn das ist dein Anteil am Leben und (der Lohn) für die
Mühe, mit der du dich unter der Sonne abmühst. \bibleverse{10}Alles, was
deine Hand mit deiner Kraft zu leisten vermag, das tu; denn in der
Unterwelt, wohin dein Weg geht, gibt es kein Schaffen und keine
Überlegung mehr, weder Erkenntnis noch Weisheit.

\hypertarget{die-abhuxe4ngigkeit-des-menschen-vom-schicksal}{%
\paragraph{Die Abhängigkeit des Menschen vom
Schicksal}\label{die-abhuxe4ngigkeit-des-menschen-vom-schicksal}}

\bibleverse{11}Wiederum habe ich unter der Sonne gesehen, daß nicht dem
Schnellsten der Sieg✲ im Wettlauf und nicht dem Tapfersten der Sieg im
Kriege zuteil wird, auch nicht den Weisen das Brot und nicht den
Verständigen der Reichtum, auch nicht den Einsichtsvollen die Gunst,
sondern sie sind alle von Zeit und Umständen abhängig.
\bibleverse{12}Der Mensch kennt ja nicht einmal die ihm bestimmte Zeit;
nein, wie die Fische, die im Unglücksnetz sich fangen, und wie die
Vögel, die von der Schlinge erfaßt werden, ebenso werden auch die
Menschenkinder zur Zeit des Unglücks umstrickt, wenn es plötzlich über
sie hereinbricht.

\hypertarget{weitere-lebenserfahrungen-und-weisheitsspruxfcche}{%
\subsubsection{15. Weitere Lebenserfahrungen und
Weisheitssprüche}\label{weitere-lebenserfahrungen-und-weisheitsspruxfcche}}

\bibleverse{13}Und doch habe ich folgenden Fall von Weisheit unter der
Sonne erlebt, und er hat einen tiefen Eindruck auf mich gemacht:
\bibleverse{14}Es war eine kleine Stadt, in der sich nur wenige Leute
befanden; da zog ein mächtiger König gegen sie heran, schloß sie rings
ein und ließ gewaltige Belagerungswerke gegen sie aufführen.
\bibleverse{15}Nun fand sich in ihr ein armer\textless sup title=``oder:
geringer''\textgreater✲, aber weiser Mann, der die Stadt durch seine
Weisheit rettete; aber kein Mensch denkt mehr an diesen armen Mann.
\bibleverse{16}Da sagte ich mir: »Weisheit ist (zwar) besser als Stärke,
aber die Weisheit des Armen wird verachtet, und seine Worte bleiben
ungehört.«

\bibleverse{17}Worte der Weisen, die man in Ruhe anhört, sind mehr
wert\textless sup title=``oder: wirken stärker''\textgreater✲ als das
Brüllen eines Herrschers unter Toren. \bibleverse{18}Weisheit ist besser
als Kriegsgerät; aber ein einziger Bösewicht kann viel Gutes verderben.

\hypertarget{section-9}{%
\section{10}\label{section-9}}

\bibleverse{1}Tote Fliegen machen ranzig und trübe das Öl des
Salbenmischers; so verderbt ein wenig Torheit den Wert der Weisheit.~--
\bibleverse{2}Der Sinn des Weisen ist auf das Rechte gerichtet und der
Sinn des Toren auf das Verkehrte; \bibleverse{3}und wo der Tor auch
gehen mag, auf Schritt und Tritt, versagt sein Verstand, so daß er sich
allen Leuten als Toren zu erkennen gibt.~-- \bibleverse{4}Wenn der Unmut
des Herrschers gegen dich aufsteigt, so verlaß darum deinen
Platz\textless sup title=``oder: Posten''\textgreater✲ nicht; denn
Gelassenheit verhütet\textless sup title=``oder: macht
gut''\textgreater✲ schwere Verfehlungen.~-- \bibleverse{5}Es gibt einen
Übelstand, den ich unter der Sonne wahrgenommen habe, nämlich ein
verfehltes Verfahren, das von einem Machthaber ausgeht:
\bibleverse{6}Toren werden auf große Höhe gestellt\textless sup
title=``=~in die höchsten Würden eingesetzt''\textgreater✲, und
Reiche\textless sup title=``oder: Edle''\textgreater✲ müssen unten
sitzen. \bibleverse{7}Ich habe Sklaven hoch zu Roß gesehen und Fürsten
wie Sklaven zu Fuß einhergehen.

\bibleverse{8}Wer eine Grube gräbt, fällt selbst hinein\textless sup
title=``oder: kann hineinfallen''\textgreater✲, und wer Gemäuer
einreißt, den kann eine Schlange beißen; \bibleverse{9}wer Steine
bricht, kann sich an ihnen verletzen, wer Holz spaltet, kann sich dabei
wehetun.~-- \bibleverse{10}Wenn eine Axt stumpf geworden ist und man die
Schneide nicht schärft, dann muß man seine Kräfte um so mehr anstrengen;
aber der Vorteil des Instandsetzens ist Weisheit.~-- \bibleverse{11}Wenn
die Schlange beißt, ehe die Beschwörung stattgefunden hat, so hat der
Beschwörer keinen Nutzen (von seiner Kunst).

\bibleverse{12}Worte aus dem Munde eines Weisen sind herzgewinnend, aber
den Toren richten die eigenen Lippen zugrunde. \bibleverse{13}Der Anfang
der Worte seines Mundes ist Torheit und das Ende seines Redens schlimmer
Unsinn. \bibleverse{14}Auch macht der Tor viele Worte, obgleich kein
Mensch weiß, was geschehen wird, und niemand ihm ansagen kann, was die
Zukunft bringt. \bibleverse{15}Die Mühe, die der Tor aufwendet, macht
ihn müde, so daß er den Weg nach der Stadt nicht mehr kennt.

\bibleverse{16}Wehe dir, Land, dessen König ein Knabe ist und dessen
Fürsten schon am Morgen schmausen! \bibleverse{17}Heil dir, du Land,
dessen König ein Sproß von edler Herkunft ist und dessen Fürsten zu
rechter Zeit tafeln, und zwar als Männer und nicht als
Zecher\textless sup title=``oder: Trunkenbolde''\textgreater✲!~--
\bibleverse{18}Infolge von Faulheit senkt sich das Gebälk (eines
Hauses), und infolge von Lässigkeit der Hände tropft das
Haus\textless sup title=``=~dringt der Regen durch das
Hausdach''\textgreater✲.~-- \bibleverse{19}Zur Belustigung veranstaltet
man Mahlzeiten, und der Wein erheitert das Leben, und für Geld kann man
alles haben.~-- \bibleverse{20}Selbst auf deinem Lager fluche dem Könige
nicht, und einen Hochgestellten schmähe auch in deinem Schlafgemach
nicht; denn die Vögel des Himmels könnten den Laut\textless sup
title=``=~das Ausgesprochene''\textgreater✲ weitertragen und ein
geflügelter Bote das Wort verraten.

\hypertarget{kluges-und-gewinnbringendes-handeln-bei-der-ungewiuxdfheit-alles-irdischen}{%
\subsubsection{16. Kluges und gewinnbringendes Handeln bei der
Ungewißheit alles
Irdischen}\label{kluges-und-gewinnbringendes-handeln-bei-der-ungewiuxdfheit-alles-irdischen}}

\hypertarget{section-10}{%
\section{11}\label{section-10}}

\bibleverse{1}Laß dein Brot\textless sup title=``=~Geld,
Vermögen''\textgreater✲ über das weite Meer fahren; denn nach Verlauf
vieler Tage wirst du es wieder heimkommen sehen; \bibleverse{2}doch
verteile es auf sieben, ja auf acht Fahrten\textless sup title=``oder:
Unternehmungen''\textgreater✲; denn du weißt nicht, was für Unglück sich
auf der Erde ereignen mag.~-- \bibleverse{3}Wenn die Wolken mit Regen
gefüllt sind, lassen sie ihn auf die Erde strömen; und wenn ein Baum
nach Süden oder nach Norden fällt, so bleibt er an der Stelle liegen,
wohin er gefallen ist.~-- \bibleverse{4}Wer (immerfort) auf den Wind
achtet, kommt nicht zum Säen, und wer (immerfort) nach den Wolken sieht,
kommt nicht zum Ernten.~-- \bibleverse{5}Gleichwie du nicht weißt,
welches der Weg des Windes ist oder wie die Gebeine im Schoße der
Schwangeren sich bilden, ebensowenig kennst du das Walten Gottes, der
alles wirkt.~-- \bibleverse{6}Am Morgen säe deinen Samen, und bis zum
Abend laß deine Hände nicht ruhen; denn du weißt nicht, was gelingen
wird, ob dieses oder jenes, oder ob gar beides zugleich gut geraten
wird.~-- \bibleverse{7}Und köstlich ist das Licht, und wohltuend ist's
für die Augen, die Sonne zu sehen; \bibleverse{8}denn wenn jemand auch
viele Jahre lebt, möge er sich doch in ihnen allen der Freude hingeben
und an die Tage der Finsternis denken, daß ihrer viele sein werden:
alles, was kommt, ist nichtig.

\hypertarget{mahnung-zum-vollen-aber-gott-wohlgefuxe4lligen-genuuxdf-des-lebens-in-der-jugend}{%
\subsubsection{17. Mahnung zum vollen, aber Gott wohlgefälligen Genuß
des Lebens in der
Jugend}\label{mahnung-zum-vollen-aber-gott-wohlgefuxe4lligen-genuuxdf-des-lebens-in-der-jugend}}

\bibleverse{9}Freue dich, Jüngling, in deiner Jugend und laß dein Herz
guter Dinge sein in den Tagen deiner Jugendzeit; wandle die Wege, zu
denen dein Herz sich hingezogen fühlt, und gehe dem nach, was deine
Augen erschaun; doch wisse wohl, daß Gott um dies alles Rechenschaft von
dir fordern wird! \bibleverse{10}Schlage dir den Unmut aus dem Sinn und
halte dir das Leid vom Leibe fern, denn Jugend und dunkles Haar sind
schnell entschwunden.

\hypertarget{section-11}{%
\section{12}\label{section-11}}

\bibleverse{1}Und bleibe deines Schöpfers eingedenk in den Tagen deiner
Jugendzeit, ehe die bösen Tage kommen und die Jahre sich einstellen, von
denen du sagen wirst: »Sie gefallen mir nicht«;

\hypertarget{schilderung-der-gebrechen-des-alters}{%
\paragraph{Schilderung der Gebrechen des
Alters}\label{schilderung-der-gebrechen-des-alters}}

\bibleverse{2}ehe noch die Sonne und das Tageslicht, der Mond und die
Sterne sich verfinstern und die Wolken wiederkehren nach dem Regen,
\bibleverse{3}in der Zeit, wo die Hüter\textless sup title=``oder:
Wächter''\textgreater✲ des Hauses zittern und die starken Männer sich
krümmen; wo die Müllerinnen die Arbeit einstellen, weil ihrer wenige
geworden sind, und die Fensterguckerinnen trübe werden; \bibleverse{4}wo
die beiden Pforten nach der Straße hin geschlossen stehen, weil die
Mühle mit weniger Geräusch geht, und man beim Hahnenschrei\textless sup
title=``oder: Vogelgezwitscher''\textgreater✲ aufsteht und aller
Liederklang verstummt; \bibleverse{5}auch vor jeder Steigung fürchtet
man sich und sieht Schrecknisse auf jedem Wege; der Mandelbaum steht in
Blüte, und die Heuschrecke\textless sup title=``oder: der
Grashüpfer''\textgreater✲ schleppt sich träge dahin, und die Kaperwürze
versagt ihre Wirkung -- denn der Mensch geht hin zu seiner ewigen
Behausung, und die Klageleute ziehen auf der Straße umher --;
\bibleverse{6}ehe noch der silberne Faden\textless sup title=``d.h.
Lebensfaden''\textgreater✲ zerreißt und die goldene Schale zerbricht und
der Krug an der Quelle in Scherben geht und das Schöpfrad zertrümmert in
den Brunnen fällt \bibleverse{7}und der Staub zur Erde zurückkehrt als
das, was er vorher gewesen ist, und der Odem\textless sup title=``oder:
Geist''\textgreater✲ zu Gott zurückkehrt, der ihn gegeben hat.

\bibleverse{8}»O Nichtigkeit der Nichtigkeiten!« ruft der Prediger aus,
»alles ist nichtig!«

\hypertarget{iii.-nachwort-des-herausgebers-uxfcber-den-verfasser-den-zweck-und-das-ergebnis-des-buches}{%
\subsection{III. Nachwort des Herausgebers über den Verfasser, den Zweck
und das Ergebnis des
Buches}\label{iii.-nachwort-des-herausgebers-uxfcber-den-verfasser-den-zweck-und-das-ergebnis-des-buches}}

\hypertarget{a-nachruhm-des-predigers}{%
\paragraph{a) Nachruhm des Predigers}\label{a-nachruhm-des-predigers}}

\bibleverse{9}Abgesehen davon, daß der Prediger ein Weiser war, hat er
das Volk auch Erkenntnis gelehrt und war ein Denker und Forscher, der
zahlreiche Sprüche verfaßt\textless sup title=``oder:
gesammelt?''\textgreater✲ hat. \bibleverse{10}Der Prediger war bemüht,
ansprechende Worte zu finden und zutreffende Weisungen
niederzuschreiben, Aussprüche der Wahrheit. \bibleverse{11}Die
Aussprüche der Weisen sind wie Treibstachel, und wie eingeschlagene
Pflöcke stehen die einzelnen Sprüche beisammen, die von einem einzigen
Hirten\textless sup title=``=~weisen Lehrer oder: Meister''\textgreater✲
herrühren.

\hypertarget{b-warnung-vor-unnuxfctzem-gruxfcbeln-aufstellung-des-schluuxdfergebnisses}{%
\paragraph{b) Warnung vor unnützem Grübeln; Aufstellung des
Schlußergebnisses}\label{b-warnung-vor-unnuxfctzem-gruxfcbeln-aufstellung-des-schluuxdfergebnisses}}

\bibleverse{12}Und ferner noch\textless sup title=``oder: im
übrigen''\textgreater✲: laß dich warnen, mein Sohn; des vielen
Bücherschreibens ist kein Ende, und das viele Studieren verursacht dem
Leibe Ermüdung.~-- \bibleverse{13}Laßt uns das Endergebnis des Ganzen
hören: Fürchte Gott und halte seine Gebote! Denn das kommt jedem
Menschen zu. \bibleverse{14}Denn Gott wird in dem Gericht, das über
alles Verborgene ergeht, das Urteil über alles Tun sprechen, es sei gut
oder böse (gewesen).
