\hypertarget{das-buch-nehemia}{%
\section{DAS BUCH NEHEMIA}\label{das-buch-nehemia}}

\hypertarget{i.-nehemias-reise-nach-dem-heiligen-lande-und-seine-wirksamkeit-in-jerusalem-11-772}{%
\subsection{I. Nehemias Reise nach dem heiligen Lande und seine
Wirksamkeit in Jerusalem
(1,1-7,72)}\label{i.-nehemias-reise-nach-dem-heiligen-lande-und-seine-wirksamkeit-in-jerusalem-11-772}}

\hypertarget{nehemias-reise-nach-jerusalem}{%
\subsubsection{1. Nehemias Reise nach
Jerusalem}\label{nehemias-reise-nach-jerusalem}}

\hypertarget{a-nehemia-als-mundschenk-des-kuxf6nigs-artaxerxes-in-susa-seine-trauer-uxfcber-das-ungluxfcck-seines-vaterlands}{%
\paragraph{a) Nehemia als Mundschenk des Königs (Artaxerxes) in Susa;
seine Trauer über das Unglück seines
Vaterlands}\label{a-nehemia-als-mundschenk-des-kuxf6nigs-artaxerxes-in-susa-seine-trauer-uxfcber-das-ungluxfcck-seines-vaterlands}}

\hypertarget{section}{%
\section{1}\label{section}}

1Die Worte\textless sup title=``=~Denkwürdigkeiten, Denkschrift,
Bericht''\textgreater✲ Nehemias, des Sohnes Hachaljas.

Es begab sich im Monat Kislew im zwanzigsten Regierungsjahre des Königs
Arthasastha\textless sup title=``=~Artaxerxes I.''\textgreater✲, als ich
mich in der Königsstadt✲ Susa befand, 2daß Hanani, einer von meinen
Brüdern, mit einigen Männern aus Juda (zu mir) kam. Als ich mich nun bei
ihnen nach den Juden, die in der Heimat zurückgeblieben und der
Wegführung✲ entgangen waren, und nach (den Verhältnissen in) Jerusalem
erkundigte, 3teilten sie mir mit: »Die Übriggebliebenen, die dort in der
Provinz\textless sup title=``=~im Bezirk Juda''\textgreater✲ der
(Wegführung in die) Gefangenschaft entgangen sind, befinden sich in
großem Elend und in schmachvoller Lage; die Mauern Jerusalems sind
niedergerissen und die Tore der Stadt mit Feuer verbrannt.« 4Als ich
diese Mitteilungen vernahm, setzte ich mich (auf den Boden) hin und fing
an zu weinen und trug tagelang Trauer unter beständigem Fasten und
Flehen und richtete an den Gott des Himmels folgendes Gebet:

\hypertarget{b-nehemias-buuxdf--und-bittgebet}{%
\paragraph{b) Nehemias Buß- und
Bittgebet}\label{b-nehemias-buuxdf--und-bittgebet}}

5»Ach HERR, du Gott des Himmels, du großer und furchtbarer Gott, der du
denen, die dich lieben und deine Gebote halten, Bundestreue und Gnade
bewahrst: 6laß doch deine Ohren aufmerken und deine Augen geöffnet sein,
daß du das Gebet deines Knechtes vernimmst, welches ich jetzt Tag und
Nacht für die Kinder Israel, deine Knechte, vor dir bete und in welchem
ich die Sünden bekenne, die wir Israeliten gegen dich begangen haben!
Denn auch ich und meines Vaters Haus haben gesündigt. 7Gar verwerflich
haben wir gegen dich gehandelt, daß wir die Gebote, Satzungen und
Anordnungen, die du deinem Knechte Mose geboten hast, nicht befolgt
haben. 8Ach, gedenke doch der Verheißung, die du deinem Knechte Mose
gegeben hast mit den Worten:\textless sup title=``5.Mose 4,27-31;
30,1-5''\textgreater✲: ›Wenn ihr treulos handelt, so werde ich euch
unter die Völker zerstreuen; 9wenn ihr aber zu mir umkehrt und meine
Gebote beobachtet und danach tut, so will ich, wenn sich auch
Versprengte von euch am Ende des Himmels befinden sollten, sie doch von
dort sammeln und sie an die Stätte zurückbringen, die ich erwählt habe,
um meinen Namen dort wohnen zu lassen!‹ 10Sie sind ja doch deine Knechte
und dein Volk, das du durch deine große Kraft und deinen starken Arm
erlöst hast! 11Ach, HERR, laß doch dein Ohr aufmerken auf das Gebet
deines Knechtes und auf das Gebet deiner Knechte, die gewillt sind (ihre
Freude daran haben), deinen Namen zu fürchten! Laß es doch heute deinem
Knecht gelingen und laß ihn Erbarmen finden bei dem Manne hier!« Ich war
nämlich Mundschenk beim König.

\hypertarget{c-nehemia-erhuxe4lt-urlaub-und-vollmachten-vom-perserkuxf6nige-artaxerxes-zur-wiederherstellung-jerusalems}{%
\paragraph{c) Nehemia erhält Urlaub und Vollmachten vom Perserkönige
Artaxerxes zur Wiederherstellung
Jerusalems}\label{c-nehemia-erhuxe4lt-urlaub-und-vollmachten-vom-perserkuxf6nige-artaxerxes-zur-wiederherstellung-jerusalems}}

\hypertarget{section-1}{%
\section{2}\label{section-1}}

1Nun begab es sich im Monat Nisan im zwanzigsten Regierungsjahre des
Königs Arthasastha, als der Wein vor mir stand, da trug ich den Wein auf
und reichte ihn dem Könige; ich hatte aber früher nie betrübt vor ihm
ausgesehen, 2und so fragte mich der König: »Warum siehst du so betrübt
aus? Du bist doch nicht krank? Das kann nichts anderes als Herzenskummer
sein!« Da geriet ich in große Furcht, 3antwortete aber doch dem König:
»Lang lebe der König! Wie sollte ich nicht traurig aussehen, da doch die
Stadt, in der sich die Gräber meiner Väter befinden, in Trümmern liegt
und ihre Tore vom Feuer verzehrt sind?« 4Als der König mich nun fragte:
»Um was bittest du denn (unter diesen Umständen)?«, da betete ich zum
Gott des Himmels 5und sagte dann zum Könige: »Wenn es dem Könige gut
dünkt und dein Knecht Gnade bei dir findet, so wollest du mich nach Juda
senden zu der Stadt, wo meine Väter begraben liegen, damit ich sie
wieder aufbaue.« 6Da erwiderte mir der König, während die Königin neben
ihm saß: »Wie lange soll denn deine Reise dauern, und wann wirst du
wieder zurückkommen?« Weil es also dem König genehm schien, mich
hinreisen zu lassen, gab ich ihm eine bestimmte Frist an 7und sagte dann
zum König: »Wenn es dem Könige beliebt, so möge man mir Geleitbriefe an
die Statthalter der Provinz auf der Westseite des Euphrats mitgeben,
damit sie mich durchreisen lassen, bis ich nach Juda gelange; 8weiter
ein Schreiben an Asaph, den königlichen Forstmeister, daß er mir Holz
verabfolgen lasse, damit man die Tore der Burg, die zum Tempel gehört,
aus Balken zimmern kann, sowie für die Mauer der Stadt und für das Haus,
das ich selber beziehen werde.« Und der König bewilligte mir dies, weil
die gütige Hand meines Gottes über mir waltete.

\hypertarget{d-nehemias-reise-durch-das-persische-gebiet-der-provinz-syrien-beginn-der-feindschaft-angesehener-muxe4nner}{%
\paragraph{d) Nehemias Reise durch das persische Gebiet der Provinz
Syrien; Beginn der Feindschaft angesehener
Männer}\label{d-nehemias-reise-durch-das-persische-gebiet-der-provinz-syrien-beginn-der-feindschaft-angesehener-muxe4nner}}

9Als ich dann zu den Statthaltern (des Gebiets) auf der Westseite des
Euphrats kam, übergab ich ihnen die königlichen Geleitbriefe; der König
hatte mir aber Heeresoberste✲ und Reiter als Bedeckung mitgegeben. 10Als
dies der Horoniter Sanballat und der ammonitische Knecht\textless sup
title=``=~Beamte, Häuptling''\textgreater✲ Tobija\textless sup
title=``vgl. 6,17-18''\textgreater✲ erfuhren, verdroß es sie gewaltig,
daß jemand gekommen war, der für das Wohl der Israeliten sorgen wollte.

\hypertarget{nehemia-in-jerusalem-seine-vorbereitungen-zum-wiederaufbau-der-stadtmauer}{%
\subsubsection{2. Nehemia in Jerusalem; seine Vorbereitungen zum
Wiederaufbau der
Stadtmauer}\label{nehemia-in-jerusalem-seine-vorbereitungen-zum-wiederaufbau-der-stadtmauer}}

\hypertarget{a-nehemias-nuxe4chtliche-besichtigung-der-stadtmauern-seine-aufforderung-an-die-volksgenossen-zur-wiederherstellung-der-mauer}{%
\paragraph{a) Nehemias nächtliche Besichtigung der Stadtmauern; seine
Aufforderung an die Volksgenossen zur Wiederherstellung der
Mauer}\label{a-nehemias-nuxe4chtliche-besichtigung-der-stadtmauern-seine-aufforderung-an-die-volksgenossen-zur-wiederherstellung-der-mauer}}

11Als ich nun in Jerusalem angekommen war und drei Tage dort zugebracht
hatte, 12machte ich mich nachts in Begleitung einiger weniger von meinen
Leuten auf, ohne jedoch jemandem mitgeteilt zu haben, was mein Gott mir
in den Sinn gegeben hatte, für Jerusalem zu tun; ich hatte auch kein
anderes Reittier bei mir als das Maultier, auf dem ich ritt. 13So ritt
ich denn bei Nacht durch das Taltor hinaus in der Richtung zur
Drachenquelle und nach dem Misttor hin und besichtigte die Mauer
Jerusalems, die zerrissen dastand, und die Tore der Stadt, die vom Feuer
vernichtet dalagen. 14Dann ritt ich weiter zum Quellentor und zum
Königsteich, und als dort für das Tier, auf dem ich saß, kein Raum mehr
zum Durchkommen war, 15stieg ich bei Nacht die Schlucht zu Fuß hinauf
und besichtigte die Mauer; dann kehrte ich um und gelangte durch das
Taltor wieder heim; 16die Vorsteher aber wußten nicht, wohin ich
gegangen war und was ich zu tun vorhatte, denn ich hatte den Juden bis
dahin noch nichts mitgeteilt, weder den Priestern noch den Vornehmen,
weder den Vorstehern noch den übrigen, die am Bau arbeiten sollten.
17Nunmehr sagte ich zu ihnen: »Ihr seht das Elend, in dem wir uns
befinden, daß Jerusalem nämlich in Trümmern liegt und seine Tore mit
Feuer vernichtet sind. Kommt\textless sup title=``oder: wohlan
denn''\textgreater✲, laßt uns die Mauer Jerusalems wieder aufbauen,
damit wir nicht länger ein Gegenstand des Spottes sind!«

\hypertarget{b-zusage-der-vorsteher-der-gemeinde-der-spott-der-drei-heidnischen-gegner-von-nehemia-zuruxfcckgewiesen}{%
\paragraph{b) Zusage der Vorsteher der Gemeinde; der Spott der drei
heidnischen Gegner von Nehemia
zurückgewiesen}\label{b-zusage-der-vorsteher-der-gemeinde-der-spott-der-drei-heidnischen-gegner-von-nehemia-zuruxfcckgewiesen}}

18Hierauf teilte ich ihnen mit, wie gütig die Hand meines Gottes über
mir gewaltet hatte, und auch die Worte, die der König an mich gerichtet
hatte. Da erklärten sie: »Ja, wir wollen darangehen und bauen!« Und sie
ermutigten sich gegenseitig dazu, das gute Werk in Angriff zu nehmen.

19Als aber der Horoniter Sanballat und der ammonitische Knecht✲ Tobija
und der Araber Gesem Kunde davon erhielten, verhöhnten und verspotteten
sie uns und sagten: »Was ist denn das für eine Sache, die ihr da
vornehmt? Ihr wollt euch wohl gegen den König empören?« 20Da gab ich
ihnen folgende Antwort: »Der Gott des Himmels, der wird es uns gelingen
lassen; und wir wollen uns als seine Knechte an den Bau machen! Ihr aber
sollt weder Anteil, noch Anrecht, noch ein Gedächtnis in Jerusalem
haben!«

\hypertarget{stuxfcckweise-ausfuxfchrung-des-mauerbaues-liste-der-am-mauerbau-beteiligten}{%
\subsubsection{3. Stückweise Ausführung des Mauerbaues; Liste der am
Mauerbau
Beteiligten}\label{stuxfcckweise-ausfuxfchrung-des-mauerbaues-liste-der-am-mauerbau-beteiligten}}

\hypertarget{section-2}{%
\section{3}\label{section-2}}

1Hierauf machte sich der Hohepriester Eljasib mit seinen Amtsgenossen,
den Priestern, daran, das Schaftor neu zu bauen; das weihten sie und
setzten seine Torflügel ein und führten dann den Bau weiter bis zum Turm
Hammea, den sie weihten, und bis zum Turm Hananeel. 2Neben ihnen bauten
die Männer von Jericho; und neben ihnen baute Sakkur, der Sohn Imris.

3Sodann das Fischtor baute die Familie Senaa; sie führten das Gebälk auf
und setzten seine Torflügel, seine Klammern\textless sup title=``oder:
Schlösser''\textgreater✲ und Riegel\textless sup title=``oder:
Querbalken''\textgreater✲ ein. 4Neben ihnen besserte Meremoth aus, der
Sohn Urias, des Sohnes des Hakkoz; und neben ihm besserte Mesullam aus,
der Sohn Berechjas, des Sohnes Mesesabeels; und neben ihm besserte Zadok
aus, der Sohn Baanas. 5Ihm zur Seite besorgten die Thekoiter die
Ausbesserung, aber die Vornehmen unter ihnen hatten ihren Nacken nicht
unter den Dienst ihres Herrn (Nehemia) gebeugt.

6Das Tor der Altstadt besserten Jojada, der Sohn Paseahs, und Mesullam,
der Sohn Besodjas aus; sie führten das Gebälk auf und setzten seine
Torflügel, seine Klammern\textless sup title=``oder:
Schlösser''\textgreater✲ und seine Riegel\textless sup title=``oder:
Querbalken''\textgreater✲ ein. 7Neben ihnen besserten der Gibeoniter
Melatja und der Meronothiter Jadon aus samt den Leuten von Gibeon und
Mizpa, die zum Gerichtsstuhl des Statthalters der Provinz auf der
Westseite des Euphrats gehörten. 8Neben ihnen besserte Ussiel aus, der
Sohn Harhajas, und die Zunft der Goldschmiede; und ihnen zur Seite
besserte Hananja, einer von der Zunft der Salbenhändler, Jerusalem bis
an die breite Mauer aus. 9Neben ihnen arbeitete Rephaja, der Sohn Hurs,
der Vorsteher der einen Hälfte des Bezirks Jerusalem. 10Neben ihnen
besserte Jedaja aus, der Sohn Harumaphs, und zwar seinem Hause
gegenüber; und neben ihm besserte Hattus aus, der Sohn Hasabnejas.
11Eine zweite Strecke besserte Malkija, der Sohn Harims, und Hasub, der
Sohn Pahath-Moabs, aus, auch\textless sup title=``oder: bis
an''\textgreater✲ den Ofenturm. 12Neben ihnen besserte Sallum aus, der
Sohn des Hallohes, der Vorsteher der andern Hälfte des Bezirks
Jerusalem, er und seine Töchter.

13Das Taltor besserten Hanun und die Bewohner von Sanoah aus; sie bauten
es auf und setzten seine Türen, seine Klammern\textless sup
title=``oder: Schlösser''\textgreater✲ und Riegel\textless sup
title=``oder: Querbalken''\textgreater✲ ein und arbeiteten noch tausend
Ellen weiter an der Mauer bis zum Misttor.~-- 14Das Misttor selbst aber
besserte Malkija aus, der Sohn Rechabs, der Vorsteher des Bezirks
Beth-Cherem; er baute es auf und setzte seine Torflügel, seine
Klammern\textless sup title=``oder: Schlösser''\textgreater✲ und
Riegel\textless sup title=``oder: Querbalken''\textgreater✲ ein.~--
15Das Quellentor besserte Sallun aus, der Sohn Kol-Hoses, der Vorsteher
des Bezirks Mizpa; er baute es auf, überdachte es und setzte seine
Torflügel, seine Klammern\textless sup title=``oder:
Schlösser''\textgreater✲ und Riegel\textless sup title=``oder:
Querbalken''\textgreater✲ ein; dazu die Mauer am Teich der Wasserleitung
beim Königsgarten und bis an die Stufen, die von der Davidsstadt
herabführen.

16Nächst ihm besserte Nehemia aus, der Sohn Asbuks, der Vorsteher der
einen Hälfte des Bezirks Beth-Zur, bis gegenüber den Davidsgräbern und
weiter bis an den Teich, der dort angelegt worden war, und bis an die
Kaserne.~-- 17Nächst ihm besserten die Leviten aus: Rehum, der Sohn
Banis. Neben ihm besserte Hasabja aus, der Vorsteher der einen Hälfte
des Bezirks Kegila, für seinen Bezirk.~-- 18Nächst ihm besserten deren
Genossen aus: Binnui, der Sohn Henadads, der Vorsteher der andern Hälfte
des Bezirks Kegila.~-- 19Neben ihm besserte Eser aus, der Sohn Jesuas,
der Vorsteher der anderen Hälfte von Mizpa, eine zweite Strecke
gegenüber dem Aufstieg zum Zeughaus am Winkel.~-- 20Nächst ihm
bergaufwärts✲ besserte Baruch, der Sohn Sabbais\textless sup
title=``oder: Sakkais''\textgreater✲, eine zweite Strecke aus vom Winkel
bis an den Eingang zum Hause des Hohenpriesters Eljasib.~-- 21Nächst ihm
besserte Meremoth, der Sohn Urias, des Sohnes des Hakkoz, eine zweite
Strecke aus vom Eingang zum Hause Eljasibs bis ans Ende des Hauses
Eljasibs.~-- 22Nächst ihm besserten die Priester aus, die Männer aus dem
Jordanbezirk.~-- 23Nächst ihnen besserten Benjamin und Hassub ihrem
Hause gegenüber aus; nächst ihnen Asarja, der Sohn Maasejas, des Sohnes
Ananjas, neben seinem Hause.~-- 24Nächst ihm besserte Binnui, der Sohn
Henadads, eine zweite Strecke aus vom Hause Asarjas bis an den Winkel
und bis an die Ecke. 25Palal, der Sohn Usais, arbeitete gegenüber dem
Winkel und dem oberen Turm, der am königlichen Palast\textless sup
title=``oder: Schloß''\textgreater✲ beim Gefängnishof vorspringt. --
Nächst ihm besserte Pedaja aus, der Sohn des Parhos, 26 27Nächst ihm
besserten die Thekoiter eine zweite Strecke aus, dem großen
vorspringenden Turm gegenüber und bis an die Mauer des Ophel.

28Oberhalb des Roßtores besserten die Priester aus, ein jeder seinem
Hause gegenüber.~-- 29Nächst ihnen besserte Zadok, der Sohn Immers,
seinem Hause gegenüber aus, und nächst ihm Semaja, der Sohn Sechanjas,
der Hüter des Osttores.~-- 30Nächst ihm besserten Hananja, der Sohn
Selemjas, und Hanun, der sechste Sohn Zalaphs, eine zweite Strecke aus.
-- Nächst ihm besserte Mesullam, der Sohn Berechjas, seiner Wohnung
gegenüber aus.~-- 31Nächst ihm besserte Malkija, einer aus der Zunft der
Goldschmiede, bis zum Hause der Tempelhörigen und der Krämer dem
Wachttor gegenüber aus und bis zum Söller an der Ecke; 32und zwischen
dem Söller an der Ecke und dem Schaftor besserten die Goldschmiede und
die Krämer aus.

\hypertarget{fortfuxfchrung-des-mauerbaues-trotz-des-spottes-und-der-feindseligkeiten-der-heidnischen-widersacher}{%
\subsubsection{4. Fortführung des Mauerbaues trotz des Spottes und der
Feindseligkeiten der heidnischen
Widersacher}\label{fortfuxfchrung-des-mauerbaues-trotz-des-spottes-und-der-feindseligkeiten-der-heidnischen-widersacher}}

33Als nun Sanballat erfuhr, daß wir (wirklich) die Mauer wieder
aufbauten, geriet er in Zorn und heftigen Ärger und spottete über die
Juden, 34indem er in Gegenwart seiner Stammesgenossen und der
Kriegsleute von Samaria sagte: »Was machen die ohnmächtigen✲ Juden da?
Wird man sie gewähren lassen? Werden sie jemals (das Dank-) Opfer
darbringen? Werden sie damit eines Tages zu Ende kommen? Werden sie die
Steine, die doch verbrannt sind, aus den Schutthaufen
lebendig\textless sup title=``=~wieder brauchbar''\textgreater✲ machen?«
35Und der Ammoniter Tobija, der neben ihm stand, sagte: »Was sie auch
bauen mögen: springt nur ein Fuchs daran hinauf, so reißt er ihre
Steinmauer auseinander!«~-- 36Höre, unser Gott, wie wir zum Spott
geworden sind! Laß ihre Schmähungen auf ihr Haupt zurückfallen und gib
sie der Plünderung preis in einem Lande, wo man sie in Gefangenschaft
hält! 37Decke ihre Verschuldung nicht zu und laß ihre Sünde vor deinem
Angesicht nicht ausgelöscht werden! Denn sie haben durch kränkende Reden
gegen die am Bau Tätigen Ärgernis erregt.~-- 38Wir aber bauten an der
Mauer weiter; und als die ganze Mauer bis zur halben Höhe fertig war,
gewann das Volk neuen Mut zur Arbeit.

\hypertarget{neue-anschluxe4ge-der-gegner-auf-den-bau-nehemias-erfolgreiche-mauxdfregeln-dagegen}{%
\paragraph{Neue Anschläge der Gegner auf den Bau; Nehemias erfolgreiche
Maßregeln
dagegen}\label{neue-anschluxe4ge-der-gegner-auf-den-bau-nehemias-erfolgreiche-mauxdfregeln-dagegen}}

\hypertarget{section-3}{%
\section{4}\label{section-3}}

1Als aber Sanballat und Tobija sowie die Araber und Ammoniter und
Asdoditer erfuhren, daß die Wiederherstellung der Mauern Jerusalems
Fortschritte machte, und daß die Lücken sich zu schließen begannen, da
gerieten sie in heftigen Zorn 2und verschworen sich alle zusammen, sie
wollten hinziehen, um Jerusalem anzugreifen und Schaden\textless sup
title=``oder: Verwirrung''\textgreater✲ darin anzurichten. 3Da beteten
wir zu unserm Gott und stellten aus Furcht\textless sup title=``oder:
zum Schutz''\textgreater✲ vor ihnen bei Tag und Nacht Wachen gegen sie
auf. 4Aber die Judäer erklärten: »Die Kraft der Lastträger ist
erschöpft, und des Schuttes ist zu viel: wir sind nicht mehr imstande,
an der Mauer zu arbeiten!« 5Unsere Widersacher aber sagten: »Sie sollen
nichts merken und nichts sehen, bis wir mitten unter sie kommen und sie
totschlagen und so dem Bauen ein Ende machen!« 6Als nun die Juden, die
in ihrer Nachbarschaft wohnten, herbeikamen und es uns wohl zehnmal
sagten, aus allen Orten, von denen sie ab und zu gingen, 7da stellte ich
hinter der Mauer an den tieferen Stellen das Volk nach den Geschlechtern
mit ihren Schwertern, Lanzen und Bogen auf. 8Bei einer Besichtigung trat
ich dann auf und sagte zu den Vornehmen und Vorstehern und zu dem
übrigen Volke: »Fürchtet euch doch nicht vor ihnen! Denkt an den HERRN,
den großen und furchtbaren Gott, und kämpft für eure Volksgenossen, eure
Söhne und Töchter, eure Frauen und Häuser!« 9Als nun unsere Feinde
erfuhren, daß die Sache\textless sup title=``=~ihr Plan''\textgreater✲
zu unserer Kenntnis gekommen war und Gott ihren Anschlag vereitelt
hatte, kehrten wir alle wieder zu der Mauer zurück, ein jeder an seine
Arbeit.

10Seit jenem Tage aber war nur die eine Hälfte meiner\textless sup
title=``d.h. der zu meinem Gefolge gehörenden''\textgreater✲ Leute am
Bau tätig, während die andere Hälfte sich mit Lanzen, Schilden, Bogen
und Panzern bereithielt und die Oberen✲ hinter der ganzen jüdischen
Bevölkerung standen, 11die an der Mauer baute. Die Handlanger aber,
welche Lasten trugen, arbeiteten in der Weise, daß sie mit der einen
Hand die Arbeit verrichteten, in der andern aber die Waffe\textless sup
title=``d.h. den Speer oder: Wurfspieß''\textgreater✲ hielten; 12und von
den Bauleuten hatte jeder sein Schwert um die Hüften gegürtet und
mauerte so; und der Trompeter✲ stand neben mir. 13Den Vornehmen aber und
Vorstehern und dem übrigen Volk hatte ich die Weisung gegeben: »Das Werk
ist groß und weit ausgedehnt, und wir sind auf der Mauer zerstreut,
einer von dem andern weit entfernt. 14An dem Punkte also, von dem her
ihr den Schall der Trompete vernehmen werdet, da müßt ihr euch bei uns
sammeln: unser Gott wird für uns kämpfen!«

15So waren wir an dem Werke tätig, und zwar so, daß die eine Hälfte der
Leute die Lanzen vom Aufgang der Morgenröte bis zum Erscheinen der
Sterne bereit hielt. 16Auch befahl ich damals dem Volke: »Jeder soll mit
seinen Leuten die Nacht über innerhalb Jerusalems verbleiben, damit sie
uns nachts als Wachen und bei Tage als Arbeiter dienen.« 17Und weder
ich, noch meine Brüder, noch meine Leute\textless sup title=``oder:
Diener''\textgreater✲, noch die Wachmannschaften, die zu meinem Gefolge
gehörten, keiner von uns kam jemals aus den Kleidern heraus: ein jeder
hatte stets seine Waffe zur Hand.

\hypertarget{erleichterung-der-notlage-des-niedrigen-volkes-durch-den-schuldenerlauxdf-nehemias-uneigennuxfctzige-amtsfuxfchrung}{%
\subsubsection{5. Erleichterung der Notlage des niedrigen Volkes durch
den Schuldenerlaß; Nehemias uneigennützige
Amtsführung}\label{erleichterung-der-notlage-des-niedrigen-volkes-durch-den-schuldenerlauxdf-nehemias-uneigennuxfctzige-amtsfuxfchrung}}

\hypertarget{a-beschwerden-und-unwille-der-unteren-volksschichten-wegen-der-hartherzigkeit-ihrer-gluxe4ubiger}{%
\paragraph{a) Beschwerden und Unwille der unteren Volksschichten wegen
der Hartherzigkeit ihrer
Gläubiger}\label{a-beschwerden-und-unwille-der-unteren-volksschichten-wegen-der-hartherzigkeit-ihrer-gluxe4ubiger}}

\hypertarget{section-4}{%
\section{5}\label{section-4}}

1Es erhob sich aber ein großes Klagegeschrei der Leute aus dem Volk und
ihrer Frauen gegen ihre jüdischen Volksgenossen. 2Die einen sagten: »Wir
müssen unsere Söhne und Töchter verpfänden, um Getreide zu erhalten,
damit wir zu essen haben und am Leben bleiben!« 3Andere sagten: »Wir
müssen unsere Felder, unsere Weinberge und Häuser verpfänden, um uns
Getreide in der Teurung zu verschaffen!« 4Wieder andere sagten: »Wir
haben Geld zur Bezahlung der Steuern für den König auf unsere Felder und
Weinberge borgen müssen. 5Wir sind aber doch von demselben Fleisch und
Blut wie unsere Volksgenossen, und unsere Kinder sind ebenso gut wie
ihre Kinder; aber trotzdem müssen wir unsere Söhne und Töchter als
Leibeigene\textless sup title=``=~in die Sklaverei''\textgreater✲
hingeben, und manche von unsern Töchtern sind schon leibeigen geworden,
und wir können nichts dagegen tun: unsere Felder und Weinberge gehören
ja anderen Leuten!«

\hypertarget{b-abstellung-der-uxfcbelstuxe4nde-durch-die-beschluxfcsse-der-volksversammlung}{%
\paragraph{b) Abstellung der Übelstände durch die Beschlüsse der
Volksversammlung}\label{b-abstellung-der-uxfcbelstuxe4nde-durch-die-beschluxfcsse-der-volksversammlung}}

6Da geriet ich in heftigen Zorn, als ich ihre lauten Klagen und diese
Reden vernahm. 7Als ich dann mit mir zu Rate gegangen war, machte ich
den Vornehmen und den Vorstehern Vorwürfe, indem ich zu ihnen sagte:
»Wucher treibt ihr ja einer mit dem andern!« Dann veranstaltete ich eine
große Volksversammlung gegen sie 8und sagte zu ihnen: »Wir haben unsere
jüdischen Volksgenossen, die an die Heidenvölker verkauft waren, soweit
es uns möglich war, losgekauft; ihr dagegen wollt nun gar eure eigenen
Volksgenossen verkaufen, so daß sie dann wieder von uns gekauft werden
müssen!« Als sie nun schwiegen und kein Wort der Entgegnung fanden,
9fuhr ich fort: »Was ihr da tut, ist unwürdig! Ihr solltet doch in der
Furcht unsers Gottes wandeln, damit wir unseren heidnischen Feinden
keine Veranlassung zu Lästerungen geben! 10Sowohl ich als auch meine
Brüder und meine Diener\textless sup title=``oder: Leute''\textgreater✲
haben ihnen Geld und Getreide geliehen: laßt uns ihnen doch dieses
Darlehen erlassen! 11Gebt ihnen doch gleich heute ihre Felder und
Weinberge, ihre Ölgärten und Häuser zurück und (erlaßt ihnen), was ihr
an Geld und Getreide, an Wein und Öl von ihnen zu fordern habt!« 12Da
antworteten sie: »Ja, wir wollen es zurückgeben und nichts mehr von
ihnen fordern: wir wollen so tun, wie du es verlangst!« Da rief ich die
Priester herbei und ließ sie\textless sup title=``d.h. die
Gläubiger''\textgreater✲ schwören, daß sie wirklich in dieser Weise
verfahren wollten. 13Dazu schüttelte ich den Bausch meines Gewandes aus
mit den Worten: »Ebenso möge Gott jeden, der dieses sein Versprechen
nicht hält, aus seinem Hause und seinem Besitz herausschütteln, damit er
ebenso ausgeschüttelt und ausgeleert✲ sei!« Da rief die ganze
Versammlung: »Ja, so sei es!« und pries den HERRN; das Volk aber tat,
wie abgemacht war.

\hypertarget{c-nehemias-uneigennuxfctzigkeit-wuxe4hrend-seiner-amtsfuxfchrung}{%
\paragraph{c) Nehemias Uneigennützigkeit während seiner
Amtsführung}\label{c-nehemias-uneigennuxfctzigkeit-wuxe4hrend-seiner-amtsfuxfchrung}}

14Außerdem habe ich und meine Brüder von dem Tage an, wo (der König)
mich zu ihrem Statthalter im Lande Juda bestellt hatte, d.h. vom
zwanzigsten bis zum zweiunddreißigsten Jahre der Regierung des Königs
Arthasastha\textless sup title=``vgl. 1,1''\textgreater✲, also zwölf
Jahre lang, keinen Anspruch auf den Unterhalt\textless sup title=``oder:
das Einkommen''\textgreater✲ des Statthalters gemacht, 15während die
früheren Statthalter, meine Vorgänger, dem Volke schwer zur Last
gefallen waren; denn sie hatten für Speise und Wein✲ täglich vierzig
Schekel Silber von ihnen bezogen, und auch ihre Dienerschaft hatte über
das Volk willkürlich geschaltet. Ich dagegen habe aus Gottesfurcht nicht
so gehandelt. 16Auch bei diesem Mauerbau habe ich mit Hand angelegt,
ohne daß wir Grundbesitz erworben hatten; und auch meine ganze
Dienerschaft ist dort zur Arbeit am Mauerbau versammelt gewesen. 17Dazu
aßen die Juden, sowohl die Vorsteher, hundertundfünfzig Mann, als auch
die, welche aus den umwohnenden heidnischen Völkerschaften zu uns auf
Besuch kamen, an meinem Tisch; 18und was täglich zubereitet wurde,
nämlich ein Rind, sechs ausgesuchte Stück Kleinvieh sowie Geflügel, das
wurde auf meine Kosten zubereitet; außerdem alle zehn Tage allerlei Wein
in Menge. Trotzdem habe ich den Unterhalt\textless sup title=``oder: das
Einkommen''\textgreater✲ des Statthalters nicht beansprucht, weil die
Fronarbeit schon schwer genug auf diesem Volke lastete.~-- 19Gedenke,
mein Gott, mir zum Guten\textless sup title=``oder: zum
Segen''\textgreater✲ alles dessen, was ich für dieses Volk getan habe!

\hypertarget{nachstellungen-der-feinde-gegen-nehemia-vollendung-des-mauerbaues}{%
\subsubsection{6. Nachstellungen der Feinde gegen Nehemia; Vollendung
des
Mauerbaues}\label{nachstellungen-der-feinde-gegen-nehemia-vollendung-des-mauerbaues}}

\hypertarget{a-ruxe4nke-und-mordanschlag-sanballats-und-seiner-genossen-ihre-zuruxfcckweisung-durch-nehemia}{%
\paragraph{a) Ränke (und Mordanschlag) Sanballats und seiner Genossen;
ihre Zurückweisung durch
Nehemia}\label{a-ruxe4nke-und-mordanschlag-sanballats-und-seiner-genossen-ihre-zuruxfcckweisung-durch-nehemia}}

\hypertarget{section-5}{%
\section{6}\label{section-5}}

1Als es nun dem Sanballat, dem Tobija und dem Araber Gesem sowie unseren
übrigen Feinden bekannt wurde, daß ich die Mauer wieder aufgebaut hätte
und daß keine Lücke\textless sup title=``oder: kein Riß''\textgreater✲
mehr in ihr geblieben wäre -- nur hatte ich bis dahin noch keine
Türflügel in die Tore eingesetzt --, 2da sandten Sanballat und Gesem zu
mir und ließen mir sagen: »Komm, laß uns in Ha-Kaphirim in der Ebene Ono
eine Zusammenkunft halten!« Sie hatten nämlich Böses gegen mich im Sinn.
3Da schickte ich Boten zu ihnen und ließ ihnen antworten: »Ich bin mit
einem bedeutenden Werke beschäftigt und kann deshalb nicht hinabkommen:
das Werk würde sofort stille stehen, wenn ich es unterbräche und zu euch
hinunterkäme.« 4Nun sandten sie auf dieselbe Weise viermal Botschaft zu
mir, ich gab ihnen aber immer dieselbe Antwort. 5Da sandte Sanballat auf
dieselbe Weise noch zum fünftenmale seinen Burschen zu mir mit einem
offenen Briefe in der Hand, 6in welchem geschrieben stand: »Unter den
Leuten geht das Gerücht um, und Gasmu sagt es auch, daß ihr, du und die
Juden, an Empörung denkt; darum bauest du die Mauer wieder auf, und du
selbst wollest dich zum König über sie machen und dergleichen mehr;
7sogar Propheten habest du auftreten lassen, die dich in Jerusalem zum
König von Juda ausrufen sollen. Nun werden solche Gerüchte aber dem
Könige zu Ohren dringen; darum komm, laß uns zusammen ratschlagen!«
8Darauf sandte ich folgenden Bescheid an ihn: »Nichts von allem, was du
behauptest, ist wirklich geschehen, sondern du hast das selbst frei
erfunden!« 9Sie alle wollten uns nämlich nur bange machen, weil sie
dachten: »Sie werden von der Arbeit schon ablassen, so daß das Werk
nicht vollendet wird.« -- Nun aber stärke meine Hände!

\hypertarget{b-entlarvung-eines-falschen-propheten}{%
\paragraph{b) Entlarvung eines falschen
Propheten}\label{b-entlarvung-eines-falschen-propheten}}

10Als ich (einmal) in die Wohnung Semajas, des Sohnes Delajas, des
Sohnes Mehetabeels, kam, der gerade durch Unreinheit verhindert war,
sagte er: »Laß uns zusammen ins Haus Gottes gehen, in das Innerste des
Tempels, und die Türen des Tempels verschließen! Denn es werden Leute
kommen, um dich zu ermorden, und zwar werden sie bei Nacht kommen, um
dich zu ermorden.« 11Doch ich entgegnete: »Ein Mann wie ich sollte
fliehen? Und wie könnte jemand, wie ich bin, in den Tempel gehen und am
Leben bleiben? Nein, ich gehe nicht hinein!« 12Ich hatte nämlich
gemerkt, daß nicht Gott ihn gesandt✲ hatte, sondern er hatte den
Gottesspruch deshalb an mich gerichtet, weil Tobija und Sanballat ihn
bestochen hatten; 13und zwar war er zu dem Zweck bestochen worden, daß
ich in Angst geraten und so handeln und mich dadurch versündigen sollte;
das hätte ihnen dann zu übler Nachrede dienen können, um mich in Verruf
zu bringen. 14Gedenke✲, mein Gott, dem Tobija und Sanballat diese ihre
Handlungsweise und auch der Prophetin Noadja und den übrigen Propheten,
die mich ängstlich zu machen suchten!

\hypertarget{c-vollendung-des-mauerbaus-verduxe4chtiger-briefwechsel-tobijas-mit-vielen-ihm-ergebenen-juden}{%
\paragraph{c) Vollendung des Mauerbaus; verdächtiger Briefwechsel
Tobijas mit vielen ihm ergebenen
Juden}\label{c-vollendung-des-mauerbaus-verduxe4chtiger-briefwechsel-tobijas-mit-vielen-ihm-ergebenen-juden}}

15Die Mauer aber wurde am fünfundzwanzigsten Tage des Monats Elul nach
Verlauf von zweiundfünfzig Tagen fertig. 16Als nun alle unsere Feinde
das erfuhren, erschraken alle heidnischen Völkerschaften rings um uns
her, und es entfiel ihnen aller Mut; denn sie erkannten, daß dieses Werk
unter der Mitwirkung unseres Gottes vollführt worden war.~-- 17Auch
ließen in jenen Tagen die vornehmen Juden zahlreiche Briefe an Tobija
abgehen, wie solche auch von Tobija an sie ankamen. 18Es gab nämlich
unter den Juden gar manche, die ihm eidlich zum Beistand verpflichtet
waren; denn er war der Schwiegersohn Sechanjas, des Sohnes Arahs, und
sein Sohn Johanan hatte die Tochter Mesullams, des Sohnes Berechjas,
geheiratet. 19Sie redeten sogar in meiner Gegenwart von seinen guten
Eigenschaften\textless sup title=``oder: Absichten''\textgreater✲ und
hinterbrachten ihm meine Äußerungen; auch sandte Tobija Briefe, um mich
einzuschüchtern.

\hypertarget{nehemias-sorge-fuxfcr-die-sicherheit-der-stadt}{%
\paragraph{Nehemias Sorge für die Sicherheit der
Stadt}\label{nehemias-sorge-fuxfcr-die-sicherheit-der-stadt}}

\hypertarget{section-6}{%
\section{7}\label{section-6}}

1Als nun die Mauer aufgebaut war und ich die Türflügel hatte einsetzen
lassen, wurden die Torwächter angestellt. 2Dann übertrug ich den
Oberbefehl über Jerusalem meinem Bruder Hanani und dem Burghauptmann
Hananja; denn dieser war ein zuverlässiger und gottesfürchtiger Mann wie
wenige. 3Ich gab ihnen die Weisung: »Die Tore Jerusalems dürfen nicht
eher geöffnet werden, als bis die Sonne heiß scheint; und während (die
Torwächter) noch dastehen, soll man die Torflügel schließen und
verriegeln. Auch sollt ihr Wachen aus den Bürgern Jerusalems aufstellen,
einen jeden auf seinem Posten und jeden vor seinem Hause.«

\hypertarget{nehemias-sorge-fuxfcr-vermehrung-der-bevuxf6lkerung-jerusalems-verzeichnis-der-vordem-mit-serubbabel-aus-der-gefangenschaft-zuruxfcckgekehrten-israeliten}{%
\subsubsection{7. Nehemias Sorge für Vermehrung der Bevölkerung
Jerusalems; Verzeichnis der vordem mit Serubbabel aus der Gefangenschaft
zurückgekehrten
Israeliten}\label{nehemias-sorge-fuxfcr-vermehrung-der-bevuxf6lkerung-jerusalems-verzeichnis-der-vordem-mit-serubbabel-aus-der-gefangenschaft-zuruxfcckgekehrten-israeliten}}

4Die Stadt war nun zwar geräumig und groß, aber die Bevölkerung in ihr
nur spärlich, und neugebaute Häuser waren nicht vorhanden. 5Da gab mein
Gott mir den Gedanken ein, die Vornehmen und Vorsteher und das Volk zu
versammeln, damit ein Geschlechtsverzeichnis von ihnen aufgenommen
würde. Da fand ich das Geschlechtsverzeichnis derer, die zuerst (oder
früher) aus der Gefangenschaft zurückgekehrt waren, und fand darin
folgende Angaben:

6Folgendes sind die Bewohner des Bezirks\textless sup title=``d.h. der
Provinz Juda''\textgreater✲, die aus der Gefangenschaft der in der
Verbannung Lebenden, die Nebukadnezar, der König von Babylon, einst
(nach Babylon) weggeführt hatte, hinaufgezogen und nach Jerusalem und
Juda zurückgekehrt sind, ein jeder in seine Ortschaft\textless sup
title=``vgl. Esr 2,1''\textgreater✲, 7und zwar sind sie dorthin gekommen
zusammen mit Serubbabel, Jesua, Nehemia, Asarja, Raamja, Nahamani,
Mordechai, Bilsan, Mispereth✲, Bigwai, Nehum und Baana.

Die Zahl der Männer des Volkes Israel betrug: 8die Familie Parhos 2172;
9die Familie Sephatja 372; 10die Familie Arah 652; 11die Familie
Pahath-Moab, nämlich die Familien Jesua und Joab, 2818; 12die Familie
Elam 1254; 13die Familie Satthu 845; 14die Familie Sakkai 760; 15die
Familie Binnui 648; 16die Familie Bebai 628; 17die Familie Asgad 2322;
18die Familie Adonikam 667; 19die Familie Bigwai 2067; 20die Familie
Adin 655; 21die Familie Ater, nämlich der Zweig Hiskia, 98; 22die
Familie Hasum 328; 23die Familie Bezai 324; 24die Familie Hariph 112;
25die Familie Gibeon 95; 26die Männer von Bethlehem und Netopha 188;
27die Männer von Anathoth 128; 28die Männer von Beth-Asmaweth 42; 29die
Männer von Kirjath-Jearim, Kephira und Beeroth 743; 30die Männer von
Rama und Geba 621; 31die Männer von Michmas 122; 32die Männer von Bethel
und Ai 123; 33die Männer von {[}dem andern{]} Nebo 52; 34die Familie des
andern Elam 1254\textless sup title=``vgl. V.12''\textgreater✲; 35die
Familie Harim 320; 36die Leute von Jericho 345; 37die Leute von Lod,
Hadid und Ono 721; 38die Familie Senaa 3930.

39Die Priester: die Familie Jedaja, nämlich des Hauses Jesua 973; 40die
Familie Immer 1052; 41die Familie Pashur 1247; 42die Familie Harim 1017.

43Die Leviten: die Familie Jesua, nämlich die Familien Kadmiel, (Bani)
und Hodawja 74;~-- 44die Sänger: die Familie Asaph 148;~-- 45die
Torhüter: die Familien Sallum, Ater, Talmon, Akkub, Hatita und Sobai
138.

46Die Tempelhörigen: die Familie Ziha, die Familie Hasupha, die Familie
Thabbaoth, 47die Familie Keros, die Familie Sia, die Familie Padon,
48die Familie Lebana, die Familie Hagaba, die Familie Salmai, 49die
Familie Hanan, die Familie Giddel, die Familie Gahar, 50die Familie
Reaja, die Familie Rezin, die Familie Nekoda, 51die Familie Gassam, die
Familie Ussa, die Familie Paseah, 52die Familie Besai, die Familie der
Mehuniter, die Familie der Nephisiter, 53die Familie Bakbuk, die Familie
Hakupha, die Familie Harhur, 54die Familie Bazluth✲, die Familie Mehida,
die Familie Harsa, 55die Familie Barkos, die Familie Sisera, die Familie
Themah, 56die Familie Neziah, die Familie Hatipha.~-- 57Die Familien der
Sklaven\textless sup title=``oder: Leibeigenen''\textgreater✲ Salomos:
die Familie Sotai, die Familie Sophereth, die Familie Perida, 58die
Familie Jaala, die Familie Darkon, die Familie Giddel, 59die Familie
Sephatja, die Familie Hattil, die Familie Pochereth-Hazzebaim, die
Familie Amon. 60Die Gesamtzahl der Tempelhörigen und der Familien der
Sklaven\textless sup title=``oder: Leibeigenen''\textgreater✲ Salomos
betrug 392.

61Und dies sind die, welche aus Thel-Melah, Thel-Harsa, Cherub-Addon und
Immer mit hinaufgezogen sind, aber ihre Familie und ihre Abkunft nicht
nachweisen konnten, ob sie nämlich aus Israel stammten: 62die Familie
Delaja, die Familie Tobija und die Familie Nekoda: 642. 63Sodann von den
Priestern: die Familie Habaja, die Familie Hakkoz, die Familie jenes
Barsillais, der eine Frau von den Töchtern des Gileaditen Barsillai
geheiratet und deren Namen angenommen hatte. 64Diese hatten zwar nach
einer Geschlechtsurkunde gesucht, aber eine solche hatte sich nicht
finden lassen; infolgedessen wurden sie als unrein\textless sup
title=``oder: untauglich''\textgreater✲ vom Priestertum ausgeschlossen;
65und der Statthalter hatte ihnen erklärt, daß sie von dem Hochheiligen
nicht essen dürften, bis wieder ein Priester für die Befragung des Urim-
und Thummim-Orakels\textless sup title=``2.Mose 28,30''\textgreater✲ da
wäre.

66Die ganze Gemeinde insgesamt belief sich auf 42360 Seelen,
67ungerechnet ihre Sklaven und Sklavinnen, deren 7337 da waren; außerdem
hatten sie noch 245 Sänger und Sängerinnen. 68Die Zahl ihrer Pferde
betrug 736, ihrer Maultiere 245, ihrer Kamele 435 und ihrer Esel 6720.

69Manche von den Familienhäuptern spendeten Beiträge für den
Gottesdienst. Der Statthalter schenkte für den Schatz: an Gold 1000
Dariken, 50 Sprengschalen, 30 Priestergewänder. 70Von den
Familienhäuptern gaben einige für den Gottesdienst: an Gold 20000
Dariken und an Silber 2200 Minen; 71und was das übrige Volk gab, betrug
an Gold 20000 Dariken und an Silber 2000 Minen und 67 Priestergewänder.

72So siedelten sich denn die Priester und die Leviten sowie die
Torhüter, die Sänger und die Tempelhörigen in Jerusalem und dessen
Gebiet an, alle übrigen Israeliten dagegen in ihren Ortschaften.

\hypertarget{ii.-esras-neuordnung-des-gottesdienstes-und-religionswesens-772-1040}{%
\subsection{II. Esras Neuordnung des Gottesdienstes und Religionswesens
(7,72-10,40)}\label{ii.-esras-neuordnung-des-gottesdienstes-und-religionswesens-772-1040}}

\hypertarget{verlesung-des-gesetzes-durch-esra-und-feier-des-laubhuxfcttenfestes}{%
\subsubsection{1. Verlesung des Gesetzes durch Esra und Feier des
Laubhüttenfestes}\label{verlesung-des-gesetzes-durch-esra-und-feier-des-laubhuxfcttenfestes}}

\hypertarget{a-buuxdffest-der-gemeinde-mit-verlesung-des-mosaischen-gesetzes}{%
\paragraph{a) Bußfest der Gemeinde mit Verlesung des mosaischen
Gesetzes}\label{a-buuxdffest-der-gemeinde-mit-verlesung-des-mosaischen-gesetzes}}

Als nun der siebte Monat herankam, während die Israeliten sich in ihren
Ortschaften befanden,

\hypertarget{section-7}{%
\section{8}\label{section-7}}

1da versammelte sich das ganze Volk bis auf den letzten Mann auf dem
Platze vor dem Wassertor und richtete an Esra, den Schriftgelehrten, die
Bitte, er möchte das Buch des mosaischen Gesetzes
herbringen\textless sup title=``oder: holen''\textgreater✲, das der HERR
den Israeliten geboten hatte. 2Da brachte denn der Priester Esra das
Gesetz vor die Versammlung\textless sup title=``oder:
Gemeinde''\textgreater✲ sowohl der Männer als der Frauen, vor alle, die
befähigt waren, es zu verstehen, am ersten Tage des siebten Monats, 3und
er las auf dem freien Platze vor dem Wassertor von Tagesanbruch bis
Mittag den Männern und Frauen, überhaupt allen, die ein Verständnis
dafür hatten, daraus vor; und das ganze Volk schenkte der Vorlesung aus
dem Gesetzbuch aufmerksames Gehör. 4Esra, der Schriftgelehrte, stand
dabei auf einem hölzernen Gerüst, das man zu diesem Zweck hergestellt
hatte, und neben ihm standen auf seiner rechten Seite Matthithja, Sema,
Anaja, Urija, Hilkija und Maaseja, zu seiner Linken dagegen Pedaja,
Misael, Malkija, Hasum, Hasbaddana, Sacharja und Mesullam. 5Esra schlug
dann das Buch vor den Augen des ganzen Volkes auf -- er stand nämlich
höher als das ganze Volk --; und als er es aufschlug, erhob sich die
ganze Versammlung. 6Danach pries Esra den HERRN, den großen Gott, und
das ganze Volk antwortete: »Amen, Amen!« unter Emporheben der Hände;
dann verneigten sie sich und warfen sich vor dem HERRN nieder, das
Angesicht zur Erde gewandt. 7Darauf erteilten die Leviten Jesua, Bani,
Serebja, Jamin, Akkub, Sabbethai, Hodija, Maaseja, Kelita, Asarja,
Josabad, Hanan und Pelaja dem Volke Belehrung über das Gesetz, während
das Volk auf seiner Stelle stehen blieb. 8So lasen sie denn aus dem
Buche, dem Gesetz Gottes, abschnittweise vor und machten den Sinn klar,
so daß sie (die Zuhörenden) das Verständnis des Vorgelesenen gewannen.

\hypertarget{b-nehemias-und-esras-aufforderung-an-das-trauernde-volk-den-tag-in-festlicher-freude-zu-begehen}{%
\paragraph{b) Nehemias und Esras Aufforderung an das trauernde Volk, den
Tag in festlicher Freude zu
begehen}\label{b-nehemias-und-esras-aufforderung-an-das-trauernde-volk-den-tag-in-festlicher-freude-zu-begehen}}

9Hierauf sagte Nehemia -- dieser war nämlich Statthalter -- und der
Priester Esra, der Schriftgelehrte, nebst den Leviten, die das Volk
unterwiesen, folgendes zu dem ganzen Volke: »Dieser Tag ist dem HERRN,
eurem Gott, heilig; seid nicht traurig und weint nicht!« Das ganze Volk
hatte nämlich beim Anhören der Worte des Gesetzes zu weinen begonnen.
10Dann fuhr er fort: »Geht hin, eßt fette Speisen und trinkt süße
Getränke und laßt auch denen, für die nichts zubereitet\textless sup
title=``oder: vorrätig''\textgreater✲ ist, Anteile\textless sup
title=``oder: Portionen''\textgreater✲ zukommen, denn der Tag ist unserm
Herrn heilig! Darum seid nicht niedergeschlagen, denn die Freude am
HERRN ist eure Stärke.« 11So beruhigten denn die Leviten das ganze Volk,
indem sie sagten: »Seid still, denn der Tag ist heilig, und seid nicht
niedergeschlagen!« 12Da ging das ganze Volk hin, um zu essen und zu
trinken und (den Dürftigen) Anteile zukommen zu lassen und ein großes
Freudenfest zu feiern; denn sie hatten die Worte verstanden, die man
ihnen kundgetan hatte.

\hypertarget{c-feier-des-laubhuxfcttenfestes-unter-bestuxe4ndiger-gesetzesverlesung}{%
\paragraph{c) Feier des Laubhüttenfestes unter beständiger
Gesetzesverlesung}\label{c-feier-des-laubhuxfcttenfestes-unter-bestuxe4ndiger-gesetzesverlesung}}

13Am zweiten Tage aber versammelten sich die Familienhäupter des ganzen
Volkes sowie die Priester und die Leviten bei Esra, dem
Schriftgelehrten, und zwar um Kenntnis vom Wortlaut\textless sup
title=``oder: Inhalt''\textgreater✲ des Gesetzes zu erhalten. 14Da
fanden sie im Gesetz, das der HERR durch Mose geboten hatte,
geschrieben\textless sup title=``3.Mose 23,40-42''\textgreater✲, die
Israeliten sollten während des Festes im siebten Monat in Laubhütten
wohnen 15und sollten in allen ihren Ortschaften und in Jerusalem
ausrufen und laut verkündigen lassen: »Zieht auf die Berge hinaus und
holt Zweige vom edlen und vom wilden Ölbaum, Zweige von Myrten, Palmen
und anderen dichtbelaubten Bäumen, um Laubhütten daraus zu bauen, wie
geschrieben steht!« 16Da zog das Volk hinaus, holte (solche Laubzweige)
und machte sich Hütten\textless sup title=``oder: Lauben''\textgreater✲
daraus, ein jeder auf seinem Dache oder in ihren Höfen sowie in den
Höfen✲ des Hauses Gottes und auf dem Platze am Wassertor und auf dem
Platze am Ephraimtor. 17So baute sich denn die ganze Gemeinde, alle, die
aus der Gefangenschaft zurückgekehrt waren, solche Laubhütten und
wohnten in den Hütten. Seit den Tagen Josuas, des Sohnes Nuns, nämlich
bis auf jenen Tag hatten die Israeliten (das Fest) nicht in dieser Weise
gefeiert; und es herrschte sehr große Freude. 18Man\textless sup
title=``oder: Esra''\textgreater✲ las dann aber aus dem Gesetzbuche
Gottes Tag für Tag vor, vom ersten bis zum letzten Tage; und sie
feierten das Fest sieben Tage lang, und am achten Tage fand
vorschriftgemäß eine Festversammlung statt.

\hypertarget{der-grouxdfe-buuxdftag-und-das-buuxdfgebet-des-volkes}{%
\subsubsection{2. Der große Bußtag und das Bußgebet des
Volkes}\label{der-grouxdfe-buuxdftag-und-das-buuxdfgebet-des-volkes}}

\hypertarget{a-feier-des-buuxdftages-unter-mehrstuxfcndiger-gesetzesverlesung-und-mehrstuxfcndiger-beichte}{%
\paragraph{a) Feier des Bußtages unter mehrstündiger Gesetzesverlesung
und mehrstündiger
Beichte}\label{a-feier-des-buuxdftages-unter-mehrstuxfcndiger-gesetzesverlesung-und-mehrstuxfcndiger-beichte}}

\hypertarget{section-8}{%
\section{9}\label{section-8}}

1Am vierundzwanzigsten Tage desselben Monats aber versammelten sich die
Israeliten unter\textless sup title=``oder: zu einem''\textgreater✲
Fasten, und zwar in Trauergewändern und mit Erde auf dem Haupte.
2Nachdem sich dann die Vollisraeliten von allen Fremden abgesondert
hatten, traten sie hin und legten ein Bekenntnis ihrer Sünden und der
Verschuldungen ihrer Väter ab. 3Hierauf erhoben sie sich auf der Stelle,
wo sie sich befanden, und man las aus dem Gesetzbuche des HERRN, ihres
Gottes, einen Vierteltag lang\textless sup title=``d.h. drei Stunden
lang''\textgreater✲ vor und sprach dann drei weitere Stunden lang
Bußgebete, während sie sich vor dem HERRN, ihrem Gott, niedergeworfen
hatten. 4Darauf traten Jesua und Bani, Kadmiel, Sebanja, Bunni, Serabja,
Bani und Kenani auf den erhöhten Platz\textless sup title=``d.h. die
Tribüne oder Kanzel''\textgreater✲ der Leviten hinauf und riefen den
HERRN, ihren Gott, mit lauter Stimme an. 5Alsdann hielten die Leviten
Jesua und Kadmiel, Bani, Hasabneja, Serebja, Hodija, Sebanja und
Pethahja folgende Ansprache:

\hypertarget{b-das-buuxdfgebet}{%
\paragraph{b) Das Bußgebet}\label{b-das-buuxdfgebet}}

\hypertarget{aa-aufforderung-zum-preise-gottes-hinweis-auf-die-wunderbaren-machttaten-und-gnadenerweise-gottes-in-der-vorzeit-bis-zur-hinfuxfchrung-seines-volkes-in-das-verheiuxdfene-land}{%
\subparagraph{aa) Aufforderung zum Preise Gottes; Hinweis auf die
wunderbaren Machttaten und Gnadenerweise Gottes in der Vorzeit bis zur
Hinführung seines Volkes in das verheißene
Land}\label{aa-aufforderung-zum-preise-gottes-hinweis-auf-die-wunderbaren-machttaten-und-gnadenerweise-gottes-in-der-vorzeit-bis-zur-hinfuxfchrung-seines-volkes-in-das-verheiuxdfene-land}}

»Ach! Preiset den HERRN, euren Gott, von Ewigkeit zu Ewigkeit! Und man
preise deinen herrlichen Namen, der über allen Lobpreis und Ruhm erhaben
ist! 6Du bist es, der da ist, HERR, du allein! Du bist es, der den
Himmel und den obersten\textless sup title=``oder:
innersten''\textgreater✲ Himmel samt ihrem ganzen Heer geschaffen hat,
die Erde mit allem, was auf ihr ist, die Meere mit allem, was in ihnen
ist; und du bist es, der dies alles am Leben erhält und den das
himmlische Heer anbetet. 7Du, HERR, bist der Gott, der Abram erwählt,
der ihn aus Ur in Chaldäa hat auswandern lassen und ihm den Namen
Abraham gegeben hat. 8Nachdem du sein Herz treu gegen dich erfunden
hattest, hast du mit ihm den Bund geschlossen, das Land der Kanaanäer,
Hethiter, Amoriter, Pherissiter, Jebusiter und Girgasiter, dies Land
seinen Nachkommen geben zu wollen; und du hast dein Wort
gehalten\textless sup title=``=~deine Verheißung erfüllt''\textgreater✲,
denn du bist gerecht.

9Als du dann das Elend unserer Väter in Ägypten sahst und ihr Geschrei
am Schilfmeere hörtest, 10hast du Zeichen und Wunder am Pharao, an allen
seinen Dienern und an dem ganzen Volke seines Landes getan; denn du
hattest erkannt, daß jene in Vermessenheit gegen sie gehandelt hatten,
und du hast dir einen Namen gemacht, wie er heute noch groß dasteht.
11Das Meer hast du vor ihnen gespalten, so daß sie trockenen Fußes
mitten durch das Meer ziehen konnten; ihre Verfolger aber hast du in die
Tiefen geschleudert wie einen Stein in gewaltige Fluten. 12Durch eine
Wolkensäule hast du sie bei Tage geleitet und durch eine Feuersäule bei
Nacht, um ihnen den Weg zu erleuchten, auf dem sie ziehen sollten. 13Auf
den Berg Sinai bist du hinabgestiegen und hast vom Himmel her mit ihnen
geredet und ihnen richtige Weisungen und zuverlässige Gesetze, gute
Satzungen und Gebote gegeben. 14Auch deinen heiligen Sabbat✲ hast du
ihnen kundgetan und ihnen Gebote, Satzungen und das Gesetz durch deinen
Knecht Mose verordnet. 15Brot vom Himmel hast du ihnen für ihren Hunger
gegeben und Wasser aus dem Felsen ihnen für ihren Durst hervorfließen
lassen und hast ihnen geboten, in das Land einzuziehen, dessen Besitz du
ihnen mit erhobener Hand zugeschworen hattest.«

\hypertarget{bb-suxfcndhaftes-verhalten-des-volkes-gegenuxfcber-den-segnungen-gottes-wuxe4hrend-der-wuxfcstenwanderung-und-bei-der-besitzergreifung-des-verheiuxdfenen-landes}{%
\subparagraph{bb) Sündhaftes Verhalten des Volkes gegenüber den
Segnungen Gottes während der Wüstenwanderung und bei der
Besitzergreifung des verheißenen
Landes}\label{bb-suxfcndhaftes-verhalten-des-volkes-gegenuxfcber-den-segnungen-gottes-wuxe4hrend-der-wuxfcstenwanderung-und-bei-der-besitzergreifung-des-verheiuxdfenen-landes}}

16»Sie aber, unsere Väter, waren übermütig und halsstarrig, so daß sie
auf deine Gebote nicht hörten; 17sie weigerten sich vielmehr zu
gehorchen und gedachten deiner Wunder nicht mehr, die du an ihnen getan
hattest: sie wurden halsstarrig und setzten es sich in ihrer
Widerspenstigkeit in den Kopf, nach Ägypten zu ihrem Sklavendienst
zurückzukehren. Doch du bist ein Gott der Vergebung, gnädig und
barmherzig, langmütig und reich an Güte: du hast sie nicht verlassen.
18Sogar als sie sich ein gegossenes Stierbild gemacht hatten und
ausriefen: ›Dies ist dein Gott, der dich aus Ägypten geführt hat!‹, und
als sie arge Lästerdinge verübten, 19hast du sie doch nach deiner großen
Barmherzigkeit in der Wüste nicht verlassen; nein, die Wolkensäule wich
nicht von ihnen bei Tage, die sie auf dem Wege führen sollte, und die
Feuersäule nicht bei Nacht, um ihnen den Weg zu erleuchten, auf dem sie
ziehen sollten. 20Du gabst ihnen auch deinen guten Geist, um sie zu
unterweisen; du versagtest ihrem Munde dein Manna nicht und gabst ihnen
Wasser für ihren Durst. 21Vierzig Jahre lang versorgtest du sie in der
Wüste, so daß sie keinen Mangel litten; ihre Kleider nutzten sich nicht
ab, und ihre Füße schwollen nicht an. 22Dazu gabst du ihnen Königreiche
und Völker zum Besitz und teiltest ihnen Gebiet für Gebiet zu, so daß
sie das Land Sihons, des Königs von Hesbon, und das Land Ogs, des Königs
von Basan, in Besitz nahmen. 23Ihre Söhne\textless sup title=``oder:
Kinder''\textgreater✲ ließest du zahlreich werden wie die Sterne am
Himmel und brachtest sie in das Land, in das sie, wie du ihren Vätern
verheißen hattest, eindringen sollten, um es in Besitz zu nehmen. 24So
zogen denn ihre Söhne in das Land ein und nahmen es in Besitz, und du
warfst die Bewohner des Landes, die Kanaanäer, vor ihnen nieder und
ließest sie in ihre Gewalt fallen, sowohl ihre Könige als auch die
Völkerschaften des Landes, damit sie mit ihnen nach Belieben verfahren
könnten. 25So eroberten sie denn feste Städte und ein fruchtbares Land
und nahmen Häuser in Besitz, die mit Gütern aller Art angefüllt waren,
ausgehauene Brunnen✲, Weinberge und Ölbaumgärten und Obstbäume in Menge;
und sie aßen und wurden satt und fett\textless sup title=``oder:
reich''\textgreater✲ und ließen sich's wohl sein im Genuß der Fülle
deiner Güter.«

\hypertarget{cc-im-besitz-des-landes-setzt-das-volk-unter-verachtung-der-propheten-und-der-guxf6ttlichen-langmut-sein-suxfcndhaftes-verhalten-fort-bis-es-von-gott-in-die-huxe4nde-der-heiden-ausgeliefert-wird}{%
\subparagraph{cc) Im Besitz des Landes setzt das Volk unter Verachtung
der Propheten und der göttlichen Langmut sein sündhaftes Verhalten fort,
bis es von Gott in die Hände der Heiden ausgeliefert
wird}\label{cc-im-besitz-des-landes-setzt-das-volk-unter-verachtung-der-propheten-und-der-guxf6ttlichen-langmut-sein-suxfcndhaftes-verhalten-fort-bis-es-von-gott-in-die-huxe4nde-der-heiden-ausgeliefert-wird}}

26»Aber sie wurden ungehorsam und lehnten sich gegen dich auf; sie
kehrten deinem Gesetz den Rücken; sie ermordeten deine Propheten, die
ihnen ins Gewissen redeten, um sie zu dir zurückzuführen, und verübten
arge Lästerdinge. 27Darum gabst du sie der Gewalt ihrer Feinde preis,
daß diese sie bedrängten. Wenn sie dann aber in ihrer Not zu dir
schrien, erhörtest du sie vom Himmel her und ließest ihnen nach deiner
großen Barmherzigkeit Retter erstehen, die sie aus der Gewalt ihrer
Bedränger erretteten. 28Sobald sie aber Ruhe hatten, fingen sie wieder
an, Böses vor dir zu tun; und wenn du sie dann wieder in die Gewalt
ihrer Feinde fallen ließest, die sie unter ihre Herrschaft knechteten,
und sie aufs neue zu dir schrien, erhörtest du sie vom Himmel her und
errettetest sie oftmals in deiner großen Barmherzigkeit. 29Obgleich du
sie aber ernstlich warnen ließest, um sie zu deinem Gesetz
zurückzuführen, waren sie doch trotzig und gehorchten deinen Geboten
nicht, sondern sündigten gegen deine Verordnungen, obwohl der Mensch
doch durch deren Beobachtung sein Leben bewahrt\textless sup
title=``oder: das Leben gewinnt''\textgreater✲; sie wollten sich kein
Joch auf ihre Schulter legen lassen und waren halsstarrig, so daß sie
nicht gehorchten. 30Obgleich du nun noch viele Jahre lang Geduld mit
ihnen hattest und sie durch deinen Geist, durch deine Propheten,
ernstlich warnen ließest, achteten sie doch nicht darauf. Da hast du sie
in die Gewalt der Völker in den heidnischen Ländern fallen lassen,
31aber sie trotzdem in deiner großen Barmherzigkeit nicht völlig
vernichtet und sie nicht verlassen; denn du bist ein gnädiger und
barmherziger Gott.«

\hypertarget{dd-bitte-um-neue-gnade-und-treue-und-um-erleichterung-der-allerdings-wohlverdienten-leiden-seit-der-assyrerherrschaft-bis-auf-die-gegenwart}{%
\subparagraph{dd) Bitte um neue Gnade und Treue und um Erleichterung der
allerdings wohlverdienten Leiden seit der Assyrerherrschaft bis auf die
Gegenwart}\label{dd-bitte-um-neue-gnade-und-treue-und-um-erleichterung-der-allerdings-wohlverdienten-leiden-seit-der-assyrerherrschaft-bis-auf-die-gegenwart}}

32»Und nun, unser Gott, du großer, starker und furchtbarer Gott, der du
den Bund und die Gnade\textless sup title=``=~deinen
Gnadenbund''\textgreater✲ bewahrst: achte nicht gering alle die Leiden,
die uns betroffen haben, unsere Könige und Obersten\textless sup
title=``oder: Fürsten''\textgreater✲, unsere Priester und Propheten,
unsere Väter und dein ganzes Volk seit der Zeit der Assyrerkönige bis
auf diesen Tag! 33Du bist allerdings gerecht gewesen bei allem, was uns
widerfahren ist; denn du hast stets Treue geübt, wir aber haben gottlos
gehandelt. 34Auch unsere Könige und Obersten, unsere Priester und unsere
Väter haben dein Gesetz nicht gehalten und deine Gebote und ernstlichen
Warnungen, die du ihnen hast zukommen lassen, unbeachtet gelassen.
35Weil sie trotz ihres Königtums\textless sup title=``oder: ihrer
königlichen Würde''\textgreater✲ und trotz der Fülle von Wohltaten, die
du ihnen erwiesen, und trotz des weiten und fruchtbaren Landes, das du
ihnen zugeteilt hattest, dir nicht gedient und sich nicht von ihrem
bösen Tun bekehrt haben,~-- 36ja, ebendarum sind wir heute Knechte, und
das Land, das du unsern Vätern geschenkt hast, damit sie seine Früchte
und Güter genössen: ach, wir sind Knechte in ihm! 37Seinen reichen
Ertrag liefert es den Königen, die du um unserer Sünden willen über uns
gesetzt hast, und sie herrschen über unsere Leiber und über unser Vieh
nach ihrem Gutdünken, so daß wir uns in großer Not befinden.«

\hypertarget{erneuerung-des-bundes-mit-gott-urkunde-uxfcber-die-leistungen-der-gemeinde-fuxfcr-gottesdienstliche-zwecke}{%
\subsubsection{3. Erneuerung des Bundes mit Gott; Urkunde über die
Leistungen der Gemeinde für gottesdienstliche
Zwecke}\label{erneuerung-des-bundes-mit-gott-urkunde-uxfcber-die-leistungen-der-gemeinde-fuxfcr-gottesdienstliche-zwecke}}

\hypertarget{a-bundeserneuerung-durch-einen-schriftlichen-und-untersiegelten-vertrag-der-vorsteher-besonders-familienhuxe4upter-des-volkes}{%
\paragraph{a) Bundeserneuerung durch einen schriftlichen und
untersiegelten Vertrag der Vorsteher (besonders Familienhäupter) des
Volkes}\label{a-bundeserneuerung-durch-einen-schriftlichen-und-untersiegelten-vertrag-der-vorsteher-besonders-familienhuxe4upter-des-volkes}}

\hypertarget{section-9}{%
\section{10}\label{section-9}}

1Auf Grund aller dieser Umstände schließen wir einen festen Vertrag und
fertigen ihn schriftlich aus; und auf der untersiegelten Urkunde stehen
die Namen unserer Obersten\textless sup title=``oder:
Fürsten''\textgreater✲, unserer Leviten und unserer Priester; 2und auf
dem untersiegelten Schriftstück stehen die Namen✲: Nehemia, der
Statthalter, der Sohn Hachaljas, und Zedekia, 3Seraja, Asarja, Jeremia,
4Pashur, Amarja, Malkija, 5Hattus, Sechanja✲, Malluch, 6Harim, Meremoth,
Obadja, 7Daniel, Ginnethon, Baruch, 8Mesullam, Abija, Mijamin, 9Maasja,
Bilgai, Semaja; dies waren die Priester. 10Sodann die Leviten: Jesua,
der Sohn Asanjas, Binnui aus der Familie Henadad, Kadmiel; 11und ihre
Genossen: Sebanja, Hodawja, Kelita, Pelaja, Hanan, 12Micha, Rehob,
Hasabja, 13Sakkur, Serebja, Sebanja, 14Hodija, Bani, Beninu. 15Sodann
die Häupter des Volkes: Parhos, Pahath-Moab, Elam, Satthu, Bani,
16Bunni, Asgad, Bebai, 17Adonija, Bigwai, Adin, 18Ater, Hiskia, Assur,
19Hodija, Hasum, Bezai, 20Hariph, Anathoth, Nobai, 21Magpias, Mesullam,
Hesir, 22Mesesabeel, Zadok, Jaddua, 23Pelatja, Hanan, Anaja, 24Hosea,
Hananja, Hassub, 25Hallohes, Pilha, Sobek, 26Rehum, Hasabna, Maaseja,
27und Ahija, Hanan, Anan, 28Malluch, Harim, Baana.

\hypertarget{b-anschluuxdf-des-volkes-an-den-bund-unter-uxfcbernahme-bestimmter-verpflichtungen}{%
\paragraph{b) Anschluß des Volkes an den Bund unter Übernahme bestimmter
Verpflichtungen}\label{b-anschluuxdf-des-volkes-an-den-bund-unter-uxfcbernahme-bestimmter-verpflichtungen}}

\hypertarget{aa-vermeidung-der-mischehen-und-der-sabbatentheiligung}{%
\subparagraph{aa) Vermeidung der Mischehen und der
Sabbatentheiligung}\label{aa-vermeidung-der-mischehen-und-der-sabbatentheiligung}}

29Das übrige Volk aber, die Priester, Leviten, Torhüter, Sänger,
Tempelhörigen und alle, die sich von der heidnischen Bevölkerung der
Landesteile abgesondert haben und sich zum Gesetz Gottes halten, samt
ihren Frauen, Söhnen und Töchtern, alle, welche Einsicht (und)
Verständnis dafür haben, 30schließen sich hiermit ihren Volksgenossen,
den Vornehmen unter ihnen, an und verpflichten sich durch einen Eid
unwiderruflich, nach dem Gesetz Gottes, das durch Mose, den Knecht
Gottes, gegeben worden ist, zu wandeln und alle Gebote Gottes, unseres
HERRN, seine Verordnungen und Satzungen zu beobachten und zu erfüllen.
31Wir wollen also weder unsere Töchter den heidnischen Landesbewohnern
zu Frauen geben noch ihre Töchter für unsere Söhne zu Frauen nehmen.
32Wenn ferner die heidnischen Landesbewohner am Sabbattage Waren und
Getreide aller Art zum Verkauf herbringen, wollen wir ihnen am Sabbat
und an einem (andern) heiligen Tage nichts abkaufen. Wir wollen ferner
in jedem siebten Jahre die Felder unbestellt liegen lassen und auf jede
Schuldforderung (in dem betreffenden Jahre) verzichten\textless sup
title=``vgl. 2.Mose 23,11; 5.Mose 15,1-2''\textgreater✲.

\hypertarget{bb-rechtzeitige-und-reichliche-leistung-aller-abgaben-und-verpflichtungen-fuxfcr-den-gottesdienst-und-die-priesterschaft}{%
\subparagraph{bb) Rechtzeitige und reichliche Leistung aller Abgaben und
Verpflichtungen für den Gottesdienst und die
Priesterschaft}\label{bb-rechtzeitige-und-reichliche-leistung-aller-abgaben-und-verpflichtungen-fuxfcr-den-gottesdienst-und-die-priesterschaft}}

33Weiter verpflichten wir uns unwiderruflich dazu, ein Drittel Schekel
jährlich als Steuer für den heiligen Dienst im Hause unsers Gottes zu
entrichten, 34nämlich für die Schaubrote und für das tägliche
Speisopfer, für das tägliche Brandopfer und für die Opfer an den
Sabbaten, an den Neumonden und den Festen, für die Heilsopfer und für
die Sündopfer, um Sühnung für Israel zu erwirken, überhaupt für den
gesamten Dienst im Hause unsers Gottes. 35Ferner haben wir, die
Priester, die Leviten und das Volk, durch das Los über die
Holzlieferungen entscheiden lassen, damit wir diese familienweise zu
bestimmten Zeiten Jahr für Jahr an das Haus unsers Gottes gelangen
lassen, zur Feuerung auf dem Altar des HERRN, unsers Gottes, wie im
Gesetz vorgeschrieben ist. 36Ebenso verpflichten wir uns, die Erstlinge
unserer Felder und die Erstlinge aller Früchte von allen Bäumen Jahr für
Jahr zum Hause des HERRN zu bringen, 37auch unsere erstgeborenen Söhne
und die Erstlinge unsers Viehes, wie es im Gesetz vorgeschrieben ist,
und zwar die Erstlinge unserer Rinder und unsers Kleinviehs in das Haus
unsers Gottes für die Priester zu bringen, die den Dienst im Hause
unsers Gottes verrichten. 38Ferner wollen wir das Beste von unserm
Schrotmehl und von unseren Heilsopfern sowie von allen Baumfrüchten, von
Wein und Öl den Priestern in die Zellen des Hauses unsers Gottes liefern
und den Zehnten von unseren Feldern an die Leviten; denn die Leviten
sind es, die den Zehnten in allen Ortschaften erheben, wo wir Ackerbau
treiben. 39Und wenn die Leviten den Zehnten einfordern, soll der
(betreffende) Priester, ein Nachkomme Aarons, die Leviten begleiten, und
die Leviten sollen dann den Zehnten von ihrem Zehnten zum Hause unsers
Gottes in die Zellen des Vorratshauses hinaufbringen. 40Denn in die
Zellen sollen sowohl die Israeliten als auch die Leviten die Abgaben vom
Getreide, vom Most und vom Öl liefern; denn dort befinden sich auch die
heiligen Geräte sowie die diensttuenden Priester, die Torhüter und die
Sänger. Und so wollen wir es dem Hause unsers Gottes an nichts fehlen
lassen.

\hypertarget{iii.-innere-zustuxe4nde-in-jerusalem-feierliche-einweihung-der-stadtmauer-kap.-11-13}{%
\subsection{III. Innere Zustände in Jerusalem; feierliche Einweihung der
Stadtmauer (Kap.
11-13)}\label{iii.-innere-zustuxe4nde-in-jerusalem-feierliche-einweihung-der-stadtmauer-kap.-11-13}}

\hypertarget{mauxdfregeln-zur-mehrung-der-einwohnerzahl-jerusalems-verzeichnisse-der-bevuxf6lkerung-jerusalems-und-des-juxfcdischen-gebiets}{%
\subsubsection{1. Maßregeln zur Mehrung der Einwohnerzahl Jerusalems;
Verzeichnisse der Bevölkerung Jerusalems und des jüdischen
Gebiets}\label{mauxdfregeln-zur-mehrung-der-einwohnerzahl-jerusalems-verzeichnisse-der-bevuxf6lkerung-jerusalems-und-des-juxfcdischen-gebiets}}

\hypertarget{a-ein-zehntel-der-landbevuxf6lkerung-wird-durch-das-los-zur-uxfcbersiedelung-nach-jerusalem-bestimmt}{%
\paragraph{a) Ein Zehntel der Landbevölkerung wird durch das Los zur
Übersiedelung nach Jerusalem
bestimmt}\label{a-ein-zehntel-der-landbevuxf6lkerung-wird-durch-das-los-zur-uxfcbersiedelung-nach-jerusalem-bestimmt}}

\hypertarget{section-10}{%
\section{11}\label{section-10}}

1Hierauf nahmen die Obersten\textless sup title=``oder: Fürsten,
Vornehmen''\textgreater✲ des Volkes ihren Wohnsitz in Jerusalem; das
übrige✲ Volk aber bestimmte durch das Los je den zehnten Mann dazu, sich
in der heiligen Stadt Jerusalem anzusiedeln, während die übrigen neun
Zehntel in den Ortschaften (des Landes) wohnen bleiben sollten. 2Das
Volk aber hieß alle Leute willkommen, die sich freiwillig zur
Niederlassung in Jerusalem erboten.

\hypertarget{b-listen-der-in-jerusalem-wohnhaften-huxe4upter-der-juduxe4er-und-benjaminiten-auch-der-priester-torhuxfcter-u.a.}{%
\paragraph{b) Listen der in Jerusalem wohnhaften Häupter der Judäer und
Benjaminiten (auch der Priester, Torhüter
u.a.)}\label{b-listen-der-in-jerusalem-wohnhaften-huxe4upter-der-juduxe4er-und-benjaminiten-auch-der-priester-torhuxfcter-u.a.}}

3Und dies sind die Häupter des Bezirks\textless sup title=``oder: der
Provinz''\textgreater✲, die sich in Jerusalem und in den Ortschaften
Judas niedergelassen haben, und zwar ein jeder auf seinem Besitztum in
den dortigen Ortschaften: die (gewöhnlichen) Israeliten, die Priester
und die Leviten, die Tempelhörigen und die Nachkommen der Leibeigenen
Salomos\textless sup title=``vgl. 1.Chr 9,2''\textgreater✲.

4In Jerusalem haben sich sowohl Judäer als auch Benjaminiten
niedergelassen, und zwar von den Judäern: Athaja, der Sohn Ussijas, des
Sohnes Sacharjas, des Sohnes Amarjas, des Sohnes Sephatjas, des Sohnes
Mahalaleels, von den Nachkommen des Perez; 5ferner Maaseja, der Sohn
Baruchs, des Sohnes Kol-Hoses, des Sohnes Hasajas, des Sohnes Adajas,
des Sohnes Jojaribs, des Sohnes Sacharjas, des Sohnes des Siloniten.
6Die Gesamtzahl der Nachkommen des Perez, die in Jerusalem wohnten,
betrug 468, tüchtige\textless sup title=``oder: wehrhafte''\textgreater✲
Männer.~-- 7Und dies sind die Benjaminiten: Sallu, der Sohn Mesullams,
des Sohnes Joeds, des Sohnes Pedajas, des Sohnes Kolajas, des Sohnes
Maasejas, des Sohnes Ithiels, des Sohnes Jesajas, 8und seine Genossen,
wehrhafte Krieger, 928 an der Zahl.~-- 9Joel aber, der Sohn Sichris, war
ihr Vorsteher; und Juda, der Sohn Hassenuas, war zweiter Aufseher über
die Stadt.~-- 10Von den Priestern: Jedaja, der Sohn Jojaribs, Jachin,
11Seraja, der Sohn Hilkijas, des Sohnes Mesullams, des Sohnes Zadoks,
des Sohnes Merajoths, des Sohnes Ahitubs, der Fürst\textless sup
title=``oder: Oberaufseher''\textgreater✲ über das Haus Gottes, 12und
ihre Genossen, die den heiligen Dienst im Hause Gottes besorgten, 822 an
Zahl; ferner Adaja, der Sohn Jerohams, des Sohnes Pelaljas, des Sohnes
Amzis, des Sohnes Sacharjas, des Sohnes Pashurs, des Sohnes Malkijas,
13und seine Genossen, 242, Familienhäupter; ferner Amassai, der Sohn
Asareels, des Sohnes Ahsais, des Sohnes Mesillemoths, des Sohnes Immers,
14und ihre Genossen, tüchtige Männer, 128 an der Zahl; ihr Vorsteher war
Sabdiel, der Sohn Haggedolims.~-- 15Ferner von den Leviten: Semaja, der
Sohn Hassubs, des Sohnes Asrikams, des Sohnes Hasabjas, des Sohnes
Bunnis; 16ferner Sabbethai und Josabad, welche die weltlichen Geschäfte
des Gotteshauses zu besorgen hatten; 17und Matthanja, der Sohn Michas,
des Sohnes Sabdis, des Sohnes Asaphs, der Leiter des Lobgesangs, der die
Danksagung beim Gebet anstimmte; und Bakbukja, der zweite im Rang unter
seinen Genossen, und Abda, der Sohn Sammuas, des Sohnes Galals, des
Sohnes Jeduthuns. 18Die Gesamtzahl der Leviten in der heiligen Stadt
betrug 284.~-- 19Die Torhüter aber waren: Akkub, Talmon und ihre
Genossen, die an den Toren Wache hielten, 172 an der Zahl.

20Die übrigen Israeliten aber, Priester und Leviten, wohnten in allen
Ortschaften Judas zerstreut, ein jeder in seinem Besitztum. 21Die
Tempelhörigen aber wohnten auf dem Ophel; Ziha und Gispa waren die
Aufseher über die Tempelhörigen.~-- 22Der Vorsteher der Leviten in
Jerusalem war Ussi, der Sohn Banis, des Sohnes Hasabjas, des Sohnes
Matthanjas, des Sohnes Michas, einer von den Nachkommen Asaphs, den
Sängern für den Dienst im Hause Gottes; 23es lag nämlich eine königliche
Verfügung in bezug auf sie vor und eine Verordnung für die Sänger
bezüglich ihrer täglichen Amtsleistungen. 24Pethahja aber, der Sohn
Mesesabeels, aus der Zahl der Nachkommen Serahs, des Sohnes Judas, war
der königliche Beamte für alle Angelegenheiten, die das Volk betrafen,
25und für die Dörfer\textless sup title=``oder: Gehöfte''\textgreater✲
auf ihren\textless sup title=``oder: den zugehörigen''\textgreater✲
Feldmarken.

\hypertarget{c-verzeichnis-von-ortschaften-die-damals-von-juduxe4ern-benjaminiten-und-leviten-besiedelt-wurden}{%
\paragraph{c) Verzeichnis von Ortschaften, die damals von Judäern,
Benjaminiten und Leviten besiedelt
wurden}\label{c-verzeichnis-von-ortschaften-die-damals-von-juduxe4ern-benjaminiten-und-leviten-besiedelt-wurden}}

Von den Judäern wohnte ein Teil in Kirjath-Arba nebst den zugehörigen
Ortschaften sowie in Dibon nebst den zugehörigen Ortschaften und in
Jekabzeel nebst den zugehörigen Gehöften; 26ferner in Jesua, Molada,
Beth-Pelet, 27in Hazar-Sual und Beerseba nebst den zugehörigen
Ortschaften, 28in Ziklag und Mechona nebst den zugehörigen Ortschaften,
29in En-Rimmon, Zora, Jarmuth, 30Sanoah, Adullam nebst den zugehörigen
Gehöften, Lachis und dessen Feldmarken und in Aseka nebst den
zugehörigen Ortschaften; sie hatten sich also von Beerseba bis zum Tal
Hinnom angesiedelt. 31Die Benjaminiten aber wohnten von Geba an in
Michmas, Ajja und Bethel nebst den zugehörigen Ortschaften, 32in
Anathoth, Nob, Ananja, 33Hazor, Rama, Hitthaim, 34Hadid, Zeboim,
Neballat, 35Lod und Ono (und) im Tal der Zimmerleute. 36Von den Leviten
aber gehörten einige judäische Abteilungen zu Benjamin.

\hypertarget{listen-von-priester--und-levitenklassen-familienhuxe4uptern-u.a.}{%
\subsubsection{2. Listen von Priester- und Levitenklassen,
Familienhäuptern
u.a.}\label{listen-von-priester--und-levitenklassen-familienhuxe4uptern-u.a.}}

\hypertarget{a-priester--und-levitenklassen-die-mit-serubbabel-und-jesua-heimkehrten}{%
\paragraph{a) Priester- und Levitenklassen, die mit Serubbabel und Jesua
heimkehrten}\label{a-priester--und-levitenklassen-die-mit-serubbabel-und-jesua-heimkehrten}}

\hypertarget{section-11}{%
\section{12}\label{section-11}}

1Folgende sind die Priester und die Leviten, die mit Serubbabel, dem
Sohne Sealthiels, und mit Jesua (nach Jerusalem) hinaufgezogen waren:
Seraja, Jeremia, Esra, 2Amarja, Malluch, Hattus, 3Sechanja, Harim,✲
Meremoth, 4Iddo, Ginnethoi, Abia, 5Mijjamin, Maadja, Bilga, 6Semaja und
Jojarib, Jedaja, 7Sallu, Amok, Hilkija, Jedaja. Das waren die Häupter
der Priester und ihrer Genossen zur Zeit Jesuas.~-- 8Die Leviten aber
waren: Jesua, Binnui, Kadmiel, Serebja, Juda, Matthanja; dieser und
seine Genossen leiteten den Dankgesang, 9während ihre Genossen Bakbukja
und Unni nach den Dienstabteilungen ihnen gegenüberstanden.

\hypertarget{die-hohepriesterliche-linie}{%
\paragraph{Die hohepriesterliche
Linie}\label{die-hohepriesterliche-linie}}

10Jesua war der Vater Jojakims, Jojakim der Vater Eljasibs, Eljasib der
Vater Jojadas, 11Jojada der Vater Johanans, Johanan der Vater Jadduas.

\hypertarget{b-priesterliche-familienhuxe4upter-aus-der-zeit-des-hohenpriesters-jojakim}{%
\paragraph{b) Priesterliche Familienhäupter aus der Zeit des
Hohenpriesters
Jojakim}\label{b-priesterliche-familienhuxe4upter-aus-der-zeit-des-hohenpriesters-jojakim}}

12Zur Zeit Jojakims✲ aber waren folgende Priester die Häupter der
Familien: von der Familie Seraja: Meraja, von Jeremia: Hananja, 13von
Esra: Mesullam, von Amarja: Johanan, 14von Malluch: Jonathan, von
Sechanja: Joseph, 15von Harim: Adna, von Merajoth : Helkai, 16von Iddo:
Sacharja, von Ginneton: Mesullam, 17von Abija: Sichri, von
Minjamin:~\ldots, von Moadja: Piltai, 18von Bilga: Sammua, von Semaja:
Jonathan, 19von Jojarib: Matthenai, von Jedaja: Ussi, 20von Sallu:
Kallai, von Amok: Eber, 21von Hilkija: Hasabja, von Jedaja: Nethaneel.

\hypertarget{c-levitenliste-bis-zur-zeit-des-hohenpriesters-johanan}{%
\paragraph{c) Levitenliste bis zur Zeit des Hohenpriesters
Johanan}\label{c-levitenliste-bis-zur-zeit-des-hohenpriesters-johanan}}

22Was die Leviten betrifft, so sind ihre Familienhäupter zur Zeit
Eljasibs, Jojadas, Johanans und Jadduas aufgezeichnet worden, von den
Priestern aber unter\textless sup title=``oder: bis zu''\textgreater✲
der Regierung des Perserkönigs Darius.~-- 23Von den Leviten sind die
Familienhäupter im Buch der Chronik\textless sup title=``oder:
Zeitgeschichte; vgl. 1.Chr 24-25''\textgreater✲ aufgezeichnet, und zwar
bis auf die Zeit Johanans, des Sohnes✲ Eljasibs. 24Die Häupter der
Leviten waren: Hasabja, Serebja und Jesua, der Sohn Kadmiels, und ihre
Genossen, die ihnen gegenüberstanden, um die Lob- und Danklieder zu
singen nach der Anordnung Davids, des Mannes Gottes, eine Abteilung
neben\textless sup title=``oder: abwechselnd mit''\textgreater✲ der
andern. 25Matthanja und Bakbukja, Obadja, Mesullam, Talmon und Akkub
hielten als Torhüter Wache bei den Vorratshäusern an den Toren. 26Diese
waren Zeitgenossen Jojakims, des Sohnes Jesuas, des Sohnes Jozadaks, und
Zeitgenossen des Statthalters Nehemia und des schriftgelehrten Priesters
Esra.

\hypertarget{feierliche-einweihung-der-stadtmauer}{%
\subsubsection{3. Feierliche Einweihung der
Stadtmauer}\label{feierliche-einweihung-der-stadtmauer}}

27Bei Gelegenheit der Einweihung der Mauer Jerusalems aber suchte man
die Leviten zu veranlassen, aus allen ihren Wohnorten nach Jerusalem zu
kommen, damit die Einweihung durch ein Freudenfest begangen würde mit
Dankliedern und Lobgesängen, mit Zimbeln, Harfen und Zithern. 28Da
versammelten sich die zu den (Tempel-)- Sängern Gehörenden sowohl aus
der ganzen Umgegend von Jerusalem als auch aus den Gehöften von Netopha
29sowie aus Beth-Gilgal und aus den Feldmarken von Geba und Asmaweth;
die Sänger hatten sich nämlich Gehöfte in der Umgegend von Jerusalem
gebaut. 30Nachdem nun die Priester und die Leviten sich selbst gereinigt
und dann auch das Volk sowie die Tore und die Mauer gereinigt hatten,
31ließ ich die Obersten\textless sup title=``oder:
Fürsten''\textgreater✲ Judas oben auf die Mauer steigen und stellte zwei
große Dankchöre und Festzüge auf, von denen der eine oben auf der Mauer
südwärts zum Misttor hin zog, 32und hinter ihnen her schritt Hosaja mit
der einen Hälfte der Obersten von Juda, 33nämlich Asarja, Esra und
Mesullam, 34Juda, Benjamin, Semaja und Jeremia; 35ferner einige
Mitglieder der Priesterschaft mit Trompeten, nämlich Sacharja, der Sohn
Jonathans, des Sohnes Semajas, des Sohnes Matthanjas, des Sohnes
Michajas, des Sohnes Sakkurs, des Sohnes Asaphs, 36und seine Genossen
Semaja und Asareel, Milalai, Gilalai, Maai, Nethaneel und Juda, Hanani,
mit den Tonwerkzeugen✲ Davids, des Mannes Gottes; und der
Schriftgelehrte Esra ging an ihrer Spitze. 37Weiter zogen sie zum
Quelltor hin, stiegen dann geradeaus auf den Stufen der Davidsstadt den
Aufgang zur Mauer hinauf und dann oberhalb des Palastes Davids bis zum
Wassertor im Osten. 38Der zweite Festchor aber zog nach der
entgegengesetzten Seite, und ich selbst ging hinter ihm her mit der
andern Hälfte (der Obersten) des Volkes oben auf der Mauer hin, am
Ofenturm vorüber bis an die Breite Mauer, 39dann am Tor Ephraim und dem
Tor der Altstadt, dem Fischtor, dem Turm Hananeel und dem Turm Mea
vorüber bis an das Schaftor; am Gefängnistor\textless sup title=``oder:
Wachttor''\textgreater✲ machten sie Halt. 40Dann nahmen beide Festchöre
beim Hause Gottes Aufstellung, auch ich und die eine Hälfte der
Obersten\textless sup title=``oder: Fürsten''\textgreater✲ mit mir.
41Hierauf bliesen die Priester Eljakim, Maaseja, Minjamin, Michaja,
Eljoenai, Sacharja und Hananja mit Trompeten, 42und Maaseja, Semaja,
Eleasar, Ussi, Johanan, Malkija, Elam und Eser, die Sänger, trugen unter
Leitung Jisrahjas Lieder vor. 43Alsdann brachte man an diesem Tage große
Schlachtopfer dar und gab sich der Freude hin, denn Gott der HERR hatte
ihnen eine große Freude bereitet; und auch die Frauen und Kinder
überließen sich der Freude, so daß man den Jubel Jerusalems bis in weite
Ferne hörte.

\hypertarget{zwei-neue-mauxdfnahmen-nehemias}{%
\subsubsection{4. Zwei neue Maßnahmen
Nehemias}\label{zwei-neue-mauxdfnahmen-nehemias}}

\hypertarget{a-anstellung-von-beamten-zur-aufsicht-uxfcber-das-einkommen-der-priester-und-leviten}{%
\paragraph{a) Anstellung von Beamten zur Aufsicht über das Einkommen der
Priester und
Leviten}\label{a-anstellung-von-beamten-zur-aufsicht-uxfcber-das-einkommen-der-priester-und-leviten}}

44An jenem Tage bestellte man auch Männer zu Aufsehern über die Zellen,
die zu Vorratskammern für die Abgaben\textless sup title=``oder:
Hebeopfer''\textgreater✲, für die Erstlinge und die Zehnten, bestimmt
waren, um in ihnen nach den Feldmarken der einzelnen Ortschaften die
gesetzlichen Abgaben für die Priester und die Leviten einzusammeln; denn
die Judäer hatten ihre Freude an den Priestern und Leviten, die für den
heiligen Dienst bestellt waren; 45und diese besorgten in der Tat den
Dienst ihres Gottes und die Beobachtung der Reinigungsvorschriften
gewissenhaft, ebenso auch die Sänger und Torhüter, nach der Anordnung
Davids und seines Sohnes Salomo. 46Denn schon vor alters, zur Zeit
Davids und Asaphs, hatte es Vorsteher der Sänger sowie Lobgesänge und
Danklieder für Gott gegeben. 47Ganz Israel aber entrichtete zur Zeit
Serubbabels und zur Zeit Nehemias die Abgaben für die Sänger und die
Torhüter, die diesen tagtäglich gebührten; sie lieferten aber die
(vorgeschriebenen) heiligen Gaben an die Leviten ab, und die Leviten
ließen die Weihegaben den Nachkommen Aarons zukommen.

\hypertarget{b-ausscheidung-der-heidnischen-besonders-ammonitischen-und-moabitischen-bestandteile-aus-der-gemeinde}{%
\paragraph{b) Ausscheidung der heidnischen (besonders ammonitischen und
moabitischen) Bestandteile aus der
Gemeinde}\label{b-ausscheidung-der-heidnischen-besonders-ammonitischen-und-moabitischen-bestandteile-aus-der-gemeinde}}

\hypertarget{section-12}{%
\section{13}\label{section-12}}

1Als an jenem Tage dem Volke aus dem Buch des mosaischen Gesetzes laut
vorgelesen wurde, fand sich darin geschrieben\textless sup
title=``5.Mose 23,4-6''\textgreater✲, daß kein Ammoniter und kein
Moabiter jemals Aufnahme in die Gemeinde Gottes finden dürfe, 2weil sie
den Israeliten nicht mit Brot und Wasser entgegengekommen waren und (ihr
König) den Bileam gegen sie in Sold genommen hatte, damit er sie
verfluche; allerdings hatte unser Gott den Fluch in Segen verwandelt.
3Als sie nun das Gesetz vernommen hatten, sonderten sie alles Mischvolk✲
aus Israel aus.

\hypertarget{nehemias-tuxe4tigkeit-wuxe4hrend-seiner-zweiten-anwesenheit-in-jerusalem}{%
\subsubsection{5. Nehemias Tätigkeit während seiner zweiten Anwesenheit
in
Jerusalem}\label{nehemias-tuxe4tigkeit-wuxe4hrend-seiner-zweiten-anwesenheit-in-jerusalem}}

\hypertarget{a-beseitigung-der-tobijazelle-am-tempel}{%
\paragraph{a) Beseitigung der Tobijazelle am
Tempel}\label{a-beseitigung-der-tobijazelle-am-tempel}}

4Vordem aber hatte der Priester Eljasib, dem die Aufsicht über die
Zellen des Hauses unsers Gottes übertragen war, ein Verwandter Tobijas,
5diesem eine große Zelle eingeräumt, in der man früher die
Speisopfer\textless sup title=``oder: das Opfermehl''\textgreater✲, den
Weihrauch, die Geräte und den Zehnten vom Getreide, Wein und Öl
untergebracht hatte, die Anteile, die den Leviten, den Sängern und den
Torhütern zukamen, sowie die Abgaben an die Priester. 6Während aber dies
alles vor sich ging, war ich nicht in Jerusalem anwesend gewesen,
sondern hatte mich im zweiunddreißigsten Regierungsjahre Arthasasthas,
des Königs von Babylon, an den königlichen Hof begeben. Als ich mir dann
nach einiger Zeit wieder Urlaub vom Könige erbeten hatte 7und wieder
nach Jerusalem gekommen war, entdeckte ich den Unfug, den Eljasib dem
Tobija zuliebe verübt hatte, indem er ihm eine Zelle in den Vorhöfen des
Hauses Gottes eingeräumt hatte. 8Dies erregte solchen Unwillen in mir,
daß ich allen Hausrat Tobijas aus der Zelle hinauswerfen ließ 9und den
Befehl gab, man solle die Zelle reinigen; darauf ließ ich dort wieder
die Geräte des Hauses Gottes, das Speisopfer\textless sup title=``oder:
Opfermehl''\textgreater✲ und den Weihrauch unterbringen.

\hypertarget{b-sicherstellung-der-richtigen-ablieferung-der-abgaben-an-die-leviten}{%
\paragraph{b) Sicherstellung der richtigen Ablieferung der Abgaben an
die
Leviten}\label{b-sicherstellung-der-richtigen-ablieferung-der-abgaben-an-die-leviten}}

10Als ich dann erfuhr, daß man den Leviten die ihnen zukommenden Anteile
nicht geliefert hatte und daß infolgedessen die Leviten und die Sänger,
die den heiligen Dienst zu verrichten hatten, sich alle auf ihre
ländlichen Besitzungen\textless sup title=``vgl. 12,28-29''\textgreater✲
entfernt hatten, 11da stellte ich die Vorsteher zur Rede und fragte sie,
warum das Haus Gottes so verwahrlost worden sei. Darauf ließ ich die
betreffenden Leute wieder zusammenholen und stellte sie wieder auf ihre
Posten. 12Als dann ganz Juda die Zehnten vom Getreide, Wein und Öl in
die Vorratskammern gebracht hatte, 13übertrug ich die Aufsicht über die
Vorräte dem Priester Selemja und dem Schriftgelehrten Zadok und von den
Leviten dem Pedaja und bestellte zu ihrer Unterstützung Hanan, den Sohn
Sakkurs, des Sohnes Matthanjas; denn sie galten als zuverlässige Männer,
und ihnen oblag es nunmehr, die Austeilung an ihre Genossen vorzunehmen.
14Gedenke mir dies, mein Gott, und laß die Wohltaten, die ich dem Hause
meines Gottes und seiner Dienerschaft erwiesen habe, nicht in
Vergessenheit geraten!

\hypertarget{c-beseitigung-der-entweihung-des-sabbats-durch-geschuxe4fts--und-handelsleute}{%
\paragraph{c) Beseitigung der Entweihung des Sabbats durch Geschäfts-
und
Handelsleute}\label{c-beseitigung-der-entweihung-des-sabbats-durch-geschuxe4fts--und-handelsleute}}

15Zu derselben Zeit sah ich in Juda Leute, die am Sabbat die Kelter
traten und Getreide vom Felde einbrachten und Esel damit beluden, auch
Wein, Trauben, Feigen und andere Ladungen aller Art aufpackten und sie
am Sabbattage nach Jerusalem hereinbrachten. Ich verwarnte sie also an
dem Tage, an welchem sie die Lebensmittel feilboten. 16Auch die Tyrier,
die im Lande wohnten, brachten Fische und allerlei andere Waren herein
und verkauften sie am Sabbat an die Juden in Jerusalem. 17Da stellte ich
die vornehmen Juden zur Rede und hielt ihnen vor: »Was ist das für eine
böse Sache, die ihr da tut, daß ihr den Sabbattag entheiligt! 18Haben
nicht eure Väter ebenso getan, und hat nicht unser Gott eben deswegen
all dieses Unglück über uns und diese Stadt ergehen lassen? Wollt ihr
denn noch größeren Zorn über Israel bringen, indem ihr den Sabbattag
entweiht?« 19Sobald es nun am Vorabend des Sabbats (auf den
Marktplätzen) an den Toren Jerusalems dunkel wurde, ließ ich die Tore
schließen und gab Befehl, sie erst nach Ablauf des Sabbats wieder zu
öffnen; auch stellte ich einige von meinen Leuten an den Toren auf,
damit keine Last\textless sup title=``oder: Ware''\textgreater✲ am
Sabbattag hereinkäme. 20Nun mußten die Händler und Verkäufer von Waren
aller Art die Nacht ein- oder zweimal draußen vor der Stadt zubringen.
21Darauf verwarnte ich sie mit den Worten: »Wozu haltet ihr euch während
der Nacht vor der Mauer auf? Wenn ihr das noch einmal tut, werde ich
euch festnehmen lassen!« Von dieser Zeit an kamen sie am Sabbat nicht
wieder. 22Darauf befahl ich den Leviten, sie sollten sich reinigen und
die Bewachung der Tore übernehmen, damit der Sabbattag heilig gehalten
würde. Auch das gedenke mir, mein Gott, und übe Gnade an mir nach deiner
großen Güte!

\hypertarget{d-mauxdfregeln-gegen-die-mischehen-verstouxdfung-eines-hohepriestersohnes}{%
\paragraph{d) Maßregeln gegen die Mischehen; Verstoßung eines
Hohepriestersohnes}\label{d-mauxdfregeln-gegen-die-mischehen-verstouxdfung-eines-hohepriestersohnes}}

23Ebenfalls in jenen Tagen sah ich mich nach den Juden um, welche
asdoditische, ammonitische und moabitische Frauen geheiratet hatten.
24Deren Kinder redeten nämlich zur Hälfte asdoditisch und konnten nicht
mehr jüdisch sprechen, sondern verstanden nur die Sprache des
betreffenden Volkes. 25Da machte ich den Männern schwere Vorwürfe,
fluchte ihnen, schlug einige von ihnen, zauste sie am Bart und beschwor
sie bei Gott: »Ihr dürft eure Töchter durchaus nicht den Söhnen jener zu
Frauen geben und dürft von den Töchtern jener keine für eure Söhne oder
für euch selbst zu Frauen nehmen! 26Hat sich nicht Salomo, der König von
Israel, gerade um solcher Frauen willen zur Sünde verleiten lassen? Zwar
hat es unter den vielen Völkern keinen König wie ihn gegeben, und er war
ein Liebling seines Gottes, so daß Gott ihn zum König über ganz Israel
machte; aber sogar ihn haben die ausländischen Frauen zur Sünde
verleitet. 27Und nun müssen wir von euch hören, daß ihr all dieses große
Unrecht begeht, durch Verheiratung mit ausländischen Frauen treulos
gegen unsern Gott zu handeln!« 28Und einer von den Söhnen des
Hohepriesters Jojada, des Sohnes Eljasibs, hatte sich mit einer Tochter
des Horoniters Sanballat verheiratet; den entfernte ich aus meiner
Umgebung. 29Gedenke ihnen, mein Gott, ihre vielfache Entweihung des
Priestertums und des Bundes, den du mit der Priesterschaft und mit den
Leviten geschlossen hast!

\hypertarget{e-schluuxdf-der-denkschrift}{%
\paragraph{e) Schluß der
Denkschrift}\label{e-schluuxdf-der-denkschrift}}

30So habe ich sie von allem ausländischen Wesen gereinigt und die
Dienstleistungen der Priester und der Leviten fest geordnet, für jeden
einzelnen bezüglich seiner Amtspflichten, 31auch für die Lieferung von
Brennholz zu den festgesetzten Zeiten und für die Erstlingsgaben (habe
ich Sorge getragen). Gedenke mir das, mein Gott, (und rechne es mir an)
zum Guten✲!
