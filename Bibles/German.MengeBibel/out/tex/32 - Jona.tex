\hypertarget{der-prophet-jona}{%
\section{DER PROPHET JONA}\label{der-prophet-jona}}

\hypertarget{jonas-berufung-ungehorsam-und-bestrafung}{%
\subsubsection{1. Jonas Berufung, Ungehorsam und
Bestrafung}\label{jonas-berufung-ungehorsam-und-bestrafung}}

\hypertarget{section}{%
\section{1}\label{section}}

\bibleverse{1}Einst erging das Wort des HERRN an Jona, den Sohn
Amitthais\textless sup title=``2.Kön 14,25''\textgreater✲,
folgendermaßen: \bibleverse{2}»Mache dich auf, begib dich nach der
großen Stadt Ninive\textless sup title=``1.Mose 10,11-12''\textgreater✲
und kündige ihr an, daß ihr böses Tun vor mich gekommen ist!«
\bibleverse{3}Aber Jona machte sich auf den Weg, um aus dem Angesicht
des HERRN hinweg nach Tharsis zu fliehen; und als er nach Joppe
hinabgegangen war und dort ein Schiff gefunden hatte, das nach Tharsis
fahren wollte, bezahlte er das Fahrgeld und stieg ein, um mit
ihnen\textless sup title=``d.h. den Schiffern''\textgreater✲ nach
Tharsis zu fahren und so dem HERRN aus den Augen zu kommen.
\bibleverse{4}Da ließ der HERR einen starken Wind auf das Meer
hinabfahren, so daß sich ein gewaltiges Unwetter auf dem Meer erhob und
das Schiff zu scheitern drohte. \bibleverse{5}Da gerieten die Leute auf
dem Schiff in Angst und schrien ein jeder zu seinem Gott um Hilfe und
warfen die Gerätschaften, die sich im Schiff befanden, ins Meer, um (das
Schiff) dadurch zu erleichtern. Jona aber war in den
Hinterraum\textless sup title=``oder: untersten Raum''\textgreater✲ des
Schiffes hinabgestiegen, hatte sich dort niedergelegt und war fest
eingeschlafen. \bibleverse{6}Da trat der Schiffshauptmann\textless sup
title=``oder: Kapitän''\textgreater✲ zu ihm mit den Worten: »Wie kannst
du nur schlafen?! Stehe auf, rufe deinen Gott an! Vielleicht nimmt sich
dieser Gott unser an, daß wir nicht untergehen.« \bibleverse{7}Dann
sagten jene zueinander: »Kommt, wir wollen Lose werfen, um zu erfahren,
durch wessen Schuld dieses Unglück uns trifft!« Als sie nun die Lose
warfen, fiel das Los auf Jona.

\bibleverse{8}Da sagten sie zu ihm: »Teile uns doch mit, du, um
dessenwillen dies Unglück uns widerfährt: Welches ist dein Gewerbe, und
woher kommst du? Wo bist du zu Haus, und was für ein Landsmann bist du?«
\bibleverse{9}Da antwortete er ihnen: »Ich bin ein Hebräer und verehre
den HERRN, den Gott des Himmels, der das Meer und das feste Land
geschaffen hat.« \bibleverse{10}Da gerieten die Männer in große Furcht
und sagten zu ihm: »Was hast du nur getan?« Die Männer wußten nämlich
bereits, daß er sich auf der Flucht vor dem HERRN befand; denn er hatte
es ihnen mitgeteilt. \bibleverse{11}Dann fragten sie ihn: »Was sollen
wir mit dir machen, damit das Meer sich beruhigt und uns nicht länger
bedroht?« Denn das Meer wurde immer noch stürmischer. \bibleverse{12}Da
erwiderte er ihnen: »Nehmt mich und werft mich ins Meer, damit das Meer
sich beruhigt und euch nicht länger bedroht! Denn ich erkenne, daß
dieser gewaltige Sturm durch meine Schuld über euch gekommen ist.«
\bibleverse{13}Nun strengten sich die Männer zwar an, das Schiff (durch
Rudern) ans Land zu bringen, vermochten es aber nicht, weil das Meer
immer ärger gegen sie tobte. \bibleverse{14}Da riefen sie den HERRN an
mit den Worten: »Ach, HERR! Laß uns doch nicht untergehen, wenn wir
diesen Mann ums Leben bringen, und rechne uns nicht unschuldig
vergossenes Blut an! Denn du bist der HERR: du hast getan, wie es dir
wohlgefallen hat.« \bibleverse{15}Darauf ergriffen sie Jona und warfen
ihn ins Meer, und sogleich legte sich das Toben des Meeres.
\bibleverse{16}Da gerieten die Männer in große Furcht vor dem HERRN; sie
brachten dem HERRN ein Schlachtopfer dar und taten Gelübde.

\hypertarget{section-1}{%
\section{2}\label{section-1}}

\bibleverse{1}Der HERR aber ließ einen großen Fisch kommen, der Jona
verschlingen sollte; und Jona befand sich im Bauche des Fisches drei
Tage und drei Nächte lang.

\hypertarget{jonas-gebet-und-errettung}{%
\subsubsection{2. Jonas Gebet und
Errettung}\label{jonas-gebet-und-errettung}}

\bibleverse{2}Da richtete Jona aus dem Leibe des Fisches folgendes Gebet
an den HERRN, seinen Gott: \bibleverse{3}»Gerufen habe ich aus meiner
Bedrängnis zum HERRN, da hat er mich erhört; aus dem Schoß der Unterwelt
habe ich um Hilfe geschrien, da hast du mein Rufen vernommen.
\bibleverse{4}Denn du hattest mich in die Tiefe geschleudert, mitten ins
Meer hinein, so daß die Fluten mich umschlossen; alle deine Wogen und
Wellen fuhren über mich dahin. \bibleverse{5}Schon dachte ich:
›Verstoßen bin ich, hinweg von deinem Angesicht: wie könnte ich je
wieder nach deinem heiligen Tempel schauen?‹ \bibleverse{6}Die Wasser
umgaben mich und gingen mir bis an die Seele\textless sup title=``oder:
ans Leben''\textgreater✲; die Tiefe\textless sup title=``oder:
Flut''\textgreater✲ umfing mich, Seegras hatte sich mir ums Haupt
geschlungen; \bibleverse{7}zu den Wurzeln der Berge\textless sup
title=``=~den tiefsten Gründen der Erde''\textgreater✲ war ich
hinabgefahren; die Riegel der Erde hatten sich auf ewig hinter mir
geschlossen: -- da hast du mein Leben aus der Grube heraufgeholt, HERR,
mein Gott! \bibleverse{8}Als meine Seele in mir verzagte\textless sup
title=``oder: mein Leben in mir zu Ende ging''\textgreater✲, da gedachte
ich des HERRN, und zu dir drang mein Gebet, zu deinem heiligen Tempel.
\bibleverse{9}Die sich an nichtige Götzen halten, verlassen den, bei
welchem das Heil für sie liegt. \bibleverse{10}Ich aber will dir laute
Danksagung als Opfer darbringen, will, was ich gelobt habe, bezahlen✲:
die Rettung kommt vom HERRN!«

\bibleverse{11}Hierauf gebot der HERR dem Fisch, und dieser spie Jona
ans Land aus.

\hypertarget{jonas-erfolgreiche-buuxdfpredigt-in-ninive}{%
\subsubsection{3. Jonas erfolgreiche Bußpredigt in
Ninive}\label{jonas-erfolgreiche-buuxdfpredigt-in-ninive}}

\hypertarget{section-2}{%
\section{3}\label{section-2}}

\bibleverse{1}Nun erging das Wort des HERRN an Jona zum zweitenmal
folgendermaßen: \bibleverse{2}»Mache dich auf, begib dich nach der
großen Stadt Ninive und laß sie die Botschaft\textless sup title=``oder:
Predigt''\textgreater✲ hören, die ich dir ansagen werde!«

\bibleverse{3}Da machte Jona sich auf den Weg und begab sich nach
Ninive, wie der HERR ihm geboten hatte. Ninive war aber eine gewaltig
große Stadt, deren Durchwanderung drei Tagereisen erforderte.
\bibleverse{4}So begann denn Jona eine Tagereise weit in die Stadt
hineinzugehen und predigte dabei mit den Worten: »Noch vierzig Tage,
dann ist Ninive zerstört!« \bibleverse{5}Da glaubten die Einwohner von
Ninive an Gott, riefen ein Fasten aus und legten Sackleinen✲ an, klein
und groß; \bibleverse{6}und als die Kunde davon zum König von Ninive
gelangte, erhob er sich von seinem Thron, legte seinen Mantel ab, hüllte
sich in ein Trauergewand und setzte sich in die Asche.
\bibleverse{7}Sodann ließ er in Ninive durch Ausruf bekanntmachen: »Auf
Befehl des Königs und seiner Großen (wird folgende Verordnung erlassen):
Menschen und Vieh, Rinder und Kleinvieh sollen durchaus nichts genießen;
dürfen nicht auf die Weide gehen und kein Wasser trinken,
\bibleverse{8}sondern sollen, sowohl Menschen als Vieh, in Sackleinen✲
gekleidet sein und mit aller Macht zu Gott rufen und umkehren ein jeder
von seinem bösen Wege✲ und von dem Unrecht ablassen, das an seinen
Händen klebt! \bibleverse{9}Vielleicht tut es Gott dann doch noch leid,
und er läßt von seinem lodernden Zorn ab, so daß wir nicht untergehen!«

\bibleverse{10}Als nun Gott sah, was sie taten, daß sie nämlich von
ihrem bösen Wege✲ umkehrten, tat ihm das Unheil leid, das er ihnen
angedroht hatte, und er ließ es nicht eintreten.

\hypertarget{jonas-miuxdfmut-und-zurechtweisung}{%
\subsubsection{4. Jonas Mißmut und
Zurechtweisung}\label{jonas-miuxdfmut-und-zurechtweisung}}

\hypertarget{section-3}{%
\section{4}\label{section-3}}

\bibleverse{1}Das verursachte aber dem Jona großen Verdruß, und er
geriet in Zorn, \bibleverse{2}so daß er folgendes Gebet an den HERRN
richtete: »Ach, HERR, das ist es ja, was ich gedacht habe, als ich noch
daheim war, und eben darum habe ich das vorige Mal die Flucht nach
Tharsis ergriffen; denn ich wußte wohl, daß du ein gnädiger und
barmherziger Gott bist, langsam zum Zorn und reich an Güte und geneigt,
dich das Unheil gereuen zu lassen. \bibleverse{3}Und nun, HERR, nimm
doch mein Leben von mir! Denn es ist besser für mich\textless sup
title=``oder: ist mir lieber''\textgreater✲, zu sterben als noch am
Leben zu bleiben.« \bibleverse{4}Aber der HERR erwiderte: »Ist es recht
von dir, so zu zürnen?«

\bibleverse{5}Hierauf ging Jona aus der Stadt hinaus und ließ sich
östlich von der Stadt nieder; er baute sich dort eine Hütte und setzte
sich unter ihr in den Schatten, um abzuwarten, wie es der Stadt ergehen
würde. \bibleverse{6}Da ließ Gott der HERR eine
Rizinusstaude\textless sup title=``oder: einen Wunderbaum''\textgreater✲
aufschießen und über Jona emporwachsen, damit er seinem Haupte Schatten
biete und ihn von seinem Unmut befreie; und Jona hatte große Freude an
dem Rizinus. \bibleverse{7}Am andern Tage aber, als die Morgenröte
aufging, ließ Gott einen Wurm entstehen\textless sup title=``oder:
kommen''\textgreater✲, der fraß die Rizinusstaude an, so daß sie
verdorrte; \bibleverse{8}und als die Sonne aufging, ließ Gott einen
schwülen Ostwind kommen; und die Sonne stach Jona auf das Haupt, so daß
er ganz ohnmächtig wurde und sich den Tod wünschte mit den Worten: »Es
ist besser für mich\textless sup title=``oder: ist mir
lieber''\textgreater✲, zu sterben als noch am Leben zu bleiben!«

\bibleverse{9}Da sagte Gott zu Jona: »Ist es wohl recht von dir, wegen
der Rizinusstaude so zornig zu sein?« Er antwortete: »Ja, mit Recht bin
ich erzürnt bis zum Sterben!« \bibleverse{10}Der HERR aber entgegnete:
»Dir tut der Rizinus leid, um den du dich doch nicht gemüht und den du
nicht großgezogen hast, der in einer Nacht entstanden und in einer Nacht
vergangen ist. \bibleverse{11}Und mir sollte die große Stadt Ninive
nicht leid tun, in der mehr als hundertzwanzigtausend Menschen leben,
die zwischen rechts und links noch nicht zu unterscheiden wissen, dazu
auch eine Menge Tiere?«
