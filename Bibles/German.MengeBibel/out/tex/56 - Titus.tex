\hypertarget{der-brief-des-apostels-paulus-an-titus}{%
\section{DER BRIEF DES APOSTELS PAULUS AN
TITUS}\label{der-brief-des-apostels-paulus-an-titus}}

\hypertarget{zuschrift-und-segenswunsch}{%
\subsubsection{Zuschrift und
Segenswunsch}\label{zuschrift-und-segenswunsch}}

\hypertarget{section}{%
\section{1}\label{section}}

\bibleverse{1} Ich, Paulus, ein Knecht Gottes und ein Apostel Jesu
Christi, (bestellt) für den Glauben der Auserwählten Gottes und für die
Erkenntnis der Wahrheit, die sich in einem gottseligen Wandel bewährt,
\bibleverse{2} (bestellt) aufgrund der Hoffnung des ewigen Lebens, das
der untrügliche Gott (schon) vor ewigen Zeiten verheißen hat~--
\bibleverse{3} kundgetan aber hat er sein Wort zur festgesetzten Zeit
durch die Predigt, mit der ich im Auftrage Gottes, unsers
Retters\textless sup title=``oder: Heilands''\textgreater✲ betraut
worden bin --: \bibleverse{4} ich sende meinen Gruß dem Titus, meinem
hinsichtlich des gemeinsamen Glaubens echten Kinde:

Gnade (sei mit dir) und Friede von Gott dem Vater und unserm
Retter\textless sup title=``oder: Heiland''\textgreater✲ Christus Jesus!

\hypertarget{verordnung-bezuxfcglich-des-ausbaues-der-gemeindeverfassung-auf-kreta-und-angabe-von-mauxdfregeln-gegen-das-umsichgreifen-der-irrlehren}{%
\subsubsection{1. Verordnung bezüglich des Ausbaues der
Gemeindeverfassung auf Kreta und Angabe von Maßregeln gegen das
Umsichgreifen der
Irrlehren}\label{verordnung-bezuxfcglich-des-ausbaues-der-gemeindeverfassung-auf-kreta-und-angabe-von-mauxdfregeln-gegen-das-umsichgreifen-der-irrlehren}}

\hypertarget{a-vorschriften-uxfcber-die-einsetzung-von-uxe4ltesten-als-gemeindeleitern}{%
\paragraph{a) Vorschriften über die Einsetzung von Ältesten als
Gemeindeleitern}\label{a-vorschriften-uxfcber-die-einsetzung-von-uxe4ltesten-als-gemeindeleitern}}

\bibleverse{5} Ich habe dich zu dem Zweck in Kreta zurückgelassen, daß
du das (von mir dort) noch nicht Erledigte in Ordnung bringen und in den
einzelnen Städten Älteste einsetzen möchtest, wie ich es dir aufgetragen
habe, \bibleverse{6} nämlich solche (Männer), die unbescholten und nur
eines Weibes Mann\textless sup title=``vgl. 1.Tim 3,2''\textgreater✲
sind und gläubige Kinder haben, denen man nicht zuchtlosen Lebenswandel
oder Unbotmäßigkeit nachsagen kann; \bibleverse{7} denn ein
Gemeindevorsteher\textless sup title=``vgl. 1.Tim 3,1-2''\textgreater✲
muß als Gottes Haushalter unbescholten sein, nicht eigenwillig, nicht
zornmütig, kein Trinker, kein Händelsucher, nicht schändlichem Gewinn
nachgehend; \bibleverse{8} vielmehr muß er gastfrei sein, allem Guten
zugetan, besonnen, gerecht\textless sup title=``oder:
rechtschaffen''\textgreater✲, gottesfürchtig, enthaltsam; \bibleverse{9}
er muß an dem zuverlässigen Wort (Gottes) festhalten, wie er es im
Unterricht empfangen hat, damit er imstande ist, aufgrund der gesunden
Lehre ebensowohl zu ermahnen als auch die Widersprechenden zu
widerlegen\textless sup title=``oder: zu überführen''\textgreater✲.

\hypertarget{b-vorschriften-uxfcber-das-verhalten-gegen-die-buxf6swilligen-verfuxfchrer-und-scheinheiligen-irrlehrer}{%
\paragraph{b) Vorschriften über das Verhalten gegen die böswilligen
Verführer und scheinheiligen
Irrlehrer}\label{b-vorschriften-uxfcber-das-verhalten-gegen-die-buxf6swilligen-verfuxfchrer-und-scheinheiligen-irrlehrer}}

\bibleverse{10} Denn es gibt viele, die sich nicht unterordnen wollen,
Schwätzer und Schwindler, besonders unter den Judenchristen;
\bibleverse{11} ihnen muß man den Mund stopfen, weil sie ganze
Häuser\textless sup title=``oder: Familien''\textgreater✲ zerrütten,
indem sie um schändlichen Gewinnes willen ungehörige Lehren vortragen.
\bibleverse{12} Hat doch ein Prophet aus ihrer eigenen Mitte gesagt:
»Die Kreter sind immer verlogen, bösartige Tiere\textless sup
title=``=~streit- und rauflustig''\textgreater✲, faule Bäuche.«
\bibleverse{13} Dies Zeugnis entspricht der Wahrheit, und aus diesem
Grunde weise sie rücksichtslos zurecht, damit sie im Glauben gesund
bleiben \bibleverse{14} und ihre Gedanken nicht auf jüdische Fabeln und
auf Gebote von Menschen richten, die der Wahrheit den Rücken kehren.
\bibleverse{15} »Für die Reinen ist alles rein«, für die Befleckten und
Ungläubigen dagegen ist nichts rein, sondern bei ihnen ist beides, ihr
Verstand\textless sup title=``oder: ihre Gesinnung''\textgreater✲ und
ihr Gewissen, befleckt. \bibleverse{16} Sie behaupten zwar, Gott zu
kennen, verleugnen ihn aber durch ihr ganzes Tun: verabscheuenswerte und
ungehorsame, zu jedem guten Werk untüchtige Menschen.

\hypertarget{ermahnungen-fuxfcr-die-gemeinde}{%
\subsubsection{2. Ermahnungen für die
Gemeinde}\label{ermahnungen-fuxfcr-die-gemeinde}}

\hypertarget{a-vorschriften-fuxfcr-die-einzelnen-stuxe4nde-in-der-gemeinde}{%
\paragraph{a) Vorschriften für die einzelnen Stände in der
Gemeinde}\label{a-vorschriften-fuxfcr-die-einzelnen-stuxe4nde-in-der-gemeinde}}

\hypertarget{section-1}{%
\section{2}\label{section-1}}

\bibleverse{1} Du aber sprich offen aus, was der gesunden Lehre
entspricht, \bibleverse{2} nämlich daß die älteren Männer nüchtern sein
sollen, ehrbar, besonnen, gesund im Glauben, in der Liebe und in der
Standhaftigkeit\textless sup title=``oder: Geduld''\textgreater✲.~--
\bibleverse{3} Ebenso ermahne die älteren Frauen, sich in ihrer Haltung
würdevoll\textless sup title=``eig. wie Frauen von priesterlichem
Stand''\textgreater✲ zu benehmen, nicht klatschsüchtig zu sein, nicht
dem übermäßigen Weingenuß ergeben, Lehrerinnen des Guten, \bibleverse{4}
damit sie die jungen Frauen zu besonnener Pflichterfüllung anleiten,
nämlich ihre Männer und ihre Kinder zu lieben, \bibleverse{5} züchtig,
keusch, tüchtige Haushälterinnen, gütig zu sein und sich ihren
Ehemännern unterzuordnen, damit das Wort Gottes nicht gelästert
werde.~-- \bibleverse{6} Die jungen Männer ermahne gleichfalls, besonnen
zu sein\textless sup title=``=~sich in Zucht zu halten''\textgreater✲,
\bibleverse{7} und biete dich selbst ihnen in jeder Beziehung als
Vorbild guter Werke\textless sup title=``oder: für ein gutes
Verhalten''\textgreater✲ dar. In der Lehre beweise Unverfälschtheit,
würdevollen Ernst, \bibleverse{8} gesunde, unanfechtbare Rede, damit
jeder Gegner sich beschämt fühlt, weil er uns nichts Schlechtes
nachsagen kann.~-- \bibleverse{9} Die Knechte\textless sup title=``oder:
Sklaven''\textgreater✲ ermahne, ihren Herren in jeder Hinsicht gehorsam
zu sein und ihnen zu Gefallen zu leben, nicht zu widersprechen,
\bibleverse{10} nichts zu veruntreuen, vielmehr volle, echte Treue zu
beweisen, damit sie der Lehre Gottes, unsers Retters\textless sup
title=``oder: Heilands''\textgreater✲, in allen Beziehungen Ehre machen.

\hypertarget{b-begruxfcndung-dieser-vorschriften-durch-den-hinweis-auf-die-in-der-welt-erschienene-gnade-gottes}{%
\paragraph{b) Begründung dieser Vorschriften durch den Hinweis auf die
in der Welt erschienene Gnade
Gottes}\label{b-begruxfcndung-dieser-vorschriften-durch-den-hinweis-auf-die-in-der-welt-erschienene-gnade-gottes}}

\bibleverse{11} Denn erschienen\textless sup title=``=~offenbar
geworden''\textgreater✲ ist die Gnade Gottes, die allen Menschen das
Heil bringt, \bibleverse{12} indem sie uns dazu erzieht, dem gottlosen
Wesen und den weltlichen Begierden abzusagen und besonnen, gerecht und
gottselig (schon) in der gegenwärtigen Weltzeit zu leben,
\bibleverse{13} indem wir dabei auf unser seliges Hoffnungsgut und auf
das Erscheinen der Herrlichkeit des großen Gottes und unsers
Retters\textless sup title=``oder: Heilands''\textgreater✲ Christus
Jesus warten, \bibleverse{14} der sich selbst für uns dahingegeben hat,
um uns von aller Gesetzlosigkeit zu erlösen und sich ein reines Volk zum
Eigentum zu schaffen, das eifrig auf gute Werke bedacht ist\textless sup
title=``2.Mose 19,5; 5.Mose 14,2''\textgreater✲. \bibleverse{15} Dies
trage ihnen vor, dazu ermahne sie und rede ihnen mit allem Nachdruck ins
Gewissen; laß dich (dabei) von niemand geringschätzig behandeln!

\hypertarget{c-vom-verhalten-gegen-die-heidnische-obrigkeit-und-die-nichtchristen-und-vom-wandel-der-christen-als-erneuerter-menschen}{%
\paragraph{c) Vom Verhalten gegen die heidnische Obrigkeit und die
Nichtchristen und vom Wandel der Christen als erneuerter
Menschen}\label{c-vom-verhalten-gegen-die-heidnische-obrigkeit-und-die-nichtchristen-und-vom-wandel-der-christen-als-erneuerter-menschen}}

\hypertarget{section-2}{%
\section{3}\label{section-2}}

\bibleverse{1} Schärfe ihnen ein, daß sie sich den obrigkeitlichen
Gewalten\textless sup title=``=~der Obrigkeit und den
Behörden''\textgreater✲ unterordnen, (ihren Befehlen) Gehorsam leisten
und zu jedem guten Werk bereit seien, \bibleverse{2} daß sie niemand
schmähen, sich friedfertig und nachgiebig benehmen und volle Sanftmut
gegen alle Menschen beweisen. \bibleverse{3} Denn einst sind auch wir
unverständig und ungehorsam gewesen und irre gegangen, indem wir
mannigfachen Begierden und Lüsten dienten und in Bosheit und Neid
dahinlebten, hassenswert und gegeneinander haßerfüllt. \bibleverse{4}
Als aber die Güte und Menschenfreundlichkeit Gottes, unsers
Retters\textless sup title=``oder: Heilands''\textgreater✲, erschienen
war, \bibleverse{5} da hat er uns -- nicht aufgrund von Werken der
Gerechtigkeit, die wir unserseits vollbracht hätten, sondern nach seiner
Barmherzigkeit -- gerettet durch das Bad der Wiedergeburt und der
Erneuerung des heiligen Geistes, \bibleverse{6} den er reichlich auf uns
ausgegossen hat durch unsern Retter\textless sup title=``oder:
Heiland''\textgreater✲ Jesus Christus, \bibleverse{7} damit wir durch
seine\textless sup title=``d.h. Christi''\textgreater✲ Gnade
gerechtgesprochen und unserer Hoffnung gemäß Erben des ewigen Lebens
würden.

\hypertarget{d-schluuxdfwort-uxfcber-das-verhalten-gegen-die-lehrverirrungen-und-deren-vertreter}{%
\paragraph{d) Schlußwort über das Verhalten gegen die Lehrverirrungen
und deren
Vertreter}\label{d-schluuxdfwort-uxfcber-das-verhalten-gegen-die-lehrverirrungen-und-deren-vertreter}}

\bibleverse{8} Zuverlässig ist das Wort, und ich will, daß du dich
hierüber mit aller Bestimmtheit aussprichst, damit die, welche zum
Glauben an Gott gekommen sind, allen Eifer darauf verwenden, sich in
guten Werken zu betätigen -- das ist etwas Schönes und für die Menschen
Segensreiches. \bibleverse{9} Dagegen mit törichten
Untersuchungen\textless sup title=``oder: Streitfragen''\textgreater✲
und Geschlechtsverzeichnissen sowie mit Streitigkeiten und Gezänk über
das Gesetz habe du nichts zu tun, denn das sind unnütze und unfruchtbare
Dinge. \bibleverse{10} Einen Menschen, der Spaltungen anrichtet, weise
nach einmaliger oder zweimaliger Verwarnung ab; \bibleverse{11} du weißt
ja, daß ein solcher Mensch auf verkehrte Wege geraten und nach seinem
eigenen Urteil\textless sup title=``vgl. Gal 2,17-18''\textgreater✲ ein
Sünder ist.

\hypertarget{persuxf6nliche-schluuxdfbemerkungen-letzte-auftruxe4ge-und-gruxfcuxdfe}{%
\subsubsection{3. Persönliche Schlußbemerkungen, letzte Aufträge und
Grüße}\label{persuxf6nliche-schluuxdfbemerkungen-letzte-auftruxe4ge-und-gruxfcuxdfe}}

\bibleverse{12} Sobald ich Artemas oder Tychikus zu dir sende, beeile
dich, zu mir nach Nikopolis zu kommen; denn ich habe mich entschlossen,
dort den Winter zuzubringen. \bibleverse{13} Zenas, den
Gesetzeslehrer\textless sup title=``oder: Juristen''\textgreater✲, und
Apollos rüste sorgfältig für ihre Winterreise aus, damit es ihnen an
nichts gebricht. \bibleverse{14} Auch unsere Leute müssen daraus eine
löbliche Liebestätigkeit zur Befriedigung unumgänglicher Bedürfnisse
lernen, damit sie nicht ohne Früchte (ihres Glaubens) bleiben.

\bibleverse{15} Alle, die hier bei mir sind, lassen dich grüßen. Grüße
die, welche im Glauben uns lieben! Die Gnade (Gottes) sei mit euch
allen!
