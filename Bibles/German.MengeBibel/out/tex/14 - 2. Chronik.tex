\hypertarget{zweites-buch-der-chronik}{%
\section{ZWEITES BUCH DER CHRONIK}\label{zweites-buch-der-chronik}}

\hypertarget{i.-das-kuxf6nigtum-salomos-kap.-1-9}{%
\subsection{I. Das Königtum Salomos (Kap.
1-9)}\label{i.-das-kuxf6nigtum-salomos-kap.-1-9}}

\hypertarget{salomos-regierungsantritt-sein-heer-und-sein-reichtum}{%
\subsubsection{1. Salomos Regierungsantritt; sein Heer und sein
Reichtum}\label{salomos-regierungsantritt-sein-heer-und-sein-reichtum}}

\hypertarget{a-salomos-opfer-in-gibeon}{%
\paragraph{a) Salomos Opfer in Gibeon}\label{a-salomos-opfer-in-gibeon}}

\hypertarget{section}{%
\section{1}\label{section}}

1Als nun Salomo, der Sohn Davids, sich in seiner Herrschaft befestigt
hatte -- der HERR, sein Gott, war nämlich mit ihm und ließ ihn überaus
mächtig werden --, 2da ließ Salomo Befehl an ganz Israel ergehen, an die
Befehlshaber der Tausendschaften und der Hundertschaften, an die Richter
und alle Fürsten von ganz Israel, die Familienhäupter; 3und dann begab
sich Salomo mit der ganzen Volksgemeinde nach der Höhe bei Gibeon; denn
dort befand sich das Offenbarungszelt Gottes, das Mose, der Knecht des
HERRN, in der Wüste hergestellt hatte~-- 4dagegen die Lade Gottes hatte
David aus Kirjath-Jearim an den Platz hinaufgebracht, den David für sie
hatte herrichten lassen; denn er hatte für sie in Jerusalem ein Zelt
aufschlagen lassen --. 5Auch der kupferne Altar, den Bezaleel, der Sohn
Uris, des Sohnes Hurs, hergestellt hatte\textless sup title=``2.Mose
38,1-7''\textgreater✲, stand dort (in Gibeon) vor der Wohnung des HERRN;
und Salomo und die Volksgemeinde suchten ihn\textless sup title=``d.h.
den HERRN''\textgreater✲ dort auf. 6Salomo opferte dann dort vor dem
HERRN auf dem kupfernen Altar, der zum Offenbarungszelt gehörte, und
brachte tausend Brandopfer auf ihm dar.

\hypertarget{b-die-gotteserscheinung-oder-das-traumgesicht-nach-dem-opfer}{%
\paragraph{b) Die Gotteserscheinung (oder das Traumgesicht) nach dem
Opfer}\label{b-die-gotteserscheinung-oder-das-traumgesicht-nach-dem-opfer}}

7In jener Nacht erschien Gott dem Salomo und sprach zu ihm: »Bitte, was
ich dir geben soll!« 8Da antwortete Salomo dem HERRN: »Du hast meinem
Vater David große Liebe erwiesen und hast mich zum König an seiner Statt
gemacht. 9So laß denn, HERR, mein Gott, deine Verheißung, die du meinem
Vater David gegeben hast, in Erfüllung gehen! denn du hast mich zum
König über ein Volk gemacht, das so zahlreich ist wie der Staub auf dem
Erdboden. 10So verleihe mir nun Weisheit und Einsicht, damit ich mich
diesem Volk gegenüber in rechter Weise zu verhalten weiß; denn wer
vermöchte sonst dieses dein großes Volk zu regieren?« 11Darauf sagte
Gott zu Salomo: »Weil du solche Gesinnung hegst und nicht um Reichtum,
Schätze und Ehre oder um den Tod deiner Feinde, auch nicht um langes
Leben gebeten, sondern dir Weisheit und Einsicht erbeten hast, um mein
Volk, zu dessen König ich dich gemacht habe, regieren zu können: 12so
soll dir die (erbetene) Weisheit und Einsicht verliehen sein; aber auch
Reichtum, Schätze und Ehre will ich dir schenken, wie keiner von den
Königen vor dir sie besessen hat und wie sie keiner nach dir jemals
besitzen wird.«

13Darauf kehrte Salomo von der Höhe bei Gibeon, von dem Platz vor dem
Offenbarungszelt, nach Jerusalem zurück und herrschte über Israel.

\hypertarget{c-salomos-reichtum-und-handel-mit-wagen-und-rossen}{%
\paragraph{c) Salomos Reichtum und Handel mit Wagen und
Rossen}\label{c-salomos-reichtum-und-handel-mit-wagen-und-rossen}}

14Salomo brachte zahlreiche Kriegswagen und Reitpferde\textless sup
title=``oder: Reiter''\textgreater✲ zusammen, so daß er 1400 Wagen und
12000 Reitpferde\textless sup title=``oder: Reiter''\textgreater✲ besaß,
die er in den Wagenstädten oder in seiner Nähe zu Jerusalem
unterbrachte. 15Und der König brachte es dahin, daß es in Jerusalem so
viel Silber und Gold gab wie Steine und daß die Zedernstämme den
Maulbeerfeigenbäumen in der Niederung an Menge gleichkamen. 16Der Bezug
der Pferde für Salomo erfolgte aus Ägypten, und zwar aus Koa; die
Händler des Königs kauften sie nämlich dort in Koa auf, 17so daß ein
Wagen bei der Ausfuhr aus Ägypten auf sechshundert Schekel Silber zu
stehen kam und ein Pferd auf einhundertundfünfzig. Ebenso wurden sie
durch ihre\textless sup title=``d.h. der Händler''\textgreater✲
Vermittlung an alle Könige der Hethiter und an die Könige von Syrien
ausgeführt\textless sup title=``vgl. 9,25-28''\textgreater✲.

\hypertarget{salomos-vertrag-mit-huram-von-tyrus-vorbereitungen-zum-tempelbau}{%
\subsubsection{2. Salomos Vertrag mit Huram von Tyrus; Vorbereitungen
zum
Tempelbau}\label{salomos-vertrag-mit-huram-von-tyrus-vorbereitungen-zum-tempelbau}}

18Darauf beschloß Salomo, dem Namen des HERRN einen Tempel und für sich
selbst einen Königspalast zu bauen.

\hypertarget{section-1}{%
\section{2}\label{section-1}}

1So ließ er denn 70000 Lastträger und 80000 Steinhauer im Gebirge (Juda)
und 3600 Aufseher über sie abzählen.

\hypertarget{a-salomos-botschaft-und-bitte-an-huram}{%
\paragraph{a) Salomos Botschaft und Bitte an
Huram}\label{a-salomos-botschaft-und-bitte-an-huram}}

2Dann schickte Salomo eine Gesandtschaft an den König Huram✲ von Tyrus
und ließ ihm sagen: »Wie du es bei meinem Vater David gehalten hast,
indem du ihm Zedern sandtest, damit er sich einen Palast zu seinem
Wohnsitz erbauen könnte, so halte es auch mit mir! 3Wisse wohl: ich bin
im Begriff, dem Namen des HERRN, meines Gottes, ein Haus zu bauen und es
ihm zu weihen, damit man wohlriechendes Räucherwerk vor ihm verbrennen
und Schaubrote regelmäßig zurichten und Brandopfer an jedem Morgen und
Abend, an den Sabbaten, sowie an den Neumonden und den (anderen) Festen
des HERRN, unseres Gottes, darbringen kann, -- das ist ja stehender
Brauch in Israel. 4Das Haus aber, das ich bauen will, muß groß sein;
denn unser Gott ist größer als alle anderen Götter. 5Freilich -- wer
vermöchte ihm ein Haus zu bauen, da doch der Himmel und aller Himmel
Himmel ihn nicht fassen können? Und wer bin ich, daß ich ihm ein Haus
bauen sollte? Es geschieht ja nur, um vor ihm (Opfer als) Räucherwerk zu
verbrennen. 6So sende mir denn nun einen Mann, der sich auf Arbeiten in
Gold und Silber, in Kupfer und Eisen, in rotem Purpur, karmesinfarbigen
Stoffen und blauem Purpur versteht und geschickt ist, Steine zu
schneiden✲, im Verein mit den Künstlern, die mir in Juda und Jerusalem
zur Verfügung stehen und die mein Vater David schon
beschafft\textless sup title=``oder: angestellt''\textgreater✲ hat.
7Sende mir auch Zedernstämme, Zypressen und Sandelholz vom Libanon; denn
ich weiß, daß deine Leute sich darauf verstehen, die Bäume auf dem
Libanon zu fällen; dabei sollen dann meine Leute den deinen behilflich
sein. 8Es muß mir aber Holz in Menge beschafft werden; denn das Haus,
das ich bauen will, soll von ungewöhnlicher Größe sein. 9Ich bin
übrigens bereit, für deine Leute, die Holzhauer, welche die Bäume
fällen, zum Unterhalt 20000 Kor Weizen, 20000 Kor Gerste, 20000 Bath
Wein und 20000 Bath Öl zu liefern.«

\hypertarget{b-hurams-antwort-und-zusage}{%
\paragraph{b) Hurams Antwort und
Zusage}\label{b-hurams-antwort-und-zusage}}

10Huram, der König von Tyrus, antwortete in einem Briefe, den er an
Salomo sandte, folgendermaßen: »Weil der HERR sein Volk liebt, hat er
dich zum König über sie gemacht.« 11Dann fuhr Huram fort: »Gepriesen sei
der HERR, der Gott Israels, der den Himmel und die Erde geschaffen hat,
daß er dem König David einen weisen Sohn geschenkt hat, der so klug und
einsichtsvoll ist, daß er dem HERRN einen Tempel und sich selbst einen
Königspalast erbauen will! 12So sende ich dir denn einen
kunstverständigen, einsichtsvollen Mann, nämlich Huram-Abi, 13den Sohn
einer Danitin, der aber einen Tyrer zum Vater hat. Er versteht sich auf
Arbeiten in Gold und Silber, in Kupfer und Eisen, in Steinen und Holz,
in rotem und blauem Purpur, in Byssus und karmesinfarbigen Stoffen; er
kann auch Schnitzarbeit\textless sup title=``oder:
Gravierungen''\textgreater✲ jeder Art und alle Kunstarbeiten, die ihm
übertragen werden, im Verein mit deinen Künstlern und den Künstlern
meines Herrn David, deines Vaters, ausführen. 14So möge denn mein Herr
den Weizen und die Gerste, das Öl und den Wein seinen Knechten senden,
wie mein Herr es versprochen hat; 15wir aber wollen Bäume, soviele du
überhaupt bedarfst, auf dem Libanon fällen und wollen sie dir als Flöße
auf dem Meer nach Japho✲ zuführen; du kannst sie dann von dort nach
Jerusalem hinaufschaffen.«

\hypertarget{c-salomo-hebt-die-nicht-israeliten-zu-fronarbeitern-aus}{%
\paragraph{c) Salomo hebt die Nicht-Israeliten zu Fronarbeitern
aus}\label{c-salomo-hebt-die-nicht-israeliten-zu-fronarbeitern-aus}}

16Nun ließ Salomo alle Fremdlinge✲, die sich im Lande Israel befanden,
zählen nach der (früheren) Zählung, die sein Vater David mit ihnen
vorgenommen hatte\textless sup title=``vgl. 1.Chr 22,2''\textgreater✲;
da fanden sich deren 153600. 17Von diesen machte er 70000 zu
Lastträgern, 80000 zu Steinhauern im Gebirge (Juda) und 3600 zu
Aufsehern, welche die Leute zur Arbeit anzuhalten hatten.

\hypertarget{beginn-des-tempelbaues-die-ausstattung-des-tempels}{%
\subsubsection{3. Beginn des Tempelbaues; die Ausstattung des
Tempels}\label{beginn-des-tempelbaues-die-ausstattung-des-tempels}}

\hypertarget{section-2}{%
\section{3}\label{section-2}}

1Hierauf begann Salomo den Tempel des HERRN in Jerusalem zu bauen auf
dem Berge Morija, wo der HERR seinem Vater David erschienen war, auf dem
Platze, den David dazu bestimmt hatte, nämlich auf der Tenne des
Jebusiters Ornan\textless sup title=``vgl. 1.Chr
21,18-22,1''\textgreater✲; 2und zwar begann er den Bau am zweiten (Tage)
im zweiten Monat, im vierten Jahre seiner Regierung.

\hypertarget{a-mauxdfe-und-schmuck-des-tempelhauses}{%
\paragraph{a) Maße und Schmuck des
Tempelhauses}\label{a-mauxdfe-und-schmuck-des-tempelhauses}}

3Folgendes sind aber die Grundmaße, an welche Salomo sich beim Bau des
Hauses Gottes hielt: die Länge betrug sechzig Ellen nach dem
alten\textless sup title=``d.h. mosaischen''\textgreater✲ Maß und die
Breite zwanzig Ellen. 4Die Halle, die sich vor der Breitseite des
Großraumes des Hauses befand, war 20 Ellen lang und 20 Ellen hoch; er
ließ sie im Inneren mit reinem Golde überziehen. 5Den großen Tempelraum
täfelte er mit Zypressenholz, überzog dieses sodann mit gediegenem Gold
und brachte Palmen und Blumengewinde darauf an. 6Weiter
überzog\textless sup title=``oder: belegte''\textgreater✲ er das
Tempelhaus mit kostbaren Steinen zum Schmuck; das Gold aber war
Parwaimgold. 7Auch überkleidete er im Tempelhause die Balken (der
Decke), die Schwellen, seine Wände und Türen mit Gold und ließ Cherube
an den Wänden einschnitzen.

\hypertarget{b-ausstattung-des-allerheiligsten}{%
\paragraph{b) Ausstattung des
Allerheiligsten}\label{b-ausstattung-des-allerheiligsten}}

8Dann stellte er den Raum des Allerheiligsten her, dessen Länge
entsprechend der Breite des Tempelhauses 20 Ellen betrug; die Breite war
auch 20 Ellen, und er ließ es mit feinem Gold im Betrage von
sechshundert Talenten überziehen; 9das Gewicht der Nägel aber betrug
fünfzig Schekel Gold. Auch die Obergemächer erhielten eine Bekleidung
von Gold. 10Sodann ließ er im Raum des Allerheiligsten zwei Cherube
herstellen, ein Werk der Bildhauerkunst, und man überzog sie mit Gold.
11Die Flügel der Cherube hatten eine Gesamtlänge von zwanzig Ellen; der
fünf Ellen lange Flügel des einen Cherubs berührte die Wand des Raumes,
während der andere, ebenfalls fünf Ellen lange Flügel an den Flügel des
anderen Cherubs stieß. 12Ebenso reichte der eine, fünf Ellen lange
Flügel des zweiten Cherubs bis an die Wand des Raumes, während der
andere, gleichfalls fünf Ellen lange Flügel an den Flügel des ersten
Cherubs stieß, 13so daß die Flügel dieser beiden Cherube ausgebreitet
zwanzig Ellen maßen. Sie selbst aber standen aufrecht auf ihren Füßen,
und ihre Gesichter waren dem Innenraum des Tempels zugewandt. 14Weiter
ließ er einen Vorhang von blauem und rotem Purpur, von karmesinfarbigem
Stoff und Byssus herstellen und brachte Cherube darauf an.

\hypertarget{c-die-beiden-ehernen-suxe4ulen-vor-dem-tempelhause}{%
\paragraph{c) Die beiden ehernen Säulen vor dem
Tempelhause}\label{c-die-beiden-ehernen-suxe4ulen-vor-dem-tempelhause}}

15Dann ließ er vor dem Tempelhause zwei Säulen herstellen; sie waren 35
Ellen lang✲, und der Knauf oben darauf maß 5 Ellen; 16auch ließ er
Kettenwerk für den unteren Saum der Knäufe anfertigen und es oben an den
Säulen anbringen; an die Ketten aber tat er hundert kunstvoll
gearbeitete Granatäpfel. 17Die Säulen ließ er dann auf der Vorderseite
des Tempels aufstellen, die eine rechts, die andere links; und die
rechtsstehende nannte er Jachin\textless sup title=``d.h. er gründet
fest''\textgreater✲, die linksstehende Boas\textless sup title=``d.h. in
ihm ist Kraft''\textgreater✲.

\hypertarget{d-herstellung-der-tempelgeruxe4te}{%
\paragraph{d) Herstellung der
Tempelgeräte}\label{d-herstellung-der-tempelgeruxe4te}}

\hypertarget{section-3}{%
\section{4}\label{section-3}}

1Weiter ließ er einen kupfernen Altar von 20 Ellen Länge, 20 Ellen
Breite und 10 Ellen Höhe herstellen.~-- 2Auch fertigte er das aus Erz
gegossene Meer\textless sup title=``=~große Wasserbecken''\textgreater✲
an, das von einem Rande bis zum andern zehn Ellen maß, ringsum gerundet
und fünf Ellen hoch; eine Schnur von dreißig Ellen war erforderlich, um
es ganz zu umspannen. 3Unten an ihm waren Gebilde von wilden
Gurken\textless sup title=``=~Koloquinten, gurkenartigen
Früchten''\textgreater✲ angebracht, die es rings umgaben, je zehn auf
die Elle; sie bildeten einen Kranz um das Becken, zwei Reihen Gurken,
die gleich bei seinem Guß mitgegossen worden waren. 4Es ruhte auf zwölf
Rindern, von denen drei nach Norden, drei nach Westen, drei nach Süden
und drei nach Osten gewandt waren; das Becken aber lag oben auf ihnen,
und die Hinterseite war bei allen Rindern nach innen gekehrt. 5Die Dicke
(der Wand) des Beckens betrug eine Handbreite, und sein Rand war wie ein
Becherrand geformt, nach Art einer blühenden Lilie. An Inhalt faßte es
dreitausend Bath\textless sup title=``vgl. 1.Kön 7,23-26''\textgreater✲.
6Er ließ auch zehn Becken\textless sup title=``oder:
Kessel''\textgreater✲ anfertigen und fünf von ihnen rechts und fünf
links aufstellen, damit man Waschungen darin vornähme; man spülte in
ihnen die Stücke ab, die zum Brandopfer gehörten; das große Wasserbecken
dagegen diente den Priestern für ihre Waschungen.~-- 7Weiter ließ er die
goldenen Leuchter, zehn an der Zahl, so anfertigen, wie es der für sie
geltenden Vorschrift entsprach, und stellte sie in den Tempel, fünf auf
die rechte und fünf auf die linke Seite.~-- 8Weiter ließ er zehn Tische
anfertigen und stellte sie im Tempel auf, fünf auf der rechten und fünf
auf der linken Seite; auch ließ er hundert goldene Sprengschalen
anfertigen.~-- 9Sodann stellte er den Vorhof der Priester und den großen
Vorhof her und die Tore für den Vorhof; die dazugehörigen Torflügel
überzog er mit Kupfer. 10Das große Wasserbecken aber stellte er auf der
Südseite (des Tempelhauses) nach Osten zu auf, an der Südostecke des
Tempels.

11Weiter fertigte Huram die Töpfe, die Schaufeln und die Sprengschalen
an und vollendete so die Arbeiten, die er für den König Salomo im
Gotteshause herzustellen hatte, 12nämlich die zwei Säulen mit den beiden
kugelförmigen Knäufen\textless sup title=``oder:
Kapitellen''\textgreater✲ oben auf den Säulen; dazu die zwei Geflechte
zur Bekleidung der beiden kugelförmigen Knäufe\textless sup
title=``oder: Kapitelle''\textgreater✲ oben auf den Säulen; 13ferner die
vierhundert Granatäpfel für die beiden Flechtwerke; zwei Reihen
Granatäpfel für jedes Flechtwerk {[}zur Bekleidung der beiden
kugelförmigen Knäufe oben auf den Säulen{]}. 14Auch fertigte er die
(zehn) Gestühle nebst den (zehn) Kesseln auf den Gestühlen an 15sowie
das eine große Wasserbecken mit den zwölf Rindern unter ihm; 16auch die
Töpfe, die Schaufeln und die Gabeln nebst allen anderen Geräten fertigte
Huram-Abiw dem König Salomo für den Tempel des HERRN aus geglättetem✲
Kupfer an. 17In der Jordanaue hatte der König sie gießen lassen an der
Furt von Adama, zwischen Sukkoth und Zereda. 18Salomo ließ aber alle
diese Geräte in sehr großer Anzahl herstellen; denn das Gewicht des
Erzes wurde nicht festgestellt.

19Weiter ließ Salomo alle die (Gold-) Geräte anfertigen, die zum Hause
Gottes gehörten, nämlich den vergoldeten Altar und die Tische, auf denen
die Schaubrote lagen; 20weiter die Leuchter samt den zugehörigen Lampen,
damit sie vorschriftsmäßig vor dem Allerheiligsten angezündet würden,
aus gediegenem Golde, 21dazu die Blüten und die Lampen und Lichtscheren
aus Gold, und zwar aus lauterem Gold; 22weiter die Messer,
Sprengschalen, Schüsseln und Räucherpfannen aus gediegenem Gold. Was
schließlich die Türen des Tempels betrifft, so waren die inneren
Türflügel am Eingang zum Allerheiligsten sowie die Flügeltüren des
Tempelhauses, die in den großen Raum führten, von Gold.

\hypertarget{e-die-in-den-schatzkammern-untergebrachten-kostbarkeiten}{%
\paragraph{e) Die in den Schatzkammern untergebrachten
Kostbarkeiten}\label{e-die-in-den-schatzkammern-untergebrachten-kostbarkeiten}}

\hypertarget{section-4}{%
\section{5}\label{section-4}}

1Als nun alle Arbeiten, die Salomo für den Tempel des HERRN hatte
herstellen lassen, fertig waren, ließ Salomo die Weihgeschenke seines
Vaters David hineinbringen und legte das Silber, das Gold und sämtliche
Geräte in den Schatzkammern des Gotteshauses nieder.

\hypertarget{die-einweihung-des-tempels}{%
\subsubsection{4. Die Einweihung des
Tempels}\label{die-einweihung-des-tempels}}

\hypertarget{a-die-uxfcberfuxfchrung-der-bundeslade-ins-allerheiligste}{%
\paragraph{a) Die Überführung der Bundeslade ins
Allerheiligste}\label{a-die-uxfcberfuxfchrung-der-bundeslade-ins-allerheiligste}}

2Damals ließ Salomo die Ältesten der Israeliten und alle Häupter der
Stämme, die Fürsten der israelitischen Geschlechter, in Jerusalem
zusammenkommen, um die Bundeslade des HERRN aus der Davidsstadt, das ist
Zion, hinaufzubringen. 3So versammelten sich denn alle Israeliten beim
König zum Fest, nämlich im Monat Ethanim, das ist der siebte Monat. 4Als
nun alle Ältesten der Israeliten sich eingefunden hatten, hoben die
Leviten die Lade auf 5und trugen sie hinauf, ebenso das Offenbarungszelt
samt allen heiligen Geräten, die sich im Zelt befanden: die levitischen
Priester trugen sie hinauf. 6Der König Salomo aber und die ganze
Volksgemeinde Israel, die bei ihm versammelt war, standen vor der Lade
und opferten so viele Stück Kleinvieh und Rinder, daß ihre Menge
geradezu unzählbar war. 7Alsdann brachten die Priester die Lade mit dem
Bundesgesetz des HERRN an den für sie bestimmten Platz, nämlich in den
Hinterraum des Tempelhauses, in das Allerheiligste, unter die Flügel der
Cherube; 8die Cherube hielten nämlich die Flügel ausgebreitet über den
Platz, wo die Lade stand, so daß die Cherube eine Decke oben über der
Lade und deren Tragstangen bildeten. 9Die Tragstangen aber waren so
lang, daß die Spitzen der Stangen im Heiligtum an der Vorderseite des
Allerheiligsten sichtbar waren; weiter draußen aber waren sie nicht zu
sehen; und sie sind dort geblieben bis auf den heutigen Tag. 10In der
Lade befand sich nichts als nur die beiden Tafeln, die Mose am Horeb
hineingelegt hatte, als der HERR mit den Israeliten nach ihrem Auszug
aus Ägypten einen Bund schloß.

\hypertarget{b-die-erscheinung-der-herrlichkeit-gottes}{%
\paragraph{b) Die Erscheinung der Herrlichkeit
Gottes}\label{b-die-erscheinung-der-herrlichkeit-gottes}}

11Als aber die Priester aus dem Heiligtum hinaustraten -- alle Priester
nämlich, die zugegen waren, hatten sich geheiligt, ohne daß man
Rücksicht auf die (Reihenfolge der) Abteilungen genommen hatte, 12und
die levitischen Sänger standen insgesamt, nämlich Asaph, Heman, Jeduthun
nebst ihren Söhnen und Amtsgenossen, in Byssus gekleidet, mit Zimbeln,
Harfen und Zithern auf der Ostseite des Altars und bei ihnen an
hundertundzwanzig Priester, die auf Trompeten bliesen; 13es hatten aber
die Trompeter und die Sänger wie ein Mann mit einer Stimme\textless sup
title=``d.h. gleichzeitig und einstimmig''\textgreater✲ einzusetzen, um
dem HERRN Lobpreis und Dank darzubringen --, sobald sich also der Schall
der Trompeten und Zimbeln und der übrigen Musikinstrumente erhob und man
das Loblied auf den HERRN anstimmte: »Denn er ist gütig, und seine Gnade
währet ewiglich«: da wurde das Haus, der Tempel des HERRN, von einer
Wolke erfüllt, 14so daß die Priester wegen der Wolke nicht hintreten
konnten, um ihren Dienst zu versehen; denn die Herrlichkeit des HERRN
erfüllte das Gotteshaus.

\hypertarget{c-des-kuxf6nigs-weihespruch-und-weiherede-an-das-volk}{%
\paragraph{c) Des Königs Weihespruch und Weiherede an das
Volk}\label{c-des-kuxf6nigs-weihespruch-und-weiherede-an-das-volk}}

\hypertarget{section-5}{%
\section{6}\label{section-5}}

1Damals sprach Salomo: »Der HERR hat erklärt, er wolle im Dunkel wohnen.
2Ich aber habe dir ein Haus zur Wohnung gebaut, eine Stätte zum Wohnsitz
für dich auf ewige Zeiten!« 3Hierauf wandte der König sich um und sprach
einen Segensspruch über die ganze Volksgemeinde Israel, wobei die ganze
Gemeinde Israel stand. 4Dann sprach er: »Gepriesen sei der HERR, der
Gott Israels, der die Verheißung, die er meinem Vater David mündlich
gegeben, nun tatsächlich erfüllt hat, da er sagte: 5›Seit der Zeit, da
ich mein Volk (Israel) aus Ägypten hinausgeführt, habe ich aus allen
Stämmen Israels nie eine Stadt dazu erwählt, daß mir daselbst ein Haus
gebaut würde, an dem mein Name haften sollte; und ich habe auch niemand
dazu erwählt, daß er Fürst über mein Volk Israel sein sollte. 6Dann aber
habe ich Jerusalem erwählt, daß mein Name daselbst wohne, und ich habe
David dazu ersehen, Herrscher über mein Volk Israel zu sein.‹ 7Nun hegte
zwar mein Vater David den Wunsch, dem Namen des HERRN, des Gottes
Israels, ein Haus zu bauen; 8aber der HERR ließ meinem Vater David
verkünden: ›Daß du den Wunsch gehegt hast, meinem Namen ein Haus zu
bauen, daran hast du wohlgetan; 9jedoch nicht du sollst das Haus bauen,
sondern dein leiblicher Sohn, der dir geboren werden wird, der soll
meinem Namen das Haus bauen.‹ 10Nun hat der HERR die Verheißung, die er
gegeben, in Erfüllung gehen lassen; denn ich bin an die Stelle meines
Vaters David getreten und habe den Thron Israels bestiegen, wie der HERR
es verheißen hat, und ich habe dem Namen des HERRN, des Gottes Israels,
den Tempel erbaut 11und habe daselbst die Lade aufgestellt, in der sich
die Urkunde des Bundes befindet, den der HERR mit den Israeliten
geschlossen hat.«

\hypertarget{d-salomos-weihegebet}{%
\paragraph{d) Salomos Weihegebet}\label{d-salomos-weihegebet}}

12Hierauf trat er angesichts der ganzen Gemeinde Israels vor den Altar
des HERRN und breitete seine Hände aus~-- 13Salomo hatte nämlich eine
Kanzel von Kupfer anfertigen und sie mitten auf dem Tempelhofe
aufstellen lassen; sie war fünf Ellen lang, fünf Ellen breit und drei
Ellen hoch; auf diese trat er nun, ließ sich angesichts der ganzen
Versammlung der Israeliten auf seine Knie nieder, breitete seine Hände
gen Himmel aus 14und betete: »HERR, du Gott Israels! Kein Gott weder im
Himmel noch auf der Erde ist dir gleich, der du den Bund und die Gnade
deinen Knechten bewahrst, die mit ganzem Herzen vor dir wandeln. 15Du
hast deinem Knechte David, meinem Vater, gehalten, was du ihm verheißen
hattest; ja, was du mündlich zugesagt hattest, das hast du tatsächlich
erfüllt, wie es heute sichtbar zu Tage liegt! 16Und nun, o HERR, Gott
Israels! Halte deinem Knechte David, meinem Vater, die Verheißung, die
du ihm gegeben hast mit den Worten: ›Es soll dir nie an einem
(Nachkommen) fehlen, der vor meinem Angesicht auf dem Throne Israels
sitze, wofern nur deine Nachkommen auf ihren Weg achthaben, daß sie nach
meinem Gesetz wandeln, wie du vor mir gewandelt bist.‹ 17Nun also, o
HERR, Gott Israels: laß deine Verheißung, die du deinem Knechte David
gegeben hast, in Erfüllung gehen!

18Wie aber? Sollte Gott wirklich bei den Menschen auf der Erde Wohnung
nehmen? Siehe, der Himmel und aller Himmel Himmel können dich nicht
fassen: wieviel weniger dieses Haus, das ich gebaut habe! 19Und doch
wende dich zu dem Gebet deines Knechtes und zu seinem Flehen, o HERR,
mein Gott, daß du hörst auf das Rufen und das Gebet, das dein Knecht an
dich richtet! 20Laß deine Augen bei Tag und bei Nacht offen stehen über
diesem Hause, über der Stätte, von der du verheißen hast, du wollest
deinen Namen daselbst wohnen lassen, daß du das Gebet erhörst, welches
dein Knecht an dieser Stätte verrichten wird! 21So erhöre denn das
Flehen deines Knechtes und deines Volkes Israel, sooft sie an dieser
Stätte beten werden! Erhöre du es von der Stätte, wo du thronst, vom
Himmel her, und wenn du es hörst, so vergib! 22Wenn sich jemand gegen
seinen Nächsten vergeht und man ihm einen Eid\textless sup title=``vgl.
1.Kön 8,31''\textgreater✲ auferlegt, den er schwören soll, und er kommt
und schwört vor deinem Altar in diesem Hause: 23so wollest du es vom
Himmel her hören und eingreifen und deinen Knechten Recht schaffen,
indem du dem Schuldigen dadurch vergiltst, daß du sein Tun auf sein
Haupt zurückfallen läßt, dem Unschuldigen aber dadurch zu seinem Recht
verhilfst, daß du ihm zuteil werden läßt nach seiner
Gerechtigkeit\textless sup title=``oder: Unschuld''\textgreater✲!

24Und wenn dein Volk Israel vom Feinde geschlagen wird, weil es sich an
dir versündigt hat, sich dann aber bekehrt und deinen Namen bekennt und
in diesem Hause zu dir betet und fleht: 25so wollest du es vom Himmel
her hören und deinem Volke Israel die Sünde vergeben und sie in das Land
zurückbringen\textless sup title=``oder: in dem Lande wohnen
lassen''\textgreater✲, das du ihnen und ihren Vätern gegeben hast!

26Wenn der Himmel verschlossen bleibt und kein Regen fällt, weil sie
sich an dir versündigt haben, und sie dann an dieser Stätte beten und
deinen Namen bekennen und sich von ihrer Sünde abkehren, weil du sie
demütigst: 27so wollest du es im Himmel hören und deinen Knechten und
deinem Volke Israel die Sünde vergeben, indem du ihnen den rechten Weg
weisest, auf dem sie wandeln sollen, und wollest Regen fallen lassen auf
dein Land, das du deinem Volk zum Erbbesitz gegeben hast!

28Wenn eine Hungersnot im Lande herrscht, wenn die Pest ausbricht, wenn
Getreidebrand oder Vergilben des Getreides, Heuschrecken oder Ungeziefer
über das Land kommen, wenn seine Feinde es in einer seiner Ortschaften
bedrängen oder sonst irgendeine Plage oder irgendeine Krankheit sie
heimsucht: 29was man alsdann erbittet und erfleht, es geschehe von einem
einzelnen Menschen oder von deinem ganzen Volke Israel, wenn ein jeder
sich in seinem Gewissen getroffen und von Reue ergriffen fühlt und er
seine Hände nach diesem Hause hin ausstreckt: 30so wollest du es vom
Himmel her hören an der Stätte, wo du thronst, und wollest Verzeihung
gewähren und einem jeden ganz nach Verdienst vergelten, wie du sein Herz
kennst -- denn du allein kennst das Herz der Menschenkinder --, 31damit
sie dich fürchten und allezeit auf deinen Wegen wandeln, solange sie in
dem Lande leben, das du unsern Vätern gegeben hast!

32Aber auch den Fremdling, der nicht zu deinem Volke Israel gehört,
sondern aus fernem Lande um deines großen Namens willen und wegen deiner
starken Hand und deines hocherhobenen Armes hergekommen ist und vor
diesem Tempel\textless sup title=``oder: nach diesem Tempel
hin''\textgreater✲ betet, 33wollest du vom Himmel her erhören, von der
Stätte her, wo du thronst, und alles das tun, um was der Fremdling dich
anruft, auf daß alle Völker der Erde deinen Namen kennen lernen und
damit sie dich ebenso fürchten wie dein Volk Israel und damit sie inne
werden, daß dieses Haus, welches ich erbaut habe, nach deinem Namen
genannt ist!

34Wenn dein Volk gegen seine Feinde in den Krieg zieht auf dem Wege, den
du sie senden wirst, und sie sich im Gebet zu dir nach dieser Stadt hin
wenden, die du erwählt hast, und nach dem Tempel hin, den ich zu Ehren
deines Namens erbaut habe: 35so wollest du ihr Gebet und Flehen vom
Himmel her hören und ihnen zu ihrem Recht verhelfen!

36Wenn sie sich an dir versündigt haben -- es gibt ja keinen Menschen,
der nicht sündigt -- und du ihnen zürnst und sie dem Feinde preisgibst,
so daß ihre Besieger sie gefangen in ein fernes oder in ein nahes Land
wegführen, 37und sie dann in dem Lande, wohin sie in die Gefangenschaft
geführt sind, in sich gehen und sich bekehren und dich im Lande ihrer
Gefangenschaft mit dem Bekenntnis anrufen: ›Wir haben gesündigt, wir
haben uns vergangen und gottlos gehandelt!‹, 38wenn sie sich also in dem
Lande ihrer Gefangenschaft, wohin man sie geschleppt hat, mit ganzem
Herzen und mit ganzer Seele dir wieder zuwenden und zu dir beten in der
Richtung nach ihrem Lande hin, das du ihren Vätern gegeben hast, und
nach der Stadt hin, die du erwählt hast, und nach dem Tempel hin, den
ich zu Ehren deines Namens erbaut habe: 39so wollest du ihr Gebet und
Flehen vom Himmel her, von der Stätte, wo du thronst, erhören und ihnen
zu ihrem Recht verhelfen und wollest deinem Volke vergeben, was sie
gegen dich gesündigt haben!

40Nun denn, mein Gott, laß doch deine Augen offenstehen und deine Ohren
aufmerken auf das Gebet an dieser Stätte! 41Und nun, mache dich auf,
HERR, mein Gott, zu deiner Ruhestätte, du selbst und deine machtvolle
Lade! Laß deine Priester, HERR, mein Gott, in Heil sich kleiden und
deine Frommen sich freuen des Glücks! 42O HERR, mein Gott, weise deinen
Gesalbten nicht ab! Gedenke der Gnadenerweise, die du deinem Knechte
David verheißen hast!«

\hypertarget{e-erscheinung-der-herrlichkeit-gottes-salomos-und-des-volkes-feierliches-opferfest-nebst-festversammlung}{%
\paragraph{e) Erscheinung der Herrlichkeit Gottes; Salomos und des
Volkes feierliches Opferfest nebst
Festversammlung}\label{e-erscheinung-der-herrlichkeit-gottes-salomos-und-des-volkes-feierliches-opferfest-nebst-festversammlung}}

\hypertarget{section-6}{%
\section{7}\label{section-6}}

1Als dann Salomo mit seinem Gebet zu Ende war, fuhr Feuer vom Himmel
herab und verzehrte das Brandopfer und die Schlachtopfer, und die
Herrlichkeit des HERRN erfüllte das Tempelhaus, 2so daß die Priester
nicht in den Tempel des HERRN hineingehen konnten, weil die Herrlichkeit
des HERRN den Tempel des HERRN erfüllte. 3Als nun alle Israeliten das
Feuer herabfahren und die Herrlichkeit des HERRN über dem Tempel
ausgebreitet sahen, knieten sie, das Angesicht zur Erde geneigt, auf das
Steinpflaster nieder, vollzogen die Anbetung und priesen den HERRN, daß
er gütig sei und daß seine Gnade ewig währe.

4Hierauf brachten der König und das ganze Volk Schlachtopfer vor dem
HERRN dar, 5und zwar ließ Salomo als Schlachtopfer 22000 Rinder und
120000 Stück Kleinvieh schlachten: damit weihten der König und das ganze
Volk den Tempel Gottes ein, 6während die Priester ihres Amtes warteten
und (ebenso) die Leviten mit den gottesdienstlichen Musikinstrumenten,
die der König David hatte anfertigen lassen, den Lobgesang auf den HERRN
vortrugen, daß seine Güte ewig währe, und die Priester ihnen gegenüber
in die Trompeten stießen, während alle Israeliten dabeistanden.~--
7Damals weihte Salomo den mittleren Teil des Vorhofes, der vor dem
Tempelhause des HERRN lag, zur Opferstätte; denn dort brachte er die
Brandopfer und die Fettstücke der Heilsopfer dar, weil der kupferne
Altar, den Salomo hatte herstellen lassen, die Brand- und Speiseopfer
und die Fettstücke (der Heilsopfer) nicht fassen konnte.

8Auf diese Weise beging Salomo damals das Fest sieben Tage lang und ganz
Israel mit ihm, eine gewaltige Festgemeinde, die zusammengekommen war
von der Gegend bei Hamath an bis an den Bach Ägyptens. 9Am achten Tage
aber hielten sie eine Festversammlung; denn die Einweihung des Altars
hatte man sieben Tage lang gefeiert, und das Fest dauerte auch sieben
Tage. 10Am dreiundzwanzigsten Tage des siebten Monats aber entließ
Salomo das Volk in ihre Heimat, fröhlich und wohlgemut wegen der
Wohltaten, mit denen der HERR (seinen Knecht) David und Salomo und sein
Volk Israel gesegnet hatte.

\hypertarget{gottes-abermalige-erscheinung-und-seine-antwort-verheiuxdfung-und-drohung-auf-salomos-gebet}{%
\subsubsection{5. Gottes abermalige Erscheinung und seine Antwort
(Verheißung und Drohung) auf Salomos
Gebet}\label{gottes-abermalige-erscheinung-und-seine-antwort-verheiuxdfung-und-drohung-auf-salomos-gebet}}

11Als nun Salomo den Tempel des HERRN und den königlichen Palast
vollendet und alles, was er im Tempel des HERRN und in seinem Palast
hatte schaffen wollen, glücklich ausgeführt hatte, 12da erschien der
HERR dem Salomo nachts im Traume und sagte zu ihm: »Ich habe dein Gebet
gehört und diesen Ort mir zur Opferstätte erkoren. 13Wenn ich den Himmel
verschließe, so daß kein Regen fällt, oder wenn ich den Heuschrecken
gebiete, das Land abzufressen, oder wenn ich die Pest unter mein Volk
sende 14und mein Volk, das nach meinem Namen genannt ist, sich dann
demütigt und (zu mir) betet und mein Angesicht sucht und sich von seinem
bösen Tun bekehrt: so will ich sie vom Himmel her erhören und ihnen ihre
Sünden vergeben und ihrem Lande Rettung schaffen. 15Fortan sollen also
meine Augen offenstehen und meine Ohren aufmerken auf die Gebete an
dieser Stätte. 16Und nunmehr habe ich dieses Haus erwählt und zu meinem
Heiligtum gemacht, damit mein Name daselbst in Ewigkeit wohnt und meine
Augen und mein Herz daselbst immerdar weilen. 17Wenn du nun vor mir
ebenso wandelst, wie dein Vater David es getan hat, so daß du alles
tust, was ich dir geboten habe, und meine Satzungen und Rechte
beobachtest, 18so will ich den Thron deines Königtums feststellen, wie
ich es deinem Vater David feierlich zugesagt habe mit den Worten: ›Es
soll dir nie an einem (Nachkommen) fehlen, der über Israel herrsche.‹
19Wenn ihr aber von mir abfallt und meine Satzungen und Gebote, die ich
euch zur Pflicht gemacht habe, verlaßt und anderen Göttern zu dienen und
sie anzubeten anfangt, 20so werde ich die Israeliten aus meinem Lande,
das ich ihnen gegeben habe, hinwegreißen und dieses Haus, das ich meinem
Namen geheiligt habe, keines Blickes mehr würdigen und werde es für alle
Völker zum Gegenstand des Hohnes und Spottes machen. 21Und so erhaben
dieser Tempel auch gewesen sein mag, so sollen doch alle, die an ihm
vorübergehen, sich entsetzen; und wenn sie fragen: ›Warum hat der HERR
diesem Lande und diesem Hause solches Geschick widerfahren lassen?‹,
22so wird man antworten: ›Zur Strafe dafür, daß sie den HERRN, den Gott
ihrer Väter, der sie aus dem Lande Ägypten hinausgeführt hatte,
verlassen und sich anderen Göttern zugewandt und sie angebetet und ihnen
gedient haben: darum hat er all dieses Unglück über sie kommen lassen!‹«

\hypertarget{weitere-unternehmungen-und-einrichtungen-salomos}{%
\subsubsection{6. Weitere Unternehmungen und Einrichtungen
Salomos}\label{weitere-unternehmungen-und-einrichtungen-salomos}}

\hypertarget{a-angaben-uxfcber-salomos-stuxe4dte--und-festungsbauten}{%
\paragraph{a) Angaben über Salomos Städte- und
Festungsbauten}\label{a-angaben-uxfcber-salomos-stuxe4dte--und-festungsbauten}}

\hypertarget{section-7}{%
\section{8}\label{section-7}}

1Nach Ablauf der zwanzig Jahre aber, während deren Salomo den Tempel des
HERRN und seinen Palast erbaut hatte, 2da befestigte Salomo die Städte,
die Huram ihm abgetreten hatte, und wies sie Israeliten zum Wohnsitz an.
3Hierauf zog Salomo gegen Hamath-Zoba und unterwarf es. 4Er befestigte
auch Thadmor in der Wüste und alle Vorratsstädte, die er in Hamath
angelegt hatte. 5Weiter baute er das obere und das untere Beth-Horon zu
festen Plätzen mit Mauern, Toren und Torriegeln aus; 6ebenso Baalath und
alle Vorratsstädte, die er besaß, sowie alle Ortschaften, in denen die
Kriegswagen und die Reitpferde\textless sup title=``oder:
Reiter''\textgreater✲ untergebracht wurden, überhaupt alle Bauten, die
Salomo in Jerusalem, auf dem Libanon und im ganzen Bereich seiner
Herrschaft auszuführen wünschte.

\hypertarget{b-salomos-fronarbeiter-und-deren-aufseher-umzug-seiner-gemahlin-der-uxe4gyptischen-prinzessin-in-den-fuxfcr-sie-erbauten-palast}{%
\paragraph{b) Salomos Fronarbeiter und deren Aufseher; Umzug seiner
Gemahlin, der ägyptischen Prinzessin, in den für sie erbauten
Palast}\label{b-salomos-fronarbeiter-und-deren-aufseher-umzug-seiner-gemahlin-der-uxe4gyptischen-prinzessin-in-den-fuxfcr-sie-erbauten-palast}}

7Alles, was noch an Nachkommen von den Hethitern, Amoritern,
Perissitern, Hewitern und Jebusitern übrig war, die nicht zu den
Israeliten gehörten~-- 8die Nachkommen von ihnen, soweit sie im Lande
noch übriggeblieben waren, weil die Israeliten sie nicht ausgerottet
hatten --, die hob Salomo zum Frondienst aus, und sie sind Fronarbeiter
geblieben bis auf den heutigen Tag. 9Von den Israeliten dagegen machte
Salomo keinen zum Leibeigenen für seine Arbeit, sondern sie dienten im
Heer als Kriegsleute, als Befehlshaber und als Obere über seine
Kriegswagen und über seine Reiterei.~-- 10Die Zahl der Oberaufseher, die
der König Salomo hatte, belief sich auf 250; sie hatten die Leute bei
den Arbeiten zu beaufsichtigen.~--

11Die Tochter des Pharaos aber brachte Salomo aus der Davidsstadt in den
Palast hinauf, den er für sie hatte erbauen lassen; denn er sagte: »Es
soll keine meiner Frauen im Hause Davids, des Königs von Israel, wohnen;
denn das sind heilige Stätten, seitdem die Lade des HERRN in sie
eingezogen ist.«

\hypertarget{c-salomos-ordnung-des-opfer--und-tempeldienstes}{%
\paragraph{c) Salomos Ordnung des Opfer- und
Tempeldienstes}\label{c-salomos-ordnung-des-opfer--und-tempeldienstes}}

12Damals brachte Salomo dem HERRN Brandopfer dar auf dem Altar, den er
dem HERRN vor der Vorhalle errichtet hatte, 13und zwar so, daß er
daselbst das opferte, was nach dem Gebot Moses an jedem Tage
erforderlich war: an den Sabbaten, an den Neumonden und dreimal im Jahr
an den Festen, nämlich am Fest der ungesäuerten Brote und am Wochenfest
und am Laubhüttenfest. 14Auch bestellte er nach der Anordnung seines
Vaters David die Abteilungen der Priester zu ihrem Dienst und die
Leviten zu ihren Amtsverrichtungen, so daß sie die Lobgesänge
anzustimmen hatten und den Priestern zur Hand gingen, wie jeder einzelne
Tag es erforderte; ebenso die Torhüter nach ihren Abteilungen für die
einzelnen Tore; denn so bestimmte es der Befehl Davids, des Mannes
Gottes; 15und man wich in keinem Punkt von dem Gebot des Königs in
betreff der Priester und der Leviten ab, auch nicht in betreff der
Schatzkammern.~-- 16So wurde denn das ganze Werk Salomos vom Tage der
Grundlegung des Tempels des HERRN an bis zu dessen Vollendung zur
Ausführung gebracht, bis der Tempel des HERRN fertig dastand.

\hypertarget{d-salomos-ophirfahrten}{%
\paragraph{d) Salomos Ophirfahrten}\label{d-salomos-ophirfahrten}}

17Damals begab sich Salomo nach Ezjon-Geber und nach Eloth\textless sup
title=``oder: Elath; 1.Kön 9,26''\textgreater✲ an der Küste des (Roten)
Meeres im Lande der Edomiter; 18Huram aber sandte ihm durch seine Leute
Schiffe und seekundige Leute, die zusammen mit Salomos Leuten nach Ophir
fuhren und von dort 450 Talente Gold\textless sup title=``vgl. 1.Chr
29,4''\textgreater✲ holten und es dem König Salomo überbrachten.

\hypertarget{salomos-kuxf6nigsherrlichkeit-und-tod}{%
\subsubsection{7. Salomos Königsherrlichkeit und
Tod}\label{salomos-kuxf6nigsherrlichkeit-und-tod}}

\hypertarget{a-besuch-der-kuxf6nigin-von-saba}{%
\paragraph{a) Besuch der Königin von
Saba}\label{a-besuch-der-kuxf6nigin-von-saba}}

\hypertarget{section-8}{%
\section{9}\label{section-8}}

1Als aber die Königin von Saba den Ruhm Salomos vernahm, kam sie, um
Salomo mit Rätselfragen auf die Probe zu stellen, nach Jerusalem mit
einem sehr großen Gefolge und mit Kamelen, die Spezereien\textless sup
title=``=~Gewürzwaren, Balsam''\textgreater✲ und Gold in Menge und
Edelsteine trugen. Als sie nun bei Salomo angekommen war, trug sie ihm
alles vor, was sie sich vorgenommen hatte; 2Salomo aber wußte ihr auf
alle Fragen Antwort zu geben, und nichts war Salomo verborgen, worüber
er ihr nicht hätte Auskunft geben können. 3Als nun die Königin von Saba
sich von der Weisheit Salomos überzeugt hatte und den Palast sah, den er
erbaut hatte, 4und die Speisen auf seiner Tafel und wie seine
Hofleute\textless sup title=``oder: Beamten''\textgreater✲ dasaßen,
ferner die Aufwartung seiner Dienerschaft und ihre Tracht, seine
Mundschenken und deren Gewandung, dazu seine Brandopfer, die er im
Tempel des HERRN darzubringen pflegte, da geriet sie vor Erstaunen ganz
außer sich 5und sagte zum König: »Wahr ist das gewesen, was ich in
meiner Heimat von dir und deiner Weisheit gehört habe. 6Ich wollte dem,
was man mir erzählte, nicht glauben, bis ich jetzt hergekommen bin und
mich mit eigenen Augen überzeugt habe. Und dabei hat man mir noch nicht
einmal die Hälfte von deiner außerordentlichen Weisheit berichtet: du
übertriffst noch das Gerücht, das ich vernommen habe. 7Beneidenswert
sind deine Leute\textless sup title=``oder: Frauen''\textgreater✲ und
beneidenswert diese deine Diener, die beständig vor dir
stehen\textless sup title=``=~um dich sind''\textgreater✲ und deine
Weisheit hören dürfen! 8Gepriesen sei der HERR, dein Gott, der
Wohlgefallen an dir gefunden hat, so daß er dich auf seinen Thron als
König für den HERRN, deinen Gott, gesetzt hat! Weil dein Gott Israel
liebt, darum hat er, um ihm für immer Bestand zu verleihen, dich zum
König über sie bestellt, damit du Recht und Gerechtigkeit übest!«~--
9Hierauf schenkte sie dem König hundertundzwanzig Talente Gold✲ und
Spezereien in großer Menge sowie Edelsteine; niemals ist eine solche
Menge von Spezereien beisammen gewesen, wie die Königin von Saba sie
damals dem König Salomo schenkte. 10{[}Allerdings brachten auch die
Leute Hurams und die Leute Salomos, wenn sie Gold aus Ophir geholt
hatten, Sandelholz und Edelsteine mit; 11und der König ließ von dem
Sandelholz Treppen sowohl für den Tempel des HERRN als auch für den
königlichen Palast herstellen sowie Zithern und Harfen für die Sänger;
aber derartiges war zuvor im Lande Juda nicht zu sehen gewesen.{]} 12Der
König Salomo aber schenkte der Königin von Saba alles, wonach sie
Verlangen trug und was sie sich erbat, abgesehen von dem Gegengeschenk
für das, was sie dem König gebracht hatte. Hierauf trat sie mit ihrem
Gefolge den Rückweg an und zog wieder heim.

\hypertarget{b-salomos-reichtum-kunst--und-prachtwerke-und-fremde-handelsartikel}{%
\paragraph{b) Salomos Reichtum, Kunst- und Prachtwerke und fremde
Handelsartikel}\label{b-salomos-reichtum-kunst--und-prachtwerke-und-fremde-handelsartikel}}

13Das Gewicht des Goldes, das für Salomo in einem Jahre einging, betrug
666 Talente Gold✲, 14ungerechnet die Einkünfte von den Karawanen✲ und
aus dem Handel der Kaufleute; dazu brachten auch alle Könige von Arabien
und die Statthalter des Landes dem Salomo Gold und Silber.~-- 15Der
König Salomo ließ auch zweihundert Langschilde von getriebenem Golde
anfertigen: mit sechshundert Schekel getriebenen Goldes überzog er jeden
Schild; 16ferner dreihundert Tartschen\textless sup title=``d.h.
Kleinschilde''\textgreater✲ von getriebenem Golde: dreihundert Schekel
Gold verwandte er auf den Überzug jeder Tartsche; der König brachte sie
dann im Libanonwaldhause unter.~-- 17Weiter ließ der König einen großen
Thron von Elfenbein anfertigen und ihn mit feinem Gold überziehen. 18Der
Thron hatte sechs Stufen, und ein goldener Fußschemel war an dem Throne
befestigt; auf beiden Seiten des Sitzplatzes befanden sich Armlehnen,
und neben den Armlehnen standen zwei Löwen; 19außerdem standen zwölf
Löwen auf den sechs Stufen zu beiden Seiten: ein derartiges Kunstwerk
ist noch nie für irgendein Königreich hergestellt worden.~-- 20Alle
Trinkgefäße des Königs Salomo bestanden aus Gold; auch alle Geräte des
Libanonwaldhauses waren von feinem Gold; Silber wurde zu Salomos Zeiten
für wertlos geachtet. 21Denn der König hatte Schiffe, die mit den Leuten
Hurams nach Tharsis fuhren; einmal in drei Jahren kamen die
Tharsisschiffe heim und brachten Gold und Silber, Elfenbein, Affen und
Pfauen mit.

\hypertarget{c-salomos-machtstellung-und-sein-dadurch-gefuxf6rderter-reichtum}{%
\paragraph{c) Salomos Machtstellung und sein dadurch geförderter
Reichtum}\label{c-salomos-machtstellung-und-sein-dadurch-gefuxf6rderter-reichtum}}

22So übertraf denn der König Salomo alle Könige der Erde an Reichtum und
Weisheit; 23und alle Könige der Erde suchten Salomo zu sehen, um sich
persönlich von seiner Weisheit zu überzeugen, die Gott ihm ins Herz
gelegt hatte. 24Dabei brachte jeder von ihnen sein Geschenk mit:
silberne und goldene Geräte\textless sup title=``oder:
Kunstwerke''\textgreater✲, Gewänder, Waffen und Spezereien✲, Rosse und
Maultiere, Jahr für Jahr.~-- 25Salomo besaß auch 4000 Gespanne Rosse und
Wagen und 12000 Reitpferde\textless sup title=``oder:
Reiter''\textgreater✲, die er in den Wagenstädten oder in seiner Nähe zu
Jerusalem untergebracht hatte\textless sup title=``vgl. 1.Kön
5,6''\textgreater✲. 26Und er herrschte über alle Könige vom Euphratstrom
an bis zum Philisterlande und bis an die Grenze Ägyptens. 27Und der
König brachte es dahin, daß es in Jerusalem soviel Silber gab wie Steine
und daß die Zedernstämme an Menge den Maulbeerfeigenbäumen in der
Niederung gleichkamen. 28Und man führte für Salomo Pferde aus Ägypten
und aus allen übrigen Ländern ein\textless sup title=``vgl.
1,14-17''\textgreater✲.

\hypertarget{d-die-quellen-der-geschichte-salomos-sein-tod}{%
\paragraph{d) Die Quellen der Geschichte Salomos; sein
Tod}\label{d-die-quellen-der-geschichte-salomos-sein-tod}}

29Die übrige Geschichte Salomos aber, die frühere wie die spätere,
findet sich bekanntlich schon aufgezeichnet in der Geschichte des
Propheten Nathan sowie in der Weissagung Ahias von Silo und in den
Gesichten\textless sup title=``oder: Offenbarungen''\textgreater✲ des
Sehers Iddo über\textless sup title=``oder: gegen''\textgreater✲
Jerobeam, den Sohn Nebats. 30Salomo hat aber vierzig Jahre lang in
Jerusalem über ganz Israel geherrscht. 31Dann legte Salomo sich zu
seinen Vätern, und man begrub ihn in der Stadt seines Vaters David. Sein
Sohn Rehabeam folgte ihm in der Regierung nach.

\hypertarget{ii.-geschichte-der-kuxf6nige-von-juda-von-rehabeam-bis-zedekia-kap.-10-36}{%
\subsection{II. Geschichte der Könige von Juda von Rehabeam bis Zedekia
(Kap.
10-36)}\label{ii.-geschichte-der-kuxf6nige-von-juda-von-rehabeam-bis-zedekia-kap.-10-36}}

\hypertarget{rehabeam-und-jerobeam-zu-sichem-die-spaltung-des-reiches}{%
\subsubsection{1. Rehabeam und Jerobeam zu Sichem; die Spaltung des
Reiches}\label{rehabeam-und-jerobeam-zu-sichem-die-spaltung-des-reiches}}

\hypertarget{a-der-reichstag-zu-sichem-bitte-der-israeliten-um-erleichterung}{%
\paragraph{a) Der Reichstag zu Sichem; Bitte der Israeliten um
Erleichterung}\label{a-der-reichstag-zu-sichem-bitte-der-israeliten-um-erleichterung}}

\hypertarget{section-9}{%
\section{10}\label{section-9}}

1Rehabeam begab sich nun nach Sichem; denn in Sichem hatten sich alle
Israeliten eingefunden, um ihn zum König zu machen. 2Sobald nun
Jerobeam, der Sohn Nebats, Kunde davon erhielt -- er befand sich nämlich
in Ägypten, wohin er vor dem König Salomo geflohen war --, da kehrte
Jerobeam aus Ägypten zurück; 3man hatte nämlich hingesandt und ihn holen
lassen. So kamen denn Jerobeam und das ganze Israel und trugen dem
Rehabeam folgendes vor: 4»Dein Vater hat uns ein hartes Joch auferlegt;
so erleichtere du uns jetzt deines Vaters harten Dienst und das schwere
Joch, das er uns auferlegt hat, so wollen wir dir untertan sein!« 5Er
antwortete ihnen: »Geduldet euch noch drei Tage, dann kommt wieder zu
mir!«

\hypertarget{b-die-beratung-rehabeams}{%
\paragraph{b) Die Beratung Rehabeams}\label{b-die-beratung-rehabeams}}

Nachdem sich nun das Volk entfernt hatte, 6beriet sich der König
Rehabeam mit den alten Räten, die seinem Vater Salomo bei dessen
Lebzeiten gedient hatten, und fragte sie: »Welche Antwort ratet ihr mir
diesen Leuten zu erteilen?« 7Sie erwiderten ihm: »Wenn du dich heute
diesen Leuten gegenüber freundlich zeigst und sie gnädig behandelst und
ihnen eine gütige Antwort erteilst, so werden sie dir stets gehorsame
Untertanen sein.« 8Aber er ließ den Rat, den ihm die Alten gegeben
hatten, unbeachtet und beriet sich mit den jungen Männern, die mit ihm
aufgewachsen waren und jetzt in seinen Diensten standen. 9Er fragte sie:
»Welche Antwort müssen wir nach eurer Ansicht diesen Leuten geben, die
von mir eine Erleichterung des Joches verlangen, das mein Vater ihnen
auferlegt hat?« 10Da gaben ihm die jungen Männer, die mit ihm
aufgewachsen waren, folgende Antwort: »So mußt du den Leuten antworten,
die von dir eine Erleichterung des schweren Joches verlangen, das dein
Vater ihnen auferlegt hat, -- so mußt du ihnen antworten: ›Mein kleiner
Finger ist dicker als meines Vaters Lenden! 11Und nun: hat mein Vater
euch ein schweres Joch aufgeladen, so will ich euer Joch noch schwerer
machen; hat mein Vater euch mit Peitschen gezüchtigt, so will ich euch
mit Skorpionen züchtigen!«

\hypertarget{c-abfall-der-zehn-stuxe4mme-jerobeams-wahl-zum-kuxf6nig-von-israel}{%
\paragraph{c) Abfall der zehn Stämme; Jerobeams Wahl zum König von
Israel}\label{c-abfall-der-zehn-stuxe4mme-jerobeams-wahl-zum-kuxf6nig-von-israel}}

12Als nun Jerobeam mit dem ganzen Volk am dritten Tage zu Rehabeam kam,
wie der König ihnen befohlen hatte mit den Worten: »Kommt am dritten
Tage wieder zu mir!«, 13gab der König ihnen eine harte Antwort; denn der
König Rehabeam ließ den Rat der Alten unbeachtet 14und gab ihnen nach
dem Rat der jungen Männer folgende Antwort: »Hat mein Vater euch ein
schweres Joch auferlegt, so will ich es noch schwerer machen; hat mein
Vater euch mit Peitschen gezüchtigt, so will ich euch mit Skorpionen✲
züchtigen.« 15So schenkte also der König dem Volke kein Gehör; denn von
Gott war es so gefügt worden, damit der HERR seine Verheißung in
Erfüllung gehen ließe, die er durch den Mund Ahias von Silo dem
Jerobeam, dem Sohne Nebats, gegeben hatte.

16Als nun ganz Israel sah, daß der König ihnen kein Gehör schenkte, ließ
das Volk dem König folgende Erklärung zugehen: »Was haben wir mit David
zu schaffen? Wir haben nichts gemein mit dem Sohne Isais. Ein jeder
begebe sich in seine Heimat, ihr Israeliten! Nun sorge für dein eigenes
Haus, David!« So begaben sich denn sämtliche Israeliten in ihre Heimat,
17so daß Rehabeam nur über die Israeliten, die in den Ortschaften Judas
wohnten, König blieb. 18Als dann der König Rehabeam Hadoram, den
Oberaufseher über die Fronarbeiter, hinsandte, warfen die Israeliten ihn
mit Steinen zu Tode. Da hatte der König Rehabeam nichts Eiligeres zu
tun, als in seinen Wagen zu springen, um nach Jerusalem zu fliehen. 19So
fiel Israel vom Hause Davids ab bis auf den heutigen Tag.

\hypertarget{d-rehabeam-steht-auf-gottes-weisung-vom-kriege-gegen-israel-ab}{%
\paragraph{d) Rehabeam steht auf Gottes Weisung vom Kriege gegen Israel
ab}\label{d-rehabeam-steht-auf-gottes-weisung-vom-kriege-gegen-israel-ab}}

\hypertarget{section-10}{%
\section{11}\label{section-10}}

1Als aber Rehabeam in Jerusalem angekommen war, bot er den Stamm Juda
und Benjamin, 180000 auserlesene Krieger, zum Kampf gegen Israel auf, um
das Königtum für Rehabeam wiederzugewinnen. 2Aber das Wort des HERRN
erging an den Gottesmann Semaja also: 3»Sage zu Rehabeam, dem Sohne
Salomos, dem König von Juda, und zu sämtlichen Israeliten in Juda und
Benjamin also: 4›So hat der HERR gesprochen: Ihr sollt nicht hinziehen,
um mit euren Brüdern Krieg zu führen: kehrt allesamt nach Hause zurück!
Denn von mir aus ist dies alles so gefügt worden.« Als sie die Weisung
des HERRN vernahmen, kehrten sie um, ohne gegen Jerobeam zu ziehen.

\hypertarget{die-regierung-rehabeams}{%
\subsubsection{2. Die Regierung
Rehabeams}\label{die-regierung-rehabeams}}

\hypertarget{a-rehabeams-festungsbauten}{%
\paragraph{a) Rehabeams
Festungsbauten}\label{a-rehabeams-festungsbauten}}

5So blieb denn Rehabeam in Jerusalem und baute Ortschaften in Juda zu
Festungen aus, 6und zwar befestigte er Bethlehem, Etam, Thekoa,
7Beth-Zur, Socho, Adullam, 8Gath, Maresa, Siph, 9Adoraim, Lachis, Aseka,
10Zorea, Ajjalon und Hebron, die in Juda und Benjamin lagen; die baute
er zu festen Plätzen aus. 11Er machte aus ihnen starke Festungen, setzte
Befehlshaber über sie und legte Vorräte von Lebensmitteln, von Öl und
Wein hinein 12sowie in jeden Platz Schilde\textless sup title=``d.h.
Großschilde''\textgreater✲ und Speere und setzte sie so in vorzüglich
festen Stand. So waren denn Juda und Benjamin in seiner Gewalt.

\hypertarget{b-zuzug-von-priestern-leviten-und-frommen-leuten-aus-dem-zehnstuxe4mmereich}{%
\paragraph{b) Zuzug von Priestern, Leviten und frommen Leuten aus dem
Zehnstämmereich}\label{b-zuzug-von-priestern-leviten-und-frommen-leuten-aus-dem-zehnstuxe4mmereich}}

13Die Priester und Leviten aber im gesamten Israel stellten sich ihm aus
allen ihren Bezirken zur Verfügung. 14Die Leviten verließen nämlich ihre
Wohnorte und ihr Besitztum und begaben sich nach Juda und Jerusalem,
weil Jerobeam samt seinen Söhnen sie ihres Amtes als Priester des HERRN
entsetzt 15und er sich eigene Priester für den Höhendienst sowie für die
Feldteufel und die Stierbilder bestellt hatte, die er hatte anfertigen
lassen. 16Ihrem Vorgange folgend, kamen dann aus allen Stämmen Israels
diejenigen, welche aufrichtig darauf bedacht waren, den HERRN, den Gott
Israels, zu suchen, nach Jerusalem, um dem HERRN, dem Gott ihrer Väter,
zu opfern. 17Diese stärkten das Reich Juda und befestigten Rehabeam, den
Sohn Salomos, in der Herrschaft drei Jahre lang; denn drei Jahre lang
wandelten sie auf dem Wege Davids und Salomos.

\hypertarget{c-familiengeschichte-rehabeams}{%
\paragraph{c) Familiengeschichte
Rehabeams}\label{c-familiengeschichte-rehabeams}}

18Rehabeam war aber mit Mahalath, einer Tochter Jerimoths, des Sohnes
Davids, und der Abihail, der Tochter Eliabs, des Sohnes Isais,
verheiratet; 19die gebar ihm Söhne, nämlich Jehus, Semarja und Saham.

20Nach ihr verheiratete er sich mit Maacha, der Tochter Absaloms, die
ihm Abia, Atthai, Sisa und Selomith gebar. 21Rehabeam hatte aber Maacha,
die Tochter Absaloms, lieber als alle seine anderen Frauen und
Nebenweiber; er hatte sich nämlich achtzehn Frauen und sechzig
Nebenweiber genommen, von denen ihm achtundzwanzig Söhne und sechzig
Töchter geboren wurden. 22Rehabeam setzte dann Abia, den Sohn der
Maacha, zum Familienhaupt, zum Fürsten unter seinen Brüdern, ein; denn
es war seine Absicht, ihn zum König zu machen. 23Dabei ging er klug zu
Werke, indem er alle seine Söhne auf die einzelnen Landesteile von Juda
und Benjamin und auf alle festen Plätze verteilte und ihnen ein
reichliches Auskommen zuwies und Frauen in Menge für sie warb.

\hypertarget{d-einfall-und-pluxfcnderung-des-uxe4gyptischen-kuxf6nigs-sisak-auftreten-des-propheten-semaja}{%
\paragraph{d) Einfall und Plünderung des ägyptischen Königs Sisak;
Auftreten des Propheten
Semaja}\label{d-einfall-und-pluxfcnderung-des-uxe4gyptischen-kuxf6nigs-sisak-auftreten-des-propheten-semaja}}

\hypertarget{section-11}{%
\section{12}\label{section-11}}

1Als Rehabeam sich aber auf dem Throne befestigt und seine Herrschaft
gesichert hatte, fiel er und ganz Israel mit ihm vom Gesetz des HERRN
ab. 2Da zog im fünften Jahre der Regierung Rehabeams der König Sisak von
Ägypten mit 1200 Wagen und 60000~Reitern gegen Jerusalem heran, weil sie
treulos am HERRN gehandelt hatten; 3und unzählbar war das Kriegsvolk,
das mit ihm aus Ägypten kam: Libyer, Sukkiter und Äthiopier. 4Er
eroberte die festen Plätze, die zu Juda gehörten, und drang bis
Jerusalem vor.

5Da trat der Prophet Semaja vor Rehabeam und vor die Fürsten Judas, die
sich vor Sisak nach Jerusalem zurückgezogen hatten, und sagte zu ihnen:
»So hat der HERR gesprochen: ›Ihr habt mich verlassen; so habe denn auch
ich euch verlassen und euch der Gewalt Sisaks preisgegeben!‹« 6Da
demütigten sich die Fürsten Judas mit dem Könige und bekannten: »Der
HERR ist gerecht!« 7Als nun der HERR sah, daß sie sich gedemütigt
hatten, erging das Wort des HERRN an Semaja folgendermaßen: »Sie haben
sich gedemütigt: ich will sie (daher) nicht vernichten, sondern ihnen in
kurzem Rettung zuteil werden lassen, und mein Grimm soll sich nicht
durch Sisak über Jerusalem ergießen! 8Doch sollen sie ihm dienstbar
werden, damit sie den Unterschied zwischen meinem Dienste und dem Dienst
der weltlichen✲ Königreiche kennenlernen.«

9So zog denn Sisak, der König von Ägypten, gegen Jerusalem heran und
raubte die Schätze des Tempels des HERRN und die Schätze des königlichen
Palastes: alles raubte er, auch die goldenen Schilde nahm er weg, die
Salomo hatte anfertigen lassen. 10An deren Stelle ließ der König
Rehabeam kupferne Schilde herstellen und übergab sie der Obhut der
Befehlshaber seiner Leibwache, die am Eingang zum königlichen Palast die
Wache hatte. 11Sooft sich nun der König in den Tempel des HERRN begab,
kamen die Leibwächter und trugen die Schilde und brachten sie dann
wieder in das Wachtzimmer der Leibwache zurück. 12Weil er sich aber
gedemütigt hatte, wandte sich der Zorn des HERRN von ihm ab, so daß er
ihn nicht gänzlich zugrunde richtete; es fand sich ja damals auch in
Juda noch manches Gute.

\hypertarget{e-schluuxdf-der-regierung-rehabeams-und-die-quellen-seiner-geschichte}{%
\paragraph{e) Schluß der Regierung Rehabeams und die Quellen seiner
Geschichte}\label{e-schluuxdf-der-regierung-rehabeams-und-die-quellen-seiner-geschichte}}

13So befestigte sich denn der König Rehabeam zu Jerusalem wieder in der
Herrschaft und regierte weiter; im Alter von einundvierzig Jahren
nämlich hatte Rehabeam den Thron bestiegen, und siebzehn Jahre regierte
er in Jerusalem, der Stadt, die der HERR aus allen Stämmen Israels
erwählt hatte, um seinen Namen dort wohnen zu lassen. Seine Mutter hieß
Naama und war eine Ammonitin. 14Er hatte aber böse gehandelt, weil er
nicht darauf bedacht gewesen war, den HERRN zu suchen.

15Die Geschichte Rehabeams aber, die frühere wie die spätere, findet
sich bekanntlich aufgezeichnet in der Geschichte des Propheten Semaja
und in der des Sehers Iddo. Die Kriege zwischen Rehabeam und Jerobeam
aber gingen ohne Unterbrechung fort. 16Als Rehabeam sich dann zu seinen
Vätern gelegt hatte, wurde er in der Davidsstadt begraben, und sein Sohn
Abia folgte ihm in der Regierung nach.

\hypertarget{die-regierung-des-kuxf6nigs-abia}{%
\subsubsection{3. Die Regierung des Königs
Abia}\label{die-regierung-des-kuxf6nigs-abia}}

\hypertarget{a-abias-krieg-mit-jerobeam-seine-ansprache-an-das-heer-jerobeams}{%
\paragraph{a) Abias Krieg mit Jerobeam; seine Ansprache an das Heer
Jerobeams}\label{a-abias-krieg-mit-jerobeam-seine-ansprache-an-das-heer-jerobeams}}

\hypertarget{section-12}{%
\section{13}\label{section-12}}

1Im achtzehnten Jahre (der Regierung) des Königs Jerobeam wurde Abia
König über Juda. 2Er regierte drei Jahre in Jerusalem; seine Mutter hieß
Michaja und war eine Tochter Uriels von Gibea. Es herrschte aber Krieg
zwischen Abia und Jerobeam. 3Und Abia eröffnete den Kampf mit einem
kriegstüchtigen Heere von 400000 auserlesenen Kriegern, während Jerobeam
800000~Mann auserlesener Kerntruppen gegen ihn ins Feld stellte.

4Da trat Abia oben auf den Berg Zemaraim, der im Berglande Ephraim
liegt, und rief: »Hört mich an, Jerobeam und ihr Israeliten alle!
5Solltet ihr wirklich nicht wissen, daß der HERR, der Gott Israels, das
Königtum über Israel dem David auf ewig verliehen hat, ihm und seinen
Nachkommen, und zwar auf Grund eines Salzbundes\textless sup
title=``vgl. 3.Mose 2,13''\textgreater✲? 6Aber Jerobeam, der Sohn
Nebats, ein Untertan Salomos, des Sohnes Davids, ist aufgetreten und hat
sich gegen seinen Herrn empört; 7und leichtfertige Männer, nichtswürdige
Leute, haben sich um ihn geschart und die Oberhand über Rehabeam, den
Sohn Salomos, gewonnen; denn Rehabeam war noch jung und unselbständig
und zu schwach, um sich ihrem Einfluß zu entziehen. 8Und nun meint ihr
dem Königtum des HERRN, das im Besitz der Nachkommen Davids ist,
entgegentreten zu können, weil ihr ein großer Haufe seid und ihr die
goldenen Stierbilder bei euch habt, die Jerobeam euch zu Göttern gemacht
hat! 9Habt ihr nicht die Priester des HERRN, die Nachkommen Aarons, und
die Leviten vertrieben und euch eigene Priester gemacht wie die
Heidenvölker? Wer irgend mit einem jungen Stiere und sieben Widdern
gekommen ist, um sich für das Priestertum weihen zu lassen, der ist von
euch zum Priester der Nichtgötter (oder Götzen) gemacht worden. 10Unser
Gott dagegen ist der HERR, und wir sind nicht von ihm abgefallen; und
als Priester dienen dem HERRN die Nachkommen Aarons, und die Leviten
verrichten ihre Amtsgeschäfte; 11sie bringen dem HERRN an jedem Morgen
und jedem Abend Brandopfer und wohlriechendes Räucherwerk dar und legen
die Schaubrote auf dem Tische von reinem Gold aus und zünden den
goldenen Leuchter mit seinen Lampen an jedem Abend an; denn wir
beobachten die Vorschriften des HERRN, unsers Gottes, genau, während ihr
ihn verlassen habt. 12Bedenkt wohl: mit uns ist Gott, der an unserer
Spitze steht, dazu seine Priester mit den Lärmtrompeten, um sie gegen
euch erschallen zu lassen! O ihr Israeliten! Kämpft nicht gegen den
HERRN, den Gott eurer Väter, denn damit werdet ihr kein Glück haben!«

\hypertarget{b-abias-sieg-uxfcber-jerobeam}{%
\paragraph{b) Abias Sieg über
Jerobeam}\label{b-abias-sieg-uxfcber-jerobeam}}

13Jerobeam aber hatte die im Hinterhalt liegenden Mannschaften eine
Schwenkung machen lassen, damit sie ihnen\textless sup title=``d.h. den
Judäern''\textgreater✲ in den Rücken kämen; und so standen sie
einesteils vorn den Judäern gegenüber, während andernteils die
Mannschaften des Hinterhalts sich in ihrem Rücken befanden. 14Als daher
die Judäer sich umwandten, sahen sie sich von vorn und von hinten
angegriffen. Da schrien sie zum HERRN, und die Priester stießen in die
Trompeten, 15und die Judäer erhoben das Kriegsgeschrei; und als sie das
getan hatten, ließ Gott Jerobeam und ganz Israel die Flucht vor Abia und
den Judäern ergreifen. 16Als nun die Israeliten vor den Judäern flohen,
gab Gott sie ihnen preis, 17so daß Abia und seine Leute ein furchtbares
Blutbad unter ihnen anrichteten und von den Israeliten 500000~Mann
auserlesener Mannschaften erschlagen liegen blieben. 18So wurde damals
die Macht der Israeliten gebrochen, während die Judäer die Oberhand
gewannen, weil sie ihr Vertrauen auf den HERRN, den Gott ihrer Väter,
gesetzt hatten. 19Abia verfolgte dann Jerobeam und nahm ihm mehrere
Städte weg, nämlich Bethel nebst den zugehörigen Ortschaften, Jesana
nebst den zugehörigen Ortschaften und Ephron nebst den zugehörigen
Ortschaften. 20Jerobeam aber kam, solange Abia lebte, nicht wieder zu
Kräften, und der HERR ließ ihn eines plötzlichen Todes sterben.

\hypertarget{c-abschluuxdf-und-quellen-der-geschichte-abias}{%
\paragraph{c) Abschluß und Quellen der Geschichte
Abias}\label{c-abschluuxdf-und-quellen-der-geschichte-abias}}

21Abia aber wurde mächtig; er nahm sich vierzehn Frauen, die ihm
zweiundzwanzig Söhne und sechzehn Töchter gebaren. 22Die übrige
Geschichte Abias aber, seine Taten und seine Reden, finden sich
aufgezeichnet in dem erbaulichen\textless sup title=``oder:
ausführlichen''\textgreater✲ Geschichtswerk des Propheten Iddo.~-- 23Als
Abia sich aber zu seinen Vätern gelegt und man ihn in der Davidsstadt
begraben hatte, folgte ihm sein Sohn Asa auf dem Throne nach; unter
dessen Regierung erfreute das Land sich zehn Jahre lang der Ruhe.

\hypertarget{die-regierung-des-kuxf6nigs-asa}{%
\subsubsection{4. Die Regierung des Königs
Asa}\label{die-regierung-des-kuxf6nigs-asa}}

\hypertarget{a-asas-einschreiten-gegen-den-guxf6tzendienst}{%
\paragraph{a) Asas Einschreiten gegen den
Götzendienst}\label{a-asas-einschreiten-gegen-den-guxf6tzendienst}}

\hypertarget{section-13}{%
\section{14}\label{section-13}}

1Asa tat, was in den Augen des HERRN, seines Gottes, gut und recht war;
2denn er beseitigte die Altäre der fremden Götter und den Höhendienst,
ließ die Malsteine zerschlagen und die Götzenpfähle\textless sup
title=``oder: heiligen Bäume''\textgreater✲ umhauen 3und wies die Judäer
darauf hin, den HERRN, den Gott ihrer Väter, zu suchen und das Gesetz
und die Gebote zu erfüllen. 4In allen Ortschaften Judas beseitigte er
den Höhendienst und die Sonnensäulen, und das Reich erfreute sich der
Ruhe unter ihm.

\hypertarget{b-hebung-der-wehrkraft-des-reiches}{%
\paragraph{b) Hebung der Wehrkraft des
Reiches}\label{b-hebung-der-wehrkraft-des-reiches}}

5Er legte sodann feste Plätze in Juda an; denn das Land hatte Ruhe, und
niemand führte in jenen Jahren Krieg mit ihm, weil der HERR ihm Ruhe
verschafft hatte. 6Darum forderte er die Judäer auf: »Laßt uns diese
Ortschaften ausbauen und sie mit Mauern und Türmen, Toren und Riegeln
rings umgeben! Noch haben wir freie Hand im Lande, weil wir den HERRN,
unsern Gott, gesucht haben; wir haben ihn gesucht, und er hat uns
ringsum Ruhe geschafft.« So machten sie sich denn an die Bauten und
führten sie glücklich aus. 7Dazu besaß Asa ein Heer, das Schild und
Speer✲ führte: aus Juda 300000 und aus Benjamin 280000~Mann, welche
Tartschen\textless sup title=``d.h. kleine Schilde''\textgreater✲
führten und den Bogen zu spannen wußten, durchweg tapfere Krieger.

\hypertarget{c-asas-sieg-uxfcber-den-kuschiten-serah}{%
\paragraph{c) Asas Sieg über den Kuschiten
Serah}\label{c-asas-sieg-uxfcber-den-kuschiten-serah}}

8Da zog der Kuschit Serah gegen sie heran mit einem Heere von einer
Million Mann und mit dreihundert Kriegswagen und drang bis Maresa vor.
9Asa zog ihm entgegen, und sie stellten sich im Tal Zephatha bei Maresa
zur Schlacht auf. 10Da rief Asa den HERRN, seinen Gott, an und betete:
»HERR! Um zu helfen, ist bei dir kein Unterschied zwischen einem Starken
und einem Schwachen. So hilf uns, HERR, unser Gott! Denn auf dich setzen
wir unser Vertrauen und in deinem Namen sind wir gegen diese Übermacht
ausgezogen. Du bist der HERR, unser Gott: kein Mensch soll dir gegenüber
sich behaupten dürfen!« 11Da schlug der HERR die Kuschiten, so daß sie
vor Asa und den Judäern die Flucht ergriffen; 12Asa aber und das Heer,
das bei ihm war, verfolgten sie bis Gerar, und es fielen von den
Kuschiten so viele, daß keiner von ihnen mit dem Leben davonkam, sondern
sie vor dem HERRN und seinem Heere völlig aufgerieben wurden. So machten
die Judäer denn eine gewaltige Beute, 13eroberten auch alle Städte rings
um Gerar -- es war nämlich ein Schrecken vom HERRN her über sie gekommen
--, und sie plünderten alle Städte; denn es befand sich eine reiche
Beute in ihnen. 14Auch der Zeltlager mit den Herden bemächtigten sie
sich, trieben Kleinvieh in Menge und Kamele als Beute weg und kehrten
dann nach Jerusalem zurück.

\hypertarget{d-die-mahnrede-des-propheten-asarja}{%
\paragraph{d) Die Mahnrede des Propheten
Asarja}\label{d-die-mahnrede-des-propheten-asarja}}

\hypertarget{section-14}{%
\section{15}\label{section-14}}

1Da kam der Geist Gottes über Asarja, den Sohn Odeds, 2so daß er
hinausging, vor Asa hintrat und zu ihm sagte: »Hört mich an, Asa und ihr
Judäer und Benjaminiten alle! Der HERR ist mit euch, solange ihr euch zu
ihm haltet, und wenn ihr ihn sucht, läßt er sich von euch finden, wenn
ihr ihn aber verlaßt, wird auch er euch verlassen. 3Lange Zeit ist
Israel ohne den wahren Gott gewesen und ohne priesterliche Belehrung und
ohne Gesetz; 4dann aber, in seiner Bedrängnis, kehrte es zum HERRN, dem
Gott Israels, zurück, und da sie ihn suchten, ließ er sich von ihnen
finden. 5Zu jenen Zeiten aber gab es keine Sicherheit für die Aus- und
Eingehenden\textless sup title=``Sach 8,10''\textgreater✲; denn
beständige Unruhen herrschten bei allen Bewohnern der verschiedenen
Landesteile: 6ein Volk wurde von dem andern bedrängt und eine Stadt von
der andern; denn Gott schreckte sie durch Leiden aller Art. 7Ihr aber,
seid stark und laßt eure Hände nicht erschlaffen\textless sup
title=``=~den Mut nicht sinken''\textgreater✲, denn euer Tun wird
belohnt werden!«

\hypertarget{e-asas-erneuerung-des-bundes-mit-gott}{%
\paragraph{e) Asas Erneuerung des Bundes mit
Gott}\label{e-asas-erneuerung-des-bundes-mit-gott}}

8Als Asa diese Worte und die Weissagung des Propheten Asarja, des Sohnes
Odeds, vernahm, gewann er neuen Mut und schaffte die Greuel\textless sup
title=``oder: Scheusale, Götzen''\textgreater✲ aus dem ganzen Lande Juda
und Benjamin und aus den Ortschaften weg, die er im Berglande Ephraim
erobert hatte, und erneuerte den Altar des HERRN, der vor der Vorhalle
(des Tempels) des HERRN stand. 9Dann versammelte er ganz Juda und
Benjamin und diejenigen Ephraimiten, Manassiten und Simeoniten, die als
Fremdlinge bei ihnen lebten; denn es waren ihm aus Israel Leute in
großer Zahl zugefallen, als sie sahen, daß der HERR, sein Gott, mit ihm
war. 10Sie kamen also im dritten Monat des fünfzehnten Regierungsjahres
Asas in Jerusalem zusammen 11und opferten dem HERRN an jenem Tage von
der Beute, die sie heimgebracht hatten, siebenhundert Rinder und
siebentausend Stück Kleinvieh. 12Hierauf verpflichteten sie sich
feierlich, den HERRN, den Gott ihrer Väter, mit ganzem Herzen und mit
ganzer Seele zu suchen; 13jeder aber, der sich nicht an den HERRN, den
Gott Israels, halten würde, sollte mit dem Tode bestraft werden, er
möchte vornehm oder gering, Mann oder Weib sein. 14Diesen Eid leisteten
sie dem HERRN mit lauter Stimme und mit Jubelgeschrei, unter Trompeten-
und Posaunenschall; 15und ganz Juda war voller Freude über den Schwur,
denn sie hatten ihn mit ganzem Herzen geleistet; und weil sie den HERRN
mit aller Aufrichtigkeit suchten, ließ er sich von ihnen finden, und der
HERR verschaffte ihnen Ruhe ringsumher.~-- 16Sogar seiner Mutter Maacha
entzog der König Asa den Rang der Königin-Mutter, weil sie der Aschera
ein Götzenbild hatte anfertigen lassen; Asa ließ ihr Götzenbild umhauen
und zerschlagen und im Kidrontal verbrennen. 17Der Höhendienst wurde
allerdings in Israel nicht abgeschafft, doch blieb das Herz Asas dem
HERRN zeitlebens ungeteilt ergeben. 18Er ließ auch die Geschenke, die
sein Vater und die er selbst geweiht hatte, in den Tempel Gottes
bringen: Silber, Gold und Geräte.

\hypertarget{f-asas-krieg-mit-baesa-von-israel-seine-zuflucht-bei-benhadad-von-syrien}{%
\paragraph{f) Asas Krieg mit Baesa von Israel; seine Zuflucht bei
Benhadad von
Syrien}\label{f-asas-krieg-mit-baesa-von-israel-seine-zuflucht-bei-benhadad-von-syrien}}

19Bis zum fünfunddreißigsten Regierungsjahre Asas fand kein Krieg statt;

\hypertarget{section-15}{%
\section{16}\label{section-15}}

1aber im sechsunddreißigsten Regierungsjahre Asas zog Baesa, der König
von Israel, gegen Juda heran und befestigte Rama, damit niemand mehr bei
Asa, dem Könige von Juda, ungehindert aus- und eingehen könne. 2Da nahm
Asa Silber und Gold aus den Schatzkammern des Tempels des HERRN und des
königlichen Palastes, sandte es an Benhadad, den König von Syrien, der
zu Damaskus wohnte✲, und ließ ihm sagen: 3»Ein Bündnis besteht zwischen
mir und dir, zwischen meinem Vater und deinem Vater! Hier sende ich dir
nun Silber und Gold. So löse denn dein Bündnis mit dem König Baesa von
Israel auf, damit er aus meinem Lande abzieht!« 4Benhadad schenkte der
Aufforderung des Königs Asa Gehör und ließ seine Heerführer gegen die
Städte Israels ziehen; sie eroberten Ijjon, Dan, Abel-Majim und alle
Vorratshäuser in den Ortschaften Naphthalis. 5Sobald nun Baesa Kunde
davon erhielt, gab er die Befestigung Ramas auf und stellte seine Arbeit
dort ein. 6Der König Asa aber bot ganz Juda auf; die mußten die Steine
und Balken wegschaffen, mit denen Baesa Rama hatte befestigen wollen;
und er ließ dann Geba und Mizpa damit befestigen.

\hypertarget{g-hananis-strafrede-an-asa-hat-uxfcble-wirkung}{%
\paragraph{g) Hananis Strafrede an Asa hat üble
Wirkung}\label{g-hananis-strafrede-an-asa-hat-uxfcble-wirkung}}

7Zu jener Zeit aber kam der Seher Hanani zu Asa, dem König von Juda, und
sagte zu ihm: »Weil du dein Vertrauen auf den König von Syrien gesetzt
und dich nicht auf den HERRN, deinen Gott, verlassen hast, darum ist das
Heer des Königs von Syrien der Vernichtung durch dich entgangen. 8Waren
nicht die Kuschiten✲ und Libyer eine gewaltige Heeresmacht mit Wagen und
Reitern in großer Zahl? Aber weil du dein Vertrauen auf den HERRN
setztest, ließ er sie in deine Gewalt fallen. 9Denn die Augen des HERRN
überschauen die ganze Erde, damit er seine Macht zum Heil für die
erweise, deren Herz ungeteilt auf ihn gerichtet ist. Du hast in diesem
Fall töricht gehandelt; denn von nun an wirst du (unablässig) Kriege zu
führen haben.« 10Asa wurde über den Seher unwillig und ließ ihn ins
Stockhaus werfen; denn er war wegen seines Auftretens aufgebracht gegen
ihn. Gleichzeitig ging Asa auch gegen einige Leute aus dem Volke
gewalttätig vor.

\hypertarget{h-asas-ende-und-ehrenvolles-begruxe4bnis}{%
\paragraph{h) Asas Ende und ehrenvolles
Begräbnis}\label{h-asas-ende-und-ehrenvolles-begruxe4bnis}}

11Die Geschichte Asas aber von Anfang bis zu Ende findet sich
bekanntlich bereits im Buch der Könige von Juda und Israel
aufgezeichnet.

12Im neununddreißigsten Jahre seiner Regierung erkrankte Asa an einem
Fußleiden, und zwar in sehr ernster Weise; aber auch in (dieser) seiner
Krankheit suchte er nicht beim HERRN Hilfe, sondern bei den Ärzten.
13Als er sich dann zu seinen Vätern gelegt und im einundvierzigsten
Jahre seiner Regierung gestorben war, 14begrub man ihn in seiner
Grabstätte, die er sich in der Davidsstadt hatte aushauen lassen. Man
legte ihn auf ein Lager, das man mit Spezereien\textless sup
title=``d.h. balsamischen Stoffen''\textgreater✲ aller Art und mit
Wohlgerüchen von kunstgerecht hergestellten Salben angefüllt hatte, und
veranstaltete ihm zu Ehren einen überaus großartigen Leichenbrand.

\hypertarget{die-regierung-des-kuxf6nigs-josaphat}{%
\subsubsection{5. Die Regierung des Königs
Josaphat}\label{die-regierung-des-kuxf6nigs-josaphat}}

\hypertarget{a-josaphats-fromme-und-gluxfcckliche-regierung}{%
\paragraph{a) Josaphats fromme und glückliche
Regierung}\label{a-josaphats-fromme-und-gluxfcckliche-regierung}}

\hypertarget{section-16}{%
\section{17}\label{section-16}}

1Sein Nachfolger auf dem Throne wurde sein Sohn Josaphat, der mit
Nachdruck gegen Israel auftrat. 2Er legte Kriegsvolk in alle festen
Plätze Judas und verteilte Besatzungen über das Land Juda und in die
Städte von Ephraim, die sein Vater Asa erobert hatte. 3Und der HERR war
mit Josaphat, weil er auf den alten Wegen seines Ahnherrn David wandelte
und nicht die Baale aufsuchte, 4sondern sich an den Gott seines Ahnherrn
hielt und nach dessen Geboten wandelte und es nicht wie die Israeliten
machte. 5Darum ließ der HERR das Königtum in seiner Hand\textless sup
title=``=~durch ihn''\textgreater✲ erstarken, so daß alle Judäer ihm
Geschenke brachten und ihm Reichtum und Ehre in hohem Maße zuteil wurde.
6Weil ihm dann der Mut auf den Wegen des HERRN wuchs, beseitigte er auch
noch den Höhendienst und die Götzenbäume\textless sup title=``vgl.
14,2''\textgreater✲ in Juda.

\hypertarget{josaphat-sorgt-fuxfcr-die-unterweisung-des-volkes-im-gesetz-des-herrn}{%
\paragraph{Josaphat sorgt für die Unterweisung des Volkes im Gesetz des
Herrn}\label{josaphat-sorgt-fuxfcr-die-unterweisung-des-volkes-im-gesetz-des-herrn}}

7In seinem dritten Regierungsjahr aber sandte er seine höchsten Beamten
Ben-Hail, Obadja, Sacharja, Nethaneel und Michaja aus, damit sie den
Städten Judas Unterweisung erteilten, 8und mit ihnen die Leviten Semaja,
Nethanja, Sebadja, Asahel, Semiramoth, Jonathan, Adonia, Tobia und
Tob-Adonia {[}die Leviten{]}, dazu mit ihnen die Priester Elisama und
Joram. 9Diese lehrten also in Juda, indem sie das Gesetzbuch des HERRN
bei sich hatten und in allen Ortschaften Judas umherzogen und unter dem
Volke lehrten.

\hypertarget{josaphats-ansehen-bei-den-nachbarvuxf6lkern-und-seine-bedeutende-kriegsmacht}{%
\paragraph{Josaphats Ansehen bei den Nachbarvölkern und seine bedeutende
Kriegsmacht}\label{josaphats-ansehen-bei-den-nachbarvuxf6lkern-und-seine-bedeutende-kriegsmacht}}

10Vom HERRN aber ging ein Schrecken über alle Königreiche in den Ländern
aus, die rings um Juda her lagen, so daß sie keinen Krieg mit Josaphat
anfingen; 11ja sogar von den Philistern kamen Gesandte, die dem Josaphat
Geschenke und Silber als die ihnen auferlegte Abgabe brachten; auch die
Araber brachten ihm Kleinvieh, nämlich 7700 Widder und 7700 Böcke. 12So
wurde Josaphat allmählich immer größer und mächtiger und legte in Juda
Burgen und Vorratsstädte an. 13Er besaß auch bedeutende Vorräte in den
Städten Judas, dazu in Jerusalem ein Heer tapferer Krieger, 14das, nach
Familien geordnet, folgende Zusammensetzung aufwies: Von Juda waren als
Befehlshaber von Tausendschaften: Adna, der Heerführer, der 300000
tapfere Krieger unter sich hatte; 15neben ihm Johanan, der Heerführer,
der 280000~Mann befehligte; 16neben ihm Amasja, der Sohn Sichris, der
sich dem HERRN freiwillig zur Verfügung gestellt hatte und 200000
tapfere Krieger unter sich hatte. 17Aus Benjamin aber waren: der tapfere
Kriegsmann Eljada, der Befehlshaber über 200000~Mann, die mit Bogen und
Schild bewaffnet waren; 18und neben ihm Josabad, der 180000
kriegsgerüstete Leute befehligte. 19Diese waren es, die im Dienst des
Königs standen, abgesehen von jenen, welche der König in die festen
Plätze von ganz Juda gelegt hatte.

\hypertarget{b-josaphats-und-ahabs-ungluxfccklicher-feldzug-gegen-die-syrer}{%
\paragraph{b) Josaphats und Ahabs unglücklicher Feldzug gegen die
Syrer}\label{b-josaphats-und-ahabs-ungluxfccklicher-feldzug-gegen-die-syrer}}

\hypertarget{aa-josaphat-und-ahab-verbuxfcnden-sich-zum-kriege-gegen-die-syrer}{%
\subparagraph{aa) Josaphat und Ahab verbünden sich zum Kriege gegen die
Syrer}\label{aa-josaphat-und-ahab-verbuxfcnden-sich-zum-kriege-gegen-die-syrer}}

\hypertarget{section-17}{%
\section{18}\label{section-17}}

1Als Josaphat aber Reichtum und Ehre in Fülle erlangt hatte,
verschwägerte er sich mit Ahab 2und begab sich dann einige Jahre später
auf Besuch zu Ahab nach Samaria hinab. Da ließ Ahab für ihn und sein
Gefolge Kleinvieh und Rinder in Menge schlachten und beredete ihn, am
Feldzuge gegen Ramoth in Gilead teilzunehmen. 3Der König Ahab von Israel
richtete nämlich an den König Josaphat von Juda die Frage: »Willst du
mit mir gegen Ramoth in Gilead ziehen?« Da erwiderte er ihm: »Ich will
sein wie du: mein Volk (oder Heer) wie dein Volk (oder Heer); ja ich
will mit dir zu Felde ziehen!«

\hypertarget{bb-der-guxfcnstige-bescheid-der-400-propheten-micha-soll-befragt-werden}{%
\subparagraph{bb) Der günstige Bescheid der 400 Propheten; Micha soll
befragt
werden}\label{bb-der-guxfcnstige-bescheid-der-400-propheten-micha-soll-befragt-werden}}

4Als Josaphat dann dem König von Israel riet, zunächst doch den Willen
des HERRN zu erforschen, 5ließ der König von Israel die Propheten
zusammenkommen, vierhundert Mann, und fragte sie: »Sollen wir gegen
Ramoth in Gilead zu Felde ziehen, oder soll ich es unterlassen?« Sie
antworteten: »Ziehe hin, denn Gott wird es dem König in die Hand geben!«

6Da fragte Josaphat: »Ist hier sonst kein Prophet des HERRN mehr, durch
den wir Auskunft erhalten könnten?« 7Der König von Israel erwiderte dem
Josaphat: »Es ist wohl noch einer da, durch den wir den HERRN befragen
könnten; aber ich habe nicht gern mit ihm zu tun, denn er weissagt mir
niemals Gutes, sondern immer nur Unglück; das ist Micha, der Sohn
Jimlas.« Aber Josaphat entgegnete: »Der König wolle nicht so reden!«

8Da rief der König von Israel einen Kammerherrn und befahl ihm,
schleunigst Micha, den Sohn Jimlas, zu holen. 9Während nun der König von
Israel und der König Josaphat von Juda ein jeder auf seinem Thron in
ihren Königsgewändern auf dem freien Platz am Eingang des Stadttores von
Samaria saßen und alle Propheten vor ihnen weissagten, 10machte sich
Zedekia, der Sohn Kenaanas, eiserne Hörner und rief aus: »So spricht der
HERR: ›Mit solchen Hörnern wirst du die Syrer niederstoßen, bis du sie
vernichtet hast!‹« 11Ebenso weissagten auch alle anderen Propheten,
indem sie riefen: »Ziehe hin nach Ramoth in Gilead: du wirst Glück
haben, denn der HERR wird es dem König in die Hand fallen lassen.«

\hypertarget{cc-michas-anfuxe4nglicher-gluxfccksspruch-sodann-seine-unheilsverkuxfcndigung}{%
\subparagraph{cc) Michas anfänglicher Glücksspruch, sodann seine
Unheilsverkündigung}\label{cc-michas-anfuxe4nglicher-gluxfccksspruch-sodann-seine-unheilsverkuxfcndigung}}

12Der Bote aber, der hingegangen war, um Micha zu holen, sagte zu ihm:
»Siehe, die (übrigen) Propheten haben dem König einstimmig Glück
verheißen; so schließe du dich doch ihrem einmütigen Ausspruche an und
prophezeie ebenfalls Glück!« 13Micha aber antwortete: »So wahr der HERR
lebt! Nur was mein Gott mir eingeben wird, das werde ich verkünden!«
14Als er nun zum König kam, fragte dieser ihn: »Micha, sollen wir gegen
Ramoth in Gilead zu Felde ziehen, oder soll ich es unterlassen?« Er
antwortete: »Zieht hin, ihr werdet Glück haben, denn sie werden euch in
die Hand geliefert werden!« 15Da entgegnete ihm der König: »Wie oft soll
ich dich beschwören, daß du mir nichts verkündest als die reine Wahrheit
im Namen des HERRN?« 16Da sagte Micha: »Ich habe ganz Israel zerstreut
auf den Bergen gesehen wie Schafe, die keinen Hirten haben; der HERR
aber sagte: ›Diese haben keinen Herrscher mehr: ein jeder (von ihnen)
möge in Frieden nach Hause zurückkehren!‹« 17Da sagte der König von
Israel zu Josaphat: »Habe ich dir nicht gesagt, daß er mir niemals
Glück, sondern nur Unheil prophezeit?« 18Micha aber fuhr fort: »Darum
vernehmet das Wort des HERRN! Ich habe den HERRN auf seinem Throne
sitzen sehen, während das ganze himmlische Heer ihm zur Rechten und zur
Linken stand. 19Da fragte der HERR: ›Wer will Ahab, den König von
Israel, betören, daß er zu Felde ziehe und bei Ramoth in Gilead falle?‹
Da erwiderte der eine dies, der andere das, 20bis endlich der (oder ein)
Geist vortrat, sich vor den HERRN stellte und sagte: ›Ich will ihn
betören!‹ Der HERR fragte ihn: ›Auf welche Weise?‹ 21Da antwortete er:
›Ich will hingehen und zum Lügengeist im Munde aller seiner Propheten
werden.‹ Da sagte der HERR: ›Du sollst ihn betören, und es wird dir auch
gelingen! Geh hin und tu so!‹ 22Nun denn, siehe! Der HERR hat diesen
deinen Propheten einen Geist der Lüge in den Mund gelegt; denn der HERR
hat Unglück für dich beschlossen.«

\hypertarget{dd-michas-miuxdfhandlung-durch-zedekia-und-seine-gefangennahme-durch-ahab}{%
\subparagraph{dd) Michas Mißhandlung durch Zedekia und seine
Gefangennahme durch
Ahab}\label{dd-michas-miuxdfhandlung-durch-zedekia-und-seine-gefangennahme-durch-ahab}}

23Da trat Zedekia, der Sohn Kenaanas, auf Micha zu und gab ihm einen
Backenstreich mit den Worten: »Auf welchem Wege sollte denn der Geist
des HERRN von mir gewichen sein, um mit dir zu reden?« 24Micha
entgegnete: »Du wirst es an jenem Tage erfahren, an welchem du dich aus
einem Gemach in das andere begeben wirst, um dich zu verstecken!«
25Hierauf befahl der König von Israel: »Nehmt Micha und führt ihn zum
Stadthauptmann Amon und zum königlichen Prinzen Joas zurück 26und meldet
dort: ›So hat der König befohlen: Werft diesen Menschen ins Gefängnis
und erhaltet ihn notdürftig mit Brot und Wasser am Leben, bis ich
wohlbehalten heimkehre!‹« 27Micha antwortete: »Wenn du wirklich
wohlbehalten heimkehrst, so hat der HERR nicht durch mich geredet!« Er
fügte dann noch hinzu: »Hört dies, ihr Völker alle!«

\hypertarget{ee-niederlage-der-verbuxfcndeten-bei-ramoth-tod-ahabs}{%
\subparagraph{ee) Niederlage der Verbündeten bei Ramoth; Tod
Ahabs}\label{ee-niederlage-der-verbuxfcndeten-bei-ramoth-tod-ahabs}}

28Als hierauf der König von Israel und Josaphat, der König von Juda,
gegen Ramoth in Gilead zu Felde gezogen waren, 29sagte der König von
Israel zu Josaphat: »Ich will mich verkleiden und so in die Schlacht
gehen; du aber magst deine gewöhnliche Kleidung anbehalten.« So
verkleidete sich denn der König von Israel, und sie zogen in die
Schlacht. 30Der König von Syrien hatte aber den Befehlshabern seiner
Kriegswagen den bestimmten Befehl erteilt: »Ihr sollt niemand angreifen,
er sei gering oder vornehm, sondern nur den König von Israel!« 31Als nun
die Befehlshaber der Kriegswagen Josaphat zu Gesicht bekamen, dachten
sie, daß er der König von Israel sei, und wandten sich von allen Seiten
gegen ihn, um ihn anzugreifen. Da erhob Josaphat ein Geschrei, und der
HERR half ihm, indem Gott sie von ihm weglockte. 32Sobald nämlich die
Befehlshaber der Wagen erkannt hatten, daß er nicht der König von Israel
sei, wandten sie sich von ihm ab. 33Ein Mann aber spannte seinen Bogen
aufs Geratewohl und traf den König von Israel zwischen den Ringelgurt
und den Panzer. Da befahl er seinem Wagenlenker: »Wende um und bringe
mich vom Schlachtfelde weg, denn ich bin verwundet!« 34Da aber der Kampf
an jenem Tage immer heftiger entbrannte, blieb der König von Israel dann
doch den Syrern gegenüber aufrecht im Wagen stehen bis zum Abend; um die
Zeit des Sonnenuntergangs aber starb er.

\hypertarget{ff-strafrede-des-propheten-jehu-an-josaphat}{%
\subparagraph{ff) Strafrede des Propheten Jehu an
Josaphat}\label{ff-strafrede-des-propheten-jehu-an-josaphat}}

\hypertarget{section-18}{%
\section{19}\label{section-18}}

1Als dann Josaphat, der König von Juda, wohlbehalten nach Hause, nach
Jerusalem, zurückgekehrt war, 2trat der Seher Jehu, der Sohn Hananis,
vor ihn hin und sagte zu ihm: »Darfst (oder mußtest) du einem Gottlosen
Hilfe leisten, und willst du denen, die den HERRN hassen, Liebe
erweisen? Aus diesem Grunde lastet nun der Zorn des HERRN auf dir.
3Jedoch ist auch Gutes an dir gefunden worden, weil du die
Astartebilder\textless sup title=``oder: Götzensäulen''\textgreater✲ aus
dem Lande weggeschafft hast und darauf bedacht gewesen bist, Gott zu
suchen.«

\hypertarget{c-josaphats-neuordnung-der-rechtspflege}{%
\paragraph{c) Josaphats Neuordnung der
Rechtspflege}\label{c-josaphats-neuordnung-der-rechtspflege}}

4Nachdem Josaphat dann eine Zeitlang in Jerusalem geblieben war, zog er
wiederum unter dem Volk umher von Beerseba bis zum Bergland Ephraim und
führte sie zum HERRN, dem Gott ihrer Väter, zurück. 5Auch bestellte er
Richter im Lande, in allen festen Städten Judas, Stadt für Stadt, 6und
gab den Richtern die Weisung: »Seht wohl zu, was ihr tut! Denn nicht im
Auftrage von Menschen sprecht ihr Recht, sondern im Auftrage des HERRN,
der bei euch ist, wenn ihr Recht sprecht. 7Darum möge ein
Schrecken\textless sup title=``=~heilige Scheu''\textgreater✲ vor dem
HERRN in euch wohnen! Gebt auf das, was ihr tut, wohl acht! Denn bei dem
HERRN, unserm Gott, findet sich weder Ungerechtigkeit noch Ansehen der
Person und keine Bestechlichkeit.« 8Auch in Jerusalem bestellte Josaphat
einige Leviten und Priester und Familienhäupter Israels für das Gericht
des HERRN und für die Rechtsstreitigkeiten der Bewohner Jerusalems.
9Dabei gab er ihnen folgende Weisungen: »So sollt ihr verfahren in der
Furcht vor dem HERRN, mit aller Treue und mit aufrichtigem Herzen! 10In
jedem Rechtsstreit, der euch von seiten eurer Volksgenossen, die in
ihren Ortschaften wohnen, vorgelegt wird, er betreffe irgendeinen Fall
von Blutschuld oder irgendein Gesetz und Gebot oder irgendwelche
Satzungen und Rechte, sollt ihr die Leute belehren, damit sie sich nicht
gegen den HERRN vergehen und kein Zorn\textless sup title=``oder:
Strafgericht''\textgreater✲ über euch und eure Volksgenossen kommt. So
müßt ihr verfahren, damit ihr euch nicht verschuldet. 11Dabei soll der
Oberpriester Amarja für alle Angelegenheiten des HERRN und Sebadja, der
Sohn Ismaels, der Fürst des Hauses Juda, für alle königlichen
Angelegenheiten euer Vorgesetzter\textless sup title=``oder:
Vorsitzender''\textgreater✲ sein; und als Beisitzer stehen euch die
Leviten zur Verfügung. Geht mit Vertrauen ans Werk! Der HERR wird mit
den Rechtschaffenen sein.«

\hypertarget{d-josaphats-sieg-uxfcber-die-ammoniter-und-moabiter-sein-handelsvertrag-mit-ahasja}{%
\paragraph{d) Josaphats Sieg über die Ammoniter und Moabiter; sein
Handelsvertrag mit
Ahasja}\label{d-josaphats-sieg-uxfcber-die-ammoniter-und-moabiter-sein-handelsvertrag-mit-ahasja}}

\hypertarget{aa-josaphats-gebet-nach-dem-einfall-der-feinde}{%
\subparagraph{aa) Josaphats Gebet nach dem Einfall der
Feinde}\label{aa-josaphats-gebet-nach-dem-einfall-der-feinde}}

\hypertarget{section-19}{%
\section{20}\label{section-19}}

1Später begab es sich, daß die Moabiter und Ammoniter und mit ihnen ein
Teil der Mehuniter\textless sup title=``1.Chr 4,41''\textgreater✲
heranzogen, um Josaphat zu bekriegen. 2Als nun Boten kamen und dem
Josaphat die Meldung brachten: »Ein großer Heerhaufe rückt von jenseits
des (Toten) Meeres aus Syrien gegen dich heran, und sie stehen schon in
Hazezon-Thamar, das ist Engedi«: 3da erschrak Josaphat und faßte den
festen Entschluß, sich an den HERRN zu wenden; und er ließ in ganz Juda
ein Fasten ausrufen. 4Als nun die Judäer zusammengekommen waren, um den
HERRN um Hilfe anzuflehen, und sie sich auch aus allen Städten Judas
versammelt hatten, um Hilfe vom HERRN zu erbitten, 5da trat Josaphat in
der Volksgemeinde Judas und Jerusalems im Tempel des HERRN vor dem neuen
Vorhofe auf 6und betete: »HERR, du Gott unserer Väter, bist du nicht der
Gott im Himmel und du nicht der Herrscher über alle Königreiche der
Heiden? Ja, in deiner Hand ist Kraft und Stärke, und niemand vermag dir
zu widerstehen. 7Hast nicht du, unser Gott, einst die Bewohner dieses
Landes vor deinem Volk Israel vertrieben und dies Land den Nachkommen
Abrahams, deines Freundes, für ewige Zeit gegeben? 8So haben sie sich
denn darin niedergelassen und dir ein Heiligtum darin für deinen Namen
gebaut, indem sie dachten: 9›Wenn Unglück über uns hereinbricht, Krieg
als Strafgericht oder die Pest oder Hungersnot, und wir dann vor dieses
Haus und vor dich hintreten -- denn dein Name wohnt in diesem Hause! --
und wir in unserer Bedrängnis zu dir schreien, so wirst du uns erhören
und erretten.« 10Und nun, siehe, die Ammoniter und Moabiter und die
Bewohner des Gebirges Seir, deren Land du einst die Israeliten, als sie
aus Ägypten kamen, nicht hast betreten lassen, sondern an denen sie
vorübergezogen sind, ohne sie zu vernichten\textless sup title=``5.Mose
2,4-6.9.18-19''\textgreater✲: 11siehe, die wollen uns das jetzt
entgelten lassen, indem sie kommen, um uns aus deinem Eigentum zu
vertreiben, das du uns zum Besitz verliehen hast. 12O unser Gott, willst
du nicht strafend gegen sie vorgehen? Denn wir sind zu schwach gegenüber
dieser gewaltigen Übermacht, die gegen uns heranzieht, und wir wissen
nicht, was wir tun sollen, sondern auf dich sind unsere Augen
gerichtet!« 13So standen die Judäer insgesamt vor dem HERRN da samt
ihren kleinen Kindern, ihren Frauen und Söhnen.

\hypertarget{bb-gottes-antwort-siegesverheiuxdfung-des-leviten-jahasiel}{%
\subparagraph{bb) Gottes Antwort: Siegesverheißung des Leviten
Jahasiel}\label{bb-gottes-antwort-siegesverheiuxdfung-des-leviten-jahasiel}}

14Da wurde Jahasiel, der Sohn Sacharjas, des Sohnes Benajas, des Sohnes
Jehiels, des Sohnes Matthanjas, der Levit aus den Nachkommen Asaphs,
inmitten der Volksgemeinde vom Geist des HERRN ergriffen, 15so daß er
ausrief: »Merkt auf, ihr Judäer alle und ihr Bewohner Jerusalems und du,
König Josaphat! So spricht der HERR zu euch: ›Ihr braucht euch nicht zu
fürchten und nicht zu erschrecken vor diesem großen Haufen; denn nicht
eure Sache ist der Kampf, sondern die Sache Gottes! 16Zieht morgen gegen
sie hinab: sie werden dann gerade die Anhöhe Ziz heraufkommen, und ihr
werdet am Ende der Schlucht vor der Steppe Jeruel auf sie stoßen. 17Ihr
sollt aber dabei nicht selbst zu kämpfen haben; nein, nehmt nur
Aufstellung, bleibt ruhig stehen und seht die Rettung an, die der HERR
euch widerfahren läßt, Juda und Jerusalem! Fürchtet euch nicht und seid
nicht verzagt! Zieht ihnen morgen entgegen: der HERR wird mit euch
sein!‹« 18Da verneigte sich Josaphat mit dem Gesicht bis zur Erde, und
alle Judäer samt den Bewohnern Jerusalems warfen sich vor dem HERRN
nieder, um den HERRN anzubeten. 19Alsdann traten die Leviten auf, die
zum Geschlecht der Kahathiten und der Korahiten gehörten, um mit überaus
lauter Stimme Loblieder auf den HERRN, den Gott Israels, zu singen.

\hypertarget{cc-der-aufbruch-gegen-die-feinde-selbstvernichtung-der-feinde-gewaltige-beute-der-juduxe4er}{%
\subparagraph{cc) Der Aufbruch gegen die Feinde; Selbstvernichtung der
Feinde; gewaltige Beute der
Judäer}\label{cc-der-aufbruch-gegen-die-feinde-selbstvernichtung-der-feinde-gewaltige-beute-der-juduxe4er}}

20Als sie dann am andern Morgen in aller Frühe nach der Steppe Thekoa
aufbrachen, trat Josaphat bei ihrem Auszuge auf und sagte: »Hört mich
an, ihr Judäer und ihr Bewohner Jersualems! Vertraut auf den HERRN,
euren Gott, so werdet ihr gesichert sein! Vertraut auf seine Propheten,
so werdet ihr siegen!« 21Sodann traf er Verabredungen mit dem Kriegsvolk
und ließ Sänger antreten, die zu Ehren des HERRN im heiligen Schmuck
Loblieder anstimmen mußten, während sie an der Spitze der gerüsteten
Krieger einherzogen und sangen: »Danket dem HERRN, denn seine Güte
währet ewiglich!« 22Sobald sie aber mit dem Jubelruf und Lobgesang
begonnen hatten, ließ der HERR feindselige Mächte gegen die Ammoniter,
Moabiter und die Bewohner des Gebirges Seir, die gegen Juda heranzogen,
in Wirksamkeit treten, so daß sie sich selbst aufrieben. 23Denn die
Ammoniter und Moabiter erhoben sich gegen die Bewohner des Gebirges
Seir, um sie niederzumachen und zu vertilgen; und als sie mit den
Bewohnern von Seir fertig waren, half einer dem andern zur gegenseitigen
Vernichtung.

24Als nun die Judäer auf der Berghöhe ankamen, von wo man die Steppe
übersehen konnte, und sie nach dem Heerhaufen ausschauten, sahen sie nur
Leichen am Boden liegen: keiner war entronnen. 25Da kam Josaphat mit
seinem Heer heran, um zu plündern, was jene bei sich gehabt hatten, und
sie fanden bei ihnen eine Menge Vieh und Habseligkeiten, Kleider und
kostbare Sachen, und erbeuteten davon so viel, daß sie es nicht
fortschaffen konnten; drei Tage lang plünderten sie: so groß war die
Beute. 26Am vierten Tage aber versammelten sie sich im ›Lobpreistal‹;
dort nämlich lobpriesen sie den HERRN; daher führt jener Ort den Namen
›Lobpreistal‹ bis auf den heutigen Tag. 27Hierauf kehrte die ganze
Mannschaft Judas und Jerusalems wieder um, mit Josaphat an ihrer Spitze,
um voller Freude nach Jerusalem zurückzukehren; denn der HERR hatte sie
an ihren Feinden Freude erleben lassen. 28So zogen sie denn in Jerusalem
ein unter Harfen-, Zither- und Trompetenklang, zum Tempel des HERRN hin.
29Auf alle Reiche der (benachbarten) Länder aber fiel ein Schrecken
Gottes, als sie vernahmen, daß der HERR selbst mit den Feinden Israels
gestritten hatte. 30Daher verlief die weitere Regierung Josaphats in
Frieden, da sein Gott ihm Ruhe ringsumher verschafft hatte.

\hypertarget{dd-abschluuxdf-der-regierung-josaphats-und-die-quellen-seiner-geschichte}{%
\subparagraph{dd) Abschluß der Regierung Josaphats und die Quellen
seiner
Geschichte}\label{dd-abschluuxdf-der-regierung-josaphats-und-die-quellen-seiner-geschichte}}

31So regierte denn Josaphat über Juda; fünfunddreißig Jahre war er beim
Regierungsantritt alt, und fünfundzwanzig Jahre regierte er in
Jerusalem; seine Mutter hieß Asuba und war eine Tochter Silhis. 32Er
wandelte auf dem Wege seines Vaters Asa, ohne davon abzuweichen, so daß
er tat, was dem HERRN wohlgefiel. 33Nur der Höhendienst wurde nicht
beseitigt, und das Volk hatte sein Herz immer noch nicht fest auf den
Gott seiner Väter gerichtet.~-- 34Die übrige Geschichte Josaphats aber,
von Anfang bis zu Ende, findet sich bekanntlich bereits aufgezeichnet in
der Geschichte Jehus, des Sohnes Hananis, die in das Buch der Könige von
Israel aufgenommen ist.

\hypertarget{ee-josaphats-buxfcndnis-mit-ahasja-von-israel-und-bestrafung-dafuxfcr-sein-tod}{%
\subparagraph{ee) Josaphats Bündnis mit Ahasja von Israel und Bestrafung
dafür; sein
Tod}\label{ee-josaphats-buxfcndnis-mit-ahasja-von-israel-und-bestrafung-dafuxfcr-sein-tod}}

35Später verbündete sich Josaphat, der König von Juda, mit dem König
Ahasja von Israel, der in seinem Tun gottlos war; 36und zwar schloß er
einen Bund mit ihm, um Schiffe zu bauen, die nach Tharsis fahren
sollten; und sie ließen wirklich Schiffe in Ezjon-Geber bauen. 37Aber
Elieser, der Sohn Dodawas\textless sup title=``oder:
Dodias''\textgreater✲, aus Maresa, sprach gegen Josaphat die Weissagung
aus: »Weil du dich in ein Bündnis mit Ahasja eingelassen hast, wird der
HERR dein Machwerk zerstören!« So scheiterten denn die Schiffe und kamen
nicht dazu, nach Tharsis zu fahren.

\hypertarget{section-20}{%
\section{21}\label{section-20}}

1Als Josaphat sich dann zu seinen Vätern gelegt und man ihn bei seinen
Vätern in der Davidsstadt begraben hatte, folgte ihm sein Sohn Joram in
der Regierung nach.

\hypertarget{die-regierung-des-kuxf6nigs-joram}{%
\subsubsection{6. Die Regierung des Königs
Joram}\label{die-regierung-des-kuxf6nigs-joram}}

\hypertarget{a-ermordung-seiner-bruxfcder}{%
\paragraph{a) Ermordung seiner
Brüder}\label{a-ermordung-seiner-bruxfcder}}

2Joram hatte Brüder, die Söhne Josaphats, nämlich Asarja, Jehiel,
Sacharja, Asarja, Michael und Sephatja; alle diese waren Söhne des
Königs Josaphat von Juda. 3Ihr Vater hatte ihnen reiche Geschenke an
Silber, Gold und Kostbarkeiten gegeben, dazu noch feste Städte in Juda;
aber die Königswürde\textless sup title=``oder: Regierung''\textgreater✲
hatte er dem Joram verliehen, weil dieser sein erstgeborener Sohn war.
4Als nun Joram die Regierung von seinem Vater übernommen und sich auf
dem Throne befestigt hatte, ließ er alle seine Brüder und auch einige
von den höchststehenden Männern\textless sup title=``oder: obersten
Beamten''\textgreater✲ Judas hinrichten.

\hypertarget{b-gottes-stellung-zum-abfall-jorams}{%
\paragraph{b) Gottes Stellung zum Abfall
Jorams}\label{b-gottes-stellung-zum-abfall-jorams}}

5Zweiundreißig Jahre war Joram alt, als er König wurde, und acht Jahre
regierte er in Jerusalem. 6Er wandelte auf dem Wege der Könige von
Israel, wie es im Hause Ahabs durchweg der Fall war -- er hatte sich
nämlich mit einer Tochter Ahabs verheiratet --, und er tat, was dem
HERRN mißfiel. 7Aber der HERR wollte das Haus Davids nicht untergehen
lassen um des Bundes willen, den er mit David geschlossen, und weil er
ihm ja verheißen hatte, daß er ihm {[}und seinen Nachkommen{]} allezeit
eine Leuchte vor seinem Angesicht verleihen wolle.

\hypertarget{c-abfall-der-edomiter-und-der-stadt-libna}{%
\paragraph{c) Abfall der Edomiter und der Stadt
Libna}\label{c-abfall-der-edomiter-und-der-stadt-libna}}

8Unter seiner Regierung fielen die Edomiter von der Oberherrschaft Judas
ab und setzten einen eigenen König über sich ein. 9Da zog Joram mit
seinen Heerführern und allen Kriegswagen hinüber; und als er nachts
aufgebrochen war, schlug er die Edomiter, die ihn und die Befehlshaber
der Wagen umzingelt hatten. 10Aber die Edomiter machten sich von der
Oberherrschaft Judas frei (und sind unabhängig geblieben) bis auf den
heutigen Tag. Damals machte sich zu derselben Zeit auch
Libna\textless sup title=``vgl. 2.Kön 8,22''\textgreater✲ von
seiner\textless sup title=``d.h. Jorams''\textgreater✲ Oberherrschaft
los, weil er vom HERRN, dem Gott seiner Väter, abgefallen war.

\hypertarget{d-das-drohende-schreiben-des-propheten-elia-an-joram}{%
\paragraph{d) Das drohende Schreiben des Propheten Elia an
Joram}\label{d-das-drohende-schreiben-des-propheten-elia-an-joram}}

11Auch er richtete den Höhendienst auf den Bergen\textless sup
title=``oder: in den Städten''\textgreater✲ Judas ein und verleitete die
Bewohner Jerusalems zum Götzendienst und führte Juda vom rechten Wege
ab. 12Da gelangte ein Schreiben vom Propheten Elia an ihn, das so
lautete: »So hat der HERR, der Gott deines Vaters✲ David, gesprochen:
›Zur Strafe dafür, daß du nicht auf den Wegen deines Vaters Josaphat und
auf den Wegen Asas, des Königs von Juda, gewandelt bist, 13sondern nach
der Weise der Könige von Israel wandelst und Juda und die Bewohner
Jerusalems nach dem Vorgange des Hauses Ahabs zum Götzendienst verleitet
hast, außerdem auch deine Brüder, das ganze Haus deines Vaters, sie, die
doch besser waren als du, hast ermorden lassen: 14siehe, so wird der
HERR eine schwere Heimsuchung über dein Volk und deine Söhne, über deine
Frauen und deinen gesamten Besitz bringen. 15Du selbst aber wirst in
eine qualvolle Krankheit verfallen durch Erkrankung deiner Eingeweide,
bis nach Jahr und Tag deine Eingeweide infolge der Krankheit
heraustreten werden.‹«

\hypertarget{e-raubzug-der-philister-und-araber}{%
\paragraph{e) Raubzug der Philister und
Araber}\label{e-raubzug-der-philister-und-araber}}

16So erweckte denn der HERR gegen Joram die Wut der Philister und der
Araber, die neben den Kuschiten✲ wohnten, 17so daß sie gegen Juda
heranzogen, in das Land einbrachen und alles Hab und Gut, das sich im
Palast des Königs vorfand, dazu auch seine Söhne und Frauen
wegschleppten, so daß ihm kein Sohn übrigblieb als Joahas✲, der jüngste
von seinen Söhnen.

\hypertarget{f-jorams-qualvolles-ende-und-unehrenvolles-begruxe4bnis}{%
\paragraph{f) Jorams qualvolles Ende und unehrenvolles
Begräbnis}\label{f-jorams-qualvolles-ende-und-unehrenvolles-begruxe4bnis}}

18Nach allem diesem aber suchte ihn der HERR mit einer unheilbaren
Krankheit in seinen Eingeweiden heim. 19Das dauerte ununterbrochen lange
Tage an, bis ihm schließlich am Ende des zweiten Jahres die Eingeweide
infolge der Krankheit heraustraten und er unter schlimmen Schmerzen
starb. Sein Volk aber veranstaltete ihm zu Ehren keinen solchen
Leichenbrand, wie seine Väter ihn erhalten hatten. 20Im Alter von
zweiunddreißig Jahren war er König geworden, und acht Jahre hatte er in
Jerusalem regiert; er ging dahin, von niemand zurückersehnt, und man
begrub ihn in der Davidsstadt, aber nicht in den Gräbern der Könige.

\hypertarget{die-regierung-des-kuxf6nigs-ahasja}{%
\subsubsection{7. Die Regierung des Königs
Ahasja}\label{die-regierung-des-kuxf6nigs-ahasja}}

\hypertarget{a-seine-gott-miuxdffuxe4llige-regierung}{%
\paragraph{a) Seine Gott mißfällige
Regierung}\label{a-seine-gott-miuxdffuxe4llige-regierung}}

\hypertarget{section-21}{%
\section{22}\label{section-21}}

1Darauf machten die Bewohner Jerusalems seinen jüngsten Sohn
Ahasja\textless sup title=``21,17 Joahas''\textgreater✲ an seiner Statt
zum König; denn alle älteren Söhne hatte die Streifschar ermordet, die
mit den Arabern in das Lager eingedrungen war, und so wurde Ahasja
König, der Sohn des Königs Joram von Juda. 2Im Alter von zweiundzwanzig
Jahren kam er auf den Thron und regierte ein Jahr in Jerusalem; seine
Mutter hieß Athalja und war die Enkelin Omris. 3Auch er wandelte auf den
Wegen des Hauses Ahabs; denn seine Mutter war für ihn eine Beraterin zu
gottlosem Handeln. 4So tat er denn, was dem HERRN mißfiel, wie das Haus
Ahabs; denn dessen Angehörige waren nach seines Vaters Tode seine
Ratgeber, zum Unheil für ihn.

\hypertarget{b-sein-buxfcndnis-mit-joram-von-israel-und-sein-tod-durch-jehu}{%
\paragraph{b) Sein Bündnis mit Joram von Israel und sein Tod durch
Jehu}\label{b-sein-buxfcndnis-mit-joram-von-israel-und-sein-tod-durch-jehu}}

5Auf ihren Rat hin unternahm er auch mit Joram, dem Sohne des Königs
Ahab von Israel, einen\textless sup title=``oder: den''\textgreater✲
Feldzug gegen den König Hasael von Syrien nach Ramoth in Gilead. Als
aber die Syrer\textless sup title=``oder: die Schützen''\textgreater✲
Joram verwundet hatten, 6kehrte dieser zurück, um sich in Jesreel von
den Wunden heilen zu lassen, die man ihm bei Rama beigebracht hatte, als
er gegen Hasael, den König von Syrien, zu Felde gezogen war. Darauf kam
Ahasja, der König von Juda, der Sohn Jorams (von Juda), um Joram, den
Sohn Ahabs, in Jesreel zu besuchen, weil dieser dort krank lag. 7Das war
aber von Gott zum Untergang Ahasjas verhängt, daß er sich zu Joram
begab. Denn als er nach seiner Ankunft dort mit Joram gegen Jehu, den
Sohn Nimsis, zog, den der HERR hatte salben lassen, damit er das Haus
Ahabs ausrotte, 8da begab es sich, als Jehu das Strafgericht am Hause
Ahabs vollzog, daß er die Fürsten\textless sup title=``oder: obersten
Beamten''\textgreater✲ von Juda und die Neffen Ahasjas, die in Ahasjas
Diensten standen, antraf und sie niederhauen ließ. 9Als er dann auch
nach Ahasja suchen ließ, der sich in Samaria versteckt hielt, und man
ihn nach seiner Festnahme zu Jehu brachte, ließ dieser ihn töten. Doch
begrub man ihn alsdann, weil man bedachte, daß er ein Sohn✲ Josaphats
sei, der sich zum HERRN mit ganzem Herzen gehalten hatte. Aber in der
Familie Ahasjas war niemand mehr vorhanden, der fähig gewesen wäre, den
Thron zu besteigen.

\hypertarget{regierung-sturz-und-tod-der-kuxf6nigin-athalja}{%
\subsubsection{8. Regierung, Sturz und Tod der Königin
Athalja}\label{regierung-sturz-und-tod-der-kuxf6nigin-athalja}}

\hypertarget{a-athaljas-thronraub-und-mordtaten-rettung-des-joas}{%
\paragraph{a) Athaljas Thronraub und Mordtaten; Rettung des
Joas}\label{a-athaljas-thronraub-und-mordtaten-rettung-des-joas}}

10Als aber Athalja, die Mutter Ahasjas, erfuhr, daß ihr Sohn tot war,
machte sie sich daran, alle, die zur königlichen Familie des Hauses Juda
gehörten, umzubringen. 11Aber Josabath, die Tochter des Königs Joram,
nahm Joas, den Sohn Ahasjas, und schaffte ihn aus der Mitte der
Königssöhne, die getötet werden sollten, heimlich beiseite, indem sie
ihn samt seiner Amme in die Bettzeugkammer brachte. So verbarg ihn
Josabath, die Tochter des Königs Joram, die Frau des Priesters Jojada --
sie war nämlich die Schwester Ahasjas -- vor Athalja, so daß diese ihn
nicht ermorden konnte. 12Er blieb dann sechs Jahre lang bei ihnen im
Hause Gottes versteckt, während Athalja das Land regierte.

\hypertarget{b-die-verschwuxf6rung-jojadas}{%
\paragraph{b) Die Verschwörung
Jojadas}\label{b-die-verschwuxf6rung-jojadas}}

\hypertarget{section-22}{%
\section{23}\label{section-22}}

1Im siebten Jahre aber faßte Jojada einen kühnen Entschluß und setzte
sich mit den Befehlshabern der Hundertschaften, nämlich mit Asarja, dem
Sohne Jerohams, Ismael, dem Sohne Johanans, Asarja, dem Sohne Obeds,
Maaseja, dem Sohne Adajas, und Elisaphat, dem Sohne Sichris, ins
Einvernehmen. 2Die zogen dann in Juda umher, brachten die Leviten in
allen Städten Judas und die israelitischen Familienhäupter auf ihre
Seite, so daß sie nach Jerusalem kamen, 3wo die ganze Versammlung im
Hause Gottes einen Bund mit dem (jungen) Könige schloß. Dabei richtete
(Jojada) folgende Worte an sie: »Hier der Königssohn soll König sein,
wie der HERR es betreffs der Nachkommen Davids verheißen\textless sup
title=``oder: bestimmt''\textgreater✲ hat. 4Folgendermaßen müßt ihr nun
zu Werke gehen: das eine Drittel von euch, den Priestern und Leviten,
die ihr am Sabbat (aus dem Tempel) abzieht, soll den Dienst als Torhüter
an den Schwellen versehen; 5ein anderes Drittel soll den königlichen
Palast, das letzte Drittel das Tor Jesod, das gesamte Kriegsvolk endlich
die Vorhöfe des Tempels des HERRN besetzen. 6Den Tempel des HERRN aber
darf niemand betreten außer den Priestern und den diensttuenden Leviten:
diese dürfen hineingehen, denn sie sind geheiligt; das gesamte übrige
Volk aber soll die Vorschrift des HERRN beobachten. 7Die Leviten sollen
sich dann rings um den König scharen, ein jeder mit seinen Waffen in der
Hand, und wer in den Tempel eindringt, soll getötet werden; ihr müßt
beständig um den König sein, wenn er (aus dem Tempel) auszieht und wenn
er (in den Palast) einzieht!«

\hypertarget{c-athaljas-gefangennahme-und-ermordung-erhebung-des-joas-zum-kuxf6nig}{%
\paragraph{c) Athaljas Gefangennahme und Ermordung; Erhebung des Joas
zum
König}\label{c-athaljas-gefangennahme-und-ermordung-erhebung-des-joas-zum-kuxf6nig}}

8Die Leviten und alle Judäer verfuhren dann genau nach der Anweisung des
Priesters Jojada: jeder nahm seine Leute zu sich, sowohl die, welche am
Sabbat abtraten, als auch die, welche am Sabbat antraten; denn der
Priester Jojada hatte die (dienstfreien) Abteilungen nicht entlassen.
9Der Priester Jojada gab dann den Hauptleuten der Hundertschaften die
Speere, Tartschen\textless sup title=``d.h. Kleinschilde''\textgreater✲
und Großschilde, die dem König David gehört hatten und die sich im
Tempel Gottes befanden. 10Nachdem er dann das ganze Kriegsvolk, und zwar
einen jeden mit der Lanze in der Hand, von der Südseite des Tempels bis
zur Nordseite des Tempels, bis an den Altar und wieder bis an den Tempel
aufgestellt hatte, 11führten sie den Königssohn heraus, legten ihm den
Stirnreif\textless sup title=``oder: die Königsbinde''\textgreater✲ um
und gaben ihm die Gesetzesrolle in die Hand. So machten sie ihn zum
König; und Jojada samt seinen Söhnen salbten ihn und riefen: »Es lebe
der König!«

12Als nun Athalja das Geschrei des Volkes hörte, das herbeigeströmt war
und dem Könige zujubelte, begab sie sich zu dem Volk in den Tempel des
HERRN. 13Dort sah sie den König auf seinem Standort am Eingang stehen
und die Hauptleute und die Trompeter neben dem König, während das
gesamte Volk des Landes voller Freude war und in die Trompeten stieß und
die Sänger mit den Musikinstrumenten da waren und das Zeichen zum
Lobpreis gaben. Da zerriß Athalja ihre Kleider und rief: »Verrat,
Verrat!« 14Aber der Priester Jojada ließ die Hauptleute der
Hundertschaften, die Befehlshaber des Heeres, vortreten und gab ihnen
den Befehl: »Führt sie hinaus zwischen den Reihen hindurch! Und wer ihr
folgt, soll mit dem Schwert niedergehauen werden!« Der Priester hatte
nämlich befohlen: »Ihr dürft sie nicht im Tempel des HERRN töten!« 15Da
legte man Hand an sie, und als sie am Eingang des Roßtores am
königlichen Palast angelangt war, tötete man sie dort.

\hypertarget{d-jojadas-mauxdfnahmen-zur-ehre-gottes-kruxf6nung-des-joas}{%
\paragraph{d) Jojadas Maßnahmen zur Ehre Gottes; Krönung des
Joas}\label{d-jojadas-mauxdfnahmen-zur-ehre-gottes-kruxf6nung-des-joas}}

16Darauf schloß Jojada zwischen dem HERRN und dem gesamten Volk und dem
König das feierliche Abkommen, daß sie das Volk des HERRN sein wollten.
17Darauf zog das ganze Volk zum Baalstempel und riß ihn nieder: seine
Altäre und Götterbilder zertrümmerten sie vollständig und erschlugen den
Baalspriester Matthan vor den Altären. 18Alsdann besetzte Jojada die
Ämter am Tempel des HERRN mit den Priestern und den Leviten, die David
für den Tempeldienst in Klassen abgeteilt hatte, damit sie dem HERRN die
Brandopfer, wie es im mosaischen Gesetz vorgeschrieben ist, unter
Freudenrufen und mit Gesängen nach der Anordnung Davids darbrächten.
19Auch stellte er die Torhüter an den Toren des Hauses des HERRN auf,
damit kein irgendwie Unreiner hereinkäme. 20Dann nahm er die Hauptleute
sowie die Vornehmen und die Männer, die eine leitende Stellung im Volke
einnahmen, aber auch das gewöhnliche Volk des Landes mit sich und führte
den König aus dem Tempel des HERRN hinab, und als sie durch das obere
Tor in das königliche Schloß gezogen waren, setzten sie den König auf
den königlichen Thron. 21Da war alles Volk im Lande voller Freude, und
die Stadt blieb ruhig; Athalja aber hatte man mit dem Schwert getötet.

\hypertarget{die-regierung-des-kuxf6nigs-joas}{%
\subsubsection{9. Die Regierung des Königs
Joas}\label{die-regierung-des-kuxf6nigs-joas}}

\hypertarget{a-eingangswort}{%
\paragraph{a) Eingangswort}\label{a-eingangswort}}

\hypertarget{section-23}{%
\section{24}\label{section-23}}

1Joas war beim Regierungsantritt sieben Jahre alt und regierte vierzig
Jahre in Jerusalem; seine Mutter hieß Zibja und stammte aus Beerseba.
2Joas tat, was dem HERRN wohlgefiel, solange der Priester Jojada lebte.
3Dieser verheiratete ihn mit zwei Frauen, und er wurde Vater von Söhnen
und Töchtern.

\hypertarget{b-ausbesserung-des-tempels-verordnung-uxfcber-die-verwaltung-und-verwendung-der-einkommenden-tempelgelder}{%
\paragraph{b) Ausbesserung des Tempels; Verordnung über die Verwaltung
und Verwendung der einkommenden
Tempelgelder}\label{b-ausbesserung-des-tempels-verordnung-uxfcber-die-verwaltung-und-verwendung-der-einkommenden-tempelgelder}}

4Späterhin entschloß sich Joas, den Tempel des HERRN wiederherzustellen.
5Er ließ also die Priester und Leviten zusammenkommen und befahl ihnen:
»Begebt euch in die Ortschaften Judas und sammelt Geld von allen
Israeliten ein, um das Haus eures Gottes Jahr für Jahr auszubessern, und
zwar müßt ihr die Sache beschleunigen!« Doch die Leviten hatten es damit
nicht eilig. 6Da ließ der König den Oberpriester Jojada kommen und sagte
zu ihm: »Warum hast du die Leviten nicht dazu angehalten, aus Juda und
Jerusalem die Abgabe einzubringen, die Mose, der Knecht des HERRN, der
israelitischen Volksgemeinde für das Gesetzeszelt auferlegt hat? 7Denn
die ruchlose Athalja und ihre Söhne haben das Gotteshaus verfallen
lassen und haben auch alle heiligen Gegenstände, die zum Hause des HERRN
gehörten, für die Baale verwandt!« 8So fertigte man denn auf Befehl des
Königs eine Lade an und stellte sie draußen vor dem Tempeltor auf.
9Sodann ließ man in Juda und Jerusalem öffentlich bekanntmachen, man
solle dem HERRN die Abgabe entrichten, die Mose, der Knecht Gottes, den
Israeliten in der Wüste auferlegt hatte. 10Da freuten sich alle
Vornehmen und das ganze Volk und brachten ihren Beitrag und warfen ihn
in die Lade, bis sie voll war. 11Sooft man nun die Lade durch die
Leviten zu der königlichen Aufsichtsbehörde bringen ließ -- wenn man
nämlich sah, daß viel Geld darin war --, so gingen der Schreiber des
Königs und der vom Oberpriester Beauftragte hin, leerten die Lade und
brachten sie dann wieder an ihren Platz zurück. So tat man Tag für Tag
und brachte eine Menge Geld zusammen. 12Der König und Jojada übergaben
dieses dann denen, die als Werkmeister am Tempel des HERRN tätig waren,
und diese stellten Steinmetzen und Zimmerleute an, um den Tempel des
HERRN wiederherzustellen, dazu auch Eisen- und Kupferschmiede für die
Ausbesserung des Tempels. 13So waren denn die Werkmeister tätig, und
unter ihrer Hand nahm die Ausführung der Arbeiten\textless sup
title=``=~die Ausbesserung des Baues''\textgreater✲ guten Fortgang, so
daß sie das Gotteshaus in allen seinen Teilen wiederherstellten und in
guten Stand setzten. 14Als sie damit fertig waren, lieferten sie das
noch übrige Geld an den König und an Jojada ab, und man ließ davon
Geräte für den Tempel des HERRN anfertigen, Geräte für den Gottesdienst
und für die Opfer, nämlich Schalen und andere goldene und silberne
Geräte. Man brachte aber, solange Jojada lebte, Brandopfer regelmäßig im
Tempel des HERRN dar.

\hypertarget{c-abfall-des-joas-von-gott-nach-jojadas-tode-sacharjas-strafrede-und-seine-steinigung}{%
\paragraph{c) Abfall des Joas von Gott nach Jojadas Tode; Sacharjas
Strafrede und seine
Steinigung}\label{c-abfall-des-joas-von-gott-nach-jojadas-tode-sacharjas-strafrede-und-seine-steinigung}}

15Als Jojada aber alt und lebenssatt geworden war, starb er; er war bei
seinem Tode hundertunddreißig Jahre alt. 16Man begrub ihn in der
Davidsstadt bei den Königen, weil er sich um Israel und auch um Gott und
seinen Tempel verdient gemacht hatte.

17Nach dem Tode Jojadas aber kamen die Fürsten Judas und warfen sich vor
dem Könige nieder; da schenkte dieser ihnen Gehör. 18So verließen sie
denn das Haus des HERRN, des Gottes ihrer Väter, und verehrten die
Standbilder der Aschera✲ und die Götzenbilder. Da brach der
Zorn\textless sup title=``oder: ein Zorngericht''\textgreater✲ Gottes
über Juda und Jerusalem infolge dieser ihrer Verschuldung herein. 19Er
sandte Propheten unter sie, um sie zum HERRN zurückzuführen; diese
redeten ihnen ins Gewissen, doch sie schenkten ihnen kein Gehör. 20Da
kam der Geist Gottes über Sacharja, den Sohn des Priesters Jojada, so
daß er vor das Volk hintrat und zu ihnen sagte: »So hat Gott gesprochen:
›Warum übertretet ihr die Gebote des HERRN? Ihr bringt euch dadurch nur
um euer Glück! Weil ihr den HERRN verlassen habt, hat er euch auch
verlassen!‹« 21Aber sie stifteten eine Verschwörung gegen ihn an und
steinigten ihn auf Befehl des Königs im Vorhof des Tempels des HERRN.
22So wenig gedachte der König Joas der Liebe, die Jojada, der Vater
jenes (Sacharja), ihm erwiesen hatte, daß er der Mörder seines Sohnes
wurde. Dieser aber rief sterbend aus: »Der HERR möge es sehen und wird
es rächen!«

\hypertarget{d-ungluxfccklicher-krieg-mit-den-syrern-ermordung-des-kuxf6nigs-durch-verschwuxf6rer-schluuxdfwort}{%
\paragraph{d) Unglücklicher Krieg mit den Syrern; Ermordung des Königs
durch Verschwörer;
Schlußwort}\label{d-ungluxfccklicher-krieg-mit-den-syrern-ermordung-des-kuxf6nigs-durch-verschwuxf6rer-schluuxdfwort}}

23Und noch vor Ablauf des Jahres zog das Heer der Syrer gegen ihn heran.
Als diese in Juda und Jerusalem eingedrungen waren, ließen sie alle
Obersten\textless sup title=``oder: Fürsten''\textgreater✲ des Volkes
grausam hinrichten und sandten alle Beute, die sie bei ihnen gemacht
hatten, an den König von Damaskus. 24Obgleich nämlich das syrische Heer
bei seinem Einfall verhältnismäßig klein war, hatte der HERR sie doch
über ein sehr zahlreiches Heer der Judäer siegen lassen, weil diese den
HERRN, den Gott ihrer Väter, verlassen hatten; so vollzogen die Syrer
das Strafgericht an Joas. 25Als sie dann von ihm weggezogen waren -- sie
ließen ihn nämlich schwer erkrankt zurück --, verschworen sich seine
Hofleute gegen ihn wegen seiner Bluttat am Sohne des Priesters Jojada
und ermordeten ihn auf seinem Bette. So fand er seinen Tod; und man
begrub ihn in der Davidsstadt, jedoch nicht in den Gräbern der Könige.
26Folgende (Hofleute) hatten aber die Verschwörung gegen ihn
angestiftet: Sabad, der Sohn der Ammonitin Simeath, und Josabad, der
Sohn der Moabitin Simrith.

27Was aber seine Söhne betrifft und die vielen Prophetensprüche gegen
ihn und den Neubau des Hauses Gottes: das alles findet sich bereits
aufgezeichnet in der erbaulichen\textless sup title=``oder:
ausführlichen''\textgreater✲ Auslegung des Buches der
Könige\textless sup title=``vgl. 13,22''\textgreater✲. Sein Sohn Amazja
folgte ihm dann in der Regierung nach.

\hypertarget{die-regierung-des-kuxf6nigs-amazja}{%
\subsubsection{10. Die Regierung des Königs
Amazja}\label{die-regierung-des-kuxf6nigs-amazja}}

\hypertarget{a-guter-regierungsanfang}{%
\paragraph{a) Guter Regierungsanfang}\label{a-guter-regierungsanfang}}

\hypertarget{section-24}{%
\section{25}\label{section-24}}

1Im Alter von fünfundzwanzig Jahren kam Amazja auf den Thron, und
neunundzwanzig Jahre regierte er in Jerusalem; seine Mutter hieß Joaddan
und stammte aus Jerusalem. 2Er tat, was dem HERRN wohlgefiel, jedoch
nicht mit ungeteiltem Herzen. 3Sobald er nun die Herrschaft fest in
Händen hatte, ließ er von seinen Dienern diejenigen hinrichten, die den
König, seinen Vater, ermordet hatten. 4Aber ihre Söhne\textless sup
title=``oder: Kinder''\textgreater✲ ließ er nicht hinrichten, sondern
verfuhr so, wie im Gesetz, im Buch Moses, geschrieben steht\textless sup
title=``5.Mose 25,16''\textgreater✲, wo der HERR ausdrücklich geboten
hat: »Väter sollen nicht wegen einer Verschuldung ihrer Kinder sterben,
und Kinder sollen nicht wegen einer Verschuldung ihrer Väter sterben,
sondern ein jeder soll nur wegen seines eigenen Vergehens sterben!«

\hypertarget{b-amazjas-sieg-uxfcber-die-edomiter-nach-ruxfccksendung-der-israelitischen-suxf6ldner-die-rache-dieser-truppen}{%
\paragraph{b) Amazjas Sieg über die Edomiter nach Rücksendung der
israelitischen Söldner; die Rache dieser
Truppen}\label{b-amazjas-sieg-uxfcber-die-edomiter-nach-ruxfccksendung-der-israelitischen-suxf6ldner-die-rache-dieser-truppen}}

5Hierauf bot Amazja die Judäer auf und ließ sie, nach Familien geordnet,
unter den Befehlshabern über die Tausendschaften und unter den
Befehlshabern über die Hundertschaften antreten, ganz Juda und Benjamin.
Als er sie dann von den Zwanzigjährigen an und darüber
aufschreiben\textless sup title=``oder: mustern''\textgreater✲ ließ,
stellte er für sie eine Gesamtzahl von 300000 auserlesenen,
felddienstfähigen Kriegern fest, die Speer und Schild✲ führten. 6Dazu
nahm er aus Israel 100000 kriegstüchtige Leute für hundert Talente
Silber in Sold. 7Da kam aber ein Gottesmann zu ihm und sagte: »O König!
Laß die israelitische Mannschaft nicht mit dir ziehen! Denn der HERR ist
nicht mit Israel, mit allen diesen Leuten aus Ephraim. 8Sondern ziehe du
(allein) und entschlossen in den Kampf: Gott wird dich sonst vor dem
Feinde zu Fall bringen; denn Gott ist stark genug, sowohl den Sieg zu
verleihen als auch zu Fall zu bringen.« 9Als nun Amazja den Gottesmann
fragte: »Was soll dann aber aus den hundert Talenten werden, die ich der
israelitischen Heerschar bereits gegeben habe?«, antwortete der
Gottesmann: »Der HERR vermag dir mehr als das zu geben!« 10Da sonderte
Amazja die Abteilung, die aus Ephraim zu ihm gekommen war, von seinem
Heere ab, damit sie wieder in ihre Heimat zögen. Da gerieten diese aber
in heftigen Zorn gegen die Judäer und kehrten in glühendem Zorn in ihre
Heimat zurück. 11Amazja aber führte sein Heer voller Mut ins Feld und
zog in das Salztal, wo er zehntausend Seiriten erschlug; 12zehntausend
andere aber, welche die Judäer lebendig gefangen genommen hatten,
führten sie auf eine Felsenspitze und stürzten sie von der Felsenspitze
hinab, so daß sie allesamt zerschmettert wurden. 13Aber die Leute der
Heeresabteilung, die Amazja zurückgeschickt hatte, so daß sie an dem
Feldzuge nicht hatten teilnehmen dürfen, waren in die Ortschaften Judas
von Samaria bis Beth-Horon eingefallen, hatten dreitausend Menschen in
ihnen erschlagen und reiche Beute gemacht.

\hypertarget{c-amazjas-abfall-von-gott-warnung-durch-einen-propheten}{%
\paragraph{c) Amazjas Abfall von Gott; Warnung durch einen
Propheten}\label{c-amazjas-abfall-von-gott-warnung-durch-einen-propheten}}

14Als aber Amazja nach dem Siege über die Edomiter heimgekehrt war,
stellte er die Götterbilder der Seiriten, die er mitgebracht hatte, als
Götter für sich auf, betete sie an und brachte ihnen Opfer dar. 15Da
entbrannte der Zorn des HERRN gegen Amazja, und er sandte einen
Propheten zu ihm, der zu ihm sagte: »Warum hältst du dich an die Götter
jenes Volkes, die doch ihr eigenes Volk nicht vor deiner Hand haben
retten können?« 16Als er aber so zu ihm redete, entgegnete ihm Amazja:
»Haben wir dich etwa zum Ratgeber des Königs bestellt? Unterlaß das!
Warum willst du geschlagen werden?« Da hörte der Prophet auf, sagte aber
noch: »Ich merke wohl, daß Gott dein Verderben beschlossen hat, weil du
so gehandelt und auf meinen Rat nicht hast hören wollen!«

\hypertarget{d-amazjas-ungluxfccklicher-krieg-mit-joas-von-israel}{%
\paragraph{d) Amazjas unglücklicher Krieg mit Joas von
Israel}\label{d-amazjas-ungluxfccklicher-krieg-mit-joas-von-israel}}

17Nachdem hierauf Amazja, der König von Juda, mit sich zu Rate gegangen
war, schickte er Gesandte an den König Joas von Israel, den Sohn des
Joahas, des Sohnes Jehus, und ließ ihm sagen: »Komm, wir wollen unsere
Kräfte miteinander messen!« 18Da ließ Joas, der König von Israel, dem
König Amazja von Juda durch eine Gesandtschaft antworten: »Der
Dornstrauch auf dem Libanon sandte zur Zeder auf dem Libanon und ließ
ihr sagen: ›Gib deine Tochter meinem Sohne zur Frau!‹ Aber da liefen die
wilden Tiere auf dem Libanon über den Dornstrauch hin und zertraten ihn.
19Du denkst, du habest ja die Edomiter geschlagen, und da reißt dein
Mut\textless sup title=``oder: Übermut''\textgreater✲ dich fort, dir
noch mehr Ruhm zu erwerben. Bleibe nur (ruhig) zu Hause: warum willst du
das Unglück herausfordern, daß du zu Fall kommst und Juda mit dir?«
20Aber Amazja wollte sich nicht warnen lassen; denn es war von Gott so
gefügt, damit er sie\textless sup title=``d.h. die Judäer''\textgreater✲
den Feinden preisgäbe, weil sie sich den Göttern der Edomiter zugewandt
hatten. 21So zog denn Joas, der König von Israel, heran, und beide maßen
ihre Kräfte miteinander, er und Amazja, der König von Juda, bei
Beth-Semes, das zu Juda gehört. 22Da wurden die Judäer von den
Israeliten geschlagen, und ein jeder floh in seine Heimat. 23Amazja
aber, den König von Juda, den Sohn des Joas, des Sohnes des Joahas, nahm
Joas, der König von Israel, in Beth-Semes gefangen und ließ, als er ihn
nach Jerusalem gebracht hatte, ein Stück der Mauer Jerusalems vom
Ephraimstor bis zum Ecktor auf einer Strecke von vierhundert Ellen
niederreißen. 24Außerdem nahm er alles Gold und Silber sowie alle
Geräte, die sich im Hause Gottes bei Obed-Edom vorfanden, und die
Schätze des königlichen Palastes, dazu auch Geiseln, und kehrte nach
Samaria zurück.

\hypertarget{e-schluuxdfwort-ermordung-des-kuxf6nigs-durch-verschwuxf6rer}{%
\paragraph{e) Schlußwort; Ermordung des Königs durch
Verschwörer}\label{e-schluuxdfwort-ermordung-des-kuxf6nigs-durch-verschwuxf6rer}}

25Amazja, der Sohn des Joas, der König von Juda, überlebte dann den
König Joas von Israel, den Sohn des Joahas, noch fünfzehn Jahre.~--
26Die übrige Geschichte Amazjas aber, von Anfang bis zu Ende, findet
sich bekanntlich bereits aufgezeichnet im Buch der Könige von Juda und
Israel. 27Seit der Zeit, da Amazja treulos vom HERRN abgefallen war,
bestand in Jerusalem eine Verschwörung gegen ihn. Er floh nach Lachis,
doch man sandte Leute nach Lachis hinter ihm her und ließ ihn dort
ermorden. 28Dann lud man ihn auf Rosse und begrub ihn bei seinen Vätern
in der Davidsstadt.

\hypertarget{die-regierung-des-kuxf6nigs-ussia}{%
\subsubsection{11. Die Regierung des Königs
Ussia}\label{die-regierung-des-kuxf6nigs-ussia}}

\hypertarget{a-guter-regierungsanfang-ussias-gluxfcck-im-krieg-und-im-frieden}{%
\paragraph{a) Guter Regierungsanfang; Ussias Glück im Krieg und im
Frieden}\label{a-guter-regierungsanfang-ussias-gluxfcck-im-krieg-und-im-frieden}}

\hypertarget{section-25}{%
\section{26}\label{section-25}}

1Hierauf nahm das ganze Volk von Juda den Ussia, obgleich er erst
sechzehn Jahre alt war, und machte ihn zum König als Nachfolger seines
Vaters Amazja. 2Er befestigte Eloth\textless sup title=``vgl.
8,17''\textgreater✲, das er an Juda zurückgebracht hatte, nachdem der
König sich zu seinen Vätern gelegt hatte. 3Im Alter von sechzehn Jahren
bestieg Ussia den Thron, und zweiundfünfzig Jahre regierte er in
Jerusalem; seine Mutter hieß Jecholja und stammte aus Jerusalem. 4Er
tat, was dem HERRN wohlgefiel, ganz wie sein Vater Amazja getan hatte.
5Er war darauf bedacht, sich an Gott zu halten, solange Sacharja lebte,
der ihn zur Furcht Gottes anleitete; und solange er den HERRN suchte,
gab Gott ihm Glück\textless sup title=``oder: Gelingen''\textgreater✲.
6Denn als er einen Feldzug gegen die Philister unternahm, riß er die
Mauern der Städte Gath, Jabne und Asdod nieder und legte feste Plätze um
Asdod her und im (übrigen) Philisterlande an. 7So half Gott ihm im Kampf
gegen die Philister und ebenso gegen die Araber, die in Gur-Baal
wohnten, und gegen die Mehuniter✲. 8Auch die Ammoniter mußten Ussia
Abgaben✲ entrichten, und sein Ruhm verbreitete sich bis nach Ägypten
hin; denn er war überaus mächtig geworden. 9Auch baute Ussia in
Jerusalem Türme am Ecktor, am Taltor und am Winkel und befestigte sie.
10Weiter ließ er in der Steppe Türme erbauen und zahlreiche Zisternen
anlegen; denn er besaß große Viehherden sowohl in der Niederung als auch
in der Ebene, dazu Ackerleute und Weingärtner im Gebirge und im Gefilde;
denn er war ein Freund der Landwirtschaft.

\hypertarget{b-ussias-sorge-fuxfcr-ein-tuxfcchtiges-heer-und-fuxfcr-die-sicherung-des-landes}{%
\paragraph{b) Ussias Sorge für ein tüchtiges Heer und für die Sicherung
des
Landes}\label{b-ussias-sorge-fuxfcr-ein-tuxfcchtiges-heer-und-fuxfcr-die-sicherung-des-landes}}

11Ussia hatte aber auch ein kriegstüchtiges Heer, das, in Scharen
gegliedert, zu Felde zog, soviele ihrer durch den Schreiber Jehiel und
den Amtmann Maaseja unter der Aufsicht Hananjas, eines der Heerführer
des Königs, gemustert\textless sup title=``oder: zum Kriegsdienst
ausgehoben''\textgreater✲ worden waren. 12Die Gesamtzahl der
Familienhäupter der kriegstüchtigen Mannschaft betrug 2600, 13unter
deren Befehl eine Heeresmacht von 307500 kriegstüchtigen, vollkräftigen
Männern stand, die dem König gegen die Feinde zu Gebote standen. 14Für
dieses ganze Heer beschaffte Ussia Schilde, Lanzen, Helme, Panzer, Bogen
und Schleudersteine. 15Auch ließ er in Jerusalem kunstvoll gebaute
Maschinen anfertigen, die auf den Türmen und Zinnen aufgestellt werden
sollten, um mit ihnen Pfeile und große Steine zu schleudern. So drang
sein Ruhm in weite Ferne; denn er erlangte wunderbare Erfolge, bis er
(überaus) mächtig geworden war.

\hypertarget{c-ussias-uxfcbergriff-in-das-priesteramt-wird-von-gott-mit-aussatz-bestraft}{%
\paragraph{c) Ussias Übergriff in das Priesteramt wird von Gott mit
Aussatz
bestraft}\label{c-ussias-uxfcbergriff-in-das-priesteramt-wird-von-gott-mit-aussatz-bestraft}}

16Als er aber zu Macht gelangt war, überhob sich sein Sinn zu gottlosem
Handeln, so daß er sich gegen den HERRN, seinen Gott, versündigte, indem
er in den Tempel des HERRN ging, um auf dem Räucheraltar Rauchopfer
darzubringen. 17Da kam der Priester Asarja hinter ihm her, begleitet von
achtzig Priestern des HERRN, vortrefflichen Männern; 18die traten dem
König Ussia mit den Worten entgegen: »Nicht dir, Ussia, steht das Recht
zu, dem HERRN Rauchopfer darzubringen, sondern nur den Priestern, den
Nachkommen Aarons, die zu diesem Dienst geweiht sind. Verlaß das
Heiligtum, denn du hast dich vergangen, und das bringt dir vor Gott, dem
HERRN, keine Ehre!« 19Da geriet Ussia in Zorn, während er noch das
Räucherfaß in der Hand hielt, um zu räuchern; als er aber seinen Zorn
gegen die Priester ausließ, kam der Aussatz an seiner Stirn vor den
Augen der Priester im Tempel des HERRN neben dem Räucheraltar zum
Ausbruch. 20Als sich nun der Oberpriester Asarja und alle Priester zu
ihm hinwandten, da war er in der Tat an der Stirn vom Aussatz befallen.
Da brachten sie ihn schleunigst von dort weg, und auch er selbst beeilte
sich hinauszukommen, weil der HERR ihn geschlagen\textless sup
title=``=~schwer heimgesucht''\textgreater✲ hatte. 21So war denn der
König Ussia aussätzig bis zu seinem Todestage und wohnte als Aussätziger
in einem Hause für sich; denn er war vom Tempel des HERRN
ausgeschlossen. Jotham aber, sein Sohn, waltete im königlichen Hause als
Familienhaupt und versah die Regierungsgeschäfte für das Land.

\hypertarget{d-ussias-tod-und-begruxe4bnis}{%
\paragraph{d) Ussias Tod und
Begräbnis}\label{d-ussias-tod-und-begruxe4bnis}}

22Die übrige Geschichte Ussias aber, von Anfang bis zu Ende\textless sup
title=``=~die frühere wie die spätere''\textgreater✲, hat der Prophet
Jesaja, der Sohn des Amoz, geschrieben. 23Als Ussia sich dann zu seinen
Vätern gelegt hatte, begrub man ihn {[}bei seinen Vätern{]} auf dem
(freien) Felde bei der Begräbnisstätte der Könige; denn man sagte: »Er
ist aussätzig!« Sein Sohn Jotham folgte ihm in der Regierung nach.

\hypertarget{die-regierung-des-kuxf6nigs-jotham}{%
\subsubsection{12. Die Regierung des Königs
Jotham}\label{die-regierung-des-kuxf6nigs-jotham}}

\hypertarget{a-gute-und-gluxfcckliche-regierung-bauten-und-erfolgreiche-kriege}{%
\paragraph{a) Gute und glückliche Regierung; Bauten und erfolgreiche
Kriege}\label{a-gute-und-gluxfcckliche-regierung-bauten-und-erfolgreiche-kriege}}

\hypertarget{section-26}{%
\section{27}\label{section-26}}

1Im Alter von fündundzwanzig Jahren wurde Jotham König und regierte
sechzehn Jahre in Jerusalem; seine Mutter hieß Jerusa und war eine
Tochter Zadoks. 2Er tat, was dem HERRN wohlgefiel, ganz wie sein Vater
Ussia getan hatte; nur drang er nicht in den Tempel des HERRN ein; das
Volk aber verhielt sich immer noch gottlos. 3Er erbaute das obere Tor am
Tempel des HERRN und ließ auch viel an der Mauer des Ophel bauen.
4Weiter befestigte er Ortschaften im Berglande Juda und legte in den
waldigen Gegenden Burgen und Türme an. 5Er führte auch Krieg mit dem
König der Ammoniter und besiegte sie, so daß die Ammoniter ihm in jenem
Jahre eine Abgabe von hundert Talenten Silber, zehntausend Kor Weizen
und zehntausend Kor Gerste entrichteten. Dieselbe Abgabe mußten ihm die
Ammoniter auch noch im zweiten und dritten Jahre entrichten. 6So wurde
Jotham immer mächtiger, weil er sich in seiner Lebensführung genau nach
dem Willen des HERRN, seines Gottes, richtete.

\hypertarget{b-schluuxdfwort}{%
\paragraph{b) Schlußwort}\label{b-schluuxdfwort}}

7Die übrige Geschichte Jothams aber sowie alle seine Kriege und
Unternehmungen finden sich bereits im Buch der Könige von Israel und
Juda aufgezeichnet. 8Im Alter von fünfundzwanzig Jahren war er zur
Regierung gekommen, und sechzehn Jahre hat er in Jerusalem regiert. 9Als
Jotham sich dann zu seinen Vätern gelegt und man ihn in der Davidsstadt
begraben hatte, folgte ihm sein Sohn Ahas in der Regierung nach.

\hypertarget{die-regierung-des-kuxf6nigs-ahas}{%
\subsubsection{13. Die Regierung des Königs
Ahas}\label{die-regierung-des-kuxf6nigs-ahas}}

\hypertarget{a-des-ahas-heidnische-greuel}{%
\paragraph{a) Des Ahas heidnische
Greuel}\label{a-des-ahas-heidnische-greuel}}

\hypertarget{section-27}{%
\section{28}\label{section-27}}

1Im Alter von zwanzig Jahren wurde Ahas König und regierte sechzehn
Jahre in Jerusalem. Er tat nicht, was dem HERRN, seinem Gott,
wohlgefiel, wie sein Ahnherr David getan hatte, 2sondern er wandelte auf
den Wegen der Könige von Israel, ja er ließ sogar für die Baale
gegossene Bilder anfertigen 3und brachte selbst Rauchopfer im Tal
Ben-Hinnom dar und verbrannte seine Söhne als Opfer nach der greulichen
Sitte der heidnischen Völker, die der HERR vor den Israeliten vertrieben
hatte. 4Er brachte auch Schlacht- und Rauchopfer auf den Höhen und
Hügeln und unter jedem dichtbelaubten Baume dar.

\hypertarget{b-schwere-heimsuchungen-durch-die-syrer-und-israeliten}{%
\paragraph{b) Schwere Heimsuchungen durch die Syrer und
Israeliten}\label{b-schwere-heimsuchungen-durch-die-syrer-und-israeliten}}

5So gab ihn denn der HERR, sein Gott, der Gewalt des Königs der Syrer
preis, die ihn besiegten und eine große Menge seiner Leute als Gefangene
wegführten und sie nach Damaskus brachten. Ebenso wurde er auch der
Gewalt des Königs von Israel preisgegeben, der ihm eine schwere
Niederlage beibrachte; 6denn Pekah, der Sohn Remaljas, ließ in Juda an
einem Tage 120000~Mann niederhauen, lauter tüchtige Krieger, weil sie
den HERRN, den Gott ihrer Väter, verlassen hatten. 7Außerdem erschlug
Sichri, ein tapferer ephraimitischer Krieger, den königlichen Prinzen
Maaseja, den Palastvorsteher✲ Asrikam und Elkana, den höchststehenden
Mann im Reiche nächst dem König. 8Dann schleppten die Israeliten von
ihren Volksgenossen 200000 Frauen, Knaben und Mädchen in die
Gefangenschaft weg; dazu nahmen sie ihnen gewaltige Beute ab, die sie
nach Samaria brachten.

\hypertarget{c-freilassung-der-juduxe4ischen-kriegsgefangenen-in-samaria-infolge-der-mahnrede-des-propheten-oded}{%
\paragraph{c) Freilassung der judäischen Kriegsgefangenen in Samaria
infolge der Mahnrede des Propheten
Oded}\label{c-freilassung-der-juduxe4ischen-kriegsgefangenen-in-samaria-infolge-der-mahnrede-des-propheten-oded}}

9Dort lebte aber ein Prophet des HERRN, namens Oded; der ging dem Heere,
das nach Samaria heimkehrte, entgegen und sagte zu ihnen: »Bedenket
wohl: nur deshalb, weil der HERR, der Gott eurer Väter, gegen die Judäer
erzürnt ist\textless sup title=``oder: war''\textgreater✲, hat er sie
euch in die Hände fallen lassen; ihr aber habt ein Blutbad unter ihnen
angerichtet mit einer Wut, die bis an den Himmel reicht✲! 10Und nun
beabsichtigt ihr, diese Kinder Judas und Jerusalems in der Knechtschaft
bei euch zu Sklaven und Sklavinnen zu erniedrigen! Aber habt ihr selbst
keine Verschuldungen gegen den HERRN, euren Gott, auf euch lasten? 11So
hört denn jetzt auf mich und schickt die Gefangenen wieder zurück, die
ihr euren Brüdern✲ geraubt habt: sonst bricht das schwere Zorngericht
des HERRN über euch herein!«

12Da traten von den Häuptern der Ephraimiten einige Männer, nämlich
Asarja, der Sohn Johanans, Berechja, der Sohn Mesillemoths, Hiskia, der
Sohn Sallums, und Amasa, der Sohn Hadlais, vor die vom Feldzug
Heimkehrenden hin 13und sagten zu ihnen: »Ihr dürft die Gefangenen nicht
hierher bringen; denn dadurch würdet ihr eine Verschuldung gegen den
HERRN auf uns laden! Ihr beabsichtigt ja, zu unseren Sünden und
Verschuldungen noch neue hinzuzufügen, und doch ist unsere Schuld schon
groß genug, und ein schweres Zorngericht droht über Israel
hereinzubrechen!« 14Da gaben die Krieger die Gefangenen und die Beute in
Gegenwart der Fürsten und der ganzen Volksgemeinde frei; 15und die
Männer, die mit Namen oben angegeben sind, machten sich daran, sich der
Gefangenen anzunehmen: sie versahen alle, die unter ihnen ungenügend
bekleidet waren, mit Kleidern aus der Beute; und nachdem sie sie mit
Kleidung und Schuhen versehen hatten, gaben sie ihnen zu essen und zu
trinken und Öl, um sich zu salben; hierauf setzten sie alle, die (zum
Gehen) zu schwach waren, auf Esel und brachten sie nach der Palmenstadt
Jericho in die Nähe ihrer Volksgenossen; alsdann kehrten sie nach
Samaria zurück.

\hypertarget{d-schwere-heimsuchungen-durch-die-edomiter-philister-und-assyrer}{%
\paragraph{d) Schwere Heimsuchungen durch die Edomiter, Philister und
Assyrer}\label{d-schwere-heimsuchungen-durch-die-edomiter-philister-und-assyrer}}

16Zu jener Zeit schickte der König Ahas eine Gesandtschaft an den König
von Assyrien, daß er ihm zu Hilfe kommen möchte. 17Denn auch die
Edomiter zogen aufs neue heran, besiegten die Judäer und schleppten
Gefangene weg. 18Dazu unternahmen die Philister Plünderungszüge gegen
die Städte in der Niederung und im Südlande von Juda und eroberten
Beth-Semes, Ajjalon, Gederoth, Socho nebst den zugehörigen Ortschaften,
dazu Thimna nebst den zugehörigen Ortschaften und Gimso nebst den
zugehörigen Ortschaften und setzten sich darin fest. 19Denn der HERR
demütigte Juda um des Königs Ahas von Juda willen, weil er es zügellos
in Juda getrieben und ganz treulos gegen den HERRN gehandelt hatte. 20Da
rückte Thilgath-Pilneser, der König von Assyrien, gegen ihn heran und
bedrängte ihn, statt ihm Beistand zu leisten; 21denn Ahas hatte den
Tempel des HERRN und den königlichen Palast und die Fürsten
ausgeplündert und alles dem König von Assyrien gegeben, ohne daß dies
ihm etwas genützt hätte.

\hypertarget{e-die-zunehmende-gottlosigkeit-des-ahas-schluuxdfwort}{%
\paragraph{e) Die zunehmende Gottlosigkeit des Ahas;
Schlußwort}\label{e-die-zunehmende-gottlosigkeit-des-ahas-schluuxdfwort}}

22Doch (selbst) während der Zeit seiner Bedrängnis versündigte er, der
König Ahas, sich durch Treulosigkeit noch mehr gegen den HERRN. 23Er
opferte nämlich den Göttern von Damaskus, die ihn doch geschlagen
hatten, und dachte dabei: »Weil die Götter der Könige von Syrien ihnen
geholfen haben, so will ich ihnen opfern, damit sie auch mir helfen«;
aber sie dienten ihm nur dazu, ihn selbst und ganz Israel ins Unglück zu
stürzen. 24Auch ließ Ahas die Geräte des Hauses Gottes zusammenbringen
und sie dann zerschlagen, verschloß hierauf die Tore des Tempels des
HERRN und legte sich Altäre an allen Ecken in Jerusalem an. 25Ebenso
richtete er in jeder einzelnen Ortschaft von Juda Opferhöhen ein, um
anderen✲ Göttern Rauchopfer darzubringen, und reizte so den HERRN, den
Gott seiner Väter, zum Zorn.

26Seine übrige Geschichte aber und alle seine Unternehmungen, die
früheren und die späteren, finden sich bereits aufgezeichnet im Buch der
Könige von Juda und Israel. 27Als Ahas sich dann zu seinen Vätern gelegt
hatte, begrub man ihn in der Stadt, in Jerusalem; denn man setzte ihn
nicht in den Gräbern der israelitischen Könige bei. Sein Sohn Hiskia
aber folgte ihm in der Regierung nach.

\hypertarget{die-regierung-des-kuxf6nigs-hiskia}{%
\subsubsection{14. Die Regierung des Königs
Hiskia}\label{die-regierung-des-kuxf6nigs-hiskia}}

\hypertarget{a-wiederherstellung-des-tempels-und-des-reinen-gottesdienstes}{%
\paragraph{a) Wiederherstellung des Tempels und des reinen
Gottesdienstes}\label{a-wiederherstellung-des-tempels-und-des-reinen-gottesdienstes}}

\hypertarget{aa-hiskias-regierungsantritt-und-gute-regierung}{%
\subparagraph{aa) Hiskias Regierungsantritt und gute
Regierung}\label{aa-hiskias-regierungsantritt-und-gute-regierung}}

\hypertarget{section-28}{%
\section{29}\label{section-28}}

1Hiskia wurde im Alter von fünfundzwanzig Jahren König und regierte
neunundzwanzig Jahre in Jerusalem; seine Mutter hieß Abija und war eine
Tochter Sacharjas. 2Er tat, was dem HERRN wohlgefiel, ganz wie sein
Ahnherr David getan hatte.

\hypertarget{bb-hiskias-mahnrede-an-die-priester-und-leviten}{%
\subparagraph{bb) Hiskias Mahnrede an die Priester und
Leviten}\label{bb-hiskias-mahnrede-an-die-priester-und-leviten}}

3Gleich im ersten Monat des ersten Jahres seiner Regierung öffnete er
die Tore des Tempels des HERRN und setzte sie wieder instand. 4Sodann
ließ er die Priester und die Leviten kommen, versammelte sie auf dem
freien Platz gegen Osten 5und hielt folgende Ansprache an sie: »Hört
mich an, ihr Nachkommen Levis! Heiligt euch jetzt und heiligt auch das
Haus des HERRN, des Gottes eurer Väter, indem ihr den Schmutz (des
Götzendienstes) aus dem Heiligtum wegschafft! 6Denn unsere Väter haben
treulos gehandelt und haben getan, was dem HERRN, unserm Gott, mißfällt,
indem sie von ihm abgefallen sind und ihre Blicke von der Wohnstätte des
HERRN abgewandt und (ihr) den Rücken zugekehrt haben. 7Sie haben sogar
die Türen der Vorhalle geschlossen und die Lampen ausgelöscht, haben
kein Räucherwerk mehr verbrannt und dem Gott Israels kein Brandopfer
mehr im Heiligtum dargebracht. 8Daher ist auch der Zorn des HERRN über
Juda und Jerusalem hereingebrochen, und er hat sie zu einem
abschreckenden Beispiel, zu einem Gegenstand des Entsetzens und der
höchsten Verachtung gemacht, wie ihr es mit eigenen Augen seht. 9Ihr
wißt ja: unsere Väter sind eben deswegen durch das Schwert gefallen, und
unsere Söhne und unsere Töchter und Frauen befinden sich in der
Gefangenschaft. 10Doch jetzt bin ich entschlossen, einen Bund mit dem
HERRN, dem Gott Israels, zu schließen, damit die Glut seines Zornes sich
von uns abwendet. 11So zeigt euch nun nicht lässig, meine Söhne! Denn
euch hat der HERR dazu ausersehen, daß ihr vor ihm stehen sollt, um
seinen Dienst zu verrichten; seine Diener sollt ihr sein und ihm
Schlacht- und Rauchopfer darbringen.«

\hypertarget{cc-reinigung-des-tempels-durch-die-leviten}{%
\subparagraph{cc) Reinigung des Tempels durch die
Leviten}\label{cc-reinigung-des-tempels-durch-die-leviten}}

12Da machten sich denn die Leviten ans Werk, nämlich Mahath, der Sohn
Amasais, und Joel, der Sohn Asarjas, von den Nachkommen\textless sup
title=``oder: von der Familie''\textgreater✲ der Kahathiten; und von den
Nachkommen Meraris: Kis, der Sohn Abdis, und Asarja, der Sohn
Jehallelels; und von den Gersoniten: Joah, der Sohn Simmas, und Eden,
der Sohn Joahs; 13und von den Nachkommen Elizaphans: Simri und Jegiel;
und von den Nachkommen Asaphs: Sacharja und Matthanja; 14und von den
Nachkommen Hemans: Jehiel und Simei; und von den Nachkommen Jeduthuns:
Semaja und Ussiel. 15Diese versammelten ihre Stammesgenossen, und
nachdem sie sich geheiligt hatten, machten sie sich nach dem Befehl des
Königs daran, den Tempel des HERRN nach den Weisungen des HERRN zu
reinigen. 16Die Priester begaben sich also in das Innere des Tempels des
HERRN, um es zu reinigen, und schafften alles Unreine, das sie im Tempel
vorfanden, in den Vorhof am Hause des HERRN hinaus, wo die Leviten es in
Empfang nahmen, um es an den Bach Kidron hinauszubringen. 17Am ersten
Tage des ersten Monats fingen sie mit der Reinigung an, und am achten
Tage des Monats waren sie bis an die Vorhalle des Tempels gekommen; acht
Tage verwandten sie dann noch darauf, den Tempel in heiligen Zustand zu
setzen; und am sechzehnten Tage des ersten Monats waren sie fertig. 18Da
begaben sie sich in den Palast des Königs Hiskia und meldeten: »Wir
haben das ganze Haus des HERRN gereinigt, dazu auch den Brandopferaltar
mit all seinen Geräten und den Schaubrottisch mit all seinen Geräten;
19auch alle Geräte, die der König Ahas während seiner Regierung infolge
seines Abfalls entweiht hat, haben wir wieder hergerichtet und
geheiligt: sie stehen jetzt vor dem Altar des HERRN!«

\hypertarget{dd-die-neue-weihe-des-tempels-mit-opfern-gebet-und-gesuxe4ngen}{%
\subparagraph{dd) Die neue Weihe des Tempels mit Opfern, Gebet und
Gesängen}\label{dd-die-neue-weihe-des-tempels-mit-opfern-gebet-und-gesuxe4ngen}}

20Da ließ der König Hiskia am andern Morgen früh die Stadtobersten
zusammenkommen und ging zum Tempel des HERRN hinauf. 21Sie ließen dann
sieben junge Stiere, sieben Widder, sieben Schafe und sieben Ziegenböcke
zum Sündopfer für das Königshaus, für das Heiligtum und für Juda
herbeibringen, und er befahl den Nachkommen Aarons, den Priestern, sie
auf dem Altar des HERRN zu opfern. 22Da schlachteten sie die Rinder, und
die Priester fingen das Blut auf und sprengten es an den Altar; sodann
schlachteten sie die Widder und sprengten das Blut an den Altar; alsdann
schlachteten sie die Schafe und sprengten das Blut an den Altar.
23Schließlich brachten sie die Böcke zum Sündopfer vor den König und vor
die versammelte Volksgemeinde, und nachdem diese ihre Hände fest auf sie
gelegt hatten, 24schlachteten die Priester sie und taten ihr Blut als
Sündopfer an den Altar, um Sühne für ganz Israel zu erwirken; denn für
ganz Israel hatte der König das Brandopfer und das Sündopfer angeordnet.
25Dabei hatte er die Leviten am Tempelhause des HERRN mit Zimbeln,
Harfen und Zithern Aufstellung nehmen lassen, wie David und der
königliche Seher Gad und der Prophet Nathan es angeordnet hatten; denn
vom HERRN war die Anordnung ausgegangen durch den Mund seiner Propheten.
26So hatten sich also die Leviten mit den Musikinstrumenten Davids und
die Priester mit den Trompeten aufgestellt. 27Da befahl Hiskia, das
Brandopfer auf dem Altar darzubringen; und sobald das Brandopfer begann,
fing auch der Gesang zu Ehren des HERRN an, und die Trompeten setzten
ein, und zwar unter Begleitung der Instrumente Davids, des Königs von
Israel. 28Die ganze Versammlung aber warf sich nieder, während die
Lieder erklangen und die Trompeten schmetterten; das alles dauerte, bis
das Brandopfer vollendet war. 29Als man aber mit dem Opfer zu Ende war,
knieten der König und alle, die bei ihm anwesend waren, zum
Gebet\textless sup title=``oder: zur Anbetung''\textgreater✲ nieder.
30Hierauf geboten der König Hiskia und die Fürsten\textless sup
title=``oder: obersten Beamten''\textgreater✲ den Leviten, zu Ehren des
HERRN den Lobgesang anzustimmen mit den Worten Davids und des Sehers
Asaph. Da trugen sie denn das Loblied mit Freuden vor, verneigten sich
und warfen sich nieder.

31Hierauf nahm Hiskia das Wort und sagte: »So habt ihr euch denn jetzt
dem HERRN geweiht: tretet nun herzu und bringt Schlacht- und Dankopfer
zum Tempel des HERRN!« Da brachten die Versammelten Schlacht- und
Dankopfer dar, und jeder, der sich dazu getrieben fühlte, brachte
Brandopfer. 32Die Zahl der Brandopfer, welche die Versammelten
darbrachten, betrug 70 Rinder, 100 Widder, 200 Schafe, diese alle als
Brandopfer für den HERRN; 33außerdem weihte man als Heilsopfer 600
Rinder und 3000 Stück Kleinvieh. 34Weil nun die Priester zu wenige
waren, so daß sie nicht allen Brandopfern die Haut abzuziehen
vermochten, halfen ihnen ihre Stammesgenossen, die Leviten, bis die
Arbeit zu Ende gebracht war und bis die Priester sich geheiligt hatten;
die Leviten waren nämlich mit größerem Eifer darauf bedacht gewesen,
sich zu heiligen, als die Priester. 35Auch Brandopfer waren in Menge zu
besorgen nebst den Fettstücken der Heilsopfer und nebst den zu den
Brandopfern gehörenden Trankopfern.

So war denn der Dienst am Tempel des HERRN wiederhergestellt; 36Hiskia
aber und das ganze Volk freuten sich über das Glück, das der HERR dem
Volke hatte zuteil werden lassen; denn die Sache war außerordentlich
schnell vor sich gegangen.

\hypertarget{b-hiskias-passahfeier}{%
\paragraph{b) Hiskias Passahfeier}\label{b-hiskias-passahfeier}}

\hypertarget{aa-einladung-zur-passahfeier}{%
\subparagraph{aa) Einladung zur
Passahfeier}\label{aa-einladung-zur-passahfeier}}

\hypertarget{section-29}{%
\section{30}\label{section-29}}

1Hierauf sandte Hiskia Boten an ganz Israel und Juda und schrieb auch
Briefe an Ephraim und Manasse, sie möchten zum Hause des HERRN in
Jerusalem kommen, um dem HERRN, dem Gott Israels, eine Passahfeier zu
veranstalten. 2Der König und seine Fürsten\textless sup title=``oder:
höchsten Beamten''\textgreater✲ nebst der ganzen Volksgemeinde in
Jerusalem hatten sich aber entschlossen, das Passah erst im zweiten
Monat zu feiern; 3denn sie konnten es damals nicht sogleich feiern, weil
die Priester sich noch nicht in hinreichender Anzahl dazu geheiligt
hatten und das Volk noch nicht in Jerusalem versammelt war. 4Als dieser
Beschluß die Zustimmung des Königs und der ganzen Volksgemeinde gefunden
hatte, 5faßten sie weiter den Beschluß, einen Aufruf in ganz Israel von
Beerseba bis Dan ergehen zu lassen, man möge kommen, um dem HERRN, dem
Gott Israels, das Passah in Jerusalem zu feiern; denn man hatte es seit
langer Zeit bei so zahlreicher Beteiligung nicht so gefeiert, wie es
vorgeschrieben war.

6So zogen denn die Eilboten mit den Briefen von der Hand des Königs und
seiner höchsten Beamten in ganz Israel und Juda umher und verkündeten
nach dem Befehle des Königs: »Ihr Israeliten! Kehrt zum HERRN, dem Gott
Abrahams, Isaaks und Israels, zurück, damit er sich denen wieder
zuwendet, die der Gewalt der Könige von Assyrien entronnen und euch noch
übriggeblieben sind; 7und macht es nicht wie eure Väter und eure
Volksgenossen, die treulos gegen den HERRN, den Gott ihrer Väter,
gehandelt haben, so daß er sie der Vernichtung preisgegeben hat, wie ihr
es seht! 8Zeigt euch also jetzt nicht halsstarrig wie eure Väter,
sondern reicht dem HERRN die Hand und kommt zu seinem Heiligtum, das er
auf ewig geheiligt hat, und dient dem HERRN, eurem Gott, damit die Glut
seines Zornes sich von euch abwendet! 9Denn wenn ihr zum HERRN umkehrt,
so werden eure Brüder und eure Kinder Erbarmen bei denen finden, die sie
in die Gefangenschaft weggeführt haben, so daß sie in dieses Land
zurückkehren können; denn der HERR, euer Gott, ist gnädig und barmherzig
und wird sein Angesicht nicht von euch wegwenden, wenn ihr zu ihm
zurückkehrt!«~-- 10So zogen denn die Eilboten von Stadt zu Stadt durch
das Land Ephraim und Manasse und bis nach Sebulon, aber man verspottete
und verhöhnte sie; 11nur einige Männer aus Asser, Manasse und Sebulon
gingen in sich und kamen nach Jerusalem. 12Auch in Juda zeigte sich die
Einwirkung Gottes dadurch, daß er ihnen einen einmütigen Sinn verlieh,
um der Aufforderung des Königs und seiner obersten Beamten nach der
Weisung des HERRN nachzukommen.

\hypertarget{bb-verlauf-der-passahfeier-in-der-ersten-woche}{%
\subparagraph{bb) Verlauf der Passahfeier in der ersten
Woche}\label{bb-verlauf-der-passahfeier-in-der-ersten-woche}}

13So versammelte sich denn eine große Volksmenge in Jerusalem, um das
Fest der ungesäuerten Brote im zweiten Monat zu feiern, eine überaus
zahlreiche Volksgemeinde. 14Sie machten sich zunächst daran, die Altäre,
die sich in Jerusalem befanden, wegzuschaffen; ebenso beseitigten sie
alle Rauchopferstätten und warfen sie in den Bach Kidron. 15Sodann
schlachteten sie das Passah am vierzehnten Tage des zweiten Monats; denn
die Priester und die Leviten fühlten sich beschämt und hatten sich
geheiligt und brachten nun Brandopfer zum Tempel des HERRN; 16sie
versahen ihre Amtsgeschäfte pflichtgemäß, wie es ihnen nach dem Gesetz
Moses, des Mannes Gottes, oblag: die Priester sprengten das Blut, das
sie aus der Hand der Leviten genommen hatten. 17Es gab nämlich viele
unter den Versammelten, die sich nicht geheiligt hatten; daher besorgten
die Leviten das Schlachten der Passahlämmer für einen jeden, der nicht
rein war, um sie dem HERRN zu weihen. 18Ja, die Mehrzahl des Volkes,
viele aus Ephraim und Manasse, Issaschar und Sebulon, hatten sich nicht
gereinigt, sondern aßen das Passah nicht in der vorgeschriebenen Weise.
Doch Hiskia hatte für sie gebetet mit den Worten: »Der HERR, der Gütige,
wolle einem jeden verzeihen, 19der ernstlich darauf bedacht ist, Gott
den HERRN, den Gott seiner Väter, zu suchen, wenn auch nicht gemäß der
für das Heiligtum erforderlichen Reinheit!« 20Und der HERR erhörte
Hiskia und ließ das Volk unversehrt.

21So feierten denn die Israeliten, die sich in Jerusalem eingefunden
hatten, das Fest der ungesäuerten Brote sieben Tage lang mit großer
Freude, und die Leviten und die Priester sangen dem HERRN Tag für Tag
Loblieder unter Begleitung der Musikinstrumente. 22Dabei richtete Hiskia
freundliche Worte an alle Leviten, welche richtiges Verständnis für den
Dienst des HERRN bewiesen; und man hielt Festmahle die sieben Tage
hindurch, indem sie Heilsopfer darbrachten und den HERRN, den Gott ihrer
Väter, priesen.

\hypertarget{cc-fortsetzung-der-festfeier-in-der-zweiten-woche}{%
\subparagraph{cc) Fortsetzung der Festfeier in der zweiten
Woche}\label{cc-fortsetzung-der-festfeier-in-der-zweiten-woche}}

23Darauf entschloß sich die ganze Volksgemeinde, noch weitere sieben
Tage zu feiern; und so begingen sie noch sieben Tage lang ein
Freudenfest. 24Denn Hiskia, der König von Juda, hatte der Volksgemeinde
tausend Stiere und siebentausend Stück Kleinvieh als Spende überwiesen,
und die Fürsten\textless sup title=``oder: höchsten
Beamten''\textgreater✲ hatten gleichfalls tausend Stiere und zehntausend
Stück Kleinvieh gespendet; und die Priester hatten sich in großer Zahl
geheiligt. 25So überließ sich denn die ganze Volksgemeinde Judas der
Fröhlichkeit und ebenso die Priester und die Leviten und die ganze Menge
derer, die aus Israel gekommen waren, auch die Fremdlinge✲, die aus den
verschiedensten Gegenden Israels gekommen waren oder in Juda wohnten.
26Und es herrschte große Freude in Jerusalem; denn seit den Tagen
Salomos, des Sohnes Davids, des Königs von Israel, war etwas Derartiges
in Jerusalem nicht vorgekommen. 27Die levitischen Priester aber erhoben
sich und segneten das Volk, und ihre laute Bitte fand Erhörung: ihr
Gebet drang zu seiner\textless sup title=``d.h. Gottes''\textgreater✲
heiligen Wohnung, in den Himmel.

\hypertarget{c-neuordnung-des-gottesdienstes-und-der-versorgung-der-priester-und-leviten}{%
\paragraph{c) Neuordnung des Gottesdienstes und der Versorgung der
Priester und
Leviten}\label{c-neuordnung-des-gottesdienstes-und-der-versorgung-der-priester-und-leviten}}

\hypertarget{aa-reinigung-des-landes-vom-guxf6tzendienst}{%
\subparagraph{aa) Reinigung des Landes vom
Götzendienst}\label{aa-reinigung-des-landes-vom-guxf6tzendienst}}

\hypertarget{section-30}{%
\section{31}\label{section-30}}

1Als nun alle diese (Festlichkeiten) zu Ende waren, zogen sämtliche
Israeliten, die sich dazu eingefunden hatten, in die Ortschaften Judas
hinaus, zertrümmerten die Malsteine, zerschlugen die Standbilder der
Aschera und zerstörten die Opferhöhen\textless sup title=``oder:
Höhenheiligtümer''\textgreater✲ und die Altäre in ganz Juda und Benjamin
sowie in Ephraim und Manasse, bis sie sie gänzlich vernichtet hatten;
darauf kehrten alle Israeliten in ihre Ortschaften zurück, ein jeder zu
seinem Besitztum.

\hypertarget{bb-erfolgreiche-sorge-fuxfcr-das-einkommen-der-priester-und-leviten}{%
\subparagraph{bb) Erfolgreiche Sorge für das Einkommen der Priester und
Leviten}\label{bb-erfolgreiche-sorge-fuxfcr-das-einkommen-der-priester-und-leviten}}

2Hierauf bestellte Hiskia die verschiedenen Abteilungen der Priester und
der Leviten, jeden einzelnen von den Priestern und den Leviten nach
Maßgabe des ihm obliegenden Dienstes bei den Brandopfern und bei den
Heilsopfern, um in den Toren des Lagers\textless sup title=``oder: der
Vorhöfe des Hauses''\textgreater✲ des HERRN Dienst zu tun und Danklieder
oder Lobgesänge vorzutragen. 3Der Beitrag des Königs aus seinem Vermögen
war für die Brandopfer bestimmt, und zwar für die Brandopfer sowohl an
jedem Morgen und Abend als auch an den Sabbaten und Neumonden und an den
Festen, wie es im Gesetz des HERRN vorgeschrieben ist. 4Sodann machte er
es dem Volke, das in Jerusalem wohnte, zur Pflicht, den Priestern und
den Leviten die ihnen gebührenden Abgaben zukommen zu lassen, damit sie
am Gesetz des HERRN festhalten könnten. 5Sobald nun dieser Befehl
bekannt wurde, brachten die Israeliten reichlich die Erstlinge vom
Getreide, Most, Öl und Honig sowie von allen (übrigen) Erzeugnissen des
Feldes dar und lieferten den Zehnten von allem in Menge ab; 6und die
Israeliten und Judäer, die in den Ortschaften Judas wohnten, brachten
ebenfalls den Zehnten vom Groß- und Kleinvieh sowie den Zehnten von den
Weihegaben, die dem HERRN, ihrem Gott, geweiht waren, und legten sie
Haufen bei Haufen hin. 7Im dritten Monat begannen sie die Haufen
aufzuschichten, und im siebten Monat waren sie damit fertig. 8Als dann
Hiskia und die Fürsten\textless sup title=``oder: höchsten
Beamten''\textgreater✲ kamen und die Haufen besichtigten, priesen sie
den HERRN und sein Volk Israel; 9und als Hiskia sich nun bei den
Priestern und den Leviten in betreff der Haufen erkundigte, 10antwortete
ihm Asarja, der Oberpriester aus dem Hause Zadok: »Seitdem man
angefangen hat, die Abgaben zum Tempel des HERRN zu bringen, haben wir
gegessen und sind satt geworden und haben noch viel übrig behalten; denn
der HERR hat sein Volk gesegnet; daher ist dieser große Vorrat da
übriggeblieben.«

11Da befahl Hiskia, Zellen\textless sup title=``oder:
Vorratskammern''\textgreater✲ im Hause des HERRN herzurichten; und als
dies geschehen war, 12brachte man die Abgaben sowie die Zehnten und die
Weihegaben gewissenhaft hinein. Zum Oberaufseher darüber wurde der Levit
Chananja bestellt und sein Bruder Simei an zweiter Stelle; 13Jehiel
aber, Asasja, Nahath, Asahel, Jerimoth, Josabad, Eliel, Jismachja,
Mahath und Benaja standen als Aufseher dem Chananja und seinem Bruder
Simei zur Seite nach der Anordnung des Königs Hiskia und Asarjas, des
Fürsten\textless sup title=``oder: Oberaufsehers''\textgreater✲ im Hause
Gottes. 14Weiter wurde der Levit Kore, der Sohn Jimnas, der Hüter des
östlichen Tores, zum Aufseher über die Gaben bestellt, die Gott
freiwillig dargebracht wurden, damit die dem HERRN gebührenden Hebopfer
und die hochheiligen Gaben abgeliefert würden. 15Ihm standen Eden,
Minjamin, Jesua, Semaja, Amarja und Sechanja in den Priesterstädten
getreulich zur Seite, um ihren Amtsbrüdern abteilungsweise, den alten
wie den jungen, ihre Anteile zuzuweisen 16mit Ausnahme der in das
Geschlechtsverzeichnis eingetragenen männlichen Personen im Alter von
drei und mehr Jahren, d.h. aller, die zum Hause des HERRN kamen, wie es
ein jeder Tag erforderte, um ihren Dienst nach ihren Obliegenheiten
abteilungsweise auszurichten.

\hypertarget{cc-aufstellung-von-verzeichnissen-der-priester-und-leviten-schluuxdfwort}{%
\subparagraph{cc) Aufstellung von Verzeichnissen der Priester und
Leviten;
Schlußwort}\label{cc-aufstellung-von-verzeichnissen-der-priester-und-leviten-schluuxdfwort}}

17Was aber das Verzeichnis der Priester anbetrifft, so war es nach ihren
Familien angelegt, und das der Leviten enthielt die Personen von zwanzig
und mehr Jahren mit Rücksicht auf ihre amtlichen Obliegenheiten
abteilungsweise, 18und zwar waren sie in das Verzeichnis eingetragen
samt all ihren kleinen Kindern, ihren Frauen, ihren Söhnen und ihren
Töchtern, also der gesamte Stand; denn in ihrer Ehrenhaftigkeit hatten
sie sich zu gewissenhafter Pflichterfüllung geheiligt. 19Auch für die
Nachkommen Aarons, die Priester, waren in den Bezirken der zu ihren
Städten gehörenden Markung, in jeder einzelnen Stadt, Männer, die mit
Namen angegeben waren, dazu bestellt, allen männlichen Personen unter
den Priestern und allen in das Verzeichnis eingetragenen Leviten Anteile
zukommen zu lassen.

20Auf diese Weise verfuhr Hiskia in ganz Juda und tat, was vor dem
HERRN, seinem Gott, gut, recht und pflichtgemäß war; 21und bei allem,
was er in betreff des Dienstes am Hause Gottes und auf Grund des
Gesetzes und des Gebotes, um seinen Gott zu suchen, vornahm, handelte er
mit voller Aufrichtigkeit und hatte daher auch glücklichen Erfolg.

\hypertarget{d-der-einfall-sanheribs-und-die-weitere-geschichte-hiskias}{%
\paragraph{d) Der Einfall Sanheribs und die weitere Geschichte
Hiskias}\label{d-der-einfall-sanheribs-und-die-weitere-geschichte-hiskias}}

\hypertarget{aa-hiskias-umsichtige-mauxdfnahmen-gegen-den-feind}{%
\subparagraph{aa) Hiskias umsichtige Maßnahmen gegen den
Feind}\label{aa-hiskias-umsichtige-mauxdfnahmen-gegen-den-feind}}

\hypertarget{section-31}{%
\section{32}\label{section-31}}

1Nach diesen Begebenheiten und dieser von Hiskia bewiesenen Treue zog
Sanherib, der König von Assyrien, heran, drang in Juda ein und belagerte
die festen Plätze in der Absicht, sie für sich zu erobern. 2Als nun
Hiskia sah, daß Sanherib herangerückt war und sich zum Angriff auf
Jerusalem anschickte, 3entschloß er sich im Einvernehmen mit seinen
höchsten Beamten und seinen Heerführern, die Wasserquellen außerhalb der
Stadt zu verschütten, und sie waren ihm dabei behilflich. 4Es wurden
also Leute in großer Zahl aufgeboten, welche die sämtlichen Quellen und
den Bach, der mitten durch das Land floß, verschütteten, indem sie
sagten: »Warum sollen die Könige von Assyrien, wenn sie kommen, Wasser
in Menge vorfinden?« 5Alsdann machte er sich mit Entschlossenheit ans
Werk und ließ die Stadtmauer überall, wo sie niedergerissen\textless sup
title=``oder: schadhaft''\textgreater✲ war, wiederherstellen und Türme
auf ihr errichten und draußen noch eine zweite Mauer aufführen; auch
befestigte er das Millo\textless sup title=``1.Chr 11,8''\textgreater✲
in der Davidsstadt und ließ Waffen\textless sup title=``oder:
Wurfgeschosse''\textgreater✲ und Schilde in Menge herstellen. 6Weiter
setzte er kriegstüchtige Anführer über das Heer ein, versammelte diese
um sich auf dem freien Platz am Stadttor und sprach ihnen Mut zu mit den
Worten: 7»Seid mutig und entschlossen! Fürchtet euch nicht und seid
unverzagt vor dem König von Assyrien und vor dem ganzen Haufen, der mit
ihm zieht! Denn mit uns ist ein Stärkerer als mit ihm: 8mit ihm ist nur
ein Arm von Fleisch, mit uns aber ist der HERR, unser Gott, der wird uns
helfen und unsere Kriege führen!« Und das Volk fühlte sich durch die
Worte Hiskias, des Königs von Juda, ermutigt.

\hypertarget{bb-sanheribs-aufforderung-zur-uxfcbergabe-der-stadt-von-lachis-aus}{%
\subparagraph{bb) Sanheribs Aufforderung zur Übergabe der Stadt von
Lachis
aus}\label{bb-sanheribs-aufforderung-zur-uxfcbergabe-der-stadt-von-lachis-aus}}

9Darnach sandte Sanherib, der König von Assyrien, während er selbst mit
seiner ganzen Heeresmacht vor Lachis stand, (einige von) seinen
Dienern\textless sup title=``oder: Oberen''\textgreater✲ nach Jerusalem
an Hiskia, den König von Juda, und an alle Judäer, die sich in Jerusalem
befanden, mit der Botschaft: 10»So läßt Sanherib, der König von
Assyrien, euch sagen: ›Worauf verlaßt ihr euch, daß ihr in Jerusalem
eingeschlossen sitzen bleibt? 11Ja, Hiskia ist es, der euch irreführt,
um euch dem Tode durch Hunger und Durst preiszugeben, indem er euch
verheißt: Der HERR, unser Gott, wird uns aus der Gewalt des Königs von
Assyrien erretten! 12Ist das nicht derselbe Hiskia, der den Höhendienst
und die Altäre eures Gottes weggeschafft und in Juda und Jerusalem das
Gebot erlassen hat: Vor einem einzigen Altar sollt ihr anbeten, und nur
auf ihm dürft ihr opfern? 13Wißt ihr nicht, wie ich und meine Väter mit
allen Völkern der (anderen) Länder verfahren sind? Haben etwa die Götter
der Völker in den (übrigen) Ländern ihr Land aus meiner Gewalt irgendwie
zu erretten vermocht? 14Wo ist unter allen Göttern dieser Völker, die
meine Väter dem Untergange geweiht haben, einer gewesen, der sein Volk
aus meiner Gewalt hätte erretten können? Wie sollte euer Gott da euch
aus meiner Gewalt zu erretten vermögen? 15So laßt euch denn jetzt nicht
von Hiskia betören und euch nicht auf solche Weise irreführen und
schenkt ihm keinen Glauben! Denn kein Gott irgendeines Volkes und
irgendeines Reiches hat (bisher) sein Volk aus meiner und meiner Väter
Gewalt zu erretten vermocht; geschweige denn, daß euer Gott euch aus
meiner Gewalt sollte erretten können!‹«

\hypertarget{cc-sanheribs-und-seiner-gesandten-hochmut}{%
\subparagraph{cc) Sanheribs und seiner Gesandten
Hochmut}\label{cc-sanheribs-und-seiner-gesandten-hochmut}}

16Seine Gesandten redeten dann noch mehr gegen Gott den HERRN und gegen
seinen Knecht Hiskia; 17auch hatte er, um den HERRN, den Gott Israels,
zu verhöhnen und zu lästern, einen Brief folgenden Inhalts geschrieben:
»Sowenig die Götter der Völker in den (übrigen) Ländern ihr Volk aus
meiner Gewalt errettet haben, ebensowenig wird der Gott Hiskias sein
Volk aus meiner Gewalt erretten!« 18Sie riefen dies auch der Bevölkerung
von Jerusalem, die auf der Mauer stand, mit lauter Stimme auf jüdisch
zu, um sie zu schrecken und in Angst zu versetzen, damit sie so die
Stadt in ihre Gewalt brächten, 19und redeten von dem Gott Jerusalems wie
von den Göttern der Heidenvölker der Erde, die doch nur Machwerke von
Menschenhänden sind.

\hypertarget{dd-hiskias-gebet-gottes-hilfe-sanheribs-vernichtung-abzug-und-tod}{%
\subparagraph{dd) Hiskias Gebet; Gottes Hilfe: Sanheribs Vernichtung,
Abzug und
Tod}\label{dd-hiskias-gebet-gottes-hilfe-sanheribs-vernichtung-abzug-und-tod}}

20Als nun infolgedessen der König Hiskia und der Prophet Jesaja, der
Sohn des Amoz, beteten und um Hilfe zum Himmel schrien, 21da sandte der
HERR einen Engel, der sämtliche Kriegsleute und Fürsten und Heerführer
im Lager des Königs von Assyrien sterben ließ, so daß jener mit Schimpf
und Schande in sein Land zurückkehrte; und als er sich dann in den
Tempel seines Gottes begab, brachten ihn dort einige von seinen
leiblichen Söhnen mit dem Schwerte um. 22So rettete der HERR den Hiskia
und die Bewohner Jerusalems aus der Gewalt des Königs Sanherib von
Assyrien und aus der Gewalt aller (seiner übrigen Feinde) und
verschaffte ihnen Ruhe auf allen Seiten. 23Und viele brachten dem HERRN
Gaben nach Jerusalem und dem König Hiskia kostbare Geschenke, so daß er
fortan bei allen Völkern in hohem Ansehen stand.

\hypertarget{ee-hiskias-krankheit-uxfcberhebung-und-buuxdfe}{%
\subparagraph{ee) Hiskias Krankheit, Überhebung und
Buße}\label{ee-hiskias-krankheit-uxfcberhebung-und-buuxdfe}}

24Als Hiskia zu jener Zeit auf den Tod erkrankte, betete er zum HERRN,
und dieser erhörte ihn und gab ihm ein Wunderzeichen. 25Aber Hiskia
bewies sich für die ihm erwiesene Wohltat nicht dankbar, sondern sein
Herz überhob sich; darum brach ein Zorngericht über ihn und über Juda
und Jerusalem herein. 26Weil aber Hiskia sich nunmehr wegen seines
Hochmuts demütigte, er samt der Bevölkerung von Jerusalem, erging das
Zorngericht des HERRN über sie nicht schon bei Lebzeiten Hiskias.

\hypertarget{ff-hiskias-reichtum-wasserversorgung-jerusalems-und-versuchung-durch-die-babylonische-gesandtschaft}{%
\subparagraph{ff) Hiskias Reichtum; Wasserversorgung Jerusalems und
Versuchung durch die babylonische
Gesandtschaft}\label{ff-hiskias-reichtum-wasserversorgung-jerusalems-und-versuchung-durch-die-babylonische-gesandtschaft}}

27Hiskia besaß sehr großen Reichtum und bedeutende Macht und legte sich
auch Schatzkammern an für Silber, Gold und Edelsteine, für
Spezereien\textless sup title=``d.h. Gewürzwaren, Balsam''\textgreater✲,
Schilde und Kostbarkeiten aller Art, 28auch Vorratshäuser für den Ertrag
von Getreide, Wein und Öl und Stallungen für allerlei Arten von Vieh und
Hürden für die Herden. 29Auch Städte legte er sich an, und er besaß
große Herden von Kleinvieh und Rindern; denn Gott hatte ihm ein sehr
bedeutendes Vermögen verliehen. 30Hiskia ist es auch gewesen, der den
oberen Ausfluß des Wassers des Gihon verschüttet und es nach der
Westseite hinunter nach der Davidsstadt geleitet hat; und bei allen
seinen Unternehmungen hatte er Glück. 31Gleichwohl bei Gelegenheit der
Gesandtschaft, welche die Fürsten von Babylon an ihn geschickt hatten,
um Erkundigungen wegen des Wunderzeichens einzuziehen, das im Lande
geschehen war, verließ ihn Gott, um ihn auf die Probe zu stellen, damit
er seine Gesinnung völlig kennenlernte.

\hypertarget{gg-abschluuxdf-der-geschichte-hiskias}{%
\subparagraph{gg) Abschluß der Geschichte
Hiskias}\label{gg-abschluuxdf-der-geschichte-hiskias}}

32Die übrige Geschichte Hiskias aber und seine frommen Taten finden sich
bekanntlich bereits aufgezeichnet in der Offenbarung des Propheten
Jesaja, des Sohnes des Amoz, und im Buch der Könige von Juda und Israel.

33Als Hiskia sich dann zu seinen Vätern gelegt hatte, begrub man ihn am
Aufstieg zu den Gräbern der Nachkommen Davids; und ganz Juda und die
Bevölkerung Jerusalems erzeigten ihm Ehre bei seinem Tode. Sein Sohn
Manasse folgte ihm dann in der Regierung nach.

\hypertarget{die-regierung-der-kuxf6nige-manasse-und-amon}{%
\subsubsection{15. Die Regierung der Könige Manasse und
Amon}\label{die-regierung-der-kuxf6nige-manasse-und-amon}}

\hypertarget{a-manasse-kuxf6nig-von-juda}{%
\paragraph{a) Manasse König von
Juda}\label{a-manasse-kuxf6nig-von-juda}}

\hypertarget{aa-manasses-abguxf6tterei}{%
\subparagraph{aa) Manasses Abgötterei}\label{aa-manasses-abguxf6tterei}}

\hypertarget{section-32}{%
\section{33}\label{section-32}}

1Im Alter von zwölf Jahren wurde Manasse König und regierte
fünfundfünfzig Jahre in Jerusalem. 2Er tat, was dem HERRN mißfiel, im
Anschluß an den greuelhaften Götzendienst der (heidnischen) Völker, die
der HERR vor den Israeliten vertrieben hatte. 3Er baute die
Höhenheiligtümer wieder auf, die sein Vater Hiskia zerstört hatte,
errichtete den Baalen Altäre, ließ Standbilder der Aschera\textless sup
title=``oder: Astarte''\textgreater✲ herstellen, betete das ganze
Sternenheer des Himmels an und erwies ihnen Verehrung. 4Er erbaute sogar
Altäre im Tempel des HERRN, von dem doch der HERR gesagt hatte: »In
Jerusalem soll mein Name für alle Zeiten wohnen!«, 5und zwar erbaute er
dem ganzen Sternenheer des Himmels Altäre in den beiden
Vorhöfen\textless sup title=``vgl. zu 2.Kön 21,5''\textgreater✲ beim
Tempel des HERRN. 6Ja er ließ sogar seine eigenen Söhne im Tale
Ben-Hinnom als Brandopfer verbrennen, trieb Zauberei, Wahrsagerei und
geheime Künste und bestellte Totenbeschwörer und Zeichendeuter: er tat
vieles, was dem HERRN mißfiel und ihn zum Zorn reizen mußte. 7Das
geschnitzte Götzenbild, das er hatte anfertigen lassen, stellte er sogar
im Hause Gottes auf, von dem doch Gott zu David und dessen Sohne Salomo
gesagt hatte: »In diesem Hause und in Jerusalem, das ich aus allen
Stämmen Israels erwählt habe, will ich meinen Namen für ewige Zeiten
wohnen lassen; 8und ich will den Fuß Israels fortan nicht wieder weichen
lassen aus dem Lande, das ich ihren Vätern angewiesen habe, wofern sie
nur darauf bedacht sind, alles zu tun, was ich ihnen geboten habe,
nämlich ganz nach dem Gesetz und den Satzungen und Verordnungen (zu
leben), die sie durch Mose erhalten haben.« 9Aber Manasse verleitete
Juda und die Bewohner Jerusalems dazu, es noch ärger zu treiben, als die
heidnischen Völker es getan hatten, die der HERR vor den Israeliten
vertilgt hatte. 10Zwar warnte der HERR den Manasse und sein Volk durch
den Mund von Propheten, aber sie achteten nicht darauf.

\hypertarget{bb-manasses-gefangenfuxfchrung-nach-babylon-seine-buuxdfe-und-heimkehr}{%
\subparagraph{bb) Manasses Gefangenführung nach Babylon, seine Buße und
Heimkehr}\label{bb-manasses-gefangenfuxfchrung-nach-babylon-seine-buuxdfe-und-heimkehr}}

11Da ließ der HERR die Heerführer des Königs von Assyrien gegen sie
anrücken; die führten Manasse mit Haken gefangen, legten ihm eherne
Fesseln an und brachten ihn nach Babylon. 12Als er sich nun in Not
befand, flehte er zum HERRN, seinem Gott, um Gnade und demütigte sich
tief vor dem Gott seiner Väter; 13und als er nun zu ihm betete, ließ er
sich von ihm erbitten, so daß er sein Flehen erhörte und ihn nach
Jerusalem in seine königliche Stellung zurückbrachte. Da erkannte
Manasse, daß der HERR (der wahre) Gott ist.

\hypertarget{cc-manasses-mauerbauten-und-bemuxfchung-um-beseitigung-des-guxf6tzendienstes}{%
\subparagraph{cc) Manasses Mauerbauten und Bemühung um Beseitigung des
Götzendienstes}\label{cc-manasses-mauerbauten-und-bemuxfchung-um-beseitigung-des-guxf6tzendienstes}}

14Später baute er noch eine äußere Mauer an der Davidsstadt, auf der
Westseite nach dem Gihon hin, im (Kidron-) Tal und bis zum Eingang ins
Fischtor, so daß er den Ophel umschloß; dabei erhöhte er sie
beträchtlich. Auch setzte er Befehlshaber in sämtlichen festen Plätzen
Judas ein. 15Sodann schaffte er die fremden Götter und das Götzenbild✲
aus dem Tempel des HERRN weg, ebenso alle Altäre, die er auf dem
Tempelberge und in Jerusalem errichtet hatte, und ließ sie vor die Stadt
hinauswerfen. 16Dagegen stellte er den Altar des HERRN wieder her und
opferte auf ihm Heils- und Dankopfer und machte den Judäern die
Verehrung des HERRN, des Gottes Israels, zur Pflicht. 17Indessen opferte
das Volk noch immer auf den Höhen, allerdings nur dem HERRN, ihrem Gott.

\hypertarget{dd-schluuxdfwort}{%
\subparagraph{dd) Schlußwort}\label{dd-schluuxdfwort}}

18Die übrige Geschichte Manasses aber und sein Gebet zu seinem Gott
sowie die Reden der Seher, die im Namen des HERRN, des Gottes Israels,
zu ihm geredet haben, das alles steht bereits in der Geschichte der
Könige von Israel geschrieben. 19Sein Gebet aber und wie er bei Gott
Erhörung fand, sowie alle seine Versündigungen und seine Untreue und die
Orte, an denen er vor seiner Demütigung Höhenaltäre erbaut und
Standbilder der Aschera und geschnitzte Götzenbilder aufgestellt hatte,
das findet sich bekanntlich aufgezeichnet in der Geschichte
Hosais\textless sup title=``oder: der Seher''\textgreater✲.~-- 20Als
Manasse sich dann zu seinen Vätern gelegt und man ihn in seinem
Hause\textless sup title=``d.h. im Garten seines Palastes''\textgreater✲
begraben hatte, folgte ihm sein Sohn Amon in der Regierung nach.

\hypertarget{b-amon-kuxf6nig-von-juda}{%
\paragraph{b) Amon König von Juda}\label{b-amon-kuxf6nig-von-juda}}

21Im Alter von zweiundzwanzig Jahren wurde Amon König und zwei Jahre
regierte er in Jerusalem. 22Er tat, was dem HERRN mißfiel, wie sein
Vater Manasse getan hatte; und allen Götzenbildern, die sein Vater
Manasse hatte anfertigen lassen, brachte Amon eifrig Opfer dar und
verehrte sie. 23Aber er demütigte sich nicht vor dem HERRN, wie sein
Vater Manasse sich gedemütigt hatte, sondern er, Amon, lud große Schuld
auf sich. 24Da verschworen sich seine eigenen Diener gegen ihn und
ermordeten ihn in seinem Palast. 25Die Landbevölkerung aber erschlug
alle, die an der Verschwörung gegen den König Amon teilgenommen hatten,
und erhob dann seinen Sohn Josia zu seinem Nachfolger auf dem Throne.

\hypertarget{die-regierung-des-kuxf6nigs-josia}{%
\subsubsection{16. Die Regierung des Königs
Josia}\label{die-regierung-des-kuxf6nigs-josia}}

\hypertarget{a-eingangswort-1}{%
\paragraph{a) Eingangswort}\label{a-eingangswort-1}}

\hypertarget{section-33}{%
\section{34}\label{section-33}}

1Im Alter von acht Jahren wurde Josia König, und einunddreißig Jahre
regierte er in Jerusalem. 2Er tat, was dem HERRN wohlgefiel: er wandelte
auf den Wegen seines Ahnherrn David, ohne nach rechts oder nach links
davon abzuweichen.

\hypertarget{b-wiederherstellung-des-reinen-gottesdienstes}{%
\paragraph{b) Wiederherstellung des reinen
Gottesdienstes}\label{b-wiederherstellung-des-reinen-gottesdienstes}}

3Im achten Jahre seiner Regierung, als er noch ein Jüngling war, fing er
an, den Gott seines Ahnherrn David zu suchen, und im zwölften Jahr
begann er, Juda und Jerusalem von dem Höhendienst und den Standbildern
der Aschera und den geschnitzten und gegossenen Bildern zu reinigen.
4Vor seinen Augen riß man die Altäre der Baale nieder, und die
Sonnensäulen, die oben auf ihnen standen, ließ er umhauen und die
Standbilder der Aschera und die geschnitzten und gegossenen Bilder
zertrümmern und zermalmen und den Staub von ihnen auf die Gräber derer
streuen, die ihnen geopfert hatten; 5die Gebeine der Priester ließ er
auf ihren Altären verbrennen und reinigte auf diese Weise Juda und
Jerusalem. 6Auch in den Städten von Manasse und Ephraim, von Simeon und
bis nach Naphthali hin -- in den dortigen Trümmerstätten ringsum~-- 7riß
er die Altäre nieder, zertrümmerte und zermalmte die Standbilder der
Aschera und die geschnitzten Bilder und ließ alle Sonnensäulen in allen
Gegenden Israels umhauen; dann kehrte er nach Jerusalem zurück.

\hypertarget{c-darlegung-des-bei-der-wiederherstellung-und-instandhaltung-des-tempels-beachteten-verfahrens}{%
\paragraph{c) Darlegung des bei der Wiederherstellung und Instandhaltung
des Tempels beachteten
Verfahrens}\label{c-darlegung-des-bei-der-wiederherstellung-und-instandhaltung-des-tempels-beachteten-verfahrens}}

8Im achtzehnten Jahre seiner Regierung aber, als er das Land und den
Tempel gereinigt hatte, sandte er Saphan, den Sohn Azaljas, und den
Stadthauptmann Maaseja und den Kanzler Joah, den Sohn des Joahas, um den
Tempel des HERRN, seines Gottes, wieder instandsetzen zu lassen. 9Als
diese nun zum Hohenpriester Hilkia kamen, übergaben sie das Geld, das im
Hause Gottes eingegangen war und das die Leviten, die Schwellenhüter,
von den Stämmen Manasse und Ephraim und von allen übrigen Israeliten
sowie von ganz Juda und Benjamin und den Bewohnern Jerusalems
eingesammelt hatten,~-- 10und zwar händigten sie es den Werkführern ein,
die für die Arbeit am Tempel des HERRN bestellt waren; diese übergaben
es dann den Arbeitern, die am Tempel des HERRN tätig waren, um den
Tempel herzustellen und auszubessern; 11und zwar übergaben sie es den
Zimmerleuten und Bauleuten zum Ankauf von behauenen Steinen und von
Hölzern für die Decken und um die Baulichkeiten, welche die Könige von
Juda hatten in Verfall geraten lassen, mit neuem Gebälk zu versehen.
12Diese Männer handelten bei ihrer Arbeit auf Treu und Glauben; und über
sie waren gesetzt: die Leviten Jahath und Obadja aus den Nachkommen
Meraris, und Sacharja und Mesullam aus den Nachkommen der Kahathiten, um
die Aufsicht zu führen; und die Leviten, soweit sie sich auf
Musikinstrumente verstanden, 13waren über die Lastträger gesetzt und
führten auch die Aufsicht über alle Arbeiter bei ihren verschiedenen
Dienstleistungen; und andere von den Leviten waren auch Schreiber,
Amtleute und Torhüter.

\hypertarget{d-bericht-uxfcber-die-auffindung-des-gesetzbuches-und-seine-erste-wirkung}{%
\paragraph{d) Bericht über die Auffindung des Gesetzbuches und seine
erste
Wirkung}\label{d-bericht-uxfcber-die-auffindung-des-gesetzbuches-und-seine-erste-wirkung}}

14Als sie nun das Geld, das im Tempel des HERRN eingegangen war,
herausnahmen, fand der Priester Hilkia das Buch mit dem von Mose
herrührenden Gesetz des HERRN. 15Da hob Hilkia an und sagte zum
Staatsschreiber Saphan: »Ich habe das Gesetzbuch im Tempel des HERRN
gefunden!« Damit übergab Hilkia das Buch dem Saphan. 16Dieser
überbrachte dann das Buch dem König und erstattete außerdem dem Könige
folgenden Bericht: »Alles, was deinen Knechten\textless sup
title=``oder: Dienern''\textgreater✲ aufgetragen war, haben sie
ausgerichtet: 17sie haben das Geld, das sich im Tempel des HERRN
vorfand, ausgeschüttet und haben es den zur Aufsicht Bestellten und den
Werkführern eingehändigt.« 18Weiter machte der Staatsschreiber Saphan
dem Könige noch die Mitteilung: »Der Priester Hilkia hat mir ein Buch
gegeben«; und Saphan las dem Könige daraus vor. 19Als nun der König den
Inhalt des Gesetzbuches vernahm, zerriß er seine Kleider 20und gab
sodann dem Hilkia und Ahikam, dem Sohne Saphans, ferner Abdon, dem Sohne
Michas, und dem Staatsschreiber Saphan und Asaja, dem Leibdiener des
Königs, folgenden Befehl: 21»Geht hin und befragt den HERRN für mich und
für die, welche in Israel und Juda noch übriggeblieben sind, in betreff
des Wortlautes\textless sup title=``oder: Inhaltes''\textgreater✲ des
Buches, das man aufgefunden hat; denn groß ist der Grimm des HERRN, der
sich über uns ergossen hat, weil unsere Väter die Weisungen des HERRN
nicht beachtet haben, um alles zu tun, was in diesem Buche geschrieben
steht.«

\hypertarget{e-befragung-und-antwort-der-prophetin-hulda}{%
\paragraph{e) Befragung und Antwort der Prophetin
Hulda}\label{e-befragung-und-antwort-der-prophetin-hulda}}

22Da begab sich Hilkia mit den Männern, die der König bezeichnet hatte,
zu der Prophetin Hulda, der Frau des Kleiderhüters Sallum, des Sohnes
Thokhaths, des Sohnes Hasras; die wohnte in Jerusalem im zweiten Bezirk.
Als sie sich nun mit ihr gemäß dem ihnen erteilten Auftrage besprachen,
23sagte sie zu ihnen: »So hat der HERR, der Gott Israels, gesprochen:
Sagt dem Manne, der euch zu mir gesandt hat: 24So hat der HERR
gesprochen: ›Wisse wohl: ich will Unglück über diesen Ort und seine
Bewohner kommen lassen, nämlich alle die Flüche, die in dem Buche
geschrieben stehen, das man dem König von Juda vorgelesen hat. 25Zur
Strafe dafür, daß sie mich verlassen und anderen Göttern geopfert haben,
um mich mit all den Machwerken ihrer Hände zum Zorn zu reizen, daher hat
sich mein Grimm über diesen Ort ergossen und wird nicht wieder
erlöschen!‹ 26Zum König von Juda aber, der euch gesandt hat, um den
HERRN zu befragen, zu dem sollt ihr sagen: So hat der HERR, der Gott
Israels, gesprochen: ›Was die Drohungen anbetrifft, die du vernommen
hast: 27weil dein Herz weich geworden ist und du dich vor Gott
gedemütigt hast, als du seine Drohungen gegen diesen Ort und gegen seine
Bewohner vernahmst, und du dich vor mir gedemütigt und deine Kleider
zerrissen und vor mir geweint hast, so habe auch ich dir Gehör
geschenkt‹ -- so lautet der Ausspruch des HERRN. 28›Darum wisse wohl:
wenn ich dich zu deinen Vätern versammle, sollst du in Frieden in deine
Grabstätte eingebracht werden, und deine Augen sollen all das Unglück,
das ich über diesen Ort und seine Bewohner bringen werde, nicht zu sehen
bekommen!‹«

\hypertarget{f-josia-schlieuxdft-den-neuen-gottesbund-im-verein-mit-den-uxe4ltesten-des-volkes-ab}{%
\paragraph{f) Josia schließt den neuen Gottesbund im Verein mit den
Ältesten des Volkes
ab}\label{f-josia-schlieuxdft-den-neuen-gottesbund-im-verein-mit-den-uxe4ltesten-des-volkes-ab}}

Als sie nun dem Könige Bericht erstattet hatten, 29sandte der König
Boten aus und ließ alle Ältesten von Juda und Jerusalem zusammenkommen.
30Hierauf ging der König zum Tempel des HERRN hinauf und mit ihm alle
Männer von Juda und die Bewohner Jerusalems, auch die Priester und die
Leviten, kurz das ganze Volk, klein und groß; und er las ihnen den
ganzen Inhalt✲ des Bundesbuches vor, das im Tempel des HERRN gefunden
worden war. 31Hierauf trat der König an seinen Standort und schloß den
Bund vor dem HERRN mit der Zusage ab, daß sie dem HERRN
nachwandeln\textless sup title=``oder: anhangen''\textgreater✲ und seine
Gebote, seine Verordnungen und seine Satzungen mit ganzem Herzen und mit
ganzer Seele beobachten wollten, um so nach den Bestimmungen des Bundes
zu handeln, die in diesem Buche geschrieben standen. 32Er ließ dann
alle, die sich in Jerusalem und Benjamin befanden, dem Bunde beitreten,
und die Einwohner von Jerusalem handelten so, wie es dem Bunde mit Gott,
dem Gott ihrer Väter, entsprach. 33So schaffte denn Josia alle
Götzengreuel aus sämtlichen Landschaften der Israeliten weg und
verpflichtete alle, die sich in Israel befanden, zur Verehrung des
HERRN, ihres Gottes. Solange Josia lebte, wichen sie nicht von der
Nachfolge des HERRN, des Gottes ihrer Väter, ab.

\hypertarget{g-josias-gesetzesstrenge-passahfeier}{%
\paragraph{g) Josias gesetzesstrenge
Passahfeier}\label{g-josias-gesetzesstrenge-passahfeier}}

\hypertarget{section-34}{%
\section{35}\label{section-34}}

1Hierauf feierte Josia dem HERRN ein Passah in Jerusalem; und zwar
schlachtete man das Passah am vierzehnten Tage des ersten Monats. 2Da
bestellte er die Priester zu ihren Obliegenheiten und weckte in ihnen
den Eifer für den Dienst im Hause des HERRN. 3Den Leviten aber, die ganz
Israel zu unterweisen hatten und die dem HERRN geheiligt waren, gebot
er: »Bringt die heilige Lade in den Tempel, den Salomo, der Sohn Davids,
der König von Israel, gebaut hat: ihr braucht sie nicht mehr auf der
Schulter zu tragen. So dient nunmehr dem HERRN, eurem Gott, und seinem
Volk Israel 4und haltet euch bereit nach euren Familien, in euren
Abteilungen\textless sup title=``=~Abteilung für
Abteilung''\textgreater✲, wie David, der König von Israel, (es
verordnet) und sein Sohn Salomo es vorgeschrieben hat. 5Stellt euch also
im Heiligtum entsprechend den Gruppen der Familien eurer Volksgenossen,
der Laien\textless sup title=``d.h. Leute aus dem Volk''\textgreater✲,
auf, und zwar (für jede Gruppe) eine Abteilung einer levitischen
Familie. 6Dann schlachtet das Passah und heiligt euch und richtet (die
Opfertiere) für eure Volksgenossen zu, so daß\textless sup title=``oder:
indem''\textgreater✲ ihr nach den Weisungen verfahrt, die der HERR durch
Mose gegeben hat.«

7Hierauf machte Josia den Leuten aus dem Volke Kleinvieh zum Geschenk,
nämlich Schaflämmer und junge Ziegenböcke, alles zu den Passahopfern für
alle Anwesenden, dreißigtausend an der Zahl, dazu dreitausend Rinder,
alles dies aus dem königlichen Besitz. 8Auch seine Obersten\textless sup
title=``oder: höchsten Beamten''\textgreater✲ spendeten dem Volke, den
Priestern und den Leviten freiwillige Gaben; nämlich Hilkia, Sacharja
und Jehiel, die Fürsten\textless sup title=``oder:
Vorsteher''\textgreater✲ des Gotteshauses, schenkten den Priestern für
die Passahopfer 2600 Stück Kleinvieh und 300 Rinder; 9und Chananja nebst
seinen Brüdern Semaja und Nethaneel sowie Hasabja, Jehiel und Josabad,
die Obersten der Leviten, spendeten den Leviten als freiwillige Gabe zu
den Passahopfern 5000 Stück Kleinvieh und 500 Rinder.

10Nach diesen Vorbereitungen für den Gottesdienst traten die Priester an
ihren Platz, ebenso die Leviten nach ihren Abteilungen, wie der König es
befohlen hatte. 11Dann schlachteten (die Leviten) das Passah, und die
Priester nahmen die Besprengungen mit dem ihnen dargereichten Blut vor,
während die Leviten (den Tieren) die Haut abzogen. 12Sie legten aber die
Stücke beiseite, welche verbrannt werden sollten, um sie den einzelnen
Gruppen der Laienfamilien zu geben, damit diese sie dem HERRN
darbrächten, wie es im mosaischen Gesetzbuch vorgeschrieben ist; und
ebenso machte man es mit den Rindern. 13Dann brieten sie das Passah
vorschriftsgemäß am Feuer, während sie die geweihten Gaben in Töpfen,
Kesseln und Schüsseln kochten und sie eilends allen Leuten aus dem Volke
hinbrachten. 14Darnach richteten sie auch für sich und für die Priester
zu; denn die Priester, die Nachkommen Aarons, hatten mit der Darbringung
der Brandopfer und der Fettstücke bis in die Nacht hinein zu tun; darum
mußten die Leviten für sich und für die Priester, die Nachkommen Aarons,
zurichten. 15Auch die Sänger, die Nachkommen Asaphs, waren auf ihrem
Posten nach der Anordnung Davids, Asaphs, Hemans und Jeduthuns, des
königlichen Sehers, und ebenso standen die Torhüter an den einzelnen
Toren; sie hatten nicht nötig, ihren Dienst zu verlassen, weil ihre
Stammesgenossen, die Leviten, für sie zurichteten.

16So war der ganze Dienst des HERRN an jenem Tage geordnet, so daß man
das Passah feierte und die Brandopfer auf dem Altar des HERRN nach der
Anordnung des Königs Josia darbrachte. 17So feierten also damals die
Israeliten, die zugegen waren, das Passah und dann das Fest der
ungesäuerten Brote sieben Tage lang. 18Es war aber ein solches Passah
wie dieses in Israel nicht gefeiert worden seit der Zeit des Propheten
Samuel, und keiner von allen Königen Israels hatte ein solches Passah
veranstaltet, wie Josia es feierte mit den Priestern und den Leviten und
ganz Juda sowie mit den Israeliten, die sich dazu eingefunden hatten,
und mit den Bewohnern Jerusalems. 19Im achtzehnten Jahre der Regierung
Josias ist dieses Passah gefeiert worden.

\hypertarget{h-necho-von-uxe4gypten-und-der-tod-josias-trauer-um-ihn}{%
\paragraph{h) Necho von Ägypten und der Tod Josias; Trauer um
ihn}\label{h-necho-von-uxe4gypten-und-der-tod-josias-trauer-um-ihn}}

20Nach allen diesen Begebenheiten, als Josia den Tempel
wiederhergestellt hatte, zog Necho, der König von Ägypten, heran, um
(dem König von Assyrien) bei Karchemis am Euphrat eine Schlacht zu
liefern, und Josia zog ihm entgegen. 21Da sandte jener Boten an ihn und
ließ ihm sagen: »Was haben wir miteinander zu schaffen, König von Juda?
Nicht gegen dich ziehe ich diesmal, sondern gegen das Herrscherhaus (von
Assyrien), mit dem ich Krieg führe, und Gott hat mir Eile geboten. Laß
also ab von Gott\textless sup title=``d.h. vom Widerstande gegen
Gott''\textgreater✲, der mit mir ist, damit er dich nicht verderbe!«
22Aber Josia ließ sich nicht zum Rückzug vor ihm bewegen, sondern faßte
den kühnen Entschluß, mit ihm zu kämpfen, ohne auf die Warnung Nechos zu
hören, die doch aus dem Munde Gottes kam. Er rückte also zum Kampf in
die Ebene von Megiddo. 23Da schossen die Bogenschützen auf den König
Josia, bis dieser seinen Dienern befahl: »Bringt mich hinweg, denn ich
bin schwer verwundet!« 24Da hoben ihn seine Diener von dem Kriegswagen
hinunter, setzten ihn auf den zweiten Wagen, den er bei sich hatte, und
brachten ihn nach Jerusalem, wo er starb und in den Gräbern seiner Väter
beigesetzt wurde. Ganz Juda und Jerusalem trauerten um Josia; 25Jeremia
aber dichtete ein Klagelied auf Josia, und alle Sänger und Sängerinnen
haben (seitdem) in ihren Klageliedern von Josia gesungen bis auf den
heutigen Tag; sie sind in Israel überall in Aufnahme gekommen und finden
sich bekanntlich in den Klageliedern aufgezeichnet.

\hypertarget{i-schluuxdfwort}{%
\paragraph{i) Schlußwort}\label{i-schluuxdfwort}}

26Die übrige Geschichte Josias aber und seine frommen, den Vorschriften
im Gesetz des HERRN entsprechenden Taten, 27überhaupt seine Geschichte
von Anfang bis zu Ende, das findet sich bekanntlich schon aufgezeichnet
im Buch der Könige von Israel und Juda.

\hypertarget{die-letzten-kuxf6nige-von-juda-untergang-des-reiches-die-babylonische-gefangenschaft-und-ihr-ende}{%
\subsubsection{17. Die letzten Könige von Juda; Untergang des Reiches;
die babylonische Gefangenschaft und ihr
Ende}\label{die-letzten-kuxf6nige-von-juda-untergang-des-reiches-die-babylonische-gefangenschaft-und-ihr-ende}}

\hypertarget{a-joahas-kuxf6nig-von-juda}{%
\paragraph{a) Joahas König von Juda}\label{a-joahas-kuxf6nig-von-juda}}

\hypertarget{section-35}{%
\section{36}\label{section-35}}

1Die Landbevölkerung nahm dann Joahas, den Sohn Josias, und machte ihn
zum König in Jerusalem als Nachfolger seines Vaters. 2Im Alter von
dreiundzwanzig Jahren wurde Joahas König und regierte drei Monate in
Jerusalem; 3dann setzte ihn der König von Ägypten in Jerusalem ab und
legte dem Lande eine Geldbuße von hundert Talenten Silber und einem
Talent Gold auf.

\hypertarget{b-jojakim-kuxf6nig-von-juda}{%
\paragraph{b) Jojakim König von
Juda}\label{b-jojakim-kuxf6nig-von-juda}}

4Dann machte der König von Ägypten den Bruder des Joahas, nämlich
Eljakim, zum König über Juda und Jerusalem und änderte seinen Namen in
Jojakim ab; seinen Bruder Joahas aber nahm Necho mit sich und brachte
ihn nach Ägypten.

5Im Alter von fünfundzwanzig Jahren wurde Jojakim König und regierte elf
Jahre in Jerusalem; er tat, was dem HERRN, seinem Gott, mißfiel. 6Da zog
Nebukadnezar, der König von Babylon, gegen ihn heran und legte ihn in
Ketten, um ihn nach Babylon bringen zu lassen. 7Auch einen Teil der
Geräte des Tempels des HERRN entführte Nebukadnezar nach Babylon und
brachte sie in seinem Tempel\textless sup title=``oder:
Palast''\textgreater✲ zu Babylon unter.~-- 8Die übrige Geschichte
Jojakims aber und die Greueltaten, die er begangen hat, und was sonst
noch Böses bei ihm vorgekommen ist, das findet sich bekanntlich bereits
aufgezeichnet im Buch der Könige von Israel und Juda. Sein Sohn Jojachin
folgte ihm dann in der Regierung nach.

\hypertarget{c-jojachin-kuxf6nig-von-juda}{%
\paragraph{c) Jojachin König von
Juda}\label{c-jojachin-kuxf6nig-von-juda}}

9Im Alter von achtzehn Jahren wurde Jojachin König und regierte drei
Monate und zehn Tage in Jerusalem; er tat, was dem HERRN mißfiel. 10Um
die Jahreswende\textless sup title=``oder: vor Ablauf des
Jahres''\textgreater✲ aber sandte der König Nebukadnezar hin und ließ
ihn nach Babylon holen samt den kostbarsten Geräten des Tempels des
HERRN und machte seinen Bruder Zedekia zum König über Juda und
Jerusalem.

\hypertarget{d-zedekia-kuxf6nig-von-juda-sein-und-seines-volkes-untergang}{%
\paragraph{d) Zedekia König von Juda; sein und seines Volkes
Untergang}\label{d-zedekia-kuxf6nig-von-juda-sein-und-seines-volkes-untergang}}

\hypertarget{aa-eingangswort-die-allgemeine-gottlosigkeit-zedekias-abfall-von-nebukadnezar}{%
\subparagraph{aa) Eingangswort; die allgemeine Gottlosigkeit; Zedekias
Abfall von
Nebukadnezar}\label{aa-eingangswort-die-allgemeine-gottlosigkeit-zedekias-abfall-von-nebukadnezar}}

11Im Alter von einundzwanzig Jahren wurde Zedekia König und regierte elf
Jahre in Jerusalem. 12Er tat, was dem HERRN, seinem Gott, mißfiel; er
demütigte sich nicht vor Jeremia, der als Prophet auf Befehl des HERRN
zu ihm redete. 13Dazu empörte er sich gegen den König Nebukadnezar, der
ihn doch einen Treueid bei Gott hatte schwören lassen, und machte seinen
Nacken steif und sein Herz verstockt, so daß er sich nicht zum HERRN,
dem Gott Israels, bekehrte. 14Ebenso begingen alle höchstgestellten
Priester und die Häupter des Volkes Sünden auf Sünden in der Weise✲ der
heidnischen Götzengreuel und entweihten das Haus, das der HERR in
Jerusalem geheiligt hatte. 15Zwar sandte der HERR, der Gott ihrer Väter,
durch seine Boten unermüdlich immer wieder (Warnungen) an sie, weil er
Erbarmen mit seinem Volke fühlte und seine Wohnstätte ihm leid tat;
16aber sie verhöhnten die Boten Gottes und verachteten seine Drohungen
und trieben ihren Spott mit seinen Propheten, bis der Grimm des HERRN
gegen sein Volk so hoch stieg, daß keine Heilung✲ mehr möglich war.

\hypertarget{bb-vernichtung-des-reiches-durch-nebukadnezar-die-babylonische-gefangenschaft}{%
\subparagraph{bb) Vernichtung des Reiches durch Nebukadnezar; die
babylonische
Gefangenschaft}\label{bb-vernichtung-des-reiches-durch-nebukadnezar-die-babylonische-gefangenschaft}}

17So ließ er denn den König der Chaldäer gegen sie heranziehen; der
erschlug ihre junge Mannschaft mit dem Schwert in ihrem heiligen Tempel:
er verschonte weder Jünglinge noch Jungfrauen, nicht Greise noch
Hochbetagte: alles ließ Gott ihm in die Hände fallen. 18Auch sämtliche
Geräte des Gotteshauses, die großen wie die kleinen, und die Schätze des
Tempels des HERRN und die Schätze des Königs und seiner Würdenträger:
alles entführte er nach Babylon. 19Den Tempel aber verbrannten sie, die
Mauern Jerusalems rissen sie nieder und ließen alle Paläste der Stadt in
Flammen aufgehen, so daß alle kostbaren Geräte darin zugrunde gingen.
20Hierauf führte er alle, die dem Blutbad entgangen waren, gefangen nach
Babylon, wo sie ihm und seinen Söhnen als Knechte dienstbar waren, bis
das Perserreich zur Herrschaft gelangte. 21So sollte das Wort des HERRN,
das durch den Mund Jeremias ausgesprochen worden war\textless sup
title=``Jer 25,11; 3.Mose 26,34''\textgreater✲, seine Erfüllung finden:
»Bis das Land seine Sabbatjahre abgetragen hätte.« Während der ganzen
Zeit seiner Verödung hatte es Ruhe, bis siebzig Jahre voll waren.

\hypertarget{e-die-heimkehrerlaubnis-des-perserkuxf6nigs-cyrus}{%
\paragraph{e) Die Heimkehrerlaubnis des Perserkönigs
Cyrus}\label{e-die-heimkehrerlaubnis-des-perserkuxf6nigs-cyrus}}

22Aber im ersten Jahre der Regierung des Kores✲, des Königs von Persien
-- damit das durch den Mund Jeremias ergangene Wort des HERRN in
Erfüllung ginge --, regte der HERR den Geist des Perserkönigs Kores dazu
an, folgende Verfügung durch sein ganzes Reich hin ausrufen und auch
durch schriftlichen Erlaß bekanntmachen zu lassen: 23»So
spricht\textless sup title=``=~Folgendes verfügt''\textgreater✲ Kores,
der König von Persien: Alle Reiche der Erde hat mir der HERR, der Gott
des Himmels, gegeben, und er ist's auch, der mir aufgetragen hat, ihm zu
Jerusalem in Juda einen Tempel zu erbauen. Wer also unter euch allen zu
seinem Volke gehört, mit dem sei sein Gott, und er ziehe hinauf!«
