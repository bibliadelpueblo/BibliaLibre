\hypertarget{der-erste-brief-des-apostels-petrus}{%
\section{DER ERSTE BRIEF DES APOSTELS
PETRUS}\label{der-erste-brief-des-apostels-petrus}}

\hypertarget{zuschrift-und-segenswunsch}{%
\subsubsection{Zuschrift und
Segenswunsch}\label{zuschrift-und-segenswunsch}}

\hypertarget{section}{%
\section{1}\label{section}}

\bibleverse{1} Ich, Petrus, ein Apostel Jesu Christi, entbiete meinen
Gruß den Fremdlingen\textless sup title=``=~fremden
Gemeindegenossen''\textgreater✲, die in Pontus, Galatien, Kappadozien,
(der römischen Provinz) Asien und Bithynien in der Zerstreuung leben
\bibleverse{2} und nach der Vorersehung Gottes des Vaters dazu
auserwählt sind, in der Heiligung\textless sup title=``oder: durch die
Heiligung''\textgreater✲ des Geistes zum Gehorsam und zur Besprengung
mit dem Blute Jesu Christi (zu gelangen): Gnade und Friede möge euch
immer reichlicher zuteil werden!

\hypertarget{eingang-lobpreis-gottes-fuxfcr-unsere-wiedergeburt-und-fuxfcr-die-herrlichkeit-der-christenhoffnung-und-des-christlichen-heils}{%
\subsubsection{1. Eingang: Lobpreis Gottes für unsere Wiedergeburt und
für die Herrlichkeit der Christenhoffnung und des christlichen
Heils}\label{eingang-lobpreis-gottes-fuxfcr-unsere-wiedergeburt-und-fuxfcr-die-herrlichkeit-der-christenhoffnung-und-des-christlichen-heils}}

\hypertarget{a-das-uns-geschenkte-heil-gibt-jubelnde-freude-wenn-auch-der-glaube-sich-in-truxfcbsal-bewuxe4hren-muuxdf}{%
\paragraph{a) Das uns geschenkte Heil gibt jubelnde Freude, wenn auch
der Glaube sich in Trübsal bewähren
muß}\label{a-das-uns-geschenkte-heil-gibt-jubelnde-freude-wenn-auch-der-glaube-sich-in-truxfcbsal-bewuxe4hren-muuxdf}}

\bibleverse{3} Gelobt sei der Gott und Vater unsers Herrn Jesus
Christus, der nach seiner großen Barmherzigkeit uns wiedergeboren hat zu
einer lebendigen\textless sup title=``oder: lebensvollen''\textgreater✲
Hoffnung durch die Auferstehung Jesu Christi von den Toten,
\bibleverse{4} zu einem unvergänglichen, unbefleckten und
unverwelklichen Erbe, das im Himmel aufbewahrt ist für euch,
\bibleverse{5} die ihr in der Kraft Gottes durch den Glauben für die
Errettung\textless sup title=``oder: das Heil =~die
Seligkeit''\textgreater✲ bewahrt werdet, die (schon jetzt) bereitsteht,
um in der letzten Zeit geoffenbart zu werden. \bibleverse{6} Darüber
jubelt ihr, mögt ihr jetzt auch eine kurze Zeit\textless sup
title=``oder: ein wenig''\textgreater✲, wenn es so sein muß, durch
mancherlei Anfechtung in Trübsal versetzt sein; \bibleverse{7} dadurch
soll sich ja die Echtheit eures Glaubens bewähren und wertvoller
erfunden werden als Gold, das vergänglich ist, aber durch Feuer in
seiner Echtheit erprobt wird, und sich (euch) zum Lobe, zur Ehre und zur
Verherrlichung bei der Offenbarung Jesu Christi erweisen. \bibleverse{8}
Ihn habt ihr lieb, obgleich ihr ihn nicht gesehen habt; an ihn glaubt
ihr\textless sup title=``oder: auf ihn setzt ihr euer
Vertrauen''\textgreater✲, obgleich ihr ihn jetzt nicht seht, und ihm
jubelt ihr mit unaussprechlicher und verklärter Freude entgegen,
\bibleverse{9} weil\textless sup title=``oder: indem''\textgreater✲ ihr
das Endziel eures Glaubens davontragt, nämlich die Errettung eurer
Seelen.

\hypertarget{b-das-von-den-propheten-vor-zeiten-vielfach-verheiuxdfene-heil-ist-nunmehr-endguxfcltig-verwirklicht-worden}{%
\paragraph{b) Das von den Propheten vor Zeiten vielfach verheißene Heil
ist nunmehr endgültig verwirklicht
worden}\label{b-das-von-den-propheten-vor-zeiten-vielfach-verheiuxdfene-heil-ist-nunmehr-endguxfcltig-verwirklicht-worden}}

\bibleverse{10} In betreff dieser Errettung haben die Propheten
nachgesonnen und nachgeforscht, die von der euch zugedachten Gnade
geweissagt haben, \bibleverse{11} indem sie ausfindig zu machen suchten,
welche oder was für eine Zeit es sei, auf welche der in ihnen wirkende
Geist Christi hinwies, wenn er ihnen die für Christus bestimmten Leiden
und seine darauf folgenden Verherrlichungen im voraus bezeugte.
\bibleverse{12} Dabei wurde ihnen geoffenbart, daß sie durch ihren
Dienst nicht sich selbst, sondern euch eben das vermitteln sollten, was
euch jetzt durch die Männer verkündigt worden ist, die euch die
Heilsbotschaft in der Kraft des vom Himmel hergesandten heiligen Geistes
gepredigt haben: Dinge, in welche auch die Engel gern
hineinschauen\textless sup title=``oder: einen Einblick
gewinnen''\textgreater✲ möchten.

\hypertarget{erste-reihe-von-allgemeinen-ermahnungen-wandelt-dem-gesteckten-ziel-zu}{%
\subsubsection{2. Erste Reihe von (allgemeinen) Ermahnungen: Wandelt dem
gesteckten Ziel
zu}\label{erste-reihe-von-allgemeinen-ermahnungen-wandelt-dem-gesteckten-ziel-zu}}

\hypertarget{a-wandelt-in-heiliger-furcht-vor-gott-in-freudigem-vertrauen-auf-das-durch-die-erluxf6sung-bewirkte-heil-und-in-hoffnung-auf-die-herrlichkeit}{%
\paragraph{a) Wandelt in heiliger Furcht vor Gott in freudigem Vertrauen
auf das durch die Erlösung bewirkte Heil und in Hoffnung auf die
Herrlichkeit}\label{a-wandelt-in-heiliger-furcht-vor-gott-in-freudigem-vertrauen-auf-das-durch-die-erluxf6sung-bewirkte-heil-und-in-hoffnung-auf-die-herrlichkeit}}

\bibleverse{13} Darum macht euch geistlich fertig zum rüstigen
Vorwärtsschreiten, seid nüchtern und setzt eure Hoffnung ausschließlich
auf die Gnade, die euch in der Offenbarung\textless sup title=``oder:
beim Offenbarwerden''\textgreater✲ Jesu Christi dargeboten wird.
\bibleverse{14} Als gehorsame (Gottes-) Kinder gestaltet euer Leben
nicht nach den Lüsten, die ihr früher während (der Zeit) eurer
Unwissenheit gehegt habt, \bibleverse{15} sondern werdet nach dem
Vorbild des Heiligen, der euch berufen hat, gleichfalls in eurem ganzen
Wandel heilig, \bibleverse{16} weil ja doch geschrieben
steht\textless sup title=``3.Mose 11,44; 19,2''\textgreater✲: »Ihr sollt
heilig sein, denn ich bin heilig!« \bibleverse{17} Und wenn ihr den als
Vater anruft, der ohne Ansehen der Person nach dem Werk✲ eines jeden
richtet, so führet euren Wandel in Furcht während der Zeit eurer
Fremdlingschaft; \bibleverse{18} ihr wißt ja, daß ihr von eurem eitlen
Wandel, den ihr von den Vätern her überkommen hattet, nicht mit
vergänglichen Dingen✲, mit Silber oder Gold, losgekauft worden seid,
\bibleverse{19} sondern mit dem kostbaren Blute Christi als eines
fehllosen und unbefleckten Lammes. \bibleverse{20} Er war zwar schon vor
Grundlegung der Welt zuvorersehen, ist aber erst am Ende der Zeiten
geoffenbart worden euch zugute; \bibleverse{21} denn durch ihn seid ihr
zum Glauben an Gott gekommen, der ihn von den Toten auferweckt und ihm
Herrlichkeit verliehen hat, so daß euer Glaube zugleich Hoffnung auf
Gott ist.

\hypertarget{b-wandelt-innerhalb-der-gemeinde-in-lauterer-bruderliebe-und-wuxfcrdig-des-euch-geschenkten-neuen-lebens}{%
\paragraph{b) Wandelt innerhalb der Gemeinde in lauterer Bruderliebe und
würdig des euch geschenkten neuen
Lebens}\label{b-wandelt-innerhalb-der-gemeinde-in-lauterer-bruderliebe-und-wuxfcrdig-des-euch-geschenkten-neuen-lebens}}

\bibleverse{22} Da\textless sup title=``oder: nachdem''\textgreater✲ ihr
eure Seelen im Gehorsam gegen die Wahrheit zu ungeheuchelter Bruderliebe
gereinigt✲ habt, so liebet einander innig\textless sup title=``oder:
beharrlich''\textgreater✲ von Herzen; \bibleverse{23} ihr seid ja nicht
aus vergänglichem, sondern aus unvergänglichem Samen
wiedergeboren\textless sup title=``oder: neugeboren''\textgreater✲,
nämlich durch das lebendige und ewigbleibende Wort Gottes.
\bibleverse{24} Denn »alles Fleisch ist wie Gras und alle seine
Herrlichkeit wie des Grases Blume; das Gras verdorrt und seine Blume
fällt ab, \bibleverse{25} das Wort des Herrn aber bleibt in
Ewigkeit«\textless sup title=``Jes 40,6-8''\textgreater✲. Dies ist aber
das Wort, das euch als Heilsbotschaft verkündigt worden ist.

\hypertarget{c-schreitet-fort-in-der-heiligung-und-erbaut-euch-als-lebendige-steine-auf-christus-den-eckstein-zu-einem-geistlichen-priestervolk}{%
\paragraph{c) Schreitet fort in der Heiligung und erbaut euch als
lebendige Steine auf Christus, den Eckstein, zu einem geistlichen
Priestervolk}\label{c-schreitet-fort-in-der-heiligung-und-erbaut-euch-als-lebendige-steine-auf-christus-den-eckstein-zu-einem-geistlichen-priestervolk}}

\hypertarget{section-1}{%
\section{2}\label{section-1}}

\bibleverse{1} So legt also alle Bosheit und alle Falschheit, die
Heuchelei, den Neid und alle Verleumdungssucht ab \bibleverse{2} und
tragt wie neugeborene Kinder nach der geistigen\textless sup
title=``oder: im Wort Gottes dargebotenen''\textgreater✲ lauteren Milch
Verlangen, damit ihr durch sie zur Errettung heranwachst, \bibleverse{3}
wenn ihr wirklich »geschmeckt habt, daß der Herr freundlich
ist«\textless sup title=``Ps 34,9''\textgreater✲. \bibleverse{4} Wenn
ihr zu ihm, dem lebendigen Stein, herantretet, der von den Menschen zwar
als unbrauchbar verworfen, bei Gott aber als ein auserwähltes Kleinod
gilt, \bibleverse{5} so werdet auch ihr selbst als lebendige Bausteine
zu einem geistlichen Hause, zu einer heiligen Priesterschaft aufgebaut,
um geistliche\textless sup title=``=~durch den Geist
gewirkte''\textgreater✲ Opfer darzubringen, die Gott durch Jesus
Christus wohlgefällig sind. \bibleverse{6} In der Schrift heißt es
ja\textless sup title=``Jes 28,16''\textgreater✲: »Seht, ich lege in
Zion einen auserwählten Stein, einen kostbaren Eckstein; und wer auf ihn
sein Vertrauen setzt\textless sup title=``oder: seinen Glauben
baut''\textgreater✲, wird nimmermehr zuschanden✲ werden.« \bibleverse{7}
Euch also, die ihr Vertrauen\textless sup title=``oder:
Glauben''\textgreater✲ besitzt, wird das kostbare Gut zuteil; für die
Ungläubigen aber ist »der Stein, den die Bauleute verworfen haben --
gerade der ist zum Eckstein geworden«\textless sup title=``Ps
118,22''\textgreater✲ \bibleverse{8} und damit »zu einem Stein des
Anstoßes« und »zum Felsen des Ärgernisses\textless sup title=``d.h. an
dem man zu Fall kommt''\textgreater✲«\textless sup title=``Jes
8,14''\textgreater✲; sie stoßen sich an ihm in ihrem Ungehorsam gegen
das Wort, wozu sie auch bestimmt sind. \bibleverse{9} Ihr dagegen seid
»das auserwählte Geschlecht, die königliche Priesterschaft, die heilige
Volksgemeinschaft, das zum Eigentum erkorene Volk«, und sollt die
Tugenden\textless sup title=``d.h. Ruhmestaten; vgl. Jes
43,21''\textgreater✲ dessen verkünden, der euch aus der Finsternis in
sein wunderbares Licht berufen hat, \bibleverse{10} euch, die ihr vordem
»ein Nicht-Volk\textless sup title=``=~kein Volk''\textgreater✲« waret,
jetzt aber »das Volk Gottes« seid, einst »ohne Gottes Erbarmen«, jetzt
aber »reich an Gotteserbarmen«\textless sup title=``vgl. Hos 1,6.9;
2,25''\textgreater✲.

\hypertarget{zweite-reihe-von-besonderen-ermahnungen-fuxfcr-einzelne-lebensverhuxe4ltnisse}{%
\subsubsection{3. Zweite Reihe von (besonderen) Ermahnungen für einzelne
Lebensverhältnisse}\label{zweite-reihe-von-besonderen-ermahnungen-fuxfcr-einzelne-lebensverhuxe4ltnisse}}

\hypertarget{a-allgemeine-aufforderung-zu-einem-reinen-wandel-vor-den-heiden-ungluxe4ubigen}{%
\paragraph{a) Allgemeine Aufforderung zu einem reinen Wandel vor den
Heiden
(=~Ungläubigen)}\label{a-allgemeine-aufforderung-zu-einem-reinen-wandel-vor-den-heiden-ungluxe4ubigen}}

\bibleverse{11} Geliebte, ich ermahne euch: Enthaltet euch, da ihr ja
»Fremdlinge und Beisassen\textless sup title=``oder:
Gäste''\textgreater✲« seid\textless sup title=``Ps 39,13''\textgreater✲,
der fleischlichen Begierden, die im Kampf gegen die Seele liegen;
\bibleverse{12} führt einen guten✲ Wandel unter den Heiden, damit sie in
allem, worin sie euch (jetzt) als Übeltäter verlästern, bei genauer
Prüfung auf Grund eurer guten✲ Werke Gott preisen am »Tage der
Gnadenheimsuchung«\textless sup title=``Jes 10,3''\textgreater✲.

\hypertarget{b-ermahnung-zum-gehorsam-gegen-die-heidnische-obrigkeit}{%
\paragraph{b) Ermahnung zum Gehorsam gegen die heidnische
Obrigkeit}\label{b-ermahnung-zum-gehorsam-gegen-die-heidnische-obrigkeit}}

\bibleverse{13} Seid jeder menschlichen Ordnung um des Herrn willen
untertan, es sei dem König\textless sup title=``oder:
Kaiser''\textgreater✲ als dem obersten Herrn \bibleverse{14} oder den
Statthaltern als denen, die von ihm zur Bestrafung der Übeltäter und
Belobigung\textless sup title=``=~lobenden Anerkennung''\textgreater✲
der recht Handelnden entsandt werden. \bibleverse{15} Denn so ist es der
Wille Gottes, daß ihr durch Gutestun\textless sup title=``oder: gutes
Verhalten''\textgreater✲ den Unverstand der törichten Menschen zum
Schweigen bringt, \bibleverse{16} und zwar als (wahrhaft) Freie und
nicht als solche, welche die Freiheit zum Deckmantel der Bosheit machen,
sondern als Knechte Gottes. \bibleverse{17} Erweiset jedermann die
schuldige Ehre, habt die Brüder lieb, »fürchtet Gott, ehret den
König\textless sup title=``oder: Kaiser''\textgreater✲«!\textless sup
title=``Spr 24,21''\textgreater✲

\hypertarget{c-ermahnungen-an-die-sklaven-zum-dulden-nach-christi-vorbild}{%
\paragraph{c) Ermahnungen an die Sklaven zum Dulden nach Christi
Vorbild}\label{c-ermahnungen-an-die-sklaven-zum-dulden-nach-christi-vorbild}}

\bibleverse{18} Ihr Dienstleute\textless sup title=``=~Gesinde,
Sklaven''\textgreater✲, seid in aller Furcht euren Herren untertan,
nicht nur den gütigen und nachsichtigen, sondern auch den
verkehrten\textless sup title=``oder: wunderlichen''\textgreater✲;
\bibleverse{19} denn das ist Gnade\textless sup title=``=~wohlgefällig
bei Gott''\textgreater✲, wenn jemand im Gedanken an Gott Trübsale✲
geduldig erträgt, sofern er unschuldig leidet. \bibleverse{20} Denn was
ist das für ein Ruhm, wenn ihr (die Schläge) geduldig aushaltet, wo ihr
euch vergeht und dann gezüchtigt werdet? Aber wenn ihr geduldig
aushaltet, wo ihr trotz eures guten Verhaltens leiden müßt, das ist
Gnade✲ bei Gott. \bibleverse{21} Denn dazu seid ihr berufen worden, weil
auch Christus für euch gelitten und euch (dadurch) ein Vorbild
hinterlassen hat, damit ihr seinen Fußtapfen nachfolget. \bibleverse{22}
Er hat keine Sünde getan, auch ist kein Trug in seinem Munde gefunden
worden; \bibleverse{23} er hat, wenn er geschmäht wurde, nicht wieder
geschmäht und, als er litt, keine Drohungen ausgestoßen, sondern es dem
anheimgestellt, der gerecht richtet. \bibleverse{24} Er hat unsere
Sünden selber mit seinem Leibe an das (Marter-) Holz hinaufgetragen,
damit wir, von den Sünden freigemacht\textless sup title=``oder: den
Sünden abgestorben''\textgreater✲, der Gerechtigkeit leben möchten:
durch seine Wunden\textless sup title=``=~sein blutiges
Leiden''\textgreater✲ seid ihr geheilt worden\textless sup title=``Jes
53,5''\textgreater✲. \bibleverse{25} Denn ihr ginget (einst) wie Schafe
in der Irre; jetzt aber seid ihr zu dem Hirten und Hüter eurer Seelen
bekehrt worden.

\hypertarget{d-ermahnungen-fuxfcr-die-ehegatten}{%
\paragraph{d) Ermahnungen für die
Ehegatten}\label{d-ermahnungen-fuxfcr-die-ehegatten}}

\hypertarget{section-2}{%
\section{3}\label{section-2}}

\bibleverse{1} Ebenso, ihr Frauen: seid euren Ehemännern untertan, damit
auch solche (Männer), die dem Wort ungehorsam sind\textless sup
title=``=~nicht glauben wollen''\textgreater✲, durch den Wandel ihrer
Frauen auch ohne Wort gewonnen werden, \bibleverse{2} wenn sie euren in
Gottesfurcht sittsamen Wandel wahrnehmen. \bibleverse{3} Euer Schmuck
sei nicht der äußerliche, nicht kunstvolles Haargeflecht und das Anlegen
goldenen Geschmeides oder das Anziehen prächtiger Gewänder,
\bibleverse{4} sondern der im Herzen\textless sup title=``=~tief
innerlich''\textgreater✲ verborgene Mensch mit dem unvergänglichen Wesen
eines sanften und stillen Geistes\textless sup title=``oder:
Sinnes''\textgreater✲, der vor Gott als kostbar gilt. \bibleverse{5} So
haben sich ja einst auch die heiligen Frauen geschmückt, die ihre
Hoffnung auf Gott setzten, indem sie sich ihren Ehemännern untertan
bewiesen. \bibleverse{6} So hat sich z.B. Sara dem Abraham gehorsam
gezeigt, indem sie ihn »Herr« nannte\textless sup title=``1.Mose
18,12''\textgreater✲. Ihre Kinder✲ seid ihr geworden, wenn ihr das Gute
tut und euch durch keine Drohung einschüchtern laßt.

\bibleverse{7} Ebenso, ihr Männer: lebt in vernünftiger Weise mit euren
Frauen zusammen als mit dem schwächeren Teil\textless sup title=``eig.
Gefäß''\textgreater✲ und erweist ihnen (die schuldige) Ehre, indem ihr
in ihnen auch Miterben der Gnadengabe des (ewigen) Lebens seht; sonst
würden ja eure (gemeinsamen) Gebete unmöglich gemacht.

\hypertarget{e-allgemeine-ermahnungen-fuxfcr-die-gemeindeglieder}{%
\paragraph{e) Allgemeine Ermahnungen für die
Gemeindeglieder}\label{e-allgemeine-ermahnungen-fuxfcr-die-gemeindeglieder}}

\bibleverse{8} Schließlich aber: seid alle einträchtig, voll Mitgefühl
und Bruderliebe, barmherzig und demütig! \bibleverse{9} Vergeltet nicht
Böses mit Bösem oder Scheltwort mit Scheltwort, sondern im Gegenteil
segnet, denn dazu seid ihr berufen, damit ihr Segen ererbet.
\bibleverse{10} Denn »wer seines Lebens froh werden will und gute Tage
zu sehen wünscht, der halte seine Zunge vom Bösen fern und seine Lippen,
daß sie nicht Trug reden; \bibleverse{11} er wende sich vom Bösen ab und
tue das Gute, er suche Frieden und jage ihm nach! \bibleverse{12} Denn
die Augen des Herrn (sind) auf die Gerechten (hingewandt), und seine
Ohren (achten) auf ihr Flehen; dagegen ist das Angesicht des Herrn gegen
die Übeltäter (gerichtet).«\textless sup title=``Ps
34,13-17''\textgreater✲

\hypertarget{f-vom-segen-des-willigen-unrechtleidens}{%
\paragraph{f) Vom Segen des willigen
Unrechtleidens}\label{f-vom-segen-des-willigen-unrechtleidens}}

\hypertarget{aa-im-leiden-seid-ihr-zeugen-fuxfcr-eure-umgebung}{%
\subparagraph{aa) Im Leiden seid ihr Zeugen für eure
Umgebung}\label{aa-im-leiden-seid-ihr-zeugen-fuxfcr-eure-umgebung}}

\bibleverse{13} Und wo ist jemand, der euch Böses zufügen
sollte\textless sup title=``oder: wollte''\textgreater✲, wenn ihr dem
Guten eifrig nachtrachtet? \bibleverse{14} Doch müßtet ihr um der
Gerechtigkeit willen auch leiden: selig seid ihr zu preisen! So fürchtet
euch denn nicht vor ihnen und laßt euch nicht erschrecken!
\bibleverse{15} Haltet nur den Herrn Christus in euren Herzen heilig und
seid allezeit bereit, euch gegen jedermann zu verantworten, der von euch
Rechenschaft über die Hoffnung fordert, die in euch lebt;
\bibleverse{16} tut es jedoch mit Sanftmut und Furcht, so daß ihr euch
ein gutes Gewissen bewahrt, damit die, welche euren guten Wandel in
Christus schmähen, mit ihren Verleumdungen gegen euch zuschanden✲
werden. \bibleverse{17} Es ist ja doch besser, wenn Gottes Wille es so
fügen sollte, für Gutestun zu leiden als für Bösestun.

\hypertarget{bb-die-segensreichen-folgen-des-unverschuldeten-leidens-christi-seine-heilsbotschaft-im-totenreich-die-bedeutung-der-taufe}{%
\subparagraph{bb) Die segensreichen Folgen des unverschuldeten Leidens
Christi (seine Heilsbotschaft im Totenreich; die Bedeutung der
Taufe)}\label{bb-die-segensreichen-folgen-des-unverschuldeten-leidens-christi-seine-heilsbotschaft-im-totenreich-die-bedeutung-der-taufe}}

\bibleverse{18} Denn auch Christus ist einmal um der Sünden willen
gestorben, als Gerechter für Ungerechte, um uns zu Gott zu führen, er,
der am\textless sup title=``oder: nach dem''\textgreater✲ Fleisch✲ zwar
getötet worden ist, aber zum Leben erweckt am\textless sup title=``oder:
nach dem''\textgreater✲ Geist\textless sup title=``Röm
1,4''\textgreater✲.

\bibleverse{19} Im Geist\textless sup title=``=~als Geist''\textgreater✲
ist er auch hingegangen und hat den Geistern im Gefängnis
gepredigt\textless sup title=``=~die Heilsbotschaft
verkündigt''\textgreater✲, \bibleverse{20} nämlich denen, welche einst
ungehorsam gewesen waren, als Gottes Langmut geduldig wartete in den
Tagen Noahs, während die Arche hergestellt wurde, in der nur wenige,
nämlich acht Seelen, Rettung fanden durchs Wasser hindurch.
\bibleverse{21} Dieses (Wasser) rettet jetzt als Gegenstück\textless sup
title=``oder: gegenbildlich''\textgreater✲ auch euch, nämlich die Taufe,
die nicht eine Beseitigung des Schmutzes am Fleisch ist, sondern eine an
Gott gerichtete Bitte um ein gutes Gewissen; (sie rettet euch) kraft der
Auferstehung Jesu Christi, \bibleverse{22} der nach seiner Himmelfahrt
zur Rechten Gottes sitzt: Engel, Gewalten und Mächte sind ihm untertan
geworden.

\hypertarget{cc-leidenswilligkeit-widersteht-dem-sinn-zum-suxfcndigen-duxe4mpft-die-luxfcste-und-verhilft-zu-einem-gottergebenen-wandel}{%
\subparagraph{cc) Leidenswilligkeit widersteht dem Sinn zum Sündigen,
dämpft die Lüste und verhilft zu einem gottergebenen
Wandel}\label{cc-leidenswilligkeit-widersteht-dem-sinn-zum-suxfcndigen-duxe4mpft-die-luxfcste-und-verhilft-zu-einem-gottergebenen-wandel}}

\hypertarget{section-3}{%
\section{4}\label{section-3}}

\bibleverse{1} Weil nun Christus am Fleisch✲ gelitten hat, so wappnet
auch ihr euch mit der gleichen Gesinnung -- denn wer leiblich gelitten
hat, ist damit zur Ruhe vor der Sünde gekommen --, \bibleverse{2} damit
ihr die noch übrige Zeit eures leiblichen Daseins\textless sup
title=``=~eures Erdenlebens''\textgreater✲ nicht mehr im Dienst
menschlicher Lüste, sondern nach dem Willen Gottes verlebt.
\bibleverse{3} Denn lang genug ist die vergangene Zeit, in der ihr den
Willen der Heiden vollbracht habt, indem ihr in Ausschweifungen und
Lüsten, in Trunkenheit, Schmausereien, Zechgelagen und verwerflichem
Götzendienst dahingelebt habt. \bibleverse{4} Darum befremdet es sie
jetzt, daß ihr euch nicht mehr mit ihnen in denselben Schlamm der
Liederlichkeit stürzt, und deshalb schmähen sie euch; \bibleverse{5}
doch sie werden sich vor dem zu verantworten haben, der sich bereithält,
Lebende und Tote zu richten. \bibleverse{6} Denn dazu ist auch Toten die
Heilsbotschaft verkündigt worden, daß sie, wenn sie auch leiblich, dem
menschlichen Lose entsprechend, dem Gericht verfallen sind, doch im
Geist\textless sup title=``oder: dem Geiste nach''\textgreater✲, dem
Wesen Gottes entsprechend, das Leben haben (sollen).

\hypertarget{dritte-reihe-von-ermahnungen-fuxfcr-gemeindeangelegenheiten-und-zur-glaubensfestigkeit-in-leiden}{%
\subsubsection{4. Dritte Reihe von Ermahnungen für
Gemeindeangelegenheiten und zur Glaubensfestigkeit in
Leiden}\label{dritte-reihe-von-ermahnungen-fuxfcr-gemeindeangelegenheiten-und-zur-glaubensfestigkeit-in-leiden}}

\hypertarget{a-mahnung-zur-bewuxe4hrung-christlichen-gemeinschaftslebens-im-hinblick-auf-das-nahe-weltende}{%
\paragraph{a) Mahnung zur Bewährung christlichen Gemeinschaftslebens im
Hinblick auf das nahe
Weltende}\label{a-mahnung-zur-bewuxe4hrung-christlichen-gemeinschaftslebens-im-hinblick-auf-das-nahe-weltende}}

\bibleverse{7} Das Ende aller Dinge steht nahe bevor. Werdet also
besonnen und nüchtern zum Gebet; \bibleverse{8} vor allem aber hegt
innige Liebe zueinander, denn »die Liebe deckt der Sünden Menge
zu«\textless sup title=``Spr 10,12; Jak 5,20''\textgreater✲.
\bibleverse{9} Seid gastfrei gegeneinander ohne Murren. \bibleverse{10}
Dienet einander, ein jeder mit der Gnadengabe, die er empfangen hat, als
gute Verwalter der mannigfachen Gnadengaben Gottes! \bibleverse{11}
Redet jemand, so seien seine Worte wie Aussprüche\textless sup
title=``=~wie die eines Sprechers''\textgreater✲ Gottes; hat jemand
Dienste (als Diakon) zu leisten, so (tue er es) in der Kraft, die Gott
verleiht, damit in allen Fällen Gott verherrlicht werde durch Jesus
Christus: sein ist die Herrlichkeit\textless sup title=``oder:
Ehre''\textgreater✲ und die Macht in alle Ewigkeit. Amen.

\hypertarget{b-mahnung-zur-bewuxe4hrung-echten-christensinnes-im-luxe4uterungsfeuer-der-leiden-im-hinblick-auf-die-zu-erlangende-herrlichkeit}{%
\paragraph{b) Mahnung zur Bewährung echten Christensinnes im
Läuterungsfeuer der Leiden im Hinblick auf die zu erlangende
Herrlichkeit}\label{b-mahnung-zur-bewuxe4hrung-echten-christensinnes-im-luxe4uterungsfeuer-der-leiden-im-hinblick-auf-die-zu-erlangende-herrlichkeit}}

\bibleverse{12} Geliebte, laßt die Feuerglut (der Leiden), die zur
Prüfung über euch ergeht, nicht befremdlich auf euch wirken, als ob euch
damit etwas Unbegreifliches widerführe, \bibleverse{13} sondern freuet
euch darüber in dem Maße, wie ihr an den Leiden Christi Anteil bekommt,
damit ihr auch bei der Offenbarung seiner Herrlichkeit euch freuen und
jubeln könnt. \bibleverse{14} Wenn ihr um des Namens Christi willen
geschmäht werdet, so seid ihr selig zu preisen; denn dann ruht der Geist
der Herrlichkeit und der (Geist) Gottes auf euch. \bibleverse{15} Keiner
nämlich von euch möge als Mörder oder Dieb oder Übeltäter oder auch nur
deshalb leiden, weil er unbefugt in fremde Angelegenheiten\textless sup
title=``oder: Rechte''\textgreater✲ eingegriffen hat; muß er aber als
Christ leiden, \bibleverse{16} so schäme er sich dessen nicht, sondern
mache vielmehr Gott durch diesen (Christen-) Namen Ehre! \bibleverse{17}
Denn die Zeit ist da, daß das Gericht beim Hause Gottes seinen Anfang
nimmt. Wenn es aber bei uns zuerst (anhebt), wie wird da das Ende bei
denen sein, die der Heilsbotschaft Gottes nicht gehorchen?
\bibleverse{18} Und »wenn der Gerechte kaum gerettet wird, wo wird da
der Gottlose und Sünder sich zeigen\textless sup title=``oder: erblickt
werden''\textgreater✲«?\textless sup title=``Spr 11,31''\textgreater✲
\bibleverse{19} Daher sollen auch die, welche nach dem Willen Gottes zu
leiden haben, ihm, dem treuen Schöpfer, ihre Seelen befehlen, und zwar
dadurch, daß sie Gutes tun.

\hypertarget{c-mahnwort-an-die-uxe4ltesten-leiter-und-an-die-juxfcngeren-in-der-gemeinde}{%
\paragraph{c) Mahnwort an die Ältesten (=~Leiter) und an die Jüngeren in
der
Gemeinde}\label{c-mahnwort-an-die-uxe4ltesten-leiter-und-an-die-juxfcngeren-in-der-gemeinde}}

\hypertarget{section-4}{%
\section{5}\label{section-4}}

\bibleverse{1} Die Ältesten nun unter euch ermahne ich als ihr
Mitältester und als der Zeuge✲ der Leiden Christi, wie auch als
Teilnehmer an der Herrlichkeit, deren Offenbarung bevorsteht:
\bibleverse{2} weidet die euch anvertraute Herde Gottes und überwacht
sie, nicht aus Zwang✲, sondern mit freudiger Bereitwilligkeit nach
Gottes Willen, auch nicht in schnöder Gewinnsucht, sondern mit
Hingebung, \bibleverse{3} auch nicht als Gewaltherrscher über die euch
anvertrauten (Gemeinden), sondern als Vorbilder für die Herde;
\bibleverse{4} dann werdet ihr auch, wenn der Erzhirte✲ erscheint, den
unverwelklichen Kranz der Herrlichkeit empfangen.~-- \bibleverse{5}
Ebenso, ihr Jüngeren: seid den Ältesten\textless sup title=``oder:
Älteren''\textgreater✲ untertan. Allesamt aber legt euch im Verkehr
miteinander das Dienstgewand der Demut an, denn »Gott widersteht den
Hoffärtigen, aber den Demütigen gibt er Gnade«\textless sup title=``Spr
3,34''\textgreater✲. \bibleverse{6} Demütigt euch also unter die
gewaltige Hand Gottes, damit er euch zu seiner Zeit erhöhe!
\bibleverse{7} Alle eure Sorge werft auf ihn, denn er sorgt für euch!

\hypertarget{d-seid-wachsam-gegenuxfcber-den-anfechtungen-des-teufels}{%
\paragraph{d) Seid wachsam gegenüber den Anfechtungen des
Teufels}\label{d-seid-wachsam-gegenuxfcber-den-anfechtungen-des-teufels}}

\bibleverse{8} Seid nüchtern, seid wachsam! Euer Widersacher, der
Teufel, geht wie ein brüllender\textless sup title=``d.h.
fraßhungriger''\textgreater✲ Löwe umher und sucht, wen er verschlingen
kann! \bibleverse{9} Dem leistet Widerstand in Glaubensfestigkeit; ihr
wißt ja, daß die gleichen Leiden euren Brüdern in der ganzen Welt
auferlegt werden. \bibleverse{10} Der Gott aller Gnade aber, der uns
berufen hat zu seiner ewigen Herrlichkeit in Christus, der wird euch
nach einer kurzen Leidenszeit vollbereiten, festigen, stärken und
gründen. \bibleverse{11} Sein ist die Macht in alle Ewigkeit! Amen.

\hypertarget{schluuxdf-des-briefes-gruxfcuxdfe-und-segenswunsch}{%
\paragraph{Schluß des Briefes; Grüße und
Segenswunsch}\label{schluuxdf-des-briefes-gruxfcuxdfe-und-segenswunsch}}

\bibleverse{12} (Dies) habe ich euch durch Silvanus, den -- wie ich
überzeugt bin -- treuen Bruder, in Kürze geschrieben, um euch zu
ermahnen und euch zu bezeugen, daß dies die wahre Gnade Gottes ist, in
der ihr stehen sollt.~-- \bibleverse{13} Es grüßt euch die miterwählte
(Gemeinde) in Babylon und mein Sohn Markus. \bibleverse{14} Grüßt
einander mit dem Liebeskuß! Friede sei mit euch allen, die ihr in
Christus seid!
