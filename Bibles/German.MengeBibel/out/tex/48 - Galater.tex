\hypertarget{der-brief-des-apostels-paulus-an-die-galater}{%
\section{DER BRIEF DES APOSTELS PAULUS AN DIE
GALATER}\label{der-brief-des-apostels-paulus-an-die-galater}}

\hypertarget{zuschrift-und-segensgruuxdf}{%
\subsubsection{Zuschrift und
Segensgruß}\label{zuschrift-und-segensgruuxdf}}

\hypertarget{section}{%
\section{1}\label{section}}

\bibleverse{1} Ich, Paulus, ein Apostel -- nicht von Menschen her
(entsandt), auch nicht durch Vermittlung✲ eines Menschen (dazu gemacht),
sondern durch Jesus Christus und Gott den Vater, der ihn von den Toten
auferweckt hat --: \bibleverse{2} ich und sämtliche Brüder, die hier bei
mir sind, senden den Gemeinden in Galatien unsern Gruß: \bibleverse{3}
Gnade sei mit euch und Friede von Gott, unserm Vater, und dem Herrn
Jesus Christus, \bibleverse{4} der sich für unsere Sünden hingegeben
hat, um uns aus der gegenwärtigen bösen Welt zu erretten, nach dem
Willen unsers Gottes und Vaters. \bibleverse{5} Diesem sei\textless sup
title=``oder: gebührt''\textgreater✲ die Ehre in alle Ewigkeit. Amen.

\hypertarget{eingang-veranlassung-des-schreibens-befremden-des-apostels-uxfcber-den-schnellen-abfall-der-galater-von-der-einen-wahren-heilsbotschaft-des-paulus-verfluchung-der-verkuxfcndiger-einer-andern-falschen-heilslehre}{%
\subsubsection{Eingang: Veranlassung des Schreibens; Befremden des
Apostels über den schnellen Abfall der Galater von der einen, wahren
Heilsbotschaft (des Paulus); Verfluchung der Verkündiger einer andern
(falschen)
Heilslehre}\label{eingang-veranlassung-des-schreibens-befremden-des-apostels-uxfcber-den-schnellen-abfall-der-galater-von-der-einen-wahren-heilsbotschaft-des-paulus-verfluchung-der-verkuxfcndiger-einer-andern-falschen-heilslehre}}

\bibleverse{6} Ich muß mich darüber wundern, daß ihr so schnell wieder
abfallt\textless sup title=``=~euch abbringen laßt''\textgreater✲ von
dem, der euch durch die Gnade Christi berufen hat, und euch einer
anderen Heilsbotschaft zuwendet, \bibleverse{7} während es doch keine
andere (Heilsbotschaft) gibt; nur daß gewisse Leute da sind, die euch
verwirren und die Heilsbotschaft Christi\textless sup title=``oder: von
Christus''\textgreater✲ verkehren✲ möchten. \bibleverse{8} Aber auch
wenn wir selbst oder ein Engel aus dem Himmel euch eine andere
Heilsbotschaft verkündigten als die, welche wir euch verkündigt haben:
Fluch über ihn! \bibleverse{9} Wie wir es schon früher ausgesprochen
haben, so wiederhole ich es jetzt noch einmal: »Wenn jemand euch eine
andere Heilsbotschaft verkündigt als die, welche ihr (von mir) empfangen
habt: Fluch über ihn!« \bibleverse{10} Suche ich jetzt nun (mit solcher
Sprache) den Beifall von Menschen zu gewinnen oder (nicht vielmehr) die
Zustimmung Gottes? Oder gehe ich etwa darauf aus, Menschen zu gefallen?
Nein, wenn ich mich noch um das Wohlgefallen von Menschen bemühte, so
wäre ich kein Knecht✲ Christi.

\hypertarget{i.-beweis-dauxdf-paulus-seine-heilsbotschaft-nicht-von-menschen-empfangen-hat-111-221}{%
\subsection{I. Beweis, daß Paulus seine Heilsbotschaft nicht von
Menschen empfangen hat
(1,11-2,21)}\label{i.-beweis-dauxdf-paulus-seine-heilsbotschaft-nicht-von-menschen-empfangen-hat-111-221}}

\hypertarget{aufstellung-der-behauptung-dauxdf-die-heilsbotschaft-des-paulus-von-gott-stamme}{%
\subsubsection{1. Aufstellung der Behauptung, daß die Heilsbotschaft des
Paulus von Gott
stamme}\label{aufstellung-der-behauptung-dauxdf-die-heilsbotschaft-des-paulus-von-gott-stamme}}

\bibleverse{11} Ich weise euch nämlich darauf hin, liebe Brüder, daß die
von mir zuverlässig verkündigte Heilsbotschaft nicht nach
Menschenart\textless sup title=``d.h. ein Menschenwerk''\textgreater✲
ist. \bibleverse{12} Ich habe sie ja auch nicht von einem Menschen
empfangen, noch sie durch Unterricht erlernt, sondern (sie) durch eine
Offenbarung Jesu Christi (erhalten).

\hypertarget{paulus-hat-bis-zu-seinem-auftreten-als-heidenapostel-keine-menschliche-unterweisung-in-der-heilsbotschaft-erhalten}{%
\subsubsection{2. Paulus hat bis zu seinem Auftreten als Heidenapostel
keine menschliche Unterweisung in der Heilsbotschaft
erhalten}\label{paulus-hat-bis-zu-seinem-auftreten-als-heidenapostel-keine-menschliche-unterweisung-in-der-heilsbotschaft-erhalten}}

\hypertarget{a-das-verhalten-des-paulus-im-judentum-vor-seiner-bekehrung-und-unmittelbar-darauf}{%
\paragraph{a) Das Verhalten des Paulus (im Judentum) vor seiner
Bekehrung und unmittelbar
darauf}\label{a-das-verhalten-des-paulus-im-judentum-vor-seiner-bekehrung-und-unmittelbar-darauf}}

\bibleverse{13} Ihr habt ja von meinem einstmaligen Verhalten im
Judentum gehört: daß ich nämlich die Gemeinde Gottes maßlos✲ verfolgt
habe und sie zu vernichten suchte \bibleverse{14} und daß ich es an
Leidenschaft für das jüdische Wesen vielen meiner Altersgenossen in
meinem Volk zuvorgetan habe, indem ich ein ganz besonderer Eiferer für
die von meinen Vätern überkommenen Überlieferungen war. \bibleverse{15}
Als es aber dem\textless sup title=``d.h. Gott''\textgreater✲, der mich
vom Mutterleibe an ausgesondert und durch seine Gnade berufen hat,
wohlgefällig war, \bibleverse{16} seinen Sohn in mir zu offenbaren,
damit ich die Heilsbotschaft von ihm unter den Heiden verkündigte, da
habe ich mich sofort nicht an Menschen von Fleisch und Blut (um Rat)
gewandt, \bibleverse{17} bin auch nicht nach Jerusalem zu meinen
Vorgängern im Apostelamt hinaufgegangen, nein, ich begab mich nach
Arabien und kehrte dann wieder nach Damaskus zurück.

\hypertarget{b-die-selbstuxe4ndige-wirksamkeit-des-paulus-wuxe4hrend-der-zeit-vor-der-apostelversammlung}{%
\paragraph{b) Die selbständige Wirksamkeit des Paulus während der Zeit
vor der
Apostelversammlung}\label{b-die-selbstuxe4ndige-wirksamkeit-des-paulus-wuxe4hrend-der-zeit-vor-der-apostelversammlung}}

\bibleverse{18} Darauf, drei Jahre später, ging ich nach Jerusalem
hinauf, um Kephas✲ kennenzulernen, und blieb fünfzehn Tage bei ihm.
\bibleverse{19} Von den übrigen Aposteln habe ich damals keinen gesehen
außer Jakobus, den Bruder des Herrn. \bibleverse{20} Was ich euch hier
schreibe -- vor Gottes Angesicht (versichere ich), daß ich die reine
Wahrheit sage! \bibleverse{21} Hierauf begab ich mich in die
Landschaften\textless sup title=``oder: Gegenden''\textgreater✲ von
Syrien und Cilicien. \bibleverse{22} Den Christengemeinden in Judäa✲
aber blieb ich persönlich unbekannt; \bibleverse{23} nur vom Hörensagen
wußten sie: »Unser ehemaliger Verfolger verkündigt jetzt als
Heilsbotschaft den Glauben, den er einst ausrotten wollte«;
\bibleverse{24} und sie priesen Gott im Hinblick auf mich.

\hypertarget{auftreten-des-paulus-bei-der-apostelberatung-in-jerusalem-feierliche-anerkennung-seines-heidenapostolischen-berufs-von-seiten-der-urapostel}{%
\subsubsection{3. Auftreten des Paulus bei der Apostelberatung in
Jerusalem; feierliche Anerkennung seines heidenapostolischen Berufs von
seiten der
Urapostel}\label{auftreten-des-paulus-bei-der-apostelberatung-in-jerusalem-feierliche-anerkennung-seines-heidenapostolischen-berufs-von-seiten-der-urapostel}}

\hypertarget{a-sein-aufenthalt-in-jerusalem}{%
\paragraph{a) Sein Aufenthalt in
Jerusalem}\label{a-sein-aufenthalt-in-jerusalem}}

\hypertarget{section-1}{%
\section{2}\label{section-1}}

\bibleverse{1} Darauf, vierzehn Jahre später, zog ich wieder nach
Jerusalem hinauf, (diesmal) mit Barnabas, und nahm auch Titus mit.
\bibleverse{2} Ich unternahm aber diese Reise aufgrund einer
(göttlichen) Offenbarung und legte ihnen, insbesondere\textless sup
title=``=~in einer Privatbesprechung''\textgreater✲ den Angesehenen, die
Heilsbotschaft dar, die ich unter den Heiden verkündige, (um feststellen
zu lassen) ob meine Arbeit vergeblich wäre oder gewesen wäre.
\bibleverse{3} Doch nicht einmal mein Begleiter Titus, der doch (ein
Heidenchrist) aus den Griechen war, wurde zur Beschneidung genötigt.
\bibleverse{4} Was aber die eingedrungenen\textless sup
title=``=~unrechtmäßig hereingekommenen''\textgreater✲ falschen Brüder
anlangt, die sich eingeschlichen hatten, um unsere Freiheit, die wir in
Christus Jesus haben, zu belauern\textless sup title=``=~feindselig
auszukundschaften''\textgreater✲ und uns ganz in die Knechtschaft (des
Gesetzes) zu bringen: \bibleverse{5} vor ihnen sind wir auch nicht für
eine Stunde\textless sup title=``=~einen Augenblick''\textgreater✲ in
Unterwürfigkeit zurückgewichen, damit die Heilsbotschaft in ihrer
Wahrheit bei euch\textless sup title=``oder: für euch''\textgreater✲
bestehen bliebe.

\hypertarget{b-das-fuxfcr-paulus-guxfcnstige-ergebnis-der-verhandlungen-mit-den-uxe4lteren-angesehenen-aposteln}{%
\paragraph{b) Das für Paulus günstige Ergebnis der Verhandlungen mit den
älteren (»angesehenen«)
Aposteln}\label{b-das-fuxfcr-paulus-guxfcnstige-ergebnis-der-verhandlungen-mit-den-uxe4lteren-angesehenen-aposteln}}

\bibleverse{6} Von seiten der Angesehenen aber, die als Häupter galten
-- übrigens ist es mir gleichgültig, von welcher Art ihr Ansehen damals
war: Gott nimmt ja auf das äußere Ansehen eines Menschen keine Rücksicht
--: mir also haben diese Angesehenen keine weitere Verpflichtung
auferlegt; \bibleverse{7} nein, im Gegenteil, weil sie einsahen, daß ich
mit der Heilsbotschaft für die Unbeschnittenen✲ betraut bin, ebenso wie
Petrus (mit der Heilsbotschaft) für die Beschnittenen✲~-- \bibleverse{8}
denn (Gott), der sich in Petrus für den Aposteldienst an den Juden mit
seiner Kraft wirksam erwiesen hat, ist mit seiner Kraft auch in mir für
die Heiden wirksam gewesen --, \bibleverse{9} und weil sie die Gnade
(Gottes) erkannten, die mir zuteil geworden ist, haben Jakobus, Kephas
und Johannes, die als Säulen (der Gemeinde) galten, mir und Barnabas den
Handschlag zum Gemeinschaftsbunde gegeben (mit der Bestimmung): wir
sollten für die Heiden, sie aber für die Juden wirken; \bibleverse{10}
nur der Armen (in Jerusalem) sollten wir gedenken, was ich mir eben
deshalb auch ernstlich habe angelegen sein lassen.

\hypertarget{berechtigtes-auftreten-des-paulus-gegen-petrus-in-antiochien}{%
\subsubsection{4. Berechtigtes Auftreten des Paulus gegen Petrus in
Antiochien}\label{berechtigtes-auftreten-des-paulus-gegen-petrus-in-antiochien}}

\hypertarget{a-der-fehltritt-des-petrus}{%
\paragraph{a) Der Fehltritt des
Petrus}\label{a-der-fehltritt-des-petrus}}

\bibleverse{11} Als (später) aber Kephas✲ nach Antiochien gekommen war,
trat ich ihm Auge in Auge entgegen, denn er war ganz offenbar im
Unrecht. \bibleverse{12} Bevor nämlich einige (Abgesandte) des Jakobus
kamen, hatte er mit den Heidenchristen Tischgemeinschaft gehalten; aber
nach der Ankunft jener hatte er sich zurückgezogen und sich (von ihnen)
abgesondert aus Furcht vor den Judenchristen. \bibleverse{13} An dieser
Heuchelei hatten sich auch die übrigen Judenchristen mit ihm beteiligt,
so daß sich sogar Barnabas durch ihre Heuchelei hatte mit fortreißen
lassen.

\hypertarget{b-die-zurechtweisende-rede-des-paulus-gegen-petrus}{%
\paragraph{b) Die zurechtweisende Rede des Paulus gegen
Petrus}\label{b-die-zurechtweisende-rede-des-paulus-gegen-petrus}}

\bibleverse{14} Als ich jedoch sah, daß sie nicht den rechten Weg in
Übereinstimmung mit der Wahrheit der Heilsbotschaft wandelten, sagte ich
zu Kephas offen im Beisein aller\textless sup title=``=~in der
Gemeindeversammlung''\textgreater✲: »Wenn du, der du doch ein Jude bist,
nach heidnischer und nicht nach jüdischer Weise lebst, wie kannst du da
die Heiden zwingen wollen, die jüdischen Bräuche✲ zu beobachten?
\bibleverse{15} Wohl sind wir von Natur✲ Juden und nicht Sünder
heidnischer Herkunft; \bibleverse{16} weil wir aber wissen, daß der
Mensch nicht aufgrund von Gesetzeswerken gerechtfertigt wird, sondern
nur durch den Glauben an Christus Jesus, haben auch wir den Glauben an
Christus Jesus angenommen, um aufgrund des Glaubens an Christus und
nicht aufgrund von Gesetzeswerken gerechtfertigt zu werden; denn
aufgrund von Gesetzeswerken wird kein Fleisch✲ gerechtfertigt werden.
\bibleverse{17} Wenn wir nun aber bei unserem Streben, in Christus
gerechtfertigt zu werden, gerade als Sünder erfunden worden wären, da
stünde ja Christus im Dienst der Sünde! Das kann nicht sein!
\bibleverse{18} Allerdings, wenn ich das, was ich niedergerissen habe,
(später) wieder aufbaue, so stelle ich mich damit selbst als Übertreter
hin. \bibleverse{19} Ich meinerseits dagegen bin durch das Gesetz für
das Gesetz gestorben, um (fortan) für Gott zu leben: ich bin mit
Christus gekreuzigt. \bibleverse{20} So lebe also nicht mehr ich selbst,
sondern Christus lebt in mir; was✲ ich jetzt aber noch im Fleisch lebe,
das lebe ich im Glauben an den Sohn Gottes, der mich geliebt und sich
selbst für mich dahingegeben hat. \bibleverse{21} Ich
verwerfe\textless sup title=``oder: vereitle''\textgreater✲ die Gnade
Gottes nicht; denn wenn Gerechtigkeit durch das Gesetz
kommt\textless sup title=``oder: käme''\textgreater✲, dann freilich
ist\textless sup title=``oder: wäre''\textgreater✲ Christus umsonst✲
gestorben.«

\hypertarget{ii.-die-rechtfertigung-durch-den-glauben-und-die-freiheit-des-christen-vom-mosaischen-gesetz-31-512}{%
\subsection{II. Die Rechtfertigung durch den Glauben und die Freiheit
des Christen vom mosaischen Gesetz
(3,1-5,12)}\label{ii.-die-rechtfertigung-durch-den-glauben-und-die-freiheit-des-christen-vom-mosaischen-gesetz-31-512}}

\hypertarget{hinweis-auf-die-erfahrung-welche-die-galater-selbst-gemacht-haben-dauxdf-der-empfang-des-geistes-eine-folge-des-glaubens-ist}{%
\subsubsection{1. Hinweis auf die Erfahrung, welche die Galater selbst
gemacht haben, daß der Empfang des Geistes eine Folge des Glaubens
ist}\label{hinweis-auf-die-erfahrung-welche-die-galater-selbst-gemacht-haben-dauxdf-der-empfang-des-geistes-eine-folge-des-glaubens-ist}}

\hypertarget{section-2}{%
\section{3}\label{section-2}}

\bibleverse{1} O ihr unverständigen Galater! Wer hat euch nur bezaubert,
da euch doch Jesus Christus vor die Augen gemalt worden ist als
Gekreuzigter? \bibleverse{2} Nur das eine möchte ich von euch erfahren:
Habt ihr den Geist aufgrund von Gesetzeswerken empfangen oder infolge
der Predigt vom Glauben? \bibleverse{3} Seid ihr wirklich so
unverständig? Im Geist habt ihr den Anfang gemacht und wollt jetzt im
Fleisch den Abschluß machen? \bibleverse{4} So Großes solltet ihr
vergeblich erlitten haben?! Ja, wenn wirklich bloß vergeblich!
\bibleverse{5} Der euch also den Geist mitteilt und Wunderkräfte in
euch\textless sup title=``oder: unter euch, in eurer
Mitte''\textgreater✲ wirkt, (tut er das) aufgrund von Gesetzeswerken
oder infolge der Predigt vom Glauben?

\hypertarget{abrahams-glaubensgerechtigkeit-unser-vorbild-der-gesetzesdienst-bringt-den-fluch-christus-befreit-vom-fluch-des-gesetzes}{%
\subsubsection{2. Abrahams Glaubensgerechtigkeit unser Vorbild; der
Gesetzesdienst bringt den Fluch, Christus befreit vom Fluch des
Gesetzes}\label{abrahams-glaubensgerechtigkeit-unser-vorbild-der-gesetzesdienst-bringt-den-fluch-christus-befreit-vom-fluch-des-gesetzes}}

\bibleverse{6} (Ja, es ist so) wie bei Abraham: »er glaubte Gott, und
das wurde ihm zur Gerechtigkeit gerechnet«\textless sup title=``1.Mose
15,6''\textgreater✲. \bibleverse{7} Ihr erkennt also: die Gläubigen, die
sind Abrahams Söhne. \bibleverse{8} Weil aber die Schrift voraussah, daß
Gott die Völker✲ um des Glaubens willen rechtfertigt, hat sie dem
Abraham die Heilsverheißung im voraus verkündigtGen
12,318,18''\textgreater✲: »In dir sollen alle Völker✲ gesegnet werden.«
\bibleverse{9} Somit empfangen die, welche aus dem Glauben
sind\textless sup title=``=~die Gläubigen''\textgreater✲ den Segen
zugleich mit dem gläubigen Abraham.

\bibleverse{10} Denn✲ alle, die aus Werken des Gesetzes
sind\textless sup title=``=~auf Gesetzeswerke bauen''\textgreater✲,
stehen unter einem\textless sup title=``oder: dem''\textgreater✲ Fluch;
denn es steht geschrieben\textless sup title=``5.Mose
27,26''\textgreater✲: »Verflucht ist jeder, der nicht in allen
(Geboten), die im Buch des Gesetzes geschrieben stehen, beharrt, um sie
(tatsächlich) zu erfüllen.« \bibleverse{11} Daß aber aufgrund des
Gesetzes niemand bei Gott gerechtfertigt wird, ist offenbar; denn »der
Gerechte wird aus Glauben\textless sup title=``oder: aufgrund des
Glaubens''\textgreater✲ leben«\textless sup title=``Hab
2,4''\textgreater✲. \bibleverse{12} Das Gesetz aber hat mit dem Glauben
nichts zu tun, sondern\textless sup title=``da gilt; 3.Mose
18,5''\textgreater✲: »Wer sie\textless sup title=``d.h. die Vorschriften
des Gesetzes''\textgreater✲ erfüllt hat, der wird durch sie leben.«

\bibleverse{13} Christus hat uns vom Fluch des Gesetzes dadurch
losgekauft, daß er für uns zum Fluch\textless sup title=``=~an unserer
Statt oder uns zuliebe ein Verfluchter''\textgreater✲ geworden ist; denn
es steht geschrieben\textless sup title=``5.Mose 21,23''\textgreater✲:
»Verflucht ist jeder, der am Holze✲ hängt.« \bibleverse{14} Es sollte
eben der dem Abraham verheißene Segen den Heiden in Christus Jesus
zuteil werden, damit wir das Verheißungsgut des Geistes\textless sup
title=``=~den verheißenen Geist''\textgreater✲ durch den Glauben
empfangen könnten.

\hypertarget{die-dem-abraham-gegebene-verheiuxdfung-ist-durch-das-spuxe4ter-gegebene-gesetz-nicht-aufgehoben}{%
\subsubsection{3. Die dem Abraham gegebene Verheißung ist durch das
später gegebene Gesetz nicht
aufgehoben}\label{die-dem-abraham-gegebene-verheiuxdfung-ist-durch-das-spuxe4ter-gegebene-gesetz-nicht-aufgehoben}}

\bibleverse{15} Liebe Brüder, ich will an menschliche Verhältnisse
erinnern: Auch die letztwillige Verfügung eines Menschen, die
rechtskräftig geworden ist, kann doch niemand umstoßen oder nachträglich
mit Zusätzen versehen. \bibleverse{16} Nun sind aber die (göttlichen)
Verheißungen dem Abraham »und seinem Samen✲« zugesprochen worden. Es
heißt nicht: »und den Samen\textless sup title=``=~den
Nachkommen''\textgreater✲« in der Mehrzahl\textless sup title=``=~als
wären es mehrere''\textgreater✲, sondern mit Bezug auf einen
einzigen\textless sup title=``=~in der Einzahl''\textgreater✲: »und
deinem Samen\textless sup title=``=~deinem Nachkommen''\textgreater✲«,
und das ist Christus. \bibleverse{17} Ich meine das aber so: Eine von
Gott bereits früher vollgültig\textless sup title=``oder:
rechtskräftig''\textgreater✲ gemachte Verfügung kann durch das Gesetz,
das erst vierhundertunddreißig Jahre später gekommen ist, nicht außer
Kraft gesetzt\textless sup title=``=~für ungültig erklärt''\textgreater✲
werden, so daß es die Verheißung aufhöbe. \bibleverse{18} Wenn nämlich
das (verheißene) Erbe vom Gesetz abhängt, so hängt es nicht mehr von der
Verheißung ab; dem Abraham aber hat Gott es durch eine Verheißung als
Gnadengabe verliehen.

\hypertarget{das-gesetz-war-dazu-bestimmt-ein-erzieher-auf-christus-hin-zu-sein}{%
\subsubsection{4. Das Gesetz war dazu bestimmt, ein Erzieher auf
Christus hin zu
sein}\label{das-gesetz-war-dazu-bestimmt-ein-erzieher-auf-christus-hin-zu-sein}}

\hypertarget{a-wesen-und-zweck-des-zur-heilsvollendung-unwirksamen-durch-engel-und-einen-mittler-verordneten-und-nur-fuxfcr-die-zwischenzeit-bestimmten-gesetzes}{%
\paragraph{a) Wesen und Zweck des zur Heilsvollendung unwirksamen (durch
Engel und einen Mittler verordneten und nur für die Zwischenzeit
bestimmten)
Gesetzes}\label{a-wesen-und-zweck-des-zur-heilsvollendung-unwirksamen-durch-engel-und-einen-mittler-verordneten-und-nur-fuxfcr-die-zwischenzeit-bestimmten-gesetzes}}

\bibleverse{19} Was soll\textless sup title=``oder: wozu
dient''\textgreater✲ nun aber das Gesetz? Der Übertretungen wegen ist es
(der Verheißung) hinzugefügt worden (für die Zwischenzeit), bis der
Same\textless sup title=``=~der Nachkomme''\textgreater✲ käme, dem die
Verheißung gilt; und zwar ist es durch Engel verordnet✲ worden unter
Mitwirkung eines Mittlers. \bibleverse{20} Ein Mittler aber vertritt
nicht einen einzigen; Gott aber ist ein einziger.

\hypertarget{b-abweisung-einer-muxf6glichen-miuxdfdeutung}{%
\paragraph{b) Abweisung einer möglichen
Mißdeutung}\label{b-abweisung-einer-muxf6glichen-miuxdfdeutung}}

\bibleverse{21} Steht demnach das Gesetz mit den Verheißungen Gottes in
Widerspruch? Nimmermehr! Ja, wenn ein Gesetz gegeben worden wäre, das
die Kraft besäße, Leben zu verleihen, dann käme die Gerechtigkeit
tatsächlich aus dem Gesetz. \bibleverse{22} Nun aber hat die Schrift
alles\textless sup title=``oder: die Gesamtheit =~die ganze
Menschheit''\textgreater✲ unter die (Herrschaft der) Sünde
zusammengeschlossen, damit das Verheißungsgut den Gläubigen aufgrund des
Glaubens an Jesus Christus zuteil würde.

\hypertarget{c-der-uxe4uuxdferlich-erzieherische-zweck-des-gesetzes}{%
\paragraph{c) Der äußerlich erzieherische Zweck des
Gesetzes}\label{c-der-uxe4uuxdferlich-erzieherische-zweck-des-gesetzes}}

\bibleverse{23} Bevor aber der Glaube kam, wurden wir unter dem Gesetz
in Verwahrung\textless sup title=``oder: Gewahrsam''\textgreater✲
gehalten und waren unter Verschluß gelegt\textless sup title=``oder: in
Schutzhaft genommen''\textgreater✲ für den Glauben, der erst später
geoffenbart werden sollte. \bibleverse{24} Somit ist das Gesetz unser
Erzieher\textless sup title=``oder: Zuchtmeister''\textgreater✲ für
Christus geworden, damit wir aufgrund des Glaubens gerechtfertigt
würden.

\hypertarget{d-alle-gluxe4ubigen-christen-sind-nunmehr-nach-dem-ende-der-herrschaft-des-gesetzes-gottes-und-abrahams-suxf6hne-und-erben}{%
\paragraph{d) Alle gläubigen Christen sind nunmehr (nach dem Ende der
Herrschaft des Gesetzes) Gottes und Abrahams Söhne und
Erben}\label{d-alle-gluxe4ubigen-christen-sind-nunmehr-nach-dem-ende-der-herrschaft-des-gesetzes-gottes-und-abrahams-suxf6hne-und-erben}}

\bibleverse{25} Seitdem nun aber der Glaube gekommen ist, stehen wir
nicht mehr unter einem Erzieher; \bibleverse{26} denn ihr alle seid
Söhne Gottes durch den Glauben an\textless sup title=``oder:
in''\textgreater✲ Christus Jesus. \bibleverse{27} Denn ihr alle, die ihr
in\textless sup title=``oder: für, oder: auf''\textgreater✲ Christus
getauft worden seid, habt (damit) Christus angezogen. \bibleverse{28} Da
gibt es nun nicht mehr Juden und Griechen\textless sup
title=``=~griechisch redende Heiden''\textgreater✲, nicht mehr Knechte✲
und Freie, nicht mehr Mann und Weib: nein, ihr seid allesamt
Einer\textless sup title=``oder: eine Einheit''\textgreater✲ in Christus
Jesus. \bibleverse{29} Wenn ihr aber Christus angehört, so seid ihr
damit ja Abrahams Nachkommenschaft\textless sup title=``oder:
Kinder''\textgreater✲, Erben gemäß der Verheißung.

\hypertarget{die-schluuxdffolgerungen-fuxfcr-die-galater}{%
\subsubsection{5. Die Schlußfolgerungen für die
Galater}\label{die-schluuxdffolgerungen-fuxfcr-die-galater}}

\hypertarget{a-an-die-stelle-der-gesetzesknechtschaft-ist-fuxfcr-die-gluxe4ubigen-die-sohnesstellung-oder-kindschaft-in-christus-getreten}{%
\paragraph{a) An die Stelle der Gesetzesknechtschaft ist für die
Gläubigen die Sohnesstellung (oder: Kindschaft) in Christus
getreten}\label{a-an-die-stelle-der-gesetzesknechtschaft-ist-fuxfcr-die-gluxe4ubigen-die-sohnesstellung-oder-kindschaft-in-christus-getreten}}

\hypertarget{section-3}{%
\section{4}\label{section-3}}

\bibleverse{1} Ich sage\textless sup title=``oder: meine''\textgreater✲
aber: Solange der Erbe noch unmündig ist, besteht zwischen ihm und einem
Knecht\textless sup title=``oder: Sklaven''\textgreater✲ kein
Unterschied, wenn er auch der Herr von allem\textless sup title=``oder:
Besitzer aller Güter''\textgreater✲ ist; \bibleverse{2} er steht
vielmehr unter Vormündern und Vermögensverwaltern bis zu dem vom Vater
festgesetzten Zeitpunkt. \bibleverse{3} So standen auch wir, solange wir
(geistlich) unmündig waren, als Sklaven unter der Herrschaft der
Elemente\textless sup title=``vgl. Kol 2,8''\textgreater✲ der Welt.
\bibleverse{4} Als aber die Erfüllung der Zeit\textless sup title=``d.h.
der festgesetzte Zeitpunkt''\textgreater✲ gekommen war, sandte Gott
seinen Sohn, der von einem Weibe geboren und dem Gesetz unterworfen
wurde; \bibleverse{5} er sollte die unter dem Gesetz Stehenden
loskaufen, damit wir die Einsetzung in die Sohnschaft\textless sup
title=``=~die Kindschaftsstellung''\textgreater✲ erlangten.
\bibleverse{6} Weil ihr jetzt aber Söhne\textless sup title=``oder:
Kinder''\textgreater✲ seid, hat Gott den Geist seines Sohnes in unsere
Herzen gesandt, der da ruft: »Abba, (lieber) Vater!«\textless sup
title=``vgl. Röm 8,15''\textgreater✲ \bibleverse{7} Mithin bist du kein
Knecht mehr, sondern ein Sohn; bist du aber ein Sohn, so bist du auch
ein Erbe durch Gott.

\hypertarget{b-klage-des-apostels-uxfcber-den-unbegreiflichen-ruxfcckfall-der-gluxe4ubigen-galater-in-gesetzesknechtschaft-und-verwerfliches-formwesen-persuxf6nlicher-aufruf-an-die-gemeinde-zur-umkehr}{%
\paragraph{b) Klage des Apostels über den unbegreiflichen Rückfall der
gläubigen Galater in Gesetzesknechtschaft und verwerfliches Formwesen;
persönlicher Aufruf an die Gemeinde zur
Umkehr}\label{b-klage-des-apostels-uxfcber-den-unbegreiflichen-ruxfcckfall-der-gluxe4ubigen-galater-in-gesetzesknechtschaft-und-verwerfliches-formwesen-persuxf6nlicher-aufruf-an-die-gemeinde-zur-umkehr}}

\bibleverse{8} Damals freilich, als ihr Gott noch nicht kanntet, habt
ihr solchen Göttern gedient, die ihrem Wesen nach gar keine Götter sind.
\bibleverse{9} Da ihr jetzt aber Gott erkannt habt oder
vielmehr\textless sup title=``=~richtiger gesagt''\textgreater✲ von Gott
erkannt worden seid: wie könnt ihr euch da nur wieder den erbärmlichen
und armseligen Elementen✲ zuwenden und Lust haben, ihnen noch einmal von
neuem als Knechte zu dienen?! \bibleverse{10} Ihr beobachtet ja Tage und
Monate\textless sup title=``oder: Neumonde''\textgreater✲, Festzeiten
und Jahre! \bibleverse{11} Ich bin um euch besorgt, daß ich vergeblich
an euch gearbeitet haben möchte.

\bibleverse{12} Werdet doch so, wie ich (bin)! Denn auch ich (bin so
gesetzesfrei geworden), wie ihr\textless sup title=``ursprünglich waret;
vgl. 1.Kor 9,21''\textgreater✲, liebe Brüder, ich bitte euch! Ihr habt
mir (ja früher) nichts zuleide getan; \bibleverse{13} ihr wißt vielmehr,
daß ich euch das erste Mal\textless sup title=``=~bei meinem ersten
Besuche''\textgreater✲, veranlaßt durch leibliche Schwäche\textless sup
title=``oder: Krankheit''\textgreater✲, die Heilsbotschaft verkündigt
habe. \bibleverse{14} Damals habt ihr mich trotz des Anstoßes, den mein
leiblicher Zustand bei euch erregen mußte, nicht mit Verachtung
behandelt und nicht mit Abscheu zurückgewiesen, sondern mich wie einen
Engel Gottes, ja wie Christus Jesus (selber) aufgenommen.
\bibleverse{15} Wo ist nun eure (damalige) glückselige Freude geblieben?
Ich muß euch ja das Zeugnis ausstellen, daß ihr euch damals, wenn es
möglich gewesen wäre, die Augen ausgerissen und sie mir geschenkt
hättet. \bibleverse{16} Sonach bin ich jetzt wohl dadurch euer Feind
geworden, daß ich euch die Wahrheit vorhalte\textless sup title=``oder:
verkünde''\textgreater✲? \bibleverse{17} O sie bewerben sich nicht in
guter Absicht eifrig um eure Gunst, nein, sie möchten euch gern (von
mir) ausschließen, damit ihr euch dann eifrig um ihre Gunst bemüht.
\bibleverse{18} Schön ist es ja, in guter Sache Gegenstand eifriger
Umwerbung zu sein, und zwar allezeit und nicht nur während meiner
Anwesenheit bei euch. \bibleverse{19} Meine lieben Kinder, um die ich
jetzt wiederum Geburtsschmerzen leide, bis Christus (endlich) in euch
Gestalt gewinnt: \bibleverse{20} ich wollte, ich wäre gerade jetzt bei
euch und könnte in anderem Ton zu euch reden; denn ich weiß mir
euretwegen keinen Rat!

\hypertarget{c-sinnbildliche-auslegung-des-alttestamentlichen-berichts-von-ismael-und-isaak-den-beiden-suxf6hnen-abrahams-zum-beweis-der-freiheit-des-christen-von-den-satzungen-des-gesetzes}{%
\paragraph{c) Sinnbildliche Auslegung des alttestamentlichen Berichts
von Ismael und Isaak, den beiden Söhnen Abrahams, zum Beweis der
Freiheit des Christen von den Satzungen des
Gesetzes}\label{c-sinnbildliche-auslegung-des-alttestamentlichen-berichts-von-ismael-und-isaak-den-beiden-suxf6hnen-abrahams-zum-beweis-der-freiheit-des-christen-von-den-satzungen-des-gesetzes}}

\bibleverse{21} Sagt mir doch, die ihr gern unter dem Gesetz stehen
möchtet: hört\textless sup title=``oder: versteht''\textgreater✲ ihr
denn das Gesetz nicht? \bibleverse{22} Es steht ja doch
geschrieben\textless sup title=``1.Mose 21,2.9''\textgreater✲, daß
Abraham zwei Söhne hatte, einen von der Magd✲ und einen von der Freien.
\bibleverse{23} Jedoch jener von der Magd war nur sein
fleischmäßig\textless sup title=``=~infolge leiblicher
Zeugung''\textgreater✲ erzeugter Sohn, dieser von der Freien aber war
ihm aufgrund der (göttlichen) Verheißung geboren. \bibleverse{24} Das
ist bildlich gesprochen\textless sup title=``oder: zu
verstehen''\textgreater✲; denn diese (beiden Frauen) stellen zwei
Bündnisse dar, das eine vom Berge Sinai, das zur Knechtschaft
gebiert\textless sup title=``d.h. die ihm Angehörigen in Knechtschaft
versetzt''\textgreater✲: das ist die Hagar; \bibleverse{25} das Wort
Hagar✲ bedeutet nämlich den Berg Sinai in Arabien, und sie\textless sup
title=``d.h. die Hagar''\textgreater✲ entspricht dem heutigen Jerusalem;
denn dieses befindet sich (auch) in Knechtschaft samt seinen Kindern.
\bibleverse{26} Das Jerusalem droben dagegen ist eine Freie, und dies
(Jerusalem) ist unsere Mutter; \bibleverse{27} denn es steht
geschrieben\textless sup title=``Jes 54,1''\textgreater✲: »Freue dich,
du Kinderlose, die du nicht Mutter wirst! Brich in Jubel aus und
frohlocke, die du keine Geburtsschmerzen zu leiden hast! Denn die
Alleinstehende hat zahlreiche Kinder, mehr als die Verehelichte.«
\bibleverse{28} Ihr aber, liebe Brüder, seid nach Isaaks Art Kinder der
Verheißung. \bibleverse{29} Wie jedoch damals der nach dem Fleisch
erzeugte Sohn den nach dem Geist\textless sup title=``=~nach göttlicher
Verheißung''\textgreater✲ erzeugten verfolgt hat, so ist es auch jetzt
der Fall. \bibleverse{30} Aber was sagt die Schrift dazu? »Verstoße die
Magd und ihren Sohn! Denn der Sohn der Magd soll nicht das gleiche
Erbrecht mit dem Sohn der Freien haben.«\textless sup title=``1.Mose
21,10''\textgreater✲

\hypertarget{d-zusammenfassung-des-bisherigen-und-abschlieuxdfende-mahnung-an-der-christlichen-freiheit-festzuhalten-die-mit-gesetz-und-beschneidung-unvereinbar-ist}{%
\paragraph{d) Zusammenfassung des Bisherigen und abschließende Mahnung,
an der christlichen Freiheit festzuhalten, die mit Gesetz und
Beschneidung unvereinbar
ist}\label{d-zusammenfassung-des-bisherigen-und-abschlieuxdfende-mahnung-an-der-christlichen-freiheit-festzuhalten-die-mit-gesetz-und-beschneidung-unvereinbar-ist}}

\bibleverse{31} Darum, liebe Brüder: Wir sind nicht Kinder einer Magd✲,
sondern der Freien!

\hypertarget{section-4}{%
\section{5}\label{section-4}}

\bibleverse{1} Zur\textless sup title=``oder: für die''\textgreater✲
Freiheit hat Christus uns frei gemacht. Stehet also fest und laßt euch
nicht wieder in das Joch der Knechtschaft spannen! \bibleverse{2} Seht,
ich, Paulus, erkläre euch: Wenn ihr euch beschneiden laßt, so wird
Christus euch nichts mehr nützen. \bibleverse{3} Nochmals bezeuge ich es
einem jeden, der sich beschneiden läßt: er ist damit zur Beobachtung des
ganzen Gesetzes verpflichtet. \bibleverse{4} Aus der Verbindung mit
Christus seid ihr ausgeschieden, wenn ihr durch das Gesetz
gerechtfertigt werden wollt: ihr seid dann aus der Gnade herausgefallen;
\bibleverse{5} denn wir erwarten durch den Geist das Hoffnungsgut der
Gerechtigkeit aufgrund des Glaubens. \bibleverse{6} Denn in Christus
Jesus hat weder die Beschneidung noch das Unbeschnittensein irgendwelche
Bedeutung, sondern nur der Glaube, der sich durch Liebe betätigt.

\hypertarget{e-wehmuxfctige-bzw.-unwillige-klage-uxfcber-verfuxfchrer-und-verfuxfchrte-in-der-gemeinde}{%
\paragraph{e) Wehmütige (bzw. unwillige) Klage über Verführer und
Verführte in der
Gemeinde}\label{e-wehmuxfctige-bzw.-unwillige-klage-uxfcber-verfuxfchrer-und-verfuxfchrte-in-der-gemeinde}}

\bibleverse{7} Ihr hattet einen so schönen Anlauf genommen: wer hat euch
aufgehalten, daß ihr der Wahrheit nicht mehr gehorchen\textless sup
title=``oder: Folge leisten''\textgreater✲ wollt? \bibleverse{8} Der
Antrieb dazu kommt nicht von dem her, der euch beruft. \bibleverse{9}
Schon ein wenig Sauerteig macht den ganzen Teig sauer. \bibleverse{10}
Ich für meine Person habe das Vertrauen zu euch im Herrn, daß ihr eure
Gesinnung nicht ändern werdet; wer euch aber irre macht: er wird sein
Strafurteil zu tragen haben, er sei, wer er wolle. \bibleverse{11} Was
mich aber betrifft, liebe Brüder: wenn ich wirklich noch die
Beschneidung predige, warum verfolgt man mich da noch? Dann ist ja das
Ärgernis des Kreuzes aus der Welt geschafft! \bibleverse{12} Möchten
doch die Leute, die euch aufwiegeln, sich sogar
verschneiden\textless sup title=``oder: entmannen''\textgreater✲
lassen!\textless sup title=``vgl. 5.Mose 23,2''\textgreater✲

\hypertarget{iii.-die-gesetzesfreiheit-in-christus-als-grundlage-eines-neuen-geistlichen-und-sittlichen-lebens-513-610}{%
\subsection{III. Die Gesetzesfreiheit in Christus als Grundlage eines
neuen geistlichen und sittlichen Lebens
(5,13-6,10)}\label{iii.-die-gesetzesfreiheit-in-christus-als-grundlage-eines-neuen-geistlichen-und-sittlichen-lebens-513-610}}

\hypertarget{warnung-vor-dem-miuxdfbrauch-der-durch-christus-gewonnenen-freiheit-betuxe4tigung-der-freiheit-durch-nuxe4chstenliebe}{%
\subsubsection{1. Warnung vor dem Mißbrauch der durch Christus
gewonnenen Freiheit; Betätigung der Freiheit durch
Nächstenliebe}\label{warnung-vor-dem-miuxdfbrauch-der-durch-christus-gewonnenen-freiheit-betuxe4tigung-der-freiheit-durch-nuxe4chstenliebe}}

\bibleverse{13} Ihr seid ja doch zur Freiheit berufen, liebe Brüder; nur
mißbraucht die Freiheit nicht als einen willkommenen Anlaß\textless sup
title=``oder: Freibrief''\textgreater✲ für das Fleisch\textless sup
title=``=~fleischliche Gelüste''\textgreater✲, sondern dienet einander
durch die Liebe! \bibleverse{14} Denn das ganze Gesetz findet seine
Erfüllung in dem einen Gebot\textless sup title=``3.Mose
19,18''\textgreater✲: »Du sollst deinen Nächsten lieben wie dich
selbst!« \bibleverse{15} Wenn ihr euch aber untereinander beißt und
freßt, so sehet zu, daß ihr nicht voneinander verschlungen werdet!

\hypertarget{wandelt-im-geist-die-werke-des-fleisches-und-die-frucht-des-geistes}{%
\subsubsection{2. Wandelt im Geist! Die Werke des Fleisches und die
Frucht des
Geistes}\label{wandelt-im-geist-die-werke-des-fleisches-und-die-frucht-des-geistes}}

\bibleverse{16} Ich meine aber so: Wandelt im Geist, dann werdet ihr
sicherlich das Gelüst des Fleisches nicht vollführen. \bibleverse{17}
Denn das Fleisch widerstrebt mit seinem Begehren dem Geist und ebenso
der Geist dem Fleisch; denn diese beiden liegen im Streit miteinander
(und dulden nicht), daß ihr das tut, was ihr tun möchtet.
\bibleverse{18} Laßt ihr euch aber vom Geist leiten, so steht ihr nicht
(mehr) unter dem Gesetz. \bibleverse{19} Offenbar aber sind die Werke
des Fleisches, nämlich Unzucht, Unsittlichkeit, Ausschweifung,
\bibleverse{20} Götzendienst, Zauberei, Feindseligkeiten, Zank,
Eifersucht, Zerwürfnisse, gemeine Selbstsucht, Zwietracht, Parteiungen,
\bibleverse{21} Neid, Trunksucht, Schwelgerei und so weiter. Von diesen
(Sünden) habe ich euch schon früher gesagt und wiederhole es jetzt, daß,
wer derartiges verübt, das Reich Gottes nicht erben wird.
\bibleverse{22} Die Frucht des Geistes dagegen besteht in Liebe, Freude,
Friede, Geduld, Freundlichkeit, Gütigkeit, Treue, \bibleverse{23}
Sanftmut, Beständigkeit\textless sup title=``oder:
Festigkeit''\textgreater✲; gegen derartige (Geistesfrüchte) kann das
Gesetz keine Anklage erheben. \bibleverse{24} Die aber Christus Jesus
angehören, haben ihr Fleisch samt seinen Leidenschaften und Begierden
gekreuzigt.

\hypertarget{einzelne-sittliche-ermahnungen-zur-bewuxe4hrung-des-neuen-lebens-im-geist-hinweis-auf-gottes-gericht}{%
\subsubsection{3. Einzelne sittliche Ermahnungen zur Bewährung des neuen
Lebens im Geist; Hinweis auf Gottes
Gericht}\label{einzelne-sittliche-ermahnungen-zur-bewuxe4hrung-des-neuen-lebens-im-geist-hinweis-auf-gottes-gericht}}

\bibleverse{25} Wenn wir nun im Geiste leben, so laßt uns im Geiste auch
wandeln! \bibleverse{26} Laßt uns nicht nach eitler Ehre begierig sein,
einander nicht (zum Streit) herausfordern, einander nicht beneiden!~--

\hypertarget{section-5}{%
\section{6}\label{section-5}}

\bibleverse{1} Liebe Brüder, wenn auch jemand sich von einem Fehltritt
hat übereilen lassen, so bringt ihr Geistesmenschen den Betreffenden mit
dem Geist der Sanftmut wieder zurecht, und gib dabei auf dich selbst
acht, damit du nicht auch in Versuchung gerätst! \bibleverse{2} Traget
einer des andern Lasten, so werdet ihr dadurch das Gesetz Christi
erfüllen. \bibleverse{3} Denn wenn jemand sich dünken läßt, er sei
etwas, obwohl er doch nichts ist, so betrügt er sich selbst in seinem
Sinn. \bibleverse{4} Jeder prüfe aber sein eigenes Werk, und dann mag er
für sich allein zu rühmen haben, aber nicht dem andern gegenüber;
\bibleverse{5} denn ein jeder wird an seiner eigenen Last zu tragen
haben.

\bibleverse{6} Wer aber Unterricht im Wort (Gottes) erhält, lasse seinen
Lehrer an allen Gütern teilnehmen!~-- \bibleverse{7} Irret euch nicht:
Gott läßt sich nicht spotten; denn was der Mensch sät, das wird er auch
ernten. \bibleverse{8} Denn wer auf sein Fleisch sät, wird vom Fleisch
Verderben ernten; wer aber auf den Geist sät, wird vom Geist ewiges
Leben ernten. \bibleverse{9} Laßt uns aber nicht müde werden, das
Rechte\textless sup title=``oder: Gute''\textgreater✲ zu tun; denn zu
seiner\textless sup title=``d.h. zur bestimmten''\textgreater✲ Zeit
werden wir ernten, wenn wir nicht ermatten. \bibleverse{10} Darum wollen
wir so, wie wir Gelegenheit haben, allen Menschen Gutes erweisen,
besonders aber den Glaubensgenossen!

\hypertarget{iv.-der-eigenhuxe4ndige-schluuxdf-des-briefes-611-18}{%
\subsection{IV. Der eigenhändige Schluß des Briefes
(6,11-18)}\label{iv.-der-eigenhuxe4ndige-schluuxdf-des-briefes-611-18}}

\hypertarget{a-letzte-beleuchtung-der-gegner}{%
\paragraph{a) Letzte Beleuchtung der
Gegner}\label{a-letzte-beleuchtung-der-gegner}}

\bibleverse{11} Sehet, mit wie großen Buchstaben ich euch (nun noch)
eigenhändig schreibe! \bibleverse{12} (Nur) solche (Leute), die im
Fleische\textless sup title=``d.h. in fleischlichen Dingen und im
äußeren Leben''\textgreater✲ etwas Besonderes vorstellen wollen, suchen
euch die Beschneidung aufzunötigen, lediglich um nicht wegen (der
Verkündigung) des Kreuzes Christi Verfolgungen zu erleiden.
\bibleverse{13} Denn trotz ihrer Beschneidung beobachten sie selbst das
Gesetz nicht, sondern dringen auf eure Beschneidung nur deshalb, um sich
eures leiblichen Menschen rühmen zu können.

\hypertarget{b-persuxf6nliches-schluuxdfbekenntnis-segenswunsch}{%
\paragraph{b) Persönliches Schlußbekenntnis;
Segenswunsch}\label{b-persuxf6nliches-schluuxdfbekenntnis-segenswunsch}}

\bibleverse{14} Mir aber soll es nicht beikommen, mich irgendeiner
anderen Sache zu rühmen als nur des Kreuzes unsers Herrn Jesus Christus,
durch das für mich die Welt gekreuzigt ist und ich für die Welt.
\bibleverse{15} Denn weder auf die Beschneidung noch auf das
Unbeschnittensein kommt es an, sondern nur auf eine »neue
Schöpfung«\textless sup title=``2.Kor 5,17''\textgreater✲;
\bibleverse{16} und alle, die nach dieser Richtschnur wandeln werden:
über die komme Friede und (göttliches) Erbarmen, nämlich\textless sup
title=``oder: das heißt''\textgreater✲ über das Israel Gottes!

\bibleverse{17} In Zukunft möge mir niemand zu schaffen machen, denn ich
trage die Malzeichen Jesu an meinem Leibe!~-- \bibleverse{18} Die Gnade
unsers Herrn Jesus Christus sei mit eurem Geiste, liebe Brüder! Amen.
