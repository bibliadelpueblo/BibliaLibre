\hypertarget{das-buch-ruth}{%
\section{DAS BUCH RUTH}\label{das-buch-ruth}}

\hypertarget{die-vorgeschichte}{%
\subsubsection{1. Die Vorgeschichte}\label{die-vorgeschichte}}

\hypertarget{a-die-schicksale-noomis-im-lande-der-moabiter}{%
\paragraph{a) Die Schicksale Noomis im Lande der
Moabiter}\label{a-die-schicksale-noomis-im-lande-der-moabiter}}

\hypertarget{section}{%
\section{1}\label{section}}

1Zu der Zeit, als die Richter (in Israel) walteten, kam einmal eine
Hungersnot über das Land. Da wanderte ein Mann aus Bethlehem in Juda mit
seiner Frau und seinen beiden Söhnen aus, um eine Zeitlang im Gebiet der
Moabiter als Fremdling zu leben. 2Der Mann hieß Elimelech, seine Frau
Noomi\textless sup title=``vgl. V.20''\textgreater✲; und seine beiden
Söhne hießen Machlon und Kiljon; sie waren Ephrathiter aus Bethlehem in
Juda. Sie kamen auch glücklich im Gebiet der Moabiter an und blieben
dort. 3Da starb Elimelech, der Mann Noomis, so daß sie mit ihren beiden
Söhnen allein zurückblieb. 4Diese nahmen sich dann moabitische Frauen,
von denen die eine Orpa, die andere Ruth\textless sup title=``d.h.
Freundin''\textgreater✲ hieß. So wohnten sie dort etwa zehn Jahre. 5Als
dann auch die beiden Söhne, Machlon und Kiljon, starben und die Frau,
ihrer beiden Söhne und ihres Mannes beraubt, allein stand, 6machte sie
sich mit ihren beiden Schwiegertöchtern auf, um aus dem Gebiet der
Moabiter in ihre Heimat zurückzukehren; denn sie hatte im Lande der
Moabiter in Erfahrung gebracht, daß der HERR sich seines Volkes gnädig
angenommen und ihm wieder Brot gegeben habe.

\hypertarget{b-aufbruch-noomis-und-ihrer-beiden-schwiegertuxf6chter-zur-ruxfcckkehr-nach-bethlehem-orpas-abschied-ruths-treue}{%
\paragraph{b) Aufbruch Noomis und ihrer beiden Schwiegertöchter zur
Rückkehr nach Bethlehem; Orpas Abschied, Ruths
Treue}\label{b-aufbruch-noomis-und-ihrer-beiden-schwiegertuxf6chter-zur-ruxfcckkehr-nach-bethlehem-orpas-abschied-ruths-treue}}

7Sie verließ also in Begleitung ihrer beiden Schwiegertöchter den Ort,
wo sie bis dahin gewohnt hatte. Als sie aber ihres Weges zogen, um ins
Land Juda zurückzukehren, 8sagte Noomi zu ihren beiden
Schwiegertöchtern: »Kehrt jetzt wieder heim, eine jede ins Haus ihrer
Mutter. Der HERR segne euch für die Liebe, die ihr den Verstorbenen und
mir erwiesen habt! 9Der HERR vergönne euch beiden, ein ruhiges Heim im
Hause eines Gatten zu finden!« Als Noomi sie hierauf geküßt hatte,
begannen jene laut zu weinen 10und sagten zu ihr: »Nein, wir wollen dich
zu deinem Volke\textless sup title=``=~in deine Heimat''\textgreater✲
begleiten!« 11Aber Noomi entgegnete: »Kehrt um, liebe Töchter! Warum
wollt ihr mit mir gehen? Darf ich etwa noch hoffen, Söhnen das Leben zu
geben, daß sie eure Männer werden könnten? 12Nein, kehrt um, liebe
Töchter! Geht heim! Ich bin ja zu alt, um mich nochmals zu verheiraten.
Und wenn ich auch dächte, noch Aussicht auf eine neue Ehe zu haben, ja
wenn ich noch in dieser Nacht das Weib eines Mannes und sogar Mutter von
Söhnen würde: 13wolltet ihr deshalb warten, bis sie erwachsen wären?
Wolltet ihr euch deshalb bis dahin einschließen und unverheiratet
bleiben? Nein, liebe Töchter! Ich bin ja euretwegen tief betrübt, daß
die Hand des HERRN mich so schwer getroffen hat!« 14Da begannen sie von
neuem laut zu weinen; dann küßte aber Orpa ihre Schwiegermutter und ging
weg, Ruth aber schloß sie fest in ihre Arme. 15Da sagte Noomi zu ihr:
»Nachdem nun deine Schwägerin zu ihrem Volk und zu ihrem Gott
zurückgekehrt ist, so kehre auch du um und folge deiner Schwägerin!«
16Aber Ruth erwiderte: »Dringe nicht in mich, dich zu verlassen und ohne
dich umzukehren; Nein, wohin du gehst, dahin will auch ich gehen, und wo
du bleibst, da bleibe ich auch: dein Volk ist mein Volk, und dein Gott
ist mein Gott! 17Wo du stirbst, da sterbe ich auch, und da will ich
begraben sein. Der HERR mache mit mir, was er will: nur der Tod soll
mich von dir scheiden!« 18Als Noomi nun sah, daß Ruth fest entschlossen
war, mit ihr zu gehen, redete sie nicht weiter auf sie ein.

\hypertarget{c-ankunft-und-empfang-der-beiden-frauen-in-bethlehem}{%
\paragraph{c) Ankunft und Empfang der beiden Frauen in
Bethlehem}\label{c-ankunft-und-empfang-der-beiden-frauen-in-bethlehem}}

19So gingen denn die beiden weiter, bis sie nach Bethlehem gelangten.
Als sie aber dort ankamen, geriet der ganze Ort ihretwegen in Aufregung,
und alle Frauen sagten: »Ist das nicht Noomi?« 20Da antwortete sie
ihnen: »Nennt mich nicht Noomi\textless sup title=``d.h. meine Wonne
oder: die Liebliche''\textgreater✲, nennt mich lieber Mara\textless sup
title=``d.h. die Bittere''\textgreater✲; denn der Allmächtige hat mich
viel Bitteres erleben lassen. 21Voll✲ bin ich weggegangen, und leer✲ hat
mich der HERR zurückkehren lassen. Warum nennt ihr mich Noomi? Hat sich
doch der HERR selbst gegen mich erklärt und der Allmächtige mich in Leid
gestürzt!« 22So kehrte Noomi und mit ihr Ruth, ihre Schwiegertochter,
die Moabitin, aus dem Lande der Moabiter heim, und zwar kamen sie in
Bethlehem an beim Beginn der Gerstenernte\textless sup title=``d.h. im
April''\textgreater✲.

\hypertarget{das-erste-zusammentreffen-ruths-mit-boas}{%
\subsubsection{2. Das erste Zusammentreffen Ruths mit
Boas}\label{das-erste-zusammentreffen-ruths-mit-boas}}

\hypertarget{a-ruth-kommt-um-uxe4hren-zu-lesen-auf-das-feld-des-boas-der-sich-nach-ihr-erkundigt-und-ihr-freundlich-begegnet}{%
\paragraph{a) Ruth kommt, um Ähren zu lesen, auf das Feld des Boas, der
sich nach ihr erkundigt und ihr freundlich
begegnet}\label{a-ruth-kommt-um-uxe4hren-zu-lesen-auf-das-feld-des-boas-der-sich-nach-ihr-erkundigt-und-ihr-freundlich-begegnet}}

\hypertarget{section-1}{%
\section{2}\label{section-1}}

1Nun besaß Noomi einen Verwandten von seiten ihres Mannes, einen sehr
wohlhabenden Mann aus der Familie Elimelechs, namens Boas. 2Als nun die
Moabitin Ruth zu Noomi sagte: »Ich will doch aufs Feld gehen und Ähren
lesen hinter einem her\textless sup title=``=~bei einem''\textgreater✲,
der es mir erlauben wird«, erwiderte diese ihr: »Geh nur, meine
Tochter!« 3So ging sie denn hin und las auf dem Felde hinter den
Schnittern her auf, und der Zufall wollte es, daß das Grundstück dem
Boas gehörte, der aus der Familie Elimelechs stammte. 4Da kam Boas
gerade aus Bethlehem und sagte zu den Schnittern: »Der HERR sei mit
euch!«; sie antworteten ihm: »Der HERR segne dich!« 5Darauf fragte Boas
den Großknecht bei seinen Schnittern: »Wem gehört das
Mädchen\textless sup title=``oder: die junge Frau''\textgreater✲ da?«
6Der Großknecht antwortete: »Es ist das moabitische Mädchen, das mit
Noomi aus dem Lande der Moabiter heimgekehrt ist. 7Sie hat uns gebeten:
›Laßt mich doch Ähren lesen und zwischen den Garben sammeln hinter den
Schnittern her!‹ So ist sie denn gekommen und hat ausgehalten vom frühen
Morgen an bis jetzt und sich keinen Augenblick Ruhe gegönnt.« 8Da sagte
Boas zu Ruth: »Hörst du wohl, meine Tochter? Du brauchst auf kein
anderes Feld zu gehen, um dort aufzulesen, und brauchst auch nicht von
hier wegzugehen, sondern schließe dich hier an meine Mägde an! 9Laß
deine Augen immer auf das Feld gerichtet sein, wo sie schneiden, und
gehe hinter ihnen her; ich habe auch den Knechten befohlen, dich nicht
zu belästigen. Und wenn du Durst hast, so gehe nur zu den Gefäßen und
trinke von dem Wasser, das die Knechte geholt haben!« 10Da warf sie sich
mit dem Angesicht vor ihm auf die Erde nieder und sagte zu ihm: »Wie
kommt's doch, daß du so freundlich gegen mich bist und dich meiner
annimmst, da ich doch eine Ausländerin bin?« 11Boas antwortete ihr: »O
es ist mir alles genau berichtet worden, was du an deiner
Schwiegermutter nach dem Tode deines Mannes getan hast: Vater, Mutter
und Heimatland hast du verlassen und bist zu einem Volk gezogen, das du
früher nicht kanntest. 12Der HERR vergelte dir dein Tun, und voller Lohn
möge dir zuteil werden vom HERRN, dem Gott Israels, unter dessen Flügeln
du Schutz zu suchen hergekommen bist!« 13Da antwortete sie: »Ich danke
dir für deine Freundlichkeit, mein Herr! Denn du hast mich getröstet und
deiner Magd mit Herzlichkeit zugesprochen, obgleich ich nicht einmal wie
eine von deinen Mägden bin.«

\hypertarget{b-ruth-wird-weiter-von-boas-freundlich-behandelt-kommt-mit-reichem-uxe4hrenertrag-nach-hause-und-erhuxe4lt-von-ihrer-schwiegermutter-auskunft-uxfcber-boas}{%
\paragraph{b) Ruth wird weiter von Boas freundlich behandelt, kommt mit
reichem Ährenertrag nach Hause und erhält von ihrer Schwiegermutter
Auskunft über
Boas}\label{b-ruth-wird-weiter-von-boas-freundlich-behandelt-kommt-mit-reichem-uxe4hrenertrag-nach-hause-und-erhuxe4lt-von-ihrer-schwiegermutter-auskunft-uxfcber-boas}}

14Zur Essenszeit sagte dann Boas zu ihr: »Komm hierher und iß mit von
dem Brot und tunke deinen Bissen in den Essig!« Als sie sich nun neben
die Schnitter gesetzt hatte, reichte er ihr geröstete Körner, und sie
aß, bis sie satt war, und behielt noch einen Teil übrig. 15Als sie dann
aufstand, um wieder zu lesen, gab Boas seinen Knechten den Befehl: »Sie
darf auch zwischen den Garben lesen, und ihr sollt ihr nichts zuleide
tun! 16Zieht vielmehr hin und wieder Halme für sie aus den
Bündeln\textless sup title=``oder: Schwaden''\textgreater✲ heraus und
laßt sie liegen; sie mag sie dann auflesen, ohne daß ihr sie scheltet!«
17So las sie denn auf dem Felde bis zum Abend, und als sie dann das
ausklopfte, was sie gesammelt hatte, war es beinahe ein Epha Gerste.
18Sie hob es auf, ging in die Stadt und zeigte ihrer Schwiegermutter,
was sie gelesen hatte; dann holte sie auch noch das hervor, was sie vom
Essen übrigbehalten hatte, und gab es ihr. 19Da sagte ihre
Schwiegermutter zu ihr: »Wo hast du heute so fleißig gelesen? Gesegnet
sei, der sich deiner so freundlich angenommen hat!« Nun erzählte sie
ihrer Schwiegermutter, auf wessen Felde sie bei der Arbeit gewesen war,
und sagte: »Der Mann, auf dessen Felde ich heute bei der Arbeit gewesen
bin, heißt Boas.« 20Da erwiderte Noomi ihrer Schwiegertochter: »Gesegnet
sei er vom HERRN, der seine Güte weder den noch Lebenden noch den schon
Toten entzogen hat!« Dann fuhr Noomi fort: »Der Mann ist mit uns nahe
verwandt; er ist einer von unsern Lösern.« 21Darauf sagte die Moabitin
Ruth: »Er hat auch zu mir gesagt: ›Halte dich nur zu meinen Leuten, bis
sie mit meiner ganzen Ernte fertig sind!‹« 22Da antwortete Noomi ihrer
Schwiegertochter Ruth: »Es ist gut, liebe Tochter, wenn du mit seinen
Mägden hinausgehst: so kann man dir auf einem andern Felde nichts
zuleide tun.« 23So hielt sie sich denn beim Ährenlesen zu den Mägden des
Boas, bis die Gersten- und Weizenernte zu Ende war; dann blieb sie bei
ihrer Schwiegermutter zu Hause.

\hypertarget{ruths-nuxe4chtliche-unterredung-mit-boas}{%
\subsubsection{3. Ruths nächtliche Unterredung mit
Boas}\label{ruths-nuxe4chtliche-unterredung-mit-boas}}

\hypertarget{a-ruth-begibt-sich-auf-noomis-rat-zur-tenne-des-boas-und-legt-sich-zu-dessen-fuxfcuxdfen-nieder}{%
\paragraph{a) Ruth begibt sich auf Noomis Rat zur Tenne des Boas und
legt sich zu dessen Füßen
nieder}\label{a-ruth-begibt-sich-auf-noomis-rat-zur-tenne-des-boas-und-legt-sich-zu-dessen-fuxfcuxdfen-nieder}}

\hypertarget{section-2}{%
\section{3}\label{section-2}}

1Da sagte ihre Schwiegermutter Noomi zu ihr: »Liebe Tochter, ich muß dir
doch ein sicheres Heim verschaffen, damit du gut versorgt bist. 2Nun
denn, Boas, dessen Mägden du dich angeschlossen hast, gehört zu unserer
Verwandtschaft; der worfelt gerade diese Nacht die Gerste auf der Tenne.
3So bade und salbe dich denn, lege deine besten Kleider an und gehe zur
Tenne hinab, laß dich aber von dem Manne nicht eher bemerken, als bis er
mit Essen und Trinken fertig ist. 4Wenn er sich dann schlafen legt, so
achte auf den Ort, wohin er sich legt; dann gehe hin, hebe die Decke zu
seinen Füßen auf und lege dich dort nieder; er wird dir dann schon
sagen, was du zu tun hast.« 5Sie antwortete ihr: »Ganz nach deiner
Weisung will ich tun!« 6Sie ging also zur Tenne hinab und machte es
genau so, wie ihre Schwiegermutter ihr angegeben hatte. 7Als Boas
nämlich gegessen und getrunken hatte und guter Dinge geworden war, legte
er sich am Ende\textless sup title=``oder: Rande''\textgreater✲ des
Getreidehaufens schlafen; da kam sie leise heran, deckte den Platz zu
seinen Füßen auf und legte sich dort nieder.

\hypertarget{b-ruth-bespricht-sich-mit-boas-erhuxe4lt-die-gewuxfcnschte-zusage-und-kehrt-mit-einem-geschenk-zu-noomi-zuruxfcck}{%
\paragraph{b) Ruth bespricht sich mit Boas, erhält die gewünschte Zusage
und kehrt mit einem Geschenk zu Noomi
zurück}\label{b-ruth-bespricht-sich-mit-boas-erhuxe4lt-die-gewuxfcnschte-zusage-und-kehrt-mit-einem-geschenk-zu-noomi-zuruxfcck}}

8Da, um Mitternacht, fuhr der Mann aus dem Schlafe auf, und als er sich
vorbeugte, sah er ein Weib zu seinen Füßen liegen. 9Als er nun fragte:
»Wer bist du?«, antwortete sie: »Ich bin Ruth, deine Magd; breite also
deinen Fittich✲ über deine Magd aus; denn du bist Löser für mich!« 10Da
erwiderte er: »Gesegnet seist du vom HERRN, meine Tochter! Du hast deine
Liebe (zu Noomi) zuletzt noch schöner betätigt als früher, indem du
nicht den jungen Männern nachgelaufen bist, sie seien arm oder reich.
11Und nun, meine Tochter, sei ohne Angst! Alles, was du wünschest, will
ich für dich tun; wissen doch alle Leute, die in unserm Ort auf dem
Marktplatze am Stadttor zusammenkommen, daß du ein sittsames Weib bist.
12Nun bin ich ja allerdings ein Löser für dich; aber es ist noch ein
anderer Löser vorhanden, der näher mit dir verwandt ist als ich.
13Bleibe über Nacht hier; morgen wird sich's dann finden: wenn er dich
lösen will, gut, so mag er es tun! Hat er aber keine Lust dazu, so will
ich dich lösen, so wahr der HERR lebt! Bleibe nur liegen bis zum
Morgen!« 14So blieb sie denn zu seinen Füßen liegen bis zum Morgen; dann
stand sie auf, ehe noch ein Mensch den andern erkennen konnte; er dachte
nämlich: »Es braucht nicht bekanntzuwerden, daß ein Weib auf die Tenne
gekommen ist.« 15Dann sagte er: »Nimm den Überwurf, den du anhast, und
halte ihn fest.« Als sie ihn nun hinhielt, maß er ihr sechs Maß Gerste
ab und lud sie ihr auf. So ging sie in die Stadt. 16Als sie nun zu ihrer
Schwiegermutter heimkam, fragte diese: »Wie ist es dir ergangen, liebe
Tochter?« Da erzählte sie ihr alles, wie der Mann sich gegen sie
verhalten hatte, 17und schloß mit den Worten: »Diese sechs Maß Gerste
hat er mir gegeben; denn er sagte: ›Du sollst nicht mit leeren Händen zu
deiner Schwiegermutter zurückkommen.‹« 18Da sagte diese: »Warte nur
ruhig ab, liebe Tochter, bis du erfährst, wie die Sache abläuft! Denn
der Mann wird nicht ruhen, bis er die Sache heute noch zur Entscheidung
gebracht hat.«

\hypertarget{die-uxf6ffentliche-verhandlung-zwischen-boas-und-dem-luxf6ser}{%
\subsubsection{4. Die öffentliche Verhandlung zwischen Boas und dem
Löser}\label{die-uxf6ffentliche-verhandlung-zwischen-boas-und-dem-luxf6ser}}

\hypertarget{section-3}{%
\section{4}\label{section-3}}

1Boas aber war zum Stadttor hinaufgegangen und hatte sich dort
niedergesetzt. Als nun gerade der Löser vorüberging, von dem Boas
geredet hatte, rief er: »Komm her und setze dich hierher, du Soundso!«
Als er nun gekommen war und sich gesetzt hatte, 2holte (Boas) zehn
Männer von den Ältesten\textless sup title=``oder:
Vornehmsten''\textgreater✲ der Stadt und sagte zu ihnen: »Setzt euch
hier nieder!« Als sie sich gesetzt hatten, 3sagte er zu dem Löser: »Das
Stück Land✲, das unserm Verwandten Elimelech gehört hat, will Noomi
verkaufen, die aus dem Lande der Moabiter zurückgekehrt ist. 4Nun habe
ich gedacht, ich wollte dir einen Vorschlag machen, nämlich: Kaufe es in
Gegenwart der hier Sitzenden und in Gegenwart der Ältesten meines
Volkes! Willst du Löser sein, so sei Löser; wo nicht, so gib mir eine
Erklärung ab, damit ich Bescheid weiß; denn außer dir ist kein Löser da,
und ich komme erst nach dir.« Da erklärte jener: »Ja, ich will Löser
sein.« 5Da fuhr Boas fort: »Sobald du das Stück Land von Noomi erwirbst,
hast du auch die Moabitin Ruth, die Witwe des Verstorbenen, erkauft, um
das Geschlecht des Verstorbenen auf seinem Erbbesitz fortzupflanzen.«
6Da antwortete der Löser: »In diesem Falle kann ich es nicht für mich
einlösen; ich würde sonst mein eigenes Besitztum schädigen; löse du für
dich, was ich lösen sollte, denn ich kann nicht Löser sein!« 7Nun
bestand ehemals in Israel beim Lösen wie beim Tauschen der Brauch, daß,
wenn man irgendeinen Handel fest abmachen wollte, der eine seinen Schuh
auszog und ihn dem andern gab; dies galt als Beglaubigung in Israel.
8Als daher der Löser zu Boas gesagt hatte: »Kaufe du es für dich!«, zog
er seinen Schuh aus. 9Darauf sagte Boas zu den Ältesten und zu allen
anwesenden Leuten: »Ihr seid heute Zeugen, daß ich alles, was dem
Elimelech, und alles, was dem Kiljon und Machlon gehört hat, von Noomi
gekauft habe; 10aber auch die Moabitin Ruth, die Witwe Machlons, habe
ich mir zum Weibe erkauft, um den Namen\textless sup title=``=~das
Geschlecht''\textgreater✲ des Verstorbenen auf seinem Erbbesitz
fortzupflanzen, damit nicht der Name des Verstorbenen aus dem Kreise
seiner Verwandten und aus dem Tor\textless sup title=``=~der
Bürgerschaft''\textgreater✲ seines Heimatorts verschwindet: dafür seid
ihr heute Zeugen!« 11Da erklärten alle im Stadttor Anwesenden und auch
die Ältesten: »Ja, wir sind Zeugen! Der HERR gebe, daß die Frau, die in
dein Haus einziehen soll, (so fruchtbar) werde wie Rahel und Lea, die
beide das Haus Israel aufgebaut haben! Werde glücklich in
Ephratha\textless sup title=``vgl. 1,2''\textgreater✲ und schaffe dir
einen Namen in Bethlehem! 12Und dein Haus werde wie das Haus des Perez,
den Thamar dem Juda geboren hat\textless sup title=``vgl. 1.Mose 38,29;
4.Mose 26,21''\textgreater✲, durch die Nachkommenschaft, die der HERR
dir von dieser jungen Frau bescheren wird!«

\hypertarget{des-boas-heirat-mit-ruth-vollzogen-und-durch-die-geburt-obeds-gesegnet-geschlechtsverzeichnis-von-perez-bis-david}{%
\subsubsection{5. Des Boas Heirat mit Ruth vollzogen und durch die
Geburt Obeds gesegnet; Geschlechtsverzeichnis von Perez bis
David}\label{des-boas-heirat-mit-ruth-vollzogen-und-durch-die-geburt-obeds-gesegnet-geschlechtsverzeichnis-von-perez-bis-david}}

13So heiratete denn Boas die Ruth, und sie wurde sein Weib; und als er
zu ihr eingegangen war, verlieh ihr der HERR Mutterglück durch die
Geburt eines Sohnes. 14Da sagten die Frauen zu Noomi: »Gepriesen sei der
HERR, der es dir heute an einem Löser nicht hat fehlen lassen! Möge sein
Name in Israel gefeiert werden, 15und möge er ein Trost für dein Herz
sein und dein Versorger im Alter! Denn deine Schwiegertochter, die dich
liebhat, ist seine Mutter, sie, die dir mehr wert ist als sieben Söhne.«
16Da nahm Noomi das Kind, legte es auf ihren Schoß und wurde seine
Wärterin. 17Die Nachbarinnen aber legten ihm einen Namen bei, indem sie
sagten: »Ein Sohn ist der Noomi geboren!«, und sie nannten ihn
Obed\textless sup title=``d.h. Diener =~Obadja, »Diener des
Herrn«''\textgreater✲. Der ist der Vater Isais, des Vaters Davids.
18Dies ist der Stammbaum des Perez: Perez war der Vater Hezrons,
19Hezron der Vater Rams, Ram der Vater Amminadabs, 20Amminadab der Vater
Nahassons, Nahasson der Vater Salmons, 21Salmon der Vater des Boas, Boas
der Vater Obeds, 22Obed der Vater Isais und Isai der Vater Davids.
