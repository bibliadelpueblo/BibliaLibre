\hypertarget{die-heilsbotschaft-nach-johannes}{%
\section{DIE HEILSBOTSCHAFT NACH
JOHANNES}\label{die-heilsbotschaft-nach-johannes}}

\hypertarget{vorwort-jesus-als-das-menschgewordene-wort-11-18}{%
\subsection{Vorwort: Jesus als das menschgewordene ›Wort‹
(1,1-18)}\label{vorwort-jesus-als-das-menschgewordene-wort-11-18}}

\hypertarget{a-wesen-wirken-und-bedeutung-des-uranfuxe4nglichen-wortes}{%
\paragraph{a) Wesen, Wirken und Bedeutung des uranfänglichen
›Wortes‹}\label{a-wesen-wirken-und-bedeutung-des-uranfuxe4nglichen-wortes}}

\hypertarget{section}{%
\section{1}\label{section}}

\bibleverse{1} Im Anfang war das Wort, und das Wort war bei Gott, und
Gott\textless sup title=``=~göttlichen Wesens''\textgreater✲ war das
Wort. \bibleverse{2} Dieses war im Anfang bei Gott. \bibleverse{3} Alle
Dinge sind durch dieses (Wort) geworden✲, und ohne dieses ist nichts
geworden (von allem), was geworden ist. \bibleverse{4} In ihm war Leben,
und das Leben war das Licht der Menschen. \bibleverse{5} Und das Licht
leuchtet in der Finsternis, doch die Finsternis hat es nicht
ergriffen\textless sup title=``oder: begriffen, oder:
angenommen''\textgreater✲.

\hypertarget{b-verhalten-der-welt-zu-dem-menschgewordenen-wort}{%
\paragraph{b) Verhalten der Welt zu dem menschgewordenen
›Wort‹}\label{b-verhalten-der-welt-zu-dem-menschgewordenen-wort}}

\bibleverse{6} Es trat ein Mann auf, von Gott gesandt, sein Name war
Johannes; \bibleverse{7} dieser kam, um Zeugnis abzulegen, Zeugnis von
dem Licht\textless sup title=``oder: für das Licht''\textgreater✲, damit
alle durch ihn zum Glauben kämen. \bibleverse{8} Er war nicht selbst das
Licht, sondern Zeugnis sollte er von dem Licht\textless sup
title=``oder: für das Licht''\textgreater✲ ablegen. \bibleverse{9} Das
Licht war da, das wahre, das jeden Menschen erleuchtet, es kam gerade in
die Welt; \bibleverse{10} es war in der Welt, und die Welt war durch ihn
(der das Licht war) geschaffen worden, doch die Welt erkannte ihn nicht.
\bibleverse{11} Er kam in das Seine\textless sup title=``=~sein
Eigentum''\textgreater✲, doch die Seinen\textless sup title=``d.h. die
ihm Eigenen''\textgreater✲ nahmen ihn nicht auf; \bibleverse{12} allen
aber, die ihn annahmen, verlieh er das Anrecht, Kinder Gottes zu werden,
nämlich denen, die an seinen Namen glauben, \bibleverse{13} die nicht
durch Geblüt oder durch den Naturtrieb des Fleisches, auch nicht durch
den Willen eines Mannes, sondern aus Gott gezeugt\textless sup
title=``oder: geboren''\textgreater✲ sind.

\hypertarget{c-das-wort-offenbart-seine-herrlichkeit-in-menschengestalt-wird-vom-tuxe4ufer-angekuxfcndigt-und-bringt-gottes-gnade-und-wahrheit}{%
\paragraph{c) Das ›Wort‹ offenbart seine Herrlichkeit in
Menschengestalt, wird vom Täufer angekündigt und bringt Gottes Gnade und
Wahrheit}\label{c-das-wort-offenbart-seine-herrlichkeit-in-menschengestalt-wird-vom-tuxe4ufer-angekuxfcndigt-und-bringt-gottes-gnade-und-wahrheit}}

\bibleverse{14} Und das Wort wurde Fleisch✲ und nahm seine Wohnung unter
uns, und wir haben seine Herrlichkeit geschaut, eine Herrlichkeit, wie
sie dem eingeborenen✲ Sohne vom Vater verliehen wird; eine mit Gnade und
Wahrheit erfüllte. \bibleverse{15} Johannes legt Zeugnis von
ihm\textless sup title=``oder: für ihn''\textgreater✲ ab und hat laut
verkündet: »Dieser war es, von dem ich gesagt habe: ›Der nach mir kommt,
ist (schon) vor mir gewesen, denn er war eher als ich\textless sup
title=``oder: als Erster über mir''\textgreater✲.‹« \bibleverse{16} Aus
seiner Fülle haben wir ja alle empfangen, und zwar Gnade über Gnade.
\bibleverse{17} Denn das Gesetz ist durch Mose gegeben worden, aber die
Gnade und die Wahrheit sind durch Jesus Christus geworden✲.
\bibleverse{18} Niemand hat Gott jemals gesehen: der eingeborene Sohn,
der an des Vaters Brust liegt, der hat Kunde (von ihm) gebracht.

\hypertarget{i.-einfuxfchrung-jesu-in-die-welt-119-51}{%
\subsection{I. Einführung Jesu in die Welt
(1,19-51)}\label{i.-einfuxfchrung-jesu-in-die-welt-119-51}}

\hypertarget{der-vorluxe4ufer-zeugnis-johannes-des-tuxe4ufers-von-sich-selbst-und-von-jesus}{%
\subsubsection{1. Der Vorläufer; Zeugnis Johannes des Täufers von sich
selbst und von
Jesus}\label{der-vorluxe4ufer-zeugnis-johannes-des-tuxe4ufers-von-sich-selbst-und-von-jesus}}

\hypertarget{a-des-tuxe4ufers-zeugnis-uxfcber-sich-selbst-vor-den-gesandten-des-hohen-rates}{%
\paragraph{a) Des Täufers Zeugnis über sich selbst vor den Gesandten des
Hohen
Rates}\label{a-des-tuxe4ufers-zeugnis-uxfcber-sich-selbst-vor-den-gesandten-des-hohen-rates}}

\bibleverse{19} Dies ist nun das Zeugnis des Johannes, als die Juden aus
Jerusalem Priester und Leviten zu ihm sandten, die ihn fragen sollten,
wer er sei. \bibleverse{20} Da bekannte er unverhohlen und erklärte
offen: »Ich bin nicht Christus\textless sup title=``=~der
Messias''\textgreater✲.« \bibleverse{21} Sie fragten ihn weiter: »Was
denn? Bist du Elia?« Er sagte: »Nein, ich bin es nicht.« »Bist du der
Prophet?« Er antwortete: »Nein.« \bibleverse{22} Da sagten sie zu ihm:
»Wer bist du denn? Wir müssen doch denen, die uns gesandt haben, eine
Antwort bringen! Wofür gibst du selbst dich aus?« \bibleverse{23} Da
antwortete er: »Ich bin die Stimme dessen, der in der Wüste ruft: ›Ebnet
dem Herrn den Weg!‹, wie der Prophet Jesaja geboten hat.«\textless sup
title=``Jes 40,3''\textgreater✲ \bibleverse{24} Die Gesandten aber
gehörten zu den Pharisäern \bibleverse{25} und fragten ihn weiter:
»Warum taufst du denn, wenn du weder Christus\textless sup title=``=~der
Messias''\textgreater✲ noch Elia, noch der Prophet bist?«
\bibleverse{26} Da antwortete Johannes ihnen: »Ich taufe nur mit Wasser;
aber mitten unter euch steht der, den ihr nicht kennt, \bibleverse{27}
der nach mir kommt und für den ich nicht gut genug bin, ihm den Riemen
seines Schuhwerks aufzubinden.« \bibleverse{28} Dies ist in Bethanien
geschehen jenseits des Jordans, wo Johannes sich aufhielt und taufte.

\hypertarget{b-des-tuxe4ufers-zeugnis-uxfcber-jesus-vor-seinen-juxfcngern}{%
\paragraph{b) Des Täufers Zeugnis über Jesus vor seinen
Jüngern}\label{b-des-tuxe4ufers-zeugnis-uxfcber-jesus-vor-seinen-juxfcngern}}

\bibleverse{29} Am folgenden Tage sah er Jesus auf sich zukommen; da
sagte er: »Seht, das Lamm Gottes, das die Sünde der Welt
hinwegnimmt!\textless sup title=``Jes 53,4.7''\textgreater✲
\bibleverse{30} Dieser ist's, von dem ich gesagt habe: ›Nach mir kommt
ein Mann, der (schon) vor mir gewesen ist; denn er war eher da als
ich\textless sup title=``oder: war als Erster über mir''\textgreater✲.‹
\bibleverse{31} Ich selbst kannte ihn nicht; aber damit er Israel
offenbart würde, deshalb bin ich gekommen, ich mit meiner Wassertaufe.«
\bibleverse{32} Weiter legte Johannes Zeugnis ab mit den Worten: »Ich
habe gesehen, daß der Geist wie eine Taube aus dem Himmel herabschwebte
und auf ihm blieb; \bibleverse{33} und ich selbst kannte ihn nicht, aber
der, welcher mich gesandt hat, um mit Wasser zu taufen, der hat zu mir
gesagt: ›Auf welchen du den Geist herabschweben und auf ihm bleiben
siehst, der ist's, der mit heiligem Geiste tauft.‹ \bibleverse{34} Nun
habe ich selbst es auch gesehen und bezeugt, daß dieser der Sohn Gottes
ist.«

\hypertarget{selbstoffenbarung-jesu-vor-seinen-ersten-fuxfcnf-juxfcngern}{%
\subsubsection{2. Selbstoffenbarung Jesu vor seinen ersten fünf
Jüngern}\label{selbstoffenbarung-jesu-vor-seinen-ersten-fuxfcnf-juxfcngern}}

\bibleverse{35} Am folgenden Tage stand Johannes wieder da mit zweien
seiner Jünger, \bibleverse{36} und indem er den Blick auf Jesus
richtete, der dort umherging, sagte er: »Seht, das Lamm Gottes!«
\bibleverse{37} Als die beiden Jünger ihn das sagen hörten, gingen sie
hinter Jesus her; \bibleverse{38} dieser wandte sich um, und als er sie
hinter sich herkommen sah, fragte er sie: »Was sucht\textless sup
title=``oder: wünscht''\textgreater✲ ihr?« Sie antworteten ihm: »Rabbi«
-- das heißt übersetzt\textless sup title=``=~auf deutsch''\textgreater✲
›Meister\textless sup title=``oder: Lehrer''\textgreater✲‹ --, »wo hast
du deine Herberge\textless sup title=``oder: Wohnung''\textgreater✲?«
\bibleverse{39} Er antwortete ihnen: »Kommt mit, so werdet ihr es
sehen!« Sie gingen also mit und sahen, wo er seine Herberge hatte, und
blieben jenen ganzen Tag bei ihm; es war um die zehnte Stunde.
\bibleverse{40} Andreas, der Bruder des Simon Petrus, war einer von den
beiden, die es von Johannes gehört hatten und hinter Jesus hergegangen
waren. \bibleverse{41} Dieser traf zuerst seinen Bruder Simon und sagte
zu ihm: »Wir haben den Messias« -- das heißt übersetzt ›den
Gesalbten\textless sup title=``oder: Christus''\textgreater✲‹ --
»gefunden.« \bibleverse{42} Er führte ihn dann zu Jesus; dieser blickte
ihn an und sagte: »Du bist Simon, der Sohn des Johannes; du sollst
Kephas -- das heißt übersetzt ›Fels‹\textless sup title=``vgl. Mt
16,18''\textgreater✲ -- heißen.«

\bibleverse{43} Am folgenden Tage wollte Jesus nach Galiläa aufbrechen;
da traf er Philippus und sagte zu ihm: »Folge mir nach!« \bibleverse{44}
Philippus stammte aber aus Bethsaida, dem Heimatort des Andreas und des
Petrus. \bibleverse{45} Philippus traf (darauf) den
Nathanael\textless sup title=``vgl. zu Mt 10,3''\textgreater✲ und
berichtete ihm: »Wir haben den gefunden, von welchem Mose im Gesetz und
die Propheten geschrieben haben, Jesus, den Sohn Josephs, aus Nazareth.«
\bibleverse{46} Da sagte Nathanael zu ihm: »Kann aus Nazareth etwas
Gutes kommen?« Philippus erwiderte ihm: »Komm mit uns sieh!«
\bibleverse{47} Als Jesus den Nathanael auf sich zukommen sah, sagte er
von ihm: »Siehe da, in Wahrheit ein Israelit, in dem kein Falsch ist!«
\bibleverse{48} Nathanael fragte ihn: »Woher kennst du mich?« Jesus
antwortete ihm mit den Worten: »Noch ehe Philippus dich rief, als du
unter dem Feigenbaum warst, habe ich dich gesehen.« \bibleverse{49} Da
antwortete ihm Nathanael: »Rabbi✲, du bist Gottes Sohn, du bist der
König von Israel!« \bibleverse{50} Jesus gab ihm zur Antwort: »Du
glaubst (an mich), weil ich dir gesagt habe, daß ich dich unter dem
Feigenbaum gesehen habe? Du wirst noch Größeres als dies zu sehen
bekommen.« \bibleverse{51} Dann fuhr er fort: »Wahrlich, wahrlich ich
sage euch: Ihr werdet den Himmel offen sehen und die Engel Gottes über
dem Menschensohn hinauf- und herabsteigen sehen.«

\hypertarget{ii.-jesus-offenbart-seine-guxf6ttliche-herrlichkeit-vor-der-welt-kap.-2-12}{%
\subsection{II. Jesus offenbart seine göttliche Herrlichkeit vor der
Welt (Kap.
2-12)}\label{ii.-jesus-offenbart-seine-guxf6ttliche-herrlichkeit-vor-der-welt-kap.-2-12}}

\hypertarget{jesu-erstes-wunderzeichen-auf-der-hochzeit-zu-kana}{%
\subsubsection{1. Jesu erstes Wunderzeichen auf der Hochzeit zu
Kana}\label{jesu-erstes-wunderzeichen-auf-der-hochzeit-zu-kana}}

\hypertarget{section-1}{%
\section{2}\label{section-1}}

\bibleverse{1} Am dritten Tage darauf fand zu Kana in Galiläa eine
Hochzeit statt, und die Mutter Jesu nahm daran teil; \bibleverse{2} aber
auch Jesus wurde mit seinen Jüngern zu der Hochzeit eingeladen.
\bibleverse{3} Als es nun an Wein mangelte, sagte die Mutter Jesu zu
ihm: »Sie haben keinen Wein (mehr)!« \bibleverse{4} Jesus antwortete
ihr: »Was kümmern dich meine Angelegenheiten, Frau? Meine Stunde ist
noch nicht gekommen.« \bibleverse{5} Seine Mutter sagte dann zu den
Aufwärtern: »Was er euch etwa sagt, das tut.« \bibleverse{6} Nun waren
dort sechs steinerne Wassergefäße aufgestellt, wie es die Sitte der
jüdischen Reinigung erforderte; jedes von ihnen faßte zwei bis drei
große Eimer. \bibleverse{7} Da sagte Jesus zu den Aufwärtern: »Füllt die
Gefäße mit Wasser!« Sie füllten sie darauf bis oben hin. \bibleverse{8}
Dann sagte er zu ihnen: »Schöpft nun davon und bringt es dem
Speisemeister!« Sie brachten es hin. \bibleverse{9} Als aber der
Speisemeister das zu Wein gewordene Wasser gekostet hatte, ohne zu
wissen, woher es gekommen war -- die Aufwärter aber, die das Wasser
geschöpft hatten, wußten es --, ließ der Speisemeister den Bräutigam
rufen \bibleverse{10} und sagte zu ihm: »Jedermann setzt doch (seinen
Gästen) zuerst den guten Wein vor und, wenn sie trunken geworden sind,
dann den geringeren; du aber hast den guten Wein bis jetzt
zurückbehalten.«

\bibleverse{11} Hiermit machte Jesus den Anfang seiner Zeichen✲ zu Kana
in Galiläa; er offenbarte dadurch seine (göttliche) Herrlichkeit, und
seine Jünger lernten an ihn glauben.

\hypertarget{jesus-zum-erstenmal-in-jerusalem-am-passahfest}{%
\subsubsection{2. Jesus zum erstenmal in Jerusalem am
Passahfest}\label{jesus-zum-erstenmal-in-jerusalem-am-passahfest}}

\hypertarget{a-jesus-in-kapernaum-seine-reise-nach-jerusalem-die-reinigung-des-tempels}{%
\paragraph{a) Jesus in Kapernaum; seine Reise nach Jerusalem; die
Reinigung des
Tempels}\label{a-jesus-in-kapernaum-seine-reise-nach-jerusalem-die-reinigung-des-tempels}}

\bibleverse{12} Hierauf zog er nach Kapernaum hinab, er, seine Mutter,
seine Brüder und seine Jünger; sie blieben dort aber nur wenige Tage;
\bibleverse{13} denn weil das Passah der Juden nahe bevorstand, zog
Jesus nach Jerusalem hinauf. \bibleverse{14} Er fand dort im Tempel die
Verkäufer von Rindern, Schafen und Tauben und die Geldwechsler sitzen.
\bibleverse{15} Da flocht er sich eine Geißel\textless sup title=``oder:
Peitsche''\textgreater✲ aus Stricken und trieb sie alle samt ihren
Schafen und Rindern aus dem Tempel hinaus, verschüttete den Wechslern
das Geld und stieß ihnen die Tische um \bibleverse{16} und rief den
Taubenhändlern zu: »Schafft das weg von hier, macht das Haus meines
Vaters nicht zu einem Kaufhaus!« \bibleverse{17} Da dachten seine Jünger
an das Schriftwort\textless sup title=``Ps 69,10''\textgreater✲: »Der
Eifer um dein Haus wird mich verzehren.« \bibleverse{18} Nun richteten
die Juden die Frage an ihn: »Welches Wunderzeichen läßt du uns sehen
(zum Beweis dafür), daß du so vorgehen darfst?« \bibleverse{19} Jesus
antwortete ihnen mit den Worten: »Brecht diesen Tempel ab, so werde ich
ihn in drei Tagen wieder erstehen lassen!« \bibleverse{20} Da sagten die
Juden: »Sechsundvierzig Jahre lang hat man an diesem Tempel gebaut, und
du willst ihn in drei Tagen wieder erstehen lassen?« \bibleverse{21}
Jesus hatte aber den Tempel seines eigenen Leibes gemeint.
\bibleverse{22} Als er nun (später) von den Toten auferweckt worden war,
dachten seine Jünger an diese seine Worte und kamen zum Glauben an die
Schrift und an den Ausspruch, den Jesus (damals) getan hatte.

\hypertarget{b-jesus-und-nikodemus}{%
\paragraph{b) Jesus und Nikodemus}\label{b-jesus-und-nikodemus}}

\hypertarget{aa-einfuxfchrung-das-wirken-jesu-in-jerusalem-inmitten-des-unglaubens-und-halbglaubens-des-volkes}{%
\subparagraph{aa) Einführung: Das Wirken Jesu in Jerusalem inmitten des
Unglaubens und Halbglaubens des
Volkes}\label{aa-einfuxfchrung-das-wirken-jesu-in-jerusalem-inmitten-des-unglaubens-und-halbglaubens-des-volkes}}

\bibleverse{23} Während er sich nun am Passahfest\textless sup
title=``vgl. V.13''\textgreater✲ in Jerusalem aufhielt, kamen viele zum
Glauben an seinen Namen, weil sie die Wunderzeichen sahen, die er tat.
\bibleverse{24} Jesus selbst aber vertraute sich ihnen nicht an, weil er
alle kannte \bibleverse{25} und von niemand ein Zeugnis\textless sup
title=``oder: eine Auskunft''\textgreater✲ über irgendeinen Menschen
nötig hatte; denn er erkannte von sich selbst aus, wie es innerlich mit
jedem Menschen stand.

\hypertarget{bb-das-gespruxe4ch-mit-nikodemus-uxfcber-die-innerliche-grundlegung-des-reiches-gottes-d.h.-uxfcber-die-wiedergeburt-den-neuen-heilsweg-und-den-rechten-glauben}{%
\subparagraph{bb) Das Gespräch mit Nikodemus über die innerliche
Grundlegung des Reiches Gottes (d.h. über die Wiedergeburt, den neuen
Heilsweg und den rechten
Glauben)}\label{bb-das-gespruxe4ch-mit-nikodemus-uxfcber-die-innerliche-grundlegung-des-reiches-gottes-d.h.-uxfcber-die-wiedergeburt-den-neuen-heilsweg-und-den-rechten-glauben}}

\hypertarget{section-2}{%
\section{3}\label{section-2}}

\bibleverse{1} Nun war da unter den Pharisäern ein Mann namens
Nikodemus, ein Mitglied des Hohen Rates der Juden; \bibleverse{2} dieser
kam zu Jesus bei Nacht und sagte zu ihm: »Rabbi✲, wir wissen: du bist
als Lehrer von Gott gekommen; denn niemand kann solche Wunderzeichen
tun, wie du sie tust, wenn Gott nicht mit ihm ist.« \bibleverse{3} Jesus
gab ihm zur Antwort: »Wahrlich, wahrlich ich sage dir: Wenn jemand nicht
von oben her\textless sup title=``oder: von neuem''\textgreater✲ geboren
wird, kann er das Reich Gottes nicht sehen.« \bibleverse{4} Nikodemus
entgegnete ihm: »Wie kann jemand geboren werden, wenn er ein Greis ist?
Kann er etwa zum zweitenmal in den Schoß seiner Mutter eingehen und
geboren werden?« \bibleverse{5} Jesus antwortete: »Wahrlich, wahrlich
ich sage dir: Wenn jemand nicht aus\textless sup title=``oder:
durch''\textgreater✲ Wasser und Geist geboren wird, kann er nicht in das
Reich Gottes eingehen. \bibleverse{6} Was aus dem\textless sup
title=``oder: vom''\textgreater✲ Fleisch geboren ist, das ist Fleisch,
und was aus dem\textless sup title=``oder: vom''\textgreater✲ Geist
geboren ist, das ist Geist. \bibleverse{7} Wundere dich nicht, daß ich
zu dir gesagt habe: Ihr müßt von oben her\textless sup title=``oder: von
neuem''\textgreater✲ geboren werden. \bibleverse{8} Der Wind weht, wo er
will, und du hörst sein Sausen wohl, weißt aber nicht, woher er kommt
und wohin er fährt. Ebenso verhält es sich auch mit jedem, der aus dem
Geist geboren ist.«

\bibleverse{9} Nikodemus entgegnete ihm mit der Frage: »Wie ist das
möglich?« \bibleverse{10} Jesus gab ihm zur Antwort: »Du bist der Lehrer
Israels und verstehst das nicht? \bibleverse{11} Wahrlich, wahrlich ich
sage dir: Wir reden, was wir wissen, und geben Zeugnis von dem, was wir
gesehen haben, und doch nehmt ihr unser Zeugnis nicht an.
\bibleverse{12} Wenn ich von den irdischen Dingen zu euch geredet habe
und ihr nicht glaubt: wie werdet ihr da glauben, wenn ich von den
himmlischen Dingen zu euch rede? \bibleverse{13} Und niemand ist in den
Himmel hinaufgestiegen außer dem einen, der aus dem Himmel herabgekommen
ist, (nämlich) der Menschensohn, der im Himmel ist. \bibleverse{14} Und
wie Mose die Schlange in der Wüste erhöht hat, so muß auch der
Menschensohn erhöht werden, \bibleverse{15} damit alle, die (an ihn)
glauben, in ihm ewiges Leben haben. \bibleverse{16} Denn so sehr hat
Gott die Welt geliebt, daß er seinen eingeborenen✲ Sohn hingegeben hat,
damit alle, die an ihn glauben, nicht verloren gehen, sondern ewiges
Leben haben. \bibleverse{17} Denn Gott hat seinen Sohn nicht dazu in die
Welt gesandt, daß er die Welt richte, sondern daß die Welt durch ihn
gerettet werde. \bibleverse{18} Wer an ihn glaubt, wird nicht gerichtet;
wer nicht (an ihn) glaubt, ist schon gerichtet, weil er nicht an den
Namen des eingeborenen Sohnes Gottes geglaubt hat. \bibleverse{19} Darin
besteht aber das Gericht, daß das Licht in die Welt gekommen ist, die
Menschen aber die Finsternis mehr geliebt haben als das Licht, denn ihre
Werke\textless sup title=``d.h. ihr ganzes Tun''\textgreater✲ waren
böse. \bibleverse{20} Denn jeder, der Nichtiges treibt, haßt das Licht
und kommt nicht zum\textless sup title=``oder: an das''\textgreater✲
Licht, damit seine Werke\textless sup title=``d.h. sein ganzes
Tun''\textgreater✲ nicht bloßgestellt\textless sup title=``oder:
aufgedeckt''\textgreater✲ werden; \bibleverse{21} wer aber die Wahrheit
tut\textless sup title=``oder: übt''\textgreater✲, der kommt
zum\textless sup title=``oder: an das''\textgreater✲ Licht, damit seine
Werke offenbar werden, denn sie sind in Gott getan.«

\hypertarget{jesus-in-juduxe4a-und-das-letzte-abschlieuxdfende-zeugnis-des-tuxe4ufers-fuxfcr-die-gottessohnschaft-jesu}{%
\subsubsection{3. Jesus in Judäa und das letzte abschließende Zeugnis
des Täufers (für die Gottessohnschaft
Jesu)}\label{jesus-in-juduxe4a-und-das-letzte-abschlieuxdfende-zeugnis-des-tuxe4ufers-fuxfcr-die-gottessohnschaft-jesu}}

\bibleverse{22} Hierauf begab sich Jesus mit seinen Jüngern in die
Landschaft Judäa und blieb dort längere Zeit mit ihnen und taufte.
\bibleverse{23} Aber auch Johannes war (damals noch) als Täufer zu Änon
in der Nähe von Salim tätig, weil es dort reichlich Wasser gab; und die
Leute kamen dorthin und ließen sich taufen; \bibleverse{24} Johannes war
nämlich damals noch nicht ins Gefängnis geworfen. \bibleverse{25} Da kam
es denn zu einem Streite von seiten der Jünger des Johannes mit einem
Juden über die Reinigung (durch die Taufe); \bibleverse{26} und sie
kamen zu Johannes und berichteten ihm: »Rabbi✲, der Mann, der jenseits
des Jordans bei dir war und für den du mit deinem Zeugnis eingetreten
bist, denke nur: der tauft (jetzt auch), und alle laufen ihm zu.«
\bibleverse{27} Da gab Johannes ihnen zur Antwort: »Kein Mensch kann
sich etwas nehmen, wenn es ihm nicht vom Himmel her gegeben ist.
\bibleverse{28} Ihr selbst könnt mir bezeugen, daß ich gesagt habe: ›Ich
bin nicht Christus\textless sup title=``=~der Messias''\textgreater✲,
sondern bin nur als sein Vorläufer gesandt.‹ \bibleverse{29} Wer die
Braut hat, ist der Bräutigam; der Freund des Bräutigams aber, der
dabeisteht und ihm zuhört, freut sich von Herzen über den Jubelruf des
Bräutigams. Diese meine Freude ist nun vollkommen geworden.
\bibleverse{30} Er muß wachsen, ich dagegen muß abnehmen.
\bibleverse{31} Er, der von oben her kommt, steht höher als alle
anderen; wer von der Erde her stammt, der gehört zur Erde und redet von
der Erde her\textless sup title=``d.h. was irdisch ist''\textgreater✲.
Er, der aus dem Himmel kommt, steht über allen anderen; \bibleverse{32}
er legt Zeugnis von dem ab, was er (im Himmel) gesehen und gehört hat,
und doch nimmt niemand sein Zeugnis an. \bibleverse{33} Wer sein Zeugnis
angenommen hat, der hat damit besiegelt✲, daß Gott wahrhaftig ist.
\bibleverse{34} Denn der, den Gott gesandt hat, redet die Worte Gottes;
denn Gott verleiht den Geist nicht nach einem Maß\textless sup
title=``d.h. vielmehr in unbegrenzter Fülle''\textgreater✲.
\bibleverse{35} Der Vater liebt den Sohn und hat alles in seine Hand
gegeben. \bibleverse{36} Wer an den Sohn glaubt, hat ewiges Leben; wer
aber dem Sohne ungehorsam bleibt, wird das Leben nicht zu sehen
bekommen, sondern der Zorn Gottes bleibt auf ihn gerichtet\textless sup
title=``=~bleibt über ihm''\textgreater✲.«

\hypertarget{jesus-in-samarien}{%
\subsubsection{4. Jesus in Samarien}\label{jesus-in-samarien}}

\hypertarget{a-jesu-ruxfcckkehr-aus-juduxe4a-sein-gespruxe4ch-mit-der-samariterin-am-jakobsbrunnen-seine-selbstoffenbarung}{%
\paragraph{a) Jesu Rückkehr aus Judäa; sein Gespräch mit der Samariterin
am Jakobsbrunnen; seine
Selbstoffenbarung}\label{a-jesu-ruxfcckkehr-aus-juduxe4a-sein-gespruxe4ch-mit-der-samariterin-am-jakobsbrunnen-seine-selbstoffenbarung}}

\hypertarget{section-3}{%
\section{4}\label{section-3}}

\bibleverse{1} Als nun der Herr erfuhr, den Pharisäern sei zu Ohren
gekommen, daß Jesus mehr Jünger gewinne und taufe als Johannes~--
\bibleverse{2} übrigens taufte Jesus nicht selbst, sondern nur seine
Jünger --, \bibleverse{3} verließ er Judäa und kehrte wieder nach
Galiläa zurück; \bibleverse{4} dabei mußte er aber seinen Weg durch
Samaria nehmen. \bibleverse{5} So kam er denn ins Gebiet einer
samaritischen Stadt namens Sychar, die nahe bei dem Felde✲ liegt, das
Jakob einst seinem Sohne Joseph geschenkt hatte. \bibleverse{6} Dort war
aber der Jakobsbrunnen. Weil nun Jesus von der Wanderung ermüdet war,
setzte er sich ohne weiteres am Brunnen nieder; es war ungefähr die
sechste Stunde\textless sup title=``=~Mittagszeit; vgl.
1,39''\textgreater✲.

\bibleverse{7} Da kam eine samaritische Frau, um Wasser zu schöpfen.
Jesus bat sie: »Gib mir zu trinken!« \bibleverse{8} Seine Jünger waren
nämlich in die Stadt weggegangen, um Lebensmittel zu kaufen.
\bibleverse{9} Da sagte die Samariterin zu ihm: »Wie kommst du dazu, da
du doch ein Jude bist, von mir, einer Samariterin, einen Trunk zu
erbitten?« -- Die Juden haben nämlich mit den Samaritern keinen
Verkehr.~-- \bibleverse{10} Jesus gab ihr zur Antwort: »Wenn du die Gabe
Gottes\textless sup title=``d.h. welche Gott gibt; vgl.
3,16''\textgreater✲ kenntest und wüßtest, wer der ist, der einen Trunk
von dir wünscht, so würdest du ihn bitten, und er würde dir lebendiges
Wasser geben.« \bibleverse{11} Da erwiderte ihm die Frau: »Herr, du hast
ja kein Gefäß\textless sup title=``=~keinen Eimer''\textgreater✲ zum
Schöpfen, und der Brunnen ist tief: woher willst du denn das lebendige
Wasser nehmen? \bibleverse{12} Du bist doch nicht mehr✲ als unser Vater
Jakob, der uns den Brunnen gegeben hat? Und er selbst hat aus ihm
getrunken samt seinen Söhnen\textless sup title=``oder:
Kindern''\textgreater✲ und seinen Herden.« \bibleverse{13} Jesus
antwortete ihr: »Jeder, der von diesem Wasser trinkt, wird wieder
dürsten; \bibleverse{14} wer aber von dem Wasser trinkt, das ich ihm
geben werde, der wird in Ewigkeit nicht wieder Durst leiden, sondern das
Wasser, das ich ihm geben werde, wird in ihm zu einer Wasserquelle
werden, die zu ewigem Leben sprudelt.« \bibleverse{15} Die Frau
antwortete ihm: »Herr, gib mir dieses Wasser, damit ich nicht wieder
durstig werde und nicht mehr hierher zu kommen brauche, um Wasser zu
holen!«

\bibleverse{16} Da sagte Jesus zu ihr: »Geh hin, rufe deinen Mann und
komm dann wieder hierher!« \bibleverse{17} Die Frau antwortete: »Ich
habe keinen Mann.« Jesus erwiderte ihr: »Du hast mit Recht gesagt: ›Ich
habe keinen Mann‹; \bibleverse{18} denn fünf Männer hast du gehabt, und
der, den du jetzt hast, ist nicht dein Ehemann; damit hast du die
Wahrheit gesagt.« \bibleverse{19} Die Frau entgegnete ihm: »Herr, ich
sehe: du bist ein Prophet. \bibleverse{20} Unsere Väter haben auf dem
Berge dort (Gott) angebetet, und ihr behauptet, in Jerusalem sei die
Stätte, wo man anbeten müsse.« \bibleverse{21} Jesus erwiderte ihr:
»Glaube mir, Frau: die Stunde kommt, in der ihr weder auf dem Berge dort
noch in Jerusalem den Vater anbeten werdet. \bibleverse{22} Ihr betet
an, was ihr nicht kennt; wir beten an, was wir kennen; denn die Rettung
ist aus den Juden. \bibleverse{23} Es kommt aber die Stunde, ja, sie ist
jetzt schon da, in der die wahren Anbeter den Vater im Geist und in
Wahrheit anbeten werden; denn auch der Vater will solche als seine
Anbeter haben. \bibleverse{24} Gott ist Geist, und die ihn anbeten,
müssen ihn im Geist und in Wahrheit anbeten.« \bibleverse{25} Da sagte
die Frau zu ihm: »Ich weiß, daß der Messias\textless sup title=``d.h.
der Gesalbte''\textgreater✲ kommt, den man Christus nennt; wenn der
kommt, wird er uns über alles Auskunft geben.« \bibleverse{26} Jesus
antwortete ihr: »Ich bin's, der mit dir redet.«

\hypertarget{b-jesus-und-die-juxfcnger}{%
\paragraph{b) Jesus und die Jünger}\label{b-jesus-und-die-juxfcnger}}

\bibleverse{27} In diesem Augenblick kamen seine Jünger und wunderten
sich darüber, daß er mit einer Frau redete; doch fragte ihn keiner: »Was
willst du (von ihr)?« oder: »Wozu redest du mit ihr?« \bibleverse{28} Da
ließ nun die Frau ihren Wasserkrug stehen, ging in die Stadt zurück und
sagte zu den Leuten dort: \bibleverse{29} »Kommt und seht einen Mann,
der mir alles gesagt hat, was ich getan habe! Sollte dieser vielleicht
Christus\textless sup title=``=~der Messias''\textgreater✲ sein?«
\bibleverse{30} Da gingen sie aus der Stadt hinaus und begaben sich zu
ihm.

\bibleverse{31} Inzwischen baten ihn seine Jünger: »Rabbi\textless sup
title=``oder: Meister''\textgreater✲, iß!« \bibleverse{32} Er antwortete
ihnen aber: »Ich habe eine Speise zu essen, von der ihr nichts wißt.«
\bibleverse{33} Da sagten die Jünger zueinander: »Hat ihm denn jemand zu
essen gebracht?« \bibleverse{34} Jesus erwiderte ihnen: »Meine Speise
ist die, daß ich den Willen dessen tue, der mich gesandt hat, und sein
Werk vollende. \bibleverse{35} Sagt ihr nicht selbst: ›Es währt noch
vier Monate, bis die Ernte kommt‹? Nun sage ich euch: Laßt eure Augen
ausschauen und seht die Felder an: sie sind (schon jetzt) weiß zur
Ernte. \bibleverse{36} Nunmehr empfängt der Schnitter\textless sup
title=``oder: Erntende''\textgreater✲ Lohn, und zwar dadurch, daß er
Frucht sammelt zu ewigem Leben, damit beide sich gemeinsam freuen, der
Sämann und der Schnitter\textless sup title=``oder:
Erntende''\textgreater✲. \bibleverse{37} Denn in diesem Falle trifft das
Sprichwort zu: ›Ein anderer ist's, der da sät, und ein anderer, der da
erntet.‹ \bibleverse{38} Ich habe euch ausgesandt, um das zu ernten,
wofür ihr nicht gearbeitet habt: andere haben die Arbeit geleistet, und
ihr seid in ihre Arbeit eingetreten.«

\hypertarget{c-jesus-und-die-samariter-wunderglaube-und-erfahrungsglaube}{%
\paragraph{c) Jesus und die Samariter -- Wunderglaube und
Erfahrungsglaube}\label{c-jesus-und-die-samariter-wunderglaube-und-erfahrungsglaube}}

\bibleverse{39} Aus jener Stadt aber wurden viele von den Samaritern an
ihn gläubig infolge der Versicherung der Frau: »Er hat mir alles gesagt,
was ich getan habe.« \bibleverse{40} Als nun die Samariter zu ihm
gekommen waren, baten sie ihn, er möchte bei ihnen bleiben; und er blieb
auch zwei Tage dort. \bibleverse{41} Da wurden noch viel mehr Leute
infolge seiner Predigt gläubig \bibleverse{42} und sagten zu der Frau:
»Wir glauben jetzt nicht mehr infolge deiner Aussage; denn wir haben
nunmehr selbst gehört und wissen, daß dieser wirklich der Retter✲ der
Welt ist.«

\hypertarget{jesus-in-galiluxe4a-heilung-des-sohnes-eines-kuxf6niglichen-d.h.-juxfcdischen-beamten-in-kapernaum-glaubenswilligkeit-und-gruxfcndung-des-glaubens-auf-das-wort}{%
\subsubsection{5. Jesus in Galiläa; Heilung des Sohnes eines königlichen
(d.h. jüdischen) Beamten in Kapernaum (Glaubenswilligkeit und Gründung
des Glaubens auf das
Wort)}\label{jesus-in-galiluxe4a-heilung-des-sohnes-eines-kuxf6niglichen-d.h.-juxfcdischen-beamten-in-kapernaum-glaubenswilligkeit-und-gruxfcndung-des-glaubens-auf-das-wort}}

\bibleverse{43} Nach Verlauf der beiden Tage aber zog Jesus von dort
weiter nach Galiläa\textless sup title=``Mt 4,12''\textgreater✲,
\bibleverse{44} wiewohl er selbst ausdrücklich erklärt hatte, daß ein
Prophet in seiner eigenen Heimat keine Anerkennung finde\textless sup
title=``Mt 13,57''\textgreater✲. \bibleverse{45} Doch als er nach
Galiläa kam, nahmen ihn die Galiläer gastlich auf, weil sie alles
gesehen hatten, was er in Jerusalem während des Festes getan hatte; denn
sie waren gleichfalls auf dem Fest gewesen. \bibleverse{46} So kam er
denn wieder nach Kana in Galiläa, wo er das Wasser in Wein verwandelt
hatte.

Es war aber in Kapernaum ein königlicher Beamter, dessen Sohn krank
darniederlag. \bibleverse{47} Als dieser hörte, daß Jesus aus Judäa nach
Galiläa gekommen sei, begab er sich zu ihm und bat ihn, er möchte (nach
Kapernaum) hinunterkommen und seinen Sohn heilen; denn dieser lag im
Sterben. \bibleverse{48} Da sagte Jesus zu ihm: »Wenn ihr nicht Zeichen
und Wunder seht, glaubt ihr überhaupt nicht!« \bibleverse{49} Der
königliche Beamte entgegnete ihm: »Herr, komm doch hinab, ehe mein Kind
stirbt!« \bibleverse{50} Jesus antwortete ihm: »Gehe heim, dein Sohn
lebt!« Der Mann glaubte der Versicherung, die Jesus ihm gegeben hatte,
und machte sich auf den Heimweg, \bibleverse{51} und schon während
seiner Rückkehr kamen ihm seine Knechte mit der Meldung entgegen, daß
sein Sohn lebe. \bibleverse{52} Da erkundigte er sich bei ihnen nach der
Stunde, in der sein Befinden sich gebessert habe. Sie antworteten ihm:
»Gestern in der siebten Stunde hat das Fieber ihn verlassen.«
\bibleverse{53} Nun erkannte der Vater, daß es in jener Stunde geschehen
war, in der Jesus zu ihm gesagt hatte: »Dein Sohn lebt«; und er wurde
gläubig mit seinem ganzen Hause. \bibleverse{54} Dies ist das zweite
Wunderzeichen, das Jesus wiederum (in Kana) nach seiner Rückkehr aus
Judäa nach Galiläa getan hat.

\hypertarget{der-erste-grouxdfe-kampf-jesu-mit-den-ungluxe4ubigen-juden-wuxe4hrend-seines-zweiten-aufenthalts-in-jerusalem}{%
\subsubsection{6. Der erste große Kampf Jesu mit den ungläubigen Juden
während seines zweiten Aufenthalts in
Jerusalem}\label{der-erste-grouxdfe-kampf-jesu-mit-den-ungluxe4ubigen-juden-wuxe4hrend-seines-zweiten-aufenthalts-in-jerusalem}}

\hypertarget{a-heilung-des-kranken-am-teich-bethesda-bei-jerusalem-und-sabbatstreit}{%
\paragraph{a) Heilung des Kranken am Teich Bethesda bei Jerusalem und
Sabbatstreit}\label{a-heilung-des-kranken-am-teich-bethesda-bei-jerusalem-und-sabbatstreit}}

\hypertarget{section-4}{%
\section{5}\label{section-4}}

\bibleverse{1} Hierauf fand ein Fest der Juden statt, und Jesus zog nach
Jerusalem hinauf. \bibleverse{2} Nun liegt in Jerusalem am Schaftor ein
Teich, der auf hebräisch Bethesda heißt und fünf Hallen hat.
\bibleverse{3} In diesen lagen Kranke in großer Zahl, Blinde, Lahme und
Schwindsüchtige {[}die auf die Bewegung des Wassers warteten.
\bibleverse{4} Ein Engel des Herrn stieg nämlich von Zeit zu Zeit in den
Teich hinab und setzte das Wasser in Bewegung. Wer dann nach der
Bewegung\textless sup title=``=~nach dem Aufwallen''\textgreater✲ des
Wassers zuerst hineinstieg, der wurde gesund, gleichviel mit welchem
Leiden er behaftet war{]}. \bibleverse{5} Nun lag dort ein Mann, der
schon achtunddreißig Jahre an seiner Krankheit gelitten hatte.
\bibleverse{6} Als Jesus diesen daliegen sah und erfuhr, daß er schon so
lange Zeit als Kranker dort zugebracht hatte, fragte er ihn: »Willst du
gesund werden?« \bibleverse{7} Der Kranke antwortete ihm: »Ach, Herr,
ich habe keinen Menschen, der mich in den Teich schafft, wenn das Wasser
in Bewegung gerät; während ich aber hingehe, steigt immer schon ein
anderer vor mir hinab.« \bibleverse{8} Jesus sagte zu ihm: »Steh auf,
nimm dein Bett✲ auf dich und bewege dich frei!« \bibleverse{9} Da wurde
der Mann sogleich gesund, nahm sein Bett auf sich und ging umher.

Es war aber (gerade) Sabbat an jenem Tage. \bibleverse{10} Daher sagten
die Juden zu dem Geheilten: »Heute ist Sabbat; da darfst du das
Bett\textless sup title=``=~die Bahre''\textgreater✲ nicht tragen!«
\bibleverse{11} Doch er antwortete ihnen: »Der Mann, der mich gesund
gemacht hat, der hat zu mir gesagt: ›Nimm dein Bett auf dich und bewege
dich frei!‹« \bibleverse{12} Sie fragten ihn: »Wer ist der Mann, der zu
dir gesagt hat: ›Nimm es auf dich und gehe umher!‹?« \bibleverse{13} Der
Geheilte wußte aber nicht, wer es war; denn Jesus hatte sich in der
Menschenmenge, die sich an dem Orte befand, unbemerkt entfernt.
\bibleverse{14} Später traf Jesus ihn im Tempel wieder und sagte zu ihm:
»Du bist nun gesund geworden; sündige fortan nicht mehr, damit dir nicht
noch Schlimmeres widerfährt!« \bibleverse{15} Da ging der Mann hin und
teilte den Juden mit, Jesus sei es, der ihn gesund gemacht habe.
\bibleverse{16} Deshalb verfolgten die Juden Jesus, weil er solche Werke
(auch) am Sabbat tat. \bibleverse{17} Jesus aber antwortete ihnen: »Mein
Vater wirkt (ununterbrochen) bis zu dieser Stunde; darum wirke ich
auch.« \bibleverse{18} Deshalb trachteten die Juden ihm um so mehr nach
dem Leben, weil er nicht nur den Sabbat brach, sondern auch Gott seinen
eigenen Vater nannte und sich damit Gott gleichstellte.

\hypertarget{b-jesu-selbstzeugnis-von-seinem-gottgleichen-wirken-und-von-seiner-gottessohnschaft-jesus-als-richter-und-lebenspender}{%
\paragraph{b) Jesu Selbstzeugnis von seinem gottgleichen Wirken und von
seiner Gottessohnschaft; Jesus als Richter und
Lebenspender}\label{b-jesu-selbstzeugnis-von-seinem-gottgleichen-wirken-und-von-seiner-gottessohnschaft-jesus-als-richter-und-lebenspender}}

\bibleverse{19} Daher sprach sich Jesus ihnen gegenüber so aus:
»Wahrlich, wahrlich ich sage euch: der Sohn vermag von sich selber aus
nichts zu tun, als was er den Vater tun sieht; denn was jener tut, das
tut in gleicher Weise auch der Sohn. \bibleverse{20} Denn der Vater hat
den Sohn lieb und läßt ihn alles sehen, was er selbst tut; und er wird
ihn noch größere Werke als diese\textless sup title=``=~die
bisherigen''\textgreater✲ sehen lassen, damit ihr euch wundert.
\bibleverse{21} Denn wie der Vater die Toten auferweckt und lebendig
macht, ebenso macht auch der Sohn lebendig, welche er will.
\bibleverse{22} Denn auch der Vater ist es nicht, der jemand richtet;
sondern er hat das Gericht ganz dem Sohne übertragen, \bibleverse{23}
damit alle den Sohn ebenso ehren, wie sie den Vater ehren. Wer den Sohn
nicht ehrt, ehrt auch den Vater nicht, der ihn gesandt hat.
\bibleverse{24} Wahrlich, wahrlich ich sage euch: Wer mein Wort hört und
dem glaubt, der mich gesandt hat, der hat ewiges Leben und kommt nicht
ins Gericht, sondern ist aus dem Tode ins Leben hinübergegangen.
\bibleverse{25} Wahrlich, wahrlich ich sage euch: Es kommt die Stunde,
ja sie ist jetzt schon da, wo die Toten die Stimme\textless sup
title=``=~den Ruf''\textgreater✲ des Sohnes Gottes hören werden, und
die, welche auf sie hören, werden leben. \bibleverse{26} Denn wie der
Vater (das) Leben in sich selbst hat, so hat er auch dem Sohne
verliehen, (das) Leben in sich selbst zu haben; \bibleverse{27} und er
hat ihm Vollmacht\textless sup title=``oder: die Macht''\textgreater✲
gegeben, Gericht abzuhalten, weil er ein Menschensohn ist.
\bibleverse{28} Wundert euch nicht hierüber! Denn die Stunde kommt, in
der alle, die in den Gräbern ruhen, seine Stimme\textless sup
title=``=~seinen Ruf''\textgreater✲ hören werden, \bibleverse{29} und es
werden hervorgehen: die einen, die das Gute getan haben, zur
Auferstehung für das Leben, die anderen aber, die das Böse betrieben
haben, zur Auferstehung für das Gericht. \bibleverse{30} Ich vermag
nichts von mir selbst aus zu tun; nein, wie ich es (vom Vater) höre, so
richte ich, und mein Gericht ist gerecht, weil ich nicht meinen Willen
(durchzuführen) suche, sondern den Willen dessen, der mich gesandt hat.«

\hypertarget{c-jesu-rede-von-den-zeugen-fuxfcr-seine-gottessohnschaft-der-unglaube-der-juden-und-seine-gruxfcnde}{%
\paragraph{c) Jesu Rede von den Zeugen für seine Gottessohnschaft, der
Unglaube der Juden und seine
Gründe}\label{c-jesu-rede-von-den-zeugen-fuxfcr-seine-gottessohnschaft-der-unglaube-der-juden-und-seine-gruxfcnde}}

\hypertarget{aa-das-zeugnis-des-johannes}{%
\subparagraph{aa) Das Zeugnis des
Johannes}\label{aa-das-zeugnis-des-johannes}}

\bibleverse{31} »Wenn ich über mich\textless sup title=``oder: für
mich''\textgreater✲ selbst Zeugnis ablege, so ist mein Zeugnis ungültig.
\bibleverse{32} (Nein) ein anderer ist es, der mit seinem Zeugnis für
mich eintritt, und ich weiß, daß das Zeugnis, das er über
mich\textless sup title=``oder: für mich''\textgreater✲ ablegt, wahr
ist. \bibleverse{33} Ihr habt zu Johannes gesandt, und er hat Zeugnis
für die Wahrheit abgelegt; \bibleverse{34} ich aber nehme das Zeugnis
von einem Menschen nicht an, sondern erwähne dies nur deshalb, damit ihr
gerettet werdet. \bibleverse{35} Jener war wirklich die Leuchte, die mit
hellem Schein brannte; ihr aber wolltet euch nur eine Zeitlang an ihrem
Lichtschein vergnügen.«

\hypertarget{bb-das-zeugnis-des-vaters}{%
\subparagraph{bb) Das Zeugnis des
Vaters}\label{bb-das-zeugnis-des-vaters}}

\bibleverse{36} »Ich aber habe ein Zeugnis, das gewichtiger ist als das
des Johannes; denn die Werke, die der Vater mir zu vollführen übertragen
hat, eben die Werke, die ich vollbringe, bezeugen von mir, daß der Vater
mich gesandt hat. \bibleverse{37} So ist also, der mich gesandt hat, der
Vater selbst, mit seinem Zeugnis für mich eingetreten. Ihr habt weder
seine Stimme jemals gehört noch seine Gestalt gesehen; \bibleverse{38}
und auch sein Wort habt ihr nicht als bleibenden Besitz in euch, weil
ihr dem nicht glaubt, den er gesandt hat. \bibleverse{39} Ihr
durchforscht (wohl) die (heiligen) Schriften, weil ihr in ihnen ewiges
Leben zu haben vermeint, und sie sind es auch wirklich, die von mir
Zeugnis ablegen; \bibleverse{40} aber trotzdem wollt ihr nicht zu mir
kommen, um wirklich Leben zu haben\textless sup title=``oder: zu
empfangen''\textgreater✲.«

\hypertarget{cc-angriff-auf-den-unglauben-und-die-ehrsucht-der-juden-zeugnis-des-mose}{%
\subparagraph{cc) Angriff auf den Unglauben und die Ehrsucht der Juden;
Zeugnis des
Mose}\label{cc-angriff-auf-den-unglauben-und-die-ehrsucht-der-juden-zeugnis-des-mose}}

\bibleverse{41} »Ehre von Menschen nehme ich nicht an, \bibleverse{42}
vielmehr habe ich bei euch erkannt, daß ihr die Liebe zu Gott nicht in
euch tragt. \bibleverse{43} Ich bin im Namen meines Vaters gekommen,
doch ihr nehmt mich nicht an; wenn ein anderer in seinem eigenen Namen
kommt\textless sup title=``oder: käme''\textgreater✲, den
werdet\textless sup title=``oder: würdet''\textgreater✲ ihr annehmen.
\bibleverse{44} Wie könnt ihr zum Glauben kommen, da ihr Ehre
voneinander annehmt, aber nach der Ehrung, die vom alleinigen Gott
kommt, kein Verlangen tragt? \bibleverse{45} Denkt nicht, daß ich euer
Ankläger beim Vater sein werde! Nein, es ist (ein anderer) da, der euch
anklagt, nämlich Mose, auf den ihr eure Hoffnung gesetzt habt.
\bibleverse{46} Denn wenn ihr Mose glaubtet, dann würdet ihr auch mir
glauben; denn ich bin es, von dem er geschrieben hat\textless sup
title=``1.Mose 3,15; 49,10; 5.Mose 18,15''\textgreater✲. \bibleverse{47}
Wenn ihr aber seinen Schriften nicht glaubt, wie solltet ihr da meinen
Worten Glauben schenken?«

\hypertarget{jesus-bietet-den-galiluxe4ern-das-brot-des-lebens-die-entscheidung-in-galiluxe4a}{%
\subsubsection{7. Jesus bietet den Galiläern das Brot des Lebens; die
Entscheidung in
Galiläa}\label{jesus-bietet-den-galiluxe4ern-das-brot-des-lebens-die-entscheidung-in-galiluxe4a}}

\hypertarget{a-jesus-speist-die-fuxfcnftausend}{%
\paragraph{a) Jesus speist die
Fünftausend}\label{a-jesus-speist-die-fuxfcnftausend}}

\hypertarget{section-5}{%
\section{6}\label{section-5}}

\bibleverse{1} Hierauf begab sich Jesus auf die andere Seite des
Galiläischen Sees, des Sees von Tiberias; \bibleverse{2} es zog ihm aber
dorthin eine große Volksmenge nach, weil sie die Wunderzeichen sahen,
die er an den Kranken tat. \bibleverse{3} Jesus stieg aber auf den Berg
hinauf und ließ sich dort mit seinen Jüngern nieder; \bibleverse{4} das
jüdische Passah stand aber nahe bevor. \bibleverse{5} Als nun Jesus sich
dort umschaute und eine große Volksmenge zu sich kommen sah, sagte er zu
Philippus: »Woher sollen wir Brote kaufen, damit diese zu essen haben?«
\bibleverse{6} So fragte er aber, um ihn auf die Probe zu stellen; denn
er selbst wußte wohl, was er tun wollte. \bibleverse{7} Philippus
antwortete ihm: »Für zweihundert Denare✲ Brot reicht für sie nicht hin,
damit jeder auch nur ein kleines Stück erhält.« \bibleverse{8} Da sagte
einer von seinen Jüngern, nämlich Andreas, der Bruder des Simon Petrus,
zu ihm: \bibleverse{9} »Es ist ein Knabe hier, der fünf Gerstenbrote und
zwei Fische (zum Verkauf bei sich) hat, doch was ist das für so viele?«
\bibleverse{10} Jesus aber sagte: »Laßt die Leute sich lagern!«, es war
nämlich dichter Rasen an dem Ort. So lagerten sich denn die Männer, etwa
fünftausend an Zahl. \bibleverse{11} Jesus nahm sodann die Brote, sprach
den Lobpreis (Gottes) und ließ sie unter die Leute austeilen, die sich
gelagert hatten; ebenso auch von den Fischen, soviel sie begehrten.
\bibleverse{12} Als sie dann satt geworden waren, sagte er zu seinen
Jüngern: »Sammelt die übriggebliebenen Brocken, damit nichts umkommt.«
\bibleverse{13} Da sammelten sie und füllten von den fünf Gerstenbroten
zwölf Körbe mit Brocken, die beim Essen übriggeblieben waren.
\bibleverse{14} Als nun die Leute das Wunderzeichen sahen, das er getan
hatte, erklärten sie: »Dieser ist wahrhaftig der Prophet, der in die
Welt kommen soll!« \bibleverse{15} Da nun Jesus erkannte, daß sie kommen
und sich seiner Person mit Gewalt bemächtigen würden, um ihn zum König
zu machen, zog er sich wieder auf den Berg zurück, er für sich allein.

\hypertarget{b-die-nachtfahrt-der-juxfcnger-und-das-wandeln-jesu-auf-dem-see}{%
\paragraph{b) Die Nachtfahrt der Jünger und das Wandeln Jesu auf dem
See}\label{b-die-nachtfahrt-der-juxfcnger-und-das-wandeln-jesu-auf-dem-see}}

\bibleverse{16} Als es dann Abend geworden war, gingen seine Jünger an
den See hinab, \bibleverse{17} stiegen in ein Boot und wollten über den
See nach Kapernaum hinüberfahren. Die Dunkelheit war bereits eingetreten
und Jesus immer noch nicht zu ihnen gekommen; \bibleverse{18} dabei ging
der See hoch, weil ein starker Wind wehte. \bibleverse{19} Als sie nun
etwa fünfundzwanzig bis dreißig Stadien\textless sup title=``d.h. eine
Stunde''\textgreater✲ weit gefahren waren, sahen sie Jesus über den See
hingehen und sich ihrem Boote nähern; da gerieten sie in Angst.
\bibleverse{20} Er aber rief ihnen zu: \bibleverse{21} »Ich bin's;
fürchtet euch nicht!« Sie wollten ihn nun in das Boot hineinnehmen, doch
sogleich befand sich das Boot am Lande, (und zwar da) wohin sie fahren
wollten.

\hypertarget{c-das-wiedersehen-mit-dem-volke-und-die-zeichenforderung-des-volkes}{%
\paragraph{c) Das Wiedersehen mit dem Volke und die Zeichenforderung des
Volkes}\label{c-das-wiedersehen-mit-dem-volke-und-die-zeichenforderung-des-volkes}}

\bibleverse{22} Am folgenden Tage überzeugte sich die Volksmenge, die am
jenseitigen Ufer des Sees stand\textless sup title=``=~zurückgeblieben
war''\textgreater✲, daß dort weiter kein Fahrzeug außer dem einen
gewesen war und daß Jesus nicht mit seinen Jüngern zusammen das Boot
bestiegen hatte, sondern daß seine Jünger allein abgefahren waren.
\bibleverse{23} Doch es kamen jetzt andere Fahrzeuge von Tiberias her in
die Nähe des Platzes, wo sie das Brot nach dem Dankgebet des Herrn
gegessen hatten. \bibleverse{24} Als die Volksmenge nun sah, daß Jesus
ebensowenig da war wie seine Jünger, stiegen auch sie in die Fahrzeuge
und kamen nach Kapernaum, um Jesus zu suchen. \bibleverse{25} Als sie
ihn dann auf der anderen Seite des Sees angetroffen hatten, fragten sie
ihn: »Rabbi\textless sup title=``oder: Meister''\textgreater✲, wann bist
du hierher gekommen?« \bibleverse{26} Jesus antwortete ihnen: »Wahrlich,
wahrlich ich sage euch: Ihr sucht mich nicht deshalb, weil ihr
Wunderzeichen gesehen, sondern weil ihr von den Broten gegessen habt und
satt geworden seid. \bibleverse{27} Verschafft euch doch nicht die
Speise, die vergänglich ist, sondern die Speise, die für
das\textless sup title=``oder: bis ins''\textgreater✲ ewige Leben
vorhält und die der Menschensohn euch geben wird; denn diesen hat Gott
der Vater besiegelt\textless sup title=``d.h.
beglaubigt''\textgreater✲.« \bibleverse{28} Da entgegneten sie ihm: »Was
sollen wir denn tun, um die Werke Gottes zu wirken?« \bibleverse{29}
Jesus antwortete ihnen mit den Worten: »Das Werk Gottes besteht darin,
daß ihr an den glaubt, den er gesandt hat.« \bibleverse{30} Da fragten
sie ihn: »Welches Zeichen tust du nun, damit wir es sehen und zum
Glauben an dich kommen? Womit kannst du dich ausweisen? \bibleverse{31}
Unsere Väter haben das Manna in der Wüste zu essen bekommen, wie
geschrieben steht\textless sup title=``2.Mose 16,4.14; Ps
78,24''\textgreater✲: ›Brot aus dem Himmel gab er ihnen zu essen.‹«

\hypertarget{d-jesu-rede-uxfcber-das-brot-des-lebens}{%
\paragraph{d) Jesu Rede über das Brot des
Lebens}\label{d-jesu-rede-uxfcber-das-brot-des-lebens}}

\hypertarget{aa-jesus-ist-das-wahre-himmelsbrot-und-gibt-es-den-gluxe4ubig-zu-ihm-kommenden-als-speise-fuxfcr-die-kuxfcnftige-auferstehung}{%
\subparagraph{aa) Jesus ist das wahre Himmelsbrot und gibt es den
gläubig zu ihm Kommenden als Speise für die künftige
Auferstehung}\label{aa-jesus-ist-das-wahre-himmelsbrot-und-gibt-es-den-gluxe4ubig-zu-ihm-kommenden-als-speise-fuxfcr-die-kuxfcnftige-auferstehung}}

\bibleverse{32} Da sagte Jesus zu ihnen: »Wahrlich, wahrlich ich sage
euch: Nicht Mose hat euch das Himmelsbrot gegeben, sondern mein Vater
gibt euch das wahre Himmelsbrot; \bibleverse{33} denn das Brot Gottes
ist das, welches\textless sup title=``oder: der, welcher''\textgreater✲
aus dem Himmel herabkommt und der Welt Leben gibt.« \bibleverse{34} Da
riefen sie ihm zu: »Herr, gib uns dieses Brot allezeit!« \bibleverse{35}
Da sagte Jesus zu ihnen: »Ich bin das Brot des Lebens! Wer zu mir kommt,
den wird nimmermehr hungern, und wer an mich glaubt, den wird niemals
wieder dürsten. \bibleverse{36} Aber ich habe euch (schon) gesagt: Ihr
habt mich wohl gesehen, glaubt aber doch nicht. \bibleverse{37} Alles,
was der Vater mir gibt, wird zu mir kommen, und wer zu mir kommt, den
werde ich nimmer hinausstoßen\textless sup title=``oder: von mir
stoßen''\textgreater✲; \bibleverse{38} denn ich bin aus dem Himmel
herabgekommen, nicht um meinen Willen auszuführen, sondern den Willen
dessen, der mich gesandt hat. \bibleverse{39} Das aber ist der Wille
dessen, der mich gesandt hat, daß ich von allem dem, was er mir gegeben
hat, nichts verloren gehen lasse, sondern es am jüngsten Tage
auferwecke. \bibleverse{40} Denn das ist der Wille meines Vaters, daß
jeder, der den Sohn sieht und an ihn glaubt, ewiges Leben habe, und ich
werde ihn am jüngsten Tage auferwecken.«

\hypertarget{bb-das-brot-des-lebens-wird-durch-den-glauben-und-der-glaube-durch-gottes-einwirkung-ziehen-zum-sohne-gewonnen}{%
\subparagraph{bb) Das Brot des Lebens wird durch den Glauben und der
Glaube durch Gottes Einwirkung (»Ziehen zum Sohne«)
gewonnen}\label{bb-das-brot-des-lebens-wird-durch-den-glauben-und-der-glaube-durch-gottes-einwirkung-ziehen-zum-sohne-gewonnen}}

\bibleverse{41} Da murrten die Juden über ihn, weil er gesagt hatte:
»Ich bin das Brot, das aus dem Himmel herabgekommen ist«,
\bibleverse{42} und sie sagten: »Ist dieser nicht Jesus, Josephs Sohn,
dessen Vater und Mutter wir kennen? Wie kann er da jetzt behaupten: ›Ich
bin aus dem Himmel herabgekommen?‹« \bibleverse{43} Jesus antwortete
ihnen mit den Worten: »Murret nicht untereinander! \bibleverse{44}
Niemand kann zu mir kommen, wenn nicht der Vater, der mich gesandt hat,
ihn zieht, und ich werde ihn dann am jüngsten Tage auferwecken.
\bibleverse{45} Es steht ja bei den Propheten geschrieben\textless sup
title=``Jes 54,13''\textgreater✲: ›Sie werden alle von Gott
gelehrt\textless sup title=``oder: unterwiesen''\textgreater✲ sein.‹
Jeder, der (es) vom Vater gehört und gelernt hat, kommt zu mir.
\bibleverse{46} Nicht als ob jemand den Vater gesehen hätte; denn nur
der eine, der von Gott her (gekommen) ist, nur der hat den Vater
gesehen. \bibleverse{47} Wahrlich, wahrlich ich sage euch: Wer da
glaubt, hat ewiges Leben! \bibleverse{48} Ich bin das Brot des Lebens.
\bibleverse{49} Eure Väter haben in der Wüste das Manna gegessen und
sind dann doch gestorben; \bibleverse{50} hier dagegen ist das Brot, das
aus dem Himmel herabkommt, damit man davon esse und nicht sterbe.
\bibleverse{51} Ich bin das lebendige Brot, das aus dem Himmel
herabgekommen ist: wenn jemand von diesem Brote ißt, so wird er ewiglich
leben; und zwar ist das Brot, das ich (zu essen) geben werde, mein
Fleisch, (das ich geben werde) für das Leben der Welt.«

\hypertarget{cc-die-harte-rede-jesu-vom-essen-und-trinken-seines-fleisches-und-blutes-dessen-genuuxdf-zur-auferstehung-fuxfchrt}{%
\subparagraph{cc) Die »harte« Rede Jesu vom Essen und Trinken seines
Fleisches und Blutes, dessen Genuß zur Auferstehung
führt}\label{cc-die-harte-rede-jesu-vom-essen-und-trinken-seines-fleisches-und-blutes-dessen-genuuxdf-zur-auferstehung-fuxfchrt}}

\bibleverse{52} Nun gerieten die Juden in Streit untereinander und
sagten: »Wie kann dieser uns sein Fleisch zu essen geben?«
\bibleverse{53} Da sagte Jesus zu ihnen: »Wahrlich, wahrlich ich sage
euch: Wenn ihr nicht das Fleisch des Menschensohnes eßt und sein Blut
trinkt, so habt ihr kein Leben in euch; \bibleverse{54} wer (dagegen)
mein Fleisch ißt und mein Blut trinkt, der hat ewiges Leben, und ich
werde ihn am jüngsten Tage auferwecken. \bibleverse{55} Denn mein
Fleisch ist wahre Speise, und mein Blut ist wahrer Trank.
\bibleverse{56} Wer mein Fleisch ißt und mein Blut trinkt, bleibt in mir
und ich in ihm. \bibleverse{57} Wie mich mein Vater, der das Leben in
sich trägt, gesandt hat und ich Leben in mir trage um des Vaters willen,
so wird auch der, welcher mich ißt, das Leben haben um meinetwillen.
\bibleverse{58} Von solcher Beschaffenheit ist das Brot, das aus dem
Himmel herabgekommen ist; es ist nicht von der Art, wie die Väter es
gegessen haben und gestorben sind; nein, wer dieses Brot ißt, wird leben
in Ewigkeit.«

\bibleverse{59} So sprach Jesus, als er in der Synagoge zu Kapernaum
lehrte.

\hypertarget{e-scheidung-der-juxfcnger-anhuxe4nger-jesu-als-wirkung-der-rede-das-petrusbekenntnis-bezuxfcglich-des-glaubens-der-zwuxf6lf-hinweis-auf-den-verrat-des-judas}{%
\paragraph{e) Scheidung der Jünger (=~Anhänger) Jesu als Wirkung der
Rede; das Petrusbekenntnis bezüglich des Glaubens der Zwölf; Hinweis auf
den Verrat des
Judas}\label{e-scheidung-der-juxfcnger-anhuxe4nger-jesu-als-wirkung-der-rede-das-petrusbekenntnis-bezuxfcglich-des-glaubens-der-zwuxf6lf-hinweis-auf-den-verrat-des-judas}}

\bibleverse{60} Viele nun von seinen Jüngern✲, die ihm zugehört hatten,
erklärten: »Das ist eine harte\textless sup title=``=~unannehmbare,
anstößige''\textgreater✲ Rede: wer kann sie anhören?« \bibleverse{61}
Weil aber Jesus bei sich\textless sup title=``oder: von
selbst''\textgreater✲ wußte, daß seine Jünger darüber murrten, sagte er
zu ihnen: »Das ist euch anstößig? \bibleverse{62} Wie nun (wird es
sein), wenn ihr den Menschensohn dahin auffahren seht, wo er vordem war?
\bibleverse{63} Der Geist ist es, der das Leben schafft, das Fleisch
hilft nichts; die Worte, die ich zu euch geredet habe, sind Geist und
sind Leben; \bibleverse{64} aber es sind unter euch auch solche, die
nicht glauben.« Jesus wußte nämlich von Anfang an, wer die waren, welche
ungläubig blieben, und wer der war, der ihn verraten würde.
\bibleverse{65} Er fuhr dann fort: »Aus diesem Grunde habe ich euch
gesagt: ›Niemand kann zu mir kommen, wenn es ihm nicht vom Vater
verliehen ist.‹«

\bibleverse{66} Von da an\textless sup title=``oder: aus diesem
Grunde''\textgreater✲ zogen sich viele seiner Jünger von ihm zurück und
begleiteten ihn nicht mehr auf seinen Wanderungen. \bibleverse{67} Daher
sagte Jesus zu den Zwölfen: »Ihr wollt doch nicht auch weggehen?«
\bibleverse{68} Simon Petrus antwortete ihm: »Herr, zu wem sollten wir
gehen? Du hast Worte ewigen Lebens; \bibleverse{69} und wir haben den
Glauben und die Erkenntnis gewonnen, daß du der Heilige Gottes bist.«
\bibleverse{70} Jesus antwortete ihnen: »Habe nicht ich selbst euch
Zwölf erwählt? Und einer von euch ist ein Teufel.« \bibleverse{71} Er
meinte damit aber den Judas, den Sohn Simons aus Kariot; denn dieser
sollte ihn verraten, (und war doch) einer von den Zwölfen.

\hypertarget{jesus-zum-drittenmal-in-jerusalem-auf-dem-laubhuxfcttenfest-und-darnach}{%
\subsubsection{8. Jesus (zum drittenmal) in Jerusalem auf dem
Laubhüttenfest und
darnach}\label{jesus-zum-drittenmal-in-jerusalem-auf-dem-laubhuxfcttenfest-und-darnach}}

\hypertarget{a-jesus-und-seine-bruxfcder-seine-reise-nach-jerusalem-zum-fest}{%
\paragraph{a) Jesus und seine Brüder; seine Reise nach Jerusalem zum
Fest}\label{a-jesus-und-seine-bruxfcder-seine-reise-nach-jerusalem-zum-fest}}

\hypertarget{section-6}{%
\section{7}\label{section-6}}

\bibleverse{1} Hierauf zog Jesus in Galiläa umher; denn in Judäa wollte
er nicht umherziehen, weil die Juden ihm nach dem Leben trachteten;
\bibleverse{2} es stand aber das jüdische Laubhüttenfest nahe bevor.
\bibleverse{3} Darum sagten seine Brüder zu ihm: »Mache dich von hier
auf den Weg und begib dich nach Judäa, damit deine Jünger✲ auch dort die
Werke sehen, die du tust; \bibleverse{4} denn niemand wirkt doch in der
Verborgenheit, wenn er sich in der Öffentlichkeit geltend machen will.
Willst du überhaupt solche Tätigkeit ausüben, so zeige dich der Welt
öffentlich«~-- \bibleverse{5} nicht einmal seine Brüder nämlich glaubten
an ihn. \bibleverse{6} Da antwortete Jesus ihnen: »Meine Zeit ist noch
nicht da; für euch freilich ist die Zeit immer gelegen. \bibleverse{7}
Euch kann die Welt nicht hassen, mich aber haßt sie, weil ich von ihr
bezeuge, daß ihr ganzes Tun böse ist. \bibleverse{8} Geht ihr nur zum
Fest hinauf, ich gehe zu diesem Fest nicht hinauf, weil meine Zeit noch
nicht erfüllt ist.« \bibleverse{9} So sprach er zu ihnen und blieb in
Galiläa. \bibleverse{10} Als dann aber seine Brüder zum Fest
hinaufgegangen waren, da ging auch er hinauf, jedoch nicht öffentlich,
sondern ganz in der Stille. \bibleverse{11} Die Juden suchten nun
während des Festes nach ihm und fragten: »Wo ist er?« \bibleverse{12}
Und unter den Volksscharen war viel Gerede über ihn; die einen sagten:
»Er ist ein guter Mann«; andere dagegen behaupteten: »Nein, er ist ein
Volksverführer«; \bibleverse{13} doch niemand redete mit voller
Offenheit über ihn aus Furcht vor den Juden.

\hypertarget{b-jesu-auftreten-und-selbstzeugnis-auf-dem-laubhuxfcttenfest-streitgespruxe4ch-uxfcber-die-herkunft-des-messias}{%
\paragraph{b) Jesu Auftreten und Selbstzeugnis auf dem Laubhüttenfest;
Streitgespräch über die Herkunft des
Messias}\label{b-jesu-auftreten-und-selbstzeugnis-auf-dem-laubhuxfcttenfest-streitgespruxe4ch-uxfcber-die-herkunft-des-messias}}

\hypertarget{aa-jesu-lehre-stammt-von-gott}{%
\subparagraph{aa) Jesu Lehre stammt von
Gott}\label{aa-jesu-lehre-stammt-von-gott}}

\bibleverse{14} Als aber die Festwoche schon zur Hälfte vorüber war,
ging Jesus zum Tempel hinauf und lehrte. \bibleverse{15} Da wunderten
sich die Juden und sagten: »Wie kommt dieser zur Schriftgelehrsamkeit,
obwohl er doch keinen Unterricht in ihr erhalten hat\textless sup
title=``=~nicht studiert hat''\textgreater✲?« \bibleverse{16} Da
antwortete ihnen Jesus mit den Worten: »Meine Lehre stammt nicht von
mir, sondern von dem, der mich gesandt hat; \bibleverse{17} wenn jemand
dessen Willen tun will, wird er inne werden, ob diese Lehre von Gott
stammt oder ob ich von mir selbst aus rede. \bibleverse{18} Wer von sich
selbst aus redet, sucht seine eigene Ehre; wer aber die Ehre dessen
sucht, der ihn gesandt hat, der ist wahrhaftig, und bei dem findet sich
keine Ungerechtigkeit\textless sup title=``=~verwerfliche
Selbstsucht''\textgreater✲. \bibleverse{19} Hat nicht Mose euch das
Gesetz gegeben? Und doch erfüllt niemand von euch das Gesetz! Warum
sucht ihr mich zu töten?« \bibleverse{20} Die Volksmenge antwortete: »Du
bist von Sinnen! Wer sucht dich denn zu töten?« \bibleverse{21} Jesus
antwortete ihnen: »Ein einziges Werk habe ich (hier in Jerusalem) getan,
und ihr seid allesamt verwundert darüber. \bibleverse{22} Mose hat euch
die Beschneidung gegeben -- von Mose stammt sie freilich nicht, sondern
von den Erzvätern --, und so beschneidet ihr denn einen Menschen (auch)
am Sabbat. \bibleverse{23} Wenn (nun) ein Mensch am Sabbat die
Beschneidung empfängt, damit das mosaische Gesetz nicht gebrochen wird:
da wollt ihr mir zürnen, weil ich einen ganzen Menschen am Sabbat gesund
gemacht habe? \bibleverse{24} Urteilt nicht nach dem äußeren Schein,
sondern gebt ein gerechtes Urteil ab!«

\hypertarget{bb-jesus-selbst-kommt-von-gott}{%
\subparagraph{bb) Jesus selbst kommt von
Gott}\label{bb-jesus-selbst-kommt-von-gott}}

\bibleverse{25} Da sagten einige von den Bewohnern Jerusalems: »Ist
dieser Mensch es nicht, den sie zu töten suchen? \bibleverse{26} Und
seht nur: er redet ganz öffentlich, und man sagt ihm kein Wort! Die
Oberen\textless sup title=``=~Mitglieder des Hohen Rates''\textgreater✲
werden doch nicht etwa zu der Erkenntnis gekommen sein, daß dieser der
Messias ist? \bibleverse{27} Freilich von diesem wissen wir, woher er
stammt; wenn aber der Messias kommt, weiß niemand, woher er stammt.«
\bibleverse{28} Da rief Jesus im Tempel, wo er lehrte, laut aus: »Ja,
ihr kennt mich und wißt, woher ich stamme! Und doch bin ich nicht von
mir selbst aus gekommen, sondern es ist der rechte Sender, der mich
gesandt hat, den ihr aber nicht kennt. \bibleverse{29} Ich kenne ihn,
weil ich von ihm her (ausgegangen) bin, und er hat mich gesandt.«
\bibleverse{30} Da suchten sie ihn festzunehmen, doch niemand legte Hand
an ihn, weil seine Stunde noch nicht gekommen war.

\hypertarget{c-der-verhaftungsplan-der-feinde-und-dessen-fehlschlagen-jesus-als-spender-des-geistes}{%
\paragraph{c) Der Verhaftungsplan der Feinde und dessen Fehlschlagen;
Jesus als Spender des
Geistes}\label{c-der-verhaftungsplan-der-feinde-und-dessen-fehlschlagen-jesus-als-spender-des-geistes}}

\bibleverse{31} Aus dem Volke kamen aber viele zum Glauben an ihn und
sagten: »Wird wohl Christus\textless sup title=``oder: der
Messias''\textgreater✲, wenn er kommt, mehr Wunderzeichen tun, als
dieser getan hat?« \bibleverse{32} Die Pharisäer erfuhren, daß das Volk
solche Ansichten im geheimen über ihn äußerte; daher schickten die
Hohenpriester und die Pharisäer Diener ab, die ihn festnehmen sollten.

\hypertarget{aa-jesus-kuxfcndigt-seinen-hingang-zu-gott-an}{%
\subparagraph{aa) Jesus kündigt seinen Hingang zu Gott
an}\label{aa-jesus-kuxfcndigt-seinen-hingang-zu-gott-an}}

\bibleverse{33} Da sagte Jesus: »Nur noch kurze Zeit bin ich bei euch,
dann gehe ich hin zu dem, der mich gesandt hat. \bibleverse{34} Ihr
werdet mich (dann) suchen, aber nicht finden, und wo ich (dann) bin,
dahin könnt ihr nicht kommen.« \bibleverse{35} Da sagten die Juden
zueinander: »Wohin will dieser gehen, daß wir ihn nicht finden können?
Will er etwa zu den Juden gehen, die unter den Griechen zerstreut leben,
und der Lehrer der Griechen werden? \bibleverse{36} Welchen Sinn hat
dieses Wort, das er ausgesprochen hat: ›Ihr werdet mich suchen, aber
nicht finden‹ und ›Wo ich (dann) bin, dahin könnt ihr nicht kommen‹?«

\hypertarget{bb-jesus-auf-dem-huxf6hepunkt-des-festes-als-spender-des-wassers-des-lebens-d.h.-des-geistes}{%
\subparagraph{bb) Jesus auf dem Höhepunkt des Festes als Spender des
Wassers des Lebens, d.h. des
Geistes}\label{bb-jesus-auf-dem-huxf6hepunkt-des-festes-als-spender-des-wassers-des-lebens-d.h.-des-geistes}}

\bibleverse{37} Am letzten, dem großen Tage✲ des Festes aber stand Jesus
da und rief laut aus: »Wen da dürstet, der komme zu mir und trinke!
\bibleverse{38} Wer an mich glaubt, aus dessen Leibe werden, wie die
Schrift gesagt hat\textless sup title=``Joel 4,18; Sach 14,8; Hes
47,1-12''\textgreater✲, Ströme lebendigen Wassers fließen.«
\bibleverse{39} Damit meinte er aber den Geist, den die, welche zum
Glauben an ihn gekommen waren, empfangen sollten; denn der (heilige)
Geist war noch nicht da, weil Jesus noch nicht zur Herrlichkeit erhoben
worden war.

\hypertarget{cc-entgegengesetzte-urteile-des-volkes-uxfcber-jesus}{%
\subparagraph{cc) Entgegengesetzte Urteile des Volkes über
Jesus}\label{cc-entgegengesetzte-urteile-des-volkes-uxfcber-jesus}}

\bibleverse{40} Nun sagten manche aus dem Volk, die diese Worte gehört
hatten: »Dieser ist wirklich der Prophet!« \bibleverse{41} Andere
sagten: »Er ist Christus\textless sup title=``=~der
Messias''\textgreater✲«; wieder andere meinten: »Christus kommt doch
nicht aus Galiläa! \bibleverse{42} Hat nicht die Schrift
gesagt\textless sup title=``2.Sam 7,12; Mi 5,1''\textgreater✲, daß
Christus aus dem Samen\textless sup title=``=~der
Nachkommenschaft''\textgreater✲ Davids und aus der Ortschaft Bethlehem,
wo David gewohnt hat, kommen soll?« \bibleverse{43} So entstand
seinetwegen eine Spaltung unter dem Volk. \bibleverse{44} Einige von
ihnen hätten ihn nun gern festgenommen, aber keiner legte Hand an ihn.

\hypertarget{dd-fehlschlagen-des-verhaftungsplans-der-fuxfchrer-spaltung-unter-den-mitgliedern-des-hohen-rates-mahnung-des-nikodemus}{%
\subparagraph{dd) Fehlschlagen des Verhaftungsplans der Führer; Spaltung
unter den Mitgliedern des Hohen Rates; Mahnung des
Nikodemus}\label{dd-fehlschlagen-des-verhaftungsplans-der-fuxfchrer-spaltung-unter-den-mitgliedern-des-hohen-rates-mahnung-des-nikodemus}}

\bibleverse{45} So kamen denn die Diener zu den Hohenpriestern und
Pharisäern zurück, und diese fragten sie: »Warum habt ihr ihn nicht
hergebracht?« \bibleverse{46} Die Diener antworteten: »Noch niemals hat
ein Mensch so geredet, wie dieser Mann redet!« \bibleverse{47} Da
erwiderten ihnen die Pharisäer: »Habt auch ihr euch irreführen lassen?
\bibleverse{48} Ist etwa irgendein Oberer\textless sup
title=``=~Mitglied des Hohen Rates''\textgreater✲ oder ein Pharisäer zum
Glauben an ihn gekommen? \bibleverse{49} Nein, nur dieses gemeine Volk,
das vom Gesetz nichts weiß -- verflucht sind sie!« \bibleverse{50} Da
sagte Nikodemus, der früher einmal zu Jesus gekommen war und ihrer
Partei angehörte: \bibleverse{51} »Verurteilt etwa unser Gesetz einen
Menschen, ohne daß man ihn zuvor verhört und seine Schuld festgestellt
hat?« \bibleverse{52} Da gaben sie ihm zur Antwort: »Stammst du
vielleicht auch aus Galiläa? Forsche doch nach und lerne begreifen, daß
aus Galiläa kein Prophet hervorgeht!«

\hypertarget{d-jesus-und-die-ehebrecherin}{%
\paragraph{d) Jesus und die
Ehebrecherin}\label{d-jesus-und-die-ehebrecherin}}

\bibleverse{53} Dann gingen sie weg, ein jeder in sein Haus;

\hypertarget{section-7}{%
\section{8}\label{section-7}}

\bibleverse{1} Jesus aber begab sich an den Ölberg. \bibleverse{2} Am
folgenden Morgen jedoch fand er sich wieder im Tempel ein, und das
gesamte Volk kam zu ihm; er setzte sich dann und lehrte sie.
\bibleverse{3} Da führten die Schriftgelehrten und Pharisäer eine Frau
herbei, die beim Ehebruch ergriffen\textless sup title=``oder:
ertappt''\textgreater✲ worden war, stellten sie in die Mitte
\bibleverse{4} und sagten zu ihm: »Meister, diese Frau ist auf frischer
Tat beim Ehebruch ergriffen\textless sup title=``oder:
ertappt''\textgreater✲ worden. \bibleverse{5} Nun hat Mose uns im Gesetz
geboten, solche Frauen zu steinigen\textless sup title=``3.Mose 20,10;
5.Mose 22,22''\textgreater✲. Was sagst nun du dazu?« \bibleverse{6} Dies
sagten sie aber, um ihn zu versuchen, damit sie einen Grund zur Anklage
gegen ihn hätten. Jesus aber bückte sich nieder und schrieb mit dem
Finger auf den Erdboden. \bibleverse{7} Als sie aber ihre Frage an ihn
mehrfach wiederholten, richtete er sich auf und sagte zu ihnen: »Wer
unter euch ohne Sünde ist, werfe den ersten Stein auf sie!«
\bibleverse{8} Hierauf bückte er sich aufs neue und schrieb auf dem
Erdboden weiter. \bibleverse{9} Als aber jene das gehört hatten, gingen
sie einer nach dem andern weg, die Ältesten zuerst bis zu den Letzten,
und Jesus blieb allein zurück mit der Frau, die in der
Mitte\textless sup title=``=~vor ihm''\textgreater✲ stand.
\bibleverse{10} Da richtete Jesus sich auf und fragte sie: »Frau, wo
sind sie\textless sup title=``d.h. deine Ankläger''\textgreater✲? Hat
keiner dich verurteilt?« Sie antwortete: »Keiner, Herr.« \bibleverse{11}
Da sagte Jesus: »Auch ich verurteile dich nicht: gehe hin und sündige
hinfort nicht mehr!«

\hypertarget{e-fortsetzung-und-huxf6hepunkt-des-kampfes-jesu-angriff-auf-das-ungluxe4ubige-judentum}{%
\paragraph{e) Fortsetzung und Höhepunkt des Kampfes; Jesu Angriff auf
das ungläubige
Judentum}\label{e-fortsetzung-und-huxf6hepunkt-des-kampfes-jesu-angriff-auf-das-ungluxe4ubige-judentum}}

\hypertarget{aa-selbstzeugnis-jesu-als-des-lichts-der-welt-und-des-sohnes-gottes}{%
\subparagraph{aa) Selbstzeugnis Jesu als des Lichts der Welt und des
Sohnes
Gottes}\label{aa-selbstzeugnis-jesu-als-des-lichts-der-welt-und-des-sohnes-gottes}}

\bibleverse{12} Nun redete Jesus aufs neue zu ihnen und sagte: »Ich bin
das Licht der Welt: wer mir nachfolgt, wird nicht in der Finsternis
wandeln, sondern das Licht des Lebens haben.« \bibleverse{13} Da sagten
die Pharisäer zu ihm: »Du legst Zeugnis über dich\textless sup
title=``oder: für dich''\textgreater✲ selbst ab: dein Zeugnis ist
ungültig.«\textless sup title=``vgl. 5,31''\textgreater✲ \bibleverse{14}
Jesus gab ihnen zur Antwort: »Auch wenn ich über mich\textless sup
title=``oder: für mich''\textgreater✲ selbst Zeugnis ablege, so ist mein
Zeugnis doch gültig, denn ich weiß, woher ich gekommen bin und wohin ich
gehe; ihr aber wißt nicht, woher ich komme und wohin ich gehe.
\bibleverse{15} Ihr richtet nach dem Fleisch, ich richte überhaupt
niemand. \bibleverse{16} Doch auch wenn ich richte, ist mein Urteil
wahr✲; denn ich stehe (mit meinem Zeugnis) nicht allein, sondern mit mir
ist der, welcher mich gesandt hat. \bibleverse{17} Nun steht doch auch
in eurem Gesetz geschrieben, daß das Zeugnis zweier Personen wahr✲
ist\textless sup title=``5.Mose 17,6; 19,15''\textgreater✲.
\bibleverse{18} Ich lege Zeugnis von mir\textless sup title=``oder: für
mich''\textgreater✲ ab, und der Vater, der mich gesandt hat, legt auch
Zeugnis von mir\textless sup title=``oder: für mich''\textgreater✲ ab.«
\bibleverse{19} Da fragten sie ihn: »Wo ist (denn) dein Vater?« Jesus
antwortete: »Weder mich noch meinen Vater kennt ihr; wenn ihr mich
kenntet, würdet ihr auch meinen Vater kennen.« \bibleverse{20} Diese
Worte sprach er aus, als er beim Opferkasten im Tempel lehrte; und
niemand legte Hand an ihn, weil seine Stunde noch nicht gekommen war.

\hypertarget{bb-jesus-bezeugt-die-tiefe-kluft-die-ihn-nach-seiner-herkunft-von-den-juden-trennt}{%
\subparagraph{bb) Jesus bezeugt die tiefe Kluft, die ihn nach seiner
Herkunft von den Juden
trennt}\label{bb-jesus-bezeugt-die-tiefe-kluft-die-ihn-nach-seiner-herkunft-von-den-juden-trennt}}

\bibleverse{21} Aufs neue sagte er dann zu ihnen: »Ich gehe weg; dann
werdet ihr mich suchen und in eurer Sünde sterben. Wohin ich gehe, dahin
könnt ihr nicht kommen.« \bibleverse{22} Da sagten die Juden: »Will er
sich etwa das Leben nehmen, weil er sagt: ›Wohin ich gehe, dahin könnt
ihr nicht kommen‹?« \bibleverse{23} Da sagte er zu ihnen: »Ihr seid von
unten her, ich bin von oben her; ihr seid aus dieser Welt, ich bin nicht
aus dieser Welt. \bibleverse{24} Darum habe ich euch gesagt, daß ihr in
euren Sünden sterben werdet; denn wenn ihr nicht glaubt, daß ich es
bin\textless sup title=``d.h. der Messias bin''\textgreater✲, so werdet
ihr in euren Sünden sterben.« \bibleverse{25} Da fragten sie ihn: »Wer
bist du denn?« Jesus antwortete ihnen: »Das, was ich von Anfang an
(gesagt habe) und auch jetzt euch sage. \bibleverse{26} Vieles hätte ich
über euch noch zu sagen und zu richten; aber der mich gesandt hat, ist
wahrhaftig, und ich -- was ich von ihm gehört habe, das rede ich zur
Welt.« \bibleverse{27} Sie verstanden nicht, daß er vom Vater zu ihnen
redete. \bibleverse{28} Da fuhr nun Jesus fort: »Wenn ihr den
Menschensohn erhöht haben werdet, dann werdet ihr erkennen, daß ich es
bin✲ und daß ich nichts von mir selbst aus tue, sondern so rede, wie der
Vater mich gelehrt hat. \bibleverse{29} Und der mich gesandt hat, ist
mit\textless sup title=``oder: bei''\textgreater✲ mir; er hat mich nicht
allein gelassen, weil ich allezeit das tue, was ihm wohlgefällig ist.«

\hypertarget{cc-jesu-zeugnis-von-seiner-gottessohnschaft-und-von-der-suxfcndenknechtschaft-der-juden-trotz-ihrer-abstammung-von-abraham}{%
\subparagraph{cc) Jesu Zeugnis von seiner Gottessohnschaft und von der
Sündenknechtschaft der Juden trotz ihrer Abstammung von
Abraham}\label{cc-jesu-zeugnis-von-seiner-gottessohnschaft-und-von-der-suxfcndenknechtschaft-der-juden-trotz-ihrer-abstammung-von-abraham}}

\bibleverse{30} Als er das sagte, kamen viele zum Glauben an ihn.
\bibleverse{31} Nun sagte Jesus zu den Juden, die an ihn gläubig
geworden waren: »Wenn ihr in meinem Wort bleibt\textless sup
title=``=~Hörer und Täter meines Wortes bleibt''\textgreater✲, so seid
ihr in Wahrheit meine Jünger \bibleverse{32} und werdet die Wahrheit
erkennen, und die Wahrheit wird euch frei machen.« \bibleverse{33} Da
entgegneten sie ihm: »Wir sind Abrahams Nachkommenschaft und haben noch
niemals jemandem als Knechte gedient; wie kannst du da sagen: ›Ihr
werdet frei werden‹?« \bibleverse{34} Jesus antwortete ihnen: »Wahrlich,
wahrlich ich sage euch: ein jeder, der Sünde tut, ist ein Knecht der
Sünde. \bibleverse{35} Der Knecht aber bleibt nicht für immer im Hause,
der Sohn dagegen bleibt für immer darin. \bibleverse{36} Wenn also der
Sohn euch frei gemacht hat, dann werdet ihr wirklich frei sein.«

\hypertarget{dd-die-ungluxe4ubigen-juden-sind-weder-abrahams-noch-gottes-kinder-sondern-kinder-des-teufels}{%
\subparagraph{dd) Die ungläubigen Juden sind weder Abrahams noch Gottes
Kinder, sondern Kinder des
Teufels}\label{dd-die-ungluxe4ubigen-juden-sind-weder-abrahams-noch-gottes-kinder-sondern-kinder-des-teufels}}

\bibleverse{37} »Ich weiß wohl, daß ihr Abrahams Nachkommenschaft seid;
aber ihr sucht mich zu töten, weil mein Wort keinen Eingang bei euch
findet. \bibleverse{38} Was ich beim\textless sup title=``d.h. bei
meinem''\textgreater✲ Vater gesehen habe, das rede ich; dementsprechend
tut auch ihr das, was ihr vom\textless sup title=``d.h. von
eurem''\textgreater✲ Vater gehört habt.« \bibleverse{39} Sie antworteten
ihm mit der Versicherung: »Unser Vater ist Abraham!« Jesus erwiderte
ihnen: »Wenn ihr Abrahams Kinder seid, so handelt auch so wie Abraham
(gehandelt hat)! \bibleverse{40} Nun aber geht ihr darauf aus, mich zu
töten, einen Mann, der euch die Wahrheit verkündigt hat, wie ich sie von
Gott gehört habe: so etwas hat Abraham nicht getan. \bibleverse{41} Ihr
vollbringt die Werke eures Vaters.« Sie erwiderten ihm: »Wir sind keine
unehelichen Kinder; wir haben nur einen einzigen Vater, nämlich Gott.«
\bibleverse{42} Da sagte Jesus zu ihnen: »Wenn Gott euer Vater wäre,
dann würdet ihr mich lieben; denn ich bin von Gott ausgegangen und (von
ihm) gekommen; ich bin nicht von mir selbst gekommen, sondern er hat
mich gesandt. \bibleverse{43} Wie geht es nun zu, daß ihr meine Art zu
reden nicht versteht? Weil ihr nicht imstande seid, das, was meine Worte
besagen, auch nur anzuhören. \bibleverse{44} Ihr stammt eben vom Teufel
als eurem Vater und wollt nach den Gelüsten eures Vaters handeln. Der
ist ein Menschenmörder von Anfang an gewesen und steht nicht in der
Wahrheit, weil die Wahrheit nicht in ihm ist. Wenn er die Lüge redet,
dann redet er aus seinem eigensten Wesen heraus, denn er ist ein Lügner
und der Vater von ihr\textless sup title=``d.h. von der
Lüge''\textgreater✲. \bibleverse{45} Weil ich dagegen die Wahrheit rede,
schenkt ihr mir keinen Glauben. \bibleverse{46} Wer von euch kann mich
einer Sünde zeihen\textless sup title=``oder: überführen''\textgreater✲?
Wenn ich die Wahrheit rede, warum schenkt ihr mir keinen Glauben?
\bibleverse{47} Wer aus Gott ist\textless sup title=``oder: von Gott
stammt''\textgreater✲, hört die Worte Gottes; deshalb hört ihr sie
nicht, weil ihr nicht von Gott seid.«

\hypertarget{ee-jesu-zeugnis-von-der-erhabenheit-bzw.-ewigkeit-seiner-person-und-von-seiner-uxfcberlegenheit-uxfcber-abraham}{%
\subparagraph{ee) Jesu Zeugnis von der Erhabenheit (bzw. Ewigkeit)
seiner Person und von seiner Überlegenheit über
Abraham}\label{ee-jesu-zeugnis-von-der-erhabenheit-bzw.-ewigkeit-seiner-person-und-von-seiner-uxfcberlegenheit-uxfcber-abraham}}

\bibleverse{48} Da gaben ihm die Juden zur Antwort: »Sagen wir nicht mit
Recht, daß du ein Samariter und von einem bösen Geist besessen bist?«
\bibleverse{49} Jesus antwortete ihnen: »Ich bin von keinem bösen Geist
besessen, sondern ehre meinen Vater; doch ihr beschimpft mich.
\bibleverse{50} Ich aber sorge nicht für meine Ehre: es ist einer da,
der (für sie) sorgt und Gericht (für sie) hält. \bibleverse{51}
Wahrlich, wahrlich ich sage euch: Wenn jemand mein Wort
bewahrt\textless sup title=``oder: hält''\textgreater✲, wird er den Tod
in Ewigkeit nicht sehen.« \bibleverse{52} Da entgegneten ihm die Juden:
»Jetzt wissen wir sicher, daß du von einem bösen Geist besessen bist.
Abraham ist gestorben und (ebenso) die Propheten, und du behauptest:
›Wenn jemand mein Wort bewahrt\textless sup title=``oder:
hält''\textgreater✲, wird er den Tod in Ewigkeit nicht schmecken.‹
\bibleverse{53} Bist du etwa größer als unser Vater Abraham, der doch
gestorben ist? Und auch die Propheten sind gestorben. Was machst du aus
dir selbst?« \bibleverse{54} Jesus antwortete: »Wenn ich mich selbst
ehrte, so wäre es mit meiner Ehre nichts; nein, mein Vater ist es, der
mich ehrt, derselbe, von dem ihr behauptet, er sei euer Gott;
\bibleverse{55} und dabei habt ihr ihn nicht erkannt. Ich aber kenne
ihn; und wenn ich sagen wollte, daß ich ihn nicht kenne, so würde ich
euch gleich sein, nämlich ein Lügner. Doch ich kenne ihn und
bewahre\textless sup title=``oder: halte''\textgreater✲ sein Wort.
\bibleverse{56} Euer Vater Abraham hat darüber gejubelt, daß er meinen
Tag\textless sup title=``=~den Tag meiner Geburt''\textgreater✲ sehen
sollte, und er hat ihn gesehen und sich darüber gefreut.«
\bibleverse{57} Da sagten die Juden zu ihm: »Du bist noch nicht fünfzig
Jahre alt und willst Abraham gesehen haben?« \bibleverse{58} Jesus
antwortete ihnen: »Wahrlich, wahrlich ich sage euch: Ehe Abraham
(geboren) ward, bin ich.« \bibleverse{59} Da hoben sie Steine auf, um
sie auf ihn zu werfen; Jesus aber verbarg sich und ging aus dem Tempel
hinaus.

\hypertarget{jesus-und-der-blindgeborene}{%
\subsubsection{9. Jesus und der
Blindgeborene}\label{jesus-und-der-blindgeborene}}

\hypertarget{a-heilung-des-blindgeborenen-am-sabbat}{%
\paragraph{a) Heilung des Blindgeborenen am
Sabbat}\label{a-heilung-des-blindgeborenen-am-sabbat}}

\hypertarget{section-8}{%
\section{9}\label{section-8}}

\bibleverse{1} Im Vorübergehen sah er alsdann einen Mann, der von Geburt
an blind war. \bibleverse{2} Da fragten ihn seine Jünger:
»Rabbi\textless sup title=``oder: Meister''\textgreater✲, wer hat
gesündigt, dieser Mann oder seine Eltern, daß er als Blinder geboren
worden ist?« \bibleverse{3} Jesus antwortete: »Weder dieser hat
gesündigt noch seine Eltern; sondern (dazu ist es geschehen) damit das
Wirken Gottes an ihm offenbar würde. \bibleverse{4} Wir müssen die Werke
dessen wirken, der mich gesandt hat, solange es Tag ist; es kommt die
Nacht, in der niemand wirken kann. \bibleverse{5} Solange ich in der
Welt bin, bin ich das Licht der Welt.« \bibleverse{6} Nach diesen Worten
spie er auf den Boden, stellte mit dem Speichel einen Teig\textless sup
title=``oder: Brei''\textgreater✲ her, legte dem Blinden den Teig auf
die Augen \bibleverse{7} und sagte zu ihm: »Gehe hin, wasche dich im
Teiche Siloah!« -- Das heißt übersetzt ›Abgesandter‹. -- Da ging er hin,
wusch sich und kam sehend zurück. \bibleverse{8} Nun sagten die Nachbarn
und die Leute, die ihn früher als Bettler gesehen hatten: »Ist dieser
nicht der Mann, der früher dasaß und bettelte?« \bibleverse{9} Die einen
sagten: »Ja, er ist's«; andere meinten: »Nein, er sieht ihm nur
ähnlich«; er selbst aber sagte: »Ja, ich bin's.« \bibleverse{10} Da
fragten sie ihn: »Auf welche Weise sind dir denn die Augen aufgetan
worden?« \bibleverse{11} Er antwortete: »Der Mann, der Jesus heißt,
stellte einen Teig her, strich ihn mir auf die Augen und sagte zu mir:
›Gehe hin an den Siloahteich und wasche dich dort!‹ Da ging ich hin,
wusch mich und konnte sehen.« \bibleverse{12} Sie fragten ihn nun: »Wo
ist der Mann?« Er antwortete: »Das weiß ich nicht.«

\hypertarget{b-die-mehrfachen-untersuchungen-der-heilung-und-das-unerschrockene-bekenntnis-des-blindgeborenen-fuxfchren-zu-seiner-ausstouxdfung-aus-der-judengemeinschaft-das-zeugnis-jesu}{%
\paragraph{b) Die mehrfachen Untersuchungen der Heilung und das
unerschrockene Bekenntnis des Blindgeborenen führen zu seiner Ausstoßung
aus der Judengemeinschaft; das Zeugnis
Jesu}\label{b-die-mehrfachen-untersuchungen-der-heilung-und-das-unerschrockene-bekenntnis-des-blindgeborenen-fuxfchren-zu-seiner-ausstouxdfung-aus-der-judengemeinschaft-das-zeugnis-jesu}}

\hypertarget{aa-das-erste-verhuxf6r-der-pharisuxe4er}{%
\subparagraph{aa) Das erste Verhör der
Pharisäer}\label{aa-das-erste-verhuxf6r-der-pharisuxe4er}}

\bibleverse{13} Man führte ihn nun zu den Pharisäern, ihn, den ehemals
Blinden. \bibleverse{14} Es war aber (gerade) Sabbat an dem Tage
gewesen, an dem Jesus den Teig hergestellt und ihm die Augen aufgetan
hatte. \bibleverse{15} Da fragten ihn nochmals auch die Pharisäer, wie
er sehend geworden sei, und er antwortete ihnen: »Er hat mir einen Teig
auf die Augen gelegt, ich habe mich dann gewaschen und kann nun sehen.«
\bibleverse{16} Da sagten einige von den Pharisäern: »Der betreffende
Mensch ist nicht von Gott her, weil er den Sabbat nicht hält«; andere
dagegen meinten: »Wie könnte ein sündiger Mensch derartige Wunderzeichen
tun?« So bestand eine Meinungsverschiedenheit unter ihnen.
\bibleverse{17} Sie fragten also den Blindgeborenen aufs neue: »Was
sagst du denn von ihm? Dir hat er doch die Augen aufgetan.« Jener
antwortete: »Er ist ein Prophet.«

\hypertarget{bb-das-verhuxf6r-der-eltern}{%
\subparagraph{bb) Das Verhör der
Eltern}\label{bb-das-verhuxf6r-der-eltern}}

\bibleverse{18} Die Juden wollten nun von ihm nicht glauben, daß er
blind gewesen und sehend geworden sei, bis sie schließlich die Eltern
des Sehendgewordenen riefen \bibleverse{19} und sie fragten: »Ist dies
euer Sohn, der, wie ihr behauptet, blind geboren worden ist? Wie kommt
es denn, daß er jetzt sehen kann?« \bibleverse{20} Da antworteten seine
Eltern: »Wir wissen, daß dies unser Sohn ist und daß er als Blinder
geboren worden ist; \bibleverse{21} wie es aber kommt, daß er jetzt
sehen kann, das wissen wir nicht, und wer ihm die Augen geöffnet hat,
wissen wir auch nicht. Befragt ihn selbst darüber: er ist alt genug; er
wird selbst Auskunft über sich geben.« \bibleverse{22} Das sagten seine
Eltern, weil sie sich vor den Juden fürchteten; denn die Juden hatten
bereits miteinander abgemacht, daß, wenn jemand Jesus als den Messias
anerkenne, er in den Bann getan werden solle. \bibleverse{23} Aus diesem
Grunde sagten seine Eltern: »Er ist alt genug: fragt ihn selbst!«

\hypertarget{cc-das-zweite-verhuxf6r-des-geheilten}{%
\subparagraph{cc) Das zweite Verhör des
Geheilten}\label{cc-das-zweite-verhuxf6r-des-geheilten}}

\bibleverse{24} So ließen sie denn den Mann, der blind gewesen war, zum
zweitenmal rufen und sagten zu ihm: »Gib Gott die Ehre! Wir wissen, daß
dieser Mensch ein Sünder ist.« \bibleverse{25} Da antwortete jener: »Ob
er ein Sünder ist, weiß ich nicht; eins aber weiß ich, daß ich blind
gewesen bin und jetzt sehen kann.« \bibleverse{26} Da fragten sie ihn
noch einmal: »Was hat er mit dir vorgenommen? Auf welche Weise hat er
dir die Augen aufgetan?« \bibleverse{27} Er antwortete ihnen: »Ich habe
es euch schon einmal gesagt, doch ihr habt nicht darauf gehört; warum
wollt ihr es noch einmal hören? Wollt etwa auch ihr seine Jünger
werden?« \bibleverse{28} Da schmähten sie ihn und sagten: »Du bist ein
Jünger von ihm, wir aber sind Jünger von Mose. \bibleverse{29} Wir
wissen, daß Gott zu Mose geredet hat; von diesem aber wissen wir nicht,
woher er stammt.« \bibleverse{30} Der Mann gab ihnen zur Antwort: »Darin
liegt eben das Verwunderliche, daß ihr nicht wißt, woher er stammt, und
mir hat er doch die Augen aufgetan. \bibleverse{31} Wir wissen, daß Gott
Sünder nicht erhört, sondern nur wenn jemand gottesfürchtig ist und
seinen Willen tut, den erhört er. \bibleverse{32} Von der Weltzeit
an\textless sup title=``=~solange die Welt steht''\textgreater✲ hat man
noch nicht vernommen, daß jemand einem Blindgeborenen die Augen aufgetan
hat. \bibleverse{33} Wenn dieser Mann nicht von Gott her wäre, so
vermöchte er nichts zu tun.« \bibleverse{34} Sie antworteten ihm: »Du
bist ganz in Sünden geboren, und du willst uns Lehren geben?« Und sie
stießen ihn (aus der Gemeinde der Gesetzesfrommen) aus.

\hypertarget{dd-der-glaube-des-geheilten-an-jesus-jesus-als-das-licht-der-nichtsehenden-und-als-die-verblendung-der-sehenden}{%
\subparagraph{dd) Der Glaube des Geheilten an Jesus; Jesus als das Licht
der Nichtsehenden und als die Verblendung der
Sehenden}\label{dd-der-glaube-des-geheilten-an-jesus-jesus-als-das-licht-der-nichtsehenden-und-als-die-verblendung-der-sehenden}}

\bibleverse{35} Jesus erfuhr von seiner Ausstoßung und sagte zu ihm, als
er ihn antraf: »Glaubst du an den Sohn Gottes?« \bibleverse{36} Jener
gab zur Antwort: »Herr, wer ist denn das? Ich möchte gern an ihn
glauben.« \bibleverse{37} Jesus antwortete ihm: »Du hast ihn gesehen,
und der mit dir redet, der ist es!« \bibleverse{38} Jener sagte: »Ich
glaube, Herr!« und warf sich vor ihm nieder. \bibleverse{39} Nun sagte
Jesus: »Zu einer Scheidung bin ich in diese Welt gekommen: die
Nichtsehenden sollen sehen können und die Sehenden blind werden.«
\bibleverse{40} Dies hörten einige von den Pharisäern, die sich in
seiner Nähe befanden, und fragten ihn: »Sind wir etwa auch blind?«
\bibleverse{41} Jesus antwortete ihnen: »Wäret ihr blind, so hättet ihr
keine Sünde; nun ihr aber behauptet: ›Wir sind sehend‹, so bleibt eure
Sünde!«

\hypertarget{die-bildrede-vom-hirten-und-dieb-und-vom-guten-hirten-und-mietling}{%
\subsubsection{10. Die Bildrede vom Hirten und Dieb und vom guten Hirten
und
Mietling}\label{die-bildrede-vom-hirten-und-dieb-und-vom-guten-hirten-und-mietling}}

\hypertarget{a-der-hirt-und-der-dieb-oder-ruxe4uber}{%
\paragraph{a) Der Hirt und der Dieb (oder:
Räuber)}\label{a-der-hirt-und-der-dieb-oder-ruxe4uber}}

\hypertarget{section-9}{%
\section{10}\label{section-9}}

\bibleverse{1} »Wahrlich, wahrlich ich sage euch: Wer nicht durch die
Tür in die Hürde der Schafe hineingeht, sondern anderswo hineinsteigt,
der ist ein Dieb und ein Räuber; \bibleverse{2} wer aber durch die Tür
hineingeht, der ist der Hirt der Schafe. \bibleverse{3} Diesem macht der
Türhüter auf, und die Schafe hören auf seine Stimme; er ruft die ihm
gehörenden Schafe mit Namen und führt sie hinaus. \bibleverse{4} Wenn er
dann alle Schafe, die ihm gehören, hinausgelassen hat, geht er vor ihnen
her, und die Schafe folgen ihm, weil sie seine Stimme kennen.
\bibleverse{5} Einem Fremden aber würden sie nicht folgen, sondern vor
ihm fliehen, weil sie die Stimme der Fremden nicht kennen.«
\bibleverse{6} Dies sagte Jesus ihnen in bildlicher Rede; sie verstanden
aber nicht, was er ihnen damit sagen wollte.

\hypertarget{b-die-erste-deutung-des-gleichnisses-jesus-als-die-offene-tuxfcr-fuxfcr-die-schafe}{%
\paragraph{b) Die erste Deutung des Gleichnisses: Jesus als die offene
Tür für die
Schafe}\label{b-die-erste-deutung-des-gleichnisses-jesus-als-die-offene-tuxfcr-fuxfcr-die-schafe}}

\bibleverse{7} Da sagte Jesus von neuem zu ihnen: »Wahrlich, wahrlich
ich sage euch: Ich bin die Tür für die Schafe! \bibleverse{8} Alle, die
vor mir gekommen sind, sind Diebe und Räuber; aber die Schafe haben
nicht auf sie gehört. \bibleverse{9} Ich bin die Tür: Wenn jemand durch
mich eingeht, wird er gerettet werden, wird ein- und ausgehen und Weide
finden. \bibleverse{10} Der Dieb kommt nur, um zu stehlen und zu
schlachten und Unheil anzurichten; ich aber bin gekommen, damit die
Schafe Leben haben und Überfluß\textless sup title=``oder: reiche
Fülle''\textgreater✲ haben.«

\hypertarget{c-die-zweite-deutung-jesus-als-der-gute-hirt-im-gegensatz-zum-mietling-und-als-begruxfcnder-der-menschheitsherde}{%
\paragraph{c) Die zweite Deutung: Jesus als der gute Hirt (im Gegensatz
zum Mietling) und als Begründer der
Menschheitsherde}\label{c-die-zweite-deutung-jesus-als-der-gute-hirt-im-gegensatz-zum-mietling-und-als-begruxfcnder-der-menschheitsherde}}

\bibleverse{11} »Ich bin der gute Hirt! Der gute Hirt gibt sein Leben
für die Schafe hin. \bibleverse{12} Der Mietling (aber), der kein Hirt
ist und dem die Schafe nicht zu eigen gehören, sieht den Wolf kommen,
verläßt die Schafe und flieht; und der Wolf fällt sie an und zerstreut
sie: \bibleverse{13} er ist ja nur ein Mietling, und ihm ist an den
Schafen nichts gelegen. \bibleverse{14} Ich bin der gute Hirt und kenne
die Meinen, und die Meinen kennen mich, \bibleverse{15} ebenso wie der
Vater mich kennt und ich den Vater kenne; und ich gebe mein Leben für
die Schafe hin.~-- \bibleverse{16} Ich habe auch noch andere Schafe, die
nicht zu dieser Hürde gehören; auch diese muß ich führen, und sie werden
auf meinen Ruf hören, und es wird eine Herde, ein Hirt sein.
\bibleverse{17} Um deswillen hat der Vater mich lieb, weil ich mein
Leben hingebe, damit ich es wieder an mich nehme; \bibleverse{18}
niemand nimmt es mir, sondern ich gebe es freiwillig hin. Ich habe
Vollmacht, es hinzugeben, und ich habe Vollmacht, es wieder an mich zu
nehmen; die Ermächtigung dazu habe ich von meinem Vater erhalten.«

\hypertarget{d-die-wirkung-der-rede}{%
\paragraph{d) Die Wirkung der Rede}\label{d-die-wirkung-der-rede}}

\bibleverse{19} Da entstand wegen dieser Worte wieder eine
Meinungsverschiedenheit unter den Juden. \bibleverse{20} Viele von ihnen
sagten nämlich: »Er ist von einem bösen Geist besessen und ist von
Sinnen: was hört ihr ihn noch an?« \bibleverse{21} Andere aber sagten:
»Das sind nicht Worte eines Besessenen; kann etwa ein böser Geist
Blinden die Augen auftun?«

\hypertarget{jesu-letzte-rechtfertigung-vor-den-juden-am-fest-der-tempelweihe-uxfcber-die-einheit-des-sohnes-mit-dem-vater}{%
\subsubsection{11. Jesu letzte Rechtfertigung vor den Juden am Fest der
Tempelweihe (über die Einheit des Sohnes mit dem
Vater)}\label{jesu-letzte-rechtfertigung-vor-den-juden-am-fest-der-tempelweihe-uxfcber-die-einheit-des-sohnes-mit-dem-vater}}

\bibleverse{22} Damals\textless sup title=``oder: danach''\textgreater✲
fand das Fest der Tempelweihe in Jerusalem statt; es war Winter,
\bibleverse{23} und Jesus ging im Tempel in der Halle Salomos auf und
ab. \bibleverse{24} Da umringten ihn die Juden und sagten zu ihm: »Wie
lange läßt du uns noch in Ungewißheit schweben? Bist du
Christus\textless sup title=``oder: der Messias''\textgreater✲, so sage
es uns frei heraus!« \bibleverse{25} Jesus antwortete ihnen: »Ich habe
es euch gesagt, doch ihr glaubt (es) nicht. Die Werke, die ich im Namen
meines Vaters vollbringe, die legen Zeugnis von mir\textless sup
title=``oder: für mich''\textgreater✲ ab; \bibleverse{26} aber ihr
glaubt nicht, weil ihr nicht zu meinen Schafen gehört. \bibleverse{27}
Meine Schafe hören auf meine Stimme, und ich kenne sie, und sie folgen
mir nach; \bibleverse{28} und ich gebe ihnen ewiges Leben, und sie
werden in alle Ewigkeit nicht umkommen\textless sup title=``oder:
verlorengehen''\textgreater✲, und niemand wird sie meiner Hand
entreißen. \bibleverse{29} Mein Vater, der sie mir gegeben hat, ist
größer als alle, und niemand vermag sie der Hand meines Vaters zu
entreißen. \bibleverse{30} Ich und der Vater sind eins!«

\bibleverse{31} Da holten die Juden wieder Steine herbei, um ihn zu
steinigen; \bibleverse{32} Jesus aber sagte zu ihnen: »Viele gute Werke
habe ich euch vom Vater her\textless sup title=``d.h. aus oder: in der
Macht meines Vaters''\textgreater✲ sehen lassen: welches von diesen
Werken ist es, wegen dessen ihr mich steinigen wollt?« \bibleverse{33}
Die Juden antworteten ihm: »Nicht wegen eines guten Werkes wollen wir
dich steinigen, sondern wegen Gotteslästerung, und zwar weil du, der du
doch (nur) ein Mensch bist, dich selbst zu Gott machst.« \bibleverse{34}
Jesus antwortete ihnen: »Steht nicht in eurem Gesetz
geschrieben\textless sup title=``Ps 82,6''\textgreater✲: ›Ich habe
gesagt: Ihr seid Götter‹? \bibleverse{35} Wenn die Schrift schon jene
(Männer), an die das Wort Gottes erging, Götter genannt hat -- und die
Schrift kann doch ihre Gültigkeit nicht verlieren --: \bibleverse{36}
wie könnt ihr da dem, welchem der Vater die Weihe erteilt und den er in
die Welt gesandt hat, Gotteslästerung vorwerfen, weil ich gesagt habe:
›Ich bin Gottes Sohn‹? \bibleverse{37} Wenn ich nicht die Werke meines
Vaters tue, so glaubt mir nicht; \bibleverse{38} wenn ich sie aber tue,
so glaubt, wenn auch nicht mir selbst, so doch meinen Werken, damit ihr
immer gewisser zu der Erkenntnis gelangt, daß der Vater in mir ist und
ich im Vater bin.« \bibleverse{39} Da suchten sie ihn wiederum
festzunehmen, doch er entkam aus ihren Händen.

\hypertarget{jesus-und-lazarus-jesus-als-die-auferstehung-und-das-leben}{%
\subsubsection{12. Jesus und Lazarus; Jesus als die Auferstehung und das
Leben}\label{jesus-und-lazarus-jesus-als-die-auferstehung-und-das-leben}}

\hypertarget{a-jesus-im-ostjordanlande-tod-des-lazarus-in-bethanien-jesu-aufbruch-nach-bethanien}{%
\paragraph{a) Jesus im Ostjordanlande; Tod des Lazarus in Bethanien;
Jesu Aufbruch nach
Bethanien}\label{a-jesus-im-ostjordanlande-tod-des-lazarus-in-bethanien-jesu-aufbruch-nach-bethanien}}

\bibleverse{40} Er zog nun wieder in das Ostjordanland an den
Ort\textless sup title=``oder: in die Gegend''\textgreater✲, wo Johannes
zuerst getauft hatte, und blieb dort. \bibleverse{41} Da kamen viele zu
ihm und sagten: »Johannes hat zwar keinerlei Wunder getan, alles aber,
was Johannes über diesen Mann gesagt hat, ist wahr gewesen.«
\bibleverse{42} Und viele kamen dort zum Glauben an ihn.

\hypertarget{section-10}{%
\section{11}\label{section-10}}

\bibleverse{1} Es lag aber ein Mann krank darnieder, Lazarus von
Bethanien, aus dem Dorfe, in welchem Maria und ihre Schwester Martha
wohnten~-- \bibleverse{2} es war die Maria, die den Herrn mit
Myrrhenbalsam gesalbt und seine Füße mit ihren Haaren getrocknet
hat\textless sup title=``vgl. 12,1-8''\textgreater✲ --: deren Bruder
Lazarus lag krank darnieder. \bibleverse{3} Da sandten die Schwestern zu
Jesus und ließen ihm sagen: »Herr, siehe, der, den du lieb hast, der ist
krank!« \bibleverse{4} Als Jesus das vernahm, sagte er: »Diese Krankheit
führt nicht zum Tode, sondern dient zur Verherrlichung Gottes, weil der
Sohn Gottes durch sie verherrlicht werden soll.« \bibleverse{5} Jesus
hatte aber die Martha und ihre Schwester und auch den Lazarus lieb.
\bibleverse{6} Als er nun von dessen Krankheit gehört hatte, blieb er
zunächst noch zwei Tage an dem Orte, wo er sich befand; \bibleverse{7}
dann erst sagte er zu seinen Jüngern: »Wir wollen wieder nach Judäa
ziehen!« \bibleverse{8} Die Jünger erwiderten ihm: »Rabbi✲, soeben erst
haben die Juden dich steinigen wollen, und nun willst du wieder dorthin
gehen?« \bibleverse{9} Jesus antwortete: »Hat der Tag nicht zwölf
Stunden? Wenn man am Tage wandert, stößt man nicht an, weil man das
Licht dieser Welt sieht; \bibleverse{10} wenn man aber bei Nacht
wandert, stößt man an, weil man kein Licht in sich hat, um zu sehen.«
\bibleverse{11} So sagte er und fuhr dann fort: »Unser Freund Lazarus
ist eingeschlafen; aber ich gehe hin, um ihn aus dem Schlaf zu wecken.«
\bibleverse{12} Da erwiderten ihm die Jünger: »Herr, wenn er
eingeschlafen ist, wird er wieder gesund werden.« \bibleverse{13} Jesus
hatte den Tod des Lazarus gemeint, sie dagegen waren der Meinung, er
rede vom gewöhnlichen Schlaf. \bibleverse{14} Da sagte Jesus ihnen denn
mit klaren Worten: »Lazarus ist gestorben, \bibleverse{15} und ich freue
mich euretwegen, daß ich nicht dort gewesen bin, damit ihr glauben
lernt. Doch nun laßt uns zu ihm gehen!« \bibleverse{16} Da sagte Thomas,
der auch den Namen ›Zwilling‹ führt, zu seinen Mitjüngern: »Laßt uns
hingehen, um mit ihm zu sterben!«

\hypertarget{b-jesu-ruxfcckkehr-nach-bethanien-sein-zusammentreffen-mit-martha-und-maria-und-sein-zeugnis-von-der-auferstehung-in-unverguxe4nglichkeit}{%
\paragraph{b) Jesu Rückkehr nach Bethanien; sein Zusammentreffen mit
Martha und Maria und sein Zeugnis von der Auferstehung in
Unvergänglichkeit}\label{b-jesu-ruxfcckkehr-nach-bethanien-sein-zusammentreffen-mit-martha-und-maria-und-sein-zeugnis-von-der-auferstehung-in-unverguxe4nglichkeit}}

\bibleverse{17} Als nun Jesus hinkam, fand er ihn schon seit vier Tagen
im Grabe liegen. \bibleverse{18} Bethanien lag aber in der Nähe von
Jerusalem, etwa fünfzehn Stadien\textless sup title=``=~kaum ein
Stündchen Wegs''\textgreater✲ von dort entfernt; \bibleverse{19} darum
hatten sich viele von den Juden bei Martha und Maria eingefunden, um sie
über den Tod ihres Bruders zu trösten. \bibleverse{20} Als nun Martha
von der Ankunft Jesu hörte, ging sie ihm entgegen; Maria aber blieb im
Hause (bei den Trauergästen) sitzen. \bibleverse{21} Da sagte Martha zu
Jesus: »Herr, wärest du hier gewesen, so wäre mein Bruder nicht
gestorben! \bibleverse{22} Doch auch so weiß ich, daß Gott dir alles
gewähren wird, um was du Gott bittest.« \bibleverse{23} Jesus erwiderte
ihr: »Dein Bruder wird auferstehen!« \bibleverse{24} Martha antwortete
ihm: »Ich weiß, daß er bei der Auferstehung am jüngsten Tage auferstehen
wird.« \bibleverse{25} Jesus entgegnete ihr: »Ich bin die Auferstehung
und das Leben; wer an mich glaubt, wird leben, wenn er auch stirbt,
\bibleverse{26} und wer da lebt und an mich glaubt\textless sup
title=``=~im Leben an mich glaubt''\textgreater✲, wird in Ewigkeit nicht
sterben! Glaubst du das?« \bibleverse{27} Sie antwortete ihm: »Ja, Herr,
ich habe den Glauben gewonnen, daß du Christus\textless sup
title=``=~der Messias''\textgreater✲ bist, der Sohn Gottes, der in die
Welt kommen soll.«

\bibleverse{28} Nach diesen Worten ging sie weg und rief ihre Schwester
Maria, indem sie ihr zuflüsterte: »Der Meister ist da und läßt dich
rufen!« \bibleverse{29} Sobald jene das gehört hatte, stand sie schnell
auf und machte sich auf den Weg zu ihm; \bibleverse{30} Jesus war aber
noch nicht in das Dorf gekommen, sondern befand sich noch an der Stelle,
wohin Martha ihm entgegengekommen war. \bibleverse{31} Als nun die
Juden, die bei Maria im Hause waren und sie zu trösten suchten, sie
schnell aufstehen und hinausgehen sahen, folgten sie ihr nach in der
Meinung, sie wolle zum Grabe gehen, um dort zu weinen. \bibleverse{32}
Als nun Maria an die Stelle kam, wo Jesus sich befand, und ihn erblickt
hatte, warf sie sich ihm zu Füßen und sagte zu ihm: »Herr, wärest du
hier gewesen, so wäre mein Bruder nicht gestorben!« \bibleverse{33} Als
nun Jesus sah, wie sie weinte und wie auch die Juden weinten, die mit
ihr gekommen waren, fühlte er sich im Geist heftig bewegt und
erschüttert.

\hypertarget{c-jesus-am-grabe-und-sein-gebet-die-auferweckung-des-lazarus}{%
\paragraph{c) Jesus am Grabe und sein Gebet; die Auferweckung des
Lazarus}\label{c-jesus-am-grabe-und-sein-gebet-die-auferweckung-des-lazarus}}

\bibleverse{34} Darauf fragte er: »Wo habt ihr ihn beigesetzt?« Sie
antworteten ihm: »Herr, komm und sieh es!« \bibleverse{35} Jesus weinte.
\bibleverse{36} Da sagten die Juden: »Seht, wie lieb hat er ihn gehabt!«
\bibleverse{37} Einige von ihnen aber sagten: »Hätte dieser, der dem
Blinden die Augen aufgetan hat, nicht auch machen können, daß dieser
hier nicht zu sterben brauchte?« \bibleverse{38} Da geriet Jesus in
seinem Innern aufs neue in heftige Erregung\textless sup title=``vgl.
V.33''\textgreater✲ und trat an das Grab; es war dies aber eine
Höhle\textless sup title=``=~ein Felsengrab''\textgreater✲, vor deren
Eingang ein Stein lag. \bibleverse{39} Jesus sagte: »Hebt den Stein
weg!« Martha, die Schwester des Verstorbenen, erwiderte ihm: »Herr, er
ist schon in Verwesung; es ist ja schon der vierte Tag seit seinem
Tode.« \bibleverse{40} Jesus entgegnete ihr: »Habe ich dir nicht gesagt,
daß, wenn du glaubst, du die Herrlichkeit Gottes sehen wirst?«
\bibleverse{41} Da hoben sie den Stein weg; Jesus aber richtete die
Augen (zum Himmel) empor und betete: »Vater, ich danke dir, daß du mich
erhört hast! \bibleverse{42} Ich wußte wohl, daß du mich allezeit
erhörst; aber um des Volkes willen, das hier rings (um mich) steht, habe
ich's gesagt, damit sie zum Glauben kommen, daß du mich gesandt hast.«
\bibleverse{43} Nach diesen Worten rief er mit lauter Stimme: »Lazarus,
komm heraus!« \bibleverse{44} Da kam der Gestorbene heraus, an den
Beinen und Armen mit Binden umwickelt, und sein Gesicht war mit einem
Schweißtuch umbunden. Jesus sagte zu ihnen: »Macht ihn los (von seinen
Hüllen) und laßt ihn (frei) gehen!«

\hypertarget{die-wirkungen-des-wunders-todesbeschluuxdf-des-hohen-rates-jesus-entweicht-nach-ephraim}{%
\subsubsection{13. Die Wirkungen des Wunders; Todesbeschluß des Hohen
Rates; Jesus entweicht nach
Ephraim}\label{die-wirkungen-des-wunders-todesbeschluuxdf-des-hohen-rates-jesus-entweicht-nach-ephraim}}

\bibleverse{45} Viele nun von den Juden, die zu Maria gekommen waren und
zugeschaut hatten bei dem, was Jesus getan hatte, wurden an ihn gläubig;
\bibleverse{46} einige von ihnen aber gingen weg zu den Pharisäern und
berichteten ihnen, was Jesus getan hatte. \bibleverse{47} Infolgedessen
beriefen die Hohenpriester und Pharisäer eine Versammlung des Hohen
Rates und sagten: »Was sollen wir tun, da dieser Mensch so viele
Wunderzeichen vollführt? \bibleverse{48} Lassen wir ihn so weiter
gewähren, so werden (schließlich) noch alle an ihn glauben, und dann
werden die Römer kommen und uns die Stätte\textless sup title=``d.h.
unser Heiligtum''\textgreater✲ und unser Volkstum beseitigen.«
\bibleverse{49} Einer aber von ihnen, nämlich Kaiphas, der in jenem
Jahre Hoherpriester war, sagte zu ihnen: »Ihr seid ganz ohne Einsicht
\bibleverse{50} und bedenkt nicht, daß es besser für euch ist, daß ein
einzelner Mensch für das Volk stirbt, und nicht das ganze Volk zugrunde
geht.« \bibleverse{51} Dies sagte er aber nicht von sich selbst aus,
sondern als Hoherpriester jenes Jahres weissagte er (unbewußt), daß
Jesus (zum Heil) für das Volk sterben würde, \bibleverse{52} und zwar
nicht für das (jüdische) Volk allein, sondern auch, damit er die (unter
den Völkern) zerstreuten Gotteskinder zu einem einheitlichen Ganzen
vereinigte. \bibleverse{53} So beratschlagten sie denn von diesem Tage
an miteinander in der Absicht, ihn zu töten. \bibleverse{54} Daher ging
Jesus nicht mehr öffentlich unter den Juden umher, sondern zog sich von
dort in die Gegend nahe bei der Wüste nach einer Stadt namens Ephraim
zurück und verweilte dort mit seinen Jüngern längere Zeit.

\bibleverse{55} Es stand aber das jüdische Passah nahe bevor, und viele
Leute zogen aus dem (ganzen) Lande schon vor dem Passah nach Jerusalem
hinauf, um sich zu heiligen\textless sup title=``oder: zu
weihen''\textgreater✲. \bibleverse{56} Sie suchten\textless sup
title=``oder: erkundigten sich''\textgreater✲ nun dort nach Jesus und
besprachen sich miteinander, während sie auf dem Tempelplatze standen:
»Was meint ihr? Er wird doch wohl nicht zum Feste kommen?«
\bibleverse{57} Die Hohenpriester und Pharisäer aber hatten mehrfach die
Verfügung ergehen lassen, wenn jemand seinen Aufenthaltsort in Erfahrung
bringe, solle er Anzeige erstatten, damit sie ihn festnehmen könnten.

\hypertarget{jesu-salbung-todesweihe-in-bethanien-lazarus-in-gefahr}{%
\subsubsection{14. Jesu Salbung (Todesweihe) in Bethanien; Lazarus in
Gefahr}\label{jesu-salbung-todesweihe-in-bethanien-lazarus-in-gefahr}}

\hypertarget{section-11}{%
\section{12}\label{section-11}}

\bibleverse{1} Jesus kam nun sechs Tage vor dem Passah nach Bethanien,
wo Lazarus wohnte, den Jesus von den Toten auferweckt hatte.
\bibleverse{2} Sie veranstalteten ihm zu Ehren dort ein Mahl, bei dem
Martha die Bedienung\textless sup title=``oder: Bewirtung''\textgreater✲
besorgte, während Lazarus sich unter denen befand, die mit ihm zu Tische
saßen. \bibleverse{3} Da nahm Maria ein Pfund Myrrhenbalsam, echte,
kostbare Nardensalbe, salbte Jesus die Füße und trocknete ihm die Füße
mit ihrem Haar ab; das ganze Haus wurde dabei vom Duft der Salbe
erfüllt. \bibleverse{4} Da sagte Judas Iskariot, einer von seinen
Jüngern, sein nachmaliger Verräter: \bibleverse{5} »Warum hat man diese
Salbe nicht für dreihundert Denare✲ verkauft und (den Erlös) den Armen
gegeben?« \bibleverse{6} Das sagte er aber nicht, weil ihm die Armen
sonderlich am Herzen lagen, sondern weil er ein Dieb war und als
Kassenführer die Einlagen veruntreute. \bibleverse{7} Da sagte Jesus:
»Laß sie in Ruhe! Sie soll (die Salbe) für den Tag meiner Bestattung
aufbewahrt haben. \bibleverse{8} Denn die Armen habt ihr allezeit bei
euch, mich aber habt ihr nicht allezeit.«

\bibleverse{9} Es erfuhr nun die zahlreiche Volksmenge der Juden, daß
Jesus dort sei; und sie kamen hin nicht nur um Jesu willen, sondern auch
um Lazarus zu sehen, den er von den Toten auferweckt hatte.
\bibleverse{10} Die Hohenpriester aber hielten Beratungen ab in der
Absicht, auch Lazarus zu töten, \bibleverse{11} weil viele Juden
seinetwegen dorthin gingen und zum Glauben an Jesus kamen.

\hypertarget{jesu-einzug-in-jerusalem-am-palmsonntag}{%
\subsubsection{15. Jesu Einzug in Jerusalem am
Palmsonntag}\label{jesu-einzug-in-jerusalem-am-palmsonntag}}

\bibleverse{12} Als dann am folgenden Tage von der Volksmenge, die zum
Fest gekommen war, ein großer Teil erfuhr, daß Jesus auf dem Wege nach
Jerusalem sei, \bibleverse{13} nahmen sie Palmenzweige, zogen hinaus ihm
entgegen und riefen laut: »Hosianna! Gepriesen\textless sup
title=``oder: gesegnet''\textgreater✲ sei, der da kommt im Namen des
Herrn und als der König Israels!«\textless sup title=``Ps
118,25-26''\textgreater✲ \bibleverse{14} Jesus hatte aber einen jungen
Esel vorgefunden und sich daraufgesetzt, wie geschrieben
steht\textless sup title=``Sach 9,9''\textgreater✲: \bibleverse{15}
»Fürchte dich nicht, Tochter Zion! Siehe, dein König kommt und reitet
auf einem Eselsfüllen.« \bibleverse{16} An dies Wort hatten seine Jünger
zunächst nicht gedacht; als Jesus aber zur Herrlichkeit eingegangen war,
da wurde es ihnen klar, daß dies mit Bezug auf ihn geschrieben stand und
daß man dies so an ihm zur Ausführung gebracht hatte. \bibleverse{17}
Die Volksmenge nun, die bei ihm gewesen war, als er Lazarus aus dem
Grabe gerufen und ihn von den Toten auferweckt hatte, hatte Zeugnis für
ihn abgelegt; \bibleverse{18} darum waren ihm auch die vielen Menschen
entgegengezogen, weil sie erfahren hatten, daß er dies Wunderzeichen
getan habe. \bibleverse{19} Da sagten die Pharisäer zueinander: »Ihr
seht, daß ihr nichts erreicht: die ganze Welt ist ja hinter ihm
hergelaufen!«

\hypertarget{das-ersuchen-der-griechischen-festteilnehmer-jesus-kuxfcndigt-sein-todesleiden-und-seine-daraufhin-erfolgende-verherrlichung-zum-weltheiland-an}{%
\subsubsection{16. Das Ersuchen der griechischen Festteilnehmer; Jesus
kündigt sein Todesleiden und seine daraufhin erfolgende Verherrlichung
zum Weltheiland
an}\label{das-ersuchen-der-griechischen-festteilnehmer-jesus-kuxfcndigt-sein-todesleiden-und-seine-daraufhin-erfolgende-verherrlichung-zum-weltheiland-an}}

\bibleverse{20} Es befanden sich aber einige Griechen unter
denen\textless sup title=``=~den Fremden''\textgreater✲, die nach
Jerusalem hinaufzuziehen pflegten, um dort ihre Anbetung\textless sup
title=``oder: Andacht''\textgreater✲ am Fest zu verrichten.
\bibleverse{21} Diese wandten sich nun an Philippus, der aus Bethsaida
in Galiläa war, mit der Bitte: »Herr, wir möchten Jesus gern sehen✲!«
\bibleverse{22} Da ging Philippus hin und sagte es dem Andreas; Andreas
und Philippus kamen alsdann und teilten es Jesus mit. \bibleverse{23}
Dieser antwortete ihnen mit den Worten: »Die Stunde der Verherrlichung
ist für den Menschensohn gekommen! \bibleverse{24} Wahrlich, wahrlich
ich sage euch: Wenn das Weizenkorn nicht in die Erde
hineinfällt\textless sup title=``=~hineingelegt wird''\textgreater✲ und
erstirbt, so bleibt es für sich allein; wenn es aber erstirbt, bringt es
reiche Frucht. \bibleverse{25} Wer sein Leben liebt, verliert es; wer
aber sein Leben in dieser Welt haßt, wird es zu ewigem Leben bewahren.
\bibleverse{26} Will jemand mir dienen, so folge er mir nach, und wo ich
bin, da wird auch mein Diener sein; wenn jemand mir dient, wird der
Vater ihn ehren\textless sup title=``oder: zu Ehren
bringen''\textgreater✲. \bibleverse{27} Jetzt ist meine Seele
erschüttert, und was soll ich sagen? (Soll ich bitten:) ›Vater, errette
mich aus dieser Stunde!‹? Nein, gerade deshalb bin ich ja in diese
Stunde gekommen: \bibleverse{28} Vater, verherrliche deinen Namen!« Da
erscholl eine Stimme aus dem Himmel: »Ich habe ihn (schon) verherrlicht
und werde ihn noch weiter\textless sup title=``oder: aufs
neue''\textgreater✲ verherrlichen!« \bibleverse{29} Da sagte die
Volksmenge, die dabeistand und zuhörte\textless sup title=``oder: es
hörte''\textgreater✲, es habe gedonnert; andere sagten: »Ein Engel hat
mit ihm geredet.« \bibleverse{30} Da nahm Jesus das Wort und sagte:
»Nicht um meinetwillen ist diese Stimme erschollen, sondern um
euretwillen. \bibleverse{31} Jetzt ergeht ein Gericht über diese Welt,
jetzt wird der Fürst dieser Welt hinausgestoßen werden, \bibleverse{32}
und ich werde, wenn ich von der Erde erhöht sein werde, alle zu mir
ziehen!« \bibleverse{33} Dies sagte er aber, um anzudeuten, welches
Todes er sterben würde. \bibleverse{34} Da entgegnete ihm die
Volksmenge: »Wir haben aus dem Gesetz\textless sup title=``=~der
Schrift''\textgreater✲ gehört, daß Christus\textless sup title=``oder:
der Messias''\textgreater✲ in Ewigkeit (am Leben) bleibt; wie kannst du
da behaupten, der Menschensohn müsse erhöht werden? Wer ist denn dieser
Menschensohn?« \bibleverse{35} Da sagte Jesus zu ihnen: »Nur noch kurze
Zeit ist das Licht unter euch. Wandelt (im Licht), solange ihr das Licht
noch habt, damit euch die Finsternis nicht überfällt✲; denn wer in der
Finsternis wandelt, weiß nicht, wohin er gelangt. \bibleverse{36}
Solange ihr das Licht noch habt, glaubt an das Licht, damit ihr Söhne✲
des Lichtes werdet!«

\hypertarget{ruxfcckblick-des-evangelisten-auf-die-uxf6ffentliche-wirksamkeit-jesu-schluuxdfurteil-uxfcber-das-ungluxe4ubige-juxfcdische-volk}{%
\subsubsection{17. Rückblick des Evangelisten auf die öffentliche
Wirksamkeit Jesu; Schlußurteil über das ungläubige jüdische
Volk}\label{ruxfcckblick-des-evangelisten-auf-die-uxf6ffentliche-wirksamkeit-jesu-schluuxdfurteil-uxfcber-das-ungluxe4ubige-juxfcdische-volk}}

So sprach Jesus, entfernte sich dann und hielt sich vor ihnen verborgen.
\bibleverse{37} Obwohl er aber so viele Wunderzeichen vor ihren Augen
getan hatte, glaubten sie doch nicht an ihn; \bibleverse{38} es sollte
sich eben das Wort des Propheten Jesaja erfüllen, das da
lautet\textless sup title=``Jes 53,1''\textgreater✲: »Herr, wer hat
unserer Botschaft✲ Glauben geschenkt, und wem ist der Arm des Herrn
offenbar geworden?« \bibleverse{39} Deshalb konnten sie nicht glauben,
weil Jesaja an einer anderen Stelle gesagt hat\textless sup title=``Jes
6,9-10''\textgreater✲: \bibleverse{40} »Er hat ihnen die Augen geblendet
und ihr Herz verhärtet, damit sie mit ihren Augen nicht sehen und mit
ihrem Herzen (nicht) zur Erkenntnis gelangen und sie sich (nicht)
bekehren sollten und ich sie (nicht) heile.« \bibleverse{41} So hat
Jesaja gesprochen, weil er seine\textless sup title=``d.h.
Jesu''\textgreater✲ Herrlichkeit schaute, und von ihm hat er geredet.
\bibleverse{42} Gleichwohl glaubten auch von den Obersten\textless sup
title=``d.h. Mitgliedern des Hohen Rates''\textgreater✲ viele an ihn,
bekannten es aber um der Pharisäer willen nicht offen, um nicht in den
Bann getan zu werden\textless sup title=``vgl. 9,22''\textgreater✲;
\bibleverse{43} denn an der Ehre bei den Menschen lag ihnen mehr als an
der Ehre bei Gott.

\hypertarget{jesu-zeugnis-uxfcber-sich-und-uxfcber-sein-verhuxe4ltnis-zu-gott}{%
\subsubsection{18. Jesu Zeugnis über sich und über sein Verhältnis zu
Gott}\label{jesu-zeugnis-uxfcber-sich-und-uxfcber-sein-verhuxe4ltnis-zu-gott}}

\bibleverse{44} Jesus aber rief mit lauter Stimme aus: »Wer an mich
glaubt, glaubt nicht an mich, sondern an den, der mich gesandt hat;
\bibleverse{45} und wer mich sieht, sieht den, der mich gesandt hat.
\bibleverse{46} Ich bin als Licht in die Welt gekommen, damit jeder, der
an mich glaubt, nicht in der Finsternis bleibt. \bibleverse{47} Und wenn
jemand meine Worte hört und sie nicht befolgt✲, so richte nicht ich ihn;
denn ich bin nicht gekommen, um die Welt zu richten, sondern um die Welt
zu retten. \bibleverse{48} Wer mich verwirft und meine Worte nicht
annimmt, der hat (damit schon) seinen Richter: das Wort, das ich
verkündet habe, wird sein Richter sein am jüngsten Tage. \bibleverse{49}
Denn ich habe nicht von mir selbst aus geredet, sondern der Vater, der
mich gesandt hat, der hat mir Auftrag gegeben, was ich sagen und was ich
reden soll, \bibleverse{50} und ich weiß, daß sein Auftrag ewiges Leben
bedeutet. Was ich also rede, das rede ich so, wie der Vater es mir
gesagt hat.«

\hypertarget{iii.-jesus-offenbart-seinen-juxfcngern-beim-abschied-seinen-weg-zur-herrlichkeit-und-ihren-weg-ebendahin-kap.-13-17}{%
\subsection{III. Jesus offenbart seinen Jüngern beim Abschied seinen Weg
zur Herrlichkeit und ihren Weg ebendahin (Kap.
13-17)}\label{iii.-jesus-offenbart-seinen-juxfcngern-beim-abschied-seinen-weg-zur-herrlichkeit-und-ihren-weg-ebendahin-kap.-13-17}}

\hypertarget{jesu-letztes-mahl-mit-seinen-juxfcngern}{%
\subsubsection{1. Jesu letztes Mahl mit seinen
Jüngern}\label{jesu-letztes-mahl-mit-seinen-juxfcngern}}

\hypertarget{a-die-fuuxdfwaschung}{%
\paragraph{a) Die Fußwaschung}\label{a-die-fuuxdfwaschung}}

\hypertarget{section-12}{%
\section{13}\label{section-12}}

\bibleverse{1} Vor dem Passahfest aber, da Jesus wohl wußte, daß für ihn
die Stunde gekommen sei, aus dieser Welt zum Vater hinüberzugehen,
bewies er den Seinen, die in der Welt waren, die Liebe, die er (bisher)
zu ihnen gehegt hatte, bis zum letzten Augenblick. \bibleverse{2} Es war
bei einem Mahl\textless sup title=``d.h. Abendessen''\textgreater✲, und
schon hatte der Teufel dem Judas Iskariot, dem Sohne Simons, den
Entschluß des Verrats eingegeben. \bibleverse{3} Weil Jesus nun wußte,
daß der Vater ihm alles in die Hände gegeben hatte und daß er von Gott
ausgegangen sei und wieder zu Gott hingehe, \bibleverse{4} erhob er sich
beim Mahl von seinem Platz, legte die Oberkleidung ab, nahm einen
linnenen Schurz und band ihn sich um. \bibleverse{5} Danach goß er
Wasser in das Waschbecken und begann seinen Jüngern die Füße zu waschen
und sie mit dem linnenen Schurz, den er sich umgebunden hatte,
abzutrocknen. \bibleverse{6} So kam er denn auch zu Simon Petrus. Dieser
sagte zu ihm: »Herr, du willst mir die Füße waschen?« \bibleverse{7}
Jesus antwortete ihm mit den Worten: »Was ich damit tue, verstehst du
jetzt noch nicht, du wirst es aber nachher verstehen.« \bibleverse{8}
Petrus entgegnete ihm: »Nun und nimmer sollst du mir die Füße waschen!«
Jesus antwortete ihm: »Wenn ich dich nicht wasche, so hast du keinen
Anteil an mir\textless sup title=``oder: keine Gemeinschaft mit
mir''\textgreater✲.« \bibleverse{9} Da sagte Simon Petrus zu ihm: »Herr,
dann nicht nur meine Füße, sondern auch die Hände und den Kopf!«
\bibleverse{10} Jesus antwortete ihm: »Wer gebadet ist\textless sup
title=``oder: sich gebadet hat''\textgreater✲, dem braucht nichts weiter
gewaschen zu werden als die Füße, sondern er ist am ganzen Körper rein;
und ihr seid rein, jedoch nicht alle.« \bibleverse{11} Er kannte nämlich
seinen Verräter wohl; deshalb sagte er: »Ihr seid nicht alle rein.«

\hypertarget{b-jesu-deutung-seines-demuxfctigen-liebesdienstes}{%
\paragraph{b) Jesu Deutung seines demütigen
Liebesdienstes}\label{b-jesu-deutung-seines-demuxfctigen-liebesdienstes}}

\bibleverse{12} Nachdem er ihnen nun die Füße gewaschen und seine
Oberkleidung wieder angelegt und seinen Platz am Tisch wieder
eingenommen hatte, sagte er zu ihnen: »Versteht ihr, was ich an euch
getan habe? \bibleverse{13} Ihr redet mich mit ›Meister‹✲ und ›Herr‹ an
und habt recht mit dieser Benennung, denn ich bin es wirklich.
\bibleverse{14} Wenn nun ich, der Herr und der Meister, euch die Füße
gewaschen habe, so seid auch ihr verpflichtet, einander die Füße zu
waschen; \bibleverse{15} denn ein Vorbild habe ich euch gegeben, damit
ihr es ebenso machet, wie ich an euch getan habe. \bibleverse{16}
Wahrlich, wahrlich ich sage euch: Ein Knecht steht nicht höher als sein
Herr, und ein Sendbote✲ nicht höher als sein Absender. \bibleverse{17}
Wenn ihr dies wißt -- selig seid ihr, wenn ihr danach handelt!
\bibleverse{18} Nicht von euch allen rede ich; ich weiß ja, wie die
beschaffen sind, welche ich erwählt habe; aber das Schriftwort muß
erfüllt werden\textless sup title=``Ps 41,10''\textgreater✲: ›Der mein
Brot ißt, hat seine Ferse gegen mich erhoben.‹ \bibleverse{19} Schon
jetzt sage ich es euch, noch bevor es geschieht, damit ihr, wenn es
geschehen ist, glaubt, daß ich es bin (den die Schrift meint).
\bibleverse{20} Wahrlich, wahrlich ich sage euch: Wer dann, wenn ich
jemand sende, ihn aufnimmt, der nimmt mich auf; wer aber mich aufnimmt,
nimmt den auf, der mich gesandt hat.«\textless sup title=``Mt
10,40''\textgreater✲

\hypertarget{c-kennzeichnung-und-entfernung-des-verruxe4ters}{%
\paragraph{c) Kennzeichnung und Entfernung des
Verräters}\label{c-kennzeichnung-und-entfernung-des-verruxe4ters}}

\bibleverse{21} Nach diesen Worten wurde Jesus im Geist aufs tiefste
erschüttert und sprach es offen aus: »Wahrlich, wahrlich ich sage euch:
Einer von euch wird mich verraten!« \bibleverse{22} Da blickten die
Jünger einander an und waren ratlos darüber, wen er meinte.
\bibleverse{23} Es hatte aber einer von seinen Jüngern bei Tisch seinen
Platz an der Brust\textless sup title=``=~an der Seite''\textgreater✲
Jesu, nämlich der, den Jesus (besonders) lieb hatte. \bibleverse{24}
Diesem gab nun Simon Petrus einen Wink und sagte ihm\textless sup
title=``=~bedeutete ihm damit''\textgreater✲: »Laß uns wissen, wen er
meint!« \bibleverse{25} Jener lehnte sich nun auch sogleich an die Brust
Jesu zurück und fragte ihn: »Herr, wer ist es?« \bibleverse{26} Da
antwortete Jesus: »Der ist es, dem ich den Bissen (in die Schüssel)
eintauchen und reichen werde.« Darauf tauchte er den Bissen ein, nahm
ihn und reichte ihn dem Judas, dem Sohne Simons aus Kariot.
\bibleverse{27} Nachdem dieser den Bissen genommen hatte, fuhr der Satan
in ihn hinein. Nun sagte Jesus zu ihm: »Was du zu tun vorhast, das tu
bald!« \bibleverse{28} Was er ihm damit hatte sagen wollen, verstand
keiner von den Tischgenossen. \bibleverse{29} Einige nämlich meinten,
weil Judas die Kasse führte, wolle Jesus ihm sagen: »Kaufe das ein, was
wir für das Fest nötig haben«, oder er solle den Armen etwas geben.
\bibleverse{30} Nachdem nun jener den Bissen genommen hatte, ging er
sogleich hinaus. Es war aber Nacht.

\hypertarget{jesu-abschieds--und-trostreden-1331-1726}{%
\subsubsection{2. Jesu Abschieds- und Trostreden
(13,31-17,26)}\label{jesu-abschieds--und-trostreden-1331-1726}}

\hypertarget{a-beginn-und-grundlage-der-abschiedsreden}{%
\paragraph{a) Beginn und Grundlage der
Abschiedsreden}\label{a-beginn-und-grundlage-der-abschiedsreden}}

\hypertarget{aa-jesu-ankuxfcndigung-seiner-verherrlichung}{%
\subparagraph{aa) Jesu Ankündigung seiner
Verherrlichung}\label{aa-jesu-ankuxfcndigung-seiner-verherrlichung}}

\bibleverse{31} Nach seinem Weggange nun sagte Jesus: »Jetzt ist der
Menschensohn verherrlicht, und Gott ist in ihm\textless sup
title=``oder: durch ihn''\textgreater✲ verherrlicht worden!
\bibleverse{32} Wenn Gott in ihm\textless sup title=``oder: durch
ihn''\textgreater✲ verherrlicht ist, so wird Gott auch ihn in sich
selbst\textless sup title=``oder: durch sich''\textgreater✲
verherrlichen, und zwar wird er ihn sofort verherrlichen.
\bibleverse{33} Liebe Kinder, nur noch kurze Zeit bin ich bei euch; dann
werdet ihr mich suchen, und, wie ich schon den Juden gesagt habe: ›Wohin
ich gehe, dahin könnt ihr nicht kommen‹, so sage ich es jetzt auch
euch.«

\hypertarget{bb-das-neue-gebot-der-liebe}{%
\subparagraph{bb) Das neue Gebot der
Liebe}\label{bb-das-neue-gebot-der-liebe}}

\bibleverse{34} »Ein neues Gebot gebe ich euch, daß ihr einander lieben
sollt; wie ich euch geliebt habe, so sollt auch ihr einander lieben.
\bibleverse{35} Daran werden alle erkennen, daß ihr meine Jünger seid,
wenn ihr Liebe untereinander habt.«

\hypertarget{cc-ankuxfcndigung-der-verleugnung-des-petrus}{%
\subparagraph{cc) Ankündigung der Verleugnung des
Petrus}\label{cc-ankuxfcndigung-der-verleugnung-des-petrus}}

\bibleverse{36} Da fragte ihn Simon Petrus: »Herr, wohin gehst du?«
Jesus antwortete ihm: »Wohin ich gehe, dahin kannst du mir jetzt nicht
folgen; du wirst mir aber später folgen.« Petrus antwortete ihm:
\bibleverse{37} »Herr, warum sollte ich dir jetzt nicht folgen können?
Mein Leben will ich für dich hingeben!« \bibleverse{38} Da antwortete
Jesus: »Dein Leben willst du für mich hingeben? Wahrlich, wahrlich ich
sage dir: Der Hahn wird nicht krähen, bevor du mich dreimal verleugnet
hast.«

\hypertarget{b-erste-trostrede}{%
\paragraph{b) Erste Trostrede}\label{b-erste-trostrede}}

\hypertarget{aa-jesu-verheiuxdfung-seiner-wiederkunft-und-der-aufnahme-der-juxfcnger-in-die-bei-gott-bereitete-stuxe4tte-jesus-der-weg-zu-gott-seine-einheit-mit-gott}{%
\subparagraph{aa) Jesu Verheißung seiner Wiederkunft und der Aufnahme
der Jünger in die bei Gott bereitete Stätte; Jesus der Weg zu Gott,
seine Einheit mit
Gott}\label{aa-jesu-verheiuxdfung-seiner-wiederkunft-und-der-aufnahme-der-juxfcnger-in-die-bei-gott-bereitete-stuxe4tte-jesus-der-weg-zu-gott-seine-einheit-mit-gott}}

\hypertarget{section-13}{%
\section{14}\label{section-13}}

\bibleverse{1} »Euer Herz erschrecke nicht! Vertrauet auf Gott und
vertrauet auf mich! \bibleverse{2} In meines Vaters Hause sind viele
Wohnungen; wenn es nicht so wäre, hätte ich es euch gesagt; denn ich
gehe hin, euch eine Stätte zu bereiten; \bibleverse{3} und wenn ich
hingegangen bin und euch eine Stätte bereitet habe, komme ich wieder und
werde euch zu mir nehmen, damit da, wo ich bin, auch ihr seid.
\bibleverse{4} Und wohin ich gehe -- den Weg dahin kennt ihr.«
\bibleverse{5} Da sagte Thomas zu ihm: »Herr, wir wissen nicht, wohin du
gehst: wie sollten wir da den Weg kennen?« \bibleverse{6} Jesus
antwortete ihm: »Ich bin der Weg und die Wahrheit und das Leben; niemand
kommt zum Vater außer durch mich. \bibleverse{7} Wenn ihr mich erkannt
hättet, würdet ihr auch meinen Vater kennen; von jetzt an kennt ihr ihn
und habt ihn gesehen.« \bibleverse{8} Philippus sagte zu ihm: »Herr,
zeige uns den Vater: das genügt uns.« \bibleverse{9} Da sagte Jesus zu
ihm: »So lange Zeit schon bin ich mit euch zusammen, und (trotzdem) hast
du mich noch nicht erkannt, Philippus? Wer mich gesehen hat, der hat den
Vater gesehen; wie kannst du sagen: ›Zeige uns den Vater!‹
\bibleverse{10} Glaubst du nicht, daß ich im Vater bin und der Vater in
mir ist? Die Worte, die ich zu euch rede, spreche ich nicht von mir
selbst aus, nein, der Vater, der dauernd in mir ist, der tut seine
Werke. \bibleverse{11} Glaubet mir, daß ich im Vater bin und der Vater
in mir ist; wo nicht, so glaubt doch um der Werke selbst willen!«

\hypertarget{bb-verheiuxdfung-der-gebetserhuxf6rung-und-der-erfolgreichsten-wirksamkeit-des-dauernden-besitzes-des-heiligen-geistes-des-wiedersehens-und-ewiger-vereinigung}{%
\subparagraph{bb) Verheißung der Gebetserhörung und der erfolgreichsten
Wirksamkeit, des dauernden Besitzes des heiligen Geistes, des
Wiedersehens und ewiger
Vereinigung}\label{bb-verheiuxdfung-der-gebetserhuxf6rung-und-der-erfolgreichsten-wirksamkeit-des-dauernden-besitzes-des-heiligen-geistes-des-wiedersehens-und-ewiger-vereinigung}}

\bibleverse{12} Wahrlich, wahrlich ich sage euch: Wer an mich glaubt,
wird die Werke, die ich tue, auch vollbringen, ja er wird noch größere
als diese vollbringen; \bibleverse{13} denn ich gehe zum Vater, und
alles, um was ihr (dann) in meinem Namen bitten werdet, das werde ich
tun, damit der Vater im Sohn\textless sup title=``oder: durch den
Sohn''\textgreater✲ verherrlicht werde. \bibleverse{14} Wenn ihr mich um
etwas in meinem Namen bitten werdet, so werde ich es tun.~--
\bibleverse{15} Wenn ihr mich liebt, so werdet ihr meine Gebote halten;
\bibleverse{16} und ich werde den Vater bitten, und er wird euch einen
anderen Helfer\textless sup title=``oder: Anwalt,
Beistand''\textgreater✲ geben, damit er bis in Ewigkeit bei euch sei:
\bibleverse{17} den Geist der Wahrheit, den die Welt nicht empfangen
kann, weil sie ihn nicht sieht\textless sup title=``d.h. kein Auge für
ihn hat''\textgreater✲ und ihn nicht erkennt; ihr aber erkennt ihn, weil
er bei euch bleibt und in euch sein wird.~-- \bibleverse{18} Ich will
euch nicht verwaist zurücklassen\textless sup title=``=~als Waisenkinder
dastehen lassen''\textgreater✲: ich komme zu euch! \bibleverse{19} Nur
noch eine kurze Zeit, dann sieht mich die Welt nicht mehr; ihr aber seht
mich, daß ich lebe, und ihr sollt auch leben! \bibleverse{20} An jenem
Tage werdet ihr erkennen, daß ich in meinem Vater bin und ihr in mir
seid und ich in euch.«

\hypertarget{cc-verheiuxdfung-der-innigsten-geistes--und-liebesgemeinschaft-mit-gott-und-jesus}{%
\subparagraph{cc) Verheißung der innigsten Geistes- und
Liebesgemeinschaft mit Gott und
Jesus}\label{cc-verheiuxdfung-der-innigsten-geistes--und-liebesgemeinschaft-mit-gott-und-jesus}}

\bibleverse{21} »Wer meine Gebote hat und sie hält✲, der ist es, der
mich liebt; wer aber mich liebt, wird von meinem Vater geliebt werden,
und auch ich werde ihn lieben und mich ihm offenbaren.« \bibleverse{22}
Da fragte ihn Judas -- nicht der Iskariot --: »Herr, wie kommt
es\textless sup title=``oder: welches ist der Grund''\textgreater✲, daß
du dich (nur) uns offenbaren willst und nicht (auch) der Welt?«
\bibleverse{23} Jesus antwortete ihm mit den Worten: »Wenn jemand mich
liebt, wird er mein Wort halten\textless sup title=``oder:
befolgen''\textgreater✲, und mein Vater wird ihn lieben, und wir werden
zu ihm kommen und Wohnung bei ihm nehmen. \bibleverse{24} Wer mich nicht
liebt, hält\textless sup title=``oder: befolgt''\textgreater✲ auch meine
Worte nicht; und doch kommt das Wort, das ihr hört, nicht von mir,
sondern vom Vater, der mich gesandt hat.«

\hypertarget{dd-zusage-der-belehrung-durch-den-heiligen-geist-friedensgruuxdf-und-aufforderung-zur-glaubenszuversicht}{%
\subparagraph{dd) Zusage der Belehrung durch den heiligen Geist;
Friedensgruß und Aufforderung zur
Glaubenszuversicht}\label{dd-zusage-der-belehrung-durch-den-heiligen-geist-friedensgruuxdf-und-aufforderung-zur-glaubenszuversicht}}

\bibleverse{25} »Dies habe ich zu euch geredet, während ich bei euch
weilte. \bibleverse{26} Der Helfer\textless sup title=``oder: Anwalt,
Beistand''\textgreater✲ aber, der heilige Geist, den der Vater in meinem
Namen senden wird, der wird euch über alles (Weitere) belehren und euch
an alles erinnern, was ich euch gesagt habe.~-- \bibleverse{27} Frieden
hinterlasse ich euch, meinen Frieden gebe ich euch; nicht so, wie die
Welt gibt, gebe ich euch. Euer Herz erschrecke nicht und verzage nicht!
\bibleverse{28} Ihr habt gehört, daß ich euch gesagt habe: ›Ich gehe hin
und komme wieder zu euch.‹ Hättet ihr mich lieb, so hättet ihr euch
gefreut, daß ich zum Vater gehe, denn der Vater ist größer als ich.
\bibleverse{29} Und schon jetzt habe ich es euch gesagt, bevor es
geschieht, damit ihr zum Glauben kommt, wenn es geschieht.
\bibleverse{30} Ich werde nicht mehr viel mit euch reden, denn es kommt
der Fürst der Welt; doch über mich hat er keine Macht\textless sup
title=``oder: kein Anrecht auf mich''\textgreater✲. \bibleverse{31}
Damit aber die Welt erkennt, daß ich den Vater liebe und so tue, wie der
Vater mir geboten hat: erhebt euch! Laßt uns von hier aufbrechen!«

\hypertarget{c-zweite-abschieds--und-trostrede}{%
\paragraph{c) Zweite Abschieds- und
Trostrede}\label{c-zweite-abschieds--und-trostrede}}

\hypertarget{aa-gleichnis-vom-weinstock-und-den-reben}{%
\subparagraph{aa) Gleichnis vom Weinstock und den
Reben}\label{aa-gleichnis-vom-weinstock-und-den-reben}}

\hypertarget{section-14}{%
\section{15}\label{section-14}}

\bibleverse{1} »Ich bin der wahre Weinstock, und mein Vater ist der
Weingärtner. \bibleverse{2} Jede Rebe an mir, die keine Frucht bringt,
entfernt er, und jede (Rebe), die Frucht bringt, reinigt er, damit sie
noch mehr Frucht bringe. \bibleverse{3} Ihr seid bereits rein infolge
des Wortes, das ich zu euch geredet habe: \bibleverse{4} bleibt in mir,
so bleibe ich in euch. Wie die Rebe nicht von sich selbst aus Frucht
bringen kann, wenn sie nicht am Weinstock bleibt, so könnt auch ihr es
nicht, wenn ihr nicht in mir bleibt. \bibleverse{5} Ich bin der
Weinstock, ihr seid die Reben: wer in mir bleibt und in wem ich bleibe,
der bringt reichlich Frucht; dagegen ohne mich könnt ihr nichts
vollbringen. \bibleverse{6} Wer nicht in mir bleibt, der wird
weggeworfen wie die Rebe und verdorrt; man sammelt sie dann und wirft
sie ins Feuer: da verbrennen sie. \bibleverse{7} Wenn ihr in mir bleibt
und meine Worte in euch bleiben, dann bittet, um was ihr wollt: es wird
euch zuteil werden. \bibleverse{8} Dadurch ist mein Vater verherrlicht,
daß ihr reichlich Frucht bringt und euch als meine Jünger erweist.«

\hypertarget{bb-das-liebesgebot-bleibt-in-der-liebesgemeinschaft-mit-mir-und-untereinander}{%
\subparagraph{bb) Das Liebesgebot: Bleibt in der Liebesgemeinschaft mit
mir und
untereinander!}\label{bb-das-liebesgebot-bleibt-in-der-liebesgemeinschaft-mit-mir-und-untereinander}}

\bibleverse{9} »Wie mich der Vater geliebt hat, so habe auch ich euch
geliebt: bleibet in meiner Liebe! \bibleverse{10} Wenn ihr meine Gebote
haltet\textless sup title=``oder: befolgt''\textgreater✲, werdet ihr in
meiner Liebe bleiben, gleichwie ich die Gebote meines Vaters
gehalten\textless sup title=``oder: befolgt''\textgreater✲ habe und
damit in seiner Liebe bleibe. \bibleverse{11} Dies habe ich zu euch
geredet, damit die Freude, wie ich sie habe, auch in euch (vorhanden)
sei und eure Freude vollkommen werde.~-- \bibleverse{12} Das ist mein
Gebot, daß ihr einander liebt, wie ich euch geliebt habe.
\bibleverse{13} Größere Liebe kann niemand haben als die, daß er sein
Leben für seine Freunde hingibt. \bibleverse{14} Ihr seid meine Freunde,
wenn ihr tut, was ich euch gebiete. \bibleverse{15} Ich nenne euch nicht
mehr Knechte, denn der Knecht hat keine Einsicht in das Tun seines
Herrn; vielmehr habe ich euch Freunde genannt, weil ich euch alles
kundgetan habe, was ich von meinem Vater gehört habe. \bibleverse{16}
Nicht ihr habt mich erwählt, sondern ich habe euch erwählt und euch dazu
bestellt, daß ihr hingehen und Frucht bringen sollt und eure Frucht eine
bleibende sei, auf daß der Vater euch alles gebe, um was ihr ihn in
meinem Namen bittet. \bibleverse{17} Dies ist mein Gebot an euch, daß
ihr einander liebet.«

\hypertarget{cc-weissagung-des-durch-den-hauxdf-der-welt-leidvollen-juxfcngerschicksals}{%
\subparagraph{cc) Weissagung des durch den Haß der Welt leidvollen
Jüngerschicksals}\label{cc-weissagung-des-durch-den-hauxdf-der-welt-leidvollen-juxfcngerschicksals}}

\bibleverse{18} »Wenn die Welt euch haßt, so bedenkt, daß sie mich noch
eher als euch gehaßt hat! \bibleverse{19} Wenn ihr aus der Welt
wärt\textless sup title=``oder: zur Welt gehörtet''\textgreater✲, so
würde die Welt euch als das zu ihr Gehörige lieben; weil ihr aber nicht
aus der Welt seid, sondern ich euch aus der Welt heraus erwählt✲ habe,
deshalb haßt euch die Welt. \bibleverse{20} Gedenkt an das Wort, das ich
euch gesagt habe\textless sup title=``vgl. 13,16''\textgreater✲: ›Ein
Knecht steht nicht höher als sein Herr.‹ Haben sie mich verfolgt, so
werden sie auch euch verfolgen; haben sie mein Wort befolgt, so werden
sie auch das eure befolgen. \bibleverse{21} Dies alles aber werden sie
euch um meines Namens willen antun, weil sie den nicht kennen, der mich
gesandt hat. \bibleverse{22} Wenn ich nicht gekommen wäre und nicht zu
ihnen geredet hätte, so wären sie frei von Verschulden; so aber haben
sie keine Entschuldigung für ihr Verschulden. \bibleverse{23} Wer mich
haßt, der haßt auch meinen Vater. \bibleverse{24} Wenn ich nicht solche
Werke unter ihnen getan hätte, wie kein anderer sie getan hat, so wären
sie frei von Verschulden; so aber haben sie (alles) gesehen und doch
sowohl mich als auch meinen Vater gehaßt. \bibleverse{25} Aber es muß
das Wort, das in ihrem Gesetz geschrieben steht\textless sup title=``Ps
35,19; 69,5''\textgreater✲, erfüllt werden: ›Sie haben mich ohne Grund
gehaßt.‹~-- \bibleverse{26} Wenn aber der Helfer\textless sup
title=``oder: Anwalt, Beistand''\textgreater✲ kommt, den ich euch vom
Vater her senden werde, der Geist der Wahrheit, der vom Vater ausgeht,
der wird Zeugnis über mich\textless sup title=``oder: für
mich''\textgreater✲ ablegen. \bibleverse{27} Doch auch ihr seid (meine)
Zeugen, weil ihr von Anfang an bei mir (gewesen) seid.

\hypertarget{section-15}{%
\section{16}\label{section-15}}

\bibleverse{1} Dies habe ich euch gesagt, damit ihr nicht Anstoß
nehmt\textless sup title=``=~im Glauben irre werdet''\textgreater✲.
\bibleverse{2} Man wird euch in den Bann tun\textless sup title=``vgl.
9,22''\textgreater✲; ja, es kommt die Stunde, wo jeder, der euch tötet,
Gott eine Opfergabe darzubringen\textless sup title=``=~einen heiligen
Dienst zu erweisen''\textgreater✲ meint. \bibleverse{3} Und so werden
sie verfahren, weil sie weder den Vater noch mich erkannt
haben\textless sup title=``oder: kennen''\textgreater✲. \bibleverse{4}
Aber ich habe euch dies gesagt, damit, wenn die Stunde der Erfüllung
kommt, ihr daran gedenkt, daß ich es euch gesagt habe.«

\hypertarget{dd-verheiuxdfung-des-heiligen-geistes-und-dessen-segensreiches-wirken-an-der-welt-und-in-den-juxfcngern}{%
\subparagraph{dd) Verheißung des heiligen Geistes und dessen
segensreiches Wirken an der Welt und in den
Jüngern}\label{dd-verheiuxdfung-des-heiligen-geistes-und-dessen-segensreiches-wirken-an-der-welt-und-in-den-juxfcngern}}

»Dies habe ich euch aber nicht gleich anfangs gesagt, weil ich noch bei
euch war. \bibleverse{5} Jetzt aber gehe ich hin zu dem, der mich
gesandt hat, und keiner von euch fragt mich: ›Wohin gehst du?‹,
\bibleverse{6} sondern weil ich dies zu euch gesagt habe, hat die
Traurigkeit euer Herz erfüllt. \bibleverse{7} Aber ich sage euch die
Wahrheit: Es ist gut für euch, daß ich weggehe. Denn wenn ich nicht
weggehe, so wird der Helfer\textless sup title=``oder: Anwalt,
Beistand''\textgreater✲ nicht zu euch kommen; wenn ich aber hingegangen
bin, werde ich ihn zu euch senden. \bibleverse{8} Und wenn er gekommen
ist, wird er der Welt die Augen öffnen über Sünde und über Gerechtigkeit
und über Gericht: \bibleverse{9} über Sünde, (die darin besteht) daß sie
nicht an mich glauben; \bibleverse{10} über Gerechtigkeit, (die darin
besteht) daß ich zum Vater hingehe und ihr mich fortan nicht mehr seht;
\bibleverse{11} über Gericht, (das darin besteht) daß der Fürst dieser
Welt gerichtet ist.

\bibleverse{12} Noch vieles hätte ich euch zu sagen, doch ihr könnt es
jetzt nicht tragen. \bibleverse{13} Wenn aber jener gekommen ist, der
Geist der Wahrheit, der wird euch in die ganze✲ Wahrheit einführen; denn
er wird nicht von sich selbst aus reden, sondern was er hört, das wird
er reden und euch das Zukünftige verkündigen. \bibleverse{14} Er wird
mich verherrlichen, denn von meinem Eigentum\textless sup title=``oder:
Gut''\textgreater✲ wird er es nehmen und euch verkündigen. Alles, was
der Vater hat, ist mein; \bibleverse{15} deshalb habe ich gesagt, daß er
es von meinem Eigentum nimmt und es euch verkündigen wird.«

\hypertarget{ee-verheiuxdfung-baldigen-wiedersehens-und-mahnung-zum-gebet-im-namen-jesu}{%
\subparagraph{ee) Verheißung baldigen Wiedersehens und Mahnung zum Gebet
im Namen
Jesu}\label{ee-verheiuxdfung-baldigen-wiedersehens-und-mahnung-zum-gebet-im-namen-jesu}}

\bibleverse{16} »Nur noch eine kurze Zeit, so seht ihr mich nicht mehr;
dann wieder eine kurze Zeit, so werdet ihr mich sehen.« \bibleverse{17}
Da sagten einige von seinen Jüngern zueinander: »Was meint er damit, daß
er zu uns sagt: ›Nur noch eine kurze Zeit, so seht ihr mich nicht mehr;
dann wieder eine kurze Zeit, so werdet ihr mich sehen‹, und weiter: ›Ich
gehe hin zum Vater‹?« \bibleverse{18} Sie sagten also: »Was meint er mit
dem Ausdruck ›eine kurze Zeit‹? Wir verstehen seine Worte nicht.«
\bibleverse{19} Jesus merkte, daß sie ihn darüber befragen wollten, und
sagte zu ihnen: »Darüber verhandelt ihr miteinander, (was das zu
bedeuten habe) daß ich gesagt habe: ›Nur noch eine kurze Zeit, so seht
ihr mich nicht mehr, dann wieder eine kurze Zeit, so werdet ihr mich
sehen‹? \bibleverse{20} Wahrlich, wahrlich ich sage euch: ihr werdet
weinen und wehklagen, die Welt aber wird sich freuen; ihr werdet traurig
sein, doch eure Traurigkeit wird zur Freude werden. \bibleverse{21} Wenn
eine Frau Mutter werden soll, so ist sie traurig, weil ihre Stunde
gekommen ist; wenn sie aber das Kind geboren hat, denkt sie nicht mehr
an die Angst um der Freude willen, daß ein Mensch in die Welt geboren
ist. \bibleverse{22} So seid auch ihr jetzt in Traurigkeit; aber ich
werde euch wiedersehen: dann wird euer Herz sich freuen, und niemand
wird euch eure Freude rauben. \bibleverse{23} Und an jenem Tage werdet
ihr mich um nichts mehr befragen. Wahrlich, wahrlich ich sage euch: Wenn
ihr den Vater um etwas bitten werdet, so wird er es euch in meinem Namen
geben. \bibleverse{24} Bisher habt ihr noch nie um etwas in meinem Namen
gebeten: bittet, so werdet ihr empfangen, damit eure Freude vollkommen
sei.«

\hypertarget{ff-verheiuxdfung-der-vollendung-der-gottesgemeinschaft-fuxfcr-die-juxfcnger-abschluuxdf-der-abschiedsreden}{%
\subparagraph{ff) Verheißung der Vollendung der Gottesgemeinschaft für
die Jünger; Abschluß der
Abschiedsreden}\label{ff-verheiuxdfung-der-vollendung-der-gottesgemeinschaft-fuxfcr-die-juxfcnger-abschluuxdf-der-abschiedsreden}}

\bibleverse{25} »Dies habe ich euch in Gleichnissen\textless sup
title=``oder: bildlichen Reden''\textgreater✲ verkündet; es kommt aber
die Stunde, da werde ich nicht mehr in Gleichnissen\textless sup
title=``oder: Bildern''\textgreater✲ zu euch reden, sondern euch mit
voller Offenheit Kunde über den Vater geben. \bibleverse{26} An jenem
Tage werdet ihr in meinem Namen bitten, und ich sage euch nicht, daß ich
den Vater für euch bitten werde; \bibleverse{27} denn er selbst, der
Vater, hat euch lieb, weil ihr mich geliebt und den Glauben gewonnen
habt, daß ich von Gott ausgegangen bin. \bibleverse{28} Ich bin vom
Vater ausgegangen und in die Welt gekommen; hinwiederum verlasse ich die
Welt und kehre zum Vater zurück.«~-- \bibleverse{29} Da sagten seine
Jünger: »Siehe, jetzt redest du frei heraus und gebrauchst keine
bildliche Rede mehr; \bibleverse{30} jetzt wissen wir, daß du alles
weißt und niemand dich erst zu befragen braucht; darum glauben wir, daß
du von Gott ausgegangen bist.« \bibleverse{31} Jesus antwortete ihnen:
»Jetzt glaubt ihr? \bibleverse{32} Wisset wohl: es kommt die Stunde, ja
sie ist schon da, daß ihr euch zerstreuen werdet, ein jeder in das
Seine\textless sup title=``=~in seinen Wohnort''\textgreater✲, und ihr
mich allein lassen werdet. Und doch bin ich (alsdann) nicht allein, denn
der Vater ist bei mir. \bibleverse{33} Dies habe ich zu euch geredet,
damit ihr in mir Frieden habet. In der Welt habt ihr
Bedrängnis\textless sup title=``oder: Not, Angst''\textgreater✲; doch
seid getrost: ich habe die Welt überwunden!«

\hypertarget{d-das-hohepriesterliche-abschiedsgebet-jesu-mit-den-seinen-und-fuxfcr-die-seinen}{%
\paragraph{d) Das (hohepriesterliche) Abschiedsgebet Jesu mit den Seinen
und für die
Seinen}\label{d-das-hohepriesterliche-abschiedsgebet-jesu-mit-den-seinen-und-fuxfcr-die-seinen}}

\hypertarget{aa-jesu-gebet-fuxfcr-sich-selbst-um-seine-verherrlichung-nach-vollendung-seines-werkes}{%
\subparagraph{aa) Jesu Gebet für sich selbst (um seine Verherrlichung
nach Vollendung seines
Werkes)}\label{aa-jesu-gebet-fuxfcr-sich-selbst-um-seine-verherrlichung-nach-vollendung-seines-werkes}}

\hypertarget{section-16}{%
\section{17}\label{section-16}}

\bibleverse{1} So redete Jesus; dann richtete er seine Augen zum Himmel
empor und betete: »Vater, die Stunde ist gekommen: verherrliche deinen
Sohn, damit der Sohn dich verherrliche! \bibleverse{2} Du hast ihm ja
Macht über alles Fleisch\textless sup title=``=~über die ganze
Menschheit''\textgreater✲ verliehen, damit er allen, die du ihm gegeben
hast, ewiges Leben gebe. \bibleverse{3} Darin besteht aber das ewige
Leben, daß sie dich, den allein wahren Gott, und den du gesandt hast,
Jesus Christus, erkennen. \bibleverse{4} Ich habe dich hier auf der Erde
verherrlicht und habe das Werk vollendet, dessen Vollführung du mir
aufgetragen hast. \bibleverse{5} Und jetzt verherrliche du mich, Vater,
bei dir selbst mit der Herrlichkeit, die ich bei dir besaß, ehe die Welt
war.«

\hypertarget{bb-fuxfcrbitte-jesu-fuxfcr-die-erhaltung-der-juxfcnger-in-der-rechten-gotteserkenntnis}{%
\subparagraph{bb) Fürbitte Jesu für die Erhaltung der Jünger in der
rechten
Gotteserkenntnis}\label{bb-fuxfcrbitte-jesu-fuxfcr-die-erhaltung-der-juxfcnger-in-der-rechten-gotteserkenntnis}}

\bibleverse{6} »Ich habe deinen Namen den Menschen geoffenbart, die du
mir aus der Welt gegeben hast. Dir gehörten sie an, und mir hast du sie
gegeben, und sie haben dein Wort bewahrt✲. \bibleverse{7} Jetzt haben
sie erkannt, daß alles, was du mir gegeben hast, von dir stammt;
\bibleverse{8} denn die Worte, die du mir gegeben hast, habe ich ihnen
gegeben, und sie haben sie angenommen und haben in Wahrheit erkannt, daß
ich von dir ausgegangen bin, und haben den Glauben gewonnen, daß du es
bist, der mich gesandt hat.

\bibleverse{9} Ich bitte für sie; nicht für die Welt bitte ich, sondern
für die, welche du mir gegeben hast; denn sie sind dein Eigentum,
\bibleverse{10} und was mein ist, ist ja alles dein, und was dein ist,
das ist mein, und ich bin in ihnen verherrlicht worden. \bibleverse{11}
Und ich bin nicht mehr in der Welt, doch sie sind✲ noch in der Welt,
während ich zu dir gehe. Heiliger Vater, erhalte sie in\textless sup
title=``oder: bei''\textgreater✲ deinem Namen, den du mir
anvertraut\textless sup title=``oder: kundzutun verliehen''\textgreater✲
hast, damit sie eins seien, so wie wir es sind. \bibleverse{12} Solange
ich in ihrer Mitte gewesen bin, habe ich sie, die du mir gegeben hast,
in\textless sup title=``oder: bei''\textgreater✲ deinem Namen erhalten
und habe sie behütet, und keiner von ihnen ist verlorengegangen außer
dem Sohne des Verderbens, damit die Schrift erfüllt würde\textless sup
title=``Ps 41,10''\textgreater✲. \bibleverse{13} Jetzt aber gehe ich zu
dir und rede dieses noch in der Welt, damit sie die Freude, wie ich sie
habe, vollkommen in sich tragen. \bibleverse{14} Ich habe ihnen dein
Wort gegeben, und die Welt hat sie gehaßt, weil sie nicht zur Welt
gehören, wie auch ich nicht der Welt angehöre. \bibleverse{15} Ich bitte
dich nicht, sie aus der Welt hinwegzunehmen, sondern sie vor dem Bösen
zu behüten. \bibleverse{16} Sie gehören nicht zur Welt, wie auch ich
nicht der Welt angehöre. \bibleverse{17} Heilige sie in deiner Wahrheit:
dein Wort ist Wahrheit. \bibleverse{18} Wie du mich in die Welt gesandt
hast, so habe auch ich sie in die Welt gesandt; \bibleverse{19} und für
sie heilige ich mich, damit auch sie in Wahrheit\textless sup
title=``oder: wahrhaft''\textgreater✲ geheiligt seien.«

\hypertarget{cc-fuxfcrbitte-fuxfcr-alle-gluxe4ubigen-oder-fuxfcr-die-ganze-gemeinde-aller-zeiten-und-aller-orte}{%
\subparagraph{cc) Fürbitte für alle Gläubigen (oder: für die ganze
Gemeinde aller Zeiten und aller
Orte)}\label{cc-fuxfcrbitte-fuxfcr-alle-gluxe4ubigen-oder-fuxfcr-die-ganze-gemeinde-aller-zeiten-und-aller-orte}}

\bibleverse{20} »Ich bitte aber nicht für diese allein, sondern auch für
die, welche durch ihr Wort zum Glauben an mich kommen (werden),
\bibleverse{21} daß sie alle eins seien; wie du, Vater, in mir bist und
ich in dir bin, so laß auch sie in uns eins sein, damit die Welt glaube,
daß du mich gesandt hast. \bibleverse{22} Ich habe auch die
Herrlichkeit, die du mir gegeben hast, ihnen gegeben, damit sie eins
seien, wie wir eins sind: \bibleverse{23} ich in ihnen und du in mir,
auf daß sie zu vollkommener Einheit gelangen, damit die Welt erkenne,
daß du mich gesandt und sie geliebt hast, wie du mich geliebt hast.
\bibleverse{24} Vater, ich will, daß da, wo ich bin, auch die bei mir
seien, die du mir gegeben hast, damit sie meine Herrlichkeit sehen, die
du mir verliehen hast; denn du hast mich schon vor der Grundlegung der
Welt geliebt. \bibleverse{25} Gerechter Vater, die Welt hat dich nicht
erkannt, ich aber habe dich erkannt, und diese haben erkannt, daß du
mich gesandt hast. \bibleverse{26} Und ich habe ihnen deinen Namen
kundgetan und werde ihn (auch weiterhin) kundtun, damit die Liebe, mit
der du mich geliebt hast, in ihnen sei und ich in ihnen.«

\hypertarget{iv.-jesu-leiden-und-tod-kap.-18-19}{%
\subsection{IV. Jesu Leiden und Tod (Kap.
18-19)}\label{iv.-jesu-leiden-und-tod-kap.-18-19}}

\hypertarget{jesus-in-gethsemane-judas-malchus-gefangennahme-jesu}{%
\subsubsection{1. Jesus in Gethsemane: Judas, Malchus, Gefangennahme
Jesu}\label{jesus-in-gethsemane-judas-malchus-gefangennahme-jesu}}

\hypertarget{section-17}{%
\section{18}\label{section-17}}

\bibleverse{1} Nachdem Jesus so gebetet hatte, ging er mit seinen
Jüngern (aus der Stadt) hinaus über den Bach Kidron hinüber an einen
Ort, wo ein Garten war, in den er mit seinen Jüngern eintrat.
\bibleverse{2} Aber auch Judas, sein Verräter, kannte diesen Ort, weil
Jesus dort oft mit seinen Jüngern zusammengekommen war. \bibleverse{3}
Nachdem nun Judas die Abteilung\textless sup title=``oder: eine
Schar''\textgreater✲ Soldaten und von den Hohenpriestern und Pharisäern
Diener erhalten hatte, kam er mit Fackeln, Laternen und Waffen dorthin.
\bibleverse{4} Wiewohl nun Jesus alles wußte, was über ihn kommen würde,
trat er doch (aus dem Garten) hinaus und fragte sie: »»»Wen sucht ihr?«
\bibleverse{5} Sie antworteten ihm: »Jesus von Nazareth.« Er sagte zu
ihnen: »Der bin ich.« Auch Judas, sein Verräter, stand bei ihnen.
\bibleverse{6} Als Jesus nun zu ihnen sagte: »Der bin ich!«, wichen sie
zurück und fielen zu Boden. \bibleverse{7} Da fragte er sie nochmals:
»Wen sucht ihr?« Sie sagten: »Jesus von Nazareth.« \bibleverse{8} Jesus
antwortete: »Ich habe euch gesagt, daß ich es bin. Wenn ihr also mich
sucht, so laßt diese hier gehen!« \bibleverse{9} So sollte sich das Wort
erfüllen, das er ausgesprochen hatte\textless sup title=``vgl.
17,12''\textgreater✲: »Ich habe keinen von denen, die du mir gegeben
hast, verloren gehen lassen.« \bibleverse{10} Da nun Simon Petrus ein
Schwert bei sich hatte, zog er es heraus, schlug damit nach dem Knechte
des Hohenpriesters und hieb ihm das rechte Ohr ab; der Knecht hieß
Malchus. \bibleverse{11} Da sagte Jesus zu Petrus: »Stecke das Schwert
in die Scheide! Soll ich den Kelch nicht trinken, den mir der Vater
gereicht hat?«

\bibleverse{12} Hierauf nahmen die Abteilung Soldaten mit ihrem
Hauptmann und die Diener der Juden Jesus fest, fesselten ihn
\bibleverse{13} und führten ihn zunächst zu Hannas ab; dieser war
nämlich der Schwiegervater des Kaiphas, der in jenem Jahre Hoherpriester
war. \bibleverse{14} Kaiphas aber war es, der den Juden den Rat gegeben
hatte, es sei besser\textless sup title=``oder: am
besten''\textgreater✲, daß ein einzelner Mensch für das Volk
sterbe\textless sup title=``vgl. 11,50''\textgreater✲.

\hypertarget{verleugnung-des-petrus-jesu-verhuxf6r-vor-dem-juxfcdischen-gericht}{%
\subsubsection{2. Verleugnung des Petrus; Jesu Verhör vor dem jüdischen
Gericht}\label{verleugnung-des-petrus-jesu-verhuxf6r-vor-dem-juxfcdischen-gericht}}

\hypertarget{a-erste-verleugnung-des-petrus}{%
\paragraph{a) Erste Verleugnung des
Petrus}\label{a-erste-verleugnung-des-petrus}}

\bibleverse{15} Simon Petrus aber und noch ein anderer Jünger waren
Jesus nachgefolgt. Dieser (andere) Jünger war aber mit dem
Hohenpriester\textless sup title=``=~im Hause des
Hohenpriesters''\textgreater✲ bekannt und ging (deshalb) gleichzeitig
mit Jesus in den Palast\textless sup title=``oder: Hof''\textgreater✲
des Hohenpriesters hinein, \bibleverse{16} während Petrus draußen vor
der Tür stehenblieb. Da ging der andere Jünger, der mit dem
Hohenpriester bekannt war, hinaus, redete mit der Türhüterin und führte
Petrus hinein. \bibleverse{17} Da sagte die Magd, welche die Tür hütete,
zu Petrus: »Gehörst du nicht auch zu den Jüngern dieses Menschen?« Er
antwortete: »Nein.« \bibleverse{18} Es standen aber die Knechte und
Diener da, hatten sich wegen der Kälte ein Kohlenfeuer angemacht und
wärmten sich daran; aber auch Petrus stand bei ihnen und wärmte sich.

\hypertarget{b-jesus-vor-den-hohenpriestern-hannas-und-kaiphas}{%
\paragraph{b) Jesus vor den Hohenpriestern Hannas und
Kaiphas}\label{b-jesus-vor-den-hohenpriestern-hannas-und-kaiphas}}

\bibleverse{19} Der Hohepriester (Hannas) befragte nun Jesus über seine
Jünger und seine Lehre. \bibleverse{20} Jesus antwortete ihm: »Ich habe
frei und offen zu aller Welt geredet; ich habe allezeit in den Synagogen
und im Tempel gelehrt, wo alle Juden zusammenkommen; im geheimen habe
ich überhaupt nicht geredet. \bibleverse{21} Warum fragst du mich? Frage
die, welche gehört haben, was ich zu ihnen geredet habe; diese wissen,
was ich gesagt habe.« \bibleverse{22} Als er das ausgesprochen hatte,
gab einer von den Dienern, der dabeistand, Jesus einen Schlag ins
Gesicht und sagte: »So antwortest du dem Hohenpriester?« \bibleverse{23}
Jesus entgegnete ihm: »Wenn ich ungehörig gesprochen habe, so gib an,
was ungehörig daran gewesen ist; wenn ich aber richtig gesprochen habe,
warum schlägst du mich?« \bibleverse{24} Darauf sandte Hannas ihn
gefesselt zum Hohenpriester Kaiphas.

\hypertarget{c-zweite-und-dritte-verleugnung-des-petrus}{%
\paragraph{c) Zweite und dritte Verleugnung des
Petrus}\label{c-zweite-und-dritte-verleugnung-des-petrus}}

\bibleverse{25} Simon Petrus aber stand (unterdessen) da und wärmte
sich. Da fragten sie ihn: »Gehörst du nicht auch zu seinen Jüngern?«
\bibleverse{26} Er leugnete aber mit einem »Nein«. Da sagte einer von
den Knechten des Hohenpriesters, ein Verwandter des Knechtes, dem Petrus
das Ohr abgehauen hatte: »Habe ich dich nicht in dem Garten bei ihm
gesehen?« \bibleverse{27} Da leugnete Petrus nochmals; und sogleich
darauf krähte der Hahn.

\hypertarget{jesu-verhuxf6r-und-bekenntnis-vor-dem-ruxf6mischen-statthalter-pilatus-seine-geiuxdfelung-verspottung-und-verurteilung}{%
\subsubsection{3. Jesu Verhör und Bekenntnis vor dem römischen
Statthalter Pilatus; seine Geißelung, Verspottung und
Verurteilung}\label{jesu-verhuxf6r-und-bekenntnis-vor-dem-ruxf6mischen-statthalter-pilatus-seine-geiuxdfelung-verspottung-und-verurteilung}}

\bibleverse{28} Man führte Jesus dann aus dem Hause des Kaiphas nach der
Statthalterei; es war früh am Morgen. Die Juden selbst gingen dabei
nicht in die Statthalterei hinein, um nicht unrein zu werden, sondern
das Passah essen zu können. \bibleverse{29} Darum kam Pilatus zu ihnen
hinaus und fragte sie: »Welche Anklage habt ihr gegen diesen Mann zu
erheben?« \bibleverse{30} Sie antworteten ihm mit den Worten: »Wenn
dieser Mensch kein Verbrecher wäre, so hätten wir ihn dir nicht
überliefert!« \bibleverse{31} Da sagte Pilatus zu ihnen: »Nehmt ihr ihn
und richtet ihn nach eurem Gesetz.« Da entgegneten ihm die Juden: »Wir
haben nicht das Recht, jemand hinzurichten«~-- \bibleverse{32} so sollte
sich das Wort Jesu erfüllen, durch das er die Art seines Todes
angedeutet hatte. \bibleverse{33} Pilatus ging nun wieder in die
Statthalterei hinein, ließ Jesus rufen und fragte ihn: »Bist du der
König der Juden?« \bibleverse{34} Jesus antwortete: »Fragst du so von
dir selbst aus, oder haben andere es dir von mir gesagt?«
\bibleverse{35} Pilatus antwortete: »Ich bin doch kein Jude! Dein Volk
und zwar die Hohenpriester haben dich mir überantwortet: was hast du
verbrochen?« \bibleverse{36} Jesus antwortete: »Mein Reich\textless sup
title=``=~mein Königtum''\textgreater✲ ist nicht von dieser Welt. Wäre
mein Reich von dieser Welt, so würden meine Diener (für mich) kämpfen,
damit ich den Juden nicht überliefert würde; nun aber ist mein Reich
nicht von hier\textless sup title=``oder: derart''\textgreater✲.«
\bibleverse{37} Da sagte Pilatus zu ihm: »Ein König bist du also?« Jesus
antwortete: »Ja, ich bin ein König. Ich bin dazu geboren und dazu in die
Welt gekommen, um für die Wahrheit Zeugnis abzulegen; jeder, der aus der
Wahrheit ist, hört auf meine Stimme.« \bibleverse{38} Darauf antwortete
ihm Pilatus: »Was ist Wahrheit?!«

Nach diesen Worten ging er wieder zu den Juden hinaus und sagte zu
ihnen: »Ich finde keinerlei Schuld an ihm. \bibleverse{39} Es ist aber
herkömmlich bei euch, daß ich euch am Passah einen (Gefangenen)
freigebe: soll ich euch also den König der Juden freigeben?«
\bibleverse{40} Da riefen sie wieder laut: »Nein, nicht diesen, sondern
den Barabbas!« Barabbas war aber ein Räuber✲.

\hypertarget{section-18}{%
\section{19}\label{section-18}}

\bibleverse{1} Da ließ nun Pilatus Jesus ergreifen und geißeln;
\bibleverse{2} dann flochten die Soldaten eine Dornenkrone, setzten sie
ihm aufs Haupt und legten ihm einen scharlachroten Mantel um;
\bibleverse{3} hierauf traten sie vor ihn hin und riefen aus: »Sei
gegrüßt, Judenkönig!« und versetzten ihm Schläge ins Gesicht.
\bibleverse{4} Pilatus kam dann wieder heraus und sagte zu ihnen: »Seht,
ich führe ihn zu euch heraus, damit ihr erkennt, daß ich keinerlei
Schuld an ihm finde.« \bibleverse{5} So kam denn Jesus heraus, indem er
die Dornenkrone und den Purpurmantel trug, und Pilatus sagte zu ihnen:
»Seht, der Mensch\textless sup title=``oder: welch ein
Mensch''\textgreater✲!« \bibleverse{6} Als ihn nun die Hohenpriester und
die Tempeldiener erblickten, schrien sie: »Ans Kreuz mit ihm, ans
Kreuz!« Pilatus entgegnete ihnen: »Nehmt ihr ihn und kreuzigt ihn! Denn
ich finde keine Schuld an ihm.« \bibleverse{7} Die Juden antworteten
ihm: »Wir haben ein Gesetz, und nach diesem Gesetz muß er sterben, weil
er sich selbst zu Gottes Sohn gemacht hat.« \bibleverse{8} Als nun
Pilatus dies Wort hörte, geriet er in noch größere Angst; \bibleverse{9}
er ging also wieder in die Statthalterei hinein und fragte Jesus: »Woher
bist du?« Jesus aber gab ihm keine Antwort. \bibleverse{10} Da sagte
Pilatus zu ihm: »Mir willst du nicht Rede stehen? Weißt du nicht, daß
ich die Macht habe, dich freizugeben, und auch die Macht habe, dich
kreuzigen zu lassen?« \bibleverse{11} Jesus antwortete ihm: »Du hättest
keine Macht über mich, wenn sie dir nicht von oben her gegeben wäre;
deshalb trifft den, welcher mich dir ausgeliefert hat, eine größere
Schuld.« \bibleverse{12} Von da an\textless sup title=``oder: aus diesem
Grunde''\textgreater✲ suchte Pilatus ihn freizugeben; aber die Juden
schrien: »Gibst du diesen frei, so bist du kein Freund des Kaisers!
Jeder, der sich selbst zum König macht, lehnt sich gegen den Kaiser
auf!« \bibleverse{13} Als Pilatus diese Worte hörte, ließ er Jesus
hinausführen und setzte sich auf den Richterstuhl an dem Platze, welcher
›Steinpflaster‹, auf hebräisch Gabbatha, heißt. \bibleverse{14} Es war
aber der Rüsttag✲ auf das Passahfest, und zwar um die sechste Stunde.
Nun sagte Pilatus zu den Juden: »Seht, da ist euer König!«
\bibleverse{15} Da schrien jene: »Weg, weg mit ihm, kreuzige ihn!«
Pilatus entgegnete ihnen: »Euren König soll ich kreuzigen lassen?« Die
Hohenpriester antworteten: »Wir haben keinen König als den Kaiser!«
\bibleverse{16} a Darauf übergab er ihnen Jesus zur Kreuzigung.

\hypertarget{jesu-kreuzigung-und-tod}{%
\subsubsection{4. Jesu Kreuzigung und
Tod}\label{jesu-kreuzigung-und-tod}}

b So übernahmen sie\textless sup title=``d.h. die
Soldaten''\textgreater✲ denn Jesus; \bibleverse{17} und dieser ging,
indem er sein Kreuz selber trug, (aus der Stadt) hinaus nach der
sogenannten ›Schädelstätte‹, die auf hebräisch Golgatha heißt;
\bibleverse{18} dort kreuzigten sie ihn und mit ihm noch zwei andere auf
beiden Seiten, Jesus aber in der Mitte. \bibleverse{19} Auch eine
Aufschrift hatte Pilatus schreiben und oben am Kreuz anbringen lassen;
sie lautete: »Jesus von Nazareth, der König der Juden.« \bibleverse{20}
Diese Aufschrift nun lasen viele von den Juden, weil der Platz, wo Jesus
gekreuzigt wurde, nahe bei der Stadt lag und die Aufschrift in
hebräischer, römischer✲ und griechischer Sprache abgefaßt war.
\bibleverse{21} Da sagten die Hohenpriester der Juden zu Pilatus:
»Schreibe nicht: ›Der König der Juden‹, sondern: ›Dieser Mensch hat
behauptet, er sei der König der Juden‹!« \bibleverse{22} Pilatus (aber)
antwortete: »Was ich geschrieben habe, das habe ich geschrieben!«

\bibleverse{23} Als nun die Soldaten Jesus gekreuzigt hatten, nahmen sie
seine Kleidungsstücke und machten vier Teile daraus, für jeden Soldaten
einen Teil, außerdem noch das Unterkleid. Dieses Unterkleid war aber
ohne Naht, von oben an in einem Stück gewebt\textless sup title=``oder:
gestrickt''\textgreater✲. \bibleverse{24} Da sagten sie zueinander: »Wir
wollen es nicht zerschneiden, sondern darum losen, wem es gehören soll«
-- so sollte das Schriftwort seine Erfüllung finden\textless sup
title=``Ps 22,19''\textgreater✲: »Sie haben meine Kleider unter sich
verteilt und über mein Gewand das Los geworfen.« Auf diese Weise
verfuhren also die Soldaten.

\bibleverse{25} Es standen aber beim Kreuze Jesu seine Mutter und die
Schwester seiner Mutter, auch Maria, die Frau des Klopas, und Maria von
Magdala. \bibleverse{26} Als nun Jesus seine Mutter und neben ihr den
Jünger, den er (besonders) lieb hatte, stehen sah, sagte er zu seiner
Mutter: »Frau, siehe dein Sohn!« \bibleverse{27} Darauf sagte er zu dem
Jünger: »Siehe deine Mutter!« Und von dieser Stunde an nahm der Jünger
sie zu sich in sein Haus.

\bibleverse{28} Darauf, weil Jesus wußte, daß nunmehr alles vollbracht
war, sagte er, damit die Schrift ganz erfüllt würde\textless sup
title=``vgl. Ps 69,22; 22,16''\textgreater✲: »Mich dürstet.«
\bibleverse{29} Es stand dort nun ein mit Essig gefülltes
Gefäß\textless sup title=``vgl. Lk 23,36''\textgreater✲. Sie umwickelten
also einen mit dem Essig getränkten Schwamm mit Ysop und hielten ihm
diesen an den Mund. \bibleverse{30} Als nun Jesus den Essig genommen
hatte, sagte er: »Es ist vollbracht!«, neigte dann das Haupt und gab den
Geist auf.

\bibleverse{31} Weil es nun Rüsttag\textless sup title=``d.h.
Freitag''\textgreater✲ war, trugen die Juden, damit die Leichen nicht
während des Sabbats am Kreuz blieben -- dieser Sabbattag war nämlich ein
hoher Festtag --, dem Pilatus die Bitte vor, es möchten
ihnen\textless sup title=``d.h. den Gekreuzigten''\textgreater✲ die
Schenkel mit Keulen zerschlagen und sie dann (vom Kreuz) herabgenommen
werden. \bibleverse{32} So kamen denn die Soldaten und zerschlugen dem
ersten die Schenkel, ebenso auch dem andern, der mit (Jesus) gekreuzigt
worden war. \bibleverse{33} Als sie aber zu Jesus kamen und sahen, daß
er bereits tot war, zerschlugen sie ihm die Schenkel nicht,
\bibleverse{34} sondern einer von den Soldaten stieß ihn mit seiner
Lanze in die Seite; da floß sogleich Blut und Wasser heraus.
\bibleverse{35} Ein Augenzeuge hat dies bezeugt\textless sup
title=``=~mit Bestimmtheit ausgesagt''\textgreater✲, und sein Zeugnis
ist zuverlässig, und jener\textless sup title=``d.h. der
Betreffende''\textgreater✲ weiß, daß er die Wahrheit sagt, damit auch
ihr zum Glauben kommet. \bibleverse{36} Dies ist nämlich geschehen,
damit das Schriftwort erfüllt würde\textless sup title=``2.Mose 12,46;
Ps 34,21''\textgreater✲: »Es soll kein Knochen an ihm zerbrochen
werden.« \bibleverse{37} Und noch eine andere Schriftstelle
lautet\textless sup title=``Sach 12,10''\textgreater✲: »Sie werden auf
den blicken, den sie durchbohrt haben.«

\hypertarget{kreuzabnahme-und-grablegung-jesu}{%
\subsubsection{5. Kreuzabnahme und Grablegung
Jesu}\label{kreuzabnahme-und-grablegung-jesu}}

\bibleverse{38} Hierauf trug Joseph von Arimathäa, der ein Jünger Jesu
war -- allerdings war er's nur im geheimen aus Furcht vor den Juden --,
dem Pilatus die Bitte vor, daß er den Leichnam Jesu vom Kreuze abnehmen
dürfe; und Pilatus gewährte ihm die Bitte. So ging er denn hin und nahm
seinen Leichnam (vom Kreuz) ab. \bibleverse{39} Aber auch Nikodemus kam,
derselbe, der zum erstenmal bei Nacht zu Jesus gekommen war✲, und
brachte eine Mischung von Myrrhe und Aloe mit, wohl hundert Pfund.
\bibleverse{40} So nahmen sie denn den Leib Jesu und banden ihn ein in
Leinwandstreifen mitsamt den wohlriechenden Stoffen, wie es Sitte der
Juden bei Bestattungen ist. \bibleverse{41} Es lag aber bei dem Platze,
wo er gekreuzigt worden war, ein Garten, und in dem Garten (befand sich)
ein neues Grab, in welchem bisher noch niemand beigesetzt worden war.
\bibleverse{42} Dorthin brachten sie nun Jesus mit Rücksicht auf den
jüdischen Rüsttag, weil das Grab sich in der Nähe befand.

\hypertarget{v.-die-offenbarungen-des-auferstandenen-kap.-20-21}{%
\subsection{V. Die Offenbarungen des Auferstandenen (Kap.
20-21)}\label{v.-die-offenbarungen-des-auferstandenen-kap.-20-21}}

\hypertarget{jesu-auferstehung}{%
\subsubsection{1. Jesu Auferstehung}\label{jesu-auferstehung}}

\hypertarget{a-maria-von-magdala-und-das-leere-grab-petrus-und-johannes-am-grabe}{%
\paragraph{a) Maria von Magdala und das leere Grab; Petrus und Johannes
am
Grabe}\label{a-maria-von-magdala-und-das-leere-grab-petrus-und-johannes-am-grabe}}

\hypertarget{section-19}{%
\section{20}\label{section-19}}

\bibleverse{1} Am ersten Tage nach dem Sabbat\textless sup title=``oder:
am ersten Tage der Woche''\textgreater✲ aber ging Maria Magdalena
frühmorgens, als es noch dunkel war, zum Grabe hin und sah, daß der
Stein vom Grabe weggenommen war. \bibleverse{2} Da eilte sie hin und kam
zu Simon Petrus und zu dem anderen Jünger, den Jesus (besonders) lieb
gehabt hatte, und sagte zu ihnen: »Man hat den Herrn aus dem Grabe
weggenommen, und wir wissen nicht, wohin man ihn gelegt hat!«
\bibleverse{3} Da gingen Petrus und der andere Jünger hinaus und machten
sich auf den Weg zum Grabe. \bibleverse{4} Die beiden liefen
miteinander, doch der andere Jünger lief voraus, schneller als Petrus,
und kam zuerst an das Grab. \bibleverse{5} Als er sich nun hineinbeugte,
sah er die leinenen Binden daliegen, ging jedoch nicht hinein.
\bibleverse{6} Nun kam auch Simon Petrus hinter ihm her und trat in das
Grab hinein; er sah dort die leinenen Binden liegen, \bibleverse{7} das
Schweißtuch aber, das auf seinem Kopf gelegen hatte, lag nicht bei den
(anderen) Leintüchern, sondern für sich zusammengefaltet\textless sup
title=``oder: aufgewickelt''\textgreater✲ an einer besonderen Stelle.
\bibleverse{8} Jetzt trat auch der andere Jünger hinein, der zuerst am
Grabe angekommen war, und sah es auch und kam zum Glauben;
\bibleverse{9} denn sie hatten die Schrift noch nicht verstanden, daß er
von den Toten auferstehen müsse. \bibleverse{10} So gingen denn die
(beiden) Jünger wieder heim.

\hypertarget{b-jesu-erscheinung-vor-maria-von-magdala}{%
\paragraph{b) Jesu Erscheinung vor Maria von
Magdala}\label{b-jesu-erscheinung-vor-maria-von-magdala}}

\bibleverse{11} Maria aber war draußen am Grabe stehengeblieben und
weinte. Mit Tränen in den Augen beugte sie sich vor in das Grab hinein;
\bibleverse{12} da sah sie dort zwei Engel in weißen Gewändern dasitzen,
den einen am Kopfende, den andern am Fußende der Stelle, wo der Leichnam
Jesu gelegen hatte. \bibleverse{13} Diese sagten zu ihr: »Frau, warum
weinst du?« Sie antwortete ihnen: »Man hat meinen Herrn weggenommen, und
ich weiß nicht, wohin man ihn gelegt hat.« \bibleverse{14} Nach diesen
Worten wandte sie sich um und sah Jesus dastehen, wußte aber nicht, daß
es Jesus war. \bibleverse{15} Da sagte Jesus zu ihr: »Frau, warum weinst
du? Wen suchst du?« Sie hielt ihn für den Hüter des Gartens und sagte zu
ihm: »Herr, wenn du ihn weggetragen hast, so sage mir doch, wohin du ihn
gebracht hast; dann will ich ihn wieder holen.« \bibleverse{16} Jesus
sagte zu ihr: »Maria!« Da wandte sie sich um und sagte auf hebräisch✲ zu
ihm: »Rabbuni!«, das heißt »Meister\textless sup title=``oder:
Lehrer''\textgreater✲«. \bibleverse{17} Jesus sagte zu ihr: »Rühre mich
nicht an, denn ich bin noch nicht zum Vater aufgefahren! Gehe aber zu
meinen Brüdern und sage ihnen: ›Ich fahre auf zu meinem Vater und eurem
Vater, zu meinem Gott und eurem Gott.‹« \bibleverse{18} Da ging Maria
Magdalena hin und verkündigte den Jüngern, sie habe den Herrn gesehen,
und er habe dies zu ihr gesagt\textless sup title=``oder: ihr
aufgetragen''\textgreater✲.

\hypertarget{jesus-und-die-juxfcnger-am-ostersonntagabend-des-thomas-unglaube-und-bekehrung}{%
\subsubsection{2. Jesus und die Jünger am Ostersonntagabend; des Thomas
Unglaube und
Bekehrung}\label{jesus-und-die-juxfcnger-am-ostersonntagabend-des-thomas-unglaube-und-bekehrung}}

\hypertarget{a-die-juxfcnger-ohne-thomas}{%
\paragraph{a) Die Jünger ohne
Thomas}\label{a-die-juxfcnger-ohne-thomas}}

\bibleverse{19} Als es nun an jenem Tage, dem ersten Wochentage, Abend
geworden war und die Türen an dem Ort, wo die Jünger sich befanden, aus
Furcht vor den Juden verschlossen waren, kam Jesus, trat mitten unter
sie und sagte zu ihnen: »Friede sei mit euch!« \bibleverse{20} Nach
diesen Worten zeigte er ihnen seine Hände und seine Seite; da freuten
sich die Jünger, weil\textless sup title=``oder: als''\textgreater✲ sie
den Herrn sahen. \bibleverse{21} Dann sagte er nochmals zu ihnen:
»Friede sei mit euch! Wie mich der Vater gesandt hat, so sende auch ich
euch.« \bibleverse{22} Nach diesen Worten hauchte er sie an und sagte zu
ihnen: »Empfanget heiligen Geist! \bibleverse{23} Wem immer ihr die
Sünden vergebt, dem sind sie vergeben, und wem ihr sie behaltet, dem
sind sie behalten.«

\hypertarget{b-die-juxfcnger-mit-thomas}{%
\paragraph{b) Die Jünger mit Thomas}\label{b-die-juxfcnger-mit-thomas}}

\bibleverse{24} Thomas aber, einer von den Zwölfen, der auch den Namen
›Zwilling‹ führt✲, war nicht bei ihnen gewesen, als Jesus gekommen war.
\bibleverse{25} Die anderen Jünger teilten ihm nun mit: »Wir haben den
Herrn gesehen!« Er aber erklärte ihnen: »Wenn ich nicht das Nägelmal in
seinen Händen sehe und meinen Finger in das Nägelmal und meine Hand in
seine Seite lege, werde ich es nimmermehr glauben!« \bibleverse{26} Acht
Tage später befanden sich seine Jünger wieder im Hause, und (diesmal)
war Thomas bei ihnen. Da kam Jesus bei verschlossenen Türen, trat mitten
unter sie und sagte: »Friede sei mit euch!« \bibleverse{27} Darauf sagte
er zu Thomas: »Reiche deinen Finger her\textless sup title=``oder: lege
deinen Finger hier auf diese Stelle''\textgreater✲ und sieh dir meine
Hände an; dann reiche deine Hand her und lege sie mir in die Seite und
sei nicht (länger) ungläubig, sondern werde gläubig!« \bibleverse{28} Da
antwortete ihm Thomas: »Mein Herr und mein Gott!« \bibleverse{29} Jesus
erwiderte ihm: »Weil du mich gesehen hast, bist du gläubig geworden.
Selig sind die, welche nicht gesehen haben und doch zum Glauben gekommen
sind!«

\hypertarget{schluuxdf-des-evangeliums}{%
\subsubsection{3. Schluß des
Evangeliums}\label{schluuxdf-des-evangeliums}}

\bibleverse{30} Noch viele andere Wunderzeichen hat Jesus vor den Augen
seiner Jünger getan, die in diesem Buche nicht aufgezeichnet stehen;
\bibleverse{31} diese aber sind niedergeschrieben worden, damit ihr
glaubt, daß Jesus der Gesalbte\textless sup title=``=~Christus, oder:
der Messias''\textgreater✲, der Sohn Gottes ist, und damit ihr durch den
Glauben Leben in seinem Namen habt.

\hypertarget{nachtrag}{%
\subsubsection{4. Nachtrag}\label{nachtrag}}

\hypertarget{a-jesus-offenbart-sich-seinen-juxfcngern-am-see-von-tiberias-der-wunderbare-fischzug}{%
\paragraph{a) Jesus offenbart sich seinen Jüngern am See von Tiberias;
der wunderbare
Fischzug}\label{a-jesus-offenbart-sich-seinen-juxfcngern-am-see-von-tiberias-der-wunderbare-fischzug}}

\hypertarget{section-20}{%
\section{21}\label{section-20}}

\bibleverse{1} Danach✲ offenbarte Jesus sich seinen Jüngern noch einmal
am See von Tiberias, und zwar offenbarte er sich auf folgende Weise:
\bibleverse{2} Es waren beisammen Simon Petrus und Thomas, der den Namen
›Zwilling‹ führt✲, Nathanael aus Kana in Galiläa, die (beiden) Söhne des
Zebedäus und noch zwei andere aus der Zahl seiner Jünger. \bibleverse{3}
Da sagte Simon Petrus zu ihnen: »Ich gehe hin und fische!« Sie
erwiderten ihm: »Dann gehen auch wir mit dir!« So gingen sie denn hinaus
und stiegen in das Boot, fingen aber in jener Nacht nichts.

\bibleverse{4} Als es bereits gegen Morgen war, stand Jesus am Ufer; die
Jünger wußten jedoch nicht, daß es Jesus war. \bibleverse{5} Da rief
Jesus ihnen zu: »Kinder, habt ihr nicht etwas (Fisch) als Zukost?« Sie
antworteten ihm: »Nein.« \bibleverse{6} Nun sagte er zu ihnen: »Werft
das Netz nach der rechten Seite des Bootes aus, so werdet ihr einen Fang
tun!« Da warfen sie es aus und konnten es vor der Menge der Fische nicht
mehr (aus dem Wasser) herausziehen. \bibleverse{7} Da sagte jener
Jünger, den Jesus (besonders) lieb hatte, zu Petrus: »Es ist der Herr!«
Als nun Simon Petrus hörte, daß es der Herr sei, gürtete er sich sein
Obergewand um -- er hatte nämlich nur ein Unterkleid angehabt -- und
sprang in den See; \bibleverse{8} die anderen Jünger aber kamen mit dem
Boote hinter ihm her -- sie waren nämlich nicht weit vom Lande, sondern
nur in einer Entfernung von etwa zweihundert Ellen -- und zogen das Netz
mit den Fischen hinter sich her.

\bibleverse{9} Als sie dann ans Land ausgestiegen waren, sahen sie ein
Kohlenfeuer (am Boden) hergerichtet und Fische darauf gelegt und Brot
(daneben). \bibleverse{10} Jesus sagte zu ihnen: »Bringt noch einige von
den Fischen her, die ihr soeben gefangen habt!« \bibleverse{11} Da stieg
Simon Petrus (in das Boot) hinein und zog das Netz ans Land, das mit
hundertunddreiundfünfzig großen Fischen gefüllt war und trotz dieser
großen Zahl nicht zerriß. \bibleverse{12} Nun sagte Jesus zu ihnen:
»Kommt her und haltet das Frühmahl!« Keiner aber von den Jüngern wagte
die Frage an ihn zu richten: »Wer bist du?« Sie wußten ja, daß es der
Herr war. \bibleverse{13} Jesus trat nun hin, nahm das Brot und gab es
ihnen, ebenso auch die Fische. \bibleverse{14} Dies war nun schon das
dritte Mal, daß Jesus sich nach seiner Auferstehung von den Toten seinen
Jüngern offenbarte.

\hypertarget{b-petrus-wieder-in-sein-ober--hirtenamt-eingesetzt-weissagung-uxfcber-das-lebensende-des-petrus-und-des-lieblingsjuxfcngers}{%
\paragraph{b) Petrus wieder in sein (Ober-) Hirtenamt eingesetzt;
Weissagung über das Lebensende des Petrus und des
Lieblingsjüngers}\label{b-petrus-wieder-in-sein-ober--hirtenamt-eingesetzt-weissagung-uxfcber-das-lebensende-des-petrus-und-des-lieblingsjuxfcngers}}

\bibleverse{15} Als sie nun das Frühmahl gehalten hatten, sagte Jesus zu
Simon Petrus: »Simon, Sohn des Johannes, liebst du mich mehr als diese?«
Er antwortete ihm: »Ja, Herr, du weißt, daß ich dich lieb habe.« Da
sagte er zu ihm: »Weide meine Lämmer!« \bibleverse{16} Darauf fragte ihn
Jesus zum zweitenmal: »Simon, Sohn des Johannes, liebst du mich?« Er
antwortete ihm: »Ja, Herr, du weißt, daß ich dich lieb habe.« Da sagte
Jesus zu ihm: »Hüte meine Schafe!« \bibleverse{17} Zum drittenmal fragte
er ihn: »Simon, Sohn des Johannes, hast du mich lieb?« Da wurde Petrus
betrübt, weil er ihn zum drittenmal fragte: »Hast du mich lieb?«, und er
antwortete ihm: »Herr, du weißt alles; du weißt auch, daß ich dich lieb
habe.« Da sagte Jesus zu ihm: »Weide meine Schafe! \bibleverse{18}
Wahrlich, wahrlich ich sage dir: Als du noch jünger warst, hast du dir
dein Gewand selbst gegürtet und bist umhergegangen, wohin du wolltest;
wenn du aber alt geworden bist, wirst du deine Arme ausstrecken, und ein
anderer wird dich gürten und dich an eine Stätte führen, wohin du nicht
willst.« \bibleverse{19} Dies sagte er aber, um anzudeuten, durch was
für eine Todesart Petrus Gott verherrlichen würde. Nach diesen Worten
sagte er zu ihm: »Folge mir nach!«

\bibleverse{20} Als Petrus sich dann umwandte, sah er den Jünger, den
Jesus (besonders) liebhatte, hinter ihnen herkommen, denselben, der sich
auch beim Abendmahl an seine Brust gelehnt und gefragt hatte: »Herr, wer
ist's, der dich verrät?« \bibleverse{21} Als nun Petrus diesen sah,
fragte er Jesus: »Herr, was wird aber mit diesem werden?«
\bibleverse{22} Jesus antwortete ihm: »Wenn es mein Wille ist, daß er
bis zu meinem Kommen (am Leben) bleibt, was geht das dich an? Folge du
mir nach!« \bibleverse{23} So verbreitete sich denn diese Rede unter den
Brüdern: »Jener\textless sup title=``=~der betreffende''\textgreater✲
Jünger stirbt nicht.« Aber Jesus hatte zu ihm nicht gesagt: »Er stirbt
nicht«, sondern: »Wenn es mein Wille ist, daß er bis zu meinem Kommen
(am Leben) bleibt, was geht das dich an?«

\hypertarget{c-zeugnis-uxfcber-den-verfasser-des-buches-und-schluuxdf}{%
\paragraph{c) Zeugnis über den Verfasser des Buches und
Schluß}\label{c-zeugnis-uxfcber-den-verfasser-des-buches-und-schluuxdf}}

\bibleverse{24} Dies ist der Jünger, der von diesen Dingen Zeugnis
ablegt und auch diese Schrift verfaßt hat, und wir wissen, daß sein
Zeugnis wahr ist. \bibleverse{25} Es gibt aber noch vieles andere, was
Jesus getan hat; wollte man das alles im einzelnen aufschreiben, so
würde nach meiner Überzeugung die Welt die Bücher nicht fassen, die dann
zu schreiben wären.
