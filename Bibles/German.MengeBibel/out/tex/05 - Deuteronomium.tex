\hypertarget{das-fuxfcnfte-buch-mose}{%
\section{DAS FÜNFTE BUCH MOSE}\label{das-fuxfcnfte-buch-mose}}

\emph{(genannt das Deuteronomium, d.h. die zweite Gesetzgebung)}

\hypertarget{i.-einleitende-reden-nebst-anhang-11-443}{%
\subsection{I. Einleitende Reden (nebst Anhang)
(1,1-4,43)}\label{i.-einleitende-reden-nebst-anhang-11-443}}

\hypertarget{geschichtlicher-ruxfcckblick-auf-die-gnadenreiche-guxf6ttliche-fuxfchrung-israels-seit-dem-aufbruch-vom-horeb}{%
\subsubsection{1. Geschichtlicher Rückblick auf die gnadenreiche
göttliche Führung Israels seit dem Aufbruch vom
Horeb}\label{geschichtlicher-ruxfcckblick-auf-die-gnadenreiche-guxf6ttliche-fuxfchrung-israels-seit-dem-aufbruch-vom-horeb}}

\hypertarget{a-uxfcberschrift-und-vorbemerkung-uxfcber-ort-und-zeit-aller-folgenden-reden-moses-ruxfcckblick-auf-die-erlebnisse-in-der-wuxfcste}{%
\paragraph{a) Überschrift und Vorbemerkung über Ort und Zeit aller
folgenden Reden Moses; Rückblick auf die Erlebnisse in der
Wüste}\label{a-uxfcberschrift-und-vorbemerkung-uxfcber-ort-und-zeit-aller-folgenden-reden-moses-ruxfcckblick-auf-die-erlebnisse-in-der-wuxfcste}}

\hypertarget{section}{%
\section{1}\label{section}}

1Dies sind die Worte\textless sup title=``oder: Reden''\textgreater✲,
die Mose an ganz Israel gerichtet hat jenseits des Jordans in der Wüste,
in der Steppe, gegenüber Suph, zwischen Paran, Thophel, Laban, Hazeroth
und Disahab. 2Elf Tagereisen lang ist der Weg vom Horeb✲ nach dem
Gebirge Seir bis Kades-Barnea.~-- 3Im vierzigsten Jahre, am ersten Tage
des elften Monats, war es: da trug Mose den Israeliten alles genau so
vor, wie der HERR es ihm für sie geboten hatte, 4nachdem er den
Amoriterkönig Sihon, der in Hesbon wohnte✲, und Og, den König von Basan,
der in Astharoth wohnte, bei Edrei geschlagen hatte. 5Jenseits des
Jordans, im Lande der Moabiter, unternahm es Mose, die folgende
Unterweisung vorzutragen:

\hypertarget{aa-gottes-befehl-zum-aufbruch-vom-horeb-sinai}{%
\subparagraph{aa) Gottes Befehl zum Aufbruch vom Horeb
(Sinai)}\label{aa-gottes-befehl-zum-aufbruch-vom-horeb-sinai}}

6»Der HERR, unser Gott, hat am Horeb so zu uns gesprochen: ›Ihr habt nun
lange genug an diesem Berge verweilt; 7brecht jetzt auf und zieht
unverweilt\textless sup title=``oder: geradeswegs''\textgreater✲ nach
dem Bergland der Amoriter und zu all ihren Nachbarn, die in der
Jordanebene, in dem Bergland, in der Niederung, im Südland und an der
Meeresküste wohnen, (also) in das Land der Kanaanäer und nach dem
Libanon bis an den großen Strom, den Euphratstrom. 8Ich übergebe euch
hiermit das Land: zieht hinein und nehmt es in Besitz, das Land, dessen
Verleihung der HERR euren Vätern Abraham, Isaak und Jakob und ihrer
Nachkommenschaft nach ihnen eidlich zugesagt hat.‹«\textless sup
title=``vgl. 1.Mose 15,18''\textgreater✲

\hypertarget{bb-einsetzung-von-anfuxfchrern-und-obmuxe4nnern-zu-gehilfen-moses-und-anweisung-der-schon-vorhandenen-richter}{%
\subparagraph{bb) Einsetzung von Anführern und Obmännern zu Gehilfen
Moses und Anweisung der schon vorhandenen
Richter}\label{bb-einsetzung-von-anfuxfchrern-und-obmuxe4nnern-zu-gehilfen-moses-und-anweisung-der-schon-vorhandenen-richter}}

9»Damals sagte ich zu euch: ›Ich allein bin nicht imstande, (die Sorge
für) eure Angelegenheiten zu tragen. 10Der HERR, euer Gott, hat euch so
zahlreich werden lassen, daß ihr schon jetzt den Sternen des Himmels an
Menge gleichkommt. 11Der HERR, der Gott eurer Väter, möge eure jetzige
Zahl noch tausendmal größer werden lassen und euch so segnen, wie er
euch verheißen hat! 12Wie vermöchte ich aber allein die Last und Bürde
eurer Angelegenheiten und eurer Streitsachen zu tragen? 13Bringt aus
euren Stämmen weise, verständige und erfahrene Männer her, damit ich sie
zu Oberhäuptern\textless sup title=``oder: Hauptleuten''\textgreater✲
über euch einsetze!‹ 14Da antwortetet ihr mir: ›Der Vorschlag, den du
gemacht hast, eignet sich trefflich zur Ausführung.‹ 15Da nahm ich eure
Stammeshäupter, weise und erfahrene Männer, und setzte sie zu
Oberhäuptern über euch ein als Oberste über die Tausendschaften oder
Hundertschaften und als Anführer von je fünfzig oder je zehn und auch
als Obmänner für eure Stämme. 16Zu gleicher Zeit gab ich euren Richtern
folgende Anweisung: ›Wenn ihr eure Volksgenossen bei Streitsachen
verhört, so entscheidet mit Gerechtigkeit, mag jemand mit einem von
seinen Volksgenossen oder mit einem bei ihm wohnenden Fremdling einen
Streit haben! 17Ihr dürft beim Rechtsprechen die Person nicht ansehen:
den Niedrigsten müßt ihr ebenso wie den Vornehmsten anhören und euch vor
niemand scheuen; denn das Gericht ist Gottes Sache. Sollte aber ein
Rechtsfall für euch zu schwierig sein, so legt ihn mir vor, damit ich
die Untersuchung dabei führe.‹ 18So gab ich euch damals Anweisung für
alles, was ihr tun solltet.«

\hypertarget{cc-wanderung-vom-horeb-bis-kades-barnea-mutlosigkeit-und-unglaube-des-volkes-nach-aussendung-der-kundschafter}{%
\subparagraph{cc) Wanderung vom Horeb bis Kades-Barnea; Mutlosigkeit und
Unglaube des Volkes nach Aussendung der
Kundschafter}\label{cc-wanderung-vom-horeb-bis-kades-barnea-mutlosigkeit-und-unglaube-des-volkes-nach-aussendung-der-kundschafter}}

19»Als wir dann vom Horeb aufgebrochen waren, durchwanderten wir jene
ganze große und schreckliche Wüste, die ihr gesehen habt, in der
Richtung nach dem Bergland der Amoriter hin, wie der HERR, unser Gott,
uns geboten hatte, und kamen bis Kades-Barnea. 20Da sagte ich zu euch:
›Ihr seid nun bis zum Bergland der Amoriter gekommen, das der HERR,
unser Gott, uns geben will. 21Ihr wißt, daß der HERR, euer Gott, euch
dies Land übergeben hat; so zieht denn hinauf und nehmt es nach dem
Befehl des HERRN, des Gottes eurer Väter, in Besitz: fürchtet euch nicht
und seid unverzagt!‹ 22Da tratet ihr alle zu mir und sagtet: ›Wir wollen
einige Männer vor uns her senden, die uns das Land auskundschaften und
uns Bericht erstatten sollen über den Weg, auf dem wir hinaufziehen
müssen, und über die Städte, zu denen wir gelangen werden.‹ 23Weil der
Vorschlag mir zusagte, wählte ich zwölf Männer aus eurer Mitte, je einen
Mann für den Stamm. 24Die zogen dann unverweilt in das Bergland hinauf
und kamen bis zum Tal Eskol\textless sup title=``d.h. Traubenbach, vgl.
4.Mose 13,23''\textgreater✲ und kundschafteten (das Land) aus. 25Sie
nahmen auch einige von den Früchten des Landes mit, brachten sie zu uns
herab und erstatteten uns Bericht mit den Worten: ›Es ist ein schönes
Land, das der HERR, unser Gott, uns geben will.‹ 26Doch ihr wolltet
nicht hinaufziehen, sondern widersetztet euch dem Befehl des HERRN,
eures Gottes, 27und sagtet murrend in euren Zelten: ›Weil der HERR uns
haßte, hat er uns aus Ägypten weggeführt, um uns in die Hand der
Amoriter fallen zu lassen, damit sie uns vernichten. 28Wohin sollen wir
hinaufziehen? Unsere Volksgenossen haben uns das Herz mutlos gemacht,
indem sie berichtet haben: Die Leute sind dort zahlreicher als wir und
höher gewachsen, die Städte groß und bis an den Himmel befestigt, und
sogar Riesen\textless sup title=``4.Mose 13,33''\textgreater✲ haben wir
dort gesehen.‹«

\hypertarget{dd-vergeblicher-ermutigungsversuch-moses}{%
\subparagraph{dd) Vergeblicher Ermutigungsversuch
Moses}\label{dd-vergeblicher-ermutigungsversuch-moses}}

29»Da sagte ich zu euch: ›Laßt euch nicht entmutigen und fürchtet euch
nicht vor ihnen! 30Der HERR, euer Gott, der vor euch einherzieht, wird
selbst für euch streiten, ganz so, wie er euch in Ägypten sichtbar
geholfen hat 31und auch in der Wüste, wo, wie ihr gesehen habt, der
HERR, euer Gott, euch getragen hat, wie ein Vater seinen Sohn trägt, auf
dem ganzen Wege, den ihr gezogen seid, bis zu eurer Ankunft an diesem
Orte.‹ 32Aber trotz alledem bliebt ihr ohne Vertrauen auf den HERRN,
euren Gott, 33der doch auf dem Wege vor euch einherzog, um einen Platz
zum Aufschlagen eurer Lager für euch ausfindig zu machen: bei Nacht im
Feuer, damit ihr auf dem Wege sehen könntet, den ihr ziehen solltet, und
bei Tag in der Wolke.«

\hypertarget{ee-das-guxf6ttliche-strafurteil-reue-des-volkes-fehlschlagen-des-eigenmuxe4chtigen-eroberungsversuches}{%
\subparagraph{ee) Das göttliche Strafurteil; Reue des Volkes;
Fehlschlagen des eigenmächtigen
Eroberungsversuches}\label{ee-das-guxf6ttliche-strafurteil-reue-des-volkes-fehlschlagen-des-eigenmuxe4chtigen-eroberungsversuches}}

34»Als nun der HERR euer lautes Reden vernahm, wurde er zornig und
schwur: 35›Wahrlich, kein einziger von diesen Männern, von diesem
nichtswürdigen Geschlecht, soll das schöne Land zu sehen bekommen,
dessen Verleihung ich euren Vätern zugeschworen habe, 36außer Kaleb, dem
Sohn Jephunnes; dieser soll es zu sehen bekommen, und ihm und seinen
Söhnen\textless sup title=``oder: Kindern''\textgreater✲ will ich das
Land geben, dessen Boden er bereits betreten hat, zum Lohn dafür, daß er
dem HERRN in allen Stücken gehorsam gewesen ist.‹ 37Auch mir zürnte der
HERR um euretwillen, so daß er aussprach: ›Auch du sollst nicht dorthin
kommen: 38Josua, der Sohn Nuns, dein Diener, der soll dorthin kommen;
ihm sprich Mut ein, denn er soll das Land den Israeliten als Erbbesitz
austeilen. 39Auch eure kleinen Kinder, von denen ihr gesagt habt, sie
würden eine Beute (der Feinde) werden, und eure jungen Söhne, die heute
noch nicht Gutes und Böses zu unterscheiden wissen, die sollen dorthin
kommen! Denn ihnen will ich das Land geben, und sie sollen es in Besitz
nehmen. 40Ihr aber -- kehrt um und zieht in die Wüste in der Richtung
nach dem Schilfmeer!‹

41Da gabt ihr mir zur Antwort: ›Wir haben uns gegen den HERRN
versündigt; wir wollen nun doch hinaufziehen und kämpfen, ganz wie der
HERR, unser Gott, uns geboten hat!‹ So gürtetet ihr euch denn sämtlich
eure Waffen um und wolltet leichtfertig in das Bergland hinaufziehen.
42Da gebot mir der HERR: ›Warne sie hinaufzuziehen und zu kämpfen! Denn
ich bin nicht in eurer Mitte; sonst werdet ihr von euren Feinden eine
Niederlage erleiden!‹ 43Ich warnte euch also, doch ihr wolltet nicht
hören, sondern widersetztet euch dem Befehl des HERRN und zogt in
Vermessenheit in das Bergland hinauf. 44Da rückten die Amoriter, die in
jenem Gebirge wohnten, euch entgegen und verfolgten euch wie ein
Bienenschwarm und zersprengten euch {[}in Seir{]} bis Horma. 45Nach
eurer Rückkehr jammertet ihr dann vor dem HERRN; er aber achtete nicht
auf euer Wehklagen und schenkte euch kein Gehör. 46So mußtet ihr denn in
Kades die lange Zeit bleiben, die ihr dort zugebracht habt.«

\hypertarget{b-ruxfcckblick-auf-den-friedlichen-zug-durch-das-land-der-edomiter-und-der-moabiter-verbot-die-ammoniter-anzugreifen-besiegung-des-amoriterkuxf6nigs-sihon}{%
\paragraph{b) Rückblick auf den friedlichen Zug durch das Land der
Edomiter und der Moabiter; Verbot, die Ammoniter anzugreifen; Besiegung
des Amoriterkönigs
Sihon}\label{b-ruxfcckblick-auf-den-friedlichen-zug-durch-das-land-der-edomiter-und-der-moabiter-verbot-die-ammoniter-anzugreifen-besiegung-des-amoriterkuxf6nigs-sihon}}

\hypertarget{section-1}{%
\section{2}\label{section-1}}

1»So machten wir denn kehrt, zogen nach der Wüste hin in der Richtung
nach dem Schilfmeer, wie der HERR mir geboten hatte, und wanderten lange
Zeit im Bogen um das Gebirge Seir herum. 2Da sagte der HERR zu mir:
3›Ihr seid nun lange genug um dieses Gebirge herumgezogen; wendet euch
jetzt nordwärts 4und gib dem Volk folgende Weisung: Ihr seid jetzt im
Begriff, das Gebiet eurer Brüder, der Nachkommen Esaus, die in Seir
wohnen, zu durchziehen, und sie werden Furcht vor euch haben; aber hütet
euch wohl, 5Krieg mit ihnen anzufangen! Denn ich werde euch nichts von
ihrem Lande geben, auch nicht einen Fußbreit, weil ich das Gebirge Seir
dem Esau als Erbbesitz gegeben habe. 6Lebensmittel für euren Bedarf
sollt ihr von ihnen für Geld kaufen, und sogar Wasser zum Trinken sollt
ihr von ihnen für Geld erstehen; 7denn der HERR, dein Gott, hat dich in
all deinen Unternehmungen gesegnet; er hat sich deiner während deiner
Wanderung durch diese große Wüste angenommen; nun schon vierzig Jahre
hindurch ist der HERR, dein Gott, mit dir gewesen, so daß es dir an
nichts gemangelt hat.‹

8So wanderten wir denn weiter, weg von unseren Brüdern, den Nachkommen
Esaus, die in Seir wohnten, weg von der Straße durch die Niederung (am
Jordan), weg von Elath und Ezjon-Geber, änderten dann die Marschrichtung
und schlugen den Weg nach der Steppe der Moabiter ein. 9Da sagte der
HERR zu mir: ›Greife die Moabiter nicht an und laß dich in keinen Krieg
mit ihnen ein! Denn ich werde dir von ihrem Lande nichts zum Besitz
geben, weil ich Ar den Nachkommen Lots als Besitztum gegeben habe.‹
10{[}Ehemals haben die Emiter darin gewohnt, ein großes, zahlreiches und
hochgewachsenes Volk wie die Enakiter\textless sup title=``=~Enakssöhne;
vgl. 4.Mose 13,33''\textgreater✲. 11Auch sie wurden, wie die Enakiter,
für Rephaiter gehalten; die Moabiter aber nannten sie Emiter. 12Und in
Seir wohnten ehemals die Horiter; aber die Nachkommen Esaus verdrängten
sie aus ihrem Besitz, rotteten sie vor sich her aus und ließen sich an
ihrer Statt nieder, gerade so wie Israel es mit dem von ihm besetzten
Lande gemacht hat, das der HERR ihnen gegeben hatte.{]}~-- 13›Macht euch
also jetzt auf und zieht über den Fluß Sered!‹ Da zogen wir über den
Fluß Sered. 14Die Zeit unserer Wanderung aber von Kades-Barnea bis zu
unserm Übergang über den Fluß Sered hat achtunddreißig Jahre betragen,
bis das ganze Geschlecht der kriegstüchtigen Männer aus dem Lager
weggestorben war, wie der HERR es ihnen zugeschworen hatte. 15Es war
aber auch die Hand des HERRN gegen sie gewesen, um sie aus dem Lager bis
auf den letzten Mann zu vertilgen.

16Als nun die kriegstüchtigen Männer sämtlich aus dem Volke weggestorben
waren, 17sagte der HERR so zu mir: 18›Du bist jetzt im Begriff, die
Grenze der Moabiter zu überschreiten, an Ar vorüber, 19und wirst nun in
die Nähe des Gebiets der Ammoniter kommen. Greife sie nicht an und laß
dich in keinen Krieg mit ihnen ein! Denn ich werde dir vom Lande der
Ammoniter nichts zum Besitz geben, weil ich es den Nachkommen Lots zum
Erbbesitz verliehen habe.‹~-- 20{[}Für ein Land der Rephaiter wird auch
dieses gehalten: Rephaiter haben ehemals in ihm gewohnt, welche die
Ammoniter aber Samsummiter nannten: 21ein großes, zahlreiches und
hochgewachsenes Volk wie die Enakiter; aber der HERR rottete sie vor
ihnen\textless sup title=``d.h. den Ammonitern''\textgreater✲ her aus,
so daß diese ihnen ihr Land wegnahmen und sich an ihrer Statt
niederließen, 22wie er es auch bei den Nachkommen Esaus, die in Seir
wohnen, getan hat, vor denen er die Horiter ausrottete, so daß sie ihnen
ihr Land wegnahmen und sich an ihrer Statt bis auf den heutigen Tag
niedergelassen haben. 23Auch die Awwiter, die in Gehöften bis Gaza hin
wohnten, wurden von den Kaphthoritern ausgerottet, die aus Kaphthor
ausgewandert waren und sich an ihrer Statt ansiedelten.{]}~-- 24›Brecht
nun unverweilt auf und geht über den Fluß Arnon! Ich gebe hiermit den
Amoriter Sihon, den König von Hesbon, und sein Land in deine Gewalt.
Mache dich sofort an die Eroberung und greife ihn mit den Waffen an!
25Vom heutigen Tage an will ich Furcht und Schrecken vor dir sich auf
die Völker unter dem ganzen Himmel lagern lassen: wenn sie nur die Kunde
von dir vernehmen, sollen sie vor dir zittern und beben!‹«

\hypertarget{besiegung-des-amoriterkuxf6nigs-sihon-und-eroberung-seines-reiches}{%
\paragraph{Besiegung des Amoriterkönigs Sihon und Eroberung seines
Reiches}\label{besiegung-des-amoriterkuxf6nigs-sihon-und-eroberung-seines-reiches}}

26»Da schickte ich Gesandte aus der Wüste Kedemoth an Sihon, den König
von Hesbon, mit folgender friedlichen Botschaft: 27›Laß mich durch dein
Land hindurchziehen! ich will überall auf der Landstraße bleiben und
weder nach rechts noch nach links davon abweichen. 28Lebensmittel zum
Unterhalt sollst du mir für Geld verkaufen und ebenso Wasser zum Trinken
mir für Geld überlassen: ich will lediglich zu Fuß hindurchziehen~--
29wie auch die Nachkommen Esaus, die in Seir wohnen, und die Moabiter,
die in Ar wohnen, es mit mir gehalten haben --, bis ich über den Jordan
in das Land hinüberziehe, das der HERR, unser Gott, uns geben will.‹
30Aber Sihon, der König von Hesbon, wollte uns den Durchzug durch sein
Land nicht gestatten; denn der HERR, dein Gott, hatte seinen Sinn hart
und sein Herz trotzig gemacht, um ihn in deine Gewalt zu geben, wie der
heutige Tag es klar zeigt. 31Da sagte der HERR zu mir: ›Du weißt, daß
ich Sihon und sein Land dir bereits preisgegeben habe; mache dich sofort
an die Besetzung seines Landes, um es einzunehmen!‹ 32Als nun Sihon uns
mit seinem ganzen Kriegsvolk entgegenzog, um bei Jahaz mit uns zu
kämpfen, 33gab ihn der HERR, unser Gott, in unsere Gewalt, so daß wir
ihn samt seinen Söhnen und seinem ganzen Kriegsvolk besiegten. 34Wir
eroberten damals alle seine Städte und vollstreckten in jeder Ortschaft
den Bann an Männern, Weibern und Kindern, ohne einen einzigen entrinnen
zu lassen; 35nur das Vieh und den Raub aus den von uns eroberten Städten
behielten wir als Beute für uns. 36Von Aroer, das am Ufer des
Arnonflusses liegt, und überhaupt von den Städten, die an dem Flusse
liegen, bis nach Gilead hin gab es keinen einzigen festen Platz, der für
uns uneinnehmbar gewesen wäre; alles gab der HERR, unser Gott, in unsere
Gewalt. 37Nur das Land der Ammoniter habt ihr unberührt gelassen, alles,
was seitwärts vom Flusse Jabbok liegt, und die Ortschaften im Berglande,
überhaupt alles, was der HERR, unser Gott, uns verboten hatte.«

\hypertarget{c-ruxfcckblick-auf-die-besiegung-des-kuxf6nigs-og-von-basan-und-die-verteilung-des-ostjordanlandes-ermutigung-josuas}{%
\paragraph{c) Rückblick auf die Besiegung des Königs Og von Basan und
die Verteilung des Ostjordanlandes; Ermutigung
Josuas}\label{c-ruxfcckblick-auf-die-besiegung-des-kuxf6nigs-og-von-basan-und-die-verteilung-des-ostjordanlandes-ermutigung-josuas}}

\hypertarget{section-2}{%
\section{3}\label{section-2}}

1»Als wir uns dann wieder aufmachten und in der Richtung nach Basan
weiterzogen, rückte uns Og, der König von Basan, mit seinem ganzen
Kriegsvolk entgegen, um bei Edrei mit uns zu kämpfen. 2Da sagte der HERR
zu mir: ›Fürchte dich nicht vor ihm! Denn ich habe ihn samt seinem
ganzen Volk und seinem Lande in deine Gewalt gegeben, und du sollst mit
ihm verfahren, wie du mit dem Amoriterkönig Sihon getan hast, der in
Hesbon wohnte.‹ 3So gab denn der HERR, unser Gott, auch Og, den König
von Basan, samt seinem ganzen Kriegsvolk in unsere Gewalt, und wir
schlugen ihn so, daß kein einziger von ihnen entkam und übrigblieb.
4Damals eroberten wir alle seine Städte; es gab keine Ortschaft, die wir
ihnen nicht entrissen hätten; sechzig Städte, die ganze Landschaft
Argob, alles, was zum Königreich Ogs in Basan gehörte, 5lauter Städte,
die mit hohen Mauern, Toren und Riegeln befestigt waren, abgesehen von
der großen Zahl der offenen Landstädte. 6Wir vollstreckten dann den Bann
an ihnen, wie wir es bei Sihon, dem König von Hesbon, gemacht hatten,
indem wir den Bann in jeder Ortschaft an Männern, Weibern und Kindern
vollstreckten; 7alles Vieh aber und den Raub aus den Städten behielten
wir als Beute für uns.«

\hypertarget{ruxfcckblick-auf-die-bis-dahin-eroberten-gebiete-und-ihre-besiedelung-durch-die-stuxe4mme-ruben-gad-und-halb-manasse}{%
\paragraph{Rückblick auf die bis dahin eroberten Gebiete und ihre
Besiedelung durch die Stämme Ruben, Gad und halb
Manasse}\label{ruxfcckblick-auf-die-bis-dahin-eroberten-gebiete-und-ihre-besiedelung-durch-die-stuxe4mme-ruben-gad-und-halb-manasse}}

8»So entrissen wir damals der Gewalt der beiden Amoriterkönige das Land
diesseits des Jordans vom Fluß Arnon bis zum Gebirge Hermon~-- 9{[}die
Phönizier nennen den Hermon Sirjon, die Amoriter dagegen nennen ihn
Senir{]} --, 10alle Städte in der Ebene und ganz Gilead sowie ganz Basan
bis Salcha und Edrei, die zum Königreich Ogs gehörenden Städte in Basan.
11{[}Denn Og, der König von Basan, war der einzige, der vom Rest der
Rephaiter noch übriggeblieben war; sein Sarg, ein Sarg von Basalt✲,
befindet sich bekanntlich zu Rabba im Ammoniterlande; seine Länge
beträgt neun Ellen und seine Breite vier Ellen, nach der gewöhnlichen
Elle gemessen.{]} 12Dieses Land nahmen wir also damals in Besitz. (Das
Land) von Aroer an, das am Fluß Arnon liegt, und die Hälfte des
Berglandes Gilead samt seinen Städten übergab ich den Stämmen Ruben und
Gad; 13das übrige Gilead aber und ganz Basan, das Reich Ogs, übergab ich
dem halben Stamm Manasse, die ganze Landschaft Argob. {[}Dieser ganze
Teil von Basan wird Land der Rephaiter genannt. 14Jair, der Sohn
Manasses, eroberte die ganze Landschaft Argob bis zum Gebiet der
Gesuriter und der Maachathiter und nannte diesen Teil von Basan nach
seinem Namen die ›Zeltdörfer Jairs‹, wie sie noch bis auf den heutigen
Tag heißen. 15Dem Machir aber gab ich Gilead, 16und den beiden Stämmen
Ruben und Gad gab ich von Gilead: das Land bis zum Fluß Arnon, bis zur
Mitte des Flusses nebst dem zugehörigen Gebiet, und bis zum Fluß Jabbok,
der Grenze der Ammoniter, 17ferner die Ebene mit dem Jordan und dem
zugehörigen Gebiet, vom See Genezareth an bis zum Meer der Steppe, dem
Salzmeer, am Fuße der Abhänge des Pisga, gegen Osten.{]}«

\hypertarget{mose-ermahnt-die-ostjordanischen-stuxe4mme-zur-kampfbereitschaft-fuxfcr-ihre-bruxfcder-ermutigung-josuas}{%
\paragraph{Mose ermahnt die ostjordanischen Stämme zur Kampfbereitschaft
für ihre Brüder; Ermutigung
Josuas}\label{mose-ermahnt-die-ostjordanischen-stuxe4mme-zur-kampfbereitschaft-fuxfcr-ihre-bruxfcder-ermutigung-josuas}}

18»Darauf gab ich ihnen damals folgende Weisung: ›Der HERR, euer Gott,
hat euch zwar dieses Land zum Besitz gegeben; aber ihr müßt nun
kampfgerüstet, soviele von euch waffenfähige Männer sind, an der Spitze
eurer Volksgenossen, der Israeliten, hinüberziehen. 19Nur eure Frauen
und kleinen Kinder und euer Vieh -- ich weiß ja, daß ihr viel Vieh
besitzt -- sollen in euren Städten, die ich euch gegeben habe,
zurückbleiben, 20bis der HERR euren Volksgenossen, ebenso wie euch, Ruhe
geschafft hat und auch sie das Land in Besitz genommen haben, das der
HERR, euer Gott, ihnen jenseits des Jordans geben wird; dann mögt ihr
wieder heimkehren, ein jeder zu seinem Besitztum, das ich euch gegeben
habe.‹~-- 21Dem Josua aber habe ich damals folgende Weisung gegeben: ›Du
hast mit eigenen Augen alles gesehen, was der HERR, euer Gott, diesen
beiden Königen hat widerfahren lassen. Ebenso wird der HERR es mit allen
anderen Königreichen machen, in die du hinüberziehen wirst. 22Fürchtet
euch nicht vor ihnen! Denn der HERR, euer Gott, wird selbst für euch
streiten.‹«

\hypertarget{mose-bittet-den-herrn-vergeblich-das-volk-uxfcber-den-jordan-fuxfchren-zu-duxfcrfen}{%
\paragraph{Mose bittet den Herrn vergeblich, das Volk über den Jordan
führen zu
dürfen}\label{mose-bittet-den-herrn-vergeblich-das-volk-uxfcber-den-jordan-fuxfchren-zu-duxfcrfen}}

23»Auch betete ich in jener Zeit zum HERRN mit den Worten: 24›O HERR,
mein Gott, du hast deinen Knecht bisher schon oft deine Größe und deine
starke Hand sehen lassen; denn wo gäbe es einen Gott im Himmel und auf
der Erde, der solche Werke und so gewaltige Taten vollbringen könnte wie
du? 25Laß mich doch hinüberziehen und das schöne Land jenseits des
Jordans sehen, dieses schöne Bergland und (besonders) den Libanon!‹
26Aber der HERR, der um euretwillen Zorn gegen mich hegte, erhörte meine
Bitte nicht, sondern antwortete mir: ›Laß es genug sein! Rede nicht noch
weiter zu mir in dieser Sache! 27Steige auf den Gipfel des Pisga hinauf
und richte deine Blicke nach Westen und Norden, nach Süden und Osten und
sieh dir das Land mit deinen Augen an; denn du wirst nicht über den
Jordan da gehen. 28Gib also dem Josua Anweisung, sprich ihm Mut ein und
stärke ihn; denn er soll an der Spitze dieses Volkes hinüberziehen, und
er soll ihnen das Land, das du sehen wirst, zum Erbbesitz geben.‹~--
29So blieben wir denn im Tale liegen, Beth-Peor gegenüber.«

\hypertarget{ermahnung-zu-strenger-beobachtung-der-guxf6ttlichen-gebote-und-warnung-vor-bilderdienst-und-abguxf6tterei}{%
\subsubsection{2. Ermahnung zu strenger Beobachtung der göttlichen
Gebote und Warnung vor Bilderdienst und
Abgötterei}\label{ermahnung-zu-strenger-beobachtung-der-guxf6ttlichen-gebote-und-warnung-vor-bilderdienst-und-abguxf6tterei}}

\hypertarget{section-3}{%
\section{4}\label{section-3}}

1»Und nun, Israel, höre auf die Satzungen und auf die Verordnungen,
deren Beobachtung ich euch lehren will, damit ihr am Leben bleibt und in
den Besitz des Landes kommt, das der HERR, der Gott eurer Väter, euch
geben will! 2Ihr sollt zu den Geboten, die ich euch zur Pflicht mache,
nichts hinzufügen und nichts davon wegnehmen, damit ihr die Gebote des
HERRN, eures Gottes, beobachtet, die ich euch zur Pflicht mache. 3Ihr
habt mit eigenen Augen gesehen, was der HERR wegen des
Baal-Peor\textless sup title=``4.Mose 25''\textgreater✲ getan hat; denn
alle Männer, die dem Baal-Peor nachgingen, hat der HERR, dein Gott, aus
deiner Mitte vertilgt, 4während ihr, die ihr am HERRN, eurem Gott,
festgehalten habt, heute noch alle am Leben seid. 5Bedenkt wohl: ich
habe euch Satzungen und Verordnungen gelehrt, wie der HERR, mein Gott,
mir geboten hat, damit ihr nach ihnen handelt in dem Lande, in das ihr
einzieht, um es in Besitz zu nehmen. 6So beobachtet sie denn und haltet
sie! Denn darin soll eure Weisheit und eure Einsicht nach dem Urteil der
übrigen Völker bestehen, die, wenn sie von all diesen Satzungen Kenntnis
erhalten werden, bekennen müssen: ›Wahrlich, ein weises und einsichtiges
Volk ist diese große Volksgemeinde!‹ 7Denn wo gäbe es sonst noch ein
großes Volk, das eine Gottheit hätte, die ihm so nahe stände, wie der
HERR, unser Gott, zu uns steht, sooft wir zu ihm rufen? 8Und wo gäbe es
sonst noch ein großes Volk, das so gerechte Satzungen und Verordnungen
hätte wie dies ganze Gesetz, das ich euch heute vorlege?«

\hypertarget{im-andenken-an-die-gestaltlose-erscheinung-gottes-am-horeb-soll-israel-sich-vor-bilderdienst-huxfcten}{%
\paragraph{Im Andenken an die gestaltlose Erscheinung Gottes am Horeb
soll Israel sich vor Bilderdienst
hüten}\label{im-andenken-an-die-gestaltlose-erscheinung-gottes-am-horeb-soll-israel-sich-vor-bilderdienst-huxfcten}}

9»Nur hüte dich und nimm dich wohl in acht, daß du die Ereignisse nicht
vergißt, die du mit eigenen Augen gesehen hast, und laß sie dir dein
ganzes Leben lang nicht aus der Erinnerung entschwinden! Nein, tu sie
deinen Kindern und Kindeskindern kund! 10(Gedenke insbesondere) des
Tages, als du am Horeb vor dem HERRN, deinem Gott, standest, als der
HERR mir gebot\textless sup title=``vgl. 2.Mose 19,10ff.''\textgreater✲:
›Versammle mir das Volk: ich will sie meine Worte hören lassen, damit
sie mich fürchten lernen, solange sie auf dem Erdboden leben, und es
auch ihre Kinder lehren!‹ 11Da tratet ihr nahe heran und stelltet euch
am Fuß des Berges auf, während der Berg bis in das Innerste des Himmels
hinein in Feuerglut brannte, von Finsternis, Gewölk und Wetterdunkel
umgeben. 12Der HERR redete dann zu euch mitten aus dem Feuer heraus; den
Schall seiner Worte vernahmt ihr wohl, aber nur den Schall; eine Gestalt
dagegen nahmt ihr nicht wahr. 13Er verkündete euch seinen Bund, den er
euch zu halten gebot, die zehn Gebote, die er dann auf zwei Steintafeln
schrieb. 14Mir aber trug der HERR damals auf, euch Satzungen und
Verordnungen zu lehren, die ihr befolgen solltet in dem Lande, in das
ihr hinüberziehen würdet, um es in Besitz zu nehmen. 15Da ihr nun an dem
Tage, als der HERR am Horeb aus dem Feuer heraus zu euch redete,
keinerlei Gestalt von ihm gesehen habt, 16so hütet euch mit aller
Sorgfalt davor, euch dadurch zu versündigen, daß ihr euch ein Gottesbild
in der Gestalt irgendeiner Bildsäule anfertigt, die Nachbildung eines
männlichen oder weiblichen Wesens, 17die Nachbildung irgendeines
vierfüßigen Tieres, das auf der Erde lebt, die Nachbildung eines
beschwingten Vogels, der am Himmel\textless sup title=``=~in der
Luft''\textgreater✲ fliegt, 18die Nachbildung irgendeines Tieres, das
auf dem Erdboden kriecht, die Nachbildung irgendeines Fisches, der im
Wasser unter der Erde lebt. 19Laß dich auch, wenn du deine Augen zum
Himmel hin erhebst, durch den Anblick der Sonne, des Mondes und der
Sterne, des ganzen Himmelsheeres, nicht dazu verführen, dich vor ihnen
niederzuwerfen und ihnen zu dienen. Denn der HERR, dein Gott, hat sie
allen anderen Völkern unter dem ganzen Himmel zur Verehrung zugewiesen:
20euch aber hat der HERR genommen und euch aus dem Eisenschmelzofen, aus
Ägypten, herausgeführt, damit ihr sein Eigentumsvolk würdet, wie ihr es
am heutigen Tage seid. 21Gegen mich aber ist der HERR um euretwillen in
solchen Zorn geraten, daß er geschworen hat, ich solle nicht den Jordan
überschreiten und nicht das schöne Land betreten, das der HERR, dein
Gott, dir zum Besitz geben will; 22sondern ich muß in diesem Lande
sterben, ohne über den Jordan zu kommen, während ihr hinüberziehen
werdet und jenes schöne Land in Besitz nehmt. 23So hütet euch nun, den
Bund, den der HERR, euer Gott, mit euch geschlossen hat, zu vergessen
und euch ein Gottesbild anzufertigen, ein Abbild von irgend etwas, das
der HERR, dein Gott, dir verboten hat; 24denn der HERR, dein Gott, ist
ein verzehrendes Feuer, ein eifersüchtiger Gott.«

\hypertarget{erneute-warnung-vor-abguxf6tterei-strafandrohung-und-gnadenverheiuxdfung}{%
\paragraph{Erneute Warnung vor Abgötterei; Strafandrohung und
Gnadenverheißung}\label{erneute-warnung-vor-abguxf6tterei-strafandrohung-und-gnadenverheiuxdfung}}

25»Wenn euch dann Kinder und Kindeskinder geboren sind und ihr euch in
dem Lande eingelebt habt und euch dann versündigt, indem ihr euch ein
Gottesbild in irgendeiner Gestalt anfertigt und somit tut, was dem
HERRN, eurem Gott, mißfällt, so daß ihr ihn erbittert, 26so rufe ich
heute den Himmel und die Erde zu Zeugen gegen euch an, daß ihr dann
unfehlbar gar bald aus dem Lande verschwinden werdet, in das ihr jetzt
über den Jordan zieht, um es in Besitz zu nehmen. Ihr werdet dann nicht
lange Zeit in ihm wohnen bleiben, sondern gänzlich daraus vertilgt
werden. 27Der HERR wird euch dann unter die Völker zerstreuen, und nur
eine geringe Zahl von euch wird übrigbleiben unter den Heidenvölkern, zu
denen der HERR euch führen wird. 28Dort werdet ihr dann Göttern dienen,
die von Menschenhänden aus Holz und Stein gemacht sind, die weder sehen
noch hören, nicht essen und nicht atmen\textless sup title=``oder: nicht
schmecken und nicht riechen''\textgreater✲ können. 29Aber von dort aus
wirst du den HERRN, deinen Gott, suchen, und du wirst ihn finden, wenn
du mit ganzem Herzen und ganzer Seele nach ihm verlangst. 30Wenn du in
Bedrängnis bist und alle diese Leiden dich in zukünftigen Tagen treffen,
so wirst du zum HERRN, deinem Gott, zurückkehren und seinen Befehlen
gehorchen. 31Denn der HERR, dein Gott, ist ein barmherziger Gott: er
wird dich nicht verlassen und dich nicht ins Verderben geraten lassen
und wird den Bund nicht vergessen, den er deinen Vätern mit einem Eide
bekräftigt hat.«

\hypertarget{die-herrlichkeit-der-guxf6ttlichen-offenbarung-und-der-gnadenerweise-verpflichtet-zum-strengsten-gehorsam}{%
\paragraph{Die Herrlichkeit der göttlichen Offenbarung und der
Gnadenerweise verpflichtet zum strengsten
Gehorsam}\label{die-herrlichkeit-der-guxf6ttlichen-offenbarung-und-der-gnadenerweise-verpflichtet-zum-strengsten-gehorsam}}

32»Denn forsche doch in den früheren Zeiten nach, die vor dir gewesen
sind, seit dem Tage, wo Gott Menschen auf der Erde geschaffen hat, und
forsche von einem Ende des Himmels bis zu dessen anderem Ende nach, ob
je etwas so Großes sich zugetragen hat oder etwas Derartiges gehört
worden ist: 33ob je ein Volk die Stimme Gottes mitten aus dem Feuer
heraus hat reden hören, wie du sie gehört hast, und dennoch am Leben
geblieben ist; 34oder ob je ein Gott auch nur den Versuch gemacht hat,
auf die Erde zu kommen, um sich ein Volk mitten aus einem andern Volk
herauszuholen durch Prüfungen, durch Zeichen und Wunder, durch Krieg,
mit starker Hand und hocherhobenem Arm und durch schreckenerregende
Großtaten, wie das alles der HERR, euer Gott, in Ägypten vor euren Augen
an euch\textless sup title=``oder: für euch''\textgreater✲ getan hat.
35Du hast es zu sehen bekommen, um zu erkennen, daß der HERR der
(einzige) Gott ist und daß es keinen anderen außer ihm gibt. 36Vom
Himmel her hat er dich seine Stimme hören lassen, um dich zu
unterweisen, und auf der Erde hat er dich sein gewaltiges Feuer sehen
lassen, und aus dem Feuer heraus hast du seine Worte vernommen. 37Weil
er also deine Väter geliebt und ihre Nachkommen nach ihnen erwählt und
in eigener Person mit seiner großen Kraft dich aus Ägypten herausgeführt
hat, 38um Völker, die dir an Größe und Macht überlegen waren, vor dir zu
vertreiben und um dich herzubringen, damit er dir ihr Land als Eigentum
gebe, wie es am heutigen Tage der Fall ist~-- 39so erkenne es heute und
nimm es dir zu Herzen, daß der HERR (allein) Gott ist oben im Himmel und
unten auf der Erde, sonst aber keiner. 40Darum beobachte seine Satzungen
und Gebote, die ich dir heute zur Pflicht mache, damit es dir und deinen
Kindern nach dir gut ergeht und du lange in dem Lande wohnen bleibst,
das der HERR, dein Gott, dir für immer geben will.«

\hypertarget{anhang-aussonderung-von-drei-freistuxe4dten-im-ostjordanlande}{%
\subsubsection{3. Anhang: Aussonderung von drei Freistädten im
Ostjordanlande}\label{anhang-aussonderung-von-drei-freistuxe4dten-im-ostjordanlande}}

41Damals sonderte Mose im Ostjordanlande drei Städte aus, 42damit ein
Totschläger, der einen andern unvorsätzlich getötet hätte, ohne vorher
mit ihm verfeindet gewesen zu sein, in eine von diesen Städten fliehen
und dadurch sein Leben retten könnte. 43Diese Städte waren: Bezer in der
Steppe, im Gebiet der Hochebene, für den Stamm Ruben, ferner Ramoth in
Gilead für den Stamm Gad und Golan in Basan für den Stamm Manasse.

\hypertarget{ii.-das-gesetz-mit-einleitungs--und-schluuxdfreden-444-3020}{%
\subsection{II. Das Gesetz mit Einleitungs- und Schlußreden
(4,44-30,20)}\label{ii.-das-gesetz-mit-einleitungs--und-schluuxdfreden-444-3020}}

\hypertarget{uxfcberschrift-und-einleitung-oder-die-bundesgrundlagen-444-1132}{%
\subsubsection{1. Überschrift und Einleitung (oder: Die
Bundesgrundlagen)
(4,44-11,32)}\label{uxfcberschrift-und-einleitung-oder-die-bundesgrundlagen-444-1132}}

\hypertarget{a-die-uxfcberschrift-oder-ankuxfcndigung-der-folgenden-gesetzespredigt-nebst-ortsangabe}{%
\paragraph{a) Die Überschrift (oder: Ankündigung der folgenden
Gesetzespredigt) nebst
Ortsangabe}\label{a-die-uxfcberschrift-oder-ankuxfcndigung-der-folgenden-gesetzespredigt-nebst-ortsangabe}}

44Und dies ist das Gesetz, das Mose den Israeliten vorlegte; 45dies sind
die Zeugnisse und die Satzungen und die Verordnungen, die Mose den
Israeliten bei ihrem Auszug aus Ägypten vortrug, 46und zwar jenseits des
Jordans in dem Tal gegenüber von Beth-Peor im Lande des ehemaligen
Amoriterkönigs Sihon, der in Hesbon gewohnt✲ hatte und den Mose und die
Israeliten bei ihrem Auszug aus Ägypten geschlagen 47und dessen Land sie
in Besitz genommen hatten ebenso wie das Land Ogs, des Königs von Basan,
(das Land) der beiden Amoriterkönige, die im Ostjordanlande gewohnt
hatten 48von Aroer am Ufer des Arnonflusses an bis zum Berge Sion -- das
ist der Hermon --, 49nebst der ganzen Steppe auf der Ostseite des
Jordans bis an das Meer der Steppe am Fuß der Abhänge des Pisga.

\hypertarget{b-erinnerung-an-die-offenbarung-gottes-am-horeb-das-grundgesetz-der-zehn-gebote-mose-von-gott-als-mittler-anerkannt}{%
\paragraph{b) Erinnerung an die Offenbarung Gottes am Horeb; das
Grundgesetz der Zehn Gebote; Mose von Gott als Mittler
anerkannt}\label{b-erinnerung-an-die-offenbarung-gottes-am-horeb-das-grundgesetz-der-zehn-gebote-mose-von-gott-als-mittler-anerkannt}}

\hypertarget{section-4}{%
\section{5}\label{section-4}}

1Da berief Mose alle Israeliten und sagte zu ihnen: »Vernimm, Israel,
die Satzungen und Verordnungen, die ich euch heute laut vortrage: lernt
sie und beobachtet sie genau! 2Der HERR, unser Gott, hat am Horeb einen
Bund mit uns geschlossen. 3Nicht mit unsern Vätern hat der HERR diesen
Bund geschlossen, sondern mit uns hier, die wir alle heute noch am Leben
sind. 4Von Angesicht zu Angesicht hat der HERR auf dem Berge aus dem
Feuer heraus mit euch geredet, 5während ich selbst damals zwischen dem
HERRN und euch stand, um euch die Worte des HERRN zu verkündigen; denn
ihr fürchtetet euch vor dem Feuer und waret nicht auf den Berg
gestiegen. Die Worte aber lauteten so:«

\hypertarget{die-zehn-gebote-oder-der-dekalog}{%
\paragraph{Die Zehn Gebote (oder: der
Dekalog)}\label{die-zehn-gebote-oder-der-dekalog}}

6»Ich bin der HERR, dein Gott\textless sup title=``oder: Ich, der HERR,
bin dein Gott''\textgreater✲, der dich aus dem Lande Ägypten, aus dem
Diensthause\textless sup title=``oder: dem Hause der
Knechtschaft''\textgreater✲, hinausgeführt hat.

7Du sollst keine anderen Götter haben neben mir!

8Du sollst dir kein Gottesbild anfertigen, irgendein Abbild von dem, was
oben im Himmel oder unten auf der Erde oder im Wasser unterhalb der Erde
ist! 9Du sollst dich vor ihnen nicht niederwerfen und ihnen nicht
dienen; denn ich, der HERR, dein Gott, bin ein eifriger\textless sup
title=``d.h. eifersüchtiger''\textgreater✲ Gott, der die Verschuldung
der Väter heimsucht an den Kindern, ja an den Enkeln und Urenkeln derer,
die mich hassen, 10der aber Gnade erweist an Tausenden von
Nachkommen\textless sup title=``oder: am tausendsten
Geschlecht''\textgreater✲ derer, die mich lieben und meine Gebote
halten.

11Du sollst den Namen des HERRN, deines Gottes, nicht
mißbrauchen!\textless sup title=``vgl. 2.Mose 20,7''\textgreater✲; denn
der HERR wird den nicht ungestraft lassen, der seinen Namen mißbraucht.

12Beobachte den Sabbattag, daß du ihn heilig hältst, wie der HERR, dein
Gott, dir geboten hat! 13Sechs Tage sollst du arbeiten und alle deine
Geschäfte verrichten! 14Aber der siebte Tag ist ein
Feiertag\textless sup title=``oder: Ruhetag''\textgreater✲ zu Ehren des
HERRN, deines Gottes: da darfst du keinerlei Arbeit\textless sup
title=``oder: Geschäft''\textgreater✲ verrichten, weder du selbst, noch
dein Sohn oder deine Tochter, weder dein Knecht noch deine Magd, weder
dein Ochs, noch dein Esel, noch all dein Vieh, noch der Fremdling, der
sich bei dir in deinen Ortschaften aufhält, damit dein Knecht und deine
Magd ausruhen können wie du selbst. 15Denke daran, daß du selbst ein
Knecht gewesen bist im Lande Ägypten und daß der HERR, dein Gott, dich
von dort mit starker Hand und hocherhobenem Arm weggeführt hat; darum
hat der HERR, dein Gott, dir geboten, den Sabbattag zu feiern.

16Ehre deinen Vater und deine Mutter, wie der HERR, dein Gott, dir
geboten hat, damit du lange lebst und damit es dir wohlergeht in dem
Lande, das der HERR, dein Gott, dir geben wird!

17Du sollst nicht töten!

18Du sollst nicht ehebrechen!

19Du sollst nicht stehlen!

20Du sollst kein falsches Zeugnis ablegen gegen deinen Nächsten!

21Du sollst nicht begehren deines Nächsten Weib und sollst dich nicht
nach dem Hause deines Nächsten gelüsten lassen, nach seinem Felde, nach
seinem Knecht und seiner Magd, nach seinem Ochsen und seinem Esel und
nach allem, was deinem Nächsten gehört!«

\hypertarget{die-vom-angsterfuxfcllten-volk-erbetene-mittlerstellung-moses-wird-von-gott-anerkannt}{%
\paragraph{Die vom angsterfüllten Volk erbetene Mittlerstellung Moses
wird von Gott
anerkannt}\label{die-vom-angsterfuxfcllten-volk-erbetene-mittlerstellung-moses-wird-von-gott-anerkannt}}

22»Diese Worte hat der HERR auf dem Berge zu eurer ganzen Versammlung
mit lauter Stimme mitten aus dem Feuer und dem dunklen Gewölk heraus
gesprochen und nichts weiter hinzugefügt; er hat sie dann auf zwei
Steintafeln geschrieben und diese mir übergeben. 23Als ihr aber die
Stimme mitten aus dem Dunkel heraus vernahmt, während der Berg in
Feuerglut brannte, da tratet ihr zu mir heran, alle eure Stammeshäupter
und eure Ältesten, 24und sagtet: ›Der HERR, unser Gott, hat uns nunmehr
seine Herrlichkeit und Größe sehen lassen, und wir haben seine Stimme
aus dem Feuer heraus gehört: heute haben wir erlebt, daß, wenn der HERR
mit Menschen redet, diese doch am Leben bleiben. 25Aber warum sollen wir
uns jetzt noch in Todesgefahr begeben? Denn dieses gewaltige Feuer wird
uns verzehren! Wenn wir selbst die Stimme des HERRN, unseres Gottes,
noch weiter hören, so werden wir sterben. 26Denn wo gäbe es in der
ganzen Menschheit jemanden, der wie wir die Stimme des lebendigen Gottes
aus dem Feuer heraus hätte reden hören und doch am Leben geblieben wäre?
27Tritt du hinzu und höre alles an, was der HERR, unser Gott, sagen
wird; berichte du uns dann alles, was der HERR, unser Gott, dir sagen
wird, so wollen wir es hören und befolgen.‹ 28Als nun der HERR diese
Worte vernahm, die ihr mit lauter Stimme an mich gerichtet hattet, sagte
der HERR zu mir: ›Ich habe die Worte gehört, die dieses Volk mit lauter
Stimme an dich gerichtet hat; sie haben recht in allem, was sie zu dir
gesagt haben. 29Möchte doch dieser ihr Vorsatz ihnen verbleiben, daß sie
mich allezeit fürchten und alle meine Gebote halten! Dann sollte es
ihnen und ihren Kindern immerdar gut ergehen. 30Gehe hin und sage ihnen:
Kehrt zu euren Zelten zurück! 31Du aber bleibe hier bei mir stehen,
damit ich dir alle Verordnungen und die Satzungen und Gebote mitteile,
die du sie lehren sollst, damit sie danach tun in dem Lande, das ich
ihnen zum Besitz geben will.‹ 32So achtet denn darauf, daß ihr so tut,
wie der HERR, euer Gott, euch geboten hat: weicht weder zur Rechten noch
zur Linken davon ab! 33Wandelt genau auf dem Wege, den der HERR, euer
Gott, euch geboten hat, damit ihr das Leben behaltet und es euch
wohlgeht und ihr lange in dem Lande wohnen bleibt, das ihr in Besitz
nehmen sollt.«

\hypertarget{c-darlegung-und-einschuxe4rfung-der-zwei-grundgebote-alleinverehrung-gottes-und-liebe-zu-gott}{%
\paragraph{c) Darlegung und Einschärfung der zwei Grundgebote:
Alleinverehrung Gottes und Liebe zu
Gott}\label{c-darlegung-und-einschuxe4rfung-der-zwei-grundgebote-alleinverehrung-gottes-und-liebe-zu-gott}}

\hypertarget{section-5}{%
\section{6}\label{section-5}}

1»Dies ist nun das Gesetz, die Satzungen und die Verordnungen, die ich
euch nach dem Befehl des HERRN, eures Gottes, lehren soll, damit ihr
nach ihnen lebt in dem Lande, zu dessen Eroberung ihr jetzt
hinüberzieht, 2auf daß ihr, du und deine Kinder und Kindeskinder, den
HERRN, euren Gott, euer ganzes Leben lang fürchtet und alle seine
Satzungen und Gebote beobachtet, die ich dir zur Pflicht mache, und auf
daß deine Tage lange Dauer haben. 3So höre sie denn, Israel, und achte
darauf, sie zu befolgen, damit es dir wohlgeht und ihr sehr zahlreich
werdet, wie der HERR, der Gott deiner Väter, es dir zugesagt hat -- in
einem von Milch und Honig überfließenden Lande.

4Höre, Israel: Der HERR ist unser Gott, der HERR allein! 5So liebe denn
den HERRN, deinen Gott, mit deinem ganzen Herzen, mit deiner ganzen
Seele und mit all deiner Kraft! 6So mögen denn diese Worte, die ich dir
heute gebiete, dir am Herzen liegen\textless sup title=``oder: ins Herz
geschrieben sein''\textgreater✲, 7und du sollst sie deinen Kindern
einschärfen und von ihnen reden, wenn du in deinem Hause sitzt und wenn
du auf der Wanderung begriffen bist, wenn du dich niederlegst und wenn
du aufstehst. 8Du sollst sie dir als ein Gedenkzeichen an\textless sup
title=``oder: auf''\textgreater✲ die Hand binden und sie als Binde
zwischen deinen Augen\textless sup title=``=~auf deiner
Stirn''\textgreater✲ tragen 9und sollst sie auf die Pfosten deines
Hauses und an deine Tore schreiben.«

\hypertarget{diene-dem-herrn-treu-und-mit-dankbarem-gehorsam-auch-in-dem-herrlichen-verheiuxdfungslande}{%
\paragraph{Diene dem Herrn treu und mit dankbarem Gehorsam auch in dem
herrlichen
Verheißungslande!}\label{diene-dem-herrn-treu-und-mit-dankbarem-gehorsam-auch-in-dem-herrlichen-verheiuxdfungslande}}

10»Auch wenn der HERR, dein Gott, dich in das Land bringen wird, das er
dir, wie er deinen Vätern Abraham, Isaak und Jakob zugeschworen hat, zu
eigen geben will, große und schöne Städte, die du nicht gebaut hast,
11und Häuser, angefüllt mit Gütern jeder Art, die du nicht angefüllt
hast, in Fels gehauene Brunnen✲, die du nicht ausgehauen hast, Weinberge
und Olivengärten, die du nicht angelegt hast, und du dich dann satt
daran ißt: 12so hüte dich wohl, den HERRN zu vergessen, der dich aus dem
Lande Ägypten, aus dem Hause der Knechtschaft, ausgeführt hat! 13Den
HERRN, deinen Gott, sollst du fürchten und ihm dienen und bei seinem
Namen schwören. 14Ihr dürft keinem andern Gott von den Göttern der
Völker, die rings um euch her wohnen, anhangen; 15denn der HERR, dein
Gott, ist ein eifriger✲ Gott in deiner Mitte; es möchte sonst der Zorn
des HERRN, deines Gottes, gegen dich entbrennen und er dich vom Erdboden
vertilgen.«

\hypertarget{stellt-nicht-durch-unglauben-und-abfall-die-langmut-gottes-auf-die-probe}{%
\paragraph{Stellt nicht durch Unglauben und Abfall die Langmut Gottes
auf die
Probe!}\label{stellt-nicht-durch-unglauben-und-abfall-die-langmut-gottes-auf-die-probe}}

16»Ihr sollt den HERRN, euren Gott, nicht versuchen, wie ihr ihn in
Massa\textless sup title=``2.Mose 17,1-7''\textgreater✲ versucht habt;
17ihr sollt vielmehr die Gebote des HERRN, eures Gottes, getreulich
beobachten sowie seine Zeugnisse und seine Verordnungen, die er dir zur
Pflicht gemacht hat; 18und du sollst das tun, was in den Augen des HERRN
recht und gut ist, damit es dir wohlergeht und du in das schöne Land,
das der HERR deinen Vätern zugeschworen hat, einziehst und es in Besitz
nimmst, 19indem du alle deine Feinde vor dir her vertreibst, wie der
HERR es verheißen hat.«

\hypertarget{belehre-auch-deine-kinder-uxfcber-die-guxf6ttlichen-erluxf6sungstaten-und-uxfcber-die-bedeutung-des-gesetzes}{%
\paragraph{Belehre auch deine Kinder über die göttlichen Erlösungstaten
und über die Bedeutung des
Gesetzes!}\label{belehre-auch-deine-kinder-uxfcber-die-guxf6ttlichen-erluxf6sungstaten-und-uxfcber-die-bedeutung-des-gesetzes}}

20»Wenn dann dein Sohn dich künftig fragt: ›Was hat es denn mit den
Zeugnissen, den Satzungen und den Verordnungen auf sich, die der HERR,
unser Gott, euch geboten hat?‹, 21so sollst du deinem Sohne antworten:
›Wir waren Knechte des Pharaos in Ägypten; aber der HERR hat uns mit
starker Hand aus Ägypten hinausgeführt, 22und der HERR hat vor unsern
Augen große und furchtbare Zeichen und Wunder in Ägypten am Pharao und
seinem ganzen Hause getan; 23uns aber hat er von dort weggeführt, um uns
hierher zu bringen, damit er uns das Land gäbe, das er unsern Vätern
zugeschworen hatte. 24Daher hat der HERR uns geboten, alle diese
Satzungen zu beobachten, indem wir den HERRN, unsern Gott, fürchteten,
damit es uns allezeit wohlergehe und er uns am Leben erhalte, wie es
noch an diesem Tage der Fall ist. 25So werden wir denn als gerecht
dastehen, wenn wir es uns angelegen sein lassen, dieses ganze Gesetz vor
dem HERRN, unserm Gott, zu erfüllen, wie er uns geboten hat.‹«

\hypertarget{d-die-abguxf6ttischen-kanaanuxe4er-und-ihr-guxf6tzendienst-sollen-ausgerottet-werden-jede-verbindung-mit-ihnen-ist-suxfcndhaft}{%
\paragraph{d) Die abgöttischen Kanaanäer und ihr Götzendienst sollen
ausgerottet werden; jede Verbindung mit ihnen ist
sündhaft}\label{d-die-abguxf6ttischen-kanaanuxe4er-und-ihr-guxf6tzendienst-sollen-ausgerottet-werden-jede-verbindung-mit-ihnen-ist-suxfcndhaft}}

\hypertarget{section-6}{%
\section{7}\label{section-6}}

1»Wenn der HERR, dein Gott, dich in das Land gebracht hat, in das du
jetzt ziehst, um es in Besitz zu nehmen, und viele Völkerschaften, die
Hethiter, Girgasiter, Amoriter, Kanaanäer, Pherissiter, Hewiter und
Jebusiter, sieben Völkerschaften, die an Zahl und Stärke dir überlegen
sind, vor dir her vertrieben hat, 2und wenn der HERR, dein Gott, sie in
deine Gewalt gegeben hat und du sie besiegt hast, so sollst du den Bann
schonungslos an ihnen vollstrecken: du darfst kein Abkommen mit ihnen
treffen und keine Gnade gegen sie üben. 3Du darfst dich auch nicht mit
ihnen verschwägern, weder deine Töchter an ihre Söhne verheiraten noch
ihre Töchter für deine Söhne zu Frauen nehmen; 4denn sie würden deine
Söhne von mir abwendig machen, so daß sie anderen Göttern dienen, und
der Zorn des HERRN würde gegen euch entbrennen und euch schnell
vertilgen. 5Vielmehr sollt ihr so mit ihnen verfahren: ihre Altäre sollt
ihr niederreißen, ihre Malsteine zertrümmern, ihre Götzenbäume umhauen
und ihre geschnitzten Götterbilder im Feuer verbrennen. 6Denn du bist
ein dem HERRN, deinem Gott, geheiligtes Volk: dich hat der HERR, dein
Gott, aus allen Völkern, die auf dem Erdboden sind, zu seinem
Eigentumsvolk erwählt. 7Nicht deshalb, weil ihr zahlreicher wärt als
alle anderen Völker, hat der HERR sich euch zugewandt und euch erwählt
-- ihr seid ja das kleinste unter allen Völkern --; 8nein, weil der HERR
Liebe zu euch hegte und weil er den Eid halten wollte, den er euren
Vätern zugeschworen hatte, deshalb hat der HERR euch mit starker Hand
weggeführt und euch aus dem Hause der Knechtschaft, aus der Gewalt des
Pharaos, des Königs von Ägypten, erlöst. 9So erkenne denn, daß der HERR,
dein Gott, der (wahre) Gott ist, der treue Gott, der den Bund und die
Gnade bis ins tausendste Glied denen bewahrt, die ihn lieben und seine
Gebote halten, 10aber denen, die ihn hassen, mit Vernichtung ihrer
eigenen Person vergilt und seinen Widersachern keinen Aufschub gewährt,
sondern ihnen an ihrer eigenen Person vergilt. 11So halte denn das
Gesetz, sowohl die Satzungen als auch die Verordnungen, deren
Beobachtung ich dir heute gebiete.«

\hypertarget{die-erfuxfcllung-des-gesetzes-wird-reichen-segen-eintragen-und-auch-im-kampf-gegen-die-kanaanuxe4er-heilsam-sein}{%
\paragraph{Die Erfüllung des Gesetzes wird reichen Segen eintragen und
auch im Kampf gegen die Kanaanäer heilsam
sein}\label{die-erfuxfcllung-des-gesetzes-wird-reichen-segen-eintragen-und-auch-im-kampf-gegen-die-kanaanuxe4er-heilsam-sein}}

12»Wenn du nun diesen Verordnungen gehorchst und sie in deinem ganzen
Tun beobachtest, so wird der HERR, dein Gott, dir dafür den Bund und die
Gnade bewahren, die er deinen Vätern zugeschworen hat, 13und wird dich
lieben und segnen und dich zahlreich werden lassen, dir Kindersegen
bescheren und deine Feldfrüchte, dein Getreide, deinen Wein und dein Öl,
die Jungen deiner Rinder und den Nachwuchs deines Kleinviehs in dem
Lande segnen, das du, wie er deinen Vätern zugeschworen hat, besitzen
sollst. 14Gesegnet wirst du vor allen Völkern sein: kein Mann und kein
Weib unter dir soll unfruchtbar sein und ebenso auch kein Stück von
deinem Vieh. 15Der HERR wird auch alle Krankheiten von dir fernhalten
und keine von den bösen Seuchen Ägyptens, die du kennst, über dich
kommen lassen, sondern alle deine Feinde damit heimsuchen.

16Du sollst aber alle Völker vernichten, die der HERR, dein Gott, in
deine Gewalt gibt: dein Auge soll sie nicht mitleidig ansehen, und du
sollst ihren Göttern nicht dienen, denn das würde ein
Fallstrick\textless sup title=``=~Anlaß zum Verderben''\textgreater✲ für
dich sein. 17Wenn du aber bei dir denken solltest: ›Diese Völkerschaften
sind mir zu stark, wie sollte ich sie vertreiben können?‹ --: 18fürchte
dich nicht vor ihnen! Denke vielmehr an das zurück, was der HERR, dein
Gott, am Pharao und an allen Ägyptern getan hat, 19an die großen
Machterweise, die du mit eigenen Augen gesehen hast, an die Zeichen und
Wunder, an die starke Hand und den hocherhobenen Arm, mit dem der HERR,
dein Gott, dich herausgeführt hat: ebenso wird der HERR, dein Gott, mit
allen Völkern verfahren, vor denen du dich jetzt fürchtest. 20Auch die
Hornissen\textless sup title=``2.Mose 23,28''\textgreater✲ wird der
HERR, dein Gott, gegen sie loslassen, bis die Übriggebliebenen und die
sich vor dir Versteckenden umgekommen sind. 21Habe also keine Angst vor
ihnen! Denn der HERR, dein Gott, ist in deiner Mitte, ein großer und
furchtbarer Gott. 22Doch der HERR, dein Gott, wird diese Völker nur nach
und nach vor dir vertreiben: du darfst sie nicht schnell vernichten,
sonst würden die wilden Tiere zu deinem Schaden überhandnehmen. 23Jedoch
wird der HERR, dein Gott, sie dir preisgeben und sie in große Bestürzung
versetzen, bis sie ausgerottet sind; 24und er wird ihre Könige in deine
Gewalt geben, und du wirst ihre Namen unter dem Himmel austilgen:
niemand soll vor dir standhalten, bis du sie ausgerottet hast. 25Ihre
geschnitzten Götzenbilder sollt ihr im Feuer verbrennen: du sollst nach
dem Silber und Gold, das sich an ihnen befindet, kein Verlangen tragen
und es nicht für dich hinnehmen, damit du dadurch nicht ins Verderben
gerätst, denn es ist ein Greuel für den HERRN, deinen Gott, 26und du
sollst einen solchen Greuel nicht in dein Haus kommen lassen, um nicht
gleich ihm dem Bann zu verfallen; du sollst es vielmehr mit Ekel
verabscheuen und es durchaus für etwas Greuelhaftes halten, denn es ist
dem Banne geweiht.«

\hypertarget{e-mahnung-zum-gehorsam-gegen-das-guxf6ttliche-gesetz-und-zur-dankbarkeit-gegen-gott-fuxfcr-die-wuxe4hrend-der-wuxfcstenwanderung-erwiesenen-wohltaten}{%
\paragraph{e) Mahnung zum Gehorsam gegen das göttliche Gesetz und zur
Dankbarkeit gegen Gott für die während der Wüstenwanderung erwiesenen
Wohltaten}\label{e-mahnung-zum-gehorsam-gegen-das-guxf6ttliche-gesetz-und-zur-dankbarkeit-gegen-gott-fuxfcr-die-wuxe4hrend-der-wuxfcstenwanderung-erwiesenen-wohltaten}}

\hypertarget{section-7}{%
\section{8}\label{section-7}}

1»Das ganze Gesetz, das ich dir heute gebiete, sollt ihr gewissenhaft
befolgen, damit ihr am Leben bleibt und zahlreich werdet und
hineinkommt, um das Land in Besitz zu nehmen, das der HERR euren Vätern
zugeschworen hat. 2Und du sollst des ganzen Weges gedenken, den der
HERR, dein Gott, dich nun vierzig Jahre lang in der Wüste hat wandern
lassen, um dich demütig zu machen und dich zu erproben, damit er
erkenne, wie es um dein Herz\textless sup title=``=~mit deiner
Gesinnung''\textgreater✲ steht, ob du nämlich seine Gebote halten wirst
oder nicht. 3So demütigte er dich denn und ließ dich Hunger leiden; dann
aber speiste er dich wieder mit dem Manna, das weder du noch deine Väter
gekannt hatten, um dich zu der Erkenntnis zu führen, daß der Mensch
nicht vom Brot allein lebt, sondern daß der Mensch von allem
lebt\textless sup title=``oder: leben kann''\textgreater✲, was vom Mund
des HERRN ausgeht. 4Die Kleider, die du anhattest, haben sich nicht
abgenutzt, und die Füße sind dir während dieser vierzig Jahre nicht
geschwollen. 5So erkenne denn in deinem Herzen, daß der HERR, dein Gott,
ebenso dein Erzieher ist, wie ein Vater seinen Sohn erzieht, 6und
befolge die Gebote des HERRN, deines Gottes, indem du auf seinen Wegen
wandelst und ihn fürchtest.«

\hypertarget{schilderung-der-herrlichkeit-des-verheiuxdfenen-mit-allen-guxfctern-reich-gesegneten-landes}{%
\paragraph{Schilderung der Herrlichkeit des verheißenen, mit allen
Gütern reich gesegneten
Landes}\label{schilderung-der-herrlichkeit-des-verheiuxdfenen-mit-allen-guxfctern-reich-gesegneten-landes}}

7»Denn der HERR, dein Gott, will dich in ein schönes Land bringen, in
ein Land mit Wasserbächen, Quellen und Grundwassern\textless sup
title=``oder: Seen?''\textgreater✲, die in der Niederung und im Gebirge
entspringen, 8ein Land mit Weizen und Gerste, mit Weinstöcken,
Feigenbäumen und Granaten, ein Land mit Ölbäumen und Honig, 9ein Land,
in welchem du dein Brot nicht kärglich zu essen brauchst, sondern an
nichts Mangel leiden wirst, ein Land, das in seinem Gestein Eisen birgt
und aus dessen Bergen du Kupfer heraushauen wirst. 10Wenn du dann
gegessen hast und satt geworden bist, so preise den HERRN, deinen Gott,
für das schöne Land, das er dir gegeben hat.«

\hypertarget{ungehorsam-und-uxfcberhebung-muxfcssen-den-untergang-herbeifuxfchren}{%
\paragraph{Ungehorsam und Überhebung müssen den Untergang
herbeiführen}\label{ungehorsam-und-uxfcberhebung-muxfcssen-den-untergang-herbeifuxfchren}}

11»Hüte dich ja, alsdann den HERRN, deinen Gott, zu vergessen, so daß du
seine Verordnungen, sowohl seine Gebote als auch seine Satzungen, deren
Befolgung ich dir heute zur Pflicht mache, nicht beobachtest. 12Laß
nicht, während du dich satt ißt und dir schöne Häuser zum Bewohnen baust
13und deine Rinder und dein Kleinvieh sich mehren und Silber und Gold
sich dir mehren und dein gesamter Besitz zunimmt, 14dein Herz sich
überheben und vergiß nicht den HERRN, deinen Gott, der dich aus dem
Lande Ägypten, aus dem Hause der Knechtschaft, herausgeführt, 15der dich
durch die große und furchtbare Wüste mit ihren feurigen\textless sup
title=``vgl. 4.Mose 21,6''\textgreater✲ Schlangen und Skorpionen
geleitet hat, durch wasserlose, dürre Gegenden, und der dir Wasser aus
dem kieselharten Felsen hat sprudeln lassen; 16der dich mit Manna, das
deine Väter nicht gekannt hatten, in der Wüste gespeist hat, um dich
demütig zu machen und auf die Probe zu stellen, damit er dir zuletzt
Gutes erweisen könnte. 17Denke dann nicht etwa bei dir selbst: ›Meine
Kraft und meine starken Arme haben mir diesen Wohlstand verschafft!‹
18Denke vielmehr daran, daß der HERR, dein Gott, es ist, der dir die
Kraft verliehen hat, solchen Wohlstand zu erwerben, weil er seinen Bund
aufrechthalten will, den er deinen Vätern zugeschworen hat {[}wie es an
diesem Tage offenbar ist{]}. 19Wenn du aber trotzdem den HERRN, deinen
Gott, vergißt und anderen Göttern nachgehst und ihnen dienst und sie
anbetest, so bezeuge ich euch heute feierlich, daß ihr unfehlbar
zugrunde gehen werdet! 20Wie die Völkerschaften, die der HERR vor euch
vernichtet, so werdet auch ihr alsdann zugrunde gehen zur Strafe dafür,
daß ihr nicht auf die Stimme des HERRN, eures Gottes, gehört habt.«

\hypertarget{f-warnung-vor-selbstgerechtigkeit-hinweis-auf-die-fruxfcheren-beweise-von-ungehorsam-und-halsstarrigkeit-des-volkes-besonders-am-horeb}{%
\paragraph{f) Warnung vor Selbstgerechtigkeit; Hinweis auf die früheren
Beweise von Ungehorsam und Halsstarrigkeit des Volkes (besonders am
Horeb)}\label{f-warnung-vor-selbstgerechtigkeit-hinweis-auf-die-fruxfcheren-beweise-von-ungehorsam-und-halsstarrigkeit-des-volkes-besonders-am-horeb}}

\hypertarget{section-8}{%
\section{9}\label{section-8}}

1»Höre, Israel! Du bist jetzt im Begriff, über den Jordan zu ziehen, um
dir drüben Völker zu unterwerfen, die größer und stärker sind als du:
große und bis an den Himmel befestigte Städte, 2ein großes und
hochgewachsenes Volk, die Enakiter\textless sup title=``4.Mose
13,33''\textgreater✲, die du selbst schon kennst und von denen du selbst
hast sagen hören: ›Wer könnte es mit den Enakitern aufnehmen?‹ 3So
sollst du denn jetzt erkennen, daß der HERR, dein Gott, selbst es ist,
der an deiner Spitze als ein verzehrendes Feuer hinüberzieht: er wird
sie vernichten, und er wird sie vor dir her niederwerfen, so daß du sie
schnell aus ihrem Besitz vertreiben und sie vernichten kannst, wie der
HERR es dir verheißen hat. 4Denke nun nicht bei dir selbst, wenn der
HERR, dein Gott, sie vor dir her vertreibt: ›Um meines Verdienstes
willen hat der HERR mich hierher geführt, damit ich dieses Land in
Besitz nehme‹ {[}während der HERR diese Völkerschaften doch wegen ihrer
Verworfenheit vor dir her ausrottet{]}. 5Nicht um deines Verdienstes
willen und nicht wegen deines aufrichtigen Herzens gelangst du in den
Besitz ihres Landes, sondern der HERR, dein Gott, rottet diese
Völkerschaften vor dir her aus wegen ihrer Verworfenheit und auch um die
Verheißung zu erfüllen, die der HERR deinen Vätern Abraham, Isaak und
Jakob zugeschworen hat. 6Bedenke also wohl, daß der HERR, dein Gott, dir
dieses schöne Land nicht um deines Verdienstes willen zum Eigentum gibt;
denn du bist ein halsstarriges Volk.«

\hypertarget{die-schwere-verschuldung-des-volkes-durch-die-anbetung-des-goldenen-stierbildes}{%
\paragraph{Die schwere Verschuldung des Volkes durch die Anbetung des
goldenen
Stierbildes}\label{die-schwere-verschuldung-des-volkes-durch-die-anbetung-des-goldenen-stierbildes}}

7»Denke daran und vergiß es nicht, daß du den HERRN, deinen Gott, in der
Wüste erzürnt hast! Von dem Tage an, da ihr aus dem Land Ägypten
ausgezogen seid, bis zu eurer Ankunft an diesem Ort habt ihr euch
widerspenstig gegen den HERRN gezeigt. 8Besonders\textless sup
title=``oder: schon''\textgreater✲ am Horeb habt ihr ihn erzürnt, und
der HERR wurde gegen euch so aufgebracht, daß er euch vertilgen wollte.
9Als ich auf den Berg gestiegen war, um die Steintafeln, die Tafeln des
Bundes, den der HERR mit euch geschlossen hatte, in Empfang zu nehmen,
da blieb ich vierzig Tage und vierzig Nächte auf dem Berge, ohne Brot zu
essen und Wasser zu trinken. 10Da übergab der HERR mir die beiden
Steintafeln, die vom Finger Gottes beschrieben waren und auf denen alle
die Worte standen, die der HERR am Tage der Versammlung an dem Berge aus
dem Feuer heraus zu euch geredet hatte. 11Als nun der HERR mir damals
nach Verlauf von vierzig Tagen und vierzig Nächten die beiden
Steintafeln, die Tafeln des Bundes, übergeben hatte, 12sagte der HERR zu
mir: ›Auf! Steige schnell von hier hinab, denn dein Volk, das du aus
Ägypten hergeführt hast, handelt sündhaft: sie sind schnell von dem Wege
abgewichen, den ich ihnen geboten habe; sie haben sich ein gegossenes
Bild gemacht.‹ 13Dann sagte der HERR weiter zu mir: ›Ich habe dieses
Volk beobachtet und erkenne wohl: es ist ein halsstarriges Volk. 14Laß
mich sie nun vertilgen und ihren Namen unter dem Himmel auslöschen! Dich
will ich dafür zu einem Volk machen, das stärker und zahlreicher ist als
sie.‹ 15Hierauf kehrte ich um und stieg vom Berge hinunter, während der
Berg im Feuer brannte und die beiden Bundestafeln in meinen beiden
Händen waren. 16Da sah ich denn wirklich, daß ihr gegen den HERRN, euren
Gott, gesündigt und euch ein gegossenes Stierbild gemacht hattet; ihr
wart schnell von dem Wege abgewichen, den der HERR euch geboten hatte.
17Da faßte ich die beiden Tafeln, schleuderte sie weg aus meinen beiden
Händen und zertrümmerte sie vor euren Augen. 18Darauf aber warf ich mich
vor dem HERRN nieder wie das erste Mal, vierzig Tage und vierzig Nächte
lang, ohne Brot zu essen und Wasser zu trinken, wegen all der Sünden,
die ihr begangen hattet, indem ihr etwas tatet, was dem HERRN mißfiel
und ihn erzürnen mußte; 19denn mir war bange vor dem Zorn und Grimm, den
der HERR gegen euch hegte, so daß er euch vertilgen wollte. Und der HERR
erhörte mich auch diesmal. 20Auch gegen Aaron war der HERR heftig
erzürnt, so daß er ihn vertilgen wollte; darum legte ich damals auch für
Aaron Fürbitte ein. 21Das Machwerk eurer Sünde aber, das Stierbild, das
ihr angefertigt hattet, nahm ich und verbrannte es im Feuer, ich
zerschlug es in Stücke und zermalmte es, bis es zu feinem Staub geworden
war; und diesen Staub warf ich in den Bach, der vom Berge herabfloß.«

\hypertarget{noch-andere-beweise-von-ungehorsam-des-volkes}{%
\paragraph{Noch andere Beweise von Ungehorsam des
Volkes}\label{noch-andere-beweise-von-ungehorsam-des-volkes}}

22»Auch bei Thabera\textless sup title=``4.Mose 11,1-3''\textgreater✲
und Massa\textless sup title=``2.Mose 17,2-7''\textgreater✲ und bei den
Lustgräbern\textless sup title=``4.Mose 11,4-34''\textgreater✲ habt ihr
den HERRN immerfort erzürnt; 23und als der HERR euch aus
Kades-Barnea\textless sup title=``4.Mose 13''\textgreater✲ aufbrechen
hieß mit dem Befehl: ›Zieht hinauf und besetzt das Land, das ich euch
bestimmt habe!‹, da habt ihr euch dem Befehl des HERRN, eures Gottes,
widersetzt und kein Vertrauen zu ihm gehabt und seiner Weisung nicht
gehorcht: 24widerspenstig seid ihr gegen den HERRN gewesen, seit ich
euch kenne!«

\hypertarget{damals-ist-die-vernichtung-israels-nur-durch-die-fuxfcrbitte-moses-abgewendet-und-die-bundeserneuerung-durch-die-gnade-gottes-bewirkt-worden}{%
\paragraph{Damals ist die Vernichtung Israels nur durch die Fürbitte
Moses abgewendet und die Bundeserneuerung durch die Gnade Gottes bewirkt
worden}\label{damals-ist-die-vernichtung-israels-nur-durch-die-fuxfcrbitte-moses-abgewendet-und-die-bundeserneuerung-durch-die-gnade-gottes-bewirkt-worden}}

25»Als ich nun jene vierzig Tage und vierzig Nächte vor dem HERRN am
Boden hingestreckt gelegen hatte -- denn der HERR hatte es
ausgesprochen, daß er euch vertilgen wolle --, 26da betete ich zum HERRN
folgendermaßen: ›O HERR, unser Gott! Vernichte nicht dieses Volk und
dein Eigentum, das du durch deine große Macht befreit, das du mit
starker Hand aus Ägypten herausgeführt hast! 27Gedenke an deine Knechte
Abraham, Isaak und Jakob! Kehre dich nicht an die Halsstarrigkeit dieses
Volkes und an seine Bosheit und Sünde, 28damit man in dem Lande, aus dem
du uns weggeführt hast, nicht sagen kann: Weil der HERR nicht imstande
war, sie in das Land zu bringen, das er ihnen zugesagt hatte, und weil
er sie haßte, hat er sie hinausgeführt, um sie in der Wüste sterben zu
lassen! 29Sie sind ja doch dein Volk und dein Eigentum, das du mit
deiner großen Kraft und deinem hocherhobenen Arm hinausgeführt hast.‹

\hypertarget{section-9}{%
\section{10}\label{section-9}}

1Damals sagte der HERR zu mir: ›Haue dir zwei Steintafeln zurecht, wie
die ersten waren, und steige zu mir auf den Berg herauf; fertige dir
auch eine hölzerne Lade an. 2Ich will dann auf die Tafeln die Worte
schreiben, die auf den ersten Tafeln gestanden haben, welche du
zertrümmert hast; dann sollst du sie in die Lade legen.‹ 3So fertigte
ich denn eine Lade von Akazienholz an, hieb zwei Steintafeln zurecht,
wie die ersten waren, und stieg mit den beiden Tafeln in der Hand auf
den Berg hinauf. 4Da schrieb er auf die Tafeln in derselben Schrift, wie
das erste Mal, die zehn Gebote, die der HERR am Tage der Versammlung an
dem Berge aus dem Feuer heraus euch zugerufen hatte, und übergab mir
(die Tafeln). 5Als ich dann umgekehrt und vom Berge wieder
hinabgestiegen war, legte ich die Tafeln in die Lade, die ich
angefertigt hatte; dort sind sie liegen geblieben, wie der HERR mir
geboten hatte.

6Die Israeliten aber brachen von Beeroth-Bene-Jaakan nach
Mosera\textless sup title=``vgl. 4.Mose 33,30-31''\textgreater✲ auf;
dort starb Aaron und wurde dort begraben, und sein Sohn Eleasar trat als
Priester an seine Stelle\textless sup title=``4.Mose
20,22-29''\textgreater✲. 7Von da zogen sie nach Gudgoda weiter und von
Gudgoda nach Jotba, einem reichbewässerten Landstrich. 8Damals sonderte
der HERR den Stamm Levi dazu aus, die Lade mit dem Bundesgesetz des
HERRN zu tragen und als seine Diener vor ihm zu stehen und in seinem
Namen zu segnen\textless sup title=``4.Mose 6,22-27''\textgreater✲, wie
es bis auf den heutigen Tag geschieht. 9Darum haben die Leviten kein
Erbteil und kein Besitztum erhalten wie ihre Brüder: der HERR ist ihr
Erbteil, wie der HERR, dein Gott, ihnen zugesagt hat.

10Als ich aber auf dem Berge ebenso lange wie das erste Mal, nämlich
vierzig Tage und vierzig Nächte, geblieben war, erhörte der HERR mich
auch diesmal; der HERR wollte dich nicht verderben 11und gebot mir:
›Mache dich auf den Weg und brich an der Spitze des Volkes auf, damit
sie ans Ziel kommen und das Land in Besitz nehmen, das ich ihnen geben
will, wie ich ihren Vätern eidlich versprochen habe!‹«

\hypertarget{g-mahnung-zu-treuer-erfuxfcllung-der-guxf6ttlichen-gebote-zur-gottesfurcht-und-gottesliebe-segen-und-fluch}{%
\paragraph{g) Mahnung zu treuer Erfüllung der göttlichen Gebote, zur
Gottesfurcht und Gottesliebe; Segen und
Fluch}\label{g-mahnung-zu-treuer-erfuxfcllung-der-guxf6ttlichen-gebote-zur-gottesfurcht-und-gottesliebe-segen-und-fluch}}

12»Und nun, Israel, was fordert der HERR, dein Gott, von dir? Doch nur,
daß du den HERRN, deinen Gott, fürchtest und immerdar auf seinen Wegen
wandelst und daß du ihn liebst und dem HERRN, deinem Gott, mit ganzem
Herzen und mit ganzer Seele dienst, 13indem du die Gebote und die
Satzungen des HERRN, die ich dir heute gebiete, zu deinem eigenen
Wohlergehen beobachtest. 14Bedenke wohl: dem HERRN, deinem Gott, gehört
der Himmel, und zwar der innerste Himmel\textless sup title=``oder: bis
zum obersten Himmel hin''\textgreater✲, die Erde und alles, was auf ihr
ist; 15und doch hat der HERR sich nur deinen Vätern in Liebe zugewandt
und hat euch, die Nachkommen jener, aus allen Völkern auserwählt, wie es
an diesem Tage offenbar ist. 16So beschneidet denn die Vorhaut eures
Herzens und zeigt euch nicht länger halsstarrig! 17Denn der HERR, euer
Gott, ist der Gott der Götter und der Herr der Herren, der große, starke
und furchtbare Gott, der die Person nicht ansieht und sich nicht
bestechen läßt, 18der den Waisen und Witwen Recht schafft und den
Fremdling liebhat, so daß er ihm Brot und Kleidung gibt. 19Darum sollt
ihr auch den Fremdling lieben; denn ihr seid selbst Fremdlinge im Lande
Ägypten gewesen. 20Den HERRN, deinen Gott, sollst du fürchten, ihm
sollst du dienen, ihm anhangen und nur bei seinem Namen schwören! 21Er
ist dein Ruhm und er dein Gott, der an dir\textless sup title=``oder:
für dich''\textgreater✲ jene großen und wunderbaren Taten vollführt hat,
die du mit eigenen Augen gesehen hast. 22Siebzig Seelen an Zahl sind
einst deine Väter nach Ägypten hinabgezogen, und jetzt hat der HERR,
dein Gott, dich so zahlreich gemacht wie die Sterne am Himmel.«

\hypertarget{hinweis-auf-die-persuxf6nlich-erlebten-vernichtenden-gerichtstaten-gottes}{%
\paragraph{Hinweis auf die persönlich erlebten vernichtenden
Gerichtstaten
Gottes}\label{hinweis-auf-die-persuxf6nlich-erlebten-vernichtenden-gerichtstaten-gottes}}

\hypertarget{section-10}{%
\section{11}\label{section-10}}

1»So liebe denn den HERRN, deinen Gott, und beobachte allezeit, was er
beobachtet wissen will, seine Satzungen, seine Verordnungen und seine
Gebote, 2und erkennet heute -- denn nicht zu euren Kindern rede ich,
welche die Zucht\textless sup title=``=~die Züchtigungen oder:
Strafgerichte''\textgreater✲ des HERRN, eures Gottes, nicht miterlebt
und seine große Macht nicht gesehen haben, seine starke Hand und seinen
hocherhobenen Arm, 3seine Wunderzeichen und seine Taten, die er in
Ägypten am Pharao, dem König von Ägypten, und an seinem ganzen Lande
vollführt hat, 4und was er der Heeresmacht der Ägypter, ihren Rossen und
Kriegswagen, hat widerfahren lassen, über die er die Wasser des
Schilfmeeres hinströmen ließ, als sie euch verfolgten, und die der HERR
so bis auf den heutigen Tag vernichtet hat; 5und was er sodann an euch
in der Wüste bis zu eurer Ankunft an diesem Ort vollbracht hat, 6und was
er an Dathan und Abiram, den Söhnen des Rubeniten Eliab, getan hat, wie
da die Erde ihren Mund auftat und sie samt ihren Familien und Zelten und
dem ganzen Bestand, der zu ihnen gehörte, inmitten aller Israeliten
verschlang --: 7nein, zu euch rede ich, die ihr mit eigenen Augen alle
die großen Taten gesehen habt, die der HERR vollbracht hat. 8So
beobachtet denn das ganze Gesetz, das ich euch heute gebiete, damit ihr
stark seid und dazu kommt, das Land in Besitz zu nehmen, in das ihr
hinüberzieht, um es zu erobern, 9und damit ihr lange in dem Lande wohnen
bleibt, dessen Verleihung der HERR euren Vätern und ihren Nachkommen
zugeschworen hat, ein Land, das von Milch und Honig überfließt.«

\hypertarget{hinweis-auf-das-herrliche-aber-von-gottes-fuxfcrsorge-vuxf6llig-abhuxe4ngige-verheiuxdfungsland}{%
\paragraph{Hinweis auf das herrliche, aber von Gottes Fürsorge völlig
abhängige
Verheißungsland}\label{hinweis-auf-das-herrliche-aber-von-gottes-fuxfcrsorge-vuxf6llig-abhuxe4ngige-verheiuxdfungsland}}

10»Denn das Land, in das du einziehst, um es in Besitz zu nehmen, ist
nicht so wie das Land Ägypten, aus dem ihr ausgezogen seid, wo du die
Saat, die du gesät hattest, wie einen Gemüsegarten (durch Schöpfräder)
mit den Füßen bewässern mußtest\textless sup title=``oder:
konntest''\textgreater✲; 11nein, das Land, in das ihr hinüberzieht, um
es in Besitz zu nehmen, ist ein Land mit Bergen und Tälern, das vom
Regen des Himmels getränkt wird, 12ein Land, für das der HERR, dein
Gott, Sorge trägt und auf das die Augen des HERRN, deines Gottes,
beständig gerichtet sind vom Anfang des Jahres bis zum Ende des Jahres.
13Wenn ihr nun den Geboten, die ich euch heute zur Pflicht mache,
getreulich nachkommt, indem ihr den HERRN, euren Gott, liebt und ihm von
ganzem Herzen und mit ganzer Seele dient, 14›so wird er eurem Lande zu
rechter Zeit Regen geben, Frühregen und Spätregen\textless sup
title=``d.h. Herbst- und Frühjahrsregen''\textgreater✲, damit du dein
Getreide, deinen Wein und dein Öl einbringen kannst; 15auch wird er auf
deinen Feldern Futter für dein Vieh wachsen lassen, so daß du zu essen
hast und satt wirst‹. 16Hütet euch (aber) wohl, daß euer Herz sich nicht
betören läßt und ihr nicht abfallt und anderen Göttern dient und sie
anbetet! 17Sonst wird der Zorn des HERRN gegen euch entbrennen, und er
wird den Himmel verschließen, so daß kein Regen mehr fällt und der
Erdboden keinen Ertrag mehr gibt und ihr schnell aus dem schönen Lande
verschwindet, das der HERR euch geben will.«

\hypertarget{nochmalige-mahnung-zu-treuem-gehorsam-vorlegung-von-segen-und-fluch}{%
\paragraph{Nochmalige Mahnung zu treuem Gehorsam; Vorlegung von Segen
und
Fluch}\label{nochmalige-mahnung-zu-treuem-gehorsam-vorlegung-von-segen-und-fluch}}

18»Laßt also diese meine Worte Eingang in euer Herz finden und euch ganz
durchdringen, bindet sie euch als ein Gedenkzeichen an\textless sup
title=``oder: auf''\textgreater✲ eure Hand und tragt sie als Binde auf
eurer Stirn\textless sup title=``vgl. 6,7''\textgreater✲, 19lehrt sie
auch eure Kinder, indem ihr davon redet, wenn ihr zu Hause sitzt oder
auf der Wanderung begriffen seid, wenn ihr euch niederlegt und wenn ihr
aufsteht; 20und schreibt sie auf die Türpfosten eurer Häuser und an eure
Tore, 21damit ihr und eure Kinder in dem Lande, dessen Verleihung der
HERR euren Vätern zugeschworen hat, so lange wohnen bleibt, wie der
Himmel über der Erde steht. 22Denn wenn ihr dieses ganze Gesetz, dessen
Beobachtung ich euch heute gebiete, gewissenhaft beobachtet, indem ihr
den HERRN, euren Gott, liebt, allezeit auf seinen Wegen wandelt und ihm
anhangt, 23so wird der HERR alle diese Völkerschaften vor euch her
ausrotten, und ihr werdet Völkerschaften aus ihrem Besitz verdrängen,
die größer und stärker sind als ihr. 24Das ganze Gebiet, das eure
Fußsohle betreten wird, soll euer Eigentum werden: von der Wüste bis an
den Libanon, von dem Strome, dem Euphratstrom, bis an das Meer im Westen
soll euer Gebiet reichen; 25niemand soll euch gegenüber standhalten
können; Furcht und Schrecken vor euch wird der HERR, euer Gott, über das
ganze Land verbreiten, das ihr betreten werdet, wie er es euch zugesagt
hat.

26Seht, ich lege euch heute Segen und Fluch zur Wahl vor: 27den Segen,
wenn ihr den Geboten des HERRN, eures Gottes, gehorcht, die ich euch
heute gebiete; 28aber den Fluch, wenn ihr den Geboten des HERRN, eures
Gottes, nicht gehorcht und von dem Wege, den ich euch heute gebiete,
abweicht, um anderen Göttern anzuhangen\textless sup title=``oder:
nachzugehen''\textgreater✲, von denen ihr vorher nichts gewußt habt.«

\hypertarget{der-garizim-als-berg-des-segens-und-der-ebal-als-berg-des-fluches-verordnet}{%
\paragraph{Der Garizim als Berg des Segens und der Ebal als Berg des
Fluches
verordnet}\label{der-garizim-als-berg-des-segens-und-der-ebal-als-berg-des-fluches-verordnet}}

29»Wenn nun der HERR, dein Gott, dich in das Land gebracht hat, in das
du jetzt ziehst, um es in Besitz zu nehmen, so sollst du den Segen auf
dem Berge Garizim und den Fluch auf dem Berge Ebal erteilen. 30Diese
liegen bekanntlich jenseits des Jordans, westlich von der nach
Sonnenuntergang führenden Straße, im Lande der Kanaanäer, die in der
Jordanebene wohnen, Gilgal gegenüber, neben dem Terebinthenhain von
More. 31Denn ihr seid im Begriff, den Jordan zu überschreiten, um in den
Besitz des Landes zu gelangen, das der HERR, euer Gott, euch geben will.
Wenn ihr es dann besetzt habt und in ihm wohnt, 32so seid auf die
Erfüllung aller Satzungen und Verordnungen bedacht, die ich euch heute
vorlege.«

\hypertarget{die-einzelgesetzgebung-kap.-12-26}{%
\subsubsection{2. Die Einzelgesetzgebung (Kap.
12-26)}\label{die-einzelgesetzgebung-kap.-12-26}}

\hypertarget{a-neue-uxfcberschrift-zum-ganzen-gesetz-aufstellung-des-grundgesetzes-nur-an-einer-von-gott-erwuxe4hlten-stuxe4tte-darf-gott-durch-opfer-verehrt-werden}{%
\paragraph{a) Neue Überschrift zum ganzen Gesetz; Aufstellung des
Grundgesetzes: »Nur an einer von Gott erwählten Stätte darf Gott durch
Opfer verehrt
werden«}\label{a-neue-uxfcberschrift-zum-ganzen-gesetz-aufstellung-des-grundgesetzes-nur-an-einer-von-gott-erwuxe4hlten-stuxe4tte-darf-gott-durch-opfer-verehrt-werden}}

\hypertarget{section-11}{%
\section{12}\label{section-11}}

1»Dies sind die Satzungen und die Verordnungen, die ihr in dem Lande,
das der HERR, der Gott eurer Väter, euch zum Eigentum bestimmt hat,
allezeit beobachten sollt, solange ihr auf dem Erdboden lebt«: 2»Ihr
sollt alle Stätten von Grund aus zerstören, an denen die Völkerschaften,
die ihr aus ihrem Besitz verdrängen werdet, ihre Götter verehrt haben,
auf den hohen Bergen wie auf den Hügeln und unter jedem dichtbelaubten
Baum. 3Ihr sollt also ihre Altäre niederreißen und ihre
Malsteine\textless sup title=``vgl. 2.Mose 34,13''\textgreater✲
zertrümmern, ihre Götzenbäume im Feuer verbrennen, ihre geschnitzten
Götterbilder zerschlagen und ihren Namen von den betreffenden Stätten
verschwinden lassen.

4Mit dem HERRN, eurem Gott, dürft ihr es nicht so halten (wie jene
Völker mit ihren Göttern); 5vielmehr nur die eine Stätte, die der HERR,
euer Gott, aus all euren Stammesgebieten erwählen wird, um seinen Namen
dorthin zu versetzen und dort Wohnung zu nehmen, die sollt ihr aufsuchen
und euch dorthin begeben; 6und dorthin sollt ihr eure Brandopfer und
Schlachtopfer, eure Zehnten und die Hebeopfer, die ihr darbringt, eure
Gelübdeopfer und freiwilligen Gaben sowie die Erstgeburten eurer Rinder
und eures Kleinviehs bringen. 7Dort sollt ihr auch eure Opfermahlzeiten
vor dem HERRN, eurem Gott, halten, ihr und eure Familien, und euch der
Freude über alles das hingeben, was ihr mit eurer Hände Arbeit beschafft
habt und womit der HERR, dein Gott, dich gesegnet hat. 8Ihr dürft es
künftig nicht mehr so machen, wie wir es heutigentags hier ein jeder
ganz nach seinem Belieben zu tun pflegen; 9denn bis jetzt seid ihr noch
nicht zum ruhigen Besitz des Erbteils gekommen, das der HERR, dein Gott,
dir geben wird. 10Wenn ihr aber den Jordan überschritten habt und in dem
Lande wohnt, das der HERR, euer Gott, euch als Erbbesitz verleihen will,
und wenn er euch Ruhe vor allen euren Feinden ringsum verschafft hat, so
daß ihr in Sicherheit wohnt, 11dann sollt ihr an die Stätte, die der
HERR, euer Gott, zur Wohnung für seinen Namen erwählen wird, alles das
bringen, was ich euch gebiete: eure Brand- und Schlachtopfer, eure
Zehnten und die Hebeopfer, die ihr darbringt, und alle eure auserlesenen
Gelübdeopfer, die ihr dem HERRN geloben werdet. 12Dort sollt ihr auch
vor dem HERRN, eurem Gott, fröhlich sein, ihr und eure Söhne und
Töchter, eure Knechte und Mägde, auch die Leviten, die in euren
Ortschaften wohnen; denn sie haben keinen eigenen Landbesitz und kein
Erbteil gleich euch.

13Hüte dich wohl, deine Brandopfer an jedem beliebigen Ort, den du dir
ersehen wirst, darzubringen! 14Vielmehr sollst du nur an der Stätte, die
der HERR in einem deiner Stammesgebiete erwählen wird, deine Brandopfer
darbringen und dort alles das verrichten, was ich dir gebiete. 15Doch
darfst du in all deinen Wohnorten ganz nach Herzenslust schlachten und
Fleisch essen, je nachdem der HERR, dein Gott, dich gesegnet hat: der
Reine wie der Unreine darf es essen, wie das Fleisch einer Gazelle oder
eines Hirsches; 16nur das Blut dürft ihr nicht genießen: auf die Erde
müßt ihr es wie Wasser schütten. 17Du darfst nicht in deinen Wohnorten
den Zehnten deines Getreides und Weins und Öls verzehren, auch nicht die
Erstgeburten deiner Rinder und deines Kleinviehs und keins von deinen
Gelübdeopfern, die du geloben wirst, auch nicht deine freiwilligen Gaben
und die Hebeopfer, die du darbringen wirst; 18sondern vor dem HERRN,
deinem Gott, sollst du sie an der Stätte verzehren, die der HERR, dein
Gott, erwählen wird, du und dein Sohn und deine Tochter, dein Knecht und
deine Magd, auch die Leviten, die in deinen Wohnorten leben, und sollst
dich vor dem HERRN, deinem Gott, an allem erfreuen, was du mit deiner
Hände Arbeit beschafft hast. 19Hüte dich, die Leviten unbeachtet zu
lassen, solange du in deinem Lande lebst!«

\hypertarget{das-schlachten-von-vieh-und-der-genuuxdf-von-fleisch-auuxdfer-dem-blut-und-dem-opferfleisch-ist-uxfcberall-gestattet}{%
\paragraph{Das Schlachten von Vieh und der Genuß von Fleisch (außer dem
Blut und dem Opferfleisch) ist überall
gestattet}\label{das-schlachten-von-vieh-und-der-genuuxdf-von-fleisch-auuxdfer-dem-blut-und-dem-opferfleisch-ist-uxfcberall-gestattet}}

20»Wenn der HERR, dein Gott, dein Gebiet erweitert, wie er dir zugesagt
hat, und du dann denkst: ›Ich möchte wohl Fleisch essen!‹, weil du
Verlangen nach Fleisch trägst, so magst du ganz nach Herzenslust Fleisch
essen. 21Wenn (in diesem Fall) die Stätte, die der HERR, dein Gott,
erwählen wird, um seinen Namen dorthin zu versetzen, weit von dir
entfernt ist, so schlachte von deinen Rindern und deinem Kleinvieh, die
der HERR dir gegeben hat, wie ich dir geboten habe, und iß davon in
deinen Wohnorten ganz nach Herzenslust. 22Jedoch sollst du es so essen,
wie man Fleisch von der Gazelle und vom Hirsch genießt: der Reine wie
der Unreine darf es ohne Unterschied essen. 23Nur halte daran fest, kein
Blut zu genießen; denn das Blut ist die Seele\textless sup title=``=~der
Sitz der Seele oder des Lebens''\textgreater✲, und du darfst die
Seele\textless sup title=``oder: das Leben''\textgreater✲ nicht zugleich
mit dem Fleisch essen. 24Du darfst es nicht genießen: schütte es
vielmehr wie Wasser auf die Erde. 25Du darfst es nicht genießen, damit
es dir und deinen Kindern nach dir gut ergeht, wenn\textless sup
title=``oder: weil''\textgreater✲ du tust, was dem HERRN wohlgefällig
ist. 26Jedoch die heiligen Gaben, die dir obliegen, und deine
Gelübdeopfer sollst du nehmen und dich mit ihnen an die Stätte begeben,
die der HERR sich erwählen wird, 27und du sollst deine Brandopfer, das
Fleisch und das Blut, auf dem Altar des HERRN, deines Gottes,
darbringen, und zwar soll das Blut deiner Schlachtopfer an den Altar des
HERRN, deines Gottes, geschüttet werden, das Fleisch aber darf von dir
gegessen werden.

28Beachte und befolge alle diese Gebote, die ich dir zur Pflicht mache,
damit es dir und deinen Kindern nach dir allezeit gut ergeht,
wenn\textless sup title=``oder: weil''\textgreater✲ du tust, was in den
Augen des HERRN, deines Gottes, gut und recht ist.«

\hypertarget{b-verbot-jeder-nachahmung-heidnischen-gottesdienstes-bestrafung-der-guxf6tzendiener}{%
\paragraph{b) Verbot jeder Nachahmung heidnischen Gottesdienstes;
Bestrafung der
Götzendiener}\label{b-verbot-jeder-nachahmung-heidnischen-gottesdienstes-bestrafung-der-guxf6tzendiener}}

29»Wenn der HERR, dein Gott, die Völkerschaften, zu deren Vertreibung du
ausziehst, vor dir her ausgerottet hat und du nach ihrer Vertreibung in
ihrem Lande wohnst, 30so hüte dich wohl, dich durch ihr Beispiel zur
Nachahmung verführen zu lassen, nachdem sie vor dir vertilgt worden
sind, und dich nach ihren Göttern zu erkundigen, indem du fragst: ›Wie
haben diese Völkerschaften ihre Götter verehrt?‹ und dann sagst: ›Ich
will es auch so machen!‹ 31So darfst du gegen den HERRN, deinen Gott,
nicht verfahren; denn alles Mögliche, was für den HERRN ein Greuel ist,
den er verabscheut, haben sie bei ihrem Götterdienst verübt; sogar ihre
Söhne und Töchter haben sie ja ihren Göttern zu Ehren im Feuer
verbrannt!«

\hypertarget{section-12}{%
\section{13}\label{section-12}}

1»Alle Gebote, die ich euch zur Pflicht mache, sollt ihr gewissenhaft
beobachten, ohne etwas hinzuzufügen oder etwas davon wegzulassen.«~--

\hypertarget{huxe4rteste-bestrafung-aller-falschen-propheten-und-der-abguxf6ttischen-nuxe4chsten-verwandten}{%
\paragraph{Härteste Bestrafung aller falschen Propheten und der
abgöttischen nächsten
Verwandten}\label{huxe4rteste-bestrafung-aller-falschen-propheten-und-der-abguxf6ttischen-nuxe4chsten-verwandten}}

2»Wenn in deiner Mitte ein Prophet oder ein Träumer\textless sup
title=``d.h. ein Mann, der Traumgesichte hat''\textgreater✲ auftritt und
dir ein Zeichen oder Wunder angibt, 3das dann auch wirklich seiner
Ankündigung entsprechend eintrifft, und hierauf die Aufforderung an dich
richtet: ›Laßt uns andere Götter verehren -- die dir bisher unbekannt
gewesen sind -- und ihnen dienen!‹, 4so sollst du den Worten eines
solchen Propheten oder eines solchen Traumsehers kein Gehör schenken;
denn der HERR, euer Gott, will euch damit nur auf die Probe stellen, um
sich zu überzeugen, ob ihr wirklich den HERRN, euren Gott, von ganzem
Herzen und mit ganzer Seele liebt. 5Dem HERRN, eurem Gott, sollt ihr
nachfolgen und ihn fürchten, seine Gebote sollt ihr beobachten und auf
seine Weisungen hören, ihm dienen und ihm anhangen! 6Jener Prophet aber
oder jener Traumseher soll den Tod erleiden! Denn er hat Abfall
gepredigt vom HERRN, eurem Gott, der euch aus dem Lande Ägypten
herausgeführt und dich aus dem Hause der Knechtschaft erlöst hat, und
ist darauf ausgegangen, dich von dem Wege abzubringen, auf dem du nach
dem Gebot des HERRN, deines Gottes, wandeln sollst: schaffe das Böse aus
deiner Mitte hinweg!

7Wenn dein Bruder, sogar dein Vollbruder, oder dein Sohn oder deine
Tochter oder das Weib an deinem Busen oder dein Freund, der dir so lieb
ist wie dein eigenes Leben, dich insgeheim verleiten will, indem er dich
auffordert: ›Laß uns hingehen und anderen Göttern dienen!‹ -- solchen
Göttern, die dir und deinen Vätern bisher unbekannt gewesen sind 8und
die den Völkern rings um euch her angehören, mögen diese in deiner Nähe
oder fern von dir wohnen, von dem einen Ende der Erde bis zum andern --:
9so sollst du ihm nicht zu Willen sein und nicht auf ihn hören, sollst
auch keinen Blick des Mitleids für ihn haben und keine Schonung gegen
ihn üben oder seine Schuld verheimlichen, 10sondern sollst ihn unbedingt
ums Leben bringen: deine Hand soll die erste sein, die sich gegen ihn
erhebt, um ihm den Tod zu geben, und danach die Hand des ganzen Volkes;
11und zwar sollst du ihn zu Tode steinigen; denn er ist darauf
ausgegangen, dich vom HERRN, deinem Gott, abwendig zu machen, der dich
aus dem Lande Ägypten, aus dem Hause der Knechtschaft, herausgeführt
hat; 12und ganz Israel soll Kunde davon erhalten und sich fürchten,
damit keiner wieder etwas so Böses in deiner Mitte verübt.«

\hypertarget{sogar-eine-ganze-abguxf6ttisch-gewordene-ortschaft-soll-dem-bann-verfallen}{%
\paragraph{Sogar eine ganze abgöttisch gewordene Ortschaft soll dem Bann
verfallen}\label{sogar-eine-ganze-abguxf6ttisch-gewordene-ortschaft-soll-dem-bann-verfallen}}

13»Wenn du von einer deiner Ortschaften, die der HERR, dein Gott, dir zu
Wohnsitzen gibt, sagen hörst, 14es seien dort nichtswürdige Leute aus
deiner Mitte aufgetreten, die ihre Mitbürger mit der Aufforderung
verführt hätten: ›Laßt uns hingehen und anderen Göttern dienen!‹ --
solchen Göttern, die euch vorher unbekannt gewesen sind --, 15so sollst
du eine genaue Untersuchung und sorgfältige Nachforschung anstellen, und
wenn die Sache sich dann in der Tat so verhält, wie dir berichtet ist,
und ein solcher Greuel in deiner Mitte wirklich verübt worden ist, 16so
sollst du die Bewohner der betreffenden Ortschaft mit der Schärfe des
Schwertes schlagen, indem du an ihr und an allem, was in ihr ist, auch
an ihrem Vieh, den Bann mit der Schärfe des Schwertes vollstreckst.
17Dann sollst du alles in ihr Erbeutete auf dem dortigen Marktplatz
zusammenhäufen und die Ortschaft samt der gesamten Beute als ein
Ganzopfer für den HERRN, deinen Gott, verbrennen, und sie soll für immer
ein Schutthaufen bleiben: sie darf nie wieder aufgebaut werden, 18und
von dem Banngut darf nichts an deiner Hand hängen bleiben, damit der
HERR von seiner Zornglut wieder abläßt und sich dir gnädig erweist und
infolge seines Erbarmens dich zahlreich werden läßt, wie er deinen
Vätern zugeschworen hat, 19wenn du nämlich auf die Stimme des HERRN,
deines Gottes, hörst, indem du alle seine Gebote hältst, die ich dir
heute zu befolgen gebiete, und du das tust, was vor dem HERRN, deinem
Gott, das Richtige\textless sup title=``oder:
wohlgefällig''\textgreater✲ ist.«

\hypertarget{c-verbot-heidnischer-trauergebruxe4uche-und-unreiner-speisen}{%
\paragraph{c) Verbot heidnischer Trauergebräuche und unreiner
Speisen}\label{c-verbot-heidnischer-trauergebruxe4uche-und-unreiner-speisen}}

\hypertarget{section-13}{%
\section{14}\label{section-13}}

1»Ihr seid Söhne\textless sup title=``oder: Kinder''\textgreater✲ für
den HERRN, euren Gott; darum dürft ihr euch wegen eines Toten keine
Einschnitte ins Fleisch machen und euch über der Stirn nicht
kahlscheren; 2denn du bist ein dem HERRN, deinem Gott, geheiligtes Volk,
und dich hat der HERR, dein Gott, aus allen Völkern, die es auf dem
ganzen Erdboden gibt, zu seinem Eigentumsvolk erwählt.~--

3Du sollst nichts Greuelhaftes essen! 4Dies sind die
Vierfüßler\textless sup title=``oder: größeren Landtiere''\textgreater✲,
die ihr essen dürft: Rind, Schaf und Ziege, 5Hirsch, Gazelle, Damwild,
Steinbock, Antilope, wilder Ochs und Bergziege 6und überhaupt alle
Vierfüßler, die gespaltene Hufe\textless sup title=``oder:
Klauen''\textgreater✲ haben, und zwar ganz durchgespaltene, also zwei
Hufe\textless sup title=``oder: Klauen''\textgreater✲, und die zugleich
Wiederkäuer unter den Vierfüßlern sind: diese dürft ihr essen. 7Dagegen
folgende dürft ihr von den Wiederkäuern und von denen, welche ganz
durchgespaltene Hufe\textless sup title=``oder: Klauen''\textgreater✲
haben, nicht essen: das Kamel, den Hasen und den Klippdachs\textless sup
title=``vgl. Psalm 104,18''\textgreater✲; denn sie sind zwar
Wiederkäuer, haben aber keine gespaltenen Hufe\textless sup
title=``oder: Klauen''\textgreater✲: als unrein sollen sie euch gelten;
8ferner das Schwein, denn es hat zwar gespaltene Klauen, ist aber kein
Wiederkäuer: als unrein soll es euch gelten; vom Fleisch dieser Tiere
dürft ihr nichts genießen und ihre toten Leiber nicht anrühren.~-- 9Dies
ist es, was ihr von allen im Wasser lebenden Tieren essen dürft: alles,
was Flossen und Schuppen hat: diese dürft ihr essen; 10aber alles, was
keine Flossen und Schuppen hat, dürft ihr nicht genießen: als unrein
soll es euch gelten.~-- 11Alle reinen Vögel dürft ihr essen; 12folgende
aber sind es, von denen ihr nichts essen dürft: der Adler, der
Lämmergeier, der Bartgeier, 13die Weihe, der Habicht, die verschiedenen
Falkenarten, 14alle Arten von Raben, 15der Strauß, die
Schwalbe\textless sup title=``oder: der Kuckuck''\textgreater✲, die
Möwe, alle Habichtarten, 16das Käuzchen, der Uhu, die Eule, 17der
Pelikan, der Aasgeier, der Sturzpelikan\textless sup title=``oder:
Kormoran''\textgreater✲, 18der Storch, die verschiedenen Arten der
Regenpfeifer\textless sup title=``oder: Reiher''\textgreater✲, der
Wiedehopf und die Fledermaus. 19Auch alle geflügelten Insekten sollen
euch als unrein gelten und dürfen nicht gegessen werden. 20Alles reine
Geflügel dürft ihr essen.~-- 21Von gefallenen\textless sup title=``oder:
verendeten''\textgreater✲ Tieren dürft ihr nichts genießen; dem
Fremdling, der in deinen Ortschaften lebt, magst du sie zum Essen geben
oder magst sie an einen Nichtisraeliten verkaufen; denn du bist ein dem
HERRN, deinem Gott, geheiligtes Volk. -- Ein Böckchen darfst du nicht in
der Milch seiner Mutter kochen.«

\hypertarget{d-vorschriften-bezuxfcglich-der-ablieferung-des-zehnten-besonders-des-drittjahr-zehnten}{%
\paragraph{d) Vorschriften bezüglich der Ablieferung des Zehnten
(besonders des
Drittjahr-Zehnten)}\label{d-vorschriften-bezuxfcglich-der-ablieferung-des-zehnten-besonders-des-drittjahr-zehnten}}

22»Den ganzen Ertrag deiner Aussaat, alles, was dir auf dem Felde
wächst, sollst du Jahr für Jahr gewissenhaft verzehnten 23und sollst den
Zehnten deines Getreides, deines Weins und deines Öls sowie die
Erstgeburten deiner Rinder und deines Kleinviehs vor dem HERRN, deinem
Gott, an der Stätte, die er erwählen wird, um seinen Namen dort wohnen
zu lassen, verzehren, damit du den HERRN, deinen Gott, allezeit fürchten
lernst. 24Wenn dir aber der Weg zu weit ist, so daß du, wenn der HERR,
dein Gott, dich gesegnet hat, den Zehnten nicht hinbringen kannst, weil
die Stätte, die der HERR, dein Gott, erwählen wird, um seinen Namen
dorthin zu versetzen, zu weit von dir entfernt liegt: 25so mache den
Zehnten zu Geld, nimm dann das Geld wohlbewahrt mit dir und begib dich
an den Ort, den der HERR, dein Gott, erwählen wird. 26Dort gib das Geld
aus für alles, wonach dein Herz Verlangen tragen mag, für Rinder und
Kleinvieh, für Wein und starke Getränke, kurz für alles, wonach dich
gelüsten mag; halte dann dort vor dem HERRN, deinem Gott, ein Mahl und
sei mit deinen Angehörigen fröhlich. 27Dabei vergiß aber auch die
Leviten nicht, die in deinen Wohnorten leben; denn sie haben keinen
eigenen Landbesitz und kein Erbteil gleich dir\textless sup title=``vgl.
12,12''\textgreater✲.

28Nach Verlauf von je drei Jahren sollst du den gesamten Zehnten deines
Ertrags von jenem Jahre für sich besonders nehmen und ihn in deinen
Wohnorten niederlegen\textless sup title=``oder:
abliefern''\textgreater✲; 29dann sollen die Leviten, die ja keinen
eigenen Landbesitz und kein Erbteil gleich dir haben, sowie die
Fremdlinge, die Witwen und die Waisen, die in deinen Wohnorten leben,
herbeikommen und sich satt essen, damit der HERR, dein Gott, dich segne
bei aller Arbeit deiner Hände, bei allem, was du unternimmst.«

\hypertarget{e-vorschriften-bezuxfcglich-des-schuldenerlasses-in-jedem-siebten-jahre-und-der-freilassung-hebruxe4ischer-sklaven}{%
\paragraph{e) Vorschriften bezüglich des Schuldenerlasses in jedem
siebten Jahre und der Freilassung hebräischer
Sklaven}\label{e-vorschriften-bezuxfcglich-des-schuldenerlasses-in-jedem-siebten-jahre-und-der-freilassung-hebruxe4ischer-sklaven}}

\hypertarget{section-14}{%
\section{15}\label{section-14}}

1»Alle sieben Jahre sollst du einen Erlaß eintreten lassen; 2und mit dem
Erlaß soll es folgendermaßen gehalten werden: Jeder Gläubiger soll das
Handdarlehen, das er seinem Nächsten gewährt hat, erlassen; er soll
seinen Nächsten und besonders seinen Volksgenossen nicht drängen; denn
man hat einen Erlaß zu Ehren des HERRN ausgerufen. 3Einen
Nichtisraeliten magst du drängen; was du aber bei einem von deinen
Volksgenossen ausstehen hast, das sollst du aus deinem Besitz fahren
lassen. 4Es sollte zwar eigentlich keine Armen bei dir geben; denn der
HERR wird dich in dem Lande, das er dir als Erbteil zum Besitz geben
wird, reichlich segnen, 5wenn du nur den Weisungen des HERRN, deines
Gottes, willig gehorchst, indem du dieses ganze Gesetz genau
beobachtest, das ich dir heute gebiete. 6Denn der HERR, dein Gott, hat
dir, wie er dir zugesagt hat, Segen verliehen, so daß du vielen
Völkerschaften wirst leihen können, während du selbst nichts zu
entleihen brauchst, und daß du über viele Völkerschaften herrschen
wirst, während sie über dich nicht herrschen sollen.«

\hypertarget{empfehlung-der-bereitwilligkeit-zur-unterstuxfctzung-armer-volksgenossen-besonders-zum-leihen-von-geld}{%
\paragraph{Empfehlung der Bereitwilligkeit zur Unterstützung armer
Volksgenossen (besonders zum Leihen von
Geld)}\label{empfehlung-der-bereitwilligkeit-zur-unterstuxfctzung-armer-volksgenossen-besonders-zum-leihen-von-geld}}

7»Wenn sich bei dir ein Armer, irgendeiner von deinen Volksgenossen, in
einer deiner Ortschaften in deinem Lande befindet, das der HERR, dein
Gott, dir geben wird, so sollst du nicht hartherzig sein und deine Hand
gegenüber deinem armen Volksgenossen nicht verschließen, 8sondern sollst
deine Hand für ihn weit auftun und ihm bereitwillig leihen nach Maßgabe
des Bedürfnisses, soviel er nötig hat. 9Hüte dich wohl, in deinem Herzen
den nichtswürdigen Gedanken aufkommen zu lassen: ›Das siebte Jahr, das
Erlaßjahr, steht nahe bevor!‹, und sieh deinen armen Volksgenossen nicht
mit unfreundlichem Blick an, so daß du ihm nichts gibst und eine Sünde
auf dir lastet, wenn er den HERRN gegen dich anruft! 10Nein, du sollst
ihm bereitwillig geben, und dein Herz soll nicht in verdrießlicher
Stimmung sein, wenn du ihm gibst; denn um solcher Handlungsweise willen
wird der HERR, dein Gott, dich in allem segnen, was du tust und
unternimmst. 11Weil es an Armen inmitten des Landes niemals fehlen wird,
darum gebe ich dir das Gebot: ›Du sollst deine Hand für deinen dürftigen
und armen Volksgenossen in deinem Lande weit auftun!‹«

\hypertarget{vorschriften-bezuxfcglich-der-freilassung-und-reichlichen-ausstattung-hebruxe4ischer-sklaven}{%
\paragraph{Vorschriften bezüglich der Freilassung und reichlichen
Ausstattung hebräischer
Sklaven}\label{vorschriften-bezuxfcglich-der-freilassung-und-reichlichen-ausstattung-hebruxe4ischer-sklaven}}

12»Wenn einer deiner Volksgenossen, ein Hebräer oder eine Hebräerin,
sich dir verkauft\textless sup title=``oder: dir verkauft
wird''\textgreater✲, so soll er dir sechs Jahre lang dienen, aber im
siebten Jahre sollst du ihn als einen Freien von dir entlassen; 13und
wenn du ihn dann freiläßt, sollst du ihn nicht mit leeren Händen ziehen
lassen, 14sondern ihn gehörig ausstatten (mit Gaben) von deinem
Kleinvieh, von deiner Tenne und von deiner Kelter: von dem, womit der
HERR, dein Gott, dich gesegnet hat, sollst du ihm geben 15und sollst
bedenken, daß du selbst einst ein Knecht im Lande Ägypten gewesen bist
und daß der HERR, dein Gott, dich (aus der Knechtschaft) erlöst hat;
deshalb gebe ich dir heute dieses Gebot. 16Wenn er aber zu dir sagen
sollte: ›Ich möchte nicht von dir weggehen!‹ -- weil er dich und die
Deinen liebgewonnen hat, da er sich bei dir wohl fühlt --, 17so nimm
eine Pfrieme und durchbohre ihm damit das Ohr in die Tür hinein: dann
soll er für immer als Knecht in deinem Dienst bleiben; und auch mit
deiner Magd sollst du es so machen. 18Du darfst keine Härte darin sehen,
daß du ihn als einen Freien von dir fortlassen mußt; denn er hat dir
sechs Jahre lang das Doppelte des Lohnes eines Taglöhners erarbeitet,
und der HERR, dein Gott, wird dich dafür segnen in allem, was du
unternimmst.«

\hypertarget{f-vorschriften-bezuxfcglich-der-heiligung-der-fehlerfreien-muxe4nnlichen-erstgeburten-von-rindern-und-von-kleinvieh}{%
\paragraph{f) Vorschriften bezüglich der Heiligung der fehlerfreien
männlichen Erstgeburten von Rindern und von
Kleinvieh}\label{f-vorschriften-bezuxfcglich-der-heiligung-der-fehlerfreien-muxe4nnlichen-erstgeburten-von-rindern-und-von-kleinvieh}}

19»Jede männliche Erstgeburt, die unter deinem Rindvieh und deinem
Kleinvieh zur Welt kommt, sollst du dem HERRN, deinem Gott, weihen: du
darfst keines von deinen erstgeborenen Rindern zur Arbeit verwenden und
die Erstgeborenen deines Kleinviehs nicht scheren: 20vor dem HERRN,
deinem Gott, sollst du und deine Familie es Jahr für Jahr an der Stätte
verzehren, die der HERR erwählen wird. 21Wenn sich jedoch ein Gebrechen
an ihm findet, so daß es lahm oder blind ist oder sonst einen häßlichen
Fehler an sich hat, so sollst du es dem HERRN, deinem Gott, nicht
schlachten\textless sup title=``oder: opfern''\textgreater✲. 22In deinen
Wohnorten magst du es verzehren, der Unreine und der Reine ohne
Unterschied, wie das Fleisch der Gazelle und des Hirsches\textless sup
title=``vgl. 12,16''\textgreater✲. 23Nur sein Blut darfst du nicht
genießen: auf die Erde mußt du es wie Wasser schütten\textless sup
title=``oder: fließen lassen''\textgreater✲.«

\hypertarget{g-vorschriften-bezuxfcglich-der-drei-juxe4hrlichen-hauptfeste}{%
\paragraph{g) Vorschriften bezüglich der drei jährlichen
Hauptfeste}\label{g-vorschriften-bezuxfcglich-der-drei-juxe4hrlichen-hauptfeste}}

\hypertarget{section-15}{%
\section{16}\label{section-15}}

1»Beobachte den Monat Abib\textless sup title=``März/April; vgl. 2.Mose
13,4''\textgreater✲ und feiere das Passah zu Ehren des HERRN, deines
Gottes; denn im Monat Abib hat der HERR, dein Gott, dich bei Nacht aus
Ägypten hinausgeführt. 2Du sollst dann für den HERRN, deinen Gott, als
Passahopfer Kleinvieh und Rinder an der Stätte schlachten\textless sup
title=``oder: opfern''\textgreater✲, die der HERR erwählen wird, um
seinen Namen dort wohnen zu lassen. 3Du darfst nichts Gesäuertes
dazu\textless sup title=``d.h. zu ihm hinzu''\textgreater✲ essen: sieben
Tage lang sollst du ungesäuertes Brot als ›Elendskost‹\textless sup
title=``oder: Notstandsbrot''\textgreater✲ dazu genießen -- denn in
ängstlicher Eile bist du aus dem Lande Ägypten weggezogen --, damit du
an den Tag deines Auszugs aus dem Lande Ägypten zurückdenkst, solange du
lebst. 4Sieben Tage lang darf bei dir kein Sauerteig in deinem ganzen
Gebiet zu finden sein, und von dem Fleisch, das du am Abend des ersten
Tages schlachtest\textless sup title=``oder: opferst''\textgreater✲,
darf nichts über Nacht bis zum folgenden Morgen übrigbleiben. 5Du darfst
das Passah nicht in irgendeinem deiner Wohnorte schlachten\textless sup
title=``oder: opfern''\textgreater✲, die der HERR, dein Gott, dir gibt,
6sondern an der Stätte, die der HERR, dein Gott, erwählen wird, um
seinen Namen dort wohnen zu lassen: dort sollst du das Passah abends bei
Sonnenuntergang zu der Zeit deines Auszugs aus Ägypten schlachten 7und
sollst es kochen und essen an der Stätte, die der HERR, dein Gott,
erwählen wird; am folgenden Morgen aber sollst du umkehren\textless sup
title=``oder: dich aufmachen''\textgreater✲ und zu deinen
Zelten\textless sup title=``=~nach Hause''\textgreater✲ zurückkehren.
8Nachdem du dann sechs Tage lang ungesäuerte Brote gegessen hast, findet
am siebten Tage eine Festversammlung zu Ehren des HERRN, deines Gottes,
statt; da darfst du keine Arbeit verrichten.

9Sieben Wochen sollst du dir abzählen: von da an, wo man die Sichel
zuerst an die Saat legt\textless sup title=``oder: im Getreidefeld
anlegt''\textgreater✲, sollst du anfangen, sieben Wochen zu zählen,
10und sollst dann das Wochenfest zu Ehren des HERRN, deines Gottes, nach
Maßgabe der freiwilligen Gaben feiern, die du von deinem Besitz
darbringen wirst, je nachdem der HERR, dein Gott, dich segnet; 11und du
sollst mit deinen Söhnen und Töchtern, deinen Knechten und Mägden und
den Leviten, die in deinen Wohnorten leben, und den Fremdlingen, den
Waisen und Witwen, die bei dir wohnen, vor dem HERRN, deinem Gott,
fröhlich sein an der Stätte, die der HERR, dein Gott, erwählen wird, um
seinen Namen dort wohnen zu lassen. 12Dabei sollst du daran gedenken,
daß du (einst) ein Knecht in Ägypten gewesen bist, und sollst diese
Satzungen gewissenhaft beobachten. 13Das Laubhüttenfest sollst du sieben
Tage lang, wenn du die Obstlese hältst, von dem Ertrag deiner Tenne und
Kelter feiern 14und sollst an diesem deinem Feste mit deinen Söhnen und
Töchtern, deinen Knechten und Mägden und den Leviten, sowie mit den
Fremdlingen, den Waisen und Witwen, die in deinen Wohnorten leben,
fröhlich sein: 15sieben Tage lang sollst du das Fest zu Ehren des HERRN,
deines Gottes, an der Stätte feiern, die der HERR erwählen wird; denn
der HERR wird dich bei deinem ganzen Ernteertrag und bei der ganzen
Arbeit deiner Hände segnen; darum sollst du dich durchaus der Freude
hingeben!«

\hypertarget{zusammenfassung-und-abschluuxdf}{%
\paragraph{Zusammenfassung und
Abschluß}\label{zusammenfassung-und-abschluuxdf}}

16»Dreimal im Jahr sollen alle Personen männlichen Geschlechts bei dir
vor dem HERRN, deinem Gott, an der Stätte erscheinen, die er erwählen
wird, nämlich am Fest der ungesäuerten Brote, am Wochenfest und am
Laubhüttenfest. Man soll aber vor dem HERRN nicht mit leeren Händen
erscheinen, 17sondern jeder mit dem, was er zu geben vermag nach Maßgabe
des Segens, den der HERR, dein Gott, dir beschert hat.«

\hypertarget{h-vorschriften-bezuxfcglich-der-rechtspflege-verbot-und-bestrafung-des-guxf6tzendienstes}{%
\paragraph{h) Vorschriften bezüglich der Rechtspflege; Verbot und
Bestrafung des
Götzendienstes}\label{h-vorschriften-bezuxfcglich-der-rechtspflege-verbot-und-bestrafung-des-guxf6tzendienstes}}

18»Richter und Obmänner\textless sup title=``vgl. 1,15''\textgreater✲
sollst du dir in allen deinen Ortschaften, die der HERR, dein Gott, dir
in jedem deiner Stämme gibt, einsetzen, damit sie dem Volke mit
Gerechtigkeit Recht sprechen. 19Du darfst das Recht nicht beugen, darfst
die Person nicht ansehen und Geschenke✲ nicht annehmen; denn Geschenke
machen die Augen der Weisesten blind und bringen die Sache derer, die im
Recht sind, zu Fall. 20Der Gerechtigkeit allein sollst du die Ehre
geben, damit du am Leben bleibst und das Land im Besitz behältst, das
der HERR, dein Gott, dir geben wird.«

\hypertarget{verbot-guxf6tzendienerischer-bruxe4uche-und-fehlerhafter-opfertiere}{%
\paragraph{Verbot götzendienerischer Bräuche und fehlerhafter
Opfertiere}\label{verbot-guxf6tzendienerischer-bruxe4uche-und-fehlerhafter-opfertiere}}

21»Du sollst dir neben dem Altar, den du dir für den HERRN, deinen Gott,
errichten wirst, keinen Baumstamm irgendwelcher Art als Götzenbaum
pflanzen 22und dir keinen Malstein aufstellen, weil der HERR, dein Gott,
ihn haßt\textless sup title=``vgl. 2.Mose 34,13''\textgreater✲.-

\hypertarget{section-16}{%
\section{17}\label{section-16}}

1Du sollst dem HERRN, deinem Gott, kein Rind und kein Stück Kleinvieh
opfern, das einen Fehler, irgend etwas Häßliches, an sich hat; denn das
ist ein Greuel für den HERRN, deinen Gott.«

\hypertarget{bestrafung-des-guxf6tzendienstes}{%
\paragraph{Bestrafung des
Götzendienstes}\label{bestrafung-des-guxf6tzendienstes}}

2»Wenn in deiner Mitte, in einem deiner Wohnorte, die der HERR, dein
Gott, dir geben wird, ein Mann oder eine Frau sich findet, die das tun,
was dem HERRN, deinem Gott, mißfällt, indem sie seinen Bund übertreten,
3so daß sie hingehen und anderen Göttern dienen und sich vor ihnen und
besonders vor der Sonne oder vor dem Mond oder vor dem ganzen
Sternenheere des Himmels niederwerfen, was ich verboten habe, 4und es
dir angezeigt wird und du es erfährst, so sollst du eine genaue
Untersuchung anstellen. Wenn sich dann der Sachbericht als zutreffend
herausstellt und solcher Greuel in Israel wirklich verübt worden ist,
5so sollst du den betreffenden Mann oder jene Frau, die etwas so
Schlimmes begangen haben, zu deinen Toren hinausführen, den Mann oder
die Frau, und sie zu Tode steinigen. 6Auf die Aussage von zwei oder drei
Zeugen hin soll ein solcher, der sterben muß, den Tod erleiden; auf die
Aussage eines einzigen Zeugen hin darf er nicht getötet werden. 7Die
Hand des Zeugen soll die erste sein, die sich zu seiner Tötung erhebt,
danach die Hand des ganzen übrigen Volkes: so sollst du das Böse aus
deiner Mitte beseitigen!«

\hypertarget{einsetzung-eines-obergerichts-am-heiligtum-fuxfcr-schwierigere-rechtsfuxe4lle}{%
\paragraph{Einsetzung eines Obergerichts am Heiligtum für schwierigere
Rechtsfälle}\label{einsetzung-eines-obergerichts-am-heiligtum-fuxfcr-schwierigere-rechtsfuxe4lle}}

8»Wenn die Entscheidung einer Rechtssache, bei der es sich um
Blutvergießen, um Eigentumsfragen\textless sup title=``oder:
Rechtsansprüche''\textgreater✲, um tödliche Mißhandlung, überhaupt um
irgendwelche Streitsachen in deinen Wohnorten handelt, für dich zu
schwierig ist, so sollst du dich aufmachen und dich an den Ort begeben,
den der HERR, dein Gott, erwählen wird. 9Wende dich dann dort an die
levitischen Priester und an den Richter, der zu jener Zeit im Amt sein
wird, und frage bei ihnen an: sie werden dir dann den Rechtsspruch
kundtun. 10Du sollst dich alsdann an den Spruch halten, den sie dir von
jenem Orte aus, den der HERR erwählen wird, kundtun werden, und sollst
genau nach ihrer Anweisung verfahren: 11nach Maßgabe der Weisung, die
sie dir geben, und nach der Rechtsentscheidung, die sie dir mitteilen
werden, sollst du handeln, ohne von dem Wortlaut, den sie dir
verkündigen werden, nach rechts oder links abzuweichen. 12Sollte aber
jemand sich so vermessen benehmen, daß er auf den Priester, der im Amt
ist, um den Dienst des HERRN, deines Gottes, daselbst zu verrichten,
oder auf den Richter nicht hören will: ein solcher Mensch soll sterben!
So sollst du das Böse aus Israel beseitigen; 13das ganze Volk aber soll
es erfahren, damit es sich fürchte und fernerhin nicht mehr vermessen
handle.«

\hypertarget{i-das-kuxf6nigsgesetz}{%
\paragraph{i) Das Königsgesetz}\label{i-das-kuxf6nigsgesetz}}

14»Wenn du in das Land gekommen bist, das der HERR, dein Gott, dir geben
wird, und es in Besitz genommen hast und darin wohnst und dann denkst:
›Ich will einen König über mich setzen, wie alle Völkerschaften rings um
mich her‹, 15so magst du immerhin einen solchen König über dich setzen,
den der HERR, dein Gott, erwählen wird: aus der Mitte deiner
Volksgenossen sollst du einen König über dich setzen; einen
nichtisraelitischen Mann, der nicht dein Bruder ist\textless sup
title=``d.h. nicht zu deinem Volk gehört''\textgreater✲, darfst du nicht
über dich setzen. 16Nur darf er sich nicht viele Rosse anschaffen und
darf das Volk nicht nach Ägypten zurückführen, um sich viele Rosse
anzuschaffen; denn der HERR hat zu euch gesagt: ›Ihr dürft auf diesem
Wege nie wieder zurückkehren!‹ 17Auch soll er sich nicht viele Frauen
nehmen, damit sein Herz sich nicht (vom HERRN) abwendet; auch Silber und
Gold soll er sich nicht im Übermaß sammeln. 18Und wenn er den
Königsthron bestiegen hat, soll er sich eine Abschrift dieses Gesetzes
aus dem Buche, das sich unter der Aufsicht der levitischen Priester
befindet, in ein Buch schreiben\textless sup title=``oder: schreiben
lassen''\textgreater✲. 19Dieses soll er immer bei sich haben und soll
täglich darin lesen, solange er lebt, um den HERRN, seinen Gott,
fürchten zu lernen und alle Vorschriften dieses Gesetzes und diese
Verordnungen gewissenhaft zu beobachten, 20damit er sich in seinem
Herzen nicht über seine Volksgenossen erhebt und damit er von dem Gebot
weder nach rechts noch nach links abweicht, auf daß er samt seinen
Söhnen lange Tage in seiner Königsherrschaft inmitten Israels
verbleibt.«

\hypertarget{k-einkuxfcnfte-und-rechte-der-levitischen-im-heiligtum-amtierenden-priester-und-uxfcberhaupt-der-leviten}{%
\paragraph{k) Einkünfte und Rechte der levitischen, im Heiligtum
amtierenden Priester und überhaupt der
Leviten}\label{k-einkuxfcnfte-und-rechte-der-levitischen-im-heiligtum-amtierenden-priester-und-uxfcberhaupt-der-leviten}}

\hypertarget{section-17}{%
\section{18}\label{section-17}}

1»Die levitischen Priester, der ganze Stamm Levi, sollen keinen eigenen
Landbesitz und kein Erbteil wie die übrigen Israeliten haben: von den
Feueropfern des HERRN und den ihm als Gebühr zustehenden Abgaben sollen
sie ihren Unterhalt haben. 2Aber eigenen Erbbesitz soll dieser Stamm
inmitten seiner Volksgenossen nicht haben: der HERR ist sein Erbbesitz,
wie er ihm zugesagt hat. 3Folgendes ist es aber, was der Priester vom
Volk, nämlich von denen zu beanspruchen hat, die ein Schlachtopfer
darbringen, sei es ein Rind oder ein Stück Kleinvieh: man soll davon dem
Priester den Bug\textless sup title=``=~das Vorderbein, oder: die
Vorderkeule''\textgreater✲ und die beiden Kinnbacken und den Magen
geben. 4Die Erstlinge von deinem Getreide, deinem Wein und deinem Öl und
die Erstlinge von der Schur deines Kleinviehs sollst du ihm geben; 5denn
ihn hat der HERR, dein Gott, aus allen deinen Stämmen erwählt, damit er
und seine Söhne allezeit zur Verfügung stehen, um den priesterlichen
Dienst im Namen des HERRN zu verrichten.~-- 6Und wenn ein Levit aus
irgendeiner deiner Ortschaften, aus ganz Israel, wo er sich als
Fremdling aufhält, an die Stätte kommt, die der HERR erwählen wird -- es
steht aber ganz in seinem Belieben, ob er kommen will --, 7so darf er im
Namen des HERRN, seines Gottes, den Dienst verrichten wie alle seine
Brüder, die Leviten, die dort im Dienst des HERRN stehen; 8den gleichen
Anteil (wie diese) sollen sie (an den Einkünften) zu ihrem Unterhalt
haben, abgesehen von dem Erlös aus seinem väterlichen Vermögen✲.«

\hypertarget{l-verordnungen-bezuxfcglich-wahrsagerei-und-zauberei-und-verheiuxdfung-echten-prophetentums-mit-angabe-seiner-kennzeichen}{%
\paragraph{l) Verordnungen bezüglich Wahrsagerei und Zauberei, und
Verheißung echten Prophetentums mit Angabe seiner
Kennzeichen}\label{l-verordnungen-bezuxfcglich-wahrsagerei-und-zauberei-und-verheiuxdfung-echten-prophetentums-mit-angabe-seiner-kennzeichen}}

9»Wenn du in das Land kommst, das der HERR, dein Gott, dir geben wird,
so sollst du dich nicht daran gewöhnen, die Greuel der dortigen
Völkerschaften nachzuahmen. 10Es soll sich niemand in deiner Mitte
finden, der seinen Sohn oder seine Tochter als Opfer verbrennen läßt,
niemand, der Wahrsagerei, Zeichendeuterei oder Beschwörungskünste und
Zauberei treibt, 11niemand, der Geister bannt oder Totengeister
beschwört, keiner, der einen Wahrsagegeist befragt oder sich an die
Toten wendet; 12denn ein jeder, der sich mit solchen Dingen befaßt, ist
für den HERRN ein Greuel, und um dieser Greuel willen vertreibt der
HERR, dein Gott, diese Völker vor dir her. 13Du sollst dem HERRN, deinem
Gott, gegenüber unsträflich dastehen! 14Denn diese Völkerschaften, die
du verdrängen wirst, hören auf Zeichendeuter und Wahrsager; dir aber
erlaubt der HERR, dein Gott, etwas Derartiges nicht. 15Einen Propheten
gleich mir wird der HERR, dein Gott, dir (jeweils) aus deiner Mitte, aus
deinen Volksgenossen, erstehen lassen: auf den sollt ihr hören! 16ganz
so, wie du den HERRN, deinen Gott, am Horeb am Tage der Versammlung
gebeten hast, als du sagtest: ›Ich möchte die Stimme des HERRN, meines
Gottes, nicht länger hören und dieses gewaltige Feuer nicht mehr sehen,
damit ich nicht sterbe!‹ 17Damals sagte der HERR zu mir: ›Sie haben mit
ihrer Bitte recht! 18Einen Propheten gleich dir will ich ihnen aus der
Mitte ihrer Volksgenossen erstehen lassen und will ihm meine Worte in
den Mund legen, und er soll ihnen alles verkünden, was ich ihm gebieten
werde. 19Wer alsdann meinen Worten, die er in meinem Namen verkünden
wird, nicht gehorcht, den will ich selbst dafür zur Rechenschaft ziehen.
20Sollte sich aber ein Prophet vermessen, in meinem Namen etwas zu
verkünden, dessen Verkündigung ich ihm nicht geboten habe, oder sollte
er im Namen anderer Götter reden: ein solcher Prophet soll sterben!‹
21Solltest du aber bei dir denken: ›Woran sollen wir das Wort erkennen,
das der HERR nicht geredet hat?‹, 22so wisse: Wenn das, was ein Prophet
im Namen des HERRN verkündet, nicht eintrifft und nicht in Erfüllung
geht, so ist das ein Wort, das der HERR nicht geredet hat; in
Vermessenheit hat der Prophet es ausgesprochen: dir braucht vor ihm
nicht bange zu sein!«

\hypertarget{m-aussonderung-von-drei-bis-sechs-freistuxe4dten-zur-milderung-der-blutrache}{%
\paragraph{m) Aussonderung von drei bis sechs Freistädten zur Milderung
der
Blutrache}\label{m-aussonderung-von-drei-bis-sechs-freistuxe4dten-zur-milderung-der-blutrache}}

\hypertarget{section-18}{%
\section{19}\label{section-18}}

1»Wenn der HERR, dein Gott, die Völkerschaften ausrottet, deren Land der
HERR, dein Gott, dir geben will, und du nach ihrer Vertreibung in ihren
Städten und Häusern wohnst, 2so sollst du dir in deinem Lande, das der
HERR, dein Gott, dir zum Besitz gibt, drei Städte aussondern. 3Du sollst
dir die Wege dahin in guten Stand setzen und das Gebiet deines Landes,
das der HERR, dein Gott, dir zu eigen geben wird, in drei
Teile\textless sup title=``oder: Bezirke''\textgreater✲ zerlegen; und
das soll dazu dienen, daß jeder Totschläger sich dorthin flüchten kann.
4Es soll aber für den Totschläger, der sich dorthin flüchten darf, um am
Leben zu bleiben, folgende Bestimmung gelten: Wer einen andern
unvorsätzlich erschlägt, ohne ihm von früher her feind gewesen zu
sein~-- 5wenn z.B. jemand mit einem andern in den Wald geht, um Holz zu
fällen, und seine Hand holt mit der Axt aus, um einen Baum umzuhauen,
und das Eisen fliegt vom Stiel ab und trifft den andern so, daß er
stirbt --: ein solcher soll in eine dieser Städte fliehen, um sein Leben
zu retten, 6damit nicht der Bluträcher, wenn er in leidenschaftliche
Erregung geraten ist, dem Totschläger nacheilt und ihn wegen der Länge
des Weges einholt und totschlägt, wiewohl er des Todes nicht schuldig
ist, weil er (dem andern) ja von früher her nicht feind gewesen war.
7Darum gebiete ich dir so: Du sollst dir drei Städte aussondern. 8Wenn
aber der HERR, dein Gott, dein Gebiet erweitert, wie er deinen Vätern
zugeschworen hat, und dir nach seiner deinen Vätern gegebenen Verheißung
das ganze Land zu eigen gegeben hat 9- sofern du nämlich auf die
Beobachtung aller dieser Gebote, die ich dir heute zur Pflicht mache,
bedacht bist, indem du den HERRN, deinen Gott, liebst und allezeit auf
seinen Wegen wandelst --, so sollst du dir zu diesen drei Städten noch
drei andere hinzufügen, 10damit in deinem Lande, das der HERR, dein
Gott, dir zu eigen geben wird, kein unschuldiges Blut vergossen wird und
dadurch Blutschuld auf dich kommt.~-- 11Wenn dagegen jemand einem andern
feind ist und ihm auflauert, ihn überfällt und niederschlägt, so daß er
stirbt, und er dann in eine dieser Städte flieht, 12so sollen die
Ältesten der Stadt, zu der er gehört, hinsenden und ihn von dort holen
lassen und ihn dem Bluträcher ausliefern, damit er den Tod erleidet.
13Du darfst keinen Blick des Mitleids für ihn haben, sondern sollst
unschuldig vergossenes Blut aus Israel hinwegschaffen: dann wird es dir
gut ergehen.«

\hypertarget{n-verbot-der-grenzverruxfcckung-vorschriften-bezuxfcglich-der-zeugenschaft-vor-gericht-und-der-bestrafung-falscher-zeugen}{%
\paragraph{n) Verbot der Grenzverrückung; Vorschriften bezüglich der
Zeugenschaft vor Gericht und der Bestrafung falscher
Zeugen}\label{n-verbot-der-grenzverruxfcckung-vorschriften-bezuxfcglich-der-zeugenschaft-vor-gericht-und-der-bestrafung-falscher-zeugen}}

14»Du sollst nicht die Grenze deines Nachbars, welche die Vorfahren
gezogen\textless sup title=``oder: abgesteckt''\textgreater✲ haben, in
deinem Erbbesitz verrücken, den du in dem Lande erhalten wirst, das der
HERR, dein Gott, dir zum Eigentum geben will.~--

15Es darf nicht ein einzelner Zeuge gegen jemand auftreten, wenn es sich
um irgendein Verbrechen oder irgendeine Verschuldung, um irgendein
Vergehen handelt, das jemand begehen kann; erst aufgrund der Aussage von
zwei oder von drei Zeugen soll eine Sache endgültig entschieden
werden.~-- 16Wenn ein gewissenloser Zeuge gegen jemand auftritt, um ihn
einer Übertretung des Gesetzes zu beschuldigen, 17so sollen die beiden
Männer, die den Rechtsstreit miteinander haben, vor den HERRN, vor die
derzeitigen Priester und die Richter treten. 18Dann sollen die Richter
die Sache gründlich untersuchen, und wenn es sich herausstellt, daß der
Zeuge ein lügnerischer Zeuge ist, daß er die Unwahrheit gegen seinen
Volksgenossen ausgesagt hat, 19so sollt ihr dieselbe Strafe über ihn
verhängen, die er über seinen Volksgenossen zu bringen gedachte: so
sollst du das Böse aus deiner Mitte beseitigen. 20Die Übrigen aber
sollen es erfahren, damit sie in Furcht geraten und hinfort eine
derartige Schlechtigkeit in deiner Mitte nicht wieder verüben. 21Und du
sollst keinen Blick des Mitleids (für den Betreffenden) haben: Leben um
Leben, Auge um Auge, Zahn um Zahn, Hand um Hand, Fuß um Fuß!«

\hypertarget{o-die-kriegsgesetze}{%
\paragraph{o) Die Kriegsgesetze}\label{o-die-kriegsgesetze}}

\hypertarget{aa-verhalten-gegen-die-eigenen-leute-gesetze-uxfcber-befreiung-vom-kriegsdienst}{%
\subparagraph{aa) Verhalten gegen die eigenen Leute; Gesetze über
Befreiung vom
Kriegsdienst}\label{aa-verhalten-gegen-die-eigenen-leute-gesetze-uxfcber-befreiung-vom-kriegsdienst}}

\hypertarget{section-19}{%
\section{20}\label{section-19}}

1»Wenn du zum Krieg gegen deine Feinde ausziehst und Rosse und
Kriegswagen, ein dir an Zahl überlegenes Heer erblickst, so fürchte dich
nicht vor ihnen! Denn der HERR, dein Gott, der dich aus dem Lande
Ägypten hergeführt hat, ist mit dir. 2Und wenn ihr zum
Kriege\textless sup title=``oder: zur Schlacht?''\textgreater✲ ausrückt,
so soll der Priester vortreten und zum Volk so sprechen: 3›Höre, Israel!
Ihr zieht heute in den Kampf gegen eure Feinde: euer Herz werde nicht
verzagt! Fürchtet euch nicht und seid ohne Angst und erschreckt nicht
vor ihnen! 4Denn der HERR, euer Gott, ist es, der mit euch zieht, um für
euch mit euren Feinden zu kämpfen und euch den Sieg zu verleihen!‹
5Hierauf sollen die Obmänner\textless sup title=``vgl.
1,15''\textgreater✲ zu dem Kriegsvolk folgendes sagen: ›Ist jemand unter
euch, der ein neues Haus gebaut und es noch nicht eingeweiht hat? Der
trete ab und kehre zu seinem Hause heim, sonst könnte er im Kriege
umkommen und ein anderer das Haus einweihen. 6Ist ferner jemand unter
euch, der einen Weinberg angelegt und ihn noch nicht in Nutznießung
genommen hat? Der trete ab und kehre zu seinem Hause heim, sonst könnte
er im Kriege umkommen und ein anderer ihn in Nutznießung nehmen. 7Ist
ferner jemand da, der sich mit einem Weibe verlobt, sie aber noch nicht
geheiratet hat? Der trete ab und kehre zu seinem Hause heim, sonst
könnte er im Kriege umkommen und ein anderer die Braut heimführen.‹
8Dann sollen die Obmänner\textless sup title=``vgl. 1,15''\textgreater✲
weiter zu dem Kriegsvolk sagen: ›Ist jemand unter euch, der Angst hat
und verzagten Herzens ist? Der trete ab und kehre zu seinem Hause heim,
damit er seine Volksgenossen nicht ebenso mutlos macht, wie er selbst
ist.‹ 9Wenn dann die Obmänner mit ihrer Ansprache an das Kriegsvolk
fertig sind, soll man Anführer an die Spitze des Kriegsvolkes stellen.«

\hypertarget{bb-verhalten-gegen-die-feinde-besonders-bei-stuxe4dtebelagerungen}{%
\subparagraph{bb) Verhalten gegen die Feinde (besonders bei
Städtebelagerungen)}\label{bb-verhalten-gegen-die-feinde-besonders-bei-stuxe4dtebelagerungen}}

10»Wenn du gegen eine Stadt heranziehst, um sie zu belagern, so sollst
du sie zu einem friedlichen Abkommen auffordern. 11Antwortet sie dir
dann in friedfertiger Weise und öffnet sie dir freiwillig die Tore, so
soll die ganze Bevölkerung, die sich darin befindet, dir fronpflichtig
und dienstbar werden. 12Will sie aber auf ein friedliches Abkommen mit
dir nicht eingehen, sondern Krieg mit dir führen, so sollst du sie
belagern; 13und wenn der HERR, dein Gott, sie in deine Gewalt gibt, so
sollst du alle männlichen Personen in ihr mit der Schärfe des Schwerts
niederhauen; 14jedoch die Weiber und Kinder, das Vieh und alles, was
sonst in der Stadt ist, ihre gesamte Beute, sollst du für dich als
geraubtes Gut hinnehmen und über das bei deinen Feinden Erbeutete, das
der HERR, dein Gott, dir gegeben hat, frei verfügen. 15So sollst du es
mit allen Städten halten, die in sehr weiter Entfernung von dir liegen
und die nicht zu den Städten der hiesigen Völkerschaften gehören.
16Dagegen von den Städten der hiesigen Völker, die der HERR, dein Gott,
dir zu eigen gibt, darfst du nichts, was Odem hat, am Leben lassen,
17sondern mußt den Bann unerbittlich an ihnen vollstrecken, nämlich an
den Hethitern und Amoritern, den Kanaanäern und Pherissitern, den
Hewitern und Jebusitern, wie der HERR, dein Gott, dir geboten hat,
18damit sie euch nicht zur Nachahmung all ihrer Greuel verleiten, die
sie im Dienst ihrer Götter verübt haben, und ihr euch nicht gegen den
HERRN, euren Gott, versündigt.

19Wenn du eine Stadt lange Zeit belagern mußt, um sie mit Waffengewalt
zu erobern, so sollst du die zu ihr gehörenden Bäume nicht verderben,
indem du die Axt an sie legst; sondern genieße ihre Früchte, sie selbst
aber sollst du nicht umhauen; denn sind etwa die Bäume des Feldes
Menschen, daß sie durch dich in Belagerungszustand versetzt werden
müßten? 20Nur solche Bäume, von denen du weißt, daß sie keine eßbaren
Früchte tragen, die darfst du vernichten und umhauen und magst von ihnen
gegen die Stadt, die mit dir im Kriege liegt, Belagerungswerke
aufführen, bis sie gefallen ist.«

\hypertarget{p-suxfchnung-eines-von-unbekannter-hand-veruxfcbten-mordes}{%
\paragraph{p) Sühnung eines von unbekannter Hand verübten
Mordes}\label{p-suxfchnung-eines-von-unbekannter-hand-veruxfcbten-mordes}}

\hypertarget{section-20}{%
\section{21}\label{section-20}}

1»Wenn man in dem Lande, das der HERR, dein Gott, dir zum Eigentum gibt,
einen Erschlagenen auf dem Felde liegend findet, von dem nicht bekannt
ist, wer ihn erschlagen hat, 2so sollen deine Ältesten und deine Richter
hinausgehen und die Entfernungen bis zu den Ortschaften abmessen, die
rings um den Erschlagenen liegen. 3Dann sollen die Ältesten derjenigen
Ortschaft, die dem Erschlagenen am nächsten liegt, eine junge Kuh
nehmen, die noch nicht zur Arbeit benutzt und noch nie ins Joch gespannt
worden ist, 4und die Ältesten der betreffenden Ortschaft sollen die Kuh
zu einem immerfließenden Bach hinführen, in dem nicht gearbeitet und an
dem nicht gesät wird, und sollen der Kuh dort das Genick brechen, so daß
das Blut in den Bach hineinfließt. 5Hierauf sollen die Priester vom
Stamm Levi herantreten; denn sie hat der HERR, dein Gott, erwählt, damit
sie ihm dienen und im Namen des HERRN segnen, und nach ihrem Ausspruch
soll bei jedem Rechtshandel und jedem Verbrechen verfahren werden. 6Dann
sollen alle Ältesten der betreffenden Ortschaft, weil sie dem
Erschlagenen am nächsten wohnen, über der Kuh, der man das Genick in den
Bach hinein gebrochen hat, ihre Hände waschen 7und mit erhobener Stimme
aussagen: ›Unsere Hände haben dieses Blut nicht vergossen, und unsere
Augen haben nichts von der Tat gesehen! 8Vergib, o HERR, deinem Volke
Israel, das du erlöst hast, und mache dein Volk Israel nicht für
unschuldig in seiner Mitte vergossenes Blut verantwortlich!‹ Dann wird
die Blutschuld für sie gesühnt sein! 9So sollst du das unschuldig
vergossene Blut aus deiner Mitte wegschaffen, indem du das tust, was in
den Augen des HERRN das Richtige ist.«

\hypertarget{q-allerlei-familienrechtliche-bestimmungen-und-menschliche-pflichten}{%
\paragraph{q) Allerlei familienrechtliche Bestimmungen und menschliche
Pflichten}\label{q-allerlei-familienrechtliche-bestimmungen-und-menschliche-pflichten}}

\hypertarget{aa-verehelichung-mit-einer-kriegsgefangenen-frau}{%
\subparagraph{aa) Verehelichung mit einer kriegsgefangenen
Frau}\label{aa-verehelichung-mit-einer-kriegsgefangenen-frau}}

10»Wenn du zum Kriege gegen deine Feinde ausziehst und der HERR, dein
Gott, sie in deine Gewalt gibt und du Gefangene von ihnen erbeutest
11und du unter den Gefangenen ein Weib von schöner Gestalt siehst und
sie liebgewinnst, so daß du sie zur Frau nehmen möchtest, 12so sollst du
sie in dein Haus hineinführen; sie schere sich dann das Haupt,
beschneide ihre Nägel, 13lege die Kleidung ab, die sie als Gefangene
getragen hat, bleibe in deinem Hause und betrauere ihre Eltern einen
Monat lang; danach darfst du zu ihr eingehen und die Ehe mit ihr
vollziehen, und sie darf als deine Frau gelten. 14Wenn du dich aber
nicht mehr zu ihr hingezogen fühlst, so hast du sie gehen zu lassen,
wohin es ihr beliebt; aber für Geld darfst du sie keinesfalls verkaufen,
darfst sie auch nicht gewalttätig (als Sklavin) behandeln, weil du
ehelich mit ihr gelebt hast.«

\hypertarget{bb-sicherstellung-der-erbrechte-des-erstgeborenen}{%
\subparagraph{bb) Sicherstellung der Erbrechte des
Erstgeborenen}\label{bb-sicherstellung-der-erbrechte-des-erstgeborenen}}

15»Wenn ein Mann zwei Frauen hat, von denen ihm die eine lieb, die
andere ungeliebt ist, und sie beide ihm Söhne gebären, die geliebte wie
die ungeliebte, und der erstgeborene Sohn von der ungeliebten Frau
stammt, 16so darf er an dem Tage, an welchem er sein Vermögen an seine
Söhne als Erbgut verteilt, nicht dem Sohne der geliebten Frau die Rechte
der Erstgeburt verleihen zum Schaden des Sohnes der ungeliebten, welcher
doch tatsächlich der Erstgeborene ist; 17sondern er muß den
Erstgeborenen, den Sohn der ungeliebten Frau, als solchen anerkennen,
indem er ihm zwei Teile\textless sup title=``=~den doppelten
Anteil''\textgreater✲ von seinem gesamten Vermögen überweist; denn
dieser ist der Erstling seiner Kraft\textless sup title=``vgl. 1.Mose
49,3''\textgreater✲: ihm steht das Erstgeburtsrecht zu.«

\hypertarget{cc-mauxdfregeln-gegen-widerspenstige-suxf6hne}{%
\subparagraph{cc) Maßregeln gegen widerspenstige
Söhne}\label{cc-mauxdfregeln-gegen-widerspenstige-suxf6hne}}

18»Wenn jemand einen störrischen und widerspenstigen Sohn hat, der auf
die Mahnungen seines Vaters und seiner Mutter nicht hört und ihnen trotz
aller Zurechtweisungen\textless sup title=``oder:
Züchtigungen''\textgreater✲ nicht gehorcht, 19so sollen seine Eltern ihn
ergreifen und ihn vor die Ältesten der betreffenden Ortschaft und zwar
an das Tor des betreffenden Ortes führen 20und sollen zu den Ältesten
der Ortschaft sagen: ›Dieser unser Sohn ist störrisch und widerspenstig;
er hört nicht auf unsere Mahnungen, ist ein Verschwender und Trinker!‹
21Dann sollen alle Männer der betreffenden Ortschaft ihn zu Tode
steinigen. So sollst du das Böse aus deiner Mitte wegschaffen, und alle
Israeliten sollen es erfahren und es sich zur Warnung dienen lassen.«

\hypertarget{dd-behandlung-des-uxf6ffentlich-aufgehuxe4ngten-leichnams-eines-hingerichteten-verbrechers}{%
\subparagraph{dd) Behandlung des öffentlich aufgehängten Leichnams eines
hingerichteten
Verbrechers}\label{dd-behandlung-des-uxf6ffentlich-aufgehuxe4ngten-leichnams-eines-hingerichteten-verbrechers}}

22»Wenn jemand ein todeswürdiges Verbrechen begangen hat und man ihn
nach seiner Tötung\textless sup title=``oder: Hinrichtung''\textgreater✲
an einen Baum hängt, 23so soll sein Leichnam nicht über Nacht an dem
Baume hängen bleiben, sondern du sollst ihn unbedingt noch an demselben
Tage begraben; denn ein Gehenkter ist von Gott verflucht, und du darfst
dein Land, das der HERR, dein Gott, dir zum Eigentum geben will, nicht
verunreinigen.«

\hypertarget{ee-einige-liebespflichten-in-notlagen-des-nuxe4chsten-ruxfcckgabe-von-fundgegenstuxe4nden}{%
\subparagraph{ee) Einige Liebespflichten in Notlagen des Nächsten;
Rückgabe von
Fundgegenständen}\label{ee-einige-liebespflichten-in-notlagen-des-nuxe4chsten-ruxfcckgabe-von-fundgegenstuxe4nden}}

\hypertarget{section-21}{%
\section{22}\label{section-21}}

1»Wenn du siehst, daß das Rind eines deiner Volksgenossen oder ein Stück
seines Kleinviehs sich verlaufen hat, so sollst du ihnen deine Hilfe
nicht versagen, sollst sie vielmehr deinem Volksgenossen zurückbringen.
2Wenn aber dein Volksgenosse nicht in deiner Nähe wohnt oder du ihn
nicht kennst, so sollst du das Tier in dein Haus aufnehmen, und es soll
bei dir bleiben, bis dein Volksgenosse es sucht: dann gib es ihm zurück.
3Ebenso sollst du es mit seinem Esel und ebenso mit einem Kleidungsstück
von ihm machen, überhaupt mit allem, was einem von deinen Volksgenossen
verlorengeht oder ihm abhanden kommt und was du findest: du darfst ihm
deine Hilfe nicht versagen.~-- 4Wenn du den Esel oder das Rind eines
deiner Volksgenossen auf dem Wege zusammengebrochen daliegen siehst,
sollst du ihm deine Hilfe nicht versagen, vielmehr sollst du das Tier
mit ihm wieder auf die Beine bringen.«

\hypertarget{ff-verschiedene-vorschriften-besonders-bezuxfcglich-vermeidung-von-naturwidrigkeiten}{%
\subparagraph{ff) Verschiedene Vorschriften, besonders bezüglich
Vermeidung von
Naturwidrigkeiten}\label{ff-verschiedene-vorschriften-besonders-bezuxfcglich-vermeidung-von-naturwidrigkeiten}}

5»Eine Frau soll keine Männerkleider tragen und ein Mann keine
Frauenkleider anziehen; denn wer dieses tut, ist für den HERRN, deinen
Gott, ein Greuel.~--

6Wenn dir bei einer Wanderung ein Vogelnest auf irgendeinem Baum oder
auf der Erde mit Jungen oder mit Eiern zu Gesicht kommt und die
Vogelmutter auf den Jungen oder auf den Eiern sitzt, so sollst du nicht
die Vogelmutter samt den Jungen nehmen; 7laß vielmehr die Mutter fliegen
und nimm dir nur die Jungen, damit es dir wohl ergeht und du lange
lebst.~--

8Wenn du ein neues Haus baust, so sollst du ein Geländer an deinem Dach
anbringen, damit du keine Blutschuld auf dein Haus bringst, wenn jemand
von ihm abstürzen sollte.~--

9Du darfst deinen Weinberg nicht mit zweierlei Gewächsen
bepflanzen\textless sup title=``vgl. 3.Mose 19,19''\textgreater✲, damit
nicht der volle Weinbergertrag, sowohl die Anpflanzung, die du gemacht
hast, als auch der Ertrag des Weinbergs, dem Heiligtum verfällt.~-- 10Du
sollst nicht mit einem Ochsen und einem Esel zusammen ackern.~-- 11Du
sollst kein Zeug von verschiedenartigen Stoffen anziehen, das aus Wolle
und Leinen zusammen hergestellt ist.~-- 12An den vier Zipfeln deines
Obergewandes, mit dem du dich umhüllst, sollst du dir Quasten
anbringen.«\textless sup title=``4.Mose 15,37-38''\textgreater✲

\hypertarget{r-behandlung-von-unzuchtsklagen-sittlichkeitsgesetze}{%
\paragraph{r) Behandlung von Unzuchtsklagen;
Sittlichkeitsgesetze}\label{r-behandlung-von-unzuchtsklagen-sittlichkeitsgesetze}}

\hypertarget{aa-wie-eine-von-ihrem-mann-des-verlusts-der-jungfruxe4ulichkeit-vor-der-ehe-beschuldigte-frau-zu-behandeln-sei}{%
\subparagraph{aa) Wie eine von ihrem Mann des Verlusts der
Jungfräulichkeit vor der Ehe beschuldigte Frau zu behandeln
sei}\label{aa-wie-eine-von-ihrem-mann-des-verlusts-der-jungfruxe4ulichkeit-vor-der-ehe-beschuldigte-frau-zu-behandeln-sei}}

13»Wenn ein Mann eine Frau geheiratet und mit ihr ehelich gelebt hat,
dann aber ihrer überdrüssig wird 14und ihr schandbare Dinge, die nur
Gerede sind, zur Last legt und sie in üblen Ruf bringt, indem er sagt:
›Diese Frau habe ich geheiratet, aber als ich mich ihr nahte, habe ich
die Zeichen der Jungfräulichkeit nicht an ihr gefunden‹, 15so sollen die
Eltern der jungen Frau die Zeichen der Jungfräulichkeit der jungen Frau
nehmen und vor die Ältesten der Ortschaft an das Tor hinausbringen;
16dann soll der Vater der jungen Frau zu den Ältesten sagen: ›Ich habe
meine Tochter diesem Mann zur Frau gegeben, aber er ist ihrer
überdrüssig geworden 17und legt ihr nun schandbare Dinge, die nichts als
Gerede sind, zur Last, indem er behauptet: Ich habe an deiner Tochter
die Zeichen der Jungfräulichkeit nicht gefunden --, und hier sind doch
die Beweise für die Jungfräulichkeit meiner Tochter!‹ Und sie (die
Eltern) sollen dabei das betreffende Stück Zeug vor den Ältesten der
Ortschaft ausbreiten. 18Hierauf sollen die Ältesten jener Ortschaft den
Mann ergreifen und ihn züchtigen; 19auch sollen sie ihm eine Geldstrafe
von hundert Silberschekeln auferlegen und diese dem Vater der jungen
Frau geben, weil er eine israelitische Jungfrau in üblen Ruf gebracht
hat. Auch soll sie ihm dann als Frau angehören, die er zeitlebens nicht
entlassen kann. 20Wenn aber jene Behauptung: ›Es ist kein Zeichen der
Jungfräulichkeit an der jungen Frau gefunden worden‹ auf Wahrheit
beruht, 21so soll man die junge Frau an den Eingang ihres Vaterhauses
führen, und die Männer der betreffenden Ortschaft sollen sie zu Tode
steinigen, weil sie eine Schandtat in Israel verübt hat, indem sie in
ihrem Vaterhause Unzucht trieb. So sollst du das Böse aus deiner Mitte
wegschaffen!«

\hypertarget{bb-vorschriften-gegen-ehebruch-schuxe4ndung-eines-verlobten-muxe4dchens-und-vergewaltigung-einer-unverlobten-jungfrau}{%
\subparagraph{bb) Vorschriften gegen Ehebruch, Schändung eines verlobten
Mädchens und Vergewaltigung einer unverlobten
Jungfrau}\label{bb-vorschriften-gegen-ehebruch-schuxe4ndung-eines-verlobten-muxe4dchens-und-vergewaltigung-einer-unverlobten-jungfrau}}

22»Wird ein Mann im Ehebruch mit der Ehefrau eines andern ertappt, so
sollen sie alle beide sterben, der Mann, der sich mit der Frau vergangen
hat, und die Frau\textless sup title=``vgl. 3.Mose 20,10''\textgreater✲.
So sollst du das Böse aus Israel wegschaffen!

23Wenn ein Mädchen, eine Jungfrau, einem Manne verlobt ist und jemand
sie innerhalb der Ortschaft trifft und ihr beiwohnt, 24so sollt ihr sie
beide zum Tor der betreffenden Ortschaft hinausführen und sie zu Tode
steinigen: das Mädchen deshalb, weil sie in der Ortschaft nicht um Hilfe
geschrien hat, und den Mann deshalb, weil er die Braut eines andern
entehrt hat. So sollst du das Böse aus deiner Mitte wegschaffen!~--
25Wenn aber der Mann das verlobte Mädchen auf dem Felde angetroffen und
sie ergriffen und ihr beigewohnt hat, so soll der Mann, der ihr Gewalt
angetan hat, allein sterben; 26dem Mädchen aber soll man nichts tun: sie
hat sich kein todeswürdiges Verbrechen zuschulden kommen lassen; denn es
verhält sich in diesem Falle ebenso, wie wenn ein Mann einen andern
überfällt und ihn ums Leben bringt. 27Er hat sie ja auf dem Felde
getroffen; und wenn das verlobte Mädchen auch geschrien hätte, so würde
doch kein Retter für sie dagewesen sein.~--

28Wenn ein Mann ein Mädchen, eine Jungfrau, die nicht verlobt ist,
antrifft und sie ergreift und ihr beiwohnt und man ihn
ertappt\textless sup title=``oder: ausfindig macht''\textgreater✲, 29so
soll der Mann, der ihr beigewohnt hat, dem Vater des Mädchens fünfzig
Schekel Silber geben, und sie soll ihm als Frau angehören; zur Strafe
dafür, daß er sie entehrt hat, darf er sie zeitlebens nicht
entlassen.~--

\hypertarget{section-22}{%
\section{23}\label{section-22}}

1Niemand darf die\textless sup title=``oder: eine''\textgreater✲ Frau
seines Vaters zum Weibe nehmen und überhaupt nicht die Bettdecke seines
Vaters aufdecken.«\textless sup title=``vgl. 27,20''\textgreater✲

\hypertarget{s-wer-in-die-gemeinde-israels-aufgenommen-werden-darf-und-wer-auszuschlieuxdfen-ist}{%
\paragraph{s) Wer in die Gemeinde Israels aufgenommen werden darf und
wer auszuschließen
ist}\label{s-wer-in-die-gemeinde-israels-aufgenommen-werden-darf-und-wer-auszuschlieuxdfen-ist}}

2»Keiner, dem die Hoden zerquetscht oder die Harnröhre abgeschnitten
ist, darf in die Gemeinde des HERRN aufgenommen werden. 3Kein Bastard
darf in die Gemeinde des HERRN aufgenommen werden; nicht einmal das
zehnte Geschlecht der Nachkommen des Betreffenden darf in die Gemeinde
des HERRN aufgenommen werden.~-- 4Kein Ammoniter und kein Moabiter darf
in die Gemeinde des HERRN aufgenommen werden; nicht einmal das zehnte
Geschlecht der Nachkommen von ihnen darf jemals in die Gemeinde des
HERRN aufgenommen werden, 5deshalb weil sie euch auf eurer Wanderung,
als ihr aus Ägypten auszogt, nicht mit Brot und Wasser entgegengekommen
sind und weil sie Bileam, den Sohn Beors, aus Pethor in Mesopotamien
gegen dich in Sold genommen haben, damit er dich verfluche\textless sup
title=``vgl. 4.Mose 22-24''\textgreater✲. 6Aber der HERR, dein Gott,
wollte Bileam nicht erhören, sondern der HERR, dein Gott, verwandelte
dir den Fluch in Segen; denn der HERR, dein Gott, hat dich lieb. 7Sei
niemals, solange du lebst, darauf bedacht, ihnen etwas Gutes oder eine
Liebe zu erweisen!~--

8Einen Edomiter sollst du nicht verabscheuen, denn er ist dein Bruder.
-- Einen Ägypter sollst du nicht verabscheuen, denn du hast als Gast in
seinem Lande gewohnt. 9Von den Kindern, die ihnen geboren werden, dürfen
die zum dritten Geschlecht Gehörenden in die Gemeinde des HERRN
aufgenommen werden.«

\hypertarget{t-uxfcber-die-reinhaltung-des-heerlagers-auf-kriegszuxfcgen}{%
\paragraph{t) Über die Reinhaltung des Heerlagers auf
Kriegszügen}\label{t-uxfcber-die-reinhaltung-des-heerlagers-auf-kriegszuxfcgen}}

10»Wenn du als Kriegsherr gegen deine Feinde ausziehst, so sollst du
dich vor allem Ungehörigen hüten. 11Ist ein Mann unter dir, der infolge
eines nächtlichen Begegnisses unrein geworden ist, so soll er aus dem
Lager hinausgehen: er darf nicht wieder in das Lager hineinkommen;
12gegen Abend soll er dann eine Waschung an sich vornehmen und darf
hierauf bei Sonnenuntergang ins Lager zurückkehren. 13Auch sollst du
außerhalb des Lagers einen Seitenplatz\textless sup title=``oder:
bestimmten Platz''\textgreater✲ haben, wohin du austreten kannst; 14und
unter deinen Geräten sollst du einen Spaten haben, mit dem du, wenn du
draußen deine Notdurft verrichtest, ein Loch graben und deinen Unrat
wieder bedecken sollst. 15Denn der HERR, dein Gott, zieht inmitten
deines Lagers einher, um dich zu erretten und deine Feinde dir
preiszugeben; darum soll dein Lager heilig sein, damit er nichts
Häßliches bei dir wahrnimmt und sich nicht von dir abwendet.«

\hypertarget{u-verschiedene-einzelne-gebote-der-menschenliebe-sittenreinheit-ehescheidung-u.a.}{%
\paragraph{u) Verschiedene einzelne Gebote (der Menschenliebe,
Sittenreinheit, Ehescheidung
u.a.)}\label{u-verschiedene-einzelne-gebote-der-menschenliebe-sittenreinheit-ehescheidung-u.a.}}

16»Einen Knecht✲, der sich vor seinem Herrn zu dir geflüchtet hat,
sollst du seinem Herrn nicht ausliefern: 17er soll bei dir, in deiner
Mitte, wohnen an dem Ort, den er in einer deiner Ortschaften erwählt, wo
es ihm gut dünkt: du sollst ihm keine Schwierigkeiten machen!~--

18Unter den Töchtern der Israeliten soll es keine der Unzucht geweihte
Dirne✲ geben, und unter den Söhnen der Israeliten soll es keinen zur
Unzucht bestimmten Buhler geben. 19Du darfst keinen Hurenlohn und kein
Hundegeld aus irgendeinem Gelübde in das Haus des HERRN, deines Gottes,
bringen; denn ein Greuel für den HERRN, deinen Gott, sind sie alle
beide.~-- 20Du sollst von deinem Volksgenossen keinen Zins verlangen,
weder für geliehenes Geld, noch für gelieferte Lebensmittel, noch für
irgend etwas anderes, das man gegen Zinsen verleihen kann. 21Von einem
Ausländer✲ magst du dir Zinsen zahlen lassen, nicht aber von einem
deiner Volksgenossen, damit der HERR, dein Gott, dich bei allen deinen
Unternehmungen segnet in dem Lande, in das du kommst, um es in Besitz zu
nehmen.~--

22Wenn du dem HERRN, deinem Gott, ein Gelübde leistest, so säume nicht,
es zu erfüllen! denn der HERR, dein Gott, wird es sicherlich von dir
fordern, und du würdest dir eine Verschuldung aufgeladen haben. 23Wenn
du aber das Geloben unterläßt, so ziehst du dir dadurch keine
Verschuldung zu; 24nur was deine Lippen ausgesprochen haben, mußt du
auch halten und ausführen, weil du dem HERRN, deinem Gott, freiwillig
gelobt hast, was du mit deinem Munde ausgesprochen hast.~--

25Wenn du in den Weinberg eines deiner Volksgenossen kommst, so magst du
Trauben nach deinem Begehren essen, bis du satt bist; aber in dein Gefäß
darfst du keine tun. 26Wenn du an das Kornfeld eines deiner
Volksgenossen kommst, so magst du Ähren mit der Hand abpflücken; aber
eine Sichel darfst du nicht an das Getreide eines deiner Volksgenossen
legen.~--

\hypertarget{section-23}{%
\section{24}\label{section-23}}

1Wenn ein Mann eine Frau nimmt und die Ehe mit ihr vollzieht, später
aber sich nicht mehr zu ihr hingezogen fühlt, weil er etwas
Häßliches\textless sup title=``oder: Widerwärtiges''\textgreater✲ an ihr
entdeckt hat, und er hat ihr einen Scheidebrief geschrieben und ihn ihr
eingehändigt und sie aus seinem Hause entlassen,~-- 2wenn sie also aus
seinem Hause weggegangen ist und die Ehe mit einem andern Manne
vollzogen hat 3und der zweite Mann ihr ebenfalls abgeneigt wird und ihr
auch einen Scheidebrief schreibt und ihn ihr einhändigt und sie so aus
seinem Hause entläßt, oder wenn der zweite Mann, der sie geheiratet hat,
stirbt: 4so darf ihr erster Mann, der sie verstoßen hatte, sie nicht
nochmals zur Frau nehmen, nachdem sie verunreinigt worden ist; denn das
würde ein Greuel in den Augen des HERRN sein, und du sollst das Land,
das der HERR, dein Gott, dir zum Erbbesitz geben will, nicht mit Sünde
beladen.

5Wenn jemand neuvermählt\textless sup title=``oder: erst kurze Zeit
verheiratet''\textgreater✲ ist, so braucht er nicht mit dem Heere ins
Feld zu ziehen, und keinerlei Verpflichtung soll ihm auferlegt werden:
er soll ein Jahr lang für sein Haus frei sein und seine Frau erfreuen,
die er geheiratet hat.~--

6Man darf nicht die Handmühle oder auch nur den oberen Mühlstein
pfänden, denn damit würde man das Leben zum Pfande nehmen.~--

7Wird jemand dabei ertappt, daß er einen von seinen Volksgenossen, von
den Israeliten, geraubt und ihn gewaltsam (als Sklaven) behandelt oder
ihn verkauft hat, so soll ein solcher Menschendieb sterben: so sollst du
das Böse aus deiner Mitte wegschaffen.~--

8Nimm dich bei der Erkrankung an Aussatz wohl in acht, daß du aufs
sorgfältigste alle Weisungen befolgst, die euch die levitischen Priester
erteilen werden: verfahrt genau so, wie ich ihnen geboten habe. 9Denke
daran, was der HERR, dein Gott, an Mirjam unterwegs bei eurem Auszug aus
Ägypten getan hat\textless sup title=``vgl. 4.Mose
12,4-15''\textgreater✲!~--

10Wenn du deinem Nächsten ein Darlehn von irgendwelchem Betrage
gewährst, so sollst du nicht in sein Haus hineingehen, um ihm ein Pfand
abzunehmen; 11nein, du sollst draußen auf der Straße stehen bleiben, und
der Mann, dem du leihst, soll das Pfand zu dir herausbringen; 12und wenn
er ein dürftiger Mann ist, so sollst du dich mit seinem Pfand nicht
schlafen legen, 13sondern ihm vielmehr das Pfand bei Sonnenuntergang
zurückgeben, damit er sich in seinem Mantel schlafen legen kann und dich
segnet; dann wird dir das als Gerechtigkeit vor dem HERRN, deinem Gott,
gelten.~--

14Bedrücke keinen dürftigen und armen Tagelöhner, der zu deinen
Volksgenossen oder zu den Fremden✲ gehört, die bei dir in deinem Lande,
in deinen Ortschaften leben! 15Noch an demselben Tage sollst du ihm
seinen Lohn geben, ehe noch die Sonne darüber untergeht; denn er ist
arm, und sein Herz sehnt sich danach; er würde sonst vielleicht den
HERRN gegen dich anrufen, und du hättest eine Verschuldung auf dich
geladen.~--

16Väter sollen nicht wegen (einer Verschuldung) ihrer Kinder mit dem
Tode bestraft werden, und Kinder sollen nicht wegen (einer Verschuldung)
ihrer Väter sterben; ein jeder soll nur wegen seines eigenen Vergehens
mit dem Tode bestraft werden!~--

17Du sollst das Recht eines Nichtisraeliten und einer Waise nicht beugen
und das Kleid einer Witwe nicht pfänden; 18du sollst vielmehr daran
gedenken, daß du selbst einst ein Knecht in Ägypten gewesen bist und daß
der HERR, dein Gott, dich von dort erlöst hat. Darum gebiete ich dir, so
zu verfahren.~--

19Wenn du beim Abernten deines Feldes eine Garbe auf dem Felde vergessen
hast, so sollst du nicht umkehren, um sie zu holen: sie soll den
Fremdlingen, den Waisen und den Witwen gehören, damit der HERR, dein
Gott, dich bei allem, was du unternimmst, segnet. 20Wenn du deine
Ölbäume abklopfst, sollst du hinterher nicht noch Nachlese an den
Zweigen halten: was an Früchten noch übrig ist, soll den Fremdlingen,
den Waisen und den Witwen zugute kommen. 21Wenn du die Lese in deinem
Weinberge hältst, sollst du hinterher nicht noch eine Nachlese
vornehmen: was (an Trauben) noch übrig ist, soll den Fremdlingen, den
Waisen und den Witwen zufallen. 22Denn du sollst daran gedenken, daß du
selbst einst ein Knecht im Lande Ägypten gewesen bist; darum gebiete ich
dir, so zu verfahren.«

\hypertarget{mauxdfhalten-bei-leibesstrafen-vor-gericht-milde-gegen-ein-arbeitendes-haustier}{%
\paragraph{Maßhalten bei Leibesstrafen vor Gericht; Milde gegen ein
arbeitendes
Haustier}\label{mauxdfhalten-bei-leibesstrafen-vor-gericht-milde-gegen-ein-arbeitendes-haustier}}

\hypertarget{section-24}{%
\section{25}\label{section-24}}

1»Wenn es zwischen Männern zu einem Rechtsstreit kommt und sie vor
Gericht getreten sind und man das Urteil über sie gesprochen hat, indem
man den Unschuldigen freigesprochen und den Schuldigen verurteilt hat,
2so soll, wenn der Schuldige Prügelstrafe verdient hat, der Richter ihn
sich hinlegen lassen, und man soll ihm in seiner\textless sup
title=``d.h. des Richters''\textgreater✲ Gegenwart eine bestimmte Anzahl
von Schlägen\textless sup title=``oder: Streichen''\textgreater✲ nach
Maßgabe seiner Verschuldung geben. 3Vierzig Schläge darf er ihm geben
lassen, aber nicht mehr, damit dein Volksgenosse nicht, wenn man
fortführe, ihm eine noch größere Zahl von Schlägen zu versetzen, in
deinen Augen verächtlich gemacht wird.~--

4Du sollst einem Ochsen, während er drischt, das Maul nicht
verbinden\textless sup title=``=~keinen Maulkorb
anlegen''\textgreater✲.«

\hypertarget{v-vorschriften-bezuxfcglich-der-schwagerehe-und-andere-gebote-besonders-bezuxfcglich-der-vergeltung-an-den-amalekitern}{%
\paragraph{v) Vorschriften bezüglich der Schwagerehe und andere Gebote
(besonders bezüglich der Vergeltung an den
Amalekitern)}\label{v-vorschriften-bezuxfcglich-der-schwagerehe-und-andere-gebote-besonders-bezuxfcglich-der-vergeltung-an-den-amalekitern}}

5»Wenn Brüder beisammen wohnen und einer von ihnen stirbt, ohne einen
Sohn zu hinterlassen, so soll sich die Ehefrau des Verstorbenen nicht
nach auswärts an einen fremden Mann verheiraten, sondern ihr Schwager
soll zu ihr eingehen und sie zu seiner Frau nehmen und die Schwagerehe
mit ihr vollziehen; 6der erste Sohn aber, den sie dann gebiert, soll auf
den Namen seines verstorbenen Bruders (in die Geschlechtsregister)
eingetragen werden, damit dessen Name in Israel nicht ausstirbt. 7Wenn
aber der Mann sich nicht dazu verstehen will, seine Schwägerin zu
heiraten, so soll seine Schwägerin ans Tor zu den Ältesten hingehen und
sagen: ›Mein Schwager weigert sich, den Namen seines Bruders in Israel
fortzupflanzen: er will die Schwagerehe nicht mit mir eingehen!‹ 8Dann
sollen die Ältesten der betreffenden Ortschaft ihn rufen lassen und ihm
Vorstellungen machen; und wenn er trotzdem darauf besteht und erklärt:
›Ich bin nicht geneigt, sie zu heiraten!‹, 9so soll seine Schwägerin vor
den Augen der Ältesten zu ihm hintreten, soll ihm den Schuh vom Fuß
ziehen, ihm ins Angesicht speien und laut ausrufen: ›So soll es dem
Mann\textless sup title=``oder: einem jeden''\textgreater✲ ergehen, der
das Haus seines Bruders nicht bauen will!‹ 10Mit einem Spottnamen soll
dann sein Haus in Israel die ›Barfüßerfamilie‹ heißen.«

\hypertarget{bestrafung-weiblicher-schamlosigkeit-ehrlichkeit-im-handel-und-wandel}{%
\paragraph{Bestrafung weiblicher Schamlosigkeit; Ehrlichkeit im Handel
und
Wandel}\label{bestrafung-weiblicher-schamlosigkeit-ehrlichkeit-im-handel-und-wandel}}

11»Wenn zwei Männer, Volksgenossen, miteinander handgemein werden und
die Frau des einen, die herbeigeeilt ist, um ihren Mann aus den Händen
dessen, der ihn schlägt, zu retten, jenen mit ihrer Hand bei den
Geschlechtsteilen faßt, 12so sollst du ihr die Hand abhauen, ohne einen
Blick des Mitleids für sie zu haben!~--

13Du sollst in deinem Beutel nicht zweierlei Gewichtsteine, einen
größeren und einen kleineren, haben; 14du sollst in deinem Hause nicht
zweierlei Hohlmaße, ein größeres und ein kleineres, haben: 15volle und
richtige Gewichte und volle und richtige Maße sollst du haben, damit du
lange in dem Lande lebst, das der HERR, dein Gott, dir geben wird;
16denn ein Greuel für den HERRN, deinen Gott, ist jeder, der solches
tut, jeder, der Unredlichkeit übt.«

\hypertarget{gebot-die-amalekiter-auszurotten}{%
\paragraph{Gebot, die Amalekiter
auszurotten}\label{gebot-die-amalekiter-auszurotten}}

17»Denke daran, was die Amalekiter dir unterwegs bei eurem Auszug aus
Ägypten angetan haben: 18daß sie dich, während du müde und matt warst,
auf der Wanderung ohne Furcht vor Gott überfallen und alle die
niedergehauen haben, welche bei dir vor Ermattung zurückgeblieben
waren\textless sup title=``2.Mose 17,8-16''\textgreater✲. 19Wenn dir
also der HERR, dein Gott, Ruhe vor allen deinen Feinden ringsum
geschafft hat in dem Lande, das der HERR, dein Gott, dir als Erbgut zu
seiner Besetzung geben wird, so sollst du das Andenken an die Amalekiter
unter dem ganzen Himmel austilgen: vergiß es nicht!«\textless sup
title=``vgl. 1.Sam 15,1-9''\textgreater✲

\hypertarget{w-anhang-enthaltend-zwei-gottesdienstliche-gebetsformeln-und-eine-zusammenfassende-verpflichtungsformel}{%
\paragraph{w) Anhang, enthaltend zwei gottesdienstliche Gebetsformeln
und eine zusammenfassende
Verpflichtungsformel}\label{w-anhang-enthaltend-zwei-gottesdienstliche-gebetsformeln-und-eine-zusammenfassende-verpflichtungsformel}}

\hypertarget{aa-das-dankbare-bekenntnis-bei-der-darbringung-der-erstlingsfruxfcchte-im-heiligtum}{%
\subparagraph{aa) Das dankbare Bekenntnis bei der Darbringung der
Erstlingsfrüchte im
Heiligtum}\label{aa-das-dankbare-bekenntnis-bei-der-darbringung-der-erstlingsfruxfcchte-im-heiligtum}}

\hypertarget{section-25}{%
\section{26}\label{section-25}}

1»Wenn du nun in das Land gekommen bist, das der HERR, dein Gott, dir zu
eigen geben wird, und du es in Besitz genommen hast und in ihm wohnst,
2so sollst du einen Teil von den Erstlingen aller Feldfrüchte, die du
von deinem Lande, das der HERR, dein Gott, dir geben wird, geerntet
hast, nehmen und sie in einen Korb legen und dich damit an die Stätte
begeben, die der HERR, dein Gott, erwählen wird, um seinen Namen
daselbst wohnen zu lassen. 3Dort sollst du dann zu dem Priester treten,
der zu jener Zeit im Amt sein wird, und zu ihm sagen: ›Ich bezeuge heute
dem HERRN, deinem Gott, daß ich wirklich in das Land gekommen bin,
dessen Verleihung an uns der HERR unsern Vätern zugeschworen hat.‹
4Hierauf soll der Priester den Korb aus deiner Hand nehmen und ihn vor
den Altar des HERRN, deines Gottes, hinstellen. 5Du aber sollst dann vor
dem HERRN, deinem Gott, folgende Worte aussprechen: ›Ein umherirrender
Aramäer war mein Stammvater; mit einer Mannschaft von wenigen Leuten zog
er nach Ägypten hinab und lebte dort als Fremdling, wuchs dort aber zu
einem großen, starken und zahlreichen Volk heran. 6Weil aber die Ägypter
uns mißhandelten und bedrückten und uns harte Zwangsarbeit auferlegten,
7schrien wir zum HERRN, dem Gott unserer Väter, um Hilfe, und der HERR
erhörte unser Flehen und sah unser Elend, unsere Mühsal und Bedrängnis;
8und der HERR führte uns mit starker Hand und hocherhobenem Arm, mit
schreckenerregender Macht und unter Zeichen und Wundern aus Ägypten
hinaus; 9er brachte uns an diesen Ort und gab uns dieses Land, ein Land,
das von Milch und Honig überfließt. 10Und nun bringe ich hier die
Erstlinge von den Früchten des Feldes, das du, HERR, mir gegeben hast.‹
Dann stelle sie vor den HERRN, deinen Gott, hin, wirf dich vor dem
HERRN, deinem Gott, anbetend nieder 11und erfreue dich mit den Leviten
und den Fremdlingen, die in deiner Mitte wohnen, an all dem Guten, das
der HERR, dein Gott, dir und deinem Hause gegeben hat.«

\hypertarget{bb-gebetsformel-bei-entrichtung-des-zehnten-an-die-leviten-und-armen}{%
\subparagraph{bb) Gebetsformel bei Entrichtung des Zehnten an die
Leviten und
Armen}\label{bb-gebetsformel-bei-entrichtung-des-zehnten-an-die-leviten-und-armen}}

12»Wenn du im dritten Jahre, dem Zehntjahr\textless sup title=``vgl.
14,28-29''\textgreater✲, den gesamten Zehnten von deinem Ernteertrag
vollständig entrichtet und ihn den Leviten, den Fremdlingen, den Waisen
und Witwen übergeben hast, damit sie ihn in deinen Ortschaften verzehren
und sich satt essen, 13so sollst du vor dem HERRN, deinem Gott, so
sprechen: ›Ich habe die heilige Abgabe aus meinem Hause hinausgeschafft
und sie den Leviten und Fremdlingen, den Waisen und Witwen genau so
übergeben, wie du mir geboten hast: ich habe keines von deinen Geboten
übertreten noch vergessen. 14Ich habe nichts davon gegessen in meiner
Trauer und nichts davon weggeschafft, während ich unrein war, und habe
nichts davon für eine Totenspeisung verwandt, nein, ich bin den
Weisungen des HERRN, meines Gottes, nachgekommen und habe mich genau an
seine Gebote gehalten. 15Blicke von deiner heiligen Wohnung, vom Himmel,
herab und segne dein Volk Israel und das Land, das du uns gegeben hast,
wie du unsern Vätern zugeschworen hast, ein Land, das von Milch und
Honig überfließt!‹«

\hypertarget{cc-zusammenfassender-abschluuxdf-der-bundespflichten-und-rechte}{%
\subparagraph{cc) Zusammenfassender Abschluß der Bundespflichten und
Rechte}\label{cc-zusammenfassender-abschluuxdf-der-bundespflichten-und-rechte}}

16»Am heutigen Tage gebietet dir der HERR, dein Gott, diese
Satzungen\textless sup title=``oder: Grundgesetze''\textgreater✲ und
Verordnungen zu befolgen: so beobachte und befolge sie denn von ganzem
Herzen und mit ganzer Seele! 17Du hast dir heute vom HERRN verkünden
lassen, daß er dein Gott sein wolle und daß es dir zukomme, auf seinen
Wegen zu wandeln, seine Satzungen, Vorschriften und Verordnungen zu
beobachten und seinen Weisungen zu gehorchen. 18Der HERR aber hat dich
heute die Erklärung aussprechen lassen, daß du sein Eigentumsvolk sein
wollest, wie er dir geboten hat, und daß es dir zukomme, alle seine
Gebote zu beobachten, 19und daß er dich über alle Völker, die er
geschaffen hat, zu Ruhm und Ehre und Ansehen erhöhen wolle und daß du
ein dem HERRN, deinem Gott, geheiligtes Volk sein wollest\textless sup
title=``oder: sollest''\textgreater✲, wie er dir geboten\textless sup
title=``oder: zugesagt''\textgreater✲ hat.«

\hypertarget{die-schluuxdfreden-kap.-27-30}{%
\subsubsection{3. Die Schlußreden (Kap.
27-30)}\label{die-schluuxdfreden-kap.-27-30}}

\hypertarget{a-aufstellung-von-denksteinen-des-gesetzes-im-westjordanlande-errichtung-eines-altars-auf-dem-berge-ebal}{%
\paragraph{a) Aufstellung von Denksteinen des Gesetzes im
Westjordanlande; Errichtung eines Altars auf dem Berge
Ebal}\label{a-aufstellung-von-denksteinen-des-gesetzes-im-westjordanlande-errichtung-eines-altars-auf-dem-berge-ebal}}

\hypertarget{section-26}{%
\section{27}\label{section-26}}

1Weiter geboten Mose und die Ältesten der Israeliten dem Volke
folgendes: »Beobachtet alle Gebote, die ich euch heute zur Pflicht
mache! 2Sobald ihr also über den Jordan in das Land gezogen seid, das
der HERR, dein Gott, dir geben wird, so richte dir große Steine auf,
überstreiche sie mit Tünchkalk 3und schreibe, sobald du hinübergezogen
bist, alle Worte dieses Gesetzes auf sie, damit du in das Land
hineinkommst, das der HERR, dein Gott, dir geben will, ein Land, das von
Milch und Honig überfließt, wie der HERR, der Gott deiner Väter, dir
verheißen hat. 4Sobald ihr also über den Jordan gezogen seid, sollt ihr
diese Steine, auf die sich mein heutiger Befehl bezieht, auf dem Berge
Ebal aufrichten und sie mit Tünchkalk überstreichen. 5Auch sollst du
dort dem HERRN, deinem Gott, einen Altar erbauen, und zwar einen Altar
von Steinen, die du mit keinem eisernen Werkzeug bearbeiten darfst: 6aus
unbehauenen Steinen sollst du den Altar des HERRN, deines Gottes,
erbauen und dem HERRN, deinem Gott, Brandopfer auf ihm darbringen; 7und
du sollst Dankopfer schlachten und ein Opfermahl dort halten und vor dem
HERRN, deinem Gott, fröhlich sein. 8Auf die Steine aber sollst du alle
Worte dieses Gesetzes schreiben, indem du sie sorgfältig eingräbst.«

\hypertarget{b-ausrufung-von-segens--und-fluchspruxfcchen-auf-den-bergen-ebal-und-garizim}{%
\paragraph{b) Ausrufung von Segens- und Fluchsprüchen auf den Bergen
Ebal und
Garizim}\label{b-ausrufung-von-segens--und-fluchspruxfcchen-auf-den-bergen-ebal-und-garizim}}

9Hierauf richteten Mose und die levitischen Priester folgende Worte an
ganz Israel: »Beobachte Schweigen, Israel, und höre zu! Am heutigen Tage
bist du zum Volk des HERRN, deines Gottes, geworden. 10So gehorche denn
den Weisungen des HERRN deines Gottes, und halte seine Gebote und seine
Satzungen, die ich dir heute gebiete!«

11An demselben Tage erteilte Mose dem Volke folgenden Befehl: 12»Wenn
ihr über den Jordan gezogen seid, so sollen die einen sich auf dem Berge
Garizim aufstellen, um das Volk zu segnen, nämlich Simeon, Levi, Juda,
Issaschar, Joseph und Benjamin; 13die anderen aber sollen sich, um den
Fluch auszusprechen, auf dem Berge Ebal aufstellen, nämlich Ruben, Gad
und Asser und Sebulon, Dan und Naphthali. 14Danach sollen die Leviten
anheben und mit hocherhobener Stimme zu allen Männern Israels sagen:

\hypertarget{die-zwuxf6lf-fluchworte}{%
\paragraph{Die zwölf Fluchworte}\label{die-zwuxf6lf-fluchworte}}

15›Verflucht sei, wer ein geschnitztes oder gegossenes Bild, einen
Greuel für den HERRN\textless sup title=``oder: vor dem
HERRN''\textgreater✲, ein Machwerk von Künstlerhand,
anfertigt\textless sup title=``oder: anfertigen läßt''\textgreater✲ und
es heimlich aufstellt!‹, und das ganze Volk soll antworten: ›So sei
es!✲‹

16›Verflucht sei, wer seinen Vater oder seine Mutter mißachtet!‹, und
das ganze Volk soll sagen: ›So sei es!‹

17›Verflucht sei, wer die Grenze seines Nächsten verrückt!‹, und das
ganze Volk soll sagen: ›So sei es!‹

18›Verflucht sei, wer einen Blinden auf dem Wege irreführt!‹, und das
ganze Volk soll sagen: ›So sei es!‹

19›Verflucht sei, wer das Recht von Fremdlingen, von Waisen und Witwen
beugt!‹, und das ganze Volk soll sagen: ›So sei es!‹

20›Verflucht sei, wer sich mit dem Weibe seines Vaters\textless sup
title=``d.h. seiner Stiefmutter''\textgreater✲ vergeht; denn er hat die
Bettdecke seines Vaters aufgedeckt!‹\textless sup title=``vgl.
23,1''\textgreater✲, und das ganze Volk soll sagen: ›So sei es!‹

21›Verflucht sei, wer irgendein Tier zur Unzucht mißbraucht!‹, und das
ganze Volk soll sagen: ›So sei es!‹

22›Verflucht sei, wer sich mit seiner Schwester, der Tochter seines
Vaters oder der Tochter seiner Mutter, vergeht!‹, und das ganze Volk
soll sagen: ›So sei es!‹

23›Verflucht sei, wer sich mit seiner Schwägerin\textless sup
title=``oder: Schwiegermutter''\textgreater✲ vergeht!‹, und das ganze
Volk soll sagen: ›So sei es!‹

24›Verflucht sei, wer seinen Nächsten heimlich erschlägt!‹, und das
ganze Volk soll sagen: ›So sei es!‹

25›Verflucht sei, wer sich durch Bestechung dazu bringen läßt, jemand zu
erschlagen, unschuldiges Blut zu vergießen!‹, und das ganze Volk soll
sagen: ›So sei es!‹

26›Verflucht sei, wer nicht die Bestimmungen dieses Gesetzes durch ihre
Erfüllung aufrechthält\textless sup title=``=~in Kraft
erhält''\textgreater✲!‹, und das ganze Volk soll sagen: ›So sei es!‹«

\hypertarget{c-ankuxfcndigung-des-segens-und-des-fluches-fuxfcr-israel}{%
\paragraph{c) Ankündigung des Segens und des Fluches für
Israel}\label{c-ankuxfcndigung-des-segens-und-des-fluches-fuxfcr-israel}}

\hypertarget{aa-verheiuxdfung-reicher-segnungen-fuxfcr-das-bundestreue-volk}{%
\subparagraph{aa) Verheißung reicher Segnungen für das bundestreue
Volk}\label{aa-verheiuxdfung-reicher-segnungen-fuxfcr-das-bundestreue-volk}}

\hypertarget{section-27}{%
\section{28}\label{section-27}}

1»Wenn du aber den Weisungen des HERRN, deines Gottes, gewissenhaft
nachkommst, indem du auf die Beobachtung aller seiner Gebote, die ich
dir heute zur Pflicht mache, bedacht bist, so wird der HERR, dein Gott,
dich über alle Völker der Erde erhöhen, 2und alle die folgenden
Segnungen werden dir zuteil werden und bei dir eintreffen, wenn du den
Weisungen des HERRN, deines Gottes, nachkommst: 3Gesegnet wirst du sein
in der Stadt und gesegnet auf dem Felde. 4Gesegnet wird die Frucht
deines Mutterleibes sein und die Frucht deines Ackers und die Frucht
deines Viehs, der Wurf deiner Rinder und der Nachwuchs deines
Kleinviehs. 5Gesegnet wird dein Fruchtkorb und dein Backtrog sein.
6Gesegnet wirst du bei deinem Eingang sein und gesegnet bei deinem
Ausgang. 7Der HERR wird deine Feinde, die sich gegen dich erheben,
niedergeworfen vor dir erliegen lassen; auf einem einzigen Wege werden
sie gegen dich zu Felde ziehen, aber auf sieben Wegen vor dir fliehen.
8Der Herr wird bei dir den Segen walten lassen in deinen Speichern und
bei allen deinen Unternehmungen und dich in dem Lande segnen, das der
HERR, dein Gott, dir geben wird. 9Der HERR wird dich zu einem ihm
geheiligten Volke erheben, wie er dir zugeschworen hat, wenn du die
Gebote des HERRN, deines Gottes, beobachtest und auf seinen Wegen
wandelst; 10da werden denn alle Völker der Erde sehen, daß du mit Recht
das Volk des HERRN genannt wirst, und werden sich vor dir fürchten.
11Und der HERR wird dich durch die Frucht deines Mutterleibes und durch
die Frucht deines Viehs und die Frucht deines Feldes zum Überfluß an
Gütern gelangen lassen in dem Lande, das der HERR, wie er deinen Vätern
zugeschworen hat, dir geben wird. 12Der HERR wird dir seine reiche
Schatzkammer, den Himmel, auftun, um deinem Lande zu rechter Zeit den
Regen zu spenden und alle Arbeiten deiner Hand zu segnen, so daß du
vielen Völkern wirst leihen können, ohne selbst etwas entlehnen zu
müssen. 13So wird der HERR dich zum Haupt und nicht zum
Schwanz\textless sup title=``=~zu einem dienenden Glied''\textgreater✲
machen, und es wird mit dir immer nur aufwärts gehen und nicht abwärts,
wenn du den Geboten des HERRN, deines Gottes, deren genaue Beobachtung
ich dir heute zur Pflicht mache, gehorsam bleibst 14und von allen
Weisungen, die ich euch heute gebiete, weder nach rechts noch nach links
abweichst, indem du anderen Göttern nachgehst, um ihnen zu dienen.«

\hypertarget{bb-erster-teil-der-verfluchungen-des-ungehorsamen-volkes}{%
\subparagraph{bb) Erster Teil der Verfluchungen des ungehorsamen
Volkes}\label{bb-erster-teil-der-verfluchungen-des-ungehorsamen-volkes}}

15»Wenn du aber den Weisungen des HERRN, deines Gottes, nicht gehorchst,
daß du auf die sorgfältige Beobachtung seiner Gebote und Satzungen, die
ich dir heute zur Pflicht mache, bedacht sein sollst, so werden alle
folgenden Flüche über dich kommen und dich treffen: 16Verflucht wirst du
sein in der Stadt und verflucht auf dem Felde. 17Verflucht wird dein
Fruchtkorb und dein Backtrog sein. 18Verflucht wird die Frucht deines
Mutterleibes und die Frucht deines Ackers sein, der Wurf deiner Rinder
und der Nachwuchs deines Kleinviehs. 19Verflucht wirst du bei deinem
Eingang sein und verflucht bei deinem Ausgang.~-- 20Der HERR wird den
Fluch, die Bestürzung und die Verwünschung gegen dich senden bei allen
Geschäften, die du unternimmst, bis du wegen deines frevelhaften Tuns,
weil du mich verlassen hast, vertilgt bist und schnellen Untergang
gefunden hast. 21Der HERR wird die Pest an dir haften lassen, bis
er\textless sup title=``oder: sie''\textgreater✲ dich aus dem Lande
ausgerottet hat, in das du jetzt ziehst, um es in Besitz zu nehmen.
22Der HERR wird dich mit Schwindsucht und Fieber, mit Entzündung und
Hitze, mit Dürre, Kornbrand und Vergilbung des Getreides heimsuchen, und
dies alles wird dich verfolgen, bis du zugrunde gegangen bist. 23Der
Himmel über deinem Haupt wird zu Erz werden und die Erde unter deinen
Füßen zu Eisen. 24Der HERR wird als Regen für dein Land Flugsand und
Staub geben: vom Himmel wird er auf dich herabfallen, bis du vertilgt
bist. 25Der HERR wird dich niedergeworfen vor deinen Feinden erliegen
lassen: auf einem einzigen Wege wirst du gegen sie ausziehen und auf
sieben Wegen vor ihnen fliehen; und du wirst für alle Reiche der Erde
ein Schreckbild sein. 26Deine Leichen werden allen Vögeln des Himmels
und den wilden Tieren zum Fraß dienen, ohne daß jemand sie verscheucht.
27Der HERR wird dich mit den Geschwüren Ägyptens und mit Pocken, mit
Aussatz\textless sup title=``oder: Krätze''\textgreater✲ und Grind
schlagen, daß du nicht wirst geheilt werden können. 28Der HERR wird dich
mit Wahnsinn, mit Blindheit und geistiger Zerrüttung schlagen, 29so daß
du am hellen Mittag umhertappen mußt, wie der Blinde im Finstern tappt;
und du wirst bei deinen Unternehmungen kein Gelingen haben, sondern
allezeit nur vergewaltigt und beraubt sein, ohne daß jemand dir hilft.
30Mit einem Weibe wirst du dich verloben, aber ein anderer Mann wird bei
ihr schlafen; ein Haus wirst du dir bauen, aber nicht in ihm wohnen;
einen Weinberg wirst du anlegen, aber seine Früchte nicht genießen.
31Dein Rind wird vor deinen Augen geschlachtet werden, ohne daß du von
ihm zu essen bekommst; dein Esel wird dir vor deinen Augen geraubt
werden und nicht wieder zu dir zurückkehren; dein Kleinvieh wird deinen
Feinden gegeben werden, ohne daß dir ein Helfer erscheint. 32Deine Söhne
und Töchter fallen einem anderen Volke (als Sklaven) in die Hände; deine
Augen müssen es ansehen und den ganzen Tag vor Sehnsucht nach ihnen
schmachten, du aber vermagst nichts dagegen zu tun. 33Die Früchte deines
Feldes und den ganzen Ertrag deiner Arbeit wird ein Volk verzehren, das
du bis dahin nicht gekannt hast, und du wirst allezeit nur der
Vergewaltigte und Mißhandelte sein: 34wahnsinnig wirst du werden beim
Anblick dessen, was deine Augen zu sehen bekommen. 35Der HERR wird dich
mit bösen Geschwüren an den Knien und Schenkeln schlagen, von denen du
nicht wirst geheilt werden können, von der Fußsohle bis zum Scheitel.
36Der HERR wird dich und deinen König, den du über dich setzen wirst, zu
einem Volke führen, das dir und deinen Vätern bis dahin unbekannt
gewesen ist; dort wirst du anderen Göttern, Götzen von Holz und Stein,
dienen müssen 37und wirst ein Gegenstand des Entsetzens, des Spottes und
Hohns bei allen Völkern werden, wohin der HERR dich führen wird.
38Aussaat wirst du in Menge aufs Feld hinausbringen, aber nur wenig
einernten; denn die Heuschrecken werden es abfressen; 39Weinberge wirst
du anpflanzen und bearbeiten, aber Wein weder trinken noch einkellern,
denn der Wurm wird ihn abfressen. 40Ölbäume wirst du überall in deinem
Gebiet haben, aber dich nicht mit Öl salben; denn deine Ölbäume werden
die Früchte abfallen lassen. 41Söhne und Töchter wirst du zeugen, aber
sie werden dir nicht verbleiben, denn sie müssen in die Gefangenschaft
wandern. 42Alle deine Bäume und die Früchte deines Ackers wird das
Ungeziefer aufzehren. 43Der Fremdling, der in deiner Mitte lebt, wird
immer höher über dich emporsteigen, während du immer tiefer hinabsinkst:
44er wird dir leihen, du aber wirst ihm nichts zu leihen haben; er wird
zum Haupt, du aber wirst zum Schwanz\textless sup title=``=~zum
dienenden Glied''\textgreater✲ werden.

45Alle diese Flüche werden über dich kommen, dich verfolgen und dich
treffen, bis du vernichtet bist, weil du den Weisungen des HERRN, deines
Gottes, nicht gehorcht hast und seinen Geboten und Satzungen, die er dir
zur Pflicht gemacht hat, nicht nachgekommen bist; 46und sie werden an
dir und deinen Nachkommen als Zeichen und Wunder bis in Ewigkeit
haften.«

\hypertarget{cc-zweiter-teil-der-verfluchungen}{%
\subparagraph{cc) Zweiter Teil der
Verfluchungen}\label{cc-zweiter-teil-der-verfluchungen}}

47»Zur Strafe dafür, daß du dem HERRN, deinem Gott, trotz des
Überflusses an allem nicht mit freudigem und bereitwilligem Herzen
gedient hast, 48wirst du deinen Feinden, die der HERR gegen dich senden
wird, dienen müssen bei Hunger und Durst, bei Mangel an Kleidung und bei
völliger Verarmung; und er wird dir ein eisernes Joch auf den Nacken
legen, bis er dich vertilgt hat. 49Der HERR wird gegen dich aus der
Ferne, vom Ende der Erde her, ein Volk heranführen, das so schnell wie
ein Adler daherfliegt, ein Volk, dessen Sprache du nicht verstehst,
50ein Volk mit wildtrotzigem Angesicht, das auch auf einen Greis keine
Rücksicht nimmt und mit keinem Kinde Erbarmen hat. 51Es wird den Ertrag
deines Viehstandes und den Ertrag deines Feldes verzehren, bis du
vertilgt bist, da es dir vom Getreide, vom Wein und Öl, vom Wurf deiner
Rinder und vom Nachwuchs deines Kleinviehs nichts übriglassen wird, bis
es dich zugrunde gerichtet hat. 52Es wird dich in all deinen Städten
belagern, bis deine hohen und starken Mauern, auf die du dein Vertrauen
setzt, in deinem ganzen Lande gefallen sind, und es wird dich in all
deinen Städten belagern, in deinem ganzen Lande, das der HERR, dein
Gott, dir gegeben hat. 53Da wirst du dann in der Angst und Bedrängnis,
in die dein Feind dich versetzen wird, deine leiblichen Kinder
verzehren, das Fleisch deiner Söhne und Töchter, die der HERR, dein
Gott, dir geschenkt hat. 54Sogar der an Wohlleben und die größte
Üppigkeit gewöhnte Mann bei dir wird dann auf seinen Bruder und auf das
Weib an seinem Busen und auf den Rest seiner Kinder, die er noch
übrigbehalten hat, voll Mißgunst blicken, 55so daß er keinem von ihnen
etwas von dem Fleisch seiner Kinder abgibt, die er, ohne sich irgendeins
übrigzulassen, in der Angst und Bedrängnis verzehrt, in die dich dein
Feind in all deinen Ortschaften versetzen wird. 56Sogar die an Wohlleben
und die größte Üppigkeit gewöhnte Frau bei dir, die vor Verzärtelung und
Verweichlichung noch nie versucht hat, ihre Fußsohle auf die Erde zu
setzen -- auch deren Auge wird auf den Mann an ihrem Busen und auf ihren
Sohn und ihre Tochter voll Mißgunst blicken 57und wird ihnen die
Nachgeburt mißgönnen, die aus ihrem Schoß hervorgeht, und die Kinder,
die sie zur Welt gebracht hat; denn bei dem Mangel an allem wird sie
diese heimlich verzehren in der Angst und Bedrängnis, in die dich dein
Feind in deinen Ortschaften versetzen wird.

58Wenn du nicht auf die Beobachtung aller Bestimmungen dieses Gesetzes,
die in diesem Buch aufgezeichnet stehen, bedacht bist, indem du diesen
ruhmvollen und furchtbaren Namen, den HERRN, deinen Gott, fürchtest,
59so wird der HERR über dich und deine Nachkommen außergewöhnliche
Unglücksschläge verhängen, schwere und andauernde Unglücksschläge und
bösartige und andauernde Krankheiten. 60Er wird dann alle Seuchen
Ägyptens bei dir einkehren\textless sup title=``oder:
auftreten''\textgreater✲ lassen, vor denen du einst Grauen empfunden
hast, und sie werden an dir haften bleiben. 61Auch alle Krankheiten und
alle Heimsuchungen, die in diesem Gesetzbuch nicht verzeichnet stehen --
auch die wird der HERR über dich kommen lassen, bis du vertilgt bist;
62und es werden Männer von euch nur in geringer Anzahl übrigbleiben,
statt daß ihr vorher den Sternen des Himmels an Menge gleichkamt, weil
du den Weisungen des HERRN, deines Gottes, nicht nachgekommen bist.
63Und wie der HERR vorher seine Freude daran gehabt hatte, euch Gutes zu
erweisen und euch zahlreich werden zu lassen, ebenso wird der HERR dann
bei euch seine Freude daran haben, euch zugrunde zu richten und zu
vertilgen, so daß ihr aus dem Lande herausgerissen werdet, in das du
jetzt einziehst, um es in Besitz zu nehmen. 64Der HERR wird dich alsdann
unter alle Völker von einem Ende der Erde bis zum andern zerstreuen, und
du wirst dort anderen Göttern dienen müssen, von denen ihr, du und deine
Väter, nichts gewußt habt, Götzen von Holz und Stein. 65Und du wirst
unter jenen Völkern zu keiner Ruhe kommen, und für deine Fußsohle wird
es keine Stätte der Rast geben, sondern der HERR wird dir dort ein immer
zitterndes Herz und vor Sehnsucht schmachtende Augen und eine
verzweifelnde Seele geben. 66Dein Leben wird dir an einem Faden zu
hangen scheinen, so daß du bei Tag und bei Nacht in Angst schwebst und
dich deines Lebens niemals sicher fühlst, 67am Morgen wirst du sagen:
›Ach, wäre es doch erst Abend!‹, und am Abend wirst du wünschen: ›Ach,
wäre es doch schon Morgen!‹ infolge der Angst deines Herzens, die du
empfinden wirst, und infolge des Anblicks der Schrecknisse, die dir vor
Augen stehen. 68Und der HERR wird dich auf Schiffen nach Ägypten
zurückkehren lassen auf dem Wege, von dem ich dir gesagt
habe\textless sup title=``vgl. 17,16''\textgreater✲: ›Du sollst ihn nie
mehr wiedersehen!‹ Dort werdet ihr euch dann euren Feinden zu Sklaven
und Sklavinnen verkaufen wollen, aber es wird sich kein Käufer finden.«

\hypertarget{d-ermahnungs--und-warnungsrede-moses-bei-abschluuxdf-des-neuen-gnadenbundes}{%
\paragraph{d) Ermahnungs- und Warnungsrede Moses bei Abschluß des neuen
Gnadenbundes}\label{d-ermahnungs--und-warnungsrede-moses-bei-abschluuxdf-des-neuen-gnadenbundes}}

69Dies sind die Worte\textless sup title=``oder:
Bestimmungen''\textgreater✲ des Bundes, den Mose auf Befehl des HERRN
mit den Israeliten im Lande der Moabiter geschlossen hat, außer dem
Bunde, den er am Horeb mit ihnen geschlossen hatte.

\hypertarget{aa-hinweis-auf-die-grouxdfen-bisherigen-guxf6ttlichen-strafgerichte-und-die-wohltaten}{%
\subparagraph{aa) Hinweis auf die großen bisherigen göttlichen
Strafgerichte und die
Wohltaten}\label{aa-hinweis-auf-die-grouxdfen-bisherigen-guxf6ttlichen-strafgerichte-und-die-wohltaten}}

\hypertarget{section-28}{%
\section{29}\label{section-28}}

1Mose berief also alle Israeliten und sagte zu ihnen: »Ihr habt selbst
alles gesehen, was der HERR vor euren Augen im Lande Ägypten dem Pharao
und allen seinen Knechten✲ und seinem ganzen Lande hat widerfahren
lassen: 2die großen Machterweise, die du mit eigenen Augen gesehen hast,
jene großen Zeichen und Wunder. 3Aber bis auf den heutigen Tag hat der
HERR euch kein Herz zum Erkennen, keine Augen zum Sehen und keine Ohren
zum Hören gegeben. 4Und doch habe ich euch vierzig Jahre lang in der
Wüste wandern lassen: eure Kleider sind euch dabei nicht am Leibe
zerfallen, und deine Schuhe haben sich an deinen Füßen nicht abgenutzt;
5ihr habt kein Brot zu essen gehabt und keinen Wein und kein starkes
Getränk getrunken, weil ihr erkennen solltet, daß ich der HERR, euer
Gott, bin. 6Als ihr dann in diese Gegend gelangtet, zogen Sihon, der
König von Hesbon, und Og, der König von Basan, uns zum Kampf entgegen,
aber wir schlugen sie 7und eroberten ihr Land und gaben es den Stämmen
Ruben und Gad und dem halben Stamm Manasse zum Erbbesitz. 8So beobachtet
denn die Bestimmungen dieses Bundes und befolgt sie, damit ihr bei allen
euren Unternehmungen glücklichen Erfolg habet!«

\hypertarget{bb-der-heutige-neue-bund-wird-fuxfcr-alle-kuxfcnftigen-geschlechter-geschlossen-und-ist-hochheilig}{%
\subparagraph{bb) Der heutige (neue) Bund wird für alle künftigen
Geschlechter geschlossen und ist
hochheilig}\label{bb-der-heutige-neue-bund-wird-fuxfcr-alle-kuxfcnftigen-geschlechter-geschlossen-und-ist-hochheilig}}

9»Ihr steht heute allesamt vor dem HERRN, eurem Gott: eure
Stammeshäupter, eure Richter✲, eure Ältesten und Obmänner, alle Männer
Israels, 10eure Kinder, eure Weiber und die Nichtisraeliten, die sich
bei dir inmitten deines Lagers befinden, von den Holzhauern an bis zu
den Wasserträgern bei dir, 11damit du in den Bund mit dem HERRN, deinem
Gott, und zwar in seine Eidgemeinschaft\textless sup title=``d.h. in
seinen mit Verfluchungen verknüpften Vertrag''\textgreater✲ eintretest,
den der HERR, dein Gott, heute mit dir schließt, 12weil er dich heute zu
seinem Volk einsetzen und er dein Gott sein will, wie er dir zugesagt
und wie er deinen Vätern Abraham, Isaak und Jakob zugeschworen hat.
13Aber nicht mit euch allein schließe ich diesen Bund und diesen
Fluchvertrag, 14sondern sowohl mit denen, die heute hier mit uns vor dem
HERRN, unserm Gott, stehen, als auch mit denen, die heute noch nicht mit
uns hier zugegen sind.«

\hypertarget{cc-warnung-vor-guxf6tzendienst-und-bruch-des-neuen-bundes-auf-welche-die-schwerste-strafe-folgt}{%
\subparagraph{cc) Warnung vor Götzendienst und Bruch des neuen Bundes,
auf welche die schwerste Strafe
folgt}\label{cc-warnung-vor-guxf6tzendienst-und-bruch-des-neuen-bundes-auf-welche-die-schwerste-strafe-folgt}}

15»Ihr wißt ja selbst, wie wir im Lande Ägypten gewohnt haben und wie
wir mitten durch die Völker gezogen sind, die ihr durchzogen habt; 16und
ihr habt ihre Scheusale und Götzen von Holz und Stein, Silber und Gold
gesehen, die es bei ihnen gibt. 17Daß nur ja kein Mann oder Weib, kein
Geschlecht oder Stamm sich unter euch befinde, dessen Herz sich heute
vom HERRN, unserm Gott, abwendet, daß er hingeht, den Göttern jener
Völker zu dienen! Daß sich unter euch nur ja keine Wurzel finde, die
Schierling und Wermut als Frucht hervorbringt, 18niemand, der, wenn er
die Worte dieses Fluchvertrags vernimmt, dann sich in seinem Herzen
glücklich preist, indem er denkt: ›Gut wird es mir ergehen, wenn ich
auch in der Verstocktheit meines Herzens wandle!‹ Das würde zur Folge
haben, daß alles, das bewässerte Land mitsamt dem trockenen,
hinweggerafft würde. 19Einem solchen Menschen wird der HERR nicht
gewillt sein zu verzeihen, nein, lodern wird alsdann der Zorn und Eifer
des HERRN gegen den betreffenden Mann, und alle Flüche, die in diesem
Buch aufgezeichnet stehen, werden auf ihn einstürmen, und der HERR wird
seinen Namen unter dem Himmel austilgen, 20und der HERR wird ihn aus
allen Stämmen Israels zum Unheil aussondern, wie es allen Flüchen des
Bundes entspricht, der in diesem Gesetzbuch aufgezeichnet steht. 21Die
späteren Geschlechter aber, eure Kinder, die nach euch erstehen werden,
und die Ausländer, die aus fernen Ländern kommen, werden fragen, wenn
sie die Unglücksschläge und Krankheiten sehen, mit denen der HERR dieses
Land heimgesucht hat, 22den Schwefel und das Salz -- eine Brandstätte
wird sein ganzer Boden sein, so daß er nicht besät werden kann und
nichts sprossen läßt und keine Pflanze in ihm aufkommt, eine Verwüstung
wie die von Sodom und Gomorrha, Adama und Zeboim, die der HERR in seinem
Zorn und Grimm dem Untergang preisgegeben hat --, 23ja, alle Völker
werden fragen: ›Warum hat der HERR diesem Lande solches Geschick
widerfahren lassen? Welchen Grund hat diese schreckliche Zornesglut?‹
24Dann wird man antworten: ›Das ist die Strafe dafür, daß sie den Bund
mit dem HERRN, dem Gott ihrer Väter, verlassen haben, den er mit ihnen
geschlossen hatte, als er sie aus dem Lande Ägypten herausführte, 25und
daß sie dazu übergegangen sind, anderen Göttern zu dienen und sich vor
ihnen niederzuwerfen, Göttern, die ihnen vorher unbekannt gewesen waren
und die er ihnen nicht zugeteilt hatte: 26darum ist der Zorn des HERRN
gegen dieses Land entbrannt, so daß er den ganzen Fluch, der in diesem
Buche aufgezeichnet steht, über das Land hat kommen lassen; 27und darum
hat der HERR sie im Zorn und Grimm und in gewaltiger Erbitterung aus
ihrem Lande herausgerissen und sie in ein anderes Land geschleudert, wie
es heutigentags der Fall ist.‹

28Das noch Verborgene steht beim HERRN, unserm Gott, aber das bereits
offenbar Gewordene ist für uns und unsere Kinder für alle Ewigkeit
bestimmt, damit wir alle Worte\textless sup title=``oder:
Bestimmungen''\textgreater✲ dieses Gesetzes erfüllen.«

\hypertarget{dd-ankuxfcndigung-von-gottes-barmherzigkeit-fuxfcr-das-verstouxdfene-aber-dann-buuxdffertig-gewordene-volk-neuer-bund-nach-kuxfcnftigen-gerichten}{%
\subparagraph{dd) Ankündigung von Gottes Barmherzigkeit für das
verstoßene, aber dann bußfertig gewordene Volk; neuer Bund nach
künftigen
Gerichten}\label{dd-ankuxfcndigung-von-gottes-barmherzigkeit-fuxfcr-das-verstouxdfene-aber-dann-buuxdffertig-gewordene-volk-neuer-bund-nach-kuxfcnftigen-gerichten}}

\hypertarget{section-29}{%
\section{30}\label{section-29}}

1»Wenn nun alle diese Worte, der Segen und der Fluch, die ich dir
vorgetragen habe, (einst) bei dir eintreffen werden und du es unter
allen Völkern, unter die der HERR, dein Gott, dich (alsdann) verstoßen
hat, dir zu Herzen nehmen wirst 2und du samt deinen Kindern von ganzem
Herzen und mit ganzer Seele zum HERRN, deinem Gott, zurückkehrst und
seinen Weisungen in allem, was ich dir heute gebiete, gehorsam bist: 3so
wird der HERR, dein Gott, dein Geschick wenden und sich deiner erbarmen
und wird dich wieder aus all den Völkern sammeln, unter die der HERR,
dein Gott, dich zerstreut hat. 4Wenn deine verstoßenen (Söhne) sich auch
am Ende des Himmels befinden sollten, wird der HERR, dein Gott, dich
doch von dort sammeln und von dort dich zurückholen; 5und der HERR, dein
Gott, wird dich in das Land zurückbringen, das deine Väter besessen
hatten, damit du es wieder in Besitz nimmst, und er wird dich
glücklicher und zahlreicher werden lassen, als deine Väter waren. 6Und
der HERR wird dir und deinen Nachkommen das Herz beschneiden, damit du
den HERRN, deinen Gott, von ganzem Herzen und mit ganzer Seele liebst um
deines Lebens willen; 7und der HERR, dein Gott, wird alle jene Flüche
über deine Feinde und über deine Widersacher hereinbrechen lassen, die
dich verfolgt haben. 8Du aber wirst wieder den Weisungen des HERRN
gehorchen und alle seine Gebote befolgen, die ich dir heute zur Pflicht
mache. 9Alsdann wird der HERR, dein Gott, dir Überfluß an Glück bei
allem, was du unternimmst, verleihen durch Kindersegen, durch reichen
Viehbesitz und reichen Ertrag deiner Felder; denn der HERR wird zu
deinem Heil wieder Freude an dir haben, wie er an deinen Vätern Freude
gehabt hat, 10wenn du nämlich den Weisungen des HERRN, deines Gottes,
gehorchst, so daß du seine Gebote und Satzungen beobachtest, alles, was
in diesem Gesetzbuch aufgezeichnet steht, wenn du dich von ganzem Herzen
und mit ganzer Seele zum HERRN, deinem Gott, bekehrst.«

\hypertarget{ee-das-von-gott-gebotene-gesetz-fordert-nichts-unmuxf6gliches-es-ist-vielmehr-leicht-verstuxe4ndlich-und-leicht-zu-befolgen}{%
\subparagraph{ee) Das von Gott gebotene Gesetz fordert nichts
Unmögliches; es ist vielmehr leicht verständlich und leicht zu
befolgen}\label{ee-das-von-gott-gebotene-gesetz-fordert-nichts-unmuxf6gliches-es-ist-vielmehr-leicht-verstuxe4ndlich-und-leicht-zu-befolgen}}

11»Denn dieses Gesetz, das ich dir heute gebiete, ist für dich nicht zu
schwer und nicht unerreichbar; 12es ist nicht im Himmel, daß du sagen
müßtest: ›Wer wird für uns in den Himmel hinaufsteigen, um es uns zu
holen und es uns zu verkündigen, damit wir es befolgen können?‹ 13Es ist
auch nicht jenseits des Meeres, daß du sagen müßtest: ›Wer wird für uns
über das Meer hinüberfahren, um es uns zu holen und es uns zu
verkündigen, damit wir es befolgen können?‹ 14Nein, ganz nahe ist dir
das Wort: in deinem Munde und in deinem Herzen hast du es, so daß du es
befolgen kannst.«

\hypertarget{ff-schluuxdfermahnung-wahl-zwischen-leben-und-tod}{%
\subparagraph{ff) Schlußermahnung; Wahl zwischen Leben und
Tod}\label{ff-schluuxdfermahnung-wahl-zwischen-leben-und-tod}}

15»Bedenke wohl: ich habe dir heute das Leben und das Glück und
(andrerseits) den Tod und das Unglück zur Wahl vorgelegt. 16Was ich dir
heute gebiete, ist: den HERRN, deinen Gott, zu lieben, auf seinen Wegen
zu wandeln und seine Gebote, seine Satzungen und Verordnungen zu
beobachten, damit du am Leben bleibst und zahlreich wirst und der HERR,
dein Gott, dich segnet in dem Lande, in das du jetzt einziehst, um es in
Besitz zu nehmen. 17Wenn aber dein Herz sich abwendet und du nicht
gehorsam bist, sondern dich dazu verführen läßt, andere Götter anzubeten
und ihnen zu dienen, 18so kündige ich euch heute schon an, daß ihr
unfehlbar zugrunde gehen werdet: ihr werdet alsdann nicht lange in dem
Lande wohnen bleiben, in das du jetzt über den Jordan ziehst, um es in
Besitz zu nehmen. 19Ich rufe heute den Himmel und die Erde zu Zeugen
gegen euch an: das Leben und den Tod habe ich euch vorgelegt, den Segen
und den Fluch. So wähle denn das Leben, damit du am Leben bleibst, du
und deine Nachkommen, 20indem du den HERRN, deinen Gott, liebst, seinen
Weisungen gehorchst und fest an ihm hältst; denn davon hängt dein Leben
und die Dauer deiner Tage ab, während deren du in dem Lande wohnst,
dessen Verleihung der HERR deinen Vätern Abraham, Isaak und Jakob
zugeschworen hat.«

\hypertarget{iii.-die-letzten-schicksale-und-abschiedsreden-moses-kap.-31-34}{%
\subsection{III. Die letzten Schicksale und Abschiedsreden Moses (Kap.
31-34)}\label{iii.-die-letzten-schicksale-und-abschiedsreden-moses-kap.-31-34}}

\hypertarget{moses-letzte-anordnungen-und-letzte-prophetische-betuxe4tigung}{%
\subsubsection{1. Moses letzte Anordnungen und letzte prophetische
Betätigung}\label{moses-letzte-anordnungen-und-letzte-prophetische-betuxe4tigung}}

\hypertarget{a-einsetzung-josuas-zum-nachfolger-moses}{%
\paragraph{a) Einsetzung Josuas zum Nachfolger
Moses}\label{a-einsetzung-josuas-zum-nachfolger-moses}}

\hypertarget{section-30}{%
\section{31}\label{section-30}}

1Nun ging Mose daran, folgende Worte an ganz Israel zu richten; 2er
sagte zu ihnen: »Ich bin heute 120~Jahre alt: ich vermag nicht mehr den
Anforderungen meines Amtes zu genügen; auch hat der HERR zu mir gesagt:
›Du sollst den Jordan da nicht überschreiten!‹ 3Der HERR, dein Gott,
wird selbst an deiner Spitze hinüberziehen: er selbst wird diese Völker
vor dir her vernichten, daß du ihr Land in Besitz nehmen kannst. Josua
wird an deiner Spitze hinüberziehen, wie der HERR geboten hat, 4und der
HERR wird mit ihnen verfahren, wie er mit Sihon und Og, den Königen der
Amoriter, und mit ihrem Lande verfahren ist, die er vernichtet hat.
5Wenn der HERR sie aber in eure Gewalt gibt, so sollt ihr mit ihnen
genau nach der Weisung verfahren, die ich euch gegeben habe. 6Seid mutig
und stark! Fürchtet euch nicht und seid ohne Angst vor ihnen! Denn der
HERR, dein Gott, zieht selbst mit dir: er wird dir seine Hilfe nicht
versagen und dich nicht verlassen.«

7Hierauf berief Mose den Josua und sagte zu ihm in Gegenwart aller
Israeliten: »Sei mutig und stark! Denn du sollst mit diesem Volk in das
Land kommen, dessen Besitz der HERR ihren Vätern zugeschworen hat, und
du sollst es als Erbbesitz unter sie verteilen. 8Der HERR selbst aber
wird vor dir herziehen; er wird mit dir sein, wird dir seine Hilfe nicht
versagen und dich nicht verlassen: fürchte dich nicht und sei ohne
Angst!«

\hypertarget{b-aufzeichnung-und-uxfcbergabe-des-gesetzes-an-die-priester-und-uxe4ltesten}{%
\paragraph{b) Aufzeichnung und Übergabe des Gesetzes an die Priester und
Ältesten}\label{b-aufzeichnung-und-uxfcbergabe-des-gesetzes-an-die-priester-und-uxe4ltesten}}

9Dann schrieb Mose dieses Gesetz nieder und übergab es den Priestern,
den Leviten, welche die Lade mit dem Bundesgesetze des HERRN zu tragen
hatten, und allen Ältesten Israels. 10Mose gab ihnen dabei folgenden
Befehl: »Alle sieben Jahre, zur festgesetzten Zeit des
Erlaßjahres\textless sup title=``vgl. 15,1''\textgreater✲, am
Laubhüttenfeste✲, 11wenn ganz Israel sich einfindet, um vor dem HERRN,
deinem Gott, an der Stätte zu erscheinen, die er erwählen wird, sollst
du dieses Gesetz vor ganz Israel laut vorlesen, so daß alle es hören.
12Laß das Volk, die Männer, die Frauen und die Kinder, auch die
Fremdlinge, die bei dir in deinen Ortschaften wohnen, sich versammeln,
damit sie es hören und kennenlernen und den HERRN, euren Gott, fürchten
und alle Bestimmungen dieses Gesetzes gewissenhaft befolgen. 13Auch ihre
Kinder, die es noch nicht kennen, sollen es hören, damit sie den HERRN,
euren Gott, fürchten lernen während der ganzen Zeit, die ihr in dem
Lande leben werdet, in das ihr jetzt über den Jordan zieht, um es in
Besitz zu nehmen.«

\hypertarget{c-gott-kuxfcndigt-den-tod-moses-an-sagt-den-kuxfcnftigen-abfall-israels-voraus-und-ordnet-die-aufschreibung-eines-zum-zeugnis-dienenden-liedes-an}{%
\paragraph{c) Gott kündigt den Tod Moses an, sagt den künftigen Abfall
Israels voraus und ordnet die Aufschreibung eines zum Zeugnis dienenden
Liedes
an}\label{c-gott-kuxfcndigt-den-tod-moses-an-sagt-den-kuxfcnftigen-abfall-israels-voraus-und-ordnet-die-aufschreibung-eines-zum-zeugnis-dienenden-liedes-an}}

14Hierauf sagte der HERR zu Mose: »Nun ist die Zeit für dich gekommen,
daß du sterben mußt. Rufe Josua und tretet in das Offenbarungszelt,
damit ich ihn zu seinem Amt bestelle!« Da gingen Mose und Josua hin und
traten in das Offenbarungszelt; 15der HERR aber erschien im Zelte in
einer Wolkensäule, und die Wolkensäule blieb am Eingang des Zeltes
stehen. 16Da sagte der HERR zu Mose: »Du bist nun im Begriff, dich zu
deinen Vätern zu legen; dann wird dieses Volk sich daranmachen, mit den
fremden Göttern inmitten des Landes, in das es jetzt einzieht,
Abgötterei zu treiben; es wird mich dann verlassen und den Bund mit mir,
den ich mit ihnen geschlossen habe, brechen. 17Da wird dann mein Zorn
gegen sie zu jener Zeit entbrennen, und ich werde sie verlassen und mein
Angesicht vor ihnen verbergen; dann wird es der Vertilgung anheimfallen,
und viele Leiden und Drangsale werden es treffen. Da wird es dann zu
jener Zeit sagen: ›Haben diese Leiden mich nicht deshalb getroffen, weil
mein Gott nicht mehr in meiner Mitte ist?‹ 18Ich aber will dann zu jener
Zeit mein Angesicht gänzlich (vor ihm) verbergen wegen all des Bösen,
das dieses Volk verübt hat, indem es sich anderen Göttern zuwandte.«

\hypertarget{gottes-befehl-an-mose-das-lied-aufzuschreiben-mose-fuxfchrt-den-befehl-aus}{%
\paragraph{Gottes Befehl an Mose, das Lied aufzuschreiben; Mose führt
den Befehl
aus}\label{gottes-befehl-an-mose-das-lied-aufzuschreiben-mose-fuxfchrt-den-befehl-aus}}

19»Und nun schreibt euch das nachfolgende Lied auf und lehre es die
Israeliten; lege es ihnen in den Mund, damit dieses Lied mir zum Zeugnis
gegen die Israeliten diene. 20Denn ich werde sie allerdings in das Land
bringen, das ich ihren Vätern zugeschworen habe, (ein Land,) das von
Milch und Honig überfließt; aber wenn sie dann essen und satt werden und
fett geworden sind, werden sie sich anderen Göttern zuwenden und ihnen
dienen, mich aber werden sie verwerfen und den Bund mit mir brechen.
21Wenn dann viele Leiden und Drangsale sie treffen, dann soll dieses
Lied -- denn es wird im Munde ihrer Nachkommen unvergessen fortleben --
Zeugnis gegen sie ablegen, daß ich ihr Sinnen und Trachten, mit dem sie
schon heute umgehen, gekannt habe, noch ehe ich sie in das Land habe
gelangen lassen, das ich (ihnen) zugeschworen habe.« 22So schrieb denn
Mose dieses Lied noch an demselben Tage auf und lehrte es die
Israeliten; 23dem Josua aber, dem Sohne Nuns, erteilte der HERR
folgenden Auftrag: »Sei mutig und stark! Denn du sollst die Israeliten
in das Land bringen, das ich ihnen zugeschworen habe, und ich selbst
will mit dir sein.«

\hypertarget{mose-uxfcbergibt-das-gesetzbuch-den-leviten-zur-aufbewahrung-und-truxe4gt-sein-lied-dem-volke-als-letzte-warnung-vor}{%
\paragraph{Mose übergibt das Gesetzbuch den Leviten zur Aufbewahrung und
trägt sein Lied dem Volke als letzte Warnung
vor}\label{mose-uxfcbergibt-das-gesetzbuch-den-leviten-zur-aufbewahrung-und-truxe4gt-sein-lied-dem-volke-als-letzte-warnung-vor}}

24Als nun Mose dieses Gesetz nach seinem Wortlaut vollständig bis zu
Ende in ein Buch geschrieben hatte, 25gab er den Leviten, welche die
Lade mit dem Bundesgesetze des HERRN zu tragen hatten, folgenden Befehl:
26»Nehmt dieses Gesetzbuch und legt es neben die Bundeslade des HERRN,
eures Gottes, damit es dort zum Zeugnis gegen euch dient. 27Denn ich
kenne deine Widerspenstigkeit und deine Halsstarrigkeit wohl. Wenn ihr
schon jetzt, während ich noch als Lebender unter euch weile,
widerspenstig gegen den HERRN gewesen seid: wieviel mehr wird es da nach
meinem Tode so sein! 28Beruft zu einer Versammlung bei mir alle Ältesten
eurer Stämme und eure Obmänner\textless sup title=``vgl.
1,15''\textgreater✲, ich will ihnen diese Worte laut vorlesen und den
Himmel und die Erde zu Zeugen gegen sie anrufen. 29Denn ich weiß, daß
ihr nach meinem Tode ganz verwerflich handeln und von dem Wege abweichen
werdet, den ich euch zur Pflicht gemacht habe. So wird denn schließlich
das Unglück über euch hereinbrechen, weil ihr tun werdet, was dem HERRN
mißfällt, indem ihr ihn durch euer ganzes Tun zum Zorn reizt.«

30Hierauf trug Mose der ganzen Gemeindeversammlung der Israeliten den
Wortlaut des folgenden Liedes bis zu Ende vor:

\hypertarget{das-lied-moses}{%
\subsubsection{2. Das Lied Moses}\label{das-lied-moses}}

\hypertarget{section-31}{%
\section{32}\label{section-31}}

1Horcht auf, ihr Himmel, denn ich will reden, und die Erde vernehme die
Worte meines Mundes! 2Wie Regen ergieße sich meine Belehrung, wie Tau
riesele meine Rede, wie Regenschauer auf junges Grün und wie
Regentropfen auf Pflanzen! 3Denn den Namen✲ des HERRN will ich
verkünden: gebt unserm Gott die Ehre!

4Er ist ein Fels, vollkommen ist sein Tun, denn alle seine Wege sind
recht; ein Gott der Treue und ohne Falsch, gerecht und wahrhaftig ist
er. 5Übel haben an ihm gehandelt, die wegen ihrer Verworfenheit nicht
seine Söhne\textless sup title=``oder: Kinder''\textgreater✲ sind, ein
verderbtes und verkehrtes Geschlecht. 6Durftest du dem HERRN so
vergelten, du törichtes und unverständiges Volk? Ist nicht er dein
Vater, der dich geschaffen? Hat nicht er dich gemacht und bereitet?

7Gedenke der Tage der Vorzeit, betrachte die Jahre von Geschlecht zu
Geschlecht! Frag deinen Vater, der wird dir's kundtun, deine Greise, die
werden dir's erzählen: 8Als der Höchste den Völkern ihren Erbbesitz
zuteilte, als er die Menschenkinder voneinander schied, da setzte er die
Gebiete\textless sup title=``oder: Grenzen''\textgreater✲ der Stämme
fest nach der Zahl der Kinder Israel. 9Denn der Anteil des HERRN ist
sein Volk, Jakob der Bezirk seines Erbguts.

10Er fand es im Bereich der Wüste, in der Einöde voll Geheul der
Wildnis; er umhegte es schützend, nahm sich seiner an, hütete es wie
seinen Augapfel. 11Wie ein Adler, der seine Brut aus dem Nest
hinausführt und über seinen Jungen flatternd schwebt, seine Fittiche
über sie breitet, sie aufnimmt, sie trägt auf seinen Schwingen: 12so
leitete der HERR allein das Volk, kein fremder Gott war mit
ihm\textless sup title=``oder: bei ihm''\textgreater✲. 13Er ließ es auf
den Höhen der Erde einherfahren, und es aß die Erträgnisse des Gefildes;
er ließ es Honig aus Felsen schlürfen und Öl aus Kieselgestein, 14Sahne
von Kühen und Milch vom Kleinvieh, dazu das Fett von Lämmern und
Widdern, Sprößlinge\textless sup title=``d.h. Stiere''\textgreater✲ von
Basan und Böcke samt dem Nierenfett des Weizens; und Traubenblut trankst
du, feurigen Wein.

15Da wurde Jeschurun fett und schlug aus

ja, fett wurdest du, wurdest dick, wurdest feist!~-- und verwarf den
Gott, der ihn geschaffen, und verachtete den Felsen seines Heils.

16Sie reizten ihn zur Eifersucht durch fremde Götter, erbitterten ihn
durch greulichen Götzendienst: 17sie opferten den Dämonen, die nicht
Gott sind, Göttern, die (vorher) ihnen unbekannt gewesen, neuen Göttern,
die erst vor kurzem aufgekommen waren, die eure Väter nicht verehrt
hatten. 18Des Felsens, der dir das Dasein gegeben, gedachtest du nicht
mehr und vergaßest den Gott, dem du das Entstehn verdanktest.

19Der HERR sah es und verwarf sie voll Unmuts über seine Söhne und
Töchter; 20er sprach: »Ich will mein Angesicht vor ihnen verbergen, will
sehen, welches ihr Ausgang sein wird; denn ein Geschlecht voll
Verkehrtheit sind sie, Kinder, in denen keine Treue wohnt. 21Sie haben
mich zur Eifersucht gereizt durch Nicht-Götter, mich erbittert durch
ihre nichtigen Götzen; so will auch ich sie zur Eifersucht reizen durch
ein Nicht-Volk, durch einen unverständigen Volksstamm sie erbittern.
22Denn ein Feuer ist durch meinen Zorn entbrannt und hat bis in die
Tiefen der Unterwelt gelodert; es hat die Erde samt ihrem Ertrag
verzehrt und die Grundfesten der Berge in Flammen gesetzt. 23Ich will
Leiden auf sie häufen, meine Pfeile gegen sie verbrauchen: 24sind sie
vor Hunger verschmachtet und von Fieberglut und giftigen Seuchen
verzehrt, so will ich den Zahn wilder Tiere gegen sie loslassen samt dem
Gift der im Staube kriechenden Schlangen. 25Draußen soll das Schwert sie
(der Angehörigen) berauben und drinnen daheim der Schrecken (sie
wegraffen), den Jüngling wie die Jungfrau, den Säugling mitsamt dem
Graukopf.«

26Ich hätte gesagt: »Zerstreuen\textless sup title=``oder:
zerschlagen''\textgreater✲ will ich sie, ihr Gedächtnis unter den
Menschen verschwinden lassen!«, 27wenn nicht Verdruß ich vom Feinde her
fürchtete, daß nämlich ihre Widersacher es falsch deuteten, daß sie
sagen möchten: »Unsere Hand hat obgesiegt, und nicht der HERR hat dies
alles vollbracht!« 28Denn ein Volk sind sie, dem alle Einsicht abgeht,
und kein Verständnis findet sich bei ihnen. 29Wären sie weise, daß sie
dies begriffen, so würden sie bedenken, welches ihr Endgeschick sein
wird. 30Wie könnte ein einziger Tausend vor sich hertreiben und zwei
Zehntausend in die Flucht schlagen, hätte nicht ihr Fels sie verkauft
und der HERR sie preisgegeben? 31Denn nicht wie unser Fels ist ihr Fels;
das müssen unsere Feinde selbst anerkennen. 32Doch vom Weinstock Sodoms
stammt ihr Weinstock und aus den Gefilden Gomorrhas: ihre Trauben sind
Gifttrauben, gallenbittre Beeren haben sie; 33Schlangengeifer ist ihr
Wein und grausiges Otterngift.

34»Liegt das nicht bei mir aufbewahrt, versiegelt in meinen
Schatzkammern? 35Mir steht die Rache und Vergeltung zu für die Zeit, da
ihr Fuß wanken wird; denn nahe ist der Tag ihres Verderbens, und eilends
kommt das ihnen bestimmte Schicksal heran.« 36Denn der HERR wird sein
Volk richten, aber seiner Knechte sich erbarmen, wenn er sieht, daß
jeder Halt geschwunden und daß dahin sind Hörige wie Freie. 37Da wird er
sagen: »Wo sind nun ihre Götter, der Fels, auf den sie sich verließen?
38Wo sind die, welche das Fett ihrer Schlachtopfer aßen, den Wein ihrer
Gußopfer tranken? Sie mögen auftreten und euch helfen, damit sie euer
Schirm sind! 39Erkennet jetzt, daß ich allein es bin und neben mir kein
andrer Gott besteht! Ich bin's, der tötet und lebendig macht, ich
verwunde, aber heile auch wieder, und niemand kann aus meiner Hand
erretten!

40Nun denn, ich erhebe meine Hand zum Himmel

und gelobe: So wahr ich ewig lebe: 41Hab' ich mein blitzendes Schwert
geschärft und hat meine Hand zum Gericht gegriffen, so werde ich Rache
an meinen Feinden nehmen und denen vergelten, die mich hassen! 42Meine
Pfeile will ich mit Blut trunken machen

und mein Schwert soll Fleisch fressen~-- mit dem Blut der Erschlagnen
und Gefangnen, vom Haupt der Fürsten des Feindes!«

43Jubelt, ihr Heidenvölker, über sein Volk! denn er wird das Blut seiner
Knechte rächen und Rache an seinen Bedrängern nehmen und entsünd'gen
sein Land, sein Volk.

\hypertarget{letzte-einschuxe4rfung-des-gesetzes-mose-empfuxe4ngt-weisungen-bezuxfcglich-seines-bevorstehenden-todes}{%
\subsubsection{3. Letzte Einschärfung des Gesetzes; Mose empfängt
Weisungen bezüglich seines bevorstehenden
Todes}\label{letzte-einschuxe4rfung-des-gesetzes-mose-empfuxe4ngt-weisungen-bezuxfcglich-seines-bevorstehenden-todes}}

44Mose ging dann hin und trug den ganzen Wortlaut dieses Liedes dem
Volke laut vor, er und Hosea✲, der Sohn Nuns\textless sup title=``vgl.
4.Mose 13,16''\textgreater✲. 45Als Mose aber dieses ganze Lied allen
Israeliten bis zu Ende vorgetragen hatte, 46sagte er zu ihnen:
»Beherzigt all diese Worte, die ich euch heute feierlich ans Herz lege!
Ihr sollt sie euren Kindern einprägen, damit sie auf die sorgfältige
Beobachtung aller Bestimmungen dieses Gesetzes bedacht sind; 47denn es
ist kein bedeutungsloses Wort für euch, sondern euer Leben hängt davon
ab, und durch die Beobachtung dieses Wortes werdet ihr ein langes
Bestehen in dem Lande gewinnen, in das ihr jetzt über den Jordan zieht,
um es in Besitz zu nehmen.«

48An demselben Tage aber sagte der HERR zu Mose\textless sup
title=``vgl. 4.Mose 27,12-14''\textgreater✲: 49»Steige auf das Gebirge
Abarim hier, auf den Berg Nebo, der im Lande der Moabiter, Jericho
gegenüber, liegt, und sieh dir das Land Kanaan an, das ich den
Israeliten zum Eigentum geben will. 50Dann sollst du auf dem Berge, auf
den du hinaufsteigen wirst, sterben und zu deinen Stammesgenossen
versammelt werden, wie dein Bruder Aaron auf dem Berge Hor gestorben und
zu seinen Stammesgenossen versammelt worden ist. 51Denn ihr habt euch
inmitten der Israeliten am Haderwasser von Kades in der Wüste Zin
treulos gegen mich erwiesen, weil ihr inmitten der Israeliten mir nicht
als dem Heiligen die Ehre gegeben habt. 52Denn nur gegenüber sollst du
in das Land, das ich den Israeliten geben will, hineinsehen, aber nicht
in das Land selbst hineinkommen!«

\hypertarget{der-abschiedssegen-moses-uxfcber-die-zwuxf6lf-stuxe4mme-israels}{%
\subsubsection{4. Der Abschiedssegen Moses über die zwölf Stämme
Israels}\label{der-abschiedssegen-moses-uxfcber-die-zwuxf6lf-stuxe4mme-israels}}

\hypertarget{a-uxfcberschrift-und-einleitung}{%
\paragraph{a) Überschrift und
Einleitung}\label{a-uxfcberschrift-und-einleitung}}

\hypertarget{section-32}{%
\section{33}\label{section-32}}

1Dies ist der Segen, den Mose, der Mann Gottes, über die Israeliten vor
seinem Tode ausgesprochen hat. 2Er sprach (damals): »Der HERR kam vom
Sinai her und erschien seinem Volk von Seir her in strahlendem Glanz; er
ließ sein Licht vom Gebirge Paran leuchten und kam aus der Mitte
heiliger Zehntausend-Scharen; zu seiner Rechten war loderndes Feuer für
sie✲. 3Ja, er umgab mit Liebe die Stämme; alle seine
Heiligen\textless sup title=``oder: Geweihten''\textgreater✲ sind in
seiner Hand; sie waren gelagert längs deiner Bahn, so daß jeder etwas
von deinen Aussprüchen empfing. 4Das Gesetz hat Mose uns verordnet als
Erbbesitz der Gemeinde Jakobs. 5So ist er denn König in
Jeschurun\textless sup title=``vgl. 32,15''\textgreater✲ geworden, als
die Häupter des Volks sich versammelten, die Stämme Israels allzumal.«

\hypertarget{b-segensspruxfcche-uxfcber-die-einzelnen-stuxe4mme}{%
\paragraph{b) Segenssprüche über die einzelnen
Stämme}\label{b-segensspruxfcche-uxfcber-die-einzelnen-stuxe4mme}}

6»Ruben lebe, er sterbe nicht, und seine Mannen seien eine
Anzahl\textless sup title=``=~eine große Zahl''\textgreater✲!« 7Und dies
ist der Segen für\textless sup title=``oder: über''\textgreater✲ Juda,
und zwar sagte er: »Erhöre, HERR, die Stimme Judas und bring ihn zurück
zu seinem Volk

mit seinen Händen hat er ja für es gestritten✲~-- und sei ihm Hilfe
wider seine Feinde!«

8Und von\textless sup title=``oder: für''\textgreater✲ Levi sagte er:
»Dein Recht-Orakel und dein Licht-Orakel\textless sup title=``vgl.
2.Mose 28,15.29-30''\textgreater✲ gehört den Männern deines Getreuen,
den du bei Massa versucht hast, für den du an den Wassern von Meriba
gestritten hast, 9jenen Männern, die von Vater und Mutter sagten: ›Ich
kenne sie nicht!‹\textless sup title=``vgl. 2.Mose 32,29''\textgreater✲
und die ihre Brüder nicht ansahn, von ihren Kindern nichts wissen
wollten; denn sie bewahrten dein Gebot und hielten an deinem Bunde fest.
10Sie sollen Jakob deine Verordnungen lehren und Israel dein Gesetz; sie
sollen Weihrauchduft zum Einatmen vor dich bringen und Ganzopfer auf
deinen Altar. 11Segne, HERR, seine Kraft\textless sup title=``oder: sein
Heer, oder: seinen Wohlstand?''\textgreater✲ und laß dir das Tun seiner
Hände gefallen! Zerschmettre seinen Gegnern die Hüften und seinen
Widersachern, daß sie nicht mehr aufstehn!«

12Von\textless sup title=``oder: für''\textgreater✲ Benjamin sagte er:
»Als Liebling des HERRN wird er in Ruhe bei ihm wohnen; er beschirmt ihn
allezeit und wohnt zwischen seinen Schultern✲.«

13Und von\textless sup title=``oder: für''\textgreater✲ Joseph sagte er:
»Gesegnet vom HERRN ist\textless sup title=``oder: sei''\textgreater✲
sein Land mit der köstlichsten Himmelsgabe, mit Tau, und mit der
Wasserflut, die drunten lagert; 14mit dem Köstlichsten, was die Sonne
hervorbringt, und dem Köstlichsten, was die Monde sprossen lassen; 15mit
dem Besten, was vom Gipfel der uralten Berge kommt, und dem Köstlichsten
der ewigen Hügel; 16mit dem Köstlichsten der Erde und ihrer Fülle! Und
das Wohlgefallen dessen, der im Dornbusch wohnte\textless sup
title=``vgl. 2.Mose 3,2''\textgreater✲, das möge kommen auf das Haupt
Josephs und auf den Scheitel des Geweihten unter seinen Brüdern! 17Ein
erstgeborener Stier ist er -- etwas Herrliches, und seine Hörner sind
die eines Wildochsen; mit ihnen stößt er Völker nieder, allesamt die
Enden der Erde. So sind die Zehntausende Ephraims und so die Tausende
Manasses.« 18Und von\textless sup title=``oder: für''\textgreater✲
Sebulon sagte er: »Freue dich, Sebulon, deiner Meerfahrten und du,
Issaschar, deiner Zelte! 19Sie laden Völker ein auf den Berg, dort
opfern sie rechte Opfer\textless sup title=``eig. Opfer der
Gerechtigkeit''\textgreater✲; denn den Reichtum der Meere saugen sie ein
und die verborgensten Schätze des Sandes\textless sup title=``oder:
Strandes''\textgreater✲.«

20Und von\textless sup title=``oder: für''\textgreater✲ Gad sagte er:
»Gepriesen sei der HERR, der Gad weiten Raum schafft! Wie eine Löwin hat
er sich gelagert und zerfleischt Arm und Schädel. 21Er hat sich das
Erstlingsgebiet ersehen; denn dort ward ihm sein Anteil vom Anführer
bestimmt. Doch ist er zu den Häuptern des Volks gekommen, hat die
Gerechtigkeit des HERRN vollstreckt und seine Strafgerichte gemeinsam
mit Israel.«

22Und von\textless sup title=``oder: für''\textgreater✲ Dan sagte er:
»Dan ist ein junger Löwe, der aus Basan hervorstürmt.«

23Und von\textless sup title=``oder: für''\textgreater✲ Naphthali sagte
er: »Naphthali ist gesättigt mit Glück und reichlich bedacht mit dem
Segen des HERRN; Meer und Südland nimm in Besitz!«

24Und von\textless sup title=``oder: für''\textgreater✲ Asser sagte er:
»Der gesegnetste unter den Söhnen sei Asser! Er sei der Liebling seiner
Brüder und tauche seinen Fuß in Öl! 25Von Eisen und Erz seien deine
Riegel, und solange du lebst, währe deine Ruhe\textless sup
title=``oder: Kraft?''\textgreater✲!«

\hypertarget{c-lobpreis-des-herrn-und-des-hochbegluxfcckten-israels}{%
\paragraph{c) Lobpreis des Herrn und des hochbeglückten
Israels}\label{c-lobpreis-des-herrn-und-des-hochbegluxfcckten-israels}}

26»Keiner ist dem Gott Jeschuruns\textless sup title=``vgl.
32,15''\textgreater✲ gleich, der über den Himmel dahinfährt als dein
Helfer und in seiner Hoheit auf den Wolken. 27Eine Zuflucht für dich ist
der Gott der Urzeit, und unter dir sind ewige Arme ausgebreitet; er hat
den Feind vor dir her vertrieben und dir geboten: ›Vertilge!‹ 28So wohnt
denn Israel in Sicherheit, abgesondert für sich der Quell Jakobs in
einem Land voll Korn und Wein, und sein Himmel träufelt Tau. 29Heil dir,
Israel! Wer ist dir gleich? Ein Volk, gerettet durch den HERRN! Er ist
der Schild, der dich schirmt, und das Schwert, das dir Ruhm\textless sup
title=``oder: Sieg''\textgreater✲ verschafft; deine Feinde müssen dir
huldigen, du aber schreitest dahin auf ihren Höhen!«

\hypertarget{mose-auf-dem-berge-nebo-sein-tod-und-begruxe4bnis-josua-wird-sein-nachfolger}{%
\subsubsection{5. Mose auf dem Berge Nebo; sein Tod und Begräbnis; Josua
wird sein
Nachfolger}\label{mose-auf-dem-berge-nebo-sein-tod-und-begruxe4bnis-josua-wird-sein-nachfolger}}

\hypertarget{section-33}{%
\section{34}\label{section-33}}

1Als Mose dann aus den Steppen der Moabiter auf den Berg Nebo, den
Gipfel des Pisga, der Jericho gegenüber\textless sup title=``=~östlich
von Jericho''\textgreater✲ liegt, gestiegen war, ließ der HERR ihn das
ganze Land sehen: Gilead bis nach Dan 2und ganz Naphthali, das Land
Ephraim und Manasse und die ganze Landschaft Juda bis an das westliche
Meer 3sowie das Südland und die Jordan-Aue, die Tiefebene der
Palmenstadt Jericho bis nach Zoar. 4Hierauf sagte der HERR zu ihm: »Dies
ist das Land, das ich Abraham, Isaak und Jakob zugeschworen habe mit den
Worten: ›Deiner Nachkommenschaft will ich es geben!‹ Ich habe es dich
mit eigenen Augen sehen lassen, aber hinüber sollst du nicht kommen!«
5So starb denn dort Mose, der Knecht des HERRN, im Lande der Moabiter
nach dem Befehl des HERRN; 6und er begrub ihn im Tal im Lande der
Moabiter, Beth-Peor gegenüber; aber niemand kennt sein Grab bis auf den
heutigen Tag. 7Mose war bei seinem Tode hundertundzwanzig Jahre alt;
seine Augen waren nicht schwach geworden, und seine Rüstigkeit war nicht
geschwunden. 8Die Israeliten beweinten Mose in den Steppen der Moabiter
dreißig Tage lang, bis die Tage des Weinens in der Trauer um ihn zu Ende
waren. 9Josua aber, der Sohn Nuns, war mit dem Geist der Weisheit
erfüllt, denn Mose hatte ihm die Hände fest aufgelegt; daher gehorchten
ihm die Israeliten und taten, wie der HERR dem Mose geboten hatte.

\hypertarget{wuxfcrdigung-der-gruxf6uxdfe-und-hohen-bedeutung-moses}{%
\paragraph{Würdigung der Größe und hohen Bedeutung
Moses}\label{wuxfcrdigung-der-gruxf6uxdfe-und-hohen-bedeutung-moses}}

10Es ist aber hinfort kein Prophet mehr in Israel aufgestanden wie Mose,
mit dem der HERR von Angesicht zu Angesicht verkehrt hätte; 11(keiner
ist mit ihm zu vergleichen) hinsichtlich aller der Zeichen und Wunder,
die der HERR ihn als seinen Gesandten in Ägypten am Pharao und all
seinen Dienern✲ und an seinem ganzen Lande hat vollführen lassen, 12und
hinsichtlich aller Erweise von gewaltiger Kraft und hinsichtlich aller
erstaunlichen Großtaten, die Mose vor den Augen von ganz Israel
vollbracht hat.
