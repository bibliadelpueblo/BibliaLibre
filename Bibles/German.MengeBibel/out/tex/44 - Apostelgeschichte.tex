\hypertarget{die-apostelgeschichte}{%
\section{DIE APOSTELGESCHICHTE}\label{die-apostelgeschichte}}

\hypertarget{i.-gruxfcndung-der-gemeinde-im-juxfcdischen-lande-und-in-syrien-durch-petrus-kap.-1-12}{%
\subsection{I. Gründung der Gemeinde im jüdischen Lande und in Syrien
durch Petrus (Kap.
1-12)}\label{i.-gruxfcndung-der-gemeinde-im-juxfcdischen-lande-und-in-syrien-durch-petrus-kap.-1-12}}

\hypertarget{a.-das-werden-und-wachsen-der-urgemeinde-im-juxfcdischen-lande-besonders-in-jerusalem-kap.-1-7}{%
\subsection{A. Das Werden und Wachsen der Urgemeinde im jüdischen Lande
(besonders in Jerusalem) (Kap.
1-7)}\label{a.-das-werden-und-wachsen-der-urgemeinde-im-juxfcdischen-lande-besonders-in-jerusalem-kap.-1-7}}

\hypertarget{jesu-letzte-anordnungen-und-seine-verheiuxdfung-an-die-juxfcnger-himmelfahrt}{%
\subsubsection{1. Jesu letzte Anordnungen und seine Verheißung an die
Jünger;
Himmelfahrt}\label{jesu-letzte-anordnungen-und-seine-verheiuxdfung-an-die-juxfcnger-himmelfahrt}}

\hypertarget{section}{%
\section{1}\label{section}}

\bibleverse{1} Meinen ersten Bericht habe ich, lieber Theophilus, über
alles das verfaßt✲, was Jesus getan und gelehrt hat von Anfang an
\bibleverse{2} bis zu dem Tage, an dem er den Aposteln, die er erwählt
hatte, durch den heiligen Geist seine (letzten) Aufträge erteilte und
dann (in den Himmel) aufgenommen wurde. \bibleverse{3} Ihnen hatte er
sich auch nach seinem Leiden durch viele Beweise als lebendig bezeugt,
indem er sich vierzig Tage lang vor ihnen sehen ließ und mit ihnen über
das Reich Gottes redete.

\bibleverse{4} Als er so mit ihnen zusammen war, gebot er ihnen, sich
von Jerusalem nicht zu entfernen, sondern (dort) die (Erfüllung der)
Verheißung des Vaters abzuwarten, »die ihr« -- so lauteten seine Worte
-- »von mir vernommen habt; \bibleverse{5} denn Johannes hat (nur) mit
Wasser getauft, ihr aber werdet mit heiligem Geist getauft werden, und
zwar nicht lange nach diesen Tagen\textless sup title=``oder: nach
wenigen Tagen von heute ab''\textgreater✲.« \bibleverse{6} Da fragten
ihn die dort Versammelten: »Herr, stellst du in dieser Zeit das
Königtum\textless sup title=``oder: das Reich''\textgreater✲ für (das
Volk) Israel wieder her?« \bibleverse{7} Er antwortete ihnen: »Euch
kommt es nicht zu, Zeiten und Fristen\textless sup title=``=~Zeit und
Stunde''\textgreater✲ zu wissen, die der Vater vermöge seiner eigenen
Machtvollkommenheit festgesetzt hat. \bibleverse{8} Ihr werdet jedoch
Kraft empfangen, wenn der heilige Geist auf euch kommt, und ihr werdet
Zeugen für mich sein in Jerusalem und in ganz Judäa und Samaria und bis
ans Ende der Erde.«

\bibleverse{9} Nach diesen Worten wurde er vor ihren Augen emporgehoben:
eine Wolke nahm ihn auf und entzog ihn ihren Blicken; \bibleverse{10}
und als sie ihm noch unverwandt nachschauten, während er zum Himmel
auffuhr, standen mit einemmal zwei Männer in weißen Gewändern bei ihnen,
\bibleverse{11} die sagten: »Ihr Männer aus Galiläa, was steht ihr da
und blickt zum Himmel empor? Dieser Jesus, der aus eurer Mitte in den
Himmel emporgehoben worden ist, wird in derselben Weise kommen, wie ihr
ihn in den Himmel habt auffahren sehen!«

\bibleverse{12} Darauf kehrten sie von dem sogenannten Ölberge, der nahe
bei Jerusalem liegt und nur einen Sabbatweg entfernt ist, nach Jerusalem
zurück. \bibleverse{13} Als sie dort angekommen waren, gingen sie in das
Obergemach (des Hauses) hinauf, wo sie sich aufzuhalten pflegten,
nämlich Petrus und Johannes und Jakobus und Andreas, Philippus und
Thomas, Bartholomäus und Matthäus, Jakobus, der Sohn des Alphäus, und
Simon der Eiferer und Judas, der Sohn des Jakobus. \bibleverse{14} Diese
alle waren dort einmütig und andauernd im Gebet vereinigt samt (einigen)
Frauen, besonders auch mit Maria, der Mutter Jesu, und mit seinen
Brüdern.

\hypertarget{ersatzwahl-eines-apostels-des-matthias-an-stelle-des-verruxe4ters-judas-iskariot}{%
\subsubsection{2. Ersatzwahl eines Apostels (des Matthias) an Stelle des
Verräters Judas
Iskariot}\label{ersatzwahl-eines-apostels-des-matthias-an-stelle-des-verruxe4ters-judas-iskariot}}

\bibleverse{15} In diesen Tagen nun trat Petrus im Kreise der Brüder auf
-- es war aber eine Schar von ungefähr einhundertundzwanzig Personen
versammelt -- und sagte: \bibleverse{16} »Liebe Brüder, das Schriftwort
mußte erfüllt werden, das der heilige Geist durch den Mund Davids im
voraus ausgesprochen hat\textless sup title=``Ps 41,10''\textgreater✲
über Judas, der denen, die Jesus gefangen nahmen, als Führer gedient
hat; \bibleverse{17} und er gehörte doch zu unserer Zahl und hatte
Anteil an diesem Dienst mit uns empfangen! \bibleverse{18} Dieser hat
sich nun zwar von seinem Sündenlohn einen Acker gekauft, ist aber
kopfüber zu Boden gestürzt und mitten auseinander geborsten, so daß alle
seine Eingeweide herausgetreten sind. \bibleverse{19} Dies ist allen
Einwohnern Jerusalems bekannt geworden, so daß auch jener Acker in ihrer
Sprache den Namen Hakeldamach, das heißt ›Blutacker‹, erhalten
hat\textless sup title=``vgl. Mt 27,5-8''\textgreater✲. \bibleverse{20}
Denn im Psalmbuch steht geschrieben\textless sup title=``Ps
69,26''\textgreater✲: ›Seine Behausung soll öde werden und kein Bewohner
darin sein‹, und ferner\textless sup title=``Ps 109,8''\textgreater✲:
›Sein Aufseheramt soll ein anderer übernehmen.‹ \bibleverse{21} Es muß
also einer von den Männern, die mit uns zusammen gezogen sind während
der ganzen Zeit, in welcher der Herr Jesus bei uns ein- und ausgegangen
ist, \bibleverse{22} nämlich von der Taufe des Johannes an bis zu dem
Tage, an dem er aus unserer Mitte hinweg (zum Himmel) emporgehoben
worden ist -- einer von diesen muß ein Zeuge✲ seiner Auferstehung im
Verein mit uns werden.« \bibleverse{23} So stellten sie denn zwei Männer
auf: Joseph, genannt Barsabbas, der den Beinamen Justus führte, und
Matthias. \bibleverse{24} Dann beteten sie mit den Worten: »Du, o Herr,
der du die Herzen aller kennst, zeige du (uns) den einen an, den du von
diesen beiden erwählt hast, \bibleverse{25} damit er die Stelle in
diesem Dienst und Apostelamt übernehme, aus welchem Judas
abgetreten\textless sup title=``oder: vorsätzlich
geschieden''\textgreater✲ ist, um an den ihm gebührenden Platz zu
kommen!« \bibleverse{26} Hierauf teilte man ihnen Lose zu, und das Los
fiel auf Matthias, der nunmehr den elf Aposteln zugeordnet wurde.

\hypertarget{das-pfingstfest}{%
\subsubsection{3. Das Pfingstfest}\label{das-pfingstfest}}

\hypertarget{a-das-pfingstwunder-die-ausgieuxdfung-des-heiligen-geistes-und-sein-gewaltiges-zeugnis-von-den-grouxdfen-taten-gottes}{%
\paragraph{a) Das Pfingstwunder: die Ausgießung des heiligen Geistes und
sein gewaltiges Zeugnis von den großen Taten
Gottes}\label{a-das-pfingstwunder-die-ausgieuxdfung-des-heiligen-geistes-und-sein-gewaltiges-zeugnis-von-den-grouxdfen-taten-gottes}}

\hypertarget{section-1}{%
\section{2}\label{section-1}}

\bibleverse{1} Als dann der Tag des Pfingstfestes herbeigekommen war,
befanden sie alle sich an einem Ort beisammen. \bibleverse{2} Da
entstand plötzlich ein Brausen\textless sup title=``oder:
Rauschen''\textgreater✲ vom Himmel her, wie wenn ein gewaltiger Wind
daherfährt, und erfüllte das ganze Haus\textless sup title=``oder:
Gemach''\textgreater✲, in welchem sie weilten; \bibleverse{3} und es
erschienen ihnen Zungen wie von Feuer, die sich (in Flämmchen)
zerteilten und von denen sich eine auf jeden von ihnen niederließ;
\bibleverse{4} und sie wurden alle mit heiligem Geist erfüllt und
begannen in anderen Zungen zu reden, wie\textless sup title=``oder: je
nachdem''\textgreater✲ der Geist es ihnen eingab
auszusprechen\textless sup title=``oder: sich vernehmen zu
lassen''\textgreater✲.

\bibleverse{5} Nun waren aber Juden in Jerusalem wohnhaft,
gottesfürchtige Männer aus allen Völkern unter dem Himmel.
\bibleverse{6} Als nun dieses Brausen\textless sup title=``oder:
Rauschen''\textgreater✲ erfolgt war, kamen sie in großer Zahl zusammen
und gerieten in Bestürzung; denn jeder hörte sie in seiner eigenen
Sprache reden. \bibleverse{7} Da wurden sie alle betroffen und fragten
voll Verwunderung: »Sind nicht diese alle, die da reden, aus Galiläa?
\bibleverse{8} Wie kommt es denn, daß wir ein jeder sie in unserer
eigenen Sprache reden hören, in der wir geboren\textless sup
title=``oder: groß geworden''\textgreater✲ sind: \bibleverse{9} Parther,
Meder und Elamiter✲ und wir Bewohner von Mesopotamien, von Judäa und
Kappadocien, von Pontus und (der Provinz) Asien, \bibleverse{10} von
Phrygien und Pamphylien, von Ägypten und der Landschaft Libyen bei
Cyrene, auch die hier ansässigen Römer, \bibleverse{11} geborene Juden
und Judengenossen\textless sup title=``=~zum Judentum übergetretene
Heiden''\textgreater✲, Kreter und Araber -- wir hören sie mit unsern
Zungen✲ die großen Taten Gottes verkünden!« \bibleverse{12} So waren sie
alle betroffen und ratlos und sagten einer zum anderen: »Was hat das zu
bedeuten?« \bibleverse{13} Andere aber spotteten und sagten: »Sie sind
voll süßen Weins!«

\hypertarget{b-die-pfingstrede-des-petrus}{%
\paragraph{b) Die Pfingstrede des
Petrus}\label{b-die-pfingstrede-des-petrus}}

\hypertarget{aa-erkluxe4rung-des-pfingstwunders-als-erfuxfcllung-des-alten-prophetenwortes-joels}{%
\subparagraph{aa) Erklärung des Pfingstwunders als Erfüllung des alten
Prophetenwortes
Joels}\label{aa-erkluxe4rung-des-pfingstwunders-als-erfuxfcllung-des-alten-prophetenwortes-joels}}

\bibleverse{14} Da trat Petrus im Verein mit den Elfen auf und redete
sie mit laut erhobener Stimme so an: »Ihr jüdischen Männer und ihr
anderen alle, die ihr in Jerusalem wohnt: dies sei euch kundgetan und
schenkt meinen Worten Gehör! \bibleverse{15} Diese Männer hier sind
nicht betrunken, wie ihr meint -- es ist ja erst die dritte Stunde des
Tages --, \bibleverse{16} nein, hier erfüllt sich die Verheißung des
Propheten Joel (3,1-5): \bibleverse{17} ›In den letzten Tagen wird es
geschehen, spricht Gott, da werde ich von meinem Geist auf alles Fleisch
ausgießen, so daß eure Söhne und eure Töchter prophetisch reden und eure
jungen Männer Gesichte schauen und eure Greise Offenbarungen in Träumen
empfangen; \bibleverse{18} ja, sogar auf meine Knechte und auf meine
Mägde werde ich in jenen Tagen von meinem Geist ausgießen, so daß sie
prophetisch reden. \bibleverse{19} Und ich werde Wunderzeichen
erscheinen lassen oben am Himmel und Wahrzeichen unten auf der Erde:
Blut und Feuer und Rauchwolken. \bibleverse{20} Die Sonne wird sich in
Finsternis verwandeln und der Mond in Blut, bevor der Tag des Herrn
kommt, der große und herrliche. \bibleverse{21} Und es wird geschehen:
Jeder, der den Namen des Herrn anruft, wird gerettet werden.‹«

\hypertarget{bb-jesus-der-gekreuzigte-auferstandene-und-von-gott-erhuxf6hte-hat-die-beiden-davidsworte-pps-168-11-und-1101}{%
\subparagraph{bb) Jesus, der Gekreuzigte, Auferstandene und von Gott
Erhöhte, hat die beiden Davidsworte \textbar pPs 16,8-11 und
110,1}\label{bb-jesus-der-gekreuzigte-auferstandene-und-von-gott-erhuxf6hte-hat-die-beiden-davidsworte-pps-168-11-und-1101}}

\bibleverse{22} »Ihr Männer von Israel, vernehmt diese Worte! Jesus von
Nazareth, einen Mann, der als Gottgesandter durch Machttaten, Wunder und
Zeichen, die Gott durch ihn in eurer Mitte getan hat, wie ihr selbst
wißt, vor euch erwiesen✲ worden ist~-- \bibleverse{23} diesen Mann, der
nach dem festgesetzten Ratschluß und der Vorherbestimmung Gottes euch
preisgegeben war, habt ihr durch die Hand der Gesetzlosen✲ ans Kreuz
nageln und hinrichten lassen. \bibleverse{24} Gott aber hat ihn
auferweckt, indem er die Wehen des Todes löste\textless sup
title=``oder: aufhob''\textgreater✲, weil er ja unmöglich vom Tode
festgehalten werden konnte. \bibleverse{25} David sagt ja mit Bezug auf
ihn\textless sup title=``Ps 16,8-11''\textgreater✲: ›Ich habe den Herrn
allezeit vor meinen Augen gesehen, denn er steht mir zur Rechten, damit
ich nicht wanke. \bibleverse{26} Deshalb freute sich mein Herz, und
meine Zunge frohlockte; zudem wird auch mein Leib in\textless sup
title=``oder: auf''\textgreater✲ Hoffnung ruhen; \bibleverse{27} denn du
wirst meine Seele✲ nicht im Totenreich belassen\textless sup
title=``oder: dem Totenreich überlassen''\textgreater✲ und nicht
zugeben, daß dein Heiliger die Verwesung sieht. \bibleverse{28} Du hast
mir Wege des Lebens\textless sup title=``oder: zum Leben''\textgreater✲
kundgetan; du wirst mich mit Freude erfüllen vor deinem Angesicht.‹

\bibleverse{29} Werte Brüder! Ich darf mit Freimütigkeit zu euch vom
Erzvater David reden: er ist gestorben und ist begraben worden, und sein
Grabmal befindet sich bis auf den heutigen Tag hier bei uns.
\bibleverse{30} Weil er nun ein Prophet war und wußte, daß Gott ihm mit
einem Eide zugeschworen hatte, es solle einer von seinen leiblichen
Nachkommen auf seinem Throne sitzen\textless sup title=``Ps
89,4-5''\textgreater✲, \bibleverse{31} so hat er vorausschauend von der
Auferstehung Christi\textless sup title=``=~des Messias''\textgreater✲
geredet, daß dieser nämlich weder dem Totenreich überlassen worden ist
noch sein Leib die Verwesung gesehen hat. \bibleverse{32} Diesen Jesus
hat Gott auferweckt: dafür sind wir alle Zeugen! \bibleverse{33} Nachdem
er nun durch die Rechte Gottes\textless sup title=``oder: zur Rechten
Gottes''\textgreater✲ erhöht worden ist und den verheißenen heiligen
Geist empfangen hat vom Vater, hat er jetzt diesen (Geist), wie ihr
selbst seht und hört, hier ausgegossen. \bibleverse{34} Denn nicht David
ist in die Himmel hinaufgefahren; wohl aber sagt er selbst\textless sup
title=``Ps 110,1''\textgreater✲: ›Der Herr hat zu meinem Herrn gesagt:
Setze dich zu meiner Rechten, \bibleverse{35} bis ich deine Feinde
hinlege zum Schemel deiner Füße!‹ \bibleverse{36} So möge denn das ganze
Haus Israel mit Sicherheit erkennen, daß Gott ihn zum Herrn und zum
Christus\textless sup title=``=~zum Messias''\textgreater✲ gemacht hat,
eben diesen Jesus, den ihr gekreuzigt habt!«

\hypertarget{c-wirkung-der-rede-erster-seelsorgerdienst-des-petrus-gruxfcndung-der-ersten-gemeinde}{%
\paragraph{c) Wirkung der Rede; erster Seelsorgerdienst des Petrus;
Gründung der ersten
Gemeinde}\label{c-wirkung-der-rede-erster-seelsorgerdienst-des-petrus-gruxfcndung-der-ersten-gemeinde}}

\bibleverse{37} Als sie das hörten, ging es ihnen wie ein Stich durchs
Herz, und sie wandten sich an Petrus und die anderen Apostel mit der
Frage: »Was sollen wir tun, werte Brüder?« \bibleverse{38} Da antwortete
ihnen Petrus: »Tut Buße\textless sup title=``vgl. Mt 3,2''\textgreater✲
und laßt euch ein jeder auf den Namen Jesu Christi zur Vergebung eurer
Sünden taufen, dann werdet ihr die Gabe des heiligen Geistes empfangen.
\bibleverse{39} Denn euch gilt die Verheißung und euren Kindern und
allen, die noch fern stehen, soviele ihrer der Herr, unser Gott, berufen
wird.« \bibleverse{40} Auch noch mit vielen anderen Worten redete er
ihnen eindringlich zu und ermahnte sie: »Laßt euch aus diesem verkehrten
Geschlecht erretten!« \bibleverse{41} Die nun sein Wort annahmen, ließen
sich taufen, und so kamen an jenem Tage etwa dreitausend Seelen (zu der
Gemeinde) hinzu.

\hypertarget{die-gemeinde-zu-jerusalem-242-67}{%
\subsubsection{4. Die Gemeinde zu Jerusalem
(2,42-6,7)}\label{die-gemeinde-zu-jerusalem-242-67}}

\hypertarget{a-das-leben-der-gluxe4ubigen-in-der-ersten-gemeinde}{%
\paragraph{a) Das Leben der Gläubigen in der ersten
Gemeinde}\label{a-das-leben-der-gluxe4ubigen-in-der-ersten-gemeinde}}

\bibleverse{42} Sie hielten aber beharrlich fest an der Lehre der
Apostel und an der (brüderlichen) Gemeinschaft, am Brechen des Brotes
und an den (gemeinsamen) Gebeten. \bibleverse{43} Und über jedermann (im
Volk) kam Furcht, und viele Wunder und Zeichen geschahen durch die
Apostel. \bibleverse{44} Alle Gläubiggewordenen aber waren
beisammen\textless sup title=``oder: hielten fest
zusammen''\textgreater✲ und hatten alles gemeinsam; \bibleverse{45} sie
verkauften ihre Besitztümer und ihre Habe und verteilten (den Erlös)
unter alle nach Maßgabe der Bedürftigkeit eines jeden; \bibleverse{46}
und indem sie am täglichen Besuch des Tempels mit Einmütigkeit
festhielten und das Brot in den einzelnen Häusern brachen, genossen sie
ihre (tägliche) Nahrung mit Frohlocken und in Herzenseinfalt,
\bibleverse{47} priesen Gott und standen mit dem ganzen Volk in gutem
Einvernehmen. Der Herr aber fügte täglich solche, die gerettet
wurden\textless sup title=``oder: gerettet werden
sollten''\textgreater✲, zu festem Anschluß hinzu.

\hypertarget{b-uxf6ffentliches-wirken-des-petrus-und-johannes}{%
\paragraph{b) Öffentliches Wirken des Petrus und
Johannes}\label{b-uxf6ffentliches-wirken-des-petrus-und-johannes}}

\hypertarget{aa-heilung-eines-lahmgeborenen-durch-petrus-und-johannes}{%
\subparagraph{aa) Heilung eines Lahmgeborenen durch Petrus und
Johannes}\label{aa-heilung-eines-lahmgeborenen-durch-petrus-und-johannes}}

\hypertarget{section-2}{%
\section{3}\label{section-2}}

\bibleverse{1} Petrus und Johannes aber gingen (eines Tages) zusammen um
die neunte Stunde, die Gebetsstunde, in den Tempel hinauf.
\bibleverse{2} Da wurde (gerade) ein Mann herbeigetragen, der von seiner
Geburt an lahm war und den man täglich an das sogenannte Schöne Tor des
Tempels hinsetzte, damit er sich dort Almosen von den Besuchern des
Tempels erbitte. \bibleverse{3} Als dieser nun Petrus und Johannes sah,
die in den Tempel hineingehen wollten, bat er sie um ein Almosen.
\bibleverse{4} Da sah Petrus samt Johannes ihn scharf an und sagte:
»Sieh uns an!« \bibleverse{5} Jener blickte sie nun aufmerksam an in der
Erwartung, eine Gabe von ihnen zu erhalten. \bibleverse{6} Petrus aber
sagte: »Silber und Gold besitze ich nicht; was ich aber habe, das gebe
ich dir: Im Namen Jesu Christi von Nazareth: Gehe umher!« \bibleverse{7}
Dann faßte er ihn bei der rechten Hand und richtete ihn auf; da wurden
seine Füße und Knöchel augenblicklich fest; \bibleverse{8} er sprang
auf, konnte stehen, ging umher und trat mit ihnen in den Tempel ein,
indem er umherging und sprang und Gott pries. \bibleverse{9} So sahen
ihn denn alle Leute, wie er umherging und Gott pries; \bibleverse{10}
sie erkannten in ihm auch den Mann, der sonst, um Almosen zu erbitten,
am Schönen Tore des Tempels gesessen hatte, und wurden voller Staunen
und Verwunderung wegen der Heilung, die an ihm vorgegangen war.
\bibleverse{11} Da er sich aber fest zu Petrus und Johannes hielt,
strömte das gesamte Volk zu ihnen bei der sogenannten Halle Salomos
zusammen, außer sich vor Staunen.

\hypertarget{bb-tempelrede-buuxdfpredigt-des-petrus-nach-der-lahmenheilung}{%
\subparagraph{bb) Tempelrede (Bußpredigt des Petrus nach der
Lahmenheilung)}\label{bb-tempelrede-buuxdfpredigt-des-petrus-nach-der-lahmenheilung}}

\bibleverse{12} Als Petrus das sah, richtete er folgende Ansprache an
das Volk: »Ihr Männer von Israel, was wundert ihr euch über diesen Mann,
oder was seht ihr uns so erstaunt an, als hätten wir durch eigene Kraft
oder Frömmigkeit bewirkt, daß er umhergehen kann? \bibleverse{13} Nein,
der Gott Abrahams, Isaaks und Jakobs, der Gott unserer Väter, hat seinen
Knecht Jesus verherrlicht, den ihr (den Heiden) ausgeliefert und vor
Pilatus verleugnet habt, als dieser seine Freigebung beschlossen hatte;
\bibleverse{14} da habt ihr den Heiligen und Gerechten verleugnet und
die Begnadigung eines Mörders verlangt, \bibleverse{15} dagegen den
Fürsten des Lebens\textless sup title=``oder: den Führer zum
Leben''\textgreater✲ hinrichten lassen. Diesen hat Gott von den Toten
auferweckt: dafür sind wir Zeugen! \bibleverse{16} Und auf Grund des✲
Glaubens an seinen Namen hat sein Name diesem Manne hier, den ihr seht
und kennt, jetzt Kraft verliehen, und der durch Jesus (in uns) gewirkte
Glaube hat ihm vor euer aller Augen diese seine gesunden Glieder
geschenkt. \bibleverse{17} Und nun, ihr Brüder: ich weiß, daß ihr aus
Unwissenheit gehandelt habt, wie auch eure Oberen; \bibleverse{18} Gott
aber hat auf diese Weise das in Erfüllung gehen lassen, was er schon
vorher durch den Mund aller Propheten verkündigt hat, daß nämlich sein
Gesalbter\textless sup title=``=~Christus, oder: der
Messias''\textgreater✲ leiden werde. \bibleverse{19} So tut denn
Buße\textless sup title=``vgl. Mt 3,2''\textgreater✲ und bekehrt euch,
damit eure Sünden vergeben werden, \bibleverse{20} auf daß Zeiten der
Erquickung vom Angesicht des Herrn kommen und er den für euch zum
Gesalbten✲ bestimmten Jesus senden kann. \bibleverse{21} Diesen muß
allerdings der Himmel aufnehmen bis zu den Zeiten der Wiederherstellung
alles dessen, was Gott durch den Mund seiner heiligen Propheten von der
Urzeit her verkündet hat. \bibleverse{22} Mose hat ja
gesagt\textless sup title=``5.Mose 18,15-19''\textgreater✲: ›Einen
Propheten wie mich wird der Herr, unser Gott, euch aus euren Brüdern
erstehen lassen: auf den sollt ihr in allem hören, was er zu euch reden
wird; \bibleverse{23} und jede Seele, die auf diesen Propheten nicht
hört, soll aus dem Volke ausgerottet werden!‹ \bibleverse{24} Aber auch
alle anderen Propheten, soviele ihrer von Samuel an und in den folgenden
Zeiten aufgetreten sind, haben diese Tage angekündigt. \bibleverse{25}
Ihr seid die Söhne der Propheten und gehört dem Bunde an, den Gott mit
euren Vätern geschlossen hat, als er dem Abraham die Verheißung
gab\textless sup title=``1.Mose 22,18''\textgreater✲: ›In deiner
Nachkommenschaft\textless sup title=``oder: durch einen von deinen
Nachkommen''\textgreater✲ sollen alle Geschlechter der Erde gesegnet
werden.‹ \bibleverse{26} Für euch zuerst hat Gott seinen Knecht erstehen
lassen und ihn gesandt, um euch dadurch\textless sup title=``oder: unter
der Bedingung''\textgreater✲ zu segnen, daß ein jeder unter euch sich
von seinen Bosheiten bekehrt.«

\hypertarget{c-das-einschreiten-der-juxfcdischen-behuxf6rde-petrus-und-johannes-im-gefuxe4ngnis-und-vor-dem-hohen-rat-ihre-freilassung}{%
\paragraph{c) Das Einschreiten der jüdischen Behörde: Petrus und
Johannes im Gefängnis und vor dem Hohen Rat; ihre
Freilassung}\label{c-das-einschreiten-der-juxfcdischen-behuxf6rde-petrus-und-johannes-im-gefuxe4ngnis-und-vor-dem-hohen-rat-ihre-freilassung}}

\hypertarget{section-3}{%
\section{4}\label{section-3}}

\bibleverse{1} Während sie noch zum Volk redeten, traten die Priester
sowie der Tempelhauptmann und die Sadduzäer an sie heran; \bibleverse{2}
diese waren nämlich unwillig darüber, daß sie dem Volk ihre Lehre
vortrugen und die an Jesus vollführte Auferstehung von den Toten
verkündeten. \bibleverse{3} Sie nahmen sie also fest und setzten sie in
Gewahrsam bis zum folgenden Morgen; es war nämlich bereits Abend.
\bibleverse{4} Viele aber von denen, welche die Ansprache (des Petrus)
gehört hatten, waren zum Glauben gekommen, so daß die Zahl der
(gläubigen) Männer sich (nunmehr) auf etwa fünftausend belief.

\bibleverse{5} Am andern Morgen aber versammelten sich ihre
Oberen\textless sup title=``d.h. die Mitglieder des Hohen
Rates''\textgreater✲ sowie die Ältesten und Schriftgelehrten in
Jerusalem, \bibleverse{6} auch der Hohepriester Hannas, ferner Kaiphas,
Johannes, Alexander und alle, die zu einer hohenpriesterlichen Familie
gehörten. \bibleverse{7} Dann ließ man sie vorführen und fragte sie:
»Durch was für eine Kraft oder durch welchen Namen habt ihr dies
(Zeichen) vollführt?« \bibleverse{8} Da wurde Petrus mit heiligem Geist
erfüllt und sagte zu ihnen: »Ihr Oberen des Volkes und ihr Ältesten!
\bibleverse{9} Wenn wir uns heute wegen der einem kranken Menschen
erwiesenen Wohltat zu verantworten haben (und gefragt werden), durch wen
dieser Mann hier gesund geworden sei, \bibleverse{10} so soll euch allen
und dem Volk Israel kundgetan sein: In der Kraft des Namens Jesu Christi
von Nazareth, den ihr gekreuzigt habt, den Gott aber von den Toten
auferweckt hat -- ja, durch dessen Namen steht der Mann hier gesund vor
euch! \bibleverse{11} Dieser (Jesus) ist der von euch Bauleuten
verworfene Stein, der zum Eckstein geworden ist\textless sup title=``Ps
118,22''\textgreater✲; \bibleverse{12} und in keinem andern ist die
Rettung\textless sup title=``oder: das Heil''\textgreater✲ zu finden;
denn es ist auch kein anderer Name unter dem Himmel den Menschen
gegeben, in dem\textless sup title=``oder: durch den''\textgreater✲ wir
gerettet werden sollen.«

\bibleverse{13} Als sie nun die freudige Zuversicht des Petrus und
Johannes sahen und in Erfahrung gebracht\textless sup title=``oder:
gemerkt''\textgreater✲ hatten, daß es Menschen ohne Schulung und ohne
gelehrte Bildung waren, verwunderten sie sich; sie erkannten auch wohl,
daß sie Begleiter✲ Jesu gewesen waren, \bibleverse{14} und weil sie den
Mann, der geheilt worden war, bei ihnen stehen sahen, wußten sie nichts
zu entgegnen. \bibleverse{15} Sie ließen sie also aus der Ratssitzung
abtreten und berieten miteinander die Frage: \bibleverse{16} »Was sollen
wir mit diesen Menschen machen? Denn daß ein unverkennbares
Wunderzeichen durch sie geschehen ist, das ist allen Einwohnern
Jerusalems bekannt geworden, und wir können es nicht bestreiten.
\bibleverse{17} Damit aber die Kunde davon sich nicht noch weiter unter
dem Volk verbreitet, wollen wir ihnen ernstlich gebieten, in Zukunft zu
keinem Menschen mehr unter Nennung dieses Namens zu reden.«
\bibleverse{18} Sie ließen sie dann wieder hereinrufen und geboten
ihnen, überhaupt nichts mehr unter Nennung des Namens Jesu verlauten zu
lassen und zu lehren. \bibleverse{19} Doch Petrus und Johannes
antworteten ihnen mit den Worten: »Urteilt selbst, ob es vor Gott recht
ist, euch mehr zu gehorchen als Gott! \bibleverse{20} Wir unserseits
können es ja unmöglich unterlassen, von dem zu reden, was wir gesehen
und gehört haben!« \bibleverse{21} Jene fügten noch weitere Drohungen
hinzu, ließen sie dann aber frei, weil sie keine Möglichkeit fanden, sie
zu bestrafen, (auch) mit Rücksicht auf das Volk, weil alle Gott wegen
des Geschehenen\textless sup title=``d.h. der vorliegenden
Heilung''\textgreater✲ priesen; \bibleverse{22} denn über vierzig Jahre
war der Mann alt, an dem dieses Heilungswunder geschehen war.

\hypertarget{d-ruxfcckkehr-der-apostel-dank--und-bittgebet-der-gemeinde}{%
\paragraph{d) Rückkehr der Apostel; Dank- und Bittgebet der
Gemeinde}\label{d-ruxfcckkehr-der-apostel-dank--und-bittgebet-der-gemeinde}}

\bibleverse{23} Nach ihrer Freilassung kehrten Petrus und Johannes zu
den Ihrigen zurück und berichteten ihnen alles, was die Hohenpriester
und die Ältesten zu ihnen gesagt hatten. \bibleverse{24} Als jene es
vernommen hatten, erhoben sie einmütig ihre Stimme zu Gott und beteten:
»Herr, du bist es, der den Himmel und die Erde, das Meer und alles, was
in ihnen ist, geschaffen hat; \bibleverse{25} du hast durch den heiligen
Geist zu unsern Vätern durch den Mund Davids, deines Knechtes,
gesagt\textless sup title=``Ps 2,1-2''\textgreater✲: ›Was soll das Toben
der Heiden und das eitle Sinnen der Völker? \bibleverse{26} Die Könige
der Erde erheben sich, und die Fürsten rotten sich zusammen gegen den
Herrn und gegen seinen Gesalbten!‹ \bibleverse{27} Ja, es haben sich in
Wahrheit gegen deinen heiligen Knecht Jesus, den du gesalbt hast, in
dieser Stadt Herodes und Pontius Pilatus mit den Heiden und den
Volksscharen\textless sup title=``oder: Stämmen''\textgreater✲ Israels
zusammengetan, \bibleverse{28} um alles auszuführen, was deine Hand✲ und
dein Ratschluß vorherbestimmt haben, daß es geschehen sollte.
\bibleverse{29} Und jetzt, Herr, blicke hin auf ihre Drohungen und
verleihe deinen Knechten Kraft, dein Wort mit allem Freimut zu
verkündigen! \bibleverse{30} Strecke deine Hand dabei zu Heilungen aus
und laß Zeichen und Wunder durch den Namen deines heiligen Knechtes
Jesus geschehen!« \bibleverse{31} Als sie so gebetet hatten, erbebte die
Stätte, wo sie versammelt waren, und sie wurden alle vom heiligen Geist
erfüllt und verkündigten das Wort Gottes unerschrocken.

\hypertarget{e-zweite-schilderung-des-inneren-lebens-der-ersten-gemeinde}{%
\paragraph{e) Zweite Schilderung des inneren Lebens der ersten
Gemeinde}\label{e-zweite-schilderung-des-inneren-lebens-der-ersten-gemeinde}}

\hypertarget{aa-die-guxfctergemeinschaft}{%
\subparagraph{aa) Die
Gütergemeinschaft}\label{aa-die-guxfctergemeinschaft}}

\bibleverse{32} Die Gesamtheit der Gläubiggewordenen aber war ein Herz
und eine Seele, und kein einziger nannte ein Stück seines Besitzes sein
ausschließliches Eigentum, sondern sie hatten alles als Gemeingut.
\bibleverse{33} Zudem legten die Apostel mit großem Nachdruck Zeugnis
von der Auferstehung des Herrn Jesus ab, und alle erfreuten sich
großer\textless sup title=``oder: allgemeiner''\textgreater✲
Beliebtheit. \bibleverse{34} Denn es gab auch keinen Notleidenden unter
ihnen; alle nämlich, welche Ländereien oder Häuser besaßen, verkauften
diese, brachten dann den Erlös aus dem Verkauf \bibleverse{35} und
stellten ihn den Aposteln zur Verfügung; davon wurde dann jedem nach
seiner Bedürftigkeit zugeteilt. \bibleverse{36} So besaß (z.B.) Joseph,
der von den Aposteln den Beinamen Barnabas, das heißt auf deutsch ›Sohn
der Tröstung\textless sup title=``=~der Tröster''\textgreater✲‹,
erhalten hatte, ein Levit, aus Cypern gebürtig, einen Acker;
\bibleverse{37} den verkaufte er, brachte dann den Geldbetrag und
stellte ihn den Aposteln zur Verfügung.

\hypertarget{bb-ein-beispiel-der-ernsten-gemeindezucht-ananias-und-sapphira}{%
\subparagraph{bb) Ein Beispiel der ernsten Gemeindezucht: Ananias und
Sapphira}\label{bb-ein-beispiel-der-ernsten-gemeindezucht-ananias-und-sapphira}}

\hypertarget{section-4}{%
\section{5}\label{section-4}}

\bibleverse{1} Ein Mann dagegen namens Ananias verkaufte im Einvernehmen
mit seiner Frau Sapphira ein Grundstück, \bibleverse{2} behielt aber
einen Teil des Erlöses unter Mitwissen seiner Frau für sich zurück: er
brachte nur einen Teil davon und stellte ihn den Aposteln zur Verfügung.
\bibleverse{3} Da sagte Petrus: »Ananias, warum hat der Satan dir das
Herz erfüllt✲, daß du den heiligen Geist belogen und einen Teil vom
Erlös des Ackers für dich zurückbehalten hast? \bibleverse{4} Blieb er
nicht dein Eigentum, wenn du ihn unverkauft gelassen hättest, und stand
dir nicht auch nach dem Verkauf die Verfügung über ihn\textless sup
title=``d.h. den Erlös''\textgreater✲ frei? Warum hast du dir eine
solche Handlungsweise in den Sinn kommen lassen? Du hast nicht Menschen
belogen, sondern Gott!« \bibleverse{5} Als Ananias diese Worte hörte,
fiel er nieder und gab seinen Geist auf. Da kam große Furcht über alle,
die es hörten\textless sup title=``oder: die zugehört
hatten''\textgreater✲. \bibleverse{6} Die jüngeren Männer aber standen
auf, hüllten die Leiche in Tücher und trugen sie zum Begräbnis hinaus.

\bibleverse{7} Nach Verlauf von etwa drei Stunden trat auch seine Frau
ein, ohne von dem Vorgefallenen etwas zu wissen. \bibleverse{8} Petrus
redete sie mit den Worten an: »Sage mir: habt ihr das Grundstück für
diesen Preis verkauft?« Sie antwortete: »Ja, für diesen Preis.«
\bibleverse{9} Da sagte Petrus zu ihr: »Warum habt ihr beide euch
verabredet, den Geist des Herrn zu versuchen? Siehe, die Füße derer, die
deinen Mann zu Grabe getragen haben, stehen vor der Tür, und sie werden
auch dich hinaustragen!« \bibleverse{10} Da fiel sie augenblicklich zu
seinen Füßen nieder und gab ihren Geist auf; und als die jungen Männer
hereinkamen, fanden sie sie als Leiche vor; sie trugen sie hinaus und
begruben sie bei ihrem Manne. \bibleverse{11} Da kam eine große Furcht
über die ganze Gemeinde und über alle, die davon hörten.

\hypertarget{cc-wundertaten-besonders-krankenheilungen-der-apostel-weiteres-wachsen-der-gemeinde}{%
\subparagraph{cc) Wundertaten (besonders Krankenheilungen) der Apostel;
weiteres Wachsen der
Gemeinde}\label{cc-wundertaten-besonders-krankenheilungen-der-apostel-weiteres-wachsen-der-gemeinde}}

\bibleverse{12} Durch die Hände der Apostel aber geschahen viele Zeichen
und Wunder unter dem Volke, und alle (Gläubigen) pflegten sich einmütig
in der Halle Salomos zu versammeln; \bibleverse{13} von den übrigen aber
wagte sich niemand dort störend an sie heranzudrängen, sondern das Volk
hielt sie hoch in Ehren. \bibleverse{14} Und immer mehr kamen solche
hinzu, die an den Herrn glaubten, ganze Scharen von Männern und Frauen;
\bibleverse{15} ja man brachte die Kranken sogar auf die Straßen hinaus
und legte sie dort auf Betten und Bahren, damit, wenn Petrus käme,
wenigstens sein Schatten auf den einen oder andern von ihnen fiele.
\bibleverse{16} Aber auch aus den rings um Jerusalem liegenden
Ortschaften strömte die Bevölkerung zusammen und brachte Kranke und von
unreinen Geistern Geplagte dorthin, die dann alle geheilt wurden.

\hypertarget{f-verhaftung-der-apostel-und-verhandlung-vor-dem-hohen-rat-geiuxdfelung-und-freilassung-der-gefangenen}{%
\paragraph{f) Verhaftung der Apostel und Verhandlung vor dem Hohen Rat;
Geißelung und Freilassung der
Gefangenen}\label{f-verhaftung-der-apostel-und-verhandlung-vor-dem-hohen-rat-geiuxdfelung-und-freilassung-der-gefangenen}}

\hypertarget{aa-die-verhaftung-befreiung-durch-einen-engel}{%
\subparagraph{aa) Die Verhaftung; Befreiung durch einen
Engel}\label{aa-die-verhaftung-befreiung-durch-einen-engel}}

\bibleverse{17} Da erhob sich aber der Hohepriester samt seinem ganzen
Anhang, nämlich die Partei\textless sup title=``oder:
Sonderrichtung''\textgreater✲ der Sadduzäer; sie waren voll von
Eifersucht, \bibleverse{18} ließen die Apostel festnehmen und sie in das
öffentliche Gefängnis setzen. \bibleverse{19} Aber ein Engel des Herrn
öffnete während der Nacht die Gefängnistüren, führte sie hinaus und
gebot ihnen: \bibleverse{20} »Geht hin, tretet offen auf und verkündigt
im Tempel dem Volk alle diese Lebensworte\textless sup title=``oder:
diese ganze Botschaft des Lebens''\textgreater✲!« \bibleverse{21} Als
sie das gehört hatten, begaben sie sich mit Tagesanbruch in den Tempel
und lehrten dort. Der Hohepriester und sein Anhang hatten sich
inzwischen eingestellt und den Hohen Rat und die gesamte Ältestenschaft
der Israeliten zur Versammlung berufen lassen und sandten nun ins
Gefängnis, um sie vorführen zu lassen. \bibleverse{22} Als aber die
Diener hinkamen, fanden sie sie im Gefängnis nicht vor und meldeten nach
ihrer Rückkehr: \bibleverse{23} »Wir haben das Gefängnis ganz fest
verschlossen gefunden, auch die Wächter standen auf ihrem Posten an den
Türen; als wir aber aufschlossen, haben wir niemand drinnen
vorgefunden.« \bibleverse{24} Als der Tempelhauptmann und die
Hohenpriester diese Meldung vernahmen, waren sie ihretwegen ratlos,
welche Bewandtnis es damit wohl haben möchte. \bibleverse{25} Da kam
einer und meldete ihnen: »Denkt nur! Die Männer, die ihr ins Gefängnis
gesetzt habt, die stehen jetzt im Tempel und lehren das Volk!«
\bibleverse{26} Da ging der Hauptmann mit den Dienern hin und holte sie
herbei, doch ohne Anwendung von Gewalt; denn sie hatten zu befürchten,
vom Volk gesteinigt zu werden.

\hypertarget{bb-der-apostel-mutiges-zeugnis-von-der-auferstehung-christi}{%
\subparagraph{bb) Der Apostel mutiges Zeugnis von der Auferstehung
Christi}\label{bb-der-apostel-mutiges-zeugnis-von-der-auferstehung-christi}}

\bibleverse{27} Sie brachten sie also herbei und stellten sie vor den
Hohen Rat; und der Hohepriester verhörte sie folgendermaßen:
\bibleverse{28} »Wir haben euch doch ausdrücklich geboten, auf
Grund\textless sup title=``oder: unter Nennung''\textgreater✲ dieses
Namens nicht zu lehren, und trotzdem habt ihr mit eurer Lehre ganz
Jerusalem erfüllt und wollt das Blut dieses Menschen auf uns bringen!«
\bibleverse{29} Da antwortete Petrus, und die Apostel erklärten: »Man
muß Gott mehr gehorchen als den Menschen! \bibleverse{30} Der Gott
unserer Väter hat Jesus auferweckt, den ihr ans Holz
gehängt\textless sup title=``=~ans Kreuz geschlagen''\textgreater✲ und
hingerichtet habt. \bibleverse{31} Diesen hat Gott durch seine rechte
Hand zum Anführer\textless sup title=``oder: Fürsten''\textgreater✲ und
Retter\textless sup title=``oder: Heiland''\textgreater✲ erhöht, um
Israel Buße und Vergebung der Sünden zu verleihen. \bibleverse{32} Für
diese Tatsachen sind wir Zeugen und auch der heilige Geist, den Gott
denen verliehen hat, die ihm gehorsam sind.« \bibleverse{33} Als jene
das hörten, ging es ihnen wie ein Stich durchs Herz, und sie waren
entschlossen, sie hinrichten zu lassen.

\hypertarget{cc-gamaliels-fuxfcrsprache-und-rat}{%
\subparagraph{cc) Gamaliels Fürsprache und
Rat}\label{cc-gamaliels-fuxfcrsprache-und-rat}}

\bibleverse{34} Doch da stand in der Versammlung ein Pharisäer namens
Gamaliel auf, ein beim ganzen Volk hochangesehener Gesetzeslehrer; er
ließ die Männer auf kurze Zeit hinausführen \bibleverse{35} und richtete
dann folgende Ansprache an die Versammlung: »Ihr Männer von Israel,
überlegt euch wohl, wie ihr mit diesen Leuten verfahren wollt!
\bibleverse{36} Denn vor einiger Zeit trat Theudas auf und gab sich für
etwas Besonderes aus, und eine Anzahl von etwa vierhundert Männern fiel
ihm zu; aber er fand den Tod, und alle seine Anhänger wurden zersprengt
und vernichtet. \bibleverse{37} Nach diesem trat Judas aus Galiläa zur
Zeit der Schätzung✲ auf und brachte einen Volkshaufen zum Aufstand unter
seiner Führung; aber auch er kam ums Leben, und sein ganzer Anhang wurde
zerstreut. \bibleverse{38} Und nunmehr gebe ich euch den Rat: Steht von
diesen Leuten ab und laßt sie gewähren! Denn wenn dieses Vorhaben oder
dieses Werk von Menschen ausgeht, so wird es zunichte werden;
\bibleverse{39} hat es aber seinen Ursprung in Gott, so werdet ihr diese
Leute nicht zu vernichten vermögen. Laßt ihr euch nur nicht gar als
Widersacher Gottes erfinden!«

\hypertarget{dd-ausgang-und-folgen-des-vorkommnisses}{%
\subparagraph{dd) Ausgang und Folgen des
Vorkommnisses}\label{dd-ausgang-und-folgen-des-vorkommnisses}}

Daraufhin folgten sie seinem Rat: \bibleverse{40} sie ließen die Apostel
wieder hereinrufen und geißeln und befahlen ihnen, auf Grund des Namens
Jesu nicht mehr zu predigen; dann gab man sie frei. \bibleverse{41} So
gingen sie denn aus dem Hohen Rat weg, hocherfreut, daß sie gewürdigt
worden waren, um des Namens (Jesu) willen Schmach zu erleiden;
\bibleverse{42} und sie hörten nicht auf, täglich im Tempel und in den
Häusern zu lehren und die Heilsbotschaft von Christus Jesus zu
verkündigen.

\hypertarget{g-trennung-des-predigtamtes-und-der-armenpflege-wahl-und-einsetzung-der-sieben-armenpfleger}{%
\paragraph{g) Trennung des Predigtamtes und der Armenpflege; Wahl und
Einsetzung der sieben
Armenpfleger}\label{g-trennung-des-predigtamtes-und-der-armenpflege-wahl-und-einsetzung-der-sieben-armenpfleger}}

\hypertarget{section-5}{%
\section{6}\label{section-5}}

\bibleverse{1} In diesen Tagen nun entstand bei der Zunahme der Zahl der
Jünger laute Unzufriedenheit der Hellenisten gegen die Hebräer, weil
ihre Witwen bei der täglichen Verpflegung\textless sup title=``oder:
Versorgung''\textgreater✲ nicht genügend berücksichtigt würden.
\bibleverse{2} So beriefen denn die Zwölf die Gesamtheit der Jünger und
sagten: »Es scheint uns nicht das Richtige zu sein, daß wir die
Verkündigung des Wortes Gottes hintansetzen, um den Tischdienst zu
besorgen. \bibleverse{3} So seht euch nun, ihr Brüder, nach sieben
bewährten, mit Geist und Weisheit erfüllten Männern aus eurer Mitte um,
damit wir sie zu diesem Dienst\textless sup title=``oder: für dieses
Amt''\textgreater✲ bestellen; \bibleverse{4} wir selbst aber wollen uns
ausschließlich dem Gebet und dem Dienst am Wort widmen.« \bibleverse{5}
Dieser Vorschlag fand den Beifall der ganzen Versammlung, und man wählte
Stephanus, einen Mann voll Glaubens und heiligen Geistes, ferner den
Philippus, Prochorus, Nikanor, Timon, Parmenas und Nikolaus, einen
Judengenossen\textless sup title=``=~zum Judentum übergetretenen
Heiden''\textgreater✲ aus Antiochia. \bibleverse{6} Diese Männer ließ
man vor die Apostel hintreten, die (für sie) beteten und ihnen die Hände
auflegten.

\bibleverse{7} Das Wort Gottes breitete sich nun immer weiter aus, und
die Zahl der Jünger vermehrte sich in Jerusalem stark; sogar eine große
Menge von Priestern wurde dem Glauben gehorsam\textless sup
title=``=~nahm den Glauben an''\textgreater✲.

\hypertarget{anklage-und-tod-des-stephanus-des-ersten-blutzeugen}{%
\subsubsection{5. Anklage und Tod des Stephanus, des ersten
Blutzeugen}\label{anklage-und-tod-des-stephanus-des-ersten-blutzeugen}}

\hypertarget{a-die-uxf6ffentliche-wirksamkeit-des-stephanus-zieht-ihm-eine-anklage-zu}{%
\paragraph{a) Die öffentliche Wirksamkeit des Stephanus zieht ihm eine
Anklage
zu}\label{a-die-uxf6ffentliche-wirksamkeit-des-stephanus-zieht-ihm-eine-anklage-zu}}

\bibleverse{8} Stephanus aber, ein Mann voll Gnade und Geisteskraft, tat
Wunder und große Zeichen unter dem Volke. \bibleverse{9} Da traten
einige Mitglieder der sogenannten Synagoge der Freigelassenen sowie der
Cyrenäer und Alexandriner und der Synagogen der Cilicier und der Provinz
Asien auf und führten mit Stephanus Streitgespräche, \bibleverse{10}
vermochten jedoch gegen die Weisheit und den Geist, mit dem er redete,
nicht aufzukommen. \bibleverse{11} Infolgedessen stifteten sie Männer
an, die aussagen mußten: »Wir haben ihn Lästerworte gegen Mose und gegen
Gott aussprechen hören.« \bibleverse{12} So wiegelten sie denn das Volk
sowie die Ältesten und die Schriftgelehrten auf, fielen dann über ihn
her, schleppten ihn mit sich und führten ihn vor den Hohen Rat.
\bibleverse{13} Hier ließen sie falsche Zeugen auftreten, die aussagten:
»Dieser Mensch hört nicht auf, Reden gegen die heilige Stätte hier und
das Gesetz zu führen. \bibleverse{14} So haben wir ihn sagen hören:
›Dieser Jesus von Nazareth wird diese Stätte zerstören und die
Gebräuche\textless sup title=``oder: Satzungen''\textgreater✲ abändern,
die Mose uns verordnet hat.‹« \bibleverse{15} Als nun alle, die im Hohen
Rate saßen, ihre Blicke gespannt auf ihn richteten, sahen sie sein
Antlitz (verklärt) wie das Angesicht eines Engels.

\hypertarget{b-verteidigungsrede-des-stephanus-eine-anklagerede-wider-den-hohen-rat-und-die-juden-dauxdf-sie-gleich-ihren-vuxe4tern-der-leitung-gottes-und-auch-dem-gesetz-widerstrebt-haben}{%
\paragraph{b) Verteidigungsrede des Stephanus -- eine Anklagerede wider
den Hohen Rat und die Juden, daß sie gleich ihren Vätern der Leitung
Gottes und auch dem Gesetz widerstrebt
haben}\label{b-verteidigungsrede-des-stephanus-eine-anklagerede-wider-den-hohen-rat-und-die-juden-dauxdf-sie-gleich-ihren-vuxe4tern-der-leitung-gottes-und-auch-dem-gesetz-widerstrebt-haben}}

\hypertarget{aa-die-zeit-der-erzvuxe4ter}{%
\subparagraph{aa) Die Zeit der
Erzväter}\label{aa-die-zeit-der-erzvuxe4ter}}

\hypertarget{section-6}{%
\section{7}\label{section-6}}

\bibleverse{1} Der Hohepriester fragte ihn nun: »Verhält sich dies so?«
\bibleverse{2} Da antwortete Stephanus: »Werte Brüder und Väter, hört
mich an! Der Gott der Herrlichkeit erschien unserm Vater Abraham, als er
noch in Mesopotamien wohnte, bevor er sich in Haran niedergelassen
hatte, \bibleverse{3} und gebot ihm: ›Verlaß dein Heimatland und deine
Verwandtschaft und ziehe in das Land, das ich dir zeigen
werde!‹\textless sup title=``1.Mose 12,1''\textgreater✲ \bibleverse{4}
Da wanderte er aus dem Lande der Chaldäer aus und ließ sich in Haran
nieder. Von dort ließ Gott ihn dann nach dem Tode seines Vaters in
dieses Land hier übersiedeln, das ihr noch jetzt bewohnt; \bibleverse{5}
doch gab er ihm keinen festen Besitz darin, auch nicht einen Fuß breit,
verhieß ihm jedoch, er wolle es ihm und seiner Nachkommenschaft
späterhin zum Eigentum geben, obgleich er damals noch kein Kind hatte.
\bibleverse{6} So lauteten aber Gottes Worte: ›Seine Nachkommen werden
als Beisassen in einem fremden Lande ansässig sein, wo man sie
vierhundert Jahre lang knechten und mißhandeln wird; \bibleverse{7} doch
das Volk, dem sie als Knechte dienen werden, will ich richten‹, sagte
Gott; ›und hierauf werden sie ausziehen und mir an dieser Stätte
dienen.‹\textless sup title=``1.Mose 15,13-14; 2.Mose
3,12''\textgreater✲ \bibleverse{8} Dann gab Gott ihm den Bund der
Beschneidung, und so wurde Abraham der Vater Isaaks, den er am achten
Tage beschnitt; Isaak wurde dann der Vater Jakobs und Jakob der Vater
der zwölf Erzväter. \bibleverse{9} Weil dann aber die Erzväter auf
Joseph neidisch wurden, verkauften sie ihn nach Ägypten; doch Gott war
mit ihm \bibleverse{10} und rettete ihn aus allen seinen Bedrängnissen
und verlieh ihm Gnade und Weisheit vor dem Pharao, dem ägyptischen
König, der ihn zum Gebieter über Ägypten und über sein ganzes Haus
einsetzte. \bibleverse{11} Da kam eine Hungersnot und große Drangsal
über das ganze Land Ägypten und Kanaan, und unsere Väter hatten nichts
zu essen. \bibleverse{12} Als aber Jakob erfuhr, daß in Ägypten Getreide
zu haben sei, sandte er unsere Väter zum erstenmal hin. \bibleverse{13}
Beim zweitenmal gab sich dann Joseph seinen Brüdern zu erkennen, und
Josephs Herkunft wurde dem Pharao bekannt. \bibleverse{14} Joseph sandte
dann hin und ließ seinen Vater Jakob und seine gesamte Verwandtschaft
holen, im ganzen fünfundsiebenzig Seelen. \bibleverse{15} So zog denn
Jakob nach Ägypten hinab, wo er starb und ebenso auch unsere Väter;
\bibleverse{16} sie wurden nach Sichem überführt und in dem Grabe
beigesetzt, das Abraham für eine Geldsumme von den Söhnen Hemors in
Sichem gekauft hatte.«

\hypertarget{bb-die-mosaische-zeit}{%
\subparagraph{bb) Die mosaische Zeit}\label{bb-die-mosaische-zeit}}

\bibleverse{17} »Je näher aber die Zeit der Verheißung kam, die Gott dem
Abraham zugesagt hatte, desto mehr wuchs das Volk in Ägypten an und
wurde zahlreich, \bibleverse{18} bis ein anderer König zur Regierung
über Ägypten kam, der von Joseph nichts wußte. \bibleverse{19} Dieser
verfuhr arglistig gegen unser Volk und mißhandelte unsere Väter, so daß
sie ihre neugeborenen Kinder aussetzen mußten, damit sie nicht am Leben
bleiben sollten. \bibleverse{20} In dieser Zeit wurde Mose geboren und
war ein ausnehmend schönes Kind; drei Monate lang wurde er im
Elternhause aufgezogen, \bibleverse{21} und als man ihn dann ausgesetzt
hatte, nahm ihn die Tochter Pharaos zu sich und erzog ihn für sich
selbst zum Sohn. \bibleverse{22} So wurde denn Mose in aller Weisheit
der Ägypter unterrichtet und war gewaltig in seinen Worten und Taten.
\bibleverse{23} Als er aber volle vierzig Jahre alt geworden war, stieg
das Verlangen in ihm auf, sich einmal nach seinen Brüdern, den
Israeliten, umzusehen; \bibleverse{24} und als er einen von ihnen sah,
dem Unrecht geschah, leistete er ihm Beistand und verschaffte dem
Mißhandelten Genugtuung, indem er den Ägypter erschlug. \bibleverse{25}
Dabei war er der Meinung, seine Volksgenossen würden zu der Einsicht
kommen, daß Gott ihnen durch seine Hand Rettung schaffen würde; doch sie
erkannten es nicht. \bibleverse{26} Am folgenden Tage kam er gerade
dazu, als zwei von ihnen einen Streit miteinander hatten; da wollte er
sie versöhnen, damit sie Frieden hielten, indem er sagte: ›Ihr Männer,
ihr seid doch Brüder: warum tut ihr einander unrecht?‹ \bibleverse{27}
Der aber, welcher seinem Genossen unrecht tat, stieß ihn zurück mit den
Worten: ›Wer hat dich zum Oberhaupt und Richter über uns eingesetzt?
\bibleverse{28} Willst du mich etwa ebenso erschlagen, wie du gestern
den Ägypter erschlagen hast?‹\textless sup title=``2.Mose
2,14-15''\textgreater✲ \bibleverse{29} Um dieses Wortes willen ergriff
Mose die Flucht und lebte als Fremdling im Lande Midian, wo ihm zwei
Söhne geboren wurden.

\bibleverse{30} Als dann wieder volle vierzig Jahre vergangen waren,
erschien ihm in der Wüste des Berges Sinai ein Engel in der Feuerflamme
eines Dornbusches. \bibleverse{31} Als Mose das sah, verwunderte er sich
über die Erscheinung; als er aber näher hinzutrat, um genauer zuzusehen,
erscholl die Stimme des Herrn: \bibleverse{32} ›Ich bin der Gott deiner
Väter, der Gott Abrahams, Isaaks und Jakobs.‹\textless sup
title=``2.Mose 3,6''\textgreater✲ Da begann Mose zu zittern und wagte
nicht, genauer hinzusehen. \bibleverse{33} Der Herr aber sagte zu ihm:
›Ziehe dir die Schuhe ab von den Füßen; denn die Stätte, auf der du
stehst, ist heiliges Land. \bibleverse{34} Ich habe die Mißhandlung
meines Volkes in Ägypten gesehen und ihr Seufzen gehört; darum bin ich
herabgekommen, sie zu erretten. Und jetzt komm: ich will dich nach
Ägypten senden!‹\textless sup title=``2.Mose 2,24; 3,7.10''\textgreater✲
\bibleverse{35} Diesen Mose, den sie verleugnet\textless sup
title=``oder: zurückgewiesen''\textgreater✲ hatten, als sie sagten: ›Wer
hat dich zum Oberhaupt und Richter (über uns) gesetzt?‹, eben diesen hat
Gott als Oberhaupt und Erlöser gesandt durch die Vermittlung des Engels,
der ihm im Dornbusch erschienen war. \bibleverse{36} Dieser ist es auch,
der sie (aus dem Lande) weggeführt hat, indem er Wunder und Zeichen im
Lande Ägypten und am Roten Meer sowie vierzig Jahre lang in der Wüste
tat. \bibleverse{37} Dieser Mose ist es, der zu den Israeliten gesagt
hat: ›Einen Propheten wie mich wird Gott euch aus euren Volksgenossen
erwecken.‹\textless sup title=``5.Mose 18,15''\textgreater✲
\bibleverse{38} Dieser ist es, der bei der Gemeindeversammlung in der
Wüste Vermittler gewesen ist zwischen dem Engel, der auf dem Berge Sinai
zu ihm redete, und zwischen unsern Vätern, derselbe, der lebendige Worte
empfing, um sie uns mitzuteilen. \bibleverse{39} Doch unsere Väter
wollten ihm nicht gehorsam sein; vielmehr stießen sie ihn von sich und
sehnten sich nach Ägypten zurück \bibleverse{40} und sagten zu Aaron:
›Mache uns Götter, die vor uns herziehen sollen! Denn von diesem Mose,
der uns aus dem Lande Ägypten herausgeführt hat, wissen wir nicht, was
aus ihm geworden ist.‹\textless sup title=``2.Mose 32,1''\textgreater✲
\bibleverse{41} So machten sie sich denn damals ein Stierbild, brachten
diesem Götzen Opfer dar und hatten ihre Freude an den
Werken\textless sup title=``oder: an dem Machwerk''\textgreater✲ ihrer
Hände. \bibleverse{42} Da wandte Gott sich von ihnen ab und gab sie
dahin, daß sie dem Sternenheer des Himmels dienten\textless sup
title=``=~Anbetung erwiesen''\textgreater✲, wie im Buch der Propheten
geschrieben steht\textless sup title=``Am 5,25-27''\textgreater✲: ›Habt
ihr etwa mir Schlachttiere und Opfergaben während der vierzig Jahre in
der Wüste dargebracht, ihr vom Hause Israel? \bibleverse{43} Nein, das
Zelt des Moloch und das Sternbild des Gottes Rephan\textless sup
title=``oder: Romphan''\textgreater✲ habt ihr getragen, die
Götzenbilder, die ihr zur Anbetung angefertigt hattet; darum werde ich
euch über Babylon hinaus (in die Verbannung) wegführen lassen.‹«

\hypertarget{cc-die-zeit-der-stiftshuxfctte-und-des-tempelbaus}{%
\subparagraph{cc) Die Zeit der Stiftshütte und des
Tempelbaus}\label{cc-die-zeit-der-stiftshuxfctte-und-des-tempelbaus}}

\bibleverse{44} »Die Hütte des Zeugnisses\textless sup title=``=~die
Stiftshütte''\textgreater✲ hatten unsere Väter in der Wüste so, wie Gott
es angeordnet hatte, als er dem Mose gebot, sie nach dem Vorbild, das er
gesehen hatte, herzustellen. \bibleverse{45} Diese (Hütte) haben unsere
Väter dann auch übernommen und sie unter Josua in das Gebiet der
Heidenvölker gebracht, die Gott vor den Augen unserer Väter vertrieb;
(und so blieb es) bis zur Zeit Davids. \bibleverse{46} Dieser fand Gnade
vor Gott und erbat es sich als Gunst, eine Wohnung für den Gott Jakobs
zu finden\textless sup title=``=~errichten zu dürfen; Ps
132,5''\textgreater✲. \bibleverse{47} Aber erst Salomo hat ihm ein Haus
erbaut. \bibleverse{48} Doch der Höchste wohnt nicht in einem Bau, der
von Menschenhand hergestellt ist, wie der Prophet sagt\textless sup
title=``Jes 66,1-2''\textgreater✲: \bibleverse{49} ›Der Himmel ist mein
Thron und die Erde der Schemel meiner Füße. Was für ein Haus wäre es,
das ihr mir bauen könntet?‹ -- sagt der Herr -- ›oder welches wäre die
Stätte der Ruhe für mich? \bibleverse{50} Hat nicht meine Hand dies
ganze Weltall geschaffen?‹«

\hypertarget{dd-schluuxdf-der-rede-anklage-des-volkes}{%
\subparagraph{dd) Schluß der Rede; Anklage des
Volkes}\label{dd-schluuxdf-der-rede-anklage-des-volkes}}

\bibleverse{51} »O ihr Halsstarrigen und an Herz und Ohren
Unbeschnittenen! Immerfort widerstrebt ihr dem heiligen Geist, wie eure
Väter, so auch ihr. \bibleverse{52} Wo ist ein Prophet gewesen, den eure
Väter nicht verfolgt haben? Und so haben sie auch die getötet, welche
das Kommen des Gerechten vorausverkündigt haben, an dem ihr jetzt zu
Verrätern und Mördern geworden seid. \bibleverse{53} Auf
Anordnung\textless sup title=``oder: durch Vermittlung''\textgreater✲
von Engeln habt ihr das Gesetz empfangen und es doch nicht gehalten!«

\hypertarget{c-der-muxe4rtyrertod-des-stephanus}{%
\paragraph{c) Der Märtyrertod des
Stephanus}\label{c-der-muxe4rtyrertod-des-stephanus}}

\bibleverse{54} Als sie das hörten, ging es ihnen wie ein Stich durchs
Herz, und sie knirschten mit den Zähnen gegen ihn. \bibleverse{55} Er
aber, voll heiligen Geistes, blickte fest\textless sup title=``oder:
unverwandt''\textgreater✲ zum Himmel empor, sah die Herrlichkeit Gottes
und Jesus zur Rechten Gottes stehen \bibleverse{56} und rief aus: »Ich
sehe die Himmel aufgetan und den Menschensohn zur Rechten Gottes
stehen!« \bibleverse{57} Da erhoben sie ein lautes Geschrei, hielten
sich die Ohren zu und stürmten einmütig auf ihn los; \bibleverse{58}
dann stießen sie ihn zur Stadt hinaus und steinigten ihn. Dabei legten
die Zeugen ihre Obergewänder\textless sup title=``oder:
Mäntel''\textgreater✲ ab zu den Füßen eines jungen Mannes mit Namen
Saulus \bibleverse{59} und steinigten den Stephanus, der betend ausrief:
»Herr Jesus, nimm meinen Geist auf!« \bibleverse{60} Alsdann auf die
Knie niedergesunken, rief er noch laut aus: »Herr, rechne ihnen diese
Sünde nicht zu!« Nach diesen Worten gab er seinen Geist auf.

\hypertarget{section-7}{%
\section{8}\label{section-7}}

\bibleverse{1} Saulus aber war mit seiner Hinrichtung durchaus
einverstanden.

\hypertarget{b.-die-ausbreitung-der-heilsbotschaft-in-samarien-und-syrien-beginn-der-heidenbekehrung-kap.-8-12}{%
\subsection{B. Die Ausbreitung der Heilsbotschaft in Samarien und
Syrien; Beginn der Heidenbekehrung (Kap.
8-12)}\label{b.-die-ausbreitung-der-heilsbotschaft-in-samarien-und-syrien-beginn-der-heidenbekehrung-kap.-8-12}}

\hypertarget{die-erste-verfolgung-der-christengemeinde-in-jerusalem-besonders-durch-saulus-und-ihre-zerstreuung}{%
\subsubsection{1. Die erste Verfolgung der Christengemeinde in Jerusalem
(besonders durch Saulus) und ihre
Zerstreuung}\label{die-erste-verfolgung-der-christengemeinde-in-jerusalem-besonders-durch-saulus-und-ihre-zerstreuung}}

b An jenem Tage kam es aber zu einer schweren Verfolgung der Gemeinde in
Jerusalem. Da zerstreuten sich alle mit Ausnahme der Apostel in die
Landbezirke von Judäa und Samaria. \bibleverse{2} Den Stephanus aber
bestatteten gottesfürchtige Männer und veranstalteten eine feierliche
Trauerfeier um ihn. \bibleverse{3} Saulus aber wütete gegen die
Gemeinde, indem er überall in die Häuser eindrang, Männer und Frauen
fortschleppte und sie ins Gefängnis werfen ließ.

\hypertarget{wirksamkeit-des-philippus-und-petrus-in-samarien}{%
\subsubsection{2. Wirksamkeit des Philippus und Petrus in
Samarien}\label{wirksamkeit-des-philippus-und-petrus-in-samarien}}

\hypertarget{a-philippus-predigt-und-heilt}{%
\paragraph{a) Philippus predigt und
heilt}\label{a-philippus-predigt-und-heilt}}

\bibleverse{4} Die Versprengten nun zogen im Lande umher und
verkündigten die Heilsbotschaft. \bibleverse{5} Dabei kam Philippus in
die Hauptstadt von Samarien hinab und predigte ihren Bewohnern den
Gottgesalbten✲. \bibleverse{6} Die Volksmenge zeigte sich allgemein für
die Predigt des Philippus empfänglich, indem sie ihm zuhörten und die
Zeichen sahen, die er tat; \bibleverse{7} denn aus vielen fuhren die
unreinen Geister, von denen sie besessen waren, mit lautem Geschrei aus,
und zahlreiche Gelähmte und Verkrüppelte wurden geheilt. \bibleverse{8}
Darüber herrschte in jener Stadt große Freude.

\hypertarget{b-der-zauberer-simon-in-samaria}{%
\paragraph{b) Der Zauberer Simon in
Samaria}\label{b-der-zauberer-simon-in-samaria}}

\bibleverse{9} Nun hatte schon vorher ein Mann namens Simon in der Stadt
gelebt, der sich mit Zauberei abgab und die Bevölkerung von Samaria
dadurch in Staunen versetzte; denn er behauptete von sich, er sei etwas
Großes. \bibleverse{10} Alle waren für ihn eingenommen, klein und groß,
und erklärten: »Dieser Mann ist die Kraft Gottes, welche die große
heißt.« \bibleverse{11} Sie waren aber deshalb für ihn so eingenommen,
weil er sie lange Zeit durch seine Zauberkünste in Erstaunen gesetzt
hatte. \bibleverse{12} Als sie jetzt aber dem Philippus Glauben
schenkten, der ihnen die Heilsbotschaft vom Reiche Gottes und vom Namen
Jesu Christi verkündigte, ließen sie sich taufen, Männer wie Frauen.
\bibleverse{13} So wurde denn Simon ebenfalls gläubig; er schloß sich
nach seiner Taufe eng an Philippus an und kam nicht aus dem Staunen
heraus, als er die Zeichen und großen Wunder sah, die da geschahen.

\hypertarget{c-wirken-des-petrus-und-johannes-in-samaria-simons-zurechtweisung-durch-petrus}{%
\paragraph{c) Wirken des Petrus und Johannes in Samaria; Simons
Zurechtweisung durch
Petrus}\label{c-wirken-des-petrus-und-johannes-in-samaria-simons-zurechtweisung-durch-petrus}}

\bibleverse{14} Als nun die Apostel in Jerusalem vernahmen, daß Samaria
das Wort Gottes angenommen habe, entsandten sie Petrus und Johannes zu
ihnen. \bibleverse{15} Diese beteten nach ihrer Ankunft für sie, daß sie
den heiligen Geist empfangen möchten; \bibleverse{16} denn dieser war
noch auf keinen von ihnen gefallen, sondern sie waren lediglich auf den
Namen des Herrn Jesus getauft worden. \bibleverse{17} Infolgedessen
legten sie (die beiden Apostel) ihnen die Hände auf, und sie empfingen
den heiligen Geist.

\bibleverse{18} Als nun Simon sah, daß durch die Handauflegung der
Apostel der heilige Geist verliehen wurde, bot er ihnen Geld an
\bibleverse{19} und bat: »Verleiht doch auch mir diese Kraft, daß jeder,
dem ich die Hände auflege, den heiligen Geist empfängt.« \bibleverse{20}
Petrus aber gab ihm zur Antwort: »Dein Geld fahre samt dir ins
Verderben, weil du gemeint hast, die Gabe Gottes durch Geld erkaufen zu
können! \bibleverse{21} Du hast keinen Anteil und kein Anrecht an dieser
Sache; denn dein Herz ist nicht aufrichtig vor Gott. \bibleverse{22}
Darum bekehre dich von dieser deiner Bosheit und bete zum Herrn, ob dir
vielleicht das Trachten deines Herzens vergeben werden mag;
\bibleverse{23} denn ich sehe, daß du in ›Galle der Bitterkeit‹ und in
›Bande der Ungerechtigkeit‹ geraten bist.«\textless sup title=``5.Mose
29,17; Jes 58,6''\textgreater✲ \bibleverse{24} Da antwortete Simon:
»Betet ihr für mich zum Herrn, daß nichts von dem, was ihr
ausgesprochen\textless sup title=``oder: angedroht''\textgreater✲ habt,
mich treffen möge!«

\bibleverse{25} Als sie nun (die beiden Apostel) das Wort des Herrn
bezeugt und gepredigt hatten, machten sie sich auf die Rückreise nach
Jerusalem und verkündigten (dabei noch) in vielen samaritischen
Ortschaften die Heilsbotschaft.

\hypertarget{bekehrung-und-taufe-des-uxe4thiopischen-hofbeamten-durch-philippus}{%
\subsubsection{3. Bekehrung und Taufe des äthiopischen Hofbeamten durch
Philippus}\label{bekehrung-und-taufe-des-uxe4thiopischen-hofbeamten-durch-philippus}}

\bibleverse{26} Ein Engel des Herrn aber gebot dem Philippus: »Mach dich
auf und begib dich um die Mittagszeit auf die Straße, die von Jerusalem
nach Gaza hinabführt und einsam ist!« \bibleverse{27} Da machte er sich
auf und ging hin. Und siehe, ein Äthiopier, ein Hofbeamter und
Würdenträger der äthiopischen Königin Kandace, der ihren gesamten Schatz
zu verwalten hatte, war nach Jerusalem gekommen, um dort anzubeten.
\bibleverse{28} Jetzt befand er sich wieder auf der Heimreise und saß
auf seinem Wagen, indem er den Propheten Jesaja las. \bibleverse{29} Da
gebot der Geist dem Philippus: »Tritt hinzu und halte dich nahe an
diesen Wagen!« \bibleverse{30} So lief denn Philippus hinzu, und als er
hörte, wie jener den Propheten Jesaja las, fragte er ihn: »Verstehst du
auch, was du liest?« \bibleverse{31} Er antwortete: »Wie sollte ich das
können, wenn mir niemand Anleitung gibt?« Dann bat er Philippus,
aufzusteigen und sich zu ihm zu setzen.

\bibleverse{32} Der Wortlaut der Schriftstelle nun, die er gerade las,
war dieser\textless sup title=``Jes 53,7-8''\textgreater✲: »Wie ein
Schaf wurde er zur Schlachtbank geführt, und wie ein Lamm vor seinem
Scherer stumm bleibt, so tat er seinen Mund nicht auf. \bibleverse{33}
In seiner Erniedrigung wurde das Strafgericht über ihn aufgehoben, und
wer wird seine Nachkommenschaft berechnen? Denn erhoben wird sein Leben
von der Erde hinweg.« \bibleverse{34} Da wandte sich der Hofbeamte an
Philippus mit der Frage: »Ich bitte dich: von wem redet hier der
Prophet? Von sich selbst oder von einem andern?« \bibleverse{35} Da tat
Philippus seinen Mund auf und verkündigte ihm, indem er an dieses
Schriftwort anknüpfte, die Heilsbotschaft von Jesus.

\bibleverse{36} Als sie nun so auf der Straße dahinfuhren, kamen sie an
ein Gewässer; da sagte der Hofbeamte: »Hier ist ja Wasser! Was steht
meiner Taufe noch im Wege?« \bibleverse{37} {[}Philippus antwortete ihm:
»Wenn du von ganzem Herzen glaubst, so darf es wohl geschehen.« Jener
antwortete: »Ich glaube, daß Jesus Christus der Sohn Gottes ist.«{]}
\bibleverse{38} Er ließ also den Wagen halten, und beide stiegen in das
Wasser hinab, Philippus sowohl wie der Hofbeamte, und er taufte ihn.
\bibleverse{39} Als sie dann wieder aus dem Wasser heraufgestiegen
waren, entrückte der Geist des Herrn den Philippus, und der Hofbeamte
sah ihn nicht mehr; denn freudig zog er auf seiner Straße weiter.
\bibleverse{40} Philippus aber befand sich in Asdod; er zog dort von Ort
zu Ort und verkündigte die Heilsbotschaft in allen Städten, bis er nach
Cäsarea kam.

\hypertarget{die-bekehrung-und-berufung-des-saulus-zum-heidenapostel}{%
\subsubsection{4. Die Bekehrung und Berufung des Saulus zum
Heidenapostel}\label{die-bekehrung-und-berufung-des-saulus-zum-heidenapostel}}

\hypertarget{a-das-erlebnis-des-saulus-auf-dem-weg-nach-damaskus}{%
\paragraph{a) Das Erlebnis des Saulus auf dem Weg nach
Damaskus}\label{a-das-erlebnis-des-saulus-auf-dem-weg-nach-damaskus}}

\hypertarget{section-8}{%
\section{9}\label{section-8}}

\bibleverse{1} Saulus aber, der noch immer Drohungen und Mord gegen die
Jünger des Herrn schnaubte, wandte sich an den Hohenpriester
\bibleverse{2} und erbat sich von ihm Briefe✲ nach Damaskus an die
dortigen Synagogen\textless sup title=``=~jüdischen
Gemeinden''\textgreater✲, um Anhänger der neuen Lehre\textless sup
title=``oder: Glaubensrichtung''\textgreater✲, die er etwa fände, Männer
wie Frauen, in Fesseln nach Jerusalem zu bringen. \bibleverse{3} Während
er nun so dahinzog und schon in die Nähe von Damaskus gekommen war,
umstrahlte ihn plötzlich ein Lichtschein vom Himmel her; \bibleverse{4}
er stürzte zu Boden und vernahm eine Stimme, die ihm zurief: »Saul,
Saul! Was verfolgst du mich?« \bibleverse{5} Er fragte: »Wer bist du,
Herr?« Jener antwortete: »Ich bin Jesus, den du verfolgst!
\bibleverse{6} Doch stehe auf und geh in die Stadt hinein: dort wird dir
gesagt werden, was du tun sollst!« \bibleverse{7} Die Männer nun, die
ihn auf der Reise begleiteten, standen sprachlos da; denn sie hörten
wohl die Stimme, sahen aber niemand. \bibleverse{8} Saulus erhob sich
dann von der Erde; obwohl jedoch seine Augen geöffnet waren, konnte er
nichts sehen: an der Hand mußte man ihn nach Damaskus hinführen,
\bibleverse{9} und er war drei Tage lang ohne Sehvermögen; auch aß und
trank er nichts.

\hypertarget{b-heilung-und-taufe-des-saulus-durch-ananias}{%
\paragraph{b) Heilung und Taufe des Saulus durch
Ananias}\label{b-heilung-und-taufe-des-saulus-durch-ananias}}

\bibleverse{10} Nun wohnte in Damaskus ein Jünger namens Ananias; zu dem
sprach der Herr in einem Gesicht: »Ananias!« Jener antwortete: »Hier bin
ich, Herr!« \bibleverse{11} Der Herr fuhr fort: »Stehe auf und begib
dich in die sogenannte Gerade Straße; erkundige dich dort im Hause des
Judas nach einem Manne namens Saulus aus Tarsus; denn siehe, er betet
\bibleverse{12} und hat in einem Gesicht gesehen, wie ein Mann namens
Ananias bei ihm eintrat und ihm die Hände auflegte, damit er sein
Augenlicht wiederbekomme.« \bibleverse{13} Ananias aber antwortete:
»Herr, ich habe von vielen Seiten über diesen Mann gehört, wieviel Böses
er deinen Heiligen in Jerusalem zugefügt hat; \bibleverse{14} und auch
hier hat er von den Hohenpriestern Vollmacht, alle in Fesseln zu legen,
die deinen Namen anrufen.« \bibleverse{15} Aber der Herr gab ihm zur
Antwort: »Gehe hin! Denn dieser Mann ist für mich ein auserwähltes
Werkzeug: er soll meinen Namen vor Heidenvölker und Könige und vor die
Kinder Israel tragen; \bibleverse{16} denn ich werde ihm zeigen, wieviel
er um meines Namens willen leiden muß.« \bibleverse{17} Da machte sich
Ananias auf den Weg, ging in das Haus und legte ihm die Hände auf mit
den Worten: »Bruder Saul, der Herr hat mich gesandt, Jesus, der dir auf
dem Wege hierher erschienen ist: du sollst wieder sehen können und mit
heiligem Geist erfüllt werden.« \bibleverse{18} Da fiel es ihm sogleich
von den Augen ab wie Schuppen: er konnte wieder sehen, stand auf und
ließ sich taufen; \bibleverse{19} a dann nahm er auch Nahrung zu sich
und kam wieder zu Kräften.

\hypertarget{erstes-apostolisches-wirken-und-leiden-des-saulus-in-damaskus-und-jerusalem}{%
\subsubsection{5. Erstes apostolisches Wirken und Leiden des Saulus in
Damaskus und
Jerusalem}\label{erstes-apostolisches-wirken-und-leiden-des-saulus-in-damaskus-und-jerusalem}}

\hypertarget{a-die-wirksamkeit-des-paulus-in-damaskus-und-seine-flucht}{%
\paragraph{a) Die Wirksamkeit des Paulus in Damaskus und seine
Flucht}\label{a-die-wirksamkeit-des-paulus-in-damaskus-und-seine-flucht}}

b Einige Tage war er nun mit den Jüngern✲ in Damaskus zusammen;
\bibleverse{20} dann predigte er sogleich in den (dortigen) Synagogen
von Jesus, daß dieser der Sohn Gottes sei. \bibleverse{21} Da gerieten
alle, die ihn hörten, in Erstaunen und sagten: »Ist das nicht derselbe
Mann, der in Jerusalem die Bekenner dieses Namens wütend verfolgt hat
und auch hierher in der Absicht gekommen war, sie in Fesseln vor die
Hohenpriester zu führen?« \bibleverse{22} Saulus aber trat nur um so
entschlossener auf und brachte die Juden, die in Damaskus wohnten,
völlig außer Fassung, indem er bewies: »Dieser ist Christus\textless sup
title=``=~der Messias''\textgreater✲!«

\bibleverse{23} Als so eine geraume Zeit vergangen war, beschlossen die
Juden gemeinsam, ihn zu ermorden; \bibleverse{24} doch ihr Anschlag
wurde dem Saulus bekannt. Da sie nun sogar die Stadttore bei Tag und bei
Nacht bewachten, um ihn zu ermorden, \bibleverse{25} nahmen ihn die
Jünger✲ und ließen ihn bei Nacht in einem Korbe über die Mauer hinab.

\hypertarget{b-paulus-zum-erstenmal-als-christ-in-jerusalem}{%
\paragraph{b) Paulus zum erstenmal als Christ in
Jerusalem}\label{b-paulus-zum-erstenmal-als-christ-in-jerusalem}}

\bibleverse{26} Als er dann nach Jerusalem gekommen war, machte er dort
Versuche, sich an die Jünger anzuschließen; aber alle fürchteten sich
vor ihm, weil sie nicht glauben konnten, daß er ein Jünger sei.
\bibleverse{27} Da nahm sich Barnabas seiner an, führte ihn zu den
Aposteln und teilte ihnen mit, wie er unterwegs den Herrn gesehen und
daß dieser zu ihm geredet habe, und wie er dann in Damaskus freimütig im
Namen Jesu öffentlich gepredigt habe. \bibleverse{28} So ging er denn in
Jerusalem mit\textless sup title=``oder: bei''\textgreater✲ ihnen ein
und aus und predigte freimütig im Namen des Herrn; \bibleverse{29} auch
unterredete er sich und hielt Streitgespräche mit den griechisch
redenden Juden\textless sup title=``vgl. 6,1''\textgreater✲, die dann
aber einen Anschlag auf sein Leben machten. \bibleverse{30} Als die
Brüder das erfuhren, schafften sie ihn nach Cäsarea hinab und ließen ihn
von da weiter nach Tarsus reisen.

\hypertarget{wundertaten-des-petrus-in-lydda-und-joppe}{%
\subsubsection{6. Wundertaten des Petrus in Lydda und
Joppe}\label{wundertaten-des-petrus-in-lydda-und-joppe}}

\bibleverse{31} So hatte nun die Gemeinde in ganz Judäa, Galiläa und
Samaria Frieden; sie baute sich innerlich auf, wandelte in der Furcht
des Herrn und wuchs auch äußerlich durch den Beistand des heiligen
Geistes.

\hypertarget{a-heilung-des-geluxe4hmten-uxe4neas-in-lydda}{%
\paragraph{a) Heilung des gelähmten Äneas in
Lydda}\label{a-heilung-des-geluxe4hmten-uxe4neas-in-lydda}}

\bibleverse{32} Da geschah es, als Petrus allenthalben umherzog, daß er
auch zu den Heiligen\textless sup title=``=~getauften
Gläubigen''\textgreater✲ kam, die in Lydda wohnten. \bibleverse{33} Er
fand dort einen Mann namens Äneas, der schon seit acht Jahren zu Bett
lag, weil er gelähmt war. \bibleverse{34} Da sagte Petrus zu ihm:
Ȁneas, Jesus Christus macht dich gesund; stehe auf und mache dir dein
Bett selbst!« Da stand er sogleich auf, \bibleverse{35} und alle
Einwohner von Lydda und (der Landschaft) Saron sahen ihn und bekehrten
sich zum Herrn.

\hypertarget{b-auferweckung-der-tabitha-in-joppe}{%
\paragraph{b) Auferweckung der Tabitha in
Joppe}\label{b-auferweckung-der-tabitha-in-joppe}}

\bibleverse{36} In Joppe aber lebte eine Jüngerin namens Tabitha, das
heißt auf deutsch ›Gazelle‹; die tat außerordentlich viel Gutes und
spendete reichlich Almosen. \bibleverse{37} Gerade in jenen Tagen aber
begab es sich, daß sie krank wurde und starb. Man wusch sie also und
bahrte sie in einem Obergemach auf. \bibleverse{38} Weil nun Lydda nahe
bei Joppe liegt, sandten die Jünger auf die Nachricht, daß Petrus dort
sei, zwei Männer an ihn ab und ließen ihn bitten: »Komm doch
unverzüglich zu uns herüber!« \bibleverse{39} Da machte sich Petrus auf
den Weg und ging mit ihnen. Nach seiner Ankunft führte man ihn in das
Obergemach hinauf; da traten alle Witwen weinend zu ihm und zeigten ihm
die Röcke und Oberkleider, welche die Gazelle angefertigt hatte, als sie
noch bei ihnen war. \bibleverse{40} Petrus ließ nun alle aus dem Zimmer
hinausgehen, kniete nieder und betete; dann wandte er sich der Toten zu
und sagte: »Tabitha, stehe auf!« Da schlug sie die Augen auf, und als
sie Petrus erblickte, setzte sie sich aufrecht hin. \bibleverse{41} Er
reichte ihr nun die Hand und half ihr auf; dann rief er die Heiligen und
die Witwen herbei und stellte sie lebend vor sie hin. \bibleverse{42}
Das wurde in ganz Joppe bekannt, und viele kamen zum Glauben an den
Herrn. \bibleverse{43} Petrus blieb dann noch geraume Zeit in Joppe bei
einem gewissen Simon, einem Gerber.

\hypertarget{anfang-der-heidenbekehrung-petrus-und-der-heidnische-hauptmann-kornelius}{%
\subsubsection{7. Anfang der Heidenbekehrung; Petrus und der heidnische
Hauptmann
Kornelius}\label{anfang-der-heidenbekehrung-petrus-und-der-heidnische-hauptmann-kornelius}}

\hypertarget{a-bekehrung-und-taufe-des-kornelius-in-cuxe4sarea}{%
\paragraph{a) Bekehrung und Taufe des Kornelius in
Cäsarea}\label{a-bekehrung-und-taufe-des-kornelius-in-cuxe4sarea}}

\hypertarget{aa-das-gesicht-des-kornelius-in-cuxe4sarea}{%
\subparagraph{aa) Das Gesicht des Kornelius in
Cäsarea}\label{aa-das-gesicht-des-kornelius-in-cuxe4sarea}}

\hypertarget{section-9}{%
\section{10}\label{section-9}}

\bibleverse{1} In Cäsarea aber lebte (damals) ein Mann namens Kornelius,
ein Hauptmann bei der sogenannten Italischen Abteilung\textless sup
title=``eig. Kohorte''\textgreater✲; \bibleverse{2} er war fromm und
gottesfürchtig mit seinem ganzen Hause, tat dem (jüdischen) Volke viel
Gutes durch seine Mildtätigkeit und betete ohne Unterlaß zu Gott.
\bibleverse{3} Dieser Mann sah (eines Tages) in einem Gesicht um die
neunte Tagesstunde deutlich einen Engel Gottes bei sich eintreten, der
ihn anredete: »Kornelius!« \bibleverse{4} Dieser blickte ihn starr an
und fragte erschrocken: »Was soll ich, Herr?« Jener antwortete ihm:
»Deine Gebete und deine Almosen\textless sup title=``oder:
Liebeswerke''\textgreater✲ sind zu Gott emporgestiegen, und er gedenkt
ihrer wohl. \bibleverse{5} Und nun sende Boten nach Joppe und laß einen
gewissen Simon mit dem Beinamen Petrus zu dir kommen; \bibleverse{6} der
ist als Gast bei einem Gerber Simon, dessen Haus am Meere liegt.«
\bibleverse{7} Als nun der Engel, der mit ihm gesprochen hatte,
verschwunden war, rief Kornelius zwei von seinen Dienern und einen
frommen Soldaten aus der Zahl der Mannschaften, die ihn persönlich zu
bedienen hatten, \bibleverse{8} teilte ihnen alles mit und sandte sie
nach Joppe.

\hypertarget{bb-gesicht-des-petrus-in-joppe-eintreffen-der-boten-des-kornelius-bei-petrus}{%
\subparagraph{bb) Gesicht des Petrus in Joppe; Eintreffen der Boten des
Kornelius bei
Petrus}\label{bb-gesicht-des-petrus-in-joppe-eintreffen-der-boten-des-kornelius-bei-petrus}}

\bibleverse{9} Am folgenden Tage aber, als diese unterwegs waren und
sich schon der Stadt näherten, stieg Petrus um die Mittagszeit auf das
Dach des Hauses hinauf, um dort zu beten. \bibleverse{10} Da wurde er
hungrig und wünschte, etwas zu genießen. Während man es ihm nun
zubereitete, kam eine Verzückung über ihn: \bibleverse{11} er sah den
Himmel offen stehen und einen Behälter herabkommen wie ein großes Stück
Leinwand, das an den vier Zipfeln zur Erde herabgelassen wurde.
\bibleverse{12} Darin befanden sich alle Arten vierfüßiger und
kriechender Tiere der Erde und Vögel des Himmels. \bibleverse{13} Nun
rief eine Stimme ihm zu: »Stehe auf, Petrus, schlachte und iß!«
\bibleverse{14} Petrus aber antwortete: »Nicht doch, Herr! Denn noch nie
habe ich etwas Unheiliges und Unreines genossen.« \bibleverse{15} Da
rief zum zweitenmal eine Stimme ihm zu: »Was Gott gereinigt hat, das
erkläre du nicht für unrein!« \bibleverse{16} Dies wiederholte sich
dreimal; dann wurde der Behälter sogleich wieder in den Himmel
emporgezogen.

\bibleverse{17} Als nun Petrus sich nicht zu erklären wußte, was die
Erscheinung, die er gesehen hatte, zu bedeuten habe, siehe, da standen
die Männer, die von Kornelius abgesandt worden waren und das Haus Simons
ausfindig gemacht hatten, am Toreingang; \bibleverse{18} dort riefen sie
und erkundigten sich, ob Simon mit dem Beinamen Petrus hier als Gast
wohne. \bibleverse{19} Während Petrus noch immer über das Gesicht
nachdachte, sagte der Geist zu ihm: »Da sind drei Männer, die dich
suchen. \bibleverse{20} So stehe nun auf, gehe hinunter und mache dich
mit ihnen ohne Bedenken auf den Weg! Denn ich habe sie gesandt.«
\bibleverse{21} Petrus stieg also zu den Männern hinunter und sagte zu
ihnen: »Ich bin der, den ihr sucht! Aus welchem Grunde seid ihr
hergekommen?« \bibleverse{22} Jene antworteten: »Ein Hauptmann
Kornelius, ein ehrenhafter, gottesfürchtiger und von der ganzen
jüdischen Bevölkerung anerkannter Mann, hat von einem heiligen Engel die
(göttliche) Weisung erhalten, dich in sein Haus kommen zu lassen und zu
hören, was du ihm zu sagen hast.« \bibleverse{23} Da lud Petrus sie zu
sich herein und nahm sie gastlich auf.

\hypertarget{cc-petrus-im-hause-des-kornelius}{%
\subparagraph{cc) Petrus im Hause des
Kornelius}\label{cc-petrus-im-hause-des-kornelius}}

Am folgenden Tage aber machte er sich mit ihnen auf den Weg; auch einige
von den Brüdern aus Joppe begleiteten ihn. \bibleverse{24} Tags darauf
kam er in Cäsarea an, wo Kornelius sie schon erwartete und alle seine
Verwandten und vertrauten Freunde zu sich eingeladen hatte.
\bibleverse{25} Als Petrus nun im Begriff stand, in das Haus
einzutreten, kam Kornelius ihm entgegen, warf sich vor ihm nieder und
bezeigte ihm seine hohe Verehrung. \bibleverse{26} Petrus aber hob ihn
auf mit den Worten: »Stehe auf! Ich bin auch nur ein Mensch.«
\bibleverse{27} Dann trat er im Gespräch mit ihm ein und traf eine
zahlreiche Versammlung an, \bibleverse{28} zu der er sagte: »Ihr wißt,
wie streng es einem Juden verboten ist, Verkehr mit jemand zu haben, der
zu einem anderen Volke gehört, oder gar bei ihm einzukehren. Doch mir
hat Gott gezeigt, daß man keinen Menschen als unheilig oder unrein
bezeichnen darf. \bibleverse{29} Deshalb habe ich mich auch auf eure
Einladung hin ohne Weigerung hier eingefunden. Ich möchte nun aber
wissen, aus welchem Grunde ihr mich habt herkommen lassen.«
\bibleverse{30} Da antwortete Kornelius: »Vor vier Tagen, genau zu
dieser Zeit, betete ich um die neunte Stunde in meinem Hause; da stand
plötzlich ein Mann in einem glänzenden Gewande vor mir \bibleverse{31}
und sagte: ›Kornelius, dein Gebet hat Erhörung gefunden, und deiner
Almosen\textless sup title=``oder: Liebeswerke''\textgreater✲ ist vor
Gott gedacht worden. \bibleverse{32} So sende nun nach Joppe und laß
Simon, der den Beinamen Petrus führt, herrufen; der ist als Gast im
Hause eines Gerbers Simon am Meer.‹ \bibleverse{33} Da habe ich auf der
Stelle zu dir gesandt, und ich bin dir dankbar dafür, daß du gekommen
bist. Jetzt haben wir nun alle uns hier vor Gottes Angesicht
eingefunden, um alles zu vernehmen, was dir vom Herrn aufgetragen worden
ist.«

\bibleverse{34} Da tat Petrus den Mund auf und sagte: »Nun erkenne ich
in Wahrheit, daß Gott nicht die Person ansieht\textless sup title=``vgl.
1.Sam 16,7''\textgreater✲, \bibleverse{35} sondern daß in jedem Volk
der, welcher ihn fürchtet und Gerechtigkeit übt, ihm
angenehm\textless sup title=``oder: für ihn zur Annahme
geeignet''\textgreater✲ ist. \bibleverse{36} Ihr kennt das Wort, das er
an die Kinder Israel hat ergehen lassen, indem er ihnen (die
Heilsbotschaft vom) Frieden durch Jesus Christus verkündigen ließ:
dieser ist der Herr über alle. \bibleverse{37} Ebenso kennt ihr die
Ereignisse, die sich im ganzen jüdischen Lande zugetragen haben und von
Galiläa nach der Taufe, die Johannes gepredigt hatte, ausgegangen sind,
\bibleverse{38} nämlich wie Gott Jesus von Nazareth mit heiligem Geist
und mit Kraft gesalbt hat, wie dieser dann umhergezogen ist und Gutes
getan und alle geheilt hat, die vom Teufel überwältigt
waren\textless sup title=``=~die unter der Herrschaft des Teufels
standen''\textgreater✲, denn Gott war mit ihm; \bibleverse{39} und wir
sind Zeugen für alles das, was er im jüdischen Lande sowie in Jerusalem
vollbracht hat. Den haben sie dann zwar ans Kreuz gehängt und getötet,
\bibleverse{40} aber Gott hat ihn am dritten Tage auferweckt und ihn
sichtbar erscheinen lassen, \bibleverse{41} nicht dem ganzen Volk,
sondern uns, den von Gott zuvor erwählten Zeugen, die wir nach seiner
Auferstehung von den Toten mit ihm zusammen gegessen und getrunken
haben. \bibleverse{42} Und er hat uns geboten, dem Volke zu verkündigen
und zu bezeugen, daß dieser der von Gott bestimmte Richter über Lebende
und Tote ist. \bibleverse{43} Für diesen\textless sup title=``oder: von
diesem''\textgreater✲ legen alle Propheten das Zeugnis ab, daß jeder,
der an ihn glaubt, Vergebung der Sünden durch seinen Namen empfängt.«

\bibleverse{44} Während Petrus noch in dieser Weise redete, fiel der
heilige Geist auf alle, die seine Ansprache hörten. \bibleverse{45} Da
gerieten die Gläubigen jüdischer Herkunft\textless sup title=``d.h. die
Judenchristen''\textgreater✲, die mit Petrus gekommen waren, in das
höchste Erstaunen darüber, daß auch auf die Heiden die Gabe des heiligen
Geistes ausgegossen war; \bibleverse{46} denn sie hörten sie mit Zungen
reden und Gott preisen. Da sprach Petrus: \bibleverse{47} »Kann wohl
jemand diesen Leuten, die den heiligen Geist ebenso wie wir empfangen
haben, das Wasser versagen, daß diese nicht getauft würden?«
\bibleverse{48} So ordnete er denn an, daß sie im Namen Jesu Christi
getauft würden. Hierauf baten sie ihn, noch einige Tage bei ihnen zu
bleiben.

\hypertarget{b-petrus-rechtfertigt-in-jerusalem-die-heidentaufe}{%
\paragraph{b) Petrus rechtfertigt in Jerusalem die
Heidentaufe}\label{b-petrus-rechtfertigt-in-jerusalem-die-heidentaufe}}

\hypertarget{section-10}{%
\section{11}\label{section-10}}

\bibleverse{1} Es erhielten aber die Apostel und die Brüder in Judäa
Kunde davon, daß auch die Heiden das Wort Gottes angenommen hätten.
\bibleverse{2} Als daher Petrus nach Jerusalem zurückgekehrt war,
stellten die Gläubigen jüdischer Herkunft ihn zur Rede \bibleverse{3}
und hielten ihm vor: »Du bist bei Nichtjuden eingekehrt und hast mit
ihnen gegessen!« \bibleverse{4} Da legte Petrus ihnen im einzelnen dar,
wie sich alles von Anfang an zugetragen hatte, und berichtete:
\bibleverse{5} »Ich befand mich in der Stadt Joppe im Gebet; da sah ich
im Zustande der Verzückung ein Gesicht: ein Behälter kam herab wie ein
großes Stück Leinwand, das an den vier Zipfeln vom Himmel herabgelassen
wurde und bis zu mir kam. \bibleverse{6} Als ich dann hineinschaute und
es mir genau ansah, erblickte ich darin die vierfüßigen Tiere der Erde,
auch die wilden, und die kriechenden Tiere und die Vögel des Himmels.
\bibleverse{7} Zugleich vernahm ich auch eine Stimme, die mir zurief:
›Stehe auf, Petrus, schlachte und iß!‹ \bibleverse{8} Ich erwiderte
darauf: ›Nicht doch, Herr! Denn noch nie ist etwas Unheiliges und
Unreines in meinen Mund gekommen.‹ \bibleverse{9} Aber eine Stimme
erscholl zum zweitenmal vom Himmel her: ›Was Gott gereinigt hat, das
erkläre du nicht für unrein!‹ \bibleverse{10} Dies wiederholte sich bis
zum drittenmal; dann wurde alles wieder in den Himmel hinaufgezogen.
\bibleverse{11} Und seht, sogleich standen vor dem Hause, in dem wir
waren, drei Männer, die aus Cäsarea zu mir gesandt waren;
\bibleverse{12} und der Geist gebot mir, ohne alles Bedenken mit ihnen
zu gehen. Es begleiteten mich aber auch diese sechs Brüder hier, und wir
traten in das Haus des Mannes ein. \bibleverse{13} Der berichtete uns
nun, wie er den Engel in seinem Hause gesehen habe, der dagestanden und
gesagt hätte: ›Sende nach Joppe und laß Simon mit dem Beinamen Petrus
holen; \bibleverse{14} der wird Worte zu dir reden, durch die du mit
deinem ganzen Hause gerettet werden wirst.‹ \bibleverse{15} Während ich
dann zu reden begann, fiel der heilige Geist auf sie ebenso wie auch auf
uns im Anfang. \bibleverse{16} Da gedachte ich an das Wort des Herrn,
wie er sagte: ›Johannes hat mit Wasser getauft, ihr aber werdet mit
heiligem Geist getauft werden.‹ \bibleverse{17} Wenn somit Gott ihnen
die gleiche Gnadengabe verliehen hat wie uns, die wir zum Glauben an den
Herrn Jesus Christus gekommen sind, wie wäre ich da imstande gewesen,
Gott zu wehren?« \bibleverse{18} Als sie das hörten, beruhigten sie sich
und priesen Gott mit den Worten: »So hat Gott also auch den Heiden die
Buße\textless sup title=``oder: Bekehrung; vgl. Mt 3,2''\textgreater✲
zum Leben verliehen!«

\hypertarget{gruxfcndung-der-ersten-heidenchristlichen-gemeinde-zu-antiochia-in-syrien-deren-hilfeleistung-fuxfcr-die-notleidenden-christen-in-juduxe4a}{%
\subsubsection{8. Gründung der ersten heidenchristlichen Gemeinde zu
Antiochia in Syrien; deren Hilfeleistung für die notleidenden Christen
in
Judäa}\label{gruxfcndung-der-ersten-heidenchristlichen-gemeinde-zu-antiochia-in-syrien-deren-hilfeleistung-fuxfcr-die-notleidenden-christen-in-juduxe4a}}

\bibleverse{19} Diejenigen (Gläubigen) nun, welche sich aus Anlaß der
Verfolgung, die wegen des Stephanus entstanden war, zerstreut hatten,
waren bis nach Phönizien, Cypern und Antiochia gezogen, ohne jedoch
jemandem das Wort\textless sup title=``=~die Heilslehre''\textgreater✲
zu verkündigen als nur Juden. \bibleverse{20} Es befanden sich aber
einige Männer aus Cypern und Cyrene unter ihnen, die sich nach ihrer
Ankunft in Antiochia auch mit den Griechen\textless sup
title=``=~griechisch redenden Heiden''\textgreater✲ besprachen und ihnen
die Heilsbotschaft vom Herrn Jesus verkündigten. \bibleverse{21} Und die
Hand des Herrn war mit ihnen, so daß eine große Anzahl gläubig wurde und
sich zum Herrn bekehrte. \bibleverse{22} Die Kunde hiervon kam der
Gemeinde in Jerusalem zu Ohren, die dann den Barnabas nach Antiochia
entsandte. \bibleverse{23} Als dieser nach seiner Ankunft dort die Gnade
Gottes wahrnahm, freute er sich und ermahnte alle, mit festem Herzen dem
Herrn treu zu bleiben; \bibleverse{24} er war nämlich ein trefflicher
Mann, erfüllt mit heiligem Geist und mit Glauben. So wurde denn eine
ansehnliche Menge für den Herrn hinzugewonnen. \bibleverse{25} Barnabas
begab sich dann von dort nach Tarsus, um Saulus aufzusuchen;
\bibleverse{26} und als er ihn gefunden hatte, nahm er ihn mit nach
Antiochia; und es fügte sich dann so, daß sie ein ganzes Jahr hindurch
als Gäste in der Gemeinde tätig waren und eine beträchtliche Menge
unterwiesen und daß man in Antiochia zuerst den Jüngern den Namen
›Christen‹ (wörtlich: Christianer) beilegte.

\bibleverse{27} In dieser Zeit kamen Propheten von Jerusalem nach
Antiochia hinab. \bibleverse{28} Einer von ihnen namens Agabus trat auf
und weissagte auf Eingebung des Geistes, daß eine große Hungersnot über
den ganzen Erdkreis kommen würde, die dann auch wirklich unter der
Regierung des (Kaisers) Klaudius eintrat. \bibleverse{29} Da beschlossen
die Jünger, jeder von ihnen solle nach Maßgabe seines Vermögens den im
jüdischen Lande wohnenden Brüdern eine Unterstützung zukommen lassen.
\bibleverse{30} Sie führten diesen Beschluß auch aus und ließen es (den
Ertrag der Sammlung) durch Vermittlung des Barnabas und Saulus an die
Ältesten (der Gemeinde) gelangen.

\hypertarget{verfolgung-der-gemeinde-in-jerusalem-durch-herodes-agrippa-hinrichtung-des-jakobus-petrus-gefangen-und-wunderbar-befreit-tod-des-herodes}{%
\subsubsection{9. Verfolgung der Gemeinde in Jerusalem durch Herodes
Agrippa; Hinrichtung des Jakobus; Petrus gefangen und wunderbar befreit;
Tod des
Herodes}\label{verfolgung-der-gemeinde-in-jerusalem-durch-herodes-agrippa-hinrichtung-des-jakobus-petrus-gefangen-und-wunderbar-befreit-tod-des-herodes}}

\hypertarget{a-tod-des-jakobus-verhaftung-des-petrus}{%
\paragraph{a) Tod des Jakobus, Verhaftung des
Petrus}\label{a-tod-des-jakobus-verhaftung-des-petrus}}

\hypertarget{section-11}{%
\section{12}\label{section-11}}

\bibleverse{1} Um jene Zeit ließ der König Herodes einige Mitglieder der
Gemeinde gefangennehmen, um seine Wut an ihnen auszulassen.
\bibleverse{2} So ließ er Jakobus, den Bruder des Johannes, mit dem
Schwert hinrichten; \bibleverse{3} und als er sah, daß sein Vorgehen den
Beifall der Juden fand, ließ er weiter auch Petrus verhaften, und zwar
während der Tage der ungesäuerten Brote. \bibleverse{4} Als er ihn nun
festgenommen hatte, ließ er ihn ins Gefängnis setzen und übertrug seine
Bewachung vier Abteilungen Soldaten von je vier Mann; nach dem Passah
wollte er ihn dann dem Volk (zur Aburteilung) vorführen lassen.
\bibleverse{5} So wurde also Petrus im Gefängnis bewacht, während von
der Gemeinde unablässig für ihn zu Gott gebetet wurde.

\hypertarget{b-wunderbare-rettung-des-petrus}{%
\paragraph{b) Wunderbare Rettung des
Petrus}\label{b-wunderbare-rettung-des-petrus}}

\bibleverse{6} Als ihn nun Herodes (zur Verurteilung) vorführen lassen
wollte, schlief Petrus in der Nacht zuvor zwischen zwei Soldaten, mit
zwei Ketten gefesselt; außerdem versahen Posten vor der Tür die
Bewachung (der Zelle). \bibleverse{7} Da stand mit einemmal ein Engel
des Herrn da, und Lichtschein erhellte den Raum. Der Engel weckte den
Petrus durch einen Stoß in die Seite und sagte zu ihm: »Stehe schnell
auf!« Zugleich fielen ihm die Ketten von den Armen ab. \bibleverse{8}
Weiter sagte der Engel zu ihm: »Gürte dich und binde dir die Sandalen
unter!« Das tat Petrus. Dann sagte der Engel zu ihm: »Wirf dir deinen
Mantel um und folge mir!« \bibleverse{9} So ging Petrus denn hinter ihm
her (aus der Zelle) hinaus, wußte aber nicht, daß das, was durch den
Engel geschah, Wirklichkeit war; er meinte vielmehr zu träumen.
\bibleverse{10} Als sie dann an dem ersten und zweiten Wachposten
vorübergegangen waren, kamen sie an das eiserne Tor, das zur Stadt
hinausführte; dieses öffnete sich ihnen von selbst, und nachdem sie
hinausgetreten waren, gingen sie eine Straße weit vorwärts; da
verschwand plötzlich der Engel neben ihm.

\bibleverse{11} Als Petrus nun zu sich kam, sagte er: »Jetzt weiß ich in
Wahrheit, daß der Herr seinen Engel gesandt und mich aus der Hand des
Herodes und vor der ganzen Gier des jüdischen Volks gerettet hat.«
\bibleverse{12} Nachdem er sich hierüber klar geworden war, ging er nach
dem Hause der Maria, der Mutter des Johannes, der den Beinamen Markus
führte; hier waren viele (Brüder) versammelt und beteten.
\bibleverse{13} Als er nun an die Tür des Hoftores geklopft hatte, kam
eine Magd namens Rhode herbei, um zu horchen, wer da sei.
\bibleverse{14} Als sie Petrus an der Stimme erkannte, schloß sie in
ihrer Freude das Tor nicht auf, sondern lief ins Haus hinein und
meldete, Petrus stehe vor dem Tor. \bibleverse{15} Jene antworteten ihr:
»Du bist von Sinnen!«, doch sie versicherte bestimmt, es verhalte sich
so. \bibleverse{16} Da sagten sie: »Es muß sein Engel sein!« Petrus aber
fuhr inzwischen beharrlich fort zu pochen; da schlossen sie auf, sahen
ihn und gerieten in das höchste Erstaunen. \bibleverse{17} Er aber gab
ihnen mit der Hand einen Wink, sie möchten sich still verhalten,
erzählte ihnen dann, wie der Herr ihn aus dem Gefängnis herausgeführt
habe, und gab ihnen den Auftrag: »Teilt dies dem Jakobus und den
(übrigen) Brüdern mit!« Darauf entfernte er sich und begab sich an einen
anderen Ort.

\hypertarget{c-zorn-des-herodes-sein-untergang-in-cuxe4sarea-durch-ein-gottesgericht}{%
\paragraph{c) Zorn des Herodes; sein Untergang in Cäsarea durch ein
Gottesgericht}\label{c-zorn-des-herodes-sein-untergang-in-cuxe4sarea-durch-ein-gottesgericht}}

\bibleverse{18} Nach Tagesanbruch aber entstand eine nicht geringe
Bestürzung unter den Soldaten, was wohl mit Petrus geschehen sei.
\bibleverse{19} Herodes wollte ihn nämlich holen lassen; und als er ihn
nicht vorfand, stellte er ein Verhör mit den Wächtern an und ließ sie
(zur Hinrichtung) abführen. Darnach begab er sich aus Judäa nach Cäsarea
hinab und verlegte dorthin seine Hofhaltung.

\bibleverse{20} Er war damals aber gegen die Einwohner von Tyrus und
Sidon erbittert; diese schickten jedoch nach gemeinsamem Beschluß eine
Gesandtschaft an ihn, gewannen Blastus, den königlichen Kammerherrn, für
sich und baten um Frieden; ihr Land war nämlich bezüglich der Zufuhr auf
das Land des Königs angewiesen. \bibleverse{21} Am festgesetzten Tage
nun legte Herodes königliche Gewandung an, nahm auf der Rednerbühne
Platz und hielt eine Ansprache an sie. \bibleverse{22} Dabei rief das
Volk ihm zu: »Ein Gott redet und nicht ein Mensch!« \bibleverse{23} Da
schlug ihn augenblicklich ein Engel des Herrn zur Strafe dafür, daß er
nicht Gott die Ehre gegeben hatte: er erkrankte am Wurmfraß und beschloß
sein Leben.

\hypertarget{d-schluuxdf-des-ersten-teils-der-apostelgeschichte-uxe4uuxdferes-und-inneres-wachstum-der-gemeinde}{%
\paragraph{d) Schluß des ersten Teils der Apostelgeschichte; äußeres und
inneres Wachstum der
Gemeinde}\label{d-schluuxdf-des-ersten-teils-der-apostelgeschichte-uxe4uuxdferes-und-inneres-wachstum-der-gemeinde}}

\bibleverse{24} Das Wort Gottes jedoch wuchs und breitete sich immer
weiter aus. \bibleverse{25} Barnabas und Saulus aber kehrten nach
Ausrichtung ihres Auftrags von Jerusalem (nach Antiochia) zurück und
nahmen auch Johannes mit dem Beinamen Markus mit sich.

\hypertarget{ii.-geschichte-der-heidenbekehrung-kap.-13-28}{%
\subsection{II. Geschichte der Heidenbekehrung (Kap.
13-28)}\label{ii.-geschichte-der-heidenbekehrung-kap.-13-28}}

\hypertarget{a.-bekehrungstuxe4tigkeit-des-paulus-und-seiner-genossen-131-2114}{%
\subsection{A. Bekehrungstätigkeit des Paulus und seiner Genossen
(13,1-21,14)}\label{a.-bekehrungstuxe4tigkeit-des-paulus-und-seiner-genossen-131-2114}}

\hypertarget{erste-bekehrungsreise-des-paulus-und-barnabas-von-antiochia-aus}{%
\subsubsection{1. Erste Bekehrungsreise des Paulus und Barnabas von
Antiochia
aus}\label{erste-bekehrungsreise-des-paulus-und-barnabas-von-antiochia-aus}}

\hypertarget{a-einsegnung-aussendung-und-abreise-der-beiden-apostel-ihre-wirksamkeit-in-cypern}{%
\paragraph{a) Einsegnung, Aussendung und Abreise der beiden Apostel;
ihre Wirksamkeit in
Cypern}\label{a-einsegnung-aussendung-und-abreise-der-beiden-apostel-ihre-wirksamkeit-in-cypern}}

\hypertarget{section-12}{%
\section{13}\label{section-12}}

\bibleverse{1} In Antiochia wirkten damals in der dortigen Gemeinde
folgende Propheten\textless sup title=``vgl. 1.Kor 12,28''\textgreater✲
und Lehrer: Barnabas, Symeon✲ mit dem Beinamen Niger, Lucius aus Cyrene,
Manaën, der mit dem Vierfürsten Herodes erzogen worden war, und Saulus.
\bibleverse{2} Als sie nun einst dem Herrn Gottesdienst hielten und
fasteten, gebot der heilige Geist: »Sondert mir doch Barnabas und Saulus
für das Werk aus, zu dem ich sie berufen habe!« \bibleverse{3} Da
fasteten und beteten sie, legten ihnen die Hände auf und ließen sie
ziehen.

\bibleverse{4} So gingen denn die beiden, vom heiligen Geist ausgesandt,
nach Seleucia hinab, fuhren von dort zu Schiff nach Cypern
\bibleverse{5} und verkündigten nach ihrer Ankunft dort in Salamis das
Wort Gottes in den Synagogen der Juden; als Gehilfen hatten sie noch
Johannes (Markus) bei sich. \bibleverse{6} Nachdem sie nun die ganze
Insel bis nach Paphos durchzogen hatten, trafen sie dort einen jüdischen
Zauberer und falschen Propheten namens Barjesus\textless sup
title=``d.h. Sohn des Jesus''\textgreater✲, \bibleverse{7} der zu der
Umgebung des (römischen) Statthalters Sergius Paulus, eines verständigen
Mannes, gehörte. Dieser ließ Barnabas und Saulus zu sich rufen und
wünschte von ihnen das Wort Gottes zu hören. \bibleverse{8} Da trat aber
Elymas, der Zauberer -- so lautet nämlich sein Name übersetzt --, ihnen
entgegen und suchte den Statthalter vom Glauben abzuhalten.
\bibleverse{9} Saulus aber, der auch Paulus heißt, blickte ihn fest an
und sagte, voll heiligen Geistes: \bibleverse{10} »O du Teufelssohn, der
du ganz voll von lauter Lug und Trug bist, du Feind aller Gerechtigkeit,
wirst du nicht aufhören, die geraden Wege des Herrn krumm zu
machen\textless sup title=``=~zu vereiteln; vgl. Hos
14,10''\textgreater✲? \bibleverse{11} Jetzt aber kommt die Hand des
Herrn über dich: du sollst blind sein und das Sonnenlicht eine Zeitlang
nicht sehen!« Da fiel augenblicklich Dunkel und Finsternis auf ihn: er
tappte umher und suchte nach jemandem, der ihn an der Hand führen
sollte. \bibleverse{12} Als der Statthalter den Vorgang sah, kam er zum
Glauben und war voll Staunens über die (Kraft der) Lehre des Herrn.

\hypertarget{b-weiterreise-nach-kleinasien-und-aufenthalt-im-pisidischen-antiochia}{%
\paragraph{b) Weiterreise nach Kleinasien und Aufenthalt im pisidischen
Antiochia}\label{b-weiterreise-nach-kleinasien-und-aufenthalt-im-pisidischen-antiochia}}

\bibleverse{13} Von Paphos gingen Paulus und seine Gefährten wieder in
See und kamen nach Perge in Pamphylien; hier trennte sich Johannes von
ihnen und kehrte nach Jerusalem zurück. \bibleverse{14} Sie aber zogen
von Perge aus landeinwärts weiter und gelangten nach dem pisidischen
Antiochia, wo sie am Sabbattage in die Synagoge gingen und sich dort
(auf den für Fremde bestimmten Plätzen) niedersetzten. \bibleverse{15}
Nach der Schriftverlesung aus dem Gesetz und den Propheten ließen ihnen
die Vorsteher der Synagoge sagen: »Werte Brüder, wenn ihr eine
erbauliche Ansprache an das Volk zu richten habt, so redet!«

\bibleverse{16} Da stand Paulus auf, gab einen Wink mit der Hand und
sagte:

»Ihr Männer von Israel und ihr, die ihr Gott fürchtet, hört mich an!
\bibleverse{17} Der Gott unsers Volkes Israel hat unsere Väter sich
erwählt und unser Volk während seines Aufenthalts in dem fremden Lande
Ägypten emporgebracht und sie dann mit hocherhobenem Arm von dort
weggeführt. \bibleverse{18} Während einer Zeit von ungefähr vierzig
Jahren hat er sie dann in der Wüste mit schonender✲ Liebe getragen,
\bibleverse{19} hat sieben Völker im Lande Kanaan vernichtet und ihnen
deren Land zum Besitz gegeben; \bibleverse{20} das hat ungefähr
vierhundertfünfzig Jahre gedauert. Hierauf gab er ihnen Richter bis auf
den Propheten Samuel. \bibleverse{21} Von da an wollten sie einen König
haben, und Gott gab ihnen Saul, den Sohn des Kis, einen Mann aus dem
Stamme Benjamin, für vierzig Jahre. \bibleverse{22} Nach dessen
Verwerfung erhob er David zum König über sie; ihm hat er dann auch das
Zeugnis erteilt: ›Ich habe David gefunden, den Sohn Isais, einen Mann
nach meinem Herzen, der in allem meinen Willen tun wird.‹\textless sup
title=``Ps 89,21; 1.Sam 13,14''\textgreater✲ \bibleverse{23} Dieser
ist's, aus dessen Nachkommenschaft Gott jetzt nach seiner Verheißung
Jesus als Retter✲ für Israel hat hervorgehen lassen, \bibleverse{24}
nachdem vor dessen Auftreten Johannes dem ganzen Volk Israel eine Taufe
der Buße verkündigt hatte. \bibleverse{25} Als aber Johannes am Abschluß
seiner Laufbahn stand, erklärte er: ›Das, wofür ihr mich haltet, bin ich
nicht; doch wisset wohl, nach mir kommt der, für den ich nicht gut genug
bin, ihm die Schuhe von den Füßen loszubinden!‹

\bibleverse{26} Werte Brüder, Söhne von Abrahams Geschlecht, und ihr
anderen hier, die ihr Gott fürchtet: an uns ist diese Heilsverkündigung
ergangen! \bibleverse{27} Denn die Bewohner Jerusalems und ihre Oberen
haben, obwohl sie diesen Jesus nicht erkannten, doch die Aussprüche der
Propheten, die an jedem Sabbat zur Verlesung kommen, durch ihr
Verdammungsurteil zur Erfüllung gebracht \bibleverse{28} und, obwohl sie
keine todeswürdige Schuld an ihm fanden, doch seine Hinrichtung von
Pilatus verlangt. \bibleverse{29} Nachdem sie schließlich alles
ausgeführt hatten, was über ihn in der Schrift steht, haben sie ihn vom
Kreuz herabgenommen und ihn in ein Grab gelegt. \bibleverse{30} Gott
aber hat ihn von den Toten auferweckt, \bibleverse{31} und mehrere Tage
hindurch ist er denen erschienen, die mit ihm aus Galiläa nach Jerusalem
hinaufgezogen waren und die jetzt Zeugen für ihn dem Volk gegenüber
sind. \bibleverse{32} Und wir bringen euch die das Heil verkündende
Botschaft, daß Gott die Verheißung, die unsern Vätern einst zuteil
geworden ist, \bibleverse{33} für uns, die Nachkommen jener, durch die
Auferweckung Jesu zur Erfüllung gebracht hat, wie ja auch im zweiten
Psalm geschrieben steht\textless sup title=``Ps 2,7''\textgreater✲: ›Du
bist mein Sohn, ich habe dich heute gezeugt.‹ \bibleverse{34} Daß er ihn
aber von den Toten auferweckt hat, um ihn nicht wieder der Verwesung
anheimfallen zu lassen, hat er mit den Worten ausgesprochen\textless sup
title=``Jes 55,3''\textgreater✲: ›Ich will euch die heiligen, dem David
verheißenen unverbrüchlichen Güter verleihen.‹ \bibleverse{35} Darum
heißt es auch an einer andern Stelle\textless sup title=``Ps
16,10''\textgreater✲: ›Du wirst nicht zulassen, daß dein Heiliger die
Verwesung sieht.‹ \bibleverse{36} Denn David selbst ist doch
entschlafen, nachdem er zu seiner Zeit dem Willen Gottes gedient hatte:
er ist zu seinen Vätern versammelt worden und hat die Verwesung gesehen;
\bibleverse{37} der aber, den Gott auferweckt hat, der hat die Verwesung
nicht gesehen.

\bibleverse{38} So soll euch denn kundgetan sein, werte Brüder, daß euch
durch diesen (Jesus) die Vergebung der Sünden verkündigt wird,
\bibleverse{39} und von allem, wovon ihr durch das mosaische Gesetz
keine Rechtfertigung habt erlangen können, wird in diesem\textless sup
title=``oder: durch diesen''\textgreater✲ ein jeder gerechtfertigt, der
da glaubt. \bibleverse{40} Darum seht wohl zu, daß {[}bei euch{]} nicht
das Prophetenwort zutreffe\textless sup title=``Hab 1,5''\textgreater✲:
\bibleverse{41} ›Seht, ihr Verächter, verwundert euch und vergeht! Denn
ein Werk vollführe ich in euren Tagen, ein Werk, das ihr gewiß nicht
glauben würdet, wenn jemand es euch erzählte.‹«

\hypertarget{verschiedener-erfolg-der-rede}{%
\paragraph{Verschiedener Erfolg der
Rede}\label{verschiedener-erfolg-der-rede}}

\bibleverse{42} Als sie dann die Synagoge verließen, sprach man die
Bitte gegen sie aus, sie möchten am nächsten Sabbat von diesen Dingen
noch weiter zu ihnen reden. \bibleverse{43} Nachdem nun die
Synagogen-Versammlung auseinandergegangen war, folgten viele von den
Juden und den gottesfürchtigen Heidenjuden dem Paulus und Barnabas nach;
diese redeten ihnen eifrig zu und ermahnten sie, in der Gnade Gottes zu
verharren. \bibleverse{44} Am folgenden Sabbat aber fand sich beinahe
die ganze Stadt ein, um das Wort Gottes zu hören. \bibleverse{45} Als
jedoch die Juden die Volksmenge sahen, wurden sie mit Eifersucht erfüllt
und widersprachen den Darlegungen des Paulus unter Schmähungen.

\bibleverse{46} Da erklärten ihnen Paulus und Barnabas mit Freimut:
»Euch (Juden) mußte das Wort Gottes zuerst verkündigt werden; weil ihr
es aber zurückstoßt und euch selbst des ewigen Lebens nicht für würdig
erachtet, so wenden wir uns nunmehr zu den Heiden! \bibleverse{47} Denn
so hat uns der Herr geboten\textless sup title=``Jes
49,6''\textgreater✲: ›Ich habe dich zum Licht der Heiden
gemacht\textless sup title=``oder: bestimmt''\textgreater✲, damit du zum
Heil werdest bis ans Ende der Erde.‹« \bibleverse{48} Als die Heiden das
hörten, freuten sie sich und priesen das Wort des Herrn; und alle,
soweit sie zum ewigen Leben verordnet waren, wurden gläubig.
\bibleverse{49} So verbreitete sich denn das Wort des Herrn durch die
ganze Gegend.

\bibleverse{50} Die Juden aber hetzten die vornehmen Frauen, die sich
zum Judentum hielten, und die angesehensten Männer der Stadt auf,
erregten eine Verfolgung gegen Paulus und Barnabas und vertrieben sie
aus dem Gebiet ihrer Stadt. \bibleverse{51} Da schüttelten diese den
Staub von ihren Füßen (zum Zeugnis) gegen sie und begaben sich nach
Ikonium; \bibleverse{52} die Jünger aber wurden mit Freude und mit
heiligem Geist erfüllt.

\hypertarget{c-aufenthalt-in-ikonium-lystra-und-derbe}{%
\paragraph{c) Aufenthalt in Ikonium, Lystra und
Derbe}\label{c-aufenthalt-in-ikonium-lystra-und-derbe}}

\hypertarget{aa-wirksamkeit-der-apostel-in-ikonium}{%
\subparagraph{aa) Wirksamkeit der Apostel in
Ikonium}\label{aa-wirksamkeit-der-apostel-in-ikonium}}

\hypertarget{section-13}{%
\section{14}\label{section-13}}

\bibleverse{1} In Ikonium gingen sie in derselben Weise (wie in
Antiochia) in die Synagoge der Juden und predigten mit solchem Erfolg,
daß sowohl von den Juden als auch von den Griechen\textless sup
title=``d.h. griechisch sprechenden Heiden''\textgreater✲ eine große
Zahl zum Glauben kam. \bibleverse{2} Von den Juden aber reizten die,
welche ungläubig geblieben waren, die heidnische Bevölkerung zur
Erbitterung gegen die Brüder an. \bibleverse{3} Dennoch blieben Paulus
und Barnabas geraume Zeit dort und predigten freimütig im Vertrauen auf
den Herrn, der für das Wort\textless sup title=``oder: die
Verkündigung''\textgreater✲ seiner Gnade dadurch Zeugnis ablegte, daß er
Zeichen und Wunder durch ihre Hände geschehen ließ. \bibleverse{4} Da
entstand eine Spaltung unter der Bevölkerung der Stadt: die einen
hielten es mit den Juden, die anderen mit den Aposteln. \bibleverse{5}
Als aber die Heiden und die Juden im Einvernehmen mit der dortigen
Obrigkeit\textless sup title=``oder: mit ihren Vorstehern''\textgreater✲
voller Wut den Beschluß faßten, sich tätlich an ihnen zu vergreifen und
sie zu steinigen, \bibleverse{6} entflohen sie, nachdem sie Kunde davon
erhalten hatten, in die lykaonischen Städte Lystra und Derbe und deren
Umgegend \bibleverse{7} und setzten auch dort die Verkündigung der
Heilsbotschaft fort.

\hypertarget{bb-heilung-eines-lahmgeborenen-und-steinigung-des-paulus-in-lystra-die-beiden-apostel-entweichen-nach-derbe}{%
\subparagraph{bb) Heilung eines Lahmgeborenen und Steinigung des Paulus
in Lystra; die beiden Apostel entweichen nach
Derbe}\label{bb-heilung-eines-lahmgeborenen-und-steinigung-des-paulus-in-lystra-die-beiden-apostel-entweichen-nach-derbe}}

\bibleverse{8} Nun wohnte da in Lystra ein Mann, der keine Kraft in
seinen Beinen hatte; er war von Geburt an lahm und hatte noch niemals
gehen können. \bibleverse{9} Dieser hörte der Predigt des Paulus zu; und
als dieser ihn fest ansah und erkannte, daß er den Glauben hatte, der zu
seiner Heilung nötig war, \bibleverse{10} rief er ihm mit lauter Stimme
zu: »Stelle dich aufrecht auf deine Füße hin!« Da sprang er auf und ging
umher. \bibleverse{11} Als nun die Volksmenge sah, was Paulus getan
hatte, erhoben sie ihre Stimme und riefen auf lykaonisch aus: »Die
Götter haben Menschengestalt angenommen und sind zu uns herabgekommen!«
\bibleverse{12} Dabei nannten sie Barnabas Zeus\textless sup
title=``oder: Jupiter''\textgreater✲ und den Paulus Hermes\textless sup
title=``oder: Merkurius''\textgreater✲, weil dieser es war, der das Wort
führte; \bibleverse{13} und der Priester des Zeus\textless sup
title=``oder: Jupiter''\textgreater✲, der vor der Stadt seinen Tempel
hatte, brachte Stiere und Kränze an das Stadttor und wollte mit den
Volksscharen Opfer darbringen. \bibleverse{14} Als die Apostel Barnabas
und Paulus das vernahmen, zerrissen sie ihre Kleider, sprangen in die
Volksmenge hinein \bibleverse{15} und riefen laut: »Ihr Männer, was tut
ihr da? Wir sind auch nur Menschen von derselben Art wie ihr und
verkündigen euch die Heilsbotschaft, damit ihr euch von diesen
Verkehrtheiten\textless sup title=``=~nichtigen Götzen''\textgreater✲ zu
dem lebendigen Gott bekehrt, der den Himmel und die Erde, das Meer und
alles, was darin ist, geschaffen hat\textless sup title=``2.Mose
20,11''\textgreater✲. \bibleverse{16} Er hat in den vergangenen Zeiten
alle Heidenvölker ihre eigenen Wege gehen lassen, \bibleverse{17} doch
sich durch seine Wohltaten nicht unbezeugt gelassen, indem er euch Regen
und fruchtbare Zeiten vom Himmel her gesandt und euch reichlich Nahrung
geschenkt und eure Herzen mit Freude erfüllt hat.« \bibleverse{18} Durch
diese Worte brachten sie die Menge nur mit Mühe davon ab, ihnen zu
opfern.

\bibleverse{19} Es kamen dann aber Juden aus Antiochia und Ikonium
herüber, welche die Einwohnerschaft umstimmten\textless sup
title=``oder: für sich gewannen''\textgreater✲; sie steinigten Paulus
und schleiften ihn zur Stadt hinaus in der Meinung, er sei tot.
\bibleverse{20} Als ihn aber die Jünger umringten, stand er auf und ging
wieder in die Stadt hinein. \bibleverse{21} a Am folgenden Tage zog er
dann mit Barnabas nach Derbe weiter.

\hypertarget{d-die-apostel-in-derbe-bestuxe4rkung-der-gegruxfcndeten-gemeinden-ruxfcckkehr-nach-antiochia-in-syrien}{%
\paragraph{d) Die Apostel in Derbe; Bestärkung der gegründeten
Gemeinden; Rückkehr nach Antiochia in
Syrien}\label{d-die-apostel-in-derbe-bestuxe4rkung-der-gegruxfcndeten-gemeinden-ruxfcckkehr-nach-antiochia-in-syrien}}

b Sie verkündigten die Heilsbotschaft (auch) in dieser Stadt und
kehrten, nachdem sie zahlreiche Jünger gewonnen hatten, nach Lystra,
Ikonium und (dem pisidischen) Antiochia zurück. \bibleverse{22} Sie
stärkten überall die Herzen der Jünger, ermahnten sie zu festem
Ausharren im Glauben und wiesen sie darauf hin, daß wir durch viele
Leiden in das Reich Gottes eingehen müssen. \bibleverse{23} Sie
erwählten ihnen für jede Gemeinde Älteste und befahlen diese unter Gebet
und Fasten dem Herrn, an den sie gläubig geworden waren. \bibleverse{24}
Als sie dann Pisidien durchzogen hatten, kamen sie nach Pamphylien,
\bibleverse{25} wo sie in Perge das Wort (des Herrn) noch predigten;
dann zogen sie nach Attalia hinab. \bibleverse{26} Von dort fuhren sie
zu Schiff nach Antiochia, von wo aus sie der Gnade Gottes für das Werk,
das sie nun glücklich vollbracht hatten, befohlen worden waren.
\bibleverse{27} Nach ihrer Ankunft beriefen sie eine Gemeindeversammlung
und berichteten über alles, was Gott durch sie vollbracht hatte,
besonders darüber, daß er den Heiden die Tür zum Glauben aufgetan habe.
\bibleverse{28} Sie blieben sodann noch längere Zeit bei den Jüngern
dort.

\hypertarget{die-apostelversammlung-oder-der-apostelkonvent-in-jerusalem-betreffend-die-freiheit-der-heidenchristen-vom-juxfcdischen-gesetz}{%
\subsubsection{2. Die Apostelversammlung (oder: der Apostelkonvent) in
Jerusalem betreffend die Freiheit der Heidenchristen vom jüdischen
Gesetz}\label{die-apostelversammlung-oder-der-apostelkonvent-in-jerusalem-betreffend-die-freiheit-der-heidenchristen-vom-juxfcdischen-gesetz}}

\hypertarget{a-die-veranlassung-des-konvents-sendung-des-paulus-und-barnabas-nach-jerusalem}{%
\paragraph{a) Die Veranlassung des Konvents; Sendung des Paulus und
Barnabas nach
Jerusalem}\label{a-die-veranlassung-des-konvents-sendung-des-paulus-und-barnabas-nach-jerusalem}}

\hypertarget{section-14}{%
\section{15}\label{section-14}}

\bibleverse{1} Da trugen einige (Gläubige), die aus Judäa (nach
Antiochia) herabgekommen waren, den Brüdern die Lehre vor: »Wenn ihr
euch nicht nach mosaischem Brauch beschneiden laßt, könnt ihr die
Rettung\textless sup title=``oder: das Heil''\textgreater✲ nicht
erlangen!« \bibleverse{2} Als nun dadurch eine Aufregung (in der
Gemeinde) und ein heftiger Streit zwischen diesen Männern und Paulus und
Barnabas entstanden war, faßte man den Beschluß, Paulus und Barnabas
nebst einigen anderen aus ihrer Mitte sollten wegen dieser Streitfrage
zu den Aposteln und Ältesten nach Jerusalem hinaufziehen. \bibleverse{3}
Diese wurden also von der Gemeinde feierlich entlassen und reisten dann
durch Phönizien und Samarien, wobei sie (überall) von der Bekehrung der
Heiden berichteten und dadurch allen Brüdern große Freude bereiteten.
\bibleverse{4} Nach ihrer Ankunft in Jerusalem wurden sie von der
Gemeinde und von den Aposteln und den Ältesten empfangen und berichteten
alles, was Gott durch sie vollführt hatte. \bibleverse{5} Da traten
einige, die zu der Partei der Pharisäer gehört hatten und gläubig
geworden waren, mit der Forderung auf, man müsse (die Heidenchristen)
beschneiden und von ihnen die Beobachtung des mosaischen Gesetzes
verlangen.

\hypertarget{b-die-verhandlungen-reden-des-petrus-und-jakobus}{%
\paragraph{b) Die Verhandlungen; Reden des Petrus und
Jakobus}\label{b-die-verhandlungen-reden-des-petrus-und-jakobus}}

\bibleverse{6} So traten denn die Apostel und die Ältesten zur Beratung
über diese Frage zusammen. \bibleverse{7} Nachdem nun eine lange,
erregte Erörterung stattgefunden hatte, stand Petrus auf und sprach zu
ihnen: »Werte Brüder! Ihr wißt, daß Gott schon vor längerer Zeit mich in
eurem Kreise dazu erwählt hat, daß die Heiden durch meinen Mund das Wort
der Heilsbotschaft vernehmen und so zum Glauben kommen sollten.
\bibleverse{8} Und Gott, der Herzenskenner, hat selbst Zeugnis für sie
dadurch abgelegt, daß er ihnen den heiligen Geist gerade so verliehen
hat wie uns: \bibleverse{9} er hat keinen Unterschied zwischen uns und
ihnen gemacht, indem er ihre Herzen durch den Glauben gereinigt hat.
\bibleverse{10} Warum versucht ihr also jetzt Gott dadurch, daß ihr den
Jüngern ein Joch auf den Nacken legen wollt, das weder unsere Väter noch
wir zu tragen vermocht haben? \bibleverse{11} Nein, durch die Gnade des
Herrn Jesus glauben wir auf dieselbe Weise die Rettung\textless sup
title=``oder: das Heil''\textgreater✲ zu erlangen wie jene auch.«
\bibleverse{12} Da schwieg die ganze Versammlung still und schenkte dem
Barnabas und Paulus Gehör, die einen Bericht über alle die Zeichen und
Wunder erstatteten, die Gott unter den Heiden durch sie getan hatte.

\bibleverse{13} Als sie damit zu Ende waren, nahm Jakobus das Wort zu
folgender Ansprache: »Werte Brüder, hört mich an! \bibleverse{14}
Symeon✲ hat berichtet, wie Gott selbst zuerst darauf bedacht gewesen
ist, ein Volk aus den Heiden für seinen Namen zu gewinnen.
\bibleverse{15} Und damit stimmen die Worte der Propheten überein; denn
es steht geschrieben\textless sup title=``Am 9,11-12''\textgreater✲:
\bibleverse{16} ›Hierauf will ich umkehren und die zerfallene Hütte
Davids wieder aufbauen; ich will ihre Trümmer wieder aufrichten und sie
selbst neu erstehen lassen, \bibleverse{17} damit die Menschen, welche
übriggeblieben sind, den Herrn suchen, auch alle Heiden, die mir als
mein Volk zu eigen gehören, spricht der Herr, der dieses vollbringt,
\bibleverse{18} wie es von Ewigkeit her kund geworden ist.‹
\bibleverse{19} Deshalb bin ich meinerseits der Ansicht, man solle
denen, die aus der Heidenwelt sich zu Gott bekehren, keine (unnötigen)
Lasten aufbürden, \bibleverse{20} sondern ihnen nur die Verpflichtung
auferlegen, sich von der Verunreinigung durch die Götzen, von der
Unzucht, vom Fleisch erstickter Tiere und vom (Genuß von) Blut
fernzuhalten. \bibleverse{21} Denn Mose hat seit alten Zeiten in jeder
Stadt seine Verkündiger, weil er ja in den Synagogen an jedem Sabbat
vorgelesen wird.«

\hypertarget{c-der-beschluuxdf-und-seine-ausfuxfchrung}{%
\paragraph{c) Der Beschluß und seine
Ausführung}\label{c-der-beschluuxdf-und-seine-ausfuxfchrung}}

\bibleverse{22} Hierauf beschlossen die Apostel und die Ältesten im
Einvernehmen mit der ganzen Gemeinde, Männer aus ihrer Mitte zu wählen
und sie mit Paulus und Barnabas nach Antiochia zu senden, nämlich Judas
mit dem Beinamen Barsabbas und Silas, zwei Männer, die unter den Brüdern
eine führende Stellung einnahmen. \bibleverse{23} Durch diese ließen sie
folgendes Schreiben überbringen: »Wir Apostel und Älteste senden als
Brüder unseren Brüdern, den Heidenchristen in Antiochia, Syrien und
Cilicien, unsern Gruß. \bibleverse{24} Da wir vernommen haben, daß
einige aus unserer Mitte zu euch gekommen sind und euch durch Reden
beunruhigt und eure Seelen in Aufregung versetzt haben, ohne daß sie
einen Auftrag dazu von uns erhalten hatten, \bibleverse{25} so haben wir
in einer Versammlung den einmütigen Beschluß gefaßt, Männer zu erwählen
und zu euch zu senden zusammen mit unserm geliebten Barnabas und Paulus,
\bibleverse{26} zwei Männer, die ihre Seele\textless sup title=``oder:
ihr Leben''\textgreater✲ für den Namen unsers Herrn Jesus Christus
eingesetzt haben. \bibleverse{27} Wir haben also Judas und Silas
abgesandt, die euch dasselbe auch noch mündlich mitteilen werden.
\bibleverse{28} Es ist nämlich des heiligen Geistes und unser Beschluß,
euch keine weitere Last aufzubürden als folgende Stücke, die unerläßlich
sind: \bibleverse{29} daß ihr euch vom Götzenopferfleisch, vom
Blutgenuß, vom Fleisch erstickter Tiere und von Unzucht fernhaltet. Wenn
ihr euch davor bewahrt, werdet ihr euch gut dabei stehen. Gehabt euch
wohl!«

\hypertarget{d-der-ausgang-judas-und-silas-in-antiochia}{%
\paragraph{d) Der Ausgang: Judas und Silas in
Antiochia}\label{d-der-ausgang-judas-und-silas-in-antiochia}}

\bibleverse{30} So wurden diese denn verabschiedet und kamen nach
Antiochia, wo sie die Gemeinde beriefen und das Schreiben übergaben.
\bibleverse{31} Als jene es gelesen hatten, freuten sie sich über den
tröstlichen Zuspruch\textless sup title=``oder: die beruhigende
Botschaft''\textgreater✲. \bibleverse{32} Judas aber und Silas, welche
Propheten\textless sup title=``=~geisterfüllte Redner; vgl.
13,1''\textgreater✲ waren, spendeten auch ihrerseits den Brüdern durch
viele Ansprachen Zuspruch und stärkten sie (im Glauben). \bibleverse{33}
Nachdem sie dann einige Zeit dort zugebracht hatten, wurden sie von den
Brüdern in Frieden\textless sup title=``oder: mit
Segenswünschen''\textgreater✲ wieder zu ihren Auftraggebern entlassen.
\bibleverse{34} Silas aber entschloß sich, dort zu bleiben.

\hypertarget{zweite-bekehrungsreise-des-paulus-1535-1822}{%
\subsubsection{3. Zweite Bekehrungsreise des Paulus
(15,35-18,22)}\label{zweite-bekehrungsreise-des-paulus-1535-1822}}

\hypertarget{a-der-streit-des-paulus-mit-barnabas-aufbruch-des-paulus-und-silas-aus-antiochia}{%
\paragraph{a) Der Streit des Paulus mit Barnabas; Aufbruch des Paulus
und Silas aus
Antiochia}\label{a-der-streit-des-paulus-mit-barnabas-aufbruch-des-paulus-und-silas-aus-antiochia}}

\bibleverse{35} Paulus und Barnabas blieben dann in Antiochia, indem sie
das Wort des Herrn lehrten und die Heilsbotschaft mit noch vielen
anderen verkündigten. \bibleverse{36} Nach einiger Zeit aber sagte
Paulus zu Barnabas: »Laß uns doch wieder ausziehen und in allen Städten,
in denen wir das Wort des Herrn verkündigt haben, uns nach den Brüdern
umsehen, wie es mit ihnen steht!« \bibleverse{37} Nun wollte Barnabas
auch den Johannes mit dem Beinamen Markus (wieder) mitnehmen;
\bibleverse{38} Paulus aber hielt es nicht für recht, einen Mann
mitzunehmen, der sich (das vorige Mal) in Pamphylien von ihnen getrennt
und sie nicht auf das Arbeitsfeld begleitet hatte. \bibleverse{39} So
kam es denn zwischen beiden zu einem hitzigen Streit, infolgedessen sie
sich voneinander trennten: Barnabas nahm den Markus zu sich und fuhr zu
Schiff nach Cypern; \bibleverse{40} Paulus dagegen wählte sich Silas zum
Begleiter und trat (mit ihm) die Landreise an, nachdem er von den
Brüdern der Gnade des Herrn anbefohlen worden war. \bibleverse{41} Er
durchzog (zunächst) Syrien und Cilicien und stärkte die dortigen
Gemeinden.

\hypertarget{b-die-landreise-durch-kleinasien-bis-nach-troas}{%
\paragraph{b) Die Landreise durch Kleinasien bis nach
Troas}\label{b-die-landreise-durch-kleinasien-bis-nach-troas}}

\hypertarget{section-15}{%
\section{16}\label{section-15}}

\bibleverse{1} Weiter kam er dann auch nach Derbe und Lystra. Und siehe,
hier war ein Jünger namens Timotheus -- der Sohn einer gläubig
gewordenen Jüdin, aber eines griechischen✲ Vaters --, \bibleverse{2} dem
von den Brüdern in Lystra und Ikonium ein empfehlendes Zeugnis
ausgestellt wurde. \bibleverse{3} Paulus wünschte diesen als Begleiter
auf der Reise zu haben; so nahm er ihn denn zu sich und vollzog die
Beschneidung an ihm mit Rücksicht auf die Juden, die in jenen Gegenden
(wohnhaft) waren; denn es wußten ja alle, daß sein Vater ein Grieche✲
war. \bibleverse{4} Auf ihrer Wanderung durch die Städte machten sie den
Gläubigen dort zur Pflicht, die von den Aposteln und Ältesten in
Jerusalem beschlossenen Satzungen✲ zu beobachten. \bibleverse{5} So
wurden denn die Gemeinden von ihnen im Glauben gestärkt und nahmen
täglich an Zahl zu.

\bibleverse{6} Sie zogen dann weiter durch Phrygien und das galatische
Land, weil sie vom heiligen Geist daran gehindert wurden, die
Heilsbotschaft in (der römischen Provinz) Asien zu verkündigen.
\bibleverse{7} Als sie aber in die Nähe von Mysien gekommen waren,
machten sie den Versuch, nach Bithynien zu gelangen, doch der Geist Jesu
gestattete es ihnen nicht; \bibleverse{8} sie zogen vielmehr an der
Grenze von Mysien hin und kamen so an die Küste nach Troas hinunter.
\bibleverse{9} Hier erschien dem Paulus nachts ein Traumgesicht: Ein
mazedonischer Mann stand da und sprach die Bitte gegen ihn aus: »Komm
nach Mazedonien herüber und hilf uns!« \bibleverse{10} Als er diese
Erscheinung gesehen hatte, suchten wir sofort eine Gelegenheit, nach
Mazedonien zu gelangen, weil wir aus ihr schlossen, daß Gott uns dazu
berufen habe, ihnen die Heilsbotschaft zu verkündigen.

\hypertarget{c-die-seereise-nach-mazedonien-paulus-in-philippi}{%
\paragraph{c) Die Seereise nach Mazedonien; Paulus in
Philippi}\label{c-die-seereise-nach-mazedonien-paulus-in-philippi}}

\bibleverse{11} So segelten wir denn von Troas ab und fuhren geradeswegs
nach Samothrake, am folgenden Tage nach Neapolis \bibleverse{12} und von
dort nach Philippi, welches die erste✲ Stadt des (dortigen)
mazedonischen Bezirks ist, eine römische Kolonie\textless sup
title=``=~Siedlung oder: Pflanzstadt''\textgreater✲.

\hypertarget{aa-bekehrung-der-purpurhuxe4ndlerin-lydia}{%
\subparagraph{aa) Bekehrung der Purpurhändlerin
Lydia}\label{aa-bekehrung-der-purpurhuxe4ndlerin-lydia}}

In dieser Stadt blieben wir einige Tage \bibleverse{13} und gingen am
Sabbattage zum Stadttor hinaus an den Fluß, wo wir eine (jüdische)
Gebetsstätte vermuteten. Wir setzten uns dort nieder und redeten zu den
Frauen, die sich da versammelt hatten.

\bibleverse{14} Unter den Zuhörerinnen befand sich auch eine
gottesfürchtige Frau namens Lydia, eine Purpurhändlerin aus der Stadt
Thyatira (in Lydien); ihr öffnete der Herr das Herz, so daß sie den
Worten des Paulus Beachtung schenkte. \bibleverse{15} Als sie sich dann
samt ihren Hausgenossen hatte taufen lassen, sprach sie die Bitte aus:
»Wenn ihr wirklich in mir eine treue Jüngerin des Herrn erkannt habt, so
kommt in mein Haus und wohnt bei mir!« So nötigte sie uns (zu sich).

\hypertarget{bb-die-wahrsagende-magd-paulus-und-silas-vor-gericht-und-im-gefuxe4ngnis}{%
\subparagraph{bb) Die wahrsagende Magd; Paulus und Silas vor Gericht und
im
Gefängnis}\label{bb-die-wahrsagende-magd-paulus-und-silas-vor-gericht-und-im-gefuxe4ngnis}}

\bibleverse{16} Als wir nun (eines Tages wieder) auf dem Wege zu der
Gebetsstätte waren, begegnete uns eine Magd✲, die von einem
Wahrsagegeist besessen war und ihrer Herrschaft durch ihr Wahrsagen viel
Geld einbrachte. \bibleverse{17} Die ging hinter Paulus und uns her und
rief laut: »Diese Männer sind Diener des höchsten Gottes, die euch den
Weg zur Rettung\textless sup title=``oder: zum Heil''\textgreater✲
verkündigen!« \bibleverse{18} Das setzte sie viele Tage hindurch fort.
Darüber wurde Paulus unwillig; er wandte sich um und sprach zu dem
Geist: »Ich gebiete dir im Namen Jesu Christi, von ihr auszufahren!«,
und er fuhr wirklich auf der Stelle aus. \bibleverse{19} Als nun die
Herrschaft sah, daß es mit ihrer Hoffnung auf Geldgewinn vorbei war,
ergriffen sie den Paulus und Silas, schleppten sie auf den Marktplatz
vor die Behörde, \bibleverse{20} führten sie vor die Stadtrichter und
sagten: »Diese Menschen stören die Ruhe in unserer Stadt; sie sind Juden
\bibleverse{21} und verkünden Gebräuche, die wir als Römer nicht
annehmen und ausüben dürfen.« \bibleverse{22} Da trat die Volksmenge
gleichfalls gegen sie auf, und die Stadtrichter ließen ihnen die Kleider
vom Leibe reißen und ordneten ihre Auspeitschung an. \bibleverse{23}
Nachdem sie ihnen dann viele Stockschläge hatten verabfolgen lassen,
setzten sie sie ins Gefängnis mit der Weisung an den Gefängnisaufseher,
er solle sie in sicherem Gewahrsam halten. \bibleverse{24} Der warf sie
auf diesen Befehl hin in die innerste Zelle des Gefängnisses und schloß
ihnen die Füße in den Block ein.

\hypertarget{cc-die-bekehrung-des-gefuxe4ngnisaufsehers}{%
\subparagraph{cc) Die Bekehrung des
Gefängnisaufsehers}\label{cc-die-bekehrung-des-gefuxe4ngnisaufsehers}}

\bibleverse{25} Um Mitternacht aber beteten Paulus und Silas und priesen
Gott in Lobliedern; die übrigen Gefangenen aber hörten ihnen zu.
\bibleverse{26} Da entstand plötzlich ein starkes Erdbeben, so daß die
Grundmauern des Gefängnisses erbebten; sofort sprangen sämtliche Türen
auf, und allen fielen die Fesseln von selbst ab. \bibleverse{27} Als nun
der Gefängnisaufseher aus dem Schlaf erwachte und die Türen der
Gefängniszellen offenstehen sah, zog er sein Schwert und wollte sich das
Leben nehmen; denn er dachte, die Gefangenen seien entflohen.
\bibleverse{28} Paulus jedoch rief mit lauter Stimme: »Tu dir kein Leid
an, denn wir sind alle noch hier!« \bibleverse{29} Da rief jener nach
Licht, stürzte in die Zelle hinein und warf sich zitternd vor Paulus und
Silas nieder; \bibleverse{30} dann führte er sie hinaus und fragte sie:
»Ihr Herren, was muß ich tun, um gerettet✲ zu werden?« \bibleverse{31}
Sie antworteten: »Glaube an den Herrn Jesus, so wirst du mit deinem
Hause gerettet werden.« \bibleverse{32} Nun verkündigten sie ihm und
allen seinen Hausgenossen das Wort des Herrn. \bibleverse{33} Da nahm er
sie noch in derselben Stunde der Nacht zu sich, wusch ihnen die blutigen
Striemen ab und ließ sich mit all den Seinen sogleich taufen.
\bibleverse{34} Danach führte er sie in seine Wohnung hinauf, ließ ihnen
den Tisch decken und frohlockte mit seinem ganzen Hause, daß er zum
Glauben an Gott gekommen war.

\hypertarget{dd-die-entlassung-des-paulus-und-silas-aus-dem-gefuxe4ngnis}{%
\subparagraph{dd) Die Entlassung des Paulus und Silas aus dem
Gefängnis}\label{dd-die-entlassung-des-paulus-und-silas-aus-dem-gefuxe4ngnis}}

\bibleverse{35} Als es dann Tag geworden war, schickten die Stadtrichter
ihre Gerichtsdiener und ließen sagen: »Laß jene Männer frei!«
\bibleverse{36} Der Gefängnisaufseher teilte dem Paulus diese Botschaft
mit: »Die Stadtrichter haben sagen lassen, ihr sollt freigelassen
werden; so geht jetzt also hinaus und zieht in Frieden weiter!«
\bibleverse{37} Paulus aber entgegnete ihnen: »Sie haben uns ohne Verhör
und Urteil öffentlich auspeitschen lassen, obgleich wir römische Bürger
sind, haben uns ins Gefängnis gesetzt und wollen uns jetzt unter der
Hand ausweisen? O nein! Sie sollen selbst herkommen und uns
hinausgeleiten!« \bibleverse{38} Die Gerichtsdiener überbrachten diese
Antwort den Stadtrichtern. Diese bekamen einen Schrecken, als sie
hörten, daß es sich um römische Bürger handle; \bibleverse{39} sie kamen
also, entschuldigten sich bei ihnen und führten sie (aus dem Gefängnis)
hinaus mit der Bitte, sie möchten die Stadt verlassen. \bibleverse{40}
Da gingen sie aus dem Gefängnis hinaus und begaben sich zu Lydia,
besuchten dann die Brüder, sprachen ihnen zu und zogen weiter.

\hypertarget{d-uxfcber-thessalonike-und-beruxf6a-nach-athen}{%
\paragraph{d) Über Thessalonike und Beröa nach
Athen}\label{d-uxfcber-thessalonike-und-beruxf6a-nach-athen}}

\hypertarget{aa-paulus-in-thessalonike}{%
\subparagraph{aa) Paulus in
Thessalonike}\label{aa-paulus-in-thessalonike}}

\hypertarget{section-16}{%
\section{17}\label{section-16}}

\bibleverse{1} Nachdem sie durch Amphipolis und Apollonia gewandert
waren, kamen sie nach Thessalonike, wo es eine Synagoge der Juden gab.
\bibleverse{2} Nach seiner Gewohnheit ging Paulus zu ihnen hinein und
besprach sich an drei Sabbaten\textless sup title=``oder: drei Wochen
lang''\textgreater✲ mit ihnen auf Grund der Schriftworte, \bibleverse{3}
die er ihnen auslegte und aus denen er dartat, daß Christus\textless sup
title=``=~der Messias''\textgreater✲ leiden und von den Toten
auferstehen mußte, und (so schloß er): »Dieser Jesus, den ich euch
verkündige, ist Christus\textless sup title=``=~der
Messias''\textgreater✲.« \bibleverse{4} Einige von ihnen ließen sich
auch überzeugen und wurden für Paulus und Silas gewonnen, ebenso auch
gottesfürchtige Griechen\textless sup title=``vgl. 14,1''\textgreater✲
in großer Zahl und nicht wenige von den vornehmsten Frauen.
\bibleverse{5} Darüber wurden aber die Juden eifersüchtig, nahmen einige
schlechte Männer aus dem Straßengesindel zu Hilfe, erregten einen
Volksauflauf und brachten die Stadt in Aufruhr; dann stellten sie sich
vor dem Hause Jasons auf und suchten dort nach Paulus und Silas, um sie
dem versammelten Volke vorzuführen. \bibleverse{6} Als man sie dort aber
nicht fand, schleppten sie den Jason und einige Brüder vor die
Oberhäupter der Stadt, wobei sie schrien: »Diese Menschen, die den
ganzen Erdkreis aufgewiegelt haben, sind jetzt auch hierher gekommen:
\bibleverse{7} Jason hat sie bei sich aufgenommen, und diese Leute
verstoßen alle gegen die Verordnungen des Kaisers, denn sie behaupten,
ein anderer sei König, nämlich Jesus.« \bibleverse{8} Durch solche Reden
versetzten sie die Volksmenge und auch die Oberhäupter der Stadt in
Aufregung; \bibleverse{9} diese ließen sich von Jason und den anderen
die erforderliche Bürgschaft stellen und gaben sie dann frei.

\hypertarget{bb-erlebnisse-des-paulus-in-beruxf6a-und-seine-reise-nach-athen}{%
\subparagraph{bb) Erlebnisse des Paulus in Beröa und seine Reise nach
Athen}\label{bb-erlebnisse-des-paulus-in-beruxf6a-und-seine-reise-nach-athen}}

\bibleverse{10} Die Brüder aber veranlaßten den Paulus und Silas
sogleich noch während der Nacht dazu, nach Beröa aufzubrechen, wo sie
sich nach ihrer Ankunft in die Synagoge der Juden begaben.
\bibleverse{11} Diese waren edler gesinnt als die Juden in Thessalonike:
sie nahmen das Wort mit aller Bereitwilligkeit an und forschten Tag für
Tag in den (heiligen) Schriften, ob dies (alles) sich so verhalte.
\bibleverse{12} So wurden denn viele von ihnen gläubig, auch von den
vornehmen griechischen Frauen und Männern nicht wenige. \bibleverse{13}
Als jedoch die Juden in Thessalonike erfuhren, daß auch in Beröa das
Wort Gottes von Paulus verkündigt worden sei, kamen sie auch dorthin und
versetzten die Volksmassen in Unruhe und Aufregung. \bibleverse{14} Da
ließen die Brüder den Paulus sogleich (aus der Stadt) weggehen, damit er
sich ans Meer begäbe, während Silas und Timotheus dort (in Beröa)
zurückblieben. \bibleverse{15} Die Geleiter des Paulus aber brachten ihn
bis Athen und kehrten dann von dort wieder zurück mit dem Auftrag an
Silas und Timotheus, sie möchten möglichst bald zu ihm kommen.

\hypertarget{e-paulus-in-athen}{%
\paragraph{e) Paulus in Athen}\label{e-paulus-in-athen}}

\hypertarget{aa-beginn-seiner-arbeit}{%
\subparagraph{aa) Beginn seiner Arbeit}\label{aa-beginn-seiner-arbeit}}

\bibleverse{16} Während Paulus nun in Athen auf sie wartete, wurde er
innerlich schmerzlich erregt, weil er die Stadt voll von Götterbildern
sah. \bibleverse{17} Er besprach sich in der Synagoge mit den Juden und
den zum Judentum übergetretenen Griechen, ebenso auf dem Markte Tag für
Tag mit denen, die er dort gerade antraf. \bibleverse{18} Aber auch
einige epikureische und stoische Philosophen✲ ließen sich mit ihm ein,
und manche sagten: »Was fällt denn diesem Schwätzer ein zu behaupten?«
Andere aber meinten: »Er scheint ein Verkünder fremder Gottheiten zu
sein« -- er verkündigte nämlich die Heilsbotschaft von Jesus und von der
Auferstehung. \bibleverse{19} So nahmen sie ihn denn mit sich, führten
ihn auf den Areshügel und fragten: »Dürfen wir erfahren, was das für
eine neue Lehre ist, die du vorträgst? \bibleverse{20} Du gibst uns
seltsame Dinge zu hören; darum möchten wir gern wissen, was dahinter
steckt.« \bibleverse{21} Alle Athener nämlich und auch die dort sich
aufhaltenden Ausländer hatten für nichts anderes so viel Zeit übrig als
dafür, irgendeine Neuigkeit zu erzählen oder zu hören.

\hypertarget{bb-rede-des-paulus-auf-dem-areshuxfcgel}{%
\subparagraph{bb) Rede des Paulus auf dem
Areshügel}\label{bb-rede-des-paulus-auf-dem-areshuxfcgel}}

\bibleverse{22} So trat denn Paulus mitten auf den Areshügel und hielt
folgende Rede: »Männer von Athen! Nach allem, was ich sehe, seid ihr in
besonderem Grade eifrige Gottesverehrer. \bibleverse{23} Denn als ich
hier umherging und mir eure Heiligtümer ansah, fand ich auch einen Altar
mit der Aufschrift: ›Einem unbekannten Gott‹. Das Wesen nun, das ihr
verehrt, ohne es zu kennen, das verkündige ich euch. \bibleverse{24} Der
Gott, der die Welt und alles, was in ihr ist, geschaffen hat, er, der
Herr des Himmels und der Erde, wohnt nicht in Tempeln, die von
Menschenhand erbaut sind, \bibleverse{25} läßt sich auch nicht von
Menschenhänden bedienen, als ob er etwas bedürfte, während er doch
selbst allen Wesen Leben und Odem und alles andere gibt. \bibleverse{26}
Er hat auch gemacht, daß das ganze Menschengeschlecht von einem einzigen
(Stammvater) her auf der ganzen Oberfläche der Erde wohnt, und hat für
sie bestimmte Zeiten ihres Bestehens und auch die Grenzen ihrer
Wohnsitze festgesetzt: \bibleverse{27} sie sollten Gott suchen, ob sie
ihn wohl wahrnehmen und finden möchten, ihn, der ja nicht fern von einem
jeden unter uns ist; \bibleverse{28} denn in ihm leben wir und bewegen
wir uns und sind wir\textless sup title=``=~haben wir unser
Dasein''\textgreater✲, wie ja auch einige von euren Dichtern gesagt
haben: ›Seines Geschlechts sind auch wir.‹ \bibleverse{29} Weil wir also
göttlichen Geschlechts sind, dürfen wir nicht meinen, die Gottheit
gleiche dem Gold oder Silber oder Stein, einem Gebilde menschlicher
Kunstfertigkeit und Überlegung. \bibleverse{30} Über die (früheren)
Zeiten der Unwissenheit hat Gott zwar hinweggesehen; jetzt aber läßt er
den Menschen ansagen, daß sie alle überall Buße tun sollen\textless sup
title=``vgl. Mt 3,2''\textgreater✲; \bibleverse{31} denn er hat einen
Tag festgesetzt, an welchem er den Erdkreis mit Gerechtigkeit richten
will durch einen Mann, den er dazu ausersehen und den er für alle durch
seine Auferweckung von den Toten beglaubigt hat.«

\hypertarget{cc-schluuxdf-und-geringer-erfolg-der-rede}{%
\subparagraph{cc) Schluß und (geringer) Erfolg der
Rede}\label{cc-schluuxdf-und-geringer-erfolg-der-rede}}

\bibleverse{32} Als sie aber von einer Auferstehung der Toten hörten,
spotteten die einen, die anderen aber sagten: »Wir wollen dich hierüber
später noch einmal hören.« \bibleverse{33} So ging denn Paulus aus ihrer
Mitte hinweg. \bibleverse{34} Einige Männer jedoch schlossen sich ihm an
und kamen zum Glauben, z.B. Dionysius, ein Mitglied des
Areopags\textless sup title=``=~des obersten
Gerichtshofes''\textgreater✲, sowie eine Frau namens Damaris und noch
mehrere andere mit ihnen.

\hypertarget{f-paulus-in-korinth}{%
\paragraph{f) Paulus in Korinth}\label{f-paulus-in-korinth}}

\hypertarget{aa-seine-handwerks--und-seine-lehrtuxe4tigkeit}{%
\subparagraph{aa) Seine Handwerks- und seine
Lehrtätigkeit}\label{aa-seine-handwerks--und-seine-lehrtuxe4tigkeit}}

\hypertarget{section-17}{%
\section{18}\label{section-17}}

\bibleverse{1} Hierauf verließ Paulus Athen und begab sich nach Korinth.
\bibleverse{2} Dort traf er einen Juden namens Aquila, der aus Pontus
stammte und erst vor kurzem mit seiner Frau Priscilla aus Italien
gekommen war, weil (der Kaiser) Klaudius alle Juden aus Rom hatte
ausweisen lassen. Paulus besuchte die beiden, \bibleverse{3} und weil er
das gleiche Handwerk betrieb wie sie, blieb er bei ihnen wohnen und
arbeitete mit ihnen zusammen; sie waren nämlich nach ihrem Handwerk
Zeltmacher. \bibleverse{4} In der Synagoge aber besprach er sich an
jedem Sabbat und suchte Juden wie Griechen\textless sup title=``vgl.
14,1''\textgreater✲ zu gewinnen\textless sup title=``oder: zu
überzeugen''\textgreater✲. \bibleverse{5} Als dann Silas und Timotheus
aus Mazedonien eingetroffen waren, widmete Paulus sich ganz der
Lehrtätigkeit und bezeugte den Juden nachdrücklich, daß Jesus der
Gottgesalbte\textless sup title=``=~Christus oder: der
Messias''\textgreater✲ sei. \bibleverse{6} Weil sie aber nichts davon
wissen wollten und Lästerreden führten, schüttelte er den Staub von
seinen Kleidern ab und sagte zu ihnen: »Euer Blut komme auf euer Haupt:
ich bin unschuldig! Von nun an wende ich mich an die Heiden!«
\bibleverse{7} Damit ging er von dort weg und begab sich in das Haus
eines Heidenjuden\textless sup title=``vgl. 13,43''\textgreater✲ namens
Titius Justus, dessen Haus an die Synagoge anstieß. \bibleverse{8}
Crispus aber, der Vorsteher der Synagoge, wurde mit seinem ganzen Hause
gläubig an den Herrn, und ebenso kamen viele Korinther, die Paulus
predigen hörten, zum Glauben und ließen sich taufen. \bibleverse{9} Der
Herr aber sagte zu Paulus bei Nacht in einem Traumgesicht: »Fürchte dich
nicht, sondern rede weiter und schweige nicht; \bibleverse{10} denn ich
bin mit dir, und niemand soll sich an dir vergreifen und dir ein Leid
antun; denn ich habe ein zahlreiches Volk in dieser Stadt.«
\bibleverse{11} So blieb denn Paulus anderthalb Jahre dort und lehrte
das Wort Gottes unter ihnen.

\hypertarget{bb-die-anklage-der-juden-vom-statthalter-gallio-zuruxfcckgewiesen}{%
\subparagraph{bb) Die Anklage der Juden vom Statthalter Gallio
zurückgewiesen}\label{bb-die-anklage-der-juden-vom-statthalter-gallio-zuruxfcckgewiesen}}

\bibleverse{12} Als aber Gallio Statthalter von Griechenland (geworden)
war, traten die Juden einmütig gegen Paulus auf und führten ihn vor den
Richterstuhl (des Statthalters) \bibleverse{13} mit der Beschuldigung:
»Dieser Mensch verleitet die Leute zu einer Gottesverehrung, die gegen
unser Gesetz verstößt.« \bibleverse{14} Als Paulus sich nun dagegen
verantworten wollte, sagte Gallio zu den Juden: »Wenn irgendein
Verbrechen oder ein böswilliges Vergehen vorläge, ihr Juden, so würde
ich eure Klage selbstverständlich angenommen haben; \bibleverse{15} wenn
es sich jedoch (nur) um Streitfragen über eine Lehre und über
Benennungen und über das für euch gültige Gesetz handelt, so müßt ihr
selbst zusehen: über solche Dinge will ich nicht Richter sein.«
\bibleverse{16} Damit wies er sie von seinem Richterstuhl weg.
\bibleverse{17} Da fielen alle über Sosthenes, den Vorsteher der
Synagoge, her und verprügelten ihn vor dem Richterstuhl; Gallio aber
kümmerte sich nicht weiter darum.

\hypertarget{g-ruxfcckkehr-des-paulus-uxfcber-ephesus-und-juduxe4a-nach-antiochia-in-syrien}{%
\paragraph{g) Rückkehr des Paulus über Ephesus und Judäa nach Antiochia
in
Syrien}\label{g-ruxfcckkehr-des-paulus-uxfcber-ephesus-und-juduxe4a-nach-antiochia-in-syrien}}

\bibleverse{18} Nachdem Paulus dann noch längere Zeit (in Korinth)
geblieben war, nahm er von den Brüdern Abschied und trat die Seefahrt
nach Syrien an, und zwar zusammen mit Priscilla und Aquila, nachdem er
sich in Kenchreä (dem östlichen Hafen Korinths) das Haupt hatte scheren
lassen, weil er ein Gelübde getan\textless sup title=``oder: zu
erfüllen''\textgreater✲ hatte. \bibleverse{19} Sie kamen dann nach
Ephesus, wo Paulus sich von jenen beiden trennte; er selbst aber ging in
die Synagoge und unterredete sich mit den Juden. \bibleverse{20} Als sie
ihn aber baten, er möchte noch länger dort bleiben, ging er nicht darauf
ein, \bibleverse{21} sondern nahm Abschied von ihnen mit den Worten:
»{[}Ich muß durchaus das bevorstehende Fest in Jerusalem feiern; aber{]}
so Gott will, werde ich später zu euch zurückkehren.« Dann fuhr er zu
Schiff von Ephesus ab, \bibleverse{22} landete in Cäsarea, ging (nach
Jerusalem) hinauf, wo er die Gemeinde begrüßte, und zog dann nach
Antiochia hinab.

\hypertarget{dritte-bekehrungsreise-des-paulus-1823-2114}{%
\subsubsection{4. Dritte Bekehrungsreise des Paulus
(18,23-21,14)}\label{dritte-bekehrungsreise-des-paulus-1823-2114}}

\hypertarget{a-antritt-der-reise-apollos-in-ephesus-und-korinth}{%
\paragraph{a) Antritt der Reise; Apollos in Ephesus und
Korinth}\label{a-antritt-der-reise-apollos-in-ephesus-und-korinth}}

\bibleverse{23} Nachdem er dort einige Zeit zugebracht hatte, brach er
wieder auf, durchwanderte von einem Ort zum anderen das galatische Land
und Phrygien und stärkte überall die Jünger✲ durch Zuspruch.

\bibleverse{24} Inzwischen war ein Jude namens Apollos, der aus
Alexandria stammte, ein gelehrter\textless sup title=``oder:
redegewandter?''\textgreater✲ Mann, der in den (heiligen) Schriften
außerordentlich bewandert war, nach Ephesus gekommen. \bibleverse{25} Er
hatte Unterweisung über den Weg\textless sup title=``d.h. in der
Lehre''\textgreater✲ des Herrn erhalten, redete mit glühender
Begeisterung und trug das auf Jesus Bezügliche richtig vor, obgleich er
nur von der Taufe des Johannes wußte. \bibleverse{26} Dieser Mann fing
dann auch an, in der Synagoge freimütig zu reden. Als Priscilla und
Aquila ihn gehört hatten, traten sie mit ihm in Verbindung und setzten
ihm die Lehre Gottes noch genauer auseinander. \bibleverse{27} Als er
dann nach Griechenland hinüberzureisen wünschte, bestärkten die Brüder
ihn in dieser Absicht und schrieben an die Jünger (in Korinth), sie
möchten ihn freundlich aufnehmen. Nach seiner Ankunft leistete er denen,
die gläubig geworden waren, durch seine Gnadengabe\textless sup
title=``oder: Begabung''\textgreater✲ die erfreulichsten Dienste;
\bibleverse{28} denn in schlagender Weise widerlegte er die Juden
öffentlich, indem er aus den (heiligen) Schriften nachwies, daß Jesus
der Messias sei.

\hypertarget{b-wirksamkeit-und-erlebnisse-des-paulus-in-ephesus}{%
\paragraph{b) Wirksamkeit und Erlebnisse des Paulus in
Ephesus}\label{b-wirksamkeit-und-erlebnisse-des-paulus-in-ephesus}}

\hypertarget{aa-bekehrung-und-taufe-der-johannesjuxfcnger}{%
\subparagraph{aa) Bekehrung und Taufe der
Johannesjünger}\label{aa-bekehrung-und-taufe-der-johannesjuxfcnger}}

\hypertarget{section-18}{%
\section{19}\label{section-18}}

\bibleverse{1} Während nun Apollos sich in Korinth aufhielt, kam Paulus,
nachdem er das Binnenland von Kleinasien durchwandert hatte, nach
Ephesus und fand dort einige Jünger vor. \bibleverse{2} Er fragte diese:
»Habt ihr den heiligen Geist empfangen, nachdem ihr gläubig geworden
waret?« Sie antworteten ihm: »Nein, wir haben überhaupt noch nichts
davon gehört, ob der heilige Geist (schon) da ist.« \bibleverse{3}
Darauf fragte er sie: »Worauf seid ihr denn getauft worden?« Sie
antworteten: »Auf die Taufe des Johannes.« \bibleverse{4} Da sagte
Paulus: »Johannes hat (nur) eine Bußtaufe vollzogen und dabei dem Volke
geboten, sie sollten an den glauben, der nach ihm kommen würde, nämlich
an Jesus.« \bibleverse{5} Als sie das hörten, ließen sie sich auf den
Namen des Herrn Jesus taufen; \bibleverse{6} und als Paulus ihnen dann
die Hände auflegte, kam der heilige Geist auf sie, und sie redeten mit
Zungen und sprachen prophetisch. \bibleverse{7} Es waren dies im ganzen
etwa zwölf Männer.

\hypertarget{bb-die-zweijuxe4hrige-lehr--und-wundertuxe4tigkeit-des-paulus-in-ephesus}{%
\subparagraph{bb) Die zweijährige Lehr- und Wundertätigkeit des Paulus
in
Ephesus}\label{bb-die-zweijuxe4hrige-lehr--und-wundertuxe4tigkeit-des-paulus-in-ephesus}}

\bibleverse{8} Paulus ging dann in die Synagoge und trat dort ein
Vierteljahr lang mit Freimut auf, indem er sich besprach und sie für das
Reich Gottes zu gewinnen suchte. \bibleverse{9} Als manche jedoch
verstockt und unzugänglich blieben und die (neue) Lehre\textless sup
title=``vgl. 18,26''\textgreater✲ vor der versammelten Menge schmähten,
sagte er sich von ihnen los, sonderte auch die Jünger von ihnen ab und
hielt nun täglich seine Vorträge\textless sup title=``oder:
Besprechungen''\textgreater✲ im Hörsaal eines gewissen Tyrannus.
\bibleverse{10} Das ging so zwei Jahre lang fort, so daß alle Bewohner
der Provinz Asien das Wort des Herrn zu hören bekamen, Juden sowohl wie
Griechen\textless sup title=``vgl. 14,1''\textgreater✲.~--
\bibleverse{11} Auch ungewöhnliche Wunder ließ Gott durch die Hände des
Paulus geschehen, \bibleverse{12} so daß man sogar Schweißtücher oder
Schürzen, die er (bei der Arbeit) an seinem Leibe getragen hatte, zu den
Kranken brachte, worauf dann die Krankheiten von ihnen wichen und die
bösen Geister ausfuhren.

\hypertarget{cc-die-uxfcberwindung-des-aberglaubens-der-beschwuxf6rer-und-zauberbuxfccher}{%
\subparagraph{cc) Die Überwindung des Aberglaubens (der Beschwörer und
Zauberbücher)}\label{cc-die-uxfcberwindung-des-aberglaubens-der-beschwuxf6rer-und-zauberbuxfccher}}

\bibleverse{13} Nun unterfingen sich aber auch einige von den
umherziehenden jüdischen Beschwörern, über Personen, die von bösen
Geistern besessen waren, den Namen des Herrn Jesus auszusprechen, indem
sie sagten: »Ich beschwöre euch bei dem Jesus, den Paulus predigt!«
\bibleverse{14} Es waren besonders sieben Söhne eines gewissen Skeuas,
eines Juden aus hohepriesterlichem Geschlecht, die das taten.
\bibleverse{15} Der böse Geist aber gab ihnen zur Antwort: »Jesus kenne
ich wohl, und auch Paulus ist mir bekannt; doch wer seid ihr?«
\bibleverse{16} Hierauf sprang der Mensch, in welchem der böse Geist
war, auf sie los, überwältigte beide und richtete sie so zu, daß sie
unbekleidet und blutig geschlagen aus jenem Hause entflohen.
\bibleverse{17} Dieses Vorkommnis wurde alsdann allen Juden und Griechen
bekannt, die in Ephesus wohnten, und Furcht befiel sie alle; der Name
des Herrn Jesus aber wurde hoch gepriesen. \bibleverse{18} Ebenso kamen
auch viele von denen, die gläubig geworden waren, und bekannten offen
und unverhohlen ihr früheres Treiben; \bibleverse{19} ja nicht wenige
von denen, die sich mit Zauberei abgegeben hatten, brachten die
Zauberbücher auf einen Haufen zusammen und verbrannten sie öffentlich.
Als man ihre Preise\textless sup title=``=~ihren Wert''\textgreater✲
zusammenrechnete, kam der Betrag von fünfzigtausend Drachmen heraus.
\bibleverse{20} So breitete sich das Wort des Herrn unaufhaltsam aus und
wurde immer stärker.

\hypertarget{dd-reisepluxe4ne-des-paulus}{%
\subparagraph{dd) Reisepläne des
Paulus}\label{dd-reisepluxe4ne-des-paulus}}

\bibleverse{21} Als dieses glücklich ausgeführt war, entschloß sich
Paulus dazu, Mazedonien und Griechenland zu durchwandern und sich dann
nach Jerusalem zu begeben, wobei er erklärte: »Nachdem ich dort gewesen
bin, muß ich auch Rom sehen.« \bibleverse{22} So sandte er denn zwei von
seinen Gehilfen, Timotheus und Erastus, nach Mazedonien ab, während er
selbst noch eine Zeitlang in der Provinz Asien verblieb.

\hypertarget{ee-der-aufruhr-der-silberarbeiter-des-demetrius}{%
\subparagraph{ee) Der Aufruhr der Silberarbeiter des
Demetrius}\label{ee-der-aufruhr-der-silberarbeiter-des-demetrius}}

\bibleverse{23} Um diese Zeit aber kam es (in Ephesus) zu großen Unruhen
wegen des Weges\textless sup title=``d.h. der christlichen Lehre; vgl.
18,25''\textgreater✲. \bibleverse{24} Ein Silberschmied nämlich,
Demetrius mit Namen, der silberne Tempel der Artemis✲ verfertigte und
den Handwerkern dadurch viel zu verdienen gab, \bibleverse{25} berief
diese und die hierbei beschäftigten Arbeiter zu einer Versammlung und
sprach sich so aus: »Ihr Männer, ihr wißt, daß wir unsern Wohlstand
diesem unserm Gewerbe verdanken. \bibleverse{26} Nun seht und hört ihr
aber, daß dieser Paulus nicht nur hier in Ephesus, sondern beinahe in
der ganzen Provinz Asien viele Leute durch sein Gerede betört hat, indem
er ihnen vorhält, das seien keine Götter, die von Menschenhänden
angefertigt würden. \bibleverse{27} Aber nicht nur dieser unser
Erwerbszweig droht in Mißachtung zu kommen\textless sup title=``oder:
Einbuße zu erleiden''\textgreater✲, sondern auch der Tempel der großen
Göttin Artemis schwebt in Gefahr, in völlige Mißachtung zu geraten; ja
es ist zu befürchten, daß sie sogar ihres hohen Ruhmes ganz verlustig
geht, während sie jetzt doch von ganz Asien, ja von aller Welt verehrt
wird.«

\bibleverse{28} Als sie das hörten, gerieten sie in volle Wut und riefen
laut: »Groß ist die Artemis✲ von Ephesus!« \bibleverse{29} Die ganze
Stadt geriet in Aufruhr, und alle stürmten einmütig ins Theater, wohin
sie auch die Mazedonier Gajus und Aristarchus, die Reisegefährten des
Paulus, mitschleppten. \bibleverse{30} Als Paulus nun (selbst) vor die
Volksmenge treten wollte, ließen die Jünger✲ es ihm nicht zu;
\bibleverse{31} auch einige von den obersten Beamten der Provinz Asien,
die seine guten Freunde waren, schickten zu ihm und ließen ihm die
Mahnung zugehen, er möchte sich nicht ins Theater begeben.
\bibleverse{32} Dort schrie nun alles wild durcheinander; denn die
Versammlung war ein Wirrwarr; die meisten wußten überhaupt nicht,
weswegen man zusammengekommen war. \bibleverse{33} Da verständigte man
aus der Volksmenge heraus den Alexander, den die Juden vorschoben.
Dieser Alexander gab auch ein Zeichen mit der Hand und wollte eine
Verteidigungsrede an das Volk richten; \bibleverse{34} als man aber
merkte\textless sup title=``oder: erfuhr''\textgreater✲, daß er ein Jude
sei, erscholl von allen wie aus einem Munde etwa zwei Stunden lang der
Ruf: »Groß ist die Artemis von Ephesus!«

\bibleverse{35} Endlich brachte der Stadtschreiber die Menge zur Ruhe
und sagte: »Ihr Männer von Ephesus! Wo gibt es wohl in der ganzen Welt
einen Menschen, der nicht wüßte, daß die Stadt Ephesus die Tempelhüterin
der großen Artemis und ihres vom Himmel herabgefallenen Bildes ist?
\bibleverse{36} Da dies also eine unbestreitbare Tatsache ist, solltet
ihr euch ja ruhig verhalten und nichts Übereiltes tun. \bibleverse{37}
Ihr habt ja doch diese Männer hierher gebracht, die weder Tempelräuber
sind noch unsere Göttin lästern. \bibleverse{38} Wenn nun Demetrius und
die Zunft der Kunsthandwerker mit ihm Grund zu einer Klage gegen jemand
haben, nun, so werden ja Gerichtstage abgehalten, und es gibt
Statthalter; dort mögen sie ihre Sache miteinander abmachen!
\bibleverse{39} Habt ihr aber außerdem noch Wünsche\textless sup
title=``oder: Anliegen''\textgreater✲, so wird das in der ordentlichen
Volksversammlung erledigt werden. \bibleverse{40} Droht uns doch wegen
der heutigen Vorkommnisse sogar eine Anklage wegen Aufruhrs, weil kein
Grund vorliegt, mit dem wir diesen Aufruhr rechtfertigen könnten.«
\bibleverse{41} Durch diese Worte brachte er die Versammlung zum
Auseinandergehen.

\hypertarget{c-reise-des-paulus-durch-mazedonien-nach-griechenland-und-zuruxfcck-nach-mazedonien-troas-und-milet}{%
\paragraph{c) Reise des Paulus durch Mazedonien nach Griechenland und
zurück nach Mazedonien, Troas und
Milet}\label{c-reise-des-paulus-durch-mazedonien-nach-griechenland-und-zuruxfcck-nach-mazedonien-troas-und-milet}}

\hypertarget{aa-reise-nach-griechenland-und-ruxfcckkehr-nach-troas}{%
\subparagraph{aa) Reise nach Griechenland und Rückkehr nach
Troas}\label{aa-reise-nach-griechenland-und-ruxfcckkehr-nach-troas}}

\hypertarget{section-19}{%
\section{20}\label{section-19}}

\bibleverse{1} Als sich nun die Unruhe gelegt hatte, ließ Paulus die
Jünger✲ zu sich kommen, hielt eine ermahnende Ansprache an sie, nahm
dann Abschied von ihnen und trat die Reise nach Mazedonien an.
\bibleverse{2} Nachdem er diese Gegenden durchzogen und (den dortigen
Gläubigen) reichen Zuspruch gespendet hatte, begab er sich nach
Griechenland. \bibleverse{3} Als er sich dann nach einem Aufenthalt von
drei Monaten nach Syrien einschiffen wollte und die Juden einen Anschlag
gegen ihn planten, entschloß er sich zur Rückkehr durch Mazedonien.
\bibleverse{4} Auf der Reise begleiteten ihn (bis nach Kleinasien):
Sopater, der Sohn des Pyrrhus aus Beröa, ferner von den Thessalonikern
Aristarchus und Sekundus, weiter aus Derbe Gajus und Timotheus; außerdem
aus der Provinz Asien Tychikus und Trophimus. \bibleverse{5} Diese
(letzten beiden) jedoch reisten uns voraus\textless sup title=``=~kamen
erst später hinzu''\textgreater✲ und erwarteten uns in Troas;
\bibleverse{6} wir selbst dagegen fuhren nach den Tagen der ungesäuerten
Brote zu Schiff von Philippi ab und kamen fünf Tage später zu ihnen nach
Troas, wo wir sieben Tage blieben.

\hypertarget{bb-abschiedsfeier-des-paulus-in-troas-wiederbelebung-des-verungluxfcckten-eutychus}{%
\subparagraph{bb) Abschiedsfeier des Paulus in Troas; Wiederbelebung des
verunglückten
Eutychus}\label{bb-abschiedsfeier-des-paulus-in-troas-wiederbelebung-des-verungluxfcckten-eutychus}}

\bibleverse{7} Als wir uns nun am ersten Tage nach dem
Sabbat\textless sup title=``oder: am ersten Tage der
Woche''\textgreater✲ versammelt hatten, um das Brot zu brechen, besprach
sich Paulus mit ihnen, weil er am folgenden Tage abreisen wollte, und
dehnte die Unterredung bis Mitternacht aus. \bibleverse{8} Zahlreiche
Lampen brannten in dem Obergemach, in dem wir versammelt waren.
\bibleverse{9} Da wurde ein Jüngling namens Eutychus, der im (offenen)
Fenster\textless sup title=``=~auf der Fensterbank''\textgreater✲ saß,
von tiefem Schlaf überwältigt, weil Paulus so lange fortredete; er
stürzte dann im Schlaf vom dritten Stockwerk hinab und wurde tot
aufgehoben. \bibleverse{10} Paulus aber ging hinunter, warf sich über
ihn, schlang die Arme um ihn und sagte: »Beunruhigt euch nicht! Seine
Seele\textless sup title=``=~das Leben''\textgreater✲ ist (wieder) in
ihm.« \bibleverse{11} Als er dann wieder hinaufgegangen war und das Brot
gebrochen hatte, nahm er einen Imbiß und unterredete sich noch lange
weiter mit ihnen, bis der Tag anbrach; dann erst machte er sich auf den
Weg. \bibleverse{12} Den Knaben✲ aber hatte man lebend weggetragen,
wodurch alle sich nicht wenig getröstet fühlten.

\hypertarget{cc-reise-des-paulus-von-troas-bis-milet}{%
\subparagraph{cc) Reise des Paulus von Troas bis
Milet}\label{cc-reise-des-paulus-von-troas-bis-milet}}

\bibleverse{13} Wir (anderen) waren unterdessen auf das Schiff
vorausgegangen und fuhren auf Assos zu in der Absicht, dort Paulus an
Bord zu nehmen; denn so hatte er es angeordnet, weil er selbst den Weg
dorthin zu Fuß machen wollte. \bibleverse{14} Als er dann in Assos mit
uns wieder zusammengetroffen war, nahmen wir ihn an Bord und gelangten
nach Mitylene. \bibleverse{15} Von dort fuhren wir weiter und kamen am
folgenden Tage auf die Höhe von Chios; tags darauf legten wir in Samos
an und gelangten {[}nach einem Aufenthalt in Trogyllion{]} am nächsten
Tage nach Milet. \bibleverse{16} Paulus hatte sich nämlich entschlossen,
an Ephesus vorüberzufahren, um keine Zeit mehr in der Provinz Asien zu
verlieren; denn er beeilte sich, um womöglich am Tage des Pfingstfestes
in Jerusalem zu sein.

\hypertarget{d-zusammenkunft-des-paulus-mit-den-uxe4ltesten-von-ephesus-in-milet-seine-abschiedsrede-und-sein-scheiden}{%
\paragraph{d) Zusammenkunft des Paulus mit den Ältesten von Ephesus in
Milet; seine Abschiedsrede und sein
Scheiden}\label{d-zusammenkunft-des-paulus-mit-den-uxe4ltesten-von-ephesus-in-milet-seine-abschiedsrede-und-sein-scheiden}}

\bibleverse{17} Von Milet aus aber sandte er Botschaft nach Ephesus und
ließ die Ältesten der Gemeinde zu sich rufen. \bibleverse{18} Als sie
sich bei ihm eingefunden hatten, richtete er folgende Ansprache an sie:
»Ihr wißt selbst, wie ich mich vom ersten Tage ab, an dem ich die
Provinz Asien betreten hatte, die ganze Zeit hindurch bei euch verhalten
\bibleverse{19} und dem Herrn gedient habe mit aller Demut und unter
Tränen und Anfechtungen, die mir aus den Nachstellungen der Juden
erwuchsen, \bibleverse{20} wie ich durchaus nichts verabsäumt habe, um
euch alles, was euch heilsam sein konnte, öffentlich und in den Häusern
zu verkündigen und zu lehren, \bibleverse{21} indem ich es sowohl Juden
als auch Griechen ans Herz legte, sich zu Gott zu bekehren und an unsern
Herrn Jesus Christus zu glauben. \bibleverse{22} Und jetzt, seht: im
Geist gebunden, reise ich nach Jerusalem, ohne zu wissen, was mir dort
widerfahren wird; \bibleverse{23} nur das bezeugt mir der heilige Geist
in jeder Stadt mit Bestimmtheit, daß Gefangenschaft und Leiden auf mich
warten. \bibleverse{24} Doch ich sehe das Leben als für mich selbst
völlig wertlos an, wenn ich nur meinen Lauf {[}mit Freuden{]} vollende
und den Dienst (zum Abschluß bringe), den ich vom Herrn Jesus empfangen
habe, nämlich Zeugnis für die Heilsbotschaft von der Gnade Gottes
abzulegen. \bibleverse{25} Und jetzt, seht: ich weiß, daß ihr mein
Angesicht nicht wiedersehen werdet, ihr alle, unter denen ich mich als
Prediger des Reiches (Gottes) bewegt habe. \bibleverse{26} Darum gebe
ich euch am heutigen Tage die feste Versicherung, daß ich den Tod
niemandes auf dem Gewissen habe; \bibleverse{27} denn ich habe es an mir
nicht fehlen lassen, euch den ganzen Ratschluß\textless sup
title=``oder: Heilsplan''\textgreater✲ Gottes zu verkündigen.
\bibleverse{28} So gebt denn acht auf euch selbst und auf die ganze
Herde, bei welcher der heilige Geist euch zu Aufsehern✲ bestellt hat,
damit ihr die Gemeinde des Herrn weidet, die er sich durch sein eigenes
Blut erworben hat. \bibleverse{29} Ich weiß, daß nach meinem Weggang
schlimme\textless sup title=``=~verderbliche, reißende''\textgreater✲
Wölfe bei euch einbrechen und die Herde nicht verschonen werden;
\bibleverse{30} ja aus eurer eigenen Mitte werden Männer auftreten und
Irrlehren vortragen, um die Jünger in ihre Gefolgschaft zu ziehen.
\bibleverse{31} Darum seid wachsam und bleibt dessen eingedenk, daß ich
drei Jahre hindurch Tag und Nacht nicht aufgehört habe, jeden einzelnen
(von euch) unter Tränen zu ermahnen. \bibleverse{32} Und nunmehr befehle
ich euch Gott und dem Wort seiner Gnade, das die Kraft besitzt,
aufzubauen und das Erbe zu verleihen unter allen, die sich haben
heiligen lassen. \bibleverse{33} Silber, Gold und Kleidung habe ich von
niemand begehrt; \bibleverse{34} ihr wißt selbst, daß für meinen
Lebensunterhalt und auch für meine Begleiter\textless sup title=``oder:
Gefährten''\textgreater✲ diese (meine) Hände gesorgt haben.
\bibleverse{35} Immer und überall habe ich euch gezeigt, daß man in
solcher Weise arbeiten und sich der Schwachen annehmen und dabei der
Worte des Herrn Jesus eingedenk sein muß; denn er hat selbst gesagt:
›Geben ist seliger als Nehmen.‹« \bibleverse{36} Nach diesen Worten
kniete er mit ihnen allen nieder und betete. \bibleverse{37} Da brachen
alle in lautes Wehklagen aus, fielen dem Paulus um den Hals und küßten
ihn; \bibleverse{38} am schmerzlichsten war für sie das Wort, das er
ausgesprochen hatte, sie würden sein Angesicht nicht wiedersehen. Sie
gaben ihm darauf das Geleit bis zum Schiff.

\hypertarget{e-weiterreise-von-milet-bis-tyrus-und-cuxe4sarea}{%
\paragraph{e) Weiterreise von Milet bis Tyrus und
Cäsarea}\label{e-weiterreise-von-milet-bis-tyrus-und-cuxe4sarea}}

\hypertarget{section-20}{%
\section{21}\label{section-20}}

\bibleverse{1} Als wir uns dann von ihnen losgerissen hatten und wieder
in See gegangen waren, kamen wir in gerader Fahrt nach Kos, am nächsten
Tage nach Rhodus und von dort nach Patara. \bibleverse{2} Als wir dort
ein Schiff fanden, das nach Phönizien bestimmt war, stiegen wir ein und
fuhren ab. \bibleverse{3} Wir bekamen Cypern in Sicht, das wir aber zur
Linken liegen ließen, steuerten auf Syrien zu und legten in Tyrus an;
denn dort hatte das Schiff seine Ladung zu löschen. \bibleverse{4} Wir
suchten nun die Jünger auf und blieben sieben Tage dort; jene warnten
den Paulus auf Eingebung des Geistes wiederholt vor der Reise nach
Jerusalem. \bibleverse{5} Als wir aber die Tage dort verlebt hatten,
machten wir uns zur Weiterfahrt auf den Weg, wobei alle (Brüder) samt
Frauen und Kindern uns das Geleit bis vor die Stadt hinaus gaben. Am
Strande knieten wir nieder und beteten; \bibleverse{6} dann nahmen wir
Abschied voneinander und gingen an Bord, während jene wieder
heimkehrten.

\bibleverse{7} Wir aber legten den letzten Teil unserer Fahrt zurück,
indem wir von Tyrus nach Ptolemais segelten; wir begrüßten auch hier die
Brüder, blieben aber nur einen Tag bei ihnen. \bibleverse{8} Am nächsten
Morgen zogen wir weiter und kamen nach Cäsarea, wo wir bei dem
Evangelisten Philippus, einem der sieben\textless sup
title=``Armenpfleger; vgl. 6,5; 8,5-40''\textgreater✲, einkehrten und
bei ihm blieben. \bibleverse{9} Dieser hatte vier unverheiratete
Töchter, welche Prophetengabe besaßen. \bibleverse{10} Während unseres
mehrtägigen Aufenthalts (in Cäsarea) kam ein Prophet namens Agabus aus
Judäa herab \bibleverse{11} und besuchte uns, er nahm den Gürtel des
Paulus, band sich Hände und Füße damit und sagte dann: »So spricht der
heilige Geist: ›Den Mann, dem dieser Gürtel gehört, werden die Juden in
Jerusalem in dieser Weise binden und ihn den Heiden in die Hände
liefern.‹« \bibleverse{12} Als wir das hörten, baten wir und die
Einheimischen den Paulus inständig, er möchte nicht nach Jerusalem
hinaufgehen. \bibleverse{13} Da antwortete Paulus: »Was weint ihr so und
macht mir das Herz schwer? Ich bin ja bereit, mich in Jerusalem nicht
nur binden zu lassen, sondern auch den Tod für den Namen des Herrn Jesus
zu erleiden!« \bibleverse{14} Weil er sich nun nicht umstimmen ließ,
beruhigten wir uns\textless sup title=``oder: hörten wir mit unseren
Vorstellungen auf''\textgreater✲ und sagten: »Des Herrn Wille geschehe!«

\hypertarget{b.-reise-des-paulus-von-cuxe4sarea-nach-jerusalem-seine-gefangenschaft-2115-2831}{%
\subsection{B. Reise des Paulus von Cäsarea nach Jerusalem; seine
Gefangenschaft
(21,15-28,31)}\label{b.-reise-des-paulus-von-cuxe4sarea-nach-jerusalem-seine-gefangenschaft-2115-2831}}

\hypertarget{paulus-in-jerusalem-und-als-gefangener-in-cuxe4sarea-2115-2632}{%
\subsubsection{1. Paulus in Jerusalem und als Gefangener in Cäsarea
(21,15-26,32)}\label{paulus-in-jerusalem-und-als-gefangener-in-cuxe4sarea-2115-2632}}

\hypertarget{a-ankunft-in-jerusalem-zusammenkunft-mit-den-aposteln-beihilfe-zur-erfuxfcllung-eines-nasiruxe4atsgeluxfcbdes}{%
\paragraph{a) Ankunft in Jerusalem; Zusammenkunft mit den Aposteln;
Beihilfe zur Erfüllung eines
Nasiräatsgelübdes}\label{a-ankunft-in-jerusalem-zusammenkunft-mit-den-aposteln-beihilfe-zur-erfuxfcllung-eines-nasiruxe4atsgeluxfcbdes}}

\bibleverse{15} Nach Ablauf dieser Tage machten wir uns reisefertig und
zogen nach Jerusalem hinauf. \bibleverse{16} Dabei begleiteten uns auch
einige Jünger✲ aus Cäsarea und brachten uns zu einem gewissen Mnason aus
Cypern, einem alten Jünger, bei dem wir als Gäste wohnen sollten.
\bibleverse{17} Nach unserer Ankunft in Jerusalem nahmen uns die Brüder
mit Freuden auf. \bibleverse{18} Gleich am folgenden Tage ging Paulus
mit uns zu Jakobus, und auch alle Ältesten fanden sich dort ein.
\bibleverse{19} Nachdem Paulus sie begrüßt hatte, erzählte er ihnen
alles im einzelnen, was Gott unter den Heiden durch seine Arbeit
vollbracht hatte.

\bibleverse{20} Als sie das gehört hatten, priesen sie Gott, sagten aber
zu ihm: »Du siehst, lieber Bruder, wie viele Tausende es unter den Juden
gibt, die gläubig geworden sind; doch alle sind sie eifrige Anhänger des
(mosaischen) Gesetzes. \bibleverse{21} Nun ist ihnen aber über dich
berichtet worden, daß du allen Juden, die unter den Heiden leben, den
Abfall von Mose predigest und ihnen empfehlest, sie möchten ihre Kinder
nicht beschneiden lassen und überhaupt die herkömmlichen Gebräuche nicht
mehr beobachten. \bibleverse{22} Was ist da nun zu tun? Jedenfalls
werden sie von deinem Hiersein erfahren. \bibleverse{23} Tu also, was
wir dir raten! Wir haben hier (gerade) vier Männer unter uns, die ein
Gelübde auf sich genommen\textless sup title=``=~zu
erfüllen''\textgreater✲ haben; \bibleverse{24} nimm diese mit dir, laß
dich mit ihnen reinigen und bezahle für sie (die zu entrichtenden
Gebühren), damit sie sich das Haupt scheren lassen können\textless sup
title=``oder: dürfen''\textgreater✲; dann werden alle einsehen, daß an
den Gerüchten, die ihnen über dich zu Ohren gekommen sind, nichts Wahres
ist, daß vielmehr auch du in der Beobachtung des Gesetzes wandelst.
\bibleverse{25} Was aber die gläubig gewordenen Heiden betrifft, so
haben wir beschlossen und ihnen (schriftlich) mitgeteilt, daß sie sich
vor Götzenopferfleisch, vor (dem Genuß von) Blut, vor dem Fleisch
erstickter Tiere und vor Unzucht zu hüten haben.« \bibleverse{26}
Daraufhin nahm Paulus die (betreffenden) Männer mit sich, ließ sich am
folgenden Tage reinigen und ging mit ihnen in den Tempel, wo er den
Abschluß der Reinigungstage anmeldete, (die so lange dauerten) bis für
einen jeden von ihnen das Löseopfer dargebracht sein würde.

\hypertarget{b-gefangennahme-des-paulus}{%
\paragraph{b) Gefangennahme des
Paulus}\label{b-gefangennahme-des-paulus}}

\hypertarget{aa-paulus-im-tempel-von-den-juden-festgenommen-der-aufstand-in-jerusalem}{%
\subparagraph{aa) Paulus im Tempel von den Juden festgenommen; der
Aufstand in
Jerusalem}\label{aa-paulus-im-tempel-von-den-juden-festgenommen-der-aufstand-in-jerusalem}}

\bibleverse{27} Als aber die sieben Tage (der Reinigungszeit) nahezu
abgelaufen waren, erblickten ihn die Juden, die aus der Provinz Asien
gekommen waren, im Tempel und brachten die ganze Volksmenge in Aufruhr;
sie nahmen ihn fest \bibleverse{28} und riefen laut: »Ihr Männer von
Israel, kommt uns zu Hilfe! Dies ist der Mensch, der überall vor allen
Leuten seine Lehre gegen unser Volk und gegen das Gesetz und gegen diese
Stätte vorträgt! Dazu hat er jetzt auch noch Griechen in den Tempel
hineingebracht und dadurch diese heilige Stätte entweiht!«
\bibleverse{29} Sie hatten nämlich vorher den Trophimus aus Ephesus in
der Stadt mit ihm zusammen gesehen und meinten nun, Paulus habe ihn in
den Tempel mitgenommen. \bibleverse{30} So geriet denn die ganze Stadt
in Bewegung, und es entstand ein Volksauflauf; man ergriff Paulus und
schleppte ihn aus dem Tempel hinaus, worauf dessen Tore sogleich
geschlossen wurden.

\hypertarget{bb-gefangennahme-des-paulus-durch-den-ruxf6mischen-obersten-lysias}{%
\subparagraph{bb) Gefangennahme des Paulus durch den römischen Obersten
Lysias}\label{bb-gefangennahme-des-paulus-durch-den-ruxf6mischen-obersten-lysias}}

\bibleverse{31} Während man nun darauf ausging, ihn totzuschlagen,
gelangte an den Obersten der römischen Abteilung die Meldung hinauf,
ganz Jerusalem sei in Aufruhr. \bibleverse{32} Dieser nahm (daher)
sofort Mannschaften und Hauptleute✲ mit sich und eilte zu ihnen hinab.
Als jene nun den Obersten und die Soldaten sahen, hörten sie auf, Paulus
zu schlagen. \bibleverse{33} Da trat der Oberst heran, bemächtigte sich
seiner, ließ ihn in zwei Ketten legen und fragte, wer er sei und was er
getan habe. \bibleverse{34} Da schrien alle in der Volksmenge
durcheinander. Weil er nun wegen des Lärms nichts Sicheres ermitteln
konnte, gab er Befehl, man solle Paulus in die Burg führen.
\bibleverse{35} Als Paulus aber an die Treppe (zur Burg hinauf) gelangt
war, mußte er wegen des gewaltsamen Andrangs der Menge von den Soldaten
getragen werden; \bibleverse{36} denn die Volksmenge zog mit unter dem
lauten Ruf: »Nieder mit ihm!« \bibleverse{37} Als nun Paulus eben in die
Burg hineingeführt werden sollte, fragte er den Obersten: »Darf ich dir
etwas sagen?« Jener erwiderte: »Du kannst Griechisch? \bibleverse{38} Da
bist du also nicht der Ägypter, der vor einiger Zeit den Aufruhr erregt
und die viertausend Mann Banditen\textless sup title=``oder:
Straßenräuber, Meuchelmörder''\textgreater✲ in die Wüste hinausgeführt
hat?« \bibleverse{39} Paulus antwortete: »Nein, ich bin ein Jude aus
Tarsus, Bürger einer namhaften Stadt in Cilicien. Erlaube mir, bitte,
zum Volke zu reden!« \bibleverse{40} Als jener ihm die Erlaubnis gegeben
hatte, gab Paulus, auf der Treppe stehend, dem Volk ein Zeichen mit der
Hand; als dann völlige Stille eingetreten war, hielt er in der
hebräischen Landessprache folgende Ansprache an sie:

\hypertarget{cc-rede-des-paulus-an-das-volk-bericht-uxfcber-seine-bekehrung-und-uxfcber-den-von-jesus-empfangenen-auftrag}{%
\subparagraph{cc) Rede des Paulus an das Volk (Bericht über seine
Bekehrung und über den von Jesus empfangenen
Auftrag)}\label{cc-rede-des-paulus-an-das-volk-bericht-uxfcber-seine-bekehrung-und-uxfcber-den-von-jesus-empfangenen-auftrag}}

\hypertarget{section-21}{%
\section{22}\label{section-21}}

\bibleverse{1} »Werte Brüder und Väter, hört jetzt meine Rechtfertigung
vor euch an!« \bibleverse{2} Als sie nun hörten, daß er in hebräischer
Sprache zu ihnen redete, verhielten sie sich noch ruhiger; und er fuhr
fort: \bibleverse{3} »Ich bin ein Jude, geboren zu Tarsus in Cilicien,
aber hier in dieser Stadt erzogen: zu den Füßen Gamaliels habe ich meine
Ausbildung in strenger Befolgung des Gesetzes unserer Väter erhalten und
bin ein ebensolcher Eiferer für Gott gewesen, wie ihr alle es noch heute
seid. \bibleverse{4} Als solcher habe ich auch diese
Glaubensrichtung\textless sup title=``oder: neue Lehre''\textgreater✲
bis auf den Tod verfolgt, indem ich Männer wie Frauen in Ketten legte
und ins Gefängnis werfen ließ, \bibleverse{5} wie mir das auch der
Hohepriester und der gesamte Rat der Ältesten bezeugen können. Von
diesen habe ich mir sogar Briefe\textless sup title=``=~schriftliche
Vollmachten''\textgreater✲ an unsere Volksgenossen geben lassen und mich
nach Damaskus begeben, um auch die Leute dort gefesselt zur Bestrafung
nach Jerusalem zu bringen. \bibleverse{6} Da geschah es, als ich mich
auf dem Wege dorthin befand und in die Nähe von Damaskus gekommen war,
daß mich zur Mittagszeit plötzlich ein helles Licht vom Himmel her
umstrahlte. \bibleverse{7} Ich stürzte zu Boden und hörte eine Stimme,
die mir zurief: ›Saul, Saul! Was verfolgst du mich?‹ \bibleverse{8} Ich
antwortete: ›Wer bist du, Herr?‹ Er sagte zu mir: ›Ich bin Jesus von
Nazareth, den du verfolgst!‹ \bibleverse{9} Meine Begleiter nahmen zwar
das Licht wahr, hörten aber die Stimme dessen nicht, der zu mir redete.
\bibleverse{10} Ich fragte dann: ›Was soll ich tun, Herr?‹ Da antwortete
mir der Herr: ›Steh auf und geh nach Damaskus! Dort wirst du Auskunft
über alles erhalten, was dir zu tun verordnet ist.‹ \bibleverse{11} Weil
ich nun, von dem Glanz jenes Lichtes geblendet, nicht sehen konnte,
wurde ich von meinen Begleitern an der Hand geführt und gelangte so nach
Damaskus. \bibleverse{12} Dort kam ein gewisser Ananias, ein
gesetzesfrommer Mann, der sich der Anerkennung aller dortigen Juden
erfreute, \bibleverse{13} zu mir, trat vor mich hin und sagte zu mir:
›Bruder Saul, werde wieder sehend!‹, und augenblicklich erhielt ich das
Augenlicht zurück und konnte ihn sehen. \bibleverse{14} Er aber fuhr
fort: ›Der Gott unserer Väter hat dich dazu bestimmt, seinen Willen zu
erkennen und den Gerechten zu sehen und einen Ruf aus seinem Munde zu
vernehmen; \bibleverse{15} denn du sollst Zeugnis für ihn vor allen
Menschen ablegen von dem, was du gesehen und gehört hast.
\bibleverse{16} Und nun -- was zögerst du noch? Stehe auf, laß dich
taufen und wasche deine Sünden ab, indem du seinen Namen anrufst!‹
\bibleverse{17} Als ich dann nach Jerusalem zurückgekehrt war und im
Tempel betete, geriet ich in eine Verzückung \bibleverse{18} und sah
ihn\textless sup title=``d.h. Jesus''\textgreater✲, der mir gebot:
›Beeile dich und verlaß Jerusalem schleunigst! Denn man wird hier dein
Zeugnis über\textless sup title=``oder: für''\textgreater✲ mich nicht
annehmen.‹ \bibleverse{19} Da entgegnete ich: ›Herr, sie wissen doch
selbst, daß ich es gewesen bin, der die an dich Gläubigen ins Gefängnis
werfen und in den Synagogen auspeitschen ließ; \bibleverse{20} und als
das Blut deines Zeugen Stephanus vergossen wurde, da habe auch ich
dabeigestanden und Freude daran gehabt und Wache bei den Mänteln✲ derer
gehalten, die ihn ums Leben brachten.‹ \bibleverse{21} Doch er
antwortete mir: ›Mache dich auf den Weg, denn ich will dich in die Ferne
zu den Heiden senden!‹«

\hypertarget{dd-die-wirkung-der-rede-paulus-in-gewahrsam-bei-dem-ruxf6mischen-obersten}{%
\subparagraph{dd) Die Wirkung der Rede; Paulus in Gewahrsam bei dem
römischen
Obersten}\label{dd-die-wirkung-der-rede-paulus-in-gewahrsam-bei-dem-ruxf6mischen-obersten}}

\bibleverse{22} Bis zu diesem Wort hatten sie ihm ruhig zugehört; nun
aber erhoben sie ein Geschrei: »Hinweg mit einem solchen Menschen von
der Erde! Er darf nicht am Leben bleiben!« \bibleverse{23} Während sie
noch so schrien und dabei ihre Mäntel abwarfen und Staub in die Luft
schleuderten, \bibleverse{24} ließ der Oberst ihn in die Burg
hineinbringen und gab Befehl, man solle ihn unter Geißelhieben✲
verhören, damit man herausbrächte, aus welchem Grunde sie so wütend
gegen ihn schrien. \bibleverse{25} Als man ihn nun schon für die
(Geißelung mit) Riemen ausgestreckt hatte, sagte Paulus zu dem
Hauptmann, der dabeistand: »Dürft ihr einen römischen Bürger geißeln,
und noch dazu, ehe ein richterliches Urteil vorliegt?« \bibleverse{26}
Als der Hauptmann das hörte, begab er sich zu dem Oberst und meldete
ihm: »Was willst du tun? Dieser Mann ist ja ein römischer Bürger!«
\bibleverse{27} Da trat der Oberst herzu und sagte zu ihm: »Sage mir:
bist du wirklich ein römischer Bürger?« Er erwiderte: »Ja.«
\bibleverse{28} Da antwortete der Oberst: »Ich habe mir dieses
Bürgerrecht für viel Geld erworben.« Paulus sagte: »Ich dagegen bin
sogar als römischer Bürger geboren!« \bibleverse{29} So ließ man denn
sofort von dem beabsichtigten peinlichen Verhör ab; aber auch der Oberst
hatte einen Schrecken bekommen, da er erfahren hatte, daß er ein
römischer Bürger sei, und weil er ihn hatte fesseln lassen.

\hypertarget{c-paulus-vor-dem-hohen-rat-der-juden}{%
\paragraph{c) Paulus vor dem Hohen Rat der
Juden}\label{c-paulus-vor-dem-hohen-rat-der-juden}}

\bibleverse{30} Weil er aber über das Vergehen, das ihm von seiten der
Juden vorgeworfen wurde, ins klare kommen wollte, ließ er ihm am
folgenden Tage die Fesseln abnehmen und ordnete eine
Versammlung\textless sup title=``oder: Sitzung''\textgreater✲ der
Hohenpriester und des ganzen Hohen Rates an; dann ließ er Paulus
hinabführen und ihn vor sie stellen.

\hypertarget{section-22}{%
\section{23}\label{section-22}}

\bibleverse{1} Paulus blickte nun den Hohen Rat fest an und sagte:
»Werte Brüder! Ich habe bis heute meinen Wandel mit durchaus reinem
Gewissen im Dienste Gottes geführt.« \bibleverse{2} Da befahl der
Hohepriester Ananias den neben ihm stehenden (Gerichtsdienern), ihn auf
den Mund zu schlagen. \bibleverse{3} Paulus aber rief ihm zu: »Dich wird
Gott schlagen, du getünchte Wand! Du sitzest da, um mich nach dem Gesetz
zu richten, und läßt mich unter Verletzung des Gesetzes schlagen?«
\bibleverse{4} Da sagten die neben ihm Stehenden: »Den Hohenpriester
Gottes schmähst du?« \bibleverse{5} Da antwortete Paulus: »Ich habe
nicht gewußt, ihr Brüder, daß er Hoherpriester ist! Es steht ja
geschrieben\textless sup title=``2.Mose 22,27''\textgreater✲: ›Einen
Obersten\textless sup title=``oder: den Fürsten''\textgreater✲ deines
Volkes sollst du nicht schmähen!‹« \bibleverse{6} Weil Paulus nun wußte,
daß der eine Teil (des Hohen Rates) aus Sadduzäern, der andere aus
Pharisäern bestand, rief er laut in die Versammlung hinein: »Werte
Brüder! Ich bin ein Pharisäer und aus pharisäischer Familie! Wegen
unserer Hoffnung, nämlich wegen der Auferstehung der Toten, stehe ich
hier vor Gericht!« \bibleverse{7} Infolge dieser seiner Äußerung
entstand ein Streit zwischen den Pharisäern und Sadduzäern, und die
Versammlung spaltete sich. \bibleverse{8} Die Sadduzäer behaupten
nämlich, es gebe keine Auferstehung, auch keine Engel und keine Geister,
während die Pharisäer beides annehmen. \bibleverse{9} So erhob sich denn
ein gewaltiges Geschrei; ja, einige Schriftgelehrte von der
pharisäischen Partei standen auf, hielten Streitreden und erklärten:
»Wir finden nichts Unrechtes an diesem Mann! Kann nicht wirklich ein
Geist oder ein Engel zu ihm geredet haben?« \bibleverse{10} Als nun der
Streit leidenschaftlich wurde und der Oberst befürchtete, Paulus möchte
von ihnen zerrissen werden, ließ er seine Mannschaft herunterkommen, ihn
aus ihrer Mitte herausreißen und in die Burg zurückführen.
\bibleverse{11} In der folgenden Nacht aber trat der Herr zu Paulus und
sagte: »Sei getrost! Denn wie du für mich in Jerusalem Zeugnis abgelegt
hast, so sollst du auch in Rom Zeuge (für mich) sein!«

\hypertarget{d-mordanschlag-der-juden-gegen-paulus}{%
\paragraph{d) Mordanschlag der Juden gegen
Paulus}\label{d-mordanschlag-der-juden-gegen-paulus}}

\bibleverse{12} Als es aber Tag geworden war, rotteten sich die Juden
zusammen und verschworen sich unter feierlicher Selbstverfluchung, weder
Speise noch Trank zu sich zu nehmen, bis sie Paulus ums Leben gebracht
hätten. \bibleverse{13} Es waren ihrer aber mehr als vierzig, die sich
zu dieser Verschwörung\textless sup title=``=~diesem
Schwurbund''\textgreater✲ zusammengetan hatten. \bibleverse{14} Diese
begaben sich nun zu den Hohenpriestern und Ältesten und sagten: »Wir
haben uns hoch und heilig verschworen, nichts zu genießen, bis wir
Paulus ums Leben gebracht haben. \bibleverse{15} Werdet ihr jetzt also
zusammen mit dem Hohen Rat bei dem Oberst vorstellig, er möge ihn zu
euch herabführen lassen, weil ihr seine Sache noch genauer zu
untersuchen gedächtet; wir halten uns dann bereit, ihn zu ermorden, noch
ehe er in eure Nähe kommt.«

\bibleverse{16} Von diesem Anschlag erhielt jedoch der Schwestersohn des
Paulus Kenntnis; er begab sich deshalb hin, verschaffte sich Eingang in
die Burg und machte dem Paulus Mitteilung von der Sache. \bibleverse{17}
Da ließ Paulus einen von den Hauptleuten zu sich rufen und bat ihn:
»Führe doch diesen jungen Mann zum Obersten, denn er hat ihm etwas zu
melden.« \bibleverse{18} Der nahm ihn mit sich, führte ihn zu dem
Obersten und meldete: »Der Gefangene Paulus hat mich zu sich rufen
lassen und mich ersucht, diesen jungen Mann zu dir zu führen, weil er
dir etwas mitzuteilen habe.« \bibleverse{19} Der Oberst nahm ihn darauf
bei der Hand, trat (mit ihm) beiseite und fragte ihn unter vier Augen:
»Was hast du mir zu melden?« \bibleverse{20} Da berichtete er: »Die
Juden haben sich verabredet, dich zu bitten, du möchtest morgen Paulus
vor den Hohen Rat hinabführen lassen, angeblich weil dieser noch eine
genauere Untersuchung seiner Sache vornehmen wolle. \bibleverse{21}
Glaube du ihnen aber nicht! Denn mehr als vierzig Männer von ihnen
trachten ihm nach dem Leben; die haben sich feierlich verschworen, weder
Speise noch Trank zu sich zu nehmen, bis sie ihn ermordet haben; und sie
halten sich jetzt schon dazu bereit und warten nur noch auf deine
Zusage.« \bibleverse{22} Der Oberst entließ darauf den jungen Mann mit
der Weisung, niemandem zu verraten, daß er ihm diese Mitteilung gemacht
habe.

\hypertarget{e-brief-des-obersten-lysias-an-den-statthalter-felix-uxfcberfuxfchrung-des-paulus-von-jerusalem-nach-cuxe4sarea}{%
\paragraph{e) Brief des Obersten Lysias an den Statthalter Felix;
Überführung des Paulus von Jerusalem nach
Cäsarea}\label{e-brief-des-obersten-lysias-an-den-statthalter-felix-uxfcberfuxfchrung-des-paulus-von-jerusalem-nach-cuxe4sarea}}

\bibleverse{23} Danach ließ er zwei von seinen Hauptleuten zu sich
kommen und befahl ihnen: »Haltet zweihundert Mann für einen Marsch nach
Cäsarea bereit, ferner siebzig Reiter und zweihundert Lanzenträger, von
der dritten Stunde der Nacht an.« \bibleverse{24} Auch Reittiere sollten
sie bereithalten, um Paulus beritten zu machen und ihn sicher zum
Statthalter Felix zu bringen. \bibleverse{25} Er schrieb außerdem einen
Brief folgenden Wortlauts: \bibleverse{26} »Ich, Klaudius Lysias, sende
dem hochedlen Statthalter Felix meinen Gruß! \bibleverse{27} Dieser Mann
war von den Juden festgenommen worden und schwebte in Gefahr, von ihnen
totgeschlagen zu werden; da griff ich mit meinen Leuten ein und befreite
ihn, weil ich erfahren hatte, daß er ein römischer Bürger sei.
\bibleverse{28} Da ich nun den Grund festzustellen wünschte, weswegen
sie ihn verklagten, führte ich ihn vor ihren Hohen Rat hinab.
\bibleverse{29} Dabei fand ich, daß man ihn wegen Streitfragen über ihr
Gesetz verklagte, daß aber keine Anschuldigung, auf welche Todesstrafe
oder Gefängnis steht, gegen ihn vorlag. \bibleverse{30} Weil dann aber
die Anzeige bei mir einging, daß ein Mordanschlag gegen den Mann geplant
werde, habe ich ihn sofort von hier weg zu dir gesandt und zugleich
seine Ankläger angewiesen, ihre Sache gegen ihn bei dir anhängig zu
machen. Lebe wohl!«

\bibleverse{31} Die Soldaten nahmen nun dem erhaltenen Befehl gemäß
Paulus mit sich und brachten ihn während der Nacht nach Antipatris;
\bibleverse{32} am folgenden Tage ließen sie dann die Reiter (allein)
mit ihm weiterziehen, während sie selbst in die Burg zurückkehrten.
\bibleverse{33} Nach ihrer Ankunft in Cäsarea händigten jene dem
Statthalter das Schreiben ein und führten ihm auch den Paulus vor.
\bibleverse{34} Nachdem der Statthalter (das Schreiben) gelesen hatte,
fragte er (Paulus), aus welcher Provinz er sei; und als er erfuhr, daß
er aus Cilicien stamme, erklärte er: \bibleverse{35} »Ich werde dich
verhören, wenn auch deine Ankläger hier eingetroffen sind.« Zugleich
befahl er, ihn in der Statthalterei des Herodes in Gewahrsam zu halten.

\hypertarget{f-paulus-als-angeklagter-gefangen-in-cuxe4sarea}{%
\paragraph{f) Paulus als Angeklagter gefangen in
Cäsarea}\label{f-paulus-als-angeklagter-gefangen-in-cuxe4sarea}}

\hypertarget{aa-gerichtsverhandlung-vor-dem-statthalter-felix}{%
\subparagraph{aa) Gerichtsverhandlung vor dem Statthalter
Felix}\label{aa-gerichtsverhandlung-vor-dem-statthalter-felix}}

\hypertarget{section-23}{%
\section{24}\label{section-23}}

\bibleverse{1} Fünf Tage später kam dann der Hohepriester Ananias mit
einigen Ältesten und einem Rechtsanwalt, einem gewissen Tertullus (nach
Cäsarea) hinab, und sie machten die Anklage gegen Paulus beim
Statthalter anhängig. \bibleverse{2} Nachdem man nun Paulus
herbeigerufen\textless sup title=``oder: vorgeführt''\textgreater✲
hatte, begann Tertullus mit der Anklagerede folgendermaßen:
\bibleverse{3} »Hochedler Felix! Daß wir durch dein Verdienst in tiefem
Frieden leben und der hiesigen Bevölkerung durch deine Fürsorge
treffliche Einrichtungen allerseits und überall zuteil werden, das
erkennen wir mit aufrichtiger Dankbarkeit an. \bibleverse{4} Um dich
aber nicht unnötigerweise zu belästigen, bitte ich dich, du wollest uns
nach deiner gewohnten Güte für kurze Zeit Gehör schenken. \bibleverse{5}
Wir haben nämlich diesen Mann als eine Pest\textless sup title=``=~als
einen gemeingefährlichen Menschen''\textgreater✲ und als einen
Unruhestifter unter allen Juden im ganzen römischen Reich und als den
Hauptführer\textless sup title=``oder: Vorkämpfer''\textgreater✲ der
Sekte der Nazaräer ermittelt; \bibleverse{6} er hat sogar den Versuch
gemacht, den Tempel zu entweihen. Dabei haben wir ihn auch festgenommen;
\bibleverse{8} und wenn du ihn jetzt verhörst, wirst du dir selbst nach
seinen Aussagen ein Urteil über alles das bilden können, was wir ihm zur
Last legen.« \bibleverse{9} Diesen Angaben schlossen sich auch die
anderen Juden an und bestätigten deren Wahrheit.

\bibleverse{10} Durch einen Wink des Statthalters aufgefordert, begann
nun Paulus seine Verteidigungsrede: »Da ich weiß, daß du schon seit
vielen Jahren Richter für die hiesige Bevölkerung bist, so gehe ich
getrosten Mutes an die Verteidigung meiner Sache vor dir.
\bibleverse{11} Wie du dich vergewissern kannst, sind erst zwölf Tage
vergangen, seitdem ich nach Jerusalem hinaufgezogen bin, um dort
anzubeten; \bibleverse{12} und weder im Tempel hat man mich bei einer
Verhandlung mit jemand oder bei der Anstiftung eines Volksauflaufs
betroffen, auch nicht in den Synagogen oder sonst irgendwo in der Stadt;
\bibleverse{13} sie sind überhaupt nicht imstande, dir Beweise für ihre
jetzigen Anklagen gegen mich zu erbringen. \bibleverse{14} Das freilich
bekenne ich dir offen, daß ich nach der Glaubensrichtung\textless sup
title=``vgl. 22,4''\textgreater✲, die sie als Sekte bezeichnen, dem Gott
unserer Väter in der Weise diene, daß ich allem, was im Gesetz und was
in den Propheten geschrieben steht, Glauben schenke \bibleverse{15} und
auf Gott dieselbe Hoffnung setze, welche auch sie selbst hegen, daß
nämlich eine Auferstehung der Gerechten wie der Ungerechten stattfinden
wird. \bibleverse{16} Darum bemühe ich mich auch, immerdar ein
unverletztes Gewissen Gott und den Menschen gegenüber zu haben.
\bibleverse{17} Nun bin ich nach einer Zwischenzeit von mehreren Jahren
hergekommen, um Almosen für mein Volk zu überbringen und Opfer
darzubringen. \bibleverse{18} Als ich mich dabei einer Weihe unterzogen
hatte, haben sie mich im Tempel angetroffen, und zwar nicht in
Begleitung eines Volkshaufens oder unter Erregung eines Aufruhrs;
\bibleverse{19} nein, einige Juden aus der Provinz Asien sind es
gewesen; diese hätten hier vor dir erscheinen und Anklage erheben
müssen, wenn sie etwas gegen mich vorzubringen haben. \bibleverse{20}
Oder laß diese hier selber angeben, welche Schuld sie (an mir) ermittelt
haben, als ich vor dem Hohen Rate stand; \bibleverse{21} es müßte denn
das eine Wort sein, das ich in ihrer Mitte stehend ausgerufen habe:
›Wegen der Auferstehung der Toten stehe ich heute als Angeklagter hier
vor euch!‹«

\bibleverse{22} Felix vertagte darauf die Entscheidung ihrer Sache, weil
er ganz genau wußte, was es mit der (in Frage stehenden)
Glaubensrichtung✲ auf sich hatte, und sagte: »Wenn der Oberst Lysias
herabkommt, werde ich eure Sache entscheiden.« \bibleverse{23} Zugleich
gab er aber dem Hauptmann die Weisung, Paulus in Gewahrsam zu halten,
doch in milder Haft, und keinen von seinen Freunden an der Erweisung von
Liebesdiensten zu hindern.

\hypertarget{bb-paulus-vor-felix-und-drusilla-verschleppung-des-prozesses-durch-felix}{%
\subparagraph{bb) Paulus vor Felix und Drusilla; Verschleppung des
Prozesses durch
Felix}\label{bb-paulus-vor-felix-und-drusilla-verschleppung-des-prozesses-durch-felix}}

\bibleverse{24} Einige Tage später aber erschien Felix mit seiner Gattin
Drusilla, einer Jüdin; er beschied Paulus vor sich und ließ sich einen
Vortrag über den Glauben an Christus Jesus halten. \bibleverse{25} Als
Paulus dabei aber über Gerechtigkeit, Enthaltsamkeit und über das
künftige Gericht redete, geriet Felix in Unruhe und sagte: »Für diesmal
kannst du gehen! Wenn ich (später) gelegene Zeit habe, will ich dich
wieder rufen lassen.« \bibleverse{26} Daneben hegte er auch die
Hoffnung, er werde Geld von Paulus erhalten; daher ließ er ihn auch
öfter rufen und unterredete sich mit ihm. \bibleverse{27} Nach Verlauf
von zwei Jahren aber erhielt Felix einen Nachfolger in Porcius Festus;
und weil Felix sich die Juden zu Dank verpflichten wollte, ließ er
Paulus als Gefangenen zurück.

\hypertarget{cc-wiederaufnahme-des-prozesses-festus-in-jerusalem-und-in-cuxe4sarea-paulus-beruft-sich-auf-den-kaiser}{%
\subparagraph{cc) Wiederaufnahme des Prozesses; Festus in Jerusalem und
in Cäsarea; Paulus beruft sich auf den
Kaiser}\label{cc-wiederaufnahme-des-prozesses-festus-in-jerusalem-und-in-cuxe4sarea-paulus-beruft-sich-auf-den-kaiser}}

\hypertarget{section-24}{%
\section{25}\label{section-24}}

\bibleverse{1} Als Festus nun die Statthalterschaft in der Provinz
angetreten hatte, begab er sich drei Tage später von Cäsarea nach
Jerusalem hinauf. \bibleverse{2} Da wurden die Hohenpriester und die
vornehmsten Juden bei ihm in der Sache gegen Paulus vorstellig und
trugen ihm ihr Anliegen vor, \bibleverse{3} wobei sie es sich als
besondere Vergünstigung wider ihn (Paulus) erbaten, daß er ihn nach
Jerusalem bringen lasse; sie planten nämlich einen Anschlag, um ihn
unterwegs zu ermorden. \bibleverse{4} Da gab Festus ihnen zur Antwort,
Paulus werde in Cäsarea in Haft gehalten und er selbst werde binnen
kurzem (wieder dahin) abreisen. \bibleverse{5} »Darum mögen«, fuhr er
fort, »Bevollmächtigte aus eurer Mitte mit mir hinabkommen und die
Anklage gegen ihn erheben, wenn eine Verschuldung bei dem Manne
vorliegt.«

\bibleverse{6} Nachdem er sich dann höchstens acht oder zehn Tage bei
ihnen aufgehalten hatte, kehrte er nach Cäsarea zurück, hielt am
folgenden Tage eine Gerichtssitzung ab und ließ Paulus vorführen.
\bibleverse{7} Als dieser erschienen war, umringten ihn die Juden, die
aus Jerusalem gekommen waren, und brachten viele schwere Beschuldigungen
gegen ihn vor, die sie aber nicht zu beweisen vermochten, \bibleverse{8}
während Paulus in seiner Verteidigung dartat: »Ich habe mich weder gegen
das jüdische Gesetz noch gegen den Tempel noch gegen den Kaiser
irgendwie vergangen.« \bibleverse{9} Weil Festus sich aber die Juden zu
Dank verpflichten wollte, legte er dem Paulus die Frage vor: »Willst du
nach Jerusalem hinaufgehen und dich dort in dieser Sache vor mir richten
lassen?«

\bibleverse{10} Da antwortete Paulus: »Ich stehe hier vor des Kaisers
Richterstuhl, und hier habe ich auch mein Urteil zu empfangen. Den Juden
habe ich nichts zuleide getan, wie du selbst ganz genau weißt.
\bibleverse{11} Wenn ich nun im Unrecht bin und ein todeswürdiges
Verbrechen begangen habe, so weigere ich mich nicht, zu sterben; wenn
aber an den Beschuldigungen, die diese gegen mich vorbringen, nichts
Wahres ist, so darf mich niemand ihnen zuliebe preisgeben. Ich lege
Berufung an den Kaiser ein!« \bibleverse{12} Darauf besprach sich Festus
mit seinen Räten und gab dann den Bescheid ab: »An den Kaiser hast du
Berufung eingelegt: vor den Kaiser sollst du kommen!«

\hypertarget{dd-herodes-agrippa-ii.-und-bernice-als-besuchsguxe4ste-bei-festus-in-cuxe4sarea-festus-teilt-dem-agrippa-die-sache-des-paulus-mit}{%
\subparagraph{dd) Herodes Agrippa II. und Bernice als Besuchsgäste bei
Festus in Cäsarea; Festus teilt dem Agrippa die Sache des Paulus
mit}\label{dd-herodes-agrippa-ii.-und-bernice-als-besuchsguxe4ste-bei-festus-in-cuxe4sarea-festus-teilt-dem-agrippa-die-sache-des-paulus-mit}}

\bibleverse{13} Einige Tage später kamen der König Agrippa und (seine
Schwester) Bernice nach Cäsarea, um dem Festus ihren Besuch zu machen.
\bibleverse{14} Während ihres mehrtägigen Aufenthalts daselbst legte
Festus dem König die Sache des Paulus vor mit den Worten: »Hier ist von
Felix ein Mann als Gefangener zurückgelassen worden, \bibleverse{15}
gegen den während meiner Anwesenheit in Jerusalem die Hohenpriester und
die Ältesten der Juden bei mir vorstellig geworden sind und dessen
Verurteilung sie von mir verlangt haben. \bibleverse{16} Ich habe ihnen
zur Antwort gegeben, bei den Römern sei es nicht üblich, einen Menschen
aus Gefälligkeit preiszugeben, bevor nicht der Angeklagte seinen
Anklägern persönlich gegenübergestanden und Gelegenheit zur Verteidigung
gegen die Anklage erhalten habe. \bibleverse{17} Als sie dann hierher
gekommen waren, habe ich unverzüglich schon am nächsten Tage eine
Gerichtssitzung abgehalten und den Mann vorführen lassen.
\bibleverse{18} Die Ankläger traten auf, brachten aber über ihn keine
Beschuldigung wegen schwerer Verbrechen vor, wie ich erwartet hatte,
\bibleverse{19} sondern sie hatten gegen ihn nur einige Streitfragen
bezüglich ihrer besonderen Gottesverehrung sowie bezüglich eines
gewissen Jesus, der bereits tot ist, von dem Paulus aber behauptete, daß
er lebe. \bibleverse{20} Da ich mich nun auf die Untersuchung dieser
Dinge nicht verstand, fragte ich ihn, ob er nicht nach Jerusalem gehen
und sich dort hierüber das Urteil sprechen lassen wollte.
\bibleverse{21} Als Paulus dann aber Berufung einlegte und bis zur
Entscheidung des Kaisers in Haft zu bleiben verlangte, habe ich
befohlen, man solle ihn weiter in Gewahrsam halten, bis ich ihn zum
Kaiser senden würde.« \bibleverse{22} Da sagte Agrippa zu Festus: »Ich
möchte den Mann gern persönlich (einmal) hören«, worauf jener erwiderte:
»Gleich morgen sollst du ihn hören!«

\hypertarget{ee-vorfuxfchrung-und-verteidigungsrede-des-paulus-vor-agrippa-und-festus}{%
\subparagraph{ee) Vorführung und Verteidigungsrede des Paulus vor
Agrippa und
Festus}\label{ee-vorfuxfchrung-und-verteidigungsrede-des-paulus-vor-agrippa-und-festus}}

\bibleverse{23} Als nun am folgenden Tage Agrippa und Bernice mit großem
Gepränge\textless sup title=``oder: Gefolge''\textgreater✲ erschienen
und mit den Heeresobersten✲ und den vornehmsten Männern der Stadt in den
Vortragssaal eingetreten waren, wurde Paulus auf Befehl des Festus
vorgeführt. \bibleverse{24} Darauf sagte Festus: »König Agrippa und ihr
anderen mit uns hier anwesenden Herrn alle! Ihr seht hier den Mann,
wegen dessen die gesamte Judenschaft mich in Jerusalem wie auch hier mit
dem lauten Ruf bestürmt hat, er dürfe nicht länger am Leben bleiben.
\bibleverse{25} Ich bin mir jedoch klar darüber geworden, daß er kein
todeswürdiges Verbrechen begangen hat. Weil er selbst aber Berufung an
den Kaiser eingelegt hat, habe ich mich für seine Hinsendung
entschieden. \bibleverse{26} Nun weiß ich aber meinem kaiserlichen Herrn
nichts Zuverlässiges über ihn zu berichten; darum habe ich ihn euch und
vornehmlich dir, König Agrippa, hier vorführen lassen, damit ich nach
erfolgtem Verhör eine Unterlage für meinen schriftlichen Bericht
erhalte. \bibleverse{27} Denn es scheint mir widersinnig zu sein, einen
Gefangenen hinzusenden, ohne zugleich die gegen ihn erhobenen
Beschuldigungen anzugeben.«

\hypertarget{verteidigungsrede-des-paulus-vor-agrippa}{%
\paragraph{Verteidigungsrede des Paulus vor
Agrippa}\label{verteidigungsrede-des-paulus-vor-agrippa}}

\hypertarget{section-25}{%
\section{26}\label{section-25}}

\bibleverse{1} Darauf sagte Agrippa zu Paulus: »Es ist dir gestattet, zu
deiner Rechtfertigung zu reden.« Da streckte Paulus die Hand aus und
hielt folgende Verteidigungsrede: \bibleverse{2} »Ich schätze mich
glücklich, König Agrippa, daß ich mich heute wegen aller
Beschuldigungen, welche die Juden gegen mich erheben, hier vor dir
verantworten darf, \bibleverse{3} weil du ja ein ausgezeichneter Kenner
aller Gebräuche und Streitfragen der Juden bist. Deshalb bitte ich dich,
mir geduldiges\textless sup title=``oder: geneigtes''\textgreater✲ Gehör
zu schenken.

\bibleverse{4} Wie sich meine Lebensführung von Jugend auf inmitten
meines Volkes, und zwar in Jerusalem, von Anfang an gestaltet hat, das
wissen alle Juden, \bibleverse{5} die mich von früher her kennen; sie
müssen, wenn sie nur wollen, mir das Zeugnis ausstellen, daß ich nach
der strengsten Richtung unserer Gottesverehrung gelebt habe, nämlich als
Pharisäer. \bibleverse{6} Und jetzt stehe ich hier als Angeklagter, um
mich richten zu lassen wegen der Hoffnung auf die (Erfüllung der)
Verheißung, die von Gott an unsere Väter ergangen ist \bibleverse{7} und
zu der unser Zwölfstämmevolk durch anhaltenden Gottesdienst bei Tag und
Nacht zu gelangen hofft: wegen dieser Hoffnung, o König, werde ich von
Juden angeklagt! \bibleverse{8} Warum gilt es denn bei euch für
unglaublich, wenn Gott Tote auferweckt? \bibleverse{9} Was mich freilich
betrifft, so habe ich es (einst) für meine Pflicht gehalten, den Namen
Jesu von Nazareth als erklärter Feind zu bekämpfen, \bibleverse{10} und
das habe ich denn auch in Jerusalem getan. Ich verschaffte mir nämlich
Vollmacht von den Hohenpriestern und ließ viele von den
Heiligen\textless sup title=``d.h. getauften Gläubigen''\textgreater✲ in
die Gefängnisse einschließen; und wenn sie hingerichtet werden sollten,
erklärte ich mich damit einverstanden. \bibleverse{11} In allen
Synagogen zwang ich sie oftmals durch Strafen zur Lästerung\textless sup
title=``oder: zum Widerruf''\textgreater✲ und verfolgte sie in maßloser
Wut sogar bis in die auswärtigen Städte.

\bibleverse{12} Als ich hierbei mit der Vollmacht und im Auftrag der
Hohenpriester nach Damaskus reiste, \bibleverse{13} sah ich unterwegs, o
König, zur Mittagszeit vom Himmel her ein Licht, das heller als der
Glanz der Sonne mich und meine Reisebegleiter umstrahlte.
\bibleverse{14} Als wir nun alle zu Boden niedergestürzt waren, hörte
ich eine Stimme, die mir in der hebräischen Volkssprache\textless sup
title=``vgl. 21,40''\textgreater✲ zurief: ›Saul, Saul! Was verfolgst du
mich? Es ist schwer für dich, gegen den Stachel auszuschlagen!‹
\bibleverse{15} Ich antwortete: ›Wer bist du, Herr?‹ Da erwiderte der
Herr: ›Ich bin Jesus, den du verfolgst. \bibleverse{16} Doch stehe auf
und tritt auf deine Füße! Denn dazu bin ich dir erschienen, dich zum
Diener und Zeugen für das zu machen, was du von mir (als Auferstandenem)
gesehen hast, und für das, was ich dich noch sehen lassen werde;
\bibleverse{17} und ich werde dich retten vor dem Volk (Israel) und vor
den Heiden, zu denen ich dich senden will: \bibleverse{18} du sollst
ihnen die Augen öffnen, damit sie sich von der Finsternis zum Licht und
von der Gewalt des Satans zu Gott bekehren, auf daß sie Vergebung der
Sünden und ein Erbteil unter denen erhalten, die durch den Glauben an
mich geheiligt worden sind.‹

\bibleverse{19} Infolgedessen bin ich, o König Agrippa, der himmlischen
Erscheinung nicht ungehorsam gewesen, \bibleverse{20} sondern habe
zuerst den Einwohnern von Damaskus und Jerusalem, dann denen im ganzen
jüdischen Lande und weiterhin den Heiden gepredigt, sie möchten Buße
tun\textless sup title=``vgl. Mt 3,2''\textgreater✲, sich zu Gott
bekehren und Werke vollbringen, die der Buße würdig sind.
\bibleverse{21} Das ist der Grund, weshalb die Juden mich im Tempel
festgenommen und mich ums Leben zu bringen versucht haben.
\bibleverse{22} Weil ich nun Gottes Beistand bis auf den heutigen Tag
gefunden habe, stehe ich da und lege Zeugnis vor hoch und niedrig ab;
dabei sage ich nichts anderes als das, wovon schon die Propheten und
Mose geweissagt haben, daß es geschehen werde, \bibleverse{23} nämlich
ob\textless sup title=``oder: daß''\textgreater✲ Christus\textless sup
title=``=~der Messias''\textgreater✲ zum Leiden bestimmt sei und
ob\textless sup title=``oder: daß''\textgreater✲ er als Erstling unter
den vom Tode Auferstandenen sowohl dem Volk (Israel) als auch den Heiden
das Licht verkünden solle.«

\hypertarget{eindruck-der-rede}{%
\paragraph{Eindruck der Rede}\label{eindruck-der-rede}}

\bibleverse{24} Als Paulus in dieser Weise zu seiner Verteidigung
redete, rief Festus mit lauter Stimme aus: »Paulus, du bist von Sinnen!
Die große Gelehrsamkeit\textless sup title=``oder: das viele
Studieren''\textgreater✲ bringt dich um den Verstand!« \bibleverse{25}
Da erwiderte Paulus: »Ich bin nicht von Sinnen, hochedler Festus,
sondern ich rede wahre und wohlüberlegte Worte! \bibleverse{26} Denn der
König versteht sich auf diese Dinge; an ihn wende ich mich darum auch
mit meiner freimütigen Rede; denn ich bin überzeugt, daß ihm nichts von
diesen Dingen verborgen geblieben ist; dies alles hat sich ja nicht in
einem Winkel abgespielt. \bibleverse{27} Glaubst du den Propheten, König
Agrippa? Ich weiß, daß du ihnen glaubst.« \bibleverse{28} Da antwortete
Agrippa dem Paulus: »Beinahe bringst du es fertig, mich zu einem
Christen zu machen!« \bibleverse{29} Paulus erwiderte: »Ich möchte zu
Gott beten, daß über kurz oder lang nicht allein du, sondern alle, die
mich heute hier hören, ebenso werden möchten wie ich es bin, abgesehen
allerdings von diesen meinen Fesseln.« \bibleverse{30} Darauf erhoben
sich der König und der Statthalter sowie Bernice und die übrigen neben
ihnen Sitzenden. \bibleverse{31} Nachdem sie sich zurückgezogen hatten,
unterhielten sie sich noch miteinander und erklärten: »Dieser Mann tut
nichts, was Todesstrafe oder Gefängnis verdient.« \bibleverse{32}
Agrippa aber erklärte dem Festus: »Dieser Mann könnte freigelassen
werden, wenn er nicht Berufung an den Kaiser eingelegt hätte.«

\hypertarget{reise-des-paulus-von-cuxe4sarea-nach-rom}{%
\subsubsection{2. Reise des Paulus von Cäsarea nach
Rom}\label{reise-des-paulus-von-cuxe4sarea-nach-rom}}

\hypertarget{a-abreise-fahrt-uxfcber-sidon-und-myra-bis-zur-insel-kreta}{%
\paragraph{a) Abreise; Fahrt über Sidon und Myra bis zur Insel
Kreta}\label{a-abreise-fahrt-uxfcber-sidon-und-myra-bis-zur-insel-kreta}}

\hypertarget{section-26}{%
\section{27}\label{section-26}}

\bibleverse{1} Als nun unsere Abfahrt nach Italien beschlossen war,
übergab man den Paulus und einige andere Gefangene einem Hauptmann der
Kaiserlichen Abteilung\textless sup title=``eig. Kohorte; vgl.
10,1''\textgreater✲ namens Julius. \bibleverse{2} Wir bestiegen dann ein
Schiff aus Adramyttium, das die Küstenplätze der römischen Provinz Asien
anlaufen sollte, und fuhren ab; in unserer Begleitung befand sich auch
noch Aristarchus, ein Mazedonier aus Thessalonike. \bibleverse{3} Am
folgenden Tage landeten wir in Sidon; und weil Julius den Paulus
menschenfreundlich behandelte, erlaubte er ihm, seine (dortigen) Freunde
zu besuchen und sich von ihnen mit dem nötigen Reisebedarf versorgen zu
lassen. \bibleverse{4} Von da fuhren wir weiter, und zwar dicht an (der
Ostseite von) Cypern hin, weil wir Gegenwind hatten. \bibleverse{5}
Nachdem wir dann die See längs der Küste von Cilicien und Pamphylien hin
durchsegelt hatten, gelangten wir nach Myra in Lycien. \bibleverse{6}
Als der Hauptmann dort ein alexandrinisches Schiff vorfand, das auf der
Fahrt nach Italien begriffen war, brachte er uns auf dieses.
\bibleverse{7} Im Verlauf vieler Tage langsamer Fahrt kamen wir mit Mühe
in die Nähe von Knidus; und weil uns der Wind dort nicht anlegen ließ,
fuhren wir an Kreta hin, und zwar bei Salome. \bibleverse{8} Nur mit
Mühe erreichten wir bei dieser Küstenfahrt einen Ort namens Schönhafen,
in dessen Nähe die Stadt Lasäa lag.

\bibleverse{9} Da inzwischen geraume Zeit verflossen war und die
Schiffahrt bereits gefährlich zu werden begann -- sogar der große
(jüdische Versöhnungs-) Fasttag war schon vorüber --, sagte Paulus
warnend zu ihnen: \bibleverse{10} »Ihr Männer, ich sehe voraus, daß ein
Weiterfahren mit Gefahr und großem Schaden nicht nur für die Ladung und
das Schiff, sondern auch für unser Leben verbunden sein wird.«
\bibleverse{11} Aber der Hauptmann schenkte dem Steuermann und dem
Schiffsherrn mehr Glauben als den Worten des Paulus; \bibleverse{12} und
weil der Hafen zum Überwintern ungeeignet war, faßte die Mehrzahl den
Beschluß, von dort weiterzufahren und womöglich zum Überwintern nach
Phönix zu gelangen, einem kretischen Hafen, der gegen den Südwest- und
Nordwestwind geschützt liegt.

\hypertarget{b-seesturm-und-schiffbruch-rettung-in-malta}{%
\paragraph{b) Seesturm und Schiffbruch; Rettung in
Malta}\label{b-seesturm-und-schiffbruch-rettung-in-malta}}

\bibleverse{13} Als nun ein schwacher Südwind einsetzte, glaubten sie,
ihr Vorhaben sicher ausführen zu können; sie lichteten daher die Anker
und fuhren ganz nahe an der Küste von Kreta hin. \bibleverse{14} Doch
schon nach kurzer Zeit brach von der Insel her ein Sturmwind los, der
sogenannte Euraquilo\textless sup title=``d.h.
Ostnordostwind''\textgreater✲. \bibleverse{15} Da nun das Schiff von
diesem fortgerissen wurde und dem Wind gegenüber machtlos war, mußten
wir uns auf gut Glück treiben lassen. \bibleverse{16} Als wir dann unter
dem Schutz eines Inselchens namens Klauda\textless sup title=``oder:
Kauda''\textgreater✲ hinfuhren, gelang es uns nur mit großer Mühe, uns
im Besitz des Rettungsbootes zu erhalten: \bibleverse{17} man zog es an
Bord herauf und brachte Schutzmittel in Anwendung, indem man das Schiff
(mit Tauen) gürtete; und weil man auf die (Sandbänke der) Syrte zu
geraten befürchtete, holte man die Segel herunter und ließ sich so
treiben. \bibleverse{18} Weil wir aber vom Sturm schwer zu leiden
hatten, warf man am folgenden Tage einen Teil der Ladung über Bord
\bibleverse{19} und ließ am dritten Tage das Schiffsgerät\textless sup
title=``oder: Takelwerk des Schiffes''\textgreater✲ notgedrungen
nachfolgen. \bibleverse{20} Als dann aber mehrere Tage hindurch weder
die Sonne noch Sterne sichtbar waren und der Sturm ungeschwächt
weitertobte, schwand uns schließlich alle Hoffnung auf Rettung.

\hypertarget{paulus-als-berater-truxf6ster-und-retter-in-seenot}{%
\paragraph{Paulus als Berater, Tröster und Retter in
Seenot}\label{paulus-als-berater-truxf6ster-und-retter-in-seenot}}

\bibleverse{21} Weil nun niemand mehr Nahrung zu sich nehmen mochte,
trat Paulus mitten unter sie und sagte: »Ihr Männer! Man hätte
allerdings auf mich hören und nicht von Kreta abfahren sollen: dann wäre
uns dieses Ungemach und dieser Schaden erspart geblieben.
\bibleverse{22} Doch, wie die Dinge jetzt einmal liegen, fordere ich
euch auf, getrosten Mutes zu sein; denn keiner von euch wird das Leben
verlieren; nur das Schiff ist verloren. \bibleverse{23} Denn in dieser
Nacht ist mir ein Engel des Gottes erschienen, dem ich angehöre und dem
ich auch diene, \bibleverse{24} und hat zu mir gesagt: ›Fürchte dich
nicht, Paulus! Du mußt vor den Kaiser treten, und wisse wohl: Gott hat
dir das Leben aller deiner Reisegefährten geschenkt!‹ \bibleverse{25}
Darum seid guten Mutes, ihr Männer! Denn ich habe die feste Zuversicht
zu Gott, daß es so kommen wird, wie mir angekündigt worden ist.
\bibleverse{26} Wir müssen aber an irgendeiner Insel stranden.«

\bibleverse{27} Als dann die vierzehnte Nacht gekommen war, seit wir im
Adriatischen Meer umhertrieben, vermuteten die Schiffsleute um
Mitternacht die Annäherung von Land. \bibleverse{28} Als sie nämlich das
Senkblei auswarfen, stellten sie zwanzig Klafter\textless sup
title=``oder: Faden''\textgreater✲ Tiefe fest; und als sie in kurzer
Entfernung wieder loteten, fanden sie nur fünfzehn Klafter.
\bibleverse{29} Weil sie nun fürchteten, wir könnten irgendwo auf
Klippen geraten, warfen sie vier Anker hinten vom Schiff aus und
erwarteten mit Sehnsucht den Anbruch des Tages. \bibleverse{30} Als nun
aber die Schiffsleute aus dem Schiff zu entfliehen suchten und (zu
diesem Zweck) das Rettungsboot ins Meer niederließen unter dem Vorgeben,
sie wollten auch vorn aus dem Schiff Anker auswerfen, \bibleverse{31}
erklärte Paulus dem Hauptmann und den Soldaten: »Wenn diese Leute nicht
im Schiff bleiben, könnt ihr unmöglich gerettet werden!« \bibleverse{32}
Daraufhin hieben die Soldaten die Taue des Bootes ab und ließen es in
die See treiben.

\bibleverse{33} Bis es aber Tag werden wollte, redete Paulus allen zu,
sie möchten Nahrung zu sich nehmen; er sagte nämlich: »Heute ist es der
vierzehnte Tag, daß ihr ohne Nahrung ununterbrochen in ängstlicher
Erwartung schwebt und nichts Rechtes zu euch genommen habt.
\bibleverse{34} Darum rate ich euch: nehmt Nahrung zu euch! Das ist zu
eurer Rettung notwendig; denn keinem von euch wird ein Haar vom Haupt
verlorengehen!« \bibleverse{35} Nach diesen Worten nahm er Brot, sagte
Gott vor aller Augen Dank, brach ein Stück ab und begann zu essen.
\bibleverse{36} Da bekamen alle neuen Mut und nahmen ebenfalls Nahrung
zu sich. \bibleverse{37} Wir waren aber unser im ganzen
zweihundertsechsundsiebzig Seelen auf dem Schiff. \bibleverse{38}
Nachdem sie sich nun satt gegessen hatten, erleichterten sie das Schiff
dadurch, daß sie die Getreideladung ins Meer warfen.

\hypertarget{schiffbruch-angesichts-der-insel-malta-rettung-der-schiffbruxfcchigen}{%
\paragraph{Schiffbruch angesichts der Insel Malta; Rettung der
Schiffbrüchigen}\label{schiffbruch-angesichts-der-insel-malta-rettung-der-schiffbruxfcchigen}}

\bibleverse{39} Als es dann (endlich) Tag wurde, erkannten sie das Land
nicht, gewahrten aber eine Bucht mit flachem Strand, auf den sie, wenn
möglich, das Schiff auflaufen zu lassen beschlossen. \bibleverse{40} So
kappten sie denn die Ankertaue und ließen sie ins Meer fallen; zugleich
machten sie die Riemen an den (beiden) Steuerrudern los, stellten das
Vordersegel vor den Wind und hielten auf den Strand zu. \bibleverse{41}
Dabei gerieten sie aber auf eine Sandbank, auf die sie das Schiff
auflaufen ließen: das Vorderteil bohrte sich tief ein und saß
unbeweglich fest, während das Hinterschiff infolge der Gewalt der Wogen
allmählich auseinanderging. \bibleverse{42} Die Soldaten faßten nun den
Plan, die Gefangenen zu töten, damit keiner von ihnen durch Schwimmen
entkäme; \bibleverse{43} der Hauptmann aber, welcher Paulus am Leben zu
erhalten wünschte, hinderte sie an der Ausführung ihres Vorhabens; er
ließ vielmehr die, welche schwimmen konnten, ins Meer springen und sich
zuerst ans Land retten; \bibleverse{44} die übrigen (mußten) dann teils
auf Brettern, teils auf irgendwelchen Gegenständen aus dem Schiff (das
Ufer gewinnen). Auf diese Weise gelang es allen, wohlbehalten ans Land
zu kommen.

\hypertarget{c-uxfcberwinterung-auf-der-insel-malta-weiterreise-nach-rom}{%
\paragraph{c) Überwinterung auf der Insel Malta; Weiterreise nach
Rom}\label{c-uxfcberwinterung-auf-der-insel-malta-weiterreise-nach-rom}}

\hypertarget{section-27}{%
\section{28}\label{section-27}}

\bibleverse{1} Jetzt, nach unserer Rettung, erfuhren wir, daß die Insel
Malta hieß. \bibleverse{2} Die fremdsprachigen Eingeborenen erwiesen uns
eine außerordentliche Menschenfreundlichkeit; denn sie zündeten einen
Holzstoß an und gaben uns allen wegen des eingetretenen Regens und wegen
der Kälte einen Platz (am Feuer).

\hypertarget{rettung-des-paulus-aus-lebensgefahr}{%
\paragraph{Rettung des Paulus aus
Lebensgefahr}\label{rettung-des-paulus-aus-lebensgefahr}}

\bibleverse{3} Als aber Paulus einen Haufen Reisig zusammenraffte und
ihn auf den Holzstoß ins Feuer legte, fuhr eine Otter infolge der Hitze
heraus und biß sich in seine Hand fest. \bibleverse{4} Als nun die
Eingeborenen das Tier an seiner Hand hängen sahen, sagten sie
zueinander: »Dieser Mensch muß ein Mörder sein, den die Göttin der
Vergeltung trotz seiner Rettung aus dem Meer nicht am Leben lassen
will.« \bibleverse{5} Er schleuderte jedoch das Tier von sich ab ins
Feuer, und es widerfuhr ihm nichts Schlimmes. \bibleverse{6} Jene
warteten zwar darauf, daß er anschwellen oder plötzlich tot niederfallen
werde; als sie aber geraume Zeit gewartet hatten und nichts Unheilvolles
an ihm vorgehen sahen, änderten sie ihre Meinung und sagten, er müsse
ein Gott sein.

\hypertarget{paulus-heilt-den-vater-des-publius-und-andere-kranke}{%
\paragraph{Paulus heilt den Vater des Publius und andere
Kranke}\label{paulus-heilt-den-vater-des-publius-und-andere-kranke}}

\bibleverse{7} Nun besaß in der Nähe jenes Ortes der vornehmste Mann✲
der Insel namens Publius Landgüter; dieser nahm uns bei sich auf und
beherbergte uns drei Tage lang freundlich. \bibleverse{8} Der Vater des
Publius aber lag gerade an Fieberanfällen und an der Ruhr krank
darnieder. Paulus ging nun zu ihm ins Zimmer, legte ihm unter Gebet die
Hände auf und machte ihn dadurch gesund. \bibleverse{9} Infolgedessen
kamen auch die anderen Inselbewohner, die an Krankheiten litten, zu ihm
und ließen sich heilen. \bibleverse{10} Dafür erwies man uns denn auch
viele Ehren und versah uns bei unserer Abfahrt mit allem, was wir nötig
hatten.

\hypertarget{weiterreise-uxfcber-syrakus-und-puteoli-nach-rom}{%
\paragraph{Weiterreise über Syrakus und Puteoli nach
Rom}\label{weiterreise-uxfcber-syrakus-und-puteoli-nach-rom}}

\bibleverse{11} Nach einem Vierteljahr fuhren wir dann auf einem
alexandrinischen Schiff ab, das auf der Insel überwintert hatte und als
Wahrzeichen das Bild der Dioskuren führte. \bibleverse{12} Wir landeten
hierauf in Syrakus, wo wir drei Tage blieben. \bibleverse{13} Von dort
fuhren wir, im Bogen segelnd, nach Regium weiter und gelangten, da am
folgenden Tage der Südwind einsetzte, schon in einer Fahrt von zwei
Tagen nach Puteoli. \bibleverse{14} Hier trafen wir Brüder an, die uns
baten, sieben Tage bei ihnen zu bleiben; so gelangten wir denn nach Rom.
\bibleverse{15} Von dort kamen uns die Brüder, die über uns schon Kunde
erhalten hatten, bis Forum Appii\textless sup title=``d.h. Markt des
Appius''\textgreater✲ und Tres Tabernä\textless sup title=``d.h. die
drei Schenken''\textgreater✲ entgegen; bei ihrem Anblick sprach Paulus
ein Dankgebet zu Gott und faßte neuen Mut.

\hypertarget{d-paulus-in-rom}{%
\paragraph{d) Paulus in Rom}\label{d-paulus-in-rom}}

\bibleverse{16} Nach unserer Ankunft in Rom {[}aber übergab der
Hauptmann seine Gefangenen dem Befehlshaber der kaiserlichen
Leibwache;{]} Paulus aber erhielt die Erlaubnis, mit dem ihn bewachenden
Soldaten eine eigene (Miets-) Wohnung zu beziehen.

\hypertarget{verhandlungen-des-paulus-mit-den-huxe4uptern-der-ruxf6mischen-juden}{%
\paragraph{Verhandlungen des Paulus mit den Häuptern der römischen
Juden}\label{verhandlungen-des-paulus-mit-den-huxe4uptern-der-ruxf6mischen-juden}}

\bibleverse{17} Nach drei Tagen lud er dann die Vornehmsten\textless sup
title=``oder: Häupter''\textgreater✲ der Juden zu sich ein; und als sie
sich eingefunden hatten, richtete er folgende Worte an sie: »Werte
Brüder! Obgleich ich nichts Feindseliges gegen unser Volk und die
Gebräuche der Väter begangen habe, bin ich doch als Gefangener von
Jerusalem her den Römern in die Hände geliefert worden. \bibleverse{18}
Diese wollten mich nach angestellter richterlicher Untersuchung
freilassen, weil keine todeswürdige Schuld bei mir vorlag;
\bibleverse{19} weil jedoch die Juden Widerspruch erhoben, sah ich mich
gezwungen, die Entscheidung des Kaisers anzurufen, nicht als ob ich
gegen mein Volk eine Anklage vorzubringen hätte. \bibleverse{20} Aus
diesem Grunde also habe ich euch zu mir gebeten, um euch zu sehen und
mich mit euch zu besprechen; denn um der Hoffnung Israels willen habe
ich diese Kette zu tragen.« \bibleverse{21} Sie gaben ihm zur Antwort:
»Wir haben weder Zuschriften über dich aus Judäa erhalten, noch ist
irgendein Bruder dagewesen, der etwas Nachteiliges über dich berichtet
oder ausgesagt hätte. \bibleverse{22} Wir halten es aber für billig, von
dir über deine Ansichten Näheres zu erfahren; denn von dieser
Sonderrichtung\textless sup title=``oder: Sekte''\textgreater✲ ist uns
(allerdings) bekannt, daß sie überall auf Widerspruch stößt.«

\bibleverse{23} So bestimmten sie ihm denn einen Tag und fanden sich (an
diesem) bei ihm in seiner Wohnung in noch größerer Anzahl ein (als das
erste Mal). Da legte er ihnen von früh morgens bis spät abends das Reich
Gottes dar und bezeugte es ihnen, indem er sie im Anschluß sowohl an das
mosaische Gesetz als an die Propheten für Jesus zu gewinnen suchte.
\bibleverse{24} Ein Teil von ihnen ließ sich auch durch seine
Darlegungen überzeugen, die anderen dagegen blieben ungläubig.
\bibleverse{25} Ohne also zu einer Einigung miteinander gelangt zu sein,
trennten sie sich, nachdem Paulus noch das eine Wort an sie gerichtet
hatte: »Treffend hat der heilige Geist durch den Propheten Jesaja zu
euren Vätern gesagt\textless sup title=``Jes 6,9-10''\textgreater✲:
\bibleverse{26} ›Gehe zu diesem Volk und sprich: Ihr werdet immerfort
hören und doch kein Verständnis erlangen, und ihr werdet immerfort sehen
und doch nicht wahrnehmen. \bibleverse{27} Denn das Herz dieses Volkes
ist verhärtet, und ihre Ohren sind schwerhörig geworden, und ihre Augen
haben sie geschlossen, damit sie mit ihren Augen nicht sehen und mit
ihren Ohren nicht hören und mit ihrem Herzen nicht zum Verständnis
gelangen, so daß sie sich bekehren und ich sie heile.‹ \bibleverse{28}
So sei euch denn kundgetan, daß diese Rettung\textless sup title=``oder:
dieses Heil''\textgreater✲ Gottes den Heiden gesandt worden ist:
\bibleverse{29} die werden ihr\textless sup title=``oder:
ihm''\textgreater✲ auch Gehör schenken!«

\hypertarget{des-paulus-zweijuxe4hriges-wirken-in-der-gefangenschaft-zu-rom}{%
\paragraph{Des Paulus zweijähriges Wirken in der Gefangenschaft zu
Rom}\label{des-paulus-zweijuxe4hriges-wirken-in-der-gefangenschaft-zu-rom}}

\bibleverse{30} Paulus blieb dann zwei volle Jahre in einer eigenen
Mietswohnung und nahm (daselbst) alle auf, die ihn besuchten;
\bibleverse{31} er verkündigte dabei das Reich Gottes und erteilte
Belehrung über den Herrn Jesus Christus mit vollem Freimut, ungehindert.
