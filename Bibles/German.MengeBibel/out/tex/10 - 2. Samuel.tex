\hypertarget{das-zweite-buch-samuel}{%
\section{DAS ZWEITE BUCH SAMUEL}\label{das-zweite-buch-samuel}}

\hypertarget{i.-davids-klage-um-saul-und-jonathan-bei-der-nachricht-von-deren-tod}{%
\subsection{I. Davids Klage um Saul und Jonathan bei der Nachricht von
deren
Tod}\label{i.-davids-klage-um-saul-und-jonathan-bei-der-nachricht-von-deren-tod}}

\hypertarget{section}{%
\section{1}\label{section}}

1Nach Sauls Tode nun, als David von dem Siege über die Amalekiter (nach
Ziklag) zurückgekehrt war und zwei Tage in Ziklag zugebracht hatte, 2da
kam plötzlich am dritten Tage ein Mann aus dem Lager von Saul her mit
zerrissenen Kleidern und mit Erde auf dem Haupt; und als er bei David
angekommen war, warf er sich vor ihm zur Erde nieder und brachte ihm
seine Huldigung dar.

\hypertarget{a-bericht-des-boten-uxfcber-sauls-letzte-augenblicke}{%
\paragraph{a) Bericht des Boten über Sauls letzte
Augenblicke}\label{a-bericht-des-boten-uxfcber-sauls-letzte-augenblicke}}

3Auf Davids Frage, woher er komme, gab er die Antwort: »Aus dem Lager
der Israeliten bin ich entronnen.« 4Als David ihn dann aufforderte:
»Erzähle mir doch, wie die Sache sich zugetragen hat!«, berichtete er:
»Das Heer ist aus der Schlacht geflohen, und viele von den Leuten sind
gefallen und ums Leben gekommen; auch Saul und sein Sohn Jonathan sind
tot.« 5Als nun David den jungen Mann, der ihm die Meldung brachte,
weiter fragte: »Woher weißt du, daß Saul und sein Sohn Jonathan tot
sind?«, 6berichtete ihm der junge Mann, der ihm die Meldung brachte:
»Ich kam ganz zufällig auf das Gebirge Gilboa; da sah ich plötzlich
Saul, der sich auf seinen Speer stützte, während die Wagen und Reiter
ihn eingeholt hatten\textless sup title=``oder: auf ihn zu
jagten''\textgreater✲. 7Als er sich nun umwandte und mich erblickte,
rief er mich an, und ich antwortete: ›Hier bin ich!‹ 8Er fragte mich
dann, wer ich sei, und ich erwiderte ihm: ›Ich bin ein Amalekiter!‹ 9Da
befahl er mir: ›Tritt her zu mir und töte mich vollends\textless sup
title=``=~gib mir den Todesstoß''\textgreater✲! Denn ein Schwindel hat
mich ergriffen, und doch bin ich noch bei vollem Bewußtsein.‹ 10Da trat
ich zu ihm und tötete ihn vollends\textless sup title=``=~gab ihm den
Todesstoß''\textgreater✲; denn ich wußte, daß er seinen Fall nicht
überleben würde. Dann nahm ich den Stirnreif\textless sup title=``=~das
Diadem''\textgreater✲ von seinem Haupt und die Spange, die sich an
seinem Arm befand, und überbringe sie hier meinem Herrn.«

\hypertarget{b-davids-trauer-tuxf6tung-des-boten}{%
\paragraph{b) Davids Trauer; Tötung des
Boten}\label{b-davids-trauer-tuxf6tung-des-boten}}

11Da faßte David seine Kleider und zerriß sie; ebenso taten alle Männer,
die bei ihm waren; 12sie hielten die Totenklage und weinten und fasteten
bis zum Abend um Saul und seinen Sohn Jonathan, um das Volk des HERRN
und um das Haus Israel, weil sie durch das Schwert gefallen waren.
13Hierauf fragte David den jungen Mann, der ihm die Meldung gebracht
hatte: »Woher bist du?« Er antwortete: »Ich bin der Sohn eines
amalekitischen Fremdlings✲.« 14Da rief David aus: »Wie? Du hast dich
nicht gescheut, deine Hand zu erheben, um dem Gesalbten des HERRN das
Leben zu nehmen?!« 15Hierauf rief David einen von seinen Leuten herbei
und befahl ihm: »Tritt herzu! Stoß ihn nieder!« Der hieb ihn nieder, daß
er starb, 16während David ihm noch zurief: »Dein Blut komme über dein
Haupt! Denn dein eigener Mund hat Zeugnis gegen dich
abgelegt\textless sup title=``=~hat das Urteil über dich
gesprochen''\textgreater✲ durch dein Bekenntnis: ›Ich habe den Gesalbten
des HERRN getötet.‹«

\hypertarget{c-davids-klagelied-uxfcber-saul-und-jonathan}{%
\paragraph{c) Davids Klagelied über Saul und
Jonathan}\label{c-davids-klagelied-uxfcber-saul-und-jonathan}}

17David stimmte dann folgendes Klagelied auf Saul und dessen Sohn
Jonathan an 18und befahl, man solle es {[}das Bogenlied{]} die Söhne
Judas lehren\textless sup title=``=~auswendig lernen
lassen''\textgreater✲; es steht bekanntlich aufgezeichnet im ›Buch des
Braven‹\textless sup title=``vgl. Jos 10,13''\textgreater✲:

19Ach, (deine) Zier, o Israel, liegt erschlagen auf deinen Höhen wie
sind die Helden gefallen! 20Verkündet es nicht zu Gath, meldet es nicht
auf Askalons Straßen, damit die Töchter der Philister sich nicht freun,
die Töchter der Heiden nicht jubeln!

21Ihr Berge Gilboas, kein Tau, kein Regen müsse noch auf euch fallen,
kein Gefilde auf euch sein, von dem Erstlingsgaben kommen! Denn dort
liegt der Schild der Helden weggeworfen\textless sup title=``oder:
besudelt''\textgreater✲, der Schild Sauls, den kein Öl mehr salben wird.

22Ohne Blut der Durchbohrten, ohne Fett der Helden ist Jonathans Bogen
nie zurückgekommen und Sauls Schwert nie ohne Beute heimgekehrt.

23Saul und Jonathan, die Geliebten und Holden, im Leben und auch im Tod
sind vereint sie geblieben; sie waren schneller als Adler, stärker als
Löwen!

24Ihr Töchter Israels, weinet um Saul, der in Purpur euch köstlich
gekleidet, der Goldschmuck auf eure Gewänder geheftet!

25Wie sind die Helden gefallen mitten im Kampf! Jonathan liegt
durchbohrt auf deinen Höhen!

26Wie ist mir leid um dich, mein Bruder Jonathan, wie warst du mir so
lieb\textless sup title=``oder: hold''\textgreater✲! Beseligend war mir
deine Liebe, mehr als Frauenliebe!

27Ach, wie sind die Helden gefallen, vernichtet die Rüstzeuge des
Kampfes!

\hypertarget{ii.-david-als-kuxf6nig-von-juda-in-hebron-21-55}{%
\subsection{II. David als König von Juda in Hebron
(2,1-5,5)}\label{ii.-david-als-kuxf6nig-von-juda-in-hebron-21-55}}

\hypertarget{david-wird-kuxf6nig-uxfcber-den-stamm-juda-isboseth-uxfcber-israel}{%
\subsubsection{1. David wird König über den Stamm Juda, Isboseth über
Israel}\label{david-wird-kuxf6nig-uxfcber-den-stamm-juda-isboseth-uxfcber-israel}}

\hypertarget{section-1}{%
\section{2}\label{section-1}}

1Hierauf fragte David beim HERRN an: »Soll ich in eine der Städte Judas
hinaufziehen?« Der HERR antwortete ihm: »Ja, ziehe hinauf.« Als David
weiter fragte: »Wohin soll ich ziehen?«, erhielt er die Antwort: »Nach
Hebron.« 2So zog denn David dorthin samt seinen beiden Frauen Ahinoam
aus Jesreel und Abigail, der Witwe Nabals, aus Karmel; 3auch seine
Kriegsleute, die bei ihm waren, ließ er alle samt ihren Familien
hinaufziehen, und sie ließen sich in den zu Hebron gehörigen Ortschaften
nieder. 4Da kamen die Männer von Juda und salbten David dort zum König
über den Stamm Juda.

\hypertarget{davids-botschaft-an-die-einwohner-von-jabes}{%
\paragraph{Davids Botschaft an die Einwohner von
Jabes}\label{davids-botschaft-an-die-einwohner-von-jabes}}

Als man dann David die Nachricht brachte, daß die Männer von Jabes in
Gilead es seien, die Saul begraben hätten, 5da sandte David Boten an die
Einwohner von Jabes in Gilead und ließ ihnen sagen: »Der Segen des HERRN
möge euch dafür zuteil werden, daß ihr Saul, eurem Herrn, diese Liebe
erwiesen und ihn begraben habt! 6So möge nun der HERR euch Güte und
Treue erweisen! Und auch ich will euch dafür belohnen, daß ihr so
gehandelt habt. 7Jetzt aber seid guten Mutes und beweist euch als
wackere Männer! Denn Saul, euer Herr, ist (zwar) tot, aber der Stamm
Juda hat mich zum König über sich gesalbt.«

\hypertarget{sauls-sohn-isboseth-wird-kuxf6nig-von-israel}{%
\paragraph{Sauls Sohn Isboseth wird König von
Israel}\label{sauls-sohn-isboseth-wird-kuxf6nig-von-israel}}

8Abner aber, der Sohn Ners, der Heerführer Sauls, nahm Sauls Sohn
Isboseth, brachte ihn nach Mahanaim hinüber 9und machte ihn dort zum
König über Gilead, über Asser, über Jesreel, Ephraim, Benjamin,
überhaupt über ganz Israel. 10Vierzig Jahre war Isboseth, der Sohn
Sauls, alt, als er König über Israel wurde, und zwei Jahre lang hat er
regiert; nur der Stamm Juda hielt zu David. 11Die Zeit aber, die David
als König über den Stamm Juda in Hebron regiert hat, betrug sieben Jahre
und sechs Monate.

\hypertarget{krieg-zwischen-david-und-isboseth}{%
\subsubsection{2. Krieg zwischen David und
Isboseth}\label{krieg-zwischen-david-und-isboseth}}

\hypertarget{a-kampfspiel-und-schlacht-bei-gibeon-joabs-sieg}{%
\paragraph{a) Kampfspiel und Schlacht bei Gibeon; Joabs
Sieg}\label{a-kampfspiel-und-schlacht-bei-gibeon-joabs-sieg}}

12Einst zog Abner, der Sohn Ners, mit den Leuten Isboseths, des Sohnes
Sauls, von Mahanaim nach Gibeon; 13ebenso rückte Joab, der Sohn der
Zeruja, mit den Leuten Davids (von Hebron) aus; beide Heere stießen dann
am Teiche von Gibeon aufeinander und lagerten sich, das eine diesseits,
das andere jenseits des Teiches. 14Da ließ Abner dem Joab sagen: »Die
jungen Leute könnten ja einmal auftreten und ein Kampfspiel vor unsern
Augen aufführen!« Joab ließ antworten: »Gut! Es soll geschehen!« 15Da
machten sie sich auf und traten einander abgezählt paarweise gegenüber:
zwölf aus Benjamin für Isboseth, den Sohn Sauls, und zwölf von den
Leuten Davids. 16Da faßte jeder seinen Gegner beim Schopf und stieß ihm
das Schwert in die Seite, so daß sie insgesamt fielen; daher nannte man
jenen Ort ›Feld der Klingen‹; er liegt bei Gibeon. 17Als sich darauf an
diesem Tage ein überaus erbitterter Kampf entspann, wurde Abner und das
Heer der Israeliten von den Leuten Davids in die Flucht geschlagen.

\hypertarget{b-asahel-der-juxfcngere-bruder-joabs-bei-der-verfolgung-von-abner-getuxf6tet}{%
\paragraph{b) Asahel, der jüngere Bruder Joabs, bei der Verfolgung von
Abner
getötet}\label{b-asahel-der-juxfcngere-bruder-joabs-bei-der-verfolgung-von-abner-getuxf6tet}}

18Es befanden sich aber dort (im Heere Davids) die drei Söhne der
Zeruja: Joab, Abisai und Asahel, von denen Asahel schnellfüßig war wie
eine Gazelle auf dem Felde. 19Als nun Asahel den Abner verfolgte, ohne
weder nach rechts noch nach links bei der Verfolgung hinter Abner
abzubiegen, 20wandte Abner sich um und rief: »Bist du es, Asahel?« Er
antwortete: »Jawohl.« 21Da rief Abner ihm zu: »Wende dich doch nach
links oder nach rechts (von mir) ab und mache dich an einen von den
Leuten und nimm dir seine Rüstung!« Aber Asahel wollte nicht von ihm
ablassen. 22Da rief Abner ihm noch einmal zu: »Gehe hinter mir weg!
Warum soll ich dich zu Boden schlagen? Wie könnte ich dann noch deinem
Bruder Joab vor die Augen treten?« 23Als er trotzdem die Verfolgung
nicht aufgeben wollte, stieß ihn Abner mit dem unteren Ende seines
Speeres in den Unterleib, so daß der Speer hinten herausdrang und er
dort zu Boden stürzte und an jener Stelle starb. Alle aber, die zu der
Stelle kamen, wo Asahel gefallen und gestorben war, blieben stehen.

\hypertarget{c-ende-der-verfolgung-fortdauer-des-krieges}{%
\paragraph{c) Ende der Verfolgung; Fortdauer des
Krieges}\label{c-ende-der-verfolgung-fortdauer-des-krieges}}

24Joab aber und Abisai setzten die Verfolgung Abners fort und waren bei
Sonnenuntergang bis zum Hügel Amma gelangt, der Giah gegenüber in der
Richtung\textless sup title=``oder: am Wege''\textgreater✲ nach der
Wüste Gibeon zu liegt. 25Hier sammelten sich die Benjaminiten hinter
Abner her, bildeten eine geschlossene Schar und setzten sich oben auf
dem Hügel fest. 26Nun rief Abner dem Joab die Worte zu: »Soll denn das
Schwert ewig fressen? Siehst du nicht, daß ein Verzweiflungskampf das
Ende sein wird? Wann wirst du endlich deinen Leuten befehlen, von der
Verfolgung ihrer Brüder abzustehen?« 27Joab erwiderte: »So wahr Gott
lebt! Hättest du nicht geredet, so hätten die Leute insgesamt erst
morgen früh von der Verfolgung ihrer Brüder abgelassen!« 28Hierauf ließ
Joab ein Zeichen mit der Posaune geben, und sofort machte das ganze Heer
halt, gab die weitere Verfolgung der Israeliten auf und setzte den Kampf
nicht fort. 29Abner marschierte nun mit seinen Leuten während der ganzen
folgenden Nacht durch die Jordanebene, überschritt dann den Jordan,
durchzog das ganze Tal Bithron und gelangte so nach Mahanaim.~-- 30Als
Joab aber die Verfolgung Abners aufgegeben und seine ganze Mannschaft
wieder gesammelt hatte, wurden von den Leuten Davids außer Asahel nur
neunzehn Mann vermißt, 31während Davids Leute von den Benjaminiten und
den übrigen Leuten Abners 360~Mann erschlagen hatten. 32Den Asahel aber
hob man auf und begrub ihn im Begräbnis seines Vaters zu Bethlehem. Joab
aber und seine Leute marschierten die ganze Nacht hindurch, bis sie bei
Tagesanbruch in Hebron ankamen.

\hypertarget{section-2}{%
\section{3}\label{section-2}}

1Der Krieg zwischen dem Hause Sauls und dem Hause Davids zog sich dann
in die Länge; aber Davids Macht nahm immerfort zu, während das Haus
Sauls immer schwächer wurde.

\hypertarget{davids-familie-in-hebron}{%
\paragraph{Davids Familie in Hebron}\label{davids-familie-in-hebron}}

2In Hebron wurden dem David folgende Söhne geboren\textless sup
title=``vgl. 1.Chr 3,1-4''\textgreater✲: sein Erstgeborener war Amnon,
von Ahinoam aus Jesreel; 3sein zweiter Sohn Kileab, von Abigail, der
Witwe Nabals, aus Karmel; der dritte Absalom, der Sohn der Maacha, der
Tochter Thalmais, des Königs von Gesur; 4der vierte Adonia, der Sohn der
Haggith; der fünfte Sephatja, der Sohn der Abital; 5der sechste
Jithream, von Davids Frau Egla. Diese wurden dem David in Hebron
geboren.

\hypertarget{abners-verrat-und-ermordung}{%
\subsubsection{3. Abners Verrat und
Ermordung}\label{abners-verrat-und-ermordung}}

\hypertarget{a-abners-zerwuxfcrfnis-mit-isboseth}{%
\paragraph{a) Abners Zerwürfnis mit
Isboseth}\label{a-abners-zerwuxfcrfnis-mit-isboseth}}

6Solange nun der Krieg zwischen dem Hause Sauls und dem Hause Davids
währte, hielt Abner treu zum Hause Sauls. 7Saul hatte aber ein Nebenweib
gehabt namens Rizpa, eine Tochter Ajas. Nun richtete (Isboseth) eines
Tages die Frage an Abner: »Warum hast du ein Verhältnis mit dem
Nebenweibe meines Vaters gehabt?« 8Über diese Äußerung Isboseths geriet
Abner in heftigen Zorn, so daß er ausrief: »Bin ich denn ein judäischer
Hundskopf?! Bis heute habe ich dem Hause deines Vaters Saul und seinen
Verwandten und Freunden treu gedient und dich nicht in Davids Hände
fallen lassen, und doch wirfst du mir heute eine Verschuldung mit einem
Weibe vor? 9So möge denn Gott mich, Abner, jetzt und künftighin strafen,
wenn ich nicht das, was der HERR dem David zugeschworen hat, jetzt für
ihn verwirkliche, 10daß ich nämlich das Königtum dem Hause Sauls nehme
und den Thron Davids über Israel und über Juda aufrichte von Dan bis
Beerseba!« 11Da vermochte jener dem Abner kein Wort mehr zu erwidern: so
fürchtete er sich vor ihm.

\hypertarget{b-abners-verhandlungen-mit-david-und-mit-den-huxe4uptern-israels}{%
\paragraph{b) Abners Verhandlungen mit David und mit den Häuptern
Israels}\label{b-abners-verhandlungen-mit-david-und-mit-den-huxe4uptern-israels}}

12Darauf sandte Abner auf der Stelle Boten zu David und ließ (ihm)
sagen: »Wem gehört das Land? Schließe einen Vertrag mit mir, dann stelle
ich mich dir zur Verfügung, um ganz Israel auf deine Seite zu bringen.«
13(David) gab zur Antwort: »Gut! Ich will einen Vertrag mit dir
schließen; doch fordere ich eins von dir, nämlich: du darfst mir nicht
vor die Augen treten, es sei denn, daß du Sauls Tochter Michal
mitbringst, wenn du herkommst, um vor mir zu erscheinen.« 14Zugleich
schickte David aber auch Gesandte an Isboseth, den Sohn Sauls, mit der
Forderung: »Gib mir meine Gattin Michal zurück, die ich mir um den Preis
von hundert Vorhäuten der Philister zur Frau gewonnen habe!« 15Da sandte
Isboseth hin und ließ sie ihrem Gatten Paltiel, dem Sohne des Lais,
wegnehmen. 16Ihr Gatte aber ging mit ihr und begleitete sie unter
beständigem Weinen bis Bachurim; dann, als Abner ihn aufforderte,
heimzugehen, kehrte er nach Hause zurück.

17Abner aber hatte mit den Ältesten von Israel Verhandlungen angeknüpft
und ihnen vorgestellt: »Schon längst habt ihr David zum König über euch
gewünscht: 18so führt es denn jetzt aus! Denn der HERR hat im Hinblick
auf David verheißen: ›Durch die Hand meines Knechtes David will ich mein
Volk Israel aus der Gewalt der Philister und aller seiner Feinde
befreien.‹« 19Ebenso hatte Abner auch mit den Benjaminiten heimlich
unterhandelt; und endlich machte er sich auf den Weg, um David in Hebron
persönlich alles mitzuteilen, was Israel und der ganze Stamm Benjamin
beschlossen hatten.

\hypertarget{c-abners-zusammenkunft-mit-david-in-hebron-seine-ermordung-durch-joab}{%
\paragraph{c) Abners Zusammenkunft mit David in Hebron; seine Ermordung
durch
Joab}\label{c-abners-zusammenkunft-mit-david-in-hebron-seine-ermordung-durch-joab}}

20Als nun Abner in Begleitung von zwanzig Männern zu David nach Hebron
kam und David ihm und seinen Begleitern ein Festmahl hatte ausrichten
lassen, 21sagte Abner zu David: »Ich will mich jetzt aufmachen und
hingehen und ganz Israel um meinen Herrn, den König, sammeln, damit sie
einen Vertrag mit dir schließen; dann kannst du als König ganz nach
Herzenswunsch regieren!« Darauf entließ David den Abner, der sich in
Frieden✲ entfernte.

22Da aber kamen gerade Davids Leute und Joab von einem Streifzug zurück
und brachten reiche Beute mit, als Abner sich bereits nicht mehr bei
David in Hebron befand, sondern dieser ihn entlassen hatte, so daß er in
Frieden heimzog. 23Als nun Joab mit der ganzen Mannschaft, die bei ihm
war, heimkam und man ihm berichtete: »Abner, der Sohn Ners, ist zum
König gekommen, und der hat ihn ruhig wieder ziehen lassen«, 24da ging
Joab zum König hinein und sagte: »Was hast du da getan? Ich weiß, Abner
ist bei dir gewesen: warum hast du ihn nun ungehindert wieder ziehen
lassen? 25Du mußt doch Abner, den Sohn Ners, so weit kennen, daß er nur
gekommen ist, um dich zu überlisten! Er will nur all dein Tun und Lassen
auskundschaften und alles, was du vorhast, erkunden.« 26Als Joab dann
David verlassen hatte, sandte er Boten hinter Abner her, die ihn von der
Zisterne Sira wieder zurückholen mußten, ohne daß David etwas davon
wußte. 27Als Abner nun nach Hebron zurückkam, führte Joab ihn abseits in
das Innere des Tores, als wollte er vertraulich mit ihm reden; dort
versetzte er ihm einen Stich in den Unterleib, daß er starb -- um die
Blutrache für seinen Bruder Asahel zu vollziehen.

\hypertarget{d-abner-von-david-betrauert-und-ehrenvoll-bestattet-davids-unschuldserkluxe4rung-abschluuxdf}{%
\paragraph{d) Abner von David betrauert und ehrenvoll bestattet; Davids
Unschuldserklärung;
Abschluß}\label{d-abner-von-david-betrauert-und-ehrenvoll-bestattet-davids-unschuldserkluxe4rung-abschluuxdf}}

28Als David hernach die Sache erfuhr, rief er aus: »Ich und mein
Königtum sind für immer schuldlos vor dem HERRN an der Ermordung Abners,
des Sohnes Ners! 29(Sein Blut) falle auf das Haupt Joabs und auf sein
ganzes Geschlecht zurück! Möge es im Hause Joabs nie an Schwindsüchtigen
und Aussätzigen fehlen, nie an solchen, die an Krücken gehen, die durchs
Schwert fallen und denen es an Brot mangelt!«

30Als nun Joab und sein Bruder Abisai Abner ermordet hatten, weil er
ihren Bruder Asahel bei Gibeon im Kampf getötet hatte, 31befahl David
dem Joab und allen Personen, die zu seiner Umgebung gehörten: »Zerreißt
eure Kleider, legt Trauergewänder an und stimmt die Totenklage an vor
Abners Bahre her!« Der König David aber ging hinter der Bahre her; 32und
als man Abner in Hebron begrub, brach der König am Grabe Abners in
lautes Weinen aus, und das ganze Volk weinte mit. 33Sodann stimmte der
König ein Klagelied um Abner an, das da lautete: Mußte Abner sterben,
wie ein Gottloser stirbt? 34Deine Hände waren nicht gefesselt und deine
Füße nicht in Ketten gelegt: nein, du bist gefallen, wie man vor
Verbrechern fällt! Da weinte alles Volk noch mehr um ihn. 35Als dann
alles Volk\textless sup title=``=~alle Anwesenden''\textgreater✲ kam, um
David zum Essen zu bewegen, solange es noch Tag war, schwur David: »Gott
strafe mich jetzt und künftig, wenn ich vor Sonnenuntergang Brot oder
irgend etwas anderes genieße!« 36Als alles Volk dies wahrnahm, gefiel es
ihnen wohl; überhaupt fand das ganze Verhalten des Königs beim gesamten
Volke Beifall; 37und das ganze Volk, auch ganz Israel, kam an jenem Tage
zu der Erkenntnis, daß die Ermordung Abners, des Sohnes Ners, nicht vom
Könige ausgegangen war. 38Auch sagte der König zu seiner Umgebung: »Wißt
ihr nicht, daß am heutigen Tage ein Fürst und Großer\textless sup
title=``oder: ein großer Feldherr''\textgreater✲ in Israel gefallen ist?
39Ich aber, obgleich zum König gesalbt, bin jetzt noch zu machtlos, und
diese Männer, die Söhne der Zeruja, sind zu hart für mich\textless sup
title=``=~mir durch ihre Gewalttätigkeit überlegen''\textgreater✲. Der
HERR lasse den, der die Freveltat begangen hat, für seinen Frevel
büßen!«

\hypertarget{ermordung-isboseths-davids-kruxf6nung-zum-kuxf6nig-von-ganz-israel}{%
\subsubsection{4. Ermordung Isboseths; Davids Krönung zum König von ganz
Israel}\label{ermordung-isboseths-davids-kruxf6nung-zum-kuxf6nig-von-ganz-israel}}

\hypertarget{section-3}{%
\section{4}\label{section-3}}

1Als nun Sauls Sohn (Isboseth) erfuhr, daß Abner in Hebron ums Leben
gekommen war, entsank ihm der Mut, und ganz Israel geriet in Bestürzung.
2Nun hatte Sauls Sohn Isboseth zwei Männer als Anführer von
Streifscharen, der eine hieß Baana, der andere Rechab; beide waren Söhne
des Benjaminiten Rimmon von Beeroth -- auch Beeroth wird nämlich zu
Benjamin gerechnet; 3jedoch waren die Einwohner von Beeroth nach
Gitthaim ausgewandert und sind dort als Schutzbürger bis auf den
heutigen Tag seßhaft geblieben.~-- 4Sauls Sohn Jonathan aber hatte einen
Sohn, der an beiden Füßen lahm war; er war fünf Jahre alt gewesen, als
die Nachricht vom Tode Sauls und Jonathans aus Jesreel anlangte; da
hatte seine Wärterin ihn auf den Arm genommen und die Flucht ergriffen;
aber infolge der Eile bei der Flucht war er ihr entfallen und dadurch
lahm geworden; sein Name war Mephiboseth. 5Die Söhne Rimmons von Beeroth
also, Rechab und Baana, machten sich auf und begaben sich zur Zeit der
Mittagshitze in das Haus Isboseths, der eben seine Mittagsruhe hielt.
6Die Türhüterin des Hauses war nämlich gerade beim Reinigen von Weizen
eingenickt und schlief; daher konnten sich die beiden Brüder in das Haus
einschleichen, 7drangen in das Schlafgemach ein, wo Isboseth auf seinem
Bette lag; sie schlugen ihn tot und hieben ihm den Kopf ab.

\hypertarget{david-bestraft-die-muxf6rder-und-ehrt-den-toten-isboseth}{%
\paragraph{David bestraft die Mörder und ehrt den toten
Isboseth}\label{david-bestraft-die-muxf6rder-und-ehrt-den-toten-isboseth}}

Dann nahmen sie seinen Kopf, eilten während der ganzen Nacht durch die
Jordanebene, 8kamen mit dem Kopfe Isboseths zu David nach Hebron und
sagten zum König: »Hier hast du den Kopf Isboseths, des Sohnes Sauls,
deines Feindes, der dir nach dem Leben getrachtet hat! Aber Gott, der
HERR, hat heute meinem Herrn, dem König, Rache an Saul und seinen
Nachkommen gewährt!« 9Da antwortete David dem Rechab und seinem Bruder
Baana, den Söhnen Rimmons von Beeroth: »So wahr der HERR lebt, der mich
aus aller Bedrängnis errettet hat: 10ich habe den Mann, der mir die
Nachricht von Sauls Tode brachte und sich für einen Glücksboten hielt,
in Ziklag ergreifen und töten lassen und ihm so seinen Botenlohn
gezahlt! 11Wieviel mehr, wenn ruchlose Menschen einen unschuldigen Mann
in seinem eigenen Hause auf seiner Lagerstätte ermordet haben: sollte
ich da jetzt nicht sein Blut von euch fordern und euch vom Erdboden
vertilgen?!« 12Hierauf gab David seinen Leibwächtern Befehl; die hieben
sie nieder, schlugen ihnen die Hände und Füße ab und hängten sie am
Teiche zu Hebron auf. Den Kopf Isboseths aber nahmen sie und begruben
ihn in der Grabstätte Abners zu Hebron.

\hypertarget{david-von-suxe4mtlichen-israeliten-in-hebron-zum-kuxf6nig-gesalbt}{%
\paragraph{David von sämtlichen Israeliten in Hebron zum König
gesalbt}\label{david-von-suxe4mtlichen-israeliten-in-hebron-zum-kuxf6nig-gesalbt}}

\hypertarget{section-4}{%
\section{5}\label{section-4}}

1Hierauf fanden sich sämtliche Stämme der Israeliten bei David in Hebron
ein und sagten: »Wir sind ja doch von deinem Gebein und Fleisch. 2Schon
früher, als Saul noch unser König war, bist du es gewesen, der Israel
ins Feld und wieder heim geführt hat; dazu hat der HERR dir verheißen:
›Du sollst mein Volk Israel weiden, und du sollst Fürst über Israel
sein.‹« 3Als so alle Ältesten der Israeliten zum König nach Hebron
gekommen waren, schloß der König David einen Vertrag mit ihnen in Hebron
vor dem Angesicht des HERRN; dann salbten sie David zum König über
Israel.

4Dreißig Jahre war David alt, als er König wurde, und vierzig Jahre hat
er regiert. 5Zu Hebron hat er sieben Jahre und sechs Monate über Juda
regiert, und in Jerusalem hat er dreiunddreißig Jahre über ganz Israel
und Juda regiert.

\hypertarget{iii.-david-in-jerusalem-56-2022}{%
\subsection{III. David in Jerusalem
(5,6-20,22)}\label{iii.-david-in-jerusalem-56-2022}}

\hypertarget{die-eroberung-jerusalems-davids-bauten-vergruxf6uxdferung-seiner-familie-siege-uxfcber-die-philister-56-25}{%
\subsubsection{1. Die Eroberung Jerusalems; Davids Bauten; Vergrößerung
seiner Familie; Siege über die Philister
5,6-25}\label{die-eroberung-jerusalems-davids-bauten-vergruxf6uxdferung-seiner-familie-siege-uxfcber-die-philister-56-25}}

\hypertarget{a-david-erobert-jerusalem-und-macht-es-zur-reichshauptstadt-und-zu-seiner-residenz}{%
\paragraph{a) David erobert Jerusalem und macht es zur Reichshauptstadt
und zu seiner
Residenz}\label{a-david-erobert-jerusalem-und-macht-es-zur-reichshauptstadt-und-zu-seiner-residenz}}

6Als hierauf der König mit seinem Heere vor Jerusalem gegen die
Jebusiter zog, welche die dortige Gegend bewohnten, sagte man zu David:
»Hier wirst du nicht eindringen, sondern die Blinden und Lahmen werden
dich vertreiben«✲; damit wollte man sagen: »David wird hier nicht
eindringen.« 7Aber David eroberte die Burg Zion {[}das ist die jetzige
›Davidsstadt‹{]}. 8An jenem Tage sagte David: »Wer die Jebusiter
schlägt, indem er den Schacht hinaufsteigt, und ›die Lahmen und die
Blinden‹, denen die Seele Davids feind ist~\ldots« Daher rührt das
Sprichwort: »Ein Blinder und ein Lahmer darf uns nicht ins
Haus\textless sup title=``=~in den Tempel?''\textgreater✲ kommen.«
9David nahm dann seinen Wohnsitz in der Burg und nannte sie ›Stadt
Davids‹; auch führte David Bauten ringsum auf, von der Burg Millo an
nach innen zu. 10Seine Macht wuchs nun immer mehr, weil der HERR, der
Gott der Heerscharen, mit ihm war.

\hypertarget{b-seine-bauten-mit-hilfe-hirams-von-tyrus-vergruxf6uxdferung-der-zahl-seiner-frauen-seine-in-jerusalem-geborenen-kinder}{%
\paragraph{b) Seine Bauten (mit Hilfe Hirams von Tyrus); Vergrößerung
der Zahl seiner Frauen; seine in Jerusalem geborenen
Kinder}\label{b-seine-bauten-mit-hilfe-hirams-von-tyrus-vergruxf6uxdferung-der-zahl-seiner-frauen-seine-in-jerusalem-geborenen-kinder}}

11Und Hiram, der König von Tyrus, schickte Gesandte an David mit
Zedernstämmen, dazu Zimmerleute und Steinmetzen, damit sie David ein
Haus\textless sup title=``=~einen Palast''\textgreater✲ bauten. 12Daran
erkannte David, daß der HERR ihn als König über Israel bestätigt und daß
er sein Königtum zu Ansehen gebracht habe um seines Volkes Israel
willen.

13Darauf nahm sich David noch mehr Nebenweiber und Frauen in Jerusalem,
nachdem er von Hebron dorthin gekommen war, und es wurden ihm noch mehr
Söhne und Töchter geboren. 14Dies sind die Namen der Söhne, die ihm in
Jerusalem geboren wurden: Sammua, Sobab, Nathan, Salomo, 15Jibhar,
Elisua, Nepheg, Japhia, 16Elisama, Eljada und Eliphelet.

\hypertarget{c-seine-zwei-siegreichen-kuxe4mpfe-mit-den-philistern}{%
\paragraph{c) Seine zwei siegreichen Kämpfe mit den
Philistern}\label{c-seine-zwei-siegreichen-kuxe4mpfe-mit-den-philistern}}

17Als aber die Philister vernahmen, daß man David zum König über (ganz)
Israel gesalbt hatte, zogen die Philister insgesamt heran, um seiner
habhaft zu werden. Aber David erhielt Kunde davon und zog nach der
Bergfeste (Adullam) hinab. 18Als nun die Philister herankamen und sich
in der Ebene Rephaim ausbreiteten, 19richtete David die Anfrage an den
HERRN: »Soll ich gegen die Philister hinaufziehen? Wirst du sie in meine
Hand geben?« Der HERR antwortete ihm: »Ziehe hinauf, ich will die
Philister unfehlbar in deine Hand geben.« 20Da zog David nach
Baal-Perazim; und als er sie dort geschlagen hatte, rief er aus: »Der
HERR hat meine Feinde vor mir her durchbrochen, wie das Wasser einen
Damm durchbricht!« Darum hat man jenem Ort den Namen
Baal-Perazim\textless sup title=``d.h. Ort der
Durchbrüche''\textgreater✲ gegeben. 21Da (die Philister) ihre
Götzenbilder dort zurückgelassen hatten, nahmen David und seine Leute
sie als Beute weg.

22Die Philister zogen dann nochmals herauf und breiteten sich in der
Ebene Rephaim aus. 23Als David nun den HERRN befragte, antwortete
dieser: »Du sollst nicht hinaufziehen ihnen entgegen, sondern umgehe
sie, damit du ihnen in den Rücken fällst! Greife sie vom Baka-Gehölz her
an! 24Sobald du dann in den Wipfeln des Baka-Gehölzes das Geräusch von
Schritten vernimmst, dann beeile dich! Denn alsdann ist der HERR vor dir
her ausgezogen, um das Heer der Philister zu schlagen.« 25Da tat David
so, wie der HERR ihm geboten hatte, und richtete ein Blutbad unter den
Philistern an von Geba✲ bis in die Gegend von Geser hin.

\hypertarget{uxfcberfuxfchrung-der-bundeslade-nach-dem-zion-in-jerusalem}{%
\subsubsection{2. Überführung der Bundeslade nach dem Zion in
Jerusalem}\label{uxfcberfuxfchrung-der-bundeslade-nach-dem-zion-in-jerusalem}}

\hypertarget{a-miuxdflingen-des-ersten-versuchs}{%
\paragraph{a) Mißlingen des ersten
Versuchs}\label{a-miuxdflingen-des-ersten-versuchs}}

\hypertarget{section-5}{%
\section{6}\label{section-5}}

1David ließ dann nochmals alle Auserlesenen in Israel zusammenkommen,
30000~Mann; 2dann machte er sich auf und zog mit dem gesamten Volke, das
bei ihm war, nach Baala in Juda, um von dort die Lade Gottes
heraufzuholen, die den Namen führt nach Gott, dem HERRN der Heerscharen,
der über den Cheruben thront. 3Man lud die Lade Gottes auf einen neuen
Wagen und brachte sie so hinweg aus dem Hause Abinadabs, das auf der
Anhöhe lag\textless sup title=``1.Sam 7,1''\textgreater✲; und zwar
lenkten Ussa und Ahjo, die Söhne Abinadabs, den neuen Wagen 4und
brachten die Lade weg aus dem Hause Abinadabs, das auf der Anhöhe lag,
indem Ussa neben der Lade Gottes herging, während Ahjo vor der Lade
einherschritt. 5David aber und alle Israeliten tanzten vor dem HERRN her
mit Aufbietung aller Kräfte und mit Liedern unter Begleitung von Zithern
und Harfen, Handpauken, Schellen und Zimbeln. 6Als sie nun so zur Tenne
Nachons gekommen waren, griff Ussa mit der Hand nach der Lade Gottes und
hielt sie fest, weil die Rinder zu Fall gekommen waren. 7Da entbrannte
der Zorn des HERRN gegen Ussa, und Gott schlug ihn dort wegen seiner
Verfehlung, so daß er dort neben der Lade Gottes starb. 8Da wurde David
tief betrübt, daß der HERR einen solchen Schlag gegen Ussa geführt
hatte; daher nannte man jenen Ort Perez-Ussa\textless sup title=``d.h.
Ussas Schlag''\textgreater✲ bis auf den heutigen Tag. 9David aber geriet
an jenem Tage in Furcht vor dem HERRN, so daß er ausrief: »Wie kann da
die Lade des HERRN zu mir kommen?!« 10Weil David also die Lade des HERRN
nicht zu sich in die Davidsstadt bringen wollte, ließ er sie abseits in
das Haus des Gathiters Obed-Edom schaffen. 11So blieb denn die Lade des
HERRN ein Vierteljahr lang im Hause des Gathiters Obed-Edom stehen; der
HERR aber segnete Obed-Edom und sein ganzes Haus.

\hypertarget{b-feierliche-uxfcberfuxfchrung-der-lade-nach-jerusalem-opfer--und-dankfest-des-volkes}{%
\paragraph{b) Feierliche Überführung der Lade nach Jerusalem; Opfer- und
Dankfest des
Volkes}\label{b-feierliche-uxfcberfuxfchrung-der-lade-nach-jerusalem-opfer--und-dankfest-des-volkes}}

12Als man nun dem König David berichtete, der HERR habe das Haus
Obed-Edoms und sein ganzes Besitztum um der Lade Gottes willen gesegnet,
da ging David hin und holte die Lade Gottes voller Freude aus dem Hause
Obed-Edoms nach der Davidsstadt hinauf 13und opferte dabei, als die
Träger der Lade des HERRN sechs Schritte gegangen waren, ein Rind und
ein Mastkalb. 14Auch tanzte David mit Aufbietung aller Kraft vor dem
HERRN her, wobei er nur mit einem leinenen Schulterkleid umgürtet war.
15So brachte David mit allen Israeliten die Lade des HERRN unter
Jauchzen und Posaunenschall (nach Jerusalem) hinauf. 16Da begab es sich,
als die Lade des HERRN in die Davidsstadt einzog, daß Sauls Tochter
Michal zum Fenster hinausschaute; als sie nun den König David so vor dem
HERRN her hüpfen und tanzen sah, empfand sie Verachtung für ihn in ihrem
Herzen. 17Nachdem man dann die Lade des HERRN hineingebracht und sie an
ihren Platz inmitten des Zeltes, das von David für sie aufgeschlagen
worden war, niedergesetzt hatte, brachte David Brand- und Heilsopfer vor
dem HERRN dar, 18segnete dann, als er mit der Darbringung der Brand- und
Heilsopfer fertig war, das Volk im Namen des HERRN der Heerscharen 19und
ließ unter das ganze Volk, an sämtliche Israeliten, sowohl an Männer wie
an Frauen, an einen jeden einen Brotkuchen, ein Stück Fleisch und einen
Rosinenkuchen austeilen. Darauf kehrte jedermann aus dem Volk nach Hause
zurück.

\hypertarget{c-davids-edles-verhalten-und-demuxfctige-erkluxe4rung-gegen-michal}{%
\paragraph{c) Davids edles Verhalten und demütige Erklärung gegen
Michal}\label{c-davids-edles-verhalten-und-demuxfctige-erkluxe4rung-gegen-michal}}

20Als nun David heimkehrte, um seine Familie zu begrüßen, trat Sauls
Tochter Michal ihm mit den Worten entgegen: »Wie würdevoll hat sich
heute der König von Israel benommen, indem er sich heute vor den Augen
der Mägde seiner Untertanen entblößt hat, wie es sonst nur gemeine Leute
tun!« 21Da erwiderte David der Michal: »Vor dem HERRN, der mich vor
deinem Vater und vor dessen ganzem Hause erwählt hat, um mich zum
Fürsten über das Volk des HERRN, über Israel, zu bestellen -- ja vor dem
HERRN will ich tanzen, 22auch wenn ich mich dadurch noch tiefer
erniedrige als diesmal, und will demütig von mir denken; aber bei den
Mägden, von denen du redest -- bei ihnen werde ich an Ehre gewinnen.«
23Michal aber, Sauls Tochter, blieb bis an ihren Todestag kinderlos.

\hypertarget{davids-plan-eines-tempelbaues-von-gott-verworfen-die-grouxdfe-verheiuxdfung-fuxfcr-david-und-sein-haus}{%
\subsubsection{3. Davids Plan eines Tempelbaues von Gott verworfen; die
große Verheißung für David und sein
Haus}\label{davids-plan-eines-tempelbaues-von-gott-verworfen-die-grouxdfe-verheiuxdfung-fuxfcr-david-und-sein-haus}}

\hypertarget{a-nathan-billigt-davids-plan-des-tempelbaus}{%
\paragraph{a) Nathan billigt Davids Plan des
Tempelbaus}\label{a-nathan-billigt-davids-plan-des-tempelbaus}}

\hypertarget{section-6}{%
\section{7}\label{section-6}}

1Als nun der König in seinem Hause✲ wohnte, nachdem der HERR ihm Ruhe
vor all seinen Feinden ringsum verschafft hatte, 2sagte der König (eines
Tages) zu dem Propheten Nathan: »Bedenke doch, ich wohne hier in einem
Zedernpalast, während die Lade Gottes hinter Zelttüchern steht!« 3Da
antwortete Nathan dem König: »Wohlan, führe alles aus, was du im Sinn
hast! Denn der HERR ist mit dir!«

\hypertarget{b-gottes-verwerfung-des-planes-nathans-prophetische-rede-der-tempelbau-soll-von-davids-sohn-ausgefuxfchrt-werden}{%
\paragraph{b) Gottes Verwerfung des Planes; Nathans prophetische Rede;
der Tempelbau soll von Davids Sohn ausgeführt
werden}\label{b-gottes-verwerfung-des-planes-nathans-prophetische-rede-der-tempelbau-soll-von-davids-sohn-ausgefuxfchrt-werden}}

4Aber noch in derselben Nacht erging das Wort des HERRN an Nathan
folgendermaßen: 5»Gehe hin und sage zu meinem Knecht, zu David: ›So hat
der HERR gesprochen: Solltest du mir ein Haus zu meiner Wohnung bauen?
6Ich habe ja doch in keinem Hause gewohnt seit der Zeit, als ich die
Israeliten aus Ägypten hergeführt habe, bis auf den heutigen Tag,
sondern bin in einer Zeltwohnung mit umhergewandert. 7Habe ich etwa,
solange ich unter allen Israeliten umhergezogen bin, zu einem von den
Richtern Israels, denen ich mein Volk Israel zu weiden geboten hatte,
jemals auch nur ein Wort gesagt: Warum habt ihr mir kein Zedernhaus
gebaut?‹ 8Darum sollst du jetzt zu meinem Knecht, zu David, folgendes
sagen: ›So hat der HERR der Heerscharen gesprochen: Ich habe dich von
der Weide hinter der Herde weggeholt, damit du Fürst über mein Volk,
über Israel, sein solltest; 9und ich bin bei allem, was du unternommen
hast, mit dir gewesen und habe alle deine Feinde vor dir her ausgerottet
und habe dir einen großen Namen geschaffen, wie ihn nur die Größten auf
Erden haben; 10und ich will meinem Volk Israel eine Stätte anweisen und
es da fest einpflanzen, damit es an seiner Stätte ruhig wohnen kann und
sich nicht mehr zu ängstigen braucht und gewalttätige Menschen es nicht
mehr bedrücken wie früher, 11seit der Zeit, wo ich Richter über mein
Volk Israel bestellt habe; sondern ich will dir Ruhe vor allen deinen
Feinden verschaffen; und der HERR verkündigt dir, daß der HERR dir ein
Haus bauen wird. 12Wenn einst deine Tage voll sind und du dich zu deinen
Vätern gelegt hast, dann will ich nach deinem Tode deinen leiblichen
Sohn zu deinem Nachfolger erheben und ihm sein Königtum bestätigen.
13Der soll dann meinem Namen ein haus bauen, und ich will seinen
Königsthron feststellen für immer.‹«

\hypertarget{c-gottes-grouxdfe-heilsverkuxfcndigung-an-david-betreffs-der-ewigen-dauer-seines-hauses}{%
\paragraph{c) Gottes große Heilsverkündigung an David betreffs der
ewigen Dauer seines
Hauses}\label{c-gottes-grouxdfe-heilsverkuxfcndigung-an-david-betreffs-der-ewigen-dauer-seines-hauses}}

14»Ich will ihm Vater sein, und er soll mir Sohn sein, so daß, wenn er
sich verfehlt, ich ihn mit einer menschlichen Rute und mit menschlichen
Schlägen züchtigen werde; 15aber meine Gnade soll nicht von ihm weichen,
wie ich sie von Saul, deinem Vorgänger, habe weichen lassen. 16Nein,
dein Haus und dein Königtum sollen für immer Bestand vor mir haben: dein
Thron soll feststehen für immer!‹«

\hypertarget{d-davids-dank--und-bittgebet}{%
\paragraph{d) Davids Dank- und
Bittgebet}\label{d-davids-dank--und-bittgebet}}

17Nachdem Nathan diesen Worten und dieser Offenbarung genau entsprechend
zu David geredet hatte, 18ging der König David (in das Gotteszelt)
hinein, warf sich vor dem HERRN nieder und betete: »Wer bin ich, HERR
mein Gott, und was ist mein Haus, daß du mich bis hierher\textless sup
title=``=~so weit''\textgreater✲ gebracht hast! 19Und dies hast du für
noch nicht genügend gehalten, HERR mein Gott; denn jetzt hast du auch
noch in bezug auf das Haus deines Knechtes Verheißungen für ferne Zeiten
gegeben, und zwar nach Menschenweise, HERR mein Gott. 20Was soll da
David noch weiter zu dir sagen? Du selbst kennst ja deinen Knecht, HERR
mein Gott. 21Um deines Knechtes willen und nach deinem Herzen✲ hast du
gehandelt, indem du all dieses Große deinem Knecht kundgetan hast.
22Darum bist du groß, HERR mein Gott; ja, niemand ist dir gleich, und es
gibt keinen Gott außer dir nach allem, was wir mit unsern Ohren
vernommen haben. 23Und wo ist ein anderes Volk, das deinem Volk Israel
gleicht? Es ist das einzige Volk auf Erden, um deswillen Gott
hingegangen ist, es sich zum Eigentumsvolk zu erkaufen und ihm einen
Namen zu schaffen und ihm zugut so große Dinge und wunderbare Taten zu
vollführen, indem du vor deinem Volk, das du dir aus Ägypten befreit
hast, Heidenvölker und ihre Götter vertrieben hast. 24So hast du dir
denn dein Volk Israel für alle Zeiten zum Volk bestätigt\textless sup
title=``oder: bestimmt''\textgreater✲, und du, HERR, bist ihr Gott
geworden. 25Und nun, HERR mein Gott: mache die Verheißung, die du in
betreff deines Knechtes und seines Hauses ausgesprochen hast, für alle
Zeiten wahr und verfahre so, wie du zugesagt hast! 26Dann wird dein Name
für immer groß sein, daß man aussprechen wird: ›Der HERR der Heerscharen
ist der Gott für Israel‹, und das Haus deines Knechtes David wird
Bestand vor dir haben. 27Denn du selbst, HERR der Heerscharen, Gott
Israels, hast deinem Knecht die Offenbarung zuteil werden lassen: ›Ich
will dir ein Haus bauen‹; darum hat dein Knecht den Mut gefunden, dieses
Gebet an dich zu richten. 28Nun denn, HERR mein Gott: du bist Gott, und
deine Worte sind Wahrheit! Nachdem du deinem Knecht diese herrliche
Zusage gemacht hast, 29so möge es dir nun auch gefallen, das Haus deines
Knechtes zu segnen, damit es immerdar vor dir bestehe! Denn du selbst,
HERR mein Gott, hast es verheißen. So wird denn das Haus deines Knechtes
durch deinen Segen auf ewig gesegnet sein!«

\hypertarget{davids-kriege-mit-den-nachbarvuxf6lkern-und-seine-obersten-beamten}{%
\subsubsection{4. Davids Kriege mit den Nachbarvölkern und seine
obersten
Beamten}\label{davids-kriege-mit-den-nachbarvuxf6lkern-und-seine-obersten-beamten}}

\hypertarget{a-davids-siege-uxfcber-die-philister-moabiter-und-syrer}{%
\paragraph{a) Davids Siege über die Philister, Moabiter und
Syrer}\label{a-davids-siege-uxfcber-die-philister-moabiter-und-syrer}}

\hypertarget{section-7}{%
\section{8}\label{section-7}}

1Nachmals besiegte David die Philister und demütigte sie und machte so
der Oberherrschaft der Philister ein Ende.~-- 2Er besiegte auch die
Moabiter und maß sie mit der Meßschnur ab, indem er sie auf die Erde
niederlegen ließ; und zwar maß er zwei Schnurlängen ab, um (die
betreffenden) zu töten, und eine volle Schnurlänge für die, welche am
Leben bleiben sollten. So wurden die Moabiter Davids tributpflichtige
Untertanen.~-- 3Sodann besiegte David Hadad-Eser, den Sohn Rehobs, den
König von Zoba, als dieser ausgezogen war, um seine Herrschaft am
Euphrat wiederherzustellen. 4Dabei machte David 1700~Reiter\textless sup
title=``oder: Wagenkämpfer''\textgreater✲ von ihm und 20000~Mann Fußvolk
zu Gefangenen und ließ die Gespanne\textless sup title=``oder:
Wagenpferde''\textgreater✲ sämtlich lähmen; nur hundert Wagenpferde
behielt er von diesen für sich.~-- 5Als dann die Syrer von Damaskus dem
König Hadad-Eser von Zoba zu Hilfe kamen, erschlug David von den Syrern
22000~Mann 6und setzte dann Vögte im damascenischen Syrien ein, so daß
die dortigen Syrer zu tributpflichtigen Untertanen Davids wurden. So
verlieh der HERR dem David den Sieg auf allen Zügen, die er unternahm.

\hypertarget{b-die-beute-und-ihre-verwendung-gluxfcckwunsch-des-kuxf6nigs-thoi}{%
\paragraph{b) Die Beute und ihre Verwendung; Glückwunsch des Königs
Thoi}\label{b-die-beute-und-ihre-verwendung-gluxfcckwunsch-des-kuxf6nigs-thoi}}

7Auch erbeutete David die goldenen Schilde, welche die Dienstleute✲
Hadad-Esers getragen hatten, und ließ sie nach Jerusalem bringen; 8und
in Betah\textless sup title=``oder: Tebah''\textgreater✲ und Berothai,
den Städten Hadad-Esers, fiel dem König David sehr viel Kupfer in die
Hände.~-- 9Als aber Thoi, der König von Hamath, erfuhr, daß David die
ganze Heeresmacht Hadad-Esers geschlagen hatte, 10sandte er seinen Sohn
Joram\textless sup title=``oder: Hadoran''\textgreater✲ zum König David,
um ihn zu begrüßen und ihm Glück zu wünschen, daß er aus dem Kriege mit
Hadad-Eser als Sieger hervorgegangen war -- Thoi hatte nämlich auf dem
Kriegsfuß mit Hadad-Eser gestanden --; und jener brachte silberne,
goldene und kupferne Geräte\textless sup title=``oder:
Kunstgegenstände''\textgreater✲ mit, 11die der König David ebenfalls dem
HERRN weihte, wie er es auch mit dem Silber und Gold machte, das ihm bei
allen unterworfenen Völkern in die Hände gefallen war, 12nämlich bei den
Syrern, Moabitern, Ammonitern, Philistern und Amalekitern, sowie mit der
Beute, die ihm bei Hadad-Eser, dem Sohne Rehobs, dem König von Zoba, in
die Hände gefallen war.

\hypertarget{c-besiegung-und-unterwerfung-der-edomiter}{%
\paragraph{c) Besiegung und Unterwerfung der
Edomiter}\label{c-besiegung-und-unterwerfung-der-edomiter}}

13Als David dann von dem Siege über die Syrer zurückkehrte, gewann er
neuen Ruhm, indem er die Edomiter im Salztal besiegte und 18000~Mann von
ihnen erschlug. 14Auch über alle Teile des Landes der Edomiter setzte er
Vögte ein, so daß das ganze Land Edom ihm untertan wurde. So verlieh der
HERR dem David den Sieg überall, wohin er zog.

\hypertarget{d-verzeichnis-der-obersten-beamten-davids}{%
\paragraph{d) Verzeichnis der obersten Beamten
Davids}\label{d-verzeichnis-der-obersten-beamten-davids}}

15So herrschte denn David über ganz Israel und ließ Recht und
Gerechtigkeit in seinem ganzen Volke walten.~-- 16Joab, der Sohn der
Zeruja, stand an der Spitze des Heeres; Josaphat, der Sohn Ahiluds, war
Kanzler; 17Zadok, der Sohn Ahitubs, und Abjathar, der Sohn Ahimelechs,
waren Priester, Seraja Staatsschreiber, 18Benaja, der Sohn Jojadas,
Befehlshaber der (Leibwache der) Krethi und Plethi; Priesterrang hatten
auch die Söhne Davids.

\hypertarget{davids-grouxdfmut-gegen-jonathans-sohn-mephiboseth}{%
\subsubsection{5. Davids Großmut gegen Jonathans Sohn
Mephiboseth}\label{davids-grouxdfmut-gegen-jonathans-sohn-mephiboseth}}

\hypertarget{a-zibas-auskunft-uxfcber-mephiboseth}{%
\paragraph{a) Zibas Auskunft über
Mephiboseth}\label{a-zibas-auskunft-uxfcber-mephiboseth}}

\hypertarget{section-8}{%
\section{9}\label{section-8}}

1David fragte: »Ist noch jemand von Sauls Familie übriggeblieben? Ich
will Barmherzigkeit an ihm üben um Jonathans willen.« 2Nun war im Hause
Sauls ein Diener namens Ziba; den berief man zu David, und der König
fragte ihn: »Bist du Ziba?« Er antwortete: »Ja, dein Knecht.« 3Der König
fragte weiter: »Ist niemand mehr da von Sauls Familie, daß ich ihm
Gottes Barmherzigkeit erweisen könnte?« Da antwortete Ziba dem König:
»Doch, es lebt noch ein Sohn Jonathans, der an den Füßen lahm ist.« 4Als
der König ihn nun fragte, wo dieser sich befinde, erwiderte Ziba dem
König: »Er befindet sich zu Lodebar im Hause Machirs, des Sohnes
Ammiels.«

\hypertarget{b-davids-grouxdfmuxfctige-verfuxfcgungen-bezuxfcglich-mephiboseths}{%
\paragraph{b) Davids großmütige Verfügungen bezüglich
Mephiboseths}\label{b-davids-grouxdfmuxfctige-verfuxfcgungen-bezuxfcglich-mephiboseths}}

5Da sandte der König David hin und ließ ihn aus Lodebar aus dem Hause
Machirs, des Sohnes Ammiels, holen. 6Als nun Mephiboseth, der Sohn
Jonathans, des Sohnes Sauls, zu David kam, warf er sich vor ihm auf sein
Angesicht nieder und brachte ihm seine Huldigung dar. Da sagte David zu
ihm: »Mephiboseth!« Er antwortete: »Hier ist dein Knecht!« 7Dann fuhr
David fort: »Fürchte dich nicht! Denn ich will dir um deines Vaters
Jonathan willen Barmherzigkeit erweisen und dir den ganzen Grundbesitz
deines Großvaters Saul zurückgeben; du selbst aber sollst allezeit an
meinem Tische speisen.« 8Da verneigte sich jener und sagte: »Was ist
dein Knecht, daß du deine Gnade einem toten Hunde zuwendest, wie ich
einer bin!«

9Darauf ließ der König den Diener Sauls, Ziba, kommen und sagte zu ihm:
»Alles, was Saul und seinem ganzen Hause gehört hat, habe ich dem Sohne
deines (früheren) Herrn zurückgegeben. 10Du aber sollst ihm das Feld
bestellen, du mit deinen Söhnen und Knechten, und sollst die Ernte
einbringen, damit (das Haus) des Sohnes deines (früheren) Herrn zu leben
hat; Mephiboseth aber, der Sohn deines (früheren) Herrn, wird regelmäßig
an meinem Tische speisen.« Ziba hatte aber fünfzehn Söhne und zwanzig
Knechte. 11Da antwortete Ziba dem König: »Ganz so, wie mein Herr und
König seinem Knechte befiehlt, wird dein Knecht es ausrichten.« So
speiste denn Mephiboseth an Davids Tisch, als wäre er einer von den
Königssöhnen\textless sup title=``=~königlichen Prinzen''\textgreater✲.
12Mephiboseth aber hatte einen kleinen Sohn namens Micha, und alle, die
im Hause Zibas wohnten\textless sup title=``d.h. alle Hausgenossen
Zibas''\textgreater✲, waren nun Knechte in Mephiboseths Diensten.
13Mephiboseth selbst aber wohnte in Jerusalem, weil er regelmäßig an der
Tafel des Königs speiste; er war aber an beiden Füßen lahm.

\hypertarget{davids-krieg-gegen-die-ammoniter-und-syrer}{%
\subsubsection{6. Davids Krieg gegen die Ammoniter und
Syrer}\label{davids-krieg-gegen-die-ammoniter-und-syrer}}

\hypertarget{a-das-schmachvolle-vergehen-der-ammoniter-gegen-davids-gesandte}{%
\paragraph{a) Das schmachvolle Vergehen der Ammoniter gegen Davids
Gesandte}\label{a-das-schmachvolle-vergehen-der-ammoniter-gegen-davids-gesandte}}

\hypertarget{section-9}{%
\section{10}\label{section-9}}

1Danach begab es sich, daß (Nahas), der König der Ammoniter, starb und
sein Sohn Hanun ihm in der Regierung nachfolgte. 2Da dachte David: »Ich
will mich freundlich gegen Hanun, den Sohn des Nahas, beweisen, wie sein
Vater sich mir gegenüber freundlich gezeigt hat.« So schickte denn David
hin, um ihm durch seine Diener✲ sein Beileid zum Tode seines Vaters
ausdrücken zu lassen. Als aber die Diener Davids im Lande der Ammoniter
angekommen waren, 3sagten die Fürsten der Ammoniter zu Hanun, ihrem
Herrn: »Glaubst du etwa, daß David Beileidsgesandte deshalb geschickt
hat, um deinem Vater eine Ehre zu erweisen? Nein, offenbar hat David
seine Diener nur deshalb zu dir geschickt, um die Stadt\textless sup
title=``=~Hauptstadt Rabba''\textgreater✲ zu erforschen und
auszukundschaften und sie dann zu zerstören.« 4Da ließ Hanun die
Gesandten Davids festnehmen, ließ ihnen den Bart halb abscheren und die
Röcke\textless sup title=``oder: Gewänder''\textgreater✲ halb
abschneiden bis unter den Gürtel und entließ sie dann. 5Als man dies dem
David meldete, schickte er ihnen Boten entgegen -- denn die Männer waren
schwer beschimpft --; und der König ließ ihnen sagen: »Bleibt in
Jericho, bis euch der Bart wieder gewachsen ist: dann kehrt heim!«

\hypertarget{b-ausbruch-des-krieges-erster-sieg-joabs}{%
\paragraph{b) Ausbruch des Krieges; erster Sieg
Joabs}\label{b-ausbruch-des-krieges-erster-sieg-joabs}}

6Als nun die Ammoniter sahen, daß sie David tödlich beleidigt hatten,
sandten sie hin und nahmen die Syrer von Beth-Rehob und die Syrer von
Zoba in Sold, 20000~Mann Fußvolk, ebenso den König von Maacha mit
1000~Mann und 12000~Mann von Istob. 7Sobald David Kunde davon erhielt,
ließ er Joab mit dem ganzen Heer, auch den Helden, ausrücken. 8Die
Ammoniter zogen dann (aus der Stadt) hinaus und stellten sich vor dem
Stadttor in Schlachtordnung auf, während die Syrer von Zoba und Rehob
und die Mannschaften von Istob und Maacha für sich im freien Felde
standen. 9Als nun Joab sah, daß ihm sowohl von vorn als auch im Rücken
ein Angriff drohe, nahm er aus allen auserlesenen israelitischen
Kriegern eine Auswahl vor und stellte sich mit ihnen den Syrern
gegenüber auf; 10den Rest des Heeres aber überwies er seinem Bruder
Abisai, und dieser mußte sich mit ihnen den Ammonitern gegenüber
aufstellen. 11Dann sagte er: »Wenn die Syrer mir überlegen sind, so
kommst du mir zu Hilfe; und wenn die Ammoniter dir überlegen sind, so
komme ich dir zu Hilfe. 12Nur Mut! Wir wollen tapfer kämpfen für unser
Volk und für die Städte unsers Gottes! Der HERR aber möge tun, was ihm
wohlgefällt!« 13Darauf rückte Joab mit der Mannschaft, die unter seinem
Befehl stand, zum Angriff gegen die Syrer vor, und diese wandten sich
vor ihm zur Flucht; 14und als die Ammoniter die Flucht der Syrer
gewahrten, flohen auch sie vor Abisai und zogen sich in die Stadt
zurück. Darauf ließ Joab vom Kampf gegen die Ammoniter ab und kehrte
nach Jerusalem zurück.

\hypertarget{c-david-selbst-im-felde-sein-sieg-uxfcber-die-mit-den-ammonitern-verbuxfcndeten-syrer}{%
\paragraph{c) David selbst im Felde; sein Sieg über die mit den
Ammonitern verbündeten
Syrer}\label{c-david-selbst-im-felde-sein-sieg-uxfcber-die-mit-den-ammonitern-verbuxfcndeten-syrer}}

15Als sich nun die Syrer von den Israeliten geschlagen sahen, sammelten
sie sich wieder in voller Zahl; 16und Hadad-Eser sandte hin und ließ die
Syrer von jenseits des Euphrats ins Feld rücken, und diese kamen unter
Führung Sobachs, des Heerführers Hadad-Esers, nach Helam. 17Auf die
Kunde hiervon bot David ganz Israel zum Kriege auf, überschritt den
Jordan und gelangte nach Helam, wo die Syrer sich ihm entgegenstellten
und ihm eine Schlacht lieferten. 18Aber die Syrer wurden von den
Israeliten in die Flucht geschlagen, und David erschlug den Syrern die
Bemannung von 700 Kriegswagen und 40000~Mann Fußvolk; auch ihren
Heerführer Sobach verwundete er, so daß er dort starb. 19Als nun
sämtliche Könige, die dem Hadad-Eser Heerfolge leisteten, sich von den
Israeliten besiegt sahen, schlossen sie Frieden mit den Israeliten und
unterwarfen sich ihnen. Infolgedessen scheuten sich die Syrer, den
Ammonitern fernerhin noch Hilfe zu leisten.

\hypertarget{davids-tiefer-fall-und-buuxdfe}{%
\subsubsection{7. Davids tiefer Fall und
Buße}\label{davids-tiefer-fall-und-buuxdfe}}

\hypertarget{a-davids-versuxfcndigung-ehebruch-mit-bathseba}{%
\paragraph{a) Davids Versündigung (Ehebruch) mit
Bathseba}\label{a-davids-versuxfcndigung-ehebruch-mit-bathseba}}

\hypertarget{section-10}{%
\section{11}\label{section-10}}

1Im folgenden Jahre aber sandte David zu der Zeit, wo die Könige ins
Feld zu ziehen pflegen, Joab samt seinen Hauptleuten und der Heeresmacht
von ganz Israel aus. Sie verwüsteten das Land der Ammoniter und
belagerten Rabba, während David in Jerusalem geblieben war.

2Da begab es sich eines Abends, daß David sich von seinem Lager erhob
und, als er auf dem Dache des königlichen Palastes umherging, vom Dache
aus eine Frau sich baden sah; die Frau war von ungewöhnlicher Schönheit.
3Als er sich nun durch Boten nach der Frau erkundigen ließ und man ihm
berichtete, daß es Bathseba, die Tochter Eliams, die Frau des Hethiters
Uria sei, 4sandte David Boten hin und ließ sie holen. Sie kam zu ihm,
und er wohnte ihr bei -- sie hatte sich aber eben von ihrer
Verunreinigung gereinigt --; darauf kehrte sie in ihre Wohnung zurück.

\hypertarget{b-urias-musterhaftes-verhalten-wuxe4hrend-seines-aufenthalts-im-palaste-davids}{%
\paragraph{b) Urias musterhaftes Verhalten während seines Aufenthalts im
Palaste
Davids}\label{b-urias-musterhaftes-verhalten-wuxe4hrend-seines-aufenthalts-im-palaste-davids}}

5Als die Frau dann guter Hoffnung wurde und dem David Mitteilung von
ihrem Zustande machte, 6da ließ David dem Joab sagen: »Schicke mir den
Hethiter Uria her!«, und Joab kam dem Befehle nach. 7Als nun Uria zu
David kam, erkundigte dieser sich nach dem Befinden Joabs, nach dem
Ergehen des Heeres und nach dem Stande des Krieges. 8Darauf sagte David
zu Uria: »Gehe jetzt in dein Haus hinunter und nimm ein Fußbad«; und als
Uria den Palast des Königs verließ, wurde eine königliche Ehrenmahlzeit
hinter ihm hergetragen; 9aber Uria legte sich am Eingang des
Königspalastes bei allen übrigen Dienern seines Herrn nieder und ging
nicht in sein Haus hinunter. 10Als man nun dem König meldete, Uria sei
nicht in sein Haus hinabgegangen, fragte ihn David: »Du bist doch von
der Reise heimgekommen: warum gehst du nicht in deine Wohnung?« 11Da
antwortete Uria dem Könige: »Die Lade sowie Israel und Juda sind in
Hütten\textless sup title=``oder: Zelten''\textgreater✲ untergebracht,
und mein Herr✲ Joab und die Diener✲ meines Herrn müssen auf freiem Felde
lagern, und da sollte ich in mein Haus gehen, um zu essen und zu
trinken, und sollte es mir bei meiner Frau wohl sein lassen? So wahr der
HERR lebt und so wahr du selbst lebst: das tue ich nicht!« 12Darauf
sagte David zu Uria: »Du magst auch heute noch hier bleiben: morgen
werde ich dich entlassen.« So blieb denn Uria an diesem Tage noch in
Jerusalem. 13Am folgenden Tage aber lud David ihn ein, bei ihm zu essen
und zu trinken, und er machte ihn trunken; aber am Abend ging Uria
wieder hin, um sich auf sein Lager bei den übrigen Leuten seines Herrn
schlafen zu legen, und ging nicht in sein Haus hinunter.

\hypertarget{c-der-uriasbrief-urias-tod-joabs-meldung-an-david-bescheid-des-kuxf6nigs}{%
\paragraph{c) Der Uriasbrief; Urias Tod; Joabs Meldung an David;
Bescheid des
Königs}\label{c-der-uriasbrief-urias-tod-joabs-meldung-an-david-bescheid-des-kuxf6nigs}}

14Am nächsten Morgen aber schrieb David einen Brief an Joab und ließ ihn
durch Uria überbringen. 15In dem Briefe hatte er folgendes geschrieben:
»Stellt Uria vornhin, wo am hitzigsten gekämpft wird, und zieht euch
dann hinter ihm zurück, damit er erschlagen wird und den Tod findet.«
16So stellte denn Joab bei der Belagerung der Stadt den Uria an eine
Stelle, von der er wußte, daß dort tapfere Gegner standen. 17Als dann
die Städter einen Ausfall machten und mit Joab handgemein wurden, fielen
manche von der Mannschaft, von den Leuten Davids; und auch der Hethiter
Uria fand dabei den Tod. 18Als hierauf Joab an David einen Bericht über
den ganzen Verlauf des Kampfes schickte, 19gab er dem Boten den Befehl:
»Wenn du dem König den ganzen Verlauf des Kampfes bis zu Ende berichtet
hast 20und der König dann vor Zorn aufbraust und dich fragt: ›Warum seid
ihr zum Angriff so nahe an die Stadt herangerückt? Wußtet ihr nicht, daß
man von der Mauer herab schießen würde? 21Wer hat Abimelech, den Sohn
Jerubbeseths, erschlagen? Hat nicht ein Weib den oberen Stein einer
Handmühle von der Mauer auf ihn herabgeworfen, so daß er in Thebez den
Tod fand\textless sup title=``Ri 9,53''\textgreater✲? Warum seid ihr so
nahe an die Mauer herangerückt?‹ -- dann sage nur: ›Auch dein Knecht,
der Hethiter Uria, ist ums Leben gekommen.‹« 22Darauf ging der Bote hin
und richtete nach seiner Ankunft den Auftrag Joabs bei David genau aus.
23Er meldete dem Könige nämlich: »Weil die Feinde uns überlegen und bis
aufs freie Feld gegen uns vorgedrungen waren, so mußten wir sie bis an
den Eingang des Stadttors zurückdrängen. 24Da aber schossen die Schützen
von der Mauer herab auf deine Leute, und dabei fielen einige von den
Leuten des Königs; auch dein Knecht, der Hethiter Uria, fand den Tod.«
25Da sagte David zu dem Boten: »Melde dem Joab folgendes: ›Laß dir
diesen Vorfall nicht leid sein! Denn das Schwert frißt eben bald diesen,
bald jenen. Setze nur deine Belagerung der Stadt entschlossen fort und
zerstöre sie!‹ So sollst du ihm Mut zusprechen!«

\hypertarget{d-bathsebas-trauer-um-den-gatten-ihre-verheiratung-mit-david}{%
\paragraph{d) Bathsebas Trauer um den Gatten; ihre Verheiratung mit
David}\label{d-bathsebas-trauer-um-den-gatten-ihre-verheiratung-mit-david}}

26Als nun die Frau Urias den Tod ihres Mannes erfuhr, hielt sie die
Totenklage um ihren Gatten; 27sobald aber die Trauerzeit vorüber war,
ließ David sie in sein Haus holen. Sie wurde also seine Frau und gebar
ihm einen Sohn. Aber die Tat, die David verübt hatte, erregte das
Mißfallen des HERRN.

\hypertarget{e-nathans-strafrede-und-unheilsverkuxfcndigung-davids-schuldbekenntnis-und-reue}{%
\paragraph{e) Nathans Strafrede und Unheilsverkündigung; Davids
Schuldbekenntnis und
Reue}\label{e-nathans-strafrede-und-unheilsverkuxfcndigung-davids-schuldbekenntnis-und-reue}}

\hypertarget{section-11}{%
\section{12}\label{section-11}}

1Hierauf sandte der HERR (den Propheten) Nathan zu David; als dieser zu
ihm gekommen war, redete er so zu ihm: »Es lebten zwei Männer in
derselben Stadt, ein reicher und ein armer. 2Der Reiche besaß Kleinvieh
und Rinder in großer Menge, 3der Arme aber hatte gar nichts als ein
einziges Lämmchen, das er sich gekauft und aufgezogen hatte und das bei
ihm und zugleich mit seinen Kindern aufwuchs; es aß von seinem Bissen
und trank aus seinem Becher, es schlief an seinem Busen\textless sup
title=``oder: auf seinem Schoß''\textgreater✲ und wurde von ihm wie eine
Tochter gehalten. 4Da kam eines Tages Besuch zu dem reichen Mann, und
weil es ihm leid tat, ein Stück von seinem eigenen Kleinvieh oder von
seinen Rindern zu nehmen, um es für den Besuch, der zu ihm gekommen war,
als Mahl zuzubereiten, nahm er das Lämmchen des armen Mannes und
richtete es für den Gast zu, der zu ihm gekommen war.« 5Da geriet David
in heftigen Zorn gegen den Mann, so daß er zu Nathan sagte: »So wahr der
HERR lebt: der Mann, der das getan hat, ist ein Kind des Todes! 6Und das
Lamm soll er vierfach erstatten zur Strafe dafür, daß er so gehandelt
und weil er kein Mitleid bewiesen hat!«

7Da erwiderte Nathan dem David: »Du bist der Mann! So hat der HERR, der
Gott Israels, gesprochen: ›Ich habe dich zum König über Israel gesalbt,
und ich habe dich aus Sauls Händen errettet, 8ich habe dir das Haus
deines Herrn gegeben und die Frauen deines Herrn dir in den Schoß gelegt
(zur Verfügung gestellt); ich habe dir das Haus Israel und Juda
übergeben, und wenn dir das noch zu wenig war, so hätte ich dir noch
dies und das hinzugefügt. 9Warum hast du dich über das Gebot des HERRN
hinweggesetzt und etwas getan, was ihm mißfällt? Den Hethiter Uria hast
du mit dem Schwert erschlagen lassen und sein Weib dir zum Weibe
genommen, nachdem du ihn selbst durch das Schwert der Ammoniter hast
umbringen lassen. 10So soll denn nun das Schwert aus deinem Hause
niemals weichen zur Strafe dafür, daß du mich mißachtet und das Weib des
Hethiters Uria dir zum Weibe genommen hast!‹ 11So hat der HERR
gesprochen: ›Siehe, ich will Unheil über dich aus deinem eigenen Hause
hervorgehen lassen und will dir deine Frauen vor deinen Augen wegnehmen
und sie einem andern geben, daß er im Angesicht dieser Sonne deinen
Frauen beiwohnen soll. 12Denn du hast im geheimen gehandelt, ich aber
will diese Drohung vor den Augen von ganz Israel und angesichts der
Sonne zur Ausführung bringen!‹«

13Da sagte David zu Nathan: »Ich habe gegen den HERRN gesündigt!« Nathan
antwortete dem David: »So hat auch der HERR dir deine Sünde vergeben: du
selbst wirst nicht sterben! 14Doch weil du den Feinden des HERRN durch
diese Tat Anlaß zur Lästerung gegeben hast, so soll auch der Sohn, der
dir geboren ist, unrettbar sterben!«

\hypertarget{f-erkrankung-und-tod-des-kindes-der-bathseba-davids-trauer-und-neuer-lebensmut-geburt-und-erziehung-salomos}{%
\paragraph{f) Erkrankung und Tod des Kindes der Bathseba; Davids Trauer
und neuer Lebensmut; Geburt und Erziehung
Salomos}\label{f-erkrankung-und-tod-des-kindes-der-bathseba-davids-trauer-und-neuer-lebensmut-geburt-und-erziehung-salomos}}

15Als Nathan hierauf nach Hause gegangen war, schlug der HERR das Kind,
welches die Frau Urias dem David geboren hatte, so daß es todkrank
wurde. 16Da suchte David Gott um des Knaben willen (im Heiligtum) auf,
und David fastete und brachte, als er heimgekommen war, die ganze Nacht
hindurch auf dem Erdboden liegend zu. 17Obgleich nun die Ältesten seines
Hauses\textless sup title=``=~seine vornehmsten Hofleute''\textgreater✲
zu ihm traten, um ihn zum Aufstehen von der Erde zu bewegen, weigerte er
sich doch und speiste nicht mit ihnen. 18Als das Kind dann am siebten
Tage starb, trugen die Hofleute Davids Bedenken, ihm den Tod des Kindes
anzuzeigen; denn sie dachten: »Solange das Kind noch am Leben war, hat
er unsere Vorstellungen, wenn wir ihm zuredeten, unbeachtet gelassen:
wie können wir ihm da jetzt den Tod des Kindes anzeigen? Es gäbe ein
Unglück ab!« 19Als David aber seine Hofleute miteinander flüstern sah
und daran merkte, daß das Kind tot war, fragte er seine Hofleute: »Ist
das Kind tot?« Sie antworteten ihm: »Ja, es ist tot.« 20Da stand David
vom Boden auf, wusch und salbte sich und legte die Trauerkleider ab;
darauf ging er in das Haus des HERRN und warf sich nieder\textless sup
title=``oder: betete andächtig''\textgreater✲. Als er dann in sein Haus
zurückgekehrt war, ließ er sich eine Mahlzeit auftragen und aß. 21Da
sagten seine Hofleute zu ihm: »Wie unerklärlich ist dein Verhalten!
Solange das Kind noch am Leben war, hast du um seinetwillen gefastet und
geweint; und jetzt, da das Kind tot ist, stehst du auf und nimmst
Nahrung zu dir.« 22Da antwortete er: »Solange das Kind noch lebte, habe
ich gefastet und geweint, weil ich dachte: Wer weiß, vielleicht erbarmt
der HERR sich meiner, daß das Kind am Leben bleibt? 23Nun es aber tot
ist, wozu soll ich da fasten? Kann ich es etwa wieder ins Leben
zurückrufen? Ich kann wohl zu ihm kommen, es aber kann nicht wieder zu
mir zurückkehren.«

24Nachdem David dann seiner Gattin Bathseba Trost zugesprochen und sich
ihr wieder in Liebe zugewandt hatte, wurde sie Mutter eines Sohnes, den
er Salomo\textless sup title=``d.h. Friedreich''\textgreater✲ nannte und
den der HERR liebhatte. 25David übergab ihn der Fürsorge✲ des Propheten
Nathan, der ihm den Namen Jedidjah\textless sup title=``d.h. Liebling
des Herrn''\textgreater✲ gab, um des HERRN willen.

\hypertarget{joab-erobert-rabba-bestrafung-der-ammoniter}{%
\subsubsection{8. Joab erobert Rabba; Bestrafung der
Ammoniter}\label{joab-erobert-rabba-bestrafung-der-ammoniter}}

26Joab aber bestürmte unterdessen Rabba, die Hauptstadt der Ammoniter,
und eroberte die Königsstadt. 27Hierauf sandte er Boten an David und
ließ ihm sagen: »Ich habe Rabba bestürmt und die Wasserstadt auch
erobert; 28so biete jetzt nun den Rest des Kriegsvolkes auf, belagere
die Stadt und erobere sie, damit nicht ich die Stadt einnehme und dann
mein Name über ihr ausgerufen wird.« 29Da bot David seine ganze
Heeresmacht auf, zog gegen Rabba, bestürmte die Stadt und eroberte sie.
30Er nahm dann ihrem Götzen Milkom die Krone vom Haupt, die ein Talent
Gold wog und mit einem kostbaren Edelstein besetzt war; der kam nunmehr
an\textless sup title=``oder: auf''\textgreater✲ Davids Haupt. Und er
führte aus der Stadt eine überaus reiche Beute weg. 31Die Bevölkerung,
die sich dort vorfand, ließ er wegführen und stellte sie als
Fronarbeiter an bei den Sägen, bei den eisernen Picken und den eisernen
Äxten und ließ sie an den Ziegelöfen arbeiten; ebenso verfuhr er mit
allen übrigen Städten der Ammoniter. Dann kehrte David mit dem gesamten
Heere nach Jerusalem zurück.

\hypertarget{david-und-absalom-kap.-13-19}{%
\subsubsection{9. David und Absalom (Kap.
13-19)}\label{david-und-absalom-kap.-13-19}}

\hypertarget{a-amnon-und-absalom}{%
\paragraph{a) Amnon und Absalom}\label{a-amnon-und-absalom}}

\hypertarget{aa-amnons-leidenschaftliche-liebe-seine-schandtat-an-seiner-halbschwester-thamar}{%
\subparagraph{aa) Amnons leidenschaftliche Liebe; seine Schandtat an
seiner Halbschwester
Thamar}\label{aa-amnons-leidenschaftliche-liebe-seine-schandtat-an-seiner-halbschwester-thamar}}

\hypertarget{section-12}{%
\section{13}\label{section-12}}

1Danach begab sich folgendes: Absalom, der Sohn Davids, hatte eine
schöne Schwester namens Thamar; in diese verliebte sich Amnon, Davids
Sohn, 2und härmte sich vor Liebe zu seiner Halbschwester Thamar so ab,
daß er krank wurde; sie war nämlich noch Jungfrau, und es schien dem
Amnon unmöglich, an sie heranzukommen. 3Nun hatte Amnon einen Freund
namens Jonadab, einen Sohn Simeas, des Bruders Davids; und dieser
Jonadab war ein sehr kluger Mann. 4Der fragte ihn: »Warum siehst du
jeden Morgen so elend aus, Königssohn? Willst du es mir nicht
anvertrauen?« Amnon antwortete ihm: »Ich liebe Thamar, die Schwester
meines Bruders Absalom.« 5Da sagte Jonadab zu ihm: »Lege dich zu Bett
und stelle dich krank. Wenn dann dein Vater kommt, um dich zu besuchen,
so sage zu ihm: ›Wenn doch meine Schwester Thamar herkäme und mir etwas
zu essen gäbe! Wenn sie das Essen vor meinen Augen zubereitete, so daß
ich zusehen kann, würde ich das Essen von ihrer Hand annehmen.«

\hypertarget{ausfuxfchrung-des-schuxe4ndlichen-anschlages}{%
\paragraph{Ausführung des schändlichen
Anschlages}\label{ausfuxfchrung-des-schuxe4ndlichen-anschlages}}

6So legte sich denn Amnon zu Bett und stellte sich krank; und als der
König kam, um ihn zu besuchen, sagte Amnon zum König: »Wenn doch meine
Schwester Thamar herkäme und vor meinen Augen ein paar Pfannkuchen
zubereitete, daß ich sie aus ihrer Hand essen könnte!« 7Da sandte David
zu Thamar ins Haus und ließ ihr sagen: »Gehe doch in die Wohnung deines
Bruders Amnon und bereite ihm das Essen!« 8So ging denn Thamar in die
Wohnung ihres Bruders Amnon, während er zu Bett lag. Sie nahm den Teig,
knetete ihn, formte Kuchen vor seinen Augen daraus und buk die Kuchen.
9Dann nahm sie die Pfanne und schüttete sie vor seinen Augen (auf einen
Teller) aus; er weigerte sich jedoch zu essen und befahl, jedermann
solle aus dem Zimmer hinausgehen. Als nun alle hinausgegangen waren,
10sagte Amnon zu Thamar: »Bringe mir das Essen in die Kammer herein,
dann will ich von deiner Hand das Essen annehmen.« Da nahm Thamar die
Kuchen, die sie zubereitet hatte, und brachte sie ihrem Bruder Amnon in
die Kammer hinein; 11aber als sie ihm diese zum Essen hinreichte,
ergriff er sie und sagte zu ihr: »Komm, lege dich zu mir, liebe
Schwester!« 12Sie erwiderte ihm: »Nicht doch, mein Bruder! Entehre mich
nicht! So etwas darf in Israel nicht geschehen! Begehe keine solche
Schandtat! 13Wohin sollte ich denn meine Schande tragen? Und du selbst
würdest in Israel als ein ehrloser Mann dastehen! Rede doch lieber mit
dem König: er wird mich dir gewiß nicht versagen.« 14Aber er wollte
nicht auf ihre Vorstellungen hören, sondern überwältigte sie und tat ihr
Gewalt an.

\hypertarget{weitere-schuxe4ndliche-versuxfcndigung-amnons-an-thamar}{%
\paragraph{Weitere schändliche Versündigung Amnons an
Thamar}\label{weitere-schuxe4ndliche-versuxfcndigung-amnons-an-thamar}}

15Hierauf aber faßte Amnon eine überaus tiefe Abneigung gegen sie, so
daß der Widerwille, den er gegen sie empfand, noch stärker war als seine
frühere Liebe zu ihr. Daher rief er ihr zu: »Mach, daß du fortkommst!«
16Sie antwortete ihm: »Nicht doch, mein Bruder! Denn dieses Unrecht,
wenn du mich jetzt von dir stießest, wäre noch größer als das andere,
das du mir angetan hast!« Er wollte aber nicht auf sie hören, 17sondern
rief seinen Burschen her, der ihn zu bedienen hatte, und befahl ihm:
»Schafft mir diese da hinaus auf die Straße und riegle die Tür hinter
ihr zu!« 18Sie trug aber ein Ärmelkleid; denn so kleideten sich ehemals
die Töchter des Königs\textless sup title=``=~die
Prinzessinnen''\textgreater✲, solange sie unverheiratet waren. Als nun
sein Leibdiener sie auf die Straße hinausgeschafft und die Tür hinter
ihr verriegelt hatte, 19streute Thamar Staub auf ihr Haupt, zerriß das
Ärmelkleid, das sie anhatte, legte sich die Hand\textless sup
title=``oder: die Hände''\textgreater✲ aufs Haupt und ging laut
schreiend davon.

\hypertarget{das-verhalten-absaloms-und-des-kuxf6nigs-nach-der-schandtat}{%
\paragraph{Das Verhalten Absaloms und des Königs nach der
Schandtat}\label{das-verhalten-absaloms-und-des-kuxf6nigs-nach-der-schandtat}}

20Da sagte ihr Bruder Absalom zu ihr: »Dein Bruder Amnon ist bei dir
gewesen? Nun denn, liebe Schwester, schweige still! Er ist ja dein
Bruder: nimm dir die Sache nicht zu Herzen!« So blieb denn Thamar einsam
im Hause ihres Bruders Absalom wohnen. 21Als dann der König David den
ganzen Vorfall vernahm, geriet er in heftigen Zorn (tat aber seinem
Sohne Amnon nichts zuleide; denn er hatte ihn lieb, weil er sein
Erstgeborener war). 22Absalom aber redete seitdem kein Wort mehr mit
Amnon, weder im Bösen noch im Guten; denn Absalom haßte Amnon, weil er
seine Schwester Thamar entehrt hatte.

\hypertarget{bb-absaloms-rache-an-amnon}{%
\subparagraph{bb) Absaloms Rache an
Amnon}\label{bb-absaloms-rache-an-amnon}}

23Nun begab es sich zwei volle Jahre später, daß Absalom in Baal-Hazor,
das bei Ephraim liegt, Schafschur hielt und alle Söhne des
Königs\textless sup title=``=~königlichen Prinzen''\textgreater✲ dazu
einlud. 24Er ging auch zum König und sagte: »Du weißt, dein Knecht hält
eben Schafschur; möchte doch der König samt seiner Umgebung deinen
Knecht dorthin begleiten!« 25Aber der König erwiderte dem Absalom:
»Nicht doch, mein Sohn! Wir wollen nicht allesamt kommen, damit wir dir
nicht zu viel Last machen.« Da drang er in ihn, aber er wollte nicht
mitgehen, sondern schlug ihm seine Bitte ab. 26Da sagte Absalom: »Wenn
also nicht, so laß wenigstens meinen Bruder Amnon mit uns gehen!« Der
König antwortete ihm: »Wozu sollte er mit dir gehen?« 27Als Absalom
jedoch auf seiner Bitte bestand, ließ er Amnon und alle Söhne des Königs
mit ihm gehen.

28(Absalom veranstaltete nun ein Gelage gleich einem Königsgelage.)
Dabei gab er seinen Dienern folgenden Befehl: »Gebt acht! Wenn Amnon vom
Wein in fröhliche Stimmung versetzt ist und ich euch zurufe: ›Haut Amnon
nieder!‹, so tötet ihn, fürchtet euch nicht! Ich bin's ja, der es euch
befohlen hat: seid mutig und zeigt euch mannhaft!« 29Da verfuhren die
Diener Absaloms mit Amnon so, wie Absalom ihnen befohlen hatte; darauf
standen die Söhne des Königs alle auf, bestiegen ein jeder sein Maultier
und ergriffen die Flucht.

\hypertarget{die-vorguxe4nge-in-davids-palast-beim-eintreffen-der-schreckensnachricht}{%
\paragraph{Die Vorgänge in Davids Palast beim Eintreffen der
Schreckensnachricht}\label{die-vorguxe4nge-in-davids-palast-beim-eintreffen-der-schreckensnachricht}}

30Während sie noch unterwegs waren, war schon das Gerücht zu David
gedrungen, Absalom habe alle Söhne des Königs ermordet, so daß auch
nicht einer von ihnen am Leben geblieben sei. 31Da stand der König auf,
zerriß seine Kleider und warf sich auf die Erde nieder; auch alle seine
Hofleute standen (um ihn her) mit zerrissenen Kleidern. 32Da nahm
Jonadab, der Sohn Simeas, des Bruders Davids, das Wort und sagte: »Mein
Herr denke doch nicht, daß man die jungen Leute, die Söhne des Königs,
allesamt ums Leben gebracht habe! Nein, Amnon allein ist tot; man hat es
dem Absalom ja ansehen können, daß das bei ihm beschlossene Sache war
seit dem Tage, als jener seine Schwester Thamar entehrt hatte. 33Darum
wolle jetzt mein Herr, der König, nicht den Gedanken hegen und nicht
aussprechen, daß alle Königssöhne tot seien; nein, Amnon allein ist
tot!«

34 35Da sagte Jonadab zum Könige: »Siehst du? Die Söhne des Königs
kommen! Wie dein Knecht gesagt hat, so verhält es sich wirklich!« 36Kaum
hatte er ausgeredet, da kamen auch schon die Königssöhne und brachen in
lautes Weinen aus; und auch der König und alle seine Hofleute erhoben
eine überaus große Wehklage.

\hypertarget{absaloms-flucht-nach-gesur-zu-seinem-grouxdfvater}{%
\paragraph{Absaloms Flucht nach Gesur zu seinem
Großvater}\label{absaloms-flucht-nach-gesur-zu-seinem-grouxdfvater}}

37Absalom aber war geflohen und hatte sich zu Thalmai, dem Sohne
Ammihuds, dem Könige von Gesur, begeben. David aber trauerte um seinen
Sohn Amnon die ganze Zeit hindurch. 38Nachdem Absalom aber geflohen war
und sich nach Gesur begeben hatte, blieb er dort drei Jahre.

\hypertarget{cc-absaloms-begnadigung-und-ruxfcckkehr}{%
\subparagraph{cc) Absaloms Begnadigung und
Rückkehr}\label{cc-absaloms-begnadigung-und-ruxfcckkehr}}

\hypertarget{joabs-eingreifen-die-unterredung-der-klugen-frau-von-thekoa-mit-david}{%
\paragraph{Joabs Eingreifen; die Unterredung der klugen Frau von Thekoa
mit
David}\label{joabs-eingreifen-die-unterredung-der-klugen-frau-von-thekoa-mit-david}}

39Als dann der Zorn des Königs gegen Absalom allmählich milder geworden
war, weil er sich über den Tod Amnons getröstet hatte,

\hypertarget{section-13}{%
\section{14}\label{section-13}}

1und als Joab, der Sohn der Zeruja, erkannte, daß das Herz des Königs
sich zu Absalom wieder hingewandt hatte, 2sandte Joab nach Thekoa, ließ
von dort eine kluge Frau holen und sagte zu ihr: »Stelle dich, als ob du
in Trauer wärest, ziehe Trauerkleider an, salbe dich nicht mit Öl und
benimm dich wie eine Frau, die schon lange Zeit um einen Toten trauert.
3Dann begib dich zum König und rede zu ihm so und so«; und Joab gab ihr
genau die Worte an, die sie sagen sollte.

\hypertarget{die-erste-rede-der-klugen-frau}{%
\paragraph{Die erste Rede der klugen
Frau}\label{die-erste-rede-der-klugen-frau}}

4Die Frau aus Thekoa ging also zum König hinein, warf sich vor ihm auf
ihr Angesicht zu Boden, brachte ihre Huldigung dar und rief aus: »Hilf
mir, o König!« 5Als nun der König sie fragte, was sie wünsche,
antwortete sie: »Ach, ich bin eine Witwe, denn mein Mann ist tot! 6Nun
hatte deine Magd zwei Söhne, die gerieten auf dem Felde in Streit
miteinander, und weil keiner da war, der sie auseinanderbrachte, schlug
der eine auf den andern los und tötete ihn. 7Und jetzt hat sich die
ganze Verwandtschaft gegen deine Magd erhoben und sagt: ›Gib den
Brudermörder heraus, damit wir ihn umbringen für das Leben seines
Bruders, den er erschlagen hat, und damit wir auch den Erben ausrotten!‹
So wollen sie also die letzte Kohle, die mir noch geblieben ist,
auslöschen, um meinem Manne weder Namen noch Nachkommen auf dem Erdboden
zu lassen!« 8Da sagte der König zu der Frau: »Gehe heim, ich selber
werde deinetwegen verfügen!« 9Die Frau aus Thekoa aber erwiderte dem
König: »Auf mir, mein Herr und König, liege die Schuld (nämlich, daß
keine Blutrache vollzogen wird) und auf meines Vaters Hause! Den König
aber und seinen Thron trifft keine Verantwortung!« 10Da sagte der König:
»Wer etwas von dir will, den bringe zu mir her: er soll dir nicht weiter
zu schaffen machen!« 11Da entgegnete sie: »Der König wolle doch des
HERRN, seines Gottes, gedenken, damit der Bluträcher nicht noch mehr
Unglück anrichtet und sie meinen Sohn nicht auch noch vertilgen!« Da
sagte er: »So wahr der HERR lebt, kein Haar soll deinem Sohne gekrümmt
werden!«

\hypertarget{neue-rede-der-klugen-frau}{%
\paragraph{Neue Rede der klugen Frau}\label{neue-rede-der-klugen-frau}}

12Nun fuhr die Frau fort: »Darf deine Magd ein Wort an meinen Herrn, den
König, richten?« Er antwortete: »Rede!« 13Da sagte die Frau: »Und warum
hegst du denn eine derartige Gesinnung gegen das Volk Gottes? Denn
nachdem der König dies Urteil gefällt hat, hat er sich selbst gleichsam
für schuldig erklärt, weil der König seinen verstoßenen Sohn nicht
zurückkehren läßt. 14Denn wir müssen zwar gewißlich sterben und sind wie
Wasser, das auf die Erde ausgegossen ist und nicht wieder gesammelt
werden kann; aber Gott wird das Leben dessen nicht dahinraffen, der
ernstlich darauf sinnt, einen Verbannten\textless sup title=``oder:
Verstoßenen''\textgreater✲ nicht fern von sich in dauernder Verbannung
zu belassen. 15Und nun, der Grund, weshalb ich hergekommen bin, um diese
Sache meinem Herrn, dem Könige, vorzutragen, ist der, daß die Leute mir
Angst gemacht haben. Da dachte aber deine Magd: ›Ich will es doch dem
König vortragen; vielleicht erfüllt der König die Bitte seiner Magd.‹
16Ja, der König wird mich erhören, um seine Magd aus der Hand des Mannes
zu erretten, der mich und zugleich meinen Sohn aus dem Erbe\textless sup
title=``oder: Eigentum''\textgreater✲ Gottes zu vertilgen sucht. 17Daher
dachte deine Magd: ›Das Wort meines Herrn, des Königs, wird mir eine
Beruhigung sein‹; denn mein Herr, der König, ist wie der Engel Gottes,
um Gutes und Böses zu unterscheiden; und der HERR, dein Gott, sei mit
dir!«

\hypertarget{der-kuxf6nig-durchschaut-den-listig-angelegten-plan}{%
\paragraph{Der König durchschaut den listig angelegten
Plan}\label{der-kuxf6nig-durchschaut-den-listig-angelegten-plan}}

18Da hub der König an und sagte zu der Frau: »Verheimliche mir nichts,
wonach ich dich jetzt fragen werde!« Die Frau antwortete: »Mein Herr,
der König, braucht nur zu reden!« 19Da fragte der König: »Hat nicht Joab
die Hand bei dieser ganzen Sache im Spiel?« Da erwiderte die Frau: »So
wahr du lebst, mein Herr und König! Es ist nicht möglich, bei allem, was
mein Herr, der König, sagt, rechts oder links vorbeizukommen! Ja, dein
Knecht Joab, er hat mir den Auftrag gegeben, und er selbst hat deiner
Magd alle diese Worte in den Mund gelegt. 20Um der Sache ein anderes
Aussehen zu geben, ist dein Knecht Joab so zu Werke gegangen; aber mein
Herr ist weise, ebenso weise wie der Engel Gottes, so daß er alles weiß,
was auf Erden vorgeht.«

\hypertarget{davids-zusage-joab-dankt-dem-kuxf6nige-fuxfcr-die-erfuxfcllung-seiner-bitte-und-holt-absalom-zuruxfcck}{%
\paragraph{Davids Zusage; Joab dankt dem Könige für die Erfüllung seiner
Bitte und holt Absalom
zurück}\label{davids-zusage-joab-dankt-dem-kuxf6nige-fuxfcr-die-erfuxfcllung-seiner-bitte-und-holt-absalom-zuruxfcck}}

21Der König sagte dann zu Joab: »Nun gut! Ich will diese deine Bitte
erfüllen! Gehe also hin und hole den jungen Mann, den Absalom, zurück!«
22Da warf sich Joab auf sein Angesicht zur Erde nieder, brachte seine
Huldigung dar und beglückwünschte den König; dann rief Joab aus: »Heute
erkennt dein Knecht, daß mein Herr, der König, mir in Gnaden zugetan
ist, weil der König die Bitte seines Knechtes erfüllt hat!« 23Darauf
machte Joab sich auf den Weg, begab sich nach Gesur und brachte Absalom
nach Jerusalem zurück. 24Der König aber befahl: »Er soll sich in seine
Wohnung begeben, mir aber nicht vor die Augen treten!« So begab sich
denn Absalom in seine Wohnung und durfte sich vor dem König nicht sehen
lassen.

\hypertarget{absaloms-schuxf6nheit-seine-kinder}{%
\paragraph{Absaloms Schönheit; seine
Kinder}\label{absaloms-schuxf6nheit-seine-kinder}}

25In ganz Israel gab es aber keinen Mann, der wegen seiner Schönheit
ebenso gefeiert gewesen wäre wie Absalom: von der Fußsohle bis zum
Scheitel war kein Fehl an ihm; 26und wenn er sich das Haupt(-haar)
scheren ließ -- das geschah nämlich nach Ablauf jeden Jahres, weil es
ihm sonst zu beschwerlich geworden wäre --, so wog sein Haupthaar
zweihundert Schekel nach königlichem Gewicht. 27Es waren aber dem
Absalom drei Söhne und eine Tochter namens Thamar geboren; die war ein
Mädchen von großer Schönheit.

\hypertarget{absalom-veranlauxdft-joab-ihn-mit-seinem-vater-fuxf6rmlich-auszusuxf6hnen}{%
\paragraph{Absalom veranlaßt Joab, ihn mit seinem Vater förmlich
auszusöhnen}\label{absalom-veranlauxdft-joab-ihn-mit-seinem-vater-fuxf6rmlich-auszusuxf6hnen}}

28Als nun Absalom zwei volle Jahre in Jerusalem zugebracht hatte, ohne
dem König vor die Augen zu treten, 29schickte er zu Joab, um ihn zum
König zu senden; aber der weigerte sich, zu ihm zu kommen; und als er
noch ein zweites Mal hinschickte, weigerte er sich wieder, zu kommen.
30Da sagte Absalom zu seinen Knechten: »Ihr wißt, Joab hat da ein
Ackerstück neben dem meinigen und hat Gerste darauf stehen; geht hin und
zündet es an!« Als nun die Knechte Absaloms das Feld in Brand gesteckt
hatten, 31machte Joab sich auf, ging zu Absalom ins Haus und fragte ihn:
»Warum haben deine Knechte mein Feld angezündet?« 32Absalom antwortete
dem Joab: »Du weißt, ich habe zu dir gesandt und dir sagen lassen: Komm
her zu mir, ich will dich zum König senden und ihm sagen lassen: ›Wozu
bin ich aus Gesur heimgekehrt? Es wäre besser für mich, ich wäre noch
dort!‹ Jetzt aber werde ich dem König vor die Augen treten; und wenn
eine Schuld auf mir liegt, so mag er mich töten!« 33Als Joab sich nun
zum Könige begeben und ihm die Sache vorgetragen hatte, ließ dieser
Absalom rufen. Als der zum König kam, warf er sich auf sein Angesicht
vor ihm zur Erde nieder; der König aber küßte Absalom.

\hypertarget{b-absaloms-empuxf6rung-davids-flucht-aus-jerusalem}{%
\paragraph{b) Absaloms Empörung; Davids Flucht aus
Jerusalem}\label{b-absaloms-empuxf6rung-davids-flucht-aus-jerusalem}}

\hypertarget{aa-absaloms-ehrgeizige-und-gunstbuhlerische-umtriebe}{%
\subparagraph{aa) Absaloms ehrgeizige und gunstbuhlerische
Umtriebe}\label{aa-absaloms-ehrgeizige-und-gunstbuhlerische-umtriebe}}

\hypertarget{section-14}{%
\section{15}\label{section-14}}

1Danach begab es sich, daß Absalom sich Wagen und Pferde anschaffte,
dazu fünfzig Mann, die als Leibdiener vor ihm herliefen. 2Auch pflegte
Absalom sich alle Morgen früh neben dem Wege nach dem Ratstor
aufzustellen; wenn dann jemand einen Rechtsstreit hatte und den König um
Entscheidung angehen wollte, rief Absalom ihn an und fragte ihn: »Aus
welcher Ortschaft bist du?« Wenn jener dann antwortete: »Dein Knecht
kommt aus dem und dem Stamme Israels«, 3so sagte Absalom zu ihm: »Deine
Sache ist allerdings gut und in Ordnung, aber beim König ist niemand,
der dir Gehör schenkt!« 4Dann fuhr Absalom fort: »Wenn man mich doch zum
Richter im Lande bestellte, daß jeder, der eine Streitsache und einen
Rechtshandel hat, zu mir käme: ich wollte ihm schon zu seinem Recht
verhelfen!« 5Und wenn jemand an ihn herantrat, um sich vor ihm huldigend
zu verneigen, so streckte er seine Hand aus, umarmte ihn und küßte ihn.
6So machte es Absalom mit allen Israeliten, die zum König kamen, um sich
Recht sprechen zu lassen; und so stahl Absalom sich die Herzen der
Israeliten.

\hypertarget{bb-absaloms-verschwuxf6rung-und-empuxf6rung-in-hebron}{%
\subparagraph{bb) Absaloms Verschwörung und Empörung in
Hebron}\label{bb-absaloms-verschwuxf6rung-und-empuxf6rung-in-hebron}}

7Nach Verlauf von vier Jahren aber sagte Absalom zum König: »Ich möchte
gern hingehen und in Hebron mein Gelübde einlösen, das ich dem HERRN
dargebracht habe. 8Denn als ich mich zu Gesur in Syrien aufhielt, hat
dein Knecht folgendes Gelübde getan: ›Wenn der HERR mich nach Jerusalem
zurückkehren läßt, so will ich dem HERRN ein Dankopfer darbringen.‹«
9Der König antwortete ihm: »Gehe hin in Frieden!« So machte Absalom sich
denn auf den Weg und ging nach Hebron. 10Er hatte aber heimlich Boten
durch alle Stämme der Israeliten gesandt und sagen lassen: »Sobald ihr
Posaunenschall hört, so ruft aus: ›Absalom ist in Hebron König
geworden!‹« 11Mit Absalom gingen aber auch zweihundert Männer aus
Jerusalem nach Hebron, die von ihm zum Opferfest eingeladen waren und
arglos mitgingen, ohne von irgend etwas zu wissen. 12Außerdem ließ
Absalom, als das Opferfest schon im Gange war, den Giloniten Ahithophel,
den Ratgeber Davids, aus seinem Wohnort Gilo holen. So gewann die
Verschwörung immer weitere Verbreitung, und immer mehr Leute schlossen
sich an Absalom an.

\hypertarget{cc-david-tritt-eiligst-die-flucht-aus-jerusalem-an-nach-zuruxfccklassung-einiger-kebsweiber}{%
\subparagraph{cc) David tritt eiligst die Flucht aus Jerusalem an nach
Zurücklassung einiger
Kebsweiber}\label{cc-david-tritt-eiligst-die-flucht-aus-jerusalem-an-nach-zuruxfccklassung-einiger-kebsweiber}}

13Als nun ein Bote bei David eintraf mit der Meldung: »Das Herz der
Israeliten hat sich Absalom zugewandt«, 14da befahl David allen seinen
Dienern, die in Jerusalem bei ihm waren: »Auf! Wir müssen fliehen! Sonst
gibt es für uns keine Rettung vor Absalom! Macht euch eilends auf den
Weg, damit er uns nicht zuvorkommt und das Unheil über uns bringt und
ein Blutbad in der Stadt anrichtet!« 15Die Diener des Königs antworteten
ihm: »Ganz wie unser königlicher Herr es für gut befindet: wir sind
deine gehorsamen Diener!« 16So zog denn der König aus, und sein ganzer
Hof befand sich in seinem Gefolge; nur zehn Kebsweiber ließ der König
zurück, um das Haus\textless sup title=``=~den Palast''\textgreater✲ zu
hüten.

\hypertarget{dd-vorbeizug-des-kriegsvolkes-am-kuxf6nig-die-treue-itthais}{%
\subparagraph{dd) Vorbeizug des Kriegsvolkes am König; die Treue
Itthais}\label{dd-vorbeizug-des-kriegsvolkes-am-kuxf6nig-die-treue-itthais}}

17Als so der König auszog und das ganze Volk\textless sup title=``oder:
der ganze Hof''\textgreater✲ ihm auf dem Fuße folgte, machte er beim
letzten Hause halt, 18während alle seine Diener neben ihm standen und
seine gesamte Leibwache\textless sup title=``die Krethi und Plethi, vgl.
zu 8,18''\textgreater✲ und alle Leute des Gathiters (Itthai),
sechshundert Mann, die ihm von Gath her gefolgt waren, an dem Könige
vorüberzogen. 19Da sagte der König zu Itthai aus Gath: »Warum willst
auch du mit uns ziehen? Kehre um und bleibe beim König (Absalom)! Du
bist ja ein Ausländer und noch dazu aus deiner Heimat verbannt. 20Erst
gestern bist du hergekommen, und heute soll ich dich schon mit uns auf
die Irrfahrt nehmen, ohne selbst zu wissen, wohin ich gehe? Kehre um und
nimm deine Landsleute mit dir zurück! Der HERR möge dir Güte und Treue
erweisen!« 21Itthai aber erwiderte dem König: »So wahr der HERR lebt,
und so wahr mein Herr und König lebt, nein! An dem Orte, an dem mein
Herr und König sein wird, es gehe zum Tode oder zum Leben, da wird auch
dein Knecht zu finden sein!« 22Da sagte David zu Itthai: »Gut denn, so
ziehe vorüber!« Da zog Itthai aus Gath vorüber mit all seinen Leuten und
dem ganzen Troß (von Kindern und Frauen), der bei ihm war. 23Die ganze
Bevölkerung aber weinte laut, während das gesamte Kriegsvolk vorüberzog.
Hierauf ging der König über den Bach Kidron, und auch das gesamte Volk
zog hinüber in der Richtung nach der Wüste (Juda) hin.

\hypertarget{ee-davids-auftrag-an-zadok-und-abjathar}{%
\subparagraph{ee) Davids Auftrag an Zadok und
Abjathar}\label{ee-davids-auftrag-an-zadok-und-abjathar}}

24Da erschienen auch Zadok (und Abjathar) mit allen Leviten, die trugen
die Bundeslade Gottes; sie setzten die Lade Gottes dort nieder, und
Abjathar brachte Opfer dar, bis alles Volk aus der Stadt vollständig
vorübergezogen war. 25Darauf sagte der König zu Zadok: »Bringe die Lade
Gottes in die Stadt zurück! Finde ich Gnade in den Augen des HERRN, so
wird er mich zurückführen und mich die Lade und seine Wohnung
wiedersehen lassen. 26Spricht er aber so zu mir: ›Ich habe kein Gefallen
an dir‹ -- nun, hier bin ich! Er tue mir, wie es ihm wohlgefällt!«
27Dann sagte der König weiter zum Priester Zadok: »Du und Abjathar,
kehrt ihr ruhig in die Stadt zurück und mit euch eure beiden Söhne,
Ahimaaz, dein Sohn, und Jonathan, der Sohn Abjathars. 28Gebt wohl acht!
Ich will in den Niederungen✲ der Wüste verweilen, bis eine Botschaft von
euch kommt und mir Nachricht gibt.« 29So brachten denn Zadok und
Abjathar die Lade Gottes nach Jerusalem zurück und blieben dort.

\hypertarget{ff-davids-marsch-uxfcber-den-uxf6lberg-sein-auftrag-an-husai}{%
\subparagraph{ff) Davids Marsch über den Ölberg; sein Auftrag an
Husai}\label{ff-davids-marsch-uxfcber-den-uxf6lberg-sein-auftrag-an-husai}}

30David aber stieg die Anhöhe am Ölberg hinan, im Gehen weinend und mit
verhülltem Haupt, und er ging barfuß; auch alles Volk, das ihn
begleitete, hatte ein jeder das Haupt verhüllt und stieg unter
fortwährendem Weinen den Berg hinan. 31Als man nun David meldete, daß
auch Ahithophel unter den Verschworenen bei Absalom sei, rief David aus:
»O HERR, mache doch die Ratschläge Ahithophels zur Torheit!« 32Als David
dann auf der Höhe angelangt war, wo man Gott anzubeten pflegt, kam ihm
plötzlich der Arkiter Husai mit zerrissenem Gewand und mit Erde auf dem
Haupt entgegen. 33David sagte zu ihm: »Wenn du mit mir weiter zögest,
würdest du mir nur zur Last fallen; 34wenn du aber in die Stadt
zurückkehrst und zu Absalom sagst: ›O König, ich will dein Diener sein!
Wie ich bisher deines Vaters Diener gewesen bin, so will ich jetzt dein
Diener sein!‹ -- so könntest du mir die Ratschläge Ahithophels
vereiteln. 35Dort sind ja auch die Priester Zadok und Abjathar bei dir;
teile also alles, was du aus dem Hause✲ des Königs erfährst, sofort den
Priestern Zadok und Abjathar mit! 36Sie haben dort ihre beiden Söhne bei
sich, Zadok den Ahimaaz und Abjathar den Jonathan; durch diese laßt mir
Nachricht von allem zukommen, was ihr in Erfahrung bringt.« 37So begab
sich denn Davids vertrauter Freund Husai nach Jerusalem zurück, als
Absalom gerade in die Stadt einzog.

\hypertarget{gg-ziba-mephiboseths-diener-beschenkt-den-kuxf6nig-sein-luxfcgenbericht-uxfcber-mephiboseth}{%
\subparagraph{gg) Ziba, Mephiboseths Diener, beschenkt den König; sein
Lügenbericht über
Mephiboseth}\label{gg-ziba-mephiboseths-diener-beschenkt-den-kuxf6nig-sein-luxfcgenbericht-uxfcber-mephiboseth}}

\hypertarget{section-15}{%
\section{16}\label{section-15}}

1David aber hatte die Höhe des Ölberges kaum etwas überschritten, als
ihm Ziba, der Diener Mephiboseths, mit einem Paar gesattelter Esel
entgegenkam, welche zweihundert Brote, hundert Rosinentrauben, hundert
Feigenkuchen und einen Schlauch Wein trugen. 2Als der König nun Ziba
fragte: »Was willst du damit?«, antwortete Ziba: »Die Esel sind für die
königliche Familie zum Reiten bestimmt, die Brote aber und das
getrocknete Obst für die Dienerschaft zum Essen und der Wein zum Trinken
für die in der Wüste Ermatteten.« 3Als der König dann weiter fragte: »Wo
ist denn der Sohn deines (früheren) Herrn?«, erwiderte Ziba dem König:
»Ja, der ist in Jerusalem geblieben; denn er denkt, jetzt werde ihm das
Haus Israel das Königtum seines Großvaters (Saul) zurückgeben.« 4Da
sagte der König zu Ziba: »So soll denn der gesamte Besitz Mephiboseths
dir gehören!« Ziba antwortete: »Ich werfe mich huldigend nieder! Mögest
du mir auch ferner gnädig gesinnt sein, mein Herr und König!«

\hypertarget{hh-simeis-unwuxfcrdiges-benehmen-gegen-den-kuxf6nig}{%
\subparagraph{hh) Simeis unwürdiges Benehmen gegen den
König}\label{hh-simeis-unwuxfcrdiges-benehmen-gegen-den-kuxf6nig}}

5Als hierauf der König David bis Bahurim gekommen war, trat dort auf
einmal ein Mann vom Geschlecht des Hauses Saul namens Simei, der Sohn
Geras, (aus dem Orte) heraus. Unter unaufhörlichen Verwünschungen kam er
heraus 6und warf mit Steinen nach David und allen Leuten des Königs
David, obgleich das ganze Volk und die gesamte Leibwache zur Rechten und
zur Linken des Königs gingen. 7Simei stieß aber schreiend folgende
Flüche aus: »Hinweg, hinweg mit dir, du Blutmensch, Bösewicht! 8Endlich
läßt der HERR alle deine Blutschuld am Hause Sauls, an dessen Stelle du
dich zum König gemacht hast, auf dich zurückfallen, und der HERR hat das
Königtum deinem Sohne Absalom übergeben! Und siehe, nun bist du ins
Unglück geraten, weil du ein Blutmensch bist!« 9Da sagte Abisai, der
Sohn der Zeruja, zum König: »Warum soll dieser tote Hund meinem Herrn,
dem König, fluchen dürfen? Laß mich doch hingehen und ihm den Kopf
abhauen!« 10Aber der König erwiderte: »Ihr Söhne der Zeruja, was habe
ich mit euch zu schaffen? Laßt ihn doch fluchen! Denn wenn der HERR es
ihm eingegeben hat, dem David zu fluchen, wer darf dann fragen: ›Warum
tust du so?‹« 11Weiter sagte David zu Abisai und allen seinen Hofleuten:
»Wenn mein eigener leiblicher Sohn mir nach dem Leben trachtet: wieviel
mehr jetzt dieser Benjaminit! Laßt ihn fluchen, denn der HERR hat es ihm
eingegeben! 12Vielleicht sieht der HERR mein Elend an und vergilt mir
Gutes dafür, daß mir heute hier geflucht wird.« 13So zog denn David mit
seinen Leuten seines Weges weiter, während Simei am Abhang des Berges
neben ihm herging, indem er unaufhörlich Flüche ausstieß, mit Steinen
nach ihm warf und Staub aufwirbelte. 14Endlich kam der König mit allem
Volk, das ihn begleitete, ermattet (am Jordan) an; dort konnte er sich
erholen.

\hypertarget{c-absalom-in-jerusalem}{%
\paragraph{c) Absalom in Jerusalem}\label{c-absalom-in-jerusalem}}

\hypertarget{aa-absalom-von-husai-getuxe4uscht}{%
\subparagraph{aa) Absalom von Husai
getäuscht}\label{aa-absalom-von-husai-getuxe4uscht}}

15Absalom aber war unterdessen mit seinem ganzen Anhang der Israeliten
nach Jerusalem gekommen; auch Ahithophel war bei ihm. 16Als nun der
Arkiter Husai, Davids vertrauter Freund, zu Absalom kam, rief Husai dem
Absalom zu: »Es lebe der König! Es lebe der König!« 17Absalom entgegnete
ihm: »Ist das deine Treue gegen deinen Freund? Warum bist du nicht mit
deinem Freunde gezogen?« 18Da antwortete Husai dem Absalom: »Nein!
Sondern wen der HERR und das Volk hier und alle Männer von Israel
erwählt haben, dem gehöre ich an, und bei dem will ich bleiben! 19Und
zweitens: Wem leiste ich denn Dienste? Doch wohl seinem Sohne? Wie ich
deinem Vater gedient habe, so will ich auch dir zur Verfügung stehen!«

\hypertarget{bb-ahithophels-erster-rat-von-absalom-befolgt}{%
\subparagraph{bb) Ahithophels erster Rat von Absalom
befolgt}\label{bb-ahithophels-erster-rat-von-absalom-befolgt}}

20Darauf sagte Absalom zu Ahithophel: »Erteilt mir euren Rat, was wir
tun sollen!« 21Ahithophel antwortete ihm: »Gehe ein zu den Kebsweibern
deines Vaters, die er hier zur Hut des Hauses✲ zurückgelassen hat. Wenn
dann ganz Israel erfährt, daß du unwiderruflich mit deinem Vater
gebrochen hast, so werden alle, die es mit dir halten, dadurch ermutigt
werden.« 22Da schlug man für Absalom ein Zelt auf dem Dache (des
Palastes) auf, und Absalom ging zu den Kebsweibern seines Vaters vor den
Augen von ganz Israel.

23Zu jener Zeit aber galt ein Rat, den Ahithophel gab, so viel wie eine
Offenbarung Gottes: so hoch galten alle Ratschläge Ahithophels sowohl
bei David als auch bei Absalom.

\hypertarget{cc-ahithophels-zweiter-trefflicher-rat-von-husai-bekuxe4mpft-und-von-absalom-verworfen}{%
\subparagraph{cc) Ahithophels zweiter trefflicher Rat von Husai bekämpft
und von Absalom
verworfen}\label{cc-ahithophels-zweiter-trefflicher-rat-von-husai-bekuxe4mpft-und-von-absalom-verworfen}}

\hypertarget{section-16}{%
\section{17}\label{section-16}}

1Nun sagte Ahithophel zu Absalom: »Ich will mir 12000~Mann auswählen und
mich noch in dieser Nacht aufmachen, um David zu verfolgen; 2ich werde
ihn dann überfallen, während er noch ermattet und mutlos ist, und werde
ihn in solchen Schrecken versetzen, daß die gesamte Mannschaft, die er
bei sich hat, die Flucht ergreift und ich den König allein erschlagen
kann. 3Dann will ich alles Volk zu dir zurückbringen, wie man eine junge
Frau zu ihrem Gatten zurückholt. Du trachtest ja doch nur einem Manne
nach dem Leben, während das ganze übrige Volk unversehrt bleiben soll.«
4Der Rat fand den Beifall Absaloms und aller Ältesten der Israeliten;
5dennoch befahl Absalom: »Man rufe noch den Arkiter Husai! wir wollen
doch auch seine Ansicht hören!« 6Als nun Husai zu Absalom gekommen war,
sagte dieser zu ihm: »So und so hat Ahithophel geraten; sollen wir
seinen Vorschlag ausführen? Wenn nicht, so rede du!« 7Da antwortete
Husai dem Absalom: »Diesmal ist der Rat, den Ahithophel erteilt hat,
nicht gut.« 8Er fuhr dann fort: »Du weißt wohl, daß dein Vater und seine
Leute Helden sind und voll wilden Mutes wie eine Bärin auf dem Felde,
der man die Jungen geraubt hat. Außerdem ist dein Vater ein Kriegsmann,
der seine Leute nicht Nachtruhe halten läßt. 9Gewiß hat er sich schon
jetzt in irgendeiner Schlucht oder sonstwo versteckt. Sollten nun gleich
im Anfang einige von unsern Leuten fallen, so wird jeder, der es hört,
behaupten: ›Die Leute, die es mit Absalom halten, haben eine Niederlage
erlitten!‹ 10Da würde dann auch der Tapferste, der ein Herz wie ein Löwe
hat, sicherlich den Mut sinken lassen; ganz Israel weiß ja, daß dein
Vater ein Held ist und wie tapfer die Männer sind, die er bei sich hat.
11Ich rate vielmehr: Laß ganz Israel von Dan bis Beerseba bei dir sich
versammeln, so zahlreich wie der Sand am Meer, und ziehe dann persönlich
in ihrer Mitte ins Feld. 12Treffen wir ihn dann an irgendeinem Ort, wo
er sich aufhält, so fallen wir über ihn her, wie der Tau auf den
Erdboden fällt, und es soll von ihm und allen Männern, die er bei sich
hat, auch nicht einer übrigbleiben! 13Zieht er sich aber in eine Stadt
zurück, so soll ganz Israel Seile an die betreffende Stadt legen, und
wir schleifen sie ins Tal hinunter, bis auch nicht ein Steinchen mehr
dort zu finden ist.« 14Da erklärten Absalom und alle Israeliten: »Der
Rat des Arkiters Husai ist besser als der Rat Ahithophels!« Der HERR
hatte es nämlich so gefügt, daß der gute Rat Ahithophels verworfen
wurde, weil der HERR Unheil über Absalom bringen wollte.

\hypertarget{dd-husai-und-die-priester-schicken-heimlich-botschaft-an-den-kuxf6nig-david-setzt-uxfcber-den-jordan}{%
\subparagraph{dd) Husai und die Priester schicken heimlich Botschaft an
den König; David setzt über den
Jordan}\label{dd-husai-und-die-priester-schicken-heimlich-botschaft-an-den-kuxf6nig-david-setzt-uxfcber-den-jordan}}

15Hierauf teilte Husai den Priestern Zadok und Abjathar mit: »So und so
hat Ahithophel dem Absalom und den Ältesten der Israeliten geraten, und
so und so habe ich geraten. 16Laßt also jetzt in aller Eile folgende
Botschaft an David gelangen: ›Bleibe über Nacht nicht mehr in den
Niederungen der Wüste (Juda), sondern setze auf jeden Fall (über den
Jordan) hinüber, damit der König nicht mit allen Leuten, die er bei sich
hat, vom Verderben ereilt wird.‹« 17Jonathan und Ahimaaz hatten aber
ihren Standort bei der Quelle Rogel\textless sup title=``d.h.
Walkerquelle''\textgreater✲, und eine Magd mußte von Zeit zu Zeit
hingehen und ihnen Nachricht bringen; dann gingen sie jedesmal hin und
erstatteten dem König Bericht; denn sie durften sich nicht sehen lassen,
daß sie in die Stadt hätten kommen können. 18Aber ein Knabe bemerkte sie
und teilte es Absalom mit. Da entfernten die beiden sich eiligst und
begaben sich in das haus eines Mannes zu Bachurim. Dieser hatte in
seinem Hofe eine Zisterne, in die sie sich hinabließen; 19die Frau nahm
dann eine Decke, breitete diese oben über die Zisterne aus und schüttete
Grütze darüber, so daß man nichts bemerken konnte. 20Als nun die Leute
Absaloms zu der Frau ins Haus kamen und fragten, wo Ahimaaz und Jonathan
seien, antwortete ihnen die Frau: »Sie sind von hier nach dem
Wasser\textless sup title=``d.h. zum Jordan''\textgreater✲
weitergegangen.« Als jene sie nun trotz alles Suchens nicht fanden,
kehrten sie nach Jerusalem zurück. 21Nach ihrem Weggang stiegen dann die
beiden aus der Zisterne herauf, gingen weiter, erstatteten dem König
David Bericht und sagten zu David: »Macht euch auf und setzt eilends
über den Jordan! Denn so und so hat Ahithophel in bezug auf euch
geraten.« 22Da machte sich David mit allen Leuten, die er bei sich
hatte, auf den Weg, und sie setzten über den Jordan. Bis der Morgen
tagte, befanden sich alle bis auf den letzten Mann auf der andern Seite
des Jordans.

\hypertarget{ee-ahithophels-selbstmord}{%
\subparagraph{ee) Ahithophels
Selbstmord}\label{ee-ahithophels-selbstmord}}

23Als nun Ahithophel sah, daß sein Rat nicht ausgeführt wurde, ließ er
seinen Esel satteln und machte sich auf den Heimweg nach seinem Wohnort.
Nachdem er dort sein Haus bestellt hatte, erhängte er sich; seine Leiche
wurde dann im Begräbnis seines Vaters beigesetzt.

\hypertarget{d-die-kriegerische-entscheidung-absaloms-niederlage-und-tod}{%
\paragraph{d) Die kriegerische Entscheidung: Absaloms Niederlage und
Tod}\label{d-die-kriegerische-entscheidung-absaloms-niederlage-und-tod}}

\hypertarget{aa-absalom-beginnt-die-verfolgung-davids-und-uxfcbertruxe4gt-dem-amasa-den-oberbefehl-david-in-mahanaim}{%
\subparagraph{aa) Absalom beginnt die Verfolgung Davids und überträgt
dem Amasa den Oberbefehl; David in
Mahanaim}\label{aa-absalom-beginnt-die-verfolgung-davids-und-uxfcbertruxe4gt-dem-amasa-den-oberbefehl-david-in-mahanaim}}

24David aber war bereits in Mahanaim angekommen, als Absalom mit allen
Israeliten über den Jordan setzte. 25An Stelle Joabs hatte Absalom dem
Amasa den Oberbefehl über das Heer übertragen. Dieser Amasa war der Sohn
eines gewissen Jithra, eines Ismaeliten, der mit Abigail, der Tochter
des Nahas, der Halbschwester der Zeruja, der Mutter Joabs, ein
Verhältnis gehabt hatte. 26So lagerten sich denn die Israeliten unter
Absalom in der Landschaft Gilead.

27Als aber David in Mahanaim angekommen war, hatten Sobi, der Sohn des
Nahas, aus Rabba, (der Hauptstadt) der Ammoniter, und Machir, der Sohn
Ammiels, aus Lodebar, und der Gileaditer Barsillai aus Rogelim
28Ruhebetten, Becken\textless sup title=``oder: Decken''\textgreater✲
und irdenes Geschirr, dazu Weizen, Gerste, Mehl, geröstetes Korn,
Bohnen, Linsen, 29Honig und Butter, Kleinvieh und Kuhkäse für David und
seine Leute zur Nahrung gebracht; denn sie hatten gedacht: »Die Leute
müssen in der Steppe hungrig, müde und durstig geworden sein.«

\hypertarget{bb-davids-milituxe4rische-anordnungen-auszug-seines-heeres}{%
\subparagraph{bb) Davids militärische Anordnungen; Auszug seines
Heeres}\label{bb-davids-milituxe4rische-anordnungen-auszug-seines-heeres}}

\hypertarget{section-17}{%
\section{18}\label{section-17}}

1Nun musterte David das Kriegsvolk, das bei ihm war, und setzte Anführer
ein über je tausend und über je hundert Mann. 2Dann ließ David das Heer
ausrücken, ein Drittel unter dem Befehl Joabs, ein Drittel unter Abisai,
dem Sohn der Zeruja, dem Bruder Joabs, und das letzte Drittel unter dem
Befehl Itthais aus Gath. Dabei sagte der König zu der Mannschaft: »Ich
will auch mit euch ausziehen!« 3Aber die Leute entgegneten: »Nein, du
darfst nicht mitziehen! Denn wenn wir geschlagen werden, wird man sich
um uns nicht kümmern; und wenn auch die Hälfte von uns fiele, würde man
sich um uns nicht kümmern, du dagegen bist so viel wert wie zehntausend
von uns. Außerdem ist es besser, wenn du uns jederzeit von der Stadt aus
Hilfe leisten kannst.« 4Da antwortete ihnen der König: »Ich will eurem
Wunsche nachkommen!« Hierauf trat der König neben das Tor, während das
ganze Heer nach Hunderten und Tausenden auszog. 5Dem Joab, Abisai und
Itthai aber erteilte der König den Befehl: »Geht mir schonend mit dem
jungen Mann, mit Absalom, um!« Und das gesamte Kriegsvolk hörte es mit
an, wie der König allen Anführern diesen Befehl in betreff Absaloms
erteilte.

\hypertarget{cc-absalom-wird-besiegt-und-von-joab-selbst-getuxf6tet-sein-grab}{%
\subparagraph{cc) Absalom wird besiegt und von Joab selbst getötet; sein
Grab}\label{cc-absalom-wird-besiegt-und-von-joab-selbst-getuxf6tet-sein-grab}}

6Als so das Heer ins Feld gegen die Israeliten gezogen war und es im
Walde Ephraim zur Schlacht kam, 7wurde dort das Heer der Israeliten von
den Leuten Davids besiegt, so daß sie an diesem Tage dort eine schwere
Niederlage, einen Verlust von zwanzigtausend Mann, erlitten. 8Der Kampf
breitete sich dann über die ganze Gegend dort aus, und der Wald brachte
an diesem Tage noch mehr Leuten den Tod, als das Schwert es getan hatte.

9Da kam Absalom zufällig den Leuten Davids zu Gesicht. Er ritt nämlich
auf einem Maultier, und als dieses unter die verschlungenen Zweige einer
großen Terebinthe geraten war, blieb er mit dem Haupt (-haar) an der
Terebinthe hangen, so daß er zwischen Himmel und Erde\textless sup
title=``=~in der Luft''\textgreater✲ schwebte, nachdem das Maultier
unter ihm davongelaufen war. 10Als das ein Mann sah, erstattete er dem
Joab die Meldung: »Ich habe soeben Absalom an einer Terebinthe hängen
sehen.« 11Da erwiderte Joab dem Manne, der ihm die Mitteilung gemacht
hatte: »Nun, wenn du ihn gesehen hast, warum hast du ihn dort nicht
gleich zur Erde heruntergeschlagen? Ich wäre dann in der Lage gewesen,
dir zehn Silberstücke und einen Gürtel\textless sup title=``oder: ein
Wehrgehänge''\textgreater✲ zu geben!« 12Der Mann aber antwortete dem
Joab: »Und wenn mir tausend Silberstücke in die Hand gezahlt würden,
wollte ich mich doch nicht an dem Sohne des Königs vergreifen; wir haben
ja mit eigenen Ohren gehört, wie der König dir und Abisai und Itthai den
bestimmten Befehl gegeben hat: ›Verfahrt mir schonend mit dem jungen
Manne, mit Absalom!‹ 13Hätte ich mich nun frevelhaft an seinem Leben
vergriffen, so wäre die ganze Sache dem König doch nicht verborgen
geblieben, und du selbst würdest dich abseits stellen\textless sup
title=``oder: gegen mich aufgetreten sein''\textgreater✲.« 14Da sagte
Joab: »Ich mag hier bei dir keine Zeit verlieren!« Hierauf nahm er drei
Speere in die Hand und stieß sie dem Absalom ins Herz, während er, noch
lebend, in den Zweigen der Terebinthe hing. 15Dann traten zehn Knappen,
Joabs Waffenträger, rings hinzu, schlugen Absalom herunter und töteten
ihn vollends. 16Darauf ließ Joab ein Zeichen mit der Posaune geben, da
standen die Leute Davids von der Verfolgung der Israeliten ab; denn dem
Volk wollte Joab Schonung erweisen. 17Darauf nahmen sie Absalom, warfen
ihn im Walde in eine große Grube und türmten einen gewaltigen
Steinhaufen über ihm auf. Alle Israeliten aber flohen, ein jeder in
seinen Wohnort.~-- 18Absalom hatte aber schon bei seinen Lebzeiten den
Denkstein, der im Königstal steht, genommen und ihn für sich
aufgerichtet; er hatte nämlich gedacht: »Ich habe keinen Sohn, der
meinen Namen in der Erinnerung erhalten könnte.« Er hatte also den
Denkstein nach seinem Namen benannt; und darum heißt er bis auf den
heutigen Tag ›Absaloms Denkmal‹.

\hypertarget{e-david-erhuxe4lt-die-nachricht-vom-tode-absaloms-seine-trauer}{%
\paragraph{e) David erhält die Nachricht vom Tode Absaloms; seine
Trauer}\label{e-david-erhuxe4lt-die-nachricht-vom-tode-absaloms-seine-trauer}}

\hypertarget{aa-der-wettlauf-des-ahimaaz-und-des-mohren}{%
\subparagraph{aa) Der Wettlauf des Ahimaaz und des
Mohren}\label{aa-der-wettlauf-des-ahimaaz-und-des-mohren}}

19Ahimaaz aber, der Sohn Zadoks, sagte (zu Joab): »Ich möchte gern
hinlaufen und dem König die Botschaft bringen, daß der HERR ihm den Sieg
über seine Feinde verliehen hat!« 20Aber Joab erwiderte ihm: »Du bist
heute nicht der rechte Mann für diese Botschaft; ein andermal magst du
Bote sein, aber heute darfst du die Botschaft nicht überbringen, weil ja
der Sohn des Königs tot ist.« 21Hierauf befahl Joab seinem Mohren: »Gehe
hin, melde dem König, was du gesehen hast!« Da warf sich der Mohr vor
Joab nieder und eilte davon. 22Aber Ahimaaz, der Sohn Zadoks, sagte
nochmals zu Joab: »Mag kommen, was da will: laß doch auch mich hinter
dem Mohren herlaufen!« Joab entgegnete: »Wozu willst du denn laufen,
mein Sohn? Dir wird doch kein Botenlohn ausgezahlt werden.« 23Er
antwortete: »Mag kommen, was da will: ich laufe!« Da sagte Joab zu ihm:
»Nun, so laufe!« Da schlug Ahimaaz den Weg durch die Jordanaue ein und
überholte den Mohren.

\hypertarget{bb-david-im-tor-von-mahanaim-sein-schmerz-uxfcber-absaloms-tod}{%
\subparagraph{bb) David im Tor von Mahanaim; sein Schmerz über Absaloms
Tod}\label{bb-david-im-tor-von-mahanaim-sein-schmerz-uxfcber-absaloms-tod}}

24David aber saß gerade inmitten der beiden Torpforten\textless sup
title=``=~inmitten der Torhalle''\textgreater✲, während der Späher auf
der Mauer nach dem Tordach zu ging. Als dieser nun Ausschau hielt, sah
er einen einzelnen Mann heranlaufen. 25Da meldete es der Späher dem
König durch Zuruf. Der König sagte: »Wenn es nur einer ist, so hat er
gute Nachricht zu überbringen.« Während nun jener immer näher kam, 26sah
der Späher noch einen zweiten Mann heranlaufen und rief ins Tor hinein:
»Ich sehe noch einen zweiten Mann allein heranlaufen!« Da sagte der
König: »Auch der bringt gute Botschaft.« 27Der Späher rief dann weiter:
»Am Laufen des ersten glaube ich Ahimaaz, den Sohn Zadoks, zu erkennen!«
Der König sagte: »Das ist ein braver Mann: der kommt gewiß mit guter
Botschaft!« 28Ahimaaz aber rief dem König zu: »Sieg!« Dann warf er sich
vor dem Könige mit dem Gesicht zur Erde nieder und rief aus: »Gepriesen
sei der HERR, dein Gott, der die Männer dahingegeben hat, die ihre Hand
gegen meinen Herrn, den König, erhoben haben!« 29Der König aber fragte:
»Geht es dem jungen Manne, dem Absalom, gut?« Ahimaaz antwortete: »Ich
sah ein großes Getümmel, als des Königs Diener Joab deinen Knecht
absandte, weiß aber nicht, was da vorging.« 30Der König entgegnete:
»Tritt ab und bleibe hier stehen!« Da trat er ab und stellte sich
abseits, 31als auch schon der Mohr ankam und ausrief: »Mein Herr, der
König, lasse sich die Freudenbotschaft melden, daß der HERR dir heute
den Sieg über alle verliehen hat, die sich gegen dich empört haben!«
32Da fragte der König den Mohren: »Geht es dem jungen Manne, dem
Absalom, gut?« Der Mohr antwortete: »Wie dem jungen Manne, so möge es
den Feinden des Königs, meines Herrn, und allen ergehen, die sich in
böser Absicht gegen dich auflehnen!«

\hypertarget{section-18}{%
\section{19}\label{section-18}}

1Da erbebte der König, stieg in das Obergemach des Torgebäudes hinauf
und weinte; im Gehen aber rief er die Worte aus: »Mein Sohn Absalom!
Mein Sohn! Mein Sohn Absalom! Wäre doch ich selber statt deiner
gestorben! O Absalom, mein Sohn, mein Sohn!«

\hypertarget{cc-uxfcble-wirkung-der-trauer-davids-auf-das-heer-joabs-tadel-david-rafft-sich-auf}{%
\subparagraph{cc) Üble Wirkung der Trauer Davids auf das Heer; Joabs
Tadel; David rafft sich
auf}\label{cc-uxfcble-wirkung-der-trauer-davids-auf-das-heer-joabs-tadel-david-rafft-sich-auf}}

2Als nun dem Joab berichtet wurde, daß der König um Absalom weine und
trauere, 3da wurde der Sieg an diesem Tage zur Trauer für das ganze
Volk, weil jedermann an diesem Tage erfuhr, daß der König um seinen Sohn
Leid trage. 4So stahl sich denn das Heer an jenem Tage zum Einzug in die
Stadt heran, wie sich ein Heer heranstiehlt, das sich mit Schmach
bedeckt hat, weil es in der Schlacht geflohen ist. 5Der König aber hatte
sich das Gesicht verhüllt und wehklagte laut: »Mein Sohn Absalom!
Absalom, mein Sohn, mein Sohn!« 6Da begab sich Joab zum König ins Haus
und sagte: »Du hast heute alle deine Knechte offen beschimpft, obgleich
sie heute dir sowie deinen Söhnen und Töchtern, deinen Frauen und
Nebenweibern das Leben gerettet haben! 7Denn du hast denen, die dich
hassen, Liebe und denen, die dich lieben, Haß erwiesen; du hast ja heute
offen gezeigt, daß dir an deinen Heerführern und Knechten nichts gelegen
ist; ja, jetzt weiß ich, daß, wenn nur Absalom noch lebte und wir
anderen alle heute tot wären, dir das gerade recht sein würde. 8Nun aber
stehe auf, laß dich öffentlich sehen und gönne deinen Knechten ein
freundliches Wort! Denn ich schwöre dir beim HERRN: Wenn du dich nicht
öffentlich sehen läßt, so bleibt kein Mann mehr diese Nacht bei dir, und
das wäre für dich schlimmer als alles Unglück, das du von deiner Jugend
an bis jetzt erlebt hast!« 9Da stand der König auf und setzte sich ins
Tor; und als es dem ganzen Volk bekannt wurde, daß der König (nunmehr)
im Tor sitze, erschienen alle Leute vor dem Könige.

\hypertarget{f-davids-ruxfcckkehr-aus-mahanaim}{%
\paragraph{f) Davids Rückkehr aus
Mahanaim}\label{f-davids-ruxfcckkehr-aus-mahanaim}}

\hypertarget{aa-umschlag-der-volksstimmung-fuxfcr-david-davids-verhandlungen-mit-den-uxe4ltesten-von-juda-und-mit-amasa}{%
\subparagraph{aa) Umschlag der Volksstimmung für David; Davids
Verhandlungen mit den Ältesten von Juda und mit
Amasa}\label{aa-umschlag-der-volksstimmung-fuxfcr-david-davids-verhandlungen-mit-den-uxe4ltesten-von-juda-und-mit-amasa}}

Als nun die Israeliten geflohen waren, ein jeder in seinen Wohnort,
10erhob das ganze Volk in allen Stämmen der Israeliten Vorwürfe gegen
sich selbst; überall hieß es: »Der König hat uns aus der Gewalt unserer
Feinde errettet, er hat uns von der Herrschaft der Philister befreit,
und jetzt hat er vor Absalom aus dem Lande fliehen müssen! 11Nun aber,
da Absalom, den wir zum König über uns gesalbt hatten, in der Schlacht
ums Leben gekommen ist: warum zögert ihr da noch, den König
zurückzuholen?« 12(Als diese Äußerungen des gesamten Volkes zum König
drangen) sandte der König David zu den Priestern Zadok und Abjathar und
ließ ihnen sagen: »Redet mit den Ältesten von Juda und gebt ihnen zu
erwägen: ›Warum wollt ihr die Letzten sein, die den König in sein Haus
zurückführen? 13Ihr seid doch meine Stammesgenossen, seid von meinem
Fleisch und Bein: warum wollt ihr also die Letzten sein, wo es gilt, den
König heimzuholen?‹ 14Und zu Amasa sollt ihr sagen: ›Du bist ja doch von
meinem Fleisch und Bein: Gott strafe mich jetzt und künftig, wenn du
nicht oberster Heerführer bei mir auf Lebenszeit an Joabs Statt wirst!‹«
15So gewann er die Herzen aller Männer von Juda, so daß sie einmütig die
Aufforderung an den König richteten: »Kehre du mit allen deinen
Dienern\textless sup title=``oder: deinem ganzen Hofe''\textgreater✲
zurück!«

\hypertarget{bb-david-tritt-die-ruxfcckkehr-an-und-wird-von-den-juduxe4ern-eingeholt-seine-milde-gegen-simei}{%
\subparagraph{bb) David tritt die Rückkehr an und wird von den Judäern
eingeholt; seine Milde gegen
Simei}\label{bb-david-tritt-die-ruxfcckkehr-an-und-wird-von-den-juduxe4ern-eingeholt-seine-milde-gegen-simei}}

16So trat denn der König den Rückweg an, und als er an den Jordan
gelangte, waren ihm die Judäer nach Gilgal entgegengekommen, um den
König einzuholen und ihn über den Jordan zu geleiten. 17Auch der
Benjaminit Simei, der Sohn Geras, war aus Bachurim mit der Mannschaft
der Judäer dem König David entgegengeeilt 18und mit ihm tausend Mann aus
dem Stamme Benjamin; außerdem auch Ziba, der Hausverwalter Sauls, mit
seinen fünfzehn Söhnen und seinen zwanzig Knechten; sie waren schon vor
der Ankunft des Königs über den Jordan gesetzt. 19Als nun die Fähre
hinübergefahren war, um die königliche Familie herüberzuholen und sich
dem König zur Verfügung zu stellen, warf sich Simei, der Sohn Geras, vor
dem König nieder, als dieser eben über den Jordan fahren wollte, 20und
richtete an den König die Worte: »Mein Herr wolle mir keine Verschuldung
anrechnen und nicht des Vergehens gedenken, das dein Knecht sich hat
zuschulden kommen lassen an dem Tage, als mein Herr, der König,
Jerusalem verließ! Der König wolle es mir nicht unversöhnlich
nachtragen! 21Dein Knecht weiß ja, daß ich mich vergangen habe; doch,
wie du siehst, bin ich heute als der erste vom ganzen Hause Joseph
herabgekommen, um meinen Herrn, den König, einzuholen.« 22Als nun
Abisai, der Sohn der Zeruja, das Wort nahm und ausrief: »Sollte Simei
nicht den Tod dafür erleiden, daß er dem Gesalbten des HERRN geflucht
hat?«, 23entgegnete David: »Ihr Söhne der Zeruja, was habe ich mit euch
zu tun, daß ihr mir heute zum Satan✲ werden wollt? Heute soll niemand in
Israel den Tod erleiden, da ich doch weiß, daß ich heute wieder König
über Israel bin!« 24Hierauf sagte der König zu Simei: »Du sollst nicht
sterben!«, und der König bekräftigte es ihm mit einem Eide.

\hypertarget{cc-mephiboseth-rechtfertigt-sich-gegen-david}{%
\subparagraph{cc) Mephiboseth rechtfertigt sich gegen
David}\label{cc-mephiboseth-rechtfertigt-sich-gegen-david}}

25Auch Mephiboseth, Sauls Enkel, war herabgekommen dem König entgegen;
er hatte aber weder seine Füße gereinigt, noch seinen Bart gepflegt,
noch seine Kleider gewaschen seit dem Tage, an dem der König weggezogen
war, bis zu dem Tage, an dem er glücklich heimkehrte. 26Als er nun von
Jerusalem her dem König entgegenkam, fragte der König ihn: »Mephiboseth,
warum bist du nicht mit mir ausgezogen?« 27Da antwortete er: »Mein Herr
und König! Mein Diener hat mich betrogen! Dein Knecht hatte sich nämlich
vorgenommen: ›Ich will mir doch meinen Esel satteln lassen und darauf
reiten, um mit dem König zu ziehen -- dein Knecht ist ja lahm --; 28aber
er hat deinen Knecht bei meinem Herrn, dem König, verleumdet. Jedoch
mein Herr, der König, gleicht (an Weisheit) dem Engel Gottes; so tu nun,
was dir gefällt! 29Denn da das ganze Haus meines Vaters nichts anderes
von meinem Herrn und König hat erwarten dürfen als den Tod und du
dennoch deinen Knecht unter deine Tischgenossen aufgenommen hast --
welches Recht hätte ich da noch, und was hätte ich da noch vom König zu
beanspruchen?« 30Der König antwortete ihm: »Was machst du da noch Worte?
Ich bestimme hiermit: Du und Ziba sollt euch in den Grundbesitz teilen!«
31Da sagte Mephiboseth zum König: »Er mag sogar das Ganze hinnehmen,
nachdem mein Herr und König glücklich heimgekehrt ist!«

\hypertarget{dd-barsillais-herzliche-unterredung-mit-david-uxfcbergang-uxfcber-den-jordan}{%
\subparagraph{dd) Barsillais herzliche Unterredung mit David; Übergang
über den
Jordan}\label{dd-barsillais-herzliche-unterredung-mit-david-uxfcbergang-uxfcber-den-jordan}}

32Auch der Gileaditer Barsillai war von Rogelim herabgekommen und mit
dem König an den Jordan gezogen, doch nur um ihn den Jordan entlang zu
geleiten. 33Barsillai war nämlich sehr alt, ein Mann von achtzig Jahren;
und er war's gewesen, der den König während seines Aufenthalts in
Mahanaim (mit Lebensmitteln) versorgt hatte, weil er ein sehr reicher
Mann war. 34Nun sagte der König zu Barsillai: »Du mußt mit mir
hinüberfahren; ich will für deinen Unterhalt bei mir in deinen alten
Tagen in Jerusalem sorgen.« 35Aber Barsillai erwiderte dem König: »Wie
viele sind noch der Tage meiner Lebensjahre, daß ich mit dem König nach
Jerusalem hinaufziehen sollte? 36Ich bin jetzt achtzig Jahre alt: wie
könnte ich da noch zwischen Gutem und Schlechtem unterscheiden? Kann
dein Knecht etwa noch schmecken, was ich esse und trinke? Oder kann ich
noch der Stimme der Sänger und Sängerinnen lauschen? Wozu sollte also
dein Knecht meinem Herrn, dem König, noch zur Last fallen? 37Nein, nur
eben über den Jordan möchte dein Knecht mit dem König fahren. Und warum
will der König mir mit so reichem Lohn vergelten? 38Laß doch deinen
Knecht heimkehren, damit ich in meiner Vaterstadt beim Grabe meines
Vaters und meiner Mutter sterbe! Aber siehe, hier ist (mein Sohn,) dein
Knecht Kimham: der mag mit meinem Herrn, dem König, hinüberfahren, und
tu an ihm, was du für gut hältst!« 39Der König antwortete: »Ja, Kimham
soll mit mir hinüberfahren, und ich will an ihm tun, was dir erfreulich
ist, und will dir jeden Wunsch erfüllen!« 40Als dann alles Kriegsvolk
über den Jordan gesetzt und auch der König hinübergefahren war, küßte
dieser den Barsillai und nahm mit Segenswünschen Abschied von ihm;
darauf kehrte jener in seinen Wohnort zurück, 41während der König nach
Gilgal weiterfuhr und Kimham ihn begleitete. Das gesamte Kriegsvolk von
Juda aber und auch die Hälfte des Kriegsvolkes von Israel war mit dem
König hinübergezogen.

\hypertarget{g-eifersucht-und-bitterer-hader-zwischen-israel-und-juda-bei-der-einholung-davids}{%
\paragraph{g) Eifersucht und bitterer Hader zwischen Israel und Juda bei
der Einholung
Davids}\label{g-eifersucht-und-bitterer-hader-zwischen-israel-und-juda-bei-der-einholung-davids}}

42Da kamen plötzlich alle Männer Israels zum König und fragten ihn:
»Warum haben unsere Volksgenossen, die Judäer, dich entführt und haben
den König mit seiner Familie und seinem ganzen Hofe über den Jordan
gebracht?« 43Da antworteten alle Judäer den Israeliten: »Der König steht
uns doch am nächsten! Warum regt ihr euch hierüber so auf? Haben wir
etwa auf Kosten des Königs gelebt? Oder hat er uns irgendein Geschenk
gemacht?« 44Aber die Israeliten entgegneten den Judäern: »Wir haben den
zehnfachen Anteil am König, und somit haben wir auch an David mehr
Anrecht als ihr: warum habt ihr uns also zurückgesetzt? Und haben wir
nicht zuerst die Absicht ausgesprochen, unsern König zurückzuholen?« Die
Worte der Judäer aber waren darauf noch leidenschaftlicher als die der
Israeliten.

\hypertarget{sebas-aufstand}{%
\subsubsection{10. Sebas Aufstand}\label{sebas-aufstand}}

\hypertarget{a-davids-anordnungen-in-jerusalem}{%
\paragraph{a) Davids Anordnungen in
Jerusalem}\label{a-davids-anordnungen-in-jerusalem}}

\hypertarget{section-19}{%
\section{20}\label{section-19}}

1Nun befand sich dort zufällig ein nichtswürdiger Mensch namens Seba,
der Sohn Bichris, ein Benjaminit; der stieß in die Posaune und rief aus:
»Wir haben keinen Anteil an David und nichts zu schaffen mit dem Sohne
Isais! Ein jeder begebe sich in seinen Wohnort, ihr Israeliten!« 2Da
fielen die Israeliten insgesamt von David ab und schlossen sich an Seba,
den Sohn Bichris, an; die Judäer aber blieben ihrem König treu (und
geleiteten ihn) vom Jordan bis nach Jerusalem.

3Als nun David in seinen Palast nach Jerusalem zurückgekommen war, ließ
er die zehn Nebenweiber, die er zur Hut des Palastes zurückgelassen
hatte\textless sup title=``vgl. 15,16; 16,21-22''\textgreater✲, in ein
besonderes Haus bringen und sorgte dort für ihren Unterhalt, hatte aber
keinen Verkehr mehr mit ihnen; so lebten sie eingesperrt bis zu ihrem
Todestag gleichsam als Witwen bei Lebzeiten (ihres Mannes).

4Darauf befahl der König dem Amasa: »Biete mir die Mannschaft von Juda
binnen drei Tagen auf und sei du selbst dann hier zur Stelle!« 5Amasa
machte sich nun daran, die Judäer aufzubieten; als er jedoch über die
ihm genau bestimmte Zeit hinaus ausblieb, 6sagte David zu Abisai: »Nun
wird Seba, der Sohn Bichris, für uns noch gefährlicher werden als
Absalom. Nimm du die Leute deines Herrn und verfolge ihn\textless sup
title=``d.h. Seba''\textgreater✲, damit er nicht etwa feste Städte für
sich gewinnt und uns viel zu schaffen macht\textless sup title=``oder:
uns entkommt''\textgreater✲!« 7Da zogen denn unter Abisais Führung Joab
mit seinen Leuten sowie die (Leibwache der) Krethi und
Plethi\textless sup title=``vgl. zu 8,18''\textgreater✲ und alle
›Kriegshelden‹ ins Feld; sie zogen aus Jerusalem aus, um Seba, den Sohn
Bichris, zu verfolgen.

\hypertarget{b-amasas-ermordung-durch-joab}{%
\paragraph{b) Amasas Ermordung durch
Joab}\label{b-amasas-ermordung-durch-joab}}

8Als sie nun bei dem großen Stein in Gibeon waren, kam Amasa ihnen zu
Gesicht. Joab aber war mit seinem Waffenrock bekleidet und hatte sich
darüber ein Schwert umgegürtet, das ihm in seiner Scheide an die Hüfte
gekoppelt war und das er, als er vorging, aus der Scheide herausfallen
ließ. 9Darauf redete Joab den Amasa mit den Worten an: »Geht es dir gut,
lieber Bruder?« Dabei faßte Joab mit der rechten Hand Amasa beim Bart,
um ihn zu küssen. 10Amasa hatte aber nicht auf das Schwert geachtet, das
Joab in der (linken) Hand hatte; so stieß Joab es ihm in den Leib, so
daß ihm die Eingeweide auf die Erde herausfielen und er starb, ohne daß
er ihm noch einen zweiten Stoß zu versetzen brauchte. Während dann Joab
und sein Bruder Abisai die Verfolgung Sebas, des Sohnes Bichris,
fortsetzten, 11mußte einer von den Leuten Joabs bei Amasa stehen bleiben
und ausrufen: »Wer es mit Joab hält und wer für David ist, folge Joab
nach!« 12Amasa aber hatte sich in seinem Blute gewälzt und lag mitten
auf der Straße. Als nun der Mann sah, daß die Leute alle stehenblieben,
schaffte er Amasa von der Straße weg aufs Feld und warf einen Mantel
über ihn, weil er sah, daß alle, die an ihn herankamen, stehenblieben.
13Nachdem er ihn aber von der Straße weggeschafft hatte, zogen alle
Leute vorüber hinter Joab her, um an der Verfolgung Sebas teilzunehmen.

\hypertarget{c-seba-von-joab-bekriegt-und-auf-anstiften-einer-klugen-frau-ermordet-joabs-ruxfcckkehr-nach-jerusalem}{%
\paragraph{c) Seba von Joab bekriegt und auf Anstiften einer klugen Frau
ermordet; Joabs Rückkehr nach
Jerusalem}\label{c-seba-von-joab-bekriegt-und-auf-anstiften-einer-klugen-frau-ermordet-joabs-ruxfcckkehr-nach-jerusalem}}

14Dieser hatte aber alle Stämme Israels bis nach Abel-Beth-Maacha
durchzogen (freilich mit geringem Erfolg); nur eben alle Bichrileute
waren hinter ihm hergekommen, ebenfalls dorthin. 15Nun kamen
jene\textless sup title=``d.h. die Leute Joabs''\textgreater✲ heran und
belagerten ihn in Abel-Beth-Maacha; sie führten gegen die Stadt einen
Wall auf, der an die Außenmauer stieß; und alle Leute Joabs unterwühlten
die Mauer, um sie zum Einsturz zu bringen. 16Da (trat) eine kluge Frau
(auf die Vormauer und) rief aus der Stadt heraus: »Hört, hört! Fordert
doch Joab auf, hierher zu kommen: ich möchte mit ihm reden!« 17Als er
nun nahe an sie herangekommen war, fragte die Frau: »Bist du Joab?« Er
antwortete ihr: »Ja, ich bin's.« Da sagte sie zu ihm: »Höre, was deine
Magd dir zu sagen hat!« Er antwortete: »Ich höre!« 18Da fuhr sie fort:
»Früher pflegte der Volksmund zu sagen: ›Fragt nur in Abel an!‹, und so
kam man glücklich ans Ziel. 19Wir gehören zu den friedlichsten,
getreusten Leuten in Israel, und du suchst eine Stadt, eine
Muttergemeinde in Israel zu zerstören? Warum willst du das Eigentum des
HERRN zugrunde richten?« 20Da antwortete Joab: »Ganz fern liegt es mir,
daß ich zerstören und daß ich zugrunde richten will. 21Die Sache liegt
nicht so, sondern ein Mann vom Gebirge Ephraim namens Seba, der Sohn
Bichris, hat sich gegen den König, gegen David, empört; liefert ihn aus,
ihn allein, so ziehe ich von der Stadt ab!« Da erwiderte die Frau dem
Joab: »Sein Kopf soll dir alsbald über die Mauer zugeworfen werden!«
22Hierauf redete die Frau (in der Stadt) mit ihrer Klugheit auf die
ganze Einwohnerschaft so lange ein, bis sie Seba, dem Sohne Bichris, den
Kopf abhieben und ihn dem Joab zuwarfen. Da ließ Joab mit der Posaune
zum Abzug blasen, und seine Leute zogen von der Stadt ab und zerstreuten
sich, ein jeder in seinen Wohnort; Joab aber kehrte nach Jerusalem zum
König zurück.

\hypertarget{iv.-nachtruxe4ge-zum-zweiten-buch-samuel-2023-2425}{%
\subsection{IV. Nachträge zum zweiten Buch Samuel
(20,23-24,25)}\label{iv.-nachtruxe4ge-zum-zweiten-buch-samuel-2023-2425}}

\hypertarget{davids-oberste-beamte}{%
\subsubsection{1. Davids oberste Beamte}\label{davids-oberste-beamte}}

23Joab war oberster Heerführer in Israel; Benaja, der Sohn Jojadas, war
Befehlshaber (der Leibwache) der Krethi und Plethi\textless sup
title=``vgl. zu 8,18''\textgreater✲; 24Adoram\textless sup title=``oder:
Adoniram''\textgreater✲ war Oberaufseher über die Fronarbeiten;
Josaphat, der Sohn Ahiluds, war Kanzler; 25Seja war Staatsschreiber;
Zadok und Abjathar waren Priester, 26und Ira, der Jairit, war ebenfalls
ein Priester Davids.

\hypertarget{suxfchnung-einer-von-saul-an-den-gibeoniten-begangenen-blutschuld}{%
\subsubsection{2. Sühnung einer von Saul an den Gibeoniten begangenen
Blutschuld}\label{suxfchnung-einer-von-saul-an-den-gibeoniten-begangenen-blutschuld}}

\hypertarget{a-darlegung-der-verschuldung-sauls-die-forderung-der-gibeoniten}{%
\paragraph{a) Darlegung der Verschuldung Sauls; die Forderung der
Gibeoniten}\label{a-darlegung-der-verschuldung-sauls-die-forderung-der-gibeoniten}}

\hypertarget{section-20}{%
\section{21}\label{section-20}}

1Unter der Regierung Davids herrschte einst eine Hungersnot drei Jahre
lang, Jahr für Jahr. Als sich David nun an den HERRN mit einer Anfrage
wandte, antwortete der HERR, auf Saul und seinem Hause laste eine
Blutschuld, weil er die Gibeoniten getötet habe. 2Da ließ der König die
Gibeoniten kommen und fragte sie -- die Gibeoniten gehörten nämlich
nicht zu den Israeliten, sondern zu dem Überrest der Amoriter; obgleich
nun die Israeliten einen Vertrag (mit ihnen geschlossen und ihn)
beschworen hatten, war Saul doch in seinem Eifer für die Israeliten und
Judäer darauf ausgegangen, sie auszurotten.~-- 3David fragte also die
Gibeoniten: »Was soll ich für euch tun, und womit soll ich Sühne
schaffen, damit ihr das Eigentumsvolk des HERRN (wieder) segnet?« 4Die
Gibeoniten antworteten ihm: »Es ist uns dem Saul und seinem Hause
gegenüber nicht um Silber und Gold zu tun, auch kommt es uns nicht zu,
jemand in Israel zu töten.« Da fragte er sie: »Was verlangt ihr, daß ich
für euch tun soll?« 5Da antworteten sie dem König: »Der Mann, der uns
hat vernichten wollen und der darauf ausgegangen ist, uns auszurotten,
damit in keinem Teil Israels unseres Bleibens mehr sein sollte:~-- 6von
dessen Nachkommen liefere man uns sieben Männer aus, daß wir sie vor dem
HERRN aufhängen\textless sup title=``oder: pfählen, vgl. 4.Mose
25,4''\textgreater✲ in Sauls-Gibea\textless sup title=``oder: in
Gibeon''\textgreater✲ auf dem Berge des HERRN.«

\hypertarget{b-davids-zusage-und-deren-ausfuxfchrung-an-sauls-geschlecht}{%
\paragraph{b) Davids Zusage und deren Ausführung an Sauls
Geschlecht}\label{b-davids-zusage-und-deren-ausfuxfchrung-an-sauls-geschlecht}}

Da sagte der König: »Ich will sie euch geben.« 7Der König verschonte
aber Mephiboseth, den Sohn Jonathans und Enkel Sauls, um des Schwures
beim HERRN willen, der zwischen David und Jonathan, dem Sohne Sauls,
bestand. 8Der König nahm vielmehr die beiden Söhne, welche Rizpa, die
Tochter Ajjas, dem Saul geboren hatte, Armoni und Mephiboseth, dazu die
fünf Söhne, welche Merab, die Tochter Sauls, dem Adriel, dem Sohne des
Meholathiters Barsillai, geboren hatte. 9Er übergab sie den Gibeoniten,
und diese hängten sie vor dem HERRN auf dem Berge auf. So kamen die
sieben zu gleicher Zeit ums Leben, und zwar wurden sie in den ersten
Tagen der Ernte, bei Beginn der Gerstenernte, getötet.

\hypertarget{c-rizpas-herrlicher-liebesbeweis-beisetzung-der-gebeine-sauls-und-seiner-nachkommen}{%
\paragraph{c) Rizpas herrlicher Liebesbeweis; Beisetzung der Gebeine
Sauls und seiner
Nachkommen}\label{c-rizpas-herrlicher-liebesbeweis-beisetzung-der-gebeine-sauls-und-seiner-nachkommen}}

10Da nahm Rizpa, die Tochter Ajjas, Sackleinwand\textless sup
title=``=~ein härenes Trauergewand''\textgreater✲ und breitete es (als
Lager) für sich auf dem Felsen aus, vom Beginn der Ernte an, bis der
Herbstregen auf die Toten niederfiel; und sie sorgte dafür, daß bei Tage
kein Raubvogel und während der Nacht kein wildes Tier an die Leichen
herankam.~-- 11Als man nun David berichtete, was Rizpa, das Nebenweib
Sauls, die Tochter Ajjas, getan hatte, 12ging er hin und ließ sich von
den Bürgern der Stadt Jabes in Gilead die Gebeine Sauls und die Gebeine
seines Sohnes Jonathan ausliefern, die sie einst vom Marktplatz in
Beth-San heimlich weggeholt hatten, wo die Philister sie damals
aufgehängt hatten, als die Philister Saul auf dem Gilboa geschlagen
hatten\textless sup title=``vgl. 1.Sam 31,10-13''\textgreater✲. 13Als
nun David die Gebeine Sauls und die seines Sohnes Jonathan von dort
hatte holen lassen, sammelte man auch die Gebeine der
Gehenkten\textless sup title=``oder: Gepfählten; vgl. V.6''\textgreater✲
14und begrub sie bei\textless sup title=``oder: mit''\textgreater✲ den
Gebeinen Sauls und seines Sohnes Jonathan im Gebiet des Stammes Benjamin
zu Zela\textless sup title=``oder: in einer Seitenkammer''\textgreater✲
im Begräbnis seines Vaters Kis. Als man so alles nach dem Befehl des
Königs ausgeführt hatte, ließ Gott sich von da an für das Land wieder
günstig stimmen.

\hypertarget{einige-heldentaten-der-krieger-davids-in-den-philisterkriegen}{%
\subsubsection{3. Einige Heldentaten der Krieger Davids in den
Philisterkriegen}\label{einige-heldentaten-der-krieger-davids-in-den-philisterkriegen}}

15Als einst wieder einmal ein Krieg zwischen den Philistern und
Israeliten ausgebrochen und David mit seinen Leuten hinabgezogen war
(und sie sich in Gob festgesetzt hatten), um mit den Philistern zu
kämpfen, und David ermüdet war, 16da war da ein Mann namens Jisbi-Benob,
einer von den Riesenkindern; der hatte einen Speer, dessen eherne Spitze
dreihundert Schekel wog, und hatte eine neue Rüstung an und gedachte
David zu erschlagen. 17Aber Abisai, der Sohn der Zeruja, kam ihm zu
Hilfe und schlug den Philister tot. Damals beschworen Davids Leute ihn
mit den Worten: »Du darfst nicht wieder mit uns in den Kampf ziehen,
damit du die Leuchte Israels nicht auslöschest!«

18Später kam es dann nochmals zum Kampf mit den Philistern bei Gob.
Damals erschlug der Husathiter Sibbechai den Saph, der auch zu den
Riesenkindern gehörte.

19Dann fand nochmals ein Kampf mit den Philistern bei Gob statt; und
Elhanan aus Bethlehem, der Sohn Jaare-Orgims\textless sup title=``vgl.
1.Chr 20,5''\textgreater✲, erschlug den Goliath aus Gath, dessen
Speerschaft wie ein Weberbaum war. 20Als es dann wiederum zum Kampf und
zwar bei Gath kam, war da ein Mann von riesiger Größe, der an jeder Hand
sechs Finger und an jedem Fuß sechs Zehen hatte, im ganzen
vierundzwanzig; auch dieser stammte aus dem Riesengeschlecht. 21Er hatte
die Israeliten verhöhnt; aber Jonathan, der Sohn Simeas, des Bruders
Davids, erschlug ihn. 22Diese vier stammten aus dem Riesengeschlecht in
Gath, und sie fielen durch die Hand Davids und seiner Krieger.

\hypertarget{davids-dank--und-siegeslied-nach-besiegung-seiner-feinde}{%
\subsubsection{4. Davids Dank- und Siegeslied nach Besiegung seiner
Feinde}\label{davids-dank--und-siegeslied-nach-besiegung-seiner-feinde}}

\hypertarget{section-21}{%
\section{22}\label{section-21}}

1Das folgende Lied richtete David an den HERRN zu der Zeit, als der HERR
ihn aus der Hand aller seiner Feinde und (besonders) aus der Hand Sauls
errettet hatte. Er betete (damals):

2Der HERR ist mein Fels, meine Burg und mein Erretter; 3Gott ist mein
Fels, zu dem ich mich flüchte, mein Schild und das Horn meines Heils,
mein fester Turm und meine Zuflucht, mein Erretter, der von Gewalttat
mich rettet. 4Den Preiswürdigen rufe ich an, den HERRN: so werd' ich von
meinen Feinden errettet.

5Denn die Wogen des Todes hatten mich umringt, die Ströme des Unheils
schreckten mich, 6die Netze des Totenreichs umfingen mich, die Schlingen
des Todes fielen über mich\textless sup title=``oder: starrten mir
entgegen''\textgreater✲. 7In meiner Angst rief ich zum HERRN und schrie
um Hilfe zu meinem Gott; da vernahm er in seinem Palast mein Rufen, und
mein Notschrei drang zu seinen Ohren.

8Da wankte und schwankte die Erde, des Himmels Grundfesten bebten und
wankten hin und her, denn er war zornentbrannt; 9Rauch stieg auf aus
seiner Nase, und fressendes Feuer drang aus seinem Munde, glühende
Kohlen sprühten von ihm aus. 10Er neigte den Himmel und fuhr herab,
Wolkennacht lag unter seinen Füßen; 11Er fuhr auf dem Cherub und flog
daher und schoß herab auf den Fittichen des Sturms; 12Finsternis machte
er rings um sich her zu seinem Gezelt, Regendunkel, dichtes Gewölk;
13aus dem Glanz vor ihm her brannten Feuerflammen. 14Dann donnerte der
HERR vom Himmel her, der Höchste ließ seine Stimme erschallen; 15er
schoß seine Pfeile ab und zerstreute sie\textless sup title=``d.h. die
Feinde''\textgreater✲, schleuderte Blitze und schreckte sie\textless sup
title=``d.h. die Feinde''\textgreater✲. 16Da wurden sichtbar die Tiefen
des Meeres und aufgedeckt die Grundfesten der Erde durch das Schelten
des HERRN, von dem Zornesschnauben seiner Nase. 17Er streckte die Hand
herab aus der Höhe, ergriff mich, zog mich heraus aus den großen Fluten,
18entriß mich meinem starken Feinde, meinen Widersachern, die zu stark
mir waren. 19Sie hatten mich überfallen an meinem Unglückstage, doch der
HERR ward mir zur Stütze; 20er führte mich heraus auf weiten Raum, riß
mich heraus, weil er Wohlgefallen an mir hatte.

21Der HERR hat mir gelohnt nach meiner Gerechtigkeit, nach der Reinheit
meiner Hände mir vergolten; 22denn ich habe innegehalten die Wege des
HERRN und bin von meinem Gott nicht treulos abgefallen; 23nein, alle
seine Rechte haben mir vor Augen gestanden, und von seinen Geboten bin
ich nicht abgewichen. 24So bin ich unsträflich vor ihm gewandelt und
hab' mich vor jeder Verschuldung gehütet; 25drum hat mir der HERR
vergolten nach meiner Gerechtigkeit, nach meiner Reinheit, die seinen
Augen sichtbar war. 26Gegen den Guten erweist du dich gütig, gegen den
Redlichen zeigst du dich redlich, 27gegen den Reinen erweist du dich
rein, doch gegen den Falschen zeigst du dich enttäuschend; 28denn du
schaffst demütigen Leuten Hilfe, aber stolze Augen erniedrigst du. 29Ja,
du bist meine Leuchte, o HERR; und mein Gott erhellt meine Finsternis.
30Denn mit dir überrenne ich Feindesscharen, mit meinem Gott überspringe
ich Mauern. 31Dieser Gott -- sein Walten ist vollkommen, die Worte des
HERRN sind lauter, ein Schild ist er allen, die zu ihm sich flüchten.
32Denn wer ist Gott außer dem HERRN und wer ein Fels als nur unser
Gott?, 33dieser Gott, der mit Kraft mich gegürtet und der meinen Weg
ohn' Anstoß gemacht; 34der mir Füße verliehen den Hirschen gleich und
mich sicher auf Bergeshöhen gestellt; 35der meine Hände streiten
gelehrt, daß meine Arme den ehernen Bogen spannten. 36Du reichtest mir
deinen schützenden Schild, und deine Gnade machte mich groß. 37Du
schafftest weiten Raum meinen Schritten unter mir, und meine Knöchel
wankten nicht. 38Ich verfolgte meine Feinde, vertilgte sie und kehrte
nicht um, bis ich sie vernichtet; 39ich vernichtete und zerschmetterte
sie, daß sie nicht wieder aufstehn konnten: sie sanken unter meine Füße
nieder. 40Und du gürtetest mich mit Kraft zum Streit, beugtest unter
mich, die sich gegen mich erhoben; 41du triebst meine Feinde vor mir in
die Flucht, und alle, die mich haßten, vernichtete ich. 42Sie blickten
nach Hilfe umher -- doch da war kein Helfer~-- zum HERRN -- doch er
hörte sie nicht; 43ich zermalmte sie wie Staub auf dem Boden, wie Kot
auf den Gassen zertrat, zerstampfte ich sie.

44Du hast mich aus meines Volkes Fehden errettet, mich zum Oberhaupt von
Völkern\textless sup title=``oder: der Heiden''\textgreater✲ eingesetzt:
Völker, die ich nicht kannte, dienen mir; 45die Söhne des Auslands
huldigen mir, aufs bloße Wort gehorchen sie mir; 46die Söhne des
Auslands sinken mutlos hin und kommen zitternd hervor aus ihren
Schlössern.

47Der HERR lebt: gepriesen sei mein Hort!, und erhaben ist der Gott, der
Fels meines Heils, 48der Gott, der mir Rache verliehen und die Völker
unter meine Herrschaft gezwungen, 49der mich von meinen Feinden frei
gemacht und über meine Widersacher mich erhöht, von dem Mann der
Gewalttat mich befreit hat! 50Drum will ich dich preisen\textless sup
title=``oder: dir danken''\textgreater✲, HERR, unter den Völkern und
deinem Namen lobsingen\textless sup title=``vgl. Röm
15,9''\textgreater✲, 51dir, der seinem Könige großes Heil verleiht und
Gnade an seinem Gesalbten übt, an David und seinem Hause ewiglich!

\hypertarget{davids-letzte-worte}{%
\subsubsection{5. Davids letzte Worte}\label{davids-letzte-worte}}

\hypertarget{section-22}{%
\section{23}\label{section-22}}

1Dies sind Davids letzte Worte:

Ausspruch Davids, des Sohnes Isais, und Ausspruch des Mannes, der hoch
erhoben ward, des Gesalbten des Gottes Jakobs und des Lieblings der
Lieder Israels: 2Der Geist des HERRN redet in mir\textless sup
title=``oder: durch mich''\textgreater✲, und sein Wort liegt auf meiner
Zunge. 3Es hat gesprochen der Gott Israels\textless sup title=``oder:
Jakobs''\textgreater✲, der Fels Israels zu mir gesagt: Wer gerecht
herrscht über die Menschen, wer da herrscht in der Furcht Gottes, 4der
ist wie das Licht, das am Morgen aufstrahlt, wie die Sonne am Morgen
ohne Gewölk: von ihrem Glanz nach dem Regen sproßt junges Grün aus der
Erde hervor.

5Ja, steht nicht so mein Haus zu Gott? Hat er doch einen ewigen Bund mit
mir geschlossen, der in allen Stücken gesichert und festgestellt ist.
Ja, all mein Glück und all mein Verlangen, sollte er das nicht sprossen✲
lassen?

6Die Bösen aber sind allesamt wie Dornen, die man wegwirft; denn mit der
Hand faßt man sie nicht an; 7nein, wer sich mit ihnen befaßt, bewehrt
sich mit Eisen und Speerschaft, und im Feuer verbrennt man sie gänzlich
an ihrer Stätte.

\hypertarget{verzeichnis-und-heldentaten-von-davids-kriegern}{%
\subsubsection{6. Verzeichnis und Heldentaten von Davids
Kriegern}\label{verzeichnis-und-heldentaten-von-davids-kriegern}}

8Dies sind die Namen der Helden✲ Davids: Joseb-Bassebeth, der
Tahchemoniter, das Haupt der Drei\textless sup title=``oder: der
Ritter''\textgreater✲; der schwang seinen Speer über achthundert
(Feinden), die er auf einmal erschlagen\textless sup title=``oder:
durchbohrt''\textgreater✲ hatte.~-- 9Nach ihm kam unter den drei
(Rittern) Eleasar, der Sohn Dodos, der Ahohiter. Er war bei David in
Pas-Dammim, als die Philister sich dort zur Schlacht versammelt hatten.
Als nun die Israeliten sich vor ihnen zurückzogen, 10war er es, der
standhielt und unter den Philistern ein Blutbad anrichtete, bis sein Arm
erlahmt war und seine Hand am Schwert kleben blieb. So verlieh der HERR
(den Israeliten) an jenem Tage einen herrlichen Sieg. Da kehrte das Heer
unter seiner Führung wieder um, doch nur, um (die Erschlagenen)
auszuplündern.~-- 11Nach ihm kam Samma, der Sohn des Harariters Age.
Einst hatten sich nämlich die Philister in Lehi gesammelt, und es war
dort ein Ackerstück mit Linsen; als nun das Heer vor den Philistern
floh, 12trat er mitten auf das Feld, behauptete es und schlug die
Philister; so verlieh (ihm) der HERR einen herrlichen Sieg.

\hypertarget{wagnis-dreier-helden}{%
\paragraph{Wagnis dreier Helden}\label{wagnis-dreier-helden}}

13Einst kamen drei von den dreißig Rittern während der Erntezeit zu
David in die Höhle von Adullam hinab, während die Schar der Philister
sich in der Ebene Rephaim gelagert hatte. 14David befand sich aber
damals in der Bergfeste, während eine Besatzung der Philister damals in
Bethlehem lag. 15Nun verspürte David ein Gelüsten und rief aus: »Wer
verschafft mir Wasser zu trinken aus dem Brunnen, der in Bethlehem am
Stadttor liegt?« 16Da schlugen sich die drei Helden durch das Lager der
Philister hindurch, schöpften Wasser aus dem Brunnen am Tor zu Bethlehem
und brachten es glücklich zu David hin. Aber dieser wollte es nicht
trinken, sondern goß es als Trankopfer für den HERRN aus 17mit den
Worten: »Der HERR behüte mich davor, daß ich so etwas tun sollte! Das
ist ja das Blut der Männer, die unter Lebensgefahr dorthin gezogen
sind!« Und er wollte es nicht trinken. Das hatten die drei Helden
ausgeführt.

\hypertarget{abisai-und-benaja}{%
\paragraph{Abisai und Benaja}\label{abisai-und-benaja}}

18Abisai aber, der Bruder Joabs, der Sohn der Zeruja, war das Haupt der
Dreißig; der schwang seinen Speer über dreihundert (Feinden), die er
erschlagen\textless sup title=``oder: durchbohrt''\textgreater✲ hatte,
und besaß hohes Ansehen unter den Dreißig. 19Unter den Dreißig genoß er
die höchste Ehre, so daß er auch ihr Oberster wurde; aber an die
(ersten) drei Helden reichte er nicht heran.~-- 20Benaja, der Sohn
Jojadas, (ein tapferer Mann), groß an Taten, stammte aus Kabzeel; er war
es, der die beiden Söhne Ariels aus Moab erschlug. Auch stieg er einmal
in eine Zisterne hinab und erschlug darin einen Löwen an einem Tage, an
dem Schnee gefallen war. 21Auch erschlug er einen Ägypter von riesiger
Größe, der einen Speer in der Hand hatte; er aber ging mit einem Stecken
auf ihn los, riß dem Ägypter den Speer aus der Hand und tötete ihn mit
seinem eigenen Speer. 22Das tat Benaja, der Sohn Jojadas; er besaß hohes
Ansehen unter den dreißig Rittern 23und war der geehrteste unter den
Dreißig, aber an die (ersten) drei Helden reichte er nicht heran. David
stellte ihn an die Spitze seiner Leibwache.

\hypertarget{eine-liste-anderer-helden-davids}{%
\paragraph{Eine Liste anderer Helden
Davids}\label{eine-liste-anderer-helden-davids}}

24Zu den dreißig (Rittern) gehörten: Asahel, der Bruder Joabs; Elhanan
aus Bethlehem, der Sohn Dodos; 25Samma aus Harod; Elika aus Harod;
26Helez aus Pelet; Ira, der Sohn des Ikkes, aus Thekoa; 27Abieser aus
Anathoth; Mebunnai\textless sup title=``oder: Sibbechai''\textgreater✲
aus Husa; 28Zalmon aus Ahoah; Maharai aus Netopha; 29Heleb, der Sohn
Baanas, aus Netopha; Itthai, der Sohn Ribais, aus Gibea im Stamme
Benjamin; 30Benaja aus Pirathon; Hiddai aus Nahale-Gaas;
31Abi-Albon\textless sup title=``oder: Abibaal''\textgreater✲ aus
Beth-Araba; Asmaweth aus Bahurim; 32Eljahba aus Saalbon; Jasen, der
Gunit; 33Jonathan, der Sohn Sammas, aus Harar; Ahiam, der Sohn Sarars,
aus Harar; 34Eliphelet, der Sohn Ahasbais, aus Beth-Maacha; Eliam, der
Sohn Ahithophels, aus Gilo; 35Hezrai aus Karmel; Paarai, der Arkiter;
36Jigal, der Sohn Nathans, aus Zoba; Bani aus Gad; 37Zelek, der
Ammoniter; Naharai aus Beeroth, der Waffenträger Joabs, des Sohnes der
Zeruja; 38Ira aus Jatthir; Gareb aus Jatthir; 39Uria, der Hethiter:
zusammen siebenunddreißig.

\hypertarget{davids-volkszuxe4hlung-und-ihre-bestrafung}{%
\subsubsection{7. Davids Volkszählung und ihre
Bestrafung}\label{davids-volkszuxe4hlung-und-ihre-bestrafung}}

\hypertarget{a-david-beschlieuxdft-die-volkszuxe4hlung-trotz-joabs-warnung}{%
\paragraph{a) David beschließt die Volkszählung trotz Joabs
Warnung}\label{a-david-beschlieuxdft-die-volkszuxe4hlung-trotz-joabs-warnung}}

\hypertarget{section-23}{%
\section{24}\label{section-23}}

1Der Zorn des HERRN aber entbrannte (einst) aufs neue gegen Israel, so
daß er David gegen das Volk reizte durch die Aufforderung: »Auf! Nimm
eine Zählung in Israel und Juda vor!« 2Da befahl der König seinem
Heerführer Joab (und den Heeresobersten bei ihm): »Durchwandre alle
Stammgebiete Israels von Dan bis Beerseba und nehmt eine Volkszählung
vor, damit ich die Zahl des Volkes erfahre!« 3Joab antwortete dem König:
»Der HERR, dein Gott, möge das Volk, so zahlreich es auch schon ist,
noch hundertmal zahlreicher werden lassen, und mein Herr, der König,
möge das selbst noch mit eigenen Augen schauen! Aber warum trägt mein
Herr, der König, Verlangen nach einer derartigen Vornahme?« 4Doch der
Befehl des Königs blieb trotz den Vorstellungen Joabs und der
Heeresobersten bestehen, und so machte sich denn Joab mit den
Heeresobersten auf Geheiß des Königs daran, die Volkszählung in Israel
vorzunehmen.

\hypertarget{b-ausfuxfchrung-der-volkszuxe4hlung-und-deren-ergebnis}{%
\paragraph{b) Ausführung der Volkszählung und deren
Ergebnis}\label{b-ausfuxfchrung-der-volkszuxe4hlung-und-deren-ergebnis}}

5Sie gingen also über den Jordan und lagerten sich bei Aroer rechts von
der Stadt, die inmitten des Flußtales (des Arnon) liegt, in der Richtung
nach Gad und nach Jaser hin. 6Dann begaben sie sich nach Gilead und bis
zum Lande der Hethiter gegen Kades hin; hierauf gelangten sie nach Dan,
bogen hierauf um nach Sidon zu, 7kamen alsdann zu der festen Stadt Tyrus
und zu allen Ortschaften der Hewiter und Kanaanäer und begaben sich
schließlich in das Südland von Juda, nach Beerseba. 8Nachdem sie so das
ganze Land durchzogen hatten, kehrten sie nach Verlauf von neun Monaten
und zwanzig Tagen nach Jerusalem zurück. 9Da teilte Joab dem König das
Ergebnis der Volkszählung mit, und zwar belief sich die Zahl der
kriegstüchtigen, schwertbewaffneten Männer in Israel auf 800000, in Juda
auf 500000~Mann.

\hypertarget{c-davids-reue-eingreifen-des-propheten-gad-david-wuxe4hlt-zur-suxfchnung-seiner-schuld-ein-volkssterben-davids-buuxdf--und-bittgebet}{%
\paragraph{c) Davids Reue; Eingreifen des Propheten Gad; David wählt zur
Sühnung seiner Schuld ein Volkssterben; Davids Buß- und
Bittgebet}\label{c-davids-reue-eingreifen-des-propheten-gad-david-wuxe4hlt-zur-suxfchnung-seiner-schuld-ein-volkssterben-davids-buuxdf--und-bittgebet}}

10Nachdem aber David die Volkszählung hatte vornehmen lassen, schlug ihm
das Gewissen; daher betete er zum HERRN: »Ich habe mich durch mein Tun
schwer versündigt; doch laß nun, o HERR, deinem Knecht seine
Verschuldung ungestraft hingehen, denn ich habe in großer Verblendung
gehandelt!« 11Als aber David am folgenden Morgen aufstand, erging das
Wort des HERRN an den Propheten Gad, den Seher Davids, folgendermaßen:
12»Gehe hin und sage zu David: ›So hat der HERR gesprochen: Dreierlei
lege ich dir vor: wähle dir eins davon, damit ich es an dir zur
Ausführung bringe!‹« 13Da begab sich Gad zu David, teilte es ihm mit und
sagte zu ihm: »Sollen dir zur Strafe drei Jahre Hungersnot über dein
Land kommen? Oder willst du drei Monate lang vor deinen Feinden fliehen
müssen und von ihnen verfolgt werden? Oder soll die Pest drei Tage lang
in deinem Lande sein? Nun gehe mit dir zu Rat und überlege, welche
Antwort ich dem bringen soll, der mich gesandt hat.« 14Da sagte David zu
Gad: »Mir ist sehr bange! Wir wollen aber lieber in die Hand des HERRN
fallen, denn sein Erbarmen ist groß; aber in die Hand von Menschen
möchte ich nicht fallen!«

15Da ließ der HERR eine Pest über Israel kommen vom Morgen an bis zum
Nachmittag, und es starben aus dem Volke von Dan bis Beerseba
siebzigtausend Menschen. 16Als aber der Engel seine Hand gegen Jerusalem
ausstreckte, um es zu vernichten, da gereute den HERRN das Unheil, und
er gebot dem Engel, der das Unglück unter dem Volke anzurichten hatte:
»Es ist genug so! Laß jetzt deine Hand ruhen!« Der Engel des HERRN
befand sich aber gerade bei der Tenne des Jebusiters Arawna. 17Als nun
David den Engel sah, der das Sterben unter dem Volke anrichtete, rief
er, zum HERRN betend, aus: »Ach, ich bin's ja, der gesündigt hat, und
ich habe mich vergangen! Diese Herde aber -- was hat sie verschuldet?
Laß doch deine Hand mich und meine Familie treffen!«

\hypertarget{d-errichtung-eines-altars-auf-der-tenne-arawnas-ende-der-pest}{%
\paragraph{d) Errichtung eines Altars auf der Tenne Arawnas; Ende der
Pest}\label{d-errichtung-eines-altars-auf-der-tenne-arawnas-ende-der-pest}}

18An jenem Tage kam dann Gad zu David und sagte zu ihm: »Gehe hinauf und
errichte dem HERRN einen Altar auf der Tenne des Jebusiters Arawna!«
19Da begab sich David nach der Aufforderung Gads, dem Befehl des HERRN
gehorsam, hinauf. 20Als nun Arawna von oben her ausschaute und den König
mit seinen Dienern auf sich zukommen sah (Arawna war nämlich gerade mit
dem Dreschen des Weizens beschäftigt), trat er hinaus und verneigte sich
vor David mit dem Angesicht bis zur Erde. 21Hierauf fragte Arawna:
»Warum kommt mein Herr, der König, zu seinem Knecht?« David antwortete:
»Um die Tenne von dir zu kaufen; ich will hier dem HERRN einen Altar
errichten, damit dem Sterben unter dem Volk Einhalt getan wird.« 22Da
sagte Arawna zu David: »Mein Herr, der König, nehme sie hin und opfere,
was ihm beliebt! Die Rinder hier stehen als Brandopfer und die
Dreschschlitten nebst den Geschirren der Rinder als Brennholz zu deiner
Verfügung: 23das alles, o König, macht Arawna dem Könige zum Geschenk.«
Dann fuhr er fort: »Der HERR, dein Gott, wolle dir gnädig sein!« 24Aber
der König erwiderte dem Arawna: »Nein! Käuflich will ich es von dir
erwerben für den vollen Preis; denn ich mag dem HERRN, meinem Gott,
keine Brandopfer darbringen, die mir geschenkt sind.« So kaufte denn
David die Tenne und die Rinder für den Preis von fünfzig Schekeln
Silber. 25David erbaute alsdann dem HERRN dort einen Altar und brachte
Brandopfer und Heilsopfer dar; hierauf wandte der HERR dem Lande seine
Gnade wieder zu, und das Sterben unter den Israeliten erreichte sein
Ende.
