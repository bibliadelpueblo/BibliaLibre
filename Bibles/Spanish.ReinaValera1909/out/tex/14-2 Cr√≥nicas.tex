\hypertarget{acceso-de-salomuxf3n-al-gobierno-su-ejuxe9rcito-y-su-riqueza}{%
\subsection{Acceso de Salomón al gobierno; su ejército y su
riqueza}\label{acceso-de-salomuxf3n-al-gobierno-su-ejuxe9rcito-y-su-riqueza}}

\hypertarget{section}{%
\section{1}\label{section}}

\bibverse{1} Y salomón hijo de David fué afirmado en su reino; y Jehová
su Dios fué con él, y le engrandeció sobremanera.

\bibverse{2} Y llamó Salomón á todo Israel, tribunos, centuriones, y
jueces, y á todos los príncipes de todo Israel, cabezas de familias.
\bibverse{3} Y fué Salomón, y con él toda esta junta, al alto que había
en Gabaón; porque allí estaba el tabernáculo del testimonio de Dios, que
Moisés siervo de Jehová había hecho en el desierto. \footnote{\textbf{1:3}
  1Cró 16,39; 1Cró 21,29} \bibverse{4} Mas David había traído el arca de
Dios de Chîriath-jearim al lugar que él le había preparado; porque él le
había tendido una tienda en Jerusalem. \footnote{\textbf{1:4} 1Cró 13,6;
  1Cró 15,3; 1Cró 15,28; 1Cró 16,1} \bibverse{5} Asimismo el altar de
bronce que había hecho Bezaleel hijo de Uri hijo de Hur, estaba allí
delante del tabernáculo de Jehová, al cual fué á consultar Salomón con
aquella junta. \footnote{\textbf{1:5} Éxod 38,1-8; 2Cró 1,3}
\bibverse{6} Subió pues Salomón allá delante de Jehová, al altar de
bronce que estaba en el tabernáculo del testimonio, y ofreció sobre él
mil holocaustos.

\hypertarget{la-apariciuxf3n-de-dios-o-sueuxf1o-despuuxe9s-del-sacrificio}{%
\subsection{La aparición de Dios (o sueño) después del
sacrificio}\label{la-apariciuxf3n-de-dios-o-sueuxf1o-despuuxe9s-del-sacrificio}}

\bibverse{7} Y aquella noche apareció Dios á Salomón, y díjole: Demanda
lo que quisieres que yo te dé.

\bibverse{8} Y Salomón dijo á Dios: Tú has hecho con David mi padre
grande misericordia, y á mí me has puesto por rey en lugar suyo.
\bibverse{9} Confírmese pues ahora, oh Jehová Dios, tu palabra dada á
David mi padre; porque tú me has puesto por rey sobre un pueblo en
muchedumbre como el polvo de la tierra. \bibverse{10} Dame ahora
sabiduría y ciencia, para salir y entrar delante de este pueblo: porque
¿quién podrá juzgar este tu pueblo tan grande?

\bibverse{11} Y dijo Dios á Salomón: Por cuanto esto fué en tu corazón,
que no pediste riquezas, hacienda, ó gloria, ni el alma de los que te
quieren mal, ni pediste muchos días, sino que has pedido para ti
sabiduría y ciencia para juzgar mi pueblo, sobre el cual te he puesto
por rey, \bibverse{12} Sabiduría y ciencia te es dada; y también te daré
riquezas, hacienda, y gloria, cual nunca hubo en los reyes que han sido
antes de ti, ni después de ti habrá tal.

\bibverse{13} Y volvió Salomón á Jerusalem del alto que estaba en
Gabaón, de ante el tabernáculo del testimonio; y reinó sobre Israel.

\hypertarget{la-riqueza-y-el-comercio-de-salomuxf3n-en-carros-y-caballos}{%
\subsection{La riqueza y el comercio de Salomón en carros y
caballos}\label{la-riqueza-y-el-comercio-de-salomuxf3n-en-carros-y-caballos}}

\bibverse{14} Y juntó Salomón carros y gente de á caballo; y tuvo mil y
cuatrocientos carros, y doce mil jinetes, los cuales puso en las
ciudades de los carros, y con el rey en Jerusalem. \footnote{\textbf{1:14}
  1Re 10,26-29} \bibverse{15} Y puso el rey plata y oro en Jerusalem
como piedras, y cedro como cabrahigos que nacen en los campos en
abundancia. \footnote{\textbf{1:15} 2Cró 9,27} \bibverse{16} Y sacaban
caballos y lienzos finos de Egipto para Salomón; pues por contrato
tomaban allí los mercaderes del rey caballos y lienzos. \bibverse{17} Y
subían, y sacaban de Egipto, un carro por seiscientas piezas de plata, y
un caballo por ciento y cincuenta: y así se sacaban por medio de ellos
para todos los reyes de los Hetheos, y para los reyes de Siria.

\hypertarget{el-tratado-de-salomuxf3n-con-hiram-de-tiro-preparativos-para-la-construcciuxf3n-del-templo}{%
\subsection{El tratado de Salomón con Hiram de Tiro; Preparativos para
la construcción del
templo}\label{el-tratado-de-salomuxf3n-con-hiram-de-tiro-preparativos-para-la-construcciuxf3n-del-templo}}

\hypertarget{section-1}{%
\section{2}\label{section-1}}

\bibverse{1} Determinó pues Salomón edificar casa al nombre de Jehová, y
otra casa para su reino. \footnote{\textbf{2:1} 1Cró 14,1} \bibverse{2}
Y contó Salomón setenta mil hombres que llevasen cargas, y ochenta mil
hombres que cortasen en el monte, y tres mil y seiscientos que los
gobernasen.

\hypertarget{mensaje-de-salomuxf3n-y-peticiuxf3n-a-hiram}{%
\subsection{Mensaje de Salomón y petición a
Hiram}\label{mensaje-de-salomuxf3n-y-peticiuxf3n-a-hiram}}

\bibverse{3} Y envió á decir Salomón á Hiram rey de Tiro: Haz conmigo
como hiciste con David mi padre, enviándole cedros para que edificara
para sí casa en que morase. \bibverse{4} He aquí yo tengo que edificar
casa al nombre de Jehová mi Dios, para consagrársela, para quemar
perfumes aromáticos delante de él, y para la colocación continua de los
panes de la proposición, y para holocaustos á mañana y tarde, y los
sábados, y nuevas lunas, y festividades de Jehová nuestro Dios; lo cual
ha de ser perpetuo en Israel.

\bibverse{5} Y la casa que tengo que edificar, ha de ser grande: porque
el Dios nuestro es grande sobre todos los dioses. \footnote{\textbf{2:5}
  2Cró 6,18; 1Re 8,27} \bibverse{6} Mas ¿quién será tan poderoso que le
edifique casa? Los cielos y los cielos de los cielos no le pueden
comprender; ¿quién pues soy yo, que le edifique casa, sino para quemar
perfumes delante de él?

\bibverse{7} Envíame pues ahora un hombre hábil, que sepa trabajar en
oro, y en plata, y en metal, y en hierro, en púrpura, y en grana, y en
cárdeno, y que sepa esculpir con los maestros que están conmigo en Judá
y en Jerusalem, los cuales previno mi padre.

\bibverse{8} Envíame también madera de cedro, de haya, de pino, del
Líbano: porque yo sé que tus siervos entienden de cortar madera en el
Líbano; y he aquí, mis siervos irán con los tuyos, \bibverse{9} Para que
me apresten mucha madera, porque la casa que tengo de edificar ha de ser
grande y portentosa. \bibverse{10} Y he aquí para los operarios tus
siervos, cortadores de la madera, he dado veinte mil coros de trigo en
grano, y veinte mil coros de cebada, y veinte mil batos de vino, y
veinte mil batos de aceite.

\hypertarget{respuesta-y-aceptaciuxf3n-de-hiram}{%
\subsection{Respuesta y aceptación de
Hiram}\label{respuesta-y-aceptaciuxf3n-de-hiram}}

\bibverse{11} Entonces Hiram rey de Tiro respondió por letras, las que
envió á Salomón: Porque Jehová amó á su pueblo, te ha puesto por rey
sobre ellos. \bibverse{12} Y además decía Hiram: Bendito sea Jehová el
Dios de Israel, que hizo los cielos y la tierra, y que dió al rey David
hijo sabio, entendido, cuerdo y prudente, que edifique casa á Jehová, y
casa para su reino.

\bibverse{13} Yo pues te he enviado un hombre hábil y entendido, que fué
de Hiram mi padre, \bibverse{14} Hijo de una mujer de las hijas de Dan,
mas su padre fué de Tiro; el cual sabe trabajar en oro, y plata, y
metal, y hierro, en piedra y en madera, en púrpura y en cárdeno, en lino
y en carmesí; asimismo para esculpir todas figuras, y sacar toda suerte
de diseño que se le propusiere, y estar con tus hombres peritos, y con
los de mi señor David tu padre.

\bibverse{15} Ahora pues, enviará mi señor á sus siervos el trigo y
cebada, y aceite y vino, que ha dicho; \bibverse{16} Y nosotros
cortaremos en el Líbano la madera que hubieres menester, y te la
traeremos en balsas por la mar hasta Joppe, y tú la harás llevar hasta
Jerusalem.

\hypertarget{salomuxf3n-eleva-a-los-no-israelitas-al-trabajo-esclavo}{%
\subsection{Salomón eleva a los no israelitas al trabajo
esclavo}\label{salomuxf3n-eleva-a-los-no-israelitas-al-trabajo-esclavo}}

\bibverse{17} Y contó Salomón todos los hombres extranjeros que estaban
en la tierra de Israel, después de haberlos ya contado David su padre, y
fueron hallados ciento cincuenta y tres mil seiscientos. \footnote{\textbf{2:17}
  Jos 9,27} \bibverse{18} Y señaló de ellos setenta mil para llevar
cargas, y ochenta mil que cortasen en el monte, y tres mil y seiscientos
por sobrestantes para hacer trabajar al pueblo.

\hypertarget{inicio-de-la-construcciuxf3n-del-templo-los-muebles-del-templo}{%
\subsection{Inicio de la construcción del templo; los muebles del
templo}\label{inicio-de-la-construcciuxf3n-del-templo-los-muebles-del-templo}}

\hypertarget{section-2}{%
\section{3}\label{section-2}}

\bibverse{1} Y comenzó Salomón á edificar la casa en Jerusalem, en el
monte Moria que había sido mostrado á David su padre, en el lugar que
David había preparado en la era de Ornán Jebuseo. \bibverse{2} Y comenzó
á edificar en el mes segundo, á dos del mes, en el cuarto año de su
reinado.

\hypertarget{dimensiones-y-decoraciones-de-la-casa-del-templo}{%
\subsection{Dimensiones y decoraciones de la casa del
templo}\label{dimensiones-y-decoraciones-de-la-casa-del-templo}}

\bibverse{3} Estas son las medidas de que Salomón fundó el edificio de
la casa de Dios. La primera medida fué, la longitud de sesenta codos; y
la anchura de veinte codos. \bibverse{4} El pórtico que estaba en la
delantera de la longitud, era de veinte codos al frente del ancho de la
casa, y su altura de ciento y veinte: y cubriólo por dentro de oro puro.
\bibverse{5} Y techó la casa mayor con madera de haya, la cual cubrió de
buen oro, é hizo resaltar sobre ella palmas y cadenas. \bibverse{6}
Cubrió también la casa de piedras preciosas por excelencia: y el oro era
oro de Parvaim. \bibverse{7} Así cubrió la casa, sus vigas, sus
umbrales, sus paredes, y sus puertas, con oro; y esculpió querubines por
las paredes.

\hypertarget{equipo-del-lugar-santuxedsimo}{%
\subsection{Equipo del lugar
santísimo}\label{equipo-del-lugar-santuxedsimo}}

\bibverse{8} Hizo asimismo la casa del lugar santísimo, cuya longitud
era de veinte codos según el ancho del frente de la casa, y su anchura
de veinte codos: y cubrióla de buen oro que ascendía á seiscientos
talentos. \bibverse{9} Y el peso de los clavos tuvo cincuenta siclos de
oro. Cubrió también de oro las salas.

\bibverse{10} Y dentro del lugar santísimo hizo dos querubines de forma
de niños, los cuales cubrieron de oro. \bibverse{11} El largo de las
alas de los querubines era de veinte codos: porque la una ala era de
cinco codos: la cual llegaba hasta la pared de la casa; y la otra ala de
cinco codos, la cual llegaba al ala del otro querubín. \bibverse{12} De
la misma manera la una ala del otro querubín era de cinco codos: la cual
llegaba hasta la pared de la casa; y la otra ala era de cinco codos, que
tocaba al ala del otro querubín. \bibverse{13} Así las alas de estos
querubines estaban extendidas por veinte codos: y ellos estaban en pie
con los rostros hacia la casa. \bibverse{14} Hizo también el velo de
cárdeno, púrpura, carmesí y lino, é hizo resaltar en él querubines.

\footnote{\textbf{3:14} Éxod 26,31}

\hypertarget{los-dos-pilares-de-bronce-frente-a-la-casa-del-templo}{%
\subsection{Los dos pilares de bronce frente a la casa del
templo}\label{los-dos-pilares-de-bronce-frente-a-la-casa-del-templo}}

\bibverse{15} Delante de la casa hizo dos columnas de treinta y cinco
codos de longitud, con sus capiteles encima, de cinco codos.
\bibverse{16} Hizo asimismo cadenas en el oratorio, y púsolas sobre los
capiteles de las columnas: é hizo cien granadas, las cuales puso en las
cadenas. \bibverse{17} Y asentó las columnas delante del templo, la una
á la mano derecha, y la otra á la izquierda; y á la de la mano derecha
llamó Jachîn, y á la de la izquierda, Boaz.

\hypertarget{fabricaciuxf3n-de-implementos-para-el-templo}{%
\subsection{Fabricación de implementos para el
templo}\label{fabricaciuxf3n-de-implementos-para-el-templo}}

\hypertarget{section-3}{%
\section{4}\label{section-3}}

\bibverse{1} Hizo además un altar de bronce de veinte codos de longitud,
y veinte codos de anchura, y diez codos de altura. \bibverse{2} También
hizo un mar de fundición, el cual tenía diez codos del un borde al otro,
enteramente redondo: su altura era de cinco codos, y una línea de
treinta codos lo ceñía alrededor. \bibverse{3} Y debajo de él había
figuras de bueyes que lo circundaban, diez en cada codo todo alrededor:
eran dos órdenes de bueyes fundidos juntamente con el mar. \bibverse{4}
Y estaba asentado sobre doce bueyes, tres de los cuales miraban al
septentrión, y tres al occidente, y tres al mediodía, y tres al oriente:
y el mar asentaba sobre ellos, y todas sus traseras estaban á la parte
de adentro. \bibverse{5} Y tenía de grueso un palmo, y el borde era de
la hechura del borde de un cáliz, ó flor de lis. Y hacía tres mil batos.
\bibverse{6} Hizo también diez fuentes, y puso cinco á la derecha y
cinco á la izquierda, para lavar y limpiar en ellas la obra del
holocausto; mas el mar era para lavarse los sacerdotes en él.

\bibverse{7} Hizo asimismo diez candeleros de oro según su forma, los
cuales puso en el templo, cinco á la derecha, y cinco á la izquierda.
\bibverse{8} Además hizo diez mesas y púsolas en el templo, cinco á la
derecha, y cinco á la izquierda: igualmente hizo cien tazones de oro.
\bibverse{9} A más de esto hizo el atrio de los sacerdotes, y el gran
atrio, y las portadas del atrio, y cubrió las puertas de ellas de
bronce. \bibverse{10} Y asentó el mar al lado derecho hacia el oriente,
enfrente del mediodía.

\bibverse{11} Hizo también Hiram calderos, y palas, y tazones; y acabó
Hiram la obra que hacía al rey Salomón para la casa de Dios;

\bibverse{12} Dos columnas, y los cordones, los capiteles sobre las
cabezas de las dos columnas, y dos redes para cubrir las dos bolas de
los capiteles que estaban encima de las columnas; \bibverse{13}
Cuatrocientas granadas en las dos redecillas, dos órdenes de granadas en
cada redecilla, para que cubriesen las dos bolas de los capiteles que
estaban encima de las columnas. \bibverse{14} Hizo también las basas,
sobre las cuales asentó las fuentes; \bibverse{15} Un mar, y doce bueyes
debajo de él; \bibverse{16} Y calderos, y palas, y garfios; y todos sus
enseres hizo Hiram su padre al rey Salomón para la casa de Jehová, de
metal purísimo. \bibverse{17} Y fundiólos el rey en los llanos del
Jordán, en tierra arcillosa, entre Suchôt y Seredat. \bibverse{18} Y
Salomón hizo todos estos vasos en grande abundancia, porque no pudo ser
hallado el peso del metal.

\bibverse{19} Así hizo Salomón todos los vasos para la casa de Dios, y
el altar de oro, y las mesas sobre las cuales se ponían los panes de la
proposición; \bibverse{20} Asimismo los candeleros y sus candilejas, de
oro puro, para que las encendiesen delante del oratorio conforme á la
costumbre. \bibverse{21} Y las flores, y las lamparillas, y las
despabiladeras se hicieron de oro, de oro perfecto; \bibverse{22}
También los platillos, y las jofainas, y las cucharas, y los
incensarios, de oro puro. Cuanto á la entrada de la casa, sus puertas
interiores para el lugar santísimo, y las puertas de la casa del templo,
de oro.

\hypertarget{los-objetos-de-valor-almacenados-en-las-cuxe1maras-del-tesoro.}{%
\subsection{Los objetos de valor almacenados en las cámaras del
tesoro.}\label{los-objetos-de-valor-almacenados-en-las-cuxe1maras-del-tesoro.}}

\hypertarget{section-4}{%
\section{5}\label{section-4}}

\bibverse{1} Y acabada que fué toda la obra que hizo Salomón para la
casa de Jehová, metió Salomón en ella las cosas que David su padre había
dedicado; y puso la plata, y el oro, y todos los vasos, en los tesoros
de la casa de Dios.

\footnote{\textbf{5:1} 1Cró 28,14-18}

\hypertarget{la-transferencia-del-arca-al-lugar-santuxedsimo}{%
\subsection{La transferencia del arca al lugar
santísimo}\label{la-transferencia-del-arca-al-lugar-santuxedsimo}}

\bibverse{2} Entonces Salomón juntó en Jerusalem los ancianos de Israel,
y todos los príncipes de las tribus, los cabezas de las familias de los
hijos de Israel, para que trajesen el arca del pacto de Jehová de la
ciudad de David, que es Sión. \bibverse{3} Y juntáronse al rey todos los
varones de Israel, á la solemnidad del mes séptimo. \bibverse{4} Y
vinieron todos los ancianos de Israel, y tomaron los Levitas el arca:
\bibverse{5} Y llevaron el arca, y el tabernáculo del testimonio, y
todos los vasos del santuario que estaban en el tabernáculo: los
sacerdotes y los Levitas los llevaron. \bibverse{6} Y el rey Salomón, y
toda la congregación de Israel que se había á él reunido delante del
arca, sacrificaron ovejas y bueyes, que por la multitud no se pudieron
contar ni numerar. \bibverse{7} Y los sacerdotes metieron el arca del
pacto de Jehová en su lugar, en el oratorio de la casa, en el lugar
santísimo, bajo las alas de los querubines: \bibverse{8} Pues los
querubines extendían las alas sobre el asiento del arca, y cubrían los
querubines por encima así el arca como sus barras. \bibverse{9} E
hicieron salir fuera las barras, de modo que se viesen las cabezas de
las barras del arca delante del oratorio, mas no se veían desde fuera: y
allí estuvieron hasta hoy. \bibverse{10} En el arca no había sino las
dos tablas que Moisés había puesto en Horeb, con las cuales Jehová había
hecho alianza con los hijos de Israel, después que salieron de Egipto.

\footnote{\textbf{5:10} Heb 9,4}

\hypertarget{la-apariciuxf3n-de-la-gloria-de-dios}{%
\subsection{La aparición de la gloria de
Dios}\label{la-apariciuxf3n-de-la-gloria-de-dios}}

\bibverse{11} Y como los sacerdotes salieron del santuario, (porque
todos los sacerdotes que se hallaron habían sido santificados, y no
guardaban sus veces; \bibverse{12} Y los Levitas cantores, todos los de
Asaph, los de Hemán, y los de Jeduthún, juntamente con sus hijos y sus
hermanos, vestidos de lino fino, estaban con címbalos y salterios y
arpas al oriente del altar; y con ellos ciento veinte sacerdotes que
tocaban trompetas:) \bibverse{13} Sonaban pues las trompetas, y cantaban
con la voz todos á una, para alabar y confesar á Jehová: y cuando
alzaban la voz con trompetas y címbalos é instrumentos de música, cuando
alababan á Jehová, diciendo: Porque es bueno, porque su misericordia es
para siempre: la casa se llenó entonces de una nube, la casa de Jehová.
\footnote{\textbf{5:13} 1Cró 16,34} \bibverse{14} Y no podían los
sacerdotes estar para ministrar, por causa de la nube; porque la gloria
de Jehová había henchido la casa de Dios. \footnote{\textbf{5:14} 2Cró
  7,1; 2Cró 7,3}

\hypertarget{el-discurso-de-ordenaciuxf3n-y-consagraciuxf3n-del-rey-al-pueblo}{%
\subsection{El discurso de ordenación y consagración del rey al
pueblo}\label{el-discurso-de-ordenaciuxf3n-y-consagraciuxf3n-del-rey-al-pueblo}}

\hypertarget{section-5}{%
\section{6}\label{section-5}}

\bibverse{1} Entonces dijo Salomón: Jehová ha dicho que él habitaría en
la oscuridad. \bibverse{2} Yo pues he edificado una casa de morada para
ti, y una habitación en que mores para siempre.

\bibverse{3} Y volviendo el rey su rostro, bendijo á toda la
congregación de Israel: y toda la congregación de Israel estaba en pie.

\bibverse{4} Y él dijo: Bendito sea Jehová Dios de Israel, el cual con
su mano ha cumplido lo que habló por su boca á David mi padre, diciendo:
\bibverse{5} Desde el día que saqué mi pueblo de la tierra de Egipto,
ninguna ciudad he elegido de todas las tribus de Israel para edificar
casa donde estuviese mi nombre, ni he escogido varón que fuese príncipe
sobre mi pueblo Israel. \bibverse{6} Mas á Jerusalem he elegido para que
en ella esté mi nombre, y á David he elegido para que fuese sobre mi
pueblo Israel. \bibverse{7} Y David mi padre tuvo en el corazón edificar
casa al nombre de Jehová Dios de Israel. \footnote{\textbf{6:7} 2Sam
  7,2-13} \bibverse{8} Mas Jehová dijo á David mi padre: Respecto á
haber tenido en tu corazón edificar casa á mi nombre, bien has hecho en
haber tenido esto en tu corazón. \bibverse{9} Empero tú no edificarás la
casa, sino tu hijo que saldrá de tus lomos, él edificará casa á mi
nombre.

\bibverse{10} Y Jehová ha cumplido su palabra que había dicho; pues
levantéme yo en lugar de David mi padre, y heme sentado en el trono de
Israel, como Jehová había dicho, y he edificado casa al nombre de Jehová
Dios de Israel. \bibverse{11} Y en ella he puesto el arca, en la cual
está el pacto de Jehová que concertó con los hijos de Israel.

\hypertarget{oraciuxf3n-de-consagraciuxf3n-de-salomuxf3n}{%
\subsection{Oración de consagración de
Salomón}\label{oraciuxf3n-de-consagraciuxf3n-de-salomuxf3n}}

\bibverse{12} Púsose luego Salomón delante del altar de Jehová, en
presencia de toda la congregación de Israel, y extendió sus manos.
\bibverse{13} Porque Salomón había hecho un púlpito de metal, de cinco
codos de largo, y de cinco codos de ancho, y de altura de tres codos, y
lo había puesto en medio del atrio: y púsose sobre él, é hincóse de
rodillas delante de toda la congregación de Israel, y extendiendo sus
manos al cielo, dijo: \bibverse{14} Jehová Dios de Israel, no hay Dios
semejante á ti en el cielo ni en la tierra, que guardas el pacto y la
misericordia á tus siervos que caminan delante de ti de todo su corazón;
\bibverse{15} Que has guardado á tu siervo David mi padre lo que le
dijiste: tú lo dijiste de tu boca, mas con tu mano lo has cumplido, como
parece este día.

\bibverse{16} Ahora pues, Jehová Dios de Israel, guarda á tu siervo
David mi padre lo que le has prometido, diciendo: No faltará de ti varón
delante de mí, que se siente en el trono de Israel, á condición que tus
hijos guarden su camino, andando en mi ley, como tú delante de mí has
andado. \bibverse{17} Ahora pues, oh Jehová Dios de Israel, verifíquese
tu palabra que dijiste á tu siervo David.

\bibverse{18} Mas ¿es verdad que Dios ha de habitar con el hombre en la
tierra? He aquí, los cielos y los cielos de los cielos no pueden
contenerte: ¿cuánto menos esta casa que he edificado? \footnote{\textbf{6:18}
  2Cró 2,5} \bibverse{19} Mas tú mirarás á la oración de tu siervo, y á
su ruego, oh Jehová Dios mío, para oir el clamor y la oración con que tu
siervo ora delante de ti. \bibverse{20} Que tus ojos estén abiertos
sobre esta casa de día y de noche, sobre el lugar del cual dijiste, Mi
nombre estará allí; que oigas la oración con que tu siervo ora en este
lugar. \bibverse{21} Asimismo que oigas el ruego de tu siervo, y de tu
pueblo Israel, cuando en este lugar hicieren oración, que tú oirás desde
los cielos, desde el lugar de tu morada: que oigas y perdones.

\bibverse{22} Si alguno pecare contra su prójimo, y él le pidiere
juramento haciéndole jurar, y el juramento viniere delante de tu altar
en esta casa, \footnote{\textbf{6:22} Éxod 22,10} \bibverse{23} Tú oirás
desde los cielos, y obrarás, y juzgarás á tus siervos, dando la paga al
impío, tornándole su proceder sobre su cabeza, y justificando al justo
en darle conforme á su justicia.

\bibverse{24} Si tu pueblo Israel cayere delante de los enemigos, por
haber prevaricado contra ti, y se convirtieren, y confesaren tu nombre,
y rogaren delante de ti en esta casa, \bibverse{25} Tú oirás desde los
cielos, y perdonarás el pecado de tu pueblo Israel, y los volverás á la
tierra que diste á ellos y á sus padres.

\bibverse{26} Si los cielos se cerraren, que no haya lluvias por haber
pecado contra ti, si oraren á ti en este lugar, y confesaren tu nombre,
y se convirtieren de sus pecados, cuando los afligieres, \footnote{\textbf{6:26}
  Deut 28,23-24} \bibverse{27} Tú los oirás en los cielos, y perdonarás
el pecado de tus siervos y de tu pueblo Israel, y les enseñarás el buen
camino para que anden en él, y darás lluvia sobre tu tierra, la cual
diste por heredad á tu pueblo.

\bibverse{28} Y si hubiere hambre en la tierra, ó si hubiere
pestilencia, si hubiere tizoncillo ó añublo, langosta ó pulgón; ó si los
cercaren sus enemigos en la tierra de su domicilio; cualquiera plaga ó
enfermedad que sea; \bibverse{29} Toda oración y todo ruego que hiciere
cualquier hombre, ó todo tu pueblo Israel, cualquiera que conociere su
llaga y su dolor en su corazón, si extendiere sus manos á esta casa,
\bibverse{30} Tú oirás desde los cielos, desde el lugar de tu
habitación, y perdonarás, y darás á cada uno conforme á sus caminos,
habiendo conocido su corazón; (porque solo tú conoces el corazón de los
hijos de los hombres;) \bibverse{31} Para que te teman y anden en tus
caminos, todos los días que vivieren sobre la haz de la tierra que tú
diste á nuestros padres.

\bibverse{32} Y también al extranjero que no fuere de tu pueblo Israel,
que hubiere venido de lejanas tierras á causa de tu grande nombre, y de
tu mano fuerte, y de tu brazo extendido, si vinieren, y oraren en esta
casa, \bibverse{33} Tú oirás desde los cielos, desde el lugar de tu
morada, y harás conforme á todas las cosas por las cuales hubiere
clamado á ti el extranjero; para que todos los pueblos de la tierra
conozcan tu nombre, y te teman como tu pueblo Israel, y sepan que tu
nombre es invocado sobre esta casa que he edificado yo.

\bibverse{34} Si tu pueblo saliere á la guerra contra sus enemigos por
el camino que tú los enviares, y oraren á ti hacia esta ciudad que tú
elegiste, hacia la casa que he edificado á tu nombre, \footnote{\textbf{6:34}
  Dan 6,11} \bibverse{35} Tú oirás desde los cielos su oración y su
ruego, y ampararás su derecho.

\bibverse{36} Si pecaren contra ti, (pues no hay hombre que no peque,) y
te airares contra ellos, y los entregares delante de sus enemigos, para
que los que los tomaren los lleven cautivos á tierra de enemigos, lejos
ó cerca, \bibverse{37} Y ellos volvieren en sí en la tierra donde fueren
llevados cautivos; si se convirtieren, y oraren á ti en la tierra de su
cautividad, y dijeren: Pecamos, hemos hecho inicuamente, impíamente
hemos obrado; \bibverse{38} Si se convirtieren á ti de todo su corazón y
de toda su alma en la tierra de su cautividad, donde los hubieren
llevado cautivos, y oraren hacia su tierra que tú diste á sus padres,
hacia la ciudad que tu elegiste, y hacia la casa que he edificado á tu
nombre; \bibverse{39} Tú oirás desde los cielos, desde el lugar de tu
morada, su oración y su ruego, y ampararás su causa, y perdonarás á tu
pueblo que pecó contra ti.

\bibverse{40} Ahora pues, oh Dios mío, ruégote estén abiertos tus ojos,
y atentos tus oídos á la oración en este lugar.

\bibverse{41} Oh Jehová Dios, levántate ahora para habitar en tu reposo,
tú y el arca de tu fortaleza; sean, oh Jehová Dios, vestidos de salud
tus sacerdotes, y gocen de bien tus santos. \footnote{\textbf{6:41} Sal
  132,8-9}

\bibverse{42} Jehová Dios, no hagas volver el rostro de tu ungido:
acuérdate de las misericordias de David tu siervo. \footnote{\textbf{6:42}
  2Sam 7,13}

\hypertarget{apariciuxf3n-de-la-gloria-de-dios-salomuxf3n-y-el-pueblo-fiesta-solemne-de-sacrificios-y-asamblea-de-celebraciuxf3n}{%
\subsection{Aparición de la gloria de Dios; Salomón y el pueblo fiesta
solemne de sacrificios y asamblea de
celebración}\label{apariciuxf3n-de-la-gloria-de-dios-salomuxf3n-y-el-pueblo-fiesta-solemne-de-sacrificios-y-asamblea-de-celebraciuxf3n}}

\hypertarget{section-6}{%
\section{7}\label{section-6}}

\bibverse{1} Y como Salomón acabó de orar, el fuego descendió de los
cielos, y consumió el holocausto y las víctimas; y la gloria de Jehová
hinchió la casa. \footnote{\textbf{7:1} Lev 9,24; 1Re 18,38; Éxod 40,34}
\bibverse{2} Y no podían entrar los sacerdotes en la casa de Jehová,
porque la gloria de Jehová había henchido la casa de Jehová.
\bibverse{3} Y como vieron todos los hijos de Israel descender el fuego
y la gloria de Jehová sobre la casa, cayeron en tierra sobre sus rostros
en el pavimento, y adoraron, confesando á Jehová y diciendo: Que es
bueno, que su misericordia es para siempre.

\bibverse{4} Entonces el rey y todo el pueblo sacrificaron víctimas
delante de Jehová. \bibverse{5} Y ofreció el rey Salomón en sacrificio
veinte y dos mil bueyes, y ciento veinte mil ovejas; y así dedicaron la
casa de Dios el rey y todo el pueblo. \bibverse{6} Y los sacerdotes
asistían en su ministerio; y los Levitas con los instrumentos de música
de Jehová, los cuales había hecho el rey David para confesar á Jehová,
que su misericordia es para siempre; cuando David alababa por mano de
ellos. Asimismo los sacerdotes tañían trompetas delante de ellos, y todo
Israel estaba en pie.

\bibverse{7} También santificó Salomón el medio del atrio que estaba
delante de la casa de Jehová, por cuanto había ofrecido allí los
holocaustos, y los sebos de los pacíficos; porque en el altar de bronce
que Salomón había hecho, no podían caber los holocaustos, y el presente,
y los sebos. \footnote{\textbf{7:7} 1Re 8,62-66}

\bibverse{8} Entonces hizo Salomón fiesta siete días, y con él todo
Israel, una grande congregación, desde la entrada de Hamath hasta el
arroyo de Egipto.

\bibverse{9} Al octavo día hicieron convocación, porque habían hecho la
dedicación del altar en siete días, y habían celebrado la solemnidad por
siete días. \bibverse{10} Y á los veintitrés del mes séptimo envió al
pueblo á sus estancias, alegres y gozosos de corazón por los beneficios
que Jehová había hecho á David, y á Salomón, y á su pueblo Israel.

\hypertarget{la-repetida-apariciuxf3n-de-dios-y-su-respuesta-promesa-y-amenaza-a-la-oraciuxf3n-de-salomuxf3n}{%
\subsection{La repetida aparición de Dios y su respuesta (promesa y
amenaza) a la oración de
Salomón}\label{la-repetida-apariciuxf3n-de-dios-y-su-respuesta-promesa-y-amenaza-a-la-oraciuxf3n-de-salomuxf3n}}

\bibverse{11} Acabó pues Salomón la casa de Jehová, y la casa del rey: y
todo lo que Salomón tuvo en voluntad de hacer en la casa de Jehová y en
su casa, fué prosperado.

\bibverse{12} Y apareció Jehová á Salomón de noche, y díjole: Yo he oído
tu oración, y he elegido para mí este lugar por casa de sacrificio.
\footnote{\textbf{7:12} Deut 12,5}

\bibverse{13} Si yo cerrare los cielos, que no haya lluvia, y si mandare
á la langosta que consuma la tierra, ó si enviare pestilencia á mi
pueblo; \bibverse{14} Si se humillare mi pueblo, sobre los cuales ni
nombre es invocado, y oraren, y buscaren mi rostro, y se convirtieren de
sus malos caminos; entonces yo oiré desde los cielos, y perdonaré sus
pecados, y sanaré su tierra. \bibverse{15} Ahora estarán abiertos mis
ojos, y atentos mis oídos, á la oración en este lugar: \bibverse{16}
Pues que ahora he elegido y santificado esta casa, para que esté en ella
mi nombre para siempre; y mis ojos y mi corazón estarán ahí para
siempre.

\bibverse{17} Y tú, si anduvieres delante de mí, como anduvo David tu
padre, é hicieres todas las cosas que yo te he mandado, y guardares mis
estatutos y mis derechos, \bibverse{18} Yo confirmaré el trono de tu
reino, como concerté con David tu padre, diciendo: No faltará varón de
ti que domine en Israel. \footnote{\textbf{7:18} 2Sam 7,12; 2Sam 7,16}

\bibverse{19} Mas si vosotros os volviereis, y dejareis mis estatutos y
mis preceptos que os he propuesto, y fuereis y sirviereis á dioses
ajenos, y los adorareis, \bibverse{20} Yo los arrancaré de mi tierra que
les he dado; y esta casa que he santificado á mi nombre, yo la echaré de
delante de mí, y pondréla por proverbio y fábula en todos los pueblos.
\bibverse{21} Y esta casa que habrá sido ilustre, será espanto á todo el
que pasare, y dirá: ¿Por qué ha hecho así Jehová á esta tierra y á esta
casa? \footnote{\textbf{7:21} Deut 29,23-26; Jer 22,8-9} \bibverse{22} Y
se responderá: Por cuanto dejaron á Jehová Dios de sus padres, el cual
los sacó de la tierra de Egipto, y han abrazado dioses ajenos, y los
adoraron y sirvieron: por eso él ha traído todo este mal sobre ellos.

\hypertarget{informaciuxf3n-sobre-las-ciudades-y-fortalezas-de-salomuxf3n}{%
\subsection{Información sobre las ciudades y fortalezas de
Salomón}\label{informaciuxf3n-sobre-las-ciudades-y-fortalezas-de-salomuxf3n}}

\hypertarget{section-7}{%
\section{8}\label{section-7}}

\bibverse{1} Y aconteció que al cabo de veinte años que Salomón había
edificado la casa de Jehová y su casa, \bibverse{2} Reedificó Salomón
las ciudades que Hiram le había dado, y estableció en ellas á los hijos
de Israel.

\bibverse{3} Después vino Salomón á Amath de Soba, y la tomó.
\bibverse{4} Y edificó á Tadmor en el desierto, y todas las ciudades de
municiones que edificó en Hamath. \bibverse{5} Asimismo reedificó á
Beth-oron la de arriba, y á Beth-oron la de abajo, ciudades
fortificadas, de muros, puertas, y barras; \bibverse{6} Y á Baalath, y á
todas las villas de munición que Salomón tenía; también todas las
ciudades de los carros y las de la gente de á caballo; y todo lo que
Salomón quiso edificar en Jerusalem, y en el Líbano, y en toda la tierra
de su señorío.

\hypertarget{los-obreros-de-salomuxf3n-y-sus-capataces-su-esposa-la-princesa-egipcia-se-traslada-al-palacio-construido-para-ella}{%
\subsection{Los obreros de Salomón y sus capataces; Su esposa, la
princesa egipcia, se traslada al palacio construido para
ella}\label{los-obreros-de-salomuxf3n-y-sus-capataces-su-esposa-la-princesa-egipcia-se-traslada-al-palacio-construido-para-ella}}

\bibverse{7} Y á todo el pueblo que había quedado de los Hetheos,
Amorrheos, Pherezeos, Heveos, y Jebuseos, que no eran de Israel,
\bibverse{8} Los hijos de los que habían quedado en la tierra después de
ellos, á los cuales los hijos de Israel no destruyeron del todo, hizo
Salomón tributarios hasta hoy. \bibverse{9} Y de los hijos de Israel no
puso Salomón siervos en su obra; porque eran hombres de guerra, y sus
príncipes y sus capitanes, y comandantes de sus carros, y su gente de á
caballo. \bibverse{10} Y tenía Salomón doscientos y cincuenta
principales de los gobernadores, los cuales mandaban en aquella gente.

\bibverse{11} Y pasó Salomón á la hija de Faraón, de la ciudad de David
á la casa que él le había edificado; porque dijo: Mi mujer no morará en
la casa de David rey de Israel, porque aquellas habitaciones donde ha
entrado el arca de Jehová, son sagradas.

\hypertarget{orden-de-sacrificio-y-servicio-en-el-templo-de-salomuxf3n}{%
\subsection{Orden de sacrificio y servicio en el templo de
Salomón}\label{orden-de-sacrificio-y-servicio-en-el-templo-de-salomuxf3n}}

\bibverse{12} Entonces ofreció Salomón holocaustos á Jehová sobre el
altar de Jehová, que había él edificado delante del pórtico, \footnote{\textbf{8:12}
  2Cró 1,3-6} \bibverse{13} Para que ofreciesen cada cosa en su día,
conforme al mandamiento de Moisés, en los sábados, en las nuevas lunas,
y en las solemnidades, tres veces en el año, á saber, en la fiesta de
los panes ázimos, en la fiesta de las semanas, y en la fiesta de las
cabañas. \footnote{\textbf{8:13} Núm 28,2; Núm 28,9; Núm 28,11; Núm
  28,17; Núm 28,26; Núm 29,12}

\bibverse{14} Y constituyó los repartimientos de los sacerdotes en sus
oficios, conforme á la ordenación de David su padre; y los Levitas por
sus órdenes, para que alabasen y ministrasen delante de los sacerdotes,
cada cosa en su día; asimismo los porteros por su orden á cada puerta:
porque así lo había mandado David, varón de Dios. \footnote{\textbf{8:14}
  1Cró 23,1-26} \bibverse{15} Y no salieron del mandamiento del rey,
cuanto á los sacerdotes y Levitas, y los tesoros, y todo negocio:

\bibverse{16} Porque toda la obra de Salomón estaba preparada desde el
día en que la casa de Jehová fué fundada hasta que se acabó, hasta que
la casa de Jehová fué acabada del todo.

\hypertarget{paseos-de-ofir-de-salomuxf3n}{%
\subsection{Paseos de Ofir de
Salomón}\label{paseos-de-ofir-de-salomuxf3n}}

\bibverse{17} Entonces Salomón fué á Esion-geber, y á Eloth, á la costa
de la mar en la tierra de Edom. \bibverse{18} Porque Hiram le había
enviado navíos por mano de sus siervos, y marineros diestros en la mar,
los cuales fueron con los siervos de Salomón á Ophir, y tomaron de allá
cuatrocientos y cincuenta talentos de oro, y los trajeron al rey
Salomón.

\hypertarget{visita-de-la-reina-de-saba}{%
\subsection{Visita de la Reina de
Saba}\label{visita-de-la-reina-de-saba}}

\hypertarget{section-8}{%
\section{9}\label{section-8}}

\bibverse{1} Y oyendo la reina de Seba la fama de Salomón, vino á
Jerusalem con un muy grande séquito, con camellos cargados de aroma, y
oro en abundancia, y piedras preciosas, para tentar á Salomón con
preguntas difíciles. Y luego que vino á Salomón, habló con él todo lo
que en su corazón tenía. \bibverse{2} Pero Salomón le declaró todas sus
palabras: ninguna cosa quedó que Salomón no le declarase. \bibverse{3} Y
viendo la reina de Seba la sabiduría de Salomón, y la casa que había
edificado, \bibverse{4} Y las viandas de su mesa, y el asiento de sus
siervos, y el estado de sus criados, y los vestidos de ellos, sus
maestresalas y sus vestidos, y su subida por donde subía á la casa de
Jehová, no quedó más espíritu en ella.

\bibverse{5} Y dijo al rey: Verdad es lo que había oído en mi tierra de
tus cosas y de tu sabiduría; \bibverse{6} Mas yo no creía las palabras
de ellos, hasta que he venido, y mis ojos han visto: y he aquí que ni
aun la mitad de la grandeza de tu sabiduría me había sido dicha; porque
tú sobrepujas la fama que yo había oído. \bibverse{7} Bienaventurados
tus hombres, y dichosos estos tus siervos, que están siempre delante de
ti, y oyen tu sabiduría. \bibverse{8} Jehová tu Dios sea bendito, el
cual se ha agradado en ti para ponerte sobre su trono por rey de Jehová
tu Dios: por cuanto tu Dios amó á Israel para afirmarlo perpetuamente,
por eso te ha puesto por rey sobre ellos, para que hagas juicio y
justicia.

\bibverse{9} Y dió al rey ciento y veinte talentos de oro, y gran copia
de aromas, y piedras preciosas: nunca hubo tales aromas como los que dió
la reina de Seba al rey Salomón.

\bibverse{10} También los siervos de Hiram y los siervos de Salomón, que
habían traído el oro de Ophir, trajeron madera de Algummim, y piedras
preciosas. \bibverse{11} E hizo el rey de la madera de Algummim gradas
en la casa de Jehová, y en las casas reales, y arpas y salterios para
los cantores: nunca en tierra de Judá se había visto madera semejante.
\bibverse{12} Y el rey Salomón dió á la reina de Seba todo lo que ella
quiso y le pidió, más de lo que había traído al rey. Después se volvió y
fuése á su tierra con sus siervos.

\hypertarget{riqueza-obras-de-arte-y-esplendor-de-salomuxf3n-y-artuxedculos-de-comercio-exterior}{%
\subsection{Riqueza, obras de arte y esplendor de Salomón y artículos de
comercio
exterior}\label{riqueza-obras-de-arte-y-esplendor-de-salomuxf3n-y-artuxedculos-de-comercio-exterior}}

\bibverse{13} Y el peso de oro que venía á Salomón cada un año, era
seiscientos sesenta y seis talentos de oro, \bibverse{14} Sin lo que
traían los mercaderes y negociantes; y también todos los reyes de Arabia
y los príncipes de la tierra traían oro y plata á Salomón. \bibverse{15}
Hizo también el rey Salomón doscientos paveses de oro de martillo, cada
uno de los cuales tenía seiscientos siclos de oro labrado: \bibverse{16}
Asimismo trescientos escudos de oro batido, teniendo cada escudo
trescientos siclos de oro: y púsolos el rey en la casa del bosque del
Líbano. \bibverse{17} Hizo además el rey un gran trono de marfil, y
cubriólo de oro puro. \bibverse{18} Y había seis gradas al trono, con un
estrado de oro al mismo, y brazos de la una parte y de la otra al lugar
del asiento, y dos leones que estaban junto á los brazos. \bibverse{19}
Había también allí doce leones sobre las seis gradas de la una parte y
de la otra. Jamás fué hecho otro semejante en reino alguno.
\bibverse{20} Toda la vajilla del rey Salomón era de oro, y toda la
vajilla de la casa del bosque del Líbano, de oro puro. En los días de
Salomón la plata no era de estima. \bibverse{21} Porque la flota del rey
iba á Tharsis con los siervos de Hiram, y cada tres años solían venir
las naves de Tharsis, y traían oro, plata, marfil, simios, y pavos.

\hypertarget{la-posiciuxf3n-de-poder-de-salomuxf3n-y-la-riqueza-que-promueve}{%
\subsection{La posición de poder de Salomón y la riqueza que
promueve}\label{la-posiciuxf3n-de-poder-de-salomuxf3n-y-la-riqueza-que-promueve}}

\bibverse{22} Y excedió el rey Salomón á todos los reyes de la tierra en
riqueza y en sabiduría. \bibverse{23} Y todos los reyes de la tierra
procuraban ver el rostro de Salomón, por oir su sabiduría, que Dios
había puesto en su corazón: \bibverse{24} Y de éstos, cada uno traía su
presente, vasos de plata, vasos de oro, vestidos, armas, aromas,
caballos y acémilas, todos los años. \bibverse{25} Tuvo también Salomón
cuatro mil caballerizas para los caballos y carros, y doce mil jinetes,
los cuales puso en las ciudades de los carros, y con el rey en
Jerusalem. \footnote{\textbf{9:25} 2Cró 1,14-17; 1Re 5,6} \bibverse{26}
Y tuvo señorío sobre todos los reyes desde el río hasta la tierra de los
Filisteos, y hasta el término de Egipto. \bibverse{27} Y puso el rey
plata en Jerusalem como piedras, y cedros como los cabrahigos que nacen
por las campiñas en abundancia. \bibverse{28} Sacaban también caballos
para Salomón, de Egipto y de todas las provincias.

\hypertarget{las-fuentes-de-la-historia-de-salomuxf3n-su-muerte}{%
\subsection{Las fuentes de la historia de Salomón; su
muerte}\label{las-fuentes-de-la-historia-de-salomuxf3n-su-muerte}}

\bibverse{29} Lo demás de los hechos de Salomón, primeros y postreros,
¿no está todo escrito en los libros de Nathán profeta, y en la profecía
de Ahías Silonita, y en las profecías del vidente Iddo contra Jeroboam
hijo de Nabat? \bibverse{30} Y reinó Salomón en Jerusalem sobre todo
Israel cuarenta años. \bibverse{31} Y durmió Salomón con sus padres, y
sepultáronlo en la ciudad de David su padre: y reinó en su lugar Roboam
su hijo.

\hypertarget{roboam-y-jeroboam-en-siquem-la-divisiuxf3n-del-imperio}{%
\subsection{Roboam y Jeroboam en Siquem; la división del
imperio}\label{roboam-y-jeroboam-en-siquem-la-divisiuxf3n-del-imperio}}

\hypertarget{section-9}{%
\section{10}\label{section-9}}

\bibverse{1} Y roboam fué á Sichêm porque en Sichêm se había juntado
todo Israel para hacerlo rey. \bibverse{2} Y como lo oyó Jeroboam hijo
de Nabat, el cual estaba en Egipto, donde había huído á causa del rey
Salomón, volvió de Egipto. \footnote{\textbf{10:2} 1Re 11,40}
\bibverse{3} Y enviaron y llamáronle. Vino pues Jeroboam, y todo Israel,
y hablaron á Roboam, diciendo: \bibverse{4} Tu padre agravó nuestro
yugo: afloja tú, pues, ahora algo de la dura servidumbre, y del grave
yugo con que tu padre nos apremió, y te serviremos.

\bibverse{5} Y él les dijo: Volved á mí de aquí á tres días. Y el pueblo
se fué.

\hypertarget{consejeruxeda-de-rehoboams}{%
\subsection{Consejería de Rehoboams}\label{consejeruxeda-de-rehoboams}}

\bibverse{6} Entonces el rey Roboam tomó consejo con los viejos, que
habían estado delante de Salomón su padre cuando vivía, y díjoles: ¿Cómo
aconsejáis vosotros que responda á este pueblo?

\bibverse{7} Y ellos le hablaron, diciendo: Si te condujeres humanamente
con este pueblo, y los agradares, y les hablares buenas palabras, ellos
te servirán perpetuamente.

\bibverse{8} Mas él, dejando el consejo que le dieron los viejos, tomó
consejo con los mancebos que se habían criado con él, y que delante de
él asistían; \bibverse{9} Y díjoles: ¿Qué aconsejáis vosotros que
respondamos á este pueblo, que me ha hablado, diciendo: Alivia algo del
yugo que tu padre puso sobre nosotros?

\bibverse{10} Entonces los mancebos que se habían criado con él, le
hablaron, diciendo: Así dirás al pueblo que te ha hablado diciendo, Tu
padre agravó nuestro yugo, mas tú descárganos: así les dirás: Lo más
menudo mío es más grueso que los lomos de mi padre. \bibverse{11} Así
que, mi padre os cargó de grave yugo, y yo añadiré á vuestro yugo: mi
padre os castigó con azotes, y yo con escorpiones.

\hypertarget{descenso-de-las-diez-tribus-elecciuxf3n-de-jeroboam-como-rey-de-israel}{%
\subsection{Descenso de las diez tribus; Elección de Jeroboam como rey
de
Israel}\label{descenso-de-las-diez-tribus-elecciuxf3n-de-jeroboam-como-rey-de-israel}}

\bibverse{12} Vino pues Jeroboam con todo el pueblo á Roboam al tercer
día: según el rey les había mandado diciendo: Volved á mí de aquí á tres
días. \bibverse{13} Y respondióles el rey ásperamente; pues dejó el rey
Roboam el consejo de los viejos, \bibverse{14} Y hablóles conforme al
consejo de los mancebos, diciendo: Mi padre agravó vuestro yugo, y yo
añadiré á vuestro yugo: mi padre os castigó con azotes, y yo con
escorpiones.

\bibverse{15} Y no escuchó el rey al pueblo; porque la causa era de
Dios, para cumplir Jehová su palabra que había hablado, por Ahías
Silonita, á Jeroboam hijo de Nabat.

\bibverse{16} Y viendo todo Israel que el rey no les había oído,
respondió el pueblo al rey, diciendo: ¿Qué parte tenemos nosotros con
David, ni herencia en el hijo de Isaí? ¡Israel, cada uno á sus
estancias! ¡David, mira ahora por tu casa! Así se fué todo Israel á sus
estancias.

\bibverse{17} Mas reinó Roboam sobre los hijos de Israel que habitaban
en las ciudades de Judá. \bibverse{18} Envió luego el rey Roboam á
Adoram, que tenía cargo de los tributos; pero le apedrearon los hijos de
Israel, y murió. Entonces se esforzó el rey Roboam, y subiendo en un
carro huyó á Jerusalem. \bibverse{19} Así se apartó Israel de la casa de
David hasta hoy.

\hypertarget{roboam-se-abstiene-de-la-guerra-contra-israel-bajo-la-direcciuxf3n-de-dios}{%
\subsection{Roboam se abstiene de la guerra contra Israel bajo la
dirección de
Dios}\label{roboam-se-abstiene-de-la-guerra-contra-israel-bajo-la-direcciuxf3n-de-dios}}

\hypertarget{section-10}{%
\section{11}\label{section-10}}

\bibverse{1} Y como vino Roboam á Jerusalem, juntó la casa de Judá y de
Benjamín, ciento y ochenta mil hombres escogidos de guerra, para pelear
contra Israel y volver el reino á Roboam. \bibverse{2} Mas fué palabra
de Jehová á Semeías varón de Dios, diciendo: \bibverse{3} Habla á Roboam
hijo de Salomón, rey de Judá, y á todos los Israelitas en Judá y
Benjamín, diciéndoles: \bibverse{4} Así ha dicho Jehová: No subáis ni
peleéis contra vuestros hermanos; vuélvase cada uno á su casa, porque yo
he hecho este negocio. Y ellos oyeron la palabra de Jehová, y
tornáronse, y no fueron contra Jeroboam.

\hypertarget{fortalezas-de-roboam}{%
\subsection{Fortalezas de Roboam}\label{fortalezas-de-roboam}}

\bibverse{5} Y habitó Roboam en Jerusalem, y edificó ciudades para
fortificar á Judá. \bibverse{6} Y edificó á Beth-lehem, y á Etham, y á
Tecoa, \bibverse{7} Y á Beth-sur, y á Sochô, y á Adullam, \bibverse{8} Y
á Gath, y á Maresa, y á Ziph, \bibverse{9} Y á Adoraim, y á Lachîs, y á
Acechâ, \bibverse{10} Y á Sora, y á Ajalón, y á Hebrón, que eran en Judá
y en Benjamín, ciudades fuertes. \bibverse{11} Fortificó también las
fortalezas, y puso en ellas capitanes, y vituallas, y vino, y aceite;
\bibverse{12} Y en todas las ciudades, escudos y lanzas. Fortificólas
pues en gran manera, y Judá y Benjamín le estaban sujetos.

\hypertarget{entrada-de-sacerdotes-levitas-y-personas-piadosas-del-reino-de-diez-tribus}{%
\subsection{Entrada de sacerdotes, levitas y personas piadosas del reino
de diez
tribus}\label{entrada-de-sacerdotes-levitas-y-personas-piadosas-del-reino-de-diez-tribus}}

\bibverse{13} Y los sacerdotes y Levitas que estaban en todo Israel, se
juntaron á él de todos sus términos. \bibverse{14} Porque los Levitas
dejaban sus ejidos y sus posesiones, y se venían á Judá y á Jerusalem:
pues Jeroboam y sus hijos los echaban del ministerio de Jehová.
\footnote{\textbf{11:14} 2Cró 13,9} \bibverse{15} Y él se hizo
sacerdotes para los altos, y para los demonios, y para los becerros que
él había hecho. \footnote{\textbf{11:15} 1Re 12,31} \bibverse{16} Tras
aquéllos acudieron también de todas las tribus de Israel los que habían
puesto su corazón en buscar á Jehová Dios de Israel; y viniéronse á
Jerusalem para sacrificar á Jehová, el Dios de sus padres. \bibverse{17}
Así fortificaron el reino de Judá, y confirmaron á Roboam hijo de
Salomón, por tres años; porque tres años anduvieron en el camino de
David y de Salomón.

\hypertarget{historia-familiar-de-rehaboam}{%
\subsection{Historia familiar de
rehaboam}\label{historia-familiar-de-rehaboam}}

\bibverse{18} Y tomóse Roboam por mujer á Mahalath, hija de Jerimoth
hijo de David, y á Abihail, hija de Eliab hijo de Esaí. \footnote{\textbf{11:18}
  1Sam 16,6} \bibverse{19} La cual le parió hijos: á Jeus, y á Samaria,
y á Zaham. \bibverse{20} Después de ella tomó á Maachâ hija de Absalom,
la cual le parió á Abías, á Athai, Ziza, y Selomith. \bibverse{21} Mas
Roboam amó á Maachâ hija de Absalom sobre todas sus mujeres y
concubinas; porque tomó diez y ocho mujeres y sesenta concubinas, y
engendró veintiocho hijos y sesenta hijas. \bibverse{22} Y puso Roboam á
Abías hijo de Maachâ por cabeza y príncipe de sus hermanos, porque
quería hacerle rey. \bibverse{23} E hízole instruir, y esparció todos
sus hijos por todas las tierras de Judá y de Benjamín, y por todas las
ciudades fuertes, y dióles vituallas en abundancia, y pidió muchas
mujeres.

\hypertarget{incursiuxf3n-y-saqueo-del-rey-egipcio-sisak-apariciuxf3n-del-profeta-semeuxedas}{%
\subsection{Incursión y saqueo del rey egipcio Sisak; Aparición del
profeta
Semeías}\label{incursiuxf3n-y-saqueo-del-rey-egipcio-sisak-apariciuxf3n-del-profeta-semeuxedas}}

\hypertarget{section-11}{%
\section{12}\label{section-11}}

\bibverse{1} Y como Roboam hubo confirmado el reino, dejó la ley de
Jehová, y con él todo Israel. \bibverse{2} Y en el quinto año del rey
Roboam subió Sisac rey de Egipto contra Jerusalem, (por cuanto se habían
rebelado contra Jehová,) \bibverse{3} Con mil y doscientos carros, y con
sesenta mil hombres de á caballo: mas el pueblo que venía con él de
Egipto, no tenía número; á saber, de Libios, Sukienos, y Etiopes.
\bibverse{4} Y tomó las ciudades fuertes de Judá, y llegó hasta
Jerusalem. \footnote{\textbf{12:4} 2Cró 11,4-10} \bibverse{5} Entonces
vino Semeías profeta á Roboam y á los príncipes de Judá, que estaban
reunidos en Jerusalem por causa de Sisac, y díjoles: Así ha dicho
Jehová: Vosotros me habéis dejado, y yo también os he dejado en manos de
Sisac.

\bibverse{6} Y los príncipes de Israel y el rey se humillaron, y
dijeron: Justo es Jehová.

\bibverse{7} Y como vió Jehová que se habían humillado, fué palabra de
Jehová á Semeías, diciendo: Hanse humillado; no los destruiré; antes los
salvaré en breve, y no se derramará mi ira contra Jerusalem por mano de
Sisac. \bibverse{8} Empero serán sus siervos; para que sepan qué es
servirme á mí, y servir á los reinos de las naciones.

\bibverse{9} Subió pues Sisac rey de Egipto á Jerusalem, y tomó los
tesoros de la casa de Jehová, y los tesoros de la casa del rey; todo lo
llevó: y tomó los paveses de oro que Salomón había hecho. \bibverse{10}
Y en lugar de ellos hizo el rey Roboam paveses de metal, y entrególos en
manos de los jefes de la guardia, los cuales custodiaban la entrada de
la casa del rey. \bibverse{11} Y cuando el rey iba á la casa de Jehová,
venían los de la guardia, y traíanlos, y después los volvían á la cámara
de la guardia. \bibverse{12} Y como él se humilló, la ira de Jehová se
apartó de él, para no destruirlo del todo: y también en Judá las cosas
fueron bien.

\hypertarget{conclusiuxf3n-del-gobierno-de-roboam-y-las-fuentes-de-su-historia}{%
\subsection{Conclusión del gobierno de Roboam y las fuentes de su
historia}\label{conclusiuxf3n-del-gobierno-de-roboam-y-las-fuentes-de-su-historia}}

\bibverse{13} Fortificado pues Roboam, reinó en Jerusalem: y era Roboam
de cuarenta y un años cuando comenzó á reinar, y diecisiete años reinó
en Jerusalem, ciudad que escogió Jehová de todas las tribus de Israel,
para poner en ella su nombre. Y el nombre de su madre fué Naama
Ammonita. \bibverse{14} E hizo lo malo, porque no apercibió su corazón
para buscar á Jehová.

\bibverse{15} Y las cosas de Roboam, primeras y postreras, ¿no están
escritas en los libros de Semeías profeta y de Iddo vidente, en la
cuenta de los linajes? Y entre Roboam y Jeroboam hubo perpetua guerra.
\footnote{\textbf{12:15} 2Cró 13,22} \bibverse{16} Y durmió Roboam con
sus padres, y fué sepultado en la ciudad de David: y reinó en su lugar
Abías su hijo.

\hypertarget{la-guerra-de-abias-con-jeroboam-su-discurso-al-ejuxe9rcito-de-jeroboam}{%
\subsection{La guerra de Abias con Jeroboam; su discurso al ejército de
Jeroboam}\label{la-guerra-de-abias-con-jeroboam-su-discurso-al-ejuxe9rcito-de-jeroboam}}

\hypertarget{section-12}{%
\section{13}\label{section-12}}

\bibverse{1} A los dieciocho años del rey Jeroboam, reinó Abías sobre
Judá. \bibverse{2} Y reinó tres años en Jerusalem. El nombre de su madre
fué Michâía hija de Uriel de Gabaa. Y hubo guerra entre Abías y
Jeroboam. \bibverse{3} Entonces ordenó Abías batalla con un ejército de
cuatrocientos mil hombres de guerra valerosos y escogidos: y Jeroboam
ordenó batalla contra él con ochocientos mil hombres escogidos, fuertes
y valerosos. \bibverse{4} Y levantóse Abías sobre el monte de Semaraim,
que es en los montes de Ephraim, y dijo: Oidme, Jeroboam y todo Israel.
\bibverse{5} ¿No sabéis vosotros, que Jehová Dios de Israel dió el reino
á David sobre Israel para siempre, á él y á sus hijos en alianza de sal?
\bibverse{6} Pero Jeroboam hijo de Nabat, siervo de Salomón hijo de
David, se levantó y rebeló contra su señor. \bibverse{7} Y se allegaron
á él hombres vanos, hijos de iniquidad, y pudieron más que Roboam hijo
de Salomón, porque Roboam era mozo y tierno de corazón, y no se defendió
de ellos.

\bibverse{8} Y ahora vosotros tratáis de fortificaros contra el reino de
Jehová en mano de los hijos de David, porque sois muchos, y tenéis con
vosotros los becerros de oro que Jeroboam os hizo por dioses.
\footnote{\textbf{13:8} 1Re 12,28} \bibverse{9} ¿No echasteis vosotros á
los sacerdotes de Jehová, á los hijos de Aarón, y á los Levitas, y os
habéis hecho sacerdotes á la manera de los pueblos de otras tierras,
para que cualquiera venga á consagrarse con un becerro y siete carneros,
y así sea sacerdote de los que no son dioses? \footnote{\textbf{13:9}
  2Cró 11,15}

\bibverse{10} Mas en cuanto á nosotros, Jehová es nuestro Dios, y no le
hemos dejado: y los sacerdotes que ministran á Jehová son los hijos de
Aarón, y los Levitas en la obra; \bibverse{11} Los cuales queman á
Jehová los holocaustos cada mañana y cada tarde, y los perfumes
aromáticos; y ponen los panes sobre la mesa limpia, y el candelero de
oro con sus candilejas para que ardan cada tarde: porque nosotros
guardamos la ordenanza de Jehová nuestro Dios; mas vosotros le habéis
dejado. \footnote{\textbf{13:11} Núm 28,3-8} \bibverse{12} Y he aquí
Dios está con nosotros por cabeza, y sus sacerdotes con las trompetas
del júbilo para que suenen contra vosotros. Oh hijos de Israel, no
peleéis contra Jehová el Dios de vuestros padres, porque no os sucederá
bien.

\footnote{\textbf{13:12} Núm 10,9}

\hypertarget{victoria-de-abias-sobre-jeroboam}{%
\subsection{Victoria de Abias sobre
Jeroboam}\label{victoria-de-abias-sobre-jeroboam}}

\bibverse{13} Pero Jeroboam hizo girar una emboscada para venir á ellos
por la espalda: y estando así delante de ellos, la emboscada estaba á
espaldas de Judá. \bibverse{14} Y como miró Judá, he aquí que tenía
batalla delante y á las espaldas; por lo que clamaron á Jehová, y los
sacerdotes tocaron las trompetas. \bibverse{15} Entonces los de Judá
alzaron grita; y así que ellos alzaron grita, Dios desbarató á Jeroboam
y á todo Israel delante de Abías y de Judá: \bibverse{16} Y huyeron los
hijos de Israel delante de Judá, y Dios los entregó en sus manos.
\bibverse{17} Y Abías y su gente hacían en ellos gran mortandad; y
cayeron heridos de Israel quinientos mil hombres escogidos.
\bibverse{18} Así fueron humillados los hijos de Israel en aquel tiempo:
mas los hijos de Judá se fortificaron, porque se apoyaban en Jehová el
Dios de sus padres. \bibverse{19} Y siguió Abías á Jeroboam, y tomóle
algunas ciudades, á Beth-el con sus aldeas, á Jesana con sus aldeas, y á
Ephraim con sus aldeas.

\bibverse{20} Y nunca más tuvo Jeroboam poderío en los días de Abías: é
hirióle Jehová, y murió.

\hypertarget{conclusiuxf3n-y-fuentes-de-la-historia-de-abias}{%
\subsection{Conclusión y fuentes de la historia de
Abias}\label{conclusiuxf3n-y-fuentes-de-la-historia-de-abias}}

\bibverse{21} Empero se fortificó Abías; y tomó catorce mujeres, y
engendró veintidós hijos, y dieciséis hijas. \bibverse{22} Lo demás de
los hechos de Abías, sus caminos y sus negocios, está escrito en la
historia de Iddo profeta. \footnote{\textbf{13:22} 2Cró 12,15}

\hypertarget{la-intervenciuxf3n-de-asa-contra-la-idolatruxeda}{%
\subsection{La intervención de Asa contra la
idolatría}\label{la-intervenciuxf3n-de-asa-contra-la-idolatruxeda}}

\hypertarget{section-13}{%
\section{14}\label{section-13}}

\bibverse{1} Y durmió Abías con sus padres, y fué sepultado en la ciudad
de David. Y reinó en su lugar su hijo Asa, en cuyos días tuvo sosiego el
país por diez años. \footnote{\textbf{14:1} 1Re 15,11-12} \bibverse{2} E
hizo Asa lo bueno y lo recto en los ojos de Jehová su Dios. \bibverse{3}
Porque quitó los altares del culto ajeno, y los altos; quebró las
imágenes, y taló los bosques; \bibverse{4} Y mandó á Judá que buscasen á
Jehová el Dios de sus padres, y pusiesen por obra la ley y sus
mandamientos. \bibverse{5} Quitó asimismo de todas las ciudades de Judá
los altos y las imágenes, y estuvo el reino quieto delante de él.

\footnote{\textbf{14:5} 2Cró 15,15}

\hypertarget{eleva-la-fuerza-defensiva-del-imperio}{%
\subsection{Eleva la fuerza defensiva del
imperio}\label{eleva-la-fuerza-defensiva-del-imperio}}

\bibverse{6} Y edificó ciudades fuertes en Judá, por cuanto había paz en
la tierra, y no había guerra contra él en aquellos tiempos; porque
Jehová le había dado reposo. \bibverse{7} Dijo por tanto á Judá:
Edifiquemos estas ciudades, y cerquémoslas de muros con torres, puertas,
y barras, ya que la tierra es nuestra: porque hemos buscado á Jehová
nuestro Dios, hémosle buscado, y él nos ha dado reposo de todas partes.
Edificaron pues, y fueron prosperados.

\hypertarget{la-victoria-de-asa-sobre-los-cusitas-serah}{%
\subsection{La victoria de Asa sobre los cusitas
Serah}\label{la-victoria-de-asa-sobre-los-cusitas-serah}}

\bibverse{8} Tuvo también Asa ejército que traía escudos y lanzas: de
Judá trescientos mil, y de Benjamín doscientos y ochenta mil que traían
escudos y flechaban arcos; todos hombres diestros.

\bibverse{9} Y salió contra ellos Zera Etiope con un ejército de mil
millares, y trescientos carros; y vino hasta Maresa. \bibverse{10}
Entonces salió Asa contra él, y ordenaron la batalla en el valle de
Sephata junto á Maresa. \bibverse{11} Y clamó Asa á Jehová su Dios, y
dijo: Jehová, no tienes tú más con el grande que con el que ninguna
fuerza tiene, para dar ayuda. Ayúdanos, oh Jehová Dios nuestro, porque
en ti nos apoyamos, y en tu nombre venimos contra este ejército. Oh
Jehová, tú eres nuestro Dios: no prevalezca contra ti el hombre.

\bibverse{12} Y Jehová deshizo los Etiopes delante de Asa y delante de
Judá; y huyeron los Etiopes. \bibverse{13} Y Asa, y el pueblo que con él
estaba, lo siguió hasta Gerar: y cayeron los Etiopes hasta no quedar en
ellos aliento; porque fueron deshechos delante de Jehová y de su
ejército. Y les tomaron muy grande despojo. \bibverse{14} Batieron
también todas las ciudades alrededor de Gerar, porque el terror de
Jehová fué sobre ellos: y saquearon todas las ciudades, porque había en
ellas gran despojo. \bibverse{15} Asimismo dieron sobre las cabañas de
los ganados, y trajeron muchas ovejas y camellos, y volviéronse á
Jerusalem.

\hypertarget{la-amonestaciuxf3n-del-profeta-azaruxedas}{%
\subsection{La amonestación del profeta
Azarías}\label{la-amonestaciuxf3n-del-profeta-azaruxedas}}

\hypertarget{section-14}{%
\section{15}\label{section-14}}

\bibverse{1} Y fué el espíritu de Dios sobre Azarías hijo de Obed;
\bibverse{2} Y salió al encuentro á Asa, y díjole: Oidme, Asa, y todo
Judá y Benjamín: Jehová es con vosotros, si vosotros fuereis con él: y
si le buscareis, será hallado de vosotros; mas si le dejareis, él
también os dejará. \bibverse{3} Muchos días ha estado Israel sin
verdadero Dios y sin sacerdote, y sin enseñador y sin ley: \footnote{\textbf{15:3}
  Os 3,4} \bibverse{4} Mas cuando en su tribulación se convirtieron á
Jehová Dios de Israel, y le buscaron, él fué hallado de ellos.
\footnote{\textbf{15:4} Jer 29,13-14} \bibverse{5} En aquellos tiempos
no hubo paz, ni para el que entraba, ni para el que salía, sino muchas
aflicciones sobre todos los habitadores de las tierras. \bibverse{6} Y
la una gente destruía á la otra, y una ciudad á otra ciudad: porque Dios
los conturbó con todas calamidades. \bibverse{7} Esforzaos empero
vosotros, y no desfallezcan vuestras manos; que salario hay para vuestra
obra.

\footnote{\textbf{15:7} 1Cor 15,58}

\hypertarget{renovaciuxf3n-de-asa-del-pacto-con-dios}{%
\subsection{Renovación de Asa del pacto con
Dios}\label{renovaciuxf3n-de-asa-del-pacto-con-dios}}

\bibverse{8} Y como oyó Asa las palabras y profecía de Obed profeta, fué
confortado, y quitó las abominaciones de toda la tierra de Judá y de
Benjamín, y de las ciudades que él había tomado en el monte de Ephraim;
y reparó el altar de Jehová que estaba delante del pórtico de Jehová.
\bibverse{9} Después hizo juntar á todo Judá y Benjamín, y con ellos los
extranjeros de Ephraim, y de Manasés, y de Simeón: porque muchos de
Israel se habían pasado á él, viendo que Jehová su Dios era con él.
\bibverse{10} Juntáronse pues en Jerusalem en el mes tercero del año
décimoquinto del reinado de Asa. \bibverse{11} Y en aquel mismo día
sacrificaron á Jehová, de los despojos que habían traído, setecientos
bueyes y siete mil ovejas. \bibverse{12} Y entraron en concierto de que
buscarían á Jehová el Dios de sus padres, de todo su corazón y de toda
su alma; \bibverse{13} Y que cualquiera que no buscase á Jehová el Dios
de Israel, muriese, grande ó pequeño, hombre ó mujer. \bibverse{14} Y
juraron á Jehová con gran voz y júbilo, á son de trompetas y de bocinas:
\bibverse{15} Del cual juramento todos los de Judá se alegraron; porque
de todo su corazón lo juraban, y de toda su voluntad lo buscaban: y fué
hallado de ellos; y dióles Jehová reposo de todas partes. \footnote{\textbf{15:15}
  2Cró 14,5-6; 2Cró 20,30}

\bibverse{16} Y aun á Maachâ madre del rey Asa, él mismo la depuso de su
dignidad, porque había hecho un ídolo en el bosque: y Asa deshizo su
ídolo, y lo desmenuzó, y quemó en el torrente de Cedrón. \footnote{\textbf{15:16}
  1Re 15,13-15} \bibverse{17} Mas con todo eso los altos no eran
quitados de Israel, aunque el corazón de Asa fué perfecto mientras
vivió. \bibverse{18} Y metió en la casa de Dios lo que su padre había
dedicado, y lo que él había consagrado, plata y oro y vasos.

\hypertarget{la-guerra-de-asa-con-baesa-de-israel-su-refugio-en-ben-adad-de-siria}{%
\subsection{La guerra de Asa con Baesa de Israel; su refugio en Ben-adad
de
Siria}\label{la-guerra-de-asa-con-baesa-de-israel-su-refugio-en-ben-adad-de-siria}}

\bibverse{19} Y no hubo guerra hasta los treinta y cinco años del
reinado de Asa.

\hypertarget{section-15}{%
\section{16}\label{section-15}}

\bibverse{1} En el año treinta y seis del reinado de Asa, subió Baasa
rey de Israel contra Judá, y edificó á Rama, para no dejar salir ni
entrar á ninguno al rey Asa, rey de Judá. \footnote{\textbf{16:1} 1Re
  15,16-22} \bibverse{2} Entonces sacó Asa la plata y el oro de los
tesoros de la casa de Jehová y de la casa real, y envió á Ben-adad rey
de Siria, que estaba en Damasco, diciendo: \bibverse{3} Haya alianza
entre mí y ti, como la hubo entre mi padre y tu padre; he aquí yo te he
enviado plata y oro, para que vengas y deshagas la alianza que tienes
con Baasa rey de Israel, á fin de que se retire de mí.

\bibverse{4} Y consintió Ben-adad con el rey Asa, y envió los capitanes
de sus ejércitos á la ciudades de Israel: y batieron á Ion, Dan, y
Abel-maim, y las ciudades fuertes de Nephtalí. \bibverse{5} Y oyendo
esto Baasa, cesó de edificar á Rama, y dejó su obra. \bibverse{6}
Entonces el rey Asa tomó á todo Judá, y lleváronse de Rama la piedra y
madera con que Baasa edificaba, y con ella edificó á Gibaa y Mizpa.

\hypertarget{el-discurso-de-castigo-de-hanani-a-asa-tiene-un-efecto-negativo}{%
\subsection{El discurso de castigo de Hanani a Asa tiene un efecto
negativo}\label{el-discurso-de-castigo-de-hanani-a-asa-tiene-un-efecto-negativo}}

\bibverse{7} En aquel tiempo vino Hanani vidente á Asa rey de Judá, y
díjole: Por cuanto te has apoyado en el rey de Siria, y no te apoyaste
en Jehová tu Dios, por eso el ejército del rey de Siria ha escapado de
tus manos. \bibverse{8} Los Etiopes y los Libios, ¿no eran un ejército
numerosísimo, con carros y muy mucha gente de á caballo? con todo,
porque te apoyaste en Jehová, él los entregó en tus manos. \footnote{\textbf{16:8}
  2Cró 14,8-12} \bibverse{9} Porque los ojos de Jehová contemplan toda
la tierra, para corroborar á los que tienen corazón perfecto para con
él. Locamente has hecho en esto; porque de aquí adelante habrá guerra
contra ti.

\bibverse{10} Y enojado Asa contra el vidente, echólo en la casa de la
cárcel, porque fué en extremo conmovido á causa de esto. Y oprimió Asa
en aquel tiempo algunos del pueblo.

\hypertarget{el-fin-de-asa-y-un-entierro-honorable}{%
\subsection{El fin de Asa y un entierro
honorable}\label{el-fin-de-asa-y-un-entierro-honorable}}

\bibverse{11} Mas he aquí, los hechos de Asa, primeros y postreros,
están escritos en el libro de los reyes de Judá y de Israel.
\bibverse{12} Y el año treinta y nueve de su reinado enfermó Asa de los
pies para arriba, y en su enfermedad no buscó á Jehová, sino á los
médicos. \bibverse{13} Y durmió Asa con sus padres, y murió en el año
cuarenta y uno de su reinado. \bibverse{14} Y sepultáronlo en sus
sepulcros que él había hecho para sí en la ciudad de David; Y pusiéronlo
en una litera, la cual hinchieron de aromas y diversas materias
odoríferas, preparadas por obra de perfumadores; é hiciéronle una quema
muy grande. \footnote{\textbf{16:14} 2Cró 21,19; Jer 34,5}

\hypertarget{el-gobierno-piadoso-y-feliz-de-josafat}{%
\subsection{El gobierno piadoso y feliz de
Josafat}\label{el-gobierno-piadoso-y-feliz-de-josafat}}

\hypertarget{section-16}{%
\section{17}\label{section-16}}

\bibverse{1} Y reinó en su lugar Josaphat su hijo, el cual prevaleció
contra Israel. \footnote{\textbf{17:1} 1Re 15,24} \bibverse{2} Y puso
ejército en todas las ciudades fuertes de Judá, y colocó gente de
guarnición, en tierra de Judá, y asimismo en las ciudades de Ephraim que
su padre Asa había tomado. \bibverse{3} Y fué Jehová con Josaphat,
porque anduvo en los primeros caminos de David su padre, y no buscó á
los Baales; \bibverse{4} Sino que buscó al Dios de su padre, y anduvo en
sus mandamientos, y no según las obras de Israel. \bibverse{5} Jehová
por tanto confirmó el reino en su mano, y todo Judá dió á Josaphat
presentes; y tuvo riquezas y gloria en abundancia. \footnote{\textbf{17:5}
  2Cró 18,1} \bibverse{6} Y animóse su corazón en los caminos de Jehová,
y quitó los altos y los bosques de Judá.

\hypertarget{josafat-instruye-al-pueblo-en-la-ley-del-seuxf1or}{%
\subsection{Josafat instruye al pueblo en la ley del
Señor}\label{josafat-instruye-al-pueblo-en-la-ley-del-seuxf1or}}

\bibverse{7} Al tercer año de su reinado envió sus príncipes Ben-hail,
Obdías, Zachârías, Nathaniel y Michêas, para que enseñasen en las
ciudades de Judá; \bibverse{8} Y con ellos á los Levitas, Semeías,
Nethanías, Zebadías, y Asael, y Semiramoth, y Jonathán, y Adonías, y
Tobías, y Tobadonías, Levitas; y con ellos á Elisama y á Joram,
sacerdotes. \bibverse{9} Y enseñaron en Judá, teniendo consigo el libro
de la ley de Jehová, y rodearon por todas las ciudades de Judá enseñando
al pueblo.

\hypertarget{la-reputaciuxf3n-de-josafat-entre-los-pueblos-vecinos-y-su-importante-poder-militar}{%
\subsection{La reputación de Josafat entre los pueblos vecinos y su
importante poder
militar}\label{la-reputaciuxf3n-de-josafat-entre-los-pueblos-vecinos-y-su-importante-poder-militar}}

\bibverse{10} Y cayó el pavor de Jehová sobre todos los reinos de las
tierras que estaban alrededor de Judá; que no osaron hacer guerra contra
Josaphat. \bibverse{11} Y traían de los Filisteos presentes á Josaphat,
y tributos de plata. Los Arabes también le trajeron ganados, siete mil y
setecientos carneros y siete mil y setecientos machos de cabrío.
\bibverse{12} Iba pues Josaphat creciendo altamente: y edificó en Judá
fortalezas y ciudades de depósitos. \bibverse{13} Tuvo además muchas
obras en las ciudades de Judá, y hombres de guerra muy valientes en
Jerusalem. \bibverse{14} Y este es el número de ellos según las casas de
sus padres: en Judá, jefes de los millares: el general Adna, y con él
trescientos mil hombres muy esforzados; \bibverse{15} Después de él, el
jefe Johanán, y con él doscientos y ochenta mil; \bibverse{16} Tras
éste, Amasías hijo de Zichri, el cual se había ofrecido voluntariamente
á Jehová, y con él doscientos mil hombres valientes; \bibverse{17} De
Benjamín, Eliada, hombre muy valeroso, y con él doscientos mil armados
de arco y escudo; \bibverse{18} Tras éste, Jozabad, y con él ciento y
ochenta mil apercibidos para la guerra. \bibverse{19} Estos eran siervos
del rey, sin los que había el rey puesto en las ciudades de guarnición
por toda Judea.

\hypertarget{josafat-y-acab-unen-fuerzas-en-una-guerra-contra-los-sirios}{%
\subsection{Josafat y Acab unen fuerzas en una guerra contra los
sirios}\label{josafat-y-acab-unen-fuerzas-en-una-guerra-contra-los-sirios}}

\hypertarget{section-17}{%
\section{18}\label{section-17}}

\bibverse{1} Tenía pues Josaphat riquezas y gloria en abundancia, y
trabó parentesco con Achâb. \footnote{\textbf{18:1} 2Cró 17,5}
\bibverse{2} Y después de algunos años descendió á Achâb á Samaria; por
lo que mató Achâb muchas ovejas y bueyes para él, y para la gente que
con él venía: y persuadióle que fuese con él á Ramoth de Galaad.
\bibverse{3} Y dijo Achâb rey de Israel á Josaphat rey de Judá: ¿Quieres
venir conmigo á Ramoth de Galaad? Y él respondió: Como yo, así también
tú; y como tu pueblo, así también mi pueblo: iremos contigo á la guerra.

\hypertarget{el-mensaje-favorable-de-los-400-profetas-micha-deberuxeda-ser-entrevistado}{%
\subsection{El mensaje favorable de los 400 profetas; Micha debería ser
entrevistado}\label{el-mensaje-favorable-de-los-400-profetas-micha-deberuxeda-ser-entrevistado}}

\bibverse{4} Además dijo Josaphat al rey de Israel: Ruégote que
consultes hoy la palabra de Jehová.

\bibverse{5} Entonces el rey de Israel juntó cuatrocientos profetas, y
díjoles: ¿Iremos á la guerra contra Ramoth de Galaad, ó estaréme yo
quieto? Y ellos dijeron: Sube, que Dios los entregará en mano del rey.

\bibverse{6} Mas Josaphat dijo: ¿Hay aún aquí algún profeta de Jehová,
para que por él preguntemos?

\bibverse{7} Y el rey de Israel respondió á Josaphat: Aun hay aquí un
hombre por el cual podemos preguntar á Jehová: mas yo le aborrezco,
porque nunca me profetiza cosa buena, sino siempre mal. Este es Michêas,
hijo de Imla. Y respondió Josaphat: No hable así el rey.

\bibverse{8} Entonces el rey de Israel llamó un eunuco, y díjole: Haz
venir luego á Michêas hijo de Imla.

\bibverse{9} Y el rey de Israel y Josaphat rey de Judá, estaban sentados
cada uno en su trono, vestidos de sus ropas; y estaban sentados en la
era á la entrada de la puerta de Samaria, y todos los profetas
profetizaban delante de ellos. \bibverse{10} Y Sedechîas hijo de
Chênaana se había hecho cuernos de hierro, y decía: Así ha dicho Jehová:
Con estos acornearás á los Siros hasta destruirlos del todo.

\bibverse{11} De esta manera profetizaban también todos los profetas,
diciendo: Sube á Ramoth de Galaad, y sé prosperado; porque Jehová la
entregará en mano del rey.

\hypertarget{la-buena-fortuna-inicial-de-micha-luego-su-anuncio-de-la-perdiciuxf3n}{%
\subsection{La buena fortuna inicial de Micha, luego su anuncio de la
perdición}\label{la-buena-fortuna-inicial-de-micha-luego-su-anuncio-de-la-perdiciuxf3n}}

\bibverse{12} Y el mensajero que había ido á llamar á Michêas, le habló,
diciendo: He aquí las palabras de los profetas á una boca anuncian al
rey bienes; yo pues te ruego que tu palabra sea como la de uno de ellos,
que hables bien.

\bibverse{13} Y dijo Michêas: Vive Jehová, que lo que mi Dios me dijere,
eso hablaré. Y vino al rey.

\bibverse{14} Y el rey le dijo: Michêas, ¿iremos á pelear contra Ramoth
de Galaad, ó estaréme yo quieto? Y él respondió: Subid, que seréis
prosperados, que serán entregados en vuestras manos.

\bibverse{15} Y el rey le dijo: ¿Hasta cuántas veces te conjuraré por el
nombre de Jehová que no me hables sino la verdad?

\bibverse{16} Entonces él dijo: He visto á todo Israel derramado por los
montes como ovejas sin pastor: y dijo Jehová: Estos no tienen señor;
vuélvase cada uno en paz á su casa.

\bibverse{17} Y el rey de Israel dijo á Josaphat: ¿No te había yo dicho
que no me profetizaría bien, sino mal?

\bibverse{18} Entonces él dijo: Oid pues palabra de Jehová: Yo he visto
á Jehová sentado en su trono, y todo el ejército de los cielos estaba á
su mano derecha y á su izquierda. \bibverse{19} Y Jehová dijo: ¿Quién
inducirá á Achâb rey de Israel, para que suba y caiga en Ramoth de
Galaad? Y uno decía así, y otro decía de otra manera. \bibverse{20} Mas
salió un espíritu, que se puso delante de Jehová, y dijo: Yo le
induciré. Y Jehová le dijo: ¿De qué modo?

\bibverse{21} Y él dijo: Saldré y seré espíritu de mentira en la boca de
todos sus profetas. Y Jehová dijo: Incita, y también prevalece: sal, y
hazlo así.

\bibverse{22} Y he aquí ahora ha puesto Jehová espíritu de mentira en la
boca de estos tus profetas; mas Jehová ha decretado el mal acerca de ti.

\hypertarget{el-maltrato-de-miqueas-por-sedequuxedas-y-su-captura-por-acab}{%
\subsection{El maltrato de Miqueas por Sedequías y su captura por
Acab}\label{el-maltrato-de-miqueas-por-sedequuxedas-y-su-captura-por-acab}}

\bibverse{23} Entonces Sedechîas hijo de Chênaana se llegó á él, é hirió
á Michêas en la mejilla, y dijo: ¿Por qué camino se apartó de mí el
espíritu de Jehová para hablarte á ti? \footnote{\textbf{18:23} 2Cró
  18,10}

\bibverse{24} Y Michêas respondió: He aquí tú lo verás aquel día, cuando
te entrarás de cámara en cámara para esconderte.

\bibverse{25} Entonces el rey de Israel dijo: Tomad á Michêas, y
volvedlo á Amón gobernador de la ciudad, y á Joas hijo del rey.
\bibverse{26} Y diréis: El rey ha dicho así: Poned á éste en la cárcel,
y sustentadle con pan de aflicción y agua de angustia, hasta que yo
vuelva en paz.

\bibverse{27} Y Michêas dijo: Si tú volvieres en paz, Jehová no ha
hablado por mí. Dijo además: Oidlo, pueblos todos.

\hypertarget{derrota-de-los-aliados-en-ramoth-muerte-de-acab}{%
\subsection{Derrota de los aliados en Ramoth; Muerte de
Acab}\label{derrota-de-los-aliados-en-ramoth-muerte-de-acab}}

\bibverse{28} Subió pues el rey de Israel, y Josaphat rey de Judá, á
Ramoth de Galaad. \bibverse{29} Y dijo el rey de Israel á Josaphat: Yo
me disfrazaré para entrar en la batalla: mas tú vístete tus vestidos. Y
disfrazóse el rey de Israel, y entró en la batalla. \bibverse{30} Había
el rey de Siria mandado á los capitanes de los carros que tenía consigo,
diciendo: No peleéis con chico ni con grande, sino sólo con el rey de
Israel.

\bibverse{31} Y como los capitanes de los carros vieron á Josaphat,
dijeron: Este es el rey de Israel. Y cercáronlo para pelear; mas
Josaphat clamó, y ayudólo Jehová, y apartólos Dios de él: \bibverse{32}
Pues viendo los capitanes de los carros que no era el rey de Israel,
desistieron de acosarle. \bibverse{33} Mas disparando uno el arco á la
ventura, hirió al rey de Israel entre las junturas y el coselete. El
entonces dijo al carretero: Vuelve tu mano, y sácame del campo, porque
estoy mal herido. \bibverse{34} Y arreció la batalla aquel día, por lo
que estuvo el rey de Israel en pie en el carro enfrente de los Siros
hasta la tarde; mas murió á puestas del sol.

\hypertarget{discurso-de-castigo-del-profeta-jehuxfa-a-josafat}{%
\subsection{Discurso de castigo del profeta Jehú a
Josafat}\label{discurso-de-castigo-del-profeta-jehuxfa-a-josafat}}

\hypertarget{section-18}{%
\section{19}\label{section-18}}

\bibverse{1} Y josaphat rey de Judá se volvió en paz á su casa en
Jerusalem. \bibverse{2} Y salióle al encuentro Jehú el vidente, hijo de
Hanani, y dijo al rey Josaphat: ¿Al impío das ayuda, y amas á los que
aborrecen á Jehová? Pues la ira de la presencia de Jehová será sobre ti
por ello. \bibverse{3} Empero se han hallado en ti buenas cosas, porque
cortaste de la tierra los bosques, y has apercibido tu corazón á buscar
á Dios.

\footnote{\textbf{19:3} 2Cró 17,3-6}

\hypertarget{la-reorganizaciuxf3n-de-la-administraciuxf3n-de-justicia-por-parte-de-josafat}{%
\subsection{La reorganización de la administración de justicia por parte
de
Josafat}\label{la-reorganizaciuxf3n-de-la-administraciuxf3n-de-justicia-por-parte-de-josafat}}

\bibverse{4} Habitó pues Josaphat en Jerusalem; mas daba vuelta y salía
al pueblo, desde Beer-seba hasta el monte de Ephraim, y reducíalos á
Jehová el Dios de sus padres. \bibverse{5} Y puso en la tierra jueces en
todas las ciudades fuertes de Judá, por todos los lugares. \bibverse{6}
Y dijo á los jueces: Mirad lo que hacéis: porque no juzgáis en lugar de
hombre, sino en lugar de Jehová, el cual está con vosotros en el negocio
del juicio. \bibverse{7} Sea pues con vosotros el temor de Jehová;
guardad y haced: porque en Jehová nuestro Dios no hay iniquidad, ni
acepción de personas, ni recibir cohecho.

\bibverse{8} Y puso también Josaphat en Jerusalem algunos de los Levitas
y sacerdotes, y de los padres de familias de Israel, para el juicio de
Jehová y para las causas. Y volviéronse á Jerusalem. \footnote{\textbf{19:8}
  Deut 17,8-9; Deut 19,17} \bibverse{9} Y mandóles, diciendo:
Procederéis asimismo con temor de Jehová, con verdad, y con corazón
íntegro. \bibverse{10} En cualquier causa que viniere á vosotros de
vuestros hermanos que habitan en las ciudades, entre sangre y sangre,
entre ley y precepto, estatutos y derechos, habéis de amonestarles que
no pequen contra Jehová, porque no venga ira sobre vosotros y sobre
vuestros hermanos. Obrando así no pecaréis. \bibverse{11} Y he aquí
Amarías sacerdote será el que os presida en todo negocio de Jehová; y
Zebadías hijo de Ismael, príncipe de la casa de Judá, en todos los
negocios del rey; también los Levitas serán oficiales en presencia de
vosotros. Esforzaos pues, y obrad; que Jehová será con el bueno.

\hypertarget{la-oraciuxf3n-de-josafat-despuuxe9s-de-que-el-enemigo-invadiuxf3}{%
\subsection{La oración de Josafat después de que el enemigo
invadió}\label{la-oraciuxf3n-de-josafat-despuuxe9s-de-que-el-enemigo-invadiuxf3}}

\hypertarget{section-19}{%
\section{20}\label{section-19}}

\bibverse{1} Pasadas estas cosas, aconteció que los hijos de Moab y de
Ammón, y con ellos otros de los Ammonitas, vinieron contra Josaphat á la
guerra. \bibverse{2} Y acudieron, y dieron aviso á Josaphat, diciendo:
Contra ti viene una grande multitud de la otra parte de la mar, y de la
Siria; y he aquí ellos están en Hasasón-tamar, que es Engedi.
\bibverse{3} Entonces él tuvo temor; y puso Josaphat su rostro para
consultar á Jehová, é hizo pregonar ayuno á todo Judá. \bibverse{4} Y
juntáronse los de Judá para pedir socorro á Jehová: y también de todas
las ciudades de Judá vinieron á pedir á Jehová.

\bibverse{5} Púsose entonces Josaphat en pie en la reunión de Judá y de
Jerusalem, en la casa de Jehová, delante del atrio nuevo; \bibverse{6} Y
dijo: Jehová Dios de nuestros padres, ¿no eres tú Dios en los cielos, y
te enseñoreas en todos los reinos de las Gentes? ¿no está en tu mano tal
fuerza y potencia, que no hay quien te resista? \footnote{\textbf{20:6}
  1Cró 29,12; 2Cró 14,10} \bibverse{7} Dios nuestro, ¿no echaste tú los
moradores de aquesta tierra delante de tu pueblo Israel, y la diste á la
simiente de Abraham tu amigo para siempre? \bibverse{8} Y ellos han
habitado en ella, y te han edificado en ella santuario á tu nombre,
diciendo: \bibverse{9} Si mal viniere sobre nosotros, ó espada de
castigo, ó pestilencia, ó hambre, presentarnos hemos delante de esta
casa, y delante de ti, (porque tu nombre está en esta casa,) y de
nuestras tribulaciones clamaremos á ti, y tú nos oirás y salvarás.
\bibverse{10} Ahora pues, he aquí los hijos de Ammón y de Moab, y los
del monte de Seir, á la tierra de los cuales no quisiste que pasase
Israel cuando venían de la tierra de Egipto, sino que se apartasen de
ellos, y no los destruyesen; \footnote{\textbf{20:10} Deut 2,4-5; Deut
  2,9; Deut 2,19} \bibverse{11} He aquí ellos nos dan el pago, viniendo
á echarnos de tu heredad, que tú nos diste á poseer. \bibverse{12} ¡Oh
Dios nuestro! ¿no los juzgarás tú? porque en nosotros no hay fuerza
contra tan grande multitud que viene contra nosotros: no sabemos lo que
hemos de hacer, mas á ti volvemos nuestros ojos.

\bibverse{13} Y todo Judá estaba en pie delante de Jehová, con sus
niños, y sus mujeres, y sus hijos.

\hypertarget{la-respuesta-de-dios-promesa-de-victoria-del-levita-jahaziel}{%
\subsection{La respuesta de Dios: promesa de victoria del levita
Jahaziel}\label{la-respuesta-de-dios-promesa-de-victoria-del-levita-jahaziel}}

\bibverse{14} Y estaba allí Jahaziel hijo de Zachârías, hijo de Benaías,
hijo de Jeiel, hijo de Mathanías, Levita de los hijos de Asaph, sobre el
cual vino el espíritu de Jehová en medio de la reunión; \bibverse{15} Y
dijo: Oid, Judá todo, y vosotros moradores de Jerusalem, y tú, rey
Josaphat. Jehová os dice así: No temáis ni os amedrentéis delante de
esta tan grande multitud; porque no es vuestra la guerra, sino de Dios.
\bibverse{16} Mañana descenderéis contra ellos: he aquí que ellos
subirán por la cuesta de Sis, y los hallaréis junto al arroyo, antes del
desierto de Jeruel. \bibverse{17} No habrá para qué vosotros peleéis en
este caso: paraos, estad quedos, y ved la salud de Jehová con vosotros.
Oh Judá y Jerusalem, no temáis ni desmayéis; salid mañana contra ellos,
que Jehová será con vosotros.

\bibverse{18} Entonces Josaphat se inclinó rostro por tierra, y asimismo
todo Judá y los moradores de Jerusalem se postraron delante de Jehová, y
adoraron á Jehová. \bibverse{19} Y levantáronse los Levitas de los hijos
de Coath y de los hijos de Coré, para alabar á Jehová el Dios de Israel
á grande y alta voz.

\hypertarget{la-salida-contra-el-enemigo-autodestrucciuxf3n-del-enemigo-enorme-botuxedn-de-los-juduxedos}{%
\subsection{La salida contra el enemigo; Autodestrucción del enemigo;
enorme botín de los
judíos}\label{la-salida-contra-el-enemigo-autodestrucciuxf3n-del-enemigo-enorme-botuxedn-de-los-juduxedos}}

\bibverse{20} Y como se levantaron por la mañana, salieron por el
desierto de Tecoa. Y mientras ellos salían, Josaphat estando en pie,
dijo: Oidme, Judá y moradores de Jerusalem. Creed á Jehová vuestro Dios,
y seréis seguros; creed á sus profetas, y seréis prosperados.
\footnote{\textbf{20:20} Is 28,16}

\bibverse{21} Y habido consejo con el pueblo, puso á algunos que
cantasen á Jehová, y alabasen en la hermosura de la santidad, mientras
que salía la gente armada, y dijesen: Glorificad á Jehová, porque su
misericordia es para siempre. \footnote{\textbf{20:21} Sal 106,1}
\bibverse{22} Y como comenzaron con clamor y con alabanza, puso Jehová
contra los hijos de Ammón, de Moab, y del monte de Seir, las emboscadas
de ellos mismos que venían contra Judá, y matáronse los unos á los
otros: \bibverse{23} Pues los hijos de Ammón y Moab se levantaron contra
los del monte de Seir, para matarlos y destruirlos; y como hubieron
acabado á los del monte de Seir, cada cual ayudó á la destrucción de su
compañero. \footnote{\textbf{20:23} 1Sam 14,20}

\bibverse{24} Y luego que vino Judá á la atalaya del desierto, miraron
hacia la multitud; mas he aquí yacían ellos en tierra muertos, que
ninguno había escapado. \bibverse{25} Viniendo entonces Josaphat y su
pueblo á despojarlos, hallaron en ellos muchas riquezas entre los
cadáveres, así vestidos como preciosos enseres, los cuales tomaron para
sí, tantos, que no los podían llevar: tres días duró el despojo, porque
era mucho. \bibverse{26} Y al cuarto día se juntaron en el valle de
Beracah; porque allí bendijeron á Jehová, y por esto llamaron el nombre
de aquel paraje el valle de Beracah, hasta hoy. \bibverse{27} Y todo
Judá y los de Jerusalem, y Josaphat á la cabeza de ellos, volvieron para
tornarse á Jerusalem con gozo, porque Jehová les había dado gozo de sus
enemigos. \bibverse{28} Y vinieron á Jerusalem con salterios, arpas, y
bocinas, á la casa de Jehová. \bibverse{29} Y fué el pavor de Dios sobre
todos los reinos de aquella tierra, cuando oyeron que Jehová había
peleado contra los enemigos de Israel. \bibverse{30} Y el reino de
Josaphat tuvo reposo; porque su Dios le dió reposo de todas partes.

\hypertarget{la-conclusiuxf3n-del-gobierno-de-josafat-y-las-fuentes-de-su-historia}{%
\subsection{La conclusión del gobierno de Josafat y las fuentes de su
historia}\label{la-conclusiuxf3n-del-gobierno-de-josafat-y-las-fuentes-de-su-historia}}

\bibverse{31} Así reinó Josaphat sobre Judá: de treinta y cinco años era
cuando comenzó á reinar, y reinó veinte y cinco años en Jerusalem. El
nombre de su madre fué Azuba, hija de Silhi. \footnote{\textbf{20:31}
  1Re 22,41-51} \bibverse{32} Y anduvo en el camino de Asa su padre, sin
apartarse de él, haciendo lo recto en los ojos de Jehová. \bibverse{33}
Con todo eso los altos no eran quitados; que el pueblo aun no había
enderezado su corazón al Dios de sus padres.

\bibverse{34} Lo demás de los hechos de Josaphat, primeros y postreros,
he aquí están escritos en las palabras de Jehú hijo de Hanani, del cual
es hecha mención en el libro de los reyes de Israel.

\hypertarget{la-alianza-de-josafat-con-ocozuxedas-de-israel-y-su-castigo-su-muerte}{%
\subsection{La alianza de Josafat con Ocozías de Israel y su castigo; su
muerte}\label{la-alianza-de-josafat-con-ocozuxedas-de-israel-y-su-castigo-su-muerte}}

\bibverse{35} Pasadas estas cosas, Josaphat rey de Judá trabó amistad
con Ochôzías rey de Israel, el cual fué dado á la impiedad:
\bibverse{36} E hizo con él compañía para aparejar navíos que fuesen á
Tharsis; y construyeron los navíos en Esion-geber. \bibverse{37}
Entonces Eliezer hijo de Dodava de Mareosah, profetizó contra Josaphat,
diciendo: Por cuanto has hecho compañía con Ochôzías, Jehová destruirá
tus obras. Y los navíos se rompieron, y no pudieron ir á Tharsis.

\hypertarget{el-gobierno-del-rey-joram}{%
\subsection{El gobierno del rey Joram}\label{el-gobierno-del-rey-joram}}

\hypertarget{section-20}{%
\section{21}\label{section-20}}

\bibverse{1} Y durmió Josaphat con sus padres, y sepultáronlo con sus
padres en la ciudad de David. Y reinó en su lugar Joram su hijo.

\hypertarget{asesinato-de-sus-hermanos}{%
\subsection{Asesinato de sus hermanos}\label{asesinato-de-sus-hermanos}}

\bibverse{2} Este tuvo hermanos, hijos de Josaphat, á Azarías, Jehiel,
Zachârías, Azarías, Michâel, y Sephatías. Todos estos fueron hijos de
Josaphat rey de Israel. \bibverse{3} Y su padre les había dado muchos
dones de oro y de plata, y cosas preciosas, y ciudades fuertes en Judá;
mas había dado el reino á Joram, porque él era el primogénito.
\bibverse{4} Fué pues elevado Joram al reino de su padre; y luego que se
hizo fuerte, mató á cuchillo á todos sus hermanos, y asimismo algunos de
los príncipes de Israel.

\hypertarget{la-posiciuxf3n-de-dios-sobre-la-apostasuxeda-de-joram}{%
\subsection{La posición de Dios sobre la apostasía de
Joram}\label{la-posiciuxf3n-de-dios-sobre-la-apostasuxeda-de-joram}}

\bibverse{5} Cuando comenzó á reinar era de treinta y dos años, y reinó
ocho años en Jerusalem. \bibverse{6} Y anduvo en el camino de los reyes
de Israel, como hizo la casa de Achâb; porque tenía por mujer la hija de
Achâb, é hizo lo malo en ojos de Jehová. \bibverse{7} Mas Jehová no
quiso destruir la casa de David, á causa de la alianza que con David
había hecho, y porque le había dicho que le daría lámpara á él y á sus
hijos perpetuamente.

\footnote{\textbf{21:7} 2Sam 7,12; 1Re 11,36; Sal 132,17}

\hypertarget{apostasuxeda-de-los-edomitas-y-la-ciudad-de-libna}{%
\subsection{Apostasía de los edomitas y la ciudad de
Libna}\label{apostasuxeda-de-los-edomitas-y-la-ciudad-de-libna}}

\bibverse{8} En los días de éste se rebeló la Idumea, para no estar bajo
el poder de Judá, y pusieron rey sobre sí. \bibverse{9} Entonces pasó
Joram con sus príncipes, y consigo todos sus carros; y levantóse de
noche, é hirió á los Idumeos que le habían cercado, y á todos los
comandantes de sus carros. \bibverse{10} Con todo eso Edom quedó
rebelado, sin estar bajo la mano de Judá hasta hoy. También se rebeló en
el mismo tiempo Libna para no estar bajo su mano; por cuanto él había
dejado á Jehová el Dios de sus padres.

\hypertarget{la-carta-amenazante-del-profeta-eluxedas-a-joram}{%
\subsection{La carta amenazante del profeta Elías a
Joram}\label{la-carta-amenazante-del-profeta-eluxedas-a-joram}}

\bibverse{11} Demás de esto hizo altos en los montes de Judá, é hizo que
los moradores de Jerusalem fornicasen, y á ello impelió á Judá.
\bibverse{12} Y viniéronle letras del profeta Elías, que decían: Jehová,
el Dios de David tu padre, ha dicho así: Por cuanto no has andado en los
caminos de Josaphat tu padre, ni en los caminos de Asa, rey de Judá,
\bibverse{13} Antes has andado en el camino de los reyes de Israel, y
has hecho que fornicase Judá, y los moradores de Jerusalem, como fornicó
la casa de Achâb; y además has muerto á tus hermanos, á la familia de tu
padre, los cuales eran mejores que tú: \bibverse{14} He aquí Jehová
herirá tu pueblo de una grande plaga, y á tus hijos y á tus mujeres, y á
toda tu hacienda; \bibverse{15} Y á ti con muchas enfermedades, con
enfermedad de tus entrañas, hasta que las entrañas se te salgan á causa
de la enfermedad de cada día.

\hypertarget{incursiones-filisteas-y-uxe1rabes}{%
\subsection{Incursiones filisteas y
árabes}\label{incursiones-filisteas-y-uxe1rabes}}

\bibverse{16} Entonces despertó Jehová contra Joram el espíritu de los
Filisteos, y de los Arabes que estaban junto á los Etiopes;
\bibverse{17} Y subieron contra Judá, é invadieron la tierra, y tomaron
toda la hacienda que hallaron en la casa del rey, y á sus hijos, y á sus
mujeres; que no le quedó hijo, sino Joachâz el menor de sus hijos.

\hypertarget{el-agonizante-final-y-el-entierro-deshonroso-de-joram}{%
\subsection{El agonizante final y el entierro deshonroso de
Joram}\label{el-agonizante-final-y-el-entierro-deshonroso-de-joram}}

\bibverse{18} Después de todo esto Jehová lo hirió en las entrañas de
una enfermedad incurable. \bibverse{19} Y aconteció que, pasando un día
tras otro, al fin, al cabo de dos años, las entrañas se le salieron con
la enfermedad, muriendo así de enfermedad muy penosa. Y no le hizo quema
su pueblo, como las había hecho á sus padres. \bibverse{20} Cuando
comenzó á reinar era de treinta y dos años, y reinó en Jerusalem ocho
años; y fuése sin ser deseado. Y sepultáronlo en la ciudad de David, mas
no en los sepulcros de los reyes. \footnote{\textbf{21:20} 2Cró 21,5;
  2Cró 24,25}

\hypertarget{el-gobierno-del-rey-ochuxf4zuxedas-su-gobierno-desaprobando-a-dios}{%
\subsection{El gobierno del rey Ochôzías; Su gobierno desaprobando a
Dios}\label{el-gobierno-del-rey-ochuxf4zuxedas-su-gobierno-desaprobando-a-dios}}

\hypertarget{section-21}{%
\section{22}\label{section-21}}

\bibverse{1} Y los moradores de Jerusalem hicieron rey en lugar suyo á
Ochôzías su hijo menor: porque la tropa que había venido con los Arabes
al campo, había muerto á todos los mayores; por lo cual reinó Ochôzías,
hijo de Joram rey de Judá. \footnote{\textbf{22:1} 2Re 8,25-29}
\bibverse{2} Cuando Ochôzías comenzó á reinar era de cuarenta y dos
años, y reinó un año en Jerusalem. El nombre de su madre fué Athalía,
hija de Omri. \bibverse{3} También él anduvo en los caminos de la casa
de Achâb: porque su madre le aconsejaba á obrar impíamente. \bibverse{4}
Hizo pues lo malo en ojos de Jehová, como la casa de Achâb; porque
después de la muerte de su padre, ellos le aconsejaron para su
perdición.

\hypertarget{su-pacto-con-joram-de-israel-y-su-muerte-por-jehuxfa}{%
\subsection{Su pacto con Joram de Israel y su muerte por
Jehú}\label{su-pacto-con-joram-de-israel-y-su-muerte-por-jehuxfa}}

\bibverse{5} Y él anduvo en los consejos de ellos, y fué á la guerra con
Joram hijo de Achâb, rey de Israel, contra Hazael rey de Siria, á Ramoth
de Galaad, donde los Siros hirieron á Joram. \bibverse{6} Y se volvió
para curarse en Jezreel de las heridas que le habían hecho en Rama,
peleando con Hazael rey de Siria. Y descendió Azarías hijo de Joram, rey
de Judá, á visitar á Joram hijo de Achâb, en Jezreel, porque allí estaba
enfermo.

\bibverse{7} Esto empero venía de Dios, para que Ochôzías fuese hollado
viniendo á Joram: porque siendo venido, salió con Joram contra Jehú hijo
de Nimsi, al cual Jehová había ungido para que talase la casa de Achâb.
\footnote{\textbf{22:7} 1Re 19,16; 2Re 9,6} \bibverse{8} Y fué que,
haciendo juicio Jehú con la casa de Achâb, halló á los príncipes de
Judá, y á los hijos de los hermanos de Ochôzías, que servían á Ochôzías,
y matólos. \footnote{\textbf{22:8} 2Re 10,12-14} \bibverse{9} Y buscando
á Ochôzías, el cual se había escondido en Samaria, tomáronlo, y
trajéronlo á Jehú, y le mataron; y diéronle sepultura, porque dijeron:
Es hijo de Josaphat, el cual buscó á Jehová de todo su corazón. Y la
casa de Ochôzías no tenía fuerzas para poder retener el reino.

\footnote{\textbf{22:9} 2Re 9,27-29}

\hypertarget{el-robo-y-el-asesinato-de-ataluxeda-rescate-de-jouxe1s}{%
\subsection{El robo y el asesinato de Atalía; Rescate de
Joás}\label{el-robo-y-el-asesinato-de-ataluxeda-rescate-de-jouxe1s}}

\bibverse{10} Entonces Athalía madre de Ochôzías, viendo que su hijo era
muerto, levantóse y destruyó toda la simiente real de la casa de Judá.
\bibverse{11} Empero Josabeth, hija del rey, tomó á Joas hijo de
Ochôzías, y arrebatólo de entre los hijos del rey, que mataban, y
guardóle á él y á su ama en la cámara de los lechos. Así pues lo
escondió Josabeth, hija del rey Joram, mujer de Joiada el sacerdote,
(porque ella era hermana de Ochôzías), de delante de Athalía, y no lo
mataron. \bibverse{12} Y estuvo con ellos escondido en la casa de Dios
seis años. Entre tanto Athalía reinaba en el país.

\hypertarget{la-conspiraciuxf3n-de-joiada}{%
\subsection{La conspiración de
Joiada}\label{la-conspiraciuxf3n-de-joiada}}

\hypertarget{section-22}{%
\section{23}\label{section-22}}

\bibverse{1} Mas el séptimo año se animó Joiada, y tomó consigo en
alianza á los centuriones, Azarías hijo de Jeroam, y á Ismael hijo de
Johanán, y á Azarías hijo de Obed, y á Maasías hijo de Adaías, y á
Elisaphat hijo de Zichri; \bibverse{2} Los cuales rodeando por Judá,
juntaron los Levitas de todas las ciudades de Judá, y á los príncipes de
las familias de Israel, y vinieron á Jerusalem. \bibverse{3} Y toda la
multitud hizo alianza con el rey en la casa de Dios. Y él les dijo: He
aquí el hijo del rey, el cual reinará, como Jehová lo tiene dicho de los
hijos de David. \bibverse{4} Lo que habéis de hacer es: la tercera parte
de vosotros, los que entran de semana, estarán de porteros con los
sacerdotes y los Levitas; \bibverse{5} Y la tercera parte, á la casa del
rey; y la tercera parte, á la puerta del fundamento: y todo el pueblo
estará en los patios de la casa de Jehová. \bibverse{6} Y ninguno entre
en la casa de Jehová, sino los sacerdotes y Levitas que sirven: éstos
entrarán, porque están consagrados; y todo el pueblo hará la guardia de
Jehová. \bibverse{7} Y los Levitas rodearán al rey por todas partes, y
cada uno tendrá sus armas en la mano; y cualquiera que entrare en la
casa, muera: y estaréis con el rey cuando entrare, y cuando saliere.

\hypertarget{captura-y-asesinato-de-athalja-elevaciuxf3n-de-jouxe1s-a-rey}{%
\subsection{Captura y asesinato de Athalja; Elevación de Joás a
rey}\label{captura-y-asesinato-de-athalja-elevaciuxf3n-de-jouxe1s-a-rey}}

\bibverse{8} Y los Levitas y todo Judá lo hicieron todo como lo había
mandado el sacerdote Joiada: y tomó cada uno los suyos, los que entraban
de semana, y los que salían el sábado: porque el sacerdote Joiada no dió
licencia á las compañías. \bibverse{9} Dió también el sacerdote Joiada á
los centuriones las lanzas, paveses y escudos que habían sido del rey
David, que estaban en la casa de Dios; \bibverse{10} Y puso en orden á
todo el pueblo, teniendo cada uno su espada en la mano, desde el rincón
derecho del templo hasta el izquierdo, hacia el altar y la casa, en
derredor del rey por todas partes. \bibverse{11} Entonces sacaron al
hijo del rey, y pusiéronle la corona y el testimonio, é hiciéronle rey;
y Joiada y sus hijos le ungieron, diciendo luego: ¡Viva el rey!

\bibverse{12} Y como Athalía oyó el estruendo de la gente que corría, y
de los que bendecían al rey, vino al pueblo á la casa de Jehová;
\bibverse{13} Y mirando, vió al rey que estaba junto á su columna á la
entrada, y los príncipes y los trompetas junto al rey, y que todo el
pueblo de la tierra hacía alegrías, y sonaban bocinas, y cantaban con
instrumentos de música los que sabían alabar. Entonces Athalía rasgó sus
vestidos, y dijo: ¡Conjuración, conjuración!

\bibverse{14} Y sacando el pontífice Joiada los centuriones y capitanes
del ejército, díjoles: Sacadla fuera del recinto; y el que la siguiere,
muera á cuchillo: porque el sacerdote había mandado que no la matasen en
la casa de Jehová. \bibverse{15} Ellos pues le echaron mano, y luego que
hubo ella pasado la entrada de la puerta de los caballos de la casa del
rey, allí la mataron.

\hypertarget{medidas-de-joiada-para-la-gloria-de-dios-coronaciuxf3n-de-jouxe1s}{%
\subsection{Medidas de Joiada para la gloria de Dios; Coronación de
Joás}\label{medidas-de-joiada-para-la-gloria-de-dios-coronaciuxf3n-de-jouxe1s}}

\bibverse{16} Y Joiada hizo pacto entre sí y todo el pueblo y el rey,
que serían pueblo de Jehová. \bibverse{17} Después de esto entró todo el
pueblo en el templo de Baal, y derribáronlo, y también sus altares; é
hicieron pedazos sus imágenes, y mataron delante de los altares á
Mathán, sacerdote de Baal. \bibverse{18} Luego ordenó Joiada los oficios
en la casa de Jehová bajo la mano de los sacerdotes y Levitas, según
David los había distribuído en la casa de Jehová, para ofrecer á Jehová
los holocaustos, como está escrito en la ley de Moisés, con gozo y
cantares, conforme á la ordenación de David. \footnote{\textbf{23:18}
  2Cró 29,30} \bibverse{19} Puso también porteros á las puertas de la
casa de Jehová, para que por ninguna vía entrase ningún inmundo.
\bibverse{20} Tomó después los centuriones, y los principales, y los que
gobernaban el pueblo; y á todo el pueblo de la tierra, y llevó al rey de
la casa de Jehová; y viniendo hasta el medio de la puerta mayor de la
casa del rey, sentaron al rey sobre el trono del reino. \bibverse{21} Y
todo el pueblo del país hizo alegrías: y la ciudad estuvo quieta, muerto
que hubieron á Athalía á cuchillo.

\hypertarget{el-gobierno-del-rey-jouxe1s}{%
\subsection{El gobierno del rey
Joás}\label{el-gobierno-del-rey-jouxe1s}}

\hypertarget{section-23}{%
\section{24}\label{section-23}}

\bibverse{1} De siete años era Joas cuando comenzó á reinar, y cuarenta
años reinó en Jerusalem. El nombre de su madre fué Sibia, de Beer-seba.
\bibverse{2} E hizo Joas lo recto en ojos de Jehová todos los días de
Joiada el sacerdote. \bibverse{3} Y tomó para él Joiada dos mujeres; y
engendró hijos é hijas.

\hypertarget{reparando-el-templo-ordenanza-sobre-la-administraciuxf3n-y-el-uso-del-dinero-entrante-para-el-templo}{%
\subsection{Reparando el templo; Ordenanza sobre la administración y el
uso del dinero entrante para el
templo}\label{reparando-el-templo-ordenanza-sobre-la-administraciuxf3n-y-el-uso-del-dinero-entrante-para-el-templo}}

\bibverse{4} Después de esto aconteció que Joas tuvo voluntad de reparar
la casa de Jehová. \bibverse{5} Y juntó los sacerdotes y los Levitas, y
díjoles: Salid por las ciudades de Judá, y juntad dinero de todo Israel,
para que cada año sea reparada la casa de vuestro Dios; y vosotros poned
diligencia en el negocio. Mas los Levitas no pusieron diligencia.
\bibverse{6} Por lo cual el rey llamó á Joiada el principal, y díjole:
¿Por qué no has procurado que los Levitas traigan de Judá y de Jerusalem
al tabernáculo del testimonio, la ofrenda que impuso Moisés siervo de
Jehová, y de la congregación de Israel? \footnote{\textbf{24:6} Éxod
  30,12-13} \bibverse{7} Porque la impía Athalía y sus hijos habían
destruído la casa de Dios, y además habían gastado en los ídolos todas
las cosas consagradas á la casa de Jehová. \footnote{\textbf{24:7} 2Cró
  22,3-4}

\bibverse{8} Mandó pues el rey que hiciesen un arca, la cual pusieron
fuera á la puerta de la casa de Jehová; \bibverse{9} E hicieron pregonar
en Judá y en Jerusalem, que trajesen á Jehová la ofrenda que Moisés
siervo de Dios había impuesto á Israel en el desierto. \footnote{\textbf{24:9}
  2Cró 24,6} \bibverse{10} Y todos los príncipes y todo el pueblo se
holgaron: y traían, y echaban en el arca hasta henchirla. \bibverse{11}
Y como venía el tiempo para llevar el arca al magistrado del rey por
mano de los Levitas, cuando veían que había mucho dinero, venía el
escriba del rey, y el que estaba puesto por el sumo sacerdote, y
llevaban el arca, y vaciábanla, y volvíanla á su lugar: y así lo hacían
de día en día, y recogían mucho dinero; \bibverse{12} El cual daba el
rey y Joiada á los que hacían la obra del servicio de la casa de Jehová,
y tomaban canteros y oficiales que reparasen la casa de Jehová, y
herreros y metalarios para componer la casa de Jehová. \bibverse{13}
Hacían pues los oficiales la obra, y por sus manos fué la obra
restaurada, y restituyeron la casa de Dios á su condición, y la
consolidaron. \bibverse{14} Y cuando hubieron acabado, trajeron lo que
quedaba del dinero al rey y á Joiada, é hicieron de él vasos para la
casa de Jehová, vasos para el servicio, morteros, cucharas, vasos de oro
y de plata. Y sacrificaban holocaustos continuamente en la casa de
Jehová todos los días de Joiada.

\hypertarget{el-alejamiento-de-jouxe1s-de-dios-despuuxe9s-de-la-muerte-de-joiada-el-discurso-de-zachuxe2ruxedas-y-su-lapidaciuxf3n}{%
\subsection{El alejamiento de Joás de Dios después de la muerte de
Joiada; El discurso de Zachârías y su
lapidación}\label{el-alejamiento-de-jouxe1s-de-dios-despuuxe9s-de-la-muerte-de-joiada-el-discurso-de-zachuxe2ruxedas-y-su-lapidaciuxf3n}}

\bibverse{15} Mas Joiada envejeció, y murió harto de días: de ciento y
treinta años era cuando murió. \bibverse{16} Y sepultáronlo en la ciudad
de David con los reyes, por cuanto había hecho bien con Israel, y para
con Dios, y con su casa.

\bibverse{17} Muerto Joiada, vinieron los príncipes de Judá, é hicieron
acatamiento al rey; y el rey los oyó. \bibverse{18} Y desampararon la
casa de Jehová el Dios de sus padres, y sirvieron á los bosques y á las
imágenes esculpidas; y la ira vino sobre Judá y Jerusalem por este su
pecado. \bibverse{19} Y envióles profetas, para que los redujesen á
Jehová, los cuales les protestaron: mas ellos no los escucharon.

\bibverse{20} Y el espíritu de Dios envistió á Zachârías, hijo de Joiada
el sacerdote, el cual estando sobre el pueblo, les dijo: Así ha dicho
Dios: ¿Por qué quebrantáis los mandamientos de Jehová? No os vendrá bien
de ello; porque por haber dejado á Jehová, él también os dejará.

\bibverse{21} Mas ellos hicieron conspiración contra él, y cubriéronle
de piedras por mandato del rey, en el patio de la casa de Jehová.
\bibverse{22} No tuvo pues memoria el rey Joas de la misericordia que su
padre Joiada había hecho con él, antes matóle su hijo; el cual dijo al
morir: Jehová lo vea, y lo requiera.

\hypertarget{guerra-desafortunada-con-los-sirios-asesinato-del-rey-por-conspiradores-palabra-final}{%
\subsection{Guerra desafortunada con los sirios; Asesinato del rey por
conspiradores; Palabra
final}\label{guerra-desafortunada-con-los-sirios-asesinato-del-rey-por-conspiradores-palabra-final}}

\bibverse{23} A la vuelta del año subió contra él el ejército de Siria;
y vinieron á Judá y á Jerusalem, y destruyeron en el pueblo á todos los
principales de él, y enviaron todos sus despojos al rey á Damasco.
\bibverse{24} Porque aunque el ejército de Siria había venido con poca
gente, Jehová les entregó en sus manos un ejército muy numeroso; por
cuanto habían dejado á Jehová el Dios de sus padres. Y con Joas hicieron
juicios.

\bibverse{25} Y yéndose de él los Siros, dejáronlo en muchas
enfermedades; y conspiraron contra él sus siervos á causa de las sangres
de los hijos de Joiada el sacerdote, é hiriéronle en su cama, y murió: y
sepultáronle en la ciudad de David, mas no lo sepultaron en los
sepulcros de los reyes. \footnote{\textbf{24:25} 2Cró 21,20}
\bibverse{26} Los que conspiraron contra él fueron Zabad, hijo de Simath
Ammonita, y Jozabad, hijo de Simrith Moabita. \bibverse{27} De sus
hijos, y de la multiplicación que hizo de las rentas, y de la
instauración de la casa de Jehová, he aquí está escrito en la historia
del libro de los reyes. Y reinó en su lugar Amasías su hijo.

\hypertarget{el-gobierno-del-rey-amasuxedas-buen-comienzo-para-el-gobierno}{%
\subsection{El gobierno del rey Amasías; Buen comienzo para el
gobierno}\label{el-gobierno-del-rey-amasuxedas-buen-comienzo-para-el-gobierno}}

\hypertarget{section-24}{%
\section{25}\label{section-24}}

\bibverse{1} De veinticinco años era Amasías cuando comenzó á reinar, y
veintinueve años reinó en Jerusalem: el nombre de su madre fué Joaddan,
de Jerusalem. \bibverse{2} Hizo él lo recto en los ojos de Jehová aunque
no de perfecto corazón. \bibverse{3} Y luego que fué confirmado en el
reino, mató á sus siervos que habían muerto al rey su padre;
\bibverse{4} Mas no mató á los hijos de ellos, según lo que está escrito
en la ley en el libro de Moisés, donde Jehová mandó, diciendo: No
morirán los padres por los hijos, ni los hijos por los padres; mas cada
uno morirá por su pecado.

\footnote{\textbf{25:4} Deut 24,16}

\hypertarget{la-victoria-de-amasuxedas-sobre-los-edomitas-despuuxe9s-de-que-los-mercenarios-israelitas-fueran-devueltos-la-venganza-de-estas-tropas}{%
\subsection{La victoria de Amasías sobre los edomitas después de que los
mercenarios israelitas fueran devueltos; la venganza de estas
tropas}\label{la-victoria-de-amasuxedas-sobre-los-edomitas-despuuxe9s-de-que-los-mercenarios-israelitas-fueran-devueltos-la-venganza-de-estas-tropas}}

\bibverse{5} Juntó luego Amasías á Judá, y con arreglo á las familias
púsoles tribunos y centuriones por todo Judá y Benjamín; y tomólos por
lista de veinte años arriba, y fueron hallados en ellos trescientos mil
escogidos para salir á la guerra, que tenían lanza y escudo.
\bibverse{6} Y de Israel tomó á sueldo cien mil hombres valientes, por
cien talentos de plata. \bibverse{7} Mas un varón de Dios vino á él,
diciéndole: Rey, no vaya contigo el ejército de Israel; porque Jehová no
es con Israel, ni con todos los hijos de Ephraim. \bibverse{8} Pero si
tú vas, si lo haces, y te esfuerzas para pelear, Dios te hará caer
delante de los enemigos; porque en Dios está la fortaleza, ó para
ayudar, ó para derribar.

\bibverse{9} Y Amasías dijo al varón de Dios: ¿Qué pues se hará de cien
talentos que he dado al ejército de Israel? Y el varón de Dios
respondió: De Jehová es darte mucho más que esto.

\bibverse{10} Entonces Amasías apartó el escuadrón de la gente que había
venido á él de Ephraim, para que se fuesen á sus casas: y ellos se
enojaron grandemente contra Judá, y volviéronse á sus casas
encolerizados.

\bibverse{11} Esforzándose entonces Amasías, sacó su pueblo, y vino al
valle de la Sal: é hirió de los hijos de Seir diez mil. \bibverse{12} Y
los hijos de Judá tomaron vivos otros diez mil, los cuales llevaron á la
cumbre de un peñasco, y de allí los despeñaron, y todos se hicieron
pedazos. \bibverse{13} Empero los del escuadrón que Amasías había
despedido, porque no fuesen con él á la guerra, derramáronse sobre las
ciudades de Judá, desde Samaria hasta Beth-oron, é hirieron de ellos
tres mil, y tomaron un grande despojo.

\hypertarget{amasuxedas-se-aparta-de-dios-advertencia-de-un-profeta}{%
\subsection{Amasías se aparta de Dios; Advertencia de un
profeta}\label{amasuxedas-se-aparta-de-dios-advertencia-de-un-profeta}}

\bibverse{14} Regresando luego Amasías de la matanza de los Idumeos,
trajo también consigo los dioses de los hijos de Seir, y púsoselos para
sí por dioses, y encorvóse delante de ellos, y quemóles perfumes.
\bibverse{15} Encendióse por tanto el furor de Jehová contra Amasías, y
envió á él un profeta, que le dijo: ¿Por qué has buscado los dioses de
gente, que no libraron á su pueblo de tus manos?

\bibverse{16} Y hablándole el profeta estas cosas, él le respondió:
¿Hante puesto á ti por consejero del rey? Déjate de eso: ¿por qué
quieres que te maten? Y al cesar, el profeta dijo luego: Yo sé que Dios
ha acordado destruirte, porque has hecho esto, y no obedeciste á mi
consejo.

\hypertarget{la-desafortunada-guerra-de-amasuxedas-con-jouxe1s-de-israel}{%
\subsection{La desafortunada guerra de Amasías con Joás de
Israel}\label{la-desafortunada-guerra-de-amasuxedas-con-jouxe1s-de-israel}}

\bibverse{17} Y Amasías rey de Judá, habido su consejo, envió á decir á
Joas, hijo de Joachâz hijo de Jehú, rey de Israel: Ven, y veámonos cara
á cara.

\bibverse{18} Entonces Joas rey de Israel envió á decir á Amasías rey de
Judá: El cardo que estaba en el Líbano, envió al cedro que estaba en el
Líbano, diciendo: Da tu hija á mi hijo por mujer. Y he aquí que las
bestias fieras que estaban en el Líbano, pasaron, y hollaron el cardo.
\bibverse{19} Tú dices: He aquí he herido á Edom; y tu corazón se
enaltece para gloriarte: ahora estáte en tu casa; ¿para qué te
entrometes en mal, para caer tú y Judá contigo?

\bibverse{20} Mas Amasías no quiso oir; porque estaba de Dios, que los
quería entregar en manos de sus enemigos, por cuanto habían buscado los
dioses de Edom. \bibverse{21} Subió pues Joas rey de Israel, y viéronse
cara á cara él y Amasías rey de Judá, en Beth-semes, la cual es de Judá.
\bibverse{22} Pero cayó Judá delante de Israel, y huyó cada uno á su
estancia.

\bibverse{23} Y Joas rey de Israel prendió en Beth-semes á Amasías rey
de Judá, hijo de Joas hijo de Joachâz, y llevólo á Jerusalem: y derribó
el muro de Jerusalem desde la puerta de Ephraim hasta la puerta del
ángulo, cuatrocientos codos. \bibverse{24} Asimismo tomó todo el oro y
plata, y todos los vasos que se hallaron en la casa de Dios en casa de
Obed-edom, y los tesoros de la casa del rey, y los hijos de los
príncipes, y volvióse á Samaria.

\hypertarget{palabra-final-asesinato-del-rey-por-conspiradores}{%
\subsection{Palabra final; Asesinato del rey por
conspiradores}\label{palabra-final-asesinato-del-rey-por-conspiradores}}

\bibverse{25} Y vivió Amasías hijo de Joas, rey de Judá, quince años
después de la muerte de Joas hijo de Joachâz rey de Israel.
\bibverse{26} Lo demás de los hechos de Amasías, primeros y postreros,
¿no está escrito en el libro de los reyes de Judá y de Israel?
\bibverse{27} Desde aquel tiempo que Amasías se apartó de Jehová,
maquinaron contra él conjuración en Jerusalem; y habiendo él huído á
Lachîs, enviaron tras él á Lachîs, y allá lo mataron; \footnote{\textbf{25:27}
  2Cró 24,25} \bibverse{28} Y trajéronlo en caballos, y sepultáronlo con
sus padres en la ciudad de Judá.

\hypertarget{el-gobierno-del-rey-ussia-buen-comienzo-para-el-gobierno-la-felicidad-de-ussia-en-la-guerra-y-la-paz}{%
\subsection{El gobierno del rey Ussia; Buen comienzo para el gobierno;
La felicidad de Ussia en la guerra y la
paz}\label{el-gobierno-del-rey-ussia-buen-comienzo-para-el-gobierno-la-felicidad-de-ussia-en-la-guerra-y-la-paz}}

\hypertarget{section-25}{%
\section{26}\label{section-25}}

\bibverse{1} Entonces todo el pueblo de Judá tomó á Uzzías, el cual era
de diez y seis años, y pusiéronlo por rey en lugar de Amasías su padre.
\bibverse{2} Edificó él á Eloth, y la restituyó á Judá después que el
rey durmió con sus padres. \bibverse{3} De diez y seis años era Uzzías
cuando comenzó á reinar, y cincuenta y dos años reinó en Jerusalem. El
nombre de su madre fué Jechôlía, de Jerusalem. \bibverse{4} E hizo lo
recto en los ojos de Jehová, conforme á todas las cosas que había hecho
Amasías su padre. \footnote{\textbf{26:4} 2Cró 25,2} \bibverse{5} Y
persistió en buscar á Dios en los días de Zachârías, entendido en
visiones de Dios; y en estos días que él buscó á Jehová, él le prosperó.

\bibverse{6} Y salió, y peleó contra los Filisteos, y rompió el muro de
Gath, y el muro de Jabnia, y el muro de Asdod; y edificó ciudades en
Asdod, y en la tierra de los Filisteos. \bibverse{7} Y dióle Dios ayuda
contra los Filisteos, y contra los Arabes que habitaban en Gur-baal, y
contra los Ammonitas. \bibverse{8} Y dieron los Ammonitas presentes á
Uzzías, y divulgóse su nombre hasta la entrada de Egipto; porque se
había hecho altamente poderoso. \bibverse{9} Edificó también Uzzías
torres en Jerusalem, junto á la puerta del ángulo, y junto á la puerta
del valle, y junto á las esquinas; y fortificólas. \bibverse{10}
Asimismo edificó torres en el desierto, y abrió muchas cisternas: porque
tuvo muchos ganados, así en los valles como en las vegas; y viñas, y
labranzas, así en los montes como en los llanos fértiles; porque era
amigo de la agricultura.

\hypertarget{la-preocupaciuxf3n-de-ussia-por-un-ejuxe9rcito-capaz-y-por-la-seguridad-del-pauxeds}{%
\subsection{La preocupación de Ussia por un ejército capaz y por la
seguridad del
país}\label{la-preocupaciuxf3n-de-ussia-por-un-ejuxe9rcito-capaz-y-por-la-seguridad-del-pauxeds}}

\bibverse{11} Tuvo también Uzzías escuadrones de guerreros, los cuales
salían á la guerra en ejército, según que estaban por lista hecha por
mano de Jehiel escriba y de Maasías gobernador, y por mano de Hananías,
uno de los príncipes del rey. \bibverse{12} Todo el número de los jefes
de familias, valientes y esforzados, era dos mil y seiscientos.
\bibverse{13} Y bajo la mano de éstos estaba el ejército de guerra, de
trescientos siete mil y quinientos guerreros poderosos y fuertes para
ayudar al rey contra los enemigos. \bibverse{14} Y aprestóles Uzzías
para todo el ejército, escudos, lanzas, almetes, coseletes, arcos, y
hondas de tirar piedras. \bibverse{15} E hizo en Jerusalem máquinas por
industria de ingenieros, para que estuviesen en las torres y en los
baluartes, para arrojar saetas y grandes piedras, y su fama se extendió
lejos, porque se ayudó maravillosamente, hasta hacerse fuerte.

\hypertarget{la-invasiuxf3n-de-ussia-al-sacerdocio-es-castigada-por-dios-con-lepra}{%
\subsection{La invasión de Ussia al sacerdocio es castigada por Dios con
lepra}\label{la-invasiuxf3n-de-ussia-al-sacerdocio-es-castigada-por-dios-con-lepra}}

\bibverse{16} Mas cuando fué fortificado, su corazón se enalteció hasta
corromperse; porque se rebeló contra Jehová su Dios, entrando en el
templo de Jehová para quemar sahumerios en el altar del perfume.
\bibverse{17} Y entró tras él el sacerdote Azarías, y con él ochenta
sacerdotes de Jehová, de los valientes. \bibverse{18} Y pusiéronse
contra el rey Uzzías, y dijéronle: No á ti, oh Uzzías, el quemar perfume
á Jehová, sino á los sacerdotes hijos de Aarón, que son consagrados para
quemarlo: sal del santuario, por que has prevaricado, y no te será para
gloria delante del Dios Jehová. \footnote{\textbf{26:18} Núm 18,7}

\bibverse{19} Y airóse Uzzías, que tenía el perfume en la mano para
quemarlo; y en esta su ira contra los sacerdotes, la lepra le salió en
la frente delante de los sacerdotes en la casa de Jehová, junto al altar
del perfume. \bibverse{20} Y miróle Azarías el sumo sacerdote, y todos
los sacerdotes, y he aquí la lepra estaba en su frente; é hiciéronle
salir apriesa de aquel lugar; y él también se dió priesa á salir, porque
Jehová lo había herido. \bibverse{21} Así el rey Uzzías fué leproso
hasta el día de su muerte, y habitó en una casa apartada, leproso, por
lo que había sido separado de la casa de Jehová; y Joatham su hijo tuvo
cargo de la casa real, gobernando al pueblo de la tierra.

\hypertarget{muerte-y-entierro-de-ussia}{%
\subsection{Muerte y entierro de
Ussia}\label{muerte-y-entierro-de-ussia}}

\bibverse{22} Lo demás de los hechos de Uzzías, primeros y postreros,
escribiólo Isaías profeta, hijo de Amós. \footnote{\textbf{26:22} 2Re
  15,5-7; Is 1,1; Is 6,1} \bibverse{23} Y durmió Uzzías con sus padres,
y sepultáronlo con sus padres en el campo de los sepulcros reales;
porque dijeron: Leproso es. Y reinó Joatham su hijo en lugar suyo.

\hypertarget{el-gobierno-del-rey-jotam-gobierno-bueno-y-feliz-edificios-y-guerras-exitosas}{%
\subsection{El gobierno del rey Jotam; Gobierno bueno y feliz; Edificios
y guerras
exitosas}\label{el-gobierno-del-rey-jotam-gobierno-bueno-y-feliz-edificios-y-guerras-exitosas}}

\hypertarget{section-26}{%
\section{27}\label{section-26}}

\bibverse{1} De veinticinco años era Joatham cuando comenzó á reinar, y
dieciséis años reinó en Jerusalem. El nombre de su madre fué Jerusa,
hija de Sadoc. \bibverse{2} E hizo lo recto en ojos de Jehová, conforme
á todas las cosas que había hecho Uzzías su padre, salvo que no entró en
el templo de Jehová. Y el pueblo falseaba aún. \bibverse{3} Edificó él
la puerta mayor de la casa de Jehová, y en el muro de la fortaleza
edificó mucho. \bibverse{4} Además edificó ciudades en las montañas de
Judá, y labró palacios y torres en los bosques. \footnote{\textbf{27:4}
  2Cró 26,10} \bibverse{5} También tuvo él guerra con el rey de los
hijos de Ammón, á los cuales venció; y diéronle los hijos de Ammón en
aquel año cien talentos de plata, y diez mil coros de trigo, y diez mil
de cebada. Esto le dieron los hijos de Ammón, y lo mismo en el segundo
año, y en el tercero. \bibverse{6} Así que Joatham fué fortificado,
porque preparó sus caminos delante de Jehová su Dios. \bibverse{7} Lo
demás de los hechos de Joatham, y todas sus guerras, y sus caminos, he
aquí está escrito en el libro de los reyes de Israel y de Judá.
\bibverse{8} Cuando comenzó á reinar era de veinticinco años, y
dieciséis reinó en Jerusalem. \bibverse{9} Y durmió Joatham con sus
padres, y sepultáronlo en la ciudad de David; y reinó en su lugar Achâz
su hijo.

\hypertarget{el-reinado-del-rey-acaz-las-abominaciones-paganas-de-acaz}{%
\subsection{El reinado del rey Acaz; Las abominaciones paganas de
Acaz}\label{el-reinado-del-rey-acaz-las-abominaciones-paganas-de-acaz}}

\hypertarget{section-27}{%
\section{28}\label{section-27}}

\bibverse{1} De veinte años era Achâz cuando comenzó á reinar, y
dieciséis años reinó en Jerusalem: mas no hizo lo recto en ojos de
Jehová, como David su padre. \bibverse{2} Antes anduvo en los caminos de
los reyes de Israel, y además hizo imágenes de fundición á los Baales.
\bibverse{3} Quemó también perfume en el valle de los hijos de Hinnom, y
quemó sus hijos por fuego, conforme á las abominaciones de las gentes
que Jehová había echado delante de los hijos de Israel. \footnote{\textbf{28:3}
  Deut 18,9-10; Deut 18,12} \bibverse{4} Asimismo sacrificó y quemó
perfumes en los altos, y en los collados, y debajo de todo árbol espeso.

\footnote{\textbf{28:4} 1Re 14,23}

\hypertarget{visitaciones-severas-de-sirios-e-israelitas}{%
\subsection{Visitaciones severas de sirios e
israelitas}\label{visitaciones-severas-de-sirios-e-israelitas}}

\bibverse{5} Por lo cual Jehová su Dios lo entregó en manos del rey de
los Siros, los cuales le derrotaron, y cogieron de él una grande presa,
que llevaron á Damasco. Fué también entregado en manos del rey de
Israel, el cual lo batió con gran mortandad. \bibverse{6} Porque Peca,
hijo de Remalías mató en Judá en un día ciento y veinte mil, todos
hombres valientes; por cuanto habían dejado á Jehová el Dios de sus
padres. \bibverse{7} Asimismo Zichri, hombre poderoso de Ephraim, mató á
Maasías hijo del rey, y á Azricam su mayordomo, y á Elcana, segundo
después del rey. \bibverse{8} Tomaron también cautivos los hijos de
Israel de sus hermanos doscientos mil, mujeres, muchachos, y muchachas,
á más de haber saqueado de ellos un gran despojo, el cual trajeron á
Samaria.

\hypertarget{liberaciuxf3n-de-los-prisioneros-de-guerra-de-judea-en-samaria-siguiendo-la-amonestaciuxf3n-del-profeta-oded}{%
\subsection{Liberación de los prisioneros de guerra de Judea en Samaria
siguiendo la amonestación del profeta
Oded}\label{liberaciuxf3n-de-los-prisioneros-de-guerra-de-judea-en-samaria-siguiendo-la-amonestaciuxf3n-del-profeta-oded}}

\bibverse{9} Había entonces allí un profeta de Jehová, que se llamaba
Obed, el cual salió delante del ejército cuando entraba en Samaria, y
díjoles: He aquí Jehová el Dios de vuestros padres, por el enojo contra
Judá, los ha entregado en vuestras manos; y vosotros los habéis muerto
con ira, que hasta el cielo ha llegado. \footnote{\textbf{28:9} Gén
  18,21; Esd 9,6} \bibverse{10} Y ahora habéis determinado sujetar á
vosotros á Judá y á Jerusalem por siervos y siervas: mas ¿no habéis
vosotros pecado contra Jehová vuestro Dios? \bibverse{11} Oidme pues
ahora, y volved á enviar los cautivos que habéis tomado de vuestros
hermanos: porque Jehová está airado contra vosotros. \bibverse{12}
Levantáronse entonces algunos varones de los principales de los hijos de
Ephraim, Azarías hijo de Johanán, y Berechîas hijo de Mesillemoth, y
Ezechîas hijo de Sallum, y Amasa hijo de Hadlai, contra los que venían
de la guerra. \bibverse{13} Y dijéronles: No metáis acá la cautividad;
porque el pecado contra Jehová será sobre nosotros. Vosotros tratáis de
añadir sobre nuestros pecados y sobre nuestras culpas, siendo asaz
grande nuestro delito, y la ira del furor sobre Israel.

\bibverse{14} Entonces el ejército dejó los cautivos y la presa delante
de los príncipes y de toda la multitud. \bibverse{15} Y levantáronse los
varones nombrados, y tomaron los cautivos, y vistieron del despojo á los
que de ellos estaban desnudos; vistiéronlos y calzáronlos, y diéronles
de comer y de beber, y ungiéronlos, y condujeron en asnos á todos los
flacos, y lleváronlos hasta Jericó, ciudad de las palmas, cerca de sus
hermanos; y ellos se volvieron á Samaria.

\hypertarget{fuertes-visitaciones-de-los-edomitas-filisteos-y-asirios}{%
\subsection{Fuertes visitaciones de los edomitas, filisteos y
asirios}\label{fuertes-visitaciones-de-los-edomitas-filisteos-y-asirios}}

\bibverse{16} En aquel tiempo envió á pedir el rey Achâz á los reyes de
Asiria que le ayudasen: \bibverse{17} Porque á más de esto, los Idumeos
habían venido y herido á los de Judá, y habían llevado cautivos.
\bibverse{18} Asimismo los Filisteos se habían derramado por las
ciudades de la llanura, y al mediodía de Judá, y habían tomado á
Beth-semes, á Ajalón, Gederoth, y Sochô con sus aldeas, Timna también
con sus aldeas, y Gimzo con sus aldeas; y habitaban en ellas.
\bibverse{19} Porque Jehová había humillado á Judá por causa de Achâz
rey de Israel: por cuanto él había desnudado á Judá, y rebeládose
gravemente contra Jehová. \bibverse{20} Y vino contra él
Tilgath-pilneser, rey de los Asirios: pues lo redujo á estrechez, y no
lo fortificó. \bibverse{21} Aunque despojó Achâz la casa de Jehová, y la
casa real, y las de los príncipes, para dar al rey de los Asirios, con
todo eso él no le ayudó.

\hypertarget{la-creciente-maldad-de-acaz-palabra-final}{%
\subsection{La creciente maldad de Acaz; Palabra
final}\label{la-creciente-maldad-de-acaz-palabra-final}}

\bibverse{22} Además el rey Achâz en el tiempo que aquél le apuraba,
añadió prevaricación contra Jehová; \bibverse{23} Porque sacrificó á los
dioses de Damasco que le habían herido, y dijo: Pues que los dioses de
los reyes de Siria les ayudan, yo también sacrificaré á ellos para que
me ayuden; bien que fueron éstos su ruina, y la de todo Israel.
\bibverse{24} A más de eso recogió Achâz los vasos de la casa de Dios, y
quebrólos, y cerró las puertas de la casa de Jehová, é hízose altares en
Jerusalem en todos los rincones. \bibverse{25} Hizo también altos en
todas las ciudades de Judá, para quemar perfumes á los dioses ajenos,
provocando así á ira á Jehová el Dios de sus padres.

\bibverse{26} Lo demás de sus hechos, y todos sus caminos, primeros y
postreros, he aquí ello está escrito en el libro de los reyes de Judá y
de Israel. \bibverse{27} Y durmió Achâz con sus padres, y sepultáronlo
en la ciudad de Jerusalem: mas no le metieron en los sepulcros de los
reyes de Israel; y reinó en su lugar Ezechîas su hijo. \footnote{\textbf{28:27}
  2Cró 21,20}

\hypertarget{el-gobierno-del-rey-ezechuxeeas-restauraciuxf3n-del-templo-y-adoraciuxf3n-pura}{%
\subsection{El gobierno del rey Ezechîas; Restauración del templo y
adoración
pura}\label{el-gobierno-del-rey-ezechuxeeas-restauraciuxf3n-del-templo-y-adoraciuxf3n-pura}}

\hypertarget{section-28}{%
\section{29}\label{section-28}}

\bibverse{1} Y ezechîas comenzó á reinar siendo de veinticinco años, y
reinó veintinueve años en Jerusalem. El nombre de su madre fué Abía,
hija de Zachârías. \footnote{\textbf{29:1} 2Re 18,1-3} \bibverse{2} E
hizo lo recto en ojos de Jehová, conforme á todas las cosas que había
hecho David su padre.

\hypertarget{la-exhortaciuxf3n-de-ezechuxeeas-a-los-sacerdotes-y-levitas}{%
\subsection{La exhortación de Ezechîas a los sacerdotes y
levitas}\label{la-exhortaciuxf3n-de-ezechuxeeas-a-los-sacerdotes-y-levitas}}

\bibverse{3} En el primer año de su reinado, en el mes primero, abrió
las puertas de la casa de Jehová, y las reparó. \bibverse{4} E hizo
venir los sacerdotes y Levitas, y juntólos en la plaza oriental.
\bibverse{5} Y díjoles: Oidme, Levitas, y santificaos ahora, y
santificaréis la casa de Jehová el Dios de vuestros padres, y sacaréis
del santuario la inmundicia. \bibverse{6} Porque nuestros padres se han
rebelado, y han hecho lo malo en ojos de Jehová nuestro Dios; que le
dejaron, y apartaron sus ojos del tabernáculo de Jehová, y le volvieron
las espaldas. \bibverse{7} Y aun cerraron las puertas del pórtico, y
apagaron las lámparas; no quemaron perfume, ni sacrificaron holocausto
en el santuario al Dios de Israel. \footnote{\textbf{29:7} 2Cró 28,24}
\bibverse{8} Por tanto la ira de Jehová ha venido sobre Judá y
Jerusalem, y los ha entregado á turbación, y á execración y escarnio,
como veis vosotros con vuestros ojos. \bibverse{9} Y he aquí nuestros
padres han caído á cuchillo, nuestros hijos y nuestras hijas y nuestras
mujeres son cautivas por esto. \bibverse{10} Ahora pues, yo he
determinado hacer alianza con Jehová el Dios de Israel, para que aparte
de nosotros la ira de su furor. \bibverse{11} Hijos míos, no os engañéis
ahora, porque Jehová os ha escogido á vosotros para que estéis delante
de él, y le sirváis, y seáis sus ministros, y le queméis perfume.

\hypertarget{purificaciuxf3n-del-templo-por-los-levitas}{%
\subsection{Purificación del templo por los
levitas}\label{purificaciuxf3n-del-templo-por-los-levitas}}

\bibverse{12} Entonces los Levitas se levantaron, Mahath hijo de Amasai,
y Joel hijo de Azarías, de los hijos de Coath; y de los hijos de Merari,
Cis hijo de Abdi, y Azarías hijo de Jehaleleel; y de los hijos de
Gersón, Joah hijo de Zimma, y Edén hijo de Joah; \bibverse{13} Y de los
hijos de Elisaphán, Simri y Jehiel; y de los hijos de Asaph, Zachârías y
Mathanías; \bibverse{14} Y de los hijos de Hemán, Jehiel y Simi; y de
los hijos de Jeduthún, Semeías y Uzziel. \bibverse{15} Estos juntaron á
sus hermanos, y santificáronse, y entraron, conforme al mandamiento del
rey y las palabras de Jehová, para limpiar la casa de Jehová.
\bibverse{16} Y entrando los sacerdotes dentro de la casa de Jehová para
limpiarla, sacaron toda la inmundicia que hallaron en el templo de
Jehová, al atrio de la casa de Jehová; la cual tomaron los Levitas, para
sacarla fuera al torrente de Cedrón. \bibverse{17} Y comenzaron á
santificar el primero del mes primero, y á los ocho del mismo mes
vinieron al pórtico de Jehová: y santificaron la casa de Jehová en ocho
días, y en el dieciséis del mes primero acabaron. \bibverse{18} Luego
pasaron al rey Ezechîas, y dijéronle: Ya hemos limpiado toda la casa de
Jehová, el altar del holocausto, y todos sus instrumentos, y la mesa de
la proposición con todos sus utensilios. \bibverse{19} Asimismo hemos
preparado y santificado todos los vasos que en su prevaricación había
maltratado el rey Achâz, cuando reinaba: y he aquí están delante del
altar de Jehová.

\hypertarget{la-nueva-consagraciuxf3n-del-templo-con-sacrificios-oraciuxf3n-y-cuxe1nticos}{%
\subsection{La nueva consagración del templo con sacrificios, oración y
cánticos}\label{la-nueva-consagraciuxf3n-del-templo-con-sacrificios-oraciuxf3n-y-cuxe1nticos}}

\bibverse{20} Y levantándose de mañana el rey Ezechîas reunió los
principales de la ciudad, y subió á la casa de Jehová. \bibverse{21} Y
presentaron siete novillos, siete carneros, siete corderos, y siete
machos de cabrío, para expiación por el reino, por el santuario y por
Judá. Y dijo á los sacerdotes hijos de Aarón, que los ofreciesen sobre
el altar de Jehová. \bibverse{22} Mataron pues los bueyes, y los
sacerdotes tomaron la sangre, y esparciéronla sobre el altar; mataron
luego los carneros, y esparcieron la sangre sobre el altar; asimismo
mataron los corderos, y esparcieron la sangre sobre el altar.
\bibverse{23} Hicieron después llegar los machos cabríos de la expiación
delante del rey y de la multitud, y pusieron sobre ellos sus manos:
\bibverse{24} Y los sacerdotes los mataron, y expiando esparcieron la
sangre de ellos sobre el altar, para reconciliar á todo Israel: porque
por todo Israel mandó el rey hacer el holocausto y la expiación.

\bibverse{25} Puso también Levitas en la casa de Jehová con címbalos, y
salterios, y arpas, conforme al mandamiento de David, y de Gad vidente
del rey, y de Nathán profeta: porque aquel mandamiento fué por mano de
Jehová, por mano de sus profetas. \footnote{\textbf{29:25} 1Cró 25,1}
\bibverse{26} Y los Levitas estaban con los instrumentos de David, y los
sacerdotes con trompetas. \bibverse{27} Entonces mandó Ezechîas
sacrificar el holocausto en el altar; y al tiempo que comenzó el
holocausto, comenzó también el cántico de Jehová, con las trompetas y
los instrumentos de David rey de Israel. \bibverse{28} Y toda la
multitud adoraba, y los cantores cantaban, y los trompetas sonaban las
trompetas; todo hasta acabarse el holocausto.

\bibverse{29} Y como acabaron de ofrecer, inclinóse el rey, y todos los
que con él estaban, y adoraron. \bibverse{30} Entonces el rey Ezechîas y
los príncipes dijeron á los Levitas que alabasen á Jehová por las
palabras de David y de Asaph vidente: y ellos alabaron con grande
alegría, é inclinándose adoraron.

\bibverse{31} Y respondiendo Ezechîas dijo: Vosotros os habéis
consagrado ahora á Jehová; llegaos pues, y presentad sacrificios y
alabanzas en la casa de Jehová. Y la multitud presentó sacrificios y
alabanzas; y todo liberal de corazón, holocaustos. \bibverse{32} Y fué
el número de los holocaustos que trajo la congregación, setenta bueyes,
cien carneros, doscientos corderos; todo para el holocausto de Jehová.
\bibverse{33} Y las ofrendas fueron seiscientos bueyes, y tres mil
ovejas. \bibverse{34} Mas los sacerdotes eran pocos, y no podían bastar
á desollar los holocaustos; y así sus hermanos los Levitas les ayudaron
hasta que acabaron la obra, y hasta que los sacerdotes se santificaron:
porque los Levitas tuvieron mayor prontitud de corazón para
santificarse, que los sacerdotes. \footnote{\textbf{29:34} 2Cró 30,3;
  2Cró 30,16-17} \bibverse{35} Así pues hubo gran multitud de
holocaustos, con sebos de pacíficos, y libaciones de cada holocausto. Y
quedó ordenado el servicio de la casa de Jehová. \footnote{\textbf{29:35}
  Lev 3,16-17; Núm 15,5; Núm 15,7; Núm 15,10} \bibverse{36} Y alegróse
Ezechîas, y todo el pueblo, de que Dios hubiese preparado el pueblo;
porque la cosa fué prestamente hecha.

\hypertarget{celebraciuxf3n-de-la-pascua-de-ezechuxeeas}{%
\subsection{Celebración de la Pascua de
Ezechîas}\label{celebraciuxf3n-de-la-pascua-de-ezechuxeeas}}

\hypertarget{section-29}{%
\section{30}\label{section-29}}

\bibverse{1} Envió también Ezechîas por todo Israel y Judá, y escribió
letras á Ephraim y Manasés, que viniesen á Jerusalem á la casa de
Jehová, para celebrar la pascua á Jehová Dios de Israel. \footnote{\textbf{30:1}
  2Cró 35,1} \bibverse{2} Y había el rey tomado consejo con sus
príncipes, y con toda la congregación en Jerusalem, para celebrar la
pascua en el mes segundo: \footnote{\textbf{30:2} 2Cró 30,15}
\bibverse{3} Porque entonces no la podían celebrar, por cuanto no había
suficientes sacerdotes santificados, ni el pueblo estaba junto en
Jerusalem. \bibverse{4} Esto agradó al rey y á toda la multitud.
\bibverse{5} Y determinaron hacer pasar pregón por todo Israel, desde
Beer-seba hasta Dan, para que viniesen á celebrar la pascua á Jehová
Dios de Israel, en Jerusalem: porque en mucho tiempo no la habían
celebrado al modo que está escrito.

\bibverse{6} Fueron pues correos con letras de mano del rey y de sus
príncipes por todo Israel y Judá, como el rey lo había mandado, y
decían: Hijos de Israel, volveos á Jehová el Dios de Abraham, de Isaac,
y de Israel, y él se volverá á las reliquias que os han quedado de la
mano de los reyes de Asiria. \bibverse{7} No seáis como vuestros padres
y como vuestros hermanos, que se rebelaron contra Jehová el Dios de sus
padres, y él los entregó á desolación, como vosotros veis. \bibverse{8}
No endurezcáis pues ahora vuestra cerviz como vuestros padres: dad la
mano á Jehová, y venid á su santuario, el cual él ha santificado para
siempre; y servid á Jehová vuestro Dios, y la ira de su furor se
apartará de vosotros. \bibverse{9} Porque si os volviereis á Jehová,
vuestros hermanos y vuestros hijos hallarán misericordia delante de los
que los tienen cautivos, y volverán á esta tierra: porque Jehová vuestro
Dios es clemente y misericordioso, y no volverá de vosotros su rostro,
si vosotros os volviereis á él.

\bibverse{10} Pasaron pues los correos de ciudad en ciudad por la tierra
de Ephraim y Manasés, hasta Zabulón: mas se reían y burlaban de ellos.
\bibverse{11} Con todo eso, algunos hombres de Aser, de Manasés, y de
Zabulón, se humillaron, y vinieron á Jerusalem. \bibverse{12} En Judá
también fué la mano de Dios para darles un corazón para cumplir el
mensaje del rey y de los príncipes, conforme á la palabra de Jehová.

\hypertarget{curso-de-pascua-en-la-primera-semana}{%
\subsection{Curso de Pascua en la primera
semana}\label{curso-de-pascua-en-la-primera-semana}}

\bibverse{13} Y juntóse en Jerusalem mucha gente para celebrar la
solemnidad de los ázimos en el mes segundo; una vasta reunión.
\bibverse{14} Y levantándose, quitaron los altares que había en
Jerusalem; quitaron también todos los altares de perfumes, y echáronlos
en el torrente de Cedrón. \bibverse{15} Entonces sacrificaron la pascua,
á los catorce del mes segundo; y los sacerdotes y los Levitas se
santificaron con vergüenza, y trajeron los holocaustos á la casa de
Jehová. \footnote{\textbf{30:15} Núm 9,11} \bibverse{16} Y pusiéronse en
su orden conforme á su costumbre, conforme á la ley de Moisés varón de
Dios; los sacerdotes esparcían la sangre que recibían de manos de los
Levitas: \footnote{\textbf{30:16} 2Cró 29,34} \bibverse{17} Porque había
muchos en la congregación que no estaban santificados, y por eso los
Levitas sacrificaban la pascua por todos los que no se habían limpiado,
para santificarlos á Jehová. \bibverse{18} Porque una gran multitud del
pueblo de Ephraim y Manasés, y de Issachâr y Zabulón, no se habían
purificado, y comieron la pascua no conforme á lo que está escrito. Mas
Ezechîas oró por ellos, diciendo: Jehová, que es bueno, sea propicio á
todo aquel que ha apercibido su corazón para buscar á Dios, \footnote{\textbf{30:18}
  Éxod 12,-1} \bibverse{19} A Jehová el Dios de sus padres, aunque no
esté purificado según la purificación del santuario.

\bibverse{20} Y oyó Jehová á Ezechîas, y sanó al pueblo. \bibverse{21}
Así celebraron los hijos de Israel que se hallaron en Jerusalem, la
solemnidad de los panes sin levadura por siete días con grande gozo: y
alababan á Jehová todos los días los Levitas y los sacerdotes, cantando
con instrumentos de fortaleza á Jehová. \bibverse{22} Y habló Ezechîas
al corazón de todos los Levitas que tenían buena inteligencia en el
servicio de Jehová. Y comieron de lo sacrificado en la solemnidad por
siete días, ofreciendo sacrificios pacíficos, y dando gracias á Jehová
el Dios de sus padres.

\hypertarget{continuaciuxf3n-de-la-celebraciuxf3n-en-la-segunda-semana}{%
\subsection{Continuación de la celebración en la segunda
semana}\label{continuaciuxf3n-de-la-celebraciuxf3n-en-la-segunda-semana}}

\bibverse{23} Y toda aquella multitud determinó que celebrasen otros
siete días; y celebraron otros siete días con alegría. \bibverse{24}
Porque Ezechîas rey de Judá había dado á la multitud mil novillos y
siete mil ovejas; y también los príncipes dieron al pueblo mil novillos
y diez mil ovejas: y muchos sacerdotes se santificaron. \footnote{\textbf{30:24}
  2Cró 35,7} \bibverse{25} Alegróse pues toda la congregación de Judá,
como también los sacerdotes y Levitas, y toda la multitud que había
venido de Israel; asimismo los extranjeros que habían venido de la
tierra de Israel, y los que habitaban en Judá. \bibverse{26} E
hiciéronse grandes alegrías en Jerusalem: porque desde los días de
Salomón hijo de David rey de Israel, no había habido cosa tal en
Jerusalem. \bibverse{27} Levantándose después los sacerdotes y Levitas,
bendijeron al pueblo: y la voz de ellos fué oída, y su oración llegó á
la habitación de su santuario, al cielo.

\hypertarget{limpiando-la-tierra-de-la-idolatruxeda}{%
\subsection{Limpiando la tierra de la
idolatría}\label{limpiando-la-tierra-de-la-idolatruxeda}}

\hypertarget{section-30}{%
\section{31}\label{section-30}}

\bibverse{1} Hechas todas estas cosas, todos los de Israel que se habían
hallado allí, salieron por las ciudades de Judá, y quebraron las
estatuas y destruyeron los bosques, y derribaron los altos y los altares
por todo Judá y Benjamín, y también en Ephraim y Manasés, hasta acabarlo
todo. Después volviéronse todos los hijos de Israel, cada uno á su
posesión y á sus ciudades.

\hypertarget{cuidado-exitoso-de-los-ingresos-de-los-sacerdotes-y-levitas}{%
\subsection{Cuidado exitoso de los ingresos de los sacerdotes y
levitas}\label{cuidado-exitoso-de-los-ingresos-de-los-sacerdotes-y-levitas}}

\bibverse{2} Y arregló Ezechîas los repartimientos de los sacerdotes y
de los Levitas conforme á sus órdenes, cada uno según su oficio, los
sacerdotes y los Levitas para el holocausto y pacíficos, para que
ministrasen, para que confesasen y alabasen á las puertas de los reales
de Jehová. \bibverse{3} La contribución del rey de su hacienda, era
holocaustos á mañana y tarde, y holocaustos para los sábados, nuevas
lunas, y solemnidades, como está escrito en la ley de Jehová.
\footnote{\textbf{31:3} Núm 28,-1; Núm 29,1-29} \bibverse{4} Mandó
también al pueblo que habitaba en Jerusalem, que diesen la porción á los
sacerdotes y Levitas, para que se esforzasen en la ley de Jehová.
\bibverse{5} Y como este edicto fué divulgado, los hijos de Israel
dieron muchas primicias de grano, vino, aceite, miel, y de todos los
frutos de la tierra: trajeron asimismo los diezmos de todas las cosas en
abundancia. \bibverse{6} También los hijos de Israel y de Judá, que
habitaban en las ciudades de Judá, dieron del mismo modo los diezmos de
las vacas y de las ovejas: y trajeron los diezmos de lo santificado, de
las cosas que habían prometido á Jehová su Dios, y pusiéronlos por
montones.

\bibverse{7} En el mes tercero comenzaron á fundar aquellos montones, y
en el mes séptimo acabaron. \bibverse{8} Y Ezechîas y los príncipes
vinieron á ver los montones, y bendijeron á Jehová, y á su pueblo
Israel. \bibverse{9} Y preguntó Ezechîas á los sacerdotes y á los
Levitas acerca de los montones. \bibverse{10} Y respondióle Azarías,
sumo sacerdote, de la casa de Sadoc, y dijo: Desde que comenzaron á
traer la ofrenda á la casa de Jehová, hemos comido y saciádonos, y nos
ha sobrado mucho: porque Jehová ha bendecido su pueblo, y ha quedado
esta muchedumbre.

\bibverse{11} Entonces mandó Ezechîas que preparasen cámaras en la casa
de Jehová; y preparáronlas. \bibverse{12} Y metieron las primicias y
diezmos y las cosas consagradas, fielmente; y dieron cargo de ello á
Chônanías Levita, el principal, y Simi su hermano fué el segundo.
\bibverse{13} Y Jehiel, Azazías, Nahath, Asael, Jerimoth, Josabad,
Eliel, Ismachîas, Mahaath, y Benaías, fueron sobrestantes bajo la mano
de Chônanías y de Simi su hermano, por mandamiento del rey Ezechîas y de
Azarías, príncipe de la casa de Dios. \bibverse{14} Y Coré hijo de Imna
Levita, portero al oriente, tenía cargo de las limosnas de Dios, y de
las ofrendas de Jehová que se daban, y de todo lo que se santificaba.
\bibverse{15} Y á su mano estaba Edén, Benjamín, Jeshua, Semaías,
Amarías, y Sechânías, en las ciudades de los sacerdotes, para dar con
fidelidad á sus hermanos sus partes conforme á sus órdenes, así al mayor
como al menor: \bibverse{16} A más de los varones anotados por sus
linajes, de tres años arriba, á todos los que entraban en la casa de
Jehová, su porción diaria por su ministerio, según sus oficios y clases;

\hypertarget{elaboraciuxf3n-de-listas-de-sacerdotes-y-levitas-palabra-final}{%
\subsection{Elaboración de listas de sacerdotes y levitas; Palabra
final}\label{elaboraciuxf3n-de-listas-de-sacerdotes-y-levitas-palabra-final}}

\bibverse{17} También á los que eran contados entre los sacerdotes por
las familias de sus padres, y á los Levitas de edad de veinte años
arriba, conforme á sus oficios y órdenes; \bibverse{18} Asimismo á los
de su generación con todos sus niños, y sus mujeres, y sus hijos é
hijas, á toda la familia; porque con fidelidad se consagraban á las
cosas santas. \bibverse{19} Del mismo modo en orden á los hijos de
Aarón, sacerdotes, que estaban en los ejidos de sus ciudades, por todas
las ciudades, los varones nombrados tenían cargo de dar sus porciones á
todos los varones de los sacerdotes, y á todo el linaje de los Levitas.

\bibverse{20} De esta manera hizo Ezechîas en todo Judá: y ejecutó lo
bueno, recto, y verdadero, delante de Jehová su Dios. \bibverse{21} En
todo cuanto comenzó en el servicio de la casa de Dios, y en la ley y
mandamientos, buscó á su Dios, é hízolo de todo corazón, y fué
prosperado. \footnote{\textbf{31:21} Sal 1,3}

\hypertarget{la-incursiuxf3n-de-senaquerib-y-el-resto-de-ezechuxeeas}{%
\subsection{La incursión de Senaquerib y el resto de
Ezechîas}\label{la-incursiuxf3n-de-senaquerib-y-el-resto-de-ezechuxeeas}}

\hypertarget{section-31}{%
\section{32}\label{section-31}}

\bibverse{1} Después de estas cosas y de esta fidelidad, vino
Sennachêrib rey de los Asirios, entró en Judá, y asentó campo contra las
ciudades fuertes, y determinó de entrar en ellas. \footnote{\textbf{32:1}
  2Cró 31,20} \bibverse{2} Viendo pues Ezechîas la venida de
Sennachêrib, y su aspecto de combatir á Jerusalem, \bibverse{3} Tuvo su
consejo con sus príncipes y con sus valerosos, sobre cegar las fuentes
de las aguas que estaban fuera de la ciudad; y ellos le apoyaron.
\bibverse{4} Juntóse pues mucho pueblo, y cegaron todas las fuentes, y
el arroyo que derrama por en medio del territorio, diciendo: ¿Por qué
han de hallar los reyes de Asiria muchas aguas cuando vinieren?

\bibverse{5} Alentóse así Ezechîas, y edificó todos los muros caídos, é
hizo alzar las torres, y otro muro por de fuera: fortificó además á
Millo en la ciudad de David, é hizo muchas espadas y paveses.
\footnote{\textbf{32:5} 2Cró 25,23} \bibverse{6} Y puso capitanes de
guerra sobre el pueblo, é hízolos reunir así en la plaza de la puerta de
la ciudad, y hablóles al corazón de ellos, diciendo: \footnote{\textbf{32:6}
  2Cró 30,22} \bibverse{7} Esforzaos y confortaos; no temáis, ni hayáis
miedo del rey de Asiria, ni de toda su multitud que con él viene; porque
más son con nosotros que con él. \footnote{\textbf{32:7} 2Re 6,16}
\bibverse{8} Con él es el brazo de carne, mas con nosotros Jehová
nuestro Dios para ayudarnos, y pelear nuestras batallas. Y afirmóse el
pueblo sobre las palabras de Ezechîas rey de Judá.

\footnote{\textbf{32:8} Jer 17,5; Jer 17,7}

\hypertarget{la-solicitud-de-senaquerib-de-entregar-la-ciudad-a-lachis}{%
\subsection{La solicitud de Senaquerib de entregar la ciudad a
Lachis}\label{la-solicitud-de-senaquerib-de-entregar-la-ciudad-a-lachis}}

\bibverse{9} Después de esto Sennachêrib rey de los Asirios, estando él
sobre Lachîs y con él toda su potencia, envió sus siervos á Jerusalem,
para decir á Ezechîas rey de Judá, y á todos los de Judá que estaban en
Jerusalem: \bibverse{10} Así ha dicho Sennachêrib rey de los Asirios:
¿En quién confiáis vosotros para estar cercados en Jerusalem?
\bibverse{11} ¿No os engaña Ezechîas para entregaros á muerte, á hambre,
y á sed, diciendo: Jehová nuestro Dios nos librará de la mano del rey de
Asiria? \bibverse{12} ¿No es Ezechîas el que ha quitado sus altos y sus
altares, y dijo á Judá y á Jerusalem: Delante de este solo altar
adoraréis, y sobre él quemaréis perfume? \bibverse{13} ¿No habéis sabido
lo que yo y mis padres hemos hecho á todos los pueblos de la tierra?
¿Pudieron los dioses de las gentes de las tierras librar su tierra de mi
mano? \bibverse{14} ¿Qué dios hubo de todos los dioses de aquellas
gentes que destruyeron mis padres, que pudiese salvar su pueblo de mis
manos? ¿Por qué podrá vuestro Dios libraros de mi mano? \bibverse{15}
Ahora pues, no os engañe Ezechîas, ni os persuada tal cosa, ni le
creáis; que si ningún dios de todas aquellas naciones y reinos pudo
librar su pueblo de mis manos, y de las manos de mis padres, ¿cuánto
menos vuestro Dios os podrá librar de mi mano?

\hypertarget{senaquerib-y-la-arrogancia-de-sus-embajadores}{%
\subsection{Senaquerib y la arrogancia de sus
embajadores}\label{senaquerib-y-la-arrogancia-de-sus-embajadores}}

\bibverse{16} Y otras cosas hablaron sus siervos contra el Dios Jehová,
y contra su siervo Ezechîas. \bibverse{17} Además de esto escribió
letras en que blasfemaba á Jehová el Dios de Israel, y hablaba contra
él, diciendo: Como los dioses de las gentes de los países no pudieron
librar su pueblo de mis manos, tampoco el Dios de Ezechîas librará al
suyo de mis manos. \bibverse{18} Y clamaron á gran voz en judaico al
pueblo de Jerusalem que estaba en los muros, para espantarlos y ponerles
temor, para tomar la ciudad. \bibverse{19} Y hablaron contra el Dios de
Jerusalem, como contra los dioses de los pueblos de la tierra, obra de
manos de hombres.

\hypertarget{oraciuxf3n-de-ezequuxedas-la-ayuda-de-dios-la-destrucciuxf3n-el-retiro-y-la-muerte-de-senaquerib}{%
\subsection{Oración de Ezequías; La ayuda de Dios: la destrucción, el
retiro y la muerte de
Senaquerib}\label{oraciuxf3n-de-ezequuxedas-la-ayuda-de-dios-la-destrucciuxf3n-el-retiro-y-la-muerte-de-senaquerib}}

\bibverse{20} Mas el rey Ezechîas, y el profeta Isaías hijo de Amós,
oraron por esto, y clamaron al cielo.

\bibverse{21} Y Jehová envió un ángel, el cual hirió á todo valiente y
esforzado, y á los jefes y capitanes en el campo del rey de Asiria.
Volvióse por tanto con vergüenza de rostro á su tierra; y entrando en el
templo de su dios, allí lo mataron á cuchillo los que habían salido de
sus entrañas. \bibverse{22} Así salvó Jehová á Ezechîas y á los
moradores de Jerusalem de las manos de Sennachêrib rey de Asiria, y de
las manos de todos: y preservólos de todas partes. \bibverse{23} Y
muchos trajeron ofrenda á Jehová á Jerusalem, y á Ezechîas rey de Judá,
ricos dones; y fué muy grande delante de todas las gentes después de
esto.

\hypertarget{la-enfermedad-la-arrogancia-y-la-penitencia-de-ezechuxeeas}{%
\subsection{La enfermedad, la arrogancia y la penitencia de
Ezechîas}\label{la-enfermedad-la-arrogancia-y-la-penitencia-de-ezechuxeeas}}

\bibverse{24} En aquel tiempo Ezechîas enfermó de muerte: y oró á
Jehová, el cual le respondió, y dióle una señal. \bibverse{25} Mas
Ezechîas no pagó conforme al bien que le había sido hecho: antes se
enalteció su corazón, y fué la ira contra él, y contra Judá y Jerusalem.
\footnote{\textbf{32:25} 2Cró 26,16} \bibverse{26} Empero Ezechîas,
después de haberse engreído su corazón, se humilló, él y los moradores
de Jerusalem; y no vino sobre ellos la ira de Jehová en los días de
Ezechîas.

\hypertarget{la-riqueza-de-ezechuxeeas-abastecimiento-de-agua-a-jerusaluxe9n-y-tentaciuxf3n-de-la-embajada-de-babilonia}{%
\subsection{La riqueza de Ezechîas; Abastecimiento de agua a Jerusalén y
tentación de la embajada de
Babilonia}\label{la-riqueza-de-ezechuxeeas-abastecimiento-de-agua-a-jerusaluxe9n-y-tentaciuxf3n-de-la-embajada-de-babilonia}}

\bibverse{27} Y tuvo Ezechîas riquezas y gloria mucha en gran manera; é
hízose de tesoros de plata y oro, de piedras preciosas, de aromas, de
escudos, y de todas alhajas de desear; \bibverse{28} Asimismo depósitos
para las rentas del grano, del vino, y aceite; establos para toda suerte
de bestias, y majadas para los ganados. \bibverse{29} Hízose también
ciudades, y hatos de ovejas y de vacas en gran copia; porque Dios le
había dado mucha hacienda. \bibverse{30} Este Ezechîas tapó los
manaderos de las aguas de Gihón la de arriba, y encaminólas abajo al
occidente de la ciudad de David. Y fué prosperado Ezechîas en todo lo
que hizo.

\bibverse{31} Empero en lo de los embajadores de los príncipes de
Babilonia, que enviaron á él para saber del prodigio que había acaecido
en aquella tierra, Dios lo dejó, para probarle, para hacer conocer todo
lo que estaba en su corazón.

\hypertarget{termina-la-historia-de-ezechuxeeas}{%
\subsection{Termina la historia de
Ezechîas}\label{termina-la-historia-de-ezechuxeeas}}

\bibverse{32} Lo demás de los hechos de Ezechîas, y de sus
misericordias, he aquí todo está escrito en la profecía de Isaías
profeta, hijo de Amós, en el libro de los reyes de Judá y de Israel.
\bibverse{33} Y durmió Ezechîas con sus padres, y sepultáronlo en los
más insignes sepulcros de los hijos de David, honrándole en su muerte
todo Judá y los de Jerusalem: y reinó en su lugar Manasés su hijo.

\hypertarget{manasuxe9s-rey-de-juduxe1-idolatruxeda-manasuxe9s}{%
\subsection{Manasés rey de Judá; Idolatría
manasés}\label{manasuxe9s-rey-de-juduxe1-idolatruxeda-manasuxe9s}}

\hypertarget{section-32}{%
\section{33}\label{section-32}}

\bibverse{1} De doce años era Manasés cuando comenzó á reinar, y
cincuenta y cinco años reinó en Jerusalem. \bibverse{2} Mas hizo lo malo
en ojos de Jehová, conforme á las abominaciones de las gentes que había
echado Jehová delante de los hijos de Israel: \footnote{\textbf{33:2}
  Deut 18,9} \bibverse{3} Porque él reedificó los altos que Ezechîas su
padre había derribado, y levantó altares á los Baales, é hizo bosques, y
adoró á todo el ejército de los cielos, y á él sirvió. \footnote{\textbf{33:3}
  2Re 18,4} \bibverse{4} Edificó también altares en la casa de Jehová,
de la cual había dicho Jehová: En Jerusalem será mi nombre
perpetuamente. \footnote{\textbf{33:4} Deut 12,5; Deut 12,11; 1Re 9,3}
\bibverse{5} Edificó asimismo altares á todo el ejército de los cielos
en los dos atrios de la casa de Jehová. \bibverse{6} Y pasó sus hijos
por fuego en el valle de los hijos de Hinnom; y miraba en los tiempos,
miraba en agüeros, era dado á adivinaciones, y consultaba pythones y
encantadores: subió de punto en hacer lo malo en ojos de Jehová, para
irritarle. \bibverse{7} A más de esto puso una imagen de fundición, que
hizo, en la casa de Dios, de la cual había dicho Dios á David y á
Salomón su hijo: En esta casa y en Jerusalem, la cual yo elegí sobre
todas las tribus de Israel, pondré mi nombre para siempre: \bibverse{8}
Y nunca más quitaré el pie de Israel de la tierra que yo entregué á
vuestros padres, á condición que guarden y hagan todas las cosas que yo
les he mandado, toda la ley, estatutos, y ordenanzas, por mano de
Moisés. \bibverse{9} Hizo pues Manasés desviarse á Judá y á los
moradores de Jerusalem, para hacer más mal que las gentes que Jehová
destruyó delante de los hijos de Israel.

\bibverse{10} Y habló Jehová á Manasés y á su pueblo, mas ellos no
escucharon: por lo cual Jehová trajo contra ellos los generales del
ejército del rey de los Asirios, los cuales aprisionaron con grillos á
Manasés, y atado con cadenas lleváronlo á Babilonia.

\hypertarget{la-gira-del-prisionero-de-manasuxe9s-a-babilonia-su-arrepentimiento-y-regreso-a-casa}{%
\subsection{La gira del prisionero de Manasés a Babilonia, su
arrepentimiento y regreso a
casa}\label{la-gira-del-prisionero-de-manasuxe9s-a-babilonia-su-arrepentimiento-y-regreso-a-casa}}

\bibverse{11} Mas luego que fué puesto en angustias, oró ante Jehová su
Dios, humillado grandemente en la presencia del Dios de sus padres.

\bibverse{12} Y habiendo á él orado, fué atendido; pues que oyó su
oración, y volviólo á Jerusalem, á su reino. Entonces conoció Manasés
que Jehová era Dios. \bibverse{13} Después de esto edificó el muro de
afuera de la ciudad de David, al occidente de Gihón, en el valle, á la
entrada de la puerta del pescado, y cercó á Ophel, y alzólo muy alto; y
puso capitanes de ejército en todas las ciudades fuertes por Judá.

\hypertarget{manasuxe9s-construyendo-muros-y-esfuerzos-para-eliminar-la-idolatruxeda}{%
\subsection{Manasés construyendo muros y esfuerzos para eliminar la
idolatría}\label{manasuxe9s-construyendo-muros-y-esfuerzos-para-eliminar-la-idolatruxeda}}

\bibverse{14} Asimismo quitó los dioses ajenos, y el ídolo de la casa de
Jehová, y todos los altares que había edificado en el monte de la casa
de Jehová y en Jerusalem, y echólos fuera de la ciudad. \bibverse{15}
Reparó luego el altar de Jehová, y sacrificó sobre él sacrificios
pacíficos y de alabanza; y mandó á Judá que sirviesen á Jehová Dios de
Israel. \bibverse{16} Empero el pueblo aun sacrificaba en los altos,
bien que á Jehová su Dios. \bibverse{17} Lo demás de los hechos de
Manasés, y su oración á su Dios, y las palabras de los videntes que le
hablaron en nombre de Jehová el Dios de Israel, he aquí todo está
escrito en los hechos de los reyes de Israel.

\bibverse{18} Su oración también, y cómo fué oído, todos sus pecados, y
su prevaricación, los lugares donde edificó altos y había puesto bosques
é ídolos antes que se humillase, he aquí estas cosas están escritas en
las palabras de los videntes. \footnote{\textbf{33:18} 2Re 21,17-18}
\bibverse{19} Y durmió Manasés con sus padres, y sepultáronlo en su
casa: y reinó en su lugar Amón su hijo.

\hypertarget{amuxf3n-rey-de-juduxe1}{%
\subsection{Amón Rey de Judá}\label{amuxf3n-rey-de-juduxe1}}

\bibverse{20} De veinte y dos años era Amón cuando comenzó á reinar, y
dos años reinó en Jerusalem. \bibverse{21} E hizo lo malo en ojos de
Jehová, como había hecho Manasés su padre: porque á todos los ídolos que
su padre Manasés había hecho, sacrificó y sirvió Amón. \bibverse{22} Mas
nunca se humilló delante de Jehová, como se humilló Manasés su padre:
antes aumentó el pecado. \bibverse{23} Y conspiraron contra él sus
siervos, y matáronlo en su casa. \footnote{\textbf{33:23} 2Cró 33,12}
\bibverse{24} Mas el pueblo de la tierra hirió á todos los que habían
conspirado contra el rey Amón; y el pueblo de la tierra puso por rey en
su lugar á Josías su hijo. \bibverse{25}

\hypertarget{el-gobierno-del-rey-josuxedas}{%
\subsection{El gobierno del rey
Josías}\label{el-gobierno-del-rey-josuxedas}}

\hypertarget{section-33}{%
\section{34}\label{section-33}}

\bibverse{1} De ocho años era Josías cuando comenzó á reinar, y treinta
y un años reinó en Jerusalem. \bibverse{2} Este hizo lo recto en ojos de
Jehová, y anduvo en los caminos de David su padre, sin apartarse á la
diestra ni á la siniestra.

\footnote{\textbf{34:2} 2Re 22,1-2; 2Cró 29,2}

\hypertarget{restauraciuxf3n-del-culto-puro}{%
\subsection{Restauración del culto
puro}\label{restauraciuxf3n-del-culto-puro}}

\bibverse{3} A los ocho años de su reinado, siendo aún muchacho, comenzó
á buscar al Dios de David su padre; y á los doce años comenzó á limpiar
á Judá y á Jerusalem de los altos, bosques, esculturas, é imágenes de
fundición. \footnote{\textbf{34:3} 2Re 23,4-20} \bibverse{4} Y
derribaron delante de él los altares de los Baales, é hizo pedazos las
imágenes del sol, que estaban puestas encima; despedazó también los
bosques, y las esculturas y estatuas de fundición, y desmenuzólas, y
esparció el polvo sobre los sepulcros de los que las habían sacrificado.
\footnote{\textbf{34:4} 2Cró 14,4; Lev 26,30} \bibverse{5} Quemó además
los huesos de los sacerdotes sobre sus altares, y limpió á Judá y á
Jerusalem. \footnote{\textbf{34:5} 1Re 13,2} \bibverse{6} Lo mismo hizo
en las ciudades de Manasés, Ephraim, y Simeón, hasta en Nephtalí, con
sus lugares asolados alrededor. \bibverse{7} Y como hubo derribado los
altares y los bosques, y quebrado y desmenuzado las esculturas, y
destruído todos los ídolos por toda la tierra de Israel, volvióse á
Jerusalem.

\hypertarget{explicar-los-procedimientos-que-se-siguen-para-restaurar-y-mantener-el-templo}{%
\subsection{Explicar los procedimientos que se siguen para restaurar y
mantener el
templo}\label{explicar-los-procedimientos-que-se-siguen-para-restaurar-y-mantener-el-templo}}

\bibverse{8} A los dieciocho años de su reinado, después de haber
limpiado la tierra, y la casa, envió á Saphán hijo de Asalías, y á
Maasías gobernador de la ciudad, y á Joah hijo de Joachâz, canciller,
para que reparasen la casa de Jehová su Dios. \footnote{\textbf{34:8}
  2Re 22,3-6} \bibverse{9} Los cuales vinieron á Hilcías, gran
sacerdote, y dieron el dinero que había sido metido en la casa de
Jehová, que los Levitas que guardaban la puerta habían recogido de mano
de Manasés y de Ephraim y de todas las reliquias de Israel, y de todo
Judá y Benjamín, habiéndose después vuelto á Jerusalem. \bibverse{10} Y
entregáronlo en mano de los que hacían la obra, que eran sobrestantes en
la casa de Jehová; los cuales lo daban á los que hacían la obra y
trabajaban en la casa de Jehová, para reparar y restaurar el templo.
\bibverse{11} Daban asimismo á los oficiales y albañiles para que
comprasen piedra de cantería, y madera para las trabazones, y para
entabladura de las casas, las cuales habían destruído los reyes de Judá.
\bibverse{12} Y estos hombres procedían con fidelidad en la obra: y eran
sus gobernadores Jahath y Abdías, Levitas de los hijos de Merari; y
Zachârías y Mesullam de los hijos de Coath, para que activasen la obra;
y de los Levitas, todos los entendidos en instrumentos de música.
\bibverse{13} También velaban sobre los ganapanes, y eran sobrestantes
de los que se ocupaban en cualquier clase de obra; y de los Levitas
había escribas, gobernadores, y porteros.

\hypertarget{informe-sobre-el-descubrimiento-del-cuxf3digo-y-su-primer-efecto}{%
\subsection{Informe sobre el descubrimiento del código y su primer
efecto}\label{informe-sobre-el-descubrimiento-del-cuxf3digo-y-su-primer-efecto}}

\bibverse{14} Y al sacar el dinero que había sido metido en la casa de
Jehová, Hilcías el sacerdote halló el libro de la ley de Jehová dada por
mano de Moisés. \bibverse{15} Y dando cuenta Hilcías, dijo á Saphán
escriba: Yo he hallado el libro de la ley en la casa de Jehová. Y dió
Hilcías el libro á Saphán.

\bibverse{16} Y Saphán lo llevó al rey, y contóle el negocio, diciendo:
Tus siervos han cumplido todo lo que les fué dado á cargo. \bibverse{17}
Han reunido el dinero que se halló en la casa de Jehová, y lo han
entregado en mano de los comisionados, y en mano de los que hacen la
obra. \bibverse{18} A más de esto, declaró Saphán escriba al rey,
diciendo: El sacerdote Hilcías me dió un libro. Y leyó Saphán en él
delante del rey.

\bibverse{19} Y luego que el rey oyó las palabras de la ley, rasgó sus
vestidos; \bibverse{20} Y mandó á Hilcías y á Ahicam hijo de Saphán, y á
Abdón hijo de Michâ, y á Saphán escriba, y á Asaía siervo del rey,
diciendo: \bibverse{21} Andad, y consultad á Jehová de mí, y de las
reliquias de Israel y de Judá, acerca de las palabras del libro que se
ha hallado; porque grande es el furor de Jehová que ha caído sobre
nosotros, por cuanto nuestros padres no guardaron la palabra de Jehová,
para hacer conforme á todo lo que está escrito en este libro.

\hypertarget{interrogatorio-y-respuesta-de-la-profetisa-hulda}{%
\subsection{Interrogatorio y respuesta de la profetisa
Hulda}\label{interrogatorio-y-respuesta-de-la-profetisa-hulda}}

\bibverse{22} Entonces Hilcías y los del rey fueron á Hulda profetisa,
mujer de Sallum, hijo de Tikvath, hijo de Hasra, guarda de las
vestimentas, la cual moraba en Jerusalem en la casa de la doctrina; y
dijéronle las palabras dichas.

\bibverse{23} Y ella respondió: Jehová el Dios de Israel ha dicho así:
Decid al varón que os ha enviado á mí, que así ha dicho Jehová:
\bibverse{24} He aquí yo traigo mal sobre este lugar, y sobre los
moradores de él, todas las maldiciones que están escritas en el libro
que leyeron delante del rey de Judá: \bibverse{25} Por cuanto me han
dejado, y han sacrificado á dioses ajenos, provocándome á ira en todas
las obras de sus manos; por tanto mi furor destilará sobre este lugar, y
no se apagará. \bibverse{26} Mas al rey de Judá, que os ha enviado á
consultar á Jehová, así le diréis: Jehová el Dios de Israel ha dicho
así: Por cuanto oiste las palabras del libro, \bibverse{27} Y tu corazón
se enterneció, y te humillaste delante de Dios al oir sus palabras sobre
este lugar, y sobre sus moradores, y te humillaste delante de mí, y
rasgaste tus vestidos, y lloraste en mi presencia, yo también te he
oído, dice Jehová. \footnote{\textbf{34:27} 2Cró 33,12} \bibverse{28} He
aquí que yo te recogeré con tus padres, y serás recogido en tus
sepulcros en paz, y tus ojos no verán todo el mal que yo traigo sobre
este lugar, y sobre los moradores de él. Y ellos refirieron al rey la
respuesta.

\hypertarget{josuxedas-concluye-el-nuevo-pacto-de-dios-en-asociaciuxf3n-con-los-ancianos-del-pueblo}{%
\subsection{Josías concluye el nuevo pacto de Dios en asociación con los
ancianos del
pueblo}\label{josuxedas-concluye-el-nuevo-pacto-de-dios-en-asociaciuxf3n-con-los-ancianos-del-pueblo}}

\bibverse{29} Entonces el rey envió y juntó todos los ancianos de Judá y
de Jerusalem. \bibverse{30} Y subió el rey á la casa de Jehová, y con él
todos los varones de Judá, y los moradores de Jerusalem, y los
sacerdotes, y los Levitas, y todo el pueblo desde el mayor hasta el más
pequeño; y leyó á oídos de ellos todas las palabras del libro del pacto
que había sido hallado en la casa de Jehová. \bibverse{31} Y estando el
rey en pie en su sitio, hizo alianza delante de Jehová de caminar en pos
de Jehová, y de guardar sus mandamientos, sus testimonios, y sus
estatutos, de todo su corazón y de toda su alma, poniendo por obra las
palabras del pacto que estaban escritas en aquel libro. \footnote{\textbf{34:31}
  2Cró 15,12; Jos 24,25} \bibverse{32} E hizo que se obligaran á ello
todos los que estaban en Jerusalem y en Benjamín: y los moradores de
Jerusalem hicieron conforme al pacto de Dios, del Dios de sus padres.
\footnote{\textbf{34:32} 2Re 23,3} \bibverse{33} Y quitó Josías todas
las abominaciones de todas las tierras de los hijos de Israel, é hizo á
todos los que se hallaron en Israel que sirviesen á Jehová su Dios. No
se apartaron de en pos de Jehová el Dios de sus padres, todo el tiempo
que él vivió.

\hypertarget{la-estricta-celebraciuxf3n-de-la-pascua-de-josuxedas}{%
\subsection{La estricta celebración de la Pascua de
Josías}\label{la-estricta-celebraciuxf3n-de-la-pascua-de-josuxedas}}

\hypertarget{section-34}{%
\section{35}\label{section-34}}

\bibverse{1} Y josías hizo pascua á Jehová en Jerusalem, y sacrificaron
la pascua á los catorce del mes primero. \bibverse{2} Y puso á los
sacerdotes en sus empleos, y confirmólos en el ministerio de la casa de
Jehová. \bibverse{3} Y dijo á los Levitas que enseñaban á todo Israel, y
que estaban dedicados á Jehová: Poned el arca del santuario en la casa
que edificó Salomón hijo de David, rey de Israel, para que no la
carguéis más sobre los hombros. Ahora serviréis á Jehová vuestro Dios, y
á su pueblo Israel. \footnote{\textbf{35:3} 1Re 6,1} \bibverse{4}
Apercibíos según las familias de vuestros padres, por vuestros órdenes,
conforme á la prescripción de David rey de Israel, y de Salomón su hijo.
\bibverse{5} Estad en el santuario según la distribución de las familias
de vuestros hermanos los hijos del pueblo, y según la división de la
familia de los Levitas. \bibverse{6} Sacrificad luego la pascua: y
después de santificaros, apercibid á vuestros hermanos, para que hagan
conforme á la palabra de Jehová dada por mano de Moisés.

\bibverse{7} Y ofreció el rey Josías á los del pueblo ovejas, corderos,
y cabritos de los rebaños, en número de treinta mil, y tres mil bueyes,
todo para la pascua, para todos los que se hallaron presentes: esto de
la hacienda del rey. \bibverse{8} También sus príncipes ofrecieron con
liberalidad al pueblo, y á los sacerdotes y Levitas. Hilcías, Zachârías
y Jehiel, príncipes de la casa de Dios, dieron á los sacerdotes para
hacer la pascua dos mil seiscientas ovejas, y trescientos bueyes.
\bibverse{9} Asimismo Chônanías, y Semeías y Nathanael sus hermanos, y
Hasabías, Jehiel, y Josabad, príncipes de los Levitas, dieron á los
Levitas para los sacrificios de la pascua cinco mil ovejas, y quinientos
bueyes.

\bibverse{10} Aprestado así el servicio, los sacerdotes se colocaron en
sus puestos, y asimismo los Levitas en sus órdenes, conforme al
mandamiento del rey. \bibverse{11} Y sacrificaron la pascua; y esparcían
los sacerdotes la sangre tomada de mano de los Levitas, y los Levitas
desollaban. \bibverse{12} Tomaron luego del holocausto, para dar
conforme á los repartimientos por las familias de los del pueblo, á fin
de que ofreciesen á Jehová, según está escrito en el libro de Moisés: y
asimismo tomaron de los bueyes. \bibverse{13} Y asaron la pascua al
fuego según la costumbre: mas lo que había sido santificado lo cocieron
en ollas, en calderos, y calderas, y repartiéronlo prestamente á todo el
pueblo. \bibverse{14} Y después aderezaron para sí y para los
sacerdotes; porque los sacerdotes, hijos de Aarón, estuvieron ocupados
hasta la noche en el sacrificio de los holocaustos y de los sebos; por
tanto, los Levitas aderezaron para sí, y para los sacerdotes hijos de
Aarón. \bibverse{15} Asimismo los cantores hijos de Asaph estaban en su
puesto, conforme al mandamiento de David, de Asaph y de Hemán, y de
Jeduthún vidente del rey; también los porteros estaban á cada puerta; y
no era menester que se apartasen de su ministerio, porque sus hermanos
los Levitas aparejaban para ellos. \footnote{\textbf{35:15} 1Cró 25,1;
  1Cró 26,1}

\bibverse{16} Así fué aprestado todo el servicio de Jehová en aquel día,
para hacer la pascua, y sacrificar los holocaustos sobre el altar de
Jehová, conforme al mandamiento del rey Josías. \bibverse{17} Y los
hijos de Israel que se hallaron allí, hicieron la pascua en aquel
tiempo, y la solemnidad de los panes sin levadura, por siete días.
\bibverse{18} Nunca tal pascua fué hecha en Israel desde los días de
Samuel el profeta; ni ningún rey de Israel hizo pascua tal como la que
hizo el rey Josías, y los sacerdotes y Levitas, y todo Judá é Israel,
los que se hallaron allí, juntamente con los moradores de Jerusalem.
\bibverse{19} Esta pascua fué celebrada en el año dieciocho del rey
Josías.

\hypertarget{necao-de-egipto-y-la-muerte-de-josuxedas-dolor-por-el}{%
\subsection{Necao de Egipto y la muerte de Josías; Dolor por
el}\label{necao-de-egipto-y-la-muerte-de-josuxedas-dolor-por-el}}

\bibverse{20} Después de todas estas cosas, luego de haber Josías
preparado la casa, Nechâo rey de Egipto subió á hacer guerra en
Carchêmis junto á Eufrates; y salió Josías contra él. \bibverse{21} Y él
le envió embajadores, diciendo: ¿Qué tenemos yo y tú, rey de Judá? Yo no
vengo contra ti hoy, sino contra la casa que me hace guerra: y Dios dijo
que me apresurase. Déjate de meterte con Dios, que es conmigo, no te
destruya.

\bibverse{22} Mas Josías no volvió su rostro de él, antes disfrazóse
para darle batalla, y no atendió á las palabras de Nechâo, que eran de
boca de Dios; y vino á darle la batalla en el campo de Megiddo.
\bibverse{23} Y los archeros tiraron al rey Josías flechas; y dijo el
rey á sus siervos: Quitadme de aquí, porque estoy herido gravemente.

\bibverse{24} Entonces sus siervos lo quitaron de aquel carro, y
pusiéronle en otro segundo carro que tenía, y lleváronle á Jerusalem, y
murió; y sepultáronle en los sepulcros de sus padres. Y todo Judá y
Jerusalem hizo duelo por Josías. \bibverse{25} Y endechó Jeremías por
Josías, y todos los cantores y cantoras recitan sus lamentaciones sobre
Josías hasta hoy; y las dieron por norma para endechar en Israel, las
cuales están escritas en las Lamentaciones. \footnote{\textbf{35:25} Jer
  22,10-11} \bibverse{26} Lo demás de los hechos de Josías, y sus
piadosas obras, conforme á lo que está escrito en la ley de Jehová,
\bibverse{27} Y sus hechos, primeros y postreros, he aquí está escrito
en el libro de los reyes de Israel y de Judá.

\hypertarget{joachuxe2z-rey-de-juduxe1}{%
\subsection{Joachâz rey de Judá}\label{joachuxe2z-rey-de-juduxe1}}

\hypertarget{section-35}{%
\section{36}\label{section-35}}

\bibverse{1} Entonces el pueblo de la tierra tomó á Joachâz hijo de
Josías, é hiciéronle rey en lugar de su padre en Jerusalem. \bibverse{2}
De veinte y tres años era Joachâz cuando comenzó á reinar, y tres meses
reinó en Jerusalem. \bibverse{3} Y el rey de Egipto lo quitó de
Jerusalem, y condenó la tierra en cien talentos de plata y uno de oro.

\hypertarget{joacim-kuxf6nig-von-juda}{%
\subsection{Joacim König von Juda}\label{joacim-kuxf6nig-von-juda}}

\bibverse{4} Y constituyó el rey de Egipto á su hermano Eliacim por rey
sobre Judá y Jerusalem, y mudóle el nombre en Joacim; y á Joachâz su
hermano tomó Nechâo, y llevólo á Egipto.

\bibverse{5} Cuando comenzó á reinar Joacim era de veinte y cinco años,
y reinó once años en Jerusalem: é hizo lo malo en ojos de Jehová su
Dios. \bibverse{6} Y subió contra él Nabucodonosor rey de Babilonia, y
atado con cadenas lo llevó á Babilonia. \bibverse{7} También llevó
Nabucodonosor á Babilonia de los vasos de la casa de Jehová, y púsolos
en su templo en Babilonia. \footnote{\textbf{36:7} Esd 1,7} \bibverse{8}
Lo demás de los hechos de Joacim, y las abominaciones que hizo, y lo que
en él se halló, he aquí está escrito en el libro de los reyes de Israel
y de Judá: y reinó en su lugar Joachîn su hijo.

\hypertarget{joachuxeen-rey-de-juduxe1}{%
\subsection{Joachîn rey de Judá}\label{joachuxeen-rey-de-juduxe1}}

\bibverse{9} De ocho años era Joachîn cuando comenzó á reinar, y reinó
tres meses y diez días en Jerusalem: é hizo lo malo en ojos de Jehová.
\bibverse{10} A la vuelta del año el rey Nabucodonosor envió, é hízolo
llevar á Babilonia juntamente con los vasos preciosos de la casa de
Jehová; y constituyó á Sedecías su hermano por rey sobre Judá y
Jerusalem.

\hypertarget{sedecuxedas-rey-de-juduxe1-la-ruina-de-uxe9l-y-de-su-gente}{%
\subsection{Sedecías, rey de Judá; la ruina de él y de su
gente}\label{sedecuxedas-rey-de-juduxe1-la-ruina-de-uxe9l-y-de-su-gente}}

\bibverse{11} De veinte y un años era Sedecías cuando comenzó á reinar,
y once años reinó en Jerusalem. \footnote{\textbf{36:11} Jer 52,1-27}
\bibverse{12} E hizo lo malo en ojos de Jehová su Dios, y no se humilló
delante de Jeremías profeta, que le hablaba de parte de Jehová.
\footnote{\textbf{36:12} Jer 37,-1; Jer 38,1-38} \bibverse{13} Rebelóse
asimismo contra Nabucodonosor, al cual había jurado por Dios; y
endureció su cerviz, y obstinó su corazón, para no volverse á Jehová el
Dios de Israel. \bibverse{14} Y también todos los príncipes de los
sacerdotes, y el pueblo, aumentaron la prevaricación, siguiendo todas
las abominaciones de las gentes, y contaminando la casa de Jehová, la
cual él había santificado en Jerusalem. \footnote{\textbf{36:14} Deut
  18,9}

\bibverse{15} Y Jehová el Dios de sus padres envió á ellos por mano de
sus mensajeros, levantándose de mañana y enviando: porque él tenía
misericordia de su pueblo, y de su habitación. \bibverse{16} Mas ellos
hacían escarnio de los mensajeros de Dios, y menospreciaban sus
palabras, burlándose de sus profetas, hasta que subió el furor de Jehová
contra su pueblo, y que no hubo remedio.

\hypertarget{destrucciuxf3n-del-imperio-por-nabucodonosor-el-cautiverio-babiluxf3nico}{%
\subsection{Destrucción del imperio por Nabucodonosor; el cautiverio
babilónico}\label{destrucciuxf3n-del-imperio-por-nabucodonosor-el-cautiverio-babiluxf3nico}}

\bibverse{17} Por lo cual trajo contra ellos al rey de los Caldeos, que
mató á cuchillo sus mancebos en la casa de su santuario, sin perdonar
joven, ni doncella, ni viejo, ni decrépito; todos los entregó en sus
manos. \bibverse{18} Asimismo todos los vasos de la casa de Dios,
grandes y chicos, los tesoros de la casa de Jehová, y los tesoros del
rey y de sus príncipes, todo lo llevó á Babilonia. \bibverse{19} Y
quemaron la casa de Dios, y rompieron el muro de Jerusalem, y
consumieron al fuego todos sus palacios, y destruyeron todos sus vasos
deseables. \bibverse{20} Los que quedaron del cuchillo, pasáronlos á
Babilonia; y fueron siervos de él y de sus hijos, hasta que vino el
reino de los Persas; \bibverse{21} Para que se cumpliese la palabra de
Jehová por la boca de Jeremías, hasta que la tierra hubo gozado sus
sábados: porque todo el tiempo de su asolamiento reposó, hasta que los
setenta años fueron cumplidos.

\footnote{\textbf{36:21} Lev 26,34; Jer 25,8-11}

\hypertarget{el-permiso-para-regresar-a-casa-del-rey-persa-ciro}{%
\subsection{El permiso para regresar a casa del rey persa
Ciro}\label{el-permiso-para-regresar-a-casa-del-rey-persa-ciro}}

\bibverse{22} Mas al primer año de Ciro rey de los Persas, para que se
cumpliese la palabra de Jehová por boca de Jeremías, Jehová excitó el
espíritu de Ciro rey de los Persas, el cual hizo pasar pregón por todo
su reino, y también por escrito, diciendo: \bibverse{23} Así dice Ciro
rey de los Persas: Jehová, el Dios de los cielos, me ha dado todos los
reinos de la tierra; y él me ha encargado que le edifique casa en
Jerusalem, que es en Judá. ¿Quién de vosotros hay de todo su pueblo?
Jehová su Dios sea con él, y suba.
