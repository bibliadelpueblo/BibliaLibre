\hypertarget{section}{%
\section{1}\label{section}}

\bibverse{1} Estos son los nombres de los hijos de Israel, que entraron
en Egipto con Jacob; cada uno entró con su familia. \bibverse{2} Rubén,
Simeón, Leví y Judá; \bibverse{3} Issachâr, Zabulón y Benjamín;
\bibverse{4} Dan y Nephtalí, Gad y Aser. \bibverse{5} Y todas las almas
de los que salieron del muslo de Jacob, fueron setenta. Y José estaba en
Egipto. \bibverse{6} Y murió José, y todos sus hermanos, y toda aquella
generación. \bibverse{7} Y los hijos de Israel crecieron, y
multiplicaron, y fueron aumentados y corroborados en extremo; y llenóse
la tierra de ellos. \bibverse{8} Levantóse entretanto un nuevo rey sobre
Egipto, que no conocía á José; el cual dijo á su pueblo: \bibverse{9} He
aquí, el pueblo de los hijos de Israel es mayor y más fuerte que
nosotros: \bibverse{10} Ahora, pues, seamos sabios para con él, porque
no se multiplique, y acontezca que viniendo guerra, él también se junte
con nuestros enemigos, y pelee contra nosotros, y se vaya de la tierra.
\bibverse{11} Entonces pusieron sobre él comisarios de tributos que los
molestasen con sus cargas; y edificaron á Faraón las ciudades de los
bastimentos, Phithom y Raamses. \bibverse{12} Empero cuanto más los
oprimían, tanto más se multiplicaban y crecían: así que estaban ellos
fastidiados de los hijos de Israel. \bibverse{13} Y los Egipcios
hicieron servir á los hijos de Israel con dureza: \bibverse{14} Y
amargaron su vida con dura servidumbre, en hacer barro y ladrillo, y en
toda labor del campo, y en todo su servicio, al cual los obligaban con
rigorismo. \bibverse{15} Y habló el rey de Egipto á las parteras de las
Hebreas, una de las cuales se llamaba Siphra, y otra Phúa, y díjoles:
\bibverse{16} Cuando parteareis á las Hebreas, y mirareis los asientos,
si fuere hijo, matadlo; y si fuere hija, entonces viva. \bibverse{17}
Mas las parteras temieron á Dios, y no hicieron como les mandó el rey de
Egipto, sino que reservaban la vida á los niños. \bibverse{18} Y el rey
de Egipto hizo llamar á las parteras, y díjoles: Por qué habéis hecho
esto, que habéis reservado la vida á los niños? \bibverse{19} Y las
parteras respondieron á Faraón: Porque las mujeres Hebreas no son como
las Egipcias: porque son robustas, y paren antes que la partera venga á
ellas. \bibverse{20} Y Dios hizo bien á las parteras: y el pueblo se
multiplicó, y se corroboraron en gran manera. \bibverse{21} Y por haber
las parteras temido á Dios, él les hizo casas. \bibverse{22} Entonces
Faraón mandó á todo su pueblo, diciendo: Echad en el río todo hijo que
naciere, y á toda hija reservad la vida.

\hypertarget{section-1}{%
\section{2}\label{section-1}}

\bibverse{1} Un varón de la familia de Leví fué, y tomó por mujer una
hija de Leví: \bibverse{2} La cual concibió, y parió un hijo: y viéndolo
que era hermoso, túvole escondido tres meses. \bibverse{3} Pero no
pudiendo ocultarle más tiempo, tomó una arquilla de juncos, y
calafateóla con pez y betún, y colocó en ella al niño, y púsolo en un
carrizal á la orilla del río: \bibverse{4} Y paróse una hermana suya á
lo lejos, para ver lo que le acontecería. \bibverse{5} Y la hija de
Faraón descendió á lavarse al río, y paseándose sus doncellas por la
ribera del río, vió ella la arquilla en el carrizal, y envió una criada
suya á que la tomase. \bibverse{6} Y como la abrió, vió al niño; y he
aquí que el niño lloraba. Y teniendo compasión de él, dijo: De los niños
de los Hebreos es éste. \bibverse{7} Entonces su hermana dijo á la hija
de Faraón: ¿Iré á llamarte un ama de las Hebreas, para que te críe este
niño? \bibverse{8} Y la hija de Faraón respondió: Ve. Entonces fué la
doncella, y llamó á la madre del niño; \bibverse{9} A la cual dijo la
hija de Faraón: Lleva este niño, y críamelo, y yo te lo pagaré. Y la
mujer tomó al niño, y criólo. \bibverse{10} Y como creció el niño, ella
lo trajo á la hija de Faraón, la cual lo prohijó, y púsole por nombre
Moisés, diciendo: Porque de las aguas lo saqué. \bibverse{11} Y en
aquellos días acaeció que, crecido ya Moisés, salió á sus hermanos, y
vió sus cargas: y observó á un Egipcio que hería á uno de los Hebreos,
sus hermanos. \bibverse{12} Y miró á todas partes, y viendo que no
parecía nadie, mató al Egipcio, y escondiólo en la arena. \bibverse{13}
Y salió al día siguiente, y viendo á dos Hebreos que reñían, dijo al que
hacía la injuria: ¿Por qué hieres á tu prójimo? \bibverse{14} Y él
respondió: ¿Quién te ha puesto á ti por príncipe y juez sobre nosotros?
¿piensas matarme como mataste al Egipcio? Entonces Moisés tuvo miedo, y
dijo: Ciertamente esta cosa es descubierta. \bibverse{15} Y oyendo
Faraón este negocio, procuró matar á Moisés: mas Moisés huyó de delante
de Faraón, y habitó en la tierra de Madián; y sentóse junto á un pozo.
\bibverse{16} Tenía el sacerdote de Madián siete hijas, las cuales
vinieron á sacar agua, para llenar las pilas y dar de beber á las ovejas
de su padre. \bibverse{17} Mas los pastores vinieron, y echáronlas:
Entonces Moisés se levantó y defendiólas, y abrevó sus ovejas.
\bibverse{18} Y volviendo ellas á Ragüel su padre, díjoles él: ¿Por qué
habéis hoy venido tan presto? \bibverse{19} Y ellas respondieron: Un
varón Egipcio nos defendió de mano de los pastores, y también nos sacó
el agua, y abrevó las ovejas. \bibverse{20} Y dijo á sus hijas: ¿Y dónde
está? ¿por qué habéis dejado ese hombre? llamadle para que coma pan.
\bibverse{21} Y Moisés acordó en morar con aquel varón; y él dió á
Moisés á su hija Séphora: \bibverse{22} La cual le parió un hijo, y él
le puso por nombre Gersom: porque dijo: Peregrino soy en tierra ajena.
\bibverse{23} Y aconteció que después de muchos días murió el rey de
Egipto, y los hijos de Israel suspiraron á causa de la servidumbre, y
clamaron: y subió á Dios el clamor de ellos con motivo de su
servidumbre. \bibverse{24} Y oyó Dios el gemido de ellos, y acordóse de
su pacto con Abraham, Isaac y Jacob. \bibverse{25} Y miró Dios á los
hijos de Israel, y reconociólos Dios.

\hypertarget{section-2}{%
\section{3}\label{section-2}}

\bibverse{1} Y apacentando Moisés las ovejas de Jethro su suegro,
sacerdote de Madián, llevó las ovejas detrás del desierto, y vino á
Horeb, monte de Dios. \bibverse{2} Y apareciósele el Angel de Jehová en
una llama de fuego en medio de una zarza: y él miró, y vió que la zarza
ardía en fuego, y la zarza no se consumía. \bibverse{3} Entonces Moisés
dijo: Iré yo ahora, y veré esta grande visión, por qué causa la zarza no
se quema. \bibverse{4} Y viendo Jehová que iba á ver, llamólo Dios de en
medio de la zarza, y dijo: ¡Moisés, Moisés! Y él respondió: Heme aquí.
\bibverse{5} Y dijo: No te llegues acá: quita tus zapatos de tus pies,
porque el lugar en que tú estás, tierra santa es. \bibverse{6} Y dijo:
Yo soy el Dios de tu padre, Dios de Abraham, Dios de Isaac, Dios de
Jacob. Entonces Moisés cubrió su rostro, porque tuvo miedo de mirar á
Dios. \bibverse{7} Y dijo Jehová: Bien he visto la aflicción de mi
pueblo que está en Egipto, y he oído su clamor á causa de sus exactores;
pues tengo conocidas sus angustias: \bibverse{8} Y he descendido para
librarlos de mano de los Egipcios, y sacarlos de aquella tierra á una
tierra buena y ancha, á tierra que fluye leche y miel, á los lugares del
Cananeo, del Hetheo, del Amorrheo, del Pherezeo, del Heveo, y del
Jebuseo. \bibverse{9} El clamor, pues, de los hijos de Israel ha venido
delante de mí, y también he visto la opresión con que los Egipcios los
oprimen. \bibverse{10} Ven por tanto ahora, y enviarte he á Faraón, para
que saques á mi pueblo, los hijos de Israel, de Egipto. \bibverse{11}
Entonces Moisés respondió á Dios: ¿Quién soy yo, para que vaya á Faraón,
y saque de Egipto á los hijos de Israel? \bibverse{12} Y él le
respondió: Ve, porque yo seré contigo; y esto te será por señal de que
yo te he enviado: luego que hubieres sacado este pueblo de Egipto,
serviréis á Dios sobre este monte. \bibverse{13} Y dijo Moisés á Dios:
He aquí que llego yo á los hijos de Israel, y les digo, El Dios de
vuestros padres me ha enviado á vosotros; si ellos me preguntaren: ¿Cuál
es su nombre? ¿qué les responderé? \bibverse{14} Y respondió Dios á
Moisés: YO SOY EL QUE SOY. Y dijo: Así dirás á los hijos de Israel: YO
SOY me ha enviado á vosotros. \bibverse{15} Y dijo más Dios á Moisés:
Así dirás á los hijos de Israel: Jehová, el Dios de vuestros padres, el
Dios de Abraham, Dios de Isaac y Dios de Jacob, me ha enviado á
vosotros. Este es mi nombre para siempre, este es mi memorial por todos
los siglos. \bibverse{16} Ve, y junta los ancianos de Israel, y diles:
Jehová, el Dios de vuestros padres, el Dios de Abraham, de Isaac, y de
Jacob, me apareció, diciendo: De cierto os he visitado, y visto lo que
se os hace en Egipto; \bibverse{17} Y he dicho: Yo os sacaré de la
aflicción de Egipto á la tierra del Cananeo, y del Hetheo, y del
Amorrheo, y del Pherezeo, y del Heveo, y del Jebuseo, á una tierra que
fluye leche y miel. \bibverse{18} Y oirán tu voz; é irás tú, y los
ancianos de Israel, al rey de Egipto, y le diréis: Jehová, el Dios de
los Hebreos, nos ha encontrado; por tanto nosotros iremos ahora camino
de tres días por el desierto, para que sacrifiquemos á Jehová nuestro
Dios. \bibverse{19} Mas yo sé que el rey de Egipto no os dejará ir sino
por mano fuerte. \bibverse{20} Empero yo extenderé mi mano, y heriré á
Egipto con todas mis maravillas que haré en él, y entonces os dejará ir.
\bibverse{21} Y yo daré á este pueblo gracia en los ojos de los
Egipcios, para que cuando os partiereis, no salgáis vacíos:
\bibverse{22} Sino que demandará cada mujer á su vecina y á su huéspeda
vasos de plata, vasos de oro, y vestidos: los cuales pondréis sobre
vuestros hijos y vuestras hijas; y despojaréis á Egipto.

\hypertarget{section-3}{%
\section{4}\label{section-3}}

\bibverse{1} Entonces Moisés respondió, y dijo: He aquí que ellos no me
creerán, ni oirán mi voz; porque dirán: No te ha aparecido Jehová.
\bibverse{2} Y Jehová dijo: ¿Qué es eso que tienes en tu mano? Y él
respondió: Una vara. \bibverse{3} Y él le dijo: Échala en tierra. Y él
la echó en tierra, y tornóse una culebra: y Moisés huía de ella.
\bibverse{4} Entonces dijo Jehová á Moisés: Extiende tu mano, y tómala
por la cola. Y él extendió su mano, y tomóla, y tornóse vara en su mano.
\bibverse{5} Por esto creerán que se te ha aparecido Jehová, el Dios de
tus padres, el Dios de Abraham, Dios de Isaac y Dios de Jacob.
\bibverse{6} Y díjole más Jehová: Mete ahora tu mano en tu seno. Y él
metió la mano en su seno; y como la sacó, he aquí que su mano estaba
leprosa como la nieve. \bibverse{7} Y dijo: Vuelve á meter tu mano en tu
seno: y él volvió á meter su mano en su seno; y volviéndola á sacar del
seno, he aquí que se había vuelto como la otra carne. \bibverse{8} Sí
aconteciere, que no te creyeren, ni obedecieren á la voz de la primera
señal, creerán á la voz de la postrera. \bibverse{9} Y si aun no
creyeren á estas dos señales, ni oyeren tu voz, tomarás de las aguas del
río, y derrámalas en tierra; y volverse han aquellas aguas que tomarás
del río, se volverán sangre en la tierra. \bibverse{10} Entonces dijo
Moisés á Jehová: ¡Ay Señor! yo no soy hombre de palabras de ayer ni de
anteayer, ni aun desde que tú hablas á tu siervo; porque soy tardo en el
habla y torpe de lengua. \bibverse{11} Y Jehová le respondió: ¿Quién dió
la boca al hombre? ¿ó quién hizo al mudo y al sordo, al que ve y al
ciego? ¿no soy yo Jehová? \bibverse{12} Ahora pues, ve, que yo seré en
tu boca, y te enseñaré lo que hayas de hablar. \bibverse{13} Y él dijo:
¡Ay Señor! envía por mano del que has de enviar. \bibverse{14} Entonces
Jehová se enojó contra Moisés, y dijo: ¿No conozco yo á tu hermano
Aarón, Levita, y que él hablará? Y aun he aquí que él te saldrá á
recibir, y en viéndote, se alegrará en su corazón. \bibverse{15} Tú
hablarás á él, y pondrás en su boca las palabras, y yo seré en tu boca y
en la suya, y os enseñaré lo que hayáis de hacer. \bibverse{16} Y él
hablará por ti al pueblo; y él te será á ti en lugar de boca, y tú serás
para él en lugar de Dios. \bibverse{17} Y tomarás esta vara en tu mano,
con la cual harás las señales. \bibverse{18} Así se fué Moisés, y
volviendo á su suegro Jethro, díjole: Iré ahora, y volveré á mis
hermanos que están en Egipto, para ver si aun viven. Y Jethro dijo á
Moisés: Ve en paz. \bibverse{19} Dijo también Jehová á Moisés en Madián:
Ve, y vuélvete á Egipto, porque han muerto todos los que procuraban tu
muerte. \bibverse{20} Entonces Moisés tomó su mujer y sus hijos, y
púsolos sobre un asno, y volvióse á tierra de Egipto: tomó también
Moisés la vara de Dios en su mano. \bibverse{21} Y dijo Jehová á Moisés:
Cuando hubiereis vuelto á Egipto, mira que hagas delante de Faraón todas
las maravillas que he puesto en tu mano: yo empero endureceré su
corazón, de modo que no dejará ir al pueblo. \bibverse{22} Y dirás á
Faraón: Jehová ha dicho así: Israel es mi hijo, mi primogénito.
\bibverse{23} Ya te he dicho que dejes ir á mi hijo, para que me sirva,
mas no has querido dejarlo ir: he aquí yo voy á matar á tu hijo, tu
primogénito. \bibverse{24} Y aconteció en el camino, que en una posada
le salió al encuentro Jehová, y quiso matarlo. \bibverse{25} Entonces
Séphora cogió un afilado pedernal, y cortó el prepucio de su hijo, y
echólo á sus pies, diciendo: A la verdad tú me eres un esposo de sangre.
\bibverse{26} Así le dejó luego ir. Y ella dijo: Esposo de sangre, á
causa de la circuncisión. \bibverse{27} Y Jehová dijo á Aarón: Ve á
recibir á Moisés al desierto. Y él fué, y encontrólo en el monte de
Dios, y besóle. \bibverse{28} Entonces contó Moisés á Aarón todas las
palabras de Jehová que le enviaba, y todas las señales que le había
dado. \bibverse{29} Y fueron Moisés y Aarón, y juntaron todos los
ancianos de los hijos de Israel: \bibverse{30} Y habló Aarón todas las
palabras que Jehová había dicho á Moisés, é hizo las señales delante de
los ojos del pueblo. \bibverse{31} Y el pueblo creyó: y oyendo que
Jehová había visitado los hijos de Israel, y que había visto su
aflicción, inclináronse y adoraron.

\hypertarget{section-4}{%
\section{5}\label{section-4}}

\bibverse{1} Después entraron Moisés y Aarón á Faraón, y le dijeron:
Jehová, el Dios de Israel, dice así: Deja ir á mi pueblo á celebrarme
fiesta en el desierto. \bibverse{2} Y Faraón respondió: ¿Quién es
Jehová, para que yo oiga su voz y deje ir á Israel? Yo no conozco á
Jehová, ni tampoco dejaré ir á Israel. \bibverse{3} Y ellos dijeron: El
Dios de los Hebreos nos ha encontrado: iremos, pues, ahora camino de
tres días por el desierto, y sacrificaremos á Jehová nuestro Dios;
porque no venga sobre nosotros con pestilencia ó con espada.
\bibverse{4} Entonces el rey de Egipto les dijo: Moisés y Aarón, ¿por
qué hacéis cesar al pueblo de su obra? idos á vuestros cargos.
\bibverse{5} Dijo también Faraón: He aquí el pueblo de la tierra es
ahora mucho, y vosotros les hacéis cesar de sus cargos. \bibverse{6} Y
mandó Faraón aquel mismo día á los cuadrilleros del pueblo que le tenían
á su cargo, y á sus gobernadores, diciendo: \bibverse{7} De aquí
adelante no daréis paja al pueblo para hacer ladrillo, como ayer y antes
de ayer; vayan ellos y recojan por sí mismos la paja: \bibverse{8} Y
habéis de ponerles la tarea del ladrillo que hacían antes, y no les
disminuiréis nada; porque están ociosos, y por eso levantan la voz
diciendo: Vamos y sacrificaremos á nuestro Dios. \bibverse{9} Agrávese
la servidumbre sobre ellos, para que se ocupen en ella, y no atiendan á
palabras de mentira. \bibverse{10} Y saliendo los cuadrilleros del
pueblo y sus gobernadores, hablaron al pueblo, diciendo: Así ha dicho
Faraón: Yo no os doy paja. \bibverse{11} Id vosotros, y recoged paja
donde la hallareis; que nada se disminuirá de vuestra tarea.
\bibverse{12} Entonces el pueblo se derramó por toda la tierra de Egipto
á coger rastrojo en lugar de paja. \bibverse{13} Y los cuadrilleros los
apremiaban, diciendo: Acabad vuestra obra, la tarea del día en su día,
como cuando se os daba paja. \bibverse{14} Y azotaban á los capataces de
los hijos de Israel, que los cuadrilleros de Faraón habían puesto sobre
ellos, diciendo: ¿Por qué no habéis cumplido vuestra tarea de ladrillo
ni ayer ni hoy, como antes? \bibverse{15} Y los capataces de los hijos
de Israel vinieron á Faraón, y se quejaron á él, diciendo: ¿Por qué lo
haces así con tus siervos? \bibverse{16} No se da paja á tus siervos, y
con todo nos dicen: Haced el ladrillo. Y he aquí tus siervos son
azotados, y tu pueblo cae en falta. \bibverse{17} Y él respondió: Estáis
ociosos, sí, ociosos, y por eso decís: Vamos y sacrifiquemos á Jehová.
\bibverse{18} Id pues ahora, y trabajad. No se os dará paja, y habéis de
dar la tarea del ladrillo. \bibverse{19} Entonces los capataces de los
hijos de Israel se vieron en aflicción, habiéndoseles dicho: No se
disminuirá nada de vuestro ladrillo, de la tarea de cada día.
\bibverse{20} Y encontrando á Moisés y á Aarón, que estaban á la vista
de ellos cuando salían de Faraón, \bibverse{21} Dijéronles: Mire Jehová
sobre vosotros, y juzgue; pues habéis hecho heder nuestro olor delante
de Faraón y de sus siervos, dándoles el cuchillo en las manos para que
nos maten. \bibverse{22} Entonces Moisés se volvió á Jehová, y dijo:
Señor, ¿por qué afliges á este pueblo? ¿para qué me enviaste?
\bibverse{23} Porque desde que yo vine á Faraón para hablarle en tu
nombre, ha afligido á este pueblo; y tú tampoco has librado á tu pueblo.

\hypertarget{section-5}{%
\section{6}\label{section-5}}

\bibverse{1} Jehová respondió á Moisés: Ahora verás lo que yo haré á
Faraón; porque con mano fuerte los ha de dejar ir; y con mano fuerte los
ha de echar de su tierra. \bibverse{2} Habló todavía Dios á Moisés, y
díjole: Yo soy JEHOVÁ; \bibverse{3} Y aparecí á Abraham, á Isaac y á
Jacob bajo el nombre de Dios Omnipotente, mas en mi nombre JEHOVÁ no me
notifiqué á ellos. \bibverse{4} Y también establecí mi pacto con ellos,
de darles la tierra de Canaán, la tierra en que fueron extranjeros, y en
la cual peregrinaron. \bibverse{5} Y asimismo yo he oído el gemido de
los hijos de Israel, á quienes hacen servir los Egipcios, y heme
acordado de mi pacto. \bibverse{6} Por tanto dirás á los hijos de
Israel: Yo JEHOVÁ; y yo os sacaré de debajo de las cargas de Egipto, y
os libraré de su servidumbre, y os redimiré con brazo extendido, y con
juicios grandes: \bibverse{7} Y os tomaré por mi pueblo y seré vuestro
Dios: y vosotros sabréis que yo soy Jehová vuestro Dios, que os saco de
debajo de las cargas de Egipto: \bibverse{8} Y os meteré en la tierra,
por la cual alcé mi mano que la daría á Abraham, á Isaac y á Jacob: y yo
os la daré por heredad. Yo JEHOVÁ. \bibverse{9} De esta manera habló
Moisés á los hijos de Israel: mas ellos no escuchaban á Moisés á causa
de la congoja de espíritu, y de la dura servidumbre. \bibverse{10} Y
habló Jehová á Moisés, diciendo: \bibverse{11} Entra, y habla á Faraón
rey de Egipto, que deje ir de su tierra á los hijos de Israel.
\bibverse{12} Y respondió Moisés delante de Jehová, diciendo: He aquí,
los hijos de Israel no me escuchan: ¿cómo pues me escuchará Faraón,
mayormente siendo yo incircunciso de labios? \bibverse{13} Entonces
Jehová habló á Moisés y á Aarón, y dióles mandamiento para los hijos de
Israel, y para Faraón rey de Egipto, para que sacasen á los hijos de
Israel de la tierra de Egipto. \bibverse{14} Estas son las cabezas de
las familias de sus padres. Los hijos de Rubén, el primogénito de
Israel: Hanoch y Phallú, Hezrón y Carmi: estas son las familias de
Rubén. \bibverse{15} Los hijos de Simeón: Jemuel, y Jamín, y Ohad, y
Jachîn, y Zoar, y Saúl, hijo de una Cananea: estas son las familias de
Simeón. \bibverse{16} Y estos son los nombres de los hijos de Leví por
sus linajes: Gersón, y Coath, y Merari. Y los años de la vida de Leví
fueron ciento treinta y siete años. \bibverse{17} Y los hijos de Gersón:
Libni, y Shimi, por sus familias. \bibverse{18} Y los hijos de Coath:
Amram, é Izhar, y Hebrón, y Uzziel. Y los años de la vida de Coath
fueron ciento treinta y tres años. \bibverse{19} Y los hijos de Merari:
Mahali, y Musi: estas son las familias de Leví por sus linajes.
\bibverse{20} Y Amram tomó por mujer á Jochêbed su tía; la cual le parió
á Aarón y á Moisés. Y los años de la vida de Amram fueron ciento treinta
y siete años. \bibverse{21} Y los hijos de Izhar: Cora, y Nepheg y
Zithri. \bibverse{22} Y los hijos de Uzziel: Misael, y Elzaphán y
Zithri. \bibverse{23} Y tomóse Aarón por mujer á Elisabeth, hija de
Aminadab, hermana de Naasón; la cual le parió á Nadab, y á Abiú, y á
Eleazar y á Ithamar. \bibverse{24} Y los hijos de Cora: Assir, y Elcana,
y Abiasaph: estas son las familias de los Coritas. \bibverse{25} Y
Eleazar, hijo de Aarón, tomó para sí mujer de las hijas de Phutiel, la
cual le parió á Phinees: y estas son las cabezas de los padres de los
Levitas por sus familias. \bibverse{26} Este es aquel Aarón y aquel
Moisés, á los cuales Jehová dijo: Sacad á los hijos de Israel de la
tierra de Egipto por sus escuadrones. \bibverse{27} Estos son los que
hablaron á Faraón rey de Egipto, para sacar de Egipto á los hijos de
Israel. Moisés y Aarón fueron estos. \bibverse{28} Cuando Jehová habló á
Moisés en la tierra de Egipto, \bibverse{29} Entonces Jehová habló á
Moisés, diciendo: Yo soy JEHOVÁ; di á Faraón rey de Egipto todas las
cosas que yo te digo á ti. \bibverse{30} Y Moisés respondió delante de
Jehová: He aquí, yo soy incircunciso de labios, ¿cómo pues me ha de oir
Faraón?

\hypertarget{section-6}{%
\section{7}\label{section-6}}

\bibverse{1} Y jehová dijo á Moisés: Mira, yo te he constituído dios
para Faraón, y tu hermano Aarón será tu profeta. \bibverse{2} Tú dirás
todas las cosas que yo te mandaré, y Aarón tu hermano hablará á Faraón,
para que deje ir de su tierra á los hijos de Israel. \bibverse{3} Y yo
endureceré el corazón de Faraón, y multiplicaré en la tierra de Egipto
mis señales y mis maravillas. \bibverse{4} Y Faraón no os oirá; mas yo
pondré mi mano sobre Egipto, y sacaré á mis ejércitos, mi pueblo, los
hijos de Israel, de la tierra de Egipto, con grandes juicios.
\bibverse{5} Y sabrán los Egipcios que yo soy Jehová, cuando extenderé
mi mano sobre Egipto, y sacaré los hijos de Israel de en medio de ellos.
\bibverse{6} E hizo Moisés y Aarón como Jehová les mandó: hiciéronlo
así. \bibverse{7} Y era Moisés de edad de ochenta años, y Aarón de edad
de ochenta y tres, cuando hablaron á Faraón. \bibverse{8} Y habló Jehová
á Moisés y á Aarón, diciendo: \bibverse{9} Si Faraón os respondiere
diciendo, Mostrad milagro; dirás á Aarón: Toma tu vara, y échala delante
de Faraón, para que se torne culebra. \bibverse{10} Vinieron, pues,
Moisés y Aarón á Faraón, é hicieron como Jehová lo había mandado: y echó
Aarón su vara delante de Faraón y de sus siervos, y tornóse culebra.
\bibverse{11} Entonces llamó también Faraón sabios y encantadores; é
hicieron también lo mismo los encantadores de Egipto con sus
encantamientos; \bibverse{12} Pues echó cada uno su vara, las cuales se
volvieron culebras: mas la vara de Aarón devoró las varas de ellos.
\bibverse{13} Y el corazón de Faraón se endureció, y no los escuchó;
como Jehová lo había dicho. \bibverse{14} Entonces Jehová dijo á Moisés:
El corazón de Faraón está agravado, que no quiere dejar ir al pueblo.
\bibverse{15} Ve por la mañana á Faraón, he aquí que él sale á las
aguas; y tú ponte á la orilla del río delante de él, y toma en tu mano
la vara que se volvió culebra, \bibverse{16} Y dile: Jehová el Dios de
los Hebreos me ha enviado á ti, diciendo: Deja ir á mi pueblo, para que
me sirvan en el desierto; y he aquí que hasta ahora no has querido oir.
\bibverse{17} Así ha dicho Jehová: En esto conocerás que yo soy Jehová:
he aquí, yo heriré con la vara que tengo en mi mano el agua que está en
el río, y se convertirá en sangre: \bibverse{18} Y los peces que hay en
el río morirán, y hederá el río, y tendrán asco los Egipcios de beber el
agua del río. \bibverse{19} Y Jehová dijo á Moisés: Di á Aarón: Toma tu
vara, y extiende tu mano sobre las aguas de Egipto, sobre sus ríos,
sobre sus arroyos y sobre sus estanques, y sobre todos sus depósitos de
aguas, para que se conviertan en sangre, y haya sangre por toda la
región de Egipto, así en los vasos de madera como en los de piedra.
\bibverse{20} Y Moisés y Aarón hicieron como Jehová lo mandó; y alzando
la vara hirió las aguas que había en el río, en presencia de Faraón y de
sus siervos; y todas las aguas que había en el río se convirtieron en
sangre. \bibverse{21} Asimismo los peces que había en el río murieron; y
el río se corrompió, que los Egipcios no podían beber de él: y hubo
sangre por toda la tierra de Egipto. \bibverse{22} Y los encantadores de
Egipto hicieron lo mismo con sus encantamientos: y el corazón de Faraón
se endureció, y no los escuchó; como Jehová lo había dicho.
\bibverse{23} Y tornando Faraón volvióse á su casa, y no puso su corazón
aun en esto. \bibverse{24} Y en todo Egipto hicieron pozos alrededor del
río para beber, porque no podían beber de las aguas del río.
\bibverse{25} Y cumpliéronse siete días después que Jehová hirió el río.

\hypertarget{section-7}{%
\section{8}\label{section-7}}

\bibverse{1} Entonces Jehová dijo á Moisés: Entra á Faraón, y dile:
Jehová ha dicho así: Deja ir á mi pueblo, para que me sirvan.
\bibverse{2} Y si no lo quisieres dejar ir, he aquí yo heriré con ranas
todos tus términos: \bibverse{3} Y el río criará ranas, las cuales
subirán, y entrarán en tu casa, y en la cámara de tu cama, y sobre tu
cama, y en las casas de tus siervos, y en tu pueblo, y en tus hornos, y
en tus artesas: \bibverse{4} Y las ranas subirán sobre ti, y sobre tu
pueblo, y sobre todos tus siervos. \bibverse{5} Y Jehová dijo á Moisés:
Di á Aarón: Extiende tu mano con tu vara sobre los ríos, arroyos, y
estanques, para que haga venir ranas sobre la tierra de Egipto.
\bibverse{6} Entonces Aarón extendió su mano sobre las aguas de Egipto,
y subieron ranas que cubrieron la tierra de Egipto. \bibverse{7} Y los
encantadores hicieron lo mismo con sus encantamientos, é hicieron venir
ranas sobre la tierra de Egipto. \bibverse{8} Entonces Faraón llamó á
Moisés y á Aarón, y díjoles: Orad á Jehová que quite las ranas de mí y
de mi pueblo; y dejaré ir al pueblo, para que sacrifique á Jehová.
\bibverse{9} Y dijo Moisés á Faraón: Gloríate sobre mí: ¿cuándo oraré
por ti, y por tus siervos, y por tu pueblo, para que las ranas sean
quitadas de ti, y de tus casas, y que solamente se queden en el río?
\bibverse{10} Y él dijo: Mañana. Y Moisés respondió: Se hará conforme á
tu palabra, para que conozcas que no hay como Jehová nuestro Dios:
\bibverse{11} Y las ranas se irán de ti, y de tus casas, y de tus
siervos, y de tu pueblo, y solamente se quedarán en el río.
\bibverse{12} Entonces salieron Moisés y Aarón de con Faraón, y clamó
Moisés á Jehová sobre el negocio de las ranas que había puesto á Faraón.
\bibverse{13} E hizo Jehová conforme á la palabra de Moisés, y murieron
las ranas de las casas, de los cortijos, y de los campos. \bibverse{14}
Y las juntaron en montones, y apestaban la tierra. \bibverse{15} Y
viendo Faraón que le habían dado reposo, agravó su corazón, y no los
escuchó; como Jehová lo había dicho. \bibverse{16} Entonces Jehová dijo
á Moisés: Di á Aarón: Extiende tu vara, y hiere el polvo de la tierra,
para que se vuelva piojos por todo el país de Egipto. \bibverse{17} Y
ellos lo hicieron así; y Aarón extendió su mano con su vara, é hirió el
polvo de la tierra, el cual se volvió piojos, así en los hombres como en
las bestias: todo el polvo de la tierra se volvió piojos en todo el país
de Egipto. \bibverse{18} Y los encantadores hicieron así también, para
sacar piojos con sus encantamientos; mas no pudieron. Y había piojos así
en los hombres como en las bestias. \bibverse{19} Entonces los magos
dijeron á Faraón: Dedo de Dios es este. Mas el corazón de Faraón se
endureció, y no los escuchó; como Jehová lo había dicho. \bibverse{20} Y
Jehová dijo á Moisés: Levántate de mañana y ponte delante de Faraón, he
aquí él sale á las aguas; y dile: Jehová ha dicho así: Deja ir á mi
pueblo, para que me sirva. \bibverse{21} Porque si no dejares ir á mi
pueblo, he aquí yo enviaré sobre ti, y sobre tus siervos, y sobre tu
pueblo, y sobre tus casas toda suerte de moscas; y las casas de los
Egipcios se henchirán de toda suerte de moscas, y asimismo la tierra
donde ellos estuvieren. \bibverse{22} Y aquel día yo apartaré la tierra
de Gosén, en la cual mi pueblo habita, para que ninguna suerte de moscas
haya en ella; á fin de que sepas que yo soy Jehová en medio de la
tierra. \bibverse{23} Y yo pondré redención entre mi pueblo y el tuyo.
Mañana será esta señal. \bibverse{24} Y Jehová lo hizo así: que vino
toda suerte de moscas molestísimas sobre la casa de Faraón, y sobre las
casas de sus siervos, y sobre todo el país de Egipto; y la tierra fué
corrompida á causa de ellas. \bibverse{25} Entonces Faraón llamó á
Moisés y á Aarón, y díjoles: Andad, sacrificad á vuestro Dios en la
tierra. \bibverse{26} Y Moisés respondió: No conviene que hagamos así,
porque sacrificaríamos á Jehová nuestro Dios la abominación de los
Egipcios. He aquí, si sacrificáramos la abominación de los Egipcios
delante de ellos, ¿no nos apedrearían? \bibverse{27} Camino de tres días
iremos por el desierto, y sacrificaremos á Jehová nuestro Dios, como él
nos dirá. \bibverse{28} Y dijo Faraón: Yo os dejaré ir para que
sacrifiquéis á Jehová vuestro Dios en el desierto, con tal que no vayáis
más lejos: orad por mí. \bibverse{29} Y respondió Moisés: He aquí, en
saliendo yo de contigo, rogaré á Jehová que las diversas suertes de
moscas se vayan de Faraón, y de sus siervos, y de su pueblo mañana; con
tal que Faraón no falte más, no dejando ir al pueblo á sacrificar á
Jehová. \bibverse{30} Entonces Moisés salió de con Faraón, y oró á
Jehová. \bibverse{31} Y Jehová hizo conforme á la palabra de Moisés; y
quitó todas aquellas moscas de Faraón, y de sus siervos, y de su pueblo,
sin que quedara una. \bibverse{32} Mas Faraón agravó aún esta vez su
corazón, y no dejó ir al pueblo.

\hypertarget{section-8}{%
\section{9}\label{section-8}}

\bibverse{1} Entonces Jehová dijo á Moisés: Entra á Faraón, y dile:
Jehová, el Dios de los Hebreos, dice así: Deja ir á mi pueblo, para que
me sirvan; \bibverse{2} Porque si no lo quieres dejar ir, y los
detuvieres aún, \bibverse{3} He aquí la mano de Jehová será sobre tus
ganados que están en el campo, caballos, asnos, camellos, vacas y
ovejas, con pestilencia gravísima: \bibverse{4} Y Jehová hará separación
entre los ganados de Israel y los de Egipto, de modo que nada muera de
todo lo de los hijos de Israel. \bibverse{5} Y Jehová señaló tiempo,
diciendo: Mañana hará Jehová esta cosa en la tierra. \bibverse{6} Y el
día siguiente Jehová hizo aquello, y murió todo el ganado de Egipto; mas
del ganado de los hijos de Israel no murió uno. \bibverse{7} Entonces
Faraón envió, y he aquí que del ganado de los hijos de Israel no había
muerto uno. Mas el corazón de Faraón se agravó, y no dejó ir al pueblo.
\bibverse{8} Y Jehová dijo á Moisés y á Aarón: Tomad puñados de ceniza
de un horno, y espárzala Moisés hacia el cielo delante de Faraón:
\bibverse{9} Y vendrá á ser polvo sobre toda la tierra de Egipto, el
cual originará sarpullido que cause tumores apostemados en los hombres y
en las bestias, por todo el país de Egipto. \bibverse{10} Y tomaron la
ceniza del horno, y pusiéronse delante de Faraón, y esparcióla Moisés
hacia el cielo; y vino un sarpullido que causaba tumores apostemados así
en los hombres como en las bestias. \bibverse{11} Y los magos no podían
estar delante de Moisés á causa de los tumores, porque hubo sarpullido
en los magos y en todos los Egipcios. \bibverse{12} Y Jehová endureció
el corazón de Faraón, y no los oyó; como Jehová lo había dicho á Moisés.
\bibverse{13} Entonces Jehová dijo á Moisés: Levántate de mañana, y
ponte delante de Faraón, y dile: Jehová, el Dios de los Hebreos, dice
así: Deja ir á mi pueblo, para que me sirva. \bibverse{14} Porque yo
enviaré esta vez todas mis plagas á tu corazón, sobre tus siervos, y
sobre tu pueblo, para que entiendas que no hay otro como yo en toda la
tierra. \bibverse{15} Porque ahora yo extenderé mi mano para herirte á
ti y á tu pueblo de pestilencia, y serás quitado de la tierra.
\bibverse{16} Y á la verdad yo te he puesto para declarar en ti mi
potencia, y que mi Nombre sea contado en toda la tierra. \bibverse{17}
¿Todavía te ensalzas tú contra mi pueblo, para no dejarlos ir?
\bibverse{18} He aquí que mañana á estas horas yo haré llover granizo
muy grave, cual nunca fué en Egipto, desde el día que se fundó hasta
ahora. \bibverse{19} Envía, pues, á recoger tu ganado, y todo lo que
tienes en el campo; porque todo hombre ó animal que se hallare en el
campo, y no fuere recogido á casa, el granizo descenderá sobre él, y
morirá. \bibverse{20} De los siervos de Faraón el que temió la palabra
de Jehová, hizo huir sus criados y su ganado á casa: \bibverse{21} Mas
el que no puso en su corazón la palabra de Jehová, dejó sus criados y
sus ganados en el campo. \bibverse{22} Y Jehová dijo á Moisés: Extiende
tu mano hacia el cielo, para que venga granizo en toda la tierra de
Egipto sobre los hombres, y sobre las bestias, y sobre toda la hierba
del campo en el país de Egipto. \bibverse{23} Y Moisés extendió su vara
hacia el cielo, y Jehová hizo tronar y granizar, y el fuego discurría
por la tierra; y llovió Jehová granizo sobre la tierra de Egipto.
\bibverse{24} Hubo pues granizo, y fuego mezclado con el granizo, tan
grande, cual nunca hubo en toda la tierra de Egipto desde que fué
habitada. \bibverse{25} Y aquel granizo hirió en toda la tierra de
Egipto todo lo que estaba en el campo, así hombres como bestias;
asimismo hirió el granizo toda la hierba del campo, y desgajó todos los
árboles del país. \bibverse{26} Solamente en la tierra de Gosén, donde
los hijos de Israel estaban, no hubo granizo. \bibverse{27} Entonces
Faraón envió á llamar á Moisés y á Aarón, y les dijo: He pecado esta
vez: Jehová es justo, y yo y mi pueblo impíos. \bibverse{28} Orad á
Jehová: y cesen los truenos de Dios y el granizo; y yo os dejaré ir, y
no os detendréis más. \bibverse{29} Y respondióle Moisés: En saliendo yo
de la ciudad extenderé mis manos á Jehová, y los truenos cesarán, y no
habrá más granizo; para que sepas que de Jehová es la tierra.
\bibverse{30} Mas yo sé que ni tú ni tus siervos temeréis todavía la
presencia del Dios Jehová. \bibverse{31} El lino, pues, y la cebada
fueron heridos; porque la cebada estaba ya espigada, y el lino en caña.
\bibverse{32} Mas el trigo y el centeno no fueron heridos; porque eran
tardíos. \bibverse{33} Y salido Moisés de con Faraón de la ciudad,
extendió sus manos á Jehová, y cesaron los truenos y el granizo; y la
lluvia no cayó más sobre la tierra. \bibverse{34} Y viendo Faraón que la
lluvia había cesado y el granizo y los truenos, perseveró en pecar, y
agravó su corazón, él y sus siervos. \bibverse{35} Y el corazón de
Faraón se endureció, y no dejó ir á los hijos de Israel; como Jehová lo
había dicho por medio de Moisés.

\hypertarget{section-9}{%
\section{10}\label{section-9}}

\bibverse{1} Y jehová dijo á Moisés: Entra á Faraón; porque yo he
agravado su corazón, y el corazón de sus siervos, para dar entre ellos
estas mis señales; \bibverse{2} Y para que cuentes á tus hijos y á tus
nietos las cosas que yo hice en Egipto, y mis señales que dí entre
ellos; y para que sepáis que yo soy Jehová. \bibverse{3} Entonces
vinieron Moisés y Aarón á Faraón, y le dijeron: Jehová, el Dios de los
Hebreos, ha dicho así: ¿Hasta cuándo no querrás humillarte delante de
mí? Deja ir á mi pueblo para que me sirvan. \bibverse{4} Y si aun
rehusas dejarlo ir, he aquí que yo traeré mañana langosta en tus
términos, \bibverse{5} La cual cubrirá la faz de la tierra, de modo que
no pueda verse la tierra; y ella comerá lo que quedó salvo, lo que os ha
quedado del granizo; comerá asimismo todo árbol que os produce fruto en
el campo: \bibverse{6} Y llenarse han tus casas, y las casas de todos
tus siervos, y las casas de todos los Egipcios, cual nunca vieron tus
padres ni tus abuelos, desde que ellos fueron sobre la tierra hasta hoy.
Y volvióse, y salió de con Faraón. \bibverse{7} Entonces los siervos de
Faraón le dijeron: ¿Hasta cuándo nos ha de ser éste por lazo? Deja ir á
estos hombres, para que sirvan á Jehová su Dios; ¿aun no sabes que
Egipto está destruído? \bibverse{8} Y Moisés y Aarón volvieron á ser
llamados á Faraón, el cual les dijo: Andad, servid á Jehová vuestro
Dios. ¿Quién y quién son los que han de ir? \bibverse{9} Y Moisés
respondió: Hemos de ir con nuestros niños y con nuestros viejos, con
nuestros hijos y con nuestras hijas: con nuestras ovejas y con nuestras
vacas hemos de ir; porque tenemos solemnidad de Jehová. \bibverse{10} Y
él les dijo: Así sea Jehová con vosotros como yo os dejaré ir á vosotros
y á vuestros niños: mirad como el mal está delante de vuestro rostro.
\bibverse{11} No será así: id ahora vosotros los varones, y servid á
Jehová: pues esto es lo que vosotros demandasteis. Y echáronlos de
delante de Faraón. \bibverse{12} Entonces Jehová dijo á Moisés: Extiende
tu mano sobre la tierra de Egipto para langosta, á fin de que suba sobre
el país de Egipto, y consuma todo lo que el granizo dejó. \bibverse{13}
Y extendió Moisés su vara sobre la tierra de Egipto, y Jehová trajo un
viento oriental sobre el país todo aquel día y toda aquella noche; y á
la mañana el viento oriental trajo la langosta: \bibverse{14} Y subió la
langosta sobre toda la tierra de Egipto, y asentóse en todos los
términos de Egipto, en gran manera grave: antes de ella no hubo langosta
semejante, ni después de ella vendrá otra tal; \bibverse{15} Y cubrió la
faz de todo el país, y oscurecióse la tierra; y consumió toda la hierba
de la tierra, y todo el fruto de los árboles que había dejado el
granizo; que no quedó cosa verde en árboles ni en hierba del campo, por
toda la tierra de Egipto. \bibverse{16} Entonces Faraón hizo llamar
apriesa á Moisés y á Aarón, y dijo: He pecado contra Jehová vuestro
Dios, y contra vosotros. \bibverse{17} Mas ruego ahora que perdones mi
pecado solamente esta vez, y que oréis á Jehová vuestro Dios que quite
de mí solamente esta muerte. \bibverse{18} Y salió de con Faraón, y oró
á Jehová. \bibverse{19} Y Jehová volvió un viento occidental fortísimo,
y quitó la langosta, y arrojóla en el mar Bermejo: ni una langosta quedó
en todo el término de Egipto. \bibverse{20} Mas Jehová endureció el
corazón de Faraón; y no envió los hijos de Israel. \bibverse{21} Y
Jehová dijo á Moisés: Extiende tu mano hacia el cielo, para que haya
tinieblas sobre la tierra de Egipto, tales que cualquiera las palpe.
\bibverse{22} Y extendió Moisés su mano hacia el cielo, y hubo densas
tinieblas tres días por toda la tierra de Egipto. \bibverse{23} Ninguno
vió á su prójimo, ni nadie se levantó de su lugar en tres días; mas
todos los hijos de Israel tenían luz en sus habitaciones. \bibverse{24}
Entonces Faraón hizo llamar á Moisés, y dijo: Id, servid á Jehová;
solamente queden vuestras ovejas y vuestras vacas: vayan también
vuestros niños con vosotros. \bibverse{25} Y Moisés respondió: Tú
también nos entregarás sacrificios y holocaustos que sacrifiquemos á
Jehová nuestro Dios. \bibverse{26} Nuestros ganados irán también con
nosotros; no quedará ni una uña; porque de ellos hemos de tomar para
servir á Jehová nuestro Dios; y no sabemos con qué hemos de servir á
Jehová, hasta que lleguemos allá. \bibverse{27} Mas Jehová endureció el
corazón de Faraón, y no quiso dejarlos ir. \bibverse{28} Y díjole
Faraón: Retírate de mí: guárdate que no veas más mi rostro, porque en
cualquier día que vieres mi rostro, morirás. \bibverse{29} Y Moisés
respondió: Bien has dicho; no veré más tu rostro.

\hypertarget{section-10}{%
\section{11}\label{section-10}}

\bibverse{1} Y jehová dijo á Moisés: Una plaga traeré aún sobre Faraón,
y sobre Egipto; después de la cual él os dejará ir de aquí; y
seguramente os echará de aquí del todo. \bibverse{2} Habla ahora al
pueblo, y que cada uno demande á su vecino, y cada una á su vecina,
vasos de plata y de oro. \bibverse{3} Y Jehová dió gracia al pueblo en
los ojos de los Egipcios. También Moisés era muy gran varón en la tierra
de Egipto, á los ojos de los siervos de Faraón, y á los ojos del pueblo.
\bibverse{4} Y dijo Moisés: Jehová ha dicho así: A la media noche yo
saldré por medio de Egipto, \bibverse{5} Y morirá todo primogénito en
tierra de Egipto, desde el primogénito de Faraón que se sienta en su
trono, hasta el primogénito de la sierva que está tras la muela; y todo
primogénito de las bestias. \bibverse{6} Y habrá gran clamor por toda la
tierra de Egipto, cual nunca fué, ni jamás será. \bibverse{7} Mas entre
todos los hijos de Israel, desde el hombre hasta la bestia, ni un perro
moverá su lengua: para que sepáis que hará diferencia Jehová entre los
Egipcios y los Israelitas. \bibverse{8} Y descenderán á mí todos estos
tus siervos, é inclinados delante de mí dirán: Sal tú, y todo el pueblo
que está bajo de ti; y después de esto yo saldré. Y salióse muy enojado
de con Faraón. \bibverse{9} Y Jehová dijo á Moisés: Faraón no os oirá,
para que mis maravillas se multipliquen en la tierra de Egipto.
\bibverse{10} Y Moisés y Aarón hicieron todos estos prodigios delante de
Faraón: mas Jehová había endurecido el corazón de Faraón, y no envió á
los hijos de Israel fuera de su país.

\hypertarget{section-11}{%
\section{12}\label{section-11}}

\bibverse{1} Y habló Jehová á Moisés y á Aarón en la tierra de Egipto,
diciendo: \bibverse{2} Este mes os será principio de los meses; será
este para vosotros el primero en los meses del año. \bibverse{3} Hablad
á toda la congregación de Israel, diciendo: En el diez de aqueste mes
tómese cada uno un cordero por las familias de los padres, un cordero
por familia: \bibverse{4} Mas si la familia fuere pequeña que no baste á
comer el cordero, entonces tomará á su vecino inmediato á su casa, y
según el número de las personas, cada uno conforme á su comer, echaréis
la cuenta sobre el cordero. \bibverse{5} El cordero será sin defecto,
macho de un año: tomaréislo de las ovejas ó de las cabras: \bibverse{6}
Y habéis de guardarlo hasta el día catorce de este mes; y lo inmolará
toda la congregación del pueblo de Israel entre las dos tardes.
\bibverse{7} Y tomarán de la sangre, y pondrán en los dos postes y en el
dintel de las casas en que lo han de comer. \bibverse{8} Y aquella noche
comerán la carne asada al fuego, y panes sin levadura: con hierbas
amargas lo comerán. \bibverse{9} Ninguna cosa comeréis de él cruda, ni
cocida en agua, sino asada al fuego; su cabeza con sus pies y sus
intestinos. \bibverse{10} Ninguna cosa dejaréis de él hasta la mañana; y
lo que habrá quedado hasta la mañana, habéis de quemarlo en el fuego.
\bibverse{11} Y así habéis de comerlo: ceñidos vuestros lomos, vuestros
zapatos en vuestros pies, y vuestro bordón en vuestra mano; y lo
comeréis apresuradamente: es la Pascua de Jehová. \bibverse{12} Pues yo
pasaré aquella noche por la tierra de Egipto, y heriré á todo
primogénito en la tierra de Egipto, así en los hombres como en las
bestias: y haré juicios en todos los dioses de Egipto. Yo JEHOVÁ.
\bibverse{13} Y la sangre os será por señal en las casas donde vosotros
estéis; y veré la sangre, y pasaré de vosotros, y no habrá en vosotros
plaga de mortandad, cuando heriré la tierra de Egipto. \bibverse{14} Y
este día os ha de ser en memoria, y habéis de celebrarlo como solemne á
Jehová durante vuestras generaciones: por estatuto perpetuo lo
celebraréis. \bibverse{15} Siete días comeréis panes sin levadura; y así
el primer día haréis que no haya levadura en vuestras casas: porque
cualquiera que comiere leudado desde el primer día hasta el séptimo,
aquella alma será cortada de Israel. \bibverse{16} El primer día habrá
santa convocación, y asimismo en el séptimo día tendréis una santa
convocación: ninguna obra se hará en ellos, excepto solamente que
aderecéis lo que cada cual hubiere de comer. \bibverse{17} Y guardaréis
la fiesta de los ázimos, porque en aqueste mismo día saqué vuestros
ejércitos de la tierra de Egipto: por tanto guardaréis este día en
vuestras generaciones por costumbre perpetua. \bibverse{18} En el mes
primero, el día catorce del mes por la tarde, comeréis los panes sin
levadura, hasta el veintiuno del mes por la tarde. \bibverse{19} Por
siete días no se hallará levadura en vuestras casas, porque cualquiera
que comiere leudado, así extranjero como natural del país, aquella alma
será cortada de la congregación de Israel. \bibverse{20} Ninguna cosa
leudada comeréis; en todas vuestras habitaciones comeréis panes sin
levadura. \bibverse{21} Y Moisés convocó á todos los ancianos de Israel,
y díjoles: Sacad, y tomaos corderos por vuestras familias, y sacrificad
la pascua. \bibverse{22} Y tomad un manojo de hisopo, y mojadle en la
sangre que estará en una jofaina, y untad el dintel y los dos postes con
la sangre que estará en la jofaina; y ninguno de vosotros salga de las
puertas de su casa hasta la mañana. \bibverse{23} Porque Jehová pasará
hiriendo á los Egipcios; y como verá la sangre en el dintel y en los dos
postes, pasará Jehová aquella puerta, y no dejará entrar al heridor en
vuestras casas para herir. \bibverse{24} Y guardaréis esto por estatuto
para vosotros y para vuestros hijos para siempre. \bibverse{25} Y será,
cuando habréis entrado en la tierra que Jehová os dará, como tiene
hablado, que guardaréis este rito. \bibverse{26} Y cuando os dijeren
vuestros hijos: ¿Qué rito es este vuestro? \bibverse{27} Vosotros
responderéis: Es la víctima de la Pascua de Jehová, el cual pasó las
casas de los hijos de Israel en Egipto, cuando hirió á los Egipcios, y
libró nuestras casas. Entonces el pueblo se inclinó y adoró.
\bibverse{28} Y los hijos de Israel se fueron, é hicieron puntualmente
así; como Jehová había mandado á Moisés y á Aarón. \bibverse{29} Y
aconteció que á la medianoche Jehová hirió á todo primogénito en la
tierra de Egipto, desde el primogénito de Faraón que se sentaba sobre su
trono, hasta el primogénito del cautivo que estaba en la cárcel, y todo
primogénito de los animales. \bibverse{30} Y levantóse aquella noche
Faraón, él y todos sus siervos, y todos los Egipcios; y había un gran
clamor en Egipto, porque no había casa donde no hubiese muerto.
\bibverse{31} E hizo llamar á Moisés y á Aarón de noche, y díjoles:
Salid de en medio de mi pueblo vosotros, y los hijos de Israel; é id,
servid á Jehová, como habéis dicho. \bibverse{32} Tomad también vuestras
ovejas y vuestras vacas, como habéis dicho, é idos; y bendecidme también
á mí. \bibverse{33} Y los Egipcios apremiaban al pueblo, dándose priesa
á echarlos de la tierra; porque decían: Todos somos muertos.
\bibverse{34} Y llevó el pueblo su masa antes que se leudase, sus masas
envueltas en sus sábanas sobre sus hombros. \bibverse{35} E hicieron los
hijos de Israel conforme al mandamiento de Moisés, demandando á los
Egipcios vasos de plata, y vasos de oro, y vestidos. \bibverse{36} Y
Jehová dió gracia al pueblo delante de los Egipcios, y prestáronles; y
ellos despojaron á los Egipcios. \bibverse{37} Y partieron los hijos de
Israel de Rameses á Succoth, como seiscientos mil hombres de á pie, sin
contar los niños. \bibverse{38} Y también subió con ellos grande
multitud de diversa suerte de gentes; y ovejas, y ganados muy muchos.
\bibverse{39} Y cocieron tortas sin levadura de la masa que habían
sacado de Egipto; porque no había leudado, por cuanto echándolos los
Egipcios, no habían podido detenerse, ni aun prepararse comida.
\bibverse{40} El tiempo que los hijos de Israel habitaron en Egipto, fué
cuatrocientos treinta años. \bibverse{41} Y pasados cuatrocientos
treinta años, en el mismo día salieron todos los ejércitos de Jehová de
la tierra de Egipto. \bibverse{42} Es noche de guardar á Jehová, por
haberlos sacado en ella de la tierra de Egipto. Esta noche deben guardar
á Jehová todos los hijos de Israel en sus generaciones. \bibverse{43} Y
Jehová dijo á Moisés y á Aarón: Esta es la ordenanza de la Pascua:
Ningún extraño comerá de ella: \bibverse{44} Mas todo siervo humano
comprado por dinero, comerá de ella después que lo hubieres
circuncidado. \bibverse{45} El extranjero y el asalariado no comerán de
ella. \bibverse{46} En una casa se comerá, y no llevarás de aquella
carne fuera de casa, ni quebraréis hueso suyo. \bibverse{47} Toda la
congregación de Israel le sacrificará. \bibverse{48} Mas si algún
extranjero peregrinare contigo, y quisiere hacer la pascua á Jehová,
séale circuncidado todo varón, y entonces se llegará á hacerla, y será
como el natural de la tierra; pero ningún incircunciso comerá de ella.
\bibverse{49} La misma ley será para el natural y para el extranjero que
peregrinare entre vosotros. \bibverse{50} Así lo hicieron todos los
hijos de Israel; como mandó Jehová á Moisés y á Aarón, así lo hicieron.
\bibverse{51} Y en aquel mismo día sacó Jehová á los hijos de Israel de
la tierra de Egipto por sus escuadrones.

\hypertarget{section-12}{%
\section{13}\label{section-12}}

\bibverse{1} Y jehová habló á Moisés, diciendo: \bibverse{2} Santifícame
todo primogénito, cualquiera que abre matriz entre los hijos de Israel,
así de los hombres como de los animales: mío es. \bibverse{3} Y Moisés
dijo al pueblo: Tened memoria de aqueste día, en el cual habéis salido
de Egipto, de la casa de servidumbre; pues Jehová os ha sacado de aquí
con mano fuerte: por tanto, no comeréis leudado. \bibverse{4} Vosotros
salís hoy en el mes de Abib. \bibverse{5} Y cuando Jehová te hubiere
metido en la tierra del Cananeo, y del Hetheo, y del Amorrheo, y del
Hebeo, y del Jebuseo, la cual juró á tus padres que te daría, tierra que
destila leche y miel, harás este servicio en aqueste mes. \bibverse{6}
Siete días comerás por leudar, y el séptimo día será fiesta á Jehová.
\bibverse{7} Por los siete días se comerán los panes sin levadura; y no
se verá contigo leudado, ni levadura en todo tu término. \bibverse{8} Y
contarás en aquel día á tu hijo, diciendo: Hácese esto con motivo de lo
que Jehová hizo conmigo cuando me sacó de Egipto. \bibverse{9} Y serte
ha como una señal sobre tu mano, y como una memoria delante de tus ojos,
para que la ley de Jehová esté en tu boca; por cuanto con mano fuerte te
sacó Jehová de Egipto. \bibverse{10} Por tanto, tú guardarás este rito
en su tiempo de año en año. \bibverse{11} Y cuando Jehová te hubiere
metido en la tierra del Cananeo, como te ha jurado á ti y á tus padres,
y cuando te la hubiere dado, \bibverse{12} Harás pasar á Jehová todo lo
que abriere la matriz, asimismo todo primerizo que abriere la matriz de
tus animales: los machos serán de Jehová. \bibverse{13} Mas todo
primogénito de asno redimirás con un cordero; y si no lo redimieres, le
degollarás: asimismo redimirás todo humano primogénito de tus hijos.
\bibverse{14} Y cuando mañana te preguntare tu hijo, diciendo: ¿Qué es
esto? decirle has: Jehová nos sacó con mano fuerte de Egipto, de casa de
servidumbre; \bibverse{15} Y endureciéndose Faraón en no dejarnos ir,
Jehová mató en la tierra de Egipto á todo primogénito, desde el
primogénito humano hasta el primogénito de la bestia: y por esta causa
yo sacrifico á Jehová todo primogénito macho, y redimo todo primogénito
de mis hijos. \bibverse{16} Serte ha, pues, como una señal sobre tu
mano, y por una memoria delante de tus ojos; ya que Jehová nos sacó de
Egipto con mano fuerte. \bibverse{17} Y luego que Faraón dejó ir al
pueblo, Dios no los llevó por el camino de la tierra de los Filisteos,
que estaba cerca; porque dijo Dios: Que quizá no se arrepienta el pueblo
cuando vieren la guerra, y se vuelvan á Egipto: \bibverse{18} Mas hizo
Dios al pueblo que rodease por el camino del desierto del mar Bermejo. Y
subieron los hijos de Israel de Egipto armados. \bibverse{19} Tomó
también consigo Moisés los huesos de José, el cual había juramentado á
los hijos de Israel, diciendo: Dios ciertamente os visitará, y haréis
subir mis huesos de aquí con vosotros. \bibverse{20} Y partidos de
Succoth, asentaron campo en Etham, á la entrada del desierto.
\bibverse{21} Y Jehová iba delante de ellos de día en una columna de
nube, para guiarlos por el camino; y de noche en una columna de fuego
para alumbrarles; á fin de que anduviesen de día y de noche.
\bibverse{22} Nunca se partió de delante del pueblo la columna de nube
de día, ni de noche la columna de fuego.

\hypertarget{section-13}{%
\section{14}\label{section-13}}

\bibverse{1} Y habló Jehová á Moisés, diciendo: \bibverse{2} Habla á los
hijos de Israel que den la vuelta, y asienten su campo delante de
Pihahiroth, entre Migdol y la mar hacia Baalzephón: delante de él
asentaréis el campo, junto á la mar. \bibverse{3} Porque Faraón dirá de
los hijos de Israel: Encerrados están en la tierra, el desierto los ha
encerrado. \bibverse{4} Y yo endureceré el corazón de Faraón para que
los siga; y seré glorificado en Faraón y en todo su ejército; y sabrán
los Egipcios que yo soy Jehová. Y ellos lo hicieron así. \bibverse{5} Y
fué dado aviso al rey de Egipto cómo el pueblo se huía: y el corazón de
Faraón y de sus siervos se volvió contra el pueblo, y dijeron: ¿Cómo
hemos hecho esto de haber dejado ir á Israel, para que no nos sirva?
\bibverse{6} Y unció su carro, y tomó consigo su pueblo; \bibverse{7} Y
tomó seiscientos carros escogidos, y todos los carros de Egipto, y los
capitanes sobre ellos. \bibverse{8} Y endureció Jehová el corazón de
Faraón rey de Egipto, y siguió á los hijos de Israel; pero los hijos de
Israel habían salido con mano poderosa. \bibverse{9} Siguiéndolos, pues,
los Egipcios, con toda la caballería y carros de Faraón, su gente de á
caballo, y todo su ejército, alcanzáronlos asentando el campo junto á la
mar, al lado de Pihahiroth, delante de Baalzephón. \bibverse{10} Y
cuando Faraón se hubo acercado, los hijos de Israel alzaron sus ojos, y
he aquí los Egipcios que venían tras ellos; por lo que temieron en gran
manera, y clamaron los hijos de Israel á Jehová. \bibverse{11} Y dijeron
á Moisés: ¿No había sepulcros en Egipto, que nos has sacado para que
muramos en el desierto? ¿Por qué lo has hecho así con nosotros, que nos
has sacado de Egipto? \bibverse{12} ¿No es esto lo que te hablamos en
Egipto, diciendo: Déjanos servir á los Egipcios? Que mejor nos fuera
servir á los Egipcios, que morir nosotros en el desierto. \bibverse{13}
Y Moisés dijo al pueblo: No temáis; estáos quedos, y ved la salud de
Jehová, que él hará hoy con vosotros; porque los Egipcios que hoy habéis
visto, nunca más para siempre los veréis. \bibverse{14} Jehová peleará
por vosotros, y vosotros estaréis quedos. \bibverse{15} Entonces Jehová
dijo á Moisés: ¿Por qué clamas á mí? di á los hijos de Israel que
marchen: \bibverse{16} Y tú alza tu vara, y extiende tu mano sobre la
mar, y divídela; y entren los hijos de Israel por medio de la mar en
seco. \bibverse{17} Y yo, he aquí yo endureceré el corazón de los
Egipcios, para que los sigan: y yo me glorificaré en Faraón, y en todo
su ejército, y en sus carros, y en su caballería; \bibverse{18} Y sabrán
los Egipcios que yo soy Jehová, cuando me glorificaré en Faraón, en sus
carros, y en su gente de á caballo. \bibverse{19} Y el ángel de Dios que
iba delante del campo de Israel, se apartó, é iba en pos de ellos; y
asimismo la columna de nube que iba delante de ellos, se apartó, y
púsose á sus espaldas: \bibverse{20} E iba entre el campo de los
Egipcios y el campo de Israel; y era nube y tinieblas para aquéllos, y
alumbraba á Israel de noche: y en toda aquella noche nunca llegaron los
unos á los otros. \bibverse{21} Y extendió Moisés su mano sobre la mar,
é hizo Jehová que la mar se retirase por recio viento oriental toda
aquella noche; y tornó la mar en seco, y las aguas quedaron divididas.
\bibverse{22} Entonces los hijos de Israel entraron por medio de la mar
en seco, teniendo las aguas como muro á su diestra y á su siniestra:
\bibverse{23} Y siguiéndolos los Egipcios, entraron tras ellos hasta el
medio de la mar, toda la caballería de Faraón, sus carros, y su gente de
á caballo. \bibverse{24} Y aconteció á la vela de la mañana, que Jehová
miró al campo de los Egipcios desde la columna de fuego y nube, y
perturbó el campo de los Egipcios. \bibverse{25} Y quitóles las ruedas
de sus carros, y trastornólos gravemente. Entonces los Egipcios dijeron:
Huyamos de delante de Israel, porque Jehová pelea por ellos contra los
Egipcios. \bibverse{26} Y Jehová dijo á Moisés: Extiende tu mano sobre
la mar, para que las aguas vuelvan sobre los Egipcios, sobre sus carros,
y sobre su caballería. \bibverse{27} Y Moisés extendió su mano sobre la
mar, y la mar se volvió en su fuerza cuando amanecía; y los Egipcios
iban hacia ella: y Jehová derribó á los Egipcios en medio de la mar.
\bibverse{28} Y volvieron las aguas, y cubrieron los carros y la
caballería, y todo el ejército de Faraón que había entrado tras ellos en
la mar; no quedó de ellos ni uno. \bibverse{29} Y los hijos de Israel
fueron por medio de la mar en seco, teniendo las aguas por muro á su
diestra y á su siniestra. \bibverse{30} Así salvó Jehová aquel día á
Israel de mano de los Egipcios; é Israel vió á los Egipcios muertos á la
orilla de la mar. \bibverse{31} Y vió Israel aquel grande hecho que
Jehová ejecutó contra los Egipcios: y el pueblo temió á Jehová, y
creyeron á Jehová y á Moisés su siervo.

\hypertarget{section-14}{%
\section{15}\label{section-14}}

\bibverse{1} Entonces cantó Moisés y los hijos de Israel este cántico á
Jehová, y dijeron: Cantaré yo á Jehová, porque se ha magnificado
grandemente, echando en la mar al caballo y al que en él subía.
\bibverse{2} Jehová es mi fortaleza, y mi canción, y hame sido por
salud: éste es mi Dios, y á éste engrandeceré; Dios de mi padre, y á
éste ensalzaré. \bibverse{3} Jehová, varón de guerra; Jehová es su
nombre. \bibverse{4} Los carros de Faraón y á su ejército echó en la
mar; y sus escogidos príncipes fueron hundidos en el mar Bermejo.
\bibverse{5} Los abismos los cubrieron; como piedra descendieron á los
profundos. \bibverse{6} Tu diestra, oh Jehová, ha sido magnificada en
fortaleza; tu diestra, oh Jehová, ha quebrantado al enemigo.
\bibverse{7} Y con la grandeza de tu poder has trastornado á los que se
levantaron contra ti: enviaste tu furor; los tragó como á hojarasca.
\bibverse{8} Con el soplo de tus narices se amontonaron las aguas;
paráronse las corrientes como en un montón; los abismos se cuajaron en
medio de la mar. \bibverse{9} El enemigo dijo: Perseguiré, prenderé,
repartiré despojos; mi alma se henchirá de ellos; sacaré mi espada,
destruirlos ha mi mano. \bibverse{10} Soplaste con tu viento, cubriólos
la mar: hundiéronse como plomo en las impetuosas aguas. \bibverse{11}
¿Quién como tú, Jehová, entre los dioses? ¿quién como tú, magnífico en
santidad, terrible en loores, hacedor de maravillas? \bibverse{12}
Extendiste tu diestra; la tierra los tragó. \bibverse{13} Condujiste en
tu misericordia á este pueblo, al cual salvaste; llevástelo con tu
fortaleza á la habitación de tu santuario. \bibverse{14} Oiránlo los
pueblos, y temblarán; apoderarse ha dolor de los moradores de Palestina.
\bibverse{15} Entonces los príncipes de Edom se turbarán; á los robustos
de Moab los ocupará temblor; abatirse han todos los moradores de Canaán.
\bibverse{16} Caiga sobre ellos temblor y espanto; á la grandeza de tu
brazo enmudezcan como una piedra; hasta que haya pasado tu pueblo, oh
Jehová, hasta que haya pasado este pueblo que tú rescataste.
\bibverse{17} Tú los introducirás y los plantarás en el monte de tu
heredad, en el lugar de tu morada, que tú has aparejado, oh Jehová; en
el santuario del Señor, que han afirmado tus manos. \bibverse{18} Jehová
reinará por los siglos de los siglos. \bibverse{19} Porque Faraón entró
cabalgando con sus carros y su gente de á caballo en la mar, y Jehová
volvió á traer las aguas de la mar sobre ellos; mas los hijos de Israel
fueron en seco por medio de la mar. \bibverse{20} Y María la profetisa,
hermana de Aarón, tomó un pandero en su mano, y todas las mujeres
salieron en pos de ella con panderos y danzas; \bibverse{21} Y María les
respondía: Cantad á Jehová; porque en extremo se ha engrandecido,
echando en la mar al caballo, y al que en él subía. \bibverse{22} E hizo
Moisés que partiese Israel del mar Bermejo, y salieron al desierto de
Shur; y anduvieron tres días por el desierto sin hallar agua.
\bibverse{23} Y llegaron á Mara, y no pudieron beber las aguas de Mara,
porque eran amargas; por eso le pusieron el nombre de Mara.
\bibverse{24} Entonces el pueblo murmuró contra Moisés, y dijo: ¿Qué
hemos de beber? \bibverse{25} Y Moisés clamó á Jehová; y Jehová le
mostró un árbol, el cual metídolo que hubo dentro de las aguas, las
aguas se endulzaron. Allí les dió estatutos y ordenanzas, y allí los
probó; \bibverse{26} Y dijo: Si oyeres atentamente la voz de Jehová tu
Dios, é hicieres lo recto delante de sus ojos, y dieres oído á sus
mandamientos, y guardares todos sus estatutos, ninguna enfermedad de las
que envié á los Egipcios te enviaré á ti; porque yo soy Jehová tu
Sanador. \bibverse{27} Y llegaron á Elim, donde había doce fuentes de
aguas, y setenta palmas; y asentaron allí junto á las aguas.

\hypertarget{section-15}{%
\section{16}\label{section-15}}

\bibverse{1} Y partiendo de Elim toda la congregación de los hijos de
Israel, vino al desierto de Sin, que está entre Elim y Sinaí, á los
quince días del segundo mes después que salieron de la tierra de Egipto.
\bibverse{2} Y toda la congregación de los hijos de Israel murmuró
contra Moisés y Aarón en el desierto; \bibverse{3} Y decíanles los hijos
de Israel: Ojalá hubiéramos muerto por mano de Jehová en la tierra de
Egipto, cuando nos sentábamos á las ollas de las carnes, cuando comíamos
pan en hartura; pues nos habéis sacado á este desierto, para matar de
hambre á toda esta multitud. \bibverse{4} Y Jehová dijo á Moisés: He
aquí yo os haré llover pan del cielo; y el pueblo saldrá, y cogerá para
cada un día, para que yo le pruebe si anda en mi ley, ó no. \bibverse{5}
Mas al sexto día aparejarán lo que han de encerrar, que será el doble de
lo que solían coger cada día. \bibverse{6} Entonces dijo Moisés y Aarón
á todos los hijos de Israel: A la tarde sabréis que Jehová os ha sacado
de la tierra de Egipto: \bibverse{7} Y á la mañana veréis la gloria de
Jehová; porque él ha oído vuestras murmuraciones contra Jehová; que
nosotros, ¿qué somos, para que vosotros murmuréis contra nosotros?
\bibverse{8} Y dijo Moisés: Jehová os dará á la tarde carne para comer,
y á la mañana pan en hartura; por cuanto Jehová ha oído vuestras
murmuraciones con que habéis murmurado contra él: que nosotros, ¿qué
somos? vuestras murmuraciones no son contra nosotros, sino contra
Jehová. \bibverse{9} Y dijo Moisés á Aarón: Di á toda la congregación de
los hijos de Israel: Acercaos á la presencia de Jehová; que él ha oído
vuestras murmuraciones. \bibverse{10} Y hablando Aarón á toda la
congregación de los hijos de Israel, miraron hacia el desierto, y he
aquí la gloria de Jehová, que apareció en la nube. \bibverse{11} Y
Jehová habló á Moisés, diciendo: \bibverse{12} Yo he oído las
murmuraciones de los hijos de Israel; háblales, diciendo: Entre las dos
tardes comeréis carne, y por la mañana os hartaréis de pan, y sabréis
que yo soy Jehová vuestro Dios. \bibverse{13} Y venida la tarde subieron
codornices que cubrieron el real; y á la mañana descendió rocío en
derredor del real. \bibverse{14} Y como el rocío cesó de descender, he
aquí sobre la haz del desierto una cosa menuda, redonda, menuda como una
helada sobre la tierra. \bibverse{15} Y viéndolo los hijos de Israel, se
dijeron unos á otros: ¿Qué es esto? porque no sabían qué era. Entonces
Moisés les dijo: Es el pan que Jehová os da para comer. \bibverse{16}
Esto es lo que Jehová ha mandado: cogeréis de él cada uno según pudiere
comer; un gomer por cabeza, conforme al número de vuestras personas,
tomaréis cada uno para los que están en su tienda. \bibverse{17} Y los
hijos de Israel lo hicieron así: y recogieron unos más, otros menos:
\bibverse{18} Y medíanlo por gomer, y no sobraba al que había recogido
mucho, ni faltaba al que había recogido poco: cada uno recogió conforme
á lo que había de comer. \bibverse{19} Y díjoles Moisés: Ninguno deje
nada de ello para mañana. \bibverse{20} Mas ellos no obedecieron á
Moisés, sino que algunos dejaron de ello para otro día, y crió gusanos,
y pudrióse; y enojóse contra ellos Moisés. \bibverse{21} Y recogíanlo
cada mañana, cada uno según lo que había de comer: y luego que el sol
calentaba, derretíase. \bibverse{22} En el sexto día recogieron doblada
comida, dos gomeres para cada uno: y todos los príncipes de la
congregación vinieron á Moisés, y se lo hicieron saber. \bibverse{23} Y
él les dijo: Esto es lo que ha dicho Jehová: Mañana es el santo sábado,
el reposo de Jehová: lo que hubiereis de cocer, cocedlo hoy, y lo que
hubiereis de cocinar, cocinadlo; y todo lo que os sobrare, guardadlo
para mañana. \bibverse{24} Y ellos lo guardaron hasta la mañana, según
que Moisés había mandado, y no se pudrió, ni hubo en él gusano.
\bibverse{25} Y dijo Moisés: Comedlo hoy, porque hoy es sábado de
Jehová: hoy no hallaréis en el campo. \bibverse{26} En los seis días lo
recogeréis; mas el séptimo día es sábado, en el cual no se hallará.
\bibverse{27} Y aconteció que algunos del pueblo salieron en el séptimo
día á recoger, y no hallaron. \bibverse{28} Y Jehová dijo á Moisés:
¿Hasta cuándo no querréis guardar mis mandamientos y mis leyes?
\bibverse{29} Mirad que Jehová os dió el sábado, y por eso os da en el
sexto día pan para dos días. Estése, pues, cada uno en su estancia, y
nadie salga de su lugar en el séptimo día. \bibverse{30} Así el pueblo
reposó el séptimo día. \bibverse{31} Y la casa de Israel lo llamó Maná;
y era como simiente de culantro, blanco, y su sabor como de hojuelas con
miel. \bibverse{32} Y dijo Moisés: Esto es lo que Jehová ha mandado:
Henchirás un gomer de él para que se guarde para vuestros descendientes,
á fin de que vean el pan que yo os dí á comer en el desierto, cuando yo
os saqué de la tierra de Egipto. \bibverse{33} Y dijo Moisés á Aarón:
Toma un vaso, y pon en él un gomer lleno de maná, y ponlo delante de
Jehová, para que sea guardado para vuestros descendientes. \bibverse{34}
Y Aarón lo puso delante del Testimonio para guardarlo, como Jehová lo
mandó á Moisés. \bibverse{35} Así comieron los hijos de Israel maná
cuarenta años, hasta que entraron en la tierra habitada: maná comieron
hasta que llegaron al término de la tierra de Canaán. \bibverse{36} Y un
gomer es la décima parte del epha.

\hypertarget{section-16}{%
\section{17}\label{section-16}}

\bibverse{1} Y toda la congregación de los hijos de Israel partió del
desierto de Sin, por sus jornadas, al mandamiento de Jehová, y asentaron
el campo en Rephidim: y no había agua para que el pueblo bebiese.
\bibverse{2} Y altercó el pueblo con Moisés, y dijeron: Danos agua que
bebamos. Y Moisés les dijo: ¿Por qué altercáis conmigo? ¿por qué tentáis
á Jehová? \bibverse{3} Así que el pueblo tuvo allí sed de agua, y
murmuró contra Moisés, y dijo: ¿Por qué nos hiciste subir de Egipto,
para matarnos de sed á nosotros, y á nuestros hijos, y á nuestros
ganados? \bibverse{4} Entonces clamó Moisés á Jehová, diciendo: ¿Qué
haré con este pueblo? de aquí á un poco me apedrearán. \bibverse{5} Y
Jehová dijo á Moisés: Pasa delante del pueblo, y toma contigo de los
ancianos de Israel; y toma también en tu mano tu vara, con que heriste
el río, y ve: \bibverse{6} He aquí que yo estoy delante de ti allí sobre
la peña en Horeb; y herirás la peña, y saldrán de ella aguas, y beberá
el pueblo. Y Moisés lo hizo así en presencia de los ancianos de Israel.
\bibverse{7} Y llamó el nombre de aquel lugar Massah y Meribah, por la
rencilla de los hijos de Israel, y porque tentaron á Jehová, diciendo:
¿Está, pues, Jehová entre nosotros, ó no? \bibverse{8} Y vino Amalec, y
peleó con Israel en Rephidim. \bibverse{9} Y dijo Moisés á Josué:
Escógenos varones, y sal, pelea con Amalec: mañana yo estaré sobre la
cumbre del collado, y la vara de Dios en mi mano. \bibverse{10} E hizo
Josué como le dijo Moisés, peleando con Amalec; y Moisés y Aarón y Hur
subieron á la cumbre del collado. \bibverse{11} Y sucedía que cuando
alzaba Moisés su mano, Israel prevalecía; mas cuando él bajaba su mano,
prevalecía Amalec. \bibverse{12} Y las manos de Moisés estaban pesadas;
por lo que tomaron una piedra, y pusiéronla debajo de él, y se sentó
sobre ella: y Aarón y Hur sustentaban sus manos, el uno de una parte y
el otro de otra; así hubo en sus manos firmeza hasta que se puso el sol.
\bibverse{13} Y Josué deshizo á Amalec y á su pueblo á filo de espada.
\bibverse{14} Y Jehová dijo á Moisés: Escribe esto para memoria en un
libro, y di á Josué que del todo tengo de raer la memoria de Amalec de
debajo del cielo. \bibverse{15} Y Moisés edificó un altar, y llamó su
nombre Jehová-nissi; \bibverse{16} Y dijo: Por cuanto la mano sobre el
trono de Jehová, Jehová tendrá guerra con Amalec de generación en
generación.

\hypertarget{section-17}{%
\section{18}\label{section-17}}

\bibverse{1} Y oyó Jethro, sacerdote de Madián, suegro de Moisés, todas
las cosas que Dios había hecho con Moisés, y con Israel su pueblo, y
cómo Jehová había sacado á Israel de Egipto: \bibverse{2} Y tomó Jethro,
suegro de Moisés, á Séphora la mujer de Moisés, después que él la envió,
\bibverse{3} Y á sus dos hijos; el uno se llamaba Gersom, porque dijo,
Peregrino he sido en tierra ajena; \bibverse{4} Y el otro se llamaba
Eliezer, porque dijo, El Dios de mi padre me ayudó, y me libró del
cuchillo de Faraón. \bibverse{5} Y Jethro, el suegro de Moisés, con sus
hijos y su mujer, llegó á Moisés en el desierto, donde tenía el campo
junto al monte de Dios; \bibverse{6} Y dijo á Moisés: Yo tu suegro
Jethro vengo á ti, con tu mujer, y sus dos hijos con ella. \bibverse{7}
Y Moisés salió á recibir á su suegro, é inclinóse, y besólo: y
preguntáronse el uno al otro cómo estaban, y vinieron á la tienda.
\bibverse{8} Y Moisés contó á su suegro todas las cosas que Jehová había
hecho á Faraón y á los Egipcios por amor de Israel, y todo el trabajo
que habían pasado en el camino, y cómo los había librado Jehová.
\bibverse{9} Y alegróse Jethro de todo el bien que Jehová había hecho á
Israel, que lo había librado de mano de los Egipcios. \bibverse{10} Y
Jethro dijo: Bendito sea Jehová, que os libró de mano de los Egipcios, y
de la mano de Faraón, y que libró al pueblo de la mano de los Egipcios.
\bibverse{11} Ahora conozco que Jehová es grande más que todos los
dioses; hasta en lo que se ensoberbecieron contra ellos. \bibverse{12} Y
tomó Jethro, suegro de Moisés, holocaustos y sacrificios para Dios: y
vino Aarón y todos los ancianos de Israel á comer pan con el suegro de
Moisés delante de Dios. \bibverse{13} Y aconteció que otro día se sentó
Moisés á juzgar al pueblo; y el pueblo estuvo delante de Moisés desde la
mañana hasta la tarde. \bibverse{14} Y viendo el suegro de Moisés todo
lo que él hacía con el pueblo, dijo: ¿Qué es esto que haces tú con el
pueblo? ¿por qué te sientas tú solo, y todo el pueblo está delante de ti
desde la mañana hasta la tarde? \bibverse{15} Y Moisés respondió á su
suegro: Porque el pueblo viene á mí para consultar á Dios: \bibverse{16}
Cuando tienen negocios, vienen á mí; y yo juzgo entre el uno y el otro,
y declaro las ordenanzas de Dios y sus leyes. \bibverse{17} Entonces el
suegro de Moisés le dijo: No haces bien: \bibverse{18} Desfallecerás del
todo, tú, y también este pueblo que está contigo; porque el negocio es
demasiado pesado para ti; no podrás hacerlo tú solo. \bibverse{19} Oye
ahora mi voz, yo te aconsejaré, y Dios será contigo. Está tú por el
pueblo delante de Dios, y somete tú los negocios á Dios. \bibverse{20} Y
enseña á ellos las ordenanzas y las leyes, y muéstrales el camino por
donde anden, y lo que han de hacer. \bibverse{21} Además inquiere tú de
entre todo el pueblo varones de virtud, temerosos de Dios, varones de
verdad, que aborrezcan la avaricia; y constituirás á éstos sobre ellos
caporales sobre mil, sobre ciento, sobre cincuenta y sobre diez.
\bibverse{22} Los cuales juzgarán al pueblo en todo tiempo; y será que
todo negocio grave lo traerán á ti, y ellos juzgarán todo negocio
pequeño: alivia así la carga de sobre ti, y llevarla han ellos contigo.
\bibverse{23} Si esto hicieres, y Dios te lo mandare, tú podrás
persistir, y todo este pueblo se irá también en paz á su lugar.
\bibverse{24} Y oyó Moisés la voz de su suegro, é hizo todo lo que dijo.
\bibverse{25} Y escogió Moisés varones de virtud de todo Israel, y
púsolos por cabezas sobre el pueblo, caporales sobre mil, sobre ciento,
sobre cincuenta, y sobre diez; \bibverse{26} Y juzgaban al pueblo en
todo tiempo: el negocio arduo traíanlo á Moisés, y ellos juzgaban todo
negocio pequeño. \bibverse{27} Y despidió Moisés á su suegro, y fuése á
su tierra.

\hypertarget{section-18}{%
\section{19}\label{section-18}}

\bibverse{1} Al mes tercero de la salida de los hijos de Israel de la
tierra de Egipto, en aquel día vinieron al desierto de Sinaí.
\bibverse{2} Porque partieron de Rephidim, y llegaron al desierto de
Sinaí, y asentaron en el desierto; y acampó allí Israel delante del
monte. \bibverse{3} Y Moisés subió á Dios; y Jehová lo llamó desde el
monte, diciendo: Así dirás á la casa de Jacob, y denunciarás á los hijos
de Israel: \bibverse{4} Vosotros visteis lo que hice á los Egipcios, y
cómo os tomé sobre alas de águilas, y os he traído á mí. \bibverse{5}
Ahora pues, si diereis oído á mi voz, y guardareis mi pacto, vosotros
seréis mi especial tesoro sobre todos los pueblos; porque mía es toda la
tierra. \bibverse{6} Y vosotros seréis mi reino de sacerdotes, y gente
santa. Estas son las palabras que dirás á los hijos de Israel.
\bibverse{7} Entonces vino Moisés, y llamó á los ancianos del pueblo, y
propuso en presencia de ellos todas estas palabras que Jehová le había
mandado. \bibverse{8} Y todo el pueblo respondió á una, y dijeron: Todo
lo que Jehová ha dicho haremos. Y Moisés refirió las palabras del pueblo
á Jehová. \bibverse{9} Y Jehová dijo á Moisés: He aquí, yo vengo á ti en
una nube espesa, para que el pueblo oiga mientras yo hablo contigo, y
también para que te crean para siempre. Y Moisés denunció las palabras
del pueblo á Jehová. \bibverse{10} Y Jehová dijo á Moisés: Ve al pueblo,
y santifícalos hoy y mañana, y laven sus vestidos; \bibverse{11} Y estén
apercibidos para el día tercero, porque al tercer día Jehová descenderá,
á ojos de todo el pueblo, sobre el monte de Sinaí. \bibverse{12} Y
señalarás término al pueblo en derredor, diciendo: Guardaos, no subáis
al monte, ni toquéis á su término: cualquiera que tocare el monte, de
seguro morirá: \bibverse{13} No le tocará mano, mas será apedreado ó
asaeteado; sea animal ó sea hombre, no vivirá. En habiendo sonado
largamente la bocina, subirán al monte. \bibverse{14} Y descendió Moisés
del monte al pueblo, y santificó al pueblo; y lavaron sus vestidos.
\bibverse{15} Y dijo al pueblo: Estad apercibidos para el tercer día; no
lleguéis á mujer. \bibverse{16} Y aconteció al tercer día cuando vino la
mañana, que vinieron truenos y relámpagos, y espesa nube sobre el monte,
y sonido de bocina muy fuerte; y estremecióse todo el pueblo que estaba
en el real. \bibverse{17} Y Moisés sacó del real al pueblo á recibir á
Dios; y pusiéronse á lo bajo del monte. \bibverse{18} Y todo el monte de
Sinaí humeaba, porque Jehová había descendido sobre él en fuego: y el
humo de él subía como el humo de un horno, y todo el monte se estremeció
en gran manera. \bibverse{19} Y el sonido de la bocina iba esforzándose
en extremo: Moisés hablaba, y Dios le respondía en voz. \bibverse{20} Y
descendió Jehová sobre el monte de Sinaí, sobre la cumbre del monte: y
llamó Jehová á Moisés á la cumbre del monte, y Moisés subió.
\bibverse{21} Y Jehová dijo á Moisés: Desciende, requiere al pueblo que
no traspasen el término por ver á Jehová, porque caerá multitud de
ellos. \bibverse{22} Y también los sacerdotes que se llegan á Jehová, se
santifiquen, porque Jehová no haga en ellos estrago. \bibverse{23} Y
Moisés dijo á Jehová: El pueblo no podrá subir al monte de Sinaí, porque
tú nos has requerido diciendo: Señala términos al monte, y santifícalo.
\bibverse{24} Y Jehová le dijo: Ve, desciende, y subirás tú, y Aarón
contigo: mas los sacerdotes y el pueblo no traspasen el término por
subir á Jehová, porque no haga en ellos estrago. \bibverse{25} Entonces
Moisés descendió al pueblo, y habló con ellos.

\hypertarget{section-19}{%
\section{20}\label{section-19}}

\bibverse{1} Y habló Dios todas estas palabras, diciendo: \bibverse{2}
Yo soy JEHOVÁ tu Dios, que te saqué de la tierra de Egipto, de casa de
siervos. \bibverse{3} No tendrás dioses ajenos delante de mí.
\bibverse{4} No te harás imagen, ni ninguna semejanza de cosa que esté
arriba en el cielo, ni abajo en la tierra, ni en las aguas debajo de la
tierra: \bibverse{5} No te inclinarás á ellas, ni las honrarás; porque
yo soy Jehová tu Dios, fuerte, celoso, que visito la maldad de los
padres sobre los hijos, sobre los terceros y sobre los cuartos, á los
que me aborrecen, \bibverse{6} Y que hago misericordia en millares á los
que me aman, y guardan mis mandamientos. \bibverse{7} No tomarás el
nombre de Jehová tu Dios en vano; porque no dará por inocente Jehová al
que tomare su nombre en vano. \bibverse{8} Acordarte has del día del
reposo, para santificarlo: \bibverse{9} Seis días trabajarás, y harás
toda tu obra; \bibverse{10} Mas el séptimo día será reposo para Jehová
tu Dios: no hagas en él obra alguna, tú, ni tu hijo, ni tu hija, ni tu
siervo, ni tu criada ni tu bestia, ni tu extranjero que está dentro de
tus puertas: \bibverse{11} Porque en seis días hizo Jehová los cielos y
la tierra, la mar y todas las cosas que en ellos hay, y reposó en el
séptimo día: por tanto Jehová bendijo el día del reposo y lo santificó.
\bibverse{12} Honra á tu padre y á tu madre, porque tus días se alarguen
en la tierra que Jehová tu Dios te da. \bibverse{13} No matarás.
\bibverse{14} No cometerás adulterio. \bibverse{15} No hurtarás.
\bibverse{16} No hablarás contra tu prójimo falso testimonio.
\bibverse{17} No codiciarás la casa de tu prójimo, no codiciarás la
mujer de tu prójimo, ni su siervo, ni su criada, ni su buey, ni su asno,
ni cosa alguna de tu prójimo. \bibverse{18} Todo el pueblo consideraba
las voces, y las llamas, y el sonido de la bocina, y el monte que
humeaba: y viéndolo el pueblo, temblaron, y pusiéronse de lejos.
\bibverse{19} Y dijeron á Moisés: Habla tú con nosotros, que nosotros
oiremos; mas no hable Dios con nosotros, porque no muramos.
\bibverse{20} Y Moisés respondió al pueblo: No temáis; que por probaros
vino Dios, y porque su temor esté en vuestra presencia para que no
pequéis. \bibverse{21} Entonces el pueblo se puso de lejos, y Moisés se
llegó á la osbcuridad, en la cual estaba Dios. \bibverse{22} Y Jehová
dijo á Moisés: Así dirás á los hijos de Israel: Vosotros habéis visto
que he hablado desde el cielo con vosotros. \bibverse{23} No hagáis
conmigo dioses de plata, ni dioses de oro os haréis. \bibverse{24} Altar
de tierra harás para mí, y sacrificarás sobre él tus holocaustos y tus
pacíficos, tus ovejas y tus vacas: en cualquier lugar donde yo hiciere
que esté la memoria de mi nombre, vendré á ti, y te bendeciré.
\bibverse{25} Y si me hicieres altar de piedras, no las labres de
cantería; porque si alzares tu pico sobre él, tú lo profanarás.
\bibverse{26} Y no subirás por gradas á mi altar, porque tu desnudez no
sea junto á él descubierta.

\hypertarget{section-20}{%
\section{21}\label{section-20}}

\bibverse{1} Y estos son los derechos que les propondrás. \bibverse{2}
Si comprares siervo hebreo, seis años servirá; mas el séptimo saldrá
horro de balde. \bibverse{3} Si entró solo, solo saldrá: si tenía mujer,
saldrá él y su mujer con él. \bibverse{4} Si su amo le hubiere dado
mujer, y ella le hubiere parido hijos ó hijas, la mujer y sus hijos
serán de su amo, y él saldrá solo. \bibverse{5} Y si el siervo dijere:
Yo amo á mi señor, á mi mujer y á mis hijos, no saldré libre:
\bibverse{6} Entonces su amo lo hará llegar á los jueces, y harále
llegar á la puerta ó al poste; y su amo le horadará la oreja con lesna,
y será su siervo para siempre. \bibverse{7} Y cuando alguno vendiere su
hija por sierva, no saldrá como suelen salir los siervos. \bibverse{8}
Si no agradare á su señor, por lo cual no la tomó por esposa, permitirle
ha que se rescate, y no la podrá vender á pueblo extraño cuando la
desechare. \bibverse{9} Mas si la hubiere desposado con su hijo, hará
con ella según la costumbre de las hijas. \bibverse{10} Si le tomare
otra, no disminuirá su alimento, ni su vestido, ni el débito conyugal.
\bibverse{11} Y si ninguna de estas tres cosas hiciere, ella saldrá de
gracia sin dinero. \bibverse{12} El que hiriere á alguno, haciéndole así
morir, él morirá. \bibverse{13} Mas el que no armó asechanzas, sino que
Dios lo puso en sus manos, entonces yo te señalaré lugar al cual ha de
huir. \bibverse{14} Además, si alguno se ensoberbeciere contra su
prójimo, y lo matare con alevosía, de mi altar lo quitarás para que
muera. \bibverse{15} Y el que hiriere á su padre ó á su madre, morirá.
\bibverse{16} Asimismo el que robare una persona, y la vendiere, ó se
hallare en sus manos, morirá. \bibverse{17} Igualmente el que maldijere
á su padre ó á su madre, morirá. \bibverse{18} Además, si algunos
riñeren, y alguno hiriere á su prójimo con piedra ó con el puño, y no
muriere, pero cayere en cama; \bibverse{19} Si se levantare y anduviere
fuera sobre su báculo, entonces será el que le hirió absuelto: solamente
le satisfará lo que estuvo parado, y hará que le curen. \bibverse{20} Y
si alguno hiriere á su siervo ó á su sierva con palo, y muriere bajo de
su mano, será castigado: \bibverse{21} Mas si durare por un día ó dos,
no será castigado, porque su dinero es. \bibverse{22} Si algunos
riñeren, é hiriesen á mujer preñada, y ésta abortare, pero sin haber
muerte, será penado conforme á lo que le impusiere el marido de la mujer
y juzgaren los árbitros. \bibverse{23} Mas si hubiere muerte, entonces
pagarás vida por vida, \bibverse{24} Ojo por ojo, diente por diente,
mano por mano, pie por pie, \bibverse{25} Quemadura por quemadura,
herida por herida, golpe por golpe. \bibverse{26} Y cuando alguno
hiriere el ojo de su siervo, ó el ojo de su sierva, y lo entortare,
darále libertad por razón de su ojo. \bibverse{27} Y si sacare el diente
de su siervo, ó el diente de su sierva, por su diente le dejará ir
libre. \bibverse{28} Si un buey acorneare hombre ó mujer, y de resultas
muriere, el buey será apedreado, y no se comerá su carne; mas el dueño
del buey será absuelto. \bibverse{29} Pero si el buey era acorneador
desde ayer y antes de ayer, y á su dueño le fué hecho requerimiento, y
no lo hubiere guardado, y matare hombre ó mujer, el buey será apedreado,
y también morirá su dueño. \bibverse{30} Si le fuere impuesto rescate,
entonces dará por el rescate de su persona cuanto le fuere impuesto.
\bibverse{31} Haya acorneado hijo, ó haya acorneado hija, conforme á
este juicio se hará con él. \bibverse{32} Si el buey acorneare siervo ó
sierva, pagará treinta siclos de plata su señor, y el buey será
apedreado. \bibverse{33} Y si alguno abriere hoyo, ó cavare cisterna, y
no la cubriere, y cayere allí buey ó asno, \bibverse{34} El dueño de la
cisterna pagará el dinero, resarciendo á su dueño, y lo que fué muerto
será suyo. \bibverse{35} Y si el buey de alguno hiriere al buey de su
prójimo, y éste muriere, entonces venderán el buey vivo, y partirán el
dinero de él, y también partirán el muerto. \bibverse{36} Mas si era
notorio que el buey era acorneador de ayer y antes de ayer, y su dueño
no lo hubiere guardado, pagará buey por buey, y el muerto será suyo.

\hypertarget{section-21}{%
\section{22}\label{section-21}}

\bibverse{1} Cuando alguno hurtare buey ú oveja, y le degollare ó
vendiere, por aquel buey pagará cinco bueyes, y por aquella oveja cuatro
ovejas. \bibverse{2} Si el ladrón fuere hallado forzando una casa, y
fuere herido y muriere, el que le hirió no será culpado de su muerte.
\bibverse{3} Si el sol hubiere sobre él salido, el matador será reo de
homicidio: el ladrón habrá de restituir cumplidamente; si no tuviere,
será vendido por su hurto. \bibverse{4} Si fuere hallado con el hurto en
la mano, sea buey ó asno ú oveja vivos, pagará el duplo. \bibverse{5} Si
alguno hiciere pacer campo ó viña, y metiere su bestia, y comiere la
tierra de otro, de lo mejor de su tierra y de lo mejor de su viña
pagará. \bibverse{6} Cuando rompiere un fuego, y hallare espinas, y
fuere quemado montón, ó haza, ó campo, el que encendió el fuego pagará
lo quemado. \bibverse{7} Cuando alguno diere á su prójimo plata ó
alhajas á guardar, y fuere hurtado de la casa de aquel hombre, si el
ladrón se hallare, pagará el doble. \bibverse{8} Si el ladrón no se
hallare, entonces el dueño de la casa será presentado á los jueces, para
ver si ha metido su mano en la hacienda de su prójimo. \bibverse{9}
Sobre todo negocio de fraude, sobre buey, sobre asno, sobre oveja, sobre
vestido, sobre toda cosa perdida, cuando uno dijere: Esto es mío, la
causa de ambos vendrá delante de los jueces; y el que los jueces
condenaren, pagará el doble á su prójimo. \bibverse{10} Si alguno
hubiere dado á su prójimo asno, ó buey, ú oveja, ó cualquier otro animal
á guardar, y se muriere, ó se perniquebrare, ó fuere llevado sin verlo
nadie; \bibverse{11} Juramento de Jehová tendrá lugar entre ambos de que
no echó su mano á la hacienda de su prójimo: y su dueño lo aceptará, y
el otro no pagará. \bibverse{12} Mas si le hubiere sido hurtado,
resarcirá á su dueño. \bibverse{13} Y si le hubiere sido arrebatado por
fiera, traerle ha testimonio, y no pagará lo arrebatado. \bibverse{14}
Pero si alguno hubiere tomado prestada bestia de su prójimo, y fuere
estropeada ó muerta, ausente su dueño, deberá pagarla. \bibverse{15} Si
el dueño estaba presente, no la pagará. Si era alquilada, él vendrá por
su alquiler. \bibverse{16} Y si alguno engañare á alguna doncella que no
fuere desposada, y durmiere con ella, deberá dotarla y tomarla por
mujer. \bibverse{17} Si su padre no quisiere dársela, él le pesará plata
conforme al dote de las vírgenes. \bibverse{18} A la hechicera no
dejarás que viva. \bibverse{19} Cualquiera que tuviere ayuntamiento con
bestia, morirá. \bibverse{20} El que sacrificare á dioses, excepto á
sólo Jehová, será muerto. \bibverse{21} Y al extranjero no engañarás, ni
angustiarás, porque extranjeros fuisteis vosotros en la tierra de
Egipto. \bibverse{22} A ninguna viuda ni huérfano afligiréis.
\bibverse{23} Que si tú llegas á afligirle, y él á mí clamare,
ciertamente oiré yo su clamor; \bibverse{24} Y mi furor se encenderá, y
os mataré á cuchillo, y vuestras mujeres serán viudas, y huérfanos
vuestros hijos. \bibverse{25} Si dieres á mi pueblo dinero emprestado,
al pobre que está contigo, no te portarás con él como logrero, ni le
impondrás usura. \bibverse{26} Si tomares en prenda el vestido de tu
prójimo, á puestas del sol se lo volverás: \bibverse{27} Porque sólo
aquello es su cubierta, es aquél el vestido para cubrir sus carnes, en
el que ha de dormir: y será que cuando él á mí clamare, yo entonces le
oiré, porque soy misericordioso. \bibverse{28} No denostarás á los
jueces, ni maldecirás al príncipe de tu pueblo. \bibverse{29} No
dilatarás la primicia de tu cosecha, ni de tu licor: me darás el
primogénito de tus hijos. \bibverse{30} Así harás con el de tu buey y de
tu oveja: siete días estará con su madre, y al octavo día me lo darás.
\bibverse{31} Y habéis de serme varones santos: y no comeréis carne
arrebatada de las fieras en el campo; á los perros la echaréis.

\hypertarget{section-22}{%
\section{23}\label{section-22}}

\bibverse{1} No admitirás falso rumor. No te concertarás con el impío
para ser testigo falso. \bibverse{2} No seguirás á los muchos para mal
hacer; ni responderás en litigio inclinándote á los más para hacer
agravios; \bibverse{3} Ni al pobre distinguirás en su causa.
\bibverse{4} Si encontrares el buey de tu enemigo ó su asno extraviado,
vuelve á llevárselo. \bibverse{5} Si vieres el asno del que te aborrece
caído debajo de su carga, ¿le dejarás entonces desamparado? Sin falta
ayudarás con él á levantarlo. \bibverse{6} No pervertirás el derecho de
tu mendigo en su pleito. \bibverse{7} De palabra de mentira te alejarás,
y no matarás al inocente y justo; porque yo no justificaré al impío.
\bibverse{8} No recibirás presente; porque el presente ciega á los que
ven, y pervierte las palabras justas. \bibverse{9} Y no angustiarás al
extranjero: pues vosotros sabéis cómo se halla el alma del extranjero,
ya que extranjeros fuisteis en la tierra de Egipto. \bibverse{10} Seis
años sembrarás tu tierra, y allegarás su cosecha: \bibverse{11} Mas el
séptimo la dejarás vacante y soltarás, para que coman los pobres de tu
pueblo; y de lo que quedare comerán las bestias del campo; así harás de
tu viña y de tu olivar. \bibverse{12} Seis días harás tus negocios, y al
séptimo día holgarás, á fin que descanse tu buey y tu asno, y tome
refrigerio el hijo de tu sierva, y el extranjero. \bibverse{13} Y en
todo lo que os he dicho seréis avisados. Y nombre de otros dioses no
mentaréis, ni se oirá de vuestra boca. \bibverse{14} Tres veces en el
año me celebraréis fiesta. \bibverse{15} La fiesta de los ázimos
guardarás: siete días comerás los panes sin levadura, como yo te mandé,
en el tiempo del mes de Abib; porque en él saliste de Egipto: y ninguno
comparecerá vacío delante de mí: \bibverse{16} También la fiesta de la
siega, los primeros frutos de tus labores que hubieres sembrado en el
campo; y la fiesta de la cosecha á la salida del año, cuando habrás
recogido tus labores del campo. \bibverse{17} Tres veces en el año
parecerá todo varón tuyo delante del Señor Jehová. \bibverse{18} No
ofrecerás con pan leudo la sangre de mi sacrificio; ni el sebo de mi
víctima quedará de la noche hasta la mañana. \bibverse{19} Las primicias
de los primeros frutos de tu tierra traerás á la casa de Jehová tu Dios.
No guisarás el cabrito con la leche de su madre. \bibverse{20} He aquí
yo envío el Angel delante de ti para que te guarde en el camino, y te
introduzca en el lugar que yo he preparado. \bibverse{21} Guárdate
delante de él, y oye su voz; no le seas rebelde; porque él no perdonará
vuestra rebelión: porque mi nombre está en él. \bibverse{22} Pero si en
verdad oyeres su voz, é hicieres todo lo que yo te dijere, seré enemigo
á tus enemigos, y afligiré á los que te afligieren. \bibverse{23} Porque
mi Angel irá delante de ti, y te introducirá al Amorrheo, y al Hetheo, y
al Pherezeo, y al Cananeo, y al Heveo, y al Jebuseo, á los cuales yo
haré destruir. \bibverse{24} No te inclinarás á sus dioses, ni los
servirás, ni harás como ellos hacen; antes los destruirás del todo, y
quebrantarás enteramente sus estatuas. \bibverse{25} Mas á Jehová
vuestro Dios serviréis, y él bendecirá tu pan y tus aguas; y yo quitaré
toda enfermedad de en medio de ti. \bibverse{26} No habrá mujer que
aborte, ni estéril en tu tierra; y yo cumpliré el número de tus días.
\bibverse{27} Yo enviaré mi terror delante de ti, y consternaré á todo
pueblo donde tú entrares, y te daré la cerviz de todos tus enemigos.
\bibverse{28} Yo enviaré la avispa delante de ti, que eche fuera al
Heveo, y al Cananeo, y al Hetheo, de delante de ti: \bibverse{29} No los
echaré de delante de ti en un año, porque no quede la tierra desierta, y
se aumenten contra ti las bestias del campo. \bibverse{30} Poco á poco
los echaré de delante de ti, hasta que te multipliques y tomes la tierra
por heredad. \bibverse{31} Y yo pondré tu término desde el mar Bermejo
hasta la mar de Palestina, y desde el desierto hasta el río: porque
pondré en vuestras manos los moradores de la tierra, y tú los echarás de
delante de ti. \bibverse{32} No harás alianza con ellos, ni con sus
dioses. \bibverse{33} En tu tierra no habitarán, no sea que te hagan
pecar contra mí sirviendo á sus dioses: porque te será de tropiezo.

\hypertarget{section-23}{%
\section{24}\label{section-23}}

\bibverse{1} Y dijo á Moisés: Sube á Jehová, tú, y Aarón, Nadab, y Abiú,
y setenta de los ancianos de Israel; y os inclinaréis desde lejos.
\bibverse{2} Mas Moisés sólo se llegará á Jehová; y ellos no se lleguen
cerca, ni suba con él el pueblo. \bibverse{3} Y Moisés vino y contó al
pueblo todas las palabras de Jehová, y todos los derechos: y todo el
pueblo respondió á una voz, y dijeron: Ejecutaremos todas las palabras
que Jehová ha dicho. \bibverse{4} Y Moisés escribió todas las palabras
de Jehová, y levantándose de mañana edificó un altar al pie del monte, y
doce columnas, según las doce tribus de Israel. \bibverse{5} Y envió á
los mancebos de los hijos de Israel, los cuales ofrecieron holocaustos,
y sacrificaron pacíficos á Jehová, becerros. \bibverse{6} Y Moisés tomó
la mitad de la sangre, y púsola en tazones, y esparció la otra mitad de
la sangre sobre el altar. \bibverse{7} Y tomó el libro de la alianza, y
leyó á oídos del pueblo, el cual dijo: Haremos todas las cosas que
Jehová ha dicho, y obedeceremos. \bibverse{8} Entonces Moisés tomó la
sangre, y roció sobre el pueblo, y dijo: He aquí la sangre de la alianza
que Jehová ha hecho con vosotros sobre todas estas cosas. \bibverse{9} Y
subieron Moisés y Aarón, Nadab y Abiú, y setenta de los ancianos de
Israel; \bibverse{10} Y vieron al Dios de Israel; y había debajo de sus
pies como un embaldosado de zafiro, semejante al cielo cuando está
sereno. \bibverse{11} Mas no extendió su mano sobre los príncipes de los
hijos de Israel: y vieron á Dios, y comieron y bebieron. \bibverse{12}
Entonces Jehová dijo á Moisés: Sube á mí al monte, y espera allá, y te
daré tablas de piedra, y la ley, y mandamientos que he escrito para
enseñarlos. \bibverse{13} Y levantóse Moisés, y Josué su ministro; y
Moisés subió al monte de Dios. \bibverse{14} Y dijo á los ancianos:
Esperadnos aquí hasta que volvamos á vosotros: y he aquí Aarón y Hur
están con vosotros: el que tuviere negocios, lléguese á ellos.
\bibverse{15} Entonces Moisés subió al monte, y una nube cubrió el
monte. \bibverse{16} Y la gloria de Jehová reposó sobre el monte Sinaí,
y la nube lo cubrió por seis días: y al séptimo día llamó á Moisés de en
medio de la nube. \bibverse{17} Y el parecer de la gloria de Jehová era
como un fuego abrasador en la cumbre del monte, á los ojos de los hijos
de Israel. \bibverse{18} Y entró Moisés en medio de la nube, y subió al
monte: y estuvo Moisés en el monte cuarenta días y cuarenta noches.

\hypertarget{section-24}{%
\section{25}\label{section-24}}

\bibverse{1} Y jehová habló á Moisés, diciendo: \bibverse{2} Di á los
hijos de Israel que tomen para mí ofrenda: de todo varón que la diere de
su voluntad, de corazón, tomaréis mi ofrenda. \bibverse{3} Y esta es la
ofrenda que tomaréis de ellos: Oro, y plata, y cobre, \bibverse{4} Y
jacinto, y púrpura, y carmesí, y lino fino, y pelo de cabras,
\bibverse{5} Y cueros de carneros teñidos de rojo, y cueros de tejones,
y madera de Sittim; \bibverse{6} Aceite para la luminaria, especias para
el aceite de la unción, y para el sahumerio aromático; \bibverse{7}
Piedras de onix, y piedras de engastes, para el ephod, y para el
racional. \bibverse{8} Y hacerme han un santuario, y yo habitaré entre
ellos. \bibverse{9} Conforme á todo lo que yo te mostrare, el diseño del
tabernáculo, y el diseño de todos sus vasos, así lo haréis.
\bibverse{10} Harán también un arca de madera de Sittim, cuya longitud
será de dos codos y medio, y su anchura de codo y medio, y su altura de
codo y medio. \bibverse{11} Y la cubrirás de oro puro; por dentro y por
fuera la cubrirás; y harás sobre ella una cornisa de oro alrededor.
\bibverse{12} Y para ella harás de fundición cuatro anillos de oro, que
pondrás á sus cuatro esquinas; dos anillos al un lado de ella, y dos
anillos al otro lado. \bibverse{13} Y harás unas varas de madera de
Sittim, las cuales cubrirás de oro: \bibverse{14} Y meterás las varas
por los anillos á los lados del arca, para llevar el arca con ellas.
\bibverse{15} Las varas se estarán en los anillos del arca: no se
quitarán de ella. \bibverse{16} Y pondrás en el arca el testimonio que
yo te daré. \bibverse{17} Y harás una cubierta de oro fino, cuya
longitud será de dos codos y medio, y su anchura de codo y medio.
\bibverse{18} Harás también dos querubines de oro, labrados á martillo
los harás, en los dos cabos de la cubierta. \bibverse{19} Harás, pues,
un querubín al extremo de un lado, y un querubín al otro extremo del
lado opuesto: de la calidad de la cubierta harás los querubines en sus
dos extremidades. \bibverse{20} Y los querubines extenderán por encima
las alas, cubriendo con sus alas la cubierta: sus caras la una enfrente
de la otra, mirando á la cubierta las caras de los querubines.
\bibverse{21} Y pondrás la cubierta encima del arca, y en el arca
pondrás el testimonio que yo te daré. \bibverse{22} Y de allí me
declararé á ti, y hablaré contigo de sobre la cubierta, de entre los dos
querubines que están sobre el arca del testimonio, todo lo que yo te
mandaré para los hijos de Israel. \bibverse{23} Harás asimismo una mesa
de madera de Sittim: su longitud será de dos codos, y de un codo su
anchura, y su altura de codo y medio. \bibverse{24} Y la cubrirás de oro
puro, y le has de hacer una cornisa de oro alrededor. \bibverse{25}
Hacerle has también una moldura alrededor, del ancho de una mano, á la
cual moldura harás una cornisa de oro en circunferencia. \bibverse{26} Y
le harás cuatro anillos de oro, los cuales pondrás á las cuatro esquinas
que corresponden á sus cuatro pies. \bibverse{27} Los anillos estarán
antes de la moldura, por lugares de las varas, para llevar la mesa.
\bibverse{28} Y harás las varas de madera de Sittim, y las cubrirás de
oro, y con ellas será llevada la mesa. \bibverse{29} Harás también sus
platos, y sus cucharas, y sus cubiertas, y sus tazones, con que se
libará: de oro fino los harás. \bibverse{30} Y pondrás sobre la mesa el
pan de la proposición delante de mí continuamente. \bibverse{31} Harás
además un candelero de oro puro; labrado á martillo se hará el
candelero: su pie, y su caña, sus copas, sus manzanas, y sus flores,
serán de lo mismo: \bibverse{32} Y saldrán seis brazos de sus lados:
tres brazos del candelero del un lado suyo, y tres brazos del candelero
del otro su lado: \bibverse{33} Tres copas en forma de almendras en el
un brazo, una manzana y una flor; y tres copas, figura de almendras, en
el otro brazo, una manzana y una flor: así pues, en los seis brazos que
salen del candelero: \bibverse{34} Y en el candelero cuatro copas en
forma de almendras, sus manzanas y sus flores. \bibverse{35} Habrá una
manzana debajo de los dos brazos de lo mismo, otra manzana debajo de los
otros dos brazos de lo mismo, y otra manzana debajo de los otros dos
brazos de lo mismo, en conformidad á los seis brazos que salen del
candelero. \bibverse{36} Sus manzanas y sus brazos serán de lo mismo,
todo ello una pieza labrada á martillo, de oro puro. \bibverse{37} Y
hacerle has siete candilejas, las cuales encenderás para que alumbren á
la parte de su delantera: \bibverse{38} También sus despabiladeras y sus
platillos, de oro puro. \bibverse{39} De un talento de oro fino lo
harás, con todos estos vasos. \bibverse{40} Y mira, y hazlos conforme á
su modelo, que te ha sido mostrado en el monte.

\hypertarget{section-25}{%
\section{26}\label{section-25}}

\bibverse{1} Y harás el tabernáculo de diez cortinas de lino torcido,
cárdeno, y púrpura, y carmesí: y harás querubines de obra delicada.
\bibverse{2} La longitud de la una cortina de veintiocho codos, y la
anchura de la misma cortina de cuatro codos: todas las cortinas tendrán
una medida. \bibverse{3} Cinco cortinas estarán juntas la una con la
otra, y cinco cortinas unidas la una con la otra. \bibverse{4} Y harás
lazadas de cárdeno en la orilla de la una cortina, en el borde, en la
juntura: y así harás en la orilla de la postrera cortina en la juntura
segunda. \bibverse{5} Cincuenta lazadas harás en la una cortina, y
cincuenta lazadas harás en el borde de la cortina que está en la segunda
juntura: las lazadas estarán contrapuestas la una á la otra.
\bibverse{6} Harás también cincuenta corchetes de oro, con los cuales
juntarás las cortinas la una con la otra, y se formará un tabernáculo.
\bibverse{7} Harás asimismo cortinas de pelo de cabras para una cubierta
sobre el tabernáculo; once cortinas harás. \bibverse{8} La longitud de
la una cortina será de treinta codos, y la anchura de la misma cortina
de cuatro codos: una medida tendrán las once cortinas. \bibverse{9} Y
juntarás las cinco cortinas aparte y las otras seis cortinas
separadamente; y doblarás la sexta cortina delante de la faz del
tabernáculo. \bibverse{10} Y harás cincuenta lazadas en la orilla de la
una cortina, al borde en la juntura, y cincuenta lazadas en la orilla de
la segunda cortina en la otra juntura. \bibverse{11} Harás asimismo
cincuenta corchetes de alambre, los cuales meterás por las lazadas: y
juntarás la tienda, para que se haga una sola cubierta. \bibverse{12} Y
el sobrante que resulta en las cortinas de la tienda, la mitad de la una
cortina que sobra, quedará á las espaldas del tabernáculo. \bibverse{13}
Y un codo de la una parte, y otro codo de la otra que sobra en la
longitud de las cortinas de la tienda, cargará sobre los lados del
tabernáculo de la una parte y de la otra, para cubrirlo. \bibverse{14}
Harás también á la tienda una cubierta de cueros de carneros, teñidos de
rojo, y una cubierta de cueros de tejones encima. \bibverse{15} Y harás
para el tabernáculo tablas de madera de Sittim, que estén derechas.
\bibverse{16} La longitud de cada tabla será de diez codos, y de codo y
medio la anchura de cada tabla. \bibverse{17} Dos quicios tendrá cada
tabla, trabadas la una con la otra; así harás todas las tablas del
tabernáculo. \bibverse{18} Harás, pues, las tablas del tabernáculo:
veinte tablas al lado del mediodía, al austro. \bibverse{19} Y harás
cuarenta basas de plata debajo de las veinte tablas; dos basas debajo de
la una tabla para sus dos quicios, y dos basas debajo de la otra tabla
para sus dos quicios. \bibverse{20} Y al otro lado del tabernáculo, á la
parte del aquilón, veinte tablas; \bibverse{21} Y sus cuarenta basas de
plata: dos basas debajo de la una tabla, y dos basas debajo de la otra
tabla. \bibverse{22} Y para el lado del tabernáculo, al occidente, harás
seis tablas. \bibverse{23} Harás además dos tablas para las esquinas del
tabernáculo en los dos ángulos posteriores; \bibverse{24} Las cuales se
unirán por abajo, y asimismo se juntarán por su alto á un gozne: así
será de las otras dos que estarán á las dos esquinas. \bibverse{25} De
suerte que serán ocho tablas, con sus basas de plata, diez y seis basas;
dos basas debajo de la una tabla, y dos basas debajo de la otra tabla.
\bibverse{26} Harás también cinco barras de madera de Sittim, para las
tablas del un lado del tabernáculo, \bibverse{27} Y cinco barras para
las tablas del otro lado del tabernáculo, y cinco barras para el otro
lado del tabernáculo, que está al occidente. \bibverse{28} Y la barra
del medio pasará por medio de las tablas, del un cabo al otro.
\bibverse{29} Y cubrirás las tablas de oro, y harás sus anillos de oro
para meter por ellos las barras: también cubrirás las barras de oro.
\bibverse{30} Y alzarás el tabernáculo conforme á su traza que te fué
mostrada en el monte. \bibverse{31} Y harás también un velo de cárdeno,
y púrpura, y carmesí, y de lino torcido: será hecho de primorosa labor,
con querubines: \bibverse{32} Y has de ponerlo sobre cuatro columnas de
madera de Sittim cubiertas de oro; sus capiteles de oro, sobre basas de
plata. \bibverse{33} Y pondrás el velo debajo de los corchetes, y
meterás allí, del velo adentro, el arca del testimonio; y aquel velo os
hará separación entre el lugar santo y el santísimo. \bibverse{34} Y
pondrás la cubierta sobre el arca del testimonio en el lugar santísimo.
\bibverse{35} Y pondrás la mesa fuera del velo, y el candelero enfrente
de la mesa al lado del tabernáculo al mediodía; y pondrás la mesa al
lado del aquilón. \bibverse{36} Y harás á la puerta del tabernáculo una
cortina de cárdeno, y púrpura, y carmesí, y lino torcido, obra de
bordador. \bibverse{37} Y harás para la cortina cinco columnas de madera
de Sittim, las cuales cubrirás de oro, con sus capiteles de oro: y
hacerlas has de fundición cinco basas de metal.

\hypertarget{section-26}{%
\section{27}\label{section-26}}

\bibverse{1} Harás también altar de madera de Sittim de cinco codos de
longitud, y de cinco codos de anchura: será cuadrado el altar, y su
altura de tres codos. \bibverse{2} Y harás sus cuernos á sus cuatro
esquinas; los cuernos serán de lo mismo; y lo cubrirás de metal.
\bibverse{3} Harás también sus calderas para echar su ceniza; y sus
paletas, y sus tazones, y sus garfios, y sus braseros: harás todos sus
vasos de metal. \bibverse{4} Y le harás un enrejado de metal de obra de
malla; y sobre el enrejado harás cuatro anillos de metal á sus cuatro
esquinas. \bibverse{5} Y lo has de poner dentro del cerco del altar
abajo; y llegará el enrejado hasta el medio del altar. \bibverse{6}
Harás también varas para el altar, varas de madera de Sittim, las cuales
cubrirás de metal. \bibverse{7} Y sus varas se meterán por los anillos:
y estarán aquellas varas á ambos lados del altar, cuando hubiere de ser
llevado. \bibverse{8} De tablas lo harás, hueco: de la manera que te fué
mostrado en el monte, así lo harás. \bibverse{9} Asimismo harás el atrio
del tabernáculo: al lado del mediodía, al austro, tendrá el atrio
cortinas de lino torcido, de cien codos de longitud cada un lado;
\bibverse{10} Sus veinte columnas, y sus veinte basas serán de metal;
los capiteles de las columnas y sus molduras, de plata. \bibverse{11} Y
de la misma manera al lado del aquilón habrá á lo largo cortinas de cien
codos de longitud, y sus veinte columnas, con sus veinte basas de metal;
los capiteles de sus columnas y sus molduras, de plata. \bibverse{12} Y
el ancho del atrio del lado occidental tendrá cortinas de cincuenta
codos; sus columnas diez, con sus diez basas. \bibverse{13} Y en el
ancho del atrio por la parte de levante, al oriente, habrá cincuenta
codos. \bibverse{14} Y las cortinas del un lado serán de quince codos;
sus columnas tres, con sus tres basas. \bibverse{15} Al otro lado quince
codos de cortinas; sus columnas tres, con sus tres basas. \bibverse{16}
Y á la puerta del atrio habrá un pabellón de veinte codos, de cárdeno, y
púrpura, y carmesí, y lino torcido, de obra de bordador: sus columnas
cuatro, con sus cuatro basas. \bibverse{17} Todas las columnas del atrio
en derredor serán ceñidas de plata; sus capiteles de plata, y sus basas
de metal. \bibverse{18} La longitud del atrio será de cien codos, y la
anchura cincuenta por un lado y cincuenta por el otro, y la altura de
cinco codos: sus cortinas de lino torcido, y sus basas de metal.
\bibverse{19} Todos los vasos del tabernáculo en todo su servicio, y
todos sus clavos, y todos los clavos del atrio, serán de metal.
\bibverse{20} Y tú mandarás á los hijos de Israel que te traigan aceite
puro de olivas, molido, para la luminaria, para hacer arder
continuamente las lámparas. \bibverse{21} En el tabernáculo del
testimonio, afuera del velo que está delante del testimonio, las pondrá
en orden Aarón y sus hijos, delante de Jehová desde la tarde hasta la
mañana, como estatuto perpetuo de los hijos de Israel por sus
generaciones.

\hypertarget{section-27}{%
\section{28}\label{section-27}}

\bibverse{1} Y tú allega á ti á Aarón tu hermano, y á sus hijos consigo,
de entre los hijos de Israel, para que sean mis sacerdotes; á Aarón,
Nadab y Abiú, Eleazar é Ithamar, hijos de Aarón. \bibverse{2} Y harás
vestidos sagrados á Aarón tu hermano, para honra y hermosura.
\bibverse{3} Y tú hablarás á todos los sabios de corazón, á quienes yo
he henchido de espíritu de sabiduría, á fin que hagan los vestidos de
Aarón, para consagrarle á que me sirva de sacerdote. \bibverse{4} Los
vestidos que harán son estos: el racional, y el ephod, y el manto, y la
túnica labrada, la mitra, y el cinturón. Hagan, pues, los sagrados
vestidos á Aarón tu hermano, y á sus hijos, para que sean mis
sacerdotes. \bibverse{5} Tomarán oro, y cárdeno, y púrpura, y carmesí, y
lino torcido. \bibverse{6} Y harán el ephod de oro y cárdeno, y púrpura,
y carmesí, y lino torcido de obra de bordador. \bibverse{7} Tendrá dos
hombreras que se junten á sus dos lados, y se juntará. \bibverse{8} Y el
artificio de su cinto que está sobre él, será de su misma obra, de lo
mismo; de oro, cárdeno, y púrpura, y carmesí, y lino torcido.
\bibverse{9} Y tomarás dos piedras oniquinas, y grabarás en ellas los
nombres de los hijos de Israel: \bibverse{10} Los seis de sus nombres en
la una piedra, y los otros seis nombres en la otra piedra, conforme al
nacimiento de ellos. \bibverse{11} De obra de escultor en piedra á modo
de grabaduras de sello, harás grabar aquellas dos piedras con los
nombres de los hijos de Israel; harásles alrededor engastes de oro.
\bibverse{12} Y pondrás aquellas dos piedras sobre los hombros del
ephod, para piedras de memoria á los hijos de Israel; y Aarón llevará
los nombres de ellos delante de Jehová en sus dos hombros por memoria.
\bibverse{13} Harás pues, engastes de oro, \bibverse{14} Y dos
cadenillas de oro fino; las cuales harás de hechura de trenza; y fijarás
las cadenas de hechura de trenza en los engastes. \bibverse{15} Harás
asimismo el racional del juicio de primorosa obra, le has de hacer
conforme á la obra del ephod, de oro, y cárdeno, y púrpura, y carmesí, y
lino torcido. \bibverse{16} Será cuadrado y doble, de un palmo de largo
y un palmo de ancho: \bibverse{17} Y lo llenarás de pedrería con cuatro
órdenes de piedras: un orden de una piedra sárdica, un topacio, y un
carbunclo; será el primer orden; \bibverse{18} El segundo orden, una
esmeralda, un zafiro, y un diamante; \bibverse{19} El tercer orden, un
rubí, un ágata, y una amatista; \bibverse{20} Y el cuarto orden, un
berilo, un onix, y un jaspe: estarán engastadas en oro en sus encajes.
\bibverse{21} Y serán aquellas piedras según los nombres de los hijos de
Israel, doce según sus nombres; como grabaduras de sello cada una con su
nombre, vendrán á ser según las doce tribus. \bibverse{22} Harás también
en el racional cadenetas de hechura de trenzas de oro fino.
\bibverse{23} Y harás en el racional dos anillos de oro, los cuales dos
anillos pondrás á las dos puntas del racional. \bibverse{24} Y pondrás
las dos trenzas de oro en los dos anillos á las dos puntas del racional:
\bibverse{25} Y los dos cabos de las dos trenzas sobre los dos engastes,
y las pondrás á los lados del ephod en la parte delantera. \bibverse{26}
Harás también dos anillos de oro, los cuales pondrás á las dos puntas
del racional, en su orilla que está al lado del ephod de la parte de
dentro. \bibverse{27} Harás asimismo dos anillos de oro, los cuales
pondrás á los dos lados del ephod abajo en la parte delantera, delante
de su juntura sobre el cinto del ephod. \bibverse{28} Y juntarán el
racional con sus anillos á los anillos del ephod con un cordón de
jacinto, para que esté sobre el cinto del ephod, y no se aparte el
racional del ephod. \bibverse{29} Y llevará Aarón los nombres de los
hijos de Israel en el racional del juicio sobre su corazón, cuando
entrare en el santuario, para memoria delante de Jehová continuamente.
\bibverse{30} Y pondrás en el racional del juicio Urim y Thummim, para
que estén sobre el corazón de Aarón cuando entrare delante de Jehová: y
llevará siempre Aarón el juicio de los hijos de Israel sobre su corazón
delante de Jehová. \bibverse{31} Harás el manto del ephod todo de
jacinto: \bibverse{32} Y en medio de él por arriba habrá una abertura,
la cual tendrá un borde alrededor de obra de tejedor, como el cuello de
un coselete, para que no se rompa. \bibverse{33} Y abajo en sus orillas
harás granadas de jacinto, y púrpura, y carmesí, por sus bordes
alrededor; y entre ellas campanillas de oro alrededor: \bibverse{34} Una
campanilla de oro y una granada, campanilla de oro y granada, por las
orillas del manto alrededor. \bibverse{35} Y estará sobre Aarón cuando
ministrare; y oiráse su sonido cuando él entrare en el santuario delante
de Jehová y cuando saliere, porque no muera. \bibverse{36} Harás además
una plancha de oro fino, y grabarás en ella grabadura de sello, SANTIDAD
Á JEHOVÁ. \bibverse{37} Y la pondrás con un cordón de jacinto, y estará
sobre la mitra; por el frente anterior de la mitra estará. \bibverse{38}
Y estará sobre la frente de Aarón: y llevará Aarón el pecado de las
cosas santas, que los hijos de Israel hubieren consagrado en todas sus
santas ofrendas; y sobre su frente estará continuamente para que hayan
gracia delante de Jehová. \bibverse{39} Y bordarás una túnica de lino, y
harás una mitra de lino; harás también un cinto de obra de recamador.
\bibverse{40} Y para los hijos de Aarón harás túnicas; también les harás
cintos, y les formarás chapeos (tiaras) para honra y hermosura.
\bibverse{41} Y con ellos vestirás á Aarón tu hermano, y á sus hijos con
él: y los ungirás, y los consagrarás, y santificarás, para que sean mis
sacerdotes. \bibverse{42} Y les harás pañetes de lino para cubrir la
carne vergonzosa; serán desde los lomos hasta los muslos: \bibverse{43}
Y estarán sobre Aarón y sobre sus hijos cuando entraren en el
tabernáculo del testimonio, ó cuando se llegaren al altar para servir en
el santuario, porque no lleven pecado, y mueran. Estatuto perpetuo para
él, y para su simiente después de él.

\hypertarget{section-28}{%
\section{29}\label{section-28}}

\bibverse{1} Y esto es lo que les harás para consagrarlos, para que sean
mis sacerdotes: Toma un becerro de la vacada, y dos carneros sin tacha;
\bibverse{2} Y panes sin levadura, y tortas sin levadura amasadas con
aceite, y hojaldres sin levadura untadas con aceite; las cuales cosas
harás de flor de harina de trigo: \bibverse{3} Y las pondrás en un
canastillo, y en el canastillo las ofrecerás, con el becerro y los dos
carneros. \bibverse{4} Y harás llegar á Aarón y á sus hijos á la puerta
del tabernáculo del testimonio, y los lavarás con agua. \bibverse{5} Y
tomarás las vestiduras, y vestirás á Aarón la túnica y el manto del
ephod, y el ephod, y el racional, y le ceñirás con el cinto del ephod;
\bibverse{6} Y pondrás la mitra sobre su cabeza, y sobre la mitra
pondrás la diadema santa. \bibverse{7} Y tomarás el aceite de la unción,
y derramarás sobre su cabeza, y le ungirás. \bibverse{8} Y harás llegar
sus hijos, y les vestirás las túnicas. \bibverse{9} Y les ceñirás el
cinto, á Aarón y á sus hijos, y les atarás los chapeos (tiaras), y
tendrán el sacerdocio por fuero perpetuo: y henchirás las manos de Aarón
y de sus hijos. \bibverse{10} Y harás llegar el becerro delante del
tabernáculo del testimonio, y Aarón y sus hijos pondrán sus manos sobre
la cabeza del becerro. \bibverse{11} Y matarás el becerro delante de
Jehová á la puerta del tabernáculo del testimonio. \bibverse{12} Y
tomarás de la sangre del becerro, y pondrás sobre los cuernos del altar
con tu dedo, y derramarás toda la demás sangre al pie del altar.
\bibverse{13} Tomarás también todo el sebo que cubre los intestinos, y
el redaño de sobre el hígado, y los dos riñones, y el sebo que está
sobre ellos, y los quemarás sobre el altar. \bibverse{14} Empero
consumirás á fuego fuera del campo la carne del becerro, y su pellejo, y
su estiércol: es expiación. \bibverse{15} Asimismo tomarás el un
carnero, y Aarón y sus hijos pondrán sus manos sobre la cabeza del
carnero. \bibverse{16} Y matarás el carnero, y tomarás su sangre, y
rociarás sobre el altar alrededor. \bibverse{17} Y cortarás el carnero
en pedazos, y lavarás sus intestinos y sus piernas, y las pondrás sobre
sus trozos y sobre su cabeza. \bibverse{18} Y quemarás todo el carnero
sobre el altar: es holocausto á Jehová, olor grato, es ofrenda quemada á
Jehová. \bibverse{19} Tomarás luego el otro carnero, y Aarón y sus hijos
pondrán sus manos sobre la cabeza del carnero: \bibverse{20} Y matarás
el carnero, y tomarás de su sangre, y pondrás sobre la ternilla de la
oreja derecha de Aarón, y sobre la ternilla de las orejas de sus hijos,
y sobre el dedo pulgar de las manos derechas de ellos, y sobre el dedo
pulgar de los pies derechos de ellos, y esparcirás la sangre sobre el
altar alrededor. \bibverse{21} Y tomarás de la sangre que hay sobre el
altar, y del aceite de la unción, y esparcirás sobre Aarón, y sobre sus
vestiduras, y sobre sus hijos, y sobre las vestimentas de éstos; y él
será santificado, y sus vestiduras, y sus hijos, y las vestimentas de
sus hijos con él. \bibverse{22} Luego tomarás del carnero el sebo, y la
cola, y el sebo que cubre los intestinos, y el redaño del hígado, y los
dos riñones, y el sebo que está sobre ellos, y la espaldilla derecha;
porque es carnero de consagraciones: \bibverse{23} También una torta de
pan, y una hojaldre amasada con aceite, y una lasaña del canastillo de
los ázimos presentado á Jehová; \bibverse{24} Y lo has de poner todo en
las manos de Aarón, y en las manos de sus hijos; y lo mecerás agitándolo
delante de Jehová. \bibverse{25} Después lo tomarás de sus manos, y lo
harás arder sobre el altar en holocausto, por olor agradable delante de
Jehová. Es ofrenda encendida á Jehová. \bibverse{26} Y tomarás el pecho
del carnero de las consagraciones, que fué inmolado para la de Aarón, y
lo mecerás por ofrenda agitada delante de Jehová; y será porción tuya.
\bibverse{27} Y apartarás el pecho de la ofrenda mecida, y la espaldilla
de la santificación, lo que fué mecido y lo que fué santificado del
carnero de las consagraciones de Aarón y de sus hijos: \bibverse{28} Y
será para Aarón y para sus hijos por estatuto perpetuo de los hijos de
Israel, porque es porción elevada; y será tomada de los hijos de Israel
de sus sacrificios pacíficos, porción de ellos elevada en ofrenda á
Jehová. \bibverse{29} Y las vestimentas santas, que son de Aarón, serán
de sus hijos después de él, para ser ungidos con ellas, y para ser con
ellas consagrados. \bibverse{30} Por siete días las vestirá el sacerdote
de sus hijos, que en su lugar viniere al tabernáculo del testimonio á
servir en el santuario. \bibverse{31} Y tomarás el carnero de las
consagraciones, y cocerás su carne en el lugar del santuario.
\bibverse{32} Y Aarón y sus hijos comerán la carne del carnero, y el pan
que está en el canastillo, á la puerta del tabernáculo del testimonio.
\bibverse{33} Y comerán aquellas cosas con las cuales se hizo expiación,
para henchir sus manos para ser santificados: mas el extranjero no
comerá, porque es cosa santa. \bibverse{34} Y si sobrare algo de la
carne de las consagraciones y del pan hasta la mañana, quemarás al fuego
lo que hubiere sobrado: no se comerá, porque es cosa santa.
\bibverse{35} Así pues harás á Aarón y á sus hijos, conforme á todas las
cosas que yo te he mandado: por siete días los consagrarás.
\bibverse{36} Y sacrificarás el becerro de la expiación en cada día para
las expiaciones; y purificarás el altar en habiendo hecho expiación por
él, y lo ungirás para santificarlo. \bibverse{37} Por siete días
expiarás el altar, y lo santificarás, y será un altar santísimo:
cualquiera cosa que tocare al altar, será santificada. \bibverse{38} Y
esto es lo que ofrecerás sobre el altar: dos corderos de un año cada
día, sin intermisión. \bibverse{39} Ofrecerás el un cordero á la mañana,
y el otro cordero ofrecerás á la caída de la tarde: \bibverse{40} Además
una décima parte de un epha de flor de harina amasada con la cuarta
parte de un hin de aceite molido: y la libación será la cuarta parte de
un hin de vino con cada cordero. \bibverse{41} Y ofrecerás el otro
cordero á la caída de la tarde, haciendo conforme á la ofrenda de la
mañana, y conforme á su libación, en olor de suavidad; será ofrenda
encendida á Jehová. \bibverse{42} Esto será holocausto continuo por
vuestras generaciones á la puerta del tabernáculo del testimonio delante
de Jehová, en el cual me concertaré con vosotros, para hablaros allí,
\bibverse{43} Y allí testificaré de mí á los hijos de Israel, y el lugar
será santificado con mi gloria. \bibverse{44} Y santificaré el
tabernáculo del testimonio y el altar: santificaré asimismo á Aarón y á
sus hijos, para que sean mis sacerdotes. \bibverse{45} Y habitaré entre
los hijos de Israel, y seré su Dios. \bibverse{46} Y conocerán que yo
soy Jehová su Dios, que los saqué de la tierra de Egipto, para habitar
en medio de ellos: Yo Jehová su Dios.

\hypertarget{section-29}{%
\section{30}\label{section-29}}

\bibverse{1} Harás asimismo un altar de sahumerio de perfume: de madera
de Sittim lo harás. \bibverse{2} Su longitud será de un codo, y su
anchura de un codo: será cuadrado: y su altura de dos codos: y sus
cuernos serán de lo mismo. \bibverse{3} Y cubrirlo has de oro puro, su
techado, y sus paredes en derredor, y sus cuernos: y le harás en
derredor una corona de oro. \bibverse{4} Le harás también dos anillos de
oro debajo de su corona á sus dos esquinas en ambos lados suyos, para
meter los varales con que será llevado. \bibverse{5} Y harás los varales
de madera de Sittim, y los cubrirás de oro. \bibverse{6} Y lo pondrás
delante del velo que está junto al arca del testimonio, delante de la
cubierta que está sobre el testimonio, donde yo te testificaré de mí.
\bibverse{7} Y quemará sobre él Aarón sahumerio de aroma cada mañana:
cuando aderezare las lámparas lo quemará. \bibverse{8} Y cuando Aarón
encenderá las lámparas al anochecer, quemará el sahumerio: rito perpetuo
delante de Jehová por vuestras edades. \bibverse{9} No ofreceréis sobre
él sahumerio extraño, ni holocausto, ni presente; ni tampoco derramaréis
sobre él libación. \bibverse{10} Y sobre sus cuernos hará Aarón
expiación una vez en el año con la sangre de la expiación para las
reconciliaciones: una vez en el año hará expiación sobre él en vuestras
edades: será muy santo á Jehová. \bibverse{11} Y habló Jehová á Moisés,
diciendo: \bibverse{12} Cuando tomares el número de los hijos de Israel
conforme á la cuenta de ellos, cada uno dará á Jehová el rescate de su
persona, cuando los contares, y no habrá en ellos mortandad por haberlos
contado. \bibverse{13} Esto dará cualquiera que pasare por la cuenta,
medio siclo conforme al siclo del santuario. El siclo es de veinte
óbolos: la mitad de un siclo será la ofrenda á Jehová. \bibverse{14}
Cualquiera que pasare por la cuenta, de veinte años arriba, dará la
ofrenda á Jehová. \bibverse{15} Ni el rico aumentará, ni el pobre
disminuirá de medio siclo, cuando dieren la ofrenda á Jehová para hacer
expiación por vuestras personas. \bibverse{16} Y tomarás de los hijos de
Israel el dinero de las expiaciones, y lo darás para la obra del
tabernáculo del testimonio: y será por memoria á los hijos de Israel
delante de Jehová, para expiar vuestras personas. \bibverse{17} Habló
más Jehová á Moisés, diciendo: \bibverse{18} Harás también una fuente de
metal, con su basa de metal, para lavar; y la has de poner entre el
tabernáculo del testimonio y el altar; y pondrás en ella agua.
\bibverse{19} Y de ella se lavarán Aarón y sus hijos sus manos y sus
pies: \bibverse{20} Cuando entraren en el tabernáculo del testimonio, se
han de lavar con agua, y no morirán: y cuando se llegaren al altar para
ministrar, para encender á Jehová la ofrenda que se ha de consumir al
fuego, \bibverse{21} También se lavarán las manos y los pies, y no
morirán. Y lo tendrán por estatuto perpetuo él y su simiente por sus
generaciones. \bibverse{22} Habló más Jehová á Moisés, diciendo:
\bibverse{23} Y tú has de tomar de las principales drogas; de mirra
excelente quinientos siclos, y de canela aromática la mitad, esto es,
doscientos y cincuenta, y de cálamo aromático doscientos y cincuenta,
\bibverse{24} Y de casia quinientos, al peso del santuario, y de aceite
de olivas un hin: \bibverse{25} Y harás de ello el aceite de la santa
unción, superior ungüento, obra de perfumador, el cual será el aceite de
la unción sagrada. \bibverse{26} Con él ungirás el tabernáculo del
testimonio, y el arca del testimonio, \bibverse{27} Y la mesa, y todos
sus vasos, y el candelero, y todos sus vasos, y el altar del perfume,
\bibverse{28} Y el altar del holocausto, todos sus vasos, y la fuente y
su basa. \bibverse{29} Así los consagrarás, y serán cosas santísimas:
todo lo que tocare en ellos, será santificado. \bibverse{30} Ungirás
también á Aarón y á sus hijos, y los consagrarás para que sean mis
sacerdotes. \bibverse{31} Y hablarás á los hijos de Israel, diciendo:
Este será mi aceite de la santa unción por vuestras edades.
\bibverse{32} Sobre carne de hombre no será untado, ni haréis otro
semejante, conforme á su composición: santo es; por santo habéis de
tenerlo vosotros. \bibverse{33} Cualquiera que compusiere ungüento
semejante, y que pusiere de él sobre extraño, será cortado de sus
pueblos. \bibverse{34} Dijo aún Jehová á Moisés: Tómate aromas, estacte
y uña olorosa y gálbano aromático é incienso limpio; de todo en igual
peso: \bibverse{35} Y harás de ello una confección aromática de obra de
perfumador, bien mezclada, pura y santa: \bibverse{36} Y molerás alguna
de ella pulverizándola, y la pondrás delante del testimonio en el
tabernáculo del testimonio, donde yo te testificaré de mí. Os será cosa
santísima. \bibverse{37} Como la confección que harás, no os haréis otra
según su composición: te será cosa sagrada para Jehová. \bibverse{38}
Cualquiera que hiciere otra como ella para olerla, será cortado de sus
pueblos.

\hypertarget{section-30}{%
\section{31}\label{section-30}}

\bibverse{1} Y habló Jehová á Moisés, diciendo: \bibverse{2} Mira, yo he
llamado por su nombre á Bezaleel, hijo de Uri, hijo de Hur, de la tribu
de Judá; \bibverse{3} Y lo he henchido de espíritu de Dios, en
sabiduría, y en inteligencia, y en ciencia, y en todo artificio,
\bibverse{4} Para inventar diseños, para trabajar en oro, y en plata, y
en metal, \bibverse{5} Y en artificio de piedras para engastarlas, y en
artificio de madera; para obrar en toda suerte de labor. \bibverse{6} Y
he aquí que yo he puesto con él á Aholiab, hijo de Ahisamac, de la tribu
de Dan: y he puesto sabiduría en el ánimo de todo sabio de corazón, para
que hagan todo lo que te he mandado: \bibverse{7} El tabernáculo del
testimonio, y el arca del testimonio, y la cubierta que está sobre ella,
y todos los vasos del tabernáculo; \bibverse{8} Y la mesa y sus vasos, y
el candelero limpio y todos sus vasos, y el altar del perfume;
\bibverse{9} Y el altar del holocausto y todos sus vasos, y la fuente y
su basa; \bibverse{10} Y los vestidos del servicio, y las santas
vestiduras para Aarón el sacerdote, y las vestiduras de sus hijos, para
que ejerzan el sacerdocio; \bibverse{11} Y el aceite de la unción, y el
perfume aromático para el santuario: harán conforme á todo lo que te he
mandado. \bibverse{12} Habló además Jehová á Moisés, diciendo:
\bibverse{13} Y tú hablarás á los hijos de Israel, diciendo: Con todo
eso vosotros guardaréis mis sábados: porque es señal entre mí y vosotros
por vuestras edades, para que sepáis que yo soy Jehová que os santifico.
\bibverse{14} Así que guardaréis el sábado, porque santo es á vosotros:
el que lo profanare, de cierto morirá; porque cualquiera que hiciere
obra alguna en él, aquella alma será cortada de en medio de sus pueblos.
\bibverse{15} Seis días se hará obra, mas el día séptimo es sábado de
reposo consagrado á Jehová; cualquiera que hiciere obra el día del
sábado, morirá ciertamente. \bibverse{16} Guardarán, pues, el sábado los
hijos de Israel: celebrándolo por sus edades por pacto perpetuo:
\bibverse{17} Señal es para siempre entre mí y los hijos de Israel;
porque en seis días hizo Jehová los cielos y la tierra, y en el séptimo
día cesó, y reposó. \bibverse{18} Y dió á Moisés, como acabó de hablar
con él en el monte de Sinaí, dos tablas del testimonio, tablas de piedra
escritas con el dedo de Dios.

\hypertarget{section-31}{%
\section{32}\label{section-31}}

\bibverse{1} Mas viendo el pueblo que Moisés tardaba en descender del
monte, allegóse entonces á Aarón, y dijéronle: Levántate, haznos dioses
que vayan delante de nosotros; porque á este Moisés, aquel varón que nos
sacó de la tierra de Egipto, no sabemos qué le haya acontecido.
\bibverse{2} Y Aarón les dijo: Apartad los zarcillos de oro que están en
las orejas de vuestras mujeres, y de vuestros hijos, y de vuestras
hijas, y traédmelos. \bibverse{3} Entonces todo el pueblo apartó los
zarcillos de oro que tenían en sus orejas, y trajéronlos á Aarón:
\bibverse{4} El cual los tomó de las manos de ellos, y formólo con
buril, é hizo de ello un becerro de fundición. Entonces dijeron: Israel,
estos son tus dioses, que te sacaron de la tierra de Egipto.
\bibverse{5} Y viendo esto Aarón, edificó un altar delante del becerro;
y pregonó Aarón, y dijo: Mañana será fiesta á Jehová. \bibverse{6} Y el
día siguiente madrugaron, y ofrecieron holocaustos, y presentaron
pacíficos: y sentóse el pueblo á comer y á beber, y levantáronse á
regocijarse. \bibverse{7} Entonces Jehová dijo á Moisés: Anda,
desciende, porque tu pueblo que sacaste de tierra de Egipto se ha
corrompido: \bibverse{8} Presto se han apartado del camino que yo les
mandé, y se han hecho un becerro de fundición, y lo han adorado, y han
sacrificado á él, y han dicho: Israel, estos son tus dioses, que te
sacaron de la tierra de Egipto. \bibverse{9} Dijo más Jehová á Moisés:
Yo he visto á este pueblo, que por cierto es pueblo de dura cerviz:
\bibverse{10} Ahora pues, déjame que se encienda mi furor en ellos, y
los consuma: y á ti yo te pondré sobre gran gente. \bibverse{11}
Entonces Moisés oró á la faz de Jehová su Dios, y dijo: Oh Jehová, ¿por
qué se encenderá tu furor en tu pueblo, que tú sacaste de la tierra de
Egipto con gran fortaleza, y con mano fuerte? \bibverse{12} ¿Por qué han
de hablar los Egipcios, diciendo: Para mal los sacó, para matarlos en
los montes, y para raerlos de sobre la haz de la tierra? Vuélvete del
furor de tu ira, y arrepiéntete del mal de tu pueblo. \bibverse{13}
Acuérdate de Abraham, de Isaac, y de Israel, tus siervos, á los cuales
has jurado por ti mismo, y dícholes: Yo multiplicaré vuestra simiente
como las estrellas del cielo; y daré á vuestra simiente toda esta tierra
que he dicho, y la tomarán por heredad para siempre. \bibverse{14}
Entonces Jehová se arrepintió del mal que dijo que había de hacer á su
pueblo. \bibverse{15} Y volvióse Moisés, y descendió del monte trayendo
en su mano las dos tablas del testimonio, las tablas escritas por ambos
lados; de una parte y de otra estaban escritas. \bibverse{16} Y las
tablas eran obra de Dios, y la escritura era escritura de Dios grabada
sobre las tablas. \bibverse{17} Y oyendo Josué el clamor del pueblo que
gritaba, dijo á Moisés: Alarido de pelea hay en el campo. \bibverse{18}
Y él respondió: No es eco de algazara de fuertes, ni eco de alaridos de
flacos: algazara de cantar oigo yo. \bibverse{19} Y aconteció, que como
llegó él al campo, y vió el becerro y las danzas, enardeciósele la ira á
Moisés, y arrojó las tablas de sus manos, y quebrólas al pie del monte.
\bibverse{20} Y tomó el becerro que habían hecho, y quemólo en el fuego,
y moliólo hasta reducirlo á polvo, que esparció sobre las aguas, y diólo
á beber á los hijos de Israel. \bibverse{21} Y dijo Moisés á Aarón: ¿Qué
te ha hecho este pueblo, que has traído sobre él tan gran pecado?
\bibverse{22} Y respondió Aarón: No se enoje mi señor; tú conoces el
pueblo, que es inclinado á mal. \bibverse{23} Porque me dijeron: Haznos
dioses que vayan delante de nosotros, que á este Moisés, el varón que
nos sacó de tierra de Egipto, no sabemos qué le ha acontecido.
\bibverse{24} Y yo les respondí: ¿Quién tiene oro? apartadlo. Y
diéronmelo, y echélo en el fuego, y salió este becerro. \bibverse{25} Y
viendo Moisés que el pueblo estaba despojado, porque Aarón lo había
despojado para vergüenza entre sus enemigos, \bibverse{26} Púsose Moisés
á la puerta del real, y dijo: ¿Quién es de Jehová? júntese conmigo. Y
juntáronse con él todos los hijos de Leví. \bibverse{27} Y él les dijo:
Así ha dicho Jehová, el Dios de Israel: Poned cada uno su espada sobre
su muslo: pasad y volved de puerta á puerta por el campo, y matad cada
uno á su hermano, y á su amigo, y á su pariente. \bibverse{28} Y los
hijos de Leví lo hicieron conforme al dicho de Moisés: y cayeron del
pueblo en aquel día como tres mil hombres. \bibverse{29} Entonces Moisés
dijo: Hoy os habéis consagrado á Jehová, porque cada uno se ha
consagrado en su hijo, y en su hermano, para que dé él hoy bendición
sobre vosotros. \bibverse{30} Y aconteció que el día siguiente dijo
Moisés al pueblo: Vosotros habéis cometido un gran pecado: mas yo subiré
ahora á Jehová; quizá le aplacaré acerca de vuestro pecado.
\bibverse{31} Entonces volvió Moisés á Jehová, y dijo: Ruégote, pues
este pueblo ha cometido un gran pecado, porque se hicieron dioses de
oro, \bibverse{32} Que perdones ahora su pecado, y si no, ráeme ahora de
tu libro que has escrito. \bibverse{33} Y Jehová respondió á Moisés: Al
que pecare contra mí, á éste raeré yo de mi libro. \bibverse{34} Ve pues
ahora, lleva á este pueblo donde te he dicho: he aquí mi ángel irá
delante de ti; que en el día de mi visitación yo visitaré en ellos su
pecado. \bibverse{35} Y Jehová hirió al pueblo, porque habían hecho el
becerro que formó Aarón.

\hypertarget{section-32}{%
\section{33}\label{section-32}}

\bibverse{1} Y jehová dijo á Moisés: Ve, sube de aquí, tú y el pueblo
que sacaste de la tierra de Egipto, á la tierra de la cual juré á
Abraham, Isaac, y Jacob, diciendo: A tu simiente la daré: \bibverse{2} Y
yo enviaré delante de ti el ángel, y echaré fuera al Cananeo y al
Amorrheo, y al Hetheo, y al Pherezeo, y al Heveo y al Jebuseo:
\bibverse{3} (A la tierra que fluye leche y miel); porque yo no subiré
en medio de ti, porque eres pueblo de dura cerviz, no sea que te consuma
en el camino. \bibverse{4} Y oyendo el pueblo esta sensible palabra,
vistieron luto, y ninguno se puso sus atavíos: \bibverse{5} Pues Jehová
dijo á Moisés: Di á los hijos de Israel: Vosotros sois pueblo de dura
cerviz: en un momento subiré en medio de ti, y te consumiré: quítate
pues ahora tus atavíos, que yo sabré lo que te tengo de hacer.
\bibverse{6} Entonces los hijos de Israel se despojaron de sus atavíos
desde el monte Horeb. \bibverse{7} Y Moisés tomó el tabernáculo, y
extendiólo fuera del campo, lejos del campo, y llamólo el Tabernáculo
del Testimonio. Y fué, que cualquiera que requería á Jehová, salía al
tabernáculo del testimonio, que estaba fuera del campo. \bibverse{8} Y
sucedía que, cuando salía Moisés al tabernáculo, todo el pueblo se
levantaba, y estaba cada cual en pie á la puerta de su tienda, y miraban
en pos de Moisés, hasta que él entraba en el tabernáculo. \bibverse{9} Y
cuando Moisés entraba en el tabernáculo, la columna de nube descendía, y
poníase á la puerta del tabernáculo, y Jehová hablaba con Moisés.
\bibverse{10} Y viendo todo el pueblo la columna de la nube, que estaba
á la puerta del tabernáculo, levantábase todo el pueblo, cada uno á la
puerta de su tienda, y adoraba. \bibverse{11} Y hablaba Jehová á Moisés
cara á cara, como habla cualquiera á su compañero. Y volvíase al campo;
mas el joven Josué, su criado, hijo de Nun, nunca se apartaba de en
medio del tabernáculo. \bibverse{12} Y dijo Moisés á Jehová: Mira, tú me
dices á mí: Saca este pueblo: y tú no me has declarado á quién has de
enviar conmigo: sin embargo tú dices: Yo te he conocido por tu nombre, y
has hallado también gracia en mis ojos. \bibverse{13} Ahora, pues, si he
hallado gracia en tus ojos, ruégote que me muestres ahora tu camino,
para que te conozca, porque halle gracia en tus ojos: y mira que tu
pueblo es aquesta gente. \bibverse{14} Y él dijo: Mi rostro irá contigo,
y te haré descansar. \bibverse{15} Y él respondió: Si tu rostro no ha de
ir conmigo, no nos saques de aquí. \bibverse{16} ¿Y en qué se conocerá
aquí que he hallado gracia en tus ojos, yo y tu pueblo, sino en andar tú
con nosotros, y que yo y tu pueblo seamos apartados de todos los pueblos
que están sobre la faz de la tierra? \bibverse{17} Y Jehová dijo á
Moisés: También haré esto que has dicho, por cuanto has hallado gracia
en mis ojos, y te he conocido por tu nombre. \bibverse{18} El entonces
dijo: Ruégote que me muestres tu gloria. \bibverse{19} Y respondióle: Yo
haré pasar todo mi bien delante de tu rostro, y proclamaré el nombre de
Jehová delante de ti; y tendré misericordia del que tendré misericordia,
y seré clemente para con el que seré clemente. \bibverse{20} Dijo más:
No podrás ver mi rostro: porque no me verá hombre, y vivirá.
\bibverse{21} Y dijo aún Jehová: He aquí lugar junto á mí, y tú estarás
sobre la peña: \bibverse{22} Y será que, cuando pasare mi gloria, yo te
pondré en una hendidura de la peña, y te cubriré con mi mano hasta que
haya pasado: \bibverse{23} Después apartaré mi mano, y verás mis
espaldas; mas no se verá mi rostro.

\hypertarget{section-33}{%
\section{34}\label{section-33}}

\bibverse{1} Y jehová dijo á Moisés: Alísate dos tablas de piedra como
las primeras, y escribiré sobre esas tablas las palabras que estaban en
las tablas primeras que quebraste. \bibverse{2} Apercíbete, pues, para
mañana, y sube por la mañana al monte de Sinaí, y estáme allí sobre la
cumbre del monte. \bibverse{3} Y no suba hombre contigo, ni parezca
alguno en todo el monte; ni ovejas ni bueyes pazcan delante del monte.
\bibverse{4} Y Moisés alisó dos tablas de piedra como las primeras; y
levantóse por la mañana, y subió al monte de Sinaí, como le mandó
Jehová, y llevó en su mano las dos tablas de piedra. \bibverse{5} Y
Jehová descendió en la nube, y estuvo allí con él, proclamando el nombre
de Jehová. \bibverse{6} Y pasando Jehová por delante de él, proclamó:
Jehová, Jehová, fuerte, misericordioso, y piadoso; tardo para la ira, y
grande en benignidad y verdad; \bibverse{7} Que guarda la misericordia
en millares, que perdona la iniquidad, la rebelión, y el pecado, y que
de ningún modo justificará al malvado; que visita la iniquidad de los
padres sobre los hijos y sobre los hijos de los hijos, sobre los
terceros, y sobre los cuartos. \bibverse{8} Entonces Moisés,
apresurándose, bajó la cabeza hacia el suelo y encorvóse; \bibverse{9} Y
dijo: Si ahora, Señor, he hallado gracia en tus ojos, vaya ahora el
Señor en medio de nosotros; porque este es pueblo de dura cerviz; y
perdona nuestra iniquidad y nuestro pecado, y poséenos. \bibverse{10} Y
él dijo: He aquí, yo hago concierto delante de todo tu pueblo: haré
maravillas que no han sido hechas en toda la tierra, ni en nación
alguna; y verá todo el pueblo en medio del cual estás tú, la obra de
Jehová; porque ha de ser cosa terrible la que yo haré contigo.
\bibverse{11} Guarda lo que yo te mando hoy; he aquí que yo echo de
delante de tu presencia al Amorrheo, y al Cananeo, y al Hetheo, y al
Pherezeo, y al Heveo, y al Jebuseo. \bibverse{12} Guárdate que no hagas
alianza con los moradores de la tierra donde has de entrar, porque no
sean por tropezadero en medio de tí: \bibverse{13} Mas derribaréis sus
altares, y quebraréis sus estatuas, y talaréis sus bosques:
\bibverse{14} Porque no te has de inclinar á dios ajeno; que Jehová,
cuyo nombre es Celoso, Dios celoso es. \bibverse{15} Por tanto no harás
alianza con los moradores de aquella tierra; porque fornicarán en pos de
sus dioses, y sacrificarán á sus dioses, y te llamarán, y comerás de sus
sacrificios; \bibverse{16} O tomando de sus hijas para tus hijos, y
fornicando sus hijas en pos de sus dioses, harán también fornicar á tus
hijos en pos de los dioses de ellas. \bibverse{17} No harás dioses de
fundición para ti. \bibverse{18} La fiesta de los ázimos guardarás:
siete días comerás por leudar, según te he mandado, en el tiempo del mes
de Abib; porque en el mes de Abib saliste de Egipto. \bibverse{19} Todo
lo que abre matriz, mío es; y de tu ganado todo primerizo de vaca ó de
oveja que fuere macho. \bibverse{20} Empero redimirás con cordero el
primerizo del asno; y si no lo redimieres, le has de cortar la cabeza.
Redimirás todo primogénito de tus hijos, y no serán vistos vacíos
delante de mí. \bibverse{21} Seis días trabajarás, mas en el séptimo día
cesarás: cesarás aun en la arada y en la siega. \bibverse{22} Y te harás
la fiesta de las semanas á los principios de la siega del trigo: y la
fiesta de la cosecha á la vuelta del año. \bibverse{23} Tres veces en el
año será visto todo varón tuyo delante del Señoreador Jehová, Dios de
Israel. \bibverse{24} Porque yo arrojaré las gentes de tu presencia, y
ensancharé tu término: y ninguno codiciará tu tierra, cuando tú subieres
para ser visto delante de Jehová tu Dios tres veces en el año.
\bibverse{25} No ofrecerás con leudo la sangre de mi sacrificio; ni
quedará de la noche para la mañana el sacrificio de la fiesta de la
pascua. \bibverse{26} La primicia de los primeros frutos de tu tierra
meterás en la casa de Jehová tu Dios. No cocerás el cabrito en la leche
de su madre. \bibverse{27} Y Jehová dijo á Moisés: Escribe tú estas
palabras; porque conforme á estas palabras he hecho la alianza contigo y
con Israel. \bibverse{28} Y él estuvo allí con Jehová cuarenta días y
cuarenta noches: no comió pan, ni bebió agua; y escribió en tablas las
palabras de la alianza, las diez palabras. \bibverse{29} Y aconteció,
que descendiendo Moisés del monte Sinaí con las dos tablas del
testimonio en su mano, mientras descendía del monte, no sabía él que la
tez de su rostro resplandecía, después que hubo con El hablado.
\bibverse{30} Y miró Aarón y todos los hijos de Israel á Moisés, y he
aquí la tez de su rostro era resplandeciente; y tuvieron miedo de
llegarse á él. \bibverse{31} Y llamólos Moisés; y Aarón y todos los
príncipes de la congregación volvieron á él, y Moisés les habló.
\bibverse{32} Y después se llegaron todos los hijos de Israel, á los
cuales mandó todas las cosas que Jehová le había dicho en el monte de
Sinaí. \bibverse{33} Y cuando hubo acabado Moisés de hablar con ellos,
puso un velo sobre su rostro. \bibverse{34} Y cuando venía Moisés
delante de Jehová para hablar con él, quitábase el velo hasta que salía;
y saliendo, hablaba con los hijos de Israel lo que le era mandado;
\bibverse{35} Y veían los hijos de Israel el rostro de Moisés, que la
tez de su rostro era resplandeciente; y volvía Moisés á poner el velo
sobre su rostro, hasta que entraba á hablar con El.

\hypertarget{section-34}{%
\section{35}\label{section-34}}

\bibverse{1} Y moisés hizo juntar toda la congregación de los hijos de
Israel, y díjoles: Estas son las cosas que Jehová ha mandado que hagáis.
\bibverse{2} Seis días se hará obra, mas el día séptimo os será santo,
sábado de reposo á Jehová: cualquiera que en él hiciere obra, morirá.
\bibverse{3} No encenderéis fuego en todas vuestras moradas en el día
del sábado. \bibverse{4} Y habló Moisés á toda la congregación de los
hijos de Israel, diciendo: Esto es lo que Jehová ha mandado, diciendo:
\bibverse{5} Tomad de entre vosotros ofrenda para Jehová: todo liberal
de corazón la traerá á Jehová: oro, plata, metal; \bibverse{6} Y
cárdeno, y púrpura, y carmesí, y lino fino, y pelo de cabras;
\bibverse{7} Y cueros rojos de carneros, y cueros de tejones, y madera
de Sittim; \bibverse{8} Y aceite para la luminaria, y especias
aromáticas para el aceite de la unción, y para el perfume aromático;
\bibverse{9} Y piedras de onix, y demás pedrería, para el ephod, y para
el racional. \bibverse{10} Y todo sabio de corazón entre vosotros,
vendrá y hará todas las cosas que Jehová ha mandado: \bibverse{11} El
tabernáculo, su tienda, y su cubierta, y sus anillos, y sus tablas, sus
barras, sus columnas, y sus basas; \bibverse{12} El arca, y sus varas,
la cubierta, y el velo de la tienda; \bibverse{13} La mesa, y sus varas,
y todos sus vasos, y el pan de la proposición. \bibverse{14} El
candelero de la luminaria, y sus vasos, y sus candilejas, y el aceite
para la luminaria; \bibverse{15} Y el altar del perfume, y sus varas, y
el aceite de la unción, y el perfume aromático, y el pabellón de la
puerta, para la entrada del tabernáculo; \bibverse{16} El altar del
holocausto, y su enrejado de metal, y sus varas, y todos sus vasos, y la
fuente con su basa; \bibverse{17} Las cortinas del atrio, sus columnas,
y sus basas, y el pabellón de la puerta del atrio; \bibverse{18} Las
estacas del tabernáculo, y las estacas del atrio, y sus cuerdas;
\bibverse{19} Las vestiduras del servicio para ministrar en el
santuario, las sagradas vestiduras de Aarón el sacerdote, y las
vestiduras de sus hijos para servir en el sacerdocio. \bibverse{20} Y
salió toda la congregación de los hijos de Israel de delante de Moisés.
\bibverse{21} Y vino todo varón á quien su corazón estimuló, y todo
aquel á quien su espíritu le dió voluntad, y trajeron ofrenda á Jehová
para la obra del tabernáculo del testimonio, y para toda su fábrica, y
para las sagradas vestiduras. \bibverse{22} Y vinieron así hombres como
mujeres, todo voluntario de corazón, y trajeron cadenas y zarcillos,
sortijas y brazaletes, y toda joya de oro; y cualquiera ofrecía ofrenda
de oro á Jehová. \bibverse{23} Todo hombre que se hallaba con jacinto, ó
púrpura, ó carmesí, ó lino fino, ó pelo de cabras, ó cueros rojos de
carneros, ó cueros de tejones, lo traía. \bibverse{24} Cualquiera que
ofrecía ofrenda de plata ó de metal, traía á Jehová la ofrenda: y todo
el que se hallaba con madera de Sittim, traíala para toda la obra del
servicio. \bibverse{25} Además todas las mujeres sabias de corazón
hilaban de sus manos, y traían lo que habían hilado: cárdeno, ó púrpura,
ó carmesí, ó lino fino. \bibverse{26} Y todas las mujeres cuyo corazón
las levantó en sabiduría, hilaron pelos de cabras. \bibverse{27} Y los
príncipes trajeron piedras de onix, y las piedras de los engastes para
el ephod y el racional; \bibverse{28} Y la especia aromática y aceite,
para la luminaria, y para el aceite de la unción, y para el perfume
aromático. \bibverse{29} De los hijos de Israel, así hombres como
mujeres, todos los que tuvieron corazón voluntario para traer para toda
la obra, que Jehová había mandado por medio de Moisés que hiciesen,
trajeron ofrenda voluntaria á Jehová. \bibverse{30} Y dijo Moisés á los
hijos de Israel: Mirad, Jehová ha nombrado á Bezaleel, hijo de Uri, hijo
de Hur, de la tribu de Judá; \bibverse{31} Y lo ha henchido de espíritu
de Dios, en sabiduría, en inteligencia, y en ciencia, y en todo
artificio, \bibverse{32} Para proyectar inventos, para trabajar en oro,
y en plata, y en metal, \bibverse{33} Y en obra de pedrería para
engastar, y en obra de madera, para trabajar en toda invención
ingeniosa. \bibverse{34} Y ha puesto en su corazón el que pueda enseñar,
así él como Aholiab, hijo de Ahisamac, de la tribu de Dan: \bibverse{35}
Y los ha henchido de sabiduría de corazón, para que hagan toda obra de
artificio, y de invención, y de recamado en jacinto, y en púrpura, y en
carmesí, y en lino fino, y en telar; para que hagan toda labor, é
inventen todo diseño.

\hypertarget{section-35}{%
\section{36}\label{section-35}}

\bibverse{1} Hizo, pues, Bezaleel y Aholiab, y todo hombre sabio de
corazón, á quien Jehová dió sabiduría é inteligencia para que supiesen
hacer toda la obra del servicio del santuario, todas las cosas que había
mandado Jehová. \bibverse{2} Y Moisés llamó á Bezaleel y á Aholiab, y á
todo varón sabio de corazón, en cuyo corazón había dado Jehová
sabiduría, y á todo hombre á quien su corazón le movió á llegarse á la
obra, para trabajar en ella; \bibverse{3} Y tomaron de delante de Moisés
toda la ofrenda que los hijos de Israel habían traído para la obra del
servicio del santuario, á fin de hacerla. Y ellos le traían aún ofrenda
voluntaria cada mañana. \bibverse{4} Vinieron, por tanto, todos los
maestros que hacían toda la obra del santuario, cada uno de la obra que
hacía. \bibverse{5} Y hablaron á Moisés, diciendo: El pueblo trae mucho
más de lo que es menester para la atención de hacer la obra que Jehová
ha mandado que se haga. \bibverse{6} Entonces Moisés mandó pregonar por
el campo, diciendo: Ningún hombre ni mujer haga más obra para ofrecer
para el santuario. Y así fué el pueblo impedido de ofrecer; \bibverse{7}
Pues tenía material abundante para hacer toda la obra, y sobraba.
\bibverse{8} Y todos los sabios de corazón entre los que hacían la obra,
hicieron el tabernáculo de diez cortinas, de lino torcido, y de jacinto,
y de púrpura y carmesí; las cuales hicieron de obra prima, con
querubines. \bibverse{9} La longitud de la una cortina era de veintiocho
codos, y la anchura de cuatro codos: todas las cortinas tenían una misma
medida. \bibverse{10} Y juntó las cinco cortinas la una con la otra:
asimismo unió las otras cinco cortinas la una con la otra. \bibverse{11}
E hizo las lazadas de color de jacinto en la orilla de la una cortina,
en el borde, á la juntura; y así hizo en la orilla al borde de la
segunda cortina, en la juntura. \bibverse{12} Cincuenta lazadas hizo en
la una cortina, y otras cincuenta en la segunda cortina, en el borde, en
la juntura; las unas lazadas enfrente de las otras. \bibverse{13} Hizo
también cincuenta corchetes de oro, con los cuales juntó las cortinas,
la una con la otra; é hízose un tabernáculo. \bibverse{14} Hizo asimismo
cortinas de pelo de cabras para la tienda sobre el tabernáculo, é
hízolas en número de once. \bibverse{15} La longitud de la una cortina
era de treinta codos, y la anchura de cuatro codos: las once cortinas
tenían una misma medida. \bibverse{16} Y juntó las cinco cortinas de por
sí, y las seis cortinas aparte. \bibverse{17} Hizo además cincuenta
lazadas en la orilla de la postrera cortina en la juntura, y otras
cincuenta lazadas en la orilla de la otra cortina en la juntura.
\bibverse{18} Hizo también cincuenta corchetes de metal para juntar la
tienda, de modo que fuese una. \bibverse{19} E hizo una cubierta para la
tienda de cueros rojos de carneros, y una cubierta encima de cueros de
tejones. \bibverse{20} Además hizo las tablas para el tabernáculo de
madera de Sittim, para estar derechas. \bibverse{21} La longitud de cada
tabla de diez codos, y de codo y medio la anchura. \bibverse{22} Cada
tabla tenía dos quicios enclavijados el uno delante del otro: así hizo
todas las tablas del tabernáculo. \bibverse{23} Hizo, pues, las tablas
para el tabernáculo: veinte tablas al lado del austro, al mediodía.
\bibverse{24} Hizo también las cuarenta basas de plata debajo de las
veinte tablas: dos basas debajo de la una tabla para sus dos quicios, y
dos basas debajo de la otra tabla para sus dos quicios. \bibverse{25} Y
para el otro lado del tabernáculo, á la parte del aquilón, hizo veinte
tablas, \bibverse{26} Con sus cuarenta basas de plata: dos basas debajo
de la una tabla, y dos basas debajo de la otra tabla. \bibverse{27} Y
para el lado occidental del tabernáculo hizo seis tablas. \bibverse{28}
Para las esquinas del tabernáculo en los dos lados hizo dos tablas,
\bibverse{29} Las cuales se juntaban por abajo, y asimismo por arriba á
un gozne: y así hizo á la una y á la otra en las dos esquinas.
\bibverse{30} Eran, pues, ocho tablas, y sus basas de plata dieciséis;
dos basas debajo de cada tabla. \bibverse{31} Hizo también las barras de
madera de Sittim; cinco para las tablas del un lado del tabernáculo,
\bibverse{32} Y cinco barras para las tablas del otro lado del
tabernáculo, y cinco barras para las tablas del lado del tabernáculo á
la parte occidental. \bibverse{33} E hizo que la barra del medio pasase
por medio de las tablas del un cabo al otro. \bibverse{34} Y cubrió las
tablas de oro, é hizo de oro los anillos de ellas por donde pasasen las
barras: cubrió también de oro las barras. \bibverse{35} Hizo asimismo el
velo de cárdeno, y púrpura, y carmesí, y lino torcido, el cual hizo con
querubines de delicada obra. \bibverse{36} Y para él hizo cuatro
columnas de madera de Sittim; y cubriólas de oro, los capiteles de las
cuales eran de oro; é hizo para ellas cuatro basas de plata de
fundición. \bibverse{37} Hizo también el velo para la puerta del
tabernáculo, de jacinto, y púrpura, y carmesí, y lino torcido, obra de
recamador; \bibverse{38} Y sus cinco columnas con sus capiteles: y
cubrió las cabezas de ellas y sus molduras de oro: pero sus cinco basas
las hizo de metal.

\hypertarget{section-36}{%
\section{37}\label{section-36}}

\bibverse{1} Hizo también Bezaleel el arca de madera de Sittim: su
longitud era de dos codos y medio, y de codo y medio su anchura, y su
altura de otro codo y medio: \bibverse{2} Y cubrióla de oro puro por de
dentro y por de fuera, é hízole una cornisa de oro en derredor.
\bibverse{3} Hízole además de fundición cuatro anillos de oro á sus
cuatro esquinas; en el un lado dos anillos y en el otro lado dos
anillos. \bibverse{4} Hizo también las varas de madera de Sittim, y
cubriólas de oro. \bibverse{5} Y metió las varas por los anillos á los
lados del arca, para llevar el arca. \bibverse{6} Hizo asimismo la
cubierta de oro puro: su longitud de dos codos y medio, y su anchura de
codo y medio. \bibverse{7} Hizo también los dos querubines de oro,
hízolos labrados á martillo, á los dos cabos de la cubierta:
\bibverse{8} El un querubín de esta parte al un cabo, y el otro querubín
de la otra parte al otro cabo de la cubierta: hizo los querubines á sus
dos cabos. \bibverse{9} Y los querubines extendían sus alas por encima,
cubriendo con sus alas la cubierta: y sus rostros el uno enfrente del
otro, hacia la cubierta los rostros de los querubines. \bibverse{10}
Hizo también la mesa de madera de Sittim; su longitud de dos codos, y su
anchura de un codo, y de codo y medio su altura; \bibverse{11} Y
cubrióla de oro puro, é hízole una cornisa de oro en derredor.
\bibverse{12} Hízole también una moldura alrededor, del ancho de una
mano, á la cual moldura hizo la cornisa de oro en circunferencia.
\bibverse{13} Hízole asimismo de fundición cuatro anillos de oro, y
púsolos á las cuatro esquinas que correspondían á los cuatro pies de
ella. \bibverse{14} Delante de la moldura estaban los anillos, por los
cuales se metiesen las varas para llevar la mesa. \bibverse{15} E hizo
las varas de madera de Sittim para llevar la mesa, y cubriólas de oro.
\bibverse{16} También hizo los vasos que habían de estar sobre la mesa,
sus platos, y sus cucharas, y sus cubiertos y sus tazones con que se
había de libar, de oro fino. \bibverse{17} Hizo asimismo el candelero de
oro puro, é hízolo labrado á martillo: su pie y su caña, sus copas, sus
manzanas y sus flores eran de lo mismo. \bibverse{18} De sus lados
salían seis brazos; tres brazos del un lado del candelero, y otros tres
brazos del otro lado del candelero: \bibverse{19} En el un brazo, tres
copas figura de almendras, una manzana y una flor; y en el otro brazo
tres copas figura de almendras, una manzana y una flor: y así en los
seis brazos que salían del candelero. \bibverse{20} Y en el candelero
había cuatro copas figura de almendras, sus manzanas y sus flores:
\bibverse{21} Y una manzana debajo de los dos brazos de lo mismo, y otra
manzana debajo de los otros dos brazos de lo mismo, y otra manzana
debajo de los otros dos brazos de lo mismo, conforme á los seis brazos
que salían de él. \bibverse{22} Sus manzanas y sus brazos eran de lo
mismo; todo era una pieza labrada á martillo, de oro puro. \bibverse{23}
Hizo asimismo sus siete candilejas, y sus despabiladeras, y sus
platillos, de oro puro; \bibverse{24} De un talento de oro puro lo hizo,
con todos sus vasos. \bibverse{25} Hizo también el altar del perfume de
madera de Sittim: un codo su longitud, y otro codo su anchura, era
cuadrado; y su altura de dos codos; y sus cuernos de la misma pieza.
\bibverse{26} Y cubriólo de oro puro, su mesa y sus paredes alrededor, y
sus cuernos: é hízole una corona de oro alrededor. \bibverse{27} Hízole
también dos anillos de oro debajo de la corona en las dos esquinas á los
dos lados, para pasar por ellos las varas con que había de ser
conducido. \bibverse{28} E hizo las varas de madera de Sittim, y
cubriólas de oro. \bibverse{29} Hizo asimismo el aceite santo de la
unción, y el fino perfume aromático, de obra de perfumador.

\hypertarget{section-37}{%
\section{38}\label{section-37}}

\bibverse{1} Igualmente hizo el altar del holocausto de madera de
Sittim: su longitud de cinco codos, y su anchura de otros cinco codos,
cuadrado, y de tres codos de altura. \bibverse{2} E hízole sus cuernos á
sus cuatro esquinas, los cuales eran de la misma pieza, y cubriólo de
metal. \bibverse{3} Hizo asimismo todos los vasos del altar: calderas, y
tenazas, y tazones, y garfios, y palas: todos sus vasos hizo de metal.
\bibverse{4} E hizo para el altar el enrejado de metal, de hechura de
red, que puso en su cerco por debajo hasta el medio del altar.
\bibverse{5} Hizo también cuatro anillos de fundición á los cuatro cabos
del enrejado de metal, para meter las varas. \bibverse{6} E hizo las
varas de madera de Sittim, y cubriólas de metal. \bibverse{7} Y metió
las varas por los anillos á los lados del altar, para llevarlo con
ellas: hueco lo hizo, de tablas. \bibverse{8} También hizo la fuente de
metal, con su basa de metal, de los espejos de las que velaban á la
puerta del tabernáculo del testimonio. \bibverse{9} Hizo asimismo el
atrio; á la parte austral del mediodía las cortinas del atrio eran de
cien codos, de lino torcido: \bibverse{10} Sus columnas veinte, con sus
veinte basas de metal: los capiteles de las columnas y sus molduras, de
plata. \bibverse{11} Y á la parte del aquilón cortinas de cien codos:
sus columnas veinte, con sus veinte basas de metal; los capiteles de las
columnas y sus molduras, de plata. \bibverse{12} A la parte del
occidente cortinas de cincuenta codos: sus columnas diez, y sus diez
basas; los capiteles de las columnas y sus molduras, de plata.
\bibverse{13} Y á la parte oriental, al levante, cortinas de cincuenta
codos: \bibverse{14} Al un lado cortinas de quince codos, sus tres
columnas, y sus tres basas; \bibverse{15} Al otro lado, de la una parte
y de la otra de la puerta del atrio, cortinas de á quince codos, sus
tres columnas, y sus tres basas. \bibverse{16} Todas las cortinas del
atrio alrededor eran de lino torcido. \bibverse{17} Y las basas de las
columnas eran de metal; los capiteles de las columnas y sus molduras, de
plata; asimismo las cubiertas de las cabezas de ellas, de plata: y todas
las columnas del atrio tenían molduras de plata. \bibverse{18} Y el
pabellón de la puerta del atrio fué de obra de recamado, de jacinto, y
púrpura, y carmesí, y lino torcido: la longitud de veinte codos, y la
altura en el ancho de cinco codos, conforme á las cortinas del atrio.
\bibverse{19} Y sus columnas fueron cuatro con sus cuatro basas de
metal: y sus capiteles de plata; y las cubiertas de los capiteles de
ellas y sus molduras, de plata. \bibverse{20} Y todas las estacas del
tabernáculo y del atrio alrededor fueron de metal. \bibverse{21} Estas
son las cuentas del tabernáculo, del tabernáculo del testimonio, lo que
fué contado de orden de Moisés por mano de Ithamar, hijo de Aarón
sacerdote, para el ministerio de los Levitas. \bibverse{22} Y Bezaleel,
hijo de Uri, hijo de Hur, de la tribu de Judá, hizo todas las cosas que
Jehová mandó á Moisés. \bibverse{23} Y con él estaba Aholiab, hijo de
Ahisamac, de la tribu de Dan, artífice, y diseñador, y recamador en
jacinto, y púrpura, y carmesí, y lino fino. \bibverse{24} Todo el oro
empleado en la obra, en toda la obra del santuario, el cual fué oro de
ofrenda, fué veintinueve talentos, y setecientos y treinta siclos, según
el siclo del santuario. \bibverse{25} Y la plata de los contados de la
congregación fué cien talentos, y mil setecientos setenta y cinco
siclos, según el siclo del santuario: \bibverse{26} Medio por cabeza,
medio siclo, según el siclo del santuario, á todos los que pasaron por
cuenta de edad de veinte años y arriba, que fueron seiscientos tres mil
quinientos cincuenta. \bibverse{27} Hubo además cien talentos de plata
para hacer de fundición las basas del santuario y las basas del velo: en
cien basas cien talentos, á talento por basa. \bibverse{28} Y de los mil
setecientos setenta y cinco siclos hizo los capiteles de las columnas, y
cubrió los capiteles de ellas, y las ciñó. \bibverse{29} Y el metal de
la ofrenda fué setenta talentos, y dos mil cuatrocientos siclos;
\bibverse{30} Del cual hizo las basas de la puerta del tabernáculo del
testimonio, y el altar de metal, y su enrejado de metal, y todos los
vasos del altar. \bibverse{31} Y las basas del atrio alrededor, y las
basas de la puerta del atrio, y todas las estacas del tabernáculo, y
todas las estacas del atrio alrededor.

\hypertarget{section-38}{%
\section{39}\label{section-38}}

\bibverse{1} Y del jacinto, y púrpura, y carmesí, hicieron las
vestimentas del ministerio para ministrar en el santuario, y asimismo
hicieron las vestiduras sagradas para Aarón; como Jehová lo había
mandado á Moisés. \bibverse{2} Hizo también el ephod de oro, de cárdeno
y púrpura y carmesí, y lino torcido. \bibverse{3} Y extendieron las
planchas de oro, y cortaron hilos para tejerlos entre el jacinto, y
entre la púrpura, y entre el carmesí, y entre el lino, con delicada
obra. \bibverse{4} Hiciéronle las hombreras que se juntasen; y uníanse
en sus dos lados. \bibverse{5} Y el cinto del ephod que estaba sobre él,
era de lo mismo, conforme á su obra; de oro, jacinto, y púrpura, y
carmesí, y lino torcido; como Jehová lo había mandado á Moisés.
\bibverse{6} Y labraron las piedras oniquinas cercadas de engastes de
oro, grabadas de grabadura de sello con los nombres de los hijos de
Israel: \bibverse{7} Y púsolas sobre las hombreras del ephod, por
piedras de memoria á los hijos de Israel; como Jehová lo había á Moisés
mandado. \bibverse{8} Hizo también el racional de primorosa obra, como
la obra del ephod, de oro, jacinto, y púrpura, y carmesí, y lino
torcido. \bibverse{9} Era cuadrado: doblado hicieron el racional: su
longitud era de un palmo, y de un palmo su anchura, doblado.
\bibverse{10} Y engastaron en él cuatro órdenes de piedras. El primer
orden era un sardio, un topacio, y un carbunclo: este el primer orden.
\bibverse{11} El segundo orden, una esmeralda, un zafiro, y un diamante.
\bibverse{12} El tercer orden, un ligurio, un ágata, y un amatista.
\bibverse{13} Y el cuarto orden, un berilo, un onix, y un jaspe:
cercadas y encajadas en sus engastes de oro. \bibverse{14} Las cuales
piedras eran conforme á los nombres de los hijos de Israel, doce según
los nombres de ellos; como grabaduras de sello, cada una con su nombre
según las doce tribus. \bibverse{15} Hicieron también sobre el racional
las cadenas pequeñas de hechura de trenza, de oro puro. \bibverse{16}
Hicieron asimismo los dos engastes y los dos anillos, de oro; los cuales
dos anillos de oro pusieron en los dos cabos del racional. \bibverse{17}
Y pusieron las dos trenzas de oro en aquellos dos anillos á los cabos
del racional. \bibverse{18} Y fijaron los dos cabos de las dos trenzas
en los dos engastes, que pusieron sobre las hombreras del ephod, en la
parte delantera de él. \bibverse{19} E hicieron dos anillos de oro, que
pusieron en los dos cabos del racional en su orilla, á la parte baja del
ephod. \bibverse{20} Hicieron además dos anillos de oro, los cuales
pusieron en las dos hombreras del ephod, abajo en la parte delantera,
delante de su juntura, sobre el cinto del ephod. \bibverse{21} Y ataron
el racional de sus anillos á los anillos del ephod con un cordón de
jacinto, para que estuviese sobre el cinto del mismo ephod, y no se
apartase el racional del ephod; como Jehová lo había mandado á Moisés.
\bibverse{22} Hizo también el manto del ephod de obra de tejedor, todo
de jacinto, \bibverse{23} Con su abertura en medio de él, como el cuello
de un coselete, con un borde en derredor de la abertura, porque no se
rompiese. \bibverse{24} E hicieron en las orillas del manto las granadas
de jacinto, y púrpura, y carmesí, y lino torcido. \bibverse{25} Hicieron
también las campanillas de oro puro, las cuales campanillas pusieron
entre las granadas por las orillas del manto alrededor entre las
granadas: \bibverse{26} Una campanilla y una granada, una campanilla y
una granada alrededor, en las orillas del manto, para ministrar; como
Jehová lo mandó á Moisés. \bibverse{27} Igualmente hicieron las túnicas
de lino fino de obra de tejedor, para Aarón y para sus hijos;
\bibverse{28} Asimismo la mitra de lino fino, y los adornos de los
chapeos (tiaras) de lino fino, y los pañetes de lino, de lino torcido;
\bibverse{29} También el cinto de lino torcido, y de jacinto, y púrpura,
y carmesí, de obra de recamador; como Jehová lo mandó á Moisés.
\bibverse{30} Hicieron asimismo la plancha de la diadema santa de oro
puro, y escribieron en ella de grabadura de sello, el rótulo, SANTIDAD Á
JEHOVÁ. \bibverse{31} Y pusieron en ella un cordón de jacinto, para
colocarla en alto sobre la mitra; como Jehová lo había mandado á Moisés.
\bibverse{32} Y fué acabada toda la obra del tabernáculo, del
tabernáculo del testimonio: é hicieron los hijos de Israel como Jehová
lo había mandado á Moisés: así lo hicieron. \bibverse{33} Y trajeron el
tabernáculo á Moisés, el tabernáculo y todos sus vasos; sus corchetes,
sus tablas, sus barras, y sus columnas, y sus basas; \bibverse{34} Y la
cubierta de pieles rojas de carneros, y la cubierta de pieles de
tejones, y el velo del pabellón; \bibverse{35} El arca del testimonio, y
sus varas, y la cubierta; \bibverse{36} La mesa, todos sus vasos, y el
pan de la proposición; \bibverse{37} El candelero limpio, sus
candilejas, las lámparas que debían mantenerse en orden, y todos sus
vasos, y el aceite para la luminaria; \bibverse{38} Y el altar de oro, y
el aceite de la unción, y el perfume aromático, y el pabellón para la
puerta del tabernáculo; \bibverse{39} El altar de metal, con su enrejado
de metal, sus varas, y todos sus vasos; y la fuente, y su basa;
\bibverse{40} Las cortinas del atrio, y sus columnas, y sus basas, y el
pabellón para la puerta del atrio, y sus cuerdas, y sus estacas, y todos
los vasos del servicio del tabernáculo, del tabernáculo del testimonio;
\bibverse{41} Las vestimentas del servicio para ministrar en el
santuario, las sagradas vestiduras para Aarón el sacerdote, y las
vestiduras de sus hijos, para ministrar en el sacerdocio. \bibverse{42}
En conformidad á todas las cosas que Jehová había mandado á Moisés, así
hicieron los hijos de Israel toda la obra. \bibverse{43} Y vió Moisés
toda la obra, y he aquí que la habían hecho como Jehová había mandado; y
bendíjolos.

\hypertarget{section-39}{%
\section{40}\label{section-39}}

\bibverse{1} Y jehová habló á Moisés, diciendo: \bibverse{2} En el
primer día del mes primero harás levantar el tabernáculo, el tabernáculo
del testimonio: \bibverse{3} Y pondrás en él el arca del testimonio, y
la cubrirás con el velo: \bibverse{4} Y meterás la mesa, y la pondrás en
orden: meterás también el candelero, y encenderás sus lámparas:
\bibverse{5} Y pondrás el altar de oro para el perfume delante del arca
del testimonio, y pondrás el pabellón delante de la puerta del
tabernáculo. \bibverse{6} Después pondrás el altar del holocausto
delante de la puerta del tabernáculo, del tabernáculo del testimonio.
\bibverse{7} Luego pondrás la fuente entre el tabernáculo del testimonio
y el altar; y pondrás agua en ella. \bibverse{8} Finalmente pondrás el
atrio en derredor, y el pabellón de la puerta del atrio. \bibverse{9} Y
tomarás el aceite de la unción, y ungirás el tabernáculo, y todo lo que
está en él; y le santificarás con todos sus vasos, y será santo.
\bibverse{10} Ungirás también el altar del holocausto y todos sus vasos:
y santificarás el altar, y será un altar santísimo. \bibverse{11}
Asimismo ungirás la fuente y su basa, y la santificarás. \bibverse{12} Y
harás llegar á Aarón y á sus hijos á la puerta del tabernáculo del
testimonio, y los lavarás con agua. \bibverse{13} Y harás vestir á Aarón
las vestiduras sagradas, y lo ungirás, y lo consagrarás, para que sea mi
sacerdote. \bibverse{14} Después harás llegar sus hijos, y les vestirás
las túnicas: \bibverse{15} Y los ungirás como ungiste á su padre, y
serán mis sacerdotes: y será que su unción les servirá por sacerdocio
perpetuo por sus generaciones. \bibverse{16} Y Moisés hizo conforme á
todo lo que Jehová le mandó; así lo hizo. \bibverse{17} Y así en el día
primero del primer mes, en el segundo año, el tabernáculo fué erigido.
\bibverse{18} Y Moisés hizo levantar el tabernáculo, y asentó sus basas,
y colocó sus tablas, y puso sus barras, é hizo alzar sus columnas.
\bibverse{19} Y extendió la tienda sobre el tabernáculo, y puso la
sobrecubierta encima del mismo; como Jehová había mandado á Moisés.
\bibverse{20} Y tomó y puso el testimonio dentro del arca, y colocó las
varas en el arca, y encima la cubierta sobre el arca: \bibverse{21} Y
metió el arca en el tabernáculo, y puso el velo de la tienda, y cubrió
el arca del testimonio; como Jehová había mandado á Moisés.
\bibverse{22} Y puso la mesa en el tabernáculo del testimonio, al lado
septentrional del pabellón, fuera del velo: \bibverse{23} Y sobre ella
puso por orden los panes delante de Jehová, como Jehová había mandado á
Moisés. \bibverse{24} Y puso el candelero en el tabernáculo del
testimonio, enfrente de la mesa, al lado meridional del pabellón.
\bibverse{25} Y encendió las lámparas delante de Jehová; como Jehová
había mandado á Moisés. \bibverse{26} Puso también el altar de oro en el
tabernáculo del testimonio, delante del velo: \bibverse{27} Y encendió
sobre él el perfume aromático; como Jehová había mandado á Moisés.
\bibverse{28} Puso asimismo la cortina de la puerta del tabernáculo.
\bibverse{29} Y colocó el altar del holocausto á la puerta del
tabernáculo, del tabernáculo del testimonio; y ofreció sobre él
holocausto y presente; como Jehová había mandado á Moisés. \bibverse{30}
Y puso la fuente entre el tabernáculo del testimonio y el altar; y puso
en ella agua para lavar. \bibverse{31} Y Moisés y Aarón y sus hijos
lavaban en ella sus manos y sus pies. \bibverse{32} Cuando entraban en
el tabernáculo del testimonio, y cuando se llegaban al altar, se
lavaban; como Jehová había mandado á Moisés. \bibverse{33} Finalmente
erigió el atrio en derredor del tabernáculo y del altar, y puso la
cortina de la puerta del atrio. Y así acabó Moisés la obra.
\bibverse{34} Entonces una nube cubrió el tabernáculo del testimonio, y
la gloria de Jehová hinchió el tabernáculo. \bibverse{35} Y no podía
Moisés entrar en el tabernáculo del testimonio, porque la nube estaba
sobre él, y la gloria de Jehová lo tenía lleno. \bibverse{36} Y cuando
la nube se alzaba del tabernáculo, los hijos de Israel se movían en
todas sus jornadas: \bibverse{37} Pero si la nube no se alzaba, no se
partían hasta el día en que ella se alzaba. \bibverse{38} Porque la nube
de Jehová estaba de día sobre el tabernáculo, y el fuego estaba de noche
en él, á vista de toda la casa de Israel, en todas sus jornadas.
