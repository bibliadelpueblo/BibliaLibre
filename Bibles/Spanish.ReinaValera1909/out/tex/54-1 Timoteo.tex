\hypertarget{bendiciones}{%
\subsection{Bendiciones}\label{bendiciones}}

\hypertarget{section}{%
\section{1}\label{section}}

\bibverse{1} Pablo, apóstol de Jesucristo por la ordenación de Dios
nuestro Salvador, y del Señor Jesucristo, nuestra esperanza; \footnote{\textbf{1:1}
  Col 1,27} \bibverse{2} A Timoteo, verdadero hijo en la fe: Gracia,
misericordia y paz de Dios nuestro Padre, y de Cristo Jesús nuestro
Señor. \footnote{\textbf{1:2} Hech 16,1-2; Tit 1,4}

\hypertarget{amonestaciuxf3n-para-luchar-por-el-mensaje-saludable-de-salvaciuxf3n-contra-los-falsos-maestros}{%
\subsection{Amonestación para luchar por el mensaje saludable de
salvación contra los falsos
maestros}\label{amonestaciuxf3n-para-luchar-por-el-mensaje-saludable-de-salvaciuxf3n-contra-los-falsos-maestros}}

\bibverse{3} Como te rogué que te quedases en Efeso, cuando partí para
Macedonia, para que requirieses á algunos que no enseñen diversa
doctrina, \footnote{\textbf{1:3} Hech 20,1} \bibverse{4} Ni presten
atención á fábulas y genealogías sin término, que antes engendran
cuestiones que la edificación de Dios que es por fe; así te encargo
ahora. \footnote{\textbf{1:4} 1Tim 4,7} \bibverse{5} Pues el fin del
mandamiento es la caridad nacida de corazón limpio, y de buena
conciencia, y de fe no fingida: \footnote{\textbf{1:5} Mat 22,37-40; Rom
  13,10; Gal 5,6} \bibverse{6} De lo cual distrayéndose algunos, se
apartaron á vanas pláticas; \footnote{\textbf{1:6} 1Tim 6,4; 1Tim 6,20}
\bibverse{7} Queriendo ser doctores de la ley, sin entender ni lo que
hablan, ni lo que afirman.

\hypertarget{la-posiciuxf3n-del-cristiano-sobre-la-ley}{%
\subsection{La posición del cristiano sobre la
ley}\label{la-posiciuxf3n-del-cristiano-sobre-la-ley}}

\bibverse{8} Sabemos empero que la ley es buena, si alguno usa de ella
legítimamente; \footnote{\textbf{1:8} Rom 7,12} \bibverse{9} Conociendo
esto, que la ley no es puesta para el justo, sino para los injustos y
para los desobedientes, para los impíos y pecadores, para los malos y
profanos, para los parricidas y matricidas, para los homicidas,
\footnote{\textbf{1:9} 1Cor 6,9-11} \bibverse{10} Para los fornicarios,
para los sodomitas, para los ladrones de hombres, para los mentirosos y
perjuros, y si hay alguna otra cosa contraria á la sana doctrina;
\bibverse{11} Según el evangelio de la gloria del Dios bendito, el cual
á mí me ha sido encargado.

\hypertarget{la-experiencia-de-la-gracia-del-apuxf3stol-y-su-llamado-a-testificar-de-la-verdad-cristiana-de-la-salvaciuxf3n-alabado-sea-la-gracia-que-jesuxfas-le-dio}{%
\subsection{La experiencia de la gracia del apóstol y su llamado a
testificar de la verdad cristiana de la salvación; Alabado sea la gracia
que Jesús le
dio}\label{la-experiencia-de-la-gracia-del-apuxf3stol-y-su-llamado-a-testificar-de-la-verdad-cristiana-de-la-salvaciuxf3n-alabado-sea-la-gracia-que-jesuxfas-le-dio}}

\bibverse{12} Y doy gracias al que me fortificó, á Cristo Jesús nuestro
Señor, de que me tuvo por fiel, poniéndome en el ministerio: \footnote{\textbf{1:12}
  Hech 9,15; 1Cor 15,9-10; Gal 1,13-16} \bibverse{13} Habiendo sido
antes blasfemo y perseguidor é injuriador: mas fuí recibido á
misericordia, porque lo hice con ignorancia en incredulidad.
\bibverse{14} Mas la gracia de nuestro Señor fué más abundante con la fe
y amor que es en Cristo Jesús. \bibverse{15} Palabra fiel y digna de ser
recibida de todos: que Cristo Jesús vino al mundo para salvar á los
pecadores, de los cuales yo soy el primero. \bibverse{16} Mas por esto
fuí recibido á misericordia, para que Jesucristo mostrase en mí el
primero toda su clemencia, para ejemplo de los que habían de creer en él
para vida eterna. \bibverse{17} Por tanto, al Rey de siglos, inmortal,
invisible, al solo sabio Dios sea honor y gloria por los siglos de los
siglos. Amén.

\hypertarget{exhortaciuxf3n-a-timoteo-a-luchar-por-la-verdad-cristiana-contra-la-herejuxeda}{%
\subsection{Exhortación a Timoteo a luchar por la verdad cristiana
contra la
herejía}\label{exhortaciuxf3n-a-timoteo-a-luchar-por-la-verdad-cristiana-contra-la-herejuxeda}}

\bibverse{18} Este mandamiento, hijo Timoteo, te encargo, para que,
conforme á las profecías pasadas de ti, milites por ellas buena milicia;
\footnote{\textbf{1:18} 1Tim 4,14; 1Tim 6,12; Jds 1,3} \bibverse{19}
Manteniendo la fe y buena conciencia, la cual echando de sí algunos,
hicieron naufragio en la fe: \footnote{\textbf{1:19} 1Tim 3,9; 1Tim 6,10}
\bibverse{20} De los cuales son Himeneo y Alejandro, los cuales entregué
á Satanás, para que aprendan á no blasfemar. \footnote{\textbf{1:20}
  1Cor 5,5; 2Tim 2,17}

\hypertarget{regulaciones-sobre-la-oraciuxf3n-en-congregaciuxf3n-para-todas-las-personas-especialmente-para-las-autoridades}{%
\subsection{Regulaciones sobre la oración en congregación para todas las
personas, especialmente para las
autoridades}\label{regulaciones-sobre-la-oraciuxf3n-en-congregaciuxf3n-para-todas-las-personas-especialmente-para-las-autoridades}}

\hypertarget{section-1}{%
\section{2}\label{section-1}}

\bibverse{1} Amonesto pues, ante todas cosas, que se hagan rogativas,
oraciones, peticiones, hacimientos de gracias, por todos los hombres;
\bibverse{2} Por los reyes y por todos los que están en eminencia, para
que vivamos quieta y reposadamente en toda piedad y honestidad.
\bibverse{3} Porque esto es bueno y agradable delante de Dios nuestro
Salvador; \bibverse{4} El cual quiere que todos los hombres sean salvos,
y que vengan al conocimiento de la verdad. \bibverse{5} Porque hay un
Dios, asimismo un mediador entre Dios y los hombres, Jesucristo hombre;
\footnote{\textbf{2:5} Heb 9,15} \bibverse{6} El cual se dió á sí mismo
en precio del rescate por todos, para testimonio en sus tiempos:
\footnote{\textbf{2:6} Gal 1,4; Gal 2,20; Tit 2,14} \bibverse{7} De lo
que yo soy puesto por predicador y apóstol, (digo verdad en Cristo, no
miento) doctor de los Gentiles en fidelidad y verdad. \footnote{\textbf{2:7}
  2Tim 1,11; Gal 2,7-8}

\hypertarget{reglas-para-la-conducta-de-hombres-y-mujeres-en-el-culto-de-la-iglesia-cristiana}{%
\subsection{Reglas para la conducta de hombres y mujeres en el culto de
la iglesia
cristiana}\label{reglas-para-la-conducta-de-hombres-y-mujeres-en-el-culto-de-la-iglesia-cristiana}}

\bibverse{8} Quiero, pues, que los hombres oren en todo lugar,
levantando manos limpias, sin ira ni contienda. \footnote{\textbf{2:8}
  Sant 1,6} \bibverse{9} Asimismo también las mujeres, ataviándose en
hábito honesto, con vergüenza y modestia; no con cabellos encrespados, ú
oro, ó perlas, ó vestidos costosos, \footnote{\textbf{2:9} 1Pe 3,3-5}
\bibverse{10} Sino de buenas obras, como conviene á mujeres que profesan
piedad. \footnote{\textbf{2:10} 1Tim 5,10} \bibverse{11} La mujer
aprenda en silencio, con toda sujeción. \footnote{\textbf{2:11} Efes
  5,22} \bibverse{12} Porque no permito á la mujer enseñar, ni tomar
autoridad sobre el hombre, sino estar en silencio. \footnote{\textbf{2:12}
  1Cor 14,34; Gén 3,16} \bibverse{13} Porque Adam fué formado el
primero, después Eva; \bibverse{14} Y Adam no fué engañado, sino la
mujer, siendo seducida, vino á ser envuelta en transgresión: \footnote{\textbf{2:14}
  Gén 3,6} \bibverse{15} Empero se salvará engendrando hijos, si
permaneciere en la fe y caridad y santidad, con modestia. \footnote{\textbf{2:15}
  1Tim 5,14; Tit 2,4; Tit 1,2-5}

\hypertarget{requisitos-para-el-cargo-de-jefe}{%
\subsection{Requisitos para el cargo de
jefe}\label{requisitos-para-el-cargo-de-jefe}}

\hypertarget{section-2}{%
\section{3}\label{section-2}}

\bibverse{1} Palabra fiel: Si alguno apetece obispado, buena obra desea.
\footnote{\textbf{3:1} Hech 20,28; Fil 1,1; Tit 1,5-9} \bibverse{2}
Conviene, pues, que el obispo sea irreprensible, marido de una mujer,
solícito, templado, compuesto, hospedador, apto para enseñar;
\bibverse{3} No amador del vino, no heridor, no codicioso de torpes
ganancias, sino moderado, no litigioso, ajeno de avaricia; \bibverse{4}
Que gobierne bien su casa, que tenga sus hijos en sujeción con toda
honestidad; \bibverse{5} (Porque el que no sabe gobernar su casa, ¿cómo
cuidará de la iglesia de Dios?) \bibverse{6} No un neófito, porque
inflándose no caiga en juicio del diablo. \bibverse{7} También conviene
que tenga buen testimonio de los extraños, porque no caiga en afrenta y
en lazo del diablo.

\hypertarget{requisitos-de-ayudantuxeda}{%
\subsection{Requisitos de ayudantía}\label{requisitos-de-ayudantuxeda}}

\bibverse{8} Los diáconos asimismo, deben ser honestos, no bilingües, no
dados á mucho vino, no amadores de torpes ganancias; \footnote{\textbf{3:8}
  Hech 6,3; Fil 1,1} \bibverse{9} Que tengan el misterio de la fe con
limpia conciencia. \footnote{\textbf{3:9} 1Tim 1,19} \bibverse{10} Y
éstos también sean antes probados; y así ministren, si fueren sin
crimen. \bibverse{11} Las mujeres asimismo, honestas, no detractoras,
templadas, fieles en todo. \footnote{\textbf{3:11} Tit 2,3}
\bibverse{12} Los diáconos sean maridos de una mujer, que gobiernen bien
sus hijos y sus casas. \bibverse{13} Porque los que bien ministraren,
ganan para sí buen grado, y mucha confianza en la fe que es en Cristo
Jesús.

\hypertarget{conclusiuxf3n-de-las-instrucciones-anteriores-refiriuxe9ndose-a-la-comunidad-como-portadora-de-la-verdad-de-la-salvaciuxf3n}{%
\subsection{Conclusión de las instrucciones anteriores refiriéndose a la
comunidad como portadora de la verdad de la
salvación}\label{conclusiuxf3n-de-las-instrucciones-anteriores-refiriuxe9ndose-a-la-comunidad-como-portadora-de-la-verdad-de-la-salvaciuxf3n}}

\bibverse{14} Esto te escribo con esperanza que iré presto á ti:
\bibverse{15} Y si no fuere tan presto, para que sepas cómo te conviene
conversar en la casa de Dios, que es la iglesia del Dios vivo, columna y
apoyo de la verdad. \bibverse{16} Y sin contradicción, grande es el
misterio de la piedad: Dios ha sido manifestado en carne; ha sido
justificado con el Espíritu; ha sido visto de los ángeles; ha sido
predicado á los Gentiles; ha sido creído en el mundo; ha sido recibido
en gloria. \footnote{\textbf{3:16} Juan 1,14; Rom 1,4; Efes 1,20-21;
  Hech 28,28; Mar 16,19}

\hypertarget{advertencia-de-la-abstinencia-hipuxf3crita-de-los-falsos-maestros}{%
\subsection{Advertencia de la abstinencia hipócrita de los falsos
maestros}\label{advertencia-de-la-abstinencia-hipuxf3crita-de-los-falsos-maestros}}

\hypertarget{section-3}{%
\section{4}\label{section-3}}

\bibverse{1} Empero el Espíritu dice manifiestamente, que en los
venideros tiempos algunos apostatarán de la fe, escuchando á espíritus
de error y á doctrinas de demonios; \footnote{\textbf{4:1} Mat 24,24;
  2Tes 2,3; 2Tim 3,1; 2Pe 3,3; 1Jn 2,18; Jds 1,18} \bibverse{2} Que con
hipocresía hablarán mentira, teniendo cauterizada la conciencia.
\bibverse{3} Que prohibirán casarse, y mandarán abstenerse de las
viandas que Dios crió para que con hacimiento de gracias participasen de
ellas los fieles, y los que han conocido la verdad. \footnote{\textbf{4:3}
  Gén 9,3; 1Cor 10,30-31; Col 2,23} \bibverse{4} Porque todo lo que Dios
crió es bueno, y nada hay que desechar, tomándose con hacimiento de
gracias: \footnote{\textbf{4:4} Gén 1,31; Mat 15,11; Hech 10,15}
\bibverse{5} Porque por la palabra de Dios y por la oración es
santificado.

\hypertarget{el-correcto-ejercicio-cristiano-de-la-piedad-y-la-bendiciuxf3n-que-se-le-prometiuxf3}{%
\subsection{El correcto ejercicio cristiano de la piedad y la bendición
que se le
prometió}\label{el-correcto-ejercicio-cristiano-de-la-piedad-y-la-bendiciuxf3n-que-se-le-prometiuxf3}}

\bibverse{6} Si esto propusieres á los hermanos, serás buen ministro de
Jesucristo, criado en las palabras de la fe y de la buena doctrina, la
cual has alcanzado. \footnote{\textbf{4:6} 2Tim 2,15} \bibverse{7} Mas
las fábulas profanas y de viejas desecha, y ejercítate para la piedad.
\footnote{\textbf{4:7} 1Tim 6,20; 2Tim 2,16; 2Tim 2,23; 2Tim 4,4; Tit
  1,14; Tit 3,9} \bibverse{8} Porque el ejercicio corporal para poco es
provechoso; mas la piedad para todo aprovecha, pues tiene promesa de
esta vida presente, y de la venidera. \footnote{\textbf{4:8} 1Tim 6,6;
  Heb 13,9} \bibverse{9} Palabra fiel es esta, y digna de ser recibida
de todos. \bibverse{10} Que por esto aun trabajamos y sufrimos oprobios,
porque esperamos en el Dios viviente, el cual es Salvador de todos los
hombres, mayormente de los que creen. \bibverse{11} Esto manda y enseña.

\hypertarget{reglas-generales-para-timoteo-especialmente-con-respecto-a-su-juventud}{%
\subsection{Reglas generales para Timoteo, especialmente con respecto a
su
juventud}\label{reglas-generales-para-timoteo-especialmente-con-respecto-a-su-juventud}}

\bibverse{12} Ninguno tenga en poco tu juventud; pero sé ejemplo de los
fieles en palabra, en conversación, en caridad, en espíritu, en fe, en
limpieza. \bibverse{13} Entre tanto que voy, ocúpate en leer, en
exhortar, en enseñar. \bibverse{14} No descuides el don que está en ti,
que te es dado por profecía con la imposición de las manos del
presbiterio. \footnote{\textbf{4:14} 1Tim 1,18; 1Tim 5,22; Hech 6,6;
  Hech 8,17; 2Tim 1,6}

\bibverse{15} Medita estas cosas; ocúpate en ellas; para que tu
aprovechamiento sea manifiesto á todos. \bibverse{16} Ten cuidado de ti
mismo y de la doctrina; persiste en ello; pues haciendo esto, á ti mismo
salvarás y á los que te oyeren.

\hypertarget{del-correcto-comportamiento-pastoral-hacia-las-diferentes-edades-de-ambos-sexos}{%
\subsection{Del correcto comportamiento pastoral hacia las diferentes
edades de ambos
sexos}\label{del-correcto-comportamiento-pastoral-hacia-las-diferentes-edades-de-ambos-sexos}}

\hypertarget{section-4}{%
\section{5}\label{section-4}}

\bibverse{1} No reprendas al anciano, sino exhórtale como á padre: á los
más jóvenes, como á hermanos; \footnote{\textbf{5:1} Lev 19,32; Tit 2,2}
\bibverse{2} A las ancianas, como á madres; á las jovencitas, como á
hermanas, con toda pureza.

\hypertarget{normas-relativas-a-las-viudas-y-su-cuidado}{%
\subsection{Normas relativas a las viudas y su
cuidado}\label{normas-relativas-a-las-viudas-y-su-cuidado}}

\bibverse{3} Honra á las viudas que en verdad son viudas. \bibverse{4}
Pero si alguna viuda tuviere hijos, ó nietos, aprendan primero á
gobernar su casa piadosamente, y á recompensar á sus padres: porque esto
es lo honesto y agradable delante de Dios. \bibverse{5} Ahora, la que en
verdad es viuda y solitaria, espera en Dios, y es diligente en
suplicaciones y oraciones noche y día. \bibverse{6} Pero la que vive en
delicias, viviendo está muerta. \bibverse{7} Denuncia pues estas cosas,
para que sean sin reprensión. \bibverse{8} Y si alguno no tiene cuidado
de los suyos, y mayormente de los de su casa, la fe negó, y es peor que
un infiel. \footnote{\textbf{5:8} Mat 15,5-6}

\bibverse{9} La viuda sea puesta en clase especial, no menos que de
sesenta años, que haya sido esposa de un solo marido. \bibverse{10} Que
tenga testimonio en buenas obras; si crió hijos; si ha ejercitado la
hospitalidad; si ha lavado los pies de los santos; si ha socorrido á los
afligidos; si ha seguido toda buena obra.

\bibverse{11} Pero viudas más jóvenes no admitas: porque después de
hacerse licenciosas contra Cristo, quieren casarse. \bibverse{12}
Condenadas ya, por haber falseado la primera fe. \bibverse{13} Y aun
también se acostumbran á ser ociosas, á andar de casa en casa; y no
solamente ociosas, sino también parleras y curiosas, hablando lo que no
conviene. \bibverse{14} Quiero pues, que las que son jóvenes se casen,
críen hijos, gobiernen la casa; que ninguna ocasión den al adversario
para maldecir. \footnote{\textbf{5:14} 1Tim 2,15; 1Cor 7,9}
\bibverse{15} Porque ya algunas han vuelto atrás en pos de Satanás.
\bibverse{16} Si algún fiel ó alguna fiel tiene viudas, manténgalas, y
no sea gravada la iglesia; á fin de que haya lo suficiente para las que
de verdad son viudas.

\hypertarget{del-comportamiento-observado-contra-los-mayores}{%
\subsection{Del comportamiento observado contra los
mayores}\label{del-comportamiento-observado-contra-los-mayores}}

\bibverse{17} Los ancianos que gobiernan bien, sean tenidos por dignos
de doblada honra; mayormente los que trabajan en predicar y enseñar.
\footnote{\textbf{5:17} Hech 14,23; Rom 12,8} \bibverse{18} Porque la
Escritura dice: No embozarás al buey que trilla; y: Digno es el obrero
de su jornal. \footnote{\textbf{5:18} 1Cor 9,9; Luc 10,7}

\bibverse{19} Contra el anciano no recibas acusación sino con dos ó tres
testigos. \footnote{\textbf{5:19} Deut 19,15; Mat 18,16} \bibverse{20} A
los que pecaren, repréndelos delante de todos, para que los otros
también teman. \footnote{\textbf{5:20} Gal 2,14} \bibverse{21} Te
requiero delante de Dios y del Señor Jesucristo, y de sus ángeles
escogidos, que guardes estas cosas sin perjuicio de nadie, que nada
hagas inclinándote á la una parte. \bibverse{22} No impongas de ligero
las manos á ninguno, ni comuniques en pecados ajenos: consérvate en
limpieza. \footnote{\textbf{5:22} 1Tim 4,14}

\hypertarget{amonestaciuxf3n-personal-para-timoteo}{%
\subsection{Amonestación personal para
Timoteo}\label{amonestaciuxf3n-personal-para-timoteo}}

\bibverse{23} No bebas de aquí adelante agua, sino usa de un poco de
vino por causa del estómago, y de tus continuas enfermedades.

\bibverse{24} Los pecados de algunos hombres, antes que vengan ellos á
juicio, son manifiestos; mas á otros les vienen después. \bibverse{25}
Asimismo las buenas obras antes son manifiestas; y las que son de otra
manera, no pueden esconderse.

\hypertarget{reglas-para-los-esclavos-cristianos}{%
\subsection{Reglas para los esclavos
cristianos}\label{reglas-para-los-esclavos-cristianos}}

\hypertarget{section-5}{%
\section{6}\label{section-5}}

\bibverse{1} Todos los que están debajo del yugo de servidumbre, tengan
á sus señores por dignos de toda honra, porque no sea blasfemado el
nombre del Señor y la doctrina. \bibverse{2} Y los que tienen amos
fieles, no los tengan en menos, por ser hermanos; antes sírvanles mejor,
por cuanto son fieles y amados, y partícipes del beneficio. Esto enseña
y exhorta. \footnote{\textbf{6:2} Efes 6,5-8; Filem 1,16}

\hypertarget{los-terribles-frutos-de-la-herejuxeda-y-los-peligros-de-la-codicia}{%
\subsection{Los terribles frutos de la herejía y los peligros de la
codicia}\label{los-terribles-frutos-de-la-herejuxeda-y-los-peligros-de-la-codicia}}

\bibverse{3} Si alguno enseña otra cosa, y no asiente á sanas palabras
de nuestro Señor Jesucristo, y á la doctrina que es conforme á la
piedad; \footnote{\textbf{6:3} Gal 1,6-9; 2Tim 1,13} \bibverse{4} Es
hinchado, nada sabe, y enloquece acerca de cuestiones y contiendas de
palabras, de las cuales nacen envidias, pleitos, maledicencias, malas
sospechas, \footnote{\textbf{6:4} 2Tim 2,14; Tit 3,10; Tit 1,3-11}
\bibverse{5} Porfías de hombres corruptos de entendimiento y privados de
la verdad, que tienen la piedad por granjería: apártate de los tales.
\footnote{\textbf{6:5} 1Tim 4,8; Mat 6,25-34; Fil 4,11-12; Heb 13,5}

\bibverse{6} Empero grande granjería es la piedad con contentamiento.
\bibverse{7} Porque nada hemos traído á este mundo, y sin duda nada
podremos sacar. \footnote{\textbf{6:7} Ecl 5,14; Job 1,21} \bibverse{8}
Así que, teniendo sustento y con qué cubrirnos, seamos contentos con
esto. \footnote{\textbf{6:8} Prov 30,8} \bibverse{9} Porque los que
quieren enriquecerse, caen en tentación y lazo, y en muchas codicias
locas y dañosas, que hunden á los hombres en perdición y muerte.
\footnote{\textbf{6:9} Prov 28,22; Mat 13,22} \bibverse{10} Porque el
amor del dinero es la raíz de todos los males: el cual codiciando
algunos, se descaminaron de la fe, y fueron traspasados de muchos
dolores. \footnote{\textbf{6:10} 1Tim 1,19; Efes 5,5}

\hypertarget{recordatorio-a-timoteo-de-perseverar-en-la-fidelidad-y-luchar-por-la-fe}{%
\subsection{Recordatorio a Timoteo de perseverar en la fidelidad y
luchar por la
fe}\label{recordatorio-a-timoteo-de-perseverar-en-la-fidelidad-y-luchar-por-la-fe}}

\bibverse{11} Mas tú, oh hombre de Dios, huye de estas cosas, y sigue la
justicia, la piedad, la fe, la caridad, la paciencia, la mansedumbre.
\footnote{\textbf{6:11} 2Tim 2,22; 2Tim 3,17} \bibverse{12} Pelea la
buena batalla de la fe, echa mano de la vida eterna, á la cual asimismo
eres llamado, habiendo hecho buena profesión delante de muchos testigos.
\footnote{\textbf{6:12} 1Tim 1,18; 1Tim 4,14; 1Cor 9,25-26; 2Tim 4,7;
  Heb 3,1} \bibverse{13} Te mando delante de Dios, que da vida á todas
las cosas, y de Jesucristo, que testificó la buena profesión delante de
Poncio Pilato, \footnote{\textbf{6:13} Juan 18,36-37; Apoc 1,5}
\bibverse{14} Que guardes el mandamiento sin mácula ni reprensión, hasta
la aparición de nuestro Señor Jesucristo: \bibverse{15} La cual á su
tiempo mostrará el Bienaventurado y solo Poderoso, Rey de reyes, y Señor
de señores; \bibverse{16} Quien sólo tiene inmortalidad, que habita en
luz inaccesible; á quien ninguno de los hombres ha visto ni puede ver:
al cual sea la honra y el imperio sempiterno. Amén. \footnote{\textbf{6:16}
  Éxod 33,20; Juan 1,18}

\hypertarget{recordatorio-para-los-ricos-hermanos}{%
\subsection{Recordatorio para los ricos
hermanos}\label{recordatorio-para-los-ricos-hermanos}}

\bibverse{17} A los ricos de este siglo manda que no sean altivos, ni
pongan la esperanza en la incertidumbre de las riquezas, sino en el Dios
vivo, que nos da todas las cosas en abundancia de que gocemos:
\footnote{\textbf{6:17} Sal 62,11; Luc 12,15-21} \bibverse{18} Que hagan
bien, que sean ricos en buenas obras, dadivosos, que con facilidad
comuniquen; \bibverse{19} Atesorando para sí buen fundamento para lo por
venir, que echen mano á la vida eterna. \footnote{\textbf{6:19} Mat
  6,20; Luc 16,9}

\hypertarget{advertencia-final-contra-la-herejuxeda}{%
\subsection{Advertencia final contra la
herejía}\label{advertencia-final-contra-la-herejuxeda}}

\bibverse{20} Oh Timoteo, guarda lo que se te ha encomendado, evitando
las profanas pláticas de vanas cosas, y los argumentos de la falsamente
llamada ciencia: \^{}\^{} \bibverse{21} La cual profesando algunos,
fueron descaminados acerca de la fe. La gracia sea contigo. Amén. La
primera epístola á Timoteo fué escrita de Laodicea, que es metrópoli de
la Frigia Pacatiana.
