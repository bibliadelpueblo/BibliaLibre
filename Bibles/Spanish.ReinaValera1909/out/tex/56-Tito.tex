\hypertarget{bendiciones}{%
\subsection{Bendiciones}\label{bendiciones}}

\hypertarget{section}{%
\section{1}\label{section}}

\bibleverse{1} Pablo, siervo de Dios, y apóstol de Jesucristo, según la
fe de los escogidos de Dios, y el conocimiento de la verdad que es según
la piedad, \bibleverse{2} Para la esperanza de la vida eterna, la cual
Dios, que no puede mentir, prometió antes de los tiempos de los siglos,
\bibleverse{3} Y manifestó á sus tiempos su palabra por la predicación,
que me es á mí encomendada por mandamiento de nuestro Salvador Dios;
\footnote{\textbf{1:3} Efes 1,9-10} \bibleverse{4} A Tito, verdadero
hijo en la común fe: Gracia, misericordia, y paz de Dios Padre, y del
Señor Jesucristo Salvador nuestro.

\footnote{\textbf{1:4} 1Tim 1,2}

\hypertarget{regulaciones-que-rigen-el-nombramiento-de-ancianos-como-luxedderes-de-la-iglesia}{%
\subsection{Regulaciones que rigen el nombramiento de ancianos como
líderes de la
iglesia}\label{regulaciones-que-rigen-el-nombramiento-de-ancianos-como-luxedderes-de-la-iglesia}}

\bibleverse{5} Por esta causa te dejé en Creta, para que corrigieses lo
que falta, y pusieses ancianos por las villas, así como yo te mandé:
\footnote{\textbf{1:5} Hech 14,23} \bibleverse{6} El que fuere sin
crimen, marido de una mujer, que tenga hijos fieles que no estén
acusados de disolución, ó contumaces. \footnote{\textbf{1:6} 1Tim 3,1-7}
\bibleverse{7} Porque es menester que el obispo sea sin crimen, como
dispensador de Dios; no soberbio, no iracundo, no amador del vino, no
heridor, no codicioso de torpes ganancias; \footnote{\textbf{1:7} 1Cor
  4,1; 2Tim 2,24} \bibleverse{8} Sino hospedador, amador de lo bueno,
templado, justo, santo, continente; \bibleverse{9} Retenedor de la fiel
palabra que es conforme á la doctrina: para que también pueda exhortar
con sana doctrina, y convencer á los que contradijeren.

\hypertarget{reglas-de-conducta-contra-seductores-maliciosos-y-falsos-maestros-hipuxf3critas}{%
\subsection{Reglas de conducta contra seductores maliciosos y falsos
maestros
hipócritas}\label{reglas-de-conducta-contra-seductores-maliciosos-y-falsos-maestros-hipuxf3critas}}

\bibleverse{10} Porque hay aún muchos contumaces, habladores de
vanidades, y engañadores de las almas, mayormente los que son de la
circuncisión, \bibleverse{11} A los cuales es preciso tapar la boca; que
trastornan casas enteras; enseñando lo que no conviene, por torpe
ganancia. \bibleverse{12} Dijo uno de ellos, propio profeta de ellos:
Los Cretenses, siempre mentirosos, malas bestias, vientres perezosos.
\bibleverse{13} Este testimonio es verdadero: por tanto, repréndelos
duramente, para que sean sanos en la fe, \bibleverse{14} No atendiendo á
fábulas judaicas, y á mandamientos de hombres que se apartan de la
verdad. \footnote{\textbf{1:14} 1Tim 4,7; 2Tim 4,4} \bibleverse{15}
Todas las cosas son limpias á los limpios; mas á los contaminados é
infieles nada es limpio: antes su alma y conciencia están contaminadas.
\footnote{\textbf{1:15} Mat 15,11; Rom 14,20} \bibleverse{16} Profésanse
conocer á Dios; mas con los hechos lo niegan, siendo abominables y
rebeldes, reprobados para toda buena obra. \footnote{\textbf{1:16} 2Tim
  3,5}

\hypertarget{regulaciones-para-las-fincas-individuales-en-la-comunidad}{%
\subsection{Regulaciones para las fincas individuales en la
comunidad}\label{regulaciones-para-las-fincas-individuales-en-la-comunidad}}

\hypertarget{section-1}{%
\section{2}\label{section-1}}

\bibleverse{1} Empero tú, habla lo que conviene á la sana doctrina:
\footnote{\textbf{2:1} 2Tim 1,13} \bibleverse{2} Que los viejos sean
templados, graves, prudentes, sanos en la fe, en la caridad, en la
paciencia. \footnote{\textbf{2:2} 1Tim 5,1} \bibleverse{3} Las viejas,
asimismo, se distingan en un porte santo; no calumniadoras, no dadas á
mucho vino, maestras de honestidad: \footnote{\textbf{2:3} 1Tim 3,11}
\bibleverse{4} Que enseñen á las mujeres jóvenes á ser prudentes, á que
amen á sus maridos, á que amen á sus hijos, \bibleverse{5} A ser
templadas, castas, que tengan cuidado de la casa, buenas, sujetas á sus
maridos; porque la palabra de Dios no sea blasfemada. \footnote{\textbf{2:5}
  Efes 5,22}

\bibleverse{6} Exhorta asimismo á los mancebos á que sean comedidos;
\bibleverse{7} Mostrándote en todo por ejemplo de buenas obras; en
doctrina haciendo ver integridad, gravedad, \bibleverse{8} Palabra sana,
é irreprensible; que el adversario se avergüence, no teniendo mal
ninguno que decir de vosotros.

\bibleverse{9} Exhorta á los siervos á que sean sujetos á sus señores,
que agraden en todo, no respondones; \footnote{\textbf{2:9} Efes 6,5-6;
  1Tim 6,1-2; 1Pe 2,18} \bibleverse{10} No defraudando, antes mostrando
toda buena lealtad, para que adornen en todo la doctrina de nuestro
Salvador Dios.

\hypertarget{justificaciuxf3n-de-estos-reglamentos-haciendo-referencia-a-la-gracia-de-dios-que-apareciuxf3-en-el-mundo}{%
\subsection{Justificación de estos reglamentos haciendo referencia a la
gracia de Dios que apareció en el
mundo}\label{justificaciuxf3n-de-estos-reglamentos-haciendo-referencia-a-la-gracia-de-dios-que-apareciuxf3-en-el-mundo}}

\bibleverse{11} Porque la gracia de Dios que trae salvación á todos los
hombres, se manifestó, \bibleverse{12} Enseñándonos que, renunciando á
la impiedad y á los deseos mundanos, vivamos en este siglo templada, y
justa, y píamente, \bibleverse{13} Esperando aquella esperanza
bienaventurada, y la manifestación gloriosa del gran Dios y Salvador
nuestro Jesucristo, \footnote{\textbf{2:13} 1Cor 1,7; Fil 3,20; 1Tes
  1,10} \bibleverse{14} Que se dió á sí mismo por nosotros para
redimirnos de toda iniquidad, y limpiar para sí un pueblo propio, celoso
de buenas obras. \footnote{\textbf{2:14} Gal 1,4; 1Tim 2,6; Éxod 19,5;
  Efes 2,10}

\bibleverse{15} Esto habla y exhorta, y reprende con toda autoridad.
Nadie te desprecie. \footnote{\textbf{2:15} 1Tim 4,12}

\hypertarget{sobre-el-comportamiento-contra-las-autoridades-paganas-y-los-no-cristianos-y-sobre-el-camino-de-los-cristianos-como-pueblo-renovado}{%
\subsection{Sobre el comportamiento contra las autoridades paganas y los
no cristianos y sobre el camino de los cristianos como pueblo
renovado}\label{sobre-el-comportamiento-contra-las-autoridades-paganas-y-los-no-cristianos-y-sobre-el-camino-de-los-cristianos-como-pueblo-renovado}}

\hypertarget{section-2}{%
\section{3}\label{section-2}}

\bibleverse{1} Amonéstales que se sujeten á los príncipes y potestades,
que obedezcan, que estén prontos á toda buena obra. \footnote{\textbf{3:1}
  Rom 13,1; 1Pe 2,13} \bibleverse{2} Que á nadie infamen, que no sean
pendencieros, sino modestos, mostrando toda mansedumbre para con todos
los hombres. \footnote{\textbf{3:2} Fil 4,5} \bibleverse{3} Porque
también éramos nosotros necios en otro tiempo, rebeldes, extraviados,
sirviendo á concupiscencias y deleites diversos, viviendo en malicia y
en envidia, aborrecibles, aborreciendo los unos á los otros. \footnote{\textbf{3:3}
  1Cor 6,11; Efes 2,2; Efes 5,8; 1Pe 4,3} \bibleverse{4} Mas cuando se
manifestó la bondad de Dios nuestro Salvador, y su amor para con los
hombres, \footnote{\textbf{3:4} Tit 2,11} \bibleverse{5} No por obras de
justicia que nosotros habíamos hecho, mas por su misericordia nos salvó,
por el lavacro de la regeneración, y de la renovación del Espíritu
Santo; \footnote{\textbf{3:5} 2Tim 1,9; Juan 3,5; Efes 5,26}
\bibleverse{6} El cual derramó en nosotros abundantemente por Jesucristo
nuestro Salvador, \footnote{\textbf{3:6} Jl 3,1} \bibleverse{7} Para
que, justificados por su gracia, seamos hechos herederos según la
esperanza de la vida eterna.

\footnote{\textbf{3:7} Rom 3,26}

\hypertarget{conclusiuxf3n-sobre-el-comportamiento-frente-a-las-aberraciones-doctrinales-y-sus-representantes}{%
\subsection{Conclusión sobre el comportamiento frente a las aberraciones
doctrinales y sus
representantes}\label{conclusiuxf3n-sobre-el-comportamiento-frente-a-las-aberraciones-doctrinales-y-sus-representantes}}

\bibleverse{8} Palabra fiel, y estas cosas quiero que afirmes, para que
los que creen á Dios procuren gobernarse en buenas obras. Estas cosas
son buenas y útiles á los hombres. \bibleverse{9} Mas las cuestiones
necias, y genealogías, y contenciones, y debates acerca de la ley,
evita; porque son sin provecho y vanas. \footnote{\textbf{3:9} 1Tim 1,4;
  1Tim 4,7; 2Tim 2,14} \bibleverse{10} Rehusa hombre hereje, después de
una y otra amonestación; \footnote{\textbf{3:10} Mat 18,15-17; 2Jn 1,10}
\bibleverse{11} Estando cierto que el tal es trastornado, y peca, siendo
condenado de su propio juicio.

\footnote{\textbf{3:11} 1Tim 6,4-5}

\hypertarget{comentarios-personales-finales-uxfaltimos-pedidos-y-saludos}{%
\subsection{Comentarios personales finales, últimos pedidos y
saludos}\label{comentarios-personales-finales-uxfaltimos-pedidos-y-saludos}}

\bibleverse{12} Cuando enviare á ti á Artemas, ó á Tichîco, procura
venir á mí, á Nicópolis: porque allí he determinado invernar.
\footnote{\textbf{3:12} Efes 6,21} \bibleverse{13} A Zenas doctor de la
ley, y á Apolos, envía delante, procurando que nada les falte.
\footnote{\textbf{3:13} Hech 18,24; 1Cor 3,5-6} \bibleverse{14} Y
aprendan asimismo los nuestros á gobernarse en buenas obras para los
usos necesarios, para que no sean sin fruto. \footnote{\textbf{3:14} Tit
  2,14; Mat 7,19}

\bibleverse{15} Todos los que están conmigo te saludan. Saluda á los que
nos aman en la fe. La gracia sea con todos vosotros. Amén. A Tito, el
cual fué el primer obispo ordenado á la iglesia de los Cretenses,
escrita de Nicópolis de Macedonia.
