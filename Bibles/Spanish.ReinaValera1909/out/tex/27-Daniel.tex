\hypertarget{la-crianza-de-daniel-en-la-corte-pagana-de-babilonia}{%
\subsection{La crianza de Daniel en la corte pagana de
Babilonia}\label{la-crianza-de-daniel-en-la-corte-pagana-de-babilonia}}

\hypertarget{section}{%
\section{1}\label{section}}

\bibverse{1} En el año tercero del reinado de Joacim rey de Judá, vino
Nabucodonosor rey de Babilonia á Jerusalem, y cercóla. \footnote{\textbf{1:1}
  2Re 24,1-2} \bibverse{2} Y el Señor entregó en sus manos á Joacim rey
de Judá, y parte de los vasos de la casa de Dios, y trájolos á tierra de
Sinar, á la casa de su dios: y metió los vasos en la casa del tesoro de
su dios.

\hypertarget{daniel-y-sus-amigos-vienen-a-babilonia-para-ser-entrenados-para-el-servicio-real}{%
\subsection{Daniel y sus amigos vienen a Babilonia para ser entrenados
para el servicio
real}\label{daniel-y-sus-amigos-vienen-a-babilonia-para-ser-entrenados-para-el-servicio-real}}

\bibverse{3} Y dijo el rey á Aspenaz, príncipe de sus eunucos, que
trajese de los hijos de Israel, del linaje real de los príncipes,
\bibverse{4} Muchachos en quienes no hubiese tacha alguna, y de buen
parecer, y enseñados en toda sabiduría, y sabios en ciencia, y de buen
entendimiento, é idóneos para estar en el palacio del rey; y que les
enseñase las letras y la lengua de los Caldeos. \bibverse{5} Y señalóles
el rey ración para cada día de la ración de la comida del rey, y del
vino de su beber: que los criase tres años, para que al fin de ellos
estuviesen delante del rey.

\bibverse{6} Y fueron entre ellos, de los hijos de Judá, Daniel,
Ananías, Misael y Azarías: \bibverse{7} A los cuales el príncipe de los
eunucos puso nombres: y puso á Daniel, Beltsasar; y á Ananías, Sadrach;
y á Misael, Mesach; y á Azarías, Abed-nego.

\hypertarget{daniel-obtuvo-permiso-para-comer-alimentos-que-se-ajustan-a-la-ley-juduxeda}{%
\subsection{Daniel obtuvo permiso para comer alimentos que se ajustan a
la ley
judía}\label{daniel-obtuvo-permiso-para-comer-alimentos-que-se-ajustan-a-la-ley-juduxeda}}

\bibverse{8} Y Daniel propuso en su corazón de no contaminarse en la
ración de la comida del rey, ni en el vino de su beber: pidió por tanto
al príncipe de los eunucos de no contaminarse. \footnote{\textbf{1:8}
  Lev 11,-1} \bibverse{9} (Y puso Dios á Daniel en gracia y en buena
voluntad con el príncipe de los eunucos.) \footnote{\textbf{1:9} Gén
  39,21} \bibverse{10} Y dijo el príncipe de los eunucos á Daniel: Tengo
temor de mi señor el rey, que señaló vuestra comida y vuestra bebida;
pues luego que él habrá visto vuestros rostros más tristes que los de
los muchachos que son semejantes á vosotros, condenaréis para con el rey
mi cabeza.

\bibverse{11} Entonces dijo Daniel á Melsar, que estaba puesto por el
príncipe de los eunucos sobre Daniel, Ananías, Misael, y Azarías:
\bibverse{12} Prueba, te ruego, tus siervos diez días, y dennos
legumbres á comer, y agua á beber. \bibverse{13} Parezcan luego delante
de ti nuestros rostros, y los rostros de los muchachos que comen de la
ración de la comida del rey; y según que vieres, harás con tus siervos.
\bibverse{14} Consintió pues con ellos en esto, y probó con ellos diez
días.

\bibverse{15} Y al cabo de los diez días pareció el rostro de ellos
mejor y más nutrido de carne, que los otros muchachos que comían de la
ración de la comida del rey. \bibverse{16} Así fué que Melsar tomaba la
ración de la comida de ellos, y el vino de su beber, y dábales
legumbres.

\hypertarget{el-entrenamiento-bendecido-por-dios-de-los-cuatro-amigos-y-su-aceptaciuxf3n-en-el-servicio-real}{%
\subsection{El entrenamiento bendecido por Dios de los cuatro amigos y
su aceptación en el servicio
real}\label{el-entrenamiento-bendecido-por-dios-de-los-cuatro-amigos-y-su-aceptaciuxf3n-en-el-servicio-real}}

\bibverse{17} Y á estos cuatro muchachos dióles Dios conocimiento é
inteligencia en todas letras y ciencia: mas Daniel tuvo entendimiento en
toda visión y sueños. \footnote{\textbf{1:17} Ezeq 28,3}

\bibverse{18} Pasados pues los días al fin de los cuales había dicho el
rey que los trajesen, el príncipe de los eunucos los trajo delante de
Nabucodonosor. \bibverse{19} Y el rey habló con ellos, y no fué hallado
entre todos ellos otro como Daniel, Ananías, Misael, y Azarías: y así
estuvieron delante del rey. \bibverse{20} Y en todo negocio de sabiduría
é inteligencia que el rey les demandó, hallólos diez veces mejores que
todos los magos y astrólogos que había en todo su reino.

\bibverse{21} Y fué Daniel hasta el año primero del rey Ciro.

\hypertarget{el-sueuxf1o-de-nabucodonosor-interpretado-por-daniel}{%
\subsection{El sueño de Nabucodonosor interpretado por
Daniel}\label{el-sueuxf1o-de-nabucodonosor-interpretado-por-daniel}}

\hypertarget{section-1}{%
\section{2}\label{section-1}}

\bibverse{1} Y en el segundo año del reinado de Nabucodonosor, soñó
Nabucodonosor sueños, y perturbóse su espíritu, y su sueño se huyó de
él. \bibverse{2} Y mandó el rey llamar magos, astrólogos, y
encantadores, y Caldeos, para que mostrasen al rey sus sueños. Vinieron
pues, y se presentaron delante del rey. \footnote{\textbf{2:2} Is
  47,12-13} \bibverse{3} Y el rey les dijo: He soñado un sueño, y mi
espíritu se ha perturbado por saber del sueño.

\bibverse{4} Entonces hablaron los Caldeos al rey en lengua aramea: Rey,
para siempre vive: di el sueño á tus siervos, y mostraremos la
declaración.

\bibverse{5} Respondió el rey y dijo á los Caldeos: El negocio se me
fué: si no me mostráis el sueño y su declaración, seréis hechos cuartos,
y vuestras casas serán puestas por muladares. \bibverse{6} Y si
mostrareis el sueño y su declaración, recibiréis de mí dones y mercedes
y grande honra: por tanto, mostradme el sueño y su declaración.

\bibverse{7} Respondieron la segunda vez, y dijeron: Diga el rey el
sueño á sus siervos, y mostraremos su declaración.

\bibverse{8} El rey respondió, y dijo: Yo conozco ciertamente que
vosotros ponéis dilaciones, porque veis que el negocio se me ha ido.
\bibverse{9} Si no me mostráis el sueño, una sola sentencia será de
vosotros. Ciertamente preparáis respuesta mentirosa y perversa que decir
delante de mí, entre tanto que se muda el tiempo: por tanto, decidme el
sueño, para que yo entienda que me podéis mostrar su declaración.

\bibverse{10} Los Caldeos respondieron delante del rey, y dijeron: No
hay hombre sobre la tierra que pueda declarar el negocio del rey: demás
de esto, ningún rey, príncipe, ni señor, preguntó cosa semejante á
ningún mago, ni astrólogo, ni Caldeo. \bibverse{11} Finalmente, el
negocio que el rey demanda, es singular, ni hay quien lo pueda declarar
delante del rey, salvo los dioses cuya morada no es con la carne.

\hypertarget{el-rey-ordena-la-ejecuciuxf3n-de-todos-los-adivinos-daniel-procura-el-aplazamiento-de-la-ejecuciuxf3n-mediante-su-promesa}{%
\subsection{El rey ordena la ejecución de todos los adivinos; Daniel
procura el aplazamiento de la ejecución mediante su
promesa}\label{el-rey-ordena-la-ejecuciuxf3n-de-todos-los-adivinos-daniel-procura-el-aplazamiento-de-la-ejecuciuxf3n-mediante-su-promesa}}

\bibverse{12} Por esto el rey con ira y con grande enojo, mandó que
matasen á todos los sabios de Babilonia. \bibverse{13} Y publicóse el
mandamiento, y los sabios eran llevados á la muerte; y buscaron á Daniel
y á sus compañeros para matarlos.

\bibverse{14} Entonces Daniel habló avisada y prudentemente á Arioch,
capitán de los de la guarda del rey, que había salido para matar los
sabios de Babilonia. \footnote{\textbf{2:14} Dan 1,17; Dan 1,20; Dan
  2,24} \bibverse{15} Habló y dijo á Arioch capitán del rey: ¿Qué es la
causa que este mandamiento se publica de parte del rey tan
apresuradamente? Entonces Arioch declaró el negocio á Daniel.
\bibverse{16} Y Daniel entró, y pidió al rey que le diese tiempo, y que
él mostraría al rey la declaración.

\hypertarget{daniel-recibe-la-revelaciuxf3n-del-misterio-de-dios-a-travuxe9s-de-un-sueuxf1o-su-acciuxf3n-de-gracias-y-oraciuxf3n}{%
\subsection{Daniel recibe la revelación del misterio de Dios a través de
un sueño; su acción de gracias y
oración}\label{daniel-recibe-la-revelaciuxf3n-del-misterio-de-dios-a-travuxe9s-de-un-sueuxf1o-su-acciuxf3n-de-gracias-y-oraciuxf3n}}

\bibverse{17} Fuése luego Daniel á su casa, y declaró el negocio á
Ananías, Misael, y Azarías, sus compañeros, \bibverse{18} Para demandar
misericordias del Dios del cielo sobre este misterio, y que Daniel y sus
compañeros no pereciesen con los otros sabios de Babilonia.
\bibverse{19} Entonces el arcano fué revelado á Daniel en visión de
noche; por lo cual bendijo Daniel al Dios del cielo. \bibverse{20} Y
Daniel habló, y dijo: Sea bendito el nombre de Dios de siglo hasta
siglo: porque suya es la sabiduría y la fortaleza: \bibverse{21} Y él es
el que muda los tiempos y las oportunidades: quita reyes, y pone reyes:
da la sabiduría á los sabios, y la ciencia á los entendidos: \footnote{\textbf{2:21}
  Dan 4,14; Dan 4,22; Dan 4,29} \bibverse{22} El revela lo profundo y lo
escondido: conoce lo que está en tinieblas, y la luz mora con él.
\bibverse{23} A ti, oh Dios de mis padres, confieso y te alabo, que me
diste sabiduría y fortaleza, y ahora me enseñaste lo que te pedimos;
pues nos has enseñado el negocio del rey.

\hypertarget{daniel-le-dice-al-rey-el-contenido-del-sueuxf1o}{%
\subsection{Daniel le dice al rey el contenido del
sueño}\label{daniel-le-dice-al-rey-el-contenido-del-sueuxf1o}}

\bibverse{24} Después de esto Daniel entró á Arioch, al cual el rey
había puesto para matar á los sabios de Babilonia; fué, y díjole así: No
mates á los sabios de Babilonia: llévame delante del rey, que yo
mostraré al rey la declaración.

\bibverse{25} Entonces Arioch llevó prestamente á Daniel delante del
rey, y díjole así: Un varón de los trasportados de Judá he hallado, el
cual declarará al rey la interpretación.

\bibverse{26} Respondió el rey, y dijo á Daniel, al cual llamaban
Beltsasar: ¿Podrás tú hacerme entender el sueño que vi, y su
declaración?

\bibverse{27} Daniel respondió delante del rey, y dijo: El misterio que
el rey demanda, ni sabios, ni astrólogos, ni magos, ni adivinos lo
pueden enseñar al rey. \bibverse{28} Mas hay un Dios en los cielos, el
cual revela los misterios, y él ha hecho saber al rey Nabucodonosor lo
que ha de acontecer á cabo de días. Tu sueño, y las visiones de tu
cabeza sobre tu cama, es esto:

\bibverse{29} Tú, oh rey, en tu cama subieron tus pensamientos por saber
lo que había de ser en lo por venir; y el que revela los misterios te
mostró lo que ha de ser. \footnote{\textbf{2:29} Dan 2,22} \bibverse{30}
Y á mí ha sido revelado este misterio, no por sabiduría que en mí haya
más que en todos los vivientes, sino para que yo notifique al rey la
declaración, y que entendieses los pensamientos de tu corazón.
\footnote{\textbf{2:30} Gén 41,16}

\bibverse{31} Tú, oh rey, veías, y he aquí una grande imagen. Esta
imagen, que era muy grande, y cuya gloria era muy sublime, estaba en pie
delante de ti, y su aspecto era terrible. \bibverse{32} La cabeza de
esta imagen era de fino oro; sus pechos y sus brazos, de plata; su
vientre y sus muslos, de metal; \bibverse{33} Sus piernas de hierro; sus
pies, en parte de hierro, y en parte de barro cocido. \bibverse{34}
Estabas mirando, hasta que una piedra fué cortada, no con mano, la cual
hirió á la imagen en sus pies de hierro y de barro cocido, y los
desmenuzó. \bibverse{35} Entonces fué también desmenuzado el hierro, el
barro cocido, el metal, la plata y el oro, y se tornaron como tamo de
las eras del verano: y levantólos el viento, y nunca más se les halló
lugar. Mas la piedra que hirió á la imagen, fué hecha un gran monte, que
hinchió toda la tierra.

\hypertarget{la-interpretaciuxf3n-de-daniel-del-sueuxf1o}{%
\subsection{La interpretación de Daniel del
sueño}\label{la-interpretaciuxf3n-de-daniel-del-sueuxf1o}}

\bibverse{36} Este es el sueño: la declaración de él diremos también en
presencia del rey. \bibverse{37} Tú, oh rey, eres rey de reyes; porque
el Dios del cielo te ha dado reino, potencia, y fortaleza, y majestad.
\footnote{\textbf{2:37} Ezeq 26,7} \bibverse{38} Y todo lo que habitan
hijos de hombres, bestias del campo, y aves del cielo, él ha entregado
en tu mano, y te ha hecho enseñorear sobre todo ello: tú eres aquella
cabeza de oro. \footnote{\textbf{2:38} Jer 27,6}

\bibverse{39} Y después de ti se levantará otro reino menor que tú; y
otro tercer reino de metal, el cual se enseñoreará de toda la tierra.
\bibverse{40} Y el reino cuarto será fuerte como hierro; y como el
hierro desmenuza y doma todas las cosas, y como el hierro que quebranta
todas estas cosas, desmenuzará y quebrantará. \bibverse{41} Y lo que
viste de los pies y los dedos, en parte de barro cocido de alfarero, y
en parte de hierro, el reino será dividido; mas habrá en él algo de
fortaleza de hierro, según que viste el hierro mezclado con el tiesto de
barro. \bibverse{42} Y por ser los dedos de los pies en parte de hierro,
y en parte de barro cocido, en parte será el reino fuerte, y en parte
será frágil. \bibverse{43} Cuanto á aquello que viste, el hierro
mezclado con tiesto de barro, mezclaránse con simiente humana, mas no se
pegarán el uno con el otro, como el hierro no se mistura con el tiesto.

\bibverse{44} Y en los días de estos reyes, levantará el Dios del cielo
un reino que nunca jamás se corromperá: y no será dejado á otro pueblo
este reino; el cual desmenuzará y consumirá todos estos reinos, y él
permanecerá para siempre. \footnote{\textbf{2:44} Dan 7,14; Dan 7,27; Is
  9,6; 1Cor 15,24; Apoc 11,15} \bibverse{45} De la manera que viste que
del monte fué cortada una piedra, no con manos, la cual desmenuzó al
hierro, al metal, al tiesto, á la plata, y al oro; el gran Dios ha
mostrado al rey lo que ha de acontecer en lo por venir: y el sueño es
verdadero, y fiel su declaración. \footnote{\textbf{2:45} Dan 2,34}

\hypertarget{el-reconocimiento-de-nabucodonosor-de-la-grandeza-del-dios-de-los-juduxedos-donaciuxf3n-y-homenaje-a-daniel}{%
\subsection{El reconocimiento de Nabucodonosor de la grandeza del Dios
de los judíos; Donación y homenaje a
Daniel}\label{el-reconocimiento-de-nabucodonosor-de-la-grandeza-del-dios-de-los-juduxedos-donaciuxf3n-y-homenaje-a-daniel}}

\bibverse{46} Entonces el rey Nabucodonosor cayó sobre su rostro, y
humillóse á Daniel, y mandó que le sacrificasen presentes y perfumes.
\bibverse{47} El rey habló á Daniel, y dijo: Ciertamente que el Dios
vuestro es Dios de dioses, y el Señor de los reyes, y el descubridor de
los misterios, pues pudiste revelar este arcano. \footnote{\textbf{2:47}
  Dan 3,29; Jos 2,11; Sal 86,8; Is 42,8-9}

\bibverse{48} Entonces el rey engrandeció á Daniel, y le dió muchos y
grandes dones, y púsolo por gobernador de toda la provincia de
Babilonia, y por príncipe de los gobernadores sobre todos los sabios de
Babilonia. \footnote{\textbf{2:48} Dan 2,6} \bibverse{49} Y Daniel
solicitó del rey, y él puso sobre los negocios de la provincia de
Babilonia á Sadrach, Mesach, y Abed-nego: y Daniel estaba á la puerta
del rey. \footnote{\textbf{2:49} Dan 3,12}

\hypertarget{nabucodonosor-hace-levantar-un-uxeddolo-y-ordena-su-adoraciuxf3n-sobre-el-castigo-de-muerte-por-fuego}{%
\subsection{Nabucodonosor hace levantar un ídolo y ordena su adoración
sobre el castigo de muerte por
fuego}\label{nabucodonosor-hace-levantar-un-uxeddolo-y-ordena-su-adoraciuxf3n-sobre-el-castigo-de-muerte-por-fuego}}

\hypertarget{section-2}{%
\section{3}\label{section-2}}

\bibverse{1} El rey Nabucodonosor hizo una estatua de oro, la altura de
la cual era de sesenta codos, su anchura de seis codos: levantóla en el
campo de Dura, en la provincia de Babilonia. \bibverse{2} Y envió el rey
Nabucodonosor á juntar los grandes, los asistentes y capitanes, oidores,
receptores, los del consejo, presidentes, y á todos los gobernadores de
las provincias, para que viniesen á la dedicación de la estatua que el
rey Nabucodonosor había levantado. \bibverse{3} Fueron pues reunidos los
grandes, los asistentes y capitanes, los oidores, receptores, los del
consejo, los presidentes, y todos los gobernadores de las provincias, á
la dedicación de la estatua que el rey Nabucodonosor había levantado: y
estaban en pie delante de la estatua que había levantado el rey
Nabucodonosor.

\bibverse{4} Y el pregonero pregonaba en alta voz: Mándase á vosotros,
oh pueblos, naciones, y lenguas, \bibverse{5} En oyendo el son de la
bocina, del pífano, del tamboril, del arpa, del salterio, de la zampoña,
y de todo instrumento músico, os postraréis y adoraréis la estatua de
oro que el rey Nabucodonosor ha levantado: \bibverse{6} Y cualquiera que
no se postrare y adorare, en la misma hora será echado dentro de un
horno de fuego ardiendo.

\bibverse{7} Por lo cual, en oyendo todos los pueblos el son de la
bocina, del pífano, del tamboril, del arpa, del salterio, de la zampoña,
y de todo instrumento músico, todos los pueblos, naciones, y lenguas, se
postraron, y adoraron la estatua de oro que el rey Nabucodonosor había
levantado.

\hypertarget{los-tres-amigos-de-daniel-se-niegan-a-adorar-la-imagen}{%
\subsection{Los tres amigos de Daniel se niegan a adorar la
imagen}\label{los-tres-amigos-de-daniel-se-niegan-a-adorar-la-imagen}}

\bibverse{8} Por esto en el mismo tiempo algunos varones Caldeos se
llegaron, y denunciaron de los Judíos, \bibverse{9} Hablando y diciendo
al rey Nabucodonosor: Rey, para siempre vive. \bibverse{10} Tú, oh rey,
pusiste ley que todo hombre en oyendo el son de la bocina, del pífano,
del tamboril, del arpa, del salterio, de la zampoña, y de todo
instrumento músico, se postrase y adorase la estatua de oro:
\bibverse{11} Y el que no se postrase y adorase, fuese echado dentro de
un horno de fuego ardiendo. \bibverse{12} Hay unos varones Judíos, los
cuales pusiste tú sobre los negocios de la provincia de Babilonia;
Sadrach, Mesach, y Abed-nego: estos varones, oh rey, no han hecho cuenta
de ti; no adoran tus dioses, no adoran la estatua de oro que tú
levantaste. \footnote{\textbf{3:12} Dan 2,49}

\bibverse{13} Entonces Nabucodonosor dijo con ira y con enojo que
trajesen á Sadrach, Mesach, y Abed-nego. Al punto fueron traídos estos
varones delante del rey. \bibverse{14} Habló Nabucodonosor, y díjoles:
¿Es verdad Sadrach, Mesach, y Abed-nego, que vosotros no honráis á mi
dios, ni adoráis la estatua de oro que he levantado? \bibverse{15} Ahora
pues, ¿estáis prestos para que en oyendo el son de la bocina, del
pífano, del tamboril, del arpa, del salterio, de la zampoña, y de todo
instrumento músico, os postréis, y adoréis la estatua que he hecho?
Porque si no la adorareis, en la misma hora seréis echados en medio de
un horno de fuego ardiendo: ¿y qué dios será aquel que os libre de mis
manos?

\bibverse{16} Sadrach, Mesach, y Abed-nego respondieron y dijeron al rey
Nabucodonosor: No cuidamos de responderte sobre este negocio.
\bibverse{17} He aquí nuestro Dios á quien honramos, puede librarnos del
horno de fuego ardiendo; y de tu mano, oh rey, nos librará. \footnote{\textbf{3:17}
  Sal 66,12} \bibverse{18} Y si no, sepas, oh rey, que tu dios no
adoraremos, ni tampoco honraremos la estatua que has levantado.
\footnote{\textbf{3:18} Éxod 20,3-5}

\hypertarget{arrojados-al-horno-los-tres-hombres-quedan-ilesos}{%
\subsection{Arrojados al horno, los tres hombres quedan
ilesos}\label{arrojados-al-horno-los-tres-hombres-quedan-ilesos}}

\bibverse{19} Entonces Nabucodonosor fué lleno de ira, y demudóse la
figura de su rostro sobre Sadrach, Mesach, y Abed-nego: así habló, y
ordenó que el horno se encendiese siete veces tanto de lo que cada vez
solía. \bibverse{20} Y mandó á hombres muy vigorosos que tenía en su
ejército, que atasen á Sadrach, Mesach, y Abed-nego, para echarlos en el
horno de fuego ardiendo. \bibverse{21} Entonces estos varones fueron
atados con sus mantos, y sus calzas, y sus turbantes, y sus vestidos, y
fueron echados dentro del horno de fuego ardiendo. \bibverse{22} Y
porque la palabra del rey daba priesa, y había procurado que se
encendiese mucho, la llama del fuego mató á aquellos que habían alzado á
Sadrach, Mesach, y Abed-nego. \bibverse{23} Y estos tres varones,
Sadrach, Mesach, y Abed-nego, cayeron atados dentro del horno de fuego
ardiendo.

\bibverse{24} Entonces el rey Nabucodonosor se espantó, y levantóse
apriesa, y habló, y dijo á los de su consejo: ¿No echaron tres varones
atados dentro del fuego? Ellos respondieron y dijeron al rey: Es verdad,
oh rey.

\bibverse{25} Respondió él y dijo: He aquí que yo veo cuatro varones
sueltos, que se pasean en medio del fuego, y ningún daño hay en ellos: y
el parecer del cuarto es semejante á hijo de los dioses. \footnote{\textbf{3:25}
  Is 43,2; Dan 3,28}

\hypertarget{el-rey-reconoce-la-grandeza-del-dios-de-los-juduxedos-ordena-su-honor-y-confirma-a-los-tres-amigos-de-daniel-en-su-alto-cargo}{%
\subsection{El rey reconoce la grandeza del Dios de los judíos, ordena
su honor y confirma a los tres amigos de Daniel en su alto
cargo}\label{el-rey-reconoce-la-grandeza-del-dios-de-los-juduxedos-ordena-su-honor-y-confirma-a-los-tres-amigos-de-daniel-en-su-alto-cargo}}

\bibverse{26} Entonces Nabucodonosor se acercó á la puerta del horno de
fuego ardiendo, y habló y dijo: Sadrach, Mesach, y Abed-nego, siervos
del alto Dios, salid y venid. Entonces Sadrach, Mesach, y Abed-nego,
salieron de en medio del fuego.

\bibverse{27} Y juntáronse los grandes, los gobernadores, los capitanes,
y los del consejo del rey, para mirar estos varones, como el fuego no se
enseñoreó de sus cuerpos, ni cabello de sus cabezas fué quemado, ni sus
ropas se mudaron, ni olor de fuego había pasado por ellos.

\bibverse{28} Nabucodonosor habló y dijo: Bendito el Dios de ellos, de
Sadrach, Mesach, y Abed-nego, que envió su ángel, y libró sus siervos
que esperaron en él, y el mandamiento del rey mudaron, y entregaron sus
cuerpos antes que sirviesen ni adorasen otro dios que su Dios.
\footnote{\textbf{3:28} Dan 6,23} \bibverse{29} Por mí pues se pone
decreto, que todo pueblo, nación, ó lengua, que dijere blasfemia contra
el Dios de Sadrach, Mesach, y Abed-nego, sea descuartizado, y su casa
sea puesta por muladar; por cuanto no hay dios que pueda librar como
éste. \footnote{\textbf{3:29} Dan 2,47}

\bibverse{30} Entonces el rey engrandeció á Sadrach, Mesach, y Abed-nego
en la provincia de Babilonia.

\hypertarget{el-segundo-sueuxf1o-de-nabucodonosor-su-humillaciuxf3n-y-exaltaciuxf3n}{%
\subsection{El segundo sueño de Nabucodonosor, su humillación y
exaltación}\label{el-segundo-sueuxf1o-de-nabucodonosor-su-humillaciuxf3n-y-exaltaciuxf3n}}

\hypertarget{section-3}{%
\section{4}\label{section-3}}

\bibverse{1} Nabucodonosor rey, á todos los pueblos, naciones, y
lenguas, que moran en toda la tierra: Paz os sea multiplicada:

\bibverse{2} Las señales y milagros que el alto Dios ha hecho conmigo,
conviene que yo las publique. \bibverse{3} ¡Cuán grandes son sus
señales, y cuán potentes sus maravillas! Su reino, reino sempiterno, y
su señorío hasta generación y generación.

\bibverse{4} Yo Nabucodonosor estaba quieto en mi casa, y floreciente en
mi palacio. \bibverse{5} Vi un sueño que me espantó, y las imaginaciones
y visiones de mi cabeza me turbaron en mi cama. \bibverse{6} Por lo cual
yo puse mandamiento para hacer venir delante de mí todos los sabios de
Babilonia, que me mostrasen la declaración del sueño. \bibverse{7} Y
vinieron magos, astrólogos, Caldeos, y adivinos: y dije el sueño delante
de ellos, mas nunca me mostraron su declaración; \bibverse{8} Hasta
tanto que entró delante de mí Daniel, cuyo nombre es Beltsasar, como el
nombre de mi dios, y en el cual hay espíritu de los dioses santos, y
dije el sueño delante de él, diciendo: \footnote{\textbf{4:8} Dan 5,11;
  Dan 5,14}

\bibverse{9} Beltsasar, príncipe de los magos, ya que he entendido que
hay en ti espíritu de los dioses santos, y que ningún misterio se te
esconde, exprésame las visiones de mi sueño que he visto, y su
declaración. \footnote{\textbf{4:9} Ezeq 28,3}

\hypertarget{nabucodonosor-comparte-el-sueuxf1o-con-daniel}{%
\subsection{Nabucodonosor comparte el sueño con
Daniel}\label{nabucodonosor-comparte-el-sueuxf1o-con-daniel}}

\bibverse{10} Aquestas las visiones de mi cabeza en mi cama: Parecíame
que veía un árbol en medio de la tierra, cuya altura era grande.
\footnote{\textbf{4:10} Ezeq 31,3-14} \bibverse{11} Crecía este árbol, y
hacíase fuerte, y su altura llegaba hasta el cielo, y su vista hasta el
cabo de toda la tierra. \bibverse{12} Su copa era hermosa, y su fruto en
abundancia, y para todos había en él mantenimiento. Debajo de él se
ponían á la sombra las bestias del campo, y en sus ramas hacían morada
las aves del cielo, y manteníase de él toda carne.

\bibverse{13} Veía en las visiones de mi cabeza en mi cama, y he aquí
que un vigilante y santo descendía del cielo. \bibverse{14} Y clamaba
fuertemente y decía así: Cortad el árbol, y desmochad sus ramas,
derribad su copa, y derramad su fruto: váyanse las bestias que están
debajo de él, y las aves de sus ramas. \footnote{\textbf{4:14} Dan 4,20}
\bibverse{15} Mas la cepa de sus raíces dejaréis en la tierra, y con
atadura de hierro y de metal entre la hierba del campo; y sea mojado con
el rocío del cielo, y su parte con las bestias en la hierba de la
tierra. \bibverse{16} Su corazón sea mudado de corazón de hombre, y
séale dado corazón de bestia, y pasen sobre él siete tiempos.

\bibverse{17} La sentencia es por decreto de los vigilantes, y por dicho
de los santos la demanda: para que conozcan los vivientes que el
Altísimo se enseñorea del reino de los hombres, y que á quien él quiere
lo da, y constituye sobre él al más bajo de los hombres. \footnote{\textbf{4:17}
  Dan 2,21}

\bibverse{18} Yo el rey Nabucodonosor he visto este sueño. Tú pues,
Beltsasar, dirás la declaración de él, porque todos los sabios de mi
reino nunca pudieron mostrarme su interpretación: mas tú puedes, porque
hay en ti espíritu de los dioses santos.

\hypertarget{la-consternaciuxf3n-de-daniel-y-la-interpretaciuxf3n-del-sueuxf1o}{%
\subsection{La consternación de Daniel y la interpretación del
sueño}\label{la-consternaciuxf3n-de-daniel-y-la-interpretaciuxf3n-del-sueuxf1o}}

\bibverse{19} Entonces Daniel, cuyo nombre era Beltsasar, estuvo
callando casi una hora, y sus pensamientos lo espantaban: El rey habló,
y dijo: Beltsasar, el sueño ni su declaración no te espante. Respondió
Beltsasar, y dijo: Señor mío, el sueño sea para tus enemigos, y su
declaración para los que mal te quieren.

\bibverse{20} El árbol que viste, que crecía y se hacía fuerte, y que su
altura llegaba hasta el cielo, y su vista por toda la tierra;
\bibverse{21} Y cuya copa era hermosa, y su fruto en abundancia, y que
para todos había mantenimiento en él; debajo del cual moraban las
bestias del campo, y en sus ramas habitaban las aves del cielo,
\bibverse{22} Tú mismo eres, oh rey, que creciste, y te hiciste fuerte,
pues creció tu grandeza, y ha llegado hasta el cielo, y tu señorío hasta
el cabo de la tierra.

\bibverse{23} Y cuanto á lo que vió el rey, un vigilante y santo que
descendía del cielo, y decía: Cortad el árbol y destruidlo: mas la cepa
de sus raíces dejaréis en la tierra, y con atadura de hierro y de metal
en la hierba del campo; y sea mojado con el rocío del cielo, y su parte
sea con las bestias del campo, hasta que pasen sobre él siete tiempos:

\bibverse{24} Esta es la declaración, oh rey, y la sentencia del
Altísimo, que ha venido sobre el rey mi señor:

\hypertarget{el-cumplimiento-de-todas-las-profecuxedas-de-daniel-la-arrogancia-de-nabucodonosor-desprecio-por-el-espuxedritu-conversiuxf3n-restauraciuxf3n}{%
\subsection{El cumplimiento de todas las profecías de Daniel: la
arrogancia de Nabucodonosor, desprecio por el espíritu, conversión,
restauración}\label{el-cumplimiento-de-todas-las-profecuxedas-de-daniel-la-arrogancia-de-nabucodonosor-desprecio-por-el-espuxedritu-conversiuxf3n-restauraciuxf3n}}

\bibverse{25} Que te echarán de entre los hombres, y con las bestias del
campo será tu morada, y con hierba del campo te apacentarán como á los
bueyes, y con rocío del cielo serás bañado; y siete tiempos pasarán
sobre ti, hasta que entiendas que el Altísimo se enseñorea en el reino
de los hombres, y que á quien él quisiere lo dará. \bibverse{26} Y lo
que dijeron, que dejasen en la tierra la cepa de las raíces del mismo
árbol, significa que tu reino se te quedará firme, luego que entiendas
que el señorío es en los cielos. \bibverse{27} Por tanto, oh rey,
aprueba mi consejo, y redime tus pecados con justicia, y tus iniquidades
con misericordias para con los pobres; que tal vez será eso una
prolongación de tu tranquilidad.

\bibverse{28} Todo aquesto vino sobre el rey Nabucodonosor.
\bibverse{29} A cabo de doce meses, andándose paseando sobre el palacio
del reino de Babilonia, \bibverse{30} Habló el rey, y dijo: ¿No es ésta
la gran Babilonia, que yo edifiqué para casa del reino, con la fuerza de
mi poder, y para gloria de mi grandeza? \footnote{\textbf{4:30} Prov
  16,18; Hech 12,23}

\bibverse{31} Aun estaba la palabra en la boca del rey, cuando cae una
voz del cielo: A ti dicen, rey Nabucodonosor; el reino es traspasado de
ti: \footnote{\textbf{4:34} Dan 3,33} \bibverse{32} Y de entre los
hombres te echan, y con las bestias del campo será tu morada, y como á
los bueyes te apacentarán: y siete tiempos pasarán sobre ti, hasta que
conozcas que el Altísimo se enseñorea en el reino de los hombres, y á
quien él quisiere lo da.

\bibverse{33} En la misma hora se cumplió la palabra sobre
Nabucodonosor, y fué echado de entre los hombres; y comía hierba como
los bueyes, y su cuerpo se bañaba con el rocío del cielo, hasta que su
pelo creció como de águila, y sus uñas como de aves.

\bibverse{34} Mas al fin del tiempo yo Nabucodonosor alcé mis ojos al
cielo, y mi sentido me fué vuelto; y bendije al Altísimo, y alabé y
glorifiqué al que vive para siempre; porque su señorío es sempiterno, y
su reino por todas las edades. \footnote{\textbf{5:1} Dan 7,1}
\bibverse{35} Y todos los moradores de la tierra por nada son contados:
y en el ejército del cielo, y en los habitantes de la tierra, hace según
su voluntad: ni hay quien estorbe su mano, y le diga: ¿Qué haces?

\bibverse{36} En el mismo tiempo mi sentido me fué vuelto, y la majestad
de mi reino, mi dignidad y mi grandeza volvieron á mí, y mis
gobernadores y mis grandes me buscaron; y fuí restituído á mi reino, y
mayor grandeza me fué añadida.

\hypertarget{el-decreto-termina-con-alabanza-por-la-grandeza-de-dios}{%
\subsection{El decreto termina con alabanza por la grandeza de
Dios}\label{el-decreto-termina-con-alabanza-por-la-grandeza-de-dios}}

\bibverse{37} Ahora yo Nabucodonosor alabo, engrandezco y glorifico al
Rey del cielo, porque todas sus obras son verdad, y sus caminos juicio;
y humillar puede á los que andan con soberbia.

\hypertarget{belsasar-consagra-los-vasos-del-templo-de-los-juduxedos}{%
\subsection{Belsasar consagra los vasos del templo de los
judíos}\label{belsasar-consagra-los-vasos-del-templo-de-los-juduxedos}}

\hypertarget{section-4}{%
\section{5}\label{section-4}}

\bibverse{1} El rey Belsasar hizo un gran banquete á mil de sus
príncipes, y en presencia de los mil bebía vino. \bibverse{2} Belsasar,
con el gusto del vino, mandó que trajesen los vasos de oro y de plata
que Nabucodonosor su padre había traído del templo de Jerusalem; para
que bebiesen con ellos el rey y sus príncipes, sus mujeres y sus
concubinas. \footnote{\textbf{5:2} Dan 1,2; 2Cró 36,10} \bibverse{3}
Entonces fueron traídos los vasos de oro que habían traído del templo de
la casa de Dios que estaba en Jerusalem, y bebieron con ellos el rey y
sus príncipes, sus mujeres y sus concubinas. \bibverse{4} Bebieron vino,
y alabaron á los dioses de oro y de plata, de metal, de hierro, de
madera, y de piedra.

\hypertarget{aparece-la-enigmuxe1tica-inscripciuxf3n-que-ninguxfan-sabio-puede-interpretar-por-consejo-de-la-reina-madre-traen-a-daniel}{%
\subsection{Aparece la enigmática inscripción, que ningún sabio puede
interpretar; por consejo de la reina madre, traen a
Daniel}\label{aparece-la-enigmuxe1tica-inscripciuxf3n-que-ninguxfan-sabio-puede-interpretar-por-consejo-de-la-reina-madre-traen-a-daniel}}

\bibverse{5} En aquella misma hora salieron unos dedos de mano de
hombre, y escribían delante del candelero sobre lo encalado de la pared
del palacio real, y el rey veía la palma de la mano que escribía.
\bibverse{6} Entonces el rey se demudó de su color, y sus pensamientos
lo turbaron, y desatáronse las ceñiduras de sus lomos, y sus rodillas se
batían la una con la otra.

\bibverse{7} El rey clamó en alta voz que hiciesen venir magos, Caldeos,
y adivinos. Habló el rey, y dijo á los sabios de Babilonia: Cualquiera
que leyere esta escritura, y me mostrare su declaración, será vestido de
púrpura, y tendrá collar de oro á su cuello; y en el reino se
enseñoreará el tercero. \footnote{\textbf{5:7} Dan 2,2; Dan 4,3}

\bibverse{8} Entonces fueron introducidos todos los sabios del rey, y no
pudieron leer la escritura, ni mostrar al rey su declaración.
\bibverse{9} Entonces el rey Belsasar fué muy turbado, y se le mudaron
sus colores y alteráronse sus príncipes.

\bibverse{10} La reina, por las palabras del rey y de sus príncipes,
entró á la sala del banquete. Y habló la reina, y dijo: Rey, para
siempre vive, no te asombren tus pensamientos, ni tus colores se
demuden: \bibverse{11} En tu reino hay un varón, en el cual mora el
espíritu de los dioses santos; y en los días de tu padre se halló en él
luz é inteligencia y sabiduría, como ciencia de los dioses: al cual el
rey Nabucodonosor, tu padre, el rey tu padre constituyó príncipe sobre
todos los magos, astrólogos, Caldeos, y adivinos: \bibverse{12} Por
cuanto fué hallado en él mayor espíritu, y ciencia, y entendimiento,
interpretando sueños, y declarando preguntas, y deshaciendo dudas, es á
saber, en Daniel; al cual el rey puso por nombre Beltsasar. Llámese pues
ahora á Daniel, y él mostrará la declaración. \footnote{\textbf{5:12}
  Ezeq 28,3}

\hypertarget{las-brillantes-promesas-del-rey-a-daniel-su-interpretaciuxf3n-del-guiuxf3n-fantasma-su-discurso-de-castigo-y-el-anuncio-de-la-desgracia}{%
\subsection{Las brillantes promesas del rey a Daniel; su interpretación
del guión fantasma, su discurso de castigo y el anuncio de la
desgracia}\label{las-brillantes-promesas-del-rey-a-daniel-su-interpretaciuxf3n-del-guiuxf3n-fantasma-su-discurso-de-castigo-y-el-anuncio-de-la-desgracia}}

\bibverse{13} Entonces Daniel fué traído delante del rey. Y habló el
rey, y dijo á Daniel: ¿Eres tú aquel Daniel de los hijos de la
cautividad de Judá, que mi padre trajo de Judea? \bibverse{14} Yo he
oído de ti que el espíritu de los dioses santos está en ti, y que en ti
se halló luz, y entendimiento y mayor sabiduría. \bibverse{15} Y ahora
fueron traídos delante de mí, sabios, astrólogos, que leyesen esta
escritura, y me mostrasen su interpretación: pero no han podido mostrar
la declaración del negocio. \bibverse{16} Yo pues he oído de ti que
puedes declarar las dudas, y desatar dificultades. Si ahora pudieres
leer esta escritura, y mostrarme su interpretación, serás vestido de
púrpura, y collar de oro tendrás en tu cuello, y en el reino serás el
tercer señor.

\bibverse{17} Entonces Daniel respondió, y dijo delante del rey: Tus
dones sean para ti, y tus presentes dalos á otro. La escritura yo la
leeré al rey, y le mostraré la declaración.

\bibverse{18} El altísimo Dios, oh rey, dió á Nabucodonosor tu padre el
reino, y la grandeza, y la gloria, y la honra: \bibverse{19} Y por la
grandeza que le dió, todos los pueblos, naciones, y lenguas, temblaban y
temían delante de él. Los que él quería mataba, y daba vida á los que
quería: engrandecía á los que quería, y á los que quería humillaba.
\bibverse{20} Mas cuando su corazón se ensoberbeció, y su espíritu se
endureció en altivez, fué depuesto del trono de su reino, y traspasaron
de él la gloria: \footnote{\textbf{5:20} Hech 12,23} \bibverse{21} Y fué
echado de entre los hijos de los hombres; y su corazón fué puesto con
las bestias, y con los asnos monteses fué su morada. Hierba le hicieron
comer, como á buey, y su cuerpo fué bañado con el rocío del cielo, hasta
que conoció que el altísimo Dios se enseñorea del reino de los hombres,
y que pondrá sobre él al que quisiere.

\bibverse{22} Y tú, su hijo Belsasar, no has humillado tu corazón,
sabiendo todo esto: \bibverse{23} Antes contra el Señor del cielo te has
ensoberbecido, é hiciste traer delante de ti los vasos de su casa, y tú
y tus príncipes, tus mujeres y tus concubinas, bebisteis vino en ellos:
demás de esto, á dioses de plata y de oro, de metal, de hierro, de
madera, y de piedra, que ni ven, ni oyen, ni saben, diste alabanza: y al
Dios en cuya mano está tu vida, y cuyos son todos tus caminos, nunca
honraste. \bibverse{24} Entonces de su presencia fué enviada la palma de
la mano que esculpió esta escritura.

\bibverse{25} Y la escritura que esculpió es: MENE, MENE, TEKEL,
UPHARSIN.

\bibverse{26} La declaración del negocio es: MENE: Contó Dios tu reino,
y halo rematado. \bibverse{27} TEKEL: Pesado has sido en balanza, y
fuiste hallado falto. \bibverse{28} PERES: Tu reino fué rompido, y es
dado á Medos y Persas.

\hypertarget{el-honor-de-daniel-final-violento-de-belsasar-y-su-imperio}{%
\subsection{El honor de Daniel; final violento de Belsasar y su
imperio}\label{el-honor-de-daniel-final-violento-de-belsasar-y-su-imperio}}

\bibverse{29} Entonces, mandándolo Belsasar, vistieron á Daniel de
púrpura, y en su cuello fué puesto un collar de oro, y pregonaron de él
que fuese el tercer señor en el reino. \footnote{\textbf{5:29} Dan 2,48;
  Gén 41,42-43}

\bibverse{30} La misma noche fué muerto Belsasar, rey de los Caldeos.
\bibverse{31} Y Darío de Media tomó el reino, siendo de sesenta y dos
años.

\hypertarget{el-levantamiento-de-daniel-durante-la-reorganizaciuxf3n-de-la-administraciuxf3n-del-reich-por-daruxedo-envidia-de-sus-compauxf1eros-funcionarios}{%
\subsection{El levantamiento de Daniel durante la reorganización de la
administración del Reich por Darío; Envidia de sus compañeros
funcionarios}\label{el-levantamiento-de-daniel-durante-la-reorganizaciuxf3n-de-la-administraciuxf3n-del-reich-por-daruxedo-envidia-de-sus-compauxf1eros-funcionarios}}

\hypertarget{section-5}{%
\section{6}\label{section-5}}

\bibverse{1} Pareció bien á Darío constituir sobre el reino ciento
veinte gobernadores, que estuviesen en todo el reino. \bibverse{2} Y
sobre ellos tres presidentes, de los cuales Daniel era el uno, á quienes
estos gobernadores diesen cuenta, porque el rey no recibiese daño.
\bibverse{3} Pero el mismo Daniel era superior á estos gobernadores y
presidentes, porque había en él más abundancia de espíritu: y el rey
pensaba de ponerlo sobre todo el reino. \footnote{\textbf{6:3} Dan 5,12}

\bibverse{4} Entonces los presidentes y gobernadores buscaban ocasiones
contra Daniel por parte del reino; mas no podían hallar alguna ocasión ó
falta, porque él era fiel, y ningún vicio ni falta fué en él hallado.
\bibverse{5} Entonces dijeron aquellos hombres: No hallaremos contra
este Daniel ocasión alguna, si no la hallamos contra él en la ley de su
Dios.

\bibverse{6} Entonces estos gobernadores y presidentes se juntaron
delante del rey, y le dijeron así: Rey Darío, para siempre vive:

\hypertarget{los-funcionarios-celosos-obtienen-un-decreto-real-sobre-un-ejercicio-de-oraciuxf3n-uxfanico-en-el-reino}{%
\subsection{Los funcionarios celosos obtienen un decreto real sobre un
ejercicio de oración único en el
reino}\label{los-funcionarios-celosos-obtienen-un-decreto-real-sobre-un-ejercicio-de-oraciuxf3n-uxfanico-en-el-reino}}

\bibverse{7} Todos los presidentes del reino, magistrados, gobernadores,
grandes, y capitanes, han acordado por consejo promulgar un real edicto,
y confirmarlo, que cualquiera que demandare petición de cualquier dios ú
hombre en el espacio de treinta días, sino de ti, oh rey, sea echado en
el foso de los leones. \bibverse{8} Ahora, oh rey, confirma el edicto, y
firma la escritura, para que no se pueda mudar, conforme á la ley de
Media y de Persia, la cual no se revoca. \footnote{\textbf{6:8} Dan
  6,16; Est 1,19; Est 8,8} \bibverse{9} Firmó pues el rey Darío la
escritura y el edicto.

\bibverse{10} Y Daniel, cuando supo que la escritura estaba firmada,
entróse en su casa, y abiertas las ventanas de su cámara que estaban
hacia Jerusalem, hincábase de rodillas tres veces al día, y oraba, y
confesaba delante de su Dios, como lo solía hacer antes.

\hypertarget{la-transgresiuxf3n-del-edicto-de-daniel-como-resultado-de-su-temor-de-dios-su-condena-a-pesar-del-dolor-del-rey}{%
\subsection{La transgresión del edicto de Daniel como resultado de su
temor de Dios; su condena a pesar del dolor del
rey}\label{la-transgresiuxf3n-del-edicto-de-daniel-como-resultado-de-su-temor-de-dios-su-condena-a-pesar-del-dolor-del-rey}}

\bibverse{11} Entonces se juntaron aquellos hombres, y hallaron á Daniel
orando y rogando delante de su Dios. \bibverse{12} Llegáronse luego, y
hablaron delante del rey acerca del edicto real: ¿No has confirmado
edicto que cualquiera que pidiere á cualquier dios ú hombre en el
espacio de treinta días, excepto á ti, oh rey, fuese echado en el foso
de los leones? Respondió el rey y dijo: Verdad es, conforme á la ley de
Media y de Persia, la cual no se abroga. \footnote{\textbf{6:12} Dan
  3,10}

\bibverse{13} Entonces respondieron y dijeron delante del rey: Daniel
que es de los hijos de la cautividad de los Judíos, no ha hecho cuenta
de ti, oh rey, ni del edicto que confirmaste; antes tres veces al día
hace su petición. \bibverse{14} El rey entonces, oyendo el negocio,
pesóle en gran manera, y sobre Daniel puso cuidado para librarlo; y
hasta puestas del sol trabajó para librarle.

\bibverse{15} Empero aquellos hombres se reunieron cerca del rey, y
dijeron al rey: Sepas, oh rey, que es ley de Media y de Persia, que
ningún decreto ú ordenanza que el rey confirmare pueda mudarse.

\bibverse{16} Entonces el rey mandó, y trajeron á Daniel, y echáronle en
el foso de los leones. Y hablando el rey dijo á Daniel: El Dios tuyo, á
quien tú continuamente sirves, él te libre.

\bibverse{17} Y fué traída una piedra, y puesta sobre la puerta del
foso, la cual selló el rey con su anillo, y con el anillo de sus
príncipes, porque el acuerdo acerca de Daniel no se mudase.
\bibverse{18} Fuése luego el rey á su palacio, y acostóse ayuno; ni
instrumentos de música fueron traídos delante de él, y se le fué el
sueño.

\hypertarget{el-rescate-de-daniel-la-alegruxeda-y-la-gracia-del-rey-castigo-de-los-envidiosos-culpables}{%
\subsection{El rescate de Daniel; la alegría y la gracia del rey;
Castigo de los envidiosos
culpables}\label{el-rescate-de-daniel-la-alegruxeda-y-la-gracia-del-rey-castigo-de-los-envidiosos-culpables}}

\bibverse{19} El rey, por tanto, se levantó muy de mañana, y fué apriesa
al foso de los leones: \bibverse{20} Y llegándose cerca del foso llamó á
voces á Daniel con voz triste: y hablando el rey dijo á Daniel: Daniel,
siervo del Dios viviente, el Dios tuyo, á quien tú continuamente sirves
¿te ha podido librar de los leones? \footnote{\textbf{6:20} Dan 3,17}

\bibverse{21} Entonces habló Daniel con el rey: oh rey, para siempre
vive. \footnote{\textbf{6:21} Dan 6,7} \bibverse{22} El Dios mío envió
su ángel, el cual cerró la boca de los leones, para que no me hiciesen
mal: porque delante de él se halló en mí justicia: y aun delante de ti,
oh rey, yo no he hecho lo que no debiese. \footnote{\textbf{6:22} Dan
  3,28; Heb 11,33}

\bibverse{23} Entonces se alegró el rey en gran manera á causa de él, y
mandó sacar á Daniel del foso: y fué Daniel sacado del foso, y ninguna
lesión se halló en él, porque creyó en su Dios. \footnote{\textbf{6:23}
  Sal 37,40}

\bibverse{24} Y mandándolo el rey fueron traídos aquellos hombres que
habían acusado á Daniel, y fueron echados en el foso de los leones,
ellos, sus hijos, y sus mujeres; y aun no habían llegado al suelo del
foso, cuando los leones se apoderaron de ellos, y quebrantaron todos sus
huesos.

\hypertarget{reconocimiento-de-la-grandeza-del-dios-de-los-juduxedos-por-un-nuevo-edicto-real-daniel-permanece-en-una-posiciuxf3n-alta}{%
\subsection{Reconocimiento de la grandeza del Dios de los judíos por un
nuevo edicto real; Daniel permanece en una posición
alta}\label{reconocimiento-de-la-grandeza-del-dios-de-los-juduxedos-por-un-nuevo-edicto-real-daniel-permanece-en-una-posiciuxf3n-alta}}

\bibverse{25} Entonces el rey Darío escribió á todos los pueblos,
naciones, y lenguas, que habitan en toda la tierra: Paz os sea
multiplicada:

\bibverse{26} De parte mía es puesta ordenanza, que en todo el señorío
de mi reino todos teman y tiemblen de la presencia del Dios de Daniel:
porque él es el Dios viviente y permanente por todos los siglos, y su
reino tal que no será deshecho, y su señorío hasta el fin. \footnote{\textbf{6:26}
  Dan 3,33} \bibverse{27} Que salva y libra, y hace señales y maravillas
en el cielo y en la tierra; el cual libró á Daniel del poder de los
leones.

\bibverse{28} Y este Daniel fué prosperado durante el reinado de Darío,
y durante el reinado de Ciro, Persa.

\hypertarget{daniels-traum-von-dem-erscheinen-eines-luxf6wen-eines-buxe4ren-eines-panthers-eines-furchtbaren-tieres-mit-zehn-huxf6rnern-sowie-eines-kleinen-horns}{%
\subsection{Daniels Traum von dem Erscheinen eines Löwen, eines Bären,
eines Panthers, eines furchtbaren Tieres mit zehn Hörnern sowie eines
kleinen
Horns}\label{daniels-traum-von-dem-erscheinen-eines-luxf6wen-eines-buxe4ren-eines-panthers-eines-furchtbaren-tieres-mit-zehn-huxf6rnern-sowie-eines-kleinen-horns}}

\hypertarget{section-6}{%
\section{7}\label{section-6}}

\bibverse{1} En el primer año de Belsasar rey de Babilonia, vió Daniel
un sueño y visiones de su cabeza en su cama: luego escribió el sueño, y
notó la suma de los negocios. \footnote{\textbf{7:1} Dan 5,1}

\bibverse{2} Habló Daniel y dijo: Veía yo en mi visión de noche, y he
aquí que los cuatro vientos del cielo combatían en la gran mar.
\footnote{\textbf{7:2} Apoc 17,15} \bibverse{3} Y cuatro bestias
grandes, diferentes la una de la otra, subían de la mar. \footnote{\textbf{7:3}
  Apoc 13,1-2}

\bibverse{4} La primera era como león, y tenía alas de águila. Yo estaba
mirando hasta tanto que sus alas fueron arrancadas, y fué quitada de la
tierra; y púsose enhiesta sobre los pies á manera de hombre, y fuéle
dado corazón de hombre. \footnote{\textbf{7:4} Dan 4,31}

\bibverse{5} Y he aquí otra segunda bestia, semejante á un oso, la cual
se puso al un lado, y tenía en su boca tres costillas entre sus dientes;
y fuéle dicho así: Levántate, traga carne mucha.

\bibverse{6} Después de esto yo miraba, y he aquí otra, semejante á un
tigre, y tenía cuatro alas de ave en sus espaldas: tenía también esta
bestia cuatro cabezas; y fuéle dada potestad.

\bibverse{7} Después de esto miraba yo en las visiones de la noche, y he
aquí la cuarta bestia, espantosa y terrible, y en grande manera fuerte;
la cual tenía unos dientes grandes de hierro: devoraba y desmenuzaba, y
las sobras hollaba con sus pies: y era muy diferente de todas las
bestias que habían sido antes de ella, y tenía diez cuernos.

\bibverse{8} Estando yo contemplando los cuernos, he aquí que otro
cuerno pequeño subía entre ellos, y delante de él fueron arrancados tres
cuernos de los primeros; y he aquí, en este cuerno había ojos como ojos
de hombre, y una boca que hablaba grandezas. \footnote{\textbf{7:8} Dan
  11,36}

\hypertarget{sesiuxf3n-de-la-corte-en-el-cielo-presidida-por-un-anciano-en-la-gloria-de-la-luz-decisiuxf3n-de-destruir-la-cuarta-bestia-y-derrocar-a-las-tres-primeras-bestias-transferencia-del-dominio-eterno-al-hijo-del-hombre}{%
\subsection{Sesión de la corte en el cielo presidida por un anciano en
la gloria de la luz; Decisión de destruir la cuarta bestia y derrocar a
las tres primeras bestias; Transferencia del dominio eterno al Hijo del
Hombre}\label{sesiuxf3n-de-la-corte-en-el-cielo-presidida-por-un-anciano-en-la-gloria-de-la-luz-decisiuxf3n-de-destruir-la-cuarta-bestia-y-derrocar-a-las-tres-primeras-bestias-transferencia-del-dominio-eterno-al-hijo-del-hombre}}

\bibverse{9} Estuve mirando hasta que fueron puestas sillas: y un
Anciano de grande edad se sentó, cuyo vestido era blanco como la nieve,
y el pelo de su cabeza como lana limpia; su silla llama de fuego, sus
ruedas fuego ardiente. \footnote{\textbf{7:9} Sal 90,2} \bibverse{10} Un
río de fuego procedía y salía de delante de él: millares de millares le
servían, y millones de millones asistían delante de él: el Juez se
sentó, y los libros se abrieron. \footnote{\textbf{7:10} Sal 68,18; Apoc
  5,11}

\bibverse{11} Yo entonces miraba á causa de la voz de las grandes
palabras que hablaba el cuerno; miraba hasta tanto que mataron la
bestia, y su cuerpo fué deshecho, y entregado para ser quemado en el
fuego. \footnote{\textbf{7:11} Apoc 19,20} \bibverse{12} Habían también
quitado á las otras bestias su señorío, y les había sido dada
prolongación de vida hasta cierto tiempo. \footnote{\textbf{7:12} Dan
  2,21}

\bibverse{13} Miraba yo en la visión de la noche, y he aquí en las nubes
del cielo como un hijo de hombre que venía, y llegó hasta el Anciano de
grande edad, é hiciéronle llegar delante de él. \footnote{\textbf{7:13}
  Luc 21,27} \bibverse{14} Y fuéle dado señorío, y gloria, y reino; y
todos los pueblos, naciones y lenguas le sirvieron; su señorío, señorío
eterno, que no será transitorio, y su reino que no se corromperá.

\hypertarget{a-peticiuxf3n-suya-daniel-recibe-informaciuxf3n-de-alguien-que-estuxe1-alluxed-sobre-los-cuatro-imperios-mundiales-especialmente-sobre-la-destrucciuxf3n-del-cuarto-reino-y-el-establecimiento-del-reino-mesiuxe1nico}{%
\subsection{A petición suya, Daniel recibe información de alguien que
está allí sobre los cuatro imperios mundiales, especialmente sobre la
destrucción del cuarto reino y el establecimiento del reino
mesiánico}\label{a-peticiuxf3n-suya-daniel-recibe-informaciuxf3n-de-alguien-que-estuxe1-alluxed-sobre-los-cuatro-imperios-mundiales-especialmente-sobre-la-destrucciuxf3n-del-cuarto-reino-y-el-establecimiento-del-reino-mesiuxe1nico}}

\bibverse{15} Mi espíritu fué turbado, yo Daniel, en medio de mi cuerpo,
y las visiones de mi cabeza me asombraron. \bibverse{16} Lleguéme á uno
de los que asistían, y preguntéle la verdad acerca de todo esto. Y
hablóme, y declaróme la interpretación de las cosas. \footnote{\textbf{7:16}
  Dan 7,10}

\bibverse{17} Estas grandes bestias, las cuales son cuatro, cuatro reyes
son, que se levantarán en la tierra. \bibverse{18} Después tomarán el
reino los santos del Altísimo, y poseerán el reino hasta el siglo, y
hasta el siglo de los siglos.

\bibverse{19} Entonces tuve deseo de saber la verdad acerca de la cuarta
bestia, que tan diferente era de todas las otras, espantosa en gran
manera, que tenía dientes de hierro, y sus uñas de metal, que devoraba y
desmenuzaba, y las sobras hollaba con sus pies: \footnote{\textbf{7:19}
  Dan 7,7} \bibverse{20} Asimismo acerca de los diez cuernos que tenía
en su cabeza, y del otro que había subido, de delante del cual habían
caído tres: y este mismo cuerno tenía ojos, y boca que hablaba
grandezas, y su parecer mayor que el de sus compañeros. \bibverse{21} Y
veía yo que este cuerno hacía guerra contra los santos, y los vencía,
\bibverse{22} Hasta tanto que vino el Anciano de grande edad, y se dió
el juicio á los santos del Altísimo; y vino el tiempo, y los santos
poseyeron el reino.

\bibverse{23} Dijo así: La cuarta bestia será un cuarto reino en la
tierra, el cual será más grande que todos los otros reinos, y á toda la
tierra devorará, y la hollará, y la despedazará. \bibverse{24} Y los
diez cuernos significan que de aquel reino se levantarán diez reyes; y
tras ellos se levantará otro, el cual será mayor que los primeros, y á
tres reyes derribará. \footnote{\textbf{7:24} Apoc 17,12} \bibverse{25}
Y hablará palabras contra el Altísimo, y á los santos del Altísimo
quebrantará, y pensará en mudar los tiempos y la ley: y entregados serán
en su mano hasta tiempo, y tiempos, y el medio de un tiempo. \footnote{\textbf{7:25}
  Apoc 13,5-6; Dan 12,7; Dan 4,13}

\bibverse{26} Empero se sentará el juez, y quitaránle su señorío, para
que sea destruído y arruinado hasta el extremo; \bibverse{27} Y que el
reino, y el señorío, y la majestad de los reinos debajo de todo el
cielo, sea dado al pueblo de los santos del Altísimo; cuyo reino es
reino eterno, y todos los señoríos le servirán y obedecerán.

\hypertarget{impresiuxf3n-de-lo-que-se-vio-en-daniel}{%
\subsection{Impresión de lo que se vio en
Daniel}\label{impresiuxf3n-de-lo-que-se-vio-en-daniel}}

\bibverse{28} Hasta aquí fué el fin de la plática. Yo Daniel, mucho me
turbaron mis pensamientos, y mi rostro se me mudó: mas guardé en mi
corazón el negocio.

\hypertarget{escena-de-la-cara-del-sueuxf1o-lucha-del-carnero-de-cuernos-desiguales-persa-y-el-macho-cabruxedo-de-un-cuerno-griego-victoria-y-fortalecimiento-de-este-uxfaltimo}{%
\subsection{Escena de la cara del sueño; Lucha del carnero de cuernos
desiguales (persa) y el macho cabrío de un cuerno (griego); Victoria y
fortalecimiento de este
último}\label{escena-de-la-cara-del-sueuxf1o-lucha-del-carnero-de-cuernos-desiguales-persa-y-el-macho-cabruxedo-de-un-cuerno-griego-victoria-y-fortalecimiento-de-este-uxfaltimo}}

\hypertarget{section-7}{%
\section{8}\label{section-7}}

\bibverse{1} En el año tercero del reinado del rey Belsasar, me apareció
una visión á mí, Daniel, después de aquella que me había aparecido
antes. \bibverse{2} Vi en visión, (y aconteció cuando vi, que yo estaba
en Susán, que es cabecera del reino en la provincia de Persia;) vi pues
en visión, estando junto al río Ulai, \bibverse{3} Y alcé mis ojos, y
miré, y he aquí un carnero que estaba delante del río, el cual tenía dos
cuernos: y aunque eran altos, el uno era más alto que el otro; y el más
alto subió á la postre. \bibverse{4} Vi que el carnero hería con los
cuernos al poniente, al norte, y al mediodía, y que ninguna bestia podía
parar delante de él, ni había quien escapase de su mano: y hacía
conforme á su voluntad, y engrandecíase.

\bibverse{5} Y estando yo considerando, he aquí un macho de cabrío venía
de la parte del poniente sobre la haz de toda la tierra, el cual no
tocaba la tierra: y tenía aquel macho de cabrío un cuerno notable entre
sus ojos: \bibverse{6} Y vino hasta el carnero que tenía los dos
cuernos, al cual había yo visto que estaba delante del río, y corrió
contra él con la ira de su fortaleza. \bibverse{7} Y vilo que llegó
junto al carnero, y levantóse contra él, é hiriólo, y quebró sus dos
cuernos, porque en el carnero no había fuerzas para parar delante de él:
derribólo por tanto en tierra, y hollólo; ni hubo quien librase al
carnero de su mano. \bibverse{8} Y engrandecióse en gran manera el macho
de cabrío; y estando en su mayor fuerza, aquel gran cuerno fué quebrado,
y en su lugar subieron otros cuatro maravillosos hacia los cuatro
vientos del cielo. \footnote{\textbf{8:8} Dan 7,6; Dan 11,4}

\hypertarget{buen-humor-y-ultraje-religioso-del-cuerno-pequeuxf1o-que-creciuxf3-en-uno-de-los-cuatro-cuernos-de-la-cabra}{%
\subsection{Buen humor y ultraje religioso del cuerno pequeño que creció
en uno de los cuatro cuernos de la
cabra}\label{buen-humor-y-ultraje-religioso-del-cuerno-pequeuxf1o-que-creciuxf3-en-uno-de-los-cuatro-cuernos-de-la-cabra}}

\bibverse{9} Y del uno de ellos salió un cuerno pequeño, el cual creció
mucho al mediodía, y al oriente, y hacia la tierra deseable. \footnote{\textbf{8:9}
  Dan 7,8; Dan 11,16} \bibverse{10} Y engrandecióse hasta el ejército
del cielo; y parte del ejército y de las estrellas echó por tierra, y
las holló. \bibverse{11} Aun contra el príncipe de la fortaleza se
engrandeció, y por él fué quitado el continuo sacrificio, y el lugar de
su santuario fué echado por tierra. \footnote{\textbf{8:11} Dan 11,31}
\bibverse{12} Y el ejército fuéle entregado á causa de la prevaricación
sobre el continuo sacrificio: y echó por tierra la verdad, é hizo cuanto
quiso, y sucedióle prósperamente.

\hypertarget{revelaciuxf3n-del-uxe1ngel-mensajero-que-la-deshonra-religiosa-del-cuerno-pequeuxf1o-duraruxe1-1150-duxedas}{%
\subsection{Revelación del ángel mensajero que la deshonra religiosa del
cuerno pequeño durará 1150
días}\label{revelaciuxf3n-del-uxe1ngel-mensajero-que-la-deshonra-religiosa-del-cuerno-pequeuxf1o-duraruxe1-1150-duxedas}}

\bibverse{13} Y oí un santo que hablaba; y otro de los santos dijo á
aquél que hablaba: ¿Hasta cuándo durará la visión del continuo
sacrificio, y la prevaricación asoladora que pone el santuario y el
ejército para ser hollados?

\bibverse{14} Y él me dijo: Hasta dos mil y trescientos días de tarde y
mañana; y el santuario será purificado.

\hypertarget{gabriel-el-arcuxe1ngel-en-forma-humana-muestra-el-rostro-de-daniel-y-anuncia-los-perversos-acontecimientos-del-uxfaltimo-rey-griego}{%
\subsection{Gabriel, el arcángel en forma humana, muestra el rostro de
Daniel y anuncia los perversos acontecimientos del último rey
griego}\label{gabriel-el-arcuxe1ngel-en-forma-humana-muestra-el-rostro-de-daniel-y-anuncia-los-perversos-acontecimientos-del-uxfaltimo-rey-griego}}

\bibverse{15} Y acaeció que estando yo Daniel considerando la visión, y
buscando su inteligencia, he aquí, como una semejanza de hombre se puso
delante de mí. \bibverse{16} Y oí una voz de hombre entre las riberas de
Ulai, que gritó y dijo: Gabriel, enseña la visión á éste.

\bibverse{17} Vino luego cerca de donde yo estaba; y con su venida me
asombré, y caí sobre mi rostro. Empero él me dijo: Entiende, hijo del
hombre, porque al tiempo se cumplirá la visión. \footnote{\textbf{8:17}
  Dan 10,9}

\bibverse{18} Y estando él hablando conmigo, caí dormido en tierra sobre
mi rostro: y él me tocó, é hízome estar en pie.

\bibverse{19} Y dijo: He aquí yo te enseñaré lo que ha de venir en el
fin de la ira: porque al tiempo se cumplirá: \bibverse{20} Aquel carnero
que viste, que tenía cuernos, son los reyes de Media y de Persia.
\bibverse{21} Y el macho cabrío es el rey de Javán: y el cuerno grande
que tenía entre sus ojos es el rey primero. \bibverse{22} Y que fué
quebrado y sucedieron cuatro en su lugar, significa que cuatro reinos
sucederán de la nación, mas no en la fortaleza de él.

\bibverse{23} Y al cabo del imperio de éstos, cuando se cumplirán los
prevaricadores, levantaráse un rey altivo de rostro, y entendido en
dudas. \bibverse{24} Y su poder se fortalecerá, mas no con fuerza suya;
y destruirá maravillosamente, y prosperará; y hará arbitrariamente, y
destruirá fuertes y al pueblo de los santos. \bibverse{25} Y con su
sagacidad hará prosperar el engaño en su mano; y en su corazón se
engrandecerá, y con paz destruirá á muchos: y contra el príncipe de los
príncipes se levantará; mas sin mano será quebrantado.

\bibverse{26} Y la visión de la tarde y la mañana que está dicha, es
verdadera: y tú guarda la visión, porque es para muchos días.
\footnote{\textbf{8:26} Dan 12,4}

\hypertarget{la-consternaciuxf3n-y-la-enfermedad-de-daniel-de-la-cara}{%
\subsection{La consternación y la enfermedad de Daniel de la
cara}\label{la-consternaciuxf3n-y-la-enfermedad-de-daniel-de-la-cara}}

\bibverse{27} Y yo Daniel fuí quebrantado, y estuve enfermo algunos
días: y cuando convalecí, hice el negocio del rey; mas estaba espantado
acerca de la visión, y no había quien la entendiese.

\hypertarget{daniel-preocupado-por-una-profecuxeda-de-jeremuxedas-decide-obtener-informaciuxf3n-de-dios-a-travuxe9s-de-una-oraciuxf3n-solemne}{%
\subsection{Daniel, preocupado por una profecía de Jeremías, decide
obtener información de Dios a través de una oración
solemne}\label{daniel-preocupado-por-una-profecuxeda-de-jeremuxedas-decide-obtener-informaciuxf3n-de-dios-a-travuxe9s-de-una-oraciuxf3n-solemne}}

\hypertarget{section-8}{%
\section{9}\label{section-8}}

\bibverse{1} En el año primero de Darío hijo de Assuero, de la nación de
los Medos, el cual fué puesto por rey sobre el reino de los Caldeos;
\bibverse{2} En el año primero de su reinado, yo Daniel miré atentamente
en los libros el número de los años, del cual habló Jehová al profeta
Jeremías, que había de concluir la asolación de Jerusalem en setenta
años. \footnote{\textbf{9:2} Jer 25,11-12} \bibverse{3} Y volví mi
rostro al Señor Dios, buscándole en oración y ruego, en ayuno, y
cilicio, y ceniza.

\hypertarget{oraciuxf3n-de-daniel-confesiuxf3n-del-pecado-y-solicitud-de-salvaciuxf3n}{%
\subsection{Oración de Daniel, confesión del pecado, y solicitud de
salvación}\label{oraciuxf3n-de-daniel-confesiuxf3n-del-pecado-y-solicitud-de-salvaciuxf3n}}

\bibverse{4} Y oré á Jehová mi Dios, y confesé, y dije: Ahora Señor,
Dios grande, digno de ser temido, que guardas el pacto y la misericordia
con los que te aman y guardan tus mandamientos;

\bibverse{5} Hemos pecado, hemos hecho iniquidad, hemos obrado
impíamente, y hemos sido rebeldes, y nos hemos apartado de tus
mandamientos y de tus juicios. \bibverse{6} No hemos obedecido á tus
siervos los profetas, que en tu nombre hablaron á nuestros reyes, y á
nuestros príncipes, á nuestros padres, y á todo el pueblo de la tierra.

\bibverse{7} Tuya es, Señor, la justicia, y nuestra la confusión de
rostro, como en el día de hoy á todo hombre de Judá, y á los moradores
de Jerusalem, y á todo Israel, á los de cerca y á los de lejos, en todas
las tierras á donde los has echado á causa de su rebelión con que contra
ti se rebelaron. \bibverse{8} Oh Jehová, nuestra es la confusión de
rostro, de nuestros reyes, de nuestros príncipes, y de nuestros padres;
porque contra ti pecamos. \bibverse{9} De Jehová nuestro Dios es el
tener misericordia, y el perdonar, aunque contra él nos hemos rebelado;
\footnote{\textbf{9:9} Sal 130,4} \bibverse{10} Y no obedecimos á la voz
de Jehová nuestro Dios, para andar en sus leyes, las cuales puso él
delante de nosotros por mano de sus siervos los profetas. \bibverse{11}
Y todo Israel traspasó tu ley apartándose para no oir tu voz: por lo
cual ha fluído sobre nosotros la maldición, y el juramento que está
escrito en la ley de Moisés, siervo de Dios; porque contra él pecamos.

\bibverse{12} Y él ha verificado su palabra que habló sobre nosotros, y
sobre nuestros jueces que nos gobernaron, trayendo sobre nosotros tan
grande mal; que nunca fué hecho debajo del cielo como el que fué hecho
en Jerusalem. \bibverse{13} Según está escrito en la ley de Moisés, todo
aqueste mal vino sobre nosotros: y no hemos rogado á la faz de Jehová
nuestro Dios, para convertirnos de nuestras maldades, y entender tu
verdad. \bibverse{14} Veló por tanto Jehová sobre el mal, y trájolo
sobre nosotros; porque justo es Jehová nuestro Dios en todas sus obras
que hizo, porque no obedecimos á su voz. \footnote{\textbf{9:14} Jer
  1,12}

\bibverse{15} Ahora pues, Señor Dios nuestro, que sacaste tu pueblo de
la tierra de Egipto con mano poderosa, y te hiciste nombre cual en este
día; hemos pecado, impíamente hemos hecho. \bibverse{16} Oh Señor, según
todas tus justicias, apártese ahora tu ira y tu furor de sobre tu ciudad
Jerusalem, tu santo monte: porque á causa de nuestros pecados, y por la
maldad de nuestros padres, Jerusalem y tu pueblo dados son en oprobio á
todos en derredor nuestro.

\bibverse{17} Ahora pues, Dios nuestro, oye la oración de tu siervo, y
sus ruegos, y haz que tu rostro resplandezca sobre tu santuario asolado,
por amor del Señor. \bibverse{18} Inclina, oh Dios mío, tu oído, y oye;
abre tus ojos, y mira nuestros asolamientos, y la ciudad sobre la cual
es llamado tu nombre: porque no derramamos nuestros ruegos ante tu
acatamiento confiados en nuestras justicias, sino en tus muchas
miseraciones. \bibverse{19} Oye, Señor; oh Señor, perdona; presta oído,
Señor, y haz; no pongas dilación, por amor de ti mismo, Dios mío: porque
tu nombre es llamado sobre tu ciudad y sobre tu pueblo. \footnote{\textbf{9:19}
  Jer 14,9}

\hypertarget{daniel-recibe-la-informaciuxf3n-deseada-de-gabriel-refiriuxe9ndose-a-las-semanas-de-auxf1os-designadas-por-jeremuxedas}{%
\subsection{Daniel recibe la información deseada de Gabriel refiriéndose
a las ``semanas de años'' designadas por
Jeremías}\label{daniel-recibe-la-informaciuxf3n-deseada-de-gabriel-refiriuxe9ndose-a-las-semanas-de-auxf1os-designadas-por-jeremuxedas}}

\bibverse{20} Aun estaba hablando, y orando, y confesando mi pecado y el
pecado de mi pueblo Israel, y derramaba mi ruego delante de Jehová mi
Dios por el monte santo de mi Dios; \bibverse{21} Aun estaba hablando en
oración, y aquel varón Gabriel, al cual había visto en visión al
principio, volando con presteza, me tocó como á la hora del sacrificio
de la tarde. \bibverse{22} E hízome entender, y habló conmigo, y dijo:
Daniel, ahora he salido para hacerte entender la declaración.
\bibverse{23} Al principio de tus ruegos salió la palabra, y yo he
venido para enseñártela, porque tú eres varón de deseos. Entiende pues
la palabra, y entiende la visión.

\bibverse{24} Setenta semanas están determinadas sobre tu pueblo y sobre
tu santa ciudad, para acabar la prevaricación, y concluir el pecado, y
expiar la iniquidad; y para traer la justicia de los siglos, y sellar la
visión y la profecía, y ungir al Santo de los santos.

\bibverse{25} Sepas pues y entiendas, que desde la salida de la palabra
para restaurar y edificar á Jerusalem hasta el Mesías Príncipe, habrá
siete semanas, y sesenta y dos semanas; tornaráse á edificar la plaza y
el muro en tiempos angustiosos. \bibverse{26} Y después de las sesenta y
dos semanas se quitará la vida al Mesías, y no por sí: y el pueblo de un
príncipe que ha de venir, destruirá á la ciudad y el santuario; con
inundación será el fin de ella, y hasta el fin de la guerra será talada
con asolamientos. \footnote{\textbf{9:26} Luc 21,24} \bibverse{27} Y en
otra semana confirmará el pacto á muchos, y á la mitad de la semana hará
cesar el sacrificio y la ofrenda: después con la muchedumbre de las
abominaciones será el desolar, y esto hasta una entera consumación; y
derramaráse la ya determinada sobre el pueblo asolado. \footnote{\textbf{9:27}
  Dan 12,11; Mat 24,15}

\hypertarget{la-preparaciuxf3n-de-daniel-para-la-visiona-mediante-el-ayuno}{%
\subsection{La preparación de Daniel para la visiona mediante el
ayuno}\label{la-preparaciuxf3n-de-daniel-para-la-visiona-mediante-el-ayuno}}

\hypertarget{section-9}{%
\section{10}\label{section-9}}

\bibverse{1} En el tercer año de Ciro rey de Persia, fué revelada
palabra á Daniel, cuyo nombre era Beltsasar; y la palabra era verdadera,
mas el tiempo fijado era largo: él empero comprendió la palabra, y tuvo
inteligencia en la visión. \footnote{\textbf{10:1} Dan 1,21; Dan 1,7}

\bibverse{2} En aquellos días yo Daniel me contristé por espacio de tres
semanas. \bibverse{3} No comí pan delicado, ni entró carne ni vino en mi
boca, ni me unté con ungüento, hasta que se cumplieron tres semanas de
días.

\hypertarget{la-apariencia-exterior-del-mensajero-celestial-efecto-de-la-apariciuxf3n-en-daniel}{%
\subsection{La apariencia exterior del mensajero celestial; Efecto de la
aparición en
Daniel}\label{la-apariencia-exterior-del-mensajero-celestial-efecto-de-la-apariciuxf3n-en-daniel}}

\bibverse{4} Y á los veinte y cuatro días del mes primero estaba yo á la
orilla del gran río Hiddekel; \bibverse{5} Y alzando mis ojos miré, y he
aquí un varón vestido de lienzos, y ceñidos sus lomos de oro de Uphaz:
\bibverse{6} Y su cuerpo era como piedra de Tarsis, y su rostro parecía
un relámpago, y sus ojos como antorchas de fuego, y sus brazos y sus
pies como de color de metal resplandeciente, y la voz de sus palabras
como la voz de ejército.

\bibverse{7} Y sólo yo, Daniel, vi aquella visión, y no la vieron los
hombres que estaban conmigo; sino que cayó sobre ellos un gran temor, y
huyeron, y escondiéronse. \bibverse{8} Quedé pues yo solo, y vi esta
gran visión, y no quedó en mí esfuerzo; antes mi fuerza se me trocó en
desmayo, sin retener vigor alguno. \bibverse{9} Empero oí la voz de sus
palabras: y oyendo la voz de sus palabras, estaba yo adormecido sobre mi
rostro, y mi rostro en tierra. \footnote{\textbf{10:9} Dan 8,17-18}

\bibverse{10} Y, he aquí, una mano me tocó, é hizo que me moviese sobre
mis rodillas, y sobre las palmas de mis manos.

\hypertarget{mensajes-persuasivos-y-alentadores-del-uxe1ngel}{%
\subsection{Mensajes persuasivos y alentadores del
ángel}\label{mensajes-persuasivos-y-alentadores-del-uxe1ngel}}

\bibverse{11} Y díjome: Daniel, varón de deseos, está atento á las
palabras que te hablaré, y levántate sobre tus pies; porque á ti he sido
enviado ahora. Y estando hablando conmigo esto, yo estaba temblando.

\bibverse{12} Y díjome: Daniel, no temas: porque desde el primer día que
diste tu corazón á entender, y á afligirte en la presencia de tu Dios,
fueron oídas tus palabras; y á causa de tus palabras yo soy venido.
\bibverse{13} Mas el príncipe del reino de Persia se puso contra mí
veintiún días: y he aquí, Miguel, uno de los principales príncipes, vino
para ayudarme, y yo quedé allí con los reyes de Persia. \bibverse{14}
Soy pues venido para hacerte saber lo que ha de venir á tu pueblo en los
postreros días; porque la visión es aún para días. \footnote{\textbf{10:14}
  Dan 9,22}

\hypertarget{daniel-fortalecido-en-su-debilidad-por-el-uxe1ngel-anuncio-de-la-lucha-del-mensajero-celestial-con-los-pruxedncipes-de-persia-y-grecia}{%
\subsection{Daniel fortalecido en su debilidad por el ángel; Anuncio de
la lucha del mensajero celestial con los príncipes de Persia y
Grecia}\label{daniel-fortalecido-en-su-debilidad-por-el-uxe1ngel-anuncio-de-la-lucha-del-mensajero-celestial-con-los-pruxedncipes-de-persia-y-grecia}}

\bibverse{15} Y estando hablando conmigo semejantes palabras, puse mis
ojos en tierra, y enmudecí. \bibverse{16} Mas he aquí, como una
semejanza de hijo de hombre tocó mis labios. Entonces abrí mi boca, y
hablé, y dije á aquel que estaba delante de mí: Señor mío, con la visión
se revolvieron mis dolores sobre mí, y no me quedó fuerza. \bibverse{17}
¿Cómo pues podrá el siervo de mi señor hablar con este mi señor? porque
al instante me faltó la fuerza, y no me ha quedado aliento.

\bibverse{18} Y aquella como semejanza de hombre me tocó otra vez, y me
confortó; \bibverse{19} Y díjome: Varón de deseos, no temas: paz á ti;
ten buen ánimo, y aliéntate. Y hablando él conmigo cobré yo vigor, y
dije: Hable mi señor, porque me has fortalecido. \footnote{\textbf{10:19}
  Apoc 1,17}

\bibverse{20} Y dijo: ¿Sabes por qué he venido á ti? Porque luego tengo
de volver para pelear con el príncipe de los Persas; y en saliendo yo,
luego viene el príncipe de Grecia. \footnote{\textbf{10:20} Dan 10,13}

\bibverse{21} Empero yo te declararé lo que está escrito en la escritura
de verdad: y ninguno hay que se esfuerce conmigo en estas cosas, sino
Miguel vuestro príncipe.

\hypertarget{section-10}{%
\section{11}\label{section-10}}

\bibverse{1} Y en el año primero de Darío el de Media, yo estuve para
animarlo y fortalecerlo.

\bibverse{2} Y ahora yo te mostraré la verdad. He aquí que aun habrá
tres reyes en Persia, y el cuarto se hará de grandes riquezas más que
todos; y fortificándose con sus riquezas, despertará á todos contra el
reino de Javán. \footnote{\textbf{11:2} Dan 10,21} \bibverse{3}
Levantaráse luego un rey valiente, el cual se enseñoreará sobre gran
dominio, y hará su voluntad. \bibverse{4} Pero cuando estará
enseñoreado, será quebrantado su reino, y repartido por los cuatro
vientos del cielo; y no á sus descendientes, ni según el señorío con que
él se enseñoreó: porque su reino será arrancado, y para otros fuera de
aquellos.

\hypertarget{resumen-de-las-batallas-de-los-reyes-egipcios-y-sirios-despuuxe9s-de-la-muerte-de-alejandro-excepto-antuxedoco-epuxedfanes}{%
\subsection{Resumen de las batallas de los reyes egipcios y sirios
después de la muerte de Alejandro, excepto Antíoco
Epífanes}\label{resumen-de-las-batallas-de-los-reyes-egipcios-y-sirios-despuuxe9s-de-la-muerte-de-alejandro-excepto-antuxedoco-epuxedfanes}}

\bibverse{5} Y haráse fuerte el rey del mediodía: mas uno de los
príncipes de aquél le sobrepujará, y se hará poderoso; su señorío será
grande señorío. \bibverse{6} Y al cabo de años se concertarán, y la hija
del rey del mediodía vendrá al rey del norte para hacer los conciertos.
Empero ella no podrá retener la fuerza del brazo: ni permanecerá él, ni
su brazo; porque será entregada ella, y los que la habían traído,
asimismo su hijo, y los que estaban de parte de ella en aquel tiempo.

\bibverse{7} Mas del renuevo de sus raíces se levantará uno sobre su
silla, y vendrá con ejército, y entrará en la fortaleza del rey del
norte, y hará en ellos á su arbitrio, y predominará. \bibverse{8} Y aun
los dioses de ellos, con sus príncipes, con sus vasos preciosos de plata
y de oro, llevará cautivos á Egipto: y por años se mantendrá él contra
el rey del norte. \bibverse{9} Así entrará en el reino el rey del
mediodía, y volverá á su tierra. \bibverse{10} Mas los hijos de aquél se
airarán, y reunirán multitud de grandes ejércitos: y vendrá á gran
priesa, é inundará, y pasará, y tornará, y llegará con ira hasta su
fortaleza.

\bibverse{11} Por lo cual se enfurecerá el rey del mediodía, y saldrá, y
peleará con el mismo rey del norte; y pondrá en campo gran multitud, y
toda aquella multitud será entregada en su mano. \bibverse{12} Y la
multitud se ensoberbecerá, elevaráse su corazón, y derribará muchos
millares; mas no prevalecerá. \bibverse{13} Y el rey del norte volverá á
poner en campo mayor multitud que primero, y á cabo del tiempo de años
vendrá á gran priesa con grande ejército y con muchas riquezas.

\bibverse{14} Y en aquellos tiempos se levantarán muchos contra el rey
del mediodía; é hijos de disipadores de tu pueblo se levantarán para
confirmar la profecía, y caerán. \bibverse{15} Vendrá pues el rey del
norte, y fundará baluartes, y tomará la ciudad fuerte; y los brazos del
mediodía no podrán permanecer, ni su pueblo escogido, ni habrá fortaleza
que pueda resistir. \bibverse{16} Y el que vendrá contra él, hará á su
voluntad, ni habrá quien se le pueda parar delante; y estará en la
tierra deseable, la cual será consumida en su poder. \footnote{\textbf{11:16}
  Dan 8,9} \bibverse{17} Pondrá luego su rostro para venir con el poder
de todo su reino; y hará con aquél cosas rectas, y darále una hija de
mujeres para trastornarla: mas no estará ni será por él. \bibverse{18}
Volverá después su rostro á las islas, y tomará muchas; mas un príncipe
le hará parar su afrenta, y aun tornará sobre él su oprobio.
\bibverse{19} Luego volverá su rostro á las fortalezas de su tierra: mas
tropezará y caerá, y no parecerá más.

\bibverse{20} Entonces sucederá en su silla uno que hará pasar exactor
por la gloria del reino; mas en pocos días será quebrantado, no en
enojo, ni en batalla.

\hypertarget{historia-del-malvado-antiochus-epiphanes}{%
\subsection{Historia del malvado Antiochus
Epiphanes}\label{historia-del-malvado-antiochus-epiphanes}}

\bibverse{21} Y sucederá en su lugar un vil, al cual no darán la honra
del reino: vendrá empero con paz, y tomará el reino con halagos.
\bibverse{22} Y con los brazos de inundación serán inundados delante de
él, y serán quebrantados; y aun también el príncipe del pacto.
\bibverse{23} Y después de los conciertos con él, él hará engaño, y
subirá, y saldrá vencedor con poca gente. \bibverse{24} Estando la
provincia en paz y en abundancia, entrará y hará lo que no hicieron sus
padres, ni los padres de sus padres; presa, y despojos, y riquezas
repartirá á sus soldados; y contra las fortalezas formará sus designios:
y esto por tiempo.

\bibverse{25} Y despertará sus fuerzas y su corazón contra el rey del
mediodía con grande ejército: y el rey del mediodía se moverá á la
guerra con grande y muy fuerte ejército; mas no prevalecerá, porque le
harán traición. \bibverse{26} Aun los que comerán su pan, le
quebrantarán; y su ejército será destruído, y caerán muchos muertos.
\bibverse{27} Y el corazón de estos dos reyes será para hacer mal, y en
una misma mesa tratarán mentira: mas no servirá de nada, porque el plazo
aun no es llegado. \bibverse{28} Y volveráse á su tierra con grande
riqueza, y su corazón será contra el pacto santo: hará pues, y volveráse
á su tierra. \footnote{\textbf{11:28} 1Macc 1,21-29}

\bibverse{29} Al tiempo señalado tornará al mediodía; mas no será la
postrera venida como la primera. \bibverse{30} Porque vendrán contra él
naves de Chîttim, y él se contristará, y se volverá, y enojaráse contra
el pacto santo, y hará: volveráse pues, y pensará en los que habrán
desamparado el santo pacto.

\hypertarget{persecuciuxf3n-de-los-juduxedos-en-jerusaluxe9n}{%
\subsection{Persecución de los judíos en
Jerusalén}\label{persecuciuxf3n-de-los-juduxedos-en-jerusaluxe9n}}

\bibverse{31} Y serán puestos brazos de su parte; y contaminarán el
santuario de fortaleza, y quitarán el continuo sacrificio, y pondrán la
abominación espantosa. \bibverse{32} Y con lisonjas hará pecar á los
violadores del pacto: mas el pueblo que conoce á su Dios, se esforzará,
y hará. \footnote{\textbf{11:32} 1Macc 2,1-6}

\bibverse{33} Y los sabios del pueblo darán sabiduría á muchos: y caerán
á cuchillo y á fuego, en cautividad y despojo, por días. \footnote{\textbf{11:33}
  Dan 12,3} \bibverse{34} Y en su caer serán ayudados de pequeño
socorro: y muchos se juntarán á ellos con lisonjas. \bibverse{35} Y
algunos de los sabios caerán para ser purgados, y limpiados, y
emblanquecidos, hasta el tiempo determinado: porque aun para esto hay
plazo.

\hypertarget{actos-violentos-atropellos-contra-el-culto-juduxedo-y-el-resultado-del-rey-antijuduxedo}{%
\subsection{Actos violentos, atropellos contra el culto judío y el
resultado del rey
antijudío}\label{actos-violentos-atropellos-contra-el-culto-juduxedo-y-el-resultado-del-rey-antijuduxedo}}

\bibverse{36} Y el rey hará á su voluntad; y se ensoberbecerá, y se
engrandecerá sobre todo dios: y contra el Dios de los dioses hablará
maravillas, y será prosperado, hasta que sea consumada la ira: porque
hecha está determinación. \footnote{\textbf{11:36} 2Tes 2,4; Dan 7,8;
  Dan 7,25; Apoc 13,5-6} \bibverse{37} Y del Dios de sus padres no se
cuidará, ni del amor de las mujeres: ni se cuidará de dios alguno,
porque sobre todo se engrandecerá. \footnote{\textbf{11:37} 1Tim 4,3}
\bibverse{38} Mas honrará en su lugar al dios Mauzim, dios que sus
padres no conocieron: honrarálo con oro, y plata, y piedras preciosas, y
con cosas de gran precio. \bibverse{39} Y con el dios ajeno que
conocerá, hará á los baluartes de Mauzim crecer en gloria: y harálos
enseñorear sobre muchos, y por interés repartirá la tierra.

\bibverse{40} Empero al cabo del tiempo el rey del mediodía se acorneará
con él; y el rey del norte levantará contra él como tempestad, con
carros y gente de á caballo, y muchos navíos; y entrará por las tierras,
é inundará, y pasará. \bibverse{41} Y vendrá á la tierra deseable, y
muchas provincias caerán; mas éstas escaparán de su mano: Edom, y Moab,
y lo primero de los hijos de Ammón. \footnote{\textbf{11:41} Dan 11,16}

\bibverse{42} Asimismo extenderá su mano á las otras tierras, y no
escapará el país de Egipto. \bibverse{43} Y se apoderará de los tesoros
de oro y plata, y de todas las cosas preciosas de Egipto, de Libia, y
Etiopía por donde pasará. \bibverse{44} Mas nuevas de oriente y del
norte lo espantarán; y saldrá con grande ira para destruir y matar
muchos. \bibverse{45} Y plantará las tiendas de su palacio entre los
mares, en el monte deseable del santuario; y vendrá hasta su fin, y no
tendrá quien le ayude.

\hypertarget{el-amanecer-del-fin-de-los-tiempos-con-su-miseria-su-retribuciuxf3n-y-la-resurrecciuxf3n-de-los-impuxedos-y-de-los-rectos}{%
\subsection{El amanecer del fin de los tiempos con su miseria, su
retribución y la resurrección de los impíos y de los
rectos}\label{el-amanecer-del-fin-de-los-tiempos-con-su-miseria-su-retribuciuxf3n-y-la-resurrecciuxf3n-de-los-impuxedos-y-de-los-rectos}}

\hypertarget{section-11}{%
\section{12}\label{section-11}}

\bibverse{1} Y en aquel tiempo se levantará Miguel, el gran príncipe que
está por los hijos de tu pueblo; y será tiempo de angustia, cual nunca
fué después que hubo gente hasta entonces: mas en aquel tiempo será
libertado tu pueblo, todos los que se hallaren escritos en el libro.
\bibverse{2} Y muchos de los que duermen en el polvo de la tierra serán
despertados, unos para vida eterna, y otros para vergüenza y confusión
perpetua. \footnote{\textbf{12:2} Juan 5,29} \bibverse{3} Y los
entendidos resplandecerán como el resplandor del firmamento; y los que
enseñan á justicia la multitud, como las estrellas á perpetua eternidad.
\footnote{\textbf{12:3} Mat 13,43; 1Cor 15,41-42}

\hypertarget{comisiuxf3n-del-uxe1ngel-a-daniel-revelaciuxf3n-sobre-la-duraciuxf3n-del-peruxedodo-de-sufrimiento-y-finalmente-una-promesa-de-salvaciuxf3n-para-daniel}{%
\subsection{Comisión del ángel a Daniel; Revelación sobre la duración
del período de sufrimiento; y finalmente una promesa de salvación para
Daniel}\label{comisiuxf3n-del-uxe1ngel-a-daniel-revelaciuxf3n-sobre-la-duraciuxf3n-del-peruxedodo-de-sufrimiento-y-finalmente-una-promesa-de-salvaciuxf3n-para-daniel}}

\bibverse{4} Tú empero Daniel, cierra las palabras y sella el libro
hasta el tiempo del fin: pasarán muchos, y multiplicaráse la ciencia.
\footnote{\textbf{12:4} Dan 12,9; Apoc 10,4}

\bibverse{5} Y yo, Daniel, miré, y he aquí otros dos que estaban, el uno
de esta parte á la orilla del río, y el otro de la otra parte á la
orilla del río. \bibverse{6} Y dijo uno al varón vestido de lienzos, que
estaba sobre las aguas del río: ¿Cuándo será el fin de estas maravillas?

\bibverse{7} Y oía al varón vestido de lienzos, que estaba sobre las
aguas del río, el cual alzó su diestra y su siniestra al cielo, y juró
por el Viviente en los siglos, que será por tiempo, tiempos, y la mitad.
Y cuando se acabare el esparcimiento del escuadrón del pueblo santo,
todas estas cosas serán cumplidas. \footnote{\textbf{12:7} Apoc 10,5-6;
  Dan 7,25}

\bibverse{8} Y yo oí, mas no entendí. Y dije: Señor mío, ¿qué será el
cumplimiento de estas cosas?

\bibverse{9} Y dijo: Anda, Daniel, que estas palabras están cerradas y
selladas hasta el tiempo del cumplimiento. \bibverse{10} Muchos serán
limpios, y emblanquecidos, y purificados; mas los impíos obrarán
impíamente, y ninguno de los impíos entenderá, pero entenderán los
entendidos.

\bibverse{11} Y desde el tiempo que fuere quitado el continuo sacrificio
hasta la abominación espantosa, habrá mil doscientos y noventa días.
\bibverse{12} Bienaventurado el que esperare, y llegare hasta mil
trescientos treinta y cinco días.

\bibverse{13} Y tú irás al fin, y reposarás, y te levantarás en tu
suerte al fin de los días.
