\hypertarget{primer-recuento-de-los-hombres-de-guerra}{%
\subsection{Primer recuento de los hombres de
guerra}\label{primer-recuento-de-los-hombres-de-guerra}}

\hypertarget{section}{%
\section{1}\label{section}}

\bibverse{1} Y habló Jehová á Moisés en el desierto de Sinaí, en el
tabernáculo del testimonio, en el primero del mes segundo, en el segundo
año de su salida de la tierra de Egipto, diciendo: \bibverse{2} Tomad el
encabezamiento de toda la congregación de los hijos de Israel por sus
familias, por las casas de sus padres, con la cuenta de los nombres,
todos los varones por sus cabezas: \footnote{\textbf{1:2} Núm 26,2-51;
  Éxod 30,12} \bibverse{3} De veinte años arriba, todos los que pueden
salir á la guerra en Israel, los contaréis tú y Aarón por sus
cuadrillas. \bibverse{4} Y estará con vosotros un varón de cada tribu,
cada uno cabeza de la casa de sus padres. \bibverse{5} Y estos son los
nombres de los varones que estarán con vosotros: De la tribu de Rubén,
Elisur hijo de Sedeur.

\bibverse{6} De Simeón, Selumiel hijo de Zurisaddai.

\bibverse{7} De Judá, Naasón hijo de Aminadab.

\bibverse{8} De Issachâr, Nathanael hijo de Suar.

\bibverse{9} De Zabulón, Eliab hijo de Helón.

\bibverse{10} De los hijos de José: de Ephraim, Elisama hijo de Ammiud;
de Manasés, Gamaliel hijo de Pedasur. \footnote{\textbf{1:10} 1Cró 7,26}

\bibverse{11} De Benjamín, Abidán hijo de Gedeón.

\bibverse{12} De Dan, Ahiezer hijo de Ammisaddai.

\bibverse{13} De Aser, Phegiel hijo de Ocrán.

\bibverse{14} De Gad, Eliasaph hijo de Dehuel.

\bibverse{15} De Nephtalí, Ahira hijo de Enán.

\bibverse{16} Estos eran los nombrados de la congregación, príncipes de
las tribus de sus padres, capitanes de los millares de Israel.
\bibverse{17} Tomó pues Moisés y Aarón á estos varones que fueron
declarados por sus nombres: \bibverse{18} Y juntaron toda la
congregación en el primero del mes segundo, y fueron reunidos sus
linajes, por las casas de sus padres, según la cuenta de los nombres, de
veinte años arriba, por sus cabezas, \bibverse{19} Como Jehová lo había
mandado á Moisés; y contólos en el desierto de Sinaí.

\hypertarget{los-resultados-del-censo}{%
\subsection{Los resultados del censo}\label{los-resultados-del-censo}}

\bibverse{20} Y los hijos de Rubén, primogénito de Israel, por sus
generaciones, por sus familias, por las casas de sus padres, conforme á
la cuenta de los nombres por sus cabezas, todos los varones de veinte
años arriba, todos los que podían salir á la guerra; \bibverse{21} Los
contados de ellos, de la tribu de Rubén, fueron cuarenta y seis mil y
quinientos.

\bibverse{22} De los hijos de Simeón, por sus generaciones, por sus
familias, por las casas de sus padres, los contados de ellos conforme á
la cuenta de los nombres por sus cabezas, todos los varones de veinte
años arriba, todos los que podían salir á la guerra; \bibverse{23} Los
contados de ellos, de la tribu de Simeón, cincuenta y nueve mil y
trescientos.

\bibverse{24} De los hijos de Gad, por sus generaciones, por sus
familias, por las casas de sus padres, conforme á la cuenta de los
nombres, de veinte años arriba, todos los que podían salir á la guerra;
\bibverse{25} Los contados de ellos, de la tribu de Gad, cuarenta y
cinco mil seiscientos y cincuenta.

\bibverse{26} De los hijos de Judá, por sus generaciones, por sus
familias, por las casas de sus padres, conforme á la cuenta de los
nombres, de veinte años arriba, todos los que podían salir á la guerra;
\bibverse{27} Los contados de ellos, de la tribu de Judá, setenta y
cuatro mil y seiscientos.

\bibverse{28} De los hijos de Issachâr, por sus generaciones, por sus
familias, por las casas de sus padres, conforme á la cuenta de los
nombres, de veinte años arriba, todos los que podían salir á la guerra;
\bibverse{29} Los contados de ellos, de la tribu de Issachâr, cincuenta
y cuatro mil y cuatrocientos.

\bibverse{30} De los hijos de Zabulón, por sus generaciones, por sus
familias, por las casas de sus padres, conforme á la cuenta de sus
nombres, de veinte años arriba, todos los que podían salir á la guerra;
\bibverse{31} Los contados de ellos, de la tribu de Zabulón, cincuenta y
siete mil y cuatrocientos.

\bibverse{32} De los hijos de José: de los hijos de Ephraim, por sus
generaciones, por sus familias, por las casas de sus padres, conforme á
la cuenta de los nombres, de veinte años arriba, todos los que podían
salir á la guerra; \bibverse{33} Los contados de ellos, de la tribu de
Ephraim, cuarenta mil y quinientos.

\bibverse{34} De los hijos de Manasés, por sus generaciones, por sus
familias, por las casas de sus padres, conforme á la cuenta de los
nombres, de veinte años arriba, todos los que podían salir á la guerra;
\bibverse{35} Los contados de ellos, de la tribu de Manasés, treinta y
dos mil y doscientos.

\bibverse{36} De los hijos de Benjamín, por sus generaciones, por sus
familias, por las casas de sus padres, conforme á la cuenta de los
nombres, de veinte años arriba, todos los que podían salir á la guerra;
\bibverse{37} Los contados de ellos, de la tribu de Benjamín, treinta y
cinco mil y cuatrocientos.

\bibverse{38} De los hijos de Dan, por sus generaciones, por sus
familias, por las casas de sus padres, conforme á la cuenta de los
nombres, de veinte años arriba, todos los que podían salir á la guerra;
\bibverse{39} Los contados de ellos, de la tribu de Dan, sesenta y dos
mil y setecientos.

\bibverse{40} De los hijos de Aser, por sus generaciones, por sus
familias, por las casas de sus padres, conforme á la cuenta de los
nombres, de veinte años arriba, todos los que podían salir á la guerra;
\bibverse{41} Los contados de ellos, de la tribu de Aser, cuarenta y un
mil y quinientos.

\bibverse{42} De los hijos de Nephtalí, por sus generaciones, por sus
familias, por las casas de sus padres, conforme á la cuenta de los
nombres, de veinte años arriba, todos los que podían salir á la guerra;
\bibverse{43} Los contados de ellos, de la tribu de Nephtalí, cincuenta
y tres mil y cuatrocientos.

\bibverse{44} Estos fueron los contados, los cuales contaron Moisés y
Aarón, con los príncipes de Israel, que eran doce, uno por cada casa de
sus padres. \bibverse{45} Y fueron todos los contados de los hijos de
Israel por las casas de sus padres, de veinte años arriba, todos los que
podían salir á la guerra en Israel; \bibverse{46} Fueron todos los
contados seiscientos tres mil quinientos y cincuenta.

\hypertarget{la-posiciuxf3n-excepcional-de-los-levitas}{%
\subsection{La posición excepcional de los
levitas}\label{la-posiciuxf3n-excepcional-de-los-levitas}}

\bibverse{47} Pero los Levitas no fueron contados entre ellos según la
tribu de sus padres. \bibverse{48} Porque habló Jehová á Moisés,
diciendo: \bibverse{49} Solamente no contarás la tribu de Leví, ni
tomarás la cuenta de ellos entre los hijos de Israel: \footnote{\textbf{1:49}
  Núm 2,33; Núm 3,15} \bibverse{50} Mas tú pondrás á los Levitas en el
tabernáculo del testimonio, y sobre todos sus vasos, y sobre todas las
cosas que le pertenecen: ellos llevarán el tabernáculo y todos sus
vasos, y ellos servirán en él, y asentarán sus tiendas alrededor del
tabernáculo. \footnote{\textbf{1:50} Núm 4,-1; Núm 3,23-38}
\bibverse{51} Y cuando el tabernáculo partiere, los Levitas lo
desarmarán; y cuando el tabernáculo parare, los Levitas lo armarán: y el
extraño que se llegare, morirá. \footnote{\textbf{1:51} Núm 3,10; Núm
  3,38} \bibverse{52} Y los hijos de Israel asentarán sus tiendas cada
uno en su escuadrón, y cada uno junto á su bandera, por sus cuadrillas;
\bibverse{53} Mas los Levitas asentarán las suyas alrededor del
tabernáculo del testimonio, y no habrá ira sobre la congregación de los
hijos de Israel: y los Levitas tendrán la guarda del tabernáculo del
testimonio.

\bibverse{54} E hicieron los hijos de Israel conforme á todas las cosas
que mandó Jehová á Moisés; así lo hicieron.

\hypertarget{el-orden-de-acampamiento-de-las-tribus.}{%
\subsection{El orden de acampamiento de las
tribus.}\label{el-orden-de-acampamiento-de-las-tribus.}}

\hypertarget{section-1}{%
\section{2}\label{section-1}}

\bibverse{1} Y habló Jehová á Moisés y á Aarón, diciendo: \bibverse{2}
Los hijos de Israel acamparán cada uno junto á su bandera, según las
enseñas de las casas de sus padres; alrededor del tabernáculo del
testimonio acamparán.

\bibverse{3} Estos acamparán al levante, al oriente: la bandera del
ejército de Judá, por sus escuadrones; y el jefe de los hijos de Judá,
Naasón hijo de Aminadab: \bibverse{4} Su hueste, con los contados de
ellos, setenta y cuatro mil y seiscientos.

\bibverse{5} Junto á él acamparán los de la tribu de Issachâr: y el jefe
de los hijos de Issachâr, Nathanael hijo de Suar; \bibverse{6} Y su
hueste, con sus contados, cincuenta y cuatro mil y cuatrocientos:

\bibverse{7} Y la tribu de Zabulón: y el jefe de los hijos de Zabulón,
Eliab hijo de Helón; \bibverse{8} Y su hueste, con sus contados,
cincuenta y siete mil y cuatrocientos.

\bibverse{9} Todos los contados en el ejército de Judá, ciento ochenta y
seis mil y cuatrocientos, por sus escuadrones, irán delante.

\bibverse{10} La bandera del ejército de Rubén al mediodía, por sus
escuadrones: y el jefe de los hijos de Rubén, Elisur hijo de Sedeur;
\bibverse{11} Y su hueste, sus contados, cuarenta y seis mil y
quinientos.

\bibverse{12} Y acamparán junto á él los de la tribu de Simeón: y el
jefe de los hijos de Simeón, Selumiel hijo de Zurisaddai; \bibverse{13}
Y su hueste, con los contados de ellos, cincuenta y nueve mil y
trescientos:

\bibverse{14} Y la tribu de Gad: y el jefe de los hijos de Gad, Eliasaph
hijo de Rehuel; \bibverse{15} Y su hueste, con los contados de ellos,
cuarenta y cinco mil seiscientos y cincuenta.

\bibverse{16} Todos los contados en el ejército de Rubén, ciento
cincuenta y un mil cuatrocientos y cincuenta, por sus escuadrones, irán
los segundos.

\bibverse{17} Luego irá el tabernáculo del testimonio, el campo de los
Levitas en medio de los ejércitos: de la manera que asientan el campo,
así caminarán, cada uno en su lugar, junto á sus banderas.

\bibverse{18} La bandera del ejército de Ephraim por sus escuadrones, al
occidente: y el jefe de los hijos de Ephraim, Elisama hijo de Ammiud;
\bibverse{19} Y su hueste, con los contados de ellos, cuarenta mil y
quinientos.

\bibverse{20} Junto á él estará la tribu de Manasés; y el jefe de los
hijos de Manasés, Gamaliel hijo de Pedasur; \bibverse{21} Y su hueste,
con los contados de ellos, treinta y dos mil y doscientos:

\bibverse{22} Y la tribu de Benjamín: y el jefe de los hijos de
Benjamín, Abidán hijo de Gedeón; \bibverse{23} Y su hueste, con los
contados de ellos, treinta y cinco mil y cuatrocientos.

\bibverse{24} Todos los contados en el ejército de Ephraim, ciento ocho
mil y ciento, por sus escuadrones, irán los terceros.

\bibverse{25} La bandera del ejército de Dan estará al aquilón, por sus
escuadrones: y el jefe de los hijos de Dan, Ahiezer hijo de Amisaddai;
\bibverse{26} Y su hueste, con los contados de ellos, sesenta y dos mil
y setecientos.

\bibverse{27} Junto á él acamparán los de la tribu de Aser: y el jefe de
los hijos de Aser, Phegiel hijo de Ocrán; \bibverse{28} Y su hueste, con
los contados de ellos, cuarenta y un mil y quinientos:

\bibverse{29} Y la tribu de Nephtalí: y el jefe de los hijos de
Nephtalí, Ahira hijo de Enán; \bibverse{30} Y su hueste, con los
contados de ellos, cincuenta y tres mil y cuatrocientos.

\bibverse{31} Todos los contados en el ejército de Dan, ciento cincuenta
y siete mil y seiscientos: irán los postreros tras sus banderas.

\bibverse{32} Estos son los contados de los hijos de Israel, por las
casas de sus padres: todos los contados por ejércitos, por sus
escuadrones, seiscientos tres mil quinientos y cincuenta. \footnote{\textbf{2:32}
  Núm 1,46} \bibverse{33} Mas los Levitas no fueron contados entre los
hijos de Israel; como Jehová lo mandó á Moisés. \footnote{\textbf{2:33}
  Núm 1,48-49}

\bibverse{34} E hicieron los hijos de Israel conforme á todas las cosas
que Jehová mandó á Moisés; así asentaron el campo por sus banderas, y
así marcharon cada uno por sus familias, según las casas de sus padres.
\footnote{\textbf{2:34} Núm 2,2}

\hypertarget{los-hijos-de-aaron}{%
\subsection{Los hijos de Aaron}\label{los-hijos-de-aaron}}

\hypertarget{section-2}{%
\section{3}\label{section-2}}

\bibverse{1} Y estas son las generaciones de Aarón y de Moisés, desde
que Jehová habló á Moisés en el monte de Sinaí. \footnote{\textbf{3:1}
  Éxod 6,23} \bibverse{2} Y estos son los nombres de los hijos de Aarón:
Nadab el primogénito, y Abiú, Eleazar, é Ithamar.

\bibverse{3} Estos son los nombres de los hijos de Aarón, sacerdotes
ungidos; cuyas manos él hinchió para administrar el sacerdocio.
\bibverse{4} Mas Nadab y Abiú murieron delante de Jehová, cuando
ofrecieron fuego extraño delante de Jehová, en el desierto de Sinaí: y
no tuvieron hijos: y Eleazar é Ithamar ejercieron el sacerdocio delante
de Aarón su padre. \footnote{\textbf{3:4} Lev 10,1-2}

\hypertarget{los-levitas-fueron-designados-para-ayudar-a-los-sacerdotes-y-servir-en-el-santuario.}{%
\subsection{Los levitas fueron designados para ayudar a los sacerdotes y
servir en el
santuario.}\label{los-levitas-fueron-designados-para-ayudar-a-los-sacerdotes-y-servir-en-el-santuario.}}

\bibverse{5} Y Jehová habló á Moisés, diciendo: \bibverse{6} Haz llegar
á la tribu de Leví, y hazla estar delante del sacerdote Aarón, para que
le ministren; \bibverse{7} Y desempeñen su cargo, y el cargo de toda la
congregación delante del tabernáculo del testimonio, para servir en el
ministerio del tabernáculo; \bibverse{8} Y guarden todas las alhajas del
tabernáculo del testimonio, y lo encargado á ellos de los hijos de
Israel, y ministren en el servicio del tabernáculo. \footnote{\textbf{3:8}
  Núm 4,-1} \bibverse{9} Y darás los Levitas á Aarón y á sus hijos: le
son enteramente dados de entre los hijos de Israel. \bibverse{10} Y
constituirás á Aarón y á sus hijos, para que ejerzan su sacerdocio: y el
extraño que se llegare, morirá.

\hypertarget{los-levitas-fueran-designados-para-redimir-al-primoguxe9nito-israelita}{%
\subsection{Los levitas fueran designados para redimir al primogénito
israelita}\label{los-levitas-fueran-designados-para-redimir-al-primoguxe9nito-israelita}}

\bibverse{11} Y habló Jehová á Moisés, diciendo: \bibverse{12} Y he aquí
yo he tomado los Levitas de entre los hijos de Israel en lugar de todos
los primogénitos que abren la matriz entre los hijos de Israel; serán
pues míos los Levitas: \footnote{\textbf{3:12} Núm 8,16; Éxod 13,2}
\bibverse{13} Porque mío es todo primogénito; desde el día que yo maté
todos los primogénitos en la tierra de Egipto, yo santifiqué á mí todos
los primogénitos en Israel, así de hombres como de animales: míos serán:
Yo Jehová.

\hypertarget{conteo-lugar-de-almacenamiento-luxedder-y-reglamentos-de-los-levitas-masculinos}{%
\subsection{Conteo, lugar de almacenamiento, líder y reglamentos de los
levitas
masculinos}\label{conteo-lugar-de-almacenamiento-luxedder-y-reglamentos-de-los-levitas-masculinos}}

\bibverse{14} Y Jehová habló á Moisés en el desierto de Sinaí, diciendo:
\bibverse{15} Cuenta los hijos de Leví por las casas de sus padres, por
sus familias: contarás todos los varones de un mes arriba.

\bibverse{16} Y Moisés los contó conforme á la palabra de Jehová, como
le fué mandado.

\bibverse{17} Y los hijos de Leví fueron estos por sus nombres: Gersón,
y Coath, y Merari.

\bibverse{18} Y los nombres de los hijos de Gersón, por sus familias,
estos: Libni, y Simei.

\bibverse{19} Y los hijos de Coath, por sus familias: Amram, é Izhar, y
Hebrón, y Uzziel.

\bibverse{20} Y los hijos de Merari, por sus familias: Mahali, y Musi.
Estas, las familias de Leví, por las casas de sus padres.

\bibverse{21} De Gersón, la familia de Libni y la de Simei: estas son
las familias de Gersón.

\bibverse{22} Los contados de ellos conforme á la cuenta de todos los
varones de un mes arriba, los contados de ellos, siete mil y quinientos.

\bibverse{23} Las familias de Gersón asentarán sus tiendas á espaldas
del tabernáculo, al occidente;

\bibverse{24} Y el jefe de la casa del padre de los Gersonitas, Eliasaph
hijo de Lael. \bibverse{25} A cargo de los hijos de Gersón, en el
tabernáculo del testimonio, estará el tabernáculo, y la tienda, y su
cubierta, y el pabellón de la puerta del tabernáculo del testimonio,
\bibverse{26} Y las cortinas del atrio, y el pabellón de la puerta del
atrio, que está junto al tabernáculo y junto al altar alrededor;
asimismo sus cuerdas para todo su servicio.

\bibverse{27} Y de Coath, la familia Amramítica, y la familia
Izeharítica, y la familia Hebronítica, y la familia Ozielítica: estas
son las familias Coathitas. \bibverse{28} Por la cuenta de todos los
varones de un mes arriba, eran ocho mil y seiscientos, que tenían la
guarda del santuario.

\bibverse{29} Las familias de los hijos de Coath acamparán al lado del
tabernáculo, al mediodía; \bibverse{30} Y el jefe de la casa del padre
de las familias de Coath, Elisaphán hijo de Uzziel. \footnote{\textbf{3:30}
  Lev 10,4} \bibverse{31} Y á cargo de ellos estará el arca, y la mesa,
y el candelero, y los altares, y los vasos del santuario con que
ministran, y el velo, con todo su servicio. \footnote{\textbf{3:31} Núm
  7,9} \bibverse{32} Y el principal de los jefes de los Levitas será
Eleazar, hijo de Aarón el sacerdote, prepósito de los que tienen la
guarda del santuario.

\bibverse{33} De Merari, la familia Mahalítica, y la familia Musítica:
estas son las familias de Merari. \bibverse{34} Y los contados de ellos
conforme á la cuenta de todos los varones de un mes arriba, fueron seis
mil y doscientos.

\bibverse{35} Y el jefe de la casa del padre de las familias de Merari,
Suriel hijo de Abihail: acamparán al lado del tabernáculo, al aquilón.
\bibverse{36} Y á cargo de los hijos de Merari estará la custodia de las
tablas del tabernáculo, y sus barras, y sus columnas, y sus basas, y
todos sus enseres, con todo su servicio: \bibverse{37} Y las columnas en
derredor del atrio, y sus basas, y sus estacas, y sus cuerdas.

\bibverse{38} Y los que acamparán delante del tabernáculo al oriente,
delante del tabernáculo del testimonio al levante, serán Moisés, y Aarón
y sus hijos, teniendo la guarda del santuario en lugar de los hijos de
Israel: y el extraño que se acercare, morirá. \footnote{\textbf{3:38}
  Núm 3,10} \bibverse{39} Todos los contados de los Levitas, que Moisés
y Aarón conforme á la palabra de Jehová contaron por sus familias, todos
los varones de un mes arriba, fueron veinte y dos mil.

\hypertarget{examen-y-resoluciuxf3n-del-primoguxe9nito-masculino-en-israel}{%
\subsection{Examen y resolución del primogénito masculino en
Israel}\label{examen-y-resoluciuxf3n-del-primoguxe9nito-masculino-en-israel}}

\bibverse{40} Y Jehová dijo á Moisés: Cuenta todos los primogénitos
varones de los hijos de Israel de un mes arriba, y toma la cuenta de los
nombres de ellos. \bibverse{41} Y tomarás los Levitas para mí, yo
Jehová, en lugar de todos los primogénitos de los hijos de Israel: y los
animales de los Levitas en lugar de todos los primogénitos de los
animales de los hijos de Israel.

\bibverse{42} Y contó Moisés, como Jehová le mandó, todos los
primogénitos de los hijos de Israel. \bibverse{43} Y todos los
primogénitos varones, conforme á la cuenta de los nombres, de un mes
arriba, los contados de ellos fueron veinte y dos mil doscientos setenta
y tres.

\bibverse{44} Y habló Jehová á Moisés, diciendo: \bibverse{45} Toma los
Levitas en lugar de todos los primogénitos de los hijos de Israel, y los
animales de los Levitas en lugar de sus animales; y los Levitas serán
míos: Yo Jehová. \bibverse{46} Y por los rescates de los doscientos y
setenta y tres, que sobrepujan á los Levitas los primogénitos de los
hijos de Israel, \footnote{\textbf{3:46} Núm 3,39; Núm 3,43}
\bibverse{47} Tomarás cinco siclos por cabeza; conforme al siclo del
santuario tomarás: el siclo tiene veinte óbolos: \bibverse{48} Y darás á
Aarón y á sus hijos el dinero por los rescates de los que de ellos
sobran.

\bibverse{49} Tomó, pues, Moisés el dinero del rescate de los que
resultaron de más de los redimidos por los Levitas: \bibverse{50} Y
recibió de los primogénitos de los hijos de Israel en dinero, mil
trescientos sesenta y cinco siclos, conforme al siclo del santuario.
\bibverse{51} Y Moisés dió el dinero de los rescates á Aarón y á sus
hijos, conforme al dicho de Jehová, según que Jehová había mandado á
Moisés.

\hypertarget{examen-de-los-levitas-aptos-para-el-servicio-incluidas-las-normas-de-servicio}{%
\subsection{Examen de los levitas aptos para el servicio, incluidas las
normas de
servicio}\label{examen-de-los-levitas-aptos-para-el-servicio-incluidas-las-normas-de-servicio}}

\hypertarget{section-3}{%
\section{4}\label{section-3}}

\bibverse{1} Y habló Jehová á Moisés y á Aarón, diciendo: \bibverse{2}
Toma la cuenta de los hijos de Coath de entre los hijos de Leví, por sus
familias, por las casas de sus padres, \bibverse{3} De edad de treinta
años arriba hasta cincuenta años, todos los que entran en compañía, para
hacer servicio en el tabernáculo del testimonio.

\bibverse{4} Este será el oficio de los hijos de Coath en el tabernáculo
del testimonio, en el lugar santísimo: \bibverse{5} Cuando se hubiere de
mudar el campo, vendrán Aarón y sus hijos, y desarmarán el velo de la
tienda, y cubrirán con él el arca del testimonio: \bibverse{6} Y pondrán
sobre ella la cubierta de pieles de tejones, y extenderán encima el paño
todo de cárdeno, y le pondrán sus varas.

\bibverse{7} Y sobre la mesa de la proposición extenderán el paño
cárdeno, y pondrán sobre ella las escudillas, y las cucharas, y las
copas, y los tazones para libar: y el pan continuo estará sobre ella.
\bibverse{8} Y extenderán sobre ella el paño de carmesí colorado, y lo
cubrirán con la cubierta de pieles de tejones; y le pondrán sus varas.

\bibverse{9} Y tomarán un paño cárdeno, y cubrirán el candelero de la
luminaria; y sus candilejas, y sus despabiladeras, y sus platillos, y
todos sus vasos del aceite con que se sirve; \footnote{\textbf{4:9} Éxod
  25,31} \bibverse{10} Y lo pondrán con todos sus vasos en una cubierta
de pieles de tejones, y lo colocarán sobre unas parihuelas.

\bibverse{11} Y sobre el altar de oro extenderán el paño cárdeno, y le
cubrirán con la cubierta de pieles de tejones, y le pondrán sus varales.

\bibverse{12} Y tomarán todos los vasos del servicio, de que hacen uso
en el santuario, y los pondrán en un paño cárdeno, y los cubrirán con
una cubierta de pieles de tejones, y los colocarán sobre unas
parihuelas.

\bibverse{13} Y quitarán la ceniza del altar, y extenderán sobre él un
paño de púrpura: \bibverse{14} Y pondrán sobre él todos sus instrumentos
de que se sirve: las paletas, los garfios, los braseros, y los tazones,
todos los vasos del altar; y extenderán sobre él la cubierta de pieles
de tejones, y le pondrán además las varas.

\bibverse{15} Y en acabando Aarón y sus hijos de cubrir el santuario y
todos los vasos del santuario, cuando el campo se hubiere de mudar,
vendrán después de ello los hijos de Coath para conducir: mas no tocarán
cosa santa, que morirán. Estas serán las cargas de los hijos de Coath en
el tabernáculo del testimonio.

\bibverse{16} Empero al cargo de Eleazar, hijo de Aarón el sacerdote,
estará el aceite de la luminaria, y el perfume aromático, y el presente
continuo, y el aceite de la unción; el cargo de todo el tabernáculo, y
de todo lo que está en él, en el santuario, y en sus vasos.

\bibverse{17} Y habló Jehová á Moisés y á Aarón, diciendo: \bibverse{18}
No cortaréis la tribu de las familias de Coath de entre los Levitas;
\bibverse{19} Mas esto haréis con ellos, para que vivan, y no mueran
cuando llegaren al lugar santísimo: Aarón y sus hijos vendrán y los
pondrán á cada uno en su oficio, y en su cargo. \bibverse{20} No
entrarán para ver, cuando cubrieren las cosas santas; que morirán.
\footnote{\textbf{4:20} 1Sam 6,19}

\bibverse{21} Y habló Jehová á Moisés diciendo: \bibverse{22} Toma
también la cuenta de los hijos de Gersón por las casas de sus padres,
por sus familias. \bibverse{23} De edad de treinta años arriba hasta
cincuenta años los contarás; todos los que entran en compañía, para
hacer servicio en el tabernáculo del testimonio.

\bibverse{24} Este será el oficio de las familias de Gersón, para
ministrar y para llevar: \bibverse{25} Llevarán las cortinas del
tabernáculo, y el tabernáculo del testimonio, su cubierta, y la cubierta
de pieles de tejones que está sobre él encima, y el pabellón de la
puerta del tabernáculo del testimonio, \bibverse{26} Y las cortinas del
atrio, y el pabellón de la puerta del atrio, que está cerca del
tabernáculo y cerca del altar alrededor, y sus cuerdas, y todos los
instrumentos de su servicio, y todo lo que será hecho para ellos: así
servirán. \bibverse{27} Según la orden de Aarón y de sus hijos será todo
el ministerio de los hijos de Gersón en todos sus cargos, y en todo su
servicio: y les encomendaréis en guarda todos sus cargos. \bibverse{28}
Este es el servicio de las familias de los hijos de Gersón en el
tabernáculo del testimonio: y el cargo de ellos estará bajo la mano de
Ithamar, hijo de Aarón el sacerdote.

\bibverse{29} Contarás los hijos de Merari por sus familias, por las
casas de sus padres. \bibverse{30} Desde el de edad de treinta años
arriba hasta el de cincuenta años, los contarás; todos los que entran en
compañía, para hacer servicio en el tabernáculo del testimonio.
\bibverse{31} Y este será el deber de su cargo para todo su servicio en
el tabernáculo del testimonio: las tablas del tabernáculo, y sus barras,
y sus columnas, y sus basas, \bibverse{32} Y las columnas del atrio
alrededor, y sus basas, y sus estacas, y sus cuerdas con todos sus
instrumentos, y todo su servicio; y contaréis por sus nombres todos los
vasos de la guarda de su cargo. \bibverse{33} Este será el servicio de
las familias de los hijos de Merari para todo su ministerio en el
tabernáculo del testimonio, bajo la mano de Ithamar, hijo de Aarón el
sacerdote.

\hypertarget{resultados-de-la-inspecciuxf3n}{%
\subsection{Resultados de la
inspección}\label{resultados-de-la-inspecciuxf3n}}

\bibverse{34} Moisés, pues, y Aarón, y los jefes de la congregación,
contaron los hijos de Coath por sus familias, y por las casas de sus
padres, \bibverse{35} Desde el de edad de treinta años arriba hasta el
de edad de cincuenta años; todos los que entran en compañía, para
ministrar en el tabernáculo del testimonio. \bibverse{36} Y fueron los
contados de ellos por sus familias, dos mil setecientos y cincuenta.
\bibverse{37} Estos fueron los contados de las familias de Coath, todos
los que ministran en el tabernáculo del testimonio, los cuales contaron
Moisés y Aarón, como lo mandó Jehová por mano de Moisés.

\bibverse{38} Y los contados de los hijos de Gersón, por sus familias, y
por las casas de sus padres, \bibverse{39} Desde el de edad de treinta
años arriba hasta el de edad de cincuenta años, todos los que entran en
compañía, para ministrar en el tabernáculo del testimonio; \bibverse{40}
Los contados de ellos por sus familias, por las casas de sus padres,
fueron dos mil seiscientos y treinta. \bibverse{41} Estos son los
contados de las familias de los hijos de Gersón, todos los que ministran
en el tabernáculo del testimonio, los cuales contaron Moisés y Aarón por
mandato de Jehová.

\bibverse{42} Y los contados de las familias de los hijos de Merari, por
sus familias, por las casas de sus padres, \bibverse{43} Desde el de
edad de treinta años arriba hasta el de edad de cincuenta años, todos
los que entran en compañía, para ministrar en el tabernáculo del
testimonio: \bibverse{44} Los contados de ellos, por sus familias,
fueron tres mil y doscientos. \bibverse{45} Estos fueron los contados de
las familias de los hijos de Merari, los cuales contaron Moisés y Aarón,
según lo mandó Jehová por mano de Moisés.

\bibverse{46} Todos los contados de los Levitas, que Moisés y Aarón y
los jefes de Israel contaron por sus familias, y por las casas de sus
padres, \bibverse{47} Desde el de edad de treinta años arriba hasta el
de edad de cincuenta años, todos los que entraban para ministrar en el
servicio, y tener cargo de obra en el tabernáculo del testimonio;
\bibverse{48} Los contados de ellos fueron ocho mil quinientos y
ochenta, \bibverse{49} Como lo mandó Jehová por mano de Moisés fueron
contados, cada uno según su oficio, y según su cargo; los cuales contó
él, como le fué mandado.

\hypertarget{extracciuxf3n-de-los-inmundos-del-campamento}{%
\subsection{Extracción de los inmundos del
campamento}\label{extracciuxf3n-de-los-inmundos-del-campamento}}

\hypertarget{section-4}{%
\section{5}\label{section-4}}

\bibverse{1} Y jehová habló á Moisés, diciendo: \bibverse{2} Manda á los
hijos de Israel que echen del campo á todo leproso, y á todos los que
padecen flujo de semen, y á todo contaminado sobre muerto: \bibverse{3}
Así hombres como mujeres echaréis, fuera del campo los echaréis; porque
no contaminen el campo de aquellos entre los cuales yo habito.
\footnote{\textbf{5:3} Núm 12,14; Núm 35,34}

\bibverse{4} E hiciéronlo así los hijos de Israel, que los echaron fuera
del campo: como Jehová dijo á Moisés, así lo hicieron los hijos de
Israel.

\hypertarget{malversaciuxf3n-y-su-expiaciuxf3n}{%
\subsection{Malversación y su
expiación}\label{malversaciuxf3n-y-su-expiaciuxf3n}}

\bibverse{5} Además habló Jehová á Moisés, diciendo: \bibverse{6} Habla
á los hijos de Israel: El hombre ó la mujer que cometiere alguno de
todos los pecados de los hombres, haciendo prevaricación contra Jehová,
y delinquiere aquella persona; \bibverse{7} Confesarán su pecado que
cometieron, y compensarán su ofensa enteramente, y añadirán su quinto
sobre ello, y lo darán á aquel contra quien pecaron. \bibverse{8} Y si
aquel hombre no tuviere pariente al cual sea resarcida la ofensa, daráse
la indemnización del agravio á Jehová, al sacerdote, á más del carnero
de las expiaciones, con el cual hará expiación por él. \bibverse{9} Y
toda ofrenda de todas las cosas santas que los hijos de Israel
presentaren al sacerdote, suya será. \footnote{\textbf{5:9} Núm 18,8}
\bibverse{10} Y lo santificado de cualquiera será suyo: asimismo lo que
cualquiera diere al sacerdote, suyo será.

\hypertarget{sacrificio-de-celo-y-agua-de-maldiciuxf3n-de-una-mujer-sospechosa-de-adulterio}{%
\subsection{Sacrificio de celo y agua de maldición de una mujer
sospechosa de
adulterio}\label{sacrificio-de-celo-y-agua-de-maldiciuxf3n-de-una-mujer-sospechosa-de-adulterio}}

\bibverse{11} Y Jehová habló á Moisés, diciendo: \bibverse{12} Habla á
los hijos de Israel, y diles: Cuando la mujer de alguno se desmandare, é
hiciere traición contra él, \bibverse{13} Que alguno se hubiere echado
con ella en carnal ayuntamiento, y su marido no lo hubiese visto por
haberse ella contaminado ocultamente, ni hubiere testigo contra ella, ni
ella hubiere sido cogida en el acto; \bibverse{14} Si viniere sobre él
espíritu de celo, y tuviere celos de su mujer, habiéndose ella
contaminado; ó viniere sobre él espíritu de celo, y tuviere celos de su
mujer, no habiéndose ella contaminado; \bibverse{15} Entonces el marido
traerá su mujer al sacerdote, y traerá su ofrenda con ella, la décima de
un epha de harina de cebada; no echará sobre ella aceite, ni pondrá
sobre ella incienso: porque es presente de celos, presente de
recordación, que trae en memoria pecado. \bibverse{16} Y el sacerdote la
hará acercar, y la hará poner delante de Jehová. \bibverse{17} Luego
tomará el sacerdote del agua santa en un vaso de barro: tomará también
el sacerdote del polvo que hubiere en el suelo del tabernáculo, y
echarálo en el agua. \bibverse{18} Y hará el sacerdote estar en pie á la
mujer delante de Jehová, y descubrirá la cabeza de la mujer, y pondrá
sobre sus manos el presente de la recordación, que es el presente de
celos: y el sacerdote tendrá en la mano las aguas amargas que acarrean
maldición. \bibverse{19} Y el sacerdote la conjurará, y le dirá: Si
ninguno hubiere dormido contigo, y si no te has apartado de tu marido á
inmundicia, libre seas de estas aguas amargas que traen maldición:
\bibverse{20} Mas si te has descarriado de tu marido, y te has
amancillado, y alguno hubiere tenido coito contigo, fuera de tu marido:
\bibverse{21} (El sacerdote conjurará á la mujer con juramento de
maldición, y dirá á la mujer): Jehová te dé en maldición y en
conjuración en medio de tu pueblo, haciendo Jehová á tu muslo que caiga,
y á tu vientre que se te hinche; \bibverse{22} Y estas aguas que dan
maldición entren en tus entrañas, y hagan hinchar tu vientre y caer tu
muslo. Y la mujer dirá: Amén, amén.

\bibverse{23} Y el sacerdote escribirá estas maldiciones en un libro, y
las borrará con las aguas amargas: \bibverse{24} Y dará á beber á la
mujer las aguas amargas que traen maldición; y las aguas que obran
maldición entrarán en ella por amargas. \bibverse{25} Después tomará el
sacerdote de la mano de la mujer el presente de los celos, y mecerálo
delante de Jehová, y lo ofrecerá delante del altar: \bibverse{26} Y
tomará el sacerdote un puñado del presente, en memoria de ella, y lo
quemará sobre el altar, y después dará á beber las aguas á la mujer.
\bibverse{27} Darále pues á beber las aguas; y será, que si fuere
inmunda y hubiere hecho traición contra su marido, las aguas que obran
maldición entrarán en ella en amargura, y su vientre se hinchará, y
caerá su muslo; y la mujer será por maldición en medio de su pueblo.
\bibverse{28} Mas si la mujer no fuere inmunda, sino que estuviere
limpia, ella será libre, y será fecunda.

\bibverse{29} Esta es la ley de los celos, cuando la mujer hiciere
traición á su marido, y se amancillare; \bibverse{30} O del marido,
sobre el cual pasare espíritu de celo, y tuviere celos de su mujer:
presentarála entonces delante de Jehová, y el sacerdote ejecutará en
ella toda esta ley. \bibverse{31} Y aquel varón será libre de iniquidad,
y la mujer llevará su pecado.

\hypertarget{normas-relativas-a-los-nazareos}{%
\subsection{Normas relativas a los
nazareos}\label{normas-relativas-a-los-nazareos}}

\hypertarget{section-5}{%
\section{6}\label{section-5}}

\bibverse{1} Y habló Jehová á Moisés, diciendo: \bibverse{2} Habla á los
hijos de Israel, y diles: El hombre, ó la mujer, cuando se apartare
haciendo voto de Nazareo, para dedicarse á Jehová, \footnote{\textbf{6:2}
  1Sam 1,11} \bibverse{3} Se abstendrá de vino y de sidra; vinagre de
vino, ni vinagre de sidra no beberá, ni beberá algún licor de uvas, ni
tampoco comerá uvas frescas ni secas. \footnote{\textbf{6:3} Luc 1,15}
\bibverse{4} Todo el tiempo de su nazareato, de todo lo que se hace de
vid de vino, desde los granillos hasta el hollejo, no comerá.

\bibverse{5} Todo el tiempo del voto de su nazareato no pasará navaja
sobre su cabeza, hasta que sean cumplidos los días de su apartamiento á
Jehová: santo será; dejará crecer las guedejas del cabello de su cabeza.
\footnote{\textbf{6:5} Jue 13,5}

\bibverse{6} Todo el tiempo que se apartare á Jehová, no entrará á
persona muerta. \bibverse{7} Por su padre, ni por su madre, por su
hermano, ni por su hermana, no se contaminará con ellos cuando murieren;
porque consagración de su Dios tiene sobre su cabeza. \bibverse{8} Todo
el tiempo de su nazareato, será santo á Jehová.

\hypertarget{regulaciones-relativas-a-la-contaminaciuxf3n-del-nazareo}{%
\subsection{Regulaciones relativas a la contaminación del
nazareo}\label{regulaciones-relativas-a-la-contaminaciuxf3n-del-nazareo}}

\bibverse{9} Y si alguno muriere muy de repente junto á él, contaminará
la cabeza de su nazareato; por tanto el día de su purificación raerá su
cabeza; al séptimo día la raerá. \footnote{\textbf{6:9} Núm 19,11}
\bibverse{10} Y el día octavo traerá dos tórtolas ó dos palominos al
sacerdote, á la puerta del tabernáculo del testimonio; \footnote{\textbf{6:10}
  Lev 5,7} \bibverse{11} Y el sacerdote hará el uno en expiación, y el
otro en holocausto: y expiarálo de lo que pecó sobre el muerto, y
santificará su cabeza en aquel día. \bibverse{12} Y consagrará á Jehová
los días de su nazareato, y traerá un cordero de un año en expiación por
la culpa; y los días primeros serán anulados, por cuanto fué contaminado
su nazareato.

\hypertarget{ordenanzas-sobre-la-ceremonia-del-sacrificio-al-final-del-nazareo}{%
\subsection{Ordenanzas sobre la ceremonia del sacrificio al final del
nazareo}\label{ordenanzas-sobre-la-ceremonia-del-sacrificio-al-final-del-nazareo}}

\bibverse{13} Esta es, pues, la ley del Nazareo el día que se cumpliere
el tiempo de su nazareato: Vendrá á la puerta del tabernáculo del
testimonio; \bibverse{14} Y ofrecerá su ofrenda á Jehová, un cordero de
un año sin tacha en holocausto, y una cordera de un año sin defecto en
expiación, y un carnero sin defecto por sacrificio de paces:
\bibverse{15} Además un canastillo de cenceñas, tortas de flor de harina
amasadas con aceite, y hojaldres cenceñas untadas con aceite, y su
presente, y sus libaciones. \bibverse{16} Y el sacerdote lo ofrecerá
delante de Jehová, y hará su expiación y su holocausto: \bibverse{17} Y
ofrecerá el carnero en sacrificio de paces á Jehová, con el canastillo
de las cenceñas; ofrecerá asimismo el sacerdote su presente, y sus
libaciones. \bibverse{18} Entonces el Nazareo raerá á la puerta del
tabernáculo del testimonio la cabeza de su nazareato, y tomará los
cabellos de la cabeza de su nazareato, y los pondrá sobre el fuego que
está debajo del sacrificio de las paces. \footnote{\textbf{6:18} Hech
  18,18} \bibverse{19} Después tomará el sacerdote la espaldilla cocida
del carnero, y una torta sin levadura del canastillo, y una hojaldre sin
levadura, y pondrálas sobre las manos del Nazareo, después que fuere
raído su nazareato: \bibverse{20} Y el sacerdote mecerá aquello, ofrenda
agitada delante de Jehová; lo cual será cosa santa del sacerdote, á más
del pecho mecido y de la espaldilla separada: y después podrá beber vino
el Nazareo.

\bibverse{21} Esta es la ley del Nazareo que hiciere voto de su ofrenda
á Jehová por su nazareato, á más de lo que su mano alcanzare: según el
voto que hiciere, así hará, conforme á la ley de su nazareato.

\hypertarget{orden-de-la-bendiciuxf3n-sacerdotal}{%
\subsection{Orden de la bendición
sacerdotal}\label{orden-de-la-bendiciuxf3n-sacerdotal}}

\bibverse{22} Y Jehová habló á Moisés, diciendo: \bibverse{23} Habla á
Aarón y á sus hijos, y diles: Así bendeciréis á los hijos de Israel,
diciéndoles: \footnote{\textbf{6:23} Lev 9,22-23} \bibverse{24} Jehová
te bendiga, y te guarde: \footnote{\textbf{6:24} Sal 121,-1}
\bibverse{25} Haga resplandecer Jehová su rostro sobre ti, y haya de ti
misericordia: \footnote{\textbf{6:25} Sal 80,4} \bibverse{26} Jehová
alce á ti su rostro, y ponga en ti paz. \footnote{\textbf{6:26} Sal
  69,17-18}

\bibverse{27} Y pondrán mi nombre sobre los hijos de Israel, y yo los
bendeciré.

\hypertarget{los-dones-de-consagraciuxf3n-de-los-jefes-tribales-para-el-santuario}{%
\subsection{Los dones de consagración de los jefes tribales para el
santuario}\label{los-dones-de-consagraciuxf3n-de-los-jefes-tribales-para-el-santuario}}

\hypertarget{section-6}{%
\section{7}\label{section-6}}

\bibverse{1} Y aconteció, que cuando Moisés hubo acabado de levantar el
tabernáculo, y ungídolo y santificádolo, con todos sus vasos; y asimismo
ungido y santificado el altar, con todos sus vasos; \footnote{\textbf{7:1}
  Éxod 40,9-10} \bibverse{2} Entonces los príncipes de Israel, las
cabezas de las casas de sus padres, los cuales eran los príncipes de las
tribus, que estaban sobre los contados, ofrecieron; \bibverse{3} Y
trajeron sus ofrendas delante de Jehová, seis carros cubiertos, y doce
bueyes; por cada dos príncipes un carro, y cada uno un buey; lo cual
ofrecieron delante del tabernáculo. \bibverse{4} Y Jehová habló á
Moisés, diciendo: \bibverse{5} Tómalo de ellos, y será para el servicio
del tabernáculo del testimonio: y lo darás á los Levitas, á cada uno
conforme á su ministerio.

\bibverse{6} Entonces Moisés recibió los carros y los bueyes, y diólos á
los Levitas. \bibverse{7} Dos carros y cuatro bueyes, dió á los hijos de
Gersón, conforme á su ministerio; \bibverse{8} Y á los hijos de Merari
dió los cuatro carros y ocho bueyes, conforme á su ministerio, bajo la
mano de Ithamar, hijo de Aarón el sacerdote. \bibverse{9} Y á los hijos
de Coath no dió; porque llevaban sobre sí en los hombros el servicio del
santuario.

\bibverse{10} Y ofrecieron los príncipes á la dedicación del altar el
día que fué ungido, ofrecieron los príncipes su ofrenda delante del
altar. \footnote{\textbf{7:10} 2Cró 7,9}

\bibverse{11} Y Jehová dijo á Moisés: Ofrecerán su ofrenda, un príncipe
un día, y otro príncipe otro día, á la dedicación del altar. \footnote{\textbf{7:11}
  Núm 1,4-16; Núm 2,3-29}

\bibverse{12} Y el que ofreció su ofrenda el primer día fué Naasón hijo
de Aminadab, de la tribu de Judá. \bibverse{13} Y fué su ofrenda un
plato de plata de peso de ciento y treinta siclos, y un jarro de plata
de setenta siclos, al siclo del santuario; ambos llenos de flor de
harina amasada con aceite para presente;

\bibverse{14} Una cuchara de oro de diez siclos, llena de perfume;

\bibverse{15} Un becerro, un carnero, un cordero de un año para
holocausto;

\bibverse{16} Un macho cabrío para expiación;

\bibverse{17} Y para sacrificio de paces, dos bueyes, cinco carneros,
cinco machos de cabrío, cinco corderos de un año. Esta fué la ofrenda de
Naasón, hijo de Aminadab.

\bibverse{18} El segundo día ofreció Nathanael hijo de Suar, príncipe de
Issachâr. \bibverse{19} Ofreció por su ofrenda un plato de plata de
ciento y treinta siclos de peso, un jarro de plata de setenta siclos, al
siclo del santuario; ambos llenos de flor de harina amasada con aceite
para presente;

\bibverse{20} Una cuchara de oro de diez siclos, llena de perfume;

\bibverse{21} Un becerro, un carnero, un cordero de un año para
holocausto;

\bibverse{22} Un macho cabrío para expiación;

\bibverse{23} Y para sacrificio de paces, dos bueyes, cinco carneros,
cinco machos de cabrío, cinco corderos de un año. Esta fué la ofrenda de
Nathanael, hijo de Suar.

\bibverse{24} El tercer día, Eliab hijo de Helón, príncipe de los hijos
de Zabulón: \bibverse{25} Y su ofrenda, un plato de plata de ciento y
treinta siclos de peso, un jarro de plata de setenta siclos, al siclo
del santuario; ambos llenos de flor de harina amasada con aceite para
presente;

\bibverse{26} Una cuchara de oro de diez siclos, llena de perfume;

\bibverse{27} Un becerro, un carnero, un cordero de un año para
holocausto;

\bibverse{28} Un macho cabrío para expiación;

\bibverse{29} Y para sacrificio de paces, dos bueyes, cinco carneros,
cinco machos de cabrío, cinco corderos de un año. Esta fué la ofrenda de
Eliab, hijo de Helón.

\bibverse{30} El cuarto día, Elisur hijo de Sedeur, príncipe de los
hijos de Rubén: \bibverse{31} Y su ofrenda, un plato de plata de ciento
y treinta siclos de peso, un jarro de plata de setenta siclos, al siclo
del santuario; ambos llenos de flor de harina amasada con aceite para
presente;

\bibverse{32} Una cuchara de oro de diez siclos, llena de perfume;

\bibverse{33} Un becerro, un carnero, un cordero de un año para
holocausto;

\bibverse{34} Un macho cabrío para expiación;

\bibverse{35} Y para sacrificio de paces, dos bueyes, cinco carneros,
cinco machos de cabrío, cinco corderos de un año. Esta fué la ofrenda de
Elisur, hijo de Sedeur.

\bibverse{36} El quinto día, Selumiel hijo de Zurisaddai, príncipe de
los hijos de Simeón: \bibverse{37} Y su ofrenda, un plato de plata de
ciento y treinta siclos de peso, un jarro de plata de setenta siclos, al
siclo del santuario; ambos llenos de flor de harina amasada con aceite
para presente;

\bibverse{38} Una cuchara de oro de diez siclos, llena de perfume;

\bibverse{39} Un becerro, un carnero, un cordero de un año para
holocausto;

\bibverse{40} Un macho cabrío para expiación;

\bibverse{41} Y para sacrificio de paces, dos bueyes, cinco carneros,
cinco machos de cabrío, cinco corderos de un año. Esta fué la ofrenda de
Selumiel, hijo de Zurisaddai.

\bibverse{42} El sexto día, Eliasaph hijo de Dehuel, príncipe de los
hijos de Gad: \bibverse{43} Y su ofrenda, un plato de plata de ciento y
treinta siclos de peso, un jarro de plata de setenta siclos, al siclo
del santuario; ambos llenos de flor de harina amasada con aceite para
presente;

\bibverse{44} Una cuchara de oro de diez siclos, llena de perfume;

\bibverse{45} Un becerro, un carnero, un cordero de un año para
holocausto;

\bibverse{46} Un macho cabrío para expiación;

\bibverse{47} Y para sacrificio de paces, dos bueyes, cinco carneros,
cinco machos de cabrío, cinco corderos de un año. Esta fué la ofrenda de
Eliasaph, hijo de Dehuel.

\bibverse{48} El séptimo día, el príncipe de los hijos de Ephraim,
Elisama hijo de Ammiud: \bibverse{49} Y su ofrenda, un plato de plata de
ciento y treinta siclos de peso, un jarro de plata de setenta siclos, al
siclo del santuario; ambos llenos de flor de harina amasada con aceite
para presente;

\bibverse{50} Una cuchara de oro de diez siclos, llena de perfume;

\bibverse{51} Un becerro, un carnero, un cordero de un año para
holocausto;

\bibverse{52} Un macho cabrío para expiación;

\bibverse{53} Y para sacrificio de paces, dos bueyes, cinco carneros,
cinco machos de cabrío, cinco corderos de un año. Esta fué la ofrenda de
Elisama, hijo de Ammiud.

\bibverse{54} El octavo día, el príncipe de los hijos de Manasés,
Gamaliel hijo de Pedasur: \bibverse{55} Y su ofrenda, un plato de plata
de ciento y treinta siclos de peso, un jarro de plata de setenta siclos,
al siclo del santuario; ambos llenos de flor de harina amasada con
aceite para presente;

\bibverse{56} Una cuchara de oro de diez siclos, llena de perfume;

\bibverse{57} Un becerro, un carnero, un cordero de un año para
holocausto;

\bibverse{58} Un macho cabrío para expiación;

\bibverse{59} Y para sacrificio de paces, dos bueyes, cinco carneros,
cinco machos de cabrío, cinco corderos de un año. Esta fué la ofrenda de
Gamaliel, hijo de Pedasur.

\bibverse{60} El noveno día, el príncipe de los hijos de Benjamín,
Abidán hijo de Gedeón: \bibverse{61} Y su ofrenda, un plato de plata de
ciento y treinta siclos de peso, un jarro de plata de setenta siclos, al
siclo del santuario; ambos llenos de flor de harina amasada con aceite
para presente;

\bibverse{62} Una cuchara de oro de diez siclos, llena de perfume;

\bibverse{63} Un becerro, un carnero, un cordero de un año para
holocausto;

\bibverse{64} Un macho cabrío para expiación;

\bibverse{65} Y para sacrificio de paces, dos bueyes, cinco carneros,
cinco machos de cabrío, cinco corderos de un año. Esta fué la ofrenda de
Abidán, hijo de Gedeón.

\bibverse{66} El décimo día, el príncipe de los hijos de Dan, Ahiezer
hijo de Ammisaddai: \bibverse{67} Y su ofrenda, un plato de plata de
ciento y treinta siclos de peso, un jarro de plata de setenta siclos, al
siclo del santuario; ambos llenos de flor de harina amasada con aceite
para presente;

\bibverse{68} Una cuchara de oro de diez siclos, llena de perfume;

\bibverse{69} Un becerro, un carnero, un cordero de un año para
holocausto;

\bibverse{70} Un macho cabrío para expiación;

\bibverse{71} Y para sacrificio de paces, dos bueyes, cinco carneros,
cinco machos de cabrío, cinco corderos de un año. Esta fué la ofrenda de
Ahiezer, hijo de Ammisaddai.

\bibverse{72} El undécimo día, el príncipe de los hijos de Aser, Pagiel
hijo de Ocrán: \bibverse{73} Y su ofrenda, un plato de plata de ciento y
treinta siclos de peso, un jarro de plata de setenta siclos, al siclo
del santuario; ambos llenos de flor de harina amasada con aceite para
presente;

\bibverse{74} Una cuchara de oro de diez siclos, llena de perfume;

\bibverse{75} Un becerro, un carnero, un cordero de un año para
holocausto;

\bibverse{76} Un macho cabrío para expiación;

\bibverse{77} Y para sacrificio de paces, dos bueyes, cinco carneros,
cinco machos de cabrío, cinco corderos de un año. Esta fué la ofrenda de
Pagiel, hijo de Ocrán.

\bibverse{78} El duodécimo día, el príncipe de los hijos de Nephtalí,
Ahira hijo de Enán: \bibverse{79} Y su ofrenda, un plato de plata de
ciento y treinta siclos de peso, un jarro de plata de setenta siclos, al
siclo del santuario; ambos llenos de flor de harina amasada con aceite
para presente;

\bibverse{80} Una cuchara de oro de diez siclos, llena de perfume;

\bibverse{81} Un becerro, un carnero, un cordero de un año para
holocausto;

\bibverse{82} Un macho cabrío para expiación;

\bibverse{83} Y para sacrificio de paces, dos bueyes, cinco carneros,
cinco machos de cabrío, cinco corderos de un año. Esta fué la ofrenda de
Ahira, hijo de Enán.

\bibverse{84} Esta fué la dedicación del altar, el día que fué ungido,
por los príncipes de Israel: doce platos de plata, doce jarros de plata,
doce cucharas de oro. \bibverse{85} Cada plato de ciento y treinta
siclos, cada jarro de setenta: toda la plata de los vasos, dos mil y
cuatrocientos siclos, al siclo del santuario. \bibverse{86} Las doce
cucharas de oro llenas de perfume, de diez siclos cada cuchara, al siclo
del santuario: todo el oro de las cucharas, ciento y veinte siclos.
\bibverse{87} Todos los bueyes para holocausto, doce becerros; doce los
carneros, doce los corderos de un año, con su presente: y doce los
machos de cabrío, para expiación. \bibverse{88} Y todos los bueyes del
sacrificio de las paces veinte y cuatro novillos, sesenta los carneros,
sesenta los machos de cabrío, sesenta los corderos de un año. Esta fué
la dedicación del altar, después que fué ungido.

\bibverse{89} Y cuando entraba Moisés en el tabernáculo del testimonio,
para hablar con El, oía la Voz que le hablaba de encima de la cubierta
que estaba sobre el arca del testimonio, de entre los dos querubines: y
hablaba con él. \footnote{\textbf{7:89} Éxod 25,21-22; 1Sam 3,3-14}

\hypertarget{las-siete-luxe1mparas-del-candelero}{%
\subsection{Las siete lámparas del
candelero}\label{las-siete-luxe1mparas-del-candelero}}

\hypertarget{section-7}{%
\section{8}\label{section-7}}

\bibverse{1} Y habló Jehová á Moisés, diciendo: \bibverse{2} Habla á
Aarón, y dile: Cuando encendieres las lámparas, las siete lámparas
alumbrarán frente á frente del candelero.

\bibverse{3} Y Aarón lo hizo así; que encendió enfrente del candelero
sus lámparas, como Jehová lo mandó á Moisés. \bibverse{4} Y esta era la
hechura del candelero: de oro labrado á martillo; desde su pie hasta sus
flores era labrado á martillo: conforme al modelo que Jehová mostró á
Moisés, así hizo el candelero.

\hypertarget{la-consagraciuxf3n-de-los-levitas-como-un-regalo-santo-a-dios}{%
\subsection{La consagración de los levitas como un regalo santo a
Dios}\label{la-consagraciuxf3n-de-los-levitas-como-un-regalo-santo-a-dios}}

\bibverse{5} Y Jehová habló á Moisés, diciendo: \bibverse{6} Toma á los
Levitas de entre los hijos de Israel, y expíalos. \footnote{\textbf{8:6}
  Mal 3,3} \bibverse{7} Y así les harás para expiarlos: rocía sobre
ellos el agua de la expiación, y haz pasar la navaja sobre toda su
carne, y lavarán sus vestidos, y serán expiados. \footnote{\textbf{8:7}
  Núm 5,17; Núm 19,9; Núm 19,17; Lev 14,8} \bibverse{8} Luego tomarán un
novillo, con su presente de flor de harina amasada con aceite; y tomarás
otro novillo para expiación. \bibverse{9} Y harás llegar los Levitas
delante del tabernáculo del testimonio, y juntarás toda la congregación
de los hijos de Israel; \bibverse{10} Y cuando habrás hecho llegar los
Levitas delante de Jehová, pondrán los hijos de Israel sus manos sobre
los Levitas; \bibverse{11} Y ofrecerá Aarón los Levitas delante de
Jehová en ofrenda de los hijos de Israel, y servirán en el ministerio de
Jehová. \footnote{\textbf{8:11} Núm 8,21}

\bibverse{12} Y los Levitas pondrán sus manos sobre las cabezas de los
novillos: y ofrecerás el uno por expiación, y el otro en holocausto á
Jehová, para expiar los Levitas. \bibverse{13} Y harás presentar los
Levitas delante de Aarón, y delante de sus hijos, y los ofrecerás en
ofrenda á Jehová. \bibverse{14} Así apartarás los Levitas de entre los
hijos de Israel; y serán míos los Levitas.

\bibverse{15} Y después de eso vendrán los Levitas á ministrar en el
tabernáculo del testimonio: los expiarás pues, y los ofrecerás en
ofrenda. \bibverse{16} Porque enteramente me son á mí dados los Levitas
de entre los hijos de Israel, en lugar de todo aquel que abre matriz;
helos tomado para mí en lugar de los primogénitos de todos los hijos de
Israel. \footnote{\textbf{8:16} Núm 3,12} \bibverse{17} Porque mío es
todo primogénito en los hijos de Israel, así de hombres como de
animales; desde el día que yo herí todo primogénito en la tierra de
Egipto, los santifiqué para mí. \footnote{\textbf{8:17} Éxod 13,2}
\bibverse{18} Y he tomado los Levitas en lugar de todos los primogénitos
en los hijos de Israel. \bibverse{19} Y yo he dado en don los Levitas á
Aarón y á sus hijos de entre los hijos de Israel, para que sirvan el
ministerio de los hijos de Israel en el tabernáculo del testimonio, y
reconcilien á los hijos de Israel; porque no haya plaga en los hijos de
Israel, llegando los hijos de Israel al santuario. \footnote{\textbf{8:19}
  Núm 3,9}

\bibverse{20} Y Moisés, y Aarón, y toda la congregación de los hijos de
Israel, hicieron de los Levitas conforme á todas las cosas que mandó
Jehová á Moisés acerca de los Levitas; así hicieron de ellos los hijos
de Israel. \bibverse{21} Y los Levitas se purificaron, y lavaron sus
vestidos; y Aarón los ofreció en ofrenda delante de Jehová, é hizo Aarón
expiación por ellos para purificarlos. \bibverse{22} Y así vinieron
después los Levitas para servir en su ministerio en el tabernáculo del
testimonio, delante de Aarón y delante de sus hijos: de la manera que
mandó Jehová á Moisés acerca de los Levitas, así hicieron con ellos.

\hypertarget{el-tiempo-del-deber-de-los-levitas}{%
\subsection{El tiempo del deber de los
levitas}\label{el-tiempo-del-deber-de-los-levitas}}

\bibverse{23} Y habló Jehová á Moisés, diciendo: \bibverse{24} Esto
cuanto á los Levitas: de veinte y cinco años arriba entrarán á hacer su
oficio en el servicio del tabernáculo del testimonio: \footnote{\textbf{8:24}
  Núm 4,3; Núm 4,23; Núm 4,30; Núm 4,47}

\bibverse{25} Mas desde los cincuenta años volverán del oficio de su
ministerio, y nunca más servirán: \bibverse{26} Pero servirán con sus
hermanos en el tabernáculo del testimonio, para hacer la guarda, bien
que no servirán en el ministerio. Así harás de los Levitas cuanto á sus
oficios.

\hypertarget{la-celebraciuxf3n-posterior-a-la-pascua-para-los-inmundos-y-los-viajeros-la-pascua-de-los-extrauxf1os}{%
\subsection{La celebración posterior a la Pascua para los inmundos y los
viajeros; la pascua de los
extraños}\label{la-celebraciuxf3n-posterior-a-la-pascua-para-los-inmundos-y-los-viajeros-la-pascua-de-los-extrauxf1os}}

\hypertarget{section-8}{%
\section{9}\label{section-8}}

\bibverse{1} Y habló Jehová á Moisés en el desierto de Sinaí, en el
segundo año de su salida de la tierra de Egipto, en el mes primero,
diciendo: \bibverse{2} Los hijos de Israel harán la pascua á su tiempo.
\bibverse{3} El décimocuarto día de este mes, entre las dos tardes, la
haréis á su tiempo: conforme á todos sus ritos, y conforme á todas sus
leyes la haréis.

\bibverse{4} Y habló Moisés á los hijos de Israel, para que hiciesen la
pascua. \bibverse{5} E hicieron la pascua en el mes primero, á los
catorce días del mes, entre las dos tardes, en el desierto de Sinaí:
conforme á todas las cosas que mandó Jehová á Moisés, así hicieron los
hijos de Israel. \bibverse{6} Y hubo algunos que estaban inmundos á
causa de muerto, y no pudieron hacer la pascua aquel día; y llegaron
delante de Moisés y delante de Aarón aquel día; \footnote{\textbf{9:6}
  Núm 19,11} \bibverse{7} Y dijéronle aquellos hombres: Nosotros somos
inmundos por causa de muerto; ¿por qué seremos impedidos de ofrecer
ofrenda á Jehová á su tiempo entre los hijos de Israel?

\bibverse{8} Y Moisés les respondió: Esperad, y oiré qué mandará Jehová
acerca de vosotros.

\bibverse{9} Y Jehová habló á Moisés, diciendo: \bibverse{10} Habla á
los hijos de Israel, diciendo: Cualquiera de vosotros ó de vuestras
generaciones, que fuere inmundo por causa de muerto ó estuviere de viaje
lejos, hará pascua á Jehová: \bibverse{11} En el mes segundo, á los
catorce días del mes, entre las dos tardes, la harán: con cenceñas y
hierbas amargas la comerán; \bibverse{12} No dejarán de él para la
mañana, ni quebrarán hueso en él: conforme á todos los ritos de la
pascua la harán. \bibverse{13} Mas el que estuviere limpio, y no
estuviere de viaje, si dejare de hacer la pascua, la tal persona será
cortada de sus pueblos: por cuanto no ofreció á su tiempo la ofrenda de
Jehová, el tal hombre llevará su pecado.

\bibverse{14} Y si morare con vosotros peregrino, é hiciere la pascua á
Jehová, conforme al rito de la pascua y conforme á sus leyes así la
hará: un mismo rito tendréis, así el peregrino como el natural de la
tierra.

\hypertarget{la-apariciuxf3n-de-la-columna-de-nubes-y-fuego-sobre-el-santuario}{%
\subsection{La aparición de la columna de nubes y fuego sobre el
santuario}\label{la-apariciuxf3n-de-la-columna-de-nubes-y-fuego-sobre-el-santuario}}

\bibverse{15} Y el día que el tabernáculo fué levantado, la nube cubrió
el tabernáculo sobre la tienda del testimonio; y á la tarde había sobre
el tabernáculo como una apariencia de fuego, hasta la mañana.
\bibverse{16} Así era continuamente: la nube lo cubría, y de noche la
apariencia de fuego. \bibverse{17} Y según que se alzaba la nube del
tabernáculo, los hijos de Israel se partían: y en el lugar donde la nube
paraba, allí alojaban los hijos de Israel. \bibverse{18} Al mandato de
Jehová los hijos de Israel se partían: y al mandato de Jehová asentaban
el campo: todos los días que la nube estaba sobre el tabernáculo, ellos
estaban quedos. \bibverse{19} Y cuando la nube se detenía sobre el
tabernáculo muchos días, entonces los hijos de Israel guardaban la
ordenanza de Jehová y no partían. \bibverse{20} Y cuando sucedía que la
nube estaba sobre el tabernáculo pocos días, al dicho de Jehová
alojaban, y al dicho de Jehová partían. \bibverse{21} Y cuando era que
la nube se detenía desde la tarde hasta la mañana, cuando á la mañana la
nube se levantaba, ellos partían: ó si había estado el día, y á la noche
la nube se levantaba, entonces partían. \bibverse{22} O si dos días, ó
un mes, ó un año, mientras la nube se detenía sobre el tabernáculo
quedándose sobre él, los hijos de Israel se estaban acampados, y no
movían: mas cuando ella se alzaba, ellos movían. \bibverse{23} Al dicho
de Jehová asentaban, y al dicho de Jehová partían, guardando la
ordenanza de Jehová, como lo había Jehová dicho por medio de Moisés.

\hypertarget{ordenanza-sobre-dos-trompetas-de-plata}{%
\subsection{Ordenanza sobre dos trompetas de
plata}\label{ordenanza-sobre-dos-trompetas-de-plata}}

\hypertarget{section-9}{%
\section{10}\label{section-9}}

\bibverse{1} Y jehová habló á Moisés, diciendo: \bibverse{2} Hazte dos
trompetas de plata; de obra de martillo las harás, las cuales te
servirán para convocar la congregación, y para hacer mover el campo.
\footnote{\textbf{10:2} Núm 31,6} \bibverse{3} Y cuando las tocaren,
toda la congregación se juntará á ti á la puerta del tabernáculo del
testimonio. \bibverse{4} Mas cuando tocaren sólo la una, entonces se
congregarán á ti los príncipes, las cabezas de los millares de Israel.
\bibverse{5} Y cuando tocareis alarma, entonces moverán el campo de los
que están alojados al oriente. \bibverse{6} Y cuando tocareis alarma la
segunda vez, entonces moverán el campo de los que están alojados al
mediodía: alarma tocarán á sus partidas. \bibverse{7} Empero cuando
hubiereis de juntar la congregación, tocaréis, mas no con sonido de
alarma.

\bibverse{8} Y los hijos de Aarón, los sacerdotes, tocarán las
trompetas; y las tendréis por estatuto perpetuo por vuestras
generaciones. \bibverse{9} Y cuando viniereis á la guerra en vuestra
tierra contra el enemigo que os molestare, tocaréis alarma con las
trompetas: y seréis en memoria delante de Jehová vuestro Dios, y seréis
salvos de vuestros enemigos.

\bibverse{10} Y en el día de vuestra alegría, y en vuestras
solemnidades, y en los principios de vuestros meses, tocaréis las
trompetas sobre vuestros holocaustos, y sobre los sacrificios de
vuestras paces, y os serán por memoria delante de vuestro Dios: Yo
Jehová vuestro Dios.

\hypertarget{salida-del-sinauxed-hacia-el-desierto-de-paran}{%
\subsection{Salida del Sinaí hacia el desierto de
Paran}\label{salida-del-sinauxed-hacia-el-desierto-de-paran}}

\bibverse{11} Y fué en el año segundo, en el mes segundo, á los veinte
del mes, que la nube se alzó del tabernáculo del testimonio.
\bibverse{12} Y movieron los hijos de Israel por sus partidas del
desierto de Sinaí; y paró la nube en el desierto de Parán.

\hypertarget{descripciuxf3n-del-pedido-de-tren}{%
\subsection{Descripción del pedido de
tren}\label{descripciuxf3n-del-pedido-de-tren}}

\bibverse{13} Y movieron la primera vez al dicho de Jehová por mano de
Moisés. \footnote{\textbf{10:13} Núm 1,1-4} \bibverse{14} Y la bandera
del campo de los hijos de Judá comenzó á marchar primero, por sus
escuadrones: y Naasón, hijo de Aminadab, era sobre su ejército.
\bibverse{15} Y sobre el ejército de la tribu de los hijos de Issachâr,
Nathanael hijo de Suar. \bibverse{16} Y sobre el ejército de la tribu de
los hijos de Zabulón, Eliab hijo de Helón. \bibverse{17} Y después que
estaba ya desarmado el tabernáculo, movieron los hijos de Gersón y los
hijos de Merari, que lo llevaban. \bibverse{18} Luego comenzó á marchar
la bandera del campo de Rubén por sus escuadrones: y Elisur, hijo de
Sedeur, era sobre su ejército. \bibverse{19} Y sobre el ejército de la
tribu de los hijos de Simeón, Selumiel hijo de Zurisaddai. \bibverse{20}
Y sobre el ejército de la tribu de los hijos de Gad, Eliasaph hijo de
Dehuel.

\bibverse{21} Luego comenzaron á marchar los Coathitas llevando el
santuario; y entre tanto que ellos llegaban, los otros acondicionaron el
tabernáculo.

\bibverse{22} Después comenzó á marchar la bandera del campo de los
hijos de Ephraim por sus escuadrones: y Elisama, hijo de Ammiud, era
sobre su ejército. \bibverse{23} Y sobre el ejército de la tribu de los
hijos de Manasés, Gamaliel hijo de Pedasur. \bibverse{24} Y sobre el
ejército de la tribu de los hijos de Benjamín, Abidán hijo de Gedeón.

\bibverse{25} Luego comenzó á marchar la bandera del campo de los hijos
de Dan por sus escuadrones, recogiendo todos los campos: y Ahiezer, hijo
de Ammisaddai, era sobre su ejército. \bibverse{26} Y sobre el ejército
de la tribu de los hijos de Aser, Pagiel hijo de Ocrán. \bibverse{27} Y
sobre el ejército de la tribu de los hijos de Nephtalí, Ahira hijo de
Enán. \bibverse{28} Estas son las partidas de los hijos de Israel por
sus ejércitos, cuando se movían.

\hypertarget{moisuxe9s-intenta-ganarse-a-su-cuuxf1ado-hobab-como-guuxeda-para-el-viaje-hacia-adelante}{%
\subsection{Moisés intenta ganarse a su cuñado Hobab como guía para el
viaje hacia
adelante}\label{moisuxe9s-intenta-ganarse-a-su-cuuxf1ado-hobab-como-guuxeda-para-el-viaje-hacia-adelante}}

\bibverse{29} Entonces dijo Moisés á Hobab, hijo de Ragüel Madianita, su
suegro: Nosotros nos partimos para el lugar del cual Jehová ha dicho: Yo
os lo daré. Ven con nosotros, y te haremos bien: porque Jehová ha
hablado bien respecto á Israel.

\bibverse{30} Y él le respondió: Yo no iré, sino que me marcharé á mi
tierra y á mi parentela.

\bibverse{31} Y él le dijo: Ruégote que no nos dejes; porque tú sabes
nuestros alojamientos en el desierto, y nos serás en lugar de ojos.
\bibverse{32} Y será, que si vinieres con nosotros, cuando tuviéremos el
bien que Jehová nos ha de hacer, nosotros te haremos bien.

\hypertarget{la-partida-del-monte-de-dios-bajo-la-guuxeda-del-arca}{%
\subsection{La partida del monte de Dios bajo la guía del
arca}\label{la-partida-del-monte-de-dios-bajo-la-guuxeda-del-arca}}

\bibverse{33} Así partieron del monte de Jehová camino de tres días; y
el arca de la alianza de Jehová fué delante de ellos camino de tres
días, buscándoles lugar de descanso. \bibverse{34} Y la nube de Jehová
iba sobre ellos de día, desde que partieron del campo. \footnote{\textbf{10:34}
  Éxod 13,21} \bibverse{35} Y fué, que en moviendo el arca, Moisés
decía: Levántate, Jehová, y sean disipados tus enemigos, y huyan de tu
presencia los que te aborrecen. \footnote{\textbf{10:35} Sal 68,2; Sal
  132,8}

\bibverse{36} Y cuando ella asentaba, decía: Vuelve, Jehová, á los
millares de millares de Israel.

\hypertarget{el-murmullo-de-la-gente-y-la-fogata-en-thabera}{%
\subsection{El murmullo de la gente y la fogata en
Thabera}\label{el-murmullo-de-la-gente-y-la-fogata-en-thabera}}

\hypertarget{section-10}{%
\section{11}\label{section-10}}

\bibverse{1} Y aconteció que el pueblo se quejó á oídos de Jehová: y
oyólo Jehová, y enardecióse su furor, y encendióse en ellos fuego de
Jehová y consumió el un cabo del campo. \footnote{\textbf{11:1} Lev 10,2}
\bibverse{2} Entonces el pueblo dió voces á Moisés, y Moisés oró á
Jehová, y soterróse el fuego. \bibverse{3} Y llamó á aquel lugar
Taberah; porque el fuego de Jehová se encendió en ellos.

\hypertarget{la-gente-se-queja-de-la-comida}{%
\subsection{La gente se queja de la
comida}\label{la-gente-se-queja-de-la-comida}}

\bibverse{4} Y el vulgo que había en medio tuvo un vivo deseo, y
volvieron, y aun lloraron los hijos de Israel, y dijeron: ¡Quién nos
diera á comer carne! \bibverse{5} Nos acordamos del pescado que comíamos
en Egipto de balde, de los cohombros, y de los melones, y de los
puerros, y de las cebollas, y de los ajos: \bibverse{6} Y ahora nuestra
alma se seca; que nada sino maná ven nuestros ojos. \bibverse{7} Y era
el maná como semilla de culantro, y su color como color de bdelio.
\footnote{\textbf{11:7} Éxod 16,14-31} \bibverse{8} Derramábase el
pueblo, y recogían, y molían en molinos, ó majaban en morteros, y lo
cocían en caldera, ó hacían de él tortas: y su sabor era como sabor de
aceite nuevo. \bibverse{9} Y cuando descendía el rocío sobre el real de
noche, el maná descendía de sobre él.

\hypertarget{el-lamento-de-moisuxe9s-ante-dios}{%
\subsection{El lamento de Moisés ante
Dios}\label{el-lamento-de-moisuxe9s-ante-dios}}

\bibverse{10} Y oyó Moisés al pueblo, que lloraba por sus familias, cada
uno á la puerta de su tienda: y el furor de Jehová se encendió en gran
manera; también pareció mal á Moisés. \bibverse{11} Y dijo Moisés á
Jehová: ¿Por qué has hecho mal á tu siervo? ¿y por qué no he hallado
gracia en tus ojos, que has puesto la carga de todo este pueblo sobre
mí? \bibverse{12} ¿Concebí yo á todo este pueblo? ¿engendrélo yo, para
que me digas: Llévalo en tu seno, como lleva la que cría al que mama, á
la tierra de la cual juraste á sus padres? \bibverse{13} ¿De dónde tengo
yo carne para dar á todo este pueblo? porque lloran á mí, diciendo:
Danos carne que comamos. \bibverse{14} No puedo yo solo soportar á todo
este pueblo, que me es pesado en demasía. \bibverse{15} Y si así lo
haces tú conmigo, yo te ruego que me des muerte, si he hallado gracia en
tus ojos; y que yo no vea mi mal.

\hypertarget{ordenanza-de-dios-nombramiento-de-setenta-asistentes-de-moisuxe9s-la-promesa-de-dios-de-donaciuxf3n-de-carne-la-respuesta-incruxe9dula-de-moisuxe9s}{%
\subsection{Ordenanza de Dios (nombramiento de setenta asistentes de
Moisés); La promesa de Dios de donación de carne; la respuesta incrédula
de
Moisés}\label{ordenanza-de-dios-nombramiento-de-setenta-asistentes-de-moisuxe9s-la-promesa-de-dios-de-donaciuxf3n-de-carne-la-respuesta-incruxe9dula-de-moisuxe9s}}

\bibverse{16} Entonces Jehová dijo á Moisés: Júntame setenta varones de
los ancianos de Israel, que tú sabes que son ancianos del pueblo y sus
principales; y tráelos á la puerta del tabernáculo del testimonio, y
esperen allí contigo. \footnote{\textbf{11:16} Éxod 18,21; Éxod 24,1}
\bibverse{17} Y yo descenderé y hablaré allí contigo; y tomaré del
espíritu que está en ti, y pondré en ellos; y llevarán contigo la carga
del pueblo, y no la llevarás tú solo.

\bibverse{18} Empero dirás al pueblo: Santificaos para mañana, y
comeréis carne: pues que habéis llorado en oídos de Jehová, diciendo:
¡Quién nos diera á comer carne! ¡cierto mejor nos iba en Egipto! Jehová,
pues, os dará carne, y comeréis. \bibverse{19} No comeréis un día, ni
dos días, ni cinco días, ni diez días, ni veinte días; \bibverse{20}
Sino hasta un mes de tiempo, hasta que os salga por las narices, y os
sea en aborrecimiento: por cuanto menospreciasteis á Jehová que está en
medio de vosotros, y llorasteis delante de él, diciendo: ¿Para qué
salimos acá de Egipto?

\bibverse{21} Entonces dijo Moisés: Seiscientos mil de á pie es el
pueblo en medio del cual yo estoy; y tú dices: Les daré carne, y comerán
el tiempo de un mes. \bibverse{22} ¿Se han de degollar para ellos ovejas
y bueyes que les basten? ¿ó se juntarán para ellos todos los peces de la
mar para que tengan abasto? \footnote{\textbf{11:22} Juan 6,7}

\bibverse{23} Entonces Jehová respondió á Moisés: ¿Hase acortado la mano
de Jehová? ahora verás si te sucede mi dicho, ó no. \footnote{\textbf{11:23}
  Is 50,2; Is 59,1}

\hypertarget{el-entusiasmo-profuxe9tico-de-los-setenta-ancianos}{%
\subsection{El entusiasmo profético de los setenta
ancianos}\label{el-entusiasmo-profuxe9tico-de-los-setenta-ancianos}}

\bibverse{24} Y salió Moisés, y dijo al pueblo las palabras de Jehová: y
juntó los setenta varones de los ancianos del pueblo, é hízolos estar
alrededor del tabernáculo. \bibverse{25} Entonces Jehová descendió en la
nube, y hablóle; y tomó del espíritu que estaba en él, y púsolo en los
setenta varones ancianos; y fué que, cuando posó sobre ellos el
espíritu, profetizaron, y no cesaron. \bibverse{26} Y habían quedado en
el campo dos varones, llamado el uno Eldad y el otro Medad, sobre los
cuales también reposó el espíritu: estaban estos entre los escritos, mas
no habían salido al tabernáculo; y profetizaron en el campo.
\bibverse{27} Y corrió un mozo, y dió aviso á Moisés, y dijo: Eldad y
Medad profetizan en el campo.

\bibverse{28} Entonces respondió Josué hijo de Nun, ministro de Moisés,
uno de sus mancebos, y dijo: Señor mío Moisés, impídelos. \footnote{\textbf{11:28}
  Núm 13,16; Éxod 24,13}

\bibverse{29} Y Moisés le respondió: ¿Tienes tú celos por mí? mas ojalá
que todo el pueblo de Jehová fuesen profetas, que Jehová pusiera su
espíritu sobre ellos. \footnote{\textbf{11:29} Mar 9,39; Jl 3,1}

\bibverse{30} Y recogióse Moisés al campo, él y los ancianos de Israel.

\hypertarget{alimentaciuxf3n-de-codornices-juicio-de-dios-las-tumbas-del-placer}{%
\subsection{Alimentación de codornices; Juicio de Dios; las tumbas del
placer}\label{alimentaciuxf3n-de-codornices-juicio-de-dios-las-tumbas-del-placer}}

\bibverse{31} Y salió un viento de Jehová, y trajo codornices de la mar,
y dejólas sobre el real, un día de camino de la una parte, y un día de
camino de la otra, en derredor del campo, y casi dos codos sobre la haz
de la tierra. \footnote{\textbf{11:31} Éxod 16,13} \bibverse{32}
Entonces el pueblo estuvo levantado todo aquel día, y toda la noche, y
todo el día siguiente, y recogiéronse codornices: el que menos, recogió
diez montones; y las tendieron para sí á lo largo en derredor del campo.
\bibverse{33} Aun estaba la carne entre los dientes de ellos, antes que
fuese mascada, cuando el furor de Jehová se encendió en el pueblo, é
hirió Jehová al pueblo con una muy grande plaga. \bibverse{34} Y llamó
el nombre de aquel lugar Kibroth-hattaavah, por cuanto allí sepultaron
al pueblo codicioso.

\bibverse{35} De Kibroth-hattaavah movió el pueblo á Haseroth, y pararon
en Haseroth.

\hypertarget{la-rebeliuxf3n-de-miriam-y-aaruxf3n-contra-moisuxe9s}{%
\subsection{La rebelión de Miriam y Aarón contra
Moisés}\label{la-rebeliuxf3n-de-miriam-y-aaruxf3n-contra-moisuxe9s}}

\hypertarget{section-11}{%
\section{12}\label{section-11}}

\bibverse{1} Y hablaron María y Aarón contra Moisés á causa de la mujer
Ethiope que había tomado: porque él había tomado mujer Ethiope.
\footnote{\textbf{12:1} Éxod 2,21} \bibverse{2} Y dijeron: ¿Solamente
por Moisés ha hablado Jehová? ¿no ha hablado también por nosotros? Y
oyólo Jehová. \bibverse{3} Y aquel varón Moisés era muy manso, más que
todos los hombres que había sobre la tierra.

\hypertarget{dios-estuxe1-defendiendo-a-moisuxe9s-el-castigo-de-miriam}{%
\subsection{Dios está defendiendo a Moisés; El castigo de
miriam}\label{dios-estuxe1-defendiendo-a-moisuxe9s-el-castigo-de-miriam}}

\bibverse{4} Y luego dijo Jehová á Moisés, y á Aarón, y á María: Salid
vosotros tres al tabernáculo del testimonio. Y salieron ellos tres.

\bibverse{5} Entonces Jehová descendió en la columna de la nube, y
púsose á la puerta del tabernáculo, y llamó á Aarón y á María; y
salieron ellos ambos. \bibverse{6} Y él les dijo: Oid ahora mis
palabras: si tuviereis profeta de Jehová, le apareceré en visión, en
sueños hablaré con él. \bibverse{7} No así á mi siervo Moisés, que es
fiel en toda mi casa: \footnote{\textbf{12:7} Heb 3,2} \bibverse{8} Boca
á boca hablaré con él, y á las claras, y no por figuras; y verá la
apariencia de Jehová: ¿por qué pues no tuvisteis temor de hablar contra
mi siervo Moisés? \footnote{\textbf{12:8} Éxod 33,11; Éxod 33,23}
\bibverse{9} Entonces el furor de Jehová se encendió en ellos; y fuése.

\bibverse{10} Y la nube se apartó del tabernáculo: y he aquí que María
era leprosa como la nieve; y miró Aarón á María, y he aquí que estaba
leprosa. \footnote{\textbf{12:10} Deut 24,9}

\hypertarget{la-intercesiuxf3n-de-aaruxf3n-y-moisuxe9s-la-respuesta-de-dios-la-curaciuxf3n-de-miriam-llegada-al-desierto-de-paran}{%
\subsection{La intercesión de Aarón y Moisés; La respuesta de Dios; La
curación de Miriam; Llegada al desierto de
Paran}\label{la-intercesiuxf3n-de-aaruxf3n-y-moisuxe9s-la-respuesta-de-dios-la-curaciuxf3n-de-miriam-llegada-al-desierto-de-paran}}

\bibverse{11} Y dijo Aarón á Moisés: ¡Ah! señor mío, no pongas ahora
sobre nosotros pecado; porque locamente lo hemos hecho, y hemos pecado.
\bibverse{12} No sea ella ahora como el que sale muerto del vientre de
su madre, consumida la mitad de su carne.

\bibverse{13} Entonces Moisés clamó á Jehová, diciendo: Ruégote, oh
Dios, que la sanes ahora.

\bibverse{14} Respondió Jehová á Moisés: Pues si su padre hubiera
escupido en su cara, ¿no se avergonzaría por siete días?: sea echada
fuera del real por siete días, y después se reunirá. \footnote{\textbf{12:14}
  Lev 13,46}

\bibverse{15} Así María fué echada del real siete días; y el pueblo no
pasó adelante hasta que se le reunió María. \bibverse{16}

\hypertarget{envuxedo-de-los-doce-exploradores}{%
\subsection{Envío de los doce
exploradores}\label{envuxedo-de-los-doce-exploradores}}

\hypertarget{section-12}{%
\section{13}\label{section-12}}

\bibverse{1} Y después movió el pueblo de Haseroth, y asentaron el campo
en el desierto de Parán. \bibverse{2} Y Jehová habló á Moisés, diciendo:

\bibverse{3} Envía tú hombres que reconozcan la tierra de Canaán, la
cual yo doy á los hijos de Israel: de cada tribu de sus padres enviaréis
un varón, cada uno príncipe entre ellos. \bibverse{4} Y Moisés los envió
desde el desierto de Parán, conforme á la palabra de Jehová: y todos
aquellos varones eran príncipes de los hijos de Israel. \bibverse{5} Los
nombres de los cuales son éstos: De la tribu de Rubén, Sammua hijo de
Zaccur. \bibverse{6} De la tribu de Simeón, Saphat hijo de Hurí.
\bibverse{7} De la tribu de Judá, Caleb hijo de Jephone. \bibverse{8} De
la tribu de Issachâr, Igal hijo de Joseph. \bibverse{9} De la tribu de
Ephraim, Oseas hijo de Nun. \footnote{\textbf{13:9} Núm 13,16; 1Cró 7,27}
\bibverse{10} De la tribu de Benjamín, Palti hijo de Raphu.
\bibverse{11} De la tribu de Zabulón, Gaddiel hijo de Sodi.
\bibverse{12} De la tribu de José, de la tribu de Manasés, Gaddi hijo de
Susi. \bibverse{13} De la tribu de Dan, Ammiel hijo de Gemalli.
\bibverse{14} De la tribu de Aser, Sethur hijo de Michâel. \bibverse{15}
De la tribu de Nephtalí, Nahabí hijo de Vapsi. \bibverse{16} De la tribu
de Gad, Geuel hijo de Machî. \bibverse{17} Estos son los nombres de los
varones que Moisés envió á reconocer la tierra: y á Oseas hijo de Nun,
le puso Moisés el nombre de Josué.

\hypertarget{la-instrucciuxf3n-de-moisuxe9s-a-los-espuxedas}{%
\subsection{La instrucción de Moisés a los
espías}\label{la-instrucciuxf3n-de-moisuxe9s-a-los-espuxedas}}

\bibverse{18} Enviólos, pues, Moisés á reconocer la tierra de Canaán,
diciéndoles: Subid por aquí, por el mediodía, y subid al monte:
\bibverse{19} Y observad la tierra qué tal es; y el pueblo que la
habita, si es fuerte ó débil, si poco ó numeroso; \bibverse{20} Qué tal
la tierra habitada, si es buena ó mala; y qué tales son las ciudades
habitadas, si de tiendas ó de fortalezas; \bibverse{21} Y cuál sea el
terreno, si es pingüe ó flaco, si en él hay ó no árboles: y esforzaos, y
coged del fruto del país. Y el tiempo era el tiempo de las primeras
uvas.

\hypertarget{explorando-la-tierra}{%
\subsection{Explorando la tierra}\label{explorando-la-tierra}}

\bibverse{22} Y ellos subieron, y reconocieron la tierra desde el
desierto de Zin hasta Rehob, entrando en Emath. \bibverse{23} Y subieron
por el mediodía, y vinieron hasta Hebrón: y allí estaban Aimán, y Sesai,
y Talmai, hijos de Anac. Hebrón fué edificada siete años antes de Zoán,
la de Egipto. \bibverse{24} Y llegaron hasta el arroyo de Escol, y de
allí cortaron un sarmiento con un racimo de uvas, el cual trajeron dos
en un palo, y de las granadas y de los higos. \bibverse{25} Y llamóse
aquel lugar Nahal-escol, por el racimo que cortaron de allí los hijos de
Israel.

\hypertarget{regreso-e-informe-de-los-emisarios}{%
\subsection{Regreso e informe de los
emisarios}\label{regreso-e-informe-de-los-emisarios}}

\bibverse{26} Y volvieron de reconocer la tierra al cabo de cuarenta
días. \bibverse{27} Y anduvieron y vinieron á Moisés y á Aarón, y á toda
la congregación de los hijos de Israel, en el desierto de Parán, en
Cades, y diéronles la respuesta, y á toda la congregación, y les
mostraron el fruto de la tierra. \bibverse{28} Y le contaron, y dijeron:
Nosotros llegamos á la tierra á la cual nos enviaste, la que ciertamente
fluye leche y miel; y este es el fruto de ella. \footnote{\textbf{13:28}
  Éxod 3,8; Éxod 3,17} \bibverse{29} Mas el pueblo que habita aquella
tierra es fuerte, y las ciudades muy grandes y fuertes; y también vimos
allí los hijos de Anac.

\bibverse{30} Amalec habita la tierra del mediodía; y el Hetheo, y el
Jebuseo, y el Amorrheo, habitan en el monte; y el Cananeo habita junto á
la mar, y á la ribera del Jordán.

\hypertarget{las-palabras-tranquilizadoras-de-caleb-y-las-palabras-desalentadoras-de-los-otros-exploradores}{%
\subsection{Las palabras tranquilizadoras de Caleb y las palabras
desalentadoras de los otros
exploradores}\label{las-palabras-tranquilizadoras-de-caleb-y-las-palabras-desalentadoras-de-los-otros-exploradores}}

\bibverse{31} Entonces Caleb hizo callar el pueblo delante de Moisés, y
dijo: Subamos luego, y poseámosla; que más podremos que ella.
\bibverse{32} Mas los varones que subieron con él, dijeron: No podremos
subir contra aquel pueblo; porque es más fuerte que nosotros.
\bibverse{33} Y vituperaron entre los hijos de Israel la tierra que
habían reconocido, diciendo: La tierra por donde pasamos para
reconocerla, es tierra que traga á sus moradores; y todo el pueblo que
vimos en medio de ella, son hombres de grande estatura. También vimos
allí gigantes, hijos de Anac, raza de los gigantes: y éramos nosotros, á
nuestro parecer, como langostas; y así les parecíamos á ellos.
\footnote{\textbf{13:33} Deut 9,2}

\hypertarget{el-efecto-del-informe-en-la-gente}{%
\subsection{El efecto del informe en la
gente}\label{el-efecto-del-informe-en-la-gente}}

\hypertarget{section-13}{%
\section{14}\label{section-13}}

\bibverse{1} Entonces toda la congregación alzaron grita, y dieron
voces: y el pueblo lloró aquella noche. \bibverse{2} Y quejáronse contra
Moisés y contra Aarón todos los hijos de Israel; y díjoles toda la
multitud: ¡Ojalá muriéramos en la tierra de Egipto; ó en este desierto
ojalá muriéramos! \bibverse{3} ¿Y por qué nos trae Jehová á esta tierra
para caer á cuchillo, y que nuestras mujeres y nuestros chiquitos sean
por presa? ¿no nos sería mejor volvernos á Egipto? \footnote{\textbf{14:3}
  Sal 106,24} \bibverse{4} Y decían el uno al otro: Hagamos un capitán,
y volvámonos á Egipto.

\bibverse{5} Entonces Moisés y Aarón cayeron sobre sus rostros delante
de toda la multitud de la congregación de los hijos de Israel.

\hypertarget{el-intento-fallido-de-apaciguamiento-de-joshua-y-caleb}{%
\subsection{El intento fallido de apaciguamiento de Joshua y
Caleb}\label{el-intento-fallido-de-apaciguamiento-de-joshua-y-caleb}}

\bibverse{6} Y Josué hijo de Nun, y Caleb hijo de Jephone, que eran de
los que habían reconocido la tierra, rompieron sus vestidos; \footnote{\textbf{14:6}
  Núm 13,16; Núm 13,30} \bibverse{7} Y hablaron á toda la congregación
de los hijos de Israel, diciendo: La tierra por donde pasamos para
reconocerla, es tierra en gran manera buena. \bibverse{8} Si Jehová se
agradare de nosotros, él nos meterá en esta tierra, y nos la entregará;
tierra que fluye leche y miel. \bibverse{9} Por tanto, no seáis rebeldes
contra Jehová, ni temáis al pueblo de aquesta tierra, porque nuestro pan
son: su amparo se ha apartado de ellos, y con nosotros está Jehová: no
los temáis.

\hypertarget{ira-de-dios-la-exitosa-intercesiuxf3n-de-moisuxe9s-el-juicio-divino}{%
\subsection{Ira de Dios; la exitosa intercesión de Moisés; el juicio
divino}\label{ira-de-dios-la-exitosa-intercesiuxf3n-de-moisuxe9s-el-juicio-divino}}

\bibverse{10} Entonces toda la multitud habló de apedrearlos con
piedras. Mas la gloria de Jehová se mostró en el tabernáculo del
testimonio á todos los hijos de Israel. \footnote{\textbf{14:10} Éxod
  17,4; Éxod 16,10}

\bibverse{11} Y Jehová dijo á Moisés: ¿Hasta cuándo me ha de irritar
este pueblo? ¿hasta cuándo no me ha de creer con todas las señales que
he hecho en medio de ellos? \bibverse{12} Yo le heriré de mortandad, y
lo destruiré, y á ti te pondré sobre gente grande y más fuerte que
ellos.

\bibverse{13} Y Moisés respondió á Jehová: Oiránlo luego los Egipcios,
porque de en medio de ellos sacaste á este pueblo con tu fortaleza:
\bibverse{14} Y lo dirán á los habitadores de esta tierra; los cuales
han oído que tú, oh Jehová, estabas en medio de este pueblo, que ojo á
ojo aparecías tú, oh Jehová, y que tu nube estaba sobre ellos, y que de
día ibas delante de ellos en columna de nube, y de noche en columna de
fuego: \bibverse{15} Y que has hecho morir á este pueblo como á un
hombre: y las gentes que hubieren oído tu fama hablarán, diciendo:
\bibverse{16} Porque no pudo Jehová meter este pueblo en la tierra de la
cual les había jurado, los mató en el desierto. \footnote{\textbf{14:16}
  Deut 9,28} \bibverse{17} Ahora, pues, yo te ruego que sea magnificada
la fortaleza del Señor, como lo hablaste, diciendo: \bibverse{18}
Jehová, tardo de ira y grande en misericordia, que perdona la iniquidad
y la rebelión, y absolviendo no absolverá al culpado; que visita la
maldad de los padres sobre los hijos hasta los terceros y hasta los
cuartos. \bibverse{19} Perdona ahora la iniquidad de este pueblo según
la grandeza de tu misericordia, y como has perdonado á este pueblo desde
Egipto hasta aquí.

\bibverse{20} Entonces Jehová dijo: yo lo he perdonado conforme á tu
dicho: \bibverse{21} Mas, ciertamente vivo yo y mi gloria hinche toda la
tierra, \footnote{\textbf{14:21} Éxod 9,16} \bibverse{22} Que todos los
que vieron mi gloria y mis señales que he hecho en Egipto y en el
desierto, y me han tentado ya diez veces, y no han oído mi voz,
\bibverse{23} No verán la tierra de la cual juré á sus padres: no,
ninguno de los que me han irritado la verá. \bibverse{24} Empero mi
siervo Caleb, por cuanto hubo en él otro espíritu, y cumplió de ir en
pos de mí, yo le meteré en la tierra donde entró, y su simiente la
recibirá en heredad. \footnote{\textbf{14:24} Jos 14,6; Jos 14,9}
\bibverse{25} Ahora bien, el Amalecita y el Cananeo habitan en el valle;
volveos mañana, y partíos al desierto, camino del mar Bermejo.

\hypertarget{el-castigo-de-dios-para-las-personas-y-los-espuxedas-se-especifica-con-muxe1s-detalle}{%
\subsection{El castigo de Dios para las personas y los espías se
especifica con más
detalle}\label{el-castigo-de-dios-para-las-personas-y-los-espuxedas-se-especifica-con-muxe1s-detalle}}

\bibverse{26} Y Jehová habló á Moisés y á Aarón, diciendo: \bibverse{27}
¿Hasta cuándo oiré esta depravada multitud que murmura contra mí, las
querellas de los hijos de Israel, que de mí se quejan? \bibverse{28}
Diles: Vivo yo, dice Jehová, que según habéis hablado á mis oídos, así
haré yo con vosotros: \bibverse{29} En este desierto caerán vuestros
cuerpos; todos vuestros contados según toda vuestra cuenta, de veinte
años arriba, los cuales habéis murmurado contra mí; \bibverse{30}
Vosotros á la verdad no entraréis en la tierra, por la cual alcé mi mano
de haceros habitar en ella; exceptuando á Caleb hijo de Jephone, y á
Josué hijo de Nun. \bibverse{31} Mas vuestros chiquitos, de los cuales
dijisteis que serían por presa, yo los introduciré, y ellos conocerán la
tierra que vosotros despreciasteis. \bibverse{32} Y en cuanto á
vosotros, vuestros cuerpos caerán en este desierto. \bibverse{33} Y
vuestros hijos andarán pastoreando en el desierto cuarenta años, y ellos
llevarán vuestras fornicaciones, hasta que vuestros cuerpos sean
consumidos en el desierto. \bibverse{34} Conforme al número de los días,
de los cuarenta días en que reconocisteis la tierra, llevaréis vuestras
iniquidades cuarenta años, un año por cada día; y conoceréis mi castigo.
\bibverse{35} Yo Jehová he hablado; así haré á toda esta multitud
perversa que se ha juntado contra mí; en este desierto serán consumidos,
y ahí morirán.

\hypertarget{muerte-repentina-de-los-espuxedas-excepto-josuuxe9-y-caleb}{%
\subsection{Muerte repentina de los espías excepto Josué y
Caleb}\label{muerte-repentina-de-los-espuxedas-excepto-josuuxe9-y-caleb}}

\bibverse{36} Y los varones que Moisés envió á reconocer la tierra, y
vueltos habían hecho murmurar contra él á toda la congregación,
desacreditando aquel país, \footnote{\textbf{14:36} 1Cor 10,5; 1Cor
  10,10; Jds 1,5} \bibverse{37} Aquellos varones que habían hablado mal
de la tierra, murieron de plaga delante de Jehová. \bibverse{38} Mas
Josué hijo de Nun, y Caleb hijo de Jephone, quedaron con vida de entre
aquellos hombres que habían ido á reconocer la tierra.

\hypertarget{arrepentimiento-del-pueblo-el-intento-fallido-de-penetrar-en-el-pauxeds-enemigo}{%
\subsection{Arrepentimiento del pueblo; el intento fallido de penetrar
en el país
enemigo}\label{arrepentimiento-del-pueblo-el-intento-fallido-de-penetrar-en-el-pauxeds-enemigo}}

\bibverse{39} Y Moisés dijo estas cosas á todos los hijos de Israel, y
el pueblo se enlutó mucho. \bibverse{40} Y levantáronse por la mañana, y
subieron á la cumbre del monte, diciendo: Henos aquí para subir al lugar
del cual ha hablado Jehová; porque hemos pecado. \footnote{\textbf{14:40}
  Núm 13,17}

\bibverse{41} Y dijo Moisés: ¿Por qué quebrantáis el dicho de Jehová?
Esto tampoco os sucederá bien. \bibverse{42} No subáis, porque Jehová no
está en medio de vosotros, no seáis heridos delante de vuestros
enemigos. \bibverse{43} Porque el Amalecita y el Cananeo están allí
delante de vosotros, y caeréis á cuchillo: pues por cuanto os habéis
retraído de seguir á Jehová, por eso no será Jehová con vosotros.

\bibverse{44} Sin embargo, se obstinaron en subir á la cima del monte:
mas el arca de la alianza de Jehová, y Moisés, no se apartaron de en
medio del campo. \bibverse{45} Y descendieron el Amalecita y el Cananeo,
que habitaban en aquel monte, é hiriéronlos y derrotáronlos,
persiguiéndolos hasta Horma.

\hypertarget{regulaciones-con-respecto-a-las-ofrendas-de-comida-y-bebida-como-adiciuxf3n-a-los-holocaustos-y-las-ofrendas-de-salvaciuxf3n}{%
\subsection{Regulaciones con respecto a las ofrendas de comida y bebida
como adición a los holocaustos y las ofrendas de
salvación}\label{regulaciones-con-respecto-a-las-ofrendas-de-comida-y-bebida-como-adiciuxf3n-a-los-holocaustos-y-las-ofrendas-de-salvaciuxf3n}}

\hypertarget{section-14}{%
\section{15}\label{section-14}}

\bibverse{1} Y jehová habló á Moisés, diciendo: \bibverse{2} Habla á los
hijos de Israel, y diles: Cuando hubiereis entrado en la tierra de
vuestras habitaciones, que yo os doy, \bibverse{3} E hiciereis ofrenda
encendida á Jehová, holocausto, ó sacrificio, por especial voto, ó de
vuestra voluntad, ó para hacer en vuestras solemnidades olor suave á
Jehová, de vacas ó de ovejas; \footnote{\textbf{15:3} Lev 7,16}
\bibverse{4} Entonces el que ofreciere su ofrenda á Jehová, traerá por
presente una décima de un epha de flor de harina, amasada con la cuarta
parte de un hin de aceite; \bibverse{5} Y de vino para la libación
ofrecerás la cuarta parte de un hin, además del holocausto ó del
sacrificio, por cada un cordero.

\bibverse{6} Y por cada carnero harás presente de dos décimas de flor de
harina, amasada con el tercio de un hin de aceite: \bibverse{7} Y de
vino para la libación ofrecerás el tercio de un hin, en olor suave á
Jehová. \bibverse{8} Y cuando ofreciereis novillo en holocausto ó
sacrificio, por especial voto, ó de paces á Jehová, \bibverse{9}
Ofrecerás con el novillo un presente de tres décimas de flor de harina,
amasada con la mitad de un hin de aceite: \bibverse{10} Y de vino para
la libación ofrecerás la mitad de un hin, en ofrenda encendida de olor
suave á Jehová. \bibverse{11} Así se hará con cada un buey, ó carnero, ó
cordero, lo mismo de ovejas que de cabras. \bibverse{12} Conforme al
número así haréis con cada uno según el número de ellos.

\bibverse{13} Todo natural hará estas cosas así, para ofrecer ofrenda
encendida de olor suave á Jehová. \bibverse{14} Y cuando habitare con
vosotros extranjero, ó cualquiera que estuviere entre vosotros por
vuestras edades, si hiciere ofrenda encendida de olor suave á Jehová,
como vosotros hiciereis, así hará él. \bibverse{15} Un mismo estatuto
tendréis, vosotros de la congregación y el extranjero que con vosotros
mora; estatuto que será perpetuo por vuestras edades: como vosotros, así
será el peregrino delante de Jehová. \footnote{\textbf{15:15} Éxod 12,49}
\bibverse{16} Una misma ley y un mismo derecho tendréis, vosotros y el
peregrino que con vosotros mora.

\hypertarget{disposiciuxf3n-sobre-los-primeros-pasteles}{%
\subsection{Disposición sobre los primeros
pasteles}\label{disposiciuxf3n-sobre-los-primeros-pasteles}}

\bibverse{17} Y habló Jehová á Moisés, diciendo: \bibverse{18} Habla á
los hijos de Israel, y diles: Cuando hubiereis entrado en la tierra á la
cual yo os llevo, \bibverse{19} Será que cuando comenzareis á comer del
pan de la tierra, ofreceréis ofrenda á Jehová. \bibverse{20} De lo
primero que amasareis, ofreceréis una torta en ofrenda; como la ofrenda
de la era, así la ofreceréis. \bibverse{21} De las primicias de vuestras
masas daréis á Jehová ofrenda por vuestras generaciones.

\hypertarget{reglas-con-respecto-a-las-ofrendas-por-el-pecado-por-obrar-mal-involuntariamente-impunidad-por-transgresiones-intencionales}{%
\subsection{Reglas con respecto a las ofrendas por el pecado por obrar
mal involuntariamente; Impunidad por transgresiones
intencionales}\label{reglas-con-respecto-a-las-ofrendas-por-el-pecado-por-obrar-mal-involuntariamente-impunidad-por-transgresiones-intencionales}}

\bibverse{22} Y cuando errareis, y no hiciereis todos estos mandamientos
que Jehová ha dicho á Moisés, \footnote{\textbf{15:22} Lev 4,2; Lev 4,13}
\bibverse{23} Todas las cosas que Jehová os ha mandado por la mano de
Moisés, desde el día que Jehová lo mandó, y en adelante por vuestras
edades, \bibverse{24} Será que, si el pecado fué hecho por yerro con
ignorancia de la congregación, toda la congregación ofrecerá un novillo
por holocausto, en olor suave á Jehová, con su presente y su libación,
conforme á la ley; y un macho cabrío en expiación. \bibverse{25} Y el
sacerdote hará expiación por toda la congregación de los hijos de
Israel; y les será perdonado, porque yerro es: y ellos traerán sus
ofrendas, ofrenda encendida á Jehová, y sus expiaciones delante de
Jehová, por sus yerros: \bibverse{26} Y será perdonado á toda la
congregación de los hijos de Israel, y al extranjero que peregrina entre
ellos, por cuanto es yerro de todo el pueblo.

\bibverse{27} Y si una persona pecare por yerro, ofrecerá una cabra de
un año por expiación. \bibverse{28} Y el sacerdote hará expiación por la
persona que habrá pecado por yerro, cuando pecare por yerro delante de
Jehová, la reconciliará, y le será perdonado. \bibverse{29} El natural
entre los hijos de Israel, y el peregrino que habitare entre ellos, una
misma ley tendréis para el que hiciere algo por yerro.

\bibverse{30} Mas la persona que hiciere algo con altiva mano, así el
natural como el extranjero, á Jehová injurió; y la tal persona será
cortada de en medio de su pueblo. \footnote{\textbf{15:30} Hech 13,38;
  Heb 10,26-27} \bibverse{31} Por cuanto tuvo en poco la palabra de
Jehová, y dió por nulo su mandamiento, enteramente será cortada la tal
persona: su iniquidad será sobre ella.

\hypertarget{informe-de-la-lapidaciuxf3n-de-un-abusador-del-suxe1bado}{%
\subsection{Informe de la lapidación de un abusador del
sábado}\label{informe-de-la-lapidaciuxf3n-de-un-abusador-del-suxe1bado}}

\bibverse{32} Y estando los hijos de Israel en el desierto, hallaron un
hombre que recogía leña en día de sábado. \bibverse{33} Y los que le
hallaron recogiendo leña trajéronle á Moisés y á Aarón, y á toda la
congregación: \bibverse{34} Y pusiéronlo en la cárcel, por que no estaba
declarado qué le habían de hacer.

\bibverse{35} Y Jehová dijo á Moisés: Irremisiblemente muera aquel
hombre; apedréelo con piedras toda la congregación fuera del campo.
\bibverse{36} Entonces lo sacó la congregación fuera del campo, y
apedreáronlo con piedras, y murió; como Jehová mandó á Moisés.

\hypertarget{ordenanza-sobre-las-borlas-para-adherir-a-las-puntas-de-la-ropa}{%
\subsection{Ordenanza sobre las borlas para adherir a las puntas de la
ropa}\label{ordenanza-sobre-las-borlas-para-adherir-a-las-puntas-de-la-ropa}}

\bibverse{37} Y Jehová habló á Moisés, diciendo: \bibverse{38} Habla á
los hijos de Israel, y diles que se hagan pezuelos (franjas) en los
remates de sus vestidos, por sus generaciones; y pongan en cada pezuelo
de los remates un cordón de cárdeno: \footnote{\textbf{15:38} Deut
  22,12; Mat 23,5}

\bibverse{39} Y serviros ha de pezuelo, para que cuando lo viereis, os
acordéis de todos los mandamientos de Jehová, para ponerlos por obra; y
no miréis en pos de vuestro corazón y de vuestros ojos, en pos de los
cuales fornicáis: \bibverse{40} Para que os acordéis, y hagáis todos mis
mandamientos, y seáis santos á vuestro Dios. \bibverse{41} Yo Jehová
vuestro Dios, que os saqué de la tierra de Egipto, para ser vuestro
Dios: Yo Jehová vuestro Dios.

\hypertarget{el-ultraje-de-coruxe9-y-los-rubenitas}{%
\subsection{El ultraje de Coré y los
rubenitas}\label{el-ultraje-de-coruxe9-y-los-rubenitas}}

\hypertarget{section-15}{%
\section{16}\label{section-15}}

\bibverse{1} Y coré, hijo de Ishar, hijo de Coath, hijo de Leví; y
Dathán y Abiram, hijos de Eliab; y Hon, hijo de Peleth, de los hijos de
Rubén, tomaron gente, \bibverse{2} Y levantáronse contra Moisés con
doscientos y cincuenta varones de los hijos de Israel, príncipes de la
congregación, de los del consejo, varones de nombre; \footnote{\textbf{16:2}
  Núm 12,1-2} \bibverse{3} Y se juntaron contra Moisés y Aarón, y les
dijeron: Básteos, porque toda la congregación, todos ellos son santos, y
en medio de ellos está Jehová: ¿por qué, pues, os levantáis vosotros
sobre la congregación de Jehová?

\hypertarget{moisuxe9s-confronta-al-grupo-de-coruxe9-y-anuncia-un-juicio-divino-en-el-santuario}{%
\subsection{Moisés confronta al grupo de Coré y anuncia un juicio divino
en el
santuario}\label{moisuxe9s-confronta-al-grupo-de-coruxe9-y-anuncia-un-juicio-divino-en-el-santuario}}

\bibverse{4} Y como lo oyó Moisés, echóse sobre su rostro; \bibverse{5}
Y habló á Coré y á todo su séquito, diciendo: Mañana mostrará Jehová
quién es suyo, y al santo harálo llegar á sí; y al que él escogiere, él
lo allegará á sí. \footnote{\textbf{16:5} 2Tim 2,19} \bibverse{6} Haced
esto: tomaos incensarios, Coré y todo su séquito: \bibverse{7} Y poned
fuego en ellos, y poned en ellos sahumerio delante de Jehová mañana; y
será que el varón á quien Jehová escogiere, aquél será el santo: básteos
esto, hijos de Leví.

\bibverse{8} Dijo más Moisés á Coré: Oid ahora, hijos de Leví:
\bibverse{9} ¿Os es poco que el Dios de Israel os haya apartado de la
congregación de Israel, haciéndoos allegar á sí para que ministraseis en
el servicio del tabernáculo de Jehová, y estuvieseis delante de la
congregación para ministrarles? \bibverse{10} ¿Y que te hizo acercar á
ti, y á todos tus hermanos los hijos de Leví contigo; para que procuréis
también el sacerdocio? \bibverse{11} Por tanto, tú y todo tu séquito
sois los que os juntáis contra Jehová: pues Aarón, ¿qué es para que
contra él murmuréis? \footnote{\textbf{16:11} Éxod 16,7}

\hypertarget{datuxe1n-y-abiram-se-burlan-de-la-invitaciuxf3n-de-moisuxe9s-moisuxe9s-oraciuxf3n-a-dios}{%
\subsection{Datán y Abiram se burlan de la invitación de Moisés; Moisés
oración a
Dios}\label{datuxe1n-y-abiram-se-burlan-de-la-invitaciuxf3n-de-moisuxe9s-moisuxe9s-oraciuxf3n-a-dios}}

\bibverse{12} Y envió Moisés á llamar á Dathán y Abiram, hijos de Eliab;
mas ellos respondieron: No iremos allá: \bibverse{13} ¿Es poco que nos
hayas hecho venir de una tierra que destila leche y miel, para hacernos
morir en el desierto, sino que también te enseñorees de nosotros
imperiosamente? \bibverse{14} Ni tampoco nos has metido tú en tierra que
fluya leche y miel, ni nos has dado heredades de tierras y viñas: ¿has
de arrancar los ojos de estos hombres? No subiremos.

\bibverse{15} Entonces Moisés se enojó en gran manera, y dijo á Jehová:
No mires á su presente: ni aun un asno he tomado de ellos, ni á ninguno
de ellos he hecho mal. \footnote{\textbf{16:15} 1Sam 12,3; Hech 20,33}

\hypertarget{moisuxe9s-convoca-a-coruxe9-y-sus-compauxf1eros-para-realizar-el-sacrificio-la-apariciuxf3n-de-la-gloria-de-dios-intercesiuxf3n-de-moisuxe9s}{%
\subsection{Moisés convoca a Coré y sus compañeros para realizar el
sacrificio; La aparición de la gloria de Dios; Intercesión de
Moisés}\label{moisuxe9s-convoca-a-coruxe9-y-sus-compauxf1eros-para-realizar-el-sacrificio-la-apariciuxf3n-de-la-gloria-de-dios-intercesiuxf3n-de-moisuxe9s}}

\bibverse{16} Después dijo Moisés á Coré: Tú y todo tu séquito, poneos
mañana delante de Jehová; tú, y ellos, y Aarón: \bibverse{17} Y tomad
cada uno su incensario, y poned sahumerio en ellos, y allegad delante de
Jehová cada uno su incensario: doscientos y cincuenta incensarios: tú
también, y Aarón, cada uno con su incensario.

\bibverse{18} Y tomaron cada uno su incensario, y pusieron en ellos
fuego, y echaron en ellos sahumerio, y pusiéronse á la puerta del
tabernáculo del testimonio con Moisés y Aarón. \bibverse{19} Ya Coré
había hecho juntar contra ellos toda la congregación á la puerta del
tabernáculo del testimonio: entonces la gloria de Jehová apareció á toda
la congregación.

\bibverse{20} Y Jehová habló á Moisés y á Aarón, diciendo: \bibverse{21}
Apartaos de entre esta congregación, y consumirlos he en un momento.

\bibverse{22} Y ellos se echaron sobre sus rostros, y dijeron: Dios,
Dios de los espíritus de toda carne, ¿no es un hombre el que pecó? ¿y
airarte has tú contra toda la congregación? \footnote{\textbf{16:22} Job
  12,10; 2Sam 24,17}

\bibverse{23} Entonces Jehová habló á Moisés, diciendo: \bibverse{24}
Habla á la congregación, diciendo: Apartaos de en derredor de la tienda
de Coré, Dathán, y Abiram.

\hypertarget{moisuxe9s-convoca-a-coruxe9-y-sus-compauxf1eros-para-realizar-el-sacrificio-la-apariciuxf3n-de-la-gloria-de-dios-intercesiuxf3n-de-moisuxe9s-1}{%
\subsection{Moisés convoca a Coré y sus compañeros para realizar el
sacrificio; La aparición de la gloria de Dios; Intercesión de
Moisés}\label{moisuxe9s-convoca-a-coruxe9-y-sus-compauxf1eros-para-realizar-el-sacrificio-la-apariciuxf3n-de-la-gloria-de-dios-intercesiuxf3n-de-moisuxe9s-1}}

\bibverse{25} Y Moisés se levantó, y fué á Dathán y á Abiram; y los
ancianos de Israel fueron en pos de él. \bibverse{26} Y él habló á la
congregación, diciendo: Apartaos ahora de las tiendas de estos impíos
hombres, y no toquéis ninguna cosa suya, por que no perezcáis en todos
sus pecados.

\bibverse{27} Y apartáronse de las tiendas de Coré, de Dathán, y de
Abiram en derredor: y Dathán y Abiram salieron y pusiéronse á las
puertas de sus tiendas, con sus mujeres, y sus hijos, y sus chiquitos.

\bibverse{28} Y dijo Moisés: En esto conoceréis que Jehová me ha enviado
para que hiciese todas estas cosas; que no de mi corazón las hice.
\bibverse{29} Si como mueren todos los hombres murieren éstos, ó si
fueren ellos visitados á la manera de todos los hombres, Jehová no me
envió. \bibverse{30} Mas si Jehová hiciere una nueva cosa, y la tierra
abriere su boca, y los tragare con todas sus cosas, y descendieren vivos
al abismo, entonces conoceréis que estos hombres irritaron á Jehová.

\bibverse{31} Y aconteció, que en acabando él de hablar todas estas
palabras, rompióse la tierra que estaba debajo de ellos: \bibverse{32} Y
abrió la tierra su boca, y tragólos á ellos, y á sus casas, y á todos
los hombres de Coré, y á toda su hacienda. \bibverse{33} Y ellos, con
todo lo que tenían, descendieron vivos al abismo, y cubriólos la tierra,
y perecieron de en medio de la congregación. \bibverse{34} Y todo
Israel, los que estaban en derredor de ellos, huyeron al grito de ellos;
porque decían: No nos trague también la tierra. \bibverse{35} Y salió
fuego de Jehová, y consumió los doscientos y cincuenta hombres que
ofrecían el sahumerio. \footnote{\textbf{16:35} Lev 10,1-2; Sal 106,18}

\hypertarget{el-uso-de-las-250-ollas-humeantes-por-parte-de-coruxe9-y-sus-compauxf1eros-como-cubierta-para-el-altar-de-sacrificios}{%
\subsection{El uso de las 250 ollas humeantes por parte de Coré y sus
compañeros como cubierta para el altar de
sacrificios}\label{el-uso-de-las-250-ollas-humeantes-por-parte-de-coruxe9-y-sus-compauxf1eros-como-cubierta-para-el-altar-de-sacrificios}}

\bibverse{36} Entonces Jehová habló á Moisés, diciendo: \bibverse{37} Di
á Eleazar, hijo de Aarón sacerdote, que tome los incensarios de en medio
del incendio, y derrame más allá el fuego; porque son santificados:
\bibverse{38} Los incensarios de estos pecadores contra sus almas: y
harán de ellos planchas extendidas para cubrir el altar: por cuanto
ofrecieron con ellos delante de Jehová, son santificados; y serán por
señal á los hijos de Israel.

\bibverse{39} Y el sacerdote Eleazar tomó los incensarios de metal con
que los quemados habían ofrecido; y extendiéronlos para cubrir el altar,
\bibverse{40} En recuerdo á los hijos de Israel que ningún extraño que
no sea de la simiente de Aarón, llegue á ofrecer sahumerio delante de
Jehová, porque no sea como Coré, y como su séquito; según se lo dijo
Jehová por mano de Moisés.

\hypertarget{castigo-a-la-comunidad-quejuxe1ndose-por-la-desapariciuxf3n-de-los-alborotadores-la-expiaciuxf3n-hecha-por-moisuxe9s-y-aaruxf3n}{%
\subsection{Castigo a la comunidad quejándose por la desaparición de los
alborotadores; la expiación hecha por Moisés y
Aarón}\label{castigo-a-la-comunidad-quejuxe1ndose-por-la-desapariciuxf3n-de-los-alborotadores-la-expiaciuxf3n-hecha-por-moisuxe9s-y-aaruxf3n}}

\bibverse{41} El día siguiente toda la congregación de los hijos de
Israel murmuró contra Moisés y Aarón, diciendo: Vosotros habéis muerto
al pueblo de Jehová.

\bibverse{42} Y aconteció que, como se juntó la congregación contra
Moisés y Aarón, miraron hacia el tabernáculo del testimonio, y he aquí
la nube lo había cubierto, y apareció la gloria de Jehová. \bibverse{43}
Y vinieron Moisés y Aarón delante del tabernáculo del testimonio.
\bibverse{44} Y Jehová habló á Moisés, diciendo: \bibverse{45} Apartaos
de en medio de esta congregación, y consumirélos en un momento. Y ellos
se echaron sobre sus rostros.

\bibverse{46} Y dijo Moisés á Aarón: Toma el incensario, y pon en él
fuego del altar, y sobre él pon perfume, y ve presto á la congregación,
y haz expiación por ellos; porque el furor ha salido de delante de la
faz de Jehová: la mortandad ha comenzado.

\bibverse{47} Entonces tomó Aarón el incensario, como Moisés dijo, y
corrió en medio de la congregación: y he aquí que la mortandad había
comenzado en el pueblo: y él puso perfume, é hizo expiación por el
pueblo. \bibverse{48} Y púsose entre los muertos y los vivos; y cesó la
mortandad. \bibverse{49} Y los que murieron en aquella mortandad fueron
catorce mil y setecientos, sin los muertos por el negocio de Coré.
\bibverse{50} Después se volvió Aarón á Moisés á la puerta del
tabernáculo del testimonio, cuando la mortandad había cesado.

\hypertarget{prueba-del-derecho-sacerdotal-de-aaruxf3n-por-los-maravillosos-peldauxf1os-de-su-cayado}{%
\subsection{Prueba del derecho sacerdotal de Aarón por los maravillosos
peldaños de su
cayado}\label{prueba-del-derecho-sacerdotal-de-aaruxf3n-por-los-maravillosos-peldauxf1os-de-su-cayado}}

\hypertarget{section-16}{%
\section{17}\label{section-16}}

\bibverse{1} Y habló Jehová á Moisés, diciendo: \bibverse{2} Habla á los
hijos de Israel, y toma de ellos una vara por cada casa de los padres,
de todos los príncipes de ellos, doce varas conforme á las casas de sus
padres; y escribirás el nombre de cada uno sobre su vara. \bibverse{3} Y
escribirás el nombre de Aarón sobre la vara de Leví; porque cada cabeza
de familia de sus padres tendrá una vara. \bibverse{4} Y las pondrás en
el tabernáculo del testimonio delante del testimonio, donde yo me
declararé á vosotros. \bibverse{5} Y será, que el varón que yo
escogiere, su vara florecerá: y haré cesar de sobre mí las quejas de los
hijos de Israel, con que murmuran contra vosotros.

\bibverse{6} Y Moisés habló á los hijos de Israel, y todos los príncipes
de ellos le dieron varas; cada príncipe por las casas de sus padres una
vara, en todas doce varas; y la vara de Aarón estaba entre las varas de
ellos. \bibverse{7} Y Moisés puso las varas delante de Jehová en el
tabernáculo del testimonio. \footnote{\textbf{17:7} Núm 14,10}

\bibverse{8} Y aconteció que el día siguiente vino Moisés al tabernáculo
del testimonio; y he aquí que la vara de Aarón de la casa de Leví había
brotado, y echado flores, y arrojado renuevos, y producido almendras.
\bibverse{9} Entonces sacó Moisés todas las varas de delante de Jehová á
todos los hijos de Israel; y ellos lo vieron, y tomaron cada uno su
vara.

\bibverse{10} Y Jehová dijo á Moisés: Vuelve la vara de Aarón delante
del testimonio, para que se guarde por señal á los hijos rebeldes; y
harás cesar sus quejas de sobre mí, porque no mueran. \bibverse{11} E
hízolo Moisés: como le mandó Jehová, así hizo. \footnote{\textbf{17:11}
  Éxod 28,38; Lev 16,13}

\bibverse{12} Entonces los hijos de Israel hablaron á Moisés, diciendo:
He aquí nosotros somos muertos, perdidos somos, todos nosotros somos
perdidos. \bibverse{13} Cualquiera que se llegare, el que se acercare al
tabernáculo de Jehová morirá: ¿acabaremos de perecer todos?

\hypertarget{ordenanzas-generales-sobre-los-deberes-de-los-sacerdotes-y-sus-ayudantes-los-levitas}{%
\subsection{Ordenanzas generales sobre los deberes de los sacerdotes y
sus ayudantes, los
levitas}\label{ordenanzas-generales-sobre-los-deberes-de-los-sacerdotes-y-sus-ayudantes-los-levitas}}

\hypertarget{section-17}{%
\section{18}\label{section-17}}

\bibverse{1} Y jehová dijo á Aarón: Tú y tus hijos, y la casa de tu
padre contigo, llevaréis el pecado del santuario: y tú y tus hijos
contigo llevaréis el pecado de vuestro sacerdocio. \footnote{\textbf{18:1}
  Éxod 28,38; Lev 16,32-33} \bibverse{2} Y á tus hermanos también, la
tribu de Leví, la tribu de tu padre, hazlos llegar á ti, y júntense
contigo, y servirte han; y tú y tus hijos contigo serviréis delante del
tabernáculo del testimonio. \footnote{\textbf{18:2} Núm 3,6-10}
\bibverse{3} Y guardarán lo que tú ordenares, y el cargo de todo el
tabernáculo: mas no llegarán á los vasos santos ni al altar, porque no
mueran ellos y vosotros. \bibverse{4} Se juntarán, pues, contigo, y
tendrán el cargo del tabernáculo del testimonio en todo el servicio del
tabernáculo; ningún extraño se ha de llegar á vosotros.

\bibverse{5} Y tendréis la guarda del santuario, y la guarda del altar,
para que no sea más la ira sobre los hijos de Israel. \bibverse{6}
Porque he aquí yo he tomado á vuestros hermanos los Levitas de entre los
hijos de Israel, dados á vosotros en don de Jehová, para que sirvan en
el ministerio del tabernáculo del testimonio. \footnote{\textbf{18:6}
  Núm 3,12; Núm 3,45} \bibverse{7} Mas tú y tus hijos contigo guardaréis
vuestro sacerdocio en todo negocio del altar, y del velo adentro, y
ministraréis. Yo os he dado en don el servicio de vuestro sacerdocio; y
el extraño que se llegare, morirá. \footnote{\textbf{18:7} Núm 1,51}

\hypertarget{los-ingresos-de-los-sacerdotes}{%
\subsection{Los ingresos de los
sacerdotes}\label{los-ingresos-de-los-sacerdotes}}

\bibverse{8} Dijo más Jehová á Aarón: He aquí yo te he dado también la
guarda de mis ofrendas: todas las cosas consagradas de los hijos de
Israel te he dado por razón de la unción, y á tus hijos, por estatuto
perpetuo. \footnote{\textbf{18:8} Lev 2,3; Lev 2,10; Lev 6,9-11; Lev
  6,19-22; Lev 7,6-10} \bibverse{9} Esto será tuyo de la ofrenda de las
cosas santas reservadas del fuego: toda ofrenda de ellos, todo presente
suyo, y toda expiación por el pecado de ellos, y toda expiación por la
culpa de ellos, que me han de presentar, será cosa muy santa para ti y
para tus hijos. \bibverse{10} En el santuario la comerás; todo varón
comerá de ella: cosa santa será para ti.

\bibverse{11} Esto también será tuyo: la ofrenda elevada de sus dones, y
todas las ofrendas agitadas de los hijos de Israel, he dado á ti, y á
tus hijos y á tus hijas contigo, por estatuto perpetuo: todo limpio en
tu casa comerá de ellas.

\bibverse{12} De aceite, y de mosto, y de trigo, todo lo más escogido,
las primicias de ello, que presentarán á Jehová, á ti las he dado.
\bibverse{13} Las primicias de todas las cosas de la tierra de ellos,
las cuales traerán á Jehová, serán tuyas: todo limpio en tu casa comerá
de ellas. \footnote{\textbf{18:13} Éxod 23,19; Deut 18,4}

\bibverse{14} Todo lo consagrado por voto en Israel será tuyo.
\footnote{\textbf{18:14} Lev 27,28} \bibverse{15} Todo lo que abriere
matriz en toda carne que ofrecerán á Jehová, así de hombres como de
animales, será tuyo: mas has de hacer redimir el primogénito del hombre:
también harás redimir el primogénito de animal inmundo. \footnote{\textbf{18:15}
  Éxod 13,12-13; Éxod 34,19-20} \bibverse{16} Y de un mes harás efectuar
el rescate de ellos, conforme á tu estimación, por precio de cinco
siclos, al siclo del santuario, que es de veinte óbolos.

\bibverse{17} Mas el primogénito de vaca, y el primogénito de oveja, y
el primogénito de cabra, no redimirás; santificados son: la sangre de
ellos rociarás sobre el altar, y quemarás la grosura de ellos, ofrenda
encendida en olor suave á Jehová. \bibverse{18} Y la carne de ellos será
tuya: como el pecho de la mecedura y como la espaldilla derecha, será
tuya. \bibverse{19} Todas las ofrendas elevadas de las cosas santas, que
los hijos de Israel ofrecieren á Jehová, helas dado para ti, y para tus
hijos y para tus hijas contigo, por estatuto perpetuo: pacto de sal
perpetuo es delante de Jehová para ti y para tu simiente contigo.

\hypertarget{asignaciuxf3n-del-diezmo-a-los-levitas-por-la-negaciuxf3n-de-la-tierra}{%
\subsection{Asignación del diezmo a los levitas por la negación de la
tierra}\label{asignaciuxf3n-del-diezmo-a-los-levitas-por-la-negaciuxf3n-de-la-tierra}}

\bibverse{20} Y Jehová dijo á Aarón: De la tierra de ellos no tendrás
heredad, ni entre ellos tendrás parte: Yo soy tu parte y tu heredad en
medio de los hijos de Israel.

\bibverse{21} Y he aquí yo he dado á los hijos de Leví todos los diezmos
en Israel por heredad, por su ministerio, por cuanto ellos sirven en el
ministerio del tabernáculo del testimonio. \footnote{\textbf{18:21} Lev
  27,30} \bibverse{22} Y no llegarán más los hijos de Israel al
tabernáculo del testimonio, porque no lleven pecado, por el cual mueran.
\bibverse{23} Mas los Levitas harán el servicio del tabernáculo del
testimonio, y ellos llevarán su iniquidad: estatuto perpetuo por
vuestras edades; y no poseerán heredad entre los hijos de Israel.
\bibverse{24} Porque á los Levitas he dado por heredad los diezmos de
los hijos de Israel, que ofrecerán á Jehová en ofrenda: por lo cual les
he dicho: Entre los hijos de Israel no poseerán heredad.

\hypertarget{el-diezmo-de-los-ingresos-de-los-levitas-a-los-sacerdotes}{%
\subsection{El diezmo de los ingresos de los levitas a los
sacerdotes}\label{el-diezmo-de-los-ingresos-de-los-levitas-a-los-sacerdotes}}

\bibverse{25} Y habló Jehová á Moisés, diciendo: \bibverse{26} Así
hablarás á los Levitas, y les dirás: Cuando tomareis de los hijos de
Israel los diezmos que os he dado de ellos por vuestra heredad, vosotros
presentaréis de ellos en ofrenda mecida á Jehová el diezmo de los
diezmos. \bibverse{27} Y se os contará vuestra ofrenda como grano de la
era, y como acopio del lagar. \bibverse{28} Así ofreceréis también
vosotros ofrenda á Jehová de todos vuestros diezmos que hubiereis
recibido de los hijos de Israel; y daréis de ellos la ofrenda de Jehová
á Aarón el sacerdote. \bibverse{29} De todos vuestros dones ofreceréis
toda ofrenda á Jehová; de todo lo mejor de ellos ofreceréis la porción
que ha de ser consagrada.

\bibverse{30} Y les dirás: Cuando ofreciereis lo mejor de ellos, será
contado á los Levitas por fruto de la era, y como fruto del lagar.
\bibverse{31} Y lo comeréis en cualquier lugar, vosotros y vuestra
familia: pues es vuestra remuneración por vuestro ministerio en el
tabernáculo del testimonio. \bibverse{32} Y cuando vosotros hubiereis
ofrecido de ello lo mejor suyo, no llevaréis por ello pecado: y no
habéis de contaminar las cosas santas de los hijos de Israel, y no
moriréis.

\hypertarget{preparaciuxf3n-y-uso-del-agua-de-limpieza}{%
\subsection{Preparación y uso del agua de
limpieza}\label{preparaciuxf3n-y-uso-del-agua-de-limpieza}}

\hypertarget{section-18}{%
\section{19}\label{section-18}}

\bibverse{1} Y jehová habló á Moisés y á Aarón, diciendo: \bibverse{2}
Esta es la ordenanza de la ley que Jehová ha prescrito, diciendo: Di á
los hijos de Israel que te traigan una vaca bermeja, perfecta, en la
cual no haya falta, sobre la cual no se haya puesto yugo: \footnote{\textbf{19:2}
  Heb 9,13; Lev 22,20} \bibverse{3} Y la daréis á Eleazar el sacerdote,
y él la sacará fuera del campo, y harála degollar en su presencia.
\bibverse{4} Y tomará Eleazar el sacerdote de su sangre con su dedo, y
rociará hacia la delantera del tabernáculo del testimonio con la sangre
de ella siete veces; \bibverse{5} Y hará quemar la vaca ante sus ojos:
su cuero y su carne y su sangre, con su estiércol, hará quemar.
\bibverse{6} Luego tomará el sacerdote palo de cedro, é hisopo, y
escarlata, y lo echará en medio del fuego en que arde la vaca.
\footnote{\textbf{19:6} Lev 14,6} \bibverse{7} El sacerdote lavará luego
sus vestidos, lavará también su carne con agua, y después entrará en el
real; y será inmundo el sacerdote hasta la tarde. \footnote{\textbf{19:7}
  Lev 16,28} \bibverse{8} Asimismo el que la quemó, lavará sus vestidos
en agua, también lavará en agua su carne, y será inmundo hasta la tarde.

\bibverse{9} Y un hombre limpio recogerá las cenizas de la vaca, y las
pondrá fuera del campo en lugar limpio, y las guardará la congregación
de los hijos de Israel para el agua de separación: es una expiación.
\bibverse{10} Y el que recogió las cenizas de la vaca, lavará sus
vestidos, y será inmundo hasta la tarde: y será á los hijos de Israel, y
al extranjero que peregrina entre ellos, por estatuto perpetuo.

\bibverse{11} El que tocare muerto de cualquiera persona humana, siete
días será inmundo: \bibverse{12} Este se purificará al tercer día con
aquesta agua, y al séptimo día será limpio; y si al tercer día no se
purificare, no será limpio al séptimo día. \bibverse{13} Cualquiera que
tocare en muerto, en persona de hombre que estuviere muerto, y no se
purificare, el tabernáculo de Jehová contaminó; y aquella persona será
cortada de Israel: por cuanto el agua de la separación no fué rociada
sobre él, inmundo será; y su inmundicia será sobre él. \footnote{\textbf{19:13}
  Lev 15,31}

\hypertarget{instrucciones-sobre-casos-especuxedficos-de-contaminaciuxf3n-y-su-tratamiento}{%
\subsection{Instrucciones sobre casos específicos de contaminación y su
tratamiento}\label{instrucciones-sobre-casos-especuxedficos-de-contaminaciuxf3n-y-su-tratamiento}}

\bibverse{14} Esta es la ley para cuando alguno muriere en la tienda:
cualquiera que entrare en la tienda y todo el que estuviere en ella,
será inmundo siete días. \bibverse{15} Y todo vaso abierto, sobre el
cual no hubiere tapadera bien ajustada, será inmundo.

\bibverse{16} Y cualquiera que tocare en muerto á cuchillo sobre la haz
del campo, ó en muerto, ó en hueso humano, ó en sepulcro, siete días
será inmundo.

\bibverse{17} Y para el inmundo tomarán de la ceniza de la quemada vaca
de la expiación, y echarán sobre ella agua viva en un vaso:
\bibverse{18} Y un hombre limpio tomará hisopo, y mojarálo en el agua, y
rociará sobre la tienda, y sobre todos los muebles, y sobre las personas
que allí estuvieren, y sobre aquel que hubiere tocado el hueso, ó el
matado, ó el muerto, ó el sepulcro: \bibverse{19} Y el limpio rociará
sobre el inmundo al tercero y al séptimo día: y cuando lo habrá
purificado al día séptimo, él lavará luego sus vestidos, y á sí mismo se
lavará con agua, y será limpio á la tarde. \bibverse{20} Y el que fuere
inmundo, y no se purificare, la tal persona será cortada de entre la
congregación, por cuanto contaminó el tabernáculo de Jehová: no fué
rociada sobre él el agua de separación: es inmundo. \bibverse{21} Y les
será por estatuto perpetuo: también el que rociare el agua de la
separación lavará sus vestidos; y el que tocare el agua de la
separación, será inmundo hasta la tarde.

\bibverse{22} Y todo lo que el inmundo tocare, será inmundo: y la
persona que lo tocare, será inmunda hasta la tarde.

\hypertarget{llegada-a-kade-y-muerte-de-miriam-reonovada-queja-del-pueblo-la-fatuxeddicia-doncaciuxf3n-de-agua-de-la-roca-para-moisuxe9s-y-aaruxf3n}{%
\subsection{Llegada a Kade y muerte de Miriam; reonovada queja del
pueblo; la fatídicia doncación de agua de la roca para Moisés y
Aarón}\label{llegada-a-kade-y-muerte-de-miriam-reonovada-queja-del-pueblo-la-fatuxeddicia-doncaciuxf3n-de-agua-de-la-roca-para-moisuxe9s-y-aaruxf3n}}

\hypertarget{section-19}{%
\section{20}\label{section-19}}

\bibverse{1} Y llegaron los hijos de Israel, toda la congregación, al
desierto de Zin, en el mes primero, y asentó el pueblo en Cades; y allí
murió María, y fué allí sepultada. \bibverse{2} Y como no hubiese agua
para la congregación, juntáronse contra Moisés y Aarón. \footnote{\textbf{20:2}
  Éxod 17,1-7} \bibverse{3} Y regañó el pueblo con Moisés, y hablaron
diciendo: ¡Ojalá que nosotros hubiéramos muerto cuando perecieron
nuestros hermanos delante de Jehová! \bibverse{4} ¿Y por qué hiciste
venir la congregación de Jehová á este desierto, para que muramos aquí
nosotros y nuestras bestias? \bibverse{5} ¿Y por qué nos has hecho subir
de Egipto, para traernos á este mal lugar? No es lugar de sementera, de
higueras, de viñas, ni granadas: ni aun de agua para beber.

\bibverse{6} Y fuéronse Moisés y Aarón de delante de la congregación á
la puerta del tabernáculo del testimonio, y echáronse sobre sus rostros;
y la gloria de Jehová apareció sobre ellos. \bibverse{7} Y habló Jehová
á Moisés, diciendo: \bibverse{8} Toma la vara, y reune la congregación,
tú y Aarón tu hermano, y hablad á la peña en ojos de ellos; y ella dará
su agua, y les sacarás aguas de la peña, y darás de beber á la
congregación, y á sus bestias.

\bibverse{9} Entonces Moisés tomó la vara de delante de Jehová, como él
le mandó. \bibverse{10} Y juntaron Moisés y Aarón la congregación
delante de la peña, y díjoles: Oid ahora, rebeldes: ¿os hemos de hacer
salir aguas de esta peña? \footnote{\textbf{20:10} Sal 106,33}
\bibverse{11} Entonces alzó Moisés su mano, é hirió la peña con su vara
dos veces: y salieron muchas aguas, y bebió la congregación, y sus
bestias.

\bibverse{12} Y Jehová dijo á Moisés y á Aarón: Por cuanto no creísteis
en mí, para santificarme en ojos de los hijos de Israel, por tanto, no
meteréis esta congregación en la tierra que les he dado.

\bibverse{13} Estas son las aguas de la rencilla, por las cuales
contendieron los hijos de Israel con Jehová, y él se santificó en ellos.
\footnote{\textbf{20:13} Sal 81,8}

\hypertarget{los-edomitas-se-niegan-a-permitir-el-paso-la-muerte-de-aaron}{%
\subsection{Los edomitas se niegan a permitir el paso; La muerte de
Aaron}\label{los-edomitas-se-niegan-a-permitir-el-paso-la-muerte-de-aaron}}

\bibverse{14} Y envió Moisés embajadores al rey de Edom desde Cades: Así
dice Israel tu hermano: Tú has sabido todo el trabajo que nos ha venido:
\footnote{\textbf{20:14} Gén 32,4; Jue 11,17; Deut 23,8}

\bibverse{15} Cómo nuestros padres descendieron á Egipto, y estuvimos en
Egipto largo tiempo, y los Egipcios nos maltrataron, y á nuestros
padres; \bibverse{16} Y clamamos á Jehová, el cual oyó nuestra voz, y
envió ángel, y sacónos de Egipto; y he aquí estamos en Cades, ciudad al
extremo de tus confines: \footnote{\textbf{20:16} Éxod 23,20}

\bibverse{17} Rogámoste que pasemos por tu tierra; no pasaremos por
labranza, ni por viña, ni beberemos agua de pozos: por el camino real
iremos, sin apartarnos á la diestra ni á la siniestra, hasta que hayamos
pasado tu término. \footnote{\textbf{20:17} Núm 21,22}

\bibverse{18} Y Edom le respondió: No pasarás por mi país, de otra
manera saldré contra ti armado.

\bibverse{19} Y los hijos de Israel dijeron: Por el camino seguido
iremos; y si bebiéremos tus aguas yo y mis ganados, daré el precio de
ellas: ciertamente sin hacer otra cosa, pasaré de seguida.

\bibverse{20} Y él respondió: No pasarás. Y salió Edom contra él con
mucho pueblo, y mano fuerte. \bibverse{21} No quiso, pues, Edom dejar
pasar á Israel por su término, y apartóse Israel de él.

\hypertarget{el-tren-de-kades-al-monte-hor-la-muerte-de-aaron}{%
\subsection{El tren de Kades al monte Hor; La muerte de
Aaron}\label{el-tren-de-kades-al-monte-hor-la-muerte-de-aaron}}

\bibverse{22} Y partidos de Cades los hijos de Israel, toda aquella
congregación, vinieron al monte de Hor. \bibverse{23} Y Jehová habló á
Moisés y Aarón en el monte de Hor, en los confines de la tierra de Edom,
diciendo: \bibverse{24} Aarón será reunido á sus pueblos; pues no
entrará en la tierra que yo di á los hijos de Israel, por cuanto
fuisteis rebeldes á mi mandamiento en las aguas de la rencilla.
\bibverse{25} Toma á Aarón y á Eleazar su hijo, y hazlos subir al monte
de Hor; \bibverse{26} Y haz desnudar á Aarón sus vestidos, y viste de
ellos á Eleazar su hijo; porque Aarón será reunido á sus pueblos, y allí
morirá. \footnote{\textbf{20:26} Lev 21,10}

\bibverse{27} Y Moisés hizo como Jehová le mandó: y subieron al monte de
Hor á ojos de toda la congregación. \bibverse{28} Y Moisés hizo desnudar
á Aarón de sus vestidos y vistiólos á Eleazar su hijo: y Aarón murió
allí en la cumbre del monte: y Moisés y Eleazar descendieron del monte.
\bibverse{29} Y viendo toda la congregación que Aarón era muerto,
hiciéronle duelo por treinta días todas las familias de Israel.

\hypertarget{batalla-victoriosa-con-el-rey-de-arad}{%
\subsection{Batalla victoriosa con el Rey de
Arad}\label{batalla-victoriosa-con-el-rey-de-arad}}

\hypertarget{section-20}{%
\section{21}\label{section-20}}

\bibverse{1} Y oyendo el Cananeo, el rey de Arad, el cual habitaba al
mediodía, que venía Israel por el camino de los centinelas, peleó con
Israel, y tomó de él presa. \bibverse{2} Entonces Israel hizo voto á
Jehová, y dijo: Si en efecto entregares á este pueblo en mi mano, yo
destruiré sus ciudades. \footnote{\textbf{21:2} Deut 13,16; Jos 6,17;
  Jue 1,17; 1Sam 15,3} \bibverse{3} Y Jehová escuchó la voz de Israel, y
entregó al Cananeo, y destruyólos á ellos y á sus ciudades; y llamó el
nombre de aquel lugar Horma.

\hypertarget{murmullos-de-la-gente-las-serpientes-venenosas-y-la-serpiente-de-bronce}{%
\subsection{Murmullos de la gente; las serpientes venenosas y la
serpiente de
bronce}\label{murmullos-de-la-gente-las-serpientes-venenosas-y-la-serpiente-de-bronce}}

\bibverse{4} Y partieron del monte de Hor, camino del mar Bermejo, para
rodear la tierra de Edom; y abatióse el ánimo del pueblo por el camino.
\bibverse{5} Y habló el pueblo contra Dios y Moisés: ¿Por qué nos
hiciste subir de Egipto para que muramos en este desierto? que ni hay
pan, ni agua, y nuestra alma tiene fastidio de este pan tan liviano.

\bibverse{6} Y Jehová envió entre el pueblo serpientes ardientes, que
mordían al pueblo: y murió mucho pueblo de Israel. \footnote{\textbf{21:6}
  1Cor 10,9} \bibverse{7} Entonces el pueblo vino á Moisés, y dijeron:
Pecado hemos por haber hablado contra Jehová, y contra ti: ruega á
Jehová que quite de nosotros estas serpientes. Y Moisés oró por el
pueblo.

\bibverse{8} Y Jehová dijo á Moisés: Hazte una serpiente ardiente, y
ponla sobre la bandera: y será que cualquiera que fuere mordido y mirare
á ella, vivirá. \bibverse{9} Y Moisés hizo una serpiente de metal, y
púsola sobre la bandera: y fué, que cuando alguna serpiente mordía á
alguno, miraba á la serpiente de metal, y vivía.

\hypertarget{el-tren-al-arnuxf3n-ya-las-estepas-de-los-moabitas-la-canciuxf3n-de-la-fuente}{%
\subsection{El tren al Arnón ya las estepas de los moabitas; la canción
de la
fuente}\label{el-tren-al-arnuxf3n-ya-las-estepas-de-los-moabitas-la-canciuxf3n-de-la-fuente}}

\bibverse{10} Y partieron los hijos de Israel, y asentaron campo en
Oboth. \bibverse{11} Y partidos de Oboth, asentaron en Ije-abarim, en el
desierto que está delante de Moab, al nacimiento del sol. \bibverse{12}
Partidos de allí, asentaron en la arroyada de Zared. \bibverse{13} De
allí movieron, y asentaron de la otra parte de Arnón, que está en el
desierto, y que sale del término del Amorrheo; porque Arnón es término
de Moab, entre Moab y el Amorrheo. \bibverse{14} Por tanto se dice en el
libro de las batallas de Jehová: Lo que hizo en el mar Bermejo, y en los
arroyos de Arnón: \bibverse{15} Y á la corriente de los arroyos que va á
parar en Ar, y descansa en el término de Moab.

\bibverse{16} Y de allí vinieron á Beer: este es el pozo del cual Jehová
dijo á Moisés: Junta el pueblo, y les daré agua.

\bibverse{17} Entonces cantó Israel esta canción: Sube, oh pozo; á él
cantad: \bibverse{18} Pozo, el cual cavaron los señores; caváronlo los
príncipes del pueblo, y el legislador, con sus bordones. Y del desierto
vinieron á Mathana:

\bibverse{19} Y de Mathana á Nahaliel: y de Nahaliel á Bamoth:
\bibverse{20} Y de Bamoth al valle que está en los campos de Moab, y á
la cumbre de Pisga, que mira á Jesimón.

\hypertarget{derrota-del-rey-amorreo-sehuxf3n-y-conquista-de-su-pauxeds-canciuxf3n-de-triunfo-de-los-israelitas}{%
\subsection{Derrota del rey amorreo Sehón y conquista de su país;
Canción de triunfo de los
israelitas}\label{derrota-del-rey-amorreo-sehuxf3n-y-conquista-de-su-pauxeds-canciuxf3n-de-triunfo-de-los-israelitas}}

\bibverse{21} Y envió Israel embajadores á Sehón, rey de los Amorrheos,
diciendo: \footnote{\textbf{21:21} Deut 2,26-37} \bibverse{22} Pasaré
por tu tierra: no nos apartaremos por los labrados, ni por las viñas; no
beberemos las aguas de los pozos: por el camino real iremos, hasta que
pasemos tu término.

\bibverse{23} Mas Sehón no dejó pasar á Israel por su término: antes
juntó Sehón todo su pueblo, y salió contra Israel en el desierto: y vino
á Jahaz, y peleó contra Israel. \bibverse{24} E hirióle Israel á filo de
espada, y tomó su tierra desde Arnón hasta Jaboc, hasta los hijos de
Ammón: porque el término de los hijos de Ammón era fuerte. \bibverse{25}
Y tomó Israel todas estas ciudades: y habitó Israel en todas las
ciudades del Amorrheo, en Hesbón y en todas sus aldeas. \bibverse{26}
Porque Hesbón era la ciudad de Sehón, rey de los Amorrheos; el cual
había tenido guerra antes con el rey de Moab, y tomado de su poder toda
su tierra hasta Arnón. \bibverse{27} Por tanto, dicen los proverbistas:
Venid á Hesbón, edifíquese y repárese la ciudad de Sehón: \bibverse{28}
Que fuego salió de Hesbón, y llama de la ciudad de Sehón, y consumió á
Ar de Moab, á los señores de los altos de Arnón. \bibverse{29} ¡Ay de
ti, Moab! Perecido has, pueblo de Chêmos: puso sus hijos en huída, y sus
hijas en cautividad, por Sehón rey de los Amorrheos. \bibverse{30} Mas
devastamos el reino de ellos; pereció Hesbón hasta Dibón, y destruimos
hasta Nopha y Medeba.

\hypertarget{mayor-avance-de-los-israelitas-derrota-del-rey-og-de-basan}{%
\subsection{Mayor avance de los israelitas; Derrota del rey Og de
Basan}\label{mayor-avance-de-los-israelitas-derrota-del-rey-og-de-basan}}

\bibverse{31} Así habitó Israel en la tierra del Amorrheo. \bibverse{32}
Y envió Moisés á reconocer á Jazer; y tomaron sus aldeas, y echaron al
Amorrheo que estaba allí. \bibverse{33} Y volvieron, y subieron camino
de Basán, y salió contra ellos Og rey de Basán, él y todo su pueblo,
para pelear en Edrei.

\bibverse{34} Entonces Jehová dijo á Moisés: No le tengas miedo, que en
tu mano lo he dado, á él y á todo su pueblo, y á su tierra; y harás de
él como hiciste de Sehón, rey de los Amorrheos, que habitaba en Hesbón.
\footnote{\textbf{21:34} Sal 136,17-22}

\bibverse{35} E hirieron á él, y á sus hijos, y á toda su gente, sin que
le quedara uno, y poseyeron su tierra.

\hypertarget{der-moabiterkuxf6nig-balak-beschlieuxdft-gesandte-an-bileam-zu-schicken}{%
\subsection{Der Moabiterkönig Balak beschließt, Gesandte an Bileam zu
schicken}\label{der-moabiterkuxf6nig-balak-beschlieuxdft-gesandte-an-bileam-zu-schicken}}

\hypertarget{section-21}{%
\section{22}\label{section-21}}

\bibverse{1} Y movieron los hijos de Israel, y asentaron en los campos
de Moab, de esta parte del Jordán de Jericó. \bibverse{2} Y vió Balac,
hijo de Zippor, todo lo que Israel había hecho al Amorrheo. \bibverse{3}
Y Moab temió mucho á causa del pueblo que era mucho; y angustióse Moab á
causa de los hijos de Israel. \bibverse{4} Y dijo Moab á los ancianos de
Madián: Ahora lamerá esta gente todos nuestros contornos, como lame el
buey la grama del campo. Y Balac, hijo de Zippor, era entonces rey de
Moab.

\bibverse{5} Por tanto envió mensajeros á Balaam hijo de Beor, á Pethor,
que está junto al río en la tierra de los hijos de su pueblo, para que
lo llamasen, diciendo: Un pueblo ha salido de Egipto, y he aquí cubre la
haz de la tierra, y habita delante de mí: \bibverse{6} Ven pues ahora,
te ruego, maldíceme este pueblo, porque es más fuerte que yo: quizá
podré yo herirlo, y echarlo de la tierra: que yo sé que el que tú
bendijeres, será bendito, y el que tú maldijeres, será maldito.

\hypertarget{la-primera-embajada-de-balac-a-balaam-sin-uxe9xito-su-mensaje-repetido}{%
\subsection{La primera embajada de Balac a Balaam sin éxito; su mensaje
repetido}\label{la-primera-embajada-de-balac-a-balaam-sin-uxe9xito-su-mensaje-repetido}}

\bibverse{7} Y fueron los ancianos de Moab, y los ancianos de Madián,
con las dádivas de adivinación en su mano, y llegaron á Balaam, y le
dijeron las palabras de Balac. \footnote{\textbf{22:7} 2Pe 2,15}

\bibverse{8} Y él les dijo: Reposad aquí esta noche, y yo os referiré
las palabras, como Jehová me hablare. Así los príncipes de Moab se
quedaron con Balaam.

\bibverse{9} Y vino Dios á Balaam, y díjole: ¿Qué varones son estos que
están contigo?

\bibverse{10} Y Balaam respondió á Dios: Balac hijo de Zippor, rey de
Moab, ha enviado á mí diciendo: \bibverse{11} He aquí este pueblo que ha
salido de Egipto, cubre la haz de la tierra: ven pues ahora, y
maldícemelo; quizá podré pelear con él, y echarlo.

\bibverse{12} Entonces dijo Dios á Balaam: No vayas con ellos, ni
maldigas al pueblo; porque es bendito.

\bibverse{13} Así Balaam se levantó por la mañana, y dijo á los
príncipes de Balac: Volveos á vuestra tierra, porque Jehová no me quiere
dejar ir con vosotros.

\bibverse{14} Y los príncipes de Moab se levantaron, y vinieron á Balac,
y dijeron: Balaam no quiso venir con nosotros.

\bibverse{15} Y tornó Balac á enviar otra vez más príncipes, y más
honorables que los otros. \bibverse{16} Los cuales vinieron á Balaam, y
dijéronle: Así dice Balac, hijo de Zippor: Ruégote que no dejes de venir
á mí: \bibverse{17} Porque sin duda te honraré mucho, y haré todo lo que
me dijeres: ven pues ahora, maldíceme á este pueblo.

\bibverse{18} Y Balaam respondió, y dijo á los siervos de Balac: Aunque
Balac me diese su casa llena de plata y oro, no puedo traspasar la
palabra de Jehová mi Dios, para hacer cosa chica ni grande.
\bibverse{19} Ruégoos por tanto ahora, que reposéis aquí esta noche,
para que yo sepa qué me vuelve á decir Jehová.

\bibverse{20} Y vino Dios á Balaam de noche, y díjole: Si vinieren á
llamarte hombres, levántate y ve con ellos: empero harás lo que yo te
dijere.

\bibverse{21} Así Balaam se levantó por la mañana, y cinchó su asna, y
fué con los príncipes de Moab.

\hypertarget{el-viaje-de-balaam-a-moab-y-el-incidente-con-el-burro}{%
\subsection{El viaje de Balaam a Moab y el incidente con el
burro}\label{el-viaje-de-balaam-a-moab-y-el-incidente-con-el-burro}}

\bibverse{22} Y el furor de Dios se encendió porque él iba; y el ángel
de Jehová se puso en el camino por adversario suyo. Iba, pues, él
montado sobre su asna, y con él dos mozos suyos. \bibverse{23} Y el asna
vió al ángel de Jehová, que estaba en el camino con su espada desnuda en
su mano; y apartóse el asna del camino, é iba por el campo. Entonces
hirió Balaam al asna para hacerla volver al camino. \footnote{\textbf{22:23}
  Gén 3,24; Jos 5,13} \bibverse{24} Mas el ángel de Jehová se puso en
una senda de viñas que tenía pared de una parte y pared de otra.
\bibverse{25} Y viendo el asna al ángel de Jehová, pegóse á la pared, y
apretó contra la pared el pie de Balaam: y él volvió á herirla.

\bibverse{26} Y el ángel de Jehová pasó más allá, y púsose en una
angostura, donde no había camino para apartarse ni á diestra ni á
siniestra. \bibverse{27} Y viendo el asna al ángel de Jehová, echóse
debajo de Balaam: y enojóse Balaam, é hirió al asna con el palo.

\bibverse{28} Entonces Jehová abrió la boca al asna, la cual dijo á
Balaam: ¿Qué te he hecho, que me has herido estas tres veces?

\bibverse{29} Y Balaam respondió al asna: Porque te has burlado de mí:
¡ojalá tuviera espada en mi mano, que ahora te mataría!

\bibverse{30} Y el asna dijo á Balaam: ¿No soy yo tu asna? sobre mí has
cabalgado desde que tú me tienes hasta este día; ¿he acostumbrado á
hacerlo así contigo? Y él respondió: No.~

\bibverse{31} Entonces Jehová abrió los ojos á Balaam, y vió al ángel de
Jehová que estaba en el camino, y tenía su espada desnuda en su mano. Y
Balaam hizo reverencia, é inclinóse sobre su rostro. \bibverse{32} Y el
ángel de Jehová le dijo: ¿Por qué has herido tu asna estas tres veces?
he aquí yo he salido para contrarrestarte, porque tu camino es perverso
delante de mí: \bibverse{33} El asna me ha visto, y hase apartado luego
de delante de mí estas tres veces: y si de mí no se hubiera apartado, yo
también ahora te mataría á ti, y á ella dejaría viva.

\bibverse{34} Entonces Balaam dijo al ángel de Jehová: He pecado, que no
sabía que tú te ponías delante de mí en el camino: mas ahora, si te
parece mal, yo me volveré.

\bibverse{35} Y el ángel de Jehová dijo á Balaam: Ve con esos hombres:
empero la palabra que yo te dijere, esa hablarás. Así Balaam fué con los
príncipes de Balac.

\hypertarget{la-llegada-de-balaam-a-balac}{%
\subsection{La llegada de Balaam a
Balac}\label{la-llegada-de-balaam-a-balac}}

\bibverse{36} Y oyendo Balac que Balaam venía, salió á recibirlo á la
ciudad de Moab, que está junto al término de Arnón, que es el cabo de
los confines. \bibverse{37} Y Balac dijo á Balaam: ¿No envié yo á ti á
llamarte? ¿por qué no has venido á mí? ¿no puedo yo honrarte?

\bibverse{38} Y Balaam respondió á Balac: He aquí yo he venido á ti: mas
¿podré ahora hablar alguna cosa? La palabra que Dios pusiere en mi boca,
esa hablaré.

\bibverse{39} Y fué Balaam con Balac, y vinieron á la ciudad de Husoth.
\bibverse{40} Y Balac hizo matar bueyes y ovejas, y envió á Balaam, y á
los príncipes que estaban con él. \bibverse{41} Y el día siguiente Balac
tomó á Balaam, é hízolo subir á los altos de Baal, y desde allí vió la
extremidad del pueblo. \footnote{\textbf{22:41} Núm 23,28}

\hypertarget{los-preparativos-para-la-revelaciuxf3n-divina-el-primer-dicho-de-balaam}{%
\subsection{Los preparativos para la revelación divina; el primer dicho
de
Balaam}\label{los-preparativos-para-la-revelaciuxf3n-divina-el-primer-dicho-de-balaam}}

\hypertarget{section-22}{%
\section{23}\label{section-22}}

\bibverse{1} Y balaam dijo á Balac: Edifícame aquí siete altares, y
prepárame aquí siete becerros y siete carneros.

\bibverse{2} Y Balac hizo como le dijo Balaam: y ofrecieron Balac y
Balaam un becerro y un carnero en cada altar. \bibverse{3} Y Balaam dijo
á Balac: Ponte junto á tu holocausto, y yo iré: quizá Jehová me vendrá
al encuentro, y cualquiera cosa que me mostrare, te la noticiaré. Y así
se fué solo.

\bibverse{4} Y vino Dios al encuentro de Balaam, y éste le dijo: Siete
altares he ordenado, y en cada altar he ofrecido un becerro y un
carnero.

\bibverse{5} Y Jehová puso palabra en la boca de Balaam, y díjole:
Vuelve á Balac, y has de hablar así.

\bibverse{6} Y volvió á él, y he aquí estaba él junto á su holocausto,
él y todos los príncipes de Moab.

\hypertarget{balaam-bendice-a-israel-desde-bamot-baal}{%
\subsection{Balaam bendice a Israel desde
Bamot-Baal}\label{balaam-bendice-a-israel-desde-bamot-baal}}

\bibverse{7} Y él tomó su parábola, y dijo: De Aram me trajo Balac, rey
de Moab, de los montes del oriente: ven, maldíceme á Jacob; y ven,
execra á Israel. \bibverse{8} ¿Por qué maldeciré yo al que Dios no
maldijo? ¿y por qué he de execrar al que Jehová no ha execrado?
\bibverse{9} Porque de la cumbre de las peñas lo veré, y desde los
collados lo miraré: he aquí un pueblo que habitará confiado, y no será
contado entre las gentes. \bibverse{10} ¿Quién contará el polvo de
Jacob, o el número de la cuarta parte de Israel? Muera mi persona de la
muerte de los rectos, y mi postrimería sea como la suya.

\bibverse{11} Entonces Balac dijo á Balaam: ¿Qué me has hecho? hete
tomado para que maldigas á mis enemigos, y he aquí has proferido
bendiciones.

\bibverse{12} Y él respondió, y dijo: ¿No observaré yo lo que Jehová
pusiere en mi boca para decirlo? \footnote{\textbf{23:12} Núm 22,38}

\hypertarget{los-preparativos-para-la-nueva-revelaciuxf3n-divina-el-segundo-dicho-de-balaam}{%
\subsection{Los preparativos para la nueva revelación divina; el segundo
dicho de
Balaam}\label{los-preparativos-para-la-nueva-revelaciuxf3n-divina-el-segundo-dicho-de-balaam}}

\bibverse{13} Y dijo Balac: Ruégote que vengas conmigo á otro lugar
desde el cual lo veas; su extremidad solamente verás, que no lo verás
todo; y desde allí me lo maldecirás.

\bibverse{14} Y llevólo al campo de Sophim, á la cumbre de Pisga, y
edificó siete altares, y ofreció un becerro y un carnero en cada altar.
\bibverse{15} Entonces él dijo á Balac: Ponte aquí junto á tu
holocausto, y yo iré á encontrar á Dios allí.

\bibverse{16} Y Jehová salió al encuentro de Balaam, y puso palabra en
su boca, y díjole: Vuelve á Balac, y así has de decir.

\bibverse{17} Y vino á él, y he aquí que él estaba junto á su
holocausto, y con él los príncipes de Moab: y díjole Balac: ¿Qué ha
dicho Jehová?

\hypertarget{balaam-bendice-a-israel-desde-el-monte-pisga}{%
\subsection{Balaam bendice a Israel desde el monte
Pisga}\label{balaam-bendice-a-israel-desde-el-monte-pisga}}

\bibverse{18} Entonces él tomó su parábola, y dijo: Balac, levántate y
oye; escucha mis palabras, hijo de Zippor: \bibverse{19} Dios no es
hombre, para que mienta; ni hijo de hombre para que se arrepienta: el
dijo, ¿y no hará?; habló, ¿y no lo ejecutará? \bibverse{20} He aquí, yo
he tomado bendición: y él bendijo, y no podré revocarla. \bibverse{21}
No ha notado iniquidad en Jacob, ni ha visto perversidad en Israel:
Jehová su Dios es con él, y júbilo de rey en él. \bibverse{22} Dios los
ha sacado de Egipto; tiene fuerzas como de unicornio. \bibverse{23}
Porque en Jacob no hay agüero, ni adivinación en Israel: como ahora,
será dicho de Jacob y de Israel: ¡Lo que ha hecho Dios! \bibverse{24} He
aquí el pueblo, que como león se levantará, y como león se erguirá: no
se echará hasta que coma la presa, y beba la sangre de los muertos.
\footnote{\textbf{23:24} Núm 24,9}

\bibverse{25} Entonces Balac dijo á Balaam: Ya que no lo maldices, ni
tampoco lo bendigas.

\bibverse{26} Y Balaam respondió, y dijo á Balac: ¿No te he dicho que
todo lo que Jehová me dijere, aquello tengo de hacer?

\hypertarget{los-preparativos-para-la-tercera-revelaciuxf3n-divina-el-tercer-dicho-de-balaam}{%
\subsection{Los preparativos para la tercera revelación divina; el
tercer dicho de
Balaam}\label{los-preparativos-para-la-tercera-revelaciuxf3n-divina-el-tercer-dicho-de-balaam}}

\bibverse{27} Y dijo Balac á Balaam: Ruégote que vengas, te llevaré á
otro lugar; por ventura parecerá bien á Dios que desde allí me lo
maldigas.

\bibverse{28} Y Balac llevó á Balaam á la cumbre de Peor, que mira hacia
Jesimón. \footnote{\textbf{23:28} Núm 25,3} \bibverse{29} Entonces
Balaam dijo á Balac: Edifícame aquí siete altares, y prepárame aquí
siete becerros y siete carneros. \footnote{\textbf{23:29} Núm 23,1}

\bibverse{30} Y Balac hizo como Balaam le dijo; y ofreció un becerro y
un carnero en cada altar.

\hypertarget{balaam-bendice-a-israel-desde-el-monte-peor}{%
\subsection{Balaam bendice a Israel desde el monte
Peor}\label{balaam-bendice-a-israel-desde-el-monte-peor}}

\hypertarget{section-23}{%
\section{24}\label{section-23}}

\bibverse{1} Y como vió Balaam que parecía bien á Jehová que él
bendijese á Israel, no fué, como la primera y segunda vez, á encuentro
de agüeros, sino que puso su rostro hacia el desierto; \bibverse{2} Y
alzando sus ojos, vió á Israel alojado por sus tribus; y el espíritu de
Dios vino sobre él. \bibverse{3} Entonces tomó su parábola, y dijo: Dijo
Balaam hijo de Beor, y dijo el varón de ojos abiertos: \footnote{\textbf{24:3}
  1Sam 9,9} \bibverse{4} Dijo el que oyó los dichos de Dios, el que vió
la visión del Omnipotente; caído, mas abiertos los ojos: \footnote{\textbf{24:4}
  Is 50,4} \bibverse{5} ¡Cuán hermosas son tus tiendas, oh Jacob, tus
habitaciones, oh Israel! \bibverse{6} Como arroyos están extendidas,
como huertos junto al río, como lináloes plantados por Jehová, como
cedros junto á las aguas. \bibverse{7} De sus manos destilarán aguas, y
su simiente será en muchas aguas: y ensalzarse ha su rey más que Agag, y
su reino será ensalzado. \bibverse{8} Dios lo sacó de Egipto; tiene
fuerzas como de unicornio: comerá á las gentes sus enemigas, y
desmenuzará sus huesos, y asaeteará con sus saetas. \bibverse{9} Se
encorvará para echarse como león, y como leona; ¿quién lo despertará?
Benditos los que te bendijeren, y malditos los que te maldijeren.
\footnote{\textbf{24:9} Núm 23,24; Gén 49,9; Gén 12,3}

\hypertarget{la-ira-de-balac-y-la-disculpa-de-balaam}{%
\subsection{La ira de Balac y la disculpa de
Balaam}\label{la-ira-de-balac-y-la-disculpa-de-balaam}}

\bibverse{10} Entonces se encendió la ira de Balac contra Balaam, y
batiendo sus palmas le dijo: Para maldecir á mis enemigos te he llamado,
y he aquí los has resueltamente bendecido ya tres veces. \bibverse{11}
Húyete, por tanto, ahora á tu lugar: yo dije que te honraría, mas he
aquí que Jehová te ha privado de honra.

\bibverse{12} Y Balaam le respondió: ¿No lo declaré yo también á tus
mensajeros que me enviaste, diciendo: \bibverse{13} Si Balac me diese su
casa llena de plata y oro, yo no podré traspasar el dicho de Jehová para
hacer cosa buena ni mala de mi arbitrio; mas lo que Jehová hablare, eso
diré yo? \bibverse{14} He aquí yo me voy ahora á mi pueblo: por tanto,
ven, te indicaré lo que este pueblo ha de hacer á tu pueblo en los
postrimeros días.

\hypertarget{cuarto-dicho-de-balaam-la-estrella-de-jacob-cuya-victoria-sobre-moab-y-edom}{%
\subsection{Cuarto dicho de Balaam: la estrella de Jacob; cuya victoria
sobre Moab y
Edom}\label{cuarto-dicho-de-balaam-la-estrella-de-jacob-cuya-victoria-sobre-moab-y-edom}}

\bibverse{15} Y tomó su parábola, y dijo: Dijo Balaam hijo de Beor, dijo
el varón de ojos abiertos: \footnote{\textbf{24:15} Núm 24,3-4}
\bibverse{16} Dijo el que oyó los dichos de Jehová, y el que sabe la
ciencia del Altísimo, el que vió la visión del Omnipotente; caído, mas
abiertos los ojos: \bibverse{17} Verélo, mas no ahora: lo miraré, mas no
de cerca: saldrá ESTRELLA de Jacob, y levantaráse cetro de Israel, y
herirá los cantones de Moab, y destruirá á todos los hijos de Seth.
\bibverse{18} Y será tomada Edom, será también tomada Seir por sus
enemigos, e Israel se portará varonilmente. \footnote{\textbf{24:18}
  2Sam 8,14; Am 9,11; Am 1,9-12} \bibverse{19} Y el de Jacob se
enseñoreará, y destruirá de la ciudad lo que quedare. \footnote{\textbf{24:19}
  Miq 5,1; Miq 5,7-8}

\hypertarget{proverbios-sobre-los-amalecitas-ceneos-y-asirios-fin-de-la-historia-de-balaam}{%
\subsection{Proverbios sobre los amalecitas, ceneos y asirios; Fin de la
historia de
Balaam}\label{proverbios-sobre-los-amalecitas-ceneos-y-asirios-fin-de-la-historia-de-balaam}}

\bibverse{20} Y viendo á Amalec, tomó su parábola, y dijo: Amalec,
cabeza de gentes; mas su postrimería perecerá para siempre. \footnote{\textbf{24:20}
  Éxod 17,14}

\bibverse{21} Y viendo al Cineo, tomó su parábola, y dijo: Fuerte es tu
habitación, pon en la peña tu nido: \footnote{\textbf{24:21} 1Sam 15,6;
  Abd 1,3} \bibverse{22} Que el Cineo será echado, cuando Assur te
llevará cautivo.

\bibverse{23} Todavía tomó su parábola, y dijo: ¡Ay! ¿quién vivirá
cuando hiciere Dios estas cosas? \bibverse{24} Y vendrán navíos de la
costa de Cittim, y afligirán á Assur, afligirán también á Eber: mas él
también perecerá para siempre. \footnote{\textbf{24:24} 1Macc 1,1}

\bibverse{25} Entonces se levantó Balaam, y se fué, y volvióse á su
lugar: y también Balac se fué por su camino.

\hypertarget{la-deuda-de-israel-a-travuxe9s-de-la-fornicaciuxf3n-y-la-idolatruxeda}{%
\subsection{La deuda de Israel a través de la fornicación y la
idolatría}\label{la-deuda-de-israel-a-travuxe9s-de-la-fornicaciuxf3n-y-la-idolatruxeda}}

\hypertarget{section-24}{%
\section{25}\label{section-24}}

\bibverse{1} Y reposó Israel en Sittim, y el pueblo empezó á fornicar
con las hijas de Moab: \bibverse{2} Las cuales llamaron al pueblo á los
sacrificios de sus dioses: y el pueblo comió, é inclinóse á sus dioses.
\bibverse{3} Y allegóse el pueblo á Baal-peor; y el furor de Jehová se
encendió contra Israel. \footnote{\textbf{25:3} Deut 4,3} \bibverse{4} Y
Jehová dijo á Moisés: Toma todos los príncipes del pueblo, y ahórcalos á
Jehová delante del sol; y la ira del furor de Jehová se apartará de
Israel. \footnote{\textbf{25:4} 2Sam 21,6; 2Sam 21,9; Deut 21,22-23}

\bibverse{5} Entonces Moisés dijo á los jueces de Israel: Matad cada uno
á aquellos de los suyos que se han allegado á Baal-peor.

\hypertarget{la-intervenciuxf3n-de-phinehas-su-enajenaciuxf3n-de-dios-con-un-sacerdocio-eterno}{%
\subsection{La intervención de Phinehas; su enajenación de Dios con un
sacerdocio
eterno}\label{la-intervenciuxf3n-de-phinehas-su-enajenaciuxf3n-de-dios-con-un-sacerdocio-eterno}}

\bibverse{6} Y he aquí un varón de los hijos de Israel vino y trajo una
Madianita á sus hermanos, á ojos de Moisés y de toda la congregación de
los hijos de Israel, llorando ellos á la puerta del tabernáculo del
testimonio. \bibverse{7} Y viólo Phinees, hijo de Eleazar, hijo de Aarón
el sacerdote, y levantóse de en medio de la congregación, y tomó una
lanza en su mano: \bibverse{8} Y fué tras el varón de Israel á la
tienda, y alanceólos á ambos, al varón de Israel, y á la mujer por su
vientre. Y cesó la mortandad de los hijos de Israel. \bibverse{9} Y
murieron de aquella mortandad veinte y cuatro mil. \footnote{\textbf{25:9}
  1Cor 10,8}

\bibverse{10} Entonces Jehová habló á Moisés, diciendo: \bibverse{11}
Phinees, hijo de Eleazar, hijo de Aarón el sacerdote, ha hecho tornar mi
furor de los hijos de Israel, llevado de celo entre ellos: por lo cual
yo no he consumido en mi celo á los hijos de Israel. \bibverse{12} Por
tanto diles: He aquí yo establezco mi pacto de paz con él; \bibverse{13}
Y tendrá él, y su simiente después de él, el pacto del sacerdocio
perpetuo; por cuanto tuvo celo por su Dios, é hizo expiación por los
hijos de Israel. \footnote{\textbf{25:13} Sal 106,30-31}

\bibverse{14} Y el nombre del varón muerto, que fué muerto con la
Madianita, era Zimri hijo de Salu, jefe de una familia de la tribu de
Simeón. \bibverse{15} Y el nombre de la mujer Madianita muerta, era
Cozbi, hija de Zur, príncipe de pueblos, padre de familia en Madián.

\hypertarget{gottes-gebot-an-den-midianitern-rache-zu-nehmen}{%
\subsection{Gottes Gebot, an den Midianitern Rache zu
nehmen}\label{gottes-gebot-an-den-midianitern-rache-zu-nehmen}}

\bibverse{16} Y Jehová habló á Moisés, diciendo: \bibverse{17}
Hostilizaréis á los Madianitas, y los heriréis: \footnote{\textbf{25:17}
  Núm 31,2-10}

\bibverse{18} Por cuanto ellos os afligieron á vosotros con sus ardides,
con que os han engañado en el negocio de Peor, y en el negocio de Cozbi,
hija del príncipe de Madián, su hermana, la cual fué muerta el día de la
mortandad por causa de Peor.

\hypertarget{el-segundo-censo-de-la-gente-en-la-llanura-de-los-moabitas-con-el-propuxf3sito-de-distribuir-la-tierra}{%
\subsection{El segundo censo de la gente en la llanura de los moabitas
con el propósito de distribuir la
tierra}\label{el-segundo-censo-de-la-gente-en-la-llanura-de-los-moabitas-con-el-propuxf3sito-de-distribuir-la-tierra}}

\hypertarget{section-25}{%
\section{26}\label{section-25}}

\bibverse{1} Y aconteció después de la mortandad, que Jehová habló á
Moisés, y á Eleazar hijo del sacerdote Aarón, diciendo: \bibverse{2}
Tomad la suma de toda la congregación de los hijos de Israel, de veinte
años arriba, por las casas de sus padres, todos los que puedan salir á
la guerra en Israel. \bibverse{3} Y Moisés y Eleazar el sacerdote
hablaron con ellos en los campos de Moab, junto al Jordán de Jericó,
diciendo: \bibverse{4} Contaréis el pueblo de veinte años arriba, como
mandó Jehová á Moisés y á los hijos de Israel, que habían salido de
tierra de Egipto.

\hypertarget{los-resultados-del-censo-1}{%
\subsection{Los resultados del censo}\label{los-resultados-del-censo-1}}

\bibverse{5} Rubén primogénito de Israel: los hijos de Rubén: Hanoc, del
cual era la familia de los Hanochîtas; de Phallú, la familia de los
Phalluitas; \footnote{\textbf{26:5} Gén 46,8-27; 1Cró 4,1-7}
\bibverse{6} De Hesrón, la familia de los Hesronitas; de Carmi, la
familia de los Carmitas. \bibverse{7} Estas son las familias de los
Rubenitas: y sus contados fueron cuarenta y tres mil setecientos y
treinta. \bibverse{8} Y los hijos de Phallú: Eliab. \bibverse{9} Y los
hijos de Eliab: Nemuel, y Dathán, y Abiram. Estos Dathán y Abiram fueron
los del consejo de la congregación, que hicieron el motín contra Moisés
y Aarón con la compañía de Coré, cuando se amotinaron contra Jehová;
\bibverse{10} Que la tierra abrió su boca y tragó á ellos y á Coré,
cuando aquella compañía murió, cuando consumió el fuego doscientos y
cincuenta varones, los cuales fueron por señal. \bibverse{11} Mas los
hijos de Coré no murieron.

\bibverse{12} Los hijos de Simeón por sus familias: de Nemuel, la
familia de los Nemuelitas; de Jamín, la familia de los Jaminitas; de
Jachîn, la familia de los Jachînitas; \bibverse{13} De Zera, la familia
de los Zeraitas; de Saul, la familia de los Saulitas. \bibverse{14}
Estas son las familias de los Simeonitas, veinte y dos mil y doscientos.

\bibverse{15} Los hijos de Gad por sus familias: de Zephón, la familia
de los Zephonitas; de Aggi, la familia de los Aggitas; de Suni, la
familia de los Sunitas; \bibverse{16} De Ozni, la familia de los
Oznitas; de Eri, la familia de los Eritas; \bibverse{17} De Aroz, la
familia de los Aroditas; de Areli, la familia de los Arelitas.
\bibverse{18} Estas son las familias de Gad, por sus contados, cuarenta
mil y quinientos.

\bibverse{19} Los hijos de Judá: Er y Onán; y Er y Onán murieron en la
tierra de Canaán. \footnote{\textbf{26:19} Gén 38,7; Gén 38,10}
\bibverse{20} Y fueron los hijos de Judá por sus familias: de Sela, la
familia de los Selaitas; de Phares, la familia de los Pharesitas; de
Zera, la familia de los Zeraitas. \bibverse{21} Y fueron los hijos de
Phares: de Hesrón, la familia de los Hesronitas; de Hamul, la familia de
los Hamulitas. \bibverse{22} Estas son las familias de Judá, por sus
contados, setenta y seis mil y quinientos.

\bibverse{23} Los hijos de Issachâr por sus familias: de Thola, la
familia de los Tholaitas; de Puá la familia de los Puanitas;
\bibverse{24} De Jasub, la familia de los Jasubitas; de Simron, la
familia de los Simronitas. \bibverse{25} Estas son las familias de
Issachâr, por sus contados, sesenta y cuatro mil y trescientos.

\bibverse{26} Los hijos de Zabulón por sus familias: de Sered, la
familia de los Sereditas; de Elón, la familia de los Elonitas; de Jalel,
la familia de los Jalelitas. \bibverse{27} Estas son las familias de los
Zabulonitas, por sus contados, sesenta mil y quinientos.

\bibverse{28} Los hijos de José por sus familias: Manasés y Ephraim.
\bibverse{29} Los hijos de Manasés: de Machîr, la familia de los
Machîritas; y Machîr engendró á Galaad; de Galaad, la familia de los
Galaaditas. \footnote{\textbf{26:29} Jos 17,1-3} \bibverse{30} Estos son
los hijos de Galaad: de Jezer, la familia de los Jezeritas; de Helec, la
familia de los Helecitas; \bibverse{31} De Asriel, la familia de los
Asrielitas: de Sechêm, la familia de los Sechêmitas; \bibverse{32} De
Semida, la familia de los Semidaitas; de Hepher, la familia de los
Hepheritas. \bibverse{33} Y Salphaad, hijo de Hepher, no tuvo hijos sino
hijas: y los nombres de las hijas de Salphaad fueron Maala, y Noa, y
Hogla, y Milca, y Tirsa. \bibverse{34} Estas son las familias de
Manasés; y sus contados, cincuenta y dos mil y setecientos.

\bibverse{35} Estos son los hijos de Ephraim por sus familias: de
Suthala, la familia de los Suthalaitas; de Bechêr, la familia de los
Bechêritas; de Tahan, la familia de los Tahanitas. \bibverse{36} Y estos
son los hijos de Suthala: de Herán, la familia de los Heranitas.
\bibverse{37} Estas son las familias de los hijos de Ephraim, por sus
contados, treinta y dos mil y quinientos. Estos son los hijos de José
por sus familias.

\bibverse{38} Los hijos de Benjamín por sus familias: de Bela, la
familia de los Belaitas; de Asbel, la familia de los Asbelitas; de
Achîram, la familia de los Achîramitas; \bibverse{39} De Supham, la
familia de los Suphamitas; de Hupham, la familia de los Huphamitas.
\bibverse{40} Y los hijos de Bela fueron Ard y Naamán: de Ard, la
familia de los Arditas; de Naamán, la familia de los Naamanitas.
\bibverse{41} Estos son los hijos de Benjamín por sus familias; y sus
contados, cuarenta y cinco mil y seiscientos.

\bibverse{42} Estos son los hijos de Dan por sus familias: de Suham, la
familia de los Suhamitas. Estas son las familias de Dan por sus
familias. \bibverse{43} Todas las familias de los Suhamitas, por sus
contados, sesenta y cuatro mil y cuatrocientos.

\bibverse{44} Los hijos de Aser por sus familias: de Imna, la familia de
los Imnaitas; de Issui, la familia de los Issuitas; de Beria, la familia
de los Beriaitas. \bibverse{45} Los hijos de Beria: de Heber, la familia
de los Heberitas; de Malchîel, la familia de los Malchîelitas.
\bibverse{46} Y el nombre de la hija de Aser fué Sera. \bibverse{47}
Estas son las familias de los hijos de Aser, por sus contados, cincuenta
y tres mil y cuatrocientos.

\bibverse{48} Los hijos de Nephtalí por sus familias: de Jahzeel, la
familia de los Jahzeelitas; de Guni, la familia de los Gunitas;
\bibverse{49} De Jeser, la familia de los Jeseritas; de Sillem, la
familia de los Sillemitas. \bibverse{50} Estas son las familias de
Nephtalí por sus familias; y sus contados, cuarenta y cinco mil y
cuatrocientos.

\bibverse{51} Estos son los contados de los hijos de Israel, seiscientos
y un mil setecientos y treinta.

\hypertarget{instrucciuxf3n-sobre-distribuciuxf3n-de-tierras}{%
\subsection{Instrucción sobre distribución de
tierras}\label{instrucciuxf3n-sobre-distribuciuxf3n-de-tierras}}

\bibverse{52} Y habló Jehová á Moisés, diciendo: \bibverse{53} A estos
se repartirá la tierra en heredad, por la cuenta de los nombres.
\bibverse{54} A los más darás mayor heredad, y á los menos menor; y á
cada uno se le dará su heredad conforme á sus contados. \bibverse{55}
Empero la tierra será repartida por suerte; y por los nombres de las
tribus de sus padres heredarán. \footnote{\textbf{26:55} Núm 33,54; Jos
  14,2} \bibverse{56} Conforme á la suerte será repartida su heredad
entre el grande y el pequeño.

\hypertarget{el-conteo-de-los-levitas}{%
\subsection{El conteo de los levitas}\label{el-conteo-de-los-levitas}}

\bibverse{57} Y los contados de los Levitas por sus familias son estos:
de Gersón, la familia de los Gersonitas; de Coath, la familia de los
Coathitas; de Merari, la familia de los Meraritas. \bibverse{58} Estas
son las familias de los Levitas: la familia de los Libnitas, la familia
de los Hebronitas, la familia de los Mahalitas, la familia de los
Musitas, la familia de los Coritas. Y Coath engendró á Amram.
\bibverse{59} Y la mujer de Amram se llamó Jochâbed, hija de Leví, la
cual nació á Leví en Egipto: ésta parió de Amram á Aarón y á Moisés, y á
María su hermana. \bibverse{60} Y á Aarón nacieron Nadab y Abiú, Eleazar
é Ithamar. \bibverse{61} Mas Nadab y Abiú murieron, cuando ofrecieron
fuego extraño delante de Jehová. \footnote{\textbf{26:61} Lev 10,1-2}
\bibverse{62} Y los contados de los Levitas fueron veinte y tres mil,
todos varones de un mes arriba: porque no fueron contados entre los
hijos de Israel, por cuanto no les había de ser dada heredad entre los
hijos de Israel.

\bibverse{63} Estos son los contados por Moisés y Eleazar el sacerdote,
los cuales contaron los hijos de Israel en los campos de Moab, junto al
Jordán de Jericó. \bibverse{64} Y entre estos ninguno hubo de los
contados por Moisés y Aarón el sacerdote, los cuales contaron á los
hijos de Israel en el desierto de Sinaí. \bibverse{65} Porque Jehová les
dijo: Han de morir en el desierto: y no quedó varón de ellos, sino Caleb
hijo de Jephone, y Josué hijo de Nun. \footnote{\textbf{26:65} Núm
  14,22-38}

\hypertarget{disposiciones-relativas-a-la-propiedad-de-los-herederos}{%
\subsection{Disposiciones relativas a la propiedad de los
herederos}\label{disposiciones-relativas-a-la-propiedad-de-los-herederos}}

\hypertarget{section-26}{%
\section{27}\label{section-26}}

\bibverse{1} Y las hijas de Salphaad, hijo de Hepher, hijo de Galaad,
hijo de Machîr, hijo de Manasés, de las familias de Manasés, hijo de
José, los nombres de las cuales eran Maala, y Noa, y Hogla, y Milca, y
Tirsa, llegaron; \footnote{\textbf{27:1} Núm 26,33; Núm 36,2; Jos 17,3-6}
\bibverse{2} Y presentáronse delante de Moisés, y delante del sacerdote
Eleazar, y delante de los príncipes, y de toda la congregación, á la
puerta del tabernáculo del testimonio, y dijeron: \bibverse{3} Nuestro
padre murió en el desierto, el cual no estuvo en la junta que se reunió
contra Jehová en la compañía de Coré: sino que en su pecado murió, y no
tuvo hijos. \footnote{\textbf{27:3} Núm 16,2; Núm 26,65} \bibverse{4}
¿Por qué será quitado el nombre de nuestro padre de entre su familia,
por no haber tenido hijo? Danos heredad entre los hermanos de nuestro
padre.

\bibverse{5} Y Moisés llevó su causa delante de Jehová. \bibverse{6} Y
Jehová respondió á Moisés, diciendo: \bibverse{7} Bien dicen las hijas
de Salphaad: has de darles posesión de heredad entre los hermanos de su
padre; y traspasarás la heredad de su padre á ellas. \bibverse{8} Y á
los hijos de Israel hablarás, diciendo: Cuando alguno muriere sin hijos,
traspasaréis su herencia á su hija: \bibverse{9} Y si no tuviere hija,
daréis su herencia á sus hermanos: \bibverse{10} Y si no tuviere
hermanos, daréis su herencia á los hermanos de su padre. \bibverse{11} Y
si su padre no tuviere hermanos, daréis su herencia á su pariente más
cercano de su linaje, el cual la poseerá: y será á los hijos de Israel
por estatuto de derecho, como Jehová mandó á Moisés.

\hypertarget{anuncio-de-muerte-inminente-a-moisuxe9s-instalaciuxf3n-de-joshua-como-su-sucesor}{%
\subsection{Anuncio de muerte inminente a Moisés; Instalación de Joshua
como su
sucesor}\label{anuncio-de-muerte-inminente-a-moisuxe9s-instalaciuxf3n-de-joshua-como-su-sucesor}}

\bibverse{12} Y Jehová dijo á Moisés: Sube á este monte Abarim, y verás
la tierra que he dado á los hijos de Israel. \footnote{\textbf{27:12}
  Deut 32,48-49} \bibverse{13} Y después que la habrás visto, tú también
serás reunido á tus pueblos, como fué reunido tu hermano Aarón:
\footnote{\textbf{27:13} Núm 20,24; Núm 20,28} \bibverse{14} Pues
fuisteis rebeldes á mi dicho en el desierto de Zin, en la rencilla de la
congregación, para santificarme en las aguas á ojos de ellos. Estas son
las aguas de la rencilla de Cades en el desierto de Zin. \footnote{\textbf{27:14}
  Núm 20,12-13}

\bibverse{15} Entonces respondió Moisés á Jehová, diciendo:
\bibverse{16} Ponga Jehová, Dios de los espíritus de toda carne, varón
sobre la congregación, \bibverse{17} Que salga delante de ellos, y que
entre delante de ellos, que los saque y los introduzca; porque la
congregación de Jehová no sea como ovejas sin pastor. \footnote{\textbf{27:17}
  Mat 9,36}

\bibverse{18} Y Jehová dijo á Moisés: Toma á Josué hijo de Nun, varón en
el cual hay espíritu, y pondrás tu mano sobre él; \footnote{\textbf{27:18}
  Deut 3,21; Deut 34,9} \bibverse{19} Y ponerlo has delante de Eleazar
el sacerdote, y delante de toda la congregación; y le darás órdenes en
presencia de ellos. \bibverse{20} Y pondrás de tu dignidad sobre él,
para que toda la congregación de los hijos de Israel le obedezcan.
\footnote{\textbf{27:20} 2Re 2,9; 2Re 2,15} \bibverse{21} Y él estará
delante de Eleazar el sacerdote, y á él preguntará por el juicio del
Urim delante de Jehová: por el dicho de él saldrán, y por el dicho de él
entrarán, él, y todos los hijos de Israel con él, y toda la
congregación. \footnote{\textbf{27:21} Éxod 28,30}

\bibverse{22} Y Moisés hizo como Jehová le había mandado; que tomó á
Josué, y le puso delante de Eleazar el sacerdote, y de toda la
congregación: \bibverse{23} Y puso sobre él sus manos, y dióle órdenes,
como Jehová había mandado por mano de Moisés.

\hypertarget{normativa-sobre-los-sacrificios-comunitarios-diarios-y-diarios}{%
\subsection{Normativa sobre los sacrificios comunitarios diarios y
diarios}\label{normativa-sobre-los-sacrificios-comunitarios-diarios-y-diarios}}

\hypertarget{section-27}{%
\section{28}\label{section-27}}

\bibverse{1} Y habló Jehová á Moisés, diciendo: \bibverse{2} Manda á los
hijos de Israel, y diles: Mi ofrenda, mi pan con mis ofrendas encendidas
en olor á mí agradable, guardaréis, ofreciéndomelo á su tiempo.
\footnote{\textbf{28:2} Lev 21,6}

\hypertarget{el-holocausto-diario-de-la-mauxf1ana-y-de-la-tarde}{%
\subsection{El holocausto diario de la mañana y de la
tarde}\label{el-holocausto-diario-de-la-mauxf1ana-y-de-la-tarde}}

\bibverse{3} Y les dirás: Esta es la ofrenda encendida que ofreceréis á
Jehová: dos corderos sin tacha de un año, cada un día, será el
holocausto continuo. \footnote{\textbf{28:3} Éxod 29,38-42} \bibverse{4}
El un cordero ofrecerás por la mañana, y el otro cordero ofrecerás entre
las dos tardes: \bibverse{5} Y la décima de un epha de flor de harina,
amasada con una cuarta de un hin de aceite molido, en presente.
\footnote{\textbf{28:5} Lev 2,1} \bibverse{6} Es holocausto continuo,
que fué hecho en el monte de Sinaí en olor de suavidad, ofrenda
encendida á Jehová. \bibverse{7} Y su libación, la cuarta de un hin con
cada cordero: derramarás libación de superior vino á Jehová en el
santuario. \bibverse{8} Y ofrecerás el segundo cordero entre las dos
tardes: conforme á la ofrenda de la mañana, y conforme á su libación
ofrecerás, ofrenda encendida en olor de suavidad á Jehová.

\hypertarget{la-ofrenda-adicional-del-suxe1bado}{%
\subsection{La ofrenda adicional del
sábado}\label{la-ofrenda-adicional-del-suxe1bado}}

\bibverse{9} Mas el día del sábado dos corderos de un año sin defecto, y
dos décimas de flor de harina amasada con aceite, por presente, con su
libación: \bibverse{10} Es el holocausto del sábado en cada sábado,
además del holocausto continuo, y su libación.

\hypertarget{el-sacrificio-adicional-en-el-duxeda-de-luna-nueva}{%
\subsection{El sacrificio adicional en el día de luna
nueva}\label{el-sacrificio-adicional-en-el-duxeda-de-luna-nueva}}

\bibverse{11} Y en los principios de vuestros meses ofreceréis en
holocausto á Jehová dos becerros de la vacada, y un carnero, y siete
corderos de un año sin defecto; \footnote{\textbf{28:11} Núm 10,10}
\bibverse{12} Y tres décimas de flor de harina amasada con aceite, por
presente con cada becerro; y dos décimas de flor de harina amasada con
aceite, por presente con cada carnero; \footnote{\textbf{28:12} Núm
  28,20; Núm 28,28; Núm 15,2-13} \bibverse{13} Y una décima de flor de
harina amasada con aceite, en ofrenda por presente con cada cordero:
holocausto de olor suave, ofrenda encendida á Jehová. \bibverse{14} Y
sus libaciones de vino, medio hin con cada becerro, y el tercio de un
hin con cada carnero, y la cuarta de un hin con cada cordero. Este es el
holocausto de cada mes por todos los meses del año. \bibverse{15} Y un
macho cabrío en expiación se ofrecerá á Jehová, además del holocausto
continuo con su libación. \footnote{\textbf{28:15} Núm 28,22}

\hypertarget{las-ofrendas-adicionales-para-los-siete-duxedas-de-la-fiesta-de-los-panes-sin-levadura}{%
\subsection{Las ofrendas adicionales para los siete días de la Fiesta de
los Panes sin
Levadura}\label{las-ofrendas-adicionales-para-los-siete-duxedas-de-la-fiesta-de-los-panes-sin-levadura}}

\bibverse{16} Mas en el mes primero, á los catorce del mes será la
pascua de Jehová. \footnote{\textbf{28:16} Lev 23,5-14} \bibverse{17} Y
á los quince días de aqueste mes, la solemnidad: por siete días se
comerán ázimos. \bibverse{18} El primer día, santa convocación; ninguna
obra servil haréis: \footnote{\textbf{28:18} Núm 28,25-26} \bibverse{19}
Y ofreceréis por ofrenda encendida en holocausto á Jehová dos becerros
de la vacada, y un carnero, y siete corderos de un año: sin defecto los
tomaréis: \bibverse{20} Y su presente de harina amasada con aceite: tres
décimas con cada becerro, y dos décimas con cada carnero ofreceréis;
\bibverse{21} Con cada uno de los siete corderos ofreceréis una décima;
\bibverse{22} Y un macho cabrío por expiación, para reconciliaros.
\footnote{\textbf{28:22} Núm 28,15} \bibverse{23} Esto ofreceréis además
del holocausto de la mañana, que es el holocausto continuo.
\bibverse{24} Conforme á esto ofreceréis cada uno de los siete días,
vianda y ofrenda encendida en olor de suavidad á Jehová; ofrecerse ha,
además del holocausto continuo, con su libación. \bibverse{25} Y el
séptimo día tendréis santa convocación: ninguna obra servil haréis.

\hypertarget{los-sacrificios-adicionales-en-la-fiesta-de-las-primicias}{%
\subsection{Los sacrificios adicionales en la fiesta de las
primicias}\label{los-sacrificios-adicionales-en-la-fiesta-de-las-primicias}}

\bibverse{26} Además el día de las primicias, cuando ofreciereis
presente nuevo á Jehová en vuestras semanas, tendréis santa convocación:
ninguna obra servil haréis: \bibverse{27} Y ofreceréis en holocausto, en
olor de suavidad á Jehová, dos becerros de la vacada, un carnero, siete
corderos de un año: \bibverse{28} Y el presente de ellos, flor de harina
amasada con aceite, tres décimas con cada becerro, dos décimas con cada
carnero, \bibverse{29} Con cada uno de los siete corderos una décima;
\bibverse{30} Un macho cabrío, para hacer expiación por vosotros.
\footnote{\textbf{28:30} Núm 28,15}

\bibverse{31} Los ofreceréis, además del holocausto continuo con sus
presentes, y sus libaciones: sin defecto los tomaréis.

\hypertarget{los-sacrificios-adicionales-el-duxeda-de-auxf1o-nuevo}{%
\subsection{Los sacrificios adicionales el día de Año
Nuevo}\label{los-sacrificios-adicionales-el-duxeda-de-auxf1o-nuevo}}

\hypertarget{section-28}{%
\section{29}\label{section-28}}

\bibverse{1} Y el séptimo mes, al primero del mes tendréis santa
convocación: ninguna obra servil haréis; os será día de sonar las
trompetas. \bibverse{2} Y ofreceréis holocausto por olor de suavidad á
Jehová, un becerro de la vacada, un carnero, siete corderos de un año
sin defecto; \bibverse{3} Y el presente de ellos, de flor de harina
amasada con aceite, tres décimas con cada becerro, dos décimas con cada
carnero, \bibverse{4} Y con cada uno de los siete corderos, una décima;
\bibverse{5} Y un macho cabrío por expiación, para reconciliaros:
\bibverse{6} Además del holocausto del mes, y su presente, y el
holocausto continuo y su presente, y sus libaciones, conforme á su ley,
por ofrenda encendida á Jehová en olor de suavidad.

\hypertarget{los-sacrificios-adicionales-en-el-gran-duxeda-de-la-expiaciuxf3n}{%
\subsection{Los sacrificios adicionales en el gran día de la
expiación}\label{los-sacrificios-adicionales-en-el-gran-duxeda-de-la-expiaciuxf3n}}

\bibverse{7} Y en el diez de este mes séptimo tendréis santa
convocación, y afligiréis vuestras almas: ninguna obra haréis:
\footnote{\textbf{29:7} Lev 23,27-32} \bibverse{8} Y ofreceréis en
holocausto á Jehová por olor de suavidad, un becerro de la vacada, un
carnero, siete corderos de un año; sin defecto los tomaréis:
\bibverse{9} Y sus presentes, flor de harina amasada con aceite, tres
décimas con cada becerro, dos décimas con cada carnero, \bibverse{10} Y
con cada uno de los siete corderos, una décima; \bibverse{11} Un macho
cabrío por expiación: además de la ofrenda de las expiaciones por el
pecado, y del holocausto continuo, y de sus presentes, y de sus
libaciones.

\hypertarget{las-ofrendas-adicionales-para-los-siete-duxedas-de-la-fiesta-de-los-tabernuxe1culos}{%
\subsection{Las ofrendas adicionales para los siete días de la Fiesta de
los
Tabernáculos}\label{las-ofrendas-adicionales-para-los-siete-duxedas-de-la-fiesta-de-los-tabernuxe1culos}}

\bibverse{12} También á los quince días del mes séptimo tendréis santa
convocación; ninguna obra servil haréis, y celebraréis solemnidad á
Jehová por siete días; \footnote{\textbf{29:12} Lev 23,34-43}
\bibverse{13} Y ofreceréis en holocausto, en ofrenda encendida á Jehová
en olor de suavidad, trece becerros de la vacada, dos carneros, catorce
corderos de un año: han de ser sin defecto; \bibverse{14} Y los
presentes de ellos, de flor de harina amasada con aceite, tres décimas
con cada uno de los trece becerros, dos décimas con cada uno de los dos
carneros, \bibverse{15} Y con cada uno de los catorce corderos, una
décima; \bibverse{16} Y un macho cabrío por expiación: además del
holocausto continuo, su presente y su libación.

\bibverse{17} Y el segundo día, doce becerros de la vacada, dos
carneros, catorce corderos de un año sin defecto; \bibverse{18} Y sus
presentes y sus libaciones con los becerros, con los carneros, y con los
corderos, según el número de ellos, conforme á la ley; \bibverse{19} Y
un macho cabrío por expiación: además del holocausto continuo, y su
presente y su libación.

\bibverse{20} Y el día tercero, once becerros, dos carneros, catorce
corderos de un año sin defecto; \bibverse{21} Y sus presentes y sus
libaciones con los becerros, con los carneros, y con los corderos, según
el número de ellos, conforme á la ley; \bibverse{22} Y un macho cabrío
por expiación: además del holocausto continuo, y su presente y su
libación.

\bibverse{23} Y el cuarto día, diez becerros, dos carneros, catorce
corderos de un año sin defecto; \bibverse{24} Sus presentes y sus
libaciones con los becerros, con los carneros, y con los corderos, según
el número de ellos, conforme á la ley; \bibverse{25} Y un macho cabrío
por expiación: además del holocausto continuo, su presente y su
libación.

\bibverse{26} Y el quinto día, nueve becerros, dos carneros, catorce
corderos de un año sin defecto; \bibverse{27} Y sus presentes y sus
libaciones con los becerros, con los carneros, y con los corderos, según
el número de ellos, conforme á la ley; \bibverse{28} Y un macho cabrío
por expiación: además del holocausto continuo, su presente y su
libación.

\bibverse{29} Y el sexto día, ocho becerros, dos carneros, catorce
corderos de un año sin defecto; \bibverse{30} Y sus presentes y sus
libaciones con los becerros, con los carneros, y con los corderos, según
el número de ellos, conforme á la ley; \bibverse{31} Y un macho cabrío
por expiación: además del holocausto continuo, su presente y sus
libaciones.

\bibverse{32} Y el séptimo día, siete becerros, dos carneros, catorce
corderos de un año sin defecto; \bibverse{33} Y sus presentes y sus
libaciones con los becerros, con los carneros, y con los corderos, según
el número de ellos, conforme á la ley; \bibverse{34} Y un macho cabrío
por expiación: además del holocausto continuo, con su presente y su
libación.

\bibverse{35} El octavo día tendréis solemnidad: ninguna obra servil
haréis: \bibverse{36} Y ofreceréis en holocausto, en ofrenda encendida
de olor suave á Jehová, un novillo, un carnero, siete corderos de un año
sin defecto; \bibverse{37} Sus presentes y sus libaciones con el
novillo, con el carnero, y con los corderos, según el número de ellos,
conforme á la ley; \bibverse{38} Y un macho cabrío por expiación: además
del holocausto continuo, con su presente y su libación.

\hypertarget{sentencia-final-de-las-leyes-de-vuxedctimas}{%
\subsection{Sentencia final de las leyes de
víctimas}\label{sentencia-final-de-las-leyes-de-vuxedctimas}}

\bibverse{39} Estas cosas ofreceréis á Jehová en vuestras solemnidades,
además de vuestros votos, y de vuestras ofrendas libres, para vuestros
holocaustos, y para vuestros presentes, y para vuestras libaciones, y
para vuestras paces. \bibverse{40}

\hypertarget{section-29}{%
\section{30}\label{section-29}}

\bibverse{1} Y moisés dijo á los hijos de Israel, conforme á todo lo que
Jehová le había mandado.

\hypertarget{reglamento-sobre-la-vinculaciuxf3n-o-nulidad-de-los-votos}{%
\subsection{Reglamento sobre la vinculación o nulidad de los
votos}\label{reglamento-sobre-la-vinculaciuxf3n-o-nulidad-de-los-votos}}

\bibverse{2} Y habló Moisés á los príncipes de las tribus de los hijos
de Israel, diciendo: Esto es lo que Jehová ha mandado.

\bibverse{3} Cuando alguno hiciere voto á Jehová, ó hiciere juramento
ligando su alma con obligación, no violará su palabra: hará conforme á
todo lo que salió de su boca. \bibverse{4} Mas la mujer, cuando hiciere
voto á Jehová, y se ligare con obligación en casa de su padre, en su
mocedad; \bibverse{5} Si su padre oyere su voto, y la obligación con que
ligó su alma, y su padre callare á ello, todos los votos de ella serán
firmes, y toda obligación con que hubiere ligado su alma, firme será.

\bibverse{6} Mas si su padre le vedare el día que oyere todos sus votos
y sus obligaciones, con que ella hubiere ligado su alma, no serán
firmes; y Jehová la perdonará, por cuanto su padre le vedó. \bibverse{7}
Empero si fuere casada, é hiciere votos, ó pronunciare de sus labios
cosa con que obligue su alma; \bibverse{8} Si su marido lo oyere, y
cuando lo oyere callare á ello, los votos de ella serán firmes, y la
obligación con que ligó su alma, firme será.

\bibverse{9} Pero si cuando su marido lo oyó, le vedó, entonces el voto
que ella hizo, y lo que pronunció de sus labios con que ligó su alma,
será nulo; y Jehová lo perdonará.

\bibverse{10} Mas todo voto de viuda, ó repudiada, con que ligare su
alma, será firme. \bibverse{11} Y si hubiere hecho voto en casa de su
marido, y hubiere ligado su alma con obligación de juramento,
\bibverse{12} Si su marido oyó, y calló á ello, y no le vedó; entonces
todos sus votos serán firmes, y toda obligación con que hubiere ligado
su alma, firme será. \bibverse{13} Mas si su marido los anuló el día que
los oyó; todo lo que salió de sus labios cuanto á sus votos, y cuanto á
la obligación de su alma, será nulo; su marido los anuló, y Jehová la
perdonará.

\hypertarget{promulgaciuxf3n-renovada-de-los-derechos-del-marido}{%
\subsection{Promulgación renovada de los derechos del
marido}\label{promulgaciuxf3n-renovada-de-los-derechos-del-marido}}

\bibverse{14} Todo voto, ó todo juramento obligándose á afligir el alma,
su marido lo confirmará, ó su marido lo anulará. \bibverse{15} Empero si
su marido callare á ello de día en día, entonces confirmó todos sus
votos, y todas las obligaciones que están sobre ella: confirmólas, por
cuanto calló á ello el día que lo oyó.

\bibverse{16} Mas si las anulare después de haberlas oído, entonces él
llevará el pecado de ella. Estas son las ordenanzas que Jehová mandó á
Moisés entre el varón y su mujer, entre el padre y su hija, durante su
mocedad en casa de su padre.

\hypertarget{guerra-de-venganza-de-los-israelitas-contra-los-madianitas}{%
\subsection{Guerra de venganza de los israelitas contra los
madianitas}\label{guerra-de-venganza-de-los-israelitas-contra-los-madianitas}}

\hypertarget{section-30}{%
\section{31}\label{section-30}}

\bibverse{1} Y jehová habló á Moisés, diciendo: \bibverse{2} Haz la
venganza de los hijos de Israel sobre los Madianitas; después serás
recogido á tus pueblos. \footnote{\textbf{31:2} Núm 25,17; Núm 27,13}

\bibverse{3} Entonces Moisés habló al pueblo, diciendo: Armaos algunos
de vosotros para la guerra, é irán contra Madián, y harán la venganza de
Jehová en Madián. \bibverse{4} Mil de cada tribu de todas las tribus de
los hijos de Israel, enviaréis á la guerra. \bibverse{5} Así fueron
dados de los millares de Israel, mil por cada tribu, doce mil á punto de
guerra. \bibverse{6} Y Moisés los envió á la guerra: mil de cada tribu
envió: y Phinees, hijo de Eleazar sacerdote, fué á la guerra con los
santos instrumentos, con las trompetas en su mano para tocar.
\bibverse{7} Y pelearon contra Madián, como Jehová lo mandó á Moisés, y
mataron á todo varón. \footnote{\textbf{31:7} Éxod 20,13} \bibverse{8}
Mataron también, entre los muertos de ellos, á los reyes de Madián: Evi,
y Recem, y Zur, y Hur, y Reba, cinco reyes de Madián: á Balaam también,
hijo de Beor, mataron á cuchillo. \footnote{\textbf{31:8} Jos 13,21-22;
  Núm 22,5} \bibverse{9} Y llevaron cautivas los hijos de Israel las
mujeres de los Madianitas, y sus chiquitos y todas sus bestias, y todos
sus ganados; y arrebataron toda su hacienda. \bibverse{10} Y abrasaron
con fuego todas sus ciudades, aldeas y castillos. \bibverse{11} Y
tomaron todo el despojo, y toda la presa, así de hombres como de
bestias. \bibverse{12} Y trajeron á Moisés, y á Eleazar el sacerdote, y
á la congregación de los hijos de Israel, los cautivos y la presa y los
despojos, al campo en los llanos de Moab, que están junto al Jordán de
Jericó.

\hypertarget{ordenanza-sobre-la-matanza-de-todos-los-niuxf1os-varones-sobre-el-trato-de-las-reclusas-y-los-niuxf1os-y-sobre-la-limpieza-que-se-debe-realizar-antes-del-regreso}{%
\subsection{Ordenanza sobre la matanza de todos los niños varones, sobre
el trato de las reclusas y los niños y sobre la limpieza que se debe
realizar antes del
regreso}\label{ordenanza-sobre-la-matanza-de-todos-los-niuxf1os-varones-sobre-el-trato-de-las-reclusas-y-los-niuxf1os-y-sobre-la-limpieza-que-se-debe-realizar-antes-del-regreso}}

\bibverse{13} Y salieron Moisés y Eleazar el sacerdote, y todos los
príncipes de la congregación, á recibirlos fuera del campo.
\bibverse{14} Y enojóse Moisés contra los capitanes del ejército, contra
los tribunos y centuriones que volvían de la guerra; \bibverse{15} Y
díjoles Moisés: ¿Todas las mujeres habéis reservado? \bibverse{16} He
aquí ellas fueron á los hijos de Israel, por consejo de Balaam, para
causar prevaricación contra Jehová en el negocio de Peor; por lo que
hubo mortandad en la congregación de Jehová. \footnote{\textbf{31:16}
  Núm 25,1; Apoc 2,14} \bibverse{17} Matad pues ahora todos los varones
entre los niños: matad también toda mujer que haya conocido varón
carnalmente. \footnote{\textbf{31:17} Jue 21,11} \bibverse{18} Y todas
las niñas entre las mujeres, que no hayan conocido ayuntamiento de
varón, os reservaréis vivas.

\bibverse{19} Y vosotros quedaos fuera del campo siete días: y todos los
que hubieren matado persona, y cualquiera que hubiere tocado muerto, os
purificaréis al tercero y al séptimo día, vosotros y vuestros cautivos.
\footnote{\textbf{31:19} Núm 19,11} \bibverse{20} Asimismo purificaréis
todo vestido, y toda prenda de pieles, y toda obra de pelos de cabra, y
todo vaso de madera.

\bibverse{21} Y Eleazar el sacerdote dijo á los hombres de guerra que
venían de la guerra: Esta es la ordenanza de la ley que Jehová ha
mandado á Moisés: \bibverse{22} Ciertamente el oro, y la plata, metal,
hierro, estaño, y plomo, \bibverse{23} Todo lo que resiste el fuego, por
fuego lo haréis pasar, y será limpio, bien que en las aguas de
purificación habrá de purificarse: mas haréis pasar por agua todo lo que
no aguanta el fuego. \bibverse{24} Además lavaréis vuestros vestidos el
séptimo día, y así seréis limpios; y después entraréis en el campo.

\hypertarget{distribuciuxf3n-de-presas-vivas-humanos-y-ganado-regalo-de-navidad-de-los-luxedderes}{%
\subsection{Distribución de presas vivas (humanos y ganado); Regalo de
Navidad de los
líderes}\label{distribuciuxf3n-de-presas-vivas-humanos-y-ganado-regalo-de-navidad-de-los-luxedderes}}

\bibverse{25} Y Jehová habló á Moisés, diciendo: \bibverse{26} Toma la
cuenta de la presa que se ha hecho, así de las personas como de las
bestias, tú y el sacerdote Eleazar, y las cabezas de los padres de la
congregación: \bibverse{27} Y partirás por mitad la presa entre los que
pelearon, los que salieron á la guerra, y toda la congregación.
\bibverse{28} Y apartarás para Jehová el tributo de los hombres de
guerra, que salieron á la guerra: de quinientos uno, así de las personas
como de los bueyes, de los asnos, y de las ovejas: \bibverse{29} De la
mitad de ellos lo tomarás; y darás á Eleazar el sacerdote la ofrenda de
Jehová. \bibverse{30} Y de la mitad perteneciente á los hijos de Israel
tomarás uno de cincuenta, de las personas, de los bueyes, de los asnos,
y de las ovejas, de todo animal; y los darás á los Levitas, que tienen
la guarda del tabernáculo de Jehová.

\bibverse{31} E hicieron Moisés y Eleazar el sacerdote como Jehová mandó
á Moisés.

\bibverse{32} Y fué la presa, el resto de la presa que tomaron los
hombres de guerra, seiscientas y setenta y cinco mil ovejas,
\bibverse{33} Y setenta y dos mil bueyes, \bibverse{34} Y setenta y un
mil asnos; \bibverse{35} Y en cuanto á personas, de mujeres que no
habían conocido ayuntamiento de varón, en todas treinta y dos mil.
\bibverse{36} Y la mitad, la parte de los que habían salido á la guerra,
fué el número de trescientas treinta y siete mil y quinientas ovejas.
\bibverse{37} Y el tributo para Jehová de las ovejas, fué seiscientas
setenta y cinco. \bibverse{38} Y de los bueyes, treinta y seis mil: y de
ellos el tributo para Jehová, setenta y dos. \bibverse{39} Y de los
asnos, treinta mil y quinientos: y de ellos el tributo para Jehová,
setenta y uno. \bibverse{40} Y de las personas, diez y seis mil: y de
ellas el tributo para Jehová, treinta y dos personas. \bibverse{41} Y
dió Moisés el tributo, por elevada ofrenda á Jehová, á Eleazar el
sacerdote, como Jehová lo mandó á Moisés. \bibverse{42} Y de la mitad
para los hijos de Israel, que apartó Moisés de los hombres que habían
ido á la guerra; \bibverse{43} (La mitad para la congregación fué: de
las ovejas, trescientas treinta y siete mil y quinientas; \bibverse{44}
Y de los bueyes, treinta y seis mil; \bibverse{45} Y de los asnos,
treinta mil y quinientos; \bibverse{46} Y de las personas, diez y seis
mil:) \bibverse{47} De la mitad, pues, para los hijos de Israel tomó
Moisés uno de cada cincuenta, así de las personas como de los animales,
y diólos á los Levitas, que tenían la guarda del tabernáculo de Jehová;
como Jehová lo había mandado á Moisés.

\bibverse{48} Y llegaron á Moisés los jefes de los millares de aquel
ejército, los tribunos y centuriones; \bibverse{49} Y dijeron á Moisés:
Tus siervos han tomado razón de los hombres de guerra que están en
nuestro poder, y ninguno ha faltado de nosotros. \bibverse{50} Por lo
cual hemos ofrecido á Jehová ofrenda, cada uno de lo que ha hallado,
vasos de oro, brazaletes, manillas, anillos, zarcillos, y cadenas, para
hacer expiación por nuestras almas delante de Jehová.

\bibverse{51} Y Moisés y el sacerdote Eleazar recibieron el oro de
ellos, alhajas, todas elaboradas. \bibverse{52} Y todo el oro de la
ofrenda que ofrecieron á Jehová de los tribunos y centuriones, fué diez
y seis mil setecientos y cincuenta siclos. \bibverse{53} Los hombres del
ejército habían pillado cada uno para sí. \bibverse{54} Recibieron,
pues, Moisés y el sacerdote Eleazar, el oro de los tribunos y
centuriones, y trajéronlo al tabernáculo del testimonio, por memoria de
los hijos de Israel delante de Jehová.

\hypertarget{la-peticiuxf3n-de-los-rubenitas-y-gaditas-fue-rechazada-por-moisuxe9s-en-un-discurso-punitivo}{%
\subsection{La petición de los rubenitas y gaditas fue rechazada por
Moisés en un discurso
punitivo}\label{la-peticiuxf3n-de-los-rubenitas-y-gaditas-fue-rechazada-por-moisuxe9s-en-un-discurso-punitivo}}

\hypertarget{section-31}{%
\section{32}\label{section-31}}

\bibverse{1} Y los hijos de Rubén y los hijos de Gad tenían una muy
grande muchedumbre de ganado; los cuales viendo la tierra de Jazer y de
Galaad, parecióles el país lugar de ganado. \bibverse{2} Y vinieron los
hijos de Gad y los hijos de Rubén, y hablaron á Moisés, y á Eleazar el
sacerdote, y á los príncipes de la congregación, diciendo: \bibverse{3}
Ataroth, y Dibón, y Jazer, y Nimra, y Hesbón, y Eleale, y Sabán, y Nebo,
y Beón, \bibverse{4} La tierra que Jehová hirió delante de la
congregación de Israel, es tierra de ganado, y tus siervos tienen
ganado. \bibverse{5} Por tanto, dijeron, si hallamos gracia en tus ojos,
dése esta tierra á tus siervos en heredad, y no nos hagas pasar el
Jordán.

\bibverse{6} Y respondió Moisés á los hijos de Gad y á los hijos de
Rubén: ¿Vendrán vuestros hermanos á la guerra, y vosotros os quedaréis
aquí? \bibverse{7} ¿Y por qué prevenís el ánimo de los hijos de Israel,
para que no pasen á la tierra que les ha dado Jehová? \bibverse{8} Así
hicieron vuestros padres, cuando los envié desde Cades-barnea para que
viesen la tierra. \footnote{\textbf{32:8} Núm 13,-1} \bibverse{9} Que
subieron hasta la arroyada de Escol, y después que vieron la tierra,
preocuparon el ánimo de los hijos de Israel, para que no viniesen á la
tierra que Jehová les había dado. \bibverse{10} Y el furor de Jehová se
encendió entonces, y juró diciendo: \bibverse{11} Que no verán los
varones que subieron de Egipto de veinte años arriba, la tierra por la
cual juré á Abraham, Isaac, y Jacob, por cuanto no fueron perfectos en
pos de mí; \bibverse{12} Excepto Caleb, hijo de Jephone Cenezeo, y Josué
hijo de Nun, que fueron perfectos en pos de Jehová. \bibverse{13} Y el
furor de Jehová se encendió en Israel, é hízolos andar errantes cuarenta
años por el desierto, hasta que fué acabada toda aquella generación, que
había hecho mal delante de Jehová.

\bibverse{14} Y he aquí vosotros habéis sucedido en lugar de vuestros
padres, prole de hombres pecadores, para añadir aún á la ira de Jehová
contra Israel. \bibverse{15} Si os volviereis de en pos de él, él
volverá otra vez á dejaros en el desierto, y destruiréis á todo este
pueblo.

\hypertarget{la-respuesta-de-los-rubenitas-y-gaditas}{%
\subsection{La respuesta de los rubenitas y
gaditas}\label{la-respuesta-de-los-rubenitas-y-gaditas}}

\bibverse{16} Entonces ellos se llegaron á él y dijeron: Edificaremos
aquí majadas para nuestro ganado, y ciudades para nuestros niños;
\bibverse{17} Y nosotros nos armaremos, é iremos con diligencia delante
de los hijos de Israel, hasta que los metamos en su lugar: y nuestros
niños quedarán en ciudades fuertes á causa de los moradores del país.
\bibverse{18} No volveremos á nuestras casas hasta que los hijos de
Israel posean cada uno su heredad. \bibverse{19} Porque no tomaremos
heredad con ellos al otro lado del Jordán ni adelante, por cuanto
tendremos ya nuestra heredad de estotra parte del Jordán al oriente.

\hypertarget{la-promesa-de-moisuxe9s-declarando-las-condiciones-otorgando-el-pauxeds-al-este-del-jorduxe1n-a-las-tribus-suplicantes}{%
\subsection{La promesa de Moisés, declarando las condiciones; Otorgando
el País al este del Jordán a las tribus
suplicantes}\label{la-promesa-de-moisuxe9s-declarando-las-condiciones-otorgando-el-pauxeds-al-este-del-jorduxe1n-a-las-tribus-suplicantes}}

\bibverse{20} Entonces les respondió Moisés: Si lo hiciereis así, si os
apercibiereis para ir delante de Jehová á la guerra, \footnote{\textbf{32:20}
  Jos 1,13-15} \bibverse{21} Y pasareis todos vosotros armados el Jordán
delante de Jehová, hasta que haya echado á sus enemigos de delante de
sí, \bibverse{22} Y sea el país sojuzgado delante de Jehová; luego
volveréis, y seréis libres de culpa para con Jehová, y para con Israel;
y esta tierra será vuestra en heredad delante de Jehová.

\bibverse{23} Mas si así no lo hiciereis, he aquí habréis pecado á
Jehová; y sabed que os alcanzará vuestro pecado. \bibverse{24} Edificaos
ciudades para vuestros niños, y majadas para vuestras ovejas, y haced lo
que ha salido de vuestra boca.

\bibverse{25} Y hablaron los hijos de Gad y los hijos de Rubén á Moisés,
diciendo: Tus siervos harán como mi señor ha mandado. \bibverse{26}
Nuestros niños, nuestras mujeres, nuestros ganados, y todas nuestras
bestias, estarán ahí en las ciudades de Galaad; \bibverse{27} Y tus
siervos, armados todos de guerra, pasarán delante de Jehová á la guerra,
de la manera que mi señor dice.

\bibverse{28} Entonces los encomendó Moisés á Eleazar el sacerdote, y á
Josué hijo de Nun, y á los príncipes de los padres de las tribus de los
hijos de Israel. \bibverse{29} Y díjoles Moisés: Si los hijos de Gad y
los hijos de Rubén, pasaren con vosotros el Jordán, armados todos de
guerra delante de Jehová, luego que el país fuere sojuzgado delante de
vosotros, les daréis la tierra de Galaad en posesión: \bibverse{30} Mas
si no pasaren armados con vosotros, entonces tendrán posesión entre
vosotros en la tierra de Canaán.

\bibverse{31} Y los hijos de Gad y los hijos de Rubén respondieron,
diciendo: Haremos lo que Jehová ha dicho á tus siervos. \bibverse{32}
Nosotros pasaremos armados delante de Jehová á la tierra de Canaán, y la
posesión de nuestra heredad será de esta parte del Jordán.

\bibverse{33} Así les dió Moisés á los hijos de Gad y á los hijos de
Rubén, y á la media tribu de Manasés hijo de José, el reino de Sehón rey
Amorrheo, y el reino de Og rey de Basán, la tierra con sus ciudades y
términos, las ciudades del país alrededor. \footnote{\textbf{32:33} Jos
  13,8-31}

\hypertarget{resumen-de-las-ciudades-reconstruidas-por-los-gaditas-y-los-rubenitas}{%
\subsection{Resumen de las ciudades reconstruidas por los gaditas y los
rubenitas}\label{resumen-de-las-ciudades-reconstruidas-por-los-gaditas-y-los-rubenitas}}

\bibverse{34} Y los hijos de Gad edificaron á Dibón, y á Ataroth, y á
Aroer, \bibverse{35} Y á Atroth-sophan, y á Jazer, y á Jogbaa,
\bibverse{36} Y á Beth-nimra, y á Betharán: ciudades fuertes, y también
majadas para ovejas. \bibverse{37} Y los hijos de Rubén edificaron á
Hesbón, y á Eleale, y á Kiriathaim, \bibverse{38} Y á Nebo, y á
Baal-meón, (mudados los nombres), y á Sibma: y pusieron nombres á las
ciudades que edificaron.

\hypertarget{los-descendientes-de-manasuxe9s-se-establecieron-en-la-ribera-oriental}{%
\subsection{Los descendientes de Manasés se establecieron en la Ribera
Oriental}\label{los-descendientes-de-manasuxe9s-se-establecieron-en-la-ribera-oriental}}

\bibverse{39} Y los hijos de Machîr hijo de Manasés fueron á Galaad, y
tomáronla, y echaron al Amorrheo que estaba en ella. \bibverse{40} Y
Moisés dió Galaad á Machîr hijo de Manasés, el cual habitó en ella.
\bibverse{41} También Jair hijo de Manasés fué y tomó sus aldeas, y
púsoles por nombre Havoth-jair. \bibverse{42} Asimismo Noba fué y tomó á
Kenath y sus aldeas, y llamóle Noba, conforme á su nombre.

\hypertarget{lista-de-los-campamentos-en-los-que-pasaron-los-israelitas-durante-los-cuarenta-auxf1os-del-desierto}{%
\subsection{Lista de los campamentos en los que pasaron los israelitas
durante los cuarenta años del
desierto}\label{lista-de-los-campamentos-en-los-que-pasaron-los-israelitas-durante-los-cuarenta-auxf1os-del-desierto}}

\hypertarget{section-32}{%
\section{33}\label{section-32}}

\bibverse{1} Estas son las estancias de los hijos de Israel, los cuales
salieron de la tierra de Egipto por sus escuadrones, bajo la conducta de
Moisés y Aarón. \bibverse{2} Y Moisés escribió sus salidas conforme á
sus jornadas por mandato de Jehová. Estas, pues, son sus estancias con
arreglo á sus partidas. \bibverse{3} De Rameses partieron en el mes
primero, á los quince días del mes primero: el segundo día de la pascua
salieron los hijos de Israel con mano alta, á ojos de todos los
Egipcios. \footnote{\textbf{33:3} Éxod 1,11; Éxod 14,8} \bibverse{4}
Estaban enterrando los Egipcios los que Jehová había muerto de ellos, á
todo primogénito; habiendo Jehová hecho también juicios en sus dioses.
\footnote{\textbf{33:4} Éxod 12,12} \bibverse{5} Partieron, pues, los
hijos de Israel de Rameses, y asentaron campo en Succoth. \footnote{\textbf{33:5}
  Éxod 12,37} \bibverse{6} Y partiendo de Succoth, asentaron en Etham,
que está al cabo del desierto. \footnote{\textbf{33:6} Éxod 13,20}
\bibverse{7} Y partiendo de Etham, volvieron sobre Pi-hahiroth, que está
delante de Baalsephon, y asentaron delante de Migdol. \footnote{\textbf{33:7}
  Éxod 14,2} \bibverse{8} Y partiendo de Pi-hahiroth, pasaron por medio
de la mar al desierto, y anduvieron camino de tres días por el desierto
de Etham, y asentaron en Mara. \footnote{\textbf{33:8} Éxod 14,22; Éxod
  15,23} \bibverse{9} Y partiendo de Mara, vinieron á Elim, donde había
doce fuentes de aguas, y setenta palmeras; y asentaron allí. \footnote{\textbf{33:9}
  Éxod 15,27} \bibverse{10} Y partidos de Elim, asentaron junto al mar
Bermejo. \bibverse{11} Y partidos del mar Bermejo, asentaron en el
desierto de Sin. \bibverse{12} Y partidos del desierto de Sin, asentaron
en Dophca. \bibverse{13} Y partidos de Dophca, asentaron en Alús.
\bibverse{14} Y partidos de Alús, asentaron en Rephidim, donde el pueblo
no tuvo aguas para beber. \footnote{\textbf{33:14} Éxod 17,1}
\bibverse{15} Y partidos de Rephidim, asentaron en el desierto de Sinaí.
\footnote{\textbf{33:15} Éxod 19,1} \bibverse{16} Y partidos del
desierto de Sinaí, asentaron en Kibroth-hataava. \footnote{\textbf{33:16}
  Núm 11,34} \bibverse{17} Y partidos de Kibroth-hataava, asentaron en
Haseroth. \footnote{\textbf{33:17} Núm 11,35} \bibverse{18} Y partidos
de Haseroth, asentaron en Ritma. \footnote{\textbf{33:18} Núm 12,16}
\bibverse{19} Y partidos de Ritma, asentaron en Rimmón-peres.
\bibverse{20} Y partidos de Rimmón-peres, asentaron en Libna.
\bibverse{21} Y partidos de Libna, asentaron en Rissa. \bibverse{22} Y
partidos de Rissa, asentaron en Ceelatha. \bibverse{23} Y partidos de
Ceelatha, asentaron en el monte de Sepher. \bibverse{24} Y partidos del
monte de Sepher, asentaron en Harada. \bibverse{25} Y partidos de
Harada, asentaron en Maceloth. \bibverse{26} Y partidos de Maceloth,
asentaron en Tahath. \bibverse{27} Y partidos de Tahath, asentaron en
Tara. \bibverse{28} Y partidos de Tara, asentaron en Mithca.
\bibverse{29} Y partidos de Mithca, asentaron en Hasmona. \bibverse{30}
Y partidos de Hasmona, asentaron en Moseroth. \bibverse{31} Y partidos
de Moseroth, asentaron en Bene-jaacán. \bibverse{32} Y partidos de
Bene-jaacán, asentaron en el monte de Gidgad. \bibverse{33} Y partidos
del monte de Gidgad, asentaron en Jotbatha. \footnote{\textbf{33:33}
  Deut 10,7} \bibverse{34} Y partidos de Jotbatha, asentaron en Abrona.
\bibverse{35} Y partidos de Abrona, asentaron en Esion-geber.
\bibverse{36} Y partidos de Esion-geber, asentaron en el desierto de
Zin, que es Cades. \bibverse{37} Y partidos de Cades, asentaron en el
monte de Hor, en la extremidad del país de Edom. \footnote{\textbf{33:37}
  Núm 20,22-29} \bibverse{38} Y subió Aarón el sacerdote al monte de
Hor, conforme al dicho de Jehová, y allí murió á los cuarenta años de la
salida de los hijos de Israel de la tierra de Egipto, en el mes quinto,
en el primero del mes. \bibverse{39} Y era Aarón de edad de ciento y
veinte y tres años, cuando murió en el monte de Hor. \bibverse{40} Y el
Cananeo, rey de Arad, que habitaba al mediodía en la tierra de Canaán,
oyó como habían venido los hijos de Israel. \bibverse{41} Y partidos del
monte de Hor, asentaron en Salmona. \bibverse{42} Y partidos de Salmona,
asentaron en Phunón. \bibverse{43} Y partidos de Phunón, asentaron en
Oboth. \footnote{\textbf{33:43} Núm 21,10} \bibverse{44} Y partidos de
Oboth, asentaron en Ije-abarim; en el término de Moab. \footnote{\textbf{33:44}
  Núm 21,11} \bibverse{45} Y partidos de Ije-abarim, asentaron en
Dibon-gad. \bibverse{46} Y partidos de Dibon-gad, asentaron en
Almon-diblathaim. \bibverse{47} Y partidos de Almon-diblathaim,
asentaron en los montes de Abarim, delante de Nebo. \footnote{\textbf{33:47}
  Núm 21,20} \bibverse{48} Y partidos de los montes de Abarim, asentaron
en los campos de Moab, junto al Jordán de Jericó. \footnote{\textbf{33:48}
  Núm 22,1; Deut 32,49} \bibverse{49} Finalmente asentaron junto al
Jordán, desde Beth-jesimoth hasta Abel-sitim, en los campos de Moab.
\footnote{\textbf{33:49} Núm 25,1}

\hypertarget{ordenanzas-provisionales-de-dios-con-respecto-a-la-conquista-y-distribuciuxf3n-de-cisjordania-de-canauxe1n}{%
\subsection{Ordenanzas provisionales de Dios con respecto a la conquista
y distribución de Cisjordania de
Canaán}\label{ordenanzas-provisionales-de-dios-con-respecto-a-la-conquista-y-distribuciuxf3n-de-cisjordania-de-canauxe1n}}

\bibverse{50} Y habló Jehová á Moisés en los campos de Moab junto al
Jordán de Jericó, diciendo: \bibverse{51} Habla á los hijos de Israel, y
diles: Cuando hubiereis pasado el Jordán á la tierra de Canaán,
\bibverse{52} Echaréis á todos los moradores del país de delante de
vosotros, y destruiréis todas sus pinturas, y todas sus imágenes de
fundición, y arruinaréis todos sus altos; \bibverse{53} Y echaréis los
moradores de la tierra, y habitaréis en ella; porque yo os la he dado
para que la poseáis. \bibverse{54} Y heredaréis la tierra por suertes
por vuestras familias: á los muchos daréis mucho por su heredad, y á los
pocos daréis menos por heredad suya: donde le saliere la suerte, allí la
tendrá cada uno: por las tribus de vuestros padres heredaréis.

\bibverse{55} Y si no echareis los moradores del país de delante de
vosotros, sucederá que los que dejareis de ellos serán por aguijones en
vuestros ojos, y por espinas en vuestros costados, y afligiros han sobre
la tierra en que vosotros habitareis. \footnote{\textbf{33:55} Jos 23,13}

\bibverse{56} Será además, que haré á vosotros como yo pensé hacerles á
ellos.

\hypertarget{establecer-los-luxedmites-de-la-tierra-de-canauxe1n-que-se-tomaruxe1n}{%
\subsection{Establecer los límites de la tierra de Canaán que se
tomarán}\label{establecer-los-luxedmites-de-la-tierra-de-canauxe1n-que-se-tomaruxe1n}}

\hypertarget{section-33}{%
\section{34}\label{section-33}}

\bibverse{1} Y jehová habló á Moisés, diciendo: \bibverse{2} Manda á los
hijos de Israel, y diles: Cuando hubiereis entrado en la tierra de
Canaán, es á saber, la tierra que os ha de caer en heredad, la tierra de
Canaán según sus términos; \bibverse{3} Tendréis el lado del mediodía
desde el desierto de Zin hasta los términos de Edom; y os será el
término del mediodía al extremo del mar salado hacia el oriente:
\footnote{\textbf{34:3} Jos 15,1} \bibverse{4} Y este término os irá
rodeando desde el mediodía hasta la subida de Acrabbim, y pasará hasta
Zin; y sus salidas serán del mediodía á Cades-barnea; y saldrá á
Hasar-addar, y pasará hasta Asmón; \bibverse{5} Y rodeará este término,
desde Asmón hasta el torrente de Egipto, y sus remates serán al
occidente.

\bibverse{6} Y el término occidental os será la gran mar: este término
os será el término occidental.

\bibverse{7} Y el término del norte será este: desde la gran mar os
señalaréis el monte de Hor; \bibverse{8} Del monte de Hor señalaréis á
la entrada de Hamath, y serán las salidas de aquel término á Sedad;
\bibverse{9} Y saldrá este término á Ziphón, y serán sus remates en
Hasar-enán: este os será el término del norte.

\bibverse{10} Y por término al oriente os señalaréis desde Hasar-enán
hasta Sepham; \bibverse{11} Y bajará este término desde Sepham á Ribla,
al oriente de Ain: y descenderá el término, y llegará á la costa de la
mar de Cinnereth al oriente; \bibverse{12} Después descenderá este
término al Jordán, y serán sus salidas al mar Salado: esta será vuestra
tierra: por sus términos alrededor.

\bibverse{13} Y mandó Moisés á los hijos de Israel, diciendo: Esta es la
tierra que heredaréis por suerte, la cual mandó Jehová que diese á las
nueve tribus, y á la media tribu: \bibverse{14} Porque la tribu de los
hijos de Rubén según las casas de sus padres, y la tribu de los hijos de
Gad según las casas de sus padres, y la media tribu de Manasés, han
tomado su herencia: \footnote{\textbf{34:14} Núm 32,33} \bibverse{15}
Dos tribus y media tomaron su heredad de esta parte del Jordán de Jericó
al oriente, al nacimiento del sol.

\hypertarget{lista-de-hombres-que-se-encargaruxe1n-de-la-distribuciuxf3n-de-la-tierra}{%
\subsection{Lista de hombres que se encargarán de la distribución de la
tierra}\label{lista-de-hombres-que-se-encargaruxe1n-de-la-distribuciuxf3n-de-la-tierra}}

\bibverse{16} Y habló Jehová á Moisés, diciendo: \bibverse{17} Estos son
los nombres de los varones que os aposesionarán la tierra: Eleazar el
sacerdote, y Josué hijo de Nun. \bibverse{18} Tomaréis también de cada
tribu un príncipe, para dar la posesión de la tierra. \bibverse{19} Y
estos son los nombres de los varones: De la tribu de Judá, Caleb hijo de
Jephone. \footnote{\textbf{34:19} Núm 13,6; Núm 13,30}

\bibverse{20} Y de la tribu de los hijos de Simeón, Samuel hijo de
Ammiud. \bibverse{21} De la tribu de Benjamín, Elidad hijo de Chislón.
\bibverse{22} Y de la tribu de los hijos de Dan, el príncipe Bucci hijo
de Jogli. \bibverse{23} De los hijos de José: de la tribu de los hijos
de Manasés, el príncipe Haniel hijo de Ephod. \bibverse{24} Y de la
tribu de los hijos de Ephraim, el príncipe Chêmuel hijo de Siphtán.
\bibverse{25} Y de la tribu de los hijos de Zabulón, el príncipe
Elisaphán hijo de Pharnach. \bibverse{26} Y de la tribu de los hijos de
Issachâr, el príncipe Paltiel hijo de Azan. \bibverse{27} Y de la tribu
de los hijos de Aser, el príncipe Ahiud hijo de Selomi. \bibverse{28} Y
de la tribu de los hijos de Nephtalí, el príncipe Pedael hijo de Ammiud.
\bibverse{29} Estos son á los que mandó Jehová que hiciesen la partición
de la herencia á los hijos de Israel en la tierra de Canaán.

\hypertarget{regulaciones-relativas-a-las-ciudades-levitas-y-las-seis-ciudades-libres-designadas-para-asesinos}{%
\subsection{Regulaciones relativas a las ciudades levitas y las seis
ciudades libres designadas para
asesinos}\label{regulaciones-relativas-a-las-ciudades-levitas-y-las-seis-ciudades-libres-designadas-para-asesinos}}

\hypertarget{section-34}{%
\section{35}\label{section-34}}

\bibverse{1} Y habló Jehová á Moisés en los campos de Moab, junto al
Jordán de Jericó, diciendo: \bibverse{2} Manda á los hijos de Israel,
que den á los Levitas de la posesión de su heredad ciudades en que
habiten: también daréis á los Levitas los ejidos de esas ciudades
alrededor de ellas. \bibverse{3} Y tendrán ellos las ciudades para
habitar, y los ejidos de ellas serán para sus animales, y para sus
ganados, y para todas sus bestias.

\bibverse{4} Y los ejidos de las ciudades que daréis á los Levitas,
serán mil codos alrededor, desde el muro de la ciudad para afuera.
\bibverse{5} Luego mediréis fuera de la ciudad á la parte del oriente
dos mil codos, y á la parte del mediodía dos mil codos, y á la parte del
occidente dos mil codos, y á la parte del norte dos mil codos, y la
ciudad en medio: esto tendrán por los ejidos de las ciudades.

\bibverse{6} Y de las ciudades que daréis á los Levitas, seis ciudades
serán de acogimiento, las cuales daréis para que el homicida se acoja
allá: y además de éstas daréis cuarenta y dos ciudades. \footnote{\textbf{35:6}
  Éxod 21,13; Deut 4,41; Deut 19,2; Deut 19,9; Jos 20,-1} \bibverse{7}
Todas las ciudades que daréis á los Levitas serán cuarenta y ocho
ciudades; ellas con sus ejidos. \bibverse{8} Y las ciudades que diereis
de la heredad de los hijos de Israel, del que mucho tomaréis mucho, y
del que poco tomaréis poco: cada uno dará de sus ciudades á los Levitas
según la posesión que heredará. \bibverse{9} Y habló Jehová á Moisés,
diciendo: \bibverse{10} Habla á los hijos de Israel, y diles: Cuando
hubiereis pasado el Jordán á la tierra de Canaán, \bibverse{11} Os
señalaréis ciudades, ciudades de acogimiento tendréis, donde huya el
homicida que hiriere á alguno de muerte por yerro. \bibverse{12} Y os
serán aquellas ciudades por acogimiento del pariente, y no morirá el
homicida hasta que esté á juicio delante de la congregación.
\bibverse{13} De las ciudades, pues, que daréis, tendréis seis ciudades
de acogimiento. \bibverse{14} Tres ciudades daréis de esta parte del
Jordán, y tres ciudades daréis en la tierra de Canaán; las cuales serán
ciudades de acogimiento. \bibverse{15} Estas seis ciudades serán para
acogimiento á los hijos de Israel, y al peregrino, y al que morare entre
ellos, para que huya allá cualquiera que hiriere de muerte á otro por
yerro.

\hypertarget{el-castigo-del-asesino}{%
\subsection{El castigo del asesino}\label{el-castigo-del-asesino}}

\bibverse{16} Y si con instrumento de hierro lo hiriere y muriere,
homicida es; el homicida morirá: \bibverse{17} Y si con piedra de mano,
de que pueda morir, lo hiriere, y muriere, homicida es; el homicida
morirá. \bibverse{18} Y si con instrumento de palo de mano, de que pueda
morir, lo hiriere, y muriere, homicida es; el homicida morirá.
\bibverse{19} El pariente del muerto, él matará al homicida: cuando lo
encontrare, él le matará. \bibverse{20} Y si por odio lo empujó, ó echó
sobre él alguna cosa por asechanzas, y muere; \bibverse{21} O por
enemistad lo hirió con su mano, y murió: el heridor morirá; es homicida;
el pariente del muerto matará al homicida, cuando lo encontrare.

\bibverse{22} Mas si casualmente lo empujó sin enemistades, ó echó sobre
él cualquier instrumento sin asechanzas, \bibverse{23} O bien, sin
verlo, hizo caer sobre él alguna piedra, de que pudo morir, y muriere, y
él no era su enemigo, ni procuraba su mal; \bibverse{24} Entonces la
congregación juzgará entre el heridor y el pariente del muerto conforme
á estas leyes: \bibverse{25} Y la congregación librará al homicida de
mano del pariente del muerto, y la congregación lo hará volver á su
ciudad de acogimiento, á la cual se había acogido; y morará en ella
hasta que muera el gran sacerdote, el cual fué ungido con el aceite
santo. \footnote{\textbf{35:25} Lev 21,10}

\bibverse{26} Y si el homicida saliere fuera del término de su ciudad de
refugio, á la cual se acogió, \bibverse{27} Y el pariente del muerto le
hallare fuera del término de la ciudad de su acogida, y el pariente del
muerto al homicida matare, no se le culpará por ello: \bibverse{28} Pues
en su ciudad de refugio deberá aquél habitar hasta que muera el gran
sacerdote: y después que muriere el gran sacerdote, el homicida volverá
á la tierra de su posesión.

\bibverse{29} Y estas cosas os serán por ordenanza de derecho por
vuestras edades, en todas vuestras habitaciones.

\bibverse{30} Cualquiera que hiriere á alguno, por dicho de testigos,
morirá el homicida: mas un solo testigo no hará fe contra alguna persona
para que muera.

\bibverse{31} Y no tomaréis precio por la vida del homicida; porque está
condenado á muerte: mas indefectiblemente morirá.

\bibverse{32} Ni tampoco tomaréis precio del que huyó á su ciudad de
refugio, para que vuelva á vivir en su tierra, hasta que muera el
sacerdote.

\bibverse{33} Y no contaminaréis la tierra donde estuviereis: porque
esta sangre amancillará la tierra: y la tierra no será expiada de la
sangre que fué derramada en ella, sino por la sangre del que la derramó.
\footnote{\textbf{35:33} Gén 9,6} \bibverse{34} No contaminéis, pues, la
tierra donde habitáis, en medio de la cual yo habito; porque yo Jehová
habito en medio de los hijos de Israel. \footnote{\textbf{35:34} Éxod
  29,45}

\hypertarget{apuxe9ndice-a-la-ley-de-reliquias}{%
\subsection{Apéndice a la ley de
reliquias}\label{apuxe9ndice-a-la-ley-de-reliquias}}

\hypertarget{section-35}{%
\section{36}\label{section-35}}

\bibverse{1} Y llegaron los príncipes de los padres de la familia de
Galaad, hijo de Machîr, hijo de Manasés, de las familias de los hijos de
José; y hablaron delante de Moisés, y de los príncipes, cabezas de
padres de los hijos de Israel, \bibverse{2} Y dijeron: Jehová mandó á mi
señor que por suerte diese la tierra á los hijos de Israel en posesión:
también ha mandado Jehová á mi señor, que dé la posesión de Salphaad
nuestro hermano á sus hijas; \footnote{\textbf{36:2} Núm 26,55; Núm
  27,6-7} \bibverse{3} Las cuales, si se casaren con algunos de los
hijos de las otras tribus de los hijos de Israel, la herencia de ellas
será así desfalcada de la herencia de nuestros padres, y será añadida á
la herencia de la tribu á que serán unidas: y será quitada de la suerte
de nuestra heredad. \bibverse{4} Y cuando viniere el jubileo de los
hijos de Israel, la heredad de ellas será añadida á la heredad de la
tribu de sus maridos; y así la heredad de ellas será quitada de la
heredad de la tribu de nuestros padres.

\hypertarget{la-nueva-regulaciuxf3n-de-aplicaciuxf3n-general-sobre-el-matrimonio-de-reliquias}{%
\subsection{La nueva regulación de aplicación general sobre el
matrimonio de
reliquias}\label{la-nueva-regulaciuxf3n-de-aplicaciuxf3n-general-sobre-el-matrimonio-de-reliquias}}

\bibverse{5} Entonces Moisés mandó á los hijos de Israel por dicho de
Jehová, diciendo: La tribu de los hijos de José habla rectamente.
\bibverse{6} Esto es lo que ha mandado Jehová acerca de las hijas de
Salphaad, diciendo: Cásense como á ellas les pluguiere, empero en la
familia de la tribu de su padre se casarán; \bibverse{7} Para que la
heredad de los hijos de Israel no sea traspasada de tribu en tribu;
porque cada uno de los hijos de Israel se allegará á la heredad de la
tribu de sus padres. \bibverse{8} Y cualquiera hija que poseyere heredad
de las tribus de los hijos de Israel, con alguno de la familia de la
tribu de su padre se casará, para que los hijos de Israel posean cada
uno la heredad de sus padres, \bibverse{9} Y no ande la heredad rodando
de una tribu á otra: mas cada una de las tribus de los hijos de Israel
se llegue á su heredad.

\bibverse{10} Como Jehová mandó á Moisés, así hicieron las hijas de
Salphaad. \bibverse{11} Y así Maala, y Tirsa, y Hogla, y Milchâ, y Noa,
hijas de Salphaad, se casaron con hijos de sus tíos: \footnote{\textbf{36:11}
  Núm 26,33} \bibverse{12} De la familia de los hijos de Manasés, hijo
de José, fueron mujeres; y la heredad de ellas quedó en la tribu de la
familia de su padre.

\bibverse{13} Estos son los mandamientos y los estatutos que mandó
Jehová por mano de Moisés á los hijos de Israel en los campos de Moab,
junto al Jordán de Jericó.
