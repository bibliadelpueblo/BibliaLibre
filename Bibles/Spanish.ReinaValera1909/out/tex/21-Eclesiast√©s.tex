\hypertarget{la-inutilidad-de-todo-esfuerzo-humano-como-resultado-de-la-constante-monotonuxeda-en-el-ciclo-de-las-cosas.}{%
\subsection{La inutilidad de todo esfuerzo humano como resultado de la
constante monotonía en el ciclo de las
cosas.}\label{la-inutilidad-de-todo-esfuerzo-humano-como-resultado-de-la-constante-monotonuxeda-en-el-ciclo-de-las-cosas.}}

\hypertarget{section}{%
\section{1}\label{section}}

\bibverse{1} Palabras del Predicador, hijo de David, rey en Jerusalem.

\bibverse{2} Vanidad de vanidades, dijo el Predicador; vanidad de
vanidades, todo vanidad. \bibverse{3} ¿Qué provecho tiene el hombre de
todo su trabajo con que se afana debajo del sol? \bibverse{4} Generación
va, y generación viene: mas la tierra siempre permanece. \footnote{\textbf{1:4}
  Sal 90,3} \bibverse{5} Y sale el sol, y pónese el sol, y con deseo
vuelve á su lugar donde torna á nacer. \bibverse{6} El viento tira hacia
el mediodía, y rodea al norte; va girando de continuo, y á sus giros
torna el viento de nuevo. \bibverse{7} Los ríos todos van á la mar, y la
mar no se hinche; al lugar de donde los ríos vinieron, allí tornan para
correr de nuevo. \bibverse{8} Todas las cosas andan en trabajo más que
el hombre pueda decir: ni los ojos viendo se hartan de ver, ni los oídos
se hinchen de oir. \bibverse{9} ¿Qué es lo que fué? Lo mismo que será.
¿Qué es lo que ha sido hecho? Lo mismo que se hará: y nada hay nuevo
debajo del sol. \bibverse{10} ¿Hay algo de que se pueda decir: He aquí
esto es nuevo? Ya fué en los siglos que nos han precedido. \bibverse{11}
No hay memoria de lo que precedió, ni tampoco de lo que sucederá habrá
memoria en los que serán después.

\hypertarget{la-inutilidad-de-luchar-por-la-sabiduruxeda-y-el-conocimiento-la-vida-humana-resulta-inuxfatil-para-el-espectador}{%
\subsection{La inutilidad de luchar por la sabiduría y el conocimiento;
La vida humana resulta inútil para el
espectador}\label{la-inutilidad-de-luchar-por-la-sabiduruxeda-y-el-conocimiento-la-vida-humana-resulta-inuxfatil-para-el-espectador}}

\bibverse{12} Yo el Predicador fuí rey sobre Israel en Jerusalem.
\footnote{\textbf{1:12} Ecl 1,1} \bibverse{13} Y dí mi corazón á
inquirir y buscar con sabiduría sobre todo lo que se hace debajo del
cielo: este penoso trabajo dió Dios á los hijos de los hombres, en que
se ocupen. \bibverse{14} Yo miré todas las obras que se hacen debajo del
sol; y he aquí, todo ello es vanidad y aflicción de espíritu.
\bibverse{15} Lo torcido no se puede enderezar; y lo falto no puede
contarse.

\hypertarget{la-buxfasqueda-de-un-conocimiento-claro-conduce-a-la-decepciuxf3n}{%
\subsection{La búsqueda de un conocimiento claro conduce a la
decepción}\label{la-buxfasqueda-de-un-conocimiento-claro-conduce-a-la-decepciuxf3n}}

\bibverse{16} Hablé yo con mi corazón, diciendo: He aquí hállome yo
engrandecido, y he crecido en sabiduría sobre todos los que fueron antes
de mí en Jerusalem; y mi corazón ha percibido muchedumbre de sabiduría y
ciencia. \bibverse{17} Y dí mi corazón á conocer la sabiduría, y también
á entender las locuras y los desvaríos: conocí que aun esto era
aflicción de espíritu. \bibverse{18} Porque en la mucha sabiduría hay
mucha molestia; y quien añade ciencia, añade dolor.

\hypertarget{la-inutilidad-de-intentar-obtener-satisfacciuxf3n-mediante-los-placeres-sensuales-y-el-disfrute-de-la-vida-o-mediante-la-actividad-creativa}{%
\subsection{La inutilidad de intentar obtener satisfacción mediante los
placeres sensuales y el disfrute de la vida o mediante la actividad
creativa}\label{la-inutilidad-de-intentar-obtener-satisfacciuxf3n-mediante-los-placeres-sensuales-y-el-disfrute-de-la-vida-o-mediante-la-actividad-creativa}}

\hypertarget{section-1}{%
\section{2}\label{section-1}}

\bibverse{1} Dije yo en mi corazón: Ven ahora, te probaré con alegría, y
gozarás de bienes. Mas he aquí esto también era vanidad. \bibverse{2} A
la risa dije: Enloqueces; y al placer: ¿De qué sirve esto?

\bibverse{3} Propuse en mi corazón agasajar mi carne con vino, y que
anduviese mi corazón en sabiduría, con retención de la necedad, hasta
ver cuál fuese el bien de los hijos de los hombres, en el cual se
ocuparan debajo del cielo todos los días de su vida. \footnote{\textbf{2:3}
  Prov 31,4} \bibverse{4} Engrandecí mis obras, edifiquéme casas,
plantéme viñas; \bibverse{5} Híceme huertos y jardines, y planté en
ellos árboles de todos frutos; \bibverse{6} Híceme estanques de aguas,
para regar de ellos el bosque donde los árboles crecían. \bibverse{7}
Poseí siervos y siervas, y tuve hijos de familia; también tuve posesión
grande de vacas y ovejas, sobre todos los que fueron antes de mí en
Jerusalem; \bibverse{8} Alleguéme también plata y oro, y tesoro preciado
de reyes y de provincias; híceme de cantores y cantoras, y los deleites
de los hijos de los hombres, instrumentos músicos y de todas suertes.
\bibverse{9} Y fuí engrandecido, y aumentado más que todos los que
fueron antes de mí en Jerusalem: á más de esto perseveró conmigo mi
sabiduría. \bibverse{10} No negué á mis ojos ninguna cosa que desearan,
ni aparté mi corazón de placer alguno, porque mi corazón gozó de todo mi
trabajo: y ésta fué mi parte de toda mi faena. \bibverse{11} Miré yo
luego todas las obras que habían hecho mis manos, y el trabajo que tomé
para hacerlas: y he aquí, todo vanidad y aflicción de espíritu, y no hay
provecho debajo del sol. \footnote{\textbf{2:11} Ecl 1,14}

\hypertarget{al-final-la-sabiduruxeda-es-tan-vacuxeda-como-la-locura-porque-el-destino-final-de-los-sabios-y-los-necios-es-el-mismo}{%
\subsection{Al final, la sabiduría es tan vacía como la locura, porque
el destino final de los sabios y los necios es el
mismo}\label{al-final-la-sabiduruxeda-es-tan-vacuxeda-como-la-locura-porque-el-destino-final-de-los-sabios-y-los-necios-es-el-mismo}}

\bibverse{12} Después torné yo á mirar para ver la sabiduría y los
desvaríos y la necedad; (porque ¿qué hombre hay que pueda seguir al rey
en lo que ya hicieron?) \footnote{\textbf{2:12} Ecl 1,17} \bibverse{13}
Y he visto que la sabiduría sobrepuja á la necedad, como la luz á las
tinieblas. \bibverse{14} El sabio tiene sus ojos en su cabeza, mas el
necio anda en tinieblas: empero también entendí yo que un mismo suceso
acaecerá al uno que al otro. \footnote{\textbf{2:14} Prov 17,24}
\bibverse{15} Entonces dije yo en mi corazón: Como sucederá al necio me
sucederá también á mí: ¿para qué pues he trabajado hasta ahora por
hacerme más sabio? Y dije en mi corazón, que también esto era vanidad.
\bibverse{16} Porque ni del sabio ni del necio habrá memoria para
siempre; pues en los días venideros ya todo será olvidado, y también
morirá el sabio como el necio.

\bibverse{17} Aborrecí por tanto la vida; porque la obra que se hace
debajo del sol me era fastidiosa; por cuanto todo es vanidad y aflicción
de espíritu.

\hypertarget{referencia-al-mal-de-que-el-sabio-debe-dejar-el-beneficio-y-el-disfrute-de-su-laboriosa-labor-a-un-heredero-quizuxe1s-insensato}{%
\subsection{Referencia al mal de que el sabio debe dejar el beneficio y
el disfrute de su laboriosa labor a un heredero quizás
insensato}\label{referencia-al-mal-de-que-el-sabio-debe-dejar-el-beneficio-y-el-disfrute-de-su-laboriosa-labor-a-un-heredero-quizuxe1s-insensato}}

\bibverse{18} Yo asimismo aborrecí todo mi trabajo que había puesto por
obra debajo del sol; el cual dejaré á otro que vendrá después de mí.
\footnote{\textbf{2:18} Ecl 2,21; Ecl 2,26; Sal 39,7} \bibverse{19} ¿Y
quién sabe si será sabio, ó necio, el que se enseñoreará de todo mi
trabajo en que yo me afané, y en que ocupé debajo del sol mi sabiduría?
Esto también es vanidad.

\bibverse{20} Tornéme por tanto á desesperanzar mi corazón acerca de
todo el trabajo en que me afané, y en que había ocupado debajo del sol
mi sabiduría. \bibverse{21} ¡Que el hombre trabaje con sabiduría, y con
ciencia, y con rectitud, y que haya de dar su hacienda á hombre que
nunca trabajó en ello! También es esto vanidad y mal grande.
\bibverse{22} Porque ¿qué tiene el hombre de todo su trabajo, y fatiga
de su corazón, con que debajo del sol él se afanara? \bibverse{23}
Porque todos sus días no son sino dolores, y sus trabajos molestias: aun
de noche su corazón no reposa. Esto también es vanidad.

\hypertarget{entonces-lo-mejor-para-el-hombre-es-disfrutar-el-momento-en-la-medida-en-que-dios-se-lo-conceda.}{%
\subsection{Entonces lo mejor para el hombre es disfrutar el momento en
la medida en que Dios se lo
conceda.}\label{entonces-lo-mejor-para-el-hombre-es-disfrutar-el-momento-en-la-medida-en-que-dios-se-lo-conceda.}}

\bibverse{24} No hay cosa mejor para el hombre sino que coma y beba, y
que su alma vea el bien de su trabajo. También tengo yo visto que esto
es de la mano de Dios. \bibverse{25} Porque ¿quién comerá, y quién se
cuidará, mejor que yo? \bibverse{26} Porque al hombre que le agrada,
Dios le da sabiduría y ciencia y gozo: mas al pecador da trabajo, el que
allegue y amontone, para que dé al que agrada á Dios. También esto es
vanidad y aflicción de espíritu. \footnote{\textbf{2:26} Prov 13,22;
  Prov 28,8}

\hypertarget{todo-tiene-su-tiempo}{%
\subsection{Todo tiene su tiempo}\label{todo-tiene-su-tiempo}}

\hypertarget{section-2}{%
\section{3}\label{section-2}}

\bibverse{1} Para todas las cosas hay sazón, y todo lo que se quiere
debajo del cielo, tiene su tiempo: \footnote{\textbf{3:1} Ecl 8,6}
\bibverse{2} Tiempo de nacer, y tiempo de morir; tiempo de plantar, y
tiempo de arrancar lo plantado; \bibverse{3} Tiempo de matar, y tiempo
de curar; tiempo de destruir, y tiempo de edificar; \bibverse{4} Tiempo
de llorar, y tiempo de reir; tiempo de endechar, y tiempo de bailar;
\bibverse{5} Tiempo de esparcir las piedras, y tiempo de allegar las
piedras; tiempo de abrazar, y tiempo de alejarse de abrazar;
\bibverse{6} Tiempo de agenciar, y tiempo de perder; tiempo de guardar,
y tiempo de arrojar; \bibverse{7} Tiempo de romper, y tiempo de coser;
tiempo de callar, y tiempo de hablar; \bibverse{8} Tiempo de amar, y
tiempo de aborrecer; tiempo de guerra, y tiempo de paz.

\hypertarget{pero-el-hombre-no-conoce-el-tiempo-establecido-por-dios-y-es-impotente-contra-uxe9l}{%
\subsection{Pero el hombre no conoce el tiempo establecido por Dios y es
impotente contra
él}\label{pero-el-hombre-no-conoce-el-tiempo-establecido-por-dios-y-es-impotente-contra-uxe9l}}

\bibverse{9} ¿Qué provecho tiene el que trabaja en lo que trabaja?
\bibverse{10} Yo he visto el trabajo que Dios ha dado á los hijos de los
hombres para que en él se ocupasen. \bibverse{11} Todo lo hizo hermoso
en su tiempo: y aun el mundo dió en su corazón, de tal manera que no
alcance el hombre la obra de Dios desde el principio hasta el cabo.
\footnote{\textbf{3:11} Ecl 8,17} \bibverse{12} Yo he conocido que no
hay mejor para ellos, que alegrarse, y hacer bien en su vida:
\footnote{\textbf{3:12} Ecl 2,24} \bibverse{13} Y también que es don de
Dios que todo hombre coma y beba, y goce el bien de toda su labor.
\bibverse{14} He entendido que todo lo que Dios hace, eso será perpetuo:
sobre aquello no se añadirá, ni de ello se disminuirá; y hácelo Dios,
para que delante de él teman los hombres. \bibverse{15} Aquello que fué,
ya es: y lo que ha de ser, fué ya; y Dios restaura lo que pasó.

\hypertarget{en-el-mundo-humano-hay-maldad-e-injusticia-pero-dios-es-el-juez-mundial}{%
\subsection{En el mundo humano hay maldad e injusticia, pero Dios es el
juez
mundial}\label{en-el-mundo-humano-hay-maldad-e-injusticia-pero-dios-es-el-juez-mundial}}

\bibverse{16} Vi más debajo del sol: en lugar del juicio, allí la
impiedad; y en lugar de la justicia, allí la iniquidad. \bibverse{17} Y
dije yo en mi corazón: Al justo y al impío juzgará Dios; porque allí hay
tiempo á todo lo que se quiere y sobre todo lo que se hace. \footnote{\textbf{3:17}
  Ecl 12,14}

\hypertarget{la-ley-de-la-impermanencia-existe-tanto-para-los-humanos-como-para-los-animales-y-exhorta-a-disfrutar-de-la-vida}{%
\subsection{La ley de la impermanencia existe tanto para los humanos
como para los animales y exhorta a disfrutar de la
vida}\label{la-ley-de-la-impermanencia-existe-tanto-para-los-humanos-como-para-los-animales-y-exhorta-a-disfrutar-de-la-vida}}

\bibverse{18} Dije en mi corazón, en orden á la condición de los hijos
de los hombres, que Dios los probaría, para que así echaran de ver ellos
mismos que son semejantes á las bestias. \bibverse{19} Porque el suceso
de los hijos de los hombres, y el suceso del animal, el mismo suceso es:
como mueren los unos, así mueren los otros; y una misma respiración
tienen todos; ni tiene más el hombre que la bestia: porque todo es
vanidad. \bibverse{20} Todo va á un lugar: todo es hecho del polvo, y
todo se tornará en el mismo polvo. \footnote{\textbf{3:20} Gén 3,19}
\bibverse{21} ¿Quién sabe que el espíritu de los hijos de los hombres
suba arriba, y que el espíritu del animal descienda debajo de la tierra?

\bibverse{22} Así que he visto que no hay cosa mejor que alegrarse el
hombre con lo que hiciere; porque esta es su parte: porque ¿quién lo
llevará para que vea lo que ha de ser después de él?

\hypertarget{la-opresiuxf3n-los-celos-y-el-trabajo-en-parte-inquieto-la-calma-en-parte-lenta-devaluxfaan-la-vida}{%
\subsection{La opresión, los celos y el trabajo en parte inquieto, la
calma en parte lenta devalúan la
vida}\label{la-opresiuxf3n-los-celos-y-el-trabajo-en-parte-inquieto-la-calma-en-parte-lenta-devaluxfaan-la-vida}}

\hypertarget{section-3}{%
\section{4}\label{section-3}}

\bibverse{1} Y tornéme yo, y vi todas las violencias que se hacen debajo
del sol: y he aquí las lágrimas de los oprimidos, y sin tener quien los
consuele; y la fuerza estaba en la mano de sus opresores, y para ellos
no había consolador. \bibverse{2} Y alabé yo los finados que ya
murieron, más que los vivientes que hasta ahora están vivos.
\bibverse{3} Y tuve por mejor que unos y otros al que no ha sido aún,
que no ha visto las malas obras que debajo del sol se hacen. \footnote{\textbf{4:3}
  Ecl 6,3} \bibverse{4} Visto he asimismo que todo trabajo y toda
excelencia de obras mueve la envidia del hombre contra su prójimo.
También esto es vanidad y aflicción de espíritu.

\bibverse{5} El necio dobla sus manos y come su carne. \bibverse{6} Mas
vale el un puño lleno con descanso, que ambos puños llenos con trabajo y
aflicción de espíritu. \footnote{\textbf{4:6} Prov 15,16}

\hypertarget{el-esfuerzo-de-la-persona-soltera-es-inuxfatil-dos-trabajadores-que-trabajan-juntos-estuxe1n-mejor}{%
\subsection{El esfuerzo de la persona soltera es inútil; dos
trabajadores que trabajan juntos están
mejor}\label{el-esfuerzo-de-la-persona-soltera-es-inuxfatil-dos-trabajadores-que-trabajan-juntos-estuxe1n-mejor}}

\bibverse{7} Yo me torné otra vez, y vi vanidad debajo del sol.
\footnote{\textbf{4:7} Ecl 2,12} \bibverse{8} Está un hombre solo y sin
sucesor; que ni tiene hijo ni hermano; mas nunca cesa de trabajar, ni
sus ojos se hartan de sus riquezas, ni se pregunta: ¿Para quién trabajo
yo, y defraudo mi alma del bien? También esto es vanidad, y duro
trabajo.

\bibverse{9} Mejores son dos que uno; porque tienen mejor paga de su
trabajo. \bibverse{10} Porque si cayeren, el uno levantará á su
compañero: mas ¡ay del solo! que cuando cayere, no habrá segundo que lo
levante. \bibverse{11} También si dos durmieren juntos, se calentarán;
mas ¿cómo se calentará uno solo? \bibverse{12} Y si alguno prevaleciere
contra el uno, dos estarán contra él; y cordón de tres dobleces no
presto se rompe.

\hypertarget{comunicaciuxf3n-de-un-evento-histuxf3rico-que-confirma-la-observaciuxf3n-del-predicador-de-que-el-favor-popular-no-es-confiable}{%
\subsection{Comunicación de un evento histórico que confirma la
observación del predicador de que el favor popular no es
confiable}\label{comunicaciuxf3n-de-un-evento-histuxf3rico-que-confirma-la-observaciuxf3n-del-predicador-de-que-el-favor-popular-no-es-confiable}}

\bibverse{13} Mejor es el muchacho pobre y sabio, que el rey viejo y
fatuo que no sabe ser aconsejado. \bibverse{14} Porque de la cárcel
salió para reinar; mientras el nacido en su reino se hizo pobre.
\footnote{\textbf{4:14} Gén 41,14}

\bibverse{15} Vi todos los vivientes debajo del sol caminando con el
muchacho sucesor, que estará en lugar de aquél. \bibverse{16} No tiene
fin todo el pueblo que fué antes de ellos: tampoco los que vendrán
después estarán con él contentos. Y esto es también vanidad y aflicción
de espíritu.

\hypertarget{recordatorio-de-tener-cuidado-al-realizar-deberes-religiosos-con-sacrificios-oraciuxf3n-y-votos}{%
\subsection{Recordatorio de tener cuidado al realizar deberes religiosos
(con sacrificios, oración y
votos)}\label{recordatorio-de-tener-cuidado-al-realizar-deberes-religiosos-con-sacrificios-oraciuxf3n-y-votos}}

\hypertarget{section-4}{%
\section{5}\label{section-4}}

\bibverse{1} Cuando fueres á la casa de Dios, guarda tu pie; y acércate
más para oir que para dar el sacrificio de los necios: porque no saben
que hacen mal. \footnote{\textbf{5:1} Sant 1,19} \bibverse{2} No te des
priesa con tu boca, ni tu corazón se apresure á proferir palabra delante
de Dios; porque Dios está en el cielo, y tú sobre la tierra: por tanto,
sean pocas tus palabras. \footnote{\textbf{5:2} Ecl 10,14; Prov 10,19}
\bibverse{3} Porque de la mucha ocupación viene el sueño, y de la
multitud de las palabras la voz del necio. \footnote{\textbf{5:3} Deut
  23,22} \bibverse{4} Cuando á Dios hicieres promesa, no tardes en
pagarla; porque no se agrada de los insensatos. Paga lo que prometieres.
\bibverse{5} Mejor es que no prometas, que no que prometas y no pagues.
\footnote{\textbf{5:5} Mal 2,7} \bibverse{6} No sueltes tu boca para
hacer pecar á tu carne; ni digas delante del ángel, que fué ignorancia.
¿Por qué harás que Dios se aire á causa de tu voz, y que destruya la
obra de tus manos? \bibverse{7} Donde los sueños son en multitud,
también lo son las vanidades y muchas las palabras; mas tú teme á Dios.

\hypertarget{las-opresiones-en-el-estado-son-lamentables-pero-comprensibles-bendiciones-de-regaluxedas-para-los-estados-agruxedcolas}{%
\subsection{Las opresiones en el estado son lamentables, pero
comprensibles; Bendiciones de regalías para los estados
agrícolas}\label{las-opresiones-en-el-estado-son-lamentables-pero-comprensibles-bendiciones-de-regaluxedas-para-los-estados-agruxedcolas}}

\bibverse{8} Si violencias de pobres, y extorsión de derecho y de
justicia vieres en la provincia, no te maravilles de esta licencia;
porque alto está mirando sobre alto, y uno más alto está sobre ellos.
\bibverse{9} Además el provecho de la tierra es para todos: el rey mismo
está sujeto á los campos.

\hypertarget{la-nulidad-y-las-quejas-de-las-riquezas}{%
\subsection{La nulidad y las quejas de las
riquezas}\label{la-nulidad-y-las-quejas-de-las-riquezas}}

\bibverse{10} El que ama el dinero, no se hartará de dinero; y el que
ama el mucho tener, no sacará fruto. También esto es vanidad.
\bibverse{11} Cuando los bienes se aumentan, también se aumentan sus
comedores. ¿Qué bien, pues, tendrá su dueño, sino verlos con sus ojos?

\bibverse{12} Dulce es el sueño del trabajador, ora coma mucho ó poco;
mas al rico no le deja dormir la hartura.

\bibverse{13} Hay una trabajosa enfermedad que he visto debajo del sol:
las riquezas guardadas de sus dueños para su mal; \bibverse{14} Las
cuales se pierden en malas ocupaciones, y á los hijos que engendraron
nada les queda en la mano. \footnote{\textbf{5:14} Job 1,21; Sal 49,18}
\bibverse{15} Como salió del vientre de su madre, desnudo, así se
vuelve, tornando como vino; y nada tuvo de su trabajo para llevar en su
mano. \bibverse{16} Este también es un gran mal, que como vino, así haya
de volver. ¿Y de qué le aprovechó trabajar al viento? \bibverse{17}
Demás de esto, todos los días de su vida comerá en tinieblas, con mucho
enojo y dolor y miseria.

\hypertarget{recomendaciuxf3n-del-disfrute-de-la-vida-ademuxe1s-del-trabajo-y-la-riqueza}{%
\subsection{Recomendación del disfrute de la vida además del trabajo y
la
riqueza}\label{recomendaciuxf3n-del-disfrute-de-la-vida-ademuxe1s-del-trabajo-y-la-riqueza}}

\bibverse{18} He aquí pues el bien que yo he visto: Que lo bueno es
comer y beber, y gozar uno del bien de todo su trabajo con que se fatiga
debajo del sol, todos los días de su vida que Dios le ha dado; porque
esta es su parte. \bibverse{19} Asimismo, á todo hombre á quien Dios dió
riquezas y hacienda, y le dió también facultad para que coma de ellas, y
tome su parte, y goce su trabajo; esto es don de Dios. \bibverse{20}
Porque no se acordará mucho de los días de su vida; pues Dios le
responderá con alegría de su corazón.

\hypertarget{alguien-tiene-bienes-ricos-pero-no-los-disfruta}{%
\subsection{Alguien tiene bienes ricos pero no los
disfruta}\label{alguien-tiene-bienes-ricos-pero-no-los-disfruta}}

\hypertarget{section-5}{%
\section{6}\label{section-5}}

\bibverse{1} Hay un mal que he visto debajo del cielo, y muy común entre
los hombres: \bibverse{2} Hombre á quien Dios dió riquezas, y hacienda,
y honra, y nada le falta de todo lo que su alma desea; mas Dios no le
dió facultad de comer de ello, sino que los extraños se lo comen. Esto
vanidad es, y enfermedad trabajosa. \footnote{\textbf{6:2} Ecl 2,18}

\bibverse{3} Si el hombre engendrare ciento, y viviere muchos años, y
los días de su edad fueren numerosos; si su alma no se hartó del bien, y
también careció de sepultura, yo digo que el abortivo es mejor que él.
\bibverse{4} Porque en vano vino, y á tinieblas va, y con tinieblas será
cubierto su nombre. \bibverse{5} Aunque no haya visto el sol, ni
conocido nada, más reposo tiene éste que aquél. \bibverse{6} Porque si
viviere aquél mil años dos veces, si no ha gozado del bien, cierto todos
van á un lugar.

\hypertarget{la-insaciabilidad-del-deseo-y-la-buxfasqueda-del-placer}{%
\subsection{La insaciabilidad del deseo y la búsqueda del
placer}\label{la-insaciabilidad-del-deseo-y-la-buxfasqueda-del-placer}}

\bibverse{7} Todo el trabajo del hombre es para su boca, y con todo eso
su alma no se harta. \bibverse{8} Porque ¿qué más tiene el sabio que el
necio? ¿qué más tiene el pobre que supo caminar entre los vivos?
\bibverse{9} Más vale vista de ojos que deseo que pasa. Y también esto
es vanidad y aflicción de espíritu.

\hypertarget{la-impotencia-humana-en-relaciuxf3n-con-la-predestinaciuxf3n-divina-de-todas-las-cosas-especialmente-la-vida-de-personas-individuales}{%
\subsection{La impotencia humana en relación con la predestinación
divina de todas las cosas (especialmente la vida de personas
individuales)}\label{la-impotencia-humana-en-relaciuxf3n-con-la-predestinaciuxf3n-divina-de-todas-las-cosas-especialmente-la-vida-de-personas-individuales}}

\bibverse{10} El que es, ya su nombre ha sido nombrado; y se sabe que es
hombre, y que no podrá contender con el que es más fuerte que él.
\bibverse{11} Ciertamente las muchas palabras multiplican la vanidad.
¿Qué más tiene el hombre? \bibverse{12} Porque ¿quién sabe cuál es el
bien del hombre en la vida, todos los días de la vida de su vanidad, los
cuales él pasa como sombra? Porque ¿quién enseñará al hombre qué será
después de él debajo del sol?

\hypertarget{advertencias-para-ser-serios-con-la-vida-y-someterse-pacientemente-a-los-decretos-divinos}{%
\subsection{Advertencias para ser serios con la vida y someterse
pacientemente a los decretos
divinos}\label{advertencias-para-ser-serios-con-la-vida-y-someterse-pacientemente-a-los-decretos-divinos}}

\hypertarget{section-6}{%
\section{7}\label{section-6}}

\bibverse{1} Mejor es la buena fama que el buen ungüento; y el día de la
muerte que el día del nacimiento. \footnote{\textbf{7:1} Prov 22,1}
\bibverse{2} Mejor es ir á la casa del luto que á la casa del convite:
porque aquello es el fin de todos los hombres; y el que vive parará
mientes. \bibverse{3} Mejor es el enojo que la risa: porque con la
tristeza del rostro se enmendará el corazón. \bibverse{4} El corazón de
los sabios, en la casa del luto; mas el corazón de los insensatos, en la
casa del placer. \bibverse{5} Mejor es oir la reprensión del sabio, que
la canción de los necios. \bibverse{6} Porque la risa del necio es como
el estrépito de las espinas debajo de la olla. Y también esto es
vanidad. \bibverse{7} Ciertamente la opresión hace enloquecer al sabio:
y el presente corrompe el corazón. \bibverse{8} Mejor es el fin del
negocio que su principio: mejor es el sufrido de espíritu que el altivo
de espíritu.

\bibverse{9} No te apresures en tu espíritu á enojarte: porque la ira en
el seno de los necios reposa. \bibverse{10} Nunca digas: ¿Qué es la
causa que los tiempos pasados fueron mejores que éstos? Porque nunca de
esto preguntarás con sabiduría.

\bibverse{11} Buena es la ciencia con herencia; y más á los que ven el
sol. \bibverse{12} Porque escudo es la ciencia, y escudo es el dinero:
mas la sabiduría excede en que da vida á sus poseedores. \footnote{\textbf{7:12}
  Prov 3,2}

\bibverse{13} Mira la obra de Dios; porque ¿quién podrá enderezar lo que
él torció? \bibverse{14} En el día del bien goza del bien; y en el día
del mal considera. Dios también hizo esto delante de lo otro, porque el
hombre no halle nada tras de él.

\hypertarget{advertencia-contra-todo-exceso-y-amonestaciuxf3n-de-la-verdadera-sabiduruxeda}{%
\subsection{Advertencia contra todo exceso y amonestación de la
verdadera
sabiduría}\label{advertencia-contra-todo-exceso-y-amonestaciuxf3n-de-la-verdadera-sabiduruxeda}}

\bibverse{15} Todo esto he visto en los días de mi vanidad. Justo hay
que perece por su justicia, y hay impío que por su maldad alarga sus
días. \bibverse{16} No seas demasiado justo, ni seas sabio con exceso:
¿por qué te destruirás? \bibverse{17} No hagas mal mucho, ni seas
insensato: ¿por qué morirás antes de tu tiempo? \bibverse{18} Bueno es
que tomes esto, y también de estotro no apartes tu mano; porque el que á
Dios teme, saldrá con todo. \bibverse{19} La sabiduría fortifica al
sabio más que diez poderosos la ciudad en que fueron. \bibverse{20}
Ciertamente no hay hombre justo en la tierra, que haga bien y nunca
peque. \footnote{\textbf{7:20} Sal 14,3} \bibverse{21} Tampoco apliques
tu corazón á todas las cosas que se hablaren, porque no oigas á tu
siervo que dice mal de ti: \bibverse{22} Porque tu corazón sabe, como tú
también dijiste mal de otros muchas veces. \bibverse{23} Todas estas
cosas probé con sabiduría, diciendo: Hacerme he sabio: mas ella se alejó
de mí. \bibverse{24} Lejos está lo que fué; y lo muy profundo ¿quién lo
hallará?

\hypertarget{las-malas-experiencias-del-predicador-con-las-mujeres}{%
\subsection{Las malas experiencias del predicador con las
mujeres}\label{las-malas-experiencias-del-predicador-con-las-mujeres}}

\bibverse{25} Yo he rodeado con mi corazón por saber, y examinar, é
inquirir la sabiduría, y la razón; y por conocer la maldad de la
insensatez, y el desvarío del error;

\bibverse{26} Y yo he hallado más amarga que la muerte la mujer, la cual
es redes, y lazos su corazón; sus manos como ligaduras. El que agrada á
Dios escapará de ella; mas el pecador será preso en ella.

\bibverse{27} He aquí, esto he hallado, dice el Predicador, pesando las
cosas una por una para hallar la razón; \bibverse{28} Lo que aun busca
mi alma, y no encuentro: un hombre entre mil he hallado; mas mujer de
todas éstas nunca hallé. \bibverse{29} He aquí, solamente he hallado
esto: que Dios hizo al hombre recto, mas ellos buscaron muchas cuentas.
\footnote{\textbf{7:29} Prov 2,7}

\hypertarget{la-conducta-del-sabio-hacia-el-gobernante-y-en-duxedas-de-opresiuxf3n}{%
\subsection{La conducta del sabio hacia el gobernante y en días de
opresión}\label{la-conducta-del-sabio-hacia-el-gobernante-y-en-duxedas-de-opresiuxf3n}}

\hypertarget{section-7}{%
\section{8}\label{section-7}}

\bibverse{1} ¿QUIÉN como el sabio? ¿y quién como el que sabe la
declaración de las cosas? La sabiduría del hombre hará relucir su
rostro, y mudaráse la tosquedad de su semblante.

\bibverse{2} Yo te aviso que guardes el mandamiento del rey y la palabra
del juramento de Dios. \bibverse{3} No te apresures á irte de delante de
él, ni en cosa mala persistas; porque él hará todo lo que quisiere:
\bibverse{4} Pues la palabra del rey es con potestad, ¿y quién le dirá,
Qué haces? \bibverse{5} El que guarda el mandamiento no experimentará
mal; y el tiempo y el juicio conoce el corazón del sabio.

\hypertarget{impotencia-e-desorientaciuxf3n-del-hombre}{%
\subsection{Impotencia e desorientación del
hombre}\label{impotencia-e-desorientaciuxf3n-del-hombre}}

\bibverse{6} Porque para todo lo que quisieres hay tiempo y juicio; mas
el trabajo del hombre es grande sobre él; \bibverse{7} Porque no sabe lo
que ha de ser; y el cuándo haya de ser, ¿quién se lo enseñará?
\footnote{\textbf{8:7} Ecl 10,14} \bibverse{8} No hay hombre que tenga
potestad sobre el espíritu para retener el espíritu, ni potestad sobre
el día de la muerte: y no valen armas en tal guerra; ni la impiedad
librará al que la posee.

\hypertarget{justos-y-malvados-generalmente-corren-el-mismo-destino-en-una-sola-violencia-pertenece-cuando-uno-tiene-pautas-para-el-disfrute-de-la-vida-en-el-trabajo}{%
\subsection{Justos y malvados generalmente corren el mismo destino en
una sola violencia; Pertenece cuando uno tiene pautas para el disfrute
de la vida en el
trabajo}\label{justos-y-malvados-generalmente-corren-el-mismo-destino-en-una-sola-violencia-pertenece-cuando-uno-tiene-pautas-para-el-disfrute-de-la-vida-en-el-trabajo}}

\bibverse{9} Todo esto he visto, y puesto he mi corazón en todo lo que
debajo del sol se hace: hay tiempo en que el hombre se enseñorea del
hombre para mal suyo. \bibverse{10} Esto vi también: que los impíos
sepultados vinieron aún en memoria; mas los que partieron del lugar
santo, fueron luego puestos en olvido en la ciudad donde con rectitud
habían obrado. Esto también es vanidad. \bibverse{11} Porque no se
ejecuta luego sentencia sobre la mala obra, el corazón de los hijos de
los hombres está en ellos lleno para hacer mal. \bibverse{12} Bien que
el pecador haga mal cien veces, y le sea dilatado el castigo, con todo
yo también sé que los que á Dios temen tendrán bien, los que temieren
ante su presencia; \footnote{\textbf{8:12} Sal 73,17-26} \bibverse{13} Y
que el impío no tendrá bien, ni le serán prolongados los días, que son
como sombra; por cuanto no temió delante de la presencia de Dios.

\bibverse{14} Hay vanidad que se hace sobre la tierra: que hay justos á
quienes sucede como si hicieran obras de impíos; y hay impíos á quienes
acaece como si hicieran obras de justos. Digo que esto también es
vanidad. \bibverse{15} Por tanto alabé yo la alegría; que no tiene el
hombre bien debajo del sol, sino que coma y beba, y se alegre; y que
esto se le quede de su trabajo los días de su vida que Dios le dió
debajo del sol. \footnote{\textbf{8:15} Ecl 2,24}

\hypertarget{el-gobierno-de-dios-en-el-gobierno-mundial-es-insondable-para-el-hombre}{%
\subsection{El gobierno de Dios en el gobierno mundial es insondable
para el
hombre}\label{el-gobierno-de-dios-en-el-gobierno-mundial-es-insondable-para-el-hombre}}

\bibverse{16} Yo pues dí mi corazón á conocer sabiduría, y á ver la
faena que se hace sobre la tierra; (porque hay quien ni de noche ni de
día ve sueño en su ojos;) \bibverse{17} Y he visto todas las obras de
Dios, que el hombre no puede alcanzar la obra que debajo del sol se
hace; por mucho que trabaje el hombre buscándola, no la hallará: aunque
diga el sabio que la sabe, no por eso podrá alcanzarla.

\hypertarget{la-misma-suerte-para-todos-en-la-vida-y-en-la-muerte-impotencia-humana-contra-la-deidad-el-disfrute-piadoso-de-la-vida-antes-de-la-muerte-establece-una-meta-para-todo-gozo-y-actividad}{%
\subsection{La misma suerte para todos en la vida y en la muerte;
impotencia humana contra la deidad; El disfrute piadoso de la vida antes
de la muerte establece una meta para todo gozo y
actividad}\label{la-misma-suerte-para-todos-en-la-vida-y-en-la-muerte-impotencia-humana-contra-la-deidad-el-disfrute-piadoso-de-la-vida-antes-de-la-muerte-establece-una-meta-para-todo-gozo-y-actividad}}

\hypertarget{section-8}{%
\section{9}\label{section-8}}

\bibverse{1} Ciertamente dado he mi corazón á todas estas cosas, para
declarar todo esto: que los justos y los sabios, y sus obras, están en
la mano de Dios; y que no sabe el hombre ni el amor ni el odio por todo
lo que pasa delante de él. \bibverse{2} Todo acontece de la misma manera
á todos: un mismo suceso ocurre al justo y al impío; al bueno y al
limpio y al no limpio; al que sacrifica, y al que no sacrifica: como el
bueno, así el que peca; el que jura, como el que teme el juramento.
\footnote{\textbf{9:2} Ecl 2,14; Job 9,22} \bibverse{3} Este mal hay
entre todo lo que se hace debajo del sol, que todos tengan un mismo
suceso, y también que el corazón de los hijos de los hombres esté lleno
de mal, y de enloquecimiento en su corazón durante su vida: y después, á
los muertos. \footnote{\textbf{9:3} Ecl 8,11} \bibverse{4} Aun hay
esperanza para todo aquél que está entre los vivos; porque mejor es
perro vivo que león muerto. \bibverse{5} Porque los que viven saben que
han de morir: mas los muertos nada saben, ni tienen más paga; porque su
memoria es puesta en olvido. \bibverse{6} También su amor, y su odio y
su envidia, feneció ya: ni tiene ya más parte en el siglo, en todo lo
que se hace debajo del sol.

\bibverse{7} Anda, y come tu pan con gozo, y bebe tu vino con alegre
corazón: porque tus obras ya son agradables á Dios. \footnote{\textbf{9:7}
  Ecl 5,17} \bibverse{8} En todo tiempo sean blancos tus vestidos, y
nunca falte ungüento sobre tu cabeza. \bibverse{9} Goza de la vida con
la mujer que amas, todos los días de la vida de tu vanidad, que te son
dados debajo del sol, todos los días de tu vanidad; porque esta es tu
parte en la vida, y en tu trabajo con que te afanas debajo del sol.
\bibverse{10} Todo lo que te viniere á la mano para hacer, hazlo según
tus fuerzas; porque en el sepulcro, adonde tú vas, no hay obra, ni
industria, ni ciencia, ni sabiduría.

\hypertarget{la-dependencia-del-hombre-del-destino}{%
\subsection{La dependencia del hombre del
destino}\label{la-dependencia-del-hombre-del-destino}}

\bibverse{11} Tornéme, y vi debajo del sol, que ni es de los ligeros la
carrera, ni la guerra de los fuertes, ni aun de los sabios el pan, ni de
los prudentes las riquezas, ni de los elocuentes el favor; sino que
tiempo y ocasión acontece á todos. \footnote{\textbf{9:11} Jer 10,23}
\bibverse{12} Porque el hombre tampoco conoce su tiempo: como los peces
que son presos en la mala red, y como las aves que se prenden en lazo,
así son enlazados los hijos de los hombres en el tiempo malo, cuando cae
de repente sobre ellos.

\hypertarget{muxe1s-experiencias-de-vida-y-dichos-de-sabiduruxeda}{%
\subsection{Más experiencias de vida y dichos de
sabiduría}\label{muxe1s-experiencias-de-vida-y-dichos-de-sabiduruxeda}}

\bibverse{13} También vi esta sabiduría debajo del sol, la cual me
parece grande: \bibverse{14} Una pequeña ciudad, y pocos hombres en
ella; y viene contra ella un gran rey, y cércala, y edifica contra ella
grandes baluartes: \bibverse{15} Y hállase en ella un hombre pobre,
sabio, el cual libra la ciudad con su sabiduría; y nadie se acordaba de
aquel pobre hombre. \bibverse{16} Entonces dije yo: Mejor es la
sabiduría que la fortaleza; aunque la ciencia del pobre sea
menospreciada, y no sean escuchadas sus palabras. \bibverse{17} Las
palabras del sabio con reposo son oídas, más que el clamor del señor
entre los necios. \bibverse{18} Mejor es la sabiduría que las armas de
guerra; mas un pecador destruye mucho bien.

\hypertarget{section-9}{%
\section{10}\label{section-9}}

\bibverse{1} Las moscas muertas hacen heder y dar mal olor el perfume
del perfumista: así una pequeña locura, al estimado por sabiduría y
honra. \bibverse{2} El corazón del sabio está á su mano derecha; mas el
corazón del necio á su mano izquierda. \bibverse{3} Y aun mientras va el
necio por el camino, fáltale su cordura, y dice á todos, que es necio.
\bibverse{4} Si el espíritu del príncipe se exaltare contra ti, no dejes
tu lugar; porque la lenidad hará cesar grandes ofensas.

\bibverse{5} Hay un mal que debajo del sol he visto, á manera de error
emanado del príncipe: \bibverse{6} La necedad está colocada en grandes
alturas, y los ricos están sentados en lugar bajo. \footnote{\textbf{10:6}
  Prov 30,21-22} \bibverse{7} Vi siervos en caballos, y príncipes que
andaban como siervos sobre la tierra. \bibverse{8} El que hiciere el
hoyo caerá en él; y el que aportillare el vallado, morderále la
serpiente. \bibverse{9} El que mudare las piedras, trabajo tendrá en
ellas: el que cortare la leña, en ella peligrará. \bibverse{10} Si se
embotare el hierro, y su filo no fuere amolado, hay que añadir entonces
más fuerza: empero excede la bondad de la sabiduría.

\bibverse{11} Muerde la serpiente cuando no está encantada, y el
lenguaraz no es mejor. \footnote{\textbf{10:11} Sal 58,5-6}
\bibverse{12} Las palabras de la boca del sabio son gracia; mas los
labios del necio causan su propia ruina. \bibverse{13} El principio de
las palabras de su boca es necedad; y el fin de su charla nocivo
desvarío. \bibverse{14} El necio multiplica palabras: no sabe hombre lo
que ha de ser; ¿y quién le hará saber lo que después de él será?

\bibverse{15} El trabajo de los necios los fatiga; porque no saben por
dónde ir á la ciudad. \bibverse{16} ¡Ay de ti, tierra, cuando tu rey es
muchacho, y tus príncipes comen de mañana! \footnote{\textbf{10:16} Is
  3,4} \bibverse{17} ¡Bienaventurada, tú, tierra, cuando tu rey es hijo
de nobles, y tus príncipes comen á su hora, por refección, y no por el
beber! \bibverse{18} Por la pereza se cae la techumbre, y por flojedad
de manos se llueve la casa. \bibverse{19} Por el placer se hace el
convite, y el vino alegra los vivos: y el dinero responde á todo.
\footnote{\textbf{10:19} Jue 9,13; Sal 104,15} \bibverse{20} Ni aun en
tu pensamiento digas mal del rey, ni en los secretos de tu cámara digas
mal del rico; porque las aves del cielo llevarán la voz, y las que
tienen alas harán saber la palabra. \footnote{\textbf{10:20} Éxod 22,27}

\hypertarget{actuaciuxf3n-inteligente-y-rentable-ante-la-incertidumbre-de-todo-lo-terrenal}{%
\subsection{Actuación inteligente y rentable ante la incertidumbre de
todo lo
terrenal}\label{actuaciuxf3n-inteligente-y-rentable-ante-la-incertidumbre-de-todo-lo-terrenal}}

\hypertarget{section-10}{%
\section{11}\label{section-10}}

\bibverse{1} Echa tu pan sobre las aguas; que después de muchos días lo
hallarás. \footnote{\textbf{11:1} Prov 19,17} \bibverse{2} Reparte á
siete, y aun á ocho: porque no sabes el mal que vendrá sobre la tierra.
\bibverse{3} Si las nubes fueren llenas de agua, sobre la tierra la
derramarán: y si el árbol cayere al mediodía, ó al norte, al lugar que
el árbol cayere, allí quedará. \bibverse{4} El que al viento mira, no
sembrará; y el que mira á las nubes, no segará. \bibverse{5} Como tú no
sabes cuál es el camino del viento, ó cómo se crían los huesos en el
vientre de la mujer preñada, así ignoras la obra de Dios, el cual hace
todas las cosas. \bibverse{6} Por la mañana siembra tu simiente, y á la
tarde no dejes reposar tu mano: porque tú no sabes cuál es lo mejor, si
esto ó lo otro, ó si ambas á dos cosas son buenas. \bibverse{7} Suave
ciertamente es la luz, y agradable á los ojos ver el sol: \bibverse{8}
Mas si el hombre viviere muchos años, y en todos ellos hubiere gozado
alegría; si después trajere á la memoria los días de las tinieblas, que
serán muchos, todo lo que le habrá pasado, dirá haber sido vanidad.

\hypertarget{recordatorio-para-disfrutar-plenamente-de-la-vida-en-la-juventud-pero-agrada-a-dios}{%
\subsection{Recordatorio para disfrutar plenamente de la vida en la
juventud, pero agrada a
Dios}\label{recordatorio-para-disfrutar-plenamente-de-la-vida-en-la-juventud-pero-agrada-a-dios}}

\bibverse{9} Alégrate, mancebo, en tu mocedad, y tome placer tu corazón
en los días de tu juventud; y anda en los caminos de tu corazón, y en la
vista de tus ojos: mas sabe, que sobre todas estas cosas te traerá Dios
á juicio. \footnote{\textbf{11:9} Ecl 8,15}

\bibverse{10} Quita pues el enojo de tu corazón, y aparta el mal de tu
carne: porque la mocedad y la juventud son vanidad.

\hypertarget{section-11}{%
\section{12}\label{section-11}}

\bibverse{1} Y acuérdate de tu Criador en los días de tu juventud, antes
que vengan los malos días, y lleguen los años, de los cuales digas, No
tengo en ellos contentamiento;

\hypertarget{descripciuxf3n-de-las-dolencias-de-la-vejez}{%
\subsection{Descripción de las dolencias de la
vejez}\label{descripciuxf3n-de-las-dolencias-de-la-vejez}}

\bibverse{2} Antes que se oscurezca el sol, y la luz, y la luna y las
estrellas, y las nubes se tornen tras la lluvia: \bibverse{3} Cuando
temblarán los guardas de la casa, y se encorvarán los hombres fuertes, y
cesarán las muelas, porque han disminuído, y se oscurecerán los que
miran por las ventanas; \bibverse{4} Y las puertas de afuera se
cerrarán, por la bajeza de la voz de la muela; y levantaráse á la voz
del ave, y todas las hijas de canción serán humilladas; \bibverse{5}
Cuando también temerán de lo alto, y los tropezones en el camino; y
florecerá el almendro, y se agravará la langosta, y perderáse el
apetito: porque el hombre va á la casa de su siglo, y los endechadores
andarán en derredor por la plaza: \bibverse{6} Antes que la cadena de
plata se quiebre, y se rompa el cuenco de oro, y el cántaro se quiebre
junto á la fuente, y la rueda sea rota sobre el pozo; \bibverse{7} Y el
polvo se torne á la tierra, como era, y el espíritu se vuelva á Dios que
lo dió.

\bibverse{8} Vanidad de vanidades, dijo el Predicador, todo vanidad.
\footnote{\textbf{12:8} Ecl 1,2}

\bibverse{9} Y cuanto más sabio fué el Predicador, tanto más enseñó
sabiduría al pueblo; é hizo escuchar, é hizo escudriñar, y compuso
muchos proverbios. \footnote{\textbf{12:9} 1Re 5,12} \bibverse{10}
Procuró el Predicador hallar palabras agradables, y escritura recta,
palabras de verdad. \bibverse{11} Las palabras de los sabios son como
aguijones; y como clavos hincados, las de los maestros de las
congregaciones, dadas por un Pastor. \footnote{\textbf{12:11} Heb 4,12}

\hypertarget{advertencia-contra-cavilaciones-inuxfatiles-lista-del-resultado-final}{%
\subsection{Advertencia contra cavilaciones inútiles; Lista del
resultado
final}\label{advertencia-contra-cavilaciones-inuxfatiles-lista-del-resultado-final}}

\bibverse{12} Ahora, hijo mío, á más de esto, sé avisado. No hay fin de
hacer muchos libros; y el mucho estudio aflicción es de la carne.

\bibverse{13} El fin de todo el discurso oído es este: Teme á Dios, y
guarda sus mandamientos; porque esto es el todo del hombre.
\bibverse{14} Porque Dios traerá toda obra á juicio, el cual se hará
sobre toda cosa oculta, buena ó mala.
