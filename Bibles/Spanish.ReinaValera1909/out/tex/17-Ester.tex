\hypertarget{la-fiesta-del-rey-persa-assuero-en-susa-para-los-dignatarios-y-altos-funcionarios-de-su-imperio}{%
\subsection{La fiesta del rey persa Assuero en Susa para los dignatarios
y altos funcionarios de su
imperio}\label{la-fiesta-del-rey-persa-assuero-en-susa-para-los-dignatarios-y-altos-funcionarios-de-su-imperio}}

\hypertarget{section}{%
\section{1}\label{section}}

\bibleverse{1} Y aconteció en los días de Assuero, (el Assuero que reinó
desde la India hasta la Etiopía sobre ciento veinte y siete provincias,)
\bibleverse{2} Que en aquellos días, asentado que fué el rey Assuero en
la silla de su reino, la cual estaba en Susán capital del reino,
\bibleverse{3} En el tercer año de su reinado hizo banquete á todos sus
príncipes y siervos, teniendo delante de él la fuerza de Persia y de
Media, gobernadores y príncipes de provincias, \bibleverse{4} Para
mostrar él las riquezas de la gloria de su reino, y el lustre de la
magnificencia de su poder, por muchos días, ciento y ochenta días.

\hypertarget{la-comida-de-los-habitantes-de-la-ciudad-real-susa-la-fiesta-de-la-reina-wasthi}{%
\subsection{La comida de los habitantes de la ciudad real Susa; la
fiesta de la reina
Wasthi}\label{la-comida-de-los-habitantes-de-la-ciudad-real-susa-la-fiesta-de-la-reina-wasthi}}

\bibleverse{5} Y cumplidos estos días, hizo el rey banquete por siete
días en el patio del huerto del palacio real á todo el pueblo, desde el
mayor hasta el menor que se halló en Susán capital del reino.
\bibleverse{6} El pabellón era de blanco, verde, y cárdeno, tendido
sobre cuerdas de lino y púrpura en sortijas de plata y columnas de
mármol: los reclinatorios de oro y de plata, sobre losado de pórfido y
de mármol, y de alabastro y de jacinto. \bibleverse{7} Y daban á beber
en vasos de oro, y vasos diferentes unos de otros, y mucho vino real,
conforme á la facultad del rey. \bibleverse{8} Y la bebida fué según
esta ley: Que nadie constriñese; porque así lo había mandado el rey á
todos los mayordomos de su casa; que se hiciese según la voluntad de
cada uno.

\bibleverse{9} Asimismo la reina Vasthi hizo banquete de mujeres, en la
casa real del rey Assuero.

\hypertarget{wasthi-se-niega-a-aparecer-en-el-saluxf3n-de-baile}{%
\subsection{Wasthi se niega a aparecer en el salón de
baile}\label{wasthi-se-niega-a-aparecer-en-el-saluxf3n-de-baile}}

\bibleverse{10} El séptimo día, estando el corazón del rey alegre del
vino, mandó á Mehumán, y á Biztha, y á Harbona, y á Bigtha, y á Abagtha,
y á Zetar, y á Carcas, siete eunucos que servían delante del rey
Assuero, \bibleverse{11} Que trajesen á la reina Vasthi delante del rey
con la corona regia, para mostrar á los pueblos y á los príncipes su
hermosura; porque era linda de aspecto. \bibleverse{12} Mas la reina
Vasthi no quiso comparecer á la orden del rey, enviada por mano de los
eunucos; y enojóse el rey muy mucho, y encendióse en él su ira.

\hypertarget{asesoramiento-y-toma-de-decisiones-sobre-quuxe9-castigo-anuncio-del-repudio-en-todo-el-imperio}{%
\subsection{Asesoramiento y toma de decisiones sobre qué castigo;
Anuncio del repudio en todo el
imperio}\label{asesoramiento-y-toma-de-decisiones-sobre-quuxe9-castigo-anuncio-del-repudio-en-todo-el-imperio}}

\bibleverse{13} Preguntó entonces el rey á los sabios que sabían los
tiempos, (porque así era la costumbre del rey para con todos los que
sabían la ley y el derecho; \footnote{\textbf{1:13} 1Cró 12,32}
\bibleverse{14} Y estaban junto á él, Carsena, y Sethar, y Admatha, y
Tharsis, y Meres, y Marsena, y Memucán, siete príncipes de Persia y de
Media que veían la cara del rey, y se sentaban los primeros del reino:)
\bibleverse{15} Qué se había de hacer según la ley con la reina Vasthi,
por cuanto no había cumplido la orden del rey Assuero, enviada por mano
de los eunucos.

\bibleverse{16} Y dijo Memucán delante del rey y de los príncipes: No
solamente contra el rey ha pecado la reina Vasthi, sino contra todos los
príncipes, y contra todos los pueblos que hay en todas las provincias
del rey Assuero. \bibleverse{17} Porque este hecho de la reina pasará á
noticia de todas las mujeres, para hacerles tener en poca estima á sus
maridos, diciendo: El rey Assuero mandó traer delante de sí á la reina
Vasthi, y ella no vino. \bibleverse{18} Y entonces dirán esto las
señoras de Persia y de Media que oyeren el hecho de la reina, á todos
los príncipes del rey: y habrá mucho menosprecio y enojo.

\bibleverse{19} Si parece bien al rey, salga mandamiento real delante de
él, y escríbase entre las leyes de Persia y de Media, y no sea
traspasado: Que no venga más Vasthi delante del rey Assuero: y dé el rey
su reino á su compañera que sea mejor que ella. \bibleverse{20} Y el
mandamiento que hará el rey será oído en todo su reino, aunque es
grande, y todas las mujeres darán honra á sus maridos, desde el mayor
hasta el menor.

\bibleverse{21} Y plugo esta palabra en ojos del rey y de los príncipes,
é hizo el rey conforme al dicho de Memucán; \bibleverse{22} Pues envió
letras á todas las provincias del rey, á cada provincia conforme á su
escribir, y á cada pueblo conforme á su lenguaje, diciendo que todo
hombre fuese señor en su casa; y háblese esto según la lengua de su
pueblo. \footnote{\textbf{1:22} Est 3,12; Est 8,9; Gén 3,16}

\hypertarget{organizaciuxf3n-de-un-gran-espectuxe1culo-nupcial-para-el-rey}{%
\subsection{Organización de un gran espectáculo nupcial para el
rey}\label{organizaciuxf3n-de-un-gran-espectuxe1culo-nupcial-para-el-rey}}

\hypertarget{section-1}{%
\section{2}\label{section-1}}

\bibleverse{1} Pasadas estas cosas, sosegada ya la ira del rey Assuero,
acordóse de Vasthi, y de lo que hizo, y de lo que fué sentenciado contra
ella. \bibleverse{2} Y dijeron los criados del rey, sus oficiales:
Busquen al rey mozas vírgenes de buen parecer; \bibleverse{3} Y ponga el
rey personas en todas las provincias de su reino, que junten todas las
mozas vírgenes de buen parecer en Susán residencia regia, en la casa de
las mujeres, al cuidado de Hegai, eunuco del rey, guarda de las mujeres,
dándoles sus atavíos; \bibleverse{4} Y la moza que agradare á los ojos
del rey, reine en lugar de Vasthi. Y la cosa plugo en ojos del rey, é
hízolo así.

\hypertarget{informaciuxf3n-sobre-la-prehistoria-de-esther}{%
\subsection{Información sobre la prehistoria de
Esther}\label{informaciuxf3n-sobre-la-prehistoria-de-esther}}

\bibleverse{5} Había un varón Judío en Susán residencia regia, cuyo
nombre era Mardochêo, hijo de Jair, hijo de Simi, hijo de Cis, del
linaje de Benjamín; \bibleverse{6} El cual había sido trasportado de
Jerusalem con los cautivos que fueron llevados con Jechônías rey de
Judá, á quien hizo trasportar Nabucodonosor rey de Babilonia.
\footnote{\textbf{2:6} 2Re 24,15-16} \bibleverse{7} Y había criado á
Hadassa, que es Esther, hija de su tío, porque no tenía padre ni madre;
y era moza de hermosa forma y de buen parecer; y como su padre y su
madre murieron, Mardochêo la había tomado por hija suya.

\footnote{\textbf{2:7} Est 2,15}

\hypertarget{el-auxf1o-de-preparaciuxf3n-de-ester-en-el-palacio-real-y-su-elevaciuxf3n-a-reina}{%
\subsection{El año de preparación de Ester en el palacio real y su
elevación a
reina}\label{el-auxf1o-de-preparaciuxf3n-de-ester-en-el-palacio-real-y-su-elevaciuxf3n-a-reina}}

\bibleverse{8} Sucedió pues, que como se divulgó el mandamiento del rey
y su acuerdo, y siendo reunidas muchas mozas en Susán residencia regia,
á cargo de Hegai, fué tomada también Esther para casa del rey, al
cuidado de Hegai, guarda de las mujeres. \bibleverse{9} Y la moza agradó
en sus ojos, y halló gracia delante de él; por lo que hizo darle
prestamente sus atavíos y sus raciones, dándole también siete
convenientes doncellas de la casa del rey; y pasóla con sus doncellas á
lo mejor de la casa de las mujeres. \bibleverse{10} Esther no declaró su
pueblo ni su nacimiento; porque Mardochêo le había mandado que no lo
declarase. \bibleverse{11} Y cada día Mardochêo se paseaba delante del
patio de la casa de las mujeres, por saber cómo iba á Esther, y qué se
hacía de ella.

\bibleverse{12} Y como llegaba el tiempo de cada una de las mozas para
venir al rey Assuero, al cabo de haber estado ya doce meses conforme á
la ley acerca de las mujeres (porque así se cumplía el tiempo de sus
atavíos, esto es, seis meses con óleo de mirra, y seis meses con cosas
aromáticas y afeites de mujeres), \bibleverse{13} Entonces la moza venía
así al rey: todo lo que ella decía se le daba, para venir con ello de la
casa de las mujeres hasta la casa del rey. \bibleverse{14} Ella venía á
la tarde, y á la mañana se volvía á la casa segunda de las mujeres, al
cargo de Saasgaz eunuco del rey, guarda de las concubinas: no venía más
al rey, salvo si el rey la quería, y era llamada por nombre.

\bibleverse{15} Y llegado que fué el tiempo de Esther, hija de Abihail
tío de Mardochêo, que él se había tomado por hija, para venir al rey,
ninguna cosa procuró sino lo que dijo Hegai eunuco del rey, guarda de
las mujeres: y ganaba Esther la gracia de todos los que la veían.

\bibleverse{16} Fué pues Esther llevada al rey Assuero á su casa real en
el mes décimo, que es el mes de Tebeth, en el año séptimo de su reinado.
\bibleverse{17} Y el rey amó á Esther sobre todas las mujeres, y halló
gracia y benevolencia delante de él más que todas las vírgenes; y puso
la corona real en su cabeza, é hízola reina en lugar de Vasthi.

\bibleverse{18} Hizo luego el rey gran banquete á todos sus príncipes y
siervos, el banquete de Esther; y alivió á las provincias, é hizo y dió
mercedes conforme á la facultad real.

\hypertarget{mardochai-descubre-una-conspiraciuxf3n-contra-el-rey-su-muxe9rito-estuxe1-registrado-en-las-cruxf3nicas-del-reino}{%
\subsection{Mardochai descubre una conspiración contra el rey; su mérito
está registrado en las crónicas del
reino}\label{mardochai-descubre-una-conspiraciuxf3n-contra-el-rey-su-muxe9rito-estuxe1-registrado-en-las-cruxf3nicas-del-reino}}

\bibleverse{19} Y cuando se juntaban las vírgenes la segunda vez,
Mardochêo estaba puesto á la puerta del rey. \bibleverse{20} Y Esther,
según le tenía mandado Mardochêo, no había declarado su nación ni su
pueblo; porque Esther hacía lo que decía Mardochêo, como cuando con él
se educaba. \footnote{\textbf{2:20} Est 2,10} \bibleverse{21} En
aquellos días, estando Mardochêo sentado á la puerta del rey, enojáronse
Bigthán y Teres, dos eunucos del rey, de la guardia de la puerta, y
procuraban poner mano en el rey Assuero. \bibleverse{22} Mas entendido
que fué esto por Mardochêo, él lo denunció á la reina Esther, y Esther
lo dijo al rey en nombre de Mardochêo. \bibleverse{23} Hízose entonces
indagación de la cosa, y fué hallada cierta; por tanto, entrambos fueron
colgados en una horca. Y escribióse el caso en el libro de las cosas de
los tiempos delante del rey.

\hypertarget{promociuxf3n-de-amuxe1n-al-muxe1s-alto-honor-mardochai-se-niega-a-doblar-las-rodillas-amuxe1n-decide-exterminar-a-todos-los-juduxedos}{%
\subsection{Promoción de Amán al más alto honor; Mardochai se niega a
doblar las rodillas; Amán decide exterminar a todos los
judíos}\label{promociuxf3n-de-amuxe1n-al-muxe1s-alto-honor-mardochai-se-niega-a-doblar-las-rodillas-amuxe1n-decide-exterminar-a-todos-los-juduxedos}}

\hypertarget{section-2}{%
\section{3}\label{section-2}}

\bibleverse{1} Después de estas cosas, el rey Assuero engrandeció á Amán
hijo de Amadatha Agageo, y ensalzólo, y puso su silla sobre todos los
príncipes que estaban con él. \bibleverse{2} Y todos los siervos del rey
que estaban á la puerta del rey, se arrodillaban é inclinaban á Amán,
porque así se lo había mandado el rey; pero Mardochêo, ni se arrodillaba
ni se humillaba. \bibleverse{3} Y los siervos del rey que estaban á la
puerta, dijeron á Mardochêo: ¿Por qué traspasas el mandamiento del rey?
\bibleverse{4} Y aconteció que, hablándole cada día de esta manera, y no
escuchándolos él, denunciáronlo á Amán, por ver si las palabras de
Mardochêo se mantendrían; porque ya él les había declarado que era
Judío. \bibleverse{5} Y vió Amán que Mardochêo ni se arrodillaba ni se
humillaba delante de él; y llenóse de ira. \bibleverse{6} Mas tuvo en
poco meter mano en solo Mardochêo; que ya le habían declarado el pueblo
de Mardochêo: y procuró Amán destruir á todos los Judíos que había en el
reino de Assuero, al pueblo de Mardochêo.

\hypertarget{amuxe1n-hace-cumplir-su-resoluciuxf3n-con-el-rey}{%
\subsection{Amán hace cumplir su resolución con el
rey}\label{amuxe1n-hace-cumplir-su-resoluciuxf3n-con-el-rey}}

\bibleverse{7} En el mes primero, que es el mes de Nisán, en el año
duodécimo del rey Assuero, fué echada Pur, esto es, la suerte, delante
de Amán, de día en día y de mes en mes; y salió el mes duodécimo, que es
el mes de Adar. \footnote{\textbf{3:7} Est 9,24} \bibleverse{8} Y dijo
Amán al rey Assuero: Hay un pueblo esparcido y dividido entre los
pueblos en todas las provincias de tu reino, y sus leyes son diferentes
de las de todo pueblo, y no observan las leyes del rey; y al rey no
viene provecho de dejarlos. \bibleverse{9} Si place al rey, escríbase
que sean destruídos; y yo pesaré diez mil talentos de plata en manos de
los que manejan la hacienda, para que sean traídos á los tesoros del
rey.

\bibleverse{10} Entonces el rey quitó su anillo de su mano, y diólo á
Amán hijo de Amadatha Agageo, enemigo de los Judíos, \bibleverse{11} Y
díjole: La plata propuesta sea para ti, y asimismo el pueblo, para que
hagas de él lo que bien te pareciere.

\hypertarget{el-exterminio-de-los-juduxedos-en-todo-el-imperio-ordenado-por-el-rey}{%
\subsection{El exterminio de los judíos en todo el imperio ordenado por
el
rey}\label{el-exterminio-de-los-juduxedos-en-todo-el-imperio-ordenado-por-el-rey}}

\bibleverse{12} Entonces fueron llamados los escribanos del rey en el
mes primero, á trece del mismo, y fué escrito conforme á todo lo que
mandó Amán, á los príncipes del rey, y á los capitanes que estaban sobre
cada provincia, y á los príncipes de cada pueblo, á cada provincia según
su escritura, y á cada pueblo según su lengua: en nombre del rey Assuero
fué escrito, y signado con el anillo del rey. \footnote{\textbf{3:12}
  Est 1,22} \bibleverse{13} Y fueron enviadas letras por mano de los
correos á todas las provincias del rey, para destruir, y matar, y
exterminar á todos los Judíos, desde el niño hasta el viejo, niños y
mujeres en un día, en el trece del mes duodécimo, que es el mes de Adar,
y para apoderarse de su despojo. \bibleverse{14} La copia del escrito
que se diese por mandamiento en cada provincia, fué publicada á todos
los pueblos, á fin de que estuviesen apercibidos para aquel día.
\bibleverse{15} Y salieron los correos de priesa por mandato del rey, y
el edicto fué dado en Susán capital del reino. Y el rey y Amán estaban
sentados á beber, y la ciudad de Susán estaba conmovida.

\hypertarget{el-dolor-de-mardochai-sus-esfuerzos-para-mover-a-ester-a-salvar-a-los-juduxedos}{%
\subsection{El dolor de Mardochai; sus esfuerzos para mover a Ester a
salvar a los
judíos}\label{el-dolor-de-mardochai-sus-esfuerzos-para-mover-a-ester-a-salvar-a-los-juduxedos}}

\hypertarget{section-3}{%
\section{4}\label{section-3}}

\bibleverse{1} Luego que supo Mardochêo todo lo que se había hecho,
rasgó sus vestidos, y vistióse de saco y de ceniza, y fuése por medio de
la ciudad clamando con grande y amargo clamor. \bibleverse{2} Y vino
hasta delante de la puerta del rey: porque no era lícito pasar adentro
de la puerta del rey con vestido de saco. \bibleverse{3} Y en cada
provincia y lugar donde el mandamiento del rey y su decreto llegaba,
tenían los Judíos grande luto, y ayuno, y lloro, y lamentación: saco y
ceniza era la cama de muchos.

\hypertarget{ester-es-informada-por-mardochai-sobre-el-desastre-inminente-y-le-pide-que-ruegue-al-rey-por-misericordia}{%
\subsection{Ester es informada por Mardochai sobre el desastre inminente
y le pide que ruegue al rey por
misericordia}\label{ester-es-informada-por-mardochai-sobre-el-desastre-inminente-y-le-pide-que-ruegue-al-rey-por-misericordia}}

\bibleverse{4} Y vinieron las doncellas de Esther y sus eunucos, y
dijéronselo: y la reina tuvo gran dolor, y envió vestidos para hacer
vestir á Mardochêo, y hacerle quitar el saco de sobre él; mas él no los
recibió. \bibleverse{5} Entonces Esther llamó á Atach, uno de los
eunucos del rey, que él había hecho estar delante de ella, y mandólo á
Mardochêo, con orden de saber qué era aquello, y por qué. \bibleverse{6}
Salió pues Atach á Mardochêo, á la plaza de la ciudad que estaba delante
de la puerta del rey. \bibleverse{7} Y Mardochêo le declaró todo lo que
le había acontecido, y dióle noticia de la plata que Amán había dicho
que pesaría para los tesoros del rey por razón de los Judíos, para
destruirlos. \bibleverse{8} Dióle también la copia de la escritura del
decreto que había sido dado en Susán para que fuesen destruídos, á fin
de que la mostrara á Esther y se lo declarase, y le encargara que fuese
al rey á suplicarle, y á pedir delante de él por su pueblo.

\hypertarget{la-negativa-de-esther-es-derrotada-por-mardochai-sin-embargo-requiere-que-los-juduxedos-mantengan-un-ayuno-estricto-a-su-favor.}{%
\subsection{La negativa de Esther es derrotada por Mardochai; Sin
embargo, requiere que los judíos mantengan un ayuno estricto a su
favor.}\label{la-negativa-de-esther-es-derrotada-por-mardochai-sin-embargo-requiere-que-los-juduxedos-mantengan-un-ayuno-estricto-a-su-favor.}}

\bibleverse{9} Y vino Atach, y contó á Esther las palabras de Mardochêo.
\bibleverse{10} Entonces Esther dijo á Atach, y mandóle decir á
Mardochêo: \bibleverse{11} Todos los siervos del rey, y el pueblo de las
provincias del rey saben, que cualquier hombre ó mujer que entra al rey
al patio de adentro sin ser llamado, por una sola ley ha de morir: salvo
aquel á quien el rey extendiere el cetro de oro, el cual vivirá: y yo no
he sido llamada para entrar al rey estos treinta días. \footnote{\textbf{4:11}
  Est 5,2; Est 8,4}

\bibleverse{12} Y dijeron á Mardochêo las palabras de Esther.
\bibleverse{13} Entonces dijo Mardochêo que respondiesen á Esther: No
pienses en tu alma, que escaparás en la casa del rey más que todos los
Judíos: \bibleverse{14} Porque si absolutamente callares en este tiempo,
respiro y libertación tendrán los Judíos de otra parte; mas tú y la casa
de tu padre pereceréis. ¿Y quién sabe si para esta hora te han hecho
llegar al reino?

\bibleverse{15} Y Esther dijo que respondiesen á Mardochêo:
\bibleverse{16} Ve, y junta á todos los Judíos que se hallan en Susán, y
ayunad por mí, y no comáis ni bebáis en tres días, noche ni día: yo
también con mis doncellas ayunaré igualmente, y así entraré al rey,
aunque no sea conforme á la ley; y si perezco, que perezca. \footnote{\textbf{4:16}
  2Re 7,4} \bibleverse{17} Entonces se fué Mardochêo, é hizo conforme á
todo lo que le mandó Esther.

\hypertarget{la-recepciuxf3n-amistosa-de-ester-por-parte-del-rey-y-el-engauxf1o-de-amuxe1n}{%
\subsection{La recepción amistosa de Ester por parte del rey y el engaño
de
Amán}\label{la-recepciuxf3n-amistosa-de-ester-por-parte-del-rey-y-el-engauxf1o-de-amuxe1n}}

\hypertarget{section-4}{%
\section{5}\label{section-4}}

\bibleverse{1} Y aconteció que al tercer día se vistió Esther su vestido
real, y púsose en el patio de adentro de la casa del rey, enfrente del
aposento del rey: y estaba el rey sentado en su solio regio en el
aposento real, enfrente de la puerta del aposento. \bibleverse{2} Y fué
que, como vió á la reina Esther que estaba en el patio, ella obtuvo
gracia en sus ojos; y el rey extendió á Esther el cetro de oro que tenía
en la mano. Entonces se llegó Esther, y tocó la punta del cetro.

\bibleverse{3} Y dijo el rey: ¿Qué tienes, reina Esther? ¿y cuál es tu
petición? Hasta la mitad del reino, se te dará.

\bibleverse{4} Y Esther dijo: Si al rey place, venga hoy el rey con Amán
al banquete que le he hecho.

\footnote{\textbf{5:4} Est 1,19}

\hypertarget{el-rey-invitado-por-ester-a-cenar-con-amuxe1n-acepta-otra-invitaciuxf3n-a-cenar}{%
\subsection{El rey, invitado por Ester a cenar con Amán, acepta otra
invitación a
cenar}\label{el-rey-invitado-por-ester-a-cenar-con-amuxe1n-acepta-otra-invitaciuxf3n-a-cenar}}

\bibleverse{5} Y respondió el rey: Daos priesa, llamad á Amán, para
hacer lo que Esther ha dicho. Vino pues el rey con Amán al banquete que
Esther dispuso.

\bibleverse{6} Y dijo el rey á Esther en el banquete del vino: ¿Cuál es
tu petición, y te será otorgada? ¿Cuál es tu demanda? Aunque sea la
mitad del reino, te será concedida.

\bibleverse{7} Entonces respondió Esther, y dijo: Mi petición y mi
demanda es: \bibleverse{8} Si he hallado gracia en los ojos del rey, y
si place al rey otorgar mi petición y hacer mi demanda, que venga el rey
con Amán al banquete que les dispondré; y mañana haré conforme á lo que
el rey ha mandado.

\hypertarget{el-altivo-engauxf1o-de-amuxe1n-su-intenciuxf3n-de-deshacerse-de-mardochai}{%
\subsection{El altivo engaño de Amán; su intención de deshacerse de
Mardochai}\label{el-altivo-engauxf1o-de-amuxe1n-su-intenciuxf3n-de-deshacerse-de-mardochai}}

\bibleverse{9} Y salió Amán aquel día contento y alegre de corazón; pero
como vió á Mardochêo á la puerta del rey, que no se levantaba ni se
movía de su lugar, llenóse contra Mardochêo de ira. \bibleverse{10} Mas
refrenóse Amán, y vino á su casa, y envió, é hizo venir sus amigos, y á
Zeres su mujer. \bibleverse{11} Y refirióles Amán la gloria de sus
riquezas, y la multitud de sus hijos, y todas las cosas con que el rey
le había engrandecido y con que le había ensalzado sobre los príncipes y
siervos del rey.

\bibleverse{12} Y añadió Amán: También la reina Esther á ninguno hizo
venir con el rey al banquete que ella dispuso, sino á mí: y aun para
mañana soy convidado de ella con el rey. \bibleverse{13} Mas todo esto
nada me sirve cada vez que veo al judío Mardochêo sentado á la puerta
del rey.

\bibleverse{14} Y díjole Zeres su mujer, y todos sus amigos: Hagan una
horca alta de cincuenta codos, y mañana di al rey que cuelguen á
Mardochêo en ella; y entra con el rey al banquete alegre. Y plugo la
cosa en los ojos de Amán, é hizo preparar la horca.

\hypertarget{mardochai-criado-en-alto-honor-por-amuxe1n}{%
\subsection{Mardochai criado en alto honor por
Amán}\label{mardochai-criado-en-alto-honor-por-amuxe1n}}

\hypertarget{section-5}{%
\section{6}\label{section-5}}

\bibleverse{1} Aquella noche se le fué el sueño al rey, y dijo que le
trajesen el libro de las memorias de las cosas de los tiempos: y
leyéronlas delante del rey. \bibleverse{2} Y hallóse escrito que
Mardochêo había denunciado de Bigthan y de Teres, dos eunucos del rey,
de la guarda de la puerta, que habían procurado meter mano en el rey
Assuero. \footnote{\textbf{6:2} Est 2,21-23} \bibleverse{3} Y dijo el
rey: ¿Qué honra ó que distinción se hizo á Mardochêo por esto? Y
respondieron los servidores del rey, sus oficiales: Nada se ha hecho con
él.

\bibleverse{4} Entonces dijo el rey: ¿Quién está en el patio? Y Amán
había venido al patio de afuera de la casa del rey, para decir al rey
que hiciese colgar á Mardochêo en la horca que él le tenía preparada.

\hypertarget{amuxe1n-involuntariamente-hace-que-el-rey-decida-sobre-un-honor-extraordinario-para-mardochai-y-que-lo-lleve-a-cabo-personalmente.}{%
\subsection{Amán involuntariamente hace que el rey decida sobre un honor
extraordinario para Mardochai y que lo lleve a cabo
personalmente.}\label{amuxe1n-involuntariamente-hace-que-el-rey-decida-sobre-un-honor-extraordinario-para-mardochai-y-que-lo-lleve-a-cabo-personalmente.}}

\bibleverse{5} Y los servidores del rey le respondieron: He aquí Amán
está en el patio. Y el rey dijo: Entre.

\bibleverse{6} Entró pues Amán, y el rey le dijo: ¿Qué se hará al hombre
cuya honra desea el rey? Y dijo Amán en su corazón: ¿A quién deseará el
rey hacer honra más que á mí?

\bibleverse{7} Y respondió Amán al rey: Al varón cuya honra desea el
rey, \bibleverse{8} Traigan el vestido real de que el rey se viste, y el
caballo en que el rey cabalga, y la corona real que está puesta en su
cabeza; \bibleverse{9} Y den el vestido y el caballo en mano de alguno
de los príncipes más nobles del rey, y vistan á aquel varón cuya honra
desea el rey, y llévenlo en el caballo por la plaza de la ciudad, y
pregonen delante de él: Así se hará al varón cuya honra desea el rey.

\bibleverse{10} Entonces el rey dijo á Amán: Date priesa, toma el
vestido y el caballo, como tú has dicho, y hazlo así con el judío
Mardochêo, que se sienta á la puerta del rey; no omitas nada de todo lo
que has dicho.

\bibleverse{11} Y Amán tomó el vestido y el caballo, y vistió á
Mardochêo, y llevólo á caballo por la plaza de la ciudad, é hizo
pregonar delante de él: Así se hará al varón cuya honra desea el rey.

\hypertarget{el-dolor-de-amuxe1n-lleno-de-presentimientos-fue-al-banquete-de-la-reina}{%
\subsection{El dolor de Amán; Lleno de presentimientos, fue al banquete
de la
reina}\label{el-dolor-de-amuxe1n-lleno-de-presentimientos-fue-al-banquete-de-la-reina}}

\bibleverse{12} Después de esto Mardochêo se volvió á la puerta del rey,
y Amán se fué corriendo á su casa, apesadumbrado y cubierta su cabeza.
\bibleverse{13} Contó luego Amán á Zeres su mujer, y á todos sus amigos,
todo lo que le había acontecido: y dijéronle sus sabios, y Zeres su
mujer: Si de la simiente de los Judíos es el Mardochêo, delante de quien
has comenzado á caer, no lo vencerás; antes caerás por cierto delante de
él. \bibleverse{14} Aun estaban ellos hablando con él, cuando los
eunucos del rey llegaron apresurados, para hacer venir á Amán al
banquete que Esther había dispuesto. \footnote{\textbf{6:14} Est 5,8}

\hypertarget{durante-la-cena-ester-revela-los-planes-de-amuxe1n-de-matar-al-rey-el-rey-se-levanta-enojado-de-la-cena}{%
\subsection{Durante la cena, Ester revela los planes de Amán de matar al
rey; el rey se levanta enojado de la
cena}\label{durante-la-cena-ester-revela-los-planes-de-amuxe1n-de-matar-al-rey-el-rey-se-levanta-enojado-de-la-cena}}

\hypertarget{section-6}{%
\section{7}\label{section-6}}

\bibleverse{1} Vino pues el rey con Amán á beber con la reina Esther.
\footnote{\textbf{7:1} Est 5,8; Est 6,14} \bibleverse{2} Y también el
segundo día dijo el rey á Esther en el convite del vino: ¿Cuál es tu
petición, reina Esther, y se te concederá? ¿Cuál es pues tu demanda?
Aunque sea la mitad del reino, pondráse por obra.

\bibleverse{3} Entonces la reina Esther respondió y dijo: Oh rey, si he
hallado gracia en tus ojos, y si al rey place, séame dada mi vida por mi
petición, y mi pueblo por mi demanda. \bibleverse{4} Porque vendidos
estamos yo y mi pueblo, para ser destruídos, para ser muertos y
exterminados. Y si para siervos y siervas fuéramos vendidos, callárame,
bien que el enemigo no compensara el daño del rey.

\bibleverse{5} Y respondió el rey Assuero, y dijo á la reina Esther:
¿Quién es, y dónde está, aquél á quien ha henchido su corazón para obrar
así?

\bibleverse{6} Y Esther dijo: El enemigo y adversario es este malvado
Amán. Entonces se turbó Amán delante del rey y de la reina.

\bibleverse{7} Levantóse luego el rey del banquete del vino en su furor,
y se fué al huerto del palacio: y quedóse Amán para procurar de la reina
Esther por su vida; porque vió que estaba resuelto para él el mal de
parte del rey.

\hypertarget{a-su-regreso-el-rey-condenuxf3-a-muerte-a-amuxe1n-e-inmediatamente-lo-hizo-colgar-en-la-estaca-erigida-para-mardochai}{%
\subsection{A su regreso, el rey condenó a muerte a Amán e
inmediatamente lo hizo colgar en la estaca erigida para
Mardochai}\label{a-su-regreso-el-rey-condenuxf3-a-muerte-a-amuxe1n-e-inmediatamente-lo-hizo-colgar-en-la-estaca-erigida-para-mardochai}}

\bibleverse{8} Volvió después el rey del huerto del palacio al aposento
del banquete del vino, y Amán había caído sobre el lecho en que estaba
Esther. Entonces dijo el rey: ¿También para forzar la reina, estando
conmigo en casa? Como esta palabra salió de la boca del rey, el rostro
de Amán fué cubierto.

\bibleverse{9} Y dijo Harbona, uno de los eunucos de delante del rey: He
aquí también la horca de cincuenta codos de altura que hizo Amán para
Mardochêo, el cual había hablado bien por el rey, está en casa de Amán.
Entonces el rey dijo: Colgadlo en ella.

\bibleverse{10} Así colgaron á Amán en la horca que él había hecho
aparejar para Mardochêo; y apaciguóse la ira del rey.

\hypertarget{el-regalo-de-ester-y-la-exaltaciuxf3n-de-mardochai-por-parte-del-rey}{%
\subsection{El regalo de Ester y la exaltación de Mardochai por parte
del
rey}\label{el-regalo-de-ester-y-la-exaltaciuxf3n-de-mardochai-por-parte-del-rey}}

\hypertarget{section-7}{%
\section{8}\label{section-7}}

\bibleverse{1} EL mismo día dió el rey Assuero á la reina Esther la casa
de Amán enemigo de los Judíos; y Mardochêo vino delante del rey, porque
Esther le declaró lo que era respecto de ella. \bibleverse{2} Y quitóse
el rey su anillo que había vuelto á tomar de Amán, y diólo á Mardochêo.
Y Esther puso á Mardochêo sobre la casa de Amán.

\footnote{\textbf{8:2} Est 3,10}

\hypertarget{establecer-y-promulgar-medidas-de-protecciuxf3n-para-los-juduxedos-contra-sus-enemigos}{%
\subsection{Establecer y promulgar medidas de protección para los judíos
contra sus
enemigos}\label{establecer-y-promulgar-medidas-de-protecciuxf3n-para-los-juduxedos-contra-sus-enemigos}}

\bibleverse{3} Volvió luego Esther á hablar delante del rey, y echóse á
sus pies, llorando y rogándole que hiciese nula la maldad de Amán
Agageo, y su designio que había formado contra los Judíos.
\bibleverse{4} Entonces extendió el rey á Esther el cetro de oro, y
Esther se levantó, y púsose en pie delante del rey. \bibleverse{5} Y
dijo: Si place al rey, y si he hallado gracia delante de él, y si la
cosa es recta delante del rey, y agradable yo en sus ojos, sea escrito
para revocar las letras del designio de Amán hijo de Amadatha Agageo,
que escribió para destruir á los Judíos que están en todas las
provincias del rey. \bibleverse{6} Porque ¿cómo podré yo ver el mal que
alcanzará á mi pueblo? ¿cómo podré yo ver la destrucción de mi nación?

\bibleverse{7} Y respondió el rey Assuero á la reina Esther, y á
Mardochêo Judío: He aquí yo he dado á Esther la casa de Amán, y á él han
colgado en la horca, por cuanto extendió su mano contra los Judíos.
\bibleverse{8} Escribid pues vosotros á los Judíos como bien os
pareciere en el nombre del rey, y selladlo con el anillo del rey; porque
la escritura que se escribe en nombre del rey, y se sella con el anillo
del rey, no es para revocarla.

\bibleverse{9} Entonces fueron llamados los escribanos del rey en el mes
tercero, que es Siván, á veintitrés del mismo; y escribióse conforme á
todo lo que mandó Mardochêo, á los Judíos, y á los sátrapas, y á los
capitanes, y á los príncipes de las provincias que había desde la India
hasta la Ethiopía, ciento veintisiete provincias; á cada provincia según
su escribir, y á cada pueblo conforme á su lengua, á los Judíos también
conforme á su escritura y lengua. \bibleverse{10} Y escribió en nombre
del rey Assuero, y selló con el anillo del rey, y envió letras por
correos de á caballo, montados en dromedarios, y en mulos hijos de
yeguas; \bibleverse{11} Con intimación de que el rey concedía á los
Judíos que estaban en todas las ciudades, que se juntasen y estuviesen á
la defensa de su vida, prontos á destruir, y matar, y acabar con todo
ejército de pueblo ó provincia que viniese contra ellos, aun niños y
mujeres, y su despojo para presa, \bibleverse{12} En un mismo día en
todas las provincias del rey Assuero, en el trece del mes duodécimo, que
es el mes de Adar. \bibleverse{13} La copia de la escritura que había de
darse por ordenanza en cada provincia, para que fuese manifiesta á todos
los pueblos, decía que los Judíos estuviesen apercibidos para aquel día,
para vengarse de sus enemigos. \bibleverse{14} Los correos pues,
cabalgando en dromedarios y en mulos, salieron apresurados y
constreñidos por el mandamiento del rey: y la ley fué dada en Susán
capital del reino.

\hypertarget{mardochai-aparece-en-susa-con-un-traje-principesco-alegruxeda-de-los-juduxedos-en-todo-el-imperio}{%
\subsection{Mardochai aparece en Susa con un traje principesco; Alegría
de los judíos en todo el
imperio}\label{mardochai-aparece-en-susa-con-un-traje-principesco-alegruxeda-de-los-juduxedos-en-todo-el-imperio}}

\bibleverse{15} Y salió Mardochêo de delante del rey con vestido real de
cárdeno y blanco, y una gran corona de oro, y un manto de lino y
púrpura: y la ciudad de Susán se alegró y regocijó. \bibleverse{16} Los
Judíos tuvieron luz y alegría, y gozo y honra. \bibleverse{17} Y en cada
provincia y en cada ciudad donde llegó el mandamiento del rey, los
Judíos tuvieron alegría y gozo, banquete y día de placer. Y muchos de
los pueblos de la tierra se hacían Judíos, porque el temor de los Judíos
había caído sobre ellos. \footnote{\textbf{8:17} Éxod 15,14-16}

\hypertarget{exterminio-de-enemigos-de-los-juduxedos-en-todo-el-imperio-el-duxeda-13-del-mes-de-adar}{%
\subsection{Exterminio de enemigos de los judíos en todo el imperio el
día 13 del mes de
Adar}\label{exterminio-de-enemigos-de-los-juduxedos-en-todo-el-imperio-el-duxeda-13-del-mes-de-adar}}

\hypertarget{section-8}{%
\section{9}\label{section-8}}

\bibleverse{1} Y en el mes duodécimo, que es el mes de Adar, á trece del
mismo, en el que tocaba se ejecutase el mandamiento del rey y su ley, el
mismo día en que esperaban los enemigos de los Judíos enseñorearse de
ellos, fué lo contrario; porque los Judíos se enseñorearon de los que
los aborrecían. \bibleverse{2} Los Judíos se juntaron en sus ciudades en
todas las provincias del rey Assuero, para meter mano sobre los que
habían procurado su mal: y nadie se puso delante de ellos, porque el
temor de ellos había caído sobre todos los pueblos. \bibleverse{3} Y
todos los príncipes de las provincias, y los virreyes, y capitanes, y
oficiales del rey, ensalzaban á los Judíos; porque el temor de Mardochêo
había caído sobre ellos. \bibleverse{4} Porque Mardochêo era grande en
la casa del rey, y su fama iba por todas las provincias; pues el varón
Mardochêo iba engrandeciéndose. \bibleverse{5} E hirieron los Judíos á
todos sus enemigos con plaga de espada, y de mortandad, y de perdición;
é hicieron en sus enemigos á su voluntad. \bibleverse{6} Y en Susán
capital del reino, mataron y destruyeron los Judíos á quinientos
hombres. \bibleverse{7} Mataron entonces á Phorsandatha, y á Dalphón, y
á Asphatha, \bibleverse{8} Y á Phoratha y á Ahalía, y á Aridatha,
\bibleverse{9} Y á Pharmastha, y á Arisai, y á Aridai, y á Vaizatha,
\bibleverse{10} Diez hijos de Amán hijo de Amadatha, enemigo de los
Judíos: mas en la presa no metieron su mano.

\hypertarget{continuaciuxf3n-de-la-matanza-el-duxeda-14-del-mes-regocijo-de-los-juduxedos-para-celebrar-su-salvaciuxf3n}{%
\subsection{Continuación de la matanza el día 14 del mes; Regocijo de
los judíos para celebrar su
salvación}\label{continuaciuxf3n-de-la-matanza-el-duxeda-14-del-mes-regocijo-de-los-juduxedos-para-celebrar-su-salvaciuxf3n}}

\bibleverse{11} El mismo día vino la cuenta de los muertos en Susán
residencia regia, delante del rey. \bibleverse{12} Y dijo el rey á la
reina Esther: En Susán, capital del reino, han muerto los Judíos y
destruído á quinientos hombres, y á diez hijos de Amán; ¿qué habrán
hecho en las otras provincias del rey? ¿Cuál pues es tu petición, y te
será concedida? ¿ó qué más es tu demanda, y será hecho? \footnote{\textbf{9:12}
  Est 5,6; Est 7,2}

\bibleverse{13} Y respondió Esther: Si place al rey, concédase también
mañana á los Judíos en Susán, que hagan conforme á la ley de hoy; y que
cuelguen en la horca á los diez hijos de Amán.

\bibleverse{14} Y mandó el rey que se hiciese así: y dióse la orden en
Susán, y colgaron á los diez hijos de Amán. \bibleverse{15} Y los Judíos
que estaban en Susán, se juntaron también el catorce del mes de Adar, y
mataron en Susán trescientos hombres: mas en la presa no metieron su
mano.

\bibleverse{16} En cuanto á los otros Judíos que estaban en las
provincias del rey, también se juntaron y pusiéronse en defensa de su
vida, y tuvieron reposo de sus enemigos, y mataron de sus contrarios
setenta y cinco mil; mas en la presa no metieron su mano.
\bibleverse{17} En el día trece del mes de Adar fué esto; y reposaron en
el día catorce del mismo, é hiciéronlo día de banquete y de alegría.

\bibleverse{18} Mas los Judíos que estaban en Susán se juntaron en el
trece y en el catorce del mismo mes; y al quince del mismo reposaron, é
hicieron aquel día día de banquete y de regocijo. \bibleverse{19} Por
tanto los Judíos aldeanos que habitan en las villas sin muro, hacen á
los catorce del mes de Adar el día de alegría y de banquete, y buen día,
y de enviar porciones cada uno á su vecino.

\hypertarget{mardochai-ordena-la-celebraciuxf3n-de-la-fiesta-de-purim-para-todos-los-futuros}{%
\subsection{Mardochai ordena la celebración de la fiesta de Purim para
todos los
futuros}\label{mardochai-ordena-la-celebraciuxf3n-de-la-fiesta-de-purim-para-todos-los-futuros}}

\bibleverse{20} Y escribió Mardochêo estas cosas, y envió letras á todos
los Judíos que estaban en todas las provincias del rey Assuero, cercanos
y distantes, \bibleverse{21} Ordenándoles que celebrasen el día
décimocuarto del mes de Adar, y el décimoquinto del mismo, cada un año,
\bibleverse{22} Como días en que los Judíos tuvieron reposo de sus
enemigos, y el mes que se les tornó de tristeza en alegría, y de luto en
día bueno; que los hiciesen días de banquete y de gozo, y de enviar
porciones cada uno á su vecino, y dádivas á los pobres. \bibleverse{23}
Y los Judíos aceptaron hacer, según habían comenzado, lo que les
escribió Mardochêo. \bibleverse{24} Porque Amán hijo de Amadatha,
Agageo, enemigo de todos los Judíos, había ideado contra los Judíos para
destruirlos, y echó Pur, que quiere decir suerte, para consumirlos y
acabar con ellos. \bibleverse{25} Mas como Esther vino á la presencia
del rey, él intimó por carta: El perverso designio que aquél trazó
contra los Judíos, recaiga sobre su cabeza; y cuélguenlo á él y á sus
hijos en la horca. \footnote{\textbf{9:25} Est 9,14; Est 7,10}

\bibleverse{26} Por esto llamaron á estos días Purim, del nombre Pur.
Por todas las palabras pues de esta carta, y por lo que ellos vieron
sobre esto, y lo que llegó á su noticia, \bibleverse{27} Establecieron y
tomaron los Judíos sobre sí, y sobre su simiente, y sobre todos los
allegados á ellos, y no será traspasado, el celebrar estos dos días
según está escrito en orden á ellos, y conforme á su tiempo cada un año;
\bibleverse{28} Y que estos dos días serían en memoria, y celebrados en
todas las naciones, y familias, y provincias, y ciudades. Estos días de
Purim no pasarán de entre los Judíos, y la memoria de ellos no cesará de
su simiente. \bibleverse{29} Y la reina Esther hija de Abihail, y
Mardochêo Judío, escribieron con toda eficacia, para confirmar esta
segunda carta de Purim. \bibleverse{30} Y envió Mardochêo letras á todos
los Judíos, á las ciento veintisiete provincias del rey Assuero, con
palabras de paz y de verdad, \bibleverse{31} Para confirmar estos días
de Purim en sus tiempos señalados, según les había constituído Mardochêo
Judío y la reina Esther, y como habían ellos tomado sobre sí y sobre su
simiente, para conmemorar el fin de los ayunos y de su clamor.
\bibleverse{32} Y el mandamiento de Esther confirmó estas palabras dadas
acerca de Purim, y escribióse en el libro.

\hypertarget{posiciuxf3n-de-poder-y-servicios-de-mardochai-para-el-bienestar-de-los-juduxedos}{%
\subsection{Posición de poder y servicios de Mardochai para el bienestar
de los
judíos}\label{posiciuxf3n-de-poder-y-servicios-de-mardochai-para-el-bienestar-de-los-juduxedos}}

\hypertarget{section-9}{%
\section{10}\label{section-9}}

\bibleverse{1} Y el rey Assuero impuso tributo sobre la tierra y las
islas de la mar. \bibleverse{2} Y toda la obra de su fortaleza, y de su
valor, y la declaración de la grandeza de Mardochêo, con que el rey le
engrandeció, ¿no está escrito en el libro de los anales de los reyes de
Media y de Persia? \bibleverse{3} Porque Mardochêo Judío fué segundo
después del rey Assuero, y grande entre los Judíos, y acepto á la
multitud de sus hermanos, procurando el bien de su pueblo, y hablando
paz para toda su simiente.
