\hypertarget{las-ordenanzas-finales-de-jesuxfas-y-su-promesa-a-los-discuxedpulos-ascensiuxf3n}{%
\subsection{Las ordenanzas finales de Jesús y su promesa a los
discípulos;
Ascensión}\label{las-ordenanzas-finales-de-jesuxfas-y-su-promesa-a-los-discuxedpulos-ascensiuxf3n}}

\hypertarget{section}{%
\section{1}\label{section}}

\bibverse{1} En el primer tratado, oh Teófilo, he hablado de todas las
cosas que Jesús comenzó á hacer y á enseñar, \footnote{\textbf{1:1} Luc
  1,3} \bibverse{2} Hasta el día en que, habiendo dado mandamientos por
el Espíritu Santo á los apóstoles que escogió, fué recibido arriba;
\footnote{\textbf{1:2} Mat 28,19-20} \bibverse{3} A los cuales, después
de haber padecido, se presentó vivo con muchas pruebas indubitables,
apareciéndoles por cuarenta días, y hablándoles del reino de Dios.
\bibverse{4} Y estando juntos, les mandó que no se fuesen de Jerusalem,
sino que esperasen la promesa del Padre, que oísteis, dijo, de mí.
\footnote{\textbf{1:4} Juan 15,26; Luc 24,49} \bibverse{5} Porque Juan á
la verdad bautizó con agua, mas vosotros seréis bautizados con el
Espíritu Santo no muchos días después de estos. \footnote{\textbf{1:5}
  Mat 3,11}

\bibverse{6} Entonces los que se habían juntado le preguntaron,
diciendo: Señor, ¿restituirás el reino á Israel en este tiempo?
\footnote{\textbf{1:6} Luc 19,11; Luc 24,21}

\bibverse{7} Y les dijo: No toca á vosotros saber los tiempos ó las
sazones que el Padre puso en su sola potestad; \footnote{\textbf{1:7}
  Mat 24,36} \bibverse{8} Mas recibiréis la virtud del Espíritu Santo
que vendrá sobre vosotros; y me sereís testigos en Jerusalem, y en toda
Judea, y Samaria, y hasta lo último de la tierra. \footnote{\textbf{1:8}
  Hech 8,11; Luc 24,48}

\bibverse{9} Y habiendo dicho estas cosas, viéndolo ellos, fué alzado; y
una nube le recibió y le quitó de sus ojos. \footnote{\textbf{1:9} Mar
  16,19; Luc 24,51} \bibverse{10} Y estando con los ojos puestos en el
cielo, entre tanto que él iba, he aquí dos varones se pusieron junto á
ellos en vestidos blancos; \footnote{\textbf{1:10} Luc 24,4}
\bibverse{11} Los cuales también les dijeron: Varones Galileos, ¿qué
estáis mirando al cielo? este mismo Jesús que ha sido tomado desde
vosotros arriba en el cielo, así vendrá como le habéis visto ir al
cielo. \footnote{\textbf{1:11} Luc 21,27}

\bibverse{12} Entonces se volvieron á Jerusalem del monte que se llama
del Olivar, el cual está cerca de Jerusalem camino de un sábado.
\footnote{\textbf{1:12} Luc 24,50; Luc 24,52-53} \bibverse{13} Y
entrados, subieron al aposento alto, donde moraban Pedro y Jacobo, y
Juan y Andrés, Felipe y Tomás, Bartolomé y Mateo, Jacobo hijo de Alfeo,
y Simón Zelotes, y Judas hermano de Jacobo. \footnote{\textbf{1:13} Luc
  6,13-16} \bibverse{14} Todos éstos perseveraban unánimes en oración y
ruego, con las mujeres, y con María la madre de Jesús, y con sus
hermanos. \footnote{\textbf{1:14} Juan 7,3}

\hypertarget{reemplazo-de-un-apuxf3stol-matuxedas-en-lugar-del-traidor-judas-iscariote}{%
\subsection{Reemplazo de un apóstol (Matías) en lugar del traidor Judas
Iscariote}\label{reemplazo-de-un-apuxf3stol-matuxedas-en-lugar-del-traidor-judas-iscariote}}

\bibverse{15} Y en aquellos días, Pedro, levantándose en medio de los
hermanos, dijo (y era la compañía junta como de ciento y veinte en
número): \footnote{\textbf{1:15} Juan 21,15-19} \bibverse{16} Varones
hermanos, convino que se cumpliese la Escritura, la cual dijo antes el
Espíritu Santo por la boca de David, de Judas, que fué guía de los que
prendieron á Jesús; \footnote{\textbf{1:16} Sal 41,10} \bibverse{17} El
cuál era contado con nosotros, y tenía suerte en este ministerio.
\bibverse{18} Este, pues, adquirió un campo del salario de su iniquidad,
y colgándose, reventó por medio, y todas sus entrañas se derramaron.
\bibverse{19} Y fué notorio á todos los moradores de Jerusalem; de tal
manera que aquel campo es llamado en su propia lengua, Acéldama, que es,
Campo de sangre. \bibverse{20} Porque está escrito en el libro de los
salmos: Sea hecha desierta su habitación, y no haya quien more en ella;
y: Tome otro su obispado.

\bibverse{21} Conviene, pues, que de estos hombres que han estado juntos
con nosotros todo el tiempo que el Señor Jesús entró y salió entre
nosotros, \footnote{\textbf{1:21} Juan 15,27} \bibverse{22} Comenzando
desde el bautismo de Juan, hasta el día que fué recibido arriba de entre
nosotros, uno sea hecho testigo con nosotros de su resurrección.

\bibverse{23} Y señalaron á dos: á José, llamado Barsabas, que tenía por
sobrenombre Justo, y á Matías. \bibverse{24} Y orando, dijeron: Tú,
Señor, que conoces los corazones de todos, muestra cuál escoges de estos
dos, \bibverse{25} Para que tome el oficio de este ministerio y
apostolado, del cual cayó Judas por transgresión, para irse á su lugar.
\bibverse{26} Y les echaron suertes, y cayó la suerte sobre Matías; y
fué contado con los once apóstoles. \footnote{\textbf{1:26} Prov 16,33}

\hypertarget{el-milagro-de-pentecostuxe9s-el-derramamiento-del-espuxedritu-santo-y-su-tremendo-testimonio-de-las-grandes-obras-de-dios}{%
\subsection{El milagro de Pentecostés: el derramamiento del Espíritu
Santo y su tremendo testimonio de las grandes obras de
Dios}\label{el-milagro-de-pentecostuxe9s-el-derramamiento-del-espuxedritu-santo-y-su-tremendo-testimonio-de-las-grandes-obras-de-dios}}

\hypertarget{section-1}{%
\section{2}\label{section-1}}

\bibverse{1} Y como se cumplieron los días de Pentecostés, estaban todos
unánimes juntos; \footnote{\textbf{2:1} Lev 23,15-21} \bibverse{2} Y de
repente vino un estruendo del cielo como de un viento recio que corría,
el cual hinchió toda la casa donde estaban sentados; \bibverse{3} Y se
les aparecieron lenguas repartidas, como de fuego, que se asentó sobre
cada uno de ellos. \footnote{\textbf{2:3} Mat 3,11} \bibverse{4} Y
fueron todos llenos del Espíritu Santo, y comenzaron á hablar en otras
lenguas, como el Espíritu les daba que hablasen. \footnote{\textbf{2:4}
  Hech 4,31; Hech 10,44-46}

\bibverse{5} Moraban entonces en Jerusalem Judíos, varones religiosos,
de todas las naciones debajo del cielo. \footnote{\textbf{2:5} Hech
  13,26} \bibverse{6} Y hecho este estruendo, juntóse la multitud; y
estaban confusos, porque cada uno les oía hablar su propia lengua.
\bibverse{7} Y estaban atónitos y maravillados, diciendo: He aquí ¿no
son Galileos todos estos que hablan? \bibverse{8} ¿Cómo, pues, les oímos
nosotros hablar cada uno en nuestra lengua en que somos nacidos?
\bibverse{9} Partos y Medos, y Elamitas, y los que habitamos en
Mesopotamia, en Judea y en Capadocia, en el Ponto y en Asia,
\bibverse{10} En Phrygia y Pamphylia, en Egipto y en las partes de
Africa que está de la otra parte de Cirene, y Romanos extranjeros, tanto
Judíos como convertidos, \bibverse{11} Cretenses y Arabes, les oímos
hablar en nuestras lenguas las maravillas de Dios. \bibverse{12} Y
estaban todos atónitos y perplejos, diciendo los unos á los otros: ¿Qué
quiere ser esto? \bibverse{13} Mas otros burlándose, decían: Que están
llenos de mosto.

\hypertarget{explicaciuxf3n-del-milagro-de-pentecostuxe9s-como-el-cumplimiento-de-la-antigua-palabra-profuxe9tica-de-joel}{%
\subsection{Explicación del milagro de Pentecostés como el cumplimiento
de la antigua palabra profética de
Joel}\label{explicaciuxf3n-del-milagro-de-pentecostuxe9s-como-el-cumplimiento-de-la-antigua-palabra-profuxe9tica-de-joel}}

\bibverse{14} Entonces Pedro, poniéndose en pie con los once, alzó su
voz, y hablóles diciendo: Varones Judíos, y todos los que habitáis en
Jerusalem, esto os sea notorio, y oid mis palabras. \bibverse{15} Porque
éstos no están borrachos, como vosotros pensáis, siendo la hora tercia
del día; \bibverse{16} Mas esto es lo que fué dicho por el profeta Joel:
\bibverse{17} Y será en los postreros días, dice Dios, derramaré de mi
Espíritu sobre toda carne, y vuestros hijos y vuestras hijas
profetizarán; y vuestros mancebos verán visiones, y vuestros viejos
soñarán sueños: \bibverse{18} Y de cierto sobre mis siervos y sobre mis
siervas en aquellos días derramaré de mi Espíritu, y profetizarán.
\bibverse{19} Y daré prodigios arriba en el cielo, y señales abajo en la
tierra, sangre y fuego y vapor de humo: \bibverse{20} El sol se volverá
en tinieblas, y la luna en sangre, antes que venga el día del Señor,
grande y manifiesto; \bibverse{21} Y será que todo aquel que invocare el
nombre del Señor, será salvo.

\hypertarget{jesuxfas-crucificado-resucitado-y-exaltado-por-dios-tiene-las-dos-palabras-de-david}{%
\subsection{Jesús, crucificado, resucitado y exaltado por Dios, tiene
las dos palabras de
David}\label{jesuxfas-crucificado-resucitado-y-exaltado-por-dios-tiene-las-dos-palabras-de-david}}

\bibverse{22} Varones Israelitas, oid estas palabras: Jesús Nazareno,
varón aprobado de Dios entre vosotros en maravillas y prodigios y
señales, que Dios hizo por él en medio de vosotros, como también
vosotros sabéis; \bibverse{23} A éste, entregado por determinado consejo
y providencia de Dios, prendisteis y matasteis por manos de los inicuos,
crucificándole; \bibverse{24} Al cual Dios levantó, sueltos los dolores
de la muerte, por cuanto era imposible ser detenido de ella.
\bibverse{25} Porque David dice de él: Veía al Señor siempre delante de
mí: porque está á mi diestra, no seré conmovido. \bibverse{26} Por lo
cual mi corazón se alegró, y gozóse mi lengua; y aun mi carne descansará
en esperanza; \bibverse{27} Que no dejarás mi alma en el infierno, ni
darás á tu Santo que vea corrupción. \bibverse{28} Hicísteme notorios
los caminos de la vida; me henchirás de gozo con tu presencia.

\bibverse{29} Varones hermanos, se os puede libremente decir del
patriarca David, que murió, y fué sepultado, y su sepulcro está con
nosotros hasta del día de hoy. \footnote{\textbf{2:29} 1Re 2,10}
\bibverse{30} Empero siendo profeta, y sabiendo que con juramento le
había Dios jurado que del fruto de su lomo, cuanto á la carne,
levantaría al Cristo que se sentaría sobre su trono; \footnote{\textbf{2:30}
  Sal 89,4-5; 2Sam 7,12-13} \bibverse{31} Viéndolo antes, habló de la
resurrección de Cristo, que su alma no fué dejada en el infierno, ni su
carne vió corrupción. \bibverse{32} A este Jesús resucitó Dios, de lo
cual todos nosotros somos testigos. \bibverse{33} Así que, levantado por
la diestra de Dios, y recibiendo del Padre la promesa del Espíritu
Santo, ha derramado esto que vosotros veis y oís. \footnote{\textbf{2:33}
  Juan 15,26} \bibverse{34} Porque David no subió á los cielos; empero
él dice: Dijo el Señor á mi Señor: Siéntate á mi diestra, \bibverse{35}
Hasta que ponga á tus enemigos por estrado de tus pies.

\bibverse{36} Sepa pues ciertísimamente toda la casa de Israel, que á
éste Jesús que vosotros crucificasteis, Dios ha hecho Señor y Cristo.

\hypertarget{efecto-del-habla-primer-ministerio-pastoral-de-pedro-fundaciuxf3n-de-la-primera-iglesia}{%
\subsection{Efecto del habla; primer ministerio pastoral de Pedro;
Fundación de la primera
iglesia}\label{efecto-del-habla-primer-ministerio-pastoral-de-pedro-fundaciuxf3n-de-la-primera-iglesia}}

\bibverse{37} Entonces oído esto, fueron compungidos de corazón, y
dijeron á Pedro y á los otros apóstoles: Varones hermanos, ¿qué haremos?
\footnote{\textbf{2:37} Hech 16,30; Luc 3,10}

\bibverse{38} Y Pedro les dice: Arrepentíos, y bautícese cada uno de
vosotros en el nombre de Jesucristo para perdón de los pecados; y
recibiréis el don del Espíritu Santo. \footnote{\textbf{2:38} Hech
  3,17-19; Luc 24,47} \bibverse{39} Porque para vosotros es la promesa,
y para vuestros hijos, y para todos los que están lejos; para cuantos el
Señor nuestro Dios llamare. \footnote{\textbf{2:39} Jl 3,5}
\bibverse{40} Y con otras muchas palabras testificaba y exhortaba,
diciendo: Sed salvos de esta perversa generación. \footnote{\textbf{2:40}
  Mat 17,17; Fil 2,15}

\bibverse{41} Así que, los que recibieron su palabra, fueron bautizados:
y fueron añadidas á ellos aquel día como tres mil personas.

\hypertarget{la-vida-de-los-creyentes-en-la-primera-iglesia}{%
\subsection{La vida de los creyentes en la primera
iglesia}\label{la-vida-de-los-creyentes-en-la-primera-iglesia}}

\bibverse{42} Y perseveraban en la doctrina de los apóstoles, y en la
comunión, y en el partimiento del pan, y en las oraciones. \footnote{\textbf{2:42}
  Hech 20,7} \bibverse{43} Y toda persona tenía temor: y muchas
maravillas y señales eran hechas por los apóstoles. \bibverse{44} Y
todos los que creían estaban juntos; y tenían todas las cosas comunes;
\bibverse{45} Y vendían las posesiones, y las haciendas, y repartíanlas
á todos, como cada uno había menester. \bibverse{46} Y perseverando
unánimes cada día en el templo, y partiendo el pan en las casas, comían
juntos con alegría y con sencillez de corazón, \bibverse{47} Alabando á
Dios, y teniendo gracia con todo el pueblo. Y el Señor añadía cada día á
la iglesia los que habían de ser salvos. \footnote{\textbf{2:47} Hech
  4,4; Hech 5,14; Hech 6,7; Hech 11,21; Hech 14,1}

\hypertarget{pedro-y-juan-curan-a-un-cojo-de-nacimiento}{%
\subsection{Pedro y Juan curan a un cojo de
nacimiento}\label{pedro-y-juan-curan-a-un-cojo-de-nacimiento}}

\hypertarget{section-2}{%
\section{3}\label{section-2}}

\bibverse{1} Pedro y Juan subían juntos al templo á la hora de oración,
la de nona. \bibverse{2} Y un hombre que era cojo desde el vientre de su
madre, era traído; al cual ponían cada día á la puerta del templo que se
llama la Hermosa, para que pidiese limosna de los que entraban en el
templo. \bibverse{3} Este, como vió á Pedro y á Juan que iban á entrar
en el templo, rogaba que le diesen limosna. \bibverse{4} Y Pedro, con
Juan, fijando los ojos en él, dijo: Mira á nosotros. \bibverse{5}
Entonces él estuvo atento á ellos, esperando recibir de ellos algo.
\bibverse{6} Y Pedro dijo: Ni tengo plata ni oro; mas lo que tengo te
doy: en el nombre de Jesucristo de Nazaret, levántate y anda.
\bibverse{7} Y tomándole por la mano derecha le levantó: y luego fueron
afirmados sus pies y tobillos; \bibverse{8} Y saltando, se puso en pie,
y anduvo; y entró con ellos en el templo, andando, y saltando, y
alabando á Dios. \bibverse{9} Y todo el pueblo le vió andar y alabar á
Dios. \bibverse{10} Y conocían que él era el que se sentaba á la limosna
á la puerta del templo, la Hermosa: y fueron llenos de asombro y de
espanto por lo que le había acontecido. \bibverse{11} Y teniendo á Pedro
y á Juan el cojo que había sido sanado, todo el pueblo concurrió á ellos
al pórtico que se llama de Salomón, atónitos.

\hypertarget{discurso-en-el-templo-sermuxf3n-penitencial-de-pedro-despuuxe9s-de-sanar-al-cojo}{%
\subsection{Discurso en el templo, sermón penitencial de Pedro después
de sanar al
cojo}\label{discurso-en-el-templo-sermuxf3n-penitencial-de-pedro-despuuxe9s-de-sanar-al-cojo}}

\bibverse{12} Y viendo esto Pedro, respondió al pueblo: Varones
Israelitas, ¿por qué os maravilláis de esto? ó ¿por qué ponéis los ojos
en nosotros, como si con nuestra virtud ó piedad hubiésemos hecho andar
á éste? \bibverse{13} El Dios de Abraham, y de Isaac, y de Jacob, el
Dios de nuestros padres ha glorificado á su Hijo Jesús, al cual vosotros
entregasteis, y negasteis delante de Pilato, juzgando él que había de
ser suelto. \bibverse{14} Mas vosotros al Santo y al Justo negasteis, y
pedisteis que se os diese un homicida; \footnote{\textbf{3:14} Mat
  27,20-21} \bibverse{15} Y matasteis al Autor de la vida, al cual Dios
ha resucitado de los muertos; de lo que nosotros somos testigos.
\bibverse{16} Y en la fe de su nombre, á éste que vosotros veis y
conocéis, ha confirmado su nombre: y la fe que por él es, ha dado á éste
esta completa sanidad en presencia de todos vosotros.

\bibverse{17} Mas ahora, hermanos, sé que por ignorancia lo habéis
hecho, como también vuestros príncipes. \bibverse{18} Empero Dios ha
cumplido así lo que había antes anunciado por boca de todos sus
profetas, que su Cristo había de padecer. \footnote{\textbf{3:18} Luc
  24,44}

\bibverse{19} Así que, arrepentíos y convertíos, para que sean borrados
vuestros pecados; pues que vendrán los tiempos del refrigerio de la
presencia del Señor, \footnote{\textbf{3:19} Hech 2,38} \bibverse{20} Y
enviará á Jesucristo, que os fué antes anunciado: \bibverse{21} Al cual
de cierto es menester que el cielo tenga hasta los tiempos de la
restauración de todas las cosas, que habló Dios por boca de sus santos
profetas que han sido desde el siglo. \bibverse{22} Porque Moisés dijo á
los padres: El Señor vuestro Dios os levantará profeta de vuestros
hermanos, como yo; á él oiréis en todas las cosas que os hablare.
\bibverse{23} Y será, que cualquiera alma que no oyere á aquel profeta,
será desarraigada del pueblo. \bibverse{24} Y todos los profetas desde
Samuel y en adelante, todos los que han hablado, han anunciado estos
días. \footnote{\textbf{3:24} 2Sam 7,12-16}

\bibverse{25} Vosotros sois los hijos de los profetas, y del pacto que
Dios concertó con nuestros padres, diciendo á Abraham: Y en tu simiente
serán benditas todas las familias de la tierra. \bibverse{26} A vosotros
primeramente, Dios, habiendo levantado á su Hijo, le envió para que os
bendijese, á fin de que cada uno se convierta de su maldad.

\hypertarget{pedro-y-juan-en-la-cuxe1rcel-y-ante-el-concilio}{%
\subsection{Pedro y Juan en la cárcel y ante el
concilio}\label{pedro-y-juan-en-la-cuxe1rcel-y-ante-el-concilio}}

\hypertarget{section-3}{%
\section{4}\label{section-3}}

\bibverse{1} Y hablando ellos al pueblo, sobrevinieron los sacerdotes, y
el magistrado del templo, y los Saduceos, \footnote{\textbf{4:1} Luc
  22,4; Luc 22,52} \bibverse{2} Resentidos de que enseñasen al pueblo, y
anunciasen en Jesús la resurrección de los muertos. \footnote{\textbf{4:2}
  Hech 23,8} \bibverse{3} Y les echaron mano, y los pusieron en la
cárcel hasta el día siguiente; porque era ya tarde. \bibverse{4} Mas
muchos de los que habían oído la palabra, creyeron; y fué el número de
los varones como cinco mil. \footnote{\textbf{4:4} Hech 2,47}

\bibverse{5} Y aconteció al día siguiente, que se juntaron en Jerusalem
los príncipes de ellos, y los ancianos, y los escribas; \bibverse{6} Y
Anás, príncipe de los sacerdotes, y Caifás, y Juan y Alejandro, y todos
los que eran del linaje sacerdotal; \bibverse{7} Y haciéndolos presentar
en medio, les preguntaron: ¿Con qué potestad, ó en qué nombre, habéis
hecho vosotros esto? \footnote{\textbf{4:7} Mat 21,33}

\bibverse{8} Entonces Pedro, lleno del Espíritu Santo, les dijo:
Príncipes del pueblo, y ancianos de Israel: \footnote{\textbf{4:8} Mat
  10,19-20} \bibverse{9} Pues que somos hoy demandados acerca del
beneficio hecho á un hombre enfermo, de qué manera éste haya sido
sanado, \bibverse{10} Sea notorio á todos vosotros, y á todo el pueblo
de Israel, que en el nombre de Jesucristo de Nazaret, al que vosotros
crucificasteis y Dios le resucitó de los muertos, por él este hombre
está en vuestra presencia sano. \footnote{\textbf{4:10} Hech 3,6; Hech
  3,13-16} \bibverse{11} Este es la piedra reprobada de vosotros los
edificadores, la cual es puesta por cabeza del ángulo. \footnote{\textbf{4:11}
  Mat 21,42} \bibverse{12} Y en ningún otro hay salud; porque no hay
otro nombre debajo del cielo, dado á los hombres, en que podamos ser
salvos. \footnote{\textbf{4:12} Hech 10,43; Mat 1,21}

\bibverse{13} Entonces viendo la constancia de Pedro y de Juan, sabido
que eran hombres sin letras é ignorantes, se maravillaban; y les
conocían que habían estado con Jesús. \bibverse{14} Y viendo al hombre
que había sido sanado, que estaba con ellos, no podían decir nada en
contra. \bibverse{15} Mas les mandaron que se saliesen fuera del
concilio; y conferían entre sí, \bibverse{16} Diciendo: ¿Qué hemos de
hacer á estos hombres? porque de cierto, señal manifiesta ha sido hecha
por ellos, notoria á todos los que moran en Jerusalem, y no lo podemos
negar. \footnote{\textbf{4:16} Juan 11,47} \bibverse{17} Todavía, porque
no se divulgue más por el pueblo, amenacémoslos que no hablen de aquí
adelante á hombre alguno en este nombre. \bibverse{18} Y llamándolos,
les intimaron que en ninguna manera hablasen ni enseñasen en el nombre
de Jesús.

\bibverse{19} Entonces Pedro y Juan, respondiendo, les dijeron: Juzgad
si es justo delante de Dios obedecer antes á vosotros que á Dios:
\bibverse{20} Porque no podemos dejar de decir lo que hemos visto y
oído.

\bibverse{21} Ellos entonces los despacharon amenazándolos, no hallando
ningún modo de castigarlos, por causa del pueblo; porque todos
glorificaban á Dios de lo que había sido hecho. \bibverse{22} Porque el
hombre en quien había sido hecho este milagro de sanidad, era de más de
cuarenta años.

\hypertarget{regreso-de-los-apuxf3stoles-acciuxf3n-de-gracias-y-suxfaplica-de-la-congregaciuxf3n}{%
\subsection{Regreso de los apóstoles; Acción de gracias y súplica de la
congregación}\label{regreso-de-los-apuxf3stoles-acciuxf3n-de-gracias-y-suxfaplica-de-la-congregaciuxf3n}}

\bibverse{23} Y sueltos, vinieron á los suyos, y contaron todo lo que
los príncipes de los sacerdotes y los ancianos les habían dicho.
\bibverse{24} Y ellos, habiéndolo oído, alzaron unánimes la voz á Dios,
y dijeron: Señor, tú eres el Dios que hiciste el cielo y la tierra, la
mar, y todo lo que en ellos hay; \bibverse{25} Que por boca de David, tu
siervo, dijiste: ¿Por qué han bramado las gentes, y los pueblos han
pensado cosas vanas? \bibverse{26} Asistieron los reyes de la tierra, y
los príncipes se juntaron en uno contra el Señor, y contra su Cristo.

\bibverse{27} Porque verdaderamente se juntaron en esta ciudad contra tu
santo Hijo Jesús, al cual ungiste, Herodes y Poncio Pilato, con los
Gentiles y los pueblos de Israel, \footnote{\textbf{4:27} Luc 23,12}
\bibverse{28} Para hacer lo que tu mano y tu consejo habían antes
determinado que había de ser hecho. \footnote{\textbf{4:28} Hech 2,23}
\bibverse{29} Y ahora, Señor, mira sus amenazas, y da á tus siervos que
con toda confianza hablen tu palabra; \footnote{\textbf{4:29} Efes 6,19}
\bibverse{30} Que extiendas tu mano á que sanidades, y milagros, y
prodigios sean hechos por el nombre de tu santo Hijo Jesús.

\bibverse{31} Y como hubieron orado, el lugar en que estaban congregados
tembló; y todos fueron llenos del Espíritu Santo, y hablaron la palabra
de Dios con confianza.

\hypertarget{la-comunidad-de-bienes}{%
\subsection{La comunidad de bienes}\label{la-comunidad-de-bienes}}

\bibverse{32} Y la multitud de los que habían creído era de un corazón y
un alma: y ninguno decía ser suyo algo de lo que poseía; mas todas las
cosas les eran comunes. \bibverse{33} Y los apóstoles daban testimonio
de la resurrección del Señor Jesús con gran esfuerzo; y gran gracia era
en todos ellos. \footnote{\textbf{4:33} Hech 2,47} \bibverse{34} Que
ningún necesitado había entre ellos: porque todos los que poseían
heredades ó casas, vendiéndolas, traían el precio de lo vendido,
\footnote{\textbf{4:34} Hech 2,45} \bibverse{35} Y lo ponían á los pies
de los apóstoles; y era repartido á cada uno según que había menester.

\bibverse{36} Entonces José, que fué llamado de los apóstoles por
sobrenombre, Bernabé, (que es interpretado, Hijo de consolación) Levita,
natural de Cipro, \footnote{\textbf{4:36} Hech 11,22-26; Hech 12,25;
  Hech 15,2; Gal 2,1; Col 4,10}

\bibverse{37} Como tuviese una heredad, la vendió, y trajo el precio, y
púsolo á los pies de los apóstoles.

\hypertarget{un-ejemplo-de-disciplina-eclesiuxe1stica-seria-ananuxedas-y-safira}{%
\subsection{Un ejemplo de disciplina eclesiástica seria: Ananías y
Safira}\label{un-ejemplo-de-disciplina-eclesiuxe1stica-seria-ananuxedas-y-safira}}

\hypertarget{section-4}{%
\section{5}\label{section-4}}

\bibverse{1} Mas un varón llamado Ananías, con Safira su mujer, vendió
una posesión, \bibverse{2} Y defraudó del precio, sabiéndolo también su
mujer; y trayendo una parte, púsola á los pies de los apóstoles.
\bibverse{3} Y dijo Pedro: Ananías, ¿por qué ha llenado Satanás tu
corazón á que mintieses al Espíritu Santo, y defraudases del precio de
la heredad? \bibverse{4} Reteniéndola, ¿no se te quedaba á ti? y
vendida, ¿no estaba en tu potestad? ¿Por qué pusiste esto en tu corazón?
No has mentido á los hombres, sino á Dios.

\bibverse{5} Entonces Ananías, oyendo estas palabras, cayó y espiró. Y
vino un gran temor sobre todos los que lo oyeron. \bibverse{6} Y
levantándose los mancebos, le tomaron, y sacándolo, sepultáronlo.
\bibverse{7} Y pasado espacio como de tres horas, sucedió que entró su
mujer, no sabiendo lo que había acontecido. \bibverse{8} Entonces Pedro
le dijo: Dime: ¿vendisteis en tanto la heredad? Y ella dijo: Sí, en
tanto.

\bibverse{9} Y Pedro le dijo: ¿Por qué os concertasteis para tentar al
Espíritu del Señor? He aquí á la puerta los pies de los que han
sepultado á tu marido, y te sacarán.

\bibverse{10} Y luego cayó á los pies de él, y espiró: y entrados los
mancebos, la hallaron muerta; y la sacaron, y la sepultaron junto á su
marido. \bibverse{11} Y vino un gran temor en toda la iglesia, y en
todos los que oyeron estas cosas.

\hypertarget{milagros-especialmente-la-curaciuxf3n-de-los-enfermos-de-los-apuxf3stoles-mayor-crecimiento-de-la-comunidad}{%
\subsection{Milagros (especialmente la curación de los enfermos) de los
apóstoles; mayor crecimiento de la
comunidad}\label{milagros-especialmente-la-curaciuxf3n-de-los-enfermos-de-los-apuxf3stoles-mayor-crecimiento-de-la-comunidad}}

\bibverse{12} Y por las manos de los apóstoles eran hechos muchos
milagros y prodigios en el pueblo; y estaban todos unánimes en el
pórtico de Salomón. \footnote{\textbf{5:12} Hech 3,11} \bibverse{13} Y
de los otros, ninguno osaba juntarse con ellos; mas el pueblo los
alababa grandemente. \bibverse{14} Y los que creían en el Señor se
aumentaban más, gran número así de hombres como de mujeres;
\bibverse{15} Tanto que echaban los enfermos por las calles, y los
ponían en camas y en lechos, para que viniendo Pedro, á lo menos su
sombra tocase á alguno de ellos. \footnote{\textbf{5:15} Hech 19,11-12}
\bibverse{16} Y aun de las ciudades vecinas concurría multitud á
Jerusalem, trayendo enfermos y atormentados de espíritus inmundos; los
cuales todos eran curados.

\hypertarget{el-arresto-liberaciuxf3n-a-travuxe9s-de-un-uxe1ngel}{%
\subsection{El arresto; Liberación a través de un
ángel}\label{el-arresto-liberaciuxf3n-a-travuxe9s-de-un-uxe1ngel}}

\bibverse{17} Entonces levantándose el príncipe de los sacerdotes, y
todos los que estaban con él, que es la secta de los Saduceos, se
llenaron de celo; \bibverse{18} Y echaron mano á los apóstoles, y
pusiéronlos en la cárcel pública. \bibverse{19} Mas el ángel del Señor,
abriendo de noche las puertas de la cárcel, y sacándolos, dijo:
\footnote{\textbf{5:19} Hech 12,7} \bibverse{20} Id, y estando en el
templo, hablad al pueblo todas las palabras de esta vida.

\bibverse{21} Y oído que hubieron esto, entraron de mañana en el templo,
y enseñaban. Entre tanto, viniendo el príncipe de los sacerdotes, y los
que eran con él, convocaron el concilio, y á todos los ancianos de los
hijos de Israel, y enviaron á la cárcel para que fuesen traídos.
\bibverse{22} Mas como llegaron los ministros, y no los hallaron en la
cárcel, volvieron, y dieron aviso, \bibverse{23} Diciendo: Por cierto,
la cárcel hemos hallado cerrada con toda seguridad, y los guardas que
estaban delante de las puertas; mas cuando abrimos, á nadie hallamos
dentro.

\bibverse{24} Y cuando oyeron estas palabras el pontífice y el
magistrado del templo y los príncipes de los sacerdotes, dudaban en qué
vendría á parar aquello. \bibverse{25} Pero viniendo uno, dióles esta
noticia: He aquí, los varones que echasteis en la cárcel, están en el
templo, y enseñan al pueblo. \bibverse{26} Entonces fué el magistrado
con los ministros, y trájolos sin violencia; porque temían del pueblo
ser apedreados.

\hypertarget{el-valiente-testimonio-del-apuxf3stol-de-la-resurrecciuxf3n-de-cristo}{%
\subsection{El valiente testimonio del apóstol de la resurrección de
Cristo}\label{el-valiente-testimonio-del-apuxf3stol-de-la-resurrecciuxf3n-de-cristo}}

\bibverse{27} Y como los trajeron, los presentaron en el concilio: y el
príncipe de los sacerdotes les preguntó, \bibverse{28} Diciendo: ¿No os
denunciamos estrechamente, que no enseñaseis en este nombre? y he aquí,
habéis llenado á Jerusalem de vuestra doctrina, y queréis echar sobre
nosotros la sangre de este hombre.

\bibverse{29} Y respondiendo Pedro y los apóstoles, dijeron: Es menester
obedecer á Dios antes que á los hombres. \footnote{\textbf{5:29} Hech
  4,19; Dan 3,16-18} \bibverse{30} El Dios de nuestros padres levantó á
Jesús, al cual vosotros matasteis colgándole en un madero. \footnote{\textbf{5:30}
  Hech 3,15} \bibverse{31} A éste ha Dios ensalzado con su diestra por
Príncipe y Salvador, para dar á Israel arrepentimiento y remisión de
pecados. \footnote{\textbf{5:31} Hech 2,33} \bibverse{32} Y nosotros
somos testigos suyos de estas cosas, y también el Espíritu Santo, el
cual ha dado Dios á los que le obedecen. \footnote{\textbf{5:32} Luc
  24,48; Juan 15,26-27}

\bibverse{33} Ellos, oyendo esto, regañaban, y consultaban matarlos.

\hypertarget{defensa-y-asesoramiento-de-gamaliel}{%
\subsection{Defensa y asesoramiento de
Gamaliel}\label{defensa-y-asesoramiento-de-gamaliel}}

\bibverse{34} Entonces levantándose en el concilio un Fariseo llamado
Gamaliel, doctor de la ley, venerable á todo el pueblo, mandó que
sacasen fuera un poco á los apóstoles. \footnote{\textbf{5:34} Hech 22,3}
\bibverse{35} Y les dijo: Varones Israelitas, mirad por vosotros acerca
de estos hombres en lo que habéis de hacer. \bibverse{36} Porque antes
de estos días se levantó Teudas, diciendo que era alguien; al que se
agregó un número de hombres como cuatrocientos: el cual fué matado; y
todos los que le creyeron fueron dispersos, y reducidos á nada.
\bibverse{37} Después de éste, se levantó Judas el Galileo en los días
del empadronamiento, y llevó mucho pueblo tras sí. Pereció también
aquél; y todos los que consintieron con él, fueron derramados.
\bibverse{38} Y ahora os digo: Dejaos de estos hombres, y dejadlos;
porque si este consejo ó esta obra es de los hombres, se desvanecerá:
\bibverse{39} Mas si es de Dios, no la podréis deshacer; no seáis tal
vez hallados resistiendo á Dios.

\bibverse{40} Y convinieron con él: y llamando á los apóstoles, después
de azotados, les intimaron que no hablasen en el nombre de Jesús, y
soltáronlos. \footnote{\textbf{5:40} Mat 10,17} \bibverse{41} Y ellos
partieron de delante del concilio, gozosos de que fuesen tenidos por
dignos de padecer afrenta por el Nombre. \footnote{\textbf{5:41} Mat
  5,10-12; 1Pe 4,13}

\bibverse{42} Y todos los días, en el templo y por las casas, no cesaban
de enseñar y predicar á Jesucristo.

\hypertarget{separaciuxf3n-de-la-oficina-de-predicaciuxf3n-y-ayuda-a-los-pobres-elecciuxf3n-y-nombramiento-de-los-siete-cuidadores-pobres}{%
\subsection{Separación de la oficina de predicación y ayuda a los
pobres; Elección y nombramiento de los siete cuidadores
pobres}\label{separaciuxf3n-de-la-oficina-de-predicaciuxf3n-y-ayuda-a-los-pobres-elecciuxf3n-y-nombramiento-de-los-siete-cuidadores-pobres}}

\hypertarget{section-5}{%
\section{6}\label{section-5}}

\bibverse{1} En aquellos días, creciendo el número de los discípulos,
hubo murmuración de los Griegos contra los Hebreos, de que sus viudas
eran menospreciadas en el ministerio cotidiano. \footnote{\textbf{6:1}
  Hech 4,35} \bibverse{2} Así que, los doce convocaron la multitud de
los discípulos, y dijeron: No es justo que nosotros dejemos la palabra
de Dios, y sirvamos á las mesas. \bibverse{3} Buscad pues, hermanos,
siete varones de vosotros de buen testimonio, llenos de Espíritu Santo y
de sabiduría, los cuales pongamos en esta obra. \bibverse{4} Y nosotros
persistiremos en la oración, y en el ministerio de la palabra.

\bibverse{5} Y plugo el parecer á toda la multitud; y eligieron á
Esteban, varón lleno de fe y de Espíritu Santo, y á Felipe, y á Prócoro,
y á Nicanor, y á Timón, y á Parmenas, y á Nicolás, prosélito de
Antioquía: \footnote{\textbf{6:5} Hech 8,5} \bibverse{6} A estos
presentaron delante de los apóstoles, los cuales orando les pusieron las
manos encima. \footnote{\textbf{6:6} Hech 1,24; Hech 13,3; Hech 14,23}

\bibverse{7} Y crecía la palabra del Señor, y el número de los
discípulos se multiplicaba mucho en Jerusalem: también una gran multitud
de los sacerdotes obedecía á la fe. \footnote{\textbf{6:7} Hech 2,47;
  Hech 19,20}

\hypertarget{acusaciuxf3n-y-muerte-de-esteban-el-primer-muxe1rtir}{%
\subsection{Acusación y muerte de Esteban, el primer
mártir}\label{acusaciuxf3n-y-muerte-de-esteban-el-primer-muxe1rtir}}

\bibverse{8} Empero Esteban, lleno de gracia y de potencia, hacía
prodigios y milagros grandes en el pueblo. \bibverse{9} Levantáronse
entonces unos de la sinagoga que se llama de los Libertinos, y Cireneos,
y Alejandrinos, y de los de Cilicia, y de Asia, disputando con Esteban.
\bibverse{10} Mas no podían resistir á la sabiduría y al Espíritu con
que hablaba. \bibverse{11} Entonces sobornaron á unos que dijesen que le
habían oído hablar palabras blasfemas contra Moisés y Dios. \footnote{\textbf{6:11}
  Mat 26,60-66} \bibverse{12} Y conmovieron al pueblo, y á los ancianos,
y á los escribas; y arremetiendo le arrebataron, y le trajeron al
concilio. \bibverse{13} Y pusieron testigos falsos, que dijesen: Este
hombre no cesa de hablar palabras blasfemas contra este lugar santo y la
ley: \bibverse{14} Porque le hemos oído decir, que este Jesús de Nazaret
destruirá este lugar, y mudará las ordenanzas que nos dió Moisés.
\footnote{\textbf{6:14} Juan 2,19}

\bibverse{15} Entonces todos los que estaban sentados en el concilio,
puestos los ojos en él, vieron su rostro como el rostro de un ángel.

\hypertarget{discurso-de-defensa-de-esteban-la-uxe9poca-de-los-patriarcas}{%
\subsection{Discurso de defensa de Esteban: la época de los
patriarcas}\label{discurso-de-defensa-de-esteban-la-uxe9poca-de-los-patriarcas}}

\hypertarget{section-6}{%
\section{7}\label{section-6}}

\bibverse{1} El príncipe de los sacerdotes dijo entonces: ¿Es esto así?

\bibverse{2} Y él dijo: Varones hermanos y padres, oid: El Dios de la
gloria apareció á nuestro padre Abraham, estando en Mesopotamia, antes
que morase en Chârán, \bibverse{3} Y le dijo: Sal de tu tierra y de tu
parentela, y ven á la tierra que te mostraré. \bibverse{4} Entonces
salió de la tierra de los Caldeos, y habitó en Chârán: y de allí, muerto
su padre, le traspasó á esta tierra, en la cual vosotros habitáis ahora;
\bibverse{5} Y no le dió herencia en ella, ni aun para asentar un pie:
mas le prometió que se la daría en posesión, y á su simiente después de
él, no teniendo hijo. \bibverse{6} Y hablóle Dios así: Que su simiente
sería extranjera en tierra ajena, y que los reducirían á servidumbre y
maltratarían, por cuatrocientos años. \footnote{\textbf{7:6} Éxod 12,40}
\bibverse{7} Mas yo juzgaré, dijo Dios, la nación á la cual serán
siervos: y después de esto saldrán y me servirán en este lugar.
\bibverse{8} Y dióle el pacto de la circuncisión: y así Abraham engendró
á Isaac, y le circuncidó al octavo día; é Isaac á Jacob, y Jacob á los
doce patriarcas.

\bibverse{9} Y los patriarcas, movidos de envidia, vendieron á José para
Egipto; mas Dios era con él, \bibverse{10} Y le libró de todas sus
tribulaciones, y le dió gracia y sabiduría en la presencia de Faraón,
rey de Egipto, el cual le puso por gobernador sobre Egipto, y sobre toda
su casa. \bibverse{11} Vino entonces hambre en toda la tierra de Egipto
y de Canaán, y grande tribulación; y nuestros padres no hallaban
alimentos. \bibverse{12} Y como oyese Jacob que había trigo en Egipto,
envió á nuestros padres la primera vez. \bibverse{13} Y en la segunda,
José fué conocido de sus hermanos, y fué sabido de Faraón el linaje de
José. \bibverse{14} Y enviando José, hizo venir á su padre Jacob, y á
toda su parentela, en número de setenta y cinco personas. \bibverse{15}
Así descendió Jacob á Egipto, donde murió él y nuestros padres;
\bibverse{16} Los cuales fueron trasladados á Sichêm, y puestos en el
sepulcro que compró Abraham á precio de dinero de los hijos de Hemor de
Sichêm.

\hypertarget{el-tiempo-del-mosaico}{%
\subsection{El tiempo del mosaico}\label{el-tiempo-del-mosaico}}

\bibverse{17} Mas como se acercaba el tiempo de la promesa, la cual Dios
había jurado á Abraham, el pueblo creció y multiplicóse en Egipto,
\bibverse{18} Hasta que se levantó otro rey en Egipto que no conocía á
José. \bibverse{19} Este, usando de astucia con nuestro linaje, maltrató
á nuestros padres, á fin de que pusiesen á peligro de muerte sus niños,
para que cesase la generación. \bibverse{20} En aquel mismo tiempo nació
Moisés, y fué agradable á Dios: y fué criado tres meses en casa de su
padre. \bibverse{21} Mas siendo puesto al peligro, la hija de Faraón le
tomó, y le crió como á hijo suyo. \bibverse{22} Y fué enseñado Moisés en
toda la sabiduría de los egipcios; y era poderoso en sus dichos y
hechos. \bibverse{23} Y cuando hubo cumplido la edad de cuarenta años,
le vino voluntad de visitar á sus hermanos los hijos de Israel.
\bibverse{24} Y como vió á uno que era injuriado, defendióle, é hiriendo
al Egipcio, vengó al injuriado. \bibverse{25} Pero él pensaba que sus
hermanos entendían que Dios les había de dar salud por su mano; mas
ellos no lo habían entendido.

\bibverse{26} Y al día siguiente, riñendo ellos, se les mostró, y los
ponía en paz, diciendo: Varones, hermanos sois, ¿por qué os injuriáis
los unos á los otros? \bibverse{27} Entonces el que injuriaba á su
prójimo, le rempujó, diciendo: ¿Quién te ha puesto por príncipe y juez
sobre nosotros? \bibverse{28} ¿Quieres tú matarme, como mataste ayer al
Egipcio? \bibverse{29} A esta palabra Moisés huyó, y se hizo extranjero
en tierra de Madián, donde engendró dos hijos. \footnote{\textbf{7:29}
  Éxod 18,3-4}

\bibverse{30} Y cumplidos cuarenta años, un ángel le apareció en el
desierto del monte Sina, en fuego de llama de una zarza. \bibverse{31}
Entonces Moisés mirando, se maravilló de la visión: y llegándose para
considerar, fué hecha á él voz del Señor: \bibverse{32} Yo soy el Dios
de tus padres, el Dios de Abraham, y el Dios de Isaac, y el Dios de
Jacob. Mas Moisés, temeroso, no osaba mirar. \bibverse{33} Y le dijo el
Señor: Quita los zapatos de tus pies, porque el lugar en que estás es
tierra santa. \bibverse{34} He visto, he visto la aflicción de mi pueblo
que está en Egipto, y he oído el gemido de ellos, y he descendido para
librarlos. Ahora pues, ven, te enviaré á Egipto.

\bibverse{35} A este Moisés, al cual habían rehusado, diciendo: ¿Quién
te ha puesto por príncipe y juez? á éste envió Dios por príncipe y
redentor con la mano del ángel que le apareció en la zarza.
\bibverse{36} Este los sacó, habiendo hecho prodigios y milagros en la
tierra de Egipto, y en el mar Bermejo, y en el desierto por cuarenta
años. \bibverse{37} Este es el Moisés, el cual dijo á los hijos de
Israel: Profeta os levantará el Señor Dios vuestro de vuestros hermanos,
como yo; á él oiréis. \bibverse{38} Este es aquél que estuvo en la
congregación en el desierto con el ángel que le hablaba en el monte
Sina, y con nuestros padres; y recibió las palabras de vida para darnos:
\footnote{\textbf{7:38} Éxod 19,-1; Deut 9,10} \bibverse{39} Al cual
nuestros padres no quisieron obedecer; antes le desecharon, y se
apartaron de corazón á Egipto, \bibverse{40} Diciendo á Aarón: Haznos
dioses que vayan delante de nosotros; porque á este Moisés, que nos sacó
de tierra de Egipto, no sabemos qué le ha acontecido. \bibverse{41} Y
entonces hicieron un becerro, y ofrecieron sacrificio al ídolo, y en las
obras de sus manos se holgaron. \bibverse{42} Y Dios se apartó, y los
entregó que sirviesen al ejército del cielo; como está escrito en el
libro de los profetas: ¿Me ofrecisteis víctimas y sacrificios en el
desierto por cuarenta años, casa de Israel? \bibverse{43} Antes,
trajisteis el tabernáculo de Moloch, y la estrella de vuestro dios
Remphan: figuras que os hicisteis para adorarlas: os transportaré pues,
más allá de Babilonia.

\hypertarget{el-tiempo-del-tabernuxe1culo-y-la-construcciuxf3n-del-templo}{%
\subsection{El tiempo del tabernáculo y la construcción del
templo}\label{el-tiempo-del-tabernuxe1culo-y-la-construcciuxf3n-del-templo}}

\bibverse{44} Tuvieron nuestros padres el tabernáculo del testimonio en
el desierto, como había ordenado Dios, hablando á Moisés que lo hiciese
según la forma que había visto. \bibverse{45} El cual recibido, metieron
también nuestros padres con Josué en la posesión de los Gentiles, que
Dios echó de la presencia de nuestros padres, hasta los días de David;
\footnote{\textbf{7:45} Jos 3,14; Jos 18,1} \bibverse{46} El cual halló
gracia delante de Dios, y pidió hallar tabernáculo para el Dios de
Jacob. \footnote{\textbf{7:46} 2Sam 7,-1; Sal 132,1-5} \bibverse{47} Mas
Salomón le edificó casa. \footnote{\textbf{7:47} 1Re 6,-1} \bibverse{48}
Si bien el Altísimo no habita en templos hechos de mano; como el profeta
dice: \bibverse{49} El cielo es mi trono, y la tierra es el estrado de
mis pies. ¿Qué casa me edificaréis? dice el Señor; ¿ó cuál es el lugar
de mi reposo? \bibverse{50} ¿No hizo mi mano todas estas cosas?

\hypertarget{fin-del-discurso-acusaciuxf3n-del-pueblo}{%
\subsection{Fin del discurso; Acusación del
pueblo}\label{fin-del-discurso-acusaciuxf3n-del-pueblo}}

\bibverse{51} Duros de cerviz, é incircuncisos de corazón y de oídos,
vosotros resistís siempre al Espíritu Santo: como vuestros padres, así
también vosotros. \bibverse{52} ¿A cuál de los profetas no persiguieron
vuestros padres? y mataron á los que antes anunciaron la venida del
Justo, del cual vosotros ahora habéis sido entregadores y matadores;
\footnote{\textbf{7:52} 2Cró 36,16; Mat 23,31} \bibverse{53} Que
recibisteis la ley por disposición de ángeles, y no la guardasteis.
\footnote{\textbf{7:53} Éxod 20,-1; Gal 3,19; Heb 2,2}

\hypertarget{el-martirio-de-esteban}{%
\subsection{El martirio de Esteban}\label{el-martirio-de-esteban}}

\bibverse{54} Y oyendo estas cosas, regañaban de sus corazones, y
crujían los dientes contra él. \bibverse{55} Mas él, estando lleno de
Espíritu Santo, puestos los ojos en el cielo, vió la gloria de Dios, y á
Jesús que estaba á la diestra de Dios, \bibverse{56} Y dijo: He aquí,
veo los cielos abiertos, y al Hijo del hombre que está á la diestra de
Dios. \footnote{\textbf{7:56} Luc 22,69}

\bibverse{57} Entonces dando grandes voces, se taparon sus oídos, y
arremetieron unánimes contra él; \bibverse{58} Y echándolo fuera de la
ciudad, le apedreaban: y los testigos pusieron sus vestidos á los pies
de un mancebo que se llamaba Saulo. \bibverse{59} Y apedrearon á
Esteban, invocando él y diciendo: Señor Jesús, recibe mi espíritu.
\footnote{\textbf{7:59} Luc 23,46}

\bibverse{60} Y puesto de rodillas, clamó á gran voz: Señor, no les
imputes este pecado. Y habiendo dicho esto, durmió.

\hypertarget{la-primera-persecuciuxf3n-de-la-comunidad-cristiana-en-jerusaluxe9n}{%
\subsection{La primera persecución de la comunidad cristiana en
Jerusalén}\label{la-primera-persecuciuxf3n-de-la-comunidad-cristiana-en-jerusaluxe9n}}

\hypertarget{section-7}{%
\section{8}\label{section-7}}

\bibverse{1} Y saulo consentía en su muerte. Y en aquel día se hizo una
grande persecución en la iglesia que estaba en Jerusalem; y todos fueron
esparcidos por las tierras de Judea y de Samaria, salvo los apóstoles.
\bibverse{2} Y llevaron á enterrar á Esteban varones piadosos, é
hicieron gran llanto sobre él. \bibverse{3} Entonces Saulo asolaba la
iglesia, entrando por las casas: y trayendo hombres y mujeres, los
entregaba en la cárcel. \footnote{\textbf{8:3} Hech 9,1; Hech 22,4; 1Cor
  15,9}

\hypertarget{felipe-predica-y-sana}{%
\subsection{Felipe predica y sana}\label{felipe-predica-y-sana}}

\bibverse{4} Mas los que fueron esparcidos, iban por todas partes
anunciando la palabra. \bibverse{5} Entonces Felipe, descendiendo á la
ciudad de Samaria, les predicaba á Cristo. \bibverse{6} Y las gentes
escuchaban atentamente unánimes las cosas que decía Felipe, oyendo y
viendo las señales que hacía. \bibverse{7} Porque de muchos que tenían
espíritus inmundos, salían éstos dando grandes voces; y muchos
paralíticos y cojos eran sanados: \footnote{\textbf{8:7} Mar 16,17}
\bibverse{8} Así que había gran gozo en aquella ciudad.

\hypertarget{el-mago-simuxf3n-en-samaria}{%
\subsection{El mago Simón en
Samaria}\label{el-mago-simuxf3n-en-samaria}}

\bibverse{9} Y había un hombre llamado Simón, el cual había sido antes
mágico en aquella ciudad, y había engañado la gente de Samaria,
diciéndose ser algún grande: \bibverse{10} Al cual oían todos
atentamente desde el más pequeño hasta el más grande, diciendo: Este es
la gran virtud de Dios. \bibverse{11} Y le estaban atentos, porque con
sus artes mágicas los había embelesado mucho tiempo. \bibverse{12} Mas
cuando creyeron á Felipe, que anunciaba el evangelio del reino de Dios y
el nombre de Jesucristo, se bautizaban hombres y mujeres. \bibverse{13}
El mismo Simón creyó también entonces, y bautizándose, se llegó á
Felipe: y viendo los milagros y grandes maravillas que se hacían, estaba
atónito.

\hypertarget{obra-de-pedro-y-juan-en-samaria}{%
\subsection{Obra de Pedro y Juan en
Samaria}\label{obra-de-pedro-y-juan-en-samaria}}

\bibverse{14} Y los apóstoles que estaban en Jerusalem, habiendo oído
que Samaria había recibido la palabra de Dios, les enviaron á Pedro y á
Juan: \bibverse{15} Los cuales venidos, oraron por ellos, para que
recibiesen el Espíritu Santo; \bibverse{16} (Porque aun no había
descendido sobre ninguno de ellos, mas solamente eran bautizados en el
nombre de Jesús.) \bibverse{17} Entonces les impusieron las manos, y
recibieron el Espíritu Santo. \bibverse{18} Y como vió Simón que por la
imposición de las manos de los apóstoles se daba el Espíritu Santo, les
ofreció dinero, \bibverse{19} Diciendo: Dadme también á mí esta
potestad, que á cualquiera que pusiere las manos encima, reciba el
Espíritu Santo. \bibverse{20} Entonces Pedro le dijo: Tu dinero perezca
contigo, que piensas que el don de Dios se gane por dinero.
\bibverse{21} No tienes tú parte ni suerte en este negocio; porque tu
corazón no es recto delante de Dios. \bibverse{22} Arrepiéntete pues de
esta tu maldad, y ruega á Dios, si quizás te será perdonado el
pensamiento de tu corazón. \bibverse{23} Porque en hiel de amargura y en
prisión de maldad veo que estás.

\bibverse{24} Respondiendo entonces Simón, dijo: Rogad vosotros por mí
al Señor, que ninguna cosa de estas que habéis dicho, venga sobre mí.

\bibverse{25} Y ellos, habiendo testificado y hablado la palabra de
Dios, se volvieron á Jerusalem, y en muchas tierras de los Samaritanos
anunciaron el evangelio.

\hypertarget{conversiuxf3n-y-bautismo-del-funcionario-de-la-corte-etuxedope-por-felipe}{%
\subsection{Conversión y bautismo del funcionario de la corte etíope por
Felipe}\label{conversiuxf3n-y-bautismo-del-funcionario-de-la-corte-etuxedope-por-felipe}}

\bibverse{26} Empero el ángel del Señor habló á Felipe, diciendo:
Levántate y ve hacia el mediodía, al camino que desciende de Jerusalem á
Gaza, el cual es desierto.

\bibverse{27} Entonces él se levantó, y fué: y he aquí un Etiope,
eunuco, gobernador de Candace, reina de los Etiopes, el cual era puesto
sobre todos sus tesoros, y había venido á adorar á Jerusalem,
\bibverse{28} Se volvía sentado en su carro, y leyendo el profeta
Isaías.

\bibverse{29} Y el Espíritu dijo á Felipe: Llégate, y júntate á este
carro.

\bibverse{30} Y acudiendo Felipe, le oyó que leía el profeta Isaías, y
dijo: Mas ¿entiendes lo que lees?

\bibverse{31} Y él dijo: ¿Y cómo podré, si alguno no me enseñare? Y rogó
á Felipe que subiese, y se sentase con él. \bibverse{32} Y el lugar de
la Escritura que leía, era éste: Como oveja á la muerte fué llevado; y
como cordero mudo delante del que le trasquila, así no abrió su boca:
\bibverse{33} En su humillación su juicio fué quitado: mas su
generación, ¿quién la contará? porque es quitada de la tierra su vida.

\bibverse{34} Y respondiendo el eunuco á Felipe, dijo: Ruégote ¿de quién
el profeta dice esto? ¿de sí, ó de otro alguno?

\bibverse{35} Entonces Felipe, abriendo su boca, y comenzando desde esta
escritura, le anunció el evangelio de Jesús. \bibverse{36} Y yendo por
el camino, llegaron á cierta agua; y dijo el eunuco: He aquí agua; ¿qué
impide que yo sea bautizado?

\bibverse{37} Y Felipe dijo: Si crees de todo corazón, bien puedes. Y
respondiendo, dijo: Creo que Jesucristo es el Hijo de Dios.
\bibverse{38} Y mandó parar el carro: y descendieron ambos al agua,
Felipe y el eunuco; y bautizóle.

\bibverse{39} Y como subieron del agua, el Espíritu del Señor arrebató á
Felipe; y no le vió más el eunuco, y se fué por su camino gozoso.
\footnote{\textbf{8:39} 1Re 18,12} \bibverse{40} Felipe empero se halló
en Azoto: y pasando, anunciaba el evangelio en todas las ciudades, hasta
que llegó á Cesarea. \footnote{\textbf{8:40} Hech 21,8-9}

\hypertarget{la-experiencia-de-saulo-camino-a-damasco}{%
\subsection{La experiencia de Saulo camino a
Damasco}\label{la-experiencia-de-saulo-camino-a-damasco}}

\hypertarget{section-8}{%
\section{9}\label{section-8}}

\bibverse{1} Y saulo, respirando aún amenazas y muerte contra los
discípulos del Señor, vino al príncipe de los sacerdotes, \footnote{\textbf{9:1}
  Hech 8,3; Hech 22,3-16; Hech 26,9-18} \bibverse{2} Y demandó de él
letras para Damasco á las sinagogas, para que si hallase algunos hombres
ó mujeres de esta secta, los trajese presos á Jerusalem. \bibverse{3} Y
yendo por el camino, aconteció que llegando cerca de Damasco,
súbitamente le cercó un resplandor de luz del cielo; \bibverse{4} Y
cayendo en tierra, oyó una voz que le decía: Saulo, Saulo, ¿por qué me
persigues?

\bibverse{5} Y él dijo: ¿Quién eres, Señor? Y él dijo: Yo soy Jesús á
quien tú persigues: dura cosa te es dar coces contra el aguijón.

\bibverse{6} El, temblando y temeroso, dijo: Señor, ¿qué quieres que
haga? Y el Señor le dice: Levántate y entra en la ciudad, y se te dirá
lo que te conviene hacer.

\bibverse{7} Y los hombres que iban con Saulo, se pararon atónitos,
oyendo á la verdad la voz, mas no viendo á nadie. \bibverse{8} Entonces
Saulo se levantó de tierra, y abriendo los ojos, no veía á nadie: así
que, llevándole por la mano, metiéronle en Damasco; \bibverse{9} Donde
estuvo tres días sin ver, y no comió, ni bebió.

\hypertarget{sanidad-y-bautismo-de-saulo-por-ananuxedas}{%
\subsection{Sanidad y bautismo de Saulo por
Ananías}\label{sanidad-y-bautismo-de-saulo-por-ananuxedas}}

\bibverse{10} Había entonces un discípulo en Damasco llamado Ananías, al
cual el Señor dijo en visión: Ananías. Y él respondió: Heme aquí, Señor.

\bibverse{11} Y el Señor le dijo: Levántate, y ve á la calle que se
llama la Derecha, y busca en casa de Judas á uno llamado Saulo, de
Tarso: porque he aquí, él ora; \bibverse{12} Y ha visto en visión un
varón llamado Ananías, que entra y le pone la mano encima, para que
reciba la vista.

\bibverse{13} Entonces Ananías respondió: Señor, he oído á muchos acerca
de este hombre, cuántos males ha hecho á tus santos en Jerusalem:
\bibverse{14} Y aun aquí tiene facultad de los príncipes de los
sacerdotes de prender á todos los que invocan tu nombre.

\bibverse{15} Y le dijo el Señor: Ve: porque instrumento escogido me es
éste, para que lleve mi nombre en presencia de los Gentiles, y de reyes,
y de los hijos de Israel: \footnote{\textbf{9:15} Hech 13,46; Hech 26,2;
  Hech 27,24} \bibverse{16} Porque yo le mostraré cuánto le sea menester
que padezca por mi nombre. \footnote{\textbf{9:16} 2Cor 11,23-28}

\bibverse{17} Ananías entonces fué, y entró en la casa, y poniéndole las
manos encima, dijo: Saulo hermano, el Señor Jesús, que te apareció en el
camino por donde venías, me ha enviado para que recibas la vista y seas
lleno de Espíritu Santo. \bibverse{18} Y luego le cayeron de los ojos
como escamas, y recibió al punto la vista: y levantándose, fué
bautizado. \bibverse{19} Y como comió, fué confortado. Y estuvo Saulo
por algunos días con los discípulos que estaban en Damasco.

\hypertarget{la-eficacia-de-pablo-en-damasco-y-su-huida}{%
\subsection{La eficacia de Pablo en Damasco y su
huida}\label{la-eficacia-de-pablo-en-damasco-y-su-huida}}

\bibverse{20} Y luego en las sinagogas predicaba á Cristo, diciendo que
éste era el Hijo de Dios. \bibverse{21} Y todos los que le oían estaban
atónitos, y decían: ¿No es éste el que asolaba en Jerusalem á los que
invocaban este nombre, y á eso vino acá, para llevarlos presos á los
príncipes de los sacerdotes? \footnote{\textbf{9:21} Hech 8,1; Hech
  26,10}

\bibverse{22} Empero Saulo mucho más se esforzaba, y confundía á los
Judíos que moraban en Damasco, afirmando que éste es el Cristo.
\footnote{\textbf{9:22} Hech 18,28} \bibverse{23} Y como pasaron muchos
días, los Judíos hicieron entre sí consejo de matarle; \bibverse{24} Mas
las asechanzas de ellos fueron entendidas de Saulo. Y ellos guardaban
las puertas de día y de noche para matarle. \bibverse{25} Entonces los
discípulos, tomándole de noche, le bajaron por el muro en una espuerta.
\footnote{\textbf{9:25} 2Cor 11,32-33}

\hypertarget{pablo-por-primera-vez-como-cristiano-en-jerusaluxe9n}{%
\subsection{Pablo por primera vez como cristiano en
Jerusalén}\label{pablo-por-primera-vez-como-cristiano-en-jerusaluxe9n}}

\bibverse{26} Y como vino á Jerusalem, tentaba de juntarse con los
discípulos; mas todos tenían miedo de él, no creyendo que era discípulo.
\footnote{\textbf{9:26} Gal 1,17-19} \bibverse{27} Entonces Bernabé,
tomándole, lo trajo á los apóstoles, y contóles cómo había visto al
Señor en el camino, y que le había hablado, y cómo en Damasco había
hablado confiadamente en el nombre de Jesús. \bibverse{28} Y entraba y
salía con ellos en Jerusalem; \bibverse{29} Y hablaba confiadamente en
el nombre del Señor: y disputaba con los Griegos; mas ellos procuraban
matarle. \bibverse{30} Lo cual, como los hermanos entendieron, le
acompañaron hasta Cesarea, y le enviaron á Tarso. \footnote{\textbf{9:30}
  Gal 1,21}

\hypertarget{milagros-de-pedro-en-lydda-y-jope}{%
\subsection{Milagros de Pedro en Lydda y
Jope}\label{milagros-de-pedro-en-lydda-y-jope}}

\bibverse{31} Las iglesias entonces tenían paz por toda Judea y Galilea
y Samaria, y eran edificadas, andando en el temor del Señor; y con
consuelo del Espíritu Santo eran multiplicadas.

\hypertarget{sanaciuxf3n-del-paralizado-eneas-en-lydda}{%
\subsection{Sanación del paralizado Eneas en
Lydda}\label{sanaciuxf3n-del-paralizado-eneas-en-lydda}}

\bibverse{32} Y aconteció que Pedro, andándolos á todos, vino también á
los santos que habitaban en Lydda. \bibverse{33} Y halló allí á uno que
se llamaba Eneas, que hacía ocho años que estaba en cama, que era
paralítico. \bibverse{34} Y le dijo Pedro: Eneas, Jesucristo te sana;
levántate, y hazte tu cama. Y luego se levantó. \bibverse{35} Y viéronle
todos los que habitaban en Lydda y en Sarona, los cuales se convirtieron
al Señor.

\hypertarget{criar-a-tabitha-en-joppe}{%
\subsection{Criar a Tabitha en Joppe}\label{criar-a-tabitha-en-joppe}}

\bibverse{36} Entonces en Joppe había una discípula llamada Tabita, que
si lo declaras, quiere decir Dorcas. Esta era llena de buenas obras y de
limosnas que hacía. \bibverse{37} Y aconteció en aquellos días que
enfermando, murió; á la cual, después de lavada, pusieron en una sala.
\bibverse{38} Y como Lydda estaba cerca de Joppe, los discípulos, oyendo
que Pedro estaba allí, le enviaron dos hombres, rogándole: No te
detengas en venir hasta nosotros. \bibverse{39} Pedro entonces
levantándose, fué con ellos: y llegado que hubo, le llevaron á la sala,
donde le rodearon todas las viudas, llorando y mostrando las túnicas y
los vestidos que Dorcas hacía cuando estaba con ellas. \bibverse{40}
Entonces echados fuera todos, Pedro puesto de rodillas, oró; y vuelto al
cuerpo, dijo: Tabita, levántate. Y ella abrió los ojos, y viendo á
Pedro, incorporóse. \bibverse{41} Y él le dió la mano, y levantóla:
entonces llamando á los santos y las viudas, la presentó viva.
\bibverse{42} Esto fué notorio por toda Joppe; y creyeron muchos en el
Señor. \bibverse{43} Y aconteció que se quedó muchos días en Joppe en
casa de un cierto Simón, curtidor.

\hypertarget{la-visiuxf3n-de-cornelio-en-cesarea}{%
\subsection{La visión de Cornelio en
Cesarea}\label{la-visiuxf3n-de-cornelio-en-cesarea}}

\hypertarget{section-9}{%
\section{10}\label{section-9}}

\bibverse{1} Y había un varón en Cesarea llamado Cornelio, centurión de
la compañía que se llamaba la Italiana, \bibverse{2} Pío y temeroso de
Dios con toda su casa, y que hacía muchas limosnas al pueblo, y oraba á
Dios siempre. \bibverse{3} Este vió en visión manifiestamente, como á la
hora nona del día, que un ángel de Dios entraba á él, y le decía:
Cornelio.

\bibverse{4} Y él, puestos en él los ojos, espantado, dijo: ¿Qué es,
Señor? Y díjole: Tus oraciones y tus limosnas han subido en memoria á la
presencia de Dios.

\bibverse{5} Envía pues ahora hombres á Joppe, y haz venir á un Simón,
que tiene por sobrenombre Pedro. \bibverse{6} Este posa en casa de un
Simón, curtidor, que tiene su casa junto á la mar: él te dirá lo que te
conviene hacer. \footnote{\textbf{10:6} Hech 9,43}

\bibverse{7} E ido el ángel que hablaba con Cornelio, llamó dos de sus
criados, y un devoto soldado de los que le asistían; \bibverse{8} A los
cuales, después de habérselo contado todo, los envió á Joppe.

\hypertarget{visiuxf3n-de-pedro-en-joppe-llegada-de-los-mensajeros-de-cornelio-a-pedro}{%
\subsection{Visión de Pedro en Joppe; Llegada de los mensajeros de
Cornelio a
Pedro}\label{visiuxf3n-de-pedro-en-joppe-llegada-de-los-mensajeros-de-cornelio-a-pedro}}

\bibverse{9} Y al día siguiente, yendo ellos su camino, y llegando cerca
de la ciudad, Pedro subió á la azotea á orar, cerca de la hora de sexta;
\bibverse{10} Y aconteció que le vino una grande hambre, y quiso comer;
pero mientras disponían, sobrevínole un éxtasis; \bibverse{11} Y vió el
cielo abierto, y que descendía un vaso, como un gran lienzo, que atado
de los cuatro cabos era bajado á la tierra; \bibverse{12} En el cual
había de todos los animales cuadrúpedos de la tierra, y reptiles, y aves
del cielo. \bibverse{13} Y le vino una voz: Levántate, Pedro, mata y
come.

\bibverse{14} Entonces Pedro dijo: Señor, no; porque ninguna cosa común
é inmunda he comido jamás. \footnote{\textbf{10:14} Ezeq 4,14; Lev 11,-1}

\bibverse{15} Y volvió la voz hacia él la segunda vez: Lo que Dios
limpió, no lo llames tú común. \footnote{\textbf{10:15} Rom 14,14}
\bibverse{16} Y esto fué hecho por tres veces; y el vaso volvió á ser
recogido en el cielo.

\bibverse{17} Y estando Pedro dudando dentro de sí qué sería la visión
que había visto, he aquí, los hombres que habían sido enviados por
Cornelio, que, preguntando por la casa de Simón, llegaron á la puerta.
\bibverse{18} Y llamando, preguntaron si un Simón que tenía por
sobrenombre Pedro, posaba allí. \bibverse{19} Y estando Pedro pensando
en la visión, le dijo el Espíritu: He aquí, tres hombres te buscan.
\bibverse{20} Levántate, pues, y desciende, y no dudes ir con ellos;
porque yo los he enviado.

\bibverse{21} Entonces Pedro, descendiendo á los hombres que eran
enviados por Cornelio, dijo: He aquí, yo soy el que buscáis: ¿cuál es la
causa por la que habéis venido?

\bibverse{22} Y ellos dijeron: Cornelio, el centurión, varón justo y
temeroso de Dios, y que tiene testimonio de toda la nación de los
Judíos, ha recibido respuesta por un santo ángel, de hacerte venir á su
casa, y oir de ti palabras.

\hypertarget{pedro-en-la-casa-de-cornelio}{%
\subsection{Pedro en la casa de
Cornelio}\label{pedro-en-la-casa-de-cornelio}}

\bibverse{23} Entonces metiéndolos dentro, los hospedó. Y al día
siguiente, levantándose, se fué con ellos; y le acompañaron algunos de
los hermanos de Joppe.

\bibverse{24} Y al otro día entraron en Cesarea. Y Cornelio los estaba
esperando, habiendo llamado á sus parientes y los amigos más familiares.
\bibverse{25} Y como Pedro entró, salió Cornelio á recibirle; y
derribándose á sus pies, adoró. \bibverse{26} Mas Pedro le levantó,
diciendo: Levántate; yo mismo también soy hombre. \footnote{\textbf{10:26}
  Hech 14,15; Apoc 19,10} \bibverse{27} Y hablando con él, entró, y
halló á muchos que se habían juntado. \bibverse{28} Y les dijo: Vosotros
sabéis que es abominable á un varón Judío juntarse ó llegarse á
extranjero; mas me ha mostrado Dios que á ningún hombre llame común ó
inmundo; \bibverse{29} Por lo cual, llamado, he venido sin dudar. Así
que pregunto: ¿por qué causa me habéis hecho venir?

\bibverse{30} Entonces Cornelio dijo: Cuatro días ha que á esta hora yo
estaba ayuno; y á la hora de nona estando orando en mi casa, he aquí, un
varón se puso delante de mí en vestido resplandeciente. \bibverse{31} Y
dijo: Cornelio, tu oración es oída, y tus limosnas han venido en memoria
en la presencia de Dios. \bibverse{32} Envía pues á Joppe, y haz venir á
un Simón, que tiene por sobrenombre Pedro; éste posa en casa de Simón,
curtidor, junto á la mar; el cual venido, te hablará. \bibverse{33} Así
que, luego envié á ti; y tú has hecho bien en venir. Ahora pues, todos
nosotros estamos aquí en la presencia de Dios, para oir todo lo que Dios
te ha mandado.

\bibverse{34} Entonces Pedro, abriendo su boca, dijo: Por verdad hallo
que Dios no hace acepción de personas; \bibverse{35} Sino que de
cualquiera nación que le teme y obra justicia, se agrada. \footnote{\textbf{10:35}
  Juan 10,16} \bibverse{36} Envió palabra Dios á los hijos de Israel,
anunciando la paz por Jesucristo; éste es el Señor de todos. \footnote{\textbf{10:36}
  Efes 2,17} \bibverse{37} Vosotros sabéis lo que fué divulgado por toda
Judea; comenzando desde Galilea después del bautismo que Juan predicó,
\footnote{\textbf{10:37} Mat 4,12-17} \bibverse{38} Cuanto á Jesús de
Nazaret; cómo le ungió Dios de Espíritu Santo y de potencia; el cual
anduvo haciendo bienes, y sanando á todos los oprimidos del diablo;
porque Dios era con él. \footnote{\textbf{10:38} Mat 3,16} \bibverse{39}
Y nosotros somos testigos de todas las cosas que hizo en la tierra de
Judea, y en Jerusalem; al cual mataron colgándole en un madero.
\bibverse{40} A éste levantó Dios al tercer día, é hizo que apareciese
manifiesto, \footnote{\textbf{10:40} 1Cor 15,4-7} \bibverse{41} No á
todo el pueblo, sino á los testigos que Dios antes había ordenado, es á
saber, á nosotros que comimos y bebimos con él, después que resucitó de
los muertos. \footnote{\textbf{10:41} Juan 14,19; Juan 14,22; Luc 24,30;
  Luc 24,43} \bibverse{42} Y nos mandó que predicásemos al pueblo, y
testificásemos que él es el que Dios ha puesto por Juez de vivos y
muertos. \footnote{\textbf{10:42} Juan 5,22} \bibverse{43} A éste dan
testimonio todos los profetas, de que todos los que en él creyeren,
recibirán perdón de pecados por su nombre. \footnote{\textbf{10:43} Is
  53,5-6; Jer 31,34}

\bibverse{44} Estando aún hablando Pedro estas palabras, el Espíritu
Santo cayó sobre todos los que oían el sermón. \bibverse{45} Y se
espantaron los fieles que eran de la circuncisión, que habían venido con
Pedro, de que también sobre los Gentiles se derramase el don del
Espíritu Santo. \bibverse{46} Porque los oían que hablaban en lenguas, y
que magnificaban á Dios. \footnote{\textbf{10:46} Hech 2,4}

\bibverse{47} Entonces respondió Pedro: ¿Puede alguno impedir el agua,
para que no sean bautizados éstos que han recibido el Espíritu Santo
también como nosotros? \bibverse{48} Y les mandó bautizar en el nombre
del Señor Jesús. Entonces le rogaron que se quedase por algunos días.

\hypertarget{pedro-justifica-el-bautismo-pagano-en-jerusaluxe9n}{%
\subsection{Pedro justifica el bautismo pagano en
Jerusalén}\label{pedro-justifica-el-bautismo-pagano-en-jerusaluxe9n}}

\hypertarget{section-10}{%
\section{11}\label{section-10}}

\bibverse{1} Y oyeron los apóstoles y los hermanos que estaban en Judea,
que también los Gentiles habían recibido la palabra de Dios.
\bibverse{2} Y como Pedro subió á Jerusalem, contendían contra él los
que eran de la circuncisión, \bibverse{3} Diciendo: ¿Por qué has entrado
á hombres incircuncisos, y has comido con ellos?

\bibverse{4} Entonces comenzando Pedro, les declaró por orden lo pasado,
diciendo: \bibverse{5} Estaba yo en la ciudad de Joppe orando, y vi en
rapto de entendimiento una visión: un vaso, como un gran lienzo, que
descendía, que por los cuatro cabos era abajado del cielo, y venía hasta
mí. \footnote{\textbf{11:5} Hech 10,9-48} \bibverse{6} En el cual como
puse los ojos, consideré y vi animales terrestres de cuatro pies, y
fieras, y reptiles, y aves del cielo. \bibverse{7} Y oí una voz que me
decía: Levántate, Pedro, mata y come. \bibverse{8} Y dije: Señor, no;
porque ninguna cosa común ó inmunda entró jamás en mi boca. \bibverse{9}
Entonces la voz me respondió del cielo segunda vez: Lo que Dios limpió,
no lo llames tú común. \bibverse{10} Y esto fué hecho por tres veces: y
volvió todo á ser tomado arriba en el cielo. \bibverse{11} Y he aquí,
luego sobrevinieron tres hombres á la casa donde yo estaba, enviados á
mí de Cesarea. \bibverse{12} Y el Espíritu me dijo que fuese con ellos
sin dudar. Y vinieron también conmigo estos seis hermanos, y entramos en
casa de un varón, \bibverse{13} El cual nos contó cómo había visto un
ángel en su casa, que se paró, y le dijo: Envía á Joppe, y haz venir á
un Simón que tiene por sobrenombre Pedro; \bibverse{14} El cual te
hablará palabras por las cuales serás salvo tú, y toda tu casa.
\bibverse{15} Y como comencé á hablar, cayó el Espíritu Santo sobre
ellos también, como sobre nosotros al principio. \bibverse{16} Entonces
me acordé del dicho del Señor, como dijo: Juan ciertamente bautizó en
agua; mas vosotros seréis bautizados en Espíritu Santo. \footnote{\textbf{11:16}
  Hech 1,5} \bibverse{17} Así que, si Dios les dió el mismo don también
como á nosotros que hemos creído en el Señor Jesucristo, ¿quién era yo
que pudiese estorbar á Dios?

\bibverse{18} Entonces, oídas estas cosas, callaron, y glorificaron á
Dios, diciendo: De manera que también á los Gentiles ha dado Dios
arrepentimiento para vida.

\hypertarget{fundaciuxf3n-de-la-primera-comunidad-cristiana-gentil-en-antioquuxeda-en-siria-su-ayuda-para-los-cristianos-necesitados-en-judea}{%
\subsection{Fundación de la primera comunidad cristiana gentil en
Antioquía en Siria; su ayuda para los cristianos necesitados en
Judea}\label{fundaciuxf3n-de-la-primera-comunidad-cristiana-gentil-en-antioquuxeda-en-siria-su-ayuda-para-los-cristianos-necesitados-en-judea}}

\bibverse{19} Y los que habían sido esparcidos por causa de la
tribulación que sobrevino en tiempo de Esteban, anduvieron hasta
Fenicia, y Cipro, y Antioquía, no hablando á nadie la palabra, sino sólo
á los Judíos. \bibverse{20} Y de ellos había unos varones Ciprios y
Cirenenses, los cuales como entraron en Antioquía, hablaron á los
Griegos, anunciando el evangelio del Señor Jesús. \bibverse{21} Y la
mano del Señor era con ellos: y creyendo, gran número se convirtió al
Señor. \footnote{\textbf{11:21} Hech 2,47} \bibverse{22} Y llegó la fama
de estas cosas á oídos de la iglesia que estaba en Jerusalem: y enviaron
á Bernabé que fuese hasta Antioquía. \footnote{\textbf{11:22} Hech 4,36}
\bibverse{23} El cual, como llegó, y vió la gracia de Dios, regocijóse;
y exhortó á todos á que permaneciesen en el propósito del corazón en el
Señor. \bibverse{24} Porque era varón bueno, y lleno de Espíritu Santo y
de fe: y mucha compañía fué agregada al Señor. \footnote{\textbf{11:24}
  Hech 5,14}

\bibverse{25} Después partió Bernabé á Tarso á buscar á Saulo; y
hallado, le trajo á Antioquía. \footnote{\textbf{11:25} Hech 9,30}
\bibverse{26} Y conversaron todo un año allí con la iglesia, y enseñaron
á mucha gente; y los discípulos fueron llamados Cristianos primeramente
en Antioquía. \footnote{\textbf{11:26} Gal 2,11}

\bibverse{27} Y en aquellos días descendieron de Jerusalem profetas á
Antioquía. \footnote{\textbf{11:27} Hech 13,1; Hech 15,32} \bibverse{28}
Y levantándose uno de ellos, llamado Agabo, daba á entender por
Espíritu, que había de haber una grande hambre en toda la tierra
habitada: la cual hubo en tiempo de Claudio. \footnote{\textbf{11:28}
  Hech 21,10}

\bibverse{29} Entonces los discípulos, cada uno conforme á lo que tenía,
determinaron enviar subsidio á los hermanos que habitaban en Judea:
\bibverse{30} Lo cual asimismo hicieron, enviándolo á los ancianos por
mano de Bernabé y de Saulo.

\hypertarget{muerte-de-santiago-arresto-de-pedro}{%
\subsection{Muerte de Santiago, arresto de
Pedro}\label{muerte-de-santiago-arresto-de-pedro}}

\hypertarget{section-11}{%
\section{12}\label{section-11}}

\bibverse{1} Y en el mismo tiempo el rey Herodes echó mano á maltratar
algunos de la iglesia. \bibverse{2} Y mató á cuchillo á Jacobo, hermano
de Juan. \footnote{\textbf{12:2} Mat 20,20-43} \bibverse{3} Y viendo que
había agradado á los Judíos, pasó adelante para prender también á Pedro.
Eran entonces los días de los ázimos. \bibverse{4} Y habiéndole preso,
púsole en la cárcel, entregándole á cuatro cuaterniones de soldados que
le guardasen; queriendo sacarle al pueblo después de la Pascua.
\bibverse{5} Así que, Pedro era guardado en la cárcel; y la iglesia
hacía sin cesar oración á Dios por él.

\hypertarget{maravillosa-salvaciuxf3n-de-pedro}{%
\subsection{Maravillosa salvación de
Pedro}\label{maravillosa-salvaciuxf3n-de-pedro}}

\bibverse{6} Y cuando Herodes le había de sacar, aquella misma noche
estaba Pedro durmiendo entre dos soldados, preso con dos cadenas, y los
guardas delante de la puerta, que guardaban la cárcel.

\bibverse{7} Y he aquí, el ángel del Señor sobrevino, y una luz
resplandeció en la cárcel; é hiriendo á Pedro en el lado, le despertó,
diciendo: Levántate prestamente. Y las cadenas se le cayeron de las
manos. \bibverse{8} Y le dijo el ángel: Cíñete, y átate tus sandalias. Y
lo hizo así. Y le dijo: Rodéate tu ropa, y sígueme. \bibverse{9} Y
saliendo, le seguía; y no sabía que era verdad lo que hacía el ángel,
mas pensaba que veía visión. \bibverse{10} Y como pasaron la primera y
la segunda guardia, vinieron á la puerta de hierro que va á la ciudad,
la cual se les abrió de suyo: y salidos, pasaron una calle; y luego el
ángel se apartó de él.

\bibverse{11} Entonces Pedro, volviendo en sí, dijo: Ahora entiendo
verdaderamente que el Señor ha enviado su ángel, y me ha librado de la
mano de Herodes, y de todo el pueblo de los Judíos que me esperaba.
\bibverse{12} Y habiendo considerado esto, llegó á casa de María la
madre de Juan, el que tenía por sobrenombre Marcos, donde muchos estaban
juntos orando. \footnote{\textbf{12:12} Hech 12,25; Hech 13,5; Hech
  13,13; Hech 15,37} \bibverse{13} Y tocando Pedro á la puerta del
patio, salió una muchacha, para escuchar, llamada Rhode: \bibverse{14}
La cual como conoció la voz de Pedro, de gozo no abrió el postigo, sino
corriendo adentro, dió nueva de que Pedro estaba al postigo.

\bibverse{15} Y ellos le dijeron: Estás loca. Mas ella afirmaba que así
era. Entonces ellos decían: Su ángel es. \bibverse{16} Mas Pedro
perseveraba en llamar: y cuando abrieron, viéronle, y se espantaron.
\bibverse{17} Mas él haciéndoles con la mano señal de que callasen, les
contó cómo el Señor le había sacado de la cárcel. Y dijo: Haced saber
esto á Jacobo y á los hermanos. Y salió, y partió á otro lugar.

\hypertarget{ira-de-herodes-su-cauxedda-en-cesarea-por-un-juicio-divino}{%
\subsection{Ira de Herodes; su caída en Cesarea por un juicio
divino}\label{ira-de-herodes-su-cauxedda-en-cesarea-por-un-juicio-divino}}

\bibverse{18} Luego que fué de día, hubo no poco alboroto entre los
soldados sobre qué se había hecho de Pedro. \bibverse{19} Mas Herodes,
como le buscó y no le halló, hecha inquisición de los guardas, los mandó
llevar. Después descendiendo de Judea á Cesarea, se quedó allí.

\bibverse{20} Y Herodes estaba enojado contra los de Tiro y los de
Sidón: mas ellos vinieron concordes á él, y sobornado Blasto, que era el
camarero del rey, pedían paz; porque las tierras de ellos eran
abastecidas por las del rey. \footnote{\textbf{12:20} 1Re 5,25; Ezeq
  27,17} \bibverse{21} Y un día señalado, Herodes vestido de ropa real,
se sentó en el tribunal, y arengóles. \bibverse{22} Y el pueblo
aclamaba: Voz de Dios, y no de hombre. \bibverse{23} Y luego el ángel
del Señor le hirió, por cuanto no dió la gloria á Dios; y espiró comido
de gusanos. \footnote{\textbf{12:23} Dan 5,20}

\bibverse{24} Mas la palabra del Señor crecía y era multiplicada.
\footnote{\textbf{12:24} Hech 6,7; Is 55,11} \bibverse{25} Y Bernabé y
Saulo volvieron de Jerusalem cumplido su servicio, tomando también
consigo á Juan, el que tenía por sobrenombre Marcos. \footnote{\textbf{12:25}
  Hech 11,29-30; Hech 13,5}

\hypertarget{consagraciuxf3n-envuxedo-y-partida-de-pablo-y-bernabuxe9-su-eficacia-en-chipre}{%
\subsection{Consagración, envío y partida de Pablo y Bernabé; su
eficacia en
Chipre}\label{consagraciuxf3n-envuxedo-y-partida-de-pablo-y-bernabuxe9-su-eficacia-en-chipre}}

\hypertarget{section-12}{%
\section{13}\label{section-12}}

\bibverse{1} Había entonces en la iglesia que estaba en Antioquía,
profetas y doctores: Bernabé, y Simón el que se llamaba Niger, y Lucio
Cireneo, y Manahén, que había sido criado con Herodes el tetrarca, y
Saulo. \footnote{\textbf{13:1} Hech 11,27; 1Cor 12,28} \bibverse{2}
Ministrando pues éstos al Señor, y ayunando, dijo el Espíritu Santo:
Apartadme á Bernabé y á Saulo para la obra para la cual los he llamado.
\footnote{\textbf{13:2} Hech 9,15}

\bibverse{3} Entonces habiendo ayunado y orado, y puesto las manos
encima de ellos, despidiéronlos. \footnote{\textbf{13:3} Hech 6,6}
\bibverse{4} Y ellos, enviados así por el Espíritu Santo, descendieron á
Seleucia; y de allí navegaron á Cipro. \bibverse{5} Y llegados á
Salamina, anunciaban la palabra de Dios en las sinagogas de los Judíos:
y tenían también á Juan en el ministerio. \footnote{\textbf{13:5} Hech
  12,12; Hech 12,25} \bibverse{6} Y habiendo atravesado toda la isla
hasta Papho, hallaron un hombre mago, falso profeta, Judío, llamado
Barjesús; \bibverse{7} El cual estaba con el procónsul Sergio Paulo,
varón prudente. Este, llamando á Bernabé y á Saulo, deseaba oir la
palabra de Dios. \bibverse{8} Mas les resistía Elimas el encantador (que
así se interpreta su nombre), procurando apartar de la fe al procónsul.
\bibverse{9} Entonces Saulo, que también es Pablo, lleno del Espíritu
Santo, poniendo en él los ojos, \bibverse{10} Dijo: Oh, lleno de todo
engaño y de toda maldad, hijo del diablo, enemigo de toda justicia, ¿no
cesarás de trastornar los caminos rectos del Señor? \bibverse{11} Ahora
pues, he aquí la mano del Señor es contra ti, y serás ciego, que no veas
el sol por tiempo. Y luego cayeron en él obscuridad y tinieblas; y
andando alrededor, buscaba quién le condujese por la mano.

\bibverse{12} Entonces el procónsul, viendo lo que había sido hecho,
creyó, maravillado de la doctrina del Señor.

\hypertarget{continuaciuxf3n-del-viaje-a-asia-menor-y-estancia-en-antioquuxeda-de-pisidia}{%
\subsection{Continuación del viaje a Asia Menor y estancia en Antioquía
de
Pisidia}\label{continuaciuxf3n-del-viaje-a-asia-menor-y-estancia-en-antioquuxeda-de-pisidia}}

\bibverse{13} Y partidos de Papho, Pablo y sus compañeros arribaron á
Perge de Pamphylia: entonces Juan, apartándose de ellos, se volvió á
Jerusalem. \bibverse{14} Y ellos pasando de Perge, llegaron á Antioquía
de Pisidia, y entrando en la sinagoga un día de sábado, sentáronse.
\bibverse{15} Y después de la lectura de la ley y de los profetas, los
príncipes de la sinagoga enviaron á ellos, diciendo: Varones hermanos,
si tenéis alguna palabra de exhortación para el pueblo, hablad.
\footnote{\textbf{13:15} Hech 15,21}

\bibverse{16} Entonces Pablo, levantándose, hecha señal de silencio con
la mano, dice: Varones Israelitas, y los que teméis á Dios, oid:
\bibverse{17} El Dios del pueblo de Israel escogió á nuestros padres, y
ensalzó al pueblo, siendo ellos extranjeros en la tierra de Egipto, y
con brazo levantado los sacó de ella. \bibverse{18} Y por tiempo como de
cuarenta años soportó sus costumbres en el desierto; \footnote{\textbf{13:18}
  Éxod 16,35} \bibverse{19} Y destruyendo siete naciones en la tierra de
Canaán, les repartió por suerte la tierra de ellas. \footnote{\textbf{13:19}
  Deut 7,1; Jos 14,2} \bibverse{20} Y después, como por cuatrocientos y
cincuenta años, dióles jueces hasta el profeta Samuel. \footnote{\textbf{13:20}
  Jue 2,16; 1Sam 3,20} \bibverse{21} Y entonces demandaron rey; y les
dió Dios á Saúl, hijo de Cis, varón de la tribu de Benjamín, por
cuarenta años. \footnote{\textbf{13:21} 1Sam 8,5; 1Sam 10,21; 1Sam 10,24}
\bibverse{22} Y quitado aquél, levantóles por rey á David, al que dió
también testimonio, diciendo: He hallado á David, hijo de Jessé, varón
conforme á mi corazón, el cual hará todo lo que yo quiero. \bibverse{23}
De la simiente de éste, Dios, conforme á la promesa, levantó á Jesús por
Salvador á Israel; \footnote{\textbf{13:23} Is 11,1} \bibverse{24}
Predicando Juan delante de la faz de su venida el bautismo de
arrepentimiento á todo el pueblo de Israel. \footnote{\textbf{13:24} Luc
  3,3} \bibverse{25} Mas como Juan cumpliese su carrera, dijo: ¿Quién
pensáis que soy? No soy yo él; mas he aquí, viene tras mí uno, cuyo
calzado de los pies no soy digno de desatar. \footnote{\textbf{13:25}
  Juan 1,20; Juan 1,27; Luc 3,16; Mar 1,7}

\bibverse{26} Varones hermanos, hijos del linaje de Abraham, y los que
entre vosotros temen á Dios, á vosotros es enviada la palabra de esta
salud. \bibverse{27} Porque los que habitaban en Jerusalem, y sus
príncipes, no conociendo á éste, y las voces de los profetas que se leen
todos los sábados, condenándole, las cumplieron. \bibverse{28} Y sin
hallar en él causa de muerte, pidieron á Pilato que le matasen.
\footnote{\textbf{13:28} Mat 27,22-23} \bibverse{29} Y habiendo cumplido
todas las cosas que de él estaban escritas, quitándolo del madero, lo
pusieron en el sepulcro. \footnote{\textbf{13:29} Mat 27,59-60}
\bibverse{30} Mas Dios le levantó de los muertos. \footnote{\textbf{13:30}
  Hech 3,15} \bibverse{31} Y él fué visto por muchos días de los que
habían subido juntamente con él de Galilea á Jerusalem, los cuales son
sus testigos al pueblo. \footnote{\textbf{13:31} Hech 1,3} \bibverse{32}
Y nosotros también os anunciamos el evangelio de aquella promesa que fué
hecha á los padres, \bibverse{33} La cual Dios ha cumplido á los hijos
de ellos, á nosotros, resucitando á Jesús: como también en el salmo
segundo está escrito: Mi hijo eres tú, yo te he engendrado hoy.

\bibverse{34} Y que le levantó de los muertos para nunca más volver á
corrupción, así lo dijo: Os daré las misericordias fieles de David.
\bibverse{35} Por eso dice también en otro lugar: No permitirás que tu
Santo vea corrupción. \bibverse{36} Porque á la verdad David, habiendo
servido en su edad á la voluntad de Dios, durmió, y fué juntado con sus
padres, y vió corrupción: \bibverse{37} Mas aquel que Dios levantó, no
vió corrupción. \bibverse{38} Séaos pues notorio, varones hermanos, que
por éste os es anunciada remisión de pecados; \bibverse{39} Y de todo lo
que por la ley de Moisés no pudisteis ser justificados, en éste es
justificado todo aquel que creyere. \footnote{\textbf{13:39} Rom 8,3-4;
  Rom 10,4} \bibverse{40} Mirad, pues, que no venga sobre vosotros lo
que está dicho en los profetas; \bibverse{41} Mirad, oh
menospreciadores, y entonteceos, y desvaneceos; porque yo obro una obra
en vuestros días, obra que no creeréis, si alguien os la contare.

\hypertarget{varios-uxe9xitos-del-discurso}{%
\subsection{Varios éxitos del
discurso}\label{varios-uxe9xitos-del-discurso}}

\bibverse{42} Y saliendo ellos de la sinagoga de los Judíos, los
Gentiles les rogaron que el sábado siguiente les hablasen estas
palabras. \bibverse{43} Y despedida la congregación, muchos de los
Judíos y de los religiosos prosélitos siguieron á Pablo y á Bernabé; los
cuales hablándoles, les persuadían que permaneciesen en la gracia de
Dios.

\bibverse{44} Y el sábado siguiente se juntó casi toda la ciudad á oir
la palabra de Dios. \bibverse{45} Mas los Judíos, visto el gentío,
llenáronse de celo, y se oponían á lo que Pablo decía, contradiciendo y
blasfemando.

\bibverse{46} Entonces Pablo y Bernabé, usando de libertad, dijeron: A
vosotros á la verdad era menester que se os hablase la palabra de Dios;
mas pues que la desecháis, y os juzgáis indignos de la vida eterna, he
aquí, nos volvemos á los Gentiles. \bibverse{47} Porque así nos ha
mandado el Señor, diciendo: Te he puesto para luz de los Gentiles, para
que seas salud hasta lo postrero de la tierra.

\bibverse{48} Y los Gentiles oyendo esto, fueron gozosos, y glorificaban
la palabra del Señor: y creyeron todos los que estaban ordenados para
vida eterna. \footnote{\textbf{13:48} Rom 8,29-30}

\bibverse{49} Y la palabra del Señor era sembrada por toda aquella
provincia. \bibverse{50} Mas los Judíos concitaron mujeres pías y
honestas, y á los principales de la ciudad, y levantaron persecución
contra Pablo y Bernabé, y los echaron de sus términos. \bibverse{51}
Ellos entonces sacudiendo en ellos el polvo de sus pies, vinieron á
Iconio. \bibverse{52} Y los discípulos estaban llenos de gozo, y del
Espíritu Santo.

\hypertarget{efectividad-de-los-apuxf3stoles-en-iconio}{%
\subsection{Efectividad de los Apóstoles en
Iconio}\label{efectividad-de-los-apuxf3stoles-en-iconio}}

\hypertarget{section-13}{%
\section{14}\label{section-13}}

\bibverse{1} Y aconteció en Iconio, que entrados juntamente en la
sinagoga de los Judíos, hablaron de tal manera, que creyó una grande
multitud de Judíos, y asimismo de Griegos. \bibverse{2} Mas los Judíos
que fueron incrédulos, incitaron y corrompieron los ánimos de los
Gentiles contra los hermanos. \bibverse{3} Con todo eso se detuvieron
allí mucho tiempo, confiados en el Señor, el cual daba testimonio á la
palabra de su gracia, dando que señales y milagros fuesen hechos por las
manos de ellos. \footnote{\textbf{14:3} Hech 19,11; Heb 2,4}
\bibverse{4} Mas el vulgo de la ciudad estaba dividido; y unos eran con
los Judíos, y otros con los apóstoles. \bibverse{5} Y haciendo ímpetu
los Judíos y los Gentiles juntamente con sus príncipes, para afrentarlos
y apedrearlos, \bibverse{6} Habiéndolo entendido, huyeron á Listra y
Derbe, ciudades de Licaonia, y por toda la tierra alrededor.
\bibverse{7} Y allí predicaban el evangelio.

\hypertarget{curaciuxf3n-de-un-cojo-y-lapidaciuxf3n-de-pablo-en-listra-los-dos-apuxf3stoles-escapan-a-derbe}{%
\subsection{Curación de un cojo y lapidación de Pablo en Listra; los dos
apóstoles escapan a
Derbe}\label{curaciuxf3n-de-un-cojo-y-lapidaciuxf3n-de-pablo-en-listra-los-dos-apuxf3stoles-escapan-a-derbe}}

\bibverse{8} Y un hombre de Listra, impotente de los pies, estaba
sentado, cojo desde el vientre de su madre, que jamás había andado.
\bibverse{9} Este oyó hablar á Pablo; el cual, como puso los ojos en él,
y vió que tenía fe para ser sano, \footnote{\textbf{14:9} Mat 9,28}
\bibverse{10} Dijo á gran voz: Levántate derecho sobre tus pies. Y
saltó, y anduvo. \bibverse{11} Entonces las gentes, visto lo que Pablo
había hecho, alzaron la voz, diciendo en lengua licaónica: Dioses
semejantes á hombres han descendido á nosotros. \bibverse{12} Y á
Bernabé llamaban Júpiter, y á Pablo, Mercurio, porque era el que llevaba
la palabra. \bibverse{13} Y el sacerdote de Júpiter, que estaba delante
de la ciudad de ellos, trayendo toros y guirnaldas delante de las
puertas, quería con el pueblo sacrificar.

\bibverse{14} Y como lo oyeron los apóstoles Bernabé y Pablo, rotas sus
ropas, se lanzaron al gentío, dando voces, \bibverse{15} Y diciendo:
Varones, ¿por qué hacéis esto? Nosotros también somos hombres semejantes
á vosotros, que os anunciamos que de estas vanidades os convirtáis al
Dios vivo, que hizo el cielo y la tierra, y la mar, y todo lo que está
en ellos: \footnote{\textbf{14:15} Hech 10,26} \bibverse{16} El cual en
las edades pasadas ha dejado á todas las gentes andar en sus caminos;
\footnote{\textbf{14:16} Hech 17,30} \bibverse{17} Si bien no se dejó á
sí mismo sin testimonio, haciendo bien, dándonos lluvias del cielo y
tiempos fructíferos, hinchiendo de mantenimiento y de alegría nuestros
corazones.

\bibverse{18} Y diciendo estas cosas, apenas apaciguaron el pueblo, para
que no les ofreciesen sacrificio. \bibverse{19} Entonces sobrevinieron
unos Judíos de Antioquía y de Iconio, que persuadieron á la multitud, y
habiendo apedreado á Pablo, le sacaron fuera de la ciudad, pensando que
estaba muerto. \footnote{\textbf{14:19} 2Cor 11,25; 2Tim 3,11}

\hypertarget{los-apuxf3stoles-en-derbe-fortalecimiento-de-las-comunidades-fundadas-regreso-a-antioquuxeda-en-siria}{%
\subsection{Los apóstoles en Derbe; Fortalecimiento de las comunidades
fundadas; Regreso a Antioquía en
Siria}\label{los-apuxf3stoles-en-derbe-fortalecimiento-de-las-comunidades-fundadas-regreso-a-antioquuxeda-en-siria}}

\bibverse{20} Mas rodeándole los discípulos, se levantó y entró en la
ciudad; y un día después, partió con Bernabé á Derbe.

\bibverse{21} Y como hubieron anunciado el evangelio á aquella ciudad, y
enseñado á muchos, volvieron á Listra, y á Iconio, y á Antioquía,
\bibverse{22} Confirmando los ánimos de los discípulos, exhortándoles á
que permaneciesen en la fe, y que es menester que por muchas
tribulaciones entremos en el reino de Dios. \bibverse{23} Y habiéndoles
constituído ancianos en cada una de las iglesias, y habiendo orado con
ayunos, los encomendaron al Señor en el cual habían creído. \footnote{\textbf{14:23}
  Hech 6,6}

\bibverse{24} Y pasando por Pisidia vinieron á Pamphylia. \bibverse{25}
Y habiendo predicado la palabra en Perge, descendieron á Atalia;
\bibverse{26} Y de allí navegaron á Antioquía, donde habían sido
encomendados á la gracia de Dios para la obra que habían acabado.
\bibverse{27} Y habiendo llegado, y reunido la iglesia, relataron cuán
grandes cosas había Dios hecho con ellos, y cómo había abierto á los
Gentiles la puerta de la fe. \footnote{\textbf{14:27} 1Cor 16,9}

\bibverse{28} Y se quedaron allí mucho tiempo con los discípulos.

\hypertarget{la-causa-de-la-convenciuxf3n-envuxedo-de-pablo-y-bernabuxe9-a-jerusaluxe9n}{%
\subsection{La causa de la Convención; Envío de Pablo y Bernabé a
Jerusalén}\label{la-causa-de-la-convenciuxf3n-envuxedo-de-pablo-y-bernabuxe9-a-jerusaluxe9n}}

\hypertarget{section-14}{%
\section{15}\label{section-14}}

\bibverse{1} Entonces algunos que venían de Judea enseñaban á los
hermanos: Que si no os circuncidáis conforme al rito de Moisés, no
podéis ser salvos. \bibverse{2} Así que, suscitada una disensión y
contienda no pequeña á Pablo y á Bernabé contra ellos, determinaron que
subiesen Pablo y Bernabé á Jerusalem, y algunos otros de ellos, á los
apóstoles y á los ancianos, sobre esta cuestión. \footnote{\textbf{15:2}
  Gal 2,1} \bibverse{3} Ellos, pues, habiendo sido acompañados de la
iglesia, pasaron por la Fenicia y Samaria, contando la conversión de los
Gentiles; y daban gran gozo á todos los hermanos. \bibverse{4} Y
llegados á Jerusalem, fueron recibidos de la iglesia y de los apóstoles
y de los ancianos: y refirieron todas las cosas que Dios había hecho con
ellos.

\bibverse{5} Mas algunos de la secta de los Fariseos, que habían creído,
se levantaron, diciendo: Que es menester circuncidarlos, y mandarles que
guarden la ley de Moisés.

\hypertarget{las-negociaciones-discursos-de-pedro-y-santiago}{%
\subsection{Las negociaciones; Discursos de Pedro y
Santiago}\label{las-negociaciones-discursos-de-pedro-y-santiago}}

\bibverse{6} Y se juntaron los apóstoles y los ancianos para conocer de
este negocio. \bibverse{7} Y habiendo habido grande contienda,
levantándose Pedro, les dijo: Varones hermanos, vosotros sabéis cómo ya
hace algún tiempo que Dios escogió que los Gentiles oyesen por mi boca
la palabra del evangelio, y creyesen. \bibverse{8} Y Dios, que conoce
los corazones, les dió testimonio, dándoles el Espíritu Santo también
como á nosotros; \bibverse{9} Y ninguna diferencia hizo entre nosotros y
ellos, purificando con la fe sus corazones. \bibverse{10} Ahora pues,
¿por qué tentáis á Dios, poniendo sobre la cerviz de los discípulos
yugo, que ni nuestros padres ni nosotros hemos podido llevar?
\footnote{\textbf{15:10} Mat 23,4; Gal 5,1} \bibverse{11} Antes por la
gracia del Señor Jesús creemos que seremos salvos, como también ellos.
\footnote{\textbf{15:11} Gal 2,16; Efes 2,4-10}

\bibverse{12} Entonces toda la multitud calló, y oyeron á Bernabé y á
Pablo, que contaban cuán grandes maravillas y señales Dios había hecho
por ellos entre los Gentiles. \bibverse{13} Y después que hubieron
callado, Jacobo respondió, diciendo: Varones hermanos, oidme:
\footnote{\textbf{15:13} Hech 21,18; Gal 2,9} \bibverse{14} Simón ha
contado cómo Dios primero visitó á los Gentiles, para tomar de ellos
pueblo para su nombre; \bibverse{15} Y con esto concuerdan las palabras
de los profetas, como está escrito: \bibverse{16} Después de esto
volveré y restauraré la habitación de David, que estaba caída; y
repararé sus ruinas, y la volveré á levantar; \bibverse{17} Para que el
resto de los hombres busque al Señor, y todos los Gentiles, sobre los
cuales es llamado mi nombre, dice el Señor, que hace todas estas cosas.

\bibverse{18} Conocidas son á Dios desde el siglo todas sus obras.
\bibverse{19} Por lo cual yo juzgo, que los que de los Gentiles se
convierten á Dios, no han de ser inquietados; \bibverse{20} Sino
escribirles que se aparten de las contaminaciones de los ídolos, y de
fornicación, y de ahogado, y de sangre. \bibverse{21} Porque Moisés
desde los tiempos antiguos tiene en cada ciudad quien le predique en las
sinagogas, donde es leído cada sábado. \footnote{\textbf{15:21} Hech
  13,15}

\hypertarget{la-resoluciuxf3n-y-su-implementaciuxf3n}{%
\subsection{La resolución y su
implementación}\label{la-resoluciuxf3n-y-su-implementaciuxf3n}}

\bibverse{22} Entonces pareció bien á los apóstoles y á los ancianos,
con toda la iglesia, elegir varones de ellos, y enviarlos á Antioquía
con Pablo y Bernabé: á Judas que tenía por sobrenombre Barsabas, y á
Silas, varones principales entre los hermanos; \bibverse{23} Y escribir
por mano de ellos: Los apóstoles y los ancianos y los hermanos, á los
hermanos de los Gentiles que están en Antioquía, y en Siria, y en
Cilicia, salud:

\bibverse{24} Por cuanto hemos oído que algunos que han salido de
nosotros, os han inquietado con palabras, trastornando vuestras almas,
mandando circuncidaros y guardar la ley, á los cuales no mandamos;
\bibverse{25} Nos ha parecido, congregados en uno, elegir varones, y
enviarlos á vosotros con nuestros amados Bernabé y Pablo, \bibverse{26}
Hombres que han expuesto sus vidas por el nombre de nuestro Señor
Jesucristo. \bibverse{27} Así que, enviamos á Judas y á Silas, los
cuales también por palabra os harán saber lo mismo. \bibverse{28} Que ha
parecido bien al Espíritu Santo, y á nosotros, no imponeros ninguna
carga más que estas cosas necesarias: \bibverse{29} Que os abstengáis de
cosas sacrificadas á ídolos, y de sangre, y de ahogado, y de
fornicación; de las cuales cosas si os guardareis, bien haréis. Pasadlo
bien.

\hypertarget{el-resultado-judas-y-silas-en-antioquuxeda}{%
\subsection{El resultado: Judas y Silas en
Antioquía}\label{el-resultado-judas-y-silas-en-antioquuxeda}}

\bibverse{30} Ellos entonces enviados, descendieron á Antioquía; y
juntando la multitud, dieron la carta. \bibverse{31} La cual, como
leyeron, fueron gozosos de la consolación. \bibverse{32} Judas también y
Silas, como ellos también eran profetas, consolaron y confirmaron á los
hermanos con abundancia de palabra. \bibverse{33} Y pasando allí algún
tiempo, fueron enviados de los hermanos á los apóstoles en paz.
\bibverse{34} Mas á Silas pareció bien el quedarse allí.

\hypertarget{la-pelea-de-pablo-con-bernabuxe9-salida-de-pablo-y-silas-de-antioquuxeda}{%
\subsection{La pelea de Pablo con Bernabé; Salida de Pablo y Silas de
Antioquía}\label{la-pelea-de-pablo-con-bernabuxe9-salida-de-pablo-y-silas-de-antioquuxeda}}

\bibverse{35} Y Pablo y Bernabé se estaban en Antioquía, enseñando la
palabra del Señor y anunciando el evangelio con otros muchos.

\bibverse{36} Y después de algunos días, Pablo dijo á Bernabé: Volvamos
á visitar á los hermanos por todas las ciudades en las cuales hemos
anunciado la palabra del Señor, cómo están. \bibverse{37} Y Bernabé
quería que tomasen consigo á Juan, el que tenía por sobrenombre Marcos;
\footnote{\textbf{15:37} Hech 1,12; Hech 1,25} \bibverse{38} Mas á Pablo
no le parecía bien llevar consigo al que se había apartado de ellos
desde Pamphylia, y no había ido con ellos á la obra. \footnote{\textbf{15:38}
  Hech 13,13}

\bibverse{39} Y hubo tal contención entre ellos, que se apartaron el uno
del otro; y Bernabé tomando á Marcos, navegó á Cipro. \bibverse{40} Y
Pablo escogiendo á Silas, partió encomendado de los hermanos á la gracia
del Señor. \bibverse{41} Y anduvo la Siria y la Cilicia, confirmando á
las iglesias.

\hypertarget{el-viaje-por-tierra-a-travuxe9s-de-asia-menor-hasta-troas}{%
\subsection{El viaje por tierra a través de Asia Menor hasta
Troas}\label{el-viaje-por-tierra-a-travuxe9s-de-asia-menor-hasta-troas}}

\hypertarget{section-15}{%
\section{16}\label{section-15}}

\bibverse{1} Después llegó á Derbe, y á Listra: y he aquí, estaba allí
un discípulo llamado Timoteo, hijo de una mujer Judía fiel, mas de padre
Griego. \footnote{\textbf{16:1} Hech 17,14; Hech 19,22; Hech 20,4; Fil
  2,19-22; 1Tes 3,2; 1Tes 3,6; 2Tim 1,5} \bibverse{2} De éste daban buen
testimonio los hermanos que estaban en Listra y en Iconio. \bibverse{3}
Este quiso Pablo que fuese con él; y tomándole, le circuncidó por causa
de los Judíos que estaban en aquellos lugares; porque todos sabían que
su padre era Griego. \bibverse{4} Y como pasaban por las ciudades, les
daban que guardasen los decretos que habían sido determinados por los
apóstoles y los ancianos que estaban en Jerusalem. \footnote{\textbf{16:4}
  Hech 15,23-29} \bibverse{5} Así que, las iglesias eran confirmadas en
fe, y eran aumentadas en número cada día.

\bibverse{6} Y pasando á Phrygia y la provincia de Galacia, les fué
prohibido por el Espíritu Santo hablar la palabra en Asia. \bibverse{7}
Y como vinieron á Misia, tentaron de ir á Bithynia; mas el Espíritu no
les dejó. \bibverse{8} Y pasando á Misia, descendieron á Troas.
\bibverse{9} Y fué mostrada á Pablo de noche una visión: Un varón
Macedonio se puso delante, rogándole, y diciendo: Pasa á Macedonia, y
ayúdanos. \bibverse{10} Y como vió la visión, luego procuramos partir á
Macedonia, dando por cierto que Dios nos llamaba para que les
anunciásemos el evangelio.

\hypertarget{el-viaje-por-mar-a-macedonia-pablo-en-filipos}{%
\subsection{El viaje por mar a Macedonia; Pablo en
Filipos}\label{el-viaje-por-mar-a-macedonia-pablo-en-filipos}}

\bibverse{11} Partidos pues de Troas, vinimos camino derecho á
Samotracia, y el día siguiente á Neápolis; \bibverse{12} Y de allí á
Filipos, que es la primera ciudad de la parte de Macedonia, y una
colonia; y estuvimos en aquella ciudad algunos días.

\hypertarget{conversiuxf3n-de-la-trader-morada-lydia}{%
\subsection{Conversión de la trader morada
Lydia}\label{conversiuxf3n-de-la-trader-morada-lydia}}

\bibverse{13} Y un día de sábado salimos de la puerta junto al río,
donde solía ser la oración; y sentándonos, hablamos á las mujeres que se
habían juntado. \bibverse{14} Entonces una mujer llamada Lidia, que
vendía púrpura en la ciudad de Tiatira, temerosa de Dios, estaba oyendo;
el corazón de la cual abrió el Señor para que estuviese atenta á lo que
Pablo decía. \bibverse{15} Y cuando fué bautizada, y su familia, nos
rogó, diciendo: Si habéis juzgado que yo sea fiel al Señor, entrad en mi
casa, y posad: y constriñónos.

\hypertarget{la-doncella-adivina-pablo-y-silas-en-la-corte-y-en-la-cuxe1rcel}{%
\subsection{La doncella adivina; Pablo y Silas en la corte y en la
cárcel}\label{la-doncella-adivina-pablo-y-silas-en-la-corte-y-en-la-cuxe1rcel}}

\bibverse{16} Y aconteció, que yendo nosotros á la oración, una muchacha
que tenía espíritu pitónico, nos salió al encuentro, la cual daba grande
ganancia á sus amos adivinando. \bibverse{17} Esta, siguiendo á Pablo y
á nosotros, daba voces, diciendo: Estos hombres son siervos del Dios
Alto, los cuales os anuncian el camino de salud. \footnote{\textbf{16:17}
  Mar 1,24; Mar 1,34} \bibverse{18} Y esto hacía por muchos días; mas
desagradando á Pablo, se volvió y dijo al espíritu: Te mando en el
nombre de Jesucristo, que salgas de ella. Y salió en la misma hora.
\footnote{\textbf{16:18} Mar 16,17}

\bibverse{19} Y viendo sus amos que había salido la esperanza de su
ganancia, prendieron á Pablo y á Silas, y los trajeron al foro, al
magistrado; \bibverse{20} Y presentándolos á los magistrados, dijeron:
Estos hombres, siendo Judíos, alborotan nuestra ciudad, \footnote{\textbf{16:20}
  Hech 17,6} \bibverse{21} Y predican ritos, los cuales no nos es lícito
recibir ni hacer, pues somos Romanos.

\bibverse{22} Y agolpóse el pueblo contra ellos: y los magistrados
rompiéndoles sus ropas, les mandaron azotar con varas. \bibverse{23} Y
después que los hubieron herido de muchos azotes, los echaron en la
cárcel, mandando al carcelero que los guardase con diligencia:
\bibverse{24} El cual, recibido este mandamiento, los metió en la cárcel
de más adentro; y les apretó los pies en el cepo.

\hypertarget{la-conversiuxf3n-del-carcelero}{%
\subsection{La conversión del
carcelero}\label{la-conversiuxf3n-del-carcelero}}

\bibverse{25} Mas á media noche, orando Pablo y Silas, cantaban himnos á
Dios: y los que estaban presos los oían. \bibverse{26} Entonces fué
hecho de repente un gran terremoto, de tal manera que los cimientos de
la cárcel se movían; y luego todas las puertas se abrieron, y las
prisiones de todos se soltaron. \bibverse{27} Y despertado el carcelero,
como vió abiertas las puertas de la cárcel, sacando la espada se quería
matar, pensando que los presos se habían huído. \bibverse{28} Mas Pablo
clamó á gran voz, diciendo: No te hagas ningún mal; que todos estamos
aquí.

\bibverse{29} El entonces pidiendo luz, entró dentro, y temblando,
derribóse á los pies de Pablo y de Silas; \bibverse{30} Y sacándolos
fuera, les dice: Señores, ¿qué es menester que yo haga para ser salvo?
\footnote{\textbf{16:30} Hech 2,37}

\bibverse{31} Y ellos dijeron: Cree en el Señor Jesucristo, y serás
salvo tú, y tu casa. \bibverse{32} Y le hablaron la palabra del Señor, y
á todos los que estaban en su casa.

\bibverse{33} Y tomándolos en aquella misma hora de la noche, les lavó
los azotes; y se bautizó luego él, y todos los suyos. \bibverse{34} Y
llevándolos á su casa, les puso la mesa: y se gozó de que con toda su
casa había creído á Dios.

\hypertarget{la-liberaciuxf3n-de-pablo-y-silas-de-la-cuxe1rcel}{%
\subsection{La liberación de Pablo y Silas de la
cárcel}\label{la-liberaciuxf3n-de-pablo-y-silas-de-la-cuxe1rcel}}

\bibverse{35} Y como fué de día, los magistrados enviaron los
alguaciles, diciendo: Deja ir á aquellos hombres.

\bibverse{36} Y el carcelero hizo saber estas palabras á Pablo: Los
magistrados han enviado á decir que seáis sueltos: así que ahora salid,
é id en paz.

\bibverse{37} Entonces Pablo les dijo: Azotados públicamente sin ser
condenados, siendo hombres Romanos, nos echaron en la cárcel; y ¿ahora
nos echan encubiertamente? No, de cierto, sino vengan ellos y sáquennos.

\bibverse{38} Y los alguaciles volvieron á decir á los magistrados estas
palabras: y tuvieron miedo, oído que eran Romanos. \bibverse{39} Y
viniendo, les rogaron; y sacándolos, les pidieron que se saliesen de la
ciudad. \bibverse{40} Entonces salidos de la cárcel, entraron en casa de
Lidia; y habiendo visto á los hermanos, los consolaron, y se salieron.

\hypertarget{pablo-en-tesaluxf3nica}{%
\subsection{Pablo en Tesalónica}\label{pablo-en-tesaluxf3nica}}

\hypertarget{section-16}{%
\section{17}\label{section-16}}

\bibverse{1} Y pasando por Amphípolis y Apolonia, llegaron á Tesalónica,
donde estaba la sinagoga de los Judíos. \footnote{\textbf{17:1} 1Tes 2,2}
\bibverse{2} Y Pablo, como acostumbraba, entró á ellos, y por tres
sábados disputó con ellos de las Escrituras, \bibverse{3} Declarando y
proponiendo, que convenía que el Cristo padeciese, y resucitase de los
muertos; y que Jesús, el cual yo os anuncio, decía él, éste era el
Cristo.

\bibverse{4} Y algunos de ellos creyeron, y se juntaron con Pablo y con
Silas; y de los Griegos religiosos grande multitud, y mujeres nobles no
pocas. \footnote{\textbf{17:4} 1Tes 1,1; 2Tes 1,1} \bibverse{5} Entonces
los Judíos que eran incrédulos, teniendo celos, tomaron consigo á
algunos ociosos, malos hombres, y juntando compañía, alborotaron la
ciudad; y acometiendo á la casa de Jasón, procuraban sacarlos al pueblo.
\bibverse{6} Mas no hallándolos, trajeron á Jasón y á algunos hermanos á
los gobernadores de la ciudad, dando voces: Estos que alborotan el
mundo, también han venido acá; \bibverse{7} A los cuales Jasón ha
recibido; y todos estos hacen contra los decretos de César, diciendo que
hay otro rey, Jesús. \footnote{\textbf{17:7} Luc 23,2} \bibverse{8} Y
alborotaron al pueblo y á los gobernadores de la ciudad, oyendo estas
cosas. \bibverse{9} Mas recibida satisfacción de Jasón y de los demás,
los soltaron.

\hypertarget{las-experiencias-de-pablo-en-berea-y-su-viaje-a-atenas}{%
\subsection{Las experiencias de Pablo en Berea y su viaje a
Atenas}\label{las-experiencias-de-pablo-en-berea-y-su-viaje-a-atenas}}

\bibverse{10} Entonces los hermanos, luego de noche, enviaron á Pablo y
á Silas á Berea; los cuales habiendo llegado, entraron en la sinagoga de
los Judíos.

\bibverse{11} Y fueron éstos más nobles que los que estaban en
Tesalónica, pues recibieron la palabra con toda solicitud, escudriñando
cada día las Escrituras, si estas cosas eran así. \bibverse{12} Así que
creyeron muchos de ellos; y mujeres Griegas de distinción, y no pocos
hombres. \bibverse{13} Mas como entendieron los Judíos de Tesalónica que
también en Berea era anunciada la palabra de Dios por Pablo, fueron, y
también allí tumultuaron al pueblo. \bibverse{14} Empero luego los
hermanos enviaron á Pablo que fuese como á la mar; y Silas y Timoteo se
quedaron allí. \footnote{\textbf{17:14} Hech 16,1} \bibverse{15} Y los
que habían tomado á cargo á Pablo, le llevaron hasta Atenas; y tomando
encargo para Silas y Timoteo, que viniesen á él lo más presto que
pudiesen, partieron.

\hypertarget{pablo-en-atenas}{%
\subsection{Pablo en Atenas}\label{pablo-en-atenas}}

\bibverse{16} Y esperándolos Pablo en Atenas, su espíritu se deshacía en
él viendo la ciudad dada á idolatría. \bibverse{17} Así que, disputaba
en la sinagoga con los Judíos y religiosos; y en la plaza cada día con
los que le ocurrían. \bibverse{18} Y algunos filósofos de los Epicúreos
y de los Estóicos, disputaban con él; y unos decían: ¿Qué quiere decir
este palabrero? Y otros: Parece que es predicador de nuevos dioses:
porque les predicaba á Jesús y la resurrección.

\bibverse{19} Y tomándole, le trajeron al Areópago, diciendo: ¿Podremos
saber qué sea esta nueva doctrina que dices? \bibverse{20} Porque pones
en nuestros oídos unas nuevas cosas: queremos pues saber qué quiere ser
esto. \bibverse{21} (Entonces todos los Atenienses y los huéspedes
extranjeros, en ninguna otra cosa entendían, sino ó en decir ó en oir
alguna cosa nueva.)

\hypertarget{discurso-de-pablo-en-el-cerro-del-areuxf3pago}{%
\subsection{Discurso de Pablo en el cerro del
Areópago}\label{discurso-de-pablo-en-el-cerro-del-areuxf3pago}}

\bibverse{22} Estando pues Pablo en medio del Areópago, dijo: Varones
Atenienses, en todo os veo como más supersticiosos; \bibverse{23} Porque
pasando y mirando vuestros santuarios, hallé también un altar en el cual
estaba esta inscripción: AL DIOS NO CONOCIDO. Aquél pues, que vosotros
honráis sin conocerle, á éste os anuncio yo. \bibverse{24} El Dios que
hizo el mundo y todas las cosas que en él hay, éste, como sea Señor del
cielo y de la tierra, no habita en templos hechos de manos, \footnote{\textbf{17:24}
  1Re 8,27} \bibverse{25} Ni es honrado con manos de hombres, necesitado
de algo; pues él da á todos vida, y respiración, y todas las cosas;
\footnote{\textbf{17:25} Sal 50,9-12} \bibverse{26} Y de una sangre ha
hecho todo el linaje de los hombres, para que habitasen sobre toda la
faz de la tierra; y les ha prefijado el orden de los tiempos, y los
términos de la habitación de ellos; \footnote{\textbf{17:26} Deut 32,8}
\bibverse{27} Para que buscasen á Dios, si en alguna manera, palpando,
le hallen; aunque cierto no está lejos de cada uno de nosotros:
\footnote{\textbf{17:27} Is 55,6} \bibverse{28} Porque en él vivimos, y
nos movemos, y somos; como también algunos de vuestros poetas dijeron:
Porque linaje de éste somos también. \bibverse{29} Siendo pues linaje de
Dios, no hemos de estimar la Divinidad ser semejante á oro, ó á plata, ó
á piedra, escultura de artificio ó de imaginación de hombres.
\footnote{\textbf{17:29} Gén 1,27; Is 40,18} \bibverse{30} Empero Dios,
habiendo disimulado los tiempos de esta ignorancia, ahora denuncia á
todos los hombres en todos los lugares que se arrepientan: \footnote{\textbf{17:30}
  Hech 14,16; Luc 24,47} \bibverse{31} Por cuanto ha establecido un día,
en el cual ha de juzgar al mundo con justicia, por aquel varón al cual
determinó; dando fe á todos con haberle levantado de los muertos.
\footnote{\textbf{17:31} Hech 10,42; Mat 25,31-33}

\bibverse{32} Y así como oyeron de la resurrección de los muertos, unos
se burlaban, y otros decían: Te oiremos acerca de esto otra vez.

\bibverse{33} Y así Pablo se salió de en medio de ellos. \bibverse{34}
Mas algunos creyeron, juntándose con él; entre los cuales también fué
Dionisio el del Areópago, y una mujer llamada Dámaris, y otros con
ellos.

\hypertarget{pablo-en-corinto}{%
\subsection{Pablo en Corinto}\label{pablo-en-corinto}}

\hypertarget{section-17}{%
\section{18}\label{section-17}}

\bibverse{1} Pasadas estas cosas, Pablo partió de Atenas, y vino á
Corinto. \bibverse{2} Y hallando á un Judío llamado Aquila, natural del
Ponto, que hacía poco que había venido de Italia, y á Priscila su mujer,
(porque Claudio había mandado que todos los Judíos saliesen de Roma) se
vino á ellos; \bibverse{3} Y porque era de su oficio, posó con ellos, y
trabajaba; porque el oficio de ellos era hacer tiendas. \footnote{\textbf{18:3}
  Hech 20,34; 1Cor 4,12} \bibverse{4} Y disputaba en la sinagoga todos
los sábados, y persuadía á Judíos y á Griegos.

\bibverse{5} Y cuando Silas y Timoteo vinieron de Macedonia, Pablo
estaba constreñido por la palabra, testificando á los Judíos que Jesús
era el Cristo. \bibverse{6} Mas contradiciendo y blasfemando ellos, les
dijo, sacudiendo sus vestidos: Vuestra sangre sea sobre vuestra cabeza;
yo, limpio; desde ahora me iré á los Gentiles. \footnote{\textbf{18:6}
  Hech 13,51; Hech 20,26}

\bibverse{7} Y partiendo de allí, entró en casa de uno llamado Justo,
temeroso de Dios, la casa del cual estaba junto á la sinagoga.
\bibverse{8} Y Crispo, el prepósito de la sinagoga, creyó al Señor con
toda su casa: y muchos de los Corintios oyendo creían, y eran
bautizados. \bibverse{9} Entonces el Señor dijo de noche en visión á
Pablo: No temas, sino habla, y no calles: \footnote{\textbf{18:9} 1Cor
  2,3} \bibverse{10} Porque yo estoy contigo, y ninguno te podrá hacer
mal; porque yo tengo mucho pueblo en esta ciudad. \footnote{\textbf{18:10}
  Jer 1,8; Juan 10,16}

\bibverse{11} Y se detuvo allí un año y seis meses, enseñándoles la
palabra de Dios.

\hypertarget{la-acusaciuxf3n-contra-los-juduxedos-fue-rechazada-por-el-gobernador-galiuxf3n}{%
\subsection{La acusación contra los judíos fue rechazada por el
gobernador
Galión}\label{la-acusaciuxf3n-contra-los-juduxedos-fue-rechazada-por-el-gobernador-galiuxf3n}}

\bibverse{12} Y siendo Galión procónsul de Acaya, los Judíos se
levantaron de común acuerdo contra Pablo, y le llevaron al tribunal,
\bibverse{13} Diciendo: Que éste persuade á los hombres á honrar á Dios
contra la ley.

\bibverse{14} Y comenzando Pablo á abrir la boca, Galión dijo á los
Judíos: Si fuera algún agravio ó algún crimen enorme, oh Judíos,
conforme á derecho yo os tolerara: \footnote{\textbf{18:14} Hech
  25,18-20} \bibverse{15} Mas si son cuestiones de palabras, y de
nombres, y de vuestra ley, vedlo vosotros; porque yo no quiero ser juez
de estas cosas. \footnote{\textbf{18:15} Juan 18,31} \bibverse{16} Y los
echó del tribunal.

\bibverse{17} Entonces todos los Griegos tomando á Sóstenes, prepósito
de la sinagoga, le herían delante del tribunal: y á Galión nada se le
daba de ello.

\hypertarget{regreso-de-pablo-vuxeda-uxe9feso-y-judea-a-antioquuxeda-en-siria}{%
\subsection{Regreso de Pablo vía Éfeso y Judea a Antioquía en
Siria}\label{regreso-de-pablo-vuxeda-uxe9feso-y-judea-a-antioquuxeda-en-siria}}

\bibverse{18} Mas Pablo habiéndose detenido aún allí muchos días,
después se despidió de los hermanos, y navegó á Siria, y con él Priscila
y Aquila, habiéndose trasquilado la cabeza en Cencreas, porque tenía
voto. \footnote{\textbf{18:18} Hech 21,24; Núm 6,2; Núm 6,5; Núm 6,13;
  Núm 6,18} \bibverse{19} Y llegó á Efeso, y los dejó allí: y él
entrando en la sinagoga, disputó con los Judíos, \bibverse{20} Los
cuales le rogaban que se quedase con ellos por más tiempo; mas no
accedió, \bibverse{21} Sino que se despidió de ellos, diciendo: Es
menester que en todo caso tenga la fiesta que viene, en Jerusalem; mas
otra vez volveré á vosotros, queriendo Dios. Y partió de Efeso.

\bibverse{22} Y habiendo arribado á Cesarea subió á Jerusalem; y después
de saludar á la iglesia, descendió á Antioquía. \footnote{\textbf{18:22}
  Hech 21,15}

\hypertarget{inicio-del-viaje-apolos-en-uxe9feso-y-corinto}{%
\subsection{Inicio del viaje; Apolos en Éfeso y
Corinto}\label{inicio-del-viaje-apolos-en-uxe9feso-y-corinto}}

\bibverse{23} Y habiendo estado allí algún tiempo, partió, andando por
orden la provincia de Galacia, y la Phrygia, confirmando á todos los
discípulos. \bibverse{24} Llegó entonces á Efeso un Judío, llamado
Apolos, natural de Alejandría, varón elocuente, poderoso en las
Escrituras. \bibverse{25} Este era instruído en el camino del Señor; y
ferviente de espíritu, hablaba y enseñaba diligentemente las cosas que
son del Señor, enseñando solamente en el bautismo de Juan. \footnote{\textbf{18:25}
  Hech 19,3} \bibverse{26} Y comenzó á hablar confiadamente en la
sinagoga: al cual como oyeron Priscila y Aquila, le tomaron, y le
declararon más particularmente el camino de Dios.

\bibverse{27} Y queriendo él pasar á Acaya, los hermanos exhortados,
escribieron á los discípulos que le recibiesen; y venido él, aprovechó
mucho por la gracia á los que habían creído: \bibverse{28} Porque con
gran vehemencia convencía públicamente á los Judíos, mostrando por las
Escrituras que Jesús era el Cristo.

\hypertarget{conversiuxf3n-y-bautismo-de-los-discuxedpulos-de-juan}{%
\subsection{Conversión y bautismo de los discípulos de
Juan}\label{conversiuxf3n-y-bautismo-de-los-discuxedpulos-de-juan}}

\hypertarget{section-18}{%
\section{19}\label{section-18}}

\bibverse{1} Y aconteció que entre tanto que Apolos estaba en Corinto,
Pablo, andadas las regiones superiores, vino á Efeso, y hallando ciertos
discípulos, \bibverse{2} Díjoles: ¿Habéis recibido el Espíritu Santo
después que creísteis? Y ellos le dijeron: Antes ni aun hemos oído si
hay Espíritu Santo. \footnote{\textbf{19:2} Hech 2,38}

\bibverse{3} Entonces dijo: ¿En qué pues sois bautizados? Y ellos
dijeron: En el bautismo de Juan.

\bibverse{4} Y dijo Pablo: Juan bautizó con bautismo de arrepentimiento,
diciendo al pueblo que creyesen en el que había de venir después de él,
es á saber, en Jesús el Cristo.

\bibverse{5} Oído que hubieron esto, fueron bautizados en el nombre del
Señor Jesús. \bibverse{6} Y habiéndoles impuesto Pablo las manos, vino
sobre ellos el Espíritu Santo; y hablaban en lenguas, y profetizaban.
\footnote{\textbf{19:6} Hech 8,17; Hech 10,44; Hech 10,46} \bibverse{7}
Y eran en todos como unos doce hombres.

\hypertarget{la-actividad-de-dos-auxf1os-de-enseuxf1anza-y-milagros-de-pablo-en-uxe9feso}{%
\subsection{La actividad de dos años de enseñanza y milagros de Pablo en
Éfeso}\label{la-actividad-de-dos-auxf1os-de-enseuxf1anza-y-milagros-de-pablo-en-uxe9feso}}

\bibverse{8} Y entrando él dentro de la sinagoga, hablaba libremente por
espacio de tres meses, disputando y persuadiendo del reino de Dios.

\bibverse{9} Mas endureciéndose algunos y no creyendo, maldiciendo el
Camino delante de la multitud, apartándose Pablo de ellos separó á los
discípulos, disputando cada día en la escuela de un cierto Tyranno.
\bibverse{10} Y esto fué por espacio de dos años; de manera que todos
los que habitaban en Asia, Judíos y Griegos, oyeron la palabra del Señor
Jesús.

\bibverse{11} Y hacía Dios singulares maravillas por manos de Pablo:
\bibverse{12} De tal manera que aun se llevaban sobre los enfermos los
sudarios y los pañuelos de su cuerpo, y las enfermedades se iban de
ellos, y los malos espíritus salían de ellos. \footnote{\textbf{19:12}
  Hech 5,15}

\hypertarget{superar-la-supersticiuxf3n-los-invocadores-y-los-libros-de-hechizos}{%
\subsection{Superar la superstición (Los invocadores y los libros de
hechizos)}\label{superar-la-supersticiuxf3n-los-invocadores-y-los-libros-de-hechizos}}

\bibverse{13} Y algunos de los Judíos, exorcistas vagabundos, tentaron á
invocar el nombre del Señor Jesús sobre los que tenían espíritus malos,
diciendo: Os conjuro por Jesús, el que Pablo predica. \footnote{\textbf{19:13}
  Luc 9,49} \bibverse{14} Y había siete hijos de un tal Sceva, Judío,
príncipe de los sacerdotes, que hacían esto.

\bibverse{15} Y respondiendo el espíritu malo, dijo: A Jesús conozco, y
sé quién es Pablo: mas vosotros ¿quiénes sois? \bibverse{16} Y el hombre
en quien estaba el espíritu malo, saltando en ellos, y enseñoreándose de
ellos, pudo más que ellos, de tal manera que huyeron de aquella casa
desnudos y heridos. \bibverse{17} Y esto fué notorio á todos, así Judíos
como Griegos, los que habitaban en Efeso: y cayó temor sobre todos
ellos, y era ensalzado el nombre del Señor Jesús. \bibverse{18} Y muchos
de los que habían creído, venían, confesando y dando cuenta de sus
hechos. \bibverse{19} Asimismo muchos de los que habían practicado vanas
artes, trajeron los libros, y los quemaron delante de todos; y echada la
cuenta del precio de ellos, hallaron ser cincuenta mil denarios.
\bibverse{20} Así crecía poderosamente la palabra del Señor, y
prevalecía. \footnote{\textbf{19:20} Hech 12,24}

\hypertarget{planes-de-viaje-de-pablo}{%
\subsection{Planes de viaje de Pablo}\label{planes-de-viaje-de-pablo}}

\bibverse{21} Y acabadas estas cosas, se propuso Pablo en espíritu
partir á Jerusalem, después de andada Macedonia y Acaya, diciendo:
Después que hubiere estado allá, me será menester ver también á Roma.
\footnote{\textbf{19:21} Hech 23,11}

\bibverse{22} Y enviando á Macedonia á dos de los que le ayudaban,
Timoteo y Erasto, él se estuvo por algún tiempo en Asia. \footnote{\textbf{19:22}
  2Tim 4,20}

\hypertarget{el-motuxedn-de-los-plateros-de-demetrio}{%
\subsection{El motín de los plateros de
Demetrio}\label{el-motuxedn-de-los-plateros-de-demetrio}}

\bibverse{23} Entonces hubo un alboroto no pequeño acerca del Camino.
\footnote{\textbf{19:23} 2Cor 1,8-9} \bibverse{24} Porque un platero
llamado Demetrio, el cual hacía de plata templecillos de Diana, daba á
los artífices no poca ganancia; \bibverse{25} A los cuales, reunidos con
los oficiales de semejante oficio, dijo: Varones, sabéis que de este
oficio tenemos ganancia; \bibverse{26} Y veis y oís que este Pablo, no
solamente en Efeso, sino á muchas gentes de casi toda el Asia, ha
apartado con persuasión, diciendo, que no son dioses los que se hacen
con las manos. \bibverse{27} Y no solamente hay peligro de que este
negocio se nos vuelva en reproche, sino también que el templo de la gran
diosa Diana sea estimado en nada, y comience á ser destruída su
majestad, la cual honra toda el Asia y el mundo.

\bibverse{28} Oídas estas cosas, llenáronse de ira, y dieron alarido,
diciendo: ¡Grande es Diana de los Efesios! \bibverse{29} Y la ciudad se
llenó de confusión; y unánimes se arrojaron al teatro, arrebatando á
Gayo y á Aristarco, Macedonios, compañeros de Pablo. \footnote{\textbf{19:29}
  Hech 20,4} \bibverse{30} Y queriendo Pablo salir al pueblo, los
discípulos no le dejaron. \bibverse{31} También algunos de los
principales de Asia, que eran sus amigos, enviaron á él rogando que no
se presentase en el teatro. \bibverse{32} Y otros gritaban otra cosa;
porque la concurrencia estaba confusa, y los más no sabían por qué se
habían juntado. \bibverse{33} Y sacaron de entre la multitud á
Alejandro, empujándole los Judíos. Entonces Alejandro, pedido silencio
con la mano, quería dar razón al pueblo. \bibverse{34} Mas como
conocieron que era Judío, fué hecha un voz de todos, que gritaron casi
por dos horas: ¡Grande es Diana de los Efesios!

\bibverse{35} Entonces el escribano, apaciguado que hubo la gente, dijo:
Varones Efesios, ¿y quién hay de los hombres que no sepa que la ciudad
de los Efesios es honradora de la gran diosa Diana, y de la imagen
venida de Júpiter? \bibverse{36} Así que, pues esto no puede ser
contradicho, conviene que os apacigüéis, y que nada hagáis
temerariamente; \bibverse{37} Pues habéis traído á estos hombres, sin
ser sacrílegos ni blasfemadores de vuestra diosa. \bibverse{38} Que si
Demetrio y los oficiales que están con él tienen negocio con alguno,
audiencias se hacen, y procónsules hay; acúsense los unos á los otros.
\bibverse{39} Y si demandáis alguna otra cosa, en legítima asamblea se
pueda decidir. \bibverse{40} Porque peligro hay de que seamos argüidos
de sedición por hoy, no habiendo ninguna causa por la cual podamos dar
razón de este concurso. Y habiendo dicho esto, despidió la concurrencia.
\bibverse{41}

\hypertarget{viaje-a-grecia-y-regresa-a-troas}{%
\subsection{Viaje a Grecia y regresa a
Troas}\label{viaje-a-grecia-y-regresa-a-troas}}

\hypertarget{section-19}{%
\section{20}\label{section-19}}

\bibverse{1} Y después que cesó el alboroto, llamando Pablo á los
discípulos habiéndoles exhortado y abrazado, se despidió, y partió para
ir á Macedonia. \bibverse{2} Y andado que hubo aquellas partes, y
exhortádoles con abundancia de palabra, vino á Grecia. \bibverse{3} Y
después de haber estado allí tres meses, y habiendo de navegar á Siria,
le fueron puestas asechanzas por los Judíos; y así tomó consejo de
volverse por Macedonia. \bibverse{4} Y le acompañaron hasta Asia Sopater
Bereense, y los Tesalonicenses, Aristarco y Segundo; y Gayo de Derbe, y
Timoteo; y de Asia, Tychîco y Trófimo. \footnote{\textbf{20:4} Hech
  17,10; Hech 19,29; Hech 16,1; Hech 21,29; Efes 6,21} \bibverse{5}
Estos yendo delante, nos esperaron en Troas. \bibverse{6} Y nosotros,
pasados los días de los panes sin levadura, navegamos de Filipos y
vinimos á ellos á Troas en cinco días, donde estuvimos siete días.

\hypertarget{celebraciuxf3n-de-despedida-de-pablo-en-troas-reanimaciuxf3n-del-fallido-eutico}{%
\subsection{Celebración de despedida de Pablo en Troas; Reanimación del
fallido
Eutico}\label{celebraciuxf3n-de-despedida-de-pablo-en-troas-reanimaciuxf3n-del-fallido-eutico}}

\bibverse{7} Y el día primero de la semana, juntos los discípulos á
partir el pan, Pablo les enseñaba, habiendo de partir al día siguiente:
y alargó el discurso hasta la media noche. \bibverse{8} Y había muchas
lámparas en el aposento alto donde estaban juntos. \bibverse{9} Y un
mancebo llamado Eutichô que estaba sentado en la ventana, tomado de un
sueño profundo, como Pablo disputaba largamente, postrado del sueño cayó
del tercer piso abajo, y fué alzado muerto. \bibverse{10} Entonces
descendió Pablo, y derribóse sobre él, y abrazándole, dijo: No os
alborotéis, que su alma está en él. \footnote{\textbf{20:10} 1Re 17,21}

\bibverse{11} Después subiendo, y partiendo el pan, y gustando, habló
largamente hasta el alba, y así partió. \bibverse{12} Y llevaron al mozo
vivo, y fueron consolados no poco.

\hypertarget{el-viaje-de-pablo-de-troas-a-mileto}{%
\subsection{El viaje de Pablo de Troas a
Mileto}\label{el-viaje-de-pablo-de-troas-a-mileto}}

\bibverse{13} Y nosotros subiendo en el navío, navegamos á Assón, para
recibir de allí á Pablo; pues así había determinado que debía él ir por
tierra. \bibverse{14} Y como se juntó con nosotros en Assón, tomándole
vinimos á Mitilene. \bibverse{15} Y navegando de allí, al día siguiente
llegamos delante de Chîo, y al otro día tomamos puerto en Samo: y
habiendo reposado en Trogilio, al día siguiente llegamos á Mileto.
\bibverse{16} Porque Pablo se había propuesto pasar adelante de Efeso;
por no detenerse en Asia: porque se apresuraba por hacer el día de
Pentecostés, si le fuese posible, en Jerusalem.

\hypertarget{encuentro-de-pablo-con-los-ancianos-de-uxe9feso-en-mileto-su-discurso-de-despedida-y-su-despedida}{%
\subsection{Encuentro de Pablo con los ancianos de Éfeso en Mileto; su
discurso de despedida y su
despedida}\label{encuentro-de-pablo-con-los-ancianos-de-uxe9feso-en-mileto-su-discurso-de-despedida-y-su-despedida}}

\bibverse{17} Y enviando desde Mileto á Efeso, hizo llamar á los
ancianos de la iglesia. \bibverse{18} Y cuando vinieron á él, les dijo:
Vosotros sabéis cómo, desde el primer día que entré en Asia, he estado
con vosotros por todo el tiempo, \footnote{\textbf{20:18} Hech 18,19;
  Hech 19,10} \bibverse{19} Sirviendo al Señor con toda humildad, y con
muchas lágrimas, y tentaciones que me han venido por las asechanzas de
los Judíos: \bibverse{20} Cómo nada que fuese útil he rehuído de
anunciaros y enseñaros, públicamente y por las casas, \bibverse{21}
Testificando á los Judíos y á los Gentiles arrepentimiento para con
Dios, y la fe en nuestro Señor Jesucristo. \bibverse{22} Y ahora, he
aquí, ligado yo en espíritu, voy á Jerusalem, sin saber lo que allá me
ha de acontecer: \bibverse{23} Mas que el Espíritu Santo por todas las
ciudades me da testimonio, diciendo que prisiones y tribulaciones me
esperan. \footnote{\textbf{20:23} Hech 9,16; Hech 21,4; Hech 21,11}
\bibverse{24} Mas de ninguna cosa hago caso, ni estimo mi vida preciosa
para mí mismo; solamente que acabe mi carrera con gozo, y el ministerio
que recibí del Señor Jesús, para dar testimonio del evangelio de la
gracia de Dios. \footnote{\textbf{20:24} Hech 21,13; 2Tim 4,7}

\bibverse{25} Y ahora, he aquí, yo sé que ninguno de todos vosotros, por
quien he pasado predicando el reino de Dios, verá más mi rostro.
\bibverse{26} Por tanto, yo os protesto el día de hoy, que yo soy limpio
de la sangre de todos: \footnote{\textbf{20:26} Hech 18,6; Ezeq 3,17-19}
\bibverse{27} Porque no he rehuído de anunciaros todo el consejo de
Dios. \bibverse{28} Por tanto mirad por vosotros, y por todo el rebaño
en que el Espíritu Santo os ha puesto por obispos, para apacentar la
iglesia del Señor, la cual ganó por su sangre. \bibverse{29} Porque yo
sé que después de mi partida entrarán en medio de vosotros lobos
rapaces, que no perdonarán al ganado; \footnote{\textbf{20:29} Mat 7,15}
\bibverse{30} Y de vosotros mismos se levantarán hombres que hablen
cosas perversas, para llevar discípulos tras sí. \footnote{\textbf{20:30}
  1Jn 2,18; 1Jn 1,2-19} \bibverse{31} Por tanto, velad, acordándoos que
por tres años de noche y de día, no he cesado de amonestar con lágrimas
á cada uno. \bibverse{32} Y ahora, hermanos, os encomiendo á Dios, y á
la palabra de su gracia: el cual es poderoso para sobreedificar, y daros
heredad con todos los santificados. \bibverse{33} La plata, ó el oro, ó
el vestido de nadie he codiciado. \bibverse{34} Antes vosotros sabéis
que para lo que me ha sido necesario, y á los que están conmigo, estas
manos me han servido. \footnote{\textbf{20:34} Hech 18,3; 1Cor 4,12;
  1Tes 2,9} \bibverse{35} En todo os he enseñado que, trabajando así, es
necesario sobrellevar á los enfermos, y tener presente las palabras del
Señor Jesús, el cual dijo: Más bienaventurada cosa es dar que recibir.

\bibverse{36} Y como hubo dicho estas cosas, se puso de rodillas, y oró
con todos ellos. \bibverse{37} Entonces hubo un gran lloro de todos: y
echándose en el cuello de Pablo, le besaban, \bibverse{38} Doliéndose en
gran manera por la palabra que dijo, que no habían de ver más su rostro.
Y le acompañaron al navío.

\hypertarget{continuaciuxf3n-del-viaje-de-mileto-a-tiro-y-cesarea}{%
\subsection{Continuación del viaje de Mileto a Tiro y
Cesarea}\label{continuaciuxf3n-del-viaje-de-mileto-a-tiro-y-cesarea}}

\hypertarget{section-20}{%
\section{21}\label{section-20}}

\bibverse{1} Y habiendo partido de ellos, navegamos y vinimos camino
derecho á Coos, y al día siguiente á Rhodas, y de allí á Pátara.
\bibverse{2} Y hallando un barco que pasaba á Fenicia, nos embarcamos, y
partimos. \bibverse{3} Y como avistamos á Cipro, dejándola á mano
izquierda, navegamos á Siria, y vinimos á Tiro: porque el barco había de
descargar allí su carga. \bibverse{4} Y nos quedamos allí siete días,
hallados los discípulos, los cuales decían á Pablo por Espíritu, que no
subiese á Jerusalem. \footnote{\textbf{21:4} Hech 20,23} \bibverse{5} Y
cumplidos aquellos días, salimos acompañándonos todos, con sus mujeres é
hijos, hasta fuera de la ciudad; y puestos de rodillas en la ribera,
oramos. \footnote{\textbf{21:5} Hech 20,36} \bibverse{6} Y abrazándonos
los unos á los otros, subimos al barco, y ellos se volvieron á sus
casas.

\bibverse{7} Y nosotros, cumplida la navegación, vinimos de Tiro á
Tolemaida; y habiendo saludado á los hermanos, nos quedamos con ellos un
día. \bibverse{8} Y otro día, partidos Pablo y los que con él estábamos,
vinimos á Cesarea: y entrando en casa de Felipe el evangelista, el cual
era uno de los siete, posamos con él. \footnote{\textbf{21:8} Hech 6,5;
  Hech 8,40}

\bibverse{9} Y éste tenía cuatro hijas, doncellas, que profetizaban.
\bibverse{10} Y parando nosotros allí por muchos días, descendió de
Judea un profeta, llamado Agabo; \bibverse{11} Y venido á nosotros, tomó
el cinto de Pablo, y atándose los pies y las manos, dijo: Esto dice el
Espíritu Santo: Así atarán los Judíos en Jerusalem al varón cuyo es este
cinto, y le entregarán en manos de los Gentiles. \footnote{\textbf{21:11}
  Hech 20,23}

\bibverse{12} Lo cual como oímos, le rogamos nosotros y los de aquel
lugar, que no subiese á Jerusalem. \footnote{\textbf{21:12} Mat 16,22}
\bibverse{13} Entonces Pablo respondió: ¿Qué hacéis llorando y
afligiéndome el corazón? porque yo no sólo estoy presto á ser atado, mas
aun á morir en Jerusalem por el nombre del Señor Jesús. \footnote{\textbf{21:13}
  Hech 20,24}

\bibverse{14} Y como no le pudimos persuadir, desistimos, diciendo:
Hágase la voluntad del Señor. \footnote{\textbf{21:14} Luc 22,42}

\hypertarget{pablo-en-jerusaluxe9n-y-preso-en-cesarea}{%
\subsection{Pablo en Jerusalén y preso en
Cesarea}\label{pablo-en-jerusaluxe9n-y-preso-en-cesarea}}

\bibverse{15} Y después de estos días, apercibidos, subimos á Jerusalem.
\bibverse{16} Y vinieron también con nosotros de Cesarea algunos de los
discípulos, trayendo consigo á un Mnasón, Cyprio, discípulo antiguo, con
el cual posásemos.

\bibverse{17} Y cuando llegamos á Jerusalem, los hermanos nos recibieron
de buena voluntad. \bibverse{18} Y al día siguiente Pablo entró con
nosotros á Jacobo, y todos los ancianos se juntaron; \footnote{\textbf{21:18}
  Hech 15,13} \bibverse{19} A los cuales, como los hubo saludado, contó
por menudo lo que Dios había hecho entre los Gentiles por su ministerio.
\bibverse{20} Y ellos como lo oyeron, glorificaron á Dios, y le dijeron:
Ya ves, hermano, cuántos millares de Judíos hay que han creído; y todos
son celadores de la ley: \bibverse{21} Mas fueron informados acerca de
ti, que enseñas á apartarse de Moisés á todos los Judíos que están entre
los Gentiles, diciéndoles que no han de circuncidar á los hijos, ni
andar según la costumbre. \footnote{\textbf{21:21} Hech 16,3}
\bibverse{22} ¿Qué hay pues? La multitud se reunirá de cierto: porque
oirán que has venido. \bibverse{23} Haz pues esto que te decimos: Hay
entre nosotros cuatro hombres que tienen voto sobre sí: \bibverse{24}
Tomando á éstos contigo, purifícate con ellos, y gasta con ellos, para
que rasuren sus cabezas, y todos entiendan que no hay nada de lo que
fueron informados acerca de ti; sino que tú también andas guardando la
ley. \bibverse{25} Empero cuanto á los que de los Gentiles han creído,
nosotros hemos escrito haberse acordado que no guarden nada de esto;
solamente que se abstengan de lo que fuere sacrificado á los ídolos, y
de sangre, y de ahogado, y de fornicación. \footnote{\textbf{21:25} Hech
  15,20; Hech 15,29}

\bibverse{26} Entonces Pablo tomó consigo aquellos hombres, y al día
siguiente, habiéndose purificado con ellos, entró en el templo, para
anunciar el cumplimiento de los días de la purificación, hasta ser
ofrecida ofrenda por cada uno de ellos. \footnote{\textbf{21:26} Núm
  6,1-20; 1Cor 9,20}

\hypertarget{pablo-arrestado-por-los-juduxedos-en-el-templo-el-levantamiento-en-jerusaluxe9n}{%
\subsection{Pablo arrestado por los judíos en el templo; el
levantamiento en
Jerusalén}\label{pablo-arrestado-por-los-juduxedos-en-el-templo-el-levantamiento-en-jerusaluxe9n}}

\bibverse{27} Y cuando estaban para acabarse los siete días, unos Judíos
de Asia, como le vieron en el templo, alborotaron todo el pueblo y le
echaron mano, \bibverse{28} Dando voces: Varones Israelitas, ayudad:
Este es el hombre que por todas partes enseña á todos contra el pueblo,
y la ley, y este lugar; y además de esto ha metido Gentiles en el
templo, y ha contaminado este lugar santo. \footnote{\textbf{21:28} Hech
  6,13; Ezeq 44,7} \bibverse{29} Porque antes habían visto con él en la
ciudad á Trófimo, Efesio, al cual pensaban que Pablo había metido en el
templo. \footnote{\textbf{21:29} Hech 20,4; 2Tim 4,20}

\bibverse{30} Así que, toda la ciudad se alborotó, y agolpóse el pueblo;
y tomando á Pablo, hiciéronle salir fuera del templo, y luego las
puertas fueron cerradas.

\hypertarget{captura-de-pablo-por-el-coronel-romano-lisias}{%
\subsection{Captura de Pablo por el coronel romano
Lisias}\label{captura-de-pablo-por-el-coronel-romano-lisias}}

\bibverse{31} Y procurando ellos matarle, fué dado aviso al tribuno de
la compañía, que toda la ciudad de Jerusalem estaba alborotada;
\bibverse{32} El cual tomando luego soldados y centuriones, corrió á
ellos. Y ellos como vieron al tribuno y á los soldados, cesaron de herir
á Pablo. \bibverse{33} Entonces llegando el tribuno, le prendió, y le
mandó atar con dos cadenas; y preguntó quién era, y qué había hecho.
\footnote{\textbf{21:33} Hech 20,23} \bibverse{34} Y entre la multitud,
unos gritaban una cosa, y otros otra: y como no podía entender nada de
cierto á causa del alboroto, le mandó llevar á la fortaleza.

\bibverse{35} Y como llegó á las gradas, aconteció que fué llevado de
los soldados á causa de la violencia del pueblo; \bibverse{36} Porque
multitud de pueblo venía detrás, gritando: Mátale. \bibverse{37} Y como
comenzaron á meter á Pablo en la fortaleza, dice al tribuno: ¿Me será
lícito hablarte algo? Y él dijo: ¿Sabes griego?

\bibverse{38} ¿No eres tú aquel Egipcio que levantaste una sedición
antes de estos días, y sacaste al desierto cuatro mil hombres
salteadores?

\bibverse{39} Entonces dijo Pablo: Yo de cierto soy hombre Judío,
ciudadano de Tarso, ciudad no obscura de Cilicia: empero ruégote que me
permitas que hable al pueblo.

\bibverse{40} Y como él se lo permitió, Pablo, estando en pie en las
gradas, hizo señal con la mano al pueblo. Y hecho grande silencio, habló
en lengua hebrea, diciendo:

\hypertarget{el-discurso-de-pablo-al-pueblo}{%
\subsection{El discurso de Pablo al
pueblo}\label{el-discurso-de-pablo-al-pueblo}}

\hypertarget{section-21}{%
\section{22}\label{section-21}}

\bibverse{1} Varones hermanos y padres, oid la razón que ahora os doy.

\bibverse{2} (Y como oyeron que les hablaba en lengua hebrea, guardaron
más silencio.) Y dijo: \footnote{\textbf{22:2} Hech 21,40}

\bibverse{3} Yo de cierto soy Judío, nacido en Tarso de Cilicia, mas
criado en esta ciudad á los pies de Gamaliel, enseñado conforme á la
verdad de la ley de la patria, celoso de Dios, como todos vosotros sois
hoy. \footnote{\textbf{22:3} Hech 5,34; Hech 9,1-29; Hech 26,9-20}
\bibverse{4} Que he perseguido este camino hasta la muerte, prendiendo y
entregando en cárceles hombres y mujeres: \footnote{\textbf{22:4} Hech
  8,3} \bibverse{5} Como también el príncipe de los sacerdotes me es
testigo, y todos los ancianos; de los cuales también tomando letras á
los hermanos, iba á Damasco para traer presos á Jerusalem aun á los que
estuviesen allí, para que fuesen castigados.

\bibverse{6} Mas aconteció que yendo yo, y llegando cerca de Damasco,
como á medio día, de repente me rodeó mucha luz del cielo: \bibverse{7}
Y caí en el suelo, y oí una voz que me decía: Saulo, Saulo, ¿por qué me
persigues? \bibverse{8} Yo entonces respondí: ¿Quién eres, Señor? Y me
dijo: Yo soy Jesús de Nazaret, á quien tú persigues.

\bibverse{9} Y los que estaban conmigo vieron á la verdad la luz, y se
espantaron; mas no oyeron la voz del que hablaba conmigo. \bibverse{10}
Y dije: ¿Qué haré, Señor? Y el Señor me dijo: Levántate, y ve á Damasco,
y allí te será dicho todo lo que te está señalado hacer. \bibverse{11} Y
como yo no viese por causa de la claridad de la luz, llevado de la mano
por los que estaban conmigo, vine á Damasco.

\bibverse{12} Entonces un Ananías, varón pío conforme á la ley, que
tenía buen testimonio de todos los Judíos que allí moraban,
\bibverse{13} Viniendo á mí, y acercándose, me dijo: Hermano Saulo,
recibe la vista. Y yo en aquella hora le miré. \bibverse{14} Y él dijo:
El Dios de nuestros padres te ha predestinado para que conocieses su
voluntad, y vieses á aquel Justo, y oyeses la voz de su boca.
\bibverse{15} Porque has de ser testigo suyo á todos los hombres, de lo
que has visto y oído. \bibverse{16} Ahora pues, ¿por qué te detienes?
Levántate, y bautízate, y lava tus pecados, invocando su nombre.

\bibverse{17} Y me aconteció, vuelto á Jerusalem, que orando en el
templo, fuí arrebatado fuera de mí. \bibverse{18} Y le vi que me decía:
Date prisa, y sal prestamente fuera de Jerusalem; porque no recibirán tu
testimonio de mí. \bibverse{19} Y yo dije: Señor, ellos saben que yo
encerraba en cárcel, y hería por las sinagogas á los que creían en ti;
\bibverse{20} Y cuando se derramaba la sangre de Esteban tu testigo, yo
también estaba presente, y consentía á su muerte, y guardaba las ropas
de los que le mataban.

\bibverse{21} Y me dijo: Ve, porque yo te tengo que enviar lejos á los
Gentiles. \footnote{\textbf{22:21} Hech 13,2}

\hypertarget{el-efecto-del-habla-pablo-bajo-custodia-con-el-coronel-romano}{%
\subsection{El efecto del habla; Pablo bajo custodia con el coronel
romano}\label{el-efecto-del-habla-pablo-bajo-custodia-con-el-coronel-romano}}

\bibverse{22} Y le oyeron hasta esta palabra: entonces alzaron la voz,
diciendo: Quita de la tierra á un tal hombre, porque no conviene que
viva. \footnote{\textbf{22:22} Hech 21,36}

\bibverse{23} Y dando ellos voces, y arrojando sus ropas y echando polvo
al aire, \bibverse{24} Mandó el tribuno que le llevasen á la fortaleza,
y ordenó que fuese examinado con azotes, para saber por qué causa
clamaban así contra él. \bibverse{25} Y como le ataron con correas,
Pablo dijo al centurión que estaba presente: ¿Os es lícito azotar á un
hombre Romano sin ser condenado? \footnote{\textbf{22:25} Hech 16,37;
  Hech 23,27}

\bibverse{26} Y como el centurión oyó esto, fué y dió aviso al tribuno,
diciendo: ¿Qué vas á hacer? porque este hombre es Romano.

\bibverse{27} Y viniendo el tribuno, le dijo: Dime, ¿eres tú Romano? Y
él dijo: Sí.

\bibverse{28} Y respondió el tribuno: Yo con grande suma alcancé esta
ciudadanía. Entonces Pablo dijo: Pero yo lo soy de nacimiento.

\bibverse{29} Así que, luego se apartaron de él los que le habían de
atormentar: y aun el tribuno también tuvo temor, entendido que era
Romano, por haberle atado.

\hypertarget{pablo-ante-el-sumo-consejo-juduxedo}{%
\subsection{Pablo ante el sumo consejo
judío}\label{pablo-ante-el-sumo-consejo-juduxedo}}

\bibverse{30} Y al día siguiente, queriendo saber de cierto la causa por
qué era acusado de los Judíos, le soltó de las prisiones, y mandó venir
á los príncipes de los sacerdotes, y á todo su concilio: y sacando á
Pablo, le presentó delante de ellos.

\hypertarget{section-22}{%
\section{23}\label{section-22}}

\bibverse{1} Entonces Pablo, poniendo los ojos en el concilio, dice:
Varones hermanos, yo con toda buena conciencia he conversado delante de
Dios hasta el día de hoy.

\bibverse{2} El príncipe de los sacerdotes, Ananías, mandó entonces á
los que estaban delante de él, que le hiriesen en la boca.

\bibverse{3} Entonces Pablo le dijo: Herirte ha Dios, pared blanqueada:
¿y estás tú sentado para juzgarme conforme á la ley, y contra la ley me
mandas herir? \footnote{\textbf{23:3} Mat 23,37}

\bibverse{4} Y los que estaban presentes dijeron: ¿Al sumo sacerdote de
Dios maldices?

\bibverse{5} Y Pablo dijo: No sabía, hermanos, que era el sumo
sacerdote; pues escrito está: Al príncipe de tu pueblo no maldecirás.

\bibverse{6} Entonces Pablo, sabiendo que la una parte era de Saduceos,
y la otra de Fariseos, clamó en el concilio: Varones hermanos, yo soy
Fariseo, hijo de Fariseo: de la esperanza y de la resurrección de los
muertos soy yo juzgado.

\bibverse{7} Y como hubo dicho esto, fué hecha disensión entre los
Fariseos y los Saduceos; y la multitud fué dividida. \bibverse{8} Porque
los Saduceos dicen que no hay resurrección, ni ángel, ni espíritu; mas
los Fariseos confiesan ambas cosas. \footnote{\textbf{23:8} Mat 22,23}
\bibverse{9} Y levantóse un gran clamor: y levantándose los escribas de
la parte de los Fariseos, contendían diciendo: Ningún mal hallamos en
este hombre; que si espíritu le ha hablado, ó ángel, no resistamos á
Dios. \footnote{\textbf{23:9} Hech 25,25; Hech 5,39}

\bibverse{10} Y habiendo grande disensión, el tribuno, teniendo temor de
que Pablo fuese despedazado de ellos, mandó venir soldados, y
arrebatarle de en medio de ellos, y llevarle á la fortaleza.

\bibverse{11} Y la noche siguiente, presentándosele el Señor, le dijo:
Confía, Pablo; que como has testificado de mí en Jerusalem, así es
menester testifiques también en Roma. \footnote{\textbf{23:11} Hech
  25,11-12; Hech 27,23-24}

\hypertarget{intento-de-asesinato-de-los-juduxedos-contra-pablo}{%
\subsection{Intento de asesinato de los judíos contra
Pablo}\label{intento-de-asesinato-de-los-juduxedos-contra-pablo}}

\bibverse{12} Y venido el día, algunos de los Judíos se juntaron, é
hicieron voto bajo de maldición, diciendo que ni comerían ni beberían
hasta que hubiesen muerto á Pablo. \bibverse{13} Y eran más de cuarenta
los que habían hecho esta conjuración; \bibverse{14} Los cuales se
fueron á los príncipes de los sacerdotes y á los ancianos, y dijeron:
Nosotros hemos hecho voto debajo de maldición, que no hemos de gustar
nada hasta que hayamos muerto á Pablo. \bibverse{15} Ahora pues,
vosotros, con el concilio, requerid al tribuno que le saque mañana á
vosotros como que queréis entender de él alguna cosa más cierta; y
nosotros, antes que él llegue, estaremos aparejados para matarle.

\bibverse{16} Entonces un hijo de la hermana de Pablo, oyendo las
asechanzas, fué, y entró en la fortaleza, y dió aviso á Pablo.
\bibverse{17} Y Pablo, llamando á uno de los centuriones, dice: Lleva á
este mancebo al tribuno, porque tiene cierto aviso que darle.

\bibverse{18} El entonces tomándole, le llevó al tribuno, y dijo: El
preso Pablo, llamándome, me rogó que trajese á ti este mancebo, que
tiene algo que hablarte.

\bibverse{19} Y el tribuno, tomándole de la mano y retirándose aparte,
le preguntó: ¿Qué es lo que tienes que decirme?

\bibverse{20} Y él dijo: Los Judíos han concertado rogarte que mañana
saques á Pablo al concilio, como que han de inquirir de él alguna cosa
más cierta. \bibverse{21} Mas tú no los creas; porque más de cuarenta
hombres de ellos le acechan, los cuales han hecho voto debajo de
maldición, de no comer ni beber hasta que le hayan muerto; y ahora están
apercibidos esperando tu promesa.

\bibverse{22} Entonces el tribuno despidió al mancebo, mandándole que á
nadie dijese que le había dado aviso de esto.

\hypertarget{carta-del-coronel-lysias-al-gobernador-fuxe9lix-traslado-de-pablo-de-jerusaluxe9n-a-cesarea}{%
\subsection{Carta del coronel Lysias al gobernador Félix; Traslado de
Pablo de Jerusalén a
Cesarea}\label{carta-del-coronel-lysias-al-gobernador-fuxe9lix-traslado-de-pablo-de-jerusaluxe9n-a-cesarea}}

\bibverse{23} Y llamados dos centuriones, mandó que apercibiesen para la
hora tercia de la noche doscientos soldados, que fuesen hasta Cesarea, y
setenta de á caballo, y doscientos lanceros;

\bibverse{24} Y que aparejasen cabalgaduras en que poniendo á Pablo, le
llevasen en salvo á Félix el Presidente. \bibverse{25} Y escribió una
carta en estos términos:

\bibverse{26} Claudio Lisias al excelentísimo gobernador Félix: Salud.

\bibverse{27} A este hombre, aprehendido de los Judíos, y que iban ellos
á matar, libré yo acudiendo con la tropa, habiendo entendido que era
Romano. \bibverse{28} Y queriendo saber la causa por qué le acusaban, le
llevé al concilio de ellos: \footnote{\textbf{23:28} Hech 22,30}
\bibverse{29} Y hallé que le acusaban de cuestiones de la ley de ellos,
y que ningún crimen tenía digno de muerte ó de prisión. \footnote{\textbf{23:29}
  Hech 18,14-15} \bibverse{30} Mas siéndome dado aviso de asechanzas que
le habían aparejado los Judíos, luego al punto le he enviado á ti,
intimando también á los acusadores que traten delante de ti lo que
tienen contra él. Pásalo bien. \footnote{\textbf{23:30} Hech 24,8}

\bibverse{31} Y los soldados, tomando á Pablo como les era mandado,
lleváronle de noche á Antipatris. \bibverse{32} Y al día siguiente,
dejando á los de á caballo que fuesen con él, se volvieron á la
fortaleza. \bibverse{33} Y como llegaron á Cesarea, y dieron la carta al
gobernador, presentaron también á Pablo delante de él. \bibverse{34} Y
el gobernador, leída la carta, preguntó de qué provincia era; y
entendiendo que de Cilicia, \bibverse{35} Te oiré, dijo, cuando vinieren
tus acusadores. Y mandó que le guardasen en el pretorio de Herodes.

\hypertarget{juicio-ante-el-gobernador-fuxe9lix}{%
\subsection{Juicio ante el gobernador
Félix}\label{juicio-ante-el-gobernador-fuxe9lix}}

\hypertarget{section-23}{%
\section{24}\label{section-23}}

\bibverse{1} Y cinco días después descendió el sumo sacerdote Ananías,
con algunos de los ancianos, y un cierto Tértulo, orador; y parecieron
delante del gobernador contra Pablo. \bibverse{2} Y citado que fué,
Tértulo comenzó á acusar, diciendo: Como por causa tuya vivamos en
grande paz, y muchas cosas sean bien gobernadas en el pueblo por tu
prudencia, \bibverse{3} Siempre y en todo lugar lo recibimos con todo
hacimiento de gracias, oh excelentísimo Félix. \bibverse{4} Empero por
no molestarte más largamente, ruégote que nos oigas brevemente conforme
á tu equidad. \bibverse{5} Porque hemos hallado que este hombre es
pestilencial, y levantador de sediciones entre todos los Judíos por todo
el mundo, y príncipe de la secta de los Nazarenos: \footnote{\textbf{24:5}
  Hech 17,6} \bibverse{6} El cual también tentó á violar el templo; y
prendiéndole, le quisimos juzgar conforme á nuestra ley: \footnote{\textbf{24:6}
  Hech 21,28-29} \bibverse{7} Mas interviniendo el tribuno Lisias, con
grande violencia le quitó de nuestras manos, \bibverse{8} Mandando á sus
acusadores que viniesen á ti; del cual tú mismo juzgando, podrás
entender todas estas cosas de que le acusamos. \footnote{\textbf{24:8}
  Hech 21,17}

\bibverse{9} Y contendían también los Judíos, diciendo ser así estas
cosas.

\bibverse{10} Entonces Pablo, haciéndole el gobernador señal que
hablase, respondió: Porque sé que muchos años ha eres gobernador de esta
nación, con buen ánimo satisfaré por mí. \bibverse{11} Porque tú puedes
entender que no hace más de doce días que subí á adorar á Jerusalem;
\bibverse{12} Y ni me hallaron en el templo disputando con ninguno, ni
haciendo concurso de multitud, ni en sinagogas, ni en la ciudad;
\bibverse{13} Ni te pueden probar las cosas de que ahora me acusan.
\bibverse{14} Esto empero te confieso, que conforme á aquel Camino que
llaman herejía, así sirvo al Dios de mis padres, creyendo todas las
cosas que en la ley y en los profetas están escritas; \bibverse{15}
Teniendo esperanza en Dios que ha de haber resurrección de los muertos,
así de justos como de injustos, la cual también ellos esperan.
\bibverse{16} Y por esto, procuro yo tener siempre conciencia sin
remordimiento acerca de Dios y acerca de los hombres. \footnote{\textbf{24:16}
  Hech 23,1} \bibverse{17} Mas pasados muchos años, vine á hacer
limosnas á mi nación, y ofrendas, \footnote{\textbf{24:17} Rom 15,25-26;
  Gal 2,10} \bibverse{18} Cuando me hallaron purificado en el templo (no
con multitud ni con alboroto) unos Judíos de Asia; \footnote{\textbf{24:18}
  Hech 21,27} \bibverse{19} Los cuales debieron comparecer delante de
ti, y acusarme, si contra mí tenían algo. \bibverse{20} O digan estos
mismos si hallaron en mí alguna cosa mal hecha, cuando yo estuve en el
concilio, \bibverse{21} Si no sea que, estando entre ellos prorrumpí en
alta voz: Acerca de la resurrección de los muertos soy hoy juzgado de
vosotros.

\bibverse{22} Entonces Félix, oídas estas cosas, estando bien informado
de esta secta, les puso dilación, diciendo: Cuando descendiere el
tribuno Lisias acabaré de conocer de vuestro negocio. \footnote{\textbf{24:22}
  Hech 23,26} \bibverse{23} Y mandó al centurión que Pablo fuese
guardado, y aliviado de las prisiones; y que no vedase á ninguno de sus
familiares servirle, ó venir á él. \footnote{\textbf{24:23} Hech 27,3}

\hypertarget{pablo-ante-felix-y-drusilla-felix-retrasuxf3-el-juicio}{%
\subsection{Pablo ante Felix y Drusilla; Felix retrasó el
juicio}\label{pablo-ante-felix-y-drusilla-felix-retrasuxf3-el-juicio}}

\bibverse{24} Y algunos días después, viniendo Félix con Drusila, su
mujer, la cual era Judía, llamó á Pablo, y oyó de él la fe que es en
Jesucristo. \bibverse{25} Y disertando él de la justicia, y de la
continencia, y del juicio venidero, espantado Félix, respondió: Ahora
vete; mas en teniendo oportunidad te llamaré: \bibverse{26} Esperando
también con esto, que de parte de Pablo le serían dados dineros, porque
le soltase; por lo cual, haciéndole venir muchas veces, hablaba con él.

\bibverse{27} Mas al cabo de dos años recibió Félix por sucesor á Porcio
Festo: y queriendo Félix ganar la gracia de los Judíos, dejó preso á
Pablo.

\hypertarget{reanudaciuxf3n-del-proceso-festo-en-jerusaluxe9n-y-cesarea-pablo-apela-al-emperador}{%
\subsection{Reanudación del proceso; Festo en Jerusalén y Cesarea; Pablo
apela al
emperador}\label{reanudaciuxf3n-del-proceso-festo-en-jerusaluxe9n-y-cesarea-pablo-apela-al-emperador}}

\hypertarget{section-24}{%
\section{25}\label{section-24}}

\bibverse{1} Festo pues, entrado en la provincia, tres días después
subió de Cesarea á Jerusalem. \bibverse{2} Y vinieron á él los príncipes
de los sacerdotes y los principales de los Judíos contra Pablo; y le
rogaron, \footnote{\textbf{25:2} Hech 24,1} \bibverse{3} Pidiendo gracia
contra él, que le hiciese traer á Jerusalem, poniendo ellos asechanzas
para matarle en el camino. \footnote{\textbf{25:3} Hech 23,15}
\bibverse{4} Mas Festo respondió, que Pablo estaba guardado en Cesarea,
y que él mismo partiría presto. \bibverse{5} Los que de vosotros pueden,
dijo, desciendan juntamente; y si hay algún crimen en este varón,
acúsenle.

\bibverse{6} Y deteniéndose entre ellos no más de ocho ó diez días,
venido á Cesarea, el siguiente día se sentó en el tribunal, y mandó que
Pablo fuese traído. \bibverse{7} El cual venido, le rodearon los Judíos
que habían venido de Jerusalem, poniendo contra Pablo muchas y graves
acusaciones, las cuales no podían probar; \bibverse{8} Alegando él por
su parte: Ni contra la ley de los Judíos, ni contra el templo, ni contra
César he pecado en nada.

\bibverse{9} Mas Festo, queriendo congraciarse con los Judíos,
respondiendo á Pablo, dijo: ¿Quieres subir á Jerusalem, y allá ser
juzgado de estas cosas delante de mí?

\bibverse{10} Y Pablo dijo: Ante el tribunal de César estoy, donde
conviene que sea juzgado. A los Judíos no he hecho injuria ninguna, como
tú sabes muy bien. \bibverse{11} Porque si alguna injuria, ó cosa alguna
digna de muerte he hecho, no rehuso morir; mas si nada hay de las cosas
de que éstos me acusan, nadie puede darme á ellos. A César apelo.
\footnote{\textbf{25:11} Hech 23,11; Hech 28,19}

\bibverse{12} Entonces Festo, habiendo hablado con el consejo,
respondió: ¿A César has apelado? á César irás.

\hypertarget{herodes-agripa-ii-y-berenice-como-invitados-en-festo-en-cesarea-festo-informa-a-agripa-de-la-causa-de-pablo}{%
\subsection{Herodes Agripa II y Berenice como invitados en Festo en
Cesarea; Festo informa a Agripa de la causa de
Pablo}\label{herodes-agripa-ii-y-berenice-como-invitados-en-festo-en-cesarea-festo-informa-a-agripa-de-la-causa-de-pablo}}

\bibverse{13} Y pasados algunos días, el rey Agripa y Bernice vinieron á
Cesarea á saludar á Festo. \bibverse{14} Y como estuvieron allí muchos
días, Festo declaró la causa de Pablo al rey, diciendo: Un hombre ha
sido dejado preso por Félix, \bibverse{15} Sobre el cual, cuando fuí á
Jerusalem, vinieron á mí los príncipes de los sacerdotes y los ancianos
de los Judíos, pidiendo condenación contra él: \bibverse{16} A los
cuales respondí, no ser costumbre de los Romanos dar alguno á la muerte
antes que el que es acusado tenga presentes sus acusadores, y haya lugar
de defenderse de la acusación. \footnote{\textbf{25:16} Hech 22,25}
\bibverse{17} Así que, habiendo venido ellos juntos acá, sin ninguna
dilación, al día siguiente, sentado en el tribunal, mandé traer al
hombre; \bibverse{18} Y estando presentes los acusadores, ningún cargo
produjeron de los que yo sospechaba: \bibverse{19} Solamente tenían
contra él ciertas cuestiones acerca de su superstición, y de un cierto
Jesús, difunto, el cual Pablo afirmaba que estaba vivo. \bibverse{20} Y
yo, dudando en cuestión semejante, dije, si quería ir á Jerusalem, y
allá ser juzgado de estas cosas. \bibverse{21} Mas apelando Pablo á ser
guardado al conocimiento de Augusto, mandé que le guardasen hasta que le
enviara á César.

\bibverse{22} Entonces Agripa dijo á Festo: Yo también quisiera oir á
ese hombre. Y él dijo: Mañana le oirás. \footnote{\textbf{25:22} Luc
  23,8}

\hypertarget{discurso-de-manifestaciuxf3n-y-defensa-de-pablo-frente-a-agripa-y-festo}{%
\subsection{Discurso de manifestación y defensa de Pablo frente a Agripa
y
Festo}\label{discurso-de-manifestaciuxf3n-y-defensa-de-pablo-frente-a-agripa-y-festo}}

\bibverse{23} Y al otro día, viniendo Agripa y Bernice con mucho
aparato, y entrando en la audiencia con los tribunos y principales
hombres de la ciudad, por mandato de Festo, fué traído Pablo.
\bibverse{24} Entonces Festo dijo: Rey Agripa, y todos los varones que
estáis aquí juntos con nosotros: veis á éste, por el cual toda la
multitud de los Judíos me ha demandado en Jerusalem y aquí, dando voces
que no conviene que viva más; \bibverse{25} Mas yo, hallando que ninguna
cosa digna de muerte ha hecho, y él mismo apelando á Augusto, he
determinado enviarle: \bibverse{26} Del cual no tengo cosa cierta que
escriba al señor; por lo que le he sacado á vosotros, y mayormente á tí,
oh rey Agripa, para que hecha información, tenga yo qué escribir.
\bibverse{27} Porque fuera de razón me parece enviar un preso, y no
informar de las causas.

\hypertarget{discurso-defensivo-de-pablo-ante-agripa}{%
\subsection{Discurso defensivo de Pablo ante
Agripa}\label{discurso-defensivo-de-pablo-ante-agripa}}

\hypertarget{section-25}{%
\section{26}\label{section-25}}

\bibverse{1} Entonces Agripa dijo á Pablo: Se te permite hablar por ti
mismo. Pablo entonces, extendiendo la mano, comenzó á responder por sí,
diciendo:

\bibverse{2} Acerca de todas las cosas de que soy acusado por los
Judíos, oh rey Agripa, me tengo por dichoso de que haya hoy de
defenderme delante de ti; \bibverse{3} Mayormente sabiendo tú todas las
costumbres y cuestiones que hay entre los Judíos: por lo cual te ruego
que me oigas con paciencia.

\bibverse{4} Mi vida pues desde la mocedad, la cual desde el principio
fué en mi nación, en Jerusalem, todos los Judíos la saben: \bibverse{5}
Los cuales tienen ya conocido que yo desde el principio, si quieren
testificarlo, conforme á la más rigurosa secta de nuestra religión he
vivido Fariseo. \footnote{\textbf{26:5} Hech 23,6; Fil 3,5} \bibverse{6}
Y ahora, por la esperanza de la promesa que hizo Dios á nuestros padres,
soy llamado en juicio; \footnote{\textbf{26:6} Hech 28,20} \bibverse{7}
A la cual promesa nuestras doce tribus, sirviendo constantemente de día
y de noche, esperan que han de llegar. Por la cual esperanza, oh rey
Agripa, soy acusado de los Judíos. \footnote{\textbf{26:7} Hech 24,15}
\bibverse{8} ¡Qué! ¿Júzgase cosa increíble entre vosotros que Dios
resucite los muertos? \footnote{\textbf{26:8} Hech 23,8}

\bibverse{9} Yo ciertamente había pensado deber hacer muchas cosas
contra el nombre de Jesús de Nazaret: \footnote{\textbf{26:9} Hech
  9,1-29; Hech 22,3-21} \bibverse{10} Lo cual también hice en Jerusalem,
y yo encerré en cárceles á muchos de los santos, recibida potestad de
los príncipes de los sacerdotes; y cuando eran matados, yo dí mi voto.
\bibverse{11} Y muchas veces, castigándolos por todas las sinagogas, los
forcé á blasfemar; y enfurecido sobremanera contra ellos, los perseguí
hasta en las ciudades extrañas.

\bibverse{12} En lo cual ocupado, yendo á Damasco con potestad y
comisión de los príncipes de los sacerdotes, \bibverse{13} En mitad del
día, oh rey, vi en el camino una luz del cielo, que sobrepujaba el
resplandor del sol, la cual me rodeó y á los que iban conmigo.
\bibverse{14} Y habiendo caído todos nosotros en tierra, oí una voz que
me hablaba, y decía en lengua hebraica: Saulo, Saulo, ¿por qué me
persigues? Dura cosa te es dar coces contra los aguijones.

\bibverse{15} Yo entonces dije: ¿Quién eres, Señor? Y el Señor dijo: Yo
soy Jesús, á quien tú persigues.

\bibverse{16} Mas levántate, y ponte sobre tus pies; porque para esto te
he aparecido, para ponerte por ministro y testigo de las cosas que has
visto, y de aquellas en que apareceré á ti: \bibverse{17} Librándote del
pueblo y de los Gentiles, á los cuales ahora te envío, \bibverse{18}
Para que abras sus ojos, para que se conviertan de las tinieblas á la
luz, y de la potestad de Satanás á Dios; para que reciban, por la fe que
es en mí, remisión de pecados y suerte entre los santificados.

\bibverse{19} Por lo cual, oh rey Agripa, no fuí rebelde á la visión
celestial: \footnote{\textbf{26:19} Gal 1,16} \bibverse{20} Antes
anuncié primeramente á los que están en Damasco, y Jerusalem, y por toda
la tierra de Judea, y á los Gentiles, que se arrepintiesen y se
convirtiesen á Dios, haciendo obras dignas de arrepentimiento.
\bibverse{21} Por causa de esto los Judíos, tomándome en el templo,
tentaron matarme. \bibverse{22} Mas ayudado del auxilio de Dios,
persevero hasta el día de hoy, dando testimonio á pequeños y á grandes,
no diciendo nada fuera de las cosas que los profetas y Moisés dijeron
que habían de venir: \footnote{\textbf{26:22} Luc 24,44-47}
\bibverse{23} Que Cristo había de padecer, y ser el primero de la
resurrección de los muertos, para anunciar luz al pueblo y á los
Gentiles. \footnote{\textbf{26:23} 1Cor 15,20}

\hypertarget{impresiuxf3n-del-discurso}{%
\subsection{Impresión del discurso}\label{impresiuxf3n-del-discurso}}

\bibverse{24} Y diciendo él estas cosas en su defensa, Festo á gran voz
dijo: Estás loco, Pablo: las muchas letras te vuelven loco.

\bibverse{25} Mas él dijo: No estoy loco, excelentísimo Festo, sino que
hablo palabras de verdad y de templanza. \bibverse{26} Pues el rey sabe
estas cosas, delante del cual también hablo confiadamente. Pues no
pienso que ignora nada de esto; pues no ha sido esto hecho en algún
rincón. \footnote{\textbf{26:26} Juan 18,20} \bibverse{27} ¿Crees, rey
Agripa, á los profetas? Yo sé que crees.

\bibverse{28} Entonces Agripa dijo á Pablo: Por poco me persuades á ser
Cristiano.

\bibverse{29} Y Pablo dijo: ¡Pluguiese á Dios que por poco ó por mucho,
no solamente tú, mas también todos los que hoy me oyen, fueseis hechos
tales cual yo soy, excepto estas prisiones!

\bibverse{30} Y como hubo dicho estas cosas, se levantó el rey, y el
presidente, y Bernice, y los que se habían sentado con ellos;
\bibverse{31} Y como se retiraron aparte, hablaban los unos á los otros,
diciendo: Ninguna cosa digna ni de muerte, ni de prisión, hace este
hombre. \bibverse{32} Y Agripa dijo á Festo: Podía este hombre ser
suelto, si no hubiera apelado á César.

\hypertarget{el-viaje-de-pablo-de-cesarea-a-roma}{%
\subsection{El viaje de Pablo de Cesarea a
Roma}\label{el-viaje-de-pablo-de-cesarea-a-roma}}

\hypertarget{section-26}{%
\section{27}\label{section-26}}

\bibverse{1} Mas como fué determinado que habíamos de navegar para
Italia, entregaron á Pablo y á algunos otros presos á un centurión,
llamado Julio, de la compañía Augusta. \footnote{\textbf{27:1} Hech
  25,12} \bibverse{2} Así que, embarcándonos en una nave Adrumentina,
partimos, estando con nosotros Aristarco, Macedonio de Tesalónica, para
navegar junto á los lugares de Asia. \footnote{\textbf{27:2} Hech 20,4}
\bibverse{3} Y otro día llegamos á Sidón; y Julio, tratando á Pablo con
humanidad, permitióle que fuese á los amigos, para ser de ellos
asistido. \footnote{\textbf{27:3} Hech 24,23; Hech 28,16} \bibverse{4} Y
haciéndonos á la vela desde allí, navegamos bajo de Cipro, porque los
vientos eran contrarios. \bibverse{5} Y habiendo pasado la mar de
Cilicia y Pamphylia, arribamos á Mira, ciudad de Licia. \bibverse{6} Y
hallando allí el centurión una nave Alejandrina que navegaba á Italia,
nos puso en ella. \bibverse{7} Y navegando muchos días despacio, y
habiendo apenas llegado delante de Gnido, no dejándonos el viento,
navegamos bajo de Creta, junto á Salmón. \bibverse{8} Y costeándola
difícilmente, llegamos á un lugar que llaman Buenos Puertos, cerca del
cual estaba la ciudad de Lasea.

\bibverse{9} Y pasado mucho tiempo, y siendo ya peligrosa la navegación,
porque ya era pasado el ayuno, Pablo amonestaba, \bibverse{10}
Diciéndoles: Varones, veo que con trabajo y mucho daño, no sólo de la
cargazón y de la nave, mas aun de nuestras personas, habrá de ser la
navegación. \bibverse{11} Mas el centurión creía más al piloto y al
patrón de la nave, que á lo que Pablo decía. \bibverse{12} Y no habiendo
puerto cómodo para invernar, muchos acordaron pasar aún de allí, por si
pudiesen arribar á Fenice é invernar allí, que es un puerto de Creta que
mira al Nordeste y Sudeste.

\hypertarget{tormenta-marina-y-naufragio-rescate-en-malta}{%
\subsection{Tormenta marina y naufragio; Rescate en
Malta}\label{tormenta-marina-y-naufragio-rescate-en-malta}}

\bibverse{13} Y soplando el austro, pareciéndoles que ya tenían lo que
deseaban, alzando velas, iban cerca de la costa de Creta. \bibverse{14}
Mas no mucho después dió en ella un viento repentino, que se llama
Euroclidón. \bibverse{15} Y siendo arrebatada la nave, y no pudiendo
resistir contra el viento, la dejamos, y éramos llevados. \bibverse{16}
Y habiendo corrido á sotavento de una pequeña isla que se llama Clauda,
apenas pudimos ganar el esquife: \bibverse{17} El cual tomado, usaban de
remedios, ciñendo la nave; y teniendo temor de que diesen en la Sirte,
abajadas las velas, eran así llevados. \bibverse{18} Mas siendo
atormentados de una vehemente tempestad, al siguiente día alijaron;
\bibverse{19} Y al tercer día nosotros con nuestras manos arrojamos los
aparejos de la nave. \bibverse{20} Y no pareciendo sol ni estrellas por
muchos días, y viniendo una tempestad no pequeña, ya era perdida toda la
esperanza de nuestra salud.

\hypertarget{pablo-como-consejero-consolador-y-salvador-en-angustia}{%
\subsection{Pablo como consejero, consolador y salvador en
angustia}\label{pablo-como-consejero-consolador-y-salvador-en-angustia}}

\bibverse{21} Entonces Pablo, habiendo ya mucho que no comíamos, puesto
en pie en medio de ellos, dijo: Fuera de cierto conveniente, oh varones,
haberme oído, y no partir de Creta, y evitar este inconveniente y daño.
\bibverse{22} Mas ahora os amonesto que tengáis buen ánimo; porque
ninguna pérdida habrá de persona de vosotros, sino solamente de la nave.
\bibverse{23} Porque esta noche ha estado conmigo el ángel del Dios del
cual yo soy, y al cual sirvo, \bibverse{24} Diciendo: Pablo, no temas;
es menester que seas presentado delante de César; y he aquí, Dios te ha
dado todos los que navegan contigo. \footnote{\textbf{27:24} Hech 23,11}
\bibverse{25} Por tanto, oh varones, tened buen ánimo; porque yo confío
en Dios que será así como me ha dicho; \bibverse{26} Si bien es menester
que demos en una isla.

\bibverse{27} Y venida la décimacuarta noche, y siendo llevados por el
mar Adriático, los marineros á la media noche sospecharon que estaban
cerca de alguna tierra; \bibverse{28} Y echando la sonda, hallaron
veinte brazas; y pasando un poco más adelante, volviendo á echar la
sonda, hallaron quince brazas. \bibverse{29} Y habiendo temor de dar en
lugares escabrosos, echando cuatro anclas de la popa, deseaban que se
hiciese de día. \bibverse{30} Entonces procurando los marineros huir de
la nave, echado que hubieron el esquife á la mar, aparentando como que
querían largar las anclas de proa, \bibverse{31} Pablo dijo al centurión
y á los soldados: Si éstos no quedan en la nave, vosotros no podéis
salvaros. \bibverse{32} Entonces los soldados cortaron los cabos del
esquife, y dejáronlo perder.

\bibverse{33} Y como comenzó á ser de día, Pablo exhortaba á todos que
comiesen, diciendo: Este es el décimocuarto día que esperáis y
permanecéis ayunos, no comiendo nada. \bibverse{34} Por tanto, os ruego
que comáis por vuestra salud: que ni aun un cabello de la cabeza de
ninguno de vosotros perecerá. \footnote{\textbf{27:34} Mat 10,30}
\bibverse{35} Y habiendo dicho esto, tomando el pan, hizo gracias á Dios
en presencia de todos, y partiendo, comenzó á comer. \footnote{\textbf{27:35}
  Juan 6,11} \bibverse{36} Entonces todos teniendo ya mejor ánimo,
comieron ellos también. \bibverse{37} Y éramos todas las personas en la
nave doscientas setenta y seis. \bibverse{38} Y satisfechos de comida,
aliviaban la nave, echando el grano á la mar.

\hypertarget{naufragio-en-la-faz-de-la-isla-de-malta-rescata-a-los-nuxe1ufragos}{%
\subsection{Naufragio en la faz de la isla de Malta; Rescata a los
náufragos}\label{naufragio-en-la-faz-de-la-isla-de-malta-rescata-a-los-nuxe1ufragos}}

\bibverse{39} Y como se hizo de día, no conocían la tierra: mas veían un
golfo que tenía orilla, al cual acordaron echar, si pudiesen, la nave.
\bibverse{40} Cortando pues las anclas, las dejaron en la mar, largando
también las ataduras de los gobernalles; y alzada la vela mayor al
viento, íbanse á la orilla. \bibverse{41} Mas dando en un lugar de dos
aguas, hicieron encallar la nave; y la proa, hincada, estaba sin
moverse, y la popa se abría con la fuerza de la mar.

\bibverse{42} Entonces el acuerdo de los soldados era que matasen los
presos, porque ninguno se fugase nadando. \bibverse{43} Mas el
centurión, queriendo salvar á Pablo, estorbó este acuerdo, y mandó que
los que pudiesen nadar, se echasen los primeros, y saliesen á tierra;
\bibverse{44} Y los demás, parte en tablas, parte en cosas de la nave. Y
así aconteció que todos se salvaron saliendo á tierra. \footnote{\textbf{27:44}
  Hech 27,22-25}

\hypertarget{invernada-en-la-isla-de-malta-continuaciuxf3n-del-viaje-a-roma}{%
\subsection{Invernada en la isla de Malta; Continuación del viaje a
Roma}\label{invernada-en-la-isla-de-malta-continuaciuxf3n-del-viaje-a-roma}}

\hypertarget{section-27}{%
\section{28}\label{section-27}}

\bibverse{1} Y cuando escapamos, entonces supimos que la isla se llamaba
Melita. \bibverse{2} Y los bárbaros nos mostraron no poca humanidad;
porque, encendido un fuego, nos recibieron á todos, á causa de la lluvia
que venía, y del frío.

\hypertarget{salvaciuxf3n-de-pablo-del-peligro-de-la-vida}{%
\subsection{Salvación de Pablo del peligro de la
vida}\label{salvaciuxf3n-de-pablo-del-peligro-de-la-vida}}

\bibverse{3} Entonces habiendo Pablo recogido algunos sarmientos, y
puéstolos en el fuego, una víbora, huyendo del calor, le acometió á la
mano. \bibverse{4} Y como los bárbaros vieron la víbora colgando de su
mano, decían los unos á los otros: Ciertamente este hombre es homicida,
á quien, escapado de la mar, la justicia no deja vivir. \bibverse{5} Mas
él, sacudiendo la víbora en el fuego, ningún mal padeció. \footnote{\textbf{28:5}
  Mar 16,18} \bibverse{6} Empero ellos estaban esperando cuándo se había
de hinchar, ó caer muerto de repente; mas habiendo esperado mucho, y
viendo que ningún mal le venía, mudados, decían que era un dios.
\footnote{\textbf{28:6} Hech 14,11}

\hypertarget{pablo-sana-al-padre-de-publio-y-a-otras-personas-enfermas}{%
\subsection{Pablo sana al padre de Publio y a otras personas
enfermas}\label{pablo-sana-al-padre-de-publio-y-a-otras-personas-enfermas}}

\bibverse{7} En aquellos lugares había heredades del principal de la
isla, llamado Publio, el cual nos recibió y hospedó tres días
humanamente. \bibverse{8} Y aconteció que el padre de Publio estaba en
cama, enfermo de fiebres y de disentería: al cual Pablo entró, y después
de haber orado, le puso las manos encima, y le sanó: \bibverse{9} Y esto
hecho, también los otros que en la isla tenían enfermedades, llegaban, y
eran sanados: \bibverse{10} Los cuales también nos honraron con muchos
obsequios; y cuando partimos, nos cargaron de las cosas necesarias.

\hypertarget{continuaciuxf3n-del-viaje-vuxeda-siracusa-y-puteoli-hasta-roma}{%
\subsection{Continuación del viaje vía Siracusa y Puteoli hasta
Roma}\label{continuaciuxf3n-del-viaje-vuxeda-siracusa-y-puteoli-hasta-roma}}

\bibverse{11} Así que, pasados tres meses, navegamos en una nave
Alejandrina que había invernado en la isla, la cual tenía por enseña á
Cástor y Pólux. \bibverse{12} Y llegados á Siracusa, estuvimos allí tres
días. \bibverse{13} De allí, costeando alrededor, vinimos á Regio; y
otro día después, soplando el austro, vinimos al segundo día á Puteolos:
\bibverse{14} Donde habiendo hallado hermanos, nos rogaron que
quedásemos con ellos siete días; y luego vinimos á Roma; \bibverse{15}
De donde, oyendo de nosotros los hermanos, nos salieron á recibir hasta
la plaza de Appio, y Las Tres Tabernas: á los cuales como Pablo vió, dió
gracias á Dios, y tomó aliento.

\hypertarget{pablo-en-roma}{%
\subsection{Pablo en Roma}\label{pablo-en-roma}}

\bibverse{16} Y como llegamos á Roma, el centurión entregó los presos al
prefecto de los ejércitos, mas á Pablo fué permitido estar por sí, con
un soldado que le guardase. \footnote{\textbf{28:16} Hech 27,3}

\hypertarget{negociaciones-de-pablo-con-los-jefes-de-los-juduxedos-romanos}{%
\subsection{Negociaciones de Pablo con los jefes de los judíos
romanos}\label{negociaciones-de-pablo-con-los-jefes-de-los-juduxedos-romanos}}

\bibverse{17} Y aconteció que tres días después, Pablo convocó á los
principales de los Judíos; á los cuales, luego que estuvieron juntos,
les dijo: Yo, varones hermanos, no habiendo hecho nada contra el pueblo,
ni contra los ritos de la patria, he sido entregado preso desde
Jerusalem en manos de los Romanos; \footnote{\textbf{28:17} Hech 23,1}
\bibverse{18} Los cuales, habiéndome examinado, me querían soltar, por
no haber en mí ninguna causa de muerte. \bibverse{19} Mas contradiciendo
los Judíos, fuí forzado á apelar á César; no que tenga de qué acusar á
mi nación. \footnote{\textbf{28:19} Hech 25,11} \bibverse{20} Así que,
por esta causa, os he llamado para veros y hablaros; porque por la
esperanza de Israel estoy rodeado de esta cadena. \footnote{\textbf{28:20}
  Hech 26,6-7}

\bibverse{21} Entonces ellos le dijeron: Nosotros ni hemos recibido
cartas tocante á ti de Judea, ni ha venido alguno de los hermanos que
haya denunciado ó hablado algún mal de ti. \bibverse{22} Mas querríamos
oir de ti lo que sientes; porque de esta secta notorio nos es que en
todos lugares es contradicha. \footnote{\textbf{28:22} Hech 24,14; Luc
  2,34}

\bibverse{23} Y habiéndole señalado un día, vinieron á él muchos á la
posada, á los cuales declaraba y testificaba el reino de Dios,
persuadiéndoles lo concerniente á Jesús, por la ley de Moisés y por los
profetas, desde la mañana hasta la tarde. \bibverse{24} Y algunos
asentían á lo que se decía, mas algunos no creían. \bibverse{25} Y como
fueron entre sí discordes, se fueron, diciendo Pablo esta palabra: Bien
ha hablado el Espíritu Santo por el profeta Isaías á nuestros padres,
\bibverse{26} Diciendo: Ve á este pueblo, y diles: De oído oiréis, y no
entenderéis; y viendo veréis, y no percibiréis: \bibverse{27} Porque el
corazón de este pueblo se ha engrosado, y de los oídos oyeron
pesadamente, y sus ojos taparon; porque no vean con los ojos, y oigan
con los oídos, y entiendan de corazón, y se conviertan, y yo los sane.

\bibverse{28} Séaos pues notorio que á los Gentiles es enviada esta
salud de Dios: y ellos oirán.

\bibverse{29} Y habiendo dicho esto, los Judíos salieron, teniendo entre
sí gran contienda.

\hypertarget{el-ministerio-de-dos-auxf1os-de-pablo-en-cautiverio-en-roma}{%
\subsection{El ministerio de dos años de Pablo en cautiverio en
Roma}\label{el-ministerio-de-dos-auxf1os-de-pablo-en-cautiverio-en-roma}}

\bibverse{30} Pablo empero, quedó dos años enteros en su casa de
alquiler, y recibía á todos los que á él venían, \bibverse{31}
Predicando el reino de Dios y enseñando lo que es del Señor Jesucristo
con toda libertad, sin impedimento.
