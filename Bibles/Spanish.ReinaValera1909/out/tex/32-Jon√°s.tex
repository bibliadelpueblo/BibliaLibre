\hypertarget{el-llamado-la-desobediencia-y-el-castigo-de-jonuxe1s}{%
\subsection{El llamado, la desobediencia y el castigo de
Jonás}\label{el-llamado-la-desobediencia-y-el-castigo-de-jonuxe1s}}

\hypertarget{section}{%
\section{1}\label{section}}

\bibverse{1} Y fué palabra de Jehová á Jonás, hijo de Amittai, diciendo:
\footnote{\textbf{1:1} 2Re 14,25} \bibverse{2} Levántate, y ve á Nínive,
ciudad grande, y pregona contra ella; porque su maldad ha subido delante
de mí.

\bibverse{3} Y Jonás se levantó para huir de la presencia de Jehová á
Tarsis, y descendió á Joppe; y halló un navío que partía para Tarsis; y
pagando su pasaje entró en él, para irse con ellos á Tarsis de delante
de Jehová.

\bibverse{4} Mas Jehová hizo levantar un gran viento en la mar, é hízose
una tan gran tempestad en la mar, que pensóse se rompería la nave.
\bibverse{5} Y los marineros tuvieron miedo, y cada uno llamaba á su
dios: y echaron á la mar los enseres que había en la nave, para
descargarla de ellos. Jonás empero se había bajado á los lados del
buque, y se había echado á dormir. \bibverse{6} Y el maestre de la nave
se llegó á él, y le dijo: ¿Qué tienes, dormilón? Levántate, y clama á tu
Dios; quizá él tendrá compasión de nosotros, y no pereceremos.

\bibverse{7} Y dijeron cada uno á su compañero: Venid, y echemos
suertes, para saber por quién nos ha venido este mal. Y echaron suertes,
y la suerte cayó sobre Jonás. \footnote{\textbf{1:7} Prov 16,33}
\bibverse{8} Entonces le dijeron ellos: Decláranos ahora por qué nos ha
venido este mal. ¿Qué oficio tienes, y de dónde vienes? ¿cuál es tu
tierra, y de qué pueblo eres?

\bibverse{9} Y él les respondió: Hebreo soy, y temo á Jehová, Dios de
los cielos, que hizo la mar y la tierra.

\bibverse{10} Y aquellos hombres temieron sobremanera, y dijéronle: ¿Por
qué has hecho esto? Porque ellos entendieron que huía de delante de
Jehová, porque se lo había declarado. \bibverse{11} Y dijéronle: ¿Qué te
haremos, para que la mar se nos quiete? porque la mar iba á más, y se
embravecía.

\bibverse{12} El les respondió: Tomadme, y echadme á la mar, y la mar se
os quietará: porque yo sé que por mí ha venido esta grande tempestad
sobre vosotros.

\bibverse{13} Y aquellos hombres trabajaron por tornar la nave á tierra;
mas no pudieron, porque la mar iba á más, y se embravecía sobre ellos.
\bibverse{14} Entonces clamaron á Jehová, y dijeron: Rogámoste ahora,
Jehová, que no perezcamos nosotros por la vida de aqueste hombre, ni
pongas sobre nosotros la sangre inocente: porque tú, Jehová, has hecho
como has querido. \bibverse{15} Y tomaron á Jonás, y echáronlo á la mar;
y la mar se quietó de su furia. \bibverse{16} Y temieron aquellos
hombres á Jehová con gran temor; y ofrecieron sacrificio á Jehová, y
prometieron votos. \bibverse{17}

\hypertarget{jonuxe1s-oraciuxf3n-y-salvaciuxf3n}{%
\subsection{Jonás oración y
salvación}\label{jonuxe1s-oraciuxf3n-y-salvaciuxf3n}}

\hypertarget{section-1}{%
\section{2}\label{section-1}}

\bibverse{1} Mas Jehová había prevenido un gran pez que tragase á Jonás:
y estuvo Jonás en el vientre del pez tres días y tres noches.
\footnote{\textbf{2:1} Mat 12,40; Mat 16,4}

\bibverse{2} Y oró Jonás desde el vientre del pez á Jehová su Dios,
\bibverse{3} Y dijo: Clamé de mi tribulación á Jehová, y él me oyó; del
vientre del sepulcro clamé, y mi voz oiste. \bibverse{4} Echásteme en el
profundo, en medio de los mares, y rodeóme la corriente; todas tus ondas
y tus olas pasaron sobre mí. \footnote{\textbf{2:4} Sal 42,8}
\bibverse{5} Y yo dije: Echado soy de delante de tus ojos: mas aun veré
tu santo templo. \footnote{\textbf{2:5} Sal 31,23} \bibverse{6} Las
aguas me rodearon hasta el alma, rodeóme el abismo; la ova se enredó á
mi cabeza. \footnote{\textbf{2:6} Sal 18,5; Sal 69,2} \bibverse{7}
Descendí á las raíces de los montes; la tierra echó sus cerraduras sobre
mí para siempre: mas tú sacaste mi vida de la sepultura, oh Jehová Dios
mío. \footnote{\textbf{2:7} Sal 103,4} \bibverse{8} Cuando mi alma
desfallecía en mí, acordéme de Jehová; y mi oración entró hasta ti en tu
santo templo. \footnote{\textbf{2:8} Sal 142,4} \bibverse{9} Los que
guardan las vanidades ilusorias, su misericordia abandonan. \footnote{\textbf{2:9}
  Sal 31,7}

\bibverse{10} Yo empero con voz de alabanza te sacrificaré; pagaré lo
que prometí. La salvación pertenece á Jehová. Y mandó Jehová al pez, y
vomitó á Jonás en tierra. \footnote{\textbf{2:10} Sal 50,14; Sal
  116,17-18}

\hypertarget{jonuxe1s-exitoso-sermuxf3n-penitencial-en-nuxednive}{%
\subsection{Jonás exitoso sermón penitencial en
Nínive}\label{jonuxe1s-exitoso-sermuxf3n-penitencial-en-nuxednive}}

\hypertarget{section-2}{%
\section{3}\label{section-2}}

\bibverse{1} Y fué palabra de Jehová segunda vez á Jonás, diciendo:
\bibverse{2} Levántate, y ve á Nínive, aquella gran ciudad, y publica en
ella el pregón que yo te diré.

\bibverse{3} Y levantóse Jonás, y fué á Nínive, conforme á la palabra de
Jehová. Y era Nínive ciudad sobremanera grande, de tres días de camino.
\footnote{\textbf{3:3} Jon 4,11} \bibverse{4} Y comenzó Jonás á entrar
por la ciudad, camino de un día, y pregonaba diciendo: De aquí á
cuarenta días Nínive será destruída.

\bibverse{5} Y los hombres de Nínive creyeron á Dios, y pregonaron
ayuno, y vistiéronse de sacos desde el mayor de ellos hasta el menor de
ellos. \bibverse{6} Y llegó el negocio hasta el rey de Nínive, y
levantóse de su silla, y echó de sí su vestido, y cubrióse de saco, y se
sentó sobre ceniza. \bibverse{7} E hizo pregonar y anunciar en Nínive,
por mandado del rey y de sus grandes, diciendo: Hombres y animales,
bueyes y ovejas, no gusten cosa alguna, no se les dé alimento, ni beban
agua: \bibverse{8} Y que se cubran de saco los hombres y los animales, y
clamen á Dios fuertemente: y conviértase cada uno de su mal camino, de
la rapiña que está en sus manos. \bibverse{9} ¿Quién sabe si se volverá
y arrepentirá Dios, y se apartará del furor de su ira, y no pereceremos?
\footnote{\textbf{3:9} Jl 2,14}

\bibverse{10} Y vió Dios lo que hicieron, que se convirtieron de su mal
camino: y arrepintióse del mal que había dicho les había de hacer, y no
lo hizo. \footnote{\textbf{3:10} Jer 18,7-8}

\hypertarget{jonuxe1s-disgusto-y-reprensiuxf3n}{%
\subsection{Jonás disgusto y
reprensión}\label{jonuxe1s-disgusto-y-reprensiuxf3n}}

\hypertarget{section-3}{%
\section{4}\label{section-3}}

\bibverse{1} Pero Jonás se apesadumbró en extremo, y enojóse.
\bibverse{2} Y oró á Jehová, y dijo: Ahora, oh Jehová, ¿no es esto lo
que yo decía estando aún en mi tierra? Por eso me precaví huyendo á
Tarsis: porque sabía yo que tú eres Dios clemente y piadoso, tardo á
enojarte, y de grande misericordia, y que te arrepientes del mal.
\footnote{\textbf{4:2} Éxod 34,6} \bibverse{3} Ahora pues, oh Jehová,
ruégote que me mates; porque mejor me es la muerte que la vida.
\footnote{\textbf{4:3} 1Re 19,4}

\bibverse{4} Y Jehová le dijo: ¿Haces tú bien en enojarte tanto?
\footnote{\textbf{4:4} Jon 4,9}

\bibverse{5} Y salióse Jonás de la ciudad, y asentó hacia el oriente de
la ciudad, é hízose allí una choza, y se sentó debajo de ella á la
sombra, hasta ver qué sería de la ciudad. \bibverse{6} Y preparó Jehová
Dios una calabacera, la cual creció sobre Jonás para que hiciese sombra
sobre su cabeza, y le defendiese de su mal: y Jonás se alegró
grandemente por la calabacera. \bibverse{7} Mas Dios preparó un gusano
al venir la mañana del día siguiente, el cual hirió á la calabacera, y
secóse. \bibverse{8} Y acaeció que al salir el sol, preparó Dios un
recio viento solano; y el sol hirió á Jonás en la cabeza, y desmayábase,
y se deseaba la muerte, diciendo: Mejor sería para mí la muerte que mi
vida.

\bibverse{9} Entonces dijo Dios á Jonás: ¿Tanto te enojas por la
calabacera? Y él respondió: Mucho me enojo, hasta la muerte.

\bibverse{10} Y dijo Jehová: Tuviste tú lástima de la calabacera, en la
cual no trabajaste, ni tú la hiciste crecer; que en espacio de una noche
nació, y en espacio de otra noche pereció: \bibverse{11} ¿Y no tendré yo
piedad de Nínive, aquella grande ciudad donde hay más de ciento y veinte
mil personas que no conocen su mano derecha ni su mano izquierda, y
muchos animales?
