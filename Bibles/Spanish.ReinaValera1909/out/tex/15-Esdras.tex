\hypertarget{section}{%
\section{1}\label{section}}

\bibverse{1} Y en el primer año de Ciro rey de Persia, para que se
cumpliese la palabra de Jehová por boca de Jeremías, excitó Jehová el
espíritu de Ciro rey de Persia, el cual hizo pasar pregón por todo su
reino, y también por escrito, diciendo: \bibverse{2} Así ha dicho Ciro
rey de Persia: Jehová Dios de los cielos me ha dado todos los reinos de
la tierra, y me ha mandado que le edifique casa en Jerusalem, que está
en Judá. \bibverse{3} ¿Quién hay entre vosotros de todo su pueblo? Sea
Dios con él, y suba á Jerusalem que está en Judá, y edifique la casa á
Jehová Dios de Israel, (él es el Dios,) la cual está en Jerusalem.
\bibverse{4} Y á cualquiera que hubiere quedado de todos los lugares
donde peregrinare, los hombres de su lugar le ayuden con plata, y oro, y
hacienda, y con bestias; con dones voluntarios para la casa de Dios, la
cual está en Jerusalem.

\bibverse{5} Entonces se levantaron los cabezas de las familias de Judá
y de Benjamín, y los sacerdotes y Levitas, todos aquellos cuyo espíritu
despertó Dios para subir á edificar la casa de Jehová, la cual está en
Jerusalem. \bibverse{6} Y todos los que estaban en sus alrededores
confortaron las manos de ellos con vasos de plata y de oro, con hacienda
y bestias, y con cosas preciosas, á más de lo que se ofreció
voluntariamente. \bibverse{7} Y el rey Ciro sacó los vasos de la casa de
Jehová, que Nabucodonosor había traspasado de Jerusalem, y puesto en la
casa de sus dioses. \bibverse{8} Sacólos pues Ciro rey de Persia, por
mano de Mitrídates tesorero, el cual los dió por cuenta á Sesbassar
príncipe de Judá. \bibverse{9} Y esta es la cuenta de ellos: treinta
tazones de oro, mil tazones de plata, veinte y nueve cuchillos,
\bibverse{10} Treinta tazas de oro, cuatrocientas y diez otras tazas de
plata, y mil otros vasos. \bibverse{11} Todos los vasos de oro y de
plata, cinco mil y cuatrocientos. Todos los hizo llevar Sesbassar con
los que subieron del cautiverio de Babilonia á Jerusalem.

\hypertarget{section-1}{%
\section{2}\label{section-1}}

\bibverse{1} Y estos son los hijos de la provincia que subieron de la
cautividad, de la transmigración que Nabucodonosor rey de Babilonia hizo
traspasar á Babilonia, y que volvieron á Jerusalem y á Judá, cada uno á
su ciudad: \bibverse{2} Los cuales vinieron con Zorobabel, Jesuá,
Nehemías, Seraías, Reelaías, Mardochêo, Bilsán, Mispar, Bigvai, Rehum y
Baana. La cuenta de los varones del pueblo de Israel:

\bibverse{3} Los hijos de Paros, dos mil ciento setenta y dos;
\bibverse{4} Los hijos de Sephatías, trescientos setenta y dos;
\bibverse{5} Los hijos de Ara, setecientos setenta y cinco; \bibverse{6}
Los hijos de Pahath-moab, de los hijos de Josué y de Joab, dos mil
ochocientos y doce; \bibverse{7} Los hijos de Elam, mil doscientos
cincuenta y cuatro; \bibverse{8} Los hijos de Zattu, novecientos
cuarenta y cinco; \bibverse{9} Los hijos de Zachâi, setecientos y
sesenta; \bibverse{10} Los hijos de Bani, seiscientos cuarenta y dos;
\bibverse{11} Los hijos de Bebai, seiscientos veinte y tres;
\bibverse{12} Los hijos de Azgad, mil doscientos veinte y dos;
\bibverse{13} Los hijos de Adonicam, seiscientos sesenta y seis;
\bibverse{14} Los hijos de Bigvai, dos mil cincuenta y seis;
\bibverse{15} Los hijos de Adin, cuatrocientos cincuenta y cuatro;
\bibverse{16} Los hijos de Ater, de Ezechîas, noventa y ocho;
\bibverse{17} Los hijos de Besai, trescientos veinte y tres;
\bibverse{18} Los hijos de Jora, ciento y doce; \bibverse{19} Los hijos
de Hasum, doscientos veinte y tres; \bibverse{20} Los hijos de Gibbar,
noventa y cinco; \bibverse{21} Los hijos de Beth-lehem, ciento veinte y
tres; \bibverse{22} Los varones de Nethopha, cincuenta y seis;
\bibverse{23} Los varones de Anathoth, ciento veinte y ocho;
\bibverse{24} Los hijos de Asmaveth, cuarenta y dos; \bibverse{25} Los
hijos de Chîriath-jearim, Cephira, y Beeroth, setecientos cuarenta y
tres; \bibverse{26} Los hijos de Rama y Gabaa, seiscientos veinte y uno;
\bibverse{27} Los varones de Michmas, ciento veinte y dos; \bibverse{28}
Los varones de Beth-el y Hai, doscientos veinte y tres; \bibverse{29}
Los hijos de Nebo, cincuenta y dos; \bibverse{30} Los hijos de Magbis,
ciento cincuenta y seis; \bibverse{31} Los hijos del otro Elam, mil
doscientos cincuenta y cuatro; \bibverse{32} Los hijos de Harim,
trescientos y veinte; \bibverse{33} Los hijos de Lod, Hadid, y Ono,
setecientos veinte y cinco; \bibverse{34} Los hijos de Jericó,
trescientos cuarenta y cinco; \bibverse{35} Los hijos de Senaa, tres mil
seiscientos y treinta;

\bibverse{36} Los sacerdotes: los hijos de Jedaía, de la casa de Jesuá,
novecientos setenta y tres; \bibverse{37} Los hijos de Immer, mil
cincuenta y dos; \bibverse{38} Los hijos de Pashur, mil doscientos
cuarenta y siete; \bibverse{39} Los hijos de Harim, mil diez y siete.

\bibverse{40} Los Levitas: los hijos de Jesuá y de Cadmiel, de los hijos
de Odovías, setenta y cuatro. \bibverse{41} Los cantores: los hijos de
Asaph, ciento veinte y ocho. \bibverse{42} Los hijos de los porteros:
los hijos de Sallum, los hijos de Ater, los hijos de Talmón, los hijos
de Accub, los hijos de Hatita, los hijos de Sobai; en todos, ciento
treinta y nueve.

\bibverse{43} Los Nethineos: los hijos de Siha, los hijos de Hasupha,
los hijos de Thabaoth, \bibverse{44} Los hijos de Chêros, los hijos de
Siaa, los hijos de Phadón; \bibverse{45} Los hijos de Lebana, los hijos
de Hagaba, los hijos de Accub; \bibverse{46} Los hijos de Hagab, los
hijos de Samlai, los hijos de Hanán; \bibverse{47} Los hijos de Giddel,
los hijos de Gaher, los hijos de Reaía; \bibverse{48} Los hijos de
Resin, los hijos de Necoda, los hijos de Gazam; \bibverse{49} Los hijos
de Uzza, los hijos de Phasea, los hijos de Besai; \bibverse{50} Los
hijos de Asena, los hijos de Meunim, los hijos de Nephusim;
\bibverse{51} Los hijos de Bacbuc, los hijos de Hacusa, los hijos de
Harhur; \bibverse{52} Los hijos de Basluth, los hijos de Mehida, los
hijos de Harsa; \bibverse{53} Los hijos de Barcos, los hijos de Sisera,
los hijos de Thema; \bibverse{54} Los hijos de Nesía, los hijos de
Hatipha.

\bibverse{55} Los hijos de los siervos de Salomón: los hijos de Sotai,
los hijos de Sophereth, los hijos de Peruda; \bibverse{56} Los hijos de
Jaala, los hijos de Darcón, los hijos de Giddel; \bibverse{57} Los hijos
de Sephatías, los hijos de Hatil, los hijos de Phochêreth-hassebaim, los
hijos de Ami. \bibverse{58} Todos los Nethineos, é hijos de los siervos
de Salomón, trescientos noventa y dos.

\bibverse{59} Y estos fueron los que subieron de Tel-mela, Tel-harsa,
Chêrub, Addan, é Immer, los cuales no pudieron mostrar la casa de sus
padres, ni su linaje, si eran de Israel: \bibverse{60} Los hijos de
Delaía, los hijos de Tobías, los hijos de Necoda, seiscientos cincuenta
y dos. \bibverse{61} Y de los hijos de los sacerdotes: los hijos de
Abaía, los hijos de Cos, los hijos de Barzillai, el cual tomó mujer de
las hijas de Barzillai Galaadita, y fué llamado del nombre de ellas.
\bibverse{62} Estos buscaron su registro de genealogías, y no fué
hallado; y fueron echados del sacerdocio. \bibverse{63} Y el Tirsatha
les dijo que no comiesen de las cosas más santas, hasta que hubiese
sacerdote con Urim y Thummim.

\bibverse{64} Toda la congregación, unida como un solo hombre, era de
cuarenta y dos mil trescientos y sesenta, \bibverse{65} Sin sus siervos
y siervas, los cuales eran siete mil trescientos treinta y siete: y
tenían doscientos cantores y cantoras. \bibverse{66} Sus caballos eran
setecientos treinta y seis; sus mulos, doscientos cuarenta y cinco;
\bibverse{67} Sus camellos, cuatrocientos treinta y cinco; asnos, seis
mil setecientos y veinte.

\bibverse{68} Y algunos de los cabezas de los padres, cuando vinieron á
la casa de Jehová la cual estaba en Jerusalem, ofrecieron
voluntariamente para la casa de Dios, para levantarla en su asiento.
\bibverse{69} Según sus fuerzas dieron al tesorero de la obra sesenta y
un mil dracmas de oro, y cinco mil libras de plata, y cien túnicas
sacerdotales.

\bibverse{70} Y habitaron los sacerdotes, y los Levitas, y los del
pueblo, y los cantores, y los porteros y los Nethineos, en sus ciudades;
y todo Israel en sus ciudades.

\hypertarget{section-2}{%
\section{3}\label{section-2}}

\bibverse{1} Y llegado el mes séptimo, y ya los hijos de Israel en las
ciudades, juntóse el pueblo como un solo hombre en Jerusalem.
\bibverse{2} Entonces se levantó Jesuá hijo de Josadec, y sus hermanos
los sacerdotes, y Zorobabel hijo de Sealthiel, y sus hermanos, y
edificaron el altar del Dios de Israel, para ofrecer sobre él
holocaustos, como está escrito en la ley de Moisés varón de Dios.
\bibverse{3} Y asentaron el altar sobre sus basas, bien que tenían miedo
de los pueblos de las tierras, y ofrecieron sobre él holocaustos á
Jehová, holocaustos á la mañana y á la tarde. \bibverse{4} Hicieron
asimismo la solemnidad de las cabañas, como está escrito, y holocaustos
cada día por cuenta, conforme al rito, cada cosa en su día; \bibverse{5}
Y á más de esto, el holocausto continuo, y las nuevas lunas, y todas las
fiestas santificadas de Jehová, y todo sacrificio espontáneo, toda
ofrenda voluntaria á Jehová. \bibverse{6} Desde el primer día del mes
séptimo comenzaron á ofrecer holocaustos á Jehová; mas el templo de
Jehová no estaba aún fundado. \bibverse{7} Y dieron dinero á los
carpinteros y oficiales; asimismo comida y bebida y aceite á los
Sidonios y Tirios, para que trajesen madera de cedro del Líbano á la mar
de Joppe, conforme á la voluntad de Ciro rey de Persia acerca de esto.

\bibverse{8} Y en el año segundo de su venida á la casa de Dios en
Jerusalem, en el mes segundo, comenzaron Zorobabel hijo de Sealthiel, y
Jesuá hijo de Josadec, y los otros sus hermanos, los sacerdotes y los
Levitas, y todos los que habían venido de la cautividad á Jerusalem; y
pusieron á los Levitas de veinte años arriba para que tuviesen cargo de
la obra de la casa de Jehová. \bibverse{9} Jesuá también, sus hijos y
sus hermanos, Cadmiel y sus hijos, hijos de Judá, como un solo hombre
asistían para dar priesa á los que hacían la obra en la casa de Dios:
los hijos de Henadad, sus hijos y sus hermanos, Levitas.

\bibverse{10} Y cuando los albañiles del templo de Jehová echaban los
cimientos, pusieron á los sacerdotes vestidos de sus ropas, con
trompetas, y á Levitas hijos de Asaph con címbalos, para que alabasen á
Jehová, según ordenanza de David rey de Israel. \bibverse{11} Y
cantaban, alabando y confesando á Jehová, y decían: Porque es bueno,
porque para siempre es su misericordia sobre Israel. Y todo el pueblo
aclamaba con grande júbilo, alabando á Jehová, porque á la casa de
Jehová se echaba el cimiento.

\bibverse{12} Y muchos de los sacerdotes y de los Levitas y de los
cabezas de los padres, ancianos que habían visto la casa primera, viendo
fundar esta casa, lloraban en alta voz, mientras muchos otros daban
grandes gritos de alegría. \bibverse{13} Y no podía discernir el pueblo
el clamor de los gritos de alegría, de la voz del lloro del pueblo:
porque clamaba el pueblo con grande júbilo, y oíase el ruido hasta de
lejos.

\hypertarget{section-3}{%
\section{4}\label{section-3}}

\bibverse{1} Y oyendo los enemigos de Judá y de Benjamín, que los
venidos de la cautividad edificaban el templo de Jehová Dios de Israel,
\bibverse{2} Llegáronse á Zorobabel, y á los cabezas de los padres, y
dijéronles: Edificaremos con vosotros, porque como vosotros buscaremos á
vuestro Dios, y á él sacrificamos desde los días de Esar-haddón rey de
Asiria, que nos hizo subir aquí.

\bibverse{3} Y díjoles Zorobabel, y Jesuá, y los demás cabezas de los
padres de Israel: No nos conviene edificar con vosotros casa á nuestro
Dios, sino que nosotros solos la edificaremos á Jehová Dios de Israel,
como nos mandó el rey Ciro, rey de Persia.

\bibverse{4} Mas el pueblo de la tierra debilitaba las manos del pueblo
de Judá, y los arredraban de edificar. \bibverse{5} Cohecharon además
contra ellos consejeros para disipar su consejo, todo el tiempo de Ciro
rey de Persia, y hasta el reinado de Darío rey de Persia. \bibverse{6} Y
en el reinado de Assuero, en el principio de su reinado, escribieron
acusaciones contra los moradores de Judá y de Jerusalem.

\bibverse{7} Y en días de Artajerjes, Bislam, Mitrídates, Tabeel, y los
demás sus compañeros, escribieron á Artajerjes rey de Persia; y la
escritura de la carta estaba hecha en siriaco, y declarada en siriaco.
\bibverse{8} Rehum canciller, y Simsai secretario, escribieron una carta
contra Jerusalem al rey Artajerjes, como se sigue. \bibverse{9} Entonces
Rehum canciller, y Simsai secretario, y los demás sus compañeros, los
Dineos, y los Apharsathachêos, Thepharleos, Apharseos, los Erchûeos, los
Babilonios, Susanchêos, Dieveos, y Elamitas; \bibverse{10} Y los demás
pueblos que el grande y glorioso Asnappar trasportó, é hizo habitar en
las ciudades de Samaria, y los demás de la otra parte del río, etcétera,
escribieron.

\bibverse{11} Este es el traslado de la carta que enviaron: Al rey
Artajerjes: Tus siervos de la otra parte del río, etcétera.

\bibverse{12} Sea notorio al rey, que los Judíos que subieron de ti á
nosotros, vinieron á Jerusalem; y edifican la ciudad rebelde y mala, y
han erigido los muros; y compuesto los fundamentos. \bibverse{13} Ahora,
notorio sea al rey, que si aquella ciudad fuere reedificada, y los muros
fueren establecidos, el tributo, pecho, y rentas no darán, y el catastro
de lo reyes será menoscabado. \bibverse{14} Ya pues que estamos
mantenidos de palacio, no nos es justo ver el menosprecio del rey: hemos
enviado por tanto, y hécholo saber al rey, \bibverse{15} Para que busque
en el libro de las historias de nuestros padres; y hallarás en el libro
de las historias, y sabrás que esta ciudad es ciudad rebelde, y
perjudicial á los reyes y á las provincias, y que de tiempo antiguo
forman en medio de ella rebeliones; por lo que esta ciudad fué
destruída. \bibverse{16} Hacemos saber al rey, que si esta ciudad fuere
edificada, y erigidos sus muros, la parte allá del río no será tuya.
\bibverse{17} El rey envió esta respuesta á Rehum canciller, y á Simsai
secretario, y á los demás sus compañeros que habitan en Samaria, y á los
demás de la parte allá del río: Paz, etc. \bibverse{18} La carta que nos
enviasteis claramente fué leída delante de mí. \bibverse{19} Y por mí
fué dado mandamiento, y buscaron, y hallaron que aquella ciudad de
tiempo antiguo se levanta contra los reyes, y se rebela, y se forma en
ella sedición: \bibverse{20} Y que reyes fuertes hubo en Jerusalem,
quienes señorearon en todo lo que está á la parte allá del río; y que
tributo, y pecho, y rentas se les daba. \bibverse{21} Ahora pues dad
orden que cesen aquellos hombres, y no sea esa ciudad edificada, hasta
que por mí sea dado mandamiento. \bibverse{22} Y mirad bien que no
hagáis error en esto: ¿por qué habrá de crecer el daño para perjuicio de
los reyes? \bibverse{23} Entonces, cuando el traslado de la carta del
rey Artajerjes fué leído delante de Rehum, y de Simsai secretario, y sus
compañeros, fueron prestamente á Jerusalem á los Judíos, é hiciéronles
cesar con poder y fuerza. \bibverse{24} Cesó entonces la obra de la casa
de Dios, la cual estaba en Jerusalem: y cesó hasta el año segundo del
reinado de Darío rey de Persia.

\hypertarget{section-4}{%
\section{5}\label{section-4}}

\bibverse{1} Y profetizaron Haggeo profeta, y Zacarías hijo de Iddo,
profetas, á los Judíos que estaban en Judá y en Jerusalem yendo en
nombre del Dios de Israel á ellos. \bibverse{2} Entonces se levantaron
Zorobabel hijo de Sealthiel, y Jesuá hijo de Josadec; y comenzaron á
edificar la casa de Dios que estaba en Jerusalem; y con ellos los
profetas de Dios que les ayudaban.

\bibverse{3} En aquel tiempo vino á ellos Tatnai, capitán de la parte
allá del río, y Sethar-boznai y sus compañeros, y dijéronles así: ¿Quién
os dió mandamiento para edificar esta casa, y restablecer estos muros?
\bibverse{4} Entonces les dijimos en orden á esto cuáles eran los
nombres de los varones que edificaban este edificio. \bibverse{5} Mas
los ojos de su Dios fueron sobre los ancianos de los Judíos, y no les
hicieron cesar hasta que el negocio viniese á Darío: y entonces
respondieron por carta sobre esto.

\bibverse{6} Traslado de la carta que Tatnai, capitán de la parte allá
del río, y Sethar-boznai, y sus compañeros los Apharsachêos, que estaban
á la parte allá del río, enviaron al rey Darío. \bibverse{7} Enviáronle
carta, y de esta manera estaba escrito en ella. Al rey Darío toda paz.

\bibverse{8} Sea notorio al rey, que fuimos á la provincia de Judea, á
la casa del gran Dios, la cual se edifica de piedra de mármol; y los
maderos son puestos en las paredes, y la obra se hace apriesa, y
prospera en sus manos. \bibverse{9} Entonces preguntamos á los ancianos,
diciéndoles así: ¿Quién os dió mandamiento para edificar esta casa, y
para restablecer estos muros? \bibverse{10} Y también les preguntamos
sus nombres para hacértelo saber, para escribirte los nombres de los
varones que estaban por cabezas de ellos. \bibverse{11} Y
respondiéronnos, diciendo así: Nosotros somos siervos del Dios del cielo
y de la tierra, y reedificamos la casa que ya muchos años antes había
sido edificada, la cual edificó y fundó el gran rey de Israel.
\bibverse{12} Mas después que nuestros padres ensañaron al Dios de los
cielos, él los entregó en mano de Nabucodonosor rey de Babilonia,
Caldeo, el cual destruyó esta casa, é hizo trasportar el pueblo á
Babilonia. \bibverse{13} Empero el primer año de Ciro rey de Babilonia,
el mismo rey Ciro dió mandamiento para que esta casa de Dios fuese
edificada. \bibverse{14} Y también los vasos de oro y de plata de la
casa de Dios, que Nabucodonosor había sacado del templo que estaba en
Jerusalem, y los había metido en el templo de Babilonia, el rey Ciro los
sacó del templo de Babilonia, y fueron entregados á Sesbassar, al cual
había puesto por gobernador; \bibverse{15} Y le dijo: Toma estos vasos,
ve y ponlos en el templo que está en Jerusalem; y la casa de Dios sea
edificada en su lugar. \bibverse{16} Entonces este Sesbassar vino, y
puso los fundamentos de la casa de Dios que estaba en Jerusalem, y desde
entonces hasta ahora se edifica, y aun no está acabada.

\bibverse{17} Y ahora, si al rey parece bien, búsquese en la casa de los
tesoros del rey que está allí en Babilonia, si es así que por el rey
Ciro había sido dado mandamiento para edificar esta casa de Dios en
Jerusalem, y envíenos á decir la voluntad del rey sobre esto.

\hypertarget{section-5}{%
\section{6}\label{section-5}}

\bibverse{1} Entonces el rey Darío dió mandamiento, y buscaron en la
casa de los libros, donde guardaban los tesoros allí en Babilonia.
\bibverse{2} Y fué hallado en Achmetta, en el palacio que está en la
provincia de Media, un libro, dentro del cual estaba escrito así:
Memoria: \bibverse{3} En el año primero del rey Ciro, el mismo rey Ciro
dió mandamiento acerca de la casa de Dios que estaba en Jerusalem, que
fuese la casa edificada para lugar en que sacrifiquen sacrificios, y que
sus paredes fuesen cubiertas; su altura de sesenta codos, y de sesenta
codos su anchura; \bibverse{4} Los órdenes, tres de piedra de mármol, y
un orden de madera nueva: y que el gasto sea dado de la casa del rey.
\bibverse{5} Y también los vasos de oro y de plata de la casa de Dios,
que Nabucodonosor sacó del templo que estaba en Jerusalem y los pasó á
Babilonia, sean devueltos y vayan al templo que está en Jerusalem, á su
lugar, y sean puestos en la casa de Dios.

\bibverse{6} Ahora pues, Tatnai, jefe del lado allá del río,
Sethar-boznai, y sus compañeros los Apharsachêos que estáis á la otra
parte del río, apartaos de ahí. \bibverse{7} Dejad la obra de la casa de
este Dios al principal de los Judíos, y á sus ancianos, para que
edifiquen la casa de este Dios en su lugar. \bibverse{8} Y por mí es
dado mandamiento de lo que habéis de hacer con los ancianos de estos
Judíos, para edificar la casa de este Dios: que de la hacienda del rey,
que tiene del tributo de la parte allá del río, los gastos sean dados
luego á aquellos varones, para que no cesen. \bibverse{9} Y lo que fuere
necesario, becerros y carneros y corderos, para holocaustos al Dios del
cielo, trigo, sal, vino y aceite, conforme á lo que dijeren los
sacerdotes que están en Jerusalem, déseles cada un día sin obstáculo
alguno; \bibverse{10} Para que ofrezcan olores de holganza al Dios del
cielo, y oren por la vida del rey y por sus hijos. \bibverse{11} También
es dado por mí mandamiento, que cualquiera que mudare este decreto, sea
derribado un madero de su casa, y enhiesto, sea colgado en él: y su casa
sea hecha muladar por esto. \bibverse{12} Y el Dios que hizo habitar
allí su nombre, destruya todo rey y pueblo que pusiere su mano para
mudar ó destruir esta casa de Dios, la cual está en Jerusalem. Yo Darío
puse el decreto: sea hecho prestamente. \bibverse{13} Entonces Tatnai,
gobernador del otro lado del río, y Sethar-boznai, y sus compañeros,
hicieron prestamente según el rey Darío había enviado.

\bibverse{14} Y los ancianos de los Judíos edificaban y prosperaban,
conforme á la profecía de Haggeo profeta, y de Zacarías hijo de Iddo.
Edificaron pues, y acabaron, por el mandamiento del Dios de Israel, y
por el mandamiento de Ciro, y de Darío, y de Artajerjes rey de Persia.
\bibverse{15} Y esta casa fué acabada al tercer día del mes de Adar, que
era el sexto año del reinado del rey Darío.

\bibverse{16} Y los hijos de Israel, los sacerdotes y los Levitas, y los
demás que habían venido de la trasportación, hicieron la dedicación de
esta casa de Dios con gozo. \bibverse{17} Y ofrecieron en la dedicación
de esta casa de Dios cien becerros, doscientos carneros, cuatrocientos
corderos; y machos de cabrío en expiación por todo Israel, doce,
conforme al número de las tribus de Israel. \bibverse{18} Y pusieron á
los sacerdotes en sus clases, y á los Levitas en sus divisiones, sobre
la obra de Dios que está en Jerusalem, conforme á lo escrito en el libro
de Moisés.

\bibverse{19} Y los de la transmigración hicieron la pascua á los
catorce del mes primero. \bibverse{20} Porque los sacerdotes y los
Levitas se habían purificado á una; todos fueron limpios: y sacrificaron
la pascua por todos los de la transmigración, y por sus hermanos los
sacerdotes, y por sí mismos. \bibverse{21} Y comieron los hijos de
Israel que habían vuelto de la transmigración, y todos los que se habían
apartado á ellos de la inmundicia de las gentes de la tierra, para
buscar á Jehová Dios de Israel. \bibverse{22} Y celebraron la solemnidad
de los panes ázimos siete días con regocijo, por cuanto Jehová los había
alegrado, y convertido el corazón del rey de Asiria á ellos, para
esforzar sus manos en la obra de la casa de Dios, del Dios de Israel.

\hypertarget{section-6}{%
\section{7}\label{section-6}}

\bibverse{1} Pasadas estas cosas, en el reinado de Artajerjes rey de
Persia, Esdras, hijo de Seraías, hijo de Azarías, hijo de Hilcías,
\bibverse{2} Hijo de Sallum, hijo de Sadoc, hijo de Achîtob,
\bibverse{3} Hijo de Amarías, hijo de Azarías, hijo de Meraioth,
\bibverse{4} Hijo de Zeraías, hijo de Uzzi, hijo de Bucci, \bibverse{5}
Hijo de Abisue, hijo de Phinees, hijo de Eleazar, hijo de Aarón, primer
sacerdote: \bibverse{6} Este Esdras subió de Babilonia, el cual era
escriba diligente en la ley de Moisés, que Jehová Dios de Israel había
dado; y concedióle el rey, según la mano de Jehová su Dios sobre él,
todo lo que pidió. \bibverse{7} Y subieron con él á Jerusalem de los
hijos de Israel, y de los sacerdotes, y Levitas, y cantores, y porteros,
y Nethineos, en el séptimo año del rey Artajerjes. \bibverse{8} Y llegó
á Jerusalem en el mes quinto, el año séptimo del rey. \bibverse{9}
Porque el día primero del primer mes fué el principio de la partida de
Babilonia, y al primero del mes quinto llegó á Jerusalem, según la buena
mano de su Dios sobre él. \bibverse{10} Porque Esdras había preparado su
corazón para inquirir la ley de Jehová, y para hacer y enseñar á Israel
mandamientos y juicios.

\bibverse{11} Y este es el traslado de la carta que dió el rey
Artajerjes á Esdras, sacerdote escriba, escriba de las palabras mandadas
de Jehová, y de sus estatutos á Israel: \bibverse{12} Artajerjes, rey de
los reyes, á Esdras sacerdote, escriba perfecto de la ley del Dios del
cielo: Salud, etc.

\bibverse{13} Por mí es dado mandamiento, que cualquiera que quisiere en
mi reino, del pueblo de Israel y de sus sacerdotes y Levitas, ir contigo
á Jerusalem, vaya. \bibverse{14} Porque de parte del rey y de sus siete
consultores eres enviado á visitar á Judea y á Jerusalem, conforme á la
ley de tu Dios que está en tu mano; \bibverse{15} Y á llevar la plata y
el oro que el rey y sus consultores voluntariamente ofrecen al Dios de
Israel, cuya morada está en Jerusalem; \bibverse{16} Y toda la plata y
el oro que hallares en toda la provincia de Babilonia, con las ofrendas
voluntarias del pueblo y de los sacerdotes, que de su voluntad
ofrecieren para la casa de su Dios que está en Jerusalem. \bibverse{17}
Comprarás pues prestamente con esta plata becerros, carneros, corderos,
con sus presentes y sus libaciones, y los ofrecerás sobre el altar de la
casa de vuestro Dios que está en Jerusalem. \bibverse{18} Y lo que á ti
y á tus hermanos pluguiere hacer de la otra plata y oro, hacedlo
conforme á la voluntad de vuestro Dios. \bibverse{19} Y los vasos que te
son entregados para el servicio de la casa de tu Dios, los restituirás
delante de Dios en Jerusalem. \bibverse{20} Y lo demás necesario para la
casa de tu Dios que te fuere menester dar, daráslo de la casa de los
tesoros del rey.

\bibverse{21} Y por mí el rey Artajerjes es dado mandamiento á todos los
tesoreros que están al otro lado del río, que todo lo que os demandare
Esdras sacerdote, escriba de la ley del Dios del cielo, concédasele
luego, \bibverse{22} Hasta cien talentos de plata, y hasta cien coros de
trigo, y hasta cien batos de vino, y hasta cien batos de aceite; y sal
sin tasa. \bibverse{23} Todo lo que es mandado por el Dios del cielo,
sea hecho prestamente para la casa del Dios del cielo: pues, ¿por qué
habría de ser su ira contra el reino del rey y de sus hijos?

\bibverse{24} Y á vosotros os hacemos saber, que á todos los sacerdotes
y Levitas, cantores, porteros, Nethineos y ministros de la casa de Dios,
ninguno pueda imponerles tributo, ó pecho, ó renta.

\bibverse{25} Y tú, Esdras, conforme á la sabiduría de tu Dios que
tienes, pon jueces y gobernadores, que gobiernen á todo el pueblo que
está del otro lado del río, á todos los que tienen noticia de las leyes
de tu Dios; y al que no la tuviere, le enseñaréis. \bibverse{26} Y
cualquiera que no hiciere la ley de tu Dios, y la ley del rey,
prestamente sea juzgado, ó á muerte, ó á desarraigo, ó á pena de la
hacienda, ó á prisión. \bibverse{27} Bendito Jehová, Dios de nuestros
padres, que puso tal cosa en el corazón del rey, para honrar la casa de
Jehová que está en Jerusalem. \bibverse{28} E inclinó hacia mí su
misericordia delante del rey y de sus consultores, y de todos los
príncipes poderosos del rey. Y yo, confortado según la mano de mi Dios
sobre mí, junté los principales de Israel para que subiesen conmigo.

\hypertarget{section-7}{%
\section{8}\label{section-7}}

\bibverse{1} Y estos son los cabezas de sus familias, y genealogía de
aquellos que subieron conmigo de Babilonia, reinando el rey Artajerjes:
\bibverse{2} De los hijos de Phinees, Gersón; de los hijos de Ithamar,
Daniel; de los hijos de David, Hattus; \bibverse{3} De los hijos de
Sechânías y de los hijos de Pharos, Zacarías, y con él, en la línea de
varones, ciento y cincuenta; \bibverse{4} De los hijos de Pahath-moab,
Elioenai, hijo de Zarahi, y con él doscientos varones; \bibverse{5} De
los hijos de Sechânías, el hijo de Jahaziel, y con él trescientos
varones; \bibverse{6} De los hijos de Adín, Ebed, hijo de Jonathán, y
con él cincuenta varones; \bibverse{7} De los hijos de Elam, Isaía, hijo
de Athalías, y con él setenta varones; \bibverse{8} Y de los hijos de
Sephatías, Zebadías, hijo de Michâel, y con él ochenta varones;
\bibverse{9} De los hijos de Joab, Obadías, hijo de Jehiel, y con él
doscientos diez y ocho varones; \bibverse{10} Y de los hijos de
Solomith, el hijo de Josiphías, y con él ciento y sesenta varones;
\bibverse{11} Y de los hijos de Bebai, Zacarías, hijo de Bebai, y con él
veintiocho varones; \bibverse{12} Y de los hijos de Azgad, Johanán, hijo
de Catán, y con él ciento y diez varones; \bibverse{13} Y de los hijos
de Adonicam, los postreros, cuyos nombres son estos, Eliphelet, Jeiel, y
Semaías, y con ellos sesenta varones; \bibverse{14} Y de los hijos de
Bigvai, Utai y Zabud, y con ellos sesenta varones.

\bibverse{15} Y juntélos junto al río que viene á Ahava, y reposamos
allí tres días: y habiendo buscado entre el pueblo y entre los
sacerdotes, no hallé allí de los hijos de Leví. \bibverse{16} Entonces
despaché á Eliezer, y á Ariel, y á Semaías, y á Elnathán, y á Jarib, y á
Elnathán, y á Nathán, y á Zacarías, y á Mesullam, principales; asimismo
á Joiarib y á Elnathán, hombres doctos; \bibverse{17} Y enviélos á Iddo,
jefe en el lugar de Casipia, y puse en boca de ellos las palabras que
habían de hablar á Iddo, y á sus hermanos los Nethineos en el lugar de
Casipia, para que nos trajesen ministros para la casa de nuestro Dios.
\bibverse{18} Y trajéronnos, según la buena mano de nuestro Dios sobre
nosotros, un varón entendido de los hijos de Mahalí, hijo de Leví, hijo
de Israel; y á Serabías con sus hijos y sus hermanos, dieciocho;
\bibverse{19} Y á Hasabías, y con él á Isaía de los hijos de Merari, á
sus hermanos y á sus hijos, veinte; \bibverse{20} Y de los Nethineos, á
quienes David con los príncipes puso para el ministerio de los Levitas,
doscientos y veinte Nethineos: todos los cuales fueron declarados por
sus nombres.

\bibverse{21} Y publiqué ayuno allí junto al río de Ahava, para
afligirnos delante de nuestro Dios, para solicitar de él camino derecho
para nosotros, y para nuestros niños, y para toda nuestra hacienda.
\bibverse{22} Porque tuve vergüenza de pedir al rey tropa y gente de á
caballo que nos defendiesen del enemigo en el camino: porque habíamos
hablado al rey, diciendo: La mano de nuestro Dios es para bien sobre
todos los que le buscan; mas su fortaleza y su furor sobre todos los que
le dejan. \bibverse{23} Ayunamos pues, y pedimos á nuestro Dios sobre
esto, y él nos fué propicio.

\bibverse{24} Aparté luego doce de los principales de los sacerdotes, á
Serebías y á Hasabías, y con ellos diez de sus hermanos; \bibverse{25} Y
peséles la plata, y el oro, y los vasos, la ofrenda que para la casa de
nuestro Dios habían ofrecido el rey, y sus consultores, y sus príncipes,
y todos los que se hallaron en Israel. \bibverse{26} Pesé pues en manos
de ellos seiscientos y cincuenta talentos de plata, y vasos de plata por
cien talentos, y cien talentos de oro; \bibverse{27} Además veinte
tazones de oro, de mil dracmas; y dos vasos de metal limpio muy bueno,
preciados como el oro. \bibverse{28} Y díjeles: Vosotros sois
consagrados á Jehová, y santos los vasos; y la plata y el oro ofrenda
voluntaria á Jehová, Dios de nuestros padres. \bibverse{29} Velad, y
guardadlos, hasta que los peséis delante de los príncipes de los
sacerdotes y Levitas, y de los jefes de los padres de Israel en
Jerusalem, en las cámaras de la casa de Jehová.

\bibverse{30} Los sacerdotes pues y Levitas recibieron el peso de la
plata y del oro y de los vasos, para traerlo á Jerusalem á la casa de
nuestro Dios.

\bibverse{31} Y partimos del río de Ahava el doce del mes primero, para
ir á Jerusalem: y la mano de nuestro Dios fué sobre nosotros, el cual
nos libró de mano de enemigo y de asechador en el camino. \bibverse{32}
Y llegamos á Jerusalem, y reposamos allí tres días. \bibverse{33} Al
cuarto día fué luego pesada la plata, y el oro, y los vasos, en la casa
de nuestro Dios, por mano de Meremoth hijo de Urías sacerdote, y con él
Eleazar hijo de Phinees; y con ellos Jozabad hijo de Jesuá, y Noadías
hijo de Binnui, Levitas; \bibverse{34} Por cuenta y por peso todo: y se
apuntó todo aquel peso en aquel tiempo.

\bibverse{35} Los que habían venido de la cautividad, los hijos de la
transmigración, ofrecieron holocaustos al Dios de Israel, doce becerros
por todo Israel, noventa y seis carneros, setenta y siete corderos, doce
machos cabríos por expiación: todo en holocausto á Jehová. \bibverse{36}
Y dieron los despachos del rey á sus gobernadores y capitanes del otro
lado del río, los cuales favorecieron al pueblo y á la casa de Dios.

\hypertarget{section-8}{%
\section{9}\label{section-8}}

\bibverse{1} Y acabadas estas cosas, los príncipes se llegaron á mí,
diciendo: El pueblo de Israel, y los sacerdotes y Levitas, no se han
apartado de los pueblos de las tierras, de los Cananeos, Hetheos,
Pherezeos, Jebuseos, Ammonitas, y Moabitas, Egipcios, y Amorrheos,
haciendo conforme á sus abominaciones. \bibverse{2} Porque han tomado de
sus hijas para sí y para sus hijos, y la simiente santa ha sido mezclada
con los pueblos de las tierras; y la mano de los príncipes y de los
gobernadores ha sido la primera en esta prevaricación.

\bibverse{3} Lo cual oyendo yo, rasgué mi vestido y mi manto, y arranqué
de los cabellos de mi cabeza y de mi barba, y sentéme atónito.
\bibverse{4} Y juntáronse á mí todos los temerosos de las palabras del
Dios de Israel, á causa de la prevaricación de los de la transmigración;
mas yo estuve sentado atónito hasta el sacrificio de la tarde.

\bibverse{5} Y al sacrificio de la tarde levantéme de mi aflicción; y
habiendo rasgado mi vestido y mi manto, postréme de rodillas, y extendí
mis palmas á Jehová mi Dios; \bibverse{6} Y dije: Dios mío, confuso y
avergonzado estoy para levantar, oh Dios mío, mi rostro á ti: porque
nuestras iniquidades se han multiplicado sobre nuestra cabeza, y
nuestros delitos han crecido hasta el cielo. \bibverse{7} Desde los días
de nuestros padres hasta este día estamos en grande culpa; y por
nuestras iniquidades nosotros, nuestros reyes, y nuestros sacerdotes,
hemos sido entregados en manos de los reyes de las tierras, á cuchillo,
á cautiverio, y á robo, y á confusión de rostro, como hoy día.
\bibverse{8} Y ahora como por un breve momento fué la misericordia de
Jehová nuestro Dios, para hacer que nos quedase un resto libre, y para
darnos estaca en el lugar de su santuario, á fin de alumbrar nuestros
ojos nuestro Dios, y darnos una poca de vida en nuestra servidumbre.
\bibverse{9} Porque siervos éramos: mas en nuestra servidumbre no nos
desamparó nuestro Dios, antes inclinó sobre nosotros misericordia
delante de los reyes de Persia, para que se nos diese vida para alzar la
casa de nuestro Dios, y para hacer restaurar sus asolamientos, y para
darnos vallado en Judá y en Jerusalem.

\bibverse{10} Mas ahora, ¿qué diremos, oh Dios nuestro, después de esto?
porque nosotros hemos dejado tus mandamientos, \bibverse{11} Los cuales
prescribiste por mano de tus siervos los profetas, diciendo: La tierra á
la cual entráis para poseerla, tierra inmunda es á causa de la
inmundicia de los pueblos de aquellas regiones, por las abominaciones de
que la han henchido de uno á otro extremo con su inmundicia.
\bibverse{12} Ahora pues, no daréis vuestras hijas á los hijos de ellos,
ni sus hijas tomaréis para vuestros hijos, ni procuraréis su paz ni su
bien para siempre; para que seáis corroborados, y comáis el bien de la
tierra, y la dejéis por heredad á vuestros hijos para siempre.

\bibverse{13} Mas después de todo lo que nos ha sobrevenido á causa de
nuestras malas obras, y á causa de nuestro grande delito; ya que tú,
Dios nuestro, estorbaste que fuésemos oprimidos bajo de nuestras
iniquidades, y nos diste este tal efugio; \bibverse{14} ¿Hemos de volver
á infringir tus mandamientos, y á emparentar con los pueblos de estas
abominaciones? ¿No te ensañarías contra nosotros hasta consumirnos, sin
que quedara resto ni escapatoria? \bibverse{15} Jehová, Dios de Israel,
tú eres justo: pues que hemos quedado algunos salvos, como este día,
henos aquí delante de ti en nuestros delitos; porque no es posible
subsistir en tu presencia á causa de esto.

\hypertarget{section-9}{%
\section{10}\label{section-9}}

\bibverse{1} Y orando Esdras y confesando, llorando y postrándose
delante de la casa de Dios, juntóse á él una muy grande multitud de
Israel, hombres y mujeres y niños; y lloraba el pueblo con gran llanto.
\bibverse{2} Entonces respondió Sechânías hijo de Jehiel, de los hijos
de Elam, y dijo á Esdras: Nosotros hemos prevaricado contra nuestro
Dios, pues tomamos mujeres extranjeras de los pueblos de la tierra: mas
hay aún esperanza para Israel sobre esto. \bibverse{3} Ahora pues
hagamos pacto con nuestro Dios, que echaremos todas las mujeres y los
nacidos de ellas, según el consejo del Señor, y de los que temen el
mandamiento de nuestro Dios: y hágase conforme á la ley. \bibverse{4}
Levántate, porque á ti toca el negocio, y nosotros seremos contigo;
esfuérzate, y ponlo por obra.

\bibverse{5} Entonces se levantó Esdras, y juramentó á los príncipes de
los sacerdotes y de los Levitas, y á todo Israel, que harían conforme á
esto; y ellos juraron. \bibverse{6} Levantóse luego Esdras de delante la
casa de Dios, y fuése á la cámara de Johanán hijo de Eliasib: é ido
allá, no comió pan ni bebió agua, porque se entristeció sobre la
prevaricación de los de la transmigración. \bibverse{7} E hicieron pasar
pregón por Judá y por Jerusalem á todos los hijos de la transmigración,
que se juntasen en Jerusalem: \bibverse{8} Y que el que no viniera
dentro de tres días, conforme al acuerdo de los príncipes y de los
ancianos, perdiese toda su hacienda, y él fuese apartado de la compañía
de los de la transmigración.

\bibverse{9} Así todos los hombres de Judá y de Benjamín se reunieron en
Jerusalem dentro de tres días, á los veinte del mes, el cual era el mes
noveno; y sentóse todo el pueblo en la plaza de la casa de Dios,
temblando con motivo de aquel negocio, y á causa de las lluvias.

\bibverse{10} Y levantóse Esdras el sacerdote, y díjoles: Vosotros
habéis prevaricado, por cuanto tomasteis mujeres extrañas, añadiendo así
sobre el pecado de Israel. \bibverse{11} Ahora pues, dad gloria á Jehová
Dios de vuestros padres, y haced su voluntad, y apartaos de los pueblos
de las tierras, y de las mujeres extranjeras.

\bibverse{12} Y respondió todo aquel concurso, y dijeron en alta voz:
Así se haga conforme á tu palabra. \bibverse{13} Mas el pueblo es mucho,
y el tiempo lluvioso, y no hay fuerza para estar en la calle: ni la obra
es de un día ni de dos, porque somos muchos los que hemos prevaricado en
este negocio. \bibverse{14} Estén ahora nuestros príncipes, los de toda
la congregación; y todos aquellos que en nuestras ciudades hubieren
tomado mujeres extranjeras, vengan á tiempos aplazados, y con ellos los
ancianos de cada ciudad, y los jueces de ellas, hasta que apartemos de
nosotros el furor de la ira de nuestro Dios sobre esto.

\bibverse{15} Fueron pues puestos sobre este negocio Jonathán hijo de
Asael, y Jaazías hijo de Tikvah; y Mesullam y Sabethai, Levitas, les
ayudaron.

\bibverse{16} E hicieron así los hijos de la transmigración. Y apartados
que fueron luego Esdras sacerdote, y los varones cabezas de familias en
la casa de sus padres, todos ellos por sus nombres, sentáronse el primer
día del mes décimo para inquirir el negocio. \bibverse{17} Y
concluyeron, con todos aquellos que habían tomado mujeres extranjeras,
al primer día del mes primero.

\bibverse{18} Y de los hijos de los sacerdotes que habían tomado mujeres
extranjeras, fueron hallados estos: De los hijos de Jesuá hijo de
Josadec, y de sus hermanos: Maasías, y Eliezer, y Jarib, y Gedalías;
\bibverse{19} Y dieron su mano en promesa de echar sus mujeres, y
ofrecieron como culpados un carnero de los rebaños por su delito.
\bibverse{20} Y de los hijos de Immer: Hanani y Zebadías. \bibverse{21}
Y de los hijos de Harím: Maasías, y Elías, y Semeías, y Jehiel, y
Uzzías. \bibverse{22} Y de los hijos de Phasur: Elioenai, Maasías,
Ismael, Nathanael, Jozabad, y Elasa. \bibverse{23} Y de los hijos de los
Levitas: Jozabad, y Simi, Kelaía (este es Kelita), Pethaía, Judá y
Eliezer. \bibverse{24} Y de los cantores, Eliasib; y de los porteros:
Sellum, y Telem, y Uri. \bibverse{25} Asimismo de Israel: De los hijos
de Pharos: Ramía é Izzías, y Malchías, y Miamim, y Eleazar, y Malchías,
y Benaías. \bibverse{26} Y de los hijos de Elam: Mathanías, Zachârías, y
Jehiel, y Abdi, y Jeremoth, y Elía. \bibverse{27} Y de los hijos de
Zattu: Elioenai, Eliasib, Mathanías, y Jeremoth, y Zabad, y Aziza.
\bibverse{28} Y de los hijos de Bebai: Johanán, Hananías, Zabbai, Atlai.
\bibverse{29} Y de los hijos de Bani: Mesullam, Malluch, y Adaías,
Jasub, y Seal, y Ramoth. \bibverse{30} Y de los hijos de Pahath-moab:
Adna, y Chêleal, Benaías, Maasías, Mathanías, Besaleel, Binnui y
Manasés. \bibverse{31} Y de los hijos de Harim: Eliezer, Issia,
Malchîas, Semeía, Simeón, \bibverse{32} Benjamín, Malluch, Semarías.
\bibverse{33} De los hijos de Hasum: Mathenai, Mathatha, Zabad,
Eliphelet, Jeremai, Manasés, Sami. \bibverse{34} De los hijos de Bani:
Maadi, Amram y Uel, \bibverse{35} Benaías, Bedías, Chêluhi,
\bibverse{36} Vanías, Meremoth, Eliasib, \bibverse{37} Mathanías,
Mathenai, y Jaasai, \bibverse{38} Y Bani, y Binnui, Simi, \bibverse{39}
Y Selemías y Nathán y Adaías, \bibverse{40} Machnadbai, Sasai, Sarai,
\bibverse{41} Azareel, y Selamías, Semarías, \bibverse{42} Sallum,
Amarías, Joseph. \bibverse{43} Y de los hijos de Nebo: Jehiel,
Matithías, Zabad, Zebina, Jadau, y Joel, Benaías.

\bibverse{44} Todos estos habían tomado mujeres extranjeras; y había
mujeres de ellos que habían parido hijos.
