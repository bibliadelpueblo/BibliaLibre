\hypertarget{normas-relativas-a-los-holocaustos}{%
\subsection{Normas relativas a los
holocaustos}\label{normas-relativas-a-los-holocaustos}}

\hypertarget{section}{%
\section{1}\label{section}}

\bibverse{1} Y llamó Jehová á Moisés, y habló con él desde el
tabernáculo del testimonio, diciendo: \bibverse{2} Habla á los hijos de
Israel, y diles: Cuando alguno de entre vosotros ofreciere ofrenda á
Jehová, de ganado vacuno ú ovejuno haréis vuestra ofrenda.

\hypertarget{holocaustos-de-ganado}{%
\subsection{Holocaustos de ganado}\label{holocaustos-de-ganado}}

\bibverse{3} Si su ofrenda fuere holocausto de vacas, macho sin tacha lo
ofrecerá: de su voluntad lo ofrecerá á la puerta del tabernáculo del
testimonio delante de Jehová. \bibverse{4} Y pondrá su mano sobre la
cabeza del holocausto; y él lo aceptará para expiarle. \footnote{\textbf{1:4}
  Éxod 29,10} \bibverse{5} Entonces degollará el becerro en la presencia
de Jehová; y los sacerdotes, hijos de Aarón, ofrecerán la sangre, y la
rociarán alrededor sobre el altar, el cual está á la puerta del
tabernáculo del testimonio. \bibverse{6} Y desollará el holocausto, y lo
dividirá en sus piezas. \bibverse{7} Y los hijos de Aarón sacerdote
pondrán fuego sobre el altar, y compondrán la leña sobre el fuego.
\bibverse{8} Luego los sacerdotes, hijos de Aarón, acomodarán las
piezas, la cabeza y el redaño, sobre la leña que está sobre el fuego,
que habrá encima del altar: \bibverse{9} Y lavará con agua sus
intestinos y sus piernas: y el sacerdote hará arder todo sobre el altar:
holocausto es, ofrenda encendida de olor suave á Jehová.

\hypertarget{holocaustos-de-rebauxf1os}{%
\subsection{Holocaustos de rebaños}\label{holocaustos-de-rebauxf1os}}

\bibverse{10} Y si su ofrenda para holocausto fuere de ovejas, de los
corderos, ó de las cabras, macho sin defecto lo ofrecerá. \bibverse{11}
Y ha de degollarlo al lado septentrional del altar delante de Jehová: y
los sacerdotes, hijos de Aarón, rociarán su sangre sobre el altar
alrededor. \bibverse{12} Y lo dividirá en sus piezas, con su cabeza y su
redaño; y el sacerdote las acomodará sobre la leña que está sobre el
fuego, que habrá encima del altar; \bibverse{13} Y lavará sus entrañas y
sus piernas con agua; y el sacerdote lo ofrecerá todo, y harálo arder
sobre el altar; holocausto es, ofrenda encendida de olor suave á Jehová.

\hypertarget{holocaustos-de-puxe1jaros}{%
\subsection{Holocaustos de pájaros}\label{holocaustos-de-puxe1jaros}}

\bibverse{14} Y si el holocausto se hubiere de ofrecer á Jehová de aves,
presentará su ofrenda de tórtolas, ó de palominos. \bibverse{15} Y el
sacerdote la ofrecerá sobre el altar, y ha de quitarle la cabeza, y hará
que arda en el altar; y su sangre será exprimida sobre la pared del
altar. \bibverse{16} Y le ha de quitar el buche y las plumas, lo cual
echará junto al altar, hacia el oriente, en el lugar de las cenizas.
\bibverse{17} Y la henderá por sus alas, mas no la dividirá en dos: y el
sacerdote la hará arder sobre el altar, sobre la leña que estará en el
fuego; holocausto es, ofrenda encendida de olor suave á Jehová.

\hypertarget{normas-relativas-a-la-ofrenda-de-comida}{%
\subsection{Normas relativas a la ofrenda de
comida}\label{normas-relativas-a-la-ofrenda-de-comida}}

\hypertarget{section-1}{%
\section{2}\label{section-1}}

\bibverse{1} Y cuando alguna persona ofreciere oblación de presente á
Jehová, su ofrenda será flor de harina, sobre la cual echará aceite, y
pondrá sobre ella incienso: \bibverse{2} Y la traerá á los sacerdotes,
hijos de Aarón; y de ello tomará el sacerdote su puño lleno de su flor
de harina y de su aceite, con todo su incienso, y lo hará arder sobre el
altar: ofrenda encendida para recuerdo, de olor suave á Jehová.
\bibverse{3} Y la sobra del presente será de Aarón y de sus hijos: es
cosa santísima de las ofrendas que se queman á Jehová.

\bibverse{4} Y cuando ofrecieres ofrenda de presente cocida en horno,
será de tortas de flor de harina sin levadura, amasadas con aceite, y
hojaldres sin levadura untadas con aceite. \bibverse{5} Mas si tu
presente fuere ofrenda de sartén, será de flor de harina sin levadura,
amasada con aceite, \bibverse{6} La cual partirás en piezas, y echarás
sobre ella aceite: es presente. \bibverse{7} Y si tu presente fuere
ofrenda cocida en cazuela, haráse de flor de harina con aceite.

\hypertarget{informaciuxf3n-general-sobre-la-preparaciuxf3n-y-oferta-de-las-comidas}{%
\subsection{Información general sobre la preparación y oferta de las
comidas}\label{informaciuxf3n-general-sobre-la-preparaciuxf3n-y-oferta-de-las-comidas}}

\bibverse{8} Y traerás á Jehová la ofrenda que se hará de estas cosas, y
la presentarás al sacerdote, el cual la llegará al altar. \bibverse{9} Y
tomará el sacerdote de aquel presente, en memoria del mismo, y harálo
arder sobre el altar; ofrenda encendida, de suave olor á Jehová.
\bibverse{10} Y lo restante del presente será de Aarón y de sus hijos:
es cosa santísima de las ofrendas que se queman á Jehová.

\bibverse{11} Ningún presente que ofreciereis á Jehová, será con
levadura: porque de ninguna cosa leuda, ni de ninguna miel, se ha de
quemar ofrenda á Jehová. \footnote{\textbf{2:11} Lev 6,10} \bibverse{12}
En la ofrenda de las primicias las ofreceréis á Jehová: mas no subirán
sobre el altar en olor de suavidad. \footnote{\textbf{2:12} Núm 18,12}
\bibverse{13} Y sazonarás toda ofrenda de tu presente con sal; y no
harás que falte jamás de tu presente la sal de la alianza de tu Dios: en
toda ofrenda tuya ofrecerás sal. \footnote{\textbf{2:13} Mar 9,49}

\hypertarget{ofrenda-de-comida-de-las-primicias-de-los-cereales}{%
\subsection{Ofrenda de comida de las primicias de los
cereales}\label{ofrenda-de-comida-de-las-primicias-de-los-cereales}}

\bibverse{14} Y si ofrecieres á Jehová presente de primicias, tostarás
al fuego las espigas verdes, y el grano desmenuzado ofrecerás por
ofrenda de tus primicias. \footnote{\textbf{2:14} Deut 26,2-3}

\bibverse{15} Y pondrás sobre ella aceite, y pondrás sobre ella
incienso: es presente. \bibverse{16} Y el sacerdote hará arder, en
memoria del don, parte de su grano desmenuzado, y de su aceite con todo
su incienso; es ofrenda encendida á Jehová.

\hypertarget{ofrendas-de-salvaciuxf3n-del-ganado}{%
\subsection{Ofrendas de salvación del
ganado}\label{ofrendas-de-salvaciuxf3n-del-ganado}}

\hypertarget{section-2}{%
\section{3}\label{section-2}}

\bibverse{1} Y si su ofrenda fuere sacrificio de paces, si hubiere de
ofrecerlo de ganado vacuno, sea macho ó hembra, sin defecto lo ofrecerá
delante de Jehová: \bibverse{2} Y pondrá su mano sobre la cabeza de su
ofrenda, y la degollará á la puerta del tabernáculo del testimonio; y
los sacerdotes, hijos de Aarón, rociarán su sangre sobre el altar en
derredor. \bibverse{3} Luego ofrecerá del sacrificio de las paces, por
ofrenda encendida á Jehová, el sebo que cubre los intestinos, y todo el
sebo que está sobre las entrañas, \bibverse{4} Y los dos riñones, y el
sebo que está sobre ellos, y sobre los ijares, y con los riñones quitará
el redaño que está sobre el hígado. \bibverse{5} Y los hijos de Aarón
harán arder esto en el altar, sobre el holocausto que estará sobre la
leña que habrá encima del fuego; es ofrenda de olor suave á Jehová.

\hypertarget{ofrendas-de-salvaciuxf3n-de-rebauxf1os}{%
\subsection{Ofrendas de salvación de
rebaños}\label{ofrendas-de-salvaciuxf3n-de-rebauxf1os}}

\bibverse{6} Mas si de ovejas fuere su ofrenda para sacrificio de paces
á Jehová, sea macho ó hembra, ofrecerála sin tacha. \bibverse{7} Si
ofreciere cordero por su ofrenda, ha de ofrecerlo delante de Jehová:
\bibverse{8} Y pondrá su mano sobre la cabeza de su ofrenda, y después
la degollará delante del tabernáculo del testimonio; y los hijos de
Aarón rociarán su sangre sobre el altar en derredor. \bibverse{9} Y del
sacrificio de las paces ofrecerá por ofrenda encendida á Jehová, su
sebo, y la cola entera, la cual quitará á raíz del espinazo, y el sebo
que cubre los intestinos, y todo el sebo que está sobre las entrañas:
\bibverse{10} Asimismo los dos riñones, y el sebo que está sobre ellos,
y el que está sobre los ijares, y con los riñones quitará el redaño de
sobre el hígado. \bibverse{11} Y el sacerdote hará arder esto sobre el
altar; vianda de ofrenda encendida á Jehová.

\bibverse{12} Y si fuere cabra su ofrenda ofrecerála delante de Jehová:
\bibverse{13} Y pondrá su mano sobre la cabeza de ella, y la degollará
delante del tabernáculo del testimonio; y los hijos de Aarón rociarán su
sangre sobre el altar en derredor. \bibverse{14} Después ofrecerá de
ella su ofrenda encendida á Jehová; el sebo que cubre los intestinos, y
todo el sebo que está sobre las entrañas, \bibverse{15} Y los dos
riñones, y el sebo que está sobre ellos, y el que está sobre los ijares,
y con los riñones quitará el redaño de sobre el hígado. \bibverse{16} Y
el sacerdote hará arder esto sobre el altar; es vianda de ofrenda que se
quema en olor de suavidad á Jehová: el sebo todo es de Jehová.

\bibverse{17} Estatuto perpetuo por vuestras edades; en todas vuestras
moradas, ningún sebo ni ninguna sangre comeréis. \footnote{\textbf{3:17}
  Gén 9,4; Lev 7,23; Lev 7,26; Lev 17,10-14; Deut 12,16; Deut 12,23;
  Hech 15,20; Hech 15,29}

\hypertarget{regulaciones-relativas-a-las-ofrendas-por-el-pecado}{%
\subsection{Regulaciones relativas a las ofrendas por el
pecado}\label{regulaciones-relativas-a-las-ofrendas-por-el-pecado}}

\hypertarget{section-3}{%
\section{4}\label{section-3}}

\bibverse{1} Y habló Jehová á Moisés, diciendo: \bibverse{2} Habla á los
hijos de Israel, diciendo: Cuando alguna persona pecare por yerro en
alguno de los mandamientos de Jehová sobre cosas que no se han de hacer,
y obrare contra alguno de ellos;

\hypertarget{sacrificio-cuando-el-sumo-sacerdote-pecuxf3}{%
\subsection{Sacrificio cuando el sumo sacerdote
pecó}\label{sacrificio-cuando-el-sumo-sacerdote-pecuxf3}}

\bibverse{3} Si sacerdote ungido pecare según el pecado del pueblo,
ofrecerá á Jehová, por su pecado que habrá cometido, un becerro sin
tacha para expiación. \bibverse{4} Y traerá el becerro á la puerta del
tabernáculo del testimonio delante de Jehová, y pondrá su mano sobre la
cabeza del becerro, y lo degollará delante de Jehová. \bibverse{5} Y el
sacerdote ungido tomará de la sangre del becerro, y la traerá al
tabernáculo del testimonio; \bibverse{6} Y mojará el sacerdote su dedo
en la sangre, y rociará de aquella sangre siete veces delante de Jehová,
hacia el velo del santuario. \bibverse{7} Y pondrá el sacerdote de la
sangre sobre los cuernos del altar del perfume aromático, que está en el
tabernáculo del testimonio delante de Jehová: y echará toda la sangre
del becerro al pie del altar del holocausto, que está á la puerta del
tabernáculo del testimonio. \footnote{\textbf{4:7} Éxod 30,1; Éxod 30,6;
  Éxod 40,6} \bibverse{8} Y tomará del becerro para la expiación todo su
sebo, el sebo que cubre los intestinos, y todo el sebo que está sobre
las entrañas, \bibverse{9} Y los dos riñones, y el sebo que está sobre
ellos, y el que está sobre los ijares, y con los riñones quitará el
redaño de sobre el hígado, \bibverse{10} De la manera que se quita del
buey del sacrificio de las paces: y el sacerdote lo hará arder sobre el
altar del holocausto. \bibverse{11} Y el cuero del becerro, y toda su
carne, con su cabeza, y sus piernas, y sus intestinos, y su estiércol,
\bibverse{12} En fin, todo el becerro sacará fuera del campo, á un lugar
limpio, donde se echan las cenizas, y lo quemará al fuego sobre la leña:
en donde se echan las cenizas será quemado. \footnote{\textbf{4:12} Lev
  6,4; Heb 13,11}

\hypertarget{sacrificio-por-el-pecado-de-toda-la-iglesia}{%
\subsection{Sacrificio por el pecado de toda la
iglesia}\label{sacrificio-por-el-pecado-de-toda-la-iglesia}}

\bibverse{13} Y si toda la congregación de Israel hubiere errado, y el
negocio estuviere oculto á los ojos del pueblo, y hubieren hecho algo
contra alguno de los mandamientos de Jehová en cosas que no se han de
hacer, y fueren culpables; \footnote{\textbf{4:13} Núm 15,24}
\bibverse{14} Luego que fuere entendido el pecado sobre que
delinquieron, la congregación ofrecerá un becerro por expiación, y lo
traerán delante del tabernáculo del testimonio. \footnote{\textbf{4:14}
  Rom 8,3} \bibverse{15} Y los ancianos de la congregación pondrán sus
manos sobre la cabeza del becerro delante de Jehová; y en presencia de
Jehová degollarán aquel becerro. \bibverse{16} Y el sacerdote ungido
meterá de la sangre del becerro en el tabernáculo del testimonio:
\bibverse{17} Y mojará el sacerdote su dedo en la misma sangre, y
rociará siete veces delante de Jehová hacia el velo. \bibverse{18} Y de
aquella sangre pondrá sobre los cuernos del altar que está delante de
Jehová en el tabernáculo del testimonio, y derramará toda la sangre al
pie del altar del holocausto, que está á la puerta del tabernáculo del
testimonio. \bibverse{19} Y le quitará todo el sebo, y harálo arder
sobre el altar. \bibverse{20} Y hará de aquel becerro como hizo con el
becerro de la expiación; lo mismo hará de él: así hará el sacerdote
expiación por ellos, y obtendrán perdón. \bibverse{21} Y sacará el
becerro fuera del campamento, y lo quemará como quemó el primer becerro;
expiación de la congregación.

\hypertarget{sacrificio-por-el-pecado-del-pruxedncipe}{%
\subsection{Sacrificio por el pecado del
príncipe}\label{sacrificio-por-el-pecado-del-pruxedncipe}}

\bibverse{22} Y cuando pecare el príncipe, é hiciere por yerro algo
contra alguno de todos los mandamientos de Jehová su Dios, sobre cosas
que no se han de hacer, y pecare; \bibverse{23} Luego que le fuere
conocido su pecado en que ha delinquido, presentará por su ofrenda un
macho cabrío sin defecto; \bibverse{24} Y pondrá su mano sobre la cabeza
del macho cabrío, y lo degollará en el lugar donde se degüella el
holocausto delante de Jehová; es expiación. \bibverse{25} Y tomará el
sacerdote con su dedo de la sangre de la expiación, y pondrá sobre los
cuernos del altar del holocausto, y derramará la sangre al pie del altar
del holocausto: \bibverse{26} Y quemará todo su sebo sobre el altar,
como el sebo del sacrificio de las paces: así hará el sacerdote por él
la expiación de su pecado, y tendrá perdón.

\hypertarget{sacrificio-por-el-pecado-de-un-israelita-comuxfan}{%
\subsection{Sacrificio por el pecado de un israelita
común}\label{sacrificio-por-el-pecado-de-un-israelita-comuxfan}}

\bibverse{27} Y si alguna persona del común del pueblo pecare por yerro,
haciendo algo contra alguno de los mandamientos de Jehová en cosas que
no se han de hacer, y delinquiere; \bibverse{28} Luego que le fuere
conocido su pecado que cometió, traerá por su ofrenda una hembra de las
cabras, una cabra sin defecto, por su pecado que habrá cometido:
\bibverse{29} Y pondrá su mano sobre la cabeza de la expiación, y la
degollará en el lugar del holocausto. \bibverse{30} Luego tomará el
sacerdote en su dedo de su sangre, y pondrá sobre los cuernos del altar
del holocausto, y derramará toda su sangre al pie del altar:
\bibverse{31} Y le quitará todo su sebo, de la manera que fué quitado el
sebo del sacrificio de las paces; y el sacerdote lo hará arder sobre el
altar en olor de suavidad á Jehová: así hará el sacerdote expiación por
él, y será perdonado. \footnote{\textbf{4:31} Lev 3,14-15}

\bibverse{32} Y si trajere cordero para su ofrenda por el pecado, hembra
sin defecto traerá: \bibverse{33} Y pondrá su mano sobre la cabeza de la
expiación, y la degollará por expiación en el lugar donde se degüella el
holocausto. \bibverse{34} Después tomará el sacerdote con su dedo de la
sangre de la expiación, y pondrá sobre los cuernos del altar del
holocausto; y derramará toda la sangre al pie del altar: \bibverse{35} Y
le quitará todo su sebo, como fué quitado el sebo del sacrificio de las
paces, y harálo el sacerdote arder en el altar sobre la ofrenda
encendida á Jehová: y le hará el sacerdote expiación de su pecado que
habrá cometido, y será perdonado.

\hypertarget{acerca-de-algunas-ofrendas-especiales-por-el-pecado}{%
\subsection{Acerca de algunas ofrendas especiales por el
pecado}\label{acerca-de-algunas-ofrendas-especiales-por-el-pecado}}

\hypertarget{section-4}{%
\section{5}\label{section-4}}

\bibverse{1} Y cuando alguna persona pecare, que hubiere oído la voz del
que juró, y él fuere testigo que vió, ó supo, si no lo denunciare, él
llevará su pecado.

\bibverse{2} Asimismo la persona que hubiere tocado en cualquiera cosa
inmunda, sea cuerpo muerto de bestia inmunda, ó cuerpo muerto de animal
inmundo, ó cuerpo muerto de reptil inmundo, bien que no lo supiere, será
inmunda y habrá delinquido: \footnote{\textbf{5:2} Lev 11,24}

\bibverse{3} O si tocare á hombre inmundo en cualquiera inmundicia suya
de que es inmundo, y no lo echare de ver; si después llega á saberlo,
será culpable.

\bibverse{4} También la persona que jurare, pronunciando con sus labios
hacer mal ó bien, en cualesquiera cosas que el hombre profiere con
juramento, y él no lo conociere; si después lo entiende, será culpado en
una de estas cosas. \bibverse{5} Y será que cuando pecare en alguna de
estas cosas, confesará aquello en que pecó: \bibverse{6} Y para su
expiación traerá á Jehová por su pecado que ha cometido, una hembra de
los rebaños, una cordera ó una cabra como ofrenda de expiación; y el
sacerdote hará expiación por él de su pecado.

\hypertarget{reemplazo-de-la-ofrenda-por-el-pecado-del-pobre}{%
\subsection{Reemplazo de la ofrenda por el pecado del
pobre}\label{reemplazo-de-la-ofrenda-por-el-pecado-del-pobre}}

\bibverse{7} Y si no le alcanzare para un cordero, traerá en expiación
por su pecado que cometió, dos tórtolas ó dos palominos á Jehová; el uno
para expiación, y el otro para holocausto. \bibverse{8} Y ha de traerlos
al sacerdote, el cual ofrecerá primero el que es para expiación, y
desunirá su cabeza de su cuello, mas no la apartará del todo:
\bibverse{9} Y rociará de la sangre de la expiación sobre la pared del
altar; y lo que sobrare de la sangre lo exprimirá al pie del altar; es
expiación. \bibverse{10} Y del otro hará holocausto conforme al rito; y
hará por él el sacerdote expiación de su pecado que cometió, y será
perdonado. \footnote{\textbf{5:10} Lev 1,14}

\bibverse{11} Mas si su posibilidad no alcanzare para dos tórtolas, ó
dos palominos, el que pecó traerá por su ofrenda la décima parte de un
epha de flor de harina por expiación. No pondrá sobre ella aceite, ni
sobre ella pondrá incienso, porque es expiación. \footnote{\textbf{5:11}
  Lev 2,1} \bibverse{12} Traerála, pues, al sacerdote, y el sacerdote
tomará de ella su puño lleno, en memoria suya, y la hará arder en el
altar sobre las ofrendas encendidas á Jehová: es expiación.
\bibverse{13} Y hará el sacerdote expiación por él de su pecado que
cometió en alguna de estas cosas, y será perdonado; y el sobrante será
del sacerdote, como el presente de vianda. \footnote{\textbf{5:13} Lev
  2,3}

\hypertarget{regulaciones-sobre-las-ofrendas-por-la-culpa-en-el-caso-de-apropiaciuxf3n-indebida-de-impuestos-sagrados}{%
\subsection{Regulaciones sobre las ofrendas por la culpa, en el caso de
apropiación indebida de impuestos
sagrados}\label{regulaciones-sobre-las-ofrendas-por-la-culpa-en-el-caso-de-apropiaciuxf3n-indebida-de-impuestos-sagrados}}

\bibverse{14} Habló más Jehová á Moisés, diciendo: \bibverse{15} Cuando
alguna persona cometiere falta, y pecare por yerro en las cosas
santificadas á Jehová, traerá su expiación á Jehová, un carnero sin
tacha de los rebaños, conforme á tu estimación, en siclos de plata del
siclo del santuario, en ofrenda por el pecado: \bibverse{16} Y pagará
aquello de las cosas santas en que hubiere pecado, y añadirá á ello el
quinto, y lo dará al sacerdote: y el sacerdote hará expiación por él con
el carnero del sacrificio por el pecado, y será perdonado.

\hypertarget{con-endeudamiento-inconsciente}{%
\subsection{Con endeudamiento
inconsciente}\label{con-endeudamiento-inconsciente}}

\bibverse{17} Finalmente, si una persona pecare, ó hiciere alguna de
todas aquellas cosas que por mandamiento de Jehová no se han de hacer,
aun sin hacerlo á sabiendas, es culpable, y llevará su pecado.
\bibverse{18} Traerá, pues, al sacerdote por expiación, según tú lo
estimes, un carnero sin tacha de los rebaños: y el sacerdote hará
expiación por él de su yerro que cometió por ignorancia, y será
perdonado. \bibverse{19} Es infracción, y ciertamente delinquió contra
Jehová. \footnote{\textbf{5:19} Is 53,10}

\hypertarget{cuando-se-dauxf1a-la-propiedad-de-otra-persona}{%
\subsection{Cuando se daña la propiedad de otra
persona}\label{cuando-se-dauxf1a-la-propiedad-de-otra-persona}}

\hypertarget{section-5}{%
\section{6}\label{section-5}}

\bibverse{1} Y habló Jehová á Moisés, diciendo: \bibverse{2} Cuando una
persona pecare, é hiciere prevaricación contra Jehová, y negare á su
prójimo lo encomendado, ó dejado en su mano, ó bien robare, ó calumniare
á su prójimo; \bibverse{3} O sea que hallando lo perdido, después lo
negare, y jurare en falso, en alguna de todas aquellas cosas en que
suele pecar el hombre: \footnote{\textbf{6:3} Éxod 28,42} \bibverse{4}
Entonces será que, puesto habrá pecado y ofendido, restituirá aquello
que robó, ó por el daño de la calumnia, ó el depósito que se le
encomendó, ó lo perdido que halló, \footnote{\textbf{6:4} Lev 4,12}
\bibverse{5} O todo aquello sobre que hubiere jurado falsamente; lo
restituirá, pues, por entero, y añadirá á ello la quinta parte, que ha
de pagar á aquel á quien pertenece en el día de su expiación.
\bibverse{6} Y por su expiación traerá á Jehová un carnero sin tacha de
los rebaños, conforme á tu estimación, al sacerdote para la expiación:
\bibverse{7} Y el sacerdote hará expiación por él delante de Jehová, y
obtendrá perdón de cualquiera de todas las cosas en que suele ofender.
\footnote{\textbf{6:7} Lev 2,-1}

\hypertarget{reglamento-para-los-sacerdotes-sobre-el-holocausto-diario}{%
\subsection{Reglamento para los sacerdotes sobre el holocausto
diario}\label{reglamento-para-los-sacerdotes-sobre-el-holocausto-diario}}

\bibverse{8} Habló aún Jehová á Moisés, diciendo: \bibverse{9} Manda á
Aarón y á sus hijos diciendo: Esta es la ley del holocausto: (es
holocausto, porque se quema sobre el altar toda la noche hasta la
mañana, y el fuego del altar arderá en él:) \bibverse{10} El sacerdote
se pondrá su vestimenta de lino, y se vestirá pañetes de lino sobre su
carne; y cuando el fuego hubiere consumido el holocausto, apartará él
las cenizas de sobre el altar, y pondrálas junto al altar. \bibverse{11}
Después se desnudará de sus vestimentas, y se pondrá otras vestiduras, y
sacará las cenizas fuera del real al lugar limpio. \bibverse{12} Y el
fuego encendido sobre el altar no ha de apagarse, sino que el sacerdote
pondrá en él leña cada mañana, y acomodará sobre él el holocausto, y
quemará sobre él los sebos de las paces. \bibverse{13} El fuego ha de
arder continuamente en el altar; no se apagará.

\hypertarget{reglamento-para-los-sacerdotes-sobre-la-ofrenda-de-alimentos}{%
\subsection{Reglamento para los sacerdotes sobre la ofrenda de
alimentos}\label{reglamento-para-los-sacerdotes-sobre-la-ofrenda-de-alimentos}}

\bibverse{14} Y esta es la ley del presente: Han de ofrecerlo los hijos
de Aarón delante de Jehová, delante del altar. \bibverse{15} Y tomará de
él un puñado de la flor de harina del presente, y de su aceite, y todo
el incienso que está sobre el presente, y harálo arder sobre el altar
por memoria, en olor suavísimo á Jehová. \bibverse{16} Y el sobrante de
ella lo comerán Aarón y sus hijos: sin levadura se comerá en el lugar
santo; en el atrio del tabernáculo del testimonio lo comerán.
\bibverse{17} No se cocerá con levadura: helo dado á ellos por su
porción de mis ofrendas encendidas; es cosa santísima, como la expiación
por el pecado, y como la expiación por la culpa. \bibverse{18} Todos los
varones de los hijos de Aarón comerán de ella. Estatuto perpetuo será
para vuestras generaciones tocante á las ofrendas encendidas de Jehová:
toda cosa que tocare en ellas será santificada.

\hypertarget{normas-relativas-a-la-ofrenda-de-comida-del-sumo-sacerdote}{%
\subsection{Normas relativas a la ofrenda de comida del sumo
sacerdote}\label{normas-relativas-a-la-ofrenda-de-comida-del-sumo-sacerdote}}

\bibverse{19} Y habló Jehová á Moisés, diciendo: \bibverse{20} Esta es
la ofrenda de Aarón y de sus hijos, que ofrecerán á Jehová el día que
serán ungidos: la décima parte de un epha de flor de harina, presente
perpetuo, la mitad á la mañana y la mitad á la tarde. \bibverse{21} En
sartén se aderezará con aceite; frita la traerás, y los pedazos cocidos
del presente ofrecerás á Jehová en olor de suavidad. \bibverse{22} Y el
sacerdote que en lugar de Aarón fuere ungido de entre sus hijos, hará la
ofrenda; estatuto perpetuo de Jehová: toda ella será quemada.
\bibverse{23} Y todo presente de sacerdote será enteramente quemado; no
se comerá.

\hypertarget{regulaciones-especialmente-para-los-sacerdotes-con-respecto-a-la-ofrenda-por-el-pecado}{%
\subsection{Regulaciones especialmente para los sacerdotes con respecto
a la ofrenda por el
pecado}\label{regulaciones-especialmente-para-los-sacerdotes-con-respecto-a-la-ofrenda-por-el-pecado}}

\bibverse{24} Y habló Jehová á Moisés, diciendo: \bibverse{25} Habla á
Aarón y á sus hijos, diciendo: Esta es la ley de la expiación: en el
lugar donde será degollado el holocausto, será degollada la expiación
por el pecado delante de Jehová: es cosa santísima. \bibverse{26} El
sacerdote que la ofreciere por expiación, la comerá: en el lugar santo
será comida, en el atrio del tabernáculo del testimonio. \bibverse{27}
Todo lo que en su carne tocare, será santificado; y si cayere de su
sangre sobre el vestido, lavarás aquello sobre que cayere, en el lugar
santo. \bibverse{28} Y la vasija de barro en que fuere cocida, será
quebrada: y si fuere cocida en vasija de metal, será fregada y lavada
con agua. \bibverse{29} Todo varón de entre los sacerdotes la comerá: es
cosa santísima. \bibverse{30} Mas no se comerá de expiación alguna, de
cuya sangre se metiere en el tabernáculo del testimonio para reconciliar
en el santuario: al fuego será quemada.

\hypertarget{regulaciones-sobre-el-sacrificio-de-culpa}{%
\subsection{Regulaciones sobre el sacrificio de
culpa}\label{regulaciones-sobre-el-sacrificio-de-culpa}}

\hypertarget{section-6}{%
\section{7}\label{section-6}}

\bibverse{1} Asimismo esta es la ley de la expiación de la culpa: es
cosa muy santa. \footnote{\textbf{7:1} Lev 5,14-26} \bibverse{2} En el
lugar donde degollaren el holocausto, degollarán la víctima por la
culpa; y rociará su sangre en derredor sobre el altar: \footnote{\textbf{7:2}
  Lev 1,3; Lev 1,5} \bibverse{3} Y de ella ofrecerá todo su sebo, la
cola, y el sebo que cubre los intestinos. \bibverse{4} Y los dos
riñones, y el sebo que está sobre ellos, y el que está sobre los ijares;
y con los riñones quitará el redaño de sobre el hígado. \footnote{\textbf{7:4}
  Lev 3,9-10} \bibverse{5} Y el sacerdote lo hará arder sobre el altar;
ofrenda encendida á Jehová: es expiación de la culpa. \bibverse{6} Todo
varón de entre los sacerdotes la comerá: será comida en el lugar santo:
es cosa muy santa.

\bibverse{7} Como la expiación por el pecado, así es la expiación de la
culpa: una misma ley tendrán: será del sacerdote que habrá hecho la
reconciliación con ella.

\hypertarget{participaciuxf3n-del-sacerdote-en-holocaustos-privados-y-ofrendas-privadas-de-comida}{%
\subsection{Participación del sacerdote en holocaustos privados y
ofrendas privadas de
comida}\label{participaciuxf3n-del-sacerdote-en-holocaustos-privados-y-ofrendas-privadas-de-comida}}

\bibverse{8} Y el sacerdote que ofreciere holocausto de alguno, el cuero
del holocausto que ofreciere, será para él. \bibverse{9} Asimismo todo
presente que se cociere en horno, y todo el que fuere aderezado en
sartén, ó en cazuela, será del sacerdote que lo ofreciere. \footnote{\textbf{7:9}
  Lev 2,4-5; Lev 2,7} \bibverse{10} Y todo presente amasado con aceite,
y seco, será de todos los hijos de Aarón, tanto al uno como al otro.

\hypertarget{regulaciones-para-diferentes-tipos-de-ofrendas-de-salvaciuxf3n}{%
\subsection{Regulaciones para diferentes tipos de ofrendas de
salvación}\label{regulaciones-para-diferentes-tipos-de-ofrendas-de-salvaciuxf3n}}

\bibverse{11} Y esta es la ley del sacrificio de las paces, que se
ofrecerá á Jehová: \bibverse{12} Si se ofreciere en hacimiento de
gracias, ofrecerá por sacrificio de hacimiento de gracias tortas sin
levadura amasadas con aceite, y hojaldres sin levadura untadas con
aceite, y flor de harina frita en tortas amasadas con aceite.
\footnote{\textbf{7:12} Lev 22,29} \bibverse{13} Con tortas de pan leudo
ofrecerá su ofrenda en el sacrificio de hacimientos de gracias de sus
paces. \bibverse{14} Y de toda la ofrenda presentará una parte por
ofrenda elevada á Jehová, y será del sacerdote que rociare la sangre de
los pacíficos. \bibverse{15} Y la carne del sacrificio de sus pacíficos
en hacimiento de gracias, se comerá en el día que fuere ofrecida: no
dejarán de ella nada para otro día.

\bibverse{16} Mas si el sacrificio de su ofrenda fuere voto, ó
voluntario, el día que ofreciere su sacrificio será comido; y lo que de
él quedare, comerse ha el día siguiente: \bibverse{17} Y lo que quedare
para el tercer día de la carne del sacrificio, será quemado en el fuego.
\bibverse{18} Y si se comiere de la carne del sacrificio de sus paces el
tercer día, el que lo ofreciere no será acepto, ni le será imputado;
abominación será, y la persona que de él comiere llevará su pecado.

\bibverse{19} Y la carne que tocare á alguna cosa inmunda, no se comerá;
al fuego será quemada; mas cualquiera limpio comerá de aquesta carne.
\bibverse{20} Y la persona que comiere la carne del sacrificio de paces,
el cual es de Jehová, estando inmunda, aquella persona será cortada de
sus pueblos. \bibverse{21} Además, la persona que tocare alguna cosa
inmunda, en inmundicia de hombre, ó en animal inmundo, ó en cualquiera
abominación inmunda, y comiere la carne del sacrificio de las paces, el
cual es de Jehová, aquella persona será cortada de sus pueblos.

\hypertarget{prohibiciuxf3n-del-consumo-de-grasas-y-sangre.}{%
\subsection{Prohibición del consumo de grasas y
sangre.}\label{prohibiciuxf3n-del-consumo-de-grasas-y-sangre.}}

\bibverse{22} Habló aún Jehová á Moisés, diciendo: \bibverse{23} Habla á
los hijos de Israel, diciendo: Ningún sebo de buey, ni de cordero, ni de
cabra, comeréis. \footnote{\textbf{7:23} Lev 3,7} \bibverse{24} El sebo
de animal mortecino, y el sebo del que fué arrebatado de fieras, se
aparejará para cualquiera otro uso, mas no lo comeréis. \footnote{\textbf{7:24}
  Éxod 22,30} \bibverse{25} Porque cualquiera que comiere sebo de
animal, del cual se ofrece á Jehová ofrenda encendida, la persona que lo
comiere, será cortada de sus pueblos. \bibverse{26} Además, ninguna
sangre comeréis en todas vuestras habitaciones, así de aves como de
bestias. \footnote{\textbf{7:26} Lev 3,17} \bibverse{27} Cualquiera
persona que comiere alguna sangre, la tal persona será cortada de sus
pueblos.

\hypertarget{disposiciones-sobre-la-participaciuxf3n-de-los-sacerdotes-en-los-sacrificios-de-salvaciuxf3n}{%
\subsection{Disposiciones sobre la participación de los sacerdotes en
los sacrificios de
salvación}\label{disposiciones-sobre-la-participaciuxf3n-de-los-sacerdotes-en-los-sacrificios-de-salvaciuxf3n}}

\bibverse{28} Habló más Jehová á Moisés, diciendo: \bibverse{29} Habla á
los hijos de Israel, diciendo: El que ofreciere sacrificio de sus paces
á Jehová, traerá su ofrenda del sacrificio de sus paces á Jehová;
\bibverse{30} Sus manos traerán las ofrendas que se han de quemar á
Jehová: traerá el sebo con el pecho: el pecho para que éste sea agitado,
como sacrificio agitado delante de Jehová; \bibverse{31} Y el sebo lo
hará arder el sacerdote en el altar; mas el pecho será de Aarón y de sus
hijos. \bibverse{32} Y daréis al sacerdote para ser elevada en ofrenda,
la espaldilla derecha de los sacrificios de vuestras paces. \footnote{\textbf{7:32}
  Lev 9,21} \bibverse{33} El que de los hijos de Aarón ofreciere la
sangre de las paces, y el sebo, de él será en porción la espaldilla
derecha; \bibverse{34} Porque he tomado de los hijos de Israel, de los
sacrificios de sus paces, el pecho que se agita, y la espaldilla elevada
en ofrenda, y lo he dado á Aarón el sacerdote y á sus hijos, por
estatuto perpetuo de los hijos de Israel.

\bibverse{35} Esta es por la unción de Aarón y la unción de sus hijos,
la parte de ellos en las ofrendas encendidas á Jehová, desde el día que
él los allegó para ser sacerdotes de Jehová: \bibverse{36} Lo cual mandó
Jehová que les diesen, desde el día que él los ungió de entre los hijos
de Israel, por estatuto perpetuo en sus generaciones. \bibverse{37} Esta
es la ley del holocausto, del presente, de la expiación por el pecado, y
de la culpa, y de las consagraciones, y del sacrificio de las paces:
\bibverse{38} La cual intimó Jehová á Moisés, en el monte de Sinaí, el
día que mandó á los hijos de Israel que ofreciesen sus ofrendas á Jehová
en el desierto de Sinaí.

\hypertarget{consagraciuxf3n-de-aaruxf3n-y-sus-cuatro-hijos}{%
\subsection{Consagración de Aarón y sus cuatro
hijos}\label{consagraciuxf3n-de-aaruxf3n-y-sus-cuatro-hijos}}

\hypertarget{section-7}{%
\section{8}\label{section-7}}

\bibverse{1} Y habló Jehová á Moisés, diciendo: \footnote{\textbf{8:1}
  Éxod 29,1-35} \bibverse{2} Toma á Aarón y á sus hijos con él, y las
vestimentas, y el aceite de la unción, y el becerro de la expiación, y
los dos carneros, y el canastillo de los ázimos; \bibverse{3} Y reune
toda la congregación á la puerta del tabernáculo del testimonio.

\bibverse{4} Hizo, pues, Moisés como Jehová le mandó, y juntóse la
congregación á la puerta del tabernáculo del testimonio. \bibverse{5} Y
dijo Moisés á la congregación: Esto es lo que Jehová ha mandado hacer.

\hypertarget{lavado-vestido-y-unciuxf3n-de-los-sacerdotes}{%
\subsection{Lavado, vestido y unción de los
sacerdotes}\label{lavado-vestido-y-unciuxf3n-de-los-sacerdotes}}

\bibverse{6} Entonces Moisés hizo llegar á Aarón y á sus hijos, y
lavólos con agua. \bibverse{7} Y puso sobre él la túnica, y ciñólo con
el cinto; vistióle después el manto, y puso sobre él el ephod, y ciñólo
con el cinto del ephod, y ajustólo con él. \bibverse{8} Púsole luego
encima el racional, y en él puso el Urim y Thummim. \bibverse{9} Después
puso la mitra sobre su cabeza; y sobre la mitra en su frente delantero
puso la plancha de oro, la diadema santa; como Jehová había mandado á
Moisés. \footnote{\textbf{8:9} Éxod 28,36; Éxod 39,30} \bibverse{10} Y
tomó Moisés el aceite de la unción, y ungió el tabernáculo, y todas las
cosas que estaban en él, y santificólas. \footnote{\textbf{8:10} Éxod
  30,25-26} \bibverse{11} Y roció de él sobre el altar siete veces, y
ungió el altar y todos sus vasos, y la fuente y su basa, para
santificarlos. \bibverse{12} Y derramó del aceite de la unción sobre la
cabeza de Aarón, y ungiólo para santificarlo. \bibverse{13} Después
Moisés hizo llegar los hijos de Aarón, y vistióles las túnicas, y
ciñólos con cintos, y ajustóles los chapeos (tiaras), como Jehová lo
había mandado á Moisés.

\hypertarget{la-ofrenda-sacerdotal-por-el-pecado}{%
\subsection{La ofrenda sacerdotal por el
pecado}\label{la-ofrenda-sacerdotal-por-el-pecado}}

\bibverse{14} Hizo luego llegar el becerro de la expiación, y Aarón y
sus hijos pusieron sus manos sobre la cabeza del becerro de la
expiación. \footnote{\textbf{8:14} Lev 4,-1} \bibverse{15} Y degollólo;
y Moisés tomó la sangre, y puso con su dedo sobre los cuernos del altar
alrededor, y purificó el altar; y echó la demás sangre al pie del altar,
y santificólo para reconciliar sobre él. \bibverse{16} Después tomó todo
el sebo que estaba sobre los intestinos, y el redaño del hígado, y los
dos riñones, y el sebo de ellos, é hízolo Moisés arder sobre el altar.
\bibverse{17} Mas el becerro, y su cuero, y su carne, y su estiércol,
quemólo al fuego fuera del real; como Jehová lo había mandado á Moisés.

\hypertarget{el-holocausto}{%
\subsection{El holocausto}\label{el-holocausto}}

\bibverse{18} Después hizo llegar el carnero del holocausto, y Aarón y
sus hijos pusieron sus manos sobre la cabeza del carnero: \bibverse{19}
Y degollólo; y roció Moisés la sangre sobre el altar en derredor.
\bibverse{20} Y cortó el carnero en trozos; y Moisés hizo arder la
cabeza, y los trozos, y el sebo. \bibverse{21} Lavó luego con agua los
intestinos y piernas, y quemó Moisés todo el carnero sobre el altar:
holocausto en olor de suavidad, ofrenda encendida á Jehová; como lo
había Jehová mandado á Moisés.

\hypertarget{ofrenda-de-iniciaciuxf3n-y-aspersiuxf3n}{%
\subsection{Ofrenda de iniciación y
aspersión}\label{ofrenda-de-iniciaciuxf3n-y-aspersiuxf3n}}

\bibverse{22} Después hizo llegar el otro carnero, el carnero de las
consagraciones, y Aarón y sus hijos pusieron sus manos sobre la cabeza
del carnero: \footnote{\textbf{8:22} Lev 7,37} \bibverse{23} Y
degollólo; y tomó Moisés de su sangre, y puso sobre la ternilla de la
oreja derecha de Aarón, y sobre el dedo pulgar de su mano derecha, y
sobre el dedo pulgar de su pie derecho. \bibverse{24} Hizo llegar luego
los hijos de Aarón, y puso Moisés de la sangre sobre la ternilla de sus
orejas derechas, y sobre los pulgares de sus manos derechas, y sobre los
pulgares de sus pies derechos: y roció Moisés la sangre sobre el altar
en derredor; \bibverse{25} Y después tomó el sebo, y la cola, y todo el
sebo que estaba sobre los intestinos, y el redaño del hígado, y los dos
riñones, y el sebo de ellos, y la espaldilla derecha; \bibverse{26} Y
del canastillo de los ázimos, que estaba delante de Jehová, tomó una
torta sin levadura, y una torta de pan de aceite, y una lasaña, y púsolo
con el sebo y con la espaldilla derecha; \bibverse{27} Y púsolo todo en
las manos de Aarón, y en las manos de sus hijos, é hízolo mecer: ofrenda
agitada delante de Jehová. \bibverse{28} Después tomó aquellas cosas
Moisés de las manos de ellos, é hízolas arder en el altar sobre el
holocausto: las consagraciones en olor de suavidad, ofrenda encendida á
Jehová. \bibverse{29} Y tomó Moisés el pecho, y meciólo, ofrenda agitada
delante de Jehová: del carnero de las consagraciones aquélla fué la
parte de Moisés; como Jehová lo había mandado á Moisés. \bibverse{30}
Luego tomó Moisés del aceite de la unción, y de la sangre que estaba
sobre el altar, y roció sobre Aarón, y sobre sus vestiduras, sobre sus
hijos, y sobre las vestiduras de sus hijos con él; y santificó á Aarón,
y sus vestiduras, y á sus hijos, y las vestiduras de sus hijos con él.

\hypertarget{regulaciones-relativas-a-la-comida-de-sacrificio-y-la-segregaciuxf3n-de-siete-duxedas}{%
\subsection{Regulaciones relativas a la comida de sacrificio y la
segregación de siete
días}\label{regulaciones-relativas-a-la-comida-de-sacrificio-y-la-segregaciuxf3n-de-siete-duxedas}}

\bibverse{31} Y dijo Moisés á Aarón y á sus hijos: Comed la carne á la
puerta del tabernáculo del testimonio; y comedla allí con el pan que
está en el canastillo de las consagraciones, según yo he mandado,
diciendo: Aarón y sus hijos la comerán. \bibverse{32} Y lo que sobrare
de la carne y del pan, habéis de quemarlo al fuego. \bibverse{33} De la
puerta del tabernáculo del testimonio no saldréis en siete días, hasta
el día que se cumplieren los días de vuestras consagraciones: porque por
siete días seréis consagrados. \bibverse{34} De la manera que hoy se ha
hecho, mandó hacer Jehová para expiaros. \bibverse{35} A la puerta,
pues, del tabernáculo del testimonio estaréis día y noche por siete
días, y guardaréis la ordenanza delante de Jehová, para que no muráis;
porque así me ha sido mandado. \bibverse{36} Y Aarón y sus hijos
hicieron todas las cosas que mandó Jehová por medio de Moisés.

\hypertarget{los-preparativos-para-el-sacrificio-de-aaruxf3n-y-sus-hijos}{%
\subsection{Los preparativos para el sacrificio de Aarón y sus
hijos}\label{los-preparativos-para-el-sacrificio-de-aaruxf3n-y-sus-hijos}}

\hypertarget{section-8}{%
\section{9}\label{section-8}}

\bibverse{1} Y fué en el día octavo, que Moisés llamó á Aarón y á sus
hijos, y á los ancianos de Israel; \bibverse{2} Y dijo á Aarón: Toma de
la vacada un becerro para expiación, y un carnero para holocausto, sin
defecto, y ofrécelos delante de Jehová. \bibverse{3} Y á los hijos de
Israel hablarás, diciendo: Tomad un macho cabrío para expiación, y un
becerro y un cordero de un año, sin tacha, para holocausto; \bibverse{4}
Asimismo un buey y un carnero para sacrificio de paces, que inmoléis
delante de Jehová; y un presente amasado con aceite: porque Jehová se
aparecerá hoy á vosotros.

\bibverse{5} Y llevaron lo que mandó Moisés delante del tabernáculo del
testimonio, y llegóse toda la congregación, y pusiéronse delante de
Jehová. \bibverse{6} Entonces Moisés dijo: Esto es lo que mandó Jehová;
hacedlo, y la gloria de Jehová se os aparecerá. \bibverse{7} Y dijo
Moisés á Aarón: Llégate al altar, y haz tu expiación, y tu holocausto, y
haz la reconciliación por ti y por el pueblo: haz también la ofrenda del
pueblo, y haz la reconciliación por ellos; como ha mandado Jehová.
\footnote{\textbf{9:7} Lev 16,6; Lev 16,11; Lev 16,15; Heb 5,3; Heb 7,27}

\hypertarget{la-ofrenda-por-el-pecado-y-el-holocausto-del-sumo-sacerdote}{%
\subsection{La ofrenda por el pecado y el holocausto del sumo
sacerdote}\label{la-ofrenda-por-el-pecado-y-el-holocausto-del-sumo-sacerdote}}

\bibverse{8} Entonces llegóse Aarón al altar; y degolló su becerro de la
expiación que era por él. \bibverse{9} Y los hijos de Aarón le trajeron
la sangre; y él mojó su dedo en la sangre, y puso sobre los cuernos del
altar, y derramó la demás sangre al pie del altar; \bibverse{10} Y el
sebo y riñones y redaño del hígado, de la expiación, hízolos arder sobre
el altar; como Jehová lo había mandado á Moisés. \bibverse{11} Mas la
carne y el cuero los quemó al fuego fuera del real. \bibverse{12}
Degolló asimismo el holocausto, y los hijos de Aarón le presentaron la
sangre, la cual roció él alrededor sobre el altar. \footnote{\textbf{9:12}
  Lev 1,10-13} \bibverse{13} Presentáronle después el holocausto, á
trozos, y la cabeza; é hízolos quemar sobre el altar. \bibverse{14}
Luego lavó los intestinos y las piernas, y quemólos sobre el holocausto
en el altar.

\hypertarget{los-cuatro-sacrificios-por-el-pueblo}{%
\subsection{Los cuatro sacrificios por el
pueblo}\label{los-cuatro-sacrificios-por-el-pueblo}}

\bibverse{15} Ofreció también la ofrenda del pueblo, y tomó el macho
cabrío que era para la expiación del pueblo, y degollólo, y lo ofreció
por el pecado como el primero. \bibverse{16} Y ofreció el holocausto, é
hizo según el rito. \bibverse{17} Ofreció asimismo el presente, é
hinchió de él su mano, y lo hizo quemar sobre el altar, además del
holocausto de la mañana. \bibverse{18} Degolló también el buey y el
carnero en sacrificio de paces, que era del pueblo: y los hijos de Aarón
le presentaron la sangre (la cual roció él sobre el altar alrededor),
\bibverse{19} Y los sebos del buey; y del carnero la cola con lo que
cubre las entrañas, y los riñones, y el redaño del hígado: \bibverse{20}
Y pusieron los sebos sobre los pechos, y él quemó los sebos sobre el
altar: \bibverse{21} Empero los pechos, con la espaldilla derecha,
meciólos Aarón por ofrenda agitada delante de Jehová; como Jehová lo
había mandado á Moisés.

\hypertarget{doble-bendiciuxf3n-del-pueblo-apariciuxf3n-de-la-gloria-del-seuxf1or-el-fuego-de-dios-consume-los-sacrificios-de-aaruxf3n}{%
\subsection{Doble bendición del pueblo; Aparición de la gloria del
Señor; el fuego de Dios consume los sacrificios de
Aarón}\label{doble-bendiciuxf3n-del-pueblo-apariciuxf3n-de-la-gloria-del-seuxf1or-el-fuego-de-dios-consume-los-sacrificios-de-aaruxf3n}}

\bibverse{22} Después alzó Aarón sus manos hacia el pueblo y bendíjolos:
y descendió de hacer la expiación, y el holocausto, y el sacrificio de
las paces. \footnote{\textbf{9:22} Núm 22,1-27} \bibverse{23} Y entraron
Moisés y Aarón en el tabernáculo del testimonio; y salieron, y
bendijeron al pueblo: y la gloria de Jehová se apareció á todo el
pueblo. \footnote{\textbf{9:23} Éxod 40,34} \bibverse{24} Y salió fuego
de delante de Jehová, y consumió el holocausto y los sebos sobre el
altar; y viéndolo todo el pueblo, alabaron, y cayeron sobre sus rostros.
\footnote{\textbf{9:24} 2Cró 7,1}

\hypertarget{el-pecado-y-la-muerte-de-nadab-y-abiuxfa}{%
\subsection{El pecado y la muerte de Nadab y
Abiú}\label{el-pecado-y-la-muerte-de-nadab-y-abiuxfa}}

\hypertarget{section-9}{%
\section{10}\label{section-9}}

\bibverse{1} Y los hijos de Aarón, Nadab y Abiú, tomaron cada uno su
incensario, y pusieron fuego en ellos, sobre el cual pusieron perfume, y
ofrecieron delante de Jehová fuego extraño, que él nunca les mandó.
\bibverse{2} Y salió fuego de delante de Jehová que los quemó, y
murieron delante de Jehová.

\bibverse{3} Entonces dijo Moisés á Aarón: Esto es lo que habló Jehová,
diciendo: En mis allegados me santificaré, y en presencia de todo el
pueblo seré glorificado. Y Aarón calló. \footnote{\textbf{10:3} 1Pe 4,17}

\bibverse{4} Y llamó Moisés á Misael, y á Elzaphán, hijos de Uzziel, tío
de Aarón, y díjoles: Llegaos y sacad á vuestros hermanos de delante del
santuario fuera del campo. \footnote{\textbf{10:4} Éxod 6,22; Hech 5,6;
  Hech 5,10} \bibverse{5} Y ellos llegaron, y sacáronlos con sus túnicas
fuera del campo, como dijo Moisés.

\hypertarget{reglas-para-los-sacerdotes-sobre-la-pruxe1ctica-del-duelo}{%
\subsection{Reglas para los sacerdotes sobre la práctica del
duelo}\label{reglas-para-los-sacerdotes-sobre-la-pruxe1ctica-del-duelo}}

\bibverse{6} Entonces Moisés dijo á Aarón, y á Eleazar y á Ithamar, sus
hijos: No descubráis vuestras cabezas, ni rasguéis vuestros vestidos,
porque no muráis, ni se levante la ira sobre toda la congregación:
empero vuestros hermanos, toda la casa de Israel, lamentarán el incendio
que Jehová ha hecho. \footnote{\textbf{10:6} Lev 21,10} \bibverse{7} Ni
saldréis de la puerta del tabernáculo del testimonio, porque moriréis;
por cuanto el aceite de la unción de Jehová está sobre vosotros. Y ellos
hicieron conforme al dicho de Moisés.

\hypertarget{prohibiciuxf3n-del-consumo-de-vino-a-los-sacerdotes-durante-su-labor-oficial}{%
\subsection{Prohibición del consumo de vino a los sacerdotes durante su
labor
oficial}\label{prohibiciuxf3n-del-consumo-de-vino-a-los-sacerdotes-durante-su-labor-oficial}}

\bibverse{8} Y Jehová habló á Aarón, diciendo: \bibverse{9} Tú, y tus
hijos contigo, no beberéis vino ni sidra, cuando hubiereis de entrar en
el tabernáculo del testimonio, porque no muráis: estatuto perpetuo por
vuestras generaciones; \bibverse{10} Y para poder discernir entre lo
santo y lo profano, y entre lo inmundo y lo limpio; \bibverse{11} Y para
enseñar á los hijos de Israel todos los estatutos que Jehová les ha
dicho por medio de Moisés.

\hypertarget{sobre-consumir-la-porciuxf3n-de-ofrendas-de-comida-y-ofrendas-de-paz-del-sacerdote}{%
\subsection{Sobre consumir la porción de ofrendas de comida y ofrendas
de paz del
sacerdote}\label{sobre-consumir-la-porciuxf3n-de-ofrendas-de-comida-y-ofrendas-de-paz-del-sacerdote}}

\bibverse{12} Y Moisés dijo á Aarón, y á Eleazar y á Ithamar, sus hijos
que habían quedado: Tomad el presente que queda de las ofrendas
encendidas á Jehová, y comedlo sin levadura junto al altar, porque es
cosa muy santa. \bibverse{13} Habéis, pues, de comerlo en el lugar
santo: porque esto es fuero para ti, y fuero para tus hijos, de las
ofrendas encendidas á Jehová, pues que así me ha sido mandado.
\footnote{\textbf{10:13} Lev 2,3} \bibverse{14} Comeréis asimismo en
lugar limpio, tú y tus hijos y tus hijas contigo, el pecho de la mecida,
y la espaldilla elevada, porque por fuero para ti, y fuero para tus
hijos, son dados de los sacrificios de las paces de los hijos de Israel.
\footnote{\textbf{10:14} Lev 7,34} \bibverse{15} Con las ofrendas de los
sebos que se han de encender, traerán la espaldilla que se ha de elevar,
y el pecho que será mecido, para que lo mezas por ofrenda agitada
delante de Jehová: y será por fuero perpetuo tuyo, y de tus hijos
contigo, como Jehová lo ha mandado.

\hypertarget{sobre-el-disfrute-de-la-carne-del-macho-cabruxedo-por-el-pecado-ofrecido-por-el-pueblo}{%
\subsection{Sobre el disfrute de la carne del macho cabrío por el pecado
ofrecido por el
pueblo}\label{sobre-el-disfrute-de-la-carne-del-macho-cabruxedo-por-el-pecado-ofrecido-por-el-pueblo}}

\bibverse{16} Y Moisés demandó el macho cabrío de la expiación, y
hallóse que era quemado: y enojóse contra Eleazar é Ithamar, los hijos
de Aarón que habían quedado, diciendo: \bibverse{17} ¿Por qué no
comisteis la expiación en el lugar santo? porque es muy santa, y dióla
él á vosotros para llevar la iniquidad de la congregación, para que sean
reconciliados delante de Jehová. \bibverse{18} Veis que su sangre no fué
metida dentro del santuario: habíais de comerla en el lugar santo, como
yo mandé. \footnote{\textbf{10:18} Lev 6,19; Lev 6,22}

\bibverse{19} Y respondió Aarón á Moisés: He aquí hoy han ofrecido su
expiación y su holocausto delante de Jehová: pero me han acontecido
estas cosas: pues si comiera yo hoy de la expiación, ¿hubiera sido
acepto á Jehová?

\bibverse{20} Y cuando Moisés oyó esto, dióse por satisfecho.

\hypertarget{ordenanzas-sobre-animales-limpios-e-inmundos}{%
\subsection{Ordenanzas sobre animales limpios e
inmundos}\label{ordenanzas-sobre-animales-limpios-e-inmundos}}

\hypertarget{section-10}{%
\section{11}\label{section-10}}

\bibverse{1} Y habló Jehová á Moisés y á Aarón, diciéndoles:
\bibverse{2} Hablad á los hijos de Israel, diciendo: Estos son los
animales que comeréis de todos los animales que están sobre la tierra.
\bibverse{3} De entre los animales, todo el de pezuña, y que tiene las
pezuñas hendidas, y que rumia, éste comeréis.

\bibverse{4} Estos empero no comeréis de los que rumian, y de los que
tienen pezuña: el camello, porque rumia mas no tiene pezuña hendida,
habéis de tenerlo por inmundo; \bibverse{5} También el conejo, porque
rumia, mas no tiene pezuña, tendréislo por inmundo; \bibverse{6}
Asimismo la liebre, porque rumia, mas no tiene pezuña, tendréisla por
inmunda; \bibverse{7} También el puerco, porque tiene pezuñas, y es de
pezuñas hendidas, mas no rumia, tendréislo por inmundo. \bibverse{8} De
la carne de ellos no comeréis, ni tocaréis su cuerpo muerto: tendréislos
por inmundos.

\bibverse{9} Esto comeréis de todas las cosas que están en las aguas:
todas las cosas que tienen aletas y escamas en las aguas de la mar, y en
los ríos, aquellas comeréis; \bibverse{10} Mas todas las cosas que no
tienen aletas ni escamas en la mar y en los ríos, así de todo reptil de
agua como de toda cosa viviente que está en las aguas, las tendréis en
abominación. \bibverse{11} Os serán, pues, en abominación: de su carne
no comeréis, y abominaréis sus cuerpos muertos. \bibverse{12} Todo lo
que no tuviere aletas y escamas en las aguas, tendréislo en abominación.

\bibverse{13} Y de las aves, éstas tendréis en abominación; no se
comerán, serán abominación: el águila, el quebrantahuesos, el esmerejón,
\bibverse{14} El milano, y el buitre según su especie; \bibverse{15}
Todo cuervo según su especie; \bibverse{16} El avestruz, y la lechuza, y
el laro, y el gavilán según su especie, \bibverse{17} Y el buho, y el
somormujo, y el ibis, \bibverse{18} Y el calamón, y el cisne, y el
onocrótalo, \bibverse{19} Y el herodión, y el caradrión, según su
especie, y la abubilla, y el murciélago.

\bibverse{20} Todo reptil alado que anduviere sobre cuatro pies,
tendréis en abominación. \bibverse{21} Empero esto comeréis de todo
reptil alado que anda sobre cuatro pies, que tuviere piernas además de
sus pies para saltar con ellas sobre la tierra; \bibverse{22} Estos
comeréis de ellos: la langosta según su especie, y el langostín según su
especie, y el aregol según su especie, y el haghab según su especie.
\bibverse{23} Todo reptil alado que tenga cuatro pies, tendréis en
abominación.

\hypertarget{disposiciones-sobre-la-contaminaciuxf3n-por-tocar-los-caduxe1veres-de-animales-inmundos-y-limpios}{%
\subsection{Disposiciones sobre la contaminación por tocar los cadáveres
de animales inmundos y
limpios}\label{disposiciones-sobre-la-contaminaciuxf3n-por-tocar-los-caduxe1veres-de-animales-inmundos-y-limpios}}

\bibverse{24} Y por estas cosas seréis inmundos: cualquiera que tocare á
sus cuerpos muertos, será inmundo hasta la tarde: \footnote{\textbf{11:24}
  Lev 5,2; Lev 14,46} \bibverse{25} Y cualquiera que llevare de sus
cuerpos muertos, lavará sus vestidos, y será inmundo hasta la tarde.

\bibverse{26} Todo animal de pezuña, pero que no tiene pezuña hendida,
ni rumia, tendréis por inmundo: cualquiera que los tocare será inmundo.
\bibverse{27} Y de todos los animales que andan á cuatro pies, tendréis
por inmundo cualquiera que ande sobre sus garras: cualquiera que tocare
sus cuerpos muertos, será inmundo hasta la tarde. \bibverse{28} Y el que
llevare sus cuerpos muertos, lavará sus vestidos, y será inmundo hasta
la tarde: habéis de tenerlos por inmundos.

\bibverse{29} Y estos tendréis por inmundos de los reptiles que van
arrastrando sobre la tierra: la comadreja, y el ratón, y la rana según
su especie, \bibverse{30} Y el erizo, y el lagarto, y el caracol, y la
babosa, y el topo. \bibverse{31} Estos tendréis por inmundos de todos
los reptiles: cualquiera que los tocare, cuando estuvieren muertos, será
inmundo hasta la tarde. \bibverse{32} Y todo aquello sobre que cayere
alguno de ellos después de muertos, será inmundo; así vaso de madera,
como vestido, ó piel, ó saco, cualquier instrumento con que se hace
obra, será metido en agua, y será inmundo hasta la tarde, y así será
limpio. \bibverse{33} Y toda vasija de barro dentro de la cual cayere
alguno de ellos, todo lo que estuviere en ella será inmundo, y
quebraréis la vasija: \bibverse{34} Toda vianda que se come, sobre la
cual viniere el agua de tales vasijas, será inmunda: y toda bebida que
se bebiere, será en todas esas vasijas inmunda: \bibverse{35} Y todo
aquello sobre que cayere algo del cuerpo muerto de ellos, será inmundo:
el horno ú hornillos se derribarán; son inmundos, y por inmundos los
tendréis. \bibverse{36} Con todo, la fuente y la cisterna donde se
recogen aguas, serán limpias: mas lo que hubiere tocado en sus cuerpos
muertos será inmundo. \bibverse{37} Y si cayere de sus cuerpos muertos
sobre alguna simiente que se haya de sembrar, será limpia. \bibverse{38}
Mas si se hubiere puesto agua en la simiente, y cayere de sus cuerpos
muertos sobre ella, tendréisla por inmunda.

\bibverse{39} Y si algún animal que tuviereis para comer se muriere, el
que tocare su cuerpo muerto será inmundo hasta la tarde: \bibverse{40} Y
el que comiere de su cuerpo muerto, lavará sus vestidos, y será inmundo
hasta la tarde: asimismo el que sacare su cuerpo muerto, lavará sus
vestidos, y será inmundo hasta la tarde.

\hypertarget{adiciuxf3n-sobre-el-consumo-de-reptiles}{%
\subsection{Adición sobre el consumo de
reptiles}\label{adiciuxf3n-sobre-el-consumo-de-reptiles}}

\bibverse{41} Y todo reptil que va arrastrando sobre la tierra, es
abominación; no se comerá. \bibverse{42} Todo lo que anda sobre el
pecho, y todo lo que anda sobre cuatro ó más pies, de todo reptil que
anda arrastrando sobre la tierra, no lo comeréis, porque es abominación.

\hypertarget{advertencias-finales}{%
\subsection{Advertencias finales}\label{advertencias-finales}}

\bibverse{43} No ensuciéis vuestras personas con ningún reptil que anda
arrastrando, ni os contaminéis con ellos, ni seáis inmundos por ellos.
\bibverse{44} Pues que yo soy Jehová vuestro Dios, vosotros por tanto os
santificaréis, y seréis santos, porque yo soy santo: así que no
ensuciéis vuestras personas con ningún reptil que anduviere arrastrando
sobre la tierra. \footnote{\textbf{11:44} Lev 19,2} \bibverse{45} Porque
yo soy Jehová, que os hago subir de la tierra de Egipto para seros por
Dios: seréis pues santos, porque yo soy santo. \footnote{\textbf{11:45}
  Lev 20,26}

\bibverse{46} Esta es la ley de los animales y de las aves, y de todo
ser viviente que se mueve en las aguas, y de todo animal que anda
arrastrando sobre la tierra; \bibverse{47} Para hacer diferencia entre
inmundo y limpio, y entre los animales que se pueden comer y los
animales que no se pueden comer.

\hypertarget{normativa-sobre-mujeres-que-han-dado-a-luz-recientemente}{%
\subsection{Normativa sobre mujeres que han dado a luz
recientemente}\label{normativa-sobre-mujeres-que-han-dado-a-luz-recientemente}}

\hypertarget{section-11}{%
\section{12}\label{section-11}}

\bibverse{1} Y habló Jehová á Moisés, diciendo: \bibverse{2} Habla á los
hijos de Israel, diciendo: La mujer cuando concibiere y pariere varón,
será inmunda siete días; conforme á los días que está separada por su
menstruo, será inmunda. \footnote{\textbf{12:2} Lev 15,19} \bibverse{3}
Y al octavo día circuncidará la carne de su prepucio. \footnote{\textbf{12:3}
  Gén 17,11-12; Juan 7,22; Luc 2,21} \bibverse{4} Mas ella permanecerá
treinta y tres días en la sangre de su purgación: ninguna cosa santa
tocará, ni vendrá al santuario, hasta que sean cumplidos los días de su
purgación. \bibverse{5} Y si pariere hembra será inmunda dos semanas,
conforme á su separación, y sesenta y seis días estará purificándose de
su sangre.

\bibverse{6} Y cuando los días de su purgación fueren cumplidos, por
hijo ó por hija, traerá un cordero de un año para holocausto, y un
palomino ó una tórtola para expiación, á la puerta del tabernáculo del
testimonio, al sacerdote: \footnote{\textbf{12:6} Lev 5,7} \bibverse{7}
Y él ofrecerá delante de Jehová, y hará expiación por ella, y será
limpia del flujo de su sangre. Esta es la ley de la que pariere varón ó
hembra.

\bibverse{8} Y si no alcanzare su mano lo suficiente para un cordero,
tomará entonces dos tórtolas ó dos palominos, uno para holocausto, y
otro para expiación: y el sacerdote hará expiación por ella, y será
limpia.

\hypertarget{erupciuxf3n-y-manchas-en-personas-sobre-la-piel-desnuda}{%
\subsection{Erupción y manchas en personas sobre la piel
desnuda}\label{erupciuxf3n-y-manchas-en-personas-sobre-la-piel-desnuda}}

\hypertarget{section-12}{%
\section{13}\label{section-12}}

\bibverse{1} Y habló Jehová á Moisés y á Aarón, diciendo: \footnote{\textbf{13:1}
  Deut 24,8} \bibverse{2} Cuando el hombre tuviere en la piel de su
carne hinchazón, ó postilla, ó mancha blanca, y hubiere en la piel de su
carne como llaga de lepra, será traído á Aarón el sacerdote, ó á uno de
los sacerdotes sus hijos: \bibverse{3} Y el sacerdote mirará la llaga en
la piel de la carne: si el pelo en la llaga se ha vuelto blanco, y
pareciere la llaga más hundida que la tez de la carne, llaga de lepra
es; y el sacerdote le reconocerá, y le dará por inmundo. \bibverse{4} Y
si en la piel de su carne hubiere mancha blanca, pero no pareciere más
hundida que la tez, ni su pelo se hubiere vuelto blanco, entonces el
sacerdote encerrará al llagado por siete días; \bibverse{5} Y al séptimo
día el sacerdote lo mirará; y si la llaga á su parecer se hubiere
estancado, no habiéndose extendido en la piel, entonces el sacerdote le
volverá á encerrar por otros siete días. \bibverse{6} Y al séptimo día
el sacerdote le reconocerá de nuevo; y si parece haberse oscurecido la
llaga, y que no ha cundido en la piel, entonces el sacerdote lo dará por
limpio: era postilla; y lavará sus vestidos, y será limpio. \bibverse{7}
Mas si hubiere ido creciendo la postilla en la piel, después que fué
mostrado al sacerdote para ser limpio, será visto otra vez del
sacerdote: \bibverse{8} Y si reconociéndolo el sacerdote, ve que la
postilla ha crecido en la piel, el sacerdote lo dará por inmundo: es
lepra.

\hypertarget{lepra-obsoleta}{%
\subsection{Lepra obsoleta}\label{lepra-obsoleta}}

\bibverse{9} Cuando hubiere llaga de lepra en el hombre, será traído al
sacerdote; \bibverse{10} Y el sacerdote mirará, y si pareciere tumor
blanco en la piel, el cual haya mudado el color del pelo, y se descubre
asimismo la carne viva, \bibverse{11} Lepra es envejecida en la piel de
su carne; y le dará por inmundo el sacerdote, y no le encerrará, porque
es inmundo.

\bibverse{12} Mas si brotare la lepra cundiendo por el cutis, y ella
cubriere toda la piel del llagado desde su cabeza hasta sus pies, á toda
vista de ojos del sacerdote; \bibverse{13} Entonces el sacerdote le
reconocerá; y si la lepra hubiere cubierto toda su carne, dará por
limpio al llagado: hase vuelto toda ella blanca; y él es limpio.
\bibverse{14} Mas el día que apareciere en él la carne viva, será
inmundo. \bibverse{15} Y el sacerdote mirará la carne viva, y lo dará
por inmundo. Es inmunda la carne viva: es lepra. \bibverse{16} Mas
cuando la carne viva se mudare y volviere blanca, entonces vendrá al
sacerdote; \bibverse{17} Y el sacerdote mirará, y si la llaga se hubiere
vuelto blanca, el sacerdote dará por limpio al que tenía la llaga, y
será limpio.

\hypertarget{lepra-despuuxe9s-de-uxfalcera-previa}{%
\subsection{Lepra después de úlcera
previa}\label{lepra-despuuxe9s-de-uxfalcera-previa}}

\bibverse{18} Y cuando en la carne, en su piel, hubiere apostema, y se
sanare, \bibverse{19} Y sucediere en el lugar de la apostema tumor
blanco, ó mancha blanca embermejecida, será mostrado al sacerdote:
\bibverse{20} Y el sacerdote mirará; y si pareciere estar más baja que
su piel, y su pelo se hubiere vuelto blanco, darálo el sacerdote por
inmundo: es llaga de lepra que se originó en la apostema. \bibverse{21}
Y si el sacerdote la considerare, y no pareciere en ella pelo blanco, ni
estuviere más baja que la piel, sino oscura, entonces el sacerdote lo
encerrará por siete días: \bibverse{22} Y si se fuere extendiendo por la
piel, entonces el sacerdote lo dará por inmundo: es llaga. \bibverse{23}
Empero si la mancha blanca se estuviere en su lugar, que no haya
cundido, es la costra de la apostema; y el sacerdote lo dará por limpio.
\footnote{\textbf{13:23} Lev 13,28}

\hypertarget{lepra-en-una-quemadura}{%
\subsection{Lepra en una quemadura}\label{lepra-en-una-quemadura}}

\bibverse{24} Asimismo cuando la carne tuviere en su piel quemadura de
fuego, y hubiere en lo sanado del fuego mancha blanquecina, bermejiza ó
blanca, \bibverse{25} El sacerdote la mirará; y si el pelo se hubiere
vuelto blanco en la mancha, y pareciere estar más hundida que la piel,
es lepra que salió en la quemadura; y el sacerdote declarará al sujeto
inmundo, por ser llaga de lepra. \bibverse{26} Mas si el sacerdote la
mirare, y no pareciere en la mancha pelo blanco, ni estuviere más baja
que la tez, sino que está oscura, le encerrará el sacerdote por siete
días; \bibverse{27} Y al séptimo día el sacerdote la reconocerá: si se
hubiere ido extendiendo por la piel, el sacerdote lo dará por inmundo:
es llaga de lepra. \bibverse{28} Empero si la mancha se estuviere en su
lugar, y no se hubiere extendido en la piel, sino que está oscura,
hinchazón es de la quemadura: darálo el sacerdote por limpio; que señal
de la quemadura es.

\hypertarget{llaga-en-la-cabeza-y-en-la-barba}{%
\subsection{Llaga en la cabeza y en la
barba}\label{llaga-en-la-cabeza-y-en-la-barba}}

\bibverse{29} Y al hombre ó mujer que le saliere llaga en la cabeza, ó
en la barba, \bibverse{30} El sacerdote mirará la llaga; y si pareciere
estar más profunda que la tez, y el pelo en ella fuera rubio y
adelgazado, entonces el sacerdote lo dará por inmundo: es tiña, es lepra
de la cabeza ó de la barba. \bibverse{31} Mas cuando el sacerdote
hubiere mirado la llaga de la tiña, y no pareciere estar más profunda
que la tez, ni fuere en ella pelo negro, el sacerdote encerrará al
llagado de la tiña por siete días: \bibverse{32} Y al séptimo día el
sacerdote mirará la llaga: y si la tiña no pareciere haberse extendido,
ni hubiere en ella pelo rubio, ni pareciere la tiña más profunda que la
tez, \bibverse{33} Entonces lo trasquilarán, mas no trasquilarán el
lugar de la tiña: y encerrará el sacerdote al que tiene la tiña por
otros siete días. \bibverse{34} Y al séptimo día mirará el sacerdote la
tiña; y si la tiña no hubiere cundido en la piel, ni pareciere estar más
profunda que la tez, el sacerdote lo dará por limpio; y lavará sus
vestidos, y será limpio. \bibverse{35} Empero si la tiña se hubiere ido
extendiendo en la piel después de su purificación, \bibverse{36}
Entonces el sacerdote la mirará; y si la tiña hubiere cundido en la
piel, no busque el sacerdote el pelo rubio, es inmundo. \bibverse{37}
Mas si le pareciere que la tiña está detenida, y que ha salido en ella
el pelo negro, la tiña está sanada; él está limpio, y por limpio lo dará
el sacerdote.

\hypertarget{erupciuxf3n-segura-lepra-de-los-calvos}{%
\subsection{Erupción segura; Lepra de los
calvos}\label{erupciuxf3n-segura-lepra-de-los-calvos}}

\bibverse{38} Asimismo el hombre ó mujer, cuando en la piel de su carne
tuviere manchas, manchas blancas, \bibverse{39} El sacerdote mirará: y
si en la piel de su carne parecieren manchas blancas algo oscurecidas,
es empeine que brotó en la piel, está limpia la persona.

\bibverse{40} Y el hombre, cuando se le pelare la cabeza, es calvo, mas
limpio. \bibverse{41} Y si á la parte de su rostro se le pelare la
cabeza, es calvo por delante, pero limpio. \bibverse{42} Mas cuando en
la calva ó en la antecalva hubiere llaga blanca rojiza, lepra es que
brota en su calva ó en su antecalva. \bibverse{43} Entonces el sacerdote
lo mirará, y si pareciere la hinchazón de la llaga blanca rojiza en su
calva ó en su antecalva, como el parecer de la lepra de la tez de la
carne, \bibverse{44} Leproso es, es inmundo; el sacerdote lo dará luego
por inmundo; en su cabeza tiene su llaga.

\hypertarget{regulaciones-generales-para-leprosos}{%
\subsection{Regulaciones generales para
leprosos}\label{regulaciones-generales-para-leprosos}}

\bibverse{45} Y el leproso en quien hubiere llaga, sus vestidos serán
deshechos y su cabeza descubierta, y embozado pregonará: ¡Inmundo!
¡inmundo! \bibverse{46} Todo el tiempo que la llaga estuviere en él,
será inmundo; estará impuro: habitará solo; fuera del real será su
morada. \footnote{\textbf{13:46} Núm 5,3}

\hypertarget{lepra-en-cosas-y-cuero}{%
\subsection{Lepra en cosas y cuero}\label{lepra-en-cosas-y-cuero}}

\bibverse{47} Y cuando en el vestido hubiere plaga de lepra, en vestido
de lana, ó en vestido de lino; \bibverse{48} O en estambre ó en trama,
de lino ó de lana, ó en piel, ó en cualquiera obra de piel;
\bibverse{49} Y que la plaga sea verde, ó bermeja, en vestido ó en piel,
ó en estambre ó en trama, ó en cualquiera obra de piel; plaga es de
lepra, y se ha de mostrar al sacerdote. \bibverse{50} Y el sacerdote
mirará la plaga, y encerrará la cosa plagada por siete días.
\bibverse{51} Y al séptimo día mirará la plaga: y si hubiere cundido la
plaga en el vestido, ó estambre, ó en la trama, ó en piel, ó en
cualquiera obra que se hace de pieles, lepra roedora es la plaga;
inmunda será. \bibverse{52} Será quemado el vestido, ó estambre ó trama,
de lana ó de lino, ó cualquiera obra de pieles en que hubiere tal plaga;
porque lepra roedora es: al fuego será quemada.

\bibverse{53} Y si el sacerdote mirare, y no pareciere que la plaga se
haya extendido en el vestido, ó estambre, ó en la trama, ó en cualquiera
obra de pieles; \bibverse{54} Entonces el sacerdote mandará que laven
donde está la plaga, y lo encerrará otra vez por siete días.
\bibverse{55} Y el sacerdote mirará después que la plaga fuere lavada; y
si pareciere que la plaga no ha mudado su aspecto, bien que no haya
cundido la plaga, inmunda es; la quemarás al fuego; corrosión es
penetrante, esté lo raído en la haz ó en el revés de aquella cosa.
\bibverse{56} Mas si el sacerdote la viere, y pareciere que la plaga se
ha oscurecido después que fué lavada, la cortará del vestido, ó de la
piel, ó del estambre, ó de la trama. \bibverse{57} Y si apareciere más
en el vestido, ó estambre, ó trama, ó en cualquiera cosa de pieles,
reverdeciendo en ella, quemarás al fuego aquello donde estuviere la
plaga. \bibverse{58} Empero el vestido, ó estambre, ó trama, ó
cualquiera cosa de piel que lavares, y que se le quitare la plaga,
lavarse ha segunda vez, y entonces será limpia.

\bibverse{59} Esta es la ley de la plaga de la lepra del vestido de lana
ó de lino, ó del estambre, ó de la trama, ó de cualquiera cosa de piel,
para que sea dada por limpia ó por inmunda.

\hypertarget{la-purificaciuxf3n-de-los-leprosos}{%
\subsection{La purificación de los
leprosos}\label{la-purificaciuxf3n-de-los-leprosos}}

\hypertarget{section-13}{%
\section{14}\label{section-13}}

\bibverse{1} Y habló Jehová á Moisés, diciendo:

\bibverse{2} Esta será la ley del leproso cuando se limpiare: Será
traído al sacerdote: \bibverse{3} Y el sacerdote saldrá fuera del real;
y mirará el sacerdote, y viendo que está sana la plaga de la lepra del
leproso, \bibverse{4} El sacerdote mandará luego que se tomen para el
que se purifica dos avecillas vivas, limpias, y palo de cedro, y grana,
é hisopo; \bibverse{5} Y mandará el sacerdote matar la una avecilla en
un vaso de barro sobre aguas vivas; \bibverse{6} Después tomará la
avecilla viva, y el palo de cedro, y la grana, y el hisopo, y lo mojará
con la avecilla viva en la sangre de la avecilla muerta sobre las aguas
vivas: \bibverse{7} Y rociará siete veces sobre el que se purifica de la
lepra, y le dará por limpio; y soltará la avecilla viva sobre la haz del
campo. \footnote{\textbf{14:7} Lev 16,22}

\bibverse{8} Y el que se purifica lavará sus vestidos, y raerá todos sus
pelos, y se ha de lavar con agua, y será limpio: y después entrará en el
real, y morará fuera de su tienda siete días. \footnote{\textbf{14:8}
  Núm 8,7} \bibverse{9} Y será, que al séptimo día raerá todos sus
pelos, su cabeza, y su barba, y las cejas de sus ojos; finalmente, raerá
todo su pelo, y lavará sus vestidos, y lavará su carne en aguas, y será
limpio.

\hypertarget{los-sacrificios-y-costumbres-del-octavo-duxeda}{%
\subsection{Los sacrificios y costumbres del octavo
día}\label{los-sacrificios-y-costumbres-del-octavo-duxeda}}

\bibverse{10} Y el día octavo tomará dos corderos sin defecto, y una
cordera de un año sin tacha; y tres décimas de flor de harina para
presente, amasada con aceite, y un log de aceite. \bibverse{11} Y el
sacerdote que le purifica presentará con aquellas cosas al que se ha de
limpiar delante de Jehová, á la puerta del tabernáculo del testimonio:

\bibverse{12} Y tomará el sacerdote el un cordero, y ofrecerálo por la
culpa, con el log de aceite, y lo mecerá como ofrenda agitada delante de
Jehová: \bibverse{13} Y degollará el cordero en el lugar donde degüellan
la víctima por el pecado y el holocausto, en el lugar del santuario:
porque como la víctima por el pecado, así también la víctima por la
culpa es del sacerdote: es cosa muy sagrada. \footnote{\textbf{14:13}
  Lev 7,7} \bibverse{14} Y tomará el sacerdote de la sangre de la
víctima por la culpa, y pondrá el sacerdote sobre la ternilla de la
oreja derecha del que se purifica, y sobre el pulgar de su mano derecha,
y sobre el pulgar de su pie derecho. \footnote{\textbf{14:14} Lev 8,23}
\bibverse{15} Asimismo tomará el sacerdote del log de aceite, y echará
sobre la palma de su mano izquierda: \bibverse{16} Y mojará su dedo
derecho en el aceite que tiene en su mano izquierda, y esparcirá del
aceite con su dedo siete veces delante de Jehová: \footnote{\textbf{14:16}
  Lev 4,6; Lev 4,17} \bibverse{17} Y de lo que quedare del aceite que
tiene en su mano, pondrá el sacerdote sobre la ternilla de la oreja
derecha del que se purifica, y sobre el pulgar de su mano derecha, y
sobre el pulgar de su pie derecho, sobre la sangre de la expiación por
la culpa: \bibverse{18} Y lo que quedare del aceite que tiene en su
mano, pondrá sobre la cabeza del que se purifica: y hará el sacerdote
expiación por él delante de Jehová.

\bibverse{19} Ofrecerá luego el sacerdote el sacrificio por el pecado, y
hará expiación por el que se ha de purificar de su inmundicia, y después
degollará el holocausto: \bibverse{20} Y hará subir el sacerdote el
holocausto y el presente sobre el altar. Así hará el sacerdote expiación
por él, y será limpio.

\hypertarget{reemplazo-de-la-ofrenda-por-el-pecado-y-el-holocausto-por-los-pobres}{%
\subsection{Reemplazo de la ofrenda por el pecado y el holocausto por
los
pobres}\label{reemplazo-de-la-ofrenda-por-el-pecado-y-el-holocausto-por-los-pobres}}

\bibverse{21} Mas si fuere pobre, que no alcanzare su mano á tanto,
entonces tomará un cordero para ser ofrecido como ofrenda agitada por la
culpa, para reconciliarse, y una décima de flor de harina amasada con
aceite para presente, y un log de aceite; \bibverse{22} Y dos tórtolas,
ó dos palominos, lo que alcanzare su mano: y el uno será para expiación
por el pecado, y el otro para holocausto;

\bibverse{23} Las cuales cosas traerá al octavo día de su purificación
al sacerdote, á la puerta del tabernáculo del testimonio delante de
Jehová. \bibverse{24} Y el sacerdote tomará el cordero de la expiación
por la culpa, y el log de aceite, y mecerálo el sacerdote como ofrenda
agitada delante de Jehová; \bibverse{25} Luego degollará el cordero de
la culpa, y tomará el sacerdote de la sangre de la culpa, y pondrá sobre
la ternilla de la oreja derecha del que se purifica, y sobre el pulgar
de su mano derecha, y sobre el pulgar de su pie derecho. \bibverse{26} Y
el sacerdote echará del aceite sobre la palma de su mano izquierda;
\bibverse{27} Y con su dedo derecho rociará el sacerdote del aceite que
tiene en su mano izquierda, siete veces delante de Jehová. \bibverse{28}
También pondrá el sacerdote del aceite que tiene en su mano sobre la
ternilla de la oreja derecha del que se purifica, y sobre el pulgar de
su mano derecha, y sobre el pulgar de su pie derecho, en el lugar de la
sangre de la culpa. \bibverse{29} Y lo que sobrare del aceite que el
sacerdote tiene en su mano, pondrálo sobre la cabeza del que se
purifica, para reconciliarlo delante de Jehová. \bibverse{30} Asimismo
ofrecerá la una de las tórtolas, ó de los palominos, lo que alcanzare su
mano: \bibverse{31} El uno de lo que alcanzare su mano, en expiación por
el pecado, y el otro en holocausto, además del presente: y hará el
sacerdote expiación por el que se ha de purificar, delante de Jehová.

\bibverse{32} Esta es la ley del que hubiere tenido plaga de lepra, cuya
mano no alcanzare lo prescrito para purificarse.

\hypertarget{normas-relativas-a-la-lepra-en-las-casas}{%
\subsection{Normas relativas a la lepra en las
casas}\label{normas-relativas-a-la-lepra-en-las-casas}}

\bibverse{33} Y habló Jehová á Moisés y á Aarón, diciendo: \bibverse{34}
Cuando hubieres entrado en la tierra de Canaán, la cual yo os doy en
posesión, y pusiere yo plaga de lepra en alguna casa de la tierra de
vuestra posesión, \bibverse{35} Vendrá aquél cuya fuere la casa, y dará
aviso al sacerdote, diciendo: Como plaga ha aparecido en mi casa.
\footnote{\textbf{14:35} Lev 13,2} \bibverse{36} Entonces mandará el
sacerdote, y despejarán la casa antes que el sacerdote entre á mirar la
plaga, porque no sea contaminado todo lo que estuviere en la casa: y
después el sacerdote entrará á reconocer la casa: \bibverse{37} Y mirará
la plaga: y si se vieren manchas en las paredes de la casa, cavernillas
verdosas ó rojas, las cuales parecieren más hundidas que la pared,
\bibverse{38} El sacerdote saldrá de la casa á la puerta de ella, y
cerrará la casa por siete días. \bibverse{39} Y al séptimo día volverá
el sacerdote, y mirará: y si la plaga hubiere crecido en las paredes de
la casa, \bibverse{40} Entonces mandará el sacerdote, y arrancarán las
piedras en que estuviere la plaga, y las echarán fuera de la ciudad, en
lugar inmundo: \bibverse{41} Y hará descostrar la casa por dentro
alrededor, y derramarán el polvo que descostraren, fuera de la ciudad en
lugar inmundo: \bibverse{42} Y tomarán otras piedras, y las pondrán en
lugar de las piedras quitadas; y tomarán otro barro, y encostrarán la
casa.

\bibverse{43} Y si la plaga volviere á reverdecer en aquella casa,
después que hizo arrancar las piedras, y descostrar la casa, y después
que fué encostrada, \bibverse{44} Entonces el sacerdote entrará y
mirará; y si pareciere haberse extendido la plaga en la casa, lepra
roedora está en la casa: inmunda es. \bibverse{45} Derribará, por tanto,
la tal casa, sus piedras, y sus maderos, y toda la mezcla de la casa; y
lo sacará fuera de la ciudad á lugar inmundo.

\bibverse{46} Y cualquiera que entrare en aquella casa todos los días
que la mandó cerrar, será inmundo hasta la tarde. \footnote{\textbf{14:46}
  Lev 11,24} \bibverse{47} Y el que durmiere en aquella casa, lavará sus
vestidos; también el que comiere en la casa, lavará sus vestidos.

\bibverse{48} Mas si entrare el sacerdote y mirare, y viere que la plaga
no se ha extendido en la casa después que fué encostrada, el sacerdote
dará la casa por limpia, porque la plaga ha sanado. \bibverse{49}
Entonces tomará para limpiar la casa dos avecillas, y palo de cedro, y
grana, é hisopo: \bibverse{50} Y degollará la una avecilla en una vasija
de barro sobre aguas vivas: \bibverse{51} Y tomará el palo de cedro, y
el hisopo, y la grana, y la avecilla viva, y mojarálo en la sangre de la
avecilla muerta y en las aguas vivas, y rociará la casa siete veces:
\bibverse{52} Y purificará la casa con la sangre de la avecilla, y con
las aguas vivas, y con la avecilla viva, y el palo de cedro, y el
hisopo, y la grana: \bibverse{53} Luego soltará la avecilla viva fuera
de la ciudad sobre la haz del campo: así hará expiación por la casa, y
será limpia. \footnote{\textbf{14:53} Lev 14,7}

\hypertarget{graduaciuxf3n}{%
\subsection{Graduación}\label{graduaciuxf3n}}

\bibverse{54} Esta es la ley acerca de toda plaga de lepra, y de tiña;
\bibverse{55} Y de la lepra del vestido, y de la casa; \bibverse{56} Y
acerca de la hinchazón, y de la postilla, y de la mancha blanca:
\bibverse{57} Para enseñar cuándo es inmundo, y cuándo limpio. Aquesta
es la ley tocante á la lepra.

\hypertarget{inmundicia-de-los-hombres}{%
\subsection{Inmundicia de los hombres}\label{inmundicia-de-los-hombres}}

\hypertarget{section-14}{%
\section{15}\label{section-14}}

\bibverse{1} Y habló Jehová á Moisés y á Aarón, diciendo: \bibverse{2}
Hablad á los hijos de Israel, y decidles: Cualquier varón, cuando su
simiente manare de su carne, será inmundo. \bibverse{3} Y esta será su
inmundicia en su flujo; sea que su carne destiló por causa de su flujo,
ó que su carne se obstruyó á causa de su flujo, él será inmundo.

\bibverse{4} Toda cama en que se acostare el que tuviere flujo, será
inmunda; y toda cosa sobre que se sentare, inmunda será. \bibverse{5} Y
cualquiera que tocare á su cama, lavará sus vestidos; lavaráse también á
sí mismo con agua, y será inmundo hasta la tarde. \bibverse{6} Y el que
se sentare sobre aquello en que se hubiere sentado el que tiene flujo,
lavará sus vestidos, se lavará también á sí mismo con agua, y será
inmundo hasta la tarde.

\bibverse{7} Asimismo el que tocare la carne del que tiene flujo, lavará
sus vestidos, y á sí mismo se lavará con agua, y será inmundo hasta la
tarde.

\bibverse{8} Y si el que tiene flujo escupiere sobre el limpio, éste
lavará sus vestidos, y después de haberse lavado con agua, será inmundo
hasta la tarde.

\bibverse{9} Y todo aparejo sobre que cabalgare el que tuviere flujo,
será inmundo. \bibverse{10} Y cualquiera que tocare cualquiera cosa que
haya estado debajo de él, será inmundo hasta la tarde; y el que la
llevare, lavará sus vestidos, y después de lavarse con agua, será
inmundo hasta la tarde.

\bibverse{11} Y todo aquel á quien tocare el que tiene flujo, y no
lavare con agua sus manos, lavará sus vestidos, y á sí mismo se lavará
con agua, y será inmundo hasta la tarde.

\bibverse{12} Y la vasija de barro en que tocare el que tiene flujo,
será quebrada; y toda vasija de madera será lavada con agua. \footnote{\textbf{15:12}
  Lev 11,33}

\bibverse{13} Y cuando se hubiere limpiado de su flujo el que tiene
flujo, se ha de contar siete días desde su purificación, y lavará sus
vestidos, y lavará su carne en aguas vivas, y será limpio.

\bibverse{14} Y el octavo día tomará dos tórtolas, ó dos palominos, y
vendrá delante de Jehová á la puerta del tabernáculo del testimonio, y
los dará al sacerdote: \bibverse{15} Y harálos el sacerdote, el uno
ofrenda por el pecado, y el otro holocausto: y le purificará el
sacerdote de su flujo delante de Jehová.

\bibverse{16} Y el hombre, cuando de él saliere derramamiento de semen,
lavará en aguas toda su carne, y será inmundo hasta la tarde.
\footnote{\textbf{15:16} Lev 22,4} \bibverse{17} Y toda vestimenta, ó
toda piel sobre la cual hubiere el derramamiento del semen, lavaráse con
agua, y será inmunda hasta la tarde. \bibverse{18} Y la mujer con quien
el varón tuviera ayuntamiento de semen, ambos se lavarán con agua, y
serán inmundos hasta la tarde.

\hypertarget{condiciones-impuras-en-las-mujeres}{%
\subsection{Condiciones impuras en las
mujeres}\label{condiciones-impuras-en-las-mujeres}}

\bibverse{19} Y cuando la mujer tuviere flujo de sangre, y su flujo
fuere en su carne, siete días estará apartada; y cualquiera que tocare
en ella, será inmundo hasta la tarde.

\bibverse{20} Y todo aquello sobre que ella se acostare mientras su
separación, será inmundo: también todo aquello sobre que se sentare,
será inmundo. \bibverse{21} Y cualquiera que tocare á su cama, lavará
sus vestidos, y después de lavarse con agua, será inmundo hasta la
tarde. \bibverse{22} También cualquiera que tocare cualquier mueble
sobre que ella se hubiere sentado, lavará sus vestidos; lavaráse luego á
sí mismo con agua, y será inmundo hasta la tarde. \bibverse{23} Y si
estuviere sobre la cama, ó sobre la silla en que ella se hubiere
sentado, el que tocare en ella será inmundo hasta la tarde.

\bibverse{24} Y si alguno durmiere con ella, y su menstruo fuere sobre
él, será inmundo por siete días; y toda cama sobre que durmiere, será
inmunda.

\bibverse{25} Y la mujer, cuando siguiere el flujo de su sangre por
muchos días fuera del tiempo de su costumbre, ó cuando tuviere flujo de
sangre más de su costumbre; todo el tiempo del flujo de su inmundicia,
será inmunda como en los días de su costumbre. \bibverse{26} Toda cama
en que durmiere todo el tiempo de su flujo, le será como la cama de su
costumbre; y todo mueble sobre que se sentare, será inmundo, como la
inmundicia de su costumbre. \bibverse{27} Cualquiera que tocare en esas
cosas será inmundo; y lavará sus vestidos, y á sí mismo se lavará con
agua, y será inmundo hasta la tarde.

\bibverse{28} Y cuando fuere libre de su flujo, se ha de contar siete
días, y después será limpia. \bibverse{29} Y el octavo día tomará
consigo dos tórtolas, ó dos palominos, y los traerá al sacerdote, á la
puerta del tabernáculo del testimonio: \footnote{\textbf{15:29} Lev
  15,14} \bibverse{30} Y el sacerdote hará el uno ofrenda por el pecado,
y el otro holocausto; y la purificará el sacerdote delante de Jehová del
flujo de su inmundicia.

\hypertarget{palabras-de-cierre}{%
\subsection{Palabras de cierre}\label{palabras-de-cierre}}

\bibverse{31} Así apartaréis los hijos de Israel de sus inmundicias, á
fin de que no mueran por sus inmundicias, ensuciando mi tabernáculo que
está entre ellos.

\bibverse{32} Esta es la ley del que tiene flujo, y del que sale
derramamiento de semen, viniendo á ser inmundo á causa de ello;
\bibverse{33} Y de la que padece su costumbre, y acerca del que tuviere
flujo, sea varón ó hembra, y del hombre que durmiere con mujer inmunda.

\hypertarget{el-gran-duxeda-anual-de-expiaciuxf3n-los-preparativos}{%
\subsection{El gran día anual de expiación; Los
preparativos}\label{el-gran-duxeda-anual-de-expiaciuxf3n-los-preparativos}}

\hypertarget{section-15}{%
\section{16}\label{section-15}}

\bibverse{1} Y habló Jehová á Moisés, después que murieron los dos hijos
de Aarón, cuando se llegaron delante de Jehová, y murieron; \bibverse{2}
Y Jehová dijo á Moisés: Di á Aarón tu hermano, que no en todo tiempo
entre en el santuario del velo adentro, delante de la cubierta que está
sobre el arca, para que no muera: porque yo apareceré en la nube sobre
la cubierta. \footnote{\textbf{16:2} Éxod 26,33-34}

\bibverse{3} Con esto entrará Aarón en el santuario: con un becerro por
expiación, y un carnero en holocausto. \footnote{\textbf{16:3} Lev 4,3;
  Lev 1,10} \bibverse{4} La túnica santa de lino se vestirá, y sobre su
carne tendrá pañetes de lino, y ceñiráse el cinto de lino; y con la
mitra de lino se cubrirá: son las santas vestiduras: con ellas, después
de lavar su carne con agua, se ha de vestir. \footnote{\textbf{16:4}
  Éxod 28,39; Éxod 28,42-43} \bibverse{5} Y de la congregación de los
hijos de Israel tomará dos machos de cabrío para expiación, y un carnero
para holocausto.

\hypertarget{las-costumbres-de-la-expiaciuxf3n}{%
\subsection{Las costumbres de la
expiación}\label{las-costumbres-de-la-expiaciuxf3n}}

\bibverse{6} Y hará allegar Aarón el becerro de la expiación, que es
suyo, y hará la reconciliación por sí y por su casa. \bibverse{7}
Después tomará los dos machos de cabrío, y los presentará delante de
Jehová á la puerta del tabernáculo del testimonio. \bibverse{8} Y echará
suertes Aarón sobre los dos machos de cabrío; la una suerte por Jehová,
y la otra suerte por Azazel. \footnote{\textbf{16:8} Lev 16,20-22; Mat
  12,43} \bibverse{9} Y hará allegar Aarón el macho cabrío sobre el cual
cayere la suerte por Jehová, y ofrecerálo en expiación. \bibverse{10}
Mas el macho cabrío, sobre el cual cayere la suerte por Azazel, lo
presentará vivo delante de Jehová, para hacer la reconciliación sobre
él, para enviarlo á Azazel al desierto.

\hypertarget{purificando-el-sacerdocio}{%
\subsection{Purificando el sacerdocio}\label{purificando-el-sacerdocio}}

\bibverse{11} Y hará llegar Aarón el becerro que era suyo para
expiación, y hará la reconciliación por sí y por su casa, y degollará en
expiación el becerro que es suyo. \bibverse{12} Después tomará el
incensario lleno de brasas de fuego, del altar de delante de Jehová, y
sus puños llenos del perfume aromático molido, y meterálo del velo
adentro: \bibverse{13} Y pondrá el perfume sobre el fuego delante de
Jehová, y la nube del perfume cubrirá la cubierta que está sobre el
testimonio, y no morirá. \bibverse{14} Tomará luego de la sangre del
becerro, y rociará con su dedo hacia la cubierta al lado oriental: hacia
la cubierta esparcirá siete veces de aquella sangre con su dedo.

\hypertarget{purificaciuxf3n-del-santuario-y-altar-del-holocausto}{%
\subsection{Purificación del santuario y altar del
holocausto}\label{purificaciuxf3n-del-santuario-y-altar-del-holocausto}}

\bibverse{15} Después degollará en expiación el macho cabrío, que era
del pueblo, y meterá la sangre de él del velo adentro; y hará de su
sangre como hizo de la sangre del becerro, y esparcirá sobre la cubierta
y delante de la cubierta: \bibverse{16} Y limpiará el santuario, de las
inmundicias de los hijos de Israel, y de sus rebeliones, y de todos sus
pecados: de la misma manera hará también al tabernáculo del testimonio,
el cual reside entre ellos en medio de sus inmundicias. \footnote{\textbf{16:16}
  Lev 17,11} \bibverse{17} Y ningún hombre estará en el tabernáculo del
testimonio cuando él entrare á hacer la reconciliación en el santuario,
hasta que él salga, y haya hecho la reconciliación por sí, y por su
casa, y por toda la congregación de Israel.

\bibverse{18} Y saldrá al altar que está delante de Jehová, y lo
expiará; y tomará de la sangre del becerro, y de la sangre del macho
cabrío, y pondrá sobre los cuernos del altar alrededor. \bibverse{19} Y
esparcirá sobre él de la sangre con su dedo siete veces, y lo limpiará,
y lo santificará de las inmundicias de los hijos de Israel.

\hypertarget{purificaciuxf3n-de-la-comunidad-envuxedo-de-la-cabra-a-asazel}{%
\subsection{Purificación de la comunidad: envío de la cabra a
Asazel}\label{purificaciuxf3n-de-la-comunidad-envuxedo-de-la-cabra-a-asazel}}

\bibverse{20} Y cuando hubiere acabado de expiar el santuario, y el
tabernáculo del testimonio, y el altar, hará llegar el macho cabrío
vivo: \bibverse{21} Y pondrá Aarón ambas manos suyas sobre la cabeza del
macho cabrío vivo, y confesará sobre él todas las iniquidades de los
hijos de Israel, y todas sus rebeliones, y todos sus pecados,
poniéndolos así sobre la cabeza del macho cabrío, y lo enviará al
desierto por mano de un hombre destinado para esto. \bibverse{22} Y
aquel macho cabrío llevará sobre sí todas las iniquidades de ellos á
tierra inhabitada: y dejará ir el macho cabrío por el desierto.

\hypertarget{las-ofrendas-finales-especialmente-la-ofrenda-del-macho-cabruxedo-por-el-pecado}{%
\subsection{Las ofrendas finales, especialmente la ofrenda del macho
cabrío por el
pecado}\label{las-ofrendas-finales-especialmente-la-ofrenda-del-macho-cabruxedo-por-el-pecado}}

\bibverse{23} Después vendrá Aarón al tabernáculo del testimonio, y se
desnudará las vestimentas de lino, que había vestido para entrar en el
santuario, y pondrálas allí. \bibverse{24} Lavará luego su carne con
agua en el lugar del santuario, y después de ponerse sus vestidos
saldrá, y hará su holocausto, y el holocausto del pueblo, y hará la
reconciliación por sí y por el pueblo. \bibverse{25} Y quemará el sebo
de la expiación sobre el altar.

\bibverse{26} Y el que hubiere llevado el macho cabrío á Azazel, lavará
sus vestidos, lavará también con agua su carne, y después entrará en el
real. \bibverse{27} Y sacará fuera del real el becerro del pecado, y el
macho cabrío de la culpa, la sangre de los cuales fué metida para hacer
la expiación en el santuario; y quemarán en el fuego sus pellejos, y sus
carnes, y su estiércol. \footnote{\textbf{16:27} Lev 4,12; Lev 6,23;
  Ezeq 43,21; Heb 13,11} \bibverse{28} Y el que los quemare, lavará sus
vestidos, lavará también su carne con agua, y después entrará en el
real.

\hypertarget{determinaciuxf3n-del-duxeda-y-disposiciones-sobre-la-celebraciuxf3n-externa-del-festival}{%
\subsection{Determinación del día y disposiciones sobre la celebración
externa del
festival}\label{determinaciuxf3n-del-duxeda-y-disposiciones-sobre-la-celebraciuxf3n-externa-del-festival}}

\bibverse{29} Y esto tendréis por estatuto perpetuo: En el mes séptimo,
á los diez del mes, afligiréis vuestras almas, y ninguna obra haréis, ni
el natural ni el extranjero que peregrina entre vosotros: \bibverse{30}
Porque en este día se os reconciliará para limpiaros; y seréis limpios
de todos vuestros pecados delante de Jehová. \bibverse{31} Sábado de
reposo es para vosotros, y afligiréis vuestras almas, por estatuto
perpetuo. \bibverse{32} Y hará la reconciliación el sacerdote que fuere
ungido, y cuya mano hubiere sido llena para ser sacerdote en lugar de su
padre; y se vestirá las vestimentas de lino, las vestiduras sagradas:
\bibverse{33} Y expiará el santuario santo, y el tabernáculo del
testimonio; expiará también el altar, y á los sacerdotes, y á todo el
pueblo de la congregación.

\bibverse{34} Y esto tendréis por estatuto perpetuo, para expiar á los
hijos de Israel de todos sus pecados una vez en el año. Y Moisés lo hizo
como Jehová le mandó.

\hypertarget{normativa-sobre-sacrificio-de-animales-domuxe9sticos-consumo-de-sangre-etc.}{%
\subsection{Normativa sobre sacrificio de animales domésticos, consumo
de sangre,
etc.}\label{normativa-sobre-sacrificio-de-animales-domuxe9sticos-consumo-de-sangre-etc.}}

\hypertarget{section-16}{%
\section{17}\label{section-16}}

\bibverse{1} Y habló Jehová á Moisés, diciendo: \bibverse{2} Habla á
Aarón y á sus hijos, y á todos los hijos de Israel, y diles: Esto es lo
que ha mandado Jehová, diciendo:

\hypertarget{unidad-del-matadero-de-animales-de-sacrificio}{%
\subsection{Unidad del matadero de animales de
sacrificio}\label{unidad-del-matadero-de-animales-de-sacrificio}}

\bibverse{3} Cualquier varón de la casa de Israel que degollare buey, ó
cordero, ó cabra, en el real, ó fuera del real, \bibverse{4} Y no lo
trajere á la puerta del tabernáculo del testimonio, para ofrecer ofrenda
á Jehová delante del tabernáculo de Jehová, sangre será imputada al tal
varón: sangre derramó; cortado será el tal varón de entre su pueblo:
\footnote{\textbf{17:4} Is 66,3} \bibverse{5} A fin de que traigan los
hijos de Israel sus sacrificios, los que sacrifican sobre la haz del
campo, para que los traigan á Jehová á la puerta del tabernáculo del
testimonio al sacerdote, y sacrifiquen ellos sacrificios de paces á
Jehová. \bibverse{6} Y el sacerdote esparcirá la sangre sobre el altar
de Jehová, á la puerta del tabernáculo del testimonio, y quemará el sebo
en olor de suavidad á Jehová. \bibverse{7} Y nunca más sacrificarán sus
sacrificios á los demonios, tras de los cuales han fornicado: tendrán
esto por estatuto perpetuo por sus edades.

\hypertarget{unidad-del-lugar-del-sacrificio-en-el-tabernaculo-del-testimonio}{%
\subsection{Unidad del lugar del sacrificio en el tabernaculo del
testimonio}\label{unidad-del-lugar-del-sacrificio-en-el-tabernaculo-del-testimonio}}

\bibverse{8} Les dirás también: Cualquier varón de la casa de Israel, ó
de los extranjeros que peregrinan entre vosotros, que ofreciere
holocausto ó sacrificio, \bibverse{9} Y no lo trajere á la puerta del
tabernáculo del testimonio, para hacerlo á Jehová, el tal varón será
igualmente cortado de sus pueblos. \footnote{\textbf{17:9} Deut 12,14}

\hypertarget{prohibiciuxf3n-de-cualquier-consumo-de-sangre}{%
\subsection{Prohibición de cualquier consumo de
sangre}\label{prohibiciuxf3n-de-cualquier-consumo-de-sangre}}

\bibverse{10} Y cualquier varón de la casa de Israel, ó de los
extranjeros que peregrinan entre ellos, que comiere alguna sangre, yo
pondré mi rostro contra la persona que comiere sangre, y le cortaré de
entre su pueblo. \footnote{\textbf{17:10} Lev 3,17} \bibverse{11} Porque
la vida de la carne en la sangre está: y yo os la he dado para expiar
vuestras personas sobre el altar: por lo cual la misma sangre expiará la
persona. \footnote{\textbf{17:11} Heb 9,22} \bibverse{12} Por tanto, he
dicho á los hijos de Israel: Ninguna persona de vosotros comerá sangre,
ni el extranjero que peregrina entre vosotros comerá sangre.

\hypertarget{tratamiento-de-la-sangre-de-caza-y-la-carne-de-animales-cauxeddos-o-desgarrados}{%
\subsection{Tratamiento de la sangre de caza y la carne de animales
caídos o
desgarrados}\label{tratamiento-de-la-sangre-de-caza-y-la-carne-de-animales-cauxeddos-o-desgarrados}}

\bibverse{13} Y cualquier varón de los hijos de Israel, ó de los
extranjeros que peregrinan entre ellos, que cogiere caza de animal ó de
ave que sea de comer, derramará su sangre y cubrirála con tierra:
\bibverse{14} Porque el alma de toda carne, su vida, está en su sangre:
por tanto he dicho á los hijos de Israel: No comeréis la sangre de
ninguna carne, porque la vida de toda carne es su sangre: cualquiera que
la comiere será cortado.

\bibverse{15} Y cualquiera persona que comiere cosa mortecina, ó
despedazada por fiera, así de los naturales como de los extranjeros,
lavará sus vestidos y á sí mismo se lavará con agua, y será inmundo
hasta la tarde; y se limpiará. \footnote{\textbf{17:15} Lev 11,40}

\bibverse{16} Y si no los lavare, ni lavare su carne, llevará su
iniquidad.

\hypertarget{leyes-de-castidad-y-matrimonio}{%
\subsection{Leyes de castidad y
matrimonio}\label{leyes-de-castidad-y-matrimonio}}

\hypertarget{section-17}{%
\section{18}\label{section-17}}

\bibverse{1} Y habló Jehová á Moisés, diciendo: \bibverse{2} Habla á los
hijos de Israel, y diles: Yo soy Jehová vuestro Dios. \bibverse{3} No
haréis como hacen en la tierra de Egipto, en la cual morasteis; ni
haréis como hacen en la tierra de Canaán, á la cual yo os conduzco; ni
andaréis en sus estatutos. \bibverse{4} Mis derechos pondréis por obra,
y mis estatutos guardaréis, andando en ellos: Yo Jehová vuestro Dios.
\bibverse{5} Por tanto mis estatutos y mis derechos guardaréis, los
cuales haciendo el hombre, vivirá en ellos: Yo Jehová. \footnote{\textbf{18:5}
  Neh 9,29; Ezeq 20,11; Rom 10,5; Gal 3,12}

\hypertarget{prohibiciuxf3n-del-incesto-lista-de-matrimonios-prohibidos}{%
\subsection{Prohibición del incesto; Lista de matrimonios
prohibidos}\label{prohibiciuxf3n-del-incesto-lista-de-matrimonios-prohibidos}}

\bibverse{6} Ningún varón se allegue á ninguna cercana de su carne, para
descubrir su desnudez: Yo Jehová.

\bibverse{7} La desnudez de tu padre, ó la desnudez de tu madre, no
descubrirás: tu madre es, no descubrirás su desnudez.

\bibverse{8} La desnudez de la mujer de tu padre no descubrirás; es la
desnudez de tu padre.

\bibverse{9} La desnudez de tu hermana, hija de tu padre, ó hija de tu
madre, nacida en casa ó nacida fuera, su desnudez no descubrirás.
\footnote{\textbf{18:9} Deut 27,22}

\bibverse{10} La desnudez de la hija de tu hijo, ó de la hija de tu
hija, su desnudez no descubrirás, porque es la desnudez tuya.

\bibverse{11} La desnudez de la hija de la mujer de tu padre, engendrada
de tu padre, tu hermana es, su desnudez no descubrirás.

\bibverse{12} La desnudez de la hermana de tu padre no descubrirás: es
parienta de tu padre.

\bibverse{13} La desnudez de la hermana de tu madre no descubrirás:
porque parienta de tu madre es.

\bibverse{14} La desnudez del hermano de tu padre no descubrirás: no
llegarás á su mujer: es mujer del hermano de tu padre.

\bibverse{15} La desnudez de tu nuera no descubrirás: mujer es de tu
hijo; no descubrirás su desnudez.

\bibverse{16} La desnudez de la mujer de tu hermano no descubrirás: es
la desnudez de tu hermano. \footnote{\textbf{18:16} Mar 6,18}

\bibverse{17} La desnudez de la mujer y de su hija no descubrirás: no
tomarás la hija de su hijo, ni la hija de su hija, para descubrir su
desnudez: son parientas, es maldad. \footnote{\textbf{18:17} Deut 27,23}

\bibverse{18} No tomarás mujer juntamente con su hermana, para hacerla
su rival, descubriendo su desnudez delante de ella en su vida.

\hypertarget{advertencia-de-fornicar-pecados}{%
\subsection{Advertencia de fornicar
pecados}\label{advertencia-de-fornicar-pecados}}

\bibverse{19} Y no llegarás á la mujer en el apartamiento de su
inmundicia, para descubrir su desnudez. \footnote{\textbf{18:19} Lev
  15,24; Ezeq 18,6; Ezeq 22,10}

\bibverse{20} Además, no tendrás acto carnal con la mujer de tu prójimo,
contaminándote en ella. \footnote{\textbf{18:20} 2Sam 11,4}

\bibverse{21} Y no des de tu simiente para hacerla pasar por el fuego á
Moloch; no contamines el nombre de tu Dios: Yo Jehová. \footnote{\textbf{18:21}
  Deut 18,10; 2Re 21,6; Sal 106,37; Jer 7,31}

\bibverse{22} No te echarás con varón como con mujer: es abominación.
\footnote{\textbf{18:22} Gén 19,5; Rom 1,27; 1Cor 6,9}

\bibverse{23} Ni con ningún animal tendrás ayuntamiento amancillándote
con él; ni mujer alguna se pondrá delante de animal para ayuntarse con
él: es confusión. \footnote{\textbf{18:23} Éxod 22,18}

\bibverse{24} En ninguna de estas cosas os amancillaréis; porque en
todas estas cosas se han ensuciado las gentes que yo echo de delante de
vosotros: \bibverse{25} Y la tierra fué contaminada; y yo visité su
maldad sobre ella, y la tierra vomitó sus moradores. \bibverse{26}
Guardad, pues, vosotros mis estatutos y mis derechos, y no hagáis
ninguna de todas estas abominaciones: ni el natural ni el extranjero que
peregrina entre vosotros. \bibverse{27} (Porque todas estas
abominaciones hicieron los hombres de la tierra, que fueron antes de
vosotros, y la tierra fué contaminada:) \bibverse{28} Y la tierra no os
vomitará, por haberla contaminado, como vomitó á la gente que fué antes
de vosotros.

\bibverse{29} Porque cualquiera que hiciere alguna de todas estas
abominaciones, las personas que las hicieren, serán cortadas de entre su
pueblo. \bibverse{30} Guardad, pues, mi ordenanza, no haciendo de las
prácticas abominables que tuvieron lugar antes de vosotros, y no os
ensuciéis en ellas: Yo Jehová vuestro Dios.

\hypertarget{todo-tipo-de-reglamentos-basados-en-los-diez-mandamientos}{%
\subsection{Todo tipo de reglamentos basados \hspace{0pt}\hspace{0pt}en
los Diez
Mandamientos}\label{todo-tipo-de-reglamentos-basados-en-los-diez-mandamientos}}

\hypertarget{section-18}{%
\section{19}\label{section-18}}

\bibverse{1} Y habló Jehová á Moisés, diciendo: \bibverse{2} Habla á
toda la congregación de los hijos de Israel, y diles: Santos seréis,
porque santo soy yo Jehová vuestro Dios.

\bibverse{3} Cada uno temerá á su madre y á su padre, y mis sábados
guardaréis: Yo Jehová vuestro Dios. \footnote{\textbf{19:3} Éxod 20,8;
  Éxod 20,12}

\bibverse{4} No os volveréis á los ídolos, ni haréis para vosotros
dioses de fundición: Yo Jehová vuestro Dios. \footnote{\textbf{19:4}
  Éxod 20,3; Éxod 34,17}

\bibverse{5} Y cuando sacrificareis sacrificio de paces á Jehová, de
vuestra voluntad lo sacrificaréis. \footnote{\textbf{19:5} Lev 22,18-20}
\bibverse{6} Será comido el día que lo sacrificareis, y el siguiente
día: y lo que quedare para el tercer día, será quemado en el fuego.
\footnote{\textbf{19:6} Lev 7,15-18} \bibverse{7} Y si se comiere el día
tercero, será abominación; no será acepto: \bibverse{8} Y el que lo
comiere, llevará su delito, por cuanto profanó lo santo de Jehová; y la
tal persona será cortada de sus pueblos.

\bibverse{9} Cuando segareis la mies de vuestra tierra, no acabarás de
segar el rincón de tu haza, ni espigarás tu tierra segada. \footnote{\textbf{19:9}
  Lev 23,22; Deut 24,19; Rut 2,2; Rut 2,15-16} \bibverse{10} Y no
rebuscarás tu viña, ni recogerás los granos caídos de tu viña; para el
pobre y para el extranjero los dejarás: Yo Jehová vuestro Dios.

\bibverse{11} No hurtaréis, y no engañaréis, ni mentiréis ninguno á su
prójimo.

\bibverse{12} Y no juraréis en mi nombre con mentira, ni profanarás el
nombre de tu Dios: Yo Jehová. \footnote{\textbf{19:12} Éxod 20,7; Mat
  5,33}

\bibverse{13} No oprimirás á tu prójimo, ni le robarás. No se detendrá
el trabajo del jornalero en tu casa hasta la mañana. \footnote{\textbf{19:13}
  Deut 24,14-15; Jer 22,13; Sant 5,4}

\bibverse{14} No maldigas al sordo, y delante del ciego no pongas
tropiezo, mas tendrás temor de tu Dios: Yo Jehová. \footnote{\textbf{19:14}
  Deut 27,18}

\bibverse{15} No harás agravio en el juicio: no tendrás respeto al
pobre, ni honrarás la cara del grande: con justicia juzgarás á tu
prójimo. \footnote{\textbf{19:15} Éxod 23,6; Deut 16,19-20}

\bibverse{16} No andarás chismeando en tus pueblos. No te pondrás contra
la sangre de tu prójimo: Yo Jehová.

\bibverse{17} No aborrecerás á tu hermano en tu corazón: ingenuamente
reprenderás á tu prójimo, y no consentirás sobre él pecado. \footnote{\textbf{19:17}
  Sal 141,5; Mat 18,15}

\bibverse{18} No te vengarás, ni guardarás rencor á los hijos de tu
pueblo: mas amarás á tu prójimo como á ti mismo: Yo Jehová. \footnote{\textbf{19:18}
  Mat 22,39; Mat 5,43-48; Luc 10,25-37; Rom 13,9; Gal 5,14; Sant 2,8;
  Juan 13,34}

\hypertarget{varias-regulaciones-religiosas-y-morales-prohibiciones-de-costumbres-paganas}{%
\subsection{Varias regulaciones religiosas y morales, prohibiciones de
costumbres
paganas}\label{varias-regulaciones-religiosas-y-morales-prohibiciones-de-costumbres-paganas}}

\bibverse{19} Mis estatutos guardaréis. A tu animal no harás ayuntar
para misturas; tu haza no sembrarás con mistura de semillas, y no te
pondrás vestidos con mezcla de diversas cosas. \footnote{\textbf{19:19}
  Deut 22,9-11}

\bibverse{20} Y cuando un hombre tuviere cópula con mujer, y ella fuere
sierva desposada con alguno, y no estuviere rescatada, ni le hubiere
sido dada libertad, ambos serán azotados: no morirán, por cuanto ella no
es libre. \bibverse{21} Y él traerá á Jehová, á la puerta del
tabernáculo del testimonio, un carnero en expiación por su culpa.
\bibverse{22} Y con el carnero de la expiación lo reconciliará el
sacerdote delante de Jehová, por su pecado que cometió: y se le
perdonará su pecado que ha cometido.

\bibverse{23} Y cuando hubiereis entrado en la tierra, y plantareis todo
árbol de comer, quitaréis su prepucio, lo primero de su fruto: tres años
os será incircunciso: su fruto no se comerá. \bibverse{24} Y el cuarto
año todo su fruto será santidad de loores á Jehová. \bibverse{25} Mas al
quinto año comeréis el fruto de él, para que os haga crecer su fruto: Yo
Jehová vuestro Dios.

\bibverse{26} No comeréis cosa alguna con sangre. No seréis agoreros, ni
adivinaréis. \footnote{\textbf{19:26} Lev 3,17}

\bibverse{27} No cortaréis en redondo las extremidades de vuestras
cabezas, ni dañarás la punta de tu barba. \footnote{\textbf{19:27} Lev
  21,5; Deut 14,1}

\bibverse{28} Y no haréis rasguños en vuestra carne por un muerto, ni
imprimiréis en vosotros señal alguna: Yo Jehová.

\bibverse{29} No contaminarás tu hija haciéndola fornicar: porque no se
prostituya la tierra, y se hincha de maldad.

\bibverse{30} Mis sábados guardaréis, y mi santuario tendréis en
reverencia: Yo Jehová.

\bibverse{31} No os volváis á los encantadores y á los adivinos: no los
consultéis ensuciándoos con ellos: Yo Jehová vuestro Dios. \footnote{\textbf{19:31}
  Lev 20,6; Deut 18,10-11; 1Sam 28,7}

\hypertarget{varios-deberes-hacia-el-vecino}{%
\subsection{Varios deberes hacia el
vecino}\label{varios-deberes-hacia-el-vecino}}

\bibverse{32} Delante de las canas te levantarás, y honrarás el rostro
del anciano, y de tu Dios tendrás temor: Yo Jehová.

\bibverse{33} Y cuando el extranjero morare contigo en vuestra tierra,
no le oprimiréis. \bibverse{34} Como á un natural de vosotros tendréis
al extranjero que peregrinare entre vosotros; y ámalo como á ti mismo;
porque peregrinos fuisteis en la tierra de Egipto: Yo Jehová vuestro
Dios.

\bibverse{35} No hagáis agravio en juicio, en medida de tierra, ni en
peso, ni en otra medida. \footnote{\textbf{19:35} Deut 25,13-16; Prov
  11,1} \bibverse{36} Balanzas justas, pesas justas, epha justo, é hin
justo tendréis: Yo Jehová vuestro Dios, que os saqué de la tierra de
Egipto.

\bibverse{37} Guardad pues todos mis estatutos, y todos mis derechos, y
ponedlos por obra: Yo Jehová.

\hypertarget{disposiciones-penales-especialmente-para-las-delitos-prohibidos}{%
\subsection{Disposiciones penales especialmente para las delitos
prohibidos}\label{disposiciones-penales-especialmente-para-las-delitos-prohibidos}}

\hypertarget{section-19}{%
\section{20}\label{section-19}}

\bibverse{1} Y habló Jehová á Moisés diciendo: \bibverse{2} Dirás
asimismo á los hijos de Israel: Cualquier varón de los hijos de Israel,
ó de los extranjeros que peregrinan en Israel, que diere de su simiente
á Moloch, de seguro morirá: el pueblo de la tierra lo apedreará con
piedras. \bibverse{3} Y yo pondré mi rostro contra el tal varón, y lo
cortaré de entre su pueblo; por cuanto dió de su simiente á Moloch,
contaminando mi santuario, y amancillando mi santo nombre. \bibverse{4}
Que si escondiere el pueblo de la tierra sus ojos de aquel varón que
hubiere dado de su simiente á Moloch, para no matarle, \bibverse{5}
Entonces yo pondré mi rostro contra aquel varón, y contra su familia, y
le cortaré de entre su pueblo, con todos los que fornicaron en pos de
él, prostituyéndose con Moloch.

\bibverse{6} Y la persona que atendiere á encantadores ó adivinos, para
prostituirse tras de ellos, yo pondré mi rostro contra la tal persona, y
cortaréla de entre su pueblo. \footnote{\textbf{20:6} Lev 19,31}

\hypertarget{amonestaciuxf3n-para-santificar-castigos-por-diversos-pecados-y-delitos}{%
\subsection{Amonestación para santificar; Castigos por diversos pecados
y
delitos}\label{amonestaciuxf3n-para-santificar-castigos-por-diversos-pecados-y-delitos}}

\bibverse{7} Santificaos, pues, y sed santos, porque yo Jehová soy
vuestro Dios. \footnote{\textbf{20:7} Lev 19,2} \bibverse{8} Y guardad
mis estatutos, y ponedlos por obra: Yo Jehová que os santifico.
\footnote{\textbf{20:8} Lev 19,37}

\bibverse{9} Porque varón que maldijere á su padre ó á su madre, de
cierto morirá: á su padre ó á su madre maldijo; su sangre será sobre él.
\footnote{\textbf{20:9} Éxod 21,17}

\bibverse{10} Y el hombre que adulterare con la mujer de otro, el que
cometiere adulterio con la mujer de su prójimo, indefectiblemente se
hará morir al adúltero y á la adúltera. \footnote{\textbf{20:10} Éxod
  20,14; Juan 8,5}

\bibverse{11} Y cualquiera que se echare con la mujer de su padre, la
desnudez de su padre descubrió: ambos han de ser muertos; su sangre será
sobre ellos.

\bibverse{12} Y cualquiera que durmiere con su nuera, ambos han de
morir: hicieron confusión; su sangre será sobre ellos.

\bibverse{13} Y cualquiera que tuviere ayuntamiento con varón como con
mujer, abominación hicieron: entrambos han de ser muertos; sobre ellos
será su sangre.

\bibverse{14} Y el que tomare mujer y á la madre de ella, comete vileza:
quemarán en fuego á él y á ellas, porque no haya vileza entre vosotros.

\bibverse{15} Y cualquiera que tuviere cópula con bestia, ha de ser
muerto; y mataréis á la bestia.

\bibverse{16} Y la mujer que se allegare á algún animal, para tener
ayuntamiento con él, á la mujer y al animal matarás: morirán
infaliblemente; será su sangre sobre ellos.

\bibverse{17} Y cualquiera que tomare á su hermana, hija de su padre ó
hija de su madre, y viere su desnudez, y ella viere la suya, cosa es
execrable; por tanto serán muertos á ojos de los hijos de su pueblo:
descubrió la desnudez de su hermana; su pecado llevará.

\bibverse{18} Y cualquiera que durmiere con mujer menstruosa, y
descubriere su desnudez, su fuente descubrió, y ella descubrió la fuente
de su sangre: ambos serán cortados de entre su pueblo.

\bibverse{19} La desnudez de la hermana de tu madre, ó de la hermana de
tu padre, no descubrirás: por cuanto descubrió su parienta, su iniquidad
llevarán. \bibverse{20} Y cualquiera que durmiere con la mujer del
hermano de su padre, la desnudez del hermano de su padre descubrió; su
pecado llevarán; morirán sin hijos.

\bibverse{21} Y el que tomare la mujer de su hermano, es suciedad; la
desnudez de su hermano descubrió; sin hijos serán.

\hypertarget{exhortaciuxf3n-a-ser-santos-para-israel-como-pueblo-apartado-para-dios}{%
\subsection{Exhortación a ser santos para Israel como pueblo apartado
para
Dios}\label{exhortaciuxf3n-a-ser-santos-para-israel-como-pueblo-apartado-para-dios}}

\bibverse{22} Guardad, pues, todos mis estatutos y todos mis derechos, y
ponedlos por obra: y no os vomitará la tierra, en la cual yo os
introduzco para que habitéis en ella. \bibverse{23} Y no andéis en las
prácticas de la gente que yo echaré de delante de vosotros: porque ellos
hicieron todas estas cosas, y los tuve en abominación. \bibverse{24}
Empero á vosotros os he dicho: Vosotros poseeréis la tierra de ellos, y
yo os la daré para que la poseáis por heredad, tierra que fluye leche y
miel: Yo Jehová vuestro Dios, que os he apartado de los pueblos.

\bibverse{25} Por tanto, vosotros haréis diferencia entre animal limpio
é inmundo, y entre ave inmunda y limpia: y no ensuciéis vuestras
personas en los animales, ni en las aves, ni en ninguna cosa que va
arrastrando por la tierra, las cuales os he apartado por inmundas.
\bibverse{26} Habéis, pues, de serme santos, porque yo Jehová soy santo,
y os he apartado de los pueblos, para que seáis míos.

\bibverse{27} Y el hombre ó la mujer en quienes hubiere espíritu
pithónico ó de adivinación, han de ser muertos: los apedrearán con
piedras; su sangre sobre ellos. \footnote{\textbf{20:27} Éxod 22,17}

\hypertarget{ordenanzas-sobre-contaminaciuxf3n-por-muerte-costumbres-de-duelo-y-obstuxe1culos-al-matrimonio}{%
\subsection{Ordenanzas sobre contaminación por muerte, costumbres de
duelo y obstáculos al
matrimonio}\label{ordenanzas-sobre-contaminaciuxf3n-por-muerte-costumbres-de-duelo-y-obstuxe1culos-al-matrimonio}}

\hypertarget{section-20}{%
\section{21}\label{section-20}}

\bibverse{1} Y jehová dijo á Moisés: Habla á los sacerdotes hijos de
Aarón, y diles que no se contaminen por un muerto en sus pueblos.
\footnote{\textbf{21:1} Ezeq 44,20-25} \bibverse{2} Mas por su pariente
cercano á sí, por su madre, ó por su padre, ó por su hijo, ó por su
hermano, \bibverse{3} O por su hermana virgen, á él cercana, la cual no
haya tenido marido, por ella se contaminará. \bibverse{4} No se
contaminará, porque es príncipe en sus pueblos, haciéndose inmundo.

\bibverse{5} No harán calva en su cabeza, ni raerán la punta de su
barba, ni en su carne harán rasguños. \footnote{\textbf{21:5} Lev
  19,27-28} \bibverse{6} Santos serán á su Dios, y no profanarán el
nombre de su Dios; porque los fuegos de Jehová y el pan de su Dios
ofrecen: por tanto serán santos.

\bibverse{7} Mujer ramera ó infame no tomarán: ni tomarán mujer
repudiada de su marido: porque es santo á su Dios. \bibverse{8} Lo
santificarás por tanto, pues el pan de tu Dios ofrece: santo será para
ti, porque santo soy yo Jehová vuestro santificador.

\bibverse{9} Y la hija del varón sacerdote, si comenzare á fornicar, á
su padre amancilla: quemada será al fuego.

\bibverse{10} Y el sumo sacerdote entre sus hermanos, sobre cuya cabeza
fué derramado el aceite de la unción, y que hinchió su mano para vestir
las vestimentas, no descubrirá su cabeza, ni romperá sus vestidos:
\bibverse{11} Ni entrará donde haya alguna persona muerta, ni por su
padre, ó por su madre se contaminará. \bibverse{12} Ni saldrá del
santuario, ni contaminará el santuario de su Dios; porque la corona del
aceite de la unción de su Dios está sobre él: Yo Jehová.

\bibverse{13} Y tomará él mujer con su virginidad. \bibverse{14} Viuda,
ó repudiada, ó infame, ó ramera, éstas no tomará: mas tomará virgen de
sus pueblos por mujer. \bibverse{15} Y no amancillará su simiente en sus
pueblos; porque yo Jehová soy el que los santifico.

\hypertarget{los-defectos-fuxedsicos-excluyendo-el-sacerdocio}{%
\subsection{Los defectos físicos excluyendo el
sacerdocio}\label{los-defectos-fuxedsicos-excluyendo-el-sacerdocio}}

\bibverse{16} Y Jehová habló á Moisés, diciendo: \bibverse{17} Habla á
Aarón, y dile: El varón de tu simiente en sus generaciones, en el cual
hubiere falta, no se allegará para ofrecer el pan de su Dios.
\bibverse{18} Porque ningún varón en el cual hubiere falta, se allegará:
varón ciego, ó cojo, ó falto, ó sobrado, \bibverse{19} O varón en el
cual hubiere quebradura de pie ó rotura de mano, \bibverse{20} O
corcovado, ó lagañoso, ó que tuviere nube en el ojo, ó que tenga sarna,
ó empeine, ó compañón relajado; \bibverse{21} Ningún varón de la
simiente de Aarón sacerdote, en el cual hubiere falta, se allegará para
ofrecer las ofrendas encendidas de Jehová. Hay falta en él; no se
allegará á ofrecer el pan de su Dios. \bibverse{22} El pan de su Dios,
de lo muy santo y las cosas santificadas, comerá. \bibverse{23} Empero
no entrará del velo adentro, ni se allegará al altar, por cuanto hay
falta en él: y no profanará mi santuario, porque yo Jehová soy el que
los santifico.

\bibverse{24} Y Moisés habló esto á Aarón, y á sus hijos, y á todos los
hijos de Israel.

\hypertarget{comportamiento-hacia-los-dones-sagrados}{%
\subsection{Comportamiento hacia los dones
sagrados}\label{comportamiento-hacia-los-dones-sagrados}}

\hypertarget{section-21}{%
\section{22}\label{section-21}}

\bibverse{1} Y habló Jehová á Moisés, diciendo: \bibverse{2} Di á Aarón
y á sus hijos, que se abstengan de las santificaciones de los hijos de
Israel, y que no profanen mi santo nombre en lo que ellos me santifican:
Yo Jehová.

\bibverse{3} Diles: Todo varón de toda vuestra simiente en vuestras
generaciones que llegare á las cosas sagradas, que los hijos de Israel
consagran á Jehová, teniendo inmundicia sobre sí, de delante de mí será
cortada su alma: Yo Jehová.

\bibverse{4} Cualquier varón de la simiente de Aarón que fuere leproso,
ó padeciere flujo, no comerá de las cosas sagradas hasta que esté
limpio: y el que tocare cualquiera cosa inmunda de mortecino, ó el varón
del cual hubiere salido derramamiento de semen; \footnote{\textbf{22:4}
  Lev 15,2; Lev 15,16} \bibverse{5} O el varón que hubiere tocado
cualquier reptil, por el cual será inmundo, ú hombre por el cual venga á
ser inmundo, conforme á cualquiera inmundicia suya; \bibverse{6} La
persona que lo tocare, será inmunda hasta la tarde, y no comerá de las
cosas sagradas antes que haya lavado su carne con agua. \bibverse{7} Y
cuando el sol se pusiere, será limpio; y después comerá las cosas
sagradas, porque su pan es. \bibverse{8} Mortecino ni despedazado por
fiera no comerá, para contaminarse en ello: Yo Jehová. \footnote{\textbf{22:8}
  Éxod 22,30}

\bibverse{9} Guarden, pues, mi ordenanza, y no lleven pecado por ello,
no sea que así mueran cuando la profanaren: Yo Jehová que los santifico.

\bibverse{10} Ningún extraño comerá cosa sagrada; el huésped del
sacerdote, ni el jornalero, no comerá cosa sagrada. \bibverse{11} Mas el
sacerdote, cuando comprare persona de su dinero, ésta comerá de ella, y
el nacido en su casa: éstos comerán de su pan. \bibverse{12} Empero la
hija del sacerdote, cuando se casare con varón extraño, ella no comerá
de la ofrenda de las cosas sagradas. \bibverse{13} Pero si la hija del
sacerdote fuere viuda, ó repudiada, y no tuviere prole, y se hubiere
vuelto á la casa de su padre, como en su mocedad, comerá del pan de su
padre; mas ningún extraño coma de él.

\bibverse{14} Y el que por yerro comiere cosa sagrada, añadirá á ella su
quinto, y darálo al sacerdote con la cosa sagrada. \bibverse{15} No
profanarán, pues, las cosas santas de los hijos de Israel, las cuales
apartan para Jehová: \bibverse{16} Y no les harán llevar la iniquidad
del pecado, comiendo las cosas santas de ellos: porque yo Jehová soy el
que los santifico. \footnote{\textbf{22:16} Lev 22,9}

\hypertarget{impecabilidad-de-los-animales-sacrificados}{%
\subsection{Impecabilidad de los animales
sacrificados}\label{impecabilidad-de-los-animales-sacrificados}}

\bibverse{17} Y habló Jehová á Moisés, diciendo: \bibverse{18} Habla á
Aarón y á sus hijos, y á todos los hijos de Israel, y diles: Cualquier
varón de la casa de Israel, ó de los extranjeros en Israel, que
ofreciere su ofrenda por todos sus votos, y por todas sus voluntarias
oblaciones que ofrecieren á Jehová en holocausto; \bibverse{19} De
vuestra voluntad ofreceréis macho sin defecto de entre las vacas, de
entre los corderos, ó de entre las cabras. \bibverse{20} Ninguna cosa en
que haya falta ofreceréis, porque no será acepto por vosotros.
\bibverse{21} Asimismo, cuando alguno ofreciere sacrificio de paces á
Jehová para presentar voto, ú ofreciendo voluntariamente, sea de vacas ó
de ovejas, sin tacha será acepto; no ha de haber en él falta.
\bibverse{22} Ciego, ó perniquebrado, ó mutilado, ó verrugoso, ó
sarnoso, ó roñoso, no ofreceréis éstos á Jehová, ni de ellos pondréis
ofrenda encendida sobre el altar de Jehová. \bibverse{23} Buey ó carnero
que tenga de más ó de menos, podrás ofrecer por ofrenda voluntaria; mas
por voto no será acepto. \bibverse{24} Herido ó magullado, rompido ó
cortado, no ofreceréis á Jehová, ni en vuestra tierra lo haréis.
\bibverse{25} Y de mano de hijo de extranjero no ofreceréis el pan de
vuestro Dios de todas estas cosas; porque su corrupción está en ellas:
hay en ellas falta, no se os aceptarán.

\hypertarget{tres-reglas-de-sacrificio-muxe1s}{%
\subsection{Tres reglas de sacrificio
más}\label{tres-reglas-de-sacrificio-muxe1s}}

\bibverse{26} Y habló Jehová á Moisés, diciendo: \bibverse{27} El buey,
ó el cordero, ó la cabra, cuando naciere, siete días estará mamando de
su madre: mas desde el octavo día en adelante será acepto para ofrenda
de sacrificio encendido á Jehová. \footnote{\textbf{22:27} Éxod 22,29}
\bibverse{28} Y sea buey ó carnero, no degollaréis en un día á él y á su
hijo. \footnote{\textbf{22:28} Deut 22,6-7}

\bibverse{29} Y cuando sacrificareis sacrificio de hacimiento de gracias
á Jehová, de vuestra voluntad lo sacrificaréis. \bibverse{30} En el
mismo día se comerá; no dejaréis de él para otro día: Yo Jehová.
\footnote{\textbf{22:30} Lev 7,15}

\bibverse{31} Guardad pues mis mandamientos, y ejecutadlos: Yo Jehová.
\bibverse{32} Y no amancilléis mi santo nombre, y yo me santificaré en
medio de los hijos de Israel: Yo Jehová que os santifico; \bibverse{33}
Que os saqué de la tierra de Egipto, para ser vuestro Dios: Yo Jehová.

\hypertarget{leyes-para-la-celebraciuxf3n-del-suxe1bado-y-las-fiestas-del-auxf1o}{%
\subsection{Leyes para la celebración del sábado y las fiestas del
año}\label{leyes-para-la-celebraciuxf3n-del-suxe1bado-y-las-fiestas-del-auxf1o}}

\hypertarget{section-22}{%
\section{23}\label{section-22}}

\bibverse{1} Y habló Jehová á Moisés, diciendo: \bibverse{2} Habla á los
hijos de Israel, y diles: Las solemnidades de Jehová, las cuales
proclamaréis santas convocaciones, aquestas serán mis solemnidades.

\hypertarget{el-suxe1bado}{%
\subsection{El sábado}\label{el-suxe1bado}}

\bibverse{3} Seis días se trabajará, y el séptimo día sábado de reposo
será, convocación santa: ninguna obra haréis; sábado es de Jehová en
todas vuestras habitaciones. \footnote{\textbf{23:3} Éxod 20,8-11}

\hypertarget{la-pascua-y-la-solemnidad-de-los-uxe1zimos}{%
\subsection{La Pascua y la solemnidad de los
ázimos}\label{la-pascua-y-la-solemnidad-de-los-uxe1zimos}}

\bibverse{4} Estas son las solemnidades de Jehová, las convocaciones
santas, á las cuales convocaréis en sus tiempos. \footnote{\textbf{23:4}
  Éxod 23,14-19} \bibverse{5} En el mes primero, á los catorce del mes,
entre las dos tardes, pascua es de Jehová. \footnote{\textbf{23:5} Éxod
  12,-1} \bibverse{6} Y á los quince días de este mes es la solemnidad
de los ázimos á Jehová: siete días comeréis ázimos. \bibverse{7} El
primer día tendréis santa convocación: ninguna obra servil haréis.
\bibverse{8} Y ofreceréis á Jehová siete días ofrenda encendida: el
séptimo día será santa convocación; ninguna obra servil haréis.

\hypertarget{ofreciendo-de-los-primeros-frutos}{%
\subsection{Ofreciendo de los primeros
frutos}\label{ofreciendo-de-los-primeros-frutos}}

\bibverse{9} Y habló Jehová á Moisés, diciendo: \bibverse{10} Habla á
los hijos de Israel, y diles: Cuando hubiereis entrado en la tierra que
yo os doy, y segareis su mies, traeréis al sacerdote un omer por
primicia de los primeros frutos de vuestra siega; \bibverse{11} El cual
mecerá el omer delante de Jehová, para que seáis aceptos: el siguiente
día del sábado lo mecerá el sacerdote. \bibverse{12} Y el día que
ofrezcáis el omer, ofreceréis un cordero de un año, sin defecto, en
holocausto á Jehová. \bibverse{13} Y su presente será dos décimas de
flor de harina amasada con aceite, ofrenda encendida á Jehová en olor
suavísimo; y su libación de vino, la cuarta parte de un hin.
\bibverse{14} Y no comeréis pan, ni grano tostado, ni espiga fresca,
hasta este mismo día, hasta que hayáis ofrecido la ofrenda de vuestro
Dios; estatuto perpetuo es por vuestras edades en todas vuestras
habitaciones.

\hypertarget{la-fiesta-de-pentecostuxe9s}{%
\subsection{La fiesta de
Pentecostés}\label{la-fiesta-de-pentecostuxe9s}}

\bibverse{15} Y os habéis de contar desde el siguiente día del sábado,
desde el día en que ofrecisteis el omer de la ofrenda mecida; siete
semanas cumplidas serán: \footnote{\textbf{23:15} Éxod 23,16; Éxod
  34,22; Núm 28,26-31; Deut 16,9-12} \bibverse{16} Hasta el siguiente
día del sábado séptimo contaréis cincuenta días; entonces ofreceréis
nuevo presente á Jehová. \bibverse{17} De vuestras habitaciones traeréis
dos panes para ofrenda mecida, que serán de dos décimas de flor de
harina, cocidos con levadura, por primicias á Jehová. \bibverse{18} Y
ofreceréis con el pan siete corderos de un año sin defecto, y un becerro
de la vacada y dos carneros: serán holocausto á Jehová, con su presente
y sus libaciones; ofrenda encendida de suave olor á Jehová.
\bibverse{19} Ofreceréis además un macho de cabrío por expiación; y dos
corderos de un año en sacrificio de paces. \bibverse{20} Y el sacerdote
los mecerá en ofrenda agitada delante de Jehová, con el pan de las
primicias, y los dos corderos: serán cosa sagrada de Jehová para el
sacerdote. \bibverse{21} Y convocaréis en este mismo día; os será santa
convocación: ninguna obra servil haréis: estatuto perpetuo en todas
vuestras habitaciones por vuestras edades.

\bibverse{22} Y cuando segareis la mies de vuestra tierra, no acabarás
de segar el rincón de tu haza, ni espigarás tu siega; para el pobre, y
para el extranjero la dejarás: Yo Jehová vuestro Dios.

\hypertarget{la-celebracion-del-auxf1o-nuevo}{%
\subsection{La celebracion del año
nuevo}\label{la-celebracion-del-auxf1o-nuevo}}

\bibverse{23} Y habló Jehová á Moisés, diciendo: \bibverse{24} Habla á
los hijos de Israel, y diles: En el mes séptimo, al primero del mes
tendréis sábado, una conmemoración al son de trompetas, y una santa
convocación. \footnote{\textbf{23:24} Núm 29,1-6; Núm 10,10}
\bibverse{25} Ninguna obra servil haréis; y ofreceréis ofrenda encendida
á Jehová.

\hypertarget{der-grouxdfe-versuxf6hnungstag}{%
\subsection{Der große
Versöhnungstag}\label{der-grouxdfe-versuxf6hnungstag}}

\bibverse{26} Y habló Jehová á Moisés, diciendo: \bibverse{27} Empero á
los diez de este mes séptimo será el día de las expiaciones: tendréis
santa convocación, y afligiréis vuestras almas, y ofreceréis ofrenda
encendida á Jehová. \bibverse{28} Ninguna obra haréis en este mismo día;
porque es día de expiaciones, para reconciliaros delante de Jehová
vuestro Dios. \bibverse{29} Porque toda persona que no se afligiere en
este mismo día, será cortada de sus pueblos. \bibverse{30} Y cualquiera
persona que hiciere obra alguna en este mismo día, yo destruiré la tal
persona de entre su pueblo. \bibverse{31} Ninguna obra haréis: estatuto
perpetuo es por vuestras edades en todas vuestras habitaciones.
\bibverse{32} Sábado de reposo será á vosotros, y afligiréis vuestras
almas, comenzando á los nueve del mes en la tarde: de tarde á tarde
holgaréis vuestro sábado.

\hypertarget{la-fiesta-de-los-tabernuxe1culos}{%
\subsection{La fiesta de los
tabernáculos}\label{la-fiesta-de-los-tabernuxe1culos}}

\bibverse{33} Y habló Jehová á Moisés, diciendo: \bibverse{34} Habla á
los hijos de Israel, y diles: A los quince días de este mes séptimo será
la solemnidad de las cabañas á Jehová por siete días. \footnote{\textbf{23:34}
  Éxod 23,16; Éxod 34,22; Núm 29,12-39; Deut 16,13-15} \bibverse{35} El
primer día habrá santa convocación: ninguna obra servil haréis.
\bibverse{36} Siete días ofreceréis ofrenda encendida á Jehová: el
octavo día tendréis santa convocación, y ofreceréis ofrenda encendida á
Jehová: es fiesta: ninguna obra servil haréis.

\hypertarget{finalizaciuxf3n-del-calendario-de-festivales}{%
\subsection{Finalización del calendario de
festivales}\label{finalizaciuxf3n-del-calendario-de-festivales}}

\bibverse{37} Estas son las solemnidades de Jehová, á las que
convocaréis santas reuniones, para ofrecer ofrenda encendida á Jehová,
holocausto y presente, sacrificio y libaciones, cada cosa en su tiempo:
\bibverse{38} Además de los sábados de Jehová y además de vuestros
dones, y á más de todos vuestros votos, y además de todas vuestras
ofrendas voluntarias, que daréis á Jehová.

\hypertarget{disposiciones-posteriores-sobre-la-fiesta-de-los-tabernuxe1culos}{%
\subsection{Disposiciones posteriores sobre la fiesta de los
tabernáculos}\label{disposiciones-posteriores-sobre-la-fiesta-de-los-tabernuxe1culos}}

\bibverse{39} Empero á los quince del mes séptimo, cuando hubiereis
allegado el fruto de la tierra, haréis fiesta á Jehová por siete días:
el primer día será sábado; sábado será también el octavo día.
\bibverse{40} Y tomaréis el primer día gajos con fruto de árbol hermoso,
ramos de palmas, y ramas de árboles espesos, y sauces de los arroyos; y
os regocijaréis delante de Jehová vuestro Dios por siete días.
\footnote{\textbf{23:40} Neh 8,14-16} \bibverse{41} Y le haréis fiesta á
Jehová por siete días cada un año; será estatuto perpetuo por vuestras
edades; en el mes séptimo la haréis. \bibverse{42} En cabañas habitaréis
siete días: todo natural de Israel habitará en cabañas; \bibverse{43}
Para que sepan vuestros descendientes que en cabañas hice yo habitar á
los hijos de Israel, cuando los saqué de la tierra de Egipto: Yo Jehová
vuestro Dios.

\bibverse{44} Así habló Moisés á los hijos de Israel sobre las
solemnidades de Jehová.

\hypertarget{reglamento-sobre-la-preparaciuxf3n-del-candelero-santo-y-el-pan-de-la-proposiciuxf3n}{%
\subsection{Reglamento sobre la preparación del candelero santo y el pan
de la
proposición}\label{reglamento-sobre-la-preparaciuxf3n-del-candelero-santo-y-el-pan-de-la-proposiciuxf3n}}

\hypertarget{section-23}{%
\section{24}\label{section-23}}

\bibverse{1} Y habló Jehová á Moisés, diciendo: \bibverse{2} Manda á los
hijos de Israel que te traigan aceite de olivas claro, molido, para la
luminaria, para hacer arder las lámparas de continuo. \bibverse{3} Fuera
del velo del testimonio, en el tabernáculo del testimonio, las aderezará
Aarón desde la tarde hasta la mañana delante de Jehová, de continuo:
estatuto perpetuo por vuestras edades. \bibverse{4} Sobre el candelero
limpio pondrá siempre en orden las lámparas delante de Jehová.

\bibverse{5} Y tomarás flor de harina, y cocerás de ella doce tortas:
cada torta será de dos décimas. \bibverse{6} Y has de ponerlas en dos
órdenes, seis en cada orden, sobre la mesa limpia delante de Jehová.
\footnote{\textbf{24:6} Éxod 25,30} \bibverse{7} Pondrás también sobre
cada orden incienso limpio, y será para el pan por perfume, ofrenda
encendida á Jehová. \bibverse{8} Cada día de sábado lo pondrá de
continuo en orden delante de Jehová, de los hijos de Israel por pacto
sempiterno. \bibverse{9} Y será de Aarón y de sus hijos, los cuales lo
comerán en el lugar santo; porque es cosa muy santa para él, de las
ofrendas encendidas á Jehová, por fuero perpetuo.

\hypertarget{castigo-por-blasfemia-homicidio-y-lesiones-corporales}{%
\subsection{Castigo por blasfemia, homicidio y lesiones
corporales;}\label{castigo-por-blasfemia-homicidio-y-lesiones-corporales}}

\bibverse{10} En aquella sazón el hijo de una mujer Israelita, el cual
era hijo de un Egipcio, salió entre los hijos de Israel; y el hijo de la
Israelita y un hombre de Israel riñeron en el real: \bibverse{11} Y el
hijo de la mujer Israelita pronunció el Nombre, y maldijo: entonces le
llevaron á Moisés. Y su madre se llamaba Selomith, hija de Dribi, de la
tribu de Dan. \bibverse{12} Y pusiéronlo en la cárcel, hasta que les
fuese declarado por palabra de Jehová. \footnote{\textbf{24:12} Núm
  15,34} \bibverse{13} Y Jehová habló á Moisés, diciendo: \bibverse{14}
Saca al blasfemo fuera del real, y todos los que le oyeron pongan sus
manos sobre la cabeza de él, y apedréelo toda la congregación.
\bibverse{15} Y á los hijos de Israel hablarás, diciendo: Cualquiera que
maldijere á su Dios, llevará su iniquidad. \bibverse{16} Y el que
blasfemare el nombre de Jehová, ha de ser muerto: toda la congregación
lo apedreará: así el extranjero como el natural, si blasfemare el
Nombre, que muera.

\bibverse{17} Asimismo el hombre que hiere de muerte á cualquiera
persona, que sufra la muerte. \footnote{\textbf{24:17} Éxod 21,12}
\bibverse{18} Y el que hiere á algún animal, ha de restituirlo: animal
por animal. \bibverse{19} Y el que causare lesión en su prójimo, según
hizo, así le sea hecho: \bibverse{20} Rotura por rotura, ojo por ojo,
diente por diente: según la lesión que habrá hecho á otro, tal se hará á
él. \bibverse{21} El que hiere algún animal, ha de restituirlo; mas el
que hiere de muerte á un hombre, que muera. \bibverse{22} Un mismo
derecho tendréis: como el extranjero, así será el natural: porque yo soy
Jehová vuestro Dios. \footnote{\textbf{24:22} Lev 19,34; Éxod 12,49}

\bibverse{23} Y habló Moisés á los hijos de Israel, y ellos sacaron al
blasfemo fuera del real, y apedreáronlo con piedras. Y los hijos de
Israel hicieron según que Jehová había mandado á Moisés. \footnote{\textbf{24:23}
  Núm 15,36}

\hypertarget{el-auxf1o-sabuxe1tico}{%
\subsection{El año sabático}\label{el-auxf1o-sabuxe1tico}}

\hypertarget{section-24}{%
\section{25}\label{section-24}}

\bibverse{1} Y jehová habló á Moisés en el monte de Sinaí, diciendo:
\bibverse{2} Habla á los hijos de Israel, y diles: Cuando hubiereis
entrado en la tierra que yo os doy, la tierra hará sábado á Jehová.
\bibverse{3} Seis años sembrarás tu tierra, y seis años podarás tu viña,
y cogerás sus frutos; \footnote{\textbf{25:3} Éxod 23,10-11; Deut
  15,1-11} \bibverse{4} Y el séptimo año la tierra tendrá sábado de
holganza, sábado á Jehová: no sembrarás tu tierra, ni podarás tu viña.
\bibverse{5} Lo que de suyo se naciere en tu tierra segada, no lo
segarás; y las uvas de tu viñedo no vendimiarás: año de holganza será á
la tierra. \bibverse{6} Mas el sábado de la tierra os será para comer á
ti, y á tu siervo, y á tu sierva, y á tu criado, y á tu extranjero que
morare contigo: \bibverse{7} Y á tu animal, y á la bestia que hubiere en
tu tierra, será todo el fruto de ella para comer.

\hypertarget{el-auxf1o-de-jubileo}{%
\subsection{El año de jubileo}\label{el-auxf1o-de-jubileo}}

\bibverse{8} Y te has de contar siete semanas de años, siete veces siete
años; de modo que los días de las siete semanas de años vendrán á serte
cuarenta y nueve años. \bibverse{9} Entonces harás pasar la trompeta de
jubilación en el mes séptimo á los diez del mes; el día de la expiación
haréis pasar la trompeta por toda vuestra tierra. \bibverse{10} Y
santificaréis el año cincuenta, y pregonaréis libertad en la tierra á
todos sus moradores: éste os será jubileo; y volveréis cada uno á su
posesión, y cada cual volverá á su familia. \footnote{\textbf{25:10} Is
  61,2; Luc 4,19} \bibverse{11} El año de los cincuenta años os será
jubileo: no sembraréis, ni segaréis lo que naciere de suyo en la tierra,
ni vendimiaréis sus viñedos: \bibverse{12} Porque es jubileo: santo será
á vosotros; el producto de la tierra comeréis.

\bibverse{13} En este año de jubileo volveréis cada uno á su posesión.

\bibverse{14} Y cuando vendiereis algo á vuestro prójimo, ó comprareis
de mano de vuestro prójimo, no engañe ninguno á su hermano:
\bibverse{15} Conforme al número de los años después del jubileo
comprarás de tu prójimo; conforme al número de los años de los frutos te
venderá él á ti. \bibverse{16} Conforme á la multitud de los años
aumentarás el precio, y conforme á la disminución de los años
disminuirás el precio; porque según el número de los rendimientos te ha
de vender él. \bibverse{17} Y no engañe ninguno á su prójimo; mas
tendrás temor de tu Dios: porque yo soy Jehová vuestro Dios.

\bibverse{18} Ejecutad, pues, mis estatutos, y guardad mis derechos, y
ponedlos por obra, y habitaréis en la tierra seguros; \footnote{\textbf{25:18}
  Lev 26,5; 1Re 5,5} \bibverse{19} Y la tierra dará su fruto, y comeréis
hasta hartura, y habitaréis en ella con seguridad. \bibverse{20} Y si
dijereis: ¿Qué comeremos el séptimo año? he aquí no hemos de sembrar, ni
hemos de coger nuestros frutos: \bibverse{21} Entonces yo os enviaré mi
bendición el sexto año, y hará fruto por tres años. \bibverse{22} Y
sembraréis el año octavo, y comeréis del fruto añejo; hasta el año
noveno, hasta que venga su fruto comeréis del añejo.

\bibverse{23} Y la tierra no se venderá rematadamente, porque la tierra
mía es; que vosotros peregrinos y extranjeros sois para conmigo.
\footnote{\textbf{25:23} Sal 39,13} \bibverse{24} Por tanto, en toda la
tierra de vuestra posesión, otorgaréis redención á la tierra.

\hypertarget{redenciuxf3n-o-recauxedda-de-la-propiedad-de-la-tierra-y-la-vivienda}{%
\subsection{Redención o recaída de la propiedad de la tierra y la
vivienda}\label{redenciuxf3n-o-recauxedda-de-la-propiedad-de-la-tierra-y-la-vivienda}}

\bibverse{25} Cuando tu hermano empobreciere, y vendiere algo de su
posesión, vendrá el rescatador, su cercano, y rescatará lo que su
hermano hubiere vendido. \bibverse{26} Y cuando el hombre no tuviere
rescatador, si alcanzare su mano, y hallare lo que basta para su
rescate; \bibverse{27} Entonces contará los años de su venta, y pagará
lo que quedare al varón á quien vendió, y volverá á su posesión.
\bibverse{28} Mas si no alcanzare su mano lo que basta para que vuelva á
él, lo que vendió estará en poder del que lo compró hasta el año del
jubileo; y al jubileo saldrá, y él volverá á su posesión.

\bibverse{29} Y el varón que vendiere casa de morada en ciudad cercada,
tendrá facultad de redimirla hasta acabarse el año de su venta: un año
será el término de poderse redimir. \bibverse{30} Y si no fuere redimida
dentro de un año entero, la casa que estuviere en la ciudad murada
quedará para siempre por de aquel que la compró, y para sus
descendientes: no saldrá en el jubileo. \bibverse{31} Mas las casas de
las aldeas que no tienen muro alrededor, serán estimadas como una haza
de tierra: tendrán redención, y saldrán en el jubileo.

\bibverse{32} Pero en cuanto á las ciudades de los Levitas, siempre
podrán redimir los Levitas las casas de las ciudades que poseyeren.
\footnote{\textbf{25:32} Núm 35,-1} \bibverse{33} Y el que comprare de
los Levitas, saldrá de la casa vendida, ó de la ciudad de su posesión,
en el jubileo: por cuanto las casas de las ciudades de los Levitas es la
posesión de ellos entre los hijos de Israel. \bibverse{34} Mas la tierra
del ejido de sus ciudades no se venderá, porque es perpetua posesión de
ellos.

\hypertarget{mandamiento-de-ayudar-a-los-israelitas-empobrecidos-compra-de-esclavos-hebreos-o-su-liberaciuxf3n-en-el-auxf1o-de-jubileo}{%
\subsection{Mandamiento de ayudar a los israelitas empobrecidos; Compra
de esclavos hebreos o su liberación en el año de
jubileo}\label{mandamiento-de-ayudar-a-los-israelitas-empobrecidos-compra-de-esclavos-hebreos-o-su-liberaciuxf3n-en-el-auxf1o-de-jubileo}}

\bibverse{35} Y cuando tu hermano empobreciere, y se acogiere á ti, tú
lo ampararás: como peregrino y extranjero vivirá contigo. \bibverse{36}
No tomarás usura de él, ni aumento; mas tendrás temor de tu Dios, y tu
hermano vivirá contigo. \bibverse{37} No le darás tu dinero á usura, ni
tu vitualla á ganancia: \bibverse{38} Yo Jehová vuestro Dios, que os
saqué de la tierra de Egipto, para daros la tierra de Canaán, para ser
vuestro Dios.

\bibverse{39} Y cuando tu hermano empobreciere, estando contigo, y se
vendiere á ti, no le harás servir como siervo: \footnote{\textbf{25:39}
  Éxod 21,2} \bibverse{40} Como criado, como extranjero estará contigo;
hasta el año del jubileo te servirá. \bibverse{41} Entonces saldrá de
contigo, él y sus hijos consigo, y volverá á su familia, y á la posesión
de sus padres se restituirá. \bibverse{42} Porque son mis siervos, los
cuales saqué yo de la tierra de Egipto: no serán vendidos á manera de
siervos. \bibverse{43} No te enseñorearás de él con dureza, mas tendrás
temor de tu Dios.

\bibverse{44} Así tu siervo como tu sierva que tuvieres, serán de las
gentes que están en vuestro alrededor: de ellos compraréis siervos y
siervas. \bibverse{45} También compraréis de los hijos de los forasteros
que viven entre vosotros, y de los que del linaje de ellos son nacidos
en vuestra tierra, que están con vosotros; los cuales tendréis por
posesión: \bibverse{46} Y los poseeréis por juro de heredad para
vuestros hijos después de vosotros, como posesión hereditaria; para
siempre os serviréis de ellos; empero en vuestros hermanos los hijos de
Israel, no os enseñorearéis cada uno sobre su hermano con dureza.

\bibverse{47} Y si el peregrino ó extranjero que está contigo,
adquiriese medios, y tu hermano que está con él empobreciere, y se
vendiere al peregrino ó extranjero que está contigo, ó á la raza de la
familia del extranjero; \bibverse{48} Después que se hubiere vendido,
podrá ser rescatado: uno de sus hermanos lo rescatará; \bibverse{49} O
su tío, ó el hijo de su tío lo rescatará, ó el cercano de su carne, de
su linaje, lo rescatará; ó si sus medios alcanzaren, él mismo se
redimirá. \bibverse{50} Y contará con el que lo compró, desde el año que
se vendió á él hasta el año del jubileo: y ha de apreciarse el dinero de
su venta conforme al número de los años, y se hará con él conforme al
tiempo de un criado asalariado. \bibverse{51} Si aun fueren muchos años,
conforme á ellos volverá para su rescate del dinero por el cual se
vendió. \bibverse{52} Y si quedare poco tiempo hasta el año del jubileo,
entonces contará con él, y devolverá su rescate conforme á sus años.
\bibverse{53} Como con tomado á salario anualmente hará con él: no se
enseñoreará en él con aspereza delante de tus ojos. \footnote{\textbf{25:53}
  Lev 25,43}

\bibverse{54} Mas si no se redimiere en esos años, en el año del jubileo
saldrá, él, y sus hijos con él. \bibverse{55} Porque mis siervos son los
hijos de Israel; son siervos míos, á los cuales saqué de la tierra de
Egipto: Yo Jehová vuestro Dios.

\hypertarget{bendiciones-y-maldiciones-dadas-por-dios-como-opciuxf3n.}{%
\subsection{Bendiciones y maldiciones dadas por Dios como
opción.}\label{bendiciones-y-maldiciones-dadas-por-dios-como-opciuxf3n.}}

\hypertarget{section-25}{%
\section{26}\label{section-25}}

\bibverse{1} No haréis para vosotros ídolos, ni escultura, ni os
levantaréis estatua, ni pondréis en vuestra tierra piedra pintada para
inclinaros á ella: porque yo soy Jehová vuestro Dios.

\bibverse{2} Guardad mis sábados, y tened en reverencia mi santuario: Yo
Jehová. \footnote{\textbf{26:2} Éxod 20,8}

\hypertarget{promesas-de-bendiciones-en-caso-de-obediencia}{%
\subsection{Promesas de bendiciones en caso de
obediencia}\label{promesas-de-bendiciones-en-caso-de-obediencia}}

\bibverse{3} Si anduviereis en mis decretos, y guardareis mis
mandamientos, y los pusiereis por obra; \bibverse{4} Yo daré vuestra
lluvia en su tiempo, y la tierra rendirá sus producciones, y el árbol
del campo dará su fruto; \bibverse{5} Y la trilla os alcanzará á la
vendimia, y la vendimia alcanzará á la sementera, y comeréis vuestro pan
en hartura, y habitaréis seguros en vuestra tierra: \footnote{\textbf{26:5}
  Am 9,13}

\bibverse{6} Y yo daré paz en la tierra, y dormiréis, y no habrá quien
os espante: y haré quitar las malas bestias de vuestra tierra, y no
pasará por vuestro país la espada: \footnote{\textbf{26:6} Job 11,19}
\bibverse{7} Y perseguiréis á vuestros enemigos, y caerán á cuchillo
delante de vosotros: \bibverse{8} Y cinco de vosotros perseguirán á
ciento, y ciento de vosotros perseguirán á diez mil, y vuestros enemigos
caerán á cuchillo delante de vosotros. \footnote{\textbf{26:8} Deut
  32,30}

\bibverse{9} Porque yo me volveré á vosotros, y os haré crecer, y os
multiplicaré, y afirmaré mi pacto con vosotros: \bibverse{10} Y comeréis
lo añejo de mucho tiempo, y sacaréis fuera lo añejo á causa de lo nuevo:
\bibverse{11} Y pondré mi morada en medio de vosotros, y mi alma no os
abominará: \bibverse{12} Y andaré entre vosotros, y yo seré vuestro
Dios, y vosotros seréis mi pueblo. \bibverse{13} Yo Jehová vuestro Dios,
que os saqué de la tierra de Egipto, para que no fueseis sus siervos; y
rompí las coyundas de vuestro yugo, y os he hecho andar el rostro alto.

\hypertarget{tuxf6dliche-krankheiten-und-schimpfliche-niederlagen}{%
\subsection{Tödliche Krankheiten und schimpfliche
Niederlagen}\label{tuxf6dliche-krankheiten-und-schimpfliche-niederlagen}}

\bibverse{14} Empero si no me oyereis, ni hiciereis todos estos mis
mandamientos, \bibverse{15} Y si abominareis mis decretos, y vuestra
alma menospreciare mis derechos, no ejecutando todos mis mandamientos, é
invalidando mi pacto; \bibverse{16} Yo también haré con vosotros esto:
enviaré sobre vosotros terror, extenuación y calentura, que consuman los
ojos y atormenten el alma: y sembraréis en balde vuestra simiente,
porque vuestros enemigos la comerán: \bibverse{17} Y pondré mi ira sobre
vosotros, y seréis heridos delante de vuestros enemigos; y los que os
aborrecen se enseñorearán de vosotros, y huiréis sin que haya quien os
persiga.

\hypertarget{sequuxeda-y-mal-crecimiento}{%
\subsection{Sequía y mal
crecimiento}\label{sequuxeda-y-mal-crecimiento}}

\bibverse{18} Y si aun con esas cosas no me oyereis, yo tornaré á
castigaros siete veces más por vuestros pecados. \bibverse{19} Y
quebrantaré la soberbia de vuestra fortaleza, y tornaré vuestro cielo
como hierro, y vuestra tierra como metal: \footnote{\textbf{26:19} Deut
  11,17; 1Re 17,1} \bibverse{20} Y vuestra fuerza se consumirá en vano;
que vuestra tierra no dará su esquilmo, y los árboles de la tierra no
darán su fruto.

\hypertarget{animales-salvajes}{%
\subsection{Animales salvajes}\label{animales-salvajes}}

\bibverse{21} Y si anduviereis conmigo en oposición, y no me quisiereis
oir, yo añadiré sobre vosotros siete veces más plagas según vuestros
pecados. \bibverse{22} Enviaré también contra vosotros bestias fieras
que os arrebaten los hijos, y destruyan vuestros animales, y os apoquen,
y vuestros caminos sean desiertos.

\hypertarget{angustia-de-guerra-combinada-con-peste-y-hambre}{%
\subsection{Angustia de guerra combinada con peste y
hambre}\label{angustia-de-guerra-combinada-con-peste-y-hambre}}

\bibverse{23} Y si con estas cosas no fuereis corregidos, sino que
anduviereis conmigo en oposición, \bibverse{24} Yo también procederé con
vosotros en oposición, y os heriré aun siete veces por vuestros pecados:
\footnote{\textbf{26:24} 2Sam 22,27} \bibverse{25} Y traeré sobre
vosotros espada vengadora, en vindicación del pacto; y os recogeréis á
vuestras ciudades; mas yo enviaré pestilencia entre vosotros, y seréis
entregados en mano del enemigo. \footnote{\textbf{26:25} Is 1,20}
\bibverse{26} Cuando yo os quebrantare el arrimo del pan, cocerán diez
mujeres vuestro pan en un horno, y os devolverán vuestro pan por peso; y
comeréis, y no os hartaréis.

\hypertarget{el-sufrimiento-del-pueblo-durante-el-exilio-en-los-pauxedses-de-sus-enemigos.}{%
\subsection{El sufrimiento del pueblo durante el exilio en los países de
sus
enemigos.}\label{el-sufrimiento-del-pueblo-durante-el-exilio-en-los-pauxedses-de-sus-enemigos.}}

\bibverse{27} Y si con esto no me oyereis, mas procediereis conmigo en
oposición, \bibverse{28} Yo procederé con vosotros en contra y con ira,
y os castigaré aun siete veces por vuestros pecados. \bibverse{29} Y
comeréis las carnes de vuestros hijos, y comeréis las carnes de vuestras
hijas: \footnote{\textbf{26:29} 2Re 6,28} \bibverse{30} Y destruiré
vuestros altos, y talaré vuestras imágenes, y pondré vuestros cuerpos
muertos sobre los cuerpos muertos de vuestros ídolos, y mi alma os
abominará: \bibverse{31} Y pondré vuestras ciudades en desierto, y
asolaré vuestros santuarios, y no oleré la fragancia de vuestro suave
perfume. \bibverse{32} Yo asolaré también la tierra, y se pasmarán de
ella vuestros enemigos que en ella moran: \bibverse{33} Y á vosotros os
esparciré por las gentes, y desenvainaré espada en pos de vosotros: y
vuestra tierra estará asolada, y yermas vuestras ciudades. \bibverse{34}
Entonces la tierra holgará sus sábados todos los días que estuviere
asolada, y vosotros en la tierra de vuestros enemigos: la tierra
descansará entonces y gozará sus sábados. \bibverse{35} Todo el tiempo
que estará asolada, holgará lo que no holgó en vuestros sábados mientras
habitabais en ella.

\bibverse{36} Y á los que quedaren de vosotros infundiré en sus
corazones tal cobardía, en la tierra de sus enemigos, que el sonido de
una hoja movida los perseguirá, y huirán como de cuchillo, y caerán sin
que nadie los persiga: \bibverse{37} Y tropezarán los unos en los otros,
como si huyeran delante de cuchillo, aunque nadie los persiga; y no
podréis resistir delante de vuestros enemigos. \bibverse{38} Y
pereceréis entre las gentes, y la tierra de vuestros enemigos os
consumirá. \bibverse{39} Y los que quedaren de vosotros decaerán en las
tierras de vuestros enemigos por su iniquidad; y por la iniquidad de sus
padres decaerán con ellos:

\hypertarget{perspectiva-sobre-el-arrepentimiento-y-la-conversiuxf3n-de-israel-en-el-exilio}{%
\subsection{Perspectiva sobre el arrepentimiento y la conversión de
Israel en el
exilio}\label{perspectiva-sobre-el-arrepentimiento-y-la-conversiuxf3n-de-israel-en-el-exilio}}

\bibverse{40} Y confesarán su iniquidad, y la iniquidad de sus padres,
por su prevaricación con que prevaricaron contra mí: y también porque
anduvieron conmigo en oposición, \footnote{\textbf{26:40} Deut 4,30;
  Deut 30,2} \bibverse{41} Yo también habré andado con ellos en contra,
y los habré metido en la tierra de sus enemigos: y entonces se humillará
su corazón incircunciso, y reconocerán su pecado; \footnote{\textbf{26:41}
  Jer 9,25; Luc 23,41} \bibverse{42} Y yo me acordaré de mi pacto con
Jacob, y asimismo de mi pacto con Isaac, y también de mi pacto con
Abraham me acordaré; y haré memoria de la tierra. \footnote{\textbf{26:42}
  Éxod 2,24; 2Re 13,23} \bibverse{43} Que la tierra estará desamparada
de ellos, y holgará sus sábados, estando yerma á causa de ellos; mas
entretanto se someterán al castigo de sus iniquidades: por cuanto
menospreciaron mis derechos, y tuvo el alma de ellos fastidio de mis
estatutos. \footnote{\textbf{26:43} Lev 26,41} \bibverse{44} Y aun con
todo esto, estando ellos en tierra de sus enemigos, yo no los desecharé,
ni los abominaré para consumirlos, invalidando mi pacto con ellos:
porque yo Jehová soy su Dios: \bibverse{45} Antes me acordaré de ellos
por el pacto antiguo, cuando los saqué de la tierra de Egipto á los ojos
de las gentes, para ser su Dios: Yo Jehová. \footnote{\textbf{26:45} Gén
  15,18; Éxod 12,33; Éxod 12,51}

\bibverse{46} Estos son los decretos, derechos y leyes que estableció
Jehová entre sí y los hijos de Israel en el monte de Sinaí por mano de
Moisés.

\hypertarget{votos-y-su-soluciuxf3n}{%
\subsection{Votos y su solución}\label{votos-y-su-soluciuxf3n}}

\hypertarget{section-26}{%
\section{27}\label{section-26}}

\bibverse{1} Y habló Jehová á Moisés, diciendo: \bibverse{2} Habla á los
hijos de Israel, y diles: Cuando alguno hiciere especial voto á Jehová,
según la estimación de las personas que se hayan de redimir, así será tu
estimación: \bibverse{3} En cuanto al varón de veinte años hasta
sesenta, tu estimación será cincuenta siclos de plata, según el siclo
del santuario. \bibverse{4} Y si fuere hembra, la estimación será
treinta siclos. \bibverse{5} Y si fuere de cinco años hasta veinte, tu
estimación será respecto al varón veinte siclos, y á la hembra diez
siclos. \bibverse{6} Y si fuere de un mes hasta cinco años, tu
estimación será en orden al varón, cinco siclos de plata; y por la
hembra será tu estimación tres siclos de plata. \bibverse{7} Mas si
fuere de sesenta años arriba, por el varón tu estimación será quince
siclos, y por la hembra diez siclos. \bibverse{8} Pero si fuere más
pobre que tu estimación, entonces comparecerá ante el sacerdote, y el
sacerdote le pondrá tasa: conforme á la facultad del votante le impondrá
tasa el sacerdote.

\bibverse{9} Y si fuere animal de que se ofrece ofrenda á Jehová, todo
lo que se diere de él á Jehová será santo. \bibverse{10} No será mudado
ni trocado, bueno por malo, ni malo por bueno: y si se permutare un
animal por otro, él y el dado por él en cambio serán sagrados.
\bibverse{11} Y si fuere algún animal inmundo, de que no se ofrece
ofrenda á Jehová, entonces el animal será puesto delante del sacerdote:
\bibverse{12} Y el sacerdote lo apreciará, sea bueno ó sea malo;
conforme á la estimación del sacerdote, así será. \bibverse{13} Y si lo
hubieren de redimir, añadirán su quinto sobre tu valuación.

\bibverse{14} Y cuando alguno santificare su casa consagrándola á
Jehová, la apreciará el sacerdote, sea buena ó sea mala: según la
apreciare el sacerdote, así quedará. \bibverse{15} Mas si el
santificante redimiere su casa, añadirá á tu valuación el quinto del
dinero de ella, y será suya.

\bibverse{16} Y si alguno santificare de la tierra de su posesión á
Jehová, tu estimación será conforme á su sembradura: un omer de
sembradura de cebada se apreciará en cincuenta siclos de plata.
\bibverse{17} Y si santificare su tierra desde el año del jubileo,
conforme á tu estimación quedará. \bibverse{18} Mas si después del
jubileo santificare su tierra, entonces el sacerdote hará la cuenta del
dinero conforme á los años que quedaren hasta el año del jubileo, y se
rebajará de tu estimación. \bibverse{19} Y si el que santificó la tierra
quisiere redimirla, añadirá á tu estimación el quinto del dinero de
ella, y quedaráse para él. \bibverse{20} Mas si él no redimiere la
tierra, y la tierra se vendiere á otro, no la redimirá más;
\bibverse{21} Sino que cuando saliere en el jubileo, la tierra será
santa á Jehová, como tierra consagrada: la posesión de ella será del
sacerdote.

\bibverse{22} Y si santificare alguno á Jehová la tierra que él compró,
que no era de la tierra de su herencia, \bibverse{23} Entonces el
sacerdote calculará con él la suma de tu estimación hasta el año del
jubileo, y aquel día dará tu señalado precio, cosa consagrada á Jehová.
\bibverse{24} En el año del jubileo, volverá la tierra á aquél de quien
él la compró, cuya es la herencia de la tierra. \footnote{\textbf{27:24}
  Lev 25,10} \bibverse{25} Y todo lo que apreciares será conforme al
siclo del santuario: el siclo tiene veinte óbolos.

\hypertarget{disposiciones-relativas-a-los-primoguxe9nitos}{%
\subsection{Disposiciones relativas a los
primogénitos}\label{disposiciones-relativas-a-los-primoguxe9nitos}}

\bibverse{26} Empero el primogénito de los animales, que por la
primogenitura es de Jehová, nadie lo santificará; sea buey ú oveja, de
Jehová es. \bibverse{27} Mas si fuere de los animales inmundos, lo
redimirán conforme á tu estimación, y añadirán sobre ella su quinto: y
si no lo redimieren, se venderá conforme á tu estimación.

\hypertarget{consagraciones-en-forma-de-prohibiciuxf3n}{%
\subsection{Consagraciones en forma de
prohibición}\label{consagraciones-en-forma-de-prohibiciuxf3n}}

\bibverse{28} Pero ninguna cosa consagrada, que alguno hubiere
santificado á Jehová de todo lo que tuviere, de hombres y animales, y de
las tierras de su posesión, no se venderá, ni se redimirá: todo lo
consagrado será cosa santísima á Jehová. \footnote{\textbf{27:28} Núm
  18,14; Núm 21,2}

\bibverse{29} Cualquier anatema (cosa consagrada) de hombres que se
consagrare, no será redimido: indefectiblemente ha de ser muerto.
\footnote{\textbf{27:29} 1Sam 15,3; 1Sam 15,9}

\hypertarget{disposiciones-relativas-a-los-diezmos-de-frutas-y-ganado}{%
\subsection{Disposiciones relativas a los diezmos de frutas y
ganado}\label{disposiciones-relativas-a-los-diezmos-de-frutas-y-ganado}}

\bibverse{30} Y todas las décimas de la tierra, así de la simiente de la
tierra como del fruto de los árboles, de Jehová son: es cosa consagrada
á Jehová. \footnote{\textbf{27:30} Núm 18,21} \bibverse{31} Y si alguno
quisiere redimir algo de sus décimas, añadirá su quinto á ello.
\bibverse{32} Y toda décima de vacas ó de ovejas, de todo lo que pasa
bajo la vara, la décima será consagrada á Jehová. \bibverse{33} No
mirará si es bueno ó malo, ni lo trocará: y si lo trocare, ello y su
trueque serán cosas sagradas; no se redimirá.

\bibverse{34} Estos son los mandamientos que ordenó Jehová á Moisés,
para los hijos de Israel, en el monte de Sinaí.
