\hypertarget{apariciuxf3n-y-eficacia-de-juan-el-bautista}{%
\subsection{Aparición y eficacia de Juan el
Bautista}\label{apariciuxf3n-y-eficacia-de-juan-el-bautista}}

\hypertarget{section}{%
\section{1}\label{section}}

\bibverse{1} Principio del evangelio de Jesucristo, Hijo de Dios.

\bibverse{2} Como está escrito en Isaías el profeta: He aquí yo envío á
mi mensajero delante de tu faz, que apareje tu camino delante de ti.
\bibverse{3} Voz del que clama en el desierto: Aparejad el camino del
Señor; enderezad sus veredas.

\bibverse{4} Bautizaba Juan en el desierto, y predicaba el bautismo del
arrepentimiento para remisión de pecados. \bibverse{5} Y salía á él toda
la provincia de Judea, y los de Jerusalem; y eran todos bautizados por
él en el río de Jordán, confesando sus pecados. \bibverse{6} Y Juan
andaba vestido de pelos de camello, y con un cinto de cuero alrededor de
sus lomos; y comía langostas y miel silvestre. \bibverse{7} Y predicaba,
diciendo: Viene tras mí el que es más poderoso que yo, al cual no soy
digno de desatar encorvado la correa de sus zapatos. \bibverse{8} Yo á
la verdad os he bautizado con agua; mas él os bautizará con Espíritu
Santo.

\hypertarget{el-bautismo-y-la-tentaciuxf3n-de-jesuxfas}{%
\subsection{El bautismo y la tentación de
Jesús}\label{el-bautismo-y-la-tentaciuxf3n-de-jesuxfas}}

\bibverse{9} Y aconteció en aquellos días, que Jesús vino de Nazaret de
Galilea, y fué bautizado por Juan en el Jordán. \footnote{\textbf{1:9}
  Luc 2,51} \bibverse{10} Y luego, subiendo del agua, vió abrirse los
cielos, y al Espíritu como paloma, que descendía sobre él. \bibverse{11}
Y hubo una voz de los cielos que decía: Tú eres mi Hijo amado; en ti
tomo contentamiento.

\bibverse{12} Y luego el Espíritu le impele al desierto. \bibverse{13} Y
estuvo allí en el desierto cuarenta días, y era tentado de Satanás; y
estaba con las fieras; y los ángeles le servían.

\hypertarget{primera-apariciuxf3n-de-jesuxfas-en-galilea}{%
\subsection{Primera aparición de Jesús en
Galilea}\label{primera-apariciuxf3n-de-jesuxfas-en-galilea}}

\bibverse{14} Mas después que Juan fué encarcelado, Jesús vino á Galilea
predicando el evangelio del reino de Dios, \bibverse{15} Y diciendo: El
tiempo es cumplido, y el reino de Dios está cerca: arrepentíos, y creed
al evangelio. \footnote{\textbf{1:15} Gal 4,4}

\hypertarget{llamando-a-los-primeros-cuatro-discuxedpulos}{%
\subsection{Llamando a los primeros cuatro
discípulos}\label{llamando-a-los-primeros-cuatro-discuxedpulos}}

\bibverse{16} Y pasando junto á la mar de Galilea, vió á Simón, y á
Andrés su hermano, que echaban la red en la mar; porque eran pescadores.
\bibverse{17} Y les dijo Jesús: Venid en pos de mí, y haré que seáis
pescadores de hombres.

\bibverse{18} Y luego, dejadas sus redes, le siguieron.

\bibverse{19} Y pasando de allí un poco más adelante, vió á Jacobo, hijo
de Zebedeo, y á Juan su hermano, también ellos en el navío, que
aderezaban las redes. \bibverse{20} Y luego los llamó: y dejando á su
padre Zebedeo en el barco con los jornaleros, fueron en pos de él.

\hypertarget{el-primer-sermuxf3n-de-jesuxfas-y-la-curaciuxf3n-de-un-hombre-poseuxeddo-en-la-sinagoga-de-capernaum}{%
\subsection{El primer sermón de Jesús y la curación de un hombre poseído
en la sinagoga de
Capernaum}\label{el-primer-sermuxf3n-de-jesuxfas-y-la-curaciuxf3n-de-un-hombre-poseuxeddo-en-la-sinagoga-de-capernaum}}

\bibverse{21} Y entraron en Capernaum; y luego los sábados, entrando en
la sinagoga, enseñaba. \bibverse{22} Y se admiraban de su doctrina;
porque les enseñaba como quien tiene potestad, y no como los escribas.
\bibverse{23} Y había en la sinagoga de ellos un hombre con espíritu
inmundo, el cual dió voces, \bibverse{24} Diciendo: ¡Ah! ¿qué tienes con
nosotros, Jesús Nazareno? ¿Has venido á destruirnos? Sé quién eres, el
Santo de Dios. \footnote{\textbf{1:24} Mar 5,7}

\bibverse{25} Y Jesús le riñó, diciendo: Enmudece, y sal de él.

\bibverse{26} Y el espíritu inmundo, haciéndole pedazos, y clamando á
gran voz, salió de él. \bibverse{27} Y todos se maravillaron, de tal
manera que inquirían entre sí, diciendo: ¿Qué es esto? ¿Qué nueva
doctrina es ésta, que con potestad aun á los espíritus inmundos manda, y
le obedecen? \bibverse{28} Y vino luego su fama por toda la provincia
alrededor de Galilea.

\hypertarget{sanaciuxf3n-de-la-suegra-de-simuxf3n-y-otros-enfermos-en-capernaum}{%
\subsection{Sanación de la suegra de Simón y otros enfermos en
Capernaum}\label{sanaciuxf3n-de-la-suegra-de-simuxf3n-y-otros-enfermos-en-capernaum}}

\bibverse{29} Y luego saliendo de la sinagoga, vinieron á casa de Simón
y de Andrés, con Jacobo y Juan. \bibverse{30} Y la suegra de Simón
estaba acostada con calentura; y le hablaron luego de ella.
\bibverse{31} Entonces llegando él, la tomó de su mano y la levantó; y
luego la dejó la calentura, y les servía.

\bibverse{32} Y cuando fué la tarde, luego que el sol se puso, traían á
él todos los que tenían mal, y endemoniados; \bibverse{33} Y toda la
ciudad se juntó á la puerta. \bibverse{34} Y sanó á muchos que estaban
enfermos de diversas enfermedades, y echó fuera muchos demonios; y no
dejaba decir á los demonios que le conocían. \footnote{\textbf{1:34}
  Hech 16,17-18}

\hypertarget{jesuxfas-deja-capernaum-su-sermuxf3n-errante-y-actividad-curativa-en-galilea}{%
\subsection{Jesús deja Capernaum; su sermón errante y actividad curativa
en
Galilea}\label{jesuxfas-deja-capernaum-su-sermuxf3n-errante-y-actividad-curativa-en-galilea}}

\bibverse{35} Y levantándose muy de mañana, aun muy de noche, salió y se
fué á un lugar desierto, y allí oraba. \footnote{\textbf{1:35} Mat
  14,23; Mat 26,36; Luc 5,16; Luc 11,1} \bibverse{36} Y le siguió Simón,
y los que estaban con él; \bibverse{37} Y hallándole, le dicen: Todos te
buscan.

\bibverse{38} Y les dice: Vamos á los lugares vecinos, para que predique
también allí; porque para esto he venido. \bibverse{39} Y predicaba en
las sinagogas de ellos en toda Galilea, y echaba fuera los demonios.

\hypertarget{jesuxfas-sana-a-un-leproso-y-escapa-a-la-soledad}{%
\subsection{Jesús sana a un leproso y escapa a la
soledad}\label{jesuxfas-sana-a-un-leproso-y-escapa-a-la-soledad}}

\bibverse{40} Y un leproso vino á él, rogándole; é hincada la rodilla,
le dice: Si quieres, puedes limpiarme.

\bibverse{41} Y Jesús, teniendo misericordia de él, extendió su mano, y
le tocó, y le dice: Quiero, sé limpio. \bibverse{42} Y así que hubo él
hablado, la lepra se fué luego de aquél, y fué limpio. \bibverse{43}
Entonces le apercibió, y despidióle luego, \footnote{\textbf{1:43} Mar
  3,12; Mar 7,36} \bibverse{44} Y le dice: Mira, no digas á nadie nada;
sino ve, muéstrate al sacerdote, y ofrece por tu limpieza lo que Moisés
mandó, para testimonio á ellos. \footnote{\textbf{1:44} Lev 14,2-32}

\bibverse{45} Mas él salido, comenzó á publicarlo mucho, y á divulgar el
hecho, de manera que ya Jesús no podía entrar manifiestamente en la
ciudad, sino que estaba fuera en los lugares desiertos; y venían á él de
todas partes.

\hypertarget{curaciuxf3n-de-un-paraluxedtico-en-capernaum-jesuxfas-perdona-los-pecados}{%
\subsection{Curación de un paralítico en Capernaum; Jesús perdona los
pecados}\label{curaciuxf3n-de-un-paraluxedtico-en-capernaum-jesuxfas-perdona-los-pecados}}

\hypertarget{section-1}{%
\section{2}\label{section-1}}

\bibverse{1} Y entró otra vez en Capernaum después de algunos días, y se
oyó que estaba en casa. \bibverse{2} Y luego se juntaron á él muchos,
que ya no cabían ni aun á la puerta; y les predicaba la palabra.
\bibverse{3} Entonces vinieron á él unos trayendo un paralítico, que era
traído por cuatro. \bibverse{4} Y como no podían llegar á él á causa del
gentío, descubrieron el techo de donde estaba, y haciendo abertura,
bajaron el lecho en que yacía el paralítico. \bibverse{5} Y viendo Jesús
la fe de ellos, dice al paralítico: Hijo, tus pecados te son perdonados.

\bibverse{6} Y estaban allí sentados algunos de los escribas, los cuales
pensando en sus corazones, \bibverse{7} Decían: ¿Por qué habla éste así?
Blasfemias dice. ¿Quién puede perdonar pecados, sino solo Dios?
\footnote{\textbf{2:7} Sal 130,4; Is 43,25}

\bibverse{8} Y conociendo luego Jesús en su espíritu que pensaban así
dentro de sí mismos, les dijo: ¿Por qué pensáis estas cosas en vuestros
corazones? \bibverse{9} ¿Qué es más fácil, decir al paralítico: Tus
pecados te son perdonados, ó decirle: Levántate, y toma tu lecho y anda?
\bibverse{10} Pues para que sepáis que el Hijo del hombre tiene potestad
en la tierra de perdonar los pecados, (dice al paralítico):
\bibverse{11} A ti te digo: Levántate, y toma tu lecho, y vete á tu
casa.

\bibverse{12} Entonces él se levantó luego, y tomando su lecho, se salió
delante de todos, de manera que todos se asombraron, y glorificaron á
Dios, diciendo: Nunca tal hemos visto.

\hypertarget{llamando-al-recaudador-de-impuestos-levi-jesuxfas-como-compauxf1ero-de-mesa-para-recaudadores-de-impuestos-y-pecadores}{%
\subsection{Llamando al recaudador de impuestos Levi; Jesús como
compañero de mesa para recaudadores de impuestos y
pecadores}\label{llamando-al-recaudador-de-impuestos-levi-jesuxfas-como-compauxf1ero-de-mesa-para-recaudadores-de-impuestos-y-pecadores}}

\bibverse{13} Y volvió á salir á la mar, y toda la gente venía á él, y
los enseñaba. \bibverse{14} Y pasando, vió á Leví, hijo de Alfeo,
sentado al banco de los públicos tributos, y le dice: Sígueme. Y
levantándose le siguió.

\bibverse{15} Y aconteció que estando Jesús á la mesa en casa de él,
muchos publicanos y pecadores estaban también á la mesa juntamente con
Jesús y con sus discípulos: porque había muchos, y le habían seguido.
\bibverse{16} Y los escribas y los Fariseos, viéndole comer con los
publicanos y con los pecadores, dijeron á sus discípulos: ¿Qué es esto,
que él come y bebe con los publicanos y con los pecadores?

\bibverse{17} Y oyéndolo Jesús, les dice: Los sanos no tienen necesidad
de médico, mas los que tienen mal. No he venido á llamar á los justos,
sino á los pecadores.

\hypertarget{la-pregunta-del-ayuno-de-los-discuxedpulos-de-juan-y-los-fariseos}{%
\subsection{La pregunta del ayuno de los discípulos de Juan y los
fariseos}\label{la-pregunta-del-ayuno-de-los-discuxedpulos-de-juan-y-los-fariseos}}

\bibverse{18} Y los discípulos de Juan, y de los Fariseos ayunaban; y
vienen, y le dicen: ¿Por qué los discípulos de Juan y los de los
Fariseos ayunan, y tus discípulos no ayunan?

\bibverse{19} Y Jesús les dice: ¿Pueden ayunar los que están de bodas,
cuando el esposo está con ellos? Entre tanto que tienen consigo al
esposo no pueden ayunar. \bibverse{20} Mas vendrán días, cuando el
esposo les será quitado, y entonces en aquellos días ayunarán.
\bibverse{21} Nadie echa remiendo de paño recio en vestido viejo; de
otra manera el mismo remiendo nuevo tira del viejo, y la rotura se hace
peor. \bibverse{22} Ni nadie echa vino nuevo en odres viejos; de otra
manera, el vino nuevo rompe los odres, y se derrama el vino, y los odres
se pierden; mas el vino nuevo en odres nuevos se ha de echar.

\hypertarget{el-arranco-de-espigas-de-los-discuxedpulos-en-suxe1bado-la-primera-disputa-de-jesuxfas-con-los-fariseos-sobre-la-santificaciuxf3n-del-duxeda-de-reposo}{%
\subsection{El arranco de espigas de los discípulos en sábado; La
primera disputa de Jesús con los fariseos sobre la santificación del día
de
reposo}\label{el-arranco-de-espigas-de-los-discuxedpulos-en-suxe1bado-la-primera-disputa-de-jesuxfas-con-los-fariseos-sobre-la-santificaciuxf3n-del-duxeda-de-reposo}}

\bibverse{23} Y aconteció que pasando él por los sembrados en sábado,
sus discípulos andando comenzaron á arrancar espigas. \bibverse{24}
Entonces los Fariseos le dijeron: He aquí, ¿por qué hacen en sábado lo
que no es lícito?

\bibverse{25} Y él les dijo: ¿Nunca leísteis qué hizo David cuando tuvo
necesidad, y tuvo hambre, él y los que con él estaban: \bibverse{26}
Cómo entró en la casa de Dios, siendo Abiathar sumo pontífice, y comió
los panes de la proposición, de los cuales no es lícito comer sino á los
sacerdotes, y aun dió á los que con él estaban?

\bibverse{27} También les dijo: El sábado por causa del hombre es hecho;
no el hombre por causa del sábado. \footnote{\textbf{2:27} Deut 5,14}

\bibverse{28} Así que el Hijo del hombre es Señor aun del sábado.

\hypertarget{sanaciuxf3n-del-hombre-con-el-brazo-paralizado-en-suxe1bado-el-segundo-argumento-sobre-la-observancia-del-suxe1bado}{%
\subsection{Sanación del hombre con el brazo paralizado en sábado; el
segundo argumento sobre la observancia del
sábado}\label{sanaciuxf3n-del-hombre-con-el-brazo-paralizado-en-suxe1bado-el-segundo-argumento-sobre-la-observancia-del-suxe1bado}}

\hypertarget{section-2}{%
\section{3}\label{section-2}}

\bibverse{1} Y otra vez entró en la sinagoga; y había allí un hombre que
tenía una mano seca. \bibverse{2} Y le acechaban si en sábado le
sanaría, para acusarle. \bibverse{3} Entonces dijo al hombre que tenía
la mano seca: Levántate en medio. \bibverse{4} Y les dice: ¿Es lícito
hacer bien en sábado, ó hacer mal? ¿salvar la vida, ó quitarla? Mas
ellos callaban. \bibverse{5} Y mirándolos alrededor con enojo,
condoleciéndose de la ceguedad de su corazón, dice al hombre: Extiende
tu mano. Y la extendió, y su mano fué restituída sana. \bibverse{6}
Entonces saliendo los Fariseos, tomaron consejo con los Herodianos
contra él, para matarle.

\hypertarget{afluencia-de-personas-muchas-curaciones-en-el-lago}{%
\subsection{Afluencia de personas; muchas curaciones en el
lago}\label{afluencia-de-personas-muchas-curaciones-en-el-lago}}

\bibverse{7} Mas Jesús se apartó á la mar con sus discípulos: y le
siguió gran multitud de Galilea, y de Judea, \bibverse{8} Y de
Jerusalem, y de Idumea, y de la otra parte del Jordán. Y los de
alrededor de Tiro y de Sidón, grande multitud, oyendo cuán grandes cosas
hacía, vinieron á él. \bibverse{9} Y dijo á sus discípulos que le
estuviese siempre apercibida la barquilla, por causa del gentío, para
que no le oprimiesen. \bibverse{10} Porque había sanado á muchos; de
manera que caían sobre él cuantos tenían plagas, por tocarle.
\bibverse{11} Y los espíritus inmundos, al verle, se postraban delante
de él, y daban voces, diciendo: Tú eres el Hijo de Dios. \footnote{\textbf{3:11}
  Luc 4,41} \bibverse{12} Mas él les reñía mucho que no le manifestasen.
\footnote{\textbf{3:12} Mar 1,43}

\hypertarget{berufung-und-namen-der-zwuxf6lf-juxfcnger}{%
\subsection{Berufung und Namen der zwölf
Jünger}\label{berufung-und-namen-der-zwuxf6lf-juxfcnger}}

\bibverse{13} Y subió al monte, y llamó á sí á los que él quiso; y
vinieron á él. \bibverse{14} Y estableció doce, para que estuviesen con
él, y para enviarlos á predicar, \bibverse{15} Y que tuviesen potestad
de sanar enfermedades, y de echar fuera demonios: \bibverse{16} A Simón,
al cual puso por nombre Pedro; \bibverse{17} Y á Jacobo, hijo de
Zebedeo, y á Juan hermano de Jacobo; y les apellidó Boanerges, que es,
Hijos del trueno; \footnote{\textbf{3:17} Luc 9,54} \bibverse{18} Y á
Andrés, y á Felipe, y á Bartolomé, y á Mateo, y á Tomás, y á Jacobo hijo
de Alfeo, y á Tadeo, y á Simón el Cananita, \bibverse{19} Y á Judas
Iscariote, el que le entregó. Y vinieron á casa.

\hypertarget{el-crecimiento-del-movimiento}{%
\subsection{El crecimiento del
movimiento}\label{el-crecimiento-del-movimiento}}

\bibverse{20} Y agolpóse de nuevo la gente, de modo que ellos ni aun
podían comer pan. \bibverse{21} Y como lo oyeron los suyos, vinieron
para prenderle: porque decían: Está fuera de sí.

\hypertarget{jesuxfas-se-defiende-de-la-blasfemia-de-beelzebul-de-los-escribas.-del-pecado-contra-el-espuxedritu-santo}{%
\subsection{Jesús se defiende de la blasfemia de Beelzebul de los
escribas. Del pecado contra el espíritu
santo}\label{jesuxfas-se-defiende-de-la-blasfemia-de-beelzebul-de-los-escribas.-del-pecado-contra-el-espuxedritu-santo}}

\bibverse{22} Y los escribas que habían venido de Jerusalem, decían que
tenía á Beelzebub, y que por el príncipe de los demonios echaba fuera
los demonios. \footnote{\textbf{3:22} Mat 9,34}

\bibverse{23} Y habiéndolos llamado, les decía en parábolas: ¿Cómo puede
Satanás echar fuera á Satanás? \bibverse{24} Y si algún reino contra sí
mismo fuere dividido, no puede permanecer el tal reino. \bibverse{25} Y
si alguna casa fuere dividida contra sí misma, no puede permanecer la
tal casa. \bibverse{26} Y si Satanás se levantare contra sí mismo, y
fuere dividido, no puede permanecer; antes tiene fin. \bibverse{27}
Nadie puede saquear las alhajas del valiente entrando en su casa, si
antes no atare al valiente y entonces saqueará su casa.

\bibverse{28} De cierto os digo que todos los pecados serán perdonados á
los hijos de los hombres, y las blasfemias cualesquiera con que
blasfemaren; \bibverse{29} Mas cualquiera que blasfemare contra el
Espíritu Santo, no tiene jamás perdón, mas está expuesto á eterno
juicio. \bibverse{30} Porque decían: Tiene espíritu inmundo. \footnote{\textbf{3:30}
  Juan 10,20}

\hypertarget{los-verdaderos-parientes-de-jesuxfas}{%
\subsection{Los verdaderos parientes de
Jesús}\label{los-verdaderos-parientes-de-jesuxfas}}

\bibverse{31} Vienen después sus hermanos y su madre, y estando fuera,
enviaron á él llamándole. \bibverse{32} Y la gente estaba sentada
alrededor de él, y le dijeron: He aquí, tu madre y tus hermanos te
buscan fuera.

\bibverse{33} Y él les respondió, diciendo: ¿Quién es mi madre y mis
hermanos? \bibverse{34} Y mirando á los que estaban sentados alrededor
de él, dijo: He aquí mi madre y hermanos. \bibverse{35} Porque
cualquiera que hiciere la voluntad de Dios, éste es mi hermano, y mi
hermana, y mi madre.

\hypertarget{paruxe1bola-del-sembrador-y-cuatro-tipos-de-campos}{%
\subsection{Parábola del sembrador y cuatro tipos de
campos}\label{paruxe1bola-del-sembrador-y-cuatro-tipos-de-campos}}

\hypertarget{section-3}{%
\section{4}\label{section-3}}

\bibverse{1} Y otra vez comenzó á enseñar junto á la mar, y se juntó á
él mucha gente; tanto, que entrándose él en un barco, se sentó en la
mar: y toda la gente estaba en tierra junto á la mar. \bibverse{2} Y les
enseñaba por parábolas muchas cosas, y les decía en su doctrina:
\bibverse{3} Oid: He aquí, el sembrador salió á sembrar. \bibverse{4} Y
aconteció sembrando, que una parte cayó junto al camino; y vinieron las
aves del cielo, y la tragaron. \bibverse{5} Y otra parte cayó en
pedregales, donde no tenía mucha tierra; y luego salió, porque no tenía
la tierra profunda: \bibverse{6} Mas salido el sol, se quemó; y por
cuanto no tenía raíz, se secó. \bibverse{7} Y otra parte cayó en
espinas; y subieron las espinas, y la ahogaron, y no dió fruto.
\bibverse{8} Y otra parte cayó en buena tierra, y dió fruto, que subió y
creció: y llevó uno á treinta, y otro á sesenta, y otro á ciento.
\bibverse{9} Entonces les dijo: El que tiene oídos para oir, oiga.

\hypertarget{analice-el-significado-y-el-propuxf3sito-de-las-paruxe1bolas}{%
\subsection{Analice el significado y el propósito de las
parábolas}\label{analice-el-significado-y-el-propuxf3sito-de-las-paruxe1bolas}}

\bibverse{10} Y cuando estuvo solo, le preguntaron los que estaban cerca
de él con los doce, sobre la parábola. \bibverse{11} Y les dijo: A
vosotros es dado saber el misterio del reino de Dios; mas á los que
están fuera, por parábolas todas las cosas; \bibverse{12} Para que
viendo, vean y no echen de ver; y oyendo, oigan y no entiendan: porque
no se conviertan, y les sean perdonados los pecados.

\bibverse{13} Y les dijo: ¿No sabéis esta parábola? ¿Cómo, pues,
entenderéis todas las parábolas?

\hypertarget{interpretaciuxf3n-de-la-paruxe1bola-del-sembrador}{%
\subsection{Interpretación de la parábola del
sembrador}\label{interpretaciuxf3n-de-la-paruxe1bola-del-sembrador}}

\bibverse{14} El que siembra es el que siembra la palabra. \bibverse{15}
Y éstos son los de junto al camino: en los que la palabra es sembrada:
mas después que la oyeron, luego viene Satanás, y quita la palabra que
fué sembrada en sus corazones. \bibverse{16} Y asimismo éstos son los
que son sembrados en pedregales: los que cuando han oído la palabra,
luego la toman con gozo; \bibverse{17} Mas no tienen raíz en sí, antes
son temporales, que en levantándose la tribulación ó la persecución por
causa de la palabra, luego se escandalizan. \bibverse{18} Y éstos son
los que son sembrados entre espinas: los que oyen la palabra;
\bibverse{19} Mas los cuidados de este siglo, y el engaño de las
riquezas, y las codicias que hay en las otras cosas, entrando, ahogan la
palabra, y se hace infructuosa. \bibverse{20} Y éstos son los que fueron
sembrados en buena tierra: los que oyen la palabra, y la reciben, y
hacen fruto, uno á treinta, otro á sesenta, y otro á ciento.

\bibverse{21} También les dijo: ¿Tráese la antorcha para ser puesta
debajo del almud, ó debajo de la cama? ¿No es para ser puesta en el
candelero? \footnote{\textbf{4:21} Mat 5,15} \bibverse{22} Porque no hay
nada oculto que no haya de ser manifestado, ni secreto que no haya de
descubrirse. \footnote{\textbf{4:22} Mat 10,26-27; Luc 12,2}
\bibverse{23} Si alguno tiene oídos para oir, oiga.

\bibverse{24} Les dijo también: Mirad lo que oís: con la medida que
medís, os medirán otros, y será añadido á vosotros los que oís.
\footnote{\textbf{4:24} Mat 7,2} \bibverse{25} Porque al que tiene, le
será dado; y al que no tiene, aun lo que tiene le será quitado.
\footnote{\textbf{4:25} Mat 13,12-13}

\hypertarget{paruxe1bolas-de-la-semilla-que-crece-tranquilamente-por-suxed-misma-y-de-la-semilla-de-mostaza}{%
\subsection{Parábolas de la semilla que crece tranquilamente por sí
misma y de la semilla de
mostaza}\label{paruxe1bolas-de-la-semilla-que-crece-tranquilamente-por-suxed-misma-y-de-la-semilla-de-mostaza}}

\bibverse{26} Decía más: Así es el reino de Dios, como si un hombre echa
simiente en la tierra; \bibverse{27} Y duerme, y se levanta de noche y
de día, y la simiente brota y crece como él no sabe. \footnote{\textbf{4:27}
  Sant 5,7} \bibverse{28} Porque de suyo fructifica la tierra, primero
hierba, luego espiga, después grano lleno en la espiga; \bibverse{29} Y
cuando el fruto fuere producido, luego se mete la hoz, porque la siega
es llegada.

\bibverse{30} Y decía: ¿A qué haremos semejante el reino de Dios? ¿ó con
qué parábola le compararemos? \bibverse{31} Es como el grano de mostaza,
que, cuando se siembra en tierra, es la más pequeña de todas las
simientes que hay en la tierra; \bibverse{32} Mas después de sembrado,
sube, y se hace la mayor de todas las legumbres, y echa grandes ramas,
de tal manera que las aves del cielo puedan morar bajo su sombra.

\bibverse{33} Y con muchas tales parábolas les hablaba la palabra,
conforme á lo que podían oir. \bibverse{34} Y sin parábola no les
hablaba; mas á sus discípulos en particular declaraba todo.

\hypertarget{jesuxfas-apacigua-la-tormenta-del-mar}{%
\subsection{Jesús apacigua la tormenta del
mar}\label{jesuxfas-apacigua-la-tormenta-del-mar}}

\bibverse{35} Y les dijo aquel día cuando fué tarde: Pasemos de la otra
parte. \bibverse{36} Y despachando la multitud, le tomaron como estaba,
en el barco; y había también con él otros barquitos. \bibverse{37} Y se
levantó una grande tempestad de viento, y echaba las olas en el barco,
de tal manera que ya se henchía. \bibverse{38} Y él estaba en la popa,
durmiendo sobre un cabezal, y le despertaron, y le dicen: ¿Maestro, no
tienes cuidado que perecemos?

\bibverse{39} Y levantándose, increpó al viento, y dijo á la mar: Calla,
enmudece. Y cesó el viento, y fué hecha grande bonanza. \bibverse{40} Y
á ellos dijo: ¿Por qué estáis así amedrentados? ¿Cómo no tenéis fe?

\bibverse{41} Y temieron con gran temor, y decían el uno al otro. ¿Quién
es éste, que aun el viento y la mar le obedecen?

\hypertarget{jesuxfas-sana-a-los-poseuxeddos-en-la-tierra-de-los-gerasenos}{%
\subsection{Jesús sana a los poseídos en la tierra de los
gerasenos}\label{jesuxfas-sana-a-los-poseuxeddos-en-la-tierra-de-los-gerasenos}}

\hypertarget{section-4}{%
\section{5}\label{section-4}}

\bibverse{1} Y vinieron de la otra parte de la mar á la provincia de los
Gadarenos. \bibverse{2} Y salido él del barco, luego le salió al
encuentro, de los sepulcros, un hombre con un espíritu inmundo,
\bibverse{3} Que tenía domicilio en los sepulcros, y ni aun con cadenas
le podía alguien atar; \bibverse{4} Porque muchas veces había sido atado
con grillos y cadenas, mas las cadenas habían sido hechas pedazos por
él, y los grillos desmenuzados; y nadie le podía domar. \bibverse{5} Y
siempre, de día y de noche, andaba dando voces en los montes y en los
sepulcros, é hiriéndose con las piedras. \bibverse{6} Y como vió á Jesús
de lejos, corrió, y le adoró. \bibverse{7} Y clamando á gran voz, dijo:
¿Qué tienes conmigo, Jesús, Hijo del Dios Altísimo? Te conjuro por Dios
que no me atormentes. \bibverse{8} Porque le decía: Sal de este hombre,
espíritu inmundo.

\bibverse{9} Y le preguntó: ¿Cómo te llamas? Y respondió diciendo:
Legión me llamo; porque somos muchos.

\bibverse{10} Y le rogaba mucho que no le enviase fuera de aquella
provincia. \bibverse{11} Y estaba allí cerca del monte una grande manada
de puercos paciendo. \bibverse{12} Y le rogaron todos los demonios,
diciendo: Envíanos á los puercos para que entremos en ellos.

\bibverse{13} Y luego Jesús se lo permitió. Y saliendo aquellos
espíritus inmundos, entraron en los puercos, y la manada cayó por un
despeñadero en la mar; los cuales eran como dos mil; y en la mar se
ahogaron. \bibverse{14} Y los que apacentaban los puercos huyeron, y
dieron aviso en la ciudad y en los campos. Y salieron para ver qué era
aquello que había acontecido.

\bibverse{15} Y vienen á Jesús, y ven al que había sido atormentado del
demonio, y que había tenido la legión, sentado y vestido, y en su juicio
cabal; y tuvieron miedo. \bibverse{16} Y les contaron los que lo habían
visto, cómo había acontecido al que había tenido el demonio, y lo de los
puercos. \bibverse{17} Y comenzaron á rogarle que se fuese de los
términos de ellos.

\bibverse{18} Y entrando él en el barco, le rogaba el que había sido
fatigado del demonio, para estar con él. \bibverse{19} Mas Jesús no le
permitió, sino le dijo: Vete á tu casa, á los tuyos, y cuéntales cuán
grandes cosas el Señor ha hecho contigo, y cómo ha tenido misericordia
de ti.

\bibverse{20} Y se fué, y comenzó á publicar en Decápolis cuán grandes
cosas Jesús había hecho con él: y todos se maravillaban. \footnote{\textbf{5:20}
  Mar 7,31}

\hypertarget{jesuxfas-sana-a-la-mujer-ensangrentada-en-capernaum-y-despierta-a-la-hija-de-jairo}{%
\subsection{Jesús sana a la mujer ensangrentada en Capernaum y despierta
a la hija de
Jairo}\label{jesuxfas-sana-a-la-mujer-ensangrentada-en-capernaum-y-despierta-a-la-hija-de-jairo}}

\bibverse{21} Y pasando otra vez Jesús en un barco á la otra parte, se
juntó á él gran compañía; y estaba junto á la mar. \bibverse{22} Y vino
uno de los príncipes de la sinagoga, llamado Jairo; y luego que le vió,
se postró á sus pies, \bibverse{23} Y le rogaba mucho, diciendo: Mi hija
está á la muerte: ven y pondrás las manos sobre ella para que sea salva,
y vivirá.

\bibverse{24} Y fué con él, y le seguía gran compañía, y le apretaban.
\bibverse{25} Y una mujer que estaba con flujo de sangre doce años
hacía, \bibverse{26} Y había sufrido mucho de muchos médicos, y había
gastado todo lo que tenía, y nada había aprovechado, antes le iba peor,
\bibverse{27} Como oyó hablar de Jesús, llegó por detrás entre la
compañía, y tocó su vestido. \bibverse{28} Porque decía: Si tocare tan
solamente su vestido, seré salva. \bibverse{29} Y luego la fuente de su
sangre se secó; y sintió en el cuerpo que estaba sana de aquel azote.

\bibverse{30} Y luego Jesús, conociendo en sí mismo la virtud que había
salido de él, volviéndose á la compañía, dijo: ¿Quién ha tocado mis
vestidos?

\bibverse{31} Y le dijeron sus discípulos: Ves que la multitud te
aprieta, y dices: ¿Quién me ha tocado?

\bibverse{32} Y él miraba alrededor para ver á la que había hecho esto.
\bibverse{33} Entonces la mujer, temiendo y temblando, sabiendo lo que
en sí había sido hecho, vino y se postró delante de él, y le dijo toda
la verdad.

\bibverse{34} Y él le dijo: Hija, tu fe te ha hecho salva: ve en paz, y
queda sana de tu azote.

\bibverse{35} Hablando aún él, vinieron de casa del príncipe de la
sinagoga, diciendo: Tu hija es muerta; ¿para qué fatigas más al Maestro?

\bibverse{36} Mas luego Jesús, oyendo esta razón que se decía, dijo al
príncipe de la sinagoga: No temas, cree solamente. \bibverse{37} Y no
permitió que alguno viniese tras él sino Pedro, y Jacobo, y Juan hermano
de Jacobo. \footnote{\textbf{5:37} Mat 17,1} \bibverse{38} Y vino á casa
del príncipe de la sinagoga, y vió el alboroto, los que lloraban y
gemían mucho. \bibverse{39} Y entrando, les dice: ¿Por qué alborotáis y
lloráis? La muchacha no es muerta, mas duerme.

\bibverse{40} Y hacían burla de él: mas él, echados fuera todos, toma al
padre y á la madre de la muchacha, y á los que estaban con él, y entra
donde la muchacha estaba. \bibverse{41} Y tomando la mano de la
muchacha, le dice: Talitha cumi; que es, si lo interpretares: Muchacha,
á ti digo, levántate. \footnote{\textbf{5:41} Luc 7,14; Hech 9,40}

\bibverse{42} Y luego la muchacha se levantó, y andaba; porque tenía
doce años. Y se espantaron de grande espanto. \bibverse{43} Mas él les
mandó mucho que nadie lo supiese, y dijo que le diesen de comer.

\hypertarget{rechazo-y-fracaso-de-jesuxfas-en-su-natal-nazaret}{%
\subsection{Rechazo y fracaso de Jesús en su natal
Nazaret}\label{rechazo-y-fracaso-de-jesuxfas-en-su-natal-nazaret}}

\hypertarget{section-5}{%
\section{6}\label{section-5}}

\bibverse{1} Y salió de allí, y vino á su tierra, y le siguieron sus
discípulos. \bibverse{2} Y llegado el sábado, comenzó á enseñar en la
sinagoga; y muchos oyéndole, estaban atónitos, diciendo: ¿De dónde tiene
éste estas cosas? ¿Y qué sabiduría es ésta que le es dada, y tales
maravillas que por sus manos son hechas? \bibverse{3} ¿No es éste el
carpintero, hijo de María, hermano de Jacobo, y de José, y de Judas, y
de Simón? ¿No están también aquí con nosotros, sus hermanas? Y se
escandalizaban en él. \footnote{\textbf{6:3} Juan 6,42}

\bibverse{4} Mas Jesús les decía: No hay profeta deshonrado sino en su
tierra, y entre sus parientes, y en su casa. \bibverse{5} Y no pudo
hacer allí alguna maravilla; solamente sanó unos pocos enfermos,
poniendo sobre ellos las manos. \bibverse{6} Y estaba maravillado de la
incredulidad de ellos. Y rodeaba las aldeas de alrededor, enseñando.

\hypertarget{enviar-e-instruir-a-los-doce-discuxedpulos}{%
\subsection{Enviar e instruir a los doce
discípulos}\label{enviar-e-instruir-a-los-doce-discuxedpulos}}

\bibverse{7} Y llamó á los doce, y comenzó á enviarlos de dos en dos: y
les dió potestad sobre los espíritus inmundos. \bibverse{8} Y les mandó
que no llevasen nada para el camino, sino solamente báculo; no alforja,
ni pan, ni dinero en la bolsa; \bibverse{9} Mas que calzasen sandalias,
y no vistiesen dos túnicas. \bibverse{10} Y les decía: Donde quiera que
entréis en una casa, posad en ella hasta que salgáis de allí.
\bibverse{11} Y todos aquellos que no os recibieren ni os oyeren,
saliendo de allí, sacudid el polvo que está debajo de vuestros pies, en
testimonio á ellos. De cierto os digo que más tolerable será el castigo
de los de Sodoma y Gomorra el día del juicio, que el de aquella ciudad.

\bibverse{12} Y saliendo, predicaban que los hombres se arrepintiesen.
\bibverse{13} Y echaban fuera muchos demonios, y ungían con aceite á
muchos enfermos, y sanaban. \footnote{\textbf{6:13} Sant 5,14; Sant
  1,5-15}

\hypertarget{el-juicio-de-herodes-sobre-jesuxfas-el-fin-de-juan-el-bautista}{%
\subsection{El juicio de Herodes sobre Jesús; el fin de Juan el
Bautista}\label{el-juicio-de-herodes-sobre-jesuxfas-el-fin-de-juan-el-bautista}}

\bibverse{14} Y oyó el rey Herodes la fama de Jesús, porque su nombre se
había hecho notorio; y dijo: Juan el que bautizaba, ha resucitado de los
muertos, y por tanto, virtudes obran en él. \bibverse{15} Otros decían:
Elías es. Y otros decían: Profeta es, ó alguno de los profetas.
\bibverse{16} Y oyéndolo Herodes, dijo: Este es Juan el que yo degollé:
él ha resucitado de los muertos. \bibverse{17} Porque el mismo Herodes
había enviado, y prendido á Juan, y le había aprisionado en la cárcel á
causa de Herodías, mujer de Felipe su hermano; pues la había tomado por
mujer. \bibverse{18} Porque Juan decía á Herodes: No te es lícito tener
la mujer de tu hermano. \bibverse{19} Mas Herodías le acechaba, y
deseaba matarle, y no podía: \bibverse{20} Porque Herodes temía á Juan,
sabiendo que era varón justo y santo, y le tenía respeto: y oyéndole,
hacía muchas cosas; y le oía de buena gana.

\bibverse{21} Y venido un día oportuno, en que Herodes, en la fiesta de
su nacimiento, daba una cena á sus príncipes y tribunos, y á los
principales de Galilea; \bibverse{22} Y entrando la hija de Herodías, y
danzando, y agradando á Herodes y á los que estaban con él á la mesa, el
rey dijo á la muchacha: Pídeme lo que quisieres, que yo te lo daré.
\bibverse{23} Y le juró: Todo lo que me pidieres te daré, hasta la mitad
de mi reino. \footnote{\textbf{6:23} Est 5,3; Est 5,6}

\bibverse{24} Y saliendo ella, dijo á su madre: ¿Qué pediré? Y ella
dijo: La cabeza de Juan Bautista.

\bibverse{25} Entonces ella entró prestamente al rey, y pidió, diciendo:
Quiero que ahora mismo me des en un plato la cabeza de Juan Bautista.

\bibverse{26} Y el rey se entristeció mucho; mas á causa del juramento,
y de los que estaban con él á la mesa, no quiso desecharla.
\bibverse{27} Y luego el rey, enviando uno de la guardia, mandó que
fuese traída su cabeza; \bibverse{28} El cual fué, y le degolló en la
cárcel, y trajo su cabeza en un plato, y la dió á la muchacha, y la
muchacha la dió á su madre.

\bibverse{29} Y oyéndolo sus discípulos, vinieron y tomaron su cuerpo, y
le pusieron en un sepulcro.

\hypertarget{regreso-de-los-doce-apuxf3stoles-jesuxfas-escapa-a-la-soledad-alimentando-a-los-cinco-mil}{%
\subsection{Regreso de los doce apóstoles; Jesús escapa a la soledad;
Alimentando a los cinco
mil}\label{regreso-de-los-doce-apuxf3stoles-jesuxfas-escapa-a-la-soledad-alimentando-a-los-cinco-mil}}

\bibverse{30} Y los apóstoles se juntaron con Jesús, y le contaron todo
lo que habían hecho, y lo que habían enseñado. \bibverse{31} Y él les
dijo: Venid vosotros aparte al lugar desierto, y reposad un poco. Porque
eran muchos los que iban y venían, que ni aun tenían lugar de comer.
\bibverse{32} Y se fueron en un barco al lugar desierto aparte.
\bibverse{33} Y los vieron ir muchos, y le conocieron; y concurrieron
allá muchos á pie de las ciudades, y llegaron antes que ellos, y se
juntaron á él. \bibverse{34} Y saliendo Jesús vió grande multitud, y
tuvo compasión de ellos, porque eran como ovejas que no tenían pastor; y
les comenzó á enseñar muchas cosas. \footnote{\textbf{6:34} Mat 9,36}
\bibverse{35} Y como ya fuese el día muy entrado, sus discípulos
llegaron á él, diciendo: El lugar es desierto, y el día ya muy entrado;
\footnote{\textbf{6:35} Mar 8,1-9} \bibverse{36} Envíalos para que vayan
á los cortijos y aldeas de alrededor, y compren para sí pan; porque no
tienen qué comer.

\bibverse{37} Y respondiendo él, les dijo: Dadles de comer vosotros. Y
le dijeron: ¿Que vayamos y compremos pan por doscientos denarios, y les
demos de comer?

\bibverse{38} Y él les dice: ¿Cuántos panes tenéis? Id, y vedlo. Y
sabiéndolo, dijeron: Cinco, y dos peces.

\bibverse{39} Y les mandó que hiciesen recostar á todos por partidas
sobre la hierba verde. \bibverse{40} Y se recostaron por partidas, de
ciento en ciento, y de cincuenta en cincuenta. \bibverse{41} Y tomados
los cinco panes y los dos peces, mirando al cielo, bendijo, y partió los
panes, y dió á sus discípulos para que los pusiesen delante: y repartió
á todos los dos peces. \footnote{\textbf{6:41} Mar 7,34} \bibverse{42} Y
comieron todos, y se hartaron. \bibverse{43} Y alzaron de los pedazos
doce cofines llenos, y de los peces. \bibverse{44} Y los que comieron
eran cinco mil hombres.

\hypertarget{regrese-a-travuxe9s-del-lago-por-la-noche-el-caminar-de-jesuxfas-sobre-el-lago-el-desembarco-en-gennesaret}{%
\subsection{Regrese a través del lago por la noche; el caminar de Jesús
sobre el lago; el desembarco en
Gennesaret}\label{regrese-a-travuxe9s-del-lago-por-la-noche-el-caminar-de-jesuxfas-sobre-el-lago-el-desembarco-en-gennesaret}}

\bibverse{45} Y luego dió priesa á sus discípulos á subir en el barco, é
ir delante de él á Bethsaida de la otra parte, entre tanto que él
despedía la multitud. \bibverse{46} Y después que los hubo despedido, se
fué al monte á orar.

\bibverse{47} Y como fué la tarde, el barco estaba en medio de la mar, y
él solo en tierra. \bibverse{48} Y los vió fatigados bogando, porque el
viento les era contrario: y cerca de la cuarta vigilia de la noche, vino
á ellos andando sobre la mar, y quería precederlos. \bibverse{49} Y
viéndole ellos, que andaba sobre la mar, pensaron que era fantasma, y
dieron voces; \bibverse{50} Porque todos le veían, y se turbaron. Mas
luego habló con ellos, y les dijo: Alentaos; yo soy, no temáis.
\bibverse{51} Y subió á ellos en el barco, y calmó el viento: y ellos en
gran manera estaban fuera de sí, y se maravillaban: \bibverse{52} Porque
aun no habían considerado lo de los panes, por cuanto estaban ofuscados
sus corazones. \footnote{\textbf{6:52} Mar 8,17}

\bibverse{53} Y cuando estuvieron de la otra parte, vinieron á tierra de
Genezaret, y tomaron puerto. \bibverse{54} Y saliendo ellos del barco,
luego le conocieron. \bibverse{55} Y recorriendo toda la tierra de
alrededor, comenzaron á traer de todas partes enfermos en lechos, á
donde oían que estaba. \bibverse{56} Y donde quiera que entraba, en
aldeas, ó ciudades, ó heredades, ponían en las calles á los que estaban
enfermos, y le rogaban que tocasen siquiera el borde de su vestido; y
todos los que le tocaban quedaban sanos.

\hypertarget{pelea-con-los-oponentes-sobre-el-lavado-de-manos-advertencia-de-estatutos-humanos-y-marcado-de-verdadera-impureza}{%
\subsection{Pelea con los oponentes sobre el lavado de manos;
Advertencia de estatutos humanos y marcado de verdadera
impureza}\label{pelea-con-los-oponentes-sobre-el-lavado-de-manos-advertencia-de-estatutos-humanos-y-marcado-de-verdadera-impureza}}

\hypertarget{section-6}{%
\section{7}\label{section-6}}

\bibverse{1} Y se juntaron á él los Fariseos, y algunos de los escribas,
que habían venido de Jerusalem; \bibverse{2} Los cuales, viendo á
algunos de sus discípulos comer pan con manos comunes, es á saber, no
lavadas, los condenaban. \bibverse{3} (Porque los Fariseos y todos los
Judíos, teniendo la tradición de los ancianos, si muchas veces no se
lavan las manos, no comen. \bibverse{4} Y volviendo de la plaza, si no
se lavaren, no comen. Y otras muchas cosas hay, que tomaron para
guardar, como las lavaduras de los vasos de beber, y de los jarros, y de
los vasos de metal, y de los lechos.) \footnote{\textbf{7:4} Mat 23,25}
\bibverse{5} Y le preguntaron los Fariseos y los escribas: ¿Por qué tus
discípulos no andan conforme á la tradición de los ancianos, sino que
comen pan con manos comunes?

\bibverse{6} Y respondiendo él, les dijo: Hipócritas, bien profetizó de
vosotros Isaías, como está escrito: Este pueblo con los labios me honra,
mas su corazón lejos está de mí. \bibverse{7} Y en vano me honran,
enseñando como doctrinas mandamientos de hombres.

\bibverse{8} Porque dejando el mandamiento de Dios, tenéis la tradición
de los hombres; las lavaduras de los jarros y de los vasos de beber: y
hacéis otras muchas cosas semejantes. \bibverse{9} Les decía también:
Bien invalidáis el mandamiento de Dios para guardar vuestra tradición.
\bibverse{10} Porque Moisés dijo: Honra á tu padre y á tu madre, y: El
que maldijere al padre ó á la madre, morirá de muerte. \bibverse{11} Y
vosotros decís: Basta si dijere un hombre al padre ó la madre: Es Corbán
(quiere decir, don mío á Dios) todo aquello con que pudiera valerte;
\bibverse{12} Y no le dejáis hacer más por su padre ó por su madre,
\bibverse{13} Invalidando la palabra de Dios con vuestra tradición que
disteis: y muchas cosas hacéis semejantes á éstas.

\bibverse{14} Y llamando á toda la multitud, les dijo: Oidme todos, y
entended: \bibverse{15} Nada hay fuera del hombre que entre en él, que
le pueda contaminar: mas lo que sale de él, aquello es lo que contamina
al hombre. \bibverse{16} Si alguno tiene oídos para oir, oiga.

\bibverse{17} Y apartado de la multitud, habiendo entrado en casa, le
preguntaron sus discípulos sobre la parábola. \bibverse{18} Y díjoles:
¿También vosotros estáis así sin entendimiento? ¿No entendéis que todo
lo de fuera que entra en el hombre, no le puede contaminar;
\bibverse{19} Porque no entra en su corazón, sino en el vientre, y sale
á la secreta? Esto decía, haciendo limpias todas las viandas.
\bibverse{20} Mas decía, que lo que del hombre sale, aquello contamina
al hombre. \bibverse{21} Porque de dentro, del corazón de los hombres,
salen los malos pensamientos, los adulterios, las fornicaciones, los
homicidios, \bibverse{22} Los hurtos, las avaricias, las maldades, el
engaño, las desvergüenzas, el ojo maligno, las injurias, la soberbia, la
insensatez. \bibverse{23} Todas estas maldades de dentro salen, y
contaminan al hombre.

\hypertarget{jesuxfas-y-la-sirofenicia-en-el-uxe1rea-de-tiro-y-siduxf3n}{%
\subsection{Jesús y la sirofenicia en el área de Tiro y
Sidón}\label{jesuxfas-y-la-sirofenicia-en-el-uxe1rea-de-tiro-y-siduxf3n}}

\bibverse{24} Y levantándose de allí, se fué á los términos de Tiro y de
Sidón; y entrando en casa, quiso que nadie lo supiese; mas no pudo
esconderse. \bibverse{25} Porque una mujer, cuya hija tenía un espíritu
inmundo, luego que oyó de él, vino y se echó á sus pies. \bibverse{26} Y
la mujer era Griega, Sirofenisa de nación; y le rogaba que echase fuera
de su hija al demonio. \bibverse{27} Mas Jesús le dijo: Deja primero
hartarse los hijos, porque no es bien tomar el pan de los hijos y
echarlo á los perrillos.

\bibverse{28} Y respondió ella, y le dijo: Sí, Señor; pero aun los
perrillos debajo de la mesa, comen de las migajas de los hijos.

\bibverse{29} Entonces le dice: Por esta palabra, ve; el demonio ha
salido de tu hija.

\bibverse{30} Y como fué á su casa, halló que el demonio había salido, y
á la hija echada sobre la cama.

\hypertarget{el-regreso-de-jesuxfas-a-galilea-en-la-orilla-oriental-del-lago-sanando-a-un-sordomudo}{%
\subsection{El regreso de Jesús a Galilea en la orilla oriental del
lago; Sanando a un
sordomudo}\label{el-regreso-de-jesuxfas-a-galilea-en-la-orilla-oriental-del-lago-sanando-a-un-sordomudo}}

\bibverse{31} Y volviendo á salir de los términos de Tiro, vino por
Sidón á la mar de Galilea, por mitad de los términos de Decápolis.
\footnote{\textbf{7:31} Mar 5,20; Mat 15,29-31}

\bibverse{32} Y le traen un sordo y tartamudo, y le ruegan que le ponga
la mano encima. \bibverse{33} Y tomándole aparte de la gente, metió sus
dedos en las orejas de él, y escupiendo, tocó su lengua; \bibverse{34} Y
mirando al cielo, gimió, y le dijo: Ephphatha: que es decir: Sé abierto.
\bibverse{35} Y luego fueron abiertos sus oídos, y fué desatada la
ligadura de su lengua, y hablaba bien. \bibverse{36} Y les mandó que no
lo dijesen á nadie; pero cuanto más les mandaba, tanto más y más lo
divulgaban. \bibverse{37} Y en gran manera se maravillaban, diciendo:
Bien lo ha hecho todo: hace á los sordos oir, y á los mudos hablar.

\hypertarget{alimentando-a-los-cuatro-mil}{%
\subsection{Alimentando a los cuatro
mil}\label{alimentando-a-los-cuatro-mil}}

\hypertarget{section-7}{%
\section{8}\label{section-7}}

\bibverse{1} En aquellos días, como hubo gran gentío, y no tenían qué
comer, Jesús llamó á sus discípulos, y les dijo: \bibverse{2} Tengo
compasión de la multitud, porque ya hace tres días que están conmigo, y
no tienen qué comer: \footnote{\textbf{8:2} Mar 6,34-44} \bibverse{3} Y
si los enviare en ayunas á sus casas, desmayarán en el camino; porque
algunos de ellos han venido de lejos.

\bibverse{4} Y sus discípulos le respondieron: ¿De dónde podrá alguien
hartar á estos de pan aquí en el desierto?

\bibverse{5} Y les preguntó: ¿Cuántos panes tenéis? Y ellos dijeron:
Siete.

\bibverse{6} Entonces mandó á la multitud que se recostase en tierra; y
tomando los siete panes, habiendo dado gracias, partió, y dió á sus
discípulos que los pusiesen delante: y los pusieron delante á la
multitud. \bibverse{7} Tenían también unos pocos pececillos: y los
bendijo, y mandó que también los pusiesen delante. \bibverse{8} Y
comieron, y se hartaron: y levantaron de los pedazos que habían sobrado,
siete espuertas. \bibverse{9} Y eran los que comieron, como cuatro mil:
y los despidió.

\hypertarget{el-rechazo-de-jesuxfas-a-la-demanda-de-seuxf1ales-de-los-fariseos}{%
\subsection{El rechazo de Jesús a la demanda de señales de los
fariseos}\label{el-rechazo-de-jesuxfas-a-la-demanda-de-seuxf1ales-de-los-fariseos}}

\bibverse{10} Y luego entrando en el barco con sus discípulos, vino á
las partes de Dalmanutha. \bibverse{11} Y vinieron los Fariseos, y
comenzaron á altercar con él, pidiéndole señal del cielo, tentándole.
\bibverse{12} Y gimiendo en su espíritu, dice: ¿Por qué pide señal esta
generación? De cierto os digo que no se dará señal á esta generación.

\bibverse{13} Y dejándolos, volvió á entrar en el barco, y se fué de la
otra parte.

\hypertarget{advertencia-de-la-levadura-de-los-fariseos-y-la-de-herodes}{%
\subsection{Advertencia de la levadura de los fariseos y la de
Herodes}\label{advertencia-de-la-levadura-de-los-fariseos-y-la-de-herodes}}

\bibverse{14} Y se habían olvidado de tomar pan, y no tenían sino un pan
consigo en el barco. \bibverse{15} Y les mandó, diciendo: Mirad,
guardaos de la levadura de los Fariseos, y de la levadura de Herodes.
\footnote{\textbf{8:15} Luc 12,1; Mar 3,6}

\bibverse{16} Y altercaban los unos con los otros diciendo: Pan no
tenemos.

\bibverse{17} Y como Jesús lo entendió, les dice: ¿Qué altercáis, porque
no tenéis pan? ¿no consideráis ni entendéis? ¿aun tenéis endurecido
vuestro corazón? \bibverse{18} ¿Teniendo ojos no veis, y teniendo oídos
no oís? ¿y no os acordáis? \footnote{\textbf{8:18} Mat 13,13; Mat 13,16}
\bibverse{19} Cuando partí los cinco panes entre cinco mil, ¿cuántas
espuertas llenas de los pedazos alzasteis? Y ellos dijeron: Doce.
\footnote{\textbf{8:19} Mar 6,41-44}

\bibverse{20} Y cuando los siete panes entre cuatro mil, ¿cuántas
espuertas llenas de los pedazos alzasteis? Y ellos dijeron: Siete.

\bibverse{21} Y les dijo: ¿Cómo aun no entendéis?

\hypertarget{curaciuxf3n-de-ciegos-en-betsaida}{%
\subsection{Curación de ciegos en
Betsaida}\label{curaciuxf3n-de-ciegos-en-betsaida}}

\bibverse{22} Y vino á Bethsaida; y le traen un ciego, y le ruegan que
le tocase. \footnote{\textbf{8:22} Mar 6,56} \bibverse{23} Entonces,
tomando la mano del ciego, le sacó fuera de la aldea; y escupiendo en
sus ojos, y poniéndole las manos encima, le preguntó si veía algo.
\footnote{\textbf{8:23} Juan 9,6}

\bibverse{24} Y él mirando, dijo: Veo los hombres, pues veo que andan
como árboles.

\bibverse{25} Luego le puso otra vez las manos sobre sus ojos, y le hizo
que mirase; y fué restablecido, y vió de lejos y claramente á todos.
\bibverse{26} Y envióle á su casa, diciendo: No entres en la aldea, ni
lo digas á nadie en la aldea. \footnote{\textbf{8:26} Mar 7,36}

\hypertarget{la-confesiuxf3n-de-pedro-del-mesuxedas}{%
\subsection{La confesión de Pedro del
Mesías}\label{la-confesiuxf3n-de-pedro-del-mesuxedas}}

\bibverse{27} Y salió Jesús y sus discípulos por las aldeas de Cesarea
de Filipo. Y en el camino preguntó á sus discípulos, diciéndoles: ¿Quién
dicen los hombres que soy yo?

\bibverse{28} Y ellos respondieron: Juan Bautista; y otros, Elías; y
otros, Alguno de los profetas.

\bibverse{29} Entonces él les dice: Y vosotros, ¿quién decís que soy yo?
Y respondiendo Pedro, le dice: Tú eres el Cristo.

\bibverse{30} Y les apercibió que no hablasen de él á ninguno.
\footnote{\textbf{8:30} Mar 9,9}

\hypertarget{el-primer-anuncio-del-sufrimiento-de-jesuxfas}{%
\subsection{El primer anuncio del sufrimiento de
Jesús}\label{el-primer-anuncio-del-sufrimiento-de-jesuxfas}}

\bibverse{31} Y comenzó á enseñarles, que convenía que el Hijo del
hombre padeciese mucho, y ser reprobado de los ancianos, y de los
príncipes de los sacerdotes, y de los escribas, y ser muerto, y
resucitar después de tres días. \bibverse{32} Y claramente decía esta
palabra. Entonces Pedro le tomó, y le comenzó á reprender. \bibverse{33}
Y él, volviéndose y mirando á sus discípulos, riñó á Pedro, diciendo:
Apártate de mí, Satanás; porque no sabes las cosas que son de Dios, sino
las que son de los hombres.

\hypertarget{proverbios-sobre-el-seguimiento-de-los-discuxedpulos-en-el-sufrimiento}{%
\subsection{Proverbios sobre el seguimiento de los discípulos en el
sufrimiento}\label{proverbios-sobre-el-seguimiento-de-los-discuxedpulos-en-el-sufrimiento}}

\bibverse{34} Y llamando á la gente con sus discípulos, les dijo:
Cualquiera que quisiere venir en pos de mí, niéguese á sí mismo, y tome
su cruz, y sígame. \bibverse{35} Porque el que quisiere salvar su vida,
la perderá; y el que perdiere su vida por causa de mí y del evangelio,
la salvará. \bibverse{36} Porque ¿qué aprovechará al hombre, si
granjeare todo el mundo, y pierde su alma? \bibverse{37} ¿O qué
recompensa dará el hombre por su alma? \bibverse{38} Porque el que se
avergonzare de mí y de mis palabras en esta generación adulterina y
pecadora, el Hijo del hombre se avergonzará también de él, cuando vendrá
en la gloria de su Padre con los santos ángeles. \footnote{\textbf{8:38}
  Mat 10,33}

\hypertarget{section-8}{%
\section{9}\label{section-8}}

\bibverse{1} También les dijo: De cierto os digo que hay algunos de los
que están aquí, que no gustarán la muerte hasta que hayan visto el reino
de Dios que viene con potencia.

\hypertarget{la-transfiguraciuxf3n-de-jesuxfas-en-la-montauxf1a-y-su-conversaciuxf3n-con-los-discuxedpulos-en-el-descenso}{%
\subsection{La transfiguración de Jesús en la montaña y su conversación
con los discípulos en el
descenso}\label{la-transfiguraciuxf3n-de-jesuxfas-en-la-montauxf1a-y-su-conversaciuxf3n-con-los-discuxedpulos-en-el-descenso}}

\bibverse{2} Y seis días después tomó Jesús á Pedro, y á Jacobo, y á
Juan, y los sacó aparte solos á un monte alto; y fué transfigurado
delante de ellos. \bibverse{3} Y sus vestidos se volvieron
resplandecientes, muy blancos, como la nieve; tanto que ningún lavador
en la tierra los puede hacer tan blancos. \bibverse{4} Y les apareció
Elías con Moisés, que hablaban con Jesús.

\bibverse{5} Entonces respondiendo Pedro, dice á Jesús: Maestro, bien
será que nos quedemos aquí, y hagamos tres pabellones: para ti uno, y
para Moisés otro, y para Elías otro; \bibverse{6} Porque no sabía lo que
hablaba; que estaban espantados.

\bibverse{7} Y vino una nube que les hizo sombra, y una voz de la nube,
que decía: Este es mi Hijo amado: á él oid.

\bibverse{8} Y luego, como miraron, no vieron más á nadie consigo, sino
á Jesús solo.

\bibverse{9} Y descendiendo ellos del monte, les mandó que á nadie
dijesen lo que habían visto, sino cuando el Hijo del hombre hubiese
resucitado de los muertos. \footnote{\textbf{9:9} Mar 8,30}
\bibverse{10} Y retuvieron la palabra en sí, altercando qué sería
aquéllo: Resucitar de los muertos.

\bibverse{11} Y le preguntaron, diciendo: ¿Qué es lo que los escribas
dicen, que es necesario que Elías venga antes?

\bibverse{12} Y respondiendo él, les dijo: Elías á la verdad, viniendo
antes, restituirá todas las cosas: y como está escrito del Hijo del
hombre, que padezca mucho y sea tenido en nada. \bibverse{13} Empero os
digo que Elías ya vino, y le hicieron todo lo que quisieron, como está
escrito de él. \footnote{\textbf{9:13} Mat 11,14; 1Re 19,2; 1Re 19,10}

\hypertarget{curaciuxf3n-de-un-niuxf1o-epiluxe9ptico-la-incapacidad-de-los-discuxedpulos}{%
\subsection{Curación de un niño epiléptico; la incapacidad de los
discípulos}\label{curaciuxf3n-de-un-niuxf1o-epiluxe9ptico-la-incapacidad-de-los-discuxedpulos}}

\bibverse{14} Y como vino á los discípulos, vió grande compañía
alrededor de ellos, y escribas que disputaban con ellos. \bibverse{15} Y
luego toda la gente, viéndole, se espantó, y corriendo á él, le
saludaron. \bibverse{16} Y preguntóles: ¿Qué disputáis con ellos?

\bibverse{17} Y respondiendo uno de la compañía, dijo: Maestro, traje á
ti mi hijo, que tiene un espíritu mudo, \bibverse{18} El cual, donde
quiera que le toma, le despedaza; y echa espumarajos, y cruje los
dientes, y se va secando: y dije á tus discípulos que le echasen fuera,
y no pudieron.

\bibverse{19} Y respondiendo él, les dijo: ¡Oh generación infiel! ¿hasta
cuándo estaré con vosotros? ¿hasta cuándo os tengo de sufrir? Traédmele.

\bibverse{20} Y se le trajeron: y como le vió, luego el espíritu le
desgarraba; y cayendo en tierra, se revolcaba, echando espumarajos.

\bibverse{21} Y Jesús preguntó á su padre: ¿Cuánto tiempo há que le
aconteció esto? Y él dijo: Desde niño:

\bibverse{22} Y muchas veces le echa en el fuego y en aguas, para
matarle; mas, si puedes algo, ayúdanos, teniendo misericordia de
nosotros.

\bibverse{23} Y Jesús le dijo: Si puedes creer, al que cree todo es
posible.

\bibverse{24} Y luego el padre del muchacho dijo clamando: Creo, ayuda
mi incredulidad.

\bibverse{25} Y como Jesús vió que la multitud se agolpaba, reprendió al
espíritu inmundo, diciéndole: Espíritu mudo y sordo, yo te mando, sal de
él, y no entres más en él.

\bibverse{26} Entonces el espíritu clamando y desgarrándole mucho,
salió; y él quedó como muerto, de modo que muchos decían: Está muerto.
\bibverse{27} Mas Jesús tomándole de la mano, enderezóle; y se levantó.

\bibverse{28} Y como él entró en casa, sus discípulos le preguntaron
aparte: ¿Por qué nosotros no pudimos echarle fuera?

\bibverse{29} Y les dijo: Este género con nada puede salir, sino con
oración y ayuno.

\hypertarget{segundo-anuncio-de-sufrimiento}{%
\subsection{Segundo anuncio de
sufrimiento}\label{segundo-anuncio-de-sufrimiento}}

\bibverse{30} Y habiendo salido de allí, caminaron por Galilea; y no
quería que nadie lo supiese. \bibverse{31} Porque enseñaba á sus
discípulos, y les decía: El Hijo del hombre será entregado en manos de
hombres, y le matarán; mas muerto él, resucitará al tercer día.
\footnote{\textbf{9:31} Mar 8,31; Mar 10,32-34}

\bibverse{32} Pero ellos no entendían esta palabra, y tenían miedo de
preguntarle. \footnote{\textbf{9:32} Luc 18,34}

\hypertarget{controversia-entre-discuxedpulos-la-exhortaciuxf3n-de-jesuxfas-a-la-humildad}{%
\subsection{Controversia entre discípulos; La exhortación de Jesús a la
humildad}\label{controversia-entre-discuxedpulos-la-exhortaciuxf3n-de-jesuxfas-a-la-humildad}}

\bibverse{33} Y llegó á Capernaum; y así que estuvo en casa, les
preguntó: ¿Qué disputabais entre vosotros en el camino?

\bibverse{34} Mas ellos callaron; porque los unos con los otros habían
disputado en el camino quién había de ser el mayor.

\bibverse{35} Entonces sentándose, llamó á los doce, y les dice: Si
alguno quiere ser el primero, será el postrero de todos, y el servidor
de todos. \footnote{\textbf{9:35} Mar 10,44; Mat 20,27} \bibverse{36} Y
tomando un niño, púsolo en medio de ellos; y tomándole en sus brazos,
les dice: \bibverse{37} El que recibiere en mi nombre uno de los tales
niños, á mí recibe; y el que á mí recibe, no recibe á mí, mas al que me
envió.

\hypertarget{enseuxf1ar-sobre-la-tolerancia}{%
\subsection{Enseñar sobre la
tolerancia}\label{enseuxf1ar-sobre-la-tolerancia}}

\bibverse{38} Y respondióle Juan, diciendo: Maestro, hemos visto á uno
que en tu nombre echaba fuera los demonios, el cual no nos sigue; y se
lo prohibimos, porque no nos sigue. \footnote{\textbf{9:38} Núm 11,27-28}

\bibverse{39} Y Jesús dijo: No se lo prohibáis; porque ninguno hay que
haga milagro en mi nombre que luego pueda decir mal de mí. \footnote{\textbf{9:39}
  1Cor 12,3} \bibverse{40} Porque el que no es contra nosotros, por
nosotros es. \footnote{\textbf{9:40} Mat 12,30; Luc 11,23} \bibverse{41}
Y cualquiera que os diere un vaso de agua en mi nombre, porque sois de
Cristo, de cierto os digo que no perderá su recompensa. \footnote{\textbf{9:41}
  Mat 10,42}

\hypertarget{advertencia-de-engauxf1o-a-la-incredulidad-y-al-pecado-dichos-de-sal}{%
\subsection{Advertencia de engaño (a la incredulidad y al pecado);
Dichos de
sal}\label{advertencia-de-engauxf1o-a-la-incredulidad-y-al-pecado-dichos-de-sal}}

\bibverse{42} Y cualquiera que escandalizare á uno de estos pequeñitos
que creen en mí, mejor le fuera si se le atase una piedra de molino al
cuello, y fuera echado en la mar. \bibverse{43} Y si tu mano te
escandalizare, córtala: mejor te es entrar á la vida manco, que teniendo
dos manos ir á la Gehenna, al fuego que no puede ser apagado;
\footnote{\textbf{9:43} Mat 5,30} \bibverse{44} Donde su gusano no
muere, y el fuego nunca se apaga. \bibverse{45} Y si tu pie te fuere
ocasión de caer, córtalo: mejor te es entrar á la vida cojo, que
teniendo dos pies ser echado en la Gehenna, al fuego que no puede ser
apagado; \bibverse{46} Donde el gusano de ellos no muere, y el fuego
nunca se apaga. \bibverse{47} Y si tu ojo te fuere ocasión de caer,
sácalo: mejor te es entrar al reino de Dios con un ojo, que teniendo dos
ojos ser echado á la Gehenna; \bibverse{48} Donde el gusano de ellos no
muere, y el fuego nunca se apaga. \bibverse{49} Porque todos serán
salados con fuego, y todo sacrificio será salado con sal. \footnote{\textbf{9:49}
  Lev 2,13} \bibverse{50} Buena es la sal; mas si la sal fuere
desabrida, ¿con qué la adobaréis? Tened en vosotros mismos sal; y tened
paz los unos con los otros. \footnote{\textbf{9:50} Mat 5,13; Luc 14,34;
  Col 4,6}

\hypertarget{jesuxfas-en-judea-y-transjordania-conversaciones-sobre-matrimonio-y-divorcio}{%
\subsection{Jesús en Judea y Transjordania; Conversaciones sobre
matrimonio y
divorcio}\label{jesuxfas-en-judea-y-transjordania-conversaciones-sobre-matrimonio-y-divorcio}}

\hypertarget{section-9}{%
\section{10}\label{section-9}}

\bibverse{1} Y partiéndose de allí, vino á los términos de Judea y tras
el Jordán: y volvió el pueblo á juntarse á él; y de nuevo les enseñaba
como solía.

\bibverse{2} Y llegándose los Fariseos, le preguntaron, para tentarle,
si era lícito al marido repudiar á su mujer.

\bibverse{3} Mas él respondiendo, les dijo: ¿Qué os mandó Moisés?

\bibverse{4} Y ellos dijeron: Moisés permitió escribir carta de
divorcio, y repudiar. \footnote{\textbf{10:4} Deut 24,1; Mat 5,31-32}

\bibverse{5} Y respondiendo Jesús, les dijo: Por la dureza de vuestro
corazón os escribió este mandamiento; \bibverse{6} Pero al principio de
la creación, varón y hembra los hizo Dios. \bibverse{7} Por esto dejará
el hombre á su padre y á su madre, y se juntará á su mujer. \footnote{\textbf{10:7}
  Gén 2,24} \bibverse{8} Y los que eran dos, serán hechos una carne: así
que no son más dos, sino una carne. \bibverse{9} Pues lo que Dios juntó,
no lo aparte el hombre.

\bibverse{10} Y en casa volvieron los discípulos á preguntarle de lo
mismo. \bibverse{11} Y les dice: Cualquiera que repudiare á su mujer, y
se casare con otra, comete adulterio contra ella: \bibverse{12} Y si la
mujer repudiare á su marido y se casare con otro, comete adulterio.

\hypertarget{jesuxfas-bendice-a-los-niuxf1os}{%
\subsection{Jesús bendice a los
niños}\label{jesuxfas-bendice-a-los-niuxf1os}}

\bibverse{13} Y le presentaban niños para que los tocase; y los
discípulos reñían á los que los presentaban. \bibverse{14} Y viéndolo
Jesús, se enojó, y les dijo: Dejad los niños venir, y no se lo
estorbéis; porque de los tales es el reino de Dios. \bibverse{15} De
cierto os digo, que el que no recibiere el reino de Dios como un niño,
no entrará en él. \footnote{\textbf{10:15} Mat 18,3} \bibverse{16} Y
tomándolos en los brazos, poniendo las manos sobre ellos, los bendecía.
\footnote{\textbf{10:16} Mar 9,36}

\hypertarget{la-conversaciuxf3n-de-jesuxfas-con-los-ricos-y-su-referencia-al-peligro-de-las-riquezas}{%
\subsection{La conversación de Jesús con los ricos y su referencia al
peligro de las
riquezas}\label{la-conversaciuxf3n-de-jesuxfas-con-los-ricos-y-su-referencia-al-peligro-de-las-riquezas}}

\bibverse{17} Y saliendo él para ir su camino, vino uno corriendo, é
hincando la rodilla delante de él, le preguntó: Maestro bueno, ¿qué haré
para poseer la vida eterna?

\bibverse{18} Y Jesús le dijo: ¿Por qué me dices bueno? Ninguno hay
bueno, sino sólo uno, Dios. \bibverse{19} Los mandamientos sabes: No
adulteres: No mates: No hurtes: No digas falso testimonio: No defraudes:
Honra á tu padre y á tu madre. \footnote{\textbf{10:19} Éxod 20,12-17}

\bibverse{20} El entonces respondiendo, le dijo: Maestro, todo esto he
guardado desde mi mocedad.

\bibverse{21} Entonces Jesús mirándole, amóle, y díjole: Una cosa te
falta: ve, vende todo lo que tienes, y da á los pobres, y tendrás tesoro
en el cielo; y ven, sígueme, tomando tu cruz.

\bibverse{22} Mas él, entristecido por esta palabra, se fué triste,
porque tenía muchas posesiones.

\bibverse{23} Entonces Jesús, mirando alrededor, dice á sus discípulos:
¡Cuán difícilmente entrarán en el reino de Dios los que tienen riquezas!

\bibverse{24} Y los discípulos se espantaron de sus palabras; mas Jesús
respondiendo, les volvió á decir: ¡Hijos, cuán difícil es entrar en el
reino de Dios, los que confían en las riquezas! \footnote{\textbf{10:24}
  Sal 62,11; 1Tim 6,17} \bibverse{25} Más fácil es pasar un camello por
el ojo de una aguja, que el rico entrar en el reino de Dios.

\bibverse{26} Y ellos se espantaban más, diciendo dentro de sí: ¿Y quién
podrá salvarse?

\bibverse{27} Entonces Jesús mirándolos, dice: Para los hombres es
imposible; mas para Dios, no; porque todas las cosas son posibles para
Dios.

\hypertarget{la-recompensa-de-seguir-a-jesuxfas-y-la-renuncia}{%
\subsection{La recompensa de seguir a Jesús y la
renuncia}\label{la-recompensa-de-seguir-a-jesuxfas-y-la-renuncia}}

\bibverse{28} Entonces Pedro comenzó á decirle: He aquí, nosotros hemos
dejado todas las cosas, y te hemos seguido.

\bibverse{29} Y respondiendo Jesús, dijo: De cierto os digo, que no hay
ninguno que haya dejado casa, ó hermanos, ó hermanas, ó padre, ó madre,
ó mujer, ó hijos, ó heredades, por causa de mí y del evangelio,
\bibverse{30} Que no reciba cien tantos ahora en este tiempo, casas, y
hermanos, y hermanas, y madres, é hijos, y heredades, con persecuciones;
y en el siglo venidero la vida eterna. \bibverse{31} Empero muchos
primeros serán postreros, y postreros primeros.

\hypertarget{salida-hacia-jerusaluxe9n-tercer-anuncio-del-sufrimiento-de-jesuxfas}{%
\subsection{Salida hacia Jerusalén; tercer anuncio del sufrimiento de
Jesús}\label{salida-hacia-jerusaluxe9n-tercer-anuncio-del-sufrimiento-de-jesuxfas}}

\bibverse{32} Y estaban en el camino subiendo á Jerusalem; y Jesús iba
delante de ellos, y se espantaban, y le seguían con miedo: entonces
volviendo á tomar á los doce aparte, les comenzó á decir las cosas que
le habían de acontecer: \bibverse{33} He aquí subimos á Jerusalem, y el
Hijo del hombre será entregado á los príncipes de los sacerdotes, y á
los escribas, y le condenarán á muerte, y le entregarán á los Gentiles:
\bibverse{34} Y le escarnecerán, y le azotarán, y escupirán en él, y le
matarán; mas al tercer día resucitará.

\hypertarget{solicitud-ambiciosa-de-los-dos-hijos-de-zebedeo}{%
\subsection{Solicitud ambiciosa de los dos hijos de
Zebedeo}\label{solicitud-ambiciosa-de-los-dos-hijos-de-zebedeo}}

\bibverse{35} Entonces Jacobo y Juan, hijos de Zebedeo, se llegaron á
él, diciendo: Maestro, querríamos que nos hagas lo que pidiéremos.

\bibverse{36} Y él les dijo: ¿Qué queréis que os haga?

\bibverse{37} Y ellos le dijeron: Danos que en tu gloria nos sentemos el
uno á tu diestra, y el otro á tu siniestra.

\bibverse{38} Entonces Jesús les dijo: No sabéis lo que pedís. ¿Podéis
beber del vaso que yo bebo, ó ser bautizados del bautismo de que yo soy
bautizado? \footnote{\textbf{10:38} Mar 14,36; Luc 12,50}

\bibverse{39} Y ellos dijeron: Podemos. Y Jesús les dijo: A la verdad,
del vaso que yo bebo, beberéis; y del bautismo de que yo soy bautizado,
seréis bautizados. \footnote{\textbf{10:39} Hech 12,2; Apoc 1,9}

\bibverse{40} Mas que os sentéis á mi diestra y á mi siniestra, no es
mío darlo, sino á quienes está aparejado.

\bibverse{41} Y como lo oyeron los diez, comenzaron á enojarse de Jacobo
y de Juan.

\bibverse{42} Mas Jesús, llamándolos, les dice: Sabéis que los que se
ven ser príncipes entre las gentes, se enseñorean de ellas, y los que
entre ellas son grandes, tienen sobre ellas potestad. \footnote{\textbf{10:42}
  Luc 22,25-27} \bibverse{43} Mas no será así entre vosotros: antes
cualquiera que quisiere hacerse grande entre vosotros, será vuestro
servidor; \footnote{\textbf{10:43} Mar 9,35; 1Pe 5,3} \bibverse{44} Y
cualquiera de vosotros que quisiere hacerse el primero, será siervo de
todos. \bibverse{45} Porque el Hijo del hombre tampoco vino para ser
servido, mas para servir, y dar su vida en rescate por muchos.

\hypertarget{curaciuxf3n-del-ciego-bartimeo-cerca-de-jericuxf3}{%
\subsection{Curación del ciego Bartimeo cerca de
Jericó}\label{curaciuxf3n-del-ciego-bartimeo-cerca-de-jericuxf3}}

\bibverse{46} Entonces vienen á Jericó: y saliendo él de Jericó y sus
discípulos y una gran compañía, Bartimeo el ciego, hijo de Timeo, estaba
sentado junto al camino mendigando. \bibverse{47} Y oyendo que era Jesús
el Nazareno, comenzó á dar voces y decir: Jesús, Hijo de David, ten
misericordia de mí. \bibverse{48} Y muchos le reñían, que callase: mas
él daba mayores voces: Hijo de David, ten misericordia de mí.

\bibverse{49} Entonces Jesús parándose, mandó llamarle: y llaman al
ciego, diciéndole: Ten confianza: levántate, te llama.

\bibverse{50} El entonces, echando su capa, se levantó, y vino á Jesús.

\bibverse{51} Y respondiendo Jesús, le dice: ¿Qué quieres que te haga? Y
el ciego le dice: Maestro, que cobre la vista.

\bibverse{52} Y Jesús le dijo: Ve, tu fe te ha salvado. Y luego cobró la
vista, y seguía á Jesús en el camino.

\hypertarget{la-entrada-de-jesuxfas-a-jerusaluxe9n}{%
\subsection{La entrada de Jesús a
Jerusalén}\label{la-entrada-de-jesuxfas-a-jerusaluxe9n}}

\hypertarget{section-10}{%
\section{11}\label{section-10}}

\bibverse{1} Y como fueron cerca de Jerusalem, de Bethphagé, y de
Bethania, al monte de las Olivas, envía dos de sus discípulos,
\footnote{\textbf{11:1} Juan 2,13} \bibverse{2} Y les dice: Id al lugar
que está delante de vosotros, y luego entrados en él, hallaréis un
pollino atado, sobre el cual ningún hombre ha subido; desatadlo y
traedlo. \bibverse{3} Y si alguien os dijere: ¿Por qué hacéis eso? decid
que el Señor lo ha menester: y luego lo enviará acá.

\bibverse{4} Y fueron, y hallaron el pollino atado á la puerta fuera,
entre dos caminos; y le desataron. \bibverse{5} Y unos de los que
estaban allí, les dijeron: ¿Qué hacéis desatando el pollino?
\bibverse{6} Ellos entonces les dijeron como Jesús había mandado: y los
dejaron.

\bibverse{7} Y trajeron el pollino á Jesús, y echaron sobre él sus
vestidos, y se sentó sobre él. \bibverse{8} Y muchos tendían sus
vestidos por el camino, y otros cortaban hojas de los árboles, y las
tendían por el camino. \bibverse{9} Y los que iban delante, y los que
iban detrás, daban voces diciendo: ¡Hosanna! Bendito el que viene en el
nombre del Señor. \bibverse{10} Bendito el reino de nuestro padre David
que viene: ¡Hosanna en las alturas!

\bibverse{11} Y entró Jesús en Jerusalem, y en el templo: y habiendo
mirado alrededor todas las cosas, y siendo ya tarde, salióse á Bethania
con los doce.

\hypertarget{la-maldiciuxf3n-de-una-higuera-estuxe9ril}{%
\subsection{La maldición de una higuera
estéril}\label{la-maldiciuxf3n-de-una-higuera-estuxe9ril}}

\bibverse{12} Y el día siguiente, como salieron de Bethania, tuvo
hambre. \bibverse{13} Y viendo de lejos una higuera que tenía hojas, se
acercó, si quizá hallaría en ella algo: y como vino á ella, nada halló
sino hojas; porque no era tiempo de higos. \bibverse{14} Entonces Jesús
respondiendo, dijo á la higuera: Nunca más coma nadie fruto de ti para
siempre. Y lo oyeron sus discípulos.

\hypertarget{la-limpieza-del-templo}{%
\subsection{La limpieza del templo}\label{la-limpieza-del-templo}}

\bibverse{15} Vienen, pues, á Jerusalem; y entrando Jesús en el templo,
comenzó á echar fuera á los que vendían y compraban en el templo; y
trastornó las mesas de los cambistas, y las sillas de los que vendían
palomas; \footnote{\textbf{11:15} Juan 2,14-16} \bibverse{16} Y no
consentía que alguien llevase vaso por el templo. \bibverse{17} Y les
enseñaba diciendo: ¿No está escrito que mi casa, casa de oración será
llamada por todas las gentes? Mas vosotros la habéis hecho cueva de
ladrones.

\bibverse{18} Y lo oyeron los escribas y los príncipes de los
sacerdotes, y procuraban cómo le matarían; porque le tenían miedo, por
cuanto todo el pueblo estaba maravillado de su doctrina.

\bibverse{19} Mas como fué tarde, Jesús salió de la ciudad.

\hypertarget{repaso-de-la-higuera-seca-con-posterior-referencia-al-poder-de-la-fe-y-la-oraciuxf3n-advertencia}{%
\subsection{Repaso de la higuera seca con posterior referencia al poder
de la fe y la oración;
advertencia}\label{repaso-de-la-higuera-seca-con-posterior-referencia-al-poder-de-la-fe-y-la-oraciuxf3n-advertencia}}

\bibverse{20} Y pasando por la mañana, vieron que la higuera se había
secado desde las raíces. \bibverse{21} Entonces Pedro acordándose, le
dice: Maestro, he aquí la higuera que maldijiste, se ha secado.

\bibverse{22} Y respondiendo Jesús, les dice: Tened fe en Dios.
\bibverse{23} Porque de cierto os digo que cualquiera que dijere á este
monte: Quítate, y échate en la mar, y no dudare en su corazón, mas
creyere que será hecho lo que dice, lo que dijere le será hecho.
\footnote{\textbf{11:23} Mar 9,23; Mat 17,20} \bibverse{24} Por tanto,
os digo que todo lo que orando pidiereis, creed que lo recibiréis, y os
vendrá. \footnote{\textbf{11:24} Mat 7,7; Juan 14,13; 1Jn 5,14; 1Jn
  1,5-15} \bibverse{25} Y cuando estuviereis orando, perdonad, si tenéis
algo contra alguno, para que vuestro Padre que está en los cielos os
perdone también á vosotros vuestras ofensas. \footnote{\textbf{11:25}
  Mat 5,23} \bibverse{26} Porque si vosotros no perdonareis, tampoco
vuestro Padre que está en los cielos os perdonará vuestras ofensas.
\footnote{\textbf{11:26} Mat 6,14-15}

\hypertarget{la-pregunta-del-sumo-consejo-sobre-la-autoridad-de-jesuxfas}{%
\subsection{La pregunta del sumo consejo sobre la autoridad de
Jesús}\label{la-pregunta-del-sumo-consejo-sobre-la-autoridad-de-jesuxfas}}

\bibverse{27} Y volvieron á Jerusalem; y andando él por el templo,
vienen á él los príncipes de los sacerdotes, y los escribas, y los
ancianos; \bibverse{28} Y le dicen: ¿Con qué facultad haces estas cosas?
¿y quién te ha dado esta facultad para hacer estas cosas?

\bibverse{29} Y Jesús respondiendo entonces, les dice: Os preguntaré
también yo una palabra; y respondedme, y os diré con qué facultad hago
estas cosas: \bibverse{30} El bautismo de Juan, ¿era del cielo, ó de los
hombres? Respondedme.

\bibverse{31} Entonces ellos pensaron dentro de sí, diciendo: Si
dijéremos, del cielo, dirá: ¿Por qué, pues, no le creísteis?
\bibverse{32} Y si dijéremos, de los hombres, tememos al pueblo: porque
todos juzgaban de Juan, que verdaderamente era profeta. \footnote{\textbf{11:32}
  Luc 7,29-30}

\bibverse{33} Y respondiendo, dicen á Jesús: No sabemos. Entonces
respondiendo Jesús, les dice: Tampoco yo os diré con qué facultad hago
estas cosas.

\hypertarget{paruxe1bola-de-los-viticultores-infieles}{%
\subsection{Parábola de los viticultores
infieles}\label{paruxe1bola-de-los-viticultores-infieles}}

\hypertarget{section-11}{%
\section{12}\label{section-11}}

\bibverse{1} Y comenzó á hablarles por parábolas: Plantó un hombre una
viña, y la cercó con seto, y cavó un lagar, y edificó una torre, y la
arrendó á labradores, y se partió lejos. \bibverse{2} Y envió un siervo
á los labradores, al tiempo, para que tomase de los labradores del fruto
de la viña. \bibverse{3} Mas ellos, tomándole, le hirieron, y le
enviaron vacío. \bibverse{4} Y volvió á enviarles otro siervo; mas
apedreándole, le hirieron en la cabeza, y volvieron á enviarle
afrentado. \bibverse{5} Y volvió á enviar otro, y á aquél mataron; y á
otros muchos, hiriendo á unos y matando á otros. \bibverse{6} Teniendo
pues aún un hijo suyo amado, enviólo también á ellos el postrero,
diciendo: Tendrán en reverencia á mi hijo. \bibverse{7} Mas aquellos
labradores dijeron entre sí: Este es el heredero; venid, matémosle, y la
heredad será nuestra. \bibverse{8} Y prendiéndole, le mataron, y echaron
fuera de la viña. \footnote{\textbf{12:8} Heb 13,12} \bibverse{9} ¿Qué,
pues, hará el señor de la viña? Vendrá, y destruirá á estos labradores,
y dará su viña á otros. \bibverse{10} ¿Ni aun esta Escritura habéis
leído: La piedra que desecharon los que edificaban, ésta es puesta por
cabeza de esquina; \bibverse{11} Por el Señor es hecho esto, y es cosa
maravillosa en nuestros ojos?

\bibverse{12} Y procuraban prenderle, porque entendían que decía á ellos
aquella parábola; mas temían á la multitud; y dejándole, se fueron.

\hypertarget{la-cuestiuxf3n-fiscal-de-los-fariseos}{%
\subsection{La cuestión fiscal de los
fariseos}\label{la-cuestiuxf3n-fiscal-de-los-fariseos}}

\bibverse{13} Y envían á él algunos de los Fariseos y de los Herodianos,
para que le sorprendiesen en alguna palabra. \bibverse{14} Y viniendo
ellos, le dicen: Maestro, sabemos que eres hombre de verdad, y que no te
cuidas de nadie; porque no miras á la apariencia de hombres, antes con
verdad enseñas el camino de Dios: ¿Es lícito dar tributo á César, ó no?
¿Daremos, ó no daremos? \bibverse{15} Entonces él, como entendía la
hipocresía de ellos, les dijo: ¿Por qué me tentáis? Traedme la moneda
para que la vea.

\bibverse{16} Y ellos se la trajeron y les dice: ¿Cúya es esta imagen y
esta inscripción? Y ellos le dijeron: De César.

\bibverse{17} Y respondiendo Jesús, les dijo: Dad lo que es de César á
César; y lo que es de Dios, á Dios. Y se maravillaron de ello.

\hypertarget{la-pregunta-sobre-la-resurrecciuxf3n-de-los-muertos}{%
\subsection{La pregunta sobre la resurrección de los
muertos}\label{la-pregunta-sobre-la-resurrecciuxf3n-de-los-muertos}}

\bibverse{18} Entonces vienen á él los Saduceos, que dicen que no hay
resurrección, y le preguntaron, diciendo: \bibverse{19} Maestro, Moisés
nos escribió, que si el hermano de alguno muriese, y dejase mujer, y no
dejase hijos, que su hermano tome su mujer, y levante linaje á su
hermano. \bibverse{20} Fueron siete hermanos: y el primero tomó mujer, y
muriendo, no dejó simiente; \bibverse{21} Y la tomó el segundo, y murió,
y ni aquél tampoco dejó simiente; y el tercero, de la misma manera.
\bibverse{22} Y la tomaron los siete, y tampoco dejaron simiente: á la
postre murió también la mujer. \bibverse{23} En la resurrección, pues,
cuando resucitaren, ¿de cuál de ellos será mujer? porque los siete la
tuvieron por mujer.

\bibverse{24} Entonces respondiendo Jesús, les dice: ¿No erráis por eso,
porque no sabéis las Escrituras, ni la potencia de Dios? \bibverse{25}
Porque cuando resucitarán de los muertos, ni se casarán, ni serán dados
en casamiento, mas son como los ángeles que están en los cielos.
\bibverse{26} Y de que los muertos hayan de resucitar, ¿no habéis leído
en el libro de Moisés cómo le habló Dios en la zarza, diciendo: Yo soy
el Dios de Abraham, y el Dios de Isaac, y el Dios de Jacob?
\bibverse{27} No es Dios de muertos, mas Dios de vivos; así que vosotros
mucho erráis.

\hypertarget{la-pregunta-de-un-escriba-sobre-el-mandamiento-muxe1s-noble}{%
\subsection{La pregunta de un escriba sobre el mandamiento más
noble}\label{la-pregunta-de-un-escriba-sobre-el-mandamiento-muxe1s-noble}}

\bibverse{28} Y llegándose uno de los escribas, que los había oído
disputar, y sabía que les había respondido bien, le preguntó: ¿Cuál es
el primer mandamiento de todos?

\bibverse{29} Y Jesús le respondió: El primer mandamiento de todos es:
Oye, Israel, el Señor nuestro Dios, el Señor uno es. \bibverse{30}
Amarás pues al Señor tu Dios de todo tu corazón, y de toda tu alma, y de
toda tu mente, y de todas tus fuerzas; este es el principal mandamiento.
\bibverse{31} Y el segundo es semejante á él: Amarás á tu prójimo como á
ti mismo. No hay otro mandamiento mayor que éstos.

\bibverse{32} Entonces el escriba le dijo: Bien, Maestro, verdad has
dicho, que uno es Dios, y no hay otro fuera de él; \bibverse{33} Y que
amarle de todo corazón, y de todo entendimiento, y de toda el alma, y de
todas las fuerzas, y amar al prójimo como á sí mismo, más es que todos
los holocaustos y sacrificios. \footnote{\textbf{12:33} 1Sam 15,22; Os
  6,6}

\bibverse{34} Jesús entonces, viendo que había respondido sabiamente, le
dice: No estás lejos del reino de Dios. Y ya ninguno osaba preguntarle.
\footnote{\textbf{12:34} Hech 26,27-29}

\hypertarget{la-contrapregunta-de-jesuxfas-sobre-el-mesuxedas-como-hijo-de-david}{%
\subsection{La contrapregunta de Jesús sobre el Mesías como hijo de
David}\label{la-contrapregunta-de-jesuxfas-sobre-el-mesuxedas-como-hijo-de-david}}

\bibverse{35} Y respondiendo Jesús decía, enseñando en el templo: ¿Cómo
dicen los escribas que el Cristo es hijo de David? \footnote{\textbf{12:35}
  Is 11,1; Rom 1,3} \bibverse{36} Porque el mismo David dijo por el
Espíritu Santo: Dijo el Señor á mi Señor: Siéntate á mi diestra, hasta
que ponga tus enemigos por estrado de tus pies. \footnote{\textbf{12:36}
  2Sam 23,2}

\bibverse{37} Luego llamándole el mismo David Señor, ¿de dónde, pues, es
su hijo? Y los que eran del común del pueblo le oían de buena gana.

\hypertarget{la-advertencia-de-jesuxfas-sobre-la-ambiciuxf3n-y-la-codicia-de-los-escribas}{%
\subsection{La advertencia de Jesús sobre la ambición y la codicia de
los
escribas}\label{la-advertencia-de-jesuxfas-sobre-la-ambiciuxf3n-y-la-codicia-de-los-escribas}}

\bibverse{38} Y les decía en su doctrina: Guardaos de los escribas, que
quieren andar con ropas largas, y aman las salutaciones en las plazas,
\bibverse{39} Y las primeras sillas en las sinagogas, y los primeros
asientos en las cenas; \bibverse{40} Que devoran las casas de las
viudas, y por pretexto hacen largas oraciones. Estos recibirán mayor
juicio. \footnote{\textbf{12:40} Sant 1,27}

\hypertarget{jesuxfas-alaba-las-dos-blancas-de-la-viuda-pobre}{%
\subsection{Jesús alaba las dos blancas de la viuda
pobre}\label{jesuxfas-alaba-las-dos-blancas-de-la-viuda-pobre}}

\bibverse{41} Y estando sentado Jesús delante del arca de la ofrenda,
miraba cómo el pueblo echaba dinero en el arca: y muchos ricos echaban
mucho. \footnote{\textbf{12:41} 2Re 12,10}

\bibverse{42} Y como vino una viuda pobre, echó dos blancas, que son un
maravedí. \bibverse{43} Entonces llamando á sus discípulos, les dice: De
cierto os digo que esta viuda pobre echó más que todos los que han
echado en el arca: \bibverse{44} Porque todos han echado de lo que les
sobra; mas ésta, de su pobreza echó todo lo que tenía, todo su alimento.

\hypertarget{los-primeros-signos-del-fin-de-los-tiempos}{%
\subsection{Los primeros signos del fin de los
tiempos}\label{los-primeros-signos-del-fin-de-los-tiempos}}

\hypertarget{section-12}{%
\section{13}\label{section-12}}

\bibverse{1} Y saliendo del templo, le dice uno de sus discípulos:
Maestro, mira qué piedras, y qué edificios.

\bibverse{2} Y Jesús respondiendo, le dijo: ¿Ves estos grandes
edificios? no quedará piedra sobre piedra que no sea derribada.

\bibverse{3} Y sentándose en el monte de las Olivas delante del templo,
le preguntaron aparte Pedro y Jacobo y Juan y Andrés: \footnote{\textbf{13:3}
  Mat 17,1} \bibverse{4} Dinos, ¿cuándo serán estas cosas? ¿y qué señal
habrá cuando todas estas cosas han de cumplirse?

\hypertarget{los-primeros-signos-del-fin-de-los-tiempos-1}{%
\subsection{Los primeros signos del fin de los
tiempos}\label{los-primeros-signos-del-fin-de-los-tiempos-1}}

\bibverse{5} Y Jesús respondiéndoles, comenzó á decir: Mirad, que nadie
os engañe; \bibverse{6} Porque vendrán muchos en mi nombre, diciendo: Yo
soy el Cristo; y engañarán á muchos.

\bibverse{7} Mas cuando oyereis de guerras y de rumores de guerras no os
turbéis, porque conviene hacerse así; mas aun no será el fin.
\bibverse{8} Porque se levantará nación contra nación, y reino contra
reino; y habrá terremotos en muchos lugares, y habrá hambres y
alborotos; principios de dolores serán estos.

\hypertarget{la-persecuciuxf3n-de-los-discuxedpulos}{%
\subsection{La persecución de los
discípulos}\label{la-persecuciuxf3n-de-los-discuxedpulos}}

\bibverse{9} Mas vosotros mirad por vosotros: porque os entregarán en
los concilios, y en sinagogas seréis azotados: y delante de presidentes
y de reyes seréis llamados por causa de mí, en testimonio á ellos.
\bibverse{10} Y á todas las gentes conviene que el evangelio sea
predicado antes. \footnote{\textbf{13:10} Mar 16,15} \bibverse{11} Y
cuando os trajeren para entregaros, no premeditéis qué habéis de decir,
ni lo penséis: mas lo que os fuere dado en aquella hora, eso hablad;
porque no sois vosotros los que habláis, sino el Espíritu Santo.

\bibverse{12} Y entregará á la muerte el hermano al hermano, y el padre
al hijo: y se levantarán los hijos contra los padres, y los matarán.
\bibverse{13} Y seréis aborrecidos de todos por mi nombre: mas el que
perseverare hasta el fin, éste será salvo.

\hypertarget{el-cluxedmax-de-la-tribulaciuxf3n-en-judea}{%
\subsection{El clímax de la tribulación en
Judea}\label{el-cluxedmax-de-la-tribulaciuxf3n-en-judea}}

\bibverse{14} Empero cuando viereis la abominación de asolamiento, que
fué dicha por el profeta Daniel, que estará donde no debe (el que lee,
entienda), entonces los que estén en Judea huyan á los montes;
\footnote{\textbf{13:14} Dan 9,27; Dan 11,31} \bibverse{15} Y el que
esté sobre el terrado, no descienda á la casa, ni entre para tomar algo
de su casa; \bibverse{16} Y el que estuviere en el campo, no vuelva
atrás á tomar su capa. \bibverse{17} Mas ¡ay de las preñadas, y de las
que criaren en aquellos días! \bibverse{18} Orad pues, que no acontezca
vuestra huída en invierno. \bibverse{19} Porque aquellos días serán de
aflicción, cual nunca fué desde el principio de la creación que crió
Dios, hasta este tiempo, ni será. \bibverse{20} Y si el Señor no hubiese
abreviado aquellos días, ninguna carne se salvaría; mas por causa de los
escogidos que él escogió, abrevió aquellos días.

\hypertarget{profecuxeda-sobre-los-falsos-profetas}{%
\subsection{Profecía sobre los falsos
profetas}\label{profecuxeda-sobre-los-falsos-profetas}}

\bibverse{21} Y entonces si alguno os dijere: He aquí, aquí está el
Cristo; ó, He aquí, allí está, no le creáis. \bibverse{22} Porque se
levantarán falsos Cristos y falsos profetas, y darán señales y
prodigios, para engañar, si se pudiese hacer, aun á los escogidos.
\bibverse{23} Mas vosotros mirad; os lo he dicho antes todo.

\hypertarget{los-uxfaltimos-augurios-y-la-apariciuxf3n-del-hijo-del-hombre-en-el-uxfaltimo-duxeda}{%
\subsection{Los últimos augurios y la aparición del Hijo del Hombre en
el último
día}\label{los-uxfaltimos-augurios-y-la-apariciuxf3n-del-hijo-del-hombre-en-el-uxfaltimo-duxeda}}

\bibverse{24} Empero en aquellos días, después de aquella aflicción, el
sol se obscurecerá, y la luna no dará su resplandor; \bibverse{25} Y las
estrellas caerán del cielo, y las virtudes que están en los cielos serán
conmovidas; \footnote{\textbf{13:25} Heb 12,26} \bibverse{26} Y entonces
verán al Hijo del hombre, que vendrá en las nubes con mucha potestad y
gloria. \bibverse{27} Y entonces enviará sus ángeles, y juntará sus
escogidos de los cuatro vientos, desde el cabo de la tierra hasta el
cabo del cielo.

\bibverse{28} De la higuera aprended la semejanza: Cuando su rama ya se
enternece, y brota hojas, conocéis que el verano está cerca:
\bibverse{29} Así también vosotros, cuando viereis hacerse estas cosas,
conoced que está cerca, á las puertas. \bibverse{30} De cierto os digo
que no pasará esta generación, que todas estas cosas no sean hechas.
\bibverse{31} El cielo y la tierra pasarán, mas mis palabras no pasarán.

\bibverse{32} Empero de aquel día y de la hora, nadie sabe; ni aun los
ángeles que están en el cielo, ni el Hijo, sino el Padre.

\hypertarget{exhortaciuxf3n-final-a-los-discuxedpulos-a-estar-alerta}{%
\subsection{Exhortación final a los discípulos a estar
alerta}\label{exhortaciuxf3n-final-a-los-discuxedpulos-a-estar-alerta}}

\bibverse{33} Mirad, velad y orad: porque no sabéis cuándo será el
tiempo. \footnote{\textbf{13:33} Luc 12,35-40}

\bibverse{34} Como el hombre que partiéndose lejos, dejó su casa, y dió
facultad á sus siervos, y á cada uno su obra, y al portero mandó que
velase: \bibverse{35} Velad pues, porque no sabéis cuándo el señor de la
casa vendrá; si á la tarde, ó á la media noche, ó al canto del gallo, ó
á la mañana; \bibverse{36} Porque cuando viniere de repente, no os halle
durmiendo. \bibverse{37} Y las cosas que á vosotros digo, á todos las
digo: Velad.

\hypertarget{intento-de-asesinato-por-parte-de-los-luxedderes-del-pueblo}{%
\subsection{Intento de asesinato por parte de los líderes del
pueblo}\label{intento-de-asesinato-por-parte-de-los-luxedderes-del-pueblo}}

\hypertarget{section-13}{%
\section{14}\label{section-13}}

\bibverse{1} Y dos días después era la Pascua y los días de los panes
sin levadura: y procuraban los príncipes de los sacerdotes y los
escribas cómo le prenderían por engaño, y le matarían. \bibverse{2} Y
decían: No en el día de la fiesta, porque no se haga alboroto del
pueblo.

\hypertarget{unciuxf3n-de-jesuxfas-en-betania}{%
\subsection{Unción de Jesús en
Betania}\label{unciuxf3n-de-jesuxfas-en-betania}}

\bibverse{3} Y estando él en Bethania en casa de Simón el leproso, y
sentado á la mesa, vino una mujer teniendo un alabastro de ungüento de
nardo espique de mucho precio; y quebrando el alabastro, derramóselo
sobre su cabeza. \footnote{\textbf{14:3} Juan 12,1-8} \bibverse{4} Y
hubo algunos que se enojaron dentro de sí, y dijeron: ¿Para qué se ha
hecho este desperdicio de ungüento? \bibverse{5} Porque podía esto ser
vendido por más de trescientos denarios, y darse á los pobres. Y
murmuraban contra ella.

\bibverse{6} Mas Jesús dijo: Dejadla; ¿por qué la fatigáis? buena obra
me ha hecho; \bibverse{7} Que siempre tendréis los pobres con vosotros,
y cuando quisiereis les podréis hacer bien; mas á mí no siempre me
tendréis. \bibverse{8} Esta ha hecho lo que podía; porque se ha
anticipado á ungir mi cuerpo para la sepultura. \bibverse{9} De cierto
os digo que donde quiera que fuere predicado este evangelio en todo el
mundo, también esto que ha hecho ésta, será dicho para memoria de ella.

\hypertarget{traiciuxf3n-de-judas}{%
\subsection{Traición de Judas}\label{traiciuxf3n-de-judas}}

\bibverse{10} Entonces Judas Iscariote, uno de los doce, vino á los
príncipes de los sacerdotes, para entregársele. \bibverse{11} Y ellos
oyéndolo se holgaron, y prometieron que le darían dineros. Y buscaba
oportunidad cómo le entregaría.

\hypertarget{preparaciuxf3n-de-la-comida-pascual}{%
\subsection{Preparación de la comida
pascual}\label{preparaciuxf3n-de-la-comida-pascual}}

\bibverse{12} Y el primer día de los panes sin levadura, cuando
sacrificaban la pascua, sus discípulos le dicen: ¿Dónde quieres que
vayamos á disponer para que comas la pascua?

\bibverse{13} Y envía dos de sus discípulos, y les dice: Id á la ciudad,
y os encontrará un hombre que lleva un cántaro de agua; seguidle;
\bibverse{14} Y donde entrare, decid al señor de la casa: El Maestro
dice: ¿Dónde está el aposento donde he de comer la pascua con mis
discípulos? \footnote{\textbf{14:14} Mar 11,3} \bibverse{15} Y él os
mostrará un gran cenáculo ya preparado: aderezad para nosotros allí.

\bibverse{16} Y fueron sus discípulos, y vinieron á la ciudad, y
hallaron como les había dicho; y aderezaron la pascua.

\hypertarget{la-uxfaltima-cena-de-jesuxfas-en-el-cuxedrculo-de-los-discuxedpulos-anuncio-de-la-traiciuxf3n-de-judas-instituciuxf3n-de-la-santa-comuniuxf3n}{%
\subsection{La última cena de Jesús en el círculo de los discípulos;
Anuncio de la traición de Judas; Institución de la santa
comunión}\label{la-uxfaltima-cena-de-jesuxfas-en-el-cuxedrculo-de-los-discuxedpulos-anuncio-de-la-traiciuxf3n-de-judas-instituciuxf3n-de-la-santa-comuniuxf3n}}

\bibverse{17} Y llegada la tarde, fué con los doce. \bibverse{18} Y como
se sentaron á la mesa y comiesen, dice Jesús: De cierto os digo que uno
de vosotros, que come conmigo, me ha de entregar.

\bibverse{19} Entonces ellos comenzaron á entristecerse, y á decirle
cada uno por sí: ¿Seré yo? Y el otro: ¿Seré yo?

\bibverse{20} Y él respondiendo les dijo: Es uno de los doce que moja
conmigo en el plato. \bibverse{21} A la verdad el Hijo del hombre va,
como está de él escrito; mas ¡ay de aquel hombre por quien el Hijo del
hombre es entregado! bueno le fuera á aquel hombre si nunca hubiera
nacido.

\bibverse{22} Y estando ellos comiendo, tomó Jesús pan, y bendiciendo,
partió y les dió, y dijo: Tomad, esto es mi cuerpo. \footnote{\textbf{14:22}
  1Cor 11,23-25}

\bibverse{23} Y tomando el vaso, habiendo hecho gracias, les dió: y
bebieron de él todos. \bibverse{24} Y les dice: Esto es mi sangre del
nuevo pacto, que por muchos es derramada. \bibverse{25} De cierto os
digo que no beberé más del fruto de la vid, hasta aquel día cuando lo
beberé nuevo en el reino de Dios.

\hypertarget{camina-a-getsemanuxed}{%
\subsection{Camina a Getsemaní}\label{camina-a-getsemanuxed}}

\bibverse{26} Y como hubieron cantado el himno, se salieron al monte de
las Olivas. \footnote{\textbf{14:26} Sal 113,1-118}

\bibverse{27} Jesús entonces les dice: Todos seréis escandalizados en mí
esta noche; porque escrito está: Heriré al pastor, y serán derramadas
las ovejas. \footnote{\textbf{14:27} Juan 16,32} \bibverse{28} Mas
después que haya resucitado, iré delante de vosotros á Galilea.
\footnote{\textbf{14:28} Mar 16,7}

\bibverse{29} Entonces Pedro le dijo: Aunque todos sean escandalizados,
mas no yo.

\bibverse{30} Y le dice Jesús: De cierto te digo que tú, hoy, en esta
noche, antes que el gallo haya cantado dos veces, me negarás tres veces.

\bibverse{31} Mas él con mayor porfía decía: Si me fuere menester morir
contigo, no te negaré. También todos decían lo mismo.

\hypertarget{el-conflicto-y-la-oraciuxf3n-de-jesuxfas-en-getsemanuxed-debilidad-de-los-discuxedpulos}{%
\subsection{El conflicto y la oración de Jesús en Getsemaní; Debilidad
de los
discípulos}\label{el-conflicto-y-la-oraciuxf3n-de-jesuxfas-en-getsemanuxed-debilidad-de-los-discuxedpulos}}

\bibverse{32} Y vienen al lugar que se llama Gethsemaní, y dice á sus
discípulos: Sentaos aquí, entre tanto que yo oro. \bibverse{33} Y toma
consigo á Pedro y á Jacobo y á Juan, y comenzó á atemorizarse, y á
angustiarse. \footnote{\textbf{14:33} Mat 17,1} \bibverse{34} Y les
dice: Está muy triste mi alma, hasta la muerte: esperad aquí y velad.
\footnote{\textbf{14:34} Juan 12,27}

\bibverse{35} Y yéndose un poco adelante, se postró en tierra, y oró que
si fuese posible, pasase de él aquella hora. \bibverse{36} Y decía:
Abba, Padre, todas las cosas son á ti posibles; traspasa de mí este
vaso; empero no lo que yo quiero, sino lo que tú. \footnote{\textbf{14:36}
  Mar 10,38}

\bibverse{37} Y vino y los halló durmiendo; y dice á Pedro: ¿Simón,
duermes? ¿No has podido velar una hora? \bibverse{38} Velad y orad, para
que no entréis en tentación: el espíritu á la verdad es presto, mas la
carne enferma.

\bibverse{39} Y volviéndose á ir, oró, y dijo las mismas palabras.
\bibverse{40} Y vuelto, los halló otra vez durmiendo, porque los ojos de
ellos estaban cargados; y no sabían qué responderle. \bibverse{41} Y
vino la tercera vez, y les dice: Dormid ya y descansad: basta, la hora
es venida; he aquí, el Hijo del hombre es entregado en manos de los
pecadores. \bibverse{42} Levantaos, vamos: he aquí, el que me entrega
está cerca.

\hypertarget{encarcelamiento-de-jesuxfas-escape-de-los-discuxedpulos}{%
\subsection{Encarcelamiento de Jesús; Escape de los
discípulos}\label{encarcelamiento-de-jesuxfas-escape-de-los-discuxedpulos}}

\bibverse{43} Y luego, aun hablando él, vino Judas, que era uno de los
doce, y con él una compañía con espadas y palos, de parte de los
príncipes de los sacerdotes, y de los escribas y de los ancianos.
\bibverse{44} Y el que le entregaba les había dado señal común,
diciendo: Al que yo besare, aquél es: prendedle, y llevadle con
seguridad. \bibverse{45} Y como vino, se acercó luego á él, y le dice:
Maestro, Maestro. Y le besó. \bibverse{46} Entonces ellos echaron en él
sus manos, y le prendieron. \bibverse{47} Y uno de los que estaban allí,
sacando la espada, hirió al siervo del sumo sacerdote, y le cortó la
oreja.

\bibverse{48} Y respondiendo Jesús, les dijo: ¿Como á ladrón habéis
salido con espadas y con palos á tomarme? \bibverse{49} Cada día estaba
con vosotros enseñando en el templo, y no me tomasteis; pero es así,
para que se cumplan las Escrituras.

\bibverse{50} Entonces dejándole todos sus discípulos, huyeron.
\bibverse{51} Empero un mancebillo le seguía cubierto de una sábana
sobre el cuerpo desnudo; y los mancebos le prendieron: \bibverse{52} Mas
él, dejando la sábana, se huyó de ellos desnudo.

\hypertarget{el-interrogatorio-la-confesiuxf3n-y-la-condena-de-jesuxfas-ante-el-sumo-sacerdote-y-el-concilio}{%
\subsection{El interrogatorio, la confesión y la condena de Jesús ante
el sumo sacerdote y el
concilio}\label{el-interrogatorio-la-confesiuxf3n-y-la-condena-de-jesuxfas-ante-el-sumo-sacerdote-y-el-concilio}}

\bibverse{53} Y trajeron á Jesús al sumo sacerdote; y se juntaron á él
todos los príncipes de los sacerdotes y los ancianos y los escribas.

\bibverse{54} Empero Pedro le siguió de lejos hasta dentro del patio del
sumo sacerdote; y estaba sentado con los servidores, y calentándose al
fuego. \bibverse{55} Y los príncipes de los sacerdotes y todo el
concilio buscaban testimonio contra Jesús, para entregarle á la muerte;
mas no lo hallaban. \bibverse{56} Porque muchos decían falso testimonio
contra él; mas sus testimonios no concertaban. \bibverse{57} Entonces
levantándose unos, dieron falso testimonio contra él, diciendo:
\bibverse{58} Nosotros le hemos oído decir: Yo derribaré este templo que
es hecho de mano, y en tres días edificaré otro echo sin mano.
\bibverse{59} Mas ni aun así se concertaba el testimonio de ellos.

\bibverse{60} Entonces el sumo sacerdote, levantándose en medio,
preguntó á Jesús, diciendo: ¿No respondes algo? ¿Qué atestiguan éstos
contra ti? \bibverse{61} Mas él callaba, y nada respondía. El sumo
sacerdote le volvió á preguntar, y le dice: ¿Eres tú el Cristo, el Hijo
del Bendito? \footnote{\textbf{14:61} Mar 15,5; Is 53,7}

\bibverse{62} Y Jesús le dijo: Yo soy; y veréis al Hijo del hombre
sentado á la diestra de la potencia de Dios, y viniendo en las nubes del
cielo. \footnote{\textbf{14:62} Dan 7,13-14}

\bibverse{63} Entonces el sumo sacerdote, rasgando sus vestidos, dijo:
¿Qué más tenemos necesidad de testigos? \bibverse{64} Oído habéis la
blasfemia: ¿qué os parece? Y ellos todos le condenaron ser culpado de
muerte. \footnote{\textbf{14:64} Juan 19,7} \bibverse{65} Y algunos
comenzaron á escupir en él, y cubrir su rostro, y á darle bofetadas, y
decirle: Profetiza. Y los servidores le herían de bofetadas.

\hypertarget{negaciuxf3n-y-arrepentimiento-de-pedro}{%
\subsection{Negación y arrepentimiento de
Pedro}\label{negaciuxf3n-y-arrepentimiento-de-pedro}}

\bibverse{66} Y estando Pedro abajo en el atrio, vino una de las criadas
del sumo sacerdote; \bibverse{67} Y como vió á Pedro que se calentaba,
mirándole, dice: Y tú con Jesús el Nazareno estabas.

\bibverse{68} Mas él negó, diciendo: No conozco, ni sé lo que dices. Y
se salió fuera á la entrada; y cantó el gallo.

\bibverse{69} Y la criada viéndole otra vez, comenzó á decir á los que
estaban allí: Este es de ellos. \bibverse{70} Mas él negó otra vez. Y
poco después, los que estaban allí dijeron otra vez á Pedro:
Verdaderamente tú eres de ellos; porque eres Galileo, y tu habla es
semejante. \bibverse{71} Y él comenzó á maldecir y á jurar: No conozco á
este hombre de quien habláis.

\bibverse{72} Y el gallo cantó la segunda vez: y Pedro se acordó de las
palabras que Jesús le había dicho: Antes que el gallo cante dos veces,
me negarás tres veces. Y pensando en esto, lloraba.

\hypertarget{el-interrogatorio-de-jesuxfas-ante-el-gobernador-romano-poncio-pilato-su-condenaciuxf3n-y-flagelaciuxf3n}{%
\subsection{El interrogatorio de Jesús ante el gobernador romano Poncio
Pilato; su condenación y
flagelación}\label{el-interrogatorio-de-jesuxfas-ante-el-gobernador-romano-poncio-pilato-su-condenaciuxf3n-y-flagelaciuxf3n}}

\hypertarget{section-14}{%
\section{15}\label{section-14}}

\bibverse{1} Y luego por la mañana, habiendo tenido consejo los
príncipes de los sacerdotes con los ancianos, y con los escribas, y con
todo el concilio, llevaron á Jesús atado, y le entregaron á Pilato.
\bibverse{2} Y Pilato le preguntó: ¿Eres tú el Rey de los Judíos? Y
respondiendo él, le dijo: Tú lo dices.

\bibverse{3} Y los príncipes de los sacerdotes le acusaban mucho.
\bibverse{4} Y le preguntó otra vez Pilato, diciendo: ¿No respondes
algo? Mira de cuántas cosas te acusan.

\bibverse{5} Mas Jesús ni aun con eso respondió; de modo que Pilato se
maravillaba.

\bibverse{6} Empero en el día de la fiesta les soltaba un preso,
cualquiera que pidiesen. \bibverse{7} Y había uno, que se llamaba
Barrabás, preso con sus compañeros de motín que habían hecho muerte en
una revuelta. \bibverse{8} Y viniendo la multitud, comenzó á pedir
hiciese como siempre les había hecho. \bibverse{9} Y Pilato les
respondió, diciendo: ¿Queréis que os suelte al Rey de los Judíos?
\bibverse{10} Porque conocía que por envidia le habían entregado los
príncipes de los sacerdotes. \footnote{\textbf{15:10} Juan 11,48}
\bibverse{11} Mas los príncipes de los sacerdotes incitaron á la
multitud, que les soltase antes á Barrabás. \bibverse{12} Y respondiendo
Pilato, les dice otra vez: ¿Qué pues queréis que haga del que llamáis
Rey de los Judíos?

\bibverse{13} Y ellos volvieron á dar voces: Crucifícale.

\bibverse{14} Mas Pilato les decía: ¿Pues qué mal ha hecho? Y ellos
daban más voces: Crucifícale.

\bibverse{15} Y Pilato, queriendo satisfacer al pueblo, les soltó á
Barrabás, y entregó á Jesús, después de azotarle, para que fuese
crucificado.

\hypertarget{la-burla-y-el-maltrato-de-jesuxfas-por-parte-de-los-soldados-romanos}{%
\subsection{La burla y el maltrato de Jesús por parte de los soldados
romanos}\label{la-burla-y-el-maltrato-de-jesuxfas-por-parte-de-los-soldados-romanos}}

\bibverse{16} Entonces los soldados le llevaron dentro de la sala, es á
saber, al Pretorio; y convocan toda la cohorte. \bibverse{17} Y le
visten de púrpura; y poniéndole una corona tejida de espinas,
\bibverse{18} Comenzaron luego á saludarle: ¡Salve, Rey de los Judíos!
\bibverse{19} Y le herían en la cabeza con una caña, y escupían en él, y
le adoraban hincadas las rodillas.

\hypertarget{el-curso-de-la-muerte-de-jesuxfas-despuuxe9s-del-guxf3lgota-su-crucifixiuxf3n-y-su-muerte}{%
\subsection{El curso de la muerte de Jesús después del Gólgota, su
crucifixión y su
muerte}\label{el-curso-de-la-muerte-de-jesuxfas-despuuxe9s-del-guxf3lgota-su-crucifixiuxf3n-y-su-muerte}}

\bibverse{20} Y cuando le hubieron escarnecido, le desnudaron la
púrpura, y le vistieron sus propios vestidos, y le sacaron para
crucificarle.

\bibverse{21} Y cargaron á uno que pasaba, Simón Cireneo, padre de
Alejandro y de Rufo, que venía del campo, para que llevase su cruz.
\bibverse{22} Y le llevan al lugar de Gólgotha, que declarado quiere
decir: Lugar de la Calavera. \bibverse{23} Y le dieron á beber vino
mezclado con mirra; mas él no lo tomó. \footnote{\textbf{15:23} Sal
  69,22}

\bibverse{24} Y cuando le hubieron crucificado, repartieron sus
vestidos, echando suertes sobre ellos, qué llevaría cada uno.
\footnote{\textbf{15:24} Sal 22,19} \bibverse{25} Y era la hora de las
tres cuando le crucificaron. \bibverse{26} Y el título escrito de su
causa era: EL REY DE LOS JUDIOS. \bibverse{27} Y crucificaron con él dos
ladrones, uno á su derecha, y el otro á su izquierda. \bibverse{28} Y se
cumplió la Escritura, que dice: Y con los inicuos fué contado.

\bibverse{29} Y los que pasaban le denostaban, meneando sus cabezas, y
diciendo: ¡Ah! tú que derribas el templo de Dios, y en tres días lo
edificas, \footnote{\textbf{15:29} Mar 14,58} \bibverse{30} Sálvate á ti
mismo, y desciende de la cruz.

\bibverse{31} Y de esta manera también los príncipes de los sacerdotes
escarneciendo, decían unos á otros, con los escribas: A otros salvó, á
sí mismo no se puede salvar. \bibverse{32} El Cristo, Rey de Israel,
descienda ahora de la cruz, para que veamos y creamos. También los que
estaban crucificados con él le denostaban.

\hypertarget{la-muerte-de-jesuxfas-el-signo-milagroso-de-su-muerte}{%
\subsection{La muerte de Jesús; el signo milagroso de su
muerte}\label{la-muerte-de-jesuxfas-el-signo-milagroso-de-su-muerte}}

\bibverse{33} Y cuando vino la hora de sexta, fueron hechas tinieblas
sobre toda la tierra hasta la hora de nona. \bibverse{34} Y á la hora de
nona, exclamó Jesús á gran voz, diciendo: Eloi, Eloi, ¿lama sabachthani?
que declarado, quiere decir: Dios mío, Dios mío, ¿por qué me has
desamparado? \footnote{\textbf{15:34} Sal 22,2}

\bibverse{35} Y oyéndole unos de los que estaban allí, decían: He aquí,
llama á Elías.

\bibverse{36} Y corrió uno, y empapando una esponja en vinagre, y
poniéndola en una caña, le dió á beber, diciendo: Dejad, veamos si
vendrá Elías á quitarle.

\bibverse{37} Mas Jesús, dando una grande voz, espiró. \bibverse{38}
Entonces el velo del templo se rasgó en dos, de alto á bajo.
\bibverse{39} Y el centurión que estaba delante de él, viendo que había
espirado así clamando, dijo: Verdaderamente este hombre era el Hijo de
Dios.

\bibverse{40} Y también estaban algunas mujeres mirando de lejos; entre
las cuales estaba María Magdalena, y María la madre de Jacobo el menor y
de José, y Salomé; \bibverse{41} Las cuales, estando aún él en Galilea,
le habían seguido, y le servían; y otras muchas que juntamente con él
habían subido á Jerusalem.

\hypertarget{entierro-de-jesuxfas}{%
\subsection{Entierro de Jesús}\label{entierro-de-jesuxfas}}

\bibverse{42} Y cuando fué la tarde, porque era la preparación, es
decir, la víspera del sábado, \bibverse{43} José de Arimatea, senador
noble, que también esperaba el reino de Dios, vino, y osadamente entró á
Pilato, y pidió el cuerpo de Jesús. \bibverse{44} Y Pilato se maravilló
que ya fuese muerto; y haciendo venir al centurión, preguntóle si era ya
muerto. \bibverse{45} Y enterado del centurión, dió el cuerpo á José:
\bibverse{46} El cual compró una sábana, y quitándole, le envolvió en la
sábana: y le puso en un sepulcro que estaba cavado en una peña; y
revolvió una piedra á la puerta del sepulcro. \bibverse{47} Y María
Magdalena, y María madre de José, miraban donde era puesto.

\hypertarget{descubrimiento-de-la-tumba-vacuxeda-en-la-mauxf1ana-de-pascua-la-revelaciuxf3n-del-uxe1ngel-a-las-mujeres}{%
\subsection{Descubrimiento de la tumba vacía en la mañana de Pascua; la
revelación del ángel a las
mujeres}\label{descubrimiento-de-la-tumba-vacuxeda-en-la-mauxf1ana-de-pascua-la-revelaciuxf3n-del-uxe1ngel-a-las-mujeres}}

\hypertarget{section-15}{%
\section{16}\label{section-15}}

\bibverse{1} Y como pasó el sábado, María Magdalena, y María madre de
Jacobo, y Salomé, compraron drogas aromáticas, para venir á ungirle.
\bibverse{2} Y muy de mañana, el primer día de la semana, vienen al
sepulcro, ya salido el sol. \bibverse{3} Y decían entre sí: ¿Quién nos
revolverá la piedra de la puerta del sepulcro? \bibverse{4} Y como
miraron, ven la piedra revuelta; que era muy grande.

\bibverse{5} Y entradas en el sepulcro, vieron un mancebo sentado al
lado derecho, cubierto de una larga ropa blanca; y se espantaron.
\bibverse{6} Mas él les dice: No os asustéis: buscáis á Jesús Nazareno,
el que fué crucificado; resucitado há, no está aquí; he aquí el lugar en
donde le pusieron. \bibverse{7} Mas id, decid á sus discípulos y á
Pedro, que él va antes que vosotros á Galilea: allí le veréis, como os
dijo. \footnote{\textbf{16:7} Mar 14,28}

\bibverse{8} Y ellas se fueron huyendo del sepulcro; porque las había
tomado temblor y espanto; ni decían nada á nadie, porque tenían miedo.

\hypertarget{jesuxfas-se-aparece-a-maruxeda-magdalena-y-a-los-dos-discuxedpulos-de-emauxfas}{%
\subsection{Jesús se aparece a María Magdalena y a los dos discípulos de
Emaús}\label{jesuxfas-se-aparece-a-maruxeda-magdalena-y-a-los-dos-discuxedpulos-de-emauxfas}}

\bibverse{9} Mas como Jesús resucitó por la mañana, el primer día de la
semana, apareció primeramente á María Magdalena, de la cual había echado
siete demonios. \bibverse{10} Yendo ella, lo hizo saber á los que habían
estado con él, que estaban tristes y llorando. \bibverse{11} Y ellos
como oyeron que vivía, y que había sido visto de ella, no lo creyeron.

\bibverse{12} Mas después apareció en otra forma á dos de ellos que iban
caminando, yendo al campo. \footnote{\textbf{16:12} Luc 24,13-35}
\bibverse{13} Y ellos fueron, y lo hicieron saber á los otros; y ni aun
á ellos creyeron.

\hypertarget{la-apariciuxf3n-de-jesuxfas-a-los-once-apuxf3stoles-y-su-mandato-misional}{%
\subsection{La aparición de Jesús a los once apóstoles y su mandato
misional}\label{la-apariciuxf3n-de-jesuxfas-a-los-once-apuxf3stoles-y-su-mandato-misional}}

\bibverse{14} Finalmente se apareció á los once mismos, estando sentados
á la mesa, y censuróles su incredulidad y dureza de corazón, que no
hubiesen creído á los que le habían visto resucitado. \bibverse{15} Y
les dijo: Id por todo el mundo; predicad el evangelio á toda criatura.
\footnote{\textbf{16:15} Mar 13,10; Mat 28,18-20} \bibverse{16} El que
creyere y fuere bautizado, será salvo; mas el que no creyere, será
condenado. \footnote{\textbf{16:16} Hech 2,38; Hech 16,31; Hech 16,33}
\bibverse{17} Y estas señales seguirán á los que creyeren: En mi nombre
echarán fuera demonios; hablarán nuevas lenguas; \footnote{\textbf{16:17}
  Hech 16,18; Hech 10,46; Hech 19,6} \bibverse{18} Quitarán serpientes,
y si bebieren cosa mortífera, no les dañará; sobre los enfermos pondrán
sus manos, y sanarán. \footnote{\textbf{16:18} Luc 10,19; Hech 28,3-6;
  Sant 5,14; Sant 1,5-15}

\hypertarget{ascensiuxf3n-de-jesuxfas}{%
\subsection{Ascensión de Jesús}\label{ascensiuxf3n-de-jesuxfas}}

\bibverse{19} Y el Señor, después que les habló, fué recibido arriba en
el cielo, y sentóse á la diestra de Dios. \^{}\^{} \bibverse{20} Y
ellos, saliendo, predicaron en todas partes, obrando con ellos el Señor,
y confirmando la palabra con las señales que se seguían. Amén.
