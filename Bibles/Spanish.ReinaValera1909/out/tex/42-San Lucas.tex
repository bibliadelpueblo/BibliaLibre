\hypertarget{introducciuxf3n}{%
\subsection{Introducción}\label{introducciuxf3n}}

\hypertarget{section}{%
\section{1}\label{section}}

\bibverse{1} Habiendo muchos tentado á poner en orden la historia de las
cosas que entre nosotros han sido ciertísimas, \bibverse{2} Como nos lo
enseñaron los que desde el principio lo vieron por sus ojos, y fueron
ministros de la palabra; \footnote{\textbf{1:2} 1Jn 1,1-4} \bibverse{3}
Me ha parecido también á mí, después de haber entendido todas las cosas
desde el principio con diligencia, escribírtelas por orden, oh muy buen
Teófilo, \footnote{\textbf{1:3} Hech 1,1; Col 4,14} \bibverse{4} Para
que conozcas la verdad de las cosas en las cuales has sido enseñado.

\hypertarget{anuncio-del-nacimiento-de-juan-el-bautista}{%
\subsection{Anuncio del nacimiento de Juan el
Bautista}\label{anuncio-del-nacimiento-de-juan-el-bautista}}

\bibverse{5} HUBO en los días de Herodes, rey de Judea, un sacerdote
llamado Zacarías, de la suerte de Abías; y su mujer, de las hijas de
Aarón, llamada Elisabet. \footnote{\textbf{1:5} 1Cró 24,10; 1Cró 24,19}
\bibverse{6} Y eran ambos justos delante de Dios, andando sin reprensión
en todos los mandamientos y estatutos del Señor. \bibverse{7} Y no
tenían hijo, porque Elisabet era estéril, y ambos eran avanzados en
días.

\bibverse{8} Y aconteció que ejerciendo Zacarías el sacerdocio delante
de Dios por el orden de su vez, \bibverse{9} Conforme á la costumbre del
sacerdocio, salió en suerte á poner el incienso, entrando en el templo
del Señor. \bibverse{10} Y toda la multitud del pueblo estaba fuera
orando á la hora del incienso.

\bibverse{11} Y se le apareció el ángel del Señor puesto en pie á la
derecha del altar del incienso. \bibverse{12} Y se turbó Zacarías
viéndole, y cayó temor sobre él. \bibverse{13} Mas el ángel le dijo:
Zacarías, no temas; porque tu oración ha sido oída, y tu mujer Elisabet
te parirá un hijo, y llamarás su nombre Juan. \bibverse{14} Y tendrás
gozo y alegría, y muchos se gozarán de su nacimiento. \bibverse{15}
Porque será grande delante de Dios, y no beberá vino ni sidra; y será
lleno del Espíritu Santo, aun desde el seno de su madre. \footnote{\textbf{1:15}
  Jue 13,4-5} \bibverse{16} Y á muchos de los hijos de Israel convertirá
al Señor Dios de ellos. \bibverse{17} Porque él irá delante de él con el
espíritu y virtud de Elías, para convertir los corazones de los padres á
los hijos, y los rebeldes á la prudencia de los justos, para aparejar al
Señor un pueblo apercibido.

\bibverse{18} Y dijo Zacarías al ángel: ¿En qué conoceré esto? porque yo
soy viejo, y mi mujer avanzada en días. \footnote{\textbf{1:18} Gén
  18,11}

\bibverse{19} Y respondiendo el ángel le dijo: Yo soy Gabriel, que estoy
delante de Dios; y soy enviado á hablarte, y á darte estas buenas
nuevas. \footnote{\textbf{1:19} Dan 8,16} \bibverse{20} Y he aquí
estarás mudo y no podrás hablar, hasta el día que esto sea hecho, por
cuanto no creíste á mis palabras, las cuales se cumplirán á su tiempo.

\bibverse{21} Y el pueblo estaba esperando á Zacarías, y se maravillaban
de que él se detuviese en el templo. \bibverse{22} Y saliendo, no les
podía hablar: y entendieron que había visto visión en el templo: y él
les hablaba por señas, y quedó mudo. \bibverse{23} Y fué, que cumplidos
los días de su oficio, se vino á su casa. \bibverse{24} Y después de
aquellos días concibió su mujer Elisabet, y se encubrió por cinco meses,
diciendo: \bibverse{25} Porque el Señor me ha hecho así en los días en
que miró para quitar mi afrenta entre los hombres. \footnote{\textbf{1:25}
  Gén 30,23}

\hypertarget{anunciando-el-nacimiento-de-jesuxfas}{%
\subsection{Anunciando el nacimiento de
Jesús}\label{anunciando-el-nacimiento-de-jesuxfas}}

\bibverse{26} Y al sexto mes, el ángel Gabriel fué enviado de Dios á una
ciudad de Galilea, llamada Nazaret, \bibverse{27} A una virgen desposada
con un varón que se llamaba José, de la casa de David: y el nombre de la
virgen era María. \bibverse{28} Y entrando el ángel á donde estaba,
dijo, ¡Salve, muy favorecida! el Señor es contigo: bendita tú entre las
mujeres.

\bibverse{29} Mas ella, cuando le vió, se turbó de sus palabras, y
pensaba qué salutación fuese ésta. \bibverse{30} Entonces el ángel le
dijo: María, no temas, porque has hallado gracia cerca de Dios.
\bibverse{31} Y he aquí, concebirás en tu seno, y parirás un hijo, y
llamarás su nombre JESUS. \footnote{\textbf{1:31} Is 7,14; Mat 1,21-23}
\bibverse{32} Este será grande, y será llamado Hijo del Altísimo: y le
dará el Señor Dios el trono de David su padre: \footnote{\textbf{1:32}
  Is 9,6} \bibverse{33} Y reinará en la casa de Jacob por siempre; y de
su reino no habrá fin.

\bibverse{34} Entonces María dijo al ángel: ¿Cómo será esto? porque no
conozco varón.

\bibverse{35} Y respondiendo el ángel le dijo: El Espíritu Santo vendrá
sobre ti, y la virtud del Altísimo te hará sombra; por lo cual también
lo Santo que nacerá, será llamado Hijo de Dios. \footnote{\textbf{1:35}
  Mat 1,18; Mat 1,20} \bibverse{36} Y he aquí, Elisabet tu parienta,
también ella ha concebido hijo en su vejez; y este es el sexto mes á
ella que es llamada la estéril: \bibverse{37} Porque ninguna cosa es
imposible para Dios.

\bibverse{38} Entonces María dijo: He aquí la sierva del Señor; hágase á
mí conforme á tu palabra. Y el ángel partió de ella.

\hypertarget{el-encuentro-de-las-dos-madres-el-saludo-de-elisabeth}{%
\subsection{El encuentro de las dos madres; El saludo de
Elisabeth}\label{el-encuentro-de-las-dos-madres-el-saludo-de-elisabeth}}

\bibverse{39} En aquellos días levantándose María, fué á la montaña con
priesa, á una ciudad de Judá; \bibverse{40} Y entró en casa de Zacarías,
y saludó á Elisabet. \bibverse{41} Y aconteció, que como oyó Elisabet la
salutación de María, la criatura saltó en su vientre; y Elisabet fué
llena del Espíritu Santo, \bibverse{42} Y exclamó á gran voz, y dijo:
Bendita tú entre las mujeres, y bendito el fruto de tu vientre.
\bibverse{43} ¿Y de dónde esto á mí, que la madre de mi Señor venga á
mí? \bibverse{44} Porque he aquí, como llegó la voz de tu salutación á
mis oídos, la criatura saltó de alegría en mi vientre. \bibverse{45} Y
bienaventurada la que creyó, porque se cumplirán las cosas que le fueron
dichas de parte del Señor. \footnote{\textbf{1:45} Luc 11,28}

\hypertarget{canciuxf3n-de-alabanza-de-maruxeda}{%
\subsection{Canción de alabanza de
María}\label{canciuxf3n-de-alabanza-de-maruxeda}}

\bibverse{46} Entonces María dijo: engrandece mi alma al Señor;
\footnote{\textbf{1:46} 1Sam 2,1-10} \bibverse{47} Y mi espíritu se
alegró en Dios mi Salvador. \bibverse{48} Porque ha mirado á la bajeza
de su criada; porque he aquí, desde ahora me dirán bienaventurada todas
las generaciones. \bibverse{49} Porque me ha hecho grandes cosas el
Poderoso; y santo es su nombre. \bibverse{50} Y su misericordia de
generación á generación á los que le temen. \footnote{\textbf{1:50} Sal
  103,13; Sal 103,17} \bibverse{51} Hizo valentía con su brazo: esparció
los soberbios del pensamiento de su corazón. \bibverse{52} Quitó los
poderosos de los tronos, y levantó á los humildes. \bibverse{53} A los
hambrientos hinchió de bienes; y á los ricos envió vacíos. \footnote{\textbf{1:53}
  Sal 34,11; Sal 107,9} \bibverse{54} Recibió á Israel su siervo,
acordándose de la misericordia, \bibverse{55} Como habló á nuestros
padres á Abraham y á su simiente para siempre.

\bibverse{56} Y se quedó María con ella como tres meses: después se
volvió á su casa.

\hypertarget{nacimiento-circuncisiuxf3n-y-juventud-de-juan-himno-de-zacaruxedas}{%
\subsection{Nacimiento, circuncisión y juventud de Juan; Himno de
Zacarías}\label{nacimiento-circuncisiuxf3n-y-juventud-de-juan-himno-de-zacaruxedas}}

\bibverse{57} Y á Elisabet se le cumplió el tiempo de parir, y parió un
hijo. \bibverse{58} Y oyeron los vecinos y los parientes que Dios había
hecho con ella grande misericordia, y se alegraron con ella.
\bibverse{59} Y aconteció, que al octavo día vinieron para circuncidar
al niño; y le llamaban del nombre de su padre, Zacarías. \footnote{\textbf{1:59}
  Gén 17,12} \bibverse{60} Y respondiendo su madre, dijo: No; sino Juan
será llamado.

\bibverse{61} Y le dijeron: ¿Por qué? nadie hay en tu parentela que se
llame de este nombre. \bibverse{62} Y hablaron por señas á su padre,
cómo le quería llamar.

\bibverse{63} Y demandando la tablilla, escribió, diciendo: Juan es su
nombre. Y todos se maravillaron.

\bibverse{64} Y luego fué abierta su boca y su lengua, y habló
bendiciendo á Dios. \bibverse{65} Y fué un temor sobre todos los vecinos
de ellos; y en todas las montañas de Judea fueron divulgadas todas estas
cosas. \bibverse{66} Y todos los que las oían, las conservaban en su
corazón, diciendo: ¿Quién será este niño? Y la mano del Señor estaba con
él.

\hypertarget{himno-profuxe9tico-de-alabanza-de-zacaruxedas}{%
\subsection{Himno profético de alabanza de
Zacarías}\label{himno-profuxe9tico-de-alabanza-de-zacaruxedas}}

\bibverse{67} Y Zacarías su padre fué lleno de Espíritu Santo, y
profetizó, diciendo: \bibverse{68} Bendito el Señor Dios de Israel, que
ha visitado y hecho redención á su pueblo, \bibverse{69} Y nos alzó un
cuerno de salvación en la casa de David su siervo, \footnote{\textbf{1:69}
  Sal 132,17} \bibverse{70} Como habló por boca de sus santos profetas
que fueron desde el principio: \bibverse{71} Salvación de nuestros
enemigos, y de mano de todos los que nos aborrecieron; \bibverse{72}
Para hacer misericordia con nuestros padres, y acordándose de su santo
pacto; \bibverse{73} Del juramento que juró á Abraham nuestro padre, que
nos había de dar, \footnote{\textbf{1:73} Gén 22,16-18; Miq 7,20}
\bibverse{74} Que sin temor librados de nuestros enemigos, le
serviríamos \footnote{\textbf{1:74} Tit 2,12; Tit 1,2-14} \bibverse{75}
En santidad y en justicia delante de él, todos los días nuestros.
\bibverse{76} Y tú, niño, profeta del Altísimo serás llamado; porque
irás ante la faz del Señor, para aparejar sus caminos; \footnote{\textbf{1:76}
  Mal 3,1} \bibverse{77} Dando conocimiento de salud á su pueblo, para
remisión de sus pecados, \footnote{\textbf{1:77} Jer 31,34}
\bibverse{78} Por las entrañas de misericordia de nuestro Dios, con que
nos visitó de lo alto el Oriente, \footnote{\textbf{1:78} Núm 24,17; Is
  60,1-2; Mal 3,20} \bibverse{79} Para dar luz á los que habitan en
tinieblas y en sombra de muerte; para encaminar nuestros pies por camino
de paz. \footnote{\textbf{1:79} Is 9,1}

\bibverse{80} Y el niño crecía, y se fortalecía en espíritu: y estuvo en
los desiertos hasta el día que se mostró á Israel. \footnote{\textbf{1:80}
  Mat 3,1}

\hypertarget{el-decreto-del-emperador-augusto-y-su-significado-para-el-nacimiento-de-jesuxfas}{%
\subsection{El decreto del emperador Augusto y su significado para el
nacimiento de
Jesús}\label{el-decreto-del-emperador-augusto-y-su-significado-para-el-nacimiento-de-jesuxfas}}

\hypertarget{section-1}{%
\section{2}\label{section-1}}

\bibverse{1} Y aconteció en aquellos días que salió edicto de parte de
Augusto César, que toda la tierra fuese empadronada. \bibverse{2} Este
empadronamiento primero fué hecho siendo Cirenio gobernador de la Siria.
\bibverse{3} E iban todos para ser empadronados, cada uno á su ciudad.
\bibverse{4} Y subió José de Galilea, de la ciudad de Nazaret, á Judea,
á la ciudad de David, que se llama Bethlehem, por cuanto era de la casa
y familia de David; \bibverse{5} Para ser empadronado con María su
mujer, desposada con él, la cual estaba en cinta.

\bibverse{6} Y aconteció que estando ellos allí, se cumplieron los días
en que ella había de parir. \bibverse{7} Y parió á su hijo primogénito,
y le envolvió en pañales, y acostóle en un pesebre, porque no había
lugar para ellos en el mesón.

\hypertarget{los-pastores-en-el-campo-y-la-apariciuxf3n-angelical}{%
\subsection{Los pastores en el campo y la aparición
angelical}\label{los-pastores-en-el-campo-y-la-apariciuxf3n-angelical}}

\bibverse{8} Y había pastores en la misma tierra, que velaban y
guardaban las vigilias de la noche sobre su ganado. \bibverse{9} Y he
aquí el ángel del Señor vino sobre ellos, y la claridad de Dios los
cercó de resplandor; y tuvieron gran temor. \bibverse{10} Mas el ángel
les dijo: No temáis; porque he aquí os doy nuevas de gran gozo, que será
para todo el pueblo: \bibverse{11} Que os ha nacido hoy, en la ciudad de
David, un Salvador, que es CRISTO el Señor. \bibverse{12} Y esto os será
por señal: hallaréis al niño envuelto en pañales, echado en un pesebre.
\bibverse{13} Y repentinamente fué con el ángel una multitud de los
ejércitos celestiales, que alababan á Dios, y decían: \footnote{\textbf{2:13}
  Sal 103,20-21; Dan 7,10} \bibverse{14} Gloria en las alturas á Dios, y
en la tierra paz, buena voluntad para con los hombres. \footnote{\textbf{2:14}
  Luc 19,38; Is 57,19; Efes 2,14; Efes 2,17}

\hypertarget{los-pastores-con-el-niuxf1o-jesuxfas-en-beluxe9n}{%
\subsection{Los pastores con el niño Jesús en
Belén}\label{los-pastores-con-el-niuxf1o-jesuxfas-en-beluxe9n}}

\bibverse{15} Y aconteció que como los ángeles se fueron de ellos al
cielo, los pastores dijeron los unos á los otros: Pasemos pues hasta
Bethlehem, y veamos esto que ha sucedido, que el Señor nos ha
manifestado. \bibverse{16} Y vinieron apriesa, y hallaron á María, y á
José, y al niño acostado en el pesebre. \bibverse{17} Y viéndolo,
hicieron notorio lo que les había sido dicho del niño. \bibverse{18} Y
todos los que oyeron, se maravillaron de lo que los pastores les decían.
\bibverse{19} Mas María guardaba todas estas cosas, confiriéndolas en su
corazón. \bibverse{20} Y se volvieron los pastores glorificando y
alabando á Dios de todas las cosas que habían oído y visto, como les
había sido dicho.

\hypertarget{circuncisiuxf3n-y-presentaciuxf3n-de-jesuxfas-en-el-templo}{%
\subsection{Circuncisión y presentación de Jesús en el
templo}\label{circuncisiuxf3n-y-presentaciuxf3n-de-jesuxfas-en-el-templo}}

\bibverse{21} Y pasados los ocho días para circuncidar al niño, llamaron
su nombre JESÚS; el cual le fué puesto por el ángel antes que él fuese
concebido en el vientre. \footnote{\textbf{2:21} Luc 1,31; Luc 1,59; Gén
  17,12}

\bibverse{22} Y como se cumplieron los días de la purificación de ella,
conforme á la ley de Moisés, le trajeron á Jerusalem para presentarle al
Señor, \footnote{\textbf{2:22} Núm 18,15-16} \bibverse{23} (Como está
escrito en la ley del Señor: Todo varón que abriere la matriz, será
llamado santo al Señor), \bibverse{24} Y para dar la ofrenda, conforme á
lo que está dicho en la ley del Señor: un par de tórtolas, ó dos
palominos.

\hypertarget{saludos-himnos-y-profecuxeda-de-simeuxf3n-envejecido}{%
\subsection{Saludos, himnos y profecía de Simeón
envejecido}\label{saludos-himnos-y-profecuxeda-de-simeuxf3n-envejecido}}

\bibverse{25} Y he aquí, había un hombre en Jerusalem, llamado Simeón, y
este hombre, justo y pío, esperaba la consolación de Israel: y el
Espíritu Santo era sobre él. \footnote{\textbf{2:25} Is 40,1; Is 49,13}
\bibverse{26} Y había recibido respuesta del Espíritu Santo, que no
vería la muerte antes que viese al Cristo del Señor. \bibverse{27} Y
vino por Espíritu al templo. Y cuando metieron al niño Jesús sus padres
en el templo, para hacer por él conforme á la costumbre de la ley,
\bibverse{28} Entonces él le tomó en sus brazos, y bendijo á Dios, y
dijo: \bibverse{29} Ahora despides, Señor, á tu siervo, conforme á tu
palabra, en paz; \bibverse{30} Porque han visto mis ojos tu salvación,
\bibverse{31} La cual has aparejado en presencia de todos los pueblos;
\bibverse{32} Luz para ser revelada á los Gentiles, y la gloria de tu
pueblo Israel.

\bibverse{33} Y José y su madre estaban maravillados de las cosas que se
decían de él. \bibverse{34} Y los bendijo Simeón, y dijo á su madre
María: He aquí, éste es puesto para caída y para levantamiento de muchos
en Israel; y para señal á la que será contradicho; \footnote{\textbf{2:34}
  Is 8,13-14; Mat 21,42; Mat 21,44; Hech 28,22; 1Cor 1,23} \bibverse{35}
Y una espada traspasará tu alma de ti misma, para que sean manifestados
los pensamientos de muchos corazones. \footnote{\textbf{2:35} Juan 19,25}

\hypertarget{la-anciana-hanna-saluda-al-niuxf1o-regreso-de-la-sagrada-familia-a-nazaret}{%
\subsection{La anciana Hanna saluda al niño; Regreso de la Sagrada
Familia a
Nazaret}\label{la-anciana-hanna-saluda-al-niuxf1o-regreso-de-la-sagrada-familia-a-nazaret}}

\bibverse{36} Estaba también allí Ana, profetisa, hija de Phanuel, de la
tribu de Aser; la cual había venido en grande edad, y había vivido con
su marido siete años desde su virginidad; \bibverse{37} Y era viuda de
hasta ochenta y cuatro años, que no se apartaba del templo, sirviendo de
noche y de día con ayunos y oraciones. \footnote{\textbf{2:37} 1Tim 5,5}
\bibverse{38} Y ésta, sobreviniendo en la misma hora, juntamente
confesaba al Señor, y hablaba de él á todos los que esperaban la
redención en Jerusalem. \footnote{\textbf{2:38} Is 52,9}

\bibverse{39} Mas como cumplieron todas las cosas según la ley del
Señor, se volvieron á Galilea, á su ciudad de Nazaret. \bibverse{40} Y
el niño crecía, y fortalecíase, y se henchía de sabiduría; y la gracia
de Dios era sobre él.

\hypertarget{el-niuxf1o-jesuxfas-de-doce-auxf1os-en-el-templo}{%
\subsection{El niño Jesús de doce años en el
templo}\label{el-niuxf1o-jesuxfas-de-doce-auxf1os-en-el-templo}}

\bibverse{41} E iban sus padres todos los años á Jerusalem en la fiesta
de la Pascua. \bibverse{42} Y cuando fué de doce años, subieron ellos á
Jerusalem conforme á la costumbre del día de la fiesta. \bibverse{43} Y
acabados los días, volviendo ellos, se quedó el niño Jesús en Jerusalem,
sin saberlo José y su madre. \footnote{\textbf{2:43} Éxod 12,18}
\bibverse{44} Y pensando que estaba en la compañía, anduvieron camino de
un día; y le buscaban entre los parientes y entre los conocidos:
\bibverse{45} Mas como no le hallasen, volvieron á Jerusalem buscándole.
\bibverse{46} Y aconteció, que tres días después le hallaron en el
templo, sentado en medio de los doctores, oyéndoles y preguntándoles.
\bibverse{47} Y todos los que le oían, se pasmaban de su entendimiento y
de sus respuestas. \bibverse{48} Y cuando le vieron, se maravillaron; y
díjole su madre: Hijo, ¿por qué nos has hecho así? He aquí, tu padre y
yo te hemos buscado con dolor.

\bibverse{49} Entonces él les dice: ¿Qué hay? ¿por qué me buscabais? ¿No
sabíais que en los negocios de mi Padre me conviene estar? \bibverse{50}
Mas ellos no entendieron las palabras que les habló. \bibverse{51} Y
descendió con ellos, y vino á Nazaret, y estaba sujeto á ellos. Y su
madre guardaba todas estas cosas en su corazón. \bibverse{52} Y Jesús
crecía en sabiduría, y en edad, y en gracia para con Dios y los hombres.
\footnote{\textbf{2:52} 1Sam 2,26}

\hypertarget{apariciuxf3n-sermuxf3n-penitencial-efectividad-y-captura-de-juan-bautista}{%
\subsection{Aparición, sermón penitencial, efectividad y captura de Juan
Bautista}\label{apariciuxf3n-sermuxf3n-penitencial-efectividad-y-captura-de-juan-bautista}}

\hypertarget{section-2}{%
\section{3}\label{section-2}}

\bibverse{1} Y en el año quince del imperio de Tiberio César, siendo
gobernador de Judea Poncio Pilato, y Herodes tetrarca de Galilea, y su
hermano Felipe tetrarca de Iturea y de la provincia de Traconite, y
Lisanias tetrarca de Abilinia, \bibverse{2} Siendo sumos sacerdotes Anás
y Caifás, vino palabra del Señor sobre Juan, hijo de Zacarías, en el
desierto. \bibverse{3} Y él vino por toda la tierra al rededor del
Jordán predicando el bautismo del arrepentimiento para la remisión de
pecados; \bibverse{4} Como está escrito en el libro de las palabras del
profeta Isaías que dice: Voz del que clama en el desierto: Aparejad el
camino del Señor, haced derechas sus sendas. \bibverse{5} Todo valle se
henchirá, y bajaráse todo monte y collado; y los caminos torcidos serán
enderezados, y los caminos ásperos allanados; \bibverse{6} Y verá toda
carne la salvación de Dios.

\bibverse{7} Y decía á las gentes que salían para ser bautizadas de él:
¡Oh generación de víboras, quién os enseñó á huir de la ira que vendrá?
\bibverse{8} Haced, pues, frutos dignos de arrepentimiento, y no
comencéis á decir en vosotros mismos: Tenemos á Abraham por padre:
porque os digo que puede Dios, aun de estas piedras, levantar hijos á
Abraham. \bibverse{9} Y ya también el hacha está puesta á la raíz de los
árboles: todo árbol pues que no hace buen fruto, es cortado, y echado en
el fuego.

\bibverse{10} Y las gentes le preguntaban, diciendo: ¿Pues qué haremos?

\bibverse{11} Y respondiendo, les dijo: El que tiene dos túnicas, dé al
que no tiene; y el que tiene qué comer, haga lo mismo.

\bibverse{12} Y vinieron también publicanos para ser bautizados, y le
dijeron: Maestro, ¿qué haremos?

\bibverse{13} Y él les dijo: No exijáis más de lo que os está ordenado.

\bibverse{14} Y le preguntaron también los soldados, diciendo: Y
nosotros, ¿qué haremos? Y les dice: No hagáis extorsión á nadie, ni
calumniéis; y contentaos con vuestras pagas.

\bibverse{15} Y estando el pueblo esperando, y pensando todos de Juan en
sus corazones, si él fuese el Cristo, \bibverse{16} Respondió Juan,
diciendo á todos: Yo, á la verdad, os bautizo en agua; mas viene quien
es más poderoso que yo, de quien no soy digno de desatar la correa de
sus zapatos: él os bautizará en Espíritu Santo y fuego; \bibverse{17}
Cuyo bieldo está en su mano, y limpiará su era, y juntará el trigo en su
alfolí, y la paja quemará en fuego que nunca se apagará.

\bibverse{18} Y amonestando, otras muchas cosas también anunciaba al
pueblo. \bibverse{19} Entonces Herodes el tetrarca, siendo reprendido
por él á causa de Herodías, mujer de Felipe su hermano, y de todas las
maldades que había hecho Herodes, \footnote{\textbf{3:19} Mat 14,3-4;
  Mar 6,17-18} \bibverse{20} Añadió también esto sobre todo, que encerró
á Juan en la cárcel.

\hypertarget{bautismo-y-consagraciuxf3n-del-mesuxedas-de-jesuxfas}{%
\subsection{Bautismo y consagración del Mesías de
Jesús}\label{bautismo-y-consagraciuxf3n-del-mesuxedas-de-jesuxfas}}

\bibverse{21} Y aconteció que, como todo el pueblo se bautizaba, también
Jesús fué bautizado; y orando, el cielo se abrió, \bibverse{22} Y
descendió el Espíritu Santo sobre él en forma corporal, como paloma, y
fué hecha una voz del cielo que decía: Tú eres mi Hijo amado, en ti me
he complacido.

\hypertarget{uxe1rbol-genealuxf3gico-de-jesuxfas}{%
\subsection{Árbol genealógico de
Jesús}\label{uxe1rbol-genealuxf3gico-de-jesuxfas}}

\bibverse{23} Y el mismo Jesús comenzaba á ser como de treinta años,
hijo de José, como se creía; que fué hijo de Elí, \footnote{\textbf{3:23}
  Luc 4,22} \bibverse{24} Que fué de Mathat, que fué de Leví, que fué de
Melchî, que fué de Janna, que fué de José, \bibverse{25} Que fué de
Mattathías, que fué de Amós, que fué de Nahum, que fué de Esli,
\bibverse{26} Que fué de Naggai, que fué de Maat, que fué de Mattathías,
que fué de Semei, que fué de José, que fué de Judá, \bibverse{27} Que
fué de Joanna, que fué de Rhesa, que fué de Zorobabel, que fué de
Salathiel, \bibverse{28} Que fué de Neri, que fué de Melchî, que fué de
Abdi, que fué de Cosam, que fué de Elmodam, que fué de Er, \bibverse{29}
Que fué de Josué, que fué de Eliezer, que fué de Joreim, que fué de
Mathat, \bibverse{30} Que fué de Leví, que fué de Simeón, que fué de
Judá, que fué de José, que fué de Jonán, que fué de Eliachîm,
\bibverse{31} Que fué de Melea, que fué de Mainán, que fué de Mattatha,
que fué de Nathán, \bibverse{32} Que fué de David, que fué de Jessé, que
fué de Obed, que fué de Booz, que fué de Salmón, que fué de Naassón,
\footnote{\textbf{3:32} Rut 4,17-22} \bibverse{33} Que fué de Aminadab,
que fué de Aram, que fué de Esrom, que fué de Phares, \footnote{\textbf{3:33}
  Gén 5,1-32; Gén 11,10-26; Gén 21,2-3; Gén 29,35}

\bibverse{34} Que fué de Judá, que fué de Jacob, que fué de Isaac, que
fué de Abraham, que fué de Thara, que fué de Nachôr, \bibverse{35} Que
fué de Saruch, que fué de Ragau, que fué de Phalec, que fué de Heber,
\bibverse{36} Que fué de Sala, que fué de Cainán, que fué de Arphaxad,
que fué de Sem, que fué de Noé, que fué de Lamech, \bibverse{37} Que fué
de Mathusala, que fué de Enoch, que fué de Jared, que fué de Maleleel,
\bibverse{38} Que fué de Cainán, que fué de Enós, que fué de Seth, que
fué de Adam, que fué de Dios.

\hypertarget{la-tentaciuxf3n-de-jesuxfas-como-prueba-de-mesuxedas}{%
\subsection{La tentación de Jesús como prueba de
Mesías}\label{la-tentaciuxf3n-de-jesuxfas-como-prueba-de-mesuxedas}}

\hypertarget{section-3}{%
\section{4}\label{section-3}}

\bibverse{1} Y jesús, lleno del Espíritu Santo, volvió del Jordán, y fué
llevado por el Espíritu al desierto \bibverse{2} Por cuarenta días, y
era tentado del diablo. Y no comió cosa en aquellos días: los cuales
pasados, tuvo hambre.

\bibverse{3} Entonces el diablo le dijo: Si eres Hijo de Dios, di á esta
piedra que se haga pan.

\bibverse{4} Y Jesús respondiéndole, dijo: Escrito está: Que no con pan
solo vivirá el hombre, mas con toda palabra de Dios.

\bibverse{5} Y le llevó el diablo á un alto monte, y le mostró en un
momento de tiempo todos los reinos de la tierra. \bibverse{6} Y le dijo
el diablo: A ti te daré toda esta potestad, y la gloria de ellos; porque
á mí es entregada, y á quien quiero la doy: \bibverse{7} Pues si tú
adorares delante de mí, serán todos tuyos.

\bibverse{8} Y respondiendo Jesús, le dijo: Vete de mí, Satanás, porque
escrito está: A tu Señor Dios adorarás, y á él solo servirás.

\bibverse{9} Y le llevó á Jerusalem, y púsole sobre las almenas del
templo, y le dijo: Si eres Hijo de Dios, échate de aquí abajo:
\bibverse{10} Porque escrito está: Que á sus ángeles mandará de ti, que
te guarden;

\bibverse{11} Y en las manos te llevarán, porque no dañes tu pie en
piedra.

\bibverse{12} Y respondiendo Jesús, le dijo: Dicho está: No tentarás al
Señor tu Dios.

\bibverse{13} Y acabada toda tentación, el diablo se fué de él por un
tiempo. \footnote{\textbf{4:13} Heb 4,15}

\hypertarget{primera-apariciuxf3n-de-jesuxfas-en-galilea-su-predicaciuxf3n-y-rechazo-en-su-natal-nazaret}{%
\subsection{Primera aparición de Jesús en Galilea; su predicación y
rechazo en su natal
Nazaret}\label{primera-apariciuxf3n-de-jesuxfas-en-galilea-su-predicaciuxf3n-y-rechazo-en-su-natal-nazaret}}

\bibverse{14} Y Jesús volvió en virtud del Espíritu á Galilea, y salió
la fama de él por toda la tierra de alrededor. \bibverse{15} Y enseñaba
en las sinagogas de ellos, y era glorificado de todos.

\bibverse{16} Y vino á Nazaret, donde había sido criado; y entró,
conforme á su costumbre, el día del sábado en la sinagoga, y se levantó
á leer. \bibverse{17} Y fuéle dado el libro del profeta Isaías; y como
abrió el libro, halló el lugar donde estaba escrito: \bibverse{18} El
Espíritu del Señor es sobre mí, por cuanto me ha ungido para dar buenas
nuevas á los pobres: me ha enviado para sanar á los quebrantados de
corazón; para pregonar á los cautivos libertad, y á los ciegos vista;
para poner en libertad á los quebrantados: \bibverse{19} Para predicar
el año agradable del Señor. \footnote{\textbf{4:19} Lev 25,10}

\bibverse{20} Y rollando el libro, lo dió al ministro, y sentóse: y los
ojos de todos en la sinagoga estaban fijos en él. \bibverse{21} Y
comenzó á decirles: Hoy se ha cumplido esta Escritura en vuestros oídos.

\bibverse{22} Y todos le daban testimonio, y estaban maravillados de las
palabras de gracia que salían de su boca, y decían: ¿No es éste el hijo
de José?

\bibverse{23} Y les dijo: Sin duda me diréis este refrán: Médico, cúrate
á ti mismo: de tantas cosas que hemos oído haber sido hechas en
Capernaum, haz también aquí en tu tierra. \bibverse{24} Y dijo: De
cierto os digo, que ningún profeta es acepto en su tierra. \footnote{\textbf{4:24}
  Juan 4,44} \bibverse{25} Mas en verdad os digo, que muchas viudas
había en Israel en los días de Elías, cuando el cielo fué cerrado por
tres años y seis meses, que hubo una grande hambre en toda la tierra;
\footnote{\textbf{4:25} 1Re 17,1; 1Re 17,9; 1Re 18,1} \bibverse{26} Pero
á ninguna de ellas fué enviado Elías, sino á Sarepta de Sidón, á una
mujer viuda. \bibverse{27} Y muchos leprosos había en Israel en tiempo
del profeta Eliseo; mas ninguno de ellos fué limpio, sino Naamán el
Siro. \footnote{\textbf{4:27} 2Re 5,14}

\bibverse{28} Entonces todos en la sinagoga fueron llenos de ira, oyendo
estas cosas; \bibverse{29} Y levantándose, le echaron fuera de la
ciudad, y le llevaron hasta la cumbre del monte sobre el cual la ciudad
de ellos estaba edificada, para despeñarle. \bibverse{30} Mas él,
pasando por medio de ellos, se fué.

\hypertarget{jesuxfas-enseuxf1a-en-la-sinagoga-de-capernaum-y-sana-a-un-poseuxeddo}{%
\subsection{Jesús enseña en la sinagoga de Capernaum y sana a un
poseído}\label{jesuxfas-enseuxf1a-en-la-sinagoga-de-capernaum-y-sana-a-un-poseuxeddo}}

\bibverse{31} Y descendió á Capernaum, ciudad de Galilea. Y los enseñaba
en los sábados. \bibverse{32} Y se maravillaban de su doctrina, porque
su palabra era con potestad. \footnote{\textbf{4:32} Mat 7,28-29; Juan
  7,46} \bibverse{33} Y estaba en la sinagoga un hombre que tenía un
espíritu de un demonio inmundo, el cual exclamó á gran voz,
\bibverse{34} Diciendo: Déjanos, ¿qué tenemos contigo, Jesús Nazareno?
¿has venido á destruirnos? Yo te conozco quién eres, el Santo de Dios.

\bibverse{35} Y Jesús le increpó, diciendo: Enmudece, y sal de él.
Entonces el demonio, derribándole en medio, salió de él, y no le hizo
daño alguno.

\bibverse{36} Y hubo espanto en todos, y hablaban unos á otros,
diciendo: ¿Qué palabra es ésta, que con autoridad y potencia manda á los
espíritus inmundos, y salen? \bibverse{37} Y la fama de él se divulgaba
de todas partes por todos los lugares de la comarca.

\hypertarget{sanaciuxf3n-de-la-suegra-de-simuxf3n-pedro-y-otros-enfermos-en-capernaum}{%
\subsection{Sanación de la suegra de Simón Pedro y otros enfermos en
Capernaum}\label{sanaciuxf3n-de-la-suegra-de-simuxf3n-pedro-y-otros-enfermos-en-capernaum}}

\bibverse{38} Y levantándose Jesús de la sinagoga, entró en casa de
Simón: y la suegra de Simón estaba con una grande fiebre; y le rogaron
por ella. \bibverse{39} E inclinándose hacia ella, riñó á la fiebre; y
la fiebre la dejó; y ella levantándose luego, les servía. \bibverse{40}
Y poniéndose el sol, todos los que tenían enfermos de diversas
enfermedades, los traían á él; y él, poniendo las manos sobre cada uno
de ellos, los sanaba. \bibverse{41} Y salían también demonios de muchos,
dando voces, y diciendo: Tú eres el Hijo de Dios. Mas riñéndolos no les
dejaba hablar; porque sabían que él era el Cristo.

\hypertarget{el-sermuxf3n-itinerante-de-jesuxfas-en-las-cercanuxedas-de-capernaum}{%
\subsection{El sermón itinerante de Jesús en las cercanías de
Capernaum}\label{el-sermuxf3n-itinerante-de-jesuxfas-en-las-cercanuxedas-de-capernaum}}

\bibverse{42} Y siendo ya de día salió, y se fué á un lugar desierto: y
las gentes le buscaban, y vinieron hasta él; y le detenían para que no
se apartase de ellos. \bibverse{43} Mas él les dijo: Que también á otras
ciudades es necesario que anuncie el evangelio del reino de Dios; porque
para esto soy enviado. \bibverse{44} Y predicaba en las sinagogas de
Galilea. \footnote{\textbf{4:44} Mat 4,23}

\hypertarget{el-sermuxf3n-de-jesuxfas-en-el-barco-maravilloso-viaje-de-pesca-de-peter-llamando-a-los-primeros-cuatro-discuxedpulos}{%
\subsection{El sermón de Jesús en el barco; maravilloso viaje de pesca
de Peter; Llamando a los primeros cuatro
discípulos}\label{el-sermuxf3n-de-jesuxfas-en-el-barco-maravilloso-viaje-de-pesca-de-peter-llamando-a-los-primeros-cuatro-discuxedpulos}}

\hypertarget{section-4}{%
\section{5}\label{section-4}}

\bibverse{1} Y aconteció, que estando él junto al lago de Genezaret, las
gentes se agolpaban sobre él para oir la palabra de Dios. \bibverse{2} Y
vió dos barcos que estaban cerca de la orilla del lago: y los
pescadores, habiendo descendido de ellos, lavaban sus redes.
\bibverse{3} Y entrado en uno de estos barcos, el cual era de Simón, le
rogó que lo desviase de tierra un poco; y sentándose, enseñaba desde el
barco á las gentes.

\bibverse{4} Y como cesó de hablar, dijo á Simón: Tira á alta mar, y
echad vuestras redes para pescar.

\bibverse{5} Y respondiendo Simón, le dijo: Maestro, habiendo trabajado
toda la noche, nada hemos tomado; mas en tu palabra echaré la red.
\bibverse{6} Y habiéndolo hecho, encerraron gran multitud de pescado,
que su red se rompía. \bibverse{7} E hicieron señas á los compañeros que
estaban en el otro barco, que viniesen á ayudarles; y vinieron, y
llenaron ambos barcos, de tal manera que se anegaban. \bibverse{8} Lo
cual viendo Simón Pedro, se derribó de rodillas á Jesús, diciendo:
Apártate de mí, Señor, porque soy hombre pecador. \footnote{\textbf{5:8}
  Luc 18,13} \bibverse{9} Porque temor le había rodeado, y á todos los
que estaban con él, de la presa de los peces que habían tomado;
\bibverse{10} Y asimismo á Jacobo y á Juan, hijos de Zebedeo, que eran
compañeros de Simón. Y Jesús dijo á Simón: No temas: desde ahora
pescarás hombres.

\bibverse{11} Y como llegaron á tierra los barcos, dejándolo todo, le
siguieron.

\hypertarget{jesuxfas-sana-a-un-leproso-y-escapa-a-la-soledad}{%
\subsection{Jesús sana a un leproso y escapa a la
soledad}\label{jesuxfas-sana-a-un-leproso-y-escapa-a-la-soledad}}

\bibverse{12} Y aconteció que estando en una ciudad, he aquí un hombre
lleno de lepra, el cual viendo á Jesús, postrándose sobre el rostro, le
rogó, diciendo: Señor, si quieres, puedes limpiarme.

\bibverse{13} Entonces, extendiendo la mano, le tocó diciendo: Quiero:
sé limpio. Y luego la lepra se fué de él.

\bibverse{14} Y él le mandó que no lo dijese á nadie: Mas ve, díjole,
muéstrate al sacerdote, y ofrece por tu limpieza, como mandó Moisés,
para testimonio á ellos.

\bibverse{15} Empero tanto más se extendía su fama: y se juntaban muchas
gentes á oir y ser sanadas de sus enfermedades. \bibverse{16} Mas él se
apartaba á los desiertos, y oraba. \footnote{\textbf{5:16} Mar 1,35}

\hypertarget{curaciuxf3n-de-un-paraluxedtico-jesuxfas-perdona-los-pecados}{%
\subsection{Curación de un paralítico; Jesús perdona los
pecados}\label{curaciuxf3n-de-un-paraluxedtico-jesuxfas-perdona-los-pecados}}

\bibverse{17} Y aconteció un día, que él estaba enseñando, y los
Fariseos y doctores de la ley estaban sentados, los cuales habían venido
de todas las aldeas de Galilea, y de Judea y Jerusalem: y la virtud del
Señor estaba allí para sanarlos. \bibverse{18} Y he aquí unos hombres,
que traían sobre un lecho un hombre que estaba paralítico; y buscaban
meterle, y ponerle delante de él. \bibverse{19} Y no hallando por donde
meterle á causa de la multitud, subieron encima de la casa, y por el
tejado le bajaron con el lecho en medio, delante de Jesús; \bibverse{20}
El cual, viendo la fe de ellos, le dice: Hombre, tus pecados te son
perdonados.

\bibverse{21} Entonces los escribas y los Fariseos comenzaron á pensar,
diciendo: ¿Quién es éste que habla blasfemias? ¿Quién puede perdonar
pecados sino sólo Dios?

\bibverse{22} Jesús entonces, conociendo los pensamientos de ellos,
respondiendo les dijo: ¿Qué pensáis en vuestros corazones? \bibverse{23}
¿Qué es más fácil, decir: Tus pecados te son perdonados, ó decir:
Levántate y anda? \bibverse{24} Pues para que sepáis que el Hijo del
hombre tiene potestad en la tierra de perdonar pecados, (dice al
paralítico): A ti digo, levántate, toma tu lecho, y vete á tu casa.
\footnote{\textbf{5:24} Juan 5,36}

\bibverse{25} Y luego, levantándose en presencia de ellos, y tomando
aquel en que estaba echado, se fué á su casa, glorificando á Dios.
\bibverse{26} Y tomó espanto á todos, y glorificaban á Dios; y fueron
llenos de temor, diciendo: Hemos visto maravillas hoy.

\hypertarget{llamando-al-recaudador-de-impuestos-levi-jesuxfas-como-compauxf1ero-de-mesa-para-recaudadores-de-impuestos-y-pecadores}{%
\subsection{Llamando al recaudador de impuestos Levi; Jesús como
compañero de mesa para recaudadores de impuestos y
pecadores}\label{llamando-al-recaudador-de-impuestos-levi-jesuxfas-como-compauxf1ero-de-mesa-para-recaudadores-de-impuestos-y-pecadores}}

\bibverse{27} Y después de estas cosas salió, y vió á un publicano
llamado Leví, sentado al banco de los públicos tributos, y le dijo:
Sígueme.

\bibverse{28} Y dejadas todas las cosas, levantándose, le siguió.
\bibverse{29} E hizo Leví gran banquete en su casa; y había mucha
compañía de publicanos y de otros, los cuales estaban á la mesa con
ellos. \bibverse{30} Y los escribas y los Fariseos murmuraban contra sus
discípulos, diciendo: ¿Por qué coméis y bebéis con los publicanos y
pecadores?

\bibverse{31} Y respondiendo Jesús, les dijo: Los que están sanos no
necesitan médico, sino los que están enfermos. \bibverse{32} No he
venido á llamar justos, sino pecadores á arrepentimiento.

\hypertarget{la-pregunta-del-ayuno-de-los-discuxedpulos-de-juan-y-los-fariseos-jesuxfas-justifica-lo-nuevo-en-su-comportamiento}{%
\subsection{La pregunta del ayuno de los discípulos de Juan y los
fariseos; Jesús justifica lo nuevo en su
comportamiento}\label{la-pregunta-del-ayuno-de-los-discuxedpulos-de-juan-y-los-fariseos-jesuxfas-justifica-lo-nuevo-en-su-comportamiento}}

\bibverse{33} Entonces ellos le dijeron: ¿Por qué los discípulos de Juan
ayunan muchas veces y hacen oraciones, y asimismo los de los Fariseos, y
tus discípulos comen y beben?

\bibverse{34} Y él les dijo: ¿Podéis hacer que los que están de bodas
ayunen, entre tanto que el esposo está con ellos? \bibverse{35} Empero
vendrán días cuando el esposo les será quitado: entonces ayunarán en
aquellos días.

\bibverse{36} Y les decía también una parábola: Nadie mete remiendo de
paño nuevo en vestido viejo; de otra manera el nuevo rompe, y al viejo
no conviene remiendo nuevo. \bibverse{37} Y nadie echa vino nuevo en
cueros viejos; de otra manera el vino nuevo romperá los cueros, y el
vino se derramará, y los cueros se perderán. \bibverse{38} Mas el vino
nuevo en cueros nuevos se ha de echar; y lo uno y lo otro se conserva.
\bibverse{39} Y ninguno que bebiere del añejo, quiere luego el nuevo;
porque dice: El añejo es mejor.

\hypertarget{el-arranco-de-espigas-de-los-discuxedpulos-en-suxe1bado-la-primera-disputa-de-jesuxfas-con-los-fariseos-sobre-la-santificaciuxf3n-del-duxeda-de-reposo}{%
\subsection{El arranco de espigas de los discípulos en sábado; La
primera disputa de Jesús con los fariseos sobre la santificación del día
de
reposo}\label{el-arranco-de-espigas-de-los-discuxedpulos-en-suxe1bado-la-primera-disputa-de-jesuxfas-con-los-fariseos-sobre-la-santificaciuxf3n-del-duxeda-de-reposo}}

\hypertarget{section-5}{%
\section{6}\label{section-5}}

\bibverse{1} Y aconteció que pasando él por los sembrados en un sábado
segundo del primero, sus discípulos arrancaban espigas, y comían,
restregándolas con las manos. \footnote{\textbf{6:1} Luc 13,10-17; Luc
  14,1-6} \bibverse{2} Y algunos de los Fariseos les dijeron: ¿Por qué
hacéis lo que no es lícito hacer en los sábados?

\bibverse{3} Y respondiendo Jesús les dijo: ¿Ni aun esto habéis leído,
qué hizo David cuando tuvo hambre, él, y los que con él estaban;
\bibverse{4} Cómo entró en la casa de Dios, y tomó los panes de la
proposición, y comió, y dió también á los que estaban con él, los cuales
no era lícito comer, sino á solos los sacerdotes? \footnote{\textbf{6:4}
  Lev 24,9} \bibverse{5} Y les decía: El Hijo del hombre es Señor aun
del sábado.

\hypertarget{sanaciuxf3n-del-hombre-con-el-brazo-paralizado-en-suxe1bado-el-segundo-argumento-sobre-la-observancia-del-suxe1bado}{%
\subsection{Sanación del hombre con el brazo paralizado en sábado; el
segundo argumento sobre la observancia del
sábado}\label{sanaciuxf3n-del-hombre-con-el-brazo-paralizado-en-suxe1bado-el-segundo-argumento-sobre-la-observancia-del-suxe1bado}}

\bibverse{6} Y aconteció también en otro sábado, que él entró en la
sinagoga y enseñaba; y estaba allí un hombre que tenía la mano derecha
seca. \bibverse{7} Y le acechaban los escribas y los Fariseos, si
sanaría en sábado, por hallar de qué le acusasen. \bibverse{8} Mas él
sabía los pensamientos de ellos; y dijo al hombre que tenía la mano
seca: Levántate, y ponte en medio. Y él levantándose, se puso en pie.
\bibverse{9} Entonces Jesús les dice: Os preguntaré una cosa: ¿Es lícito
en sábados hacer bien, ó hacer mal? ¿salvar la vida, ó quitarla?
\bibverse{10} Y mirándolos á todos alrededor, dice al hombre: Extiende
tu mano. Y él lo hizo así, y su mano fué restaurada. \bibverse{11} Y
ellos se llenaron de rabia; y hablaban los unos á los otros qué harían á
Jesús.

\hypertarget{llamadas-y-nombres-de-los-doce-apuxf3stoles-afluencia-de-personas-muchas-curaciones}{%
\subsection{Llamadas y nombres de los doce apóstoles; Afluencia de
personas; muchas
curaciones}\label{llamadas-y-nombres-de-los-doce-apuxf3stoles-afluencia-de-personas-muchas-curaciones}}

\bibverse{12} Y aconteció en aquellos días, que fué al monte á orar, y
pasó la noche orando á Dios. \bibverse{13} Y como fué de día, llamó á
sus discípulos, y escogió doce de ellos, á los cuales también llamó
apóstoles: \footnote{\textbf{6:13} Mat 10,2-4; Hech 1,13} \bibverse{14}
A Simón, al cual también llamó Pedro, y á Andrés su hermano, Jacobo y
Juan, Felipe y Bartolomé, \bibverse{15} Mateo y Tomás, Jacobo hijo de
Alfeo, y Simón el que se llama Celador, \bibverse{16} Judas hermano de
Jacobo, y Judas Iscariote, que también fué el traidor.

\bibverse{17} Y descendió con ellos, y se paró en un lugar llano, y la
compañía de sus discípulos, y una grande multitud de pueblo de toda
Judea y de Jerusalem, y de la costa de Tiro y de Sidón, que habían
venido á oirle, y para ser sanados de sus enfermedades; \bibverse{18} Y
los que habían sido atormentados de espíritus inmundos: y estaban
curados. \bibverse{19} Y toda la gente procuraba tocarle; porque salía
de él virtud, y sanaba á todos.

\hypertarget{el-sermuxf3n-del-monte}{%
\subsection{El sermón del monte}\label{el-sermuxf3n-del-monte}}

\bibverse{20} Y alzando él los ojos á sus discípulos, decía:
Bienaventurados vosotros los pobres; porque vuestro es el reino de Dios.
\bibverse{21} Bienaventurados los que ahora tenéis hambre; porque seréis
saciados. Bienaventurados los que ahora lloráis, porque reiréis.
\footnote{\textbf{6:21} Apoc 7,16-17} \bibverse{22} Bienaventurados
seréis, cuando los hombres os aborrecieren, y cuando os apartaren de sí,
y os denostaren, y desecharen vuestro nombre como malo, por el Hijo del
hombre. \footnote{\textbf{6:22} Juan 15,18-19} \bibverse{23} Gozaos en
aquel día, y alegraos; porque he aquí vuestro galardón es grande en los
cielos; porque así hacían sus padres á los profetas. \bibverse{24} Mas
¡ay de vosotros, ricos! porque tenéis vuestro consuelo. \footnote{\textbf{6:24}
  Mat 19,23; Sant 5,1} \bibverse{25} ¡Ay de vosotros, los que estáis
hartos! porque tendréis hambre. ¡Ay de vosotros, los que ahora reís!
porque lamentaréis y lloraréis. \bibverse{26} ¡Ay de vosotros, cuando
todos los hombres dijeren bien de vosotros! porque así hacían sus padres
á los falsos profetas.

\hypertarget{mandamiento-de-amar-al-enemigo-renuncia-a-represalias}{%
\subsection{Mandamiento de amar al enemigo; Renuncia a
represalias}\label{mandamiento-de-amar-al-enemigo-renuncia-a-represalias}}

\bibverse{27} Mas á vosotros los que oís, digo: Amad á vuestros
enemigos, haced bien á los que os aborrecen; \bibverse{28} Bendecid á
los que os maldicen, y orad por los que os calumnian. \footnote{\textbf{6:28}
  1Cor 4,12} \bibverse{29} Y al que te hiriere en la mejilla, dale
también la otra; y al que te quitare la capa, ni aun el sayo le
defiendas. \bibverse{30} Y á cualquiera que te pidiere, da; y al que
tomare lo que es tuyo, no vuelvas á pedir.

\bibverse{31} Y como queréis que os hagan los hombres, así hacedles
también vosotros:

\bibverse{32} Porque si amáis á los que os aman, ¿qué gracias tendréis?
porque también los pecadores aman á los que los aman. \bibverse{33} Y si
hiciereis bien á los que os hacen bien, ¿qué gracias tendréis? porque
también los pecadores hacen lo mismo. \bibverse{34} Y si prestareis á
aquellos de quienes esperáis recibir, ¿qué gracias tendréis? porque
también los pecadores prestan á los pecadores, para recibir otro tanto.
\footnote{\textbf{6:34} Lev 25,35-36} \bibverse{35} Amad, pues, á
vuestros enemigos, y haced bien, y prestad, no esperando de ello nada; y
será vuestro galardón grande, y seréis hijos del Altísimo: porque él es
benigno para con los ingratos y malos. \bibverse{36} Sed pues
misericordiosos, como también vuestro Padre es misericordioso.

\hypertarget{advertencia-contra-el-juicio-y-la-hipuxf3crita-voluntad-de-mejorar}{%
\subsection{Advertencia contra el juicio y la hipócrita voluntad de
mejorar}\label{advertencia-contra-el-juicio-y-la-hipuxf3crita-voluntad-de-mejorar}}

\bibverse{37} No juzguéis, y no seréis juzgados: no condenéis, y no
seréis condenados: perdonad, y seréis perdonados.

\bibverse{38} Dad, y se os dará; medida buena, apretada, remecida, y
rebosando darán en vuestro seno: porque con la misma medida que
midiereis, os será vuelto á medir. \footnote{\textbf{6:38} Mar 4,24}

\bibverse{39} Y les decía una parábola: ¿Puede el ciego guiar al ciego?
¿No caerán ambos en el hoyo? \footnote{\textbf{6:39} Mat 15,14}
\bibverse{40} El discípulo no es sobre su maestro; mas cualquiera que
fuere como el maestro, será perfecto. \footnote{\textbf{6:40} Mat
  10,24-25; Juan 15,20} \bibverse{41} ¿Por qué miras la paja que está en
el ojo de tu hermano, y la viga que está en tu propio ojo no consideras?
\bibverse{42} ¿O cómo puedes decir á tu hermano: Hermano, deja, echaré
fuera la paja que está en tu ojo, no mirando tú la viga que está en tu
ojo? Hipócrita, echa primero fuera de tu ojo la viga, y entonces verás
bien para sacar la paja que está en el ojo de tu hermano.

\hypertarget{tanto-la-obediencia-de-la-fe-como-la-incredulidad-de-la-gente-viene-del-corazuxf3n-como-el-fruto-de-la-especie-del-uxe1rbol}{%
\subsection{Tanto la obediencia de la fe como la incredulidad de la
gente viene del corazón, como el fruto de la especie del
árbol}\label{tanto-la-obediencia-de-la-fe-como-la-incredulidad-de-la-gente-viene-del-corazuxf3n-como-el-fruto-de-la-especie-del-uxe1rbol}}

\bibverse{43} Porque no es buen árbol el que da malos frutos; ni árbol
malo el que da buen fruto. \bibverse{44} Porque cada árbol por su fruto
es conocido: que no cogen higos de los espinos, ni vendimian uvas de las
zarzas. \bibverse{45} El buen hombre del buen tesoro de su corazón saca
bien; y el mal hombre del mal tesoro de su corazón saca mal; porque de
la abundancia del corazón habla su boca.

\hypertarget{sea-un-hacedor-de-la-palabra-no-solo-un-oyente}{%
\subsection{Sea un hacedor de la palabra, no solo un
oyente}\label{sea-un-hacedor-de-la-palabra-no-solo-un-oyente}}

\bibverse{46} ¿Por qué me llamáis, Señor, Señor, y no hacéis lo que
digo? \bibverse{47} Todo aquel que viene á mí, y oye mis palabras, y las
hace, os enseñaré á quién es semejante: \bibverse{48} Semejante es al
hombre que edifica una casa, el cual cavó y ahondó, y puso el fundamento
sobre la peña; y cuando vino una avenida, el río dió con ímpetu en
aquella casa, mas no la pudo menear: porque estaba fundada sobre la
peña. \bibverse{49} Mas el que oyó y no hizo, semejante es al hombre que
edificó su casa sobre tierra, sin fundamento; en la cual el río dió con
ímpetu, y luego cayó; y fué grande la ruina de aquella casa.

\hypertarget{sanaciuxf3n-del-siervo-del-centuriuxf3n-de-capernaum}{%
\subsection{Sanación del siervo del centurión de
Capernaum}\label{sanaciuxf3n-del-siervo-del-centuriuxf3n-de-capernaum}}

\hypertarget{section-6}{%
\section{7}\label{section-6}}

\bibverse{1} Y como acabó todas sus palabras oyéndole el pueblo, entró
en Capernaum. \bibverse{2} Y el siervo de un centurión, al cual tenía él
en estima, estaba enfermo y á punto de morir. \bibverse{3} Y como oyó
hablar de Jesús, envió á él los ancianos de los Judíos, rogándole que
viniese y librase á su siervo. \bibverse{4} Y viniendo ellos á Jesús,
rogáronle con diligencia, diciéndole: Porque es digno de concederle
esto; \bibverse{5} Que ama nuestra nación, y él nos edificó una
sinagoga. \bibverse{6} Y Jesús fué con ellos. Mas como ya no estuviesen
lejos de su casa, envió el centurión amigos á él, diciéndole: Señor, no
te incomodes, que no soy digno que entres debajo de mi tejado;
\bibverse{7} Por lo cual ni aun me tuve por digno de venir á ti; mas di
la palabra, y mi siervo será sano. \bibverse{8} Porque también yo soy
hombre puesto en potestad, que tengo debajo de mí soldados; y digo á
éste: Ve, y va; y al otro: Ven, y viene; y á mi siervo: Haz esto, y lo
hace.

\bibverse{9} Lo cual oyendo Jesús, se maravilló de él, y vuelto, dijo á
las gentes que le seguían: Os digo que ni aun en Israel he hallado tanta
fe. \bibverse{10} Y vueltos á casa los que habían sido enviados,
hallaron sano al siervo que había estado enfermo.

\hypertarget{criar-al-joven-en-nain}{%
\subsection{Criar al joven en Nain}\label{criar-al-joven-en-nain}}

\bibverse{11} Y aconteció después, que él iba á la ciudad que se llama
Naín, é iban con él muchos de sus discípulos, y gran compañía.
\bibverse{12} Y como llegó cerca de la puerta de la ciudad, he aquí que
sacaban fuera á un difunto, unigénito de su madre, la cual también era
viuda: y había con ella grande compañía de la ciudad. \footnote{\textbf{7:12}
  1Re 17,17} \bibverse{13} Y como el Señor la vió, compadecióse de ella,
y le dice: No llores. \bibverse{14} Y acercándose, tocó el féretro: y
los que lo llevaban, pararon. Y dice: Mancebo, á ti digo, levántate.
\bibverse{15} Entonces se incorporó el que había muerto, y comenzó á
hablar. Y dióle á su madre. \footnote{\textbf{7:15} 1Re 17,23; 2Re 4,36}

\bibverse{16} Y todos tuvieron miedo, y glorificaban á Dios, diciendo:
Que un gran profeta se ha levantado entre nosotros; y que Dios ha
visitado á su pueblo. \footnote{\textbf{7:16} Luc 1,68; Mat 16,14}
\bibverse{17} Y salió esta fama de él por toda Judea, y por toda la
tierra de alrededor.

\hypertarget{embajada-de-juan-el-bautista-la-respuesta-y-el-testimonio-de-jesuxfas-sobre-juan}{%
\subsection{Embajada de Juan el Bautista; La respuesta y el testimonio
de Jesús sobre
Juan}\label{embajada-de-juan-el-bautista-la-respuesta-y-el-testimonio-de-jesuxfas-sobre-juan}}

\bibverse{18} Y sus discípulos dieron á Juan las nuevas de todas estas
cosas: y llamó Juan á dos de sus discípulos, \bibverse{19} Y envió á
Jesús, diciendo: ¿Eres tú aquél que había de venir, ó esperaremos á
otro? \bibverse{20} Y como los hombres vinieron á él, dijeron: Juan el
Bautista nos ha enviado á ti, diciendo: ¿Eres tú aquél que había de
venir, ó esperaremos á otro?

\bibverse{21} Y en la misma hora sanó á muchos de enfermedades y plagas,
y de espíritus malos; y á muchos ciegos dió la vista. \bibverse{22} Y
respondiendo Jesús, les dijo: Id, dad las nuevas á Juan de lo que habéis
visto y oído: que los ciegos ven, los cojos andan, los leprosos son
limpiados, los sordos oyen, los muertos resucitan, á los pobres es
anunciado el evangelio: \bibverse{23} Y bienaventurado es el que no
fuere escandalizado en mí.

\bibverse{24} Y como se fueron los mensajeros de Juan, comenzó á hablar
de Juan á las gentes: ¿Qué salisteis á ver al desierto? ¿una caña que es
agitada por el viento? \bibverse{25} Mas ¿qué salisteis á ver? ¿un
hombre cubierto de vestidos delicados? He aquí, los que están en vestido
precioso, y viven en delicias, en los palacios de los reyes están.
\bibverse{26} Mas ¿qué salisteis á ver? ¿un profeta? También os digo, y
aun más que profeta. \footnote{\textbf{7:26} Luc 1,76} \bibverse{27}
Este es de quien está escrito: He aquí, envío mi mensajero delante de tu
faz, el cual aparejará tu camino delante de ti.

\bibverse{28} Porque os digo que entre los nacidos de mujeres, no hay
mayor profeta que Juan el Bautista: mas el más pequeño en el reino de
los cielos es mayor que él.

\bibverse{29} Y todo el pueblo oyéndole, y los publicanos, justificaron
á Dios, bautizándose con el bautismo de Juan. \footnote{\textbf{7:29}
  Luc 3,7; Luc 3,12; Mat 21,32} \bibverse{30} Mas los Fariseos y los
sabios de la ley, desecharon el consejo de Dios contra sí mismos, no
siendo bautizados de él. \footnote{\textbf{7:30} Hech 13,46}

\bibverse{31} Y dice el Señor: ¿A quién, pues, compararé los hombres de
esta generación, y á qué son semejantes? \bibverse{32} Semejantes son á
los muchachos sentados en la plaza, y que dan voces los unos á los
otros, y dicen: Os tañimos con flautas, y no bailasteis: os endechamos,
y no llorasteis. \bibverse{33} Porque vino Juan el Bautista, que ni
comía pan, ni bebía vino, y decís: Demonio tiene. \bibverse{34} Vino el
Hijo del hombre, que come y bebe, y decís: He aquí un hombre comilón, y
bebedor de vino, amigo de publicanos y de pecadores. \footnote{\textbf{7:34}
  Luc 15,2} \bibverse{35} Mas la sabiduría es justificada de todos sus
hijos. \footnote{\textbf{7:35} 1Cor 1,24-30}

\hypertarget{la-unciuxf3n-de-jesuxfas-por-la-gran-pecadora}{%
\subsection{La unción de Jesús por la gran
pecadora}\label{la-unciuxf3n-de-jesuxfas-por-la-gran-pecadora}}

\bibverse{36} Y le rogó uno de los Fariseos, que comiese con él. Y
entrado en casa del Fariseo, sentóse á la mesa. \footnote{\textbf{7:36}
  Luc 11,37} \bibverse{37} Y he aquí una mujer que había sido pecadora
en la ciudad, como entendió que estaba á la mesa en casa de aquel
Fariseo, trajo un alabastro de ungüento, \footnote{\textbf{7:37} Mar
  14,3-9} \bibverse{38} Y estando detrás á sus pies, comenzó llorando á
regar con lágrimas sus pies, y los limpiaba con los cabellos de su
cabeza; y besaba sus pies, y los ungía con el ungüento. \bibverse{39} Y
como vió esto el Fariseo que le había convidado, habló entre sí,
diciendo: Este, si fuera profeta, conocería quién y cuál es la mujer que
le toca, que es pecadora.

\bibverse{40} Entonces respondiendo Jesús, le dijo: Simón, una cosa
tengo que decirte. Y él dice: Di, Maestro.

\bibverse{41} Un acreedor tenía dos deudores: el uno le debía quinientos
denarios, y el otro cincuenta; \bibverse{42} Y no teniendo ellos de qué
pagar, perdonó á ambos. Di, pues, ¿cuál de éstos le amará más?

\bibverse{43} Y respondiendo Simón, dijo: Pienso que aquél al cual
perdonó más. Y él le dijo: Rectamente has juzgado.

\bibverse{44} Y vuelto á la mujer, dijo á Simón: ¿Ves esta mujer? Entré
en tu casa, no diste agua para mis pies; mas ésta ha regado mis pies con
lágrimas, y los ha limpiado con los cabellos. \footnote{\textbf{7:44}
  Gén 18,4} \bibverse{45} No me diste beso, mas ésta, desde que entré,
no ha cesado de besar mis pies. \footnote{\textbf{7:45} Rom 16,16}
\bibverse{46} No ungiste mi cabeza con óleo; mas ésta ha ungido con
ungüento mis pies. \bibverse{47} Por lo cual te digo que sus muchos
pecados son perdonados, porque amó mucho; mas al que se perdona poco,
poco ama. \bibverse{48} Y á ella dijo: Los pecados te son perdonados.

\bibverse{49} Y los que estaban juntamente sentados á la mesa,
comenzaron á decir entre sí: ¿Quién es éste, que también perdona
pecados? \footnote{\textbf{7:49} Luc 5,21}

\bibverse{50} Y dijo á la mujer: Tu fe te ha salvado, ve en paz.
\footnote{\textbf{7:50} Luc 8,48; Luc 17,19; Luc 18,42}

\hypertarget{el-compauxf1ero-constante-de-jesuxfas-en-sus-andanzas-las-sirvientas-galileas}{%
\subsection{El compañero constante de Jesús en sus andanzas; las
sirvientas
galileas}\label{el-compauxf1ero-constante-de-jesuxfas-en-sus-andanzas-las-sirvientas-galileas}}

\hypertarget{section-7}{%
\section{8}\label{section-7}}

\bibverse{1} Y aconteció después, que él caminaba por todas las ciudades
y aldeas, predicando y anunciando el evangelio del reino de Dios, y los
doce con él, \bibverse{2} Y algunas mujeres que habían sido curadas de
malos espíritus y de enfermedades: María, que se llamaba Magdalena, de
la cual habían salido siete demonios, \footnote{\textbf{8:2} Mar
  15,40-41; Mar 16,9} \bibverse{3} Y Juana, mujer de Chuza, procurador
de Herodes, y Susana, y otras muchas que le servían de sus haciendas.

\hypertarget{paruxe1bola-del-sembrador-y-el-campo-de-cuatro-tipos}{%
\subsection{Parábola del sembrador y el campo de cuatro
tipos}\label{paruxe1bola-del-sembrador-y-el-campo-de-cuatro-tipos}}

\bibverse{4} Y como se juntó una grande compañía, y los que estaban en
cada ciudad vinieron á él, dijo por una parábola: \bibverse{5} Uno que
sembraba, salió á sembrar su simiente; y sembrando, una parte cayó junto
al camino, y fué hollada; y las aves del cielo la comieron. \bibverse{6}
Y otra parte cayó sobre la piedra; y nacida, se secó, porque no tenía
humedad. \bibverse{7} Y otra parte cayó entre las espinas; y naciendo
las espinas juntamente, la ahogaron. \bibverse{8} Y otra parte cayó en
buena tierra, y cuando fué nacida, llevó fruto á ciento por uno.
Diciendo estas cosas clamaba: El que tiene oídos para oir, oiga.

\hypertarget{significado-y-propuxf3sito-de-las-paruxe1bolas-interpretaciuxf3n-de-la-paruxe1bola-del-sembrador}{%
\subsection{Significado y propósito de las parábolas; Interpretación de
la parábola del
sembrador}\label{significado-y-propuxf3sito-de-las-paruxe1bolas-interpretaciuxf3n-de-la-paruxe1bola-del-sembrador}}

\bibverse{9} Y sus discípulos le preguntaron, diciendo, qué era esta
parábola.

\bibverse{10} Y él dijo: A vosotros es dado conocer los misterios del
reino de Dios; mas á los otros por parábolas, para que viendo no vean, y
oyendo no entiendan.

\bibverse{11} Es pues ésta la parábola: La simiente es la palabra de
Dios. \bibverse{12} Y los de junto al camino, éstos son los que oyen; y
luego viene el diablo, y quita la palabra de su corazón, porque no crean
y se salven. \bibverse{13} Y los de sobre la piedra, son los que
habiendo oído, reciben la palabra con gozo; mas éstos no tienen raíces;
que á tiempo creen, y en el tiempo de la tentación se apartan.
\bibverse{14} Y la que cayó entre las espinas, éstos son los que oyeron;
mas yéndose, son ahogados de los cuidados y de las riquezas y de los
pasatiempos de la vida, y no llevan fruto. \bibverse{15} Mas la que en
buena tierra, éstos son los que con corazón bueno y recto retienen la
palabra oída, y llevan fruto en paciencia. \footnote{\textbf{8:15} Hech
  16,14}

\bibverse{16} Ninguno que enciende la antorcha la cubre con vasija, ó la
pone debajo de la cama; mas la pone en un candelero, para que los que
entran vean la luz. \footnote{\textbf{8:16} Mat 5,15} \bibverse{17}
Porque no hay cosa oculta, que no haya de ser manifestada; ni cosa
escondida, que no haya de ser entendida, y de venir á luz. \footnote{\textbf{8:17}
  Mat 10,26; 1Cor 4,5} \bibverse{18} Mirad pues cómo oís; porque á
cualquiera que tuviere, le será dado; y á cualquiera que no tuviere, aun
lo que parece tener le será quitado. \footnote{\textbf{8:18} Mat 25,29}

\hypertarget{los-verdaderos-parientes-de-jesuxfas}{%
\subsection{Los verdaderos parientes de
Jesús}\label{los-verdaderos-parientes-de-jesuxfas}}

\bibverse{19} Y vinieron á él su madre y hermanos; y no podían llegar á
él por causa de la multitud. \bibverse{20} Y le fué dado aviso,
diciendo: Tu madre y tus hermanos están fuera, que quieren verte.

\bibverse{21} El entonces respondiendo, les dijo: Mi madre y mis
hermanos son los que oyen la palabra de Dios, y la ejecutan.

\hypertarget{jesuxfas-apacigua-la-tormenta-del-mar}{%
\subsection{Jesús apacigua la tormenta del
mar}\label{jesuxfas-apacigua-la-tormenta-del-mar}}

\bibverse{22} Y aconteció un día que él entró en un barco con sus
discípulos, y les dijo: Pasemos á la otra parte del lago. Y partieron.
\bibverse{23} Pero mientras ellos navegaban, él se durmió. Y sobrevino
una tempestad de viento en el lago; y henchían de agua, y peligraban.
\bibverse{24} Y llegándose á él, le despertaron, diciendo: ¡Maestro,
Maestro, que perecemos! Y despertado él, increpó al viento y á la
tempestad del agua; y cesaron, y fué hecha bonanza. \bibverse{25} Y les
dijo: ¿Qué es de vuestra fe? Y atemorizados, se maravillaban, diciendo
los unos á los otros: ¿Quién es éste, que aun á los vientos y al agua
manda, y le obedecen?

\hypertarget{jesuxfas-sana-a-los-poseuxeddos-en-la-tierra-de-los-gergesen}{%
\subsection{Jesús sana a los poseídos en la tierra de los
Gergesen}\label{jesuxfas-sana-a-los-poseuxeddos-en-la-tierra-de-los-gergesen}}

\bibverse{26} Y navegaron á la tierra de los Gadarenos, que está delante
de Galilea. \bibverse{27} Y saliendo él á tierra, le vino al encuentro
de la ciudad un hombre que tenía demonios ya de mucho tiempo; y no
vestía vestido, ni estaba en casa, sino por los sepulcros. \bibverse{28}
El cual, como vió á Jesús, exclamó y se postró delante de él, y dijo á
gran voz: ¿Qué tengo yo contigo, Jesús, Hijo del Dios Altísimo? Ruégote
que no me atormentes. \bibverse{29} (Porque mandaba al espíritu inmundo
que saliese del hombre: porque ya de mucho tiempo le arrebataba; y le
guardaban preso con cadenas y grillos; mas rompiendo las prisiones, era
agitado del demonio por los desiertos.)

\bibverse{30} Y le preguntó Jesús, diciendo: ¿Qué nombre tienes? Y él
dijo: Legión. Porque muchos demonios habían entrado en él.

\bibverse{31} Y le rogaban que no les mandase ir al abismo.

\bibverse{32} Y había allí un hato de muchos puercos que pacían en el
monte; y le rogaron que los dejase entrar en ellos; y los dejó.
\bibverse{33} Y salidos los demonios del hombre, entraron en los
puercos; y el hato se arrojó de un despeñadero en el lago, y ahogóse.

\bibverse{34} Y los pastores, como vieron lo que había acontecido,
huyeron, y yendo dieron aviso en la ciudad y por las heredades.

\bibverse{35} Y salieron á ver lo que había acontecido; y vinieron á
Jesús, y hallaron sentado al hombre de quien habían salido los demonios,
vestido, y en su juicio, á los pies de Jesús; y tuvieron miedo.
\bibverse{36} Y les contaron los que lo habían visto, cómo había sido
salvado aquel endemoniado. \bibverse{37} Entonces toda la multitud de la
tierra de los Gadarenos alrededor, le rogaron que se fuese de ellos;
porque tenían gran temor. Y él, subiendo en el barco, volvióse.
\bibverse{38} Y aquel hombre, de quien habían salido los demonios, le
rogó para estar con él; mas Jesús le despidió, diciendo: \bibverse{39}
Vuélvete á tu casa, y cuenta cuán grandes cosas ha hecho Dios contigo. Y
él se fué, publicando por toda la ciudad cuán grandes cosas había hecho
Jesús con él.

\hypertarget{jesuxfas-sana-a-la-mujer-ensangrentada-y-despierta-a-la-hija-de-jairo}{%
\subsection{Jesús sana a la mujer ensangrentada y despierta a la hija de
Jairo}\label{jesuxfas-sana-a-la-mujer-ensangrentada-y-despierta-a-la-hija-de-jairo}}

\bibverse{40} Y aconteció que volviendo Jesús, recibióle la gente;
porque todos le esperaban. \bibverse{41} Y he aquí un varón, llamado
Jairo, y que era príncipe de la sinagoga, vino, y cayendo á los pies de
Jesús, le rogaba que entrase en su casa; \bibverse{42} Porque tenía una
hija única, como de doce años, y ella se estaba muriendo. Y yendo, le
apretaba la compañía. \bibverse{43} Y una mujer, que tenía flujo de
sangre hacía ya doce años, la cual había gastado en médicos toda su
hacienda, y por ninguno había podido ser curada, \bibverse{44}
Llegándose por las espaldas, tocó el borde de su vestido; y luego se
estancó el flujo de su sangre.

\bibverse{45} Entonces Jesús dijo: ¿Quién es el que me ha tocado? Y
negando todos, dijo Pedro y los que estaban con él: Maestro, la compañía
te aprieta y oprime, y dices: ¿Quién es el que me ha tocado?

\bibverse{46} Y Jesús dijo: Me ha tocado alguien; porque yo he conocido
que ha salido virtud de mí. \bibverse{47} Entonces, como la mujer vió
que no se había ocultado, vino temblando, y postrándose delante de él
declaróle delante de todo el pueblo la causa por qué le había tocado, y
cómo luego había sido sana. \bibverse{48} Y él le dijo: Hija, tu fe te
ha salvado: ve en paz. \footnote{\textbf{8:48} Luc 7,50}

\bibverse{49} Estando aún él hablando, vino uno del príncipe de la
sinagoga á decirle: Tu hija es muerta, no des trabajo al Maestro.

\bibverse{50} Y oyéndolo Jesús, le respondió: No temas: cree solamente,
y será salva.

\bibverse{51} Y entrado en casa, no dejó entrar á nadie consigo, sino á
Pedro, y á Jacobo, y á Juan, y al padre y á la madre de la moza.
\bibverse{52} Y lloraban todos, y la plañían. Y él dijo: No lloréis; no
es muerta, sino que duerme. \footnote{\textbf{8:52} Luc 7,13}

\bibverse{53} Y hacían burla de él, sabiendo que estaba muerta.
\bibverse{54} Mas él, tomándola de la mano, clamó, diciendo: Muchacha,
levántate. \bibverse{55} Entonces su espíritu volvió, y se levantó
luego: y él mandó que le diesen de comer. \bibverse{56} Y sus padres
estaban atónitos; á los cuales él mandó, que á nadie dijesen lo que
había sido hecho.

\hypertarget{enviar-a-los-doce-discuxedpulos-y-darles-instrucciones}{%
\subsection{Enviar a los doce discípulos y darles
instrucciones}\label{enviar-a-los-doce-discuxedpulos-y-darles-instrucciones}}

\hypertarget{section-8}{%
\section{9}\label{section-8}}

\bibverse{1} Y juntando á sus doce discípulos, les dió virtud y potestad
sobre todos los demonios, y que sanasen enfermedades. \footnote{\textbf{9:1}
  Luc 10,1-12} \bibverse{2} Y los envió á que predicasen el reino de
Dios, y que sanasen á los enfermos. \bibverse{3} Y les dice: No toméis
nada para el camino, ni báculo, ni alforja, ni pan, ni dinero; ni
tengáis dos vestidos cada uno. \bibverse{4} Y en cualquiera casa en que
entrareis, quedad allí, y de allí salid. \bibverse{5} Y todos los que no
os recibieren, saliéndoos de aquella ciudad, aun el polvo sacudid de
vuestros pies en testimonio contra ellos.

\bibverse{6} Y saliendo, rodeaban por todas las aldeas, anunciando el
evangelio, y sanando por todas partes.

\hypertarget{conclusiuxf3n-de-la-obra-de-jesuxfas-en-galilea}{%
\subsection{Conclusión de la obra de Jesús en
Galilea}\label{conclusiuxf3n-de-la-obra-de-jesuxfas-en-galilea}}

\bibverse{7} Y oyó Herodes el tetrarca todas las cosas que hacía; y
estaba en duda, porque decían algunos: Juan ha resucitado de los
muertos; \bibverse{8} Y otros: Elías ha aparecido; y otros: Algún
profeta de los antiguos ha resucitado. \bibverse{9} Y dijo Herodes: A
Juan yo degollé: ¿quién pues será éste, de quien yo oigo tales cosas? Y
procuraba verle.

\hypertarget{regreso-de-los-apuxf3stoles-jesuxfas-se-retira-a-la-soledad-alimentando-a-los-cinco-mil}{%
\subsection{Regreso de los apóstoles; Jesús se retira a la soledad;
Alimentando a los cinco
mil}\label{regreso-de-los-apuxf3stoles-jesuxfas-se-retira-a-la-soledad-alimentando-a-los-cinco-mil}}

\bibverse{10} Y vueltos los apóstoles, le contaron todas las cosas que
habían hecho. Y tomándolos, se retiró aparte á un lugar desierto de la
ciudad que se llama Bethsaida.

\bibverse{11} Y como lo entendieron las gentes, le siguieron; y él las
recibió, y les hablaba del reino de Dios, y sanaba á los que tenían
necesidad de cura. \bibverse{12} Y el día había comenzado á declinar; y
llegándose los doce, le dijeron: Despide á las gentes, para que yendo á
las aldeas y heredades de alrededor, procedan á alojarse y hallen
viandas; porque aquí estamos en lugar desierto.

\bibverse{13} Y les dice: Dadles vosotros de comer. Y dijeron ellos: No
tenemos más que cinco panes y dos pescados, si no vamos nosotros á
comprar viandas para toda esta compañía.

\bibverse{14} Y eran como cinco mil hombres. Entonces dijo á sus
discípulos: Hacedlos sentar en ranchos, de cincuenta en cincuenta.

\bibverse{15} Y así lo hicieron, haciéndolos sentar á todos.
\bibverse{16} Y tomando los cinco panes y los dos pescados, mirando al
cielo los bendijo, y partió, y dió á sus discípulos para que pusiesen
delante de las gentes. \bibverse{17} Y comieron todos, y se hartaron; y
alzaron lo que les sobró, doce cestos de pedazos.

\hypertarget{la-confesiuxf3n-de-pedro-del-mesuxedas-y-el-primer-anuncio-del-sufrimiento-de-jesuxfas}{%
\subsection{La confesión de Pedro del Mesías y el primer anuncio del
sufrimiento de
Jesús}\label{la-confesiuxf3n-de-pedro-del-mesuxedas-y-el-primer-anuncio-del-sufrimiento-de-jesuxfas}}

\bibverse{18} Y aconteció que estando él solo orando, estaban con él los
discípulos; y les preguntó diciendo: ¿Quién dicen las gentes que soy?

\bibverse{19} Y ellos respondieron, y dijeron: Juan el Bautista; y
otros, Elías; y otros, que algún profeta de los antiguos ha resucitado.

\bibverse{20} Y les dijo: ¿Y vosotros, quién decís que soy? Entonces
respondiendo Simón Pedro, dijo: El Cristo de Dios.

\bibverse{21} Mas él, conminándolos, mandó que á nadie dijesen esto;
\bibverse{22} Diciendo: Es necesario que el Hijo del hombre padezca
muchas cosas, y sea desechado de los ancianos, y de los príncipes de los
sacerdotes, y de los escribas, y que sea muerto, y resucite al tercer
día.

\hypertarget{proverbios-sobre-seguir-por-los-discuxedpulos}{%
\subsection{Proverbios sobre seguir por los
discípulos}\label{proverbios-sobre-seguir-por-los-discuxedpulos}}

\bibverse{23} Y decía á todos: Si alguno quiere venir en pos de mí,
niéguese á sí mismo, y tome su cruz cada día, y sígame. \bibverse{24}
Porque cualquiera que quisiere salvar su vida, la perderá; y cualquiera
que perdiere su vida por causa de mí, éste la salvará. \footnote{\textbf{9:24}
  Luc 17,33; Mat 10,39; Juan 12,25} \bibverse{25} Porque ¿qué aprovecha
al hombre, si granjeare todo el mundo, y se pierda él á sí mismo, ó
corra peligro de sí? \bibverse{26} Porque el que se avergonzare de mí y
de mis palabras, de este tal el Hijo del hombre se avergonzará cuando
viniere en su gloria, y del Padre, y de los santos ángeles.
\bibverse{27} Y os digo en verdad, que hay algunos de los que están
aquí, que no gustarán la muerte, hasta que vean el reino de Dios.

\hypertarget{la-transfiguraciuxf3n-de-jesuxfas-en-la-montauxf1a}{%
\subsection{La transfiguración de Jesús en la
montaña}\label{la-transfiguraciuxf3n-de-jesuxfas-en-la-montauxf1a}}

\bibverse{28} Y aconteció como ocho días después de estas palabras, que
tomó á Pedro y á Juan y á Jacobo, y subió al monte á orar. \bibverse{29}
Y entre tanto que oraba, la apariencia de su rostro se hizo otra, y su
vestido blanco y resplandeciente. \bibverse{30} Y he aquí dos varones
que hablaban con él, los cuales eran Moisés y Elías; \bibverse{31} Que
aparecieron en majestad, y hablaban de su salida, la cual había de
cumplir en Jerusalem.

\bibverse{32} Y Pedro y los que estaban con él, estaban cargados de
sueño: y como despertaron, vieron su majestad, y á aquellos dos varones
que estaban con él. \bibverse{33} Y aconteció, que apartándose ellos de
él, Pedro dice á Jesús: Maestro, bien es que nos quedemos aquí: y
hagamos tres pabellones, uno para ti, y uno para Moisés, y uno para
Elías; no sabiendo lo que se decía.

\bibverse{34} Y estando él hablando esto, vino una nube que los cubrió;
y tuvieron temor, entrando ellos en la nube. \bibverse{35} Y vino una
voz de la nube, que decía: Este es mi Hijo amado; á él oid. \footnote{\textbf{9:35}
  Luc 3,22; Deut 18,15; Deut 18,19; Sal 2,7} \bibverse{36} Y pasada
aquella voz, Jesús fué hallado solo: y ellos callaron; y por aquellos
días no dijeron nada á nadie de lo que habían visto.

\hypertarget{curaciuxf3n-de-un-niuxf1o-epiluxe9ptico}{%
\subsection{Curación de un niño
epiléptico}\label{curaciuxf3n-de-un-niuxf1o-epiluxe9ptico}}

\bibverse{37} Y aconteció al día siguiente, que apartándose ellos del
monte, gran compañía les salió al encuentro. \bibverse{38} Y he aquí, un
hombre de la compañía clamó, diciendo: Maestro, ruégote que veas á mi
hijo; que es el único que tengo: \bibverse{39} Y he aquí un espíritu le
toma, y de repente da voces; y le despedaza y hace echar espuma, y
apenas se aparta de él quebrantándole. \bibverse{40} Y rogué á tus
discípulos que le echasen fuera, y no pudieron.

\bibverse{41} Y respondiendo Jesús, dice: ¡Oh generación infiel y
perversa! ¿hasta cuándo tengo de estar con vosotros, y os sufriré? Trae
tu hijo acá.

\bibverse{42} Y como aun se acercaba, el demonio le derribó y despedazó:
mas Jesús increpó al espíritu inmundo, y sanó al muchacho, y se lo
volvió á su padre. \bibverse{43} Y todos estaban atónitos de la grandeza
de Dios. Y maravillándose todos de todas las cosas que hacía, dijo á sus
discípulos: \footnote{\textbf{9:43} Luc 18,31-34}

\hypertarget{segundo-anuncio-del-sufrimiento-de-jesuxfas}{%
\subsection{Segundo anuncio del sufrimiento de
Jesús}\label{segundo-anuncio-del-sufrimiento-de-jesuxfas}}

\bibverse{44} Poned vosotros en vuestros oídos estas palabras; porque ha
de acontecer que el Hijo del hombre será entregado en manos de hombres.
\bibverse{45} Mas ellos no entendían esta palabra, y les era encubierta
para que no la entendiesen; y temían preguntarle de esta palabra.

\hypertarget{la-arrogancia-de-los-discuxedpulos-enseuxf1anza-sobre-la-humildad-y-la-tolerancia}{%
\subsection{La arrogancia de los discípulos; Enseñanza sobre la humildad
y la
tolerancia}\label{la-arrogancia-de-los-discuxedpulos-enseuxf1anza-sobre-la-humildad-y-la-tolerancia}}

\bibverse{46} Entonces entraron en disputa, cuál de ellos sería el
mayor. \bibverse{47} Mas Jesús, viendo los pensamientos del corazón de
ellos, tomó un niño, y púsole junto á sí, \bibverse{48} Y les dice:
Cualquiera que recibiere este niño en mi nombre, á mí recibe; y
cualquiera que me recibiere á mí, recibe al que me envió; porque el que
fuere el menor entre todos vosotros, éste será el grande. \footnote{\textbf{9:48}
  Mat 10,40}

\bibverse{49} Entonces respondiendo Juan, dijo: Maestro, hemos visto á
uno que echaba fuera demonios en tu nombre; y se lo prohibimos, porque
no sigue con nosotros.

\bibverse{50} Jesús le dijo: No se lo prohibáis; porque el que no es
contra nosotros, por nosotros es.

\hypertarget{salida-para-el-viaje-el-inhuxf3spito-pueblo-samaritano}{%
\subsection{Salida para el viaje; el inhóspito pueblo
samaritano}\label{salida-para-el-viaje-el-inhuxf3spito-pueblo-samaritano}}

\bibverse{51} Y aconteció que, como se cumplió el tiempo en que había de
ser recibido arriba, él afirmó su rostro para ir á Jerusalem.
\footnote{\textbf{9:51} Mar 10,32} \bibverse{52} Y envió mensajeros
delante de sí, los cuales fueron y entraron en una ciudad de los
Samaritanos, para prevenirle. \bibverse{53} Mas no le recibieron, porque
era su traza de ir á Jerusalem. \bibverse{54} Y viendo esto sus
discípulos Jacobo y Juan, dijeron: Señor, ¿quieres que mandemos que
descienda fuego del cielo, y los consuma, como hizo Elías? \footnote{\textbf{9:54}
  2Re 1,10-12}

\bibverse{55} Entonces volviéndose él, los reprendió, diciendo: Vosotros
no sabéis de qué espíritu sois; \bibverse{56} Porque el Hijo del hombre
no ha venido para perder las almas de los hombres, sino para salvarlas.
Y se fueron á otra aldea.

\hypertarget{tres-seguidores-diferentes-de-jesuxfas-proverbios-sobre-seguir}{%
\subsection{Tres seguidores diferentes de Jesús; Proverbios sobre
seguir}\label{tres-seguidores-diferentes-de-jesuxfas-proverbios-sobre-seguir}}

\bibverse{57} Y aconteció que yendo ellos, uno le dijo en el camino:
Señor, te seguiré donde quiera que fueres.

\bibverse{58} Y le dijo Jesús: Las zorras tienen cuevas, y las aves de
los cielos nidos; mas el Hijo del hombre no tiene donde recline la
cabeza.

\bibverse{59} Y dijo á otro: Sígueme. Y él dijo: Señor, déjame que
primero vaya y entierre á mi padre.

\bibverse{60} Y Jesús le dijo: Deja los muertos que entierren á sus
muertos; y tú, ve, y anuncia el reino de Dios.

\bibverse{61} Entonces también dijo otro: Te seguiré, Señor; mas déjame
que me despida primero de los que están en mi casa. \footnote{\textbf{9:61}
  1Re 19,20}

\bibverse{62} Y Jesús le dijo: Ninguno que poniendo su mano al arado
mira atrás, es apto para el reino de Dios. \footnote{\textbf{9:62} Fil
  3,13}

\hypertarget{enviar-a-los-setenta-discuxedpulos-y-darles-instrucciones-arrepentimiento-sobre-las-ciudades-galileas-impenitentes}{%
\subsection{Enviar a los setenta discípulos y darles instrucciones;
Arrepentimiento sobre las ciudades galileas
impenitentes}\label{enviar-a-los-setenta-discuxedpulos-y-darles-instrucciones-arrepentimiento-sobre-las-ciudades-galileas-impenitentes}}

\hypertarget{section-9}{%
\section{10}\label{section-9}}

\bibverse{1} Y después de estas cosas, designó el Señor aun otros
setenta, los cuales envió de dos en dos delante de sí, á toda ciudad y
lugar á donde él había de venir. \footnote{\textbf{10:1} Mar 6,7}
\bibverse{2} Y les decía: La mies á la verdad es mucha, mas los obreros
pocos; por tanto, rogad al Señor de la mies que envíe obreros á su mies.
\footnote{\textbf{10:2} Juan 4,35; Mat 9,37-38} \bibverse{3} Andad, he
aquí yo os envío como corderos en medio de lobos. \bibverse{4} No
llevéis bolsa, ni alforja, ni calzado; y á nadie saludéis en el camino.
\footnote{\textbf{10:4} Luc 9,3-5; 2Re 4,29} \bibverse{5} En cualquiera
casa donde entrareis, primeramente decid: Paz sea á esta casa.
\footnote{\textbf{10:5} Juan 20,19} \bibverse{6} Y si hubiere allí algún
hijo de paz, vuestra paz reposará sobre él; y si no, se volverá á
vosotros. \bibverse{7} Y posad en aquella misma casa, comiendo y
bebiendo lo que os dieren; porque el obrero digno es de su salario. No
os paséis de casa en casa. \footnote{\textbf{10:7} Éxod 24,1}
\bibverse{8} Y en cualquier ciudad donde entrareis, y os recibieren,
comed lo que os pusieren delante; \bibverse{9} Y sanad los enfermos que
en ella hubiere, y decidles: Se ha llegado á vosotros el reino de Dios.
\bibverse{10} Mas en cualquier ciudad donde entrareis, y no os
recibieren, saliendo por sus calles, decid: \bibverse{11} Aun el polvo
que se nos ha pegado de vuestra ciudad á nuestros pies, sacudimos en
vosotros: esto empero sabed, que el reino de los cielos se ha llegado á
vosotros. \bibverse{12} Y os digo que los de Sodoma tendrán más remisión
aquel día, que aquella ciudad.

\bibverse{13} ¡Ay de ti, Corazín! ¡Ay de ti, Bethsaida! que si en Tiro y
en Sidón hubieran sido hechas las maravillas que se han hecho en
vosotras, ya días ha que, sentados en cilicio y ceniza, se habrían
arrepentido. \bibverse{14} Por tanto, Tiro y Sidón tendrán más remisión
que vosotras en el juicio. \bibverse{15} Y tú, Capernaum, que hasta los
cielos estás levantada, hasta los infiernos serás abajada. \bibverse{16}
El que á vosotros oye, á mí oye; y el que á vosotros desecha, á mí
desecha; y el que á mí desecha, desecha al que me envió.

\hypertarget{regreso-de-los-setenta-discuxedpulos-alegruxeda-de-jesuxfas-y-beatificaciuxf3n-de-los-discuxedpulos}{%
\subsection{Regreso de los setenta discípulos; Alegría de Jesús y
beatificación de los
discípulos}\label{regreso-de-los-setenta-discuxedpulos-alegruxeda-de-jesuxfas-y-beatificaciuxf3n-de-los-discuxedpulos}}

\bibverse{17} Y volvieron los setenta con gozo, diciendo: Señor, aun los
demonios se nos sujetan en tu nombre.

\bibverse{18} Y les dijo: Yo veía á Satanás, como un rayo, que caía del
cielo. \footnote{\textbf{10:18} Juan 12,31; Apoc 12,8-9} \bibverse{19}
He aquí os doy potestad de hollar sobre las serpientes y sobre los
escorpiones, y sobre toda fuerza del enemigo, y nada os dañará.
\footnote{\textbf{10:19} Mar 16,18; Sal 91,13} \bibverse{20} Mas no os
gocéis de esto, que los espíritus se os sujetan; antes gozaos de que
vuestros nombres están escritos en los cielos. \footnote{\textbf{10:20}
  Éxod 32,32; Is 4,3; Fil 4,3; Apoc 3,5}

\bibverse{21} En aquella misma hora Jesús se alegró en espíritu, y dijo:
Yo te alabo, oh Padre, Señor del cielo y de la tierra, que escondiste
estas cosas á los sabios y entendidos, y las has revelado á los
pequeños: así, Padre, porque así te agradó. \footnote{\textbf{10:21}
  1Cor 2,7}

\bibverse{22} Todas las cosas me son entregadas de mi Padre: y nadie
sabe quién sea el Hijo sino el Padre; ni quién sea el Padre, sino el
Hijo, y á quien el Hijo lo quisiere revelar.

\bibverse{23} Y vuelto particularmente á los discípulos, dijo:
Bienaventurados los ojos que ven lo que vosotros veis: \footnote{\textbf{10:23}
  Mat 13,16-17} \bibverse{24} Porque os digo que muchos profetas y reyes
desearon ver lo que vosotros veis, y no lo vieron; y oir lo que oís, y
no lo oyeron. \footnote{\textbf{10:24} 1Pe 1,10}

\hypertarget{jesuxfas-y-el-maestro-de-la-ley-seres-de-caridad-historia-del-buen-samaritano}{%
\subsection{Jesús y el maestro de la ley; Seres de caridad; Historia del
buen
samaritano}\label{jesuxfas-y-el-maestro-de-la-ley-seres-de-caridad-historia-del-buen-samaritano}}

\bibverse{25} Y he aquí, un doctor de la ley se levantó, tentándole y
diciendo: Maestro, ¿haciendo qué cosa poseeré la vida eterna?
\footnote{\textbf{10:25} Luc 18,18-20}

\bibverse{26} Y él le dijo: ¿Qué está escrito en la ley? ¿cómo lees?

\bibverse{27} Y él respondiendo, dijo: Amarás al Señor tu Dios de todo
tu corazón, y de toda tu alma, y de todas tus fuerzas, y de todo tu
entendimiento; y á tu prójimo como á ti mismo.

\bibverse{28} Y díjole: Bien has respondido: haz esto, y vivirás.

\bibverse{29} Mas él, queriéndose justificar á sí mismo, dijo á Jesús:
¿Y quién es mi prójimo?

\bibverse{30} Y respondiendo Jesús, dijo: Un hombre descendía de
Jerusalem á Jericó, y cayó en manos de ladrones, los cuales le
despojaron; é hiriéndole, se fueron, dejándole medio muerto.
\bibverse{31} Y aconteció, que descendió un sacerdote por aquel camino,
y viéndole, se pasó de un lado. \bibverse{32} Y asimismo un Levita,
llegando cerca de aquel lugar, y viéndole, se pasó de un lado.
\bibverse{33} Mas un Samaritano que transitaba, viniendo cerca de él, y
viéndole, fué movido á misericordia; \bibverse{34} Y llegándose, vendó
sus heridas, echándoles aceite y vino; y poniéndole sobre su
cabalgadura, llevóle al mesón, y cuidó de él. \bibverse{35} Y otro día
al partir, sacó dos denarios, y diólos al huésped, y le dijo: Cuídamele;
y todo lo que de más gastares, yo cuando vuelva te lo pagaré.
\bibverse{36} ¿Quién, pues, de estos tres te parece que fué el prójimo
de aquél que cayó en manos de los ladrones?

\bibverse{37} Y él dijo: El que usó con él de misericordia. Entonces
Jesús le dijo: Ve, y haz tú lo mismo. \footnote{\textbf{10:37} Juan
  13,17}

\hypertarget{marta-y-maruxeda-en-betania}{%
\subsection{Marta y María en
Betania}\label{marta-y-maruxeda-en-betania}}

\bibverse{38} Y aconteció que yendo, entró él en una aldea: y una mujer
llamada Marta, le recibió en su casa. \footnote{\textbf{10:38} Juan
  11,1; Juan 12,2-3} \bibverse{39} Y ésta tenía una hermana que se
llamaba María, la cual sentándose á los pies de Jesús, oía su palabra.
\bibverse{40} Empero Marta se distraía en muchos servicios; y
sobreviniendo, dice: Señor, ¿no tienes cuidado que mi hermana me deja
servir sola? Dile pues, que me ayude.

\bibverse{41} Pero respondiendo Jesús, le dijo: Marta, Marta, cuidadosa
estás, y con las muchas cosas estás turbada: \bibverse{42} Empero una
cosa es necesaria; y María escogió la buena parte, la cual no le será
quitada.

\hypertarget{jesuxfas-enseuxf1a-a-los-discuxedpulos-a-orar-el-padre-nuestro}{%
\subsection{Jesús enseña a los discípulos a orar: el Padre
Nuestro}\label{jesuxfas-enseuxf1a-a-los-discuxedpulos-a-orar-el-padre-nuestro}}

\hypertarget{section-10}{%
\section{11}\label{section-10}}

\bibverse{1} Y aconteció que estando él orando en un lugar, como acabó,
uno de sus discípulos le dijo: Señor, enséñanos á orar, como también
Juan enseñó á sus discípulos.

\bibverse{2} Y les dijo: Cuando orareis, decid: Padre nuestro que estás
en los cielos; sea tu nombre santificado. Venga tu reino. Sea hecha tu
voluntad, como en el cielo, así también en la tierra. \bibverse{3} El
pan nuestro de cada día, dánoslo hoy. \bibverse{4} Y perdónanos nuestros
pecados, porque también nosotros perdonamos á todos los que nos deben. Y
no nos metas en tentación, mas líbranos del malo.

\hypertarget{la-paruxe1bola-del-amigo-suplicante-jesuxfas-anima-a-la-oraciuxf3n-persistente-y-persistente}{%
\subsection{La parábola del amigo suplicante; Jesús anima a la oración
persistente y
persistente}\label{la-paruxe1bola-del-amigo-suplicante-jesuxfas-anima-a-la-oraciuxf3n-persistente-y-persistente}}

\bibverse{5} Díjoles también: ¿Quién de vosotros tendrá un amigo, é irá
á él á media noche, y le dirá: Amigo, préstame tres panes, \bibverse{6}
Porque un amigo mío ha venido á mí de camino, y no tengo qué ponerle
delante; \bibverse{7} Y el de dentro respondiendo, dijere: No me seas
molesto; la puerta está ya cerrada, y mis niños están conmigo en cama;
no puedo levantarme, y darte? \bibverse{8} Os digo, que aunque no se
levante á darle por ser su amigo, cierto por su importunidad se
levantará, y le dará todo lo que habrá menester.

\bibverse{9} Y yo os digo: Pedid, y se os dará; buscad, y hallaréis;
llamad, y os será abierto. \bibverse{10} Porque todo aquel que pide,
recibe; y el que busca, halla; y al que llama, se abre. \footnote{\textbf{11:10}
  Luc 13,25}

\bibverse{11} ¿Y cuál padre de vosotros, si su hijo le pidiere pan, le
dará una piedra? ó, si pescado, ¿en lugar de pescado, le dará una
serpiente? \bibverse{12} O, si le pidiere un huevo, ¿le dará un
escorpión? \bibverse{13} Pues si vosotros, siendo malos, sabéis dar
buenas dádivas á vuestros hijos, ¿cuánto más vuestro Padre celestial
dará el Espíritu Santo á los que lo pidieren de él?

\hypertarget{jesuxfas-se-defiende-de-la-blasfemia-de-los-fariseos-contra-beelzebul-paruxe1bola-de-recauxedda}{%
\subsection{Jesús se defiende de la blasfemia de los fariseos contra
Beelzebul; Parábola de
recaída}\label{jesuxfas-se-defiende-de-la-blasfemia-de-los-fariseos-contra-beelzebul-paruxe1bola-de-recauxedda}}

\bibverse{14} Y estaba él lanzando un demonio, el cual era mudo: y
aconteció que salido fuera el demonio, el mudo habló, y las gentes se
maravillaron. \bibverse{15} Mas algunos de ellos decían: En Beelzebub,
príncipe de los demonios, echa fuera los demonios. \bibverse{16} Y
otros, tentando, pedían de él señal del cielo. \bibverse{17} Mas él,
conociendo los pensamientos de ellos, les dijo: Todo reino dividido
contra sí mismo, es asolado; y una casa dividida contra sí misma, cae.
\bibverse{18} Y si también Satanás está dividido contra sí mismo, ¿cómo
estará en pie su reino? porque decís que en Beelzebub echo yo fuera los
demonios. \bibverse{19} Pues si yo echo fuera los demonios en Beelzebub,
¿vuestros hijos en quién los echan fuera? Por tanto, ellos serán
vuestros jueces. \bibverse{20} Mas si por el dedo de Dios echo yo fuera
los demonios, cierto el reino de Dios ha llegado á vosotros. \footnote{\textbf{11:20}
  Éxod 8,15}

\bibverse{21} Cuando el fuerte armado guarda su atrio, en paz está lo
que posee. \bibverse{22} Mas si sobreviniendo otro más fuerte que él, le
venciere, le toma todas sus armas en que confiaba, y reparte sus
despojos.

\bibverse{23} El que no es conmigo, contra mí es; y el que conmigo no
recoge, desparrama. \footnote{\textbf{11:23} Luc 9,50}

\bibverse{24} Cuando el espíritu inmundo saliere del hombre, anda por
lugares secos, buscando reposo; y no hallándolo, dice: Me volveré á mi
casa de donde salí. \bibverse{25} Y viniendo, la halla barrida y
adornada. \bibverse{26} Entonces va, y toma otros siete espíritus peores
que él; y entrados, habitan allí: y lo postrero del tal hombre es peor
que lo primero.

\hypertarget{beatificaciuxf3n-de-jesuxfas-de-los-verdaderos-hijos-de-dios}{%
\subsection{Beatificación de Jesús de los verdaderos hijos de
Dios}\label{beatificaciuxf3n-de-jesuxfas-de-los-verdaderos-hijos-de-dios}}

\bibverse{27} Y aconteció que diciendo estas cosas, una mujer de la
compañía, levantando la voz, le dijo: Bienaventurado el vientre que te
trajo, y los pechos que mamaste. \footnote{\textbf{11:27} Luc 1,28; Luc
  1,48}

\bibverse{28} Y él dijo: Antes bienaventurados los que oyen la palabra
de Dios, y la guardan. \footnote{\textbf{11:28} Luc 8,15; Luc 8,21}

\hypertarget{el-discurso-de-jesuxfas-contra-los-que-peduxedan-seuxf1ales-el-signo-de-jonuxe1s}{%
\subsection{El discurso de Jesús contra los que pedían señales; el signo
de
Jonás}\label{el-discurso-de-jesuxfas-contra-los-que-peduxedan-seuxf1ales-el-signo-de-jonuxe1s}}

\bibverse{29} Y juntándose las gentes á él, comenzó á decir: Esta
generación mala es: señal busca, mas señal no le será dada, sino la
señal de Jonás. \bibverse{30} Porque como Jonás fué señal á los
Ninivitas, así también será el Hijo del hombre á esta generación.
\bibverse{31} La reina del Austro se levantará en juicio con los hombres
de esta generación, y los condenará; porque vino de los fines de la
tierra á oir la sabiduría de Salomón; y he aquí más que Salomón en este
lugar. \footnote{\textbf{11:31} 1Re 10,1} \bibverse{32} Los hombres de
Nínive se levantarán en juicio con esta generación, y la condenarán;
porque á la predicación de Jonás se arrepintieron; y he aquí más que
Jonás en este lugar. \footnote{\textbf{11:32} Jon 3,5}

\bibverse{33} Nadie pone en oculto la antorcha encendida, ni debajo del
almud, sino en el candelero, para que los que entran vean la luz.
\footnote{\textbf{11:33} Luc 8,16} \bibverse{34} La antorcha del cuerpo
es el ojo: pues si tu ojo fuere simple, también todo tu cuerpo será
resplandeciente; mas si fuere malo, también tu cuerpo será tenebroso.
\bibverse{35} Mira pues, si la lumbre que en ti hay, es tinieblas.
\bibverse{36} Así que, siendo todo tu cuerpo resplandeciente, no
teniendo alguna parte de tinieblas, será todo luminoso, como cuando una
antorcha de resplandor te alumbra.

\hypertarget{contra-la-hipocresuxeda-de-los-fariseos}{%
\subsection{Contra la hipocresía de los
fariseos}\label{contra-la-hipocresuxeda-de-los-fariseos}}

\bibverse{37} Y luego que hubo hablado, rogóle un Fariseo que comiese
con él: y entrado Jesús, se sentó á la mesa. \bibverse{38} Y el Fariseo,
como lo vió, maravillóse de que no se lavó antes de comer. \footnote{\textbf{11:38}
  Mat 15,2} \bibverse{39} Y el Señor le dijo: Ahora vosotros los
Fariseos lo de fuera del vaso y del plato limpiáis; mas lo interior de
vosotros está lleno de rapiña y de maldad. \bibverse{40} Necios, ¿el que
hizo lo de fuera, no hizo también lo de dentro? \bibverse{41} Empero de
lo que os resta, dad limosna; y he aquí todo os será limpio.
\bibverse{42} Mas ¡ay de vosotros, Fariseos! que diezmáis la menta, y la
ruda, y toda hortaliza; mas el juicio y la caridad de Dios pasáis de
largo. Pues estas cosas era necesario hacer, y no dejar las otras.
\bibverse{43} ¡Ay de vosotros, Fariseos! que amáis las primeras sillas
en las sinagogas, y las salutaciones en las plazas. \footnote{\textbf{11:43}
  Luc 14,7} \bibverse{44} ¡Ay de vosotros, escribas y Fariseos,
hipócritas! que sois como sepulcros que no se ven, y los hombres que
andan encima no lo saben.

\hypertarget{gritos-de-defensa-sobre-la-malicia-de-los-maestros-de-la-ley}{%
\subsection{Gritos de defensa sobre la malicia de los maestros de la
ley}\label{gritos-de-defensa-sobre-la-malicia-de-los-maestros-de-la-ley}}

\bibverse{45} Y respondiendo uno de los doctores de la ley, le dice:
Maestro, cuando dices esto, también nos afrentas á nosotros.

\bibverse{46} Y él dijo: ¡Ay de vosotros también, doctores de la ley!
que cargáis á los hombres con cargas que no pueden llevar; mas vosotros
ni aun con un dedo tocáis las cargas. \bibverse{47} ¡Ay de vosotros! que
edificáis los sepulcros de los profetas, y los mataron vuestros padres.
\bibverse{48} De cierto dais testimonio que consentís en los hechos de
vuestros padres; porque á la verdad ellos los mataron, mas vosotros
edificáis sus sepulcros. \bibverse{49} Por tanto, la sabiduría de Dios
también dijo: Enviaré á ellos profetas y apóstoles; y de ellos á unos
matarán y á otros perseguirán; \bibverse{50} Para que de esta generación
sea demandada la sangre de todos los profetas, que ha sido derramada
desde la fundación del mundo; \bibverse{51} Desde la sangre de Abel,
hasta la sangre de Zacarías, que murió entre el altar y el templo: así
os digo, será demandada de esta generación. \bibverse{52} ¡Ay de
vosotros, doctores de la ley! que habéis quitado la llave de la ciencia;
vosotros mismos no entrasteis, y á los que entraban impedisteis.

\bibverse{53} Y diciéndoles estas cosas, los escribas y los Fariseos
comenzaron á apretarle en gran manera, y á provocarle á que hablase de
muchas cosas; \bibverse{54} Acechándole, y procurando cazar algo de su
boca para acusarle. \footnote{\textbf{11:54} Luc 20,20}

\hypertarget{advertencia-de-hipocresuxeda-farisea}{%
\subsection{Advertencia de hipocresía
farisea}\label{advertencia-de-hipocresuxeda-farisea}}

\hypertarget{section-11}{%
\section{12}\label{section-11}}

\bibverse{1} En esto, juntándose muchas gentes, tanto que unos á otros
se hollaban, comenzó á decir á sus discípulos, primeramente: Guardaos de
la levadura de los Fariseos, que es hipocresía. \footnote{\textbf{12:1}
  Mat 16,6; Mar 8,15} \bibverse{2} Porque nada hay encubierto, que no
haya de ser descubierto; ni oculto, que no haya de ser sabido.
\footnote{\textbf{12:2} Luc 8,17} \bibverse{3} Por tanto, las cosas que
dijisteis en tinieblas, á la luz serán oídas; y lo que hablasteis al
oído en las cámaras, será pregonado en los terrados.

\hypertarget{advertencia-de-miedo-al-hombre}{%
\subsection{Advertencia de miedo al
hombre}\label{advertencia-de-miedo-al-hombre}}

\bibverse{4} Mas os digo, amigos míos: No temáis de los que matan el
cuerpo, y después no tienen más que hacer. \bibverse{5} Mas os enseñaré
á quién temáis: temed á aquel que después de haber quitado la vida,
tiene poder de echar en la Gehenna: así os digo: á éste temed.

\bibverse{6} ¿No se venden cinco pajarillos por dos blancas? pues ni uno
de ellos está olvidado delante de Dios. \bibverse{7} Y aun los cabellos
de vuestra cabeza están todos contados. No temáis pues: de más estima
sois que muchos pajarillos. \footnote{\textbf{12:7} Luc 21,18}

\bibverse{8} Y os digo que todo aquel que me confesare delante de los
hombres, también el Hijo del hombre le confesará delante de los ángeles
de Dios; \bibverse{9} Mas el que me negare delante de los hombres, será
negado delante de los ángeles de Dios.

\hypertarget{advertencia-de-blasfemia-contra-el-espuxedritu-santo-referencia-a-la-asistencia-del-espuxedritu}{%
\subsection{Advertencia de blasfemia contra el Espíritu Santo;
Referencia a la asistencia del
Espíritu}\label{advertencia-de-blasfemia-contra-el-espuxedritu-santo-referencia-a-la-asistencia-del-espuxedritu}}

\bibverse{10} Y todo aquel que dice palabra contra el Hijo del hombre,
le será perdonado; mas al que blasfemare contra el Espíritu Santo, no le
será perdonado. \footnote{\textbf{12:10} Mat 12,32; Mar 3,28-29}
\bibverse{11} Y cuando os trajeren á las sinagogas, y á los magistrados
y potestades, no estéis solícitos cómo ó qué hayáis de responder, ó qué
hayáis de decir; \footnote{\textbf{12:11} Luc 21,14-15; Mat 10,19-20}
\bibverse{12} Porque el Espíritu Santo os enseñará en la misma hora lo
que será necesario decir.

\bibverse{13} Y díjole uno de la compañía: Maestro, di á mi hermano que
parta conmigo la herencia.

\bibverse{14} Mas él le dijo: Hombre, ¿quién me puso por juez ó partidor
sobre vosotros?

\hypertarget{advertencia-de-codicia-paruxe1bola-del-rico-tonto}{%
\subsection{Advertencia de codicia; Parábola del rico
tonto}\label{advertencia-de-codicia-paruxe1bola-del-rico-tonto}}

\bibverse{15} Y díjoles: Mirad, y guardaos de toda avaricia; porque la
vida del hombre no consiste en la abundancia de los bienes que posee.
\footnote{\textbf{12:15} 1Tim 6,9-10}

\bibverse{16} Y refirióles una parábola, diciendo: La heredad de un
hombre rico había llevado mucho; \bibverse{17} Y él pensaba dentro de
sí, diciendo: ¿Qué haré, porque no tengo donde juntar mis frutos?
\bibverse{18} Y dijo: Esto haré: derribaré mis alfolíes, y los edificaré
mayores, y allí juntaré todos mis frutos y mis bienes; \bibverse{19} Y
diré á mi alma: Alma, muchos bienes tienes almacenados para muchos años;
repósate, come, bebe, huélgate.

\bibverse{20} Y díjole Dios: Necio, esta noche vuelven á pedir tu alma;
y lo que has prevenido, ¿de quién será? \footnote{\textbf{12:20} Heb
  9,27} \bibverse{21} Así es el que hace para sí tesoro, y no es rico en
Dios. \footnote{\textbf{12:21} Mat 6,20}

\hypertarget{procurad-el-reino-de-dios-y-todas-estas-cosas-os-seruxe1n-auxf1adidas}{%
\subsection{Procurad el reino de Dios, y todas estas cosas os serán
añadidas}\label{procurad-el-reino-de-dios-y-todas-estas-cosas-os-seruxe1n-auxf1adidas}}

\bibverse{22} Y dijo á sus discípulos: Por tanto os digo: No estéis
afanosos de vuestra vida, qué comeréis; ni del cuerpo, qué vestiréis.
\bibverse{23} La vida más es que la comida, y el cuerpo que el vestido.
\bibverse{24} Considerad los cuervos, que ni siembran, ni siegan; que ni
tienen cillero, ni alfolí; y Dios los alimenta. ¿Cuánto de más estima
sois vosotros que las aves? \bibverse{25} ¿Y quién de vosotros podrá con
afán añadir á su estatura un codo? \bibverse{26} Pues si no podéis aun
lo que es menos, ¿para qué estaréis afanosos de lo demás? \bibverse{27}
Considerad los lirios, cómo crecen: no labran, ni hilan; y os digo, que
ni Salomón con toda su gloria se vistió como uno de ellos. \bibverse{28}
Y si así viste Dios á la hierba, que hoy está en el campo, y mañana es
echada en el horno; ¿cuánto más á vosotros, hombres de poca fe?

\bibverse{29} Vosotros, pues, no procuréis qué hayáis de comer, ó qué
hayáis de beber; ni estéis en ansiosa perplejidad. \bibverse{30} Porque
todas estas cosas buscan las gentes del mundo; que vuestro Padre sabe
que necesitáis estas cosas. \bibverse{31} Mas procurad el reino de Dios,
y todas estas cosas os serán añadidas.

\bibverse{32} No temáis, manada pequeña; porque al Padre ha placido
daros el reino. \footnote{\textbf{12:32} Luc 22,29; Is 41,14}
\bibverse{33} Vended lo que poseéis, y dad limosna; haceos bolsas que no
se envejecen, tesoro en los cielos que nunca falta; donde ladrón no
llega, ni polilla corrompe. \footnote{\textbf{12:33} Luc 18,22}
\bibverse{34} Porque donde está vuestro tesoro, allí también estará
vuestro corazón.

\hypertarget{recordatorios-de-vigilancia-y-preparaciuxf3n-igualdad-de-disposiciuxf3n-de-los-siervos-fieles-y-del-maligno}{%
\subsection{Recordatorios de vigilancia y preparación; Igualdad de
disposición de los siervos fieles y del
maligno}\label{recordatorios-de-vigilancia-y-preparaciuxf3n-igualdad-de-disposiciuxf3n-de-los-siervos-fieles-y-del-maligno}}

\bibverse{35} Estén ceñidos vuestros lomos, y vuestras antorchas
encendidas; \footnote{\textbf{12:35} Éxod 12,11; 1Pe 1,13; Mat 25,1-13}
\bibverse{36} Y vosotros semejantes á hombres que esperan cuando su
señor ha de volver de las bodas; para que cuando viniere y llamare,
luego le abran. \footnote{\textbf{12:36} Apoc 3,20} \bibverse{37}
Bienaventurados aquellos siervos, á los cuales cuando el Señor viniere,
hallare velando: de cierto os digo, que se ceñirá, y hará que se sienten
á la mesa, y pasando les servirá. \bibverse{38} Y aunque venga á la
segunda vigilia, y aunque venga á la tercera vigilia, y los hallare así,
bienaventurados son los tales siervos. \bibverse{39} Esto empero sabed,
que si supiese el padre de familia á qué hora había de venir el ladrón,
velaría ciertamente, y no dejaría minar su casa. \footnote{\textbf{12:39}
  1Tes 5,2} \bibverse{40} Vosotros pues también, estad apercibidos;
porque á la hora que no pensáis, el Hijo del hombre vendrá.

\bibverse{41} Entonces Pedro le dijo: Señor, ¿dices esta parábola á
nosotros, ó también á todos?

\bibverse{42} Y dijo el Señor: ¿Quién es el mayordomo fiel y prudente,
al cual el señor pondrá sobre su familia, para que á tiempo les dé su
ración? \bibverse{43} Bienaventurado aquel siervo, al cual, cuando el
señor viniere, hallare haciendo así. \bibverse{44} En verdad os digo,
que él le pondrá sobre todos sus bienes. \bibverse{45} Mas si el tal
siervo dijere en su corazón: Mi señor tarda en venir: y comenzare á
herir á los siervos y á las criadas, y á comer y á beber y á
embriagarse; \bibverse{46} Vendrá el señor de aquel siervo el día que no
espera, y á la hora que no sabe, y le apartará, y pondrá su parte con
los infieles. \bibverse{47} Porque el siervo que entendió la voluntad de
su señor, y no se apercibió, ni hizo conforme á su voluntad, será
azotado mucho. \bibverse{48} Mas el que no entendió, é hizo cosas dignas
de azotes, será azotado poco: porque á cualquiera que fué dado mucho,
mucho será vuelto á demandar de él; y al que encomendaron mucho, más le
será pedido.

\hypertarget{jesuxfas-trae-fuego-y-conflicto-a-la-tierra}{%
\subsection{Jesús trae fuego y conflicto a la
tierra}\label{jesuxfas-trae-fuego-y-conflicto-a-la-tierra}}

\bibverse{49} Fuego vine á meter en la tierra: ¿y qué quiero, si ya está
encendido? \bibverse{50} Empero de bautismo me es necesario ser
bautizado: y ¡cómo me angustio hasta que sea cumplido! \footnote{\textbf{12:50}
  Luc 18,31; Mat 20,22; Mat 26,38} \bibverse{51} ¿Pensáis que he venido
á la tierra á dar paz? No, os digo; mas disensión. \bibverse{52} Porque
estarán de aquí adelante cinco en una casa divididos; tres contra dos, y
dos contra tres. \bibverse{53} El padre estará dividido contra el hijo,
y el hijo contra el padre; la madre contra la hija, y la hija contra la
madre; la suegra contra su nuera, y la nuera contra su suegra.

\hypertarget{los-signos-de-los-tiempos-te-instan-a-tomar-en-serio-la-salvaciuxf3n-de-tu-alma}{%
\subsection{Los signos de los tiempos te instan a tomar en serio la
salvación de tu
alma}\label{los-signos-de-los-tiempos-te-instan-a-tomar-en-serio-la-salvaciuxf3n-de-tu-alma}}

\bibverse{54} Y decía también á las gentes: Cuando veis la nube que sale
del poniente, luego decís: Agua viene; y es así. \bibverse{55} Y cuando
sopla el austro, decís: Habrá calor; y lo hay. \bibverse{56}
¡Hipócritas! Sabéis examinar la faz del cielo y de la tierra; ¿y cómo no
reconocéis este tiempo?

\bibverse{57} ¿Y por qué aun de vosotros mismos no juzgáis lo que es
justo? \bibverse{58} Pues cuando vas al magistrado con tu adversario,
procura en el camino librarte de él; porque no te arrastre al juez, y el
juez te entregue al alguacil, y el alguacil te meta en la cárcel.
\footnote{\textbf{12:58} Mat 5,25-26}

\bibverse{59} Te digo que no saldrás de allá, hasta que hayas pagado
hasta el último maravedí.

\hypertarget{recordatorios-de-penitencia}{%
\subsection{Recordatorios de
penitencia}\label{recordatorios-de-penitencia}}

\hypertarget{section-12}{%
\section{13}\label{section-12}}

\bibverse{1} Y en este mismo tiempo estaban allí unos que le contaban
acerca de los Galileos, cuya sangre Pilato había mezclado con sus
sacrificios. \bibverse{2} Y respondiendo Jesús, les dijo: ¿Pensáis que
estos Galileos, porque han padecido tales cosas, hayan sido más
pecadores que todos los Galileos? \bibverse{3} No, os digo; antes si no
os arrepintiereis, todos pereceréis igualmente. \bibverse{4} O aquellos
dieciocho, sobre los cuales cayó la torre en Siloé, y los mató, ¿pensáis
que ellos fueron más deudores que todos los hombres que habitan en
Jerusalem? \bibverse{5} No, os digo; antes si no os arrepintiereis,
todos pereceréis asimismo.

\bibverse{6} Y dijo esta parábola: Tenía uno una higuera plantada en su
viña, y vino á buscar fruto en ella, y no lo halló. \footnote{\textbf{13:6}
  Mat 21,19} \bibverse{7} Y dijo al viñero: He aquí tres años ha que
vengo á buscar fruto en esta higuera, y no lo hallo; córtala, ¿por qué
ocupará aún la tierra? \bibverse{8} El entonces respondiendo, le dijo:
Señor, déjala aún este año, hasta que la excave, y estercole.
\bibverse{9} Y si hiciere fruto, bien; y si no, la cortarás después.
\footnote{\textbf{13:9} Luc 3,9}

\hypertarget{una-curaciuxf3n-de-los-enfermos-en-suxe1bado-disputa-por-guardar-el-duxeda-de-reposo}{%
\subsection{Una curación de los enfermos en sábado; Disputa por guardar
el día de
reposo}\label{una-curaciuxf3n-de-los-enfermos-en-suxe1bado-disputa-por-guardar-el-duxeda-de-reposo}}

\bibverse{10} Y enseñaba en una sinagoga en sábado. \footnote{\textbf{13:10}
  Luc 6,6-11} \bibverse{11} Y he aquí una mujer que tenía espíritu de
enfermedad dieciocho años, y andaba agobiada, que en ninguna manera se
podía enhestar. \bibverse{12} Y como Jesús la vió, llamóla, y díjole:
Mujer, libre eres de tu enfermedad. \bibverse{13} Y puso las manos sobre
ella; y luego se enderezó, y glorificaba á Dios.

\bibverse{14} Y respondiendo el príncipe de la sinagoga, enojado de que
Jesús hubiese curado en sábado, dijo á la compañía: Seis días hay en que
es necesario obrar: en estos, pues, venid y sed curados, y no en día de
sábado. \footnote{\textbf{13:14} Éxod 20,9-10}

\bibverse{15} Entonces el Señor le respondió, y dijo: Hipócrita, cada
uno de vosotros ¿no desata en sábado su buey ó su asno del pesebre, y lo
lleva á beber? \footnote{\textbf{13:15} Luc 14,5} \bibverse{16} Y á esta
hija de Abraham, que he aquí Satanás la había ligado dieciocho años, ¿no
convino desatarla de esta ligadura en día de sábado? \footnote{\textbf{13:16}
  Luc 19,9}

\bibverse{17} Y diciendo estas cosas, se avergonzaban todos sus
adversarios: mas todo el pueblo se gozaba de todas las cosas gloriosas
que eran por él hechas.

\hypertarget{paruxe1bolas-del-grano-de-mostaza-y-levadura}{%
\subsection{Parábolas del grano de mostaza y
levadura}\label{paruxe1bolas-del-grano-de-mostaza-y-levadura}}

\bibverse{18} Y dijo: ¿A qué es semejante el reino de Dios, y á qué le
compararé? \bibverse{19} Semejante es al grano de la mostaza, que
tomándolo un hombre lo metió en su huerto; y creció, y fué hecho árbol
grande, y las aves del cielo hicieron nidos en sus ramas.

\bibverse{20} Y otra vez dijo: ¿A qué compararé el reino de Dios?
\bibverse{21} Semejante es á la levadura, que tomó una mujer, y la
escondió en tres medidas de harina, hasta que todo hubo fermentado.

\bibverse{22} Y pasaba por todas las ciudades y aldeas, enseñando, y
caminando á Jerusalem. \bibverse{23} Y díjole uno: Señor, ¿son pocos los
que se salvan? Y él les dijo:

\hypertarget{porfiad-uxe1-entrar-por-la-puerta-angosta}{%
\subsection{Porfiad á entrar por la puerta
angosta}\label{porfiad-uxe1-entrar-por-la-puerta-angosta}}

\bibverse{24} Porfiad á entrar por la puerta angosta; porque os digo que
muchos procurarán entrar, y no podrán. \bibverse{25} Después que el
padre de familia se levantare, y cerrare la puerta, y comenzareis á
estar fuera, y llamar á la puerta, diciendo: Señor, Señor, ábrenos; y
respondiendo os dirá: No os conozco de dónde seáis. \footnote{\textbf{13:25}
  Mat 25,11-12} \bibverse{26} Entonces comenzaréis á decir: Delante de
ti hemos comido y bebido, y en nuestras plazas enseñaste; \footnote{\textbf{13:26}
  Mat 7,22-23} \bibverse{27} Y os dirá: Dígoos que no os conozco de
dónde seáis; apartaos de mí todos los obreros de iniquidad.
\bibverse{28} Allí será el llanto y el crujir de dientes, cuando viereis
á Abraham, y á Isaac, y á Jacob, y á todos los profetas en el reino de
Dios, y vosotros excluídos. \footnote{\textbf{13:28} Mat 8,11-12}
\bibverse{29} Y vendrán del Oriente y del Occidente, del Norte y del
Mediodía, y se sentarán á la mesa en el reino de Dios. \footnote{\textbf{13:29}
  Luc 14,15} \bibverse{30} Y he aquí, son postreros los que eran los
primeros; y son primeros los que eran los postreros. \footnote{\textbf{13:30}
  Mat 19,30}

\hypertarget{ay-de-jerusaluxe9n-que-apedrea-a-los-profetas}{%
\subsection{Ay de Jerusalén, que apedrea a los
profetas}\label{ay-de-jerusaluxe9n-que-apedrea-a-los-profetas}}

\bibverse{31} Aquel mismo día llegaron unos de los Fariseos, diciéndole:
Sal, y vete de aquí, porque Herodes te quiere matar.

\bibverse{32} Y les dijo: Id, y decid á aquella zorra: He aquí, echo
fuera demonios y acabo sanidades hoy y mañana, y al tercer día soy
consumado. \bibverse{33} Empero es menester que hoy, y mañana, y pasado
mañana camine; porque no es posible que profeta muera fuera de
Jerusalem.

\bibverse{34} ¡Jerusalem, Jerusalem! que matas á los profetas, y
apedreas á los que son enviados á ti: ¡cuántas veces quise juntar tus
hijos, como la gallina sus pollos debajo de sus alas, y no quisiste!
\bibverse{35} He aquí, os es dejada vuestra casa desierta: y os digo que
no me veréis, hasta que venga tiempo cuando digáis: Bendito el que viene
en nombre del Señor. \footnote{\textbf{13:35} Sal 118,26}

\hypertarget{nueva-disputa-sobre-la-curaciuxf3n-de-los-enfermos-en-suxe1bado}{%
\subsection{Nueva disputa sobre la curación de los enfermos en
sábado}\label{nueva-disputa-sobre-la-curaciuxf3n-de-los-enfermos-en-suxe1bado}}

\hypertarget{section-13}{%
\section{14}\label{section-13}}

\bibverse{1} Y aconteció que entrando en casa de un príncipe de los
Fariseos un sábado á comer pan, ellos le acechaban. \footnote{\textbf{14:1}
  Luc 6,6-11; Luc 11,37} \bibverse{2} Y he aquí un hombre hidrópico
estaba delante de él. \bibverse{3} Y respondiendo Jesús, habló á los
doctores de la ley y á los Fariseos, diciendo: ¿Es lícito sanar en
sábado?

\bibverse{4} Y ellos callaron. Entonces él tomándole, le sanó, y
despidióle.

\bibverse{5} Y respondiendo á ellos dijo: ¿El asno ó el buey de cuál de
vosotros caerá en algún pozo, y no lo sacará luego en día de sábado?
\footnote{\textbf{14:5} Luc 13,5; Mat 12,11}

\bibverse{6} Y no le podían replicar á estas cosas.

\hypertarget{cualquiera-que-se-ensalza-seruxe1-humillado-y-el-que-se-humilla-seruxe1-ensalzado}{%
\subsection{Cualquiera que se ensalza, será humillado; y el que se
humilla, será
ensalzado}\label{cualquiera-que-se-ensalza-seruxe1-humillado-y-el-que-se-humilla-seruxe1-ensalzado}}

\bibverse{7} Y observando cómo escogían los primeros asientos á la mesa,
propuso una parábola á los convidados, diciéndoles: \bibverse{8} Cuando
fueres convidado de alguno á bodas, no te sientes en el primer lugar, no
sea que otro más honrado que tú esté por él convidado, \bibverse{9} Y
viniendo el que te llamó á ti y á él, te diga: Da lugar á éste: y
entonces comiences con vergüenza á tener el lugar último. \bibverse{10}
Mas cuando fueres convidado, ve, y siéntate en el postrer lugar; porque
cuando viniere el que te llamó, te diga: Amigo, sube arriba: entonces
tendrás gloria delante de los que juntamente se asientan á la mesa.
\bibverse{11} Porque cualquiera que se ensalza, será humillado; y el que
se humilla, será ensalzado. \footnote{\textbf{14:11} Luc 18,14; Mat
  23,12; Sant 4,6; Sant 1,4-10}

\bibverse{12} Y dijo también al que le había convidado: Cuando haces
comida ó cena, no llames á tus amigos, ni á tus hermanos, ni á tus
parientes, ni á vecinos ricos; porque también ellos no te vuelvan á
convidar, y te sea hecha compensación. \bibverse{13} Mas cuando haces
banquete, llama á los pobres, los mancos, los cojos, los ciegos;
\bibverse{14} Y serás bienaventurado; porque no te pueden retribuir; mas
te será recompensado en la resurrección de los justos. \footnote{\textbf{14:14}
  Juan 5,29}

\hypertarget{la-paruxe1bola-de-la-gran-cena}{%
\subsection{La parábola de la gran
cena}\label{la-paruxe1bola-de-la-gran-cena}}

\bibverse{15} Y oyendo esto uno de los que juntamente estaban sentados á
la mesa, le dijo: Bienaventurado el que comerá pan en el reino de los
cielos. \footnote{\textbf{14:15} Luc 13,29}

\bibverse{16} El entonces le dijo: Un hombre hizo una grande cena, y
convidó á muchos. \bibverse{17} Y á la hora de la cena envió á su siervo
á decir á los convidados: Venid, que ya está todo aparejado.
\bibverse{18} Y comenzaron todos á una á excusarse. El primero le dijo:
He comprado una hacienda, y necesito salir y verla; te ruego que me des
por excusado.

\bibverse{19} Y el otro dijo: He comprado cinco yuntas de bueyes, y voy
á probarlos; ruégote que me des por excusado.

\bibverse{20} Y el otro dijo: Acabo de casarme, y por tanto no puedo ir.
\footnote{\textbf{14:20} 1Cor 7,33}

\bibverse{21} Y vuelto el siervo, hizo saber estas cosas á su señor.
Entonces enojado el padre de la familia, dijo á su siervo: Ve presto por
las plazas y por las calles de la ciudad, y mete acá los pobres, los
mancos, y cojos, y ciegos.

\bibverse{22} Y dijo el siervo: Señor, hecho es como mandaste, y aun hay
lugar.

\bibverse{23} Y dijo el señor al siervo: Ve por los caminos y por los
vallados, y fuérzalos á entrar, para que se llene mi casa. \bibverse{24}
Porque os digo que ninguno de aquellos hombres que fueron llamados,
gustará mi cena.

\hypertarget{sobre-las-condiciones-del-discipulado-y-la-seriedad-de-seguir-a-jesuxfas}{%
\subsection{Sobre las condiciones del discipulado y la seriedad de
seguir a
Jesús}\label{sobre-las-condiciones-del-discipulado-y-la-seriedad-de-seguir-a-jesuxfas}}

\bibverse{25} Y muchas gentes iban con él; y volviéndose les dijo:
\bibverse{26} Si alguno viene á mí, y no aborrece á su padre, y madre, y
mujer, é hijos, y hermanos, y hermanas, y aun también su vida, no puede
ser mi discípulo. \bibverse{27} Y cualquiera que no trae su cruz, y
viene en pos de mí, no puede ser mi discípulo. \footnote{\textbf{14:27}
  Luc 9,23} \bibverse{28} Porque ¿cuál de vosotros, queriendo edificar
una torre, no cuenta primero sentado los gastos, si tiene lo que
necesita para acabarla? \bibverse{29} Porque después que haya puesto el
fundamento, y no pueda acabarla, todos los que lo vieren, no comiencen á
hacer burla de él, \bibverse{30} Diciendo: Este hombre comenzó á
edificar, y no pudo acabar. \bibverse{31} ¿O cuál rey, habiendo de ir á
hacer guerra contra otro rey, sentándose primero no consulta si puede
salir al encuentro con diez mil al que viene contra él con veinte mil?
\bibverse{32} De otra manera, cuando aun el otro está lejos, le ruega
por la paz, enviándole embajada. \bibverse{33} Así pues, cualquiera de
vosotros que no renuncia á todas las cosas que posee, no puede ser mi
discípulo.

\bibverse{34} Buena es la sal; mas si aun la sal fuere desvanecida, ¿con
qué se adobará? \footnote{\textbf{14:34} Mat 5,13; Mar 9,50}

\bibverse{35} Ni para la tierra, ni para el muladar es buena; fuera la
arrojan. Quien tiene oídos para oir, oiga.

\hypertarget{paruxe1bolas-de-la-oveja-perdida-y-la-dracma-perdida}{%
\subsection{Parábolas de la oveja perdida y la dracma
perdida}\label{paruxe1bolas-de-la-oveja-perdida-y-la-dracma-perdida}}

\hypertarget{section-14}{%
\section{15}\label{section-14}}

\bibverse{1} Y se llegaban á él todos los publicanos y pecadores á
oirle. \bibverse{2} Y murmuraban los Fariseos y los escribas, diciendo:
Este á los pecadores recibe, y con ellos come.

\bibverse{3} Y él les propuso esta parábola, diciendo: \bibverse{4} ¿Qué
hombre de vosotros, teniendo cien ovejas, si perdiere una de ellas, no
deja las noventa y nueve en el desierto, y va á la que se perdió, hasta
que la halle? \footnote{\textbf{15:4} Luc 19,10; Juan 10,11-12}
\bibverse{5} Y hallada, la pone sobre sus hombros gozoso; \bibverse{6} Y
viniendo á casa, junta á los amigos y á los vecinos, diciéndoles: Dadme
el parabién, porque he hallado mi oveja que se había perdido.
\bibverse{7} Os digo, que así habrá más gozo en el cielo de un pecador
que se arrepiente, que de noventa y nueve justos, que no necesitan
arrepentimiento.

\bibverse{8} ¿O qué mujer que tiene diez dracmas, si perdiere una
dracma, no enciende el candil, y barre la casa, y busca con diligencia
hasta hallarla? \bibverse{9} Y cuando la hubiere hallado, junta las
amigas y las vecinas, diciendo: Dadme el parabién, porque he hallado la
dracma que había perdido. \bibverse{10} Así os digo que hay gozo delante
de los ángeles de Dios por un pecador que se arrepiente.

\hypertarget{la-paruxe1bola-del-hijo-pruxf3digo}{%
\subsection{La parábola del hijo
pródigo}\label{la-paruxe1bola-del-hijo-pruxf3digo}}

\bibverse{11} Y dijo: Un hombre tenía dos hijos; \bibverse{12} Y el
menor de ellos dijo á su padre: Padre, dame la parte de la hacienda que
me pertenece: y les repartió la hacienda. \bibverse{13} Y no muchos días
después, juntándolo todo el hijo menor, partió lejos á una provincia
apartada; y allí desperdició su hacienda viviendo perdidamente.
\footnote{\textbf{15:13} Prov 29,3} \bibverse{14} Y cuando todo lo hubo
malgastado, vino una grande hambre en aquella provincia, y comenzóle á
faltar. \bibverse{15} Y fué y se llegó á uno de los ciudadanos de
aquella tierra, el cual le envió á su hacienda para que apacentase los
puercos. \bibverse{16} Y deseaba henchir su vientre de las algarrobas
que comían los puercos; mas nadie se las daba. \bibverse{17} Y volviendo
en sí, dijo: ¡Cuántos jornaleros en casa de mi padre tienen abundancia
de pan, y yo aquí perezco de hambre! \bibverse{18} Me levantaré, é iré á
mi padre, y le diré: Padre, he pecado contra el cielo, y contra ti;
\footnote{\textbf{15:18} Jer 3,12-13; Sal 51,6} \bibverse{19} Ya no soy
digno de ser llamado tu hijo; hazme como á uno de tus jornaleros.

\bibverse{20} Y levantándose, vino á su padre. Y como aun estuviese
lejos, viólo su padre, y fué movido á misericordia, y corrió, y echóse
sobre su cuello, y besóle. \bibverse{21} Y el hijo le dijo: Padre, he
pecado contra el cielo, y contra ti, y ya no soy digno de ser llamado tu
hijo.

\bibverse{22} Mas el padre dijo á sus siervos: Sacad el principal
vestido, y vestidle; y poned un anillo en su mano, y zapatos en sus
pies. \bibverse{23} Y traed el becerro grueso, y matadlo, y comamos, y
hagamos fiesta: \bibverse{24} Porque este mi hijo muerto era, y ha
revivido; habíase perdido, y es hallado. Y comenzaron á regocijarse.

\bibverse{25} Y su hijo el mayor estaba en el campo; el cual como vino,
y llegó cerca de casa, oyó la sinfonía y las danzas; \bibverse{26} Y
llamando á uno de los criados, preguntóle qué era aquello. \bibverse{27}
Y él le dijo: Tu hermano ha venido; y tu padre ha muerto el becerro
grueso, por haberle recibido salvo. \bibverse{28} Entonces se enojó, y
no quería entrar. Salió por tanto su padre, y le rogaba que entrase.
\footnote{\textbf{15:28} Mat 20,15} \bibverse{29} Mas él respondiendo,
dijo al padre: He aquí tantos años te sirvo, no habiendo traspasado
jamás tu mandamiento, y nunca me has dado un cabrito para gozarme con
mis amigos: \bibverse{30} Mas cuando vino éste tu hijo, que ha consumido
tu hacienda con rameras, has matado para él el becerro grueso.

\bibverse{31} El entonces le dijo: Hijo, tú siempre estás conmigo, y
todas mis cosas son tuyas. \bibverse{32} Mas era menester hacer fiesta y
holgarnos, porque este tu hermano muerto era, y ha revivido; habíase
perdido, y es hallado.

\hypertarget{paruxe1bola-del-mayordomo-infiel-pero-sabio}{%
\subsection{Parábola del mayordomo infiel pero
sabio}\label{paruxe1bola-del-mayordomo-infiel-pero-sabio}}

\hypertarget{section-15}{%
\section{16}\label{section-15}}

\bibverse{1} Y dijo también á sus discípulos: Había un hombre rico, el
cual tenía un mayordomo, y éste fué acusado delante de él como disipador
de sus bienes. \bibverse{2} Y le llamó, y le dijo: ¿Qué es esto que oigo
de ti? Da cuenta de tu mayordomía, porque ya no podrás más ser
mayordomo.

\bibverse{3} Entonces el mayordomo dijo dentro de sí: ¿Qué haré? que mi
señor me quita la mayordomía. Cavar, no puedo; mendigar, tengo
vergüenza. \bibverse{4} Yo sé lo que haré para que cuando fuere quitado
de la mayordomía, me reciban en sus casas. \bibverse{5} Y llamando á
cada uno de los deudores de su señor, dijo al primero: ¿Cuánto debes á
mi señor? \bibverse{6} Y él dijo: Cien barriles de aceite. Y le dijo:
Toma tu obligación, y siéntate presto, y escribe cincuenta. \bibverse{7}
Después dijo á otro: ¿Y tú, cuánto debes? Y él dijo: Cien coros de
trigo. Y él le dijo: Toma tu obligación, y escribe ochenta.

\bibverse{8} Y alabó el señor al mayordomo malo por haber hecho
discretamente; porque los hijos de este siglo son en su generación más
sagaces que los hijos de luz. \bibverse{9} Y yo os digo: Haceos amigos
de las riquezas de maldad, para que cuando faltareis, os reciban en las
moradas eternas. \footnote{\textbf{16:9} Luc 14,14; Mat 6,20; Mat 19,21}

\hypertarget{de-lealtad}{%
\subsection{De lealtad}\label{de-lealtad}}

\bibverse{10} El que es fiel en lo muy poco, también en lo más es fiel:
y el que en lo muy poco es injusto, también en lo más es injusto.
\footnote{\textbf{16:10} Luc 19,17} \bibverse{11} Pues si en las malas
riquezas no fuisteis fieles, ¿quién os confiará lo verdadero?
\bibverse{12} Y si en lo ajeno no fuisteis fieles, ¿quién os dará lo que
es vuestro? \bibverse{13} Ningún siervo puede servir á dos señores;
porque ó aborrecerá al uno y amará al otro, ó se allegará al uno y
menospreciará al otro. No podéis servir á Dios y á las riquezas.
\footnote{\textbf{16:13} Mat 6,24}

\hypertarget{jesuxfas-reprende-a-los-fariseos-codiciosos-y-burladores}{%
\subsection{Jesús reprende a los fariseos codiciosos y
burladores}\label{jesuxfas-reprende-a-los-fariseos-codiciosos-y-burladores}}

\bibverse{14} Y oían también todas estas cosas los Fariseos, los cuales
eran avaros, y se burlaban de él. \bibverse{15} Y díjoles: Vosotros sois
los que os justificáis á vosotros mismos delante de los hombres; mas
Dios conoce vuestros corazones; porque lo que los hombres tienen por
sublime, delante de Dios es abominación.

\bibverse{16} La ley y los profetas hasta Juan: desde entonces el reino
de Dios es anunciado, y quienquiera se esfuerza á entrar en él.
\footnote{\textbf{16:16} Mat 11,12-13} \bibverse{17} Empero más fácil
cosa es pasar el cielo y la tierra, que frustrarse un tilde de la ley.
\footnote{\textbf{16:17} Mat 5,18}

\bibverse{18} Cualquiera que repudia á su mujer, y se casa con otra,
adultera: y el que se casa con la repudiada del marido, adultera.
\footnote{\textbf{16:18} Mat 5,32; Mat 19,9}

\hypertarget{historia-del-rico-y-el-pobre-luxe1zaro}{%
\subsection{Historia del rico y el pobre
Lázaro}\label{historia-del-rico-y-el-pobre-luxe1zaro}}

\bibverse{19} Había un hombre rico, que se vestía de púrpura y de lino
fino, y hacía cada día banquete con esplendidez. \bibverse{20} Había
también un mendigo llamado Lázaro, el cual estaba echado á la puerta de
él, lleno de llagas, \bibverse{21} Y deseando hartarse de las migajas
que caían de la mesa del rico; y aun los perros venían y le lamían las
llagas. \bibverse{22} Y aconteció que murió el mendigo, y fué llevado
por los ángeles al seno de Abraham: y murió también el rico, y fué
sepultado. \bibverse{23} Y en el infierno alzó sus ojos, estando en los
tormentos, y vió á Abraham de lejos, y á Lázaro en su seno.
\bibverse{24} Entonces él, dando voces, dijo: Padre Abraham, ten
misericordia de mí, y envía á Lázaro que moje la punta de su dedo en
agua, y refresque mi lengua; porque soy atormentado en esta llama.

\bibverse{25} Y díjole Abraham: Hijo, acuérdate que recibiste tus bienes
en tu vida, y Lázaro también males; mas ahora éste es consolado aquí, y
tú atormentado. \bibverse{26} Y además de todo esto, una grande sima
está constituída entre nosotros y vosotros, que los que quisieren pasar
de aquí á vosotros, no pueden, ni de allá pasar acá.

\bibverse{27} Y dijo: Ruégote pues, padre, que le envíes á la casa de mi
padre; \bibverse{28} Porque tengo cinco hermanos; para que les
testifique, porque no vengan ellos también á este lugar de tormento.

\bibverse{29} Y Abraham le dice: A Moisés y á los profetas tienen:
óiganlos. \footnote{\textbf{16:29} 2Tim 3,16}

\bibverse{30} El entonces dijo: No, padre Abraham: mas si alguno fuere á
ellos de los muertos, se arrepentirán.

\bibverse{31} Mas Abraham le dijo: Si no oyen á Moisés y á los profetas,
tampoco se persuadirán, si alguno se levantare de los muertos.

\hypertarget{advertencia-contra-la-seducciuxf3n-y-recordatorio-de-perdonar}{%
\subsection{Advertencia contra la seducción y recordatorio de
perdonar}\label{advertencia-contra-la-seducciuxf3n-y-recordatorio-de-perdonar}}

\hypertarget{section-16}{%
\section{17}\label{section-16}}

\bibverse{1} Y á SUS discípulos dice: Imposible es que no vengan
escándalos; mas ¡ay de aquél por quien vienen! \bibverse{2} Mejor le
fuera, si le pusiesen al cuello una piedra de molino, y le lanzasen en
el mar, que escandalizar á uno de estos pequeñitos. \bibverse{3} Mirad
por vosotros: si pecare contra ti tu hermano, repréndele; y si se
arrepintiere, perdónale. \bibverse{4} Y si siete veces al día pecare
contra ti, y siete veces al día se volviere á ti, diciendo, pésame,
perdónale. \footnote{\textbf{17:4} Mat 18,15; Mat 18,21-22}

\hypertarget{del-poder-de-la-fe}{%
\subsection{Del poder de la fe}\label{del-poder-de-la-fe}}

\bibverse{5} Y dijeron los apóstoles al Señor: Auméntanos la fe.

\bibverse{6} Entonces el Señor dijo: Si tuvieseis fe como un grano de
mostaza, diréis á este sicómoro: Desarráigate, y plántate en el mar; y
os obedecerá.

\hypertarget{paruxe1bola-del-seuxf1or-y-su-siervo-comprometido-a-trabajar}{%
\subsection{Parábola del Señor y su siervo comprometido a
trabajar}\label{paruxe1bola-del-seuxf1or-y-su-siervo-comprometido-a-trabajar}}

\bibverse{7} ¿Y quién de vosotros tiene un siervo que ara ó apacienta,
que vuelto del campo le diga luego: Pasa, siéntate á la mesa?
\bibverse{8} ¿No le dice antes: Adereza qué cene, y arremángate, y
sírveme hasta que haya comido y bebido; y después de esto, come tú y
bebe? \bibverse{9} ¿Da gracias al siervo porque hizo lo que le había
sido mandado? Pienso que no. \bibverse{10} Así también vosotros, cuando
hubiereis hecho todo lo que os es mandado, decid: Siervos inútiles
somos, porque lo que debíamos hacer, hicimos. \footnote{\textbf{17:10}
  1Cor 9,16}

\hypertarget{curaciuxf3n-de-diez-leprosos-el-samaritano-agradecido}{%
\subsection{Curación de diez leprosos; el samaritano
agradecido}\label{curaciuxf3n-de-diez-leprosos-el-samaritano-agradecido}}

\bibverse{11} Y aconteció que yendo él á Jerusalem, pasaba por medio de
Samaria y de Galilea. \footnote{\textbf{17:11} Luc 9,51; Luc 13,22}
\bibverse{12} Y entrando en una aldea, viniéronle al encuentro diez
hombres leprosos, los cuales se pararon de lejos, \footnote{\textbf{17:12}
  Lev 13,45-46} \bibverse{13} Y alzaron la voz, diciendo: Jesús,
Maestro, ten misericordia de nosotros.

\bibverse{14} Y como él los vió, les dijo: Id, mostraos á los
sacerdotes. Y aconteció, que yendo ellos, fueron limpios. \bibverse{15}
Entonces uno de ellos, como se vió que estaba limpio, volvió,
glorificando á Dios á gran voz; \bibverse{16} Y derribóse sobre el
rostro á sus pies, dándole gracias: y éste era Samaritano.

\bibverse{17} Y respondiendo Jesús, dijo: ¿No son diez los que fueron
limpios? ¿Y los nueve dónde están? \bibverse{18} ¿No hubo quien volviese
y diese gloria á Dios sino este extranjero? \bibverse{19} Y díjole:
Levántate, vete; tu fe te ha salvado. \footnote{\textbf{17:19} Luc 7,50}

\hypertarget{de-la-venida-del-reino-de-dios}{%
\subsection{De la venida del reino de
Dios}\label{de-la-venida-del-reino-de-dios}}

\bibverse{20} Y preguntado por los Fariseos, cuándo había de venir el
reino de Dios, les respondió y dijo: El reino de Dios no vendrá con
advertencia; \footnote{\textbf{17:20} Juan 18,36} \bibverse{21} Ni
dirán: Helo aquí, ó helo allí: porque he aquí el reino de Dios entre
vosotros está.

\bibverse{22} Y dijo á sus discípulos: Tiempo vendrá, cuando desearéis
ver uno de los días del Hijo del hombre, y no lo veréis. \bibverse{23} Y
os dirán: Helo aquí, ó helo allí. No vayáis, ni sigáis. \footnote{\textbf{17:23}
  Luc 21,8} \bibverse{24} Porque como el relámpago, relampagueando desde
una parte de debajo del cielo, resplandece hasta la otra debajo del
cielo, así también será el Hijo del hombre en su día. \bibverse{25} Mas
primero es necesario que padezca mucho, y sea reprobado de esta
generación. \bibverse{26} Y como fué en los días de Noé, así también
será en los días del Hijo del hombre. \bibverse{27} Comían, bebían, los
hombres tomaban mujeres, y las mujeres maridos, hasta el día que entró
Noé en el arca; y vino el diluvio, y destruyó á todos. \footnote{\textbf{17:27}
  Gén 6,1-8} \bibverse{28} Asimismo también como fué en los días de Lot;
comían, bebían, compraban, vendían, plantaban, edificaban; \bibverse{29}
Mas el día que Lot salió de Sodoma, llovió del cielo fuego y azufre, y
destruyó á todos: \bibverse{30} Como esto será el día en que el Hijo del
hombre se manifestará. \bibverse{31} En aquel día, el que estuviere en
el terrado, y sus alhajas en casa, no descienda á tomarlas: y el que en
el campo, asimismo no vuelva atrás. \bibverse{32} Acordaos de la mujer
de Lot. \footnote{\textbf{17:32} Gén 19,26} \bibverse{33} Cualquiera que
procurare salvar su vida, la perderá; y cualquiera que la perdiere, la
salvará. \footnote{\textbf{17:33} Luc 9,24} \bibverse{34} Os digo que en
aquella noche estarán dos en una cama; el uno será tomado, y el otro
será dejado. \bibverse{35} Dos mujeres estarán moliendo juntas: la una
será tomada, y la otra dejada. \bibverse{36} Dos estarán en el campo; el
uno será tomado, y el otro dejado.

\bibverse{37} Y respondiendo, le dicen: ¿Dónde, Señor? Y él les dijo:
Donde estuviere el cuerpo, allá se juntarán también las águilas.

\hypertarget{paruxe1bola-del-juez-impuxedo-y-la-viuda-suplicante}{%
\subsection{Parábola del juez impío y la viuda
suplicante}\label{paruxe1bola-del-juez-impuxedo-y-la-viuda-suplicante}}

\hypertarget{section-17}{%
\section{18}\label{section-17}}

\bibverse{1} Y propúsoles también una parábola sobre que es necesario
orar siempre, y no desmayar, \footnote{\textbf{18:1} 1Tes 5,17}
\bibverse{2} Diciendo: Había un juez en una ciudad, el cual ni temía á
Dios, ni respetaba á hombre. \bibverse{3} Había también en aquella
ciudad una viuda, la cual venía á él diciendo: Hazme justicia de mi
adversario. \bibverse{4} Pero él no quiso por algún tiempo; mas después
de esto dijo dentro de sí: Aunque ni temo á Dios, ni tengo respeto á
hombre, \bibverse{5} Todavía, porque esta viuda me es molesta, le haré
justicia, porque al fin no venga y me muela.

\bibverse{6} Y dijo el Señor: Oid lo que dice el juez injusto.
\bibverse{7} ¿Y Dios no hará justicia á sus escogidos, que claman á él
día y noche, aunque sea longánime acerca de ellos? \bibverse{8} Os digo
que los defenderá presto. Empero cuando el Hijo del hombre viniere,
¿hallará fe en la tierra?

\hypertarget{la-paruxe1bola-del-fariseo-y-el-recaudador-de-impuestos}{%
\subsection{La parábola del fariseo y el recaudador de
impuestos}\label{la-paruxe1bola-del-fariseo-y-el-recaudador-de-impuestos}}

\bibverse{9} Y dijo también á unos que confiaban de sí como justos, y
menospreciaban á los otros, esta parábola: \footnote{\textbf{18:9} Rom
  10,3; Mat 5,6} \bibverse{10} Dos hombres subieron al templo á orar: el
uno Fariseo, el otro publicano. \bibverse{11} El Fariseo, en pie, oraba
consigo de esta manera: Dios, te doy gracias, que no soy como los otros
hombres, ladrones, injustos, adúlteros, ni aun como este publicano;
\bibverse{12} Ayuno dos veces á la semana, doy diezmos de todo lo que
poseo. \footnote{\textbf{18:12} Mat 23,23} \bibverse{13} Mas el
publicano estando lejos no quería ni aun alzar los ojos al cielo, sino
que hería su pecho, diciendo: Dios, sé propicio á mí pecador.
\footnote{\textbf{18:13} Sal 51,19} \bibverse{14} Os digo que éste
descendió á su casa justificado antes que el otro; porque cualquiera que
se ensalza, será humillado; y el que se humilla, será ensalzado.
\footnote{\textbf{18:14} Luc 14,11; Mat 21,3; Mat 23,12}

\hypertarget{jesuxfas-bendice-a-los-niuxf1os}{%
\subsection{Jesús bendice a los
niños}\label{jesuxfas-bendice-a-los-niuxf1os}}

\bibverse{15} Y traían á él los niños para que los tocase; lo cual
viendo los discípulos les reñían. \bibverse{16} Mas Jesús llamándolos,
dijo: Dejad los niños venir á mí, y no los impidáis; porque de tales es
el reino de Dios. \bibverse{17} De cierto os digo, que cualquiera que no
recibiere el reino de Dios como un niño, no entrará en él.

\hypertarget{del-peligro-de-la-riqueza}{%
\subsection{Del peligro de la riqueza}\label{del-peligro-de-la-riqueza}}

\bibverse{18} Y preguntóle un príncipe, diciendo: Maestro bueno, ¿qué
haré para poseer la vida eterna?

\bibverse{19} Y Jesús le dijo: ¿Por qué me llamas bueno? ninguno hay
bueno sino sólo Dios. \bibverse{20} Los mandamientos sabes: No matarás:
No adulterarás: No hurtarás: No dirás falso testimonio: Honra á tu padre
y á tu madre. \footnote{\textbf{18:20} Éxod 20,12-16}

\bibverse{21} Y él dijo: Todas estas cosas he guardado desde mi
juventud.

\bibverse{22} Y Jesús, oído esto, le dijo: Aun te falta una cosa: vende
todo lo que tienes, y da á los pobres, y tendrás tesoro en el cielo; y
ven, sígueme.

\bibverse{23} Entonces él, oídas estas cosas, se puso muy triste, porque
era muy rico.

\bibverse{24} Y viendo Jesús que se había entristecido mucho, dijo:
¡Cuán dificultosamente entrarán en el reino de Dios los que tienen
riquezas! \footnote{\textbf{18:24} Luc 19,9} \bibverse{25} Porque más
fácil cosa es entrar un camello por el ojo de una aguja, que un rico
entrar en el reino de Dios.

\bibverse{26} Y los que lo oían, dijeron: ¿Y quién podrá ser salvo?

\bibverse{27} Y él les dijo: Lo que es imposible para con los hombres,
posible es para Dios.

\hypertarget{la-recompensa-de-seguir-a-jesuxfas}{%
\subsection{La recompensa de seguir a
Jesús}\label{la-recompensa-de-seguir-a-jesuxfas}}

\bibverse{28} Entonces Pedro dijo: He aquí, nosotros hemos dejado las
posesiones nuestras, y te hemos seguido.

\bibverse{29} Y él les dijo: De cierto os digo, que nadie hay que haya
dejado casa, padres, ó hermanos, ó mujer, ó hijos, por el reino de Dios,
\bibverse{30} Que no haya de recibir mucho más en este tiempo, y en el
siglo venidero la vida eterna.

\hypertarget{huele-a-jerusaluxe9n-tercer-anuncio-del-sufrimiento-de-jesuxfas}{%
\subsection{Huele a Jerusalén; tercer anuncio del sufrimiento de
Jesús}\label{huele-a-jerusaluxe9n-tercer-anuncio-del-sufrimiento-de-jesuxfas}}

\bibverse{31} Y Jesús, tomando á los doce, les dijo: He aquí subimos á
Jerusalem, y serán cumplidas todas las cosas que fueron escritas por los
profetas, del Hijo del hombre. \bibverse{32} Porque será entregado á las
gentes, y será escarnecido, é injuriado, y escupido. \bibverse{33} Y
después que le hubieren azotado, le matarán: mas al tercer día
resucitará.

\bibverse{34} Pero ellos nada de estas cosas entendían, y esta palabra
les era encubierta, y no entendían lo que se decía. \footnote{\textbf{18:34}
  Luc 9,45; Luc 24,45}

\hypertarget{la-curaciuxf3n-del-ciego-en-jericuxf3}{%
\subsection{La curación del ciego en
Jericó}\label{la-curaciuxf3n-del-ciego-en-jericuxf3}}

\bibverse{35} Y aconteció que acercándose él á Jericó, un ciego estaba
sentado junto al camino mendigando; \bibverse{36} El cual como oyó la
gente que pasaba, preguntó qué era aquello. \bibverse{37} Y dijéronle
que pasaba Jesús Nazareno. \bibverse{38} Entonces dió voces, diciendo:
Jesús, Hijo de David, ten misericordia de mí. \bibverse{39} Y los que
iban delante, le reñían que callase; mas él clamaba mucho más: Hijo de
David, ten misericordia de mí.

\bibverse{40} Jesús entonces parándose, mandó traerle á sí: y como él
llegó, le preguntó, \bibverse{41} Diciendo: ¿Qué quieres que te haga? Y
él dijo: Señor, que vea.

\bibverse{42} Y Jesús le dijo: Ve, tu fe te ha hecho salvo.

\bibverse{43} Y luego vió, y le seguía, glorificando á Dios: y todo el
pueblo como lo vió, dió á Dios alabanza.

\hypertarget{la-visita-de-jesuxfas-al-principal-recaudador-de-impuestos-zaqueo-en-jericuxf3}{%
\subsection{La visita de Jesús al principal recaudador de impuestos
Zaqueo en
Jericó}\label{la-visita-de-jesuxfas-al-principal-recaudador-de-impuestos-zaqueo-en-jericuxf3}}

\hypertarget{section-18}{%
\section{19}\label{section-18}}

\bibverse{1} Y habiendo entrado Jesús, iba pasando por Jericó;
\bibverse{2} Y he aquí un varón llamado Zaqueo, el cual era el principal
de los publicanos, y era rico; \bibverse{3} Y procuraba ver á Jesús
quién fuese; mas no podía á causa de la multitud, porque era pequeño de
estatura. \bibverse{4} Y corriendo delante, subióse á un árbol sicómoro
para verle; porque había de pasar por allí. \bibverse{5} Y como vino á
aquel lugar Jesús, mirando, le vió, y díjole: Zaqueo, date priesa,
desciende, porque hoy es necesario que pose en tu casa. \bibverse{6}
Entonces él descendió apriesa, y le recibió gozoso. \bibverse{7} Y
viendo esto, todos murmuraban, diciendo que había entrado á posar con un
hombre pecador. \footnote{\textbf{19:7} Luc 15,2}

\bibverse{8} Entonces Zaqueo, puesto en pie, dijo al Señor: He aquí,
Señor, la mitad de mis bienes doy á los pobres; y si en algo he
defraudado á alguno, lo vuelvo con el cuatro tanto. \footnote{\textbf{19:8}
  Éxod 21,37; Ezeq 33,14-16}

\bibverse{9} Y Jesús le dijo: Hoy ha venido la salvación á esta casa;
por cuanto él también es hijo de Abraham. \footnote{\textbf{19:9} Luc
  13,16} \bibverse{10} Porque el Hijo del hombre vino á buscar y á
salvar lo que se había perdido. \footnote{\textbf{19:10} Luc 5,32; Ezeq
  34,16; 1Tim 1,15}

\hypertarget{la-paruxe1bola-de-las-minas-confiadas}{%
\subsection{La parábola de las minas
confiadas}\label{la-paruxe1bola-de-las-minas-confiadas}}

\bibverse{11} Y oyendo ellos estas cosas, prosiguió Jesús y dijo una
parábola, por cuanto estaba cerca de Jerusalem, y porque pensaban que
luego había de ser manifestado el reino de Dios. \bibverse{12} Dijo
pues: Un hombre noble partió á una provincia lejos, para tomar para sí
un reino, y volver. \bibverse{13} Mas llamados diez siervos suyos, les
dió diez minas, y díjoles: Negociad entre tanto que vengo. \bibverse{14}
Empero sus ciudadanos le aborrecían, y enviaron tras de él una embajada,
diciendo: No queremos que éste reine sobre nosotros. \footnote{\textbf{19:14}
  Juan 1,11}

\bibverse{15} Y aconteció, que vuelto él, habiendo tomado el reino,
mandó llamar á sí á aquellos siervos á los cuales había dado el dinero,
para saber lo que había negociado cada uno. \bibverse{16} Y vino el
primero, diciendo: Señor, tu mina ha ganado diez minas.

\bibverse{17} Y él le dice: Está bien, buen siervo; pues que en lo poco
has sido fiel, tendrás potestad sobre diez ciudades.

\bibverse{18} Y vino otro, diciendo: Señor, tu mina ha hecho cinco
minas.

\bibverse{19} Y también á éste dijo: Tú también sé sobre cinco ciudades.

\bibverse{20} Y vino otro, diciendo: Señor, he aquí tu mina, la cual he
tenido guardada en un pañizuelo: \bibverse{21} Porque tuve miedo de ti,
que eres hombre recio; tomas lo que no pusiste, y siegas lo que no
sembraste.

\bibverse{22} Entonces él le dijo: Mal siervo, de tu boca te juzgo.
Sabías que yo era hombre recio, que tomo lo que no puse, y que siego lo
que no sembré; \bibverse{23} ¿Por qué, pues, no diste mi dinero al
banco, y yo viniendo lo demandara con el logro? \bibverse{24} Y dijo á
los que estaban presentes: Quitadle la mina, y dadla al que tiene las
diez minas.

\bibverse{25} Y ellos le dijeron: Señor, tiene diez minas. \bibverse{26}
Pues yo os digo que á cualquiera que tuviere, le será dado; mas al que
no tuviere, aun lo que tiene le será quitado. \footnote{\textbf{19:26}
  Luc 8,18; Mat 13,12} \bibverse{27} Y también á aquellos mis enemigos
que no querían que yo reinase sobre ellos, traedlos acá, y degolladlos
delante de mí.

\hypertarget{jesuxfas-a-las-puertas-de-jerusaluxe9n-su-entrada-en-jerusaluxe9n}{%
\subsection{Jesús a las puertas de Jerusalén; su entrada en
Jerusalén}\label{jesuxfas-a-las-puertas-de-jerusaluxe9n-su-entrada-en-jerusaluxe9n}}

\bibverse{28} Y dicho esto, iba delante subiendo á Jerusalem.

\bibverse{29} Y aconteció, que llegando cerca de Bethfagé, y de
Bethania, al monte que se llama de las Olivas, envió dos de sus
discípulos, \bibverse{30} Diciendo: Id á la aldea de enfrente; en la
cual como entrareis, hallaréis un pollino atado, en el que ningún hombre
se ha sentado jamás; desatadlo, y traedlo. \bibverse{31} Y si alguien os
preguntare, ¿por qué lo desatáis? le responderéis así: Porque el Señor
lo ha menester.

\bibverse{32} Y fueron los que habían sido enviados, y hallaron como les
dijo. \bibverse{33} Y desatando ellos el pollino, sus dueños les
dijeron: ¿Por qué desatáis el pollino? \bibverse{34} Y ellos dijeron:
Porque el Señor lo ha menester. \bibverse{35} Y trajéronlo á Jesús; y
habiendo echado sus vestidos sobre el pollino, pusieron á Jesús encima.
\bibverse{36} Y yendo él tendían sus capas por el camino.

\bibverse{37} Y como llegasen ya cerca de la bajada del monte de las
Olivas, toda la multitud de los discípulos, gozándose, comenzaron á
alabar á Dios á gran voz por todas las maravillas que habían visto,
\bibverse{38} Diciendo: ¡Bendito el rey que viene en el nombre del
Señor: paz en el cielo, y gloria en lo altísimo! \footnote{\textbf{19:38}
  Luc 2,14; Sal 118,26}

\bibverse{39} Entonces algunos de los Fariseos de la compañía, le
dijeron: Maestro, reprende á tus discípulos.

\bibverse{40} Y él respondiendo, les dijo: Os digo que si éstos
callaren, las piedras clamarán.

\hypertarget{jesuxfas-llora-por-jerusaluxe9n-y-profecuxeda-de-la-destrucciuxf3n-de-jerusaluxe9n}{%
\subsection{Jesús llora por Jerusalén y profecía de la destrucción de
Jerusalén}\label{jesuxfas-llora-por-jerusaluxe9n-y-profecuxeda-de-la-destrucciuxf3n-de-jerusaluxe9n}}

\bibverse{41} Y como llegó cerca, viendo la ciudad, lloró sobre ella,
\bibverse{42} Diciendo: ¡Oh si también tú conocieses, á lo menos en este
tu día, lo que toca á tu paz! mas ahora está encubierto de tus ojos.
\bibverse{43} Porque vendrán días sobre ti, que tus enemigos te cercarán
con baluarte, y te pondrán cerco, y de todas partes te pondrán en
estrecho, \bibverse{44} Y te derribarán á tierra, y á tus hijos dentro
de ti; y no dejarán sobre ti piedra sobre piedra; por cuanto no
conociste el tiempo de tu visitación. \footnote{\textbf{19:44} Luc 21,6}

\hypertarget{jesuxfas-limpiando-el-templo}{%
\subsection{Jesús limpiando el
templo}\label{jesuxfas-limpiando-el-templo}}

\bibverse{45} Y entrando en el templo, comenzó á echar fuera á todos los
que vendían y compraban en él. \bibverse{46} Diciéndoles: Escrito está:
Mi casa, casa de oración es; mas vosotros la habéis hecho cueva de
ladrones.

\bibverse{47} Y enseñaba cada día en el templo; mas los príncipes de los
sacerdotes, y los escribas, y los principales del pueblo procuraban
matarle. \bibverse{48} Y no hallaban qué hacerle, porque todo el pueblo
estaba suspenso oyéndole.

\hypertarget{la-pregunta-del-sumo-consejo-sobre-la-autoridad-de-jesuxfas}{%
\subsection{La pregunta del sumo consejo sobre la autoridad de
Jesús}\label{la-pregunta-del-sumo-consejo-sobre-la-autoridad-de-jesuxfas}}

\hypertarget{section-19}{%
\section{20}\label{section-19}}

\bibverse{1} Y aconteció un día, que enseñando él al pueblo en el
templo, y anunciando el evangelio, llegáronse los príncipes de los
sacerdotes y los escribas, con los ancianos; \bibverse{2} Y le hablaron,
diciendo: Dinos: ¿con qué potestad haces estas cosas? ¿ó quién es el que
te ha dado esta potestad?

\bibverse{3} Respondiendo entonces Jesús, les dijo: Os preguntaré yo
también una palabra; respondedme: \bibverse{4} El bautismo de Juan, ¿era
del cielo, ó de los hombres?

\bibverse{5} Mas ellos pensaban dentro de sí, diciendo: Si dijéremos,
del cielo, dirá: ¿Por qué, pues, no le creísteis? \footnote{\textbf{20:5}
  Luc 7,29-30} \bibverse{6} Y si dijéremos, de los hombres, todo el
pueblo nos apedreará: porque están ciertos que Juan era profeta.
\bibverse{7} Y respondieron que no sabían de dónde.

\bibverse{8} Entonces Jesús les dijo: Ni yo os digo con qué potestad
hago estas cosas.

\hypertarget{la-paruxe1bola-de-los-viticultores-infieles}{%
\subsection{La parábola de los viticultores
infieles}\label{la-paruxe1bola-de-los-viticultores-infieles}}

\bibverse{9} Y comenzó á decir al pueblo esta parábola: Un hombre plantó
una viña, y arrendóla á labradores, y se ausentó por mucho tiempo.
\bibverse{10} Y al tiempo, envió un siervo á los labradores, para que le
diesen del fruto de la viña; mas los labradores le hirieron, y enviaron
vacío. \bibverse{11} Y volvió á enviar otro siervo; mas ellos á éste
también, herido y afrentado, le enviaron vacío. \bibverse{12} Y volvió á
enviar al tercer siervo; mas ellos también á éste echaron herido.
\bibverse{13} Entonces el señor de la viña dijo: ¿Qué haré? Enviaré mi
hijo amado: quizás cuando á éste vieren, tendrán respeto.

\bibverse{14} Mas los labradores, viéndole, pensaron entre sí, diciendo:
Este es el heredero; venid, matémosle para que la heredad sea nuestra.
\bibverse{15} Y echáronle fuera de la viña, y le mataron. ¿Qué pues, les
hará el señor de la viña? \bibverse{16} Vendrá, y destruirá á estos
labradores, y dará su viña á otros. Y como ellos lo oyeron, dijeron:
¡Dios nos libre!

\bibverse{17} Mas él mirándolos, dice: ¿Qué pues es lo que está escrito:
La piedra que condenaron los edificadores, ésta fué por cabeza de
esquina? \bibverse{18} Cualquiera que cayere sobre aquella piedra, será
quebrantado; mas sobre el que la piedra cayere, le desmenuzará.

\bibverse{19} Y procuraban los príncipes de los sacerdotes y los
escribas echarle mano en aquella hora, porque entendieron que contra
ellos había dicho esta parábola: mas temieron al pueblo. \footnote{\textbf{20:19}
  Luc 19,48}

\hypertarget{la-cuestiuxf3n-fiscal-de-los-fariseos}{%
\subsection{La cuestión fiscal de los
fariseos}\label{la-cuestiuxf3n-fiscal-de-los-fariseos}}

\bibverse{20} Y acechándole enviaron espías que se simulasen justos,
para sorprenderle en palabras, para que le entregasen al principado y á
la potestad del presidente. \footnote{\textbf{20:20} Luc 11,54}
\bibverse{21} Los cuales le preguntaron, diciendo: Maestro, sabemos que
dices y enseñas bien, y que no tienes respeto á persona; antes enseñas
el camino de Dios con verdad. \bibverse{22} ¿Nos es lícito dar tributo á
César, ó no?

\bibverse{23} Mas él, entendiendo la astucia de ellos, les dijo: ¿Por
qué me tentáis? \bibverse{24} Mostradme la moneda. ¿De quién tiene la
imagen y la inscripción? Y respondiendo dijeron: De César.

\bibverse{25} Entonces les dijo: Pues dad á César lo que es de César; y
lo que es de Dios, á Dios. \footnote{\textbf{20:25} Rom 13,1; Rom 13,7;
  Hech 5,29}

\bibverse{26} Y no pudieron reprender sus palabras delante del pueblo:
antes maravillados de su respuesta, callaron.

\hypertarget{sobre-la-resurrecciuxf3n-de-los-muertos}{%
\subsection{Sobre la resurrección de los
muertos}\label{sobre-la-resurrecciuxf3n-de-los-muertos}}

\bibverse{27} Y llegándose unos de los Saduceos, los cuales niegan haber
resurrección, le preguntaron, \bibverse{28} Diciendo: Maestro, Moisés
nos escribió: Si el hermano de alguno muriere teniendo mujer, y muriere
sin hijos, que su hermano tome la mujer, y levante simiente á su
hermano. \bibverse{29} Fueron, pues, siete hermanos: y el primero tomó
mujer, y murió sin hijos. \bibverse{30} Y la tomó el segundo, el cual
también murió sin hijos. \bibverse{31} Y la tomó el tercero: asimismo
también todos siete: y murieron sin dejar prole. \bibverse{32} Y á la
postre de todos murió también la mujer. \bibverse{33} En la
resurrección, pues, ¿mujer de cuál de ellos será? porque los siete la
tuvieron por mujer.

\bibverse{34} Entonces respondiendo Jesús, les dijo: Los hijos de este
siglo se casan, y son dados en casamiento: \bibverse{35} Mas los que
fueren tenidos por dignos de aquel siglo y de la resurrección de los
muertos, ni se casan, ni son dados en casamiento: \bibverse{36} Porque
no pueden ya más morir: porque son iguales á los ángeles, y son hijos de
Dios, cuando son hijos de la resurrección. \bibverse{37} Y que los
muertos hayan de resucitar, aun Moisés lo enseñó en el pasaje de la
zarza, cuando llama al Señor: Dios de Abraham, y Dios de Isaac, y Dios
de Jacob. \bibverse{38} Porque Dios no es Dios de muertos, mas de vivos:
porque todos viven á él. \footnote{\textbf{20:38} Rom 14,8}

\bibverse{39} Y respondiéndole unos de los escribas, dijeron: Maestro,
bien has dicho. \bibverse{40} Y no osaron más preguntarle algo.

\hypertarget{la-contrapregunta-de-jesuxfas-sobre-el-mesuxedas-como-hijo-de-david}{%
\subsection{La contrapregunta de Jesús sobre el Mesías como hijo de
David}\label{la-contrapregunta-de-jesuxfas-sobre-el-mesuxedas-como-hijo-de-david}}

\bibverse{41} Y él les dijo: ¿Cómo dicen que el Cristo es hijo de David?
\bibverse{42} Y el mismo David dice en el libro de los Salmos: Dijo el
Señor á mi Señor: Siéntate á mi diestra, \bibverse{43} Entre tanto que
pongo tus enemigos por estrado de tus pies.

\bibverse{44} Así que David le llama Señor: ¿cómo pues es su hijo?

\hypertarget{advertencia-de-jesuxfas-sobre-la-ambiciuxf3n-y-la-codicia-de-los-escribas}{%
\subsection{Advertencia de Jesús sobre la ambición y la codicia de los
escribas}\label{advertencia-de-jesuxfas-sobre-la-ambiciuxf3n-y-la-codicia-de-los-escribas}}

\bibverse{45} Y oyéndole todo el pueblo, dijo á sus discípulos:
\bibverse{46} Guardaos de los escribas, que quieren andar con ropas
largas, y aman las salutaciones en las plazas, y las primeras sillas en
las sinagogas, y los primeros asientos en las cenas; \bibverse{47} Que
devoran las casas de las viudas, poniendo por pretexto la larga oración:
éstos recibirán mayor condenación.

\hypertarget{jesuxfas-alaba-las-dos-blancas-de-la-viuda-pobre}{%
\subsection{Jesús alaba las dos blancas de la viuda
pobre}\label{jesuxfas-alaba-las-dos-blancas-de-la-viuda-pobre}}

\hypertarget{section-20}{%
\section{21}\label{section-20}}

\bibverse{1} Y mirando, vió á los ricos que echaban sus ofrendas en el
gazofilacio. \bibverse{2} Y vió también una viuda pobrecilla, que echaba
allí dos blancas. \bibverse{3} Y dijo: De verdad os digo, que esta pobre
viuda echó más que todos: \footnote{\textbf{21:3} 2Cor 8,12}
\bibverse{4} Porque todos estos, de lo que les sobra echaron para las
ofrendas de Dios; mas ésta de su pobreza echó todo el sustento que
tenía.

\hypertarget{el-discurso-de-jesuxfas-en-el-monte-de-los-olivos-a-los-apuxf3stoles-sobre-la-destrucciuxf3n-del-templo-y-jerusaluxe9n-el-fin-de-este-mundo-y-su-apariciuxf3n-en-el-uxfaltimo-duxeda}{%
\subsection{El discurso de Jesús en el Monte de los Olivos a los
apóstoles sobre la destrucción del templo y Jerusalén, el fin de este
mundo y su aparición en el último
día}\label{el-discurso-de-jesuxfas-en-el-monte-de-los-olivos-a-los-apuxf3stoles-sobre-la-destrucciuxf3n-del-templo-y-jerusaluxe9n-el-fin-de-este-mundo-y-su-apariciuxf3n-en-el-uxfaltimo-duxeda}}

\bibverse{5} Y á unos que decían del templo, que estaba adornado de
hermosas piedras y dones, dijo: \bibverse{6} Estas cosas que veis, días
vendrán que no quedará piedra sobre piedra que no sea destruída.

\bibverse{7} Y le preguntaron, diciendo: Maestro, ¿cuándo será esto? ¿y
qué señal habrá cuando estas cosas hayan de comenzar á ser hechas?

\hypertarget{los-primeros-signos-del-fin}{%
\subsection{Los primeros signos del
fin}\label{los-primeros-signos-del-fin}}

\bibverse{8} El entonces dijo: Mirad, no seáis engañados; porque vendrán
muchos en mi nombre, diciendo: Yo soy; y, El tiempo está cerca: por
tanto, no vayáis en pos de ellos. \bibverse{9} Empero cuando oyereis
guerras y sediciones, no os espantéis; porque es necesario que estas
cosas acontezcan primero: mas no luego será el fin.

\bibverse{10} Entonces les dijo: Se levantará gente contra gente, y
reino contra reino; \bibverse{11} Y habrá grandes terremotos, y en
varios lugares hambres y pestilencias: y habrá espantos y grandes
señales del cielo.

\hypertarget{las-persecuciones-de-los-discuxedpulos}{%
\subsection{Las persecuciones de los
discípulos}\label{las-persecuciones-de-los-discuxedpulos}}

\bibverse{12} Mas antes de todas estas cosas os echarán mano, y
perseguirán, entregándoos á las sinagogas y á las cárceles, siendo
llevados á los reyes y á los gobernadores por causa de mi nombre.
\footnote{\textbf{21:12} Mat 10,18-22; Mat 10,30} \bibverse{13} Y os
será para testimonio. \bibverse{14} Poned pues en vuestros corazones no
pensar antes cómo habéis de responder: \bibverse{15} Porque yo os daré
boca y sabiduría, á la cual no podrán resistir ni contradecir todos los
que se os opondrán. \footnote{\textbf{21:15} Hech 6,10} \bibverse{16}
Mas seréis entregados aun de vuestros padres, y hermanos, y parientes, y
amigos; y matarán á algunos de vosotros. \bibverse{17} Y seréis
aborrecidos de todos por causa de mi nombre. \bibverse{18} Mas un pelo
de vuestra cabeza no perecerá.

\bibverse{19} En vuestra paciencia poseeréis vuestras almas. \footnote{\textbf{21:19}
  Heb 10,36}

\hypertarget{la-destrucciuxf3n-de-jerusaluxe9n-y-la-difuxedcil-situaciuxf3n-del-pueblo-juduxedo}{%
\subsection{La destrucción de Jerusalén y la difícil situación del
pueblo
judío}\label{la-destrucciuxf3n-de-jerusaluxe9n-y-la-difuxedcil-situaciuxf3n-del-pueblo-juduxedo}}

\bibverse{20} Y cuando viereis á Jerusalem cercada de ejércitos, sabed
entonces que su destrucción ha llegado. \bibverse{21} Entonces los que
estuvieren en Judea, huyan á los montes; y los que en medio de ella,
váyanse; y los que estén en los campos, no entren en ella. \bibverse{22}
Porque estos son días de venganza: para que se cumplan todas las cosas
que están escritas. \bibverse{23} Mas ¡ay de las preñadas, y de las que
crían en aquellos días! porque habrá apuro grande sobre la tierra é ira
en este pueblo. \bibverse{24} Y caerán á filo de espada, y serán
llevados cautivos á todas las naciones: y Jerusalem será hollada de las
gentes, hasta que los tiempos de las gentes sean cumplidos. \footnote{\textbf{21:24}
  Is 63,18; Rom 11,25; Apoc 11,2}

\hypertarget{las-uxfaltimas-seuxf1ales-del-fin-y-la-apariciuxf3n-del-hijo-del-hombre}{%
\subsection{Las últimas señales del fin y la aparición del Hijo del
Hombre}\label{las-uxfaltimas-seuxf1ales-del-fin-y-la-apariciuxf3n-del-hijo-del-hombre}}

\bibverse{25} Entonces habrá señales en el sol, y en la luna, y en las
estrellas; y en la tierra angustia de gentes por la confusión del sonido
de la mar y de las ondas: \footnote{\textbf{21:25} Apoc 6,12-13}
\bibverse{26} Secándose los hombres á causa del temor y expectación de
las cosas que sobrevendrán á la redondez de la tierra: porque las
virtudes de los cielos serán conmovidas. \bibverse{27} Y entonces verán
al Hijo del hombre, que vendrá en una nube con potestad y majestad
grande. \footnote{\textbf{21:27} Dan 7,13} \bibverse{28} Y cuando estas
cosas comenzaren á hacerse, mirad, y levantad vuestras cabezas, porque
vuestra redención está cerca. \footnote{\textbf{21:28} Fil 4,4-5}

\bibverse{29} Y díjoles una parábola: Mirad la higuera y todos los
árboles: \bibverse{30} Cuando ya brotan, viéndolo, de vosotros mismos
entendéis que el verano está ya cerca. \bibverse{31} Así también
vosotros, cuando viereis hacerse estas cosas, entended que está cerca el
reino de Dios. \bibverse{32} De cierto os digo, que no pasará esta
generación hasta que todo sea hecho. \bibverse{33} El cielo y la tierra
pasarán; mas mis palabras no pasarán.

\hypertarget{una-advertencia-final-sobre-la-sobriedad-y-la-vigilancia}{%
\subsection{Una advertencia final sobre la sobriedad y la
vigilancia}\label{una-advertencia-final-sobre-la-sobriedad-y-la-vigilancia}}

\bibverse{34} Y mirad por vosotros, que vuestros corazones no sean
cargados de glotonería y embriaguez, y de los cuidados de esta vida, y
venga de repente sobre vosotros aquel día. \footnote{\textbf{21:34} Mar
  4,19} \bibverse{35} Porque como un lazo vendrá sobre todos los que
habitan sobre la faz de toda la tierra. \footnote{\textbf{21:35} 1Tes
  5,3} \bibverse{36} Velad pues, orando en todo tiempo, que seáis
tenidos por dignos de evitar todas estas cosas que han de venir, y de
estar en pie delante del Hijo del hombre. \footnote{\textbf{21:36} Mar
  13,33}

\bibverse{37} Y enseñaba de día en el templo; y de noche saliendo,
estábase en el monte que se llama de las Olivas. \bibverse{38} Y todo el
pueblo venía á él por la mañana, para oirle en el templo.

\hypertarget{intento-de-asesinato-por-parte-de-los-luxedderes-del-pueblo}{%
\subsection{Intento de asesinato por parte de los líderes del
pueblo}\label{intento-de-asesinato-por-parte-de-los-luxedderes-del-pueblo}}

\hypertarget{section-21}{%
\section{22}\label{section-21}}

\bibverse{1} Y estaba cerca el día de la fiesta de los ázimos, que se
llama la Pascua. \bibverse{2} Y los príncipes de los sacerdotes y los
escribas buscaban cómo le matarían; mas tenían miedo del pueblo.

\hypertarget{traiciuxf3n-de-judas}{%
\subsection{Traición de Judas}\label{traiciuxf3n-de-judas}}

\bibverse{3} Y entró Satanás en Judas, por sobrenombre Iscariote, el
cual era uno del número de los doce; \footnote{\textbf{22:3} Juan 13,2;
  Juan 13,27} \bibverse{4} Y fué, y habló con los príncipes de los
sacerdotes, y con los magistrados, de cómo se lo entregaría.
\bibverse{5} Los cuales se holgaron, y concertaron de darle dinero.
\bibverse{6} Y prometió, y buscaba oportunidad para entregarle á ellos
sin bulla.

\hypertarget{preparaciuxf3n-de-la-cena-de-pascua}{%
\subsection{Preparación de la cena de
Pascua}\label{preparaciuxf3n-de-la-cena-de-pascua}}

\bibverse{7} Y vino el día de los ázimos, en el cual era necesario matar
la pascua. \bibverse{8} Y envió á Pedro y á Juan, diciendo: Id,
aparejadnos la pascua para que comamos.

\bibverse{9} Y ellos le dijeron: ¿Dónde quieres que aparejemos?

\bibverse{10} Y él les dijo: He aquí cuando entrareis en la ciudad, os
encontrará un hombre que lleva un cántaro de agua: seguidle hasta la
casa donde entrare, \bibverse{11} Y decid al padre de la familia de la
casa: El Maestro te dice: ¿Dónde está el aposento donde tengo de comer
la pascua con mis discípulos? \bibverse{12} Entonces él os mostrará un
gran cenáculo aderezado; aparejad allí.

\bibverse{13} Fueron pues, y hallaron como les había dicho; y aparejaron
la pascua. \footnote{\textbf{22:13} Luc 19,32}

\hypertarget{la-uxfaltima-cena-de-jesuxfas-en-el-cuxedrculo-de-los-discuxedpulos-instituciuxf3n-de-la-santa-comuniuxf3n}{%
\subsection{La última cena de Jesús en el círculo de los discípulos;
Institución de la santa
comunión}\label{la-uxfaltima-cena-de-jesuxfas-en-el-cuxedrculo-de-los-discuxedpulos-instituciuxf3n-de-la-santa-comuniuxf3n}}

\bibverse{14} Y como fué hora, sentóse á la mesa, y con él los
apóstoles. \bibverse{15} Y les dijo: En gran manera he deseado comer con
vosotros esta pascua antes que padezca; \bibverse{16} Porque os digo que
no comeré más de ella, hasta que se cumpla en el reino de Dios.
\bibverse{17} Y tomando el vaso, habiendo dado gracias, dijo: Tomad
esto, y partidlo entre vosotros; \bibverse{18} Porque os digo, que no
beberé más del fruto de la vid, hasta que el reino de Dios venga.

\bibverse{19} Y tomando el pan, habiendo dado gracias, partió, y les
dió, diciendo: Esto es mi cuerpo, que por vosotros es dado: haced esto
en memoria de mí. \footnote{\textbf{22:19} 1Cor 11,23-25} \bibverse{20}
Asimismo también el vaso, después que hubo cenado, diciendo: Este vaso
es el nuevo pacto en mi sangre, que por vosotros se derrama.
\bibverse{21} Con todo eso, he aquí la mano del que me entrega, conmigo
en la mesa. \bibverse{22} Y á la verdad el Hijo del hombre va, según lo
que está determinado; empero ¡ay de aquel hombre por el cual es
entregado!

\bibverse{23} Ellos entonces comenzaron á preguntar entre sí, cuál de
ellos sería el que había de hacer esto.

\hypertarget{palabras-de-despedida-a-los-discuxedpulos}{%
\subsection{Palabras de despedida a los
discípulos}\label{palabras-de-despedida-a-los-discuxedpulos}}

\bibverse{24} Y hubo entre ellos una contienda, quién de ellos parecía
ser el mayor. \footnote{\textbf{22:24} Luc 9,46; Mat 20,25-28; Mar
  10,42-45} \bibverse{25} Entonces él les dijo: Los reyes de las gentes
se enseñorean de ellas; y los que sobre ellas tienen potestad, son
llamados bienhechores: \bibverse{26} Mas vosotros, no así: antes el que
es mayor entre vosotros, sea como el más mozo; y el que es príncipe,
como el que sirve. \bibverse{27} Porque, ¿cuál es mayor, el que se
sienta á la mesa, ó el que sirve? ¿No es el que se sienta á la mesa? Y
yo soy entre vosotros como el que sirve.

\bibverse{28} Empero vosotros sois los que habéis permanecido conmigo en
mis tentaciones: \footnote{\textbf{22:28} Juan 6,67-68} \bibverse{29} Yo
pues os ordeno un reino, como mi Padre me lo ordenó á mí, \bibverse{30}
Para que comáis y bebáis en mi mesa en mi reino, y os sentéis sobre
tronos juzgando á las doce tribus de Israel.

\hypertarget{advertencia-al-pedro-seguro-de-suxed-mismo-y-profecuxeda-de-su-negaciuxf3n}{%
\subsection{Advertencia al Pedro seguro de sí mismo y profecía de su
negación}\label{advertencia-al-pedro-seguro-de-suxed-mismo-y-profecuxeda-de-su-negaciuxf3n}}

\bibverse{31} Dijo también el Señor: Simón, Simón, he aquí Satanás os ha
pedido para zarandaros como á trigo; \footnote{\textbf{22:31} 2Cor 2,11}
\bibverse{32} Mas yo he rogado por ti que tu fe no falte: y tú, una vez
vuelto, confirma á tus hermanos. \footnote{\textbf{22:32} Juan 17,11;
  Juan 17,15}

\bibverse{33} Y él le dijo: Señor, pronto estoy á ir contigo aun á
cárcel y á muerte.

\bibverse{34} Y él dijo: Pedro, te digo que el gallo no cantará hoy
antes que tú niegues tres veces que me conoces.

\hypertarget{referencia-al-tiempo-que-los-discuxedpulos-vivieron-con-seguridad-y-al-futuro-serio-y-difuxedcil}{%
\subsection{Referencia al tiempo que los discípulos vivieron con
seguridad y al futuro serio y
difícil}\label{referencia-al-tiempo-que-los-discuxedpulos-vivieron-con-seguridad-y-al-futuro-serio-y-difuxedcil}}

\bibverse{35} Y á ellos dijo: Cuando os envié sin bolsa, y sin alforja,
y sin zapatos, ¿os faltó algo? Y ellos dijeron: Nada. \footnote{\textbf{22:35}
  Luc 9,3; Luc 10,4}

\bibverse{36} Y les dijo: Pues ahora, el que tiene bolsa, tómela, y
también la alforja, y el que no tiene, venda su capa y compre espada.
\bibverse{37} Porque os digo, que es necesario que se cumpla todavía en
mí aquello que está escrito: Y con los malos fué contado: porque lo que
está escrito de mí, cumplimiento tiene.

\bibverse{38} Entonces ellos dijeron: Señor, he aquí dos espadas. Y él
les dijo: Basta.

\hypertarget{la-lucha-del-alma-de-jesuxfas-y-la-oraciuxf3n-en-el-monte-de-los-olivos}{%
\subsection{La lucha del alma de Jesús y la oración en el Monte de los
Olivos}\label{la-lucha-del-alma-de-jesuxfas-y-la-oraciuxf3n-en-el-monte-de-los-olivos}}

\bibverse{39} Y saliendo, se fué, como solía, al monte de las Olivas; y
sus discípulos también le siguieron. \bibverse{40} Y como llegó á aquel
lugar, les dijo: Orad que no entréis en tentación.

\bibverse{41} Y él se apartó de ellos como un tiro de piedra; y puesto
de rodillas oró, \bibverse{42} Diciendo: Padre, si quieres, pasa este
vaso de mí; empero no se haga mi voluntad, sino la tuya.

\bibverse{43} Y le apareció un ángel del cielo confortándole.
\bibverse{44} Y estando en agonía, oraba más intensamente: y fué su
sudor como grandes gotas de sangre que caían hasta la tierra.

\bibverse{45} Y como se levantó de la oración, y vino á sus discípulos,
hallólos durmiendo de tristeza; \bibverse{46} Y les dijo: ¿Por qué
dormís? Levantaos, y orad que no entréis en tentación.

\hypertarget{captura-de-jesuxfas}{%
\subsection{Captura de Jesús}\label{captura-de-jesuxfas}}

\bibverse{47} Estando él aún hablando, he aquí una turba; y el que se
llamaba Judas, uno de los doce, iba delante de ellos; y llegóse á Jesús
para besarlo. \bibverse{48} Entonces Jesús le dijo: Judas, ¿con beso
entregas al Hijo del hombre?

\bibverse{49} Y viendo los que estaban con él lo que había de ser, le
dijeron: Señor, ¿heriremos á cuchillo? \bibverse{50} Y uno de ellos
hirió á un siervo del príncipe de los sacerdotes, y le quitó la oreja
derecha.

\bibverse{51} Entonces respondiendo Jesús, dijo: Dejad hasta aquí. Y
tocando su oreja, le sanó. \bibverse{52} Y Jesús dijo á los que habían
venido á él, los príncipes de los sacerdotes, y los magistrados del
templo, y los ancianos: ¿Como á ladrón habéis salido con espadas y con
palos? \bibverse{53} Habiendo estado con vosotros cada día en el templo,
no extendisteis las manos contra mí; mas ésta es vuestra hora, y la
potestad de las tinieblas. \footnote{\textbf{22:53} Juan 7,30; Juan 8,20}

\hypertarget{negaciuxf3n-y-arrepentimiento-de-pedro}{%
\subsection{Negación y arrepentimiento de
Pedro}\label{negaciuxf3n-y-arrepentimiento-de-pedro}}

\bibverse{54} Y prendiéndole trajéronle, y metiéronle en casa del
príncipe de los sacerdotes. Y Pedro le seguía de lejos. \bibverse{55} Y
habiendo encendido fuego en medio de la sala, y sentándose todos
alrededor, se sentó también Pedro entre ellos. \bibverse{56} Y como una
criada le vió que estaba sentado al fuego, fijóse en él, y dijo: Y éste
con él estaba.

\bibverse{57} Entonces él lo negó, diciendo: Mujer, no le conozco.

\bibverse{58} Y un poco después, viéndole otro, dijo: Y tú de ellos
eras. Y Pedro dijo: Hombre, no soy.

\bibverse{59} Y como una hora pasada otro afirmaba, diciendo:
Verdaderamente también éste estaba con él, porque es Galileo.

\bibverse{60} Y Pedro dijo: Hombre, no sé qué dices. Y luego, estando él
aún hablando, el gallo cantó. \bibverse{61} Entonces, vuelto el Señor,
miró á Pedro: y Pedro se acordó de la palabra del Señor como le había
dicho: Antes que el gallo cante, me negarás tres veces. \bibverse{62} Y
saliendo fuera Pedro, lloró amargamente.

\hypertarget{burlarse-y-maltratar-a-jesuxfas-interrogatorio-ante-el-sumo-consejo}{%
\subsection{Burlarse y maltratar a Jesús; Interrogatorio ante el sumo
consejo}\label{burlarse-y-maltratar-a-jesuxfas-interrogatorio-ante-el-sumo-consejo}}

\bibverse{63} Y los hombres que tenían á Jesús, se burlaban de él
hiriéndole; \bibverse{64} Y cubriéndole, herían su rostro, y
preguntábanle, diciendo: Profetiza quién es el que te hirió.
\bibverse{65} Y decían otras muchas cosas injuriándole.

\bibverse{66} Y cuando fué de día, se juntaron los ancianos del pueblo,
y los príncipes de los sacerdotes, y los escribas, y le trajeron á su
concilio, \bibverse{67} Diciendo: ¿Eres tú el Cristo? dínoslo. Y les
dijo: Si os lo dijere, no creeréis; \footnote{\textbf{22:67} Juan 3,12}

\bibverse{68} Y también si os preguntare, no me responderéis, ni me
soltaréis: \bibverse{69} Mas después de ahora el Hijo del hombre se
asentará á la diestra de la potencia de Dios.

\bibverse{70} Y dijeron todos: ¿Luego tú eres Hijo de Dios? Y él les
dijo: Vosotros decís que yo soy.

\bibverse{71} Entonces ellos dijeron: ¿Qué más testimonio deseamos?
porque nosotros lo hemos oído de su boca.

\hypertarget{la-acusaciuxf3n-de-los-juduxedos-y-el-interrogatorio-de-jesuxfas-ante-pilato}{%
\subsection{La acusación de los judíos y el interrogatorio de Jesús ante
Pilato}\label{la-acusaciuxf3n-de-los-juduxedos-y-el-interrogatorio-de-jesuxfas-ante-pilato}}

\hypertarget{section-22}{%
\section{23}\label{section-22}}

\bibverse{1} Levantándose entonces toda la multitud de ellos, lleváronle
á Pilato. \bibverse{2} Y comenzaron á acusarle, diciendo: A éste hemos
hallado que pervierte la nación, y que veda dar tributo á César,
diciendo que él es el Cristo, el rey. \footnote{\textbf{23:2} Luc 20,25;
  Hech 24,5}

\bibverse{3} Entonces Pilato le preguntó, diciendo: ¿Eres tú el Rey de
los Judíos? Y respondiéndole él, dijo: Tú lo dices.

\bibverse{4} Y Pilato dijo á los príncipes de los sacerdotes, y á las
gentes: Ninguna culpa hallo en este hombre.

\bibverse{5} Mas ellos porfiaban, diciendo: Alborota al pueblo,
enseñando por toda Judea, comenzando desde Galilea hasta aquí.

\bibverse{6} Entonces Pilato, oyendo de Galilea, preguntó si el hombre
era Galileo. \bibverse{7} Y como entendió que era de la jurisdicción de
Herodes, le remitió á Herodes, el cual también estaba en Jerusalem en
aquellos días.

\hypertarget{jesus-antes-herodes}{%
\subsection{Jesus antes Herodes}\label{jesus-antes-herodes}}

\bibverse{8} Y Herodes, viendo á Jesús, holgóse mucho, porque hacía
mucho que deseaba verle; porque había oído de él muchas cosas, y tenía
esperanza que le vería hacer alguna señal. \footnote{\textbf{23:8} Luc
  9,9} \bibverse{9} Y le preguntaba con muchas palabras; mas él nada le
respondió: \bibverse{10} Y estaban los príncipes de los sacerdotes y los
escribas acusándole con gran porfía. \bibverse{11} Mas Herodes con su
corte le menospreció, y escarneció, vistiéndole de una ropa rica; y
volvióle á enviar á Pilato. \bibverse{12} Y fueron hechos amigos entre
sí Pilato y Herodes en el mismo día; porque antes eran enemigos entre
sí.

\hypertarget{jesuxfas-de-nuevo-ante-pilato}{%
\subsection{Jesús de nuevo ante
Pilato}\label{jesuxfas-de-nuevo-ante-pilato}}

\bibverse{13} Entonces Pilato, convocando los príncipes de los
sacerdotes, y los magistrados, y el pueblo, \bibverse{14} Les dijo: Me
habéis presentado á éste por hombre que desvía al pueblo: y he aquí,
preguntando yo delante de vosotros, no he hallado culpa alguna en este
hombre de aquéllas de que le acusáis. \bibverse{15} Y ni aun Herodes;
porque os remití á él, y he aquí, ninguna cosa digna de muerte ha hecho.
\bibverse{16} Le soltaré, pues, castigado.

\hypertarget{jesuxfas-y-barrabuxe1s-la-condenacion}{%
\subsection{Jesús y Barrabás; la
condenacion}\label{jesuxfas-y-barrabuxe1s-la-condenacion}}

\bibverse{17} Y tenía necesidad de soltarles uno en cada fiesta.
\bibverse{18} Mas toda la multitud dió voces á una, diciendo: Quita á
éste, y suéltanos á Barrabás: \bibverse{19} (El cual había sido echado
en la cárcel por una sedición hecha en la ciudad, y una muerte.)

\bibverse{20} Y hablóles otra vez Pilato, queriendo soltar á Jesús.
\bibverse{21} Pero ellos volvieron á dar voces, diciendo: Crucifícale,
crucifícale.

\bibverse{22} Y él les dijo la tercera vez: ¿Pues qué mal ha hecho éste?
Ninguna culpa de muerte he hallado en él: le castigaré, pues, y le
soltaré. \bibverse{23} Mas ellos instaban á grandes voces, pidiendo que
fuese crucificado. Y las voces de ellos y de los príncipes de los
sacerdotes crecían. \bibverse{24} Entonces Pilato juzgó que se hiciese
lo que ellos pedían; \bibverse{25} Y les soltó á aquél que había sido
echado en la cárcel por sedición y una muerte, al cual habían pedido; y
entregó á Jesús á la voluntad de ellos.

\hypertarget{el-camino-de-la-muerte-de-jesuxfas-al-guxf3lgota-y-sus-palabras-a-las-mujeres-de-luto-de-jerusaluxe9n-su-crucifixiuxf3n-y-su-muerte}{%
\subsection{El camino de la muerte de Jesús al Gólgota y sus palabras a
las mujeres de luto de Jerusalén; su crucifixión y su
muerte}\label{el-camino-de-la-muerte-de-jesuxfas-al-guxf3lgota-y-sus-palabras-a-las-mujeres-de-luto-de-jerusaluxe9n-su-crucifixiuxf3n-y-su-muerte}}

\bibverse{26} Y llevándole, tomaron á un Simón Cireneo, que venía del
campo, y le pusieron encima la cruz para que la llevase tras Jesús.
\bibverse{27} Y le seguía una grande multitud de pueblo, y de mujeres,
las cuales le lloraban y lamentaban. \bibverse{28} Mas Jesús, vuelto á
ellas, les dice: Hijas de Jerusalem, no me lloréis á mí, mas llorad por
vosotras mismas, y por vuestros hijos. \bibverse{29} Porque he aquí
vendrán días en que dirán: Bienaventuradas las estériles, y los vientres
que no engendraron, y los pechos que no criaron. \footnote{\textbf{23:29}
  Luc 21,23} \bibverse{30} Entonces comenzarán á decir á los montes:
Caed sobre nosotros: y á los collados: Cubridnos. \footnote{\textbf{23:30}
  Os 10,8; Apoc 6,16; Apoc 9,6} \bibverse{31} Porque si en el árbol
verde hacen estas cosas, ¿en el seco, qué se hará? \footnote{\textbf{23:31}
  1Pe 4,17}

\bibverse{32} Y llevaban también con él otros dos, malhechores, á ser
muertos. \bibverse{33} Y como vinieron al lugar que se llama de la
Calavera, le crucificaron allí, y á los malhechores, uno á la derecha, y
otro á la izquierda.

\bibverse{34} Y Jesús decía: Padre, perdónalos, porque no saben lo que
hacen. Y partiendo sus vestidos, echaron suertes.

\bibverse{35} Y el pueblo estaba mirando; y se burlaban de él los
príncipes con ellos, diciendo: A otros hizo salvos: sálvese á sí, si
éste es el Mesías, el escogido de Dios.

\bibverse{36} Escarnecían de él también los soldados, llegándose y
presentándole vinagre, \bibverse{37} Y diciendo: Si tú eres el Rey de
los Judíos, sálvate á ti mismo.

\bibverse{38} Y había también sobre él un título escrito con letras
griegas, y latinas, y hebraicas: ESTE ES EL REY DE LOS JUDIOS.

\hypertarget{jesuxfas-y-los-dos-ladrones}{%
\subsection{Jesús y los dos
ladrones}\label{jesuxfas-y-los-dos-ladrones}}

\bibverse{39} Y uno de los malhechores que estaban colgados, le
injuriaba, diciendo: Si tú eres el Cristo, sálvate á ti mismo y á
nosotros.

\bibverse{40} Y respondiendo el otro, reprendióle, diciendo: ¿Ni aun tú
temes á Dios, estando en la misma condenación? \bibverse{41} Y nosotros,
á la verdad, justamente padecemos; porque recibimos lo que merecieron
nuestros hechos: mas éste ningún mal hizo. \bibverse{42} Y dijo á Jesús:
Acuérdate de mí cuando vinieres á tu reino. \footnote{\textbf{23:42} Mat
  16,28}

\bibverse{43} Entonces Jesús le dijo: De cierto te digo, que hoy estarás
conmigo en el paraíso. \footnote{\textbf{23:43} 2Cor 12,4; Apoc 14,13}

\hypertarget{la-muerte-de-jesuxfas-las-seuxf1ales-milagrosas-de-su-muerte}{%
\subsection{La muerte de Jesús; las señales milagrosas de su
muerte}\label{la-muerte-de-jesuxfas-las-seuxf1ales-milagrosas-de-su-muerte}}

\bibverse{44} Y cuando era como la hora de sexta, fueron hechas
tinieblas sobre toda la tierra hasta la hora de nona. \bibverse{45} Y el
sol se obscureció: y el velo del templo se rompió por medio. \footnote{\textbf{23:45}
  Éxod 36,35} \bibverse{46} Entonces Jesús, clamando á gran voz, dijo:
Padre, en tus manos encomiendo mi espíritu. Y habiendo dicho esto,
espiró. \footnote{\textbf{23:46} Sal 31,6; Hech 7,58}

\bibverse{47} Y como el centurión vió lo que había acontecido, dió
gloria á Dios, diciendo: Verdaderamente este hombre era justo.
\bibverse{48} Y toda la multitud de los que estaban presentes á este
espectáculo, viendo lo que había acontecido, se volvían hiriendo sus
pechos. \bibverse{49} Mas todos sus conocidos, y las mujeres que le
habían seguido desde Galilea, estaban lejos mirando estas cosas.
\footnote{\textbf{23:49} Luc 8,2-3}

\hypertarget{el-entierro-de-jesuxfas}{%
\subsection{El entierro de Jesús}\label{el-entierro-de-jesuxfas}}

\bibverse{50} Y he aquí un varón llamado José, el cual era senador,
varón bueno y justo, \bibverse{51} (El cual no había consentido en el
consejo ni en los hechos de ellos), de Arimatea, ciudad de la Judea, el
cual también esperaba el reino de Dios; \bibverse{52} Este llegó á
Pilato, y pidió el cuerpo de Jesús. \bibverse{53} Y quitado, lo envolvió
en una sábana, y le puso en un sepulcro abierto en una peña, en el cual
ninguno había aún sido puesto. \bibverse{54} Y era día de la víspera de
la Pascua; y estaba para rayar el sábado. \bibverse{55} Y las mujeres
que con él habían venido de Galilea, siguieron también y vieron el
sepulcro, y cómo fué puesto su cuerpo. \bibverse{56} Y vueltas,
aparejaron drogas aromáticas y ungüentos; y reposaron el sábado,
conforme al mandamiento. \footnote{\textbf{23:56} Éxod 20,10}

\hypertarget{descubrimiento-de-la-tumba-vacuxeda-en-la-mauxf1ana-de-pascua-la-revelaciuxf3n-a-las-mujeres}{%
\subsection{Descubrimiento de la tumba vacía en la mañana de Pascua; la
revelación a las
mujeres}\label{descubrimiento-de-la-tumba-vacuxeda-en-la-mauxf1ana-de-pascua-la-revelaciuxf3n-a-las-mujeres}}

\hypertarget{section-23}{%
\section{24}\label{section-23}}

\bibverse{1} Y el primer día de la semana, muy de mañana, vinieron al
sepulcro, trayendo las drogas aromáticas que habían aparejado, y algunas
otras mujeres con ellas. \bibverse{2} Y hallaron la piedra revuelta del
sepulcro. \bibverse{3} Y entrando, no hallaron el cuerpo del Señor
Jesús. \bibverse{4} Y aconteció, que estando ellas espantadas de esto,
he aquí se pararon junto á ellas dos varones con vestiduras
resplandecientes; \bibverse{5} Y como tuviesen ellas temor, y bajasen el
rostro á tierra, les dijeron: ¿Por qué buscáis entre los muertos al que
vive?

\bibverse{6} No está aquí, mas ha resucitado: acordaos de lo que os
habló, cuando aun estaba en Galilea, \bibverse{7} Diciendo: Es menester
que el Hijo del hombre sea entregado en manos de hombres pecadores, y
que sea crucificado, y resucite al tercer día.

\bibverse{8} Entonces ellas se acordaron de sus palabras, \bibverse{9} Y
volviendo del sepulcro, dieron nuevas de todas estas cosas á los once, y
á todos los demás. \bibverse{10} Y eran María Magdalena, y Juana, y
María madre de Jacobo, y las demás con ellas, las que dijeron estas
cosas á los apóstoles. \footnote{\textbf{24:10} Luc 8,2-3} \bibverse{11}
Mas á ellos les parecían como locura las palabras de ellas, y no las
creyeron. \bibverse{12} Pero levantándose Pedro, corrió al sepulcro: y
como miró dentro, vió solos los lienzos echados; y se fué maravillándose
de lo que había sucedido.

\hypertarget{los-discuxedpulos-de-emauxfas}{%
\subsection{Los discípulos de
Emaús}\label{los-discuxedpulos-de-emauxfas}}

\bibverse{13} Y he aquí, dos de ellos iban el mismo día á una aldea que
estaba de Jerusalem sesenta estadios, llamada Emmaús. \bibverse{14} E
iban hablando entre sí de todas aquellas cosas que habían acaecido.
\bibverse{15} Y aconteció que yendo hablando entre sí, y preguntándose
el uno al otro, el mismo Jesús se llegó, é iba con ellos juntamente.
\bibverse{16} Mas los ojos de ellos estaban embargados, para que no le
conociesen. \bibverse{17} Y díjoles: ¿Qué pláticas son estas que tratáis
entre vosotros andando, y estáis tristes?

\bibverse{18} Y respondiendo el uno, que se llamaba Cleofas, le dijo:
¿Tú sólo peregrino eres en Jerusalem, y no has sabido las cosas que en
ella han acontecido estos días?

\bibverse{19} Entonces él les dijo: ¿Qué cosas? Y ellos le dijeron: De
Jesús Nazareno, el cual fué varón profeta, poderoso en obra y en palabra
delante de Dios y de todo el pueblo; \footnote{\textbf{24:19} Mat 21,11}

\bibverse{20} Y cómo le entregaron los príncipes de los sacerdotes y
nuestros príncipes á condenación de muerte, y le crucificaron.
\bibverse{21} Mas nosotros esperábamos que él era el que había de
redimir á Israel: y ahora sobre todo esto, hoy es el tercer día que esto
ha acontecido. \bibverse{22} Aunque también unas mujeres de los nuestros
nos han espantado, las cuales antes del día fueron al sepulcro:
\bibverse{23} Y no hallando su cuerpo, vinieron diciendo que también
habían visto visión de ángeles, los cuales dijeron que él vive.
\bibverse{24} Y fueron algunos de los nuestros al sepulcro, y hallaron
así como las mujeres habían dicho; mas á él no le vieron.

\bibverse{25} Entonces él les dijo: ¡Oh insensatos, y tardos de corazón
para creer todo lo que los profetas han dicho! \bibverse{26} ¿No era
necesario que el Cristo padeciera estas cosas, y que entrara en su
gloria? \bibverse{27} Y comenzando desde Moisés, y de todos los
profetas, declarábales en todas las Escrituras lo que de él decían.
\footnote{\textbf{24:27} Deut 18,15; Sal 22,-1; Is 53,-1}

\bibverse{28} Y llegaron á la aldea á donde iban: y él hizo como que iba
más lejos.

\bibverse{29} Mas ellos le detuvieron por fuerza, diciendo: Quédate con
nosotros, porque se hace tarde, y el día ya ha declinado. Entró pues á
estarse con ellos.

\bibverse{30} Y aconteció, que estando sentado con ellos á la mesa,
tomando el pan, bendijo, y partió, y dióles. \bibverse{31} Entonces
fueron abiertos los ojos de ellos, y le conocieron; mas él se
desapareció de los ojos de ellos. \bibverse{32} Y decían el uno al otro:
¿No ardía nuestro corazón en nosotros, mientras nos hablaba en el
camino, y cuando nos abría las Escrituras? \bibverse{33} Y levantándose
en la misma hora, tornáronse á Jerusalem, y hallaron á los once
reunidos, y á los que estaban con ellos. \bibverse{34} Que decían: Ha
resucitado el Señor verdaderamente, y ha aparecido á Simón. \footnote{\textbf{24:34}
  1Cor 15,4-5} \bibverse{35} Entonces ellos contaban las cosas que les
habían acontecido en el camino, y cómo había sido conocido de ellos al
partir el pan.

\hypertarget{jesuxfas-se-apareciuxf3-al-cuxedrculo-de-los-discuxedpulos-la-noche-del-domingo-de-pascua-su-mandato-misionero-y-despedida-de-los-discuxedpulos}{%
\subsection{Jesús se apareció al círculo de los discípulos la noche del
domingo de Pascua; su mandato misionero y despedida de los
discípulos}\label{jesuxfas-se-apareciuxf3-al-cuxedrculo-de-los-discuxedpulos-la-noche-del-domingo-de-pascua-su-mandato-misionero-y-despedida-de-los-discuxedpulos}}

\bibverse{36} Y entre tanto que ellos hablaban estas cosas, él se puso
en medio de ellos, y les dijo: Paz á vosotros.

\bibverse{37} Entonces ellos espantados y asombrados, pensaban que veían
espíritu.

\bibverse{38} Mas él les dice: ¿Por qué estáis turbados, y suben
pensamientos á vuestros corazones? \bibverse{39} Mirad mis manos y mis
pies, que yo mismo soy: palpad, y ved; que el espíritu ni tiene carne ni
huesos, como veis que yo tengo. \bibverse{40} Y en diciendo esto, les
mostró las manos y los pies. \footnote{\textbf{24:40} Juan 20,20}
\bibverse{41} Y no creyéndolo aún ellos de gozo, y maravillados,
díjoles: ¿Tenéis aquí algo de comer?

\bibverse{42} Entonces ellos le presentaron parte de un pez asado, y un
panal de miel. \bibverse{43} Y él tomó, y comió delante de ellos.
\bibverse{44} Y él les dijo: Estas son las palabras que os hablé,
estando aún con vosotros: que era necesario que se cumpliesen todas las
cosas que están escritas de mí en la ley de Moisés, y en los profetas, y
en los salmos. \footnote{\textbf{24:44} Luc 9,22; Luc 18,31-33}

\bibverse{45} Entonces les abrió el sentido, para que entendiesen las
Escrituras; \footnote{\textbf{24:45} Luc 9,45} \bibverse{46} Y díjoles:
Así está escrito, y así fué necesario que el Cristo padeciese, y
resucitase de los muertos al tercer día; \footnote{\textbf{24:46} Os
  6,2; Juan 12,16} \bibverse{47} Y que se predicase en su nombre el
arrepentimiento y la remisión de pecados en todas las naciones,
comenzando de Jerusalem. \footnote{\textbf{24:47} Hech 2,38; Hech 17,30}
\bibverse{48} Y vosotros sois testigos de estas cosas. \bibverse{49} Y
he aquí, yo enviaré la promesa de mi Padre sobre vosotros: mas vosotros
asentad en la ciudad de Jerusalem, hasta que seáis investidos de
potencia de lo alto. \footnote{\textbf{24:49} Juan 15,26; Juan 16,7;
  Hech 2,1-4}

\hypertarget{ascensiuxf3n-de-jesuxfas}{%
\subsection{Ascensión de jesús}\label{ascensiuxf3n-de-jesuxfas}}

\bibverse{50} Y sacólos fuera hasta Bethania, y alzando sus manos, los
bendijo. \bibverse{51} Y aconteció que bendiciéndolos, se fué de ellos;
y era llevado arriba al cielo. \bibverse{52} Y ellos, después de haberle
adorado, se volvieron á Jerusalem con gran gozo; \bibverse{53} Y estaban
siempre en el templo, alabando y bendiciendo á Dios. Amén.
